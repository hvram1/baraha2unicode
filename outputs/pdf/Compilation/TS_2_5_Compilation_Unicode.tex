\documentclass[17pt]{extarticle}
\usepackage{babel}
\usepackage{fontspec}
\usepackage{polyglossia}
\usepackage{extsizes}



\setmainlanguage{sanskrit}
\setotherlanguages{english} %% or other languages
\setlength{\parindent}{0pt}
\pagestyle{myheadings}
\newfontfamily\devanagarifont[Script=Devanagari]{AdishilaVedic}


\newcommand{\VAR}[1]{}
\newcommand{\BLOCK}[1]{}




\begin{document}
\begin{titlepage}
    \begin{center}
 
\begin{sanskrit}
    { \Huge
    कृष्ण यजुर्वेदीय तैत्तिरीय संहिता,पद,जटा,घन पाठः 
    }
    \\
    \vspace{2.5cm}
    \mbox{ \Huge
    2.5      द्वितीयकाण्डे पञ्चमः प्रश्नः - इष्टिविधानं   }
\end{sanskrit}
\end{center}

\end{titlepage}
\tableofcontents
\pagebreak

\markright{ TS 2.5.1.1  \hfill https://www.vedavms.in \hfill}
\addcontentsline{toc}{section}{ TS 2.5.1.1 }
\section*{ TS 2.5.1.1 }

\textbf{TS 2.5.1.1 } \newline
\textbf{Samhita Paata} \newline

वि॒श्वरू॑पो॒ वै त्वा॒ष्ट्रः पु॒रोहि॑तो दे॒वाना॑मासीथ् स्व॒स्रीयोऽसु॑राणां॒ तस्य॒ त्रीणि॑ शी॒र्॒.षाण्या॑सन्थ्-सोम॒पानꣳ॑ सुरा॒पान॑-म॒न्नाद॑नꣳ॒॒ स प्र॒त्यक्षं॑ दे॒वेभ्यो॑ भा॒गम॑वदत् प॒रोक्ष॒मसु॑रेभ्यः॒ सर्व॑स्मै॒ वै प्र॒त्यक्षं॑ भा॒गं ॅव॑दन्ति॒ यस्मा॑ ए॒व प॒रोक्षं॒ ॅवद॑न्ति॒ तस्य॑ भा॒ग उ॑दि॒तस्तस्मा॒दिन्द्रो॑ ऽबिभेदी॒दृङ् वै रा॒ष्ट्रं ॅवि प॒र्याव॑र्तय॒तीति॒ तस्य॒ वज्र॑मा॒दाय॑ शी॒र्॒.षाण्य॑च्छिन॒द्यथ् सो॑म॒पान॒ - [  ] \newline

\textbf{Pada Paata} \newline

वि॒श्वरू॑प॒ इति॑ वि॒श्व - रू॒पः॒ । वै । त्वा॒ष्ट्रः । पु॒रोहि॑त॒ इति॑ पु॒रः-हि॒तः॒ । दे॒वाना᳚म् । आ॒सी॒त् । स्व॒स्रीयः॑ । असु॑राणाम् । तस्य॑ । त्रीणि॑ । शी॒र्.॒षाणि॑ । आ॒स॒न्न् । सो॒म॒पान॒मिति॑ सोम - पान᳚म् । सु॒रा॒पान॒मिति॑ सुरा - पान᳚म् । अ॒न्नाद॑न॒मित्य॑न्न - अद॑नम् । सः । प्र॒त्यक्ष॒मिति॑ प्रति - अक्ष᳚म् । दे॒वेभ्यः॑ । भा॒गम् । अ॒व॒द॒त् । प॒रोक्ष॒मिति॑ परः - अक्ष᳚म् । असु॑रेभ्यः । सर्व॑स्मै । वै । प्र॒त्यक्ष॒मिति॑ प्रति - अक्ष᳚म् । भा॒गम् । व॒द॒न्ति॒ । यस्मै᳚ । ए॒व । प॒रोक्ष॒मिति॑ परः - अक्ष᳚म् । वद॑न्ति । तस्य॑ । भा॒गः । उ॒दि॒तः । तस्मा᳚त् । इन्द्रः॑ । अ॒बि॒भे॒त् । ई॒दृङ् । वै । रा॒ष्ट्रम् । वीति॑ । प॒र्याव॑र्तय॒तीति॑ परि - आव॑र्तयति । इति॑ । तस्य॑ । वज्र᳚म् । आ॒दायेत्या᳚ - दाय॑ । शी॒र्.॒षाणि॑ । अ॒च्छि॒न॒त् । यत् । सो॒म॒पान॒मिति॑ सोम-पान᳚म् ।  \newline


\textbf{Krama Paata} \newline

वि॒श्वरू॑पो॒ वै । वि॒श्वरू॑प॒ इति॑ वि॒श्व - रू॒पः॒ । वै त्वा॒ष्ट्रः । त्वा॒ष्ट्रः पु॒रोहि॑तः । पु॒रोहि॑तो दे॒वाना᳚म् । पु॒रोहि॑त॒ इति॑ पु॒रः - हि॒तः॒ । दे॒वाना॑मासीत् । आ॒सी॒थ् स्व॒स्रीयः॑ । स्व॒स्रीयो ऽसु॑राणाम् । असु॑राणा॒म् तस्य॑ । तस्य॒ त्रीणि॑ । त्रीणि॑ शी॒र्.॒षाणि॑ । शी॒र्.॒षाण्या॑सन्न् । आ॒स॒न्थ् सो॒म॒पान᳚म् । सो॒म॒पानꣳ॑ सुरा॒पान᳚म् । सो॒म॒पान॒मिति॑ सोम - पान᳚म् । सु॒रा॒पान॑म॒न्नाद॑नम् । सु॒रा॒पान॒मिति॑ सुरा - पान᳚म् । अ॒न्नाद॑नꣳ॒॒ सः । अ॒न्नाद॑न॒मित्य॑न्न - अद॑नम् । स प्र॒त्यक्ष᳚म् । प्र॒त्यक्ष॑म् दे॒वेभ्यः॑ । प्र॒त्यक्ष॒मिति॑ प्रति - अक्ष᳚म् । दे॒वेभ्यो॑ भा॒गम् । भा॒गम॑वदत् । अ॒व॒द॒त् प॒रोक्ष᳚म् । प॒रोक्ष॒मसु॑रेभ्यः । प॒रोक्ष॒मिति॑ परः - अक्ष᳚म् । असु॑रेभ्यः॒ सर्व॑स्मै । सर्व॑स्मै॒ वै । वै प्र॒त्यक्ष᳚म् । प्र॒त्यक्ष॑म् भा॒गम् । प्र॒त्यक्ष॒मिति॑ प्रति - अक्ष᳚म् । भा॒गम् ॅव॑दन्ति । व॒द॒न्ति॒ यस्मै᳚ । यस्मा॑ ए॒व । ए॒व प॒रोक्ष᳚म् । प॒रोक्ष॒म् ॅवद॑न्ति । प॒रोक्ष॒मिति॑ परः - अक्ष᳚म् । वद॑न्ति॒ तस्य॑ । तस्य॑ भा॒गः । भा॒ग उ॑दि॒तः । उ॒दि॒तस्तस्मा᳚त् । तस्मा॒दिन्द्रः॑ । इन्द्रो॑ ऽबिभेत् । अ॒बि॒भे॒दी॒दृङ्ङ् । ई॒दृङ् वै । वै रा॒ष्ट्रम् । रा॒ष्ट्रम् ॅवि । वि प॒र्याव॑र्तयति । प॒र्याव॑र्तय॒तीति॑ । प॒र्याव॑र्तय॒तीति॑ परि - आव॑र्तयति । इति॒ तस्य॑ । तस्य॒ वज्र᳚म् । वज्र॑मा॒दाय॑ । आ॒दाय॑ शी॒र्.॒षाणि॑ । आ॒दायेत्या᳚ - दाय॑ । शी॒र्.॒षाण्य॑च्छिनत् । अ॒च्छि॒न॒द् यत् । यथ् सो॑म॒पान᳚म् । सो॒म॒पान॒मासी᳚त् । सो॒म॒पान॒मिति॑ सोम - पान᳚म् \newline

\textbf{Jatai Paata} \newline

1. वि॒श्वरू॑पो॒ वै वै वि॒श्वरू॑पो वि॒श्वरू॑पो॒ वै । \newline
2. वि॒श्वरू॑प॒ इति॑ वि॒श्व - रू॒पः॒ । \newline
3. वै त्वा॒ष्ट्र स्त्वा॒ष्ट्रो वै वै त्वा॒ष्ट्रः । \newline
4. त्वा॒ष्ट्रः पु॒रोहि॑तः पु॒रोहि॑त स्त्वा॒ष्ट्र स्त्वा॒ष्ट्रः पु॒रोहि॑तः । \newline
5. पु॒रोहि॑तो दे॒वाना᳚म् दे॒वाना᳚म् पु॒रोहि॑तः पु॒रोहि॑तो दे॒वाना᳚म् । \newline
6. पु॒रोहि॑त॒ इति॑ पु॒रः - हि॒तः॒ । \newline
7. दे॒वाना॑ मासी दासीद् दे॒वाना᳚म् दे॒वाना॑ मासीत् । \newline
8. आ॒सी॒थ् स्व॒स्रीयः॑ स्व॒स्रीय॑ आसी दासीथ् स्व॒स्रीयः॑ । \newline
9. स्व॒स्रीयो ऽसु॑राणा॒ मसु॑राणाꣳ स्व॒स्रीयः॑ स्व॒स्रीयो ऽसु॑राणाम् । \newline
10. असु॑राणा॒म् तस्य॒ तस्या सु॑राणा॒ मसु॑राणा॒म् तस्य॑ । \newline
11. तस्य॒ त्रीणि॒ त्रीणि॒ तस्य॒ तस्य॒ त्रीणि॑ । \newline
12. त्रीणि॑ शी॒र्॒.षाणि॑ शी॒र्॒.षाणि॒ त्रीणि॒ त्रीणि॑ शी॒र्॒.षाणि॑ । \newline
13. शी॒र्॒.षा ण्या॑सन् नासञ् छी॒र्॒.षाणि॑ शी॒र्॒.षा ण्या॑सन्न् । \newline
14. आ॒स॒न् थ्सो॒म॒पानꣳ॑ सोम॒पान॑ मासन् नासन् थ्सोम॒पान᳚म् । \newline
15. सो॒म॒पानꣳ॑ सुरा॒पानꣳ॑ सुरा॒पानꣳ॑ सोम॒पानꣳ॑ सोम॒पानꣳ॑ सुरा॒पान᳚म् । \newline
16. सो॒म॒पान॒मिति॑ सोम - पान᳚म् । \newline
17. सु॒रा॒पान॑ म॒न्नाद॑न म॒न्नाद॑नꣳ सुरा॒पानꣳ॑ सुरा॒पान॑ म॒न्नाद॑नम् । \newline
18. सु॒रा॒पान॒मिति॑ सुरा - पान᳚म् । \newline
19. अ॒न्नाद॑नꣳ॒॒ स सो᳚ ऽन्नाद॑न म॒न्नाद॑नꣳ॒॒ सः । \newline
20. अ॒न्नाद॑न॒मित्य॑न्न - अद॑नम् । \newline
21. स प्र॒त्यक्ष॑म् प्र॒त्यक्षꣳ॒॒ स स प्र॒त्यक्ष᳚म् । \newline
22. प्र॒त्यक्ष॑म् दे॒वेभ्यो॑ दे॒वेभ्यः॑ प्र॒त्यक्ष॑म् प्र॒त्यक्ष॑म् दे॒वेभ्यः॑ । \newline
23. प्र॒त्यक्ष॒मिति॑ प्रति - अक्ष᳚म् । \newline
24. दे॒वेभ्यो॑ भा॒गम् भा॒गम् दे॒वेभ्यो॑ दे॒वेभ्यो॑ भा॒गम् । \newline
25. भा॒ग म॑वद दवदद् भा॒गम् भा॒ग म॑वदत् । \newline
26. अ॒व॒द॒त् प॒रोक्ष॑म् प॒रोक्ष॑ मवद दवदत् प॒रोक्ष᳚म् । \newline
27. प॒रोक्ष॒ मसु॑रे॒भ्यो ऽसु॑रेभ्यः प॒रोक्ष॑म् प॒रोक्ष॒ मसु॑रेभ्यः । \newline
28. प॒रोक्ष॒मिति॑ परः - अक्ष᳚म् । \newline
29. असु॑रेभ्यः॒ सर्व॑स्मै॒ सर्व॑स्मा॒ असु॑रे॒भ्यो ऽसु॑रेभ्यः॒ सर्व॑स्मै । \newline
30. सर्व॑स्मै॒ वै वै सर्व॑स्मै॒ सर्व॑स्मै॒ वै । \newline
31. वै प्र॒त्यक्ष॑म् प्र॒त्यक्षं॒ ॅवै वै प्र॒त्यक्ष᳚म् । \newline
32. प्र॒त्यक्ष॑म् भा॒गम् भा॒गम् प्र॒त्यक्ष॑म् प्र॒त्यक्ष॑म् भा॒गम् । \newline
33. प्र॒त्यक्ष॒मिति॑ प्रति - अक्ष᳚म् । \newline
34. भा॒गं ॅव॑दन्ति वदन्ति भा॒गम् भा॒गं ॅव॑दन्ति । \newline
35. व॒द॒न्ति॒ यस्मै॒ यस्मै॑ वदन्ति वदन्ति॒ यस्मै᳚ । \newline
36. यस्मा॑ ए॒वैव यस्मै॒ यस्मा॑ ए॒व । \newline
37. ए॒व प॒रोक्ष॑म् प॒रोक्ष॑ मे॒वैव प॒रोक्ष᳚म् । \newline
38. प॒रोक्षं॒ ॅवद॑न्ति॒ वद॑न्ति प॒रोक्ष॑म् प॒रोक्षं॒ ॅवद॑न्ति । \newline
39. प॒रोक्ष॒मिति॑ परः - अक्ष᳚म् । \newline
40. वद॑न्ति॒ तस्य॒ तस्य॒ वद॑न्ति॒ वद॑न्ति॒ तस्य॑ । \newline
41. तस्य॑ भा॒गो भा॒ग स्तस्य॒ तस्य॑ भा॒गः । \newline
42. भा॒ग उ॑दि॒त उ॑दि॒तो भा॒गो भा॒ग उ॑दि॒तः । \newline
43. उ॒दि॒त स्तस्मा॒त् तस्मा॑ दुदि॒त उ॑दि॒त स्तस्मा᳚त् । \newline
44. तस्मा॒ दिन्द्र॒ इन्द्र॒ स्तस्मा॒त् तस्मा॒ दिन्द्रः॑ । \newline
45. इन्द्रो॑ ऽबिभे दबिभे॒ दिन्द्र॒ इन्द्रो॑ ऽबिभेत् । \newline
46. अ॒बि॒भे॒ दी॒दृङ् ई॒दृङ् ङ॑बिभे दबिभे दी॒दृङ् । \newline
47. ई॒दृङ् वै वा ई॒दृङ् ई॒दृङ् वै । \newline
48. वै रा॒ष्ट्रꣳ रा॒ष्ट्रं ॅवै वै रा॒ष्ट्रम् । \newline
49. रा॒ष्ट्रं ॅवि वि रा॒ष्ट्रꣳ रा॒ष्ट्रं ॅवि । \newline
50. वि प॒र्याव॑र्तयति प॒र्याव॑र्तयति॒ वि वि प॒र्याव॑र्तयति । \newline
51. प॒र्याव॑र्तय॒तीतीति॑ प॒र्याव॑र्तयति प॒र्याव॑र्तय॒तीति॑ । \newline
52. प॒र्याव॑र्तय॒तीति॑ परि - आव॑र्तयति । \newline
53. इति॒ तस्य॒ तस्ये तीति॒ तस्य॑ । \newline
54. तस्य॒ वज्रं॒ ॅवज्र॒म् तस्य॒ तस्य॒ वज्र᳚म् । \newline
55. वज्र॑ मा॒दाया॒ दाय॒ वज्रं॒ ॅवज्र॑ मा॒दाय॑ । \newline
56. आ॒दाय॑ शी॒र्॒.षाणि॑ शी॒र्॒.षा ण्या॒दाया॒ दाय॑ शी॒र्॒.षाणि॑ । \newline
57. आ॒दायेत्या᳚ - दाय॑ । \newline
58. शी॒र्॒.षा ण्य॑च्छिन दच्छिनच् छी॒र्॒.षाणि॑ शी॒र्॒.षा ण्य॑च्छिनत् । \newline
59. अ॒च्छि॒न॒द् यद् यद॑च्छिन दच्छिन॒द् यत् । \newline
60. यथ् सो॑म॒पानꣳ॑ सोम॒पानं॒ ॅयद् यथ् सो॑म॒पान᳚म् । \newline
61. सो॒म॒पान॒ मासी॒ दासी᳚थ् सोम॒पानꣳ॑ सोम॒पान॒ मासी᳚त् । \newline
62. सो॒म॒पान॒मिति॑ सोम - पान᳚म् । \newline

\textbf{Ghana Paata } \newline

1. वि॒श्वरू॑पो॒ वै वै वि॒श्वरू॑पो वि॒श्वरू॑पो॒ वै त्वा॒ष्ट्र स्त्वा॒ष्ट्रो वै वि॒श्वरू॑पो वि॒श्वरू॑पो॒ वै त्वा॒ष्ट्रः । \newline
2. वि॒श्वरू॑प॒ इति॑ वि॒श्व - रू॒पः॒ । \newline
3. वै त्वा॒ष्ट्र स्त्वा॒ष्ट्रो वै वै त्वा॒ष्ट्रः पु॒रोहि॑तः पु॒रोहि॑त स्त्वा॒ष्ट्रो वै वै त्वा॒ष्ट्रः पु॒रोहि॑तः । \newline
4. त्वा॒ष्ट्रः पु॒रोहि॑तः पु॒रोहि॑त स्त्वा॒ष्ट्र स्त्वा॒ष्ट्रः पु॒रोहि॑तो दे॒वाना᳚म् दे॒वाना᳚म् पु॒रोहि॑त स्त्वा॒ष्ट्र स्त्वा॒ष्ट्रः पु॒रोहि॑तो दे॒वाना᳚म् । \newline
5. पु॒रोहि॑तो दे॒वाना᳚म् दे॒वाना᳚म् पु॒रोहि॑तः पु॒रोहि॑तो दे॒वाना॑ मासी दासीद् दे॒वाना᳚म् पु॒रोहि॑तः पु॒रोहि॑तो दे॒वाना॑ मासीत् । \newline
6. पु॒रोहि॑त॒ इति॑ पु॒रः - हि॒तः॒ । \newline
7. दे॒वाना॑ मासी दासीद् दे॒वाना᳚म् दे॒वाना॑ मासीथ् स्व॒स्रीयः॑ स्व॒स्रीय॑ आसीद् दे॒वाना᳚म् दे॒वाना॑ मासीथ् स्व॒स्रीयः॑ । \newline
8. आ॒सी॒थ् स्व॒स्रीयः॑ स्व॒स्रीय॑ आसी दासीथ् स्व॒स्रीयो ऽसु॑राणा॒ मसु॑राणाꣳ स्व॒स्रीय॑ आसी दासीथ् स्व॒स्रीयो ऽसु॑राणाम् । \newline
9. स्व॒स्रीयो ऽसु॑राणा॒ मसु॑राणाꣳ स्व॒स्रीयः॑ स्व॒स्रीयो ऽसु॑राणा॒म् तस्य॒ तस्या सु॑राणाꣳ स्व॒स्रीयः॑ स्व॒स्रीयो ऽसु॑राणा॒म् तस्य॑ । \newline
10. असु॑राणा॒म् तस्य॒ तस्या सु॑राणा॒ मसु॑राणा॒म् तस्य॒ त्रीणि॒ त्रीणि॒ तस्या सु॑राणा॒ मसु॑राणा॒म् तस्य॒ त्रीणि॑ । \newline
11. तस्य॒ त्रीणि॒ त्रीणि॒ तस्य॒ तस्य॒ त्रीणि॑ शी॒र्॒.षाणि॑ शी॒र्॒.षाणि॒ त्रीणि॒ तस्य॒ तस्य॒ त्रीणि॑ शी॒र्॒.षाणि॑ । \newline
12. त्रीणि॑ शी॒र्॒.षाणि॑ शी॒र्॒.षाणि॒ त्रीणि॒ त्रीणि॑ शी॒र्॒.षा ण्या॑सन् नासञ् छी॒र्॒.षाणि॒ त्रीणि॒ त्रीणि॑ शी॒र्॒.षा ण्या॑सन्न् । \newline
13. शी॒र्॒.षा ण्या॑सन् नासञ् छी॒र्॒.षाणि॑ शी॒र्॒.षा ण्या॑सन् थ्सोम॒पानꣳ॑ सोम॒पान॑ मासञ् छी॒र्॒.षाणि॑ शी॒र्॒.षा ण्या॑सन् थ्सोम॒पान᳚म् । \newline
14. आ॒स॒न् थ्सो॒म॒पानꣳ॑ सोम॒पान॑ मासन् नासन् थ्सोम॒पानꣳ॑ सुरा॒पानꣳ॑ सुरा॒पानꣳ॑ सोम॒पान॑ मासन् नासन् थ्सोम॒पानꣳ॑ सुरा॒पान᳚म् । \newline
15. सो॒म॒पानꣳ॑ सुरा॒पानꣳ॑ सुरा॒पानꣳ॑ सोम॒पानꣳ॑ सोम॒पानꣳ॑ सुरा॒पान॑ म॒न्नाद॑न म॒न्नाद॑नꣳ सुरा॒पानꣳ॑ सोम॒पानꣳ॑ सोम॒पानꣳ॑ सुरा॒पान॑ म॒न्नाद॑नम् । \newline
16. सो॒म॒पान॒मिति॑ सोम - पान᳚म् । \newline
17. सु॒रा॒पान॑ म॒न्नाद॑न म॒न्नाद॑नꣳ सुरा॒पानꣳ॑ सुरा॒पान॑ म॒न्नाद॑नꣳ॒॒ स सो᳚ ऽन्नाद॑नꣳ सुरा॒पानꣳ॑ सुरा॒पान॑ म॒न्नाद॑नꣳ॒॒ सः । \newline
18. सु॒रा॒पान॒मिति॑ सुरा - पान᳚म् । \newline
19. अ॒न्नाद॑नꣳ॒॒ स सो᳚ ऽन्नाद॑न म॒न्नाद॑नꣳ॒॒ स प्र॒त्यक्ष॑म् प्र॒त्यक्षꣳ॒॒ सो᳚ ऽन्नाद॑न म॒न्नाद॑नꣳ॒॒ स प्र॒त्यक्ष᳚म् । \newline
20. अ॒न्नाद॑न॒मित्य॑न्न - अद॑नम् । \newline
21. स प्र॒त्यक्ष॑म् प्र॒त्यक्षꣳ॒॒ स स प्र॒त्यक्ष॑म् दे॒वेभ्यो॑ दे॒वेभ्यः॑ प्र॒त्यक्षꣳ॒॒ स स प्र॒त्यक्ष॑म् दे॒वेभ्यः॑ । \newline
22. प्र॒त्यक्ष॑म् दे॒वेभ्यो॑ दे॒वेभ्यः॑ प्र॒त्यक्ष॑म् प्र॒त्यक्ष॑म् दे॒वेभ्यो॑ भा॒गम् भा॒गम् दे॒वेभ्यः॑ प्र॒त्यक्ष॑म् प्र॒त्यक्ष॑म् दे॒वेभ्यो॑ भा॒गम् । \newline
23. प्र॒त्यक्ष॒मिति॑ प्रति - अक्ष᳚म् । \newline
24. दे॒वेभ्यो॑ भा॒गम् भा॒गम् दे॒वेभ्यो॑ दे॒वेभ्यो॑ भा॒ग म॑वद दवदद् भा॒गम् दे॒वेभ्यो॑ दे॒वेभ्यो॑ भा॒ग म॑वदत् । \newline
25. भा॒ग म॑वद दवदद् भा॒गम् भा॒ग म॑वदत् प॒रोक्ष॑म् प॒रोक्ष॑ मवदद् भा॒गम् भा॒ग म॑वदत् प॒रोक्ष᳚म् । \newline
26. अ॒व॒द॒त् प॒रोक्ष॑म् प॒रोक्ष॑ मवद दवदत् प॒रोक्ष॒ मसु॑रे॒भ्यो ऽसु॑रेभ्यः प॒रोक्ष॑ मवद दवदत् प॒रोक्ष॒ मसु॑रेभ्यः । \newline
27. प॒रोक्ष॒ मसु॑रे॒भ्यो ऽसु॑रेभ्यः प॒रोक्ष॑म् प॒रोक्ष॒ मसु॑रेभ्यः॒ सर्व॑स्मै॒ सर्व॑स्मा॒ असु॑रेभ्यः प॒रोक्ष॑म् प॒रोक्ष॒ मसु॑रेभ्यः॒ सर्व॑स्मै । \newline
28. प॒रोक्ष॒मिति॑ परः - अक्ष᳚म् । \newline
29. असु॑रेभ्यः॒ सर्व॑स्मै॒ सर्व॑स्मा॒ असु॑रे॒भ्यो ऽसु॑रेभ्यः॒ सर्व॑स्मै॒ वै वै सर्व॑स्मा॒ असु॑रे॒भ्यो ऽसु॑रेभ्यः॒ सर्व॑स्मै॒ वै । \newline
30. सर्व॑स्मै॒ वै वै सर्व॑स्मै॒ सर्व॑स्मै॒ वै प्र॒त्यक्ष॑म् प्र॒त्यक्षं॒ ॅवै सर्व॑स्मै॒ सर्व॑स्मै॒ वै प्र॒त्यक्ष᳚म् । \newline
31. वै प्र॒त्यक्ष॑म् प्र॒त्यक्षं॒ ॅवै वै प्र॒त्यक्ष॑म् भा॒गम् भा॒गम् प्र॒त्यक्षं॒ ॅवै वै प्र॒त्यक्ष॑म् भा॒गम् । \newline
32. प्र॒त्यक्ष॑म् भा॒गम् भा॒गम् प्र॒त्यक्ष॑म् प्र॒त्यक्ष॑म् भा॒गं ॅव॑दन्ति वदन्ति भा॒गम् प्र॒त्यक्ष॑म् प्र॒त्यक्ष॑म् भा॒गं ॅव॑दन्ति । \newline
33. प्र॒त्यक्ष॒मिति॑ प्रति - अक्ष᳚म् । \newline
34. भा॒गं ॅव॑दन्ति वदन्ति भा॒गम् भा॒गं ॅव॑दन्ति॒ यस्मै॒ यस्मै॑ वदन्ति भा॒गम् भा॒गं ॅव॑दन्ति॒ यस्मै᳚ । \newline
35. व॒द॒न्ति॒ यस्मै॒ यस्मै॑ वदन्ति वदन्ति॒ यस्मा॑ ए॒वैव यस्मै॑ वदन्ति वदन्ति॒ यस्मा॑ ए॒व । \newline
36. यस्मा॑ ए॒वैव यस्मै॒ यस्मा॑ ए॒व प॒रोक्ष॑म् प॒रोक्ष॑ मे॒व यस्मै॒ यस्मा॑ ए॒व प॒रोक्ष᳚म् । \newline
37. ए॒व प॒रोक्ष॑म् प॒रोक्ष॑ मे॒वैव प॒रोक्षं॒ ॅवद॑न्ति॒ वद॑न्ति प॒रोक्ष॑ मे॒वैव प॒रोक्षं॒ ॅवद॑न्ति । \newline
38. प॒रोक्षं॒ ॅवद॑न्ति॒ वद॑न्ति प॒रोक्ष॑म् प॒रोक्षं॒ ॅवद॑न्ति॒ तस्य॒ तस्य॒ वद॑न्ति प॒रोक्ष॑म् प॒रोक्षं॒ ॅवद॑न्ति॒ तस्य॑ । \newline
39. प॒रोक्ष॒मिति॑ परः - अक्ष᳚म् । \newline
40. वद॑न्ति॒ तस्य॒ तस्य॒ वद॑न्ति॒ वद॑न्ति॒ तस्य॑ भा॒गो भा॒ग स्तस्य॒ वद॑न्ति॒ वद॑न्ति॒ तस्य॑ भा॒गः । \newline
41. तस्य॑ भा॒गो भा॒ग स्तस्य॒ तस्य॑ भा॒ग उ॑दि॒त उ॑दि॒तो भा॒ग स्तस्य॒ तस्य॑ भा॒ग उ॑दि॒तः । \newline
42. भा॒ग उ॑दि॒त उ॑दि॒तो भा॒गो भा॒ग उ॑दि॒त स्तस्मा॒त् तस्मा॑ दुदि॒तो भा॒गो भा॒ग उ॑दि॒त स्तस्मा᳚त् । \newline
43. उ॒दि॒त स्तस्मा॒त् तस्मा॑ दुदि॒त उ॑दि॒त स्तस्मा॒ दिन्द्र॒ इन्द्र॒ स्तस्मा॑ दुदि॒त उ॑दि॒त स्तस्मा॒ दिन्द्रः॑ । \newline
44. तस्मा॒ दिन्द्र॒ इन्द्र॒ स्तस्मा॒त् तस्मा॒ दिन्द्रो॑ ऽबिभे दबिभे॒ दिन्द्र॒ स्तस्मा॒त् तस्मा॒ दिन्द्रो॑ ऽबिभेत् । \newline
45. इन्द्रो॑ ऽबिभे दबिभे॒ दिन्द्र॒ इन्द्रो॑ ऽबिभे दी॒दृङ् ङी॒दृङ् ङ॑बिभे॒दिन्द्र॒ इन्द्रो॑ ऽबिभेदी॒दृङ् । \newline
46. अ॒बि॒भे॒ दी॒दृङ् ङी॒दृङ् ङ॑बिभे दबिभे दी॒दृङ् वै वा ई॒दृङ् ङ॑बिभे दबिभे दी॒दृङ् वै । \newline
47. ई॒दृङ् वै वा ई॒दृङ् ङी॒दृङ् वै रा॒ष्ट्रꣳ रा॒ष्ट्रं ॅवा ई॒दृङ् ङी॒दृङ् वै रा॒ष्ट्रम् । \newline
48. वै रा॒ष्ट्रꣳ रा॒ष्ट्रं ॅवै वै रा॒ष्ट्रं ॅवि वि रा॒ष्ट्रं ॅवै वै रा॒ष्ट्रं ॅवि । \newline
49. रा॒ष्ट्रं ॅवि वि रा॒ष्ट्रꣳ रा॒ष्ट्रं ॅवि प॒र्याव॑र्तयति प॒र्याव॑र्तयति॒ वि रा॒ष्ट्रꣳ रा॒ष्ट्रं ॅवि प॒र्याव॑र्तयति । \newline
50. वि प॒र्याव॑र्तयति प॒र्याव॑र्तयति॒ वि वि प॒र्याव॑र्तय॒ तीतीति॑ प॒र्याव॑र्तयति॒ वि वि प॒र्याव॑र्तय॒तीति॑ । \newline
51. प॒र्याव॑र्तय॒ तीतीति॑ प॒र्याव॑र्तयति प॒र्याव॑र्तय॒ तीति॒ तस्य॒ तस्ये ति॑ प॒र्याव॑र्तयति प॒र्याव॑र्तय॒ तीति॒ तस्य॑ । \newline
52. प॒र्याव॑र्तय॒तीति॑ परि - आव॑र्तयति । \newline
53. इति॒ तस्य॒ तस्ये तीति॒ तस्य॒ वज्रं॒ ॅवज्र॒म् तस्ये तीति॒ तस्य॒ वज्र᳚म् । \newline
54. तस्य॒ वज्रं॒ ॅवज्र॒म् तस्य॒ तस्य॒ वज्र॑ मा॒दाया॒ दाय॒ वज्र॒म् तस्य॒ तस्य॒ वज्र॑ मा॒दाय॑ । \newline
55. वज्र॑ मा॒दाया॒ दाय॒ वज्रं॒ ॅवज्र॑ मा॒दाय॑ शी॒र्॒.षाणि॑ शी॒र्॒.षा ण्या॒दाय॒ वज्रं॒ ॅवज्र॑ मा॒दाय॑ शी॒र्॒.षाणि॑ । \newline
56. आ॒दाय॑ शी॒र्॒.षाणि॑ शी॒र्॒.षा ण्या॒दाया॒ दाय॑ शी॒र्॒.षा ण्य॑च्छिन दच्छिन च्छी॒र्॒.षा ण्या॒दाया॒ दाय॑ शी॒र्॒.षाण्य॑च्छिनत् । \newline
57. आ॒दायेत्या᳚ - दाय॑ । \newline
58. शी॒र्॒.षा ण्य॑च्छिन दच्छिन च्छी॒र्॒.षाणि॑ शी॒र्॒.षा ण्य॑च्छिन॒द् यद् यद॑च्छिन च्छी॒र्॒.षाणि॑ शी॒र्॒.षा ण्य॑च्छिन॒द् यत् । \newline
59. अ॒च्छि॒न॒द् यद् यद॑च्छिन दच्छिन॒द् यथ् सो॑म॒पानꣳ॑ सोम॒पानं॒ ॅयद॑च्छिन दच्छिन॒द् यथ् सो॑म॒पान᳚म् । \newline
60. यथ् सो॑म॒पानꣳ॑ सोम॒पानं॒ ॅयद् यथ् सो॑म॒पान॒ मासी॒ दासी᳚थ् सोम॒पानं॒ ॅयद् यथ् सो॑म॒पान॒ मासी᳚त् । \newline
61. सो॒म॒पान॒ मासी॒ दासी᳚थ् सोम॒पानꣳ॑ सोम॒पान॒ मासी॒थ् स स आसी᳚थ् सोम॒पानꣳ॑ सोम॒पान॒ मासी॒थ् सः । \newline
62. सो॒म॒पान॒मिति॑ सोम - पान᳚म् । \newline
\pagebreak
\markright{ TS 2.5.1.2  \hfill https://www.vedavms.in \hfill}
\addcontentsline{toc}{section}{ TS 2.5.1.2 }
\section*{ TS 2.5.1.2 }

\textbf{TS 2.5.1.2 } \newline
\textbf{Samhita Paata} \newline

-मासी॒थ् स क॒पिञ्ज॑लो ऽभव॒द् यथ् सु॑रा॒पानꣳ॒॒ स क॑ल॒विङ्को॒ यद॒न्नाद॑नꣳ॒॒ स ति॑त्ति॒रिस्तस्या᳚ञ्ज॒लिना᳚ ब्रह्मह॒त्यामुपा॑गृह्णा॒त् ताꣳ सं॑ॅवथ्स॒रम॑बिभ॒स्तं भू॒तान्य॒भ्य॑क्रोश॒न् ब्रह्म॑ह॒न्निति॒ स पृ॑थि॒वीमुपा॑सीदद॒स्यै ब्र॑ह्मह॒त्यायै॒ तृती॑यं॒ प्रति॑ गृहा॒णेति॒ साऽब्र॑वी॒द्वरं॑ ॅवृणै खा॒तात् प॑राभवि॒ष्यन्ती॑ मन्ये॒ ततो॒ मा परा॑ भूव॒मिति॑पु॒रा ते॑-  [  ] \newline

\textbf{Pada Paata} \newline

आसी᳚त् । सः । क॒पिञ्ज॑लः । अ॒भ॒व॒त् । यत् । सु॒रा॒पान॒मिति॑ सुरा - पान᳚म् । सः । क॒ल॒विङ्कः॑ । यत् । अ॒न्नाद॑न॒मित्य॑न्न-अद॑नम् । सः । ति॒त्ति॒रिः । तस्य॑ । अ॒ञ्ज॒लिना᳚ । ब्र॒ह्म॒ह॒त्यामिति॑ ब्रह्म-ह॒त्याम् । उपेति॑ । अ॒गृ॒ह्णा॒त् । ताम् । सं॒ॅव॒थ्स॒रमिति॑ सं - व॒थ्स॒रम् । अ॒बि॒भः॒ । तम् । भू॒तानि॑ । अ॒भीति॑ । अ॒क्रो॒श॒न्न् । ब्रह्म॑ह॒न्निति॒ ब्रह्म॑ - ह॒न्न् । इति॑ । सः । पृ॒थि॒वीम् । उपेति॑ ।  अ॒सी॒द॒त् । अ॒स्यै । ब्र॒ह्म॒ह॒त्याया॒ इति॑ ब्रह्म - ह॒त्यायै᳚ । तृती॑यम् । प्रतीति॑ । गृ॒हा॒ण॒ । इति॑ । सा । अ॒ब्र॒वी॒त् । वर᳚म् । वृ॒णै॒ । खा॒तात् । प॒रा॒भ॒वि॒ष्यन्तीति॑ परा - भ॒वि॒ष्यन्ती᳚ । म॒न्ये॒ । ततः॑ । मा । परेति॑ । भू॒व॒म् । इति॑ । पु॒रा । ते॒ ।  \newline


\textbf{Krama Paata} \newline

आसी॒थ् सः । स क॒पिञ्ज॑लः । क॒पिञ्ज॑लो ऽभवत् । अ॒भ॒व॒द् यत् । यथ् सु॑रा॒पान᳚म् । सु॒रा॒पानꣳ॒॒ सः । सु॒रा॒पान॒मिति॑ सुरा - पान᳚म् । स क॑ल॒विङ्कः॑ । क॒ल॒विङ्को॒ यत् । यद॒न्नाद॑नम् । अ॒न्नाद॑नꣳ॒॒ सः । अ॒न्नाद॑न॒मित्य॑न्न - अद॑नम् । स ति॑त्ति॒रिः । ति॒त्ति॒रिस्तस्य॑ । तस्या᳚ऽञ्ज॒लिना᳚ । अ॒ञ्ज॒लिना᳚ ब्रह्मह॒त्याम् । ब्र॒ह्म॒ह॒त्यामुप॑ । ब्र॒ह्म॒ह॒त्यामिति॑ ब्रह्म - ह॒त्याम् । उपा॑गृह्णात् । अ॒गृ॒ह्णा॒त् ताम् । ताꣳ स॑म्ॅवथ्स॒रम् । स॒म्ॅव॒थ्स॒रम॑बिभः । स॒म्ॅव॒थ्स॒रमिति॑ सम् - व॒थ्स॒रम् । अ॒बि॒भ॒स्तम् । तम् भू॒तानि॑ । भू॒तान्य॒भि । अ॒भ्य॑क्रोशन्न् । अ॒क्रो॒श॒न् ब्रह्म॑हन्न् । ब्रह्म॑ह॒न्निति॑ । ब्रह्म॑ह॒न्निति॒ ब्रह्म॑ - ह॒न्न्॒ । इति॒ सः । स पृ॑थि॒वीम् । पृ॒थि॒वीमुप॑ । उपा॑सीदत् । अ॒सी॒द॒द॒स्यै । अ॒स्यै ब्र॑ह्मह॒त्यायै᳚ । ब्र॒ह्म॒ह॒त्यायै॒ तृती॑यम् । ब्र॒ह्म॒ह॒त्याया॒ इति॑ ब्रह्म - ह॒त्यायै᳚ । तृती॑य॒म् प्रति॑ । प्रति॑ गृहाण । गृ॒हा॒णेति॑ । इति॒ सा । सा ऽब्र॑वीत् । अ॒ब्र॒वी॒द् वर᳚म् । वर॑म् ॅवृणै । वृ॒णै॒ खा॒तात् । खा॒तात् प॑राभवि॒ष्यन्ती᳚ । प॒रा॒भ॒वि॒ष्यन्ती॑ मन्ये । प॒रा॒भ॒वि॒ष्यन्तीति॑ परा - भ॒वि॒ष्यन्ती᳚ । म॒न्ये॒ ततः॑ । ततो॒ मा । मा परा᳚ । परा॑ भूवम् । भू॒व॒मिति॑ । इति॑ पु॒रा । पु॒रा ते᳚ । ते॒ स॒म्ॅव॒थ्स॒रात् \newline

\textbf{Jatai Paata} \newline

1. आसी॒थ् स स आसी॒ दासी॒थ् सः । \newline
2. स क॒पिञ्ज॑लः क॒पिञ्ज॑लः॒ स स क॒पिञ्ज॑लः । \newline
3. क॒पिञ्ज॑लो ऽभवदभवत् क॒पिञ्ज॑लः क॒पिञ्ज॑लो ऽभवत् । \newline
4. अ॒भ॒व॒द् यद् यद॑भव दभव॒द् यत् । \newline
5. यथ् सु॑रा॒पानꣳ॑ सुरा॒पानं॒ ॅयद् यथ् सु॑रा॒पान᳚म् । \newline
6. सु॒रा॒पानꣳ॒॒ स स सु॑रा॒पानꣳ॑ सुरा॒पानꣳ॒॒ सः । \newline
7. सु॒रा॒पान॒मिति॑ सुरा - पान᳚म् । \newline
8. स क॑ल॒विङ्कः॑ कल॒विङ्कः॒ स स क॑ल॒विङ्कः॑ । \newline
9. क॒ल॒विङ्को॒ यद् यत् क॑ल॒विङ्कः॑ कल॒विङ्को॒ यत् । \newline
10. यद॒न्नाद॑न म॒न्नाद॑नं॒ ॅयद् यद॒न्नाद॑नम् । \newline
11. अ॒न्नाद॑नꣳ॒॒ स सो᳚ ऽन्नाद॑न म॒न्नाद॑नꣳ॒॒ सः । \newline
12. अ॒न्नाद॑न॒मित्य॑न्न - अद॑नम् । \newline
13. स ति॑त्ति॒रि स्ति॑त्ति॒रिः स स ति॑त्ति॒रिः । \newline
14. ति॒त्ति॒रि स्तस्य॒ तस्य॑ तित्ति॒रि स्ति॑त्ति॒रि स्तस्य॑ । \newline
15. तस्या᳚ ञ्ज॒लिना᳚ ऽञ्ज॒लिना॒ तस्य॒ तस्या᳚ ञ्ज॒लिना᳚ । \newline
16. अ॒ञ्ज॒लिना᳚ ब्रह्मह॒त्याम् ब्र॑ह्मह॒त्या म॑ञ्ज॒लिना᳚ ऽञ्ज॒लिना᳚ ब्रह्मह॒त्याम् । \newline
17. ब्र॒ह्म॒ह॒त्या मुपोप॑ ब्रह्मह॒त्याम् ब्र॑ह्मह॒त्या मुप॑ । \newline
18. ब्र॒ह्म॒ह॒त्यामिति॑ ब्रह्म - ह॒त्याम् । \newline
19. उपा॑गृह्णा दगृह्णा॒ दुपोपा॑ गृह्णात् । \newline
20. अ॒गृ॒ह्णा॒त् ताम् ता म॑गृह्णा दगृह्णा॒त् ताम् । \newline
21. ताꣳ सं॑ॅवथ्स॒रꣳ सं॑ॅवथ्स॒रम् ताम् ताꣳ सं॑ॅवथ्स॒रम् । \newline
22. सं॒ॅव॒थ्स॒र म॑बिभ रबिभः संॅवथ्स॒रꣳ सं॑ॅवथ्स॒र म॑बिभः । \newline
23. सं॒ॅव॒थ्स॒रमिति॑ सं - व॒थ्स॒रम् । \newline
24. अ॒बि॒भ॒ स्तम् त म॑बिभ रबिभ॒ स्तम् । \newline
25. तम् भू॒तानि॑ भू॒तानि॒ तम् तम् भू॒तानि॑ । \newline
26. भू॒ता न्य॒भ्य॑भि भू॒तानि॑ भू॒ता न्य॒भि । \newline
27. अ॒भ्य॑क्रोशन् नक्रोशन् न॒भ्या᳚(1॒)भ्य॑क्रोशन्न् । \newline
28. अ॒क्रो॒श॒न् ब्रह्म॑ह॒न् ब्रह्म॑हन् नक्रोशन् नक्रोश॒न् ब्रह्म॑हन्न् । \newline
29. ब्रह्म॑ह॒न् नितीति॒ ब्रह्म॑ह॒न् ब्रह्म॑ह॒न् निति॑ । \newline
30. ब्रह्म॑ह॒न्निति॒ ब्रह्म॑ - ह॒न्न् । \newline
31. इति॒ स स इतीति॒ सः । \newline
32. स पृ॑थि॒वीम् पृ॑थि॒वीꣳ स स पृ॑थि॒वीम् । \newline
33. पृ॒थि॒वी मुपोप॑ पृथि॒वीम् पृ॑थि॒वी मुप॑ । \newline
34. उपा॑सी ददसीद॒ दुपोपा॑ सीदत् । \newline
35. अ॒सी॒ द॒द॒स्या अ॒स्या अ॑सीद दसीद द॒स्यै । \newline
36. अ॒स्यै ब्र॑ह्मह॒त्यायै᳚ ब्रह्मह॒त्याया॑ अ॒स्या अ॒स्यै ब्र॑ह्मह॒त्यायै᳚ । \newline
37. ब्र॒ह्म॒ह॒त्यायै॒ तृती॑य॒म् तृती॑यम् ब्रह्मह॒त्यायै᳚ ब्रह्मह॒त्यायै॒ तृती॑यम् । \newline
38. ब्र॒ह्म॒ह॒त्याया॒ इति॑ ब्रह्म - ह॒त्यायै᳚ । \newline
39. तृती॑य॒म् प्रति॒ प्रति॒ तृती॑य॒म् तृती॑य॒म् प्रति॑ । \newline
40. प्रति॑ गृहाण गृहाण॒ प्रति॒ प्रति॑ गृहाण । \newline
41. गृ॒हा॒णे तीति॑ गृहाण गृहा॒णे ति॑ । \newline
42. इति॒ सा सेतीति॒ सा । \newline
43. सा ऽब्र॑वी दब्रवी॒थ् सा सा ऽब्र॑वीत् । \newline
44. अ॒ब्र॒वी॒द् वरं॒ ॅवर॑ मब्रवी दब्रवी॒द् वर᳚म् । \newline
45. वरं॑ ॅवृणै वृणै॒ वरं॒ ॅवरं॑ ॅवृणै । \newline
46. वृ॒णै॒ खा॒तात् खा॒ताद् वृ॑णै वृणै खा॒तात् । \newline
47. खा॒तात् प॑राभवि॒ष्यन्ती॑ पराभवि॒ष्यन्ती॑ खा॒तात् खा॒तात् प॑राभवि॒ष्यन्ती᳚ । \newline
48. प॒रा॒भ॒वि॒ष्यन्ती॑ मन्ये मन्ये पराभवि॒ष्यन्ती॑ पराभवि॒ष्यन्ती॑ मन्ये । \newline
49. प॒रा॒भ॒वि॒ष्यन्तीति॑ परा - भ॒वि॒ष्यन्ती᳚ । \newline
50. म॒न्ये॒ तत॒ स्ततो॑ मन्ये मन्ये॒ ततः॑ । \newline
51. ततो॒ मा मा तत॒ स्ततो॒ मा । \newline
52. मा परा॒ परा॒ मा मा परा᳚ । \newline
53. परा॑ भूवम् भूव॒म् परा॒ परा॑ भूवम् । \newline
54. भू॒व॒ मितीति॑ भूवम् भूव॒ मिति॑ । \newline
55. इति॑ पु॒रा पु॒रेतीति॑ पु॒रा । \newline
56. पु॒रा ते॑ ते पु॒रा पु॒रा ते᳚ । \newline
57. ते॒ सं॒ॅव॒थ्स॒राथ् सं॑ॅवथ्स॒रात् ते॑ ते संॅवथ्स॒रात् । \newline

\textbf{Ghana Paata } \newline

1. आसी॒थ् स स आसी॒ दासी॒थ् स क॒पिञ्ज॑लः क॒पिञ्ज॑लः॒ स आसी॒ दासी॒थ् स क॒पिञ्ज॑लः । \newline
2. स क॒पिञ्ज॑लः क॒पिञ्ज॑लः॒ स स क॒पिञ्ज॑लो ऽभव दभवत् क॒पिञ्ज॑लः॒ स स क॒पिञ्ज॑लो ऽभवत् । \newline
3. क॒पिञ्ज॑लो ऽभव दभवत् क॒पिञ्ज॑लः क॒पिञ्ज॑लो ऽभव॒द् यद् यद॑भवत् क॒पिञ्ज॑लः क॒पिञ्ज॑लो ऽभव॒द् यत् । \newline
4. अ॒भ॒व॒द् यद् यद॑भव दभव॒द् यथ् सु॑रा॒पानꣳ॑ सुरा॒पानं॒ ॅयद॑भव दभव॒द् यथ् सु॑रा॒पान᳚म् । \newline
5. यथ् सु॑रा॒पानꣳ॑ सुरा॒पानं॒ ॅयद् यथ् सु॑रा॒पानꣳ॒॒ स स सु॑रा॒पानं॒ ॅयद् यथ् सु॑रा॒पानꣳ॒॒ सः । \newline
6. सु॒रा॒पानꣳ॒॒ स स सु॑रा॒पानꣳ॑ सुरा॒पानꣳ॒॒ स क॑ल॒विङ्कः॑ कल॒विङ्कः॒ स सु॑रा॒पानꣳ॑ सुरा॒पानꣳ॒॒ स क॑ल॒विङ्कः॑ । \newline
7. सु॒रा॒पान॒मिति॑ सुरा - पान᳚म् । \newline
8. स क॑ल॒विङ्कः॑ कल॒विङ्कः॒ स स क॑ल॒विङ्को॒ यद् यत् क॑ल॒विङ्कः॒ स स क॑ल॒विङ्को॒ यत् । \newline
9. क॒ल॒विङ्को॒ यद् यत् क॑ल॒विङ्कः॑ कल॒विङ्को॒ यद॒न्नाद॑न म॒न्नाद॑नं॒ ॅयत् क॑ल॒विङ्कः॑ कल॒विङ्को॒ यद॒न्नाद॑नम् । \newline
10. यद॒न्नाद॑न म॒न्नाद॑नं॒ ॅयद् यद॒न्नाद॑नꣳ॒॒ स सो᳚ ऽन्नाद॑नं॒ ॅयद् यद॒न्नाद॑नꣳ॒॒ सः । \newline
11. अ॒न्नाद॑नꣳ॒॒ स सो᳚ ऽन्नाद॑न म॒न्नाद॑नꣳ॒॒ स ति॑त्ति॒रि स्ति॑त्ति॒रिः सो᳚ ऽन्नाद॑न म॒न्नाद॑नꣳ॒॒ स ति॑त्ति॒रिः । \newline
12. अ॒न्नाद॑न॒मित्य॑न्न - अद॑नम् । \newline
13. स ति॑त्ति॒रि स्ति॑त्ति॒रिः स स ति॑त्ति॒रि स्तस्य॒ तस्य॑ तित्ति॒रिः स स ति॑त्ति॒रि स्तस्य॑ । \newline
14. ति॒त्ति॒रि स्तस्य॒ तस्य॑ तित्ति॒रि स्ति॑त्ति॒रि स्तस्या᳚ ञ्ज॒लिना᳚ ऽञ्ज॒लिना॒ तस्य॑ तित्ति॒रि स्ति॑त्ति॒रि स्तस्या᳚ ञ्ज॒लिना᳚ । \newline
15. तस्या᳚ञ्ज॒लिना᳚ ऽञ्ज॒लिना॒ तस्य॒ तस्या᳚ञ्ज॒लिना᳚ ब्रह्मह॒त्याम् ब्र॑ह्मह॒त्या म॑ञ्ज॒लिना॒ तस्य॒ तस्या᳚ ञ्ज॒लिना᳚ ब्रह्मह॒त्याम् । \newline
16. अ॒ञ्ज॒लिना᳚ ब्रह्मह॒त्याम् ब्र॑ह्मह॒त्या म॑ञ्ज॒लिना᳚ ऽञ्ज॒लिना᳚ ब्रह्मह॒त्या मुपोप॑ ब्रह्मह॒त्या म॑ञ्ज॒लिना᳚ ऽञ्ज॒लिना᳚ ब्रह्मह॒त्या मुप॑ । \newline
17. ब्र॒ह्म॒ह॒त्या मुपोप॑ ब्रह्मह॒त्याम् ब्र॑ह्मह॒त्या मुपा॑गृह्णा दगृह्णा॒दुप॑ ब्रह्मह॒त्याम् ब्र॑ह्मह॒त्या मुपा॑गृह्णात् । \newline
18. ब्र॒ह्म॒ह॒त्यामिति॑ ब्रह्म - ह॒त्याम् । \newline
19. उपा॑गृह्णा दगृह्णा॒ दुपोपा॑गृह्णा॒त् ताम् ता म॑गृह्णा॒ दुपोपा॑गृह्णा॒त् ताम् । \newline
20. अ॒गृ॒ह्णा॒त् ताम् ता म॑गृह्णा दगृह्णा॒त् ताꣳ सं॑ॅवथ्स॒रꣳ सं॑ॅवथ्स॒रम् ता म॑गृह्णा दगृह्णा॒त् ताꣳ सं॑ॅवथ्स॒रम् । \newline
21. ताꣳ सं॑ॅवथ्स॒रꣳ सं॑ॅवथ्स॒रम् ताम् ताꣳ सं॑ॅवथ्स॒र म॑बिभ रबिभः संॅवथ्स॒रम् ताम् ताꣳ सं॑ॅवथ्स॒र म॑बिभः । \newline
22. सं॒ॅव॒थ्स॒र म॑बिभ रबिभः संॅवथ्स॒रꣳ सं॑ॅवथ्स॒र म॑बिभ॒ स्तम् त म॑बिभः संॅवथ्स॒रꣳ सं॑ॅवथ्स॒र म॑बिभ॒ स्तम् । \newline
23. सं॒ॅव॒थ्स॒रमिति॑ सं - व॒थ्स॒रम् । \newline
24. अ॒बि॒भ॒ स्तम् त म॑बिभ रबिभ॒ स्तम् भू॒तानि॑ भू॒तानि॒ त म॑बिभ रबिभ॒ स्तम् भू॒तानि॑ । \newline
25. तम् भू॒तानि॑ भू॒तानि॒ तम् तम् भू॒तान्य॒भ्य॑भि भू॒तानि॒ तम् तम् भू॒तान्य॒भि । \newline
26. भू॒तान्य॒भ्य॑भि भू॒तानि॑ भू॒तान्य॒भ्य॑क्रोशन् नक्रोशन् न॒भि भू॒तानि॑ भू॒तान्य॒भ्य॑क्रोशन्न् । \newline
27. अ॒भ्य॑क्रोशन् नक्रोशन् न॒भ्या᳚(1॒)भ्य॑क्रोश॒न् ब्रह्म॑ह॒न् ब्रह्म॑हन् नक्रोशन् न॒भ्या᳚(1॒)भ्य॑क्रोश॒न् ब्रह्म॑हन्न् । \newline
28. अ॒क्रो॒श॒न् ब्रह्म॑ह॒न् ब्रह्म॑हन् नक्रोशन् नक्रोश॒न् ब्रह्म॑ह॒न् नितीति॒ ब्रह्म॑हन् नक्रोशन् नक्रोश॒न् ब्रह्म॑ह॒न् निति॑ । \newline
29. ब्रह्म॑ह॒न् नितीति॒ ब्रह्म॑ह॒न् ब्रह्म॑ह॒न् निति॒ स स इति॒ ब्रह्म॑ह॒न् ब्रह्म॑ह॒न् निति॒ सः । \newline
30. ब्रह्म॑ह॒न्निति॒ ब्रह्म॑ - ह॒न्न् । \newline
31. इति॒ स स इतीति॒ स पृ॑थि॒वीम् पृ॑थि॒वीꣳ स इतीति॒ स पृ॑थि॒वीम् । \newline
32. स पृ॑थि॒वीम् पृ॑थि॒वीꣳ स स पृ॑थि॒वी मुपोप॑ पृथि॒वीꣳ स स पृ॑थि॒वी मुप॑ । \newline
33. पृ॒थि॒वी मुपोप॑ पृथि॒वीम् पृ॑थि॒वी मुपा॑सी ददसीद॒ दुप॑ पृथि॒वीम् पृ॑थि॒वी मुपा॑सीदत् । \newline
34. उपा॑सी ददसीद॒ दुपोपा॑सी दद॒स्या अ॒स्या अ॑सीद॒ दुपोपा॑सी दद॒स्यै । \newline
35. अ॒सी॒द॒द॒स्या अ॒स्या अ॑सीद दसीद द॒स्यै ब्र॑ह्मह॒त्यायै᳚ ब्रह्मह॒त्याया॑ अ॒स्या अ॑सीद दसीद द॒स्यै ब्र॑ह्मह॒त्यायै᳚ । \newline
36. अ॒स्यै ब्र॑ह्मह॒त्यायै᳚ ब्रह्मह॒त्याया॑ अ॒स्या अ॒स्यै ब्र॑ह्मह॒त्यायै॒ तृती॑य॒म् तृती॑यम् ब्रह्मह॒त्याया॑ अ॒स्या अ॒स्यै ब्र॑ह्मह॒त्यायै॒ तृती॑यम् । \newline
37. ब्र॒ह्म॒ह॒त्यायै॒ तृती॑य॒म् तृती॑यम् ब्रह्मह॒त्यायै᳚ ब्रह्मह॒त्यायै॒ तृती॑य॒म् प्रति॒ प्रति॒ तृती॑यम् ब्रह्मह॒त्यायै᳚ ब्रह्मह॒त्यायै॒ तृती॑य॒म् प्रति॑ । \newline
38. ब्र॒ह्म॒ह॒त्याया॒ इति॑ ब्रह्म - ह॒त्यायै᳚ । \newline
39. तृती॑य॒म् प्रति॒ प्रति॒ तृती॑य॒म् तृती॑य॒म् प्रति॑ गृहाण गृहाण॒ प्रति॒ तृती॑य॒म् तृती॑य॒म् प्रति॑ गृहाण । \newline
40. प्रति॑ गृहाण गृहाण॒ प्रति॒ प्रति॑ गृहा॒णे तीति॑ गृहाण॒ प्रति॒ प्रति॑ गृहा॒णे ति॑ । \newline
41. गृ॒हा॒णे तीति॑ गृहाण गृहा॒णे ति॒ सा सेति॑ गृहाण गृहा॒णे ति॒ सा । \newline
42. इति॒ सा सेतीति॒ सा ऽब्र॑वी दब्रवी॒थ् सेतीति॒ सा ऽब्र॑वीत् । \newline
43. सा ऽब्र॑वी दब्रवी॒थ् सा सा ऽब्र॑वी॒द् वरं॒ ॅवर॑ मब्रवी॒थ् सा सा ऽब्र॑वी॒द् वर᳚म् । \newline
44. अ॒ब्र॒वी॒द् वरं॒ ॅवर॑ मब्रवी दब्रवी॒द् वरं॑ ॅवृणै वृणै॒ वर॑ मब्रवी दब्रवी॒द् वरं॑ ॅवृणै । \newline
45. वरं॑ ॅवृणै वृणै॒ वरं॒ ॅवरं॑ ॅवृणै खा॒तात् खा॒ताद् वृ॑णै॒ वरं॒ ॅवरं॑ ॅवृणै खा॒तात् । \newline
46. वृ॒णै॒ खा॒तात् खा॒ताद् वृ॑णै वृणै खा॒तात् प॑राभवि॒ष्यन्ती॑ पराभवि॒ष्यन्ती॑ खा॒ताद् वृ॑णै वृणै खा॒तात् प॑राभवि॒ष्यन्ती᳚ । \newline
47. खा॒तात् प॑राभवि॒ष्यन्ती॑ पराभवि॒ष्यन्ती॑ खा॒तात् खा॒तात् प॑राभवि॒ष्यन्ती॑ मन्ये मन्ये पराभवि॒ष्यन्ती॑ खा॒तात् खा॒तात् प॑राभवि॒ष्यन्ती॑ मन्ये । \newline
48. प॒रा॒भ॒वि॒ष्यन्ती॑ मन्ये मन्ये पराभवि॒ष्यन्ती॑ पराभवि॒ष्यन्ती॑ मन्ये॒ तत॒ स्ततो॑ मन्ये पराभवि॒ष्यन्ती॑ पराभवि॒ष्यन्ती॑ मन्ये॒ ततः॑ । \newline
49. प॒रा॒भ॒वि॒ष्यन्तीति॑ परा - भ॒वि॒ष्यन्ती᳚ । \newline
50. म॒न्ये॒ तत॒ स्ततो॑ मन्ये मन्ये॒ ततो॒ मा मा ततो॑ मन्ये मन्ये॒ ततो॒ मा । \newline
51. ततो॒ मा मा तत॒ स्ततो॒ मा परा॒ परा॒ मा तत॒ स्ततो॒ मा परा᳚ । \newline
52. मा परा॒ परा॒ मा मा परा॑ भूवम् भूव॒म् परा॒ मा मा परा॑ भूवम् । \newline
53. परा॑ भूवम् भूव॒म् परा॒ परा॑ भूव॒ मितीति॑ भूव॒म् परा॒ परा॑ भूव॒ मिति॑ । \newline
54. भू॒व॒ मितीति॑ भूवम् भूव॒ मिति॑ पु॒रा पु॒रेति॑ भूवम् भूव॒ मिति॑ पु॒रा । \newline
55. इति॑ पु॒रा पु॒रेतीति॑ पु॒रा ते॑ ते पु॒रेतीति॑ पु॒रा ते᳚ । \newline
56. पु॒रा ते॑ ते पु॒रा पु॒रा ते॑ संॅवथ्स॒राथ् सं॑ॅवथ्स॒रात् ते॑ पु॒रा पु॒रा ते॑ संॅवथ्स॒रात् । \newline
57. ते॒ सं॒ॅव॒थ्स॒राथ् सं॑ॅवथ्स॒रात् ते॑ ते संॅवथ्स॒रा दप्यपि॑ संॅवथ्स॒रात् ते॑ ते संॅवथ्स॒रा दपि॑ । \newline
\pagebreak
\markright{ TS 2.5.1.3  \hfill https://www.vedavms.in \hfill}
\addcontentsline{toc}{section}{ TS 2.5.1.3 }
\section*{ TS 2.5.1.3 }

\textbf{TS 2.5.1.3 } \newline
\textbf{Samhita Paata} \newline

संॅवथ्स॒रादपि॑ रोहा॒दित्य॑ब्रवी॒त् तस्मा᳚त् पु॒रा सं॑ॅवथ्स॒रात् पृ॑थि॒व्यै खा॒तमपि॑ रोहति॒ वारे॑वृतꣳ॒॒ ह्य॑स्यै॒ तृती॑यं ब्रह्मह॒त्यायै॒ प्रत्य॑गृह्णा॒त् तथ् स्वकृ॑त॒मिरि॑णमभव॒त् तस्मा॒दाहि॑ताग्निः श्र॒द्धादे॑वः॒ स्वकृ॑त॒ इरि॑णे॒ नाव॑ स्येद्-ब्रह्मह॒त्यायै॒ ह्ये॑ष वर्णः॒ स वन॒स्पती॒नुपा॑सीदद॒स्यै ब्र॑ह्मह॒त्यायै॒ तृती॑यं॒ प्रति॑ गृह्णी॒तेति॒ ते᳚ऽब्रुव॒न् वरं॑ ॅवृणामहै वृ॒क्णात् - [  ] \newline

\textbf{Pada Paata} \newline

सं॒ॅव॒थ्स॒रादिति॑ सं - व॒थ्स॒रात् । अपीति॑ । रो॒हा॒त् । इति॑ । अ॒ब्र॒वी॒त् । तस्मा᳚त् । पु॒रा । सं॒ॅव॒थ्स॒रादिति॑ सं - व॒थ्स॒रात् । पृ॒थि॒व्यै । खा॒तम् । अपीति॑ ।  रो॒ह॒ति॒ । वारे॑वृत॒मिति॒ वारे᳚ - वृ॒त॒म् । हि ।  अ॒स्यै॒ । तृती॑यम् । ब्र॒ह्म॒ह॒त्याया॒ इति॑ ब्रह्म - ह॒त्यायै᳚ । प्रतीति॑ । अ॒गृ॒ह्णा॒त् । तत् । स्वकृ॑त॒मिति॒ स्व - कृ॒त॒म् । इरि॑णम् । अ॒भ॒व॒त् । तस्मा᳚त् । आहि॑ताग्नि॒रित्याहि॑त - अ॒ग्निः॒ । श्र॒द्धादे॑व॒ इति॑ श्र॒द्धा - दे॒वः॒ । स्वकृ॑त॒ इति॒ स्व - कृ॒ते॒ । इरि॑णे । न । अवेति॑ । स्ये॒त् । ब्र॒ह्म॒ह॒त्याया॒ इति॑ ब्रह्म - ह॒त्यायै᳚ । हि । ए॒षः । वर्णः॑ । सः । वन॒स्पतीन्॑ । उपेति॑ । अ॒सी॒द॒त् । अ॒स्यै । ब्र॒ह्म॒ह॒त्याया॒ इति॑ ब्रह्म - ह॒त्यायै᳚ । तृती॑यम् । प्रतीति॑ । गृ॒ह्णी॒त॒ । इति॑ । ते । अ॒ब्रु॒व॒न्न् । वर᳚म् । वृ॒णा॒म॒है॒ । वृ॒क्णात् ।  \newline


\textbf{Krama Paata} \newline

स॒म्ॅव॒थ्स॒रादपि॑ । स॒म्ॅव॒थ्स॒रादिति॑ सम् - व॒थ्स॒रात् । अपि॑ रोहात् । रो॒हा॒दिति॑ । इत्य॑ब्रवीत् । अ॒ब्र॒वी॒त् तस्मा᳚त् । तस्मा᳚त् पु॒रा । पु॒रा स॑म्ॅवथ्स॒रात् । स॒म्ॅव॒थ्स॒रात् पृ॑थि॒व्यै । स॒म्ॅव॒थ्स॒रादिति॑ सम् - व॒थ्स॒रात् । पृ॒थि॒व्यै खा॒तम् । खा॒तमपि॑ । अपि॑ रोहति । रो॒ह॒ति॒ वारे॑वृतम् । वारे॑वृतꣳ॒॒ हि । वारे॑वृत॒ मिति॒ वारे᳚ - वृ॒त॒म् । ह्य॑स्यै । अ॒स्यै॒ तृती॑यम् । तृती॑यम् ब्रह्मह॒त्यायै᳚ । ब्र॒ह्म॒ह॒त्यायै॒ प्रति॑ । ब्र॒ह्म॒ह॒त्याया॒ इति॑ ब्रह्म - ह॒त्यायै᳚ । प्रत्य॑गृह्णात् । अ॒गृ॒ह्णा॒त् तत् । तथ् स्वकृ॑तम् । स्वकृ॑त॒मिरि॑णम् । स्वकृ॑त॒मिति॒ स्व - कृ॒त॒म् । इरि॑णमभवत् । अ॒भ॒व॒त् तस्मा᳚त् । तस्मा॒दाहि॑ताग्निः । आहि॑ताग्निः श्र॒द्धादे॑वः । आहि॑ताग्नि॒रित्याहि॑त - अ॒ग्निः॒ । श्र॒द्धादे॑वः॒ स्वकृ॑ते । श्र॒द्धादे॑व॒ इति॑ श्र॒द्धा - दे॒वः॒ । स्वकृ॑त॒ इरि॑णे । स्वकृ॑त॒ इति॒ स्व - कृ॒ते॒ । इरि॑णे॒ न । नाव॑ । अव॑ स्येत् । स्ये॒द् ब्र॒ह्म॒ह॒त्यायै᳚ । ब्र॒ह्म॒ह॒त्यायै॒ हि । ब्र॒ह्म॒ह॒त्याया॒ इति॑ ब्रह्म - ह॒त्यायै᳚ । ह्ये॑षः । ए॒ष वर्णः॑ । वर्णः॒ सः । स वन॒स्पतीन्॑ । वन॒स्पती॒नुप॑ । उपा॑सीदत् । अ॒सी॒द॒द॒स्यै । अ॒स्यै ब्र॑ह्मह॒त्यायै᳚ । ब्र॒ह्म॒ह॒त्यायै॒ तृती॑यम् । ब्र॒ह्म॒ह॒त्याया॒ इति॑ ब्रह्म - ह॒त्यायै᳚ । तृती॑य॒म् प्रति॑ । प्रति॑ गृह्णीत । गृ॒ह्णी॒तेति॑ । इति॒ ते । ते᳚ ऽब्रुवन्न् । अ॒ब्रु॒व॒न् वर᳚म् । वर॑म् ॅवृणामहै । वृ॒णा॒म॒है॒ वृ॒क्णात् । वृ॒क्णात् प॑राभवि॒ष्यन्तः॑ \newline

\textbf{Jatai Paata} \newline

1. सं॒ॅव॒थ्स॒रा दप्यपि॑ संॅवथ्स॒राथ् सं॑ॅवथ्स॒रा दपि॑ । \newline
2. सं॒ॅव॒थ्स॒रादिति॑ सं - व॒थ्स॒रात् । \newline
3. अपि॑ रोहाद् रोहा॒ दप्यपि॑ रोहात् । \newline
4. रो॒हा॒ दितीति॑ रोहाद् रोहा॒दिति॑ । \newline
5. इत्य॑ब्रवी दब्रवी॒ दिती त्य॑ब्रवीत् । \newline
6. अ॒ब्र॒वी॒त् तस्मा॒त् तस्मा॑ दब्रवी दब्रवी॒त् तस्मा᳚त् । \newline
7. तस्मा᳚त् पु॒रा पु॒रा तस्मा॒त् तस्मा᳚त् पु॒रा । \newline
8. पु॒रा सं॑ॅवथ्स॒राथ् सं॑ॅवथ्स॒रात् पु॒रा पु॒रा सं॑ॅवथ्स॒रात् । \newline
9. सं॒ॅव॒थ्स॒रात् पृ॑थि॒व्यै पृ॑थि॒व्यै सं॑ॅवथ्स॒राथ् सं॑ॅवथ्स॒रात् पृ॑थि॒व्यै । \newline
10. सं॒ॅव॒थ्स॒रादिति॑ सं - व॒थ्स॒रात् । \newline
11. पृ॒थि॒व्यै खा॒तम् खा॒तम् पृ॑थि॒व्यै पृ॑थि॒व्यै खा॒तम् । \newline
12. खा॒त मप्यपि॑ खा॒तम् खा॒त मपि॑ । \newline
13. अपि॑ रोहति रोह॒ त्यप्यपि॑ रोहति । \newline
14. रो॒ह॒ति॒ वारे॑वृतं॒ ॅवारे॑वृतꣳ रोहति रोहति॒ वारे॑वृतम् । \newline
15. वारे॑वृतꣳ॒॒ हि हि वारे॑वृतं॒ ॅवारे॑वृतꣳ॒॒ हि । \newline
16. वारे॑वृत॒मिति॒ वारे᳚ - वृ॒त॒म् । \newline
17. ह्य॑स्या अस्यै॒ हि ह्य॑स्यै । \newline
18. अ॒स्यै॒ तृती॑य॒म् तृती॑य मस्या अस्यै॒ तृती॑यम् । \newline
19. तृती॑यम् ब्रह्मह॒त्यायै᳚ ब्रह्मह॒त्यायै॒ तृती॑य॒म् तृती॑यम् ब्रह्मह॒त्यायै᳚ । \newline
20. ब्र॒ह्म॒ह॒त्यायै॒ प्रति॒ प्रति॑ ब्रह्मह॒त्यायै᳚ ब्रह्मह॒त्यायै॒ प्रति॑ । \newline
21. ब्र॒ह्म॒ह॒त्याया॒ इति॑ ब्रह्म - ह॒त्यायै᳚ । \newline
22. प्रत्य॑गृह्णा दगृह्णा॒त् प्रति॒ प्रत्य॑गृह्णात् । \newline
23. अ॒गृ॒ह्णा॒त् तत् तद॑गृह्णा दगृह्णा॒त् तत् । \newline
24. तथ् स्वकृ॑तꣳ॒॒ स्वकृ॑त॒म् तत् तथ् स्वकृ॑तम् । \newline
25. स्वकृ॑त॒ मिरि॑ण॒ मिरि॑णꣳ॒॒ स्वकृ॑तꣳ॒॒ स्वकृ॑त॒ मिरि॑णम् । \newline
26. स्वकृ॑त॒मिति॒ स्व - कृ॒त॒म् । \newline
27. इरि॑ण मभव दभव॒ दिरि॑ण॒ मिरि॑ण मभवत् । \newline
28. अ॒भ॒व॒त् तस्मा॒त् तस्मा॑ दभव दभव॒त् तस्मा᳚त् । \newline
29. तस्मा॒ दाहि॑ताग्नि॒ राहि॑ताग्नि॒ स्तस्मा॒त् तस्मा॒ दाहि॑ताग्निः । \newline
30. आहि॑ताग्निः श्र॒द्धादे॑वः श्र॒द्धादे॑व॒ आहि॑ताग्नि॒ राहि॑ताग्निः श्र॒द्धादे॑वः । \newline
31. आहि॑ताग्नि॒रित्याहि॑त - अ॒ग्निः॒ । \newline
32. श्र॒द्धादे॑वः॒ स्वकृ॑ते॒ स्वकृ॑ते श्र॒द्धादे॑वः श्र॒द्धादे॑वः॒ स्वकृ॑ते । \newline
33. श्र॒द्धादे॑व॒ इति॑ श्र॒द्धा - दे॒वः॒ । \newline
34. स्वकृ॑त॒ इरि॑ण॒ इरि॑णे॒ स्वकृ॑ते॒ स्वकृ॑त॒ इरि॑णे । \newline
35. स्वकृ॑त॒ इति॒ स्व - कृ॒ते॒ । \newline
36. इरि॑णे॒ न ने रि॑ण॒ इरि॑णे॒ न । \newline
37. नावाव॒ न नाव॑ । \newline
38. अव॑ स्येथ् स्ये॒ दवाव॑ स्येत् । \newline
39. स्ये॒द् ब्र॒ह्म॒ह॒त्यायै᳚ ब्रह्मह॒त्यायै᳚ स्येथ् स्येद् ब्रह्मह॒त्यायै᳚ । \newline
40. ब्र॒ह्म॒ह॒त्यायै॒ हि हि ब्र॑ह्मह॒त्यायै᳚ ब्रह्मह॒त्यायै॒ हि । \newline
41. ब्र॒ह्म॒ह॒त्याया॒ इति॑ ब्रह्म - ह॒त्यायै᳚ । \newline
42. ह्ये॑ष ए॒ष हि ह्ये॑षः । \newline
43. ए॒ष वर्णो॒ वर्ण॑ ए॒ष ए॒ष वर्णः॑ । \newline
44. वर्णः॒ स स वर्णो॒ वर्णः॒ सः । \newline
45. स वन॒स्पती॒न्॒. वन॒स्पती॒न् थ्स स वन॒स्पतीन्॑ । \newline
46. वन॒स्पती॒ नुपोप॒ वन॒स्पती॒न्॒. वन॒स्पती॒ नुप॑ । \newline
47. उपा॑सी ददसीद॒ दुपोपा॑ सीदत् । \newline
48. अ॒सी॒द॒ द॒स्या अ॒स्या अ॑सीद दसीद द॒स्यै । \newline
49. अ॒स्यै ब्र॑ह्मह॒त्यायै᳚ ब्रह्मह॒त्याया॑ अ॒स्या अ॒स्यै ब्र॑ह्मह॒त्यायै᳚ । \newline
50. ब्र॒ह्म॒ह॒त्यायै॒ तृती॑य॒म् तृती॑यम् ब्रह्मह॒त्यायै᳚ ब्रह्मह॒त्यायै॒ तृती॑यम् । \newline
51. ब्र॒ह्म॒ह॒त्याया॒ इति॑ ब्रह्म - ह॒त्यायै᳚ । \newline
52. तृती॑य॒म् प्रति॒ प्रति॒ तृती॑य॒म् तृती॑य॒म् प्रति॑ । \newline
53. प्रति॑ गृह्णीत गृह्णीत॒ प्रति॒ प्रति॑ गृह्णीत । \newline
54. गृ॒ह्णी॒ते तीति॑ गृह्णीत गृह्णी॒ते ति॑ । \newline
55. इति॒ ते त इतीति॒ ते । \newline
56. ते᳚ ऽब्रुवन् नब्रुव॒न् ते ते᳚ ऽब्रुवन्न् । \newline
57. अ॒ब्रु॒व॒न्॒. वरं॒ ॅवर॑ मब्रुवन् नब्रुव॒न्॒. वर᳚म् । \newline
58. वरं॑ ॅवृणामहै वृणामहै॒ वरं॒ ॅवरं॑ ॅवृणामहै । \newline
59. वृ॒णा॒म॒है॒ वृ॒क्णाद् वृ॒क्णाद् वृ॑णामहै वृणामहै वृ॒क्णात् । \newline
60. वृ॒क्णात् प॑राभवि॒ष्यन्तः॑ पराभवि॒ष्यन्तो॑ वृ॒क्णाद् वृ॒क्णात् प॑राभवि॒ष्यन्तः॑ । \newline

\textbf{Ghana Paata } \newline

1. सं॒ॅव॒थ्स॒रा दप्यपि॑ संॅवथ्स॒राथ् सं॑ॅवथ्स॒रा दपि॑ रोहाद् रोहा॒दपि॑ संॅवथ्स॒राथ् सं॑ॅवथ्स॒रा दपि॑ रोहात् । \newline
2. सं॒ॅव॒थ्स॒रादिति॑ सं - व॒थ्स॒रात् । \newline
3. अपि॑ रोहाद् रोहा॒ दप्यपि॑ रोहा॒ दितीति॑ रोहा॒ दप्यपि॑ रोहा॒दिति॑ । \newline
4. रो॒हा॒ दितीति॑ रोहाद् रोहा॒ दित्य॑ब्रवी दब्रवी॒ दिति॑ रोहाद् रोहा॒ दित्य॑ब्रवीत् । \newline
5. इत्य॑ब्रवीदब्रवी॒ दितीत्य॑ब्रवी॒त् तस्मा॒त् तस्मा॑ दब्रवी॒ दितीत्य॑ब्रवी॒त् तस्मा᳚त् । \newline
6. अ॒ब्र॒वी॒त् तस्मा॒त् तस्मा॑ दब्रवी दब्रवी॒त् तस्मा᳚त् पु॒रा पु॒रा तस्मा॑ दब्रवी दब्रवी॒त् तस्मा᳚त् पु॒रा । \newline
7. तस्मा᳚त् पु॒रा पु॒रा तस्मा॒त् तस्मा᳚त् पु॒रा सं॑ॅवथ्स॒राथ् सं॑ॅवथ्स॒रात् पु॒रा तस्मा॒त् तस्मा᳚त् पु॒रा सं॑ॅवथ्स॒रात् । \newline
8. पु॒रा सं॑ॅवथ्स॒राथ् सं॑ॅवथ्स॒रात् पु॒रा पु॒रा सं॑ॅवथ्स॒रात् पृ॑थि॒व्यै पृ॑थि॒व्यै सं॑ॅवथ्स॒रात् पु॒रा पु॒रा सं॑ॅवथ्स॒रात् पृ॑थि॒व्यै । \newline
9. सं॒ॅव॒थ्स॒रात् पृ॑थि॒व्यै पृ॑थि॒व्यै सं॑ॅवथ्स॒राथ् सं॑ॅवथ्स॒रात् पृ॑थि॒व्यै खा॒तम् खा॒तम् पृ॑थि॒व्यै सं॑ॅवथ्स॒राथ् सं॑ॅवथ्स॒रात् पृ॑थि॒व्यै खा॒तम् । \newline
10. सं॒ॅव॒थ्स॒रादिति॑ सं - व॒थ्स॒रात् । \newline
11. पृ॒थि॒व्यै खा॒तम् खा॒तम् पृ॑थि॒व्यै पृ॑थि॒व्यै खा॒त मप्यपि॑ खा॒तम् पृ॑थि॒व्यै पृ॑थि॒व्यै खा॒त मपि॑ । \newline
12. खा॒त मप्यपि॑ खा॒तम् खा॒त मपि॑ रोहति रोह॒त्यपि॑ खा॒तम् खा॒त मपि॑ रोहति । \newline
13. अपि॑ रोहति रोह॒त्यप्यपि॑ रोहति॒ वारे॑वृतं॒ ॅवारे॑वृतꣳ रोह॒त्यप्यपि॑ रोहति॒ वारे॑वृतम् । \newline
14. रो॒ह॒ति॒ वारे॑वृतं॒ ॅवारे॑वृतꣳ रोहति रोहति॒ वारे॑वृतꣳ॒॒ हि हि वारे॑वृतꣳ रोहति रोहति॒ वारे॑वृतꣳ॒॒ हि । \newline
15. वारे॑वृतꣳ॒॒ हि हि वारे॑वृतं॒ ॅवारे॑वृतꣳ॒॒ ह्य॑स्या अस्यै॒ हि वारे॑वृतं॒ ॅवारे॑वृतꣳ॒॒ ह्य॑स्यै । \newline
16. वारे॑वृत॒मिति॒ वारे᳚ - वृ॒त॒म् । \newline
17. ह्य॑स्या अस्यै॒ हि ह्य॑स्यै॒ तृती॑य॒म् तृती॑य मस्यै॒ हि ह्य॑स्यै॒ तृती॑यम् । \newline
18. अ॒स्यै॒ तृती॑य॒म् तृती॑य मस्या अस्यै॒ तृती॑यम् ब्रह्मह॒त्यायै᳚ ब्रह्मह॒त्यायै॒ तृती॑य मस्या अस्यै॒ तृती॑यम् ब्रह्मह॒त्यायै᳚ । \newline
19. तृती॑यम् ब्रह्मह॒त्यायै᳚ ब्रह्मह॒त्यायै॒ तृती॑य॒म् तृती॑यम् ब्रह्मह॒त्यायै॒ प्रति॒ प्रति॑ ब्रह्मह॒त्यायै॒ तृती॑य॒म् तृती॑यम् ब्रह्मह॒त्यायै॒ प्रति॑ । \newline
20. ब्र॒ह्म॒ह॒त्यायै॒ प्रति॒ प्रति॑ ब्रह्मह॒त्यायै᳚ ब्रह्मह॒त्यायै॒ प्रत्य॑गृह्णा दगृह्णा॒त् प्रति॑ ब्रह्मह॒त्यायै᳚ ब्रह्मह॒त्यायै॒ प्रत्य॑गृह्णात् । \newline
21. ब्र॒ह्म॒ह॒त्याया॒ इति॑ ब्रह्म - ह॒त्यायै᳚ । \newline
22. प्रत्य॑गृह्णा दगृह्णा॒त् प्रति॒ प्रत्य॑गृह्णा॒त् तत् तद॑गृह्णा॒त् प्रति॒ प्रत्य॑गृह्णा॒त् तत् । \newline
23. अ॒गृ॒ह्णा॒त् तत् तद॑गृह्णा दगृह्णा॒त् तथ् स्वकृ॑तꣳ॒॒ स्वकृ॑त॒म् तद॑गृह्णा दगृह्णा॒त् तथ् स्वकृ॑तम् । \newline
24. तथ् स्वकृ॑तꣳ॒॒ स्वकृ॑त॒म् तत् तथ् स्वकृ॑त॒ मिरि॑ण॒ मिरि॑णꣳ॒॒ स्वकृ॑त॒म् तत् तथ् स्वकृ॑त॒ मिरि॑णम् । \newline
25. स्वकृ॑त॒ मिरि॑ण॒ मिरि॑णꣳ॒॒ स्वकृ॑तꣳ॒॒ स्वकृ॑त॒ मिरि॑ण मभव दभव॒ दिरि॑णꣳ॒॒ स्वकृ॑तꣳ॒॒ स्वकृ॑त॒ मिरि॑ण मभवत् । \newline
26. स्वकृ॑त॒मिति॒ स्व - कृ॒त॒म् । \newline
27. इरि॑ण मभव दभव॒ दिरि॑ण॒ मिरि॑ण मभव॒त् तस्मा॒त् तस्मा॑ दभव॒ दिरि॑ण॒ मिरि॑ण मभव॒त् तस्मा᳚त् । \newline
28. अ॒भ॒व॒त् तस्मा॒त् तस्मा॑ दभव दभव॒त् तस्मा॒ दाहि॑ताग्नि॒ राहि॑ताग्नि॒ स्तस्मा॑ दभव दभव॒त् तस्मा॒ दाहि॑ताग्निः । \newline
29. तस्मा॒ दाहि॑ताग्नि॒ राहि॑ताग्नि॒ स्तस्मा॒त् तस्मा॒ दाहि॑ताग्निः श्र॒द्धादे॑वः श्र॒द्धादे॑व॒ आहि॑ताग्नि॒ स्तस्मा॒त् तस्मा॒ दाहि॑ताग्निः श्र॒द्धादे॑वः । \newline
30. आहि॑ताग्निः श्र॒द्धादे॑वः श्र॒द्धादे॑व॒ आहि॑ताग्नि॒ राहि॑ताग्निः श्र॒द्धादे॑वः॒ स्वकृ॑ते॒ स्वकृ॑ते श्र॒द्धादे॑व॒ आहि॑ताग्नि॒ राहि॑ताग्निः श्र॒द्धादे॑वः॒ स्वकृ॑ते । \newline
31. आहि॑ताग्नि॒रित्याहि॑त - अ॒ग्निः॒ । \newline
32. श्र॒द्धादे॑वः॒ स्वकृ॑ते॒ स्वकृ॑ते श्र॒द्धादे॑वः श्र॒द्धादे॑वः॒ स्वकृ॑त॒ इरि॑ण॒ इरि॑णे॒ स्वकृ॑ते श्र॒द्धादे॑वः श्र॒द्धादे॑वः॒ स्वकृ॑त॒ इरि॑णे । \newline
33. श्र॒द्धादे॑व॒ इति॑ श्र॒द्धा - दे॒वः॒ । \newline
34. स्वकृ॑त॒ इरि॑ण॒ इरि॑णे॒ स्वकृ॑ते॒ स्वकृ॑त॒ इरि॑णे॒ न ने रि॑णे॒ स्वकृ॑ते॒ स्वकृ॑त॒ इरि॑णे॒ न । \newline
35. स्वकृ॑त॒ इति॒ स्व - कृ॒ते॒ । \newline
36. इरि॑णे॒ न ने रि॑ण॒ इरि॑णे॒ नावाव॒ ने रि॑ण॒ इरि॑णे॒ नाव॑ । \newline
37. नावाव॒ न नाव॑ स्येथ् स्ये॒दव॒ न नाव॑ स्येत् । \newline
38. अव॑ स्येथ् स्ये॒दवाव॑ स्येद् ब्रह्मह॒त्यायै᳚ ब्रह्मह॒त्यायै᳚ स्ये॒दवाव॑ स्येद् ब्रह्मह॒त्यायै᳚ । \newline
39. स्ये॒द् ब्र॒ह्म॒ह॒त्यायै᳚ ब्रह्मह॒त्यायै᳚ स्येथ् स्येद् ब्रह्मह॒त्यायै॒ हि हि ब्र॑ह्मह॒त्यायै᳚ स्येथ् स्येद् ब्रह्मह॒त्यायै॒ हि । \newline
40. ब्र॒ह्म॒ह॒त्यायै॒ हि हि ब्र॑ह्मह॒त्यायै᳚ ब्रह्मह॒त्यायै॒ ह्ये॑ष ए॒ष हि ब्र॑ह्मह॒त्यायै᳚ ब्रह्मह॒त्यायै॒ ह्ये॑षः । \newline
41. ब्र॒ह्म॒ह॒त्याया॒ इति॑ ब्रह्म - ह॒त्यायै᳚ । \newline
42. ह्ये॑ष ए॒ष हि ह्ये॑ष वर्णो॒ वर्ण॑ ए॒ष हि ह्ये॑ष वर्णः॑ । \newline
43. ए॒ष वर्णो॒ वर्ण॑ ए॒ष ए॒ष वर्णः॒ स स वर्ण॑ ए॒ष ए॒ष वर्णः॒ सः । \newline
44. वर्णः॒ स स वर्णो॒ वर्णः॒ स वन॒स्पती॒न्॒. वन॒स्पती॒न् थ्स वर्णो॒ वर्णः॒ स वन॒स्पतीन्॑ । \newline
45. स वन॒स्पती॒न्॒. वन॒स्पती॒न् थ्स स वन॒स्पती॒ नुपोप॒ वन॒स्पती॒न् थ्स स वन॒स्पती॒ नुप॑ । \newline
46. वन॒स्पती॒ नुपोप॒ वन॒स्पती॒न्॒. वन॒स्पती॒ नुपा॑सीद दसीद॒ दुप॒ वन॒स्पती॒न्॒. वन॒स्पती॒ नुपा॑सीदत् । \newline
47. उपा॑सीद दसीद॒ दुपोपा॑सीद द॒स्या अ॒स्या अ॑सीद॒ दुपोपा॑सीद द॒स्यै । \newline
48. अ॒सी॒द॒द॒स्या अ॒स्या अ॑सीद दसीद द॒स्यै ब्र॑ह्मह॒त्यायै᳚ ब्रह्मह॒त्याया॑ अ॒स्या अ॑सीद दसीद द॒स्यै ब्र॑ह्मह॒त्यायै᳚ । \newline
49. अ॒स्यै ब्र॑ह्मह॒त्यायै᳚ ब्रह्मह॒त्याया॑ अ॒स्या अ॒स्यै ब्र॑ह्मह॒त्यायै॒ तृती॑य॒म् तृती॑यम् ब्रह्मह॒त्याया॑ अ॒स्या अ॒स्यै ब्र॑ह्मह॒त्यायै॒ तृती॑यम् । \newline
50. ब्र॒ह्म॒ह॒त्यायै॒ तृती॑य॒म् तृती॑यम् ब्रह्मह॒त्यायै᳚ ब्रह्मह॒त्यायै॒ तृती॑य॒म् प्रति॒ प्रति॒ तृती॑यम् ब्रह्मह॒त्यायै᳚ ब्रह्मह॒त्यायै॒ तृती॑य॒म् प्रति॑ । \newline
51. ब्र॒ह्म॒ह॒त्याया॒ इति॑ ब्रह्म - ह॒त्यायै᳚ । \newline
52. तृती॑य॒म् प्रति॒ प्रति॒ तृती॑य॒म् तृती॑य॒म् प्रति॑ गृह्णीत गृह्णीत॒ प्रति॒ तृती॑य॒म् तृती॑य॒म् प्रति॑ गृह्णीत । \newline
53. प्रति॑ गृह्णीत गृह्णीत॒ प्रति॒ प्रति॑ गृह्णी॒ते तीति॑ गृह्णीत॒ प्रति॒ प्रति॑ गृह्णी॒ते ति॑ । \newline
54. गृ॒ह्णी॒ते तीति॑ गृह्णीत गृह्णी॒ते ति॒ ते त इति॑ गृह्णीत गृह्णी॒ते ति॒ ते । \newline
55. इति॒ ते त इतीति॒ ते᳚ ऽब्रुवन् नब्रुव॒न् त इतीति॒ ते᳚ ऽब्रुवन्न् । \newline
56. ते᳚ ऽब्रुवन् नब्रुव॒न् ते ते᳚ ऽब्रुव॒न्॒. वरं॒ ॅवर॑ मब्रुव॒न् ते ते᳚ ऽब्रुव॒न्॒. वर᳚म् । \newline
57. अ॒ब्रु॒व॒न्॒. वरं॒ ॅवर॑ मब्रुवन् नब्रुव॒न्॒. वरं॑ ॅवृणामहै वृणामहै॒ वर॑ मब्रुवन् नब्रुव॒न्॒. वरं॑ ॅवृणामहै । \newline
58. वरं॑ ॅवृणामहै वृणामहै॒ वरं॒ ॅवरं॑ ॅवृणामहै वृ॒क्णाद् वृ॒क्णाद् वृ॑णामहै॒ वरं॒ ॅवरं॑ ॅवृणामहै वृ॒क्णात् । \newline
59. वृ॒णा॒म॒है॒ वृ॒क्णाद् वृ॒क्णाद् वृ॑णामहै वृणामहै वृ॒क्णात् प॑राभवि॒ष्यन्तः॑ पराभवि॒ष्यन्तो॑ वृ॒क्णाद् वृ॑णामहै वृणामहै वृ॒क्णात् प॑राभवि॒ष्यन्तः॑ । \newline
60. वृ॒क्णात् प॑राभवि॒ष्यन्तः॑ पराभवि॒ष्यन्तो॑ वृ॒क्णाद् वृ॒क्णात् प॑राभवि॒ष्यन्तो॑ मन्यामहे मन्यामहे पराभवि॒ष्यन्तो॑ वृ॒क्णाद् वृ॒क्णात् प॑राभवि॒ष्यन्तो॑ मन्यामहे । \newline
\pagebreak
\markright{ TS 2.5.1.4  \hfill https://www.vedavms.in \hfill}
\addcontentsline{toc}{section}{ TS 2.5.1.4 }
\section*{ TS 2.5.1.4 }

\textbf{TS 2.5.1.4 } \newline
\textbf{Samhita Paata} \newline

प॑राभवि॒ष्यन्तो॑ मन्यामहे॒ ततो॒ मा परा॑ भू॒मेत्या॒व्रश्च॑नाद्वो॒ भूयाꣳ॑स॒ उत्ति॑ष्ठा॒नित्य॑ब्रवी॒त् तस्मा॑दा॒व्रश्च॑नाद्-वृ॒क्षाणां॒ भूयाꣳ॑स॒ उत्ति॑ष्ठन्ति॒ वारे॑वृतꣳ॒॒ ह्ये॑षां॒ तृती॑यं ब्रह्मह॒त्यायै॒ प्रत्य॑गृण्ह॒न्थ्स नि॑र्या॒सो॑ ऽभव॒त् तस्मा᳚न्निर्या॒सस्य॒ नाऽऽ*श्यं॑ ब्रह्मह॒त्यायै॒ ह्ये॑ष वर्णोऽथो॒ खलु॒ य ए॒व लोहि॑तो॒ यो वा॒ऽऽव्रश्च॑नान्नि॒र्येष॑ति॒ तस्य॒ नाऽऽश्यं॑ - [  ] \newline

\textbf{Pada Paata} \newline

प॒रा॒भ॒वि॒ष्यन्त॒ इति॑ परा-भ॒वि॒ष्यन्तः॑ । म॒न्या॒म॒हे॒ । ततः॑ । मा । परेति॑ । भू॒म॒ । इति॑ । आ॒व्रश्च॑ना॒दित्या᳚-व्रश्च॑नात् । वः॒ । भूयाꣳ॑सः । उदिति॑ । ति॒ष्ठा॒न् । इति॑ । अ॒ब्र॒वी॒त् । तस्मा᳚त् । आ॒व्रश्च॑ना॒दित्या᳚ - व्रश्च॑नात् । वृ॒क्षाणा᳚म् । भूयाꣳ॑सः । उदिति॑ । ति॒ष्ठ॒न्ति॒ । वारे॑वृत॒मिति॒ वारे᳚ - वृ॒त॒म् । हि । ए॒षा॒म् । तृती॑यम् । ब्र॒ह्म॒ह॒त्याया॒ इति॑ ब्रह्म - ह॒त्यायै᳚ । प्रतीति॑ । अ॒गृ॒ह्ण॒न्न् । सः । नि॒र्या॒स इति॑ निः - या॒सः । अ॒भ॒व॒त् । तस्मा᳚त् । नि॒र्या॒सस्येति॑ निः - या॒सस्य॑ । न । आ॒श्य᳚म् । ब्र॒ह्म॒ह॒त्याया॒ इति॑ ब्रह्म-ह॒त्यायै᳚ । हि । ए॒षः । वर्णः॑ । अथो॒ इति॑ । खलु॑ । यः । ए॒व । लोहि॑तः । यः । वा॒ । आ॒व्रश्च॑ना॒दित्या᳚ - व्रश्च॑नात् । नि॒र्येष॒तीति॑ निः - येष॑ति । तस्य॑ । न । आ॒श्य᳚म् ।  \newline


\textbf{Krama Paata} \newline

प॒रा॒भ॒वि॒ष्यन्तो॑ मन्यामहे । प॒रा॒भ॒वि॒ष्यन्त॒ इति॑ परा - भ॒वि॒ष्यन्तः॑ । म॒न्या॒म॒हे॒ ततः॑ । ततो॒ मा । मा परा᳚ । परा॑ भूम । भू॒मेति॑ । इत्या॒व्रश्च॑नात् । आ॒व्रश्च॑नाद् वः । आ॒व्रश्च॑ना॒दित्या᳚ - व्रश्च॑नात् । वो॒ भूयाꣳ॑सः । भूयाꣳ॑स॒ उत् । उत् ति॑ष्ठान् । ति॒ष्ठा॒निति॑ । इत्य॑ब्रवीत् । अ॒ब्र॒वी॒त् तस्मा᳚त् । तस्मा॑दा॒व्रश्च॑नात् । आ॒व्रश्च॑नाद् वृ॒क्षाणा᳚म् । आ॒व्रश्च॑ना॒दित्या᳚ - व्रश्च॑नात् । वृ॒क्षाणा॒म् भूयाꣳ॑सः । भूयाꣳ॑स॒ उत् । उत् ति॑ष्ठन्ति । ति॒ष्ठ॒न्ति॒ वारे॑वृतम् । वारे॑वृतꣳ॒॒ हि । वारे॑वृत॒मिति॒ वारे᳚ - वृ॒त॒म् । ह्ये॑षाम् । ए॒षा॒म् तृती॑यम् । तृती॑यम् ब्रह्मह॒त्यायै᳚ । ब्र॒ह्म॒ह॒त्यायै॒ प्रति॑ । ब्र॒ह्म॒ह॒त्याया॒ इति॑ ब्रह्म - ह॒त्यायै᳚ । प्रत्य॑गृह्णन्न् । अ॒गृ॒ह्ण॒न्थ् सः । स नि॑र्या॒सः । नि॒र्या॒सो॑ ऽभवत् । नि॒र्या॒स इति॑ निः - या॒सः । अ॒भ॒व॒त् तस्मा᳚त् । तस्मा᳚न्निर्या॒सस्य॑ । नि॒र्या॒सस्य॒ न । नि॒र्या॒सस्येति॑ निः - या॒सस्य॑ । नाश्य᳚म् । आ॒श्य॑म् ब्रह्मह॒त्यायै᳚ । ब्र॒ह्म॒ह॒त्यायै॒ हि । ब्र॒ह्म॒ह॒त्याया॒ इति॑ ब्रह्म - ह॒त्यायै᳚ । ह्ये॑षः । ए॒ष वर्णः॑ । वर्णोऽथो᳚ । अथो॒ खलु॑ । अथो॒ इत्यथो᳚ । खलु॒ यः । य ए॒व । ए॒व लोहि॑तः । लोहि॑तो॒ यः । यो वा᳚ । वा॒ ऽऽव्रश्च॑नात् । आ॒व्रश्च॑नान्नि॒र्येष॑ति । आ॒व्रश्च॑ना॒दित्या᳚ - व्रश्च॑नात् । नि॒र्येष॑ति॒ तस्य॑ । नि॒र्येष॒तीति॑ निः - येष॑ति । तस्य॒ न । नाश्य᳚म् । आ॒श्य॑म् काम᳚म् \newline

\textbf{Jatai Paata} \newline

1. प॒रा॒भ॒वि॒ष्यन्तो॑ मन्यामहे मन्यामहे पराभवि॒ष्यन्तः॑ पराभवि॒ष्यन्तो॑ मन्यामहे । \newline
2. प॒रा॒भ॒वि॒ष्यन्त॒ इति॑ परा - भ॒वि॒ष्यन्तः॑ । \newline
3. म॒न्या॒म॒हे॒ तत॒ स्ततो॑ मन्यामहे मन्यामहे॒ ततः॑ । \newline
4. ततो॒ मा मा तत॒ स्ततो॒ मा । \newline
5. मा परा॒ परा॒ मा मा परा᳚ । \newline
6. परा॑ भूम भूम॒ परा॒ परा॑ भूम । \newline
7. भू॒मे तीति॑ भूम भू॒मे ति॑ । \newline
8. इत्या॒व्रश्च॑ना दा॒व्रश्च॑ना॒ दिती त्या॒व्रश्च॑नात् । \newline
9. आ॒व्रश्च॑नाद् वो व आ॒व्रश्च॑ना दा॒व्रश्च॑नाद् वः । \newline
10. आ॒व्रश्च॑ना॒दित्या᳚ - व्रश्च॑नात् । \newline
11. वो॒ भूयाꣳ॑सो॒ भूयाꣳ॑सो वो वो॒ भूयाꣳ॑सः । \newline
12. भूयाꣳ॑स॒ उदुद् भूयाꣳ॑सो॒ भूयाꣳ॑स॒ उत् । \newline
13. उत् ति॑ष्ठान् तिष्ठा॒ नुदुत् ति॑ष्ठान् । \newline
14. ति॒ष्ठा॒ नितीति॑ तिष्ठान् तिष्ठा॒ निति॑ । \newline
15. इत्य॑ब्रवी दब्रवी॒ दिती त्य॑ब्रवीत् । \newline
16. अ॒ब्र॒वी॒त् तस्मा॒त् तस्मा॑ दब्रवी दब्रवी॒त् तस्मा᳚त् । \newline
17. तस्मा॑ दा॒व्रश्च॑ना दा॒व्रश्च॑ना॒त् तस्मा॒त् तस्मा॑ दा॒व्रश्च॑नात् । \newline
18. आ॒व्रश्च॑नाद् वृ॒क्षाणां᳚ ॅवृ॒क्षाणा॑ मा॒व्रश्च॑ना दा॒व्रश्च॑नाद् वृ॒क्षाणा᳚म् । \newline
19. आ॒व्रश्च॑ना॒दित्या᳚ - व्रश्च॑नात् । \newline
20. वृ॒क्षाणा॒म् भूयाꣳ॑सो॒ भूयाꣳ॑सो वृ॒क्षाणां᳚ ॅवृ॒क्षाणा॒म् भूयाꣳ॑सः । \newline
21. भूयाꣳ॑स॒ उदुद् भूयाꣳ॑सो॒ भूयाꣳ॑स॒ उत् । \newline
22. उत् ति॑ष्ठन्ति तिष्ठ॒ न्त्युदुत् ति॑ष्ठन्ति । \newline
23. ति॒ष्ठ॒न्ति॒ वारे॑वृतं॒ ॅवारे॑वृतम् तिष्ठन्ति तिष्ठन्ति॒ वारे॑वृतम् । \newline
24. वारे॑वृतꣳ॒॒ हि हि वारे॑वृतं॒ ॅवारे॑वृतꣳ॒॒ हि । \newline
25. वारे॑वृत॒मिति॒ वारे᳚ - वृ॒त॒म् । \newline
26. ह्ये॑षा मेषाꣳ॒॒ हि ह्ये॑षाम् । \newline
27. ए॒षा॒म् तृती॑य॒म् तृती॑य मेषा मेषा॒म् तृती॑यम् । \newline
28. तृती॑यम् ब्रह्मह॒त्यायै᳚ ब्रह्मह॒त्यायै॒ तृती॑य॒म् तृती॑यम् ब्रह्मह॒त्यायै᳚ । \newline
29. ब्र॒ह्म॒ह॒त्यायै॒ प्रति॒ प्रति॑ ब्रह्मह॒त्यायै᳚ ब्रह्मह॒त्यायै॒ प्रति॑ । \newline
30. ब्र॒ह्म॒ह॒त्याया॒ इति॑ ब्रह्म - ह॒त्यायै᳚ । \newline
31. प्रत्य॑गृह्णन् नगृह्ण॒न् प्रति॒ प्रत्य॑गृह्णन्न् । \newline
32. अ॒गृ॒ह्ण॒न् थ्स सो॑ ऽगृह्णन् नगृह्ण॒न् थ्सः । \newline
33. स नि॑र्या॒सो नि॑र्या॒सः स स नि॑र्या॒सः । \newline
34. नि॒र्या॒सो॑ ऽभव दभवन् निर्या॒सो नि॑र्या॒सो॑ ऽभवत् । \newline
35. नि॒र्या॒स इति॑ निः - या॒सः । \newline
36. अ॒भ॒व॒त् तस्मा॒त् तस्मा॑ दभव दभव॒त् तस्मा᳚त् । \newline
37. तस्मा᳚न् निर्या॒सस्य॑ निर्या॒सस्य॒ तस्मा॒त् तस्मा᳚न् निर्या॒सस्य॑ । \newline
38. नि॒र्या॒सस्य॒ न न नि॑र्या॒सस्य॑ निर्या॒सस्य॒ न । \newline
39. नि॒र्या॒सस्येति॑ निः - या॒सस्य॑ । \newline
40. नाश्य॑ मा॒श्य॑म् न नाश्य᳚म् । \newline
41. आ॒श्य॑म् ब्रह्मह॒त्यायै᳚ ब्रह्मह॒त्याया॑ आ॒श्य॑ मा॒श्य॑म् ब्रह्मह॒त्यायै᳚ । \newline
42. ब्र॒ह्म॒ह॒त्यायै॒ हि हि ब्र॑ह्मह॒त्यायै᳚ ब्रह्मह॒त्यायै॒ हि । \newline
43. ब्र॒ह्म॒ह॒त्याया॒ इति॑ ब्रह्म - ह॒त्यायै᳚ । \newline
44. ह्ये॑ष ए॒ष हि ह्ये॑षः । \newline
45. ए॒ष वर्णो॒ वर्ण॑ ए॒ष ए॒ष वर्णः॑ । \newline
46. वर्णो ऽथो॒ अथो॒ वर्णो॒ वर्णो ऽथो᳚ । \newline
47. अथो॒ खलु॒ खल्वथो॒ अथो॒ खलु॑ । \newline
48. अथो॒ इत्यथो᳚ । \newline
49. खलु॒ यो यः खलु॒ खलु॒ यः । \newline
50. य ए॒वैव यो य ए॒व । \newline
51. ए॒व लोहि॑तो॒ लोहि॑त ए॒वैव लोहि॑तः । \newline
52. लोहि॑तो॒ यो यो लोहि॑तो॒ लोहि॑तो॒ यः । \newline
53. यो वा॑ वा॒ यो यो वा᳚ । \newline
54. वा॒ ऽऽव्रश्च॑ना दा॒व्रश्च॑नाद् वा वा॒ ऽऽव्रश्च॑नात् । \newline
55. आ॒व्रश्च॑नान् नि॒र्येष॑ति नि॒र्येष॑ त्या॒व्रश्च॑ना दा॒व्रश्च॑नान् नि॒र्येष॑ति । \newline
56. आ॒व्रश्च॑ना॒दित्या᳚ - व्रश्च॑नात् । \newline
57. नि॒र्येष॑ति॒ तस्य॒ तस्य॑ नि॒र्येष॑ति नि॒र्येष॑ति॒ तस्य॑ । \newline
58. नि॒र्येष॒तीति॑ निः - येष॑ति । \newline
59. तस्य॒ न न तस्य॒ तस्य॒ न । \newline
60. नाश्य॑ मा॒श्य॑म् न नाश्य᳚म् । \newline
61. आ॒श्य॑म् काम॒म् काम॑ मा॒श्य॑ मा॒श्य॑म् काम᳚म् । \newline

\textbf{Ghana Paata } \newline

1. प॒रा॒भ॒वि॒ष्यन्तो॑ मन्यामहे मन्यामहे पराभवि॒ष्यन्तः॑ पराभवि॒ष्यन्तो॑ मन्यामहे॒ तत॒ स्ततो॑ मन्यामहे पराभवि॒ष्यन्तः॑ पराभवि॒ष्यन्तो॑ मन्यामहे॒ ततः॑ । \newline
2. प॒रा॒भ॒वि॒ष्यन्त॒ इति॑ परा - भ॒वि॒ष्यन्तः॑ । \newline
3. म॒न्या॒म॒हे॒ तत॒ स्ततो॑ मन्यामहे मन्यामहे॒ ततो॒ मा मा ततो॑ मन्यामहे मन्यामहे॒ ततो॒ मा । \newline
4. ततो॒ मा मा तत॒ स्ततो॒ मा परा॒ परा॒ मा तत॒ स्ततो॒ मा परा᳚ । \newline
5. मा परा॒ परा॒ मा मा परा॑ भूम भूम॒ परा॒ मा मा परा॑ भूम । \newline
6. परा॑ भूम भूम॒ परा॒ परा॑ भू॒मे तीति॑ भूम॒ परा॒ परा॑ भू॒मे ति॑ । \newline
7. भू॒मे तीति॑ भूम भू॒मे त्या॒व्रश्च॑ना दा॒व्रश्च॑ना॒ दिति॑ भूम भू॒मे त्या॒व्रश्च॑नात् । \newline
8. इत्या॒व्रश्च॑ना दा॒व्रश्च॑ना॒ दिती त्या॒व्रश्च॑नाद् वो व आ॒व्रश्च॑ना॒ दितीत्या॒व्रश्च॑नाद् वः । \newline
9. आ॒व्रश्च॑नाद् वो व आ॒व्रश्च॑ना दा॒व्रश्च॑नाद् वो॒ भूयाꣳ॑सो॒ भूयाꣳ॑सो व आ॒व्रश्च॑ना दा॒व्रश्च॑नाद् वो॒ भूयाꣳ॑सः । \newline
10. आ॒व्रश्च॑ना॒दित्या᳚ - व्रश्च॑नात् । \newline
11. वो॒ भूयाꣳ॑सो॒ भूयाꣳ॑सो वो वो॒ भूयाꣳ॑स॒ उदुद् भूयाꣳ॑सो वो वो॒ भूयाꣳ॑स॒ उत् । \newline
12. भूयाꣳ॑स॒ उदुद् भूयाꣳ॑सो॒ भूयाꣳ॑स॒ उत् ति॑ष्ठान् तिष्ठा॒ नुद् भूयाꣳ॑सो॒ भूयाꣳ॑स॒ उत् ति॑ष्ठान् । \newline
13. उत् ति॑ष्ठान् तिष्ठा॒ नुदुत् ति॑ष्ठा॒ नितीति॑ तिष्ठा॒ नुदुत् ति॑ष्ठा॒ निति॑ । \newline
14. ति॒ष्ठा॒ नितीति॑ तिष्ठान् तिष्ठा॒ नित्य॑ब्रवी दब्रवी॒ दिति॑ तिष्ठान् तिष्ठा॒ नित्य॑ब्रवीत् । \newline
15. इत्य॑ब्रवी दब्रवी॒ दितीत्य॑ब्रवी॒त् तस्मा॒त् तस्मा॑ दब्रवी॒ दितीत्य॑ब्रवी॒त् तस्मा᳚त् । \newline
16. अ॒ब्र॒वी॒त् तस्मा॒त् तस्मा॑ दब्रवी दब्रवी॒त् तस्मा॑ दा॒व्रश्च॑ना दा॒व्रश्च॑ना॒त् तस्मा॑ दब्रवी दब्रवी॒त् तस्मा॑ दा॒व्रश्च॑नात् । \newline
17. तस्मा॑ दा॒व्रश्च॑ना दा॒व्रश्च॑ना॒त् तस्मा॒त् तस्मा॑ दा॒व्रश्च॑नाद् वृ॒क्षाणां᳚ ॅवृ॒क्षाणा॑ मा॒व्रश्च॑ना॒त् तस्मा॒त् तस्मा॑ दा॒व्रश्च॑नाद् वृ॒क्षाणा᳚म् । \newline
18. आ॒व्रश्च॑नाद् वृ॒क्षाणां᳚ ॅवृ॒क्षाणा॑ मा॒व्रश्च॑ना दा॒व्रश्च॑नाद् वृ॒क्षाणा॒म् भूयाꣳ॑सो॒ भूयाꣳ॑सो वृ॒क्षाणा॑ मा॒व्रश्च॑ना दा॒व्रश्च॑नाद् वृ॒क्षाणा॒म् भूयाꣳ॑सः । \newline
19. आ॒व्रश्च॑ना॒दित्या᳚ - व्रश्च॑नात् । \newline
20. वृ॒क्षाणा॒म् भूयाꣳ॑सो॒ भूयाꣳ॑सो वृ॒क्षाणां᳚ ॅवृ॒क्षाणा॒म् भूयाꣳ॑स॒ उदुद् भूयाꣳ॑सो वृ॒क्षाणां᳚ ॅवृ॒क्षाणा॒म् भूयाꣳ॑स॒ उत् । \newline
21. भूयाꣳ॑स॒ उदुद् भूयाꣳ॑सो॒ भूयाꣳ॑स॒ उत् ति॑ष्ठन्ति तिष्ठ॒न्त्युद् भूयाꣳ॑सो॒ भूयाꣳ॑स॒ उत् ति॑ष्ठन्ति । \newline
22. उत् ति॑ष्ठन्ति तिष्ठ॒न्त्युदुत् ति॑ष्ठन्ति॒ वारे॑वृतं॒ ॅवारे॑वृतम् तिष्ठ॒न्त्युदुत् ति॑ष्ठन्ति॒ वारे॑वृतम् । \newline
23. ति॒ष्ठ॒न्ति॒ वारे॑वृतं॒ ॅवारे॑वृतम् तिष्ठन्ति तिष्ठन्ति॒ वारे॑वृतꣳ॒॒ हि हि वारे॑वृतम् तिष्ठन्ति तिष्ठन्ति॒ वारे॑वृतꣳ॒॒ हि । \newline
24. वारे॑वृतꣳ॒॒ हि हि वारे॑वृतं॒ ॅवारे॑वृतꣳ॒॒ ह्ये॑षा मेषाꣳ॒॒ हि वारे॑वृतं॒ ॅवारे॑वृतꣳ॒॒ ह्ये॑षाम् । \newline
25. वारे॑वृत॒मिति॒ वारे᳚ - वृ॒त॒म् । \newline
26. ह्ये॑षा मेषाꣳ॒॒ हि ह्ये॑षा॒म् तृती॑य॒म् तृती॑य मेषाꣳ॒॒ हि ह्ये॑षा॒म् तृती॑यम् । \newline
27. ए॒षा॒म् तृती॑य॒म् तृती॑य मेषा मेषा॒म् तृती॑यम् ब्रह्मह॒त्यायै᳚ ब्रह्मह॒त्यायै॒ तृती॑य मेषा मेषा॒म् तृती॑यम् ब्रह्मह॒त्यायै᳚ । \newline
28. तृती॑यम् ब्रह्मह॒त्यायै᳚ ब्रह्मह॒त्यायै॒ तृती॑य॒म् तृती॑यम् ब्रह्मह॒त्यायै॒ प्रति॒ प्रति॑ ब्रह्मह॒त्यायै॒ तृती॑य॒म् तृती॑यम् ब्रह्मह॒त्यायै॒ प्रति॑ । \newline
29. ब्र॒ह्म॒ह॒त्यायै॒ प्रति॒ प्रति॑ ब्रह्मह॒त्यायै᳚ ब्रह्मह॒त्यायै॒ प्रत्य॑गृह्णन् नगृह्ण॒न् प्रति॑ ब्रह्मह॒त्यायै᳚ ब्रह्मह॒त्यायै॒ प्रत्य॑गृह्णन्न् । \newline
30. ब्र॒ह्म॒ह॒त्याया॒ इति॑ ब्रह्म - ह॒त्यायै᳚ । \newline
31. प्रत्य॑गृह्णन् नगृह्ण॒न् प्रति॒ प्रत्य॑गृह्ण॒न् थ्स सो॑ ऽगृह्ण॒न् प्रति॒ प्रत्य॑गृह्ण॒न् थ्सः । \newline
32. अ॒गृ॒ह्ण॒न् थ्स सो॑ ऽगृह्णन् नगृह्ण॒न् थ्स नि॑र्या॒सो नि॑र्या॒सः सो॑ ऽगृह्णन् नगृह्ण॒न् थ्स नि॑र्या॒सः । \newline
33. स नि॑र्या॒सो नि॑र्या॒सः स स नि॑र्या॒सो॑ ऽभव दभवन् निर्या॒सः स स नि॑र्या॒सो॑ ऽभवत् । \newline
34. नि॒र्या॒सो॑ ऽभव दभवन् निर्या॒सो नि॑र्या॒सो॑ ऽभव॒त् तस्मा॒त् तस्मा॑ दभवन् निर्या॒सो नि॑र्या॒सो॑ ऽभव॒त् तस्मा᳚त् । \newline
35. नि॒र्या॒स इति॑ निः - या॒सः । \newline
36. अ॒भ॒व॒त् तस्मा॒त् तस्मा॑ दभव दभव॒त् तस्मा᳚न् निर्या॒सस्य॑ निर्या॒सस्य॒ तस्मा॑ दभव दभव॒त् तस्मा᳚न् निर्या॒सस्य॑ । \newline
37. तस्मा᳚न् निर्या॒सस्य॑ निर्या॒सस्य॒ तस्मा॒त् तस्मा᳚न् निर्या॒सस्य॒ न न नि॑र्या॒सस्य॒ तस्मा॒त् तस्मा᳚न् निर्या॒सस्य॒ न । \newline
38. नि॒र्या॒सस्य॒ न न नि॑र्या॒सस्य॑ निर्या॒सस्य॒ नाश्य॑ मा॒श्य॑म् न नि॑र्या॒सस्य॑ निर्या॒सस्य॒ नाश्य᳚म् । \newline
39. नि॒र्या॒सस्येति॑ निः - या॒सस्य॑ । \newline
40. नाश्य॑ मा॒श्य॑म् न नाश्य॑म् ब्रह्मह॒त्यायै᳚ ब्रह्मह॒त्याया॑ आ॒श्य॑म् न नाश्य॑म् ब्रह्मह॒त्यायै᳚ । \newline
41. आ॒श्य॑म् ब्रह्मह॒त्यायै᳚ ब्रह्मह॒त्याया॑ आ॒श्य॑ मा॒श्य॑म् ब्रह्मह॒त्यायै॒ हि हि ब्र॑ह्मह॒त्याया॑ आ॒श्य॑ मा॒श्य॑म् ब्रह्मह॒त्यायै॒ हि । \newline
42. ब्र॒ह्म॒ह॒त्यायै॒ हि हि ब्र॑ह्मह॒त्यायै᳚ ब्रह्मह॒त्यायै॒ ह्ये॑ष ए॒ष हि ब्र॑ह्मह॒त्यायै᳚ ब्रह्मह॒त्यायै॒ ह्ये॑षः । \newline
43. ब्र॒ह्म॒ह॒त्याया॒ इति॑ ब्रह्म - ह॒त्यायै᳚ । \newline
44. ह्ये॑ष ए॒ष हि ह्ये॑ष वर्णो॒ वर्ण॑ ए॒ष हि ह्ये॑ष वर्णः॑ । \newline
45. ए॒ष वर्णो॒ वर्ण॑ ए॒ष ए॒ष वर्णो ऽथो॒ अथो॒ वर्ण॑ ए॒ष ए॒ष वर्णो ऽथो᳚ । \newline
46. वर्णो ऽथो॒ अथो॒ वर्णो॒ वर्णो ऽथो॒ खलु॒ खल्वथो॒ वर्णो॒ वर्णो ऽथो॒ खलु॑ । \newline
47. अथो॒ खलु॒ खल्वथो॒ अथो॒ खलु॒ यो यः खल्वथो॒ अथो॒ खलु॒ यः । \newline
48. अथो॒ इत्यथो᳚ । \newline
49. खलु॒ यो यः खलु॒ खलु॒ य ए॒वैव यः खलु॒ खलु॒ य ए॒व । \newline
50. य ए॒वैव यो य ए॒व लोहि॑तो॒ लोहि॑त ए॒व यो य ए॒व लोहि॑तः । \newline
51. ए॒व लोहि॑तो॒ लोहि॑त ए॒वैव लोहि॑तो॒ यो यो लोहि॑त ए॒वैव लोहि॑तो॒ यः । \newline
52. लोहि॑तो॒ यो यो लोहि॑तो॒ लोहि॑तो॒ यो वा॑ वा॒ यो लोहि॑तो॒ लोहि॑तो॒ यो वा᳚ । \newline
53. यो वा॑ वा॒ यो यो वा॒ ऽऽव्रश्च॑ना दा॒व्रश्च॑नाद् वा॒ यो यो वा॒ ऽऽव्रश्च॑नात् । \newline
54. वा॒ ऽऽव्रश्च॑ना दा॒व्रश्च॑नाद् वा वा॒ ऽऽव्रश्च॑नान् नि॒र्येष॑ति नि॒र्येष॑ त्या॒व्रश्च॑नाद् वा वा॒ ऽऽव्रश्च॑नान् नि॒र्येष॑ति । \newline
55. आ॒व्रश्च॑नान् नि॒र्येष॑ति नि॒र्येष॑ त्या॒व्रश्च॑ना दा॒व्रश्च॑नान् नि॒र्येष॑ति॒ तस्य॒ तस्य॑ नि॒र्येष॑ त्या॒व्रश्च॑ना दा॒व्रश्च॑नान् नि॒र्येष॑ति॒ तस्य॑ । \newline
56. आ॒व्रश्च॑ना॒दित्या᳚ - व्रश्च॑नात् । \newline
57. नि॒र्येष॑ति॒ तस्य॒ तस्य॑ नि॒र्येष॑ति नि॒र्येष॑ति॒ तस्य॒ न न तस्य॑ नि॒र्येष॑ति नि॒र्येष॑ति॒ तस्य॒ न । \newline
58. नि॒र्येष॒तीति॑ निः - येष॑ति । \newline
59. तस्य॒ न न तस्य॒ तस्य॒ नाश्य॑ मा॒श्य॑म् न तस्य॒ तस्य॒ नाश्य᳚म् । \newline
60. नाश्य॑ मा॒श्य॑म् न नाश्य॑म् काम॒म् काम॑ मा॒श्य॑म् न नाश्य॑म् काम᳚म् । \newline
61. आ॒श्य॑म् काम॒म् काम॑ मा॒श्य॑ मा॒श्य॑म् काम॑ म॒न्यस्या॒ न्यस्य॒ काम॑ मा॒श्य॑ मा॒श्य॑म् काम॑ म॒न्यस्य॑ । \newline
\pagebreak
\markright{ TS 2.5.1.5  \hfill https://www.vedavms.in \hfill}
\addcontentsline{toc}{section}{ TS 2.5.1.5 }
\section*{ TS 2.5.1.5 }

\textbf{TS 2.5.1.5 } \newline
\textbf{Samhita Paata} \newline

काम॑म॒न्यस्य॒सस्त्री॑षꣳसा॒द-मुपा॑सीदद॒स्यै ब्र॑ह्मह॒त्यायै॒ तृती॑यं॒ प्रति॑ गृह्णी॒तेति॒ ता अ॑ब्रुव॒न् वरं॑ ॅवृणामहा॒ ऋत्वि॑यात् प्र॒जां ॅवि॑न्दामहै॒ काम॒मा विज॑नितोः॒ सं भ॑वा॒मेति॒ तस्मा॒दृत्वि॑या॒थ् स्त्रियः॑ प्र॒जां ॅवि॑न्दन्ते॒ काम॒मा विज॑नितोः॒ संभ॑वन्ति॒ वारे॑वृतꣳ॒॒ ह्या॑सां॒ तृती॑यं ब्रह्मह॒त्यायै॒ प्रत्य॑गृह्ण॒न्थ् सा मल॑वद्वासा अभव॒त् तस्मा॒न्-मल॑वद्-वाससा॒ न संॅव॑देत॒ - [  ] \newline

\textbf{Pada Paata} \newline

काम᳚म् । अ॒न्यस्य॑ । सः । स्त्री॒षꣳ॒॒सा॒दमिति॑ स्त्री - सꣳ॒॒सा॒दम् । उपेति॑ । अ॒सी॒द॒त् । अ॒स्यै । ब्र॒ह्म॒ह॒त्याया॒ इति॑ ब्रह्म - ह॒त्यायै᳚ । तृती॑यम् । प्रतीति॑ । गृ॒ह्णी॒त॒ । इति॑ । ताः । अ॒ब्रु॒व॒न्न् । वर᳚म् । वृ॒णा॒म॒है॒ । ऋत्वि॑यात् । प्र॒जामिति॑ प्र - जाम् । वि॒न्दा॒म॒है॒ । काम᳚म् । एति॑ । विज॑नितो॒रिति॒ वि - ज॒नि॒तोः॒ । समिति॑ । भ॒वा॒म॒ । इति॑ । तस्मा᳚त् । ऋत्वि॑यात् । स्त्रियः॑ । प्र॒जामिति॑ प्र-जाम् । वि॒न्द॒न्ते॒ । काम᳚म् । एति॑ । विज॑नितो॒रिति॒ वि - ज॒नि॒तोः॒ । समिति॑ । भ॒व॒न्ति॒ । वारे॑वृत॒मिति॒ वारे᳚ - वृ॒त॒म् । हि । आ॒सा॒म् । तृती॑यम् । ब्र॒ह्म॒ह॒त्याया॒ इति॑ ब्रह्म - ह॒त्यायै᳚ । प्रतीति॑ । अ॒गृ॒ह्ण॒न्न् । सा । मल॑वद्वासा॒ इति॒ मल॑वत् - वा॒साः॒ । अ॒भ॒व॒त् । तस्मा᳚त् । मल॑वद्वास॒सेति॒ मल॑वत् - वा॒स॒सा॒ । न । समिति॑ । व॒दे॒त॒ ।  \newline


\textbf{Krama Paata} \newline

काम॑म॒न्यस्य॑ । अ॒न्यस्य॒ सः । स स्त्री॑षꣳसा॒दम् । स्त्री॒षꣳ॒॒सा॒दमुप॑ । स्त्री॒षꣳ॒॒सा॒दमिति॑ स्त्री - सꣳ॒॒सा॒दम् । उपा॑सीदत् । अ॒सी॒द॒द॒स्यै । अ॒स्यै ब्र॑ह्मह॒त्यायै᳚ । ब्र॒ह्म॒ह॒त्यायै॒ तृती॑यम् । ब्र॒ह्म॒ह॒त्याया॒ इति॑ ब्रह्म - ह॒त्यायै᳚ । तृती॑य॒म् प्रति॑ । प्रति॑ गृह्णीत । गृ॒ह्णी॒तेति॑ । इति॒ ताः । ता अ॑ब्रुवन्न् । अ॒ब्रु॒व॒न् वर᳚म् । वर॑म् ॅवृणामहै । वृ॒णा॒म॒हा॒ ऋत्वि॑यात् । ऋत्वि॑यात् प्र॒जाम् । प्र॒जाम् ॅवि॑न्दामहै । प्र॒जामिति॑ प्र - जाम् । वि॒न्दा॒म॒है॒ काम᳚म् । काम॒मा । आ विज॑नितोः । विज॑नितोः॒ सम् । विज॑नितो॒रिति॒ वि - ज॒नि॒तोः॒ । सम् भ॑वाम । भ॒वा॒मेति॑ । इति॒ तस्मा᳚त् । तस्मा॒दृत्वि॑यात् । ऋत्वि॑या॒थ् स्त्रियः॑ । स्त्रियः॑ प्र॒जाम् । प्र॒जाम् ॅवि॑न्दन्ते । प्र॒जामिति॑ प्र - जाम् । वि॒न्द॒न्ते॒ काम᳚म् । काम॒मा । आ विज॑नितोः । विज॑नितोः॒ सम् । विज॑नितो॒रिति॒ वि - ज॒नि॒तोः॒ । सम् भ॑वन्ति । भ॒व॒न्ति॒ वारे॑वृतम् । वारे॑वृतꣳ॒॒ हि । वारे॑वृत॒मिति॒ वारे᳚ - वृ॒त॒म् । ह्या॑साम् । आ॒सा॒म् तृती॑यम् । तृती॑यम् ब्रह्मह॒त्यायै᳚ । ब्र॒ह्म॒ह॒त्यायै॒ प्रति॑ । ब्र॒ह्म॒ह॒त्याया॒ इति॑ ब्रह्म - ह॒त्यायै᳚ । प्रत्य॑गृह्णन्न् । अ॒गृ॒ह्ण॒न्थ् सा । सा मल॑वद्वासाः । मल॑वद्वासा अभवत् । मल॑वद्वासा॒ इति॒ मल॑वत् - वा॒साः॒ । अ॒भ॒व॒त् तस्मा᳚त् । तस्मा॒न् मल॑वद्वाससा । मल॑वद्वाससा॒ न । मल॑वद्वास॒सेति॒ मल॑वत् - वा॒स॒सा॒ । न सम् । सम् ॅव॑देत । व॒दे॒त॒ न \newline

\textbf{Jatai Paata} \newline

1. काम॑ म॒न्यस्या॒ न्यस्य॒ काम॒म् काम॑ म॒न्यस्य॑ । \newline
2. अ॒न्यस्य॒ स सो᳚ ऽन्यस्या॒ न्यस्य॒ सः । \newline
3. स स्त्री॑षꣳसा॒दꣳ स्त्री॑षꣳसा॒दꣳ स स स्त्री॑षꣳसा॒दम् । \newline
4. स्त्री॒षꣳ॒॒सा॒द मुपोप॑ स्त्रीषꣳसा॒दꣳ स्त्री॑षꣳसा॒द मुप॑ । \newline
5. स्त्री॒षꣳ॒॒सा॒दमिति॑ स्त्री - सꣳ॒॒सा॒दम् । \newline
6. उपा॑सीद दसीद॒ दुपोपा॑सीदत् । \newline
7. अ॒सी॒द॒ द॒स्या अ॒स्या अ॑सीद दसीद द॒स्यै । \newline
8. अ॒स्यै ब्र॑ह्मह॒त्यायै᳚ ब्रह्मह॒त्याया॑ अ॒स्या अ॒स्यै ब्र॑ह्मह॒त्यायै᳚ । \newline
9. ब्र॒ह्म॒ह॒त्यायै॒ तृती॑य॒म् तृती॑यम् ब्रह्मह॒त्यायै᳚ ब्रह्मह॒त्यायै॒ तृती॑यम् । \newline
10. ब्र॒ह्म॒ह॒त्याया॒ इति॑ ब्रह्म - ह॒त्यायै᳚ । \newline
11. तृती॑य॒म् प्रति॒ प्रति॒ तृती॑य॒म् तृती॑य॒म् प्रति॑ । \newline
12. प्रति॑ गृह्णीत गृह्णीत॒ प्रति॒ प्रति॑ गृह्णीत । \newline
13. गृ॒ह्णी॒ते तीति॑ गृह्णीत गृह्णी॒ते ति॑ । \newline
14. इति॒ ता स्ता इतीति॒ ताः । \newline
15. ता अ॑ब्रुवन् नब्रुव॒न् ता स्ता अ॑ब्रुवन्न् । \newline
16. अ॒ब्रु॒व॒न्॒. वरं॒ ॅवर॑ मब्रुवन् नब्रुव॒न्॒. वर᳚म् । \newline
17. वरं॑ ॅवृणामहै वृणामहै॒ वरं॒ ॅवरं॑ ॅवृणामहै । \newline
18. वृ॒णा॒म॒हा॒ ऋत्वि॑या॒ दृत्वि॑याद् वृणामहै वृणामहा॒ ऋत्वि॑यात् । \newline
19. ऋत्वि॑यात् प्र॒जाम् प्र॒जा मृत्वि॑या॒ दृत्वि॑यात् प्र॒जाम् । \newline
20. प्र॒जां ॅवि॑न्दामहै विन्दामहै प्र॒जाम् प्र॒जां ॅवि॑न्दामहै । \newline
21. प्र॒जामिति॑ प्र - जाम् । \newline
22. वि॒न्दा॒म॒है॒ काम॒म् कामं॑ ॅविन्दामहै विन्दामहै॒ काम᳚म् । \newline
23. काम॒ मा काम॒म् काम॒ मा । \newline
24. आ विज॑नितो॒र् विज॑नितो॒रा विज॑नितोः । \newline
25. विज॑नितोः॒ सꣳ सं ॅविज॑नितो॒र् विज॑नितोः॒ सम् । \newline
26. विज॑नितो॒रिति॒ वि - ज॒नि॒तोः॒ । \newline
27. सम् भ॑वाम भवाम॒ सꣳ सम् भ॑वाम । \newline
28. भ॒वा॒मे तीति॑ भवाम भवा॒मे ति॑ । \newline
29. इति॒ तस्मा॒त् तस्मा॒ दितीति॒ तस्मा᳚त् । \newline
30. तस्मा॒ दृत्वि॑या॒ दृत्वि॑या॒त् तस्मा॒त् तस्मा॒ दृत्वि॑यात् । \newline
31. ऋत्वि॑या॒थ् स्त्रियः॒ स्त्रिय॒ ऋत्वि॑या॒ दृत्वि॑या॒थ् स्त्रियः॑ । \newline
32. स्त्रियः॑ प्र॒जाम् प्र॒जाꣳ स्त्रियः॒ स्त्रियः॑ प्र॒जाम् । \newline
33. प्र॒जां ॅवि॑न्दन्ते विन्दन्ते प्र॒जाम् प्र॒जां ॅवि॑न्दन्ते । \newline
34. प्र॒जामिति॑ प्र - जाम् । \newline
35. वि॒न्द॒न्ते॒ काम॒म् कामं॑ ॅविन्दन्ते विन्दन्ते॒ काम᳚म् । \newline
36. काम॒ मा काम॒म् काम॒ मा । \newline
37. आ विज॑नितो॒र् विज॑नितो॒रा विज॑नितोः । \newline
38. विज॑नितोः॒ सꣳ सं ॅविज॑नितो॒र् विज॑नितोः॒ सम् । \newline
39. विज॑नितो॒रिति॒ वि - ज॒नि॒तोः॒ । \newline
40. सम् भ॑वन्ति भवन्ति॒ सꣳ सम् भ॑वन्ति । \newline
41. भ॒व॒न्ति॒ वारे॑वृतं॒ ॅवारे॑वृतम् भवन्ति भवन्ति॒ वारे॑वृतम् । \newline
42. वारे॑वृतꣳ॒॒ हि हि वारे॑वृतं॒ ॅवारे॑वृतꣳ॒॒ हि । \newline
43. वारे॑वृत॒मिति॒ वारे᳚ - वृ॒त॒म् । \newline
44. ह्या॑सा मासाꣳ॒॒ हि ह्या॑साम् । \newline
45. आ॒सा॒म् तृती॑य॒म् तृती॑य मासा मासा॒म् तृती॑यम् । \newline
46. तृती॑यम् ब्रह्मह॒त्यायै᳚ ब्रह्मह॒त्यायै॒ तृती॑य॒म् तृती॑यम् ब्रह्मह॒त्यायै᳚ । \newline
47. ब्र॒ह्म॒ह॒त्यायै॒ प्रति॒ प्रति॑ ब्रह्मह॒त्यायै᳚ ब्रह्मह॒त्यायै॒ प्रति॑ । \newline
48. ब्र॒ह्म॒ह॒त्याया॒ इति॑ ब्रह्म - ह॒त्यायै᳚ । \newline
49. प्रत्य॑गृह्णन् नगृह्ण॒न् प्रति॒ प्रत्य॑गृह्णन्न् । \newline
50. अ॒गृ॒ह्ण॒न् थ्सा सा ऽगृ॑ह्णन् नगृह्ण॒न् थ्सा । \newline
51. सा मल॑वद्वासा॒ मल॑वद्वासाः॒ सा सा मल॑वद्वासाः । \newline
52. मल॑वद्वासा अभवदभव॒न् मल॑वद्वासा॒ मल॑वद्वासा अभवत् । \newline
53. मल॑वद्वासा॒ इति॒ मल॑वत् - वा॒साः॒ । \newline
54. अ॒भ॒व॒त् तस्मा॒त् तस्मा॑ दभव दभव॒त् तस्मा᳚त् । \newline
55. तस्मा॒न् मल॑वद्वाससा॒ मल॑वद्वाससा॒ तस्मा॒त् तस्मा॒न् मल॑वद्वाससा । \newline
56. मल॑वद्वाससा॒ न न मल॑वद्वाससा॒ मल॑वद्वाससा॒ न । \newline
57. मल॑वद्वास॒सेति॒ मल॑वत् - वा॒स॒सा॒ । \newline
58. न सꣳ सम् न न सम् । \newline
59. सं ॅव॑देत वदेत॒ सꣳ सं ॅव॑देत । \newline
60. व॒दे॒त॒ न न व॑देत वदेत॒ न । \newline

\textbf{Ghana Paata } \newline

1. काम॑ म॒न्यस्या॒ न्यस्य॒ काम॒म् काम॑ म॒न्यस्य॒ स सो᳚ ऽन्यस्य॒ काम॒म् काम॑ म॒न्यस्य॒ सः । \newline
2. अ॒न्यस्य॒ स सो᳚ ऽन्यस्या॒न्यस्य॒ स स्त्री॑षꣳसा॒दꣳ स्त्री॑षꣳसा॒दꣳ सो᳚ ऽन्यस्या॒न्यस्य॒ स स्त्री॑षꣳसा॒दम् । \newline
3. स स्त्री॑षꣳसा॒दꣳ स्त्री॑षꣳसा॒दꣳ स स स्त्री॑षꣳसा॒द मुपोप॑ स्त्रीषꣳसा॒दꣳ स स स्त्री॑षꣳसा॒द मुप॑ । \newline
4. स्त्री॒षꣳ॒॒सा॒द मुपोप॑ स्त्रीषꣳसा॒दꣳ स्त्री॑षꣳसा॒द मुपा॑सीद दसीद॒ दुप॑ स्त्रीषꣳसा॒दꣳ स्त्री॑षꣳसा॒द मुपा॑सीदत् । \newline
5. स्त्री॒षꣳ॒॒सा॒दमिति॑ स्त्री - सꣳ॒॒सा॒दम् । \newline
6. उपा॑सीद दसीद॒ दुपोपा॑सीद द॒स्या अ॒स्या अ॑सीद॒ दुपोपा॑सीद द॒स्यै । \newline
7. अ॒सी॒द॒ द॒स्या अ॒स्या अ॑सीद दसीद द॒स्यै ब्र॑ह्मह॒त्यायै᳚ ब्रह्मह॒त्याया॑ अ॒स्या अ॑सीद दसीद द॒स्यै ब्र॑ह्मह॒त्यायै᳚ । \newline
8. अ॒स्यै ब्र॑ह्मह॒त्यायै᳚ ब्रह्मह॒त्याया॑ अ॒स्या अ॒स्यै ब्र॑ह्मह॒त्यायै॒ तृती॑य॒म् तृती॑यम् ब्रह्मह॒त्याया॑ अ॒स्या अ॒स्यै ब्र॑ह्मह॒त्यायै॒ तृती॑यम् । \newline
9. ब्र॒ह्म॒ह॒त्यायै॒ तृती॑य॒म् तृती॑यम् ब्रह्मह॒त्यायै᳚ ब्रह्मह॒त्यायै॒ तृती॑य॒म् प्रति॒ प्रति॒ तृती॑यम् ब्रह्मह॒त्यायै᳚ ब्रह्मह॒त्यायै॒ तृती॑य॒म् प्रति॑ । \newline
10. ब्र॒ह्म॒ह॒त्याया॒ इति॑ ब्रह्म - ह॒त्यायै᳚ । \newline
11. तृती॑य॒म् प्रति॒ प्रति॒ तृती॑य॒म् तृती॑य॒म् प्रति॑ गृह्णीत गृह्णीत॒ प्रति॒ तृती॑य॒म् तृती॑य॒म् प्रति॑ गृह्णीत । \newline
12. प्रति॑ गृह्णीत गृह्णीत॒ प्रति॒ प्रति॑ गृह्णी॒ते तीति॑ गृह्णीत॒ प्रति॒ प्रति॑ गृह्णी॒ते ति॑ । \newline
13. गृ॒ह्णी॒ते तीति॑ गृह्णीत गृह्णी॒ते ति॒ तास्ता इति॑ गृह्णीत गृह्णी॒ते ति॒ ताः । \newline
14. इति॒ ता स्ता इतीति॒ ता अ॑ब्रुवन् नब्रुव॒न् ता इतीति॒ ता अ॑ब्रुवन्न् । \newline
15. ता अ॑ब्रुवन् नब्रुव॒न् ता स्ता अ॑ब्रुव॒न् वरं॒ ॅवर॑ मब्रुव॒न् तास्ता अ॑ब्रुव॒न् वर᳚म् । \newline
16. अ॒ब्रु॒व॒न् वरं॒ ॅवर॑ मब्रुवन् नब्रुव॒न् वरं॑ ॅवृणामहै वृणामहै॒ वर॑ मब्रुवन् नब्रुव॒न् वरं॑ ॅवृणामहै । \newline
17. वरं॑ ॅवृणामहै वृणामहै॒ वरं॒ ॅवरं॑ ॅवृणामहा॒ ऋत्वि॑या॒ दृत्वि॑याद् वृणामहै॒ वरं॒ ॅवरं॑ ॅवृणामहा॒ ऋत्वि॑यात् । \newline
18. वृ॒णा॒म॒हा॒ ऋत्वि॑या॒ दृत्वि॑याद् वृणामहै वृणामहा॒ ऋत्वि॑यात् प्र॒जाम् प्र॒जा मृत्वि॑याद् वृणामहै वृणामहा॒ ऋत्वि॑यात् प्र॒जाम् । \newline
19. ऋत्वि॑यात् प्र॒जाम् प्र॒जा मृत्वि॑या॒ दृत्वि॑यात् प्र॒जां ॅवि॑न्दामहै विन्दामहै प्र॒जा मृत्वि॑या॒ दृत्वि॑यात् प्र॒जां ॅवि॑न्दामहै । \newline
20. प्र॒जां ॅवि॑न्दामहै विन्दामहै प्र॒जाम् प्र॒जां ॅवि॑न्दामहै॒ काम॒म् कामं॑ ॅविन्दामहै प्र॒जाम् प्र॒जां ॅवि॑न्दामहै॒ काम᳚म् । \newline
21. प्र॒जामिति॑ प्र - जाम् । \newline
22. वि॒न्दा॒म॒है॒ काम॒म् कामं॑ ॅविन्दामहै विन्दामहै॒ काम॒ मा कामं॑ ॅविन्दामहै विन्दामहै॒ काम॒ मा । \newline
23. काम॒ मा काम॒म् काम॒ मा विज॑नितो॒र् विज॑नितो॒रा काम॒म् काम॒ मा विज॑नितोः । \newline
24. आ विज॑नितो॒र् विज॑नितो॒रा विज॑नितोः॒ सꣳ सं ॅविज॑नितो॒रा विज॑नितोः॒ सम् । \newline
25. विज॑नितोः॒ सꣳ सं ॅविज॑नितो॒र् विज॑नितोः॒ सम् भ॑वाम भवाम॒ सं ॅविज॑नितो॒र् विज॑नितोः॒ सम् भ॑वाम । \newline
26. विज॑नितो॒रिति॒ वि - ज॒नि॒तोः॒ । \newline
27. सम् भ॑वाम भवाम॒ सꣳ सम् भ॑वा॒मे तीति॑ भवाम॒ सꣳ सम् भ॑वा॒मे ति॑ । \newline
28. भ॒वा॒मे तीति॑ भवाम भवा॒मे ति॒ तस्मा॒त् तस्मा॒दिति॑ भवाम भवा॒मे ति॒ तस्मा᳚त् । \newline
29. इति॒ तस्मा॒त् तस्मा॒ दितीति॒ तस्मा॒ दृत्वि॑या॒ दृत्वि॑या॒त् तस्मा॒ दितीति॒ तस्मा॒ दृत्वि॑यात् । \newline
30. तस्मा॒ दृत्वि॑या॒ दृत्वि॑या॒त् तस्मा॒त् तस्मा॒ दृत्वि॑या॒थ् स्त्रियः॒ स्त्रिय॒ ऋत्वि॑या॒त् तस्मा॒त् तस्मा॒ दृत्वि॑या॒थ् स्त्रियः॑ । \newline
31. ऋत्वि॑या॒थ् स्त्रियः॒ स्त्रिय॒ ऋत्वि॑या॒ दृत्वि॑या॒थ् स्त्रियः॑ प्र॒जाम् प्र॒जाꣳ स्त्रिय॒ ऋत्वि॑या॒ दृत्वि॑या॒थ् स्त्रियः॑ प्र॒जाम् । \newline
32. स्त्रियः॑ प्र॒जाम् प्र॒जाꣳ स्त्रियः॒ स्त्रियः॑ प्र॒जां ॅवि॑न्दन्ते विन्दन्ते प्र॒जाꣳ स्त्रियः॒ स्त्रियः॑ प्र॒जां ॅवि॑न्दन्ते । \newline
33. प्र॒जां ॅवि॑न्दन्ते विन्दन्ते प्र॒जाम् प्र॒जां ॅवि॑न्दन्ते॒ काम॒म् कामं॑ ॅविन्दन्ते प्र॒जाम् प्र॒जां ॅवि॑न्दन्ते॒ काम᳚म् । \newline
34. प्र॒जामिति॑ प्र - जाम् । \newline
35. वि॒न्द॒न्ते॒ काम॒म् कामं॑ ॅविन्दन्ते विन्दन्ते॒ काम॒ मा कामं॑ ॅविन्दन्ते विन्दन्ते॒ काम॒ मा । \newline
36. काम॒ मा काम॒म् काम॒ मा विज॑नितो॒र् विज॑नितो॒रा काम॒म् काम॒ मा विज॑नितोः । \newline
37. आ विज॑नितो॒र् विज॑नितो॒रा विज॑नितोः॒ सꣳ सं ॅविज॑नितो॒रा विज॑नितोः॒ सम् । \newline
38. विज॑नितोः॒ सꣳ सं ॅविज॑नितो॒र् विज॑नितोः॒ सम् भ॑वन्ति भवन्ति॒ सं ॅविज॑नितो॒र् विज॑नितोः॒ सम् भ॑वन्ति । \newline
39. विज॑नितो॒रिति॒ वि - ज॒नि॒तोः॒ । \newline
40. सम् भ॑वन्ति भवन्ति॒ सꣳ सम् भ॑वन्ति॒ वारे॑वृतं॒ ॅवारे॑वृतम् भवन्ति॒ सꣳ सम् भ॑वन्ति॒ वारे॑वृतम् । \newline
41. भ॒व॒न्ति॒ वारे॑वृतं॒ ॅवारे॑वृतम् भवन्ति भवन्ति॒ वारे॑वृतꣳ॒॒ हि हि वारे॑वृतम् भवन्ति भवन्ति॒ वारे॑वृतꣳ॒॒ हि । \newline
42. वारे॑वृतꣳ॒॒ हि हि वारे॑वृतं॒ ॅवारे॑वृतꣳ॒॒ ह्या॑सा मासाꣳ॒॒ हि वारे॑वृतं॒ ॅवारे॑वृतꣳ॒॒ ह्या॑साम् । \newline
43. वारे॑वृत॒मिति॒ वारे᳚ - वृ॒त॒म् । \newline
44. ह्या॑सा मासाꣳ॒॒ हि ह्या॑सा॒म् तृती॑य॒म् तृती॑य मासाꣳ॒॒ हि ह्या॑सा॒म् तृती॑यम् । \newline
45. आ॒सा॒म् तृती॑य॒म् तृती॑य मासा मासा॒म् तृती॑यम् ब्रह्मह॒त्यायै᳚ ब्रह्मह॒त्यायै॒ तृती॑य मासा मासा॒म् तृती॑यम् ब्रह्मह॒त्यायै᳚ । \newline
46. तृती॑यम् ब्रह्मह॒त्यायै᳚ ब्रह्मह॒त्यायै॒ तृती॑य॒म् तृती॑यम् ब्रह्मह॒त्यायै॒ प्रति॒ प्रति॑ ब्रह्मह॒त्यायै॒ तृती॑य॒म् तृती॑यम् ब्रह्मह॒त्यायै॒ प्रति॑ । \newline
47. ब्र॒ह्म॒ह॒त्यायै॒ प्रति॒ प्रति॑ ब्रह्मह॒त्यायै᳚ ब्रह्मह॒त्यायै॒ प्रत्य॑गृह्णन् नगृह्ण॒न् प्रति॑ ब्रह्मह॒त्यायै᳚ ब्रह्मह॒त्यायै॒ प्रत्य॑गृह्णन्न् । \newline
48. ब्र॒ह्म॒ह॒त्याया॒ इति॑ ब्रह्म - ह॒त्यायै᳚ । \newline
49. प्रत्य॑गृह्णन् नगृह्ण॒न् प्रति॒ प्रत्य॑गृह्ण॒न् थ्सा सा ऽगृ॑ह्ण॒न् प्रति॒ प्रत्य॑गृह्ण॒न् थ्सा । \newline
50. अ॒गृ॒ह्ण॒न् थ्सा सा ऽगृ॑ह्णन् नगृह्ण॒न् थ्सा मल॑वद्वासा॒ मल॑वद्वासाः॒ सा ऽगृ॑ह्णन् नगृह्ण॒न् थ्सा मल॑वद्वासाः । \newline
51. सा मल॑वद्वासा॒ मल॑वद्वासाः॒ सा सा मल॑वद्वासा अभव दभव॒न् मल॑वद्वासाः॒ सा सा मल॑वद्वासा अभवत् । \newline
52. मल॑वद्वासा अभव दभव॒न् मल॑वद्वासा॒ मल॑वद्वासा अभव॒त् तस्मा॒त् तस्मा॑ दभव॒न् मल॑वद्वासा॒ मल॑वद्वासा अभव॒त् तस्मा᳚त् । \newline
53. मल॑वद्वासा॒ इति॒ मल॑वत् - वा॒साः॒ । \newline
54. अ॒भ॒व॒त् तस्मा॒त् तस्मा॑ दभव दभव॒त् तस्मा॒न् मल॑वद्वाससा॒ मल॑वद्वाससा॒ तस्मा॑ दभव दभव॒त् तस्मा॒न् मल॑वद्वाससा । \newline
55. तस्मा॒न् मल॑वद्वाससा॒ मल॑वद्वाससा॒ तस्मा॒त् तस्मा॒न् मल॑वद्वाससा॒ न न मल॑वद्वाससा॒ तस्मा॒त् तस्मा॒न् मल॑वद्वाससा॒ न । \newline
56. मल॑वद्वाससा॒ न न मल॑वद्वाससा॒ मल॑वद्वाससा॒ न सꣳ सम् न मल॑वद्वाससा॒ मल॑वद्वाससा॒ न सम् । \newline
57. मल॑वद्वास॒सेति॒ मल॑वत् - वा॒स॒सा॒ । \newline
58. न सꣳ सम् न न सं ॅव॑देत वदेत॒ सम् न न सं ॅव॑देत । \newline
59. सं ॅव॑देत वदेत॒ सꣳ सं ॅव॑देत॒ न न व॑देत॒ सꣳ सं ॅव॑देत॒ न । \newline
60. व॒दे॒त॒ न न व॑देत वदेत॒ न स॒ह स॒ह न व॑देत वदेत॒ न स॒ह । \newline
\pagebreak
\markright{ TS 2.5.1.6  \hfill https://www.vedavms.in \hfill}
\addcontentsline{toc}{section}{ TS 2.5.1.6 }
\section*{ TS 2.5.1.6 }

\textbf{TS 2.5.1.6 } \newline
\textbf{Samhita Paata} \newline

न स॒हाऽऽसी॑त॒ नास्या॒ अन्न॑मद्याद्-ब्रह्मह॒त्यायै॒ ह्ये॑षा वर्णं॑ प्रति॒मुच्या ऽऽस्तेऽथो॒ खल्वा॑हुर॒भ्यञ्ज॑नं॒ ॅवाव स्त्रि॒या अन्न॑म॒भ्यञ्ज॑नमे॒व न प्र॑ति॒गृह्यं॒ काम॑म॒न्यदिति॒ यां मल॑वद्-वाससꣳ स॒भंव॑न्ति॒ यस्ततो॒ जाय॑ते॒ सो॑ऽभिश॒स्तो यामर॑ण्ये॒ तस्यै᳚ स्ते॒नो यां परा॑चीं॒ तस्यै᳚ ह्रीतमु॒ख्य॑पग॒ल्भो या स्नाति॒ तस्या॑ अ॒फ्सु मारु॑को॒ या - [  ] \newline

\textbf{Pada Paata} \newline

न । स॒ह । आ॒सी॒त॒ । न । अ॒स्याः॒ । अन्न᳚म् । अ॒द्या॒त् । ब्र॒ह्म॒ह॒त्याया॒ इति॑ ब्रह्म - ह॒त्यायै᳚ । हि । ए॒षा । वर्ण᳚म् । प्र॒ति॒मुच्येति॑ प्रति-मुच्य॑ । आस्ते᳚ । अथो॒ इति॑ । खलु॑ । आ॒हुः॒ । अ॒भ्यञ्ज॑न॒मित्य॑भि - अञ्ज॑नम् । वाव । स्त्रि॒याः । अन्न᳚म् । अ॒भ्यञ्ज॑न॒मित्य॑भि - अञ्ज॑नम् । ए॒व । न । प्र॒ति॒गृह्य॒मिति॑ प्रति - गृह्य᳚म् । काम᳚म् । अ॒न्यत् । इति॑ । याम् । मल॑वद्वासस॒मिति॒ मल॑वत् - वा॒स॒स॒म् । स॒भंव॒न्तीति॑ सं - भव॑न्ति । यः । ततः॑ । जाय॑ते । सः । अ॒भि॒श॒स्त इत्य॑भि-श॒स्तः । याम् । अर॑ण्ये । तस्यै᳚ । स्ते॒नः । याम् । परा॑चीम् । तस्यै᳚ । ह्री॒त॒मु॒खीति॑ ह्रीत - मु॒खी । अ॒प॒ग॒ल्भ इत्य॑प - ग॒ल्भः । या । स्नाति॑ । तस्याः᳚ । अ॒फ्सित्य॑प् - सु । मारु॑कः । या ।  \newline


\textbf{Krama Paata} \newline

न स॒ह । स॒हासी॑त । आ॒सी॒त॒ न । नास्याः᳚ । अ॒स्या॒ अन्न᳚म् । अन्न॑मद्यात् । अ॒द्या॒द् ब्र॒ह्म॒ह॒त्यायै᳚ । ब्र॒ह्म॒ह॒त्यायै॒ हि । ब्र॒ह्म॒ह॒त्याया॒ इति॑ ब्रह्म - ह॒त्यायै᳚ । ह्ये॑षा । ए॒षा वर्ण᳚म् । वर्ण॑म् प्रति॒मुच्य॑ । प्र॒ति॒मुच्यास्ते᳚ । प्र॒ति॒मुच्येति॑ प्रति - मुच्य॑ । आस्तेऽथो᳚ । अथो॒ खलु॑ । अथो॒ इत्यथो᳚ । खल्वा॑हुः । 
आ॒हु॒र॒भ्यञ्ज॑नम् । अ॒भ्यञ्ज॑न॒म् ॅवाव । अ॒भ्यञ्ज॑न॒मित्य॑भि - अञ्ज॑नम् । वाव स्त्रि॒याः । स्त्रि॒या अन्न᳚म् । अन्न॑म॒भ्यञ्ज॑नम् । अ॒भ्यञ्ज॑नमे॒व । अ॒भ्यञ्ज॑न॒मित्य॑भि - अञ्ज॑नम् । ए॒व न । 

न प्र॑ति॒गृह्य᳚म् । प्र॒ति॒गृह्य॒म् काम᳚म् । प्र॒ति॒गृह्य॒मिति॑ प्रति - गृह्य᳚म् । काम॑म॒न्यत् । अ॒न्यदिति॑ । इति॒ याम् । याम् मल॑वद्वाससम् । मल॑वद्वाससꣳ स॒म्भव॑न्ति । मल॑वद्वासस॒मिति॒ मल॑वत् - वा॒स॒स॒म् । स॒म्भव॑न्ति॒ यः । स॒म्भव॒न्तीति॑ सम् - भव॑न्ति । यस्ततः॑ । ततो॒ जाय॑ते । जाय॑ते॒ सः । सो॑ऽभिश॒स्तः । अ॒भि॒श॒स्तो याम् । अ॒भि॒श॒स्त इत्य॑भि - श॒स्तः । यामर॑ण्ये । अर॑ण्ये॒ तस्यै᳚ । तस्यै᳚ स्ते॒नः । स्ते॒नो याम् । याम् परा॑चीम् । परा॑ची॒म् तस्यै᳚ । तस्यै᳚ ह्रीतमु॒खी । ह्री॒त॒मु॒ख्य॑पग॒ल्भः । ह्री॒त॒मु॒खीति॑ ह्रीत - मु॒खी । अ॒प॒ग॒ल्भो या । अ॒प॒ग॒ल्भ इत्य॑प - ग॒ल्भः । या स्नाति॑ । स्नाति॒ तस्याः᳚ । तस्या॑ अ॒फ्सु । अ॒फ्सु मारु॑कः । अ॒फ्स्वित्य॑प् - सु । मारु॑को॒ या । या ऽभ्य॒ङ्क्ते \newline

\textbf{Jatai Paata} \newline

1. न स॒ह स॒ह न न स॒ह । \newline
2. स॒हासी॑ तासीत स॒ह स॒हासी॑त । \newline
3. आ॒सी॒त॒ न नासी॑ तासीत॒ न । \newline
4. नास्या॑ अस्या॒ न नास्याः᳚ । \newline
5. अ॒स्या॒ अन्न॒ मन्न॑ मस्या अस्या॒ अन्न᳚म् । \newline
6. अन्न॑ मद्या दद्या॒ दन्न॒ मन्न॑ मद्यात् । \newline
7. अ॒द्या॒द् ब्र॒ह्म॒ह॒त्यायै᳚ ब्रह्मह॒त्याया॑ अद्या दद्याद् ब्रह्मह॒त्यायै᳚ । \newline
8. ब्र॒ह्म॒ह॒त्यायै॒ हि हि ब्र॑ह्मह॒त्यायै᳚ ब्रह्मह॒त्यायै॒ हि । \newline
9. ब्र॒ह्म॒ह॒त्याया॒ इति॑ ब्रह्म - ह॒त्यायै᳚ । \newline
10. ह्ये॑षैषा हि ह्ये॑षा । \newline
11. ए॒षा वर्णं॒ ॅवर्ण॑ मे॒षैषा वर्ण᳚म् । \newline
12. वर्ण॑म् प्रति॒मुच्य॑ प्रति॒मुच्य॒ वर्णं॒ ॅवर्ण॑म् प्रति॒मुच्य॑ । \newline
13. प्र॒ति॒मुच्यास्त॒ आस्ते᳚ प्रति॒मुच्य॑ प्रति॒मुच्यास्ते᳚ । \newline
14. प्र॒ति॒मुच्येति॑ प्रति - मुच्य॑ । \newline
15. आस्ते ऽथो॒ अथो॒ आस्त॒ आस्ते ऽथो᳚ । \newline
16. अथो॒ खलु॒ खल्वथो॒ अथो॒ खलु॑ । \newline
17. अथो॒ इत्यथो᳚ । \newline
18. खल्वा॑हु राहुः॒ खलु॒ खल्वा॑हुः । \newline
19. आ॒हु॒ र॒भ्यञ्ज॑न म॒भ्यञ्ज॑न माहु राहु र॒भ्यञ्ज॑नम् । \newline
20. अ॒भ्यञ्ज॑नं॒ ॅवाव वावा भ्यञ्ज॑न म॒भ्यञ्ज॑नं॒ ॅवाव । \newline
21. अ॒भ्यञ्ज॑न॒मित्य॑भि - अञ्ज॑नम् । \newline
22. वाव स्त्रि॒याः स्त्रि॒या वाव वाव स्त्रि॒याः । \newline
23. स्त्रि॒या अन्न॒ मन्नꣳ॑ स्त्रि॒याः स्त्रि॒या अन्न᳚म् । \newline
24. अन्न॑ म॒भ्यञ्ज॑न म॒भ्यञ्ज॑न॒ मन्न॒ मन्न॑ म॒भ्यञ्ज॑नम् । \newline
25. अ॒भ्यञ्ज॑न मे॒वैवा भ्यञ्ज॑न म॒भ्यञ्ज॑न मे॒व । \newline
26. अ॒भ्यञ्ज॑न॒मित्य॑भि - अञ्ज॑नम् । \newline
27. ए॒व न नैवैव न । \newline
28. न प्र॑ति॒गृह्य॑म् प्रति॒गृह्य॒न्न न प्र॑ति॒गृह्य᳚म् । \newline
29. प्र॒ति॒गृह्य॒म् काम॒म् काम॑म् प्रति॒गृह्य॑म् प्रति॒गृह्य॒म् काम᳚म् । \newline
30. प्र॒ति॒गृह्य॒मिति॑ प्रति - गृह्य᳚म् । \newline
31. काम॑ म॒न्य द॒न्यत् काम॒म् काम॑ म॒न्यत् । \newline
32. अ॒न्य दिती त्य॒न्य द॒न्य दिति॑ । \newline
33. इति॒ यां ॅया मितीति॒ याम् । \newline
34. याम् मल॑वद्वासस॒म् मल॑वद्वाससं॒ ॅयां ॅयाम् मल॑वद्वाससम् । \newline
35. मल॑वद्वाससꣳ स॒म्भव॑न्ति स॒म्भव॑न्ति॒ मल॑वद्वासस॒म् मल॑वद्वाससꣳ स॒म्भव॑न्ति । \newline
36. मल॑वद्वासस॒मिति॒ मल॑वत् - वा॒स॒स॒म् । \newline
37. स॒म्भव॑न्ति॒ यो यः स॒म्भव॑न्ति स॒म्भव॑न्ति॒ यः । \newline
38. स॒म्भव॒न्तीति॑ सं - भव॑न्ति । \newline
39. य स्तत॒ स्ततो॒ यो य स्ततः॑ । \newline
40. ततो॒ जाय॑ते॒ जाय॑ते॒ तत॒स्ततो॒ जाय॑ते । \newline
41. जाय॑ते॒ स स जाय॑ते॒ जाय॑ते॒ सः । \newline
42. सो॑ ऽभिश॒स्तो॑ ऽभिश॒स्तः स सो॑ ऽभिश॒स्तः । \newline
43. अ॒भि॒श॒स्तो यां ॅया म॑भिश॒स्तो॑ ऽभिश॒स्तो याम् । \newline
44. अ॒भि॒श॒स्त इत्य॑भि - श॒स्तः । \newline
45. या मर॒ण्ये ऽर॑ण्ये॒ यां ॅया मर॑ण्ये । \newline
46. अर॑ण्ये॒ तस्यै॒ तस्या॒ अर॒ण्ये ऽर॑ण्ये॒ तस्यै᳚ । \newline
47. तस्यै᳚ स्ते॒नः स्ते॒न स्तस्यै॒ तस्यै᳚ स्ते॒नः । \newline
48. स्ते॒नो यां ॅयाꣳ स्ते॒नः स्ते॒नो याम् । \newline
49. याम् परा॑ची॒म् परा॑चीं॒ ॅयां ॅयाम् परा॑चीम् । \newline
50. परा॑ची॒म् तस्यै॒ तस्यै॒ परा॑ची॒म् परा॑ची॒म् तस्यै᳚ । \newline
51. तस्यै᳚ ह्रीतमु॒खी ह्री॑तमु॒खी तस्यै॒ तस्यै᳚ ह्रीतमु॒खी । \newline
52. ह्री॒त॒मु॒ ख्य॑पग॒ल्भो॑ ऽपग॒ल्भो ह्री॑तमु॒खी ह्री॑तमु॒ ख्य॑पग॒ल्भः । \newline
53. ह्री॒त॒मु॒खीति॑ ह्रीत - मु॒खी । \newline
54. अ॒प॒ग॒ल्भो या या ऽप॑ग॒ल्भो॑ ऽपग॒ल्भो या । \newline
55. अ॒प॒ग॒ल्भ इत्य॑प - ग॒ल्भः । \newline
56. या स्नाति॒ स्नाति॒ या या स्नाति॑ । \newline
57. स्नाति॒ तस्या॒ स्तस्याः॒ स्नाति॒ स्नाति॒ तस्याः᳚ । \newline
58. तस्या॑ अ॒फ्स्व॑फ्सु तस्या॒ स्तस्या॑ अ॒फ्सु । \newline
59. अ॒फ्सु मारु॑को॒ मारु॑को॒ ऽफ्स्व॑फ्सु मारु॑कः । \newline
60. अ॒फ्सित्य॑प् - सु । \newline
61. मारु॑को॒ या या मारु॑को॒ मारु॑को॒ या । \newline
62. या ऽभ्य॒ङ्क्ते᳚ ऽभ्य॒ङ्क्ते या या ऽभ्य॒ङ्क्ते । \newline

\textbf{Ghana Paata } \newline

1. न स॒ह स॒ह न न स॒हासी॑ता सीत स॒ह न न स॒हासी॑त । \newline
2. स॒हासी॑ता सीत स॒ह स॒हासी॑त॒ न नासी॑त स॒ह स॒हासी॑त॒ न । \newline
3. आ॒सी॒त॒ न नासी॑ता सीत॒ नास्या॑ अस्या॒ नासी॑ता सीत॒ नास्याः᳚ । \newline
4. नास्या॑ अस्या॒ न नास्या॒ अन्न॒ मन्न॑ मस्या॒ न नास्या॒ अन्न᳚म् । \newline
5. अ॒स्या॒ अन्न॒ मन्न॑ मस्या अस्या॒ अन्न॑ मद्या दद्या॒ दन्न॑ मस्या अस्या॒ अन्न॑ मद्यात् । \newline
6. अन्न॑ मद्या दद्या॒ दन्न॒ मन्न॑ मद्याद् ब्रह्मह॒त्यायै᳚ ब्रह्मह॒त्याया॑ अद्या॒ दन्न॒ मन्न॑ मद्याद् ब्रह्मह॒त्यायै᳚ । \newline
7. अ॒द्या॒द् ब्र॒ह्म॒ह॒त्यायै᳚ ब्रह्मह॒त्याया॑ अद्या दद्याद् ब्रह्मह॒त्यायै॒ हि हि ब्र॑ह्मह॒त्याया॑ अद्या दद्याद् ब्रह्मह॒त्यायै॒ हि । \newline
8. ब्र॒ह्म॒ह॒त्यायै॒ हि हि ब्र॑ह्मह॒त्यायै᳚ ब्रह्मह॒त्यायै॒ ह्ये॑षैषा हि ब्र॑ह्मह॒त्यायै᳚ ब्रह्मह॒त्यायै॒ ह्ये॑षा । \newline
9. ब्र॒ह्म॒ह॒त्याया॒ इति॑ ब्रह्म - ह॒त्यायै᳚ । \newline
10. ह्ये॑षैषा हि ह्ये॑षा वर्णं॒ ॅवर्ण॑ मे॒षा हि ह्ये॑षा वर्ण᳚म् । \newline
11. ए॒षा वर्णं॒ ॅवर्ण॑ मे॒षैषा वर्ण॑म् प्रति॒मुच्य॑ प्रति॒मुच्य॒ वर्ण॑ मे॒षैषा वर्ण॑म् प्रति॒मुच्य॑ । \newline
12. वर्ण॑म् प्रति॒मुच्य॑ प्रति॒मुच्य॒ वर्णं॒ ॅवर्ण॑म् प्रति॒मुच्यास्त॒ आस्ते᳚ प्रति॒मुच्य॒ वर्णं॒ ॅवर्ण॑म् प्रति॒मुच्यास्ते᳚ । \newline
13. प्र॒ति॒मुच्यास्त॒ आस्ते᳚ प्रति॒मुच्य॑ प्रति॒मुच्यास्ते ऽथो॒ अथो॒ आस्ते᳚ प्रति॒मुच्य॑ प्रति॒मुच्यास्ते ऽथो᳚ । \newline
14. प्र॒ति॒मुच्येति॑ प्रति - मुच्य॑ । \newline
15. आस्ते ऽथो॒ अथो॒ आस्त॒ आस्ते ऽथो॒ खलु॒ खल्वथो॒ आस्त॒ आस्ते ऽथो॒ खलु॑ । \newline
16. अथो॒ खलु॒ खल्वथो॒ अथो॒ खल्वा॑हु राहुः॒ खल्वथो॒ अथो॒ खल्वा॑हुः । \newline
17. अथो॒ इत्यथो᳚ । \newline
18. खल्वा॑हुराहुः॒ खलु॒ खल्वा॑हु र॒भ्यञ्ज॑न म॒भ्यञ्ज॑न माहुः॒ खलु॒ खल्वा॑हु र॒भ्यञ्ज॑नम् । \newline
19. आ॒हु॒ र॒भ्यञ्ज॑न म॒भ्यञ्ज॑न माहुराहु र॒भ्यञ्ज॑नं॒ ॅवाव वावाभ्यञ्ज॑न माहुराहु र॒भ्यञ्ज॑नं॒ ॅवाव । \newline
20. अ॒भ्यञ्ज॑नं॒ ॅवाव वावाभ्यञ्ज॑न म॒भ्यञ्ज॑नं॒ ॅवाव स्त्रि॒याः स्त्रि॒या वावाभ्यञ्ज॑न म॒भ्यञ्ज॑नं॒ ॅवाव स्त्रि॒याः । \newline
21. अ॒भ्यञ्ज॑न॒मित्य॑भि - अञ्ज॑नम् । \newline
22. वाव स्त्रि॒याः स्त्रि॒या वाव वाव स्त्रि॒या अन्न॒ मन्नꣳ॑ स्त्रि॒या वाव वाव स्त्रि॒या अन्न᳚म् । \newline
23. स्त्रि॒या अन्न॒ मन्नꣳ॑ स्त्रि॒याः स्त्रि॒या अन्न॑ म॒भ्यञ्ज॑न म॒भ्यञ्ज॑न॒ मन्नꣳ॑ स्त्रि॒याः स्त्रि॒या अन्न॑ म॒भ्यञ्ज॑नम् । \newline
24. अन्न॑ म॒भ्यञ्ज॑न म॒भ्यञ्ज॑न॒ मन्न॒ मन्न॑ म॒भ्यञ्ज॑न मे॒वैवा भ्यञ्ज॑न॒ मन्न॒ मन्न॑ म॒भ्यञ्ज॑न मे॒व । \newline
25. अ॒भ्यञ्ज॑न मे॒वैवा भ्यञ्ज॑न म॒भ्यञ्ज॑न मे॒व न नैवाभ्यञ्ज॑न म॒भ्यञ्ज॑न मे॒व न । \newline
26. अ॒भ्यञ्ज॑न॒मित्य॑भि - अञ्ज॑नम् । \newline
27. ए॒व न नैवैव न प्र॑ति॒गृह्य॑म् प्रति॒गृह्य॒म् नैवैव न प्र॑ति॒गृह्य᳚म् । \newline
28. न प्र॑ति॒गृह्य॑म् प्रति॒गृह्य॒म् न न प्र॑ति॒गृह्य॒म् काम॒म् काम॑म् प्रति॒गृह्य॒म् न न प्र॑ति॒गृह्य॒म् काम᳚म् । \newline
29. प्र॒ति॒गृह्य॒म् काम॒म् काम॑म् प्रति॒गृह्य॑म् प्रति॒गृह्य॒म् काम॑ म॒न्य द॒न्यत् काम॑म् प्रति॒गृह्य॑म् प्रति॒गृह्य॒म् काम॑ म॒न्यत् । \newline
30. प्र॒ति॒गृह्य॒मिति॑ प्रति - गृह्य᳚म् । \newline
31. काम॑ म॒न्य द॒न्यत् काम॒म् काम॑ म॒न्य दितीत्य॒न्यत् काम॒म् काम॑ म॒न्यदिति॑ । \newline
32. अ॒न्य दितीत्य॒न्य द॒न्यदिति॒ यां ॅया मित्य॒न्य द॒न्यदिति॒ याम् । \newline
33. इति॒ यां ॅया मितीति॒ याम् मल॑वद्वासस॒म् मल॑वद्वाससं॒ ॅया मितीति॒ याम् मल॑वद्वाससम् । \newline
34. याम् मल॑वद्वासस॒म् मल॑वद्वाससं॒ ॅयां ॅयाम् मल॑वद्वाससꣳ स॒म्भव॑न्ति स॒म्भव॑न्ति॒ मल॑वद्वाससं॒ ॅयां ॅयाम् मल॑वद्वाससꣳ स॒म्भव॑न्ति । \newline
35. मल॑वद्वाससꣳ स॒म्भव॑न्ति स॒म्भव॑न्ति॒ मल॑वद्वासस॒म् मल॑वद्वाससꣳ स॒म्भव॑न्ति॒ यो यः स॒म्भव॑न्ति॒ मल॑वद्वासस॒म् मल॑वद्वाससꣳ स॒म्भव॑न्ति॒ यः । \newline
36. मल॑वद्वासस॒मिति॒ मल॑वत् - वा॒स॒स॒म् । \newline
37. स॒म्भव॑न्ति॒ यो यः स॒म्भव॑न्ति स॒म्भव॑न्ति॒ यस्तत॒ स्ततो॒ यः स॒म्भव॑न्ति स॒म्भव॑न्ति॒ य स्ततः॑ । \newline
38. स॒म्भव॒न्तीति॑ सं - भव॑न्ति । \newline
39. य स्तत॒ स्ततो॒ यो य स्ततो॒ जाय॑ते॒ जाय॑ते॒ ततो॒ यो य स्ततो॒ जाय॑ते । \newline
40. ततो॒ जाय॑ते॒ जाय॑ते॒ तत॒ स्ततो॒ जाय॑ते॒ स स जाय॑ते॒ तत॒ स्ततो॒ जाय॑ते॒ सः । \newline
41. जाय॑ते॒ स स जाय॑ते॒ जाय॑ते॒ सो॑ ऽभिश॒स्तो॑ ऽभिश॒स्तः स जाय॑ते॒ जाय॑ते॒ सो॑ ऽभिश॒स्तः । \newline
42. सो॑ ऽभिश॒स्तो॑ ऽभिश॒स्तः स सो॑ ऽभिश॒स्तो यां ॅया म॑भिश॒स्तः स सो॑ ऽभिश॒स्तो याम् । \newline
43. अ॒भि॒श॒स्तो यां ॅया म॑भिश॒स्तो॑ ऽभिश॒स्तो या मर॒ण्ये ऽर॑ण्ये॒ या म॑भिश॒स्तो॑ ऽभिश॒स्तो या मर॑ण्ये । \newline
44. अ॒भि॒श॒स्त इत्य॑भि - श॒स्तः । \newline
45. या मर॒ण्ये ऽर॑ण्ये॒ यां ॅया मर॑ण्ये॒ तस्यै॒ तस्या॒ अर॑ण्ये॒ यां ॅया मर॑ण्ये॒ तस्यै᳚ । \newline
46. अर॑ण्ये॒ तस्यै॒ तस्या॒ अर॒ण्ये ऽर॑ण्ये॒ तस्यै᳚ स्ते॒नः स्ते॒न स्तस्या॒ अर॒ण्ये ऽर॑ण्ये॒ तस्यै᳚ स्ते॒नः । \newline
47. तस्यै᳚ स्ते॒नः स्ते॒न स्तस्यै॒ तस्यै᳚ स्ते॒नो यां ॅयाꣳ स्ते॒न स्तस्यै॒ तस्यै᳚ स्ते॒नो याम् । \newline
48. स्ते॒नो यां ॅयाꣳ स्ते॒नः स्ते॒नो याम् परा॑ची॒म् परा॑चीं॒ ॅयाꣳ स्ते॒नः स्ते॒नो याम् परा॑चीम् । \newline
49. याम् परा॑ची॒म् परा॑चीं॒ ॅयां ॅयाम् परा॑ची॒म् तस्यै॒ तस्यै॒ परा॑चीं॒ ॅयां ॅयाम् परा॑ची॒म् तस्यै᳚ । \newline
50. परा॑ची॒म् तस्यै॒ तस्यै॒ परा॑ची॒म् परा॑ची॒म् तस्यै᳚ ह्रीतमु॒खी ह्री॑तमु॒खी तस्यै॒ परा॑ची॒म् परा॑ची॒म् तस्यै᳚ ह्रीतमु॒खी । \newline
51. तस्यै᳚ ह्रीतमु॒खी ह्री॑तमु॒खी तस्यै॒ तस्यै᳚ ह्रीतमु॒ख्य॑पग॒ल्भो॑ ऽपग॒ल्भो ह्री॑तमु॒खी तस्यै॒ तस्यै᳚ ह्रीतमु॒ख्य॑पग॒ल्भः । \newline
52. ह्री॒त॒मु॒ख्य॑पग॒ल्भो॑ ऽपग॒ल्भो ह्री॑तमु॒खी ह्री॑तमु॒ख्य॑पग॒ल्भो या या ऽप॑ग॒ल्भो ह्री॑तमु॒खी ह्री॑तमु॒ख्य॑पग॒ल्भो या । \newline
53. ह्री॒त॒मु॒खीति॑ ह्रीत - मु॒खी । \newline
54. अ॒प॒ग॒ल्भो या या ऽप॑ग॒ल्भो॑ ऽपग॒ल्भो या स्नाति॒ स्नाति॒ या ऽप॑ग॒ल्भो॑ ऽपग॒ल्भो या स्नाति॑ । \newline
55. अ॒प॒ग॒ल्भ इत्य॑प - ग॒ल्भः । \newline
56. या स्नाति॒ स्नाति॒ या या स्नाति॒ तस्या॒ स्तस्याः॒ स्नाति॒ या या स्नाति॒ तस्याः᳚ । \newline
57. स्नाति॒ तस्या॒ स्तस्याः॒ स्नाति॒ स्नाति॒ तस्या॑ अ॒फ्स्व॑फ्सु तस्याः॒ स्नाति॒ स्नाति॒ तस्या॑ अ॒फ्सु । \newline
58. तस्या॑ अ॒फ्स्व॑फ्सु तस्या॒ स्तस्या॑ अ॒फ्सु मारु॑को॒ मारु॑को॒ ऽफ्सु तस्या॒ स्तस्या॑ अ॒फ्सु मारु॑कः । \newline
59. अ॒फ्सु मारु॑को॒ मारु॑को॒ ऽफ्स्व॑फ्सु मारु॑को॒ या या मारु॑को॒ ऽफ्स्व॑फ्सु मारु॑को॒ या । \newline
60. अ॒फ्सित्य॑प् - सु । \newline
61. मारु॑को॒ या या मारु॑को॒ मारु॑को॒ या ऽभ्य॒ङ्क्ते᳚ ऽभ्य॒ङ्क्ते या मारु॑को॒ मारु॑को॒ या ऽभ्य॒ङ्क्ते । \newline
62. या ऽभ्य॒ङ्क्ते᳚ ऽभ्य॒ङ्क्ते या या ऽभ्य॒ङ्क्ते तस्यै॒ तस्या॑ अभ्य॒ङ्क्ते या या ऽभ्य॒ङ्क्ते तस्यै᳚ । \newline
\pagebreak
\markright{ TS 2.5.1.7  \hfill https://www.vedavms.in \hfill}
\addcontentsline{toc}{section}{ TS 2.5.1.7 }
\section*{ TS 2.5.1.7 }

\textbf{TS 2.5.1.7 } \newline
\textbf{Samhita Paata} \newline

ऽभ्य॒ङ्क्ते तस्यै॑ दु॒श्चर्मा॒ या प्र॑लि॒खते॒ तस्यै॑ खल॒तिर॑पमा॒री याऽऽङ्क्ते तस्यै॑ का॒णो या द॒तो धाव॑ते॒ तस्यै᳚ श्या॒वद॒न्॒. या न॒खानि॑ निकृ॒न्तते॒ तस्यै॑ कुन॒खी या कृ॒णत्ति॒ तस्यै᳚ क्ली॒बो या रज्जुꣳ॑ सृ॒जति॒ तस्या॑ उ॒द्-बन्धु॑को॒ या प॒र्णेन॒ पिब॑ति॒ तस्या॑ उ॒न्मादु॑को॒ या ख॒र्वेण॒ पिब॑ति॒ तस्यै॑ ख॒र्वस्ति॒स्रो रात्री᳚र्व्र॒तं च॑रेदञ्ज॒लिना॑ वा॒ पिबे॒दख॑र्वेण वा॒ ( ) पात्रे॑ण प्र॒जायै॑ गोपी॒थाय॑ ॥ \newline

\textbf{Pada Paata} \newline

अ॒भ्य॒ङ्क्त इत्य॑भि - अ॒ङ्क्ते । तस्यै᳚ । दु॒श्चर्मेति॑ दुः - चर्मा᳚ । या । प्र॒लि॒खत॒ इति॑ प्र - लि॒खते᳚ । तस्यै᳚ । ख॒ल॒तिः । अ॒प॒मा॒रीत्य॑प - मा॒री । या । आ॒ङ्क्त इत्या᳚ - अ॒ङ्क्ते । तस्यै᳚ । का॒णः । या । द॒तः । धाव॑ते । तस्यै᳚ । श्या॒वद॒न्निति॑ श्या॒व - द॒न्न् । या । न॒खानि॑ । नि॒कृ॒न्तत॒ इति॑ नि - कृ॒न्तते᳚ । तस्यै᳚ । कु॒न॒खी । या । कृ॒णत्ति॑ । तस्यै᳚ । क्ली॒बः । या । रज्जु᳚म् । सृ॒जति॑ । तस्याः᳚ । उ॒द्बन्धु॑क॒ इत्यु॑त् - बन्धु॑कः । या । प॒र्णेन॑ । पिब॑ति । तस्याः᳚ । उ॒न्मादु॑क॒ इत्यु॑त् - मादु॑कः । या । ख॒र्वेण॑ । पिब॑ति । तस्यै᳚ । ख॒र्वः । ति॒स्रः । रात्रीः᳚ । व्र॒तम् । च॒रे॒त् । अ॒ञ्ज॒लिना᳚ । वा॒ । पिबे᳚त् । अख॑र्वेण । वा॒ ( ) । पात्रे॑ण । प्र॒जाया॒ इति॑ प्र - जायै᳚ । गो॒पी॒थाय॑ ॥  \newline


\textbf{Krama Paata} \newline

अ॒भ्य॒ङ्क्ते तस्यै᳚ । अ॒भ्य॒ङ्क्त इत्य॑भि - अ॒ङ्क्ते । तस्यै॑ दु॒श्चर्मा᳚ । दु॒श्चर्मा॒ या । दु॒श्चर्मेति॑ दुः - चर्मा᳚ । या प्र॑लि॒खते᳚ । प्र॒लि॒खते॒ तस्यै᳚ । प्र॒लि॒खत॒ इति॑ प्र - लि॒खते᳚ । तस्यै॑ खल॒तिः । ख॒ल॒तिर॑पमा॒री । अ॒प॒मा॒री या । अ॒प॒मा॒रीत्य॑प - मा॒री । या ऽऽङ्क्ते । आ॒ङ्क्ते तस्यै᳚ । आ॒ङ्क्त इत्या᳚ - अ॒ङ्क्ते । तस्यै॑ का॒णः । का॒णो या । या द॒तः । द॒तो धाव॑ते । धाव॑ते॒ तस्यै᳚ । तस्यै᳚ श्या॒वदन्न्॑ । श्या॒वद॒न्॒. या । श्या॒वद॒न्निति॑ श्या॒व - द॒न्न्॒ । या न॒खानि॑ । न॒खानि॑ निकृ॒न्तते᳚ । नि॒कृ॒न्तते॒ तस्यै᳚ । नि॒कृ॒न्तत॒ इति॑ नि - कृ॒न्तते᳚ । तस्यै॑ कुन॒खी । कु॒न॒खी या । या कृ॒णत्ति॑ । कृ॒णत्ति॒ तस्यै᳚ । तस्यै᳚ क्ली॒बः । क्ली॒बो या । या रज्जु᳚म् । रज्जुꣳ॑ सृ॒जति॑ । सृ॒जति॒ तस्याः᳚ । तस्या॑ उ॒द्बन्धु॑कः । उ॒द्बन्धु॑को॒ या । उ॒द्बन्धु॑क॒ इत्यु॑त् - बन्धु॑कः । या प॒र्णेन॑ । प॒र्णेन॒ पिब॑ति । पिब॑ति॒ तस्याः᳚ । तस्या॑ उ॒न्मादु॑कः । उ॒न्मादु॑को॒ या । उ॒न्मादु॑क॒ इत्यु॑त् - मादु॑कः । या ख॒र्वेण॑ । ख॒र्वेण॒ पिब॑ति । पिब॑ति॒ तस्यै᳚ । तस्यै॑ ख॒र्वः । ख॒र्वस्ति॒स्रः । ति॒स्रो रात्रीः᳚ । रात्री᳚र् व्र॒तम् । व्र॒तम् च॑रेत् । च॒रे॒द॒ञ्ज॒लिना᳚ । अ॒ञ्ज॒लिना॑ वा । वा॒ पिबे᳚त् । पिबे॒दख॑र्वेण ( ) । अख॑र्वेण वा । वा॒ पात्रे॑ण । पात्रे॑ण प्र॒जायै᳚ । प्र॒जायै॑ गोपी॒थाय॑ । प्र॒जाया॒ इति॑ प्र - जायै᳚ । गो॒पी॒थायेति॑ गोपी॒थाय॑ । \newline

\textbf{Jatai Paata} \newline

1. अ॒भ्य॒ङ्क्ते तस्यै॒ तस्या॑ अभ्य॒ङ्क्ते᳚ ऽभ्य॒ङ्क्ते तस्यै᳚ । \newline
2. अ॒भ्य॒ङ्क्त इत्य॑भि - अ॒ङ्क्ते । \newline
3. तस्यै॑ दु॒श्चर्मा॑ दु॒श्चर्मा॒ तस्यै॒ तस्यै॑ दु॒श्चर्मा᳚ । \newline
4. दु॒श्चर्मा॒ या या दु॒श्चर्मा॑ दु॒श्चर्मा॒ या । \newline
5. दु॒श्चर्मेति॑ दुः - चर्मा᳚ । \newline
6. या प्र॑लि॒खते᳚ प्रलि॒खते॒ या या प्र॑लि॒खते᳚ । \newline
7. प्र॒लि॒खते॒ तस्यै॒ तस्यै᳚ प्रलि॒खते᳚ प्रलि॒खते॒ तस्यै᳚ । \newline
8. प्र॒लि॒खत॒ इति॑ प्र - लि॒खते᳚ । \newline
9. तस्यै॑ खल॒तिः ख॑ल॒ति स्तस्यै॒ तस्यै॑ खल॒तिः । \newline
10. ख॒ल॒ति र॑पमा॒र्य॑पमा॒री ख॑ल॒तिः ख॑ल॒ति र॑पमा॒री । \newline
11. अ॒प॒मा॒री या या ऽप॑मा॒ र्य॑पमा॒री या । \newline
12. अ॒प॒मा॒रीत्य॑प - मा॒री । \newline
13. या ऽऽङ्क्त आ॒ङ्क्ते या या ऽऽङ्क्ते । \newline
14. आ॒ङ्क्ते तस्यै॒ तस्या॑ आ॒ङ्क्त आ॒ङ्क्ते तस्यै᳚ । \newline
15. आ॒ङ्क्त इत्या᳚ - अ॒ङ्क्ते । \newline
16. तस्यै॑ का॒णः का॒ण स्तस्यै॒ तस्यै॑ का॒णः । \newline
17. का॒णो या या का॒णः का॒णो या । \newline
18. या द॒तो द॒तो या या द॒तः । \newline
19. द॒तो धाव॑ते॒ धाव॑ते द॒तो द॒तो धाव॑ते । \newline
20. धाव॑ते॒ तस्यै॒ तस्यै॒ धाव॑ते॒ धाव॑ते॒ तस्यै᳚ । \newline
21. तस्यै᳚ श्या॒वद॑ञ् छ्या॒वद॒न् तस्यै॒ तस्यै᳚ श्या॒वदन्न्॑ । \newline
22. श्या॒वद॒न्॒. या या श्या॒वद॑ञ् छ्या॒वद॒न्॒. या । \newline
23. श्या॒वद॒न्निति॑ श्या॒व - द॒न्न् । \newline
24. या न॒खानि॑ न॒खानि॒ या या न॒खानि॑ । \newline
25. न॒खानि॑ निकृ॒न्तते॑ निकृ॒न्तते॑ न॒खानि॑ न॒खानि॑ निकृ॒न्तते᳚ । \newline
26. नि॒कृ॒न्तते॒ तस्यै॒ तस्यै॑ निकृ॒न्तते॑ निकृ॒न्तते॒ तस्यै᳚ । \newline
27. नि॒कृ॒न्तत॒ इति॑ नि - कृ॒न्तते᳚ । \newline
28. तस्यै॑ कुन॒खी कु॑न॒खी तस्यै॒ तस्यै॑ कुन॒खी । \newline
29. कु॒न॒खी या या कु॑न॒खी कु॑न॒खी या । \newline
30. या कृ॒णत्ति॑ कृ॒णत्ति॒ या या कृ॒णत्ति॑ । \newline
31. कृ॒णत्ति॒ तस्यै॒ तस्यै॑ कृ॒णत्ति॑ कृ॒णत्ति॒ तस्यै᳚ । \newline
32. तस्यै᳚ क्ली॒बः क्ली॒ब स्तस्यै॒ तस्यै᳚ क्ली॒बः । \newline
33. क्ली॒बो या या क्ली॒बः क्ली॒बो या । \newline
34. या रज्जुꣳ॒॒ रज्जुं॒ ॅया या रज्जु᳚म् । \newline
35. रज्जुꣳ॑ सृ॒जति॑ सृ॒जति॒ रज्जुꣳ॒॒ रज्जुꣳ॑ सृ॒जति॑ । \newline
36. सृ॒जति॒ तस्या॒ स्तस्याः᳚ सृ॒जति॑ सृ॒जति॒ तस्याः᳚ । \newline
37. तस्या॑ उ॒द्बन्धु॑क उ॒द्बन्धु॑क॒ स्तस्या॒ स्तस्या॑ उ॒द्बन्धु॑कः । \newline
38. उ॒द्बन्धु॑को॒ या योद्बन्धु॑क उ॒द्बन्धु॑को॒ या । \newline
39. उ॒द्बन्धु॑क॒ इत्यु॑त् - बन्धु॑कः । \newline
40. या प॒र्णेन॑ प॒र्णेन॒ या या प॒र्णेन॑ । \newline
41. प॒र्णेन॒ पिब॑ति॒ पिब॑ति प॒र्णेन॑ प॒र्णेन॒ पिब॑ति । \newline
42. पिब॑ति॒ तस्या॒ स्तस्याः॒ पिब॑ति॒ पिब॑ति॒ तस्याः᳚ । \newline
43. तस्या॑ उ॒न्मादु॑क उ॒न्मादु॑क॒ स्तस्या॒ स्तस्या॑ उ॒न्मादु॑कः । \newline
44. उ॒न्मादु॑को॒ या योन्मादु॑क उ॒न्मादु॑को॒ या । \newline
45. उ॒न्मादु॑क॒ इत्यु॑त् - मादु॑कः । \newline
46. या ख॒र्वेण॑ ख॒र्वेण॒ या या ख॒र्वेण॑ । \newline
47. ख॒र्वेण॒ पिब॑ति॒ पिब॑ति ख॒र्वेण॑ ख॒र्वेण॒ पिब॑ति । \newline
48. पिब॑ति॒ तस्यै॒ तस्यै॒ पिब॑ति॒ पिब॑ति॒ तस्यै᳚ । \newline
49. तस्यै॑ ख॒र्वः ख॒र्व स्तस्यै॒ तस्यै॑ ख॒र्वः । \newline
50. ख॒र्व स्ति॒स्र स्ति॒स्रः ख॒र्वः ख॒र्व स्ति॒स्रः । \newline
51. ति॒स्रो रात्री॒ रात्री᳚ स्ति॒स्र स्ति॒स्रो रात्रीः᳚ । \newline
52. रात्री᳚र् व्र॒तं ॅव्र॒तꣳ रात्री॒ रात्री᳚र् व्र॒तम् । \newline
53. व्र॒तम् च॑रेच् चरेद् व्र॒तं ॅव्र॒तम् च॑रेत् । \newline
54. च॒रे॒ द॒ञ्ज॒लिना᳚ ऽञ्ज॒लिना॑ चरेच् चरे दञ्ज॒लिना᳚ । \newline
55. अ॒ञ्ज॒लिना॑ वा वा ऽञ्ज॒लिना᳚ ऽञ्ज॒लिना॑ वा । \newline
56. वा॒ पिबे॒त् पिबे᳚द् वा वा॒ पिबे᳚त् । \newline
57. पिबे॒ दख॑र्वे॒णा ख॑र्वेण॒ पिबे॒त् पिबे॒ दख॑र्वेण । \newline
58. अख॑र्वेण वा॒ वा ऽख॑र्वे॒णा ख॑र्वेण वा । \newline
59. वा॒ पात्रे॑ण॒ पात्रे॑ण वा वा॒ पात्रे॑ण । \newline
60. पात्रे॑ण प्र॒जायै᳚ प्र॒जायै॒ पात्रे॑ण॒ पात्रे॑ण प्र॒जायै᳚ । \newline
61. प्र॒जायै॑ गोपी॒थाय॑ गोपी॒थाय॑ प्र॒जायै᳚ प्र॒जायै॑ गोपी॒थाय॑ । \newline
62. प्र॒जाया॒ इति॑ प्र - जायै᳚ । \newline
63. गो॒पी॒थायेति॑ गोपी॒थाय॑ । \newline

\textbf{Ghana Paata } \newline

1. अ॒भ्य॒ङ्क्ते तस्यै॒ तस्या॑ अभ्य॒ङ्क्ते᳚ ऽभ्य॒ङ्क्ते तस्यै॑ दु॒श्चर्मा॑ दु॒श्चर्मा॒ तस्या॑ अभ्य॒ङ्क्ते᳚ ऽभ्य॒ङ्क्ते तस्यै॑ दु॒श्चर्मा᳚ । \newline
2. अ॒भ्य॒ङ्क्त इत्य॑भि - अ॒ङ्क्ते । \newline
3. तस्यै॑ दु॒श्चर्मा॑ दु॒श्चर्मा॒ तस्यै॒ तस्यै॑ दु॒श्चर्मा॒ या या दु॒श्चर्मा॒ तस्यै॒ तस्यै॑ दु॒श्चर्मा॒ या । \newline
4. दु॒श्चर्मा॒ या या दु॒श्चर्मा॑ दु॒श्चर्मा॒ या प्र॑लि॒खते᳚ प्रलि॒खते॒ या दु॒श्चर्मा॑ दु॒श्चर्मा॒ या प्र॑लि॒खते᳚ । \newline
5. दु॒श्चर्मेति॑ दुः - चर्मा᳚ । \newline
6. या प्र॑लि॒खते᳚ प्रलि॒खते॒ या या प्र॑लि॒खते॒ तस्यै॒ तस्यै᳚ प्रलि॒खते॒ या या प्र॑लि॒खते॒ तस्यै᳚ । \newline
7. प्र॒लि॒खते॒ तस्यै॒ तस्यै᳚ प्रलि॒खते᳚ प्रलि॒खते॒ तस्यै॑ खल॒तिः ख॑ल॒ति स्तस्यै᳚ प्रलि॒खते᳚ प्रलि॒खते॒ तस्यै॑ खल॒तिः । \newline
8. प्र॒लि॒खत॒ इति॑ प्र - लि॒खते᳚ । \newline
9. तस्यै॑ खल॒तिः ख॑ल॒ति स्तस्यै॒ तस्यै॑ खल॒ति र॑पमा॒र्य॑पमा॒री ख॑ल॒ति स्तस्यै॒ तस्यै॑ खल॒ति र॑पमा॒री । \newline
10. ख॒ल॒ति र॑पमा॒र्य॑पमा॒री ख॑ल॒तिः ख॑ल॒ति र॑पमा॒री या या ऽप॑मा॒री ख॑ल॒तिः ख॑ल॒ति र॑पमा॒री या । \newline
11. अ॒प॒मा॒री या या ऽप॑मा॒र्य॑पमा॒री या ऽऽङ्क्त आ॒ङ्क्ते या ऽप॑मा॒र्य॑पमा॒री या ऽऽङ्क्ते । \newline
12. अ॒प॒मा॒रीत्य॑प - मा॒री । \newline
13. या ऽऽङ्क्त आ॒ङ्क्ते या या ऽऽङ्क्ते तस्यै॒ तस्या॑ आ॒ङ्क्ते या या ऽऽङ्क्ते तस्यै᳚ । \newline
14. आ॒ङ्क्ते तस्यै॒ तस्या॑ आ॒ङ्क्त आ॒ङ्क्ते तस्यै॑ का॒णः का॒ण स्तस्या॑ आ॒ङ्क्त आ॒ङ्क्ते तस्यै॑ का॒णः । \newline
15. आ॒ङ्क्त इत्या᳚ - अ॒ङ्क्ते । \newline
16. तस्यै॑ का॒णः का॒ण स्तस्यै॒ तस्यै॑ का॒णो या या का॒ण स्तस्यै॒ तस्यै॑ का॒णो या । \newline
17. का॒णो या या का॒णः का॒णो या द॒तो द॒तो या का॒णः का॒णो या द॒तः । \newline
18. या द॒तो द॒तो या या द॒तो धाव॑ते॒ धाव॑ते द॒तो या या द॒तो धाव॑ते । \newline
19. द॒तो धाव॑ते॒ धाव॑ते द॒तो द॒तो धाव॑ते॒ तस्यै॒ तस्यै॒ धाव॑ते द॒तो द॒तो धाव॑ते॒ तस्यै᳚ । \newline
20. धाव॑ते॒ तस्यै॒ तस्यै॒ धाव॑ते॒ धाव॑ते॒ तस्यै᳚ श्या॒वद॑ञ् छ्या॒वद॒न् तस्यै॒ धाव॑ते॒ धाव॑ते॒ तस्यै᳚ श्या॒वदन्न्॑ । \newline
21. तस्यै᳚ श्या॒वद॑ञ् छ्या॒वद॒न् तस्यै॒ तस्यै᳚ श्या॒वद॒न्॒. या या श्या॒वद॒न् तस्यै॒ तस्यै᳚ श्या॒वद॒न्॒. या । \newline
22. श्या॒वद॒न्॒. या या श्या॒वद॑ञ् छ्या॒वद॒न्॒. या न॒खानि॑ न॒खानि॒ या श्या॒वद॑ञ् छ्या॒वद॒न्॒. या न॒खानि॑ । \newline
23. श्या॒वद॒न्निति॑ श्या॒व - द॒न्न् । \newline
24. या न॒खानि॑ न॒खानि॒ या या न॒खानि॑ निकृ॒न्तते॑ निकृ॒न्तते॑ न॒खानि॒ या या न॒खानि॑ निकृ॒न्तते᳚ । \newline
25. न॒खानि॑ निकृ॒न्तते॑ निकृ॒न्तते॑ न॒खानि॑ न॒खानि॑ निकृ॒न्तते॒ तस्यै॒ तस्यै॑ निकृ॒न्तते॑ न॒खानि॑ न॒खानि॑ निकृ॒न्तते॒ तस्यै᳚ । \newline
26. नि॒कृ॒न्तते॒ तस्यै॒ तस्यै॑ निकृ॒न्तते॑ निकृ॒न्तते॒ तस्यै॑ कुन॒खी कु॑न॒खी तस्यै॑ निकृ॒न्तते॑ निकृ॒न्तते॒ तस्यै॑ कुन॒खी । \newline
27. नि॒कृ॒न्तत॒ इति॑ नि - कृ॒न्तते᳚ । \newline
28. तस्यै॑ कुन॒खी कु॑न॒खी तस्यै॒ तस्यै॑ कुन॒खी या या कु॑न॒खी तस्यै॒ तस्यै॑ कुन॒खी या । \newline
29. कु॒न॒खी या या कु॑न॒खी कु॑न॒खी या कृ॒णत्ति॑ कृ॒णत्ति॒ या कु॑न॒खी कु॑न॒खी या कृ॒णत्ति॑ । \newline
30. या कृ॒णत्ति॑ कृ॒णत्ति॒ या या कृ॒णत्ति॒ तस्यै॒ तस्यै॑ कृ॒णत्ति॒ या या कृ॒णत्ति॒ तस्यै᳚ । \newline
31. कृ॒णत्ति॒ तस्यै॒ तस्यै॑ कृ॒णत्ति॑ कृ॒णत्ति॒ तस्यै᳚ क्ली॒बः क्ली॒ब स्तस्यै॑ कृ॒णत्ति॑ कृ॒णत्ति॒ तस्यै᳚ क्ली॒बः । \newline
32. तस्यै᳚ क्ली॒बः क्ली॒ब स्तस्यै॒ तस्यै᳚ क्ली॒बो या या क्ली॒ब स्तस्यै॒ तस्यै᳚ क्ली॒बो या । \newline
33. क्ली॒बो या या क्ली॒बः क्ली॒बो या रज्जुꣳ॒॒ रज्जुं॒ ॅया क्ली॒बः क्ली॒बो या रज्जु᳚म् । \newline
34. या रज्जुꣳ॒॒ रज्जुं॒ ॅया या रज्जुꣳ॑ सृ॒जति॑ सृ॒जति॒ रज्जुं॒ ॅया या रज्जुꣳ॑ सृ॒जति॑ । \newline
35. रज्जुꣳ॑ सृ॒जति॑ सृ॒जति॒ रज्जुꣳ॒॒ रज्जुꣳ॑ सृ॒जति॒ तस्या॒ स्तस्याः᳚ सृ॒जति॒ रज्जुꣳ॒॒ रज्जुꣳ॑ सृ॒जति॒ तस्याः᳚ । \newline
36. सृ॒जति॒ तस्या॒ स्तस्याः᳚ सृ॒जति॑ सृ॒जति॒ तस्या॑ उ॒द्बन्धु॑क उ॒द्बन्धु॑क॒ स्तस्याः᳚ सृ॒जति॑ सृ॒जति॒ तस्या॑ उ॒द्बन्धु॑कः । \newline
37. तस्या॑ उ॒द्बन्धु॑क उ॒द्बन्धु॑क॒ स्तस्या॒ स्तस्या॑ उ॒द्बन्धु॑को॒ या योद्बन्धु॑क॒ स्तस्या॒ स्तस्या॑ उ॒द्बन्धु॑को॒ या । \newline
38. उ॒द्बन्धु॑को॒ या योद्बन्धु॑क उ॒द्बन्धु॑को॒ या प॒र्णेन॑ प॒र्णेन॒ योद्बन्धु॑क उ॒द्बन्धु॑को॒ या प॒र्णेन॑ । \newline
39. उ॒द्बन्धु॑क॒ इत्यु॑त् - बन्धु॑कः । \newline
40. या प॒र्णेन॑ प॒र्णेन॒ या या प॒र्णेन॒ पिब॑ति॒ पिब॑ति प॒र्णेन॒ या या प॒र्णेन॒ पिब॑ति । \newline
41. प॒र्णेन॒ पिब॑ति॒ पिब॑ति प॒र्णेन॑ प॒र्णेन॒ पिब॑ति॒ तस्या॒ स्तस्याः॒ पिब॑ति प॒र्णेन॑ प॒र्णेन॒ पिब॑ति॒ तस्याः᳚ । \newline
42. पिब॑ति॒ तस्या॒ स्तस्याः॒ पिब॑ति॒ पिब॑ति॒ तस्या॑ उ॒न्मादु॑क उ॒न्मादु॑क॒ स्तस्याः॒ पिब॑ति॒ पिब॑ति॒ तस्या॑ उ॒न्मादु॑कः । \newline
43. तस्या॑ उ॒न्मादु॑क उ॒न्मादु॑क॒ स्तस्या॒ स्तस्या॑ उ॒न्मादु॑को॒ या योन्मादु॑क॒ स्तस्या॒ स्तस्या॑ उ॒न्मादु॑को॒ या । \newline
44. उ॒न्मादु॑को॒ या योन्मादु॑क उ॒न्मादु॑को॒ या ख॒र्वेण॑ ख॒र्वेण॒ योन्मादु॑क उ॒न्मादु॑को॒ या ख॒र्वेण॑ । \newline
45. उ॒न्मादु॑क॒ इत्यु॑त् - मादु॑कः । \newline
46. या ख॒र्वेण॑ ख॒र्वेण॒ या या ख॒र्वेण॒ पिब॑ति॒ पिब॑ति ख॒र्वेण॒ या या ख॒र्वेण॒ पिब॑ति । \newline
47. ख॒र्वेण॒ पिब॑ति॒ पिब॑ति ख॒र्वेण॑ ख॒र्वेण॒ पिब॑ति॒ तस्यै॒ तस्यै॒ पिब॑ति ख॒र्वेण॑ ख॒र्वेण॒ पिब॑ति॒ तस्यै᳚ । \newline
48. पिब॑ति॒ तस्यै॒ तस्यै॒ पिब॑ति॒ पिब॑ति॒ तस्यै॑ ख॒र्वः ख॒र्व स्तस्यै॒ पिब॑ति॒ पिब॑ति॒ तस्यै॑ ख॒र्वः । \newline
49. तस्यै॑ ख॒र्वः ख॒र्व स्तस्यै॒ तस्यै॑ ख॒र्व स्ति॒स्र स्ति॒स्रः ख॒र्व स्तस्यै॒ तस्यै॑ ख॒र्व स्ति॒स्रः । \newline
50. ख॒र्व स्ति॒स्र स्ति॒स्रः ख॒र्वः ख॒र्व स्ति॒स्रो रात्री॒ रात्री᳚ स्ति॒स्रः ख॒र्वः ख॒र्व स्ति॒स्रो रात्रीः᳚ । \newline
51. ति॒स्रो रात्री॒ रात्री᳚ स्ति॒स्र स्ति॒स्रो रात्री᳚र् व्र॒तं ॅव्र॒तꣳ रात्री᳚ स्ति॒स्र स्ति॒स्रो रात्री᳚र् व्र॒तम् । \newline
52. रात्री᳚र् व्र॒तं ॅव्र॒तꣳ रात्री॒ रात्री᳚र् व्र॒तम् च॑रेच् चरेद् व्र॒तꣳ रात्री॒ रात्री᳚र् व्र॒तम् च॑रेत् । \newline
53. व्र॒तम् च॑रेच् चरेद् व्र॒तं ॅव्र॒तम् च॑रे दञ्ज॒लिना᳚ ऽञ्ज॒लिना॑ चरेद् व्र॒तं ॅव्र॒तम् च॑रे दञ्ज॒लिना᳚ । \newline
54. च॒रे॒द॒ञ्ज॒लिना᳚ ऽञ्ज॒लिना॑ चरेच् चरेदञ्ज॒लिना॑ वा वा ऽञ्ज॒लिना॑ चरेच् चरेदञ्ज॒लिना॑ वा । \newline
55. अ॒ञ्ज॒लिना॑ वा वा ऽञ्ज॒लिना᳚ ऽञ्ज॒लिना॑ वा॒ पिबे॒त् पिबे᳚द् वा ऽञ्ज॒लिना᳚ ऽञ्ज॒लिना॑ वा॒ पिबे᳚त् । \newline
56. वा॒ पिबे॒त् पिबे᳚द् वा वा॒ पिबे॒ दख॑र्वे॒णा ख॑र्वेण॒ पिबे᳚द् वा वा॒ पिबे॒ दख॑र्वेण । \newline
57. पिबे॒दख॑र्वे॒णा ख॑र्वेण॒ पिबे॒त् पिबे॒दख॑र्वेण वा॒ वा ऽख॑र्वेण॒ पिबे॒त् पिबे॒दख॑र्वेण वा । \newline
58. अख॑र्वेण वा॒ वा ऽख॑र्वे॒णा ख॑र्वेण वा॒ पात्रे॑ण॒ पात्रे॑ण॒ वा ऽख॑र्वे॒णा ख॑र्वेण वा॒ पात्रे॑ण । \newline
59. वा॒ पात्रे॑ण॒ पात्रे॑ण वा वा॒ पात्रे॑ण प्र॒जायै᳚ प्र॒जायै॒ पात्रे॑ण वा वा॒ पात्रे॑ण प्र॒जायै᳚ । \newline
60. पात्रे॑ण प्र॒जायै᳚ प्र॒जायै॒ पात्रे॑ण॒ पात्रे॑ण प्र॒जायै॑ गोपी॒थाय॑ गोपी॒थाय॑ प्र॒जायै॒ पात्रे॑ण॒ पात्रे॑ण प्र॒जायै॑ गोपी॒थाय॑ । \newline
61. प्र॒जायै॑ गोपी॒थाय॑ गोपी॒थाय॑ प्र॒जायै᳚ प्र॒जायै॑ गोपी॒थाय॑ । \newline
62. प्र॒जाया॒ इति॑ प्र - जायै᳚ । \newline
63. गो॒पी॒थायेति॑ गोपी॒थाय॑ । \newline
\pagebreak
\markright{ TS 2.5.2.1  \hfill https://www.vedavms.in \hfill}
\addcontentsline{toc}{section}{ TS 2.5.2.1 }
\section*{ TS 2.5.2.1 }

\textbf{TS 2.5.2.1 } \newline
\textbf{Samhita Paata} \newline

त्वष्टा॑ ह॒तपु॑त्रो॒ वीन्द्रꣳ॒॒ सोम॒माऽह॑र॒त् तस्मि॒न्निन्द्र॑ उपह॒वमै᳚च्छत॒ तं नोपा᳚ह्वयत पु॒त्रं मे॑ऽवधी॒रिति॒ स य॑ज्ञ्वेश॒सं कृ॒त्वा प्रा॒सहा॒ सोम॑मपिब॒त् तस्य॒ यद॒त्यशि॑ष्यत॒ तत् त्वष्टा॑ऽऽहव॒नीय॒मुप॒ प्राव॑र्तय॒थ् स्वाहेन्द्र॑शत्रु-र्वर्द्ध॒स्वेति॒ यदव॑र्तय॒त् तद्-वृ॒त्रस्य॑ वृत्र॒त्वं ॅयदब्र॑वी॒थ् स्वाहेन्द्र॑शत्रु-र्वर्द्ध॒स्वेति॒ तस्मा॑द॒स्ये- [  ] \newline

\textbf{Pada Paata} \newline

त्वष्टा᳚ । ह॒तपु॑त्र॒ इति॑ ह॒त-पु॒त्रः॒ । वीन्द्र॒मिति॒ वि - इ॒न्द्र॒म् । सोम᳚म् । एति॑ । अ॒ह॒र॒त् । तस्मिन्न्॑ । इन्द्रः॑ । उ॒प॒ह॒वमित्यु॑प-ह॒वम् । ऐ॒च्छ॒त॒ । तम् । न । उपेति॑ । अ॒ह्व॒य॒त॒ । पु॒त्रम् । मे॒ । अ॒व॒धीः॒ । इति॑ । सः । य॒ज्ञ्॒वे॒श॒समिति॑ यज्ञ् - वे॒श॒सम् । कृ॒त्वा । प्रा॒सहेति॑ प्र - सहा᳚ । सोम᳚म् । अ॒पि॒ब॒त् । तस्य॑ । यत् । अ॒त्यशि॑ष्य॒तेत्य॑ति - अशि॑ष्यत । तत् । त्वष्टा᳚ । आ॒ह॒व॒नीय॒मित्या᳚-ह॒व॒नीय᳚म् । उप॑ । प्रेति॑ । अ॒व॒र्त॒य॒त् । स्वाहा᳚ । इन्द्र॑शत्रु॒रितीन्द्र॑ - श॒त्रुः॒ । व॒र्द्ध॒स्व॒ । इति॑ । यत् । अव॑र्तयत् । तत् । वृ॒त्रस्य॑ । वृ॒त्र॒त्वमिति॑ वृत्र-त्वम् । यत् । अब्र॑वीत् । स्वाहा᳚ । इन्द्र॑शत्रु॒रितीन्द्र॑ - श॒त्रुः॒ । व॒र्द्ध॒स्व॒ । इति॑ । तस्मा᳚त् । अ॒स्य॒ ।  \newline


\textbf{Krama Paata} \newline

त्वष्टा॑ ह॒तपु॑त्रः । ह॒तपु॑त्रो॒ वीन्द्र᳚म् । ह॒तपु॑त्र॒ इति॑ ह॒त - पु॒त्रः॒ । वीन्द्रꣳ॒॒ सोम᳚म् । वीन्द्र॒मिति॒ वि - इ॒न्द्र॒म् । सोम॒मा । आ ऽह॑रत् । अ॒ह॒र॒त् तस्मिन्न्॑ । तस्मि॒न्निन्द्रः॑ । इन्द्र॑ उपह॒वम् । उ॒प॒ह॒वमै᳚च्छत । उ॒प॒ह॒वमित्यु॑प - ह॒वम् । ऐ॒च्छ॒त॒ तम् । तम् न । नोप॑ । उपा᳚ह्वयत । अ॒ह्व॒य॒त॒ पु॒त्रम् । पु॒त्रम् मे᳚ । मे॒ ऽव॒धीः॒ । अ॒व॒धी॒रिति॑ । इति॒ सः । स य॑ज्ञ्वेश॒सम् । य॒ज्ञ्॒वे॒श॒सम् कृ॒त्वा । य॒ज्ञ्॒वे॒श॒समिति॑ यज्ञ् - वे॒श॒सम् । कृ॒त्वा प्रा॒सहा᳚ । प्रा॒सहा॒ सोम᳚म् । प्रा॒सहेति॑ प्र - सहा᳚ । सोम॑मपिबत् । अ॒पि॒ब॒त् तस्य॑ । तस्य॒ यत् । यद॒त्यशि॑ष्यत । अ॒त्यशि॑ष्यत॒ तत् । अ॒त्यशि॑ष्य॒तेत्य॑ति - अशि॑ष्यत । तत् त्वष्टा᳚ । त्वष्टा॑ ऽऽहव॒नीय᳚म् । आ॒ह॒व॒नीय॒मुप॑ । आ॒ह॒व॒नीय॒मित्या᳚ - ह॒व॒नीय᳚म् । उप॒ प्र । प्राव॑र्तयत् । अ॒व॒र्त॒य॒थ् स्वाहा᳚ । स्वाहेन्द्र॑शत्रुः । इन्द्र॑शत्रुर् वर्द्धस्व । इन्द्र॑शत्रु॒रितीन्द्र॑ - श॒त्रुः॒ । व॒र्द्ध॒स्वेति॑ । इति॒ यत् । यदव॑र्तयत् । अव॑र्तय॒त् तत् । तद् वृ॒त्रस्य॑ । वृ॒त्रस्य॑ वृत्र॒त्वम् । वृ॒त्र॒त्वम् ॅयत् । वृ॒त्र॒त्वमिति॑ वृत्र - त्वम् । यदब्र॑वीत् । अब्र॑वी॒थ् स्वाहा᳚ । स्वाहेन्द्र॑शत्रुः । इन्द्र॑शत्रुर् वर्द्धस्व । इन्द्र॑शत्रु॒रितीन्द्र॑ - श॒त्रुः॒ । व॒र्द्ध॒स्वेति॑ । इति॒ तस्मा᳚त् । तस्मा॑दस्य । अ॒स्येन्द्रः॑ \newline

\textbf{Jatai Paata} \newline

1. त्वष्टा॑ ह॒तपु॑त्रो ह॒तपु॑त्र॒ स्त्वष्टा॒ त्वष्टा॑ ह॒तपु॑त्रः । \newline
2. ह॒तपु॑त्रो॒ वीन्द्रं॒ ॅवीन्द्रꣳ॑ ह॒तपु॑त्रो ह॒तपु॑त्रो॒ वीन्द्र᳚म् । \newline
3. ह॒तपु॑त्र॒ इति॑ ह॒त - पु॒त्रः॒ । \newline
4. वीन्द्रꣳ॒॒ सोमꣳ॒॒ सोमं॒ ॅवीन्द्रं॒ ॅवीन्द्रꣳ॒॒ सोम᳚म् । \newline
5. वीन्द्र॒मिति॒ वि - इ॒न्द्र॒म् । \newline
6. सोम॒ मा सोमꣳ॒॒ सोम॒ मा । \newline
7. आ ऽह॑र दहर॒दा ऽह॑रत् । \newline
8. अ॒ह॒र॒त् तस्मिꣳ॒॒ स्तस्मि॑न् नहर दहर॒त् तस्मिन्न्॑ । \newline
9. तस्मि॒न् निन्द्र॒ इन्द्र॒ स्तस्मिꣳ॒॒ स्तस्मि॒न् निन्द्रः॑ । \newline
10. इन्द्र॑ उपह॒व मु॑पह॒व मिन्द्र॒ इन्द्र॑ उपह॒वम् । \newline
11. उ॒प॒ह॒व मै᳚च्छ तैच्छतो पह॒व मु॑पह॒व मै᳚च्छत । \newline
12. उ॒प॒ह॒वमित्यु॑प - ह॒वम् । \newline
13. ऐ॒च्छ॒त॒ तम् त मै᳚च्छ तैच्छत॒ तम् । \newline
14. तम् न न तम् तम् न । \newline
15. नोपोप॒ न नोप॑ । \newline
16. उपा᳚ ह्वयता ह्वय॒तोपोपा᳚ ह्वयत । \newline
17. अ॒ह्व॒य॒त॒ पु॒त्रम् पु॒त्र म॑ह्वयता ह्वयत पु॒त्रम् । \newline
18. पु॒त्रम् मे॑ मे पु॒त्रम् पु॒त्रम् मे᳚ । \newline
19. मे॒ ऽव॒धी॒र॒व॒धी॒र् मे॒ मे॒ ऽव॒धीः॒ । \newline
20. अ॒व॒धी॒ रिती त्य॑वधी रवधी॒ रिति॑ । \newline
21. इति॒ स स इतीति॒ सः । \newline
22. स य॑ज्ञ्वेश॒सं ॅय॑ज्ञ्वेश॒सꣳ स स य॑ज्ञ्वेश॒सम् । \newline
23. य॒ज्ञ्॒वे॒श॒सम् कृ॒त्वा कृ॒त्वा य॑ज्ञ्वेश॒सं ॅय॑ज्ञ्वेश॒सम् कृ॒त्वा । \newline
24. य॒ज्ञ्॒वे॒श॒समिति॑ यज्ञ् - वे॒श॒सम् । \newline
25. कृ॒त्वा प्रा॒सहा᳚ प्रा॒सहा॑ कृ॒त्वा कृ॒त्वा प्रा॒सहा᳚ । \newline
26. प्रा॒सहा॒ सोमꣳ॒॒ सोम॑म् प्रा॒सहा᳚ प्रा॒सहा॒ सोम᳚म् । \newline
27. प्रा॒सहेति॑ प्र - सहा᳚ । \newline
28. सोम॑ मपिब दपिब॒थ् सोमꣳ॒॒ सोम॑ मपिबत् । \newline
29. अ॒पि॒ब॒त् तस्य॒ तस्या॑पिब दपिब॒त् तस्य॑ । \newline
30. तस्य॒ यद् यत् तस्य॒ तस्य॒ यत् । \newline
31. यद॒त्यशि॑ष्यता॒ त्यशि॑ष्यत॒ यद् यद॒त्यशि॑ष्यत । \newline
32. अ॒त्यशि॑ष्यत॒ तत् तद॒त्यशि॑ष्यता॒ त्यशि॑ष्यत॒ तत् । \newline
33. अ॒त्यशि॑ष्य॒तेत्य॑ति - अशि॑ष्यत । \newline
34. तत् त्वष्टा॒ त्वष्टा॒ तत् तत् त्वष्टा᳚ । \newline
35. त्वष्टा॑ ऽऽहव॒नीय॑ माहव॒नीय॒म् त्वष्टा॒ त्वष्टा॑ ऽऽहव॒नीय᳚म् । \newline
36. आ॒ह॒व॒नीय॒ मुपोपा॑ हव॒नीय॑ माहव॒नीय॒ मुप॑ । \newline
37. आ॒ह॒व॒नीय॒मित्या᳚ - ह॒व॒नीय᳚म् । \newline
38. उप॒ प्र प्रोपोप॒ प्र । \newline
39. प्राव॑र्तय दवर्तय॒त् प्र प्राव॑र्तयत् । \newline
40. अ॒व॒र्त॒य॒थ् स्वाहा॒ स्वाहा॑ ऽवर्तय दवर्तय॒थ् स्वाहा᳚ । \newline
41. स्वाहेन्द्र॑शत्रु॒ रिन्द्र॑शत्रुः॒ स्वाहा॒ स्वाहेन्द्र॑शत्रुः । \newline
42. इन्द्र॑शत्रुर् वर्द्धस्व वर्द्ध॒स्वे न्द्र॑शत्रु॒ रिन्द्र॑शत्रुर् वर्द्धस्व । \newline
43. इन्द्र॑शत्रु॒रितीन्द्र॑ - श॒त्रुः॒ । \newline
44. व॒र्द्ध॒स्वे तीति॑ वर्द्धस्व वर्द्ध॒स्वे ति॑ । \newline
45. इति॒ यद् यदितीति॒ यत् । \newline
46. यदव॑र्तय॒ दव॑र्तय॒द् यद् यदव॑र्तयत् । \newline
47. अव॑र्तय॒त् तत् तदव॑र्तय॒ दव॑र्तय॒त् तत् । \newline
48. तद् वृ॒त्रस्य॑ वृ॒त्रस्य॒ तत् तद् वृ॒त्रस्य॑ । \newline
49. वृ॒त्रस्य॑ वृत्र॒त्वं ॅवृ॑त्र॒त्वं ॅवृ॒त्रस्य॑ वृ॒त्रस्य॑ वृत्र॒त्वम् । \newline
50. वृ॒त्र॒त्वं ॅयद् यद् वृ॑त्र॒त्वं ॅवृ॑त्र॒त्वं ॅयत् । \newline
51. वृ॒त्र॒त्वमिति॑ वृत्र - त्वम् । \newline
52. यदब्र॑वी॒ दब्र॑वी॒द् यद् यदब्र॑वीत् । \newline
53. अब्र॑वी॒थ् स्वाहा॒ स्वाहा ऽब्र॑वी॒ दब्र॑वी॒थ् स्वाहा᳚ । \newline
54. स्वाहेन्द्र॑शत्रु॒ रिन्द्र॑शत्रुः॒ स्वाहा॒ स्वाहेन्द्र॑शत्रुः । \newline
55. इन्द्र॑शत्रुर् वर्द्धस्व वर्द्ध॒स्वे न्द्र॑शत्रु॒ रिन्द्र॑शत्रुर् वर्द्धस्व । \newline
56. इन्द्र॑शत्रु॒रितीन्द्र॑ - श॒त्रुः॒ । \newline
57. व॒र्द्ध॒स्वे तीति॑ वर्द्धस्व वर्द्ध॒स्वे ति॑ । \newline
58. इति॒ तस्मा॒त् तस्मा॒ दितीति॒ तस्मा᳚त् । \newline
59. तस्मा॑ दस्यास्य॒ तस्मा॒त् तस्मा॑ दस्य । \newline
60. अ॒स्ये न्द्र॒ इन्द्रो᳚ ऽस्या॒स्ये न्द्रः॑ । \newline

\textbf{Ghana Paata } \newline

1. त्वष्टा॑ ह॒तपु॑त्रो ह॒तपु॑त्र॒ स्त्वष्टा॒ त्वष्टा॑ ह॒तपु॑त्रो॒ वीन्द्रं॒ ॅवीन्द्रꣳ॑ ह॒तपु॑त्र॒ स्त्वष्टा॒ त्वष्टा॑ ह॒तपु॑त्रो॒ वीन्द्र᳚म् । \newline
2. ह॒तपु॑त्रो॒ वीन्द्रं॒ ॅवीन्द्रꣳ॑ ह॒तपु॑त्रो ह॒तपु॑त्रो॒ वीन्द्रꣳ॒॒ सोमꣳ॒॒ सोमं॒ ॅवीन्द्रꣳ॑ ह॒तपु॑त्रो ह॒तपु॑त्रो॒ वीन्द्रꣳ॒॒ सोम᳚म् । \newline
3. ह॒तपु॑त्र॒ इति॑ ह॒त - पु॒त्रः॒ । \newline
4. वीन्द्रꣳ॒॒ सोमꣳ॒॒ सोमं॒ ॅवीन्द्रं॒ ॅवीन्द्रꣳ॒॒ सोम॒ मा सोमं॒ ॅवीन्द्रं॒ ॅवीन्द्रꣳ॒॒ सोम॒ मा । \newline
5. वीन्द्र॒मिति॒ वि - इ॒न्द्र॒म् । \newline
6. सोम॒ मा सोमꣳ॒॒ सोम॒ मा ऽह॑र दहर॒दा सोमꣳ॒॒ सोम॒ मा ऽह॑रत् । \newline
7. आ ऽह॑र दहर॒दा ऽह॑र॒त् तस्मिꣳ॒॒ स्तस्मि॑न् नहर॒दा ऽह॑र॒त् तस्मिन्न्॑ । \newline
8. अ॒ह॒र॒त् तस्मिꣳ॒॒ स्तस्मि॑न् नहर दहर॒त् तस्मि॒न् निन्द्र॒ इन्द्र॒ स्तस्मि॑न् नहर दहर॒त् तस्मि॒न् निन्द्रः॑ । \newline
9. तस्मि॒न् निन्द्र॒ इन्द्र॒ स्तस्मिꣳ॒॒ स्तस्मि॒न् निन्द्र॑ उपह॒व मु॑पह॒व मिन्द्र॒ स्तस्मिꣳ॒॒ स्तस्मि॒न् निन्द्र॑ उपह॒वम् । \newline
10. इन्द्र॑ उपह॒व मु॑पह॒व मिन्द्र॒ इन्द्र॑ उपह॒व मै᳚च्छ तैच्छतोपह॒व मिन्द्र॒ इन्द्र॑ उपह॒व मै᳚च्छत । \newline
11. उ॒प॒ह॒व मै᳚च्छ तैच्छतोपह॒व मु॑पह॒व मै᳚च्छत॒ तम् त मै᳚च्छतोपह॒व मु॑पह॒व मै᳚च्छत॒ तम् । \newline
12. उ॒प॒ह॒वमित्यु॑प - ह॒वम् । \newline
13. ऐ॒च्छ॒त॒ तम् त मै᳚च्छतैच्छत॒ तम् न न त मै᳚च्छतैच्छत॒ तम् न । \newline
14. तम् न न तम् तम् नो पोप॒ न तम् तम् नोप॑ । \newline
15. नोपोप॒ न नोपा᳚ह्वयता ह्वय॒तोप॒ न नोपा᳚ह्वयत । \newline
16. उपा᳚ह्वयता ह्वय॒तोपोपा᳚ ह्वयत पु॒त्रम् पु॒त्र म॑ह्वय॒तोपोपा᳚ ह्वयत पु॒त्रम् । \newline
17. अ॒ह्व॒य॒त॒ पु॒त्रम् पु॒त्र म॑ह्वयता ह्वयत पु॒त्रम् मे॑ मे पु॒त्र म॑ह्वयता ह्वयत पु॒त्रम् मे᳚ । \newline
18. पु॒त्रम् मे॑ मे पु॒त्रम् पु॒त्रम् मे॑ ऽवधी रवधीर् मे पु॒त्रम् पु॒त्रम् मे॑ ऽवधीः । \newline
19. मे॒ ऽव॒धी॒ र॒व॒धी॒र् मे॒ मे॒ ऽव॒धी॒ रितीत्य॑वधीर् मे मे ऽवधी॒रिति॑ । \newline
20. अ॒व॒धी॒ रितीत्य॑वधी रवधी॒रिति॒ स स इत्य॑वधी रवधी॒रिति॒ सः । \newline
21. इति॒ स स इतीति॒ स य॑ज्ञ्वेश॒सं ॅय॑ज्ञ्वेश॒सꣳ स इतीति॒ स य॑ज्ञ्वेश॒सम् । \newline
22. स य॑ज्ञ्वेश॒सं ॅय॑ज्ञ्वेश॒सꣳ स स य॑ज्ञ्वेश॒सम् कृ॒त्वा कृ॒त्वा य॑ज्ञ्वेश॒सꣳ स स य॑ज्ञ्वेश॒सम् कृ॒त्वा । \newline
23. य॒ज्ञ्॒वे॒श॒सम् कृ॒त्वा कृ॒त्वा य॑ज्ञ्वेश॒सं ॅय॑ज्ञ्वेश॒सम् कृ॒त्वा प्रा॒सहा᳚ प्रा॒सहा॑ कृ॒त्वा य॑ज्ञ्वेश॒सं ॅय॑ज्ञ्वेश॒सम् कृ॒त्वा प्रा॒सहा᳚ । \newline
24. य॒ज्ञ्॒वे॒श॒समिति॑ यज्ञ् - वे॒श॒सम् । \newline
25. कृ॒त्वा प्रा॒सहा᳚ प्रा॒सहा॑ कृ॒त्वा कृ॒त्वा प्रा॒सहा॒ सोमꣳ॒॒ सोम॑म् प्रा॒सहा॑ कृ॒त्वा कृ॒त्वा प्रा॒सहा॒ सोम᳚म् । \newline
26. प्रा॒सहा॒ सोमꣳ॒॒ सोम॑म् प्रा॒सहा᳚ प्रा॒सहा॒ सोम॑ मपिब दपिब॒थ् सोम॑म् प्रा॒सहा᳚ प्रा॒सहा॒ सोम॑ मपिबत् । \newline
27. प्रा॒सहेति॑ प्र - सहा᳚ । \newline
28. सोम॑ मपिब दपिब॒थ् सोमꣳ॒॒ सोम॑ मपिब॒त् तस्य॒ तस्या॑पिब॒थ् सोमꣳ॒॒ सोम॑ मपिब॒त् तस्य॑ । \newline
29. अ॒पि॒ब॒त् तस्य॒ तस्या॑पिब दपिब॒त् तस्य॒ यद् यत् तस्या॑पिब दपिब॒त् तस्य॒ यत् । \newline
30. तस्य॒ यद् यत् तस्य॒ तस्य॒ यद॒त्यशि॑ष्यता॒ त्यशि॑ष्यत॒ यत् तस्य॒ तस्य॒ यद॒त्यशि॑ष्यत । \newline
31. यद॒त्यशि॑ष्यता॒ त्यशि॑ष्यत॒ यद् यद॒त्यशि॑ष्यत॒ तत् तद॒त्यशि॑ष्यत॒ यद् यद॒त्यशि॑ष्यत॒ तत् । \newline
32. अ॒त्यशि॑ष्यत॒ तत् तद॒त्यशि॑ष्यता॒ त्यशि॑ष्यत॒ तत् त्वष्टा॒ त्वष्टा॒ तद॒त्यशि॑ष्यता॒ त्यशि॑ष्यत॒ तत् त्वष्टा᳚ । \newline
33. अ॒त्यशि॑ष्य॒तेत्य॑ति - अशि॑ष्यत । \newline
34. तत् त्वष्टा॒ त्वष्टा॒ तत् तत् त्वष्टा॑ ऽऽहव॒नीय॑ माहव॒नीय॒म् त्वष्टा॒ तत् तत् त्वष्टा॑ ऽऽहव॒नीय᳚म् । \newline
35. त्वष्टा॑ ऽऽहव॒नीय॑ माहव॒नीय॒म् त्वष्टा॒ त्वष्टा॑ ऽऽहव॒नीय॒ मुपोपा॑हव॒नीय॒म् त्वष्टा॒ त्वष्टा॑ ऽऽहव॒नीय॒ मुप॑ । \newline
36. आ॒ह॒व॒नीय॒ मुपोपा॑हव॒नीय॑ माहव॒नीय॒ मुप॒ प्र प्रोपा॑हव॒नीय॑ माहव॒नीय॒ मुप॒ प्र । \newline
37. आ॒ह॒व॒नीय॒मित्या᳚ - ह॒व॒नीय᳚म् । \newline
38. उप॒ प्र प्रोपोप॒ प्राव॑र्तय दवर्तय॒त् प्रोपोप॒ प्राव॑र्तयत् । \newline
39. प्राव॑र्तय दवर्तय॒त् प्र प्राव॑र्तय॒थ् स्वाहा॒ स्वाहा॑ ऽवर्तय॒त् प्र प्राव॑र्तय॒थ् स्वाहा᳚ । \newline
40. अ॒व॒र्त॒य॒थ् स्वाहा॒ स्वाहा॑ ऽवर्तय दवर्तय॒थ् स्वाहेन्द्र॑शत्रु॒ रिन्द्र॑शत्रुः॒ स्वाहा॑ ऽवर्तय दवर्तय॒थ् स्वाहेन्द्र॑शत्रुः । \newline
41. स्वाहेन्द्र॑शत्रु॒ रिन्द्र॑शत्रुः॒ स्वाहा॒ स्वाहेन्द्र॑शत्रुर् वर्द्धस्व वर्द्ध॒स्वे न्द्र॑शत्रुः॒ स्वाहा॒ स्वाहेन्द्र॑शत्रुर् वर्द्धस्व । \newline
42. इन्द्र॑शत्रुर् वर्द्धस्व वर्द्ध॒स्वे न्द्र॑शत्रु॒ रिन्द्र॑शत्रुर् वर्द्ध॒स्वे तीति॑ वर्द्ध॒स्वे न्द्र॑शत्रु॒ रिन्द्र॑शत्रुर् वर्द्ध॒स्वे ति॑ । \newline
43. इन्द्र॑शत्रु॒रितीन्द्र॑ - श॒त्रुः॒ । \newline
44. व॒र्द्ध॒स्वे तीति॑ वर्द्धस्व वर्द्ध॒स्वे ति॒ यद् यदिति॑ वर्द्धस्व वर्द्ध॒स्वे ति॒ यत् । \newline
45. इति॒ यद् यदितीति॒ यदव॑र्तय॒ दव॑र्तय॒द् यदितीति॒ यदव॑र्तयत् । \newline
46. यदव॑र्तय॒ दव॑र्तय॒द् यद् यदव॑र्तय॒त् तत् तदव॑र्तय॒द् यद् यदव॑र्तय॒त् तत् । \newline
47. अव॑र्तय॒त् तत् तदव॑र्तय॒ दव॑र्तय॒त् तद् वृ॒त्रस्य॑ वृ॒त्रस्य॒ तदव॑र्तय॒ दव॑र्तय॒त् तद् वृ॒त्रस्य॑ । \newline
48. तद् वृ॒त्रस्य॑ वृ॒त्रस्य॒ तत् तद् वृ॒त्रस्य॑ वृत्र॒त्वं ॅवृ॑त्र॒त्वं ॅवृ॒त्रस्य॒ तत् तद् वृ॒त्रस्य॑ वृत्र॒त्वम् । \newline
49. वृ॒त्रस्य॑ वृत्र॒त्वं ॅवृ॑त्र॒त्वं ॅवृ॒त्रस्य॑ वृ॒त्रस्य॑ वृत्र॒त्वं ॅयद् यद् वृ॑त्र॒त्वं ॅवृ॒त्रस्य॑ वृ॒त्रस्य॑ वृत्र॒त्वं ॅयत् । \newline
50. वृ॒त्र॒त्वं ॅयद् यद् वृ॑त्र॒त्वं ॅवृ॑त्र॒त्वं ॅयदब्र॑वी॒ दब्र॑वी॒द् यद् वृ॑त्र॒त्वं ॅवृ॑त्र॒त्वं ॅयदब्र॑वीत् । \newline
51. वृ॒त्र॒त्वमिति॑ वृत्र - त्वम् । \newline
52. यदब्र॑वी॒ दब्र॑वी॒द् यद् यदब्र॑वी॒थ् स्वाहा॒ स्वाहा ऽब्र॑वी॒द् यद् यदब्र॑वी॒थ् स्वाहा᳚ । \newline
53. अब्र॑वी॒थ् स्वाहा॒ स्वाहा ऽब्र॑वी॒ दब्र॑वी॒थ् स्वाहेन्द्र॑शत्रु॒ रिन्द्र॑शत्रुः॒ स्वाहा ऽब्र॑वी॒ दब्र॑वी॒थ् स्वाहेन्द्र॑शत्रुः । \newline
54. स्वाहेन्द्र॑शत्रु॒ रिन्द्र॑शत्रुः॒ स्वाहा॒ स्वाहेन्द्र॑शत्रुर् वर्द्धस्व वर्द्ध॒स्वे न्द्र॑शत्रुः॒ स्वाहा॒ स्वाहेन्द्र॑शत्रुर् वर्द्धस्व । \newline
55. इन्द्र॑शत्रुर् वर्द्धस्व वर्द्ध॒स्वे न्द्र॑शत्रु॒ रिन्द्र॑शत्रुर् वर्द्ध॒स्वे तीति॑ वर्द्ध॒स्वे न्द्र॑शत्रु॒ रिन्द्र॑शत्रुर् वर्द्ध॒स्वे ति॑ । \newline
56. इन्द्र॑शत्रु॒रितीन्द्र॑ - श॒त्रुः॒ । \newline
57. व॒र्द्ध॒स्वे तीति॑ वर्द्धस्व वर्द्ध॒स्वे ति॒ तस्मा॒त् तस्मा॒दिति॑ वर्द्धस्व वर्द्ध॒स्वे ति॒ तस्मा᳚त् । \newline
58. इति॒ तस्मा॒त् तस्मा॒ दितीति॒ तस्मा॑ दस्यास्य॒ तस्मा॒दितीति॒ तस्मा॑दस्य । \newline
59. तस्मा॑ दस्यास्य॒ तस्मा॒त् तस्मा॑द॒स्ये न्द्र॒ इन्द्रो᳚ ऽस्य॒ तस्मा॒त् तस्मा॑द॒स्ये न्द्रः॑ । \newline
60. अ॒स्ये न्द्र॒ इन्द्रो᳚ ऽस्या॒स्ये न्द्रः॒ शत्रुः॒ शत्रु॒रिन्द्रो᳚ ऽस्या॒स्ये न्द्रः॒ शत्रुः॑ । \newline
\pagebreak
\markright{ TS 2.5.2.2  \hfill https://www.vedavms.in \hfill}
\addcontentsline{toc}{section}{ TS 2.5.2.2 }
\section*{ TS 2.5.2.2 }

\textbf{TS 2.5.2.2 } \newline
\textbf{Samhita Paata} \newline

-न्द्रः॒ शत्रु॑रभव॒थ् स स॒भंव॑न्न॒ग्नीषोमा॑व॒भि सम॑भव॒थ् स इ॑षुमा॒त्रमि॑षुमात्रं॒ ॅविष्व॑ङ्ङवर्द्धत॒ स इ॒मां ॅलो॒कान॑वृणो॒द्यदि॒मां ॅलो॒कानवृ॑णो॒त् तद्-वृ॒त्रस्य॑ वृत्र॒त्वं तस्मा॒दिन्द्रो॑ऽबिभे॒थ् स प्र॒जाप॑ति॒मुपा॑धाव॒-च्छत्रु॑र्मेऽ ज॒नीति॒ तस्मै॒ वज्रꣳ॑ सि॒क्त्वा प्राय॑च्छदे॒तेन॑ ज॒हीति॒ तेना॒भ्या॑यत॒ ताव॑ब्रूताम॒ग्नीषोमौ॒ मा - [  ] \newline

\textbf{Pada Paata} \newline

इन्द्रः॑ । शत्रुः॑ । अ॒भ॒व॒त् । सः । स॒भंव॒न्निति॑ सं - भवन्न्॑ । अ॒ग्नीषोमा॒वित्य॒ग्नी - सोमौ᳚ । अ॒भि । समिति॑ । अ॒भ॒व॒त् । सः । इ॒षु॒मा॒त्रमि॑षुमात्र॒मिती॑षुमा॒त्रं - इ॒षु॒मा॒त्र॒म् । विष्वङ्॑ । अ॒व॒र्द्ध॒त॒ । सः । इ॒मान् । लो॒कान् । अ॒वृ॒णो॒त् । यत् । इ॒मान् । लो॒कान् । अवृ॑णोत् । तत् । वृ॒त्रस्य॑ । वृ॒त्र॒त्वमिति॑ वृत्र - त्वम् । तस्मा᳚त् । इन्द्रः॑ । अ॒बि॒भे॒त् । सः । प्र॒जाप॑ति॒मिति॑ प्र॒जा-प॒ति॒म् । उपेति॑ । अ॒धा॒व॒त् । शत्रुः॑ । मे॒ । अ॒ज॒नि॒ । इति॑ । तस्मै᳚ । वज्र᳚म् । सि॒क्त्वा । प्रेति॑ । अ॒य॒च्छ॒त् । ए॒तेन॑ । ज॒हि॒ । इति॑ । तेन॑ । अ॒भीति॑ । आ॒य॒त॒ । तौ । अ॒ब्रू॒ता॒म् । अ॒ग्नीषोमा॒वित्य॒ग्नी - सोमौ᳚ । मा ।  \newline


\textbf{Krama Paata} \newline

इन्द्रः॒ शत्रुः॑ । शत्रु॑रभवत् । अ॒भ॒व॒थ् सः । स स॒म्भवन्न्॑ । स॒म्भव॑न्न॒ग्नीषोमौ᳚ । स॒म्भव॒न्निति॑ सम् - भवन्न्॑ । अ॒ग्नीषोमा॑व॒भि । अ॒ग्नीषोमा॒वित्य॒ग्नी - सोमौ᳚ । अ॒भि सम् । सम॑भवत् । अ॒भ॒व॒थ् सः । स इ॑षुमा॒त्रमि॑षुमात्रम् । इ॒षु॒मा॒त्रमि॑षुमात्र॒म् ॅविष्वङ्ङ्॑ । इ॒षु॒मा॒त्रमि॑षुमात्र॒मिती॑षुमा॒त्रम् - इ॒षु॒मा॒त्र॒म् । विष्व॑ङ्ङवर्द्धत । अ॒व॒र्द्ध॒त॒ सः । स इ॒मान् । इ॒मान् ॅलो॒कान् । लो॒कान॑वृणोत् । अ॒वृ॒णो॒द् यत् । यदि॒मान् । इ॒मान् ॅलो॒कान् । लो॒कानवृ॑णोत् । अवृ॑णो॒त् तत् । तद् वृ॒त्रस्य॑ । वृ॒त्रस्य॑ वृत्र॒त्वम् । वृ॒त्र॒त्वम् तस्मा᳚त् । वृ॒त्र॒त्वमिति॑ वृत्र - त्वम् । तस्मा॒दिन्द्रः॑ । इन्द्रो॑ऽ बिभेत् । अ॒बि॒भे॒थ् सः । स प्र॒जाप॑तिम् । प्र॒जाप॑ति॒मुप॑ । प्र॒जाप॑ति॒मिति॑ प्र॒जा - प॒ति॒म् । उपा॑धावत् । अ॒धा॒व॒च्छत्रुः॑ । शत्रु॑र् मे । मे॒ ऽज॒नि॒ । अ॒ज॒नीति॑ । इति॒ तस्मै᳚ । तस्मै॒ वज्र᳚म् । वज्रꣳ॑ सि॒क्त्वा । सि॒क्त्वा प्र । प्राय॑च्छत् । अ॒य॒च्छ॒दे॒तेन॑ । ए॒तेन॑ जहि । ज॒हीति॑ । इति॒ तेन॑ । तेना॒भि । अ॒भ्या॑यत । आ॒य॒त॒ तौ । ताव॑ब्रूताम् । अ॒ब्रू॒ता॒म॒ग्नीषोमौ᳚ । अ॒ग्नीषोमौ॒ मा । अ॒ग्नीषोमा॒वित्य॒ग्नी - सोमौ᳚ । मा प्र \newline

\textbf{Jatai Paata} \newline

1. इन्द्रः॒ शत्रुः॒ शत्रु॒ रिन्द्र॒ इन्द्रः॒ शत्रुः॑ । \newline
2. शत्रु॑ रभव दभव॒च् छत्रुः॒ शत्रु॑ रभवत् । \newline
3. अ॒भ॒व॒थ् स सो॑ ऽभव दभव॒थ् सः । \newline
4. स स॒म्भवन्᳚ थ्स॒म्भव॒न् थ्स स स॒म्भवन्न्॑ । \newline
5. स॒म्भव॑न् न॒ग्नीषोमा॑ व॒ग्नीषोमौ॑ स॒म्भवन्᳚ थ्स॒म्भव॑न् न॒ग्नीषोमौ᳚ । \newline
6. स॒म्भव॒न्निति॑ सं - भवन्न्॑ । \newline
7. अ॒ग्नीषोमा॑ व॒भ्या᳚(1॒)भ्य॑ग्नीषोमा॑ व॒ग्नीषोमा॑ व॒भि । \newline
8. अ॒ग्नीषोमा॒वित्य॒ग्नी - सोमौ᳚ । \newline
9. अ॒भि सꣳ स म॒भ्य॑भि सम् । \newline
10. स म॑भव दभव॒थ् सꣳ स म॑भवत् । \newline
11. अ॒भ॒व॒थ् स सो॑ ऽभव दभव॒थ् सः । \newline
12. स इ॑षुमा॒त्रमि॑षुमात्र मिषुमा॒त्रमि॑षुमात्रꣳ॒॒ स स इ॑षुमा॒त्रमि॑षुमात्रम् । \newline
13. इ॒षु॒मा॒त्रमि॑षुमात्रं॒ ॅविष्व॒ङ् विष्व॑ङ् ङिषुमा॒त्रमि॑षुमात्र मिषुमा॒त्रमि॑षुमात्रं॒ ॅविष्वङ्॑ । \newline
14. इ॒षु॒मा॒त्रमि॑षुमात्र॒मिती॑षुमा॒त्रं - इ॒षु॒मा॒त्र॒म् । \newline
15. विष्व॑ङ् ङवर्द्धता वर्द्धत॒ विष्व॒ङ् विष्व॑ङ् ङवर्द्धत । \newline
16. अ॒व॒र्द्ध॒त॒ स सो॑ ऽवर्द्धता वर्द्धत॒ सः । \newline
17. स इ॒मा नि॒मान् थ्स स इ॒मान् । \newline
18. इ॒मान् ॅलो॒कान् ॅलो॒का नि॒मा नि॒मान् ॅलो॒कान् । \newline
19. लो॒का न॑वृणो दवृणो ल्लो॒कान् ॅलो॒का न॑वृणोत् । \newline
20. अ॒वृ॒णो॒द् यद् यद॑वृणो दवृणो॒द् यत् । \newline
21. यदि॒मा नि॒मान्. यद् यदि॒मान् । \newline
22. इ॒मान् ॅलो॒कान् ॅलो॒का नि॒मा नि॒मान् ॅलो॒कान् । \newline
23. लो॒का नवृ॑णो॒ दवृ॑णो ल्लो॒कान् ॅलो॒का नवृ॑णोत् । \newline
24. अवृ॑णो॒त् तत् तदवृ॑णो॒ दवृ॑णो॒त् तत् । \newline
25. तद् वृ॒त्रस्य॑ वृ॒त्रस्य॒ तत् तद् वृ॒त्रस्य॑ । \newline
26. वृ॒त्रस्य॑ वृत्र॒त्वं ॅवृ॑त्र॒त्वं ॅवृ॒त्रस्य॑ वृ॒त्रस्य॑ वृत्र॒त्वम् । \newline
27. वृ॒त्र॒त्वम् तस्मा॒त् तस्मा᳚द् वृत्र॒त्वं ॅवृ॑त्र॒त्वम् तस्मा᳚त् । \newline
28. वृ॒त्र॒त्वमिति॑ वृत्र - त्वम् । \newline
29. तस्मा॒ दिन्द्र॒ इन्द्र॒ स्तस्मा॒त् तस्मा॒ दिन्द्रः॑ । \newline
30. इन्द्रो॑ ऽबिभे दबिभे॒ दिन्द्र॒ इन्द्रो॑ ऽबिभेत् । \newline
31. अ॒बि॒भे॒थ् स सो॑ ऽबिभे दबिभे॒थ् सः । \newline
32. स प्र॒जाप॑तिम् प्र॒जाप॑तिꣳ॒॒ स स प्र॒जाप॑तिम् । \newline
33. प्र॒जाप॑ति॒ मुपोप॑ प्र॒जाप॑तिम् प्र॒जाप॑ति॒ मुप॑ । \newline
34. प्र॒जाप॑ति॒मिति॑ प्र॒जा - प॒ति॒म् । \newline
35. उपा॑ धाव दधाव॒ दुपोपा॑ धावत् । \newline
36. अ॒धा॒व॒च् छत्रुः॒ शत्रु॑ रधाव दधाव॒च् छत्रुः॑ । \newline
37. शत्रु॑र् मे मे॒ शत्रुः॒ शत्रु॑र् मे । \newline
38. मे॒ ऽज॒न्य॒ज॒नि॒ मे॒ मे॒ ऽज॒नि॒ । \newline
39. अ॒ज॒नीती त्य॑ज न्यज॒नीति॑ । \newline
40. इति॒ तस्मै॒ तस्मा॒ इतीति॒ तस्मै᳚ । \newline
41. तस्मै॒ वज्रं॒ ॅवज्र॒म् तस्मै॒ तस्मै॒ वज्र᳚म् । \newline
42. वज्रꣳ॑ सि॒क्त्वा सि॒क्त्वा वज्रं॒ ॅवज्रꣳ॑ सि॒क्त्वा । \newline
43. सि॒क्त्वा प्र प्र सि॒क्त्वा सि॒क्त्वा प्र । \newline
44. प्राय॑च्छ दयच्छ॒त् प्र प्राय॑च्छत् । \newline
45. अ॒य॒च्छ॒ दे॒ते नै॒तेना॑ यच्छ दयच्छ दे॒तेन॑ । \newline
46. ए॒तेन॑ जहि जह्ये॒ते नै॒तेन॑ जहि । \newline
47. ज॒हीतीति॑ जहि ज॒हीति॑ । \newline
48. इति॒ तेन॒ तेने तीति॒ तेन॑ । \newline
49. तेना॒भ्य॑भि तेन॒ तेना॒भि । \newline
50. अ॒भ्या॑यता यता॒भ्या᳚(1॒)भ्या॑यत । \newline
51. आ॒य॒त॒ तौ ता वा॑यतायत॒ तौ । \newline
52. ता व॑ब्रूता मब्रूता॒म् तौ ता व॑ब्रूताम् । \newline
53. अ॒ब्रू॒ता॒ म॒ग्नीषोमा॑ व॒ग्नीषोमा॑ वब्रूता मब्रूता म॒ग्नीषोमौ᳚ । \newline
54. अ॒ग्नीषोमौ॒ मा मा ऽग्नीषोमा॑ व॒ग्नीषोमौ॒ मा । \newline
55. अ॒ग्नीषोमा॒वित्य॒ग्नी - सोमौ᳚ । \newline
56. मा प्र प्र मा मा प्र । \newline

\textbf{Ghana Paata } \newline

1. इन्द्रः॒ शत्रुः॒ शत्रु॒रिन्द्र॒ इन्द्रः॒ शत्रु॑रभव दभव॒च् छत्रु॒रिन्द्र॒ इन्द्रः॒ शत्रु॑ रभवत् । \newline
2. शत्रु॑रभव दभव॒च् छत्रुः॒ शत्रु॑ रभव॒थ् स सो॑ ऽभव॒च्छत्रुः॒ शत्रु॑ रभव॒थ् सः । \newline
3. अ॒भ॒व॒थ् स सो॑ ऽभव दभव॒थ् स स॒म्भवन्᳚ थ्स॒म्भव॒न् थ्सो॑ ऽभव दभव॒थ् स स॒म्भवन्न्॑ । \newline
4. स स॒म्भवन्᳚ थ्स॒म्भव॒न् थ्स स स॒म्भव॑न् न॒ग्नीषोमा॑ व॒ग्नीषोमौ॑ स॒म्भव॒न् थ्स स स॒म्भव॑न् न॒ग्नीषोमौ᳚ । \newline
5. स॒म्भव॑न् न॒ग्नीषोमा॑ व॒ग्नीषोमौ॑ स॒म्भवन्᳚ थ्स॒म्भव॑न् न॒ग्नीषोमा॑ व॒भ्या᳚(1॒)भ्य॑ग्नीषोमौ॑ स॒म्भवन्᳚ थ्स॒म्भव॑न् न॒ग्नीषोमा॑ व॒भि । \newline
6. स॒म्भव॒न्निति॑ सं - भवन्न्॑ । \newline
7. अ॒ग्नीषोमा॑ व॒भ्या᳚(1॒)भ्य॑ग्नीषोमा॑ व॒ग्नीषोमा॑ व॒भि सꣳ स म॒भ्य॑ग्नीषोमा॑ व॒ग्नीषोमा॑ व॒भि सम् । \newline
8. अ॒ग्नीषोमा॒वित्य॒ग्नी - सोमौ᳚ । \newline
9. अ॒भि सꣳ स म॒भ्य॑भि स म॑भव दभव॒थ् स म॒भ्य॑भि स म॑भवत् । \newline
10. स म॑भव दभव॒थ् सꣳ स म॑भव॒थ् स सो॑ ऽभव॒थ् सꣳ स म॑भव॒थ् सः । \newline
11. अ॒भ॒व॒थ् स सो॑ ऽभव दभव॒थ् स इ॑षुमा॒त्रमि॑षुमात्र मिषुमा॒त्रमि॑षुमात्रꣳ॒॒ सो॑ ऽभव दभव॒थ् स इ॑षुमा॒त्रमि॑षुमात्रम् । \newline
12. स इ॑षुमा॒त्रमि॑षुमात्र मिषुमा॒त्रमि॑षुमात्रꣳ॒॒ स स इ॑षुमा॒त्रमि॑षुमात्रं॒ ॅविष्व॒ङ् 
विष्व॑ङ् ङिषुमा॒त्रमि॑षुमात्रꣳ॒॒ स स इ॑षुमा॒त्रमि॑षुमात्रं॒ ॅविष्वङ्॑ । \newline
13. इ॒षु॒मा॒त्रमि॑षुमात्रं॒ ॅविष्व॒ङ् विष्व॑ङ् ङिषुमा॒त्रमि॑षुमात्र मिषुमा॒त्रमि॑षुमात्रं॒ ॅविष्व॑ङ् ङवर्द्धता वर्द्धत॒ विष्व॑ङ् ङिषुमा॒त्रमि॑षुमात्र मिषुमा॒त्रमि॑षुमात्रं॒ ॅविष्व॑ङ् ङवर्द्धत । \newline
14. इ॒षु॒मा॒त्रमि॑षुमात्र॒मिती॑षुमा॒त्रं - इ॒षु॒मा॒त्र॒म् । \newline
15. विष्व॑ङ् ङवर्द्धता वर्द्धत॒ विष्व॒ङ् विष्व॑ङ् ङवर्द्धत॒ स सो॑ ऽवर्द्धत॒ विष्व॒ङ् विष्व॑ङ् ङवर्द्धत॒ सः । \newline
16. अ॒व॒र्द्ध॒त॒ स सो॑ ऽवर्द्धता वर्द्धत॒ स इ॒मा नि॒मान् थ्सो॑ ऽवर्द्धता वर्द्धत॒ स इ॒मान् । \newline
17. स इ॒मा नि॒मान् थ्स स इ॒मान् ॅलो॒कान् ॅलो॒का नि॒मान् थ्स स इ॒मान् ॅलो॒कान् । \newline
18. इ॒मान् ॅलो॒कान् ॅलो॒का नि॒मा नि॒मान् ॅलो॒का न॑वृणो दवृणो ल्लो॒का नि॒मा नि॒मान् ॅलो॒का न॑वृणोत् । \newline
19. लो॒का न॑वृणो दवृणो ल्लो॒कान् ॅलो॒का न॑वृणो॒द् यद् यद॑वृणो ल्लो॒कान् ॅलो॒का न॑वृणो॒द् यत् । \newline
20. अ॒वृ॒णो॒द् यद् यद॑वृणो दवृणो॒द् यदि॒मा नि॒मान्. यद॑वृणो दवृणो॒द् यदि॒मान् । \newline
21. यदि॒मा नि॒मान्. यद् यदि॒मान् ॅलो॒कान् ॅलो॒का नि॒मान्. यद् यदि॒मान् ॅलो॒कान् । \newline
22. इ॒मान् ॅलो॒कान् ॅलो॒का नि॒मा नि॒मान् ॅलो॒का नवृ॑णो॒ दवृ॑णो ल्लो॒का नि॒मा नि॒मान् ॅलो॒का नवृ॑णोत् । \newline
23. लो॒का नवृ॑णो॒ दवृ॑णो ल्लो॒कान् ॅलो॒का नवृ॑णो॒त् तत् तदवृ॑णो ल्लो॒कान् ॅलो॒का नवृ॑णो॒त् तत् । \newline
24. अवृ॑णो॒त् तत् तदवृ॑णो॒ दवृ॑णो॒त् तद् वृ॒त्रस्य॑ वृ॒त्रस्य॒ तदवृ॑णो॒ दवृ॑णो॒त् तद् वृ॒त्रस्य॑ । \newline
25. तद् वृ॒त्रस्य॑ वृ॒त्रस्य॒ तत् तद् वृ॒त्रस्य॑ वृत्र॒त्वं ॅवृ॑त्र॒त्वं ॅवृ॒त्रस्य॒ तत् तद् वृ॒त्रस्य॑ वृत्र॒त्वम् । \newline
26. वृ॒त्रस्य॑ वृत्र॒त्वं ॅवृ॑त्र॒त्वं ॅवृ॒त्रस्य॑ वृ॒त्रस्य॑ वृत्र॒त्वम् तस्मा॒त् तस्मा᳚द् वृत्र॒त्वं ॅवृ॒त्रस्य॑ वृ॒त्रस्य॑ वृत्र॒त्वम् तस्मा᳚त् । \newline
27. वृ॒त्र॒त्वम् तस्मा॒त् तस्मा᳚द् वृत्र॒त्वं ॅवृ॑त्र॒त्वम् तस्मा॒दिन्द्र॒ इन्द्र॒ स्तस्मा᳚द् वृत्र॒त्वं ॅवृ॑त्र॒त्वम् तस्मा॒दिन्द्रः॑ । \newline
28. वृ॒त्र॒त्वमिति॑ वृत्र - त्वम् । \newline
29. तस्मा॒दिन्द्र॒ इन्द्र॒ स्तस्मा॒त् तस्मा॒दिन्द्रो॑ ऽबिभे दबिभे॒ दिन्द्र॒ स्तस्मा॒त् तस्मा॒दिन्द्रो॑ ऽबिभेत् । \newline
30. इन्द्रो॑ ऽबिभे दबिभे॒ दिन्द्र॒ इन्द्रो॑ ऽबिभे॒थ् स सो॑ ऽबिभे॒दिन्द्र॒ इन्द्रो॑ ऽबिभे॒थ् सः । \newline
31. अ॒बि॒भे॒थ् स सो॑ ऽबिभे दबिभे॒थ् स प्र॒जाप॑तिम् प्र॒जाप॑तिꣳ॒॒ सो॑ ऽबिभे दबिभे॒थ् स प्र॒जाप॑तिम् । \newline
32. स प्र॒जाप॑तिम् प्र॒जाप॑तिꣳ॒॒ स स प्र॒जाप॑ति॒ मुपोप॑ प्र॒जाप॑तिꣳ॒॒ स स प्र॒जाप॑ति॒ मुप॑ । \newline
33. प्र॒जाप॑ति॒ मुपोप॑ प्र॒जाप॑तिम् प्र॒जाप॑ति॒ मुपा॑धाव दधाव॒ दुप॑ प्र॒जाप॑तिम् प्र॒जाप॑ति॒ मुपा॑धावत् । \newline
34. प्र॒जाप॑ति॒मिति॑ प्र॒जा - प॒ति॒म् । \newline
35. उपा॑धाव दधाव॒ दुपोपा॑धाव॒च् छत्रुः॒ शत्रु॑रधाव॒ दुपोपा॑धाव॒च् छत्रुः॑ । \newline
36. अ॒धा॒व॒च् छत्रुः॒ शत्रु॑ रधाव दधाव॒च् छत्रु॑र् मे मे॒ शत्रु॑ रधाव दधाव॒च् छत्रु॑र् मे । \newline
37. शत्रु॑र् मे मे॒ शत्रुः॒ शत्रु॑र् मे ऽजन्यजनि मे॒ शत्रुः॒ शत्रु॑र् मे ऽजनि । \newline
38. मे॒ ऽज॒न्य॒ज॒नि॒ मे॒ मे॒ ऽज॒नीती त्य॑जनि मे मे ऽज॒नीति॑ । \newline
39. अ॒ज॒नीती त्य॑ज न्यज॒नीति॒ तस्मै॒ तस्मा॒ इत्य॑ज न्यज॒नीति॒ तस्मै᳚ । \newline
40. इति॒ तस्मै॒ तस्मा॒ इतीति॒ तस्मै॒ वज्रं॒ ॅवज्र॒म् तस्मा॒ इतीति॒ तस्मै॒ वज्र᳚म् । \newline
41. तस्मै॒ वज्रं॒ ॅवज्र॒म् तस्मै॒ तस्मै॒ वज्रꣳ॑ सि॒क्त्वा सि॒क्त्वा वज्र॒म् तस्मै॒ तस्मै॒ वज्रꣳ॑ सि॒क्त्वा । \newline
42. वज्रꣳ॑ सि॒क्त्वा सि॒क्त्वा वज्रं॒ ॅवज्रꣳ॑ सि॒क्त्वा प्र प्र सि॒क्त्वा वज्रं॒ ॅवज्रꣳ॑ सि॒क्त्वा प्र । \newline
43. सि॒क्त्वा प्र प्र सि॒क्त्वा सि॒क्त्वा प्राय॑च्छ दयच्छ॒त् प्र सि॒क्त्वा सि॒क्त्वा प्राय॑च्छत् । \newline
44. प्राय॑च्छ दयच्छ॒त् प्र प्राय॑च्छ दे॒ते नै॒तेना॑ यच्छ॒त् प्र प्राय॑च्छ दे॒तेन॑ । \newline
45. अ॒य॒च्छ॒ दे॒तेनै॒तेना॑ यच्छ दयच्छ दे॒तेन॑ जहि जह्ये॒तेना॑ यच्छ दयच्छ दे॒तेन॑ जहि । \newline
46. ए॒तेन॑ जहि जह्ये॒तेनै॒तेन॑ ज॒हीतीति॑ जह्ये॒तेनै॒तेन॑ ज॒हीति॑ । \newline
47. ज॒हीतीति॑ जहि ज॒हीति॒ तेन॒ तेने ति॑ जहि ज॒हीति॒ तेन॑ । \newline
48. इति॒ तेन॒ तेने तीति॒ तेना॒भ्य॑भि तेने तीति॒ तेना॒भि । \newline
49. तेना॒भ्य॑भि तेन॒ तेना॒भ्या॑यता यता॒भि तेन॒ तेना॒भ्या॑यत । \newline
50. अ॒भ्या॑यता यता॒भ्या᳚(1॒)भ्या॑यत॒ तौ ता वा॑यता॒भ्या᳚(1॒)भ्या॑यत॒ तौ । \newline
51. आ॒य॒त॒ तौ ता वा॑यतायत॒ ता व॑ब्रूता मब्रूता॒म् ता वा॑यतायत॒ ता व॑ब्रूताम् । \newline
52. ता व॑ब्रूता मब्रूता॒म् तौ ता व॑ब्रूता म॒ग्नीषोमा॑ व॒ग्नीषोमा॑ वब्रूता॒म् तौ ता व॑ब्रूता म॒ग्नीषोमौ᳚ । \newline
53. अ॒ब्रू॒ता॒ म॒ग्नीषोमा॑ व॒ग्नीषोमा॑ वब्रूता मब्रूता म॒ग्नीषोमौ॒ मा मा ऽग्नीषोमा॑ वब्रूता मब्रूता म॒ग्नीषोमौ॒ मा । \newline
54. अ॒ग्नीषोमौ॒ मा मा ऽग्नीषोमा॑ व॒ग्नीषोमौ॒ मा प्र प्र मा ऽग्नीषोमा॑ व॒ग्नीषोमौ॒ मा प्र । \newline
55. अ॒ग्नीषोमा॒वित्य॒ग्नी - सोमौ᳚ । \newline
56. मा प्र प्र मा मा प्र हार्॑. हाः॒ प्र मा मा प्र हाः᳚ । \newline
\pagebreak
\markright{ TS 2.5.2.3  \hfill https://www.vedavms.in \hfill}
\addcontentsline{toc}{section}{ TS 2.5.2.3 }
\section*{ TS 2.5.2.3 }

\textbf{TS 2.5.2.3 } \newline
\textbf{Samhita Paata} \newline

प्रहा॑रा॒वम॒न्तः स्व॒ इति॒ मम॒ वै यु॒वꣳस्थ॒ इत्य॑ब्रवी॒न् माम॒भ्येत॒मिति॒ तौ भा॑ग॒धेय॑मैच्छेतां॒ ताभ्या॑-मे॒तम॑ग्नीषो॒मीय॒-मेका॑दशकपालं पू॒र्णमा॑से॒ प्राय॑च्छ॒त् ताव॑ब्रूताम॒भि संद॑ष्टौ॒ वै स्वो॒ न श॑क्नुव॒ ऐतु॒मिति॒ स इन्द्र॑ आ॒त्मनः॑ शीतरू॒राव॑जनय॒त् तच्छी॑तरू॒रयो॒र्जन्म॒ य ए॒वꣳ शी॑तरू॒रयो॒र्जन्म॒ वेद॒ - [  ] \newline

\textbf{Pada Paata} \newline

प्रेति॑ । हाः॒ । आ॒वम् । अ॒न्तः । स्वः॒ । इति॑ । मम॑ । वै । यु॒वम् । स्थः॒ । इति॑ । अ॒ब्र॒वी॒त् । माम् । अ॒भि । एति॑ । इ॒त॒म् । इति॑ । तौ । भा॒ग॒धेय॒मिति॑ भाग - धेय᳚म् । ऐ॒च्छे॒ता॒म् । ताभ्या᳚म् । ए॒तम् । अ॒ग्नी॒षो॒मीय॒मित्य॑ग्नी - सो॒मीय᳚म् । एका॑दशकपाल॒मित्येका॑दश - क॒पा॒ल॒म् । पू॒र्णमा॑स॒ इति॑ पू॒र्ण - मा॒से॒ । प्रेति॑ । अ॒य॒च्छ॒त् । तौ । अ॒ब्रू॒ता॒म् । अ॒भीति॑ । संद॑ष्टा॒विति॒ सं - द॒ष्टौ॒ । वै । स्वः॒ । न । श॒क्नु॒वः॒ । ऐतु॒मित्या - ए॒तु॒म् । इति॑ । सः । इन्द्रः॑ । आ॒त्मनः॑ । शी॒त॒रू॒राविति॑ शीत - रू॒रौ । अ॒ज॒न॒य॒त् । तत् । शी॒त॒रू॒रयो॒रिति॑ शीत - रू॒रयोः᳚ । जन्म॑ । यः । ए॒वं । शी॒त॒रू॒रयो॒रिति॑ शीत - रू॒रयोः᳚ । जन्म॑ । वेद॑ ।  \newline


\textbf{Krama Paata} \newline

प्र हाः᳚ । हा॒रा॒वम् । आ॒वम॒न्तः । अ॒न्तः स्वः॑ । स्व॒ इति॑ । इति॒ मम॑ । मम॒ वै । वै यु॒वम् । यु॒वꣳ स्थः॑ । स्थ॒ इति॑ । इत्य॑ब्रवीत् । अ॒ब्र॒वी॒न् माम् । माम॒भि । अ॒भ्या । एत᳚म् । इ॒त॒मिति॑ । इति॒ तौ । तौ भा॑ग॒धेय᳚म् । भा॒ग॒धेय॑मैच्छेताम् । भा॒ग॒धेय॒मिति॑ भाग - धेय᳚म् । ऐ॒च्छे॒ता॒म् ताभ्या᳚म् । ताभ्या॑मे॒तम् । ए॒तम॑ग्नीषो॒मीय᳚म् । अ॒ग्नी॒षो॒मीय॒मेका॑दशकपालम् । अ॒ग्नी॒षो॒मीय॒मित्य॑ग्नी - सो॒मीय᳚म् । एका॑दशकपालम् पू॒र्णमा॑से । एका॑दशकपाल॒मित्येका॑दश - क॒पा॒ल॒म् । पू॒र्णमा॑से॒ प्र । पू॒र्णमा॑स॒ इति॑ पू॒र्ण - मा॒से॒ । प्राय॑च्छत् । अ॒य॒च्छ॒त् तौ । ताव॑ब्रूताम् । अ॒ब्रू॒ता॒म॒भि । अ॒भि सन्द॑ष्टौ । सन्द॑ष्टौ॒ वै । सन्द॑ष्टा॒विति॒ सम् - द॒ष्टौ॒ । वै स्वः॑ । स्वो॒ न । न श॑क्नुवः । श॒क्नु॒व॒ ऐतु᳚म् । ऐतु॒मिति॑ । ऐतु॒मित्या - ए॒तु॒म् । इति॒ सः । स इन्द्रः॑ । इन्द्र॑ आ॒त्मनः॑ । आ॒त्मनः॑ शीतरू॒रौ । शी॒त॒रू॒राव॑जनयत् । शी॒त॒रू॒राविति॑ शीत - रू॒रौ॒ । अ॒ज॒न॒य॒त् तत् । तच्छी॑तरू॒रयोः᳚ । शी॒त॒रू॒रयो॒र् जन्म॑ । शी॒त॒रू॒रयो॒रिति॑ शीत - रू॒रयोः᳚ । जन्म॒ यः । य ए॒वम् । ए॒वꣳ शी॑तरू॒रयोः᳚ । शी॒त॒रू॒रयो॒र् जन्म॑ । शी॒त॒रू॒रयो॒रिति॑ शीत - रू॒रयोः᳚ । जन्म॒ वेद॑ । वेद॒ न \newline

\textbf{Jatai Paata} \newline

1. प्र हार्॑. हाः॒ प्र प्र हाः᳚ । \newline
2. हा॒ रा॒व मा॒वꣳ हार्॑. हा रा॒वम् । \newline
3. आ॒व म॒न्त र॒न्त रा॒व मा॒व म॒न्तः । \newline
4. अ॒न्तः स्वः॑ स्वो॒ ऽन्त र॒न्तः स्वः॑ । \newline
5. स्व॒ इतीति॑ स्वः स्व॒ इति॑ । \newline
6. इति॒ मम॒ ममे तीति॒ मम॑ । \newline
7. मम॒ वै वै मम॒ मम॒ वै । \newline
8. वै यु॒वं ॅयु॒वं ॅवै वै यु॒वम् । \newline
9. यु॒वꣳ स्थः॑ स्थो यु॒वं ॅयु॒वꣳ स्थः॑ । \newline
10. स्थ॒ इतीति॑ स्थः स्थ॒ इति॑ । \newline
11. इत्य॑ब्रवी दब्रवी॒ दिती त्य॑ब्रवीत् । \newline
12. अ॒ब्र॒वी॒न् माम् मा म॑ब्रवी दब्रवी॒न् माम् । \newline
13. मा म॒भ्य॑भि माम् मा म॒भि । \newline
14. अ॒भ्या ऽभ्य॑भ्या । \newline
15. एत॑ मित॒ मेत᳚म् । \newline
16. इ॒त॒ मितीती॑त मित॒ मिति॑ । \newline
17. इति॒ तौ ता वितीति॒ तौ । \newline
18. तौ भा॑ग॒धेय॑म् भाग॒धेय॒म् तौ तौ भा॑ग॒धेय᳚म् । \newline
19. भा॒ग॒धेय॑ मैच्छेता मैच्छेताम् भाग॒धेय॑म् भाग॒धेय॑ मैच्छेताम् । \newline
20. भा॒ग॒धेय॒मिति॑ भाग - धेय᳚म् । \newline
21. ऐ॒च्छे॒ता॒म् ताभ्या॒म् ताभ्या॑ मैच्छेता मैच्छेता॒म् ताभ्या᳚म् । \newline
22. ताभ्या॑ मे॒त मे॒तम् ताभ्या॒म् ताभ्या॑ मे॒तम् । \newline
23. ए॒त म॑ग्नीषो॒मीय॑ मग्नीषो॒मीय॑ मे॒त मे॒त म॑ग्नीषो॒मीय᳚म् । \newline
24. अ॒ग्नी॒षो॒मीय॒ मेका॑दशकपाल॒ मेका॑दशकपाल मग्नीषो॒मीय॑ मग्नीषो॒मीय॒ मेका॑दशकपालम् । \newline
25. अ॒ग्नी॒षो॒मीय॒मित्य॑ग्नी - सो॒मीय᳚म् । \newline
26. एका॑दशकपालम् पू॒र्णमा॑से पू॒र्णमा॑स॒ एका॑दशकपाल॒ मेका॑दशकपालम् पू॒र्णमा॑से । \newline
27. एका॑दशकपाल॒मित्येका॑दश - क॒पा॒ल॒म् । \newline
28. पू॒र्णमा॑से॒ प्र प्र पू॒र्णमा॑से पू॒र्णमा॑से॒ प्र । \newline
29. पू॒र्णमा॑स॒ इति॑ पू॒र्ण - मा॒से॒ । \newline
30. प्राय॑च्छ दयच्छ॒त् प्र प्राय॑च्छत् । \newline
31. अ॒य॒च्छ॒त् तौ ता व॑यच्छ दयच्छ॒त् तौ । \newline
32. ता व॑ब्रूता मब्रूता॒म् तौ ता व॑ब्रूताम् । \newline
33. अ॒ब्रू॒ता॒ म॒भ्या᳚(1॒)भ्य॑ब्रूता मब्रूता म॒भि । \newline
34. अ॒भि सन्द॑ष्टौ॒ सन्द॑ष्टा व॒भ्य॑भि सन्द॑ष्टौ । \newline
35. सन्द॑ष्टौ॒ वै वै सन्द॑ष्टौ॒ सन्द॑ष्टौ॒ वै । \newline
36. सन्द॑ष्टा॒विति॒ सं - द॒ष्टौ॒ । \newline
37. वै स्वः॑ स्वो॒ वै वै स्वः॑ । \newline
38. स्वो॒ न न स्वः॑ स्वो॒ न । \newline
39. न श॑क्नुवः शक्नुवो॒ न न श॑क्नुवः । \newline
40. श॒क्नु॒व॒ ऐतु॒ मैतुꣳ॑ शक्नुवः शक्नुव॒ ऐतु᳚म् । \newline
41. ऐतु॒ मितीत्यैतु॒ मैतु॒ मिति॑ । \newline
42. ऐतु॒मित्या - ए॒तु॒म् । \newline
43. इति॒ स स इतीति॒ सः । \newline
44. स इन्द्र॒ इन्द्रः॒ स स इन्द्रः॑ । \newline
45. इन्द्र॑ आ॒त्मन॑ आ॒त्मन॒ इन्द्र॒ इन्द्र॑ आ॒त्मनः॑ । \newline
46. आ॒त्मनः॑ शीतरू॒रौ शी॑तरू॒रा वा॒त्मन॑ आ॒त्मनः॑ शीतरू॒रौ । \newline
47. शी॒त॒रू॒रा व॑जनय दजनयच् छीतरू॒रौ शी॑तरू॒रा व॑जनयत् । \newline
48. शी॒त॒रू॒राविति॑ शीत - रू॒रौ । \newline
49. अ॒ज॒न॒य॒त् तत् तद॑जनय दजनय॒त् तत् । \newline
50. तच्छी॑तरू॒रयोः᳚ शीतरू॒रयो॒ स्तत् तच्छी॑तरू॒रयोः᳚ । \newline
51. शी॒त॒रू॒रयो॒र् जन्म॒ जन्म॑ शीतरू॒रयोः᳚ शीतरू॒रयो॒र् जन्म॑ । \newline
52. शी॒त॒रू॒रयो॒रिति॑ शीत - रू॒रयोः᳚ । \newline
53. जन्म॒ यो यो जन्म॒ जन्म॒ यः । \newline
54. य ए॒व मे॒वं ॅयो य ए॒वं । \newline
55. ए॒वꣳ शी॑तरू॒रयोः᳚ शीतरू॒रयो॑ रे॒व मे॒वꣳ शी॑तरू॒रयोः᳚ । \newline
56. शी॒त॒रू॒रयो॒र् जन्म॒ जन्म॑ शीतरू॒रयोः᳚ शीतरू॒रयो॒र् जन्म॑ । \newline
57. शी॒त॒रू॒रयो॒रिति॑ शीत - रू॒रयोः᳚ । \newline
58. जन्म॒ वेद॒ वेद॒ जन्म॒ जन्म॒ वेद॑ । \newline
59. वेद॒ न न वेद॒ वेद॒ न । \newline

\textbf{Ghana Paata } \newline

1. प्र हार्॑. हाः॒ प्र प्र हा॑ रा॒व मा॒वꣳ हाः॒ प्र प्र हा॑ रा॒वम् । \newline
2. हा॒ रा॒व मा॒वꣳ हार्॑. हा रा॒व म॒न्त र॒न्त रा॒वꣳ हार्॑. हा रा॒व म॒न्तः । \newline
3. आ॒व म॒न्त र॒न्त रा॒व मा॒व म॒न्तः स्वः॑ स्वो॒ ऽन्त रा॒व मा॒व म॒न्तः स्वः॑ । \newline
4. अ॒न्तः स्वः॑ स्वो॒ ऽन्त र॒न्तः स्व॒ इतीति॑ स्वो॒ ऽन्त र॒न्तः स्व॒ इति॑ । \newline
5. स्व॒ इतीति॑ स्वः स्व॒ इति॒ मम॒ ममे ति॑ स्वः स्व॒ इति॒ मम॑ । \newline
6. इति॒ मम॒ ममे तीति॒ मम॒ वै वै ममे तीति॒ मम॒ वै । \newline
7. मम॒ वै वै मम॒ मम॒ वै यु॒वं ॅयु॒वं ॅवै मम॒ मम॒ वै यु॒वम् । \newline
8. वै यु॒वं ॅयु॒वं ॅवै वै यु॒वꣳ स्थः॑ स्थो यु॒वं ॅवै वै यु॒वꣳ स्थः॑ । \newline
9. यु॒वꣳ स्थः॑ स्थो यु॒वं ॅयु॒वꣳ स्थ॒ इतीति॑ स्थो यु॒वं ॅयु॒वꣳ स्थ॒ इति॑ । \newline
10. स्थ॒ इतीति॑ स्थः स्थ॒ इत्य॑ब्रवी दब्रवी॒दिति॑ स्थः स्थ॒ इत्य॑ब्रवीत् । \newline
11. इत्य॑ब्रवी दब्रवी॒ दितीत्य॑ब्रवी॒न् माम् मा म॑ब्रवी॒ दितीत्य॑ब्रवी॒न् माम् । \newline
12. अ॒ब्र॒वी॒न् माम् मा म॑ब्रवी दब्रवी॒न् मा म॒भ्य॑भि मा म॑ब्रवी दब्रवी॒न् मा म॒भि । \newline
13. मा म॒भ्य॑भि माम् मा म॒भ्या ऽभि माम् मा म॒भ्या । \newline
14. अ॒भ्या ऽभ्य॑भ्येत॑ मित॒ मा ऽभ्य॑भ्येत᳚म् । \newline
15. एत॑ मित॒ मेत॒ मितीती॑त॒ मेत॒ मिति॑ । \newline
16. इ॒त॒ मितीती॑त मित॒ मिति॒ तौ ता विती॑त मित॒ मिति॒ तौ । \newline
17. इति॒ तौ ता वितीति॒ तौ भा॑ग॒धेय॑म् भाग॒धेय॒म् ता वितीति॒ तौ भा॑ग॒धेय᳚म् । \newline
18. तौ भा॑ग॒धेय॑म् भाग॒धेय॒म् तौ तौ भा॑ग॒धेय॑ मैच्छेता मैच्छेताम् भाग॒धेय॒म् तौ तौ भा॑ग॒धेय॑ मैच्छेताम् । \newline
19. भा॒ग॒धेय॑ मैच्छेता मैच्छेताम् भाग॒धेय॑म् भाग॒धेय॑ मैच्छेता॒म् ताभ्या॒म् ताभ्या॑ मैच्छेताम् भाग॒धेय॑म् भाग॒धेय॑ मैच्छेता॒म् ताभ्या᳚म् । \newline
20. भा॒ग॒धेय॒मिति॑ भाग - धेय᳚म् । \newline
21. ऐ॒च्छे॒ता॒म् ताभ्या॒म् ताभ्या॑ मैच्छेता मैच्छेता॒म् ताभ्या॑ मे॒त मे॒तम् ताभ्या॑ मैच्छेता मैच्छेता॒म् ताभ्या॑ मे॒तम् । \newline
22. ताभ्या॑ मे॒त मे॒तम् ताभ्या॒म् ताभ्या॑ मे॒त म॑ग्नीषो॒मीय॑ मग्नीषो॒मीय॑ मे॒तम् ताभ्या॒म् ताभ्या॑ मे॒त म॑ग्नीषो॒मीय᳚म् । \newline
23. ए॒त म॑ग्नीषो॒मीय॑ मग्नीषो॒मीय॑ मे॒त मे॒त म॑ग्नीषो॒मीय॒ मेका॑दशकपाल॒ मेका॑दशकपाल मग्नीषो॒मीय॑ मे॒त मे॒त म॑ग्नीषो॒मीय॒ मेका॑दशकपालम् । \newline
24. अ॒ग्नी॒षो॒मीय॒ मेका॑दशकपाल॒ मेका॑दशकपाल मग्नीषो॒मीय॑ मग्नीषो॒मीय॒ मेका॑दशकपालम् पू॒र्णमा॑से पू॒र्णमा॑स॒ एका॑दशकपाल मग्नीषो॒मीय॑ मग्नीषो॒मीय॒ मेका॑दशकपालम् पू॒र्णमा॑से । \newline
25. अ॒ग्नी॒षो॒मीय॒मित्य॑ग्नी - सो॒मीय᳚म् । \newline
26. एका॑दशकपालम् पू॒र्णमा॑से पू॒र्णमा॑स॒ एका॑दशकपाल॒ मेका॑दशकपालम् पू॒र्णमा॑से॒ प्र प्र पू॒र्णमा॑स॒ एका॑दशकपाल॒ मेका॑दशकपालम् पू॒र्णमा॑से॒ प्र । \newline
27. एका॑दशकपाल॒मित्येका॑दश - क॒पा॒ल॒म् । \newline
28. पू॒र्णमा॑से॒ प्र प्र पू॒र्णमा॑से पू॒र्णमा॑से॒ प्राय॑च्छ दयच्छ॒त् प्र पू॒र्णमा॑से पू॒र्णमा॑से॒ प्राय॑च्छत् । \newline
29. पू॒र्णमा॑स॒ इति॑ पू॒र्ण - मा॒से॒ । \newline
30. प्राय॑च्छ दयच्छ॒त् प्र प्राय॑च्छ॒त् तौ ता व॑यच्छ॒त् प्र प्राय॑च्छ॒त् तौ । \newline
31. अ॒य॒च्छ॒त् तौ ता व॑यच्छ दयच्छ॒त् ता व॑ब्रूता मब्रूता॒म् ता व॑यच्छ दयच्छ॒त् ता व॑ब्रूताम् । \newline
32. ता व॑ब्रूता मब्रूता॒म् तौ ता व॑ब्रूता म॒भ्या᳚(1॒)भ्य॑ब्रूता॒म् तौ ता व॑ब्रूता म॒भि । \newline
33. अ॒ब्रू॒ता॒ म॒भ्या᳚(1॒)भ्य॑ब्रूता मब्रूता म॒भि सन्द॑ष्टौ॒ सन्द॑ष्टा व॒भ्य॑ब्रूता मब्रूता म॒भि सन्द॑ष्टौ । \newline
34. अ॒भि सन्द॑ष्टौ॒ सन्द॑ष्टा व॒भ्य॑भि सन्द॑ष्टौ॒ वै वै सन्द॑ष्टा व॒भ्य॑भि सन्द॑ष्टौ॒ वै । \newline
35. सन्द॑ष्टौ॒ वै वै सन्द॑ष्टौ॒ सन्द॑ष्टौ॒ वै स्वः॑ स्वो॒ वै सन्द॑ष्टौ॒ सन्द॑ष्टौ॒ वै स्वः॑ । \newline
36. सन्द॑ष्टा॒विति॒ सं - द॒ष्टौ॒ । \newline
37. वै स्वः॑ स्वो॒ वै वै स्वो॒ न न स्वो॒ वै वै स्वो॒ न । \newline
38. स्वो॒ न न स्वः॑ स्वो॒ न श॑क्नुवः शक्नुवो॒ न स्वः॑ स्वो॒ न श॑क्नुवः । \newline
39. न श॑क्नुवः शक्नुवो॒ न न श॑क्नुव॒ ऐतु॒ मैतुꣳ॑ शक्नुवो॒ न न श॑क्नुव॒ ऐतु᳚म् । \newline
40. श॒क्नु॒व॒ ऐतु॒ मैतुꣳ॑ शक्नुवः शक्नुव॒ ऐतु॒ मितीत्यैतुꣳ॑ शक्नुवः शक्नुव॒ ऐतु॒ मिति॑ । \newline
41. ऐतु॒ मितीत्यैतु॒ मैतु॒ मिति॒ स स इत्यैतु॒ मैतु॒ मिति॒ सः । \newline
42. ऐतु॒मित्या - ए॒तु॒म् । \newline
43. इति॒ स स इतीति॒ स इन्द्र॒ इन्द्रः॒ स इतीति॒ स इन्द्रः॑ । \newline
44. स इन्द्र॒ इन्द्रः॒ स स इन्द्र॑ आ॒त्मन॑ आ॒त्मन॒ इन्द्रः॒ स स इन्द्र॑ आ॒त्मनः॑ । \newline
45. इन्द्र॑ आ॒त्मन॑ आ॒त्मन॒ इन्द्र॒ इन्द्र॑ आ॒त्मनः॑ शीतरू॒रौ शी॑तरू॒रा वा॒त्मन॒ इन्द्र॒ इन्द्र॑ आ॒त्मनः॑ शीतरू॒रौ । \newline
46. आ॒त्मनः॑ शीतरू॒रौ शी॑तरू॒रा वा॒त्मन॑ आ॒त्मनः॑ शीतरू॒रा व॑जनय दजनयच् छीतरू॒रा वा॒त्मन॑ आ॒त्मनः॑ शीतरू॒रा व॑जनयत् । \newline
47. शी॒त॒रू॒रा व॑जनय दजनयच् छीतरू॒रौ शी॑तरू॒रा व॑जनय॒त् तत् तद॑जनयच् छीतरू॒रौ शी॑तरू॒रा व॑जनय॒त् तत् । \newline
48. शी॒त॒रू॒राविति॑ शीत - रू॒रौ । \newline
49. अ॒ज॒न॒य॒त् तत् तद॑जनय दजनय॒त् तच् छी॑तरू॒रयोः᳚ शीतरू॒रयो॒ स्तद॑जनय दजनय॒त् तच् छी॑तरू॒रयोः᳚ । \newline
50. तच्छी॑तरू॒रयोः᳚ शीतरू॒रयो॒ स्तत् तच्छी॑तरू॒रयो॒र् जन्म॒ जन्म॑ शीतरू॒रयो॒ स्तत् तच्छी॑तरू॒रयो॒र् जन्म॑ । \newline
51. शी॒त॒रू॒रयो॒र् जन्म॒ जन्म॑ शीतरू॒रयोः᳚ शीतरू॒रयो॒र् जन्म॒ यो यो जन्म॑ शीतरू॒रयोः᳚ शीतरू॒रयो॒र् जन्म॒ यः । \newline
52. शी॒त॒रू॒रयो॒रिति॑ शीत - रू॒रयोः᳚ । \newline
53. जन्म॒ यो यो जन्म॒ जन्म॒ य ए॒व मे॒वं ॅयो जन्म॒ जन्म॒ य ए॒वं । \newline
54. य ए॒व मे॒वं ॅयो य ए॒वꣳ शी॑तरू॒रयोः᳚ शीतरू॒रयो॑ रे॒वं ॅयो य ए॒वꣳ शी॑तरू॒रयोः᳚ । \newline
55. ए॒वꣳ शी॑तरू॒रयोः᳚ शीतरू॒रयो॑ रे॒व मे॒वꣳ शी॑तरू॒रयो॒र् जन्म॒ जन्म॑ शीतरू॒रयो॑ रे॒व मे॒वꣳ शी॑तरू॒रयो॒र् जन्म॑ । \newline
56. शी॒त॒रू॒रयो॒र् जन्म॒ जन्म॑ शीतरू॒रयोः᳚ शीतरू॒रयो॒र् जन्म॒ वेद॒ वेद॒ जन्म॑ शीतरू॒रयोः᳚ शीतरू॒रयो॒र् जन्म॒ वेद॑ । \newline
57. शी॒त॒रू॒रयो॒रिति॑ शीत - रू॒रयोः᳚ । \newline
58. जन्म॒ वेद॒ वेद॒ जन्म॒ जन्म॒ वेद॒ न न वेद॒ जन्म॒ जन्म॒ वेद॒ न । \newline
59. वेद॒ न न वेद॒ वेद॒ नैन॑ मेन॒म् न वेद॒ वेद॒ नैन᳚म् । \newline
\pagebreak
\markright{ TS 2.5.2.4  \hfill https://www.vedavms.in \hfill}
\addcontentsline{toc}{section}{ TS 2.5.2.4 }
\section*{ TS 2.5.2.4 }

\textbf{TS 2.5.2.4 } \newline
\textbf{Samhita Paata} \newline

नैनꣳ॑ शीतरू॒रौ ह॑त॒स्ताभ्या॑मेनम॒भ्य॑नय॒त् तस्मा᳚-ज्जञ्ज॒भ्यमा॑नाद॒ग्नीषोमौ॒ निर॑क्रामतां प्राणापा॒नौ वा ए॑नं॒ तद॑जहितां प्रा॒णो वै दक्षो॑ऽपा॒नः क्रतु॒स्तस्मा᳚-ज्जञ्ज॒भ्यमा॑नो ब्रूया॒न्मयि॑ दक्षक्र॒तू इति॑ प्राणापा॒नावे॒वाऽऽत्मन् ध॑त्ते॒ सर्व॒मायु॑रेति॒ स दे॒वता॑ वृ॒त्रान्नि॒र्॒.हूय॒ वार्त्र॑घ्नꣳ ह॒विः पू॒र्णमा॑से॒ निर॑वप॒द् घ्नन्ति॒ वा ए॑नं पू॒र्णमा॑स॒ आ - [  ] \newline

\textbf{Pada Paata} \newline

न । ए॒न॒म् । शी॒त॒रू॒राविति॑ शीत - रू॒रौ । ह॒तः॒ । ताभ्या᳚म् । ए॒न॒म् । अ॒भीति॑ । अ॒न॒य॒त् । तस्मा᳚त् । ज॒ञ्ज॒भ्यमा॑नात् । अ॒ग्नीषोमा॒वित्य॒ग्नी - सोमौ᳚ । निरिति॑ । अ॒क्रा॒म॒ता॒म् । प्रा॒णा॒पा॒नाविति॑ प्राण - अ॒पा॒नौ । वै । ए॒न॒म् । तत् । अ॒ज॒हि॒ता॒म् । प्रा॒ण इति॑ प्र - अ॒नः । वै । दक्षः॑ । अ॒पा॒न इत्य॑प - अ॒नः । क्रतुः॑ । तस्मा᳚त् । ज॒ञ्ज॒भ्यमा॑नः । ब्रू॒या॒त् । मयि॑ । द॒क्ष॒क्र॒तू इति॑ दक्ष-क्र॒तू । इति॑ । प्रा॒णा॒पा॒नाविति॑ प्राण - अ॒पा॒नौ । ए॒व । आ॒त्मन्न् । ध॒त्ते॒ । सर्व᳚म् । आयुः॑ । ए॒ति॒ । सः । दे॒वताः᳚ । वृ॒त्रात् । नि॒र्॒.हूयेति॑ निः - हूय॑ । वार्त्र॑घ्न॒मिति॒ वार्त्र॑ - घ्न॒म् । ह॒विः । पू॒र्णमा॑स॒ इति॑ पू॒र्ण - मा॒से॒ । निरिति॑ । अ॒व॒प॒त् । घ्नन्ति॑ । वै । ए॒न॒म् । पू॒र्णमा॑स॒ इति॑ पू॒र्ण - मा॒से॒ । एति॑ ।  \newline


\textbf{Krama Paata} \newline

नैन᳚म् । ए॒नꣳ॒॒ शी॒त॒रू॒रौ । शी॒त॒रू॒रौ ह॑तः । शी॒त॒रू॒राविति॑ शीत - रू॒रौ । ह॒त॒स्ताभ्या᳚म् । ताभ्या॑मेनम् । ए॒न॒म॒भि । अ॒भ्य॑नयत् । अ॒न॒य॒त् तस्मा᳚त् । तस्मा᳚ज्जञ्ज॒भ्यमा॑नात् । ज॒ञ्ज॒भ्यमा॑नाद॒ग्नीषोमौ᳚ । अ॒ग्नीषोमौ॒ निः । अ॒ग्नीषोमा॒वित्य॒ग्नी - सोमौ᳚ । निर॑क्रामताम् । अ॒क्रा॒म॒ता॒म् प्रा॒णा॒पा॒नौ । प्रा॒णा॒पा॒नौ वै । प्रा॒णा॒पा॒नाविति॑ प्राण - अ॒पा॒नौ । वा ए॑नम् । ए॒न॒म् तत् । तद॑जहिताम् । अ॒ज॒हि॒ता॒म् प्रा॒णः । प्रा॒णो वै । प्रा॒ण इति॑ प्र - अ॒नः । वै दक्षः॑ । दक्षो॑ ऽपा॒नः । अ॒पा॒नः क्रतुः॑ । अ॒पा॒न इत्य॑प - अ॒नः । क्रतु॒स्तस्मा᳚त् । तस्मा᳚ज्जञ्ज॒भ्यमा॑नः । ज॒ञ्ज॒भ्यमा॑नो ब्रूयात् । ब्रू॒या॒न् मयि॑ । मयि॑ दक्षक्र॒तू । द॒क्ष॒क्र॒तू इति॑ । द॒क्ष॒क्र॒तू इति॑ दक्ष - क्र॒तू । इति॑ प्राणापा॒नौ । प्रा॒णा॒पा॒नावे॒व । प्रा॒णा॒पा॒नाविति॑ प्राण - अ॒पा॒नौ । ए॒वात्मन्न् । आ॒त्मन् ध॑त्ते । ध॒त्ते॒ सर्व᳚म् । सर्व॒मायुः॑ । आयु॑रेति । ए॒ति॒ सः । स दे॒वताः᳚ । दे॒वता॑ वृ॒त्रात् । वृ॒त्रान्नि॒र्॒.हूय॑ । नि॒र्॒.हूय॒ वार्त्र॑घ्नम् । नि॒र्.॒हूयेति॑ निः - हूय॑ । वार्त्र॑घ्नꣳ ह॒विः । वार्त्र॑घ्न॒मिति॒ वार्त्र॑ - घ्न॒म् । ह॒विः पू॒र्णमा॑से । पू॒र्णमा॑से॒ निः । पू॒र्णमा॑स॒ इति॑ पू॒र्ण - मा॒से॒ । निर॑वपत् । अ॒व॒प॒द् घ्नन्ति॑ । घ्नन्ति॒ वै । वा ए॑नम् । ए॒न॒म् पू॒र्णमा॑से । पू॒र्णमा॑स॒ आ । पू॒र्णमा॑स॒ इति॑ पू॒र्ण - मा॒से॒ । आ ऽमा॑वा॒स्या॑याम् \newline

\textbf{Jatai Paata} \newline

1. नैन॑ मेन॒म् न नैन᳚म् । \newline
2. ए॒नꣳ॒॒ शी॒त॒रू॒रौ शी॑तरू॒रा वे॑न मेनꣳ शीतरू॒रौ । \newline
3. शी॒त॒रू॒रौ ह॑तो हतः शीतरू॒रौ शी॑तरू॒रौ ह॑तः । \newline
4. शी॒त॒रू॒राविति॑ शीत - रू॒रौ । \newline
5. ह॒त॒ स्ताभ्या॒म् ताभ्याꣳ॑ हतो हत॒ स्ताभ्या᳚म् । \newline
6. ताभ्या॑ मेन मेन॒म् ताभ्या॒म् ताभ्या॑ मेनम् । \newline
7. ए॒न॒ म॒भ्या᳚(1॒)भ्ये॑न मेन म॒भि । \newline
8. अ॒भ्य॑नय दनयद॒भ्या᳚(1॒)भ्य॑नयत् । \newline
9. अ॒न॒य॒त् तस्मा॒त् तस्मा॑ दनय दनय॒त् तस्मा᳚त् । \newline
10. तस्मा᳚ज् जञ्ज॒भ्यमा॑नाज् जञ्ज॒भ्यमा॑ना॒त् तस्मा॒त् तस्मा᳚ज् जञ्ज॒भ्यमा॑नात् । \newline
11. ज॒ञ्ज॒भ्यमा॑ना द॒ग्नीषोमा॑ व॒ग्नीषोमौ॑ जञ्ज॒भ्यमा॑नाज् जञ्ज॒भ्यमा॑ना द॒ग्नीषोमौ᳚ । \newline
12. अ॒ग्नीषोमौ॒ निर् णि र॒ग्नीषोमा॑ व॒ग्नीषोमौ॒ निः । \newline
13. अ॒ग्नीषोमा॒वित्य॒ग्नी - सोमौ᳚ । \newline
14. निर॑क्रामता मक्रामता॒म् निर् णिर॑क्रामताम् । \newline
15. अ॒क्रा॒म॒ता॒म् प्रा॒णा॒पा॒नौ प्रा॑णापा॒ना व॑क्रामता मक्रामताम् प्राणापा॒नौ । \newline
16. प्रा॒णा॒पा॒नौ वै वै प्रा॑णापा॒नौ प्रा॑णापा॒नौ वै । \newline
17. प्रा॒णा॒पा॒नाविति॑ प्राण - अ॒पा॒नौ । \newline
18. वा ए॑न मेनं॒ ॅवै वा ए॑नम् । \newline
19. ए॒न॒म् तत् तदे॑न मेन॒म् तत् । \newline
20. तद॑जहिता मजहिता॒म् तत् तद॑जहिताम् । \newline
21. अ॒ज॒हि॒ता॒म् प्रा॒णः प्रा॒णो॑ ऽजहिता मजहिताम् प्रा॒णः । \newline
22. प्रा॒णो वै वै प्रा॒णः प्रा॒णो वै । \newline
23. प्रा॒ण इति॑ प्र - अ॒नः । \newline
24. वै दक्षो॒ दक्षो॒ वै वै दक्षः॑ । \newline
25. दक्षो॑ ऽपा॒नो॑ ऽपा॒नो दक्षो॒ दक्षो॑ ऽपा॒नः । \newline
26. अ॒पा॒नः क्रतुः॒ क्रतु॑ रपा॒नो॑ ऽपा॒नः क्रतुः॑ । \newline
27. अ॒पा॒न इत्य॑प - अ॒नः । \newline
28. क्रतु॒ स्तस्मा॒त् तस्मा॒त् क्रतुः॒ क्रतु॒ स्तस्मा᳚त् । \newline
29. तस्मा᳚ज् जञ्ज॒भ्यमा॑नो जञ्ज॒भ्यमा॑न॒ स्तस्मा॒त् तस्मा᳚ज् जञ्ज॒भ्यमा॑नः । \newline
30. ज॒ञ्ज॒भ्यमा॑नो ब्रूयाद् ब्रूयाज् जञ्ज॒भ्यमा॑नो जञ्ज॒भ्यमा॑नो ब्रूयात् । \newline
31. ब्रू॒या॒न् मयि॒ मयि॑ ब्रूयाद् ब्रूया॒न् मयि॑ । \newline
32. मयि॑ दक्षक्र॒तू द॑क्षक्र॒तू मयि॒ मयि॑ दक्षक्र॒तू । \newline
33. द॒क्ष॒क्र॒तू इतीति॑ दक्षक्र॒तू द॑क्षक्र॒तू इति॑ । \newline
34. द॒क्ष॒क्र॒तू इति॑ दक्ष - क्र॒तू । \newline
35. इति॑ प्राणापा॒नौ प्रा॑णापा॒ना वितीति॑ प्राणापा॒नौ । \newline
36. प्रा॒णा॒पा॒ना वे॒वैव प्रा॑णापा॒नौ प्रा॑णापा॒ना वे॒व । \newline
37. प्रा॒णा॒पा॒नाविति॑ प्राण - अ॒पा॒नौ । \newline
38. ए॒वात्मन् ना॒त्मन् ने॒वैवात्मन्न् । \newline
39. आ॒त्मन् ध॑त्ते धत्त आ॒त्मन् ना॒त्मन् ध॑त्ते । \newline
40. ध॒त्ते॒ सर्वꣳ॒॒ सर्व॑म् धत्ते धत्ते॒ सर्व᳚म् । \newline
41. सर्व॒ मायु॒ रायुः॒ सर्वꣳ॒॒ सर्व॒ मायुः॑ । \newline
42. आयु॑ रेत्ये॒त्यायु॒ रायु॑रेति । \newline
43. ए॒ति॒ स स ए᳚त्येति॒ सः । \newline
44. स दे॒वता॑ दे॒वताः॒ स स दे॒वताः᳚ । \newline
45. दे॒वता॑ वृ॒त्राद् वृ॒त्राद् दे॒वता॑ दे॒वता॑ वृ॒त्रात् । \newline
46. वृ॒त्रान् नि॒र्॒.हूय॑ नि॒र्॒.हूय॑ वृ॒त्राद् वृ॒त्रान् नि॒र्॒.हूय॑ । \newline
47. नि॒र्॒.हूय॒ वार्त्र॑घ्नं॒ ॅवार्त्र॑घ्नम् नि॒र्॒.हूय॑ नि॒र्॒.हूय॒ वार्त्र॑घ्नम् । \newline
48. नि॒र्॒.हूयेति॑ निः - हूय॑ । \newline
49. वार्त्र॑घ्नꣳ ह॒विर्. ह॒विर् वार्त्र॑घ्नं॒ ॅवार्त्र॑घ्नꣳ ह॒विः । \newline
50. वार्त्र॑घ्न॒मिति॒ वार्त्र॑ - घ्न॒म् । \newline
51. ह॒विः पू॒र्णमा॑से पू॒र्णमा॑से ह॒विर्. ह॒विः पू॒र्णमा॑से । \newline
52. पू॒र्णमा॑से॒ निर् णिष् पू॒र्णमा॑से पू॒र्णमा॑से॒ निः । \newline
53. पू॒र्णमा॑स॒ इति॑ पू॒र्ण - मा॒से॒ । \newline
54. निर॑वप दवप॒न् निर् णिर॑वपत् । \newline
55. अ॒व॒प॒द् घ्नन्ति॒ घ्नन्त्य॑वप दवप॒द् घ्नन्ति॑ । \newline
56. घ्नन्ति॒ वै वै घ्नन्ति॒ घ्नन्ति॒ वै । \newline
57. वा ए॑न मेनं॒ ॅवै वा ए॑नम् । \newline
58. ए॒न॒म् पू॒र्णमा॑से पू॒र्णमा॑स एन मेनम् पू॒र्णमा॑से । \newline
59. पू॒र्णमा॑स॒ आ पू॒र्णमा॑से पू॒र्णमा॑स॒ आ । \newline
60. पू॒र्णमा॑स॒ इति॑ पू॒र्ण - मा॒से॒ । \newline
61. आ ऽमा॑वा॒स्या॑या ममावा॒स्या॑या॒ मा ऽमा॑वा॒स्या॑याम् । \newline

\textbf{Ghana Paata } \newline

1. नैन॑ मेन॒म् न नैनꣳ॑ शीतरू॒रौ शी॑तरू॒रा वे॑न॒म् न नैनꣳ॑ शीतरू॒रौ । \newline
2. ए॒नꣳ॒॒ शी॒त॒रू॒रौ शी॑तरू॒रा वे॑न मेनꣳ शीतरू॒रौ ह॑तो हतः शीतरू॒रा वे॑न मेनꣳ शीतरू॒रौ ह॑तः । \newline
3. शी॒त॒रू॒रौ ह॑तो हतः शीतरू॒रौ शी॑तरू॒रौ ह॑त॒ स्ताभ्या॒म् ताभ्याꣳ॑ हतः शीतरू॒रौ शी॑तरू॒रौ ह॑त॒ स्ताभ्या᳚म् । \newline
4. शी॒त॒रू॒राविति॑ शीत - रू॒रौ । \newline
5. ह॒त॒ स्ताभ्या॒म् ताभ्याꣳ॑ हतो हत॒ स्ताभ्या॑ मेन मेन॒म् ताभ्याꣳ॑ हतो हत॒ स्ताभ्या॑ मेनम् । \newline
6. ताभ्या॑ मेन मेन॒म् ताभ्या॒म् ताभ्या॑ मेन म॒भ्या᳚(1॒)भ्ये॑न॒म् ताभ्या॒म् ताभ्या॑ मेन म॒भि । \newline
7. ए॒न॒ म॒भ्या᳚(1॒)भ्ये॑न मेन म॒भ्य॑नय दनय द॒भ्ये॑न मेन म॒भ्य॑नयत् । \newline
8. अ॒भ्य॑नय दनय द॒भ्या᳚(1॒)भ्य॑नय॒त् तस्मा॒त् तस्मा॑ दनय द॒भ्या᳚(1॒)भ्य॑नय॒त् तस्मा᳚त् । \newline
9. अ॒न॒य॒त् तस्मा॒त् तस्मा॑ दनय दनय॒त् तस्मा᳚ज् जञ्ज॒भ्यमा॑नाज् जञ्ज॒भ्यमा॑ना॒त् तस्मा॑दनय दनय॒त् तस्मा᳚ज् जञ्ज॒भ्यमा॑नात् । \newline
10. तस्मा᳚ज् जञ्ज॒भ्यमा॑नाज् जञ्ज॒भ्यमा॑ना॒त् तस्मा॒त् तस्मा᳚ज् जञ्ज॒भ्यमा॑ना द॒ग्नीषोमा॑ व॒ग्नीषोमौ॑ जञ्ज॒भ्यमा॑ना॒त् तस्मा॒त् तस्मा᳚ज् जञ्ज॒भ्यमा॑ना द॒ग्नीषोमौ᳚ । \newline
11. ज॒ञ्ज॒भ्यमा॑ना द॒ग्नीषोमा॑ व॒ग्नीषोमौ॑ जञ्ज॒भ्यमा॑नाज् जञ्ज॒भ्यमा॑ना द॒ग्नीषोमौ॒ निर् णिर॒ग्नीषोमौ॑ जञ्ज॒भ्यमा॑नाज् जञ्ज॒भ्यमा॑ना द॒ग्नीषोमौ॒ निः । \newline
12. अ॒ग्नीषोमौ॒ निर् णिर॒ग्नीषोमा॑ व॒ग्नीषोमौ॒ निर॑क्रामता मक्रामता॒म् निर॒ग्नीषोमा॑ व॒ग्नीषोमौ॒ निर॑क्रामताम् । \newline
13. अ॒ग्नीषोमा॒वित्य॒ग्नी - सोमौ᳚ । \newline
14. निर॑क्रामता मक्रामता॒म् निर् णिर॑क्रामताम् प्राणापा॒नौ प्रा॑णापा॒ना व॑क्रामता॒म् निर् णिर॑क्रामताम् प्राणापा॒नौ । \newline
15. अ॒क्रा॒म॒ता॒म् प्रा॒णा॒पा॒नौ प्रा॑णापा॒ना व॑क्रामता मक्रामताम् प्राणापा॒नौ वै वै प्रा॑णापा॒ना व॑क्रामता मक्रामताम् प्राणापा॒नौ वै । \newline
16. प्रा॒णा॒पा॒नौ वै वै प्रा॑णापा॒नौ प्रा॑णापा॒नौ वा ए॑न मेनं॒ ॅवै प्रा॑णापा॒नौ प्रा॑णापा॒नौ वा ए॑नम् । \newline
17. प्रा॒णा॒पा॒नाविति॑ प्राण - अ॒पा॒नौ । \newline
18. वा ए॑न मेनं॒ ॅवै वा ए॑न॒म् तत् तदे॑नं॒ ॅवै वा ए॑न॒म् तत् । \newline
19. ए॒न॒म् तत् तदे॑न मेन॒म् तद॑जहिता मजहिता॒म् तदे॑न मेन॒म् तद॑जहिताम् । \newline
20. तद॑जहिता मजहिता॒म् तत् तद॑जहिताम् प्रा॒णः प्रा॒णो॑ ऽजहिता॒म् तत् तद॑जहिताम् प्रा॒णः । \newline
21. अ॒ज॒हि॒ता॒म् प्रा॒णः प्रा॒णो॑ ऽजहिता मजहिताम् प्रा॒णो वै वै प्रा॒णो॑ ऽजहिता मजहिताम् प्रा॒णो वै । \newline
22. प्रा॒णो वै वै प्रा॒णः प्रा॒णो वै दक्षो॒ दक्षो॒ वै प्रा॒णः प्रा॒णो वै दक्षः॑ । \newline
23. प्रा॒ण इति॑ प्र - अ॒नः । \newline
24. वै दक्षो॒ दक्षो॒ वै वै दक्षो॑ ऽपा॒नो॑ ऽपा॒नो दक्षो॒ वै वै दक्षो॑ ऽपा॒नः । \newline
25. दक्षो॑ ऽपा॒नो॑ ऽपा॒नो दक्षो॒ दक्षो॑ ऽपा॒नः क्रतुः॒ क्रतु॑रपा॒नो दक्षो॒ दक्षो॑ ऽपा॒नः क्रतुः॑ । \newline
26. अ॒पा॒नः क्रतुः॒ क्रतु॑रपा॒नो॑ ऽपा॒नः क्रतु॒ स्तस्मा॒त् तस्मा॒त् क्रतु॑रपा॒नो॑ ऽपा॒नः क्रतु॒ स्तस्मा᳚त् । \newline
27. अ॒पा॒न इत्य॑प - अ॒नः । \newline
28. क्रतु॒ स्तस्मा॒त् तस्मा॒त् क्रतुः॒ क्रतु॒ स्तस्मा᳚ज् जञ्ज॒भ्यमा॑नो जञ्ज॒भ्यमा॑न॒ स्तस्मा॒त् क्रतुः॒ क्रतु॒ स्तस्मा᳚ज् जञ्ज॒भ्यमा॑नः । \newline
29. तस्मा᳚ज् जञ्ज॒भ्यमा॑नो जञ्ज॒भ्यमा॑न॒ स्तस्मा॒त् तस्मा᳚ज् जञ्ज॒भ्यमा॑नो ब्रूयाद् ब्रूयाज् जञ्ज॒भ्यमा॑न॒ स्तस्मा॒त् तस्मा᳚ज् जञ्ज॒भ्यमा॑नो ब्रूयात् । \newline
30. ज॒ञ्ज॒भ्यमा॑नो ब्रूयाद् ब्रूयाज् जञ्ज॒भ्यमा॑नो जञ्ज॒भ्यमा॑नो ब्रूया॒न् मयि॒ मयि॑ ब्रूयाज् जञ्ज॒भ्यमा॑नो जञ्ज॒भ्यमा॑नो ब्रूया॒न् मयि॑ । \newline
31. ब्रू॒या॒न् मयि॒ मयि॑ ब्रूयाद् ब्रूया॒न् मयि॑ दक्षक्र॒तू द॑क्षक्र॒तू मयि॑ ब्रूयाद् ब्रूया॒न् मयि॑ दक्षक्र॒तू । \newline
32. मयि॑ दक्षक्र॒तू द॑क्षक्र॒तू मयि॒ मयि॑ दक्षक्र॒तू इतीति॑ दक्षक्र॒तू मयि॒ मयि॑ दक्षक्र॒तू इति॑ । \newline
33. द॒क्ष॒क्र॒तू इतीति॑ दक्षक्र॒तू द॑क्षक्र॒तू इति॑ प्राणापा॒नौ प्रा॑णापा॒ना विति॑ दक्षक्र॒तू द॑क्षक्र॒तू इति॑ प्राणापा॒नौ । \newline
34. द॒क्ष॒क्र॒तू इति॑ दक्ष - क्र॒तू । \newline
35. इति॑ प्राणापा॒नौ प्रा॑णापा॒ना वितीति॑ प्राणापा॒ना वे॒वैव प्रा॑णापा॒ना वितीति॑ प्राणापा॒ना वे॒व । \newline
36. प्रा॒णा॒पा॒ना वे॒वैव प्रा॑णापा॒नौ प्रा॑णापा॒ना वे॒वात्मन् ना॒त्मन् ने॒व प्रा॑णापा॒नौ प्रा॑णापा॒ना वे॒वात्मन्न् । \newline
37. प्रा॒णा॒पा॒नाविति॑ प्राण - अ॒पा॒नौ । \newline
38. ए॒वात्मन् ना॒त्मन् ने॒वैवात्मन् ध॑त्ते धत्त आ॒त्मन् ने॒वैवात्मन् ध॑त्ते । \newline
39. आ॒त्मन् ध॑त्ते धत्त आ॒त्मन् ना॒त्मन् ध॑त्ते॒ सर्वꣳ॒॒ सर्व॑म् धत्त आ॒त्मन् ना॒त्मन् ध॑त्ते॒ सर्व᳚म् । \newline
40. ध॒त्ते॒ सर्वꣳ॒॒ सर्व॑म् धत्ते धत्ते॒ सर्व॒ मायु॒ रायुः॒ सर्व॑म् धत्ते धत्ते॒ सर्व॒ मायुः॑ । \newline
41. सर्व॒ मायु॒रायुः॒ सर्वꣳ॒॒ सर्व॒ मायु॑ रेत्ये॒त्यायुः॒ सर्वꣳ॒॒ सर्व॒ मायु॑रेति । \newline
42. आयु॑ रेत्ये॒त्यायु॒ रायु॑रेति॒ स स ए॒त्यायु॒ रायु॑रेति॒ सः । \newline
43. ए॒ति॒ स स ए᳚त्येति॒ स दे॒वता॑ दे॒वताः॒ स ए᳚त्येति॒ स दे॒वताः᳚ । \newline
44. स दे॒वता॑ दे॒वताः॒ स स दे॒वता॑ वृ॒त्राद् वृ॒त्राद् दे॒वताः॒ स स दे॒वता॑ वृ॒त्रात् । \newline
45. दे॒वता॑ वृ॒त्राद् वृ॒त्राद् दे॒वता॑ दे॒वता॑ वृ॒त्रान् नि॒र्॒.हूय॑ नि॒र्॒.हूय॑ वृ॒त्राद् दे॒वता॑ दे॒वता॑ वृ॒त्रान् नि॒र्॒.हूय॑ । \newline
46. वृ॒त्रान् नि॒र्॒.हूय॑ नि॒र्॒.हूय॑ वृ॒त्राद् वृ॒त्रान् नि॒र्॒.हूय॒ वार्त्र॑घ्नं॒ ॅवार्त्र॑घ्नम् नि॒र्॒.हूय॑ वृ॒त्राद् वृ॒त्रान् नि॒र्॒.हूय॒ वार्त्र॑घ्नम् । \newline
47. नि॒र्॒.हूय॒ वार्त्र॑घ्नं॒ ॅवार्त्र॑घ्नम् नि॒र्॒.हूय॑ नि॒र्॒.हूय॒ वार्त्र॑घ्नꣳ ह॒विर्. ह॒विर् वार्त्र॑घ्नम् नि॒र्॒.हूय॑ नि॒र्॒.हूय॒ वार्त्र॑घ्नꣳ ह॒विः । \newline
48. नि॒र्॒.हूयेति॑ निः - हूय॑ । \newline
49. वार्त्र॑घ्नꣳ ह॒विर्. ह॒विर् वार्त्र॑घ्नं॒ ॅवार्त्र॑घ्नꣳ ह॒विः पू॒र्णमा॑से पू॒र्णमा॑से ह॒विर् वार्त्र॑घ्नं॒ ॅवार्त्र॑घ्नꣳ ह॒विः पू॒र्णमा॑से । \newline
50. वार्त्र॑घ्न॒मिति॒ वार्त्र॑ - घ्न॒म् । \newline
51. ह॒विः पू॒र्णमा॑से पू॒र्णमा॑से ह॒विर्. ह॒विः पू॒र्णमा॑से॒ निर् णिष् पू॒र्णमा॑से ह॒विर्. ह॒विः पू॒र्णमा॑से॒ निः । \newline
52. पू॒र्णमा॑से॒ निर् णिष् पू॒र्णमा॑से पू॒र्णमा॑से॒ निर॑वप दवप॒न् निष् पू॒र्णमा॑से पू॒र्णमा॑से॒ निर॑वपत् । \newline
53. पू॒र्णमा॑स॒ इति॑ पू॒र्ण - मा॒से॒ । \newline
54. निर॑वप दवप॒न् निर् णिर॑वप॒द् घ्नन्ति॒ घ्नन्त्य॑वप॒न् निर् णिर॑वप॒द् घ्नन्ति॑ । \newline
55. अ॒व॒प॒द् घ्नन्ति॒ घ्नन्त्य॑वप दवप॒द् घ्नन्ति॒ वै वै घ्नन्त्य॑वप दवप॒द् घ्नन्ति॒ वै । \newline
56. घ्नन्ति॒ वै वै घ्नन्ति॒ घ्नन्ति॒ वा ए॑न मेनं॒ ॅवै घ्नन्ति॒ घ्नन्ति॒ वा ए॑नम् । \newline
57. वा ए॑न मेनं॒ ॅवै वा ए॑नम् पू॒र्णमा॑से पू॒र्णमा॑स एनं॒ ॅवै वा ए॑नम् पू॒र्णमा॑से । \newline
58. ए॒न॒म् पू॒र्णमा॑से पू॒र्णमा॑स एन मेनम् पू॒र्णमा॑स॒ आ पू॒र्णमा॑स एन मेनम् पू॒र्णमा॑स॒ आ । \newline
59. पू॒र्णमा॑स॒ आ पू॒र्णमा॑से पू॒र्णमा॑स॒ आ ऽमा॑वा॒स्या॑या ममावा॒स्या॑या॒ मा पू॒र्णमा॑से पू॒र्णमा॑स॒ आ ऽमा॑वा॒स्या॑याम् । \newline
60. पू॒र्णमा॑स॒ इति॑ पू॒र्ण - मा॒से॒ । \newline
61. आ ऽमा॑वा॒स्या॑या ममावा॒स्या॑या॒ मा ऽमा॑वा॒स्या॑याम् प्याययन्ति प्यायय न्त्यमावा॒स्या॑या॒ मा ऽमा॑वा॒स्या॑याम् प्याययन्ति । \newline
\pagebreak
\markright{ TS 2.5.2.5  \hfill https://www.vedavms.in \hfill}
\addcontentsline{toc}{section}{ TS 2.5.2.5 }
\section*{ TS 2.5.2.5 }

\textbf{TS 2.5.2.5 } \newline
\textbf{Samhita Paata} \newline

ऽमा॑वा॒स्या॑यां प्याययन्ति॒ तस्मा॒द्-वार्त्र॑घ्नी पू॒र्णमा॒से ऽनू᳚च्येते॒ वृध॑न्वती अमावा॒स्या॑यां॒ तथ् सꣳ॒॒स्थाप्य॒ वार्त्र॑घ्नꣳ ह॒विर्वज्र॑मा॒दाय॒ पुन॑र॒भ्या॑यत॒ ते अ॑ब्रूतां॒ द्यावा॑पृथि॒वी मा प्र हा॑रा॒वयो॒र्वै श्रि॒त इति॒ ते अ॑ब्रूतां॒ ॅवरं॑ ॅवृणावहै॒ नक्ष॑त्रविहिता॒-ऽहमसा॒नीत्य॒साव॑ब्रवी- च्चि॒त्रवि॑हिता॒- ऽहमिती॒यं तस्मा॒न्नक्ष॑त्रविहिता॒ऽसौ चि॒त्रवि॑हिते॒ऽयं ॅय ए॒वं द्यावा॑पृथि॒व्यो - [  ] \newline

\textbf{Pada Paata} \newline

अ॒मा॒वा॒स्या॑या॒मित्य॑मा - वा॒स्या॑याम् । प्या॒य॒य॒न्ति॒ । तस्मा᳚त् । वार्त्र॑घ्नी॒ इति॒ वार्त्र॑ - घ्नी॒ । पू॒र्णमा॑स॒ इति॑ पू॒र्ण - मा॒से॒ । अन्विति॑ । उ॒च्ये॒ते॒ इति॑ । वृध॑न्वती॒ इति॒ वृधन्न्॑ - व॒ती॒ । अ॒मा॒वा॒स्या॑या॒मित्य॑मा - वा॒स्या॑याम् । तत् । सꣳ॒॒स्थाप्येति॑ सं - स्थाप्य॑ । वार्त्र॑घ्न॒मिति॒ वार्त्र॑ - घ्न॒म् । ह॒विः । वज्र᳚म् । आ॒दायेत्या᳚ - दाय॑ । पुनः॑ । अ॒भीति॑ । आ॒य॒त॒ । ते इति॑ । अ॒ब्रू॒ता॒म् । द्यावा॑पृथि॒वी इति॒ द्यावा᳚ - पृ॒थि॒वी । मा । प्रेति॑ । हाः॒ । आ॒वयोः᳚ । वै । श्रि॒तः । इति॑ । ते इति॑ । अ॒ब्रू॒ता॒म् । वर᳚म् । वृ॒णा॒व॒है॒ । नक्ष॑त्रविहि॒तेति॒ नक्ष॑त्र - वि॒हि॒ता॒ । अ॒हम् । असा॑नि । इति॑ । अ॒सौ । अ॒ब्र॒वी॒त् । चि॒त्रवि॑हि॒तेति॑ चि॒त्र - वि॒हि॒ता॒ । अ॒हम् । इति॑ । इ॒यम् । तस्मा᳚त् । नक्ष॑त्रविहि॒तेति॒ नक्ष॑त्र - वि॒हि॒ता॒ । अ॒सौ । चि॒त्रवि॑हि॒तेति॑ चि॒त्र - वि॒हि॒ता॒ । इ॒यम् । यः । ए॒वम् । द्यावा॑पृथि॒व्योरिति॒ द्यावा᳚ - पृ॒थि॒व्योः ।  \newline


\textbf{Krama Paata} \newline

अ॒मा॒वा॒स्या॑याम् प्याययन्ति । अ॒मा॒वा॒स्या॑या॒मित्य॑मा - वा॒स्या॑याम् । प्या॒य॒य॒न्ति॒ तस्मा᳚त् । तस्मा॒द् वार्त्र॑घ्नी । वार्त्र॑घ्नी पू॒र्णमा॑से । वार्त्र॑घ्नी॒ इति॒ वार्त्र॑ - घ्नी॒ । पू॒र्णमा॒से ऽनु॑ । पू॒र्णमा॑स॒ इति॑ पू॒र्ण - मा॒से॒ । अनू᳚च्येते । उ॒च्ये॒ते॒ वृध॑न्वती । उ॒च्ये॒ते॒ इत्यु॑च्येते । वृध॑न्वती अमावा॒स्या॑याम् । वृध॑न्वती॒ इति॒ वृधन्न्॑ - व॒ती॒ । अ॒मा॒वा॒स्या॑या॒म् तत् । अ॒मा॒वा॒स्या॑या॒मित्य॑मा - वा॒स्या॑याम् । तथ् सꣳ॒॒स्थाप्य॑ । सꣳ॒॒स्थाप्य॒ वार्त्र॑घ्नम् । सꣳ॒॒स्थाप्येति॑ सम् - स्थाप्य॑ । वार्त्र॑घ्नꣳ ह॒विः । वार्त्र॑घ्न॒मिति॒ वार्त्र॑ - घ्न॒म् । ह॒विर् वज्र᳚म् । वज्र॑मा॒दाय॑ । आ॒दाय॒ पुनः॑ । आ॒दायेत्या᳚ - दाय॑ । पुन॑र॒भि । अ॒भ्या॑यत । आ॒य॒त॒ ते । ते अ॑ब्रूताम् । ते इति॒ ते । अ॒ब्रू॒ता॒म् द्यावा॑पृथि॒वी । द्यावा॑पृथि॒वी मा । द्यावा॑पृथि॒वी इति॒ द्यावा᳚ - पृ॒थि॒वी । मा प्र । प्र हाः᳚ । हा॒रा॒वयोः᳚ । आ॒वयो॒र् वै । वै श्रि॒तः । श्रि॒त इति॑ । इति॒ ते । ते अ॑ब्रूताम् । ते इति॒ ते । अ॒ब्रू॒ता॒म् ॅवर᳚म् । वर॑म् ॅवृणावहै । वृ॒णा॒व॒है॒ नक्ष॑त्रविहिता । नक्ष॑त्रविहिता॒ऽहम् । नक्ष॑त्रविहि॒तेति॒ नक्ष॑त्र - वि॒हि॒ता॒ । अ॒हमसा॑नि । असा॒नीति॑ । इत्य॒सौ । अ॒साव॑ब्रवीत् । अ॒ब्र॒वी॒च्चि॒त्रवि॑हिता । चि॒त्रवि॑हिता॒ ऽहम् । चि॒त्रवि॑हि॒तेति॑ चि॒त्र - वि॒हि॒ता॒ । अ॒हमिति॑ । इती॒यम् । इ॒यम् तस्मा᳚त् । तस्मा॒न्नक्ष॑त्रविहिता । नक्ष॑त्रविहिता॒ऽसौ । नक्ष॑त्रविहि॒तेति॒ नक्ष॑त्र - वि॒हि॒ता॒ । अ॒सौ चि॒त्रवि॑हिता । चि॒त्रवि॑हिते॒यम् । चि॒त्रवि॑हि॒तेति॑ चि॒त्र - वि॒हि॒ता॒ । इ॒यम् ॅयः । य ए॒वम् । ए॒वम् द्यावा॑पृथि॒व्योः । द्यावा॑पृथि॒व्योर् वर᳚म् । द्यावा॑पृथि॒व्योरिति॒ द्यावा᳚ - पृ॒थि॒व्योः \newline

\textbf{Jatai Paata} \newline

1. अ॒मा॒वा॒स्या॑याम् प्याययन्ति प्याययन् त्यमावा॒स्या॑या ममावा॒स्या॑याम् प्याययन्ति । \newline
2. अ॒मा॒वा॒स्या॑या॒मित्य॑मा - वा॒स्या॑याम् । \newline
3. प्या॒य॒य॒न्ति॒ तस्मा॒त् तस्मा᳚त् प्याययन्ति प्याययन्ति॒ तस्मा᳚त् । \newline
4. तस्मा॒द् वार्त्र॑घ्नी॒ वार्त्र॑घ्नी॒ तस्मा॒त् तस्मा॒द् वार्त्र॑घ्नी । \newline
5. वार्त्र॑घ्नी पू॒र्णमा॑से पू॒र्णमा॑से॒ वार्त्र॑घ्नी॒ वार्त्र॑घ्नी पू॒र्णमा॑से । \newline
6. वार्त्र॑घ्नी॒ इति॒ वार्त्र॑ - घ्नी॒ । \newline
7. पू॒र्णमा॒से ऽन्वनु॑ पू॒र्णमा॑से पू॒र्णमा॒से ऽनु॑ । \newline
8. पू॒र्णमा॑स॒ इति॑ पू॒र्ण - मा॒से॒ । \newline
9. अनू᳚च्येते उच्येते॒ अन्वनू᳚च्येते । \newline
10. उ॒च्ये॒ते॒ वृध॑न्वती॒ वृध॑न्वती उच्येते उच्येते॒ वृध॑न्वती । \newline
11. उ॒च्ये॒ते॒ इत्यु॑च्येते । \newline
12. वृध॑न्वती अमावा॒स्या॑या ममावा॒स्या॑यां॒ ॅवृध॑न्वती॒ वृध॑न्वती अमावा॒स्या॑याम् । \newline
13. वृध॑न्वती॒ इति॒ वृधन्न्॑ - व॒ती॒ । \newline
14. अ॒मा॒वा॒स्या॑या॒म् तत् तद॑मावा॒स्या॑या ममावा॒स्या॑या॒म् तत् । \newline
15. अ॒मा॒वा॒स्या॑या॒मित्य॑मा - वा॒स्या॑याम् । \newline
16. तथ् सꣳ॒॒स्थाप्य॑ सꣳ॒॒स्थाप्य॒ तत् तथ् सꣳ॒॒स्थाप्य॑ । \newline
17. सꣳ॒॒स्थाप्य॒ वार्त्र॑घ्नं॒ ॅवार्त्र॑घ्नꣳ सꣳ॒॒स्थाप्य॑ सꣳ॒॒स्थाप्य॒ वार्त्र॑घ्नम् । \newline
18. सꣳ॒॒स्थाप्येति॑ सं - स्थाप्य॑ । \newline
19. वार्त्र॑घ्नꣳ ह॒विर्. ह॒विर् वार्त्र॑घ्नं॒ ॅवार्त्र॑घ्नꣳ ह॒विः । \newline
20. वार्त्र॑घ्न॒मिति॒ वार्त्र॑ - घ्न॒म् । \newline
21. ह॒विर् वज्रं॒ ॅवज्रꣳ॑ ह॒विर्. ह॒विर् वज्र᳚म् । \newline
22. वज्र॑ मा॒दाया॒ दाय॒ वज्रं॒ ॅवज्र॑ मा॒दाय॑ । \newline
23. आ॒दाय॒ पुनः॒ पुन॑ रा॒दाया॒ दाय॒ पुनः॑ । \newline
24. आ॒दायेत्या᳚ - दाय॑ । \newline
25. पुन॑ र॒भ्य॑भि पुनः॒ पुन॑ र॒भि । \newline
26. अ॒भ्या॑यता यता॒भ्या᳚(1॒)भ्या॑यत । \newline
27. आ॒य॒त॒ ते ते आ॑यता यत॒ ते । \newline
28. ते अ॑ब्रूता मब्रूता॒म् ते ते अ॑ब्रूताम् । \newline
29. ते इति॒ ते । \newline
30. अ॒ब्रू॒ता॒म् द्यावा॑पृथि॒वी द्यावा॑पृथि॒वी अ॑ब्रूता मब्रूता॒म् द्यावा॑पृथि॒वी । \newline
31. द्यावा॑पृथि॒वी मा मा द्यावा॑पृथि॒वी द्यावा॑पृथि॒वी मा । \newline
32. द्यावा॑पृथि॒वी इति॒ द्यावा᳚ - पृ॒थि॒वी । \newline
33. मा प्र प्र मा मा प्र । \newline
34. प्र हार्॑. हाः॒ प्र प्र हाः᳚ । \newline
35. हा॒ रा॒वयो॑ रा॒वयोर्॑. हार्. हा रा॒वयोः᳚ । \newline
36. आ॒वयो॒र् वै वा आ॒वयो॑ रा॒वयो॒र् वै । \newline
37. वै श्रि॒तः श्रि॒तो वै वै श्रि॒तः । \newline
38. श्रि॒त इतीति॑ श्रि॒तः श्रि॒त इति॑ । \newline
39. इति॒ ते ते इतीति॒ ते । \newline
40. ते अ॑ब्रूता मब्रूता॒म् ते ते अ॑ब्रूताम् । \newline
41. ते इति॒ ते । \newline
42. अ॒ब्रू॒तां॒ ॅवरं॒ ॅवर॑ मब्रूता मब्रूतां॒ ॅवर᳚म् । \newline
43. वरं॑ ॅवृणावहै वृणावहै॒ वरं॒ ॅवरं॑ ॅवृणावहै । \newline
44. वृ॒णा॒व॒है॒ नक्ष॑त्रविहिता॒ नक्ष॑त्रविहिता वृणावहै वृणावहै॒ नक्ष॑त्रविहिता । \newline
45. नक्ष॑त्रविहिता॒ ऽह म॒हम् नक्ष॑त्रविहिता॒ नक्ष॑त्रविहिता॒ ऽहम् । \newline
46. नक्ष॑त्रविहि॒तेति॒ नक्ष॑त्र - वि॒हि॒ता॒ । \newline
47. अ॒ह मसा॒ न्यसा᳚न्य॒ह म॒ह मसा॑नि । \newline
48. असा॒नीती त्यसा॒ न्यसा॒नीति॑ । \newline
49. इत्य॒सा व॒सा वितीत्य॒सौ । \newline
50. अ॒सा व॑ब्रवी दब्रवी द॒सा व॒सा व॑ब्रवीत् । \newline
51. अ॒ब्र॒वी॒च् चि॒त्रवि॑हिता चि॒त्रवि॑हिता ऽब्रवी दब्रवीच् चि॒त्रवि॑हिता । \newline
52. चि॒त्रवि॑हिता॒ ऽह म॒हम् चि॒त्रवि॑हिता चि॒त्रवि॑हिता॒ ऽहम् । \newline
53. चि॒त्रवि॑हि॒तेति॑ चि॒त्र - वि॒हि॒ता॒ । \newline
54. अ॒ह मितीत्य॒ह म॒ह मिति॑ । \newline
55. इती॒य मि॒य मितीती॒यम् । \newline
56. इ॒यम् तस्मा॒त् तस्मा॑दि॒य मि॒यम् तस्मा᳚त् । \newline
57. तस्मा॒न् नक्ष॑त्रविहिता॒ नक्ष॑त्रविहिता॒ तस्मा॒त् तस्मा॒न् नक्ष॑त्रविहिता । \newline
58. नक्ष॑त्रविहिता॒ ऽसा व॒सौ नक्ष॑त्रविहिता॒ नक्ष॑त्रविहिता॒ ऽसौ । \newline
59. नक्ष॑त्रविहि॒तेति॒ नक्ष॑त्र - वि॒हि॒ता॒ । \newline
60. अ॒सौ चि॒त्रवि॑हिता चि॒त्रवि॑हिता॒ ऽसा व॒सौ चि॒त्रवि॑हिता । \newline
61. चि॒त्रवि॑हिते॒य मि॒यम् चि॒त्रवि॑हिता चि॒त्रवि॑हिते॒यम् । \newline
62. चि॒त्रवि॑हि॒तेति॑ चि॒त्र - वि॒हि॒ता॒ । \newline
63. इ॒यं ॅयो य इ॒य मि॒यं ॅयः । \newline
64. य ए॒व मे॒वं ॅयो य ए॒वम् । \newline
65. ए॒वम् द्यावा॑पृथि॒व्योर् द्यावा॑पृथि॒व्यो रे॒व मे॒वम् द्यावा॑पृथि॒व्योः । \newline
66. द्यावा॑पृथि॒व्योर् वरं॒ ॅवर॒म् द्यावा॑पृथि॒व्योर् द्यावा॑पृथि॒व्योर् वर᳚म् । \newline
67. द्यावा॑पृथि॒व्योरिति॒ द्यावा᳚ - पृ॒थि॒व्योः । \newline

\textbf{Ghana Paata } \newline

1. अ॒मा॒वा॒स्या॑याम् प्याययन्ति प्यायय न्त्यमावा॒स्या॑या ममावा॒स्या॑याम् प्याययन्ति॒ तस्मा॒त् तस्मा᳚त् प्यायय न्त्यमावा॒स्या॑या ममावा॒स्या॑याम् प्याययन्ति॒ तस्मा᳚त् । \newline
2. अ॒मा॒वा॒स्या॑या॒मित्य॑मा - वा॒स्या॑याम् । \newline
3. प्या॒य॒य॒न्ति॒ तस्मा॒त् तस्मा᳚त् प्याययन्ति प्याययन्ति॒ तस्मा॒द् वार्त्र॑घ्नी॒ वार्त्र॑घ्नी॒ तस्मा᳚त् प्याययन्ति प्याययन्ति॒ तस्मा॒द् वार्त्र॑घ्नी । \newline
4. तस्मा॒द् वार्त्र॑घ्नी॒ वार्त्र॑घ्नी॒ तस्मा॒त् तस्मा॒द् वार्त्र॑घ्नी पू॒र्णमा॑से पू॒र्णमा॑से॒ वार्त्र॑घ्नी॒ तस्मा॒त् तस्मा॒द् वार्त्र॑घ्नी पू॒र्णमा॑से । \newline
5. वार्त्र॑घ्नी पू॒र्णमा॑से पू॒र्णमा॑से॒ वार्त्र॑घ्नी॒ वार्त्र॑घ्नी पू॒र्णमा॒से ऽन्वनु॑ पू॒र्णमा॑से॒ वार्त्र॑घ्नी॒ वार्त्र॑घ्नी पू॒र्णमा॒से ऽनु॑ । \newline
6. वार्त्र॑घ्नी॒ इति॒ वार्त्र॑ - घ्नी॒ । \newline
7. पू॒र्णमा॒से ऽन्वनु॑ पू॒र्णमा॑से पू॒र्णमा॒से ऽनू᳚च्येते उच्येते॒ अनु॑ पू॒र्णमा॑से पू॒र्णमा॒से ऽनू᳚च्येते । \newline
8. पू॒र्णमा॑स॒ इति॑ पू॒र्ण - मा॒से॒ । \newline
9. अनू᳚च्येते उच्येते॒ अन्वनू᳚च्येते॒ वृध॑न्वती॒ वृध॑न्वती उच्येते॒ अन्वनू᳚च्येते॒ वृध॑न्वती । \newline
10. उ॒च्ये॒ते॒ वृध॑न्वती॒ वृध॑न्वती उच्येते उच्येते॒ वृध॑न्वती अमावा॒स्या॑या ममावा॒स्या॑यां॒ ॅवृध॑न्वती उच्येते उच्येते॒ वृध॑न्वती अमावा॒स्या॑याम् । \newline
11. उ॒च्ये॒ते॒ इत्यु॑च्येते । \newline
12. वृध॑न्वती अमावा॒स्या॑या ममावा॒स्या॑यां॒ ॅवृध॑न्वती॒ वृध॑न्वती अमावा॒स्या॑या॒म् तत् तद॑मावा॒स्या॑यां॒ ॅवृध॑न्वती॒ वृध॑न्वती अमावा॒स्या॑या॒म् तत् । \newline
13. वृध॑न्वती॒ इति॒ वृधन्न्॑ - व॒ती॒ । \newline
14. अ॒मा॒वा॒स्या॑या॒म् तत् तद॑मावा॒स्या॑या ममावा॒स्या॑या॒म् तथ् सꣳ॒॒स्थाप्य॑ सꣳ॒॒स्थाप्य॒ तद॑मावा॒स्या॑या ममावा॒स्या॑या॒म् तथ् सꣳ॒॒स्थाप्य॑ । \newline
15. अ॒मा॒वा॒स्या॑या॒मित्य॑मा - वा॒स्या॑याम् । \newline
16. तथ् सꣳ॒॒स्थाप्य॑ सꣳ॒॒स्थाप्य॒ तत् तथ् सꣳ॒॒स्थाप्य॒ वार्त्र॑घ्नं॒ ॅवार्त्र॑घ्नꣳ सꣳ॒॒स्थाप्य॒ तत् तथ् सꣳ॒॒स्थाप्य॒ वार्त्र॑घ्नम् । \newline
17. सꣳ॒॒स्थाप्य॒ वार्त्र॑घ्नं॒ ॅवार्त्र॑घ्नꣳ सꣳ॒॒स्थाप्य॑ सꣳ॒॒स्थाप्य॒ वार्त्र॑घ्नꣳ ह॒विर्. ह॒विर् वार्त्र॑घ्नꣳ सꣳ॒॒स्थाप्य॑ सꣳ॒॒स्थाप्य॒ वार्त्र॑घ्नꣳ ह॒विः । \newline
18. सꣳ॒॒स्थाप्येति॑ सं - स्थाप्य॑ । \newline
19. वार्त्र॑घ्नꣳ ह॒विर्. ह॒विर् वार्त्र॑घ्नं॒ ॅवार्त्र॑घ्नꣳ ह॒विर् वज्रं॒ ॅवज्रꣳ॑ ह॒विर् वार्त्र॑घ्नं॒ ॅवार्त्र॑घ्नꣳ ह॒विर् वज्र᳚म् । \newline
20. वार्त्र॑घ्न॒मिति॒ वार्त्र॑ - घ्न॒म् । \newline
21. ह॒विर् वज्रं॒ ॅवज्रꣳ॑ ह॒विर्. ह॒विर् वज्र॑ मा॒दाया॒दाय॒ वज्रꣳ॑ ह॒विर्. ह॒विर् वज्र॑ मा॒दाय॑ । \newline
22. वज्र॑ मा॒दाया॒दाय॒ वज्रं॒ ॅवज्र॑ मा॒दाय॒ पुनः॒ पुन॑ रा॒दाय॒ वज्रं॒ ॅवज्र॑ मा॒दाय॒ पुनः॑ । \newline
23. आ॒दाय॒ पुनः॒ पुन॑ रा॒दाया॒दाय॒ पुन॑ र॒भ्य॑भि पुन॑ रा॒दाया॒दाय॒ पुन॑ र॒भि । \newline
24. आ॒दायेत्या᳚ - दाय॑ । \newline
25. पुन॑ र॒भ्य॑भि पुनः॒ पुन॑ र॒भ्या॑यतायता॒भि पुनः॒ पुन॑ र॒भ्या॑यत । \newline
26. अ॒भ्या॑यताय ता॒भ्या᳚(1॒)भ्या॑यत॒ ते ते आ॑यता॒भ्या᳚(1॒)भ्या॑यत॒ ते । \newline
27. आ॒य॒त॒ ते ते आ॑यतायत॒ ते अ॑ब्रूता मब्रूता॒म् ते आ॑यतायत॒ ते अ॑ब्रूताम् । \newline
28. ते अ॑ब्रूता मब्रूता॒म् ते ते अ॑ब्रूता॒म् द्यावा॑पृथि॒वी द्यावा॑पृथि॒वी अ॑ब्रूता॒म् ते ते अ॑ब्रूता॒म् द्यावा॑पृथि॒वी । \newline
29. ते इति॒ ते । \newline
30. अ॒ब्रू॒ता॒म् द्यावा॑पृथि॒वी द्यावा॑पृथि॒वी अ॑ब्रूता मब्रूता॒म् द्यावा॑पृथि॒वी मा मा द्यावा॑पृथि॒वी अ॑ब्रूता मब्रूता॒म् द्यावा॑पृथि॒वी मा । \newline
31. द्यावा॑पृथि॒वी मा मा द्यावा॑पृथि॒वी द्यावा॑पृथि॒वी मा प्र प्र मा द्यावा॑पृथि॒वी द्यावा॑पृथि॒वी मा प्र । \newline
32. द्यावा॑पृथि॒वी इति॒ द्यावा᳚ - पृ॒थि॒वी । \newline
33. मा प्र प्र मा मा प्र हार्॑. हाः॒ प्र मा मा प्र हाः᳚ । \newline
34. प्र हार्॑. हाः॒ प्र प्र हा॑ रा॒वयो॑ रा॒वयोर्॑. हाः॒ प्र प्र हा॑ रा॒वयोः᳚ । \newline
35. हा॒ रा॒वयो॑ रा॒वयोर्॑. हार्. हा रा॒वयो॒र् वै वा आ॒वयोर्॑. हार्. हा रा॒वयो॒र् वै । \newline
36. आ॒वयो॒र् वै वा आ॒वयो॑ रा॒वयो॒र् वै श्रि॒तः श्रि॒तो वा आ॒वयो॑ रा॒वयो॒र् वै श्रि॒तः । \newline
37. वै श्रि॒तः श्रि॒तो वै वै श्रि॒त इतीति॑ श्रि॒तो वै वै श्रि॒त इति॑ । \newline
38. श्रि॒त इतीति॑ श्रि॒तः श्रि॒त इति॒ ते ते इति॑ श्रि॒तः श्रि॒त इति॒ ते । \newline
39. इति॒ ते ते इतीति॒ ते अ॑ब्रूता मब्रूता॒म् ते इतीति॒ ते अ॑ब्रूताम् । \newline
40. ते अ॑ब्रूता मब्रूता॒म् ते ते अ॑ब्रूतां॒ ॅवरं॒ ॅवर॑ मब्रूता॒म् ते ते अ॑ब्रूतां॒ ॅवर᳚म् । \newline
41. ते इति॒ ते । \newline
42. अ॒ब्रू॒तां॒ ॅवरं॒ ॅवर॑ मब्रूता मब्रूतां॒ ॅवरं॑ ॅवृणावहै वृणावहै॒ वर॑ मब्रूता मब्रूतां॒ ॅवरं॑ ॅवृणावहै । \newline
43. वरं॑ ॅवृणावहै वृणावहै॒ वरं॒ ॅवरं॑ ॅवृणावहै॒ नक्ष॑त्रविहिता॒ नक्ष॑त्रविहिता वृणावहै॒ वरं॒ ॅवरं॑ ॅवृणावहै॒ नक्ष॑त्रविहिता । \newline
44. वृ॒णा॒व॒है॒ नक्ष॑त्रविहिता॒ नक्ष॑त्रविहिता वृणावहै वृणावहै॒ नक्ष॑त्रविहिता॒ ऽह म॒हम् नक्ष॑त्रविहिता वृणावहै वृणावहै॒ नक्ष॑त्रविहिता॒ ऽहम् । \newline
45. नक्ष॑त्रविहिता॒ ऽह म॒हन्नक्ष॑त्रविहिता॒ नक्ष॑त्रविहिता॒ ऽह मसा॒ न्यसा᳚न्य॒हम् नक्ष॑त्रविहिता॒ नक्ष॑त्रविहिता॒ ऽह मसा॑नि । \newline
46. नक्ष॑त्रविहि॒तेति॒ नक्ष॑त्र - वि॒हि॒ता॒ । \newline
47. अ॒ह मसा॒न्यसा᳚न्य॒ह म॒ह मसा॒नीती त्यसा᳚न्य॒ह म॒ह मसा॒नीति॑ । \newline
48. असा॒नीती त्यसा॒ न्यसा॒नी त्य॒सा व॒सा वित्यसा॒ न्यसा॒नी त्य॒सौ । \newline
49. इत्य॒सा व॒सा वितीत्य॒सा व॑ब्रवी दब्रवी द॒सा वितीत्य॒सा व॑ब्रवीत् । \newline
50. अ॒सा व॑ब्रवी दब्रवी द॒सा व॒सा व॑ब्रवीच् चि॒त्रवि॑हिता चि॒त्रवि॑हिता ऽब्रवीद॒सा व॒सा व॑ब्रवीच् चि॒त्रवि॑हिता । \newline
51. अ॒ब्र॒वी॒च् चि॒त्रवि॑हिता चि॒त्रवि॑हिता ऽब्रवी दब्रवीच् चि॒त्रवि॑हिता॒ ऽह म॒हम् चि॒त्रवि॑हिता ऽब्रवी दब्रवीच् चि॒त्रवि॑हिता॒ ऽहम् । \newline
52. चि॒त्रवि॑हिता॒ ऽह म॒हम् चि॒त्रवि॑हिता चि॒त्रवि॑हिता॒ ऽह मितीत्य॒हम् चि॒त्रवि॑हिता चि॒त्रवि॑हिता॒ ऽह मिति॑ । \newline
53. चि॒त्रवि॑हि॒तेति॑ चि॒त्र - वि॒हि॒ता॒ । \newline
54. अ॒ह मितीत्य॒ह म॒ह मिती॒य मि॒य मित्य॒ह म॒ह मिती॒यम् । \newline
55. इती॒य मि॒य मितीती॒यम् तस्मा॒त् तस्मा॑दि॒य मितीती॒यम् तस्मा᳚त् । \newline
56. इ॒यम् तस्मा॒त् तस्मा॑दि॒य मि॒यम् तस्मा॒न् नक्ष॑त्रविहिता॒ नक्ष॑त्रविहिता॒ तस्मा॑दि॒य मि॒यम् तस्मा॒न् नक्ष॑त्रविहिता । \newline
57. तस्मा॒न् नक्ष॑त्रविहिता॒ नक्ष॑त्रविहिता॒ तस्मा॒त् तस्मा॒न् नक्ष॑त्रविहिता॒ ऽसा व॒सौ नक्ष॑त्रविहिता॒ तस्मा॒त् तस्मा॒न् नक्ष॑त्रविहिता॒ ऽसौ । \newline
58. नक्ष॑त्रविहिता॒ ऽसा व॒सौ नक्ष॑त्रविहिता॒ नक्ष॑त्रविहिता॒ ऽसौ चि॒त्रवि॑हिता चि॒त्रवि॑हिता॒ ऽसौ नक्ष॑त्रविहिता॒ नक्ष॑त्रविहिता॒ ऽसौ चि॒त्रवि॑हिता । \newline
59. नक्ष॑त्रविहि॒तेति॒ नक्ष॑त्र - वि॒हि॒ता॒ । \newline
60. अ॒सौ चि॒त्रवि॑हिता चि॒त्रवि॑हिता॒ ऽसा व॒सौ चि॒त्रवि॑हिते॒य मि॒यम् चि॒त्रवि॑हिता॒ ऽसा व॒सौ चि॒त्रवि॑हिते॒यम् । \newline
61. चि॒त्रवि॑हिते॒य मि॒यम् चि॒त्रवि॑हिता चि॒त्रवि॑हिते॒यं ॅयो य इ॒यम् चि॒त्रवि॑हिता चि॒त्रवि॑हिते॒यं ॅयः । \newline
62. चि॒त्रवि॑हि॒तेति॑ चि॒त्र - वि॒हि॒ता॒ । \newline
63. इ॒यं ॅयो य इ॒य मि॒यं ॅय ए॒व मे॒वं ॅय इ॒य मि॒यं ॅय ए॒वम् । \newline
64. य ए॒व मे॒वं ॅयो य ए॒वम् द्यावा॑पृथि॒व्योर् द्यावा॑पृथि॒व्यो रे॒वं ॅयो य ए॒वम् द्यावा॑पृथि॒व्योः । \newline
65. ए॒वम् द्यावा॑पृथि॒व्योर् द्यावा॑पृथि॒व्यो रे॒व मे॒वम् द्यावा॑पृथि॒व्योर् वरं॒ ॅवर॒म् द्यावा॑पृथि॒व्यो रे॒व मे॒वम् द्यावा॑पृथि॒व्योर् वर᳚म् । \newline
66. द्यावा॑पृथि॒व्योर् वरं॒ ॅवर॒म् द्यावा॑पृथि॒व्योर् द्यावा॑पृथि॒व्योर् वरं॒ ॅवेद॒ वेद॒ वर॒म् द्यावा॑पृथि॒व्योर् द्यावा॑पृथि॒व्योर् वरं॒ ॅवेद॑ । \newline
67. द्यावा॑पृथि॒व्योरिति॒ द्यावा᳚ - पृ॒थि॒व्योः । \newline
\pagebreak
\markright{ TS 2.5.2.6  \hfill https://www.vedavms.in \hfill}
\addcontentsline{toc}{section}{ TS 2.5.2.6 }
\section*{ TS 2.5.2.6 }

\textbf{TS 2.5.2.6 } \newline
\textbf{Samhita Paata} \newline

-र्वरं॒ ॅवेदैनं॒ ॅवरो॑ गच्छति॒ स आ॒भ्यामे॒व प्रसू॑त॒ इन्द्रो॑ वृ॒त्रम॑ह॒न् ते दे॒वा वृ॒त्रꣳ ह॒त्वाऽग्नीषोमा॑वब्रुवन्. ह॒व्यं नो॑ वहत॒मिति॒ ताव॑ब्रूता॒मप॑तेजसौ॒ वै त्यौ वृ॒त्रे वै त्ययो॒स्तेज॒ इति॒ ते᳚ऽब्रुव॒न् क इ॒दमच्छै॒तीति॒ गौरित्य॑ब्रुव॒न् गौर्वाव सर्व॑स्य मि॒त्रमिति॒ साऽब्र॑वी॒ - [  ] \newline

\textbf{Pada Paata} \newline

वर᳚म् । वेद॑ । एति॑ । ए॒न॒म् । वरः॑ । ग॒च्छ॒ति॒ । सः । आ॒भ्याम् । ए॒व । प्रसू॑त॒ इति॒ प्र - सू॒तः॒ । इन्द्रः॑ । वृ॒त्रम् । अ॒ह॒न्न् । ते । दे॒वाः । वृ॒त्रम् । ह॒त्वा । अ॒ग्नीषोमा॒वित्य॒ग्नी - सोमौ᳚ । अ॒ब्रु॒व॒न्न् । ह॒व्यम् । नः॒ । व॒ह॒त॒म् । इति॑ । तौ । अ॒ब्रू॒ता॒म् । अप॑तेजसा॒वित्यप॑-ते॒ज॒सौ॒ । वै । त्यौ । वृ॒त्रे । वै । त्ययोः᳚ । तेजः॑ । इति॑ । ते । अ॒ब्रु॒व॒न्न् । कः । इ॒दम् । अच्छ॑ । ए॒ति॒ । इति॑ । गौः । इति॑ । अ॒ब्रु॒व॒न्न् । गौः । वाव । सर्व॑स्य । मि॒त्रम् । इति॑ । सा । अ॒ब्र॒वी॒त् ।  \newline


\textbf{Krama Paata} \newline

वर॒म् ॅवेद॑ । वेदा । ऐन᳚म् । ए॒न॒म् ॅवरः॑ । वरो॑ गच्छति । ग॒च्छ॒ति॒ सः । स आ॒भ्याम् । आ॒भ्यामे॒व । ए॒व प्रसू॑तः । प्रसू॑त॒ इन्द्रः॑ । प्रसू॑त॒ इति॒ प्र - सू॒तः॒ । इन्द्रो॑ वृ॒त्रम् । वृ॒त्रम॑हन्न् । अ॒ह॒न् ते । ते दे॒वाः । दे॒वा वृ॒त्रम् । वृ॒त्रꣳ ह॒त्वा । ह॒त्वा ऽग्नीषोमौ᳚ । अ॒ग्नीषोमा॑वब्रुन्न् । अ॒ग्नीषोमा॒वित्य॒ग्नी - सोमौ᳚ । अ॒बु॒व॒न्॒. ह॒व्यम् । ह॒व्यम् नः॑ । नो॒ व॒ह॒त॒म् । व॒ह॒त॒मिति॑ । इति॒ तौ । ताव॑ब्रूताम् । अ॒ब्रू॒ता॒मप॑तेजसौ । अप॑तेजसौ॒ वै । अप॑तेजसा॒वित्यप॑ - ते॒ज॒सौ॒ । वै त्यौ । त्यौ वृ॒त्रे । वृ॒त्रे वै । वै त्ययोः᳚ । त्ययो॒ स्तेजः॑ । तेज॒ इति॑ । इति॒ ते । ते᳚ ऽब्रुवन्न् । अ॒ब्रु॒व॒न् कः । क इ॒दम् । इ॒दमच्छ॑ । अच्छै॑ति । ए॒तीति॑ । इति॒ गौः । गौरिति॑ । इत्य॑ब्रुवन्न् । अ॒ब्रु॒व॒न् गौः । गौर् वाव । वाव सर्व॑स्य । सर्व॑स्य मि॒त्रम् । मि॒त्र मिति॑ । इति॒ सा । सा ऽब्र॑वीत् । अ॒ब्र॒वी॒द् वर᳚म् \newline

\textbf{Jatai Paata} \newline

1. वरं॒ ॅवेद॒ वेद॒ वरं॒ ॅवरं॒ ॅवेद॑ । \newline
2. वेदा वेद॒ वेदा । \newline
3. ऐन॑ मेन॒ मैन᳚म् । \newline
4. ए॒नं॒ ॅवरो॒ वर॑ एन मेनं॒ ॅवरः॑ । \newline
5. वरो॑ गच्छति गच्छति॒ वरो॒ वरो॑ गच्छति । \newline
6. ग॒च्छ॒ति॒ स स ग॑च्छति गच्छति॒ सः । \newline
7. स आ॒भ्या मा॒भ्याꣳ स स आ॒भ्याम् । \newline
8. आ॒भ्या मे॒वैवाभ्या मा॒भ्या मे॒व । \newline
9. ए॒व प्रसू॑तः॒ प्रसू॑त ए॒वैव प्रसू॑तः । \newline
10. प्रसू॑त॒ इन्द्र॒ इन्द्रः॒ प्रसू॑तः॒ प्रसू॑त॒ इन्द्रः॑ । \newline
11. प्रसू॑त॒ इति॒ प्र - सू॒तः॒ । \newline
12. इन्द्रो॑ वृ॒त्रं ॅवृ॒त्र मिन्द्र॒ इन्द्रो॑ वृ॒त्रम् । \newline
13. वृ॒त्र म॑हन् नहन् वृ॒त्रं ॅवृ॒त्र म॑हन्न् । \newline
14. अ॒ह॒न् ते ते॑ ऽहन् नह॒न् ते । \newline
15. ते दे॒वा दे॒वा स्ते ते दे॒वाः । \newline
16. दे॒वा वृ॒त्रं ॅवृ॒त्रम् दे॒वा दे॒वा वृ॒त्रम् । \newline
17. वृ॒त्रꣳ ह॒त्वा ह॒त्वा वृ॒त्रं ॅवृ॒त्रꣳ ह॒त्वा । \newline
18. ह॒त्वा ऽग्नीषोमा॑ व॒ग्नीषोमौ॑ ह॒त्वा ह॒त्वा ऽग्नीषोमौ᳚ । \newline
19. अ॒ग्नीषोमा॑ वब्रुवन् नब्रुवन् न॒ग्नीषोमा॑ व॒ग्नीषोमा॑ वब्रुवन्न् । \newline
20. अ॒ग्नीषोमा॒वित्य॒ग्नी - सोमौ᳚ । \newline
21. अ॒ब्रु॒व॒न्॒. ह॒व्यꣳ ह॒व्य म॑ब्रुवन् नब्रुवन्. ह॒व्यम् । \newline
22. ह॒व्यम् नो॑ नो ह॒व्यꣳ ह॒व्यम् नः॑ । \newline
23. नो॒ व॒ह॒तं॒ ॅव॒ह॒त॒म् नो॒ नो॒ व॒ह॒त॒म् । \newline
24. व॒ह॒त॒ मितीति॑ वहतं ॅवहत॒ मिति॑ । \newline
25. इति॒ तौ ता वितीति॒ तौ । \newline
26. ता व॑ब्रूता मब्रूता॒म् तौ ता व॑ब्रूताम् । \newline
27. अ॒ब्रू॒ता॒ मप॑तेजसा॒ वप॑तेजसा वब्रूता मब्रूता॒ मप॑तेजसौ । \newline
28. अप॑तेजसौ॒ वै वा अप॑तेजसा॒ वप॑तेजसौ॒ वै । \newline
29. अप॑तेजसा॒वित्यप॑ - ते॒ज॒सौ॒ । \newline
30. वै त्यौ त्यौ वै वै त्यौ । \newline
31. त्यौ वृ॒त्रे वृ॒त्रे त्यौ त्यौ वृ॒त्रे । \newline
32. वृ॒त्रे वै वै वृ॒त्रे वृ॒त्रे वै । \newline
33. वै त्ययो॒ स्त्ययो॒र् वै वै त्ययोः᳚ । \newline
34. त्ययो॒ स्तेज॒ स्तेज॒ स्त्ययो॒ स्त्ययो॒ स्तेजः॑ । \newline
35. तेज॒ इतीति॒ तेज॒ स्तेज॒ इति॑ । \newline
36. इति॒ ते त इतीति॒ ते । \newline
37. ते᳚ ऽब्रुवन् नब्रुव॒न् ते ते᳚ ऽब्रुवन्न् । \newline
38. अ॒ब्रु॒व॒न् कः को᳚ ऽब्रुवन् नब्रुव॒न् कः । \newline
39. क इ॒द मि॒दम् कः क इ॒दम् । \newline
40. इ॒द मच्छाच्छे॒ द मि॒द मच्छ॑ । \newline
41. अच्छै᳚ त्ये॒त्यच्छा च्छै॑ति । \newline
42. ए॒तीती त्ये᳚त्ये॒ तीति॑ । \newline
43. इति॒ गौर् गौरितीति॒ गौः । \newline
44. गौरितीति॒ गौर् गौरिति॑ । \newline
45. इत्य॑ब्रुवन् नब्रुव॒न् निती त्य॑ब्रुवन्न् । \newline
46. अ॒ब्रु॒व॒न् गौर् गौ र॑ब्रुवन् नब्रुव॒न् गौः । \newline
47. गौर् वाव वाव गौर् गौर् वाव । \newline
48. वाव सर्व॑स्य॒ सर्व॑स्य॒ वाव वाव सर्व॑स्य । \newline
49. सर्व॑स्य मि॒त्रम् मि॒त्रꣳ सर्व॑स्य॒ सर्व॑स्य मि॒त्रम् । \newline
50. मि॒त्र मितीति॑ मि॒त्रम् मि॒त्र मिति॑ । \newline
51. इति॒ सा सेतीति॒ सा । \newline
52. सा ऽब्र॑वी दब्रवी॒थ् सा सा ऽब्र॑वीत् । \newline
53. अ॒ब्र॒वी॒द् वरं॒ ॅवर॑ मब्रवी दब्रवी॒द् वर᳚म् । \newline

\textbf{Ghana Paata } \newline

1. वरं॒ ॅवेद॒ वेद॒ वरं॒ ॅवरं॒ ॅवेदा वेद॒ वरं॒ ॅवरं॒ ॅवेदा । \newline
2. वेदा वेद॒ वेदैन॑ मेन॒ मा वेद॒ वेदैन᳚म् । \newline
3. ऐन॑ मेन॒ मैनं॒ ॅवरो॒ वर॑ एन॒ मैनं॒ ॅवरः॑ । \newline
4. ए॒नं॒ ॅवरो॒ वर॑ एन मेनं॒ ॅवरो॑ गच्छति गच्छति॒ वर॑ एन मेनं॒ ॅवरो॑ गच्छति । \newline
5. वरो॑ गच्छति गच्छति॒ वरो॒ वरो॑ गच्छति॒ स स ग॑च्छति॒ वरो॒ वरो॑ गच्छति॒ सः । \newline
6. ग॒च्छ॒ति॒ स स ग॑च्छति गच्छति॒ स आ॒भ्या मा॒भ्याꣳ स ग॑च्छति गच्छति॒ स आ॒भ्याम् । \newline
7. स आ॒भ्या मा॒भ्याꣳ स स आ॒भ्या मे॒वैवाभ्याꣳ स स आ॒भ्या मे॒व । \newline
8. आ॒भ्या मे॒वैवाभ्या मा॒भ्या मे॒व प्रसू॑तः॒ प्रसू॑त ए॒वाभ्या मा॒भ्या मे॒व प्रसू॑तः । \newline
9. ए॒व प्रसू॑तः॒ प्रसू॑त ए॒वैव प्रसू॑त॒ इन्द्र॒ इन्द्रः॒ प्रसू॑त ए॒वैव प्रसू॑त॒ इन्द्रः॑ । \newline
10. प्रसू॑त॒ इन्द्र॒ इन्द्रः॒ प्रसू॑तः॒ प्रसू॑त॒ इन्द्रो॑ वृ॒त्रं ॅवृ॒त्र मिन्द्रः॒ प्रसू॑तः॒ प्रसू॑त॒ इन्द्रो॑ वृ॒त्रम् । \newline
11. प्रसू॑त॒ इति॒ प्र - सू॒तः॒ । \newline
12. इन्द्रो॑ वृ॒त्रं ॅवृ॒त्र मिन्द्र॒ इन्द्रो॑ वृ॒त्र म॑हन् नहन् वृ॒त्र मिन्द्र॒ इन्द्रो॑ वृ॒त्र म॑हन्न् । \newline
13. वृ॒त्र म॑हन् नहन् वृ॒त्रं ॅवृ॒त्र म॑ह॒न् ते ते॑ ऽहन् वृ॒त्रं ॅवृ॒त्र म॑ह॒न् ते । \newline
14. अ॒ह॒न् ते ते॑ ऽहन् नह॒न् ते दे॒वा दे॒वा स्ते॑ ऽहन् नह॒न् ते दे॒वाः । \newline
15. ते दे॒वा दे॒वा स्ते ते दे॒वा वृ॒त्रं ॅवृ॒त्रम् दे॒वा स्ते ते दे॒वा वृ॒त्रम् । \newline
16. दे॒वा वृ॒त्रं ॅवृ॒त्रम् दे॒वा दे॒वा वृ॒त्रꣳ ह॒त्वा ह॒त्वा वृ॒त्रम् दे॒वा दे॒वा वृ॒त्रꣳ ह॒त्वा । \newline
17. वृ॒त्रꣳ ह॒त्वा ह॒त्वा वृ॒त्रं ॅवृ॒त्रꣳ ह॒त्वा ऽग्नीषोमा॑ व॒ग्नीषोमौ॑ ह॒त्वा वृ॒त्रं ॅवृ॒त्रꣳ ह॒त्वा ऽग्नीषोमौ᳚ । \newline
18. ह॒त्वा ऽग्नीषोमा॑ व॒ग्नीषोमौ॑ ह॒त्वा ह॒त्वा ऽग्नीषोमा॑ वब्रुवन् नब्रुवन् न॒ग्नीषोमौ॑ ह॒त्वा ह॒त्वा ऽग्नीषोमा॑ वब्रुवन्न् । \newline
19. अ॒ग्नीषोमा॑ वब्रुवन् नब्रुवन् न॒ग्नीषोमा॑ व॒ग्नीषोमा॑ वब्रुवन्. ह॒व्यꣳ ह॒व्य म॑ब्रुवन् न॒ग्नीषोमा॑ व॒ग्नीषोमा॑ वब्रुवन्. ह॒व्यम् । \newline
20. अ॒ग्नीषोमा॒वित्य॒ग्नी - सोमौ᳚ । \newline
21. अ॒ब्रु॒व॒न्॒. ह॒व्यꣳ ह॒व्य म॑ब्रुवन् नब्रुवन्. ह॒व्यम् नो॑ नो ह॒व्य म॑ब्रुवन् नब्रुवन्. ह॒व्यम् नः॑ । \newline
22. ह॒व्यम् नो॑ नो ह॒व्यꣳ ह॒व्यम् नो॑ वहतं ॅवहतन्नो ह॒व्यꣳ ह॒व्यम् नो॑ वहतम् । \newline
23. नो॒ व॒ह॒तं॒ ॅव॒ह॒त॒म् नो॒ नो॒ व॒ह॒त॒ मितीति॑ वहतम् नो नो वहत॒ मिति॑ । \newline
24. व॒ह॒त॒ मितीति॑ वहतं ॅवहत॒ मिति॒ तौ ता विति॑ वहतं ॅवहत॒ मिति॒ तौ । \newline
25. इति॒ तौ ता वितीति॒ ता व॑ब्रूता मब्रूता॒म् ता वितीति॒ ता व॑ब्रूताम् । \newline
26. ता व॑ब्रूता मब्रूता॒म् तौ ता व॑ब्रूता॒ मप॑तेजसा॒ वप॑तेजसा वब्रूता॒म् तौ ता व॑ब्रूता॒ मप॑तेजसौ । \newline
27. अ॒ब्रू॒ता॒ मप॑तेजसा॒ वप॑तेजसा वब्रूता मब्रूता॒ मप॑तेजसौ॒ वै वा अप॑तेजसा वब्रूता मब्रूता॒ मप॑तेजसौ॒ वै । \newline
28. अप॑तेजसौ॒ वै वा अप॑तेजसा॒ वप॑तेजसौ॒ वै त्यौ त्यौ वा अप॑तेजसा॒ वप॑तेजसौ॒ वै त्यौ । \newline
29. अप॑तेजसा॒वित्यप॑ - ते॒ज॒सौ॒ । \newline
30. वै त्यौ त्यौ वै वै त्यौ वृ॒त्रे वृ॒त्रे त्यौ वै वै त्यौ वृ॒त्रे । \newline
31. त्यौ वृ॒त्रे वृ॒त्रे त्यौ त्यौ वृ॒त्रे वै वै वृ॒त्रे त्यौ त्यौ वृ॒त्रे वै । \newline
32. वृ॒त्रे वै वै वृ॒त्रे वृ॒त्रे वै त्ययो॒ स्त्ययो॒र् वै वृ॒त्रे वृ॒त्रे वै त्ययोः᳚ । \newline
33. वै त्ययो॒ स्त्ययो॒र् वै वै त्ययो॒ स्तेज॒ स्तेज॒ स्त्ययो॒र् वै वै त्ययो॒ स्तेजः॑ । \newline
34. त्ययो॒ स्तेज॒ स्तेज॒ स्त्ययो॒ स्त्ययो॒ स्तेज॒ इतीति॒ तेज॒ स्त्ययो॒ स्त्ययो॒ स्तेज॒ इति॑ । \newline
35. तेज॒ इतीति॒ तेज॒ स्तेज॒ इति॒ ते त इति॒ तेज॒ स्तेज॒ इति॒ ते । \newline
36. इति॒ ते त इतीति॒ ते᳚ ऽब्रुवन् नब्रुव॒न् त इतीति॒ ते᳚ ऽब्रुवन्न् । \newline
37. ते᳚ ऽब्रुवन् नब्रुव॒न् ते ते᳚ ऽब्रुव॒न् कः को᳚ ऽब्रुव॒न् ते ते᳚ ऽब्रुव॒न् कः । \newline
38. अ॒ब्रु॒व॒न् कः को᳚ ऽब्रुवन् नब्रुव॒न् क इ॒द मि॒दम् को᳚ ऽब्रुवन् नब्रुव॒न् क इ॒दम् । \newline
39. क इ॒द मि॒दम् कः क इ॒द मच्छाच्छे॒ दम् कः क इ॒द मच्छ॑ । \newline
40. इ॒द मच्छाच्छे॒ द मि॒द मच्छै᳚ त्ये॒ त्यच्छे॒ द मि॒द मच्छै॑ति । \newline
41. अच्छै᳚ त्ये॒ त्यच्छा च्छै॒तीती त्ये॒ त्यच्छा च्छै॒तीति॑ । \newline
42. ए॒तीतीत्ये᳚ त्ये॒तीति॒ गौर् गौरित्ये᳚ त्ये॒तीति॒ गौः । \newline
43. इति॒ गौर् गौरितीति॒ गौरितीति॒ गौरितीति॒ गौरिति॑ । \newline
44. गौरितीति॒ गौर् गौरि त्य॑ब्रुवन् नब्रुव॒न् निति॒ गौर् गौरि त्य॑ब्रुवन्न् । \newline
45. इत्य॑ब्रुवन् नब्रुव॒न् नितीत्य॑ब्रुव॒न् गौर् गौर॑ब्रुव॒न् नितीत्य॑ब्रुव॒न् गौः । \newline
46. अ॒ब्रु॒व॒न् गौर् गौर॑ब्रुवन् नब्रुव॒न् गौर् वाव वाव गौर॑ब्रुवन् नब्रुव॒न् गौर् वाव । \newline
47. गौर् वाव वाव गौर् गौर् वाव सर्व॑स्य॒ सर्व॑स्य॒ वाव गौर् गौर् वाव सर्व॑स्य । \newline
48. वाव सर्व॑स्य॒ सर्व॑स्य॒ वाव वाव सर्व॑स्य मि॒त्रम् मि॒त्रꣳ सर्व॑स्य॒ वाव वाव सर्व॑स्य मि॒त्रम् । \newline
49. सर्व॑स्य मि॒त्रम् मि॒त्रꣳ सर्व॑स्य॒ सर्व॑स्य मि॒त्र मितीति॑ मि॒त्रꣳ सर्व॑स्य॒ सर्व॑स्य मि॒त्र मिति॑ । \newline
50. मि॒त्र मितीति॑ मि॒त्रम् मि॒त्र मिति॒ सा सेति॑ मि॒त्रम् मि॒त्र मिति॒ सा । \newline
51. इति॒ सा सेतीति॒ सा ऽब्र॑वी दब्रवी॒थ् सेतीति॒ सा ऽब्र॑वीत् । \newline
52. सा ऽब्र॑वी दब्रवी॒थ् सा सा ऽब्र॑वी॒द् वरं॒ ॅवर॑ मब्रवी॒थ् सा सा ऽब्र॑वी॒द् वर᳚म् । \newline
53. अ॒ब्र॒वी॒द् वरं॒ ॅवर॑ मब्रवी दब्रवी॒द् वरं॑ ॅवृणै वृणै॒ वर॑ मब्रवी दब्रवी॒द् वरं॑ ॅवृणै । \newline
\pagebreak
\markright{ TS 2.5.2.7  \hfill https://www.vedavms.in \hfill}
\addcontentsline{toc}{section}{ TS 2.5.2.7 }
\section*{ TS 2.5.2.7 }

\textbf{TS 2.5.2.7 } \newline
\textbf{Samhita Paata} \newline

-द्वरं॑ ॅवृणै॒ मय्ये॒व स॒तोऽभये॑न भुनजाद्ध्वा॒ इति॒ तद्-गौराऽह॑र॒त् तस्मा॒द्-गवि॑ स॒तोभये॑न भुञ्जत ए॒तद्वा अ॒ग्नेस्तेजो॒ यद् घृ॒तमे॒तथ् सोम॑स्य॒ यत् पयो॒ य ए॒वम॒ग्नीषोम॑यो॒ स्तेजो॒ वेद॑ तेज॒स्व्ये॑व भ॑वति ब्रह्मवा॒दिनो॑ वदन्ति किंदेव॒त्यं॑ पौर्णमा॒समिति॑ प्राजाप॒त्यमिति॑ ब्रूया॒त् तेनेन्द्रं॑ ज्ये॒ष्ठं पु॒त्रं नि॒रवा॑सायय॒दिति॒ तस्मा᳚- ( ) -ज्ज्ये॒ष्ठं पु॒त्रं धने॑न नि॒रव॑साययन्ति ॥ \newline

\textbf{Pada Paata} \newline

वर᳚म् । वृ॒णै॒ । मयि॑ । ए॒व । स॒ता । उ॒भये॑न । भु॒न॒जा॒द्ध्वै॒ । इति॑ । तत् । गौः । एति॑ । अ॒ह॒र॒त् । तस्मा᳚त् । गवि॑ । स॒ता । उ॒भये॑न । भु॒ञ्ज॒ते॒ । ए॒तत् । वै । अ॒ग्नेः । तेजः॑ । यत् । घृ॒तम् । ए॒तत् । सोम॑स्य । यत् । पयः॑ । यः । ए॒वं । अ॒ग्नीषोम॑यो॒रित्य॒ग्नी-सोम॑योः । तेजः॑ । वेद॑ । ते॒ज॒स्वी । ए॒व । भ॒व॒ति॒ । ब्र॒ह्म॒वा॒दिन॒ इति॑ ब्रह्म - वा॒दिनः॑ । व॒द॒न्ति॒ । कि॒दें॒व॒त्य॑मिति॑ किं - दे॒व॒त्य᳚म् । पौ॒र्ण॒मा॒समिति॑ पौर्ण - मा॒सम् । इति॑ । प्रा॒जा॒प॒त्यमिति॑ प्रजा - प॒त्यम् । इति॑ । ब्रू॒या॒त् । तेन॑ । इन्द्र᳚म् । ज्ये॒ष्ठम् । पु॒त्रम् । नि॒रवा॑सायय॒दिति॑ निः-अवा॑साययत् । इति॑ । तस्मा᳚त् ( ) । ज्ये॒ष्ठम् । पु॒त्रम् । धने॑न । नि॒रव॑सायय॒न्तीति॑ निः - अव॑साययन्ति ॥  \newline


\textbf{Krama Paata} \newline

वर॑म् ॅवृणै । वृ॒णै॒ मयि॑ । मय्ये॒व । ए॒व स॒ता । स॒तोभये॑न । उ॒भये॑न भुनजाद्ध्वै । भु॒न॒जा॒द्ध्वा॒ इति॑ । इति॒ तत् । तद् गौः । गौरा । आ ऽह॑रत् । अ॒ह॒र॒त् तस्मा᳚त् । तस्मा॒द् गवि॑ । गवि॑ स॒ता । स॒तोभये॑न । उ॒भये॑न भुञ्जते । भु॒ञ्ज॒त॒ ए॒तत् । ए॒तद् वै । वा अ॒ग्नेः । अ॒ग्नेस्तेजः॑ । तेजो॒ यत् । यद् घृ॒तम् । घृ॒तमे॒तत् । ए॒तथ् सोम॑स्य । सोम॑स्य॒ यत् । यत् पयः॑ । पयो॒ यः । य ए॒वम् । ए॒वम॒ग्नीषोम॑योः । अ॒ग्नीषोम॑यो॒ स्तेजः॑ । अ॒ग्नीषोम॑यो॒रित्य॒ग्नी - सोम॑योः । तेजो॒ वेद॑ । वेद॑ तेज॒स्वी । ते॒ज॒स्व्ये॑व । ए॒व भ॑वति । भ॒व॒ति॒ ब्र॒ह्म॒वा॒दिनः॑ । ब्र॒ह्म॒वा॒दिनो॑ वदन्ति । ब्र॒ह्म॒वा॒दिन॒ इति॑ ब्रह्म - वा॒दिनः॑ । व॒द॒न्ति॒ कि॒न्दे॒व॒त्य᳚म् । कि॒न्दे॒व॒त्य॑म् पौर्णमा॒सम् । कि॒न्दे॒व॒त्य॑मिति॑ किम् - दे॒व॒त्य᳚म् । पौ॒र्ण॒मा॒समिति॑ । पौ॒र्ण॒मा॒समिति॑ पौर्ण - मा॒सम् । इति॑ प्राजाप॒त्यम् । प्रा॒जा॒प॒त्यमिति॑ । प्रा॒जा॒प॒त्यमिति॑ प्राजा - प॒त्यम् । इति॑ ब्रूयात् । ब्रू॒या॒त् तेन॑ । तेनेन्द्र᳚म् । इन्द्र॑म् ज्ये॒ष्ठम् । ज्ये॒ष्ठम् पु॒त्रम् । पु॒त्रम् नि॒रवा॑साययत् । नि॒रवा॑सायय॒दिति॑ । नि॒रवा॑सायय॒दिति॑ निः - अवा॑साययत् । इति॒ तस्मा᳚त् ( ) । तस्मा᳚ज्ज्ये॒ष्ठम् । ज्ये॒ष्ठम् पु॒त्रम् । पु॒त्रम् धने॑न । धने॑न नि॒रव॑साययन्ति । नि॒रव॑सायय॒न्तीति॑ निः - अव॑साययन्ति । \newline

\textbf{Jatai Paata} \newline

1. वरं॑ ॅवृणै वृणै॒ वरं॒ ॅवरं॑ ॅवृणै । \newline
2. वृ॒णै॒ मयि॒ मयि॑ वृणै वृणै॒ मयि॑ । \newline
3. मय्ये॒वैव मयि॒ मय्ये॒व । \newline
4. ए॒व स॒ता स॒तैवैव स॒ता । \newline
5. स॒तो भये॑नो॒भये॑न स॒ता स॒तोभये॑न । \newline
6. उ॒भये॑न भुनजाद्ध्वै भुनजाद्ध्वा उ॒भये॑नो॒ भये॑न भुनजाद्ध्वै । \newline
7. भु॒न॒जा॒द्ध्वा॒ इतीति॑ भुनजाद्ध्वै भुनजाद्ध्वा॒ इति॑ । \newline
8. इति॒ तत् तदितीति॒ तत् । \newline
9. तद् गौर् गौ स्तत् तद् गौः । \newline
10. गौरा गौर् गौरा । \newline
11. आ ऽह॑र दहर॒दा ऽह॑रत् । \newline
12. अ॒ह॒र॒त् तस्मा॒त् तस्मा॑ दहर दहर॒त् तस्मा᳚त् । \newline
13. तस्मा॒द् गवि॒ गवि॒ तस्मा॒त् तस्मा॒द् गवि॑ । \newline
14. गवि॑ स॒ता स॒ता गवि॒ गवि॑ स॒ता । \newline
15. स॒तो भये॑नो॒ भये॑न स॒ता स॒तो भये॑न । \newline
16. उ॒भये॑न भुञ्जते भुञ्जत उ॒भये॑नो॒ भये॑न भुञ्जते । \newline
17. भु॒ञ्ज॒त॒ ए॒त दे॒तद् भु॑ञ्जते भुञ्जत ए॒तत् । \newline
18. ए॒तद् वै वा ए॒त दे॒तद् वै । \newline
19. वा अ॒ग्ने र॒ग्नेर् वै वा अ॒ग्नेः । \newline
20. अ॒ग्ने स्तेज॒ स्तेजो॒ ऽग्ने र॒ग्ने स्तेजः॑ । \newline
21. तेजो॒ यद् यत् तेज॒ स्तेजो॒ यत् । \newline
22. यद् घृ॒तम् घृ॒तं ॅयद् यद् घृ॒तम् । \newline
23. घृ॒त मे॒त दे॒तद् घृ॒तम् घृ॒त मे॒तत् । \newline
24. ए॒तथ् सोम॑स्य॒ सोम॑ स्यै॒त दे॒तथ् सोम॑स्य । \newline
25. सोम॑स्य॒ यद् यथ् सोम॑स्य॒ सोम॑स्य॒ यत् । \newline
26. यत् पयः॒ पयो॒ यद् यत् पयः॑ । \newline
27. पयो॒ यो यः पयः॒ पयो॒ यः । \newline
28. य ए॒व मे॒वं यो य ए॒वं । \newline
29. ए॒व म॒ग्नीषोम॑यो र॒ग्नीषोम॑यो रे॒व मे॒व म॒ग्नीषोम॑योः । \newline
30. अ॒ग्नीषोम॑यो॒ स्तेज॒ स्तेजो॒ ऽग्नीषोम॑यो र॒ग्नीषोम॑यो॒ स्तेजः॑ । \newline
31. अ॒ग्नीषोम॑यो॒रित्य॒ग्नी - सोम॑योः । \newline
32. तेजो॒ वेद॒ वेद॒ तेज॒ स्तेजो॒ वेद॑ । \newline
33. वेद॑ तेज॒स्वी ते॑ज॒स्वी वेद॒ वेद॑ तेज॒स्वी । \newline
34. ते॒ज॒स्व्ये॑वैव ते॑ज॒स्वी ते॑ज॒स्व्ये॑व । \newline
35. ए॒व भ॑वति भव त्ये॒वैव भ॑वति । \newline
36. भ॒व॒ति॒ ब्र॒ह्म॒वा॒दिनो᳚ ब्रह्मवा॒दिनो॑ भवति भवति ब्रह्मवा॒दिनः॑ । \newline
37. ब्र॒ह्म॒वा॒दिनो॑ वदन्ति वदन्ति ब्रह्मवा॒दिनो᳚ ब्रह्मवा॒दिनो॑ वदन्ति । \newline
38. ब्र॒ह्म॒वा॒दिन॒ इति॑ ब्रह्म - वा॒दिनः॑ । \newline
39. व॒द॒न्ति॒ कि॒न्दे॒व॒त्य॑म् किन्देव॒त्यं॑ ॅवदन्ति वदन्ति किन्देव॒त्य᳚म् । \newline
40. कि॒न्दे॒व॒त्य॑म् पौर्णमा॒सम् पौ᳚र्णमा॒सम् कि॑न्देव॒त्य॑म् किन्देव॒त्य॑म् पौर्णमा॒सम् । \newline
41. कि॒न्दे॒व॒त्य॑मिति॑ किं - दे॒व॒त्य᳚म् । \newline
42. पौ॒र्ण॒मा॒स मितीति॑ पौर्णमा॒सम् पौ᳚र्णमा॒स मिति॑ । \newline
43. पौ॒र्ण॒मा॒समिति॑ पौर्ण - मा॒सम् । \newline
44. इति॑ प्राजाप॒त्यम् प्रा॑जाप॒त्य मितीति॑ प्राजाप॒त्यम् । \newline
45. प्रा॒जा॒प॒त्य मितीति॑ प्राजाप॒त्यम् प्रा॑जाप॒त्य मिति॑ । \newline
46. प्रा॒जा॒प॒त्यमिति॑ प्राजा - प॒त्यम् । \newline
47. इति॑ ब्रूयाद् ब्रूया॒ दितीति॑ ब्रूयात् । \newline
48. ब्रू॒या॒त् तेन॒ तेन॑ ब्रूयाद् ब्रूया॒त् तेन॑ । \newline
49. तेने न्द्र॒ मिन्द्र॒म् तेन॒ तेने न्द्र᳚म् । \newline
50. इन्द्र॑म् ज्ये॒ष्ठम् ज्ये॒ष्ठ मिन्द्र॒ मिन्द्र॑म् ज्ये॒ष्ठम् । \newline
51. ज्ये॒ष्ठम् पु॒त्रम् पु॒त्रम् ज्ये॒ष्ठम् ज्ये॒ष्ठम् पु॒त्रम् । \newline
52. पु॒त्रम् नि॒रवा॑साययन् नि॒रवा॑साययत् पु॒त्रम् पु॒त्रम् नि॒रवा॑साययत् । \newline
53. नि॒रवा॑सायय॒ दितीति॑ नि॒रवा॑साययन् नि॒रवा॑सायय॒ दिति॑ । \newline
54. नि॒रवा॑सायय॒दिति॑ निः - अवा॑साययत् । \newline
55. इति॒ तस्मा॒त् तस्मा॒ दितीति॒ तस्मा᳚त् । \newline
56. तस्मा᳚ज् ज्ये॒ष्ठम् ज्ये॒ष्ठम् तस्मा॒त् तस्मा᳚ज् ज्ये॒ष्ठम् । \newline
57. ज्ये॒ष्ठम् पु॒त्रम् पु॒त्रम् ज्ये॒ष्ठम् ज्ये॒ष्ठम् पु॒त्रम् । \newline
58. पु॒त्रम् धने॑न॒ धने॑न पु॒त्रम् पु॒त्रम् धने॑न । \newline
59. धने॑न नि॒रव॑साययन्ति नि॒रव॑साययन्ति॒ धने॑न॒ धने॑न नि॒रव॑साययन्ति । \newline
60. नि॒रव॑सायय॒न्तीति॑ निः - अव॑साययन्ति । \newline

\textbf{Ghana Paata } \newline

1. वरं॑ ॅवृणै वृणै॒ वरं॒ ॅवरं॑ ॅवृणै॒ मयि॒ मयि॑ वृणै॒ वरं॒ ॅवरं॑ ॅवृणै॒ मयि॑ । \newline
2. वृ॒णै॒ मयि॒ मयि॑ वृणै वृणै॒ मय्ये॒वैव मयि॑ वृणै वृणै॒ मय्ये॒व । \newline
3. मय्ये॒वैव मयि॒ मय्ये॒व स॒ता स॒तैव मयि॒ मय्ये॒व स॒ता । \newline
4. ए॒व स॒ता स॒तैवैव स॒तोभये॑नो॒ भये॑न स॒तैवैव स॒तोभये॑न । \newline
5. स॒तोभये॑नो॒ भये॑न स॒ता स॒तोभये॑न भुनजाद्ध्वै भुनजाद्ध्वा उ॒भये॑न स॒ता स॒तोभये॑न भुनजाद्ध्वै । \newline
6. उ॒भये॑न भुनजाद्ध्वै भुनजाद्ध्वा उ॒भये॑नो॒भये॑न भुनजाद्ध्वा॒ इतीति॑ भुनजाद्ध्वा उ॒भये॑नो॒ भये॑न भुनजाद्ध्वा॒ इति॑ । \newline
7. भु॒न॒जा॒द्ध्वा॒ इतीति॑ भुनजाद्ध्वै भुनजाद्ध्वा॒ इति॒ तत् तदिति॑ भुनजाद्ध्वै भुनजाद्ध्वा॒ इति॒ तत् । \newline
8. इति॒ तत् तदितीति॒ तद् गौर् गौ स्तदितीति॒ तद् गौः । \newline
9. तद् गौर् गौ स्तत् तद् गौरा गौ स्तत् तद् गौरा । \newline
10. गौरा गौर् गौरा ऽह॑र दहर॒दा गौर् गौरा ऽह॑रत् । \newline
11. आ ऽह॑र दहर॒दा ऽह॑र॒त् तस्मा॒त् तस्मा॑ दहर॒दा ऽह॑र॒त् तस्मा᳚त् । \newline
12. अ॒ह॒र॒त् तस्मा॒त् तस्मा॑ दहर दहर॒त् तस्मा॒द् गवि॒ गवि॒ तस्मा॑ दहर दहर॒त् तस्मा॒द् गवि॑ । \newline
13. तस्मा॒द् गवि॒ गवि॒ तस्मा॒त् तस्मा॒द् गवि॑ स॒ता स॒ता गवि॒ तस्मा॒त् तस्मा॒द् गवि॑ स॒ता । \newline
14. गवि॑ स॒ता स॒ता गवि॒ गवि॑ स॒तोभये॑नो॒ भये॑न स॒ता गवि॒ गवि॑ स॒तोभये॑न । \newline
15. स॒तोभये॑नो॒ भये॑न स॒ता स॒तोभये॑न भुञ्जते भुञ्जत उ॒भये॑न स॒ता स॒तोभये॑न भुञ्जते । \newline
16. उ॒भये॑न भुञ्जते भुञ्जत उ॒भये॑नो॒ भये॑न भुञ्जत ए॒तदे॒तद् भु॑ञ्जत उ॒भये॑नो॒ भये॑न भुञ्जत ए॒तत् । \newline
17. भु॒ञ्ज॒त॒ ए॒त दे॒तद् भु॑ञ्जते भुञ्जत ए॒तद् वै वा ए॒तद् भु॑ञ्जते भुञ्जत ए॒तद् वै । \newline
18. ए॒तद् वै वा ए॒त दे॒तद् वा अ॒ग्ने र॒ग्नेर् वा ए॒त दे॒तद् वा अ॒ग्नेः । \newline
19. वा अ॒ग्ने र॒ग्नेर् वै वा अ॒ग्ने स्तेज॒ स्तेजो॒ ऽग्नेर् वै वा अ॒ग्ने स्तेजः॑ । \newline
20. अ॒ग्ने स्तेज॒ स्तेजो॒ ऽग्ने र॒ग्ने स्तेजो॒ यद् यत् तेजो॒ ऽग्ने र॒ग्ने स्तेजो॒ यत् । \newline
21. तेजो॒ यद् यत् तेज॒ स्तेजो॒ यद् घृ॒तम् घृ॒तं ॅयत् तेज॒ स्तेजो॒ यद् घृ॒तम् । \newline
22. यद् घृ॒तम् घृ॒तं ॅयद् यद् घृ॒त मे॒त दे॒तद् घृ॒तं ॅयद् यद् घृ॒त मे॒तत् । \newline
23. घृ॒त मे॒त दे॒तद् घृ॒तम् घृ॒त मे॒तथ् सोम॑स्य॒ सोम॑स्यै॒तद् घृ॒तम् घृ॒त मे॒तथ् सोम॑स्य । \newline
24. ए॒तथ् सोम॑स्य॒ सोम॑स्यै॒त दे॒तथ् सोम॑स्य॒ यद् यथ् सोम॑स्यै॒त दे॒तथ् सोम॑स्य॒ यत् । \newline
25. सोम॑स्य॒ यद् यथ् सोम॑स्य॒ सोम॑स्य॒ यत् पयः॒ पयो॒ यथ् सोम॑स्य॒ सोम॑स्य॒ यत् पयः॑ । \newline
26. यत् पयः॒ पयो॒ यद् यत् पयो॒ यो यः पयो॒ यद् यत् पयो॒ यः । \newline
27. पयो॒ यो यः पयः॒ पयो॒ य ए॒व मे॒वं यः पयः॒ पयो॒ य ए॒वं । \newline
28. य ए॒व मे॒वं यो य ए॒व म॒ग्नीषोम॑यो र॒ग्नीषोम॑यो रे॒वं यो य ए॒व म॒ग्नीषोम॑योः । \newline
29. ए॒व म॒ग्नीषोम॑यो र॒ग्नीषोम॑यो रे॒व मे॒व म॒ग्नीषोम॑यो॒ स्तेज॒ स्तेजो॒ ऽग्नीषोम॑यो रे॒व मे॒व म॒ग्नीषोम॑यो॒ स्तेजः॑ । \newline
30. अ॒ग्नीषोम॑यो॒ स्तेज॒ स्तेजो॒ ऽग्नीषोम॑यो र॒ग्नीषोम॑यो॒ स्तेजो॒ वेद॒ वेद॒ तेजो॒ ऽग्नीषोम॑यो र॒ग्नीषोम॑यो॒ स्तेजो॒ वेद॑ । \newline
31. अ॒ग्नीषोम॑यो॒रित्य॒ग्नी - सोम॑योः । \newline
32. तेजो॒ वेद॒ वेद॒ तेज॒ स्तेजो॒ वेद॑ तेज॒स्वी ते॑ज॒स्वी वेद॒ तेज॒ स्तेजो॒ वेद॑ तेज॒स्वी । \newline
33. वेद॑ तेज॒स्वी ते॑ज॒स्वी वेद॒ वेद॑ तेज॒स्व्ये॑वैव ते॑ज॒स्वी वेद॒ वेद॑ तेज॒स्व्ये॑व । \newline
34. ते॒ज॒स्व्ये॑वैव ते॑ज॒स्वी ते॑ज॒स्व्ये॑व भ॑वति भवत्ये॒व ते॑ज॒स्वी ते॑ज॒स्व्ये॑व भ॑वति । \newline
35. ए॒व भ॑वति भवत्ये॒वैव भ॑वति ब्रह्मवा॒दिनो᳚ ब्रह्मवा॒दिनो॑ भवत्ये॒वैव भ॑वति ब्रह्मवा॒दिनः॑ । \newline
36. भ॒व॒ति॒ ब्र॒ह्म॒वा॒दिनो᳚ ब्रह्मवा॒दिनो॑ भवति भवति ब्रह्मवा॒दिनो॑ वदन्ति वदन्ति ब्रह्मवा॒दिनो॑ भवति भवति ब्रह्मवा॒दिनो॑ वदन्ति । \newline
37. ब्र॒ह्म॒वा॒दिनो॑ वदन्ति वदन्ति ब्रह्मवा॒दिनो᳚ ब्रह्मवा॒दिनो॑ वदन्ति किन्देव॒त्य॑म् किन्देव॒त्यं॑ ॅवदन्ति ब्रह्मवा॒दिनो᳚ ब्रह्मवा॒दिनो॑ वदन्ति किन्देव॒त्य᳚म् । \newline
38. ब्र॒ह्म॒वा॒दिन॒ इति॑ ब्रह्म - वा॒दिनः॑ । \newline
39. व॒द॒न्ति॒ कि॒न्दे॒व॒त्य॑म् किन्देव॒त्यं॑ ॅवदन्ति वदन्ति किन्देव॒त्य॑म् पौर्णमा॒सम् पौ᳚र्णमा॒सम् कि॑न्देव॒त्यं॑ ॅवदन्ति वदन्ति किन्देव॒त्य॑म् पौर्णमा॒सम् । \newline
40. कि॒न्दे॒व॒त्य॑म् पौर्णमा॒सम् पौ᳚र्णमा॒सम् कि॑न्देव॒त्य॑म् किन्देव॒त्य॑म् पौर्णमा॒स मितीति॑ पौर्णमा॒सम् कि॑न्देव॒त्य॑म् किन्देव॒त्य॑म् पौर्णमा॒स मिति॑ । \newline
41. कि॒न्दे॒व॒त्य॑मिति॑ किं - दे॒व॒त्य᳚म् । \newline
42. पौ॒र्ण॒मा॒स मितीति॑ पौर्णमा॒सम् पौ᳚र्णमा॒स मिति॑ प्राजाप॒त्यम् प्रा॑जाप॒त्य मिति॑ पौर्णमा॒सम् पौ᳚र्णमा॒स मिति॑ प्राजाप॒त्यम् । \newline
43. पौ॒र्ण॒मा॒समिति॑ पौर्ण - मा॒सम् । \newline
44. इति॑ प्राजाप॒त्यम् प्रा॑जाप॒त्य मितीति॑ प्राजाप॒त्य मितीति॑ प्राजाप॒त्य मितीति॑ प्राजाप॒त्य मिति॑ । \newline
45. प्रा॒जा॒प॒त्य मितीति॑ प्राजाप॒त्यम् प्रा॑जाप॒त्य मिति॑ ब्रूयाद् ब्रूया॒दिति॑ प्राजाप॒त्यम् प्रा॑जाप॒त्य मिति॑ ब्रूयात् । \newline
46. प्रा॒जा॒प॒त्यमिति॑ प्राजा - प॒त्यम् । \newline
47. इति॑ ब्रूयाद् ब्रूया॒दितीति॑ ब्रूया॒त् तेन॒ तेन॑ ब्रूया॒दितीति॑ ब्रूया॒त् तेन॑ । \newline
48. ब्रू॒या॒त् तेन॒ तेन॑ ब्रूयाद् ब्रूया॒त् तेने न्द्र॒ मिन्द्र॒म् तेन॑ ब्रूयाद् ब्रूया॒त् तेने न्द्र᳚म् । \newline
49. तेने न्द्र॒ मिन्द्र॒म् तेन॒ तेने न्द्र॑म् ज्ये॒ष्ठम् ज्ये॒ष्ठ मिन्द्र॒म् तेन॒ तेने न्द्र॑म् ज्ये॒ष्ठम् । \newline
50. इन्द्र॑म् ज्ये॒ष्ठम् ज्ये॒ष्ठ मिन्द्र॒ मिन्द्र॑म् ज्ये॒ष्ठम् पु॒त्रम् पु॒त्रम् ज्ये॒ष्ठ मिन्द्र॒ मिन्द्र॑म् ज्ये॒ष्ठम् पु॒त्रम् । \newline
51. ज्ये॒ष्ठम् पु॒त्रम् पु॒त्रम् ज्ये॒ष्ठम् ज्ये॒ष्ठम् पु॒त्रम् नि॒रवा॑साययन् नि॒रवा॑साययत् पु॒त्रम् ज्ये॒ष्ठम् ज्ये॒ष्ठम् पु॒त्रम् नि॒रवा॑साययत् । \newline
52. पु॒त्रम् नि॒रवा॑साययन् नि॒रवा॑साययत् पु॒त्रम् पु॒त्रम् नि॒रवा॑सायय॒ दितीति॑ नि॒रवा॑साययत् पु॒त्रम् पु॒त्रम् नि॒रवा॑सायय॒ दिति॑ । \newline
53. नि॒रवा॑सायय॒ दितीति॑ नि॒रवा॑साययन् नि॒रवा॑सायय॒ दिति॒ तस्मा॒त् तस्मा॒दिति॑ नि॒रवा॑साययन् नि॒रवा॑सायय॒ दिति॒ तस्मा᳚त् । \newline
54. नि॒रवा॑सायय॒दिति॑ निः - अवा॑साययत् । \newline
55. इति॒ तस्मा॒त् तस्मा॒ दितीति॒ तस्मा᳚ज् ज्ये॒ष्ठम् ज्ये॒ष्ठम् तस्मा॒ दितीति॒ तस्मा᳚ज् ज्ये॒ष्ठम् । \newline
56. तस्मा᳚ज् ज्ये॒ष्ठम् ज्ये॒ष्ठम् तस्मा॒त् तस्मा᳚ज् ज्ये॒ष्ठम् पु॒त्रम् पु॒त्रम् ज्ये॒ष्ठम् तस्मा॒त् तस्मा᳚ज् ज्ये॒ष्ठम् पु॒त्रम् । \newline
57. ज्ये॒ष्ठम् पु॒त्रम् पु॒त्रम् ज्ये॒ष्ठम् ज्ये॒ष्ठम् पु॒त्रम् धने॑न॒ धने॑न पु॒त्रम् ज्ये॒ष्ठम् ज्ये॒ष्ठम् पु॒त्रम् धने॑न । \newline
58. पु॒त्रम् धने॑न॒ धने॑न पु॒त्रम् पु॒त्रम् धने॑न नि॒रव॑साययन्ति नि॒रव॑साययन्ति॒ धने॑न पु॒त्रम् पु॒त्रम् धने॑न नि॒रव॑साययन्ति । \newline
59. धने॑न नि॒रव॑साययन्ति नि॒रव॑साययन्ति॒ धने॑न॒ धने॑न नि॒रव॑साययन्ति । \newline
60. नि॒रव॑सायय॒न्तीति॑ निः - अव॑साययन्ति । \newline
\pagebreak
\markright{ TS 2.5.3.1  \hfill https://www.vedavms.in \hfill}
\addcontentsline{toc}{section}{ TS 2.5.3.1 }
\section*{ TS 2.5.3.1 }

\textbf{TS 2.5.3.1 } \newline
\textbf{Samhita Paata} \newline

इन्द्रं॑ ॅवृ॒त्रं ज॑घ्नि॒वाꣳसं॒ मृधो॒ऽभि प्रावे॑पन्त॒ स ए॒तं ॅवै॑मृ॒धं पू॒ऎणमा॑सेऽनुनिर्वा॒प्य॑मपश्य॒त् तं निर॑वप॒त् तेन॒ वै स मृधोऽपा॑हत॒ यद्वै॑मृ॒धः पू॒र्णमा॑सेऽनुनिर्वा॒प्यो॑ भव॑ति॒ मृध॑ ए॒व तेन॒ यज॑मा॒नो ऽप॑ हत॒ इन्द्रो॑ वृ॒त्रꣳ ह॒त्वा दे॒वता॑भिश्चेन्द्रि॒येण॑ च॒ व्या᳚र्द्ध्यत॒ स ए॒तमा᳚ग्ने॒य-म॒ष्टाक॑पाल-ममावा॒स्या॑यामपश्यदै॒न्द्रं दधि॒ - [  ] \newline

\textbf{Pada Paata} \newline

इन्द्र᳚म् । वृ॒त्रम् । ज॒घ्नि॒वाꣳस᳚म् । मृधः॑ । अ॒भि । प्रेति॑ । अ॒वे॒प॒न्त॒ । सः । ए॒तम् । वै॒मृ॒धम् । पू॒र्णमा॑स॒ इति॑ पू॒र्ण - मा॒से॒ । अ॒नु॒नि॒र्वा॒प्य॑मित्य॑नु - नि॒र्वा॒प्य᳚म् । अ॒प॒श्य॒त् । तम् । निरिति॑ । अ॒व॒प॒त् । तेन॑ । वै । सः । मृधः॑ । अपेति॑ । अ॒ह॒त॒ । यत् । वै॒मृ॒धः । पू॒र्णमा॑स॒ इति॑ पू॒र्ण - मा॒से॒ । अ॒नु॒नि॒र्वा॒प्य॑ इत्य॑नु - नि॒र्वा॒प्यः॑ । भव॑ति । मृधः॑ । ए॒व । तेन॑ । यज॑मानः । अपेति॑ । ह॒ते॒ । इन्द्रः॑ । वृ॒त्रम् । ह॒त्वा । दे॒वता॑भिः । च॒ । इ॒न्द्रि॒येण॑ । च॒ । वीति॑ । आ॒र्द्ध्य॒त॒ । सः । ए॒तम् । आ॒ग्ने॒यम् । अ॒ष्टाक॑पाल॒मित्य॒ष्टा - क॒पा॒ल॒म् । अ॒मा॒वा॒स्या॑या॒मित्य॑मा - वा॒स्या॑याम् । अ॒प॒श्य॒त् । ऐ॒न्द्रम् । दधि॑ ।  \newline


\textbf{Krama Paata} \newline

इन्द्र॑म् ॅवृ॒त्रम् । वृ॒त्रम् ज॑घ्नि॒वाꣳस᳚म् । ज॒घ्नि॒वाꣳस॒म् मृधः॑ । मृधो॒ऽभि । अ॒भि प्र । प्रावे॑पन्त । अ॒वे॒प॒न्त॒ सः । स ए॒तम् । ए॒तम् ॅवै॑मृ॒धम् । वै॒मृ॒धम् पू॒र्णमा॑से । पू॒र्णमा॑से ऽनुनिर्वा॒प्य᳚म् । पू॒णामा॑स॒ इति॑ पू॒र्ण - मा॒से॒ । अ॒नु॒नि॒र्वा॒प्य॑मपश्यत् । अ॒नु॒नि॒र्वा॒प्य॑मित्य॑नु - नि॒र्वा॒प्य᳚म् । अ॒प॒श्य॒त् तम् । तम् निः । निर॑वपत् । अ॒व॒प॒त् तेन॑ । तेन॒ वै । वै सः । स मृधः॑ । मृधोऽप॑ । अपा॑हत । अ॒ह॒त॒ यत् । यद् वै॑मृ॒धः । वै॒मृ॒धः पू॒र्णमा॑से । पू॒र्णमा॑से ऽनुनिर्वा॒प्यः॑ । पू॒र्णमा॑स॒ इति॑ पू॒र्ण - मा॒से॒ । अ॒नु॒नि॒र्वा॒प्यो॑ भव॑ति । अ॒नु॒नि॒र्वा॒प्य॑ इत्य॑नु - नि॒र्वा॒प्यः॑ । भव॑ति॒ मृधः॑ । मृध॑ ए॒व । ए॒व तेन॑ । तेन॒ यज॑मानः । यज॑मा॒नो ऽप॑ । अप॑ हते । ह॒त॒ इन्द्रः॑ । इन्द्रो॑ वृ॒त्रम् । वृ॒त्रꣳ ह॒त्वा । ह॒त्वा दे॒वता॑भिः । दे॒वता॑भिश्च । चे॒न्द्रि॒येण॑ । इ॒न्द्रि॒येण॑ च । च॒ वि । व्या᳚र्द्ध्यत । आ॒र्द्ध्य॒त॒ सः । स ए॒तम् । ए॒तमा᳚ग्ने॒यम् । आ॒ग्ने॒यम॒ष्टाक॑पालम् । अ॒ष्टाक॑पालममावा॒स्या॑याम् । अ॒ष्टाक॑पाल॒मित्य॒ष्टा - क॒पा॒ल॒म् । अ॒मा॒वा॒स्या॑यामपश्यत् । अ॒मा॒वा॒स्या॑या॒मित्य॑मा - वा॒स्या॑याम् । अ॒प॒श्य॒दै॒न्द्रम् । ऐ॒न्द्रम् दधि॑ । दधि॒ तम् \newline

\textbf{Jatai Paata} \newline

1. इन्द्रं॑ ॅवृ॒त्रं ॅवृ॒त्र मिन्द्र॒ मिन्द्रं॑ ॅवृ॒त्रम् । \newline
2. वृ॒त्रम् ज॑घ्नि॒वाꣳस॑म् जघ्नि॒वाꣳसं॑ ॅवृ॒त्रं ॅवृ॒त्रम् ज॑घ्नि॒वाꣳस᳚म् । \newline
3. ज॒घ्नि॒वाꣳस॒म् मृधो॒ मृधो॑ जघ्नि॒वाꣳस॑म् जघ्नि॒वाꣳस॒म् मृधः॑ । \newline
4. मृधो॒ ऽभ्य॑भि मृधो॒ मृधो॒ ऽभि । \newline
5. अ॒भि प्र प्राभ्य॑भि प्र । \newline
6. प्रावे॑पन्ता वेपन्त॒ प्र प्रावे॑पन्त । \newline
7. अ॒वे॒प॒न्त॒ स सो॑ ऽवेपन्ता वेपन्त॒ सः । \newline
8. स ए॒त मे॒तꣳ स स ए॒तम् । \newline
9. ए॒तं ॅवै॑मृ॒धं ॅवै॑मृ॒ध मे॒त मे॒तं ॅवै॑मृ॒धम् । \newline
10. वै॒मृ॒धम् पू॒र्णमा॑से पू॒र्णमा॑से वैमृ॒धं ॅवै॑मृ॒धम् पू॒र्णमा॑से । \newline
11. पू॒र्णमा॑से ऽनुनिर्वा॒प्य॑ मनुनिर्वा॒प्य॑म् पू॒र्णमा॑से पू॒र्णमा॑से ऽनुनिर्वा॒प्य᳚म् । \newline
12. पू॒र्णमा॑स॒ इति॑ पू॒र्ण - मा॒से॒ । \newline
13. अ॒नु॒नि॒र्वा॒प्य॑ मपश्य दपश्य दनुनिर्वा॒प्य॑ मनुनिर्वा॒प्य॑ मपश्यत् । \newline
14. अ॒नु॒नि॒र्वा॒प्य॑मित्य॑नु - नि॒र्वा॒प्य᳚म् । \newline
15. अ॒प॒श्य॒त् तम् त म॑पश्य दपश्य॒त् तम् । \newline
16. तम् निर् णिष् टम् तम् निः । \newline
17. निर॑वप दवप॒न् निर् णिर॑वपत् । \newline
18. अ॒व॒प॒त् तेन॒ तेना॑वप दवप॒त् तेन॑ । \newline
19. तेन॒ वै वै तेन॒ तेन॒ वै । \newline
20. वै स स वै वै सः । \newline
21. स मृधो॒ मृधः॒ स स मृधः॑ । \newline
22. मृधो ऽपाप॒ मृधो॒ मृधो ऽप॑ । \newline
23. अपा॑हता ह॒ता पापा॑हत । \newline
24. अ॒ह॒त॒ यद् यद॑हता हत॒ यत् । \newline
25. यद् वै॑मृ॒धो वै॑मृ॒धो यद् यद् वै॑मृ॒धः । \newline
26. वै॒मृ॒धः पू॒र्णमा॑से पू॒र्णमा॑से वैमृ॒धो वै॑मृ॒धः पू॒र्णमा॑से । \newline
27. पू॒र्णमा॑से ऽनुनिर्वा॒प्यो॑ ऽनुनिर्वा॒प्यः॑ पू॒र्णमा॑से पू॒र्णमा॑से ऽनुनिर्वा॒प्यः॑ । \newline
28. पू॒र्णमा॑स॒ इति॑ पू॒र्ण - मा॒से॒ । \newline
29. अ॒नु॒नि॒र्वा॒प्यो॑ भव॑ति॒ भव॑ त्यनुनिर्वा॒प्यो॑ ऽनुनिर्वा॒प्यो॑ भव॑ति । \newline
30. अ॒नु॒नि॒र्वा॒प्य॑ इत्य॑नु - नि॒र्वा॒प्यः॑ । \newline
31. भव॑ति॒ मृधो॒ मृधो॒ भव॑ति॒ भव॑ति॒ मृधः॑ । \newline
32. मृध॑ ए॒वैव मृधो॒ मृध॑ ए॒व । \newline
33. ए॒व तेन॒ तेनै॒वैव तेन॑ । \newline
34. तेन॒ यज॑मानो॒ यज॑मान॒ स्तेन॒ तेन॒ यज॑मानः । \newline
35. यज॑मा॒नो ऽपाप॒ यज॑मानो॒ यज॑मा॒नो ऽप॑ । \newline
36. अप॑ हते ह॒ते ऽपाप॑ हते । \newline
37. ह॒त॒ इन्द्र॒ इन्द्रो॑ हते हत॒ इन्द्रः॑ । \newline
38. इन्द्रो॑ वृ॒त्रं ॅवृ॒त्र मिन्द्र॒ इन्द्रो॑ वृ॒त्रम् । \newline
39. वृ॒त्रꣳ ह॒त्वा ह॒त्वा वृ॒त्रं ॅवृ॒त्रꣳ ह॒त्वा । \newline
40. ह॒त्वा दे॒वता॑भिर् दे॒वता॑भिर्. ह॒त्वा ह॒त्वा दे॒वता॑भिः । \newline
41. दे॒वता॑भिश्च च दे॒वता॑भिर् दे॒वता॑भिश्च । \newline
42. चे॒ न्द्रि॒येणे᳚ न्द्रि॒येण॑ च चे न्द्रि॒येण॑ । \newline
43. इ॒न्द्रि॒येण॑ च चे न्द्रि॒येणे᳚ न्द्रि॒येण॑ च । \newline
44. च॒ वि वि च॑ च॒ वि । \newline
45. व्या᳚र्द्ध्यता र्द्ध्यत॒ वि व्या᳚र्द्ध्यत । \newline
46. आ॒र्द्ध्य॒त॒ स स आ᳚र्द्ध्यता र्द्ध्यत॒ सः । \newline
47. स ए॒त मे॒तꣳ स स ए॒तम् । \newline
48. ए॒त मा᳚ग्ने॒य मा᳚ग्ने॒य मे॒त मे॒त मा᳚ग्ने॒यम् । \newline
49. आ॒ग्ने॒य म॒ष्टाक॑पाल म॒ष्टाक॑पाल माग्ने॒य मा᳚ग्ने॒य म॒ष्टाक॑पालम् । \newline
50. अ॒ष्टाक॑पाल ममावा॒स्या॑या ममावा॒स्या॑या म॒ष्टाक॑पाल म॒ष्टाक॑पाल ममावा॒स्या॑याम् । \newline
51. अ॒ष्टाक॑पाल॒मित्य॒ष्टा - क॒पा॒ल॒म् । \newline
52. अ॒मा॒वा॒स्या॑या मपश्य दपश्य दमावा॒स्या॑या ममावा॒स्या॑या मपश्यत् । \newline
53. अ॒मा॒वा॒स्या॑या॒मित्य॑मा - वा॒स्या॑याम् । \newline
54. अ॒प॒श्य॒ दै॒न्द्र मै॒न्द्र म॑पश्य दपश्य दै॒न्द्रम् । \newline
55. ऐ॒न्द्रम् दधि॒ दध्यै॒न्द्र मै॒न्द्रम् दधि॑ । \newline
56. दधि॒ तम् तम् दधि॒ दधि॒ तम् । \newline

\textbf{Ghana Paata } \newline

1. इन्द्रं॑ ॅवृ॒त्रं ॅवृ॒त्र मिन्द्र॒ मिन्द्रं॑ ॅवृ॒त्रम् ज॑घ्नि॒वाꣳस॑म् जघ्नि॒वाꣳसं॑ ॅवृ॒त्र मिन्द्र॒ मिन्द्रं॑ ॅवृ॒त्रम् ज॑घ्नि॒वाꣳस᳚म् । \newline
2. वृ॒त्रम् ज॑घ्नि॒वाꣳस॑म् जघ्नि॒वाꣳसं॑ ॅवृ॒त्रं ॅवृ॒त्रम् ज॑घ्नि॒वाꣳस॒म् मृधो॒ मृधो॑ जघ्नि॒वाꣳसं॑ ॅवृ॒त्रं ॅवृ॒त्रम् ज॑घ्नि॒वाꣳस॒म् मृधः॑ । \newline
3. ज॒घ्नि॒वाꣳस॒म् मृधो॒ मृधो॑ जघ्नि॒वाꣳस॑म् जघ्नि॒वाꣳस॒म् मृधो॒ ऽभ्य॑भि मृधो॑ जघ्नि॒वाꣳस॑म् जघ्नि॒वाꣳस॒म् मृधो॒ ऽभि । \newline
4. मृधो॒ ऽभ्य॑भि मृधो॒ मृधो॒ ऽभि प्र प्राभि मृधो॒ मृधो॒ ऽभि प्र । \newline
5. अ॒भि प्र प्राभ्य॑भि प्रावे॑पन्ता वेपन्त॒ प्राभ्य॑भि प्रावे॑पन्त । \newline
6. प्रावे॑पन्ता वेपन्त॒ प्र प्रावे॑पन्त॒ स सो॑ ऽवेपन्त॒ प्र प्रावे॑पन्त॒ सः । \newline
7. अ॒वे॒प॒न्त॒ स सो॑ ऽवेपन्ता वेपन्त॒ स ए॒त मे॒तꣳ सो॑ ऽवेपन्ता वेपन्त॒ स ए॒तम् । \newline
8. स ए॒त मे॒तꣳ स स ए॒तं ॅवै॑मृ॒धं ॅवै॑मृ॒ध मे॒तꣳ स स ए॒तं ॅवै॑मृ॒धम् । \newline
9. ए॒तं ॅवै॑मृ॒धं ॅवै॑मृ॒ध मे॒त मे॒तं ॅवै॑मृ॒धम् पू॒र्णमा॑से पू॒र्णमा॑से वैमृ॒ध मे॒त मे॒तं ॅवै॑मृ॒धम् पू॒र्णमा॑से । \newline
10. वै॒मृ॒धम् पू॒र्णमा॑से पू॒र्णमा॑से वैमृ॒धं ॅवै॑मृ॒धम् पू॒र्णमा॑से ऽनुनिर्वा॒प्य॑ मनुनिर्वा॒प्य॑म् पू॒र्णमा॑से वैमृ॒धं ॅवै॑मृ॒धम् पू॒र्णमा॑से ऽनुनिर्वा॒प्य᳚म् । \newline
11. पू॒र्णमा॑से ऽनुनिर्वा॒प्य॑ मनुनिर्वा॒प्य॑म् पू॒र्णमा॑से पू॒र्णमा॑से ऽनुनिर्वा॒प्य॑ मपश्य दपश्य दनुनिर्वा॒प्य॑म् पू॒र्णमा॑से पू॒र्णमा॑से ऽनुनिर्वा॒प्य॑ मपश्यत् । \newline
12. पू॒र्णमा॑स॒ इति॑ पू॒र्ण - मा॒से॒ । \newline
13. अ॒नु॒नि॒र्वा॒प्य॑ मपश्य दपश्य दनुनिर्वा॒प्य॑ मनुनिर्वा॒प्य॑ मपश्य॒त् तम् त म॑पश्य दनुनिर्वा॒प्य॑ मनुनिर्वा॒प्य॑ मपश्य॒त् तम् । \newline
14. अ॒नु॒नि॒र्वा॒प्य॑मित्य॑नु - नि॒र्वा॒प्य᳚म् । \newline
15. अ॒प॒श्य॒त् तम् त म॑पश्य दपश्य॒त् तम् निर् णिष्ट म॑पश्य दपश्य॒त् तम् निः । \newline
16. तम् निर् णिष् टम् तम् निर॑वप दवप॒न् निष् टम् तम् निर॑वपत् । \newline
17. निर॑वप दवप॒न् निर् णिर॑वप॒त् तेन॒ तेना॑वप॒न् निर् णिर॑वप॒त् तेन॑ । \newline
18. अ॒व॒प॒त् तेन॒ तेना॑वप दवप॒त् तेन॒ वै वै तेना॑वप दवप॒त् तेन॒ वै । \newline
19. तेन॒ वै वै तेन॒ तेन॒ वै स स वै तेन॒ तेन॒ वै सः । \newline
20. वै स स वै वै स मृधो॒ मृधः॒ स वै वै स मृधः॑ । \newline
21. स मृधो॒ मृधः॒ स स मृधो ऽपाप॒ मृधः॒ स स मृधो ऽप॑ । \newline
22. मृधो ऽपाप॒ मृधो॒ मृधो ऽपा॑हता ह॒ताप॒ मृधो॒ मृधो ऽपा॑हत । \newline
23. अपा॑हता ह॒तापापा॑ हत॒ यद् यद॑ह॒ता पापा॑हत॒ यत् । \newline
24. अ॒ह॒त॒ यद् यद॑हता हत॒ यद् वै॑मृ॒धो वै॑मृ॒धो यद॑हता हत॒ यद् वै॑मृ॒धः । \newline
25. यद् वै॑मृ॒धो वै॑मृ॒धो यद् यद् वै॑मृ॒धः पू॒र्णमा॑से पू॒र्णमा॑से वैमृ॒धो यद् यद् वै॑मृ॒धः पू॒र्णमा॑से । \newline
26. वै॒मृ॒धः पू॒र्णमा॑से पू॒र्णमा॑से वैमृ॒धो वै॑मृ॒धः पू॒र्णमा॑से ऽनुनिर्वा॒प्यो॑ ऽनुनिर्वा॒प्यः॑ पू॒र्णमा॑से वैमृ॒धो वै॑मृ॒धः पू॒र्णमा॑से ऽनुनिर्वा॒प्यः॑ । \newline
27. पू॒र्णमा॑से ऽनुनिर्वा॒प्यो॑ ऽनुनिर्वा॒प्यः॑ पू॒र्णमा॑से पू॒र्णमा॑से ऽनुनिर्वा॒प्यो॑ भव॑ति॒ भव॑ त्यनुनिर्वा॒प्यः॑ पू॒र्णमा॑से पू॒र्णमा॑से ऽनुनिर्वा॒प्यो॑ भव॑ति । \newline
28. पू॒र्णमा॑स॒ इति॑ पू॒र्ण - मा॒से॒ । \newline
29. अ॒नु॒नि॒र्वा॒प्यो॑ भव॑ति॒ भव॑ त्यनुनिर्वा॒प्यो॑ ऽनुनिर्वा॒प्यो॑ भव॑ति॒ मृधो॒ मृधो॒ भव॑ त्यनुनिर्वा॒प्यो॑ ऽनुनिर्वा॒प्यो॑ भव॑ति॒ मृधः॑ । \newline
30. अ॒नु॒नि॒र्वा॒प्य॑ इत्य॑नु - नि॒र्वा॒प्यः॑ । \newline
31. भव॑ति॒ मृधो॒ मृधो॒ भव॑ति॒ भव॑ति॒ मृध॑ ए॒वैव मृधो॒ भव॑ति॒ भव॑ति॒ मृध॑ ए॒व । \newline
32. मृध॑ ए॒वैव मृधो॒ मृध॑ ए॒व तेन॒ तेनै॒व मृधो॒ मृध॑ ए॒व तेन॑ । \newline
33. ए॒व तेन॒ तेनै॒वैव तेन॒ यज॑मानो॒ यज॑मान॒ स्तेनै॒वैव तेन॒ यज॑मानः । \newline
34. तेन॒ यज॑मानो॒ यज॑मान॒ स्तेन॒ तेन॒ यज॑मा॒नो ऽपाप॒ यज॑मान॒ स्तेन॒ तेन॒ यज॑मा॒नो ऽप॑ । \newline
35. यज॑मा॒नो ऽपाप॒ यज॑मानो॒ यज॑मा॒नो ऽप॑ हते ह॒ते ऽप॒ यज॑मानो॒ यज॑मा॒नो ऽप॑ हते । \newline
36. अप॑ हते ह॒ते ऽपाप॑ हत॒ इन्द्र॒ इन्द्रो॑ ह॒ते ऽपाप॑ हत॒ इन्द्रः॑ । \newline
37. ह॒त॒ इन्द्र॒ इन्द्रो॑ हते हत॒ इन्द्रो॑ वृ॒त्रं ॅवृ॒त्र मिन्द्रो॑ हते हत॒ इन्द्रो॑ वृ॒त्रम् । \newline
38. इन्द्रो॑ वृ॒त्रं ॅवृ॒त्र मिन्द्र॒ इन्द्रो॑ वृ॒त्रꣳ ह॒त्वा ह॒त्वा वृ॒त्र मिन्द्र॒ इन्द्रो॑ वृ॒त्रꣳ ह॒त्वा । \newline
39. वृ॒त्रꣳ ह॒त्वा ह॒त्वा वृ॒त्रं ॅवृ॒त्रꣳ ह॒त्वा दे॒वता॑भिर् दे॒वता॑भिर्. ह॒त्वा वृ॒त्रं ॅवृ॒त्रꣳ ह॒त्वा दे॒वता॑भिः । \newline
40. ह॒त्वा दे॒वता॑भिर् दे॒वता॑भिर्. ह॒त्वा ह॒त्वा दे॒वता॑भिश्च च दे॒वता॑भिर्. ह॒त्वा ह॒त्वा दे॒वता॑भिश्च । \newline
41. दे॒वता॑भिश्च च दे॒वता॑भिर् दे॒वता॑भिश्चे न्द्रि॒येणे᳚ न्द्रि॒येण॑ च दे॒वता॑भिर् दे॒वता॑भिश्चे न्द्रि॒येण॑ । \newline
42. चे॒ न्द्रि॒येणे᳚ न्द्रि॒येण॑ च चे न्द्रि॒येण॑ च चे न्द्रि॒येण॑ च चे न्द्रि॒येण॑ च । \newline
43. इ॒न्द्रि॒येण॑ च चे न्द्रि॒येणे᳚ न्द्रि॒येण॑ च॒ वि वि चे᳚ न्द्रि॒येणे᳚ न्द्रि॒येण॑ च॒ वि । \newline
44. च॒ वि वि च॑ च॒ व्या᳚र्द्ध्यता र्द्ध्यत॒ वि च॑ च॒ व्या᳚र्द्ध्यत । \newline
45. व्या᳚र्द्ध्यता र्द्ध्यत॒ वि व्या᳚र्द्ध्यत॒ स स आ᳚र्द्ध्यत॒ वि व्या᳚र्द्ध्यत॒ सः । \newline
46. आ॒र्द्ध्य॒त॒ स स आ᳚र्द्ध्यता र्द्ध्यत॒ स ए॒त मे॒तꣳ स आ᳚र्द्ध्यता र्द्ध्यत॒ स ए॒तम् । \newline
47. स ए॒त मे॒तꣳ स स ए॒त मा᳚ग्ने॒य मा᳚ग्ने॒य मे॒तꣳ स स ए॒त मा᳚ग्ने॒यम् । \newline
48. ए॒त मा᳚ग्ने॒य मा᳚ग्ने॒य मे॒त मे॒त मा᳚ग्ने॒य म॒ष्टाक॑पाल म॒ष्टाक॑पाल माग्ने॒य मे॒त मे॒त मा᳚ग्ने॒य म॒ष्टाक॑पालम् । \newline
49. आ॒ग्ने॒य म॒ष्टाक॑पाल म॒ष्टाक॑पाल माग्ने॒य मा᳚ग्ने॒य म॒ष्टाक॑पाल ममावा॒स्या॑या ममावा॒स्या॑या म॒ष्टाक॑पाल माग्ने॒य मा᳚ग्ने॒य म॒ष्टाक॑पाल ममावा॒स्या॑याम् । \newline
50. अ॒ष्टाक॑पाल ममावा॒स्या॑या ममावा॒स्या॑या म॒ष्टाक॑पाल म॒ष्टाक॑पाल ममावा॒स्या॑या मपश्यदपश्य दमावा॒स्या॑या म॒ष्टाक॑पाल म॒ष्टाक॑पाल ममावा॒स्या॑या मपश्यत् । \newline
51. अ॒ष्टाक॑पाल॒मित्य॒ष्टा - क॒पा॒ल॒म् । \newline
52. अ॒मा॒वा॒स्या॑या मपश्य दपश्य दमावा॒स्या॑या ममावा॒स्या॑या मपश्यदै॒न्द्र मै॒न्द्र म॑पश्य दमावा॒स्या॑या ममावा॒स्या॑या मपश्य दै॒न्द्रम् । \newline
53. अ॒मा॒वा॒स्या॑या॒मित्य॑मा - वा॒स्या॑याम् । \newline
54. अ॒प॒श्य॒ दै॒न्द्र मै॒न्द्र म॑पश्य दपश्य दै॒न्द्रम् दधि॒ दध्यै॒न्द्र म॑पश्य दपश्य दै॒न्द्रम् दधि॑ । \newline
55. ऐ॒न्द्रम् दधि॒ दध्यै॒न्द्र मै॒न्द्रम् दधि॒ तम् तम् दध्यै॒न्द्र मै॒न्द्रम् दधि॒ तम् । \newline
56. दधि॒ तम् तम् दधि॒ दधि॒ तम् निर् णिष् टम् दधि॒ दधि॒ तम् निः । \newline
\pagebreak
\markright{ TS 2.5.3.2  \hfill https://www.vedavms.in \hfill}
\addcontentsline{toc}{section}{ TS 2.5.3.2 }
\section*{ TS 2.5.3.2 }

\textbf{TS 2.5.3.2 } \newline
\textbf{Samhita Paata} \newline

तं निर॑वप॒त् तेन॒ वै स दे॒वता᳚श्चेन्द्रि॒यं चावा॑रुन्ध॒यदा᳚ग्ने॒यो᳚ ऽष्टाक॑पालो ऽमावा॒स्या॑यां॒ भव॑त्यै॒न्द्रं दधि॑ दे॒वता᳚श्चै॒व तेने᳚न्द्रि॒यं च॒ यज॑मा॒नोऽव॑ रुन्ध॒ इन्द्र॑स्य वृ॒त्रं ज॒घ्नुष॑ इन्द्रि॒यं ॅवी॒र्यं॑ पृथि॒वीमनु॒ व्या᳚र्च्छ॒त् तदोष॑धयो वी॒रुधो॑ऽभव॒न्थ्स प्र॒जाप॑ति॒मुपा॑धावद्-वृ॒त्रं मे॑ ज॒घ्नुष॑ इन्द्रि॒यं ॅवी॒र्यं॑ - [  ] \newline

\textbf{Pada Paata} \newline

तम् । निरिति॑ । अ॒व॒प॒त् । तेन॑ । वै । सः । दे॒वताः᳚ । च॒ । इ॒न्द्रि॒यम् । च॒ । अवेति॑ । अ॒रु॒न्ध॒ । यत् । आ॒ग्ने॒यः । अ॒ष्टाक॑पाल॒ इत्य॒ष्टा - क॒पा॒लः॒ । अ॒मा॒वा॒स्या॑या॒मित्य॑मा - वा॒स्या॑याम् । भव॑ति । ऐ॒न्द्रम् । दधि॑ । दे॒वताः᳚ । च॒ । ए॒व । तेन॑ । इ॒न्द्रि॒यम् । च॒ । यज॑मानः । अवेति॑ । रु॒न्धे॒ । इन्द्र॑स्य । वृ॒त्रम् । ज॒घ्नुषः॑ । इ॒न्द्रि॒यम् । वी॒र्य᳚म् । पृ॒थि॒वीम् । अनु॑ । वीति॑ । आ॒र्च्छ॒त् । तत् । ओष॑धयः । वी॒रुधः॑ । अ॒भ॒व॒न्न् । सः । प्र॒जाप॑ति॒मिति॑ प्र॒जा - प॒ति॒म् । उपेति॑ । अ॒धा॒व॒त् । वृ॒त्रं । मे॒ । ज॒घ्नुषः॑ । इ॒न्द्रि॒यम् । वी॒र्य᳚म् ।  \newline


\textbf{Krama Paata} \newline

तम् निः । निर॑वपत् । अ॒व॒प॒त् तेन॑ । तेन॒ वै । वै सः । स दे॒वताः᳚ । दे॒वता᳚श्च । चे॒न्द्रि॒यम् । इ॒न्द्रि॒यम् च॑ । चाव॑ । अवा॑रुन्ध । अ॒रु॒न्ध॒ यत् । यदा᳚ग्ने॒यः । आ॒ग्ने॒यो᳚ ऽष्टाक॑पालः । अ॒ष्टाक॑पालो ऽमावा॒स्या॑याम् । अ॒ष्टाक॑पाल॒ इत्य॒ष्टा - क॒पा॒लः॒ । अ॒मा॒वा॒स्या॑या॒म् भव॑ति । अ॒मा॒वा॒स्या॑या॒मित्य॑मा - वा॒स्या॑याम् । भव॑त्यै॒न्द्रम् । ऐ॒न्द्रम् दधि॑ । दधि॑ दे॒वताः᳚ । दे॒वता᳚श्च । चै॒व । ए॒व तेन॑ । तेने᳚न्द्रि॒यम् । इ॒न्द्रि॒यम् च॑ । च॒ यज॑मानः । यज॑मा॒नो ऽव॑ । अव॑ रुन्धे । रु॒न्ध॒ इन्द्र॑स्य । इन्द्र॑स्य वृ॒त्रम् । वृ॒त्रम् ज॒घ्नुषः॑ । ज॒घ्नुष॑ इन्द्रि॒यम् । इ॒न्द्रि॒यम् ॅवी॒र्य᳚म् । वी॒र्य॑म् पृथि॒वीम् । पृ॒थि॒वीमनु॑ । अनु॒ वि । व्या᳚र्च्छत् । आ॒र्च्छ॒त् तत् । तदोष॑धयः । ओष॑धयो वी॒रुधः॑ । वी॒रुधो॑ ऽभवन्न् । अ॒भ॒व॒न्थ् सः । स प्र॒जाप॑तिम् । प्र॒जाप॑ति॒मुप॑ । प्र॒जाप॑ति॒मिति॑ प्र॒जा - प॒ति॒म् । उपा॑धावत् । अ॒धा॒व॒द् वृ॒त्रम् । वृ॒त्रम् मे᳚ । मे॒ ज॒घ्नुषः॑ । ज॒घ्नुष॑ इन्द्रि॒यम् । इ॒न्द्रि॒यम् ॅवी॒र्य᳚म् । वी॒र्य॑म् पृथि॒वीम् \newline

\textbf{Jatai Paata} \newline

1. तम् निर् णिष् टम् तम् निः । \newline
2. निर॑वप दवप॒न् निर् णिर॑वपत् । \newline
3. अ॒व॒प॒त् तेन॒ तेना॑वप दवप॒त् तेन॑ । \newline
4. तेन॒ वै वै तेन॒ तेन॒ वै । \newline
5. वै स स वै वै सः । \newline
6. स दे॒वता॑ दे॒वताः॒ स स दे॒वताः᳚ । \newline
7. दे॒वता᳚श्च च दे॒वता॑ दे॒वता᳚श्च । \newline
8. चे॒ न्द्रि॒य मि॑न्द्रि॒यम् च॑ चे न्द्रि॒यम् । \newline
9. इ॒न्द्रि॒यम् च॑ चे न्द्रि॒य मि॑न्द्रि॒यम् च॑ । \newline
10. चावाव॑ च॒ चाव॑ । \newline
11. अवा॑रुन्धा रु॒न्धा वावा॑रुन्ध । \newline
12. अ॒रु॒न्ध॒ यद् यद॑रुन्धा रुन्ध॒ यत् । \newline
13. यदा᳚ग्ने॒य आ᳚ग्ने॒यो यद् यदा᳚ग्ने॒यः । \newline
14. आ॒ग्ने॒यो᳚ ऽष्टाक॑पालो॒ ऽष्टाक॑पाल आग्ने॒य आ᳚ग्ने॒यो᳚ ऽष्टाक॑पालः । \newline
15. अ॒ष्टाक॑पालो ऽमावा॒स्या॑या ममावा॒स्या॑या म॒ष्टाक॑पालो॒ ऽष्टाक॑पालो ऽमावा॒स्या॑याम् । \newline
16. अ॒ष्टाक॑पाल॒ इत्य॒ष्टा - क॒पा॒लः॒ । \newline
17. अ॒मा॒वा॒स्या॑या॒म् भव॑ति॒ भव॑ त्यमावा॒स्या॑या ममावा॒स्या॑या॒म् भव॑ति । \newline
18. अ॒मा॒वा॒स्या॑या॒मित्य॑मा - वा॒स्या॑याम् । \newline
19. भव॑ त्यै॒न्द्र मै॒न्द्रम् भव॑ति॒ भव॑ त्यै॒न्द्रम् । \newline
20. ऐ॒न्द्रम् दधि॒ दध्यै॒न्द्र मै॒न्द्रम् दधि॑ । \newline
21. दधि॑ दे॒वता॑ दे॒वता॒ दधि॒ दधि॑ दे॒वताः᳚ । \newline
22. दे॒वता᳚श्च च दे॒वता॑ दे॒वता᳚श्च । \newline
23. चै॒वैव च॑ चै॒व । \newline
24. ए॒व तेन॒ तेनै॒वैव तेन॑ । \newline
25. तेने᳚ न्द्रि॒य मि॑न्द्रि॒यम् तेन॒ तेने᳚ न्द्रि॒यम् । \newline
26. इ॒न्द्रि॒यम् च॑ चे न्द्रि॒य मि॑न्द्रि॒यम् च॑ । \newline
27. च॒ यज॑मानो॒ यज॑मानश्च च॒ यज॑मानः । \newline
28. यज॑मा॒नो ऽवाव॒ यज॑मानो॒ यज॑मा॒नो ऽव॑ । \newline
29. अव॑ रुन्धे रु॒न्धे ऽवाव॑ रुन्धे । \newline
30. रु॒न्ध॒ इन्द्र॒स्ये न्द्र॑स्य रुन्धे रुन्ध॒ इन्द्र॑स्य । \newline
31. इन्द्र॑स्य वृ॒त्रं ॅवृ॒त्र मिन्द्र॒स्ये न्द्र॑स्य वृ॒त्रम् । \newline
32. वृ॒त्रम् ज॒घ्नुषो॑ ज॒घ्नुषो॑ वृ॒त्रं ॅवृ॒त्रम् ज॒घ्नुषः॑ । \newline
33. ज॒घ्नुष॑ इन्द्रि॒य मि॑न्द्रि॒यम् ज॒घ्नुषो॑ ज॒घ्नुष॑ इन्द्रि॒यम् । \newline
34. इ॒न्द्रि॒यं ॅवी॒र्यं॑ ॅवी॒र्य॑ मिन्द्रि॒य मि॑न्द्रि॒यं ॅवी॒र्य᳚म् । \newline
35. वी॒र्य॑म् पृथि॒वीम् पृ॑थि॒वीं ॅवी॒र्यं॑ ॅवी॒र्य॑म् पृथि॒वीम् । \newline
36. पृ॒थि॒वी मन्वनु॑ पृथि॒वीम् पृ॑थि॒वी मनु॑ । \newline
37. अनु॒ वि व्यन्वनु॒ वि । \newline
38. व्या᳚र्च्छ दार्च्छ॒द् वि व्या᳚र्च्छत् । \newline
39. आ॒र्च्छ॒त् तत् तदा᳚र्च्छ दार्च्छ॒त् तत् । \newline
40. तदोष॑धय॒ ओष॑धय॒ स्तत् तदोष॑धयः । \newline
41. ओष॑धयो वी॒रुधो॑ वी॒रुध॒ ओष॑धय॒ ओष॑धयो वी॒रुधः॑ । \newline
42. वी॒रुधो॑ ऽभवन् नभवन्. वी॒रुधो॑ वी॒रुधो॑ ऽभवन्न् । \newline
43. अ॒भ॒व॒न् थ्स सो॑ ऽभवन् नभव॒न् थ्सः । \newline
44. स प्र॒जाप॑तिम् प्र॒जाप॑तिꣳ॒॒ स स प्र॒जाप॑तिम् । \newline
45. प्र॒जाप॑ति॒ मुपोप॑ प्र॒जाप॑तिम् प्र॒जाप॑ति॒ मुप॑ । \newline
46. प्र॒जाप॑ति॒मिति॑ प्र॒जा - प॒ति॒म् । \newline
47. उपा॑धाव दधाव॒ दुपोपा॑धावत् । \newline
48. अ॒धा॒व॒द् वृ॒त्रं वृ॒त्र म॑धाव दधावद् वृ॒त्रं । \newline
49. वृ॒त्रं मे॑ मे वृ॒त्रं वृ॒त्रं मे᳚ । \newline
50. मे॒ ज॒घ्नुषो॑ ज॒घ्नुषो॑ मे मे ज॒घ्नुषः॑ । \newline
51. ज॒घ्नुष॑ इन्द्रि॒य मि॑न्द्रि॒यम् ज॒घ्नुषो॑ ज॒घ्नुष॑ इन्द्रि॒यम् । \newline
52. इ॒न्द्रि॒यं ॅवी॒र्यं॑ ॅवी॒र्य॑ मिन्द्रि॒य मि॑न्द्रि॒यं ॅवी॒र्य᳚म् । \newline
53. वी॒र्य॑म् पृथि॒वीम् पृ॑थि॒वीं ॅवी॒र्यं॑ ॅवी॒र्य॑म् पृथि॒वीम् । \newline

\textbf{Ghana Paata } \newline

1. तम् निर् णिष् टम् तम् निर॑वप दवप॒न् निष् टम् तम् निर॑वपत् । \newline
2. निर॑वप दवप॒न् निर् णिर॑वप॒त् तेन॒ तेना॑वप॒न् निर् णिर॑वप॒त् तेन॑ । \newline
3. अ॒व॒प॒त् तेन॒ तेना॑वप दवप॒त् तेन॒ वै वै तेना॑वप दवप॒त् तेन॒ वै । \newline
4. तेन॒ वै वै तेन॒ तेन॒ वै स स वै तेन॒ तेन॒ वै सः । \newline
5. वै स स वै वै स दे॒वता॑ दे॒वताः॒ स वै वै स दे॒वताः᳚ । \newline
6. स दे॒वता॑ दे॒वताः॒ स स दे॒वता᳚श्च च दे॒वताः॒ स स दे॒वता᳚श्च । \newline
7. दे॒वता᳚श्च च दे॒वता॑ दे॒वता᳚श्चे न्द्रि॒य मि॑न्द्रि॒यम् च॑ दे॒वता॑ दे॒वता᳚श्चे न्द्रि॒यम् । \newline
8. चे॒ न्द्रि॒य मि॑न्द्रि॒यम् च॑ चे न्द्रि॒यम् च॑ चे न्द्रि॒यम् च॑ चे न्द्रि॒यम् च॑ । \newline
9. इ॒न्द्रि॒यम् च॑ चे न्द्रि॒य मि॑न्द्रि॒यम् चावाव॑ चे न्द्रि॒य मि॑न्द्रि॒यम् चाव॑ । \newline
10. चावाव॑ च॒ चावा॑रुन्धा रु॒न्धाव॑ च॒ चावा॑रुन्ध । \newline
11. अवा॑रुन्धा रु॒न्धा वावा॑रुन्ध॒ यद् यद॑रु॒न्धा वावा॑रुन्ध॒ यत् । \newline
12. अ॒रु॒न्ध॒ यद् यद॑रुन्धारुन्ध॒ यदा᳚ग्ने॒य आ᳚ग्ने॒यो यद॑रुन्धारुन्ध॒ यदा᳚ग्ने॒यः । \newline
13. यदा᳚ग्ने॒य आ᳚ग्ने॒यो यद् यदा᳚ग्ने॒यो᳚ ऽष्टाक॑पालो॒ ऽष्टाक॑पाल आग्ने॒यो यद् यदा᳚ग्ने॒यो᳚ ऽष्टाक॑पालः । \newline
14. आ॒ग्ने॒यो᳚ ऽष्टाक॑पालो॒ ऽष्टाक॑पाल आग्ने॒य आ᳚ग्ने॒यो᳚ ऽष्टाक॑पालो ऽमावा॒स्या॑या ममावा॒स्या॑या म॒ष्टाक॑पाल आग्ने॒य आ᳚ग्ने॒यो᳚ ऽष्टाक॑पालो ऽमावा॒स्या॑याम् । \newline
15. अ॒ष्टाक॑पालो ऽमावा॒स्या॑या ममावा॒स्या॑या म॒ष्टाक॑पालो॒ ऽष्टाक॑पालो ऽमावा॒स्या॑या॒म् भव॑ति॒ भव॑ त्यमावा॒स्या॑या म॒ष्टाक॑पालो॒ ऽष्टाक॑पालो ऽमावा॒स्या॑या॒म् भव॑ति । \newline
16. अ॒ष्टाक॑पाल॒ इत्य॒ष्टा - क॒पा॒लः॒ । \newline
17. अ॒मा॒वा॒स्या॑या॒म् भव॑ति॒ भव॑ त्यमावा॒स्या॑या ममावा॒स्या॑या॒म् भव॑त्यै॒न्द्र मै॒न्द्रम् भव॑ त्यमावा॒स्या॑या ममावा॒स्या॑या॒म् भव॑त्यै॒न्द्रम् । \newline
18. अ॒मा॒वा॒स्या॑या॒मित्य॑मा - वा॒स्या॑याम् । \newline
19. भव॑त्यै॒न्द्र मै॒न्द्रम् भव॑ति॒ भव॑ त्यै॒न्द्रम् दधि॒ दध्यै॒न्द्रम् भव॑ति॒ भव॑ त्यै॒न्द्रम् दधि॑ । \newline
20. ऐ॒न्द्रम् दधि॒ दध्यै॒न्द्र मै॒न्द्रम् दधि॑ दे॒वता॑ दे॒वता॒ दध्यै॒न्द्र मै॒न्द्रम् दधि॑ दे॒वताः᳚ । \newline
21. दधि॑ दे॒वता॑ दे॒वता॒ दधि॒ दधि॑ दे॒वता᳚श्च च दे॒वता॒ दधि॒ दधि॑ दे॒वता᳚श्च । \newline
22. दे॒वता᳚श्च च दे॒वता॑ दे॒वता᳚ श्चै॒वैव च॑ दे॒वता॑ दे॒वता᳚श्चै॒व । \newline
23. चै॒वैव च॑ चै॒व तेन॒ तेनै॒व च॑ चै॒व तेन॑ । \newline
24. ए॒व तेन॒ तेनै॒वैव तेने᳚ न्द्रि॒य मि॑न्द्रि॒यम् तेनै॒वैव तेने᳚ न्द्रि॒यम् । \newline
25. तेने᳚ न्द्रि॒य मि॑न्द्रि॒यम् तेन॒ तेने᳚ न्द्रि॒यम् च॑ चे न्द्रि॒यम् तेन॒ तेने᳚ न्द्रि॒यम् च॑ । \newline
26. इ॒न्द्रि॒यम् च॑ चे न्द्रि॒य मि॑न्द्रि॒यम् च॒ यज॑मानो॒ यज॑मानश्चे न्द्रि॒य मि॑न्द्रि॒यम् च॒ यज॑मानः । \newline
27. च॒ यज॑मानो॒ यज॑मानश्च च॒ यज॑मा॒नो ऽवाव॒ यज॑मानश्च च॒ यज॑मा॒नो ऽव॑ । \newline
28. यज॑मा॒नो ऽवाव॒ यज॑मानो॒ यज॑मा॒नो ऽव॑ रुन्धे रु॒न्धे ऽव॒ यज॑मानो॒ यज॑मा॒नो ऽव॑ रुन्धे । \newline
29. अव॑ रुन्धे रु॒न्धे ऽवाव॑ रुन्ध॒ इन्द्र॒स्ये न्द्र॑स्य रु॒न्धे ऽवाव॑ रुन्ध॒ इन्द्र॑स्य । \newline
30. रु॒न्ध॒ इन्द्र॒स्ये न्द्र॑स्य रुन्धे रुन्ध॒ इन्द्र॑स्य वृ॒त्रं ॅवृ॒त्र मिन्द्र॑स्य रुन्धे रुन्ध॒ इन्द्र॑स्य वृ॒त्रम् । \newline
31. इन्द्र॑स्य वृ॒त्रं ॅवृ॒त्र मिन्द्र॒स्ये न्द्र॑स्य वृ॒त्रम् ज॒घ्नुषो॑ ज॒घ्नुषो॑ वृ॒त्र मिन्द्र॒स्ये न्द्र॑स्य वृ॒त्रम् ज॒घ्नुषः॑ । \newline
32. वृ॒त्रम् ज॒घ्नुषो॑ ज॒घ्नुषो॑ वृ॒त्रं ॅवृ॒त्रम् ज॒घ्नुष॑ इन्द्रि॒य मि॑न्द्रि॒यम् ज॒घ्नुषो॑ वृ॒त्रं ॅवृ॒त्रम् ज॒घ्नुष॑ इन्द्रि॒यम् । \newline
33. ज॒घ्नुष॑ इन्द्रि॒य मि॑न्द्रि॒यम् ज॒घ्नुषो॑ ज॒घ्नुष॑ इन्द्रि॒यं ॅवी॒र्यं॑ ॅवी॒र्य॑ मिन्द्रि॒यम् ज॒घ्नुषो॑ ज॒घ्नुष॑ इन्द्रि॒यं ॅवी॒र्य᳚म् । \newline
34. इ॒न्द्रि॒यं ॅवी॒र्यं॑ ॅवी॒र्य॑ मिन्द्रि॒य मि॑न्द्रि॒यं ॅवी॒र्य॑म् पृथि॒वीम् पृ॑थि॒वीं ॅवी॒र्य॑ मिन्द्रि॒य मि॑न्द्रि॒यं ॅवी॒र्य॑म् पृथि॒वीम् । \newline
35. वी॒र्य॑म् पृथि॒वीम् पृ॑थि॒वीं ॅवी॒र्यं॑ ॅवी॒र्य॑म् पृथि॒वी मन्वनु॑ पृथि॒वीं ॅवी॒र्यं॑ ॅवी॒र्य॑म् पृथि॒वी मनु॑ । \newline
36. पृ॒थि॒वी मन्वनु॑ पृथि॒वीम् पृ॑थि॒वी मनु॒ वि व्यनु॑ पृथि॒वीम् पृ॑थि॒वी मनु॒ वि । \newline
37. अनु॒ वि व्यन्वनु॒ व्या᳚र्च्छ दार्च्छ॒द् व्यन्वनु॒ व्या᳚र्च्छत् । \newline
38. व्या᳚र्च्छ दार्च्छ॒द् वि व्या᳚र्च्छ॒त् तत् तदा᳚र्च्छ॒द् वि व्या᳚र्च्छ॒त् तत् । \newline
39. आ॒र्च्छ॒त् तत् तदा᳚र्च्छ दार्च्छ॒त् तदोष॑धय॒ ओष॑धय॒ स्तदा᳚र्च्छ दार्च्छ॒त् तदोष॑धयः । \newline
40. तदोष॑धय॒ ओष॑धय॒ स्तत् तदोष॑धयो वी॒रुधो॑ वी॒रुध॒ ओष॑धय॒ स्तत् तदोष॑धयो वी॒रुधः॑ । \newline
41. ओष॑धयो वी॒रुधो॑ वी॒रुध॒ ओष॑धय॒ ओष॑धयो वी॒रुधो॑ ऽभवन् नभवन्. वी॒रुध॒ ओष॑धय॒ ओष॑धयो वी॒रुधो॑ ऽभवन्न् । \newline
42. वी॒रुधो॑ ऽभवन् नभवन्. वी॒रुधो॑ वी॒रुधो॑ ऽभव॒न् थ्स सो॑ ऽभवन्. वी॒रुधो॑ वी॒रुधो॑ ऽभव॒न् थ्सः । \newline
43. अ॒भ॒व॒न् थ्स सो॑ ऽभवन् नभव॒न् थ्स प्र॒जाप॑तिम् प्र॒जाप॑तिꣳ॒॒ सो॑ ऽभवन् नभव॒न् थ्स प्र॒जाप॑तिम् । \newline
44. स प्र॒जाप॑तिम् प्र॒जाप॑तिꣳ॒॒ स स प्र॒जाप॑ति॒ मुपोप॑ प्र॒जाप॑तिꣳ॒॒ स स प्र॒जाप॑ति॒ मुप॑ । \newline
45. प्र॒जाप॑ति॒ मुपोप॑ प्र॒जाप॑तिम् प्र॒जाप॑ति॒ मुपा॑धाव दधाव॒दुप॑ प्र॒जाप॑तिम् प्र॒जाप॑ति॒ मुपा॑धावत् । \newline
46. प्र॒जाप॑ति॒मिति॑ प्र॒जा - प॒ति॒म् । \newline
47. उपा॑धाव दधाव॒ दुपोपा॑ धावद् वृ॒त्रं वृ॒त्र म॑धाव॒ दुपोपा॑धावद् वृ॒त्रं । \newline
48. अ॒धा॒व॒द् वृ॒त्रं वृ॒त्र म॑धाव दधावद् वृ॒त्रं मे॑ मे वृ॒त्र म॑धाव दधावद् वृ॒त्रं मे᳚ । \newline
49. वृ॒त्रं मे॑ मे वृ॒त्रं वृ॒त्रं मे॑ ज॒घ्नुषो॑ ज॒घ्नुषो॑ मे वृ॒त्रं वृ॒त्रं मे॑ ज॒घ्नुषः॑ । \newline
50. मे॒ ज॒घ्नुषो॑ ज॒घ्नुषो॑ मे मे ज॒घ्नुष॑ इन्द्रि॒य मि॑न्द्रि॒यम् ज॒घ्नुषो॑ मे मे ज॒घ्नुष॑ इन्द्रि॒यम् । \newline
51. ज॒घ्नुष॑ इन्द्रि॒य मि॑न्द्रि॒यम् ज॒घ्नुषो॑ ज॒घ्नुष॑ इन्द्रि॒यं ॅवी॒र्यं॑ ॅवी॒र्य॑ मिन्द्रि॒यम् ज॒घ्नुषो॑ ज॒घ्नुष॑ इन्द्रि॒यं ॅवी॒र्य᳚म् । \newline
52. इ॒न्द्रि॒यं ॅवी॒र्यं॑ ॅवी॒र्य॑ मिन्द्रि॒य मि॑न्द्रि॒यं ॅवी॒र्य॑म् पृथि॒वीम् पृ॑थि॒वीं ॅवी॒र्य॑ मिन्द्रि॒य मि॑न्द्रि॒यं ॅवी॒र्य॑म् पृथि॒वीम् । \newline
53. वी॒र्य॑म् पृथि॒वीम् पृ॑थि॒वीं ॅवी॒र्यं॑ ॅवी॒र्य॑म् पृथि॒वी मन्वनु॑ पृथि॒वीं ॅवी॒र्यं॑ ॅवी॒र्य॑म् पृथि॒वी मनु॑ । \newline
\pagebreak
\markright{ TS 2.5.3.3  \hfill https://www.vedavms.in \hfill}
\addcontentsline{toc}{section}{ TS 2.5.3.3 }
\section*{ TS 2.5.3.3 }

\textbf{TS 2.5.3.3 } \newline
\textbf{Samhita Paata} \newline

पृथि॒वीमनु॒ व्या॑र॒त् तदोष॑धयो वी॒रुधो॑ऽभूव॒न्निति॒ स प्र॒जाप॑तिः प॒शून॑ब्रवीदे॒तद॑स्मै॒ सं न॑य॒तेति॒ तत् प॒शव॒ ओष॑धी॒भ्यो ऽध्या॒त्मन्थ् सम॑नय॒न् तत् प्रत्य॑दुह॒न॒. यथ् स॒मन॑य॒न् तथ् सां᳚ ना॒य्यस्य॑ सांनाय्य॒त्वं ॅयत् प्र॒त्यदु॑ह॒न् तत् प्र॑ति॒धुषः॑ प्रतिधु॒क्त्वꣳ सम॑नैषुः॒ प्रत्य॑धुक्ष॒न् न तु मयि॑ श्रयत॒ इत्य॑ब्रवीदे॒तद॑स्मै - [  ] \newline

\textbf{Pada Paata} \newline

पृ॒थि॒वीम् । अनु॑ । वीति॑ । आ॒र॒त् । तत् । ओष॑धयः । वी॒रुधः॑ । अ॒भू॒व॒न्न् । इति॑ । सः । प्र॒जाप॑ति॒रिति॑ प्र॒जा - प॒तिः॒ । प॒शून् । अ॒ब्र॒वी॒त् । ए॒तत् । अ॒स्मै॒ । समिति॑ । न॒य॒त॒ । इति॑ । तत् । प॒शवः॑ । ओष॑धीभ्य॒ इत्योष॑धि - भ्यः॒ । अधीति॑ । आ॒त्मन्न् । समिति॑ । अ॒न॒य॒न्न् । तत् । प्रतीति॑ । अ॒दु॒ह॒न्न् । यत् । स॒मन॑य॒न्निति॑ सं - अन॑यन्न् । तत् । सा॒नां॒य्यस्येति॑ सां - ना॒य्यस्य॑ । सा॒नां॒य्य॒त्वमिति॑ सांनाय्य - त्वम् । यत् । प्र॒त्यदु॑ह॒न्निति॑ प्रति - अदु॑हन्न् । तत् । प्र॒ति॒धुष॒ इति॑ प्रति - धुषः॑ । प्र॒ति॒धु॒क्त्वमिति॑ प्रतिधुक् - त्वम् । समिति॑ । अ॒नै॒षुः॒ । प्रतीति॑ । अ॒धु॒क्ष॒न्न् । न । तु । मयि॑ । श्र॒य॒ते॒ । इति॑ । अ॒ब्र॒वी॒त् । ए॒तत् । अ॒स्मै॒ ।  \newline


\textbf{Krama Paata} \newline

पृ॒थि॒वीमनु॑ । अनु॒ वि । व्या॑रत् । आ॒र॒त् तत् । तदोष॑धयः । ओष॑धयो वी॒रुधः॑ । वी॒रुधो॑ ऽभूवन्न् । अ॒भू॒व॒न्निति॑ । इति॒ सः । स प्र॒जाप॑तिः । प्र॒जाप॑तिः प॒शून् । प्र॒जाप॑ति॒रिति॑ प्र॒जा - प॒तिः॒ । प॒शून॑ब्रवीत् । अ॒ब्र॒वी॒दे॒तत् । ए॒तद॑स्मै । अ॒स्मै॒ सम् । सम् न॑यत । न॒य॒तेति॑ । इति॒ तत् । तत् प॒शवः॑ । प॒शव॒ ओष॑धीभ्यः । ओष॑धी॒भ्यो ऽधि॑ । ओष॑धीभ्य॒ इत्योष॑धि - भ्यः॒ । अद्ध्या॒त्मन्न् । आ॒त्मन्थ् सम् । सम॑नयन्न् । अ॒न॒य॒न् तत् । तत् प्रति॑ । प्रत्य॑दुहन्न् । अ॒दु॒ह॒न्॒. यत् । यथ् स॒मन॑यन्न् । स॒मन॑य॒न् तत् । स॒मन॑य॒न्निति॑ सम् - अन॑यन्न् । तथ् सा᳚न्ना॒य्यस्य॑ । सा॒न्ना॒य्यस्य॑ सान्नाय्य॒त्वम् । सा॒न्ना॒य्यस्येति॑ साम् - ना॒य्यस्य॑ । सा॒न्ना॒य्य॒त्वम् ॅयत् । सा॒न्ना॒य्य॒त्वमिति॑ सान्नाय्य - त्वम् । यत् प्र॒त्यदु॑हन्न् । प्र॒त्यदु॑ह॒न् तत् । प्र॒त्यदु॑ह॒न्निति॑ प्रति - अदु॑हन्न् । तत् प्र॑ति॒धुषः॑ । प्र॒ति॒धुषः॑ प्रतिधु॒क्त्वम् । प्र॒ति॒धुष॒ इति॑ प्रति - धुषः॑ । प्र॒ति॒धु॒क्त्वꣳ सम् । प्र॒ति॒धु॒क्त्वमिति॑ प्रतिधुक् - त्वम् । सम॑नैषुः । अ॒नै॒षुः॒ प्रति॑ । प्रत्य॑धुक्षन्न् । अ॒धु॒क्ष॒न् न । न तु । तु मयि॑ । मयि॑ श्रयते । श्र॒य॒त॒ इति॑ । इत्य॑ब्रवीत् । अ॒ब्र॒वी॒दे॒तत् । ए॒तद॑स्मै । अ॒स्मै॒ शृ॒तम् \newline

\textbf{Jatai Paata} \newline

1. पृ॒थि॒वी मन्वनु॑ पृथि॒वीम् पृ॑थि॒वी मनु॑ । \newline
2. अनु॒ वि व्यन्वनु॒ वि । \newline
3. व्या॑रदार॒द् वि व्या॑रत् । \newline
4. आ॒र॒त् तत् तदा॑र दार॒त् तत् । \newline
5. तदोष॑धय॒ ओष॑धय॒ स्तत् तदोष॑धयः । \newline
6. ओष॑धयो वी॒रुधो॑ वी॒रुध॒ ओष॑धय॒ ओष॑धयो वी॒रुधः॑ । \newline
7. वी॒रुधो॑ ऽभूवन् नभूवन्. वी॒रुधो॑ वी॒रुधो॑ ऽभूवन्न् । \newline
8. अ॒भू॒व॒न् निती त्य॑भूवन् नभूव॒न् निति॑ । \newline
9. इति॒ स स इतीति॒ सः । \newline
10. स प्र॒जाप॑तिः प्र॒जाप॑तिः॒ स स प्र॒जाप॑तिः । \newline
11. प्र॒जाप॑तिः प॒शून् प॒शून् प्र॒जाप॑तिः प्र॒जाप॑तिः प॒शून् । \newline
12. प्र॒जाप॑ति॒रिति॑ प्र॒जा - प॒तिः॒ । \newline
13. प॒शू न॑ब्रवी दब्रवीत् प॒शून् प॒शू न॑ब्रवीत् । \newline
14. अ॒ब्र॒वी॒ दे॒त दे॒त द॑ब्रवी दब्रवी दे॒तत् । \newline
15. ए॒तद॑स्मा अस्मा ए॒त दे॒त द॑स्मै । \newline
16. अ॒स्मै॒ सꣳ स म॑स्मा अस्मै॒ सम् । \newline
17. सम् न॑यत नयत॒ सꣳ सम् न॑यत । \newline
18. न॒य॒ते तीति॑ नयत नय॒ते ति॑ । \newline
19. इति॒ तत् तदितीति॒ तत् । \newline
20. तत् प॒शवः॑ प॒शव॒ स्तत् तत् प॒शवः॑ । \newline
21. प॒शव॒ ओष॑धीभ्य॒ ओष॑धीभ्यः प॒शवः॑ प॒शव॒ ओष॑धीभ्यः । \newline
22. ओष॑धी॒भ्यो ऽध्यध्योष॑धीभ्य॒ ओष॑धी॒भ्यो ऽधि॑ । \newline
23. ओष॑धीभ्य॒ इत्योष॑धि - भ्यः॒ । \newline
24. अध्या॒त्मन् ना॒त्मन् नध्य ध्या॒त्मन्न् । \newline
25. आ॒त्मन् थ्सꣳ स मा॒त्मन् ना॒त्मन् थ्सम् । \newline
26. स म॑नयन् ननय॒न् थ्सꣳ स म॑नयन्न् । \newline
27. अ॒न॒य॒न् तत् तद॑नयन् ननय॒न् तत् । \newline
28. तत् प्रति॒ प्रति॒ तत् तत् प्रति॑ । \newline
29. प्रत्य॑दुहन् नदुह॒न् प्रति॒ प्रत्य॑दुहन्न् । \newline
30. अ॒दु॒ह॒न्॒. यद् यद॑दुहन् नदुह॒न्॒. यत् । \newline
31. यथ् स॒मन॑यन् थ्स॒मन॑य॒न्॒. यद् यथ् स॒मन॑यन्न् । \newline
32. स॒मन॑य॒न् तत् तथ् स॒मन॑यन् थ्स॒मन॑य॒न् तत् । \newline
33. स॒मन॑य॒न्निति॑ सं - अन॑यन्न् । \newline
34. तथ् सा᳚न्ना॒य्यस्य॑ सान्ना॒य्यस्य॒ तत् तथ् सा᳚न्ना॒य्यस्य॑ । \newline
35. सा॒न्ना॒य्यस्य॑ सान्नाय्य॒त्वꣳ सा᳚न्नाय्य॒त्वꣳ सा᳚न्ना॒य्यस्य॑ सान्ना॒य्यस्य॑ सान्नाय्य॒त्वम् । \newline
36. सा॒न्ना॒य्यस्येति॑ सां - ना॒य्यस्य॑ । \newline
37. सा॒न्ना॒य्य॒त्वं ॅयद् यथ् सा᳚न्नाय्य॒त्वꣳ सा᳚न्नाय्य॒त्वं ॅयत् । \newline
38. सा॒न्ना॒य्य॒त्वमिति॑ सान्नाय्य - त्वम् । \newline
39. यत् प्र॒त्यदु॑हन् प्र॒त्यदु॑ह॒न्॒. यद् यत् प्र॒त्यदु॑हन्न् । \newline
40. प्र॒त्यदु॑ह॒न् तत् तत् प्र॒त्यदु॑हन् प्र॒त्यदु॑ह॒न् तत् । \newline
41. प्र॒त्यदु॑ह॒न्निति॑ प्रति - अदु॑हन्न् । \newline
42. तत् प्र॑ति॒धुषः॑ प्रति॒धुष॒ स्तत् तत् प्र॑ति॒धुषः॑ । \newline
43. प्र॒ति॒धुषः॑ प्रतिधु॒क्त्वम् प्र॑तिधु॒क्त्वम् प्र॑ति॒धुषः॑ प्रति॒धुषः॑ प्रतिधु॒क्त्वम् । \newline
44. प्र॒ति॒धुष॒ इति॑ प्रति - धुषः॑ । \newline
45. प्र॒ति॒धु॒क्त्वꣳ सꣳ सम् प्र॑तिधु॒क्त्वम् प्र॑तिधु॒क्त्वꣳ सम् । \newline
46. प्र॒ति॒धु॒क्त्वमिति॑ प्रतिधुक् - त्वम् । \newline
47. स म॑नैषुरनैषुः॒ सꣳ स म॑नैषुः । \newline
48. अ॒नै॒षुः॒ प्रति॒ प्रत्य॑नैषु रनैषुः॒ प्रति॑ । \newline
49. प्रत्य॑धुक्षन् नधुक्ष॒न् प्रति॒ प्रत्य॑धुक्षन्न् । \newline
50. अ॒धु॒क्ष॒न् न नाधु॑क्षन् नधुक्ष॒न् न । \newline
51. न तु तु न न तु । \newline
52. तु मयि॒ मयि॒ तु तु मयि॑ । \newline
53. मयि॑ श्रयते श्रयते॒ मयि॒ मयि॑ श्रयते । \newline
54. श्र॒य॒त॒ इतीति॑ श्रयते श्रयत॒ इति॑ । \newline
55. इत्य॑ब्रवी दब्रवी॒ दिती त्य॑ब्रवीत् । \newline
56. अ॒ब्र॒वी॒ दे॒त दे॒त द॑ब्रवी दब्रवी दे॒तत् । \newline
57. ए॒त द॑स्मा अस्मा ए॒त दे॒त द॑स्मै । \newline
58. अ॒स्मै॒ शृ॒तꣳ शृ॒त म॑स्मा अस्मै शृ॒तम् । \newline

\textbf{Ghana Paata } \newline

1. पृ॒थि॒वी मन्वनु॑ पृथि॒वीम् पृ॑थि॒वी मनु॒ वि व्यनु॑ पृथि॒वीम् पृ॑थि॒वी मनु॒ वि । \newline
2. अनु॒ वि व्यन्वनु॒ व्या॑रदार॒द् व्यन्वनु॒ व्या॑रत् । \newline
3. व्या॑रदार॒द् वि व्या॑र॒त् तत् तदा॑र॒द् वि व्या॑र॒त् तत् । \newline
4. आ॒र॒त् तत् तदा॑रदार॒त् तदोष॑धय॒ ओष॑धय॒ स्तदा॑र दार॒त् तदोष॑धयः । \newline
5. तदोष॑धय॒ ओष॑धय॒ स्तत् तदोष॑धयो वी॒रुधो॑ वी॒रुध॒ ओष॑धय॒ स्तत् तदोष॑धयो वी॒रुधः॑ । \newline
6. ओष॑धयो वी॒रुधो॑ वी॒रुध॒ ओष॑धय॒ ओष॑धयो वी॒रुधो॑ ऽभूवन् नभूवन्. वी॒रुध॒ ओष॑धय॒ ओष॑धयो वी॒रुधो॑ ऽभूवन्न् । \newline
7. वी॒रुधो॑ ऽभूवन् नभूवन्. वी॒रुधो॑ वी॒रुधो॑ ऽभूव॒न् नितीत्य॑भूवन्. वी॒रुधो॑ वी॒रुधो॑ ऽभूव॒न् निति॑ । \newline
8. अ॒भू॒व॒न् नितीत्य॑भूवन् नभूव॒न् निति॒ स स इत्य॑भूवन् नभूव॒न् निति॒ सः । \newline
9. इति॒ स स इतीति॒ स प्र॒जाप॑तिः प्र॒जाप॑तिः॒ स इतीति॒ स प्र॒जाप॑तिः । \newline
10. स प्र॒जाप॑तिः प्र॒जाप॑तिः॒ स स प्र॒जाप॑तिः प॒शून् प॒शून् प्र॒जाप॑तिः॒ स स प्र॒जाप॑तिः प॒शून् । \newline
11. प्र॒जाप॑तिः प॒शून् प॒शून् प्र॒जाप॑तिः प्र॒जाप॑तिः प॒शू न॑ब्रवी दब्रवीत् प॒शून् प्र॒जाप॑तिः प्र॒जाप॑तिः प॒शू न॑ब्रवीत् । \newline
12. प्र॒जाप॑ति॒रिति॑ प्र॒जा - प॒तिः॒ । \newline
13. प॒शू न॑ब्रवी दब्रवीत् प॒शून् प॒शू न॑ब्रवी दे॒त दे॒त द॑ब्रवीत् प॒शून् प॒शू न॑ब्रवी दे॒तत् । \newline
14. अ॒ब्र॒वी॒ दे॒त दे॒त द॑ब्रवी दब्रवी दे॒तद॑स्मा अस्मा ए॒त द॑ब्रवी दब्रवी दे॒तद॑स्मै । \newline
15. ए॒तद॑स्मा अस्मा ए॒त दे॒तद॑स्मै॒ सꣳ स म॑स्मा ए॒त दे॒तद॑स्मै॒ सम् । \newline
16. अ॒स्मै॒ सꣳ स म॑स्मा अस्मै॒ सन्न॑यत नयत॒ स म॑स्मा अस्मै॒ सन्न॑यत । \newline
17. सम् न॑यत नयत॒ सꣳ सम् न॑य॒ते तीति॑ नयत॒ सꣳ सम् न॑य॒ते ति॑ । \newline
18. न॒य॒ते तीति॑ नयत नय॒ते ति॒ तत् तदिति॑ नयत नय॒ते ति॒ तत् । \newline
19. इति॒ तत् तदितीति॒ तत् प॒शवः॑ प॒शव॒ स्तदितीति॒ तत् प॒शवः॑ । \newline
20. तत् प॒शवः॑ प॒शव॒ स्तत् तत् प॒शव॒ ओष॑धीभ्य॒ ओष॑धीभ्यः प॒शव॒ स्तत् तत् प॒शव॒ ओष॑धीभ्यः । \newline
21. प॒शव॒ ओष॑धीभ्य॒ ओष॑धीभ्यः प॒शवः॑ प॒शव॒ ओष॑धी॒भ्यो ऽध्य ध्योष॑धीभ्यः प॒शवः॑ प॒शव॒ ओष॑धी॒भ्यो ऽधि॑ । \newline
22. ओष॑धी॒भ्यो ऽध्य ध्योष॑धीभ्य॒ ओष॑धी॒भ्यो ऽध्या॒त्मन् ना॒त्मन् नध्योष॑धीभ्य॒ ओष॑धी॒भ्यो ऽध्या॒त्मन्न् । \newline
23. ओष॑धीभ्य॒ इत्योष॑धि - भ्यः॒ । \newline
24. अध्या॒त्मन् ना॒त्मन् नध्य ध्या॒त्मन् थ्सꣳ स मा॒त्मन् नध्य ध्या॒त्मन् थ्सम् । \newline
25. आ॒त्मन् थ्सꣳ स मा॒त्मन् ना॒त्मन् थ्स म॑नयन् ननय॒न् थ्स मा॒त्मन् ना॒त्मन् थ्स म॑नयन्न् । \newline
26. स म॑नयन् ननय॒न् थ्सꣳ स म॑नय॒न् तत् तद॑नय॒न् थ्सꣳ स म॑नय॒न् तत् । \newline
27. अ॒न॒य॒न् तत् तद॑नयन् ननय॒न् तत् प्रति॒ प्रति॒ तद॑नयन् ननय॒न् तत् प्रति॑ । \newline
28. तत् प्रति॒ प्रति॒ तत् तत् प्रत्य॑दुहन् नदुह॒न् प्रति॒ तत् तत् प्रत्य॑दुहन्न् । \newline
29. प्रत्य॑दुहन् नदुह॒न् प्रति॒ प्रत्य॑दुह॒न्॒. यद् यद॑दुह॒न् प्रति॒ प्रत्य॑दुह॒न्॒. यत् । \newline
30. अ॒दु॒ह॒न्॒. यद् यद॑दुहन् नदुह॒न्॒. यथ् स॒मन॑यन् थ्स॒मन॑य॒न्॒. यद॑दुहन् नदुह॒न्॒. यथ् स॒मन॑यन्न् । \newline
31. यथ् स॒मन॑यन् थ्स॒मन॑य॒न्॒. यद् यथ् स॒मन॑य॒न् तत् तथ् स॒मन॑य॒न्॒. यद् यथ् स॒मन॑य॒न् तत् । \newline
32. स॒मन॑य॒न् तत् तथ् स॒मन॑यन् थ्स॒मन॑य॒न् तथ् सा᳚न्ना॒य्यस्य॑ सान्ना॒य्यस्य॒ तथ् स॒मन॑यन् थ्स॒मन॑य॒न् तथ् सा᳚न्ना॒य्यस्य॑ । \newline
33. स॒मन॑य॒न्निति॑ सं - अन॑यन्न् । \newline
34. तथ् सा᳚न्ना॒य्यस्य॑ सान्ना॒य्यस्य॒ तत् तथ् सा᳚न्ना॒य्यस्य॑ सान्नाय्य॒त्वꣳ सा᳚न्नाय्य॒त्वꣳ सा᳚न्ना॒य्यस्य॒ तत् तथ् सा᳚न्ना॒य्यस्य॑ सान्नाय्य॒त्वम् । \newline
35. सा॒न्ना॒य्यस्य॑ सान्नाय्य॒त्वꣳ सा᳚न्नाय्य॒त्वꣳ सा᳚न्ना॒य्यस्य॑ सान्ना॒य्यस्य॑ सान्नाय्य॒त्वं ॅयद् यथ् सा᳚न्नाय्य॒त्वꣳ सा᳚न्ना॒य्यस्य॑ सान्ना॒य्यस्य॑ सान्नाय्य॒त्वं ॅयत् । \newline
36. सा॒न्ना॒य्यस्येति॑ सां - ना॒य्यस्य॑ । \newline
37. सा॒न्ना॒य्य॒त्वं ॅयद् यथ् सा᳚न्नाय्य॒त्वꣳ सा᳚न्नाय्य॒त्वं ॅयत् प्र॒त्यदु॑हन् प्र॒त्यदु॑ह॒न्॒. यथ् सा᳚न्नाय्य॒त्वꣳ सा᳚न्नाय्य॒त्वं ॅयत् प्र॒त्यदु॑हन्न् । \newline
38. सा॒न्ना॒य्य॒त्वमिति॑ सान्नाय्य - त्वम् । \newline
39. यत् प्र॒त्यदु॑हन् प्र॒त्यदु॑ह॒न्॒. यद् यत् प्र॒त्यदु॑ह॒न् तत् तत् प्र॒त्यदु॑ह॒न्॒. यद् यत् प्र॒त्यदु॑ह॒न् तत् । \newline
40. प्र॒त्यदु॑ह॒न् तत् तत् प्र॒त्यदु॑हन् प्र॒त्यदु॑ह॒न् तत् प्र॑ति॒धुषः॑ प्रति॒धुष॒ स्तत् प्र॒त्यदु॑हन् प्र॒त्यदु॑ह॒न् तत् प्र॑ति॒धुषः॑ । \newline
41. प्र॒त्यदु॑ह॒न्निति॑ प्रति - अदु॑हन्न् । \newline
42. तत् प्र॑ति॒धुषः॑ प्रति॒धुष॒ स्तत् तत् प्र॑ति॒धुषः॑ प्रतिधु॒क्त्वम् प्र॑तिधु॒क्त्वम् प्र॑ति॒धुष॒ स्तत् तत् प्र॑ति॒धुषः॑ प्रतिधु॒क्त्वम् । \newline
43. प्र॒ति॒धुषः॑ प्रतिधु॒क्त्वम् प्र॑तिधु॒क्त्वम् प्र॑ति॒धुषः॑ प्रति॒धुषः॑ प्रतिधु॒क्त्वꣳ सꣳ सम् प्र॑तिधु॒क्त्वम् प्र॑ति॒धुषः॑ प्रति॒धुषः॑ प्रतिधु॒क्त्वꣳ सम् । \newline
44. प्र॒ति॒धुष॒ इति॑ प्रति - धुषः॑ । \newline
45. प्र॒ति॒धु॒क्त्वꣳ सꣳ सम् प्र॑तिधु॒क्त्वम् प्र॑तिधु॒क्त्वꣳ स म॑नैषु रनैषुः॒ सम् प्र॑तिधु॒क्त्वम् प्र॑तिधु॒क्त्वꣳ स म॑नैषुः । \newline
46. प्र॒ति॒धु॒क्त्वमिति॑ प्रतिधुक् - त्वम् । \newline
47. स म॑नैषु रनैषुः॒ सꣳ स म॑नैषुः॒ प्रति॒ प्रत्य॑नैषुः॒ सꣳ स म॑नैषुः॒ प्रति॑ । \newline
48. अ॒नै॒षुः॒ प्रति॒ प्रत्य॑नैषु रनैषुः॒ प्रत्य॑धुक्षन् नधुक्ष॒न् प्रत्य॑नैषु रनैषुः॒ प्रत्य॑धुक्षन्न् । \newline
49. प्रत्य॑धुक्षन् नधुक्ष॒न् प्रति॒ प्रत्य॑धुक्ष॒न् न नाधु॑क्ष॒न् प्रति॒ प्रत्य॑धुक्ष॒न् न । \newline
50. अ॒धु॒क्ष॒न् न नाधु॑क्षन् नधुक्ष॒न् न तु तु नाधु॑क्षन् नधुक्ष॒न् न तु । \newline
51. न तु तु न न तु मयि॒ मयि॒ तु न न तु मयि॑ । \newline
52. तु मयि॒ मयि॒ तु तु मयि॑ श्रयते श्रयते॒ मयि॒ तु तु मयि॑ श्रयते । \newline
53. मयि॑ श्रयते श्रयते॒ मयि॒ मयि॑ श्रयत॒ इतीति॑ श्रयते॒ मयि॒ मयि॑ श्रयत॒ इति॑ । \newline
54. श्र॒य॒त॒ इतीति॑ श्रयते श्रयत॒ इत्य॑ब्रवी दब्रवी॒दिति॑ श्रयते श्रयत॒ इत्य॑ब्रवीत् । \newline
55. इत्य॑ब्रवी दब्रवी॒ दितीत्य॑ब्रवी दे॒त दे॒त द॑ब्रवी॒ दितीत्य॑ब्रवी दे॒तत् । \newline
56. अ॒ब्र॒वी॒ दे॒त दे॒त द॑ब्रवीदब्रवी दे॒तद॑स्मा अस्मा ए॒त द॑ब्रवी दब्रवी दे॒तद॑स्मै । \newline
57. ए॒तद॑स्मा अस्मा ए॒त दे॒तद॑स्मै शृ॒तꣳ शृ॒त म॑स्मा ए॒त दे॒तद॑स्मै शृ॒तम् । \newline
58. अ॒स्मै॒ शृ॒तꣳ शृ॒त म॑स्मा अस्मै शृ॒तम् कु॑रुत कुरुत शृ॒त म॑स्मा अस्मै शृ॒तम् कु॑रुत । \newline
\pagebreak
\markright{ TS 2.5.3.4  \hfill https://www.vedavms.in \hfill}
\addcontentsline{toc}{section}{ TS 2.5.3.4 }
\section*{ TS 2.5.3.4 }

\textbf{TS 2.5.3.4 } \newline
\textbf{Samhita Paata} \newline

शृ॒तं कु॑रु॒तेत्य॑ब्रवी॒त् तद॑स्मै शृ॒त-म॑कुर्वन्निन्द्रि॒यं ॅवावास्मि॑न् वी॒र्यं॑ तद॑श्रय॒न् तच्छृ॒तस्य॑ शृत॒त्वꣳ सम॑नैषुः॒ प्रत्य॑धुक्षञ्छृ॒तम॑क्र॒न् न तु मा॑ धिनो॒तीत्य॑ब्रवीदे॒तद॑स्मै॒ दधि॑ कुरु॒तेत्य॑ब्रवी॒त् तद॑स्मै॒ दद्ध्य॑कुर्व॒न् तदे॑नमधिनो॒त् तद्द॒द्ध्नो द॑धि॒त्वं ब्र॑ह्मवा॒दिनो॑ वदन्ति द॒द्ध्नः पूर्व॑स्याव॒देयं॒ - [  ] \newline

\textbf{Pada Paata} \newline

शृ॒तम् । कु॒रु॒त॒ । इति॑ । अ॒ब्र॒वी॒त् । तत् । अ॒स्मै॒ । शृ॒तम् । अ॒कु॒र्व॒न्न् । इ॒न्द्रि॒यम् । वाव । अ॒स्मि॒न्न् । वी॒र्य᳚म् । तत् । अ॒श्र॒य॒न्न् । तत् । शृ॒तस्य॑ । शृ॒त॒त्वमिति॑ शृत - त्वम् । समिति॑ । अ॒नै॒षुः॒ । प्रतीति॑ । अ॒धु॒क्ष॒न्न् । शृ॒तम् । अ॒क्र॒न्न् । न । तु । मा॒ । धि॒नो॒ति॒ । इति॑ । अ॒ब्र॒वी॒त् । ए॒तत् । अ॒स्मै॒ । दधि॑ । कु॒रु॒त॒ । इति॑ । अ॒ब्र॒वी॒त् । तत् । अ॒स्मै॒ । दधि॑ । अ॒कु॒र्व॒न्न् । तत् । ए॒न॒म् । अ॒धि॒नो॒त् । तत् । द॒द्ध्नः । द॒धि॒त्वमिति॑ दधि - त्वम् । ब्र॒ह्म॒वा॒दिन॒ इति॑ ब्रह्म-वा॒दिनः॑ । व॒द॒न्ति॒ । द॒द्ध्नः । पूर्व॑स्य । अ॒व॒देय॒मित्य॑व - देय᳚म् ।  \newline


\textbf{Krama Paata} \newline

शृ॒तम् कु॑रुत । कु॒रु॒तेति॑ । इत्य॑ब्रवीत् । अ॒ब्र॒वी॒त् तत् । तद॑स्मै । अ॒स्मै॒ शृ॒तम् । शृ॒तम॑कुर्वन्न् । अ॒कु॒र्व॒न्नि॒न्द्रि॒यम् । इ॒न्द्रि॒यम् ॅवाव । वावास्मिन्न्॑ । अ॒स्मि॒न् वी॒र्य᳚म् । वी॒र्य॑म् तत् । तद॑श्रयन्न् । अ॒श्र॒य॒न् तत् । तच्छृ॒तस्य॑ । शृ॒तस्य॑ शृत॒त्वम् । शृ॒त॒त्वꣳ सम् । शृ॒त॒त्वमिति॑ शृत - त्वम् । सम॑नैषुः । अ॒नै॒षुः॒ प्रति॑ । प्रत्य॑धुक्षन्न् । अ॒धु॒क्ष॒ञ्छृ॒तम् । शृ॒तम॑क्रन्न् । अ॒क्र॒न् न । न तु । तु मा᳚ । मा॒ धि॒नो॒ति॒ । धि॒नो॒तीति॑ । इत्य॑ब्रवीत् । अ॒ब्र॒वी॒दे॒तत् । ए॒तद॑स्मै । अ॒स्मै॒ दधि॑ । दधि॑ कुरुत । कु॒रु॒तेति॑ । इत्य॑ब्रवीत् । अ॒ब्र॒वी॒त् तत् । तद॑स्मै । अ॒स्मै॒ दधि॑ । दद्ध्य॑कुर्वन्न् । अ॒कु॒र्व॒न् तत् । तदे॑नम् । ए॒न॒म॒धि॒नो॒त्॒ । अ॒धि॒नो॒त् तत् । तद् द॒द्ध्नः । द॒द्ध्नो द॑धि॒त्वम् । द॒धि॒त्वम् ब्र॑ह्मवा॒दिनः॑ । द॒धि॒त्वमिति॑ दधि - त्वम् । ब्र॒ह्म॒वा॒दिनो॑ वदन्ति । ब्र॒ह्म॒वा॒दिन॒ इति॑ ब्रह्म - वा॒दिनः॑ । व॒द॒न्ति॒ द॒द्ध्नः । द॒द्ध्नः पूर्व॑स्य । पूर्व॑स्याव॒देय᳚म् । अ॒व॒देय॒म् दधि॑ । अ॒व॒देय॒मित्य॑व - देय᳚म् \newline

\textbf{Jatai Paata} \newline

1. शृ॒तम् कु॑रुत कुरुत शृ॒तꣳ शृ॒तम् कु॑रुत । \newline
2. कु॒रु॒ते तीति॑ कुरुत कुरु॒ते ति॑ । \newline
3. इत्य॑ब्रवी दब्रवी॒ दिती त्य॑ब्रवीत् । \newline
4. अ॒ब्र॒वी॒त् तत् तद॑ब्रवी दब्रवी॒त् तत् । \newline
5. तद॑स्मा अस्मै॒ तत् तद॑स्मै । \newline
6. अ॒स्मै॒ शृ॒तꣳ शृ॒त म॑स्मा अस्मै शृ॒तम् । \newline
7. शृ॒त म॑कुर्वन् नकुर्वञ् छृ॒तꣳ शृ॒त म॑कुर्वन्न् । \newline
8. अ॒कु॒र्व॒न् नि॒न्द्रि॒य मि॑न्द्रि॒य म॑कुर्वन् नकुर्वन् निन्द्रि॒यम् । \newline
9. इ॒न्द्रि॒यं ॅवाव वावे न्द्रि॒य मि॑न्द्रि॒यं ॅवाव । \newline
10. वावास्मि॑न् नस्मि॒न्॒. वाव वावास्मिन्न्॑ । \newline
11. अ॒स्मि॒न् वी॒र्यं॑ ॅवी॒र्य॑ मस्मिन् नस्मिन् वी॒र्य᳚म् । \newline
12. वी॒र्य॑म् तत् तद् वी॒र्यं॑ ॅवी॒र्य॑म् तत् । \newline
13. तद॑श्रयन् नश्रय॒न् तत् तद॑श्रयन्न् । \newline
14. अ॒श्र॒य॒न् तत् तद॑श्रयन् नश्रय॒न् तत् । \newline
15. तच् छृ॒तस्य॑ शृ॒तस्य॒ तत् तच् छृ॒तस्य॑ । \newline
16. शृ॒तस्य॑ शृत॒त्वꣳ शृ॑त॒त्वꣳ शृ॒तस्य॑ शृ॒तस्य॑ शृत॒त्वम् । \newline
17. शृ॒त॒त्वꣳ सꣳ सꣳ शृ॑त॒त्वꣳ शृ॑त॒त्वꣳ सम् । \newline
18. शृ॒त॒त्वमिति॑ शृत - त्वम् । \newline
19. स म॑नैषु रनैषुः॒ सꣳ स म॑नैषुः । \newline
20. अ॒नै॒षुः॒ प्रति॒ प्रत्य॑नैषु रनैषुः॒ प्रति॑ । \newline
21. प्रत्य॑धुक्षन् नधुक्ष॒न् प्रति॒ प्रत्य॑धुक्षन्न् । \newline
22. अ॒धु॒क्ष॒ञ् छृ॒तꣳ शृ॒त म॑धुक्षन् नधुक्षञ् छृ॒तम् । \newline
23. शृ॒त म॑क्रन् नक्रञ् छृ॒तꣳ शृ॒त म॑क्रन्न् । \newline
24. अ॒क्र॒न् न नाक्र॑न् नक्र॒न् न । \newline
25. न तु तु न न तु । \newline
26. तु मा॑ मा॒ तु तु मा᳚ । \newline
27. मा॒ धि॒नो॒ति॒ धि॒नो॒ति॒ मा॒ मा॒ धि॒नो॒ति॒ । \newline
28. धि॒नो॒तीतीति॑ धिनोति धिनो॒तीति॑ । \newline
29. इत्य॑ब्रवी दब्रवी॒ दिती त्य॑ब्रवीत् । \newline
30. अ॒ब्र॒वी॒ दे॒त दे॒त द॑ब्रवी दब्रवी दे॒तत् । \newline
31. ए॒तद॑स्मा अस्मा ए॒त दे॒त द॑स्मै । \newline
32. अ॒स्मै॒ दधि॒ दध्य॑स्मा अस्मै॒ दधि॑ । \newline
33. दधि॑ कुरुत कुरुत॒ दधि॒ दधि॑ कुरुत । \newline
34. कु॒रु॒ते तीति॑ कुरुत कुरु॒ते ति॑ । \newline
35. इत्य॑ब्रवी दब्रवी॒ दिती त्य॑ब्रवीत् । \newline
36. अ॒ब्र॒वी॒त् तत् तद॑ब्रवी दब्रवी॒त् तत् । \newline
37. तद॑स्मा अस्मै॒ तत् तद॑स्मै । \newline
38. अ॒स्मै॒ दधि॒ दध्य॑स्मा अस्मै॒ दधि॑ । \newline
39. दध्य॑कुर्वन् नकुर्व॒न् दधि॒ दध्य॑कुर्वन्न् । \newline
40. अ॒कु॒र्व॒न् तत् तद॑कुर्वन् नकुर्व॒न् तत् । \newline
41. तदे॑न मेन॒म् तत् तदे॑नम् । \newline
42. ए॒न॒ म॒धि॒नो॒ द॒धि॒नो॒ दे॒न॒ मे॒न॒ म॒धि॒नो॒त् । \newline
43. अ॒धि॒नो॒त् तत् तद॑धिनो दधिनो॒त् तत् । \newline
44. तद् द॒द्ध्नो द॒द्ध्न स्तत् तद् द॒द्ध्नः । \newline
45. द॒द्ध्नो द॑धि॒त्वम् द॑धि॒त्वम् द॒द्ध्नो द॒द्ध्नो द॑धि॒त्वम् । \newline
46. द॒धि॒त्वम् ब्र॑ह्मवा॒दिनो᳚ ब्रह्मवा॒दिनो॑ दधि॒त्वम् द॑धि॒त्वम् ब्र॑ह्मवा॒दिनः॑ । \newline
47. द॒धि॒त्वमिति॑ दधि - त्वम् । \newline
48. ब्र॒ह्म॒वा॒दिनो॑ वदन्ति वदन्ति ब्रह्मवा॒दिनो᳚ ब्रह्मवा॒दिनो॑ वदन्ति । \newline
49. ब्र॒ह्म॒वा॒दिन॒ इति॑ ब्रह्म - वा॒दिनः॑ । \newline
50. व॒द॒न्ति॒ द॒द्ध्नो द॒द्ध्नो व॑दन्ति वदन्ति द॒द्ध्नः । \newline
51. द॒द्ध्नः पूर्व॑स्य॒ पूर्व॑स्य द॒द्ध्नो द॒द्ध्नः पूर्व॑स्य । \newline
52. पूर्व॑स्याव॒देय॑ मव॒देय॒म् पूर्व॑स्य॒ पूर्व॑स्याव॒देय᳚म् । \newline
53. अ॒व॒देय॒म् दधि॒ दध्य॑व॒देय॑ मव॒देय॒म् दधि॑ । \newline
54. अ॒व॒देय॒मित्य॑व - देय᳚म् । \newline

\textbf{Ghana Paata } \newline

1. शृ॒तम् कु॑रुत कुरुत शृ॒तꣳ शृ॒तम् कु॑रु॒ते तीति॑ कुरुत शृ॒तꣳ शृ॒तम् कु॑रु॒ते ति॑ । \newline
2. कु॒रु॒ते तीति॑ कुरुत कुरु॒ते त्य॑ब्रवी दब्रवी॒ दिति॑ कुरुत कुरु॒ते त्य॑ब्रवीत् । \newline
3. इत्य॑ब्रवी दब्रवी॒ दिती त्य॑ब्रवी॒त् तत् तद॑ब्रवी॒ दिती त्य॑ब्रवी॒त् तत् । \newline
4. अ॒ब्र॒वी॒त् तत् तद॑ब्रवी दब्रवी॒त् तद॑स्मा अस्मै॒ तद॑ब्रवी दब्रवी॒त् तद॑स्मै । \newline
5. तद॑स्मा अस्मै॒ तत् तद॑स्मै शृ॒तꣳ शृ॒त म॑स्मै॒ तत् तद॑स्मै शृ॒तम् । \newline
6. अ॒स्मै॒ शृ॒तꣳ शृ॒त म॑स्मा अस्मै शृ॒त म॑कुर्वन् नकुर्वञ् छृ॒त म॑स्मा अस्मै शृ॒त म॑कुर्वन्न् । \newline
7. शृ॒त म॑कुर्वन् नकुर्वञ् छृ॒तꣳ शृ॒त म॑कुर्वन् निन्द्रि॒य मि॑न्द्रि॒य म॑कुर्वञ् छृ॒तꣳ शृ॒त म॑कुर्वन् निन्द्रि॒यम् । \newline
8. अ॒कु॒र्व॒न् नि॒न्द्रि॒य मि॑न्द्रि॒य म॑कुर्वन् नकुर्वन् निन्द्रि॒यं ॅवाव वावे न्द्रि॒य म॑कुर्वन् नकुर्वन् निन्द्रि॒यं ॅवाव । \newline
9. इ॒न्द्रि॒यं ॅवाव वावे न्द्रि॒य मि॑न्द्रि॒यं ॅवावास्मि॑न् नस्मि॒न्॒. वावे न्द्रि॒य मि॑न्द्रि॒यं ॅवावास्मिन्न्॑ । \newline
10. वावास्मि॑न् नस्मि॒न्॒. वाव वावास्मि॑न् वी॒र्यं॑ ॅवी॒र्य॑ मस्मि॒न्॒. वाव वावास्मि॑न् वी॒र्य᳚म् । \newline
11. अ॒स्मि॒न् वी॒र्यं॑ ॅवी॒र्य॑ मस्मिन् नस्मिन् वी॒र्य॑म् तत् तद् वी॒र्य॑ मस्मिन् नस्मिन् वी॒र्य॑म् तत् । \newline
12. वी॒र्य॑म् तत् तद् वी॒र्यं॑ ॅवी॒र्य॑म् तद॑श्रयन् नश्रय॒न् तद् वी॒र्यं॑ ॅवी॒र्य॑म् तद॑श्रयन्न् । \newline
13. तद॑श्रयन् नश्रय॒न् तत् तद॑श्रय॒न् तत् तद॑श्रय॒न् तत् तद॑श्रय॒न् तत् । \newline
14. अ॒श्र॒य॒न् तत् तद॑श्रयन् नश्रय॒न् तच् छृ॒तस्य॑ शृ॒तस्य॒ तद॑श्रयन् नश्रय॒न् तच् छृ॒तस्य॑ । \newline
15. तच् छृ॒तस्य॑ शृ॒तस्य॒ तत् तच् छृ॒तस्य॑ शृत॒त्वꣳ शृ॑त॒त्वꣳ शृ॒तस्य॒ तत् तच् छृ॒तस्य॑ शृत॒त्वम् । \newline
16. शृ॒तस्य॑ शृत॒त्वꣳ शृ॑त॒त्वꣳ शृ॒तस्य॑ शृ॒तस्य॑ शृत॒त्वꣳ सꣳ सꣳ शृ॑त॒त्वꣳ शृ॒तस्य॑ शृ॒तस्य॑ शृत॒त्वꣳ सम् । \newline
17. शृ॒त॒त्वꣳ सꣳ सꣳ शृ॑त॒त्वꣳ शृ॑त॒त्वꣳ स म॑नैषु रनैषुः॒ सꣳ शृ॑त॒त्वꣳ शृ॑त॒त्वꣳ स म॑नैषुः । \newline
18. शृ॒त॒त्वमिति॑ शृत - त्वम् । \newline
19. स म॑नैषु रनैषुः॒ सꣳ स म॑नैषुः॒ प्रति॒ प्रत्य॑नैषुः॒ सꣳ स म॑नैषुः॒ प्रति॑ । \newline
20. अ॒नै॒षुः॒ प्रति॒ प्रत्य॑नैषु रनैषुः॒ प्रत्य॑धुक्षन् नधुक्ष॒न् प्रत्य॑नैषु रनैषुः॒ प्रत्य॑धुक्षन्न् । \newline
21. प्रत्य॑धुक्षन् नधुक्ष॒न् प्रति॒ प्रत्य॑धुक्षञ् छृ॒तꣳ शृ॒त म॑धुक्ष॒न् प्रति॒ प्रत्य॑धुक्षञ् छृ॒तम् । \newline
22. अ॒धु॒क्ष॒ञ् छृ॒तꣳ शृ॒त म॑धुक्षन् नधुक्षञ् छृ॒त म॑क्रन् नक्रञ् छृ॒त म॑धुक्षन् नधुक्षञ् छृ॒त म॑क्रन्न् । \newline
23. शृ॒त म॑क्रन् नक्रञ् छृ॒तꣳ शृ॒त म॑क्र॒न् न नाक्र॑ञ् छृ॒तꣳ शृ॒त म॑क्र॒न् न । \newline
24. अ॒क्र॒न् न नाक्र॑न् नक्र॒न् न तु तु नाक्र॑न् नक्र॒न् न तु । \newline
25. न तु तु न न तु मा॑ मा॒ तु न न तु मा᳚ । \newline
26. तु मा॑ मा॒ तु तु मा॑ धिनोति धिनोति मा॒ तु तु मा॑ धिनोति । \newline
27. मा॒ धि॒नो॒ति॒ धि॒नो॒ति॒ मा॒ मा॒ धि॒नो॒तीतीति॑ धिनोति मा मा धिनो॒तीति॑ । \newline
28. धि॒नो॒तीतीति॑ धिनोति धिनो॒ती त्य॑ब्रवी दब्रवी॒ दिति॑ धिनोति धिनो॒ती त्य॑ब्रवीत् । \newline
29. इत्य॑ब्रवी दब्रवी॒ दिती त्य॑ब्रवी दे॒त दे॒त द॑ब्रवी॒ दिती त्य॑ब्रवी दे॒तत् । \newline
30. अ॒ब्र॒वी॒ दे॒त दे॒त द॑ब्रवी दब्रवी दे॒त द॑स्मा अस्मा ए॒त द॑ब्रवी दब्रवी दे॒तद॑स्मै । \newline
31. ए॒तद॑स्मा अस्मा ए॒त दे॒तद॑स्मै॒ दधि॒ दध्य॑स्मा ए॒त दे॒तद॑स्मै॒ दधि॑ । \newline
32. अ॒स्मै॒ दधि॒ दध्य॑स्मा अस्मै॒ दधि॑ कुरुत कुरुत॒ दध्य॑स्मा अस्मै॒ दधि॑ कुरुत । \newline
33. दधि॑ कुरुत कुरुत॒ दधि॒ दधि॑ कुरु॒ते तीति॑ कुरुत॒ दधि॒ दधि॑ कुरु॒ते ति॑ । \newline
34. कु॒रु॒ते तीति॑ कुरुत कुरु॒ते त्य॑ब्रवी दब्रवी॒दिति॑ कुरुत कुरु॒ते त्य॑ब्रवीत् । \newline
35. इत्य॑ब्रवी दब्रवी॒ दिती त्य॑ब्रवी॒त् तत् तद॑ब्रवी॒ दितीत्य॑ब्रवी॒त् तत् । \newline
36. अ॒ब्र॒वी॒त् तत् तद॑ब्रवी दब्रवी॒त् तद॑स्मा अस्मै॒ तद॑ब्रवी दब्रवी॒त् तद॑स्मै । \newline
37. तद॑स्मा अस्मै॒ तत् तद॑स्मै॒ दधि॒ दध्य॑स्मै॒ तत् तद॑स्मै॒ दधि॑ । \newline
38. अ॒स्मै॒ दधि॒ दध्य॑स्मा अस्मै॒ दध्य॑कुर्वन् नकुर्व॒न् दध्य॑स्मा अस्मै॒ दध्य॑कुर्वन्न् । \newline
39. दध्य॑कुर्वन् नकुर्व॒न् दधि॒ दध्य॑कुर्व॒न् तत् तद॑कुर्व॒न् दधि॒ दध्य॑कुर्व॒न् तत् । \newline
40. अ॒कु॒र्व॒न् तत् तद॑कुर्वन् नकुर्व॒न् तदे॑न मेन॒म् तद॑कुर्वन् नकुर्व॒न् तदे॑नम् । \newline
41. तदे॑न मेन॒म् तत् तदे॑न मधिनो दधिनो देन॒म् तत् तदे॑न मधिनोत् । \newline
42. ए॒न॒ म॒धि॒नो॒ द॒धि॒नो॒ दे॒न॒ मे॒न॒ म॒धि॒नो॒त् तत् तद॑धिनो देन मेन मधिनो॒त् तत् । \newline
43. अ॒धि॒नो॒त् तत् तद॑धिनो दधिनो॒त् तद् द॒द्ध्नो द॒द्ध्न स्तद॑धिनो दधिनो॒त् तद् द॒द्ध्नः । \newline
44. तद् द॒द्ध्नो द॒द्ध्न स्तत् तद् द॒द्ध्नो द॑धि॒त्वम् द॑धि॒त्वम् द॒द्ध्न स्तत् तद् द॒द्ध्नो द॑धि॒त्वम् । \newline
45. द॒द्ध्नो द॑धि॒त्वम् द॑धि॒त्वम् द॒द्ध्नो द॒द्ध्नो द॑धि॒त्वम् ब्र॑ह्मवा॒दिनो᳚ ब्रह्मवा॒दिनो॑ दधि॒त्वम् द॒द्ध्नो द॒द्ध्नो द॑धि॒त्वम् ब्र॑ह्मवा॒दिनः॑ । \newline
46. द॒धि॒त्वम् ब्र॑ह्मवा॒दिनो᳚ ब्रह्मवा॒दिनो॑ दधि॒त्वम् द॑धि॒त्वम् ब्र॑ह्मवा॒दिनो॑ वदन्ति वदन्ति ब्रह्मवा॒दिनो॑ दधि॒त्वम् द॑धि॒त्वम् ब्र॑ह्मवा॒दिनो॑ वदन्ति । \newline
47. द॒धि॒त्वमिति॑ दधि - त्वम् । \newline
48. ब्र॒ह्म॒वा॒दिनो॑ वदन्ति वदन्ति ब्रह्मवा॒दिनो᳚ ब्रह्मवा॒दिनो॑ वदन्ति द॒द्ध्नो द॒द्ध्नो व॑दन्ति ब्रह्मवा॒दिनो᳚ ब्रह्मवा॒दिनो॑ वदन्ति द॒द्ध्नः । \newline
49. ब्र॒ह्म॒वा॒दिन॒ इति॑ ब्रह्म - वा॒दिनः॑ । \newline
50. व॒द॒न्ति॒ द॒द्ध्नो द॒द्ध्नो व॑दन्ति वदन्ति द॒द्ध्नः पूर्व॑स्य॒ पूर्व॑स्य द॒द्ध्नो व॑दन्ति वदन्ति द॒द्ध्नः पूर्व॑स्य । \newline
51. द॒द्ध्नः पूर्व॑स्य॒ पूर्व॑स्य द॒द्ध्नो द॒द्ध्नः पूर्व॑स्याव॒देय॑ मव॒देय॒म् पूर्व॑स्य द॒द्ध्नो द॒द्ध्नः पूर्व॑स्याव॒देय᳚म् । \newline
52. पूर्व॑स्याव॒देय॑ मव॒देय॒म् पूर्व॑स्य॒ पूर्व॑स्याव॒देय॒म् दधि॒ दध्य॑व॒देय॒म् पूर्व॑स्य॒ पूर्व॑स्याव॒देय॒म् दधि॑ । \newline
53. अ॒व॒देय॒म् दधि॒ दध्य॑व॒देय॑ मव॒देय॒म् दधि॒ हि हि दध्य॑व॒देय॑ मव॒देय॒म् दधि॒ हि । \newline
54. अ॒व॒देय॒मित्य॑व - देय᳚म् । \newline
\pagebreak
\markright{ TS 2.5.3.5  \hfill https://www.vedavms.in \hfill}
\addcontentsline{toc}{section}{ TS 2.5.3.5 }
\section*{ TS 2.5.3.5 }

\textbf{TS 2.5.3.5 } \newline
\textbf{Samhita Paata} \newline

दधि॒ हि पूर्वं॑ क्रि॒यत॒ इत्यना॑दृत्य॒ तच्छृ॒तस्यै॒व पूर्व॒स्याव॑ द्येदिन्द्रि॒यमे॒वास्मि॑न् वी॒र्यꣳ॑ श्रि॒त्वा द॒द्ध्नो परि॑ष्टाद्धिनोति यथापू॒र्वमुपै॑ति॒ यत् पू॒तीकै᳚र्वा पर्णव॒ल्कैर्वा॑ ऽऽत॒ञ्च्याथ् सौ॒म्यं तद्यत् क्व॑लै राक्ष॒सं तद्यत् त॑ण्डु॒लैर्वै᳚श्वदे॒वं तद्यदा॒तञ्च॑नेन मानु॒षं तद्-यद्-द॒द्ध्ना तथ् सेन्द्रं॑ द॒द्ध्ना ऽऽत॑नक्ति - [  ] \newline

\textbf{Pada Paata} \newline

दधि॑ । हि । पूर्व᳚म् । क्रि॒यते᳚ । इति॑ । अना॑दृ॒त्येत्यना᳚ - दृ॒त्य॒ । तत् । शृ॒तस्य॑ । ए॒व । पूर्व॑स्य । अवेति॑ । द्ये॒त् । इ॒न्द्रि॒यम् । ए॒व । अ॒स्मि॒न्न् । वी॒र्य᳚म् । श्रि॒त्वा । द॒द्ध्ना । उ॒परि॑ष्टात् । धि॒नो॒ति॒ । य॒था॒पू॒र्वमिति॑ यथा - पू॒र्वम् । उपेति॑ । ए॒ति॒ । यत् । पू॒तीकैः᳚ । वा॒ । प॒र्ण॒व॒ल्कैरिति॑ पर्ण - व॒ल्कैः । वा॒ । आ॒त॒ञ्च्यादित्या᳚ - त॒ञ्च्यात् । सौ॒म्यम् । तत् । यत् । क्व॑लैः । रा॒क्ष॒सम् । तत् । यत् । त॒ण्डु॒लैः । वै॒श्व॒दे॒वमिति॑ वैश्व - दे॒वम् । तत् । यत् । आ॒तञ्च॑ने॒नेत्या᳚ - तञ्च॑नेन । मा॒नु॒षम् । तत् । यत् । द॒द्ध्ना । तत् । सेन्द्र॒मिति॒ स - इ॒न्द्र॒म् । द॒द्ध्ना । एति॑ । त॒न॒क्ति॒ ।  \newline


\textbf{Krama Paata} \newline

दधि॒ हि । हि पूर्व᳚म् । पूर्व॑म् क्रि॒यते᳚ । क्रि॒यत॒ इति॑ । इत्यना॑दृत्य । अना॑दृत्य॒ तत् । अना॑दृ॒त्येत्यना᳚ - दृ॒त्य॒ । तच्छृ॒तस्य॑ । शृ॒तस्यै॒व । ए॒व पूर्व॑स्य । पूर्व॒स्याव॑ । अव॑ द्येत् । द्ये॒दि॒न्द्रि॒यम् । इ॒न्द्रि॒यमे॒व । ए॒वास्मिन्न्॑ । अ॒स्मि॒न् वी॒र्य᳚म् । वी॒र्यꣳ॑ श्रि॒त्वा । श्रि॒त्वा द॒द्ध्ना । द॒द्ध्नोपरि॑ष्टात् । उ॒परि॑ष्टाद् धिनोति । धि॒नो॒ति॒ य॒था॒पू॒र्वम् । य॒था॒पू॒र्वमुप॑ । य॒था॒पू॒र्वमिति॑ यथा - पू॒र्वम् । उपै॑ति । ए॒ति॒ यत् । यत् पू॒तीकैः᳚ । पू॒तीकै᳚र् वा । वा॒ प॒र्ण॒व॒ल्कैः । प॒र्ण॒व॒ल्कैर् वा᳚ । प॒र्ण॒व॒ल्कैरिति॑ पर्ण - व॒ल्कैः । वा॒ ऽऽत॒ञ्च्यात् । आ॒त॒ञ्च्याथ् सौ॒म्यम् । आ॒त॒ञ्च्यादित्या᳚ - त॒ञ्च्यात् । सौ॒म्यम् तत् । तद् यत् । यत् क्व॑लैः । क्व॑लै राक्ष॒सम् । रा॒क्ष॒सम् तत् । तद् यत् । यत् त॑ण्डु॒लैः । त॒ण्डु॒लैर् वै᳚श्वदे॒वम् । वै॒श्व॒दे॒वम् तत् । वै॒श्व॒दे॒मिति॑ वैश्व - दे॒वम् । तद् यत् । यदा॒तञ्च॑नेन । आ॒तञ्च॑नेन मानु॒षम् । आ॒तञ्च॑ने॒नेत्या᳚ - तञ्च॑नेन । मा॒नु॒षम् तत् । तद् यत् । यद् द॒द्ध्ना । द॒द्ध्ना तत् । तथ् सेन्द्र᳚म् । सेन्द्र॑म् द॒द्ध्ना । सेन्द्र॒मिति॒ स - इ॒न्द्र॒म् । द॒द्ध्ना ऽऽ त॑नक्ति । आ त॑नक्ति । त॒न॒क्ति॒ से॒न्द्र॒त्वाय॑ \newline

\textbf{Jatai Paata} \newline

1. दधि॒ हि हि दधि॒ दधि॒ हि । \newline
2. हि पूर्व॒म् पूर्वꣳ॒॒ हि हि पूर्व᳚म् । \newline
3. पूर्व॑म् क्रि॒यते᳚ क्रि॒यते॒ पूर्व॒म् पूर्व॑म् क्रि॒यते᳚ । \newline
4. क्रि॒यत॒ इतीति॑ क्रि॒यते᳚ क्रि॒यत॒ इति॑ । \newline
5. इत्यना॑दृ॒त्या ना॑दृ॒त्ये तीत्यना॑दृत्य । \newline
6. अना॑दृत्य॒ तत् तदना॑दृ॒त्या ना॑दृत्य॒ तत् । \newline
7. अना॑दृ॒त्येत्यना᳚ - दृ॒त्य॒ । \newline
8. तच्छृ॒तस्य॑ शृ॒तस्य॒ तत् तच्छृ॒तस्य॑ । \newline
9. शृ॒तस्यै॒वैव शृ॒तस्य॑ शृ॒तस्यै॒व । \newline
10. ए॒व पूर्व॑स्य॒ पूर्व॑स्यै॒वैव पूर्व॑स्य । \newline
11. पूर्व॒स्यावाव॒ पूर्व॑स्य॒ पूर्व॒स्याव॑ । \newline
12. अव॑ द्येद् द्ये॒दवाव॑ द्येत् । \newline
13. द्ये॒दि॒न्द्रि॒य मि॑न्द्रि॒यम् द्ये᳚द् द्येदिन्द्रि॒यम् । \newline
14. इ॒न्द्रि॒य मे॒वैवे न्द्रि॒य मि॑न्द्रि॒य मे॒व । \newline
15. ए॒वास्मि॑न् नस्मिन् ने॒वैवास्मिन्न्॑ । \newline
16. अ॒स्मि॒न् वी॒र्यं॑ ॅवी॒र्य॑ मस्मिन् नस्मिन् वी॒र्य᳚म् । \newline
17. वी॒र्यꣳ॑ श्रि॒त्वा श्रि॒त्वा वी॒र्यं॑ ॅवी॒र्यꣳ॑ श्रि॒त्वा । \newline
18. श्रि॒त्वा द॒द्ध्ना द॒द्ध्ना श्रि॒त्वा श्रि॒त्वा द॒द्ध्ना । \newline
19. द॒द्ध्नो परि॑ष्टा दु॒परि॑ष्टाद् द॒द्ध्ना द॒द्ध्नो परि॑ष्टात् । \newline
20. उ॒परि॑ष्टाद् धिनोति धिनो त्यु॒परि॑ष्टा दु॒परि॑ष्टाद् धिनोति । \newline
21. धि॒नो॒ति॒ य॒था॒पू॒र्वं ॅय॑थापू॒र्वम् धि॑नोति धिनोति यथापू॒र्वम् । \newline
22. य॒था॒पू॒र्व मुपोप॑ यथापू॒र्वं ॅय॑थापू॒र्व मुप॑ । \newline
23. य॒था॒पू॒र्वमिति॑ यथा - पू॒र्वम् । \newline
24. उपै᳚त्ये॒त्युपोपै॑ति । \newline
25. ए॒ति॒ यद् यदे᳚त्येति॒ यत् । \newline
26. यत् पू॒तीकैः᳚ पू॒तीकै॒र् यद् यत् पू॒तीकैः᳚ । \newline
27. पू॒तीकै᳚र् वा वा पू॒तीकैः᳚ पू॒तीकै᳚र् वा । \newline
28. वा॒ प॒र्ण॒व॒ल्कैः प॑र्णव॒ल्कैर् वा॑ वा पर्णव॒ल्कैः । \newline
29. प॒र्ण॒व॒ल्कैर् वा॑ वा पर्णव॒ल्कैः प॑र्णव॒ल्कैर् वा᳚ । \newline
30. प॒र्ण॒व॒ल्कैरिति॑ पर्ण - व॒ल्कैः । \newline
31. वा॒ ऽऽत॒ञ्च्या दा॑त॒ञ्च्याद् वा॑ वा ऽऽत॒ञ्च्यात् । \newline
32. आ॒त॒ञ्च्याथ् सौ॒म्यꣳ सौ॒म्य मा॑त॒ञ्च्या दा॑त॒ञ्च्याथ् सौ॒म्यम् । \newline
33. आ॒त॒ञ्च्यादित्या᳚ - त॒ञ्च्यात् । \newline
34. सौ॒म्यम् तत् तथ् सौ॒म्यꣳ सौ॒म्यम् तत् । \newline
35. तद् यद् यत् तत् तद् यत् । \newline
36. यत् क्व॑लैः॒ क्व॑लै॒र् यद् यत् क्व॑लैः । \newline
37. क्व॑लै राक्ष॒सꣳ रा᳚क्ष॒सम् क्व॑लैः॒ क्व॑लै राक्ष॒सम् । \newline
38. रा॒क्ष॒सम् तत् तद् रा᳚क्ष॒सꣳ रा᳚क्ष॒सम् तत् । \newline
39. तद् यद् यत् तत् तद् यत् । \newline
40. यत् त॑ण्डु॒लै स्त॑ण्डु॒लैर् यद् यत् त॑ण्डु॒लैः । \newline
41. त॒ण्डु॒लैर् वै᳚श्वदे॒वं ॅवै᳚श्वदे॒वम् त॑ण्डु॒लै स्त॑ण्डु॒लैर् वै᳚श्वदे॒वम् । \newline
42. वै॒श्व॒दे॒वम् तत् तद् वै᳚श्वदे॒वं ॅवै᳚श्वदे॒वम् तत् । \newline
43. वै॒श्व॒दे॒वमिति॑ वैश्व - दे॒वम् । \newline
44. तद् यद् यत् तत् तद् यत् । \newline
45. यदा॒तञ्च॑नेना॒ तञ्च॑नेन॒ यद् यदा॒तञ्च॑नेन । \newline
46. आ॒तञ्च॑नेन मानु॒षम् मा॑नु॒ष मा॒तञ्च॑नेना॒ तञ्च॑नेन मानु॒षम् । \newline
47. आ॒तञ्च॑ने॒नेत्या᳚ - तञ्च॑नेन । \newline
48. मा॒नु॒षम् तत् तन् मा॑नु॒षम् मा॑नु॒षम् तत् । \newline
49. तद् यद् यत् तत् तद् यत् । \newline
50. यद् द॒द्ध्ना द॒द्ध्ना यद् यद् द॒द्ध्ना । \newline
51. द॒द्ध्ना तत् तद् द॒द्ध्ना द॒द्ध्ना तत् । \newline
52. तथ् सेन्द्रꣳ॒॒ सेन्द्र॒म् तत् तथ् सेन्द्र᳚म् । \newline
53. सेन्द्र॑म् द॒द्ध्ना द॒द्ध्ना सेन्द्रꣳ॒॒ सेन्द्र॑म् द॒द्ध्ना । \newline
54. सेन्द्र॒मिति॒ स - इ॒न्द्र॒म् । \newline
55. द॒द्ध्ना ऽऽत॑नक्ति तन॒क्त्या द॒द्ध्ना द॒द्ध्ना ऽऽत॑नक्ति । \newline
56. आ त॑नक्ति तन॒क्त्या त॑नक्ति । \newline
57. त॒न॒क्ति॒ से॒न्द्र॒त्वाय॑ सेन्द्र॒त्वाय॑ तनक्ति तनक्ति सेन्द्र॒त्वाय॑ । \newline

\textbf{Ghana Paata } \newline

1. दधि॒ हि हि दधि॒ दधि॒ हि पूर्व॒म् पूर्वꣳ॒॒ हि दधि॒ दधि॒ हि पूर्व᳚म् । \newline
2. हि पूर्व॒म् पूर्वꣳ॒॒ हि हि पूर्व॑म् क्रि॒यते᳚ क्रि॒यते॒ पूर्वꣳ॒॒ हि हि पूर्व॑म् क्रि॒यते᳚ । \newline
3. पूर्व॑म् क्रि॒यते᳚ क्रि॒यते॒ पूर्व॒म् पूर्व॑म् क्रि॒यत॒ इतीति॑ क्रि॒यते॒ पूर्व॒म् पूर्व॑म् क्रि॒यत॒ इति॑ । \newline
4. क्रि॒यत॒ इतीति॑ क्रि॒यते᳚ क्रि॒यत॒ इत्यना॑दृ॒त्या ना॑दृ॒त्ये ति॑ क्रि॒यते᳚ क्रि॒यत॒ इत्यना॑दृत्य । \newline
5. इत्यना॑दृ॒त्या ना॑दृ॒त्ये तीत्यना॑दृत्य॒ तत् तदना॑दृ॒त्ये तीत्यना॑दृत्य॒ तत् । \newline
6. अना॑दृत्य॒ तत् तदना॑दृ॒त्या ना॑दृत्य॒ तच्छृ॒तस्य॑ शृ॒तस्य॒ तदना॑दृ॒त्या ना॑दृत्य॒ तच्छृ॒तस्य॑ । \newline
7. अना॑दृ॒त्येत्यना᳚ - दृ॒त्य॒ । \newline
8. तच्छृ॒तस्य॑ शृ॒तस्य॒ तत् तच्छृ॒तस्यै॒ वैव शृ॒तस्य॒ तत् तच्छृ॒तस्यै॒व । \newline
9. शृ॒तस्यै॒ वैव शृ॒तस्य॑ शृ॒तस्यै॒व पूर्व॑स्य॒ पूर्व॑स्यै॒व शृ॒तस्य॑ शृ॒तस्यै॒व पूर्व॑स्य । \newline
10. ए॒व पूर्व॑स्य॒ पूर्व॑स्यै॒ वैव पूर्व॒स्या वाव॒ पूर्व॑स्यै॒वैव पूर्व॒स्याव॑ । \newline
11. पूर्व॒स्या वाव॒ पूर्व॑स्य॒ पूर्व॒स्याव॑ द्येद् द्ये॒दव॒ पूर्व॑स्य॒ पूर्व॒स्याव॑ द्येत् । \newline
12. अव॑ द्येद् द्ये॒दवाव॑ द्येदिन्द्रि॒य मि॑न्द्रि॒यम् द्ये॒दवाव॑ द्येदिन्द्रि॒यम् । \newline
13. द्ये॒दि॒न्द्रि॒य मि॑न्द्रि॒यम् द्ये᳚द् द्येदिन्द्रि॒य मे॒वैवे न्द्रि॒यम् द्ये᳚द् द्येदिन्द्रि॒य मे॒व । \newline
14. इ॒न्द्रि॒य मे॒वैवे न्द्रि॒य मि॑न्द्रि॒य मे॒वास्मि॑न् नस्मिन् ने॒वे न्द्रि॒य मि॑न्द्रि॒य मे॒वास्मिन्न्॑ । \newline
15. ए॒वास्मि॑न् नस्मिन् ने॒वैवास्मि॑न् वी॒र्यं॑ ॅवी॒र्य॑ मस्मिन् ने॒वैवास्मि॑न् वी॒र्य᳚म् । \newline
16. अ॒स्मि॒न् वी॒र्यं॑ ॅवी॒र्य॑ मस्मिन् नस्मिन् वी॒र्यꣳ॑ श्रि॒त्वा श्रि॒त्वा वी॒र्य॑ मस्मिन् नस्मिन् वी॒र्यꣳ॑ श्रि॒त्वा । \newline
17. वी॒र्यꣳ॑ श्रि॒त्वा श्रि॒त्वा वी॒र्यं॑ ॅवी॒र्यꣳ॑ श्रि॒त्वा द॒द्ध्ना द॒द्ध्ना श्रि॒त्वा वी॒र्यं॑ ॅवी॒र्यꣳ॑ श्रि॒त्वा द॒द्ध्ना । \newline
18. श्रि॒त्वा द॒द्ध्ना द॒द्ध्ना श्रि॒त्वा श्रि॒त्वा द॒द्ध्नो परि॑ष्टा दु॒परि॑ष्टाद् द॒द्ध्ना श्रि॒त्वा श्रि॒त्वा द॒द्ध्नो परि॑ष्टात् । \newline
19. द॒द्ध्नो परि॑ष्टा दु॒परि॑ष्टाद् द॒द्ध्ना द॒द्ध्नो परि॑ष्टाद् धिनोति धिनो त्यु॒परि॑ष्टाद् द॒द्ध्ना द॒द्ध्नो परि॑ष्टाद् धिनोति । \newline
20. उ॒परि॑ष्टाद् धिनोति धिनो त्यु॒परि॑ष्टा दु॒परि॑ष्टाद् धिनोति यथापू॒र्वं ॅय॑थापू॒र्वम् धि॑नो त्यु॒परि॑ष्टा दु॒परि॑ष्टाद् धिनोति यथापू॒र्वम् । \newline
21. धि॒नो॒ति॒ य॒था॒पू॒र्वं ॅय॑थापू॒र्वम् धि॑नोति धिनोति यथापू॒र्व मुपोप॑ यथापू॒र्वम् धि॑नोति धिनोति यथापू॒र्व मुप॑ । \newline
22. य॒था॒पू॒र्व मुपोप॑ यथापू॒र्वं ॅय॑थापू॒र्व मुपै᳚त्ये॒ त्युप॑ यथापू॒र्वं ॅय॑थापू॒र्व मुपै॑ति । \newline
23. य॒था॒पू॒र्वमिति॑ यथा - पू॒र्वम् । \newline
24. उपै᳚त्ये॒ त्युपोपै॑ति॒ यद् यदे॒ त्युपोपै॑ति॒ यत् । \newline
25. ए॒ति॒ यद् यदे᳚त्येति॒ यत् पू॒तीकैः᳚ पू॒तीकै॒र् यदे᳚त्येति॒ यत् पू॒तीकैः᳚ । \newline
26. यत् पू॒तीकैः᳚ पू॒तीकै॒र् यद् यत् पू॒तीकै᳚र् वा वा पू॒तीकै॒र् यद् यत् पू॒तीकै᳚र् वा । \newline
27. पू॒तीकै᳚र् वा वा पू॒तीकैः᳚ पू॒तीकै᳚र् वा पर्णव॒ल्कैः प॑र्णव॒ल्कैर् वा॑ पू॒तीकैः᳚ पू॒तीकै᳚र् वा पर्णव॒ल्कैः । \newline
28. वा॒ प॒र्ण॒व॒ल्कैः प॑र्णव॒ल्कैर् वा॑ वा पर्णव॒ल्कैर् वा॑ वा पर्णव॒ल्कैर् वा॑ वा पर्णव॒ल्कैर् वा᳚ । \newline
29. प॒र्ण॒व॒ल्कैर् वा॑ वा पर्णव॒ल्कैः प॑र्णव॒ल्कैर् वा॑ ऽऽत॒ञ्च्या दा॑त॒ञ्च्याद् वा॑ पर्णव॒ल्कैः प॑र्णव॒ल्कैर् वा॑ ऽऽत॒ञ्च्यात् । \newline
30. प॒र्ण॒व॒ल्कैरिति॑ पर्ण - व॒ल्कैः । \newline
31. वा॒ ऽऽत॒ञ्च्या दा॑त॒ञ्च्याद् वा॑ वा ऽऽत॒ञ्च्याथ् सौ॒म्यꣳ सौ॒म्य मा॑त॒ञ्च्याद् वा॑ वा ऽऽत॒ञ्च्याथ् सौ॒म्यम् । \newline
32. आ॒त॒ञ्च्याथ् सौ॒म्यꣳ सौ॒म्य मा॑त॒ञ्च्या दा॑त॒ञ्च्याथ् सौ॒म्यम् तत् तथ् सौ॒म्य मा॑त॒ञ्च्या दा॑त॒ञ्च्याथ् सौ॒म्यम् तत् । \newline
33. आ॒त॒ञ्च्यादित्या᳚ - त॒ञ्च्यात् । \newline
34. सौ॒म्यम् तत् तथ् सौ॒म्यꣳ सौ॒म्यम् तद् यद् यत् तथ् सौ॒म्यꣳ सौ॒म्यम् तद् यत् । \newline
35. तद् यद् यत् तत् तद् यत् क्व॑लैः॒ क्व॑लै॒र् यत् तत् तद् यत् क्व॑लैः । \newline
36. यत् क्व॑लैः॒ क्व॑लै॒र् यद् यत् क्व॑लै राक्ष॒सꣳ रा᳚क्ष॒सम् क्व॑लै॒र् यद् यत् क्व॑लै राक्ष॒सम् । \newline
37. क्व॑लै राक्ष॒सꣳ रा᳚क्ष॒सम् क्व॑लैः॒ क्व॑लै राक्ष॒सम् तत् तद् रा᳚क्ष॒सम् क्व॑लैः॒ क्व॑लै राक्ष॒सम् तत् । \newline
38. रा॒क्ष॒सम् तत् तद् रा᳚क्ष॒सꣳ रा᳚क्ष॒सम् तद् यद् यत् तद् रा᳚क्ष॒सꣳ रा᳚क्ष॒सम् तद् यत् । \newline
39. तद् यद् यत् तत् तद् यत् त॑ण्डु॒लै स्त॑ण्डु॒लैर् यत् तत् तद् यत् त॑ण्डु॒लैः । \newline
40. यत् त॑ण्डु॒लै स्त॑ण्डु॒लैर् यद् यत् त॑ण्डु॒लैर् वै᳚श्वदे॒वं ॅवै᳚श्वदे॒वम् त॑ण्डु॒लैर् यद् यत् त॑ण्डु॒लैर् वै᳚श्वदे॒वम् । \newline
41. त॒ण्डु॒लैर् वै᳚श्वदे॒वं ॅवै᳚श्वदे॒वम् त॑ण्डु॒लै स्त॑ण्डु॒लैर् वै᳚श्वदे॒वम् तत् तद् वै᳚श्वदे॒वम् त॑ण्डु॒लै स्त॑ण्डु॒लैर् वै᳚श्वदे॒वम् तत् । \newline
42. वै॒श्व॒दे॒वम् तत् तद् वै᳚श्वदे॒वं ॅवै᳚श्वदे॒वम् तद् यद् यत् तद् वै᳚श्वदे॒वं ॅवै᳚श्वदे॒वम् तद् यत् । \newline
43. वै॒श्व॒दे॒वमिति॑ वैश्व - दे॒वम् । \newline
44. तद् यद् यत् तत् तद् यदा॒तञ्च॑नेना॒ तञ्च॑नेन॒ यत् तत् तद् यदा॒तञ्च॑नेन । \newline
45. यदा॒तञ्च॑नेना॒ तञ्च॑नेन॒ यद् यदा॒तञ्च॑नेन मानु॒षम् मा॑नु॒ष मा॒तञ्च॑नेन॒ यद् यदा॒तञ्च॑नेन मानु॒षम् । \newline
46. आ॒तञ्च॑नेन मानु॒षम् मा॑नु॒ष मा॒तञ्च॑नेना॒ तञ्च॑नेन मानु॒षम् तत् तन् मा॑नु॒ष मा॒तञ्च॑नेना॒ तञ्च॑नेन मानु॒षम् तत् । \newline
47. आ॒तञ्च॑ने॒नेत्या᳚ - तञ्च॑नेन । \newline
48. मा॒नु॒षम् तत् तन् मा॑नु॒षम् मा॑नु॒षम् तद् यद् यत् तन् मा॑नु॒षम् मा॑नु॒षम् तद् यत् । \newline
49. तद् यद् यत् तत् तद् यद् द॒द्ध्ना द॒द्ध्ना यत् तत् तद् यद् द॒द्ध्ना । \newline
50. यद् द॒द्ध्ना द॒द्ध्ना यद् यद् द॒द्ध्ना तत् तद् द॒द्ध्ना यद् यद् द॒द्ध्ना तत् । \newline
51. द॒द्ध्ना तत् तद् द॒द्ध्ना द॒द्ध्ना तथ् सेन्द्रꣳ॒॒ सेन्द्र॒म् तद् द॒द्ध्ना द॒द्ध्ना तथ् सेन्द्र᳚म् । \newline
52. तथ् सेन्द्रꣳ॒॒ सेन्द्र॒म् तत् तथ् सेन्द्र॑म् द॒द्ध्ना द॒द्ध्ना सेन्द्र॒म् तत् तथ् सेन्द्र॑म् द॒द्ध्ना । \newline
53. सेन्द्र॑म् द॒द्ध्ना द॒द्ध्ना सेन्द्रꣳ॒॒ सेन्द्र॑म् द॒द्ध्ना ऽऽत॑नक्ति तन॒क्त्या द॒द्ध्ना सेन्द्रꣳ॒॒ सेन्द्र॑म् द॒द्ध्ना ऽऽत॑नक्ति । \newline
54. सेन्द्र॒मिति॒ स - इ॒न्द्र॒म् । \newline
55. द॒द्ध्ना ऽऽत॑नक्ति तन॒क्त्या द॒द्ध्ना द॒द्ध्ना ऽऽत॑नक्ति सेन्द्र॒त्वाय॑ सेन्द्र॒त्वाय॑ तन॒क्त्या द॒द्ध्ना द॒द्ध्ना ऽऽत॑नक्ति सेन्द्र॒त्वाय॑ । \newline
56. आ त॑नक्ति तन॒क्त्या त॑नक्ति सेन्द्र॒त्वाय॑ सेन्द्र॒त्वाय॑ तन॒क्त्या त॑नक्ति सेन्द्र॒त्वाय॑ । \newline
57. त॒न॒क्ति॒ से॒न्द्र॒त्वाय॑ सेन्द्र॒त्वाय॑ तनक्ति तनक्ति सेन्द्र॒त्वाया᳚ ग्निहोत्रोच्छेष॒ण म॑ग्निहोत्रोच्छेष॒णꣳ से᳚न्द्र॒त्वाय॑ तनक्ति तनक्ति सेन्द्र॒त्वाया᳚ ग्निहोत्रोच्छेष॒णम् । \newline
\pagebreak
\markright{ TS 2.5.3.6  \hfill https://www.vedavms.in \hfill}
\addcontentsline{toc}{section}{ TS 2.5.3.6 }
\section*{ TS 2.5.3.6 }

\textbf{TS 2.5.3.6 } \newline
\textbf{Samhita Paata} \newline

सेन्द्र॒त्वाया᳚-ग्निहोत्रोच्छेष॒णम॒भ्या-त॑नक्ति य॒ज्ञ्स्य॒ संत॑त्या॒ इन्द्रो॑ वृ॒त्रꣳ ह॒त्वा परां᳚ परा॒वत॑-मगच्छ॒-दपा॑राध॒मिति॒ मन्य॑मान॒स्तं दे॒वताः॒ प्रैष॑मैच्छ॒न्थ् सो᳚ऽब्रवीत् प्र॒जाप॑ति॒र्यः प्र॑थ॒मो॑ऽनुवि॒न्दति॒ तस्य॑ प्रथ॒मं भा॑ग॒धेय॒मिति॒ तं पि॒तरोऽन्व॑विन्द॒न् तस्मा᳚त् पि॒तृभ्यः॑ पूर्वे॒द्युः क्रि॑यते॒ सो॑ऽमावा॒स्यां᳚ प्रत्याऽग॑च्छ॒त् तं दे॒वा अ॒भि सम॑गच्छन्ता॒मा वै नो॒-  [  ] \newline

\textbf{Pada Paata} \newline

से॒न्द्र॒त्वायेति॑ सेन्द्र - त्वाय॑ । अ॒ग्नि॒हो॒त्रो॒च्छे॒ष॒णमित्य॑ग्निहोत्र - उ॒च्छे॒ष॒णम् । अ॒भ्यात॑न॒क्तीत्य॑भि - आत॑नक्ति । य॒ज्ञ्स्य॑ । संत॑त्या॒ इति॒ सं-त॒त्यै॒ । इन्द्रः॑ । वृ॒त्रम् । ह॒त्वा । परा᳚म् । प॒रा॒वत॒मिति॑ परा - वत᳚म् । अ॒ग॒च्छ॒त् । अपेति॑ । अ॒रा॒ध॒म् । इति॑ । मन्य॑मानः । तम् । दे॒वताः᳚ । प्रैष॒मिति॑ प्र - एष᳚म् । ऐ॒च्छ॒न्न् । सः । अ॒ब्र॒वी॒त् । प्र॒जाप॑ति॒रिति॑ प्र॒जा - प॒तिः॒ । यः । प्र॒थ॒मः । अ॒नु॒वि॒न्दतीत्य॑नु - वि॒न्दति॑ । तस्य॑ । प्र॒थ॒मम् । भा॒ग॒धेय॒मिति॑ भाग - धेय᳚म् । इति॑ । तम् । पि॒तरः॑ । अन्विति॑ । अ॒वि॒न्द॒न्न् । तस्मा᳚त् । पि॒तृभ्य॒ इति॑ पि॒तृ - भ्यः॒ । पू॒र्वे॒द्युः । क्रि॒य॒ते॒ । सः । अ॒मा॒वा॒स्या॑मित्य॑मा - वा॒स्या᳚म् । प्रति॑ । एति॑ । अ॒ग॒च्छ॒त् । तम् । दे॒वाः । अ॒भि । समिति॑ । अ॒ग॒च्छ॒न्त॒ । अ॒मा । वै । नः॒ ।  \newline


\textbf{Krama Paata} \newline

से॒न्द्र॒त्वाया᳚ग्निहोत्रोच्छेष॒णम् । से॒न्द्र॒त्वायेति॑ सेन्द्र - त्वाय॑ । अ॒ग्नि॒हो॒त्रो॒च्छे॒ष॒णम॒भ्यात॑नक्ति । अ॒ग्नि॒हो॒त्रो॒च्छे॒ष॒णमित्य॑ग्निहोत्र - उ॒च्छे॒ष॒णम् । अ॒भ्यात॑नक्ति य॒ज्ञ्स्य॑ । अ॒भ्यात॑न॒क्तीत्य॑भि - आत॑नक्ति । य॒ज्ञ्स्य॒ सन्त॑त्यै । सन्त॑त्या॒ इन्द्रः॑ । सन्त॑त्या॒ इति॒ सम् - त॒त्यै॒ । इन्द्रो॑ वृ॒त्रम् । वृ॒त्रꣳ ह॒त्वा । ह॒त्वा परा᳚म् । परा᳚म् परा॒वत᳚म् । प॒रा॒वत॑मगच्छत् । प॒रा॒वत॒मिति॑ परा - वत᳚म् । अ॒ग॒च्छ॒दप॑ । अपा॑राधम् । अ॒रा॒ध॒मिति॑ । इति॒ मन्य॑मानः । मन्य॑मान॒स्तम् । तम् दे॒वताः᳚ । दे॒वताः॒ प्रैष᳚म् । प्रैष॑मैच्छन्न् । प्रैष॒मिति॑ प्र - एष᳚म् । ऐ॒च्छ॒न्थ् सः । सो᳚ ऽब्रवीत् । अ॒ब्र॒वी॒त् प्र॒जाप॑तिः । प्र॒जाप॑ति॒र् यः । प्र॒जाप॑ति॒रिति॑ प्र॒जा - प॒तिः॒ । यः प्र॑थ॒मः । प्र॒थ॒मो॑ ऽनुवि॒न्दति॑ । अ॒नु॒वि॒न्दति॒ तस्य॑ । अ॒नु॒वि॒न्दतीत्य॑नु - वि॒न्दति॑ । तस्य॑ प्रथ॒मम् । प्र॒थ॒मम् भा॑ग॒धेय᳚म् । भा॒ग॒धेय॒मिति॑ । भा॒ग॒धेय॒मिति॑ भाग - धेय᳚म् । इति॒ तम् । तम् पि॒तरः॑ । पि॒तरो ऽनु॑ । अन्व॑विन्दन्न् । अ॒वि॒न्द॒न् तस्मा᳚त् । तस्मा᳚त् पि॒तृभ्यः॑ । पि॒तृभ्यः॑ पूर्वे॒द्युः । पि॒तृभ्य॒ इति॑ पि॒तृ - भ्यः॒ । पू॒र्वे॒द्युः क्रि॑यते । क्रि॒य॒ते॒ सः । सो॑ ऽमावा॒स्या᳚म् । अ॒मा॒वा॒स्या᳚म् प्रति॑ । अ॒मा॒वा॒स्या॑मित्य॑मा - वा॒स्या᳚म् । प्रत्या । आ ऽग॑च्छत् । अ॒ग॒च्छ॒त् तम् । तम् दे॒वाः । दे॒वा अ॒भि । अ॒भि सम् । सम॑गच्छन्त । अ॒ग॒च्छ॒न्ता॒मा । अ॒मा वै । वै नः॑ ( ) । नो॒ऽद्य \newline

\textbf{Jatai Paata} \newline

1. से॒न्द्र॒त्वाया᳚ ग्निहोत्रोच्छेष॒ण म॑ग्निहोत्रोच्छेष॒णꣳ से᳚न्द्र॒त्वाय॑ सेन्द्र॒त्वाया᳚ ग्निहोत्रोच्छेष॒णम् । \newline
2. से॒न्द्र॒त्वायेति॑ सेन्द्र - त्वाय॑ । \newline
3. अ॒ग्नि॒हो॒त्रो॒च्छे॒ष॒ण म॒भ्यात॑नक् त्य॒भ्यात॑नक् त्यग्निहोत्रोच्छेष॒ण म॑ग्निहोत्रोच्छेष॒ण म॒भ्यात॑नक्ति । \newline
4. अ॒ग्नि॒हो॒त्रो॒च्छे॒ष॒णमित्य॑ग्निहोत्र - उ॒च्छे॒ष॒णम् । \newline
5. अ॒भ्यात॑नक्ति य॒ज्ञ्स्य॑ य॒ज्ञ्स्या॒ भ्यात॑नक् त्य॒भ्यात॑नक्ति य॒ज्ञ्स्य॑ । \newline
6. अ॒भ्यात॑न॒क्तीत्य॑भि - आत॑नक्ति । \newline
7. य॒ज्ञ्स्य॒ सन्त॑त्यै॒ सन्त॑त्यै य॒ज्ञ्स्य॑ य॒ज्ञ्स्य॒ सन्त॑त्यै । \newline
8. सन्त॑त्या॒ इन्द्र॒ इन्द्रः॒ सन्त॑त्यै॒ सन्त॑त्या॒ इन्द्रः॑ । \newline
9. सन्त॑त्या॒ इति॒ सं - त॒त्यै॒ । \newline
10. इन्द्रो॑ वृ॒त्रं ॅवृ॒त्र मिन्द्र॒ इन्द्रो॑ वृ॒त्रम् । \newline
11. वृ॒त्रꣳ ह॒त्वा ह॒त्वा वृ॒त्रं ॅवृ॒त्रꣳ ह॒त्वा । \newline
12. ह॒त्वा परा॒म् पराꣳ॑ ह॒त्वा ह॒त्वा परा᳚म् । \newline
13. परा᳚म् परा॒वत॑म् परा॒वत॒म् परा॒म् परा᳚म् परा॒वत᳚म् । \newline
14. प॒रा॒वत॑ मगच्छ दगच्छत् परा॒वत॑म् परा॒वत॑ मगच्छत् । \newline
15. प॒रा॒वत॒मिति॑ परा - वत᳚म् । \newline
16. अ॒ग॒च्छ॒ दपापा॑गच्छ दगच्छ॒दप॑ । \newline
17. अपा॑राध मराध॒ मपापा॑राधम् । \newline
18. अ॒रा॒ध॒ मितीत्य॑राध मराध॒ मिति॑ । \newline
19. इति॒ मन्य॑मानो॒ मन्य॑मान॒ इतीति॒ मन्य॑मानः । \newline
20. मन्य॑मान॒ स्तम् तम् मन्य॑मानो॒ मन्य॑मान॒ स्तम् । \newline
21. तम् दे॒वता॑ दे॒वता॒ स्तम् तम् दे॒वताः᳚ । \newline
22. दे॒वताः॒ प्रैष॒म् प्रैष॑म् दे॒वता॑ दे॒वताः॒ प्रैष᳚म् । \newline
23. प्रैष॑ मैच्छन् नैच्छ॒न् प्रैष॒म् प्रैष॑ मैच्छन्न् । \newline
24. प्रैष॒मिति॑ प्र - एष᳚म् । \newline
25. ऐ॒च्छ॒न् थ्स स ऐ᳚च्छन् नैच्छ॒न् थ्सः । \newline
26. सो᳚ ऽब्रवी दब्रवी॒थ् स सो᳚ ऽब्रवीत् । \newline
27. अ॒ब्र॒वी॒त् प्र॒जाप॑तिः प्र॒जाप॑ति रब्रवी दब्रवीत् प्र॒जाप॑तिः । \newline
28. प्र॒जाप॑ति॒र् यो यः प्र॒जाप॑तिः प्र॒जाप॑ति॒र् यः । \newline
29. प्र॒जाप॑ति॒रिति॑ प्र॒जा - प॒तिः॒ । \newline
30. यः प्र॑थ॒मः प्र॑थ॒मो यो यः प्र॑थ॒मः । \newline
31. प्र॒थ॒मो॑ ऽनुवि॒न्द त्य॑नुवि॒न्दति॑ प्रथ॒मः प्र॑थ॒मो॑ ऽनुवि॒न्दति॑ । \newline
32. अ॒नु॒वि॒न्दति॒ तस्य॒ तस्या॑नुवि॒न्द त्य॑नुवि॒न्दति॒ तस्य॑ । \newline
33. अ॒नु॒वि॒न्दतीत्य॑नु - वि॒न्दति॑ । \newline
34. तस्य॑ प्रथ॒मम् प्र॑थ॒मम् तस्य॒ तस्य॑ प्रथ॒मम् । \newline
35. प्र॒थ॒मम् भा॑ग॒धेय॑म् भाग॒धेय॑म् प्रथ॒मम् प्र॑थ॒मम् भा॑ग॒धेय᳚म् । \newline
36. भा॒ग॒धेय॒ मितीति॑ भाग॒धेय॑म् भाग॒धेय॒ मिति॑ । \newline
37. भा॒ग॒धेय॒मिति॑ भाग - धेय᳚म् । \newline
38. इति॒ तम् त मितीति॒ तम् । \newline
39. तम् पि॒तरः॑ पि॒तर॒ स्तम् तम् पि॒तरः॑ । \newline
40. पि॒तरो ऽन्वनु॑ पि॒तरः॑ पि॒तरो ऽनु॑ । \newline
41. अन्व॑विन्दन् नविन्द॒न् नन्वन्व॑विन्दन्न् । \newline
42. अ॒वि॒न्द॒न् तस्मा॒त् तस्मा॑ दविन्दन् नविन्द॒न् तस्मा᳚त् । \newline
43. तस्मा᳚त् पि॒तृभ्यः॑ पि॒तृभ्य॒ स्तस्मा॒त् तस्मा᳚त् पि॒तृभ्यः॑ । \newline
44. पि॒तृभ्यः॑ पूर्वे॒द्युः पू᳚र्वे॒द्युः पि॒तृभ्यः॑ पि॒तृभ्यः॑ पूर्वे॒द्युः । \newline
45. पि॒तृभ्य॒ इति॑ पि॒तृ - भ्यः॒ । \newline
46. पू॒र्वे॒द्युः क्रि॑यते क्रियते पूर्वे॒द्युः पू᳚र्वे॒द्युः क्रि॑यते । \newline
47. क्रि॒य॒ते॒ स स क्रि॑यते क्रियते॒ सः । \newline
48. सो॑ ऽमावा॒स्या॑ ममावा॒स्याꣳ॑ स सो॑ ऽमावा॒स्या᳚म् । \newline
49. अ॒मा॒वा॒स्या᳚म् प्रति॒ प्रत्य॑मावा॒स्या॑ ममावा॒स्या᳚म् प्रति॑ । \newline
50. अ॒मा॒वा॒स्या॑मित्य॑मा - वा॒स्या᳚म् । \newline
51. प्रत्या प्रति॒ प्रत्या । \newline
52. आ ऽग॑च्छ दगच्छ॒दा ऽग॑च्छत् । \newline
53. अ॒ग॒च्छ॒त् तम् त म॑गच्छ दगच्छ॒त् तम् । \newline
54. तम् दे॒वा दे॒वा स्तम् तम् दे॒वाः । \newline
55. दे॒वा अ॒भ्य॑भि दे॒वा दे॒वा अ॒भि । \newline
56. अ॒भि सꣳ स म॒भ्य॑भि सम् । \newline
57. स म॑गच्छन्ता गच्छन्त॒ सꣳ स म॑गच्छन्त । \newline
58. अ॒ग॒च्छ॒न्ता॒मा ऽमा ऽग॑च्छन्ता गच्छन्ता॒मा । \newline
59. अ॒मा वै वा अ॒मा ऽमा वै । \newline
60. वै नो॑ नो॒ वै वै नः॑ । \newline
61. नो॒ ऽद्याद्य नो॑ नो॒ ऽद्य । \newline

\textbf{Ghana Paata } \newline

1. से॒न्द्र॒त्वाया᳚ ग्निहोत्रोच्छेष॒ण म॑ग्निहोत्रोच्छेष॒णꣳ से᳚न्द्र॒त्वाय॑ सेन्द्र॒त्वाया᳚ ग्निहोत्रोच्छेष॒ण म॒भ्यात॑न क्त्य॒भ्यात॑नक् त्यग्निहोत्रोच्छेष॒णꣳ से᳚न्द्र॒त्वाय॑ सेन्द्र॒त्वाया᳚ ग्निहोत्रोच्छेष॒ण म॒भ्यात॑नक्ति । \newline
2. से॒न्द्र॒त्वायेति॑ सेन्द्र - त्वाय॑ । \newline
3. अ॒ग्नि॒हो॒त्रो॒च्छे॒ष॒ण म॒भ्यात॑न क्त्य॒भ्यात॑न क्त्यग्निहोत्रोच्छेष॒ण म॑ग्निहोत्रोच्छेष॒ण म॒भ्यात॑नक्ति य॒ज्ञ्स्य॑ य॒ज्ञ्स्या॒भ्यात॑न क्त्यग्निहोत्रोच्छेष॒ण म॑ग्निहोत्रोच्छेष॒ण म॒भ्यात॑नक्ति य॒ज्ञ्स्य॑ । \newline
4. अ॒ग्नि॒हो॒त्रो॒च्छे॒ष॒णमित्य॑ग्निहोत्र - उ॒च्छे॒ष॒णम् । \newline
5. अ॒भ्यात॑नक्ति य॒ज्ञ्स्य॑ य॒ज्ञ्स्या॒भ्यात॑न क्त्य॒भ्यात॑नक्ति य॒ज्ञ्स्य॒ सन्त॑त्यै॒ सन्त॑त्यै य॒ज्ञ्स्या॒भ्यात॑न क्त्य॒भ्यात॑नक्ति य॒ज्ञ्स्य॒ सन्त॑त्यै । \newline
6. अ॒भ्यात॑न॒क्तीत्य॑भि - आत॑नक्ति । \newline
7. य॒ज्ञ्स्य॒ सन्त॑त्यै॒ सन्त॑त्यै य॒ज्ञ्स्य॑ य॒ज्ञ्स्य॒ सन्त॑त्या॒ इन्द्र॒ इन्द्रः॒ सन्त॑त्यै य॒ज्ञ्स्य॑ य॒ज्ञ्स्य॒ सन्त॑त्या॒ इन्द्रः॑ । \newline
8. सन्त॑त्या॒ इन्द्र॒ इन्द्रः॒ सन्त॑त्यै॒ सन्त॑त्या॒ इन्द्रो॑ वृ॒त्रं ॅवृ॒त्र मिन्द्रः॒ सन्त॑त्यै॒ सन्त॑त्या॒ इन्द्रो॑ वृ॒त्रम् । \newline
9. सन्त॑त्या॒ इति॒ सं - त॒त्यै॒ । \newline
10. इन्द्रो॑ वृ॒त्रं ॅवृ॒त्र मिन्द्र॒ इन्द्रो॑ वृ॒त्रꣳ ह॒त्वा ह॒त्वा वृ॒त्र मिन्द्र॒ इन्द्रो॑ वृ॒त्रꣳ ह॒त्वा । \newline
11. वृ॒त्रꣳ ह॒त्वा ह॒त्वा वृ॒त्रं ॅवृ॒त्रꣳ ह॒त्वा परा॒म् पराꣳ॑ ह॒त्वा वृ॒त्रं ॅवृ॒त्रꣳ ह॒त्वा परा᳚म् । \newline
12. ह॒त्वा परा॒म् पराꣳ॑ ह॒त्वा ह॒त्वा परा᳚म् परा॒वत॑म् परा॒वत॒म् पराꣳ॑ ह॒त्वा ह॒त्वा परा᳚म् परा॒वत᳚म् । \newline
13. परा᳚म् परा॒वत॑म् परा॒वत॒म् परा॒म् परा᳚म् परा॒वत॑ मगच्छ दगच्छत् परा॒वत॒म् परा॒म् परा᳚म् परा॒वत॑ मगच्छत् । \newline
14. प॒रा॒वत॑ मगच्छदगच्छत् परा॒वत॑म् परा॒वत॑ मगच्छ॒ दपापा॑गच्छत् परा॒वत॑म् परा॒वत॑ मगच्छ॒दप॑ । \newline
15. प॒रा॒वत॒मिति॑ परा - वत᳚म् । \newline
16. अ॒ग॒च्छ॒ दपापा॑गच्छ दगच्छ॒ दपा॑राध मराध॒ मपा॑गच्छ दगच्छ॒ दपा॑राधम् । \newline
17. अपा॑राध मराध॒ मपापा॑राध॒ मितीत्य॑राध॒ मपापा॑राध॒ मिति॑ । \newline
18. अ॒रा॒ध॒ मितीत्य॑राध मराध॒ मिति॒ मन्य॑मानो॒ मन्य॑मान॒ इत्य॑राध मराध॒ मिति॒ मन्य॑मानः । \newline
19. इति॒ मन्य॑मानो॒ मन्य॑मान॒ इतीति॒ मन्य॑मान॒ स्तम् तम् मन्य॑मान॒ इतीति॒ मन्य॑मान॒ स्तम् । \newline
20. मन्य॑मान॒ स्तम् तम् मन्य॑मानो॒ मन्य॑मान॒ स्तम् दे॒वता॑ दे॒वता॒ स्तम् मन्य॑मानो॒ मन्य॑मान॒ स्तम् दे॒वताः᳚ । \newline
21. तम् दे॒वता॑ दे॒वता॒ स्तम् तम् दे॒वताः॒ प्रैष॒म् प्रैष॑म् दे॒वता॒ स्तम् तम् दे॒वताः॒ प्रैष᳚म् । \newline
22. दे॒वताः॒ प्रैष॒म् प्रैष॑म् दे॒वता॑ दे॒वताः॒ प्रैष॑ मैच्छन् नैच्छ॒न् प्रैष॑म् दे॒वता॑ दे॒वताः॒ प्रैष॑ मैच्छन्न् । \newline
23. प्रैष॑ मैच्छन् नैच्छ॒न् प्रैष॒म् प्रैष॑ मैच्छ॒न् थ्स स ऐ᳚च्छ॒न् प्रैष॒म् प्रैष॑ मैच्छ॒न् थ्सः । \newline
24. प्रैष॒मिति॑ प्र - एष᳚म् । \newline
25. ऐ॒च्छ॒न् थ्स स ऐ᳚च्छन् नैच्छ॒न् थ्सो᳚ ऽब्रवी दब्रवी॒थ् स ऐ᳚च्छन् नैच्छ॒न् थ्सो᳚ ऽब्रवीत् । \newline
26. सो᳚ ऽब्रवी दब्रवी॒थ् स सो᳚ ऽब्रवीत् प्र॒जाप॑तिः प्र॒जाप॑ति रब्रवी॒थ् स सो᳚ ऽब्रवीत् प्र॒जाप॑तिः । \newline
27. अ॒ब्र॒वी॒त् प्र॒जाप॑तिः प्र॒जाप॑ति रब्रवी दब्रवीत् प्र॒जाप॑ति॒र् यो यः प्र॒जाप॑ति रब्रवी दब्रवीत् प्र॒जाप॑ति॒र् यः । \newline
28. प्र॒जाप॑ति॒र् यो यः प्र॒जाप॑तिः प्र॒जाप॑ति॒र् यः प्र॑थ॒मः प्र॑थ॒मो यः प्र॒जाप॑तिः प्र॒जाप॑ति॒र् यः प्र॑थ॒मः । \newline
29. प्र॒जाप॑ति॒रिति॑ प्र॒जा - प॒तिः॒ । \newline
30. यः प्र॑थ॒मः प्र॑थ॒मो यो यः प्र॑थ॒मो॑ ऽनुवि॒न्द त्य॑नुवि॒न्दति॑ प्रथ॒मो यो यः प्र॑थ॒मो॑ ऽनुवि॒न्दति॑ । \newline
31. प्र॒थ॒मो॑ ऽनुवि॒न्द त्य॑नुवि॒न्दति॑ प्रथ॒मः प्र॑थ॒मो॑ ऽनुवि॒न्दति॒ तस्य॒ तस्या॑नुवि॒न्दति॑ प्रथ॒मः प्र॑थ॒मो॑ ऽनुवि॒न्दति॒ तस्य॑ । \newline
32. अ॒नु॒वि॒न्दति॒ तस्य॒ तस्या॑नुवि॒न्द त्य॑नुवि॒न्दति॒ तस्य॑ प्रथ॒मम् प्र॑थ॒मम् तस्या॑नुवि॒न्द त्य॑नुवि॒न्दति॒ तस्य॑ प्रथ॒मम् । \newline
33. अ॒नु॒वि॒न्दतीत्य॑नु - वि॒न्दति॑ । \newline
34. तस्य॑ प्रथ॒मम् प्र॑थ॒मम् तस्य॒ तस्य॑ प्रथ॒मम् भा॑ग॒धेय॑म् भाग॒धेय॑म् प्रथ॒मम् तस्य॒ तस्य॑ प्रथ॒मम् भा॑ग॒धेय᳚म् । \newline
35. प्र॒थ॒मम् भा॑ग॒धेय॑म् भाग॒धेय॑म् प्रथ॒मम् प्र॑थ॒मम् भा॑ग॒धेय॒ मितीति॑ भाग॒धेय॑म् प्रथ॒मम् प्र॑थ॒मम् भा॑ग॒धेय॒ मिति॑ । \newline
36. भा॒ग॒धेय॒ मितीति॑ भाग॒धेय॑म् भाग॒धेय॒ मिति॒ तम् त मिति॑ भाग॒धेय॑म् भाग॒धेय॒ मिति॒ तम् । \newline
37. भा॒ग॒धेय॒मिति॑ भाग - धेय᳚म् । \newline
38. इति॒ तम् त मितीति॒ तम् पि॒तरः॑ पि॒तर॒ स्त मितीति॒ तम् पि॒तरः॑ । \newline
39. तम् पि॒तरः॑ पि॒तर॒ स्तम् तम् पि॒तरो ऽन्वनु॑ पि॒तर॒ स्तम् तम् पि॒तरो ऽनु॑ । \newline
40. पि॒तरो ऽन्वनु॑ पि॒तरः॑ पि॒तरो ऽन्व॑विन्दन् नविन्द॒न् ननु॑ पि॒तरः॑ पि॒तरो ऽन्व॑विन्दन्न् । \newline
41. अन्व॑विन्दन् नविन्द॒न् नन्वन्व॑विन्द॒न् तस्मा॒त् तस्मा॑ दविन्द॒न् नन्वन्व॑विन्द॒न् तस्मा᳚त् । \newline
42. अ॒वि॒न्द॒न् तस्मा॒त् तस्मा॑ दविन्दन् नविन्द॒न् तस्मा᳚त् पि॒तृभ्यः॑ पि॒तृभ्य॒ स्तस्मा॑ दविन्दन् नविन्द॒न् तस्मा᳚त् पि॒तृभ्यः॑ । \newline
43. तस्मा᳚त् पि॒तृभ्यः॑ पि॒तृभ्य॒ स्तस्मा॒त् तस्मा᳚त् पि॒तृभ्यः॑ पूर्वे॒द्युः पू᳚र्वे॒द्युः पि॒तृभ्य॒ स्तस्मा॒त् तस्मा᳚त् पि॒तृभ्यः॑ पूर्वे॒द्युः । \newline
44. पि॒तृभ्यः॑ पूर्वे॒द्युः पू᳚र्वे॒द्युः पि॒तृभ्यः॑ पि॒तृभ्यः॑ पूर्वे॒द्युः क्रि॑यते क्रियते पूर्वे॒द्युः पि॒तृभ्यः॑ पि॒तृभ्यः॑ पूर्वे॒द्युः क्रि॑यते । \newline
45. पि॒तृभ्य॒ इति॑ पि॒तृ - भ्यः॒ । \newline
46. पू॒र्वे॒द्युः क्रि॑यते क्रियते पूर्वे॒द्युः पू᳚र्वे॒द्युः क्रि॑यते॒ स स क्रि॑यते पूर्वे॒द्युः पू᳚र्वे॒द्युः क्रि॑यते॒ सः । \newline
47. क्रि॒य॒ते॒ स स क्रि॑यते क्रियते॒ सो॑ ऽमावा॒स्या॑ ममावा॒स्याꣳ॑ स क्रि॑यते क्रियते॒ सो॑ ऽमावा॒स्या᳚म् । \newline
48. सो॑ ऽमावा॒स्या॑ ममावा॒स्याꣳ॑ स सो॑ ऽमावा॒स्या᳚म् प्रति॒ प्रत्य॑मावा॒स्याꣳ॑ स सो॑ ऽमावा॒स्या᳚म् प्रति॑ । \newline
49. अ॒मा॒वा॒स्या᳚म् प्रति॒ प्रत्य॑मावा॒स्या॑ ममावा॒स्या᳚म् प्रत्या प्रत्य॑मावा॒स्या॑ ममावा॒स्या᳚म् प्रत्या । \newline
50. अ॒मा॒वा॒स्या॑मित्य॑मा - वा॒स्या᳚म् । \newline
51. प्रत्या प्रति॒ प्रत्या ऽग॑च्छ दगच्छ॒दा प्रति॒ प्रत्या ऽग॑च्छत् । \newline
52. आ ऽग॑च्छ दगच्छ॒दा ऽग॑च्छ॒त् तम् त म॑गच्छ॒दा ऽग॑च्छ॒त् तम् । \newline
53. अ॒ग॒च्छ॒त् तम् त म॑गच्छ दगच्छ॒त् तम् दे॒वा दे॒वास्त म॑गच्छ दगच्छ॒त् तम् दे॒वाः । \newline
54. तम् दे॒वा दे॒वा स्तम् तम् दे॒वा अ॒भ्य॑भि दे॒वा स्तम् तम् दे॒वा अ॒भि । \newline
55. दे॒वा अ॒भ्य॑भि दे॒वा दे॒वा अ॒भि सꣳ स म॒भि दे॒वा दे॒वा अ॒भि सम् । \newline
56. अ॒भि सꣳ स म॒भ्य॑भि स म॑गच्छन्ता गच्छन्त॒ स म॒भ्य॑भि स म॑गच्छन्त । \newline
57. स म॑गच्छन्ता गच्छन्त॒ सꣳ स म॑गच्छन्ता॒मा ऽमा ऽग॑च्छन्त॒ सꣳ स म॑गच्छन्ता॒मा । \newline
58. अ॒ग॒च्छ॒न्ता॒मा ऽमा ऽग॑च्छन्ता गच्छन्ता॒मा वै वा अ॒मा ऽग॑च्छन्ता गच्छन्ता॒मा वै । \newline
59. अ॒मा वै वा अ॒मा ऽमा वै नो॑ नो॒ वा अ॒मा ऽमा वै नः॑ । \newline
60. वै नो॑ नो॒ वै वै नो॒ ऽद्याद्य नो॒ वै वै नो॒ ऽद्य । \newline
61. नो॒ ऽद्याद्य नो॑ नो॒ ऽद्य वसु॒ वस्व॒द्य नो॑ नो॒ ऽद्य वसु॑ । \newline
\pagebreak
\markright{ TS 2.5.3.7  \hfill https://www.vedavms.in \hfill}
\addcontentsline{toc}{section}{ TS 2.5.3.7 }
\section*{ TS 2.5.3.7 }

\textbf{TS 2.5.3.7 } \newline
\textbf{Samhita Paata} \newline

ऽद्य वसु॑ वस॒तीतीन्द्रो॒ हि दे॒वानां॒ ॅवसु॒ तद॑मावा॒स्या॑या अमावास्य॒त्वं ब्र॑ह्मवा॒दिनो॑ वदन्ति किंदेव॒त्यꣳ॑ सांना॒य्यमिति॑ वैश्वदे॒वमिति॑ ब्रूया॒द्-विश्वे॒ हि तद्दे॒वा भा॑ग॒धेय॑म॒भि स॒मग॑च्छ॒न्तेत्यथो॒ खल्वै॒न्द्रमित्ये॒व ब्रू॑या॒दिन्द्रं॒ ॅवाव ते तद्-भि॑ष॒ज्यन्तो॒ऽभि सम॑गच्छ॒न्तेति॑ ॥ \newline

\textbf{Pada Paata} \newline

अ॒द्य । वसु॑ । व॒स॒ति॒ । इति॑ । इन्द्रः॑ । हि । दे॒वाना᳚म् । वसु॑ । तत् । अ॒मा॒वा॒स्या॑या॒ इत्य॑मा - वा॒स्या॑याः । अ॒मा॒वा॒स्य॒त्वमित्य॑मावास्य - त्वम् । ब्र॒ह्म॒वा॒दिन॒ इति॑ ब्रह्म-वा॒दिनः॑ । व॒द॒न्ति॒ । किं॒दे॒व॒त्य॑मिति॑ किं - दे॒व॒त्य᳚म् । सा॒नां॒य्यमिति॑ सां - ना॒य्यम् । इति॑ । वै॒श्व॒दे॒वमिति॑ वैश्व - दे॒वम् । इति॑ । ब्रू॒या॒त् । विश्वे᳚ । हि । तत् । दे॒वाः । भा॒ग॒धेय॒मिति॑ भाग - धेय᳚म् । अ॒भीति॑ । स॒मग॑च्छ॒न्तेति॑ सं - अग॑च्छन्त । इति॑ । अथो॒ इति॑ । खलु॑ । ऐ॒न्द्रम् । इति॑ । ए॒व । ब्रू॒या॒त् । इन्द्र᳚म् । वाव । ते । तत् । भि॒ष॒ज्यन्तः॑ । अ॒भि । समिति॑ । अ॒ग॒च्छ॒न्त॒ । इति॑ ॥  \newline


\textbf{Krama Paata} \newline

अ॒द्य वसु॑ । वसु॑ वसति । व॒स॒तीति॑ । इतीन्द्रः॑ । इन्द्रो॒ हि । हि दे॒वाना᳚म् । दे॒वाना॒म् ॅवसु॑ । वसु॒ तत् । तद॑मावा॒स्या॑याः । अ॒मा॒वा॒स्या॑या अमावास्य॒त्वम् । अ॒मा॒वा॒स्या॑या॒ इत्य॑मा - वा॒स्या॑याः । अ॒मा॒वा॒स्य॒त्वम् ब्र॑ह्मवा॒दिनः॑ । अ॒मा॒वा॒स्य॒त्वमित्य॑मावास्य - त्वम् । ब्र॒ह्म॒वा॒दिनो॑ वदन्ति । ब्र॒ह्म॒वा॒दिन॒ इति॑ ब्रह्म - वा॒दिनः॑ । व॒द॒न्ति॒ कि॒म्दे॒व॒त्य᳚म् । कि॒म्दे॒व॒त्यꣳ॑ सान्ना॒य्यम् । कि॒म्दे॒व॒त्य॑मिति॑ किम् - दे॒व॒त्य᳚म् । सा॒न्ना॒य्यमिति॑ । सा॒न्ना॒य्यमिति॑ साम् - ना॒य्यम् । इति॑ वैश्वदे॒वम् । वै॒श्व॒दे॒वमिति॑ । वै॒श्व॒दे॒वमिति॑ वैश्व - दे॒वम् । इति॑ ब्रूयात् । ब्रू॒या॒द् विश्वे᳚ । विश्वे॒ हि । हि तत् । तद् दे॒वाः । दे॒वा भा॑ग॒धेय᳚म् । भा॒ग॒धेय॑म॒भि । भा॒ग॒धेय॒मिति॑ भाग - धेय᳚म् । अ॒भि स॒मग॑च्छन्त । स॒मग॑च्छ॒न्तेति॑ । स॒मग॑च्छ॒न्तेति॑ सम् - अग॑च्छन्त । इत्यथो᳚ । अथो॒ खलु॑ । अथो॒ इत्यथो᳚ । खल्वै॒न्द्रम् । ऐ॒न्द्रमिति॑ । इत्ये॒व । ए॒व ब्रू॑यात् । ब्रू॒या॒दिन्द्र᳚म् । इन्द्र॒म् ॅवाव । वाव ते । ते तत् । तद् भि॑ष॒ज्यन्तः॑ । भि॒ष॒ज्यन्तो॒ऽभि । अ॒भि सम् । सम॑गच्छन्त । अ॒ग॒च्छ॒न्तेति॑ । इतीतीति॑ । \newline

\textbf{Jatai Paata} \newline

1. अ॒द्य वसु॒ वस्व॒द्याद्य वसु॑ । \newline
2. वसु॑ वसति वसति॒ वसु॒ वसु॑ वसति । \newline
3. व॒स॒तीतीति॑ वसति वस॒तीति॑ । \newline
4. इतीन्द्र॒ इन्द्र॒ इतीतीन्द्रः॑ । \newline
5. इन्द्रो॒ हि हीन्द्र॒ इन्द्रो॒ हि । \newline
6. हि दे॒वाना᳚म् दे॒वानाꣳ॒॒ हि हि दे॒वाना᳚म् । \newline
7. दे॒वानां॒ ॅवसु॒ वसु॑ दे॒वाना᳚म् दे॒वानां॒ ॅवसु॑ । \newline
8. वसु॒ तत् तद् वसु॒ वसु॒ तत् । \newline
9. तद॑मावा॒स्या॑या अमावा॒स्या॑या॒स्तत् तद॑मावा॒स्या॑याः । \newline
10. अ॒मा॒वा॒स्या॑या अमावास्य॒त्व म॑मावास्य॒त्व म॑मावा॒स्या॑या अमावा॒स्या॑या अमावास्य॒त्वम् । \newline
11. अ॒मा॒वा॒स्या॑या॒ इत्य॑मा - वा॒स्या॑याः । \newline
12. अ॒मा॒वा॒स्य॒त्वम् ब्र॑ह्मवा॒दिनो᳚ ब्रह्मवा॒दिनो॑ ऽमावास्य॒त्व म॑मावास्य॒त्वम् ब्र॑ह्मवा॒दिनः॑ । \newline
13. अ॒मा॒वा॒स्य॒त्वमित्य॑मावास्य - त्वम् । \newline
14. ब्र॒ह्म॒वा॒दिनो॑ वदन्ति वदन्ति ब्रह्मवा॒दिनो᳚ ब्रह्मवा॒दिनो॑ वदन्ति । \newline
15. ब्र॒ह्म॒वा॒दिन॒ इति॑ ब्रह्म - वा॒दिनः॑ । \newline
16. व॒द॒न्ति॒ कि॒न्दे॒व॒त्य॑म् किन्देव॒त्यं॑ ॅवदन्ति वदन्ति किन्देव॒त्य᳚म् । \newline
17. कि॒न्दे॒व॒त्यꣳ॑ सान्ना॒य्यꣳ सा᳚न्ना॒य्यम् कि॑न्देव॒त्य॑म् किन्देव॒त्यꣳ॑ सान्ना॒य्यम् । \newline
18. कि॒न्दे॒व॒त्य॑मिति॑ किं - दे॒व॒त्य᳚म् । \newline
19. सा॒न्ना॒य्य मितीति॑ सान्ना॒य्यꣳ सा᳚न्ना॒य्य मिति॑ । \newline
20. सा॒न्ना॒य्यमिति॑ सां - ना॒य्यम् । \newline
21. इति॑ वैश्वदे॒वं ॅवै᳚श्वदे॒व मितीति॑ वैश्वदे॒वम् । \newline
22. वै॒श्व॒दे॒व मितीति॑ वैश्वदे॒वं ॅवै᳚श्वदे॒व मिति॑ । \newline
23. वै॒श्व॒दे॒वमिति॑ वैश्व - दे॒वम् । \newline
24. इति॑ ब्रूयाद् ब्रूया॒ दितीति॑ ब्रूयात् । \newline
25. ब्रू॒या॒द् विश्वे॒ विश्वे᳚ ब्रूयाद् ब्रूया॒द् विश्वे᳚ । \newline
26. विश्वे॒ हि हि विश्वे॒ विश्वे॒ हि । \newline
27. हि तत् तद्धि हि तत् । \newline
28. तद् दे॒वा दे॒वा स्तत् तद् दे॒वाः । \newline
29. दे॒वा भा॑ग॒धेय॑म् भाग॒धेय॑म् दे॒वा दे॒वा भा॑ग॒धेय᳚म् । \newline
30. भा॒ग॒धेय॑ म॒भ्य॑भि भा॑ग॒धेय॑म् भाग॒धेय॑ म॒भि । \newline
31. भा॒ग॒धेय॒मिति॑ भाग - धेय᳚म् । \newline
32. अ॒भि स॒मग॑च्छन्त स॒मग॑च्छन्ता॒भ्य॑भि स॒मग॑च्छन्त । \newline
33. स॒मग॑च्छ॒न्ते तीति॑ स॒मग॑च्छन्त स॒मग॑च्छ॒न्ते ति॑ । \newline
34. स॒मग॑च्छ॒न्तेति॑ सं - अग॑च्छन्त । \newline
35. इत्यथो॒ अथो॒ इतीत्यथो᳚ । \newline
36. अथो॒ खलु॒ खल्वथो॒ अथो॒ खलु॑ । \newline
37. अथो॒ इत्यथो᳚ । \newline
38. खल्वै॒न्द्र मै॒न्द्रम् खलु॒ खल्वै॒न्द्रम् । \newline
39. ऐ॒न्द्र मितीत्यै॒न्द्र मै॒न्द्र मिति॑ । \newline
40. इत्ये॒वैवे तीत्ये॒व । \newline
41. ए॒व ब्रू॑याद् ब्रूया दे॒वैव ब्रू॑यात् । \newline
42. ब्रू॒या॒ दिन्द्र॒ मिन्द्र॑म् ब्रूयाद् ब्रूया॒ दिन्द्र᳚म् । \newline
43. इन्द्रं॒ ॅवाव वावे न्द्र॒ मिन्द्रं॒ ॅवाव । \newline
44. वाव ते ते वाव वाव ते । \newline
45. ते तत् तत् ते ते तत् । \newline
46. तद् भि॑ष॒ज्यन्तो॑ भिष॒ज्यन्त॒ स्तत् तद् भि॑ष॒ज्यन्तः॑ । \newline
47. भि॒ष॒ज्यन्तो॒ ऽभ्य॑भि भि॑ष॒ज्यन्तो॑ भिष॒ज्यन्तो॒ ऽभि । \newline
48. अ॒भि सꣳ स म॒भ्य॑भि सम् । \newline
49. स म॑गच्छन्ता गच्छन्त॒ सꣳ स म॑गच्छन्त । \newline
50. अ॒ग॒च्छ॒न्ते तीत्य॑गच्छन्ता गच्छ॒न्ते ति॑ । \newline
51. इतीतीति॑ । \newline

\textbf{Ghana Paata } \newline

1. अ॒द्य वसु॒ वस्व॒द्याद्य वसु॑ वसति वसति॒ वस्व॒द्याद्य वसु॑ वसति । \newline
2. वसु॑ वसति वसति॒ वसु॒ वसु॑ वस॒तीतीति॑ वसति॒ वसु॒ वसु॑ वस॒तीति॑ । \newline
3. व॒स॒तीतीति॑ वसति वस॒तीतीन्द्र॒ इन्द्र॒ इति॑ वसति वस॒तीतीन्द्रः॑ । \newline
4. इतीन्द्र॒ इन्द्र॒ इतीतीन्द्रो॒ हि हीन्द्र॒ इतीतीन्द्रो॒ हि । \newline
5. इन्द्रो॒ हि हीन्द्र॒ इन्द्रो॒ हि दे॒वाना᳚म् दे॒वानाꣳ॒॒ हीन्द्र॒ इन्द्रो॒ हि दे॒वाना᳚म् । \newline
6. हि दे॒वाना᳚म् दे॒वानाꣳ॒॒ हि हि दे॒वानां॒ ॅवसु॒ वसु॑ दे॒वानाꣳ॒॒ हि हि दे॒वानां॒ ॅवसु॑ । \newline
7. दे॒वानां॒ ॅवसु॒ वसु॑ दे॒वाना᳚म् दे॒वानां॒ ॅवसु॒ तत् तद् वसु॑ दे॒वाना᳚म् दे॒वानां॒ ॅवसु॒ तत् । \newline
8. वसु॒ तत् तद् वसु॒ वसु॒ तद॑मावा॒स्या॑या अमावा॒स्या॑या॒ स्तद् वसु॒ वसु॒ तद॑मावा॒स्या॑याः । \newline
9. तद॑मावा॒स्या॑या अमावा॒स्या॑या॒ स्तत् तद॑मावा॒स्या॑या अमावास्य॒त्व म॑मावास्य॒त्व म॑मावा॒स्या॑या॒ स्तत् तद॑मावा॒स्या॑या अमावास्य॒त्वम् । \newline
10. अ॒मा॒वा॒स्या॑या अमावास्य॒त्व म॑मावास्य॒त्व म॑मावा॒स्या॑या अमावा॒स्या॑या अमावास्य॒त्वम् ब्र॑ह्मवा॒दिनो᳚ ब्रह्मवा॒दिनो॑ ऽमावास्य॒त्व म॑मावा॒स्या॑या अमावा॒स्या॑या अमावास्य॒त्वम् ब्र॑ह्मवा॒दिनः॑ । \newline
11. अ॒मा॒वा॒स्या॑या॒ इत्य॑मा - वा॒स्या॑याः । \newline
12. अ॒मा॒वा॒स्य॒त्वम् ब्र॑ह्मवा॒दिनो᳚ ब्रह्मवा॒दिनो॑ ऽमावास्य॒त्व म॑मावास्य॒त्वम् ब्र॑ह्मवा॒दिनो॑ वदन्ति वदन्ति ब्रह्मवा॒दिनो॑ ऽमावास्य॒त्व म॑मावास्य॒त्वम् ब्र॑ह्मवा॒दिनो॑ वदन्ति । \newline
13. अ॒मा॒वा॒स्य॒त्वमित्य॑मावास्य - त्वम् । \newline
14. ब्र॒ह्म॒वा॒दिनो॑ वदन्ति वदन्ति ब्रह्मवा॒दिनो᳚ ब्रह्मवा॒दिनो॑ वदन्ति किन्देव॒त्य॑म् किन्देव॒त्यं॑ ॅवदन्ति ब्रह्मवा॒दिनो᳚ ब्रह्मवा॒दिनो॑ वदन्ति किन्देव॒त्य᳚म् । \newline
15. ब्र॒ह्म॒वा॒दिन॒ इति॑ ब्रह्म - वा॒दिनः॑ । \newline
16. व॒द॒न्ति॒ कि॒न्दे॒व॒त्य॑म् किन्देव॒त्यं॑ ॅवदन्ति वदन्ति किन्देव॒त्यꣳ॑ सान्ना॒य्यꣳ सा᳚न्ना॒य्यम् कि॑न्देव॒त्यं॑ ॅवदन्ति वदन्ति किन्देव॒त्यꣳ॑ सान्ना॒य्यम् । \newline
17. कि॒न्दे॒व॒त्यꣳ॑ सान्ना॒य्यꣳ सा᳚न्ना॒य्यम् कि॑न्देव॒त्य॑म् किन्देव॒त्यꣳ॑ सान्ना॒य्य मितीति॑ सान्ना॒य्यम् कि॑न्देव॒त्य॑म् किन्देव॒त्यꣳ॑ सान्ना॒य्य मिति॑ । \newline
18. कि॒न्दे॒व॒त्य॑मिति॑ किं - दे॒व॒त्य᳚म् । \newline
19. सा॒न्ना॒य्य मितीति॑ सान्ना॒य्यꣳ सा᳚न्ना॒य्य मिति॑ वैश्वदे॒वं ॅवै᳚श्वदे॒व मिति॑ सान्ना॒य्यꣳ सा᳚न्ना॒य्य मिति॑ वैश्वदे॒वम् । \newline
20. सा॒न्ना॒य्यमिति॑ सां - ना॒य्यम् । \newline
21. इति॑ वैश्वदे॒वं ॅवै᳚श्वदे॒व मितीति॑ वैश्वदे॒व मितीति॑ वैश्वदे॒व मितीति॑ वैश्वदे॒व मिति॑ । \newline
22. वै॒श्व॒दे॒व मितीति॑ वैश्वदे॒वं ॅवै᳚श्वदे॒व मिति॑ ब्रूयाद् ब्रूया॒दिति॑ वैश्वदे॒वं ॅवै᳚श्वदे॒व मिति॑ ब्रूयात् । \newline
23. वै॒श्व॒दे॒वमिति॑ वैश्व - दे॒वम् । \newline
24. इति॑ ब्रूयाद् ब्रूया॒दितीति॑ ब्रूया॒द् विश्वे॒ विश्वे᳚ ब्रूया॒दितीति॑ ब्रूया॒द् विश्वे᳚ । \newline
25. ब्रू॒या॒द् विश्वे॒ विश्वे᳚ ब्रूयाद् ब्रूया॒द् विश्वे॒ हि हि विश्वे᳚ ब्रूयाद् ब्रूया॒द् विश्वे॒ हि । \newline
26. विश्वे॒ हि हि विश्वे॒ विश्वे॒ हि तत् तद्धि विश्वे॒ विश्वे॒ हि तत् । \newline
27. हि तत् तद्धि हि तद् दे॒वा दे॒वा स्तद्धि हि तद् दे॒वाः । \newline
28. तद् दे॒वा दे॒वा स्तत् तद् दे॒वा भा॑ग॒धेय॑म् भाग॒धेय॑म् दे॒वा स्तत् तद् दे॒वा भा॑ग॒धेय᳚म् । \newline
29. दे॒वा भा॑ग॒धेय॑म् भाग॒धेय॑म् दे॒वा दे॒वा भा॑ग॒धेय॑ म॒भ्य॑भि भा॑ग॒धेय॑म् दे॒वा दे॒वा भा॑ग॒धेय॑ म॒भि । \newline
30. भा॒ग॒धेय॑ म॒भ्य॑भि भा॑ग॒धेय॑म् भाग॒धेय॑ म॒भि स॒मग॑च्छन्त स॒मग॑च्छन्ता॒भि भा॑ग॒धेय॑म् भाग॒धेय॑ म॒भि स॒मग॑च्छन्त । \newline
31. भा॒ग॒धेय॒मिति॑ भाग - धेय᳚म् । \newline
32. अ॒भि स॒मग॑च्छन्त स॒मग॑च्छन्ता॒भ्य॑भि स॒मग॑च्छ॒न्ते तीति॑ स॒मग॑च्छन्ता॒भ्य॑भि स॒मग॑च्छ॒न्ते ति॑ । \newline
33. स॒मग॑च्छ॒न्ते तीति॑ स॒मग॑च्छन्त स॒मग॑च्छ॒न्ते त्यथो॒ अथो॒ इति॑ स॒मग॑च्छन्त स॒मग॑च्छ॒न्ते त्यथो᳚ । \newline
34. स॒मग॑च्छ॒न्तेति॑ सं - अग॑च्छन्त । \newline
35. इत्यथो॒ अथो॒ इतीत्यथो॒ खलु॒ खल्वथो॒ इतीत्यथो॒ खलु॑ । \newline
36. अथो॒ खलु॒ खल्वथो॒ अथो॒ खल्वै॒न्द्र मै॒न्द्रम् खल्वथो॒ अथो॒ खल्वै॒न्द्रम् । \newline
37. अथो॒ इत्यथो᳚ । \newline
38. खल्वै॒न्द्र मै॒न्द्रम् खलु॒ खल्वै॒न्द्र मितीत्यै॒न्द्रम् खलु॒ खल्वै॒न्द्र मिति॑ । \newline
39. ऐ॒न्द्र मितीत्यै॒न्द्र मै॒न्द्र मित्ये॒वैवे त्यै॒न्द्र मै॒न्द्र मित्ये॒व । \newline
40. इत्ये॒वैवे तीत्ये॒व ब्रू॑याद् ब्रूयादे॒वे तीत्ये॒व ब्रू॑यात् । \newline
41. ए॒व ब्रू॑याद् ब्रूया दे॒वैव ब्रू॑या॒ दिन्द्र॒ मिन्द्र॑म् ब्रूया दे॒वैव ब्रू॑या॒ दिन्द्र᳚म् । \newline
42. ब्रू॒या॒ दिन्द्र॒ मिन्द्र॑म् ब्रूयाद् ब्रूया॒ दिन्द्रं॒ ॅवाव वावे न्द्र॑म् ब्रूयाद् ब्रूया॒ दिन्द्रं॒ ॅवाव । \newline
43. इन्द्रं॒ ॅवाव वावे न्द्र॒ मिन्द्रं॒ ॅवाव ते ते वावे न्द्र॒ मिन्द्रं॒ ॅवाव ते । \newline
44. वाव ते ते वाव वाव ते तत् तत् ते वाव वाव ते तत् । \newline
45. ते तत् तत् ते ते तद् भि॑ष॒ज्यन्तो॑ भिष॒ज्यन्त॒ स्तत् ते ते तद् भि॑ष॒ज्यन्तः॑ । \newline
46. तद् भि॑ष॒ज्यन्तो॑ भिष॒ज्यन्त॒ स्तत् तद् भि॑ष॒ज्यन्तो॒ ऽभ्य॑भि भि॑ष॒ज्यन्त॒ स्तत् तद् भि॑ष॒ज्यन्तो॒ ऽभि । \newline
47. भि॒ष॒ज्यन्तो॒ ऽभ्य॑भि भि॑ष॒ज्यन्तो॑ भिष॒ज्यन्तो॒ ऽभि सꣳ स म॒भि भि॑ष॒ज्यन्तो॑ भिष॒ज्यन्तो॒ ऽभि सम् । \newline
48. अ॒भि सꣳ स म॒भ्य॑भि स म॑गच्छन्ता गच्छन्त॒ स म॒भ्य॑भि स म॑गच्छन्त । \newline
49. स म॑गच्छन्ता गच्छन्त॒ सꣳ स म॑गच्छ॒न्ते तीत्य॑गच्छन्त॒ सꣳ स म॑गच्छ॒न्ते ति॑ । \newline
50. अ॒ग॒च्छ॒न्ते तीत्य॑गच्छन्ता गच्छ॒न्ते ति॑ । \newline
51. इतीतीति॑ । \newline
\pagebreak
\markright{ TS 2.5.4.1  \hfill https://www.vedavms.in \hfill}
\addcontentsline{toc}{section}{ TS 2.5.4.1 }
\section*{ TS 2.5.4.1 }

\textbf{TS 2.5.4.1 } \newline
\textbf{Samhita Paata} \newline

ब्र॒ह्म॒वा॒दिनो॑ वदन्ति॒ स त्वै द॑॑र्.शपूर्णमा॒सौ य॑जेत॒ य ए॑नौ॒ सेन्द्रौ॒ यजे॒तेति॑ वैमृ॒धः पू॒र्णमा॑से ऽनुनिर्वा॒प्यो॑ भवति॒ तेन॑ पू॒र्णमा॑सः॒ सेन्द्र॑ ऐ॒न्द्रं दद्ध्य॑मावा॒स्या॑यां॒ तेना॑मावा॒स्या॑ सेन्द्रा॒ य ए॒वं ॅवि॒द्वान् द॑र्.शपूर्णमा॒सौ यज॑ते॒ सेन्द्रा॑वे॒वैनौ॑ यजते॒ श्वः श्वो᳚ऽस्मा ईजा॒नाय॒ वसी॑यो भवति दे॒वा वै यद्-य॒ज्ञे ऽकु॑र्वत॒तदसु॑रा अकुर्वत॒ ते दे॒वा ए॒ता - [  ] \newline

\textbf{Pada Paata} \newline

ब्र॒ह्म॒वा॒दिन॒ इति॑ ब्रह्म - वा॒दिनः॑ । व॒द॒न्ति॒ । सः । तु । वै । द॒र्.॒श॒पू॒र्ण॒मा॒साविति॑ दर्.श - पू॒र्ण॒मा॒सौ । य॒जे॒त॒ । यः । ए॒नौ॒ । सेन्द्रा॒विति॒ स - इ॒न्द्रौ॒ । यजे॑त । इति॑ । वै॒मृ॒धः । पू॒र्णमा॑स॒ इति॑ पू॒र्ण - मा॒से॒ । अ॒नु॒नि॒र्वा॒प्य॑ इत्य॑नु - नि॒र्वा॒प्यः॑ । भ॒व॒ति॒ । तेन॑ । पू॒र्णमा॑स॒ इति॑ पू॒र्ण - मा॒सः॒ । सेन्द्र॒ इति॒ स - इ॒न्द्रः॒ । ऐ॒न्द्रम् । दधि॑ । अ॒मा॒वा॒स्या॑या॒मित्य॑मा - वा॒स्या॑याम् । तेन॑ । अ॒मा॒वा॒स्येत्य॑मा - वा॒स्या᳚ । सेन्द्रेति॒ स - इ॒न्द्रा॒ । यः । ए॒वम् । वि॒द्वान् । द॒र्.॒श॒पू॒र्ण॒मा॒साविति॑ दर्.श - पू॒र्ण॒मा॒सौ । यज॑ते । सेन्द्रा॒विति॒ स - इ॒न्द्रौ॒ । ए॒व । ए॒नौ॒ । य॒ज॒ते॒ । श्वः श्व॒ इति॒ श्वः - श्वः॒ । अ॒स्मै॒ । ई॒जा॒नाय॑ । वसी॑यः । भ॒व॒ति॒ । दे॒वाः । वै । यत् । य॒ज्ञे । अकु॑र्वत । तत् । असु॑राः । अ॒कु॒र्व॒त॒ । ते । दे॒वाः । ए॒ताम् ।  \newline


\textbf{Krama Paata} \newline

ब्र॒ह्म॒वा॒दिनो॑ वदन्ति । ब्र॒ह्म॒वा॒दिन॒ इति॑ ब्रह्म - वा॒दिनः॑ । व॒द॒न्ति॒ सः । स तु । त्वै । 
वै द॑र्.शपूर्णमा॒सौ । द॒र्॒.श॒पू॒र्ण॒मा॒सौ य॑जेत । द॒र्॒.श॒पू॒र्ण॒मा॒साविति॑ दर्.श - पू॒र्ण॒मा॒सौ । य॒जे॒त॒ यः । य ए॑नौ । ए॒नौ॒ सेन्द्रौ᳚ । सेन्द्रौ॒ यजे॑त । सेन्द्रा॒विति॒ स - इ॒न्द्रौ॒ । यजे॒तेति॑ । इति॑ वैमृ॒धः । वै॒मृ॒धः पू॒र्णमा॑से । पू॒र्णमा॑से ऽनुनिर्वा॒प्यः॑ । पू॒र्णमा॑स॒ इति॑ पू॒र्ण - मा॒से॒ । अ॒नु॒नि॒र्वा॒प्यो॑ भवति । अ॒नु॒नि॒र्वा॒प्य॑ इत्य॑नु - नि॒र्वा॒प्यः॑ । भ॒व॒ति॒ तेन॑ । तेन॑ पू॒र्णमा॑सः । पू॒र्णमा॑सः॒ सेन्द्रः॑ । पू॒र्णमा॑स॒ इति॑ पू॒र्ण - मा॒सः॒ । सेन्द्र॑ ऐ॒न्द्रम् । सेन्द्र॒ इति॒ स - इ॒न्द्रः॒ । ऐ॒न्द्रम् दधि॑ । दद्ध्य॑मावा॒स्या॑याम् । अ॒मा॒वा॒स्या॑या॒म् तेन॑ । अ॒मा॒वा॒स्या॑या॒मित्य॑मा - वा॒स्या॑याम् । तेना॑मावा॒स्या᳚ । अ॒मा॒वा॒स्या॑ सेन्द्रा᳚ । अ॒मा॒वा॒स्येत्य॑मा - वा॒स्या᳚ । सेन्द्रा॒ यः । सेन्द्रेति॒ स - इ॒न्द्रा॒ । य ए॒वम् । ए॒वम् ॅवि॒द्वान् । वि॒द्वान् द॑र्.शपूर्णमा॒सौ । द॒र्॒.श॒पू॒र्ण॒मा॒सौ यज॑ते । द॒र्॒.श॒पू॒र्ण॒मा॒साविति॑ दर्.श - पू॒र्ण॒मा॒सौ । यज॑ते॒ सेन्द्रौ᳚ । सेन्द्रा॑वे॒व । सेन्द्रा॒विति॒ स - इ॒न्द्रौ॒ । ए॒वैनौ᳚ । ए॒नौ॒ य॒ज॒ते॒ । य॒ज॒ते॒ श्वःश्वः॑ । श्वःश्वो᳚ ऽस्मै । श्वःश्व॒ इति॒ श्वः - श्वः॒ । अ॒स्मा॒ ई॒जा॒नाय॑ । ई॒जा॒नाय॒ वसी॑यः । वसी॑यो भवति । भ॒व॒ति॒ दे॒वाः । दे॒वा वै । वै यत् । यद् य॒ज्ञे । य॒ज्ञे ऽकु॑र्वत । अकु॑र्वत॒ तत् । तदसु॑राः । असु॑रा अकुर्वत । अ॒कु॒र्व॒त॒ ते । ते दे॒वाः । दे॒वा ए॒ताम् । ए॒तामिष्टि᳚म् \newline

\textbf{Jatai Paata} \newline

1. ब्र॒ह्म॒वा॒दिनो॑ वदन्ति वदन्ति ब्रह्मवा॒दिनो᳚ ब्रह्मवा॒दिनो॑ वदन्ति । \newline
2. ब्र॒ह्म॒वा॒दिन॒ इति॑ ब्रह्म - वा॒दिनः॑ । \newline
3. व॒द॒न्ति॒ स स व॑दन्ति वदन्ति॒ सः । \newline
4. स तु तु स स तु । \newline
5. त्वै वै तु त्वै । \newline
6. वै द॑र्.शपूर्णमा॒सौ द॑र्.शपूर्णमा॒सौ वै वै द॑र्.शपूर्णमा॒सौ । \newline
7. द॒र्॒.श॒पू॒र्ण॒मा॒सौ य॑जेत यजेत दर्.शपूर्णमा॒सौ द॑र्.शपूर्णमा॒सौ य॑जेत । \newline
8. द॒र्॒.श॒पू॒र्ण॒मा॒साविति॑ दर्.श - पू॒र्ण॒मा॒सौ । \newline
9. य॒जे॒त॒ यो यो य॑जेत यजेत॒ यः । \newline
10. य ए॑ना वेनौ॒ यो य ए॑नौ । \newline
11. ए॒नौ॒ सेन्द्रौ॒ सेन्द्रा॑ वेना वेनौ॒ सेन्द्रौ᳚ । \newline
12. सेन्द्रौ॒ यजे॑त॒ यजे॑त॒ सेन्द्रौ॒ सेन्द्रौ॒ यजे॑त । \newline
13. सेन्द्रा॒विति॒ स - इ॒न्द्रौ॒ । \newline
14. यजे॒ते तीति॒ यजे॑त॒ यजे॒ते ति॑ । \newline
15. इति॑ वैमृ॒धो वै॑मृ॒ध इतीति॑ वैमृ॒धः । \newline
16. वै॒मृ॒धः पू॒र्णमा॑से पू॒र्णमा॑से वैमृ॒धो वै॑मृ॒धः पू॒र्णमा॑से । \newline
17. पू॒र्णमा॑से ऽनुनिर्वा॒प्यो॑ ऽनुनिर्वा॒प्यः॑ पू॒र्णमा॑से पू॒र्णमा॑से ऽनुनिर्वा॒प्यः॑ । \newline
18. पू॒र्णमा॑स॒ इति॑ पू॒र्ण - मा॒से॒ । \newline
19. अ॒नु॒नि॒र्वा॒प्यो॑ भवति भव त्यनुनिर्वा॒प्यो॑ ऽनुनिर्वा॒प्यो॑ भवति । \newline
20. अ॒नु॒नि॒र्वा॒प्य॑ इत्य॑नु - नि॒र्वा॒प्यः॑ । \newline
21. भ॒व॒ति॒ तेन॒ तेन॑ भवति भवति॒ तेन॑ । \newline
22. तेन॑ पू॒र्णमा॑सः पू॒र्णमा॑स॒ स्तेन॒ तेन॑ पू॒र्णमा॑सः । \newline
23. पू॒र्णमा॑सः॒ सेन्द्रः॒ सेन्द्रः॑ पू॒र्णमा॑सः पू॒र्णमा॑सः॒ सेन्द्रः॑ । \newline
24. पू॒र्णमा॑स॒ इति॑ पू॒र्ण - मा॒सः॒ । \newline
25. सेन्द्र॑ ऐ॒न्द्र मै॒न्द्रꣳ सेन्द्रः॒ सेन्द्र॑ ऐ॒न्द्रम् । \newline
26. सेन्द्र॒ इति॒ स - इ॒न्द्रः॒ । \newline
27. ऐ॒न्द्रम् दधि॒ दध्यै॒न्द्र मै॒न्द्रम् दधि॑ । \newline
28. दध्य॑मावा॒स्या॑या ममावा॒स्या॑या॒म् दधि॒ दध्य॑मावा॒स्या॑याम् । \newline
29. अ॒मा॒वा॒स्या॑या॒म् तेन॒ तेना॑मावा॒स्या॑या ममावा॒स्या॑या॒म् तेन॑ । \newline
30. अ॒मा॒वा॒स्या॑या॒मित्य॑मा - वा॒स्या॑याम् । \newline
31. तेना॑मावा॒स्या॑ ऽमावा॒स्या॑ तेन॒ तेना॑मावा॒स्या᳚ । \newline
32. अ॒मा॒वा॒स्या॑ सेन्द्रा॒ सेन्द्रा॑ ऽमावा॒स्या॑ ऽमावा॒स्या॑ सेन्द्रा᳚ । \newline
33. अ॒मा॒वा॒स्येत्य॑मा - वा॒स्या᳚ । \newline
34. सेन्द्रा॒ यो यः सेन्द्रा॒ सेन्द्रा॒ यः । \newline
35. सेन्द्रेति॒ स - इ॒न्द्रा॒ । \newline
36. य ए॒व मे॒वं ॅयो य ए॒वम् । \newline
37. ए॒वं ॅवि॒द्वान्. वि॒द्वा ने॒व मे॒वं ॅवि॒द्वान् । \newline
38. वि॒द्वान् द॑र्.शपूर्णमा॒सौ द॑र्.शपूर्णमा॒सौ वि॒द्वान्. वि॒द्वान् द॑र्.शपूर्णमा॒सौ । \newline
39. द॒र्॒.श॒पू॒र्ण॒मा॒सौ यज॑ते॒ यज॑ते दर्.शपूर्णमा॒सौ द॑र्.शपूर्णमा॒सौ यज॑ते । \newline
40. द॒र्॒.श॒पू॒र्ण॒मा॒साविति॑ दर्.श - पू॒र्ण॒मा॒सौ । \newline
41. यज॑ते॒ सेन्द्रौ॒ सेन्द्रौ॒ यज॑ते॒ यज॑ते॒ सेन्द्रौ᳚ । \newline
42. सेन्द्रा॑ वे॒वैव सेन्द्रौ॒ सेन्द्रा॑ वे॒व । \newline
43. सेन्द्रा॒विति॒ स - इ॒न्द्रौ॒ । \newline
44. ए॒वैना॑ वेना वे॒वैवैनौ᳚ । \newline
45. ए॒नौ॒ य॒ज॒ते॒ य॒ज॒त॒ ए॒ना॒ वे॒नौ॒ य॒ज॒ते॒ । \newline
46. य॒ज॒ते॒ श्वःश्वः॒ श्वःश्वो॑ यजते यजते॒ श्वःश्वः॑ । \newline
47. श्वःश्वो᳚ ऽस्मा अस्मै॒ श्वःश्वः॒ श्वःश्वो᳚ ऽस्मै । \newline
48. श्वःश्व॒ इति॒ श्वः - श्वः॒ । \newline
49. अ॒स्मा॒ ई॒जा॒नाये॑ जा॒नाया᳚स्मा अस्मा ईजा॒नाय॑ । \newline
50. ई॒जा॒नाय॒ वसी॑यो॒ वसी॑य ईजा॒नाये॑ जा॒नाय॒ वसी॑यः । \newline
51. वसी॑यो भवति भवति॒ वसी॑यो॒ वसी॑यो भवति । \newline
52. भ॒व॒ति॒ दे॒वा दे॒वा भ॑वति भवति दे॒वाः । \newline
53. दे॒वा वै वै दे॒वा दे॒वा वै । \newline
54. वै यद् यद् वै वै यत् । \newline
55. यद् य॒ज्ञे य॒ज्ञे यद् यद् य॒ज्ञे । \newline
56. य॒ज्ञे ऽकु॑र्व॒ता कु॑र्वत य॒ज्ञे य॒ज्ञे ऽकु॑र्वत । \newline
57. अकु॑र्वत॒ तत् तदकु॑र्व॒ता कु॑र्वत॒ तत् । \newline
58. तदसु॑रा॒ असु॑रा॒ स्तत् तदसु॑राः । \newline
59. असु॑रा अकुर्वता कुर्व॒ता सु॑रा॒ असु॑रा अकुर्वत । \newline
60. अ॒कु॒र्व॒त॒ ते ते॑ ऽकुर्वता कुर्वत॒ ते । \newline
61. ते दे॒वा दे॒वा स्ते ते दे॒वाः । \newline
62. दे॒वा ए॒ता मे॒ताम् दे॒वा दे॒वा ए॒ताम् । \newline
63. ए॒ता मिष्टि॒ मिष्टि॑ मे॒ता मे॒ता मिष्टि᳚म् । \newline

\textbf{Ghana Paata } \newline

1. ब्र॒ह्म॒वा॒दिनो॑ वदन्ति वदन्ति ब्रह्मवा॒दिनो᳚ ब्रह्मवा॒दिनो॑ वदन्ति॒ स स व॑दन्ति ब्रह्मवा॒दिनो᳚ ब्रह्मवा॒दिनो॑ वदन्ति॒ सः । \newline
2. ब्र॒ह्म॒वा॒दिन॒ इति॑ ब्रह्म - वा॒दिनः॑ । \newline
3. व॒द॒न्ति॒ स स व॑दन्ति वदन्ति॒ स तु तु स व॑दन्ति वदन्ति॒ स तु । \newline
4. स तु तु स स त्वै वै तु स स त्वै । \newline
5. त्वै वै तु त्वै द॑र्.शपूर्णमा॒सौ द॑र्.शपूर्णमा॒सौ वै तु त्वै द॑र्.शपूर्णमा॒सौ । \newline
6. वै द॑र्.शपूर्णमा॒सौ द॑र्.शपूर्णमा॒सौ वै वै द॑र्.शपूर्णमा॒सौ य॑जेत यजेत दर्.शपूर्णमा॒सौ वै वै द॑र्.शपूर्णमा॒सौ य॑जेत । \newline
7. द॒र्॒.श॒पू॒र्ण॒मा॒सौ य॑जेत यजेत दर्.शपूर्णमा॒सौ द॑र्.शपूर्णमा॒सौ य॑जेत॒ यो यो य॑जेत दर्.शपूर्णमा॒सौ द॑र्.शपूर्णमा॒सौ य॑जेत॒ यः । \newline
8. द॒र्॒.श॒पू॒र्ण॒मा॒साविति॑ दर्.श - पू॒र्ण॒मा॒सौ । \newline
9. य॒जे॒त॒ यो यो य॑जेत यजेत॒ य ए॑ना वेनौ॒ यो य॑जेत यजेत॒ य ए॑नौ । \newline
10. य ए॑ना वेनौ॒ यो य ए॑नौ॒ सेन्द्रौ॒ सेन्द्रा॑ वेनौ॒ यो य ए॑नौ॒ सेन्द्रौ᳚ । \newline
11. ए॒नौ॒ सेन्द्रौ॒ सेन्द्रा॑ वेना वेनौ॒ सेन्द्रौ॒ यजे॑त॒ यजे॑त॒ सेन्द्रा॑ वेना वेनौ॒ सेन्द्रौ॒ यजे॑त । \newline
12. सेन्द्रौ॒ यजे॑त॒ यजे॑त॒ सेन्द्रौ॒ सेन्द्रौ॒ यजे॒ते तीति॒ यजे॑त॒ सेन्द्रौ॒ सेन्द्रौ॒ यजे॒ते ति॑ । \newline
13. सेन्द्रा॒विति॒ स - इ॒न्द्रौ॒ । \newline
14. यजे॒ते तीति॒ यजे॑त॒ यजे॒ते ति॑ वैमृ॒धो वै॑मृ॒ध इति॒ यजे॑त॒ यजे॒ते ति॑ वैमृ॒धः । \newline
15. इति॑ वैमृ॒धो वै॑मृ॒ध इतीति॑ वैमृ॒धः पू॒र्णमा॑से पू॒र्णमा॑से वैमृ॒ध इतीति॑ वैमृ॒धः पू॒र्णमा॑से । \newline
16. वै॒मृ॒धः पू॒र्णमा॑से पू॒र्णमा॑से वैमृ॒धो वै॑मृ॒धः पू॒र्णमा॑से ऽनुनिर्वा॒प्यो॑ ऽनुनिर्वा॒प्यः॑ पू॒र्णमा॑से वैमृ॒धो वै॑मृ॒धः पू॒र्णमा॑से ऽनुनिर्वा॒प्यः॑ । \newline
17. पू॒र्णमा॑से ऽनुनिर्वा॒प्यो॑ ऽनुनिर्वा॒प्यः॑ पू॒र्णमा॑से पू॒र्णमा॑से ऽनुनिर्वा॒प्यो॑ भवति भव त्यनुनिर्वा॒प्यः॑ पू॒र्णमा॑से पू॒र्णमा॑से ऽनुनिर्वा॒प्यो॑ भवति । \newline
18. पू॒र्णमा॑स॒ इति॑ पू॒र्ण - मा॒से॒ । \newline
19. अ॒नु॒नि॒र्वा॒प्यो॑ भवति भव त्यनुनिर्वा॒प्यो॑ ऽनुनिर्वा॒प्यो॑ भवति॒ तेन॒ तेन॑ भव त्यनुनिर्वा॒प्यो॑ ऽनुनिर्वा॒प्यो॑ भवति॒ तेन॑ । \newline
20. अ॒नु॒नि॒र्वा॒प्य॑ इत्य॑नु - नि॒र्वा॒प्यः॑ । \newline
21. भ॒व॒ति॒ तेन॒ तेन॑ भवति भवति॒ तेन॑ पू॒र्णमा॑सः पू॒र्णमा॑स॒ स्तेन॑ भवति भवति॒ तेन॑ पू॒र्णमा॑सः । \newline
22. तेन॑ पू॒र्णमा॑सः पू॒र्णमा॑स॒ स्तेन॒ तेन॑ पू॒र्णमा॑सः॒ सेन्द्रः॒ सेन्द्रः॑ पू॒र्णमा॑स॒ स्तेन॒ तेन॑ पू॒र्णमा॑सः॒ सेन्द्रः॑ । \newline
23. पू॒र्णमा॑सः॒ सेन्द्रः॒ सेन्द्रः॑ पू॒र्णमा॑सः पू॒र्णमा॑सः॒ सेन्द्र॑ ऐ॒न्द्र मै॒न्द्रꣳ सेन्द्रः॑ पू॒र्णमा॑सः पू॒र्णमा॑सः॒ सेन्द्र॑ ऐ॒न्द्रम् । \newline
24. पू॒र्णमा॑स॒ इति॑ पू॒र्ण - मा॒सः॒ । \newline
25. सेन्द्र॑ ऐ॒न्द्र मै॒न्द्रꣳ सेन्द्रः॒ सेन्द्र॑ ऐ॒न्द्रम् दधि॒ दध्यै॒न्द्रꣳ सेन्द्रः॒ सेन्द्र॑ ऐ॒न्द्रम् दधि॑ । \newline
26. सेन्द्र॒ इति॒ स - इ॒न्द्रः॒ । \newline
27. ऐ॒न्द्रम् दधि॒ दध्यै॒न्द्र मै॒न्द्रम् दध्य॑मावा॒स्या॑या ममावा॒स्या॑या॒म् दध्यै॒न्द्र मै॒न्द्रम् दध्य॑मावा॒स्या॑याम् । \newline
28. दध्य॑मावा॒स्या॑या ममावा॒स्या॑या॒म् दधि॒ दध्य॑मावा॒स्या॑या॒म् तेन॒ तेना॑मावा॒स्या॑या॒म् दधि॒ दध्य॑मावा॒स्या॑या॒म् तेन॑ । \newline
29. अ॒मा॒वा॒स्या॑या॒म् तेन॒ तेना॑मावा॒स्या॑या ममावा॒स्या॑या॒म् तेना॑मावा॒स्या॑ ऽमावा॒स्या॑ तेना॑मावा॒स्या॑या ममावा॒स्या॑या॒म् तेना॑मावा॒स्या᳚ । \newline
30. अ॒मा॒वा॒स्या॑या॒मित्य॑मा - वा॒स्या॑याम् । \newline
31. तेना॑मावा॒स्या॑ ऽमावा॒स्या॑ तेन॒ तेना॑मावा॒स्या॑ सेन्द्रा॒ सेन्द्रा॑ ऽमावा॒स्या॑ तेन॒ तेना॑मावा॒स्या॑ सेन्द्रा᳚ । \newline
32. अ॒मा॒वा॒स्या॑ सेन्द्रा॒ सेन्द्रा॑ ऽमावा॒स्या॑ ऽमावा॒स्या॑ सेन्द्रा॒ यो यः सेन्द्रा॑ ऽमावा॒स्या॑ ऽमावा॒स्या॑ सेन्द्रा॒ यः । \newline
33. अ॒मा॒वा॒स्येत्य॑मा - वा॒स्या᳚ । \newline
34. सेन्द्रा॒ यो यः सेन्द्रा॒ सेन्द्रा॒ य ए॒व मे॒वं ॅयः सेन्द्रा॒ सेन्द्रा॒ य ए॒वम् । \newline
35. सेन्द्रेति॒ स - इ॒न्द्रा॒ । \newline
36. य ए॒व मे॒वं ॅयो य ए॒वं ॅवि॒द्वान्. वि॒द्वा ने॒वं ॅयो य ए॒वं ॅवि॒द्वान् । \newline
37. ए॒वं ॅवि॒द्वान्. वि॒द्वा ने॒व मे॒वं ॅवि॒द्वान् द॑र्.शपूर्णमा॒सौ द॑र्.शपूर्णमा॒सौ वि॒द्वा ने॒व मे॒वं ॅवि॒द्वान् द॑र्.शपूर्णमा॒सौ । \newline
38. वि॒द्वान् द॑र्.शपूर्णमा॒सौ द॑र्.शपूर्णमा॒सौ वि॒द्वान्. वि॒द्वान् द॑र्.शपूर्णमा॒सौ यज॑ते॒ यज॑ते दर्.शपूर्णमा॒सौ वि॒द्वान्. वि॒द्वान् द॑र्.शपूर्णमा॒सौ यज॑ते । \newline
39. द॒र्॒.श॒पू॒र्ण॒मा॒सौ यज॑ते॒ यज॑ते दर्.शपूर्णमा॒सौ द॑र्.शपूर्णमा॒सौ यज॑ते॒ सेन्द्रौ॒ सेन्द्रौ॒ यज॑ते दर्.शपूर्णमा॒सौ द॑र्.शपूर्णमा॒सौ यज॑ते॒ सेन्द्रौ᳚ । \newline
40. द॒र्॒.श॒पू॒र्ण॒मा॒साविति॑ दर्.श - पू॒र्ण॒मा॒सौ । \newline
41. यज॑ते॒ सेन्द्रौ॒ सेन्द्रौ॒ यज॑ते॒ यज॑ते॒ सेन्द्रा॑ वे॒वैव सेन्द्रौ॒ यज॑ते॒ यज॑ते॒ सेन्द्रा॑ वे॒व । \newline
42. सेन्द्रा॑ वे॒वैव सेन्द्रौ॒ सेन्द्रा॑ वे॒वैना॑ वेना वे॒व सेन्द्रौ॒ सेन्द्रा॑ वे॒वैनौ᳚ । \newline
43. सेन्द्रा॒विति॒ स - इ॒न्द्रौ॒ । \newline
44. ए॒वैना॑ वेना वे॒वैवैनौ॑ यजते यजत एना वे॒वैवैनौ॑ यजते । \newline
45. ए॒नौ॒ य॒ज॒ते॒ य॒ज॒त॒ ए॒ना॒ वे॒नौ॒ य॒ज॒ते॒ श्वःश्वः॒ श्वःश्वो॑ यजत एना वेनौ यजते॒ श्वःश्वः॑ । \newline
46. य॒ज॒ते॒ श्वःश्वः॒ श्वःश्वो॑ यजते यजते॒ श्वःश्वो᳚ ऽस्मा अस्मै॒ श्वःश्वो॑ यजते यजते॒ श्वःश्वो᳚ ऽस्मै । \newline
47. श्वःश्वो᳚ ऽस्मा अस्मै॒ श्वःश्वः॒ श्वःश्वो᳚ ऽस्मा ईजा॒ना ये॑जा॒नाया᳚स्मै॒ श्वःश्वः॒ श्वःश्वो᳚ ऽस्मा ईजा॒नाय॑ । \newline
48. श्वःश्व॒ इति॒ श्वः - श्वः॒ । \newline
49. अ॒स्मा॒ ई॒जा॒ना ये॑जा॒नाया᳚स्मा अस्मा ईजा॒नाय॒ वसी॑यो॒ वसी॑य ईजा॒नाया᳚स्मा अस्मा ईजा॒नाय॒ वसी॑यः । \newline
50. ई॒जा॒नाय॒ वसी॑यो॒ वसी॑य ईजा॒ना ये॑जा॒नाय॒ वसी॑यो भवति भवति॒ वसी॑य ईजा॒ना ये॑जा॒नाय॒ वसी॑यो भवति । \newline
51. वसी॑यो भवति भवति॒ वसी॑यो॒ वसी॑यो भवति दे॒वा दे॒वा भ॑वति॒ वसी॑यो॒ वसी॑यो भवति दे॒वाः । \newline
52. भ॒व॒ति॒ दे॒वा दे॒वा भ॑वति भवति दे॒वा वै वै दे॒वा भ॑वति भवति दे॒वा वै । \newline
53. दे॒वा वै वै दे॒वा दे॒वा वै यद् यद् वै दे॒वा दे॒वा वै यत् । \newline
54. वै यद् यद् वै वै यद् य॒ज्ञे य॒ज्ञे यद् वै वै यद् य॒ज्ञे । \newline
55. यद् य॒ज्ञे य॒ज्ञे यद् यद् य॒ज्ञे ऽकु॑र्व॒ता कु॑र्वत य॒ज्ञे यद् यद् य॒ज्ञे ऽकु॑र्वत । \newline
56. य॒ज्ञे ऽकु॑र्व॒ता कु॑र्वत य॒ज्ञे य॒ज्ञे ऽकु॑र्वत॒ तत् तदकु॑र्वत य॒ज्ञे य॒ज्ञे ऽकु॑र्वत॒ तत् । \newline
57. अकु॑र्वत॒ तत् तदकु॑र्व॒ता कु॑र्वत॒ तदसु॑रा॒ असु॑रा॒ स्तदकु॑र्व॒ता कु॑र्वत॒ तदसु॑राः । \newline
58. तदसु॑रा॒ असु॑रा॒ स्तत् तदसु॑रा अकुर्वता कुर्व॒तासु॑रा॒ स्तत् तदसु॑रा अकुर्वत । \newline
59. असु॑रा अकुर्वता कुर्व॒तासु॑रा॒ असु॑रा अकुर्वत॒ ते ते॑ ऽकुर्व॒ता सु॑रा॒ असु॑रा अकुर्वत॒ ते । \newline
60. अ॒कु॒र्व॒त॒ ते ते॑ ऽकुर्वता कुर्वत॒ ते दे॒वा दे॒वास्ते॑ ऽकुर्वता कुर्वत॒ ते दे॒वाः । \newline
61. ते दे॒वा दे॒वा स्ते ते दे॒वा ए॒ता मे॒ताम् दे॒वा स्ते ते दे॒वा ए॒ताम् । \newline
62. दे॒वा ए॒ता मे॒ताम् दे॒वा दे॒वा ए॒ता मिष्टि॒ मिष्टि॑ मे॒ताम् दे॒वा दे॒वा ए॒ता मिष्टि᳚म् । \newline
63. ए॒ता मिष्टि॒ मिष्टि॑ मे॒ता मे॒ता मिष्टि॑ मपश्यन् नपश्य॒न् निष्टि॑ मे॒ता मे॒ता मिष्टि॑ मपश्यन्न् । \newline
\pagebreak
\markright{ TS 2.5.4.2  \hfill https://www.vedavms.in \hfill}
\addcontentsline{toc}{section}{ TS 2.5.4.2 }
\section*{ TS 2.5.4.2 }

\textbf{TS 2.5.4.2 } \newline
\textbf{Samhita Paata} \newline

-मिष्टि॑-मपश्यन्-नाग्नावैष्ण॒व-मेका॑दशकपालꣳ॒॒ सर॑स्वत्यै च॒रुꣳ सर॑स्वते च॒रुं तां पौ᳚र्णमा॒सꣳ सꣳ॒॒स्थाप्यानु॒ निर॑वप॒न् ततो॑ दे॒वा अभ॑व॒न् पराऽसु॑रा॒ यो भ्रातृ॑व्यवा॒न्थ्स्याथ् स पौ᳚र्णमा॒सꣳ स॒ꣳ॒॒स्थाप्यै॒तामिष्टि॒मनु॒ निर्व॑पेत् पौर्णमा॒सेनै॒व वज्रं॒ भ्रातृ॑व्याय प्र॒हृत्या᳚ऽऽग्नावैष्ण॒वेन॑ दे॒वता᳚श्च य॒ज्ञ्ं च॒ भ्रातृ॑व्यस्य वृङ्क्ते मिथु॒नान् प॒शून्थ् सा॑रस्व॒ताभ्यां॒ ॅयाव॑दे॒वास्यास्ति॒ तथ् - [  ] \newline

\textbf{Pada Paata} \newline

इष्टि᳚म् । अ॒प॒श्य॒न्न् । आ॒ग्ना॒वै॒ष्ण॒वमित्या᳚ग्ना - वै॒ष्ण॒वम् । एका॑दशकपाल॒मित्येका॑दश - क॒पा॒ला॒म् । सर॑स्वत्यै । च॒रुम् । सर॑स्वते । च॒रुम् । ताम् । पौ॒र्ण॒मा॒समिति॑ पौर्ण - मा॒सम् । सꣳ॒॒स्थाप्येति॑ सं - स्थाप्य॑ । अनु॑ । निरिति॑ । अ॒व॒प॒न्न् । ततः॑ । दे॒वाः । अभ॑वन्न् । परेति॑ । असु॑राः । यः । भ्रातृ॑व्यवा॒निति॒ भ्रातृ॑व्य - वा॒न् । स्यात् । सः । पौ॒र्ण॒मा॒समिति॑ पौर्ण - मा॒सम् । सꣳ॒॒स्थाप्येति॑ सं-स्थाप्य॑ । ए॒ताम् । इष्टि᳚म् । अनु॑ । निरिति॑ । व॒पे॒त् । पौ॒र्ण॒मा॒सेनेति॑ पौर्ण - मा॒सेन॑ । ए॒व । वज्र᳚म् । भ्रातृ॑व्याय । प्र॒हृत्येति॑ प्र - हृत्य॑ । आ॒ग्ना॒वै॒ष्ण॒वेनेत्या᳚ग्ना-वै॒ष्ण॒वेन॑ । दे॒वताः᳚ । च॒ । य॒ज्ञ्म् । च॒ । भ्रातृ॑व्यस्य । वृ॒ङ्क्ते॒ । मि॒थु॒नान् । प॒शून् । सा॒र॒स्व॒ताभ्या᳚म् । याव॑त् । ए॒व । अ॒स्य॒ । अस्ति॑ । तत् ।  \newline


\textbf{Krama Paata} \newline

इष्टि॑मपश्यन्न् । अ॒प॒श्य॒न्ना॒ग्ना॒वै॒ष्ण॒वम् । आ॒ग्ना॒वै॒ष्ण॒वमेका॑दशकपालम् । आ॒ग्ना॒वै॒ष्ण॒वमित्या᳚ग्ना - वै॒ष्ण॒वम् । एका॑दशकपालꣳ॒॒ सर॑स्वत्यै । एका॑दशकपाल॒मित्येका॑दश - क॒पा॒ल॒म् । सर॑स्वत्यै च॒रुम् । च॒रुꣳ सर॑स्वते । सर॑स्वते च॒रुम् । च॒रुम् ताम् । ताम् पौ᳚र्णमा॒सम् । पौ॒र्ण॒मा॒सꣳ सꣳ॒॒स्थाप्य॑ । पौ॒र्ण॒मा॒समिति॑ पौर्ण - मा॒सम् । सꣳ॒॒स्थाप्यानु॑ । सꣳ॒॒स्थाप्येति॑ सम् - स्थाप्य॑ । अनु॒ निः । निर॑वपन्न् । अ॒व॒प॒न् ततः॑ । ततो॑ दे॒वाः । दे॒वा अभ॑वन्न् । अभ॑व॒न् परा᳚ । परा ऽसु॑राः । असु॑रा॒ यः । यो भ्रातृ॑व्यवान् । भ्रातृ॑व्यवा॒न्थ् स्यात् । भ्रातृ॑व्यवा॒निति॒ भ्रातृ॑व्य - वा॒न्॒ । स्याथ् सः । स पौ᳚र्णमा॒सम् । पौ॒र्ण॒मा॒सꣳ सꣳ॒॒स्थाप्य॑ । पौ॒र्ण॒मा॒समिति॑ पौर्ण - मा॒सम् । सꣳ॒॒स्थाप्यै॒ताम् । सꣳ॒॒स्थाप्येति॑ सम् - स्थाप्य॑ । ए॒तामिष्टि᳚म् । इष्टि॒मनु॑ । अनु॒ निः । निर् व॑पेत् । व॒पे॒त् पौ॒र्ण॒मा॒सेन॑ । पौ॒र्ण॒मा॒सेनै॒व । पौ॒र्ण॒मा॒सेनेति॑ पौर्ण - मा॒सेन॑ । ए॒व वज्र᳚म् । वज्र॒म् भ्रातृ॑व्याय । भ्रातृ॑व्याय प्र॒हृत्य॑ । प्र॒हृत्या᳚ग्नावैष्ण॒वेन॑ । प्र॒हृत्येति॑ प्र - हृत्य॑ । आ॒ग्ना॒वै॒ष्ण॒वेन॑ दे॒वताः᳚ । आ॒ग्ना॒वै॒ष्ण॒वेनेत्या᳚ग्ना - वै॒ष्ण॒वेन॑ । दे॒वता᳚श्च । च॒ य॒ज्ञ्म् । य॒ज्ञ्म् च॑ । च॒ भ्रातृ॑व्यस्य । भ्रातृ॑व्यस्य वृङ्क्ते । वृ॒ङ्क्ते॒ मि॒थु॒नान् । मि॒थु॒नान् प॒शून् । प॒शून्थ् सा॑रस्व॒ताभ्या᳚म् । सा॒र॒स्व॒ताभ्या॒म् ॅयाव॑त् । याव॑दे॒व । ए॒वास्य॑ । अ॒स्यास्ति॑ । अस्ति॒ तत् । तथ् सर्व᳚म् \newline

\textbf{Jatai Paata} \newline

1. इष्टि॑ मपश्यन् नपश्य॒न् निष्टि॒ मिष्टि॑ मपश्यन्न् । \newline
2. अ॒प॒श्य॒न् ना॒ग्ना॒वै॒ष्ण॒व मा᳚ग्नावैष्ण॒व म॑पश्यन् नपश्यन् नाग्नावैष्ण॒वम् । \newline
3. आ॒ग्ना॒वै॒ष्ण॒व मेका॑दशकपाल॒ मेका॑दशकपाल माग्नावैष्ण॒व मा᳚ग्नावैष्ण॒व मेका॑दशकपालम् । \newline
4. आ॒ग्ना॒वै॒ष्ण॒वमित्या᳚ग्ना - वै॒ष्ण॒वम् । \newline
5. एका॑दशकपालꣳ॒॒ सर॑स्वत्यै॒ सर॑स्वत्या॒ एका॑दशकपाल॒ मेका॑दशकपालꣳ॒॒ सर॑स्वत्यै । \newline
6. एका॑दशकपाल॒मित्येका॑दश - क॒पा॒ल॒म् । \newline
7. सर॑स्वत्यै च॒रुम् च॒रुꣳ सर॑स्वत्यै॒ सर॑स्वत्यै च॒रुम् । \newline
8. च॒रुꣳ सर॑स्वते॒ सर॑स्वते च॒रुम् च॒रुꣳ सर॑स्वते । \newline
9. सर॑स्वते च॒रुम् च॒रुꣳ सर॑स्वते॒ सर॑स्वते च॒रुम् । \newline
10. च॒रुम् ताम् ताम् च॒रुम् च॒रुम् ताम् । \newline
11. ताम् पौ᳚र्णमा॒सम् पौ᳚र्णमा॒सम् ताम् ताम् पौ᳚र्णमा॒सम् । \newline
12. पौ॒र्ण॒मा॒सꣳ सꣳ॒॒स्थाप्य॑ सꣳ॒॒स्थाप्य॑ पौर्णमा॒सम् पौ᳚र्णमा॒सꣳ सꣳ॒॒स्थाप्य॑ । \newline
13. पौ॒र्ण॒मा॒समिति॑ पौर्ण - मा॒सम् । \newline
14. सꣳ॒॒स्थाप्यान्वनु॑ सꣳ॒॒स्थाप्य॑ सꣳ॒॒स्थाप्यानु॑ । \newline
15. सꣳ॒॒स्थाप्येति॑ सं - स्थाप्य॑ । \newline
16. अनु॒ निर् णि रन्वनु॒ निः । \newline
17. निर॑वपन् नवप॒न् निर् णिर॑वपन्न् । \newline
18. अ॒व॒प॒न् तत॒ स्ततो॑ ऽवपन् नवप॒न् ततः॑ । \newline
19. ततो॑ दे॒वा दे॒वा स्तत॒ स्ततो॑ दे॒वाः । \newline
20. दे॒वा अभ॑व॒न् नभ॑वन् दे॒वा दे॒वा अभ॑वन्न् । \newline
21. अभ॑व॒न् परा॒ परा ऽभ॑व॒न् नभ॑व॒न् परा᳚ । \newline
22. परा ऽसु॑रा॒ असु॑राः॒ परा॒ परा ऽसु॑राः । \newline
23. असु॑रा॒ यो यो ऽसु॑रा॒ असु॑रा॒ यः । \newline
24. यो भ्रातृ॑व्यवा॒न् भ्रातृ॑व्यवा॒न्॒. यो यो भ्रातृ॑व्यवान् । \newline
25. भ्रातृ॑व्यवा॒न् थ्स्याथ् स्याद् भ्रातृ॑व्यवा॒न् भ्रातृ॑व्यवा॒न् थ्स्यात् । \newline
26. भ्रातृ॑व्यवा॒निति॒ भ्रातृ॑व्य - वा॒न् । \newline
27. स्याथ् स स स्याथ् स्याथ् सः । \newline
28. स पौ᳚र्णमा॒सम् पौ᳚र्णमा॒सꣳ स स पौ᳚र्णमा॒सम् । \newline
29. पौ॒र्ण॒मा॒सꣳ सꣳ॒॒स्थाप्य॑ सꣳ॒॒स्थाप्य॑ पौर्णमा॒सम् पौ᳚र्णमा॒सꣳ सꣳ॒॒स्थाप्य॑ । \newline
30. पौ॒र्ण॒मा॒समिति॑ पौर्ण - मा॒सम् । \newline
31. सꣳ॒॒स्थाप्यै॒ता मे॒ताꣳ सꣳ॒॒स्थाप्य॑ सꣳ॒॒स्थाप्यै॒ताम् । \newline
32. सꣳ॒॒स्थाप्येति॑ सं - स्थाप्य॑ । \newline
33. ए॒ता मिष्टि॒ मिष्टि॑ मे॒ता मे॒ता मिष्टि᳚म् । \newline
34. इष्टि॒ मन्वन्विष्टि॒ मिष्टि॒ मनु॑ । \newline
35. अनु॒ निर् णिरन्वनु॒ निः । \newline
36. निर् व॑पेद् वपे॒न् निर् णिर् व॑पेत् । \newline
37. व॒पे॒त् पौ॒र्ण॒मा॒सेन॑ पौर्णमा॒सेन॑ वपेद् वपेत् पौर्णमा॒सेन॑ । \newline
38. पौ॒र्ण॒मा॒सेनै॒वैव पौ᳚र्णमा॒सेन॑ पौर्णमा॒सेनै॒व । \newline
39. पौ॒र्ण॒मा॒सेनेति॑ पौर्ण - मा॒सेन॑ । \newline
40. ए॒व वज्रं॒ ॅवज्र॑ मे॒वैव वज्र᳚म् । \newline
41. वज्र॒म् भ्रातृ॑व्याय॒ भ्रातृ॑व्याय॒ वज्रं॒ ॅवज्र॒म् भ्रातृ॑व्याय । \newline
42. भ्रातृ॑व्याय प्र॒हृत्य॑ प्र॒हृत्य॒ भ्रातृ॑व्याय॒ भ्रातृ॑व्याय प्र॒हृत्य॑ । \newline
43. प्र॒हृत्या᳚ग्नावैष्ण॒वेना᳚ ग्नावैष्ण॒वेन॑ प्र॒हृत्य॑ प्र॒हृत्या᳚ग्नावैष्ण॒वेन॑ । \newline
44. प्र॒हृत्येति॑ प्र - हृत्य॑ । \newline
45. आ॒ग्ना॒वै॒ष्ण॒वेन॑ दे॒वता॑ दे॒वता॑ आग्नावैष्ण॒वेना᳚ ग्नावैष्ण॒वेन॑ दे॒वताः᳚ । \newline
46. आ॒ग्ना॒वै॒ष्ण॒वेनेत्या᳚ग्ना - वै॒ष्ण॒वेन॑ । \newline
47. दे॒वता᳚श्च च दे॒वता॑ दे॒वता᳚श्च । \newline
48. च॒ य॒ज्ञ्ं ॅय॒ज्ञ्म् च॑ च य॒ज्ञ्म् । \newline
49. य॒ज्ञ्म् च॑ च य॒ज्ञ्ं ॅय॒ज्ञ्म् च॑ । \newline
50. च॒ भ्रातृ॑व्यस्य॒ भ्रातृ॑व्यस्य च च॒ भ्रातृ॑व्यस्य । \newline
51. भ्रातृ॑व्यस्य वृङ्क्ते वृङ्क्ते॒ भ्रातृ॑व्यस्य॒ भ्रातृ॑व्यस्य वृङ्क्ते । \newline
52. वृ॒ङ्क्ते॒ मि॒थु॒नान् मि॑थु॒नान् वृ॑ङ्क्ते वृङ्क्ते मिथु॒नान् । \newline
53. मि॒थु॒नान् प॒शून् प॒शून् मि॑थु॒नान् मि॑थु॒नान् प॒शून् । \newline
54. प॒शून् थ्सा॑रस्व॒ताभ्याꣳ॑ सारस्व॒ताभ्या᳚म् प॒शून् प॒शून् थ्सा॑रस्व॒ताभ्या᳚म् । \newline
55. सा॒र॒स्व॒ताभ्यां॒ ॅयाव॒द् याव॑थ् सारस्व॒ताभ्याꣳ॑ सारस्व॒ताभ्यां॒ ॅयाव॑त् । \newline
56. याव॑दे॒वैव याव॒द् याव॑दे॒व । \newline
57. ए॒वास्या᳚ स्यै॒वै वास्य॑ । \newline
58. अ॒स्यास्त्य स्त्य॑स्या॒स्यास्ति॑ । \newline
59. अस्ति॒ तत् तदस्त्यस्ति॒ तत् । \newline
60. तथ् सर्वꣳ॒॒ सर्व॒म् तत् तथ् सर्व᳚म् । \newline

\textbf{Ghana Paata } \newline

1. इष्टि॑ मपश्यन् नपश्य॒न् निष्टि॒ मिष्टि॑ मपश्यन् नाग्नावैष्ण॒व मा᳚ग्नावैष्ण॒व म॑पश्य॒न् निष्टि॒ मिष्टि॑ मपश्यन् नाग्नावैष्ण॒वम् । \newline
2. अ॒प॒श्य॒न् ना॒ग्ना॒वै॒ष्ण॒व मा᳚ग्नावैष्ण॒व म॑पश्यन् नपश्यन् नाग्नावैष्ण॒व मेका॑दशकपाल॒ मेका॑दशकपाल माग्नावैष्ण॒व म॑पश्यन् नपश्यन् नाग्नावैष्ण॒व मेका॑दशकपालम् । \newline
3. आ॒ग्ना॒वै॒ष्ण॒व मेका॑दशकपाल॒ मेका॑दशकपाल माग्नावैष्ण॒व मा᳚ग्नावैष्ण॒व मेका॑दशकपालꣳ॒॒ सर॑स्वत्यै॒ सर॑स्वत्या॒ एका॑दशकपाल माग्नावैष्ण॒व मा᳚ग्नावैष्ण॒व मेका॑दशकपालꣳ॒॒ सर॑स्वत्यै । \newline
4. आ॒ग्ना॒वै॒ष्ण॒वमित्या᳚ग्ना - वै॒ष्ण॒वम् । \newline
5. एका॑दशकपालꣳ॒॒ सर॑स्वत्यै॒ सर॑स्वत्या॒ एका॑दशकपाल॒ मेका॑दशकपालꣳ॒॒ सर॑स्वत्यै च॒रुम् च॒रुꣳ सर॑स्वत्या॒ एका॑दशकपाल॒ मेका॑दशकपालꣳ॒॒ सर॑स्वत्यै च॒रुम् । \newline
6. एका॑दशकपाल॒मित्येका॑दश - क॒पा॒ल॒म् । \newline
7. सर॑स्वत्यै च॒रुम् च॒रुꣳ सर॑स्वत्यै॒ सर॑स्वत्यै च॒रुꣳ सर॑स्वते॒ सर॑स्वते च॒रुꣳ सर॑स्वत्यै॒ सर॑स्वत्यै च॒रुꣳ सर॑स्वते । \newline
8. च॒रुꣳ सर॑स्वते॒ सर॑स्वते च॒रुम् च॒रुꣳ सर॑स्वते च॒रुम् च॒रुꣳ सर॑स्वते च॒रुम् च॒रुꣳ सर॑स्वते च॒रुम् । \newline
9. सर॑स्वते च॒रुम् च॒रुꣳ सर॑स्वते॒ सर॑स्वते च॒रुम् ताम् ताम् च॒रुꣳ सर॑स्वते॒ सर॑स्वते च॒रुम् ताम् । \newline
10. च॒रुम् ताम् ताम् च॒रुम् च॒रुम् ताम् पौ᳚र्णमा॒सम् पौ᳚र्णमा॒सम् ताम् च॒रुम् च॒रुम् ताम् पौ᳚र्णमा॒सम् । \newline
11. ताम् पौ᳚र्णमा॒सम् पौ᳚र्णमा॒सम् ताम् ताम् पौ᳚र्णमा॒सꣳ सꣳ॒॒स्थाप्य॑ सꣳ॒॒स्थाप्य॑ पौर्णमा॒सम् ताम् ताम् पौ᳚र्णमा॒सꣳ सꣳ॒॒स्थाप्य॑ । \newline
12. पौ॒र्ण॒मा॒सꣳ सꣳ॒॒स्थाप्य॑ सꣳ॒॒स्थाप्य॑ पौर्णमा॒सम् पौ᳚र्णमा॒सꣳ सꣳ॒॒स्थाप्यान्वनु॑ सꣳ॒॒स्थाप्य॑ पौर्णमा॒सम् पौ᳚र्णमा॒सꣳ सꣳ॒॒स्थाप्यानु॑ । \newline
13. पौ॒र्ण॒मा॒समिति॑ पौर्ण - मा॒सम् । \newline
14. सꣳ॒॒स्थाप्यान्वनु॑ सꣳ॒॒स्थाप्य॑ सꣳ॒॒स्थाप्यानु॒ निर् णिरनु॑ सꣳ॒॒स्थाप्य॑ सꣳ॒॒स्थाप्यानु॒ निः । \newline
15. सꣳ॒॒स्थाप्येति॑ सं - स्थाप्य॑ । \newline
16. अनु॒ निर् णिरन्वनु॒ निर॑वपन् नवप॒न् निरन्वनु॒ निर॑वपन्न् । \newline
17. निर॑वपन् नवप॒न् निर् णिर॑वप॒न् तत॒ स्ततो॑ ऽवप॒न् निर् णिर॑वप॒न् ततः॑ । \newline
18. अ॒व॒प॒न् तत॒ स्ततो॑ ऽवपन् नवप॒न् ततो॑ दे॒वा दे॒वा स्ततो॑ ऽवपन् नवप॒न् ततो॑ दे॒वाः । \newline
19. ततो॑ दे॒वा दे॒वा स्तत॒ स्ततो॑ दे॒वा अभ॑व॒न् नभ॑वन् दे॒वा स्तत॒ स्ततो॑ दे॒वा अभ॑वन्न् । \newline
20. दे॒वा अभ॑व॒न् नभ॑वन् दे॒वा दे॒वा अभ॑व॒न् परा॒ परा ऽभ॑वन् दे॒वा दे॒वा अभ॑व॒न् परा᳚ । \newline
21. अभ॑व॒न् परा॒ परा ऽभ॑व॒न् नभ॑व॒न् परा ऽसु॑रा॒ असु॑राः॒ परा ऽभ॑व॒न् नभ॑व॒न् परा ऽसु॑राः । \newline
22. परा ऽसु॑रा॒ असु॑राः॒ परा॒ परा ऽसु॑रा॒ यो यो ऽसु॑राः॒ परा॒ परा ऽसु॑रा॒ यः । \newline
23. असु॑रा॒ यो यो ऽसु॑रा॒ असु॑रा॒ यो भ्रातृ॑व्यवा॒न् भ्रातृ॑व्यवा॒न्॒. यो ऽसु॑रा॒ असु॑रा॒ यो भ्रातृ॑व्यवान् । \newline
24. यो भ्रातृ॑व्यवा॒न् भ्रातृ॑व्यवा॒न्॒. यो यो भ्रातृ॑व्यवा॒न् थ्स्याथ् स्याद् भ्रातृ॑व्यवा॒न्॒. यो यो भ्रातृ॑व्यवा॒न् थ्स्यात् । \newline
25. भ्रातृ॑व्यवा॒न् थ्स्याथ् स्याद् भ्रातृ॑व्यवा॒न् भ्रातृ॑व्यवा॒न् थ्स्याथ् स स स्याद् भ्रातृ॑व्यवा॒न् भ्रातृ॑व्यवा॒न् थ्स्याथ् सः । \newline
26. भ्रातृ॑व्यवा॒निति॒ भ्रातृ॑व्य - वा॒न् । \newline
27. स्याथ् स स स्याथ् स्याथ् स पौ᳚र्णमा॒सम् पौ᳚र्णमा॒सꣳ स स्याथ् स्याथ् स पौ᳚र्णमा॒सम् । \newline
28. स पौ᳚र्णमा॒सम् पौ᳚र्णमा॒सꣳ स स पौ᳚र्णमा॒सꣳ सꣳ॒॒स्थाप्य॑ सꣳ॒॒स्थाप्य॑ पौर्णमा॒सꣳ स स पौ᳚र्णमा॒सꣳ सꣳ॒॒स्थाप्य॑ । \newline
29. पौ॒र्ण॒मा॒सꣳ सꣳ॒॒स्थाप्य॑ सꣳ॒॒स्थाप्य॑ पौर्णमा॒सम् पौ᳚र्णमा॒सꣳ सꣳ॒॒स्थाप्यै॒ता मे॒ताꣳ सꣳ॒॒स्थाप्य॑ पौर्णमा॒सम् पौ᳚र्णमा॒सꣳ सꣳ॒॒स्थाप्यै॒ताम् । \newline
30. पौ॒र्ण॒मा॒समिति॑ पौर्ण - मा॒सम् । \newline
31. सꣳ॒॒स्थाप्यै॒ता मे॒ताꣳ सꣳ॒॒स्थाप्य॑ सꣳ॒॒स्थाप्यै॒ता मिष्टि॒ मिष्टि॑ मे॒ताꣳ सꣳ॒॒स्थाप्य॑ सꣳ॒॒स्थाप्यै॒ता मिष्टि᳚म् । \newline
32. सꣳ॒॒स्थाप्येति॑ सं - स्थाप्य॑ । \newline
33. ए॒ता मिष्टि॒ मिष्टि॑ मे॒ता मे॒ता मिष्टि॒ मन्वन्विष्टि॑ मे॒ता मे॒ता मिष्टि॒ मनु॑ । \newline
34. इष्टि॒ मन्वन्विष्टि॒ मिष्टि॒ मनु॒ निर् णिरन्विष्टि॒ मिष्टि॒ मनु॒ निः । \newline
35. अनु॒ निर् णिरन्वनु॒ निर् व॑पेद् वपे॒न् निरन्वनु॒ निर् व॑पेत् । \newline
36. निर् व॑पेद् वपे॒न् निर् णिर् व॑पेत् पौर्णमा॒सेन॑ पौर्णमा॒सेन॑ वपे॒न् निर् णिर् व॑पेत् पौर्णमा॒सेन॑ । \newline
37. व॒पे॒त् पौ॒र्ण॒मा॒सेन॑ पौर्णमा॒सेन॑ वपेद् वपेत् पौर्णमा॒सेनै॒वैव पौ᳚र्णमा॒सेन॑ वपेद् वपेत् पौर्णमा॒सेनै॒व । \newline
38. पौ॒र्ण॒मा॒सेनै॒वैव पौ᳚र्णमा॒सेन॑ पौर्णमा॒सेनै॒व वज्रं॒ ॅवज्र॑ मे॒व पौ᳚र्णमा॒सेन॑ पौर्णमा॒सेनै॒व वज्र᳚म् । \newline
39. पौ॒र्ण॒मा॒सेनेति॑ पौर्ण - मा॒सेन॑ । \newline
40. ए॒व वज्रं॒ ॅवज्र॑ मे॒वैव वज्र॒म् भ्रातृ॑व्याय॒ भ्रातृ॑व्याय॒ वज्र॑ मे॒वैव वज्र॒म् भ्रातृ॑व्याय । \newline
41. वज्र॒म् भ्रातृ॑व्याय॒ भ्रातृ॑व्याय॒ वज्रं॒ ॅवज्र॒म् भ्रातृ॑व्याय प्र॒हृत्य॑ प्र॒हृत्य॒ भ्रातृ॑व्याय॒ वज्रं॒ ॅवज्र॒म् भ्रातृ॑व्याय प्र॒हृत्य॑ । \newline
42. भ्रातृ॑व्याय प्र॒हृत्य॑ प्र॒हृत्य॒ भ्रातृ॑व्याय॒ भ्रातृ॑व्याय प्र॒हृत्या᳚ ग्नावैष्ण॒वेना᳚ ग्नावैष्ण॒वेन॑ प्र॒हृत्य॒ भ्रातृ॑व्याय॒ भ्रातृ॑व्याय प्र॒हृत्या᳚ ग्नावैष्ण॒वेन॑ । \newline
43. प्र॒हृत्या᳚ग्नावैष्ण॒वेना᳚ ग्नावैष्ण॒वेन॑ प्र॒हृत्य॑ प्र॒हृत्या᳚ग्नावैष्ण॒वेन॑ दे॒वता॑ दे॒वता॑ आग्नावैष्ण॒वेन॑ प्र॒हृत्य॑ प्र॒हृत्या᳚ग्नावैष्ण॒वेन॑ दे॒वताः᳚ । \newline
44. प्र॒हृत्येति॑ प्र - हृत्य॑ । \newline
45. आ॒ग्ना॒वै॒ष्ण॒वेन॑ दे॒वता॑ दे॒वता॑ आग्नावैष्ण॒वेना᳚ ग्नावैष्ण॒वेन॑ दे॒वता᳚श्च च दे॒वता॑ आग्नावैष्ण॒वेना᳚ग्नावैष्ण॒वेन॑ दे॒वता᳚श्च । \newline
46. आ॒ग्ना॒वै॒ष्ण॒वेनेत्या᳚ग्ना - वै॒ष्ण॒वेन॑ । \newline
47. दे॒वता᳚श्च च दे॒वता॑ दे॒वता᳚श्च य॒ज्ञ्ं ॅय॒ज्ञ्म् च॑ दे॒वता॑ दे॒वता᳚श्च य॒ज्ञ्म् । \newline
48. च॒ य॒ज्ञ्ं ॅय॒ज्ञ्म् च॑ च य॒ज्ञ्म् च॑ च य॒ज्ञ्म् च॑ च य॒ज्ञ्म् च॑ । \newline
49. य॒ज्ञ्म् च॑ च य॒ज्ञ्ं ॅय॒ज्ञ्म् च॒ भ्रातृ॑व्यस्य॒ भ्रातृ॑व्यस्य च य॒ज्ञ्ं ॅय॒ज्ञ्म् च॒ भ्रातृ॑व्यस्य । \newline
50. च॒ भ्रातृ॑व्यस्य॒ भ्रातृ॑व्यस्य च च॒ भ्रातृ॑व्यस्य वृङ्क्ते वृङ्क्ते॒ भ्रातृ॑व्यस्य च च॒ भ्रातृ॑व्यस्य वृङ्क्ते । \newline
51. भ्रातृ॑व्यस्य वृङ्क्ते वृङ्क्ते॒ भ्रातृ॑व्यस्य॒ भ्रातृ॑व्यस्य वृङ्क्ते मिथु॒नान् मि॑थु॒नान् वृ॑ङ्क्ते॒ भ्रातृ॑व्यस्य॒ भ्रातृ॑व्यस्य वृङ्क्ते मिथु॒नान् । \newline
52. वृ॒ङ्क्ते॒ मि॒थु॒नान् मि॑थु॒नान् वृ॑ङ्क्ते वृङ्क्ते मिथु॒नान् प॒शून् प॒शून् मि॑थु॒नान् वृ॑ङ्क्ते वृङ्क्ते मिथु॒नान् प॒शून् । \newline
53. मि॒थु॒नान् प॒शून् प॒शून् मि॑थु॒नान् मि॑थु॒नान् प॒शून् थ्सा॑रस्व॒ताभ्याꣳ॑ सारस्व॒ताभ्या᳚म् प॒शून् मि॑थु॒नान् मि॑थु॒नान् प॒शून् थ्सा॑रस्व॒ताभ्या᳚म् । \newline
54. प॒शून् थ्सा॑रस्व॒ताभ्याꣳ॑ सारस्व॒ताभ्या᳚म् प॒शून् प॒शून् थ्सा॑रस्व॒ताभ्यां॒ ॅयाव॒द् याव॑थ् सारस्व॒ताभ्या᳚म् प॒शून् प॒शून् थ्सा॑रस्व॒ताभ्यां॒ ॅयाव॑त् । \newline
55. सा॒र॒स्व॒ताभ्यां॒ ॅयाव॒द् याव॑थ् सारस्व॒ताभ्याꣳ॑ सारस्व॒ताभ्यां॒ ॅयाव॑दे॒वैव याव॑थ् सारस्व॒ताभ्याꣳ॑ सारस्व॒ताभ्यां॒ ॅयाव॑दे॒व । \newline
56. याव॑दे॒वैव याव॒द् याव॑दे॒वास्या᳚ स्यै॒व याव॒द् याव॑दे॒वास्य॑ । \newline
57. ए॒वास्या᳚ स्यै॒वैवास्या स्त्यस्त्य॑ स्यै॒वैवा स्यास्ति॑ । \newline
58. अ॒स्या स्त्यस्त्य॑ स्या॒स्यास्ति॒ तत् तदस्त्य॑ स्या॒स्यास्ति॒ तत् । \newline
59. अस्ति॒ तत् तदस्त्यस्ति॒ तथ् सर्वꣳ॒॒ सर्व॒म् तदस्त्यस्ति॒ तथ् सर्व᳚म् । \newline
60. तथ् सर्वꣳ॒॒ सर्व॒म् तत् तथ् सर्वं॑ ॅवृङ्क्ते वृङ्क्ते॒ सर्व॒म् तत् तथ् सर्वं॑ ॅवृङ्क्ते । \newline
\pagebreak
\markright{ TS 2.5.4.3  \hfill https://www.vedavms.in \hfill}
\addcontentsline{toc}{section}{ TS 2.5.4.3 }
\section*{ TS 2.5.4.3 }

\textbf{TS 2.5.4.3 } \newline
\textbf{Samhita Paata} \newline

सर्वं॑ ॅवृङ्क्ते पौर्णमा॒सीमे॒व य॑जेत॒ भ्रातृ॑व्यवा॒न्नामा॑वा॒स्याꣳ॑ ह॒त्वा भ्रातृ॑व्यं॒ नाऽऽप्या॑ययति साकंप्रस्था॒यीये॑न यजेत प॒शुका॑मो॒यस्मै॒ वा अल्पे॑ना॒ऽऽहर॑न्ति॒ नाऽऽत्मना॒ तृप्य॑ति॒ नान्यस्मै॑ ददाति॒ यस्मै॑ मह॒ता तृप्य॑त्या॒त्मना॒ ददा᳚त्य॒न्यस्मै॑ मह॒ता पू॒र्णꣳ हो॑त॒व्यं॑ तृ॒प्त ए॒वैन॒मिन्द्रः॑ प्र॒जया॑ प॒शुभि॑स्तर्पयति दारुपा॒त्रेण॑ जुहोति॒ न हि मृ॒न्मय॒माहु॑तिमान॒श औदु॑म्बरं - [  ] \newline

\textbf{Pada Paata} \newline

सर्व᳚म् । वृ॒ङ्क्ते॒ । पौ॒र्ण॒मा॒सीमिति॑ पौर्ण - मा॒सीम् । ए॒व । य॒जे॒त॒ । भ्रातृ॑व्यवा॒निति॒ भ्रातृ॑व्य - वा॒न् । न । अ॒मा॒वा॒स्या॑मित्य॑मा-वा॒स्या᳚म् । ह॒त्वा । भ्रातृ॑व्यम् । न । एति॑ । प्या॒य॒य॒ति॒ । सा॒क॒प्रं॒स्था॒यीये॒नेति॑ साकं - प्र॒स्था॒यीये॑न । य॒जे॒त॒ । प॒शुका॑म॒ इति॑ प॒शु - का॒मः॒ । यस्मै᳚ । वै । अल्पे॑न । आ॒हर॒न्तीत्या᳚ - हर॑न्ति । न । आ॒त्मना᳚ । तृप्य॑ति । न । अ॒न्यस्मै᳚ । द॒दा॒ति॒ । यस्मै᳚ । म॒ह॒ता । तृप्य॑ति । आ॒त्मना᳚ । ददा॑ति । अ॒न्यस्मै᳚ । म॒ह॒ता । पू॒र्णम् । हो॒त॒व्य᳚म् । तृ॒प्तः । ए॒व । ए॒न॒म् । इन्द्रः॑ । प्र॒जयेति॑ प्र - जया᳚ । प॒शुभि॒रिति॑ प॒शु-भिः॒ । त॒र्प॒य॒ति॒ । दा॒रु॒पा॒त्रेणेति॑ दारु - पा॒त्रेण॑ । जु॒हो॒ति॒ । न । हि । मृ॒न्मय॒मिति॑ मृत् - मय᳚म् । आहु॑ति॒मित्या - हु॒ति॒म् । आ॒न॒शे । औदु॑बंरम् ।  \newline


\textbf{Krama Paata} \newline

सर्व॑म् ॅवृङ्क्ते । वृ॒ङ्क्ते॒ पौ॒र्ण॒मा॒सीम् । पौ॒र्ण॒मा॒सीमे॒व । पौ॒र्ण॒मा॒सीमिति॑ पौर्ण - मा॒सीम् । ए॒व य॑जेत । य॒जे॒त॒ भ्रातृ॑व्यवान् । भ्रातृ॑व्यवा॒न् न । भ्रातृ॑व्यवा॒निति॒ भ्रातृ॑व्य - वा॒न्॒ । नामा॑वा॒स्या᳚म् । अ॒मा॒वा॒स्याꣳ॑ ह॒त्वा । अ॒मा॒वा॒स्या॑मित्य॑मा - वा॒स्या᳚म् । ह॒त्वा भ्रातृ॑व्यम् । भ्रातृ॑व्य॒म् न । ना । आ प्या॑ययति । प्या॒य॒य॒ति॒ सा॒क॒म्प्र॒स्था॒यीये॑न । सा॒क॒म्प्र॒स्था॒यीये॑न यजेत । सा॒क॒म्प्र॒स्था॒यीये॒नेति॑ साकम् - प्र॒स्था॒यीये॑न । य॒जे॒त॒ प॒शुका॑मः । प॒शुका॑मो॒ यस्मै᳚ । प॒शुका॑म॒ इति॑ प॒शु - का॒मः॒ । यस्मै॒ वै । वा अल्पे॑न । अल्पे॑ना॒हर॑न्ति । आ॒हर॑न्ति॒ न । आ॒हर॒न्तीत्या᳚ - हर॑न्ति । नात्मना᳚ । आ॒त्मना॒ तृप्य॑ति । तृप्य॑ति॒ न । नान्यस्मै᳚ । अ॒न्यस्मै॑ ददाति । द॒दा॒ति॒ यस्मै᳚ । यस्मै॑ मह॒ता । म॒ह॒ता तृप्य॑ति । तृप्य॑त्या॒त्मना᳚ । आ॒त्मना॒ ददा॑ति । ददा᳚त्य॒न्यस्मै᳚ । अ॒न्यस्मै॑ मह॒ता । म॒ह॒ता पू॒र्णम् । पू॒र्णꣳ हो॑त॒व्य᳚म् । हो॒त॒व्य॑म् तृ॒प्तः । तृ॒प्त ए॒व । ए॒वैन᳚म् । ए॒न॒मिन्द्रः॑ । इन्द्रः॑ प्र॒जया᳚ । प्र॒जया॑ प॒शुभिः॑ । प्र॒जयेति॑ प्र - जया᳚ । प॒शुभि॑स्तर्पयति । प॒शुभि॒रिति॑ प॒शु - भिः॒ । त॒र्प॒य॒ति॒ दा॒रु॒पा॒त्रेण॑ । दा॒रु॒पा॒त्रेण॑ जुहोति । दा॒रु॒पा॒त्रेणेति॑ दारु - पा॒त्रेण॑ । जु॒हो॒ति॒ न । न हि । हि मृ॒न्मय᳚म् । मृ॒न्मय॒माहु॑तिम् । मृ॒न्मय॒मिति॑ मृत् - मय᳚म् । आहु॑तिमान॒शे । आहु॑ति॒मित्या - हु॒ति॒म् । आ॒न॒श औदु॑म्बरम् । 
औदु॑म्बरम् भवति \newline

\textbf{Jatai Paata} \newline

1. सर्वं॑ ॅवृङ्क्ते वृङ्क्ते॒ सर्वꣳ॒॒ सर्वं॑ ॅवृङ्क्ते । \newline
2. वृ॒ङ्क्ते॒ पौ॒र्ण॒मा॒सीम् पौ᳚र्णमा॒सीं ॅवृ॑ङ्क्ते वृङ्क्ते पौर्णमा॒सीम् । \newline
3. पौ॒र्ण॒मा॒सी मे॒वैव पौ᳚र्णमा॒सीम् पौ᳚र्णमा॒सी मे॒व । \newline
4. पौ॒र्ण॒मा॒सीमिति॑ पौर्ण - मा॒सीम् । \newline
5. ए॒व य॑जेत यजेतै॒वैव य॑जेत । \newline
6. य॒जे॒त॒ भ्रातृ॑व्यवा॒न् भ्रातृ॑व्यवान्. यजेत यजेत॒ भ्रातृ॑व्यवान् । \newline
7. भ्रातृ॑व्यवा॒न् न न भ्रातृ॑व्यवा॒न् भ्रातृ॑व्यवा॒न् न । \newline
8. भ्रातृ॑व्यवा॒निति॒ भ्रातृ॑व्य - वा॒न् । \newline
9. नामा॑वा॒स्या॑ ममावा॒स्या᳚म् न नामा॑वा॒स्या᳚म् । \newline
10. अ॒मा॒वा॒स्याꣳ॑ ह॒त्वा ह॒त्वा ऽमा॑वा॒स्या॑ ममावा॒स्याꣳ॑ ह॒त्वा । \newline
11. अ॒मा॒वा॒स्या॑मित्य॑मा - वा॒स्या᳚म् । \newline
12. ह॒त्वा भ्रातृ॑व्य॒म् भ्रातृ॑व्यꣳ ह॒त्वा ह॒त्वा भ्रातृ॑व्यम् । \newline
13. भ्रातृ॑व्य॒म् न न भ्रातृ॑व्य॒म् भ्रातृ॑व्य॒म् न । \newline
14. ना न ना । \newline
15. आ प्या॑ययति प्यायय॒त्या प्या॑ययति । \newline
16. प्या॒य॒य॒ति॒ सा॒कं॒प्र॒स्था॒यीये॑न साकंप्रस्था॒यीये॑न प्याययति प्याययति साकंप्रस्था॒यीये॑न । \newline
17. सा॒कं॒प्र॒स्था॒यीये॑न यजेत यजेत साकंप्रस्था॒यीये॑न साकंप्रस्था॒यीये॑न यजेत । \newline
18. सा॒कं॒प्र॒स्था॒यीये॒नेति॑ साकं - प्र॒स्था॒यीये॑न । \newline
19. य॒जे॒त॒ प॒शुका॑मः प॒शुका॑मो यजेत यजेत प॒शुका॑मः । \newline
20. प॒शुका॑मो॒ यस्मै॒ यस्मै॑ प॒शुका॑मः प॒शुका॑मो॒ यस्मै᳚ । \newline
21. प॒शुका॑म॒ इति॑ प॒शु - का॒मः॒ । \newline
22. यस्मै॒ वै वै यस्मै॒ यस्मै॒ वै । \newline
23. वा अल्पे॒ना ल्पे॑न॒ वै वा अल्पे॑न । \newline
24. अल्पे॑ना॒ हर॑ न्त्या॒हर॒ न्त्यल्पे॒ना ल्पे॑ना॒ हर॑न्ति । \newline
25. आ॒हर॑न्ति॒ न नाहर॑ न्त्या॒हर॑न्ति॒ न । \newline
26. आ॒हर॒न्तीत्या᳚ - हर॑न्ति । \newline
27. नात्मना॒ ऽऽत्मना॒ न नात्मना᳚ । \newline
28. आ॒त्मना॒ तृप्य॑ति॒ तृप्य॑ त्या॒त्मना॒ ऽऽत्मना॒ तृप्य॑ति । \newline
29. तृप्य॑ति॒ न न तृप्य॑ति॒ तृप्य॑ति॒ न । \newline
30. नान्यस्मा॑ अ॒न्यस्मै॒ न नान्यस्मै᳚ । \newline
31. अ॒न्यस्मै॑ ददाति ददा त्य॒न्यस्मा॑ अ॒न्यस्मै॑ ददाति । \newline
32. द॒दा॒ति॒ यस्मै॒ यस्मै॑ ददाति ददाति॒ यस्मै᳚ । \newline
33. यस्मै॑ मह॒ता म॑ह॒ता यस्मै॒ यस्मै॑ मह॒ता । \newline
34. म॒ह॒ता तृप्य॑ति॒ तृप्य॑ति मह॒ता म॑ह॒ता तृप्य॑ति । \newline
35. तृप्य॑ त्या॒त्मना॒ ऽऽत्मना॒ तृप्य॑ति॒ तृप्य॑ त्या॒त्मना᳚ । \newline
36. आ॒त्मना॒ ददा॑ति॒ ददा᳚ त्या॒त्मना॒ ऽऽत्मना॒ ददा॑ति । \newline
37. ददा᳚ त्य॒न्यस्मा॑ अ॒न्यस्मै॒ ददा॑ति॒ ददा᳚ त्य॒न्यस्मै᳚ । \newline
38. अ॒न्यस्मै॑ मह॒ता म॑ह॒ता ऽन्यस्मा॑ अ॒न्यस्मै॑ मह॒ता । \newline
39. म॒ह॒ता पू॒र्णम् पू॒र्णम् म॑ह॒ता म॑ह॒ता पू॒र्णम् । \newline
40. पू॒र्णꣳ हो॑त॒व्यꣳ॑ होत॒व्य॑म् पू॒र्णम् पू॒र्णꣳ हो॑त॒व्य᳚म् । \newline
41. हो॒त॒व्य॑म् तृ॒प्त स्तृ॒प्तो हो॑त॒व्यꣳ॑ होत॒व्य॑म् तृ॒प्तः । \newline
42. तृ॒प्त ए॒वैव तृ॒प्त स्तृ॒प्त ए॒व । \newline
43. ए॒वैन॑ मेन मे॒वैवैन᳚म् । \newline
44. ए॒न॒ मिन्द्र॒ इन्द्र॑ एण मेन॒ मिन्द्रः॑ । \newline
45. इन्द्रः॑ प्र॒जया᳚ प्र॒जयेन्द्र॒ इन्द्रः॑ प्र॒जया᳚ । \newline
46. प्र॒जया॑ प॒शुभिः॑ प॒शुभिः॑ प्र॒जया᳚ प्र॒जया॑ प॒शुभिः॑ । \newline
47. प्र॒जयेति॑ प्र - जया᳚ । \newline
48. प॒शुभि॑ स्तर्पयति तर्पयति प॒शुभिः॑ प॒शुभि॑ स्तर्पयति । \newline
49. प॒शुभि॒रिति॑ प॒शु - भिः॒ । \newline
50. त॒र्प॒य॒ति॒ दा॒रु॒पा॒त्रेण॑ दारुपा॒त्रेण॑ तर्पयति तर्पयति दारुपा॒त्रेण॑ । \newline
51. दा॒रु॒पा॒त्रेण॑ जुहोति जुहोति दारुपा॒त्रेण॑ दारुपा॒त्रेण॑ जुहोति । \newline
52. दा॒रु॒पा॒त्रेणेति॑ दारु - पा॒त्रेण॑ । \newline
53. जु॒हो॒ति॒ न न जु॑होति जुहोति॒ न । \newline
54. न हि हि न न हि । \newline
55. हि मृ॒न्मय॑म् मृ॒न्मयꣳ॒॒ हि हि मृ॒न्मय᳚म् । \newline
56. मृ॒न्मय॒ माहु॑ति॒ माहु॑तिम् मृ॒न्मय॑म् मृ॒न्मय॒ माहु॑तिम् । \newline
57. मृ॒न्मय॒मिति॑ मृत् - मय᳚म् । \newline
58. आहु॑ति मान॒श आ॑न॒श आहु॑ति॒ माहु॑ति मान॒शे । \newline
59. आहु॑ति॒मित्या - हु॒ति॒म् । \newline
60. आ॒न॒श औदुं॑बर॒ मौदुं॑बर मान॒श आ॑न॒श औदुं॑बरम् । \newline
61. औदुं॑बरम् भवति भव॒त्यौदुं॑बर॒ मौदुं॑बरम् भवति । \newline

\textbf{Ghana Paata } \newline

1. सर्वं॑ ॅवृङ्क्ते वृङ्क्ते॒ सर्वꣳ॒॒ सर्वं॑ ॅवृङ्क्ते पौर्णमा॒सीम् पौ᳚र्णमा॒सीं ॅवृ॑ङ्क्ते॒ सर्वꣳ॒॒ सर्वं॑ ॅवृङ्क्ते पौर्णमा॒सीम् । \newline
2. वृ॒ङ्क्ते॒ पौ॒र्ण॒मा॒सीम् पौ᳚र्णमा॒सीं ॅवृ॑ङ्क्ते वृङ्क्ते पौर्णमा॒सी मे॒वैव पौ᳚र्णमा॒सीं ॅवृ॑ङ्क्ते वृङ्क्ते पौर्णमा॒सी मे॒व । \newline
3. पौ॒र्ण॒मा॒सी मे॒वैव पौ᳚र्णमा॒सीम् पौ᳚र्णमा॒सी मे॒व य॑जेत यजेतै॒व पौ᳚र्णमा॒सीम् पौ᳚र्णमा॒सी मे॒व य॑जेत । \newline
4. पौ॒र्ण॒मा॒सीमिति॑ पौर्ण - मा॒सीम् । \newline
5. ए॒व य॑जेत यजेतै॒वैव य॑जेत॒ भ्रातृ॑व्यवा॒न् भ्रातृ॑व्यवान्. यजेतै॒वैव य॑जेत॒ भ्रातृ॑व्यवान् । \newline
6. य॒जे॒त॒ भ्रातृ॑व्यवा॒न् भ्रातृ॑व्यवान्. यजेत यजेत॒ भ्रातृ॑व्यवा॒न् न न भ्रातृ॑व्यवान्. यजेत यजेत॒ भ्रातृ॑व्यवा॒न् न । \newline
7. भ्रातृ॑व्यवा॒न् न न भ्रातृ॑व्यवा॒न् भ्रातृ॑व्यवा॒न् नामा॑वा॒स्या॑ ममावा॒स्या᳚न्न भ्रातृ॑व्यवा॒न् भ्रातृ॑व्यवा॒न् नामा॑वा॒स्या᳚म् । \newline
8. भ्रातृ॑व्यवा॒निति॒ भ्रातृ॑व्य - वा॒न् । \newline
9. नामा॑वा॒स्या॑ ममावा॒स्या᳚न्न नामा॑वा॒स्याꣳ॑ ह॒त्वा ह॒त्वा ऽमा॑वा॒स्या᳚न्न नामा॑वा॒स्याꣳ॑ ह॒त्वा । \newline
10. अ॒मा॒वा॒स्याꣳ॑ ह॒त्वा ह॒त्वा ऽमा॑वा॒स्या॑ ममावा॒स्याꣳ॑ ह॒त्वा भ्रातृ॑व्य॒म् भ्रातृ॑व्यꣳ ह॒त्वा ऽमा॑वा॒स्या॑ ममावा॒स्याꣳ॑ ह॒त्वा भ्रातृ॑व्यम् । \newline
11. अ॒मा॒वा॒स्या॑मित्य॑मा - वा॒स्या᳚म् । \newline
12. ह॒त्वा भ्रातृ॑व्य॒म् भ्रातृ॑व्यꣳ ह॒त्वा ह॒त्वा भ्रातृ॑व्य॒न्न न भ्रातृ॑व्यꣳ ह॒त्वा ह॒त्वा भ्रातृ॑व्य॒न्न । \newline
13. भ्रातृ॑व्य॒म् न न भ्रातृ॑व्य॒म् भ्रातृ॑व्य॒म् ना न भ्रातृ॑व्य॒म् भ्रातृ॑व्य॒म् ना । \newline
14. ना न ना प्या॑ययति प्यायय॒त्या न ना प्या॑ययति । \newline
15. आ प्या॑ययति प्यायय॒त्या प्या॑ययति साकंप्रस्था॒यीये॑न साकंप्रस्था॒यीये॑न प्यायय॒त्या प्या॑ययति साकंप्रस्था॒यीये॑न । \newline
16. प्या॒य॒य॒ति॒ सा॒कं॒प्र॒स्था॒यीये॑न साकंप्रस्था॒यीये॑न प्याययति प्याययति साकंप्रस्था॒यीये॑न यजेत यजेत साकंप्रस्था॒यीये॑न प्याययति प्याययति साकंप्रस्था॒यीये॑न यजेत । \newline
17. सा॒कं॒प्र॒स्था॒यीये॑न यजेत यजेत साकंप्रस्था॒यीये॑न साकंप्रस्था॒यीये॑न यजेत प॒शुका॑मः प॒शुका॑मो यजेत साकंप्रस्था॒यीये॑न साकंप्रस्था॒यीये॑न यजेत प॒शुका॑मः । \newline
18. सा॒कं॒प्र॒स्था॒यीये॒नेति॑ साकं - प्र॒स्था॒यीये॑न । \newline
19. य॒जे॒त॒ प॒शुका॑मः प॒शुका॑मो यजेत यजेत प॒शुका॑मो॒ यस्मै॒ यस्मै॑ प॒शुका॑मो यजेत यजेत प॒शुका॑मो॒ यस्मै᳚ । \newline
20. प॒शुका॑मो॒ यस्मै॒ यस्मै॑ प॒शुका॑मः प॒शुका॑मो॒ यस्मै॒ वै वै यस्मै॑ प॒शुका॑मः प॒शुका॑मो॒ यस्मै॒ वै । \newline
21. प॒शुका॑म॒ इति॑ प॒शु - का॒मः॒ । \newline
22. यस्मै॒ वै वै यस्मै॒ यस्मै॒ वा अल्पे॒नाल्पे॑न॒ वै यस्मै॒ यस्मै॒ वा अल्पे॑न । \newline
23. वा अल्पे॒नाल्पे॑न॒ वै वा अल्पे॑ना॒ हर॑न्त्या॒हर॒ न्त्यल्पे॑न॒ वै वा अल्पे॑ना॒ हर॑न्ति । \newline
24. अल्पे॑ना॒ हर॑न्त्या॒हर॒ न्त्यल्पे॒नाल्पे॑ना॒ हर॑न्ति॒ न नाहर॒ न्त्यल्पे॒नाल्पे॑ना॒ हर॑न्ति॒ न । \newline
25. आ॒हर॑न्ति॒ न नाहर॑न्त्या॒ हर॑न्ति॒ नात्मना॒ ऽऽत्मना॒ नाहर॑न्त्या॒ हर॑न्ति॒ नात्मना᳚ । \newline
26. आ॒हर॒न्तीत्या᳚ - हर॑न्ति । \newline
27. नात्मना॒ ऽऽत्मना॒ न नात्मना॒ तृप्य॑ति॒ तृप्य॑ त्या॒त्मना॒ न नात्मना॒ तृप्य॑ति । \newline
28. आ॒त्मना॒ तृप्य॑ति॒ तृप्य॑ त्या॒त्मना॒ ऽऽत्मना॒ तृप्य॑ति॒ न न तृप्य॑ त्या॒त्मना॒ ऽऽत्मना॒ तृप्य॑ति॒ न । \newline
29. तृप्य॑ति॒ न न तृप्य॑ति॒ तृप्य॑ति॒ नान्यस्मा॑ अ॒न्यस्मै॒ न तृप्य॑ति॒ तृप्य॑ति॒ नान्यस्मै᳚ । \newline
30. नान्यस्मा॑ अ॒न्यस्मै॒ न नान्यस्मै॑ ददाति ददात्य॒न्यस्मै॒ न नान्यस्मै॑ ददाति । \newline
31. अ॒न्यस्मै॑ ददाति ददा त्य॒न्यस्मा॑ अ॒न्यस्मै॑ ददाति॒ यस्मै॒ यस्मै॑ ददा त्य॒न्यस्मा॑ अ॒न्यस्मै॑ ददाति॒ यस्मै᳚ । \newline
32. द॒दा॒ति॒ यस्मै॒ यस्मै॑ ददाति ददाति॒ यस्मै॑ मह॒ता म॑ह॒ता यस्मै॑ ददाति ददाति॒ यस्मै॑ मह॒ता । \newline
33. यस्मै॑ मह॒ता म॑ह॒ता यस्मै॒ यस्मै॑ मह॒ता तृप्य॑ति॒ तृप्य॑ति मह॒ता यस्मै॒ यस्मै॑ मह॒ता तृप्य॑ति । \newline
34. म॒ह॒ता तृप्य॑ति॒ तृप्य॑ति मह॒ता म॑ह॒ता तृप्य॑ त्या॒त्मना॒ ऽऽत्मना॒ तृप्य॑ति मह॒ता म॑ह॒ता तृप्य॑ त्या॒त्मना᳚ । \newline
35. तृप्य॑ त्या॒त्मना॒ ऽऽत्मना॒ तृप्य॑ति॒ तृप्य॑ त्या॒त्मना॒ ददा॑ति॒ ददा᳚त्या॒त्मना॒ तृप्य॑ति॒ तृप्य॑ त्या॒त्मना॒ ददा॑ति । \newline
36. आ॒त्मना॒ ददा॑ति॒ ददा᳚ त्या॒त्मना॒ ऽऽत्मना॒ ददा᳚ त्य॒न्यस्मा॑ अ॒न्यस्मै॒ ददा᳚ त्या॒त्मना॒ ऽऽत्मना॒ ददा᳚ त्य॒न्यस्मै᳚ । \newline
37. ददा᳚ त्य॒न्यस्मा॑ अ॒न्यस्मै॒ ददा॑ति॒ ददा᳚ त्य॒न्यस्मै॑ मह॒ता म॑ह॒ता ऽन्यस्मै॒ ददा॑ति॒ ददा᳚ त्य॒न्यस्मै॑ मह॒ता । \newline
38. अ॒न्यस्मै॑ मह॒ता म॑ह॒ता ऽन्यस्मा॑ अ॒न्यस्मै॑ मह॒ता पू॒र्णम् पू॒र्णम् म॑ह॒ता ऽन्यस्मा॑ अ॒न्यस्मै॑ मह॒ता पू॒र्णम् । \newline
39. म॒ह॒ता पू॒र्णम् पू॒र्णम् म॑ह॒ता म॑ह॒ता पू॒र्णꣳ हो॑त॒व्यꣳ॑ होत॒व्य॑म् पू॒र्णम् म॑ह॒ता म॑ह॒ता पू॒र्णꣳ हो॑त॒व्य᳚म् । \newline
40. पू॒र्णꣳ हो॑त॒व्यꣳ॑ होत॒व्य॑म् पू॒र्णम् पू॒र्णꣳ हो॑त॒व्य॑म् तृ॒प्त स्तृ॒प्तो हो॑त॒व्य॑म् पू॒र्णम् पू॒र्णꣳ हो॑त॒व्य॑म् तृ॒प्तः । \newline
41. हो॒त॒व्य॑म् तृ॒प्त स्तृ॒प्तो हो॑त॒व्यꣳ॑ होत॒व्य॑म् तृ॒प्त ए॒वैव तृ॒प्तो हो॑त॒व्यꣳ॑ होत॒व्य॑म् तृ॒प्त ए॒व । \newline
42. तृ॒प्त ए॒वैव तृ॒प्त स्तृ॒प्त ए॒वैन॑ मेन मे॒व तृ॒प्त स्तृ॒प्त ए॒वैन᳚म् । \newline
43. ए॒वैन॑ मेन मे॒वैवैन॒ मिन्द्र॒ इन्द्र॑ एण मे॒वैवैन॒ मिन्द्रः॑ । \newline
44. ए॒न॒ मिन्द्र॒ इन्द्र॑ एण मेन॒ मिन्द्रः॑ प्र॒जया᳚ प्र॒जयेन्द्र॑ एण मेन॒ मिन्द्रः॑ प्र॒जया᳚ । \newline
45. इन्द्रः॑ प्र॒जया᳚ प्र॒जयेन्द्र॒ इन्द्रः॑ प्र॒जया॑ प॒शुभिः॑ प॒शुभिः॑ प्र॒जयेन्द्र॒ इन्द्रः॑ प्र॒जया॑ प॒शुभिः॑ । \newline
46. प्र॒जया॑ प॒शुभिः॑ प॒शुभिः॑ प्र॒जया᳚ प्र॒जया॑ प॒शुभि॑ स्तर्पयति तर्पयति प॒शुभिः॑ प्र॒जया᳚ प्र॒जया॑ प॒शुभि॑ स्तर्पयति । \newline
47. प्र॒जयेति॑ प्र - जया᳚ । \newline
48. प॒शुभि॑ स्तर्पयति तर्पयति प॒शुभिः॑ प॒शुभि॑ स्तर्पयति दारुपा॒त्रेण॑ दारुपा॒त्रेण॑ तर्पयति प॒शुभिः॑ प॒शुभि॑ स्तर्पयति दारुपा॒त्रेण॑ । \newline
49. प॒शुभि॒रिति॑ प॒शु - भिः॒ । \newline
50. त॒र्प॒य॒ति॒ दा॒रु॒पा॒त्रेण॑ दारुपा॒त्रेण॑ तर्पयति तर्पयति दारुपा॒त्रेण॑ जुहोति जुहोति दारुपा॒त्रेण॑ तर्पयति तर्पयति दारुपा॒त्रेण॑ जुहोति । \newline
51. दा॒रु॒पा॒त्रेण॑ जुहोति जुहोति दारुपा॒त्रेण॑ दारुपा॒त्रेण॑ जुहोति॒ न न जु॑होति दारुपा॒त्रेण॑ दारुपा॒त्रेण॑ जुहोति॒ न । \newline
52. दा॒रु॒पा॒त्रेणेति॑ दारु - पा॒त्रेण॑ । \newline
53. जु॒हो॒ति॒ न न जु॑होति जुहोति॒ न हि हि न जु॑होति जुहोति॒ न हि । \newline
54. न हि हि न न हि मृ॒न्मय॑म् मृ॒न्मयꣳ॒॒ हि न न हि मृ॒न्मय᳚म् । \newline
55. हि मृ॒न्मय॑म् मृ॒न्मयꣳ॒॒ हि हि मृ॒न्मय॒ माहु॑ति॒ माहु॑तिम् मृ॒न्मयꣳ॒॒ हि हि मृ॒न्मय॒ माहु॑तिम् । \newline
56. मृ॒न्मय॒ माहु॑ति॒ माहु॑तिम् मृ॒न्मय॑म् मृ॒न्मय॒ माहु॑ति मान॒श आ॑न॒श आहु॑तिम् मृ॒न्मय॑म् मृ॒न्मय॒ माहु॑ति मान॒शे । \newline
57. मृ॒न्मय॒मिति॑ मृत् - मय᳚म् । \newline
58. आहु॑ति मान॒श आ॑न॒श आहु॑ति॒ माहु॑ति मान॒श औदुं॑बर॒ मौदुं॑बर मान॒श आहु॑ति॒ माहु॑ति मान॒श औदुं॑बरम् । \newline
59. आहु॑ति॒मित्या - हु॒ति॒म् । \newline
60. आ॒न॒श औदुं॑बर॒ मौदुं॑बर मान॒श आ॑न॒श औदुं॑बरम् भवति भव॒त्यौदुं॑बर मान॒श आ॑न॒श औदुं॑बरम् भवति । \newline
61. औदुं॑बरम् भवति भव॒त्यौदुं॑बर॒ मौदुं॑बरम् भव॒त्यूर् गूर्ग् भ॑व॒त्यौदुं॑बर॒ मौदुं॑बरम् भव॒त्यूर्क् । \newline
\pagebreak
\markright{ TS 2.5.4.4  \hfill https://www.vedavms.in \hfill}
\addcontentsline{toc}{section}{ TS 2.5.4.4 }
\section*{ TS 2.5.4.4 }

\textbf{TS 2.5.4.4 } \newline
\textbf{Samhita Paata} \newline

भव॒त्यूर्ग्वा उ॑दु॒म्बर॒ ऊर्क् प॒शव॑ ऊ॒र्जैवास्मा॒ ऊर्जं॑ प॒शूनव॑ रुन्धे॒ नाग॑तश्रीर्महे॒न्द्रं ॅय॑जेत॒ त्रयो॒ वै ग॒तश्रि॑यः शुश्रु॒वान् ग्रा॑म॒णी रा॑ज॒न्य॑स्तेषां᳚ महे॒न्द्रो दे॒वता॒ यो वै स्वां दे॒वता॑मति॒ यज॑ते॒ प्रस्वायै॑ दे॒वता॑यैच्यवते॒ न परां॒ प्राप्नो॑ति॒ पापी॑यान् भवति संॅवथ्स॒र-मिन्द्रं॑ ॅयजेत संॅवथ्स॒रꣳ हि व्र॒तं नाति॒ स्वै - [  ] \newline

\textbf{Pada Paata} \newline

भ॒व॒ति॒ । ऊर्क् । वै । उ॒दु॒बंरः॑ । ऊर्क् । प॒शवः॑ । ऊ॒र्जा । ए॒व । अ॒स्मै॒ । ऊर्ज᳚म् । प॒शून् । अवेति॑ । रु॒न्धे॒ । न । अग॑तश्री॒रित्यग॑त - श्रीः॒ । म॒हे॒न्द्रमिति॑ महा - इ॒न्द्रम् । य॒जे॒त॒ । त्रयः॑ । वै । ग॒तश्रि॑य॒ इति॑ ग॒त - श्रि॒यः॒ । शु॒श्रु॒वान् । ग्रा॒म॒णीरिति॑ ग्राम - नीः । रा॒ज॒न्यः॑ । तेषा᳚म् । म॒हे॒न्द्र इति॑ महा-इ॒न्द्रः । दे॒वता᳚ । यः । वै । स्वाम् । दे॒वता᳚म् । अ॒ति॒यज॑त॒ इत्य॑ति - यज॑ते । प्रेति॑ । स्वायै᳚ । दे॒वता॑यै । च्य॒व॒ते॒ । न । परा᳚म् । प्रेति॑ । आ॒प्नो॒ति॒ । पापी॑यान् । भ॒व॒ति॒ । सं॒ॅव॒थ्स॒रमिति॑ सं-व॒थ्स॒रम् । इन्द्र᳚म् । य॒जे॒त॒ । सं॒ॅव॒थ्स॒रमिति॑ सं - व॒थ्स॒रम् । हि । व्र॒तम् । न । अतीति॑ । स्वा ।  \newline


\textbf{Krama Paata} \newline

भ॒व॒त्यूर्क् । ऊर्ग्वै । वा उ॑दु॒म्बरः॑ । उ॒दु॒म्बर॒ ऊर्क् । ऊर्क् प॒शवः॑ । प॒शव॑ ऊ॒र्जा । ऊ॒र्जैव । ए॒वास्मै᳚ । अ॒स्मा॒ ऊर्ज᳚म् । ऊर्ज॑म् प॒शून् । प॒शूनव॑ । अव॑ रुन्धे । रु॒न्धे॒ न । नाग॑तश्रीः । अग॑तश्रीर् महे॒न्द्रम् । अग॑तश्री॒रित्यग॑त - श्रीः॒ । म॒हे॒न्द्रम् ॅय॑जेत । म॒हे॒न्द्रमिति॑ महा - इ॒न्द्रम् । य॒जे॒त॒ त्रयः॑ । त्रयो॒ वै । वै ग॒तश्रि॑यः । ग॒तश्रि॑यः शुश्रु॒वान् । ग॒तश्रि॑य॒ इति॑ ग॒त - श्रि॒यः॒ । शु॒श्रु॒वान् ग्रा॑म॒णीः । ग्रा॒म॒णी रा॑ज॒न्यः॑ । ग्रा॒म॒णीरिति॑ ग्राम - नीः । रा॒ज॒न्य॑स्तेषा᳚म् । तेषा᳚म् महे॒न्द्रः । म॒हे॒न्द्रो दे॒वता᳚ । म॒हे॒न्द्र इति॑ महा - इ॒न्द्रः । दे॒वता॒ यः । यो वै । वै स्वाम् । स्वाम् दे॒वता᳚म् । दे॒वता॑मति॒यज॑ते । अ॒ति॒यज॑ते॒ प्र । अ॒ति॒यज॑त॒ इत्य॑ति - यज॑ते । प्र स्वायै᳚ । स्वायै॑ दे॒वता॑यै । दे॒वता॑यै च्यवते । च्य॒व॒ते॒ न । न परा᳚म् । परा॒म् प्र । प्राप्नो॑ति । आ॒प्नो॒ति॒ पापी॑यान् । पापी॑यान् भवति । भ॒व॒ति॒ स॒म्ॅव॒थ्स॒रम् । स॒म्ॅव॒थ्स॒रमिन्द्र᳚म् । स॒म्ॅव॒थ्स॒रमिति॑ सम् - व॒थ्स॒रम् । इन्द्र॑म् ॅयजेत । य॒जे॒त॒ स॒म्ॅव॒थ्स॒रम् । स॒म्ॅव॒थ्स॒रꣳ हि । स॒म्ॅव॒थ्स॒रमिति॑ सम् - व॒थ्स॒रम् । हि व्र॒तम् । व्र॒तम् न । नाति॑ । अति॒ स्वा ( ) । स्वैव \newline

\textbf{Jatai Paata} \newline

1. भ॒व॒ त्यूर् गूर्ग् भ॑वति भव॒ त्यूर्क् । \newline
2. ऊर्ग् वै वा ऊर् गूर्ग् वै । \newline
3. वा उ॑दुं॒बर॑ उदुं॒बरो॒ वै वा उ॑दुं॒बरः॑ । \newline
4. उ॒दुं॒बर॒ ऊर् गूर् गु॑दुं॒बर॑ उदुं॒बर॒ ऊर्क् । \newline
5. ऊर्क् प॒शवः॑ प॒शव॒ ऊर् गूर्क् प॒शवः॑ । \newline
6. प॒शव॑ ऊ॒र्जोर्जा प॒शवः॑ प॒शव॑ ऊ॒र्जा । \newline
7. ऊ॒र्जैवैवो र्जोर्जैव । \newline
8. ए॒वा स्मा॑ अस्मा ए॒वैवा स्मै᳚ । \newline
9. अ॒स्मा॒ ऊर्ज॒ मूर्ज॑ मस्मा अस्मा॒ ऊर्ज᳚म् । \newline
10. ऊर्ज॑म् प॒शून् प॒शू नूर्ज॒ मूर्ज॑म् प॒शून् । \newline
11. प॒शू नवाव॑ प॒शून् प॒शू नव॑ । \newline
12. अव॑ रुन्धे रु॒न्धे ऽवाव॑ रुन्धे । \newline
13. रु॒न्धे॒ न न रु॑न्धे रुन्धे॒ न । \newline
14. नाग॑तश्री॒ रग॑तश्री॒र् न नाग॑तश्रीः । \newline
15. अग॑तश्रीर् महे॒न्द्रम् म॑हे॒न्द्र मग॑तश्री॒ रग॑तश्रीर् महे॒न्द्रम् । \newline
16. अग॑तश्री॒रित्यग॑त - श्रीः॒ । \newline
17. म॒हे॒न्द्रं ॅय॑जेत यजेत महे॒न्द्रम् म॑हे॒न्द्रं ॅय॑जेत । \newline
18. म॒हे॒न्द्रमिति॑ महा - इ॒न्द्रम् । \newline
19. य॒जे॒त॒ त्रय॒ स्त्रयो॑ यजेत यजेत॒ त्रयः॑ । \newline
20. त्रयो॒ वै वै त्रय॒ स्त्रयो॒ वै । \newline
21. वै ग॒तश्रि॑यो ग॒तश्रि॑यो॒ वै वै ग॒तश्रि॑यः । \newline
22. ग॒तश्रि॑यः शुश्रु॒वाञ् छु॑श्रु॒वान् ग॒तश्रि॑यो ग॒तश्रि॑यः शुश्रु॒वान् । \newline
23. ग॒तश्रि॑य॒ इति॑ ग॒त - श्रि॒यः॒ । \newline
24. शु॒श्रु॒वान् ग्रा॑म॒णीर् ग्रा॑म॒णीः शु॑श्रु॒वाञ् छु॑श्रु॒वान् ग्रा॑म॒णीः । \newline
25. ग्रा॒म॒णी रा॑ज॒न्यो॑ राज॒न्यो᳚ ग्राम॒णीर् ग्रा॑म॒णी रा॑ज॒न्यः॑ । \newline
26. ग्रा॒म॒णीरिति॑ ग्राम - नीः । \newline
27. रा॒ज॒न्य॑ स्तेषा॒म् तेषाꣳ॑ राज॒न्यो॑ राज॒न्य॑ स्तेषा᳚म् । \newline
28. तेषा᳚म् महे॒न्द्रो म॑हे॒न्द्र स्तेषा॒म् तेषा᳚म् महे॒न्द्रः । \newline
29. म॒हे॒न्द्रो दे॒वता॑ दे॒वता॑ महे॒न्द्रो म॑हे॒न्द्रो दे॒वता᳚ । \newline
30. म॒हे॒न्द्र इति॑ महा - इ॒न्द्रः । \newline
31. दे॒वता॒ यो यो दे॒वता॑ दे॒वता॒ यः । \newline
32. यो वै वै यो यो वै । \newline
33. वै स्वाꣳ स्वां ॅवै वै स्वाम् । \newline
34. स्वाम् दे॒वता᳚म् दे॒वताꣳ॒॒ स्वाꣳ स्वाम् दे॒वता᳚म् । \newline
35. दे॒वता॑ मति॒यज॑ते ऽति॒यज॑ते दे॒वता᳚म् दे॒वता॑ मति॒यज॑ते । \newline
36. अ॒ति॒यज॑ते॒ प्र प्राति॒यज॑ते ऽति॒यज॑ते॒ प्र । \newline
37. अ॒ति॒यज॑त॒ इत्य॑ति - यज॑ते । \newline
38. प्र स्वायै॒ स्वायै॒ प्र प्र स्वायै᳚ । \newline
39. स्वायै॑ दे॒वता॑यै दे॒वता॑यै॒ स्वायै॒ स्वायै॑ दे॒वता॑यै । \newline
40. दे॒वता॑यै च्यवते च्यवते दे॒वता॑यै दे॒वता॑यै च्यवते । \newline
41. च्य॒व॒ते॒ न न च्य॑वते च्यवते॒ न । \newline
42. न परा॒म् परा॒म् न न परा᳚म् । \newline
43. परा॒म् प्र प्र परा॒म् परा॒म् प्र । \newline
44. प्राप्नो᳚ त्याप्नोति॒ प्र प्राप्नो॑ति । \newline
45. आ॒प्नो॒ति॒ पापी॑या॒न् पापी॑या नाप्नो त्याप्नोति॒ पापी॑यान् । \newline
46. पापी॑यान् भवति भवति॒ पापी॑या॒न् पापी॑यान् भवति । \newline
47. भ॒व॒ति॒ सं॒ॅव॒थ्स॒रꣳ सं॑ॅवथ्स॒रम् भ॑वति भवति संॅवथ्स॒रम् । \newline
48. सं॒ॅव॒थ्स॒र मिन्द्र॒ मिन्द्रꣳ॑ संॅवथ्स॒रꣳ सं॑ॅवथ्स॒र मिन्द्र᳚म् । \newline
49. सं॒ॅव॒थ्स॒रमिति॑ सं - व॒थ्स॒रम् । \newline
50. इन्द्रं॑ ॅयजेत यजे॒ते न्द्र॒ मिन्द्रं॑ ॅयजेत । \newline
51. य॒जे॒त॒ सं॒ॅव॒थ्स॒रꣳ सं॑ॅवथ्स॒रं ॅय॑जेत यजेत संॅवथ्स॒रम् । \newline
52. सं॒ॅव॒थ्स॒रꣳ हि हि सं॑ॅवथ्स॒रꣳ सं॑ॅवथ्स॒रꣳ हि । \newline
53. सं॒ॅव॒थ्स॒रमिति॑ सं - व॒थ्स॒रम् । \newline
54. हि व्र॒तं ॅव्र॒तꣳ हि हि व्र॒तम् । \newline
55. व्र॒तम् न न व्र॒तं ॅव्र॒तम् न । \newline
56. नात्यति॒ न नाति॑ । \newline
57. अति॒ स्वा स्वा ऽत्यति॒ स्वा । \newline
58. स्वैवैव स्वा स्वैव । \newline

\textbf{Ghana Paata } \newline

1. भ॒व॒त्यूर् गूर्ग् भ॑वति भव॒त्यूर्ग् वै वा ऊर्ग् भ॑वति भव॒त्यूर्ग् वै । \newline
2. ऊर्ग् वै वा ऊर् गूर्ग् वा उ॑दुं॒बर॑ उदुं॒बरो॒ वा ऊर् गूर्ग् वा उ॑दुं॒बरः॑ । \newline
3. वा उ॑दुं॒बर॑ उदुं॒बरो॒ वै वा उ॑दुं॒बर॒ ऊर् गूर् गु॑दुं॒बरो॒ वै वा उ॑दुं॒बर॒ ऊर्क् । \newline
4. उ॒दुं॒बर॒ ऊर्गूर् गु॑दुं॒बर॑ उदुं॒बर॒ ऊर्क् प॒शवः॑ प॒शव॒ ऊर्गु॑दुं॒बर॑ उदुं॒बर॒ ऊर्क् प॒शवः॑ । \newline
5. ऊर्क् प॒शवः॑ प॒शव॒ ऊर् गूर्क् प॒शव॑ ऊ॒र्जोर्जा प॒शव॒ ऊर् गूर्क् प॒शव॑ ऊ॒र्जा । \newline
6. प॒शव॑ ऊ॒र्जोर्जा प॒शवः॑ प॒शव॑ ऊ॒र्जै वैवोर्जा प॒शवः॑ प॒शव॑ ऊ॒र्जैव । \newline
7. ऊ॒र्जै वैवोर्जोर् जैवास्मा॑ अस्मा ए॒वोर्जोर् जैवास्मै᳚ । \newline
8. ए॒वास्मा॑ अस्मा ए॒वैवास्मा॒ ऊर्ज॒ मूर्ज॑ मस्मा ए॒वैवास्मा॒ ऊर्ज᳚म् । \newline
9. अ॒स्मा॒ ऊर्ज॒ मूर्ज॑ मस्मा अस्मा॒ ऊर्ज॑म् प॒शून् प॒शू नूर्ज॑ मस्मा अस्मा॒ ऊर्ज॑म् प॒शून् । \newline
10. ऊर्ज॑म् प॒शून् प॒शू नूर्ज॒ मूर्ज॑म् प॒शू नवाव॑ प॒शू नूर्ज॒ मूर्ज॑म् प॒शू नव॑ । \newline
11. प॒शू नवाव॑ प॒शून् प॒शू नव॑ रुन्धे रु॒न्धे ऽव॑ प॒शून् प॒शू नव॑ रुन्धे । \newline
12. अव॑ रुन्धे रु॒न्धे ऽवाव॑ रुन्धे॒ न न रु॒न्धे ऽवाव॑ रुन्धे॒ न । \newline
13. रु॒न्धे॒ न न रु॑न्धे रुन्धे॒ नाग॑तश्री॒ रग॑तश्री॒र् न रु॑न्धे रुन्धे॒ नाग॑तश्रीः । \newline
14. नाग॑तश्री॒ रग॑तश्री॒र् न नाग॑तश्रीर् महे॒न्द्रम् म॑हे॒न्द्र मग॑तश्री॒र् न नाग॑तश्रीर् महे॒न्द्रम् । \newline
15. अग॑तश्रीर् महे॒न्द्रम् म॑हे॒न्द्र मग॑तश्री॒ रग॑तश्रीर् महे॒न्द्रं ॅय॑जेत यजेत महे॒न्द्र मग॑तश्री॒ रग॑तश्रीर् महे॒न्द्रं ॅय॑जेत । \newline
16. अग॑तश्री॒रित्यग॑त - श्रीः॒ । \newline
17. म॒हे॒न्द्रं ॅय॑जेत यजेत महे॒न्द्रम् म॑हे॒न्द्रं ॅय॑जेत॒ त्रय॒ स्त्रयो॑ यजेत महे॒न्द्रम् म॑हे॒न्द्रं ॅय॑जेत॒ त्रयः॑ । \newline
18. म॒हे॒न्द्रमिति॑ महा - इ॒न्द्रम् । \newline
19. य॒जे॒त॒ त्रय॒ स्त्रयो॑ यजेत यजेत॒ त्रयो॒ वै वै त्रयो॑ यजेत यजेत॒ त्रयो॒ वै । \newline
20. त्रयो॒ वै वै त्रय॒ स्त्रयो॒ वै ग॒तश्रि॑यो ग॒तश्रि॑यो॒ वै त्रय॒ स्त्रयो॒ वै ग॒तश्रि॑यः । \newline
21. वै ग॒तश्रि॑यो ग॒तश्रि॑यो॒ वै वै ग॒तश्रि॑यः शुश्रु॒वाञ् छु॑श्रु॒वान् ग॒तश्रि॑यो॒ वै वै ग॒तश्रि॑यः शुश्रु॒वान् । \newline
22. ग॒तश्रि॑यः शुश्रु॒वाञ् छु॑श्रु॒वान् ग॒तश्रि॑यो ग॒तश्रि॑यः शुश्रु॒वान् ग्रा॑म॒णीर् ग्रा॑म॒णीः शु॑श्रु॒वान् ग॒तश्रि॑यो ग॒तश्रि॑यः शुश्रु॒वान् ग्रा॑म॒णीः । \newline
23. ग॒तश्रि॑य॒ इति॑ ग॒त - श्रि॒यः॒ । \newline
24. शु॒श्रु॒वान् ग्रा॑म॒णीर् ग्रा॑म॒णीः शु॑श्रु॒वाञ् छु॑श्रु॒वान् ग्रा॑म॒णी रा॑ज॒न्यो॑ राज॒न्यो᳚ ग्राम॒णीः शु॑श्रु॒वाञ् छु॑श्रु॒वान् ग्रा॑म॒णी रा॑ज॒न्यः॑ । \newline
25. ग्रा॒म॒णी रा॑ज॒न्यो॑ राज॒न्यो᳚ ग्राम॒णीर् ग्रा॑म॒णी रा॑ज॒न्य॑ स्तेषा॒म् तेषाꣳ॑ राज॒न्यो᳚ ग्राम॒णीर् ग्रा॑म॒णी रा॑ज॒न्य॑ स्तेषा᳚म् । \newline
26. ग्रा॒म॒णीरिति॑ ग्राम - नीः । \newline
27. रा॒ज॒न्य॑ स्तेषा॒म् तेषाꣳ॑ राज॒न्यो॑ राज॒न्य॑ स्तेषा᳚म् महे॒न्द्रो म॑हे॒न्द्र स्तेषाꣳ॑ राज॒न्यो॑ राज॒न्य॑ स्तेषा᳚म् महे॒न्द्रः । \newline
28. तेषा᳚म् महे॒न्द्रो म॑हे॒न्द्र स्तेषा॒म् तेषा᳚म् महे॒न्द्रो दे॒वता॑ दे॒वता॑ महे॒न्द्र स्तेषा॒म् तेषा᳚म् महे॒न्द्रो दे॒वता᳚ । \newline
29. म॒हे॒न्द्रो दे॒वता॑ दे॒वता॑ महे॒न्द्रो म॑हे॒न्द्रो दे॒वता॒ यो यो दे॒वता॑ महे॒न्द्रो म॑हे॒न्द्रो दे॒वता॒ यः । \newline
30. म॒हे॒न्द्र इति॑ महा - इ॒न्द्रः । \newline
31. दे॒वता॒ यो यो दे॒वता॑ दे॒वता॒ यो वै वै यो दे॒वता॑ दे॒वता॒ यो वै । \newline
32. यो वै वै यो यो वै स्वाꣳ स्वां ॅवै यो यो वै स्वाम् । \newline
33. वै स्वाꣳ स्वां ॅवै वै स्वाम् दे॒वता᳚म् दे॒वताꣳ॒॒ स्वां ॅवै वै स्वाम् दे॒वता᳚म् । \newline
34. स्वाम् दे॒वता᳚म् दे॒वताꣳ॒॒ स्वाꣳ स्वाम् दे॒वता॑ मति॒यज॑ते ऽति॒यज॑ते दे॒वताꣳ॒॒ स्वाꣳ स्वाम् दे॒वता॑ मति॒यज॑ते । \newline
35. दे॒वता॑ मति॒यज॑ते ऽति॒यज॑ते दे॒वता᳚म् दे॒वता॑ मति॒यज॑ते॒ प्र प्राति॒यज॑ते दे॒वता᳚म् दे॒वता॑ मति॒यज॑ते॒ प्र । \newline
36. अ॒ति॒यज॑ते॒ प्र प्राति॒यज॑ते ऽति॒यज॑ते॒ प्र स्वायै॒ स्वायै॒ प्राति॒यज॑ते ऽति॒यज॑ते॒ प्र स्वायै᳚ । \newline
37. अ॒ति॒यज॑त॒ इत्य॑ति - यज॑ते । \newline
38. प्र स्वायै॒ स्वायै॒ प्र प्र स्वायै॑ दे॒वता॑यै दे॒वता॑यै॒ स्वायै॒ प्र प्र स्वायै॑ दे॒वता॑यै । \newline
39. स्वायै॑ दे॒वता॑यै दे॒वता॑यै॒ स्वायै॒ स्वायै॑ दे॒वता॑यै च्यवते च्यवते दे॒वता॑यै॒ स्वायै॒ स्वायै॑ दे॒वता॑यै च्यवते । \newline
40. दे॒वता॑यै च्यवते च्यवते दे॒वता॑यै दे॒वता॑यै च्यवते॒ न न च्य॑वते दे॒वता॑यै दे॒वता॑यै च्यवते॒ न । \newline
41. च्य॒व॒ते॒ न न च्य॑वते च्यवते॒ न परा॒म् परा॒म् न च्य॑वते च्यवते॒ न परा᳚म् । \newline
42. न परा॒म् परा॒म् न न परा॒म् प्र प्र परा॒म् न न परा॒म् प्र । \newline
43. परा॒म् प्र प्र परा॒म् परा॒म् प्राप्नो᳚ त्याप्नोति॒ प्र परा॒म् परा॒म् प्राप्नो॑ति । \newline
44. प्राप्नो᳚ त्याप्नोति॒ प्र प्राप्नो॑ति॒ पापी॑या॒न् पापी॑या नाप्नोति॒ प्र प्राप्नो॑ति॒ पापी॑यान् । \newline
45. आ॒प्नो॒ति॒ पापी॑या॒न् पापी॑या नाप्नो त्याप्नोति॒ पापी॑यान् भवति भवति॒ पापी॑या नाप्नो त्याप्नोति॒ पापी॑यान् भवति । \newline
46. पापी॑यान् भवति भवति॒ पापी॑या॒न् पापी॑यान् भवति संॅवथ्स॒रꣳ सं॑ॅवथ्स॒रम् भ॑वति॒ पापी॑या॒न् पापी॑यान् भवति संॅवथ्स॒रम् । \newline
47. भ॒व॒ति॒ सं॒ॅव॒थ्स॒रꣳ सं॑ॅवथ्स॒रम् भ॑वति भवति संॅवथ्स॒र मिन्द्र॒ मिन्द्रꣳ॑ संॅवथ्स॒रम् भ॑वति भवति संॅवथ्स॒र मिन्द्र᳚म् । \newline
48. सं॒ॅव॒थ्स॒र मिन्द्र॒ मिन्द्रꣳ॑ संॅवथ्स॒रꣳ सं॑ॅवथ्स॒र मिन्द्रं॑ ॅयजेत यजे॒ते न्द्रꣳ॑ संॅवथ्स॒रꣳ सं॑ॅवथ्स॒र मिन्द्रं॑ ॅयजेत । \newline
49. सं॒ॅव॒थ्स॒रमिति॑ सं - व॒थ्स॒रम् । \newline
50. इन्द्रं॑ ॅयजेत यजे॒ते न्द्र॒ मिन्द्रं॑ ॅयजेत संॅवथ्स॒रꣳ सं॑ॅवथ्स॒रं ॅय॑जे॒ते न्द्र॒ मिन्द्रं॑ ॅयजेत संॅवथ्स॒रम् । \newline
51. य॒जे॒त॒ सं॒ॅव॒थ्स॒रꣳ सं॑ॅवथ्स॒रं ॅय॑जेत यजेत संॅवथ्स॒रꣳ हि हि सं॑ॅवथ्स॒रं ॅय॑जेत यजेत संॅवथ्स॒रꣳ हि । \newline
52. सं॒ॅव॒थ्स॒रꣳ हि हि सं॑ॅवथ्स॒रꣳ सं॑ॅवथ्स॒रꣳ हि व्र॒तं ॅव्र॒तꣳ हि सं॑ॅवथ्स॒रꣳ सं॑ॅवथ्स॒रꣳ हि व्र॒तम् । \newline
53. सं॒ॅव॒थ्स॒रमिति॑ सं - व॒थ्स॒रम् । \newline
54. हि व्र॒तं ॅव्र॒तꣳ हि हि व्र॒तन्न न व्र॒तꣳ हि हि व्र॒तम् न । \newline
55. व्र॒तम् न न व्र॒तं ॅव्र॒तम् नात्यति॒ न व्र॒तं ॅव्र॒तम् नाति॑ । \newline
56. नात्यति॒ न नाति॒ स्वा स्वा ऽति॒ न नाति॒ स्वा । \newline
57. अति॒ स्वा स्वा ऽत्यति॒ स्वैवैव स्वा ऽत्यति॒ स्वैव । \newline
58. स्वैवैव स्वा स्वैवैन॑ मेन मे॒व स्वा स्वैवैन᳚म् । \newline
\pagebreak
\markright{ TS 2.5.4.5  \hfill https://www.vedavms.in \hfill}
\addcontentsline{toc}{section}{ TS 2.5.4.5 }
\section*{ TS 2.5.4.5 }

\textbf{TS 2.5.4.5 } \newline
\textbf{Samhita Paata} \newline

-वैनं॑ दे॒वते॒ज्यमा॑ना॒ भूत्या॑ इन्धे॒ वसी॑यान् भवति संॅवथ्स॒रस्य॑ प॒रस्ता॑द॒ग्नये᳚ व्र॒तप॑तये पुरो॒डाश॑म॒ष्टाक॑पालं॒ निर्व॑पेथ् संॅवथ्स॒रमे॒वैनं॑ ॅवृ॒त्रं ज॑घ्नि॒वाꣳ स॑म॒ग्नि-र्व्र॒तप॑ति-र्व्र॒तमा ल॑म्भयति॒ ततोऽधि॒ कामं॑ ॅयजेत ॥ \newline

\textbf{Pada Paata} \newline

ए॒व । ए॒न॒म् । दे॒वता᳚ । इ॒ज्यमा॑ना । भूत्यै᳚ । इ॒न्धे॒ । वसी॑यान् । भ॒व॒ति॒ । सं॒ॅव॒थ्स॒रस्येति॑ सं - व॒थ्स॒रस्य॑ । प॒रस्ता᳚त् । अ॒ग्नये᳚ । व्र॒तप॑तय॒ इति॑ व्र॒त - प॒त॒ये॒ । पु॒रो॒डाश᳚म् । अ॒ष्टाक॑पाल॒मित्य॒ष्टा - क॒पा॒ल॒म् । निरिति॑ । व॒पे॒त् । सं॒ॅव॒थ्स॒रमिति॑ सं - व॒थ्स॒रम् । ए॒व । ए॒न॒म् । वृ॒त्रम् । ज॒घ्नि॒वाꣳस᳚म् । अ॒ग्निः । व्र॒तप॑ति॒रिति॑ व्र॒त-प॒तिः॒ । व्र॒तम् । एति॑ । ल॒भं॒य॒ति॒ । ततः॑ । अधीति॑ । काम᳚म् । य॒जे॒त॒ ॥(ए॒तां - त - दौदु॑म्बरꣳ॒॒ - स्वा -  \newline


\textbf{Krama Paata} \newline

ए॒वैन᳚म् । ए॒न॒म् दे॒वता᳚ । दे॒वते॒ज्यमा॑ना । इ॒ज्यमा॑ना॒ भूत्यै᳚ । भूत्या॑ इन्धे । इ॒न्धे॒ वसी॑यान् । वसी॑यान् भवति । भ॒व॒ति॒ स॒म्ॅव॒थ्स॒रस्य॑ । स॒म्ॅव॒थ्स॒रस्य॑ प॒रस्ता᳚त् । स॒म्ॅव॒थ्स॒रस्येति॑ सम् - व॒थ्स॒रस्य॑ । प॒रस्ता॑द॒ग्नये᳚ । अ॒ग्नये᳚ व्र॒तप॑तये । व्र॒तप॑तये पुरो॒डाश᳚म् । व्र॒तप॑तय॒ इति॑ व्र॒त - प॒त॒ये॒ । पु॒रो॒डाश॑म॒ष्टाक॑पालम् । अ॒ष्टाक॑पाल॒म् निः । अ॒ष्टाक॑पाल॒मित्य॒ष्टा - क॒पा॒ल॒म् । निर् व॑पेत् । व॒पे॒थ् स॒म्ॅव॒थ्स॒रम् । स॒म्ॅव॒थ्स॒रमे॒व । स॒म्ॅव॒थ्स॒रमिति॑ सम् - व॒थ्स॒रम् । ए॒वैन᳚म् । ए॒न॒म् ॅवृ॒त्रम् । वृ॒त्रम् ज॑घ्नि॒वाꣳस᳚म् । ज॒घ्नि॒वाꣳस॑म॒ग्निः । अ॒ग्निर् व्र॒तप॑तिः । व्र॒तप॑तिर् व्र॒तम् । व्र॒तप॑ति॒रिति॑ व्र॒त - प॒तिः॒ । व्र॒तमा । आ ल॑म्भयति । ल॒म्भ॒य॒ति॒ ततः॑ । ततो ऽधि॑ । अधि॒ काम᳚म् । काम॑म् ॅयजेत । य॒जे॒तेति॑ यजेत । \newline

\textbf{Jatai Paata} \newline

1. ए॒वैन॑ मेन मे॒वैवैन᳚म् । \newline
2. ए॒न॒म् दे॒वता॑ दे॒वतै॑न मेनम् दे॒वता᳚ । \newline
3. दे॒व ते॒ज्यमा॑ ने॒ज्यमा॑ना दे॒वता॑ दे॒व ते॒ज्यमा॑ना । \newline
4. इ॒ज्यमा॑ना॒ भूत्यै॒ भूत्या॑ इ॒ज्यमा॑ ने॒ज्यमा॑ना॒ भूत्यै᳚ । \newline
5. भूत्या॑ इन्ध इन्धे॒ भूत्यै॒ भूत्या॑ इन्धे । \newline
6. इ॒न्धे॒ वसी॑या॒न्॒. वसी॑या निन्ध इन्धे॒ वसी॑यान् । \newline
7. वसी॑यान् भवति भवति॒ वसी॑या॒न्॒. वसी॑यान् भवति । \newline
8. भ॒व॒ति॒ सं॒ॅव॒थ्स॒रस्य॑ संॅवथ्स॒रस्य॑ भवति भवति संॅवथ्स॒रस्य॑ । \newline
9. सं॒ॅव॒थ्स॒रस्य॑ प॒रस्ता᳚त् प॒रस्ता᳚थ् संॅवथ्स॒रस्य॑ संॅवथ्स॒रस्य॑ प॒रस्ता᳚त् । \newline
10. सं॒ॅव॒थ्स॒रस्येति॑ सं - व॒थ्स॒रस्य॑ । \newline
11. प॒रस्ता॑द॒ग्नये॒ ऽग्नये॑ प॒रस्ता᳚त् प॒रस्ता॑ द॒ग्नये᳚ । \newline
12. अ॒ग्नये᳚ व्र॒तप॑तये व्र॒तप॑तये॒ ऽग्नये॒ ऽग्नये᳚ व्र॒तप॑तये । \newline
13. व्र॒तप॑तये पुरो॒डाश॑म् पुरो॒डाशं॑ ॅव्र॒तप॑तये व्र॒तप॑तये पुरो॒डाश᳚म् । \newline
14. व्र॒तप॑तय॒ इति॑ व्र॒त - प॒त॒ये॒ । \newline
15. पु॒रो॒डाश॑ म॒ष्टाक॑पाल म॒ष्टाक॑पालम् पुरो॒डाश॑म् पुरो॒डाश॑ म॒ष्टाक॑पालम् । \newline
16. अ॒ष्टाक॑पाल॒म् निर् णिर॒ष्टाक॑पाल म॒ष्टाक॑पाल॒म् निः । \newline
17. अ॒ष्टाक॑पाल॒मित्य॒ष्टा - क॒पा॒ल॒म् । \newline
18. निर् व॑पेद् वपे॒न् निर् णिर् व॑पेत् । \newline
19. व॒पे॒थ् सं॒ॅव॒थ्स॒रꣳ सं॑ॅवथ्स॒रं ॅव॑पेद् वपेथ् संॅवथ्स॒रम् । \newline
20. सं॒ॅव॒थ्स॒र मे॒वैव सं॑ॅवथ्स॒रꣳ सं॑ॅवथ्स॒र मे॒व । \newline
21. सं॒ॅव॒थ्स॒रमिति॑ सं - व॒थ्स॒रम् । \newline
22. ए॒वैन॑ मेन मे॒वैवैन᳚म् । \newline
23. ए॒नं॒ ॅवृ॒त्रं ॅवृ॒त्र मे॑न मेनं ॅवृ॒त्रम् । \newline
24. वृ॒त्रम् ज॑घ्नि॒वाꣳस॑म् जघ्नि॒वाꣳसं॑ ॅवृ॒त्रं ॅवृ॒त्रम् ज॑घ्नि॒वाꣳस᳚म् । \newline
25. ज॒घ्नि॒वाꣳस॑ म॒ग्नि र॒ग्निर् ज॑घ्नि॒वाꣳस॑म् जघ्नि॒वाꣳस॑ म॒ग्निः । \newline
26. अ॒ग्निर् व्र॒तप॑तिर् व्र॒तप॑ति र॒ग्नि र॒ग्निर् व्र॒तप॑तिः । \newline
27. व्र॒तप॑तिर् व्र॒तं ॅव्र॒तं ॅव्र॒तप॑तिर् व्र॒तप॑तिर् व्र॒तम् । \newline
28. व्र॒तप॑ति॒रिति॑ व्र॒त - प॒तिः॒ । \newline
29. व्र॒त मा व्र॒तं ॅव्र॒त मा । \newline
30. आ लं॑भयति लंभय॒त्या लं॑भयति । \newline
31. लं॒भ॒य॒ति॒ तत॒ स्ततो॑ लंभयति लंभयति॒ ततः॑ । \newline
32. ततो ऽध्यधि॒ तत॒ स्ततो ऽधि॑ । \newline
33. अधि॒ काम॒म् काम॒ मध्यधि॒ काम᳚म् । \newline
34. कामं॑ ॅयजेत यजेत॒ काम॒म् कामं॑ ॅयजेत । \newline
35. य॒जे॒तेति॑ यजेत । \newline

\textbf{Ghana Paata } \newline

1. ए॒वैन॑ मेन मे॒वैवैन॑म् दे॒वता॑ दे॒वतै॑न मे॒वैवैन॑म् दे॒वता᳚ । \newline
2. ए॒न॒म् दे॒वता॑ दे॒वतै॑न मेनम् दे॒वते॒ज्यमा॑ ने॒ज्यमा॑ना दे॒वतै॑न मेनम् दे॒वते॒ज्यमा॑ना । \newline
3. दे॒वते॒ज्यमा॑ ने॒ज्यमा॑ना दे॒वता॑ दे॒वते॒ज्यमा॑ना॒ भूत्यै॒ भूत्या॑ इ॒ज्यमा॑ना दे॒वता॑ दे॒वते॒ज्यमा॑ना॒ भूत्यै᳚ । \newline
4. इ॒ज्यमा॑ना॒ भूत्यै॒ भूत्या॑ इ॒ज्यमा॑ ने॒ज्यमा॑ना॒ भूत्या॑ इन्ध इन्धे॒ भूत्या॑ इ॒ज्यमा॑ ने॒ज्यमा॑ना॒ भूत्या॑ इन्धे । \newline
5. भूत्या॑ इन्ध इन्धे॒ भूत्यै॒ भूत्या॑ इन्धे॒ वसी॑या॒न्॒. वसी॑या निन्धे॒ भूत्यै॒ भूत्या॑ इन्धे॒ वसी॑यान् । \newline
6. इ॒न्धे॒ वसी॑या॒न्॒. वसी॑या निन्ध इन्धे॒ वसी॑यान् भवति भवति॒ वसी॑या निन्ध इन्धे॒ वसी॑यान् भवति । \newline
7. वसी॑यान् भवति भवति॒ वसी॑या॒न्॒. वसी॑यान् भवति संॅवथ्स॒रस्य॑ संॅवथ्स॒रस्य॑ भवति॒ वसी॑या॒न्॒. वसी॑यान् भवति संॅवथ्स॒रस्य॑ । \newline
8. भ॒व॒ति॒ सं॒ॅव॒थ्स॒रस्य॑ संॅवथ्स॒रस्य॑ भवति भवति संॅवथ्स॒रस्य॑ प॒रस्ता᳚त् प॒रस्ता᳚थ् संॅवथ्स॒रस्य॑ भवति भवति संॅवथ्स॒रस्य॑ प॒रस्ता᳚त् । \newline
9. सं॒ॅव॒थ्स॒रस्य॑ प॒रस्ता᳚त् प॒रस्ता᳚थ् संॅवथ्स॒रस्य॑ संॅवथ्स॒रस्य॑ प॒रस्ता॑ द॒ग्नये॒ ऽग्नये॑ प॒रस्ता᳚थ् संॅवथ्स॒रस्य॑ संॅवथ्स॒रस्य॑ प॒रस्ता॑ द॒ग्नये᳚ । \newline
10. सं॒ॅव॒थ्स॒रस्येति॑ सं - व॒थ्स॒रस्य॑ । \newline
11. प॒रस्ता॑ द॒ग्नये॒ ऽग्नये॑ प॒रस्ता᳚त् प॒रस्ता॑ द॒ग्नये᳚ व्र॒तप॑तये व्र॒तप॑तये॒ ऽग्नये॑ प॒रस्ता᳚त् प॒रस्ता॑ द॒ग्नये᳚ व्र॒तप॑तये । \newline
12. अ॒ग्नये᳚ व्र॒तप॑तये व्र॒तप॑तये॒ ऽग्नये॒ ऽग्नये᳚ व्र॒तप॑तये पुरो॒डाश॑म् पुरो॒डाशं॑ ॅव्र॒तप॑तये॒ ऽग्नये॒ ऽग्नये᳚ व्र॒तप॑तये पुरो॒डाश᳚म् । \newline
13. व्र॒तप॑तये पुरो॒डाश॑म् पुरो॒डाशं॑ ॅव्र॒तप॑तये व्र॒तप॑तये पुरो॒डाश॑ म॒ष्टाक॑पाल म॒ष्टाक॑पालम् पुरो॒डाशं॑ ॅव्र॒तप॑तये व्र॒तप॑तये पुरो॒डाश॑ म॒ष्टाक॑पालम् । \newline
14. व्र॒तप॑तय॒ इति॑ व्र॒त - प॒त॒ये॒ । \newline
15. पु॒रो॒डाश॑ म॒ष्टाक॑पाल म॒ष्टाक॑पालम् पुरो॒डाश॑म् पुरो॒डाश॑ म॒ष्टाक॑पाल॒म् निर् णिर॒ष्टाक॑पालम् पुरो॒डाश॑म् पुरो॒डाश॑ म॒ष्टाक॑पाल॒म् निः । \newline
16. अ॒ष्टाक॑पाल॒म् निर् णिर॒ष्टाक॑पाल म॒ष्टाक॑पाल॒म् निर् व॑पेद् वपे॒न् निर॒ष्टाक॑पाल म॒ष्टाक॑पाल॒म् निर् व॑पेत् । \newline
17. अ॒ष्टाक॑पाल॒मित्य॒ष्टा - क॒पा॒ल॒म् । \newline
18. निर् व॑पेद् वपे॒न् निर् णिर् व॑पेथ् संॅवथ्स॒रꣳ सं॑ॅवथ्स॒रं ॅव॑पे॒न् निर् णिर् व॑पेथ् संॅवथ्स॒रम् । \newline
19. व॒पे॒थ् सं॒ॅव॒थ्स॒रꣳ सं॑ॅवथ्स॒रं ॅव॑पेद् वपेथ् संॅवथ्स॒र मे॒वैव सं॑ॅवथ्स॒रं ॅव॑पेद् वपेथ् संॅवथ्स॒र मे॒व । \newline
20. सं॒ॅव॒थ्स॒र मे॒वैव सं॑ॅवथ्स॒रꣳ सं॑ॅवथ्स॒र मे॒वैन॑ मेन मे॒व सं॑ॅवथ्स॒रꣳ सं॑ॅवथ्स॒र मे॒वैन᳚म् । \newline
21. सं॒ॅव॒थ्स॒रमिति॑ सं - व॒थ्स॒रम् । \newline
22. ए॒वैन॑ मेन मे॒वैवैनं॑ ॅवृ॒त्रं ॅवृ॒त्र मे॑न मे॒वैवैनं॑ ॅवृ॒त्रम् । \newline
23. ए॒नं॒ ॅवृ॒त्रं ॅवृ॒त्र मे॑न मेनं ॅवृ॒त्रम् ज॑घ्नि॒वाꣳस॑म् जघ्नि॒वाꣳसं॑ ॅवृ॒त्र मे॑न मेनं ॅवृ॒त्रम् ज॑घ्नि॒वाꣳस᳚म् । \newline
24. वृ॒त्रम् ज॑घ्नि॒वाꣳस॑म् जघ्नि॒वाꣳसं॑ ॅवृ॒त्रं ॅवृ॒त्रम् ज॑घ्नि॒वाꣳस॑ म॒ग्निर॒ग्निर् ज॑घ्नि॒वाꣳसं॑ ॅवृ॒त्रं ॅवृ॒त्रम् ज॑घ्नि॒वाꣳस॑ म॒ग्निः । \newline
25. ज॒घ्नि॒वाꣳस॑ म॒ग्नि र॒ग्निर् ज॑घ्नि॒वाꣳस॑म् जघ्नि॒वाꣳस॑ म॒ग्निर् व्र॒तप॑तिर् व्र॒तप॑ति र॒ग्निर् ज॑घ्नि॒वाꣳस॑म् जघ्नि॒वाꣳस॑ म॒ग्निर् व्र॒तप॑तिः । \newline
26. अ॒ग्निर् व्र॒तप॑तिर् व्र॒तप॑ति र॒ग्नि र॒ग्निर् व्र॒तप॑तिर् व्र॒तं ॅव्र॒तं ॅव्र॒तप॑ति र॒ग्नि र॒ग्निर् व्र॒तप॑तिर् व्र॒तम् । \newline
27. व्र॒तप॑तिर् व्र॒तं ॅव्र॒तं ॅव्र॒तप॑तिर् व्र॒तप॑तिर् व्र॒त मा व्र॒तं ॅव्र॒तप॑तिर् व्र॒तप॑तिर् व्र॒त मा । \newline
28. व्र॒तप॑ति॒रिति॑ व्र॒त - प॒तिः॒ । \newline
29. व्र॒त मा व्र॒तं ॅव्र॒त मा लं॑भयति लंभय॒त्या व्र॒तं ॅव्र॒त मा लं॑भयति । \newline
30. आ लं॑भयति लंभय॒त्या लं॑भयति॒ तत॒ स्ततो॑ लंभय॒त्या लं॑भयति॒ ततः॑ । \newline
31. लं॒भ॒य॒ति॒ तत॒ स्ततो॑ लंभयति लंभयति॒ ततो ऽध्यधि॒ ततो॑ लंभयति लंभयति॒ ततो ऽधि॑ । \newline
32. ततो ऽध्यधि॒ तत॒ स्ततो ऽधि॒ काम॒म् काम॒ मधि॒ तत॒ स्ततो ऽधि॒ काम᳚म् । \newline
33. अधि॒ काम॒म् काम॒ मध्यधि॒ कामं॑ ॅयजेत यजेत॒ काम॒ मध्यधि॒ कामं॑ ॅयजेत । \newline
34. कामं॑ ॅयजेत यजेत॒ काम॒म् कामं॑ ॅयजेत । \newline
35. य॒जे॒तेति॑ यजेत । \newline
\pagebreak
\markright{ TS 2.5.5.1  \hfill https://www.vedavms.in \hfill}
\addcontentsline{toc}{section}{ TS 2.5.5.1 }
\section*{ TS 2.5.5.1 }

\textbf{TS 2.5.5.1 } \newline
\textbf{Samhita Paata} \newline

नासो॑मयाजी॒ सं न॑ये॒दना॑गतं॒ ॅवा ए॒तस्य॒ पयो॒ योऽसो॑मयाजी॒ यदसो॑मयाजी सं॒ नय᳚त् परिमो॒ष ए॒व सोऽनृ॑तं करो॒त्यथो॒ परै॒व सि॑च्यते सोमया॒ज्ये॑व सं न॑ये॒त् पयो॒ वै सोमः॒ पयः॑ सांना॒य्यं पय॑सै॒व पय॑ आ॒त्मन् ध॑त्ते॒ वि वा ए॒तं प्र॒जया॑ प॒शुभि॑रर्द्धयति व॒र्द्धय॑त्यस्य॒ भ्रातृ॑व्यं॒ ॅयस्य॑ ह॒विर्निरु॑प्तं पु॒रस्ता᳚च्च॒न्द्रमा॑ - [  ] \newline

\textbf{Pada Paata} \newline

न । असो॑मया॒जीत्यसो॑म - या॒जी॒ । समिति॑ । न॒ये॒त् । अना॑गत॒मित्यना᳚ - ग॒त॒म् । वै । ए॒तस्य॑ । पयः॑ । यः । असो॑मया॒जीत्यसो॑म - या॒जी॒ । यत् । असो॑मया॒जीत्यसो॑म-या॒जी॒ । स॒नंये॒दिति॑ सं - नये᳚त् । प॒रि॒मो॒ष इति॑ परि - मो॒षः । ए॒व । सः । अनृ॑तम् । क॒रो॒ति॒ । अथो॒ इति॑ । परेति॑ । ए॒व । सि॒च्य॒ते॒ । सो॒म॒या॒जीति॑ सोम - या॒जी । ए॒व । समिति॑ । न॒ये॒त् । पयः॑ । वै । सोमः॑ । पयः॑ । सा॒नां॒य्यमिति॑ सां - ना॒य्यम् । पय॑सा । ए॒व । पयः॑ । आ॒त्मन्न् । ध॒त्ते॒ । वीति॑ । वै । ए॒तम् । प्र॒जयेति॑ प्र - जया᳚ । प॒शुभि॒रिति॑ प॒शु - भिः॒ । अ॒र्द्ध॒य॒ति॒ । व॒र्द्धय॑ति । अ॒स्य॒ । भ्रातृ॑व्यम् । यस्य॑ । ह॒विः । निरु॑प्त॒मिति॒ निः - उ॒प्त॒म् । पु॒रस्ता᳚त् । च॒न्द्रमाः᳚ ।  \newline


\textbf{Krama Paata} \newline

नासो॑मयाजी । असो॑मयाजी॒ सम् । असो॑मया॒जीत्यसो॑म - या॒जी॒ । सम् न॑येत् । न॒ये॒दना॑गतम् । अना॑गत॒म् ॅवै । अना॑गत॒मित्यना᳚ - ग॒त॒म् । वा ए॒तस्य॑ । ए॒तस्य॒ पयः॑ । पयो॒ यः । यो ऽसो॑मयाजी । असो॑मयाजी॒ यत् । असो॑मया॒जीत्यसो॑म - या॒जी॒ । यदसो॑मयाजी । असो॑मयाजी स॒न्नये᳚त् । असो॑मया॒जीत्यसो॑म - या॒जी॒ । स॒न्नये᳚त् परिमो॒षः । स॒न्नये॒दिति॑ सम् - नये᳚त् । प॒रि॒मो॒ष ए॒व । प॒रि॒मो॒ष इति॑ परि - मो॒षः । ए॒व सः । सो ऽनृ॑तम् । अनृ॑तम् करोति । क॒रो॒त्यथो᳚ । अथो॒ परा᳚ । अथो॒ इत्यथो᳚ । परै॒व । ए॒व सि॑च्यते । सि॒च्य॒ते॒ सो॒म॒या॒जी । सो॒म॒या॒ज्ये॑व । सो॒म॒या॒जीति॑ सोम - या॒जी । ए॒व सम् । सम् न॑येत् । न॒ये॒त् पयः॑ । पयो॒ वै । वै सोमः॑ । सोमः॒ पयः॑ । पयः॑ सान्ना॒य्यम् । सा॒न्ना॒य्यम् पय॑सा । सा॒न्ना॒य्यमिति॑ साम् - ना॒य्यम् । पय॑सै॒व । ए॒व पयः॑ । पय॑ आ॒त्मन्न् । आ॒त्मन् ध॑त्ते । ध॒त्ते॒ वि । वि वै । वा ए॒तम् । ए॒तम् प्र॒जया᳚ । प्र॒जया॑ प॒शुभिः॑ । प्र॒जयेति॑ प्र - जया᳚ । प॒शुभि॑रर्द्धयति । प॒शुभि॒रिति॑ प॒शु - भिः॒ । अ॒र्द्ध॒य॒ति॒ व॒र्द्धय॑ति । व॒र्द्धय॑त्यस्य । अ॒स्य॒ भ्रातृ॑व्यम् । भ्रातृ॑व्य॒म् ॅयस्य॑ । यस्य॑ ह॒विः । ह॒विर् निरु॑प्तम् । निरु॑प्तम् पु॒रस्ता᳚त् । निरु॑प्त॒मिति॒ निः - उ॒प्त॒म् । पु॒रस्ता᳚च्च॒न्द्रमाः᳚ । च॒न्द्रमा॑ अ॒भि \newline

\textbf{Jatai Paata} \newline

1. नासो॑मया॒ ज्यसो॑मयाजी॒ न नासो॑मयाजी । \newline
2. असो॑मयाजी॒ सꣳ स मसो॑मया॒ ज्यसो॑मयाजी॒ सम् । \newline
3. असो॑मया॒जीत्यसो॑म - या॒जी॒ । \newline
4. सम् न॑येन् नये॒थ् सꣳ सम् न॑येत् । \newline
5. न॒ये॒ दना॑गत॒ मना॑गतम् नयेन् नये॒ दना॑गतम् । \newline
6. अना॑गतं॒ ॅवै वा अना॑गत॒ मना॑गतं॒ ॅवै । \newline
7. अना॑गत॒मित्यना᳚ - ग॒त॒म् । \newline
8. वा ए॒तस्यै॒तस्य॒ वै वा ए॒तस्य॑ । \newline
9. ए॒तस्य॒ पयः॒ पय॑ ए॒तस्यै॒तस्य॒ पयः॑ । \newline
10. पयो॒ यो यः पयः॒ पयो॒ यः । \newline
11. यो ऽसो॑मया॒ ज्यसो॑मयाजी॒ यो यो ऽसो॑मयाजी । \newline
12. असो॑मयाजी॒ यद् यदसो॑मया॒ ज्यसो॑मयाजी॒ यत् । \newline
13. असो॑मया॒जीत्यसो॑म - या॒जी॒ । \newline
14. यदसो॑मया॒ ज्यसो॑मयाजी॒ यद् यदसो॑मयाजी । \newline
15. असो॑मयाजी स॒न्नये᳚थ् स॒न्नये॒ दसो॑मया॒ ज्यसो॑मयाजी स॒न्नये᳚त् । \newline
16. असो॑मया॒जीत्यसो॑म - या॒जी॒ । \newline
17. स॒न्नये᳚त् परिमो॒षः प॑रिमो॒षः स॒न्नये᳚थ् स॒न्नये᳚त् परिमो॒षः । \newline
18. स॒न्नये॒दिति॑ सं - नये᳚त् । \newline
19. प॒रि॒मो॒ष ए॒वैव प॑रिमो॒षः प॑रिमो॒ष ए॒व । \newline
20. प॒रि॒मो॒ष इति॑ परि - मो॒षः । \newline
21. ए॒व स स ए॒वैव सः । \newline
22. सो ऽनृ॑त॒ मनृ॑तꣳ॒॒ स सो ऽनृ॑तम् । \newline
23. अनृ॑तम् करोति करो॒ त्यनृ॑त॒ मनृ॑तम् करोति । \newline
24. क॒रो॒ त्यथो॒ अथो॑ करोति करो॒ त्यथो᳚ । \newline
25. अथो॒ परा॒ परा ऽथो॒ अथो॒ परा᳚ । \newline
26. अथो॒ इत्यथो᳚ । \newline
27. परै॒वैव परा॒ परै॒व । \newline
28. ए॒व सि॑च्यते सिच्यत ए॒वैव सि॑च्यते । \newline
29. सि॒च्य॒ते॒ सो॒म॒या॒जी सो॑मया॒जी सि॑च्यते सिच्यते सोमया॒जी । \newline
30. सो॒म॒या॒ज्ये॑वैव सो॑मया॒जी सो॑मया॒ज्ये॑व । \newline
31. सो॒म॒या॒जीति॑ सोम - या॒जी । \newline
32. ए॒व सꣳ स मे॒वैव सम् । \newline
33. सम् न॑येन् नये॒थ् सꣳ सम् न॑येत् । \newline
34. न॒ये॒त् पयः॒ पयो॑ नयेन् नये॒त् पयः॑ । \newline
35. पयो॒ वै वै पयः॒ पयो॒ वै । \newline
36. वै सोमः॒ सोमो॒ वै वै सोमः॑ । \newline
37. सोमः॒ पयः॒ पयः॒ सोमः॒ सोमः॒ पयः॑ । \newline
38. पयः॑ सान्ना॒य्यꣳ सा᳚न्ना॒य्यम् पयः॒ पयः॑ सान्ना॒य्यम् । \newline
39. सा॒न्ना॒य्यम् पय॑सा॒ पय॑सा सान्ना॒य्यꣳ सा᳚न्ना॒य्यम् पय॑सा । \newline
40. सा॒न्ना॒य्यमिति॑ सां - ना॒य्यम् । \newline
41. पय॑सै॒वैव पय॑सा॒ पय॑सै॒व । \newline
42. ए॒व पयः॒ पय॑ ए॒वैव पयः॑ । \newline
43. पय॑ आ॒त्मन् ना॒त्मन् पयः॒ पय॑ आ॒त्मन्न् । \newline
44. आ॒त्मन् ध॑त्ते धत्त आ॒त्मन् ना॒त्मन् ध॑त्ते । \newline
45. ध॒त्ते॒ वि वि ध॑त्ते धत्ते॒ वि । \newline
46. वि वै वै वि वि वै । \newline
47. वा ए॒त मे॒तं ॅवै वा ए॒तम् । \newline
48. ए॒तम् प्र॒जया᳚ प्र॒जयै॒त मे॒तम् प्र॒जया᳚ । \newline
49. प्र॒जया॑ प॒शुभिः॑ प॒शुभिः॑ प्र॒जया᳚ प्र॒जया॑ प॒शुभिः॑ । \newline
50. प्र॒जयेति॑ प्र - जया᳚ । \newline
51. प॒शुभि॑ रर्द्धय त्यर्द्धयति प॒शुभिः॑ प॒शुभि॑ रर्द्धयति । \newline
52. प॒शुभि॒रिति॑ प॒शु - भिः॒ । \newline
53. अ॒र्द्ध॒य॒ति॒ व॒र्द्धय॑ति व॒र्द्धय॑ त्यर्द्धय त्यर्द्धयति व॒र्द्धय॑ति । \newline
54. व॒र्द्धय॑ त्यस्यास्य व॒र्द्धय॑ति व॒र्द्धय॑ त्यस्य । \newline
55. अ॒स्य॒ भ्रातृ॑व्य॒म् भ्रातृ॑व्य मस्यास्य॒ भ्रातृ॑व्यम् । \newline
56. भ्रातृ॑व्यं॒ ॅयस्य॒ यस्य॒ भ्रातृ॑व्य॒म् भ्रातृ॑व्यं॒ ॅयस्य॑ । \newline
57. यस्य॑ ह॒विर्. ह॒विर् यस्य॒ यस्य॑ ह॒विः । \newline
58. ह॒विर् निरु॑प्त॒म् निरु॑प्तꣳ ह॒विर्. ह॒विर् निरु॑प्तम् । \newline
59. निरु॑प्तम् पु॒रस्ता᳚त् पु॒रस्ता॒न् निरु॑प्त॒म् निरु॑प्तम् पु॒रस्ता᳚त् । \newline
60. निरु॑प्त॒मिति॒ निः - उ॒प्त॒म् । \newline
61. पु॒रस्ता᳚च् च॒न्द्रमा᳚श्च॒न्द्रमाः᳚ पु॒रस्ता᳚त् पु॒रस्ता᳚च् च॒न्द्रमाः᳚ । \newline
62. च॒न्द्रमा॑ अ॒भ्य॑भि च॒न्द्रमा᳚ श्च॒न्द्रमा॑ अ॒भि । \newline

\textbf{Ghana Paata } \newline

1. नासो॑मया॒ ज्यसो॑मयाजी॒ न नासो॑मयाजी॒ सꣳ स मसो॑मयाजी॒ न नासो॑मयाजी॒ सम् । \newline
2. असो॑मयाजी॒ सꣳ स मसो॑मया॒ ज्यसो॑मयाजी॒ सम् न॑येन् नये॒थ् स मसो॑मया॒ ज्यसो॑मयाजी॒ सम् न॑येत् । \newline
3. असो॑मया॒जीत्यसो॑म - या॒जी॒ । \newline
4. सम् न॑येन् नये॒थ् सꣳ सम् न॑ये॒ दना॑गत॒ मना॑गतम् नये॒थ् सꣳ सम् न॑ये॒ दना॑गतम् । \newline
5. न॒ये॒ दना॑गत॒ मना॑गतम् नयेन् नये॒ दना॑गतं॒ ॅवै वा अना॑गतम् नयेन् नये॒ दना॑गतं॒ ॅवै । \newline
6. अना॑गतं॒ ॅवै वा अना॑गत॒ मना॑गतं॒ ॅवा ए॒तस्यै॒तस्य॒ वा अना॑गत॒ मना॑गतं॒ ॅवा ए॒तस्य॑ । \newline
7. अना॑गत॒मित्यना᳚ - ग॒त॒म् । \newline
8. वा ए॒तस्यै॒तस्य॒ वै वा ए॒तस्य॒ पयः॒ पय॑ ए॒तस्य॒ वै वा ए॒तस्य॒ पयः॑ । \newline
9. ए॒तस्य॒ पयः॒ पय॑ ए॒तस्यै॒तस्य॒ पयो॒ यो यः पय॑ ए॒तस्यै॒तस्य॒ पयो॒ यः । \newline
10. पयो॒ यो यः पयः॒ पयो॒ यो ऽसो॑मया॒ ज्यसो॑मयाजी॒ यः पयः॒ पयो॒ यो ऽसो॑मयाजी । \newline
11. यो ऽसो॑मया॒ ज्यसो॑मयाजी॒ यो यो ऽसो॑मयाजी॒ यद् यदसो॑मयाजी॒ यो यो ऽसो॑मयाजी॒ यत् । \newline
12. असो॑मयाजी॒ यद् यदसो॑मया॒ ज्यसो॑मयाजी॒ यदसो॑मया॒ ज्यसो॑मयाजी॒ यदसो॑मया॒ ज्यसो॑मयाजी॒ यदसो॑मयाजी । \newline
13. असो॑मया॒जीत्यसो॑म - या॒जी॒ । \newline
14. यदसो॑मया॒ ज्यसो॑मयाजी॒ यद् यदसो॑मयाजी स॒न्नये᳚थ् स॒न्नये॒ दसो॑मयाजी॒ यद् यदसो॑मयाजी स॒न्नये᳚त् । \newline
15. असो॑मयाजी स॒न्नये᳚थ् स॒न्नये॒ दसो॑मया॒ ज्यसो॑मयाजी स॒न्नये᳚त् परिमो॒षः प॑रिमो॒षः स॒न्नये॒ दसो॑मया॒ ज्यसो॑मयाजी स॒न्नये᳚त् परिमो॒षः । \newline
16. असो॑मया॒जीत्यसो॑म - या॒जी॒ । \newline
17. स॒न्नये᳚त् परिमो॒षः प॑रिमो॒षः स॒न्नये᳚थ् स॒न्नये᳚त् परिमो॒ष ए॒वैव प॑रिमो॒षः स॒न्नये᳚थ् स॒न्नये᳚त् परिमो॒ष ए॒व । \newline
18. स॒न्नये॒दिति॑ सं - नये᳚त् । \newline
19. प॒रि॒मो॒ष ए॒वैव प॑रिमो॒षः प॑रिमो॒ष ए॒व स स ए॒व प॑रिमो॒षः प॑रिमो॒ष ए॒व सः । \newline
20. प॒रि॒मो॒ष इति॑ परि - मो॒षः । \newline
21. ए॒व स स ए॒वैव सो ऽनृ॑त॒ मनृ॑तꣳ॒॒ स ए॒वैव सो ऽनृ॑तम् । \newline
22. सो ऽनृ॑त॒ मनृ॑तꣳ॒॒ स सो ऽनृ॑तम् करोति करो॒ त्यनृ॑तꣳ॒॒ स सो ऽनृ॑तम् करोति । \newline
23. अनृ॑तम् करोति करो॒ त्यनृ॑त॒ मनृ॑तम् करो॒त्यथो॒ अथो॑ करो॒ त्यनृ॑त॒ मनृ॑तम् करो॒त्यथो᳚ । \newline
24. क॒रो॒त्यथो॒ अथो॑ करोति करो॒त्यथो॒ परा॒ परा ऽथो॑ करोति करो॒त्यथो॒ परा᳚ । \newline
25. अथो॒ परा॒ परा ऽथो॒ अथो॒ परै॒वैव परा ऽथो॒ अथो॒ परै॒व । \newline
26. अथो॒ इत्यथो᳚ । \newline
27. परै॒वैव परा॒ परै॒व सि॑च्यते सिच्यत ए॒व परा॒ परै॒व सि॑च्यते । \newline
28. ए॒व सि॑च्यते सिच्यत ए॒वैव सि॑च्यते सोमया॒जी सो॑मया॒जी सि॑च्यत ए॒वैव सि॑च्यते सोमया॒जी । \newline
29. सि॒च्य॒ते॒ सो॒म॒या॒जी सो॑मया॒जी सि॑च्यते सिच्यते सोमया॒ज्ये॑वैव सो॑मया॒जी सि॑च्यते सिच्यते सोमया॒ज्ये॑व । \newline
30. सो॒म॒या॒ज्ये॑वैव सो॑मया॒जी सो॑मया॒ज्ये॑व सꣳ स मे॒व सो॑मया॒जी सो॑मया॒ज्ये॑व सम् । \newline
31. सो॒म॒या॒जीति॑ सोम - या॒जी । \newline
32. ए॒व सꣳ स मे॒वैव सम् न॑येन् नये॒थ् स मे॒वैव सम् न॑येत् । \newline
33. सम् न॑येन् नये॒थ् सꣳ सम् न॑ये॒त् पयः॒ पयो॑ नये॒थ् सꣳ सम् न॑ये॒त् पयः॑ । \newline
34. न॒ये॒त् पयः॒ पयो॑ नयेन् नये॒त् पयो॒ वै वै पयो॑ नयेन् नये॒त् पयो॒ वै । \newline
35. पयो॒ वै वै पयः॒ पयो॒ वै सोमः॒ सोमो॒ वै पयः॒ पयो॒ वै सोमः॑ । \newline
36. वै सोमः॒ सोमो॒ वै वै सोमः॒ पयः॒ पयः॒ सोमो॒ वै वै सोमः॒ पयः॑ । \newline
37. सोमः॒ पयः॒ पयः॒ सोमः॒ सोमः॒ पयः॑ सान्ना॒य्यꣳ सा᳚न्ना॒य्यम् पयः॒ सोमः॒ सोमः॒ पयः॑ सान्ना॒य्यम् । \newline
38. पयः॑ सान्ना॒य्यꣳ सा᳚न्ना॒य्यम् पयः॒ पयः॑ सान्ना॒य्यम् पय॑सा॒ पय॑सा सान्ना॒य्यम् पयः॒ पयः॑ सान्ना॒य्यम् पय॑सा । \newline
39. सा॒न्ना॒य्यम् पय॑सा॒ पय॑सा सान्ना॒य्यꣳ सा᳚न्ना॒य्यम् पय॑सै॒वैव पय॑सा सान्ना॒य्यꣳ सा᳚न्ना॒य्यम् पय॑सै॒व । \newline
40. सा॒न्ना॒य्यमिति॑ सां - ना॒य्यम् । \newline
41. पय॑सै॒वैव पय॑सा॒ पय॑सै॒व पयः॒ पय॑ ए॒व पय॑सा॒ पय॑सै॒व पयः॑ । \newline
42. ए॒व पयः॒ पय॑ ए॒वैव पय॑ आ॒त्मन् ना॒त्मन् पय॑ ए॒वैव पय॑ आ॒त्मन्न् । \newline
43. पय॑ आ॒त्मन् ना॒त्मन् पयः॒ पय॑ आ॒त्मन् ध॑त्ते धत्त आ॒त्मन् पयः॒ पय॑ आ॒त्मन् ध॑त्ते । \newline
44. आ॒त्मन् ध॑त्ते धत्त आ॒त्मन् ना॒त्मन् ध॑त्ते॒ वि वि ध॑त्त आ॒त्मन् ना॒त्मन् ध॑त्ते॒ वि । \newline
45. ध॒त्ते॒ वि वि ध॑त्ते धत्ते॒ वि वै वै वि ध॑त्ते धत्ते॒ वि वै । \newline
46. वि वै वै वि वि वा ए॒त मे॒तं ॅवै वि वि वा ए॒तम् । \newline
47. वा ए॒त मे॒तं ॅवै वा ए॒तम् प्र॒जया᳚ प्र॒जयै॒तं ॅवै वा ए॒तम् प्र॒जया᳚ । \newline
48. ए॒तम् प्र॒जया᳚ प्र॒जयै॒त मे॒तम् प्र॒जया॑ प॒शुभिः॑ प॒शुभिः॑ प्र॒जयै॒त मे॒तम् प्र॒जया॑ प॒शुभिः॑ । \newline
49. प्र॒जया॑ प॒शुभिः॑ प॒शुभिः॑ प्र॒जया᳚ प्र॒जया॑ प॒शुभि॑ रर्द्धय त्यर्द्धयति प॒शुभिः॑ प्र॒जया᳚ प्र॒जया॑ प॒शुभि॑ रर्द्धयति । \newline
50. प्र॒जयेति॑ प्र - जया᳚ । \newline
51. प॒शुभि॑ रर्द्धय त्यर्द्धयति प॒शुभिः॑ प॒शुभि॑ रर्द्धयति व॒र्द्धय॑ति व॒र्द्धय॑ त्यर्द्धयति प॒शुभिः॑ प॒शुभि॑ रर्द्धयति व॒र्द्धय॑ति । \newline
52. प॒शुभि॒रिति॑ प॒शु - भिः॒ । \newline
53. अ॒र्द्ध॒य॒ति॒ व॒र्द्धय॑ति व॒र्द्धय॑ त्यर्द्धय त्यर्द्धयति व॒र्द्धय॑ त्यस्यास्य व॒र्द्धय॑ त्यर्द्धय त्यर्द्धयति व॒र्द्धय॑त्यस्य । \newline
54. व॒र्द्धय॑ त्यस्यास्य व॒र्द्धय॑ति व॒र्द्धय॑ त्यस्य॒ भ्रातृ॑व्य॒म् भ्रातृ॑व्य मस्य व॒र्द्धय॑ति व॒र्द्धय॑ त्यस्य॒ भ्रातृ॑व्यम् । \newline
55. अ॒स्य॒ भ्रातृ॑व्य॒म् भ्रातृ॑व्य मस्यास्य॒ भ्रातृ॑व्यं॒ ॅयस्य॒ यस्य॒ भ्रातृ॑व्य मस्यास्य॒ भ्रातृ॑व्यं॒ ॅयस्य॑ । \newline
56. भ्रातृ॑व्यं॒ ॅयस्य॒ यस्य॒ भ्रातृ॑व्य॒म् भ्रातृ॑व्यं॒ ॅयस्य॑ ह॒विर्. ह॒विर् यस्य॒ भ्रातृ॑व्य॒म् भ्रातृ॑व्यं॒ ॅयस्य॑ ह॒विः । \newline
57. यस्य॑ ह॒विर्. ह॒विर् यस्य॒ यस्य॑ ह॒विर् निरु॑प्त॒म् निरु॑प्तꣳ ह॒विर् यस्य॒ यस्य॑ ह॒विर् निरु॑प्तम् । \newline
58. ह॒विर् निरु॑प्त॒म् निरु॑प्तꣳ ह॒विर्. ह॒विर् निरु॑प्तम् पु॒रस्ता᳚त् पु॒रस्ता॒न् निरु॑प्तꣳ ह॒विर्. ह॒विर् निरु॑प्तम् पु॒रस्ता᳚त् । \newline
59. निरु॑प्तम् पु॒रस्ता᳚त् पु॒रस्ता॒न् निरु॑प्त॒म् निरु॑प्तम् पु॒रस्ता᳚च् च॒न्द्रमा᳚ श्च॒न्द्रमाः᳚ पु॒रस्ता॒न् निरु॑प्त॒म् निरु॑प्तम् पु॒रस्ता᳚च् च॒न्द्रमाः᳚ । \newline
60. निरु॑प्त॒मिति॒ निः - उ॒प्त॒म् । \newline
61. पु॒रस्ता᳚च् च॒न्द्रमा᳚ श्च॒न्द्रमाः᳚ पु॒रस्ता᳚त् पु॒रस्ता᳚च् च॒न्द्रमा॑ अ॒भ्य॑भि च॒न्द्रमाः᳚ पु॒रस्ता᳚त् पु॒रस्ता᳚च् च॒न्द्रमा॑ अ॒भि । \newline
62. च॒न्द्रमा॑ अ॒भ्य॑भि च॒न्द्रमा᳚ श्च॒न्द्रमा॑ अ॒भ्यु॑देत्यु॒दे त्य॒भि च॒न्द्रमा᳚ श्च॒न्द्रमा॑ अ॒भ्यु॑देति॑ । \newline
\pagebreak
\markright{ TS 2.5.5.2  \hfill https://www.vedavms.in \hfill}
\addcontentsline{toc}{section}{ TS 2.5.5.2 }
\section*{ TS 2.5.5.2 }

\textbf{TS 2.5.5.2 } \newline
\textbf{Samhita Paata} \newline

अ॒भ्यु॑देति॑ त्रे॒धा त॑ण्डु॒लान्. वि भ॑जे॒द्ये म॑द्ध्य॒माः स्युस्तान॒ग्नये॑ दा॒त्रे पु॑रो॒डाश॑म॒ष्टाक॑पालं कुर्या॒द्ये स्थवि॑ष्ठा॒स्तानिन्द्रा॑य प्रदा॒त्रे द॒धꣳश्च॒रुं ॅयेऽणि॑ष्ठा॒स्तान्. विष्ण॑वे शिपिवि॒ष्टाय॑ शृ॒ते च॒रुम॒ग्निरे॒वास्मै᳚ प्र॒जां प्र॑ज॒नय॑ति वृ॒द्धामिन्द्रः॒ प्रय॑च्छति य॒ज्ञो वै विष्णुः॑ प॒शवः॒ शिपि॑र्य॒ज्ञ् ए॒व प॒शुषु॒ प्रति॑तिष्ठति॒ न द्वे - [  ] \newline

\textbf{Pada Paata} \newline

अ॒भीति॑ । उ॒देतीत्यु॑त् - एति॑ । त्रे॒धा । त॒ण्डु॒लान् । वीति॑ । भ॒जे॒त् । ये । म॒द्ध्य॒माः । स्युः । तान् । अ॒ग्नये᳚ । दा॒त्रे । पु॒रो॒डाश᳚म् । अ॒ष्टाक॑पाल॒मित्य॒ष्टा - क॒पा॒ल॒म् । कु॒र्या॒त् । ये । स्थवि॑ष्ठाः । तान् । इन्द्रा॑य । प्र॒दा॒त्र इति॑ प्र - दा॒त्रे । द॒धन्न् । च॒रुम् । ये । अणि॑ष्ठाः । तान् । विष्ण॑वे । शि॒पि॒वि॒ष्टायेति॑ शिपि - वि॒ष्टाय॑ । शृ॒ते । च॒रुम् । अ॒ग्निः । ए॒व । अ॒स्मै॒ । प्र॒जामिति॑ प्र - जाम् । प्र॒ज॒नय॒तीति॑ प्र - ज॒नय॑ति । वृ॒द्धाम् । इन्द्रः॑ । प्रेति॑ । य॒च्छ॒ति॒ । य॒ज्ञ्ः । वै । विष्णुः॑ । प॒शवः॑ । शिपिः॑ । य॒ज्ञे । ए॒व । प॒शुषु॑ । प्रतीति॑ । ति॒ष्ठ॒ति॒ । न । द्वे इति॑ ।  \newline


\textbf{Krama Paata} \newline

अ॒भ्यु॑देति॑ । उ॒देति॑ त्रे॒धा । उ॒देतीत्यु॑त् - एति॑ । त्रे॒धा त॑ण्डु॒लान् । त॒ण्डु॒लान्. वि । वि भ॑जेत् । भ॒जे॒द् ये । ये म॑द्ध्य॒माः । म॒द्ध्य॒माः स्युः । स्युस्तान् । तान॒ग्नये᳚ । अ॒ग्नये॑ दा॒त्रे । दा॒त्रे पु॑रो॒डाश᳚म् । पु॒रो॒डाश॑म॒ष्टाक॑पालम् । अ॒ष्टाक॑पालम् कुर्यात् । अ॒षा॑कपाल॒मित्य॒ष्टा - क॒पा॒ल॒म् । कु॒र्या॒द् ये । ये स्थवि॑ष्ठाः । स्थवि॑ष्ठा॒स्तान् । तानिन्द्रा॑य । इन्द्रा॑य प्रदा॒त्रे । प्र॒दा॒त्रे द॒धन्न् । प्र॒दा॒त्र इति॑ प्र - दा॒त्रे । द॒धꣳश्च॒रुम् । च॒रुम् ॅये । ये ऽणि॑ष्ठाः । अणि॑ष्ठा॒स्तान् । तान्. विष्ण॑वे । विष्ण॑वे शिपिवि॒ष्टाय॑ । शि॒पि॒वि॒ष्टाय॑ शृ॒ते । शि॒पि॒वि॒ष्टायेति॑ शिपि - वि॒ष्टाय॑ । शृ॒ते च॒रुम् । च॒रुम॒ग्निः । अ॒ग्निरे॒व । ए॒वास्मै᳚ । अ॒स्मै॒ प्र॒जाम् । प्र॒जाम् प्र॑ज॒नय॑ति । प्र॒जामिति॑ प्र - जाम् । प्र॒ज॒नय॑ति वृ॒द्धाम् । प्र॒ज॒नय॒तीति॑ प्र - ज॒नय॑ति । वृ॒द्धामिन्द्रः॑ । इन्द्रः॒ प्र । प्र य॑च्छति । य॒च्छ॒ति॒ य॒ज्ञ्ः । य॒ज्ञो वै । वै विष्णुः॑ । विष्णुः॑ प॒शवः॑ । प॒शवः॒ शिपिः॑ । शिपि॑र् य॒ज्ञे । य॒ज्ञ् ए॒व । ए॒व प॒शुषु॑ । प॒शुषु॒ प्रति॑ । प्रति॑ तिष्ठति । ति॒ष्ठ॒ति॒ न । न द्वे । द्वे य॑जेत । द्वे इति॒ द्वे \newline

\textbf{Jatai Paata} \newline

1. अ॒भ्यु॑दे त्यु॒देत्य॒भ्या᳚(1॒)भ्यु॑देति॑ । \newline
2. उ॒देति॑ त्रे॒धा त्रे॒धोदे त्यु॒देति॑ त्रे॒धा । \newline
3. उ॒देतीत्यु॑त् - एति॑ । \newline
4. त्रे॒धा त॑ण्डु॒लान् त॑ण्डु॒लान् त्रे॒धा त्रे॒धा त॑ण्डु॒लान् । \newline
5. त॒ण्डु॒लान्. वि वि त॑ण्डु॒लान् त॑ण्डु॒लान्. वि । \newline
6. वि भ॑जेद् भजे॒द् वि वि भ॑जेत् । \newline
7. भ॒जे॒द् ये ये भ॑जेद् भजे॒द् ये । \newline
8. ये म॑द्ध्य॒मा म॑द्ध्य॒मा ये ये म॑द्ध्य॒माः । \newline
9. म॒द्ध्य॒माः स्युः स्युर् म॑द्ध्य॒मा म॑द्ध्य॒माः स्युः । \newline
10. स्यु स्ताꣳ स्तान् थ्स्युः स्यु स्तान् । \newline
11. ता न॒ग्नये॒ ऽग्नये॒ ताꣳ स्ता न॒ग्नये᳚ । \newline
12. अ॒ग्नये॑ दा॒त्रे दा॒त्रे᳚ ऽग्नये॒ ऽग्नये॑ दा॒त्रे । \newline
13. दा॒त्रे पु॑रो॒डाश॑म् पुरो॒डाश॑म् दा॒त्रे दा॒त्रे पु॑रो॒डाश᳚म् । \newline
14. पु॒रो॒डाश॑ म॒ष्टाक॑पाल म॒ष्टाक॑पालम् पुरो॒डाश॑म् पुरो॒डाश॑ म॒ष्टाक॑पालम् । \newline
15. अ॒ष्टाक॑पालम् कुर्यात् कुर्या द॒ष्टाक॑पाल म॒ष्टाक॑पालम् कुर्यात् । \newline
16. अ॒ष्टाक॑पाल॒मित्य॒ष्टा - क॒पा॒ल॒म् । \newline
17. कु॒र्या॒द् ये ये कु॑र्यात् कुर्या॒द् ये । \newline
18. ये स्थवि॑ष्ठाः॒ स्थवि॑ष्ठा॒ ये ये स्थवि॑ष्ठाः । \newline
19. स्थवि॑ष्ठा॒ स्ताꣳ स्तान् थ्स्थवि॑ष्ठाः॒ स्थवि॑ष्ठा॒ स्तान् । \newline
20. ता निन्द्रा॒ये न्द्रा॑य॒ ताꣳ स्ता निन्द्रा॑य । \newline
21. इन्द्रा॑य प्रदा॒त्रे प्र॑दा॒त्र इन्द्रा॒ये न्द्रा॑य प्रदा॒त्रे । \newline
22. प्र॒दा॒त्रे द॒धन् द॒धन् प्र॑दा॒त्रे प्र॑दा॒त्रे द॒धन्न् । \newline
23. प्र॒दा॒त्र इति॑ प्र - दा॒त्रे । \newline
24. द॒धꣳश् च॒रुम् च॒रुम् द॒धन् द॒धꣳश् च॒रुम् । \newline
25. च॒रुं ॅये ये च॒रुम् च॒रुं ॅये । \newline
26. ये ऽणि॑ष्ठा॒ अणि॑ष्ठा॒ ये ये ऽणि॑ष्ठाः । \newline
27. अणि॑ष्ठा॒ स्ताꣳ स्ता नणि॑ष्ठा॒ अणि॑ष्ठा॒ स्तान् । \newline
28. तान्. विष्ण॑वे॒ विष्ण॑वे॒ ताꣳ स्तान्. विष्ण॑वे । \newline
29. विष्ण॑वे शिपिवि॒ष्टाय॑ शिपिवि॒ष्टाय॒ विष्ण॑वे॒ विष्ण॑वे शिपिवि॒ष्टाय॑ । \newline
30. शि॒पि॒वि॒ष्टाय॑ शृ॒ते शृ॒ते शि॑पिवि॒ष्टाय॑ शिपिवि॒ष्टाय॑ शृ॒ते । \newline
31. शि॒पि॒वि॒ष्टायेति॑ शिपि - वि॒ष्टाय॑ । \newline
32. शृ॒ते च॒रुम् च॒रुꣳ शृ॒ते शृ॒ते च॒रुम् । \newline
33. च॒रु म॒ग्नि र॒ग्नि श्च॒रुम् च॒रु म॒ग्निः । \newline
34. अ॒ग्नि रे॒वैवाग्नि र॒ग्नि रे॒व । \newline
35. ए॒वास्मा॑ अस्मा ए॒वैवास्मै᳚ । \newline
36. अ॒स्मै॒ प्र॒जाम् प्र॒जा म॑स्मा अस्मै प्र॒जाम् । \newline
37. प्र॒जाम् प्र॑ज॒नय॑ति प्रज॒नय॑ति प्र॒जाम् प्र॒जाम् प्र॑ज॒नय॑ति । \newline
38. प्र॒जामिति॑ प्र - जाम् । \newline
39. प्र॒ज॒नय॑ति वृ॒द्धां ॅवृ॒द्धाम् प्र॑ज॒नय॑ति प्रज॒नय॑ति वृ॒द्धाम् । \newline
40. प्र॒ज॒नय॒तीति॑ प्र - ज॒नय॑ति । \newline
41. वृ॒द्धा मिन्द्र॒ इन्द्रो॑ वृ॒द्धां ॅवृ॒द्धा मिन्द्रः॑ । \newline
42. इन्द्रः॒ प्र प्रे न्द्र॒ इन्द्रः॒ प्र । \newline
43. प्र य॑च्छति यच्छति॒ प्र प्र य॑च्छति । \newline
44. य॒च्छ॒ति॒ य॒ज्ञो य॒ज्ञो य॑च्छति यच्छति य॒ज्ञ्ः । \newline
45. य॒ज्ञो वै वै य॒ज्ञो य॒ज्ञो वै । \newline
46. वै विष्णु॒र् विष्णु॒र् वै वै विष्णुः॑ । \newline
47. विष्णुः॑ प॒शवः॑ प॒शवो॒ विष्णु॒र् विष्णुः॑ प॒शवः॑ । \newline
48. प॒शवः॒ शिपिः॒ शिपिः॑ प॒शवः॑ प॒शवः॒ शिपिः॑ । \newline
49. शिपि॑र् य॒ज्ञे य॒ज्ञे शिपिः॒ शिपि॑र् य॒ज्ञे । \newline
50. य॒ज्ञ् ए॒वैव य॒ज्ञे य॒ज्ञ् ए॒व । \newline
51. ए॒व प॒शुषु॑ प॒शु ष्वे॒वैव प॒शुषु॑ । \newline
52. प॒शुषु॒ प्रति॒ प्रति॑ प॒शुषु॑ प॒शुषु॒ प्रति॑ । \newline
53. प्रति॑ तिष्ठति तिष्ठति॒ प्रति॒ प्रति॑ तिष्ठति । \newline
54. ति॒ष्ठ॒ति॒ न न ति॑ष्ठति तिष्ठति॒ न । \newline
55. न द्वे द्वे न न द्वे । \newline
56. द्वे य॑जेत यजेत॒ द्वे द्वे य॑जेत । \newline
57. द्वे इति॒ द्वे । \newline

\textbf{Ghana Paata } \newline

1. अ॒भ्यु॑दे त्यु॒दे त्य॒भ्या᳚(1॒)भ्यु॑देति॑ त्रे॒धा त्रे॒धो देत्य॒भ्या᳚(1॒)भ्यु॑देति॑ त्रे॒धा । \newline
2. उ॒देति॑ त्रे॒धा त्रे॒धोदे त्यु॒देति॑ त्रे॒धा त॑ण्डु॒लान् त॑ण्डु॒लान् त्रे॒धोदे त्यु॒देति॑ त्रे॒धा त॑ण्डु॒लान् । \newline
3. उ॒देतीत्यु॑त् - एति॑ । \newline
4. त्रे॒धा त॑ण्डु॒लान् त॑ण्डु॒लान् त्रे॒धा त्रे॒धा त॑ण्डु॒लान्. वि वि त॑ण्डु॒लान् त्रे॒धा त्रे॒धा त॑ण्डु॒लान्. वि । \newline
5. त॒ण्डु॒लान्. वि वि त॑ण्डु॒लान् त॑ण्डु॒लान्. वि भ॑जेद् भजे॒द् वि त॑ण्डु॒लान् त॑ण्डु॒लान्. वि भ॑जेत् । \newline
6. वि भ॑जेद् भजे॒द् वि वि भ॑जे॒द् ये ये भ॑जे॒द् वि वि भ॑जे॒द् ये । \newline
7. भ॒जे॒द् ये ये भ॑जेद् भजे॒द् ये म॑द्ध्य॒मा म॑द्ध्य॒मा ये भ॑जेद् भजे॒द् ये म॑द्ध्य॒माः । \newline
8. ये म॑द्ध्य॒मा म॑द्ध्य॒मा ये ये म॑द्ध्य॒माः स्युः स्युर् म॑द्ध्य॒मा ये ये म॑द्ध्य॒माः स्युः । \newline
9. म॒द्ध्य॒माः स्युः स्युर् म॑द्ध्य॒मा म॑द्ध्य॒माः स्यु स्ताꣳ स्तान् थ्स्युर् म॑द्ध्य॒मा म॑द्ध्य॒माः स्युस्तान् । \newline
10. स्यु स्ताꣳ स्तान् थ्स्युः स्युस्ता न॒ग्नये॒ ऽग्नये॒ तान् थ्स्युः स्यु स्ता न॒ग्नये᳚ । \newline
11. ता न॒ग्नये॒ ऽग्नये॒ ताꣳ स्ता न॒ग्नये॑ दा॒त्रे दा॒त्रे᳚ ऽग्नये॒ ताꣳ स्ता न॒ग्नये॑ दा॒त्रे । \newline
12. अ॒ग्नये॑ दा॒त्रे दा॒त्रे᳚ ऽग्नये॒ ऽग्नये॑ दा॒त्रे पु॑रो॒डाश॑म् पुरो॒डाश॑म् दा॒त्रे᳚ ऽग्नये॒ ऽग्नये॑ दा॒त्रे पु॑रो॒डाश᳚म् । \newline
13. दा॒त्रे पु॑रो॒डाश॑म् पुरो॒डाश॑म् दा॒त्रे दा॒त्रे पु॑रो॒डाश॑ म॒ष्टाक॑पाल म॒ष्टाक॑पालम् पुरो॒डाश॑म् दा॒त्रे दा॒त्रे पु॑रो॒डाश॑ म॒ष्टाक॑पालम् । \newline
14. पु॒रो॒डाश॑ म॒ष्टाक॑पाल म॒ष्टाक॑पालम् पुरो॒डाश॑म् पुरो॒डाश॑ म॒ष्टाक॑पालम् कुर्यात् कुर्या द॒ष्टाक॑पालम् पुरो॒डाश॑म् पुरो॒डाश॑ म॒ष्टाक॑पालम् कुर्यात् । \newline
15. अ॒ष्टाक॑पालम् कुर्यात् कुर्या द॒ष्टाक॑पाल म॒ष्टाक॑पालम् कुर्या॒द् ये ये कु॑र्या द॒ष्टाक॑पाल म॒ष्टाक॑पालम् कुर्या॒द् ये । \newline
16. अ॒ष्टाक॑पाल॒मित्य॒ष्टा - क॒पा॒ल॒म् । \newline
17. कु॒र्या॒द् ये ये कु॑र्यात् कुर्या॒द् ये स्थवि॑ष्ठाः॒ स्थवि॑ष्ठा॒ ये कु॑र्यात् कुर्या॒द् ये स्थवि॑ष्ठाः । \newline
18. ये स्थवि॑ष्ठाः॒ स्थवि॑ष्ठा॒ ये ये स्थवि॑ष्ठा॒ स्ताꣳ स्तान् थ्स्थवि॑ष्ठा॒ ये ये स्थवि॑ष्ठा॒ स्तान् । \newline
19. स्थवि॑ष्ठा॒ स्ताꣳ स्तान् थ्स्थवि॑ष्ठाः॒ स्थवि॑ष्ठा॒ स्ता निन्द्रा॒ये न्द्रा॑य॒ तान् थ्स्थवि॑ष्ठाः॒ स्थवि॑ष्ठा॒ स्ता निन्द्रा॑य । \newline
20. ता निन्द्रा॒ये न्द्रा॑य॒ ताꣳ स्ता निन्द्रा॑य प्रदा॒त्रे प्र॑दा॒त्र इन्द्रा॑य॒ ताꣳ स्ता निन्द्रा॑य प्रदा॒त्रे । \newline
21. इन्द्रा॑य प्रदा॒त्रे प्र॑दा॒त्र इन्द्रा॒ये न्द्रा॑य प्रदा॒त्रे द॒धन् द॒धन् प्र॑दा॒त्र इन्द्रा॒ये न्द्रा॑य प्रदा॒त्रे द॒धन्न् । \newline
22. प्र॒दा॒त्रे द॒धन् द॒धन् प्र॑दा॒त्रे प्र॑दा॒त्रे द॒धꣳ श्च॒रुम् च॒रुम् द॒धन् प्र॑दा॒त्रे प्र॑दा॒त्रे द॒धꣳ श्च॒रुम् । \newline
23. प्र॒दा॒त्र इति॑ प्र - दा॒त्रे । \newline
24. द॒धꣳ श्च॒रुम् च॒रुम् द॒धन् द॒धꣳ श्च॒रुं ॅये ये च॒रुम् द॒धन् द॒धꣳ श्च॒रुं ॅये । \newline
25. च॒रुं ॅये ये च॒रुम् च॒रुं ॅये ऽणि॑ष्ठा॒ अणि॑ष्ठा॒ ये च॒रुम् च॒रुं ॅये ऽणि॑ष्ठाः । \newline
26. ये ऽणि॑ष्ठा॒ अणि॑ष्ठा॒ ये ये ऽणि॑ष्ठा॒ स्ताꣳ स्ता नणि॑ष्ठा॒ ये ये ऽणि॑ष्ठा॒ स्तान् । \newline
27. अणि॑ष्ठा॒ स्ताꣳ स्ता नणि॑ष्ठा॒ अणि॑ष्ठा॒ स्तान्. विष्ण॑वे॒ विष्ण॑वे॒ ता नणि॑ष्ठा॒ अणि॑ष्ठा॒ स्तान्. विष्ण॑वे । \newline
28. तान्. विष्ण॑वे॒ विष्ण॑वे॒ ताꣳ स्तान्. विष्ण॑वे शिपिवि॒ष्टाय॑ शिपिवि॒ष्टाय॒ विष्ण॑वे॒ ताꣳ स्तान्. विष्ण॑वे शिपिवि॒ष्टाय॑ । \newline
29. विष्ण॑वे शिपिवि॒ष्टाय॑ शिपिवि॒ष्टाय॒ विष्ण॑वे॒ विष्ण॑वे शिपिवि॒ष्टाय॑ शृ॒ते शृ॒ते शि॑पिवि॒ष्टाय॒ विष्ण॑वे॒ विष्ण॑वे शिपिवि॒ष्टाय॑ शृ॒ते । \newline
30. शि॒पि॒वि॒ष्टाय॑ शृ॒ते शृ॒ते शि॑पिवि॒ष्टाय॑ शिपिवि॒ष्टाय॑ शृ॒ते च॒रुम् च॒रुꣳ शृ॒ते शि॑पिवि॒ष्टाय॑ शिपिवि॒ष्टाय॑ शृ॒ते च॒रुम् । \newline
31. शि॒पि॒वि॒ष्टायेति॑ शिपि - वि॒ष्टाय॑ । \newline
32. शृ॒ते च॒रुम् च॒रुꣳ शृ॒ते शृ॒ते च॒रु म॒ग्नि र॒ग्नि श्च॒रुꣳ शृ॒ते शृ॒ते च॒रु म॒ग्निः । \newline
33. च॒रु म॒ग्नि र॒ग्नि श्च॒रुम् च॒रु म॒ग्नि रे॒वैवाग्नि श्च॒रुम् च॒रु म॒ग्निरे॒व । \newline
34. अ॒ग्नि रे॒वैवाग्नि र॒ग्नि रे॒वास्मा॑ अस्मा ए॒वाग्नि र॒ग्नि रे॒वास्मै᳚ । \newline
35. ए॒वास्मा॑ अस्मा ए॒वैवास्मै᳚ प्र॒जाम् प्र॒जा म॑स्मा ए॒वैवास्मै᳚ प्र॒जाम् । \newline
36. अ॒स्मै॒ प्र॒जाम् प्र॒जा म॑स्मा अस्मै प्र॒जाम् प्र॑ज॒नय॑ति प्रज॒नय॑ति प्र॒जा म॑स्मा अस्मै प्र॒जाम् प्र॑ज॒नय॑ति । \newline
37. प्र॒जाम् प्र॑ज॒नय॑ति प्रज॒नय॑ति प्र॒जाम् प्र॒जाम् प्र॑ज॒नय॑ति वृ॒द्धां ॅवृ॒द्धाम् प्र॑ज॒नय॑ति प्र॒जाम् प्र॒जाम् प्र॑ज॒नय॑ति वृ॒द्धाम् । \newline
38. प्र॒जामिति॑ प्र - जाम् । \newline
39. प्र॒ज॒नय॑ति वृ॒द्धां ॅवृ॒द्धाम् प्र॑ज॒नय॑ति प्रज॒नय॑ति वृ॒द्धा मिन्द्र॒ इन्द्रो॑ वृ॒द्धाम् प्र॑ज॒नय॑ति प्रज॒नय॑ति वृ॒द्धा मिन्द्रः॑ । \newline
40. प्र॒ज॒नय॒तीति॑ प्र - ज॒नय॑ति । \newline
41. वृ॒द्धा मिन्द्र॒ इन्द्रो॑ वृ॒द्धां ॅवृ॒द्धा मिन्द्रः॒ प्र प्रे न्द्रो॑ वृ॒द्धां ॅवृ॒द्धा मिन्द्रः॒ प्र । \newline
42. इन्द्रः॒ प्र प्रे न्द्र॒ इन्द्रः॒ प्र य॑च्छति यच्छति॒ प्रे न्द्र॒ इन्द्रः॒ प्र य॑च्छति । \newline
43. प्र य॑च्छति यच्छति॒ प्र प्र य॑च्छति य॒ज्ञो य॒ज्ञो य॑च्छति॒ प्र प्र य॑च्छति य॒ज्ञ्ः । \newline
44. य॒च्छ॒ति॒ य॒ज्ञो य॒ज्ञो य॑च्छति यच्छति य॒ज्ञो वै वै य॒ज्ञो य॑च्छति यच्छति य॒ज्ञो वै । \newline
45. य॒ज्ञो वै वै य॒ज्ञो य॒ज्ञो वै विष्णु॒र् विष्णु॒र् वै य॒ज्ञो य॒ज्ञो वै विष्णुः॑ । \newline
46. वै विष्णु॒र् विष्णु॒र् वै वै विष्णुः॑ प॒शवः॑ प॒शवो॒ विष्णु॒र् वै वै विष्णुः॑ प॒शवः॑ । \newline
47. विष्णुः॑ प॒शवः॑ प॒शवो॒ विष्णु॒र् विष्णुः॑ प॒शवः॒ शिपिः॒ शिपिः॑ प॒शवो॒ विष्णु॒र् विष्णुः॑ प॒शवः॒ शिपिः॑ । \newline
48. प॒शवः॒ शिपिः॒ शिपिः॑ प॒शवः॑ प॒शवः॒ शिपि॑र् य॒ज्ञे य॒ज्ञे शिपिः॑ प॒शवः॑ प॒शवः॒ शिपि॑र् य॒ज्ञे । \newline
49. शिपि॑र् य॒ज्ञे य॒ज्ञे शिपिः॒ शिपि॑र् य॒ज्ञ् ए॒वैव य॒ज्ञे शिपिः॒ शिपि॑र् य॒ज्ञ् ए॒व । \newline
50. य॒ज्ञ् ए॒वैव य॒ज्ञे य॒ज्ञ् ए॒व प॒शुषु॑ प॒शुष्वे॒व य॒ज्ञे य॒ज्ञ् ए॒व प॒शुषु॑ । \newline
51. ए॒व प॒शुषु॑ प॒शुष्वे॒वैव प॒शुषु॒ प्रति॒ प्रति॑ प॒शुष्वे॒वैव प॒शुषु॒ प्रति॑ । \newline
52. प॒शुषु॒ प्रति॒ प्रति॑ प॒शुषु॑ प॒शुषु॒ प्रति॑ तिष्ठति तिष्ठति॒ प्रति॑ प॒शुषु॑ प॒शुषु॒ प्रति॑ तिष्ठति । \newline
53. प्रति॑ तिष्ठति तिष्ठति॒ प्रति॒ प्रति॑ तिष्ठति॒ न न ति॑ष्ठति॒ प्रति॒ प्रति॑ तिष्ठति॒ न । \newline
54. ति॒ष्ठ॒ति॒ न न ति॑ष्ठति तिष्ठति॒ न द्वे द्वे न ति॑ष्ठति तिष्ठति॒ न द्वे । \newline
55. न द्वे द्वे न न द्वे य॑जेत यजेत॒ द्वे न न द्वे य॑जेत । \newline
56. द्वे य॑जेत यजेत॒ द्वे द्वे य॑जेत॒ यद् यद् य॑जेत॒ द्वे द्वे य॑जेत॒ यत् । \newline
57. द्वे इति॒ द्वे । \newline
\pagebreak
\markright{ TS 2.5.5.3  \hfill https://www.vedavms.in \hfill}
\addcontentsline{toc}{section}{ TS 2.5.5.3 }
\section*{ TS 2.5.5.3 }

\textbf{TS 2.5.5.3 } \newline
\textbf{Samhita Paata} \newline

य॑जेत॒ यत् पूर्व॑या संप्र॒ति यजे॒तोत्त॑रया छ॒म्बट्कु॑र्या॒द्यदुत्त॑रया सम्प्र॒ति यजे॑त॒ पूर्व॑या छ॒म्बट्कु॑र्या॒न्नेष्टि॒र्भव॑ति॒ न य॒ज्ञ्स्तदनु॑ ह्रीतमु॒ख्य॑पग॒ल्भो जा॑यत॒ एका॑मे॒व य॑जेत प्रग॒ल्भो᳚ऽस्य जाय॒ते ऽना॑दृत्य॒ तद् द्वे ए॒व य॑जेत यज्ञ् मु॒खमे॒व पूर्व॑या॒ऽऽलभ॑ते॒ यज॑त॒ उत्त॑रया दे॒वता॑ ए॒व पूर्व॑या ऽवरु॒न्ध इ॑न्द्रि॒य-मुत्त॑रया देवलो॒कमे॒व - [  ] \newline

\textbf{Pada Paata} \newline

य॒जे॒त॒ । यत् । पूर्व॑या । सं॒प्र॒तीति॑ सं - प्र॒ति । यजे॑त । उत्त॑र॒येत्युत् - त॒र॒या॒ । छ॒बंट् । कु॒र्या॒त् । यत् । उत्त॑र॒येत्युत् - त॒र॒या॒ । सं॒प्र॒तीति॑ सं-प्र॒ति । यजे॑त । पूर्व॑या । छ॒बंट् । कु॒र्या॒त् । न । इष्टिः॑ । भव॑ति । न । य॒ज्ञ्ः । तत् । अन्विति॑ । ह्री॒त॒मु॒खीति॑ ह्रीत - मु॒खी । अ॒प॒ग॒ल्भ इत्य॑प - ग॒ल्भः । जा॒य॒ते॒ । एका᳚म् । ए॒व । य॒जे॒त॒ । प्र॒ग॒ल्भ इति॑ प्र - ग॒ल्भः । अ॒स्य॒ । जा॒य॒ते॒ । अना॑दृ॒त्येत्यना᳚-दृ॒त्य॒ । तत् । द्वे इति॑ । ए॒व । य॒जे॒त॒ । य॒ज्ञ्॒मु॒खमिति॑ यज्ञ् - मु॒खम् । ए॒व । पूर्व॑या । आ॒लभ॑त॒ इत्या᳚ - लभ॑ते । यज॑ते । उत्त॑र॒येत्युत् - त॒र॒या॒ । दे॒वताः᳚ । ए॒व । पूर्व॑या । अ॒व॒रु॒न्ध इत्य॑व - रु॒न्धे । इ॒न्द्रि॒यम् । उत्त॑र॒येत्युत् - त॒र॒या॒ । दे॒व॒लो॒कमिति॑ देव - लो॒कम् । ए॒व ।  \newline


\textbf{Krama Paata} \newline

य॒जे॒त॒ यत् । यत् पूर्व॑या । पूर्व॑या सम्प्र॒ति । स॒म्प्र॒ति यजे॑त । स॒म्प्र॒तीति॑ सम् - प्र॒ति । यजे॒तोत्त॑रया । उत्त॑रया छ॒म्बट् । उत्त॑र॒येत्युत् - त॒र॒या॒ । छ॒म्बट् कु॑र्यात् । कु॒र्या॒द् यत् । यदुत्त॑रया । उत्त॑रया सम्प्र॒ति । उत्त॑र॒येत्युत् - त॒र॒या॒ । स॒म्प्र॒ति यजे॑त । स॒म्प्र॒तीति॑ सम् - प्र॒ति । यजे॑त॒ पूर्व॑या । पूर्व॑या छ॒म्बट् । छ॒म्बट् कु॑र्यात् । कु॒र्या॒न् न । नेष्टिः॑ । इष्टि॒र् भव॑ति । भव॑ति॒ न । न य॒ज्ञ्ः । य॒ज्ञ्स्तत् । तदनु॑ । अनु॑ ह्रीतमु॒खी । ह्री॒त॒मु॒ख्य॑पग॒ल्भः । ह्री॒त॒मु॒खीति॑ ह्रीत - मु॒खी । अ॒प॒ग॒ल्भो जा॑यते । अ॒प॒ग॒ल्भ इत्य॑प - ग॒ल्भः । जा॒य॒त॒ एका᳚म् । एका॑मे॒व । ए॒व य॑जेत । य॒जे॒त॒ प्र॒ग॒ल्भः । प्र॒ग॒ल्भो᳚ऽस्य । प्र॒ग॒ल्भ इति॑ प्र - ग॒ल्भः । अ॒स्य॒ जा॒य॒ते॒ । जा॒य॒तेऽना॑दृत्य । अना॑दृत्य॒ तत् । अना॑दृ॒त्येत्यना᳚ - दृ॒त्य॒ । तद् द्वे । द्वे ए॒व । द्वे इति॒ द्वे । ए॒व य॑जेत । य॒जे॒त॒ य॒ज्ञ्॒मु॒खम् । य॒ज्ञ्॒मु॒खमे॒व । य॒ज्ञ्॒मु॒खमिति॑ यज्ञ् - मु॒खम् । ए॒व पूर्व॑या । पूर्व॑या॒ ऽऽलभ॑ते । आ॒लभ॑ते॒ यज॑ते । आ॒लभ॑त॒ इत्या᳚ - लभ॑ते । यज॑त॒ उत्त॑रया । उत्त॑रया दे॒वताः᳚ । उत्त॑र॒येत्युत् - त॒र॒या॒ । दे॒वता॑ ए॒व । ए॒व पूर्व॑या । पूर्व॑या ऽवरु॒न्धे । अ॒व॒रु॒न्ध इ॑न्द्रि॒यम् । अ॒व॒रु॒न्ध इत्य॑व - रु॒न्धे । इ॒न्द्रि॒यमुत्त॑रया । उत्त॑रया देवलो॒कम् । उत्त॑र॒येत्युत् - त॒र॒या॒ । दे॒व॒लो॒कमे॒व । दे॒व॒लो॒कमिति॑ देव - लो॒कम् । ए॒व पूर्व॑या \newline

\textbf{Jatai Paata} \newline

1. य॒जे॒त॒ यद् यद् य॑जेत यजेत॒ यत् । \newline
2. यत् पूर्व॑या॒ पूर्व॑या॒ यद् यत् पूर्व॑या । \newline
3. पूर्व॑या संप्र॒ति सं॑प्र॒ति पूर्व॑या॒ पूर्व॑या संप्र॒ति । \newline
4. सं॒प्र॒ति यजे॑त॒ यजे॑त संप्र॒ति सं॑प्र॒ति यजे॑त । \newline
5. सं॒प्र॒तीति॑ सं - प्र॒ति । \newline
6. यजे॒तोत्त॑ र॒योत्त॑रया॒ यजे॑त॒ यजे॒तोत्त॑रया । \newline
7. उत्त॑रया छं॒बट् छं॒ब डुत्त॑र॒यो त्त॑रया छं॒बट् । \newline
8. उत्त॑र॒येत्युत् - त॒र॒या॒ । \newline
9. छं॒बट् कु॑र्यात् कुर्याच् छं॒बट् छं॒बट् कु॑र्यात् । \newline
10. कु॒र्या॒द् यद् यत् कु॑र्यात् कुर्या॒द् यत् । \newline
11. यदुत्त॑र॒ योत्त॑रया॒ यद् यदुत्त॑रया । \newline
12. उत्त॑रया संप्र॒ति सं॑प्र॒ त्युत्त॑र॒यो त्त॑रया संप्र॒ति । \newline
13. उत्त॑र॒येत्युत् - त॒र॒या॒ । \newline
14. सं॒प्र॒ति यजे॑त॒ यजे॑त संप्र॒ति सं॑प्र॒ति यजे॑त । \newline
15. सं॒प्र॒तीति॑ सं - प्र॒ति । \newline
16. यजे॑त॒ पूर्व॑या॒ पूर्व॑या॒ यजे॑त॒ यजे॑त॒ पूर्व॑या । \newline
17. पूर्व॑या छं॒बट् छं॒बट् पूर्व॑या॒ पूर्व॑या छं॒बट् । \newline
18. छं॒बट् कु॑र्यात् कुर्याच् छं॒बट् छं॒बट् कु॑र्यात् । \newline
19. कु॒र्या॒न् न न कु॑र्यात् कुर्या॒न् न । \newline
20. ने ष्टि॒ रिष्टि॒र् न ने ष्टिः॑ । \newline
21. इष्टि॒र् भव॑ति॒ भव॒तीष्टि॒ रिष्टि॒र् भव॑ति । \newline
22. भव॑ति॒ न न भव॑ति॒ भव॑ति॒ न । \newline
23. न य॒ज्ञो य॒ज्ञो न न य॒ज्ञ्ः । \newline
24. य॒ज्ञ् स्तत् तद् य॒ज्ञो य॒ज्ञ् स्तत् । \newline
25. तदन्वनु॒ तत् तदनु॑ । \newline
26. अनु॑ ह्रीतमु॒खी ह्री॑तमु॒ ख्यन्वनु॑ ह्रीतमु॒खी । \newline
27. ह्री॒त॒मु॒ ख्य॑पग॒ल्भो॑ ऽपग॒ल्भो ह्री॑तमु॒खी ह्री॑तमु॒ ख्य॑पग॒ल्भः । \newline
28. ह्री॒त॒मु॒खीति॑ ह्रीत - मु॒खी । \newline
29. अ॒प॒ग॒ल्भो जा॑यते जायते ऽपग॒ल्भो॑ ऽपग॒ल्भो जा॑यते । \newline
30. अ॒प॒ग॒ल्भ इत्य॑प - ग॒ल्भः । \newline
31. जा॒य॒त॒ एका॒ मेका᳚म् जायते जायत॒ एका᳚म् । \newline
32. एका॑ मे॒वैवैका॒ मेका॑ मे॒व । \newline
33. ए॒व य॑जेत यजेतै॒वैव य॑जेत । \newline
34. य॒जे॒त॒ प्र॒ग॒ल्भः प्र॑ग॒ल्भो य॑जेत यजेत प्रग॒ल्भः । \newline
35. प्र॒ग॒ल्भो᳚ ऽस्यास्य प्रग॒ल्भः प्र॑ग॒ल्भो᳚ ऽस्य । \newline
36. प्र॒ग॒ल्भ इति॑ प्र - ग॒ल्भः । \newline
37. अ॒स्य॒ जा॒य॒ते॒ जा॒य॒ते॒ ऽस्या॒स्य॒ जा॒य॒ते॒ । \newline
38. जा॒य॒ते ऽना॑दृ॒त्या ना॑दृत्य जायते जाय॒ते ऽना॑दृत्य । \newline
39. अना॑दृत्य॒ तत् तदना॑दृ॒ त्याना॑दृत्य॒ तत् । \newline
40. अना॑दृ॒त्येत्यना᳚ - दृ॒त्य॒ । \newline
41. तद् द्वे द्वे तत् तद् द्वे । \newline
42. द्वे ए॒वैव द्वे द्वे ए॒व । \newline
43. द्वे इति॒ द्वे । \newline
44. ए॒व य॑जेत यजेतै॒वैव य॑जेत । \newline
45. य॒जे॒त॒ य॒ज्ञ्॒मु॒खं ॅय॑ज्ञ्मु॒खं ॅय॑जेत यजेत यज्ञ्मु॒खम् । \newline
46. य॒ज्ञ्॒मु॒ख मे॒वैव य॑ज्ञ्मु॒खं ॅय॑ज्ञ्मु॒ख मे॒व । \newline
47. य॒ज्ञ्॒मु॒खमिति॑ यज्ञ् - मु॒खम् । \newline
48. ए॒व पूर्व॑या॒ पूर्व॑यै॒वैव पूर्व॑या । \newline
49. पूर्व॑या॒ ऽऽलभ॑त आ॒लभ॑ते॒ पूर्व॑या॒ पूर्व॑या॒ ऽऽलभ॑ते । \newline
50. आ॒लभ॑ते॒ यज॑ते॒ यज॑त आ॒लभ॑त आ॒लभ॑ते॒ यज॑ते । \newline
51. आ॒लभ॑त॒ इत्या᳚ - लभ॑ते । \newline
52. यज॑त॒ उत्त॑र॒यो त्त॑रया॒ यज॑ते॒ यज॑त॒ उत्त॑रया । \newline
53. उत्त॑रया दे॒वता॑ दे॒वता॒ उत्त॑र॒यो त्त॑रया दे॒वताः᳚ । \newline
54. उत्त॑र॒येत्युत् - त॒र॒या॒ । \newline
55. दे॒वता॑ ए॒वैव दे॒वता॑ दे॒वता॑ ए॒व । \newline
56. ए॒व पूर्व॑या॒ पूर्व॑यै॒वैव पूर्व॑या । \newline
57. पूर्व॑या ऽवरु॒न्धे॑ ऽवरु॒न्धे पूर्व॑या॒ पूर्व॑या ऽवरु॒न्धे । \newline
58. अ॒व॒रु॒न्ध इ॑न्द्रि॒य मि॑न्द्रि॒य म॑वरु॒न्धे॑ ऽवरु॒न्ध इ॑न्द्रि॒यम् । \newline
59. अ॒व॒रु॒न्ध इत्य॑व - रु॒न्धे । \newline
60. इ॒न्द्रि॒य मुत्त॑र॒यो त्त॑रयेन्द्रि॒य मि॑न्द्रि॒य मुत्त॑रया । \newline
61. उत्त॑रया देवलो॒कम् दे॑वलो॒क मुत्त॑र॒यो त्त॑रया देवलो॒कम् । \newline
62. उत्त॑र॒येत्युत् - त॒र॒या॒ । \newline
63. दे॒व॒लो॒क मे॒वैव दे॑वलो॒कम् दे॑वलो॒क मे॒व । \newline
64. दे॒व॒लो॒कमिति॑ देव - लो॒कम् । \newline
65. ए॒व पूर्व॑या॒ पूर्व॑यै॒वैव पूर्व॑या । \newline

\textbf{Ghana Paata } \newline

1. य॒जे॒त॒ यद् यद् य॑जेत यजेत॒ यत् पूर्व॑या॒ पूर्व॑या॒ यद् य॑जेत यजेत॒ यत् पूर्व॑या । \newline
2. यत् पूर्व॑या॒ पूर्व॑या॒ यद् यत् पूर्व॑या संप्र॒ति सं॑प्र॒ति पूर्व॑या॒ यद् यत् पूर्व॑या संप्र॒ति । \newline
3. पूर्व॑या संप्र॒ति सं॑प्र॒ति पूर्व॑या॒ पूर्व॑या संप्र॒ति यजे॑त॒ यजे॑त संप्र॒ति पूर्व॑या॒ पूर्व॑या संप्र॒ति यजे॑त । \newline
4. सं॒प्र॒ति यजे॑त॒ यजे॑त संप्र॒ति सं॑प्र॒ति यजे॒तोत्त॑र॒यो त्त॑रया॒ यजे॑त संप्र॒ति सं॑प्र॒ति यजे॒तोत्त॑रया । \newline
5. सं॒प्र॒तीति॑ सं - प्र॒ति । \newline
6. यजे॒तोत्त॑र॒यो त्त॑रया॒ यजे॑त॒ यजे॒तोत्त॑रया छं॒बट् छं॒ब डुत्त॑रया॒ यजे॑त॒ यजे॒तोत्त॑रया छं॒बट् । \newline
7. उत्त॑रया छं॒बट् छं॒ब डुत्त॑र॒यो त्त॑रया छं॒बट् कु॑र्यात् कुर्याच् छं॒ब डुत्त॑र॒यो त्त॑रया छं॒बट् कु॑र्यात् । \newline
8. उत्त॑र॒येत्युत् - त॒र॒या॒ । \newline
9. छं॒बट् कु॑र्यात् कुर्याच् छं॒बट् छं॒बट् कु॑र्या॒द् यद् यत् कु॑र्याच् छं॒बट् छं॒बट् कु॑र्या॒द् यत् । \newline
10. कु॒र्या॒द् यद् यत् कु॑र्यात् कुर्या॒द् यदुत्त॑र॒यो त्त॑रया॒ यत् कु॑र्यात् कुर्या॒द् यदुत्त॑रया । \newline
11. यदुत्त॑र॒यो त्त॑रया॒ यद् यदुत्त॑रया संप्र॒ति सं॑प्र॒ त्युत्त॑रया॒ यद् यदुत्त॑रया संप्र॒ति । \newline
12. उत्त॑रया संप्र॒ति सं॑प्र॒ त्युत्त॑र॒यो त्त॑रया संप्र॒ति यजे॑त॒ यजे॑त संप्र॒ त्युत्त॑र॒यो त्त॑रया संप्र॒ति यजे॑त । \newline
13. उत्त॑र॒येत्युत् - त॒र॒या॒ । \newline
14. सं॒प्र॒ति यजे॑त॒ यजे॑त संप्र॒ति सं॑प्र॒ति यजे॑त॒ पूर्व॑या॒ पूर्व॑या॒ यजे॑त संप्र॒ति सं॑प्र॒ति यजे॑त॒ पूर्व॑या । \newline
15. सं॒प्र॒तीति॑ सं - प्र॒ति । \newline
16. यजे॑त॒ पूर्व॑या॒ पूर्व॑या॒ यजे॑त॒ यजे॑त॒ पूर्व॑या छं॒बट् छं॒बट् पूर्व॑या॒ यजे॑त॒ यजे॑त॒ पूर्व॑या छं॒बट् । \newline
17. पूर्व॑या छं॒बट् छं॒बट् पूर्व॑या॒ पूर्व॑या छं॒बट् कु॑र्यात् कुर्याच् छं॒बट् पूर्व॑या॒ पूर्व॑या छं॒बट् कु॑र्यात् । \newline
18. छं॒बट् कु॑र्यात् कुर्याच् छं॒बट् छं॒बट् कु॑र्या॒न् न न कु॑र्याच् छं॒बट् छं॒बट् कु॑र्या॒न् न । \newline
19. कु॒र्या॒न् न न कु॑र्यात् कुर्या॒न् ने ष्टि॒रिष्टि॒र् न कु॑र्यात् कुर्या॒न् ने ष्टिः॑ । \newline
20. ने ष्टि॒रिष्टि॒र् न ने ष्टि॒र् भव॑ति॒ भव॒तीष्टि॒र् न ने ष्टि॒र् भव॑ति । \newline
21. इष्टि॒र् भव॑ति॒ भव॒तीष्टि॒ रिष्टि॒र् भव॑ति॒ न न भव॒तीष्टि॒ रिष्टि॒र् भव॑ति॒ न । \newline
22. भव॑ति॒ न न भव॑ति॒ भव॑ति॒ न य॒ज्ञो य॒ज्ञो न भव॑ति॒ भव॑ति॒ न य॒ज्ञ्ः । \newline
23. न य॒ज्ञो य॒ज्ञो न न य॒ज्ञ् स्तत् तद् य॒ज्ञो न न य॒ज्ञ् स्तत् । \newline
24. य॒ज्ञ् स्तत् तद् य॒ज्ञो य॒ज्ञ् स्तदन्वनु॒ तद् य॒ज्ञो य॒ज्ञ् स्तदनु॑ । \newline
25. तदन्वनु॒ तत् तदनु॑ ह्रीतमु॒खी ह्री॑तमु॒ख्यनु॒ तत् तदनु॑ ह्रीतमु॒खी । \newline
26. अनु॑ ह्रीतमु॒खी ह्री॑तमु॒ख्यन्वनु॑ ह्रीतमु॒ख्य॑पग॒ल्भो॑ ऽपग॒ल्भो ह्री॑तमु॒ख्यन्वनु॑ ह्रीतमु॒ख्य॑पग॒ल्भः । \newline
27. ह्री॒त॒मु॒ख्य॑ पग॒ल्भो॑ ऽपग॒ल्भो ह्री॑तमु॒खी ह्री॑तमु॒ख्य॑ पग॒ल्भो जा॑यते जायते ऽपग॒ल्भो ह्री॑तमु॒खी ह्री॑तमु॒ख्य॑ पग॒ल्भो जा॑यते । \newline
28. ह्री॒त॒मु॒खीति॑ ह्रीत - मु॒खी । \newline
29. अ॒प॒ग॒ल्भो जा॑यते जायते ऽपग॒ल्भो॑ ऽपग॒ल्भो जा॑यत॒ एका॒ मेका᳚म् जायते ऽपग॒ल्भो॑ ऽपग॒ल्भो जा॑यत॒ एका᳚म् । \newline
30. अ॒प॒ग॒ल्भ इत्य॑प - ग॒ल्भः । \newline
31. जा॒य॒त॒ एका॒ मेका᳚म् जायते जायत॒ एका॑ मे॒वैवैका᳚म् जायते जायत॒ एका॑ मे॒व । \newline
32. एका॑ मे॒वैवैका॒ मेका॑ मे॒व य॑जेत यजेतै॒वैका॒ मेका॑ मे॒व य॑जेत । \newline
33. ए॒व य॑जेत यजेतै॒वैव य॑जेत प्रग॒ल्भः प्र॑ग॒ल्भो य॑जेतै॒वैव य॑जेत प्रग॒ल्भः । \newline
34. य॒जे॒त॒ प्र॒ग॒ल्भः प्र॑ग॒ल्भो य॑जेत यजेत प्रग॒ल्भो᳚ ऽस्यास्य प्रग॒ल्भो य॑जेत यजेत प्रग॒ल्भो᳚ ऽस्य । \newline
35. प्र॒ग॒ल्भो᳚ ऽस्यास्य प्रग॒ल्भः प्र॑ग॒ल्भो᳚ ऽस्य जायते जायते ऽस्य प्रग॒ल्भः प्र॑ग॒ल्भो᳚ ऽस्य जायते । \newline
36. प्र॒ग॒ल्भ इति॑ प्र - ग॒ल्भः । \newline
37. अ॒स्य॒ जा॒य॒ते॒ जा॒य॒ते॒ ऽस्या॒स्य॒ जा॒य॒ते ऽना॑दृ॒त्या ना॑दृत्य जायते ऽस्यास्य जाय॒ते ऽना॑दृत्य । \newline
38. जा॒य॒ते ऽना॑दृ॒त्या ना॑दृत्य जायते जाय॒ते ऽना॑दृत्य॒ तत् तदना॑दृत्य जायते जाय॒ते ऽना॑दृत्य॒ तत् । \newline
39. अना॑दृत्य॒ तत् तदना॑दृ॒त्या ना॑दृत्य॒ तद् द्वे द्वे तदना॑दृ॒त्या ना॑दृत्य॒ तद् द्वे । \newline
40. अना॑दृ॒त्येत्यना᳚ - दृ॒त्य॒ । \newline
41. तद् द्वे द्वे तत् तद् द्वे ए॒वैव द्वे तत् तद् द्वे ए॒व । \newline
42. द्वे ए॒वैव द्वे द्वे ए॒व य॑जेत यजेतै॒व द्वे द्वे ए॒व य॑जेत । \newline
43. द्वे इति॒ द्वे । \newline
44. ए॒व य॑जेत यजेतै॒वैव य॑जेत यज्ञ्मु॒खं ॅय॑ज्ञ्मु॒खं ॅय॑जेतै॒वैव य॑जेत यज्ञ्मु॒खम् । \newline
45. य॒जे॒त॒ य॒ज्ञ्॒मु॒खं ॅय॑ज्ञ्मु॒खं ॅय॑जेत यजेत यज्ञ्मु॒ख मे॒वैव य॑ज्ञ्मु॒खं ॅय॑जेत यजेत यज्ञ्मु॒ख मे॒व । \newline
46. य॒ज्ञ्॒मु॒ख मे॒वैव य॑ज्ञ्मु॒खं ॅय॑ज्ञ्मु॒ख मे॒व पूर्व॑या॒ पूर्व॑यै॒व य॑ज्ञ्मु॒खं ॅय॑ज्ञ्मु॒ख मे॒व पूर्व॑या । \newline
47. य॒ज्ञ्॒मु॒खमिति॑ यज्ञ् - मु॒खम् । \newline
48. ए॒व पूर्व॑या॒ पूर्व॑यै॒वैव पूर्व॑या॒ ऽऽलभ॑त आ॒लभ॑ते॒ पूर्व॑यै॒वैव पूर्व॑या॒ ऽऽलभ॑ते । \newline
49. पूर्व॑या॒ ऽऽलभ॑त आ॒लभ॑ते॒ पूर्व॑या॒ पूर्व॑या॒ ऽऽलभ॑ते॒ यज॑ते॒ यज॑त आ॒लभ॑ते॒ पूर्व॑या॒ पूर्व॑या॒ ऽऽलभ॑ते॒ यज॑ते । \newline
50. आ॒लभ॑ते॒ यज॑ते॒ यज॑त आ॒लभ॑त आ॒लभ॑ते॒ यज॑त॒ उत्त॑र॒योत्त॑रया॒ यज॑त आ॒लभ॑त आ॒लभ॑ते॒ यज॑त॒ उत्त॑रया । \newline
51. आ॒लभ॑त॒ इत्या᳚ - लभ॑ते । \newline
52. यज॑त॒ उत्त॑र॒योत्त॑रया॒ यज॑ते॒ यज॑त॒ उत्त॑रया दे॒वता॑ दे॒वता॒ उत्त॑रया॒ यज॑ते॒ यज॑त॒ उत्त॑रया दे॒वताः᳚ । \newline
53. उत्त॑रया दे॒वता॑ दे॒वता॒ उत्त॑र॒योत्त॑रया दे॒वता॑ ए॒वैव दे॒वता॒ उत्त॑र॒योत्त॑रया दे॒वता॑ ए॒व । \newline
54. उत्त॑र॒येत्युत् - त॒र॒या॒ । \newline
55. दे॒वता॑ ए॒वैव दे॒वता॑ दे॒वता॑ ए॒व पूर्व॑या॒ पूर्व॑यै॒व दे॒वता॑ दे॒वता॑ ए॒व पूर्व॑या । \newline
56. ए॒व पूर्व॑या॒ पूर्व॑यै॒वैव पूर्व॑या ऽवरु॒न्धे॑ ऽवरु॒न्धे पूर्व॑यै॒वैव पूर्व॑या ऽवरु॒न्धे । \newline
57. पूर्व॑या ऽवरु॒न्धे॑ ऽवरु॒न्धे पूर्व॑या॒ पूर्व॑या ऽवरु॒न्ध इ॑न्द्रि॒य मि॑न्द्रि॒य म॑वरु॒न्धे पूर्व॑या॒ पूर्व॑या ऽवरु॒न्ध इ॑न्द्रि॒यम् । \newline
58. अ॒व॒रु॒न्ध इ॑न्द्रि॒य मि॑न्द्रि॒य म॑वरु॒न्धे॑ ऽवरु॒न्ध इ॑न्द्रि॒य मुत्त॑र॒योत्त॑र येन्द्रि॒य म॑वरु॒न्धे॑ ऽवरु॒न्ध इ॑न्द्रि॒य मुत्त॑रया । \newline
59. अ॒व॒रु॒न्ध इत्य॑व - रु॒न्धे । \newline
60. इ॒न्द्रि॒य मुत्त॑र॒योत्त॑रयेन् द्रि॒य मि॑न्द्रि॒य मुत्त॑रया देवलो॒कम् दे॑वलो॒क मुत्त॑र येन्द्रि॒य मि॑न्द्रि॒य मुत्त॑रया देवलो॒कम् । \newline
61. उत्त॑रया देवलो॒कम् दे॑वलो॒क मुत्त॑र॒योत्त॑रया देवलो॒क मे॒वैव दे॑वलो॒क मुत्त॑र॒योत्त॑रया देवलो॒क मे॒व । \newline
62. उत्त॑र॒येत्युत् - त॒र॒या॒ । \newline
63. दे॒व॒लो॒क मे॒वैव दे॑वलो॒कम् दे॑वलो॒क मे॒व पूर्व॑या॒ पूर्व॑यै॒व दे॑वलो॒कम् दे॑वलो॒क मे॒व पूर्व॑या । \newline
64. दे॒व॒लो॒कमिति॑ देव - लो॒कम् । \newline
65. ए॒व पूर्व॑या॒ पूर्व॑यै॒वैव पूर्व॑या ऽभि॒जय॑ त्यभि॒जय॑ति॒ पूर्व॑यै॒वैव पूर्व॑या ऽभि॒जय॑ति । \newline
\pagebreak
\markright{ TS 2.5.5.4  \hfill https://www.vedavms.in \hfill}
\addcontentsline{toc}{section}{ TS 2.5.5.4 }
\section*{ TS 2.5.5.4 }

\textbf{TS 2.5.5.4 } \newline
\textbf{Samhita Paata} \newline

पूर्व॑याऽभि॒जय॑ति मनुष्यलो॒कमुत्त॑रया॒ भूय॑सो यज्ञ्क्र॒तूनुपै᳚त्ये॒षा वै सु॒मना॒ नामेष्टि॒र्यम॒द्येजा॒नं प॒श्चाच्च॒न्द्रमा॑ अ॒भ्यु॑देत्य॒स्मिन्ने॒वास्मै॑ लो॒केऽर्द्धु॑कं भवति दाक्षायण य॒ज्ञेन॑ सुव॒र्गका॑मो यजेत पू॒र्णमा॑से॒ सं न॑येन्-मैत्रावरु॒ण्या ऽऽमिक्ष॑या ऽमावा॒स्या॑यां ॅयजेत पू॒र्णमा॑से॒ वै दे॒वानाꣳ॑ सु॒तस्तेषा॑मे॒तम॑र्द्धमा॒सं प्रसु॑त॒स्तेषां᳚ मैत्रावरु॒णी व॒शाऽमा॑वा॒स्या॑या-मनूब॒न्ध्या॑ यत् - [  ] \newline

\textbf{Pada Paata} \newline

पूर्व॑या । अ॒भि॒जय॒तीत्य॑भि - जय॑ति । म॒नु॒ष्य॒लो॒कमिति॑ मनुष्य - लो॒कम् । उत्त॑र॒येत्युत् - त॒र॒या॒ । भूय॑सः । य॒ज्ञ्॒क्र॒तूनिति॑ यज्ञ् - क्र॒तून् । उपेति॑ । ए॒ति॒ । ए॒षा । वै । सु॒मना॒ इति॑ सु - मनाः᳚ । नाम॑ । इष्टिः॑ । यम् । अ॒द्य । ई॒जा॒नम् । प॒श्चात् । च॒न्द्रमाः᳚ । अ॒भीति॑ । उ॒देतीत्यु॑त् - एति॑ । अ॒स्मिन्न् । ए॒व । अ॒स्मै॒ । लो॒के । अर्द्धु॑कम् । भ॒व॒ति॒ । दा॒क्षा॒य॒ण॒य॒ज्ञेनेति॑ दाक्षायण - य॒ज्ञेन॑ । सु॒व॒र्गका॑म॒ इति॑ सुव॒र्ग - का॒मः॒ । य॒जे॒त॒ । पू॒र्णमा॑स॒ इति॑ पू॒र्ण - मा॒से॒ । समिति॑ । न॒ये॒त् । मै॒त्रा॒व॒रु॒ण्येति॑ मैत्रा - व॒रु॒ण्या । आ॒मिक्ष॑या । अ॒मा॒वा॒स्या॑या॒मित्य॑मा - वा॒स्या॑याम् । य॒जे॒त॒ । पू॒र्णमा॑स॒ इति॑ पू॒र्ण - मा॒से॒ । वै । दे॒वाना᳚म् । सु॒तः । तेषा᳚म् । ए॒तम् । अ॒र्द्ध॒मा॒समित्य॑र्द्ध - मा॒सम् । प्रसु॑त॒ इति॒ प्र - सु॒तः॒ । तेषा᳚म् । मै॒त्रा॒वरु॒णीति॑ मैत्रा - व॒रु॒णी । व॒शा । अ॒मा॒वा॒स्या॑या॒मित्य॑मा - वा॒स्या॑याम् । अ॒नू॒ब॒न्ध्येत्य॑नु - ब॒न्ध्या᳚ । यत् ।  \newline


\textbf{Krama Paata} \newline

पूर्व॑या ऽभि॒जय॑ति । अ॒भि॒जय॑ति मनुष्यलो॒कम् । अ॒भि॒जय॒तीत्य॑भि - जय॑ति । म॒नु॒ष्य॒लो॒कमुत्त॑रया । म॒नु॒ष्य॒लो॒कमिति॑ मनुष्य - लो॒कम् । उत्त॑रया॒ भूय॑सः । उत्त॑र॒येत्युत् - त॒र॒या॒ । भूय॑सो यज्ञ्क्र॒तून् । य॒ज्ञ्॒क्र॒तूनुप॑ । य॒ज्ञ्॒क्र॒तूनिति॑ यज्ञ् - क्र॒तून् । उपै॑ति । ए॒त्ये॒षा । ए॒षा वै । वै सु॒मनाः᳚ । सु॒मना॒ नाम॑ । सु॒मना॒ इति॑ सु - मनाः᳚ । नामेष्टिः॑ । इष्टि॒र् यम् । यम॒द्य । अ॒द्येजा॒नम् । ई॒जा॒नम् प॒श्चात् । प॒श्चाच्च॒न्दमाः᳚ । च॒न्द्रमा॑ अ॒भि । अ॒भ्यु॑देति॑ । उ॒देत्य॒स्मिन्न् । उ॒देतीत्यु॑त् - एति॑ । अ॒स्मिन्ने॒व । ए॒वास्मै᳚ । अ॒स्मै॒ लो॒के । लो॒के ऽर्द्धु॑कम् । अर्द्धु॑कम् भवति । भ॒व॒ति॒ दा॒क्षा॒य॒ण॒य॒ज्ञेन॑ । दा॒क्षा॒य॒ण॒य॒ज्ञेन॑ सुव॒र्गका॑मः । दा॒क्षा॒य॒ण॒य॒ज्ञेनेति॑ दाक्षायण - य॒ज्ञेन॑ । सु॒व॒र्गका॑मो यजेत । सु॒व॒र्गका॑म॒ इति॑ सुव॒र्ग - का॒मः॒ । य॒जे॒त॒ पू॒र्णमा॑से । पू॒र्णमा॑से॒ सम् । पू॒र्णमा॑स॒ इति॑ पू॒र्ण - मा॒से॒ । सम् न॑येत् । न॒ये॒न् मै॒त्रा॒व॒रु॒ण्या । मै॒त्रा॒व॒रु॒ण्या ऽऽमिक्ष॑या । मै॒त्रा॒व॒रु॒ण्येति॑ मैत्रा - व॒रु॒ण्या । आ॒मिक्ष॑या ऽमावा॒स्या॑याम् । अ॒मा॒वा॒स्या॑याम् ॅयजेत । अ॒मा॒वा॒स्या॑या॒मित्य॑मा - वा॒स्या॑याम् । य॒जे॒त॒ पू॒र्णमा॑से । पू॒र्णमा॑से॒ वै । पू॒र्णमा॑स॒ इति॑ पू॒र्ण - मा॒से॒ । वै दे॒वाना᳚म् । दे॒वानाꣳ॑ सु॒तः । सु॒तस्तेषा᳚म् । तेषा॑मे॒तम् । ए॒तम॑र्द्धमा॒सम् । अ॒र्द्ध॒मा॒सम् प्रसु॑तः । अ॒र्द्ध॒मा॒समित्य॑र्द्ध - मा॒सम् । प्रसु॑त॒स्तेषा᳚म् । प्रसु॑त॒ इति॒ प्र - सु॒तः॒ । तेषा᳚म् मैत्रावरु॒णी । मै॒त्रा॒व॒रु॒णी व॒शा । मै॒त्रा॒व॒रु॒णीति॑ मैत्रा - व॒रु॒णी । व॒शाऽमा॑वा॒स्या॑याम् । अ॒मा॒वा॒स्या॑यामनूब॒न्द्ध्या᳚ । अ॒मा॒वा॒स्या॑या॒मित्य॑मा - वा॒स्या॑याम् । अ॒नू॒ब॒न्द्ध्या॑ यत् । अ॒नू॒ब॒न्द्ध्येत्य॑नु - ब॒न्द्ध्या᳚ । यत् पू᳚र्वे॒द्युः \newline

\textbf{Jatai Paata} \newline

1. पूर्व॑या ऽभि॒जय॑ त्यभि॒जय॑ति॒ पूर्व॑या॒ पूर्व॑या ऽभि॒जय॑ति । \newline
2. अ॒भि॒जय॑ति मनुष्यलो॒कम् म॑नुष्यलो॒क म॑भि॒जय॑ त्यभि॒जय॑ति मनुष्यलो॒कम् । \newline
3. अ॒भि॒जय॒तीत्य॑भि - जय॑ति । \newline
4. म॒नु॒ष्य॒लो॒क मुत्त॑र॒योत्त॑रया मनुष्यलो॒कम् म॑नुष्यलो॒क मुत्त॑रया । \newline
5. म॒नु॒ष्य॒लो॒कमिति॑ मनुष्य - लो॒कम् । \newline
6. उत्त॑रया॒ भूय॑सो॒ भूय॑स॒ उत्त॑र॒यो त्त॑रया॒ भूय॑सः । \newline
7. उत्त॑र॒येत्युत् - त॒र॒या॒ । \newline
8. भूय॑सो यज्ञ्क्र॒तून्. य॑ज्ञ्क्र॒तून् भूय॑सो॒ भूय॑सो यज्ञ्क्र॒तून् । \newline
9. य॒ज्ञ्॒क्र॒तू नुपोप॑ यज्ञ्क्र॒तून्. य॑ज्ञ्क्र॒तू नुप॑ । \newline
10. य॒ज्ञ्॒क्र॒तूनिति॑ यज्ञ् - क्र॒तून् । \newline
11. उपै᳚त्ये॒ त्युपोपै॑ति । \newline
12. ए॒त्ये॒षैषै त्ये᳚त्ये॒षा । \newline
13. ए॒षा वै वा ए॒षैषा वै । \newline
14. वै सु॒मनाः᳚ सु॒मना॒ वै वै सु॒मनाः᳚ । \newline
15. सु॒मना॒ नाम॒ नाम॑ सु॒मनाः᳚ सु॒मना॒ नाम॑ । \newline
16. सु॒मना॒ इति॑ सु - मनाः᳚ । \newline
17. नामे ष्टि॒ रिष्टि॒र् नाम॒ नामे ष्टिः॑ । \newline
18. इष्टि॒र् यं ॅय मिष्टि॒ रिष्टि॒र् यम् । \newline
19. य म॒द्याद्य यं ॅय म॒द्य । \newline
20. अ॒द्येजा॒न मी॑जा॒न म॒द्या द्येजा॒नम् । \newline
21. ई॒जा॒नम् प॒श्चात् प॒श्चा दी॑जा॒न मी॑जा॒नम् प॒श्चात् । \newline
22. प॒श्चाच् च॒न्द्रमा᳚ श्च॒न्द्रमाः᳚ प॒श्चात् प॒श्चाच् च॒न्द्रमाः᳚ । \newline
23. च॒न्द्रमा॑ अ॒भ्य॑भि च॒न्द्रमा᳚ श्च॒न्द्रमा॑ अ॒भि । \newline
24. अ॒भ्यु॑दे त्यु॒दे त्य॒भ्या᳚(1॒)भ्यु॑देति॑ । \newline
25. उ॒दे त्य॒स्मिन् न॒स्मिन् नु॒दे त्यु॒दे त्य॒स्मिन्न् । \newline
26. उ॒देतीत्यु॑त् - एति॑ । \newline
27. अ॒स्मिन् ने॒वैवास्मिन् न॒स्मिन् ने॒व । \newline
28. ए॒वास्मा॑ अस्मा ए॒वैवास्मै᳚ । \newline
29. अ॒स्मै॒ लो॒के लो॒के᳚ ऽस्मा अस्मै लो॒के । \newline
30. लो॒के ऽर्द्धु॑क॒ मर्द्धु॑कम् ॅलो॒के लो॒के ऽर्द्धु॑कम् । \newline
31. अर्द्धु॑कम् भवति भव॒ त्यर्द्धु॑क॒ मर्द्धु॑कम् भवति । \newline
32. भ॒व॒ति॒ दा॒क्षा॒य॒ण॒य॒ज्ञेन॑ दाक्षायणय॒ज्ञेन॑ भवति भवति दाक्षायणय॒ज्ञेन॑ । \newline
33. दा॒क्षा॒य॒ण॒य॒ज्ञेन॑ सुव॒र्गका॑मः सुव॒र्गका॑मो दाक्षायणय॒ज्ञेन॑ दाक्षायणय॒ज्ञेन॑ सुव॒र्गका॑मः । \newline
34. दा॒क्षा॒य॒ण॒य॒ज्ञेनेति॑ दाक्षायण - य॒ज्ञेन॑ । \newline
35. सु॒व॒र्गका॑मो यजेत यजेत सुव॒र्गका॑मः सुव॒र्गका॑मो यजेत । \newline
36. सु॒व॒र्गका॑म॒ इति॑ सुव॒र्ग - का॒मः॒ । \newline
37. य॒जे॒त॒ पू॒र्णमा॑से पू॒र्णमा॑से यजेत यजेत पू॒र्णमा॑से । \newline
38. पू॒र्णमा॑से॒ सꣳ सम् पू॒र्णमा॑से पू॒र्णमा॑से॒ सम् । \newline
39. पू॒र्णमा॑स॒ इति॑ पू॒र्ण - मा॒से॒ । \newline
40. सम् न॑येन् नये॒थ् सꣳ सम् न॑येत् । \newline
41. न॒ये॒न् मै॒त्रा॒व॒रु॒ण्या मै᳚त्रावरु॒ण्या न॑येन् नयेन् मैत्रावरु॒ण्या । \newline
42. मै॒त्रा॒व॒रु॒ण्या ऽऽमिक्ष॑या॒ ऽऽमिक्ष॑या मैत्रावरु॒ण्या मै᳚त्रावरु॒ण्या ऽऽमिक्ष॑या । \newline
43. मै॒त्रा॒व॒रु॒ण्येति॑ मैत्रा - व॒रु॒ण्या । \newline
44. आ॒मिक्ष॑या ऽमावा॒स्या॑या ममावा॒स्या॑या मा॒मिक्ष॑या॒ ऽऽमिक्ष॑या ऽमावा॒स्या॑याम् । \newline
45. अ॒मा॒वा॒स्या॑यां ॅयजेत यजेता मावा॒स्या॑या ममावा॒स्या॑यां ॅयजेत । \newline
46. अ॒मा॒वा॒स्या॑या॒मित्य॑मा - वा॒स्या॑याम् । \newline
47. य॒जे॒त॒ पू॒र्णमा॑से पू॒र्णमा॑से यजेत यजेत पू॒र्णमा॑से । \newline
48. पू॒र्णमा॑से॒ वै वै पू॒र्णमा॑से पू॒र्णमा॑से॒ वै । \newline
49. पू॒र्णमा॑स॒ इति॑ पू॒र्ण - मा॒से॒ । \newline
50. वै दे॒वाना᳚म् दे॒वानां॒ ॅवै वै दे॒वाना᳚म् । \newline
51. दे॒वानाꣳ॑ सु॒तः सु॒तो दे॒वाना᳚म् दे॒वानाꣳ॑ सु॒तः । \newline
52. सु॒त स्तेषा॒म् तेषाꣳ॑ सु॒तः सु॒त स्तेषा᳚म् । \newline
53. तेषा॑ मे॒त मे॒तम् तेषा॒म् तेषा॑ मे॒तम् । \newline
54. ए॒त म॑र्द्धमा॒स म॑र्द्धमा॒स मे॒त मे॒त म॑र्द्धमा॒सम् । \newline
55. अ॒र्द्ध॒मा॒सम् प्रसु॑तः॒ प्रसु॑तो ऽर्द्धमा॒स म॑र्द्धमा॒सम् प्रसु॑तः । \newline
56. अ॒र्द्ध॒मा॒समित्य॑र्द्ध - मा॒सम् । \newline
57. प्रसु॑त॒ स्तेषा॒म् तेषा॒म् प्रसु॑तः॒ प्रसु॑त॒ स्तेषा᳚म् । \newline
58. प्रसु॑त॒ इति॒ प्र - सु॒तः॒ । \newline
59. तेषा᳚म् मैत्रावरु॒णी मै᳚त्रावरु॒णी तेषा॒म् तेषा᳚म् मैत्रावरु॒णी । \newline
60. मै॒त्रा॒व॒रु॒णी व॒शा व॒शा मै᳚त्रावरु॒णी मै᳚त्रावरु॒णी व॒शा । \newline
61. मै॒त्रा॒व॒रु॒णीति॑ मैत्रा - व॒रु॒णी । \newline
62. व॒शा ऽमा॑वा॒स्या॑या ममावा॒स्या॑यां ॅव॒शा व॒शा ऽमा॑वा॒स्या॑याम् । \newline
63. अ॒मा॒वा॒स्या॑या मनूब॒न्ध्या॑ ऽनूब॒न्ध्या॑ ऽमावा॒स्या॑या ममावा॒स्या॑या मनूब॒न्ध्या᳚ । \newline
64. अ॒मा॒वा॒स्या॑या॒मित्य॑मा - वा॒स्या॑याम् । \newline
65. अ॒नू॒ब॒न्ध्या॑ यद् यद॑नूब॒न्ध्या॑ ऽनूब॒न्ध्या॑ यत् । \newline
66. अ॒नू॒ब॒न्ध्येत्य॑नु - ब॒न्ध्या᳚ । \newline
67. यत् पू᳚र्वे॒द्युः पू᳚र्वे॒द्युर् यद् यत् पू᳚र्वे॒द्युः । \newline

\textbf{Ghana Paata } \newline

1. पूर्व॑या ऽभि॒जय॑ त्यभि॒जय॑ति॒ पूर्व॑या॒ पूर्व॑या ऽभि॒जय॑ति मनुष्यलो॒कम् म॑नुष्यलो॒क म॑भि॒जय॑ति॒ पूर्व॑या॒ पूर्व॑या ऽभि॒जय॑ति मनुष्यलो॒कम् । \newline
2. अ॒भि॒जय॑ति मनुष्यलो॒कम् म॑नुष्यलो॒क म॑भि॒जय॑ त्यभि॒जय॑ति मनुष्यलो॒क मुत्त॑र॒योत्त॑रया मनुष्यलो॒क म॑भि॒जय॑ त्यभि॒जय॑ति मनुष्यलो॒क मुत्त॑रया । \newline
3. अ॒भि॒जय॒तीत्य॑भि - जय॑ति । \newline
4. म॒नु॒ष्य॒लो॒क मुत्त॑र॒योत्त॑रया मनुष्यलो॒कम् म॑नुष्यलो॒क मुत्त॑रया॒ भूय॑सो॒ भूय॑स॒ उत्त॑रया मनुष्यलो॒कम् म॑नुष्यलो॒क मुत्त॑रया॒ भूय॑सः । \newline
5. म॒नु॒ष्य॒लो॒कमिति॑ मनुष्य - लो॒कम् । \newline
6. उत्त॑रया॒ भूय॑सो॒ भूय॑स॒ उत्त॑र॒योत्त॑रया॒ भूय॑सो यज्ञ्क्र॒तून्. य॑ज्ञ्क्र॒तून् भूय॑स॒ उत्त॑र॒योत्त॑रया॒ भूय॑सो यज्ञ्क्र॒तून् । \newline
7. उत्त॑र॒येत्युत् - त॒र॒या॒ । \newline
8. भूय॑सो यज्ञ्क्र॒तून्. य॑ज्ञ्क्र॒तून् भूय॑सो॒ भूय॑सो यज्ञ्क्र॒तू नुपोप॑ यज्ञ्क्र॒तून् भूय॑सो॒ भूय॑सो यज्ञ्क्र॒तू नुप॑ । \newline
9. य॒ज्ञ्॒क्र॒तू नुपोप॑ यज्ञ्क्र॒तून्. य॑ज्ञ्क्र॒तू नुपै᳚त्ये॒त्युप॑ यज्ञ्क्र॒तून्. य॑ज्ञ्क्र॒तू नुपै॑ति । \newline
10. य॒ज्ञ्॒क्र॒तूनिति॑ यज्ञ् - क्र॒तून् । \newline
11. उपै᳚त्ये॒ त्युपोपै᳚ त्ये॒षैषै त्युपोपै᳚ त्ये॒षा । \newline
12. ए॒त्ये॒षैषै त्ये᳚त्ये॒षा वै वा ए॒षै त्ये᳚ त्ये॒षा वै । \newline
13. ए॒षा वै वा ए॒षैषा वै सु॒मनाः᳚ सु॒मना॒ वा ए॒षैषा वै सु॒मनाः᳚ । \newline
14. वै सु॒मनाः᳚ सु॒मना॒ वै वै सु॒मना॒ नाम॒ नाम॑ सु॒मना॒ वै वै सु॒मना॒ नाम॑ । \newline
15. सु॒मना॒ नाम॒ नाम॑ सु॒मनाः᳚ सु॒मना॒ नामे ष्टि॒रिष्टि॒र् नाम॑ सु॒मनाः᳚ सु॒मना॒ नामे ष्टिः॑ । \newline
16. सु॒मना॒ इति॑ सु - मनाः᳚ । \newline
17. नामे ष्टि॒रिष्टि॒र् नाम॒ नामे ष्टि॒र् यं ॅय मिष्टि॒र् नाम॒ नामे ष्टि॒र् यम् । \newline
18. इष्टि॒र् यं ॅय मिष्टि॒ रिष्टि॒र् य म॒द्याद्य य मिष्टि॒ रिष्टि॒र् य म॒द्य । \newline
19. य म॒द्याद्य यं ॅय म॒द्येजा॒न मी॑जा॒न म॒द्य यं ॅय म॒द्येजा॒नम् । \newline
20. अ॒द्येजा॒न मी॑जा॒न म॒द्या द्येजा॒नम् प॒श्चात् प॒श्चा दी॑जा॒न म॒द्या द्येजा॒नम् प॒श्चात् । \newline
21. ई॒जा॒नम् प॒श्चात् प॒श्चादी॑जा॒न मी॑जा॒नम् प॒श्चाच् च॒न्द्रमा᳚ श्च॒न्द्रमाः᳚ प॒श्चा दी॑जा॒न मी॑जा॒नम् प॒श्चाच् च॒न्द्रमाः᳚ । \newline
22. प॒श्चाच् च॒न्द्रमा᳚ श्च॒न्द्रमाः᳚ प॒श्चात् प॒श्चाच् च॒न्द्रमा॑ अ॒भ्य॑भि च॒न्द्रमाः᳚ प॒श्चात् प॒श्चाच् च॒न्द्रमा॑ अ॒भि । \newline
23. च॒न्द्रमा॑ अ॒भ्य॑भि च॒न्द्रमा᳚ श्च॒न्द्रमा॑ अ॒भ्यु॑दे त्यु॒दे त्य॒भि च॒न्द्रमा᳚ श्च॒न्द्रमा॑ अ॒भ्यु॑देति॑ । \newline
24. अ॒भ्यु॑दे त्यु॒दे त्य॒भ्या᳚(1॒)भ्यु॑दे त्य॒स्मिन् न॒स्मिन् नु॒दे त्य॒भ्या᳚(1॒)भ्यु॑दे त्य॒स्मिन्न् । \newline
25. उ॒दे त्य॒स्मिन् न॒स्मिन् नु॒दे त्यु॒दे त्य॒स्मिन् ने॒वैवास्मिन् नु॒दे त्यु॒दे त्य॒स्मिन् ने॒व । \newline
26. उ॒देतीत्यु॑त् - एति॑ । \newline
27. अ॒स्मिन् ने॒वैवास्मिन् न॒स्मिन् ने॒वास्मा॑ अस्मा ए॒वास्मिन् न॒स्मिन् ने॒वास्मै᳚ । \newline
28. ए॒वास्मा॑ अस्मा ए॒वैवास्मै॑ लो॒के लो॒के᳚ ऽस्मा ए॒वैवास्मै॑ लो॒के । \newline
29. अ॒स्मै॒ लो॒के लो॒के᳚ ऽस्मा अस्मै लो॒के ऽर्द्धु॑क॒ मर्द्धु॑कम् ॅलो॒के᳚ ऽस्मा अस्मै लो॒के ऽर्द्धु॑कम् । \newline
30. लो॒के ऽर्द्धु॑क॒ मर्द्धु॑कम् ॅलो॒के लो॒के ऽर्द्धु॑कम् भवति भव॒ त्यर्द्धु॑कम् ॅलो॒के लो॒के ऽर्द्धु॑कम् भवति । \newline
31. अर्द्धु॑कम् भवति भव॒ त्यर्द्धु॑क॒ मर्द्धु॑कम् भवति दाक्षायणय॒ज्ञेन॑ दाक्षायणय॒ज्ञेन॑ भव॒ त्यर्द्धु॑क॒ मर्द्धु॑कम् भवति दाक्षायणय॒ज्ञेन॑ । \newline
32. भ॒व॒ति॒ दा॒क्षा॒य॒ण॒य॒ज्ञेन॑ दाक्षायणय॒ज्ञेन॑ भवति भवति दाक्षायणय॒ज्ञेन॑ सुव॒र्गका॑मः सुव॒र्गका॑मो दाक्षायणय॒ज्ञेन॑ भवति भवति दाक्षायणय॒ज्ञेन॑ सुव॒र्गका॑मः । \newline
33. दा॒क्षा॒य॒ण॒य॒ज्ञेन॑ सुव॒र्गका॑मः सुव॒र्गका॑मो दाक्षायणय॒ज्ञेन॑ दाक्षायणय॒ज्ञेन॑ सुव॒र्गका॑मो यजेत यजेत सुव॒र्गका॑मो दाक्षायणय॒ज्ञेन॑ दाक्षायणय॒ज्ञेन॑ सुव॒र्गका॑मो यजेत । \newline
34. दा॒क्षा॒य॒ण॒य॒ज्ञेनेति॑ दाक्षायण - य॒ज्ञेन॑ । \newline
35. सु॒व॒र्गका॑मो यजेत यजेत सुव॒र्गका॑मः सुव॒र्गका॑मो यजेत पू॒र्णमा॑से पू॒र्णमा॑से यजेत सुव॒र्गका॑मः सुव॒र्गका॑मो यजेत पू॒र्णमा॑से । \newline
36. सु॒व॒र्गका॑म॒ इति॑ सुव॒र्ग - का॒मः॒ । \newline
37. य॒जे॒त॒ पू॒र्णमा॑से पू॒र्णमा॑से यजेत यजेत पू॒र्णमा॑से॒ सꣳ सम् पू॒र्णमा॑से यजेत यजेत पू॒र्णमा॑से॒ सम् । \newline
38. पू॒र्णमा॑से॒ सꣳ सम् पू॒र्णमा॑से पू॒र्णमा॑से॒ सम् न॑येन् नये॒थ् सम् पू॒र्णमा॑से पू॒र्णमा॑से॒ सम् न॑येत् । \newline
39. पू॒र्णमा॑स॒ इति॑ पू॒र्ण - मा॒से॒ । \newline
40. सम् न॑येन् नये॒थ् सꣳ सम् न॑येन् मैत्रावरु॒ण्या मै᳚त्रावरु॒ण्या न॑ये॒थ् सꣳ सम् न॑येन् मैत्रावरु॒ण्या । \newline
41. न॒ये॒न् मै॒त्रा॒व॒रु॒ण्या मै᳚त्रावरु॒ण्या न॑येन् नयेन् मैत्रावरु॒ण्या ऽऽमिक्ष॑या॒ ऽऽमिक्ष॑या मैत्रावरु॒ण्या न॑येन् नयेन् मैत्रावरु॒ण्या ऽऽमिक्ष॑या । \newline
42. मै॒त्रा॒व॒रु॒ण्या ऽऽमिक्ष॑या॒ ऽऽमिक्ष॑या मैत्रावरु॒ण्या मै᳚त्रावरु॒ण्या ऽऽमिक्ष॑या ऽमावा॒स्या॑या ममावा॒स्या॑या मा॒मिक्ष॑या मैत्रावरु॒ण्या मै᳚त्रावरु॒ण्या ऽऽमिक्ष॑या ऽमावा॒स्या॑याम् । \newline
43. मै॒त्रा॒व॒रु॒ण्येति॑ मैत्रा - व॒रु॒ण्या । \newline
44. आ॒मिक्ष॑या ऽमावा॒स्या॑या ममावा॒स्या॑या मा॒मिक्ष॑या॒ ऽऽमिक्ष॑या ऽमावा॒स्या॑यां ॅयजेत यजेतामावा॒स्या॑या मा॒मिक्ष॑या॒ ऽऽमिक्ष॑या ऽमावा॒स्या॑यां ॅयजेत । \newline
45. अ॒मा॒वा॒स्या॑यां ॅयजेत यजेता मावा॒स्या॑या ममावा॒स्या॑यां ॅयजेत पू॒र्णमा॑से पू॒र्णमा॑से यजेता मावा॒स्या॑या ममावा॒स्या॑यां ॅयजेत पू॒र्णमा॑से । \newline
46. अ॒मा॒वा॒स्या॑या॒मित्य॑मा - वा॒स्या॑याम् । \newline
47. य॒जे॒त॒ पू॒र्णमा॑से पू॒र्णमा॑से यजेत यजेत पू॒र्णमा॑से॒ वै वै पू॒र्णमा॑से यजेत यजेत पू॒र्णमा॑से॒ वै । \newline
48. पू॒र्णमा॑से॒ वै वै पू॒र्णमा॑से पू॒र्णमा॑से॒ वै दे॒वाना᳚म् दे॒वानां॒ ॅवै पू॒र्णमा॑से पू॒र्णमा॑से॒ वै दे॒वाना᳚म् । \newline
49. पू॒र्णमा॑स॒ इति॑ पू॒र्ण - मा॒से॒ । \newline
50. वै दे॒वाना᳚म् दे॒वानां॒ ॅवै वै दे॒वानाꣳ॑ सु॒तः सु॒तो दे॒वानां॒ ॅवै वै दे॒वानाꣳ॑ सु॒तः । \newline
51. दे॒वानाꣳ॑ सु॒तः सु॒तो दे॒वाना᳚म् दे॒वानाꣳ॑ सु॒त स्तेषा॒म् तेषाꣳ॑ सु॒तो दे॒वाना᳚म् दे॒वानाꣳ॑ सु॒त स्तेषा᳚म् । \newline
52. सु॒त स्तेषा॒म् तेषाꣳ॑ सु॒तः सु॒त स्तेषा॑ मे॒त मे॒तम् तेषाꣳ॑ सु॒तः सु॒त स्तेषा॑ मे॒तम् । \newline
53. तेषा॑ मे॒त मे॒तम् तेषा॒म् तेषा॑ मे॒त म॑र्द्धमा॒स म॑र्द्धमा॒स मे॒तम् तेषा॒म् तेषा॑ मे॒त म॑र्द्धमा॒सम् । \newline
54. ए॒त म॑र्द्धमा॒स म॑र्द्धमा॒स मे॒त मे॒त म॑र्द्धमा॒सम् प्रसु॑तः॒ प्रसु॑तो ऽर्द्धमा॒स मे॒त मे॒त म॑र्द्धमा॒सम् प्रसु॑तः । \newline
55. अ॒र्द्ध॒मा॒सम् प्रसु॑तः॒ प्रसु॑तो ऽर्द्धमा॒स म॑र्द्धमा॒सम् प्रसु॑त॒ स्तेषा॒म् तेषा॒म् प्रसु॑तो ऽर्द्धमा॒स म॑र्द्धमा॒सम् प्रसु॑त॒ स्तेषा᳚म् । \newline
56. अ॒र्द्ध॒मा॒समित्य॑र्द्ध - मा॒सम् । \newline
57. प्रसु॑त॒ स्तेषा॒म् तेषा॒म् प्रसु॑तः॒ प्रसु॑त॒ स्तेषा᳚म् मैत्रावरु॒णी मै᳚त्रावरु॒णी तेषा॒म् प्रसु॑तः॒ प्रसु॑त॒ स्तेषा᳚म् मैत्रावरु॒णी । \newline
58. प्रसु॑त॒ इति॒ प्र - सु॒तः॒ । \newline
59. तेषा᳚म् मैत्रावरु॒णी मै᳚त्रावरु॒णी तेषा॒म् तेषा᳚म् मैत्रावरु॒णी व॒शा व॒शा मै᳚त्रावरु॒णी तेषा॒म् तेषा᳚म् मैत्रावरु॒णी व॒शा । \newline
60. मै॒त्रा॒व॒रु॒णी व॒शा व॒शा मै᳚त्रावरु॒णी मै᳚त्रावरु॒णी व॒शा ऽमा॑वा॒स्या॑या ममावा॒स्या॑यां ॅव॒शा मै᳚त्रावरु॒णी मै᳚त्रावरु॒णी व॒शा ऽमा॑वा॒स्या॑याम् । \newline
61. मै॒त्रा॒व॒रु॒णीति॑ मैत्रा - व॒रु॒णी । \newline
62. व॒शा ऽमा॑वा॒स्या॑या ममावा॒स्या॑यां ॅव॒शा व॒शा ऽमा॑वा॒स्या॑या मनूब॒न्ध्या॑ ऽनूब॒न्ध्या॑ ऽमावा॒स्या॑यां ॅव॒शा व॒शा ऽमा॑वा॒स्या॑या मनूब॒न्ध्या᳚ । \newline
63. अ॒मा॒वा॒स्या॑या मनूब॒न्ध्या॑ ऽनूब॒न्ध्या॑ ऽमावा॒स्या॑या ममावा॒स्या॑या मनूब॒न्ध्या॑ यद् यद॑नूब॒न्ध्या॑ ऽमावा॒स्या॑या ममावा॒स्या॑या मनूब॒न्ध्या॑ यत् । \newline
64. अ॒मा॒वा॒स्या॑या॒मित्य॑मा - वा॒स्या॑याम् । \newline
65. अ॒नू॒ब॒न्ध्या॑ यद् यद॑नूब॒न्ध्या॑ ऽनूब॒न्ध्या॑ यत् पू᳚र्वे॒द्युः पू᳚र्वे॒द्युर् यद॑नूब॒न्ध्या॑ ऽनूब॒न्ध्या॑ यत् पू᳚र्वे॒द्युः । \newline
66. अ॒नू॒ब॒न्ध्येत्य॑नु - ब॒न्ध्या᳚ । \newline
67. यत् पू᳚र्वे॒द्युः पू᳚र्वे॒द्युर् यद् यत् पू᳚र्वे॒द्युर् यज॑ते॒ यज॑ते पूर्वे॒द्युर् यद् यत् पू᳚र्वे॒द्युर् यज॑ते । \newline
\pagebreak
\markright{ TS 2.5.5.5  \hfill https://www.vedavms.in \hfill}
\addcontentsline{toc}{section}{ TS 2.5.5.5 }
\section*{ TS 2.5.5.5 }

\textbf{TS 2.5.5.5 } \newline
\textbf{Samhita Paata} \newline

पू᳚र्वे॒द्यु र्यज॑ते॒ वेदि॑मे॒व तत् क॑रोति॒ यद् व॒थ्सान॑पाक॒रोति॑ सदोहविर्द्धा॒ने ए॒व सं मि॑नोति॒ यद्यज॑ते दे॒वैरे॒व सु॒त्याꣳ सं पा॑दयति॒ स ए॒तम॑र्द्धमा॒सꣳ स॑ध॒मादं॑ दे॒वैः सोमं॑ पिबति॒ यन्-मै᳚त्रावरु॒ण्या ऽऽमिक्ष॑या ऽमावा॒स्या॑यां॒ ॅयज॑ते॒ यैवासौ दे॒वानां᳚ ॅव॒शाऽनू॑ब॒न्ध्या॑ सो ए॒वैषैतस्य॑ सा॒क्षाद्वा ए॒ष दे॒वान॒भ्यारो॑हति॒ य ए॑षां ॅय॒ज्ञ् - [  ] \newline

\textbf{Pada Paata} \newline

पू॒र्वे॒द्युः । यज॑ते । वेदि᳚म् । ए॒व । तत् । क॒रो॒ति॒ । यत् । व॒थ्सान् । अ॒पा॒क॒रोतीत्य॑प-आ॒क॒रोति॑ । स॒दो॒ह॒वि॒र्द्धा॒ने इति॑ सदः-ह॒वि॒र्द्धा॒ने । ए॒व । समिति॑ । मि॒नो॒ति॒ । यत् । यज॑ते । दे॒वैः । ए॒व । सु॒त्याम् । समिति॑ । पा॒द॒य॒ति॒ । सः । ए॒तम् । अ॒र्द्ध॒मा॒समित्य॑र्द्ध - मा॒सम् । स॒ध॒माद॒मिति॑ सध - माद᳚म् । दे॒वैः । सोम᳚म् । पि॒ब॒ति॒ । यत् । मै॒त्रा॒व॒रु॒ण्येति॑ मैत्रा - व॒रु॒ण्या । आ॒मिक्ष॑या । अ॒मा॒वा॒स्या॑या॒मित्य॑मा - वा॒स्या॑याम् । यज॑ते । या । ए॒व । अ॒सौ । दे॒वाना᳚म् । व॒शा । अ॒नू॒ब॒न्ध्येत्य॑नु - ब॒न्ध्या᳚ । सो इति॑ । ए॒व । ए॒षा । ए॒तस्य॑ । सा॒क्षादिति॑ स - अ॒क्षात् । वै । ए॒षः । दे॒वान् । अ॒भ्यारो॑ह॒तीत्य॑भि - आरो॑हति । यः । ए॒षा॒म् । य॒ज्ञ्म् ।  \newline


\textbf{Krama Paata} \newline

पू॒र्वे॒द्युर् यज॑ते । यज॑ते॒ वेदि᳚म् । वेदि॑मे॒व । ए॒व तत् । तत् क॑रोति । क॒रो॒ति॒ यत् । यद् व॒थ्सान् । व॒थ्सान॑पाक॒रोति॑ । अ॒पा॒क॒रोति॑ सदोहविर्द्धा॒ने । अ॒पा॒क॒रोरीत्य॑प - आ॒क॒रोति॑ । स॒दो॒ह॒वि॒र्द्धा॒ने ए॒व । स॒दो॒ह॒वि॒र्द्धा॒ने इति॑ सदः - ह॒वि॒र्द्धा॒ने । ए॒व सम् । सम् मि॑नोति । मि॒नो॒ति॒ यत् । यद् यज॑ते । यज॑ते दे॒वैः । दे॒वैरे॒व । ए॒व सु॒त्याम् । सु॒त्याꣳ सम् । सम् पा॑दयति । पा॒द॒य॒ति॒ सः । स ए॒तम् । ए॒तम॑र्द्धमा॒सम् । अ॒र्द्ध॒मा॒सꣳ स॑ध॒माद᳚म् । अ॒र्द्ध॒मा॒समित्य॑र्द्ध - मा॒सम् । स॒ध॒माद॑म् दे॒वैः । स॒ध॒माद॒मिति॑ सध - माद᳚म् । दे॒वैः सोम᳚म् । सोम॑म् पिबति । पि॒ब॒ति॒ यत् । यन् मै᳚त्रावरु॒ण्या । मै॒त्रा॒व॒रु॒ण्या ऽऽमिक्ष॑या । मै॒त्रा॒व॒रु॒ण्येति॑ मैत्रा - व॒रु॒ण्या । आ॒मिक्ष॑या ऽमावा॒स्या॑याम् । अ॒मा॒वा॒स्या॑या॒म् ॅयज॑ते । अ॒मा॒वा॒स्या॑या॒मित्य॑मा - वा॒स्या॑याम् । यज॑ते॒ या । यैव । ए॒वासौ । अ॒सौ दे॒वाना᳚म् । दे॒वाना᳚म् ॅव॒शा । व॒शाऽनू॑ब॒न्द्ध्या᳚ । अ॒नू॒ब॒न्द्ध्या॑ सो । अ॒नू॒ब॒न्द्धेत्य॑नु - ब॒न्द्ध्या᳚ । सो ए॒व । सो इति॒ सो । ए॒वैषा । ए॒षैतस्य॑ । ए॒तस्य॑ सा॒क्षात् । सा॒क्षाद् वै । सा॒क्षादिति॑ स - अ॒क्षात् । वा ए॒षः । ए॒ष दे॒वान् । दे॒वान॒भ्यारो॑हति । अ॒भ्यारो॑हति॒ यः । अ॒भ्यारो॑ह॒तीत्य॑भि - आरो॑हति । य ए॑षाम् । ए॒षा॒म् ॅय॒ज्ञ्म् । य॒ज्ञ्म॑भ्या॒रोह॑ति \newline

\textbf{Jatai Paata} \newline

1. पू॒र्वे॒द्युर् यज॑ते॒ यज॑ते पूर्वे॒द्युः पू᳚र्वे॒द्युर् यज॑ते । \newline
2. यज॑ते॒ वेदिं॒ ॅवेदिं॒ ॅयज॑ते॒ यज॑ते॒ वेदि᳚म् । \newline
3. वेदि॑ मे॒वैव वेदिं॒ ॅवेदि॑ मे॒व । \newline
4. ए॒व तत् तदे॒वैव तत् । \newline
5. तत् क॑रोति करोति॒ तत् तत् क॑रोति । \newline
6. क॒रो॒ति॒ यद् यत् क॑रोति करोति॒ यत् । \newline
7. यद् व॒थ्सान्. व॒थ्सान्. यद् यद् व॒थ्सान् । \newline
8. व॒थ्सा न॑पाक॒रो त्य॑पाक॒रोति॑ व॒थ्सान्. व॒थ्सा न॑पाक॒रोति॑ । \newline
9. अ॒पा॒क॒रोति॑ सदोहविर्द्धा॒ने स॑दोहविर्द्धा॒ने अ॑पाक॒रो त्य॑पाक॒रोति॑ सदोहविर्द्धा॒ने । \newline
10. अ॒पा॒क॒रोतीत्य॑प - आ॒क॒रोति॑ । \newline
11. स॒दो॒ह॒वि॒र्द्धा॒ने ए॒वैव स॑दोहविर्द्धा॒ने स॑दोहविर्द्धा॒ने ए॒व । \newline
12. स॒दो॒ह॒वि॒र्द्धा॒ने इति॑ सदः - ह॒वि॒र्द्धा॒ने । \newline
13. ए॒व सꣳ स मे॒वैव सम् । \newline
14. सम् मि॑नोति मिनोति॒ सꣳ सम् मि॑नोति । \newline
15. मि॒नो॒ति॒ यद् यन् मि॑नोति मिनोति॒ यत् । \newline
16. यद् यज॑ते॒ यज॑ते॒ यद् यद् यज॑ते । \newline
17. यज॑ते दे॒वैर् दे॒वैर् यज॑ते॒ यज॑ते दे॒वैः । \newline
18. दे॒वै रे॒वैव दे॒वैर् दे॒वै रे॒व । \newline
19. ए॒व सु॒त्याꣳ सु॒त्या मे॒वैव सु॒त्याम् । \newline
20. सु॒त्याꣳ सꣳ सꣳ सु॒त्याꣳ सु॒त्याꣳ सम् । \newline
21. सम् पा॑दयति पादयति॒ सꣳ सम् पा॑दयति । \newline
22. पा॒द॒य॒ति॒ स स पा॑दयति पादयति॒ सः । \newline
23. स ए॒त मे॒तꣳ स स ए॒तम् । \newline
24. ए॒त म॑र्द्धमा॒स म॑र्द्धमा॒स मे॒त मे॒त म॑र्द्धमा॒सम् । \newline
25. अ॒र्द्ध॒मा॒सꣳ स॑ध॒मादꣳ॑ सध॒माद॑ मर्द्धमा॒स म॑र्द्धमा॒सꣳ स॑ध॒माद᳚म् । \newline
26. अ॒र्द्ध॒मा॒समित्य॑र्द्ध - मा॒सम् । \newline
27. स॒ध॒माद॑म् दे॒वैर् दे॒वैः स॑ध॒मादꣳ॑ सध॒माद॑म् दे॒वैः । \newline
28. स॒ध॒माद॒मिति॑ सध - माद᳚म् । \newline
29. दे॒वैः सोमꣳ॒॒ सोम॑म् दे॒वैर् दे॒वैः सोम᳚म् । \newline
30. सोम॑म् पिबति पिबति॒ सोमꣳ॒॒ सोम॑म् पिबति । \newline
31. पि॒ब॒ति॒ यद् यत् पि॑बति पिबति॒ यत् । \newline
32. यन् मै᳚त्रावरु॒ण्या मै᳚त्रावरु॒ण्या यद् यन् मै᳚त्रावरु॒ण्या । \newline
33. मै॒त्रा॒व॒रु॒ण्या ऽऽमिक्ष॑या॒ ऽऽमिक्ष॑या मैत्रावरु॒ण्या मै᳚त्रावरु॒ण्या ऽऽमिक्ष॑या । \newline
34. मै॒त्रा॒व॒रु॒ण्येति॑ मैत्रा - व॒रु॒ण्या । \newline
35. आ॒मिक्ष॑या ऽमावा॒स्या॑या ममावा॒स्या॑या मा॒मिक्ष॑या॒ ऽऽमिक्ष॑या ऽमावा॒स्या॑याम् । \newline
36. अ॒मा॒वा॒स्या॑यां॒ ॅयज॑ते॒ यज॑ते ऽमावा॒स्या॑या ममावा॒स्या॑यां॒ ॅयज॑ते । \newline
37. अ॒मा॒वा॒स्या॑या॒मित्य॑मा - वा॒स्या॑याम् । \newline
38. यज॑ते॒ या या यज॑ते॒ यज॑ते॒ या । \newline
39. यैवैव या यैव । \newline
40. ए॒वासा व॒सा वे॒वैवासौ । \newline
41. अ॒सौ दे॒वाना᳚म् दे॒वाना॑ म॒सा व॒सौ दे॒वाना᳚म् । \newline
42. दे॒वानां᳚ ॅव॒शा व॒शा दे॒वाना᳚म् दे॒वानां᳚ ॅव॒शा । \newline
43. व॒शा ऽनू॑ब॒न्ध्या॑ ऽनूब॒न्ध्या॑ व॒शा व॒शा ऽनू॑ब॒न्ध्या᳚ । \newline
44. अ॒नू॒ब॒न्ध्या॑ सो सो अ॑नूब॒न्ध्या॑ ऽनूब॒न्ध्या॑ सो । \newline
45. अ॒नू॒ब॒न्ध्येत्य॑नु - ब॒न्ध्या᳚ । \newline
46. सो ए॒वैव सो सो ए॒व । \newline
47. सो इति॒ सो । \newline
48. ए॒वै षैषै वैवैषा । \newline
49. ए॒षैतस्यै॒ तस्यै॒ षैषैतस्य॑ । \newline
50. ए॒तस्य॑ सा॒क्षाथ् सा॒क्षा दे॒तस्यै॒तस्य॑ सा॒क्षात् । \newline
51. सा॒क्षाद् वै वै सा॒क्षाथ् सा॒क्षाद् वै । \newline
52. सा॒क्षादिति॑ स - अ॒क्षात् । \newline
53. वा ए॒ष ए॒ष वै वा ए॒षः । \newline
54. ए॒ष दे॒वान् दे॒वा ने॒ष ए॒ष दे॒वान् । \newline
55. दे॒वा न॒भ्यारो॑ह त्य॒भ्यारो॑हति दे॒वान् दे॒वा न॒भ्यारो॑हति । \newline
56. अ॒भ्यारो॑हति॒ यो यो᳚ ऽभ्यारो॑ह त्य॒भ्यारो॑हति॒ यः । \newline
57. अ॒भ्यारो॑ह॒तीत्य॑भि - आरो॑हति । \newline
58. य ए॑षा मेषां॒ ॅयो य ए॑षाम् । \newline
59. ए॒षां॒ ॅय॒ज्ञ्ं ॅय॒ज्ञ् मे॑षा मेषां ॅय॒ज्ञ्म् । \newline
60. य॒ज्ञ् म॑भ्या॒रोह॑ त्यभ्या॒रोह॑ति य॒ज्ञ्ं ॅय॒ज्ञ् म॑भ्या॒रोह॑ति । \newline

\textbf{Ghana Paata } \newline

1. पू॒र्वे॒द्युर् यज॑ते॒ यज॑ते पूर्वे॒द्युः पू᳚र्वे॒द्युर् यज॑ते॒ वेदिं॒ ॅवेदिं॒ ॅयज॑ते पूर्वे॒द्युः पू᳚र्वे॒द्युर् यज॑ते॒ वेदि᳚म् । \newline
2. यज॑ते॒ वेदिं॒ ॅवेदिं॒ ॅयज॑ते॒ यज॑ते॒ वेदि॑ मे॒वैव वेदिं॒ ॅयज॑ते॒ यज॑ते॒ वेदि॑ मे॒व । \newline
3. वेदि॑ मे॒वैव वेदिं॒ ॅवेदि॑ मे॒व तत् तदे॒व वेदिं॒ ॅवेदि॑ मे॒व तत् । \newline
4. ए॒व तत् तदे॒वैव तत् क॑रोति करोति॒ तदे॒वैव तत् क॑रोति । \newline
5. तत् क॑रोति करोति॒ तत् तत् क॑रोति॒ यद् यत् क॑रोति॒ तत् तत् क॑रोति॒ यत् । \newline
6. क॒रो॒ति॒ यद् यत् क॑रोति करोति॒ यद् व॒थ्सान्. व॒थ्सान्. यत् क॑रोति करोति॒ यद् व॒थ्सान् । \newline
7. यद् व॒थ्सान्. व॒थ्सान्. यद् यद् व॒थ्सा न॑पाक॒रो त्य॑पाक॒रोति॑ व॒थ्सान्. यद् यद् व॒थ्सा न॑पाक॒रोति॑ । \newline
8. व॒थ्सा न॑पाक॒रो त्य॑पाक॒रोति॑ व॒थ्सान्. व॒थ्सा न॑पाक॒रोति॑ सदोहविर्द्धा॒ने स॑दोहविर्द्धा॒ने अ॑पाक॒रोति॑ व॒थ्सान्. व॒थ्सा न॑पाक॒रोति॑ सदोहविर्द्धा॒ने । \newline
9. अ॒पा॒क॒रोति॑ सदोहविर्द्धा॒ने स॑दोहविर्द्धा॒ने अ॑पाक॒रो त्य॑पाक॒रोति॑ सदोहविर्द्धा॒ने ए॒वैव स॑दोहविर्द्धा॒ने अ॑पाक॒रो त्य॑पाक॒रोति॑ सदोहविर्द्धा॒ने ए॒व । \newline
10. अ॒पा॒क॒रोतीत्य॑प - आ॒क॒रोति॑ । \newline
11. स॒दो॒ह॒वि॒र्द्धा॒ने ए॒वैव स॑दोहविर्द्धा॒ने स॑दोहविर्द्धा॒ने ए॒व सꣳ स मे॒व स॑दोहविर्द्धा॒ने स॑दोहविर्द्धा॒ने ए॒व सम् । \newline
12. स॒दो॒ह॒वि॒र्द्धा॒ने इति॑ सदः - ह॒वि॒र्द्धा॒ने । \newline
13. ए॒व सꣳ स मे॒वैव सम् मि॑नोति मिनोति॒ स मे॒वैव सम् मि॑नोति । \newline
14. सम् मि॑नोति मिनोति॒ सꣳ सम् मि॑नोति॒ यद् यन् मि॑नोति॒ सꣳ सम् मि॑नोति॒ यत् । \newline
15. मि॒नो॒ति॒ यद् यन् मि॑नोति मिनोति॒ यद् यज॑ते॒ यज॑ते॒ यन् मि॑नोति मिनोति॒ यद् यज॑ते । \newline
16. यद् यज॑ते॒ यज॑ते॒ यद् यद् यज॑ते दे॒वैर् दे॒वैर् यज॑ते॒ यद् यद् यज॑ते दे॒वैः । \newline
17. यज॑ते दे॒वैर् दे॒वैर् यज॑ते॒ यज॑ते दे॒वै रे॒वैव दे॒वैर् यज॑ते॒ यज॑ते दे॒वैरे॒व । \newline
18. दे॒वै रे॒वैव दे॒वैर् दे॒वै रे॒व सु॒त्याꣳ सु॒त्या मे॒व दे॒वैर् दे॒वै रे॒व सु॒त्याम् । \newline
19. ए॒व सु॒त्याꣳ सु॒त्या मे॒वैव सु॒त्याꣳ सꣳ सꣳ सु॒त्या मे॒वैव सु॒त्याꣳ सम् । \newline
20. सु॒त्याꣳ सꣳ सꣳ सु॒त्याꣳ सु॒त्याꣳ सम् पा॑दयति पादयति॒ सꣳ सु॒त्याꣳ सु॒त्याꣳ सम् पा॑दयति । \newline
21. सम् पा॑दयति पादयति॒ सꣳ सम् पा॑दयति॒ स स पा॑दयति॒ सꣳ सम् पा॑दयति॒ सः । \newline
22. पा॒द॒य॒ति॒ स स पा॑दयति पादयति॒ स ए॒त मे॒तꣳ स पा॑दयति पादयति॒ स ए॒तम् । \newline
23. स ए॒त मे॒तꣳ स स ए॒त म॑र्द्धमा॒स म॑र्द्धमा॒स मे॒तꣳ स स ए॒त म॑र्द्धमा॒सम् । \newline
24. ए॒त म॑र्द्धमा॒स म॑र्द्धमा॒स मे॒त मे॒त म॑र्द्धमा॒सꣳ स॑ध॒मादꣳ॑ सध॒माद॑ मर्द्धमा॒स मे॒त मे॒त म॑र्द्धमा॒सꣳ स॑ध॒माद᳚म् । \newline
25. अ॒र्द्ध॒मा॒सꣳ स॑ध॒मादꣳ॑ सध॒माद॑ मर्द्धमा॒स म॑र्द्धमा॒सꣳ स॑ध॒माद॑म् दे॒वैर् दे॒वैः स॑ध॒माद॑ मर्द्धमा॒स म॑र्द्धमा॒सꣳ स॑ध॒माद॑म् दे॒वैः । \newline
26. अ॒र्द्ध॒मा॒समित्य॑र्द्ध - मा॒सम् । \newline
27. स॒ध॒माद॑म् दे॒वैर् दे॒वैः स॑ध॒मादꣳ॑ सध॒माद॑म् दे॒वैः सोमꣳ॒॒ सोम॑म् दे॒वैः स॑ध॒मादꣳ॑ सध॒माद॑म् दे॒वैः सोम᳚म् । \newline
28. स॒ध॒माद॒मिति॑ सध - माद᳚म् । \newline
29. दे॒वैः सोमꣳ॒॒ सोम॑म् दे॒वैर् दे॒वैः सोम॑म् पिबति पिबति॒ सोम॑म् दे॒वैर् दे॒वैः सोम॑म् पिबति । \newline
30. सोम॑म् पिबति पिबति॒ सोमꣳ॒॒ सोम॑म् पिबति॒ यद् यत् पि॑बति॒ सोमꣳ॒॒ सोम॑म् पिबति॒ यत् । \newline
31. पि॒ब॒ति॒ यद् यत् पि॑बति पिबति॒ यन् मै᳚त्रावरु॒ण्या मै᳚त्रावरु॒ण्या यत् पि॑बति पिबति॒ यन् मै᳚त्रावरु॒ण्या । \newline
32. यन् मै᳚त्रावरु॒ण्या मै᳚त्रावरु॒ण्या यद् यन् मै᳚त्रावरु॒ण्या ऽऽमिक्ष॑या॒ ऽऽमिक्ष॑या मैत्रावरु॒ण्या यद् यन् मै᳚त्रावरु॒ण्या ऽऽमिक्ष॑या । \newline
33. मै॒त्रा॒व॒रु॒ण्या ऽऽमिक्ष॑या॒ ऽऽमिक्ष॑या मैत्रावरु॒ण्या मै᳚त्रावरु॒ण्या ऽऽमिक्ष॑या ऽमावा॒स्या॑या ममावा॒स्या॑या मा॒मिक्ष॑या मैत्रावरु॒ण्या मै᳚त्रावरु॒ण्या ऽऽमिक्ष॑या ऽमावा॒स्या॑याम् । \newline
34. मै॒त्रा॒व॒रु॒ण्येति॑ मैत्रा - व॒रु॒ण्या । \newline
35. आ॒मिक्ष॑या ऽमावा॒स्या॑या ममावा॒स्या॑या मा॒मिक्ष॑या॒ ऽऽमिक्ष॑या ऽमावा॒स्या॑यां॒ ॅयज॑ते॒ यज॑ते ऽमावा॒स्या॑या मा॒मिक्ष॑या॒ ऽऽमिक्ष॑या ऽमावा॒स्या॑यां॒ ॅयज॑ते । \newline
36. अ॒मा॒वा॒स्या॑यां॒ ॅयज॑ते॒ यज॑ते ऽमावा॒स्या॑या ममावा॒स्या॑यां॒ ॅयज॑ते॒ या या यज॑ते ऽमावा॒स्या॑या ममावा॒स्या॑यां॒ ॅयज॑ते॒ या । \newline
37. अ॒मा॒वा॒स्या॑या॒मित्य॑मा - वा॒स्या॑याम् । \newline
38. यज॑ते॒ या या यज॑ते॒ यज॑ते॒ यैवैव या यज॑ते॒ यज॑ते॒ यैव । \newline
39. यैवैव या यैवासा व॒सा वे॒व या यैवासौ । \newline
40. ए॒वासा व॒सा वे॒वैवासौ दे॒वाना᳚म् दे॒वाना॑ म॒सा वे॒वैवासौ दे॒वाना᳚म् । \newline
41. अ॒सौ दे॒वाना᳚म् दे॒वाना॑ म॒सा व॒सौ दे॒वानां᳚ ॅव॒शा व॒शा दे॒वाना॑ म॒सा व॒सौ दे॒वानां᳚ ॅव॒शा । \newline
42. दे॒वानां᳚ ॅव॒शा व॒शा दे॒वाना᳚म् दे॒वानां᳚ ॅव॒शा ऽनू॑ब॒न्ध्या॑ ऽनूब॒न्ध्या॑ व॒शा दे॒वाना᳚म् दे॒वानां᳚ ॅव॒शा ऽनू॑ब॒न्ध्या᳚ । \newline
43. व॒शा ऽनू॑ब॒न्ध्या॑ ऽनूब॒न्ध्या॑ व॒शा व॒शा ऽनू॑ब॒न्ध्या॑ सो सो अ॑नूब॒न्ध्या॑ व॒शा व॒शा ऽनू॑ब॒न्ध्या॑ सो । \newline
44. अ॒नू॒ब॒न्ध्या॑ सो सो अ॑नूब॒न्ध्या॑ ऽनूब॒न्ध्या॑ सो ए॒वैव सो अ॑नूब॒न्ध्या॑ ऽनूब॒न्ध्या॑ सो ए॒व । \newline
45. अ॒नू॒ब॒न्ध्येत्य॑नु - ब॒न्ध्या᳚ । \newline
46. सो ए॒वैव सो सो ए॒वैषैषैव सो सो ए॒वैषा । \newline
47. सो इति॒ सो । \newline
48. ए॒वैषै षैवैवै षैतस्यै॒तस्यै॒ षैवैवै षैतस्य॑ । \newline
49. ए॒षैतस्यै॒ तस्यै॒षैषै तस्य॑ सा॒क्षाथ् सा॒क्षा दे॒तस्यै॒षैषै तस्य॑ सा॒क्षात् । \newline
50. ए॒तस्य॑ सा॒क्षाथ् सा॒क्षा दे॒तस्यै॒तस्य॑ सा॒क्षाद् वै वै सा॒क्षा दे॒तस्यै॒तस्य॑ सा॒क्षाद् वै । \newline
51. सा॒क्षाद् वै वै सा॒क्षाथ् सा॒क्षाद् वा ए॒ष ए॒ष वै सा॒क्षाथ् सा॒क्षाद् वा ए॒षः । \newline
52. सा॒क्षादिति॑ स - अ॒क्षात् । \newline
53. वा ए॒ष ए॒ष वै वा ए॒ष दे॒वान् दे॒वा ने॒ष वै वा ए॒ष दे॒वान् । \newline
54. ए॒ष दे॒वान् दे॒वा ने॒ष ए॒ष दे॒वा न॒भ्यारो॑ह त्य॒भ्यारो॑हति दे॒वा ने॒ष ए॒ष दे॒वा न॒भ्यारो॑हति । \newline
55. दे॒वा न॒भ्यारो॑ह त्य॒भ्यारो॑हति दे॒वान् दे॒वा न॒भ्यारो॑हति॒ यो यो᳚ ऽभ्यारो॑हति दे॒वान् दे॒वा न॒भ्यारो॑हति॒ यः । \newline
56. अ॒भ्यारो॑हति॒ यो यो᳚ ऽभ्यारो॑ह त्य॒भ्यारो॑हति॒ य ए॑षा मेषां॒ ॅयो᳚ ऽभ्यारो॑ह त्य॒भ्यारो॑हति॒ य ए॑षाम् । \newline
57. अ॒भ्यारो॑ह॒तीत्य॑भि - आरो॑हति । \newline
58. य ए॑षा मेषां॒ ॅयो य ए॑षां ॅय॒ज्ञ्ं ॅय॒ज्ञ् मे॑षां॒ ॅयो य ए॑षां ॅय॒ज्ञ्म् । \newline
59. ए॒षां॒ ॅय॒ज्ञ्ं ॅय॒ज्ञ् मे॑षा मेषां ॅय॒ज्ञ् म॑भ्या॒रोह॑ त्यभ्या॒रोह॑ति य॒ज्ञ् मे॑षा मेषां ॅय॒ज्ञ् म॑भ्या॒रोह॑ति । \newline
60. य॒ज्ञ् म॑भ्या॒रोह॑ त्यभ्या॒रोह॑ति य॒ज्ञ्ं ॅय॒ज्ञ् म॑भ्या॒रोह॑ति॒ यथा॒ यथा᳚ ऽभ्या॒रोह॑ति य॒ज्ञ्ं ॅय॒ज्ञ् म॑भ्या॒रोह॑ति॒ यथा᳚ । \newline
\pagebreak
\markright{ TS 2.5.5.6  \hfill https://www.vedavms.in \hfill}
\addcontentsline{toc}{section}{ TS 2.5.5.6 }
\section*{ TS 2.5.5.6 }

\textbf{TS 2.5.5.6 } \newline
\textbf{Samhita Paata} \newline

-म॑भ्या॒रोह॑ति॒ यथा॒ खलु॒वै श्रेया॑न॒भ्यारू॑ढः का॒मय॑ते॒ तथा॑ करोति॒ यद्य॑व॒विद्ध्य॑ति॒ पापी॑यान् भवति॒ यदि॒ नाव॒विद्ध्य॑ति स॒दृङ् व्या॒वृत्का॑म ए॒तेन॑ य॒ज्ञेन॑ यजेत क्षु॒रप॑वि॒र्ह्ये॑ष य॒ज्ञ्स्ता॒जक् पुण्यो॑ वा॒ भव॑ति॒ प्र वा॑ मीयते॒ तस्यै॒तद्व्र॒तं नानृ॑तं ॅवदे॒न्न माꣳ॒॒ सम॑श्नीया॒न्न स्त्रिय॒मुपे॑या॒न्नास्य॒ पल्पू॑लनेन॒ वासः॑ पल्पूलयेयु ( ) -रे॒तद्धि दे॒वाः सर्वं॒ न कु॒र्वन्ति॑ ॥ \newline

\textbf{Pada Paata} \newline

अ॒भ्या॒रोह॒तीत्य॑भि - आ॒रोह॑ति । यथा᳚ । खलु॑ । वै । श्रेयान्॑ । अ॒भ्यारू॑ढ॒ इत्य॑भि - आरू॑ढः । का॒मय॑ते । तथा᳚ । क॒रो॒ति॒ । यदि॑ । अ॒व॒विद्ध्य॒तीत्य॑व - विद्ध्य॑ति । पापी॑यान् । भ॒व॒ति॒ । यदि॑ । न । अ॒व॒विद्ध्य॒तीत्य॑व - विद्ध्य॑ति । स॒दृङ्ङिति॑ स - दृङ् । व्या॒वृत्का॑म॒ इति॑ व्या॒वृत् - का॒मः॒ । ए॒तेन॑ । य॒ज्ञेन॑ । य॒जे॒त॒ । क्षु॒रप॑वि॒रिति॑ क्षु॒र - प॒विः॒ । हि । ए॒षः । य॒ज्ञ्ः । ता॒जक् । पुण्यः॑ । वा॒ । भव॑ति । प्रेति॑ । वा॒ । मी॒य॒ते॒ । तस्य॑ । ए॒तत् । व्र॒तम् । न । अनृ॑तम् । व॒दे॒त् । न । माꣳ॒॒सम् । अ॒श्नी॒या॒त् । न । स्त्रिय᳚म् । उपेति॑ । इ॒या॒त् । न । अ॒स्य॒ । पल्पू॑लनेन । वासः॑ । प॒ल्पू॒ल॒ये॒युः॒ ( ) । ए॒तत् । हि । दे॒वाः । सर्व᳚म् । न । कु॒र्वन्ति॑ ॥  \newline


\textbf{Krama Paata} \newline

अ॒भ्या॒रोह॑ति॒ यथा᳚ । अ॒भ्या॒रोह॒तीत्य॑भि - आ॒रोह॑ति । यथा॒ खलु॑ । खलु॒ वै । वै श्रेयान्॑ । श्रेया॑न॒भ्यारू॑ढः । अ॒भ्यारू॑ढः का॒मय॑ते । अ॒भ्यारू॑ढ॒ इत्य॑भि - आरू॑ढः । का॒मय॑ते॒ तथा᳚ । तथा॑ करोति । क॒रो॒ति॒ यदि॑ । यद्य॑व॒विद्ध्य॑ति । अ॒व॒विद्ध्य॑ति॒ पापी॑यान् । अ॒व॒विद्ध्य॒तीत्य॑व - विद्ध्य॑ति । पापी॑यान् भवति । भ॒व॒ति॒ यदि॑ । यदि॒ न । नाव॒विद्ध्य॑ति । अ॒व॒विद्ध्य॑ति स॒दृङ्ङ् । अ॒व॒विद्ध्य॒तीत्य॑व - विद्ध्य॑ति । स॒दृङ् व्या॒वृत्का॑मः । स॒दृङ्ङिति॑ स - दृङ्ङ् । व्या॒वृत्का॑म ए॒तेन॑ । व्या॒वृत्का॑म॒ इति॑ व्या॒वृत् - का॒मः॒ । ए॒तेन॑ य॒ज्ञेन॑ । य॒ज्ञेन॑ यजेत । य॒जे॒त॒ क्षु॒रप॑विः । क्षु॒रप॑वि॒र्॒. हि । क्षु॒रप॑वि॒रिति॑ क्षु॒र - प॒विः॒ । ह्ये॑षः । ए॒ष य॒ज्ञ्ः । य॒ज्ञ् स्ता॒जक् । ता॒जक् पुण्यः॑ । पुण्यो॑ वा । वा॒ भव॑ति । भव॑ति॒ प्र । प्र वा᳚ । वा॒ मी॒य॒ते॒ । मी॒य॒ते॒ तस्य॑ । तस्यै॒तत् । ए॒तद् व्र॒तम् । व्र॒तम् न । नानृ॑तम् । अनृ॑तम् ॅवदेत् । व॒दे॒न् न । न माꣳ॒॒सम् । माꣳ॒॒सम॑श्ञीयात् । अ॒श्ञी॒या॒न् न । न स्त्रिय᳚म् । स्त्रिय॒मुप॑ । उपे॑यात् । इ॒या॒न् न । नास्य॑ । अ॒स्य॒ पल्पू॑लनेन । पल्पू॑लनेन॒ वासः॑ । वासः॑ पल्पूलयेयुः ( ) । प॒ल्पू॒ल॒ये॒यु॒रे॒तत् । ए॒तद्धि । हि दे॒वाः । दे॒वाः सर्व᳚म् । सर्व॒म् न । न कु॒र्वन्ति॑ । कु॒र्वन्तीति॑ कु॒र्वन्ति॑ । \newline

\textbf{Jatai Paata} \newline

1. अ॒भ्या॒रोह॑ति॒ यथा॒ यथा᳚ ऽभ्या॒रोह॑ त्यभ्या॒रोह॑ति॒ यथा᳚ । \newline
2. अ॒भ्या॒रोह॒तीत्य॑भि - आ॒रोह॑ति । \newline
3. यथा॒ खलु॒ खलु॒ यथा॒ यथा॒ खलु॑ । \newline
4. खलु॒ वै वै खलु॒ खलु॒ वै । \newline
5. वै श्रेया॒ञ् छ्रेया॒न्॒. वै वै श्रेयान्॑ । \newline
6. श्रेया॑ न॒भ्यारू॑ढो॒ ऽभ्यारू॑ढः॒ श्रेया॒ञ् छ्रेया॑ न॒भ्यारू॑ढः । \newline
7. अ॒भ्यारू॑ढः का॒मय॑ते का॒मय॑ते॒ ऽभ्यारू॑ढो॒ ऽभ्यारू॑ढः का॒मय॑ते । \newline
8. अ॒भ्यारू॑ढ॒ इत्य॑भि - आरू॑ढः । \newline
9. का॒मय॑ते॒ तथा॒ तथा॑ का॒मय॑ते का॒मय॑ते॒ तथा᳚ । \newline
10. तथा॑ करोति करोति॒ तथा॒ तथा॑ करोति । \newline
11. क॒रो॒ति॒ यदि॒ यदि॑ करोति करोति॒ यदि॑ । \newline
12. यद्य॑व॒विद्ध्य॑ त्यव॒विद्ध्य॑ति॒ यदि॒ यद्य॑व॒विद्ध्य॑ति । \newline
13. अ॒व॒विद्ध्य॑ति॒ पापी॑या॒न् पापी॑या नव॒विद्ध्य॑ त्यव॒विद्ध्य॑ति॒ पापी॑यान् । \newline
14. अ॒व॒विद्ध्य॒तीत्य॑व - विद्ध्य॑ति । \newline
15. पापी॑यान् भवति भवति॒ पापी॑या॒न् पापी॑यान् भवति । \newline
16. भ॒व॒ति॒ यदि॒ यदि॑ भवति भवति॒ यदि॑ । \newline
17. यदि॒ न न यदि॒ यदि॒ न । \newline
18. नाव॒विद्ध्य॑ त्यव॒विद्ध्य॑ति॒ न नाव॒विद्ध्य॑ति । \newline
19. अ॒व॒विद्ध्य॑ति स॒दृङ् ख्स॒दृङ् ङ॑व॒विद्ध्य॑ त्यव॒विद्ध्य॑ति स॒दृङ् । \newline
20. अ॒व॒विद्ध्य॒तीत्य॑व - विद्ध्य॑ति । \newline
21. स॒दृङ् व्या॒वृत्का॑मो व्या॒वृत्का॑मः स॒दृङ् ख्स॒दृङ् व्या॒वृत्का॑मः । \newline
22. स॒दृङ्ङिति॑ स - दृङ् । \newline
23. व्या॒वृत्का॑म ए॒तेनै॒तेन॑ व्या॒वृत्का॑मो व्या॒वृत्का॑म ए॒तेन॑ । \newline
24. व्या॒वृत्का॑म॒ इति॑ व्या॒वृत् - का॒मः॒ । \newline
25. ए॒तेन॑ य॒ज्ञेन॑ य॒ज्ञे नै॒ते नै॒तेन॑ य॒ज्ञेन॑ । \newline
26. य॒ज्ञेन॑ यजेत यजेत य॒ज्ञेन॑ य॒ज्ञेन॑ यजेत । \newline
27. य॒जे॒त॒ क्षु॒रप॑विः क्षु॒रप॑विर् यजेत यजेत क्षु॒रप॑विः । \newline
28. क्षु॒रप॑वि॒र्॒. हि हि क्षु॒रप॑विः क्षु॒रप॑वि॒र्॒. हि । \newline
29. क्षु॒रप॑वि॒रिति॑ क्षु॒र - प॒विः॒ । \newline
30. ह्ये॑ष ए॒ष हि ह्ये॑षः । \newline
31. ए॒ष य॒ज्ञो य॒ज्ञ् ए॒ष ए॒ष य॒ज्ञ्ः । \newline
32. य॒ज्ञ् स्ता॒जक् ता॒जग् य॒ज्ञो य॒ज्ञ् स्ता॒जक् । \newline
33. ता॒जक् पुण्यः॒ पुण्य॑ स्ता॒जक् ता॒जक् पुण्यः॑ । \newline
34. पुण्यो॑ वा वा॒ पुण्यः॒ पुण्यो॑ वा । \newline
35. वा॒ भव॑ति॒ भव॑ति वा वा॒ भव॑ति । \newline
36. भव॑ति॒ प्र प्र भव॑ति॒ भव॑ति॒ प्र । \newline
37. प्र वा॑ वा॒ प्र प्र वा᳚ । \newline
38. वा॒ मी॒य॒ते॒ मी॒य॒ते॒ वा॒ वा॒ मी॒य॒ते॒ । \newline
39. मी॒य॒ते॒ तस्य॒ तस्य॑ मीयते मीयते॒ तस्य॑ । \newline
40. तस्यै॒त दे॒तत् तस्य॒ तस्यै॒तत् । \newline
41. ए॒तद् व्र॒तं ॅव्र॒त मे॒त दे॒तद् व्र॒तम् । \newline
42. व्र॒तम् न न व्र॒तं ॅव्र॒तम् न । \newline
43. नानृ॑त॒ मनृ॑त॒म् न नानृ॑तम् । \newline
44. अनृ॑तं ॅवदेद् वदे॒ दनृ॑त॒ मनृ॑तं ॅवदेत् । \newline
45. व॒दे॒न् न न व॑देद् वदे॒न् न । \newline
46. न माꣳ॒॒सम् माꣳ॒॒सम् न न माꣳ॒॒सम् । \newline
47. माꣳ॒॒स म॑श्ञीया दश्ञीयान् माꣳ॒॒सम् माꣳ॒॒स म॑श्ञीयात् । \newline
48. अ॒श्ञी॒या॒न् न नाश्ञी॑या दश्ञीया॒न् न । \newline
49. न स्त्रियꣳ॒॒ स्त्रिय॒म् न न स्त्रिय᳚म् । \newline
50. स्त्रिय॒ मुपोप॒ स्त्रियꣳ॒॒ स्त्रिय॒ मुप॑ । \newline
51. उपे॑ यादिया॒ दुपोपे॑ यात् । \newline
52. इ॒या॒न् न ने या॑दिया॒न् न । \newline
53. नास्या᳚स्य॒ न नास्य॑ । \newline
54. अ॒स्य॒ पल्पू॑लनेन॒ पल्पू॑लनेना स्यास्य॒ पल्पू॑लनेन । \newline
55. पल्पू॑लनेन॒ वासो॒ वासः॒ पल्पू॑लनेन॒ पल्पू॑लनेन॒ वासः॑ । \newline
56. वासः॑ पल्पूलयेयुः पल्पूलयेयु॒र् वासो॒ वासः॑ पल्पूलयेयुः । \newline
57. प॒ल्पू॒ल॒ये॒यु॒ रे॒त दे॒तत् प॑ल्पूलयेयुः पल्पूलयेयु रे॒तत् । \newline
58. ए॒तद्धि ह्ये॑त दे॒तद्धि । \newline
59. हि दे॒वा दे॒वा हि हि दे॒वाः । \newline
60. दे॒वाः सर्वꣳ॒॒ सर्व॑म् दे॒वा दे॒वाः सर्व᳚म् । \newline
61. सर्व॒म् न न सर्वꣳ॒॒ सर्व॒म् न । \newline
62. न कु॒र्वन्ति॑ कु॒र्वन्ति॒ न न कु॒र्वन्ति॑ । \newline
63. कु॒र्वन्तीति॑ कु॒र्वन्ति॑ । \newline

\textbf{Ghana Paata } \newline

1. अ॒भ्या॒रोह॑ति॒ यथा॒ यथा᳚ ऽभ्या॒रोह॑ त्यभ्या॒रोह॑ति॒ यथा॒ खलु॒ खलु॒ यथा᳚ ऽभ्या॒रोह॑ त्यभ्या॒रोह॑ति॒ यथा॒ खलु॑ । \newline
2. अ॒भ्या॒रोह॒तीत्य॑भि - आ॒रोह॑ति । \newline
3. यथा॒ खलु॒ खलु॒ यथा॒ यथा॒ खलु॒ वै वै खलु॒ यथा॒ यथा॒ खलु॒ वै । \newline
4. खलु॒ वै वै खलु॒ खलु॒ वै श्रेया॒ञ् छ्रेया॒न्॒. वै खलु॒ खलु॒ वै श्रेयान्॑ । \newline
5. वै श्रेया॒ञ् छ्रेया॒न्॒. वै वै श्रेया॑ न॒भ्यारू॑ढो॒ ऽभ्यारू॑ढः॒ श्रेया॒न्॒. वै वै श्रेया॑ न॒भ्यारू॑ढः । \newline
6. श्रेया॑ न॒भ्यारू॑ढो॒ ऽभ्यारू॑ढः॒ श्रेया॒ञ् छ्रेया॑ न॒भ्यारू॑ढः का॒मय॑ते का॒मय॑ते॒ ऽभ्यारू॑ढः॒ श्रेया॒ञ् छ्रेया॑ न॒भ्यारू॑ढः का॒मय॑ते । \newline
7. अ॒भ्यारू॑ढः का॒मय॑ते का॒मय॑ते॒ ऽभ्यारू॑ढो॒ ऽभ्यारू॑ढः का॒मय॑ते॒ तथा॒ तथा॑ का॒मय॑ते॒ ऽभ्यारू॑ढो॒ ऽभ्यारू॑ढः का॒मय॑ते॒ तथा᳚ । \newline
8. अ॒भ्यारू॑ढ॒ इत्य॑भि - आरू॑ढः । \newline
9. का॒मय॑ते॒ तथा॒ तथा॑ का॒मय॑ते का॒मय॑ते॒ तथा॑ करोति करोति॒ तथा॑ का॒मय॑ते का॒मय॑ते॒ तथा॑ करोति । \newline
10. तथा॑ करोति करोति॒ तथा॒ तथा॑ करोति॒ यदि॒ यदि॑ करोति॒ तथा॒ तथा॑ करोति॒ यदि॑ । \newline
11. क॒रो॒ति॒ यदि॒ यदि॑ करोति करोति॒ यद्य॑व॒विद्ध्य॑ त्यव॒विद्ध्य॑ति॒ यदि॑ करोति करोति॒ यद्य॑व॒विद्ध्य॑ति । \newline
12. यद्य॑व॒विद्ध्य॑ त्यव॒विद्ध्य॑ति॒ यदि॒ यद्य॑व॒विद्ध्य॑ति॒ पापी॑या॒न् पापी॑या नव॒विद्ध्य॑ति॒ यदि॒ यद्य॑व॒विद्ध्य॑ति॒ पापी॑यान् । \newline
13. अ॒व॒विद्ध्य॑ति॒ पापी॑या॒न् पापी॑या नव॒विद्ध्य॑ त्यव॒विद्ध्य॑ति॒ पापी॑यान् भवति भवति॒ पापी॑या नव॒विद्ध्य॑ त्यव॒विद्ध्य॑ति॒ पापी॑यान् भवति । \newline
14. अ॒व॒विद्ध्य॒तीत्य॑व - विद्ध्य॑ति । \newline
15. पापी॑यान् भवति भवति॒ पापी॑या॒न् पापी॑यान् भवति॒ यदि॒ यदि॑ भवति॒ पापी॑या॒न् पापी॑यान् भवति॒ यदि॑ । \newline
16. भ॒व॒ति॒ यदि॒ यदि॑ भवति भवति॒ यदि॒ न न यदि॑ भवति भवति॒ यदि॒ न । \newline
17. यदि॒ न न यदि॒ यदि॒ नाव॒विद्ध्य॑ त्यव॒विद्ध्य॑ति॒ न यदि॒ यदि॒ नाव॒विद्ध्य॑ति । \newline
18. नाव॒विद्ध्य॑ त्यव॒विद्ध्य॑ति॒ न नाव॒विद्ध्य॑ति स॒दृङ् ख्स॒दृङ् ङ॑व॒विद्ध्य॑ति॒ न नाव॒विद्ध्य॑ति स॒दृङ् । \newline
19. अ॒व॒विद्ध्य॑ति स॒दृङ् ख्स॒दृङ् ङ॑व॒विद्ध्य॑ त्यव॒विद्ध्य॑ति स॒दृङ् व्या॒वृत्का॑मो व्या॒वृत्का॑मः स॒दृङ् ङ॑व॒विद्ध्य॑ त्यव॒विद्ध्य॑ति स॒दृङ् व्या॒वृत्का॑मः । \newline
20. अ॒व॒विद्ध्य॒तीत्य॑व - विद्ध्य॑ति । \newline
21. स॒दृङ् व्या॒वृत्का॑मो व्या॒वृत्का॑मः स॒दृङ् ख्स॒दृङ् व्या॒वृत्का॑म ए॒तेनै॒तेन॑ व्या॒वृत्का॑मः स॒दृङ् ख्स॒दृङ् व्या॒वृत्का॑म ए॒तेन॑ । \newline
22. स॒दृङ्ङिति॑ स - दृङ् । \newline
23. व्या॒वृत्का॑म ए॒तेनै॒तेन॑ व्या॒वृत्का॑मो व्या॒वृत्का॑म ए॒तेन॑ य॒ज्ञेन॑ य॒ज्ञेनै॒तेन॑ व्या॒वृत्का॑मो व्या॒वृत्का॑म ए॒तेन॑ य॒ज्ञेन॑ । \newline
24. व्या॒वृत्का॑म॒ इति॑ व्या॒वृत् - का॒मः॒ । \newline
25. ए॒तेन॑ य॒ज्ञेन॑ य॒ज्ञे नै॒ते नै॒तेन॑ य॒ज्ञेन॑ यजेत यजेत य॒ज्ञे नै॒ते नै॒तेन॑ य॒ज्ञेन॑ यजेत । \newline
26. य॒ज्ञेन॑ यजेत यजेत य॒ज्ञेन॑ य॒ज्ञेन॑ यजेत क्षु॒रप॑विः क्षु॒रप॑विर् यजेत य॒ज्ञेन॑ य॒ज्ञेन॑ यजेत क्षु॒रप॑विः । \newline
27. य॒जे॒त॒ क्षु॒रप॑विः क्षु॒रप॑विर् यजेत यजेत क्षु॒रप॑वि॒र्॒. हि हि क्षु॒रप॑विर् यजेत यजेत क्षु॒रप॑वि॒र्॒. हि । \newline
28. क्षु॒रप॑वि॒र्॒. हि हि क्षु॒रप॑विः क्षु॒रप॑वि॒र् ह्ये॑ष ए॒ष हि क्षु॒रप॑विः क्षु॒रप॑वि॒र् ह्ये॑षः । \newline
29. क्षु॒रप॑वि॒रिति॑ क्षु॒र - प॒विः॒ । \newline
30. ह्ये॑ष ए॒ष हि ह्ये॑ष य॒ज्ञो य॒ज्ञ् ए॒ष हि ह्ये॑ष य॒ज्ञ्ः । \newline
31. ए॒ष य॒ज्ञो य॒ज्ञ् ए॒ष ए॒ष य॒ज्ञ् स्ता॒जक् ता॒जग् य॒ज्ञ् ए॒ष ए॒ष य॒ज्ञ् स्ता॒जक् । \newline
32. य॒ज्ञ् स्ता॒जक् ता॒जग् य॒ज्ञो य॒ज्ञ् स्ता॒जक् पुण्यः॒ पुण्य॑ स्ता॒जग् य॒ज्ञो य॒ज्ञ् स्ता॒जक् पुण्यः॑ । \newline
33. ता॒जक् पुण्यः॒ पुण्य॑ स्ता॒जक् ता॒जक् पुण्यो॑ वा वा॒ पुण्य॑ स्ता॒जक् ता॒जक् पुण्यो॑ वा । \newline
34. पुण्यो॑ वा वा॒ पुण्यः॒ पुण्यो॑ वा॒ भव॑ति॒ भव॑ति वा॒ पुण्यः॒ पुण्यो॑ वा॒ भव॑ति । \newline
35. वा॒ भव॑ति॒ भव॑ति वा वा॒ भव॑ति॒ प्र प्र भव॑ति वा वा॒ भव॑ति॒ प्र । \newline
36. भव॑ति॒ प्र प्र भव॑ति॒ भव॑ति॒ प्र वा॑ वा॒ प्र भव॑ति॒ भव॑ति॒ प्र वा᳚ । \newline
37. प्र वा॑ वा॒ प्र प्र वा॑ मीयते मीयते वा॒ प्र प्र वा॑ मीयते । \newline
38. वा॒ मी॒य॒ते॒ मी॒य॒ते॒ वा॒ वा॒ मी॒य॒ते॒ तस्य॒ तस्य॑ मीयते वा वा मीयते॒ तस्य॑ । \newline
39. मी॒य॒ते॒ तस्य॒ तस्य॑ मीयते मीयते॒ तस्यै॒त दे॒तत् तस्य॑ मीयते मीयते॒ तस्यै॒तत् । \newline
40. तस्यै॒त दे॒तत् तस्य॒ तस्यै॒तद् व्र॒तं ॅव्र॒त मे॒तत् तस्य॒ तस्यै॒तद् व्र॒तम् । \newline
41. ए॒तद् व्र॒तं ॅव्र॒त मे॒त दे॒तद् व्र॒तन्न न व्र॒त मे॒त दे॒तद् व्र॒तन्न । \newline
42. व्र॒तम् न न व्र॒तं ॅव्र॒तम् नानृ॑त॒ मनृ॑त॒म् न व्र॒तं ॅव्र॒तम् नानृ॑तम् । \newline
43. नानृ॑त॒ मनृ॑त॒म् न नानृ॑तं ॅवदेद् वदे॒ दनृ॑त॒म् न नानृ॑तं ॅवदेत् । \newline
44. अनृ॑तं ॅवदेद् वदे॒ दनृ॑त॒ मनृ॑तं ॅवदे॒न् न न व॑दे॒ दनृ॑त॒ मनृ॑तं ॅवदे॒न् न । \newline
45. व॒दे॒न् न न व॑देद् वदे॒न् न माꣳ॒॒सम् माꣳ॒॒सम् न व॑देद् वदे॒न् न माꣳ॒॒सम् । \newline
46. न माꣳ॒॒सम् माꣳ॒॒सम् न न माꣳ॒॒स म॑श्ञीया दश्ञीयान् माꣳ॒॒सन्न न माꣳ॒॒स म॑श्ञीयात् । \newline
47. माꣳ॒॒स म॑श्ञीया दश्ञीयान् माꣳ॒॒सम् माꣳ॒॒स म॑श्ञीया॒न् न नाश्ञी॑यान् माꣳ॒॒सम् माꣳ॒॒स म॑श्ञीया॒न् न । \newline
48. अ॒श्ञी॒या॒न् न नाश्ञी॑या दश्ञीया॒न् न स्त्रियꣳ॒॒ स्त्रिय॒म् नाश्ञी॑या दश्ञीया॒न् न स्त्रिय᳚म् । \newline
49. न स्त्रियꣳ॒॒ स्त्रिय॒म् न न स्त्रिय॒ मुपोप॒ स्त्रिय॒म् न न स्त्रिय॒ मुप॑ । \newline
50. स्त्रिय॒ मुपोप॒ स्त्रियꣳ॒॒ स्त्रिय॒ मुपे॑ यादिया॒दुप॒ स्त्रियꣳ॒॒ स्त्रिय॒ मुपे॑ यात् । \newline
51. उपे॑ यादिया॒ दुपोपे॑ या॒न् न ने या॒दुपोपे॑ या॒न् न । \newline
52. इ॒या॒न् न ने या॑दिया॒न् नास्या᳚स्य॒ ने या॑दिया॒न् नास्य॑ । \newline
53. नास्या᳚स्य॒ न नास्य॒ पल्पू॑लनेन॒ पल्पू॑लनेनास्य॒ न नास्य॒ पल्पू॑लनेन । \newline
54. अ॒स्य॒ पल्पू॑लनेन॒ पल्पू॑लनेना स्यास्य॒ पल्पू॑लनेन॒ वासो॒ वासः॒ पल्पू॑लनेना स्यास्य॒ पल्पू॑लनेन॒ वासः॑ । \newline
55. पल्पू॑लनेन॒ वासो॒ वासः॒ पल्पू॑लनेन॒ पल्पू॑लनेन॒ वासः॑ पल्पूलयेयुः पल्पूलयेयु॒र् वासः॒ पल्पू॑लनेन॒ पल्पू॑लनेन॒ वासः॑ पल्पूलयेयुः । \newline
56. वासः॑ पल्पूलयेयुः पल्पूलयेयु॒र् वासो॒ वासः॑ पल्पूलयेयु रे॒त दे॒तत् प॑ल्पूलयेयु॒र् वासो॒ वासः॑ पल्पूलयेयु रे॒तत् । \newline
57. प॒ल्पू॒ल॒ये॒यु॒ रे॒त दे॒तत् प॑ल्पूलयेयुः पल्पूलयेयु रे॒तद्धि ह्ये॑तत् प॑ल्पूलयेयुः पल्पूलयेयु रे॒तद्धि । \newline
58. ए॒तद्धि ह्ये॑त दे॒तद्धि दे॒वा दे॒वा ह्ये॑त दे॒तद्धि दे॒वाः । \newline
59. हि दे॒वा दे॒वा हि हि दे॒वाः सर्वꣳ॒॒ सर्व॑म् दे॒वा हि हि दे॒वाः सर्व᳚म् । \newline
60. दे॒वाः सर्वꣳ॒॒ सर्व॑म् दे॒वा दे॒वाः सर्व॒न्न न सर्व॑म् दे॒वा दे॒वाः सर्व॒न्न । \newline
61. सर्व॒न्न न सर्वꣳ॒॒ सर्व॒न्न कु॒र्वन्ति॑ कु॒र्वन्ति॒ न सर्वꣳ॒॒ सर्व॒न्न कु॒र्वन्ति॑ । \newline
62. न कु॒र्वन्ति॑ कु॒र्वन्ति॒ न न कु॒र्वन्ति॑ । \newline
63. कु॒र्वन्तीति॑ कु॒र्वन्ति॑ । \newline
\pagebreak
\markright{ TS 2.5.6.1  \hfill https://www.vedavms.in \hfill}
\addcontentsline{toc}{section}{ TS 2.5.6.1 }
\section*{ TS 2.5.6.1 }

\textbf{TS 2.5.6.1 } \newline
\textbf{Samhita Paata} \newline

ए॒ष वै दे॑वर॒थो यद्-द॑र्.शपूर्णमा॒सौ यो द॑र्.शपूर्णमा॒सावि॒ष्ट्वा सोमे॑न॒ यज॑ते॒ रथ॑स्पष्ट ए॒वाव॒साने॒ वरे॑ दे॒वाना॒मव॑ स्यत्ये॒तानि॒ वा अङ्गा॒परूꣳ॑षि संॅवथ्स॒रस्य॒ यद्-द॑र्.शपूर्णमा॒सौ य ए॒वं ॅवि॒द्वान् द॑र्.शपूर्णमा॒सौ यज॒ते-ऽङ्गा॒परूꣳ॑ष्ये॒व सं॑ॅवथ्स॒रस्य॒ प्रति॑ दधात्ये॒ ते वै सं॑ॅवथ्स॒रस्य॒ चक्षु॑षी॒ यद्-द॑र्.शपूर्णमा॒सौ य ए॒वं ॅवि॒द्वान् द॑र्.शपूर्णमा॒सौ यज॑ते॒ ताभ्या॑मे॒व सु॑व॒र्गं ॅलो॒कमनु॑ पश्य - [  ] \newline

\textbf{Pada Paata} \newline

ए॒षः । वै । दे॒व॒र॒थ इति॑ देव - र॒थः । यत् । द॒र्.॒श॒पू॒र्ण॒मा॒साविति॑ दर्.श - पू॒र्ण॒मा॒सौ । यः । द॒र्.॒श॒पू॒र्ण॒मा॒साविति॑ दर्.श - पू॒र्ण॒मा॒सौ । इ॒ष्ट्वा । सोमे॑न । यज॑ते । रथ॑स्पष्ट॒ इति॒ रथ॑ - स्प॒ष्टे॒ । ए॒व । अ॒व॒सान॒ इत्य॑व-साने᳚ । वरे᳚ । दे॒वाना᳚म् । अवेति॑ । स्य॒ति॒ । ए॒तानि॑ । वै । अङ्गा॒परूꣳ॒॒षीत्यङ्गा᳚ - परूꣳ॑षि । सं॒ॅव॒थ्स॒रस्येति॑ सं - व॒थ्स॒रस्य॑ । यत् । द॒र्.॒श॒पू॒र्ण॒मा॒साविति॑ दर्.श-पू॒र्ण॒मा॒सौ । यः । ए॒वम् । वि॒द्वान् । द॒र्.॒श॒पू॒र्ण॒मा॒साविति॑ दर्.श - पू॒र्ण॒मा॒सौ । यज॑ते । अङ्गा॒परूꣳ॒॒षीत्यङ्गा᳚ - परूꣳ॑षि । ए॒व । सं॒ॅव॒थ्स॒रस्येति॑ सं - व॒थ्स॒रस्य॑ । प्रतीति॑ । द॒धा॒ति॒ । ए॒ते इति॑ । वै । सं॒ॅव॒थ्स॒रस्येति॑ सं - व॒थ्स॒रस्य॑ । चक्षु॑षी॒ इति॑ । यत् । द॒र्.॒श॒पू॒र्ण॒मा॒साविति॑ दर्.श - पू॒र्ण॒मा॒सौ । यः । ए॒वम् । वि॒द्वान् । द॒र्.॒श॒पू॒र्ण॒मा॒साविति॑ दर्.श - पू॒र्ण॒मा॒सौ । यज॑ते । ताभ्या᳚म् । ए॒व । सु॒व॒र्गमिति॑ सुवः - गम् । लो॒कम् । अन्विति॑ । प॒श्य॒ति॒ ।  \newline


\textbf{Krama Paata} \newline

ए॒ष वै । वै दे॑वर॒थः । दे॒व॒र॒थो यत् । दे॒व॒र॒थ इति॑ देव - र॒थः । यद् द॑र्.शपूर्णमा॒सौ । द॒र्॒.श॒पू॒र्ण॒मा॒सौ यः । द॒र्॒.श॒पू॒र्ण॒मा॒साविति॑ दर्.श - पू॒र्ण॒मा॒सौ । यो द॑र्.शपूर्णमा॒सौ । द॒र्॒.श॒पू॒र्ण॒मा॒सावि॒ष्ट्वा । द॒र्॒.श॒पू॒र्ण॒मा॒साविति॑ दर्.श - पू॒र्ण॒मा॒सौ । इ॒ष्ट्वा सोमे॑न । सोमे॑न॒ यज॑ते । यज॑ते॒ रथ॑स्पष्टे । रथ॑स्पष्ट ए॒व । रथ॑स्पष्ट॒ इति॒ रथ॑ - स्प॒ष्टे॒ । ए॒वाव॒साने᳚ । अ॒व॒साने॒ वरे᳚ । अ॒व॒सान॒ इत्य॑व - साने᳚ । वरे॑ दे॒वाना᳚म् । दे॒वाना॒मव॑ । अव॑ स्यति । स्य॒त्ये॒तानि॑ । ए॒तानि॒ वै । 
वा अङ्गा॒परूꣳ॑षि । अङ्गा॒परूꣳ॑षि सम्ॅवथ्स॒रस्य॑ । अङ्गा॒परूꣳ॒॒षीत्यङ्गा᳚ - परूꣳ॑षि । स॒म्ॅव॒थ्स॒रस्य॒ यत् । स॒म्ॅव॒थ्स॒रस्येति॑ सम् - व॒थ्स॒रस्य॑ । यद् द॑र्.शपूर्णमा॒सौ । द॒र्॒.श॒पू॒र्ण॒मा॒सौ यः । द॒र्॒.श॒पू॒र्ण॒मा॒साविति॑ दर्.श - पू॒र्ण॒मा॒सौ । य ए॒वम् । ए॒वम् ॅवि॒द्वान् । वि॒द्वान् द॑र्.शपूर्णमा॒सौ । द॒र्॒.श॒पू॒र्ण॒मा॒सौ यज॑ते । द॒र्॒.श॒पू॒र्ण॒मा॒साविति॑ दर्.श - पू॒र्ण॒मा॒सौ । यज॒तेऽङ्गा॒परूꣳ॑षि । अङ्गा॒परूꣳ॑ष्ये॒व । अङ्गा॒परूꣳ॒॒षीत्यङ्गा᳚ - परूꣳ॑षि । ए॒व स॑म्ॅवथ्स॒रस्य॑ । स॒म्ॅव॒थ्स॒रस्य॒ प्रति॑ । स॒म्ॅव॒थ्स॒रस्येति॑ सम् - व॒थ्स॒रस्य॑ । प्रति॑ दधाति । द॒धा॒त्ये॒ते । ए॒ते वै । ए॒ते इत्ये॒ते । वै स॑म्ॅवथ्स॒रस्य॑ । स॒म्ॅव॒थ्स॒रस्य॒ चक्षु॑षी । स॒म्ॅव॒थ्स॒रस्येति॑ सम् - व॒थ्स॒रस्य॑ । चक्षु॑षी॒ यत् । चक्षु॑षी॒ इति॒ चक्षु॑षी । यद् द॑र्.शपूर्णमा॒सौ । द॒र्॒.श॒पू॒र्ण॒मा॒सौ यः । द॒र्॒.श॒पू॒र्ण॒मा॒साविति॑ दर्.श - पू॒र्ण॒मा॒सौ । य ए॒वम् । ए॒वम् ॅवि॒द्वान् । वि॒द्वान् द॑र्.शपूर्णमा॒सौ । द॒र्॒.श॒पू॒र्ण॒मा॒सौ यज॑ते । द॒र्॒.श॒पू॒र्ण॒मा॒साविति॑ दर्.श - पू॒र्ण॒मा॒सौ । यज॑ते॒ ताभ्या᳚म् । ताभ्या॑मे॒व । ए॒व सु॑व॒र्गम् । सु॒व॒र्गम् ॅलो॒कम् । सु॒व॒र्गमिति॑ सुवः - गम् । लो॒कमनु॑ । अनु॑ पश्यति । प॒श्य॒त्ये॒षा \newline

\textbf{Jatai Paata} \newline

1. ए॒ष वै वा ए॒ष ए॒ष वै । \newline
2. वै दे॑वर॒थो दे॑वर॒थो वै वै दे॑वर॒थः । \newline
3. दे॒व॒र॒थो यद् यद् दे॑वर॒थो दे॑वर॒थो यत् । \newline
4. दे॒व॒र॒थ इति॑ देव - र॒थः । \newline
5. यद् द॑र्.शपूर्णमा॒सौ द॑र्.शपूर्णमा॒सौ यद् यद् द॑र्.शपूर्णमा॒सौ । \newline
6. द॒र्॒.श॒पू॒र्ण॒मा॒सौ यो यो द॑र्.शपूर्णमा॒सौ द॑र्.शपूर्णमा॒सौ यः । \newline
7. द॒र्॒.श॒पू॒र्ण॒मा॒साविति॑ दर्.श - पू॒र्ण॒मा॒सौ । \newline
8. यो द॑र्.शपूर्णमा॒सौ द॑र्.शपूर्णमा॒सौ यो यो द॑र्.शपूर्णमा॒सौ । \newline
9. द॒र्॒.श॒पू॒र्ण॒मा॒सा वि॒ष्ट्वेष्ट्वा द॑र्.शपूर्णमा॒सौ द॑र्.शपूर्णमा॒सा वि॒ष्ट्वा । \newline
10. द॒र्॒.श॒पू॒र्ण॒मा॒साविति॑ दर्.श - पू॒र्ण॒मा॒सौ । \newline
11. इ॒ष्ट्वा सोमे॑न॒ सोमे॑ने॒ ष्ट्वे ष्ट्वा सोमे॑न । \newline
12. सोमे॑न॒ यज॑ते॒ यज॑ते॒ सोमे॑न॒ सोमे॑न॒ यज॑ते । \newline
13. यज॑ते॒ रथ॑स्पष्टे॒ रथ॑स्पष्टे॒ यज॑ते॒ यज॑ते॒ रथ॑स्पष्टे । \newline
14. रथ॑स्पष्ट ए॒वैव रथ॑स्पष्टे॒ रथ॑स्पष्ट ए॒व । \newline
15. रथ॑स्पष्ट॒ इति॒ रथ॑ - स्प॒ष्टे॒ । \newline
16. ए॒वाव॒साने॑ ऽव॒सान॑ ए॒वैवा व॒साने᳚ । \newline
17. अ॒व॒साने॒ वरे॒ वरे॑ ऽव॒साने॑ ऽव॒साने॒ वरे᳚ । \newline
18. अ॒व॒सान॒ इत्य॑व - साने᳚ । \newline
19. वरे॑ दे॒वाना᳚म् दे॒वानां॒ ॅवरे॒ वरे॑ दे॒वाना᳚म् । \newline
20. दे॒वाना॒ मवाव॑ दे॒वाना᳚म् दे॒वाना॒ मव॑ । \newline
21. अव॑ स्यति स्य॒ त्यवाव॑ स्यति । \newline
22. स्य॒ त्ये॒ता न्ये॒तानि॑ स्यति स्य त्ये॒तानि॑ । \newline
23. ए॒तानि॒ वै वा ए॒ता न्ये॒तानि॒ वै । \newline
24. वा अङ्गा॒परूꣳ॒॒ ष्यङ्गा॒परूꣳ॑षि॒ वै वा अङ्गा॒परूꣳ॑षि । \newline
25. अङ्गा॒परूꣳ॑षि संॅवथ्स॒रस्य॑ संॅवथ्स॒रस्या ङ्गा॒परूꣳ॒॒ ष्यङ्गा॒परूꣳ॑षि संॅवथ्स॒रस्य॑ । \newline
26. अङ्गा॒परूꣳ॒॒षीत्यङ्गा᳚ - परूꣳ॑षि । \newline
27. सं॒ॅव॒थ्स॒रस्य॒ यद् यथ् सं॑ॅवथ्स॒रस्य॑ संॅवथ्स॒रस्य॒ यत् । \newline
28. सं॒ॅव॒थ्स॒रस्येति॑ सं - व॒थ्स॒रस्य॑ । \newline
29. यद् द॑र्.शपूर्णमा॒सौ द॑र्.शपूर्णमा॒सौ यद् यद् द॑र्.शपूर्णमा॒सौ । \newline
30. द॒र्॒.श॒पू॒र्ण॒मा॒सौ यो यो द॑र्.शपूर्णमा॒सौ द॑र्.शपूर्णमा॒सौ यः । \newline
31. द॒र्॒.श॒पू॒र्ण॒मा॒साविति॑ दर्.श - पू॒र्ण॒मा॒सौ । \newline
32. य ए॒व मे॒वं ॅयो य ए॒वम् । \newline
33. ए॒वं ॅवि॒द्वान्. वि॒द्वा ने॒व मे॒वं ॅवि॒द्वान् । \newline
34. वि॒द्वान् द॑र्.शपूर्णमा॒सौ द॑र्.शपूर्णमा॒सौ वि॒द्वान्. वि॒द्वान् द॑र्.शपूर्णमा॒सौ । \newline
35. द॒र्॒.श॒पू॒र्ण॒मा॒सौ यज॑ते॒ यज॑ते दर्.शपूर्णमा॒सौ द॑र्.शपूर्णमा॒सौ यज॑ते । \newline
36. द॒र्॒.श॒पू॒र्ण॒मा॒साविति॑ दर्.श - पू॒र्ण॒मा॒सौ । \newline
37. यज॒ते ऽङ्गा॒परूꣳ॒॒ ष्यङ्गा॒परूꣳ॑षि॒ यज॑ते॒ यज॒ते ऽङ्गा॒परूꣳ॑षि । \newline
38. अङ्गा॒परूꣳ॑ ष्ये॒वैवाङ्गा॒परूꣳ॒॒ ष्यङ्गा॒परूꣳ॑ष्ये॒व । \newline
39. अङ्गा॒परूꣳ॒॒षीत्यङ्गा᳚ - परूꣳ॑षि । \newline
40. ए॒व सं॑ॅवथ्स॒रस्य॑ संॅवथ्स॒रस्यै॒वैव सं॑ॅवथ्स॒रस्य॑ । \newline
41. सं॒ॅव॒थ्स॒रस्य॒ प्रति॒ प्रति॑ संॅवथ्स॒रस्य॑ संॅवथ्स॒रस्य॒ प्रति॑ । \newline
42. सं॒ॅव॒थ्स॒रस्येति॑ सं - व॒थ्स॒रस्य॑ । \newline
43. प्रति॑ दधाति दधाति॒ प्रति॒ प्रति॑ दधाति । \newline
44. द॒धा॒त्ये॒ते ए॒ते द॑धाति दधात्ये॒ते । \newline
45. ए॒ते वै वा ए॒ते ए॒ते वै । \newline
46. ए॒ते इत्ये॒ते । \newline
47. वै सं॑ॅवथ्स॒रस्य॑ संॅवथ्स॒रस्य॒ वै वै सं॑ॅवथ्स॒रस्य॑ । \newline
48. सं॒ॅव॒थ्स॒रस्य॒ चक्षु॑षी॒ चक्षु॑षी संॅवथ्स॒रस्य॑ संॅवथ्स॒रस्य॒ चक्षु॑षी । \newline
49. सं॒ॅव॒थ्स॒रस्येति॑ सं - व॒थ्स॒रस्य॑ । \newline
50. चक्षु॑षी॒ यद् यच् चक्षु॑षी॒ चक्षु॑षी॒ यत् । \newline
51. चक्षु॑षी॒ इति॒ चक्षु॑षी । \newline
52. यद् द॑र्.शपूर्णमा॒सौ द॑र्.शपूर्णमा॒सौ यद् यद् द॑र्.शपूर्णमा॒सौ । \newline
53. द॒र्॒.श॒पू॒र्ण॒मा॒सौ यो यो द॑र्.शपूर्णमा॒सौ द॑र्.शपूर्णमा॒सौ यः । \newline
54. द॒र्॒.श॒पू॒र्ण॒मा॒साविति॑ दर्.श - पू॒र्ण॒मा॒सौ । \newline
55. य ए॒व मे॒वं ॅयो य ए॒वम् । \newline
56. ए॒वं ॅवि॒द्वान्. वि॒द्वा ने॒व मे॒वं ॅवि॒द्वान् । \newline
57. वि॒द्वान् द॑र्.शपूर्णमा॒सौ द॑र्.शपूर्णमा॒सौ वि॒द्वान्. वि॒द्वान् द॑र्.शपूर्णमा॒सौ । \newline
58. द॒र्॒.श॒पू॒र्ण॒मा॒सौ यज॑ते॒ यज॑ते दर्.शपूर्णमा॒सौ द॑र्.शपूर्णमा॒सौ यज॑ते । \newline
59. द॒र्॒.श॒पू॒र्ण॒मा॒साविति॑ दर्.श - पू॒र्ण॒मा॒सौ । \newline
60. यज॑ते॒ ताभ्या॒म् ताभ्यां॒ ॅयज॑ते॒ यज॑ते॒ ताभ्या᳚म् । \newline
61. ताभ्या॑ मे॒वैव ताभ्या॒म् ताभ्या॑ मे॒व । \newline
62. ए॒व सु॑व॒र्गꣳ सु॑व॒र्ग मे॒वैव सु॑व॒र्गम् । \newline
63. सु॒व॒र्गम् ॅलो॒कम् ॅलो॒कꣳ सु॑व॒र्गꣳ सु॑व॒र्गम् ॅलो॒कम् । \newline
64. सु॒व॒र्गमिति॑ सुवः - गम् । \newline
65. लो॒क मन्वनु॑ लो॒कम् ॅलो॒क मनु॑ । \newline
66. अनु॑ पश्यति पश्य॒ त्यन्वनु॑ पश्यति । \newline
67. प॒श्य॒ त्ये॒षैषा प॑श्यति पश्य त्ये॒षा । \newline

\textbf{Ghana Paata } \newline

1. ए॒ष वै वा ए॒ष ए॒ष वै दे॑वर॒थो दे॑वर॒थो वा ए॒ष ए॒ष वै दे॑वर॒थः । \newline
2. वै दे॑वर॒थो दे॑वर॒थो वै वै दे॑वर॒थो यद् यद् दे॑वर॒थो वै वै दे॑वर॒थो यत् । \newline
3. दे॒व॒र॒थो यद् यद् दे॑वर॒थो दे॑वर॒थो यद् द॑र्.शपूर्णमा॒सौ द॑र्.शपूर्णमा॒सौ यद् दे॑वर॒थो दे॑वर॒थो यद् द॑र्.शपूर्णमा॒सौ । \newline
4. दे॒व॒र॒थ इति॑ देव - र॒थः । \newline
5. यद् द॑र्.शपूर्णमा॒सौ द॑र्.शपूर्णमा॒सौ यद् यद् द॑र्.शपूर्णमा॒सौ यो यो द॑र्.शपूर्णमा॒सौ यद् यद् द॑र्.शपूर्णमा॒सौ यः । \newline
6. द॒र्॒.श॒पू॒र्ण॒मा॒सौ यो यो द॑र्.शपूर्णमा॒सौ द॑र्.शपूर्णमा॒सौ यो द॑र्.शपूर्णमा॒सौ द॑र्.शपूर्णमा॒सौ यो द॑र्.शपूर्णमा॒सौ द॑र्.शपूर्णमा॒सौ यो द॑र्.शपूर्णमा॒सौ । \newline
7. द॒र्॒.श॒पू॒र्ण॒मा॒साविति॑ दर्.श - पू॒र्ण॒मा॒सौ । \newline
8. यो द॑र्.शपूर्णमा॒सौ द॑र्.शपूर्णमा॒सौ यो यो द॑र्.शपूर्णमा॒सा वि॒ष्ट्वे ष्ट्वा द॑र्.शपूर्णमा॒सौ यो यो द॑र्.शपूर्णमा॒सा वि॒ष्ट्वा । \newline
9. द॒र्॒.श॒पू॒र्ण॒मा॒सा वि॒ष्ट्वे ष्ट्वा द॑र्.शपूर्णमा॒सौ द॑र्.शपूर्णमा॒सा वि॒ष्ट्वा सोमे॑न॒ सोमे॑ने॒ ष्ट्वा द॑र्.शपूर्णमा॒सौ द॑र्.शपूर्णमा॒सा वि॒ष्ट्वा सोमे॑न । \newline
10. द॒र्॒.श॒पू॒र्ण॒मा॒साविति॑ दर्.श - पू॒र्ण॒मा॒सौ । \newline
11. इ॒ष्ट्वा सोमे॑न॒ सोमे॑ने॒ ष्ट्वेष्ट्वा सोमे॑न॒ यज॑ते॒ यज॑ते॒ सोमे॑ने॒ ष्ट्वे ष्ट्वा सोमे॑न॒ यज॑ते । \newline
12. सोमे॑न॒ यज॑ते॒ यज॑ते॒ सोमे॑न॒ सोमे॑न॒ यज॑ते॒ रथ॑स्पष्टे॒ रथ॑स्पष्टे॒ यज॑ते॒ सोमे॑न॒ सोमे॑न॒ यज॑ते॒ रथ॑स्पष्टे । \newline
13. यज॑ते॒ रथ॑स्पष्टे॒ रथ॑स्पष्टे॒ यज॑ते॒ यज॑ते॒ रथ॑स्पष्ट ए॒वैव रथ॑स्पष्टे॒ यज॑ते॒ यज॑ते॒ रथ॑स्पष्ट ए॒व । \newline
14. रथ॑स्पष्ट ए॒वैव रथ॑स्पष्टे॒ रथ॑स्पष्ट ए॒वा व॒साने॑ ऽव॒सान॑ ए॒व रथ॑स्पष्टे॒ रथ॑स्पष्ट ए॒वाव॒साने᳚ । \newline
15. रथ॑स्पष्ट॒ इति॒ रथ॑ - स्प॒ष्टे॒ । \newline
16. ए॒वा व॒साने॑ ऽव॒सान॑ ए॒वै वाव॒साने॒ वरे॒ वरे॑ ऽव॒सान॑ ए॒वै वाव॒साने॒ वरे᳚ । \newline
17. अ॒व॒साने॒ वरे॒ वरे॑ ऽव॒साने॑ ऽव॒साने॒ वरे॑ दे॒वाना᳚म् दे॒वानां॒ ॅवरे॑ ऽव॒साने॑ ऽव॒साने॒ वरे॑ दे॒वाना᳚म् । \newline
18. अ॒व॒सान॒ इत्य॑व - साने᳚ । \newline
19. वरे॑ दे॒वाना᳚म् दे॒वानां॒ ॅवरे॒ वरे॑ दे॒वाना॒ मवाव॑ दे॒वानां॒ ॅवरे॒ वरे॑ दे॒वाना॒ मव॑ । \newline
20. दे॒वाना॒ मवाव॑ दे॒वाना᳚म् दे॒वाना॒ मव॑ स्यति स्य॒त्यव॑ दे॒वाना᳚म् दे॒वाना॒ मव॑ स्यति । \newline
21. अव॑ स्यति स्य॒ त्यवाव॑ स्य त्ये॒ता न्ये॒तानि॑ स्य॒ त्यवाव॑ स्य त्ये॒तानि॑ । \newline
22. स्य॒ त्ये॒ता न्ये॒तानि॑ स्यति स्य त्ये॒तानि॒ वै वा ए॒तानि॑ स्यति स्य त्ये॒तानि॒ वै । \newline
23. ए॒तानि॒ वै वा ए॒ता न्ये॒तानि॒ वा अङ्गा॒परूꣳ॒॒ ष्यङ्गा॒परूꣳ॑षि॒ वा ए॒ता न्ये॒तानि॒ वा अङ्गा॒परूꣳ॑षि । \newline
24. वा अङ्गा॒परूꣳ॒॒ ष्यङ्गा॒परूꣳ॑षि॒ वै वा अङ्गा॒परूꣳ॑षि संॅवथ्स॒रस्य॑ संॅवथ्स॒रस्या ङ्गा॒परूꣳ॑षि॒ वै वा अङ्गा॒परूꣳ॑षि संॅवथ्स॒रस्य॑ । \newline
25. अङ्गा॒परूꣳ॑षि संॅवथ्स॒रस्य॑ संॅवथ्स॒रस्या ङ्गा॒परूꣳ॒॒ ष्यङ्गा॒परूꣳ॑षि संॅवथ्स॒रस्य॒ यद् यथ् सं॑ॅवथ्स॒रस्या ङ्गा॒परूꣳ॒॒ ष्यङ्गा॒परूꣳ॑षि संॅवथ्स॒रस्य॒ यत् । \newline
26. अङ्गा॒परूꣳ॒॒षीत्यङ्गा᳚ - परूꣳ॑षि । \newline
27. सं॒ॅव॒थ्स॒रस्य॒ यद् यथ् सं॑ॅवथ्स॒रस्य॑ संॅवथ्स॒रस्य॒ यद् द॑र्.शपूर्णमा॒सौ द॑र्.शपूर्णमा॒सौ यथ् सं॑ॅवथ्स॒रस्य॑ संॅवथ्स॒रस्य॒ यद् द॑र्.शपूर्णमा॒सौ । \newline
28. सं॒ॅव॒थ्स॒रस्येति॑ सं - व॒थ्स॒रस्य॑ । \newline
29. यद् द॑र्.शपूर्णमा॒सौ द॑र्.शपूर्णमा॒सौ यद् यद् द॑र्.शपूर्णमा॒सौ यो यो द॑र्.शपूर्णमा॒सौ यद् यद् द॑र्.शपूर्णमा॒सौ यः । \newline
30. द॒र्॒.श॒पू॒र्ण॒मा॒सौ यो यो द॑र्.शपूर्णमा॒सौ द॑र्.शपूर्णमा॒सौ य ए॒व मे॒वं ॅयो द॑र्.शपूर्णमा॒सौ द॑र्.शपूर्णमा॒सौ य ए॒वम् । \newline
31. द॒र्॒.श॒पू॒र्ण॒मा॒साविति॑ दर्.श - पू॒र्ण॒मा॒सौ । \newline
32. य ए॒व मे॒वं ॅयो य ए॒वं ॅवि॒द्वान्. वि॒द्वा ने॒वं ॅयो य ए॒वं ॅवि॒द्वान् । \newline
33. ए॒वं ॅवि॒द्वान्. वि॒द्वा ने॒व मे॒वं ॅवि॒द्वान् द॑र्.शपूर्णमा॒सौ द॑र्.शपूर्णमा॒सौ वि॒द्वा ने॒व मे॒वं ॅवि॒द्वान् द॑र्.शपूर्णमा॒सौ । \newline
34. वि॒द्वान् द॑र्.शपूर्णमा॒सौ द॑र्.शपूर्णमा॒सौ वि॒द्वान्. वि॒द्वान् द॑र्.शपूर्णमा॒सौ यज॑ते॒ यज॑ते दर्.शपूर्णमा॒सौ वि॒द्वान्. वि॒द्वान् द॑र्.शपूर्णमा॒सौ यज॑ते । \newline
35. द॒र्॒.श॒पू॒र्ण॒मा॒सौ यज॑ते॒ यज॑ते दर्.शपूर्णमा॒सौ द॑र्.शपूर्णमा॒सौ यज॒ते ऽङ्गा॒परूꣳ॒॒ष्यङ्गा॒परूꣳ॑षि॒ यज॑ते दर्.शपूर्णमा॒सौ द॑र्.शपूर्णमा॒सौ यज॒ते ऽङ्गा॒परूꣳ॑षि । \newline
36. द॒र्॒.श॒पू॒र्ण॒मा॒साविति॑ दर्.श - पू॒र्ण॒मा॒सौ । \newline
37. यज॒ते ऽङ्गा॒परूꣳ॒॒ ष्यङ्गा॒परूꣳ॑षि॒ यज॑ते॒ यज॒ते ऽङ्गा॒परूꣳ॑ ष्ये॒वैवाङ्गा॒परूꣳ॑षि॒ यज॑ते॒ यज॒ते ऽङ्गा॒परूꣳ॑ष्ये॒व । \newline
38. अङ्गा॒परूꣳ॑ ष्ये॒वैवा ङ्गा॒परूꣳ॒॒ ष्यङ्गा॒परूꣳ॑ ष्ये॒व सं॑ॅवथ्स॒रस्य॑ संॅवथ्स॒रस्यै॒वा ङ्गा॒परूꣳ॒॒ ष्यङ्गा॒परूꣳ॑ ष्ये॒व सं॑ॅवथ्स॒रस्य॑ । \newline
39. अङ्गा॒परूꣳ॒॒षीत्यङ्गा᳚ - परूꣳ॑षि । \newline
40. ए॒व सं॑ॅवथ्स॒रस्य॑ संॅवथ्स॒र स्यै॒वैव सं॑ॅवथ्स॒रस्य॒ प्रति॒ प्रति॑ संॅवथ्स॒र स्यै॒वैव सं॑ॅवथ्स॒रस्य॒ प्रति॑ । \newline
41. सं॒ॅव॒थ्स॒रस्य॒ प्रति॒ प्रति॑ संॅवथ्स॒रस्य॑ संॅवथ्स॒रस्य॒ प्रति॑ दधाति दधाति॒ प्रति॑ संॅवथ्स॒रस्य॑ संॅवथ्स॒रस्य॒ प्रति॑ दधाति । \newline
42. सं॒ॅव॒थ्स॒रस्येति॑ सं - व॒थ्स॒रस्य॑ । \newline
43. प्रति॑ दधाति दधाति॒ प्रति॒ प्रति॑ दधात्ये॒ते ए॒ते द॑धाति॒ प्रति॒ प्रति॑ दधात्ये॒ते । \newline
44. द॒धा॒त्ये॒ते ए॒ते द॑धाति दधात्ये॒ते वै वा ए॒ते द॑धाति दधात्ये॒ते वै । \newline
45. ए॒ते वै वा ए॒ते ए॒ते वै सं॑ॅवथ्स॒रस्य॑ संॅवथ्स॒रस्य॒ वा ए॒ते ए॒ते वै सं॑ॅवथ्स॒रस्य॑ । \newline
46. ए॒ते इत्ये॒ते । \newline
47. वै सं॑ॅवथ्स॒रस्य॑ संॅवथ्स॒रस्य॒ वै वै सं॑ॅवथ्स॒रस्य॒ चक्षु॑षी॒ चक्षु॑षी संॅवथ्स॒रस्य॒ वै वै सं॑ॅवथ्स॒रस्य॒ चक्षु॑षी । \newline
48. सं॒ॅव॒थ्स॒रस्य॒ चक्षु॑षी॒ चक्षु॑षी संॅवथ्स॒रस्य॑ संॅवथ्स॒रस्य॒ चक्षु॑षी॒ यद् यच् चक्षु॑षी संॅवथ्स॒रस्य॑ संॅवथ्स॒रस्य॒ चक्षु॑षी॒ यत् । \newline
49. सं॒ॅव॒थ्स॒रस्येति॑ सं - व॒थ्स॒रस्य॑ । \newline
50. चक्षु॑षी॒ यद् यच् चक्षु॑षी॒ चक्षु॑षी॒ यद् द॑र्.शपूर्णमा॒सौ द॑र्.शपूर्णमा॒सौ यच् चक्षु॑षी॒ चक्षु॑षी॒ यद् द॑र्.शपूर्णमा॒सौ । \newline
51. चक्षु॑षी॒ इति॒ चक्षु॑षी । \newline
52. यद् द॑र्.शपूर्णमा॒सौ द॑र्.शपूर्णमा॒सौ यद् यद् द॑र्.शपूर्णमा॒सौ यो यो द॑र्.शपूर्णमा॒सौ यद् यद् द॑र्.शपूर्णमा॒सौ यः । \newline
53. द॒र्॒.श॒पू॒र्ण॒मा॒सौ यो यो द॑र्.शपूर्णमा॒सौ द॑र्.शपूर्णमा॒सौ य ए॒व मे॒वं ॅयो द॑र्.शपूर्णमा॒सौ द॑र्.शपूर्णमा॒सौ य ए॒वम् । \newline
54. द॒र्॒.श॒पू॒र्ण॒मा॒साविति॑ दर्.श - पू॒र्ण॒मा॒सौ । \newline
55. य ए॒व मे॒वं ॅयो य ए॒वं ॅवि॒द्वान्. वि॒द्वा ने॒वं ॅयो य ए॒वं ॅवि॒द्वान् । \newline
56. ए॒वं ॅवि॒द्वान्. वि॒द्वा ने॒व मे॒वं ॅवि॒द्वान् द॑र्.शपूर्णमा॒सौ द॑र्.शपूर्णमा॒सौ वि॒द्वा ने॒व मे॒वं ॅवि॒द्वान् द॑र्.शपूर्णमा॒सौ । \newline
57. वि॒द्वान् द॑र्.शपूर्णमा॒सौ द॑र्.शपूर्णमा॒सौ वि॒द्वान्. वि॒द्वान् द॑र्.शपूर्णमा॒सौ यज॑ते॒ यज॑ते दर्.शपूर्णमा॒सौ वि॒द्वान्. वि॒द्वान् द॑र्.शपूर्णमा॒सौ यज॑ते । \newline
58. द॒र्॒.श॒पू॒र्ण॒मा॒सौ यज॑ते॒ यज॑ते दर्.शपूर्णमा॒सौ द॑र्.शपूर्णमा॒सौ यज॑ते॒ ताभ्या॒म् ताभ्यां॒ ॅयज॑ते दर्.शपूर्णमा॒सौ द॑र्.शपूर्णमा॒सौ यज॑ते॒ ताभ्या᳚म् । \newline
59. द॒र्॒.श॒पू॒र्ण॒मा॒साविति॑ दर्.श - पू॒र्ण॒मा॒सौ । \newline
60. यज॑ते॒ ताभ्या॒म् ताभ्यां॒ ॅयज॑ते॒ यज॑ते॒ ताभ्या॑ मे॒वैव ताभ्यां॒ ॅयज॑ते॒ यज॑ते॒ ताभ्या॑ मे॒व । \newline
61. ताभ्या॑ मे॒वैव ताभ्या॒म् ताभ्या॑ मे॒व सु॑व॒र्गꣳ सु॑व॒र्ग मे॒व ताभ्या॒म् ताभ्या॑ मे॒व सु॑व॒र्गम् । \newline
62. ए॒व सु॑व॒र्गꣳ सु॑व॒र्ग मे॒वैव सु॑व॒र्गम् ॅलो॒कम् ॅलो॒कꣳ सु॑व॒र्ग मे॒वैव सु॑व॒र्गम् ॅलो॒कम् । \newline
63. सु॒व॒र्गम् ॅलो॒कम् ॅलो॒कꣳ सु॑व॒र्गꣳ सु॑व॒र्गम् ॅलो॒क मन्वनु॑ लो॒कꣳ सु॑व॒र्गꣳ सु॑व॒र्गम् ॅलो॒क मनु॑ । \newline
64. सु॒व॒र्गमिति॑ सुवः - गम् । \newline
65. लो॒क मन्वनु॑ लो॒कम् ॅलो॒क मनु॑ पश्यति पश्य॒ त्यनु॑ लो॒कम् ॅलो॒क मनु॑ पश्यति । \newline
66. अनु॑ पश्यति पश्य॒ त्यन्वनु॑ पश्यत्ये॒षैषा प॑श्य॒ त्यन्वनु॑ पश्य त्ये॒षा । \newline
67. प॒श्य॒ त्ये॒षैषा प॑श्यति पश्य त्ये॒षा वै वा ए॒षा प॑श्यति पश्य त्ये॒षा वै । \newline
\pagebreak
\markright{ TS 2.5.6.2  \hfill https://www.vedavms.in \hfill}
\addcontentsline{toc}{section}{ TS 2.5.6.2 }
\section*{ TS 2.5.6.2 }

\textbf{TS 2.5.6.2 } \newline
\textbf{Samhita Paata} \newline

-त्ये॒षा वै दे॒वानां॒ ॅविक्रा᳚न्ति॒ र्यद्-द॑र्.शपूर्णमा॒सौ य ए॒वं ॅवि॒द्वान् द॑र्.शपूर्णमा॒सौ यज॑ते दे॒वाना॑मे॒व विक्रा᳚न्ति॒मनु॒ विक्र॑मत ए॒ष वै दे॑व॒यानः॒ पन्था॒ यद्-द॑र्.शपूर्णमा॒सौ य ए॒वं ॅवि॒द्वान् द॑र्.शपूर्णमा॒सौ यज॑ते॒ य ए॒व दे॑व॒यानः॒ पन्था॒स्तꣳ स॒मारो॑हत्ये॒तौ वै दे॒वानाꣳ॒॒ हरी॒ यद्-द॑र्.शपूर्णमा॒सौ य ए॒वं ॅवि॒द्वान् द॑र्.श्पूर्णमा॒सौ यज॑ते॒ यावे॒व दे॒वानाꣳ॒॒ हरी॒ ताभ्या॑ - [  ] \newline

\textbf{Pada Paata} \newline

ए॒षा । वै । दे॒वाना᳚म् । विक्रा᳚न्ति॒रिति॒ वि - क्रा॒न्तिः॒ । यत् । द॒र्.॒श॒पू॒र्ण॒मा॒साविति॑ दर्.श - पू॒र्ण॒मा॒सौ । यः । ए॒वम् । वि॒द्वान् । द॒र्.॒श॒पू॒र्ण॒मा॒साविति॑ दर्.श - पू॒र्ण॒मा॒सौ । यज॑ते । दे॒वाना᳚म् । ए॒व । विक्रा᳚न्ति॒मिति॒ वि - क्रा॒न्ति॒म् । अनु॑ । वीति॑ । क्र॒म॒ते॒ । ए॒षः । वै । दे॒व॒यान॒ इति॑ देव - यानः॑ । पन्थाः᳚ । यत् । द॒र्.॒श॒पू॒र्ण॒मा॒साविति॑ दर्.श - पू॒र्ण॒मा॒सौ । यः । ए॒वम् । वि॒द्वान् । द॒र्.॒श॒पू॒र्ण॒मा॒साविति॑ दर्.श - पू॒र्ण॒मा॒सौ । यज॑ते । यः । ए॒व । दे॒व॒यान॒ इति॑ देव-यानः॑ । पन्थाः᳚ । तम् । स॒मारो॑ह॒तीति॑ सं - आरो॑हति । ए॒तौ । वै । दे॒वाना᳚म् । हरी॒ इति॑ । यत् । द॒र्.॒श॒पू॒र्ण॒मा॒साविति॑ दर्.श - पू॒र्ण॒मा॒सौ । यः । ए॒वम् । वि॒द्वान् । द॒र्.॒श॒पू॒र्ण॒मा॒साविति॑ दर्.श - पू॒र्ण॒मा॒सौ । यज॑ते । यौ । ए॒व । दे॒वाना᳚म् । हरी॒ इति॑ । ताभ्या᳚म् ।  \newline


\textbf{Krama Paata} \newline

ए॒षा वै । वै दे॒वाना᳚म् । दे॒वाना॒म् ॅविक्रा᳚न्तिः । विक्रा᳚न्ति॒र् यत् । विक्रा᳚न्ति॒रिति॒ वि - क्रा॒न्तिः॒ । यद् द॑र्.शपूर्णमा॒सौ । द॒र्॒.श॒पू॒र्ण॒मा॒सौ यः । द॒र्॒.श॒पू॒र्ण॒मा॒साविति॑ दर्.श - पू॒र्ण॒मा॒सौ । य ए॒वम् । ए॒वम् ॅवि॒द्वान् । वि॒द्वान् द॑र्.शपूर्णमा॒सौ । द॒र्॒.श॒पू॒र्ण॒मा॒सौ यज॑ते । द॒र्॒.श॒पू॒र्ण॒मा॒साविति॑ दर्.श - पू॒र्ण॒मा॒सौ । यज॑ते दे॒वाना᳚म् । दे॒वाना॑मे॒व । ए॒व विक्रा᳚न्तिम् । विक्रा᳚न्ति॒मनु॑ । विक्रा᳚न्ति॒मिति॒ वि - क्रा॒न्ति॒म् । अनु॒ वि । वि क्र॑मते । क्र॒म॒त॒ ए॒षः । ए॒ष वै । वै दे॑व॒यानः॑ । दे॒व॒यानः॒ पन्थाः᳚ । दे॒व॒यान॒ इति॑ देव - यानः॑ । पन्था॒ यत् । यद् द॑र्.शपूर्णमा॒सौ । द॒र्॒.श॒पू॒र्ण॒मा॒सौ यः । द॒र्॒.श॒पू॒र्ण॒मा॒साविति॑ दर्.श - पू॒र्ण॒मा॒सौ । य ए॒वम् । ए॒वम् ॅवि॒द्वान् । वि॒द्वान् द॑र्.शपूर्णमा॒सौ । द॒र्॒.श॒पू॒र्ण॒मा॒सौ यज॑ते । द॒र्॒.श॒पू॒र्ण॒मा॒साविति॑ दर्.श - पू॒र्ण॒मा॒सौ । यज॑ते॒ यः । य ए॒व । ए॒व दे॑व॒यानः॑ । दे॒व॒यानः॒ पन्थाः᳚ । दे॒व॒यान॒ इति॑ देव - यानः॑ । पन्था॒स्तम् । तꣳ स॒मारो॑हति । स॒मारो॑हत्ये॒तौ । स॒मारो॑ह॒तीति॑ सम् - आरो॑हति । ए॒तौ वै । वै दे॒वाना᳚म् । दे॒वानाꣳ॒॒ हरी᳚ । हरी॒ यत् । हरी॒ इति॒ हरी᳚ । यद् द॑र्.शपूर्णमा॒सौ । द॒र्॒.श॒पू॒र्ण॒मा॒सौ यः । द॒र्॒.श॒पू॒र्ण॒मा॒साविति॑ दर्.श - पू॒र्ण॒मा॒सौ । य ए॒वम् । ए॒वम् ॅवि॒द्वान् । वि॒द्वान् द॑र्.शपूर्णमा॒सौ । द॒र्॒.श॒पू॒र्ण॒मा॒सौ यज॑ते । द॒र्॒.श॒पू॒र्ण॒मा॒साविति॑ दर्.श - पू॒र्ण॒मा॒सौ । यज॑ते॒ यौ । यावे॒व । ए॒व दे॒वाना᳚म् । दे॒वानाꣳ॒॒ हरी᳚ । हरी॒ ताभ्या᳚म् । हरी॒ इति॒ हरी᳚ । ताभ्या॑मे॒व \newline

\textbf{Jatai Paata} \newline

1. ए॒षा वै वा ए॒षैषा वै । \newline
2. वै दे॒वाना᳚म् दे॒वानां॒ ॅवै वै दे॒वाना᳚म् । \newline
3. दे॒वानां॒ ॅविक्रा᳚न्ति॒र् विक्रा᳚न्तिर् दे॒वाना᳚म् दे॒वानां॒ ॅविक्रा᳚न्तिः । \newline
4. विक्रा᳚न्ति॒र् यद् यद् विक्रा᳚न्ति॒र् विक्रा᳚न्ति॒र् यत् । \newline
5. विक्रा᳚न्ति॒रिति॒ वि - क्रा॒न्तिः॒ । \newline
6. यद् द॑र्.शपूर्णमा॒सौ द॑र्.शपूर्णमा॒सौ यद् यद् द॑र्.शपूर्णमा॒सौ । \newline
7. द॒र्॒.श॒पू॒र्ण॒मा॒सौ यो यो द॑र्.शपूर्णमा॒सौ द॑र्.शपूर्णमा॒सौ यः । \newline
8. द॒र्॒.श॒पू॒र्ण॒मा॒साविति॑ दर्.श - पू॒र्ण॒मा॒सौ । \newline
9. य ए॒व मे॒वं ॅयो य ए॒वम् । \newline
10. ए॒वं ॅवि॒द्वान्. वि॒द्वा ने॒व मे॒वं ॅवि॒द्वान् । \newline
11. वि॒द्वान् द॑र्.शपूर्णमा॒सौ द॑र्.शपूर्णमा॒सौ वि॒द्वान्. वि॒द्वान् द॑र्.शपूर्णमा॒सौ । \newline
12. द॒र्॒.श॒पू॒र्ण॒मा॒सौ यज॑ते॒ यज॑ते दर्.शपूर्णमा॒सौ द॑र्.शपूर्णमा॒सौ यज॑ते । \newline
13. द॒र्॒.श॒पू॒र्ण॒मा॒साविति॑ दर्.श - पू॒र्ण॒मा॒सौ । \newline
14. यज॑ते दे॒वाना᳚म् दे॒वानां॒ ॅयज॑ते॒ यज॑ते दे॒वाना᳚म् । \newline
15. दे॒वाना॑ मे॒वैव दे॒वाना᳚म् दे॒वाना॑ मे॒व । \newline
16. ए॒व विक्रा᳚न्तिं॒ ॅविक्रा᳚न्ति मे॒वैव विक्रा᳚न्तिम् । \newline
17. विक्रा᳚न्ति॒ मन्वनु॒ विक्रा᳚न्तिं॒ ॅविक्रा᳚न्ति॒ मनु॑ । \newline
18. विक्रा᳚न्ति॒मिति॒ वि - क्रा॒न्ति॒म् । \newline
19. अनु॒ वि व्यन्वनु॒ वि । \newline
20. वि क्र॑मते क्रमते॒ वि वि क्र॑मते । \newline
21. क्र॒म॒त॒ ए॒ष ए॒ष क्र॑मते क्रमत ए॒षः । \newline
22. ए॒ष वै वा ए॒ष ए॒ष वै । \newline
23. वै दे॑व॒यानो॑ देव॒यानो॒ वै वै दे॑व॒यानः॑ । \newline
24. दे॒व॒यानः॒ पन्थाः॒ पन्था॑ देव॒यानो॑ देव॒यानः॒ पन्थाः᳚ । \newline
25. दे॒व॒यान॒ इति॑ देव - यानः॑ । \newline
26. पन्था॒ यद् यत् पन्थाः॒ पन्था॒ यत् । \newline
27. यद् द॑र्.शपूर्णमा॒सौ द॑र्.शपूर्णमा॒सौ यद् यद् द॑र्.शपूर्णमा॒सौ । \newline
28. द॒र्॒.श॒पू॒र्ण॒मा॒सौ यो यो द॑र्.शपूर्णमा॒सौ द॑र्.शपूर्णमा॒सौ यः । \newline
29. द॒र्॒.श॒पू॒र्ण॒मा॒साविति॑ दर्.श - पू॒र्ण॒मा॒सौ । \newline
30. य ए॒व मे॒वं ॅयो य ए॒वम् । \newline
31. ए॒वं ॅवि॒द्वान्. वि॒द्वा ने॒व मे॒वं ॅवि॒द्वान् । \newline
32. वि॒द्वान् द॑र्.शपूर्णमा॒सौ द॑र्.शपूर्णमा॒सौ वि॒द्वान्. वि॒द्वान् द॑र्.शपूर्णमा॒सौ । \newline
33. द॒र्॒.श॒पू॒र्ण॒मा॒सौ यज॑ते॒ यज॑ते दर्.शपूर्णमा॒सौ द॑र्.शपूर्णमा॒सौ यज॑ते । \newline
34. द॒र्॒.श॒पू॒र्ण॒मा॒साविति॑ दर्.श - पू॒र्ण॒मा॒सौ । \newline
35. यज॑ते॒ यो यो यज॑ते॒ यज॑ते॒ यः । \newline
36. य ए॒वैव यो य ए॒व । \newline
37. ए॒व दे॑व॒यानो॑ देव॒यान॑ ए॒वैव दे॑व॒यानः॑ । \newline
38. दे॒व॒यानः॒ पन्थाः॒ पन्था॑ देव॒यानो॑ देव॒यानः॒ पन्थाः᳚ । \newline
39. दे॒व॒यान॒ इति॑ देव - यानः॑ । \newline
40. पन्था॒ स्तम् तम् पन्थाः॒ पन्था॒ स्तम् । \newline
41. तꣳ स॒मारो॑हति स॒मारो॑हति॒ तम् तꣳ स॒मारो॑हति । \newline
42. स॒मारो॑ह त्ये॒ता वे॒तौ स॒मारो॑हति स॒मारो॑ह त्ये॒तौ । \newline
43. स॒मारो॑ह॒तीति॑ सं - आरो॑हति । \newline
44. ए॒तौ वै वा ए॒ता वे॒तौ वै । \newline
45. वै दे॒वाना᳚म् दे॒वानां॒ ॅवै वै दे॒वाना᳚म् । \newline
46. दे॒वानाꣳ॒॒ हरी॒ हरी॑ दे॒वाना᳚म् दे॒वानाꣳ॒॒ हरी᳚ । \newline
47. हरी॒ यद् य द्धरी॒ हरी॒ यत् । \newline
48. हरी॒ इति॒ हरी᳚ । \newline
49. यद् द॑र्.शपूर्णमा॒सौ द॑र्.शपूर्णमा॒सौ यद् यद् द॑र्.शपूर्णमा॒सौ । \newline
50. द॒र्॒.श॒पू॒र्ण॒मा॒सौ यो यो द॑र्.शपूर्णमा॒सौ द॑र्.शपूर्णमा॒सौ यः । \newline
51. द॒र्॒.श॒पू॒र्ण॒मा॒साविति॑ दर्.श - पू॒र्ण॒मा॒सौ । \newline
52. य ए॒व मे॒वं ॅयो य ए॒वम् । \newline
53. ए॒वं ॅवि॒द्वान्. वि॒द्वा ने॒व मे॒वं ॅवि॒द्वान् । \newline
54. वि॒द्वान् द॑र्.शपूर्णमा॒सौ द॑र्.शपूर्णमा॒सौ वि॒द्वान्. वि॒द्वान् द॑र्.शपूर्णमा॒सौ । \newline
55. द॒र्॒.श॒पू॒र्ण॒मा॒सौ यज॑ते॒ यज॑ते दर्.शपूर्णमा॒सौ द॑र्.शपूर्णमा॒सौ यज॑ते । \newline
56. द॒र्॒.श॒पू॒र्ण॒मा॒साविति॑ दर्.श - पू॒र्ण॒मा॒सौ । \newline
57. यज॑ते॒ यौ यौ यज॑ते॒ यज॑ते॒ यौ । \newline
58. या वे॒वैव यौ या वे॒व । \newline
59. ए॒व दे॒वाना᳚म् दे॒वाना॑ मे॒वैव दे॒वाना᳚म् । \newline
60. दे॒वानाꣳ॒॒ हरी॒ हरी॑ दे॒वाना᳚म् दे॒वानाꣳ॒॒ हरी᳚ । \newline
61. हरी॒ ताभ्या॒म् ताभ्याꣳ॒॒ हरी॒ हरी॒ ताभ्या᳚म् । \newline
62. हरी॒ इति॒ हरी᳚ । \newline
63. ताभ्या॑ मे॒वैव ताभ्या॒म् ताभ्या॑ मे॒व । \newline

\textbf{Ghana Paata } \newline

1. ए॒षा वै वा ए॒षैषा वै दे॒वाना᳚म् दे॒वानां॒ ॅवा ए॒षैषा वै दे॒वाना᳚म् । \newline
2. वै दे॒वाना᳚म् दे॒वानां॒ ॅवै वै दे॒वानां॒ ॅविक्रा᳚न्ति॒र् विक्रा᳚न्तिर् दे॒वानां॒ ॅवै वै दे॒वानां॒ ॅविक्रा᳚न्तिः । \newline
3. दे॒वानां॒ ॅविक्रा᳚न्ति॒र् विक्रा᳚न्तिर् दे॒वाना᳚म् दे॒वानां॒ ॅविक्रा᳚न्ति॒र् यद् यद् विक्रा᳚न्तिर् दे॒वाना᳚म् दे॒वानां॒ ॅविक्रा᳚न्ति॒र् यत् । \newline
4. विक्रा᳚न्ति॒र् यद् यद् विक्रा᳚न्ति॒र् विक्रा᳚न्ति॒र् यद् द॑र्.शपूर्णमा॒सौ द॑र्.शपूर्णमा॒सौ यद् विक्रा᳚न्ति॒र् विक्रा᳚न्ति॒र् यद् द॑र्.शपूर्णमा॒सौ । \newline
5. विक्रा᳚न्ति॒रिति॒ वि - क्रा॒न्तिः॒ । \newline
6. यद् द॑र्.शपूर्णमा॒सौ द॑र्.शपूर्णमा॒सौ यद् यद् द॑र्.शपूर्णमा॒सौ यो यो द॑र्.शपूर्णमा॒सौ यद् यद् द॑र्.शपूर्णमा॒सौ यः । \newline
7. द॒र्॒.श॒पू॒र्ण॒मा॒सौ यो यो द॑र्.शपूर्णमा॒सौ द॑र्.शपूर्णमा॒सौ य ए॒व मे॒वं ॅयो द॑र्.शपूर्णमा॒सौ द॑र्.शपूर्णमा॒सौ य ए॒वम् । \newline
8. द॒र्॒.श॒पू॒र्ण॒मा॒साविति॑ दर्.श - पू॒र्ण॒मा॒सौ । \newline
9. य ए॒व मे॒वं ॅयो य ए॒वं ॅवि॒द्वान्. वि॒द्वा ने॒वं ॅयो य ए॒वं ॅवि॒द्वान् । \newline
10. ए॒वं ॅवि॒द्वान्. वि॒द्वा ने॒व मे॒वं ॅवि॒द्वान् द॑र्.शपूर्णमा॒सौ द॑र्.शपूर्णमा॒सौ वि॒द्वा ने॒व मे॒वं ॅवि॒द्वान् द॑र्.शपूर्णमा॒सौ । \newline
11. वि॒द्वान् द॑र्.शपूर्णमा॒सौ द॑र्.शपूर्णमा॒सौ वि॒द्वान्. वि॒द्वान् द॑र्.शपूर्णमा॒सौ यज॑ते॒ यज॑ते दर्.शपूर्णमा॒सौ वि॒द्वान्. वि॒द्वान् द॑र्.शपूर्णमा॒सौ यज॑ते । \newline
12. द॒र्॒.श॒पू॒र्ण॒मा॒सौ यज॑ते॒ यज॑ते दर्.शपूर्णमा॒सौ द॑र्.शपूर्णमा॒सौ यज॑ते दे॒वाना᳚म् दे॒वानां॒ ॅयज॑ते दर्.शपूर्णमा॒सौ द॑र्.शपूर्णमा॒सौ यज॑ते दे॒वाना᳚म् । \newline
13. द॒र्॒.श॒पू॒र्ण॒मा॒साविति॑ दर्.श - पू॒र्ण॒मा॒सौ । \newline
14. यज॑ते दे॒वाना᳚म् दे॒वानां॒ ॅयज॑ते॒ यज॑ते दे॒वाना॑ मे॒वैव दे॒वानां॒ ॅयज॑ते॒ यज॑ते दे॒वाना॑ मे॒व । \newline
15. दे॒वाना॑ मे॒वैव दे॒वाना᳚म् दे॒वाना॑ मे॒व विक्रा᳚न्तिं॒ ॅविक्रा᳚न्ति मे॒व दे॒वाना᳚म् दे॒वाना॑ मे॒व विक्रा᳚न्तिम् । \newline
16. ए॒व विक्रा᳚न्तिं॒ ॅविक्रा᳚न्ति मे॒वैव विक्रा᳚न्ति॒ मन्वनु॒ विक्रा᳚न्ति मे॒वैव विक्रा᳚न्ति॒ मनु॑ । \newline
17. विक्रा᳚न्ति॒ मन्वनु॒ विक्रा᳚न्तिं॒ ॅविक्रा᳚न्ति॒ मनु॒ वि व्यनु॒ विक्रा᳚न्तिं॒ ॅविक्रा᳚न्ति॒ मनु॒ वि । \newline
18. विक्रा᳚न्ति॒मिति॒ वि - क्रा॒न्ति॒म् । \newline
19. अनु॒ वि व्यन्वनु॒ वि क्र॑मते क्रमते॒ व्यन्वनु॒ वि क्र॑मते । \newline
20. वि क्र॑मते क्रमते॒ वि वि क्र॑मत ए॒ष ए॒ष क्र॑मते॒ वि वि क्र॑मत ए॒षः । \newline
21. क्र॒म॒त॒ ए॒ष ए॒ष क्र॑मते क्रमत ए॒ष वै वा ए॒ष क्र॑मते क्रमत ए॒ष वै । \newline
22. ए॒ष वै वा ए॒ष ए॒ष वै दे॑व॒यानो॑ देव॒यानो॒ वा ए॒ष ए॒ष वै दे॑व॒यानः॑ । \newline
23. वै दे॑व॒यानो॑ देव॒यानो॒ वै वै दे॑व॒यानः॒ पन्थाः॒ पन्था॑ देव॒यानो॒ वै वै दे॑व॒यानः॒ पन्थाः᳚ । \newline
24. दे॒व॒यानः॒ पन्थाः॒ पन्था॑ देव॒यानो॑ देव॒यानः॒ पन्था॒ यद् यत् पन्था॑ देव॒यानो॑ देव॒यानः॒ पन्था॒ यत् । \newline
25. दे॒व॒यान॒ इति॑ देव - यानः॑ । \newline
26. पन्था॒ यद् यत् पन्थाः॒ पन्था॒ यद् द॑र्.शपूर्णमा॒सौ द॑र्.शपूर्णमा॒सौ यत् पन्थाः॒ पन्था॒ यद् द॑र्.शपूर्णमा॒सौ । \newline
27. यद् द॑र्.शपूर्णमा॒सौ द॑र्.शपूर्णमा॒सौ यद् यद् द॑र्.शपूर्णमा॒सौ यो यो द॑र्.शपूर्णमा॒सौ यद् यद् द॑र्.शपूर्णमा॒सौ यः । \newline
28. द॒र्॒.श॒पू॒र्ण॒मा॒सौ यो यो द॑र्.शपूर्णमा॒सौ द॑र्.शपूर्णमा॒सौ य ए॒व मे॒वं ॅयो द॑र्.शपूर्णमा॒सौ द॑र्.शपूर्णमा॒सौ य ए॒वम् । \newline
29. द॒र्॒.श॒पू॒र्ण॒मा॒साविति॑ दर्.श - पू॒र्ण॒मा॒सौ । \newline
30. य ए॒व मे॒वं ॅयो य ए॒वं ॅवि॒द्वान्. वि॒द्वा ने॒वं ॅयो य ए॒वं ॅवि॒द्वान् । \newline
31. ए॒वं ॅवि॒द्वान्. वि॒द्वा ने॒व मे॒वं ॅवि॒द्वान् द॑र्.शपूर्णमा॒सौ द॑र्.शपूर्णमा॒सौ वि॒द्वा ने॒व मे॒वं ॅवि॒द्वान् द॑र्.शपूर्णमा॒सौ । \newline
32. वि॒द्वान् द॑र्.शपूर्णमा॒सौ द॑र्.शपूर्णमा॒सौ वि॒द्वान्. वि॒द्वान् द॑र्.शपूर्णमा॒सौ यज॑ते॒ यज॑ते दर्.शपूर्णमा॒सौ वि॒द्वान्. वि॒द्वान् द॑र्.शपूर्णमा॒सौ यज॑ते । \newline
33. द॒र्॒.श॒पू॒र्ण॒मा॒सौ यज॑ते॒ यज॑ते दर्.शपूर्णमा॒सौ द॑र्.शपूर्णमा॒सौ यज॑ते॒ यो यो यज॑ते दर्.शपूर्णमा॒सौ द॑र्.शपूर्णमा॒सौ यज॑ते॒ यः । \newline
34. द॒र्॒.श॒पू॒र्ण॒मा॒साविति॑ दर्.श - पू॒र्ण॒मा॒सौ । \newline
35. यज॑ते॒ यो यो यज॑ते॒ यज॑ते॒ य ए॒वैव यो यज॑ते॒ यज॑ते॒ य ए॒व । \newline
36. य ए॒वैव यो य ए॒व दे॑व॒यानो॑ देव॒यान॑ ए॒व यो य ए॒व दे॑व॒यानः॑ । \newline
37. ए॒व दे॑व॒यानो॑ देव॒यान॑ ए॒वैव दे॑व॒यानः॒ पन्थाः॒ पन्था॑ देव॒यान॑ ए॒वैव दे॑व॒यानः॒ पन्थाः᳚ । \newline
38. दे॒व॒यानः॒ पन्थाः॒ पन्था॑ देव॒यानो॑ देव॒यानः॒ पन्था॒ स्तम् तम् पन्था॑ देव॒यानो॑ देव॒यानः॒ पन्था॒ स्तम् । \newline
39. दे॒व॒यान॒ इति॑ देव - यानः॑ । \newline
40. पन्था॒स्तम् तम् पन्थाः॒ पन्था॒स्तꣳ स॒मारो॑हति स॒मारो॑हति॒ तम् पन्थाः॒ पन्था॒स्तꣳ स॒मारो॑हति । \newline
41. तꣳ स॒मारो॑हति स॒मारो॑हति॒ तम् तꣳ स॒मारो॑ह त्ये॒ता वे॒तौ स॒मारो॑हति॒ तम् तꣳ स॒मारो॑ह त्ये॒तौ । \newline
42. स॒मारो॑ह त्ये॒ता वे॒तौ स॒मारो॑हति स॒मारो॑ह त्ये॒तौ वै वा ए॒तौ स॒मारो॑हति स॒मारो॑ह त्ये॒तौ वै । \newline
43. स॒मारो॑ह॒तीति॑ सं - आरो॑हति । \newline
44. ए॒तौ वै वा ए॒ता वे॒तौ वै दे॒वाना᳚म् दे॒वानां॒ ॅवा ए॒ता वे॒तौ वै दे॒वाना᳚म् । \newline
45. वै दे॒वाना᳚म् दे॒वानां॒ ॅवै वै दे॒वानाꣳ॒॒ हरी॒ हरी॑ दे॒वानां॒ ॅवै वै दे॒वानाꣳ॒॒ हरी᳚ । \newline
46. दे॒वानाꣳ॒॒ हरी॒ हरी॑ दे॒वाना᳚म् दे॒वानाꣳ॒॒ हरी॒ यद् यद्धरी॑ दे॒वाना᳚म् दे॒वानाꣳ॒॒ हरी॒ यत् । \newline
47. हरी॒ यद् यद्धरी॒ हरी॒ यद् द॑र्.शपूर्णमा॒सौ द॑र्.शपूर्णमा॒सौ यद्धरी॒ हरी॒ यद् द॑र्.शपूर्णमा॒सौ । \newline
48. हरी॒ इति॒ हरी᳚ । \newline
49. यद् द॑र्.शपूर्णमा॒सौ द॑र्.शपूर्णमा॒सौ यद् यद् द॑र्.शपूर्णमा॒सौ यो यो द॑र्.शपूर्णमा॒सौ यद् यद् द॑र्.शपूर्णमा॒सौ यः । \newline
50. द॒र्॒.श॒पू॒र्ण॒मा॒सौ यो यो द॑र्.शपूर्णमा॒सौ द॑र्.शपूर्णमा॒सौ य ए॒व मे॒वं ॅयो द॑र्.शपूर्णमा॒सौ द॑र्.शपूर्णमा॒सौ य ए॒वम् । \newline
51. द॒र्॒.श॒पू॒र्ण॒मा॒साविति॑ दर्.श - पू॒र्ण॒मा॒सौ । \newline
52. य ए॒व मे॒वं ॅयो य ए॒वं ॅवि॒द्वान्. वि॒द्वा ने॒वं ॅयो य ए॒वं ॅवि॒द्वान् । \newline
53. ए॒वं ॅवि॒द्वान्. वि॒द्वा ने॒व मे॒वं ॅवि॒द्वान् द॑र्.शपूर्णमा॒सौ द॑र्.शपूर्णमा॒सौ वि॒द्वा ने॒व मे॒वं ॅवि॒द्वान् द॑र्.शपूर्णमा॒सौ । \newline
54. वि॒द्वान् द॑र्.शपूर्णमा॒सौ द॑र्.शपूर्णमा॒सौ वि॒द्वान्. वि॒द्वान् द॑र्.शपूर्णमा॒सौ यज॑ते॒ यज॑ते दर्.शपूर्णमा॒सौ वि॒द्वान्. वि॒द्वान् द॑र्.शपूर्णमा॒सौ यज॑ते । \newline
55. द॒र्॒.श॒पू॒र्ण॒मा॒सौ यज॑ते॒ यज॑ते दर्.शपूर्णमा॒सौ द॑र्.शपूर्णमा॒सौ यज॑ते॒ यौ यौ यज॑ते दर्.शपूर्णमा॒सौ द॑र्.शपूर्णमा॒सौ यज॑ते॒ यौ । \newline
56. द॒र्॒.श॒पू॒र्ण॒मा॒साविति॑ दर्.श - पू॒र्ण॒मा॒सौ । \newline
57. यज॑ते॒ यौ यौ यज॑ते॒ यज॑ते॒ या वे॒वैव यौ यज॑ते॒ यज॑ते॒ या वे॒व । \newline
58. या वे॒वैव यौ या वे॒व दे॒वाना᳚म् दे॒वाना॑ मे॒व यौ या वे॒व दे॒वाना᳚म् । \newline
59. ए॒व दे॒वाना᳚म् दे॒वाना॑ मे॒वैव दे॒वानाꣳ॒॒ हरी॒ हरी॑ दे॒वाना॑ मे॒वैव दे॒वानाꣳ॒॒ हरी᳚ । \newline
60. दे॒वानाꣳ॒॒ हरी॒ हरी॑ दे॒वाना᳚म् दे॒वानाꣳ॒॒ हरी॒ ताभ्या॒म् ताभ्याꣳ॒॒ हरी॑ दे॒वाना᳚म् दे॒वानाꣳ॒॒ हरी॒ ताभ्या᳚म् । \newline
61. हरी॒ ताभ्या॒म् ताभ्याꣳ॒॒ हरी॒ हरी॒ ताभ्या॑ मे॒वैव ताभ्याꣳ॒॒ हरी॒ हरी॒ ताभ्या॑ मे॒व । \newline
62. हरी॒ इति॒ हरी᳚ । \newline
63. ताभ्या॑ मे॒वैव ताभ्या॒म् ताभ्या॑ मे॒वैभ्य॑ एभ्य ए॒व ताभ्या॒म् ताभ्या॑ मे॒वैभ्यः॑ । \newline
\pagebreak
\markright{ TS 2.5.6.3  \hfill https://www.vedavms.in \hfill}
\addcontentsline{toc}{section}{ TS 2.5.6.3 }
\section*{ TS 2.5.6.3 }

\textbf{TS 2.5.6.3 } \newline
\textbf{Samhita Paata} \newline

-मे॒वैभ्यो॑ ह॒व्यं ॅव॑हत्ये॒तद्वै दे॒वाना॑मा॒स्यं॑ ॅयद्-द॑र्.शपूर्णमा॒सौ य ए॒वं ॅवि॒द्वान् द॑र्.शपूर्णमा॒सौ यज॑ते सा॒क्षादे॒व दे॒वाना॑मा॒स्ये॑ जुहोत्ये॒ष वै ह॑विर्द्धा॒नी यो द॑र्.शपूर्णमासया॒जी सा॒यंप्रा॑तरग्निहो॒त्रं जु॑होति॒ यज॑ते दर्.शपूर्णमा॒सा-वह॑रहर्.-हविर्द्धा॒निनाꣳ॑ सु॒तो य ए॒वं ॅवि॒द्वान् द॑र्.शपूर्णमा॒सौ यज॑ते हविर्द्धा॒न्य॑स्मीति॒ सर्व॑मे॒वास्य॑ बर्.हि॒ष्यं॑ द॒त्तं भ॑वति दे॒वावा अह॑ - [  ] \newline

\textbf{Pada Paata} \newline

ए॒व । ए॒भ्यः॒ । ह॒व्यम् । व॒ह॒ति॒ । ए॒तत् । वै । दे॒वाना᳚म् । आ॒स्य᳚म् । यत् । द॒र्.॒श॒पू॒र्ण॒मा॒साविति॑ दर्.श-पू॒र्ण॒मा॒सौ । यः । ए॒वम् । वि॒द्वान् । द॒र्.॒श॒पू॒र्ण॒मा॒साविति॑ दर्.श - पू॒र्ण॒मा॒सौ । यज॑ते । सा॒क्षादिति॑ स - अ॒क्षात् । ए॒व । दे॒वाना᳚म् । आ॒स्ये᳚ । जु॒हो॒ति॒ । ए॒षः । वै । ह॒वि॒र्द्धा॒नीति॑ हविः - धा॒नी । यः । द॒र्.॒श॒पू॒र्ण॒मा॒स॒या॒जीति॑ दर्.शपूर्णमास - या॒जी । सा॒यंप्रा॑त॒रिति॑ सा॒यं - प्रा॒तः॒ । अ॒ग्नि॒हो॒त्रमित्य॑ग्नि - हो॒त्रम् । जु॒हो॒ति॒ । यज॑ते । द॒र्.॒श॒पू॒र्ण॒मा॒साविति॑ दर्.श - पू॒र्ण॒मा॒सौ । अह॑रह॒रित्यहः॑ - अ॒हः॒ । ह॒वि॒र्द्धा॒निना॒मिति॑ हविः - धा॒निना᳚म् । सु॒तः । यः । ए॒वम् । वि॒द्वान् । द॒र्.॒श॒पू॒र्ण॒मा॒साविति॑ दर्.श - पू॒र्ण॒मा॒सौ । यज॑ते । ह॒वि॒र्द्धा॒नीति॑ हविः - धा॒नी । अ॒स्मि॒ । इति॑ । सर्व᳚म् । ए॒व । अ॒स्य॒ । ब॒र्॒.हि॒ष्य᳚म् । द॒त्तम् । भ॒व॒ति॒ । दे॒वाः । वै । अहः॑ ।  \newline


\textbf{Krama Paata} \newline

ए॒वैभ्यः॑ । ए॒भ्यो॒ ह॒व्यम् । ह॒व्यम् ॅव॑हति । व॒ह॒त्ये॒तत् । ए॒तद् वै । वै दे॒वाना᳚म् । दे॒वाना॑मा॒स्य᳚म् । आ॒स्य॑म् ॅयत् । यद् द॑र्.शपूर्णमा॒सौ । द॒र्॒.श॒पू॒र्ण॒मा॒सौ यः । द॒र्॒.श॒पू॒र्ण॒मा॒साविति॑ दर्.श - पू॒र्ण॒मा॒सौ । य ए॒वम् । ए॒वम् ॅवि॒द्वान् । वि॒द्वान् द॑र्.शपूर्णमा॒सौ । द॒र्॒.श॒पू॒र्ण॒मा॒सौ यज॑ते । द॒र्॒.श॒पू॒र्ण॒मा॒साविति॑ दर्.श - पू॒र्ण॒मा॒सौ । यज॑ते सा॒क्षात् । सा॒क्षादे॒व । सा॒क्षादिति॑ स - अ॒क्षात् । ए॒व दे॒वाना᳚म् । दे॒वाना॑मा॒स्ये᳚ । आ॒स्ये॑ जुहोति । जु॒हो॒त्ये॒षः । ए॒ष वै । वै ह॑विर्द्धा॒नी । ह॒वि॒र्द्धा॒नी यः । ह॒वि॒र्द्धा॒नीति॑ हविः - धा॒नी । यो द॑र्.शपूर्णमासया॒जी । द॒र्॒.श॒पू॒र्ण॒मा॒स॒या॒जी सा॒यम्प्रा॑तः । द॒र्॒.श॒पू॒र्ण॒मा॒स॒या॒जीति॑ दर्.शपूर्णमास - या॒जी । सा॒यम्प्रा॑तरग्निहो॒त्रम् । सा॒यम्प्रा॑त॒रिति॑ सा॒यम् - प्रा॒तः॒ । अ॒ग्नि॒हो॒त्रम् जु॑होति । अ॒ग्नि॒हो॒त्रमित्य॑ग्नि - हो॒त्रम् । जु॒हो॒ति॒ यज॑ते । यज॑ते दर्.शपूर्णमा॒सौ । द॒र्॒.श॒पू॒र्ण॒मा॒सावह॑रहः । द॒र्॒.श॒पू॒र्ण॒मा॒साविति॑ दर्.श - पू॒र्ण॒मा॒सौ । अह॑रहर्. हविर्द्धा॒निना᳚म् । अह॑रह॒रित्यहः॑ - अ॒हः॒ । ह॒वि॒र्द्धा॒निनाꣳ॑ सु॒तः । ह॒वि॒र्द्धा॒निना॒मिति॑ हविः - धा॒निना᳚म् । सु॒तो यः । य ए॒वम् । ए॒वम् ॅवि॒द्वान् । वि॒द्वान् द॑र्.शपूर्णमा॒सौ । द॒र्॒.श॒पू॒र्ण॒मा॒सौ यज॑ते । द॒र्॒.श॒पू॒र्ण॒मा॒साविति॑ दर्.श - पू॒र्ण॒मा॒सौ । यज॑ते हविर्द्धा॒नी । ह॒वि॒र्द्धा॒न्य॑स्मि । ह॒वि॒र्द्धा॒नीति॑ हविः - धा॒नी । अ॒स्मीति॑ । इति॒ सर्व᳚म् । सर्व॑मे॒व । ए॒वास्य॑ । अ॒स्य॒ ब॒र्॒.हि॒ष्य᳚म् । ब॒र्॒.हि॒ष्य॑म् द॒त्तम् । द॒त्तम् भ॑वति । भ॒व॒ति॒ दे॒वाः । दे॒वा वै । वा अहः॑ । अह॑र् य॒ज्ञिय᳚म् \newline

\textbf{Jatai Paata} \newline

1. ए॒वैभ्य॑ एभ्य ए॒वैवैभ्यः॑ । \newline
2. ए॒भ्यो॒ ह॒व्यꣳ ह॒व्य मे᳚भ्य एभ्यो ह॒व्यम् । \newline
3. ह॒व्यं ॅव॑हति वहति ह॒व्यꣳ ह॒व्यं ॅव॑हति । \newline
4. व॒ह॒ त्ये॒त दे॒तद् व॑हति वह त्ये॒तत् । \newline
5. ए॒तद् वै वा ए॒त दे॒तद् वै । \newline
6. वै दे॒वाना᳚म् दे॒वानां॒ ॅवै वै दे॒वाना᳚म् । \newline
7. दे॒वाना॑ मा॒स्य॑ मा॒स्य॑म् दे॒वाना᳚म् दे॒वाना॑ मा॒स्य᳚म् । \newline
8. आ॒स्यं॑ ॅयद् यदा॒स्य॑ मा॒स्यं॑ ॅयत् । \newline
9. यद् द॑र्.शपूर्णमा॒सौ द॑र्.शपूर्णमा॒सौ यद् यद् द॑र्.शपूर्णमा॒सौ । \newline
10. द॒र्॒.श॒पू॒र्ण॒मा॒सौ यो यो द॑र्.शपूर्णमा॒सौ द॑र्.शपूर्णमा॒सौ यः । \newline
11. द॒र्॒.श॒पू॒र्ण॒मा॒साविति॑ दर्.श - पू॒र्ण॒मा॒सौ । \newline
12. य ए॒व मे॒वं ॅयो य ए॒वम् । \newline
13. ए॒वं ॅवि॒द्वान्. वि॒द्वा ने॒व मे॒वं ॅवि॒द्वान् । \newline
14. वि॒द्वान् द॑र्.शपूर्णमा॒सौ द॑र्.शपूर्णमा॒सौ वि॒द्वान्. वि॒द्वान् द॑र्.शपूर्णमा॒सौ । \newline
15. द॒र्॒.श॒पू॒र्ण॒मा॒सौ यज॑ते॒ यज॑ते दर्.शपूर्णमा॒सौ द॑र्.शपूर्णमा॒सौ यज॑ते । \newline
16. द॒र्॒.श॒पू॒र्ण॒मा॒साविति॑ दर्.श - पू॒र्ण॒मा॒सौ । \newline
17. यज॑ते सा॒क्षाथ् सा॒क्षाद् यज॑ते॒ यज॑ते सा॒क्षात् । \newline
18. सा॒क्षा दे॒वैव सा॒क्षाथ् सा॒क्षा दे॒व । \newline
19. सा॒क्षादिति॑ स - अ॒क्षात् । \newline
20. ए॒व दे॒वाना᳚म् दे॒वाना॑ मे॒वैव दे॒वाना᳚म् । \newline
21. दे॒वाना॑ मा॒स्य॑ आ॒स्ये॑ दे॒वाना᳚म् दे॒वाना॑ मा॒स्ये᳚ । \newline
22. आ॒स्ये॑ जुहोति जुहो त्या॒स्य॑ आ॒स्ये॑ जुहोति । \newline
23. जु॒हो॒ त्ये॒ष ए॒ष जु॑होति जुहो त्ये॒षः । \newline
24. ए॒ष वै वा ए॒ष ए॒ष वै । \newline
25. वै ह॑विर्द्धा॒नी ह॑विर्द्धा॒नी वै वै ह॑विर्द्धा॒नी । \newline
26. ह॒वि॒र्द्धा॒नी यो यो ह॑विर्द्धा॒नी ह॑विर्द्धा॒नी यः । \newline
27. ह॒वि॒र्द्धा॒नीति॑ हविः - धा॒नी । \newline
28. यो द॑र्.शपूर्णमासया॒जी द॑र्.शपूर्णमासया॒जी यो यो द॑र्.शपूर्णमासया॒जी । \newline
29. द॒र्॒.श॒पू॒र्ण॒मा॒स॒या॒जी सा॒यंप्रा॑तः सा॒यंप्रा॑तर् दर्.शपूर्णमासया॒जी द॑र्.शपूर्णमासया॒जी सा॒यंप्रा॑तः । \newline
30. द॒र्॒.श॒पू॒र्ण॒मा॒स॒या॒जीति॑ दर्.शपूर्णमास - या॒जी । \newline
31. सा॒यंप्रा॑त रग्निहो॒त्र म॑ग्निहो॒त्रꣳ सा॒यंप्रा॑तः सा॒यंप्रा॑त रग्निहो॒त्रम् । \newline
32. सा॒यंप्रा॑त॒रिति॑ सा॒यं - प्रा॒तः॒ । \newline
33. अ॒ग्नि॒हो॒त्रम् जु॑होति जुहोत्यग्निहो॒त्र म॑ग्निहो॒त्रम् जु॑होति । \newline
34. अ॒ग्नि॒हो॒त्रमित्य॑ग्नि - हो॒त्रम् । \newline
35. जु॒हो॒ति॒ यज॑ते॒ यज॑ते जुहोति जुहोति॒ यज॑ते । \newline
36. यज॑ते दर्.शपूर्णमा॒सौ द॑र्.शपूर्णमा॒सौ यज॑ते॒ यज॑ते दर्.शपूर्णमा॒सौ । \newline
37. द॒र्॒.श॒पू॒र्ण॒मा॒सा वह॑रह॒ रह॑रहर् दर्.शपूर्णमा॒सौ द॑र्.शपूर्णमा॒सा वह॑रहः । \newline
38. द॒र्॒.श॒पू॒र्ण॒मा॒साविति॑ दर्.श - पू॒र्ण॒मा॒सौ । \newline
39. अह॑रहर्. हविर्द्धा॒निनाꣳ॑ हविर्द्धा॒निना॒ मह॑रह॒ रह॑रहर्. हविर्द्धा॒निना᳚म् । \newline
40. अह॑रह॒रित्यहः॑ - अ॒हः॒ । \newline
41. ह॒वि॒र्द्धा॒निनाꣳ॑ सु॒तः सु॒तो ह॑विर्द्धा॒निनाꣳ॑ हविर्द्धा॒निनाꣳ॑ सु॒तः । \newline
42. ह॒वि॒र्द्धा॒निना॒मिति॑ हविः - धा॒निना᳚म् । \newline
43. सु॒तो यो यः सु॒तः सु॒तो यः । \newline
44. य ए॒व मे॒वं ॅयो य ए॒वम् । \newline
45. ए॒वं ॅवि॒द्वान्. वि॒द्वा ने॒व मे॒वं ॅवि॒द्वान् । \newline
46. वि॒द्वान् द॑र्.शपूर्णमा॒सौ द॑र्.शपूर्णमा॒सौ वि॒द्वान्. वि॒द्वान् द॑र्.शपूर्णमा॒सौ । \newline
47. द॒र्॒.श॒पू॒र्ण॒मा॒सौ यज॑ते॒ यज॑ते दर्.शपूर्णमा॒सौ द॑र्.शपूर्णमा॒सौ यज॑ते । \newline
48. द॒र्॒.श॒पू॒र्ण॒मा॒साविति॑ दर्.श - पू॒र्ण॒मा॒सौ । \newline
49. यज॑ते हविर्द्धा॒नी ह॑विर्द्धा॒नी यज॑ते॒ यज॑ते हविर्द्धा॒नी । \newline
50. ह॒वि॒र्द्धा॒ न्य॑स्म्यस्मि हविर्द्धा॒नी ह॑विर्द्धा॒ न्य॑स्मि । \newline
51. ह॒वि॒र्द्धा॒नीति॑ हविः - धा॒नी । \newline
52. अ॒स्मीती त्य॑स्म्य॒ स्मीति॑ । \newline
53. इति॒ सर्वꣳ॒॒ सर्व॒ मितीति॒ सर्व᳚म् । \newline
54. सर्व॑ मे॒वैव सर्वꣳ॒॒ सर्व॑ मे॒व । \newline
55. ए॒वास्या᳚ स्यै॒वैवास्य॑ । \newline
56. अ॒स्य॒ ब॒र्॒.हि॒ष्य॑म् बर्.हि॒ष्य॑ मस्यास्य बर्.हि॒ष्य᳚म् । \newline
57. ब॒र्॒.हि॒ष्य॑म् द॒त्तम् द॒त्तम् ब॑र्.हि॒ष्य॑म् बर्.हि॒ष्य॑म् द॒त्तम् । \newline
58. द॒त्तम् भ॑वति भवति द॒त्तम् द॒त्तम् भ॑वति । \newline
59. भ॒व॒ति॒ दे॒वा दे॒वा भ॑वति भवति दे॒वाः । \newline
60. दे॒वा वै वै दे॒वा दे॒वा वै । \newline
61. वा अह॒ रह॒र् वै वा अहः॑ । \newline
62. अह॑र् य॒ज्ञियं॑ ॅय॒ज्ञिय॒ मह॒ रह॑र् य॒ज्ञिय᳚म् । \newline

\textbf{Ghana Paata } \newline

1. ए॒वैभ्य॑ एभ्य ए॒वैवैभ्यो॑ ह॒व्यꣳ ह॒व्य मे᳚भ्य ए॒वैवैभ्यो॑ ह॒व्यम् । \newline
2. ए॒भ्यो॒ ह॒व्यꣳ ह॒व्य मे᳚भ्य एभ्यो ह॒व्यं ॅव॑हति वहति ह॒व्य मे᳚भ्य एभ्यो ह॒व्यं ॅव॑हति । \newline
3. ह॒व्यं ॅव॑हति वहति ह॒व्यꣳ ह॒व्यं ॅव॑ह त्ये॒त दे॒तद् व॑हति ह॒व्यꣳ ह॒व्यं ॅव॑ह त्ये॒तत् । \newline
4. व॒ह॒ त्ये॒त दे॒तद् व॑हति वह त्ये॒तद् वै वा ए॒तद् व॑हति वह त्ये॒तद् वै । \newline
5. ए॒तद् वै वा ए॒त दे॒तद् वै दे॒वाना᳚म् दे॒वानां॒ ॅवा ए॒त दे॒तद् वै दे॒वाना᳚म् । \newline
6. वै दे॒वाना᳚म् दे॒वानां॒ ॅवै वै दे॒वाना॑ मा॒स्य॑ मा॒स्य॑म् दे॒वानां॒ ॅवै वै दे॒वाना॑ मा॒स्य᳚म् । \newline
7. दे॒वाना॑ मा॒स्य॑ मा॒स्य॑म् दे॒वाना᳚म् दे॒वाना॑ मा॒स्यं॑ ॅयद् यदा॒स्य॑म् दे॒वाना᳚म् दे॒वाना॑ मा॒स्यं॑ ॅयत् । \newline
8. आ॒स्यं॑ ॅयद् यदा॒स्य॑ मा॒स्यं॑ ॅयद् द॑र्.शपूर्णमा॒सौ द॑र्.शपूर्णमा॒सौ यदा॒स्य॑ मा॒स्यं॑ ॅयद् द॑र्.शपूर्णमा॒सौ । \newline
9. यद् द॑र्.शपूर्णमा॒सौ द॑र्.शपूर्णमा॒सौ यद् यद् द॑र्.शपूर्णमा॒सौ यो यो द॑र्.शपूर्णमा॒सौ यद् यद् द॑र्.शपूर्णमा॒सौ यः । \newline
10. द॒र्॒.श॒पू॒र्ण॒मा॒सौ यो यो द॑र्.शपूर्णमा॒सौ द॑र्.शपूर्णमा॒सौ य ए॒व मे॒वं ॅयो द॑र्.शपूर्णमा॒सौ द॑र्.शपूर्णमा॒सौ य ए॒वम् । \newline
11. द॒र्॒.श॒पू॒र्ण॒मा॒साविति॑ दर्.श - पू॒र्ण॒मा॒सौ । \newline
12. य ए॒व मे॒वं ॅयो य ए॒वं ॅवि॒द्वान्. वि॒द्वा ने॒वं ॅयो य ए॒वं ॅवि॒द्वान् । \newline
13. ए॒वं ॅवि॒द्वान्. वि॒द्वा ने॒व मे॒वं ॅवि॒द्वान् द॑र्.शपूर्णमा॒सौ द॑र्.शपूर्णमा॒सौ वि॒द्वा ने॒व मे॒वं ॅवि॒द्वान् द॑र्.शपूर्णमा॒सौ । \newline
14. वि॒द्वान् द॑र्.शपूर्णमा॒सौ द॑र्.शपूर्णमा॒सौ वि॒द्वान्. वि॒द्वान् द॑र्.शपूर्णमा॒सौ यज॑ते॒ यज॑ते दर्.शपूर्णमा॒सौ वि॒द्वान्. वि॒द्वान् द॑र्.शपूर्णमा॒सौ यज॑ते । \newline
15. द॒र्॒.श॒पू॒र्ण॒मा॒सौ यज॑ते॒ यज॑ते दर्.शपूर्णमा॒सौ द॑र्.शपूर्णमा॒सौ यज॑ते सा॒क्षाथ् सा॒क्षाद् यज॑ते दर्.शपूर्णमा॒सौ द॑र्.शपूर्णमा॒सौ यज॑ते सा॒क्षात् । \newline
16. द॒र्॒.श॒पू॒र्ण॒मा॒साविति॑ दर्.श - पू॒र्ण॒मा॒सौ । \newline
17. यज॑ते सा॒क्षाथ् सा॒क्षाद् यज॑ते॒ यज॑ते सा॒क्षा दे॒वैव सा॒क्षाद् यज॑ते॒ यज॑ते सा॒क्षा दे॒व । \newline
18. सा॒क्षा दे॒वैव सा॒क्षाथ् सा॒क्षा दे॒व दे॒वाना᳚म् दे॒वाना॑ मे॒व सा॒क्षाथ् सा॒क्षा दे॒व दे॒वाना᳚म् । \newline
19. सा॒क्षादिति॑ स - अ॒क्षात् । \newline
20. ए॒व दे॒वाना᳚म् दे॒वाना॑ मे॒वैव दे॒वाना॑ मा॒स्य॑ आ॒स्ये॑ दे॒वाना॑ मे॒वैव दे॒वाना॑ मा॒स्ये᳚ । \newline
21. दे॒वाना॑ मा॒स्य॑ आ॒स्ये॑ दे॒वाना᳚म् दे॒वाना॑ मा॒स्ये॑ जुहोति जुहोत्या॒स्ये॑ दे॒वाना᳚म् दे॒वाना॑ मा॒स्ये॑ जुहोति । \newline
22. आ॒स्ये॑ जुहोति जुहोत्या॒स्य॑ आ॒स्ये॑ जुहोत्ये॒ष ए॒ष जु॑होत्या॒स्य॑ आ॒स्ये॑ जुहोत्ये॒षः । \newline
23. जु॒हो॒त्ये॒ष ए॒ष जु॑होति जुहोत्ये॒ष वै वा ए॒ष जु॑होति जुहोत्ये॒ष वै । \newline
24. ए॒ष वै वा ए॒ष ए॒ष वै ह॑विर्द्धा॒नी ह॑विर्द्धा॒नी वा ए॒ष ए॒ष वै ह॑विर्द्धा॒नी । \newline
25. वै ह॑विर्द्धा॒नी ह॑विर्द्धा॒नी वै वै ह॑विर्द्धा॒नी यो यो ह॑विर्द्धा॒नी वै वै ह॑विर्द्धा॒नी यः । \newline
26. ह॒वि॒र्द्धा॒नी यो यो ह॑विर्द्धा॒नी ह॑विर्द्धा॒नी यो द॑र्.शपूर्णमासया॒जी द॑र्.शपूर्णमासया॒जी यो ह॑विर्द्धा॒नी ह॑विर्द्धा॒नी यो द॑र्.शपूर्णमासया॒जी । \newline
27. ह॒वि॒र्द्धा॒नीति॑ हविः - धा॒नी । \newline
28. यो द॑र्.शपूर्णमासया॒जी द॑र्.शपूर्णमासया॒जी यो यो द॑र्.शपूर्णमासया॒जी सा॒यंप्रा॑तः सा॒यंप्रा॑तर् दर्.शपूर्णमासया॒जी यो यो द॑र्.शपूर्णमासया॒जी सा॒यंप्रा॑तः । \newline
29. द॒र्॒.श॒पू॒र्ण॒मा॒स॒या॒जी सा॒यंप्रा॑तः सा॒यंप्रा॑तर् दर्.शपूर्णमासया॒जी द॑र्.शपूर्णमासया॒जी सा॒यंप्रा॑त रग्निहो॒त्र म॑ग्निहो॒त्रꣳ सा॒यंप्रा॑तर् दर्.शपूर्णमासया॒जी द॑र्.शपूर्णमासया॒जी सा॒यंप्रा॑त रग्निहो॒त्रम् । \newline
30. द॒र्॒.श॒पू॒र्ण॒मा॒स॒या॒जीति॑ दर्.शपूर्णमास - या॒जी । \newline
31. सा॒यंप्रा॑त रग्निहो॒त्र म॑ग्निहो॒त्रꣳ सा॒यंप्रा॑तः सा॒यंप्रा॑त रग्निहो॒त्रम् जु॑होति जुहो त्यग्निहो॒त्रꣳ सा॒यंप्रा॑तः सा॒यंप्रा॑त रग्निहो॒त्रम् जु॑होति । \newline
32. सा॒यंप्रा॑त॒रिति॑ सा॒यं - प्रा॒तः॒ । \newline
33. अ॒ग्नि॒हो॒त्रम् जु॑होति जुहो त्यग्निहो॒त्र म॑ग्निहो॒त्रम् जु॑होति॒ यज॑ते॒ यज॑ते जुहो त्यग्निहो॒त्र म॑ग्निहो॒त्रम् जु॑होति॒ यज॑ते । \newline
34. अ॒ग्नि॒हो॒त्रमित्य॑ग्नि - हो॒त्रम् । \newline
35. जु॒हो॒ति॒ यज॑ते॒ यज॑ते जुहोति जुहोति॒ यज॑ते दर्.शपूर्णमा॒सौ द॑र्.शपूर्णमा॒सौ यज॑ते जुहोति जुहोति॒ यज॑ते दर्.शपूर्णमा॒सौ । \newline
36. यज॑ते दर्.शपूर्णमा॒सौ द॑र्.शपूर्णमा॒सौ यज॑ते॒ यज॑ते दर्.शपूर्णमा॒सा वह॑रह॒ रह॑रहर् दर्.शपूर्णमा॒सौ यज॑ते॒ यज॑ते दर्.शपूर्णमा॒सा वह॑रहः । \newline
37. द॒र्॒.श॒पू॒र्ण॒मा॒सा वह॑रह॒ रह॑रहर् दर्.शपूर्णमा॒सौ द॑र्.शपूर्णमा॒सा वह॑रहर्. हविर्द्धा॒निनाꣳ॑ हविर्द्धा॒निना॒ मह॑रहर् दर्.शपूर्णमा॒सौ द॑र्.शपूर्णमा॒सा वह॑रहर्. हविर्द्धा॒निना᳚म् । \newline
38. द॒र्॒.श॒पू॒र्ण॒मा॒साविति॑ दर्.श - पू॒र्ण॒मा॒सौ । \newline
39. अह॑रहर्. हविर्द्धा॒निनाꣳ॑ हविर्द्धा॒निना॒ मह॑रह॒ रह॑रहर्. हविर्द्धा॒निनाꣳ॑ सु॒तः सु॒तो ह॑विर्द्धा॒निना॒ मह॑रह॒ रह॑रहर्. हविर्द्धा॒निनाꣳ॑ सु॒तः । \newline
40. अह॑रह॒रित्यहः॑ - अ॒हः॒ । \newline
41. ह॒वि॒र्द्धा॒निनाꣳ॑ सु॒तः सु॒तो ह॑विर्द्धा॒निनाꣳ॑ हविर्द्धा॒निनाꣳ॑ सु॒तो यो यः सु॒तो ह॑विर्द्धा॒निनाꣳ॑ हविर्द्धा॒निनाꣳ॑ सु॒तो यः । \newline
42. ह॒वि॒र्द्धा॒निना॒मिति॑ हविः - धा॒निना᳚म् । \newline
43. सु॒तो यो यः सु॒तः सु॒तो य ए॒व मे॒वं ॅयः सु॒तः सु॒तो य ए॒वम् । \newline
44. य ए॒व मे॒वं ॅयो य ए॒वं ॅवि॒द्वान्. वि॒द्वा ने॒वं ॅयो य ए॒वं ॅवि॒द्वान् । \newline
45. ए॒वं ॅवि॒द्वान्. वि॒द्वा ने॒व मे॒वं ॅवि॒द्वान् द॑र्.शपूर्णमा॒सौ द॑र्.शपूर्णमा॒सौ वि॒द्वा ने॒व मे॒वं ॅवि॒द्वान् द॑र्.शपूर्णमा॒सौ । \newline
46. वि॒द्वान् द॑र्.शपूर्णमा॒सौ द॑र्.शपूर्णमा॒सौ वि॒द्वान्. वि॒द्वान् द॑र्.शपूर्णमा॒सौ यज॑ते॒ यज॑ते दर्.शपूर्णमा॒सौ वि॒द्वान्. वि॒द्वान् द॑र्.शपूर्णमा॒सौ यज॑ते । \newline
47. द॒र्॒.श॒पू॒र्ण॒मा॒सौ यज॑ते॒ यज॑ते दर्.शपूर्णमा॒सौ द॑र्.शपूर्णमा॒सौ यज॑ते हविर्द्धा॒नी ह॑विर्द्धा॒नी यज॑ते दर्.शपूर्णमा॒सौ द॑र्.शपूर्णमा॒सौ यज॑ते हविर्द्धा॒नी । \newline
48. द॒र्॒.श॒पू॒र्ण॒मा॒साविति॑ दर्.श - पू॒र्ण॒मा॒सौ । \newline
49. यज॑ते हविर्द्धा॒नी ह॑विर्द्धा॒नी यज॑ते॒ यज॑ते हविर्द्धा॒ न्य॑स्म्यस्मि हविर्द्धा॒नी यज॑ते॒ यज॑ते हविर्द्धा॒न्य॑स्मि । \newline
50. ह॒वि॒र्द्धा॒ न्य॑स्म्यस्मि हविर्द्धा॒नी ह॑विर्द्धा॒ न्य॑स्मीती त्य॑स्मि हविर्द्धा॒नी ह॑विर्द्धा॒ न्य॑स्मीति॑ । \newline
51. ह॒वि॒र्द्धा॒नीति॑ हविः - धा॒नी । \newline
52. अ॒स्मीती त्य॑स्म्य॒स्मीति॒ सर्वꣳ॒॒ सर्व॒ मित्य॑स्म्य॒स्मीति॒ सर्व᳚म् । \newline
53. इति॒ सर्वꣳ॒॒ सर्व॒ मितीति॒ सर्व॑ मे॒वैव सर्व॒ मितीति॒ सर्व॑ मे॒व । \newline
54. सर्व॑ मे॒वैव सर्वꣳ॒॒ सर्व॑ मे॒वास्या᳚स्यै॒व सर्वꣳ॒॒ सर्व॑ मे॒वास्य॑ । \newline
55. ए॒वास्या᳚ स्यै॒वैवास्य॑ बर्.हि॒ष्य॑म् बर्.हि॒ष्य॑ मस्यै॒वैवास्य॑ बर्.हि॒ष्य᳚म् । \newline
56. अ॒स्य॒ ब॒र्॒.हि॒ष्य॑म् बर्.हि॒ष्य॑ मस्यास्य बर्.हि॒ष्य॑म् द॒त्तम् द॒त्तम् ब॑र्.हि॒ष्य॑ मस्यास्य बर्.हि॒ष्य॑म् द॒त्तम् । \newline
57. ब॒र्॒.हि॒ष्य॑म् द॒त्तम् द॒त्तम् ब॑र्.हि॒ष्य॑म् बर्.हि॒ष्य॑म् द॒त्तम् भ॑वति भवति द॒त्तम् ब॑र्.हि॒ष्य॑म् बर्.हि॒ष्य॑म् द॒त्तम् भ॑वति । \newline
58. द॒त्तम् भ॑वति भवति द॒त्तम् द॒त्तम् भ॑वति दे॒वा दे॒वा भ॑वति द॒त्तम् द॒त्तम् भ॑वति दे॒वाः । \newline
59. भ॒व॒ति॒ दे॒वा दे॒वा भ॑वति भवति दे॒वा वै वै दे॒वा भ॑वति भवति दे॒वा वै । \newline
60. दे॒वा वै वै दे॒वा दे॒वा वा अह॒ रह॒र् वै दे॒वा दे॒वा वा अहः॑ । \newline
61. वा अह॒ रह॒र् वै वा अह॑र् य॒ज्ञियं॑ ॅय॒ज्ञिय॒ मह॒र् वै वा अह॑र् य॒ज्ञिय᳚म् । \newline
62. अह॑र् य॒ज्ञियं॑ ॅय॒ज्ञिय॒ मह॒ रह॑र् य॒ज्ञिय॒म् न न य॒ज्ञिय॒ मह॒ रह॑र् य॒ज्ञिय॒म् न । \newline
\pagebreak
\markright{ TS 2.5.6.4  \hfill https://www.vedavms.in \hfill}
\addcontentsline{toc}{section}{ TS 2.5.6.4 }
\section*{ TS 2.5.6.4 }

\textbf{TS 2.5.6.4 } \newline
\textbf{Samhita Paata} \newline

-र्य॒ज्ञियं॒ नावि॑न्द॒न् ते द॑र्.शपूर्णमा॒साव॑पुन॒न् तौ वा ए॒तौ पू॒तौ मेद्ध्यौ॒ यद्-द॑र्.शपूर्णमा॒सौ य ए॒वं ॅवि॒द्वान् द॑र्.शपूर्णमा॒सौ यज॑ते पू॒तावे॒वैनौ॒ मेद्ध्यौ॑ यजते॒ नामा॑वा॒स्या॑यां च पौर्णमा॒स्यां च॒ स्त्रिय-मुपे॑या॒द्य- दु॑पे॒यान्निरि॑न्द्रियः स्या॒त् सोम॑स्य॒ वै राज्ञो᳚ऽर्द्धमा॒सस्य॒ रात्र॑यः॒ पत्न॑य आस॒न् तासा॑ममावा॒स्यां᳚ च पौर्णमा॒सीं च॒ नोपै॒त् - [  ] \newline

\textbf{Pada Paata} \newline

य॒ज्ञिय᳚म् । न । अ॒वि॒न्द॒न्न् । ते । द॒र्.॒श॒पू॒र्ण॒मा॒साविति॑ दर्.श-पू॒र्ण॒मा॒सौ । अ॒पु॒न॒न्न् । तौ । वै । ए॒तौ । पू॒तौ । मेद्ध्यौ᳚ । यत् । द॒र्.॒श॒पू॒र्ण॒मा॒साविति॑ दर्.श - पू॒र्ण॒मा॒सौ । यः । ए॒वम् । वि॒द्वान् । द॒र्.॒श॒पू॒र्ण॒मा॒साविति॑ दर्.श-पू॒र्ण॒मा॒सौ । यज॑ते । पू॒तौ । ए॒व । ए॒नौ॒ । मेद्ध्यौ᳚ । य॒ज॒ते॒ । न । अ॒मा॒वा॒स्या॑या॒मित्य॑मा - वा॒स्या॑याम् । च॒ । पौ॒र्ण॒मा॒स्यामिति॑ पौर्ण - मा॒स्याम् । च॒ । स्त्रिय᳚म् । उपेति॑ । इ॒या॒त् । यत् । उ॒पे॒यादित्यु॑प - इ॒यात् । निरि॑न्द्रिय॒ इति॒ निः - इ॒न्द्रि॒यः॒ । स्या॒त् । सोम॑स्य । वै । राज्ञ्ः॑ । अ॒र्द्ध॒मा॒सस्येत्य॑र्द्ध - मा॒सस्य॑ । रात्र॑यः । पत्न॑यः । आ॒स॒न्न् । तासा᳚म् । अ॒मा॒वा॒स्या॑मित्य॑मा-वा॒स्या᳚म् । च॒ । पौ॒र्ण॒मा॒सीमिति॑ पौर्ण-मा॒सीम् । च॒ । न । उपेति॑ । ऐ॒त् ।  \newline


\textbf{Krama Paata} \newline

य॒ज्ञिय॒म् न । नावि॑न्दन्न् । अ॒वि॒न्द॒न् ते । ते द॑र्.शपूर्णमा॒सौ । द॒र्॒.श॒पू॒र्ण॒मा॒साव॑पुनन्न् । द॒र्॒.श॒पू॒र्ण॒मा॒साविति॑ दर्.श - पू॒र्ण॒मा॒सौ । अ॒पु॒न॒न् तौ । तौ वै । वा ए॒तौ । ए॒तौ पू॒तौ । पू॒तौ मेद्ध्यौ᳚ । मेद्ध्यौ॒ यत् । यद् द॑र्.शपूर्णमा॒सौ । द॒र्॒.श॒पू॒र्ण॒मा॒सौ यः । द॒र्॒.श॒पू॒र्ण॒मा॒साविति॑ दर्.श - पू॒र्ण॒मा॒सौ । य ए॒वम् । ए॒वम् ॅवि॒द्वान् । वि॒द्वान् द॑र्.शपूर्णमा॒सौ । द॒र्॒.श॒पू॒र्ण॒मा॒सौ यज॑ते । द॒र्॒.श॒पू॒र्ण॒मा॒साविति॑ दर्.श - पू॒र्ण॒मा॒सौ । यज॑ते पू॒तौ । पू॒तावे॒व । ए॒वैनौ᳚ । ए॒नौ॒ मेद्ध्यौ᳚ । मेद्ध्यौ॑ यजते । य॒ज॒ते॒ न । नामा॑वा॒स्या॑याम् । अ॒मा॒वा॒स्या॑याम् च । अ॒मा॒वा॒स्या॑या॒मित्य॑मा - वा॒स्या॑याम् । च॒ पौ॒र्ण॒मा॒स्याम् । पौ॒र्ण॒मा॒स्याम् च॑ । पौ॒र्ण॒मा॒स्यामिति॑ पौर्ण - मा॒स्याम् । च॒ स्त्रिय᳚म् । स्त्रिय॒मुप॑ । उपे॑यात् । इ॒या॒द् यत् । यदु॑पे॒यात् । उ॒पे॒यान् निरि॑न्द्रियः । उ॒पे॒यादित्यु॑प - इ॒यात् । निरि॑न्द्रियः स्यात् । निरि॑न्द्रिय॒ इति॒ निः - इ॒न्द्रि॒यः॒ । स्या॒थ् सोम॑स्य । सोम॑स्य॒ वै । वै राज्ञ्ः॑ । राज्ञो᳚ ऽर्द्धमा॒सस्य॑ । अ॒र्द्ध॒मा॒सस्य॒ रात्र॑यः । अ॒र्द्ध॒मा॒सस्येत्य॑र्द्ध - मा॒सस्य॑ । रात्र॑यः॒ पत्न॑यः । पत्न॑य आसन्न् । आ॒स॒न् तासा᳚म् । तासा॑ममावा॒स्या᳚म् । अ॒मा॒वा॒स्या᳚म् च । अ॒मा॒वा॒स्या॑मित्य॑मा - वा॒स्या᳚म् । च॒ पौ॒र्ण॒मा॒सीम् । पौ॒र्ण॒मा॒सीम् च॑ । पौ॒र्ण॒मा॒सीमिति॑ पौर्ण - मा॒सीम् । च॒ न । नोप॑ । उपै᳚त् । ऐ॒त् ते \newline

\textbf{Jatai Paata} \newline

1. य॒ज्ञिय॒म् न न य॒ज्ञियं॑ ॅय॒ज्ञिय॒म् न । \newline
2. नावि॑न्दन् नविन्द॒न् न नावि॑न्दन्न् । \newline
3. अ॒वि॒न्द॒न् ते ते॑ ऽविन्दन् नविन्द॒न् ते । \newline
4. ते द॑र्.शपूर्णमा॒सौ द॑र्.शपूर्णमा॒सौ ते ते द॑र्.शपूर्णमा॒सौ । \newline
5. द॒र्॒.श॒पू॒र्ण॒मा॒सा व॑पुनन् नपुनन् दर्.शपूर्णमा॒सौ द॑र्.शपूर्णमा॒सा व॑पुनन्न् । \newline
6. द॒र्॒.श॒पू॒र्ण॒मा॒साविति॑ दर्.श - पू॒र्ण॒मा॒सौ । \newline
7. अ॒पु॒न॒न् तौ ता व॑पुनन् नपुन॒न् तौ । \newline
8. तौ वै वै तौ तौ वै । \newline
9. वा ए॒ता वे॒तौ वै वा ए॒तौ । \newline
10. ए॒तौ पू॒तौ पू॒ता वे॒ता वे॒तौ पू॒तौ । \newline
11. पू॒तौ मेद्ध्यौ॒ मेद्ध्यौ॑ पू॒तौ पू॒तौ मेद्ध्यौ᳚ । \newline
12. मेद्ध्यौ॒ यद् यन् मेद्ध्यौ॒ मेद्ध्यौ॒ यत् । \newline
13. यद् द॑र्.शपूर्णमा॒सौ द॑र्.शपूर्णमा॒सौ यद् यद् द॑र्.शपूर्णमा॒सौ । \newline
14. द॒र्॒.श॒पू॒र्ण॒मा॒सौ यो यो द॑र्.शपूर्णमा॒सौ द॑र्.शपूर्णमा॒सौ यः । \newline
15. द॒र्॒.श॒पू॒र्ण॒मा॒साविति॑ दर्.श - पू॒र्ण॒मा॒सौ । \newline
16. य ए॒व मे॒वं ॅयो य ए॒वम् । \newline
17. ए॒वं ॅवि॒द्वान्. वि॒द्वा ने॒व मे॒वं ॅवि॒द्वान् । \newline
18. वि॒द्वान् द॑र्.शपूर्णमा॒सौ द॑र्.शपूर्णमा॒सौ वि॒द्वान्. वि॒द्वान् द॑र्.शपूर्णमा॒सौ । \newline
19. द॒र्॒.श॒पू॒र्ण॒मा॒सौ यज॑ते॒ यज॑ते दर्.शपूर्णमा॒सौ द॑र्.शपूर्णमा॒सौ यज॑ते । \newline
20. द॒र्॒.श॒पू॒र्ण॒मा॒साविति॑ दर्.श - पू॒र्ण॒मा॒सौ । \newline
21. यज॑ते पू॒तौ पू॒तौ यज॑ते॒ यज॑ते पू॒तौ । \newline
22. पू॒ता वे॒वैव पू॒तौ पू॒ता वे॒व । \newline
23. ए॒वैना॑ वेना वे॒वैवैनौ᳚ । \newline
24. ए॒नौ॒ मेद्ध्यौ॒ मेद्ध्या॑ वेना वेनौ॒ मेद्ध्यौ᳚ । \newline
25. मेद्ध्यौ॑ यजते यजते॒ मेद्ध्यौ॒ मेद्ध्यौ॑ यजते । \newline
26. य॒ज॒ते॒ न न य॑जते यजते॒ न । \newline
27. नामा॑वा॒स्या॑या ममावा॒स्या॑या॒न्न नामा॑वा॒स्या॑याम् । \newline
28. अ॒मा॒वा॒स्या॑याम् च चामावा॒स्या॑या ममावा॒स्या॑याम् च । \newline
29. अ॒मा॒वा॒स्या॑या॒मित्य॑मा - वा॒स्या॑याम् । \newline
30. च॒ पौ॒र्ण॒मा॒स्याम् पौ᳚र्णमा॒स्याम् च॑ च पौर्णमा॒स्याम् । \newline
31. पौ॒र्ण॒मा॒स्याम् च॑ च पौर्णमा॒स्याम् पौ᳚र्णमा॒स्याम् च॑ । \newline
32. पौ॒र्ण॒मा॒स्यामिति॑ पौर्ण - मा॒स्याम् । \newline
33. च॒ स्त्रियꣳ॒॒ स्त्रिय॑म् च च॒ स्त्रिय᳚म् । \newline
34. स्त्रिय॒ मुपोप॒ स्त्रियꣳ॒॒ स्त्रिय॒ मुप॑ । \newline
35. उपे॑ यादिया॒ दुपोपे॑ यात् । \newline
36. इ॒या॒द् यद् यदि॑या दिया॒द् यत् । \newline
37. यदु॑पे॒या दु॑पे॒याद् यद् यदु॑पे॒यात् । \newline
38. उ॒पे॒यान् निरि॑न्द्रियो॒ निरि॑न्द्रिय उपे॒या दु॑पे॒यान् निरि॑न्द्रियः । \newline
39. उ॒पे॒यादित्यु॑प - इ॒यात् । \newline
40. निरि॑न्द्रियः स्याथ् स्या॒न् निरि॑न्द्रियो॒ निरि॑न्द्रियः स्यात् । \newline
41. निरि॑न्द्रिय॒ इति॒ निः - इ॒न्द्रि॒यः॒ । \newline
42. स्या॒थ् सोम॑स्य॒ सोम॑स्य स्याथ् स्या॒थ् सोम॑स्य । \newline
43. सोम॑स्य॒ वै वै सोम॑स्य॒ सोम॑स्य॒ वै । \newline
44. वै राज्ञो॒ राज्ञो॒ वै वै राज्ञ्ः॑ । \newline
45. राज्ञो᳚ ऽर्द्धमा॒सस्या᳚ र्द्धमा॒सस्य॒ राज्ञो॒ राज्ञो᳚ ऽर्द्धमा॒सस्य॑ । \newline
46. अ॒र्द्ध॒मा॒सस्य॒ रात्र॑यो॒ रात्र॑यो ऽर्द्धमा॒सस्या᳚ र्द्धमा॒सस्य॒ रात्र॑यः । \newline
47. अ॒र्द्ध॒मा॒सस्येत्य॑र्द्ध - मा॒सस्य॑ । \newline
48. रात्र॑यः॒ पत्न॑यः॒ पत्न॑यो॒ रात्र॑यो॒ रात्र॑यः॒ पत्न॑यः । \newline
49. पत्न॑य आसन् नास॒न् पत्न॑यः॒ पत्न॑य आसन्न् । \newline
50. आ॒स॒न् तासा॒म् तासा॑ मासन् नास॒न् तासा᳚म् । \newline
51. तासा॑ ममावा॒स्या॑ ममावा॒स्या᳚म् तासा॒म् तासा॑ ममावा॒स्या᳚म् । \newline
52. अ॒मा॒वा॒स्या᳚म् च चामावा॒स्या॑ ममावा॒स्या᳚म् च । \newline
53. अ॒मा॒वा॒स्या॑मित्य॑मा - वा॒स्या᳚म् । \newline
54. च॒ पौ॒र्ण॒मा॒सीम् पौ᳚र्णमा॒सीम् च॑ च पौर्णमा॒सीम् । \newline
55. पौ॒र्ण॒मा॒सीम् च॑ च पौर्णमा॒सीम् पौ᳚र्णमा॒सीम् च॑ । \newline
56. पौ॒र्ण॒मा॒सीमिति॑ पौर्ण - मा॒सीम् । \newline
57. च॒ न न च॑ च॒ न । \newline
58. नोपोप॒ न नोप॑ । \newline
59. उपै॑ दै॒दुपो पै᳚त् । \newline
60. ऐ॒त् ते ते ऐ॑दै॒त् ते । \newline

\textbf{Ghana Paata } \newline

1. य॒ज्ञिय॒म् न न य॒ज्ञियं॑ ॅय॒ज्ञिय॒म् नावि॑न्दन् नविन्द॒न् न य॒ज्ञियं॑ ॅय॒ज्ञिय॒म् नावि॑न्दन्न् । \newline
2. नावि॑न्दन् नविन्द॒न् न नावि॑न्द॒न् ते ते॑ ऽविन्द॒न् न नावि॑न्द॒न् ते । \newline
3. अ॒वि॒न्द॒न् ते ते॑ ऽविन्दन् नविन्द॒न् ते द॑र्.शपूर्णमा॒सौ द॑र्.शपूर्णमा॒सौ ते॑ ऽविन्दन् नविन्द॒न् ते द॑र्.शपूर्णमा॒सौ । \newline
4. ते द॑र्.शपूर्णमा॒सौ द॑र्.शपूर्णमा॒सौ ते ते द॑र्.शपूर्णमा॒सा व॑पुनन् नपुनन् दर्.शपूर्णमा॒सौ ते ते द॑र्.शपूर्णमा॒सा व॑पुनन्न् । \newline
5. द॒र्॒.श॒पू॒र्ण॒मा॒सा व॑पुनन् नपुनन् दर्.शपूर्णमा॒सौ द॑र्.शपूर्णमा॒सा व॑पुन॒न् तौ ता व॑पुनन् दर्.शपूर्णमा॒सौ द॑र्.शपूर्णमा॒सा व॑पुन॒न् तौ । \newline
6. द॒र्॒.श॒पू॒र्ण॒मा॒साविति॑ दर्.श - पू॒र्ण॒मा॒सौ । \newline
7. अ॒पु॒न॒न् तौ ता व॑पुनन् नपुन॒न् तौ वै वै ता व॑पुनन् नपुन॒न् तौ वै । \newline
8. तौ वै वै तौ तौ वा ए॒ता वे॒तौ वै तौ तौ वा ए॒तौ । \newline
9. वा ए॒ता वे॒तौ वै वा ए॒तौ पू॒तौ पू॒ता वे॒तौ वै वा ए॒तौ पू॒तौ । \newline
10. ए॒तौ पू॒तौ पू॒ता वे॒ता वे॒तौ पू॒तौ मेद्ध्यौ॒ मेद्ध्यौ॑ पू॒ता वे॒ता वे॒तौ पू॒तौ मेद्ध्यौ᳚ । \newline
11. पू॒तौ मेद्ध्यौ॒ मेद्ध्यौ॑ पू॒तौ पू॒तौ मेद्ध्यौ॒ यद् यन् मेद्ध्यौ॑ पू॒तौ पू॒तौ मेद्ध्यौ॒ यत् । \newline
12. मेद्ध्यौ॒ यद् यन् मेद्ध्यौ॒ मेद्ध्यौ॒ यद् द॑र्.शपूर्णमा॒सौ द॑र्.शपूर्णमा॒सौ यन् मेद्ध्यौ॒ मेद्ध्यौ॒ यद् द॑र्.शपूर्णमा॒सौ । \newline
13. यद् द॑र्.शपूर्णमा॒सौ द॑र्.शपूर्णमा॒सौ यद् यद् द॑र्.शपूर्णमा॒सौ यो यो द॑र्.शपूर्णमा॒सौ यद् यद् द॑र्.शपूर्णमा॒सौ यः । \newline
14. द॒र्॒.श॒पू॒र्ण॒मा॒सौ यो यो द॑र्.शपूर्णमा॒सौ द॑र्.शपूर्णमा॒सौ य ए॒व मे॒वं ॅयो द॑र्.शपूर्णमा॒सौ द॑र्.शपूर्णमा॒सौ य ए॒वम् । \newline
15. द॒र्॒.श॒पू॒र्ण॒मा॒साविति॑ दर्.श - पू॒र्ण॒मा॒सौ । \newline
16. य ए॒व मे॒वं ॅयो य ए॒वं ॅवि॒द्वान्. वि॒द्वा ने॒वं ॅयो य ए॒वं ॅवि॒द्वान् । \newline
17. ए॒वं ॅवि॒द्वान्. वि॒द्वा ने॒व मे॒वं ॅवि॒द्वान् द॑र्.शपूर्णमा॒सौ द॑र्.शपूर्णमा॒सौ वि॒द्वा ने॒व मे॒वं ॅवि॒द्वान् द॑र्.शपूर्णमा॒सौ । \newline
18. वि॒द्वान् द॑र्.शपूर्णमा॒सौ द॑र्.शपूर्णमा॒सौ वि॒द्वान्. वि॒द्वान् द॑र्.शपूर्णमा॒सौ यज॑ते॒ यज॑ते दर्.शपूर्णमा॒सौ वि॒द्वान्. वि॒द्वान् द॑र्.शपूर्णमा॒सौ यज॑ते । \newline
19. द॒र्॒.श॒पू॒र्ण॒मा॒सौ यज॑ते॒ यज॑ते दर्.शपूर्णमा॒सौ द॑र्.शपूर्णमा॒सौ यज॑ते पू॒तौ पू॒तौ यज॑ते दर्.शपूर्णमा॒सौ द॑र्.शपूर्णमा॒सौ यज॑ते पू॒तौ । \newline
20. द॒र्॒.श॒पू॒र्ण॒मा॒साविति॑ दर्.श - पू॒र्ण॒मा॒सौ । \newline
21. यज॑ते पू॒तौ पू॒तौ यज॑ते॒ यज॑ते पू॒ता वे॒वैव पू॒तौ यज॑ते॒ यज॑ते पू॒ता वे॒व । \newline
22. पू॒ता वे॒वैव पू॒तौ पू॒ता वे॒वैना॑ वेना वे॒व पू॒तौ पू॒ता वे॒वैनौ᳚ । \newline
23. ए॒वैना॑ वेना वे॒वैवैनौ॒ मेद्ध्यौ॒ मेद्ध्या॑ वेना वे॒वैवैनौ॒ मेद्ध्यौ᳚ । \newline
24. ए॒नौ॒ मेद्ध्यौ॒ मेद्ध्या॑ वेना वेनौ॒ मेद्ध्यौ॑ यजते यजते॒ मेद्ध्या॑ वेना वेनौ॒ मेद्ध्यौ॑ यजते । \newline
25. मेद्ध्यौ॑ यजते यजते॒ मेद्ध्यौ॒ मेद्ध्यौ॑ यजते॒ न न य॑जते॒ मेद्ध्यौ॒ मेद्ध्यौ॑ यजते॒ न । \newline
26. य॒ज॒ते॒ न न य॑जते यजते॒ नामा॑वा॒स्या॑या ममावा॒स्या॑या॒म् न य॑जते यजते॒ नामा॑वा॒स्या॑याम् । \newline
27. नामा॑वा॒स्या॑या ममावा॒स्या॑या॒म् न नामा॑वा॒स्या॑याम् च चामावा॒स्या॑या॒म् न नामा॑वा॒स्या॑याम् च । \newline
28. अ॒मा॒वा॒स्या॑याम् च चामावा॒स्या॑या ममावा॒स्या॑याम् च पौर्णमा॒स्याम् पौ᳚र्णमा॒स्याम् चा॑मावा॒स्या॑या ममावा॒स्या॑याम् च पौर्णमा॒स्याम् । \newline
29. अ॒मा॒वा॒स्या॑या॒मित्य॑मा - वा॒स्या॑याम् । \newline
30. च॒ पौ॒र्ण॒मा॒स्याम् पौ᳚र्णमा॒स्याम् च॑ च पौर्णमा॒स्याम् च॑ च पौर्णमा॒स्याम् च॑ च पौर्णमा॒स्याम् च॑ । \newline
31. पौ॒र्ण॒मा॒स्याम् च॑ च पौर्णमा॒स्याम् पौ᳚र्णमा॒स्याम् च॒ स्त्रियꣳ॒॒ स्त्रिय॑म् च पौर्णमा॒स्याम् पौ᳚र्णमा॒स्याम् च॒ स्त्रिय᳚म् । \newline
32. पौ॒र्ण॒मा॒स्यामिति॑ पौर्ण - मा॒स्याम् । \newline
33. च॒ स्त्रियꣳ॒॒ स्त्रिय॑म् च च॒ स्त्रिय॒ मुपोप॒ स्त्रिय॑म् च च॒ स्त्रिय॒ मुप॑ । \newline
34. स्त्रिय॒ मुपोप॒ स्त्रियꣳ॒॒ स्त्रिय॒ मुपे॑ यादिया॒ दुप॒ स्त्रियꣳ॒॒ स्त्रिय॒ मुपे॑ यात् । \newline
35. उपे॑ यादिया॒ दुपोपे॑ या॒द् यद् यदि॑या॒ दुपोपे॑ या॒द् यत् । \newline
36. इ॒या॒द् यद् यदि॑या दिया॒द् यदु॑पे॒या दु॑पे॒याद् यदि॑या दिया॒द् यदु॑पे॒यात् । \newline
37. यदु॑पे॒या दु॑पे॒याद् यद् यदु॑पे॒यान् निरि॑न्द्रियो॒ निरि॑न्द्रिय उपे॒याद् यद् यदु॑पे॒यान् निरि॑न्द्रियः । \newline
38. उ॒पे॒यान् निरि॑न्द्रियो॒ निरि॑न्द्रिय उपे॒या दु॑पे॒यान् निरि॑न्द्रियः स्याथ् स्या॒न् निरि॑न्द्रिय उपे॒या दु॑पे॒यान् निरि॑न्द्रियः स्यात् । \newline
39. उ॒पे॒यादित्यु॑प - इ॒यात् । \newline
40. निरि॑न्द्रियः स्याथ् स्या॒न् निरि॑न्द्रियो॒ निरि॑न्द्रियः स्या॒थ् सोम॑स्य॒ सोम॑स्य स्या॒न् निरि॑न्द्रियो॒ निरि॑न्द्रियः स्या॒थ् सोम॑स्य । \newline
41. निरि॑न्द्रिय॒ इति॒ निः - इ॒न्द्रि॒यः॒ । \newline
42. स्या॒थ् सोम॑स्य॒ सोम॑स्य स्याथ् स्या॒थ् सोम॑स्य॒ वै वै सोम॑स्य स्याथ् स्या॒थ् सोम॑स्य॒ वै । \newline
43. सोम॑स्य॒ वै वै सोम॑स्य॒ सोम॑स्य॒ वै राज्ञो॒ राज्ञो॒ वै सोम॑स्य॒ सोम॑स्य॒ वै राज्ञ्ः॑ । \newline
44. वै राज्ञो॒ राज्ञो॒ वै वै राज्ञो᳚ ऽर्द्धमा॒सस्या᳚ र्द्धमा॒सस्य॒ राज्ञो॒ वै वै राज्ञो᳚ ऽर्द्धमा॒सस्य॑ । \newline
45. राज्ञो᳚ ऽर्द्धमा॒सस्या᳚ र्द्धमा॒सस्य॒ राज्ञो॒ राज्ञो᳚ ऽर्द्धमा॒सस्य॒ रात्र॑यो॒ रात्र॑यो ऽर्द्धमा॒सस्य॒ राज्ञो॒ राज्ञो᳚ ऽर्द्धमा॒सस्य॒ रात्र॑यः । \newline
46. अ॒र्द्ध॒मा॒सस्य॒ रात्र॑यो॒ रात्र॑यो ऽर्द्धमा॒सस्या᳚ र्द्धमा॒सस्य॒ रात्र॑यः॒ पत्न॑यः॒ पत्न॑यो॒ रात्र॑यो ऽर्द्धमा॒सस्या᳚र्द्धमा॒सस्य॒ रात्र॑यः॒ पत्न॑यः । \newline
47. अ॒र्द्ध॒मा॒सस्येत्य॑र्द्ध - मा॒सस्य॑ । \newline
48. रात्र॑यः॒ पत्न॑यः॒ पत्न॑यो॒ रात्र॑यो॒ रात्र॑यः॒ पत्न॑य आसन् नास॒न् पत्न॑यो॒ रात्र॑यो॒ रात्र॑यः॒ पत्न॑य आसन्न् । \newline
49. पत्न॑य आसन् नास॒न् पत्न॑यः॒ पत्न॑य आस॒न् तासा॒म् तासा॑ मास॒न् पत्न॑यः॒ पत्न॑य आस॒न् तासा᳚म् । \newline
50. आ॒स॒न् तासा॒म् तासा॑ मासन् नास॒न् तासा॑ ममावा॒स्या॑ ममावा॒स्या᳚म् तासा॑ मासन् नास॒न् तासा॑ ममावा॒स्या᳚म् । \newline
51. तासा॑ ममावा॒स्या॑ ममावा॒स्या᳚म् तासा॒म् तासा॑ ममावा॒स्या᳚म् च चामावा॒स्या᳚म् तासा॒म् तासा॑ ममावा॒स्या᳚म् च । \newline
52. अ॒मा॒वा॒स्या᳚म् च चामावा॒स्या॑ ममावा॒स्या᳚म् च पौर्णमा॒सीम् पौ᳚र्णमा॒सीम् चा॑मावा॒स्या॑ ममावा॒स्या᳚म् च पौर्णमा॒सीम् । \newline
53. अ॒मा॒वा॒स्या॑मित्य॑मा - वा॒स्या᳚म् । \newline
54. च॒ पौ॒र्ण॒मा॒सीम् पौ᳚र्णमा॒सीम् च॑ च पौर्णमा॒सीम् च॑ च पौर्णमा॒सीम् च॑ च पौर्णमा॒सीम् च॑ । \newline
55. पौ॒र्ण॒मा॒सीम् च॑ च पौर्णमा॒सीम् पौ᳚र्णमा॒सीम् च॒ न न च॑ पौर्णमा॒सीम् पौ᳚र्णमा॒सीम् च॒ न । \newline
56. पौ॒र्ण॒मा॒सीमिति॑ पौर्ण - मा॒सीम् । \newline
57. च॒ न न च॑ च॒ नोपोप॒ न च॑ च॒ नोप॑ । \newline
58. नोपोप॒ न नोपै॑ दै॒दुप॒ न नोपै᳚त् । \newline
59. उपै॑ दै॒ दुपोपै॒त् ते ते ऐ॒दुपोपै॒त् ते । \newline
60. ऐ॒त् ते ते ऐ॑दै॒त् ते ए॑न मेन॒म् ते ऐ॑दै॒त् ते ए॑नम् । \newline
\pagebreak
\markright{ TS 2.5.6.5  \hfill https://www.vedavms.in \hfill}
\addcontentsline{toc}{section}{ TS 2.5.6.5 }
\section*{ TS 2.5.6.5 }

\textbf{TS 2.5.6.5 } \newline
\textbf{Samhita Paata} \newline

ते ए॑नम॒भि सम॑नह्येतां॒ तं ॅयक्ष्म॑ आर्च्छ॒द्-राजा॑नं॒ ॅयक्ष्म॑ आर॒दिति॒ तद्-रा॑जय॒क्ष्मस्य॒ जन्म॒ यत् पापी॑या॒नभ॑व॒त् तत् पा॑पय॒क्ष्मस्य॒ यज्जा॒याभ्या॒मवि॑न्द॒त् तज्जा॒येन्य॑स्य॒ य ए॒वमे॒तेषां॒ ॅयक्ष्मा॑णां॒ जन्म॒ वेद॒ नैन॑मे॒ते यक्ष्मा॑विन्दन्ति॒ स ए॒ते ए॒व न॑म॒स्यन्नुपा॑धाव॒त् ते अ॑ब्रूतां॒ ॅवरं॑ ॅवृणावहा आ॒वं दे॒वानां᳚ भाग॒धे अ॑सावा॒ - [  ] \newline

\textbf{Pada Paata} \newline

ते इति॑ । ए॒न॒म् । अ॒भि । समिति॑ । अ॒न॒ह्ये॒ता॒म् । तम् । यक्ष्मः॑ । आ॒र्च्छ॒त् । राजा॑नम् । यक्ष्मः॑ । आ॒र॒त् । इति॑ । तत् । रा॒ज॒य॒क्ष्मस्येति॑ राज - य॒क्ष्मस्य॑ । जन्म॑ । यत् । पापी॑यान् । अभ॑वत् । तत् । पा॒प॒य॒क्ष्मस्येति॑ पाप - य॒क्ष्मस्य॑ । यत् । जा॒याभ्या᳚म् । अवि॑न्दत् । तत् । जा॒येन्य॑स्य । यः । ए॒वम् । ए॒तेषा᳚म् । यक्ष्मा॑णाम् । जन्म॑ । वेद॑ । न । ए॒न॒म् । ए॒ते । यक्ष्माः᳚ । वि॒न्द॒न्ति॒ । सः । ए॒ते इति॑ । ए॒व । न॒म॒स्यन्न् । उपेति॑ । अ॒धा॒व॒त् । ते इति॑ । अ॒ब्रू॒ता॒म् । वर᳚म् । वृ॒णा॒व॒है॒ । आ॒वम् । दे॒वाना᳚म् । भा॒ग॒धे इति॑ भाग - धे । अ॒सा॒व॒ ।  \newline


\textbf{Krama Paata} \newline

ते ए॑नम् । ते इति॒ ते । ए॒न॒म॒भि । अ॒भि सम् । सम॑नह्येताम् । अ॒न॒ह्ये॒ता॒म् तम् । तम् ॅयक्ष्मः॑ । यक्ष्म॑ आर्च्छत् । आ॒र्च्छ॒द् राजा॑नम् । राजा॑न॒म् ॅयक्ष्मः॑ । यक्ष्म॑ आरत् । आ॒र॒दिति॑ । इति॒ तत् । तद् रा॑जय॒क्ष्मस्य॑ । रा॒ज॒य॒क्ष्मस्य॒ जन्म॑ । रा॒ज॒य॒क्ष्मस्येति॑ राज - य॒क्ष्मस्य॑ । जन्म॒ यत् । यत् पापी॑यान् । पापी॑या॒नभ॑वत् । अभ॑व॒त् तत् । तत् पा॑पय॒क्ष्मस्य॑ । पा॒प॒य॒क्ष्मस्य॒ यत् । पा॒प॒य॒क्ष्मस्येति॑ पाप - य॒क्ष्मस्य॑ । यज्जा॒याभ्या᳚म् । जा॒याभ्या॒मवि॑न्दत् । अवि॑न्द॒त् तत् । तज्जा॒येन्य॑स्य । जा॒येन्य॑स्य॒ यः । य ए॒वम् । ए॒वमे॒तेषा᳚म् । ए॒तेषा॒म् ॅयक्ष्मा॑णाम् । यक्ष्मा॑णा॒म् जन्म॑ । जन्म॒ वेद॑ । वेद॒ न । नैन᳚म् । ए॒न॒मे॒ते । ए॒ते यक्ष्माः᳚ । यक्ष्मा॑ विन्दन्ति । वि॒न्द॒न्ति॒ सः । स ए॒ते । ए॒ते ए॒व । ए॒ते इत्ये॒ते । ए॒व न॑म॒स्यन्न् । न॒म॒स्यन्नुप॑ । उपा॑धावत् । अ॒धा॒व॒त् ते । ते अ॑ब्रूताम् । ते इति॒ ते । अ॒ब्रू॒ता॒म् ॅवर᳚म् । वर॑म् ॅवृणावहै । वृ॒णा॒व॒हा॒ आ॒वम् । आ॒वम् दे॒वाना᳚म् । दे॒वाना᳚म् भाग॒धे । भा॒ग॒धे अ॑साव । भा॒ग॒धे इति॑ भाग - धे । अ॒सा॒वा॒वत् \newline

\textbf{Jatai Paata} \newline

1. ते ए॑न मेन॒म् ते ते ए॑नम् । \newline
2. ते इति॒ ते । \newline
3. ए॒न॒ म॒भ्या᳚(1॒)भ्ये॑न मेन म॒भि । \newline
4. अ॒भि सꣳ स म॒भ्य॑भि सम् । \newline
5. स म॑नह्येता मनह्येताꣳ॒॒ सꣳ स म॑नह्येताम् । \newline
6. अ॒न॒ह्ये॒ता॒म् तम् त म॑नह्येता मनह्येता॒म् तम् । \newline
7. तं ॅयक्ष्मो॒ यक्ष्म॒ स्तम् तं ॅयक्ष्मः॑ । \newline
8. यक्ष्म॑ आर्च्छ दार्च्छ॒द् यक्ष्मो॒ यक्ष्म॑ आर्च्छत् । \newline
9. आ॒र्च्छ॒द् राजा॑नꣳ॒॒ राजा॑न मार्च्छ दार्च्छ॒द् राजा॑नम् । \newline
10. राजा॑नं॒ ॅयक्ष्मो॒ यक्ष्मो॒ राजा॑नꣳ॒॒ राजा॑नं॒ ॅयक्ष्मः॑ । \newline
11. यक्ष्म॑ आर दार॒द् यक्ष्मो॒ यक्ष्म॑ आरत् । \newline
12. आ॒र॒दिती त्या॑र दार॒दिति॑ । \newline
13. इति॒ तत् तदितीति॒ तत् । \newline
14. तद् रा॑जय॒क्ष्मस्य॑ राजय॒क्ष्मस्य॒ तत् तद् रा॑जय॒क्ष्मस्य॑ । \newline
15. रा॒ज॒य॒क्ष्मस्य॒ जन्म॒ जन्म॑ राजय॒क्ष्मस्य॑ राजय॒क्ष्मस्य॒ जन्म॑ । \newline
16. रा॒ज॒य॒क्ष्मस्येति॑ राज - य॒क्ष्मस्य॑ । \newline
17. जन्म॒ यद् यज् जन्म॒ जन्म॒ यत् । \newline
18. यत् पापी॑या॒न् पापी॑या॒न्॒. यद् यत् पापी॑यान् । \newline
19. पापी॑या॒ नभ॑व॒दभ॑व॒त् पापी॑या॒न् पापी॑या॒ नभ॑वत् । \newline
20. अभ॑व॒त् तत् तदभ॑व॒ दभ॑व॒त् तत् । \newline
21. तत् पा॑पय॒क्ष्मस्य॑ पापय॒क्ष्मस्य॒ तत् तत् पा॑पय॒क्ष्मस्य॑ । \newline
22. पा॒प॒य॒क्ष्मस्य॒ यद् यत् पा॑पय॒क्ष्मस्य॑ पापय॒क्ष्मस्य॒ यत् । \newline
23. पा॒प॒य॒क्ष्मस्येति॑ पाप - य॒क्ष्मस्य॑ । \newline
24. यज् जा॒याभ्या᳚म् जा॒याभ्यां॒ ॅयद् यज् जा॒याभ्या᳚म् । \newline
25. जा॒याभ्या॒ मवि॑न्द॒दवि॑न्दज् जा॒याभ्या᳚म् जा॒याभ्या॒ मवि॑न्दत् । \newline
26. अवि॑न्द॒त् तत् तदवि॑न्द॒ दवि॑न्द॒त् तत् । \newline
27. तज् जा॒येन्य॑स्य जा॒येन्य॑स्य॒ तत् तज् जा॒येन्य॑स्य । \newline
28. जा॒येन्य॑स्य॒ यो यो जा॒येन्य॑स्य जा॒येन्य॑स्य॒ यः । \newline
29. य ए॒व मे॒वं ॅयो य ए॒वम् । \newline
30. ए॒व मे॒तेषा॑ मे॒तेषा॑ मे॒व मे॒व मे॒तेषा᳚म् । \newline
31. ए॒तेषां॒ ॅयक्ष्मा॑णां॒ ॅयक्ष्मा॑णा मे॒तेषा॑ मे॒तेषां॒ ॅयक्ष्मा॑णाम् । \newline
32. यक्ष्मा॑णा॒म् जन्म॒ जन्म॒ यक्ष्मा॑णां॒ ॅयक्ष्मा॑णा॒म् जन्म॑ । \newline
33. जन्म॒ वेद॒ वेद॒ जन्म॒ जन्म॒ वेद॑ । \newline
34. वेद॒ न न वेद॒ वेद॒ न । \newline
35. नैन॑ मेन॒म् न नैन᳚म् । \newline
36. ए॒न॒ मे॒त ए॒त ए॑न मेन मे॒ते । \newline
37. ए॒ते यक्ष्मा॒ यक्ष्मा॑ ए॒त ए॒ते यक्ष्माः᳚ । \newline
38. यक्ष्मा॑ विन्दन्ति विन्दन्ति॒ यक्ष्मा॒ यक्ष्मा॑ विन्दन्ति । \newline
39. वि॒न्द॒न्ति॒ स स वि॑न्दन्ति विन्दन्ति॒ सः । \newline
40. स ए॒ते ए॒ते स स ए॒ते । \newline
41. ए॒ते ए॒वैवैते ए॒ते ए॒व । \newline
42. ए॒ते इत्ये॒ते । \newline
43. ए॒व न॑म॒स्यन् न॑म॒स्यन् ने॒वैव न॑म॒स्यन्न् । \newline
44. न॒म॒स्यन् नुपोप॑ नम॒स्यन् न॑म॒स्यन् नुप॑ । \newline
45. उपा॑धाव दधाव॒ दुपोपा॑ धावत् । \newline
46. अ॒धा॒व॒त् ते ते अ॑धाव दधाव॒त् ते । \newline
47. ते अ॑ब्रूता मब्रूता॒म् ते ते अ॑ब्रूताम् । \newline
48. ते इति॒ ते । \newline
49. अ॒ब्रू॒तां॒ ॅवरं॒ ॅवर॑ मब्रूता मब्रूतां॒ ॅवर᳚म् । \newline
50. वरं॑ ॅवृणावहै वृणावहै॒ वरं॒ ॅवरं॑ ॅवृणावहै । \newline
51. वृ॒णा॒व॒हा॒ आ॒व मा॒वं ॅवृ॑णावहै वृणावहा आ॒वम् । \newline
52. आ॒वम् दे॒वाना᳚म् दे॒वाना॑ मा॒व मा॒वम् दे॒वाना᳚म् । \newline
53. दे॒वाना᳚म् भाग॒धे भा॑ग॒धे दे॒वाना᳚म् दे॒वाना᳚म् भाग॒धे । \newline
54. भा॒ग॒धे अ॑सावा साव भाग॒धे भा॑ग॒धे अ॑साव । \newline
55. भा॒ग॒धे इति॑ भाग - धे । \newline
56. अ॒सा॒ वा॒व दा॒व द॑सावा सावा॒वत् । \newline

\textbf{Ghana Paata } \newline

1. ते ए॑न मेन॒म् ते ते ए॑न म॒भ्या᳚(1॒)भ्ये॑न॒म् ते ते ए॑न म॒भि । \newline
2. ते इति॒ ते । \newline
3. ए॒न॒ म॒भ्या᳚(1॒)भ्ये॑न मेन म॒भि सꣳ स म॒भ्ये॑न मेन म॒भि सम् । \newline
4. अ॒भि सꣳ स म॒भ्य॑भि स म॑नह्येता मनह्येताꣳ॒॒ स म॒भ्य॑भि स म॑नह्येताम् । \newline
5. स म॑नह्येता मनह्येताꣳ॒॒ सꣳ स म॑नह्येता॒म् तम् त म॑नह्येताꣳ॒॒ सꣳ स म॑नह्येता॒म् तम् । \newline
6. अ॒न॒ह्ये॒ता॒म् तम् त म॑नह्येता मनह्येता॒म् तं ॅयक्ष्मो॒ यक्ष्म॒ स्त म॑नह्येता मनह्येता॒म् तं ॅयक्ष्मः॑ । \newline
7. तं ॅयक्ष्मो॒ यक्ष्म॒ स्तम् तं ॅयक्ष्म॑ आर्च्छ दार्च्छ॒द् यक्ष्म॒ स्तम् तं ॅयक्ष्म॑ आर्च्छत् । \newline
8. यक्ष्म॑ आर्च्छ दार्च्छ॒द् यक्ष्मो॒ यक्ष्म॑ आर्च्छ॒द् राजा॑नꣳ॒॒ राजा॑न मार्च्छ॒द् यक्ष्मो॒ यक्ष्म॑ आर्च्छ॒द् राजा॑नम् । \newline
9. आ॒र्च्छ॒द् राजा॑नꣳ॒॒ राजा॑न मार्च्छ दार्च्छ॒द् राजा॑नं॒ ॅयक्ष्मो॒ यक्ष्मो॒ राजा॑न मार्च्छ दार्च्छ॒द् राजा॑नं॒ ॅयक्ष्मः॑ । \newline
10. राजा॑नं॒ ॅयक्ष्मो॒ यक्ष्मो॒ राजा॑नꣳ॒॒ राजा॑नं॒ ॅयक्ष्म॑ आर दार॒द् यक्ष्मो॒ राजा॑नꣳ॒॒ राजा॑नं॒ ॅयक्ष्म॑ आरत् । \newline
11. यक्ष्म॑ आर दार॒द् यक्ष्मो॒ यक्ष्म॑ आर॒ दिती त्या॑र॒द् यक्ष्मो॒ यक्ष्म॑ आर॒दिति॑ । \newline
12. आ॒र॒ दिती त्या॑र दार॒दिति॒ तत् तदि त्या॑रदार॒ दिति॒ तत् । \newline
13. इति॒ तत् तदितीति॒ तद् रा॑जय॒क्ष्मस्य॑ राजय॒क्ष्मस्य॒ तदितीति॒ तद् रा॑जय॒क्ष्मस्य॑ । \newline
14. तद् रा॑जय॒क्ष्मस्य॑ राजय॒क्ष्मस्य॒ तत् तद् रा॑जय॒क्ष्मस्य॒ जन्म॒ जन्म॑ राजय॒क्ष्मस्य॒ तत् तद् रा॑जय॒क्ष्मस्य॒ जन्म॑ । \newline
15. रा॒ज॒य॒क्ष्मस्य॒ जन्म॒ जन्म॑ राजय॒क्ष्मस्य॑ राजय॒क्ष्मस्य॒ जन्म॒ यद् यज् जन्म॑ राजय॒क्ष्मस्य॑ राजय॒क्ष्मस्य॒ जन्म॒ यत् । \newline
16. रा॒ज॒य॒क्ष्मस्येति॑ राज - य॒क्ष्मस्य॑ । \newline
17. जन्म॒ यद् यज् जन्म॒ जन्म॒ यत् पापी॑या॒न् पापी॑या॒न्॒. यज् जन्म॒ जन्म॒ यत् पापी॑यान् । \newline
18. यत् पापी॑या॒न् पापी॑या॒न्॒. यद् यत् पापी॑या॒ नभ॑व॒द भ॑व॒त् पापी॑या॒न्॒. यद् यत् पापी॑या॒ नभ॑वत् । \newline
19. पापी॑या॒ नभ॑व॒ दभ॑व॒त् पापी॑या॒न् पापी॑या॒ नभ॑व॒त् तत् तदभ॑व॒त् पापी॑या॒न् पापी॑या॒ नभ॑व॒त् तत् । \newline
20. अभ॑व॒त् तत् तदभ॑व॒ दभ॑व॒त् तत् पा॑पय॒क्ष्मस्य॑ पापय॒क्ष्मस्य॒ तदभ॑व॒ दभ॑व॒त् तत् पा॑पय॒क्ष्मस्य॑ । \newline
21. तत् पा॑पय॒क्ष्मस्य॑ पापय॒क्ष्मस्य॒ तत् तत् पा॑पय॒क्ष्मस्य॒ यद् यत् पा॑पय॒क्ष्मस्य॒ तत् तत् पा॑पय॒क्ष्मस्य॒ यत् । \newline
22. पा॒प॒य॒क्ष्मस्य॒ यद् यत् पा॑पय॒क्ष्मस्य॑ पापय॒क्ष्मस्य॒ यज् जा॒याभ्या᳚म् जा॒याभ्यां॒ ॅयत् पा॑पय॒क्ष्मस्य॑ पापय॒क्ष्मस्य॒ यज् जा॒याभ्या᳚म् । \newline
23. पा॒प॒य॒क्ष्मस्येति॑ पाप - य॒क्ष्मस्य॑ । \newline
24. यज् जा॒याभ्या᳚म् जा॒याभ्यां॒ ॅयद् यज् जा॒याभ्या॒ मवि॑न्द॒ दवि॑न्दज् जा॒याभ्यां॒ ॅयद् यज् जा॒याभ्या॒ मवि॑न्दत् । \newline
25. जा॒याभ्या॒ मवि॑न्द॒ दवि॑न्दज् जा॒याभ्या᳚म् जा॒याभ्या॒ मवि॑न्द॒त् तत् तदवि॑न्दज् जा॒याभ्या᳚म् जा॒याभ्या॒ मवि॑न्द॒त् तत् । \newline
26. अवि॑न्द॒त् तत् तदवि॑न्द॒ दवि॑न्द॒त् तज् जा॒येन्य॑स्य जा॒येन्य॑स्य॒ तदवि॑न्द॒ दवि॑न्द॒त् तज् जा॒येन्य॑स्य । \newline
27. तज् जा॒येन्य॑स्य जा॒येन्य॑स्य॒ तत् तज् जा॒येन्य॑स्य॒ यो यो जा॒येन्य॑स्य॒ तत् तज् जा॒येन्य॑स्य॒ यः । \newline
28. जा॒येन्य॑स्य॒ यो यो जा॒येन्य॑स्य जा॒येन्य॑स्य॒ य ए॒व मे॒वं ॅयो जा॒येन्य॑स्य जा॒येन्य॑स्य॒ य ए॒वम् । \newline
29. य ए॒व मे॒वं ॅयो य ए॒व मे॒तेषा॑ मे॒तेषा॑ मे॒वं ॅयो य ए॒व मे॒तेषा᳚म् । \newline
30. ए॒व मे॒तेषा॑ मे॒तेषा॑ मे॒व मे॒व मे॒तेषां॒ ॅयक्ष्मा॑णां॒ ॅयक्ष्मा॑णा मे॒तेषा॑ मे॒व मे॒व मे॒तेषां॒ ॅयक्ष्मा॑णाम् । \newline
31. ए॒तेषां॒ ॅयक्ष्मा॑णां॒ ॅयक्ष्मा॑णा मे॒तेषा॑ मे॒तेषां॒ ॅयक्ष्मा॑णा॒म् जन्म॒ जन्म॒ यक्ष्मा॑णा मे॒तेषा॑ मे॒तेषां॒ ॅयक्ष्मा॑णा॒म् जन्म॑ । \newline
32. यक्ष्मा॑णा॒म् जन्म॒ जन्म॒ यक्ष्मा॑णां॒ ॅयक्ष्मा॑णा॒म् जन्म॒ वेद॒ वेद॒ जन्म॒ यक्ष्मा॑णां॒ ॅयक्ष्मा॑णा॒म् जन्म॒ वेद॑ । \newline
33. जन्म॒ वेद॒ वेद॒ जन्म॒ जन्म॒ वेद॒ न न वेद॒ जन्म॒ जन्म॒ वेद॒ न । \newline
34. वेद॒ न न वेद॒ वेद॒ नैन॑ मेन॒म् न वेद॒ वेद॒ नैन᳚म् । \newline
35. नैन॑ मेन॒म् न नैन॑ मे॒त ए॒त ए॑न॒म् न नैन॑ मे॒ते । \newline
36. ए॒न॒ मे॒त ए॒त ए॑न मेन मे॒ते यक्ष्मा॒ यक्ष्मा॑ ए॒त ए॑न मेन मे॒ते यक्ष्माः᳚ । \newline
37. ए॒ते यक्ष्मा॒ यक्ष्मा॑ ए॒त ए॒ते यक्ष्मा॑ विन्दन्ति विन्दन्ति॒ यक्ष्मा॑ ए॒त ए॒ते यक्ष्मा॑ विन्दन्ति । \newline
38. यक्ष्मा॑ विन्दन्ति विन्दन्ति॒ यक्ष्मा॒ यक्ष्मा॑ विन्दन्ति॒ स स वि॑न्दन्ति॒ यक्ष्मा॒ यक्ष्मा॑ विन्दन्ति॒ सः । \newline
39. वि॒न्द॒न्ति॒ स स वि॑न्दन्ति विन्दन्ति॒ स ए॒ते ए॒ते स वि॑न्दन्ति विन्दन्ति॒ स ए॒ते । \newline
40. स ए॒ते ए॒ते स स ए॒ते ए॒वै वैते स स ए॒ते ए॒व । \newline
41. ए॒ते ए॒वैवैते ए॒ते ए॒व न॑म॒स्यन् न॑म॒स्यन् ने॒वैते ए॒ते ए॒व न॑म॒स्यन्न् । \newline
42. ए॒ते इत्ये॒ते । \newline
43. ए॒व न॑म॒स्यन् न॑म॒स्यन् ने॒वैव न॑म॒स्यन् नुपोप॑ नम॒स्यन् ने॒वैव न॑म॒स्यन् नुप॑ । \newline
44. न॒म॒स्यन् नुपोप॑ नम॒स्यन् न॑म॒स्यन् नुपा॑धाव दधाव॒दुप॑ नम॒स्यन् न॑म॒स्यन् नुपा॑धावत् । \newline
45. उपा॑धाव दधाव॒ दुपोपा॑ धाव॒त् ते ते अ॑धाव॒ दुपोपा॑ धाव॒त् ते । \newline
46. अ॒धा॒व॒त् ते ते अ॑धाव दधाव॒त् ते अ॑ब्रूता मब्रूता॒म् ते अ॑धाव दधाव॒त् ते अ॑ब्रूताम् । \newline
47. ते अ॑ब्रूता मब्रूता॒म् ते ते अ॑ब्रूतां॒ ॅवरं॒ ॅवर॑ मब्रूता॒म् ते ते अ॑ब्रूतां॒ ॅवर᳚म् । \newline
48. ते इति॒ ते । \newline
49. अ॒ब्रू॒तां॒ ॅवरं॒ ॅवर॑ मब्रूता मब्रूतां॒ ॅवरं॑ ॅवृणावहै वृणावहै॒ वर॑ मब्रूता मब्रूतां॒ ॅवरं॑ ॅवृणावहै । \newline
50. वरं॑ ॅवृणावहै वृणावहै॒ वरं॒ ॅवरं॑ ॅवृणावहा आ॒व मा॒वं ॅवृ॑णावहै॒ वरं॒ ॅवरं॑ ॅवृणावहा आ॒वम् । \newline
51. वृ॒णा॒व॒हा॒ आ॒व मा॒वं ॅवृ॑णावहै वृणावहा आ॒वम् दे॒वाना᳚म् दे॒वाना॑ मा॒वं ॅवृ॑णावहै वृणावहा आ॒वम् दे॒वाना᳚म् । \newline
52. आ॒वम् दे॒वाना᳚म् दे॒वाना॑ मा॒व मा॒वम् दे॒वाना᳚म् भाग॒धे भा॑ग॒धे दे॒वाना॑ मा॒व मा॒वम् दे॒वाना᳚म् भाग॒धे । \newline
53. दे॒वाना᳚म् भाग॒धे भा॑ग॒धे दे॒वाना᳚म् दे॒वाना᳚म् भाग॒धे अ॑सावासाव भाग॒धे दे॒वाना᳚म् दे॒वाना᳚म् भाग॒धे अ॑साव । \newline
54. भा॒ग॒धे अ॑सावा साव भाग॒धे भा॑ग॒धे अ॑सावा॒ वदा॒व द॑साव भाग॒धे भा॑ग॒धे अ॑सावा॒वत् । \newline
55. भा॒ग॒धे इति॑ भाग - धे । \newline
56. अ॒सा॒वा॒ वदा॒व द॑सावा सावा॒ वदध्यध्या॒ वद॑सावा सावा॒व दधि॑ । \newline
\pagebreak
\markright{ TS 2.5.6.6  \hfill https://www.vedavms.in \hfill}
\addcontentsline{toc}{section}{ TS 2.5.6.6 }
\section*{ TS 2.5.6.6 }

\textbf{TS 2.5.6.6 } \newline
\textbf{Samhita Paata} \newline

-वदधि॑ दे॒वा इ॑ज्यान्ता॒ इति॒ तस्मा᳚थ् स॒दृशी॑नाꣳ॒॒ रात्री॑णाममावा॒स्या॑यां च पौर्णमा॒स्यां च॑ दे॒वा इ॑ज्यन्त ए॒ते हि दे॒वानां᳚ भाग॒धे भा॑ग॒धा अ॑स्मै मनु॒ष्या॑ भवन्ति॒ य ए॒वं ॅवेद॑ भू॒तानि॒ क्षुध॑मघ्नन्थ् स॒द्यो म॑नु॒ष्या॑ अर्द्धमा॒से दे॒वा मा॒सि पि॒तरः॑ संॅवथ्स॒रे वन॒स्पत॑य॒-स्तस्मा॒-दह॑रह-र्मनु॒ष्या॑ अश॑नमिच्छन्ते ऽर्द्धमा॒से दे॒वा इ॑ज्यन्ते मा॒सि पि॒तृभ्यः॑ क्रियते संॅवथ्स॒रे वन॒स्पत॑यः॒ फलं॑ ( ) गृह्णन्ति॒ य ए॒वं ॅवेद॒ हन्ति॒ क्षुधं॒ भ्रातृ॑व्यं ॥ \newline

\textbf{Pada Paata} \newline

आ॒वत् । अधीति॑ । दे॒वाः । इ॒ज्या॒न्तै॒ । इति॑ । तस्मा᳚त् । स॒दृशी॑नाम् । रात्री॑णाम् । अ॒मा॒वा॒स्या॑या॒मित्य॑मा - वा॒स्या॑याम् । च॒ । पौ॒र्ण॒मा॒स्यामिति॑ पौर्ण-मा॒स्याम् । च॒ । दे॒वाः । इ॒ज्य॒न्ते॒ । ए॒ते इति॑ । हि । दे॒वाना᳚म् । भा॒ग॒धे इति॑ भाग - धे । भा॒ग॒धा इति॑ भाग - धाः । अ॒स्मै॒ । म॒नु॒ष्याः᳚ । भ॒व॒न्ति॒ । यः । ए॒वम् । वेद॑ । भू॒तानि॑ । क्षुध᳚म् । अ॒घ्न॒न्न् । स॒द्यः । म॒नु॒ष्याः᳚ । अ॒र्द्ध॒मा॒स इत्य॑र्द्ध - मा॒से । दे॒वाः । मा॒सि । पि॒तरः॑ । सं॒ॅव॒थ्स॒र इति॑ सं - व॒थ्स॒रे । वन॒स्पत॑यः । तस्मा᳚त् । अह॑रह॒रित्यहः॑ - अ॒हः॒ । म॒नु॒ष्याः᳚ । अश॑नम् । इ॒च्छ॒न्ते॒ । अ॒र्द्ध॒मा॒स इत्य॑र्द्ध-मा॒से । दे॒वाः । इ॒ज्य॒न्ते॒ । मा॒सि । पि॒तृभ्य॒ इति॑ पि॒तृ - भ्यः॒ । क्रि॒य॒ते॒ । सं॒ॅव॒थ्स॒र इति॑ सं - व॒थ्स॒रे । वन॒स्पत॑यः । फल᳚म् ( ) । गृ॒ह्ण॒न्ति॒ । यः । ए॒वम् । वेद॑ । हन्ति॑ । क्षुध᳚म् । भ्रातृ॑व्यम् ॥  \newline


\textbf{Krama Paata} \newline

आ॒वदधि॑ । अधि॑ दे॒वाः । दे॒वा इ॑ज्यान्तै । इ॒ज्या॒न्ता॒ इति॑ । इति॒ तस्मा᳚त् । तस्मा᳚थ् स॒दृशी॑नाम् । स॒दृशी॑नाꣳ॒॒ रात्री॑णाम् । रात्री॑णाममावा॒स्या॑याम् । अ॒मा॒वा॒स्या॑याम् च । अ॒मा॒वा॒स्या॑या॒मित्य॑मा - वा॒स्या॑याम् । च॒ पौ॒र्ण॒मा॒स्याम् । पौ॒र्ण॒मा॒स्याम् च॑ । पौ॒र्ण॒मा॒स्यामिति॑ पौर्ण - मा॒स्याम् । च॒ दे॒वाः । दे॒वा इ॑ज्यन्ते । इ॒ज्य॒न्त॒ ए॒ते । ए॒ते हि । ए॒ते इत्ये॒ते । हि दे॒वाना᳚म् । दे॒वाना᳚म् भाग॒धे । भा॒ग॒धे भा॑ग॒धाः । भा॒ग॒धे इति॑ भाग - धे । भा॒ग॒धा अ॑स्मै । भा॒ग॒धा इति॑ भाग - धाः । अ॒स्मै॒ म॒नु॒ष्याः᳚ । म॒नु॒ष्या॑ भवन्ति । भ॒व॒न्ति॒ यः । य ए॒वम् । ए॒वम् ॅवेद॑ । वेद॑ भू॒तानि॑ । भू॒तानि॒ क्षुध᳚म् । क्षुध॑मघ्नन्न् । अ॒घ्न॒न्थ् स॒द्यः । स॒द्यो म॑नु॒ष्याः᳚ । म॒नु॒ष्या॑ अर्द्धमा॒से । अ॒र्द्ध॒मा॒से दे॒वाः । अ॒र्द्ध॒मा॒स इत्य॑र्द्ध - मा॒से । दे॒वा मा॒सि । मा॒सि पि॒तरः॑ । पि॒तरः॑ सम्ॅवथ्स॒रे । स॒म्ॅव॒थ्स॒रे वन॒स्पत॑यः । स॒म्ॅव॒थ्स॒र इति॑ सम् - व॒थ्स॒रे । वन॒स्पत॑य॒स्तस्मा᳚त् । तस्मा॒दह॑रहः । अह॑रहर्,मनु॒ष्याः᳚ । अह॑रह॒रित्यहः॑ - अ॒हः॒ । म॒नु॒ष्या॑ अश॑नम् । अश॑नमिच्छन्ते । इ॒च्छ॒न्ते॒ ऽर्द्ध॒मा॒से । अ॒र्द्ध॒मा॒से दे॒वाः । अ॒र्द्ध॒मा॒स इत्य॑र्द्ध - मा॒से । दे॒वा इ॑ज्यन्ते । इ॒ज्य॒न्ते॒ मा॒सि । मा॒सि पि॒तृभ्यः॑ । पि॒तृभ्यः॑ क्रियते । पि॒तृभ्य॒ इति॑ पि॒तृ - भ्यः॒ । क्रि॒य॒ते॒ स॒म्ॅव॒थ्स॒रे । स॒म्ॅव॒थ्स॒रे वन॒स्पत॑यः । स॒म्ॅव॒थ्स॒र इति॑ सम् - व॒थ्स॒रे । वन॒स्पत॑यः॒ फल᳚म् ( ) । फल॑म् गृह्णन्ति । गृ॒ह्ण॒न्ति॒ यः । य ए॒वम् । ए॒वम् ॅवेद॑ । वेद॒ हन्ति॑ । हन्ति॒ क्षुध᳚म् । क्षुध॒म् भ्रातृ॑व्यम् । भ्रातृ॑व्य॒मिति॒ भ्रातृ॑व्यम् । \newline

\textbf{Jatai Paata} \newline

1. आ॒वद ध्य ध्या॒वदा॒ वदधि॑ । \newline
2. अधि॑ दे॒वा दे॒वा अध्यधि॑ दे॒वाः । \newline
3. दे॒वा इ॑ज्यान्ता इज्यान्तै दे॒वा दे॒वा इ॑ज्यान्तै । \newline
4. इ॒ज्या॒न्ता॒ इतीती᳚ ज्यान्ता इज्यान्ता॒ इति॑ । \newline
5. इति॒ तस्मा॒त् तस्मा॒ दितीति॒ तस्मा᳚त् । \newline
6. तस्मा᳚थ् स॒दृशी॑नाꣳ स॒दृशी॑ना॒म् तस्मा॒त् तस्मा᳚थ् स॒दृशी॑नाम् । \newline
7. स॒दृशी॑नाꣳ॒॒ रात्री॑णाꣳ॒॒ रात्री॑णाꣳ स॒दृशी॑नाꣳ स॒दृशी॑नाꣳ॒॒ रात्री॑णाम् । \newline
8. रात्री॑णा ममावा॒स्या॑या ममावा॒स्या॑याꣳ॒॒ रात्री॑णाꣳ॒॒ रात्री॑णा ममावा॒स्या॑याम् । \newline
9. अ॒मा॒वा॒स्या॑याम् च चामावा॒स्या॑या ममावा॒स्या॑याम् च । \newline
10. अ॒मा॒वा॒स्या॑या॒मित्य॑मा - वा॒स्या॑याम् । \newline
11. च॒ पौ॒र्ण॒मा॒स्याम् पौ᳚र्णमा॒स्याम् च॑ च पौर्णमा॒स्याम् । \newline
12. पौ॒र्ण॒मा॒स्याम् च॑ च पौर्णमा॒स्याम् पौ᳚र्णमा॒स्याम् च॑ । \newline
13. पौ॒र्ण॒मा॒स्यामिति॑ पौर्ण - मा॒स्याम् । \newline
14. च॒ दे॒वा दे॒वाश्च॑ च दे॒वाः । \newline
15. दे॒वा इ॑ज्यन्त इज्यन्ते दे॒वा दे॒वा इ॑ज्यन्ते । \newline
16. इ॒ज्य॒न्त॒ ए॒ते ए॒ते इ॑ज्यन्त इज्यन्त ए॒ते । \newline
17. ए॒ते हि ह्ये॑ते ए॒ते हि । \newline
18. ए॒ते इत्ये॒ते । \newline
19. हि दे॒वाना᳚म् दे॒वानाꣳ॒॒ हि हि दे॒वाना᳚म् । \newline
20. दे॒वाना᳚म् भाग॒धे भा॑ग॒धे दे॒वाना᳚म् दे॒वाना᳚म् भाग॒धे । \newline
21. भा॒ग॒धे भा॑ग॒धा भा॑ग॒धा भा॑ग॒धे भा॑ग॒धे भा॑ग॒धाः । \newline
22. भा॒ग॒धे इति॑ भाग - धे । \newline
23. भा॒ग॒धा अ॑स्मा अस्मै भाग॒धा भा॑ग॒धा अ॑स्मै । \newline
24. भा॒ग॒धा इति॑ भाग - धाः । \newline
25. अ॒स्मै॒ म॒नु॒ष्या॑ मनु॒ष्या॑ अस्मा अस्मै मनु॒ष्याः᳚ । \newline
26. म॒नु॒ष्या॑ भवन्ति भवन्ति मनु॒ष्या॑ मनु॒ष्या॑ भवन्ति । \newline
27. भ॒व॒न्ति॒ यो यो भ॑वन्ति भवन्ति॒ यः । \newline
28. य ए॒व मे॒वं ॅयो य ए॒वम् । \newline
29. ए॒वं ॅवेद॒ वेदै॒व मे॒वं ॅवेद॑ । \newline
30. वेद॑ भू॒तानि॑ भू॒तानि॒ वेद॒ वेद॑ भू॒तानि॑ । \newline
31. भू॒तानि॒ क्षुध॒म् क्षुध॑म् भू॒तानि॑ भू॒तानि॒ क्षुध᳚म् । \newline
32. क्षुध॑ मघ्नन् नघ्न॒न् क्षुध॒म् क्षुध॑ मघ्नन्न् । \newline
33. अ॒घ्न॒न् थ्स॒द्यः स॒द्यो᳚ ऽघ्नन् नघ्नन् थ्स॒द्यः । \newline
34. स॒द्यो म॑नु॒ष्या॑ मनु॒ष्याः᳚ स॒द्यः स॒द्यो म॑नु॒ष्याः᳚ । \newline
35. म॒नु॒ष्या॑ अर्द्धमा॒से᳚ ऽर्द्धमा॒से म॑नु॒ष्या॑ मनु॒ष्या॑ अर्द्धमा॒से । \newline
36. अ॒र्द्ध॒मा॒से दे॒वा दे॒वा अ॑र्द्धमा॒से᳚ ऽर्द्धमा॒से दे॒वाः । \newline
37. अ॒र्द्ध॒मा॒स इत्य॑र्द्ध - मा॒से । \newline
38. दे॒वा मा॒सि मा॒सि दे॒वा दे॒वा मा॒सि । \newline
39. मा॒सि पि॒तरः॑ पि॒तरो॑ मा॒सि मा॒सि पि॒तरः॑ । \newline
40. पि॒तरः॑ संॅवथ्स॒रे सं॑ॅवथ्स॒रे पि॒तरः॑ पि॒तरः॑ संॅवथ्स॒रे । \newline
41. सं॒ॅव॒थ्स॒रे वन॒स्पत॑यो॒ वन॒स्पत॑यः संॅवथ्स॒रे सं॑ॅवथ्स॒रे वन॒स्पत॑यः । \newline
42. सं॒ॅव॒थ्स॒र इति॑ सं - व॒थ्स॒रे । \newline
43. वन॒स्पत॑य॒ स्तस्मा॒त् तस्मा॒द् वन॒स्पत॑यो॒ वन॒स्पत॑य॒ स्तस्मा᳚त् । \newline
44. तस्मा॒ दह॑रह॒ रह॑रह॒ स्तस्मा॒त् तस्मा॒ दह॑रहः । \newline
45. अह॑रहर् मनु॒ष्या॑ मनु॒ष्या॑ अह॑रह॒ रह॑रहर् मनु॒ष्याः᳚ । \newline
46. अह॑रह॒रित्यहः॑ - अ॒हः॒ । \newline
47. म॒नु॒ष्या॑ अश॑न॒ मश॑नम् मनु॒ष्या॑ मनु॒ष्या॑ अश॑नम् । \newline
48. अश॑न मिच्छन्त इच्छ॒न्ते ऽश॑न॒ मश॑न मिच्छन्ते । \newline
49. इ॒च्छ॒न्ते॒ ऽर्द्ध॒मा॒से᳚ ऽर्द्धमा॒स इ॑च्छन्त इच्छन्ते ऽर्द्धमा॒से । \newline
50. अ॒र्द्ध॒मा॒से दे॒वा दे॒वा अ॑र्द्धमा॒से᳚ ऽर्द्धमा॒से दे॒वाः । \newline
51. अ॒र्द्ध॒मा॒स इत्य॑र्द्ध - मा॒से । \newline
52. दे॒वा इ॑ज्यन्त इज्यन्ते दे॒वा दे॒वा इ॑ज्यन्ते । \newline
53. इ॒ज्य॒न्ते॒ मा॒सि मा॒सीज्य॑न्त इज्यन्ते मा॒सि । \newline
54. मा॒सि पि॒तृभ्यः॑ पि॒तृभ्यो॑ मा॒सि मा॒सि पि॒तृभ्यः॑ । \newline
55. पि॒तृभ्यः॑ क्रियते क्रियते पि॒तृभ्यः॑ पि॒तृभ्यः॑ क्रियते । \newline
56. पि॒तृभ्य॒ इति॑ पि॒तृ - भ्यः॒ । \newline
57. क्रि॒य॒ते॒ सं॒ॅव॒थ्स॒रे सं॑ॅवथ्स॒रे क्रि॑यते क्रियते संॅवथ्स॒रे । \newline
58. सं॒ॅव॒थ्स॒रे वन॒स्पत॑यो॒ वन॒स्पत॑यः संॅवथ्स॒रे सं॑ॅवथ्स॒रे वन॒स्पत॑यः । \newline
59. सं॒ॅव॒थ्स॒र इति॑ सं - व॒थ्स॒रे । \newline
60. वन॒स्पत॑यः॒ फल॒म् फलं॒ ॅवन॒स्पत॑यो॒ वन॒स्पत॑यः॒ फल᳚म् । \newline
61. फल॑म् गृह्णन्ति गृह्णन्ति॒ फल॒म् फल॑म् गृह्णन्ति । \newline
62. गृ॒ह्ण॒न्ति॒ यो यो गृ॑ह्णन्ति गृह्णन्ति॒ यः । \newline
63. य ए॒व मे॒वं ॅयो य ए॒वम् । \newline
64. ए॒वं ॅवेद॒ वेदै॒व मे॒वं ॅवेद॑ । \newline
65. वेद॒ हन्ति॒ हन्ति॒ वेद॒ वेद॒ हन्ति॑ । \newline
66. हन्ति॒ क्षुध॒म् क्षुधꣳ॒॒ हन्ति॒ हन्ति॒ क्षुध᳚म् । \newline
67. क्षुध॒म् भ्रातृ॑व्य॒म् भ्रातृ॑व्य॒म् क्षुध॒म् क्षुध॒म् भ्रातृ॑व्यम् । \newline
68. भ्रातृ॑व्य॒मिति॒ भ्रातृ॑व्यम् । \newline

\textbf{Ghana Paata } \newline

1. आ॒व दध्यध्या॒ वदा॒वदधि॑ दे॒वा दे॒वा अध्या॒ वदा॒व दधि॑ दे॒वाः । \newline
2. अधि॑ दे॒वा दे॒वा अध्यधि॑ दे॒वा इ॑ज्यान्ता इज्यान्तै दे॒वा अध्यधि॑ दे॒वा इ॑ज्यान्तै । \newline
3. दे॒वा इ॑ज्यान्ता इज्यान्तै दे॒वा दे॒वा इ॑ज्यान्ता॒ इतीती᳚ज्यान्तै दे॒वा दे॒वा इ॑ज्यान्ता॒ इति॑ । \newline
4. इ॒ज्या॒न्ता॒ इतीती᳚ज्यान्ता इज्यान्ता॒ इति॒ तस्मा॒त् तस्मा॒दिती᳚ज्यान्ता इज्यान्ता॒ इति॒ तस्मा᳚त् । \newline
5. इति॒ तस्मा॒त् तस्मा॒दितीति॒ तस्मा᳚थ् स॒दृशी॑नाꣳ स॒दृशी॑ना॒म् तस्मा॒दितीति॒ तस्मा᳚थ् स॒दृशी॑नाम् । \newline
6. तस्मा᳚थ् स॒दृशी॑नाꣳ स॒दृशी॑ना॒म् तस्मा॒त् तस्मा᳚थ् स॒दृशी॑नाꣳ॒॒ रात्री॑णाꣳ॒॒ रात्री॑णाꣳ स॒दृशी॑ना॒म् तस्मा॒त् तस्मा᳚थ् स॒दृशी॑नाꣳ॒॒ रात्री॑णाम् । \newline
7. स॒दृशी॑नाꣳ॒॒ रात्री॑णाꣳ॒॒ रात्री॑णाꣳ स॒दृशी॑नाꣳ स॒दृशी॑नाꣳ॒॒ रात्री॑णा ममावा॒स्या॑या ममावा॒स्या॑याꣳ॒॒ रात्री॑णाꣳ स॒दृशी॑नाꣳ स॒दृशी॑नाꣳ॒॒ रात्री॑णा ममावा॒स्या॑याम् । \newline
8. रात्री॑णा ममावा॒स्या॑या ममावा॒स्या॑याꣳ॒॒ रात्री॑णाꣳ॒॒ रात्री॑णा ममावा॒स्या॑याम् च चामावा॒स्या॑याꣳ॒॒ रात्री॑णाꣳ॒॒ रात्री॑णा ममावा॒स्या॑याम् च । \newline
9. अ॒मा॒वा॒स्या॑याम् च चामावा॒स्या॑या ममावा॒स्या॑याम् च पौर्णमा॒स्याम् पौ᳚र्णमा॒स्याम् चा॑मावा॒स्या॑या ममावा॒स्या॑याम् च पौर्णमा॒स्याम् । \newline
10. अ॒मा॒वा॒स्या॑या॒मित्य॑मा - वा॒स्या॑याम् । \newline
11. च॒ पौ॒र्ण॒मा॒स्याम् पौ᳚र्णमा॒स्याम् च॑ च पौर्णमा॒स्याम् च॑ च पौर्णमा॒स्याम् च॑ च पौर्णमा॒स्याम् च॑ । \newline
12. पौ॒र्ण॒मा॒स्याम् च॑ च पौर्णमा॒स्याम् पौ᳚र्णमा॒स्याम् च॑ दे॒वा दे॒वाश्च॑ पौर्णमा॒स्याम् पौ᳚र्णमा॒स्याम् च॑ दे॒वाः । \newline
13. पौ॒र्ण॒मा॒स्यामिति॑ पौर्ण - मा॒स्याम् । \newline
14. च॒ दे॒वा दे॒वाश्च॑ च दे॒वा इ॑ज्यन्त इज्यन्ते दे॒वाश्च॑ च दे॒वा इ॑ज्यन्ते । \newline
15. दे॒वा इ॑ज्यन्त इज्यन्ते दे॒वा दे॒वा इ॑ज्यन्त ए॒ते ए॒ते इ॑ज्यन्ते दे॒वा दे॒वा इ॑ज्यन्त ए॒ते । \newline
16. इ॒ज्य॒न्त॒ ए॒ते ए॒ते इ॑ज्यन्त इज्यन्त ए॒ते हि ह्ये॑ते इ॑ज्यन्त इज्यन्त ए॒ते हि । \newline
17. ए॒ते हि ह्ये॑ते ए॒ते हि दे॒वाना᳚म् दे॒वानाꣳ॒॒ ह्ये॑ते ए॒ते हि दे॒वाना᳚म् । \newline
18. ए॒ते इत्ये॒ते । \newline
19. हि दे॒वाना᳚म् दे॒वानाꣳ॒॒ हि हि दे॒वाना᳚म् भाग॒धे भा॑ग॒धे दे॒वानाꣳ॒॒ हि हि दे॒वाना᳚म् भाग॒धे । \newline
20. दे॒वाना᳚म् भाग॒धे भा॑ग॒धे दे॒वाना᳚म् दे॒वाना᳚म् भाग॒धे भा॑ग॒धा भा॑ग॒धा भा॑ग॒धे दे॒वाना᳚म् दे॒वाना᳚म् भाग॒धे भा॑ग॒धाः । \newline
21. भा॒ग॒धे भा॑ग॒धा भा॑ग॒धा भा॑ग॒धे भा॑ग॒धे भा॑ग॒धा अ॑स्मा अस्मै भाग॒धा भा॑ग॒धे भा॑ग॒धे भा॑ग॒धा अ॑स्मै । \newline
22. भा॒ग॒धे इति॑ भाग - धे । \newline
23. भा॒ग॒धा अ॑स्मा अस्मै भाग॒धा भा॑ग॒धा अ॑स्मै मनु॒ष्या॑ मनु॒ष्या॑ अस्मै भाग॒धा भा॑ग॒धा अ॑स्मै मनु॒ष्याः᳚ । \newline
24. भा॒ग॒धा इति॑ भाग - धाः । \newline
25. अ॒स्मै॒ म॒नु॒ष्या॑ मनु॒ष्या॑ अस्मा अस्मै मनु॒ष्या॑ भवन्ति भवन्ति मनु॒ष्या॑ अस्मा अस्मै मनु॒ष्या॑ भवन्ति । \newline
26. म॒नु॒ष्या॑ भवन्ति भवन्ति मनु॒ष्या॑ मनु॒ष्या॑ भवन्ति॒ यो यो भ॑वन्ति मनु॒ष्या॑ मनु॒ष्या॑ भवन्ति॒ यः । \newline
27. भ॒व॒न्ति॒ यो यो भ॑वन्ति भवन्ति॒ य ए॒व मे॒वं ॅयो भ॑वन्ति भवन्ति॒ य ए॒वम् । \newline
28. य ए॒व मे॒वं ॅयो य ए॒वं ॅवेद॒ वेदै॒वं ॅयो य ए॒वं ॅवेद॑ । \newline
29. ए॒वं ॅवेद॒ वेदै॒व मे॒वं ॅवेद॑ भू॒तानि॑ भू॒तानि॒ वेदै॒व मे॒वं ॅवेद॑ भू॒तानि॑ । \newline
30. वेद॑ भू॒तानि॑ भू॒तानि॒ वेद॒ वेद॑ भू॒तानि॒ क्षुध॒म् क्षुध॑म् भू॒तानि॒ वेद॒ वेद॑ भू॒तानि॒ क्षुध᳚म् । \newline
31. भू॒तानि॒ क्षुध॒म् क्षुध॑म् भू॒तानि॑ भू॒तानि॒ क्षुध॑ मघ्नन् नघ्न॒न् क्षुध॑म् भू॒तानि॑ भू॒तानि॒ क्षुध॑ मघ्नन्न् । \newline
32. क्षुध॑ मघ्नन् नघ्न॒न् क्षुध॒म् क्षुध॑ मघ्नन् थ्स॒द्यः स॒द्यो᳚ ऽघ्न॒न् क्षुध॒म् क्षुध॑ मघ्नन् थ्स॒द्यः । \newline
33. अ॒घ्न॒न् थ्स॒द्यः स॒द्यो᳚ ऽघ्नन् नघ्नन् थ्स॒द्यो म॑नु॒ष्या॑ मनु॒ष्याः᳚ स॒द्यो᳚ ऽघ्नन् नघ्नन् थ्स॒द्यो म॑नु॒ष्याः᳚ । \newline
34. स॒द्यो म॑नु॒ष्या॑ मनु॒ष्याः᳚ स॒द्यः स॒द्यो म॑नु॒ष्या॑ अर्द्धमा॒से᳚ ऽर्द्धमा॒से म॑नु॒ष्याः᳚ स॒द्यः स॒द्यो म॑नु॒ष्या॑ अर्द्धमा॒से । \newline
35. म॒नु॒ष्या॑ अर्द्धमा॒से᳚ ऽर्द्धमा॒से म॑नु॒ष्या॑ मनु॒ष्या॑ अर्द्धमा॒से दे॒वा दे॒वा अ॑र्द्धमा॒से म॑नु॒ष्या॑ मनु॒ष्या॑ अर्द्धमा॒से दे॒वाः । \newline
36. अ॒र्द्ध॒मा॒से दे॒वा दे॒वा अ॑र्द्धमा॒से᳚ ऽर्द्धमा॒से दे॒वा मा॒सि मा॒सि दे॒वा अ॑र्द्धमा॒से᳚ ऽर्द्धमा॒से दे॒वा मा॒सि । \newline
37. अ॒र्द्ध॒मा॒स इत्य॑र्द्ध - मा॒से । \newline
38. दे॒वा मा॒सि मा॒सि दे॒वा दे॒वा मा॒सि पि॒तरः॑ पि॒तरो॑ मा॒सि दे॒वा दे॒वा मा॒सि पि॒तरः॑ । \newline
39. मा॒सि पि॒तरः॑ पि॒तरो॑ मा॒सि मा॒सि पि॒तरः॑ संॅवथ्स॒रे सं॑ॅवथ्स॒रे पि॒तरो॑ मा॒सि मा॒सि पि॒तरः॑ संॅवथ्स॒रे । \newline
40. पि॒तरः॑ संॅवथ्स॒रे सं॑ॅवथ्स॒रे पि॒तरः॑ पि॒तरः॑ संॅवथ्स॒रे वन॒स्पत॑यो॒ वन॒स्पत॑यः संॅवथ्स॒रे पि॒तरः॑ पि॒तरः॑ संॅवथ्स॒रे वन॒स्पत॑यः । \newline
41. सं॒ॅव॒थ्स॒रे वन॒स्पत॑यो॒ वन॒स्पत॑यः संॅवथ्स॒रे सं॑ॅवथ्स॒रे वन॒स्पत॑य॒ स्तस्मा॒त् तस्मा॒द् वन॒स्पत॑यः संॅवथ्स॒रे सं॑ॅवथ्स॒रे वन॒स्पत॑य॒ स्तस्मा᳚त् । \newline
42. सं॒ॅव॒थ्स॒र इति॑ सं - व॒थ्स॒रे । \newline
43. वन॒स्पत॑य॒ स्तस्मा॒त् तस्मा॒द् वन॒स्पत॑यो॒ वन॒स्पत॑य॒ स्तस्मा॒ दह॑रह॒ रह॑रह॒ स्तस्मा॒द् वन॒स्पत॑यो॒ वन॒स्पत॑य॒ स्तस्मा॒ दह॑रहः । \newline
44. तस्मा॒ दह॑रह॒ रह॑रह॒ स्तस्मा॒त् तस्मा॒ दह॑रहर् मनु॒ष्या॑ मनु॒ष्या॑ अह॑रह॒ स्तस्मा॒त् तस्मा॒ दह॑रहर् मनु॒ष्याः᳚ । \newline
45. अह॑रहर् मनु॒ष्या॑ मनु॒ष्या॑ अह॑रह॒ रह॑रहर् मनु॒ष्या॑ अश॑न॒ मश॑नम् मनु॒ष्या॑ अह॑रह॒ रह॑रहर् मनु॒ष्या॑ अश॑नम् । \newline
46. अह॑रह॒रित्यहः॑ - अ॒हः॒ । \newline
47. म॒नु॒ष्या॑ अश॑न॒ मश॑नम् मनु॒ष्या॑ मनु॒ष्या॑ अश॑न मिच्छन्त इच्छ॒न्ते ऽश॑नम् मनु॒ष्या॑ मनु॒ष्या॑ अश॑न मिच्छन्ते । \newline
48. अश॑न मिच्छन्त इच्छ॒न्ते ऽश॑न॒ मश॑न मिच्छन्ते ऽर्द्धमा॒से᳚ ऽर्द्धमा॒स इ॑च्छ॒न्ते ऽश॑न॒ मश॑न मिच्छन्ते ऽर्द्धमा॒से । \newline
49. इ॒च्छ॒न्ते॒ ऽर्द्ध॒मा॒से᳚ ऽर्द्धमा॒स इ॑च्छन्त इच्छन्ते ऽर्द्धमा॒से दे॒वा दे॒वा अ॑र्द्धमा॒स इ॑च्छन्त इच्छन्ते ऽर्द्धमा॒से दे॒वाः । \newline
50. अ॒र्द्ध॒मा॒से दे॒वा दे॒वा अ॑र्द्धमा॒से᳚ ऽर्द्धमा॒से दे॒वा इ॑ज्यन्त इज्यन्ते दे॒वा अ॑र्द्धमा॒से᳚ ऽर्द्धमा॒से दे॒वा इ॑ज्यन्ते । \newline
51. अ॒र्द्ध॒मा॒स इत्य॑र्द्ध - मा॒से । \newline
52. दे॒वा इ॑ज्यन्त इज्यन्ते दे॒वा दे॒वा इ॑ज्यन्ते मा॒सि मा॒सीज्य॑न्ते दे॒वा दे॒वा इ॑ज्यन्ते मा॒सि । \newline
53. इ॒ज्य॒न्ते॒ मा॒सि मा॒सीज्य॑न्त इज्यन्ते मा॒सि पि॒तृभ्यः॑ पि॒तृभ्यो॑ मा॒सीज्य॑न्त इज्यन्ते मा॒सि पि॒तृभ्यः॑ । \newline
54. मा॒सि पि॒तृभ्यः॑ पि॒तृभ्यो॑ मा॒सि मा॒सि पि॒तृभ्यः॑ क्रियते क्रियते पि॒तृभ्यो॑ मा॒सि मा॒सि पि॒तृभ्यः॑ क्रियते । \newline
55. पि॒तृभ्यः॑ क्रियते क्रियते पि॒तृभ्यः॑ पि॒तृभ्यः॑ क्रियते संॅवथ्स॒रे सं॑ॅवथ्स॒रे क्रि॑यते पि॒तृभ्यः॑ पि॒तृभ्यः॑ क्रियते संॅवथ्स॒रे । \newline
56. पि॒तृभ्य॒ इति॑ पि॒तृ - भ्यः॒ । \newline
57. क्रि॒य॒ते॒ सं॒ॅव॒थ्स॒रे सं॑ॅवथ्स॒रे क्रि॑यते क्रियते संॅवथ्स॒रे वन॒स्पत॑यो॒ वन॒स्पत॑यः संॅवथ्स॒रे क्रि॑यते क्रियते संॅवथ्स॒रे वन॒स्पत॑यः । \newline
58. सं॒ॅव॒थ्स॒रे वन॒स्पत॑यो॒ वन॒स्पत॑यः संॅवथ्स॒रे सं॑ॅवथ्स॒रे वन॒स्पत॑यः॒ फल॒म् फलं॒ ॅवन॒स्पत॑यः संॅवथ्स॒रे सं॑ॅवथ्स॒रे वन॒स्पत॑यः॒ फल᳚म् । \newline
59. सं॒ॅव॒थ्स॒र इति॑ सं - व॒थ्स॒रे । \newline
60. वन॒स्पत॑यः॒ फल॒म् फलं॒ ॅवन॒स्पत॑यो॒ वन॒स्पत॑यः॒ फल॑म् गृह्णन्ति गृह्णन्ति॒ फलं॒ ॅवन॒स्पत॑यो॒ वन॒स्पत॑यः॒ फल॑म् गृह्णन्ति । \newline
61. फल॑म् गृह्णन्ति गृह्णन्ति॒ फल॒म् फल॑म् गृह्णन्ति॒ यो यो गृ॑ह्णन्ति॒ फल॒म् फल॑म् गृह्णन्ति॒ यः । \newline
62. गृ॒ह्ण॒न्ति॒ यो यो गृ॑ह्णन्ति गृह्णन्ति॒ य ए॒व मे॒वं ॅयो गृ॑ह्णन्ति गृह्णन्ति॒ य ए॒वम् । \newline
63. य ए॒व मे॒वं ॅयो य ए॒वं ॅवेद॒ वेदै॒वं ॅयो य ए॒वं ॅवेद॑ । \newline
64. ए॒वं ॅवेद॒ वेदै॒व मे॒वं ॅवेद॒ हन्ति॒ हन्ति॒ वेदै॒व मे॒वं ॅवेद॒ हन्ति॑ । \newline
65. वेद॒ हन्ति॒ हन्ति॒ वेद॒ वेद॒ हन्ति॒ क्षुध॒म् क्षुधꣳ॒॒ हन्ति॒ वेद॒ वेद॒ हन्ति॒ क्षुध᳚म् । \newline
66. हन्ति॒ क्षुध॒म् क्षुधꣳ॒॒ हन्ति॒ हन्ति॒ क्षुध॒म् भ्रातृ॑व्य॒म् भ्रातृ॑व्य॒म् क्षुधꣳ॒॒ हन्ति॒ हन्ति॒ क्षुध॒म् भ्रातृ॑व्यम् । \newline
67. क्षुध॒म् भ्रातृ॑व्य॒म् भ्रातृ॑व्य॒म् क्षुध॒म् क्षुध॒म् भ्रातृ॑व्यम् । \newline
68. भ्रातृ॑व्य॒मिति॒ भ्रातृ॑व्यम् । \newline
\pagebreak
\markright{ TS 2.5.7.1  \hfill https://www.vedavms.in \hfill}
\addcontentsline{toc}{section}{ TS 2.5.7.1 }
\section*{ TS 2.5.7.1 }

\textbf{TS 2.5.7.1 } \newline
\textbf{Samhita Paata} \newline

दे॒वा वै नर्चि न यजु॑ष्यश्रयन्त॒ ते साम॑न्ने॒वाश्र॑यन्त॒ हिं क॑रोति॒ सामै॒वाक॒॒र्॒.हिम् क॑रोति॒यत्रै॒वदे॒वा अश्र॑यन्त॒ तत॑ ए॒वैना॒न् प्रयु॑ङ्क्ते॒ हिं क॑रोति वा॒च ए॒वैष योगो॒ हिं क॑रोति प्र॒जा ए॒व तद्-यज॑मानः सृजते॒ त्रिः प्र॑थ॒मामन्वा॑ह॒ त्रिरु॑त्त॒मां ॅय॒ज्ञ्स्यै॒व तद् ब॒र्॒.सम् - [  ] \newline

\textbf{Pada Paata} \newline

दे॒वाः । वै । न । ऋ॒चि । न । यजु॑षि । अ॒श्र॒य॒न्त॒ । ते । सामन्न्॑ । ए॒व । अ॒श्र॒य॒न्त॒ । हिम् । क॒रो॒ति॒ । साम॑ । ए॒व । अ॒कः॒ । हिम् । क॒रो॒ति॒ । यत्र॑ । ए॒व । दे॒वाः । अश्र॑यन्त । ततः॑ । ए॒व । ए॒ना॒न् । प्रेति॑ । यु॒ङ्क्ते॒ । हिम् । क॒रो॒ति॒ । वा॒चः । ए॒व । ए॒षः । योगः॑ । हिम् । क॒रो॒ति॒ । प्र॒जा इति॑ प्र - जाः । ए॒व । तत् । यज॑मानः । सृ॒ज॒ते॒ । त्रिः । प्र॒थ॒माम् । अन्विति॑ । आ॒ह॒ । त्रिः । उ॒त्त॒मामित्यु॑त् - त॒माम् । य॒ज्ञ्स्य॑ । ए॒व । तत् । ब॒र्॒.सम् ।  \newline


\textbf{Krama Paata} \newline

दे॒वा वै । वै न । नर्चि । ऋ॒चि न । न यजु॑षि । यजु॑ष्यश्रयन्त । अ॒श्र॒य॒न्त॒ ते । ते सामन्न्॑ । साम॑न्ने॒व । ए॒वाश्र॑यन्त । अ॒श्र॒य॒न्त॒ हिम् । हिम् क॑रोति । क॒रो॒ति॒ साम॑ । सामै॒व । ए॒वाकः॑ । अ॒क॒र्॒. हिम् । हिम् क॑रोति । क॒रो॒ति॒ यत्र॑ । यत्रै॒व । ए॒व दे॒वाः । दे॒वा अश्र॑यन्त । अश्र॑यन्त॒ ततः॑ । तत॑ ए॒व । ए॒वैनान्॑ । ए॒ना॒न् प्र । प्र यु॑ङ्क्ते । यु॒ङ्क्ते॒ हिम् । हिम् क॑रोति । क॒रो॒ति॒ वा॒चः । वा॒च ए॒व । ए॒वैषः । ए॒ष योगः॑ । योगो॒ हिम् । हिम् क॑रोति । क॒रो॒ति॒ प्र॒जाः । प्र॒जा ए॒व । प्र॒जा इति॑ प्र - जाः । ए॒व तत् । तद् यज॑मानः । यज॑मानः सृजते । सृ॒ज॒ते॒ त्रिः । त्रिः प्र॑थ॒माम् । प्र॒थ॒मामनु॑ । अन्वा॑ह । आ॒ह॒ त्रिः । त्रिरु॑त्त॒माम् । उ॒त्त॒माम् ॅय॒ज्ञ्स्य॑ । उ॒त्त॒मामित्यु॑त् - त॒माम् । य॒ज्ञ्स्यै॒व । ए॒व तत् । तद् ब॒र्॒.सम् । ब॒र्॒.सम् न॑ह्यति \newline

\textbf{Jatai Paata} \newline

1. दे॒वा वै वै दे॒वा दे॒वा वै । \newline
2. वै न न वै वै न । \newline
3. न र्च्यृ॑चि न न र्चि । \newline
4. ऋ॒चि न न र्च्यृ॑चि न । \newline
5. न यजु॑षि॒ यजु॑षि॒ न न यजु॑षि । \newline
6. यजु॑ष्यश्रयन्ता श्रयन्त॒ यजु॑षि॒ यजु॑ष्य श्रयन्त । \newline
7. अ॒श्र॒य॒न्त॒ ते ते᳚ ऽश्रयन्ता श्रयन्त॒ ते । \newline
8. ते साम॒न् थ्साम॒न् ते ते सामन्न्॑ । \newline
9. साम॑न् ने॒वैव साम॒न् थ्साम॑न् ने॒व । \newline
10. ए॒वा श्र॑यन्ता श्रयन्तै॒वैवा श्र॑यन्त । \newline
11. अ॒श्र॒य॒न्त॒ हिꣳ हि म॑श्रयन्ता श्रयन्त॒ हिम् । \newline
12. हिम् क॑रोति करोति॒ हिꣳ हिम् क॑रोति । \newline
13. क॒रो॒ति॒ साम॒ साम॑ करोति करोति॒ साम॑ । \newline
14. सामै॒वैव साम॒ सामै॒व । \newline
15. ए॒वाक॑ रक रे॒वैवाकः॑ । \newline
16. अ॒क॒र्॒. हिꣳ हि म॑क रक॒र्॒. हिम् । \newline
17. हिम् क॑रोति करोति॒ हिꣳ हिम् क॑रोति । \newline
18. क॒रो॒ति॒ यत्र॒ यत्र॑ करोति करोति॒ यत्र॑ । \newline
19. यत्रै॒वैव यत्र॒ यत्रै॒व । \newline
20. ए॒व दे॒वा दे॒वा ए॒वैव दे॒वाः । \newline
21. दे॒वा अश्र॑य॒न्ता श्र॑यन्त दे॒वा दे॒वा अश्र॑यन्त । \newline
22. अश्र॑यन्त॒ तत॒स्ततो ऽश्र॑य॒न्ता श्र॑यन्त॒ ततः॑ । \newline
23. तत॑ ए॒वैव तत॒ स्तत॑ ए॒व । \newline
24. ए॒वैना॑ नेना ने॒वैवैनान्॑ । \newline
25. ए॒ना॒न् प्र प्रैना॑ नेना॒न् प्र । \newline
26. प्र यु॑ङ्क्ते युङ्क्ते॒ प्र प्र यु॑ङ्क्ते । \newline
27. यु॒ङ्क्ते॒ हिꣳ हिं ॅयु॑ङ्क्ते युङ्क्ते॒ हिम् । \newline
28. हिम् क॑रोति करोति॒ हिꣳ हिम् क॑रोति । \newline
29. क॒रो॒ति॒ वा॒चो वा॒चः क॑रोति करोति वा॒चः । \newline
30. वा॒च ए॒वैव वा॒चो वा॒च ए॒व । \newline
31. ए॒वैष ए॒ष ए॒वैवैषः । \newline
32. ए॒ष योगो॒ योग॑ ए॒ष ए॒ष योगः॑ । \newline
33. योगो॒ हिꣳ हिं ॅयोगो॒ योगो॒ हिम् । \newline
34. हिम् क॑रोति करोति॒ हिꣳ हिम् क॑रोति । \newline
35. क॒रो॒ति॒ प्र॒जाः प्र॒जाः क॑रोति करोति प्र॒जाः । \newline
36. प्र॒जा ए॒वैव प्र॒जाः प्र॒जा ए॒व । \newline
37. प्र॒जा इति॑ प्र - जाः । \newline
38. ए॒व तत् तदे॒वैव तत् । \newline
39. तद् यज॑मानो॒ यज॑मान॒ स्तत् तद् यज॑मानः । \newline
40. यज॑मानः सृजते सृजते॒ यज॑मानो॒ यज॑मानः सृजते । \newline
41. सृ॒ज॒ते॒ त्रि स्त्रिः सृ॑जते सृजते॒ त्रिः । \newline
42. त्रिः प्र॑थ॒माम् प्र॑थ॒माम् त्रि स्त्रिः प्र॑थ॒माम् । \newline
43. प्र॒थ॒मा मन्वनु॑ प्रथ॒माम् प्र॑थ॒मा मनु॑ । \newline
44. अन्वा॑ हा॒हा न्वन्वा॑ह । \newline
45. आ॒ह॒ त्रि स्त्रि रा॑हाह॒ त्रिः । \newline
46. त्रिरु॑त्त॒मा मु॑त्त॒माम् त्रि स्त्रिरु॑त्त॒माम् । \newline
47. उ॒त्त॒मां ॅय॒ज्ञ्स्य॑ य॒ज्ञ्स्यो᳚त्त॒मा मु॑त्त॒मां ॅय॒ज्ञ्स्य॑ । \newline
48. उ॒त्त॒मामित्यु॑त् - त॒माम् । \newline
49. य॒ज्ञ्स्यै॒वैव य॒ज्ञ्स्य॑ य॒ज्ञ्स्यै॒व । \newline
50. ए॒व तत् तदे॒वैव तत् । \newline
51. तद् ब॒र्॒.सम् ब॒र्॒.सम् तत् तद् ब॒र्॒.सम् । \newline
52. ब॒र्॒.सम् न॑ह्यति नह्यति ब॒र्॒.सम् ब॒र्॒.सम् न॑ह्यति । \newline

\textbf{Ghana Paata } \newline

1. दे॒वा वै वै दे॒वा दे॒वा वै न न वै दे॒वा दे॒वा वै न । \newline
2. वै न न वै वै न र्च्यृ॑चि न वै वै न र्चि । \newline
3. न र्च्यृ॑चि न न र्चि न न र्चि न न र्चि न । \newline
4. ऋ॒चि न न र्च्यृ॑चि न यजु॑षि॒ यजु॑षि॒ न र्च्यृ॑चि न यजु॑षि । \newline
5. न यजु॑षि॒ यजु॑षि॒ न न यजु॑ ष्यश्रयन्ता श्रयन्त॒ यजु॑षि॒ न न यजु॑ ष्यश्रयन्त । \newline
6. यजु॑ष्यश्रयन्ता श्रयन्त॒ यजु॑षि॒ यजु॑ष्यश्रयन्त॒ ते ते᳚ ऽश्रयन्त॒ यजु॑षि॒ यजु॑ष्यश्रयन्त॒ ते । \newline
7. अ॒श्र॒य॒न्त॒ ते ते᳚ ऽश्रयन्ता श्रयन्त॒ ते साम॒न् थ्साम॒न् ते᳚ ऽश्रयन्ता श्रयन्त॒ ते सामन्न्॑ । \newline
8. ते साम॒न् थ्साम॒न् ते ते साम॑न् ने॒वैव साम॒न् ते ते साम॑न् ने॒व । \newline
9. साम॑न् ने॒वैव साम॒न् थ्साम॑न् ने॒वाश्र॑यन्ता श्रयन्तै॒व साम॒न् थ्साम॑न् ने॒वाश्र॑यन्त । \newline
10. ए॒वाश्र॑यन्ता श्रयन्तै॒वैवा श्र॑यन्त॒ हिꣳ हि म॑श्रयन्तै॒वैवा श्र॑यन्त॒ हिम् । \newline
11. अ॒श्र॒य॒न्त॒ हिꣳ हि म॑श्रयन्ता श्रयन्त॒ हिम् क॑रोति करोति॒ हि म॑श्रयन्ता श्रयन्त॒ हिम् क॑रोति । \newline
12. हिम् क॑रोति करोति॒ हिꣳ हिम् क॑रोति॒ साम॒ साम॑ करोति॒ हिꣳ हिम् क॑रोति॒ साम॑ । \newline
13. क॒रो॒ति॒ साम॒ साम॑ करोति करोति॒ सामै॒वैव साम॑ करोति करोति॒ सामै॒व । \newline
14. सामै॒वैव साम॒ सामै॒वाक॑ रक रे॒व साम॒ सामै॒वाकः॑ । \newline
15. ए॒वाक॑ रक रे॒वैवाक॒र्॒. हिꣳ हि म॑क रे॒वैवाक॒र्॒. हिम् । \newline
16. अ॒क॒र्॒. हिꣳ हि म॑क रक॒र्॒. हिम् क॑रोति करोति॒ हि म॑क रक॒र्॒. हिम् क॑रोति । \newline
17. हिम् क॑रोति करोति॒ हिꣳ हिम् क॑रोति॒ यत्र॒ यत्र॑ करोति॒ हिꣳ हिम् क॑रोति॒ यत्र॑ । \newline
18. क॒रो॒ति॒ यत्र॒ यत्र॑ करोति करोति॒ यत्रै॒वैव यत्र॑ करोति करोति॒ यत्रै॒व । \newline
19. यत्रै॒वैव यत्र॒ यत्रै॒व दे॒वा दे॒वा ए॒व यत्र॒ यत्रै॒व दे॒वाः । \newline
20. ए॒व दे॒वा दे॒वा ए॒वैव दे॒वा अश्र॑य॒न्ता श्र॑यन्त दे॒वा ए॒वैव दे॒वा अश्र॑यन्त । \newline
21. दे॒वा अश्र॑य॒न्ता श्र॑यन्त दे॒वा दे॒वा अश्र॑यन्त॒ तत॒ स्ततो ऽश्र॑यन्त दे॒वा दे॒वा अश्र॑यन्त॒ ततः॑ । \newline
22. अश्र॑यन्त॒ तत॒ स्ततो ऽश्र॑य॒न्ता श्र॑यन्त॒ तत॑ ए॒वैव ततो ऽश्र॑य॒न्ता श्र॑यन्त॒ तत॑ ए॒व । \newline
23. तत॑ ए॒वैव तत॒ स्तत॑ ए॒वैना॑ नेना ने॒व तत॒ स्तत॑ ए॒वैनान्॑ । \newline
24. ए॒वैना॑ नेना ने॒वैवैना॒न् प्र प्रैना॑ ने॒वैवैना॒न् प्र । \newline
25. ए॒ना॒न् प्र प्रैना॑ नेना॒न् प्र यु॑ङ्क्ते युङ्क्ते॒ प्रैना॑ नेना॒न् प्र यु॑ङ्क्ते । \newline
26. प्र यु॑ङ्क्ते युङ्क्ते॒ प्र प्र यु॑ङ्क्ते॒ हिꣳ हिं ॅयु॑ङ्क्ते॒ प्र प्र यु॑ङ्क्ते॒ हिम् । \newline
27. यु॒ङ्क्ते॒ हिꣳ हिं ॅयु॑ङ्क्ते युङ्क्ते॒ हिम् क॑रोति करोति॒ हिं ॅयु॑ङ्क्ते युङ्क्ते॒ हिम् क॑रोति । \newline
28. हिम् क॑रोति करोति॒ हिꣳ हिम् क॑रोति वा॒चो वा॒चः क॑रोति॒ हिꣳ हिम् क॑रोति वा॒चः । \newline
29. क॒रो॒ति॒ वा॒चो वा॒चः क॑रोति करोति वा॒च ए॒वैव वा॒चः क॑रोति करोति वा॒च ए॒व । \newline
30. वा॒च ए॒वैव वा॒चो वा॒च ए॒वैष ए॒ष ए॒व वा॒चो वा॒च ए॒वैषः । \newline
31. ए॒वैष ए॒ष ए॒वैवैष योगो॒ योग॑ ए॒ष ए॒वैवैष योगः॑ । \newline
32. ए॒ष योगो॒ योग॑ ए॒ष ए॒ष योगो॒ हिꣳ हिं ॅयोग॑ ए॒ष ए॒ष योगो॒ हिम् । \newline
33. योगो॒ हिꣳ हिं ॅयोगो॒ योगो॒ हिम् क॑रोति करोति॒ हिं ॅयोगो॒ योगो॒ हिम् क॑रोति । \newline
34. हिम् क॑रोति करोति॒ हिꣳ हिम् क॑रोति प्र॒जाः प्र॒जाः क॑रोति॒ हिꣳ हिम् क॑रोति प्र॒जाः । \newline
35. क॒रो॒ति॒ प्र॒जाः प्र॒जाः क॑रोति करोति प्र॒जा ए॒वैव प्र॒जाः क॑रोति करोति प्र॒जा ए॒व । \newline
36. प्र॒जा ए॒वैव प्र॒जाः प्र॒जा ए॒व तत् तदे॒व प्र॒जाः प्र॒जा ए॒व तत् । \newline
37. प्र॒जा इति॑ प्र - जाः । \newline
38. ए॒व तत् तदे॒वैव तद् यज॑मानो॒ यज॑मान॒ स्तदे॒वैव तद् यज॑मानः । \newline
39. तद् यज॑मानो॒ यज॑मान॒ स्तत् तद् यज॑मानः सृजते सृजते॒ यज॑मान॒ स्तत् तद् यज॑मानः सृजते । \newline
40. यज॑मानः सृजते सृजते॒ यज॑मानो॒ यज॑मानः सृजते॒ त्रि स्त्रिः सृ॑जते॒ यज॑मानो॒ यज॑मानः सृजते॒ त्रिः । \newline
41. सृ॒ज॒ते॒ त्रि स्त्रिः सृ॑जते सृजते॒ त्रिः प्र॑थ॒माम् प्र॑थ॒माम् त्रिः सृ॑जते सृजते॒ त्रिः प्र॑थ॒माम् । \newline
42. त्रिः प्र॑थ॒माम् प्र॑थ॒माम् त्रि स्त्रिः प्र॑थ॒मा मन्वनु॑ प्रथ॒माम् त्रि स्त्रिः प्र॑थ॒मा मनु॑ । \newline
43. प्र॒थ॒मा मन्वनु॑ प्रथ॒माम् प्र॑थ॒मा मन्वा॑हा॒हानु॑ प्रथ॒माम् प्र॑थ॒मा मन्वा॑ह । \newline
44. अन्वा॑हा॒हान्व न्वा॑ह॒ त्रि स्त्रिरा॒हा न्वन्वा॑ह॒ त्रिः । \newline
45. आ॒ह॒ त्रि स्त्रिरा॑हाह॒ त्रिरु॑त्त॒मा मु॑त्त॒माम् त्रिरा॑हाह॒ त्रिरु॑त्त॒माम् । \newline
46. त्रिरु॑त्त॒मा मु॑त्त॒माम् त्रि स्त्रिरु॑त्त॒मां ॅय॒ज्ञ्स्य॑ य॒ज्ञ्स्यो᳚त्त॒माम् त्रि स्त्रिरु॑त्त॒मां ॅय॒ज्ञ्स्य॑ । \newline
47. उ॒त्त॒मां ॅय॒ज्ञ्स्य॑ य॒ज्ञ्स्यो᳚त्त॒मा मु॑त्त॒मां ॅय॒ज्ञ्स्यै॒वैव य॒ज्ञ्स्यो᳚त्त॒मा मु॑त्त॒मां ॅय॒ज्ञ्स्यै॒व । \newline
48. उ॒त्त॒मामित्यु॑त् - त॒माम् । \newline
49. य॒ज्ञ्स्यै॒वैव य॒ज्ञ्स्य॑ य॒ज्ञ्स्यै॒व तत् तदे॒व य॒ज्ञ्स्य॑ य॒ज्ञ्स्यै॒व तत् । \newline
50. ए॒व तत् तदे॒वैव तद् ब॒र्॒.सम् ब॒र्॒.सम् तदे॒वैव तद् ब॒र्॒.सम् । \newline
51. तद् ब॒र्॒.सम् ब॒र्॒.सम् तत् तद् ब॒र्॒.सम् न॑ह्यति नह्यति ब॒र्॒.सम् तत् तद् ब॒र्॒.सम् न॑ह्यति । \newline
52. ब॒र्॒.सम् न॑ह्यति नह्यति ब॒र्॒.सम् ब॒र्॒.सम् न॑ह्य॒ त्यप्र॑स्रꣳसा॒या प्र॑स्रꣳसाय नह्यति ब॒र्॒.सम् ब॒र्॒.सम् न॑ह्य॒त्यप्र॑स्रꣳसाय । \newline
\pagebreak
\markright{ TS 2.5.7.2  \hfill https://www.vedavms.in \hfill}
\addcontentsline{toc}{section}{ TS 2.5.7.2 }
\section*{ TS 2.5.7.2 }

\textbf{TS 2.5.7.2 } \newline
\textbf{Samhita Paata} \newline

न॑ह्य॒त्यप्र॑स्रꣳसाय॒ संत॑त॒मन्वा॑ह प्रा॒णाना॑म॒न्नाद्य॑स्य॒ संत॑त्या॒ अथो॒ रक्ष॑सा॒मप॑हत्यै॒ राथ॑तंरीं प्रथ॒मामन्वा॑ह॒ राथ॑तंरो॒ वा अ॒यं ॅलो॒क इ॒ममे॒व लो॒कम॒भि ज॑यति॒ त्रिर्वि गृ॑ह्णाति॒ त्रय॑ इ॒मे लो॒का इ॒माने॒व लो॒कान॒भि ज॑यति॒ बार्.ह॑तीमुत्त॒मा-मन्वा॑ह॒ बार्.ह॑तो॒ वा अ॒सौ लो॒को॑ऽमुमे॒व लो॒कम॒भि ज॑यति॒ प्र वो॒ - [  ] \newline

\textbf{Pada Paata} \newline

न॒ह्य॒ति॒ । अप्र॑स्रꣳसा॒येत्यप्र॑ - स्रꣳ॒॒सा॒य॒ । संत॑त॒मिति॒ सं - त॒त॒म् । अन्विति॑ । आ॒ह॒ । प्रा॒णाना॒मिति॑ प्र - अ॒नाना᳚म् । अ॒न्नाद्य॒स्येत्य॑न्न - अद्य॑स्य । संत॑त्या॒ इति॒ सं - त॒त्यै॒ । अथो॒ इति॑ । रक्ष॑साम् । अप॑हत्या॒ इत्यप॑ - ह॒त्यै॒ । राथ॑न्तरी॒मिति॒ राथं᳚ - त॒री॒म् । प्र॒थ॒माम् । अन्विति॑ । आ॒ह॒ । राथ॑न्तर॒ इति॒ राथं᳚-त॒रः॒ । वै । अ॒यम् । लो॒कः । इ॒मम् । ए॒व । लो॒कम् । अ॒भीति॑ । ज॒य॒ति॒ । त्रिः । वीति॑ । गृ॒ह्णा॒ति॒ । त्रयः॑ । इ॒मे । लो॒काः । इ॒मान् । ए॒व । लो॒कान् । अ॒भीति॑ । ज॒य॒ति॒ । बार्.ह॑तीम् । उ॒त्त॒मामित्यु॑त् - त॒माम् । अन्विति॑ । आ॒ह॒ । बार्.ह॑तः । वै । अ॒सौ । लो॒कः । अ॒मुम् । ए॒व । लो॒कम् । अ॒भीति॑ । ज॒य॒ति॒ । प्रेति॑ । वः॒ ।  \newline


\textbf{Krama Paata} \newline

न॒ह्य॒त्यप्र॑स्रꣳसाय । अप्र॑स्रꣳसाय॒ सन्त॑तम् । अप्र॑स्रꣳसा॒येत्यप्र॑ - स्रꣳ॒॒सा॒य॒ । सन्त॑त॒मनु॑ । सन्त॑त॒मिति॒ सम् - त॒त॒म् । अन्वा॑ह । आ॒ह॒ प्रा॒णाना᳚म् । प्रा॒णाना॑म॒न्नाद्य॑स्य । प्रा॒णाना॒मिति॑ प्र - अ॒नाना᳚म् । अ॒न्नाद्य॑स्य॒ सन्त॑त्यै । अ॒न्नाद्य॒स्येत्य॑न्न - अद्य॑स्य । सन्त॑त्या॒ अथो᳚ । सन्त॑त्या॒ इति॒ सम् - त॒त्यै॒ । अथो॒ रक्ष॑साम् । अथो॒ इत्यथो᳚ । रक्ष॑सा॒मप॑हत्यै । अप॑हत्यै॒ राथ॑न्तरीम् । अप॑हत्या॒ इत्यप॑ - ह॒त्यै॒ । राथ॑न्तरीम् प्रथ॒माम् । राथ॑न्तरी॒मिति॒ राथ᳚म् - त॒री॒म् । प्र॒थ॒मामनु॑ । अन्वा॑ह । आ॒ह॒ राथ॑न्तरः । राथ॑न्तरो॒ वै । राथ॑न्तर॒ इति॒ राथ᳚म् - त॒रः॒ । वा अ॒यम् । अ॒यम् ॅलो॒कः । लो॒क इ॒मम् । इ॒ममे॒व । ए॒व लो॒कम् । लो॒कम॒भि । अ॒भि ज॑यति । ज॒य॒ति॒ त्रिः । त्रिर् वि । वि गृ॑ह्णाति । गृ॒ह्णा॒ति॒ त्रयः॑ । त्रय॑ इ॒मे । इ॒मे लो॒काः । लो॒का इ॒मान् । इ॒माने॒व । ए॒व लो॒कान् । लो॒कान॒भि । अ॒भि ज॑यति । ज॒य॒ति॒ बार्.ह॑तीम् । बार्.ह॑तीमुत्त॒माम् । उ॒त्त॒मामनु॑ । उ॒त्त॒मामित्यु॑त् - त॒माम् । अन्वा॑ह । आ॒ह॒ बार्.ह॑तः । बार्.ह॑तो॒ वै । वा अ॒सौ । अ॒सौ लो॒कः । लो॒को॑ऽमुम् । अ॒मुमे॒व । ए॒व लो॒कम् । लो॒कम॒भि । अ॒भि ज॑यति । ज॒य॒ति॒ प्र । प्र वः॑ । वो॒ वाजाः᳚ \newline

\textbf{Jatai Paata} \newline

1. न॒ह्य॒ त्यप्र॑स्रꣳसा॒या प्र॑स्रꣳसाय नह्यति नह्य॒ त्यप्र॑स्रꣳसाय । \newline
2. अप्र॑स्रꣳसाय॒ सन्त॑तꣳ॒॒ सन्त॑त॒ मप्र॑स्रꣳसा॒या प्र॑स्रꣳसाय॒ सन्त॑तम् । \newline
3. अप्र॑स्रꣳसा॒येत्यप्र॑ - स्रꣳ॒॒सा॒य॒ । \newline
4. सन्त॑त॒ मन्वनु॒ सन्त॑तꣳ॒॒ सन्त॑त॒ मनु॑ । \newline
5. सन्त॑त॒मिति॒ सं - त॒त॒म् । \newline
6. अन्वा॑ हा॒हा न्वन्वा॑ह । \newline
7. आ॒ह॒ प्रा॒णाना᳚म् प्रा॒णाना॑ माहाह प्रा॒णाना᳚म् । \newline
8. प्रा॒णाना॑ म॒न्नाद्य॑स्या॒ न्नाद्य॑स्य प्रा॒णाना᳚म् प्रा॒णाना॑ म॒न्नाद्य॑स्य । \newline
9. प्रा॒णाना॒मिति॑ प्र - अ॒नाना᳚म् । \newline
10. अ॒न्नाद्य॑स्य॒ सन्त॑त्यै॒ सन्त॑त्या अ॒न्नाद्य॑स्या॒ न्नाद्य॑स्य॒ सन्त॑त्यै । \newline
11. अ॒न्नाद्य॒स्येत्य॑न्न - अद्य॑स्य । \newline
12. सन्त॑त्या॒ अथो॒ अथो॒ सन्त॑त्यै॒ सन्त॑त्या॒ अथो᳚ । \newline
13. सन्त॑त्या॒ इति॒ सं - त॒त्यै॒ । \newline
14. अथो॒ रक्ष॑साꣳ॒॒ रक्ष॑सा॒ मथो॒ अथो॒ रक्ष॑साम् । \newline
15. अथो॒ इत्यथो᳚ । \newline
16. रक्ष॑सा॒ मप॑हत्या॒ अप॑हत्यै॒ रक्ष॑साꣳ॒॒ रक्ष॑सा॒ मप॑हत्यै । \newline
17. अप॑हत्यै॒ राथ॑न्तरीꣳ॒॒ राथ॑न्तरी॒ मप॑हत्या॒ अप॑हत्यै॒ राथ॑न्तरीम् । \newline
18. अप॑हत्या॒ इत्यप॑ - ह॒त्यै॒ । \newline
19. राथ॑न्तरीम् प्रथ॒माम् प्र॑थ॒माꣳ राथ॑न्तरीꣳ॒॒ राथ॑न्तरीम् प्रथ॒माम् । \newline
20. राथ॑न्तरी॒मिति॒ राथं᳚ - त॒री॒म् । \newline
21. प्र॒थ॒मा मन्वनु॑ प्रथ॒माम् प्र॑थ॒मा मनु॑ । \newline
22. अन्वा॑ हा॒हा न्वन्वा॑ह । \newline
23. आ॒ह॒ राथ॑न्तरो॒ राथ॑न्तर आहाह॒ राथ॑न्तरः । \newline
24. राथ॑न्तरो॒ वै वै राथ॑न्तरो॒ राथ॑न्तरो॒ वै । \newline
25. राथ॑न्तर॒ इति॒ राथं᳚ - त॒रः॒ । \newline
26. वा अ॒य म॒यं ॅवै वा अ॒यम् । \newline
27. अ॒यम् ॅलो॒को लो॒को॑ ऽय म॒यम् ॅलो॒कः । \newline
28. लो॒क इ॒म मि॒मम् ॅलो॒को लो॒क इ॒मम् । \newline
29. इ॒म मे॒वैवे म मि॒म मे॒व । \newline
30. ए॒व लो॒कम् ॅलो॒क मे॒वैव लो॒कम् । \newline
31. लो॒क म॒भ्य॑भि लो॒कम् ॅलो॒क म॒भि । \newline
32. अ॒भि ज॑यति जय त्य॒भ्य॑भि ज॑यति । \newline
33. ज॒य॒ति॒ त्रि स्त्रिर् ज॑यति जयति॒ त्रिः । \newline
34. त्रिर् वि वि त्रि स्त्रिर् वि । \newline
35. वि गृ॑ह्णाति गृह्णाति॒ वि वि गृ॑ह्णाति । \newline
36. गृ॒ह्णा॒ति॒ त्रय॒ स्त्रयो॑ गृह्णाति गृह्णाति॒ त्रयः॑ । \newline
37. त्रय॑ इ॒म इ॒मे त्रय॒ स्त्रय॑ इ॒मे । \newline
38. इ॒मे लो॒का लो॒का इ॒म इ॒मे लो॒काः । \newline
39. लो॒का इ॒मा नि॒मान् ॅलो॒का लो॒का इ॒मान् । \newline
40. इ॒मा ने॒वैवे मा नि॒मा ने॒व । \newline
41. ए॒व लो॒कान् ॅलो॒का ने॒वैव लो॒कान् । \newline
42. लो॒का न॒भ्य॑भि लो॒कान् ॅलो॒का न॒भि । \newline
43. अ॒भि ज॑यति जय त्य॒भ्य॑भि ज॑यति । \newline
44. ज॒य॒ति॒ बार्.ह॑ती॒म् बार्.ह॑तीम् जयति जयति॒ बार्.ह॑तीम् । \newline
45. बार्.ह॑ती मुत्त॒मा मु॑त्त॒माम् बार्.ह॑ती॒म् बार्.ह॑ती मुत्त॒माम् । \newline
46. उ॒त्त॒मा मन्वनू᳚त्त॒मा मु॑त्त॒मा मनु॑ । \newline
47. उ॒त्त॒मामित्यु॑त् - त॒माम् । \newline
48. अन्वा॑ हा॒हा न्वन्वा॑ह । \newline
49. आ॒ह॒ बार्.ह॑तो॒ बार्.ह॑त आहाह॒ बार्.ह॑तः । \newline
50. बार्.ह॑तो॒ वै वै बार्.ह॑तो॒ बार्.ह॑तो॒ वै । \newline
51. वा अ॒सा व॒सौ वै वा अ॒सौ । \newline
52. अ॒सौ लो॒को लो॒को॑ ऽसा व॒सौ लो॒कः । \newline
53. लो॒को॑ ऽमु म॒मुम् ॅलो॒को लो॒को॑ ऽमुम् । \newline
54. अ॒मु मे॒वैवामु म॒मु मे॒व । \newline
55. ए॒व लो॒कम् ॅलो॒क मे॒वैव लो॒कम् । \newline
56. लो॒क म॒भ्य॑भि लो॒कम् ॅलो॒क म॒भि । \newline
57. अ॒भि ज॑यति जय त्य॒भ्य॑भि ज॑यति । \newline
58. ज॒य॒ति॒ प्र प्र ज॑यति जयति॒ प्र । \newline
59. प्र वो॑ वः॒ प्र प्र वः॑ । \newline
60. वो॒ वाजा॒ वाजा॑ वो वो॒ वाजाः᳚ । \newline

\textbf{Ghana Paata } \newline

1. न॒ह्य॒त्यप्र॑स्रꣳसा॒या प्र॑स्रꣳसाय नह्यति नह्य॒त्यप्र॑स्रꣳसाय॒ सन्त॑तꣳ॒॒ सन्त॑त॒ मप्र॑स्रꣳसाय नह्यति नह्य॒त्यप्र॑स्रꣳसाय॒ सन्त॑तम् । \newline
2. अप्र॑स्रꣳसाय॒ सन्त॑तꣳ॒॒ सन्त॑त॒ मप्र॑स्रꣳसा॒या प्र॑स्रꣳसाय॒ सन्त॑त॒ मन्वनु॒ सन्त॑त॒ मप्र॑स्रꣳसा॒याप्र॑स्रꣳसाय॒ सन्त॑त॒ मनु॑ । \newline
3. अप्र॑स्रꣳसा॒येत्यप्र॑ - स्रꣳ॒॒सा॒य॒ । \newline
4. सन्त॑त॒ मन्वनु॒ सन्त॑तꣳ॒॒ सन्त॑त॒ मन्वा॑हा॒हानु॒ सन्त॑तꣳ॒॒ सन्त॑त॒ मन्वा॑ह । \newline
5. सन्त॑त॒मिति॒ सं - त॒त॒म् । \newline
6. अन्वा॑हा॒हा न्वन्वा॑ह प्रा॒णाना᳚म् प्रा॒णाना॑ मा॒हा न्वन्वा॑ह प्रा॒णाना᳚म् । \newline
7. आ॒ह॒ प्रा॒णाना᳚म् प्रा॒णाना॑ माहाह प्रा॒णाना॑ म॒न्नाद्य॑स्या॒ न्नाद्य॑स्य प्रा॒णाना॑ माहाह प्रा॒णाना॑ म॒न्नाद्य॑स्य । \newline
8. प्रा॒णाना॑ म॒न्नाद्य॑स्या॒ न्नाद्य॑स्य प्रा॒णाना᳚म् प्रा॒णाना॑ म॒न्नाद्य॑स्य॒ सन्त॑त्यै॒ सन्त॑त्या अ॒न्नाद्य॑स्य प्रा॒णाना᳚म् प्रा॒णाना॑ म॒न्नाद्य॑स्य॒ सन्त॑त्यै । \newline
9. प्रा॒णाना॒मिति॑ प्र - अ॒नाना᳚म् । \newline
10. अ॒न्नाद्य॑स्य॒ सन्त॑त्यै॒ सन्त॑त्या अ॒न्नाद्य॑स्या॒ न्नाद्य॑स्य॒ सन्त॑त्या॒ अथो॒ अथो॒ सन्त॑त्या अ॒न्नाद्य॑स्या॒ न्नाद्य॑स्य॒ सन्त॑त्या॒ अथो᳚ । \newline
11. अ॒न्नाद्य॒स्येत्य॑न्न - अद्य॑स्य । \newline
12. सन्त॑त्या॒ अथो॒ अथो॒ सन्त॑त्यै॒ सन्त॑त्या॒ अथो॒ रक्ष॑साꣳ॒॒ रक्ष॑सा॒ मथो॒ सन्त॑त्यै॒ सन्त॑त्या॒ अथो॒ रक्ष॑साम् । \newline
13. सन्त॑त्या॒ इति॒ सं - त॒त्यै॒ । \newline
14. अथो॒ रक्ष॑साꣳ॒॒ रक्ष॑सा॒ मथो॒ अथो॒ रक्ष॑सा॒ मप॑हत्या॒ अप॑हत्यै॒ रक्ष॑सा॒ मथो॒ अथो॒ रक्ष॑सा॒ मप॑हत्यै । \newline
15. अथो॒ इत्यथो᳚ । \newline
16. रक्ष॑सा॒ मप॑हत्या॒ अप॑हत्यै॒ रक्ष॑साꣳ॒॒ रक्ष॑सा॒ मप॑हत्यै॒ राथ॑न्तरीꣳ॒॒ राथ॑न्तरी॒ मप॑हत्यै॒ रक्ष॑साꣳ॒॒ रक्ष॑सा॒ मप॑हत्यै॒ राथ॑न्तरीम् । \newline
17. अप॑हत्यै॒ राथ॑न्तरीꣳ॒॒ राथ॑न्तरी॒ मप॑हत्या॒ अप॑हत्यै॒ राथ॑न्तरीम् प्रथ॒माम् प्र॑थ॒माꣳ राथ॑न्तरी॒ मप॑हत्या॒ अप॑हत्यै॒ राथ॑न्तरीम् प्रथ॒माम् । \newline
18. अप॑हत्या॒ इत्यप॑ - ह॒त्यै॒ । \newline
19. राथ॑न्तरीम् प्रथ॒माम् प्र॑थ॒माꣳ राथ॑न्तरीꣳ॒॒ राथ॑न्तरीम् प्रथ॒मा मन्वनु॑ प्रथ॒माꣳ राथ॑न्तरीꣳ॒॒ राथ॑न्तरीम् प्रथ॒मा मनु॑ । \newline
20. राथ॑न्तरी॒मिति॒ राथं᳚ - त॒री॒म् । \newline
21. प्र॒थ॒मा मन्वनु॑ प्रथ॒माम् प्र॑थ॒मा मन्वा॑हा॒हानु॑ प्रथ॒माम् प्र॑थ॒मा मन्वा॑ह । \newline
22. अन्वा॑हा॒हा न्वन्वा॑ह॒ राथ॑न्तरो॒ राथ॑न्तर आ॒हा न्वन्वा॑ह॒ राथ॑न्तरः । \newline
23. आ॒ह॒ राथ॑न्तरो॒ राथ॑न्तर आहाह॒ राथ॑न्तरो॒ वै वै राथ॑न्तर आहाह॒ राथ॑न्तरो॒ वै । \newline
24. राथ॑न्तरो॒ वै वै राथ॑न्तरो॒ राथ॑न्तरो॒ वा अ॒य म॒यं ॅवै राथ॑न्तरो॒ राथ॑न्तरो॒ वा अ॒यम् । \newline
25. राथ॑न्तर॒ इति॒ राथं᳚ - त॒रः॒ । \newline
26. वा अ॒य म॒यं ॅवै वा अ॒यम् ॅलो॒को लो॒को॑ ऽयं ॅवै वा अ॒यम् ॅलो॒कः । \newline
27. अ॒यम् ॅलो॒को लो॒को॑ ऽय म॒यम् ॅलो॒क इ॒म मि॒मम् ॅलो॒को॑ ऽय म॒यम् ॅलो॒क इ॒मम् । \newline
28. लो॒क इ॒म मि॒मम् ॅलो॒को लो॒क इ॒म मे॒वैवे मम् ॅलो॒को लो॒क इ॒म मे॒व । \newline
29. इ॒म मे॒वैवे म मि॒म मे॒व लो॒कम् ॅलो॒क मे॒वे म मि॒म मे॒व लो॒कम् । \newline
30. ए॒व लो॒कम् ॅलो॒क मे॒वैव लो॒क म॒भ्य॑भि लो॒क मे॒वैव लो॒क म॒भि । \newline
31. लो॒क म॒भ्य॑भि लो॒कम् ॅलो॒क म॒भि ज॑यति जयत्य॒भि लो॒कम् ॅलो॒क म॒भि ज॑यति । \newline
32. अ॒भि ज॑यति जयत्य॒भ्य॑भि ज॑यति॒ त्रि स्त्रिर् ज॑यत्य॒भ्य॑भि ज॑यति॒ त्रिः । \newline
33. ज॒य॒ति॒ त्रि स्त्रिर् ज॑यति जयति॒ त्रिर् वि वि त्रिर् ज॑यति जयति॒ त्रिर् वि । \newline
34. त्रिर् वि वि त्रि स्त्रिर् वि गृ॑ह्णाति गृह्णाति॒ वि त्रि स्त्रिर् वि गृ॑ह्णाति । \newline
35. वि गृ॑ह्णाति गृह्णाति॒ वि वि गृ॑ह्णाति॒ त्रय॒ स्त्रयो॑ गृह्णाति॒ वि वि गृ॑ह्णाति॒ त्रयः॑ । \newline
36. गृ॒ह्णा॒ति॒ त्रय॒ स्त्रयो॑ गृह्णाति गृह्णाति॒ त्रय॑ इ॒म इ॒मे त्रयो॑ गृह्णाति गृह्णाति॒ त्रय॑ इ॒मे । \newline
37. त्रय॑ इ॒म इ॒मे त्रय॒ स्त्रय॑ इ॒मे लो॒का लो॒का इ॒मे त्रय॒ स्त्रय॑ इ॒मे लो॒काः । \newline
38. इ॒मे लो॒का लो॒का इ॒म इ॒मे लो॒का इ॒मा नि॒मान् ॅलो॒का इ॒म इ॒मे लो॒का इ॒मान् । \newline
39. लो॒का इ॒मा नि॒मान् ॅलो॒का लो॒का इ॒मा ने॒वैवे मान् ॅलो॒का लो॒का इ॒मा ने॒व । \newline
40. इ॒मा ने॒वैवे मा नि॒मा ने॒व लो॒कान् ॅलो॒का ने॒वे मा नि॒मा ने॒व लो॒कान् । \newline
41. ए॒व लो॒कान् ॅलो॒का ने॒वैव लो॒का न॒भ्य॑भि लो॒का ने॒वैव लो॒का न॒भि । \newline
42. लो॒का न॒भ्य॑भि लो॒कान् ॅलो॒का न॒भि ज॑यति जयत्य॒भि लो॒कान् ॅलो॒का न॒भि ज॑यति । \newline
43. अ॒भि ज॑यति जयत्य॒भ्य॑भि ज॑यति॒ बार्.ह॑ती॒म् बार्.ह॑तीम् जयत्य॒भ्य॑भि ज॑यति॒ बार्.ह॑तीम् । \newline
44. ज॒य॒ति॒ बार्.ह॑ती॒म् बार्.ह॑तीम् जयति जयति॒ बार्.ह॑ती मुत्त॒मा मु॑त्त॒माम् बार्.ह॑तीम् जयति जयति॒ बार्.ह॑ती मुत्त॒माम् । \newline
45. बार्.ह॑ती मुत्त॒मा मु॑त्त॒माम् बार्.ह॑ती॒म् बार्.ह॑ती मुत्त॒मा मन्वनू᳚त्त॒माम् बार्.ह॑ती॒म् बार्.ह॑ती मुत्त॒मा मनु॑ । \newline
46. उ॒त्त॒मा मन्वनू᳚त्त॒मा मु॑त्त॒मा मन्वा॑हा॒हा नू᳚त्त॒मा मु॑त्त॒मा मन्वा॑ह । \newline
47. उ॒त्त॒मामित्यु॑त् - त॒माम् । \newline
48. अन्वा॑हा॒हा न्वन्वा॑ह॒ बार्.ह॑तो॒ बार्.ह॑त आ॒हा न्वन्वा॑ह॒ बार्.ह॑तः । \newline
49. आ॒ह॒ बार्.ह॑तो॒ बार्.ह॑त आहाह॒ बार्.ह॑तो॒ वै वै बार्.ह॑त आहाह॒ बार्.ह॑तो॒ वै । \newline
50. बार्.ह॑तो॒ वै वै बार्.ह॑तो॒ बार्.ह॑तो॒ वा अ॒सा व॒सौ वै बार्.ह॑तो॒ बार्.ह॑तो॒ वा अ॒सौ । \newline
51. वा अ॒सा व॒सौ वै वा अ॒सौ लो॒को लो॒को॑ ऽसौ वै वा अ॒सौ लो॒कः । \newline
52. अ॒सौ लो॒को लो॒को॑ ऽसा व॒सौ लो॒को॑ ऽमु म॒मुम् ॅलो॒को॑ ऽसा व॒सौ लो॒को॑ ऽमुम् । \newline
53. लो॒को॑ ऽमु म॒मुम् ॅलो॒को लो॒को॑ ऽमु मे॒वैवामुम् ॅलो॒को लो॒को॑ ऽमु मे॒व । \newline
54. अ॒मु मे॒वैवामु म॒मु मे॒व लो॒कम् ॅलो॒क मे॒वामु म॒मु मे॒व लो॒कम् । \newline
55. ए॒व लो॒कम् ॅलो॒क मे॒वैव लो॒क म॒भ्य॑भि लो॒क मे॒वैव लो॒क म॒भि । \newline
56. लो॒क म॒भ्य॑भि लो॒कम् ॅलो॒क म॒भि ज॑यति जयत्य॒भि लो॒कम् ॅलो॒क म॒भि ज॑यति । \newline
57. अ॒भि ज॑यति जयत्य॒भ्य॑भि ज॑यति॒ प्र प्र ज॑यत्य॒भ्य॑भि ज॑यति॒ प्र । \newline
58. ज॒य॒ति॒ प्र प्र ज॑यति जयति॒ प्र वो॑ वः॒ प्र ज॑यति जयति॒ प्र वः॑ । \newline
59. प्र वो॑ वः॒ प्र प्र वो॒ वाजा॒ वाजा॑ वः॒ प्र प्र वो॒ वाजाः᳚ । \newline
60. वो॒ वाजा॒ वाजा॑ वो वो॒ वाजा॒ इतीति॒ वाजा॑ वो वो॒ वाजा॒ इति॑ । \newline
\pagebreak
\markright{ TS 2.5.7.3  \hfill https://www.vedavms.in \hfill}
\addcontentsline{toc}{section}{ TS 2.5.7.3 }
\section*{ TS 2.5.7.3 }

\textbf{TS 2.5.7.3 } \newline
\textbf{Samhita Paata} \newline

वाजा॒ इत्यनि॑रुक्तां प्राजाप॒त्यामन्वा॑ह य॒ज्ञो वै प्र॒जाप॑तिर्य॒ज्ञ्मे॒व प्र॒जाप॑ति॒मा र॑भते॒ प्रवो॒ वाजा॒ इत्यन्वा॒हान्नं॒ ॅवै वाजोऽन्न॑मे॒वाव॑ रुन्धे॒ प्रवो॒ वाजा॒ इत्यन्वा॑ह॒ तस्मा᳚त् प्रा॒चीनꣳ॒॒ रेतो॑ धीय॒तेऽग्न॒ आ या॑हि वी॒तय॒ इत्या॑ह॒ तस्मा᳚त् प्र॒तीचीः᳚ प्र॒जा जा॑यन्ते॒ प्रवो॒ वाजा॒ - [  ] \newline

\textbf{Pada Paata} \newline

वाजाः᳚ । इति॑ । अनि॑रुक्ता॒मित्यनिः॑ - उ॒क्ता॒म् । प्रा॒जा॒प॒त्यामिति॑ प्राजा - प॒त्याम् । अन्विति॑ । आ॒ह॒ । य॒ज्ञ्ः । वै । प्र॒जाप॑ति॒रिति॑ प्र॒जा - प॒तिः॒ । य॒ज्ञ्म् । ए॒व । प्र॒जाप॑ति॒मिति॑ प्र॒जा-प॒ति॒म् । एति॑ । र॒भ॒ते॒ । प्रेति॑ । वः॒ । वाजाः᳚ । इति॑ । अन्विति॑ । आ॒ह॒ । अन्न᳚म् । वै । वाजः॑ । अन्न᳚म् । ए॒व । अवेति॑ । रु॒न्धे॒ । प्रेति॑ । वः॒ । वाजाः᳚ । इति॑ । अन्विति॑ । आ॒ह॒ । तस्मा᳚त् । प्रा॒चीन᳚म् । रेतः॑ । धी॒य॒ते॒ । अग्ने᳚ । एति॑ । या॒हि॒ । वी॒तये᳚ । इति॑ । आ॒ह॒ । तस्मा᳚त् । प्र॒तीचीः᳚ । प्र॒जा इति॑ प्र - जाः । जा॒य॒न्ते॒ । प्रेति॑ । वः॒ । वाजाः᳚ ।  \newline


\textbf{Krama Paata} \newline

वाजा॒ इति॑ । इत्यनि॑रुक्ताम् । अनि॑रुक्ताम् प्राजाप॒त्याम् । अनि॑रुक्ता॒मित्यनिः॑ - उ॒क्ता॒म् । प्रा॒जा॒प॒त्यामनु॑ । प्रा॒जा॒प॒त्यामिति॑ प्राजा - प॒त्याम् । अन्वा॑ह । आ॒ह॒ य॒ज्ञ्ः । य॒ज्ञो वै । वै प्र॒जाप॑तिः । प्र॒जाप॑तिर् य॒ज्ञ्म् । प्र॒जाप॑ति॒रिति॑ प्र॒जा - प॒तिः॒ । य॒ज्ञ्मे॒व । ए॒व प्र॒जाप॑तिम् । प्र॒जाप॑ति॒ मा । प्र॒जाप॑ति॒मिति॑ प्र॒जा - प॒ति॒म् । आ र॑भते । र॒भ॒ते॒ प्र । प्र वः॑ । वो॒ वाजाः᳚ । वाजा॒ इति॑ । इत्यनु॑ । अन्वा॑ह । आ॒हान्न᳚म् । अन्न॒म् ॅवै । वै वाजः॑ । वाजोऽन्न᳚म् । अन्न॑मे॒व । ए॒वाव॑ । अव॑ रुन्धे । रु॒न्धे॒ प्र । प्र वः॑ । वो॒ वाजाः᳚ । वाजा॒ इति॑ । इत्यनु॑ । अन्वा॑ह । आ॒ह॒ तस्मा᳚त् । तस्मा᳚त् प्रा॒चीन᳚म् । प्रा॒चीनꣳ॒॒ रेतः॑ । रेतो॑ धीयते । धी॒य॒ते ऽग्ने᳚ । अग्न॒ आ । आ या॑हि । या॒हि॒ वी॒तये᳚ । वी॒तय॒ इति॑ । इत्या॑ह । आ॒ह॒ तस्मा᳚त् । तस्मा᳚त् प्र॒तीचीः᳚ । प्र॒तीचीः᳚ प्र॒जाः । प्र॒जा जा॑यन्ते । प्र॒जा इति॑ प्र - जाः । जा॒य॒न्ते॒ प्र । प्र वः॑ । वो॒ वाजाः᳚ । वाजा॒ इति॑ \newline

\textbf{Jatai Paata} \newline

1. वाजा॒ इतीति॒ वाजा॒ वाजा॒ इति॑ । \newline
2. इत्यनि॑रुक्ता॒ मनि॑रुक्ता॒ मिती त्यनि॑रुक्ताम् । \newline
3. अनि॑रुक्ताम् प्राजाप॒त्याम् प्रा॑जाप॒त्या मनि॑रुक्ता॒ मनि॑रुक्ताम् प्राजाप॒त्याम् । \newline
4. अनि॑रुक्ता॒मित्यनिः॑ - उ॒क्ता॒म् । \newline
5. प्रा॒जा॒प॒त्या मन्वनु॑ प्राजाप॒त्याम् प्रा॑जाप॒त्या मनु॑ । \newline
6. प्रा॒जा॒प॒त्यामिति॑ प्राजा - प॒त्याम् । \newline
7. अन्वा॑ हा॒हा न्वन्वा॑ह । \newline
8. आ॒ह॒ य॒ज्ञो य॒ज्ञ् आ॑हाह य॒ज्ञ्ः । \newline
9. य॒ज्ञो वै वै य॒ज्ञो य॒ज्ञो वै । \newline
10. वै प्र॒जाप॑तिः प्र॒जाप॑ति॒र् वै वै प्र॒जाप॑तिः । \newline
11. प्र॒जाप॑तिर् य॒ज्ञ्ं ॅय॒ज्ञ्म् प्र॒जाप॑तिः प्र॒जाप॑तिर् य॒ज्ञ्म् । \newline
12. प्र॒जाप॑ति॒रिति॑ प्र॒जा - प॒तिः॒ । \newline
13. य॒ज्ञ् मे॒वैव य॒ज्ञ्ं ॅय॒ज्ञ् मे॒व । \newline
14. ए॒व प्र॒जाप॑तिम् प्र॒जाप॑ति मे॒वैव प्र॒जाप॑तिम् । \newline
15. प्र॒जाप॑ति॒ मा प्र॒जाप॑तिम् प्र॒जाप॑ति॒ मा । \newline
16. प्र॒जाप॑ति॒मिति॑ प्र॒जा - प॒ति॒म् । \newline
17. आ र॑भते रभत॒ आ र॑भते । \newline
18. र॒भ॒ते॒ प्र प्र र॑भते रभते॒ प्र । \newline
19. प्र वो॑ वः॒ प्र प्र वः॑ । \newline
20. वो॒ वाजा॒ वाजा॑ वो वो॒ वाजाः᳚ । \newline
21. वाजा॒ इतीति॒ वाजा॒ वाजा॒ इति॑ । \newline
22. इत्यन्वन्विती त्यनु॑ । \newline
23. अन्वा॑ हा॒हा न्वन्वा॑ह । \newline
24. आ॒हान्न॒ मन्न॑ माहा॒हा न्न᳚म् । \newline
25. अन्नं॒ ॅवै वा अन्न॒ मन्नं॒ ॅवै । \newline
26. वै वाजो॒ वाजो॒ वै वै वाजः॑ । \newline
27. वाजो ऽन्न॒ मन्नं॒ ॅवाजो॒ वाजो ऽन्न᳚म् । \newline
28. अन्न॑ मे॒वैवान्न॒ मन्न॑ मे॒व । \newline
29. ए॒वावा वै॒वैवाव॑ । \newline
30. अव॑ रुन्धे रु॒न्धे ऽवाव॑ रुन्धे । \newline
31. रु॒न्धे॒ प्र प्र रु॑न्धे रुन्धे॒ प्र । \newline
32. प्र वो॑ वः॒ प्र प्र वः॑ । \newline
33. वो॒ वाजा॒ वाजा॑ वो वो॒ वाजाः᳚ । \newline
34. वाजा॒ इतीति॒ वाजा॒ वाजा॒ इति॑ । \newline
35. इत्यन्वन्विती त्यनु॑ । \newline
36. अन्वा॑ हा॒हा न्वन्वा॑ह । \newline
37. आ॒ह॒ तस्मा॒त् तस्मा॑ दाहाह॒ तस्मा᳚त् । \newline
38. तस्मा᳚त् प्रा॒चीन॑म् प्रा॒चीन॒म् तस्मा॒त् तस्मा᳚त् प्रा॒चीन᳚म् । \newline
39. प्रा॒चीनꣳ॒॒ रेतो॒ रेतः॑ प्रा॒चीन॑म् प्रा॒चीनꣳ॒॒ रेतः॑ । \newline
40. रेतो॑ धीयते धीयते॒ रेतो॒ रेतो॑ धीयते । \newline
41. धी॒य॒ते ऽग्ने ऽग्ने॑ धीयते धीय॒ते ऽग्ने᳚ । \newline
42. अग्न॒ आ ऽग्ने ऽग्न॒ आ । \newline
43. आ या॑हि या॒ह्या या॑हि । \newline
44. या॒हि॒ वी॒तये॑ वी॒तये॑ याहि याहि वी॒तये᳚ । \newline
45. वी॒तय॒ इतीति॑ वी॒तये॑ वी॒तय॒ इति॑ । \newline
46. इत्या॑हा॒हे तीत्या॑ह । \newline
47. आ॒ह॒ तस्मा॒त् तस्मा॑ दाहाह॒ तस्मा᳚त् । \newline
48. तस्मा᳚त् प्र॒तीचीः᳚ प्र॒तीची॒ स्तस्मा॒त् तस्मा᳚त् प्र॒तीचीः᳚ । \newline
49. प्र॒तीचीः᳚ प्र॒जाः प्र॒जाः प्र॒तीचीः᳚ प्र॒तीचीः᳚ प्र॒जाः । \newline
50. प्र॒जा जा॑यन्ते जायन्ते प्र॒जाः प्र॒जा जा॑यन्ते । \newline
51. प्र॒जा इति॑ प्र - जाः । \newline
52. जा॒य॒न्ते॒ प्र प्र जा॑यन्ते जायन्ते॒ प्र । \newline
53. प्र वो॑ वः॒ प्र प्र वः॑ । \newline
54. वो॒ वाजा॒ वाजा॑ वो वो॒ वाजाः᳚ । \newline
55. वाजा॒ इतीति॒ वाजा॒ वाजा॒ इति॑ । \newline

\textbf{Ghana Paata } \newline

1. वाजा॒ इतीति॒ वाजा॒ वाजा॒ इत्यनि॑रुक्ता॒ मनि॑रुक्ता॒ मिति॒ वाजा॒ वाजा॒ इत्यनि॑रुक्ताम् । \newline
2. इत्यनि॑रुक्ता॒ मनि॑रुक्ता॒ मितीत्यनि॑रुक्ताम् प्राजाप॒त्याम् प्रा॑जाप॒त्या मनि॑रुक्ता॒ मितीत्यनि॑रुक्ताम् प्राजाप॒त्याम् । \newline
3. अनि॑रुक्ताम् प्राजाप॒त्याम् प्रा॑जाप॒त्या मनि॑रुक्ता॒ मनि॑रुक्ताम् प्राजाप॒त्या मन्वनु॑ प्राजाप॒त्या मनि॑रुक्ता॒ मनि॑रुक्ताम् प्राजाप॒त्या मनु॑ । \newline
4. अनि॑रुक्ता॒मित्यनिः॑ - उ॒क्ता॒म् । \newline
5. प्रा॒जा॒प॒त्या मन्वनु॑ प्राजाप॒त्याम् प्रा॑जाप॒त्या मन्वा॑हा॒हानु॑ प्राजाप॒त्याम् प्रा॑जाप॒त्या मन्वा॑ह । \newline
6. प्रा॒जा॒प॒त्यामिति॑ प्राजा - प॒त्याम् । \newline
7. अन्वा॑हा॒हा न्वन्वा॑ह य॒ज्ञो य॒ज्ञ् आ॒हा न्वन्वा॑ह य॒ज्ञ्ः । \newline
8. आ॒ह॒ य॒ज्ञो य॒ज्ञ् आ॑हाह य॒ज्ञो वै वै य॒ज्ञ् आ॑हाह य॒ज्ञो वै । \newline
9. य॒ज्ञो वै वै य॒ज्ञो य॒ज्ञो वै प्र॒जाप॑तिः प्र॒जाप॑ति॒र् वै य॒ज्ञो य॒ज्ञो वै प्र॒जाप॑तिः । \newline
10. वै प्र॒जाप॑तिः प्र॒जाप॑ति॒र् वै वै प्र॒जाप॑तिर् य॒ज्ञ्ं ॅय॒ज्ञ्म् प्र॒जाप॑ति॒र् वै वै प्र॒जाप॑तिर् य॒ज्ञ्म् । \newline
11. प्र॒जाप॑तिर् य॒ज्ञ्ं ॅय॒ज्ञ्म् प्र॒जाप॑तिः प्र॒जाप॑तिर् य॒ज्ञ् मे॒वैव य॒ज्ञ्म् प्र॒जाप॑तिः प्र॒जाप॑तिर् य॒ज्ञ् मे॒व । \newline
12. प्र॒जाप॑ति॒रिति॑ प्र॒जा - प॒तिः॒ । \newline
13. य॒ज्ञ् मे॒वैव य॒ज्ञ्ं ॅय॒ज्ञ् मे॒व प्र॒जाप॑तिम् प्र॒जाप॑ति मे॒व य॒ज्ञ्ं ॅय॒ज्ञ् मे॒व प्र॒जाप॑तिम् । \newline
14. ए॒व प्र॒जाप॑तिम् प्र॒जाप॑ति मे॒वैव प्र॒जाप॑ति॒ मा प्र॒जाप॑ति मे॒वैव प्र॒जाप॑ति॒ मा । \newline
15. प्र॒जाप॑ति॒ मा प्र॒जाप॑तिम् प्र॒जाप॑ति॒ मा र॑भते रभत॒ आ प्र॒जाप॑तिम् प्र॒जाप॑ति॒ मा र॑भते । \newline
16. प्र॒जाप॑ति॒मिति॑ प्र॒जा - प॒ति॒म् । \newline
17. आ र॑भते रभत॒ आ र॑भते॒ प्र प्र र॑भत॒ आ र॑भते॒ प्र । \newline
18. र॒भ॒ते॒ प्र प्र र॑भते रभते॒ प्र वो॑ वः॒ प्र र॑भते रभते॒ प्र वः॑ । \newline
19. प्र वो॑ वः॒ प्र प्र वो॒ वाजा॒ वाजा॑ वः॒ प्र प्र वो॒ वाजाः᳚ । \newline
20. वो॒ वाजा॒ वाजा॑ वो वो॒ वाजा॒ इतीति॒ वाजा॑ वो वो॒ वाजा॒ इति॑ । \newline
21. वाजा॒ इतीति॒ वाजा॒ वाजा॒ इत्यन्वन्विति॒ वाजा॒ वाजा॒ इत्यनु॑ । \newline
22. इत्यन्वन्विती त्यन्वा॑हा॒हा न्विती त्यन्वा॑ह । \newline
23. अन्वा॑हा॒हा न्वन्वा॒हान्न॒ मन्न॑ मा॒हा न्वन्वा॒हा न्न᳚म् । \newline
24. आ॒हान्न॒ मन्न॑ माहा॒हान्नं॒ ॅवै वा अन्न॑ माहा॒हान्नं॒ ॅवै । \newline
25. अन्नं॒ ॅवै वा अन्न॒ मन्नं॒ ॅवै वाजो॒ वाजो॒ वा अन्न॒ मन्नं॒ ॅवै वाजः॑ । \newline
26. वै वाजो॒ वाजो॒ वै वै वाजो ऽन्न॒ मन्नं॒ ॅवाजो॒ वै वै वाजो ऽन्न᳚म् । \newline
27. वाजो ऽन्न॒ मन्नं॒ ॅवाजो॒ वाजो ऽन्न॑ मे॒वैवान्नं॒ ॅवाजो॒ वाजो ऽन्न॑ मे॒व । \newline
28. अन्न॑ मे॒वैवान्न॒ मन्न॑ मे॒वावा वै॒वान्न॒ मन्न॑ मे॒वाव॑ । \newline
29. ए॒वावा वै॒वैवाव॑ रुन्धे रु॒न्धे ऽवै॒वैवाव॑ रुन्धे । \newline
30. अव॑ रुन्धे रु॒न्धे ऽवाव॑ रुन्धे॒ प्र प्र रु॒न्धे ऽवाव॑ रुन्धे॒ प्र । \newline
31. रु॒न्धे॒ प्र प्र रु॑न्धे रुन्धे॒ प्र वो॑ वः॒ प्र रु॑न्धे रुन्धे॒ प्र वः॑ । \newline
32. प्र वो॑ वः॒ प्र प्र वो॒ वाजा॒ वाजा॑ वः॒ प्र प्र वो॒ वाजाः᳚ । \newline
33. वो॒ वाजा॒ वाजा॑ वो वो॒ वाजा॒ इतीति॒ वाजा॑ वो वो॒ वाजा॒ इति॑ । \newline
34. वाजा॒ इतीति॒ वाजा॒ वाजा॒ इत्यन्वन्विति॒ वाजा॒ वाजा॒ इत्यनु॑ । \newline
35. इत्यन्वन्विती त्यन्वा॑हा॒हा न्विती त्यन्वा॑ह । \newline
36. अन्वा॑हा॒हा न्वन्वा॑ह॒ तस्मा॒त् तस्मा॑ दा॒हान्वन्वा॑ह॒ तस्मा᳚त् । \newline
37. आ॒ह॒ तस्मा॒त् तस्मा॑ दाहाह॒ तस्मा᳚त् प्रा॒चीन॑म् प्रा॒चीन॒म् तस्मा॑ दाहाह॒ तस्मा᳚त् प्रा॒चीन᳚म् । \newline
38. तस्मा᳚त् प्रा॒चीन॑म् प्रा॒चीन॒म् तस्मा॒त् तस्मा᳚त् प्रा॒चीनꣳ॒॒ रेतो॒ रेतः॑ प्रा॒चीन॒म् तस्मा॒त् तस्मा᳚त् प्रा॒चीनꣳ॒॒ रेतः॑ । \newline
39. प्रा॒चीनꣳ॒॒ रेतो॒ रेतः॑ प्रा॒चीन॑म् प्रा॒चीनꣳ॒॒ रेतो॑ धीयते धीयते॒ रेतः॑ प्रा॒चीन॑म् प्रा॒चीनꣳ॒॒ रेतो॑ धीयते । \newline
40. रेतो॑ धीयते धीयते॒ रेतो॒ रेतो॑ धीय॒ते ऽग्ने ऽग्ने॑ धीयते॒ रेतो॒ रेतो॑ धीय॒ते ऽग्ने᳚ । \newline
41. धी॒य॒ते ऽग्ने ऽग्ने॑ धीयते धीय॒ते ऽग्न॒ आ ऽग्ने॑ धीयते धीय॒ते ऽग्न॒ आ । \newline
42. अग्न॒ आ ऽग्ने ऽग्न॒ आ या॑हि या॒ह्या ऽग्ने ऽग्न॒ आ या॑हि । \newline
43. आ या॑हि या॒ह्या या॑हि वी॒तये॑ वी॒तये॑ या॒ह्या या॑हि वी॒तये᳚ । \newline
44. या॒हि॒ वी॒तये॑ वी॒तये॑ याहि याहि वी॒तय॒ इतीति॑ वी॒तये॑ याहि याहि वी॒तय॒ इति॑ । \newline
45. वी॒तय॒ इतीति॑ वी॒तये॑ वी॒तय॒ इत्या॑हा॒हे ति॑ वी॒तये॑ वी॒तय॒ इत्या॑ह । \newline
46. इत्या॑हा॒हे तीत्या॑ह॒ तस्मा॒त् तस्मा॑ दा॒हे तीत्या॑ह॒ तस्मा᳚त् । \newline
47. आ॒ह॒ तस्मा॒त् तस्मा॑ दाहाह॒ तस्मा᳚त् प्र॒तीचीः᳚ प्र॒तीची॒ स्तस्मा॑ दाहाह॒ तस्मा᳚त् प्र॒तीचीः᳚ । \newline
48. तस्मा᳚त् प्र॒तीचीः᳚ प्र॒तीची॒ स्तस्मा॒त् तस्मा᳚त् प्र॒तीचीः᳚ प्र॒जाः प्र॒जाः प्र॒तीची॒ स्तस्मा॒त् तस्मा᳚त् प्र॒तीचीः᳚ प्र॒जाः । \newline
49. प्र॒तीचीः᳚ प्र॒जाः प्र॒जाः प्र॒तीचीः᳚ प्र॒तीचीः᳚ प्र॒जा जा॑यन्ते जायन्ते प्र॒जाः प्र॒तीचीः᳚ प्र॒तीचीः᳚ प्र॒जा जा॑यन्ते । \newline
50. प्र॒जा जा॑यन्ते जायन्ते प्र॒जाः प्र॒जा जा॑यन्ते॒ प्र प्र जा॑यन्ते प्र॒जाः प्र॒जा जा॑यन्ते॒ प्र । \newline
51. प्र॒जा इति॑ प्र - जाः । \newline
52. जा॒य॒न्ते॒ प्र प्र जा॑यन्ते जायन्ते॒ प्र वो॑ वः॒ प्र जा॑यन्ते जायन्ते॒ प्र वः॑ । \newline
53. प्र वो॑ वः॒ प्र प्र वो॒ वाजा॒ वाजा॑ वः॒ प्र प्र वो॒ वाजाः᳚ । \newline
54. वो॒ वाजा॒ वाजा॑ वो वो॒ वाजा॒ इतीति॒ वाजा॑ वो वो॒ वाजा॒ इति॑ । \newline
55. वाजा॒ इतीति॒ वाजा॒ वाजा॒ इत्यन्वन्विति॒ वाजा॒ वाजा॒ इत्यनु॑ । \newline
\pagebreak
\markright{ TS 2.5.7.4  \hfill https://www.vedavms.in \hfill}
\addcontentsline{toc}{section}{ TS 2.5.7.4 }
\section*{ TS 2.5.7.4 }

\textbf{TS 2.5.7.4 } \newline
\textbf{Samhita Paata} \newline

इत्यन्वा॑ह॒ मासा॒ वै वाजा॑ अर्द्धमा॒सा अ॒भिद्य॑वो दे॒वा ह॒विष्म॑न्तो॒ गौर्घृ॒ताची॑ य॒ज्ञो दे॒वाञ्जि॑गाति॒ यज॑मानः सुम्न॒यु-रि॒दम॑सी॒दम॒सीत्ये॒व य॒ज्ञ्स्य॑ प्रि॒यं धामाव॑ रुन्धे॒ यं का॒मये॑त॒ सर्व॒मायु॑रिया॒दिति॒ प्र वो॒ वाजा॒ इति॒ तस्या॒नूच्याग्न॒ आ या॑हि वी॒तय॒ इति॒ संत॑त॒मुत्त॑र-मर्द्ध॒र्चमा ल॑भेत - [  ] \newline

\textbf{Pada Paata} \newline

इति॑ । अन्विति॑ । आ॒ह॒ । मासाः᳚ । वै । वाजाः᳚ । अ॒द्‌र्ध॒मा॒सा इत्य॑द्‌र्ध - मा॒साः । अ॒भिद्य॑व॒ इत्य॒भि - द्य॒वः॒ । दे॒वाः । ह॒विष्म॑न्तः । गौः । घृ॒ताची᳚ । य॒ज्ञ्ः । दे॒वान् । जि॒गा॒ति॒ । यज॑मानः । सु॒म्न॒युरिति॑ सुम्न - युः । इ॒दम् । अ॒सि॒ । इ॒दम् । अ॒सि॒ । इति॑ । ए॒व । य॒ज्ञ्स्य॑ । प्रि॒यम् । धाम॑ । अवेति॑ । रु॒न्धे॒ । यम् । का॒मये॑त । सर्व᳚म् । आयुः॑ । इ॒या॒त् । इति॑ । प्रेति॑ । वः॒ । वाजाः᳚ । इति॑ । तस्य॑ । अ॒नूच्येत्य॑नु - उच्य॑ । अग्ने᳚ । एति॑ । या॒हि॒ । वी॒तये᳚ । इति॑ । संत॑त॒मिति॒ सं - त॒त॒म् । उत्त॑र॒मित्युत् - त॒र॒म् । अ॒द्‌र्ध॒र्चमित्य॑द्‌र्ध - ऋ॒चम् । एति॑ । ल॒भे॒त॒ ।  \newline


\textbf{Krama Paata} \newline

इत्यनु॑ । अन्वा॑ह । आ॒ह॒ मासाः᳚ । मासा॒ वै । वै वाजाः᳚ । वाजा॑ अर्द्धमा॒साः । अ॒र्द्ध॒मा॒सा अ॒भिद्य॑वः । अ॒र्द्ध॒मा॒सा इत्य॑र्द्ध - मा॒साः । अ॒भिद्य॑वो दे॒वाः । अ॒भिद्य॑व॒ इत्य॒भि - द्य॒वः॒ । दे॒वा ह॒विष्म॑न्तः । ह॒विष्म॑न्तो॒ गौः । गौर् घृ॒ताची᳚ । घृ॒ताची॑ य॒ज्ञ्ः । य॒ज्ञो दे॒वान् । दे॒वान् जि॑गाति । जि॒गा॒ति॒ यज॑मानः । यज॑मानः सुम्न॒युः । सु॒म्न॒युरि॒दम् । सु॒म्न॒युरिति॑ सुम्न - युः । इ॒दम॑सि । अ॒सी॒दम् । इ॒दम॑सि । असीति॑ । इत्ये॒व । ए॒व य॒ज्ञ्स्य॑ । य॒ज्ञ्स्य॑ प्रि॒यम् । प्रि॒यम् धाम॑ । धामाव॑ । अव॑ रुन्धे । रु॒न्धे॒ यम् । यम् का॒मये॑त । का॒मये॑त॒ सर्व᳚म् । सर्व॒मायुः॑ । आयु॑रियात् । इ॒या॒दिति॑ । इति॒ प्र । प्र वः॑ । वो॒ वाजाः᳚ । वाजा॒ इति॑ । इति॒ तस्य॑ । तस्या॒नूच्य॑ । अ॒नूच्याग्ने᳚ । अ॒नूच्येत्य॑नु - उच्य॑ । अग्न॒ आ । आ या॑हि । या॒हि॒ वी॒तये᳚ । वी॒तय॒ इति॑ । इति॒ सन्त॑तम् । सन्त॑त॒मुत्त॑रम् । सन्त॑त॒मिति॒ सम् - त॒त॒म् । उत्त॑रमर्द्ध॒र्चम् । उत्त॑र॒मित्युत् - त॒र॒म् । अ॒र्द्ध॒र्चमा । अ॒र्द्ध॒र्चमित्य॑र्द्ध - ऋ॒चम् । आ ल॑भेत । ल॒भे॒त॒ प्रा॒णेन॑ \newline

\textbf{Jatai Paata} \newline

1. इत्यन्वन्विती त्यनु॑ । \newline
2. अन्वा॑ हा॒हा न्वन्वा॑ह । \newline
3. आ॒ह॒ मासा॒ मासा॑ आहाह॒ मासाः᳚ । \newline
4. मासा॒ वै वै मासा॒ मासा॒ वै । \newline
5. वै वाजा॒ वाजा॒ वै वै वाजाः᳚ । \newline
6. वाजा॑ अर्द्धमा॒सा अ॑र्द्धमा॒सा वाजा॒ वाजा॑ अर्द्धमा॒साः । \newline
7. अ॒र्द्ध॒मा॒सा अ॒भिद्य॑वो॒ ऽभिद्य॑वो ऽर्द्धमा॒सा अ॑र्द्धमा॒सा अ॒भिद्य॑वः । \newline
8. अ॒र्द्ध॒मा॒सा इत्य॑र्द्ध - मा॒साः । \newline
9. अ॒भिद्य॑वो दे॒वा दे॒वा अ॒भिद्य॑वो॒ ऽभिद्य॑वो दे॒वाः । \newline
10. अ॒भिद्य॑व॒ इत्य॒भि - द्य॒वः॒ । \newline
11. दे॒वा ह॒विष्म॑न्तो ह॒विष्म॑न्तो दे॒वा दे॒वा ह॒विष्म॑न्तः । \newline
12. ह॒विष्म॑न्तो॒ गौर् गौर्. ह॒विष्म॑न्तो ह॒विष्म॑न्तो॒ गौः । \newline
13. गौर् घृ॒ताची॑ घृ॒ताची॒ गौर् गौर् घृ॒ताची᳚ । \newline
14. घृ॒ताची॑ य॒ज्ञो य॒ज्ञो घृ॒ताची॑ घृ॒ताची॑ य॒ज्ञ्ः । \newline
15. य॒ज्ञो दे॒वान् दे॒वान्. य॒ज्ञो य॒ज्ञो दे॒वान् । \newline
16. दे॒वान् जि॑गाति जिगाति दे॒वान् दे॒वान् जि॑गाति । \newline
17. जि॒गा॒ति॒ यज॑मानो॒ यज॑मानो जिगाति जिगाति॒ यज॑मानः । \newline
18. यज॑मानः सुम्न॒युः सु॑म्न॒युर् यज॑मानो॒ यज॑मानः सुम्न॒युः । \newline
19. सु॒म्न॒युरि॒द मि॒दꣳ सु॑म्न॒युः सु॑म्न॒युरि॒दम् । \newline
20. सु॒म्न॒युरिति॑ सुम्न - युः । \newline
21. इ॒द म॑स्यसी॒द मि॒द म॑सि । \newline
22. अ॒सी॒द मि॒द म॑स्यसी॒दम् । \newline
23. इ॒द म॑स्यसी॒द मि॒द म॑सि । \newline
24. अ॒सीती त्य॑स्य॒सीति॑ । \newline
25. इत्ये॒वैवे तीत्ये॒व । \newline
26. ए॒व य॒ज्ञ्स्य॑ य॒ज्ञ्स्यै॒वैव य॒ज्ञ्स्य॑ । \newline
27. य॒ज्ञ्स्य॑ प्रि॒यम् प्रि॒यं ॅय॒ज्ञ्स्य॑ य॒ज्ञ्स्य॑ प्रि॒यम् । \newline
28. प्रि॒यम् धाम॒ धाम॑ प्रि॒यम् प्रि॒यम् धाम॑ । \newline
29. धामावाव॒ धाम॒ धामाव॑ । \newline
30. अव॑ रुन्धे रु॒न्धे ऽवाव॑ रुन्धे । \newline
31. रु॒न्धे॒ यं ॅयꣳ रु॑न्धे रुन्धे॒ यम् । \newline
32. यम् का॒मये॑त का॒मये॑त॒ यं ॅयम् का॒मये॑त । \newline
33. का॒मये॑त॒ सर्वꣳ॒॒ सर्व॑म् का॒मये॑त का॒मये॑त॒ सर्व᳚म् । \newline
34. सर्व॒ मायु॒ रायुः॒ सर्वꣳ॒॒ सर्व॒ मायुः॑ । \newline
35. आयु॑ रिया दिया॒ दायु॒ रायु॑ रियात् । \newline
36. इ॒या॒ दितीती॑या दिया॒ दिति॑ । \newline
37. इति॒ प्र प्रे तीति॒ प्र । \newline
38. प्र वो॑ वः॒ प्र प्र वः॑ । \newline
39. वो॒ वाजा॒ वाजा॑ वो वो॒ वाजाः᳚ । \newline
40. वाजा॒ इतीति॒ वाजा॒ वाजा॒ इति॑ । \newline
41. इति॒ तस्य॒ तस्ये तीति॒ तस्य॑ । \newline
42. तस्या॒नूच्या॒ नूच्य॒ तस्य॒ तस्या॒ नूच्य॑ । \newline
43. अ॒नूच्याग्ने ऽग्ने॒ ऽनूच्या॒ नूच्याग्ने᳚ । \newline
44. अ॒नूच्येत्य॑नु - उच्य॑ । \newline
45. अग्न॒ आ ऽग्ने ऽग्न॒ आ । \newline
46. आ या॑हि या॒ह्या या॑हि । \newline
47. या॒हि॒ वी॒तये॑ वी॒तये॑ याहि याहि वी॒तये᳚ । \newline
48. वी॒तय॒ इतीति॑ वी॒तये॑ वी॒तय॒ इति॑ । \newline
49. इति॒ सन्त॑तꣳ॒॒ सन्त॑त॒ मितीति॒ सन्त॑तम् । \newline
50. सन्त॑त॒ मुत्त॑र॒ मुत्त॑रꣳ॒॒ सन्त॑तꣳ॒॒ सन्त॑त॒ मुत्त॑रम् । \newline
51. सन्त॑त॒मिति॒ सं - त॒त॒म् । \newline
52. उत्त॑र मर्द्ध॒र्च म॑र्द्ध॒र्च मुत्त॑र॒ मुत्त॑र मर्द्ध॒र्चम् । \newline
53. उत्त॑र॒मित्युत् - त॒र॒म् । \newline
54. अ॒र्द्ध॒र्च मा ऽर्द्ध॒र्च म॑र्द्ध॒र्च मा । \newline
55. अ॒र्द्ध॒र्चमित्य॑र्द्ध - ऋ॒चम् । \newline
56. आ ल॑भेत लभे॒ता ल॑भेत । \newline
57. ल॒भे॒त॒ प्रा॒णेन॑ प्रा॒णेन॑ लभेत लभेत प्रा॒णेन॑ । \newline

\textbf{Ghana Paata } \newline

1. इत्यन्वन्विती त्यन्वा॑हा॒हा न्विती त्यन्वा॑ह । \newline
2. अन्वा॑हा॒हा न्वन्वा॑ह॒ मासा॒ मासा॑ आ॒हान्वन्वा॑ह॒ मासाः᳚ । \newline
3. आ॒ह॒ मासा॒ मासा॑ आहाह॒ मासा॒ वै वै मासा॑ आहाह॒ मासा॒ वै । \newline
4. मासा॒ वै वै मासा॒ मासा॒ वै वाजा॒ वाजा॒ वै मासा॒ मासा॒ वै वाजाः᳚ । \newline
5. वै वाजा॒ वाजा॒ वै वै वाजा॑ अर्द्धमा॒सा अ॑र्द्धमा॒सा वाजा॒ वै वै वाजा॑ अर्द्धमा॒साः । \newline
6. वाजा॑ अर्द्धमा॒सा अ॑र्द्धमा॒सा वाजा॒ वाजा॑ अर्द्धमा॒सा अ॒भिद्य॑वो॒ ऽभिद्य॑वो ऽर्द्धमा॒सा वाजा॒ वाजा॑ अर्द्धमा॒सा अ॒भिद्य॑वः । \newline
7. अ॒र्द्ध॒मा॒सा अ॒भिद्य॑वो॒ ऽभिद्य॑वो ऽर्द्धमा॒सा अ॑र्द्धमा॒सा अ॒भिद्य॑वो दे॒वा दे॒वा अ॒भिद्य॑वो ऽर्द्धमा॒सा अ॑र्द्धमा॒सा अ॒भिद्य॑वो दे॒वाः । \newline
8. अ॒र्द्ध॒मा॒सा इत्य॑र्द्ध - मा॒साः । \newline
9. अ॒भिद्य॑वो दे॒वा दे॒वा अ॒भिद्य॑वो॒ ऽभिद्य॑वो दे॒वा ह॒विष्म॑न्तो ह॒विष्म॑न्तो दे॒वा अ॒भिद्य॑वो॒ ऽभिद्य॑वो दे॒वा ह॒विष्म॑न्तः । \newline
10. अ॒भिद्य॑व॒ इत्य॒भि - द्य॒वः॒ । \newline
11. दे॒वा ह॒विष्म॑न्तो ह॒विष्म॑न्तो दे॒वा दे॒वा ह॒विष्म॑न्तो॒ गौर् गौर्. ह॒विष्म॑न्तो दे॒वा दे॒वा ह॒विष्म॑न्तो॒ गौः । \newline
12. ह॒विष्म॑न्तो॒ गौर् गौर्. ह॒विष्म॑न्तो ह॒विष्म॑न्तो॒ गौर् घृ॒ताची॑ घृ॒ताची॒ गौर्. ह॒विष्म॑न्तो ह॒विष्म॑न्तो॒ गौर् घृ॒ताची᳚ । \newline
13. गौर् घृ॒ताची॑ घृ॒ताची॒ गौर् गौर् घृ॒ताची॑ य॒ज्ञो य॒ज्ञो घृ॒ताची॒ गौर् गौर् घृ॒ताची॑ य॒ज्ञ्ः । \newline
14. घृ॒ताची॑ य॒ज्ञो य॒ज्ञो घृ॒ताची॑ घृ॒ताची॑ य॒ज्ञो दे॒वान् दे॒वान्. य॒ज्ञो घृ॒ताची॑ घृ॒ताची॑ य॒ज्ञो दे॒वान् । \newline
15. य॒ज्ञो दे॒वान् दे॒वान्. य॒ज्ञो य॒ज्ञो दे॒वान् जि॑गाति जिगाति दे॒वान्. य॒ज्ञो य॒ज्ञो दे॒वान् जि॑गाति । \newline
16. दे॒वान् जि॑गाति जिगाति दे॒वान् दे॒वान् जि॑गाति॒ यज॑मानो॒ यज॑मानो जिगाति दे॒वान् दे॒वान् जि॑गाति॒ यज॑मानः । \newline
17. जि॒गा॒ति॒ यज॑मानो॒ यज॑मानो जिगाति जिगाति॒ यज॑मानः सुम्न॒युः सु॑म्न॒युर् यज॑मानो जिगाति जिगाति॒ यज॑मानः सुम्न॒युः । \newline
18. यज॑मानः सुम्न॒युः सु॑म्न॒युर् यज॑मानो॒ यज॑मानः सुम्न॒युरि॒द मि॒दꣳ सु॑म्न॒युर् यज॑मानो॒ यज॑मानः सुम्न॒युरि॒दम् । \newline
19. सु॒म्न॒युरि॒द मि॒दꣳ सु॑म्न॒युः सु॑म्न॒युरि॒द म॑स्यसी॒दꣳ सु॑म्न॒युः सु॑म्न॒युरि॒द म॑सि । \newline
20. सु॒म्न॒युरिति॑ सुम्न - युः । \newline
21. इ॒द म॑स्यसी॒द मि॒द म॑सी॒द मि॒द म॑सी॒द मि॒द म॑सी॒दम् । \newline
22. अ॒सी॒द मि॒द म॑स्यसी॒द म॑स्यसी॒द म॑स्यसी॒द म॑सि । \newline
23. इ॒द म॑स्यसी॒द मि॒द म॒सीतीत्य॑सी॒द मि॒द म॒सीति॑ । \newline
24. अ॒सीती त्य॑स्य॒सी त्ये॒वैवे त्य॑स्य॒सी त्ये॒व । \newline
25. इत्ये॒वैवे तीत्ये॒व य॒ज्ञ्स्य॑ य॒ज्ञ्स्यै॒वे तीत्ये॒व य॒ज्ञ्स्य॑ । \newline
26. ए॒व य॒ज्ञ्स्य॑ य॒ज्ञ्स्यै॒वैव य॒ज्ञ्स्य॑ प्रि॒यम् प्रि॒यं ॅय॒ज्ञ्स्यै॒वैव य॒ज्ञ्स्य॑ प्रि॒यम् । \newline
27. य॒ज्ञ्स्य॑ प्रि॒यम् प्रि॒यं ॅय॒ज्ञ्स्य॑ य॒ज्ञ्स्य॑ प्रि॒यम् धाम॒ धाम॑ प्रि॒यं ॅय॒ज्ञ्स्य॑ य॒ज्ञ्स्य॑ प्रि॒यम् धाम॑ । \newline
28. प्रि॒यम् धाम॒ धाम॑ प्रि॒यम् प्रि॒यम् धामावाव॒ धाम॑ प्रि॒यम् प्रि॒यम् धामाव॑ । \newline
29. धामावाव॒ धाम॒ धामाव॑ रुन्धे रु॒न्धे ऽव॒ धाम॒ धामाव॑ रुन्धे । \newline
30. अव॑ रुन्धे रु॒न्धे ऽवाव॑ रुन्धे॒ यं ॅयꣳ रु॒न्धे ऽवाव॑ रुन्धे॒ यम् । \newline
31. रु॒न्धे॒ यं ॅयꣳ रु॑न्धे रुन्धे॒ यम् का॒मये॑त का॒मये॑त॒ यꣳ रु॑न्धे रुन्धे॒ यम् का॒मये॑त । \newline
32. यम् का॒मये॑त का॒मये॑त॒ यं ॅयम् का॒मये॑त॒ सर्वꣳ॒॒ सर्व॑म् का॒मये॑त॒ यं ॅयम् का॒मये॑त॒ सर्व᳚म् । \newline
33. का॒मये॑त॒ सर्वꣳ॒॒ सर्व॑म् का॒मये॑त का॒मये॑त॒ सर्व॒ मायु॒ रायुः॒ सर्व॑म् का॒मये॑त का॒मये॑त॒ सर्व॒ मायुः॑ । \newline
34. सर्व॒ मायु॒रायुः॒ सर्वꣳ॒॒ सर्व॒ मायु॑ रिया दिया॒ दायुः॒ सर्वꣳ॒॒ सर्व॒ मायु॑रियात् । \newline
35. आयु॑ रिया दिया॒ दायु॒ रायु॑ रिया॒ दितीती॑या॒ दायु॒ रायु॑ रिया॒ दिति॑ । \newline
36. इ॒या॒ दितीती॑या दिया॒ दिति॒ प्र प्रे ती॑या दिया॒ दिति॒ प्र । \newline
37. इति॒ प्र प्रे तीति॒ प्र वो॑ वः॒ प्रे तीति॒ प्र वः॑ । \newline
38. प्र वो॑ वः॒ प्र प्र वो॒ वाजा॒ वाजा॑ वः॒ प्र प्र वो॒ वाजाः᳚ । \newline
39. वो॒ वाजा॒ वाजा॑ वो वो॒ वाजा॒ इतीति॒ वाजा॑ वो वो॒ वाजा॒ इति॑ । \newline
40. वाजा॒ इतीति॒ वाजा॒ वाजा॒ इति॒ तस्य॒ तस्ये ति॒ वाजा॒ वाजा॒ इति॒ तस्य॑ । \newline
41. इति॒ तस्य॒ तस्ये तीति॒ तस्या॒नूच्या॒ नूच्य॒ तस्ये तीति॒ तस्या॒नूच्य॑ । \newline
42. तस्या॒नूच्या॒ नूच्य॒ तस्य॒ तस्या॒ नूच्याग्ने ऽग्ने॒ ऽनूच्य॒ तस्य॒ तस्या॒ नूच्याग्ने᳚ । \newline
43. अ॒नूच्याग्ने ऽग्ने॒ ऽनूच्या॒ नूच्याग्न॒ आ ऽग्ने॒ ऽनूच्या॒ नूच्याग्न॒ आ । \newline
44. अ॒नूच्येत्य॑नु - उच्य॑ । \newline
45. अग्न॒ आ ऽग्ने ऽग्न॒ आ या॑हि या॒ह्या ऽग्ने ऽग्न॒ आ या॑हि । \newline
46. आ या॑हि या॒ह्या या॑हि वी॒तये॑ वी॒तये॑ या॒ह्या या॑हि वी॒तये᳚ । \newline
47. या॒हि॒ वी॒तये॑ वी॒तये॑ याहि याहि वी॒तय॒ इतीति॑ वी॒तये॑ याहि याहि वी॒तय॒ इति॑ । \newline
48. वी॒तय॒ इतीति॑ वी॒तये॑ वी॒तय॒ इति॒ सन्त॑तꣳ॒॒ सन्त॑त॒ मिति॑ वी॒तये॑ वी॒तय॒ इति॒ सन्त॑तम् । \newline
49. इति॒ सन्त॑तꣳ॒॒ सन्त॑त॒ मितीति॒ सन्त॑त॒ मुत्त॑र॒ मुत्त॑रꣳ॒॒ सन्त॑त॒ मितीति॒ सन्त॑त॒ मुत्त॑रम् । \newline
50. सन्त॑त॒ मुत्त॑र॒ मुत्त॑रꣳ॒॒ सन्त॑तꣳ॒॒ सन्त॑त॒ मुत्त॑र मर्द्ध॒र्च म॑र्द्ध॒र्च मुत्त॑रꣳ॒॒ सन्त॑तꣳ॒॒ सन्त॑त॒ मुत्त॑र मर्द्ध॒र्चम् । \newline
51. सन्त॑त॒मिति॒ सं - त॒त॒म् । \newline
52. उत्त॑र मर्द्ध॒र्च म॑र्द्ध॒र्च मुत्त॑र॒ मुत्त॑र मर्द्ध॒र्च मा ऽर्द्ध॒र्च मुत्त॑र॒ मुत्त॑र मर्द्ध॒र्च मा । \newline
53. उत्त॑र॒मित्युत् - त॒र॒म् । \newline
54. अ॒र्द्ध॒र्च मा ऽर्द्ध॒र्च म॑र्द्ध॒र्च मा ल॑भेत लभे॒ता ऽर्द्ध॒र्च म॑र्द्ध॒र्च मा ल॑भेत । \newline
55. अ॒र्द्ध॒र्चमित्य॑र्द्ध - ऋ॒चम् । \newline
56. आ ल॑भेत लभे॒ता ल॑भेत प्रा॒णेन॑ प्रा॒णेन॑ लभे॒ता ल॑भेत प्रा॒णेन॑ । \newline
57. ल॒भे॒त॒ प्रा॒णेन॑ प्रा॒णेन॑ लभेत लभेत प्रा॒णेनै॒वैव प्रा॒णेन॑ लभेत लभेत प्रा॒णेनै॒व । \newline
\pagebreak
\markright{ TS 2.5.7.5  \hfill https://www.vedavms.in \hfill}
\addcontentsline{toc}{section}{ TS 2.5.7.5 }
\section*{ TS 2.5.7.5 }

\textbf{TS 2.5.7.5 } \newline
\textbf{Samhita Paata} \newline

प्रा॒णेनै॒वास्या॑पा॒नं दा॑धार॒ सर्व॒मायु॑रेति॒ यो वा अ॑र॒त्निꣳ सा॑मिधे॒नीनां॒ ॅवेदा॑र॒त्नावे॒व भ्रातृ॑व्यं कुरुतेऽर्द्ध॒र्चौ सं द॑धात्ये॒ष वा अ॑र॒त्निः सा॑मिधे॒नीनां॒ ॅय ए॒वं ॅवेदा॑र॒त्नावे॒व भ्रातृ॑व्यं कुरुत॒ ऋष॑र्. ऋषे॒र्वा ए॒ता निर्मि॑ता॒ यथ् सा॑मिधे॒न्य॑स्ता यदसं॑ ॅयुक्ताः॒ स्युः प्र॒जया॑ प॒शुभि॒ र्यज॑मानस्य॒ वि ति॑ष्ठेरन्नर्द्ध॒र्चौ सं द॑धाति॒ सं ( ) ॅयु॑नक्त्ये॒वैना॒स्ता अ॑स्मै॒ संॅयु॑क्ता॒ अव॑रुद्धाः॒ सर्वा॑मा॒शिषं॑ दुह्रे ॥ \newline

\textbf{Pada Paata} \newline

प्रा॒णेनेति॑ प्र - अ॒नेन॑ । ए॒व । अ॒स्य॒ । अ॒पा॒नमित्य॑प - अ॒नम् । दा॒धा॒र॒ । सर्व᳚म् । आयुः॑ । ए॒ति॒ । यः । वै । अ॒र॒त्निम् । सा॒मि॒धे॒नीना॒मिति॑ सां - इ॒धे॒नीना᳚म् । वेद॑ । अ॒र॒त्नौ । ए॒व । भ्रातृ॑व्यम् । कु॒रु॒ते॒ । अ॒द्‌र्ध॒र्चावित्य॑द्‌र्ध - ऋ॒चौ । समिति॑ । द॒धा॒ति॒ । ए॒षः । वै । अ॒र॒त्निः । सा॒मि॒धे॒नीना॒मिति॑ सां - इ॒धे॒नीना᳚म् । यः । ए॒वम् । वेद॑ । अ॒र॒त्नौ । ए॒व । भ्रातृ॑व्यम् । कु॒रु॒ते॒ । ऋषेर्॑. ऋषे॒रित्यृषेः᳚ - ऋ॒षेः॒ । वै । ए॒ताः । निर्मि॑ता॒ इति॒ निः-मि॒ताः॒ । यत् । सा॒मि॒धे॒न्य॑ इति॑ सां - इ॒धे॒न्यः॑ । ताः । यत् । असं॑ॅयुक्ता॒ इत्यसं᳚ - यु॒क्ताः॒ । स्युः । प्र॒जयेति॑ प्र - जया᳚ । प॒शुभि॒रिति॑ प॒शु - भिः॒ । यज॑मानस्य । वीति॑ । ति॒ष्ठे॒र॒न्न् । अ॒द्‌र्ध॒र्चावित्य॑द्‌र्ध - ऋ॒चौ । समिति॑ । द॒धा॒ति॒ । समिति॑ ( ) । यु॒न॒क्ति॒ । ए॒व । ए॒नाः॒ । ताः । अ॒स्मै॒ । संॅयु॑क्ता॒ इति॒ सं - यु॒क्ताः॒ । अव॑रुद्धा॒ इत्यव॑ - रु॒द्धाः॒ । सर्वा᳚म् । आ॒शिष॒मित्या᳚ - शिष᳚म् । दु॒ह्रे॒ ॥  \newline


\textbf{Krama Paata} \newline

प्रा॒णेनै॒व । प्रा॒णेनेति॑ प्र - अ॒नेन॑ । ए॒वास्य॑ । अ॒स्या॒पा॒नम् । अ॒पा॒नम् दा॑धार । अ॒पा॒नमित्य॑प - अ॒नम् । दा॒धा॒र॒ सर्व᳚म् । सर्व॒मायुः॑ । आयु॑रेति । ए॒ति॒ यः । यो वै । वा अ॑र॒त्निम् । अ॒र॒त्निꣳ सा॑मिधे॒नीना᳚म् । सा॒मि॒धे॒नीना॒म् ॅवेद॑ । सा॒मि॒धे॒नीना॒मिति॑ साम् - इ॒धे॒नीना᳚म् । वेदा॑र॒त्नौ । अ॒र॒त्नावे॒व । ए॒व भ्रातृ॑व्यम् । भ्रातृ॑व्यम् कुरुते । कु॒रु॒ते॒ ऽर्द्ध॒र्चौ । अ॒र्द्ध॒र्चौ सम् । अ॒र्द्ध॒र्चावित्य॑र्द्ध - ऋ॒चौ । सम् द॑धाति । द॒धा॒त्ये॒षः । ए॒ष वै । वा अ॑र॒त्निः । अ॒र॒त्निः सा॑मिधे॒नीना᳚म् । सा॒मि॒धे॒नीना॒म् ॅयः । सा॒मि॒धे॒नीना॒मिति॑ साम् - इ॒धे॒नीना᳚म् । य ए॒वम् । ए॒वम् ॅवेद॑ । वेदा॑र॒त्नौ । अ॒र॒त्नावे॒व । ए॒व भ्रातृ॑व्यम् । भ्रातृ॑व्यम् कुरुते । कु॒रु॒त॒ ऋषेर्॑.ऋषेः । ऋषेर्॑.ऋषे॒र्. वै । ऋषेर्॑. ऋषे॒रित्यृषेः᳚ - ऋ॒षेः॒ । वा ए॒ताः । ए॒ता निर्मि॑ताः । निर्मि॑ता॒ यत् । निर्मि॑ता॒ इति॒ निः - मि॒ताः॒ । यथ् सा॑मिधे॒न्यः॑ । सा॒मि॒धे॒न्य॑स्ताः । सा॒मि॒धे॒न्य॑ इति॑ साम् - इ॒धे॒न्यः॑ । ता यत् । यदस॑म्ॅयुक्ताः । अस॑म्ॅयुक्ताः॒ स्युः । अस॑म्ॅयुक्ता॒ इत्यस᳚म् - यु॒क्ताः॒ । स्युः प्र॒जया᳚ । प्र॒जया॑ प॒शुभिः॑ । प्र॒जयेति॑ प्र - जया᳚ । प॒शुभि॒र् यज॑मानस्य । प॒शुभि॒रिति॑ प॒शु - भिः॒ । यज॑मानस्य॒ वि । वि ति॑ष्ठेरन्न् । ति॒ष्ठे॒र॒न्न॒र्द्ध॒र्चौ । अ॒र्द्ध॒र्चौ सम् । अ॒र्द्ध॒र्चावित्य॑र्द्ध - ऋ॒चौ । सम् द॑धाति । द॒धा॒ति॒ सम् ( ) । सम् ॅयु॑नक्ति । यु॒न॒क्त्ये॒व । ए॒वैनाः᳚ । ए॒ना॒स्ताः । ता अ॑स्मै । अ॒स्मै॒ सम्ॅयु॑क्ताः । सम्ॅयु॑क्ता॒ अव॑रुद्धाः । सम्ॅयु॑क्ता॒ इति॒ सम् - यु॒क्ताः॒ । अव॑रुद्धाः॒ सर्वा᳚म् । अव॑रुद्धा॒ इत्यव॑ - रु॒द्धाः॒ । सर्वा॑मा॒शिष᳚म् । आ॒शिष॑म् दुह्रे । आ॒शिष॒मित्या᳚ - शिष᳚म् । दु॒ह्र॒ इति॑ दुह्रे । \newline

\textbf{Jatai Paata} \newline

1. प्रा॒णे नै॒वैव प्रा॒णेन॑ प्रा॒णे नै॒व । \newline
2. प्रा॒णेनेति॑ प्र - अ॒नेन॑ । \newline
3. ए॒वास्या᳚ स्यै॒वैवास्य॑ । \newline
4. अ॒स्या॒ पा॒न म॑पा॒न म॑स्यास्या पा॒नम् । \newline
5. अ॒पा॒नम् दा॑धार दाधारा पा॒न म॑पा॒नम् दा॑धार । \newline
6. अ॒पा॒नमित्य॑प - अ॒नम् । \newline
7. दा॒धा॒र॒ सर्वꣳ॒॒ सर्व॑म् दाधार दाधार॒ सर्व᳚म् । \newline
8. सर्व॒ मायु॒रायुः॒ सर्वꣳ॒॒ सर्व॒ मायुः॑ । \newline
9. आयु॑ रेत्ये॒त्यायु॒ रायु॑रेति । \newline
10. ए॒ति॒ यो य ए᳚त्येति॒ यः । \newline
11. यो वै वै यो यो वै । \newline
12. वा अ॑र॒त्नि म॑र॒त्निं ॅवै वा अ॑र॒त्निम् । \newline
13. अ॒र॒त्निꣳ सा॑मिधे॒नीनाꣳ॑ सामिधे॒नीना॑ मर॒त्नि म॑र॒त्निꣳ सा॑मिधे॒नीना᳚म् । \newline
14. सा॒मि॒धे॒नीनां॒ ॅवेद॒ वेद॑ सामिधे॒नीनाꣳ॑ सामिधे॒नीनां॒ ॅवेद॑ । \newline
15. सा॒मि॒धे॒नीना॒मिति॑ सां - इ॒धे॒नीना᳚म् । \newline
16. वेदा॑र॒त्ना व॑र॒त्नौ वेद॒ वेदा॑र॒त्नौ । \newline
17. अ॒र॒त्ना वे॒वैवा र॒त्ना व॑र॒त्ना वे॒व । \newline
18. ए॒व भ्रातृ॑व्य॒म् भ्रातृ॑व्य मे॒वैव भ्रातृ॑व्यम् । \newline
19. भ्रातृ॑व्यम् कुरुते कुरुते॒ भ्रातृ॑व्य॒म् भ्रातृ॑व्यम् कुरुते । \newline
20. कु॒रु॒ते॒ ऽर्द्ध॒र्चा व॑र्द्ध॒र्चौ कु॑रुते कुरुते ऽर्द्ध॒र्चौ । \newline
21. अ॒र्द्ध॒र्चौ सꣳ स म॑र्द्ध॒र्चा व॑र्द्ध॒र्चौ सम् । \newline
22. अ॒र्द्ध॒र्चावित्य॑र्द्ध - ऋ॒चौ । \newline
23. सम् द॑धाति दधाति॒ सꣳ सम् द॑धाति । \newline
24. द॒धा॒ त्ये॒ष ए॒ष द॑धाति दधा त्ये॒षः । \newline
25. ए॒ष वै वा ए॒ष ए॒ष वै । \newline
26. वा अ॑र॒त्निर॑ र॒त्निर् वै वा अ॑र॒त्निः । \newline
27. अ॒र॒त्निः सा॑मिधे॒नीनाꣳ॑ सामिधे॒नीना॑ मर॒त्नि र॑र॒त्निः सा॑मिधे॒नीना᳚म् । \newline
28. सा॒मि॒धे॒नीनां॒ ॅयो यः सा॑मिधे॒नीनाꣳ॑ सामिधे॒नीनां॒ ॅयः । \newline
29. सा॒मि॒धे॒नीना॒मिति॑ सां - इ॒धे॒नीना᳚म् । \newline
30. य ए॒व मे॒वं ॅयो य ए॒वम् । \newline
31. ए॒वं ॅवेद॒ वेदै॒व मे॒वं ॅवेद॑ । \newline
32. वेदा॑ र॒त्ना व॑र॒त्नौ वेद॒ वेदा॑ र॒त्नौ । \newline
33. अ॒र॒त्ना वे॒वैवा र॒त्ना व॑र॒त्ना वे॒व । \newline
34. ए॒व भ्रातृ॑व्य॒म् भ्रातृ॑व्य मे॒वैव भ्रातृ॑व्यम् । \newline
35. भ्रातृ॑व्यम् कुरुते कुरुते॒ भ्रातृ॑व्य॒म् भ्रातृ॑व्यम् कुरुते । \newline
36. कु॒रु॒त॒ ऋषेर्॑.ऋषे॒र्॒. ऋषेर्॑.ऋषेः कुरुते कुरुत॒ ऋषेर्॑.ऋषेः । \newline
37. ऋषेर्॑.ऋषे॒र् वै वा ऋषेर्॑.ऋषे॒र्॒. ऋषेर्॑.ऋषे॒र् वै । \newline
38. ऋषेर्॑.ऋषे॒रित्यृषेः᳚ - ऋ॒षेः॒ । \newline
39. वा ए॒ता ए॒ता वै वा ए॒ताः । \newline
40. ए॒ता निर्मि॑ता॒ निर्मि॑ता ए॒ता ए॒ता निर्मि॑ताः । \newline
41. निर्मि॑ता॒ यद् यन् निर्मि॑ता॒ निर्मि॑ता॒ यत् । \newline
42. निर्मि॑ता॒ इति॒ निः - मि॒ताः॒ । \newline
43. यथ् सा॑मिधे॒न्यः॑ सामिधे॒न्यो॑ यद् यथ् सा॑मिधे॒न्यः॑ । \newline
44. सा॒मि॒धे॒न्य॑ स्ता स्ताः सा॑मिधे॒न्यः॑ सामिधे॒न्य॑ स्ताः । \newline
45. सा॒मि॒धे॒न्य॑ इति॑ सां - इ॒धे॒न्यः॑ । \newline
46. ता यद् यत् ता स्ता यत् । \newline
47. यदसं॑ॅयुक्ता॒ असं॑ॅयुक्ता॒ यद् यदसं॑ॅयुक्ताः । \newline
48. असं॑ॅयुक्ताः॒ स्युः स्यु रसं॑ॅयुक्ता॒ असं॑ॅयुक्ताः॒ स्युः । \newline
49. असं॑ॅयुक्ता॒ इत्यसं᳚ - यु॒क्ताः॒ । \newline
50. स्युः प्र॒जया᳚ प्र॒जया॒ स्युः स्युः प्र॒जया᳚ । \newline
51. प्र॒जया॑ प॒शुभिः॑ प॒शुभिः॑ प्र॒जया᳚ प्र॒जया॑ प॒शुभिः॑ । \newline
52. प्र॒जयेति॑ प्र - जया᳚ । \newline
53. प॒शुभि॒र् यज॑मानस्य॒ यज॑मानस्य प॒शुभिः॑ प॒शुभि॒र् यज॑मानस्य । \newline
54. प॒शुभि॒रिति॑ प॒शु - भिः॒ । \newline
55. यज॑मानस्य॒ वि वि यज॑मानस्य॒ यज॑मानस्य॒ वि । \newline
56. वि ति॑ष्ठेरन् तिष्ठेर॒न्॒. वि वि ति॑ष्ठेरन्न् । \newline
57. ति॒ष्ठे॒र॒न् न॒र्द्ध॒र्चा व॑र्द्ध॒र्चौ ति॑ष्ठेरन् तिष्ठेरन् नर्द्ध॒र्चौ । \newline
58. अ॒र्द्ध॒र्चौ सꣳ स म॑र्द्ध॒र्चा व॑र्द्ध॒र्चौ सम् । \newline
59. अ॒र्द्ध॒र्चावित्य॑र्द्ध - ऋ॒चौ । \newline
60. सम् द॑धाति दधाति॒ सꣳ सम् द॑धाति । \newline
61. द॒धा॒ति॒ सꣳ सम् द॑धाति दधाति॒ सम् । \newline
62. सं ॅयु॑नक्ति युनक्ति॒ सꣳ सं ॅयु॑नक्ति । \newline
63. यु॒न॒क् त्ये॒वैव यु॑नक्ति युनक् त्ये॒व । \newline
64. ए॒वैना॑ एना ए॒वैवैनाः᳚ । \newline
65. ए॒ना॒ स्ता स्ता ए॑ना एना॒ स्ताः । \newline
66. ता अ॑स्मा अस्मै॒ ता स्ता अ॑स्मै । \newline
67. अ॒स्मै॒ संॅयु॑क्ताः॒ संॅयु॑क्ता अस्मा अस्मै॒ संॅयु॑क्ताः । \newline
68. संॅयु॑क्ता॒ अव॑रुद्धा॒ अव॑रुद्धाः॒ संॅयु॑क्ताः॒ संॅयु॑क्ता॒ अव॑रुद्धाः । \newline
69. संॅयु॑क्ता॒ इति॒ सं - यु॒क्ताः॒ । \newline
70. अव॑रुद्धाः॒ सर्वाꣳ॒॒ सर्वा॒ मव॑रुद्धा॒ अव॑रुद्धाः॒ सर्वा᳚म् । \newline
71. अव॑रुद्धा॒ इत्यव॑ - रु॒द्धाः॒ । \newline
72. सर्वा॑ मा॒शिष॑ मा॒शिषꣳ॒॒ सर्वाꣳ॒॒ सर्वा॑ मा॒शिष᳚म् । \newline
73. आ॒शिष॑म् दुह्रे दुह्र आ॒शिष॑ मा॒शिष॑म् दुह्रे । \newline
74. आ॒शिष॒मित्या᳚ - शिष᳚म् । \newline
75. दु॒ह्र॒ इति॑ दुह्रे । \newline

\textbf{Ghana Paata } \newline

1. प्रा॒णेनै॒वैव प्रा॒णेन॑ प्रा॒णेनै॒वास्या᳚ स्यै॒व प्रा॒णेन॑ प्रा॒णेनै॒वास्य॑ । \newline
2. प्रा॒णेनेति॑ प्र - अ॒नेन॑ । \newline
3. ए॒वास्या᳚ स्यै॒वैवा स्या॑पा॒न म॑पा॒न म॑स्यै॒वैवा स्या॑पा॒नम् । \newline
4. अ॒स्या॒पा॒न म॑पा॒न म॑स्यास्या पा॒नम् दा॑धार दाधारापा॒न म॑स्यास्या पा॒नम् दा॑धार । \newline
5. अ॒पा॒नम् दा॑धार दाधारापा॒न म॑पा॒नम् दा॑धार॒ सर्वꣳ॒॒ सर्व॑म् दाधारापा॒न म॑पा॒नम् दा॑धार॒ सर्व᳚म् । \newline
6. अ॒पा॒नमित्य॑प - अ॒नम् । \newline
7. दा॒धा॒र॒ सर्वꣳ॒॒ सर्व॑म् दाधार दाधार॒ सर्व॒ मायु॒रायुः॒ सर्व॑म् दाधार दाधार॒ सर्व॒ मायुः॑ । \newline
8. सर्व॒ मायु॒रायुः॒ सर्वꣳ॒॒ सर्व॒ मायु॑ रेत्ये॒ त्यायुः॒ सर्वꣳ॒॒ सर्व॒ मायु॑रेति । \newline
9. आयु॑ रेत्ये॒त्यायु॒ रायु॑रेति॒ यो य ए॒त्यायु॒ रायु॑रेति॒ यः । \newline
10. ए॒ति॒ यो य ए᳚त्येति॒ यो वै वै य ए᳚त्येति॒ यो वै । \newline
11. यो वै वै यो यो वा अ॑र॒त्नि म॑र॒त्निं ॅवै यो यो वा अ॑र॒त्निम् । \newline
12. वा अ॑र॒त्नि म॑र॒त्निं ॅवै वा अ॑र॒त्निꣳ सा॑मिधे॒नीनाꣳ॑ सामिधे॒नीना॑ मर॒त्निं ॅवै वा अ॑र॒त्निꣳ सा॑मिधे॒नीना᳚म् । \newline
13. अ॒र॒त्निꣳ सा॑मिधे॒नीनाꣳ॑ सामिधे॒नीना॑ मर॒त्नि म॑र॒त्निꣳ सा॑मिधे॒नीनां॒ ॅवेद॒ वेद॑ सामिधे॒नीना॑ मर॒त्नि म॑र॒त्निꣳ सा॑मिधे॒नीनां॒ ॅवेद॑ । \newline
14. सा॒मि॒धे॒नीनां॒ ॅवेद॒ वेद॑ सामिधे॒नीनाꣳ॑ सामिधे॒नीनां॒ ॅवेदा॑र॒त्ना व॑र॒त्नौ वेद॑ सामिधे॒नीनाꣳ॑ सामिधे॒नीनां॒ ॅवेदा॑र॒त्नौ । \newline
15. सा॒मि॒धे॒नीना॒मिति॑ सां - इ॒धे॒नीना᳚म् । \newline
16. वेदा॑र॒त्ना व॑र॒त्नौ वेद॒ वेदा॑र॒त्ना वे॒वैवार॒त्नौ वेद॒ वेदा॑र॒त्ना वे॒व । \newline
17. अ॒र॒त्ना वे॒वैवार॒त्ना व॑र॒त्ना वे॒व भ्रातृ॑व्य॒म् भ्रातृ॑व्य मे॒वार॒त्ना व॑र॒त्ना वे॒व भ्रातृ॑व्यम् । \newline
18. ए॒व भ्रातृ॑व्य॒म् भ्रातृ॑व्य मे॒वैव भ्रातृ॑व्यम् कुरुते कुरुते॒ भ्रातृ॑व्य मे॒वैव भ्रातृ॑व्यम् कुरुते । \newline
19. भ्रातृ॑व्यम् कुरुते कुरुते॒ भ्रातृ॑व्य॒म् भ्रातृ॑व्यम् कुरुते ऽर्द्ध॒र्चा व॑र्द्ध॒र्चौ कु॑रुते॒ भ्रातृ॑व्य॒म् भ्रातृ॑व्यम् कुरुते ऽर्द्ध॒र्चौ । \newline
20. कु॒रु॒ते॒ ऽर्द्ध॒र्चा व॑र्द्ध॒र्चौ कु॑रुते कुरुते ऽर्द्ध॒र्चौ सꣳ स म॑र्द्ध॒र्चौ कु॑रुते कुरुते ऽर्द्ध॒र्चौ सम् । \newline
21. अ॒र्द्ध॒र्चौ सꣳ स म॑र्द्ध॒र्चा व॑र्द्ध॒र्चौ सम् द॑धाति दधाति॒ स म॑र्द्ध॒र्चा व॑र्द्ध॒र्चौ सम् द॑धाति । \newline
22. अ॒र्द्ध॒र्चावित्य॑र्द्ध - ऋ॒चौ । \newline
23. सम् द॑धाति दधाति॒ सꣳ सम् द॑धात्ये॒ष ए॒ष द॑धाति॒ सꣳ सम् द॑धात्ये॒षः । \newline
24. द॒धा॒त्ये॒ष ए॒ष द॑धाति दधात्ये॒ष वै वा ए॒ष द॑धाति दधात्ये॒ष वै । \newline
25. ए॒ष वै वा ए॒ष ए॒ष वा अ॑र॒त्नि र॑र॒त्निर् वा ए॒ष ए॒ष वा अ॑र॒त्निः । \newline
26. वा अ॑र॒त्नि र॑र॒त्निर् वै वा अ॑र॒त्निः सा॑मिधे॒नीनाꣳ॑ सामिधे॒नीना॑ मर॒त्निर् वै वा अ॑र॒त्निः सा॑मिधे॒नीना᳚म् । \newline
27. अ॒र॒त्निः सा॑मिधे॒नीनाꣳ॑ सामिधे॒नीना॑ मर॒त्नि र॑र॒त्निः सा॑मिधे॒नीनां॒ ॅयो यः सा॑मिधे॒नीना॑ मर॒त्नि र॑र॒त्निः सा॑मिधे॒नीनां॒ ॅयः । \newline
28. सा॒मि॒धे॒नीनां॒ ॅयो यः सा॑मिधे॒नीनाꣳ॑ सामिधे॒नीनां॒ ॅय ए॒व मे॒वं ॅयः सा॑मिधे॒नीनाꣳ॑ सामिधे॒नीनां॒ ॅय ए॒वम् । \newline
29. सा॒मि॒धे॒नीना॒मिति॑ सां - इ॒धे॒नीना᳚म् । \newline
30. य ए॒व मे॒वं ॅयो य ए॒वं ॅवेद॒ वेदै॒वं ॅयो य ए॒वं ॅवेद॑ । \newline
31. ए॒वं ॅवेद॒ वेदै॒व मे॒वं ॅवेदा॑र॒त्ना व॑र॒त्नौ वेदै॒व मे॒वं ॅवेदा॑र॒त्नौ । \newline
32. वेदा॑र॒त्ना व॑र॒त्नौ वेद॒ वेदा॑र॒त्ना वे॒वैवार॒त्नौ वेद॒ वेदा॑र॒त्ना वे॒व । \newline
33. अ॒र॒त्ना वे॒वैवार॒त्ना व॑र॒त्ना वे॒व भ्रातृ॑व्य॒म् भ्रातृ॑व्य मे॒वार॒त्ना व॑र॒त्ना वे॒व भ्रातृ॑व्यम् । \newline
34. ए॒व भ्रातृ॑व्य॒म् भ्रातृ॑व्य मे॒वैव भ्रातृ॑व्यम् कुरुते कुरुते॒ भ्रातृ॑व्य मे॒वैव भ्रातृ॑व्यम् कुरुते । \newline
35. भ्रातृ॑व्यम् कुरुते कुरुते॒ भ्रातृ॑व्य॒म् भ्रातृ॑व्यम् कुरुत॒ ऋषेर्॑.ऋषे॒र्॒. ऋषेर्॑.ऋषेः कुरुते॒ भ्रातृ॑व्य॒म् भ्रातृ॑व्यम् कुरुत॒ ऋषेर्॑.ऋषेः । \newline
36. कु॒रु॒त॒ ऋषेर्॑.ऋषे॒र्॒. ऋषेर्॑.ऋषेः कुरुते कुरुत॒ ऋषेर्॑.ऋषे॒र् वै वा ऋषेर्॑.ऋषेः कुरुते कुरुत॒ ऋषेर्॑.ऋषे॒र् वै । \newline
37. ऋषेर्॑.ऋषे॒र् वै वा ऋषेर्॑.ऋषे॒र्॒. ऋषेर्॑.ऋषे॒र् वा ए॒ता ए॒ता वा ऋषेर्॑.ऋषे॒र्॒. ऋषेर्॑.ऋषे॒र् वा ए॒ताः । \newline
38. ऋषेर्॑.ऋषे॒रित्यृषेः᳚ - ऋ॒षेः॒ । \newline
39. वा ए॒ता ए॒ता वै वा ए॒ता निर्मि॑ता॒ निर्मि॑ता ए॒ता वै वा ए॒ता निर्मि॑ताः । \newline
40. ए॒ता निर्मि॑ता॒ निर्मि॑ता ए॒ता ए॒ता निर्मि॑ता॒ यद् यन् निर्मि॑ता ए॒ता ए॒ता निर्मि॑ता॒ यत् । \newline
41. निर्मि॑ता॒ यद् यन् निर्मि॑ता॒ निर्मि॑ता॒ यथ् सा॑मिधे॒न्यः॑ सामिधे॒न्यो॑ यन् निर्मि॑ता॒ निर्मि॑ता॒ यथ् सा॑मिधे॒न्यः॑ । \newline
42. निर्मि॑ता॒ इति॒ निः - मि॒ताः॒ । \newline
43. यथ् सा॑मिधे॒न्यः॑ सामिधे॒न्यो॑ यद् यथ् सा॑मिधे॒न्य॑ स्ता स्ताः सा॑मिधे॒न्यो॑ यद् यथ् सा॑मिधे॒न्य॑स्ताः । \newline
44. सा॒मि॒धे॒न्य॑ स्ता स्ताः सा॑मिधे॒न्यः॑ सामिधे॒न्य॑ स्ता यद् यत् ताः सा॑मिधे॒न्यः॑ सामिधे॒न्य॑ स्ता यत् । \newline
45. सा॒मि॒धे॒न्य॑ इति॑ सां - इ॒धे॒न्यः॑ । \newline
46. ता यद् यत् ता स्ता यदसं॑ॅयुक्ता॒ असं॑ॅयुक्ता॒ यत् ता स्ता यदसं॑ॅयुक्ताः । \newline
47. यदसं॑ॅयुक्ता॒ असं॑ॅयुक्ता॒ यद् यदसं॑ॅयुक्ताः॒ स्युः स्यु रसं॑ॅयुक्ता॒ यद् यदसं॑ॅयुक्ताः॒ स्युः । \newline
48. असं॑ॅयुक्ताः॒ स्युः स्युरसं॑ॅयुक्ता॒ असं॑ॅयुक्ताः॒ स्युः प्र॒जया᳚ प्र॒जया॒ स्यु रसं॑ॅयुक्ता॒ असं॑ॅयुक्ताः॒ स्युः प्र॒जया᳚ । \newline
49. असं॑ॅयुक्ता॒ इत्यसं᳚ - यु॒क्ताः॒ । \newline
50. स्युः प्र॒जया᳚ प्र॒जया॒ स्युः स्युः प्र॒जया॑ प॒शुभिः॑ प॒शुभिः॑ प्र॒जया॒ स्युः स्युः प्र॒जया॑ प॒शुभिः॑ । \newline
51. प्र॒जया॑ प॒शुभिः॑ प॒शुभिः॑ प्र॒जया᳚ प्र॒जया॑ प॒शुभि॒र् यज॑मानस्य॒ यज॑मानस्य प॒शुभिः॑ प्र॒जया᳚ प्र॒जया॑ प॒शुभि॒र् यज॑मानस्य । \newline
52. प्र॒जयेति॑ प्र - जया᳚ । \newline
53. प॒शुभि॒र् यज॑मानस्य॒ यज॑मानस्य प॒शुभिः॑ प॒शुभि॒र् यज॑मानस्य॒ वि वि यज॑मानस्य प॒शुभिः॑ प॒शुभि॒र् यज॑मानस्य॒ वि । \newline
54. प॒शुभि॒रिति॑ प॒शु - भिः॒ । \newline
55. यज॑मानस्य॒ वि वि यज॑मानस्य॒ यज॑मानस्य॒ वि ति॑ष्ठेरन् तिष्ठेर॒न्॒. वि यज॑मानस्य॒ यज॑मानस्य॒ वि ति॑ष्ठेरन्न् । \newline
56. वि ति॑ष्ठेरन् तिष्ठेर॒न्॒. वि वि ति॑ष्ठेरन् नर्द्ध॒र्चा व॑र्द्ध॒र्चौ ति॑ष्ठेर॒न्॒. वि वि ति॑ष्ठेरन् नर्द्ध॒र्चौ । \newline
57. ति॒ष्ठे॒र॒न् न॒र्द्ध॒र्चा व॑र्द्ध॒र्चौ ति॑ष्ठेरन् तिष्ठेरन् नर्द्ध॒र्चौ सꣳ स म॑र्द्ध॒र्चौ ति॑ष्ठेरन् तिष्ठेरन् नर्द्ध॒र्चौ सम् । \newline
58. अ॒र्द्ध॒र्चौ सꣳ स म॑र्द्ध॒र्चा व॑र्द्ध॒र्चौ सम् द॑धाति दधाति॒ स म॑र्द्ध॒र्चा व॑र्द्ध॒र्चौ सम् द॑धाति । \newline
59. अ॒र्द्ध॒र्चावित्य॑र्द्ध - ऋ॒चौ । \newline
60. सम् द॑धाति दधाति॒ सꣳ सम् द॑धाति॒ सꣳ सम् द॑धाति॒ सꣳ सम् द॑धाति॒ सम् । \newline
61. द॒धा॒ति॒ सꣳ सम् द॑धाति दधाति॒ सं ॅयु॑नक्ति युनक्ति॒ सम् द॑धाति दधाति॒ सं ॅयु॑नक्ति । \newline
62. सं ॅयु॑नक्ति युनक्ति॒ सꣳ सं ॅयु॑नक्त्ये॒वैव यु॑नक्ति॒ सꣳ सं ॅयु॑नक्त्ये॒व । \newline
63. यु॒न॒क्त्ये॒वैव यु॑नक्ति युनक्त्ये॒वैना॑ एना ए॒व यु॑नक्ति युनक्त्ये॒वैनाः᳚ । \newline
64. ए॒वैना॑ एना ए॒वैवैना॒ स्ता स्ता ए॑ना ए॒वैवैना॒ स्ताः । \newline
65. ए॒ना॒ स्ता स्ता ए॑ना एना॒ स्ता अ॑स्मा अस्मै॒ ता ए॑ना एना॒ स्ता अ॑स्मै । \newline
66. ता अ॑स्मा अस्मै॒ ता स्ता अ॑स्मै॒ संॅयु॑क्ताः॒ संॅयु॑क्ता अस्मै॒ ता स्ता अ॑स्मै॒ संॅयु॑क्ताः । \newline
67. अ॒स्मै॒ संॅयु॑क्ताः॒ संॅयु॑क्ता अस्मा अस्मै॒ संॅयु॑क्ता॒ अव॑रुद्धा॒ अव॑रुद्धाः॒ संॅयु॑क्ता अस्मा अस्मै॒ संॅयु॑क्ता॒ अव॑रुद्धाः । \newline
68. संॅयु॑क्ता॒ अव॑रुद्धा॒ अव॑रुद्धाः॒ संॅयु॑क्ताः॒ संॅयु॑क्ता॒ अव॑रुद्धाः॒ सर्वाꣳ॒॒ सर्वा॒ मव॑रुद्धाः॒ संॅयु॑क्ताः॒ संॅयु॑क्ता॒ अव॑रुद्धाः॒ सर्वा᳚म् । \newline
69. संॅयु॑क्ता॒ इति॒ सं - यु॒क्ताः॒ । \newline
70. अव॑रुद्धाः॒ सर्वाꣳ॒॒ सर्वा॒ मव॑रुद्धा॒ अव॑रुद्धाः॒ सर्वा॑ मा॒शिष॑ मा॒शिषꣳ॒॒ सर्वा॒ मव॑रुद्धा॒ अव॑रुद्धाः॒ सर्वा॑ मा॒शिष᳚म् । \newline
71. अव॑रुद्धा॒ इत्यव॑ - रु॒द्धाः॒ । \newline
72. सर्वा॑ मा॒शिष॑ मा॒शिषꣳ॒॒ सर्वाꣳ॒॒ सर्वा॑ मा॒शिष॑म् दुह्रे दुह्र आ॒शिषꣳ॒॒ सर्वाꣳ॒॒ सर्वा॑ मा॒शिष॑म् दुह्रे । \newline
73. आ॒शिष॑म् दुह्रे दुह्र आ॒शिष॑ मा॒शिष॑म् दुह्रे । \newline
74. आ॒शिष॒मित्या᳚ - शिष᳚म् । \newline
75. दु॒ह्र॒ इति॑ दुह्रे । \newline
\pagebreak
\markright{ TS 2.5.8.1  \hfill https://www.vedavms.in \hfill}
\addcontentsline{toc}{section}{ TS 2.5.8.1 }
\section*{ TS 2.5.8.1 }

\textbf{TS 2.5.8.1 } \newline
\textbf{Samhita Paata} \newline

अय॑ज्ञो॒ वा ए॒ष यो॑ऽसा॒माऽग्न॒ आ या॑हि वी॒तय॒ इत्या॑ह रथंत॒रस्यै॒ष वर्ण॒स्तं त्वा॑ स॒मिद्भि॑रङ्गिर॒ इत्या॑ह वामदे॒व्यस्यै॒ष वर्णो॑ बृ॒हद॑ग्ने सु॒वीर्य॒मित्या॑ह बृह॒त ए॒ष वर्णो॒ यदे॒तं तृ॒चम॒न्वाह॑ य॒ज्ञ्मे॒व तथ् साम॑न्वन्तं करोत्य॒ग्निर॒मुष्मि॑न् ॅलो॒क आसी॑दादि॒त्यो᳚ऽस्मिन् तावि॒मौ लो॒कावशा᳚न्ता - [  ] \newline

\textbf{Pada Paata} \newline

अय॑ज्ञ्ः । वै । ए॒षः । यः । अ॒सा॒मा । अग्ने᳚ । एति॑ । या॒हि॒ । वी॒तये᳚ । इति॑ । आ॒ह॒ । र॒थ॒न्त॒रस्येति॑ रथं-त॒रस्य॑ । ए॒षः । वर्णः॑ । तम् । त्वा॒ । स॒मिद्भि॒रिति॑ स॒मित् - भिः॒ । अ॒ङ्गि॒रः॒ । इति॑ । आ॒ह॒ । वा॒म॒दे॒व्यस्येति॑ वाम - दे॒व्यस्य॑ । ए॒षः । वर्णः॑ । बृ॒हत् । अ॒ग्ने॒ । सु॒वीर्य॒मिति॑ सु - वीर्य᳚म् । इति॑ । आ॒ह॒ । बृ॒ह॒तः । ए॒षः । वर्णः॑ । यत् । ए॒तम् । तृ॒चम् । अ॒न्वाहेत्य॑नु - आह॑ । य॒ज्ञ्म् । ए॒व । तत् । साम॑न्वन्त॒मिति॒ सामन्न्॑ - व॒न्त॒म् । क॒रो॒ति॒ । अ॒ग्निः । अ॒मुष्मिन्न्॑ । लो॒के । आसी᳚त् । आ॒दि॒त्यः । अ॒स्मिन्न् । तौ । इ॒मौ । लो॒कौ । अशा᳚न्तौ ।  \newline


\textbf{Krama Paata} \newline

अय॑ज्ञो॒ वै । वा ए॒षः । ए॒ष यः । यो॑ऽसा॒मा । अ॒सा॒मा ऽग्ने᳚ । अग्न॒ आ । आ या॑हि । या॒हि॒ वी॒तये᳚ । वी॒तय॒ इति॑ । इत्या॑ह । आ॒ह॒ र॒थ॒न्त॒रस्य॑ । र॒थ॒न्त॒रस्यै॒षः । र॒थ॒न्त॒रस्येति॑ रथम् - त॒रस्य॑ । ए॒ष वर्णः॑ । वर्ण॒स्तम् । तम् त्वा᳚ । त्वा॒ स॒मिद्भिः॑ । स॒मिद्भि॑रङ्गिरः । स॒मिद्भि॒रिति॑ स॒मित् - भिः॒ । अ॒ङ्गि॒र॒ इति॑ । इत्या॑ह । आ॒ह॒ वा॒म॒दे॒व्यस्य॑ । वा॒म॒दे॒व्यस्यै॒षः । वा॒म॒दे॒व्यस्येति॑ वाम - दे॒व्यस्य॑ । ए॒ष वर्णः॑ । वर्णो॑ बृ॒हत् । बृ॒हद॑ग्ने । अ॒ग्ने॒ सु॒वीर्य᳚म् । सु॒वीर्य॒मिति॑ । सु॒वीर्य॒मिति॑ सु - वीर्य᳚म् । इत्या॑ह । आ॒ह॒ बृ॒ह॒तः । बृ॒ह॒त ए॒षः । ए॒ष वर्णः॑ । वर्णो॒ यत् । यदे॒तम् । ए॒तम् तृ॒चम् । तृ॒चम॒न्वाह॑ । अ॒न्वाह॑ य॒ज्ञ्म् । अ॒न्वाहेत्य॑नु - आह॑ । य॒ज्ञ्मे॒व । ए॒व तत् । तथ् साम॑न्वन्तम् । साम॑न्वन्तम् करोति । साम॑न्वन्त॒मिति॒ सामन्न्॑ - व॒न्त॒म् । क॒रो॒त्य॒ग्निः । अ॒ग्निर॒मुष्मिन्न्॑ । अ॒मुष्मि॑न् ॅलो॒के । लो॒क आसी᳚त् । आसी॑दादि॒त्यः । आ॒दि॒त्यो᳚ ऽस्मिन्न् । अ॒स्मिन् तौ । तावि॒मौ । इ॒मौ लो॒कौ । लो॒कावशा᳚न्तौ । अशा᳚न्तावास्ताम् \newline

\textbf{Jatai Paata} \newline

1. अय॑ज्ञो॒ वै वा अय॒ज्ञो ऽय॑ज्ञो॒ वै । \newline
2. वा ए॒ष ए॒ष वै वा ए॒षः । \newline
3. ए॒ष यो य ए॒ष ए॒ष यः । \newline
4. यो॑ ऽसा॒मा ऽसा॒मा यो यो॑ ऽसा॒मा । \newline
5. अ॒सा॒मा ऽग्ने ऽग्ने॑ ऽसा॒मा ऽसा॒मा ऽग्ने᳚ । \newline
6. अग्न॒ आ ऽग्ने ऽग्न॒ आ । \newline
7. आ या॑हि या॒ह्या या॑हि । \newline
8. या॒हि॒ वी॒तये॑ वी॒तये॑ याहि याहि वी॒तये᳚ । \newline
9. वी॒तय॒ इतीति॑ वी॒तये॑ वी॒तय॒ इति॑ । \newline
10. इत्या॑हा॒हे तीत्या॑ह । \newline
11. आ॒ह॒ र॒थ॒न्त॒रस्य॑ रथन्त॒रस्या॑ हाह रथन्त॒रस्य॑ । \newline
12. र॒थ॒न्त॒र स्यै॒ष ए॒ष र॑थन्त॒रस्य॑ रथन्त॒र स्यै॒षः । \newline
13. र॒थ॒न्त॒रस्येति॑ रथं - त॒रस्य॑ । \newline
14. ए॒ष वर्णो॒ वर्ण॑ ए॒ष ए॒ष वर्णः॑ । \newline
15. वर्ण॒ स्तम् तं ॅवर्णो॒ वर्ण॒ स्तम् । \newline
16. तम् त्वा᳚ त्वा॒ तम् तम् त्वा᳚ । \newline
17. त्वा॒ स॒मिद्भिः॑ स॒मिद्भि॑ स्त्वा त्वा स॒मिद्भिः॑ । \newline
18. स॒मिद्भि॑ रङ्गिरो अङ्गिरः स॒मिद्भिः॑ स॒मिद्भि॑ रङ्गिरः । \newline
19. स॒मिद्भि॒रिति॑ स॒मित् - भिः॒ । \newline
20. अ॒ङ्गि॒र॒ इती त्य॑ङ्गिरो ऽङ्गिर॒ इति॑ । \newline
21. इत्या॑हा॒हे तीत्या॑ह । \newline
22. आ॒ह॒ वा॒म॒दे॒व्यस्य॑ वामदे॒व्यस्या॑ हाह वामदे॒व्यस्य॑ । \newline
23. वा॒म॒दे॒व्य स्यै॒ष ए॒ष वा॑मदे॒व्यस्य॑ वामदे॒व्य स्यै॒षः । \newline
24. वा॒म॒दे॒व्यस्येति॑ वाम - दे॒व्यस्य॑ । \newline
25. ए॒ष वर्णो॒ वर्ण॑ ए॒ष ए॒ष वर्णः॑ । \newline
26. वर्णो॑ बृ॒हद् बृ॒हद् वर्णो॒ वर्णो॑ बृ॒हत् । \newline
27. बृ॒ह द॑ग्ने ऽग्ने बृ॒हद् बृ॒ह द॑ग्ने । \newline
28. अ॒ग्ने॒ सु॒वीर्यꣳ॑ सु॒वीर्य॑ मग्ने ऽग्ने सु॒वीर्य᳚म् । \newline
29. सु॒वीर्य॒ मितीति॑ सु॒वीर्यꣳ॑ सु॒वीर्य॒ मिति॑ । \newline
30. सु॒वीर्य॒मिति॑ सु - वीर्य᳚म् । \newline
31. इत्या॑हा॒हे तीत्या॑ह । \newline
32. आ॒ह॒ बृ॒ह॒तो बृ॑ह॒त आ॑हाह बृह॒तः । \newline
33. बृ॒ह॒त ए॒ष ए॒ष बृ॑ह॒तो बृ॑ह॒त ए॒षः । \newline
34. ए॒ष वर्णो॒ वर्ण॑ ए॒ष ए॒ष वर्णः॑ । \newline
35. वर्णो॒ यद् यद् वर्णो॒ वर्णो॒ यत् । \newline
36. यदे॒त मे॒तं ॅयद् यदे॒तम् । \newline
37. ए॒तम् तृ॒चम् तृ॒च मे॒त मे॒तम् तृ॒चम् । \newline
38. तृ॒च म॒न्वाहा॒ न्वाह॑ तृ॒चम् तृ॒च म॒न्वाह॑ । \newline
39. अ॒न्वाह॑ य॒ज्ञ्ं ॅय॒ज्ञ् म॒न्वाहा॒ न्वाह॑ य॒ज्ञ्म् । \newline
40. अ॒न्वाहेत्य॑नु - आह॑ । \newline
41. य॒ज्ञ् मे॒वैव य॒ज्ञ्ं ॅय॒ज्ञ् मे॒व । \newline
42. ए॒व तत् तदे॒वैव तत् । \newline
43. तथ् साम॑न्वन्तꣳ॒॒ साम॑न्वन्त॒म् तत् तथ् साम॑न्वन्तम् । \newline
44. साम॑न्वन्तम् करोति करोति॒ साम॑न्वन्तꣳ॒॒ साम॑न्वन्तम् करोति । \newline
45. साम॑न्वन्त॒मिति॒ सामन्न्॑ - व॒न्त॒म् । \newline
46. क॒रो॒ त्य॒ग्नि र॒ग्निः क॑रोति करो त्य॒ग्निः । \newline
47. अ॒ग्नि र॒मुष्मि॑न् न॒मुष्मि॑न् न॒ग्नि र॒ग्नि र॒मुष्मिन्न्॑ । \newline
48. अ॒मुष्मि॑न् ॅलो॒के लो॒के॑ ऽमुष्मि॑न् न॒मुष्मि॑न् ॅलो॒के । \newline
49. लो॒क आसी॒ दासी᳚ ल्लो॒के लो॒क आसी᳚त् । \newline
50. आसी॑ दादि॒त्य आ॑दि॒त्य आसी॒ दासी॑ दादि॒त्यः । \newline
51. आ॒दि॒त्यो᳚ ऽस्मिन् न॒स्मिन् ना॑दि॒त्य आ॑दि॒त्यो᳚ ऽस्मिन्न् । \newline
52. अ॒स्मिन् तौ ता व॒स्मिन् न॒स्मिन् तौ । \newline
53. ता वि॒मा वि॒मौ तौ ता वि॒मौ । \newline
54. इ॒मौ लो॒कौ लो॒का वि॒मा वि॒मौ लो॒कौ । \newline
55. लो॒का वशा᳚न्ता॒ वशा᳚न्तौ लो॒कौ लो॒का वशा᳚न्तौ । \newline
56. अशा᳚न्ता वास्ता मास्ता॒ मशा᳚न्ता॒ वशा᳚न्ता वास्ताम् । \newline

\textbf{Ghana Paata } \newline

1. अय॑ज्ञो॒ वै वा अय॒ज्ञो ऽय॑ज्ञो॒ वा ए॒ष ए॒ष वा अय॒ज्ञो ऽय॑ज्ञो॒ वा ए॒षः । \newline
2. वा ए॒ष ए॒ष वै वा ए॒ष यो य ए॒ष वै वा ए॒ष यः । \newline
3. ए॒ष यो य ए॒ष ए॒ष यो॑ ऽसा॒मा ऽसा॒मा य ए॒ष ए॒ष यो॑ ऽसा॒मा । \newline
4. यो॑ ऽसा॒मा ऽसा॒मा यो यो॑ ऽसा॒मा ऽग्ने ऽग्ने॑ ऽसा॒मा यो यो॑ ऽसा॒मा ऽग्ने᳚ । \newline
5. अ॒सा॒मा ऽग्ने ऽग्ने॑ ऽसा॒मा ऽसा॒मा ऽग्न॒ आ ऽग्ने॑ ऽसा॒मा ऽसा॒मा ऽग्न॒ आ । \newline
6. अग्न॒ आ ऽग्ने ऽग्न॒ आ या॑हि या॒ह्या ऽग्ने ऽग्न॒ आ या॑हि । \newline
7. आ या॑हि या॒ह्या या॑हि वी॒तये॑ वी॒तये॑ या॒ह्या या॑हि वी॒तये᳚ । \newline
8. या॒हि॒ वी॒तये॑ वी॒तये॑ याहि याहि वी॒तय॒ इतीति॑ वी॒तये॑ याहि याहि वी॒तय॒ इति॑ । \newline
9. वी॒तय॒ इतीति॑ वी॒तये॑ वी॒तय॒ इत्या॑हा॒हे ति॑ वी॒तये॑ वी॒तय॒ इत्या॑ह । \newline
10. इत्या॑हा॒हे तीत्या॑ह रथन्त॒रस्य॑ रथन्त॒रस्या॒हे तीत्या॑ह रथन्त॒रस्य॑ । \newline
11. आ॒ह॒ र॒थ॒न्त॒रस्य॑ रथन्त॒रस्या॑हाह रथन्त॒रस्यै॒ष ए॒ष र॑थन्त॒रस्या॑हाह रथन्त॒रस्यै॒षः । \newline
12. र॒थ॒न्त॒रस्यै॒ष ए॒ष र॑थन्त॒रस्य॑ रथन्त॒रस्यै॒ष वर्णो॒ वर्ण॑ ए॒ष र॑थन्त॒रस्य॑ रथन्त॒रस्यै॒ष वर्णः॑ । \newline
13. र॒थ॒न्त॒रस्येति॑ रथं - त॒रस्य॑ । \newline
14. ए॒ष वर्णो॒ वर्ण॑ ए॒ष ए॒ष वर्ण॒ स्तम् तं ॅवर्ण॑ ए॒ष ए॒ष वर्ण॒ स्तम् । \newline
15. वर्ण॒ स्तम् तं ॅवर्णो॒ वर्ण॒ स्तम् त्वा᳚ त्वा॒ तं ॅवर्णो॒ वर्ण॒ स्तम् त्वा᳚ । \newline
16. तम् त्वा᳚ त्वा॒ तम् तम् त्वा॑ स॒मिद्भिः॑ स॒मिद्भि॑ स्त्वा॒ तम् तम् त्वा॑ स॒मिद्भिः॑ । \newline
17. त्वा॒ स॒मिद्भिः॑ स॒मिद्भि॑ स्त्वा त्वा स॒मिद्भि॑ रङ्गिरो अङ्गिरः स॒मिद्भि॑ स्त्वा त्वा स॒मिद्भि॑ रङ्गिरः । \newline
18. स॒मिद्भि॑ रङ्गिरो अङ्गिरः स॒मिद्भिः॑ स॒मिद्भि॑ रङ्गिर॒ इती त्य॑ङ्गिरः स॒मिद्भिः॑ स॒मिद्भि॑ रङ्गिर॒ इति॑ । \newline
19. स॒मिद्भि॒रिति॑ स॒मित् - भिः॒ । \newline
20. अ॒ङ्गि॒र॒ इती त्य॑ङ्गिरो ऽङ्गिर॒ इत्या॑हा॒हे त्य॑ङ्गिरो ऽङ्गिर॒ इत्या॑ह । \newline
21. इत्या॑हा॒हे तीत्या॑ह वामदे॒व्यस्य॑ वामदे॒व्यस्या॒हे तीत्या॑ह वामदे॒व्यस्य॑ । \newline
22. आ॒ह॒ वा॒म॒दे॒व्यस्य॑ वामदे॒व्यस्या॑हाह वामदे॒व्यस्यै॒ष ए॒ष वा॑मदे॒व्यस्या॑हाह वामदे॒व्यस्यै॒षः । \newline
23. वा॒म॒दे॒व्यस्यै॒ष ए॒ष वा॑मदे॒व्यस्य॑ वामदे॒व्यस्यै॒ष वर्णो॒ वर्ण॑ ए॒ष वा॑मदे॒व्यस्य॑ वामदे॒व्यस्यै॒ष वर्णः॑ । \newline
24. वा॒म॒दे॒व्यस्येति॑ वाम - दे॒व्यस्य॑ । \newline
25. ए॒ष वर्णो॒ वर्ण॑ ए॒ष ए॒ष वर्णो॑ बृ॒हद् बृ॒हद् वर्ण॑ ए॒ष ए॒ष वर्णो॑ बृ॒हत् । \newline
26. वर्णो॑ बृ॒हद् बृ॒हद् वर्णो॒ वर्णो॑ बृ॒हद॑ग्ने ऽग्ने बृ॒हद् वर्णो॒ वर्णो॑ बृ॒हद॑ग्ने । \newline
27. बृ॒हद॑ग्ने ऽग्ने बृ॒हद् बृ॒हद॑ग्ने सु॒वीर्यꣳ॑ सु॒वीर्य॑ मग्ने बृ॒हद् बृ॒हद॑ग्ने सु॒वीर्य᳚म् । \newline
28. अ॒ग्ने॒ सु॒वीर्यꣳ॑ सु॒वीर्य॑ मग्ने ऽग्ने सु॒वीर्य॒ मितीति॑ सु॒वीर्य॑ मग्ने ऽग्ने सु॒वीर्य॒ मिति॑ । \newline
29. सु॒वीर्य॒ मितीति॑ सु॒वीर्यꣳ॑ सु॒वीर्य॒ मित्या॑हा॒हे ति॑ सु॒वीर्यꣳ॑ सु॒वीर्य॒ मित्या॑ह । \newline
30. सु॒वीर्य॒मिति॑ सु - वीर्य᳚म् । \newline
31. इत्या॑हा॒हे तीत्या॑ह बृह॒तो बृ॑ह॒त आ॒हे तीत्या॑ह बृह॒तः । \newline
32. आ॒ह॒ बृ॒ह॒तो बृ॑ह॒त आ॑हाह बृह॒त ए॒ष ए॒ष बृ॑ह॒त आ॑हाह बृह॒त ए॒षः । \newline
33. बृ॒ह॒त ए॒ष ए॒ष बृ॑ह॒तो बृ॑ह॒त ए॒ष वर्णो॒ वर्ण॑ ए॒ष बृ॑ह॒तो बृ॑ह॒त ए॒ष वर्णः॑ । \newline
34. ए॒ष वर्णो॒ वर्ण॑ ए॒ष ए॒ष वर्णो॒ यद् यद् वर्ण॑ ए॒ष ए॒ष वर्णो॒ यत् । \newline
35. वर्णो॒ यद् यद् वर्णो॒ वर्णो॒ यदे॒त मे॒तं ॅयद् वर्णो॒ वर्णो॒ यदे॒तम् । \newline
36. यदे॒त मे॒तं ॅयद् यदे॒तम् तृ॒चम् तृ॒च मे॒तं ॅयद् यदे॒तम् तृ॒चम् । \newline
37. ए॒तम् तृ॒चम् तृ॒च मे॒त मे॒तम् तृ॒च म॒न्वाहा॒ न्वाह॑ तृ॒च मे॒त मे॒तम् तृ॒च म॒न्वाह॑ । \newline
38. तृ॒च म॒न्वाहा॒ न्वाह॑ तृ॒चम् तृ॒च म॒न्वाह॑ य॒ज्ञ्ं ॅय॒ज्ञ् म॒न्वाह॑ तृ॒चम् तृ॒च म॒न्वाह॑ य॒ज्ञ्म् । \newline
39. अ॒न्वाह॑ य॒ज्ञ्ं ॅय॒ज्ञ् म॒न्वाहा॒ न्वाह॑ य॒ज्ञ् मे॒वैव य॒ज्ञ् म॒न्वाहा॒ न्वाह॑ य॒ज्ञ् मे॒व । \newline
40. अ॒न्वाहेत्य॑नु - आह॑ । \newline
41. य॒ज्ञ् मे॒वैव य॒ज्ञ्ं ॅय॒ज्ञ् मे॒व तत् तदे॒व य॒ज्ञ्ं ॅय॒ज्ञ् मे॒व तत् । \newline
42. ए॒व तत् तदे॒वैव तथ् साम॑न्वन्तꣳ॒॒ साम॑न्वन्त॒म् तदे॒वैव तथ् साम॑न्वन्तम् । \newline
43. तथ् साम॑न्वन्तꣳ॒॒ साम॑न्वन्त॒म् तत् तथ् साम॑न्वन्तम् करोति करोति॒ साम॑न्वन्त॒म् तत् तथ् साम॑न्वन्तम् करोति । \newline
44. साम॑न्वन्तम् करोति करोति॒ साम॑न्वन्तꣳ॒॒ साम॑न्वन्तम् करो त्य॒ग्नि र॒ग्निः क॑रोति॒ साम॑न्वन्तꣳ॒॒ साम॑न्वन्तम् करोत्य॒ग्निः । \newline
45. साम॑न्वन्त॒मिति॒ सामन्न्॑ - व॒न्त॒म् । \newline
46. क॒रो॒ त्य॒ग्नि र॒ग्निः क॑रोति करो त्य॒ग्नि र॒मुष्मि॑न् न॒मुष्मि॑न् न॒ग्निः क॑रोति करो त्य॒ग्नि र॒मुष्मिन्न्॑ । \newline
47. अ॒ग्नि र॒मुष्मि॑न् न॒मुष्मि॑न् न॒ग्नि र॒ग्नि र॒मुष्मि॑न् ॅलो॒के लो॒के॑ ऽमुष्मि॑न् न॒ग्नि र॒ग्नि र॒मुष्मि॑न् ॅलो॒के । \newline
48. अ॒मुष्मि॑न् ॅलो॒के लो॒के॑ ऽमुष्मि॑न् न॒मुष्मि॑न् ॅलो॒क आसी॒ दासी᳚ ल्लो॒के॑ ऽमुष्मि॑न् न॒मुष्मि॑न् ॅलो॒क आसी᳚त् । \newline
49. लो॒क आसी॒ दासी᳚ ल्लो॒के लो॒क आसी॑ दादि॒त्य आ॑दि॒त्य आसी᳚ ल्लो॒के लो॒क आसी॑ दादि॒त्यः । \newline
50. आसी॑ दादि॒त्य आ॑दि॒त्य आसी॒ दासी॑ दादि॒त्यो᳚ ऽस्मिन् न॒स्मिन् ना॑दि॒त्य आसी॒ दासी॑ दादि॒त्यो᳚ ऽस्मिन्न् । \newline
51. आ॒दि॒त्यो᳚ ऽस्मिन् न॒स्मिन् ना॑दि॒त्य आ॑दि॒त्यो᳚ ऽस्मिन् तौ ता व॒स्मिन् ना॑दि॒त्य आ॑दि॒त्यो᳚ ऽस्मिन् तौ । \newline
52. अ॒स्मिन् तौ ता व॒स्मिन् न॒स्मिन् ता वि॒मा वि॒मौ ता व॒स्मिन् न॒स्मिन् ता वि॒मौ । \newline
53. ता वि॒मा वि॒मौ तौ ता वि॒मौ लो॒कौ लो॒का वि॒मौ तौ ता वि॒मौ लो॒कौ । \newline
54. इ॒मौ लो॒कौ लो॒का वि॒मा वि॒मौ लो॒का वशा᳚न्ता॒ वशा᳚न्तौ लो॒का वि॒मा वि॒मौ लो॒का वशा᳚न्तौ । \newline
55. लो॒का वशा᳚न्ता॒ वशा᳚न्तौ लो॒कौ लो॒का वशा᳚न्ता वास्ता मास्ता॒ मशा᳚न्तौ लो॒कौ लो॒का वशा᳚न्ता वास्ताम् । \newline
56. अशा᳚न्ता वास्ता मास्ता॒ मशा᳚न्ता॒ वशा᳚न्ता वास्ता॒म् ते त आ᳚स्ता॒ मशा᳚न्ता॒ वशा᳚न्ता वास्ता॒म् ते । \newline
\pagebreak
\markright{ TS 2.5.8.2  \hfill https://www.vedavms.in \hfill}
\addcontentsline{toc}{section}{ TS 2.5.8.2 }
\section*{ TS 2.5.8.2 }

\textbf{TS 2.5.8.2 } \newline
\textbf{Samhita Paata} \newline

वास्तां॒ ते दे॒वा अ॑ब्रुव॒न्नेते॒मौ वि पर्यू॑हा॒मेत्यग्न॒ आ या॑हि वी॒तय॒ इत्य॒स्मिन् ॅलो॒के᳚ऽग्निम॑दधु र्बृ॒हद॑ग्ने सु॒वीर्य॒मित्य॒मुष्मि॑न् ॅलो॒क आ॑दि॒त्यं ततो॒ वा इ॒मौ लो॒काव॑शाम्यतां॒ ॅयदे॒वम॒न्वाहा॒नयो᳚ र्लो॒कयोः॒ शान्त्यै॒ शाम्य॑तोऽस्मा इ॒मौ लो॒कौ य ए॒वं ॅवेद॒ पञ्च॑दश सामिधे॒नीरन्वा॑ह॒ पञ्च॑दश॒ - [  ] \newline

\textbf{Pada Paata} \newline

आ॒स्ता॒म् । ते । दे॒वाः । अ॒ब्रु॒व॒न्न् । एति॑ । इ॒त॒ । इ॒मौ । वि । परीति॑ । ऊ॒हा॒म॒ । इति॑ । अग्ने᳚ । एति॑ । या॒हि॒ । वी॒तये᳚ । इति॑ । अ॒स्मिन्न् । लो॒के । अ॒ग्निम् । अ॒द॒धुः॒ । बृ॒हत् । अ॒ग्ने॒ । सु॒वीर्य॒मिति॑ सु - वीर्य᳚म् । इति॑ । अ॒मुष्मिन्न्॑ । लो॒के । आ॒दि॒त्यम् । ततः॑ । वै । इ॒मौ । लो॒कौ । अ॒शा॒म्य॒ता॒म् । यत् । ए॒वम् । अ॒न्वाहेत्य॑नु - आह॑ । अ॒नयोः᳚ । लो॒कयोः᳚ । शान्त्यै᳚ । शाम्य॑तः । अ॒स्मै॒ । इ॒मौ । लो॒कौ । यः । ए॒वम् । वेद॑ । पञ्च॑द॒शेति॒ पञ्च॑ - द॒श॒ । सा॒मि॒धे॒नीरिति॑ सां - इ॒धे॒नीः । अन्विति॑ । आ॒ह॒ । पञ्च॑द॒शेति॒ पञ्च॑ - द॒श॒ ।  \newline


\textbf{Krama Paata} \newline

आ॒स्ता॒म् ते । ते दे॒वाः । दे॒वा अ॑ब्रुवन्न् । अ॒ब्रु॒व॒न्ना । एत॑ । इ॒ते॒मौ । इ॒मौ वि । वि परि॑ । पर्यू॑हाम । ऊ॒हा॒मेति॑ । इत्यग्ने᳚ । अग्न॒ आ । आ या॑हि । या॒हि॒ वी॒तये᳚ । वी॒तय॒ इति॑ । इत्य॒स्मिन्न् । अ॒स्मिन् ॅलो॒के । लो॒के᳚ ऽग्निम् । अ॒ग्निम॑दधुः । अ॒द॒धु॒र् बृ॒हत् । बृ॒हद॑ग्ने । अ॒ग्ने॒ सु॒वीर्य᳚म् । सु॒वीर्य॒मिति॑ । सु॒वीर्य॒मिति॑ सु - वीर्य᳚म् । इत्य॒मुष्मिन्न्॑ । अ॒मुष्मि॑न् ॅलो॒के । लो॒क आ॑दि॒त्यम् । आ॒दि॒त्यम् ततः॑ । ततो॒ वै । वा इ॒मौ । इ॒मौ लो॒कौ । लो॒काव॑शाम्यताम् । अ॒शा॒म्य॒ता॒म् ॅयत् । यदे॒वम् । ए॒वम॒न्वाह॑ । अ॒न्वाहा॒नयोः᳚ । अ॒न्वाहेत्य॑नु - आह॑ । अ॒नयो᳚र् लो॒कयोः᳚ । लो॒कयोः॒ शान्त्यै᳚ । शान्त्यै॒ शाम्य॑तः । शाम्य॑तो ऽस्मै । अ॒स्मा॒ इ॒मौ । इ॒मौ लो॒कौ । लो॒कौ यः । य ए॒वम् । ए॒वम् ॅवेद॑ । वेद॒ पञ्च॑दश । पञ्च॑दश सामिधे॒नीः । पञ्च॑द॒शेति॒ पञ्च॑ - द॒श॒ । सा॒मि॒धे॒नीरनु॑ । सा॒मि॒धे॒नीरिति॑ साम् - इ॒धे॒नीः । अन्वा॑ह । आ॒ह॒ पञ्च॑दश । पञ्च॑दश॒ वै । पञ्च॑द॒शेति॒ पञ्च॑ - द॒श॒ \newline

\textbf{Jatai Paata} \newline

1. आ॒स्ता॒म् ते त आ᳚स्ता मास्ता॒म् ते । \newline
2. ते दे॒वा दे॒वा स्ते ते दे॒वाः । \newline
3. दे॒वा अ॑ब्रुवन् नब्रुवन् दे॒वा दे॒वा अ॑ब्रुवन्न् । \newline
4. अ॒ब्रु॒व॒न् ना ऽब्रु॑वन् नब्रुव॒न् ना । \newline
5. एते॒ तेत॑ । \newline
6. इ॒ते॒ मा वि॒मा वि॑ते ते॒ मौ । \newline
7. इ॒मौ वि वीमा वि॒मौ वि । \newline
8. वि परि॒ परि॒ वि वि परि॑ । \newline
9. पर्यू॑हामो हाम॒ परि॒ पर्यू॑हाम । \newline
10. ऊ॒हा॒मे तीत्यू॑हामो हा॒मे ति॑ । \newline
11. इत्यग्ने ऽग्न॒ इती त्यग्ने᳚ । \newline
12. अग्न॒ आ ऽग्ने ऽग्न॒ आ । \newline
13. आ या॑हि या॒ह्या या॑हि । \newline
14. या॒हि॒ वी॒तये॑ वी॒तये॑ याहि याहि वी॒तये᳚ । \newline
15. वी॒तय॒ इतीति॑ वी॒तये॑ वी॒तय॒ इति॑ । \newline
16. इत्य॒स्मिन् न॒स्मिन् निती त्य॒स्मिन्न् । \newline
17. अ॒स्मिन् ॅलो॒के लो॒के᳚ ऽस्मिन् न॒स्मिन् ॅलो॒के । \newline
18. लो॒के᳚ ऽग्नि म॒ग्निम् ॅलो॒के लो॒के᳚ ऽग्निम् । \newline
19. अ॒ग्नि म॑दधु रदधु र॒ग्नि म॒ग्नि म॑दधुः । \newline
20. अ॒द॒धु॒र् बृ॒हद् बृ॒ह द॑दधु रदधुर् बृ॒हत् । \newline
21. बृ॒ह द॑ग्ने ऽग्ने बृ॒हद् बृ॒ह द॑ग्ने । \newline
22. अ॒ग्ने॒ सु॒वीर्यꣳ॑ सु॒वीर्य॑ मग्ने ऽग्ने सु॒वीर्य᳚म् । \newline
23. सु॒वीर्य॒ मितीति॑ सु॒वीर्यꣳ॑ सु॒वीर्य॒ मिति॑ । \newline
24. सु॒वीर्य॒मिति॑ सु - वीर्य᳚म् । \newline
25. इत्य॒मुष्मि॑न् न॒मुष्मि॒न् निती त्य॒मुष्मिन्न्॑ । \newline
26. अ॒मुष्मि॑न् ॅलो॒के लो॒के॑ ऽमुष्मि॑न् न॒मुष्मि॑न् ॅलो॒के । \newline
27. लो॒क आ॑दि॒त्य मा॑दि॒त्यम् ॅलो॒के लो॒क आ॑दि॒त्यम् । \newline
28. आ॒दि॒त्यम् तत॒ स्तत॑ आदि॒त्य मा॑दि॒त्यम् ततः॑ । \newline
29. ततो॒ वै वै तत॒ स्ततो॒ वै । \newline
30. वा इ॒मा वि॒मौ वै वा इ॒मौ । \newline
31. इ॒मौ लो॒कौ लो॒का वि॒मा वि॒मौ लो॒कौ । \newline
32. लो॒का व॑शाम्यता मशाम्यताम् ॅलो॒कौ लो॒का व॑शाम्यताम् । \newline
33. अ॒शा॒म्य॒तां॒ ॅयद् यद॑शाम्यता मशाम्यतां॒ ॅयत् । \newline
34. यदे॒व मे॒वं ॅयद् यदे॒वम् । \newline
35. ए॒व म॒न्वाहा॒ न्वाहै॒व मे॒व म॒न्वाह॑ । \newline
36. अ॒न्वाहा॒ नयो॑ र॒नयो॑ र॒न्वाहा॒ न्वाहा॒ नयोः᳚ । \newline
37. अ॒न्वाहेत्य॑नु - आह॑ । \newline
38. अ॒नयो᳚र् लो॒कयो᳚र् लो॒कयो॑ र॒नयो॑ र॒नयो᳚र् लो॒कयोः᳚ । \newline
39. लो॒कयोः॒ शान्त्यै॒ शान्त्यै॑ लो॒कयो᳚र् लो॒कयोः॒ शान्त्यै᳚ । \newline
40. शान्त्यै॒ शाम्य॑तः॒ शाम्य॑तः॒ शान्त्यै॒ शान्त्यै॒ शाम्य॑तः । \newline
41. शाम्य॑तो ऽस्मा अस्मै॒ शाम्य॑तः॒ शाम्य॑तो ऽस्मै । \newline
42. अ॒स्मा॒ इ॒मा वि॒मा व॑स्मा अस्मा इ॒मौ । \newline
43. इ॒मौ लो॒कौ लो॒का वि॒मा वि॒मौ लो॒कौ । \newline
44. लो॒कौ यो यो लो॒कौ लो॒कौ यः । \newline
45. य ए॒व मे॒वं ॅयो य ए॒वम् । \newline
46. ए॒वं ॅवेद॒ वेदै॒व मे॒वं ॅवेद॑ । \newline
47. वेद॒ पञ्च॑दश॒ पञ्च॑दश॒ वेद॒ वेद॒ पञ्च॑दश । \newline
48. पञ्च॑दश सामिधे॒नीः सा॑मिधे॒नीः पञ्च॑दश॒ पञ्च॑दश सामिधे॒नीः । \newline
49. पञ्च॑द॒शेति॒ पञ्च॑ - द॒श॒ । \newline
50. सा॒मि॒धे॒नी रन्वनु॑ सामिधे॒नीः सा॑मिधे॒नी रनु॑ । \newline
51. सा॒मि॒धे॒नीरिति॑ सां - इ॒धे॒नीः । \newline
52. अन्वा॑ हा॒हा न्वन्वा॑ह । \newline
53. आ॒ह॒ पञ्च॑दश॒ पञ्च॑दशा हाह॒ पञ्च॑दश । \newline
54. पञ्च॑दश॒ वै वै पञ्च॑दश॒ पञ्च॑दश॒ वै । \newline
55. पञ्च॑द॒शेति॒ पञ्च॑ - द॒श॒ । \newline

\textbf{Ghana Paata } \newline

1. आ॒स्ता॒म् ते त आ᳚स्ता मास्ता॒म् ते दे॒वा दे॒वा स्त आ᳚स्ता मास्ता॒म् ते दे॒वाः । \newline
2. ते दे॒वा दे॒वा स्ते ते दे॒वा अ॑ब्रुवन् नब्रुवन् दे॒वा स्ते ते दे॒वा अ॑ब्रुवन्न् । \newline
3. दे॒वा अ॑ब्रुवन् नब्रुवन् दे॒वा दे॒वा अ॑ब्रुव॒न् ना ऽब्रु॑वन् दे॒वा दे॒वा अ॑ब्रुव॒न् ना । \newline
4. अ॒ब्रु॒व॒न् ना ऽब्रु॑वन् नब्रुव॒न् नेते॒ ता ऽब्रु॑वन् नब्रुव॒न् नेत॑ । \newline
5. एते॒ तेते॒ मा वि॒मा वि॒तेते॒ मौ । \newline
6. इ॒ते॒ मा वि॒मा वि॑ते ते॒ मौ वि वीमा वि॑ते ते॒ मौ वि । \newline
7. इ॒मौ वि वीमा वि॒मौ वि परि॒ परि॒ वीमा वि॒मौ वि परि॑ । \newline
8. वि परि॒ परि॒ वि वि पर्यू॑हा मोहाम॒ परि॒ वि वि पर्यू॑हाम । \newline
9. पर्यू॑हा मोहाम॒ परि॒ पर्यू॑हा॒मे तीत्यू॑हाम॒ परि॒ पर्यू॑हा॒मे ति॑ । \newline
10. ऊ॒हा॒मे तीत्यू॑हा मोहा॒मे त्यग्ने ऽग्न॒ इत्यू॑हा मोहा॒मे त्यग्ने᳚ । \newline
11. इत्यग्ने ऽग्न॒ इतीत्यग्न॒ आ ऽग्न॒ इतीत्यग्न॒ आ । \newline
12. अग्न॒ आ ऽग्ने ऽग्न॒ आ या॑हि या॒ह्या ऽग्ने ऽग्न॒ आ या॑हि । \newline
13. आ या॑हि या॒ह्या या॑हि वी॒तये॑ वी॒तये॑ या॒ह्या या॑हि वी॒तये᳚ । \newline
14. या॒हि॒ वी॒तये॑ वी॒तये॑ याहि याहि वी॒तय॒ इतीति॑ वी॒तये॑ याहि याहि वी॒तय॒ इति॑ । \newline
15. वी॒तय॒ इतीति॑ वी॒तये॑ वी॒तय॒ इत्य॒स्मिन् न॒स्मिन् निति॑ वी॒तये॑ वी॒तय॒ इत्य॒स्मिन्न् । \newline
16. इत्य॒स्मिन् न॒स्मिन् नितीत्य॒स्मिन् ॅलो॒के लो॒के᳚ ऽस्मिन् नितीत्य॒स्मिन् ॅलो॒के । \newline
17. अ॒स्मिन् ॅलो॒के लो॒के᳚ ऽस्मिन् न॒स्मिन् ॅलो॒के᳚ ऽग्नि म॒ग्निम् ॅलो॒के᳚ ऽस्मिन् न॒स्मिन् ॅलो॒के᳚ ऽग्निम् । \newline
18. लो॒के᳚ ऽग्नि म॒ग्निम् ॅलो॒के लो॒के᳚ ऽग्नि म॑दधु रदधु र॒ग्निम् ॅलो॒के लो॒के᳚ ऽग्नि म॑दधुः । \newline
19. अ॒ग्नि म॑दधु रदधु र॒ग्नि म॒ग्नि म॑दधुर् बृ॒हद् बृ॒ह द॑दधु र॒ग्नि म॒ग्नि म॑दधुर् बृ॒हत् । \newline
20. अ॒द॒धु॒र् बृ॒हद् बृ॒ह द॑दधु रदधुर् बृ॒हद॑ग्ने ऽग्ने बृ॒ह द॑दधु रदधुर् बृ॒हद॑ग्ने । \newline
21. बृ॒हद॑ग्ने ऽग्ने बृ॒हद् बृ॒हद॑ग्ने सु॒वीर्यꣳ॑ सु॒वीर्य॑ मग्ने बृ॒हद् बृ॒हद॑ग्ने सु॒वीर्य᳚म् । \newline
22. अ॒ग्ने॒ सु॒वीर्यꣳ॑ सु॒वीर्य॑ मग्ने ऽग्ने सु॒वीर्य॒ मितीति॑ सु॒वीर्य॑ मग्ने ऽग्ने सु॒वीर्य॒ मिति॑ । \newline
23. सु॒वीर्य॒ मितीति॑ सु॒वीर्यꣳ॑ सु॒वीर्य॒ मित्य॒मुष्मि॑न् न॒मुष्मि॒न् निति॑ सु॒वीर्यꣳ॑ सु॒वीर्य॒ मित्य॒मुष्मिन्न्॑ । \newline
24. सु॒वीर्य॒मिति॑ सु - वीर्य᳚म् । \newline
25. इत्य॒मुष्मि॑न् न॒मुष्मि॒न् निती त्य॒मुष्मि॑न् ॅलो॒के लो॒के॑ ऽमुष्मि॒न् निती त्य॒मुष्मि॑न् ॅलो॒के । \newline
26. अ॒मुष्मि॑न् ॅलो॒के लो॒के॑ ऽमुष्मि॑न् न॒मुष्मि॑न् ॅलो॒क आ॑दि॒त्य मा॑दि॒त्यम् ॅलो॒के॑ ऽमुष्मि॑न् न॒मुष्मि॑न् ॅलो॒क आ॑दि॒त्यम् । \newline
27. लो॒क आ॑दि॒त्य मा॑दि॒त्यम् ॅलो॒के लो॒क आ॑दि॒त्यम् तत॒ स्तत॑ आदि॒त्यम् ॅलो॒के लो॒क आ॑दि॒त्यम् ततः॑ । \newline
28. आ॒दि॒त्यम् तत॒ स्तत॑ आदि॒त्य मा॑दि॒त्यम् ततो॒ वै वै तत॑ आदि॒त्य मा॑दि॒त्यम् ततो॒ वै । \newline
29. ततो॒ वै वै तत॒ स्ततो॒ वा इ॒मा वि॒मौ वै तत॒ स्ततो॒ वा इ॒मौ । \newline
30. वा इ॒मा वि॒मौ वै वा इ॒मौ लो॒कौ लो॒का वि॒मौ वै वा इ॒मौ लो॒कौ । \newline
31. इ॒मौ लो॒कौ लो॒का वि॒मा वि॒मौ लो॒का व॑शाम्यता मशाम्यताम् ॅलो॒का वि॒मा वि॒मौ लो॒का व॑शाम्यताम् । \newline
32. लो॒का व॑शाम्यता मशाम्यताम् ॅलो॒कौ लो॒का व॑शाम्यतां॒ ॅयद् यद॑शाम्यताम् ॅलो॒कौ लो॒का व॑शाम्यतां॒ ॅयत् । \newline
33. अ॒शा॒म्य॒तां॒ ॅयद् यद॑शाम्यता मशाम्यतां॒ ॅयदे॒व मे॒वं ॅयद॑शाम्यता मशाम्यतां॒ ॅयदे॒वम् । \newline
34. यदे॒व मे॒वं ॅयद् यदे॒व म॒न्वाहा॒ न्वाहै॒वं ॅयद् यदे॒व म॒न्वाह॑ । \newline
35. ए॒व म॒न्वाहा॒ न्वाहै॒व मे॒व म॒न्वाहा॒ नयो॑ र॒नयो॑ र॒न्वाहै॒व मे॒व म॒न्वाहा॒नयोः᳚ । \newline
36. अ॒न्वाहा॒ नयो॑ र॒नयो॑ र॒न्वाहा॒ न्वाहा॒ नयो᳚र् लो॒कयो᳚र् लो॒कयो॑ र॒नयो॑ र॒न्वाहा॒ न्वाहा॒ नयो᳚र् लो॒कयोः᳚ । \newline
37. अ॒न्वाहेत्य॑नु - आह॑ । \newline
38. अ॒नयो᳚र् लो॒कयो᳚र् लो॒कयो॑ र॒नयो॑ र॒नयो᳚र् लो॒कयोः॒ शान्त्यै॒ शान्त्यै॑ लो॒कयो॑ र॒नयो॑ र॒नयो᳚र् लो॒कयोः॒ शान्त्यै᳚ । \newline
39. लो॒कयोः॒ शान्त्यै॒ शान्त्यै॑ लो॒कयो᳚र् लो॒कयोः॒ शान्त्यै॒ शाम्य॑तः॒ शाम्य॑तः॒ शान्त्यै॑ लो॒कयो᳚र् लो॒कयोः॒ शान्त्यै॒ शाम्य॑तः । \newline
40. शान्त्यै॒ शाम्य॑तः॒ शाम्य॑तः॒ शान्त्यै॒ शान्त्यै॒ शाम्य॑तो ऽस्मा अस्मै॒ शाम्य॑तः॒ शान्त्यै॒ शान्त्यै॒ शाम्य॑तो ऽस्मै । \newline
41. शाम्य॑तो ऽस्मा अस्मै॒ शाम्य॑तः॒ शाम्य॑तो ऽस्मा इ॒मा वि॒मा व॑स्मै॒ शाम्य॑तः॒ शाम्य॑तो ऽस्मा इ॒मौ । \newline
42. अ॒स्मा॒ इ॒मा वि॒मा व॑स्मा अस्मा इ॒मौ लो॒कौ लो॒का वि॒मा व॑स्मा अस्मा इ॒मौ लो॒कौ । \newline
43. इ॒मौ लो॒कौ लो॒का वि॒मा वि॒मौ लो॒कौ यो यो लो॒का वि॒मा वि॒मौ लो॒कौ यः । \newline
44. लो॒कौ यो यो लो॒कौ लो॒कौ य ए॒व मे॒वं ॅयो लो॒कौ लो॒कौ य ए॒वम् । \newline
45. य ए॒व मे॒वं ॅयो य ए॒वं ॅवेद॒ वेदै॒वं ॅयो य ए॒वं ॅवेद॑ । \newline
46. ए॒वं ॅवेद॒ वेदै॒व मे॒वं ॅवेद॒ पञ्च॑दश॒ पञ्च॑दश॒ वेदै॒व मे॒वं ॅवेद॒ पञ्च॑दश । \newline
47. वेद॒ पञ्च॑दश॒ पञ्च॑दश॒ वेद॒ वेद॒ पञ्च॑दश सामिधे॒नीः सा॑मिधे॒नीः पञ्च॑दश॒ वेद॒ वेद॒ पञ्च॑दश सामिधे॒नीः । \newline
48. पञ्च॑दश सामिधे॒नीः सा॑मिधे॒नीः पञ्च॑दश॒ पञ्च॑दश सामिधे॒नी रन्वनु॑ सामिधे॒नीः पञ्च॑दश॒ पञ्च॑दश सामिधे॒नी रनु॑ । \newline
49. पञ्च॑द॒शेति॒ पञ्च॑ - द॒श॒ । \newline
50. सा॒मि॒धे॒नी रन्वनु॑ सामिधे॒नीः सा॑मिधे॒नी रन्वा॑हा॒हानु॑ सामिधे॒नीः सा॑मिधे॒नी रन्वा॑ह । \newline
51. सा॒मि॒धे॒नीरिति॑ सां - इ॒धे॒नीः । \newline
52. अन्वा॑हा॒हा न्वन्वा॑ह॒ पञ्च॑दश॒ पञ्च॑दशा॒हा न्वन्वा॑ह॒ पञ्च॑दश । \newline
53. आ॒ह॒ पञ्च॑दश॒ पञ्च॑दशाहाह॒ पञ्च॑दश॒ वै वै पञ्च॑दशाहाह॒ पञ्च॑दश॒ वै । \newline
54. पञ्च॑दश॒ वै वै पञ्च॑दश॒ पञ्च॑दश॒ वा अ॑र्द्धमा॒सस्या᳚ र्द्धमा॒सस्य॒ वै पञ्च॑दश॒ पञ्च॑दश॒ वा अ॑र्द्धमा॒सस्य॑ । \newline
55. पञ्च॑द॒शेति॒ पञ्च॑ - द॒श॒ । \newline
\pagebreak
\markright{ TS 2.5.8.3  \hfill https://www.vedavms.in \hfill}
\addcontentsline{toc}{section}{ TS 2.5.8.3 }
\section*{ TS 2.5.8.3 }

\textbf{TS 2.5.8.3 } \newline
\textbf{Samhita Paata} \newline

वा अ॑र्द्धमा॒सस्य॒ रात्र॑योऽर्द्धमास॒शः सं॑ॅवथ्स॒र आ᳚प्यते॒ तासां॒ त्रीणि॑ च श॒तानि॑ ष॒ष्टिश्चा॒क्षरा॑णि॒ ताव॑तीः संॅवथ्स॒रस्य॒ रात्र॑योऽक्षर॒श ए॒व सं॑ॅवथ्स॒रमा᳚प्नोति नृ॒मेध॑श्च॒ परु॑च्छेपश्च ब्रह्म॒वाद्य॑मवदेताम॒स्मिन् दारा॑वा॒र्द्रे᳚ऽग्निं ज॑नयाव यत॒रो नौ॒ ब्रह्मी॑या॒निति॑ नृ॒मेधो॒ऽभ्य॑वद॒थ् स धू॒मम॑जनय॒त् परु॑च्छेपो॒ ऽभ्य॑वद॒थ् सो᳚ऽग्निम॑जनय॒दृष॒ इत्य॑ब्रवी॒द्-  [  ] \newline

\textbf{Pada Paata} \newline

वै । अ॒र्द्ध॒मा॒सस्येत्य॑र्द्ध - मा॒सस्य॑ । रात्र॑यः । अ॒र्द्ध॒मा॒स॒श इत्य॑र्द्धमास - शः । सं॒ॅव॒थ्स॒र इति॑ सं - व॒थ्स॒रः । आ॒प्य॒ते॒ । तासा᳚म् । त्रीणि॑ । च॒ । श॒तानि॑ । ष॒ष्टिः । च॒ । अ॒क्षरा॑णि । ताव॑तीः । सं॒ॅव॒थ्स॒रस्येति॑ सं - व॒थ्स॒रस्य॑ । रात्र॑यः । अ॒क्ष॒र॒श इत्य॑क्षर-शः । ए॒व । सं॒ॅव॒थ्स॒रमिति॑ सं - व॒थ्स॒रम् । आ॒प्नो॒ति॒ । नृ॒मेध॒ इति॑ नृ - मेधः॑ । च॒ । परु॑च्छेपः । च॒ । ब्र॒ह्म॒वाद्य॒मिति॑ ब्रह्म - वाद्य᳚म् । अ॒व॒दे॒ता॒म् । अ॒स्मिन्न् । दारौ᳚ । आ॒र्द्रे । अ॒ग्निम् । ज॒न॒या॒व॒ । य॒त॒रः । नौ॒ । ब्रह्मी॑यान् । इति॑ । नृ॒मेध॒ इति॑ नृ - मेधः॑ । अ॒भीति॑ । अ॒व॒द॒त् । सः । धू॒मम् । अ॒ज॒न॒य॒त् । परु॑च्छेपः । अ॒भीति॑ । अ॒व॒द॒त् । सः । अ॒ग्निम् । अ॒ज॒न॒य॒त् । ऋषे᳚ । इति॑ । अ॒ब्र॒वी॒त् ।  \newline


\textbf{Krama Paata} \newline

वा अ॑र्द्धमा॒सस्य॑ । अ॒र्द्ध॒मा॒सस्य॒ रात्र॑यः । अ॒र्द्ध॒मा॒सस्येत्य॑र्द्ध - मा॒सस्य॑ । रात्र॑यो ऽर्द्धमास॒शः । अ॒र्द्ध॒मा॒स॒शः स॑म्ॅवथ्स॒रः । अ॒र्द्ध॒मा॒स॒श इत्य॑र्द्धमास - शः । स॒म्ॅव॒थ्स॒र आ᳚प्यते । स॒म्ॅव॒थ्स॒र इति॑ सम् - व॒थ्स॒रः । आ॒प्य॒ते॒ तासा᳚म् । तासा॒म् त्रीणि॑ । त्रीणि॑ च । च॒ श॒तानि॑ । श॒तानि॑ ष॒ष्टिः । ष॒ष्टिश्च॑ । चा॒क्षरा॑णि । अ॒क्षरा॑णि॒ ताव॑तीः । ताव॑तीः सम्ॅवथ्स॒रस्य॑ । स॒म्ॅव॒थ्स॒रस्य॒ रात्र॑यः । स॒म्ॅव॒थ्स॒र॒स्येति॑ सम् - व॒थ्स॒रस्य॑ । रात्र॑यो ऽक्षर॒शः । अ॒क्ष॒र॒श ए॒व । अ॒क्ष॒र॒श इत्य॑क्षर - शः । ए॒व स॑म्ॅवथ्स॒रम् । स॒म्ॅव॒थ्स॒रमा᳚प्नोति । स॒म्ॅव॒थ्स॒रमिति॑ सम् - व॒थ्स॒रम् । आ॒प्नो॒ति॒ नृ॒मेधः॑ । नृ॒मेध॑श्च । नृ॒मेध॒ इति॑ नृ - मेधः॑ । च॒ परु॑च्छेपः । परु॑च्छेपश्च । च॒ ब्र॒ह्म॒वाद्य᳚म् । ब्र॒ह्म॒वाद्य॑मवदेताम् । ब्र॒ह्म॒वाद्य॒मिति॑ ब्रह्म - वाद्य᳚म् । अ॒व॒दे॒ता॒म॒स्मिन्न् । अ॒स्मिन् दारौ᳚ । दारा॑वा॒र्द्रे । आ॒र्द्रे᳚ ऽग्निम् । अ॒ग्निम् ज॑नयाव । ज॒न॒या॒व॒ य॒त॒रः । य॒त॒रो नौ᳚ । नौ॒ ब्रह्मी॑यान् । ब्रह्मी॑या॒निति॑ । इति॑ नृ॒मेधः॑ । नृ॒मेधो॒ऽभि । नृ॒मेध॒ इति॑ नृ - मेधः॑ । अ॒भ्य॑वदत् । अ॒व॒द॒थ् सः । स धू॒मम् । धू॒मम॑जनयत् । अ॒ज॒न॒य॒त् परु॑च्छेपः । परु॑च्छेपो॒ऽभि । अ॒भ्य॑वदत् । अ॒व॒द॒थ् सः । सो᳚ऽग्निम् । अ॒ग्निम॑जनयत् । अ॒ज॒न॒य॒दृषे᳚ । ऋष॒ इति॑ । इत्य॑ब्रवीत् । अ॒ब्र॒वी॒द् यत् \newline

\textbf{Jatai Paata} \newline

1. वा अ॑र्द्धमा॒सस्या᳚ र्द्धमा॒सस्य॒ वै वा अ॑र्द्धमा॒सस्य॑ । \newline
2. अ॒र्द्ध॒मा॒सस्य॒ रात्र॑यो॒ रात्र॑यो ऽर्द्धमा॒सस्या᳚ र्द्धमा॒सस्य॒ रात्र॑यः । \newline
3. अ॒र्द्ध॒मा॒सस्येत्य॑र्द्ध - मा॒सस्य॑ । \newline
4. रात्र॑यो ऽर्द्धमास॒शो᳚ ऽर्द्धमास॒शो रात्र॑यो॒ रात्र॑यो ऽर्द्धमास॒शः । \newline
5. अ॒र्द्ध॒मा॒स॒शः सं॑ॅवथ्स॒रः सं॑ॅवथ्स॒रो᳚ ऽर्द्धमास॒शो᳚ ऽर्द्धमास॒शः सं॑ॅवथ्स॒रः । \newline
6. अ॒र्द्ध॒मा॒स॒श इत्य॑र्द्धमास - शः । \newline
7. सं॒ॅव॒थ्स॒र आ᳚प्यत आप्यते संॅवथ्स॒रः सं॑ॅवथ्स॒र आ᳚प्यते । \newline
8. सं॒ॅव॒थ्स॒र इति॑ सं - व॒थ्स॒रः । \newline
9. आ॒प्य॒ते॒ तासा॒म् तासा॑ माप्यत आप्यते॒ तासा᳚म् । \newline
10. तासा॒म् त्रीणि॒ त्रीणि॒ तासा॒म् तासा॒म् त्रीणि॑ । \newline
11. त्रीणि॑ च च॒ त्रीणि॒ त्रीणि॑ च । \newline
12. च॒ श॒तानि॑ श॒तानि॑ च च श॒तानि॑ । \newline
13. श॒तानि॑ ष॒ष्टि ष्ष॒ष्टिः श॒तानि॑ श॒तानि॑ ष॒ष्टिः । \newline
14. ष॒ष्टिश्च॑ च ष॒ष्टि ष्ष॒ष्टिश्च॑ । \newline
15. चा॒क्षरा᳚ ण्य॒क्षरा॑णि च चा॒क्षरा॑णि । \newline
16. अ॒क्षरा॑णि॒ ताव॑ती॒ स्ताव॑ती र॒क्षरा᳚ ण्य॒क्षरा॑णि॒ ताव॑तीः । \newline
17. ताव॑तीः संॅवथ्स॒रस्य॑ संॅवथ्स॒रस्य॒ ताव॑ती॒ स्ताव॑तीः संॅवथ्स॒रस्य॑ । \newline
18. सं॒ॅव॒थ्स॒रस्य॒ रात्र॑यो॒ रात्र॑यः संॅवथ्स॒रस्य॑ संॅवथ्स॒रस्य॒ रात्र॑यः । \newline
19. सं॒ॅव॒थ्स॒रस्येति॑ सं - व॒थ्स॒रस्य॑ । \newline
20. रात्र॑यो ऽक्षर॒शो᳚ ऽक्षर॒शो रात्र॑यो॒ रात्र॑यो ऽक्षर॒शः । \newline
21. अ॒क्ष॒र॒श ए॒वैवा क्ष॑र॒शो᳚ ऽक्षर॒श ए॒व । \newline
22. अ॒क्ष॒र॒श इत्य॑क्षर - शः । \newline
23. ए॒व सं॑ॅवथ्स॒रꣳ सं॑ॅवथ्स॒र मे॒वैव सं॑ॅवथ्स॒रम् । \newline
24. सं॒ॅव॒थ्स॒र मा᳚प्नो त्याप्नोति संॅवथ्स॒रꣳ सं॑ॅवथ्स॒र मा᳚प्नोति । \newline
25. सं॒ॅव॒थ्स॒रमिति॑ सं - व॒थ्स॒रम् । \newline
26. आ॒प्नो॒ति॒ नृ॒मेधो॑ नृ॒मेध॑ आप्नो त्याप्नोति नृ॒मेधः॑ । \newline
27. नृ॒मेध॑श्च च नृ॒मेधो॑ नृ॒मेध॑श्च । \newline
28. नृ॒मेध॒ इति॑ नृ - मेधः॑ । \newline
29. च॒ परु॑च्छेपः॒ परु॑च्छेपश्च च॒ परु॑च्छेपः । \newline
30. परु॑च्छेपश्च च॒ परु॑च्छेपः॒ परु॑च्छेपश्च । \newline
31. च॒ ब्र॒ह्म॒वाद्य॑म् ब्रह्म॒वाद्य॑म् च च ब्रह्म॒वाद्य᳚म् । \newline
32. ब्र॒ह्म॒वाद्य॑ मवदेता मवदेताम् ब्रह्म॒वाद्य॑म् ब्रह्म॒वाद्य॑ मवदेताम् । \newline
33. ब्र॒ह्म॒वाद्य॒मिति॑ ब्रह्म - वाद्य᳚म् । \newline
34. अ॒व॒दे॒ता॒ म॒स्मिन् न॒स्मिन् न॑वदेता मवदेता म॒स्मिन्न् । \newline
35. अ॒स्मिन् दारौ॒ दारा॑ व॒स्मिन् न॒स्मिन् दारौ᳚ । \newline
36. दारा॑ वा॒र्द्र आ॒र्द्रे दारौ॒ दारा॑ वा॒र्द्रे । \newline
37. आ॒र्द्रे᳚ ऽग्नि म॒ग्नि मा॒र्द्र आ॒र्द्रे᳚ ऽग्निम् । \newline
38. अ॒ग्निम् ज॑नयाव जनयावा॒ग्नि म॒ग्निम् ज॑नयाव । \newline
39. ज॒न॒या॒व॒ य॒त॒रो य॑त॒रो ज॑नयाव जनयाव यत॒रः । \newline
40. य॒त॒रो नौ॑ नौ यत॒रो य॑त॒रो नौ᳚ । \newline
41. नौ॒ ब्रह्मी॑या॒न् ब्रह्मी॑यान् नौ नौ॒ ब्रह्मी॑यान् । \newline
42. ब्रह्मी॑या॒ नितीति॒ ब्रह्मी॑या॒न् ब्रह्मी॑या॒ निति॑ । \newline
43. इति॑ नृ॒मेधो॑ नृ॒मेध॒ इतीति॑ नृ॒मेधः॑ । \newline
44. नृ॒मेधो॒ ऽभ्य॑भि नृ॒मेधो॑ नृ॒मेधो॒ ऽभि । \newline
45. नृ॒मेध॒ इति॑ नृ - मेधः॑ । \newline
46. अ॒भ्य॑व ददव दद॒भ्या᳚(1॒)भ्य॑वदत् । \newline
47. अ॒व॒द॒थ् स सो॑ ऽवद दवद॒थ् सः । \newline
48. स धू॒मम् धू॒मꣳ स स धू॒मम् । \newline
49. धू॒म म॑जनय दजनयद् धू॒मम् धू॒म म॑जनयत् । \newline
50. अ॒ज॒न॒य॒त् परु॑च्छेपः॒ परु॑च्छेपो ऽजनय दजनय॒त् परु॑च्छेपः । \newline
51. परु॑च्छेपो॒ ऽभ्य॑भि परु॑च्छेपः॒ परु॑च्छेपो॒ ऽभि । \newline
52. अ॒भ्य॑व ददव दद॒भ्या᳚(1॒)भ्य॑वदत् । \newline
53. अ॒व॒द॒थ् स सो॑ ऽवद दवद॒थ् सः । \newline
54. सो᳚ ऽग्नि म॒ग्निꣳ स सो᳚ ऽग्निम् । \newline
55. अ॒ग्नि म॑जनय दजनय द॒ग्नि म॒ग्नि म॑जनयत् । \newline
56. अ॒ज॒न॒य॒ दृष॒ ऋषे॑ ऽजनय दजनय॒ दृषे᳚ । \newline
57. ऋष॒ इतीत्यृष॒ ऋष॒ इति॑ । \newline
58. इत्य॑ब्रवी दब्रवी॒ दिती त्य॑ब्रवीत् । \newline
59. अ॒ब्र॒वी॒द् यद् यद॑ब्रवी दब्रवी॒द् यत् । \newline

\textbf{Ghana Paata } \newline

1. वा अ॑र्द्धमा॒सस्या᳚ र्द्धमा॒सस्य॒ वै वा अ॑र्द्धमा॒सस्य॒ रात्र॑यो॒ रात्र॑यो ऽर्द्धमा॒सस्य॒ वै वा अ॑र्द्धमा॒सस्य॒ रात्र॑यः । \newline
2. अ॒र्द्ध॒मा॒सस्य॒ रात्र॑यो॒ रात्र॑यो ऽर्द्धमा॒सस्या᳚ र्द्धमा॒सस्य॒ रात्र॑यो ऽर्द्धमास॒शो᳚ ऽर्द्धमास॒शो रात्र॑यो ऽर्द्धमा॒सस्या᳚ र्द्धमा॒सस्य॒ रात्र॑यो ऽर्द्धमास॒शः । \newline
3. अ॒र्द्ध॒मा॒सस्येत्य॑र्द्ध - मा॒सस्य॑ । \newline
4. रात्र॑यो ऽर्द्धमास॒शो᳚ ऽर्द्धमास॒शो रात्र॑यो॒ रात्र॑यो ऽर्द्धमास॒शः सं॑ॅवथ्स॒रः सं॑ॅवथ्स॒रो᳚ ऽर्द्धमास॒शो रात्र॑यो॒ रात्र॑यो ऽर्द्धमास॒शः सं॑ॅवथ्स॒रः । \newline
5. अ॒र्द्ध॒मा॒स॒शः सं॑ॅवथ्स॒रः सं॑ॅवथ्स॒रो᳚ ऽर्द्धमास॒शो᳚ ऽर्द्धमास॒शः सं॑ॅवथ्स॒र आ᳚प्यत आप्यते संॅवथ्स॒रो᳚ ऽर्द्धमास॒शो᳚ ऽर्द्धमास॒शः सं॑ॅवथ्स॒र आ᳚प्यते । \newline
6. अ॒र्द्ध॒मा॒स॒श इत्य॑र्द्धमास - शः । \newline
7. सं॒ॅव॒थ्स॒र आ᳚प्यत आप्यते संॅवथ्स॒रः सं॑ॅवथ्स॒र आ᳚प्यते॒ तासा॒म् तासा॑ माप्यते संॅवथ्स॒रः सं॑ॅवथ्स॒र आ᳚प्यते॒ तासा᳚म् । \newline
8. सं॒ॅव॒थ्स॒र इति॑ सं - व॒थ्स॒रः । \newline
9. आ॒प्य॒ते॒ तासा॒म् तासा॑ माप्यत आप्यते॒ तासा॒म् त्रीणि॒ त्रीणि॒ तासा॑ माप्यत आप्यते॒ तासा॒म् त्रीणि॑ । \newline
10. तासा॒म् त्रीणि॒ त्रीणि॒ तासा॒म् तासा॒म् त्रीणि॑ च च॒ त्रीणि॒ तासा॒म् तासा॒म् त्रीणि॑ च । \newline
11. त्रीणि॑ च च॒ त्रीणि॒ त्रीणि॑ च श॒तानि॑ श॒तानि॑ च॒ त्रीणि॒ त्रीणि॑ च श॒तानि॑ । \newline
12. च॒ श॒तानि॑ श॒तानि॑ च च श॒तानि॑ ष॒ष्टि ष्ष॒ष्टिः श॒तानि॑ च च श॒तानि॑ ष॒ष्टिः । \newline
13. श॒तानि॑ ष॒ष्टि ष्ष॒ष्टिः श॒तानि॑ श॒तानि॑ ष॒ष्टिश्च॑ च ष॒ष्टिः श॒तानि॑ श॒तानि॑ ष॒ष्टिश्च॑ । \newline
14. ष॒ष्टिश्च॑ च ष॒ष्टि ष्ष॒ष्टि श्चा॒क्षरा᳚ ण्य॒क्षरा॑णि च ष॒ष्टि ष्ष॒ष्टि श्चा॒क्षरा॑णि । \newline
15. चा॒क्षरा᳚ ण्य॒क्षरा॑णि च चा॒क्षरा॑णि॒ ताव॑ती॒ स्ताव॑ती र॒क्षरा॑णि च चा॒क्षरा॑णि॒ ताव॑तीः । \newline
16. अ॒क्षरा॑णि॒ ताव॑ती॒ स्ताव॑ती र॒क्षरा᳚ ण्य॒क्षरा॑णि॒ ताव॑तीः संॅवथ्स॒रस्य॑ संॅवथ्स॒रस्य॒ ताव॑तीर॒क्षरा᳚ण्य॒क्षरा॑णि॒ ताव॑तीः संॅवथ्स॒रस्य॑ । \newline
17. ताव॑तीः संॅवथ्स॒रस्य॑ संॅवथ्स॒रस्य॒ ताव॑ती॒ स्ताव॑तीः संॅवथ्स॒रस्य॒ रात्र॑यो॒ रात्र॑यः संॅवथ्स॒रस्य॒ ताव॑ती॒ स्ताव॑तीः संॅवथ्स॒रस्य॒ रात्र॑यः । \newline
18. सं॒ॅव॒थ्स॒रस्य॒ रात्र॑यो॒ रात्र॑यः संॅवथ्स॒रस्य॑ संॅवथ्स॒रस्य॒ रात्र॑यो ऽक्षर॒शो᳚ ऽक्षर॒शो रात्र॑यः संॅवथ्स॒रस्य॑ संॅवथ्स॒रस्य॒ रात्र॑यो ऽक्षर॒शः । \newline
19. सं॒ॅव॒थ्स॒रस्येति॑ सं - व॒थ्स॒रस्य॑ । \newline
20. रात्र॑यो ऽक्षर॒शो᳚ ऽक्षर॒शो रात्र॑यो॒ रात्र॑यो ऽक्षर॒श ए॒वैवा क्ष॑र॒शो रात्र॑यो॒ रात्र॑यो ऽक्षर॒श ए॒व । \newline
21. अ॒क्ष॒र॒श ए॒वैवा क्ष॑र॒शो᳚ ऽक्षर॒श ए॒व सं॑ॅवथ्स॒रꣳ सं॑ॅवथ्स॒र मे॒वा क्ष॑र॒शो᳚ ऽक्षर॒श ए॒व सं॑ॅवथ्स॒रम् । \newline
22. अ॒क्ष॒र॒श इत्य॑क्षर - शः । \newline
23. ए॒व सं॑ॅवथ्स॒रꣳ सं॑ॅवथ्स॒र मे॒वैव सं॑ॅवथ्स॒र मा᳚प्नो त्याप्नोति संॅवथ्स॒र मे॒वैव सं॑ॅवथ्स॒र मा᳚प्नोति । \newline
24. सं॒ॅव॒थ्स॒र मा᳚प्नो त्याप्नोति संॅवथ्स॒रꣳ सं॑ॅवथ्स॒र मा᳚प्नोति नृ॒मेधो॑ नृ॒मेध॑ आप्नोति संॅवथ्स॒रꣳ सं॑ॅवथ्स॒र मा᳚प्नोति नृ॒मेधः॑ । \newline
25. सं॒ॅव॒थ्स॒रमिति॑ सं - व॒थ्स॒रम् । \newline
26. आ॒प्नो॒ति॒ नृ॒मेधो॑ नृ॒मेध॑ आप्नो त्याप्नोति नृ॒मेध॑श्च च नृ॒मेध॑ आप्नो त्याप्नोति नृ॒मेध॑श्च । \newline
27. नृ॒मेध॑श्च च नृ॒मेधो॑ नृ॒मेध॑श्च॒ परु॑च्छेपः॒ परु॑च्छेपश्च नृ॒मेधो॑ नृ॒मेध॑श्च॒ परु॑च्छेपः । \newline
28. नृ॒मेध॒ इति॑ नृ - मेधः॑ । \newline
29. च॒ परु॑च्छेपः॒ परु॑च्छेपश्च च॒ परु॑च्छेपश्च च॒ परु॑च्छेपश्च च॒ परु॑च्छेपश्च । \newline
30. परु॑च्छेपश्च च॒ परु॑च्छेपः॒ परु॑च्छेपश्च ब्रह्म॒वाद्य॑म् ब्रह्म॒वाद्य॑म् च॒ परु॑च्छेपः॒ परु॑च्छेपश्च ब्रह्म॒वाद्य᳚म् । \newline
31. च॒ ब्र॒ह्म॒वाद्य॑म् ब्रह्म॒वाद्य॑म् च च ब्रह्म॒वाद्य॑ मवदेता मवदेताम् ब्रह्म॒वाद्य॑म् च च ब्रह्म॒वाद्य॑ मवदेताम् । \newline
32. ब्र॒ह्म॒वाद्य॑ मवदेता मवदेताम् ब्रह्म॒वाद्य॑म् ब्रह्म॒वाद्य॑ मवदेता म॒स्मिन् न॒स्मिन् न॑वदेताम् ब्रह्म॒वाद्य॑म् ब्रह्म॒वाद्य॑ मवदेता म॒स्मिन्न् । \newline
33. ब्र॒ह्म॒वाद्य॒मिति॑ ब्रह्म - वाद्य᳚म् । \newline
34. अ॒व॒दे॒ता॒ म॒स्मिन् न॒स्मिन् न॑वदेता मवदेता म॒स्मिन् दारौ॒ दारा॑ व॒स्मिन् न॑वदेता मवदेता म॒स्मिन् दारौ᳚ । \newline
35. अ॒स्मिन् दारौ॒ दारा॑ व॒स्मिन् न॒स्मिन् दारा॑ वा॒र्द्र आ॒र्द्रे दारा॑ व॒स्मिन् न॒स्मिन् दारा॑ वा॒र्द्रे । \newline
36. दारा॑ वा॒र्द्र आ॒र्द्रे दारौ॒ दारा॑ वा॒र्द्रे᳚ ऽग्नि म॒ग्नि मा॒र्द्रे दारौ॒ दारा॑ वा॒र्द्रे᳚ ऽग्निम् । \newline
37. आ॒र्द्रे᳚ ऽग्नि म॒ग्नि मा॒र्द्र आ॒र्द्रे᳚ ऽग्निम् ज॑नयाव जनयावा॒ग्नि मा॒र्द्र आ॒र्द्रे᳚ ऽग्निम् ज॑नयाव । \newline
38. अ॒ग्निम् ज॑नयाव जनयावा॒ग्नि म॒ग्निम् ज॑नयाव यत॒रो य॑त॒रो ज॑नयावा॒ग्नि म॒ग्निम् ज॑नयाव यत॒रः । \newline
39. ज॒न॒या॒व॒ य॒त॒रो य॑त॒रो ज॑नयाव जनयाव यत॒रो नौ॑ नौ यत॒रो ज॑नयाव जनयाव यत॒रो नौ᳚ । \newline
40. य॒त॒रो नौ॑ नौ यत॒रो य॑त॒रो नौ॒ ब्रह्मी॑या॒न् ब्रह्मी॑यान् नौ यत॒रो य॑त॒रो नौ॒ ब्रह्मी॑यान् । \newline
41. नौ॒ ब्रह्मी॑या॒न् ब्रह्मी॑यान् नौ नौ॒ ब्रह्मी॑या॒ नितीति॒ ब्रह्मी॑यान् नौ नौ॒ ब्रह्मी॑या॒ निति॑ । \newline
42. ब्रह्मी॑या॒ नितीति॒ ब्रह्मी॑या॒न् ब्रह्मी॑या॒ निति॑ नृ॒मेधो॑ नृ॒मेध॒ इति॒ ब्रह्मी॑या॒न् ब्रह्मी॑या॒ निति॑ नृ॒मेधः॑ । \newline
43. इति॑ नृ॒मेधो॑ नृ॒मेध॒ इतीति॑ नृ॒मेधो॒ ऽभ्य॑भि नृ॒मेध॒ इतीति॑ नृ॒मेधो॒ ऽभि । \newline
44. नृ॒मेधो॒ ऽभ्य॑भि नृ॒मेधो॑ नृ॒मेधो॒ ऽभ्य॑व ददव दद॒भि नृ॒मेधो॑ नृ॒मेधो॒ ऽभ्य॑वदत् । \newline
45. नृ॒मेध॒ इति॑ नृ - मेधः॑ । \newline
46. अ॒भ्य॑व ददव दद॒भ्या᳚(1॒)भ्य॑वद॒थ् स सो॑ ऽवदद॒भ्या᳚(1॒)भ्य॑वद॒थ् सः । \newline
47. अ॒व॒द॒थ् स सो॑ ऽवद दवद॒थ् स धू॒मम् धू॒मꣳ सो॑ ऽवद दवद॒थ् स धू॒मम् । \newline
48. स धू॒मम् धू॒मꣳ स स धू॒म म॑जनय दजनयद् धू॒मꣳ स स धू॒म म॑जनयत् । \newline
49. धू॒म म॑जनय दजनयद् धू॒मम् धू॒म म॑जनय॒त् परु॑च्छेपः॒ परु॑च्छेपो ऽजनयद् धू॒मम् धू॒म म॑जनय॒त् परु॑च्छेपः । \newline
50. अ॒ज॒न॒य॒त् परु॑च्छेपः॒ परु॑च्छेपो ऽजनय दजनय॒त् परु॑च्छेपो॒ ऽभ्य॑भि परु॑च्छेपो ऽजनय दजनय॒त् परु॑च्छेपो॒ ऽभि । \newline
51. परु॑च्छेपो॒ ऽभ्य॑भि परु॑च्छेपः॒ परु॑च्छेपो॒ ऽभ्य॑व ददव दद॒भि परु॑च्छेपः॒ परु॑च्छेपो॒ ऽभ्य॑वदत् । \newline
52. अ॒भ्य॑व ददव दद॒भ्या᳚(1॒)भ्य॑वद॒थ् स सो॑ ऽवदद॒भ्या᳚(1॒)भ्य॑वद॒थ् सः । \newline
53. अ॒व॒द॒थ् स सो॑ ऽवद दवद॒थ् सो᳚ ऽग्नि म॒ग्निꣳ सो॑ ऽवद दवद॒थ् सो᳚ ऽग्निम् । \newline
54. सो᳚ ऽग्नि म॒ग्निꣳ स सो᳚ ऽग्नि म॑जनय दजनय द॒ग्निꣳ स सो᳚ ऽग्नि म॑जनयत् । \newline
55. अ॒ग्नि म॑जनय दजनय द॒ग्नि म॒ग्नि म॑जनय॒दृष॒ ऋषे॑ ऽजनय द॒ग्नि म॒ग्नि म॑जनय॒दृषे᳚ । \newline
56. अ॒ज॒न॒य॒दृष॒ ऋषे॑ ऽजनय दजनय॒दृष॒ इतीत्यृषे॑ ऽजनय दजनय॒दृष॒ इति॑ । \newline
57. ऋष॒ इतीत्यृष॒ ऋष॒ इत्य॑ब्रवी दब्रवी॒ दित्यृष॒ ऋष॒ इत्य॑ब्रवीत् । \newline
58. इत्य॑ब्रवी दब्रवी॒दिती त्य॑ब्रवी॒द् यद् यद॑ब्रवी॒ दिती त्य॑ब्रवी॒द् यत् । \newline
59. अ॒ब्र॒वी॒द् यद् यद॑ब्रवी दब्रवी॒द् यथ् स॒माव॑थ् स॒माव॒द् यद॑ब्रवी दब्रवी॒द् यथ् स॒माव॑त् । \newline
\pagebreak
\markright{ TS 2.5.8.4  \hfill https://www.vedavms.in \hfill}
\addcontentsline{toc}{section}{ TS 2.5.8.4 }
\section*{ TS 2.5.8.4 }

\textbf{TS 2.5.8.4 } \newline
\textbf{Samhita Paata} \newline

यथ् स॒माव॑द्वि॒द्व क॒था त्वम॒ग्निमजी॑जनो॒ नाहमिति॑ सामिधे॒नीना॑मे॒वाहं ॅवर्णं॑ ॅवे॒देत्य॑ब्रवी॒द्यद् घृ॒तव॑त् प॒दम॑नू॒च्यते॒ स आ॑सां॒ ॅवर्ण॒स्तं त्वा॑ स॒मिद्भि॑रङ्गिर॒ इत्या॑ह सामिधे॒नीष्वे॒व तज्ज्योति॑ र्जनयति॒ स्त्रिय॒स्तेन॒ यदृचः॒ स्त्रिय॒स्तेन॒ यद्-गा॑य॒त्रियः॒ स्त्रिय॒स्तेन॒ यथ् सा॑मिधे॒न्यो॑ वृष॑ण्वती॒-मन्वा॑ह॒ - [  ] \newline

\textbf{Pada Paata} \newline

यत् । स॒माव॑त् । वि॒द्व । क॒था । त्वम् । अ॒ग्निम् । अजी॑जनः । न । अ॒हम् । इति॑ । सा॒मि॒धे॒नीना॒मिति॑ सां - इ॒धे॒नीना᳚म् । ए॒व । अ॒हम् । वर्ण᳚म् । वे॒द॒ । इति॑ । अ॒ब्र॒वी॒त् । यत् । घृ॒तव॒दिति॑ घृ॒त-व॒त् । प॒दम् । अ॒नू॒च्यत॒ इत्य॑नु - उ॒च्यते᳚ । सः । आ॒सा॒म् । वर्णः॑ । तम् । त्वा॒ । स॒मिद्भि॒रिति॑ स॒मित् - भिः॒ । अ॒ङ्गि॒रः॒ । इति॑ । आ॒ह॒ । सा॒मि॒धे॒नीष्विति॑ सां - इ॒धे॒नीषु॑ । ए॒व । तत् । ज्योतिः॑ । ज॒न॒य॒ति॒ । स्त्रियः॑ । तेन॑ । यत् । ऋचः॑ । स्त्रियः॑ । तेन॑ । यत् । गा॒य॒त्रियः॑ । स्त्रियः॑ । तेन॑ । यत् । सा॒मि॒धे॒न्य॑ इति॑ सां - इ॒धे॒न्यः॑ । वृष॑ण्वती॒मिति॒ वृषण्॑ - व॒ती॒म् । अन्विति॑ । आ॒ह॒ ।  \newline


\textbf{Krama Paata} \newline

यथ् स॒माव॑त् । स॒माव॑द् वि॒द्व । वि॒द्व क॒था । क॒था त्वम् । त्वम॒ग्निम् । अ॒ग्निमजी॑जनः । अजी॑जनो॒ न । नाहम् । अ॒हमिति॑ । इति॑ सामिधे॒नीना᳚म् । सा॒मि॒धे॒नीना॑मे॒व । सा॒मि॒धे॒नीना॒मिति॑ साम् - इ॒धे॒नीना᳚म् । ए॒वाहम् । अ॒हम् ॅवर्ण᳚म् । वर्ण॑म् ॅवेद । वे॒देति॑ । इत्य॑ब्रवीत् । अ॒ब्र॒वी॒द् यत् । यद् घृ॒तव॑त् । घृ॒तव॑त् प॒दम् । घृ॒तव॒दिति॑ घृ॒त - व॒त्॒ । प॒दम॑नू॒च्यते᳚ । अ॒नू॒च्यते॒ सः । अ॒नू॒च्यत॒ इत्य॑नु - उ॒च्यते᳚ । स आ॑साम् । आ॒सा॒म् ॅवर्णः॑ । वर्ण॒स्तम् । तम् त्वा᳚ । त्वा॒ स॒मिद्भिः॑ । स॒मिद्भि॑रङ्गिरः । स॒मिद्भि॒रिति॑ स॒मित् - भिः॒ । अ॒ङ्गि॒र॒ इति॑ । इत्या॑ह । आ॒ह॒ सा॒मि॒धे॒नीषु॑ । सा॒मि॒धे॒नीष्वे॒व । सा॒मि॒धे॒नीष्विति॑ साम् - इ॒धे॒नीषु॑ । ए॒व तत् । तज्ज्योतिः॑ । ज्योति॑र् जनयति । ज॒न॒य॒ति॒ स्त्रियः॑ । स्त्रिय॒स्तेन॑ । तेन॒ यत् । यदृचः॑ । ऋचः॒ स्त्रियः॑ । स्त्रिय॒ स्तेन॑ । तेन॒ यत् । यद् गा॑य॒त्रियः॑ । गा॒य॒त्रियः॒ स्त्रियः॑ । स्त्रिय॒ स्तेन॑ । तेन॒ यत् । यथ् सा॑मिधे॒न्यः॑ । सा॒मि॒धे॒न्यो॑ वृष॑ण्वतीम् । सा॒मि॒धे॒न्य॑ इति॑ साम् - इ॒धे॒न्यः॑ । वृष॑ण्वती॒मनु॑ । वृष॑ण्वती॒मिति॒ वृषण्ण्॑ - व॒ती॒म् । अन्वा॑ह । आ॒ह॒ तेन॑ \newline

\textbf{Jatai Paata} \newline

1. यथ् स॒माव॑थ् स॒माव॒द् यद् यथ् स॒माव॑त् । \newline
2. स॒माव॑द् वि॒द्व वि॒द्व स॒माव॑थ् स॒माव॑द् वि॒द्व । \newline
3. वि॒द्व क॒था क॒था वि॒द्व वि॒द्व क॒था । \newline
4. क॒था त्वम् त्वम् क॒था क॒था त्वम् । \newline
5. त्व म॒ग्नि म॒ग्निम् त्वम् त्व म॒ग्निम् । \newline
6. अ॒ग्नि मजी॑ज॒नो ऽजी॑जनो॒ ऽग्नि म॒ग्नि मजी॑जनः । \newline
7. अजी॑जनो॒ न नाजी॑ज॒नो ऽजी॑जनो॒ न । \newline
8. नाह म॒हम् न नाहम् । \newline
9. अ॒ह मितीत्य॒ह म॒ह मिति॑ । \newline
10. इति॑ सामिधे॒नीनाꣳ॑ सामिधे॒नीना॒ मितीति॑ सामिधे॒नीना᳚म् । \newline
11. सा॒मि॒धे॒नीना॑ मे॒वैव सा॑मिधे॒नीनाꣳ॑ सामिधे॒नीना॑ मे॒व । \newline
12. सा॒मि॒धे॒नीना॒मिति॑ सां - इ॒धे॒नीना᳚म् । \newline
13. ए॒वाह म॒ह मे॒वैवाहम् । \newline
14. अ॒हं ॅवर्णं॒ ॅवर्ण॑ म॒ह म॒हं ॅवर्ण᳚म् । \newline
15. वर्णं॑ ॅवेद वेद॒ वर्णं॒ ॅवर्णं॑ ॅवेद । \newline
16. वे॒दे तीति॑ वेद वे॒दे ति॑ । \newline
17. इत्य॑ब्रवी दब्रवी॒ दिती त्य॑ब्रवीत् । \newline
18. अ॒ब्र॒वी॒द् यद् यद॑ब्रवी दब्रवी॒द् यत् । \newline
19. यद् घृ॒तव॑द् घृ॒तव॒द् यद् यद् घृ॒तव॑त् । \newline
20. घृ॒तव॑त् प॒दम् प॒दम् घृ॒तव॑द् घृ॒तव॑त् प॒दम् । \newline
21. घृ॒तव॒दिति॑ घृ॒त - व॒त् । \newline
22. प॒द म॑नू॒च्यते॑ ऽनू॒च्यते॑ प॒दम् प॒द म॑नू॒च्यते᳚ । \newline
23. अ॒नू॒च्यते॒ स सो॑ ऽनू॒च्यते॑ ऽनू॒च्यते॒ सः । \newline
24. अ॒नू॒च्यत॒ इत्य॑नु - उ॒च्यते᳚ । \newline
25. स आ॑सा मासाꣳ॒॒ स स आ॑साम् । \newline
26. आ॒सां॒ ॅवर्णो॒ वर्ण॑ आसा मासां॒ ॅवर्णः॑ । \newline
27. वर्ण॒ स्तम् तं ॅवर्णो॒ वर्ण॒ स्तम् । \newline
28. तम् त्वा᳚ त्वा॒ तम् तम् त्वा᳚ । \newline
29. त्वा॒ स॒मिद्भिः॑ स॒मिद्भि॑ स्त्वा त्वा स॒मिद्भिः॑ । \newline
30. स॒मिद्भि॑ रङ्गिरो अङ्गिरः स॒मिद्भिः॑ स॒मिद्भि॑ रङ्गिरः । \newline
31. स॒मिद्भि॒रिति॑ स॒मित् - भिः॒ । \newline
32. अ॒ङ्गि॒र॒ इती त्य॑ङ्गिरो अङ्गिर॒ इति॑ । \newline
33. इत्या॑हा॒हे तीत्या॑ह । \newline
34. आ॒ह॒ सा॒मि॒धे॒नीषु॑ सामिधे॒नी ष्वा॑हाह सामिधे॒नीषु॑ । \newline
35. सा॒मि॒धे॒नी ष्वे॒वैव सा॑मिधे॒नीषु॑ सामिधे॒नी ष्वे॒व । \newline
36. सा॒मि॒धे॒नीष्विति॑ सां - इ॒धे॒नीषु॑ । \newline
37. ए॒व तत् तदे॒वैव तत् । \newline
38. तज् ज्योति॒र् ज्योति॒ स्तत् तज् ज्योतिः॑ । \newline
39. ज्योति॑र् जनयति जनयति॒ ज्योति॒र् ज्योति॑र् जनयति । \newline
40. ज॒न॒य॒ति॒ स्त्रियः॒ स्त्रियो॑ जनयति जनयति॒ स्त्रियः॑ । \newline
41. स्त्रिय॒ स्तेन॒ तेन॒ स्त्रियः॒ स्त्रिय॒ स्तेन॑ । \newline
42. तेन॒ यद् यत् तेन॒ तेन॒ यत् । \newline
43. यदृच॒ ऋचो॒ यद् यदृचः॑ । \newline
44. ऋचः॒ स्त्रियः॒ स्त्रिय॒ ऋच॒ ऋचः॒ स्त्रियः॑ । \newline
45. स्त्रिय॒ स्तेन॒ तेन॒ स्त्रियः॒ स्त्रिय॒ स्तेन॑ । \newline
46. तेन॒ यद् यत् तेन॒ तेन॒ यत् । \newline
47. यद् गा॑य॒त्रियो॑ गाय॒त्रियो॒ यद् यद् गा॑य॒त्रियः॑ । \newline
48. गा॒य॒त्रियः॒ स्त्रियः॒ स्त्रियो॑ गाय॒त्रियो॑ गाय॒त्रियः॒ स्त्रियः॑ । \newline
49. स्त्रिय॒ स्तेन॒ तेन॒ स्त्रियः॒ स्त्रिय॒ स्तेन॑ । \newline
50. तेन॒ यद् यत् तेन॒ तेन॒ यत् । \newline
51. यथ् सा॑मिधे॒न्यः॑ सामिधे॒न्यो॑ यद् यथ् सा॑मिधे॒न्यः॑ । \newline
52. सा॒मि॒धे॒न्यो॑ वृष॑ण्वतीं॒ ॅवृष॑ण्वतीꣳ सामिधे॒न्यः॑ सामिधे॒न्यो॑ वृष॑ण्वतीम् । \newline
53. सा॒मि॒धे॒न्य॑ इति॑ सां - इ॒धे॒न्यः॑ । \newline
54. वृष॑ण्वती॒ मन्वनु॒ वृष॑ण्वतीं॒ ॅवृष॑ण्वती॒ मनु॑ । \newline
55. वृष॑ण्वती॒मिति॒ वृषण्॑ - व॒ती॒म् । \newline
56. अन्वा॑ हा॒हा न्वन्वा॑ह । \newline
57. आ॒ह॒ तेन॒ तेना॑ हाह॒ तेन॑ । \newline

\textbf{Ghana Paata } \newline

1. यथ् स॒माव॑थ् स॒माव॒द् यद् यथ् स॒माव॑द् वि॒द्व वि॒द्व स॒माव॒द् यद् यथ् स॒माव॑द् वि॒द्व । \newline
2. स॒माव॑द् वि॒द्व वि॒द्व स॒माव॑थ् स॒माव॑द् वि॒द्व क॒था क॒था वि॒द्व स॒माव॑थ् स॒माव॑द् वि॒द्व क॒था । \newline
3. वि॒द्व क॒था क॒था वि॒द्व वि॒द्व क॒था त्वम् त्वम् क॒था वि॒द्व वि॒द्व क॒था त्वम् । \newline
4. क॒था त्वम् त्वम् क॒था क॒था त्व म॒ग्नि म॒ग्निम् त्वम् क॒था क॒था त्व म॒ग्निम् । \newline
5. त्व म॒ग्नि म॒ग्निम् त्वम् त्व म॒ग्नि मजी॑ज॒नो ऽजी॑जनो॒ ऽग्निम् त्वम् त्व म॒ग्नि मजी॑जनः । \newline
6. अ॒ग्नि मजी॑ज॒नो ऽजी॑जनो॒ ऽग्नि म॒ग्नि मजी॑जनो॒ न नाजी॑जनो॒ ऽग्नि म॒ग्नि मजी॑जनो॒ न । \newline
7. अजी॑जनो॒ न नाजी॑ज॒नो ऽजी॑जनो॒ नाह म॒हम् नाजी॑ज॒नो ऽजी॑जनो॒ नाहम् । \newline
8. नाह म॒हम् न नाह मितीत्य॒हम् न नाह मिति॑ । \newline
9. अ॒ह मितीत्य॒ह म॒ह मिति॑ सामिधे॒नीनाꣳ॑ सामिधे॒नीना॒ मित्य॒ह म॒ह मिति॑ सामिधे॒नीना᳚म् । \newline
10. इति॑ सामिधे॒नीनाꣳ॑ सामिधे॒नीना॒ मितीति॑ सामिधे॒नीना॑ मे॒वैव सा॑मिधे॒नीना॒ मितीति॑ सामिधे॒नीना॑ मे॒व । \newline
11. सा॒मि॒धे॒नीना॑ मे॒वैव सा॑मिधे॒नीनाꣳ॑ सामिधे॒नीना॑ मे॒वाह म॒ह मे॒व सा॑मिधे॒नीनाꣳ॑ सामिधे॒नीना॑ मे॒वाहम् । \newline
12. सा॒मि॒धे॒नीना॒मिति॑ सां - इ॒धे॒नीना᳚म् । \newline
13. ए॒वाह म॒ह मे॒वैवाहं ॅवर्णं॒ ॅवर्ण॑ म॒ह मे॒वैवाहं ॅवर्ण᳚म् । \newline
14. अ॒हं ॅवर्णं॒ ॅवर्ण॑ म॒ह म॒हं ॅवर्णं॑ ॅवेद वेद॒ वर्ण॑ म॒ह म॒हं ॅवर्णं॑ ॅवेद । \newline
15. वर्णं॑ ॅवेद वेद॒ वर्णं॒ ॅवर्णं॑ ॅवे॒दे तीति॑ वेद॒ वर्णं॒ ॅवर्णं॑ ॅवे॒दे ति॑ । \newline
16. वे॒दे तीति॑ वेद वे॒दे त्य॑ब्रवी दब्रवी॒ दिति॑ वेद वे॒दे त्य॑ब्रवीत् । \newline
17. इत्य॑ब्रवी दब्रवी॒ दिती त्य॑ब्रवी॒द् यद् यद॑ब्रवी॒ दिती त्य॑ब्रवी॒द् यत् । \newline
18. अ॒ब्र॒वी॒द् यद् यद॑ब्रवी दब्रवी॒द् यद् घृ॒तव॑द् घृ॒तव॒द् यद॑ब्रवी दब्रवी॒द् यद् घृ॒तव॑त् । \newline
19. यद् घृ॒तव॑द् घृ॒तव॒द् यद् यद् घृ॒तव॑त् प॒दम् प॒दम् घृ॒तव॒द् यद् यद् घृ॒तव॑त् प॒दम् । \newline
20. घृ॒तव॑त् प॒दम् प॒दम् घृ॒तव॑द् घृ॒तव॑त् प॒द म॑नू॒च्यते॑ ऽनू॒च्यते॑ प॒दम् घृ॒तव॑द् घृ॒तव॑त् प॒द म॑नू॒च्यते᳚ । \newline
21. घृ॒तव॒दिति॑ घृ॒त - व॒त् । \newline
22. प॒द म॑नू॒च्यते॑ ऽनू॒च्यते॑ प॒दम् प॒द म॑नू॒च्यते॒ स सो॑ ऽनू॒च्यते॑ प॒दम् प॒द म॑नू॒च्यते॒ सः । \newline
23. अ॒नू॒च्यते॒ स सो॑ ऽनू॒च्यते॑ ऽनू॒च्यते॒ स आ॑सा मासाꣳ॒॒ सो॑ ऽनू॒च्यते॑ ऽनू॒च्यते॒ स आ॑साम् । \newline
24. अ॒नू॒च्यत॒ इत्य॑नु - उ॒च्यते᳚ । \newline
25. स आ॑सा मासाꣳ॒॒ स स आ॑सां॒ ॅवर्णो॒ वर्ण॑ आसाꣳ॒॒ स स आ॑सां॒ ॅवर्णः॑ । \newline
26. आ॒सां॒ ॅवर्णो॒ वर्ण॑ आसा मासां॒ ॅवर्ण॒ स्तम् तं ॅवर्ण॑ आसा मासां॒ ॅवर्ण॒ स्तम् । \newline
27. वर्ण॒ स्तम् तं ॅवर्णो॒ वर्ण॒ स्तम् त्वा᳚ त्वा॒ तं ॅवर्णो॒ वर्ण॒ स्तम् त्वा᳚ । \newline
28. तम् त्वा᳚ त्वा॒ तम् तम् त्वा॑ स॒मिद्भिः॑ स॒मिद्भि॑ स्त्वा॒ तम् तम् त्वा॑ स॒मिद्भिः॑ । \newline
29. त्वा॒ स॒मिद्भिः॑ स॒मिद्भि॑ स्त्वा त्वा स॒मिद्भि॑ रङ्गिरो अङ्गिरः स॒मिद्भि॑ स्त्वा त्वा स॒मिद्भि॑ रङ्गिरः । \newline
30. स॒मिद्भि॑ रङ्गिरो अङ्गिरः स॒मिद्भिः॑ स॒मिद्भि॑ रङ्गिर॒ इती त्य॑ङ्गिरः स॒मिद्भिः॑ स॒मिद्भि॑ रङ्गिर॒ इति॑ । \newline
31. स॒मिद्भि॒रिति॑ स॒मित् - भिः॒ । \newline
32. अ॒ङ्गि॒र॒ इती त्य॑ङ्गिरो अङ्गिर॒ इत्या॑हा॒हे त्य॑ङ्गिरो अङ्गिर॒ इत्या॑ह । \newline
33. इत्या॑हा॒हे तीत्या॑ह सामिधे॒नीषु॑ सामिधे॒नी ष्वा॒हे तीत्या॑ह सामिधे॒नीषु॑ । \newline
34. आ॒ह॒ सा॒मि॒धे॒नीषु॑ सामिधे॒नी ष्वा॑हाह सामिधे॒नी ष्वे॒वैव सा॑मिधे॒नी ष्वा॑हाह सामिधे॒नी ष्वे॒व । \newline
35. सा॒मि॒धे॒नी ष्वे॒वैव सा॑मिधे॒नीषु॑ सामिधे॒नी ष्वे॒व तत् तदे॒व सा॑मिधे॒नीषु॑ सामिधे॒नी ष्वे॒व तत् । \newline
36. सा॒मि॒धे॒नीष्विति॑ सां - इ॒धे॒नीषु॑ । \newline
37. ए॒व तत् तदे॒वैव तज् ज्योति॒र् ज्योति॒ स्तदे॒वैव तज् ज्योतिः॑ । \newline
38. तज् ज्योति॒र् ज्योति॒ स्तत् तज् ज्योति॑र् जनयति जनयति॒ ज्योति॒ स्तत् तज् ज्योति॑र् जनयति । \newline
39. ज्योति॑र् जनयति जनयति॒ ज्योति॒र् ज्योति॑र् जनयति॒ स्त्रियः॒ स्त्रियो॑ जनयति॒ ज्योति॒र् ज्योति॑र् जनयति॒ स्त्रियः॑ । \newline
40. ज॒न॒य॒ति॒ स्त्रियः॒ स्त्रियो॑ जनयति जनयति॒ स्त्रिय॒ स्तेन॒ तेन॒ स्त्रियो॑ जनयति जनयति॒ स्त्रिय॒ स्तेन॑ । \newline
41. स्त्रिय॒ स्तेन॒ तेन॒ स्त्रियः॒ स्त्रिय॒ स्तेन॒ यद् यत् तेन॒ स्त्रियः॒ स्त्रिय॒ स्तेन॒ यत् । \newline
42. तेन॒ यद् यत् तेन॒ तेन॒ यदृच॒ ऋचो॒ यत् तेन॒ तेन॒ यदृचः॑ । \newline
43. यदृच॒ ऋचो॒ यद् यदृचः॒ स्त्रियः॒ स्त्रिय॒ ऋचो॒ यद् यदृचः॒ स्त्रियः॑ । \newline
44. ऋचः॒ स्त्रियः॒ स्त्रिय॒ ऋच॒ ऋचः॒ स्त्रिय॒ स्तेन॒ तेन॒ स्त्रिय॒ ऋच॒ ऋचः॒ स्त्रिय॒ स्तेन॑ । \newline
45. स्त्रिय॒ स्तेन॒ तेन॒ स्त्रियः॒ स्त्रिय॒ स्तेन॒ यद् यत् तेन॒ स्त्रियः॒ स्त्रिय॒ स्तेन॒ यत् । \newline
46. तेन॒ यद् यत् तेन॒ तेन॒ यद् गा॑य॒त्रियो॑ गाय॒त्रियो॒ यत् तेन॒ तेन॒ यद् गा॑य॒त्रियः॑ । \newline
47. यद् गा॑य॒त्रियो॑ गाय॒त्रियो॒ यद् यद् गा॑य॒त्रियः॒ स्त्रियः॒ स्त्रियो॑ गाय॒त्रियो॒ यद् यद् गा॑य॒त्रियः॒ स्त्रियः॑ । \newline
48. गा॒य॒त्रियः॒ स्त्रियः॒ स्त्रियो॑ गाय॒त्रियो॑ गाय॒त्रियः॒ स्त्रिय॒ स्तेन॒ तेन॒ स्त्रियो॑ गाय॒त्रियो॑ गाय॒त्रियः॒ स्त्रिय॒ स्तेन॑ । \newline
49. स्त्रिय॒ स्तेन॒ तेन॒ स्त्रियः॒ स्त्रिय॒ स्तेन॒ यद् यत् तेन॒ स्त्रियः॒ स्त्रिय॒ स्तेन॒ यत् । \newline
50. तेन॒ यद् यत् तेन॒ तेन॒ यथ् सा॑मिधे॒न्यः॑ सामिधे॒न्यो॑ यत् तेन॒ तेन॒ यथ् सा॑मिधे॒न्यः॑ । \newline
51. यथ् सा॑मिधे॒न्यः॑ सामिधे॒न्यो॑ यद् यथ् सा॑मिधे॒न्यो॑ वृष॑ण्वतीं॒ ॅवृष॑ण्वतीꣳ सामिधे॒न्यो॑ यद् यथ् सा॑मिधे॒न्यो॑ वृष॑ण्वतीम् । \newline
52. सा॒मि॒धे॒न्यो॑ वृष॑ण्वतीं॒ ॅवृष॑ण्वतीꣳ सामिधे॒न्यः॑ सामिधे॒न्यो॑ वृष॑ण्वती॒ मन्वनु॒ वृष॑ण्वतीꣳ सामिधे॒न्यः॑ सामिधे॒न्यो॑ वृष॑ण्वती॒ मनु॑ । \newline
53. सा॒मि॒धे॒न्य॑ इति॑ सां - इ॒धे॒न्यः॑ । \newline
54. वृष॑ण्वती॒ मन्वनु॒ वृष॑ण्वतीं॒ ॅवृष॑ण्वती॒ मन्वा॑हा॒हानु॒ वृष॑ण्वतीं॒ ॅवृष॑ण्वती॒ मन्वा॑ह । \newline
55. वृष॑ण्वती॒मिति॒ वृषण्॑ - व॒ती॒म् । \newline
56. अन्वा॑हा॒हा न्वन्वा॑ह॒ तेन॒ तेना॒हा न्वन्वा॑ह॒ तेन॑ । \newline
57. आ॒ह॒ तेन॒ तेना॑हाह॒ तेन॒ पुꣳस्व॑तीः॒ पुꣳस्व॑ती॒ स्तेना॑हाह॒ तेन॒ पुꣳस्व॑तीः । \newline
\pagebreak
\markright{ TS 2.5.8.5  \hfill https://www.vedavms.in \hfill}
\addcontentsline{toc}{section}{ TS 2.5.8.5 }
\section*{ TS 2.5.8.5 }

\textbf{TS 2.5.8.5 } \newline
\textbf{Samhita Paata} \newline

तेन॒ पुꣳस्व॑ती॒स्तेन॒ सेन्द्रा॒स्तेन॑ मिथु॒ना अ॒ग्निर्दे॒वानां᳚ दू॒त आसी॑दु॒शना॑ का॒व्योऽसु॑राणां॒ तौ प्र॒जाप॑तिं प्र॒श्नमै॑ताꣳ॒॒ स प्र॒जाप॑तिर॒ग्निं दू॒तं ॅवृ॑णीमह॒ इत्य॒भि प॒र्याव॑र्तत॒ ततो॑ दे॒वा अभ॑व॒न् पराऽसु॑रा॒ यस्यै॒वं ॅवि॒दुषो॒ऽग्निं दू॒तं ॅवृ॑णीमह॒ इत्य॒न्वाह॒ भव॑त्या॒त्मना॒ परा᳚ऽस्य॒ भ्रातृ॑व्यो भवत्यद्ध्व॒रव॑ती॒मन्वा॑ह॒ भ्रातृ॑व्यमे॒वैतया᳚ - [  ] \newline

\textbf{Pada Paata} \newline

तेन॑ । पुꣳस्व॑तीः । तेन॑ । सेन्द्रा॒ इति॒ स - इ॒न्द्राः॒ । तेन॑ । मि॒थु॒नाः । अ॒ग्निः । दे॒वाना᳚म् । दू॒तः । आसी᳚त् । उ॒शना᳚ । का॒व्यः । असु॑राणाम् । तौ । प्र॒जाप॑ति॒मिति॑ प्र॒जा - प॒ति॒म् । प्र॒श्नम् । ऐ॒ता॒म् । सः । प्र॒जाप॑ति॒रिति॑ प्र॒जा - प॒तिः॒ । अ॒ग्निम् । दू॒तम् । वृ॒णी॒म॒हे॒ । इति॑ । अ॒भीति॑ । प॒र्याव॑र्त॒तेति॑ परि - आव॑र्तत । ततः॑ । दे॒वाः । अभ॑वन्न् । परेति॑ । असु॑राः । यस्य॑ । ए॒वम् । वि॒दुषः॑ । अ॒ग्निम् । दू॒तम् । वृ॒णी॒म॒हे॒ । इति॑ । अ॒न्वाहेत्य॑नु - आह॑ । भव॑ति । आ॒त्मना᳚ । परेति॑ । अ॒स्य॒ । भ्रातृ॑व्यः । भ॒व॒ति॒ । अ॒द्ध्व॒रव॑ती॒मित्य॑द्ध्व॒र-व॒ती॒म् । अन्वति॑ । आ॒ह॒ । भ्रातृ॑व्यम् । ए॒व । ए॒तया᳚ ।  \newline


\textbf{Krama Paata} \newline

तेन॒ पुꣳस्व॑तीः । पुꣳस्व॑ती॒ स्तेन॑ । तेन॒ सेन्द्राः᳚ । सेन्द्रा॒स्तेन॑ । सेन्द्रा॒ इति॒ स - इ॒न्द्राः॒ । तेन॑ मिथु॒नाः । मि॒थु॒ना अ॒ग्निः । अ॒ग्निर् दे॒वाना᳚म् । दे॒वाना᳚म् दू॒तः । दू॒त आसी᳚त् । आसी॑दु॒शना᳚ । उ॒शना॑ का॒व्यः । का॒व्यो ऽसु॑राणाम् । असु॑राणा॒म् तौ । तौ प्र॒जाप॑तिम् । प्र॒जाप॑तिम् प्र॒श्ञम् । प्र॒जाप॑ति॒मिति॑ प्र॒जा - प॒ति॒म् । प्र॒श्ञमै॑ताम् । ऐ॒ताꣳ॒॒ सः । स प्र॒जाप॑तिः । प्र॒जाप॑तिर॒ग्निम् । प्र॒जाप॑ति॒रिति॑ प्र॒जा - प॒तिः॒ । अ॒ग्निम् दू॒तम् । दू॒तम् ॅवृ॑णीमहे । वृ॒णी॒म॒ह॒ इति॑ । इत्य॒भि । अ॒भि प॒र्याव॑र्तत । प॒र्याव॑र्तत॒ ततः॑ । प॒र्याव॑र्त॒तेति॑ परि - आव॑र्तत । ततो॑ दे॒वाः । दे॒वा अभ॑वन्न् । अभ॑व॒न् परा᳚ । परा ऽसु॑राः । असु॑रा॒ यस्य॑ । यस्यै॒वम् । ए॒वम् ॅवि॒दुषः॑ । वि॒दुषो॒ ऽग्निम् । अ॒ग्निम् दू॒तम् । दू॒तम् ॅवृ॑णीमहे । वृ॒णी॒म॒ह॒ इति॑ । इत्य॒न्वाह॑ । अ॒न्वाह॒ भव॑ति । अ॒न्वाहेत्य॑नु - आह॑ । भव॑त्या॒त्मना᳚ । आ॒त्मना॒ परा᳚ । परा᳚ऽस्य । अ॒स्य॒ भातृ॑व्यः । भ्रातृ॑व्यो भवति । भ॒व॒त्य॒द्ध्व॒रव॑तीम् । अ॒द्ध्व॒रव॑ती॒मनु॑ । अ॒द्ध्व॒रव॑ती॒ मित्य॑द्ध्व॒र - व॒ती॒म् । अन्वा॑ह । आ॒ह॒ भ्रातृ॑व्यम् । भ्रातृ॑व्यमे॒व । ए॒वैतया᳚ । ए॒तया᳚ ध्वरति \newline

\textbf{Jatai Paata} \newline

1. तेन॒ पुꣳस्व॑तीः॒ पुꣳस्व॑ती॒स्तेन॒ तेन॒ पुꣳस्व॑तीः । \newline
2. पुꣳस्व॑ती॒ स्तेन॒ तेन॒ पुꣳस्व॑तीः॒ पुꣳस्व॑ती॒ स्तेन॑ । \newline
3. तेन॒ सेन्द्राः॒ सेन्द्रा॒ स्तेन॒ तेन॒ सेन्द्राः᳚ । \newline
4. सेन्द्रा॒ स्तेन॒ तेन॒ सेन्द्राः॒ सेन्द्रा॒ स्तेन॑ । \newline
5. सेन्द्रा॒ इति॒ स - इ॒न्द्राः॒ । \newline
6. तेन॑ मिथु॒ना मि॑थु॒ना स्तेन॒ तेन॑ मिथु॒नाः । \newline
7. मि॒थु॒ना अ॒ग्नि र॒ग्निर् मि॑थु॒ना मि॑थु॒ना अ॒ग्निः । \newline
8. अ॒ग्निर् दे॒वाना᳚म् दे॒वाना॑ म॒ग्नि र॒ग्निर् दे॒वाना᳚म् । \newline
9. दे॒वाना᳚म् दू॒तो दू॒तो दे॒वाना᳚म् दे॒वाना᳚म् दू॒तः । \newline
10. दू॒त आसी॒ दासी᳚द् दू॒तो दू॒त आसी᳚त् । \newline
11. आसी॑ दु॒शनो॒ शना ऽऽसी॒दासी॑ दु॒शना᳚ । \newline
12. उ॒शना॑ का॒व्यः का॒व्य उ॒शनो॒ शना॑ का॒व्यः । \newline
13. का॒व्यो ऽसु॑राणा॒ मसु॑राणाम् का॒व्यः का॒व्यो ऽसु॑राणाम् । \newline
14. असु॑राणा॒म् तौ ता वसु॑राणा॒ मसु॑राणा॒म् तौ । \newline
15. तौ प्र॒जाप॑तिम् प्र॒जाप॑ति॒म् तौ तौ प्र॒जाप॑तिम् । \newline
16. प्र॒जाप॑तिम् प्र॒श्ञम् प्र॒श्ञम् प्र॒जाप॑तिम् प्र॒जाप॑तिम् प्र॒श्ञम् । \newline
17. प्र॒जाप॑ति॒मिति॑ प्र॒जा - प॒ति॒म् । \newline
18. प्र॒श्ञ मै॑ता मैताम् प्र॒श्ञम् प्र॒श्ञ मै॑ताम् । \newline
19. ऐ॒ताꣳ॒॒ स स ऐ॑ता मैताꣳ॒॒ सः । \newline
20. स प्र॒जाप॑तिः प्र॒जाप॑तिः॒ स स प्र॒जाप॑तिः । \newline
21. प्र॒जाप॑ति र॒ग्नि म॒ग्निम् प्र॒जाप॑तिः प्र॒जाप॑ति र॒ग्निम् । \newline
22. प्र॒जाप॑ति॒रिति॑ प्र॒जा - प॒तिः॒ । \newline
23. अ॒ग्निम् दू॒तम् दू॒त म॒ग्नि म॒ग्निम् दू॒तम् । \newline
24. दू॒तं ॅवृ॑णीमहे वृणीमहे दू॒तम् दू॒तं ॅवृ॑णीमहे । \newline
25. वृ॒णी॒म॒ह॒ इतीति॑ वृणीमहे वृणीमह॒ इति॑ । \newline
26. इत्य॒भ्य॑भी तीत्य॒भि । \newline
27. अ॒भि प॒र्याव॑र्तत प॒र्याव॑र्तता॒ भ्य॑भि प॒र्याव॑र्तत । \newline
28. प॒र्याव॑र्तत॒ तत॒स्ततः॑ प॒र्याव॑र्तत प॒र्याव॑र्तत॒ ततः॑ । \newline
29. प॒र्याव॑र्त॒तेति॑ परि - आव॑र्तत । \newline
30. ततो॑ दे॒वा दे॒वा स्तत॒ स्ततो॑ दे॒वाः । \newline
31. दे॒वा अभ॑व॒न् नभ॑वन् दे॒वा दे॒वा अभ॑वन्न् । \newline
32. अभ॑व॒न् परा॒ परा ऽभ॑व॒न् नभ॑व॒न् परा᳚ । \newline
33. परा ऽसु॑रा॒ असु॑राः॒ परा॒ परा ऽसु॑राः । \newline
34. असु॑रा॒ यस्य॒ यस्या सु॑रा॒ असु॑रा॒ यस्य॑ । \newline
35. यस्यै॒व मे॒वं ॅयस्य॒ यस्यै॒वम् । \newline
36. ए॒वं ॅवि॒दुषो॑ वि॒दुष॑ ए॒व मे॒वं ॅवि॒दुषः॑ । \newline
37. वि॒दुषो॒ ऽग्नि म॒ग्निं ॅवि॒दुषो॑ वि॒दुषो॒ ऽग्निम् । \newline
38. अ॒ग्निम् दू॒तम् दू॒त म॒ग्नि म॒ग्निम् दू॒तम् । \newline
39. दू॒तं ॅवृ॑णीमहे वृणीमहे दू॒तम् दू॒तं ॅवृ॑णीमहे । \newline
40. वृ॒णी॒म॒ह॒ इतीति॑ वृणीमहे वृणीमह॒ इति॑ । \newline
41. इत्य॒न्वाहा॒ न्वाहे तीत्य॒न्वाह॑ । \newline
42. अ॒न्वाह॒ भव॑ति॒ भव॑ त्य॒न्वाहा॒ न्वाह॒ भव॑ति । \newline
43. अ॒न्वाहेत्य॑नु - आह॑ । \newline
44. भव॑ त्या॒त्मना॒ ऽऽत्मना॒ भव॑ति॒ भव॑ त्या॒त्मना᳚ । \newline
45. आ॒त्मना॒ परा॒ परा॒ ऽऽत्मना॒ ऽऽत्मना॒ परा᳚ । \newline
46. परा᳚ ऽस्यास्य॒ परा॒ परा᳚ ऽस्य । \newline
47. अ॒स्य॒ भ्रातृ॑व्यो॒ भ्रातृ॑व्यो ऽस्यास्य॒ भ्रातृ॑व्यः । \newline
48. भ्रातृ॑व्यो भवति भवति॒ भ्रातृ॑व्यो॒ भ्रातृ॑व्यो भवति । \newline
49. भ॒व॒ त्य॒द्ध्व॒रव॑ती मद्ध्व॒रव॑तीम् भवति भव त्यद्ध्व॒रव॑तीम् । \newline
50. अ॒द्ध्व॒रव॑ती॒ मन्वन्व॑द्ध्व॒रव॑ती मद्ध्व॒रव॑ती॒ मनु॑ । \newline
51. अ॒द्ध्व॒रव॑ती॒मित्य॑द्ध्व॒र - व॒ती॒म् । \newline
52. अन्वा॑ हा॒हा न्वन्वा॑ह । \newline
53. आ॒ह॒ भ्रातृ॑व्य॒म् भ्रातृ॑व्य माहाह॒ भ्रातृ॑व्यम् । \newline
54. भ्रातृ॑व्य मे॒वैव भ्रातृ॑व्य॒म् भ्रातृ॑व्य मे॒व । \newline
55. ए॒वैत यै॒त यै॒वै वैतया᳚ । \newline
56. ए॒तया᳚ ध्वरति ध्वर त्ये॒तयै॒तया᳚ ध्वरति । \newline

\textbf{Ghana Paata } \newline

1. तेन॒ पुꣳस्व॑तीः॒ पुꣳस्व॑ती॒ स्तेन॒ तेन॒ पुꣳस्व॑ती॒ स्तेन॒ तेन॒ पुꣳस्व॑ती॒ स्तेन॒ तेन॒ पुꣳस्व॑ती॒ स्तेन॑ । \newline
2. पुꣳस्व॑ती॒ स्तेन॒ तेन॒ पुꣳस्व॑तीः॒ पुꣳस्व॑ती॒ स्तेन॒ सेन्द्राः॒ सेन्द्रा॒ स्तेन॒ पुꣳस्व॑तीः॒ पुꣳस्व॑ती॒ स्तेन॒ सेन्द्राः᳚ । \newline
3. तेन॒ सेन्द्राः॒ सेन्द्रा॒ स्तेन॒ तेन॒ सेन्द्रा॒ स्तेन॒ तेन॒ सेन्द्रा॒ स्तेन॒ तेन॒ सेन्द्रा॒ स्तेन॑ । \newline
4. सेन्द्रा॒ स्तेन॒ तेन॒ सेन्द्राः॒ सेन्द्रा॒ स्तेन॑ मिथु॒ना मि॑थु॒ना स्तेन॒ सेन्द्राः॒ सेन्द्रा॒ स्तेन॑ मिथु॒नाः । \newline
5. सेन्द्रा॒ इति॒ स - इ॒न्द्राः॒ । \newline
6. तेन॑ मिथु॒ना मि॑थु॒ना स्तेन॒ तेन॑ मिथु॒ना अ॒ग्नि र॒ग्निर् मि॑थु॒ना स्तेन॒ तेन॑ मिथु॒ना अ॒ग्निः । \newline
7. मि॒थु॒ना अ॒ग्नि र॒ग्निर् मि॑थु॒ना मि॑थु॒ना अ॒ग्निर् दे॒वाना᳚म् दे॒वाना॑ म॒ग्निर् मि॑थु॒ना मि॑थु॒ना अ॒ग्निर् दे॒वाना᳚म् । \newline
8. अ॒ग्निर् दे॒वाना᳚म् दे॒वाना॑ म॒ग्नि र॒ग्निर् दे॒वाना᳚म् दू॒तो दू॒तो दे॒वाना॑ म॒ग्नि र॒ग्निर् दे॒वाना᳚म् दू॒तः । \newline
9. दे॒वाना᳚म् दू॒तो दू॒तो दे॒वाना᳚म् दे॒वाना᳚म् दू॒त आसी॒ दासी᳚द् दू॒तो दे॒वाना᳚म् दे॒वाना᳚म् दू॒त आसी᳚त् । \newline
10. दू॒त आसी॒ दासी᳚द् दू॒तो दू॒त आसी॑ दु॒शनो॒शना ऽऽसी᳚द् दू॒तो दू॒त आसी॑दु॒शना᳚ । \newline
11. आसी॑ दु॒शनो॒शना ऽऽसी॒ दासी॑दु॒शना॑ का॒व्यः का॒व्य उ॒शना ऽऽसी॒ दासी॑दु॒शना॑ का॒व्यः । \newline
12. उ॒शना॑ का॒व्यः का॒व्य उ॒शनो॒शना॑ का॒व्यो ऽसु॑राणा॒ मसु॑राणाम् का॒व्य उ॒शनो॒शना॑ का॒व्यो ऽसु॑राणाम् । \newline
13. का॒व्यो ऽसु॑राणा॒ मसु॑राणाम् का॒व्यः का॒व्यो ऽसु॑राणा॒म् तौ ता वसु॑राणाम् का॒व्यः का॒व्यो ऽसु॑राणा॒म् तौ । \newline
14. असु॑राणा॒म् तौ ता वसु॑राणा॒ मसु॑राणा॒म् तौ प्र॒जाप॑तिम् प्र॒जाप॑ति॒म् ता वसु॑राणा॒ मसु॑राणा॒म् तौ प्र॒जाप॑तिम् । \newline
15. तौ प्र॒जाप॑तिम् प्र॒जाप॑ति॒म् तौ तौ प्र॒जाप॑तिम् प्र॒श्ञम् प्र॒श्ञम् प्र॒जाप॑ति॒म् तौ तौ प्र॒जाप॑तिम् प्र॒श्ञम् । \newline
16. प्र॒जाप॑तिम् प्र॒श्ञम् प्र॒श्ञम् प्र॒जाप॑तिम् प्र॒जाप॑तिम् प्र॒श्ञ मै॑ता मैताम् प्र॒श्ञम् प्र॒जाप॑तिम् प्र॒जाप॑तिम् प्र॒श्ञ मै॑ताम् । \newline
17. प्र॒जाप॑ति॒मिति॑ प्र॒जा - प॒ति॒म् । \newline
18. प्र॒श्ञ मै॑ता मैताम् प्र॒श्ञम् प्र॒श्ञ मै॑ताꣳ॒॒ स स ऐ॑ताम् प्र॒श्ञम् प्र॒श्ञ मै॑ताꣳ॒॒ सः । \newline
19. ऐ॒ताꣳ॒॒ स स ऐ॑ता मैताꣳ॒॒ स प्र॒जाप॑तिः प्र॒जाप॑तिः॒ स ऐ॑ता मैताꣳ॒॒ स प्र॒जाप॑तिः । \newline
20. स प्र॒जाप॑तिः प्र॒जाप॑तिः॒ स स प्र॒जाप॑ति र॒ग्नि म॒ग्निम् प्र॒जाप॑तिः॒ स स प्र॒जाप॑ति र॒ग्निम् । \newline
21. प्र॒जाप॑तिर॒ग्नि म॒ग्निम् प्र॒जाप॑तिः प्र॒जाप॑ति र॒ग्निम् दू॒तम् दू॒त म॒ग्निम् प्र॒जाप॑तिः प्र॒जाप॑ति र॒ग्निम् दू॒तम् । \newline
22. प्र॒जाप॑ति॒रिति॑ प्र॒जा - प॒तिः॒ । \newline
23. अ॒ग्निम् दू॒तम् दू॒त म॒ग्नि म॒ग्निम् दू॒तं ॅवृ॑णीमहे वृणीमहे दू॒त म॒ग्नि म॒ग्निम् दू॒तं ॅवृ॑णीमहे । \newline
24. दू॒तं ॅवृ॑णीमहे वृणीमहे दू॒तम् दू॒तं ॅवृ॑णीमह॒ इतीति॑ वृणीमहे दू॒तम् दू॒तं ॅवृ॑णीमह॒ इति॑ । \newline
25. वृ॒णी॒म॒ह॒ इतीति॑ वृणीमहे वृणीमह॒ इत्य॒भ्य॑भीति॑ वृणीमहे वृणीमह॒ इत्य॒भि । \newline
26. इत्य॒भ्य॑भीती त्य॒भि प॒र्याव॑र्तत प॒र्याव॑र्तता॒भीती त्य॒भि प॒र्याव॑र्तत । \newline
27. अ॒भि प॒र्याव॑र्तत प॒र्याव॑र्तता॒भ्य॑भि प॒र्याव॑र्तत॒ तत॒ स्ततः॑ प॒र्याव॑र्तता॒भ्य॑भि प॒र्याव॑र्तत॒ ततः॑ । \newline
28. प॒र्याव॑र्तत॒ तत॒ स्ततः॑ प॒र्याव॑र्तत प॒र्याव॑र्तत॒ ततो॑ दे॒वा दे॒वा स्ततः॑ प॒र्याव॑र्तत प॒र्याव॑र्तत॒ ततो॑ दे॒वाः । \newline
29. प॒र्याव॑र्त॒तेति॑ परि - आव॑र्तत । \newline
30. ततो॑ दे॒वा दे॒वा स्तत॒ स्ततो॑ दे॒वा अभ॑व॒न् नभ॑वन् दे॒वा स्तत॒ स्ततो॑ दे॒वा अभ॑वन्न् । \newline
31. दे॒वा अभ॑व॒न् नभ॑वन् दे॒वा दे॒वा अभ॑व॒न् परा॒ परा ऽभ॑वन् दे॒वा दे॒वा अभ॑व॒न् परा᳚ । \newline
32. अभ॑व॒न् परा॒ परा ऽभ॑व॒न् नभ॑व॒न् परा ऽसु॑रा॒ असु॑राः॒ परा ऽभ॑व॒न् नभ॑व॒न् परा ऽसु॑राः । \newline
33. परा ऽसु॑रा॒ असु॑राः॒ परा॒ परा ऽसु॑रा॒ यस्य॒ यस्यासु॑राः॒ परा॒ परा ऽसु॑रा॒ यस्य॑ । \newline
34. असु॑रा॒ यस्य॒ यस्यासु॑रा॒ असु॑रा॒ यस्यै॒व मे॒वं ॅयस्यासु॑रा॒ असु॑रा॒ यस्यै॒वम् । \newline
35. यस्यै॒व मे॒वं ॅयस्य॒ यस्यै॒वं ॅवि॒दुषो॑ वि॒दुष॑ ए॒वं ॅयस्य॒ यस्यै॒वं ॅवि॒दुषः॑ । \newline
36. ए॒वं ॅवि॒दुषो॑ वि॒दुष॑ ए॒व मे॒वं ॅवि॒दुषो॒ ऽग्नि म॒ग्निं ॅवि॒दुष॑ ए॒व मे॒वं ॅवि॒दुषो॒ ऽग्निम् । \newline
37. वि॒दुषो॒ ऽग्नि म॒ग्निं ॅवि॒दुषो॑ वि॒दुषो॒ ऽग्निम् दू॒तम् दू॒त म॒ग्निं ॅवि॒दुषो॑ वि॒दुषो॒ ऽग्निम् दू॒तम् । \newline
38. अ॒ग्निम् दू॒तम् दू॒त म॒ग्नि म॒ग्निम् दू॒तं ॅवृ॑णीमहे वृणीमहे दू॒त म॒ग्नि म॒ग्निम् दू॒तं ॅवृ॑णीमहे । \newline
39. दू॒तं ॅवृ॑णीमहे वृणीमहे दू॒तम् दू॒तं ॅवृ॑णीमह॒ इतीति॑ वृणीमहे दू॒तम् दू॒तं ॅवृ॑णीमह॒ इति॑ । \newline
40. वृ॒णी॒म॒ह॒ इतीति॑ वृणीमहे वृणीमह॒ इत्य॒न्वाहा॒न्वाहे ति॑ वृणीमहे वृणीमह॒ इत्य॒न्वाह॑ । \newline
41. इत्य॒न्वाहा॒न्वा हेतीत्य॒न्वाह॒ भव॑ति॒ भव॑त्य॒न्वा हेतीत्य॒न्वाह॒ भव॑ति । \newline
42. अ॒न्वाह॒ भव॑ति॒ भव॑ त्य॒न्वाहा॒न्वाह॒ भव॑त्या॒त्मना॒ ऽऽत्मना॒ भव॑ त्य॒न्वाहा॒न्वाह॒ भव॑त्या॒त्मना᳚ । \newline
43. अ॒न्वाहेत्य॑नु - आह॑ । \newline
44. भव॑ त्या॒त्मना॒ ऽऽत्मना॒ भव॑ति॒ भव॑ त्या॒त्मना॒ परा॒ परा॒ ऽऽत्मना॒ भव॑ति॒ भव॑ त्या॒त्मना॒ परा᳚ । \newline
45. आ॒त्मना॒ परा॒ परा॒ ऽऽत्मना॒ ऽऽत्मना॒ परा᳚ ऽस्यास्य॒ परा॒ ऽऽत्मना॒ ऽऽत्मना॒ परा᳚ ऽस्य । \newline
46. परा᳚ ऽस्यास्य॒ परा॒ परा᳚ ऽस्य॒ भ्रातृ॑व्यो॒ भ्रातृ॑व्यो ऽस्य॒ परा॒ परा᳚ ऽस्य॒ भ्रातृ॑व्यः । \newline
47. अ॒स्य॒ भ्रातृ॑व्यो॒ भ्रातृ॑व्यो ऽस्यास्य॒ भ्रातृ॑व्यो भवति भवति॒ भ्रातृ॑व्यो ऽस्यास्य॒ भ्रातृ॑व्यो भवति । \newline
48. भ्रातृ॑व्यो भवति भवति॒ भ्रातृ॑व्यो॒ भ्रातृ॑व्यो भव त्यद्ध्व॒रव॑ती मद्ध्व॒रव॑तीम् भवति॒ भ्रातृ॑व्यो॒ भ्रातृ॑व्यो भव त्यद्ध्व॒रव॑तीम् । \newline
49. भ॒व॒ त्य॒द्ध्व॒रव॑ती मद्ध्व॒रव॑तीम् भवति भव त्यद्ध्व॒रव॑ती॒ मन्वन्व॑द्ध्व॒रव॑तीम् भवति भव त्यद्ध्व॒रव॑ती॒ मनु॑ । \newline
50. अ॒द्ध्व॒रव॑ती॒ मन्वन्व॑द्ध्व॒रव॑ती मद्ध्व॒रव॑ती॒ मन्वा॑हा॒हा न्व॑द्ध्व॒रव॑ती मद्ध्व॒रव॑ती॒ मन्वा॑ह । \newline
51. अ॒द्ध्व॒रव॑ती॒मित्य॑द्ध्व॒र - व॒ती॒म् । \newline
52. अन्वा॑हा॒हा न्वन्वा॑ह॒ भ्रातृ॑व्य॒म् भ्रातृ॑व्य मा॒हान्वन्वा॑ह॒ भ्रातृ॑व्यम् । \newline
53. आ॒ह॒ भ्रातृ॑व्य॒म् भ्रातृ॑व्य माहाह॒ भ्रातृ॑व्य मे॒वैव भ्रातृ॑व्य माहाह॒ भ्रातृ॑व्य मे॒व । \newline
54. भ्रातृ॑व्य मे॒वैव भ्रातृ॑व्य॒म् भ्रातृ॑व्य मे॒वैत यै॒त यै॒व भ्रातृ॑व्य॒म् भ्रातृ॑व्य मे॒वैतया᳚ । \newline
55. ए॒वैत यै॒त यै॒वै वैतया᳚ ध्वरति ध्वर त्ये॒तयै॒वै वैतया᳚ ध्वरति । \newline
56. ए॒तया᳚ ध्वरति ध्वर त्ये॒तयै॒तया᳚ ध्वरति शो॒चिष्के॑शः शो॒चिष्के॑शो ध्वर त्ये॒तयै॒तया᳚ ध्वरति शो॒चिष्के॑शः । \newline
\pagebreak
\markright{ TS 2.5.8.6  \hfill https://www.vedavms.in \hfill}
\addcontentsline{toc}{section}{ TS 2.5.8.6 }
\section*{ TS 2.5.8.6 }

\textbf{TS 2.5.8.6 } \newline
\textbf{Samhita Paata} \newline

ध्वरति शो॒चिष्के॑श॒स्तमी॑मह॒ इत्या॑ह प॒वित्र॑मे॒वैतद्-यज॑मानमे॒वैतया॑ पवयति॒ समि॑द्धो अग्न आहु॒तेत्या॑ह परि॒धिमे॒वैतं परि॑ दधा॒त्यस्क॑न्दाय॒ यदत॑ ऊ॒र्द्ध्वम॑भ्याद॒द्ध्याद्यथा॑ बहिः परि॒धि स्कन्द॑ति ता॒दृगे॒व तत् त्रयो॒ वा अ॒ग्नयो॑ हव्य॒वाह॑नो दे॒वानां᳚ कव्य॒वाह॑नः पितृ॒णाꣳ स॒हर॑क्षा॒ असु॑राणां॒ त ए॒तर्ह्या शꣳ॑सन्ते॒ मां ॅव॑रिष्यते॒ मा - [  ] \newline

\textbf{Pada Paata} \newline

ध्व॒र॒ति॒ । शो॒चिष्के॑श॒ इति॑ शो॒चिः - के॒शः॒ । तम् । ई॒म॒हे॒ । इति॑ । आ॒ह॒ । प॒वित्र᳚म् । ए॒व । ए॒तत् । यज॑मानम् । ए॒व । ए॒तया᳚ । प॒व॒य॒ति॒ । समि॑द्ध॒ इति॒ सं - इ॒द्धः॒ । अ॒ग्ने॒ । आ॒हु॒तेत्या᳚ - हु॒त॒ । इति॑ । आ॒ह॒ । प॒रि॒धिमिति॑ परि - धिम् । ए॒व । ए॒तम् । परीति॑ । द॒धा॒ति॒ । अस्क॑न्दाय । यत् । अतः॑ । ऊ॒द्‌र्ध्वम् । अ॒भ्या॒द॒द्ध्यादित्य॑भि - आ॒द॒द्ध्यात् । यथा᳚ । ब॒हिः॒ प॒रि॒धीति॑ बहिः - प॒रि॒धि । स्कन्द॑ति । ता॒दृक् । ए॒व । तत् । त्रयः॑ । वै । अ॒ग्नयः॑ । ह॒व्य॒वाह॑न॒ इति॑ हव्य - वाह॑नः । दे॒वाना᳚म् । क॒व्य॒वाह॑न॒ इति॑ कव्य - वाह॑नः । पि॒तृ॒णाम् । स॒हर॑क्षा॒ इति॑ स॒ह - र॒क्षाः॒ । असु॑राणाम् । ते । ए॒तर्.हि॑ । एति॑ । शꣳ॒॒स॒न्ते॒ । माम् । व॒रि॒ष्य॒ते॒ । माम् ।  \newline


\textbf{Krama Paata} \newline

ध्व॒र॒ति॒ शो॒चिष्के॑शः । शो॒चिष्के॑श॒स्तम् । शो॒चिष्के॑श॒ इति॑ शो॒चिः - के॒शः॒ । तमी॑महे । ई॒म॒ह॒ इति॑ । इत्या॑ह । आ॒ह॒ प॒वित्र᳚म् । प॒वित्र॑मे॒व । ए॒वैतत् । ए॒तद् यज॑मानम् । यज॑मानमे॒व । ए॒वैतया᳚ । ए॒तया॑ पवयति । प॒व॒य॒ति॒ समि॑द्धः । समि॑द्धो अग्ने । समि॑द्ध॒ इति॒ सम् - इ॒द्धः॒ । अ॒ग्न॒ आ॒हु॒त॒ । आ॒हु॒तेति॑ । आ॒हु॒तेत्या᳚ - हु॒त॒ । इत्या॑ह । आ॒ह॒ प॒रि॒धिम् । प॒रि॒धिमे॒व । प॒रि॒धिमिति॑ परि - धिम् । ए॒वैतम् । ए॒तम् परि॑ । परि॑ दधाति । द॒धा॒त्यस्क॑न्दाय । अस्क॑न्दाय॒ यत् । यदतः॑ । अत॑ ऊ॒र्द्ध्वम् । ऊ॒र्द्ध्वम॑भ्याद॒द्ध्यात् । अ॒भ्या॒द॒द्ध्याद् यथा᳚ । अ॒भ्या॒द॒द्ध्यादित्य॑भि - आ॒द॒द्ध्यात् । यथा॑ बहिःपरि॒धि । ब॒हिः॒प॒रि॒धि स्कन्द॑ति । ब॒हिः॒प॒रि॒धीति॑ बहिः - प॒रि॒धि । स्कन्द॑ति ता॒दृक् । ता॒दृगे॒व । ए॒व तत् । तत् त्रयः॑ । त्रयो॒ वै । वा अ॒ग्नयः॑ । अ॒ग्नयो॑ हव्य॒वाह॑नः । ह॒व्य॒वाह॑नो दे॒वाना᳚म् । ह॒व्य॒वाह॑न॒ इति॑ हव्य - वाह॑नः । दे॒वाना᳚म् कव्य॒वाह॑नः । क॒व्य॒वाह॑नः पितृ॒णाम् । क॒व्य॒वाह॑न॒ इति॑ कव्य - वाह॑नः । पि॒तृ॒णाꣳ स॒हर॑क्षाः । स॒हर॑क्षा॒ असु॑राणाम् । स॒हर॑क्षा॒ इति॑ स॒ह - र॒क्षाः॒ । असु॑राणा॒म् ते । त ए॒तर्.हि॑ । ए॒तर्.ह्या । आ शꣳ॑सन्ते । शꣳ॒॒स॒न्ते॒ माम् । माम् ॅव॑रिष्यते । व॒रि॒ष्य॒ते॒ माम् ( ) । मामिति॑ \newline

\textbf{Jatai Paata} \newline

1. ध्व॒र॒ति॒ शो॒चिष्के॑शः शो॒चिष्के॑शो ध्वरति ध्वरति शो॒चिष्के॑शः । \newline
2. शो॒चिष्के॑श॒ स्तम् तꣳ शो॒चिष्के॑शः शो॒चिष्के॑श॒ स्तम् । \newline
3. शो॒चिष्के॑श॒ इति॑ शो॒चिः - के॒शः॒ । \newline
4. त मी॑मह ईमहे॒ तम् त मी॑महे । \newline
5. ई॒म॒ह॒ इतीती॑मह ईमह॒ इति॑ । \newline
6. इत्या॑ हा॒हे तीत्या॑ह । \newline
7. आ॒ह॒ प॒वित्र॑म् प॒वित्र॑ माहाह प॒वित्र᳚म् । \newline
8. प॒वित्र॑ मे॒वैव प॒वित्र॑म् प॒वित्र॑ मे॒व । \newline
9. ए॒वैत दे॒त दे॒वैवैतत् । \newline
10. ए॒तद् यज॑मानं॒ ॅयज॑मान मे॒त दे॒तद् यज॑मानम् । \newline
11. यज॑मान मे॒वैव यज॑मानं॒ ॅयज॑मान मे॒व । \newline
12. ए॒वैतयै॒तयै॒वै वैतया᳚ । \newline
13. ए॒तया॑ पवयति पवय त्ये॒त यै॒तया॑ पवयति । \newline
14. प॒व॒य॒ति॒ समि॑द्धः॒ समि॑द्धः पवयति पवयति॒ समि॑द्धः । \newline
15. समि॑द्धो अग्ने ऽग्ने॒ समि॑द्धः॒ समि॑द्धो अग्ने । \newline
16. समि॑द्ध॒ इति॒ सं - इ॒द्धः॒ । \newline
17. अ॒ग्न॒ आ॒हु॒ता॒ हु॒ता॒ग्ने॒ ऽग्न॒ आ॒हु॒त॒ । \newline
18. आ॒हु॒ते तीत्या॑हुता हु॒ते ति॑ । \newline
19. आ॒हु॒तेत्या᳚ - हु॒त॒ । \newline
20. इत्या॑हा॒हे तीत्या॑ह । \newline
21. आ॒ह॒ प॒रि॒धिम् प॑रि॒धि मा॑हाह परि॒धिम् । \newline
22. प॒रि॒धि मे॒वैव प॑रि॒धिम् प॑रि॒धि मे॒व । \newline
23. प॒रि॒धिमिति॑ परि - धिम् । \newline
24. ए॒वैत मे॒त मे॒वैवैतम् । \newline
25. ए॒तम् परि॒ पर्ये॒त मे॒तम् परि॑ । \newline
26. परि॑ दधाति दधाति॒ परि॒ परि॑ दधाति । \newline
27. द॒धा॒ त्यस्क॑न्दा॒या स्क॑न्दाय दधाति दधा॒ त्यस्क॑न्दाय । \newline
28. अस्क॑न्दाय॒ यद् यदस्क॑न्दा॒या स्क॑न्दाय॒ यत् । \newline
29. यदतो ऽतो॒ यद् यदतः॑ । \newline
30. अत॑ ऊ॒र्द्ध्व मू॒र्द्ध्व मतो ऽत॑ ऊ॒र्द्ध्वम् । \newline
31. ऊ॒र्द्ध्व म॑भ्याद॒द्ध्या द॑भ्याद॒द्ध्या दू॒र्द्ध्व मू॒र्द्ध्व म॑भ्याद॒द्ध्यात् । \newline
32. अ॒भ्या॒द॒द्ध्याद् यथा॒ यथा᳚ ऽभ्याद॒द्ध्या द॑भ्याद॒द्ध्याद् यथा᳚ । \newline
33. अ॒भ्या॒द॒द्ध्यादित्य॑भि - आ॒द॒द्ध्यात् । \newline
34. यथा॑ बहिःपरि॒धि ब॑हिःपरि॒धि यथा॒ यथा॑ बहिःपरि॒धि । \newline
35. ब॒हिः॒प॒रि॒धि स्कन्द॑ति॒ स्कन्द॑ति बहिःपरि॒धि ब॑हिःपरि॒धि स्कन्द॑ति । \newline
36. ब॒हिः॒प॒रि॒धीति॑ बहिः - प॒रि॒धि । \newline
37. स्कन्द॑ति ता॒दृक् ता॒दृख् स्कन्द॑ति॒ स्कन्द॑ति ता॒दृक् । \newline
38. ता॒दृ गे॒वैव ता॒दृक् ता॒दृ गे॒व । \newline
39. ए॒व तत् तदे॒वैव तत् । \newline
40. तत् त्रय॒ स्त्रय॒ स्तत् तत् त्रयः॑ । \newline
41. त्रयो॒ वै वै त्रय॒ स्त्रयो॒ वै । \newline
42. वा अ॒ग्नयो॒ ऽग्नयो॒ वै वा अ॒ग्नयः॑ । \newline
43. अ॒ग्नयो॑ हव्य॒वाह॑नो हव्य॒वाह॑नो॒ ऽग्नयो॒ ऽग्नयो॑ हव्य॒वाह॑नः । \newline
44. ह॒व्य॒वाह॑नो दे॒वाना᳚म् दे॒वानाꣳ॑ हव्य॒वाह॑नो हव्य॒वाह॑नो दे॒वाना᳚म् । \newline
45. ह॒व्य॒वाह॑न॒ इति॑ हव्य - वाह॑नः । \newline
46. दे॒वाना᳚म् कव्य॒वाह॑नः कव्य॒वाह॑नो दे॒वाना᳚म् दे॒वाना᳚म् कव्य॒वाह॑नः । \newline
47. क॒व्य॒वाह॑नः पितृ॒णाम् पि॑तृ॒णाम् क॑व्य॒वाह॑नः कव्य॒वाह॑नः पितृ॒णाम् । \newline
48. क॒व्य॒वाह॑न॒ इति॑ कव्य - वाह॑नः । \newline
49. पि॒तृ॒णाꣳ स॒हर॑क्षाः स॒हर॑क्षाः पितृ॒णाम् पि॑तृ॒णाꣳ स॒हर॑क्षाः । \newline
50. स॒हर॑क्षा॒ असु॑राणा॒ मसु॑राणाꣳ स॒हर॑क्षाः स॒हर॑क्षा॒ असु॑राणाम् । \newline
51. स॒हर॑क्षा॒ इति॑ स॒ह - र॒क्षाः॒ । \newline
52. असु॑राणा॒म् ते ते ऽसु॑राणा॒ मसु॑राणा॒म् ते । \newline
53. त ए॒तर् ह्ये॒तर्.हि॒ ते त ए॒तर्.हि॑ । \newline
54. ए॒तर् ह्यैतर् ह्ये॒तर् ह्या । \newline
55. आ शꣳ॑सन्ते शꣳसन्त॒ आ शꣳ॑सन्ते । \newline
56. शꣳ॒॒स॒न्ते॒ माम् माꣳ शꣳ॑सन्ते शꣳसन्ते॒ माम् । \newline
57. मां ॅव॑रिष्यते वरिष्यते॒ माम् मां ॅव॑रिष्यते । \newline
58. व॒रि॒ष्य॒ते॒ माम् मां ॅव॑रिष्यते वरिष्यते॒ माम् । \newline
59. मा मितीति॒ माम् मा मिति॑ । \newline

\textbf{Ghana Paata } \newline

1. ध्व॒र॒ति॒ शो॒चिष्के॑शः शो॒चिष्के॑शो ध्वरति ध्वरति शो॒चिष्के॑श॒ स्तम् तꣳ शो॒चिष्के॑शो ध्वरति ध्वरति शो॒चिष्के॑श॒ स्तम् । \newline
2. शो॒चिष्के॑श॒ स्तम् तꣳ शो॒चिष्के॑शः शो॒चिष्के॑श॒ स्त मी॑मह ईमहे॒ तꣳ शो॒चिष्के॑शः शो॒चिष्के॑श॒ स्त मी॑महे । \newline
3. शो॒चिष्के॑श॒ इति॑ शो॒चिः - के॒शः॒ । \newline
4. त मी॑मह ईमहे॒ तम् त मी॑मह॒ इतीती॑महे॒ तम् त मी॑मह॒ इति॑ । \newline
5. ई॒म॒ह॒ इतीती॑मह ईमह॒ इत्या॑हा॒हे ती॑मह ईमह॒ इत्या॑ह । \newline
6. इत्या॑हा॒हे तीत्या॑ह प॒वित्र॑म् प॒वित्र॑ मा॒हे तीत्या॑ह प॒वित्र᳚म् । \newline
7. आ॒ह॒ प॒वित्र॑म् प॒वित्र॑ माहाह प॒वित्र॑ मे॒वैव प॒वित्र॑ माहाह प॒वित्र॑ मे॒व । \newline
8. प॒वित्र॑ मे॒वैव प॒वित्र॑म् प॒वित्र॑ मे॒वैत दे॒त दे॒व प॒वित्र॑म् प॒वित्र॑ मे॒वैतत् । \newline
9. ए॒वैत दे॒त दे॒वै वैतद् यज॑मानं॒ ॅयज॑मान मे॒त दे॒वैवैतद् यज॑मानम् । \newline
10. ए॒तद् यज॑मानं॒ ॅयज॑मान मे॒तदे॒तद् यज॑मान मे॒वैव यज॑मान मे॒तदे॒तद् यज॑मान मे॒व । \newline
11. यज॑मान मे॒वैव यज॑मानं॒ ॅयज॑मान मे॒वैत यै॒त यै॒व यज॑मानं॒ ॅयज॑मान मे॒वैतया᳚ । \newline
12. ए॒वैत यै॒त यै॒वै वैतया॑ पवयति पवय त्ये॒त यै॒वै वैतया॑ पवयति । \newline
13. ए॒तया॑ पवयति पवय त्ये॒त यै॒तया॑ पवयति॒ समि॑द्धः॒ समि॑द्धः पवय त्ये॒त यै॒तया॑ पवयति॒ समि॑द्धः । \newline
14. प॒व॒य॒ति॒ समि॑द्धः॒ समि॑द्धः पवयति पवयति॒ समि॑द्धो अग्ने ऽग्ने॒ समि॑द्धः पवयति पवयति॒ समि॑द्धो अग्ने । \newline
15. समि॑द्धो अग्ने ऽग्ने॒ समि॑द्धः॒ समि॑द्धो अग्न आहुता हुताग्ने॒ समि॑द्धः॒ समि॑द्धो अग्न आहुत । \newline
16. समि॑द्ध॒ इति॒ सं - इ॒द्धः॒ । \newline
17. अ॒ग्न॒ आ॒हु॒ता॒ हु॒ता॒ग्ने॒ ऽग्न॒ आ॒हु॒ते तीत्या॑हुताग्ने ऽग्न आहु॒ते ति॑ । \newline
18. आ॒हु॒ते तीत्या॑हुताहु॒ते त्या॑हा॒हे त्या॑हुताहु॒ते त्या॑ह । \newline
19. आ॒हु॒तेत्या᳚ - हु॒त॒ । \newline
20. इत्या॑हा॒हे तीत्या॑ह परि॒धिम् प॑रि॒धि मा॒हे तीत्या॑ह परि॒धिम् । \newline
21. आ॒ह॒ प॒रि॒धिम् प॑रि॒धि मा॑हाह परि॒धि मे॒वैव प॑रि॒धि मा॑हाह परि॒धि मे॒व । \newline
22. प॒रि॒धि मे॒वैव प॑रि॒धिम् प॑रि॒धि मे॒वैत मे॒त मे॒व प॑रि॒धिम् प॑रि॒धि मे॒वैतम् । \newline
23. प॒रि॒धिमिति॑ परि - धिम् । \newline
24. ए॒वैत मे॒त मे॒वैवैतम् परि॒ पर्ये॒त मे॒वैवैतम् परि॑ । \newline
25. ए॒तम् परि॒ पर्ये॒त मे॒तम् परि॑ दधाति दधाति॒ पर्ये॒त मे॒तम् परि॑ दधाति । \newline
26. परि॑ दधाति दधाति॒ परि॒ परि॑ दधा॒ त्यस्क॑न्दा॒या स्क॑न्दाय दधाति॒ परि॒ परि॑ दधा॒ त्यस्क॑न्दाय । \newline
27. द॒धा॒ त्यस्क॑न्दा॒या स्क॑न्दाय दधाति दधा॒ त्यस्क॑न्दाय॒ यद् यदस्क॑न्दाय दधाति दधा॒ त्यस्क॑न्दाय॒ यत् । \newline
28. अस्क॑न्दाय॒ यद् यदस्क॑न्दा॒या स्क॑न्दाय॒ यदतो ऽतो॒ यदस्क॑न्दा॒या स्क॑न्दाय॒ यदतः॑ । \newline
29. यदतो ऽतो॒ यद् यदत॑ ऊ॒र्द्ध्व मू॒र्द्ध्व मतो॒ यद् यदत॑ ऊ॒र्द्ध्वम् । \newline
30. अत॑ ऊ॒र्द्ध्व मू॒र्द्ध्व मतो ऽत॑ ऊ॒र्द्ध्व म॑भ्याद॒द्ध्या द॑भ्याद॒द्ध्या दू॒र्द्ध्व मतो ऽत॑ ऊ॒र्द्ध्व म॑भ्याद॒द्ध्यात् । \newline
31. ऊ॒र्द्ध्व म॑भ्याद॒द्ध्या द॑भ्याद॒द्ध्या दू॒र्द्ध्व मू॒र्द्ध्व म॑भ्याद॒द्ध्याद् यथा॒ यथा᳚ ऽभ्याद॒द्ध्या दू॒र्द्ध्व मू॒र्द्ध्व म॑भ्याद॒द्ध्याद् यथा᳚ । \newline
32. अ॒भ्या॒द॒द्ध्याद् यथा॒ यथा᳚ ऽभ्याद॒द्ध्या द॑भ्याद॒द्ध्याद् यथा॑ बहिःपरि॒धि ब॑हिःपरि॒धि यथा᳚ ऽभ्याद॒द्ध्या द॑भ्याद॒द्ध्याद् यथा॑ बहिःपरि॒धि । \newline
33. अ॒भ्या॒द॒द्ध्यादित्य॑भि - आ॒द॒द्ध्यात् । \newline
34. यथा॑ बहिःपरि॒धि ब॑हिःपरि॒धि यथा॒ यथा॑ बहिःपरि॒धि स्कन्द॑ति॒ स्कन्द॑ति बहिःपरि॒धि यथा॒ यथा॑ बहिःपरि॒धि स्कन्द॑ति । \newline
35. ब॒हिः॒प॒रि॒धि स्कन्द॑ति॒ स्कन्द॑ति बहिःपरि॒धि ब॑हिःपरि॒धि स्कन्द॑ति ता॒दृक् ता॒दृख् स्कन्द॑ति बहिःपरि॒धि ब॑हिःपरि॒धि स्कन्द॑ति ता॒दृक् । \newline
36. ब॒हिः॒प॒रि॒धीति॑ बहिः - प॒रि॒धि । \newline
37. स्कन्द॑ति ता॒दृक् ता॒दृख् स्कन्द॑ति॒ स्कन्द॑ति ता॒दृगे॒वैव ता॒दृख् स्कन्द॑ति॒ स्कन्द॑ति ता॒दृगे॒व । \newline
38. ता॒दृगे॒वैव ता॒दृक् ता॒दृगे॒व तत् तदे॒व ता॒दृक् ता॒दृगे॒व तत् । \newline
39. ए॒व तत् तदे॒वैव तत् त्रय॒ स्त्रय॒ स्त दे॒वैव तत् त्रयः॑ । \newline
40. तत् त्रय॒ स्त्रय॒ स्तत् तत् त्रयो॒ वै वै त्रय॒ स्तत् तत् त्रयो॒ वै । \newline
41. त्रयो॒ वै वै त्रय॒ स्त्रयो॒ वा अ॒ग्नयो॒ ऽग्नयो॒ वै त्रय॒ स्त्रयो॒ वा अ॒ग्नयः॑ । \newline
42. वा अ॒ग्नयो॒ ऽग्नयो॒ वै वा अ॒ग्नयो॑ हव्य॒वाह॑नो हव्य॒वाह॑नो॒ ऽग्नयो॒ वै वा अ॒ग्नयो॑ हव्य॒वाह॑नः । \newline
43. अ॒ग्नयो॑ हव्य॒वाह॑नो हव्य॒वाह॑नो॒ ऽग्नयो॒ ऽग्नयो॑ हव्य॒वाह॑नो दे॒वाना᳚म् दे॒वानाꣳ॑ हव्य॒वाह॑नो॒ ऽग्नयो॒ ऽग्नयो॑ हव्य॒वाह॑नो दे॒वाना᳚म् । \newline
44. ह॒व्य॒वाह॑नो दे॒वाना᳚म् दे॒वानाꣳ॑ हव्य॒वाह॑नो हव्य॒वाह॑नो दे॒वाना᳚म् कव्य॒वाह॑नः कव्य॒वाह॑नो दे॒वानाꣳ॑ हव्य॒वाह॑नो हव्य॒वाह॑नो दे॒वाना᳚म् कव्य॒वाह॑नः । \newline
45. ह॒व्य॒वाह॑न॒ इति॑ हव्य - वाह॑नः । \newline
46. दे॒वाना᳚म् कव्य॒वाह॑नः कव्य॒वाह॑नो दे॒वाना᳚म् दे॒वाना᳚म् कव्य॒वाह॑नः पितृ॒णाम् पि॑तृ॒णाम् क॑व्य॒वाह॑नो दे॒वाना᳚म् दे॒वाना᳚म् कव्य॒वाह॑नः पितृ॒णाम् । \newline
47. क॒व्य॒वाह॑नः पितृ॒णाम् पि॑तृ॒णाम् क॑व्य॒वाह॑नः कव्य॒वाह॑नः पितृ॒णाꣳ स॒हर॑क्षाः स॒हर॑क्षाः पितृ॒णाम् क॑व्य॒वाह॑नः कव्य॒वाह॑नः पितृ॒णाꣳ स॒हर॑क्षाः । \newline
48. क॒व्य॒वाह॑न॒ इति॑ कव्य - वाह॑नः । \newline
49. पि॒तृ॒णाꣳ स॒हर॑क्षाः स॒हर॑क्षाः पितृ॒णाम् पि॑तृ॒णाꣳ स॒हर॑क्षा॒ असु॑राणा॒ मसु॑राणाꣳ स॒हर॑क्षाः पितृ॒णाम् पि॑तृ॒णाꣳ स॒हर॑क्षा॒ असु॑राणाम् । \newline
50. स॒हर॑क्षा॒ असु॑राणा॒ मसु॑राणाꣳ स॒हर॑क्षाः स॒हर॑क्षा॒ असु॑राणा॒म् ते ते ऽसु॑राणाꣳ स॒हर॑क्षाः स॒हर॑क्षा॒ असु॑राणा॒म् ते । \newline
51. स॒हर॑क्षा॒ इति॑ स॒ह - र॒क्षाः॒ । \newline
52. असु॑राणा॒म् ते ते ऽसु॑राणा॒ मसु॑राणा॒म् त ए॒तर् ह्ये॒तर्.हि॒ ते ऽसु॑राणा॒ मसु॑राणा॒म् त ए॒तर्.हि॑ । \newline
53. त ए॒तर् ह्ये॒तर्.हि॒ ते त ए॒तर् ह्यैतर्.हि॒ ते त ए॒तर् ह्या । \newline
54. ए॒तर् ह्यैतर् ह्ये॒तर् ह्या शꣳ॑सन्ते शꣳसन्त॒ ऐतर् ह्ये॒तर् ह्या शꣳ॑सन्ते । \newline
55. आ शꣳ॑सन्ते शꣳसन्त॒ आ शꣳ॑सन्ते॒ माम् माꣳ शꣳ॑सन्त॒ आ शꣳ॑सन्ते॒ माम् । \newline
56. शꣳ॒॒स॒न्ते॒ माम् माꣳ शꣳ॑सन्ते शꣳसन्ते॒ मां ॅव॑रिष्यते वरिष्यते॒ माꣳ शꣳ॑सन्ते शꣳसन्ते॒ मां ॅव॑रिष्यते । \newline
57. मां ॅव॑रिष्यते वरिष्यते॒ माम् मां ॅव॑रिष्यते॒ माम् मां ॅव॑रिष्यते॒ माम् मां ॅव॑रिष्यते॒ माम् । \newline
58. व॒रि॒ष्य॒ते॒ माम् मां ॅव॑रिष्यते वरिष्यते॒ मा मितीति॒ मां ॅव॑रिष्यते वरिष्यते॒ मा मिति॑ । \newline
59. मा मितीति॒ माम् मा मिति॑ वृणी॒द्ध्वं ॅवृ॑णी॒द्ध्व मिति॒ माम् मा मिति॑ वृणी॒द्ध्वम् । \newline
\pagebreak
\markright{ TS 2.5.8.7  \hfill https://www.vedavms.in \hfill}
\addcontentsline{toc}{section}{ TS 2.5.8.7 }
\section*{ TS 2.5.8.7 }

\textbf{TS 2.5.8.7 } \newline
\textbf{Samhita Paata} \newline

-मिति॑ वृणी॒द्ध्वꣳ ह॑व्य॒वाह॑न॒मित्या॑ह॒ य ए॒व दे॒वानां॒ तं ॅवृ॑णीत आर्.षे॒यं ॅवृ॑णीते॒ बन्धो॑रे॒व नैत्यथो॒ संत॑त्यै प॒रस्ता॑द॒र्वाचो॑ वृणीते॒ तस्मा᳚त् प॒रस्ता॑द॒र्वाञ्चो॑ मनु॒ष्या᳚न् पि॒तरोऽनु॒ प्र पि॑पते ॥ \newline

\textbf{Pada Paata} \newline

इति॑ । वृ॒णी॒द्ध्वम् । ह॒व्य॒वाह॑न॒मिति॑ हव्य - वाह॑नम् । इति॑ । आ॒ह॒ । यः । ए॒व । दे॒वाना᳚म् । तम् । वृ॒णी॒ते॒ । आ॒र॒.षे॒यम् । वृ॒णी॒ते॒ । बन्धोः᳚ । ए॒व । न । ए॒ति॒ । अथो॒ इति॑ । संत॑त्या॒ इति॒ सं - त॒त्यै॒ । प॒रस्ता᳚त् । अ॒र्वाचः॑ । वृ॒णी॒ते॒ । तस्मा᳚त् । प॒रस्ता᳚त् । अ॒र्वाञ्चः॑ । म॒नु॒ष्यान्॑ । पि॒तरः॑ । अनु॑ । प्रेति॑ । पि॒प॒ते॒ ॥  \newline


\textbf{Krama Paata} \newline

इति॑ वृणी॒द्ध्वम् । वृ॒णी॒द्ध्वꣳ ह॑व्य॒वाह॑नम् । ह॒व्य॒वाह॑न॒मिति॑ । ह॒व्य॒वाह॑न॒मिति॑ हव्य - वाह॑नम् । इत्या॑ह । आ॒ह॒ यः । य ए॒व । ए॒व दे॒वाना᳚म् । दे॒वाना॒म् तम् । तम् ॅवृ॑णीते । वृ॒णी॒त॒ आ॒र्॒.षे॒यम् । आ॒र्॒.षे॒यम् ॅवृ॑णीते । वृ॒णी॒ते॒ बन्धोः᳚ । बन्धो॑रे॒व । ए॒व न । नैति॑ । ए॒त्यथो᳚ । अथो॒ सन्त॑त्यै । अथो॒ इत्यथो᳚ । सन्त॑त्यै प॒रस्ता᳚त् । सन्त॑त्या॒ इति॒ सम् - त॒त्यै॒ । प॒रस्ता॑द॒र्वाचः॑ । अ॒र्वाचो॑ वृणीते । वृ॒णी॒ते॒ तस्मा᳚त् । तस्मा᳚त् प॒रस्ता᳚त् । प॒रस्ता॑द॒र्वाञ्चः॑ । अ॒र्वाञ्चो॑ मनु॒ष्यान्॑ । म॒नु॒ष्या᳚न् पि॒तरः॑ । पि॒तरोऽनु॑ । अनु॒ प्र । प्र पि॑पते । पि॒प॒त॒ इति॑ पिपते । \newline

\textbf{Jatai Paata} \newline

1. इति॑ वृणी॒द्ध्वं ॅवृ॑णी॒द्ध्व मितीति॑ वृणी॒द्ध्वम् । \newline
2. वृ॒णी॒द्ध्वꣳ ह॑व्य॒वाह॑नꣳ हव्य॒वाह॑नं ॅवृणी॒द्ध्वं ॅवृ॑णी॒द्ध्वꣳ ह॑व्य॒वाह॑नम् । \newline
3. ह॒व्य॒वाह॑न॒ मितीति॑ हव्य॒वाह॑नꣳ हव्य॒वाह॑न॒ मिति॑ । \newline
4. ह॒व्य॒वाह॑न॒मिति॑ हव्य - वाह॑नम् । \newline
5. इत्या॑हा॒हे तीत्या॑ह । \newline
6. आ॒ह॒ यो य आ॑हाह॒ यः । \newline
7. य ए॒वैव यो य ए॒व । \newline
8. ए॒व दे॒वाना᳚म् दे॒वाना॑ मे॒वैव दे॒वाना᳚म् । \newline
9. दे॒वाना॒म् तम् तम् दे॒वाना᳚म् दे॒वाना॒म् तम् । \newline
10. तं ॅवृ॑णीते वृणीते॒ तम् तं ॅवृ॑णीते । \newline
11. वृ॒णी॒त॒ आ॒र्॒.षे॒य मा॑र.षे॒यं ॅवृ॑णीते वृणीत आर्.षे॒यम् । \newline
12. आ॒र्॒.षे॒यं ॅवृ॑णीते वृणीत आर्.षे॒य मा॑र्.षे॒यं ॅवृ॑णीते । \newline
13. वृ॒णी॒ते॒ बन्धो॒र् बन्धो᳚र् वृणीते वृणीते॒ बन्धोः᳚ । \newline
14. बन्धो॑ रे॒वैव बन्धो॒र् बन्धो॑ रे॒व । \newline
15. ए॒व न नैवैव न । \newline
16. नैत्ये॑ति॒ न नैति॑ । \newline
17. ए॒त्यथो॒ अथो॑ एत्ये॒त्यथो᳚ । \newline
18. अथो॒ सन्त॑त्यै॒ सन्त॑त्या॒ अथो॒ अथो॒ सन्त॑त्यै । \newline
19. अथो॒ इत्यथो᳚ । \newline
20. सन्त॑त्यै प॒रस्ता᳚त् प॒रस्ता॒थ् सन्त॑त्यै॒ सन्त॑त्यै प॒रस्ता᳚त् । \newline
21. सन्त॑त्या॒ इति॒ सं - त॒त्यै॒ । \newline
22. प॒रस्ता॑ द॒र्वाचो॒ ऽर्वाचः॑ प॒रस्ता᳚त् प॒रस्ता॑ द॒र्वाचः॑ । \newline
23. अ॒र्वाचो॑ वृणीते वृणीते॒ ऽर्वाचो॒ ऽर्वाचो॑ वृणीते । \newline
24. वृ॒णी॒ते॒ तस्मा॒त् तस्मा᳚द् वृणीते वृणीते॒ तस्मा᳚त् । \newline
25. तस्मा᳚त् प॒रस्ता᳚त् प॒रस्ता॒त् तस्मा॒त् तस्मा᳚त् प॒रस्ता᳚त् । \newline
26. प॒रस्ता॑ द॒र्वाञ्चो॒ ऽर्वाञ्चः॑ प॒रस्ता᳚त् प॒रस्ता॑ द॒र्वाञ्चः॑ । \newline
27. अ॒र्वाञ्चो॑ मनु॒ष्या᳚न् मनु॒ष्या॑ न॒र्वाञ्चो॒ ऽर्वाञ्चो॑ मनु॒ष्यान्॑ । \newline
28. म॒नु॒ष्या᳚न् पि॒तरः॑ पि॒तरो॑ मनु॒ष्या᳚न् मनु॒ष्या᳚न् पि॒तरः॑ । \newline
29. पि॒तरो ऽन्वनु॑ पि॒तरः॑ पि॒तरो ऽनु॑ । \newline
30. अनु॒ प्र प्राण्वनु॒ प्र । \newline
31. प्र पि॑पते पिपते॒ प्र प्र पि॑पते । \newline
32. पि॒प॒त इति॑ पिपते । \newline

\textbf{Ghana Paata } \newline

1. इति॑ वृणी॒द्ध्वं ॅवृ॑णी॒द्ध्व मितीति॑ वृणी॒द्ध्वꣳ ह॑व्य॒वाह॑नꣳ हव्य॒वाह॑नं ॅवृणी॒द्ध्व मितीति॑ वृणी॒द्ध्वꣳ ह॑व्य॒वाह॑नम् । \newline
2. वृ॒णी॒द्ध्वꣳ ह॑व्य॒वाह॑नꣳ हव्य॒वाह॑नं ॅवृणी॒द्ध्वं ॅवृ॑णी॒द्ध्वꣳ ह॑व्य॒वाह॑न॒ मितीति॑ हव्य॒वाह॑नं ॅवृणी॒द्ध्वं ॅवृ॑णी॒द्ध्वꣳ ह॑व्य॒वाह॑न॒ मिति॑ । \newline
3. ह॒व्य॒वाह॑न॒ मितीति॑ हव्य॒वाह॑नꣳ हव्य॒वाह॑न॒ मित्या॑हा॒हे ति॑ हव्य॒वाह॑नꣳ हव्य॒वाह॑न॒ मित्या॑ह । \newline
4. ह॒व्य॒वाह॑न॒मिति॑ हव्य - वाह॑नम् । \newline
5. इत्या॑हा॒हे तीत्या॑ह॒ यो य आ॒हे तीत्या॑ह॒ यः । \newline
6. आ॒ह॒ यो य आ॑हाह॒ य ए॒वैव य आ॑हाह॒ य ए॒व । \newline
7. य ए॒वैव यो य ए॒व दे॒वाना᳚म् दे॒वाना॑ मे॒व यो य ए॒व दे॒वाना᳚म् । \newline
8. ए॒व दे॒वाना᳚म् दे॒वाना॑ मे॒वैव दे॒वाना॒म् तम् तम् दे॒वाना॑ मे॒वैव दे॒वाना॒म् तम् । \newline
9. दे॒वाना॒म् तम् तम् दे॒वाना᳚म् दे॒वाना॒म् तं ॅवृ॑णीते वृणीते॒ तम् दे॒वाना᳚म् दे॒वाना॒म् तं ॅवृ॑णीते । \newline
10. तं ॅवृ॑णीते वृणीते॒ तम् तं ॅवृ॑णीत आर.षे॒य मा॑र.षे॒यं ॅवृ॑णीते॒ तम् तं ॅवृ॑णीत आर.षे॒यम् । \newline
11. वृ॒णी॒त॒ आ॒र॒.षे॒य मा॑र.षे॒यं ॅवृ॑णीते वृणीत आर.षे॒यं ॅवृ॑णीते वृणीत आर.षे॒यं ॅवृ॑णीते वृणीत आर्.षे॒यं ॅवृ॑णीते । \newline
12. आ॒र॒.षे॒यं ॅवृ॑णीते वृणीत आर.षे॒य मा॑र.षे॒यं ॅवृ॑णीते॒ बन्धो॒र् बन्धो᳚र् वृणीत आर.षे॒य मा॑र.षे॒यं ॅवृ॑णीते॒ बन्धोः᳚ । \newline
13. वृ॒णी॒ते॒ बन्धो॒र् बन्धो᳚र् वृणीते वृणीते॒ बन्धो॑ रे॒वैव बन्धो᳚र् वृणीते वृणीते॒ बन्धो॑रे॒व । \newline
14. बन्धो॑ रे॒वैव बन्धो॒र् बन्धो॑रे॒व न नैव बन्धो॒र् बन्धो॑रे॒व न । \newline
15. ए॒व न नैवैव नैत्ये॑ति॒ नैवैव नैति॑ । \newline
16. नैत्ये॑ति॒ न नैत्यथो॒ अथो॑ एति॒ न नैत्यथो᳚ । \newline
17. ए॒त्यथो॒ अथो॑ एत्ये॒त्यथो॒ सन्त॑त्यै॒ सन्त॑त्या॒ अथो॑ एत्ये॒त्यथो॒ सन्त॑त्यै । \newline
18. अथो॒ सन्त॑त्यै॒ सन्त॑त्या॒ अथो॒ अथो॒ सन्त॑त्यै प॒रस्ता᳚त् प॒रस्ता॒थ् सन्त॑त्या॒ अथो॒ अथो॒ सन्त॑त्यै प॒रस्ता᳚त् । \newline
19. अथो॒ इत्यथो᳚ । \newline
20. सन्त॑त्यै प॒रस्ता᳚त् प॒रस्ता॒थ् सन्त॑त्यै॒ सन्त॑त्यै प॒रस्ता॑ द॒र्वाचो॒ ऽर्वाचः॑ प॒रस्ता॒थ् सन्त॑त्यै॒ सन्त॑त्यै प॒रस्ता॑ द॒र्वाचः॑ । \newline
21. सन्त॑त्या॒ इति॒ सं - त॒त्यै॒ । \newline
22. प॒रस्ता॑ द॒र्वाचो॒ ऽर्वाचः॑ प॒रस्ता᳚त् प॒रस्ता॑ द॒र्वाचो॑ वृणीते वृणीते॒ ऽर्वाचः॑ प॒रस्ता᳚त् प॒रस्ता॑ द॒र्वाचो॑ वृणीते । \newline
23. अ॒र्वाचो॑ वृणीते वृणीते॒ ऽर्वाचो॒ ऽर्वाचो॑ वृणीते॒ तस्मा॒त् तस्मा᳚द् वृणीते॒ ऽर्वाचो॒ ऽर्वाचो॑ वृणीते॒ तस्मा᳚त् । \newline
24. वृ॒णी॒ते॒ तस्मा॒त् तस्मा᳚द् वृणीते वृणीते॒ तस्मा᳚त् प॒रस्ता᳚त् प॒रस्ता॒त् तस्मा᳚द् वृणीते वृणीते॒ तस्मा᳚त् प॒रस्ता᳚त् । \newline
25. तस्मा᳚त् प॒रस्ता᳚त् प॒रस्ता॒त् तस्मा॒त् तस्मा᳚त् प॒रस्ता॑ द॒र्वाञ्चो॒ ऽर्वाञ्चः॑ प॒रस्ता॒त् तस्मा॒त् तस्मा᳚त् प॒रस्ता॑ द॒र्वाञ्चः॑ । \newline
26. प॒रस्ता॑ द॒र्वाञ्चो॒ ऽर्वाञ्चः॑ प॒रस्ता᳚त् प॒रस्ता॑ द॒र्वाञ्चो॑ मनु॒ष्या᳚न् मनु॒ष्या॑ न॒र्वाञ्चः॑ प॒रस्ता᳚त् प॒रस्ता॑ द॒र्वाञ्चो॑ मनु॒ष्यान्॑ । \newline
27. अ॒र्वाञ्चो॑ मनु॒ष्या᳚न् मनु॒ष्या॑ न॒र्वाञ्चो॒ ऽर्वाञ्चो॑ मनु॒ष्या᳚न् पि॒तरः॑ पि॒तरो॑ मनु॒ष्या॑ न॒र्वाञ्चो॒ ऽर्वाञ्चो॑ मनु॒ष्या᳚न् पि॒तरः॑ । \newline
28. म॒नु॒ष्या᳚न् पि॒तरः॑ पि॒तरो॑ मनु॒ष्या᳚न् मनु॒ष्या᳚न् पि॒तरो ऽन्वनु॑ पि॒तरो॑ मनु॒ष्या᳚न् मनु॒ष्या᳚न् पि॒तरो ऽनु॑ । \newline
29. पि॒तरो ऽन्वनु॑ पि॒तरः॑ पि॒तरो ऽनु॒ प्र प्राणु॑ पि॒तरः॑ पि॒तरो ऽनु॒ प्र । \newline
30. अनु॒ प्र प्राण्वनु॒ प्र पि॑पते पिपते॒ प्राण्वनु॒ प्र पि॑पते । \newline
31. प्र पि॑पते पिपते॒ प्र प्र पि॑पते । \newline
32. पि॒प॒त इति॑ पिपते । \newline
\pagebreak
\markright{ TS 2.5.9.1  \hfill https://www.vedavms.in \hfill}
\addcontentsline{toc}{section}{ TS 2.5.9.1 }
\section*{ TS 2.5.9.1 }

\textbf{TS 2.5.9.1 } \newline
\textbf{Samhita Paata} \newline

अग्ने॑ म॒हाꣳ अ॒सीत्या॑ह म॒हान्. ह्ये॑ष यद॒ग्नि र्ब्रा᳚ह्म॒णेत्या॑ह ब्राह्म॒णो ह्ये॑ष भा॑र॒तेत्या॑है॒ष हि दे॒वेभ्यो॑ ह॒व्यं भर॑ति दे॒वेद्ध॒ इत्या॑ह दे॒वा ह्ये॑तमैन्ध॑त॒ मन्वि॑द्ध॒ इत्या॑ह॒ मनु॒र्ह्ये॑तमुत्त॑रो दे॒वेभ्य॒ ऐन्धर्.षि॑ष्टुत॒ इत्या॒हर्.ष॑यो॒ ह्ये॑तमस्तु॑व॒न् विप्रा॑नुमदित॒ इत्या॑ह॒ - [  ] \newline

\textbf{Pada Paata} \newline

अग्ने᳚ । म॒हान् । अ॒सि॒ । इति॑ । आ॒ह॒ । म॒हान् । हि । ए॒षः । यत् । अ॒ग्निः । ब्रा॒ह्म॒ण॒ । इति॑ । आ॒ह॒ । ब्रा॒ह्म॒णः । हि । ए॒षः । भा॒र॒त॒ । इति॑ । आ॒ह॒ । ए॒षः । हि । दे॒वेभ्यः॑ । ह॒व्यम् । भर॑ति । दे॒वेद्ध॒ इति॑ दे॒व - इ॒द्धः॒ । इति॑ । आ॒ह॒ । दे॒वाः । हि । ए॒तम् । ऐन्ध॑त । मन्वि॑द्ध॒ इति॒ मनु॑ - इ॒द्धः॒ । इति॑ । आ॒ह॒ । मनुः॑ । हि । ए॒तम् । उत्त॑र॒ इत्युत् - त॒रः॒ । दे॒वेभ्यः॑ । ऐन्ध॑ । ऋषि॑ष्टुत॒ इत्यृषि॑ - स्तु॒तः॒ । इति॑ । आ॒ह॒ । ऋष॑यः । हि । ए॒तम् । अस्तु॑वन्न् । विप्रा॑नुमदित॒ इति॒ विप्र॑ - अ॒नु॒म॒दि॒तः॒ । इति॑ । आ॒ह॒ ।  \newline


\textbf{Krama Paata} \newline

अग्ने॑ म॒हान् । म॒हाꣳ अ॑सि । अ॒सीति॑ । इत्या॑ह । आ॒ह॒ म॒हान् । म॒हान्. हि । ह्ये॑षः । ए॒ष यत् । यद॒ग्निः । अ॒ग्निर् ब्रा᳚ह्मण । ब्रा॒ह्म॒णेति॑ । इत्या॑ह । आ॒ह॒ ब्रा॒ह्म॒णः । ब्रा॒ह्म॒णो हि । ह्ये॑षः । ए॒ष भा॑रत । भा॒र॒तेति॑ । इत्या॑ह । आ॒है॒षः । ए॒ष हि । हि दे॒वेभ्यः॑ । दे॒वेभ्यो॑ ह॒व्यम् । ह॒व्यम् भर॑ति । भर॑ति दे॒वेद्धः॑ । दे॒वेद्ध॒ इति॑ । दे॒वेद्ध॒ इति॑ दे॒व - इ॒द्धः॒ । इत्या॑ह । आ॒ह॒ दे॒वाः । दे॒वा हि । ह्ये॑तम् । ए॒तमैन्ध॑त । ऐन्ध॑त॒ मन्वि॑द्धः । मन्वि॑द्ध॒ इति॑ । मन्वि॑द्ध॒ इति॒ मनु॑ - इ॒द्धः॒ । इत्या॑ह । आ॒ह॒ मनुः॑ । मनु॒र्. हि । ह्ये॑तम् । ए॒तमुत्त॑रः । उत्त॑रो दे॒वेभ्यः॑ । उत्त॑र॒ इत्युत् - त॒रः॒ । दे॒वेभ्य॒ ऐन्ध॑ । ऐन्धर्.षि॑ष्टुतः । ऋषि॑ष्टुत॒ इति॑ । ऋषि॑ष्टुत॒ इत्यृषि॑ - स्तु॒तः॒ । इत्या॑ह । आ॒हर्.ष॑यः । ऋष॑यो॒ हि । ह्ये॑तम् । ए॒तमस्तु॑वन्न् । अस्तु॑व॒न् विप्रा॑नुमदितः । विप्रा॑नुमदित॒ इति॑ । विप्रा॑नुमदित॒ इति॒ विप्र॑ - अ॒नु॒म॒दि॒तः॒ । इत्या॑ह । आ॒ह॒ विप्राः᳚ \newline

\textbf{Jatai Paata} \newline

1. अग्ने॑ म॒हान् म॒हाꣳ अग्ने ऽग्ने॑ म॒हान् । \newline
2. म॒हाꣳ अ॑स्यसि म॒हान् म॒हाꣳ अ॑सि । \newline
3. अ॒सीती त्य॑स्य॒सीति॑ । \newline
4. इत्या॑हा॒हे तीत्या॑ह । \newline
5. आ॒ह॒ म॒हान् म॒हा ना॑हाह म॒हान् । \newline
6. म॒हान्. हि हि म॒हान् म॒हान्. हि । \newline
7. ह्ये॑ष ए॒ष हि ह्ये॑षः । \newline
8. ए॒ष यद् यदे॒ष ए॒ष यत् । \newline
9. यद॒ग्नि र॒ग्निर् यद् यद॒ग्निः । \newline
10. अ॒ग्निर् ब्रा᳚ह्मण ब्राह्मणा॒ग्नि र॒ग्निर् ब्रा᳚ह्मण । \newline
11. ब्रा॒ह्म॒णे तीति॑ ब्राह्मण ब्राह्म॒णे ति॑ । \newline
12. इत्या॑हा॒हे तीत्या॑ह । \newline
13. आ॒ह॒ ब्रा॒ह्म॒णो ब्रा᳚ह्म॒ण आ॑हाह ब्राह्म॒णः । \newline
14. ब्रा॒ह्म॒णो हि हि ब्रा᳚ह्म॒णो ब्रा᳚ह्म॒णो हि । \newline
15. ह्ये॑ष ए॒ष हि ह्ये॑षः । \newline
16. ए॒ष भा॑रत भारतै॒ष ए॒ष भा॑रत । \newline
17. भा॒र॒ते तीति॑ भारत भार॒ते ति॑ । \newline
18. इत्या॑हा॒हे तीत्या॑ह । \newline
19. आ॒है॒ष ए॒ष आ॑हाहै॒षः । \newline
20. ए॒ष हि ह्ये॑ष ए॒ष हि । \newline
21. हि दे॒वेभ्यो॑ दे॒वेभ्यो॒ हि हि दे॒वेभ्यः॑ । \newline
22. दे॒वेभ्यो॑ ह॒व्यꣳ ह॒व्यम् दे॒वेभ्यो॑ दे॒वेभ्यो॑ ह॒व्यम् । \newline
23. ह॒व्यम् भर॑ति॒ भर॑ति ह॒व्यꣳ ह॒व्यम् भर॑ति । \newline
24. भर॑ति दे॒वेद्धो॑ दे॒वेद्धो॒ भर॑ति॒ भर॑ति दे॒वेद्धः॑ । \newline
25. दे॒वेद्ध॒ इतीति॑ दे॒वेद्धो॑ दे॒वेद्ध॒ इति॑ । \newline
26. दे॒वेद्ध॒ इति॑ दे॒व - इ॒द्धः॒ । \newline
27. इत्या॑हा॒हे तीत्या॑ह । \newline
28. आ॒ह॒ दे॒वा दे॒वा आ॑हाह दे॒वाः । \newline
29. दे॒वा हि हि दे॒वा दे॒वा हि । \newline
30. ह्ये॑त मे॒तꣳ हि ह्ये॑तम् । \newline
31. ए॒त मैन्ध॒ तैन्ध॑तै॒त मे॒त मैन्ध॑त । \newline
32. ऐन्ध॑त॒ मन्वि॑द्धो॒ मन्वि॑द्ध॒ ऐन्ध॒ तैन्ध॑त॒ मन्वि॑द्धः । \newline
33. मन्वि॑द्ध॒ इतीति॒ मन्वि॑द्धो॒ मन्वि॑द्ध॒ इति॑ । \newline
34. मन्वि॑द्ध॒ इति॒ मनु॑ - इ॒द्धः॒ । \newline
35. इत्या॑हा॒हे तीत्या॑ह । \newline
36. आ॒ह॒ मनु॒र् मनु॑ राहाह॒ मनुः॑ । \newline
37. मनु॒र्॒. हि हि मनु॒र् मनु॒र्॒. हि । \newline
38. ह्ये॑त मे॒तꣳ हि ह्ये॑तम् । \newline
39. ए॒त मुत्त॑र॒ उत्त॑र ए॒त मे॒त मुत्त॑रः । \newline
40. उत्त॑रो दे॒वेभ्यो॑ दे॒वेभ्य॒ उत्त॑र॒ उत्त॑रो दे॒वेभ्यः॑ । \newline
41. उत्त॑र॒ इत्युत् - त॒रः॒ । \newline
42. दे॒वेभ्य॒ ऐन्धैन्ध॑ दे॒वेभ्यो॑ दे॒वेभ्य॒ ऐन्ध॑ । \newline
43. ऐन्ध र्.षि॑ष्टुत॒ ऋषि॑ष्टुत॒ ऐन्धैन्ध र्.षि॑ष्टुतः । \newline
44. ऋषि॑ष्टुत॒ इती त्यृषि॑ष्टुत॒ ऋषि॑ष्टुत॒ इति॑ । \newline
45. ऋषि॑ष्टुत॒ इत्यृषि॑ - स्तु॒तः॒ । \newline
46. इत्या॑हा॒हे तीत्या॑ह । \newline
47. आ॒ह र्.ष॑य॒ ऋष॑य आहा॒ह र्.ष॑यः । \newline
48. ऋष॑यो॒ हि ह्यृष॑य॒ ऋष॑यो॒ हि । \newline
49. ह्ये॑त मे॒तꣳ हि ह्ये॑तम् । \newline
50. ए॒त मस्तु॑व॒न् नस्तु॑वन् ने॒त मे॒त मस्तु॑वन्न् । \newline
51. अस्तु॑व॒न् विप्रा॑नुमदितो॒ विप्रा॑नुमदि॒तो ऽस्तु॑व॒न् नस्तु॑व॒न् विप्रा॑नुमदितः । \newline
52. विप्रा॑नुमदित॒ इतीति॒ विप्रा॑नुमदितो॒ विप्रा॑नुमदित॒ इति॑ । \newline
53. विप्रा॑नुमदित॒ इति॒ विप्र॑ - अ॒नु॒म॒दि॒तः॒ । \newline
54. इत्या॑हा॒हे तीत्या॑ह । \newline
55. आ॒ह॒ विप्रा॒ विप्रा॑ आहाह॒ विप्राः᳚ । \newline

\textbf{Ghana Paata } \newline

1. अग्ने॑ म॒हान् म॒हाꣳ अग्ने ऽग्ने॑ म॒हाꣳ अ॑स्यसि म॒हाꣳ अग्ने ऽग्ने॑ म॒हाꣳ अ॑सि । \newline
2. म॒हाꣳ अ॑स्यसि म॒हान् म॒हाꣳ अ॒सीती त्य॑सि म॒हान् म॒हाꣳ अ॒सीति॑ । \newline
3. अ॒सीती त्य॑स्य॒सी त्या॑हा॒हे त्य॑स्य॒सीत्या॑ह । \newline
4. इत्या॑हा॒हे तीत्या॑ह म॒हान् म॒हा ना॒हे तीत्या॑ह म॒हान् । \newline
5. आ॒ह॒ म॒हान् म॒हा ना॑हाह म॒हान्. हि हि म॒हा ना॑हाह म॒हान्. हि । \newline
6. म॒हान्. हि हि म॒हान् म॒हान् ह्ये॑ष ए॒ष हि म॒हान् म॒हान् ह्ये॑षः । \newline
7. ह्ये॑ष ए॒ष हि ह्ये॑ष यद् यदे॒ष हि ह्ये॑ष यत् । \newline
8. ए॒ष यद् यदे॒ष ए॒ष यद॒ग्नि र॒ग्निर् यदे॒ष ए॒ष यद॒ग्निः । \newline
9. यद॒ग्नि र॒ग्निर् यद् यद॒ग्निर् ब्रा᳚ह्मण ब्राह्मणा॒ग्निर् यद् यद॒ग्निर् ब्रा᳚ह्मण । \newline
10. अ॒ग्निर् ब्रा᳚ह्मण ब्राह्मणा॒ग्नि र॒ग्निर् ब्रा᳚ह्म॒णे तीति॑ ब्राह्मणा॒ग्नि र॒ग्निर् ब्रा᳚ह्म॒णे ति॑ । \newline
11. ब्रा॒ह्म॒णे तीति॑ ब्राह्मण ब्राह्म॒णे त्या॑हा॒हे ति॑ ब्राह्मण ब्राह्म॒णे त्या॑ह । \newline
12. इत्या॑हा॒हे तीत्या॑ह ब्राह्म॒णो ब्रा᳚ह्म॒ण आ॒हे तीत्या॑ह ब्राह्म॒णः । \newline
13. आ॒ह॒ ब्रा॒ह्म॒णो ब्रा᳚ह्म॒ण आ॑हाह ब्राह्म॒णो हि हि ब्रा᳚ह्म॒ण आ॑हाह ब्राह्म॒णो हि । \newline
14. ब्रा॒ह्म॒णो हि हि ब्रा᳚ह्म॒णो ब्रा᳚ह्म॒णो ह्ये॑ष ए॒ष हि ब्रा᳚ह्म॒णो ब्रा᳚ह्म॒णो ह्ये॑षः । \newline
15. ह्ये॑ष ए॒ष हि ह्ये॑ष भा॑रत भारतै॒ष हि ह्ये॑ष भा॑रत । \newline
16. ए॒ष भा॑रत भारतै॒ष ए॒ष भा॑र॒ते तीति॑ भारतै॒ष ए॒ष भा॑र॒ते ति॑ । \newline
17. भा॒र॒ते तीति॑ भारत भार॒ते त्या॑हा॒हे ति॑ भारत भार॒ते त्या॑ह । \newline
18. इत्या॑हा॒हे तीत्या॑है॒ष ए॒ष आ॒हे तीत्या॑है॒षः । \newline
19. आ॒है॒ष ए॒ष आ॑हाहै॒ष हि ह्ये॑ष आ॑हाहै॒ष हि । \newline
20. ए॒ष हि ह्ये॑ष ए॒ष हि दे॒वेभ्यो॑ दे॒वेभ्यो॒ ह्ये॑ष ए॒ष हि दे॒वेभ्यः॑ । \newline
21. हि दे॒वेभ्यो॑ दे॒वेभ्यो॒ हि हि दे॒वेभ्यो॑ ह॒व्यꣳ ह॒व्यम् दे॒वेभ्यो॒ हि हि दे॒वेभ्यो॑ ह॒व्यम् । \newline
22. दे॒वेभ्यो॑ ह॒व्यꣳ ह॒व्यम् दे॒वेभ्यो॑ दे॒वेभ्यो॑ ह॒व्यम् भर॑ति॒ भर॑ति ह॒व्यम् दे॒वेभ्यो॑ दे॒वेभ्यो॑ ह॒व्यम् भर॑ति । \newline
23. ह॒व्यम् भर॑ति॒ भर॑ति ह॒व्यꣳ ह॒व्यम् भर॑ति दे॒वेद्धो॑ दे॒वेद्धो॒ भर॑ति ह॒व्यꣳ ह॒व्यम् भर॑ति दे॒वेद्धः॑ । \newline
24. भर॑ति दे॒वेद्धो॑ दे॒वेद्धो॒ भर॑ति॒ भर॑ति दे॒वेद्ध॒ इतीति॑ दे॒वेद्धो॒ भर॑ति॒ भर॑ति दे॒वेद्ध॒ इति॑ । \newline
25. दे॒वेद्ध॒ इतीति॑ दे॒वेद्धो॑ दे॒वेद्ध॒ इत्या॑हा॒हे ति॑ दे॒वेद्धो॑ दे॒वेद्ध॒ इत्या॑ह । \newline
26. दे॒वेद्ध॒ इति॑ दे॒व - इ॒द्धः॒ । \newline
27. इत्या॑हा॒हे तीत्या॑ह दे॒वा दे॒वा आ॒हे तीत्या॑ह दे॒वाः । \newline
28. आ॒ह॒ दे॒वा दे॒वा आ॑हाह दे॒वा हि हि दे॒वा आ॑हाह दे॒वा हि । \newline
29. दे॒वा हि हि दे॒वा दे॒वा ह्ये॑त मे॒तꣳ हि दे॒वा दे॒वा ह्ये॑तम् । \newline
30. ह्ये॑त मे॒तꣳ हि ह्ये॑त मैन्ध॒ तैन्ध॑ तै॒तꣳ हि ह्ये॑त मैन्ध॑त । \newline
31. ए॒त मैन्ध॒ तैन्ध॑तै॒त मे॒त मैन्ध॑त॒ मन्वि॑द्धो॒ मन्वि॑द्ध॒ ऐन्ध॑तै॒त मे॒त मैन्ध॑त॒ मन्वि॑द्धः । \newline
32. ऐन्ध॑त॒ मन्वि॑द्धो॒ मन्वि॑द्ध॒ ऐन्ध॒तैन्ध॑त॒ मन्वि॑द्ध॒ इतीति॒ मन्वि॑द्ध॒ ऐन्ध॒तैन्ध॑त॒ मन्वि॑द्ध॒ इति॑ । \newline
33. मन्वि॑द्ध॒ इतीति॒ मन्वि॑द्धो॒ मन्वि॑द्ध॒ इत्या॑हा॒हे ति॒ मन्वि॑द्धो॒ मन्वि॑द्ध॒ इत्या॑ह । \newline
34. मन्वि॑द्ध॒ इति॒ मनु॑ - इ॒द्धः॒ । \newline
35. इत्या॑हा॒हे तीत्या॑ह॒ मनु॒र् मनु॑रा॒हे तीत्या॑ह॒ मनुः॑ । \newline
36. आ॒ह॒ मनु॒र् मनु॑ राहाह॒ मनु॒र्॒. हि हि मनु॑ राहाह॒ मनु॒र्॒. हि । \newline
37. मनु॒र्॒. हि हि मनु॒र् मनु॒र् ह्ये॑त मे॒तꣳ हि मनु॒र् मनु॒र् ह्ये॑तम् । \newline
38. ह्ये॑त मे॒तꣳ हि ह्ये॑त मुत्त॑र॒ उत्त॑र ए॒तꣳ हि ह्ये॑त मुत्त॑रः । \newline
39. ए॒त मुत्त॑र॒ उत्त॑र ए॒त मे॒त मुत्त॑रो दे॒वेभ्यो॑ दे॒वेभ्य॒ उत्त॑र ए॒त मे॒त मुत्त॑रो दे॒वेभ्यः॑ । \newline
40. उत्त॑रो दे॒वेभ्यो॑ दे॒वेभ्य॒ उत्त॑र॒ उत्त॑रो दे॒वेभ्य॒ ऐन्धैन्ध॑ दे॒वेभ्य॒ उत्त॑र॒ उत्त॑रो दे॒वेभ्य॒ ऐन्ध॑ । \newline
41. उत्त॑र॒ इत्युत् - त॒रः॒ । \newline
42. दे॒वेभ्य॒ ऐन्धैन्ध॑ दे॒वेभ्यो॑ दे॒वेभ्य॒ ऐन्ध र्.षि॑ष्टुत॒ ऋषि॑ष्टुत॒ ऐन्ध॑ दे॒वेभ्यो॑ दे॒वेभ्य॒ ऐन्ध र्.षि॑ष्टुतः । \newline
43. ऐन्ध र्.षि॑ष्टुत॒ ऋषि॑ष्टुत॒ ऐन्धैन्ध र्.षि॑ष्टुत॒ इती त्यृषि॑ष्टुत॒ ऐन्धैन्ध र्.षि॑ष्टुत॒ इति॑ । \newline
44. ऋषि॑ष्टुत॒ इतीत्यृषि॑ष्टुत॒ ऋषि॑ष्टुत॒ इत्या॑हा॒हे त्यृषि॑ष्टुत॒ ऋषि॑ष्टुत॒ इत्या॑ह । \newline
45. ऋषि॑ष्टुत॒ इत्यृषि॑ - स्तु॒तः॒ । \newline
46. इत्या॑हा॒हे तीत्या॒ह र्.ष॑य॒ ऋष॑य आ॒हे तीत्या॒ह र्.ष॑यः । \newline
47. आ॒ह र्.ष॑य॒ ऋष॑य आहा॒ह र्.ष॑यो॒ हि ह्यृष॑य आहा॒ह र्.ष॑यो॒ हि । \newline
48. ऋष॑यो॒ हि ह्यृष॑य॒ ऋष॑यो॒ ह्ये॑त मे॒तꣳ ह्यृष॑य॒ ऋष॑यो॒ ह्ये॑तम् । \newline
49. ह्ये॑त मे॒तꣳ हि ह्ये॑त मस्तु॑व॒न् नस्तु॑वन् ने॒तꣳ हि ह्ये॑त मस्तु॑वन्न् । \newline
50. ए॒त मस्तु॑व॒न् नस्तु॑वन् ने॒त मे॒त मस्तु॑व॒न् विप्रा॑नुमदितो॒ विप्रा॑नुमदि॒तो ऽस्तु॑वन् ने॒त मे॒त मस्तु॑व॒न् विप्रा॑नुमदितः । \newline
51. अस्तु॑व॒न् विप्रा॑नुमदितो॒ विप्रा॑नुमदि॒तो ऽस्तु॑व॒न् नस्तु॑व॒न् विप्रा॑नुमदित॒ इतीति॒ विप्रा॑नुमदि॒तो ऽस्तु॑व॒न् नस्तु॑व॒न् विप्रा॑नुमदित॒ इति॑ । \newline
52. विप्रा॑नुमदित॒ इतीति॒ विप्रा॑नुमदितो॒ विप्रा॑नुमदित॒ इत्या॑हा॒हे ति॒ विप्रा॑नुमदितो॒ विप्रा॑नुमदित॒ इत्या॑ह । \newline
53. विप्रा॑नुमदित॒ इति॒ विप्र॑ - अ॒नु॒म॒दि॒तः॒ । \newline
54. इत्या॑हा॒हे तीत्या॑ह॒ विप्रा॒ विप्रा॑ आ॒हे तीत्या॑ह॒ विप्राः᳚ । \newline
55. आ॒ह॒ विप्रा॒ विप्रा॑ आहाह॒ विप्रा॒ हि हि विप्रा॑ आहाह॒ विप्रा॒ हि । \newline
\pagebreak
\markright{ TS 2.5.9.2  \hfill https://www.vedavms.in \hfill}
\addcontentsline{toc}{section}{ TS 2.5.9.2 }
\section*{ TS 2.5.9.2 }

\textbf{TS 2.5.9.2 } \newline
\textbf{Samhita Paata} \newline

विप्रा॒ ह्ये॑ते यच्छु॑श्रु॒वाꣳसः॑ कविश॒स्त इत्या॑ह क॒वयो॒ ह्ये॑ते यच्छु॑श्रु॒वाꣳसो॒ ब्रह्म॑सꣳशित॒ इत्या॑ह॒ ब्रह्म॑सꣳशितो॒ ह्ये॑ष घृ॒ताह॑वन॒ इत्या॑ह घृताहु॒तिर्ह्य॑स्य प्रि॒यत॑मा प्र॒णीर्य॒ज्ञाना॒मित्या॑ह प्र॒णीर्ह्ये॑ष य॒ज्ञानाꣳ॑ र॒थीर॑द्ध्व॒राणा॒मित्या॑है॒ष हि दे॑वर॒थो॑ऽतूर्तो॒ होतेत्या॑ह॒ न ह्ये॑तं कश्च॒न - [  ] \newline

\textbf{Pada Paata} \newline

विप्राः᳚ । हि । ए॒ते । यत् । शु॒श्रु॒वाꣳसः॑ । क॒वि॒श॒स्त इति॑ कवि - श॒स्तः । इति॑ । आ॒ह॒ । क॒वयः॑ । हि । ए॒ते । यत् । शु॒श्रु॒वाꣳसः॑ । ब्रह्म॑सꣳशित॒ इति॒ ब्रह्म॑ - सꣳ॒॒शि॒तः॒ । इति॑ । आ॒ह॒ । ब्रह्म॑सꣳशित॒ इति॒ ब्रह्म॑ - सꣳ॒॒शि॒तः॒ । हि । ए॒षः । घृ॒ताह॑वन॒ इति॑ घृ॒त - आ॒ह॒व॒नः॒ । इति॑ । आ॒ह॒ । घृ॒ता॒हु॒तिरिति॑ घृत-आ॒हु॒तिः । हि । अ॒स्य॒ । प्रि॒यत॒मेति॑ प्रि॒य - त॒मा॒ । प्र॒णीरिति॑ प्र - नीः । य॒ज्ञाना᳚म् । इति॑ । आ॒ह॒ । प्र॒णीरिति॑ प्र-नीः । हि । ए॒षः । य॒ज्ञाना᳚म् । र॒थीः । अ॒द्ध्व॒राणा᳚म् । इति॑ । आ॒ह॒ । ए॒षः । हि । दे॒व॒र॒थ इति॑ देव - र॒थः । अ॒तूर्तः॑ । होता᳚ । इति॑ । आ॒ह॒ । न । हि । ए॒तम् । कः । च॒न ।  \newline


\textbf{Krama Paata} \newline

विप्रा॒ हि । ह्ये॑ते । ए॒ते यत् । यच्छु॑श्रु॒वाꣳसः॑ । शु॒श्रु॒वाꣳसः॑ कविश॒स्तः । क॒वि॒श॒स्त इति॑ । क॒वि॒श॒स्त इति॑ कवि - श॒स्तः । इत्या॑ह । आ॒ह॒ क॒वयः॑ । क॒वयो॒ हि । ह्ये॑ते । ए॒ते यत् । यच्छु॑श्रु॒वाꣳसः॑ । शु॒श्रु॒वाꣳसो॒ ब्रह्म॑सꣳशितः । ब्रह्म॑सꣳशित॒ इति॑ । ब्रह्म॑सꣳशित॒ इति॒ ब्रह्म॑ - सꣳ॒॒शि॒तः॒ । इत्या॑ह । आ॒ह॒ ब्रह्म॑सꣳशितः । ब्रह्म॑सꣳशितो॒ हि । ब्रह्म॑सꣳशित॒ इति॒ ब्रह्म॑ - सꣳ॒॒शि॒तः॒ । ह्ये॑षः । ए॒ष घृ॒ताह॑वनः । घृ॒ताह॑वन॒ इति॑ । घृ॒ताह॑वन॒ इति॑ घृ॒त - आ॒ह॒व॒नः॒ । इत्या॑ह । आ॒ह॒ घृ॒ता॒हु॒तिः । घृ॒ता॒हु॒तिर्. हि । घृ॒ता॒हु॒तिरिति॑ घृत - आ॒हु॒तिः । ह्य॑स्य । अ॒स्य॒ प्रि॒यत॑मा । प्रि॒यात॑मा प्र॒णीः । प्रि॒यत॒मेति॑ प्रि॒य - त॒मा॒ । प्र॒णीर् य॒ज्ञाना᳚म् । प्र॒णीरिति॑ प्र - नीः । य॒ज्ञाना॒मिति॑ । इत्या॑ह । आ॒ह॒ प्र॒णीः । प्र॒णीर्. हि । प्र॒णीरिति॑ प्र - नीः । ह्ये॑षः । ए॒ष य॒ज्ञाना᳚म् । य॒ज्ञानाꣳ॑ र॒थीः । र॒थीर॑द्ध्व॒राणा᳚म् । अ॒द्ध्व॒राणा॒मिति॑ । इत्या॑ह । आ॒है॒षः । ए॒ष हि । हि दे॑वर॒थः । दे॒व॒र॒थो॑ ऽतूर्तः॑ । दे॒व॒र॒थ इति॑ देव - र॒थः । अ॒तूर्तो॒ होता᳚ । होतेति॑ । इत्या॑ह । आ॒ह॒ न । न हि । ह्ये॑तम् । ए॒तम् कः । कश्च॒न । च॒न तर॑ति \newline

\textbf{Jatai Paata} \newline

1. विप्रा॒ हि हि विप्रा॒ विप्रा॒ हि । \newline
2. ह्ये॑त ए॒ते हि ह्ये॑ते । \newline
3. ए॒ते यद् यदे॒त ए॒ते यत् । \newline
4. यच् छु॑श्रु॒वाꣳसः॑ शुश्रु॒वाꣳसो॒ यद् यच् छु॑श्रु॒वाꣳसः॑ । \newline
5. शु॒श्रु॒वाꣳसः॑ कविश॒स्तः क॑विश॒स्तः शु॑श्रु॒वाꣳसः॑ शुश्रु॒वाꣳसः॑ कविश॒स्तः । \newline
6. क॒वि॒श॒स्त इतीति॑ कविश॒स्तः क॑विश॒स्त इति॑ । \newline
7. क॒वि॒श॒स्त इति॑ कवि - श॒स्तः । \newline
8. इत्या॑हा॒हे तीत्या॑ह । \newline
9. आ॒ह॒ क॒वयः॑ क॒वय॑ आहाह क॒वयः॑ । \newline
10. क॒वयो॒ हि हि क॒वयः॑ क॒वयो॒ हि । \newline
11. ह्ये॑त ए॒ते हि ह्ये॑ते । \newline
12. ए॒ते यद् यदे॒त ए॒ते यत् । \newline
13. यच् छु॑श्रु॒वाꣳसः॑ शुश्रु॒वाꣳसो॒ यद् यच् छु॑श्रु॒वाꣳसः॑ । \newline
14. शु॒श्रु॒वाꣳसो॒ ब्रह्म॑सꣳशितो॒ ब्रह्म॑सꣳशितः शुश्रु॒वाꣳसः॑ शुश्रु॒वाꣳसो॒ ब्रह्म॑सꣳशितः । \newline
15. ब्रह्म॑सꣳशित॒ इतीति॒ ब्रह्म॑सꣳशितो॒ ब्रह्म॑सꣳशित॒ इति॑ । \newline
16. ब्रह्म॑सꣳशित॒ इति॒ ब्रह्म॑ - सꣳ॒॒शि॒तः॒ । \newline
17. इत्या॑हा॒हे तीत्या॑ह । \newline
18. आ॒ह॒ ब्रह्म॑सꣳशितो॒ ब्रह्म॑सꣳशित आहाह॒ ब्रह्म॑सꣳशितः । \newline
19. ब्रह्म॑सꣳशितो॒ हि हि ब्रह्म॑सꣳशितो॒ ब्रह्म॑सꣳशितो॒ हि । \newline
20. ब्रह्म॑सꣳशित॒ इति॒ ब्रह्म॑ - सꣳ॒॒शि॒तः॒ । \newline
21. ह्ये॑ष ए॒ष हि ह्ये॑षः । \newline
22. ए॒ष घृ॒ताह॑वनो घृ॒ताह॑वन ए॒ष ए॒ष घृ॒ताह॑वनः । \newline
23. घृ॒ताह॑वन॒ इतीति॑ घृ॒ताह॑वनो घृ॒ताह॑वन॒ इति॑ । \newline
24. घृ॒ताह॑वन॒ इति॑ घृ॒त - आ॒ह॒व॒नः॒ । \newline
25. इत्या॑हा॒हे तीत्या॑ह । \newline
26. आ॒ह॒ घृ॑ताहु॒तिर् घृ॑ताहु॒ति रा॑हाह घृताहु॒तिः । \newline
27. घृ॒ता॒हु॒तिर्. हि हि घृ॑ताहु॒तिर् घृ॑ताहु॒तिर्. हि । \newline
28. घृ॒ता॒हु॒तिरिति॑ घृत - आ॒हु॒तिः । \newline
29. ह्य॑स्यास्य॒ हि ह्य॑स्य । \newline
30. अ॒स्य॒ प्रि॒यत॑मा प्रि॒यत॑मा ऽस्यास्य प्रि॒यत॑मा । \newline
31. प्रि॒यत॑मा प्र॒णीः प्र॒णीः प्रि॒यत॑मा प्रि॒यत॑मा प्र॒णीः । \newline
32. प्रि॒यत॒मेति॑ प्रि॒य - त॒मा॒ । \newline
33. प्र॒णीर् य॒ज्ञानां᳚ ॅय॒ज्ञाना᳚म् प्र॒णीः प्र॒णीर् य॒ज्ञाना᳚म् । \newline
34. प्र॒णीरिति॑ प्र - नीः । \newline
35. य॒ज्ञाना॒ मितीति॑ य॒ज्ञानां᳚ ॅय॒ज्ञाना॒ मिति॑ । \newline
36. इत्या॑हा॒हे तीत्या॑ह । \newline
37. आ॒ह॒ प्र॒णीः प्र॒णी रा॑हाह प्र॒णीः । \newline
38. प्र॒णीर्. हि हि प्र॒णीः प्र॒णीर्. हि । \newline
39. प्र॒णीरिति॑ प्र - नीः । \newline
40. ह्ये॑ष ए॒ष हि ह्ये॑षः । \newline
41. ए॒ष य॒ज्ञानां᳚ ॅय॒ज्ञाना॑ मे॒ष ए॒ष य॒ज्ञाना᳚म् । \newline
42. य॒ज्ञानाꣳ॑ र॒थी र॒थीर् य॒ज्ञानां᳚ ॅय॒ज्ञानाꣳ॑ र॒थीः । \newline
43. र॒थी र॑द्ध्व॒राणा॑ मद्ध्व॒राणाꣳ॑ र॒थी र॒थी र॑द्ध्व॒राणा᳚म् । \newline
44. अ॒द्ध्व॒राणा॒ मिती त्य॑द्ध्व॒राणा॑ मद्ध्व॒राणा॒ मिति॑ । \newline
45. इत्या॑हा॒हे तीत्या॑ह । \newline
46. आ॒है॒ष ए॒ष आ॑हाहै॒षः । \newline
47. ए॒ष हि ह्ये॑ष ए॒ष हि । \newline
48. हि दे॑वर॒थो दे॑वर॒थो हि हि दे॑वर॒थः । \newline
49. दे॒व॒र॒थो॑ ऽतूर्तो॒ ऽतूर्तो॑ देवर॒थो दे॑वर॒थो॑ ऽतूर्तः॑ । \newline
50. दे॒व॒र॒थ इति॑ देव - र॒थः । \newline
51. अ॒तूर्तो॒ होता॒ होता॒ ऽतूर्तो॒ ऽतूर्तो॒ होता᳚ । \newline
52. होतेतीति॒ होता॒ होतेति॑ । \newline
53. इत्या॑हा॒हे तीत्या॑ह । \newline
54. आ॒ह॒ न नाहा॑ह॒ न । \newline
55. न हि हि न न हि । \newline
56. ह्ये॑त मे॒तꣳ हि ह्ये॑तम् । \newline
57. ए॒तम् कः क ए॒त मे॒तम् कः । \newline
58. कश्च॒न च॒न कः कश्च॒न । \newline
59. च॒न तर॑ति॒ तर॑ति च॒न च॒न तर॑ति । \newline

\textbf{Ghana Paata } \newline

1. विप्रा॒ हि हि विप्रा॒ विप्रा॒ ह्ये॑त ए॒ते हि विप्रा॒ विप्रा॒ ह्ये॑ते । \newline
2. ह्ये॑त ए॒ते हि ह्ये॑ते यद् यदे॒ते हि ह्ये॑ते यत् । \newline
3. ए॒ते यद् यदे॒त ए॒ते यच्छु॑श्रु॒वाꣳसः॑ शुश्रु॒वाꣳसो॒ यदे॒त ए॒ते यच्छु॑श्रु॒वाꣳसः॑ । \newline
4. यच्छु॑श्रु॒वाꣳसः॑ शुश्रु॒वाꣳसो॒ यद् यच्छु॑श्रु॒वाꣳसः॑ कविश॒स्तः क॑विश॒स्तः शु॑श्रु॒वाꣳसो॒ यद् यच्छु॑श्रु॒वाꣳसः॑ कविश॒स्तः । \newline
5. शु॒श्रु॒वाꣳसः॑ कविश॒स्तः क॑विश॒स्तः शु॑श्रु॒वाꣳसः॑ शुश्रु॒वाꣳसः॑ कविश॒स्त इतीति॑ कविश॒स्तः शु॑श्रु॒वाꣳसः॑ शुश्रु॒वाꣳसः॑ कविश॒स्त इति॑ । \newline
6. क॒वि॒श॒स्त इतीति॑ कविश॒स्तः क॑विश॒स्त इत्या॑हा॒हे ति॑ कविश॒स्तः क॑विश॒स्त इत्या॑ह । \newline
7. क॒वि॒श॒स्त इति॑ कवि - श॒स्तः । \newline
8. इत्या॑हा॒हे तीत्या॑ह क॒वयः॑ क॒वय॑ आ॒हे तीत्या॑ह क॒वयः॑ । \newline
9. आ॒ह॒ क॒वयः॑ क॒वय॑ आहाह क॒वयो॒ हि हि क॒वय॑ आहाह क॒वयो॒ हि । \newline
10. क॒वयो॒ हि हि क॒वयः॑ क॒वयो॒ ह्ये॑त ए॒ते हि क॒वयः॑ क॒वयो॒ ह्ये॑ते । \newline
11. ह्ये॑त ए॒ते हि ह्ये॑ते यद् यदे॒ते हि ह्ये॑ते यत् । \newline
12. ए॒ते यद् यदे॒त ए॒ते यच्छु॑श्रु॒वाꣳसः॑ शुश्रु॒वाꣳसो॒ यदे॒त ए॒ते यच्छु॑श्रु॒वाꣳसः॑ । \newline
13. यच्छु॑श्रु॒वाꣳसः॑ शुश्रु॒वाꣳसो॒ यद् यच्छु॑श्रु॒वाꣳसो॒ ब्रह्म॑सꣳशितो॒ ब्रह्म॑सꣳशितः शुश्रु॒वाꣳसो॒ यद् यच्छु॑श्रु॒वाꣳसो॒ ब्रह्म॑सꣳशितः । \newline
14. शु॒श्रु॒वाꣳसो॒ ब्रह्म॑सꣳशितो॒ ब्रह्म॑सꣳशितः शुश्रु॒वाꣳसः॑ शुश्रु॒वाꣳसो॒ ब्रह्म॑सꣳशित॒ इतीति॒ ब्रह्म॑सꣳशितः शुश्रु॒वाꣳसः॑ शुश्रु॒वाꣳसो॒ ब्रह्म॑सꣳशित॒ इति॑ । \newline
15. ब्रह्म॑सꣳशित॒ इतीति॒ ब्रह्म॑सꣳशितो॒ ब्रह्म॑सꣳशित॒ इत्या॑हा॒हे ति॒ ब्रह्म॑सꣳशितो॒ ब्रह्म॑सꣳशित॒ इत्या॑ह । \newline
16. ब्रह्म॑सꣳशित॒ इति॒ ब्रह्म॑ - सꣳ॒॒शि॒तः॒ । \newline
17. इत्या॑हा॒हे तीत्या॑ह॒ ब्रह्म॑सꣳशितो॒ ब्रह्म॑सꣳशित आ॒हे तीत्या॑ह॒ ब्रह्म॑सꣳशितः । \newline
18. आ॒ह॒ ब्रह्म॑सꣳशितो॒ ब्रह्म॑सꣳशित आहाह॒ ब्रह्म॑सꣳशितो॒ हि हि ब्रह्म॑सꣳशित आहाह॒ ब्रह्म॑सꣳशितो॒ हि । \newline
19. ब्रह्म॑सꣳशितो॒ हि हि ब्रह्म॑सꣳशितो॒ ब्रह्म॑सꣳशितो॒ ह्ये॑ष ए॒ष हि ब्रह्म॑सꣳशितो॒ ब्रह्म॑सꣳशितो॒ ह्ये॑षः । \newline
20. ब्रह्म॑सꣳशित॒ इति॒ ब्रह्म॑ - सꣳ॒॒शि॒तः॒ । \newline
21. ह्ये॑ष ए॒ष हि ह्ये॑ष घृ॒ताह॑वनो घृ॒ताह॑वन ए॒ष हि ह्ये॑ष घृ॒ताह॑वनः । \newline
22. ए॒ष घृ॒ताह॑वनो घृ॒ताह॑वन ए॒ष ए॒ष घृ॒ताह॑वन॒ इतीति॑ घृ॒ताह॑वन ए॒ष ए॒ष घृ॒ताह॑वन॒ इति॑ । \newline
23. घृ॒ताह॑वन॒ इतीति॑ घृ॒ताह॑वनो घृ॒ताह॑वन॒ इत्या॑हा॒हे ति॑ घृ॒ताह॑वनो घृ॒ताह॑वन॒ इत्या॑ह । \newline
24. घृ॒ताह॑वन॒ इति॑ घृ॒त - आ॒ह॒व॒नः॒ । \newline
25. इत्या॑हा॒हे तीत्या॑ह घृताहु॒तिर् घृ॑ताहु॒तिरा॒हे तीत्या॑ह घृताहु॒तिः । \newline
26. आ॒ह॒ घृ॒ता॒हु॒तिर् घृ॑ताहु॒ति रा॑हाह घृताहु॒तिर्. हि हि घृ॑ताहु॒ति रा॑हाह घृताहु॒तिर्. हि । \newline
27. घृ॒ता॒हु॒तिर्. हि हि घृ॑ताहु॒तिर् घृ॑ताहु॒तिर् ह्य॑स्यास्य॒ हि घृ॑ताहु॒तिर् घृ॑ताहु॒तिर् ह्य॑स्य । \newline
28. घृ॒ता॒हु॒तिरिति॑ घृत - आ॒हु॒तिः । \newline
29. ह्य॑स्यास्य॒ हि ह्य॑स्य प्रि॒यत॑मा प्रि॒यत॑मा ऽस्य॒ हि ह्य॑स्य प्रि॒यत॑मा । \newline
30. अ॒स्य॒ प्रि॒यत॑मा प्रि॒यत॑मा ऽस्यास्य प्रि॒यत॑मा प्र॒णीः प्र॒णीः प्रि॒यत॑मा ऽस्यास्य प्रि॒यत॑मा प्र॒णीः । \newline
31. प्रि॒यत॑मा प्र॒णीः प्र॒णीः प्रि॒यत॑मा प्रि॒यत॑मा प्र॒णीर् य॒ज्ञानां᳚ ॅय॒ज्ञाना᳚म् प्र॒णीः प्रि॒यत॑मा प्रि॒यत॑मा प्र॒णीर् य॒ज्ञाना᳚म् । \newline
32. प्रि॒यत॒मेति॑ प्रि॒य - त॒मा॒ । \newline
33. प्र॒णीर् य॒ज्ञानां᳚ ॅय॒ज्ञाना᳚म् प्र॒णीः प्र॒णीर् य॒ज्ञाना॒ मितीति॑ य॒ज्ञाना᳚म् प्र॒णीः प्र॒णीर् य॒ज्ञाना॒ मिति॑ । \newline
34. प्र॒णीरिति॑ प्र - नीः । \newline
35. य॒ज्ञाना॒ मितीति॑ य॒ज्ञानां᳚ ॅय॒ज्ञाना॒ मित्या॑हा॒हे ति॑ य॒ज्ञानां᳚ ॅय॒ज्ञाना॒ मित्या॑ह । \newline
36. इत्या॑हा॒हे तीत्या॑ह प्र॒णीः प्र॒णीरा॒हे तीत्या॑ह प्र॒णीः । \newline
37. आ॒ह॒ प्र॒णीः प्र॒णी रा॑हाह प्र॒णीर्. हि हि प्र॒णी रा॑हाह प्र॒णीर्. हि । \newline
38. प्र॒णीर्. हि हि प्र॒णीः प्र॒णीर् ह्ये॑ष ए॒ष हि प्र॒णीः प्र॒णीर् ह्ये॑षः । \newline
39. प्र॒णीरिति॑ प्र - नीः । \newline
40. ह्ये॑ष ए॒ष हि ह्ये॑ष य॒ज्ञानां᳚ ॅय॒ज्ञाना॑ मे॒ष हि ह्ये॑ष य॒ज्ञाना᳚म् । \newline
41. ए॒ष य॒ज्ञानां᳚ ॅय॒ज्ञाना॑ मे॒ष ए॒ष य॒ज्ञानाꣳ॑ र॒थी र॒थीर् य॒ज्ञाना॑ मे॒ष ए॒ष य॒ज्ञानाꣳ॑ र॒थीः । \newline
42. य॒ज्ञानाꣳ॑ र॒थी र॒थीर् य॒ज्ञानां᳚ ॅय॒ज्ञानाꣳ॑ र॒थी र॑द्ध्व॒राणा॑ मद्ध्व॒राणाꣳ॑ र॒थीर् य॒ज्ञानां᳚ ॅय॒ज्ञानाꣳ॑ र॒थी र॑द्ध्व॒राणा᳚म् । \newline
43. र॒थी र॑द्ध्व॒राणा॑ मद्ध्व॒राणाꣳ॑ र॒थी र॒थी र॑द्ध्व॒राणा॒ मिती त्य॑द्ध्व॒राणाꣳ॑ र॒थी र॒थी र॑द्ध्व॒राणा॒ मिति॑ । \newline
44. अ॒द्ध्व॒राणा॒ मिती त्य॑द्ध्व॒राणा॑ मद्ध्व॒राणा॒ मित्या॑हा॒हे त्य॑द्ध्व॒राणा॑ मद्ध्व॒राणा॒ मित्या॑ह । \newline
45. इत्या॑हा॒हे तीत्या॑है॒ष ए॒ष आ॒हे तीत्या॑है॒षः । \newline
46. आ॒है॒ष ए॒ष आ॑हाहै॒ष हि ह्ये॑ष आ॑हाहै॒ष हि । \newline
47. ए॒ष हि ह्ये॑ष ए॒ष हि दे॑वर॒थो दे॑वर॒थो ह्ये॑ष ए॒ष हि दे॑वर॒थः । \newline
48. हि दे॑वर॒थो दे॑वर॒थो हि हि दे॑वर॒थो॑ ऽतूर्तो॒ ऽतूर्तो॑ देवर॒थो हि हि दे॑वर॒थो॑ ऽतूर्तः॑ । \newline
49. दे॒व॒र॒थो॑ ऽतूर्तो॒ ऽतूर्तो॑ देवर॒थो दे॑वर॒थो॑ ऽतूर्तो॒ होता॒ होता॒ ऽतूर्तो॑ देवर॒थो दे॑वर॒थो॑ ऽतूर्तो॒ होता᳚ । \newline
50. दे॒व॒र॒थ इति॑ देव - र॒थः । \newline
51. अ॒तूर्तो॒ होता॒ होता॒ ऽतूर्तो॒ ऽतूर्तो॒ होतेतीति॒ होता॒ ऽतूर्तो॒ ऽतूर्तो॒ होतेति॑ । \newline
52. होतेतीति॒ होता॒ होते त्या॑हा॒हे ति॒ होता॒ होते त्या॑ह । \newline
53. इत्या॑हा॒हे तीत्या॑ह॒ न नाहे तीत्या॑ह॒ न । \newline
54. आ॒ह॒ न नाहा॑ह॒ न हि हि नाहा॑ह॒ न हि । \newline
55. न हि हि न न ह्ये॑त मे॒तꣳ हि न न ह्ये॑तम् । \newline
56. ह्ये॑त मे॒तꣳ हि ह्ये॑तम् कः क ए॒तꣳ हि ह्ये॑तम् कः । \newline
57. ए॒तम् कः क ए॒त मे॒तम् कश्च॒न च॒न क ए॒त मे॒तम् कश्च॒न । \newline
58. कश्च॒न च॒न कः कश्च॒न तर॑ति॒ तर॑ति च॒न कः कश्च॒न तर॑ति । \newline
59. च॒न तर॑ति॒ तर॑ति च॒न च॒न तर॑ति॒ तूर्णि॒ स्तूर्णि॒ स्तर॑ति च॒न च॒न तर॑ति॒ तूर्णिः॑ । \newline
\pagebreak
\markright{ TS 2.5.9.3  \hfill https://www.vedavms.in \hfill}
\addcontentsline{toc}{section}{ TS 2.5.9.3 }
\section*{ TS 2.5.9.3 }

\textbf{TS 2.5.9.3 } \newline
\textbf{Samhita Paata} \newline

तर॑ति॒ तूर्णि॑र्. हव्य॒वाडित्या॑ह॒ सर्वꣳ॒॒ह्ये॑ष तर॒त्यास्पात्रं॑ जु॒हूर्दे॒वाना॒मित्या॑ह जु॒हूर्ह्ये॑ष दे॒वानां᳚ चम॒सो दे॑व॒पान॒ इत्या॑ह चम॒सो ह्ये॑ष दे॑व॒पानो॒ऽराꣳ इ॑वाग्ने ने॒मिर्दे॒वाꣳस्त्वं प॑रि॒भूर॒सीत्या॑ह दे॒वान् ह्ये॑ष प॑रि॒भूर्यद्-ब्रू॒यादा व॑ह दे॒वान् दे॑वय॒ते यज॑माना॒येति॒ भ्रातृ॑व्यमस्मै- [  ] \newline

\textbf{Pada Paata} \newline

तर॑ति । तूर्णिः॑ । ह॒व्य॒वाडिति॑ हव्य - वाट् । इति॑ । आ॒ह॒ । सर्व᳚म् । हि । ए॒षः । तर॑ति । आस्पात्र᳚म् । जु॒हूः । दे॒वाना᳚म् । इति॑ । आ॒ह॒ । जु॒हूः । हि । ए॒षः । दे॒वाना᳚म् । च॒म॒सः । दे॒व॒पान॒ इति॑ देव - पानः॑ । इति॑ । आ॒ह॒ । च॒म॒सः । हि । ए॒षः । दे॒व॒पान॒ इति॑ देव - पानः॑ । अ॒रान् । इ॒व॒ । अ॒ग्ने॒ । ने॒मिः । दे॒वान् । त्वम् । प॒रि॒भूरिति॑ परि-भूः । अ॒सि॒ । इति॑ । आ॒ह॒ । दे॒वान् । हि । ए॒षः । प॒रि॒भूरिति॑ परि - भूः । यत् । ब्रू॒यात् । एति॑ । व॒ह॒ । दे॒वान् । दे॒व॒य॒त इति॑ देव - य॒ते । यज॑मानाय । इति॑ । भ्रातृ॑व्यम् । अ॒स्मै॒ ।  \newline


\textbf{Krama Paata} \newline

तर॑ति॒ तूर्णिः॑ । तूर्णि॑र्. हव्य॒वाट् । ह॒व्य॒वाडिति॑ । ह॒व्य॒वाडिति॑ हव्य - वाट् । इत्या॑ह । आ॒ह॒ सर्व᳚म् । सर्वꣳ॒॒ हि । ह्ये॑षः । ए॒ष तर॑ति । तर॒त्यास्पात्र᳚म् । आस्पात्र॑म् जु॒हूः । जु॒हूर् दे॒वाना᳚म् । दे॒वाना॒मिति॑ । इत्या॑ह । आ॒ह॒ जु॒हूः । जु॒हूर्. हि । ह्ये॑षः । ए॒ष दे॒वाना᳚म् । दे॒वाना᳚म् चम॒सः । च॒म॒सो दे॑व॒पानः॑ । दे॒व॒पान॒ इति॑ । दे॒व॒पान॒ इति॑ देव - पानः॑ । इत्या॑ह । आ॒ह॒ च॒म॒सः । च॒म॒सो हि । ह्ये॑षः । ए॒ष दे॑व॒पानः॑ । दे॒व॒पानो॒ऽरान् । दे॒व॒पान॒ इति॑ देव - पानः॑ । अ॒राꣳ इ॑व । इ॒वा॒ग्ने॒ । अ॒ग्ने॒ ने॒मिः । ने॒मिर् दे॒वान् । दे॒वाꣳस्त्वम् । त्वम् प॑रि॒भूः । प॒रि॒भूर॑सि । प॒रि॒भूरिति॑ परि - भूः । अ॒सीति॑ । इत्या॑ह । आ॒ह॒ दे॒वान् । दे॒वान्. हि । ह्ये॑षः । ए॒ष प॑रि॒भूः । प॒रि॒भूर् यत् । प॒रि॒भूरिति॑ परि - भूः । य॒द् ब्रू॒यात् । ब्रू॒यादा । आ व॑ह । व॒ह॒ दे॒वान् । दे॒वान् दे॑वय॒ते । दे॒व॒य॒ते यज॑मानाय । दे॒व॒य॒त इति॑ देव - य॒ते । यज॑माना॒येति॑ । इति॒ भ्रातृ॑व्यम् । भ्रातृ॑व्यमस्मै । अ॒स्मै॒ ज॒न॒ये॒त्॒ \newline

\textbf{Jatai Paata} \newline

1. तर॑ति॒ तूर्णि॒ स्तूर्णि॒ स्तर॑ति॒ तर॑ति॒ तूर्णिः॑ । \newline
2. तूर्णि॑र्. हव्य॒वा ड्ढ॑व्य॒वाट् तूर्णि॒स्तूर्णि॑र्. हव्य॒वाट् । \newline
3. ह॒व्य॒वा डितीति॑ हव्य॒वा ड्ढ॑व्य॒वाडिति॑ । \newline
4. ह॒व्य॒वाडिति॑ हव्य - वाट् । \newline
5. इत्या॑हा॒हे तीत्या॑ह । \newline
6. आ॒ह॒ सर्वꣳ॒॒ सर्व॑ माहाह॒ सर्व᳚म् । \newline
7. सर्वꣳ॒॒ हि हि सर्वꣳ॒॒ सर्वꣳ॒॒ हि । \newline
8. ह्ये॑ष ए॒ष हि ह्ये॑षः । \newline
9. ए॒ष तर॑ति॒ तर॑ त्ये॒ष ए॒ष तर॑ति । \newline
10. तर॒ त्यास्पात्र॒ मास्पात्र॒म् तर॑ति॒ तर॒ त्यास्पात्र᳚म् । \newline
11. आस्पात्र॑म् जु॒हूर् जु॒हू रास्पात्र॒ मास्पात्र॑म् जु॒हूः । \newline
12. जु॒हूर् दे॒वाना᳚म् दे॒वाना᳚म् जु॒हूर् जु॒हूर् दे॒वाना᳚म् । \newline
13. दे॒वाना॒ मितीति॑ दे॒वाना᳚म् दे॒वाना॒ मिति॑ । \newline
14. इत्या॑हा॒हे तीत्या॑ह । \newline
15. आ॒ह॒ जु॒हूर् जु॒हू रा॑हाह जु॒हूः । \newline
16. जु॒हूर्. हि हि जु॒हूर् जु॒हूर्. हि । \newline
17. ह्ये॑ष ए॒ष हि ह्ये॑षः । \newline
18. ए॒ष दे॒वाना᳚म् दे॒वाना॑ मे॒ष ए॒ष दे॒वाना᳚म् । \newline
19. दे॒वाना᳚म् चम॒स श्च॑म॒सो दे॒वाना᳚म् दे॒वाना᳚म् चम॒सः । \newline
20. च॒म॒सो दे॑व॒पानो॑ देव॒पान॑ श्चम॒स श्च॑म॒सो दे॑व॒पानः॑ । \newline
21. दे॒व॒पान॒ इतीति॑ देव॒पानो॑ देव॒पान॒ इति॑ । \newline
22. दे॒व॒पान॒ इति॑ देव - पानः॑ । \newline
23. इत्या॑हा॒हे तीत्या॑ह । \newline
24. आ॒ह॒ च॒म॒स श्च॑म॒स आ॑हाह चम॒सः । \newline
25. च॒म॒सो हि हि च॑म॒स श्च॑म॒सो हि । \newline
26. ह्ये॑ष ए॒ष हि ह्ये॑षः । \newline
27. ए॒ष दे॑व॒पानो॑ देव॒पान॑ ए॒ष ए॒ष दे॑व॒पानः॑ । \newline
28. दे॒व॒पानो॒ ऽराꣳ अ॒रान् दे॑व॒पानो॑ देव॒पानो॒ ऽरान् । \newline
29. दे॒व॒पान॒ इति॑ देव - पानः॑ । \newline
30. अ॒राꣳ इ॑वे वा॒राꣳ अ॒राꣳ इ॑व । \newline
31. इ॒वा॒ग्ने॒ ऽग्न॒ इ॒वे॒ वा॒ग्ने॒ । \newline
32. अ॒ग्ने॒ ने॒मिर् ने॒मि र॑ग्ने ऽग्ने ने॒मिः । \newline
33. ने॒मिर् दे॒वान् दे॒वान् ने॒मिर् ने॒मिर् दे॒वान् । \newline
34. दे॒वाꣳ स्त्वम् त्वम् दे॒वान् दे॒वाꣳ स्त्वम् । \newline
35. त्वम् प॑रि॒भूः प॑रि॒भू स्त्वम् त्वम् प॑रि॒भूः । \newline
36. प॒रि॒भू र॑स्यसि परि॒भूः प॑रि॒भू र॑सि । \newline
37. प॒रि॒भूरिति॑ परि - भूः । \newline
38. अ॒सीती त्य॑स्य॒सीति॑ । \newline
39. इत्या॑हा॒हे तीत्या॑ह । \newline
40. आ॒ह॒ दे॒वान् दे॒वा ना॑हाह दे॒वान् । \newline
41. दे॒वान्. हि हि दे॒वान् दे॒वान्. हि । \newline
42. ह्ये॑ष ए॒ष हि ह्ये॑षः । \newline
43. ए॒ष प॑रि॒भूः प॑रि॒भू रे॒ष ए॒ष प॑रि॒भूः । \newline
44. प॒रि॒भूर् यद् यत् प॑रि॒भूः प॑रि॒भूर् यत् । \newline
45. प॒रि॒भूरिति॑ परि - भूः । \newline
46. यद् ब्रू॒याद् ब्रू॒याद् यद् यद् ब्रू॒यात् । \newline
47. ब्रू॒यादा ब्रू॒याद् ब्रू॒यादा । \newline
48. आ व॑ह व॒हा व॑ह । \newline
49. व॒ह॒ दे॒वान् दे॒वान्. व॑ह वह दे॒वान् । \newline
50. दे॒वान् दे॑वय॒ते दे॑वय॒ते दे॒वान् दे॒वान् दे॑वय॒ते । \newline
51. दे॒व॒य॒ते यज॑मानाय॒ यज॑मानाय देवय॒ते दे॑वय॒ते यज॑मानाय । \newline
52. दे॒व॒य॒त इति॑ देव - य॒ते । \newline
53. यज॑माना॒ये तीति॒ यज॑मानाय॒ यज॑माना॒ये ति॑ । \newline
54. इति॒ भ्रातृ॑व्य॒म् भ्रातृ॑व्य॒ मितीति॒ भ्रातृ॑व्यम् । \newline
55. भ्रातृ॑व्य मस्मा अस्मै॒ भ्रातृ॑व्य॒म् भ्रातृ॑व्य मस्मै । \newline
56. अ॒स्मै॒ ज॒न॒ये॒ज् ज॒न॒ये॒ द॒स्मा॒ अ॒स्मै॒ ज॒न॒ये॒त् । \newline

\textbf{Ghana Paata } \newline

1. तर॑ति॒ तूर्णि॒ स्तूर्णि॒ स्तर॑ति॒ तर॑ति॒ तूर्णि॑र्. हव्य॒वा ड्ढ॑व्य॒वाट् तूर्णि॒ स्तर॑ति॒ तर॑ति॒ तूर्णि॑र्. हव्य॒वाट् । \newline
2. तूर्णि॑र्. हव्य॒वा ड्ढ॑व्य॒वाट् तूर्णि॒ स्तूर्णि॑र्. हव्य॒वाडितीति॑ हव्य॒वाट् तूर्णि॒ स्तूर्णि॑र्. हव्य॒वाडिति॑ । \newline
3. ह॒व्य॒वाडितीति॑ हव्य॒वा ड्ढ॑व्य॒वा डित्या॑हा॒हे ति॑ हव्य॒वा ड्ढ॑व्य॒वा डित्या॑ह । \newline
4. ह॒व्य॒वाडिति॑ हव्य - वाट् । \newline
5. इत्या॑हा॒हे तीत्या॑ह॒ सर्वꣳ॒॒ सर्व॑ मा॒हे तीत्या॑ह॒ सर्व᳚म् । \newline
6. आ॒ह॒ सर्वꣳ॒॒ सर्व॑ माहाह॒ सर्वꣳ॒॒ हि हि सर्व॑ माहाह॒ सर्वꣳ॒॒ हि । \newline
7. सर्वꣳ॒॒ हि हि सर्वꣳ॒॒ सर्वꣳ॒॒ ह्ये॑ष ए॒ष हि सर्वꣳ॒॒ सर्वꣳ॒॒ ह्ये॑षः । \newline
8. ह्ये॑ष ए॒ष हि ह्ये॑ष तर॑ति॒ तर॑त्ये॒ष हि ह्ये॑ष तर॑ति । \newline
9. ए॒ष तर॑ति॒ तर॑त्ये॒ष ए॒ष तर॒ त्यास्पात्र॒ मास्पात्र॒म् तर॑त्ये॒ष ए॒ष तर॒ त्यास्पात्र᳚म् । \newline
10. तर॒त्यास्पात्र॒ मास्पात्र॒म् तर॑ति॒ तर॒ त्यास्पात्र॑म् जु॒हूर् जु॒हू रास्पात्र॒म् तर॑ति॒ तर॒ त्यास्पात्र॑म् जु॒हूः । \newline
11. आस्पात्र॑म् जु॒हूर् जु॒हू रास्पात्र॒ मास्पात्र॑म् जु॒हूर् दे॒वाना᳚म् दे॒वाना᳚म् जु॒हू रास्पात्र॒ मास्पात्र॑म् जु॒हूर् दे॒वाना᳚म् । \newline
12. जु॒हूर् दे॒वाना᳚म् दे॒वाना᳚म् जु॒हूर् जु॒हूर् दे॒वाना॒ मितीति॑ दे॒वाना᳚म् जु॒हूर् जु॒हूर् दे॒वाना॒ मिति॑ । \newline
13. दे॒वाना॒ मितीति॑ दे॒वाना᳚म् दे॒वाना॒ मित्या॑हा॒हे ति॑ दे॒वाना᳚म् दे॒वाना॒ मित्या॑ह । \newline
14. इत्या॑हा॒हे तीत्या॑ह जु॒हूर् जु॒हूरा॒हे तीत्या॑ह जु॒हूः । \newline
15. आ॒ह॒ जु॒हूर् जु॒हू रा॑हाह जु॒हूर्. हि हि जु॒हू रा॑हाह जु॒हूर्. हि । \newline
16. जु॒हूर्. हि हि जु॒हूर् जु॒हूर् ह्ये॑ष ए॒ष हि जु॒हूर् जु॒हूर् ह्ये॑षः । \newline
17. ह्ये॑ष ए॒ष हि ह्ये॑ष दे॒वाना᳚म् दे॒वाना॑ मे॒ष हि ह्ये॑ष दे॒वाना᳚म् । \newline
18. ए॒ष दे॒वाना᳚म् दे॒वाना॑ मे॒ष ए॒ष दे॒वाना᳚म् चम॒स श्च॑म॒सो दे॒वाना॑ मे॒ष ए॒ष दे॒वाना᳚म् चम॒सः । \newline
19. दे॒वाना᳚म् चम॒स श्च॑म॒सो दे॒वाना᳚म् दे॒वाना᳚म् चम॒सो दे॑व॒पानो॑ देव॒पान॑ श्चम॒सो दे॒वाना᳚म् दे॒वाना᳚म् चम॒सो दे॑व॒पानः॑ । \newline
20. च॒म॒सो दे॑व॒पानो॑ देव॒पान॑ श्चम॒स श्च॑म॒सो दे॑व॒पान॒ इतीति॑ देव॒पान॑ श्चम॒स श्च॑म॒सो दे॑व॒पान॒ इति॑ । \newline
21. दे॒व॒पान॒ इतीति॑ देव॒पानो॑ देव॒पान॒ इत्या॑हा॒हे ति॑ देव॒पानो॑ देव॒पान॒ इत्या॑ह । \newline
22. दे॒व॒पान॒ इति॑ देव - पानः॑ । \newline
23. इत्या॑हा॒हे तीत्या॑ह चम॒स श्च॑म॒स आ॒हे तीत्या॑ह चम॒सः । \newline
24. आ॒ह॒ च॒म॒स श्च॑म॒स आ॑हाह चम॒सो हि हि च॑म॒स आ॑हाह चम॒सो हि । \newline
25. च॒म॒सो हि हि च॑म॒स श्च॑म॒सो ह्ये॑ष ए॒ष हि च॑म॒स श्च॑म॒सो ह्ये॑षः । \newline
26. ह्ये॑ष ए॒ष हि ह्ये॑ष दे॑व॒पानो॑ देव॒पान॑ ए॒ष हि ह्ये॑ष दे॑व॒पानः॑ । \newline
27. ए॒ष दे॑व॒पानो॑ देव॒पान॑ ए॒ष ए॒ष दे॑व॒पानो॒ ऽराꣳ अ॒रान् दे॑व॒पान॑ ए॒ष ए॒ष दे॑व॒पानो॒ ऽरान् । \newline
28. दे॒व॒पानो॒ ऽराꣳ अ॒रान् दे॑व॒पानो॑ देव॒पानो॒ ऽराꣳ इ॑वे वा॒रान् दे॑व॒पानो॑ देव॒पानो॒ ऽराꣳ इ॑व । \newline
29. दे॒व॒पान॒ इति॑ देव - पानः॑ । \newline
30. अ॒राꣳ इ॑वे वा॒राꣳ अ॒राꣳ इ॑वाग्ने ऽग्न इवा॒राꣳ अ॒राꣳ इ॑वाग्ने । \newline
31. इ॒वा॒ग्ने॒ ऽग्न॒ इ॒वे॒ वा॒ग्ने॒ ने॒मिर् ने॒मिर॑ग्न इवे वाग्ने ने॒मिः । \newline
32. अ॒ग्ने॒ ने॒मिर् ने॒मि र॑ग्ने ऽग्ने ने॒मिर् दे॒वान् दे॒वान् ने॒मि र॑ग्ने ऽग्ने ने॒मिर् दे॒वान् । \newline
33. ने॒मिर् दे॒वान् दे॒वान् ने॒मिर् ने॒मिर् दे॒वाꣳ स्त्वम् त्वम् दे॒वान् ने॒मिर् ने॒मिर् दे॒वाꣳ स्त्वम् । \newline
34. दे॒वाꣳ स्त्वम् त्वम् दे॒वान् दे॒वाꣳ स्त्वम् प॑रि॒भूः प॑रि॒ भूस्त्वम् दे॒वान् दे॒वाꣳ स्त्वम् प॑रि॒भूः । \newline
35. त्वम् प॑रि॒भूः प॑रि॒भू स्त्वम् त्वम् प॑रि॒भू र॑स्यसि परि॒भू स्त्वम् त्वम् प॑रि॒भू र॑सि । \newline
36. प॒रि॒भू र॑स्यसि परि॒भूः प॑रि॒भू र॒सीती त्य॑सि परि॒भूः प॑रि॒भू र॒सीति॑ । \newline
37. प॒रि॒भूरिति॑ परि - भूः । \newline
38. अ॒सीती त्य॑स्य॒सी त्या॑हा॒हे त्य॑स्य॒सीत्या॑ह । \newline
39. इत्या॑हा॒हे तीत्या॑ह दे॒वान् दे॒वा ना॒हे तीत्या॑ह दे॒वान् । \newline
40. आ॒ह॒ दे॒वान् दे॒वा ना॑हाह दे॒वान्. हि हि दे॒वा ना॑हाह दे॒वान्. हि । \newline
41. दे॒वान्. हि हि दे॒वान् दे॒वान् ह्ये॑ष ए॒ष हि दे॒वान् दे॒वान् ह्ये॑षः । \newline
42. ह्ये॑ष ए॒ष हि ह्ये॑ष प॑रि॒भूः प॑रि॒भू रे॒ष हि ह्ये॑ष प॑रि॒भूः । \newline
43. ए॒ष प॑रि॒भूः प॑रि॒भू रे॒ष ए॒ष प॑रि॒भूर् यद् यत् प॑रि॒भू रे॒ष ए॒ष प॑रि॒भूर् यत् । \newline
44. प॒रि॒भूर् यद् यत् प॑रि॒भूः प॑रि॒भूर् यद् ब्रू॒याद् ब्रू॒याद् यत् प॑रि॒भूः प॑रि॒भूर् यद् ब्रू॒यात् । \newline
45. प॒रि॒भूरिति॑ परि - भूः । \newline
46. यद् ब्रू॒याद् ब्रू॒याद् यद् यद् ब्रू॒यादा ब्रू॒याद् यद् यद् ब्रू॒यादा । \newline
47. ब्रू॒यादा ब्रू॒याद् ब्रू॒यादा व॑ह व॒हा ब्रू॒याद् ब्रू॒यादा व॑ह । \newline
48. आ व॑ह व॒हा व॑ह दे॒वान् दे॒वान्. व॒हा व॑ह दे॒वान् । \newline
49. व॒ह॒ दे॒वान् दे॒वान्. व॑ह वह दे॒वान् दे॑वय॒ते दे॑वय॒ते दे॒वान्. व॑ह वह दे॒वान् दे॑वय॒ते । \newline
50. दे॒वान् दे॑वय॒ते दे॑वय॒ते दे॒वान् दे॒वान् दे॑वय॒ते यज॑मानाय॒ यज॑मानाय देवय॒ते दे॒वान् दे॒वान् दे॑वय॒ते यज॑मानाय । \newline
51. दे॒व॒य॒ते यज॑मानाय॒ यज॑मानाय देवय॒ते दे॑वय॒ते यज॑माना॒ये तीति॒ यज॑मानाय देवय॒ते दे॑वय॒ते यज॑माना॒ये ति॑ । \newline
52. दे॒व॒य॒त इति॑ देव - य॒ते । \newline
53. यज॑माना॒ये तीति॒ यज॑मानाय॒ यज॑माना॒ये ति॒ भ्रातृ॑व्य॒म् भ्रातृ॑व्य॒ मिति॒ यज॑मानाय॒ यज॑माना॒ये ति॒ भ्रातृ॑व्यम् । \newline
54. इति॒ भ्रातृ॑व्य॒म् भ्रातृ॑व्य॒ मितीति॒ भ्रातृ॑व्य मस्मा अस्मै॒ भ्रातृ॑व्य॒ मितीति॒ भ्रातृ॑व्य मस्मै । \newline
55. भ्रातृ॑व्य मस्मा अस्मै॒ भ्रातृ॑व्य॒म् भ्रातृ॑व्य मस्मै जनयेज् जनयेदस्मै॒ भ्रातृ॑व्य॒म् भ्रातृ॑व्य मस्मै जनयेत् । \newline
56. अ॒स्मै॒ ज॒न॒ये॒ज् ज॒न॒ये॒ द॒स्मा॒ अ॒स्मै॒ ज॒न॒ये॒दा ज॑नये दस्मा अस्मै जनये॒दा । \newline
\pagebreak
\markright{ TS 2.5.9.4  \hfill https://www.vedavms.in \hfill}
\addcontentsline{toc}{section}{ TS 2.5.9.4 }
\section*{ TS 2.5.9.4 }

\textbf{TS 2.5.9.4 } \newline
\textbf{Samhita Paata} \newline

जनये॒दा व॑ह दे॒वान्. यज॑माना॒येत्या॑ह॒ यज॑मानमे॒वैतेन॑ वर्द्धयत्य॒ग्निम॑ग्न॒ आ व॑ह॒ सोम॒मा व॒हेत्या॑ह दे॒वता॑ ए॒व तद्-य॑थापू॒र्वमुप॑ ह्वयत॒ आ चा᳚ग्ने दे॒वान्. वह॑ सु॒यजा॑ च यज जातवेद॒ इत्या॑हा॒ग्निमे॒व तथ् सꣳ श्य॑ति॒ सो᳚ऽस्य॒ सꣳशि॑तो दे॒वेभ्यो॑ ह॒व्यं ॅव॑हत्य॒ग्निर्. होते - [  ] \newline

\textbf{Pada Paata} \newline

ज॒न॒ये॒त् । एति॑ । व॒ह॒ । दे॒वान् । यज॑मानाय । इति॑ । आ॒ह॒ । यज॑मानम् । ए॒व । ए॒तेन॑ । व॒द्‌र्ध॒य॒ति॒ । अ॒ग्निम् । अ॒ग्ने॒ । एति॑ । व॒ह॒ । सोम᳚म् । एति॑ । व॒ह॒ । इति॑ । आ॒ह॒ । दे॒वताः᳚ । ए॒व । तत् । य॒था॒पू॒र्वमिति॑ यथा - पू॒र्वम् । उपेति॑ । ह्व॒य॒ते॒ । एति॑ । च॒ । अ॒ग्ने॒ । दे॒वान् । वह॑ । सु॒यजेति॑ सु - यजा᳚ । च॒ । य॒ज॒ । जा॒त॒वे॒द॒ इति॑ जात-वे॒दः॒ । इति॑ । आ॒ह॒ । अ॒ग्निम् । ए॒व । तत् । समिति॑ । श्य॒ति॒ । सः । अ॒स्य॒ । सꣳशि॑त॒ इति॒ सं - शि॒तः॒ । दे॒वेभ्यः॑ । ह॒व्यम् । व॒ह॒ति॒ । अ॒ग्निः । होता᳚ ।  \newline


\textbf{Krama Paata} \newline

ज॒न॒ये॒दा । आ व॑ह । व॒ह॒ दे॒वान् । दे॒वान्. यज॑मानाय । यज॑माना॒येति॑ । इत्या॑ह । आ॒ह॒ यज॑मानम् । यज॑मानमे॒व । ए॒वैतेन॑ । ए॒तेन॑ वर्द्धयति । व॒र्द्ध॒य॒त्य॒ग्निम् । अ॒ग्निम॑ग्ने । अ॒ग्न॒ आ । आ व॑ह । व॒ह॒ सोम᳚म् । सोम॒मा । आ व॑ह । व॒हेति॑ । इत्या॑ह । आ॒ह॒ दे॒वताः᳚ । दे॒वता॑ ए॒व । ए॒व तत् । तद् य॑थापू॒र्वम् । य॒था॒पू॒र्वमुप॑ । य॒था॒पू॒र्वमिति॑ यथा - पू॒र्वम् । उप॑ ह्वयते । ह्व॒य॒त॒ आ । आ च॑ । चा॒ग्ने॒ । अ॒ग्ने॒ दे॒वान् । दे॒वान्. वह॑ । वह॑ सु॒यजा᳚ । सु॒यजा॑ च । सु॒यजेति॑ सु - यजा᳚ । च॒ य॒ज॒ । य॒ज॒ जा॒त॒वे॒दः॒ । जा॒त॒वे॒द॒ इति॑ । जा॒त॒वे॒द॒ इति॑ जात - वे॒दः॒ । इत्या॑ह । आ॒हा॒ग्निम् । 
अ॒ग्निमे॒व । ए॒व तत् । तथ् सम् । सꣳ श्य॑ति । श्य॒ति॒ सः । सो᳚ऽस्य । अ॒स्य॒ सꣳशि॑तः । सꣳशि॑तो दे॒वेभ्यः॑ । सꣳशि॑त॒ इति॒ सम् - शि॒तः॒ । दे॒वेभ्यो॑ ह॒व्यम् । ह॒व्यम् ॅव॑हति । व॒ह॒त्य॒ग्निः । अ॒ग्निर्. होता᳚ । होतेति॑ \newline

\textbf{Jatai Paata} \newline

1. ज॒न॒ये॒दा ज॑नयेज् जनये॒दा । \newline
2. आ व॑ह व॒हा व॑ह । \newline
3. व॒ह॒ दे॒वान् दे॒वान्. व॑ह वह दे॒वान् । \newline
4. दे॒वान्. यज॑मानाय॒ यज॑मानाय दे॒वान् दे॒वान्. यज॑मानाय । \newline
5. यज॑माना॒ये तीति॒ यज॑मानाय॒ यज॑माना॒ये ति॑ । \newline
6. इत्या॑हा॒हे तीत्या॑ह । \newline
7. आ॒ह॒ यज॑मानं॒ ॅयज॑मान माहाह॒ यज॑मानम् । \newline
8. यज॑मान मे॒वैव यज॑मानं॒ ॅयज॑मान मे॒व । \newline
9. ए॒वैते नै॒ते नै॒वै वैतेन॑ । \newline
10. ए॒तेन॑ वर्द्धयति वर्द्धय त्ये॒तेनै॒तेन॑ वर्द्धयति । \newline
11. व॒र्द्ध॒य॒ त्य॒ग्नि म॒ग्निं ॅव॑र्द्धयति वर्द्धय त्य॒ग्निम् । \newline
12. अ॒ग्नि म॑ग्ने ऽग्ने॒ ऽग्नि म॒ग्नि म॑ग्ने । \newline
13. अ॒ग्न॒ आ ऽग्ने᳚ ऽग्न॒ आ । \newline
14. आ व॑ह व॒हा व॑ह । \newline
15. व॒ह॒ सोमꣳ॒॒ सोमं॑ ॅवह वह॒ सोम᳚म् । \newline
16. सोम॒ मा सोमꣳ॒॒ सोम॒ मा । \newline
17. आ व॑ह व॒हा व॑ह । \newline
18. व॒हे तीति॑ वह व॒हे ति॑ । \newline
19. इत्या॑हा॒हे तीत्या॑ह । \newline
20. आ॒ह॒ दे॒वता॑ दे॒वता॑ आहाह दे॒वताः᳚ । \newline
21. दे॒वता॑ ए॒वैव दे॒वता॑ दे॒वता॑ ए॒व । \newline
22. ए॒व तत् तदे॒वैव तत् । \newline
23. तद् य॑थापू॒र्वं ॅय॑थापू॒र्वम् तत् तद् य॑थापू॒र्वम् । \newline
24. य॒था॒पू॒र्व मुपोप॑ यथापू॒र्वं ॅय॑थापू॒र्व मुप॑ । \newline
25. य॒था॒पू॒र्वमिति॑ यथा - पू॒र्वम् । \newline
26. उप॑ ह्वयते ह्वयत॒ उपोप॑ ह्वयते । \newline
27. ह्व॒य॒त॒ आ ह्व॑यते ह्वयत॒ आ । \newline
28. आ च॒ चा च॑ । \newline
29. चा॒ग्ने॒ ऽग्ने॒ च॒ चा॒ग्ने॒ । \newline
30. अ॒ग्ने॒ दे॒वान् दे॒वा न॑ग्ने ऽग्ने दे॒वान् । \newline
31. दे॒वान्. वह॒ वह॑ दे॒वान् दे॒वान्. वह॑ । \newline
32. वह॑ सु॒यजा॑ सु॒यजा॒ वह॒ वह॑ सु॒यजा᳚ । \newline
33. सु॒यजा॑ च च सु॒यजा॑ सु॒यजा॑ च । \newline
34. सु॒यजेति॑ सु - यजा᳚ । \newline
35. च॒ य॒ज॒ य॒ज॒ च॒ च॒ य॒ज॒ । \newline
36. य॒ज॒ जा॒त॒वे॒दो॒ जा॒त॒वे॒दो॒ य॒ज॒ य॒ज॒ जा॒त॒वे॒दः॒ । \newline
37. जा॒त॒वे॒द॒ इतीति॑ जातवेदो जातवेद॒ इति॑ । \newline
38. जा॒त॒वे॒द॒ इति॑ जात - वे॒दः॒ । \newline
39. इत्या॑हा॒हे तीत्या॑ह । \newline
40. आ॒हा॒ग्नि म॒ग्नि मा॑हाहा॒ग्निम् । \newline
41. अ॒ग्नि मे॒वैवाग्नि म॒ग्नि मे॒व । \newline
42. ए॒व तत् तदे॒वैव तत् । \newline
43. तथ् सꣳ सम् तत् तथ् सम् । \newline
44. सꣳ श्य॑ति श्यति॒ सꣳ सꣳ श्य॑ति । \newline
45. श्य॒ति॒ स स श्य॑ति श्यति॒ सः । \newline
46. सो᳚ ऽस्यास्य॒ स सो᳚ ऽस्य । \newline
47. अ॒स्य॒ सꣳशि॑तः॒ सꣳशि॑तो ऽस्यास्य॒ सꣳशि॑तः । \newline
48. सꣳशि॑तो दे॒वेभ्यो॑ दे॒वेभ्यः॒ सꣳशि॑तः॒ सꣳशि॑तो दे॒वेभ्यः॑ । \newline
49. सꣳशि॑त॒ इति॒ सं - शि॒तः॒ । \newline
50. दे॒वेभ्यो॑ ह॒व्यꣳ ह॒व्यम् दे॒वेभ्यो॑ दे॒वेभ्यो॑ ह॒व्यम् । \newline
51. ह॒व्यं ॅव॑हति वहति ह॒व्यꣳ ह॒व्यं ॅव॑हति । \newline
52. व॒ह॒ त्य॒ग्नि र॒ग्निर् व॑हति वह त्य॒ग्निः । \newline
53. अ॒ग्निर्. होता॒ होता॒ ऽग्नि र॒ग्निर्. होता᳚ । \newline
54. होतेतीति॒ होता॒ होतेति॑ । \newline

\textbf{Ghana Paata } \newline

1. ज॒न॒ये॒दा ज॑नयेज् जनये॒दा व॑ह व॒हा ज॑नयेज् जनये॒दा व॑ह । \newline
2. आ व॑ह व॒हा व॑ह दे॒वान् दे॒वान्. व॒हा व॑ह दे॒वान् । \newline
3. व॒ह॒ दे॒वान् दे॒वान्. व॑ह वह दे॒वान्. यज॑मानाय॒ यज॑मानाय दे॒वान्. व॑ह वह दे॒वान्. यज॑मानाय । \newline
4. दे॒वान्. यज॑मानाय॒ यज॑मानाय दे॒वान् दे॒वान्. यज॑माना॒ये तीति॒ यज॑मानाय दे॒वान् दे॒वान्. यज॑माना॒ये ति॑ । \newline
5. यज॑माना॒ये तीति॒ यज॑मानाय॒ यज॑माना॒ये त्या॑हा॒हे ति॒ यज॑मानाय॒ यज॑माना॒ये त्या॑ह । \newline
6. इत्या॑हा॒हे तीत्या॑ह॒ यज॑मानं॒ ॅयज॑मान मा॒हे तीत्या॑ह॒ यज॑मानम् । \newline
7. आ॒ह॒ यज॑मानं॒ ॅयज॑मान माहाह॒ यज॑मान मे॒वैव यज॑मान माहाह॒ यज॑मान मे॒व । \newline
8. यज॑मान मे॒वैव यज॑मानं॒ ॅयज॑मान मे॒वैते नै॒तेनै॒व यज॑मानं॒ ॅयज॑मान मे॒वैतेन॑ । \newline
9. ए॒वैते नै॒ते नै॒वैवैतेन॑ वर्द्धयति वर्द्धय त्ये॒ते नै॒वैवैतेन॑ वर्द्धयति । \newline
10. ए॒तेन॑ वर्द्धयति वर्द्धय त्ये॒तेनै॒तेन॑ वर्द्धय त्य॒ग्नि म॒ग्निं ॅव॑र्द्धय त्ये॒ते नै॒तेन॑ वर्द्धय त्य॒ग्निम् । \newline
11. व॒र्द्ध॒य॒ त्य॒ग्नि म॒ग्निं ॅव॑र्द्धयति वर्द्धय त्य॒ग्नि म॑ग्ने ऽग्ने॒ ऽग्निं ॅव॑र्द्धयति वर्द्धय त्य॒ग्नि म॑ग्ने । \newline
12. अ॒ग्नि म॑ग्ने ऽग्ने॒ ऽग्नि म॒ग्नि म॑ग्न॒ आ ऽग्ने॒ ऽग्नि म॒ग्नि म॑ग्न॒ आ । \newline
13. अ॒ग्न॒ आ ऽग्ने᳚ ऽग्न॒ आ व॑ह व॒हा ऽग्ने᳚ ऽग्न॒ आ व॑ह । \newline
14. आ व॑ह व॒हा व॑ह॒ सोमꣳ॒॒ सोमं॑ ॅव॒हा व॑ह॒ सोम᳚म् । \newline
15. व॒ह॒ सोमꣳ॒॒ सोमं॑ ॅवह वह॒ सोम॒ मा सोमं॑ ॅवह वह॒ सोम॒ मा । \newline
16. सोम॒ मा सोमꣳ॒॒ सोम॒ मा व॑ह व॒हा सोमꣳ॒॒ सोम॒ मा व॑ह । \newline
17. आ व॑ह व॒हा व॒हे तीति॑ व॒हा व॒हे ति॑ । \newline
18. व॒हे तीति॑ वह व॒हे त्या॑हा॒हे ति॑ वह व॒हे त्या॑ह । \newline
19. इत्या॑हा॒हे तीत्या॑ह दे॒वता॑ दे॒वता॑ आ॒हे तीत्या॑ह दे॒वताः᳚ । \newline
20. आ॒ह॒ दे॒वता॑ दे॒वता॑ आहाह दे॒वता॑ ए॒वैव दे॒वता॑ आहाह दे॒वता॑ ए॒व । \newline
21. दे॒वता॑ ए॒वैव दे॒वता॑ दे॒वता॑ ए॒व तत् तदे॒व दे॒वता॑ दे॒वता॑ ए॒व तत् । \newline
22. ए॒व तत् तदे॒वैव तद् य॑थापू॒र्वं ॅय॑थापू॒र्वम् तदे॒वैव तद् य॑थापू॒र्वम् । \newline
23. तद् य॑थापू॒र्वं ॅय॑थापू॒र्वम् तत् तद् य॑थापू॒र्व मुपोप॑ यथापू॒र्वम् तत् तद् य॑थापू॒र्व मुप॑ । \newline
24. य॒था॒पू॒र्व मुपोप॑ यथापू॒र्वं ॅय॑थापू॒र्व मुप॑ ह्वयते ह्वयत॒ उप॑ यथापू॒र्वं ॅय॑थापू॒र्व मुप॑ ह्वयते । \newline
25. य॒था॒पू॒र्वमिति॑ यथा - पू॒र्वम् । \newline
26. उप॑ ह्वयते ह्वयत॒ उपोप॑ ह्वयत॒ आ ह्व॑यत॒ उपोप॑ ह्वयत॒ आ । \newline
27. ह्व॒य॒त॒ आ ह्व॑यते ह्वयत॒ आ च॒ चा ह्व॑यते ह्वयत॒ आ च॑ । \newline
28. आ च॒ चा चा᳚ग्ने ऽग्ने॒ चा चा᳚ग्ने । \newline
29. चा॒ग्ने॒ ऽग्ने॒ च॒ चा॒ग्ने॒ दे॒वान् दे॒वा न॑ग्ने च चाग्ने दे॒वान् । \newline
30. अ॒ग्ने॒ दे॒वान् दे॒वा न॑ग्ने ऽग्ने दे॒वान्. वह॒ वह॑ दे॒वा न॑ग्ने ऽग्ने दे॒वान्. वह॑ । \newline
31. दे॒वान्. वह॒ वह॑ दे॒वान् दे॒वान्. वह॑ सु॒यजा॑ सु॒यजा॒ वह॑ दे॒वान् दे॒वान्. वह॑ सु॒यजा᳚ । \newline
32. वह॑ सु॒यजा॑ सु॒यजा॒ वह॒ वह॑ सु॒यजा॑ च च सु॒यजा॒ वह॒ वह॑ सु॒यजा॑ च । \newline
33. सु॒यजा॑ च च सु॒यजा॑ सु॒यजा॑ च यज यज च सु॒यजा॑ सु॒यजा॑ च यज । \newline
34. सु॒यजेति॑ सु - यजा᳚ । \newline
35. च॒ य॒ज॒ य॒ज॒ च॒ च॒ य॒ज॒ जा॒त॒वे॒दो॒ जा॒त॒वे॒दो॒ य॒ज॒ च॒ च॒ य॒ज॒ जा॒त॒वे॒दः॒ । \newline
36. य॒ज॒ जा॒त॒वे॒दो॒ जा॒त॒वे॒दो॒ य॒ज॒ य॒ज॒ जा॒त॒वे॒द॒ इतीति॑ जातवेदो यज यज जातवेद॒ इति॑ । \newline
37. जा॒त॒वे॒द॒ इतीति॑ जातवेदो जातवेद॒ इत्या॑हा॒हे ति॑ जातवेदो जातवेद॒ इत्या॑ह । \newline
38. जा॒त॒वे॒द॒ इति॑ जात - वे॒दः॒ । \newline
39. इत्या॑हा॒हे तीत्या॑हा॒ग्नि म॒ग्नि मा॒हे तीत्या॑हा॒ग्निम् । \newline
40. आ॒हा॒ग्नि म॒ग्नि मा॑हाहा॒ग्नि मे॒वैवाग्नि मा॑हाहा॒ग्नि मे॒व । \newline
41. अ॒ग्नि मे॒वैवाग्नि म॒ग्नि मे॒व तत् तदे॒वाग्नि म॒ग्नि मे॒व तत् । \newline
42. ए॒व तत् तदे॒वैव तथ् सꣳ सम् तदे॒वैव तथ् सम् । \newline
43. तथ् सꣳ सम् तत् तथ् सꣳ श्य॑ति श्यति॒ सम् तत् तथ् सꣳ श्य॑ति । \newline
44. सꣳ श्य॑ति श्यति॒ सꣳ सꣳ श्य॑ति॒ स स श्य॑ति॒ सꣳ सꣳ श्य॑ति॒ सः । \newline
45. श्य॒ति॒ स स श्य॑ति श्यति॒ सो᳚ ऽस्यास्य॒ स श्य॑ति श्यति॒ सो᳚ ऽस्य । \newline
46. सो᳚ ऽस्यास्य॒ स सो᳚ ऽस्य॒ सꣳशि॑तः॒ सꣳशि॑तो ऽस्य॒ स सो᳚ ऽस्य॒ सꣳशि॑तः । \newline
47. अ॒स्य॒ सꣳशि॑तः॒ सꣳशि॑तो ऽस्यास्य॒ सꣳशि॑तो दे॒वेभ्यो॑ दे॒वेभ्यः॒ सꣳशि॑तो ऽस्यास्य॒ सꣳशि॑तो दे॒वेभ्यः॑ । \newline
48. सꣳशि॑तो दे॒वेभ्यो॑ दे॒वेभ्यः॒ सꣳशि॑तः॒ सꣳशि॑तो दे॒वेभ्यो॑ ह॒व्यꣳ ह॒व्यम् दे॒वेभ्यः॒ सꣳशि॑तः॒ सꣳशि॑तो दे॒वेभ्यो॑ ह॒व्यम् । \newline
49. सꣳशि॑त॒ इति॒ सं - शि॒तः॒ । \newline
50. दे॒वेभ्यो॑ ह॒व्यꣳ ह॒व्यम् दे॒वेभ्यो॑ दे॒वेभ्यो॑ ह॒व्यं ॅव॑हति वहति ह॒व्यम् दे॒वेभ्यो॑ दे॒वेभ्यो॑ ह॒व्यं ॅव॑हति । \newline
51. ह॒व्यं ॅव॑हति वहति ह॒व्यꣳ ह॒व्यं ॅव॑हत्य॒ग्नि र॒ग्निर् व॑हति ह॒व्यꣳ ह॒व्यं ॅव॑हत्य॒ग्निः । \newline
52. व॒ह॒त्य॒ग्नि र॒ग्निर् व॑हति वहत्य॒ग्निर्. होता॒ होता॒ ऽग्निर् व॑हति वहत्य॒ग्निर्. होता᳚ । \newline
53. अ॒ग्निर्. होता॒ होता॒ ऽग्नि र॒ग्निर्. होतेतीति॒ होता॒ ऽग्नि र॒ग्निर्. होतेति॑ । \newline
54. होतेतीति॒ होता॒ होतेत्या॑हा॒हे ति॒ होता॒ होतेत्या॑ह । \newline
\pagebreak
\markright{ TS 2.5.9.5  \hfill https://www.vedavms.in \hfill}
\addcontentsline{toc}{section}{ TS 2.5.9.5 }
\section*{ TS 2.5.9.5 }

\textbf{TS 2.5.9.5 } \newline
\textbf{Samhita Paata} \newline

-त्या॑हा॒ग्निर्वै दे॒वानाꣳ॒॒ होता॒ य ए॒व दे॒वानाꣳ॒॒ होता॒ तं ॅवृ॑णीते॒स्मो व॒यमित्या॑हा॒ऽऽत्मान॑मे॒व स॒त्त्वं ग॑मयति सा॒धु ते॑ यजमान दे॒वतेत्या॑हा॒ऽऽशिष॑मे॒वैतामा शा᳚स्ते॒ यद्ब्रू॒याद्-यो᳚ऽग्निꣳ होता॑र॒मवृ॑था॒ इत्य॒ग्निनो॑भ॒यतो॒ यज॑मानं॒ परि॑ गृह्णीयात् प्र॒मायु॑कः स्याद्-यजमानदेव॒त्या॑ वै जु॒हूर्भ्रा॑तृव्य देव॒त्यो॑प॒भृद् - [  ] \newline

\textbf{Pada Paata} \newline

इति॑ । आ॒ह॒ । अ॒ग्निः । वै । दे॒वाना᳚म् । होता᳚ । यः । ए॒व । दे॒वाना᳚म् । होता᳚ । तम् । वृ॒णी॒ते॒ । स्मः । व॒यम् । इति॑ । आ॒ह॒ । आ॒त्मान᳚म् । ए॒व । स॒त्त्वमिति॑ सत् - त्वम् । ग॒म॒य॒ति॒ । सा॒धु । ते॒ । य॒ज॒मा॒न॒ । दे॒वता᳚ । इति॑ । आ॒ह॒ । आ॒शिष॒मित्या᳚ - शिष᳚म् । ए॒व । ए॒ताम् । एति॑ । शा॒स्ते॒ । यत् । ब्रू॒यात् । यः । अ॒ग्निम् । होता॑रम् । अवृ॑थाः । इति॑ । अ॒ग्निना᳚ । उ॒भ॒यतः॑ । यज॑मानम् । परीति॑ । गृ॒ह्णी॒या॒त् । प्र॒मायु॑क॒ इति॑ प्र - मायु॑कः । स्या॒त् । य॒ज॒मा॒न॒दे॒व॒त्येति॑ यजमान - दे॒व॒त्या᳚ । वै । जु॒हूः । भ्रा॒तृ॒व्य॒दे॒व॒त्येति॑ भ्रातृव्य - दे॒व॒त्या᳚ । उ॒प॒भृदित्यु॑प - भृत् ।  \newline


\textbf{Krama Paata} \newline

इत्या॑ह । आ॒हा॒ग्निः । अ॒ग्निर् वै । वै दे॒वाना᳚म् । दे॒वानाꣳ॒॒ होता᳚ । होता॒ यः । य ए॒व । ए॒व दे॒वाना᳚म् । दे॒वानाꣳ॒॒ होता᳚ । होता॒ तम् । तम् ॅवृ॑णीते । वृ॒णी॒ते॒ स्मः । स्मो व॒यम् । व॒यमिति॑ । इत्या॑ह । आ॒हा॒त्मान᳚म् । आ॒त्मान॑मे॒व । ए॒व स॒त्त्वम् । स॒त्त्वम् ग॑मयति । स॒त्त्वमिति॑ सत् - त्वम् । ग॒म॒य॒ति॒ सा॒धु । सा॒धु ते᳚ । ते॒ य॒ज॒मा॒न॒ । य॒ज॒मा॒न॒ दे॒वता᳚ । दे॒वतेति॑ । इत्या॑ह । आ॒हा॒शिष᳚म् । आ॒शिष॑मे॒व । आ॒शिष॒मित्या᳚ - शिष᳚म् । ए॒वैताम् । ए॒तामा । आ शा᳚स्ते । शा॒स्ते॒ यत् । यद् ब्रू॒यात् । ब्रू॒याद् यः । यो᳚ऽग्निम् । अ॒ग्निꣳ होता॑रम् । होता॑र॒मवृ॑थाः । अवृ॑था॒ इति॑ । इत्य॒ग्निना᳚ । अ॒ग्निनो॑भ॒यतः॑ । उ॒भ॒यतो॒ यज॑मानम् । यज॑मान॒म् परि॑ । परि॑ गृह्णीयात् । गृ॒ह्णी॒या॒त् प्र॒मायु॑कः । प्र॒मायु॑कः स्यात् । प्र॒मायु॑क॒ इति॑ प्र - मायु॑कः । स्या॒द् य॒ज॒मा॒न॒दे॒व॒त्या᳚ । य॒ज॒मा॒न॒दे॒व॒त्या॑ वै । य॒ज॒मा॒न॒दे॒व॒त्येति॑ यजमान - दे॒व॒त्या᳚ । वै जु॒हूः । जु॒हूर् भ्रा॑तृव्यदेव॒त्या᳚ । भ्रा॒तृ॒व्य॒दे॒व॒त्यो॑प॒भृत् । भ्रा॒तृ॒व्य॒दे॒व॒त्येति॑ भ्रातृव्य - दे॒व॒त्या᳚ । उ॒प॒भृद् यत् । उ॒प॒भृदित्यु॑प - भृत् \newline

\textbf{Jatai Paata} \newline

1. इत्या॑हा॒हे तीत्या॑ह । \newline
2. आ॒हा॒ग्नि र॒ग्नि रा॑हाहा॒ग्निः । \newline
3. अ॒ग्निर् वै वा अ॒ग्नि र॒ग्निर् वै । \newline
4. वै दे॒वाना᳚म् दे॒वानां॒ ॅवै वै दे॒वाना᳚म् । \newline
5. दे॒वानाꣳ॒॒ होता॒ होता॑ दे॒वाना᳚म् दे॒वानाꣳ॒॒ होता᳚ । \newline
6. होता॒ यो यो होता॒ होता॒ यः । \newline
7. य ए॒वैव यो य ए॒व । \newline
8. ए॒व दे॒वाना᳚म् दे॒वाना॑ मे॒वैव दे॒वाना᳚म् । \newline
9. दे॒वानाꣳ॒॒ होता॒ होता॑ दे॒वाना᳚म् दे॒वानाꣳ॒॒ होता᳚ । \newline
10. होता॒ तम् तꣳ होता॒ होता॒ तम् । \newline
11. तं ॅवृ॑णीते वृणीते॒ तम् तं ॅवृ॑णीते । \newline
12. वृ॒णी॒ते॒ स्मः स्मो वृ॑णीते वृणीते॒ स्मः । \newline
13. स्मो व॒यं ॅव॒यꣳ स्मः स्मो व॒यम् । \newline
14. व॒य मितीति॑ व॒यं ॅव॒य मिति॑ । \newline
15. इत्या॑हा॒हे तीत्या॑ह । \newline
16. आ॒हा॒त्मान॑ मा॒त्मान॑ माहाहा॒त्मान᳚म् । \newline
17. आ॒त्मान॑ मे॒वैवात्मान॑ मा॒त्मान॑ मे॒व । \newline
18. ए॒व स॒त्त्वꣳ स॒त्त्व मे॒वैव स॒त्त्वम् । \newline
19. स॒त्त्वम् ग॑मयति गमयति स॒त्त्वꣳ स॒त्त्वम् ग॑मयति । \newline
20. स॒त्त्वमिति॑ सत् - त्वम् । \newline
21. ग॒म॒य॒ति॒ सा॒धु सा॒धु ग॑मयति गमयति सा॒धु । \newline
22. सा॒धु ते॑ ते सा॒धु सा॒धु ते᳚ । \newline
23. ते॒ य॒ज॒मा॒न॒ य॒ज॒मा॒न॒ ते॒ ते॒ य॒ज॒मा॒न॒ । \newline
24. य॒ज॒मा॒न॒ दे॒वता॑ दे॒वता॑ यजमान यजमान दे॒वता᳚ । \newline
25. दे॒वतेतीति॑ दे॒वता॑ दे॒वतेति॑ । \newline
26. इत्या॑हा॒हे तीत्या॑ह । \newline
27. आ॒हा॒शिष॑ मा॒शिष॑ माहाहा॒शिष᳚म् । \newline
28. आ॒शिष॑ मे॒वैवाशिष॑ मा॒शिष॑ मे॒व । \newline
29. आ॒शिष॒मित्या᳚ - शिष᳚म् । \newline
30. ए॒वैता मे॒ता मे॒वैवैताम् । \newline
31. ए॒ता मैता मे॒ता मा । \newline
32. आ शा᳚स्ते शास्त॒ आ शा᳚स्ते । \newline
33. शा॒स्ते॒ यद् यच्छा᳚स्ते शास्ते॒ यत् । \newline
34. यद् ब्रू॒याद् ब्रू॒याद् यद् यद् ब्रू॒यात् । \newline
35. ब्रू॒याद् यो यो ब्रू॒याद् ब्रू॒याद् यः । \newline
36. यो᳚ ऽग्नि म॒ग्निं ॅयो यो᳚ ऽग्निम् । \newline
37. अ॒ग्निꣳ होता॑रꣳ॒॒ होता॑र म॒ग्नि म॒ग्निꣳ होता॑रम् । \newline
38. होता॑र॒ मवृ॑था॒ अवृ॑था॒ होता॑रꣳ॒॒ होता॑र॒ मवृ॑थाः । \newline
39. अवृ॑था॒ इती त्यवृ॑था॒ अवृ॑था॒ इति॑ । \newline
40. इत्य॒ग्निना॒ ऽग्निनेती त्य॒ग्निना᳚ । \newline
41. अ॒ग्निनो॑ भ॒यत॑ उभ॒यतो॒ ऽग्निना॒ ऽग्निनो॑ भ॒यतः॑ । \newline
42. उ॒भ॒यतो॒ यज॑मानं॒ ॅयज॑मान मुभ॒यत॑ उभ॒यतो॒ यज॑मानम् । \newline
43. यज॑मान॒म् परि॒ परि॒ यज॑मानं॒ ॅयज॑मान॒म् परि॑ । \newline
44. परि॑ गृह्णीयाद् गृह्णीया॒त् परि॒ परि॑ गृह्णीयात् । \newline
45. गृ॒ह्णी॒या॒त् प्र॒मायु॑कः प्र॒मायु॑को गृह्णीयाद् गृह्णीयात् प्र॒मायु॑कः । \newline
46. प्र॒मायु॑कः स्याथ् स्यात् प्र॒मायु॑कः प्र॒मायु॑कः स्यात् । \newline
47. प्र॒मायु॑क॒ इति॑ प्र - मायु॑कः । \newline
48. स्या॒द् य॒ज॒मा॒न॒दे॒व॒त्या॑ यजमानदेव॒त्या᳚ स्याथ् स्याद् यजमानदेव॒त्या᳚ । \newline
49. य॒ज॒मा॒न॒दे॒व॒त्या॑ वै वै य॑जमानदेव॒त्या॑ यजमानदेव॒त्या॑ वै । \newline
50. य॒ज॒मा॒न॒दे॒व॒त्येति॑ यजमान - दे॒व॒त्या᳚ । \newline
51. वै जु॒हूर् जु॒हूर् वै वै जु॒हूः । \newline
52. जु॒हूर् भ्रा॑तृव्यदेव॒त्या᳚ भ्रातृव्यदेव॒त्या॑ जु॒हूर् जु॒हूर् भ्रा॑तृव्यदेव॒त्या᳚ । \newline
53. भ्रा॒तृ॒व्य॒दे॒व॒ त्यो॑प॒भृ दु॑प॒भृद् भ्रा॑तृव्यदेव॒त्या᳚ भ्रातृव्यदेव॒ त्यो॑प॒भृत् । \newline
54. भ्रा॒तृ॒व्य॒दे॒व॒त्येति॑ भ्रातृव्य - दे॒व॒त्या᳚ । \newline
55. उ॒प॒भृद् यद् यदु॑प॒भृ दु॑प॒भृद् यत् । \newline
56. उ॒प॒भृदित्यु॑प - भृत् । \newline

\textbf{Ghana Paata } \newline

1. इत्या॑हा॒हे तीत्या॑हा॒ ग्नि र॒ग्नि रा॒हे तीत्या॑हा॒ग्निः । \newline
2. आ॒हा॒ग्नि र॒ग्नि रा॑हाहा॒ग्निर् वै वा अ॒ग्नि रा॑हाहा॒ग्निर् वै । \newline
3. अ॒ग्निर् वै वा अ॒ग्नि र॒ग्निर् वै दे॒वाना᳚म् दे॒वानां॒ ॅवा अ॒ग्नि र॒ग्निर् वै दे॒वाना᳚म् । \newline
4. वै दे॒वाना᳚म् दे॒वानां॒ ॅवै वै दे॒वानाꣳ॒॒ होता॒ होता॑ दे॒वानां॒ ॅवै वै दे॒वानाꣳ॒॒ होता᳚ । \newline
5. दे॒वानाꣳ॒॒ होता॒ होता॑ दे॒वाना᳚म् दे॒वानाꣳ॒॒ होता॒ यो यो होता॑ दे॒वाना᳚म् दे॒वानाꣳ॒॒ होता॒ यः । \newline
6. होता॒ यो यो होता॒ होता॒ य ए॒वैव यो होता॒ होता॒ य ए॒व । \newline
7. य ए॒वैव यो य ए॒व दे॒वाना᳚म् दे॒वाना॑ मे॒व यो य ए॒व दे॒वाना᳚म् । \newline
8. ए॒व दे॒वाना᳚म् दे॒वाना॑ मे॒वैव दे॒वानाꣳ॒॒ होता॒ होता॑ दे॒वाना॑ मे॒वैव दे॒वानाꣳ॒॒ होता᳚ । \newline
9. दे॒वानाꣳ॒॒ होता॒ होता॑ दे॒वाना᳚म् दे॒वानाꣳ॒॒ होता॒ तम् तꣳ होता॑ दे॒वाना᳚म् दे॒वानाꣳ॒॒ होता॒ तम् । \newline
10. होता॒ तम् तꣳ होता॒ होता॒ तं ॅवृ॑णीते वृणीते॒ तꣳ होता॒ होता॒ तं ॅवृ॑णीते । \newline
11. तं ॅवृ॑णीते वृणीते॒ तम् तं ॅवृ॑णीते॒ स्मः स्मो वृ॑णीते॒ तम् तं ॅवृ॑णीते॒ स्मः । \newline
12. वृ॒णी॒ते॒ स्मः स्मो वृ॑णीते वृणीते॒ स्मो व॒यं ॅव॒यꣳ स्मो वृ॑णीते वृणीते॒ स्मो व॒यम् । \newline
13. स्मो व॒यं ॅव॒यꣳ स्मः स्मो व॒य मितीति॑ व॒यꣳ स्मः स्मो व॒य मिति॑ । \newline
14. व॒य मितीति॑ व॒यं ॅव॒य मित्या॑हा॒हे ति॑ व॒यं ॅव॒य मित्या॑ह । \newline
15. इत्या॑हा॒हे तीत्या॑हा॒त्मान॑ मा॒त्मान॑ मा॒हे तीत्या॑हा॒त्मान᳚म् । \newline
16. आ॒हा॒त्मान॑ मा॒त्मान॑ माहाहा॒त्मान॑ मे॒वैवात्मान॑ माहाहा॒त्मान॑ मे॒व । \newline
17. आ॒त्मान॑ मे॒वैवात्मान॑ मा॒त्मान॑ मे॒व स॒त्त्वꣳ स॒त्त्व मे॒वात्मान॑ मा॒त्मान॑ मे॒व स॒त्त्वम् । \newline
18. ए॒व स॒त्त्वꣳ स॒त्त्व मे॒वैव स॒त्त्वम् ग॑मयति गमयति स॒त्त्व मे॒वैव स॒त्त्वम् ग॑मयति । \newline
19. स॒त्त्वम् ग॑मयति गमयति स॒त्त्वꣳ स॒त्त्वम् ग॑मयति सा॒धु सा॒धु ग॑मयति स॒त्त्वꣳ स॒त्त्वम् ग॑मयति सा॒धु । \newline
20. स॒त्त्वमिति॑ सत् - त्वम् । \newline
21. ग॒म॒य॒ति॒ सा॒धु सा॒धु ग॑मयति गमयति सा॒धु ते॑ ते सा॒धु ग॑मयति गमयति सा॒धु ते᳚ । \newline
22. सा॒धु ते॑ ते सा॒धु सा॒धु ते॑ यजमान यजमान ते सा॒धु सा॒धु ते॑ यजमान । \newline
23. ते॒ य॒ज॒मा॒न॒ य॒ज॒मा॒न॒ ते॒ ते॒ य॒ज॒मा॒न॒ दे॒वता॑ दे॒वता॑ यजमान ते ते यजमान दे॒वता᳚ । \newline
24. य॒ज॒मा॒न॒ दे॒वता॑ दे॒वता॑ यजमान यजमान दे॒वतेतीति॑ दे॒वता॑ यजमान यजमान दे॒वतेति॑ । \newline
25. दे॒वतेतीति॑ दे॒वता॑ दे॒वतेत्या॑हा॒हे ति॑ दे॒वता॑ दे॒वतेत्या॑ह । \newline
26. इत्या॑हा॒हे तीत्या॑हा॒शिष॑ मा॒शिष॑ मा॒हे तीत्या॑हा॒शिष᳚म् । \newline
27. आ॒हा॒शिष॑ मा॒शिष॑ माहाहा॒शिष॑ मे॒वैवाशिष॑ माहाहा॒शिष॑ मे॒व । \newline
28. आ॒शिष॑ मे॒वैवाशिष॑ मा॒शिष॑ मे॒वैता मे॒ता मे॒वाशिष॑ मा॒शिष॑ मे॒वैताम् । \newline
29. आ॒शिष॒मित्या᳚ - शिष᳚म् । \newline
30. ए॒वैता मे॒ता मे॒वैवैता मैता मे॒वैवैता मा । \newline
31. ए॒ता मैता मे॒ता मा शा᳚स्ते शास्त॒ ऐता मे॒ता मा शा᳚स्ते । \newline
32. आ शा᳚स्ते शास्त॒ आ शा᳚स्ते॒ यद् यच्छा᳚स्त॒ आ शा᳚स्ते॒ यत् । \newline
33. शा॒स्ते॒ यद् यच्छा᳚स्ते शास्ते॒ यद् ब्रू॒याद् ब्रू॒याद् यच्छा᳚स्ते शास्ते॒ यद् ब्रू॒यात् । \newline
34. यद् ब्रू॒याद् ब्रू॒याद् यद् यद् ब्रू॒याद् यो यो ब्रू॒याद् यद् यद् ब्रू॒याद् यः । \newline
35. ब्रू॒याद् यो यो ब्रू॒याद् ब्रू॒याद् यो᳚ ऽग्नि म॒ग्निं ॅयो ब्रू॒याद् ब्रू॒याद् यो᳚ ऽग्निम् । \newline
36. यो᳚ ऽग्नि म॒ग्निं ॅयो यो᳚ ऽग्निꣳ होता॑रꣳ॒॒ होता॑र म॒ग्निं ॅयो यो᳚ ऽग्निꣳ होता॑रम् । \newline
37. अ॒ग्निꣳ होता॑रꣳ॒॒ होता॑र म॒ग्नि म॒ग्निꣳ होता॑र॒ मवृ॑था॒ अवृ॑था॒ होता॑र म॒ग्नि म॒ग्निꣳ होता॑र॒ मवृ॑थाः । \newline
38. होता॑र॒ मवृ॑था॒ अवृ॑था॒ होता॑रꣳ॒॒ होता॑र॒ मवृ॑था॒ इतीत्यवृ॑था॒ होता॑रꣳ॒॒ होता॑र॒ मवृ॑था॒ इति॑ । \newline
39. अवृ॑था॒ इतीत्यवृ॑था॒ अवृ॑था॒ इत्य॒ग्निना॒ ऽग्निनेत्यवृ॑था॒ अवृ॑था॒ इत्य॒ग्निना᳚ । \newline
40. इत्य॒ग्निना॒ ऽग्निनेती त्य॒ग्निनो॑भ॒यत॑ उभ॒यतो॒ ऽग्निनेती त्य॒ग्निनो॑भ॒यतः॑ । \newline
41. अ॒ग्निनो॑भ॒यत॑ उभ॒यतो॒ ऽग्निना॒ ऽग्निनो॑भ॒यतो॒ यज॑मानं॒ ॅयज॑मान मुभ॒यतो॒ ऽग्निना॒ ऽग्निनो॑भ॒यतो॒ यज॑मानम् । \newline
42. उ॒भ॒यतो॒ यज॑मानं॒ ॅयज॑मान मुभ॒यत॑ उभ॒यतो॒ यज॑मान॒म् परि॒ परि॒ यज॑मान मुभ॒यत॑ उभ॒यतो॒ यज॑मान॒म् परि॑ । \newline
43. यज॑मान॒म् परि॒ परि॒ यज॑मानं॒ ॅयज॑मान॒म् परि॑ गृह्णीयाद् गृह्णीया॒त् परि॒ यज॑मानं॒ ॅयज॑मान॒म् परि॑ गृह्णीयात् । \newline
44. परि॑ गृह्णीयाद् गृह्णीया॒त् परि॒ परि॑ गृह्णीयात् प्र॒मायु॑कः प्र॒मायु॑को गृह्णीया॒त् परि॒ परि॑ गृह्णीयात् प्र॒मायु॑कः । \newline
45. गृ॒ह्णी॒या॒त् प्र॒मायु॑कः प्र॒मायु॑को गृह्णीयाद् गृह्णीयात् प्र॒मायु॑कः स्याथ् स्यात् प्र॒मायु॑को गृह्णीयाद् गृह्णीयात् प्र॒मायु॑कः स्यात् । \newline
46. प्र॒मायु॑कः स्याथ् स्यात् प्र॒मायु॑कः प्र॒मायु॑कः स्याद् यजमानदेव॒त्या॑ यजमानदेव॒त्या᳚ स्यात् प्र॒मायु॑कः प्र॒मायु॑कः स्याद् यजमानदेव॒त्या᳚ । \newline
47. प्र॒मायु॑क॒ इति॑ प्र - मायु॑कः । \newline
48. स्या॒द् य॒ज॒मा॒न॒दे॒व॒त्या॑ यजमानदेव॒त्या᳚ स्याथ् स्याद् यजमानदेव॒त्या॑ वै वै य॑जमानदेव॒त्या᳚ स्याथ् स्याद् यजमानदेव॒त्या॑ वै । \newline
49. य॒ज॒मा॒न॒दे॒व॒त्या॑ वै वै य॑जमानदेव॒त्या॑ यजमानदेव॒त्या॑ वै जु॒हूर् जु॒हूर् वै य॑जमानदेव॒त्या॑ यजमानदेव॒त्या॑ वै जु॒हूः । \newline
50. य॒ज॒मा॒न॒दे॒व॒त्येति॑ यजमान - दे॒व॒त्या᳚ । \newline
51. वै जु॒हूर् जु॒हूर् वै वै जु॒हूर् भ्रा॑तृव्यदेव॒त्या᳚ भ्रातृव्यदेव॒त्या॑ जु॒हूर् वै वै जु॒हूर् भ्रा॑तृव्यदेव॒त्या᳚ । \newline
52. जु॒हूर् भ्रा॑तृव्यदेव॒त्या᳚ भ्रातृव्यदेव॒त्या॑ जु॒हूर् जु॒हूर् भ्रा॑तृव्यदेव॒त्यो॑ प॒भृदु॑प॒भृद् भ्रा॑तृव्यदेव॒त्या॑ जु॒हूर् जु॒हूर् भ्रा॑तृव्यदेव॒त्यो॑प॒भृत् । \newline
53. भ्रा॒तृ॒व्य॒दे॒व॒त्यो॑ प॒भृ दु॑प॒भृद् भ्रा॑तृव्यदेव॒त्या᳚ भ्रातृव्यदेव॒त्यो॑ प॒भृद् यद् यदु॑प॒भृद् भ्रा॑तृव्यदेव॒त्या᳚ भ्रातृव्यदेव॒त्यो॑ प॒भृद् यत् । \newline
54. भ्रा॒तृ॒व्य॒दे॒व॒त्येति॑ भ्रातृव्य - दे॒व॒त्या᳚ । \newline
55. उ॒प॒भृद् यद् यदु॑प॒भृ दु॑प॒भृद् यद् द्वे द्वे यदु॑प॒भृ दु॑प॒भृद् यद् द्वे । \newline
56. उ॒प॒भृदित्यु॑प - भृत् । \newline
\pagebreak
\markright{ TS 2.5.9.6  \hfill https://www.vedavms.in \hfill}
\addcontentsline{toc}{section}{ TS 2.5.9.6 }
\section*{ TS 2.5.9.6 }

\textbf{TS 2.5.9.6 } \newline
\textbf{Samhita Paata} \newline

यद्-द्वे इ॑व ब्रू॒याद् भ्रातृ॑व्यमस्मै जनयेद् घृ॒तव॑तीमद्ध्वर्यो॒ स्रुच॒माऽस्य॒स्वेत्या॑ह॒ यज॑मान मे॒वैतेन॑ वर्द्धयति देवा॒युव॒मित्या॑ह दे॒वान्. ह्ये॑षाऽव॑ति वि॒श्ववा॑रा॒मित्या॑ह॒ विश्वꣳ॒॒ ह्ये॑षाऽव॒तीडा॑महै दे॒वाꣳ ई॒डेन्या᳚न्नम॒स्याम॑ नम॒स्यान्॑ यजा॑म य॒ज्ञिया॒नित्या॑हमनु॒ष्या॑ वा ई॒डेन्याः᳚ पि॒तरो॑ नम॒स्या॑ दे॒वा य॒ज्ञिया॑ दे॒वता॑ ए॒व ( ) तद्-य॑थाभा॒गं ॅय॑जति ॥ \newline

\textbf{Pada Paata} \newline

यत् । द्वे इति॑ । इ॒व॒ । ब्रू॒यात् । भ्रातृ॑व्यम् । अ॒स्मै॒ । ज॒न॒ये॒त् । घृ॒तव॑ती॒मिति॑ घृ॒त - व॒ती॒म् । अ॒द्ध्व॒र्यो॒ इति॑ । स्रुच᳚म् । एति॑ । अ॒स्य॒स्व॒ । इति॑ । आ॒ह॒ । यज॑मानम् । ए॒व । ए॒तेन॑ । व॒द्‌र्ध॒य॒ति॒ । दे॒वा॒युव॒मिति॑ देव - युव᳚म् । इति॑ । आ॒ह॒ । दे॒वान् । हि । ए॒षा । अव॑ति । वि॒श्ववा॑रा॒मिति॑ वि॒श्व-वा॒रा॒म् । इति॑ । आ॒ह॒ । विश्व᳚म् । हि । ए॒षा । अव॑ति । ईडा॑महै । दे॒वान् । ई॒डेन्यान्॑ । न॒म॒स्याम॑ । न॒म॒स्यान्॑ । यजा॑म । य॒ज्ञियान्॑ । इति॑ । आ॒ह॒ । म॒नु॒ष्याः᳚ । वै । ई॒डेन्याः᳚ । पि॒तरः॑ । न॒म॒स्याः᳚ । दे॒वाः । य॒ज्ञियाः᳚ । दे॒वताः᳚ । ए॒व ( ) । तत् । य॒था॒भा॒गमिति॑ यथा - भा॒गम् । य॒ज॒ति॒ ॥  \newline


\textbf{Krama Paata} \newline

यद् द्वे । द्वे इ॑व । द्वे इति॒ द्वे । इ॒व॒ ब्रू॒यात् । ब्रू॒याद् भ्रातृ॑व्यम् । भ्रातृ॑व्यमस्मै । अ॒स्मै॒ ज॒न॒ये॒त्॒ । ज॒न॒ये॒द् घृ॒तव॑तीम् । घृ॒तव॑तीमद्ध्वर्यो । घृ॒तव॑ती॒मिति॑ घृ॒त - व॒ती॒म् । अ॒द्ध्व॒र्यो॒ स्रुच᳚म् । अ॒द्ध्व॒र्यो॒ इत्य॑द्ध्वर्यो । स्रुच॒मा । आ ऽस्य॑स्व । अ॒स्य॒स्वेति॑ । इत्या॑ह । आ॒ह॒ यज॑मानम् । यज॑मानमे॒व । ए॒वैतेन॑ । ए॒तेन॑ वर्द्धयति । व॒र्द्ध॒य॒ति॒ दे॒वा॒युव᳚म् । दे॒वा॒युव॒मिति॑ । दे॒वा॒युव॒मिति॑ देव - युव᳚म् । इत्या॑ह । आ॒ह॒ दे॒वान् । दे॒वान्. हि । ह्ये॑षा । ए॒षाऽव॑ति । अव॑ति वि॒श्ववा॑राम् । वि॒श्ववा॑रा॒मिति॑ । वि॒श्ववा॑रा॒मिति॑ वि॒श्व - वा॒रा॒म् । इत्या॑ह । आ॒ह॒ विश्व᳚म् । विश्वꣳ॒॒ हि । ह्ये॑षा । ए॒षाऽव॑ति । अव॒तीडा॑महै । ईडा॑महै दे॒वान् । दे॒वाꣳ ई॒डेन्यान्॑ । ई॒डेन्या᳚न् नम॒स्याम॑ । न॒म॒स्याम॑ नम॒स्यान्॑ । न॒म॒स्यान्॑. यजा॑म । यजा॑म य॒ज्ञियान्॑ । य॒ज्ञिया॒निति॑ । इत्या॑ह । आ॒ह॒ म॒नु॒ष्याः᳚ । म॒नु॒ष्या॑ वै । वा ई॒डेन्याः᳚ । ई॒डेन्याः᳚ पि॒तरः॑ । पि॒तरो॑ नम॒स्याः᳚ । न॒म॒स्या॑ दे॒वाः । दे॒वा य॒ज्ञियाः᳚ । य॒ज्ञिया॑ दे॒वताः᳚ । दे॒वता॑ ए॒व ( ) । ए॒व तत् । तद् य॑थाभा॒गम् । य॒था॒भा॒गम् ॅय॑जति । य॒था॒भा॒गमिति॑ यथा - भा॒गम् । य॒ज॒तीति॑ यजति । \newline

\textbf{Jatai Paata} \newline

1. यद् द्वे द्वे यद् यद् द्वे । \newline
2. द्वे इ॑वे व॒ द्वे द्वे इ॑व । \newline
3. द्वे इति॒ द्वे । \newline
4. इ॒व॒ ब्रू॒याद् ब्रू॒या दि॑वे व ब्रू॒यात् । \newline
5. ब्रू॒याद् भ्रातृ॑व्य॒म् भ्रातृ॑व्यम् ब्रू॒याद् ब्रू॒याद् भ्रातृ॑व्यम् । \newline
6. भ्रातृ॑व्य मस्मा अस्मै॒ भ्रातृ॑व्य॒म् भ्रातृ॑व्य मस्मै । \newline
7. अ॒स्मै॒ ज॒न॒ये॒ज् ज॒न॒ये॒ द॒स्मा॒ अ॒स्मै॒ ज॒न॒ये॒त् । \newline
8. ज॒न॒ये॒द् घृ॒तव॑तीम् घृ॒तव॑तीम् जनयेज् जनयेद् घृ॒तव॑तीम् । \newline
9. घृ॒तव॑ती मद्ध्वर्यो अद्ध्वर्यो घृ॒तव॑तीम् घृ॒तव॑ती मद्ध्वर्यो । \newline
10. घृ॒तव॑ती॒मिति॑ घृ॒त - व॒ती॒म् । \newline
11. अ॒द्ध्व॒र्यो॒ स्रुचꣳ॒॒ स्रुच॑ मद्ध्वर्यो अद्ध्वर्यो॒ स्रुच᳚म् । \newline
12. अ॒द्ध्व॒र्यो॒ इत्य॑द्ध्वर्यो । \newline
13. स्रुच॒ मा स्रुचꣳ॒॒ स्रुच॒ मा । \newline
14. आ ऽस्य॑स्वा स्य॒स्वा ऽस्य॑स्व । \newline
15. अ॒स्य॒स्वे तीत्य॑स्यस्वा स्य॒स्वे ति॑ । \newline
16. इत्या॑हा॒हे तीत्या॑ह । \newline
17. आ॒ह॒ यज॑मानं॒ ॅयज॑मान माहाह॒ यज॑मानम् । \newline
18. यज॑मान मे॒वैव यज॑मानं॒ ॅयज॑मान मे॒व । \newline
19. ए॒वैते नै॒ते नै॒वै वैतेन॑ । \newline
20. ए॒तेन॑ वर्द्धयति वर्द्धय त्ये॒तेनै॒तेन॑ वर्द्धयति । \newline
21. व॒र्द्ध॒य॒ति॒ दे॒वा॒युव॑म् देवा॒युवं॑ ॅवर्द्धयति वर्द्धयति देवा॒युव᳚म् । \newline
22. दे॒वा॒युव॒ मितीति॑ देवा॒युव॑म् देवा॒युव॒ मिति॑ । \newline
23. दे॒वा॒युव॒मिति॑ देव - युव᳚म् । \newline
24. इत्या॑हा॒हे तीत्या॑ह । \newline
25. आ॒ह॒ दे॒वान् दे॒वा ना॑हाह दे॒वान् । \newline
26. दे॒वान्. हि हि दे॒वान् दे॒वान्. हि । \newline
27. ह्ये॑षैषा हि ह्ये॑षा । \newline
28. ए॒षा ऽव॒त्यव॑ त्ये॒षैषा ऽव॑ति । \newline
29. अव॑ति वि॒श्ववा॑रां ॅवि॒श्ववा॑रा॒ मव॒त्यव॑ति वि॒श्ववा॑राम् । \newline
30. वि॒श्ववा॑रा॒ मितीति॑ वि॒श्ववा॑रां ॅवि॒श्ववा॑रा॒ मिति॑ । \newline
31. वि॒श्ववा॑रा॒मिति॑ वि॒श्व - वा॒रा॒म् । \newline
32. इत्या॑हा॒हे तीत्या॑ह । \newline
33. आ॒ह॒ विश्वं॒ ॅविश्व॑ माहाह॒ विश्व᳚म् । \newline
34. विश्वꣳ॒॒ हि हि विश्वं॒ ॅविश्वꣳ॒॒ हि । \newline
35. ह्ये॑षैषा हि ह्ये॑षा । \newline
36. ए॒षा ऽव॒त्यव॑ त्ये॒षैषा ऽव॑ति । \newline
37. अव॒तीडा॑महा॒ ईडा॑महा॒ अव॒ त्यव॒तीडा॑महै । \newline
38. ईडा॑महै दे॒वान् दे॒वाꣳ ईडा॑महा॒ ईडा॑महै दे॒वान् । \newline
39. दे॒वाꣳ ई॒डेन्या॑ नी॒डेन्या᳚न् दे॒वान् दे॒वाꣳ ई॒डेन्यान्॑ । \newline
40. ई॒डेन्या᳚न् नम॒स्याम॑ नम॒स्या मे॒डेन्या॑ नी॒डेन्या᳚न् नम॒स्याम॑ । \newline
41. न॒म॒स्याम॑ नम॒स्या᳚न् नम॒स्या᳚न् नम॒स्याम॑ नम॒स्याम॑ नम॒स्यान्॑ । \newline
42. न॒म॒स्यान्॑. यजा॑म॒ यजा॑म नम॒स्या᳚न् नम॒स्यान्॑. यजा॑म । \newline
43. यजा॑म य॒ज्ञियान्॑. य॒ज्ञिया॒न्॒. यजा॑म॒ यजा॑म य॒ज्ञियान्॑ । \newline
44. य॒ज्ञिया॒ नितीति॑ य॒ज्ञियान्॑. य॒ज्ञिया॒ निति॑ । \newline
45. इत्या॑हा॒हे तीत्या॑ह । \newline
46. आ॒ह॒ म॒नु॒ष्या॑ मनु॒ष्या॑ आहाह मनु॒ष्याः᳚ । \newline
47. म॒नु॒ष्या॑ वै वै म॑नु॒ष्या॑ मनु॒ष्या॑ वै । \newline
48. वा ई॒डेन्या॑ ई॒डेन्या॒ वै वा ई॒डेन्याः᳚ । \newline
49. ई॒डेन्याः᳚ पि॒तरः॑ पि॒तर॑ ई॒डेन्या॑ ई॒डेन्याः᳚ पि॒तरः॑ । \newline
50. पि॒तरो॑ नम॒स्या॑ नम॒स्याः᳚ पि॒तरः॑ पि॒तरो॑ नम॒स्याः᳚ । \newline
51. न॒म॒स्या॑ दे॒वा दे॒वा न॑म॒स्या॑ नम॒स्या॑ दे॒वाः । \newline
52. दे॒वा य॒ज्ञिया॑ य॒ज्ञिया॑ दे॒वा दे॒वा य॒ज्ञियाः᳚ । \newline
53. य॒ज्ञिया॑ दे॒वता॑ दे॒वता॑ य॒ज्ञिया॑ य॒ज्ञिया॑ दे॒वताः᳚ । \newline
54. दे॒वता॑ ए॒वैव दे॒वता॑ दे॒वता॑ ए॒व । \newline
55. ए॒व तत् तदे॒वैव तत् । \newline
56. तद् य॑थाभा॒गं ॅय॑थाभा॒गम् तत् तद् य॑थाभा॒गम् । \newline
57. य॒था॒भा॒गं ॅय॑जति यजति यथाभा॒गं ॅय॑थाभा॒गं ॅय॑जति । \newline
58. य॒था॒भा॒गमिति॑ यथा - भा॒गम् । \newline
59. य॒ज॒तीति॑ यजति । \newline

\textbf{Ghana Paata } \newline

1. यद् द्वे द्वे यद् यद् द्वे इ॑वे व॒ द्वे यद् यद् द्वे इ॑व । \newline
2. द्वे इ॑वे व॒ द्वे द्वे इ॑व ब्रू॒याद् ब्रू॒यादि॑व॒ द्वे द्वे इ॑व ब्रू॒यात् । \newline
3. द्वे इति॒ द्वे । \newline
4. इ॒व॒ ब्रू॒याद् ब्रू॒यादि॑वे व ब्रू॒याद् भ्रातृ॑व्य॒म् भ्रातृ॑व्यम् ब्रू॒यादि॑वे व ब्रू॒याद् भ्रातृ॑व्यम् । \newline
5. ब्रू॒याद् भ्रातृ॑व्य॒म् भ्रातृ॑व्यम् ब्रू॒याद् ब्रू॒याद् भ्रातृ॑व्य मस्मा अस्मै॒ भ्रातृ॑व्यम् ब्रू॒याद् ब्रू॒याद् भ्रातृ॑व्य मस्मै । \newline
6. भ्रातृ॑व्य मस्मा अस्मै॒ भ्रातृ॑व्य॒म् भ्रातृ॑व्य मस्मै जनयेज् जनयेदस्मै॒ भ्रातृ॑व्य॒म् भ्रातृ॑व्य मस्मै जनयेत् । \newline
7. अ॒स्मै॒ ज॒न॒ये॒ज् ज॒न॒ये॒द॒स्मा॒ अ॒स्मै॒ ज॒न॒ये॒द् घृ॒तव॑तीम् घृ॒तव॑तीम् जनयेदस्मा अस्मै जनयेद् घृ॒तव॑तीम् । \newline
8. ज॒न॒ये॒द् घृ॒तव॑तीम् घृ॒तव॑तीम् जनयेज् जनयेद् घृ॒तव॑ती मद्ध्वर्यो अद्ध्वर्यो घृ॒तव॑तीम् जनयेज् जनयेद् घृ॒तव॑ती मद्ध्वर्यो । \newline
9. घृ॒तव॑ती मद्ध्वर्यो अद्ध्वर्यो घृ॒तव॑तीम् घृ॒तव॑ती मद्ध्वर्यो॒ स्रुचꣳ॒॒ स्रुच॑ मद्ध्वर्यो घृ॒तव॑तीम् घृ॒तव॑ती मद्ध्वर्यो॒ स्रुच᳚म् । \newline
10. घृ॒तव॑ती॒मिति॑ घृ॒त - व॒ती॒म् । \newline
11. अ॒द्ध्व॒र्यो॒ स्रुचꣳ॒॒ स्रुच॑ मद्ध्वर्यो अद्ध्वर्यो॒ स्रुच॒ मा स्रुच॑ मद्ध्वर्यो अद्ध्वर्यो॒ स्रुच॒ मा । \newline
12. अ॒द्ध्व॒र्यो॒ इत्य॑द्ध्वर्यो । \newline
13. स्रुच॒ मा स्रुचꣳ॒॒ स्रुच॒ मा ऽस्य॑स्वा स्य॒स्वा स्रुचꣳ॒॒ स्रुच॒ मा ऽस्य॑स्व । \newline
14. आ ऽस्य॑स्वा स्य॒स्वा ऽस्य॒स्वे तीत्य॑स्य॒स्वा ऽस्य॒स्वे ति॑ । \newline
15. अ॒स्य॒स्वे तीत्य॑स्यस्वा स्य॒स्वे त्या॑हा॒हे त्य॑स्यस्वा स्य॒स्वे त्या॑ह । \newline
16. इत्या॑हा॒हे तीत्या॑ह॒ यज॑मानं॒ ॅयज॑मान मा॒हे तीत्या॑ह॒ यज॑मानम् । \newline
17. आ॒ह॒ यज॑मानं॒ ॅयज॑मान माहाह॒ यज॑मान मे॒वैव यज॑मान माहाह॒ यज॑मान मे॒व । \newline
18. यज॑मान मे॒वैव यज॑मानं॒ ॅयज॑मान मे॒वैते नै॒तेनै॒व यज॑मानं॒ ॅयज॑मान मे॒वैतेन॑ । \newline
19. ए॒वैतेनै॒ते नै॒वैवैतेन॑ वर्द्धयति वर्द्धय त्ये॒ते नै॒वैवैतेन॑ वर्द्धयति । \newline
20. ए॒तेन॑ वर्द्धयति वर्द्धय त्ये॒तेनै॒तेन॑ वर्द्धयति देवा॒युव॑म् देवा॒युवं॑ ॅवर्द्ध यत्ये॒तेनै॒तेन॑ वर्द्धयति देवा॒युव᳚म् । \newline
21. व॒र्द्ध॒य॒ति॒ दे॒वा॒युव॑म् देवा॒युवं॑ ॅवर्द्धयति वर्द्धयति देवा॒युव॒ मितीति॑ देवा॒युवं॑ ॅवर्द्धयति वर्द्धयति देवा॒युव॒ मिति॑ । \newline
22. दे॒वा॒युव॒ मितीति॑ देवा॒युव॑म् देवा॒युव॒ मित्या॑हा॒हे ति॑ देवा॒युव॑म् देवा॒युव॒ मित्या॑ह । \newline
23. दे॒वा॒युव॒मिति॑ देव - युव᳚म् । \newline
24. इत्या॑हा॒हे तीत्या॑ह दे॒वान् दे॒वा ना॒हे तीत्या॑ह दे॒वान् । \newline
25. आ॒ह॒ दे॒वान् दे॒वा ना॑हाह दे॒वान्. हि हि दे॒वा ना॑हाह दे॒वान्. हि । \newline
26. दे॒वान्. हि हि दे॒वान् दे॒वान्. ह्ये॑षैषा हि दे॒वान् दे॒वान्. ह्ये॑षा । \newline
27. ह्ये॑षैषा हि ह्ये॑षा ऽव॒त्यव॑ त्ये॒षा हि ह्ये॑षा ऽव॑ति । \newline
28. ए॒षा ऽव॒त्यव॑ त्ये॒षैषा ऽव॑ति वि॒श्ववा॑रां ॅवि॒श्ववा॑रा॒ मव॑त्ये॒षैषा ऽव॑ति वि॒श्ववा॑राम् । \newline
29. अव॑ति वि॒श्ववा॑रां ॅवि॒श्ववा॑रा॒ मव॒त्यव॑ति वि॒श्ववा॑रा॒ मितीति॑ वि॒श्ववा॑रा॒ मव॒त्यव॑ति वि॒श्ववा॑रा॒ मिति॑ । \newline
30. वि॒श्ववा॑रा॒ मितीति॑ वि॒श्ववा॑रां ॅवि॒श्ववा॑रा॒ मित्या॑हा॒हे ति॑ वि॒श्ववा॑रां ॅवि॒श्ववा॑रा॒ मित्या॑ह । \newline
31. वि॒श्ववा॑रा॒मिति॑ वि॒श्व - वा॒रा॒म् । \newline
32. इत्या॑हा॒हे तीत्या॑ह॒ विश्वं॒ ॅविश्व॑ मा॒हे तीत्या॑ह॒ विश्व᳚म् । \newline
33. आ॒ह॒ विश्वं॒ ॅविश्व॑ माहाह॒ विश्वꣳ॒॒ हि हि विश्व॑ माहाह॒ विश्वꣳ॒॒ हि । \newline
34. विश्वꣳ॒॒ हि हि विश्वं॒ ॅविश्वꣳ॒॒ ह्ये॑षैषा हि विश्वं॒ ॅविश्वꣳ॒॒ ह्ये॑षा । \newline
35. ह्ये॑षैषा हि ह्ये॑षा ऽव॒त्यव॑ त्ये॒षा हि ह्ये॑षा ऽव॑ति । \newline
36. ए॒षा ऽव॒त्यव॑ त्ये॒षैषा ऽव॒तीडा॑महा॒ ईडा॑महा॒ अव॑ त्ये॒षैषा ऽव॒तीडा॑महै । \newline
37. अव॒तीडा॑महा॒ ईडा॑महा॒ अव॒ त्यव॒तीडा॑महै दे॒वान् दे॒वाꣳ ईडा॑महा॒ अव॒ त्यव॒तीडा॑महै दे॒वान् । \newline
38. ईडा॑महै दे॒वान् दे॒वाꣳ ईडा॑महा॒ ईडा॑महै दे॒वाꣳ ई॒डेन्या॑ नी॒डेन्या᳚न् दे॒वाꣳ ईडा॑महा॒ ईडा॑महै दे॒वाꣳ ई॒डेन्यान्॑ । \newline
39. दे॒वाꣳ ई॒डेन्या॑ नी॒डेन्या᳚न् दे॒वान् दे॒वाꣳ ई॒डेन्या᳚न् नम॒स्याम॑ नम॒स्या मे॒डेन्या᳚न् दे॒वान् दे॒वाꣳ ई॒डेन्या᳚न् नम॒स्याम॑ । \newline
40. ई॒डेन्या᳚न् नम॒स्याम॑ नम॒स्या मे॒डेन्या॑ नी॒डेन्या᳚न् नम॒स्याम॑ नम॒स्या᳚न् नम॒स्या᳚न् नम॒स्या मे॒डेन्या॑ नी॒डेन्या᳚न् नम॒स्याम॑ नम॒स्यान्॑ । \newline
41. न॒म॒स्याम॑ नम॒स्या᳚न् नम॒स्या᳚न् नम॒स्याम॑ नम॒स्याम॑ नम॒स्यान्॑. यजा॑म॒ यजा॑म नम॒स्या᳚न् नम॒स्याम॑ नम॒स्याम॑ नम॒स्यान्॑. यजा॑म । \newline
42. न॒म॒स्यान्॑. यजा॑म॒ यजा॑म नम॒स्या᳚न् नम॒स्यान्॑. यजा॑म य॒ज्ञियान्॑. य॒ज्ञियान्॑. यजा॑म नम॒स्या᳚न् नम॒स्यान्॑. यजा॑म य॒ज्ञियान्॑ । \newline
43. यजा॑म य॒ज्ञियान्॑. य॒ज्ञिया॒न्॒. यजा॑म॒ यजा॑म य॒ज्ञिया॒ नितीति॑ य॒ज्ञिया॒न्॒. यजा॑म॒ यजा॑म य॒ज्ञिया॒ निति॑ । \newline
44. य॒ज्ञिया॒ नितीति॑ य॒ज्ञियान्॑. य॒ज्ञिया॒ नित्या॑हा॒हे ति॑ य॒ज्ञियान्॑. य॒ज्ञिया॒ नित्या॑ह । \newline
45. इत्या॑हा॒हे तीत्या॑ह मनु॒ष्या॑ मनु॒ष्या॑ आ॒हे तीत्या॑ह मनु॒ष्याः᳚ । \newline
46. आ॒ह॒ म॒नु॒ष्या॑ मनु॒ष्या॑ आहाह मनु॒ष्या॑ वै वै म॑नु॒ष्या॑ आहाह मनु॒ष्या॑ वै । \newline
47. म॒नु॒ष्या॑ वै वै म॑नु॒ष्या॑ मनु॒ष्या॑ वा ई॒डेन्या॑ ई॒डेन्या॒ वै म॑नु॒ष्या॑ मनु॒ष्या॑ वा ई॒डेन्याः᳚ । \newline
48. वा ई॒डेन्या॑ ई॒डेन्या॒ वै वा ई॒डेन्याः᳚ पि॒तरः॑ पि॒तर॑ ई॒डेन्या॒ वै वा ई॒डेन्याः᳚ पि॒तरः॑ । \newline
49. ई॒डेन्याः᳚ पि॒तरः॑ पि॒तर॑ ई॒डेन्या॑ ई॒डेन्याः᳚ पि॒तरो॑ नम॒स्या॑ नम॒स्याः᳚ पि॒तर॑ ई॒डेन्या॑ ई॒डेन्याः᳚ पि॒तरो॑ नम॒स्याः᳚ । \newline
50. पि॒तरो॑ नम॒स्या॑ नम॒स्याः᳚ पि॒तरः॑ पि॒तरो॑ नम॒स्या॑ दे॒वा दे॒वा न॑म॒स्याः᳚ पि॒तरः॑ पि॒तरो॑ नम॒स्या॑ दे॒वाः । \newline
51. न॒म॒स्या॑ दे॒वा दे॒वा न॑म॒स्या॑ नम॒स्या॑ दे॒वा य॒ज्ञिया॑ य॒ज्ञिया॑ दे॒वा न॑म॒स्या॑ नम॒स्या॑ दे॒वा य॒ज्ञियाः᳚ । \newline
52. दे॒वा य॒ज्ञिया॑ य॒ज्ञिया॑ दे॒वा दे॒वा य॒ज्ञिया॑ दे॒वता॑ दे॒वता॑ य॒ज्ञिया॑ दे॒वा दे॒वा य॒ज्ञिया॑ दे॒वताः᳚ । \newline
53. य॒ज्ञिया॑ दे॒वता॑ दे॒वता॑ य॒ज्ञिया॑ य॒ज्ञिया॑ दे॒वता॑ ए॒वैव दे॒वता॑ य॒ज्ञिया॑ य॒ज्ञिया॑ दे॒वता॑ ए॒व । \newline
54. दे॒वता॑ ए॒वैव दे॒वता॑ दे॒वता॑ ए॒व तत् तदे॒व दे॒वता॑ दे॒वता॑ ए॒व तत् । \newline
55. ए॒व तत् तदे॒वैव तद् य॑थाभा॒गं ॅय॑थाभा॒गम् तदे॒वैव तद् य॑थाभा॒गम् । \newline
56. तद् य॑थाभा॒गं ॅय॑थाभा॒गम् तत् तद् य॑थाभा॒गं ॅय॑जति यजति यथाभा॒गम् तत् तद् य॑थाभा॒गं ॅय॑जति । \newline
57. य॒था॒भा॒गं ॅय॑जति यजति यथाभा॒गं ॅय॑थाभा॒गं ॅय॑जति । \newline
58. य॒था॒भा॒गमिति॑ यथा - भा॒गम् । \newline
59. य॒ज॒तीति॑ यजति । \newline
\pagebreak
\markright{ TS 2.5.10.1  \hfill https://www.vedavms.in \hfill}
\addcontentsline{toc}{section}{ TS 2.5.10.1 }
\section*{ TS 2.5.10.1 }

\textbf{TS 2.5.10.1 } \newline
\textbf{Samhita Paata} \newline

त्रीꣳस्तृ॒चाननु॑ ब्रूयाद्-राज॒न्य॑स्य॒ त्रयो॒ वा अ॒न्ये रा॑ज॒न्या᳚त् पुरु॑षा ब्राह्म॒णो वैश्यः॑ शू॒द्रस्ताने॒वास्मा॒ अनु॑कान् करोति॒ पञ्च॑द॒शानु॑ ब्रूयाद्-राज॒न्य॑स्य पञ्चद॒शो वै रा॑ज॒न्यः॑ स्व ए॒वैनꣳ॒॒ स्तोमे॒ प्रति॑ष्ठापयति त्रि॒ष्टुभा॒ परि॑ दद्ध्यादिन्द्रि॒यं ॅवै त्रि॒ष्टुगि॑न्द्रि॒यका॑मः॒ खलु॒ वै रा॑ज॒न्यो॑ यजते त्रि॒ष्टुभै॒वास्मा॑ इन्द्रि॒यं परि॑ गृह्णाति॒ यदि॑ का॒मये॑त-  [  ] \newline

\textbf{Pada Paata} \newline

त्रीन् । तृ॒चान् । अन्विति॑ । ब्रू॒या॒त् । रा॒ज॒न्य॑स्य । त्रयः॑ । वै । अ॒न्ये । रा॒ज॒न्या᳚त् । पुरु॑षाः । ब्रा॒ह्म॒णः । वैश्यः॑ । शू॒द्रः । तान् । ए॒व । अ॒स्मै॒ । अनु॑का॒नित्यनु॑ - का॒न् । क॒रो॒ति॒ । पञ्च॑द॒शेति॒ पञ्च॑ - द॒श॒ । अन्विति॑ । ब्रू॒या॒त् । रा॒ज॒न्य॑स्य । प॒ञ्च॒द॒श इति॑ पञ्च-द॒शः । वै । रा॒ज॒न्यः॑ । स्वे । ए॒व । ए॒न॒म् । स्तोमे᳚ । प्रतीति॑ । स्था॒प॒य॒ति॒ । त्रि॒ष्टुभा᳚ । परीति॑ । द॒द्ध्या॒त् । इ॒न्द्रि॒यम् । वै । त्रि॒ष्टुक् । इ॒न्द्रि॒यका॑म॒ इती᳚न्द्रि॒य - का॒मः॒ । खलु॑ । वै । रा॒ज॒न्यः॑ । य॒ज॒ते॒ । त्रि॒ष्टुभा᳚ । ए॒व । अ॒स्मै॒ । इ॒न्द्रि॒यम् । परीति॑ । गृ॒ह्णा॒ति॒ । यदि॑ । का॒मये॑त ।  \newline


\textbf{Krama Paata} \newline

त्रीꣳस्तृ॒चान् । तृ॒चाननु॑ । अनु॑ ब्रूयात् । ब्रू॒या॒द् रा॒ज॒न्य॑स्य । रा॒ज॒न्य॑स्य॒ त्रयः॑ । त्रयो॒ वै । वा अ॒न्ये । अ॒न्ये रा॑ज॒न्या᳚त् । रा॒ज॒न्या᳚त् पुरु॑षाः । पुरु॑षा ब्राह्म॒णः । ब्रा॒ह्म॒णो वैश्यः॑ । वैश्यः॑ शू॒द्रः । शू॒द्रस्तान् । ताने॒व । ए॒वास्मै᳚ । अ॒स्मा॒ अनु॑कान् । अनु॑कान् करोति । अनु॑का॒नित्यनु॑ - का॒न्॒ । क॒रो॒ति॒ पञ्च॑दश । पञ्च॑द॒शानु॑ । पञ्च॑द॒शेति॒ पञ्च॑ - द॒श॒ । अनु॑ ब्रूयात् । ब्रू॒या॒द् रा॒ज॒न्य॑स्य । रा॒ज॒न्य॑स्य पञ्चद॒शः । प॒ञ्च॒द॒शो वै । प॒ञ्च॒द॒श इति॑ पञ्च - द॒शः । वै रा॑ज॒न्यः॑ । रा॒ज॒न्यः॑ स्वे । स्व ए॒व । ए॒वैन᳚म् । ए॒नꣳ॒॒ स्तोमे᳚ । स्तोमे॒ प्रति॑ । प्रति॑ ष्ठापयति । स्था॒प॒य॒ति॒ त्रि॒ष्टुभा᳚ । त्रि॒ष्टुभा॒ परि॑ । परि॑ दद्ध्यात् । द॒द्ध्या॒दि॒न्द्रि॒यम् । इ॒न्द्रि॒यम् ॅवै । वै त्रि॒ष्टुक् । त्रि॒ष्टुगि॑न्द्रि॒यका॑मः । इ॒न्द्रि॒यका॑मः॒ खलु॑ । इ॒न्द्रि॒यका॑म॒ इती᳚न्द्रि॒य - का॒मः॒ । खलु॒ वै । वै रा॑ज॒न्यः॑ । रा॒ज॒न्यो॑ यजते । य॒ज॒ते॒ त्रि॒ष्टुभा᳚ । त्रि॒ष्टुभै॒व । ए॒वास्मै᳚ । अ॒स्मा॒ इ॒न्द्रि॒यम् । इ॒न्द्रि॒यम् परि॑ । परि॑ गृह्णाति । गृ॒ह्णा॒ति॒ यदि॑ । यदि॑ का॒मये॑त । का॒मये॑त ब्रह्मवर्च॒सम् \newline

\textbf{Jatai Paata} \newline

1. त्रीꣳ स्तृ॒चान् तृ॒चान् त्रीꣳ स्त्रीꣳ स्तृ॒चान् । \newline
2. तृ॒चा नन्वनु॑ तृ॒चान् तृ॒चा ननु॑ । \newline
3. अनु॑ ब्रूयाद् ब्रूया॒ दन्वनु॑ ब्रूयात् । \newline
4. ब्रू॒या॒द् रा॒ज॒न्य॑स्य राज॒न्य॑स्य ब्रूयाद् ब्रूयाद् राज॒न्य॑स्य । \newline
5. रा॒ज॒न्य॑स्य॒ त्रय॒स्त्रयो॑ राज॒न्य॑स्य राज॒न्य॑स्य॒ त्रयः॑ । \newline
6. त्रयो॒ वै वै त्रय॒ स्त्रयो॒ वै । \newline
7. वा अ॒न्ये᳚ ऽन्ये वै वा अ॒न्ये । \newline
8. अ॒न्ये रा॑ज॒न्या᳚द् राज॒न्या॑ द॒न्ये᳚ ऽन्ये रा॑ज॒न्या᳚त् । \newline
9. रा॒ज॒न्या᳚त् पुरु॑षाः॒ पुरु॑षा राज॒न्या᳚द् राज॒न्या᳚त् पुरु॑षाः । \newline
10. पुरु॑षा ब्राह्म॒णो ब्रा᳚ह्म॒णः पुरु॑षाः॒ पुरु॑षा ब्राह्म॒णः । \newline
11. ब्रा॒ह्म॒णो वैश्यो॒ वैश्यो᳚ ब्राह्म॒णो ब्रा᳚ह्म॒णो वैश्यः॑ । \newline
12. वैश्यः॑ शू॒द्रः शू॒द्रो वैश्यो॒ वैश्यः॑ शू॒द्रः । \newline
13. शू॒द्र स्ताꣳ स्ताञ् छू॒द्रः शू॒द्र स्तान् । \newline
14. ता ने॒वैव ताꣳ स्ता ने॒व । \newline
15. ए॒वास्मा॑ अस्मा ए॒वैवास्मै᳚ । \newline
16. अ॒स्मा॒ अनु॑का॒ ननु॑का नस्मा अस्मा॒ अनु॑कान् । \newline
17. अनु॑कान् करोति करो॒त्यनु॑का॒ ननु॑कान् करोति । \newline
18. अनु॑का॒नित्यनु॑ - का॒न् । \newline
19. क॒रो॒ति॒ पञ्च॑दश॒ पञ्च॑दश करोति करोति॒ पञ्च॑दश । \newline
20. पञ्च॑द॒शान्वनु॒ पञ्च॑दश॒ पञ्च॑द॒शानु॑ । \newline
21. पञ्च॑द॒शेति॒ पञ्च॑ - द॒श॒ । \newline
22. अनु॑ ब्रूयाद् ब्रूया॒ दन्वनु॑ ब्रूयात् । \newline
23. ब्रू॒या॒द् रा॒ज॒न्य॑स्य राज॒न्य॑स्य ब्रूयाद् ब्रूयाद् राज॒न्य॑स्य । \newline
24. रा॒ज॒न्य॑स्य पञ्चद॒शः प॑ञ्चद॒शो रा॑ज॒न्य॑स्य राज॒न्य॑स्य पञ्चद॒शः । \newline
25. प॒ञ्च॒द॒शो वै वै प॑ञ्चद॒शः प॑ञ्चद॒शो वै । \newline
26. प॒ञ्च॒द॒श इति॑ पञ्च - द॒शः । \newline
27. वै रा॑ज॒न्यो॑ राज॒न्यो॑ वै वै रा॑ज॒न्यः॑ । \newline
28. रा॒ज॒न्यः॑ स्वे स्वे रा॑ज॒न्यो॑ राज॒न्यः॑ स्वे । \newline
29. स्व ए॒वैव स्वे स्व ए॒व । \newline
30. ए॒वैन॑ मेन मे॒वैवैन᳚म् । \newline
31. ए॒नꣳ॒॒ स्तोमे॒ स्तोम॑ एन मेनꣳ॒॒ स्तोमे᳚ । \newline
32. स्तोमे॒ प्रति॒ प्रति॒ स्तोमे॒ स्तोमे॒ प्रति॑ । \newline
33. प्रति॑ ष्ठापयति स्थापयति॒ प्रति॒ प्रति॑ ष्ठापयति । \newline
34. स्था॒प॒य॒ति॒ त्रि॒ष्टुभा᳚ त्रि॒ष्टुभा᳚ स्थापयति स्थापयति त्रि॒ष्टुभा᳚ । \newline
35. त्रि॒ष्टुभा॒ परि॒ परि॑ त्रि॒ष्टुभा᳚ त्रि॒ष्टुभा॒ परि॑ । \newline
36. परि॑ दद्ध्याद् दद्ध्या॒त् परि॒ परि॑ दद्ध्यात् । \newline
37. द॒द्ध्या॒ दि॒न्द्रि॒य मि॑न्द्रि॒यम् द॑द्ध्याद् दद्ध्या दिन्द्रि॒यम् । \newline
38. इ॒न्द्रि॒यं ॅवै वा इ॑न्द्रि॒य मि॑न्द्रि॒यं ॅवै । \newline
39. वै त्रि॒ष्टुक् त्रि॒ष्टुग् वै वै त्रि॒ष्टुक् । \newline
40. त्रि॒ष्टु गि॑न्द्रि॒यका॑म इन्द्रि॒यका॑म स्त्रि॒ष्टुक् त्रि॒ष्टु गि॑न्द्रि॒यका॑मः । \newline
41. इ॒न्द्रि॒यका॑मः॒ खलु॒ खल्वि॑न्द्रि॒यका॑म इन्द्रि॒यका॑मः॒ खलु॑ । \newline
42. इ॒न्द्रि॒यका॑म॒ इती᳚न्द्रि॒य - का॒मः॒ । \newline
43. खलु॒ वै वै खलु॒ खलु॒ वै । \newline
44. वै रा॑ज॒न्यो॑ राज॒न्यो॑ वै वै रा॑ज॒न्यः॑ । \newline
45. रा॒ज॒न्यो॑ यजते यजते राज॒न्यो॑ राज॒न्यो॑ यजते । \newline
46. य॒ज॒ते॒ त्रि॒ष्टुभा᳚ त्रि॒ष्टुभा॑ यजते यजते त्रि॒ष्टुभा᳚ । \newline
47. त्रि॒ष्टुभै॒वैव त्रि॒ष्टुभा᳚ त्रि॒ष्टुभै॒व । \newline
48. ए॒वास्मा॑ अस्मा ए॒वैवास्मै᳚ । \newline
49. अ॒स्मा॒ इ॒न्द्रि॒य मि॑न्द्रि॒य म॑स्मा अस्मा इन्द्रि॒यम् । \newline
50. इ॒न्द्रि॒यम् परि॒ परी᳚न्द्रि॒य मि॑न्द्रि॒यम् परि॑ । \newline
51. परि॑ गृह्णाति गृह्णाति॒ परि॒ परि॑ गृह्णाति । \newline
52. गृ॒ह्णा॒ति॒ यदि॒ यदि॑ गृह्णाति गृह्णाति॒ यदि॑ । \newline
53. यदि॑ का॒मये॑त का॒मये॑त॒ यदि॒ यदि॑ का॒मये॑त । \newline
54. का॒मये॑त ब्रह्मवर्च॒सम् ब्र॑ह्मवर्च॒सम् का॒मये॑त का॒मये॑त ब्रह्मवर्च॒सम् । \newline

\textbf{Ghana Paata } \newline

1. त्रीꣳ स्तृ॒चान् तृ॒चान् त्रीꣳ स्त्रीꣳ स्तृ॒चा नन्वनु॑ तृ॒चान् त्रीꣳ स्त्रीꣳ स्तृ॒चा ननु॑ । \newline
2. तृ॒चा नन्वनु॑ तृ॒चान् तृ॒चा ननु॑ ब्रूयाद् ब्रूया॒दनु॑ तृ॒चान् तृ॒चा ननु॑ ब्रूयात् । \newline
3. अनु॑ ब्रूयाद् ब्रूया॒दन्वनु॑ ब्रूयाद् राज॒न्य॑स्य राज॒न्य॑स्य ब्रूया॒दन्वनु॑ ब्रूयाद् राज॒न्य॑स्य । \newline
4. ब्रू॒या॒द् रा॒ज॒न्य॑स्य राज॒न्य॑स्य ब्रूयाद् ब्रूयाद् राज॒न्य॑स्य॒ त्रय॒ स्त्रयो॑ राज॒न्य॑स्य ब्रूयाद् ब्रूयाद् राज॒न्य॑स्य॒ त्रयः॑ । \newline
5. रा॒ज॒न्य॑स्य॒ त्रय॒ स्त्रयो॑ राज॒न्य॑स्य राज॒न्य॑स्य॒ त्रयो॒ वै वै त्रयो॑ राज॒न्य॑स्य राज॒न्य॑स्य॒ त्रयो॒ वै । \newline
6. त्रयो॒ वै वै त्रय॒ स्त्रयो॒ वा अ॒न्ये᳚ ऽन्ये वै त्रय॒ स्त्रयो॒ वा अ॒न्ये । \newline
7. वा अ॒न्ये᳚ ऽन्ये वै वा अ॒न्ये रा॑ज॒न्या᳚द् राज॒न्या॑ द॒न्ये वै वा अ॒न्ये रा॑ज॒न्या᳚त् । \newline
8. अ॒न्ये रा॑ज॒न्या᳚द् राज॒न्या॑ द॒न्ये᳚ ऽन्ये रा॑ज॒न्या᳚त् पुरु॑षाः॒ पुरु॑षा राज॒न्या॑ द॒न्ये᳚ ऽन्ये रा॑ज॒न्या᳚त् पुरु॑षाः । \newline
9. रा॒ज॒न्या᳚त् पुरु॑षाः॒ पुरु॑षा राज॒न्या᳚द् राज॒न्या᳚त् पुरु॑षा ब्राह्म॒णो ब्रा᳚ह्म॒णः पुरु॑षा राज॒न्या᳚द् राज॒न्या᳚त् पुरु॑षा ब्राह्म॒णः । \newline
10. पुरु॑षा ब्राह्म॒णो ब्रा᳚ह्म॒णः पुरु॑षाः॒ पुरु॑षा ब्राह्म॒णो वैश्यो॒ वैश्यो᳚ ब्राह्म॒णः पुरु॑षाः॒ पुरु॑षा ब्राह्म॒णो वैश्यः॑ । \newline
11. ब्रा॒ह्म॒णो वैश्यो॒ वैश्यो᳚ ब्राह्म॒णो ब्रा᳚ह्म॒णो वैश्यः॑ शू॒द्रः शू॒द्रो वैश्यो᳚ ब्राह्म॒णो ब्रा᳚ह्म॒णो वैश्यः॑ शू॒द्रः । \newline
12. वैश्यः॑ शू॒द्रः शू॒द्रो वैश्यो॒ वैश्यः॑ शू॒द्र स्ताꣳ स्ताञ् छू॒द्रो वैश्यो॒ वैश्यः॑ शू॒द्र स्तान् । \newline
13. शू॒द्र स्ताꣳ स्ताञ् छू॒द्रः शू॒द्र स्ता ने॒वैव ताञ् छू॒द्रः शू॒द्र स्ता ने॒व । \newline
14. ता ने॒वैव ताꣳ स्ता ने॒वास्मा॑ अस्मा ए॒व ताꣳ स्ता ने॒वास्मै᳚ । \newline
15. ए॒वास्मा॑ अस्मा ए॒वैवास्मा॒ अनु॑का॒ ननु॑का नस्मा ए॒वैवास्मा॒ अनु॑कान् । \newline
16. अ॒स्मा॒ अनु॑का॒ ननु॑का नस्मा अस्मा॒ अनु॑कान् करोति करो॒ त्यनु॑का नस्मा अस्मा॒ अनु॑कान् करोति । \newline
17. अनु॑कान् करोति करो॒ त्यनु॑का॒ ननु॑कान् करोति॒ पञ्च॑दश॒ पञ्च॑दश करो॒ त्यनु॑का॒ ननु॑कान् करोति॒ पञ्च॑दश । \newline
18. अनु॑का॒नित्यनु॑ - का॒न् । \newline
19. क॒रो॒ति॒ पञ्च॑दश॒ पञ्च॑दश करोति करोति॒ पञ्च॑द॒शा न्वनु॒ पञ्च॑दश करोति करोति॒ पञ्च॑द॒शानु॑ । \newline
20. पञ्च॑द॒शा न्वनु॒ पञ्च॑दश॒ पञ्च॑द॒शानु॑ ब्रूयाद् ब्रूया॒दनु॒ पञ्च॑दश॒ पञ्च॑द॒शानु॑ ब्रूयात् । \newline
21. पञ्च॑द॒शेति॒ पञ्च॑ - द॒श॒ । \newline
22. अनु॑ ब्रूयाद् ब्रूया॒ दन्वनु॑ ब्रूयाद् राज॒न्य॑स्य राज॒न्य॑स्य ब्रूया॒ दन्वनु॑ ब्रूयाद् राज॒न्य॑स्य । \newline
23. ब्रू॒या॒द् रा॒ज॒न्य॑स्य राज॒न्य॑स्य ब्रूयाद् ब्रूयाद् राज॒न्य॑स्य पञ्चद॒शः प॑ञ्चद॒शो रा॑ज॒न्य॑स्य ब्रूयाद् ब्रूयाद् राज॒न्य॑स्य पञ्चद॒शः । \newline
24. रा॒ज॒न्य॑स्य पञ्चद॒शः प॑ञ्चद॒शो रा॑ज॒न्य॑स्य राज॒न्य॑स्य पञ्चद॒शो वै वै प॑ञ्चद॒शो रा॑ज॒न्य॑स्य राज॒न्य॑स्य पञ्चद॒शो वै । \newline
25. प॒ञ्च॒द॒शो वै वै प॑ञ्चद॒शः प॑ञ्चद॒शो वै रा॑ज॒न्यो॑ राज॒न्यो॑ वै प॑ञ्चद॒शः प॑ञ्चद॒शो वै रा॑ज॒न्यः॑ । \newline
26. प॒ञ्च॒द॒श इति॑ पञ्च - द॒शः । \newline
27. वै रा॑ज॒न्यो॑ राज॒न्यो॑ वै वै रा॑ज॒न्यः॑ स्वे स्वे रा॑ज॒न्यो॑ वै वै रा॑ज॒न्यः॑ स्वे । \newline
28. रा॒ज॒न्यः॑ स्वे स्वे रा॑ज॒न्यो॑ राज॒न्यः॑ स्व ए॒वैव स्वे रा॑ज॒न्यो॑ राज॒न्यः॑ स्व ए॒व । \newline
29. स्व ए॒वैव स्वे स्व ए॒वैन॑ मेन मे॒व स्वे स्व ए॒वैन᳚म् । \newline
30. ए॒वैन॑ मेन मे॒वैवैनꣳ॒॒ स्तोमे॒ स्तोम॑ एन मे॒वैवैनꣳ॒॒ स्तोमे᳚ । \newline
31. ए॒नꣳ॒॒ स्तोमे॒ स्तोम॑ एन मेनꣳ॒॒ स्तोमे॒ प्रति॒ प्रति॒ स्तोम॑ एन मेनꣳ॒॒ स्तोमे॒ प्रति॑ । \newline
32. स्तोमे॒ प्रति॒ प्रति॒ स्तोमे॒ स्तोमे॒ प्रति॑ ष्ठापयति स्थापयति॒ प्रति॒ स्तोमे॒ स्तोमे॒ प्रति॑ ष्ठापयति । \newline
33. प्रति॑ ष्ठापयति स्थापयति॒ प्रति॒ प्रति॑ ष्ठापयति त्रि॒ष्टुभा᳚ त्रि॒ष्टुभा᳚ स्थापयति॒ प्रति॒ प्रति॑ ष्ठापयति त्रि॒ष्टुभा᳚ । \newline
34. स्था॒प॒य॒ति॒ त्रि॒ष्टुभा᳚ त्रि॒ष्टुभा᳚ स्थापयति स्थापयति त्रि॒ष्टुभा॒ परि॒ परि॑ त्रि॒ष्टुभा᳚ स्थापयति स्थापयति त्रि॒ष्टुभा॒ परि॑ । \newline
35. त्रि॒ष्टुभा॒ परि॒ परि॑ त्रि॒ष्टुभा᳚ त्रि॒ष्टुभा॒ परि॑ दद्ध्याद् दद्ध्या॒त् परि॑ त्रि॒ष्टुभा᳚ त्रि॒ष्टुभा॒ परि॑ दद्ध्यात् । \newline
36. परि॑ दद्ध्याद् दद्ध्या॒त् परि॒ परि॑ दद्ध्या दिन्द्रि॒य मि॑न्द्रि॒यम् द॑द्ध्या॒त् परि॒ परि॑ दद्ध्या दिन्द्रि॒यम् । \newline
37. द॒द्ध्या॒ दि॒न्द्रि॒य मि॑न्द्रि॒यम् द॑द्ध्याद् दद्ध्या दिन्द्रि॒यं ॅवै वा इ॑न्द्रि॒यम् द॑द्ध्याद् दद्ध्या दिन्द्रि॒यं ॅवै । \newline
38. इ॒न्द्रि॒यं ॅवै वा इ॑न्द्रि॒य मि॑न्द्रि॒यं ॅवै त्रि॒ष्टुक् त्रि॒ष्टुग् वा इ॑न्द्रि॒य मि॑न्द्रि॒यं ॅवै त्रि॒ष्टुक् । \newline
39. वै त्रि॒ष्टुक् त्रि॒ष्टुग् वै वै त्रि॒ष्टु गि॑न्द्रि॒यका॑म इन्द्रि॒यका॑म स्त्रि॒ष्टुग् वै वै त्रि॒ष्टु गि॑न्द्रि॒यका॑मः । \newline
40. त्रि॒ष्टु गि॑न्द्रि॒यका॑म इन्द्रि॒यका॑म स्त्रि॒ष्टुक् त्रि॒ष्टु गि॑न्द्रि॒यका॑मः॒ खलु॒ खल्वि॑न्द्रि॒यका॑म स्त्रि॒ष्टुक् त्रि॒ष्टु गि॑न्द्रि॒यका॑मः॒ खलु॑ । \newline
41. इ॒न्द्रि॒यका॑मः॒ खलु॒ खल्वि॑न्द्रि॒यका॑म इन्द्रि॒यका॑मः॒ खलु॒ वै वै खल्वि॑न्द्रि॒यका॑म इन्द्रि॒यका॑मः॒ खलु॒ वै । \newline
42. इ॒न्द्रि॒यका॑म॒ इती᳚न्द्रि॒य - का॒मः॒ । \newline
43. खलु॒ वै वै खलु॒ खलु॒ वै रा॑ज॒न्यो॑ राज॒न्यो॑ वै खलु॒ खलु॒ वै रा॑ज॒न्यः॑ । \newline
44. वै रा॑ज॒न्यो॑ राज॒न्यो॑ वै वै रा॑ज॒न्यो॑ यजते यजते राज॒न्यो॑ वै वै रा॑ज॒न्यो॑ यजते । \newline
45. रा॒ज॒न्यो॑ यजते यजते राज॒न्यो॑ राज॒न्यो॑ यजते त्रि॒ष्टुभा᳚ त्रि॒ष्टुभा॑ यजते राज॒न्यो॑ राज॒न्यो॑ यजते त्रि॒ष्टुभा᳚ । \newline
46. य॒ज॒ते॒ त्रि॒ष्टुभा᳚ त्रि॒ष्टुभा॑ यजते यजते त्रि॒ष्टुभै॒वैव त्रि॒ष्टुभा॑ यजते यजते त्रि॒ष्टुभै॒व । \newline
47. त्रि॒ष्टुभै॒वैव त्रि॒ष्टुभा᳚ त्रि॒ष्टुभै॒ वास्मा॑ अस्मा ए॒व त्रि॒ष्टुभा᳚ त्रि॒ष्टुभै॒ वास्मै᳚ । \newline
48. ए॒वास्मा॑ अस्मा ए॒वैवास्मा॑ इन्द्रि॒य मि॑न्द्रि॒य म॑स्मा ए॒वैवास्मा॑ इन्द्रि॒यम् । \newline
49. अ॒स्मा॒ इ॒न्द्रि॒य मि॑न्द्रि॒य म॑स्मा अस्मा इन्द्रि॒यम् परि॒ परी᳚न्द्रि॒य म॑स्मा अस्मा इन्द्रि॒यम् परि॑ । \newline
50. इ॒न्द्रि॒यम् परि॒ परी᳚न्द्रि॒य मि॑न्द्रि॒यम् परि॑ गृह्णाति गृह्णाति॒ परी᳚न्द्रि॒य मि॑न्द्रि॒यम् परि॑ गृह्णाति । \newline
51. परि॑ गृह्णाति गृह्णाति॒ परि॒ परि॑ गृह्णाति॒ यदि॒ यदि॑ गृह्णाति॒ परि॒ परि॑ गृह्णाति॒ यदि॑ । \newline
52. गृ॒ह्णा॒ति॒ यदि॒ यदि॑ गृह्णाति गृह्णाति॒ यदि॑ का॒मये॑त का॒मये॑त॒ यदि॑ गृह्णाति गृह्णाति॒ यदि॑ का॒मये॑त । \newline
53. यदि॑ का॒मये॑त का॒मये॑त॒ यदि॒ यदि॑ का॒मये॑त ब्रह्मवर्च॒सम् ब्र॑ह्मवर्च॒सम् का॒मये॑त॒ यदि॒ यदि॑ का॒मये॑त ब्रह्मवर्च॒सम् । \newline
54. का॒मये॑त ब्रह्मवर्च॒सम् ब्र॑ह्मवर्च॒सम् का॒मये॑त का॒मये॑त ब्रह्मवर्च॒स म॑स्त्वस्तु ब्रह्मवर्च॒सम् का॒मये॑त का॒मये॑त ब्रह्मवर्च॒स म॑स्तु । \newline
\pagebreak
\markright{ TS 2.5.10.2  \hfill https://www.vedavms.in \hfill}
\addcontentsline{toc}{section}{ TS 2.5.10.2 }
\section*{ TS 2.5.10.2 }

\textbf{TS 2.5.10.2 } \newline
\textbf{Samhita Paata} \newline

ब्रह्मवर्च॒सम॒स्त्विति॑ गायत्रि॒या परि॑ दद्ध्याद्-ब्रह्मवर्च॒सं ॅवै गा॑य॒त्री ब्र॑ह्मवर्च॒समे॒व भ॑वति स॒प्तद॒शानु॑ ब्रूया॒द्-वैश्य॑स्य सप्तद॒शो वै वैश्यः॒ स्व ए॒वैनꣳ॒॒ स्तोमे॒ प्रति॑ ष्ठापयति॒जग॑त्या॒ परि॑ दद्ध्या॒ज्जाग॑ता॒ वै प॒शवः॑ प॒शुका॑मः॒ खलु॒ वै वैश्यो॑ यजते॒ जग॑त्यै॒वास्मै॑ प॒शून् परि॑ गृह्णा॒त्ये क॑विꣳ शति॒मनु॑ ब्रूयात् प्रति॒ष्ठाका॑मस्यै कविꣳ॒॒शः स्तोमा॑नां प्रति॒ष्ठा प्रति॑ष्ठित्यै॒ - [  ] \newline

\textbf{Pada Paata} \newline

ब्र॒ह्म॒व॒र्च॒समिति॑ ब्रह्म-व॒र्च॒सम् । अ॒स्तु॒ । इति॑ । गा॒य॒त्रि॒या । परीति॑ । द॒द्ध्या॒त् । ब्र॒ह्म॒व॒र्च॒समिति॑ ब्रह्म - व॒र्च॒सम् । वै । गा॒य॒त्री । ब्र॒ह्म॒व॒र्च॒समिति॑ ब्रह्म - व॒र्च॒सम् । ए॒व । भ॒व॒ति॒ । स॒प्तद॒शेति॑ स॒प्त - द॒श॒ । अन्विति॑ । ब्रू॒या॒त् । वैश्य॑स्य । स॒प्त॒द॒श इति॑ सप्त - द॒शः । वै । वैश्यः॑ । स्वे । ए॒व । ए॒न॒म् । स्तोमे᳚ । प्रतीति॑ । स्था॒प॒य॒ति॒ । जग॑त्या । परीति॑ । द॒द्ध्या॒त् । जाग॑ताः । वै । प॒शवः॑ । प॒शुका॑म॒ इति॑ प॒शु - का॒मः॒ । खलु॑ । वै । वैश्यः॑ । य॒ज॒ते॒ । जग॑त्या । ए॒व । अ॒स्मै॒ । प॒शून् । परीति॑ । गृ॒ह्णा॒ति । एक॑विꣳशति॒मित्येक॑ - विꣳ॒॒श॒ति॒म् । अन्विति॑ । ब्रू॒या॒त् । प्र॒ति॒ष्ठाका॑म॒स्येति॑ प्रति॒ष्ठा - का॒म॒स्य॒ । ए॒क॒विꣳ॒॒श इत्ये॑क-विꣳ॒॒शः । स्तोमा॑नाम् । प्र॒ति॒ष्ठेति॑ प्रति - स्था । प्रति॑ष्ठित्या॒ इति॒ प्रति॑ - स्थि॒त्यै॒ ।  \newline


\textbf{Krama Paata} \newline

ब्र॒ह्म॒व॒र्च॒सम॑स्तु । ब्र॒ह्म॒व॒र्च॒समिति॑ ब्रह्म - व॒र्च॒सम् । अ॒स्त्विति॑ । इति॑ गायत्रि॒या । गा॒य॒त्रि॒या परि॑ । परि॑ दद्ध्यात् । द॒द्ध्या॒त् ब्र॒ह्म॒व॒र्च॒सम् । ब्र॒ह्म॒व॒र्च॒सम् ॅवै । ब्र॒ह्म॒व॒र्च॒समिति॑ ब्रह्म - व॒र्च॒सम् । वै गा॑य॒त्री । गा॒य॒त्री ब्र॑ह्मवर्च॒सम् । ब्र॒ह्म॒व॒र्च॒समे॒व । ब्र॒ह्म॒व॒र्च॒समिति॑ ब्रह्म - व॒र्च॒सम् । ए॒व भ॑वति । भ॒व॒ति॒ स॒प्तद॑श । स॒प्तद॒शानु॑ । स॒प्तद॒शेति॑ स॒प्त - द॒श॒ । अनु॑ ब्रूयात् । ब्रू॒या॒द् वैश्य॑स्य । वैश्य॑स्य सप्तद॒शः । स॒प्त॒द॒शो वै । स॒प्त॒द॒श इति॑ सप्त - द॒शः । वै वैश्यः॑ । वैश्यः॒ स्वे । स्व ए॒व । ए॒वैन᳚म् । ए॒नꣳ॒॒ स्तोमे᳚ । स्तोमे॒ प्रति॑ । प्रति॑ ष्ठापयति । स्था॒प॒य॒ति॒ जग॑त्या । जग॑त्या॒ परि॑ । परि॑ दद्ध्यात् । द॒द्ध्या॒ज्जाग॑ताः । जाग॑ता॒ वै । वै प॒शवः॑ । प॒शवः॑ प॒शुका॑मः । प॒शुका॑मः॒ खलु॑ । प॒शुका॑म॒ इति॑ प॒शु - का॒मः॒ । खलु॒ वै । वै वैश्यः॑ । वैश्यो॑ यजते । य॒ज॒ते॒ जग॑त्या । जग॑त्यै॒व । ए॒वास्मै᳚ । अ॒स्मै॒ प॒शून् । प॒शून् परि॑ । परि॑ गृह्णाति । गृ॒ह्णा॒त्येक॑विꣳशतिम् । एक॑विꣳशति॒मनु॑ । एक॑विꣳशति॒मित्येक॑ - विꣳ॒॒श॒ति॒म् । अनु॑ ब्रूयात् । ब्रू॒या॒त् प्र॒ति॒ष्ठाका॑मस्य । प्र॒ति॒ष्ठाका॑मस्यैकविꣳ॒॒शः । प्र॒ति॒ष्ठाका॑म॒स्येति॑ प्रति॒ष्ठा - का॒म॒स्य॒ । ए॒क॒विꣳ॒॒शः स्तोमा॑नाम् । ए॒क॒विꣳ॒॒श इत्ये॑क - विꣳ॒॒शः । स्तोमा॑नाम् प्रति॒ष्ठा । प्र॒ति॒ष्ठा प्रति॑ष्ठित्यै । प्र॒ति॒ष्ठेति॑ प्रति - स्था । प्रति॑ष्ठित्यै॒ चतु॑र्विꣳशतिम् । प्रति॑ष्ठित्या॒ इति॒ प्रति॑ - स्थि॒त्यै॒ \newline

\textbf{Jatai Paata} \newline

1. ब्र॒ह्म॒व॒र्च॒स म॑स्त्वस्तु ब्रह्मवर्च॒सम् ब्र॑ह्मवर्च॒स म॑स्तु । \newline
2. ब्र॒ह्म॒व॒र्च॒समिति॑ ब्रह्म - व॒र्च॒सम् । \newline
3. अ॒स्त्विती त्य॑स्त्व॒ स्त्विति॑ । \newline
4. इति॑ गायत्रि॒या गा॑यत्रि॒येतीति॑ गायत्रि॒या । \newline
5. गा॒य॒त्रि॒या परि॒ परि॑ गायत्रि॒या गा॑यत्रि॒या परि॑ । \newline
6. परि॑ दद्ध्याद् दद्ध्या॒त् परि॒ परि॑ दद्ध्यात् । \newline
7. द॒द्ध्या॒द् ब्र॒ह्म॒व॒र्च॒सम् ब्र॑ह्मवर्च॒सम् द॑द्ध्याद् दद्ध्याद् ब्रह्मवर्च॒सम् । \newline
8. ब्र॒ह्म॒व॒र्च॒सं ॅवै वै ब्र॑ह्मवर्च॒सम् ब्र॑ह्मवर्च॒सं ॅवै । \newline
9. ब्र॒ह्म॒व॒र्च॒समिति॑ ब्रह्म - व॒र्च॒सम् । \newline
10. वै गा॑य॒त्री गा॑य॒त्री वै वै गा॑य॒त्री । \newline
11. गा॒य॒त्री ब्र॑ह्मवर्च॒सम् ब्र॑ह्मवर्च॒सम् गा॑य॒त्री गा॑य॒त्री ब्र॑ह्मवर्च॒सम् । \newline
12. ब्र॒ह्म॒व॒र्च॒स मे॒वैव ब्र॑ह्मवर्च॒सम् ब्र॑ह्मवर्च॒स मे॒व । \newline
13. ब्र॒ह्म॒व॒र्च॒समिति॑ ब्रह्म - व॒र्च॒सम् । \newline
14. ए॒व भ॑वति भव त्ये॒वैव भ॑वति । \newline
15. भ॒व॒ति॒ स॒प्तद॑श स॒प्तद॑श भवति भवति स॒प्तद॑श । \newline
16. स॒प्तद॒शान्वनु॑ स॒प्तद॑श स॒प्तद॒शानु॑ । \newline
17. स॒प्तद॒शेति॑ स॒प्त - द॒श॒ । \newline
18. अनु॑ ब्रूयाद् ब्रूया॒ दन्वनु॑ ब्रूयात् । \newline
19. ब्रू॒या॒द् वैश्य॑स्य॒ वैश्य॑स्य ब्रूयाद् ब्रूया॒द् वैश्य॑स्य । \newline
20. वैश्य॑स्य सप्तद॒शः स॑प्तद॒शो वैश्य॑स्य॒ वैश्य॑स्य सप्तद॒शः । \newline
21. स॒प्त॒द॒शो वै वै स॑प्तद॒शः स॑प्तद॒शो वै । \newline
22. स॒प्त॒द॒श इति॑ सप्त - द॒शः । \newline
23. वै वैश्यो॒ वैश्यो॒ वै वै वैश्यः॑ । \newline
24. वैश्यः॒ स्वे स्वे वैश्यो॒ वैश्यः॒ स्वे । \newline
25. स्व ए॒वैव स्वे स्व ए॒व । \newline
26. ए॒वैन॑ मेन मे॒वैवैन᳚म् । \newline
27. ए॒नꣳ॒॒ स्तोमे॒ स्तोम॑ एन मेनꣳ॒॒ स्तोमे᳚ । \newline
28. स्तोमे॒ प्रति॒ प्रति॒ स्तोमे॒ स्तोमे॒ प्रति॑ । \newline
29. प्रति॑ ष्ठापयति स्थापयति॒ प्रति॒ प्रति॑ ष्ठापयति । \newline
30. स्था॒प॒य॒ति॒ जग॑त्या॒ जग॑त्या स्थापयति स्थापयति॒ जग॑त्या । \newline
31. जग॑त्या॒ परि॒ परि॒ जग॑त्या॒ जग॑त्या॒ परि॑ । \newline
32. परि॑ दद्ध्याद् दद्ध्या॒त् परि॒ परि॑ दद्ध्यात् । \newline
33. द॒द्ध्या॒ज् जाग॑ता॒ जाग॑ता दद्ध्याद् दद्ध्या॒ज् जाग॑ताः । \newline
34. जाग॑ता॒ वै वै जाग॑ता॒ जाग॑ता॒ वै । \newline
35. वै प॒शवः॑ प॒शवो॒ वै वै प॒शवः॑ । \newline
36. प॒शवः॑ प॒शुका॑मः प॒शुका॑मः प॒शवः॑ प॒शवः॑ प॒शुका॑मः । \newline
37. प॒शुका॑मः॒ खलु॒ खलु॑ प॒शुका॑मः प॒शुका॑मः॒ खलु॑ । \newline
38. प॒शुका॑म॒ इति॑ प॒शु - का॒मः॒ । \newline
39. खलु॒ वै वै खलु॒ खलु॒ वै । \newline
40. वै वैश्यो॒ वैश्यो॒ वै वै वैश्यः॑ । \newline
41. वैश्यो॑ यजते यजते॒ वैश्यो॒ वैश्यो॑ यजते । \newline
42. य॒ज॒ते॒ जग॑त्या॒ जग॑त्या यजते यजते॒ जग॑त्या । \newline
43. जग॑त्यै॒वैव जग॑त्या॒ जग॑त्यै॒व । \newline
44. ए॒वास्मा॑ अस्मा ए॒वैवास्मै᳚ । \newline
45. अ॒स्मै॒ प॒शून् प॒शू न॑स्मा अस्मै प॒शून् । \newline
46. प॒शून् परि॒ परि॑ प॒शून् प॒शून् परि॑ । \newline
47. परि॑ गृह्णाति गृह्णाति॒ परि॒ परि॑ गृह्णाति । \newline
48. गृ॒ह्णा॒ त्येक॑विꣳशति॒ मेक॑विꣳशतिम् गृह्णाति गृह्णा॒ त्येक॑विꣳशतिम् । \newline
49. एक॑विꣳशति॒ मन्वन्वेक॑विꣳशति॒ मेक॑विꣳशति॒ मनु॑ । \newline
50. एक॑विꣳशति॒मित्येक॑ - विꣳ॒॒श॒ति॒म् । \newline
51. अनु॑ ब्रूयाद् ब्रूया॒ दन्वनु॑ ब्रूयात् । \newline
52. ब्रू॒या॒त् प्र॒ति॒ष्ठाका॑मस्य प्रति॒ष्ठाका॑मस्य ब्रूयाद् ब्रूयात् प्रति॒ष्ठाका॑मस्य । \newline
53. प्र॒ति॒ष्ठाका॑म स्यैकविꣳ॒॒श ए॑कविꣳ॒॒शः प्र॑ति॒ष्ठाका॑मस्य प्रति॒ष्ठाका॑म स्यैकविꣳ॒॒शः । \newline
54. प्र॒ति॒ष्ठाका॑म॒स्येति॑ प्रति॒ष्ठा - का॒म॒स्य॒ । \newline
55. ए॒क॒विꣳ॒॒शः स्तोमा॑नाꣳ॒॒ स्तोमा॑ना मेकविꣳ॒॒श ए॑कविꣳ॒॒शः स्तोमा॑नाम् । \newline
56. ए॒क॒विꣳ॒॒श इत्ये॑क - विꣳ॒॒शः । \newline
57. स्तोमा॑नाम् प्रति॒ष्ठा प्र॑ति॒ष्ठा स्तोमा॑नाꣳ॒॒ स्तोमा॑नाम् प्रति॒ष्ठा । \newline
58. प्र॒ति॒ष्ठा प्रति॑ष्ठित्यै॒ प्रति॑ष्ठित्यै प्रति॒ष्ठा प्र॑ति॒ष्ठा प्रति॑ष्ठित्यै । \newline
59. प्र॒ति॒ष्ठेति॑ प्रति - स्था । \newline
60. प्रति॑ष्ठित्यै॒ चतु॑र्विꣳशति॒म् चतु॑र्विꣳशति॒म् प्रति॑ष्ठित्यै॒ प्रति॑ष्ठित्यै॒ चतु॑र्विꣳशतिम् । \newline
61. प्रति॑ष्ठित्या॒ इति॒ प्रति॑ - स्थि॒त्यै॒ । \newline

\textbf{Ghana Paata } \newline

1. ब्र॒ह्म॒व॒र्च॒स म॑स्त्वस्तु ब्रह्मवर्च॒सम् ब्र॑ह्मवर्च॒स म॒स्त्विती त्य॑स्तु ब्रह्मवर्च॒सम् ब्र॑ह्मवर्च॒स म॒स्त्विति॑ । \newline
2. ब्र॒ह्म॒व॒र्च॒समिति॑ ब्रह्म - व॒र्च॒सम् । \newline
3. अ॒स्त्विती त्य॑स्त्व॒ स्त्विति॑ गायत्रि॒या गा॑यत्रि॒ये त्य॑स्त्व॒ स्त्विति॑ गायत्रि॒या । \newline
4. इति॑ गायत्रि॒या गा॑यत्रि॒येतीति॑ गायत्रि॒या परि॒ परि॑ गायत्रि॒येतीति॑ गायत्रि॒या परि॑ । \newline
5. गा॒य॒त्रि॒या परि॒ परि॑ गायत्रि॒या गा॑यत्रि॒या परि॑ दद्ध्याद् दद्ध्या॒त् परि॑ गायत्रि॒या गा॑यत्रि॒या परि॑ दद्ध्यात् । \newline
6. परि॑ दद्ध्याद् दद्ध्या॒त् परि॒ परि॑ दद्ध्याद् ब्रह्मवर्च॒सम् ब्र॑ह्मवर्च॒सम् द॑द्ध्या॒त् परि॒ परि॑ दद्ध्याद् ब्रह्मवर्च॒सम् । \newline
7. द॒द्ध्या॒द् ब्र॒ह्म॒व॒र्च॒सम् ब्र॑ह्मवर्च॒सम् द॑द्ध्याद् दद्ध्याद् ब्रह्मवर्च॒सं ॅवै वै ब्र॑ह्मवर्च॒सम् द॑द्ध्याद् दद्ध्याद् ब्रह्मवर्च॒सं ॅवै । \newline
8. ब्र॒ह्म॒व॒र्च॒सं ॅवै वै ब्र॑ह्मवर्च॒सम् ब्र॑ह्मवर्च॒सं ॅवै गा॑य॒त्री गा॑य॒त्री वै ब्र॑ह्मवर्च॒सम् ब्र॑ह्मवर्च॒सं ॅवै गा॑य॒त्री । \newline
9. ब्र॒ह्म॒व॒र्च॒समिति॑ ब्रह्म - व॒र्च॒सम् । \newline
10. वै गा॑य॒त्री गा॑य॒त्री वै वै गा॑य॒त्री ब्र॑ह्मवर्च॒सम् ब्र॑ह्मवर्च॒सम् गा॑य॒त्री वै वै गा॑य॒त्री ब्र॑ह्मवर्च॒सम् । \newline
11. गा॒य॒त्री ब्र॑ह्मवर्च॒सम् ब्र॑ह्मवर्च॒सम् गा॑य॒त्री गा॑य॒त्री ब्र॑ह्मवर्च॒स मे॒वैव ब्र॑ह्मवर्च॒सम् गा॑य॒त्री गा॑य॒त्री ब्र॑ह्मवर्च॒स मे॒व । \newline
12. ब्र॒ह्म॒व॒र्च॒स मे॒वैव ब्र॑ह्मवर्च॒सम् ब्र॑ह्मवर्च॒स मे॒व भ॑वति भवत्ये॒व ब्र॑ह्मवर्च॒सम् ब्र॑ह्मवर्च॒स मे॒व भ॑वति । \newline
13. ब्र॒ह्म॒व॒र्च॒समिति॑ ब्रह्म - व॒र्च॒सम् । \newline
14. ए॒व भ॑वति भवत्ये॒वैव भ॑वति स॒प्तद॑श स॒प्तद॑श भव त्ये॒वैव भ॑वति स॒प्तद॑श । \newline
15. भ॒व॒ति॒ स॒प्तद॑श स॒प्तद॑श भवति भवति स॒प्तद॒शा न्वनु॑ स॒प्तद॑श भवति भवति स॒प्तद॒शानु॑ । \newline
16. स॒प्तद॒शा न्वनु॑ स॒प्तद॑श स॒प्तद॒शानु॑ ब्रूयाद् ब्रूया॒दनु॑ स॒प्तद॑श स॒प्तद॒शानु॑ ब्रूयात् । \newline
17. स॒प्तद॒शेति॑ स॒प्त - द॒श॒ । \newline
18. अनु॑ ब्रूयाद् ब्रूया॒दन्वनु॑ ब्रूया॒द् वैश्य॑स्य॒ वैश्य॑स्य ब्रूया॒द न्वनु॑ ब्रूया॒द् वैश्य॑स्य । \newline
19. ब्रू॒या॒द् वैश्य॑स्य॒ वैश्य॑स्य ब्रूयाद् ब्रूया॒द् वैश्य॑स्य सप्तद॒शः स॑प्तद॒शो वैश्य॑स्य ब्रूयाद् ब्रूया॒द् वैश्य॑स्य सप्तद॒शः । \newline
20. वैश्य॑स्य सप्तद॒शः स॑प्तद॒शो वैश्य॑स्य॒ वैश्य॑स्य सप्तद॒शो वै वै स॑प्तद॒शो वैश्य॑स्य॒ वैश्य॑स्य सप्तद॒शो वै । \newline
21. स॒प्त॒द॒शो वै वै स॑प्तद॒शः स॑प्तद॒शो वै वैश्यो॒ वैश्यो॒ वै स॑प्तद॒शः स॑प्तद॒शो वै वैश्यः॑ । \newline
22. स॒प्त॒द॒श इति॑ सप्त - द॒शः । \newline
23. वै वैश्यो॒ वैश्यो॒ वै वै वैश्यः॒ स्वे स्वे वैश्यो॒ वै वै वैश्यः॒ स्वे । \newline
24. वैश्यः॒ स्वे स्वे वैश्यो॒ वैश्यः॒ स्व ए॒वैव स्वे वैश्यो॒ वैश्यः॒ स्व ए॒व । \newline
25. स्व ए॒वैव स्वे स्व ए॒वैन॑ मेन मे॒व स्वे स्व ए॒वैन᳚म् । \newline
26. ए॒वैन॑ मेन मे॒वैवैनꣳ॒॒ स्तोमे॒ स्तोम॑ एन मे॒वैवैनꣳ॒॒ स्तोमे᳚ । \newline
27. ए॒नꣳ॒॒ स्तोमे॒ स्तोम॑ एन मेनꣳ॒॒ स्तोमे॒ प्रति॒ प्रति॒ स्तोम॑ एन मेनꣳ॒॒ स्तोमे॒ प्रति॑ । \newline
28. स्तोमे॒ प्रति॒ प्रति॒ स्तोमे॒ स्तोमे॒ प्रति॑ ष्ठापयति स्थापयति॒ प्रति॒ स्तोमे॒ स्तोमे॒ प्रति॑ ष्ठापयति । \newline
29. प्रति॑ ष्ठापयति स्थापयति॒ प्रति॒ प्रति॑ ष्ठापयति॒ जग॑त्या॒ जग॑त्या स्थापयति॒ प्रति॒ प्रति॑ ष्ठापयति॒ जग॑त्या । \newline
30. स्था॒प॒य॒ति॒ जग॑त्या॒ जग॑त्या स्थापयति स्थापयति॒ जग॑त्या॒ परि॒ परि॒ जग॑त्या स्थापयति स्थापयति॒ जग॑त्या॒ परि॑ । \newline
31. जग॑त्या॒ परि॒ परि॒ जग॑त्या॒ जग॑त्या॒ परि॑ दद्ध्याद् दद्ध्या॒त् परि॒ जग॑त्या॒ जग॑त्या॒ परि॑ दद्ध्यात् । \newline
32. परि॑ दद्ध्याद् दद्ध्या॒त् परि॒ परि॑ दद्ध्या॒ज् जाग॑ता॒ जाग॑ता दद्ध्या॒त् परि॒ परि॑ दद्ध्या॒ज् जाग॑ताः । \newline
33. द॒द्ध्या॒ज् जाग॑ता॒ जाग॑ता दद्ध्याद् दद्ध्या॒ज् जाग॑ता॒ वै वै जाग॑ता दद्ध्याद् दद्ध्या॒ज् जाग॑ता॒ वै । \newline
34. जाग॑ता॒ वै वै जाग॑ता॒ जाग॑ता॒ वै प॒शवः॑ प॒शवो॒ वै जाग॑ता॒ जाग॑ता॒ वै प॒शवः॑ । \newline
35. वै प॒शवः॑ प॒शवो॒ वै वै प॒शवः॑ प॒शुका॑मः प॒शुका॑मः प॒शवो॒ वै वै प॒शवः॑ प॒शुका॑मः । \newline
36. प॒शवः॑ प॒शुका॑मः प॒शुका॑मः प॒शवः॑ प॒शवः॑ प॒शुका॑मः॒ खलु॒ खलु॑ प॒शुका॑मः प॒शवः॑ प॒शवः॑ प॒शुका॑मः॒ खलु॑ । \newline
37. प॒शुका॑मः॒ खलु॒ खलु॑ प॒शुका॑मः प॒शुका॑मः॒ खलु॒ वै वै खलु॑ प॒शुका॑मः प॒शुका॑मः॒ खलु॒ वै । \newline
38. प॒शुका॑म॒ इति॑ प॒शु - का॒मः॒ । \newline
39. खलु॒ वै वै खलु॒ खलु॒ वै वैश्यो॒ वैश्यो॒ वै खलु॒ खलु॒ वै वैश्यः॑ । \newline
40. वै वैश्यो॒ वैश्यो॒ वै वै वैश्यो॑ यजते यजते॒ वैश्यो॒ वै वै वैश्यो॑ यजते । \newline
41. वैश्यो॑ यजते यजते॒ वैश्यो॒ वैश्यो॑ यजते॒ जग॑त्या॒ जग॑त्या यजते॒ वैश्यो॒ वैश्यो॑ यजते॒ जग॑त्या । \newline
42. य॒ज॒ते॒ जग॑त्या॒ जग॑त्या यजते यजते॒ जग॑त्यै॒वैव जग॑त्या यजते यजते॒ जग॑त्यै॒व । \newline
43. जग॑त्यै॒वैव जग॑त्या॒ जग॑त्यै॒वास्मा॑ अस्मा ए॒व जग॑त्या॒ जग॑त्यै॒वास्मै᳚ । \newline
44. ए॒वास्मा॑ अस्मा ए॒वैवास्मै॑ प॒शून् प॒शू न॑स्मा ए॒वैवास्मै॑ प॒शून् । \newline
45. अ॒स्मै॒ प॒शून् प॒शू न॑स्मा अस्मै प॒शून् परि॒ परि॑ प॒शू न॑स्मा अस्मै प॒शून् परि॑ । \newline
46. प॒शून् परि॒ परि॑ प॒शून् प॒शून् परि॑ गृह्णाति गृह्णाति॒ परि॑ प॒शून् प॒शून् परि॑ गृह्णाति । \newline
47. परि॑ गृह्णाति गृह्णाति॒ परि॒ परि॑ गृह्णा॒ त्येक॑विꣳशति॒ मेक॑विꣳशतिम् गृह्णाति॒ परि॒ परि॑ 
गृह्णा॒ त्येक॑विꣳशतिम् । \newline
48. गृ॒ह्णा॒ त्येक॑विꣳशति॒ मेक॑विꣳशतिम् गृह्णाति गृह्णा॒ त्येक॑विꣳशति॒ मन्व न्वेक॑विꣳशतिम् गृह्णाति गृह्णा॒ त्येक॑विꣳशति॒ मनु॑ । \newline
49. एक॑विꣳशति॒ मन्व न्वेक॑विꣳशति॒ मेक॑विꣳशति॒ मनु॑ ब्रूयाद् ब्रूया॒ दन्वेक॑विꣳशति॒ मेक॑विꣳशति॒ मनु॑ ब्रूयात् । \newline
50. एक॑विꣳशति॒मित्येक॑ - विꣳ॒॒श॒ति॒म् । \newline
51. अनु॑ ब्रूयाद् ब्रूया॒दन्वनु॑ ब्रूयात् प्रति॒ष्ठाका॑मस्य प्रति॒ष्ठाका॑मस्य ब्रूया॒दन्वनु॑ ब्रूयात् प्रति॒ष्ठाका॑मस्य । \newline
52. ब्रू॒या॒त् प्र॒ति॒ष्ठाका॑मस्य प्रति॒ष्ठाका॑मस्य ब्रूयाद् ब्रूयात् प्रति॒ष्ठाका॑म स्यैकविꣳ॒॒श ए॑कविꣳ॒॒शः प्र॑ति॒ष्ठाका॑मस्य ब्रूयाद् ब्रूयात् प्रति॒ष्ठाका॑म स्यैकविꣳ॒॒शः । \newline
53. प्र॒ति॒ष्ठाका॑म स्यैकविꣳ॒॒श ए॑कविꣳ॒॒शः प्र॑ति॒ष्ठाका॑मस्य प्रति॒ष्ठाका॑म स्यैकविꣳ॒॒शः स्तोमा॑नाꣳ॒॒ स्तोमा॑ना मेकविꣳ॒॒शः प्र॑ति॒ष्ठाका॑मस्य प्रति॒ष्ठाका॑म स्यैकविꣳ॒॒शः स्तोमा॑नाम् । \newline
54. प्र॒ति॒ष्ठाका॑म॒स्येति॑ प्रति॒ष्ठा - का॒म॒स्य॒ । \newline
55. ए॒क॒विꣳ॒॒शः स्तोमा॑नाꣳ॒॒ स्तोमा॑ना मेकविꣳ॒॒श ए॑कविꣳ॒॒शः स्तोमा॑नाम् प्रति॒ष्ठा प्र॑ति॒ष्ठा स्तोमा॑ना मेकविꣳ॒॒श ए॑कविꣳ॒॒शः स्तोमा॑नाम् प्रति॒ष्ठा । \newline
56. ए॒क॒विꣳ॒॒श इत्ये॑क - विꣳ॒॒शः । \newline
57. स्तोमा॑नाम् प्रति॒ष्ठा प्र॑ति॒ष्ठा स्तोमा॑नाꣳ॒॒ स्तोमा॑नाम् प्रति॒ष्ठा प्रति॑ष्ठित्यै॒ प्रति॑ष्ठित्यै प्रति॒ष्ठा स्तोमा॑नाꣳ॒॒ स्तोमा॑नाम् प्रति॒ष्ठा प्रति॑ष्ठित्यै । \newline
58. प्र॒ति॒ष्ठा प्रति॑ष्ठित्यै॒ प्रति॑ष्ठित्यै प्रति॒ष्ठा प्र॑ति॒ष्ठा प्रति॑ष्ठित्यै॒ चतु॑र्विꣳशति॒म् चतु॑र्विꣳशति॒म् प्रति॑ष्ठित्यै प्रति॒ष्ठा प्र॑ति॒ष्ठा प्रति॑ष्ठित्यै॒ चतु॑र्विꣳशतिम् । \newline
59. प्र॒ति॒ष्ठेति॑ प्रति - स्था । \newline
60. प्रति॑ष्ठित्यै॒ चतु॑र्विꣳशति॒म् चतु॑र्विꣳशति॒म् प्रति॑ष्ठित्यै॒ प्रति॑ष्ठित्यै॒ चतु॑र्विꣳशति॒ मन्वनु॒ चतु॑र्विꣳशति॒म् प्रति॑ष्ठित्यै॒ प्रति॑ष्ठित्यै॒ चतु॑र्विꣳशति॒ मनु॑ । \newline
61. प्रति॑ष्ठित्या॒ इति॒ प्रति॑ - स्थि॒त्यै॒ । \newline
\pagebreak
\markright{ TS 2.5.10.3  \hfill https://www.vedavms.in \hfill}
\addcontentsline{toc}{section}{ TS 2.5.10.3 }
\section*{ TS 2.5.10.3 }

\textbf{TS 2.5.10.3 } \newline
\textbf{Samhita Paata} \newline

चतु॑र्विꣳशति॒मनु॑ ब्रूयाद्-ब्रह्मवर्च॒स-का॑मस्य॒ चतु॑र्विꣳशत्यक्षरा गाय॒त्री गा॑य॒त्री ब्र॑ह्मवर्च॒सं गा॑यत्रि॒यैवास्मै᳚ ब्रह्मवर्च॒समव॑ रुन्धे त्रिꣳ॒॒शत॒मनु॑ ब्रूया॒दन्न॑कामस्य त्रिꣳ॒॒शद॑क्षरा वि॒राडन्नं॑ ॅवि॒राड् वि॒राजै॒वास्मा॑ अ॒न्नाद्य॒मव॑ रुन्धे॒ द्वात्रिꣳ॑शत॒मनु॑ ब्रूयात् प्रति॒ष्ठाका॑मस्य॒ द्वात्रिꣳ॑शदक्षरा ऽनु॒ष्टुग॑नु॒ष्टुप् छन्द॑सां प्रति॒ष्ठा प्रति॑ष्ठित्यै॒ षट्त्रिꣳ॑शत॒मनु॑ ब्रूयात् प॒शुका॑मस्य॒ षट्त्रिꣳ॑शदक्षरा बृह॒ती बार्.ह॑ताः प॒शवो॑ बृह॒त्यैवास्मै॑ प॒शू - [  ] \newline

\textbf{Pada Paata} \newline

चतु॑र्विꣳशति॒मिति॒ चतुः॑ - विꣳ॒॒श॒ति॒म् । अन्विति॑ । ब्रू॒या॒त् । ब्र॒ह्म॒व॒र्च॒सका॑म॒स्येति॑ ब्रह्मवर्च॒स - का॒म॒स्य॒ । चतु॑र्विꣳशत्यक्ष॒रेति॒ चतु॑र्विꣳशति - अ॒क्ष॒रा॒ । गा॒य॒त्री । गा॒य॒त्री । ब्र॒ह्म॒व॒र्च॒समिति॑ ब्रह्म - व॒र्च॒सम् । गा॒य॒त्रि॒या । ए॒व । अ॒स्मै॒ । ब्र॒ह्म॒व॒र्च॒समिति॑ ब्रह्म - व॒र्च॒सम् । अवेति॑ । रु॒न्धे॒ । त्रिꣳ॒॒शत᳚म् । अन्विति॑ । ब्रू॒या॒त् । अन्न॑काम॒स्येत्यन्न॑ - का॒म॒स्य॒ । त्रिꣳ॒॒शद॑क्ष॒रेति॑ त्रिꣳ॒॒शत् - अ॒क्ष॒रा॒ । वि॒राडिति॑ वि - राट् । अन्न᳚म् । वि॒राडिति॑ वि - राट् । वि॒राजेति॑ वि - राजा᳚ । ए॒व । अ॒स्मै॒ । अ॒न्नाद्य॒मित्य॑न्न - अद्य᳚म् । अवेति॑ । रु॒न्धे॒ । द्वात्रिꣳ॑शतम् । अन्विति॑ । ब्रू॒या॒त् । प्र॒ति॒ष्ठाका॑म॒स्येति॑ प्रति॒ष्ठा - का॒म॒स्य॒ । द्वात्रिꣳ॑शदक्ष॒रेति॒ द्वात्रिꣳ॑शत् - अ॒क्ष॒रा॒ । अ॒नु॒ष्टुगित्य॑नु - स्तुक् । अ॒नु॒ष्टुबित्य॑नु - स्तुप् । छन्द॑साम् । प्र॒ति॒ष्ठेति॑ प्रति - स्था । प्रति॑ष्ठित्या॒ इति॒ प्रति॑ - स्थि॒त्यै॒ । षट्त्रिꣳ॑शत॒मिति॒ षट्-त्रिꣳ॒॒श॒त॒म् । अन्विति॑ । ब्रू॒या॒त् । प॒शुका॑म॒स्येति॑ प॒शु - का॒म॒स्य॒ । षट्त्रिꣳ॑शदक्ष॒रेति॒ षट्त्रिꣳ॑शत् - अ॒क्ष॒रा॒ । बृ॒ह॒ती । बार्.ह॑ताः । प॒शवः॑ । बृ॒ह॒त्या । ए॒व । अ॒स्मै॒ । प॒शून् ।  \newline


\textbf{Krama Paata} \newline

चतु॑र्विꣳशति॒मनु॑ । चतु॑र्विꣳशति॒मिति॒ चतुः॑ - विꣳ॒॒श॒ति॒म् । अनु॑ ब्रूयात् । ब्रू॒या॒द् ब्र॒ह्म॒व॒र्च॒सका॑मस्य । ब्र॒ह्म॒व॒र्च॒सका॑मस्य॒ चतु॑र्विꣳशत्यक्षरा । ब्र॒ह्म॒व॒र्च॒सका॑म॒स्येति॑ ब्रह्मवर्च॒स - का॒म॒स्य॒ । चतु॑र्विꣳशत्यक्षरा गाय॒त्री । चतु॑र्विꣳशत्यक्ष॒रेति॒ चतु॑र्विꣳशति - अ॒क्ष॒रा॒ । गा॒य॒त्री गा॑य॒त्री । गा॒य॒त्री ब्र॑ह्मवर्च॒सम् । ब्र॒ह्म॒व॒र्च॒सम् गा॑यत्रि॒या । ब्र॒ह्म॒व॒र्च॒समिति॑ ब्रह्म - व॒र्च॒सम् । गा॒य॒त्रि॒यैव । ए॒वास्मै᳚ । अ॒स्मै॒ ब्र॒ह्म॒व॒र्च॒सम् । ब्र॒ह्म॒व॒र्च॒समव॑ । ब्र॒ह्म॒व॒र्च॒समिति॑ ब्रह्म - व॒र्च॒सम् । अव॑ रुन्धे । रु॒न्धे॒ त्रिꣳ॒॒शत᳚म् । त्रिꣳ॒॒शत॒मनु॑ । अनु॑ ब्रूयात् । ब्रू॒या॒दन्न॑कामस्य । अन्न॑कामस्य त्रिꣳ॒॒शद॑क्षरा । अन्न॑काम॒स्येत्यन्न॑ - का॒म॒स्य॒ । त्रिꣳ॒॒शद॑क्षरा वि॒राट् । त्रिꣳ॒॒शद॑क्ष॒रेति॑ त्रिꣳ॒॒शत् - अ॒क्ष॒रा॒ । वि॒राडन्न᳚म् । वि॒राडिति॑ वि - राट् । अन्न॑म् ॅवि॒राट् । वि॒राड् वि॒राजा᳚ । वि॒राडिति॑ वि - राट् । वि॒राजै॒व । वि॒राजेति॑ वि - राजा᳚ । ए॒वास्मै᳚ । अ॒स्मा॒ अ॒न्नाद्य᳚म् । अ॒न्नाद्य॒मव॑ । अ॒न्नाद्य॒मित्य॑न्न - अद्य᳚म् । अव॑ रुन्धे । रु॒न्धे॒ द्वात्रिꣳ॑शतम् । द्वात्रिꣳ॑शत॒मनु॑ । अनु॑ ब्रूयात् । ब्रू॒या॒त् प्र॒ति॒ष्ठाका॑मस्य । प्र॒ति॒ष्ठाका॑मस्य॒ द्वात्रिꣳ॑शदक्षरा । प्र॒ति॒ष्ठाका॑म॒स्येति॑ प्रति॒ष्ठा - का॒म॒स्य॒ । द्वात्रिꣳ॑शदक्षरा ऽनु॒ष्टुक् । द्वात्रिꣳ॑शदक्ष॒रेति॒ द्वात्रिꣳ॑शत् - अ॒क्ष॒रा॒ । अ॒नु॒ष्टुग॑नु॒ष्टुप् । अ॒नु॒ष्टुगित्य॑नु - स्तुक् । अ॒नु॒ष्टुप् छन्द॑साम् । अ॒नु॒ष्टुबित्य॑नु - स्तुप् । छन्द॑साम् प्रति॒ष्ठा । प्र॒ति॒ष्ठा प्रति॑ष्ठित्यै । प्र॒ति॒ष्ठेति॑ प्रति - स्था । प्रति॑ष्ठित्यै॒ षट्त्रिꣳ॑शतम् । प्रति॑ष्ठित्या॒ इति॒ प्रति॑ - स्थि॒त्यै॒ । षट्त्रिꣳ॑शत॒मनु॑ । षट्त्रिꣳ॑शत॒मिति॒ षट् - त्रिꣳ॒॒श॒त॒म् । अनु॑ ब्रूयात् । ब्रू॒या॒त् प॒शुका॑मस्य । प॒शुका॑मस्य॒ षट्त्रिꣳ॑शदक्षरा । प॒शुका॑म॒स्येति॑ प॒शु - का॒म॒स्य॒ । षट्त्रिꣳ॑शदक्षरा बृह॒ती । षट्त्रिꣳ॑शदक्ष॒रेति॒ षट्त्रिꣳ॑शत् - अ॒क्ष॒रा॒ । बृ॒ह॒ती बार्.ह॑ताः । बार्.ह॑ताः प॒शवः॑ । प॒शवो॑ बृह॒त्या । बृ॒ह॒त्यैव । ए॒वास्मै᳚ । अ॒स्मै॒ प॒शून् ( ) । प॒शूनव॑ \newline

\textbf{Jatai Paata} \newline

1. चतु॑र्विꣳशति॒ मन्वनु॒ चतु॑र्विꣳशति॒म् चतु॑र्विꣳशति॒ मनु॑ । \newline
2. चतु॑र्विꣳशति॒मिति॒ चतुः॑ - विꣳ॒॒श॒ति॒म् । \newline
3. अनु॑ ब्रूयाद् ब्रूया॒ दन्वनु॑ ब्रूयात् । \newline
4. ब्रू॒या॒द् ब्र॒ह्म॒व॒र्च॒सका॑मस्य ब्रह्मवर्च॒सका॑मस्य ब्रूयाद् ब्रूयाद् ब्रह्मवर्च॒सका॑मस्य । \newline
5. ब्र॒ह्म॒व॒र्च॒सका॑मस्य॒ चतु॑र्विꣳशत्यक्षरा॒ चतु॑र्विꣳशत्यक्षरा ब्रह्मवर्च॒सका॑मस्य ब्रह्मवर्च॒सका॑मस्य॒ चतु॑र्विꣳशत्यक्षरा । \newline
6. ब्र॒ह्म॒व॒र्च॒सका॑म॒स्येति॑ ब्रह्मवर्च॒स - का॒म॒स्य॒ । \newline
7. चतु॑र्विꣳशत्यक्षरा गाय॒त्री गा॑य॒त्री चतु॑र्विꣳशत्यक्षरा॒ चतु॑र्विꣳशत्यक्षरा गाय॒त्री । \newline
8. चतु॑र्विꣳशत्यक्ष॒रेति॒ चतु॑र्विꣳशति - अ॒क्ष॒रा॒ । \newline
9. गा॒य॒त्री गा॑य॒त्री । \newline
10. गा॒य॒त्री ब्र॑ह्मवर्च॒सम् ब्र॑ह्मवर्च॒सम् गा॑य॒त्री गा॑य॒त्री ब्र॑ह्मवर्च॒सम् । \newline
11. ब्र॒ह्म॒व॒र्च॒सम् गा॑यत्रि॒या गा॑यत्रि॒या ब्र॑ह्मवर्च॒सम् ब्र॑ह्मवर्च॒सम् गा॑यत्रि॒या । \newline
12. ब्र॒ह्म॒व॒र्च॒समिति॑ ब्रह्म - व॒र्च॒सम् । \newline
13. गा॒य॒त्रि॒यैवैव गा॑यत्रि॒या गा॑यत्रि॒यैव । \newline
14. ए॒वास्मा॑ अस्मा ए॒वैवास्मै᳚ । \newline
15. अ॒स्मै॒ ब्र॒ह्म॒व॒र्च॒सम् ब्र॑ह्मवर्च॒स म॑स्मा अस्मै ब्रह्मवर्च॒सम् । \newline
16. ब्र॒ह्म॒व॒र्च॒स मवाव॑ ब्रह्मवर्च॒सम् ब्र॑ह्मवर्च॒स मव॑ । \newline
17. ब्र॒ह्म॒व॒र्च॒समिति॑ ब्रह्म - व॒र्च॒सम् । \newline
18. अव॑ रुन्धे रु॒न्धे ऽवाव॑ रुन्धे । \newline
19. रु॒न्धे॒ त्रिꣳ॒॒शत॑म् त्रिꣳ॒॒शतꣳ॑ रुन्धे रुन्धे त्रिꣳ॒॒शत᳚म् । \newline
20. त्रिꣳ॒॒शत॒ मन्वनु॑ त्रिꣳ॒॒शत॑म् त्रिꣳ॒॒शत॒ मनु॑ । \newline
21. अनु॑ ब्रूयाद् ब्रूया॒ दन्वनु॑ ब्रूयात् । \newline
22. ब्रू॒या॒ दन्न॑काम॒स्या न्न॑कामस्य ब्रूयाद् ब्रूया॒ दन्न॑कामस्य । \newline
23. अन्न॑कामस्य त्रिꣳ॒॒शद॑क्षरा त्रिꣳ॒॒शद॑क्ष॒रा ऽन्न॑काम॒स्या न्न॑कामस्य त्रिꣳ॒॒शद॑क्षरा । \newline
24. अन्न॑काम॒स्येत्यन्न॑ - का॒म॒स्य॒ । \newline
25. त्रिꣳ॒॒शद॑क्षरा वि॒राड् वि॒राट् त्रिꣳ॒॒शद॑क्षरा त्रिꣳ॒॒शद॑क्षरा वि॒राट् । \newline
26. त्रिꣳ॒॒शद॑क्ष॒रेति॑ त्रिꣳ॒॒शत् - अ॒क्ष॒रा॒ । \newline
27. वि॒राडन्न॒ मन्नं॑ ॅवि॒राड् वि॒राडन्न᳚म् । \newline
28. वि॒राडिति॑ वि - राट् । \newline
29. अन्नं॑ ॅवि॒राड् वि॒राडन्न॒ मन्नं॑ ॅवि॒राट् । \newline
30. वि॒राड् वि॒राजा॑ वि॒राजा॑ वि॒राड् वि॒राड् वि॒राजा᳚ । \newline
31. वि॒राडिति॑ वि - राट् । \newline
32. वि॒राजै॒वैव वि॒राजा॑ वि॒राजै॒व । \newline
33. वि॒राजेति॑ वि - राजा᳚ । \newline
34. ए॒वास्मा॑ अस्मा ए॒वैवास्मै᳚ । \newline
35. अ॒स्मा॒ अ॒न्नाद्य॑ म॒न्नाद्य॑ मस्मा अस्मा अ॒न्नाद्य᳚म् । \newline
36. अ॒न्नाद्य॒ मवावा॒न्नाद्य॑ म॒न्नाद्य॒ मव॑ । \newline
37. अ॒न्नाद्य॒मित्य॑न्न - अद्य᳚म् । \newline
38. अव॑ रुन्धे रु॒न्धे ऽवाव॑ रुन्धे । \newline
39. रु॒न्धे॒ द्वात्रिꣳ॑शत॒म् द्वात्रिꣳ॑शतꣳ रुन्धे रुन्धे॒ द्वात्रिꣳ॑शतम् । \newline
40. द्वात्रिꣳ॑शत॒ मन्वनु॒ द्वात्रिꣳ॑शत॒म् द्वात्रिꣳ॑शत॒ मनु॑ । \newline
41. अनु॑ ब्रूयाद् ब्रूया॒ दन्वनु॑ ब्रूयात् । \newline
42. ब्रू॒या॒त् प्र॒ति॒ष्ठाका॑मस्य प्रति॒ष्ठाका॑मस्य ब्रूयाद् ब्रूयात् प्रति॒ष्ठाका॑मस्य । \newline
43. प्र॒ति॒ष्ठाका॑मस्य॒ द्वात्रिꣳ॑शदक्षरा॒ द्वात्रिꣳ॑शदक्षरा प्रति॒ष्ठाका॑मस्य प्रति॒ष्ठाका॑मस्य॒ द्वात्रिꣳ॑शदक्षरा । \newline
44. प्र॒ति॒ष्ठाका॑म॒स्येति॑ प्रति॒ष्ठा - का॒म॒स्य॒ । \newline
45. द्वात्रिꣳ॑शदक्षरा ऽनु॒ष्टुग॑नु॒ष्टुग् द्वात्रिꣳ॑शदक्षरा॒ द्वात्रिꣳ॑शदक्षरा ऽनु॒ष्टुक् । \newline
46. द्वात्रिꣳ॑शदक्ष॒रेति॒ द्वात्रिꣳ॑शत् - अ॒क्ष॒रा॒ । \newline
47. अ॒नु॒ष्टु ग॑नु॒ष्टु ब॑नु॒ष्टु ब॑नु॒ष्टु ग॑नु॒ष्टु ग॑नु॒ष्टुप् । \newline
48. अ॒नु॒ष्टुगित्य॑नु - स्तुक् । \newline
49. अ॒नु॒ष्टुप् छन्द॑सा॒म् छन्द॑सा मनु॒ष्टु ब॑नु॒ष्टुप् छन्द॑साम् । \newline
50. अ॒नु॒ष्टुबित्य॑नु - स्तुप् । \newline
51. छन्द॑साम् प्रति॒ष्ठा प्र॑ति॒ष्ठा छन्द॑सा॒म् छन्द॑साम् प्रति॒ष्ठा । \newline
52. प्र॒ति॒ष्ठा प्रति॑ष्ठित्यै॒ प्रति॑ष्ठित्यै प्रति॒ष्ठा प्र॑ति॒ष्ठा प्रति॑ष्ठित्यै । \newline
53. प्र॒ति॒ष्ठेति॑ प्रति - स्था । \newline
54. प्रति॑ष्ठित्यै॒ षट्त्रिꣳ॑शतꣳ॒॒ षट्त्रिꣳ॑शत॒म् प्रति॑ष्ठित्यै॒ प्रति॑ष्ठित्यै॒ षट्त्रिꣳ॑शतम् । \newline
55. प्रति॑ष्ठित्या॒ इति॒ प्रति॑ - स्थि॒त्यै॒ । \newline
56. षट्त्रिꣳ॑शत॒ मन्वनु॒ षट्त्रिꣳ॑शतꣳ॒॒ षट्त्रिꣳ॑शत॒ मनु॑ । \newline
57. षट्त्रिꣳ॑शत॒मिति॒ षट् - त्रिꣳ॒॒श॒त॒म् । \newline
58. अनु॑ ब्रूयाद् ब्रूया॒ दन्वनु॑ ब्रूयात् । \newline
59. ब्रू॒या॒त् प॒शुका॑मस्य प॒शुका॑मस्य ब्रूयाद् ब्रूयात् प॒शुका॑मस्य । \newline
60. प॒शुका॑मस्य॒ षट्त्रिꣳ॑शदक्षरा॒ षट्त्रिꣳ॑शदक्षरा प॒शुका॑मस्य प॒शुका॑मस्य॒ षट्त्रिꣳ॑शदक्षरा । \newline
61. प॒शुका॑म॒स्येति॑ प॒शु - का॒म॒स्य॒ । \newline
62. षट्त्रिꣳ॑शदक्षरा बृह॒ती बृ॑ह॒ती षट्त्रिꣳ॑शदक्षरा॒ षट्त्रिꣳ॑शदक्षरा बृह॒ती । \newline
63. षट्त्रिꣳ॑शदक्ष॒रेति॒ षट्त्रिꣳ॑शत् - अ॒क्ष॒रा॒ । \newline
64. बृ॒ह॒ती बार्.ह॑ता॒ बार्.ह॑ता बृह॒ती बृ॑ह॒ती बार्.ह॑ताः । \newline
65. बार्.ह॑ताः प॒शवः॑ प॒शवो॒ बार्.ह॑ता॒ बार्.ह॑ताः प॒शवः॑ । \newline
66. प॒शवो॑ बृह॒त्या बृ॑ह॒त्या प॒शवः॑ प॒शवो॑ बृह॒त्या । \newline
67. बृ॒ह॒त्यैवैव बृ॑ह॒त्या बृ॑ह॒त्यैव । \newline
68. ए॒वास्मा॑ अस्मा ए॒वैवास्मै᳚ । \newline
69. अ॒स्मै॒ प॒शून् प॒शू न॑स्मा अस्मै प॒शून् । \newline
70. प॒शू नवाव॑ प॒शून् प॒शू नव॑ । \newline

\textbf{Ghana Paata } \newline

1. चतु॑र्विꣳशति॒ मन्वनु॒ चतु॑र्विꣳशति॒म् चतु॑र्विꣳशति॒ मनु॑ ब्रूयाद् ब्रूया॒दनु॒ चतु॑र्विꣳशति॒म् चतु॑र्विꣳशति॒ मनु॑ ब्रूयात् । \newline
2. चतु॑र्विꣳशति॒मिति॒ चतुः॑ - विꣳ॒॒श॒ति॒म् । \newline
3. अनु॑ ब्रूयाद् ब्रूया॒दन्वनु॑ ब्रूयाद् ब्रह्मवर्च॒सका॑मस्य ब्रह्मवर्च॒सका॑मस्य ब्रूया॒दन्वनु॑ ब्रूयाद् ब्रह्मवर्च॒सका॑मस्य । \newline
4. ब्रू॒या॒द् ब्र॒ह्म॒व॒र्च॒सका॑मस्य ब्रह्मवर्च॒सका॑मस्य ब्रूयाद् ब्रूयाद् ब्रह्मवर्च॒सका॑मस्य॒ चतु॑र्विꣳशत्यक्षरा॒ चतु॑र्विꣳशत्यक्षरा ब्रह्मवर्च॒सका॑मस्य ब्रूयाद् ब्रूयाद् ब्रह्मवर्च॒सका॑मस्य॒ चतु॑र्विꣳशत्यक्षरा । \newline
5. ब्र॒ह्म॒व॒र्च॒सका॑मस्य॒ चतु॑र्विꣳशत्यक्षरा॒ चतु॑र्विꣳशत्यक्षरा ब्रह्मवर्च॒सका॑मस्य ब्रह्मवर्च॒सका॑मस्य॒ चतु॑र्विꣳशत्यक्षरा गाय॒त्री गा॑य॒त्री चतु॑र्विꣳशत्यक्षरा ब्रह्मवर्च॒सका॑मस्य ब्रह्मवर्च॒सका॑मस्य॒ चतु॑र्विꣳशत्यक्षरा गाय॒त्री । \newline
6. ब्र॒ह्म॒व॒र्च॒सका॑म॒स्येति॑ ब्रह्मवर्च॒स - का॒म॒स्य॒ । \newline
7. चतु॑र्विꣳशत्यक्षरा गाय॒त्री गा॑य॒त्री चतु॑र्विꣳशत्यक्षरा॒ चतु॑र्विꣳशत्यक्षरा गाय॒त्री । \newline
8. चतु॑र्विꣳशत्यक्ष॒रेति॒ चतु॑र्विꣳशति - अ॒क्ष॒रा॒ । \newline
9. गा॒य॒त्री गा॑य॒त्री । \newline
10. गा॒य॒त्री ब्र॑ह्मवर्च॒सम् ब्र॑ह्मवर्च॒सम् गा॑य॒त्री गा॑य॒त्री ब्र॑ह्मवर्च॒सम् गा॑यत्रि॒या गा॑यत्रि॒या ब्र॑ह्मवर्च॒सम् गा॑य॒त्री गा॑य॒त्री ब्र॑ह्मवर्च॒सम् गा॑यत्रि॒या । \newline
11. ब्र॒ह्म॒व॒र्च॒सम् गा॑यत्रि॒या गा॑यत्रि॒या ब्र॑ह्मवर्च॒सम् ब्र॑ह्मवर्च॒सम् गा॑यत्रि॒यैवैव गा॑यत्रि॒या ब्र॑ह्मवर्च॒सम् ब्र॑ह्मवर्च॒सम् गा॑यत्रि॒यैव । \newline
12. ब्र॒ह्म॒व॒र्च॒समिति॑ ब्रह्म - व॒र्च॒सम् । \newline
13. गा॒य॒त्रि॒यैवैव गा॑यत्रि॒या गा॑यत्रि॒यैवास्मा॑ अस्मा ए॒व गा॑यत्रि॒या गा॑यत्रि॒यैवास्मै᳚ । \newline
14. ए॒वास्मा॑ अस्मा ए॒वैवास्मै᳚ ब्रह्मवर्च॒सम् ब्र॑ह्मवर्च॒स म॑स्मा ए॒वैवास्मै᳚ ब्रह्मवर्च॒सम् । \newline
15. अ॒स्मै॒ ब्र॒ह्म॒व॒र्च॒सम् ब्र॑ह्मवर्च॒स म॑स्मा अस्मै ब्रह्मवर्च॒स मवाव॑ ब्रह्मवर्च॒स म॑स्मा अस्मै ब्रह्मवर्च॒स मव॑ । \newline
16. ब्र॒ह्म॒व॒र्च॒स मवाव॑ ब्रह्मवर्च॒सम् ब्र॑ह्मवर्च॒स मव॑ रुन्धे रु॒न्धे ऽव॑ ब्रह्मवर्च॒सम् ब्र॑ह्मवर्च॒स मव॑ रुन्धे । \newline
17. ब्र॒ह्म॒व॒र्च॒समिति॑ ब्रह्म - व॒र्च॒सम् । \newline
18. अव॑ रुन्धे रु॒न्धे ऽवाव॑ रुन्धे त्रिꣳ॒॒शत॑म् त्रिꣳ॒॒शतꣳ॑ रु॒न्धे ऽवाव॑ रुन्धे त्रिꣳ॒॒शत᳚म् । \newline
19. रु॒न्धे॒ त्रिꣳ॒॒शत॑म् त्रिꣳ॒॒शतꣳ॑ रुन्धे रुन्धे त्रिꣳ॒॒शत॒ मन्वनु॑ त्रिꣳ॒॒शतꣳ॑ रुन्धे रुन्धे त्रिꣳ॒॒शत॒ मनु॑ । \newline
20. त्रिꣳ॒॒शत॒ मन्वनु॑ त्रिꣳ॒॒शत॑म् त्रिꣳ॒॒शत॒ मनु॑ ब्रूयाद् ब्रूया॒दनु॑ त्रिꣳ॒॒शत॑म् त्रिꣳ॒॒शत॒ मनु॑ ब्रूयात् । \newline
21. अनु॑ ब्रूयाद् ब्रूया॒ दन्वनु॑ ब्रूया॒ दन्न॑काम॒स्या न्न॑कामस्य ब्रूया॒दन्वनु॑ ब्रूया॒ दन्न॑कामस्य । \newline
22. ब्रू॒या॒ दन्न॑काम॒स्या न्न॑कामस्य ब्रूयाद् ब्रूया॒ दन्न॑कामस्य त्रिꣳ॒॒शद॑क्षरा त्रिꣳ॒॒शद॑क्ष॒रा ऽन्न॑कामस्य ब्रूयाद् ब्रूया॒ दन्न॑कामस्य त्रिꣳ॒॒शद॑क्षरा । \newline
23. अन्न॑कामस्य त्रिꣳ॒॒शद॑क्षरा त्रिꣳ॒॒शद॑क्ष॒रा ऽन्न॑काम॒स्या न्न॑कामस्य त्रिꣳ॒॒शद॑क्षरा वि॒राड् वि॒राट् त्रिꣳ॒॒शद॑क्ष॒रा ऽन्न॑काम॒स्या न्न॑कामस्य त्रिꣳ॒॒शद॑क्षरा वि॒राट् । \newline
24. अन्न॑काम॒स्येत्यन्न॑ - का॒म॒स्य॒ । \newline
25. त्रिꣳ॒॒शद॑क्षरा वि॒राड् वि॒राट् त्रिꣳ॒॒शद॑क्षरा त्रिꣳ॒॒शद॑क्षरा वि॒राडन्न॒ मन्नं॑ ॅवि॒राट् त्रिꣳ॒॒शद॑क्षरा त्रिꣳ॒॒शद॑क्षरा वि॒राडन्न᳚म् । \newline
26. त्रिꣳ॒॒शद॑क्ष॒रेति॑ त्रिꣳ॒॒शत् - अ॒क्ष॒रा॒ । \newline
27. वि॒राडन्न॒ मन्नं॑ ॅवि॒राड् वि॒राडन्नं॑ ॅवि॒राड् वि॒राडन्नं॑ ॅवि॒राड् वि॒राडन्नं॑ ॅवि॒राट् । \newline
28. वि॒राडिति॑ वि - राट् । \newline
29. अन्नं॑ ॅवि॒राड् वि॒राडन्न॒ मन्नं॑ ॅवि॒राड् वि॒राजा॑ वि॒राजा॑ वि॒राडन्न॒ मन्नं॑ ॅवि॒राड् वि॒राजा᳚ । \newline
30. वि॒राड् वि॒राजा॑ वि॒राजा॑ वि॒राड् वि॒राड् वि॒राजै॒वैव वि॒राजा॑ वि॒राड् वि॒राड् वि॒राजै॒व । \newline
31. वि॒राडिति॑ वि - राट् । \newline
32. वि॒राजै॒वैव वि॒राजा॑ वि॒राजै॒वास्मा॑ अस्मा ए॒व वि॒राजा॑ वि॒राजै॒वास्मै᳚ । \newline
33. वि॒राजेति॑ वि - राजा᳚ । \newline
34. ए॒वास्मा॑ अस्मा ए॒वैवास्मा॑ अ॒न्नाद्य॑ म॒न्नाद्य॑ मस्मा ए॒वैवास्मा॑ अ॒न्नाद्य᳚म् । \newline
35. अ॒स्मा॒ अ॒न्नाद्य॑ म॒न्नाद्य॑ मस्मा अस्मा अ॒न्नाद्य॒ मवावा॒न्नाद्य॑ मस्मा अस्मा अ॒न्नाद्य॒ मव॑ । \newline
36. अ॒न्नाद्य॒ मवावा॒न्नाद्य॑ म॒न्नाद्य॒ मव॑ रुन्धे रु॒न्धे ऽवा॒न्नाद्य॑ म॒न्नाद्य॒ मव॑ रुन्धे । \newline
37. अ॒न्नाद्य॒मित्य॑न्न - अद्य᳚म् । \newline
38. अव॑ रुन्धे रु॒न्धे ऽवाव॑ रुन्धे॒ द्वात्रिꣳ॑शत॒म् द्वात्रिꣳ॑शतꣳ रु॒न्धे ऽवाव॑ रुन्धे॒ द्वात्रिꣳ॑शतम् । \newline
39. रु॒न्धे॒ द्वात्रिꣳ॑शत॒म् द्वात्रिꣳ॑शतꣳ रुन्धे रुन्धे॒ द्वात्रिꣳ॑शत॒ मन्वनु॒ द्वात्रिꣳ॑शतꣳ रुन्धे रुन्धे॒ द्वात्रिꣳ॑शत॒ मनु॑ । \newline
40. द्वात्रिꣳ॑शत॒ मन्वनु॒ द्वात्रिꣳ॑शत॒म् द्वात्रिꣳ॑शत॒ मनु॑ ब्रूयाद् ब्रूया॒दनु॒ द्वात्रिꣳ॑शत॒म् द्वात्रिꣳ॑शत॒ मनु॑ ब्रूयात् । \newline
41. अनु॑ ब्रूयाद् ब्रूया॒दन्वनु॑ ब्रूयात् प्रति॒ष्ठाका॑मस्य प्रति॒ष्ठाका॑मस्य ब्रूया॒दन्वनु॑ ब्रूयात् प्रति॒ष्ठाका॑मस्य । \newline
42. ब्रू॒या॒त् प्र॒ति॒ष्ठाका॑मस्य प्रति॒ष्ठाका॑मस्य ब्रूयाद् ब्रूयात् प्रति॒ष्ठाका॑मस्य॒ द्वात्रिꣳ॑शदक्षरा॒ द्वात्रिꣳ॑शदक्षरा प्रति॒ष्ठाका॑मस्य ब्रूयाद् ब्रूयात् प्रति॒ष्ठाका॑मस्य॒ द्वात्रिꣳ॑शदक्षरा । \newline
43. प्र॒ति॒ष्ठाका॑मस्य॒ द्वात्रिꣳ॑शदक्षरा॒ द्वात्रिꣳ॑शदक्षरा प्रति॒ष्ठाका॑मस्य प्रति॒ष्ठाका॑मस्य॒ द्वात्रिꣳ॑शदक्षरा ऽनु॒ष्टु ग॑नु॒ष्टुग् द्वात्रिꣳ॑शदक्षरा प्रति॒ष्ठाका॑मस्य प्रति॒ष्ठाका॑मस्य॒ द्वात्रिꣳ॑शदक्षरा ऽनु॒ष्टुक् । \newline
44. प्र॒ति॒ष्ठाका॑म॒स्येति॑ प्रति॒ष्ठा - का॒म॒स्य॒ । \newline
45. द्वात्रिꣳ॑शदक्षरा ऽनु॒ष्टु ग॑नु॒ष्टुग् द्वात्रिꣳ॑शदक्षरा॒ द्वात्रिꣳ॑शदक्षरा ऽनु॒ष्टु ग॑नु॒ष्टु ब॑नु॒ष्टु ब॑नु॒ष्टुग् द्वात्रिꣳ॑शदक्षरा॒ द्वात्रिꣳ॑शदक्षरा ऽनु॒ष्टु ग॑नु॒ष्टुप् । \newline
46. द्वात्रिꣳ॑शदक्ष॒रेति॒ द्वात्रिꣳ॑शत् - अ॒क्ष॒रा॒ । \newline
47. अ॒नु॒ष्टु ग॑नु॒ष्टु ब॑नु॒ष्टु ब॑नु॒ष्टु ग॑नु॒ष्टु ग॑नु॒ष्टुप् छन्द॑सा॒म् छन्द॑सा मनु॒ष्टु ब॑नु॒ष्टु ग॑नु॒ष्टु ग॑नु॒ष्टुप् छन्द॑साम् । \newline
48. अ॒नु॒ष्टुगित्य॑नु - स्तुक् । \newline
49. अ॒नु॒ष्टुप् छन्द॑सा॒म् छन्द॑सा मनु॒ष्टु ब॑नु॒ष्टुप् छन्द॑साम् प्रति॒ष्ठा प्र॑ति॒ष्ठा छन्द॑सा मनु॒ष्टु ब॑नु॒ष्टुप् छन्द॑साम् प्रति॒ष्ठा । \newline
50. अ॒नु॒ष्टुबित्य॑नु - स्तुप् । \newline
51. छन्द॑साम् प्रति॒ष्ठा प्र॑ति॒ष्ठा छन्द॑सा॒म् छन्द॑साम् प्रति॒ष्ठा प्रति॑ष्ठित्यै॒ प्रति॑ष्ठित्यै प्रति॒ष्ठा छन्द॑सा॒म् छन्द॑साम् प्रति॒ष्ठा प्रति॑ष्ठित्यै । \newline
52. प्र॒ति॒ष्ठा प्रति॑ष्ठित्यै॒ प्रति॑ष्ठित्यै प्रति॒ष्ठा प्र॑ति॒ष्ठा प्रति॑ष्ठित्यै॒ षट्त्रिꣳ॑शतꣳ॒॒ षट्त्रिꣳ॑शत॒म् प्रति॑ष्ठित्यै प्रति॒ष्ठा प्र॑ति॒ष्ठा प्रति॑ष्ठित्यै॒ षट्त्रिꣳ॑शतम् । \newline
53. प्र॒ति॒ष्ठेति॑ प्रति - स्था । \newline
54. प्रति॑ष्ठित्यै॒ षट्त्रिꣳ॑शतꣳ॒॒ षट्त्रिꣳ॑शत॒म् प्रति॑ष्ठित्यै॒ प्रति॑ष्ठित्यै॒ षट्त्रिꣳ॑शत॒ मन्वनु॒ षट्त्रिꣳ॑शत॒म् प्रति॑ष्ठित्यै॒ प्रति॑ष्ठित्यै॒ षट्त्रिꣳ॑शत॒ मनु॑ । \newline
55. प्रति॑ष्ठित्या॒ इति॒ प्रति॑ - स्थि॒त्यै॒ । \newline
56. षट्त्रिꣳ॑शत॒ मन्वनु॒ षट्त्रिꣳ॑शतꣳ॒॒ षट्त्रिꣳ॑शत॒ मनु॑ ब्रूयाद् ब्रूया॒दनु॒ षट्त्रिꣳ॑शतꣳ॒॒ षट्त्रिꣳ॑शत॒ मनु॑ ब्रूयात् । \newline
57. षट्त्रिꣳ॑शत॒मिति॒ षट् - त्रिꣳ॒॒श॒त॒म् । \newline
58. अनु॑ ब्रूयाद् ब्रूया॒ दन्वनु॑ ब्रूयात् प॒शुका॑मस्य प॒शुका॑मस्य ब्रूया॒ दन्वनु॑ ब्रूयात् प॒शुका॑मस्य । \newline
59. ब्रू॒या॒त् प॒शुका॑मस्य प॒शुका॑मस्य ब्रूयाद् ब्रूयात् प॒शुका॑मस्य॒ षट्त्रिꣳ॑शदक्षरा॒ षट्त्रिꣳ॑शदक्षरा प॒शुका॑मस्य ब्रूयाद् ब्रूयात् प॒शुका॑मस्य॒ षट्त्रिꣳ॑शदक्षरा । \newline
60. प॒शुका॑मस्य॒ षट्त्रिꣳ॑शदक्षरा॒ षट्त्रिꣳ॑शदक्षरा प॒शुका॑मस्य प॒शुका॑मस्य॒ षट्त्रिꣳ॑शदक्षरा बृह॒ती बृ॑ह॒ती षट्त्रिꣳ॑शदक्षरा प॒शुका॑मस्य प॒शुका॑मस्य॒ षट्त्रिꣳ॑शदक्षरा बृह॒ती । \newline
61. प॒शुका॑म॒स्येति॑ प॒शु - का॒म॒स्य॒ । \newline
62. षट्त्रिꣳ॑शदक्षरा बृह॒ती बृ॑ह॒ती षट्त्रिꣳ॑शदक्षरा॒ षट्त्रिꣳ॑शदक्षरा बृह॒ती बार्.ह॑ता॒ बार्.ह॑ता बृह॒ती षट्त्रिꣳ॑शदक्षरा॒ षट्त्रिꣳ॑शदक्षरा बृह॒ती बार्.ह॑ताः । \newline
63. षट्त्रिꣳ॑शदक्ष॒रेति॒ षट्त्रिꣳ॑शत् - अ॒क्ष॒रा॒ । \newline
64. बृ॒ह॒ती बार्.ह॑ता॒ बार्.ह॑ता बृह॒ती बृ॑ह॒ती बार्.ह॑ताः प॒शवः॑ प॒शवो॒ बार्.ह॑ता बृह॒ती बृ॑ह॒ती बार्.ह॑ताः प॒शवः॑ । \newline
65. बार्.ह॑ताः प॒शवः॑ प॒शवो॒ बार्.ह॑ता॒ बार्.ह॑ताः प॒शवो॑ बृह॒त्या बृ॑ह॒त्या प॒शवो॒ बार्.ह॑ता॒ बार्.ह॑ताः प॒शवो॑ बृह॒त्या । \newline
66. प॒शवो॑ बृह॒त्या बृ॑ह॒त्या प॒शवः॑ प॒शवो॑ बृह॒त्यैवैव बृ॑ह॒त्या प॒शवः॑ प॒शवो॑ बृह॒त्यैव । \newline
67. बृ॒ह॒ त्यैवैव बृ॑ह॒त्या बृ॑ह॒ त्यैवास्मा॑ अस्मा ए॒व बृ॑ह॒त्या बृ॑ह॒ त्यैवास्मै᳚ । \newline
68. ए॒वास्मा॑ अस्मा ए॒वैवास्मै॑ प॒शून् प॒शू न॑स्मा ए॒वैवास्मै॑ प॒शून् । \newline
69. अ॒स्मै॒ प॒शून् प॒शू न॑स्मा अस्मै प॒शू नवाव॑ प॒शू न॑स्मा अस्मै प॒शू नव॑ । \newline
70. प॒शू नवाव॑ प॒शून् प॒शू नव॑ रुन्धे रु॒न्धे ऽव॑ प॒शून् प॒शू नव॑ रुन्धे । \newline
\pagebreak
\markright{ TS 2.5.10.4  \hfill https://www.vedavms.in \hfill}
\addcontentsline{toc}{section}{ TS 2.5.10.4 }
\section*{ TS 2.5.10.4 }

\textbf{TS 2.5.10.4 } \newline
\textbf{Samhita Paata} \newline

-नव॑ रुन्धे॒ चतु॑श्चत्वारिꣳशत॒मनु॑ ब्रूयादिन्द्रि॒यका॑मस्य॒ चतु॑श्चत्वारिꣳशदक्षरा त्रि॒ष्टुगि॑न्द्रि॒यं त्रि॒ष्टुप् त्रि॒ष्टुभै॒वास्मा॑ इन्द्रि॒यमव॑ रुन्धे॒ ऽष्टाच॑त्वारिꣳ शत॒मनु॑ ब्रूयात् प॒शुका॑मस्या॒ष्टाच॑त्वारिꣳशदक्षरा॒ जग॑ती॒ जाग॑ताः प॒शवो॒जग॑त्यै॒वास्मै॑ प॒शूनव॑ रुन्धे॒ सर्वा॑णि॒ छन्दाꣳ॒॒ स्यनु॑ ब्रूयाद्-बहुया॒जिनः॒ सर्वा॑णि॒ वा ए॒तस्य॒ छन्दाꣳ॒॒स्य व॑रुद्धानि॒ यो ब॑हुया॒ज्यप॑रिमित॒मनु॑ ब्रूया॒दप॑रिमित॒स्या व॑रुध्यै ॥ \newline

\textbf{Pada Paata} \newline

अवेति॑ । रु॒न्धे॒ । चतु॑श्चत्वारिꣳशत॒मिति॒ चतुः॑ - च॒त्वा॒रिꣳ॒॒श॒त॒म् । अन्विति॑ । ब्रू॒या॒त् । इ॒न्द्रि॒यका॑म॒स्येती᳚न्द्रि॒य - का॒म॒स्य॒ । चतु॑श्चत्वारिꣳशदक्ष॒रेति॒ चतु॑श्चत्वारिꣳशत् - अ॒क्ष॒रा॒ । त्रि॒ष्टुक् । इ॒न्द्रि॒यम् । त्रि॒ष्टुप् । त्रि॒ष्टुभा᳚ । ए॒व । अ॒स्मै॒ । इ॒न्द्रि॒यम् । अवेति॑ । रु॒न्धे॒ । अ॒ष्टाच॑त्वारिꣳशत॒मित्य॒ष्टा - च॒त्वा॒रिꣳ॒॒श॒त॒म् । अन्विति॑ । ब्रू॒या॒त् । प॒शुका॑म॒स्येति॑ प॒शु - का॒म॒स्य॒ । अ॒ष्टाच॑त्वारिꣳशदक्ष॒रेत्य॒ष्टाच॑त्वारिꣳशत् - अ॒क्ष॒रा॒ । जग॑ती । जाग॑ताः । प॒शवः॑ । जग॑त्या । ए॒व । अ॒स्मै॒ । प॒शून् । अवेति॑ । रु॒न्धे॒ । सर्वा॑णि । छन्दाꣳ॑सि । अन्विति॑ । ब्रू॒या॒त् । ब॒हु॒या॒जिन॒ इति॑ बहु - या॒जिनः॑ । सर्वा॑णि । वै । ए॒तस्य॑ । छन्दाꣳ॑सि । अव॑रुद्धा॒नीत्यव॑ - रु॒द्धा॒नि॒ । यः । ब॒हु॒या॒जीति॑ बहु - या॒जी । अप॑रिमित॒मित्यप॑रि - मि॒त॒म् । अन्विति॑ । ब्रू॒या॒त् । अप॑रिमित॒स्येत्यप॑रि - मि॒त॒स्य॒ । अव॑रुद्ध्या॒ इत्यव॑ - रु॒द्ध्यै॒ ॥  \newline


\textbf{Krama Paata} \newline

अव॑ रुन्धे । रु॒न्धे॒ चतु॑श्चत्वारिꣳशतम् । चतु॑श्चत्वारिꣳशत॒मनु॑ । चतु॑श्चत्वारिꣳशत॒मिति॒ चतुः॑ - च॒त्वा॒रिꣳ॒॒श॒त॒म् । अनु॑ ब्रूयात् । ब्रू॒या॒दि॒न्द्रि॒यका॑मस्य । इ॒न्द्रि॒यका॑मस्य॒ चतु॑श्चत्वारिꣳशदक्षरा । इ॒न्द्रि॒यका॑म॒स्येती᳚न्द्रि॒य - का॒म॒स्य॒ । चतु॑श्चत्वारिꣳशदक्षरा त्रि॒ष्टुक् । चतु॑श्चत्वारिꣳशदक्ष॒रेति॒ चतु॑श्चत्वारिꣳशत् - अ॒क्ष॒रा॒ । त्रि॒ष्टुगि॑न्द्रि॒यम् । इ॒न्द्रि॒यम् त्रि॒ष्टुप् । त्रि॒ष्टुप् त्रि॒ष्टुभा᳚ । त्रि॒ष्टुभै॒व । ए॒वास्मै᳚ । अ॒स्मा॒ इ॒न्द्रि॒यम् । इ॒न्द्रि॒यमव॑ । अव॑ रुन्धे । रु॒न्धे॒ऽष्टाच॑त्वारिꣳशतम् । अ॒ष्टाच॑त्वारिꣳशत॒मनु॑ । 
अ॒ष्टाच॑त्वारिꣳशत॒मित्य॒ष्टा - च॒त्वा॒रिꣳ॒॒श॒त॒म् । अनु॑ ब्रूयात् । ब्रू॒या॒त् प॒शुका॑मस्य । प॒शुका॑मस्या॒ष्टाच॑त्वारिꣳशदक्षरा । प॒शुका॑म॒स्येति॑ प॒शु - का॒म॒स्य॒ । अ॒ष्टाच॑त्वारिꣳशदक्षरा॒ जग॑ती । अ॒ष्टाच॑त्वारिꣳशदक्ष॒रेत्य॒ष्टाच॑त्वारिꣳशत् - अ॒क्ष॒रा॒ । जग॑ती॒ जाग॑ताः । जाग॑ताः प॒शवः॑ । प॒शवो॒ जग॑त्या । जग॑त्यै॒व । ए॒वास्मै᳚ । अ॒स्मै॒ प॒शून् । प॒शूनव॑ । अव॑ रुन्धे । रु॒न्धे॒ सर्वा॑णि । सर्वा॑णि॒ छन्दाꣳ॑सि । छन्दाꣳ॒॒स्यनु॑ । अनु॑ ब्रूयात् । ब्रू॒या॒द् ब॒हु॒या॒जिनः॑ । ब॒हु॒या॒जिनः॒ सर्वा॑णि । ब॒हु॒या॒जिन॒ इति॑ बहु - या॒जिनः॑ । सर्वा॑णि॒ वै । वा ए॒तस्य॑ । ए॒तस्य॒ छन्दाꣳ॑सि । छन्दाꣳ॒॒स्यव॑रुद्धानि । अव॑रुद्धानि॒ यः । अव॑रुद्धा॒नीत्यव॑ - रु॒द्धा॒नि॒ । यो ब॑हुया॒जी । ब॒हु॒या॒ज्यप॑रिमितम् । ब॒हु॒या॒जीति॑ बहु - या॒जी । अप॑रिमित॒मनु॑ । अप॑रिमित॒मित्यप॑रि - मि॒त॒म् । अनु॑ ब्रूयात् । ब्रू॒या॒दप॑रिमितस्य । अप॑रिमित॒स्यावरु॑द्ध्यै । अप॑रिमित॒स्येतप॑रि - मि॒त॒स्य॒ । अव॑रुद्ध्या॒ इत्यव॑ - रु॒द्ध्यै॒ । \newline

\textbf{Jatai Paata} \newline

1. अव॑ रुन्धे रु॒न्धे ऽवाव॑ रुन्धे । \newline
2. रु॒न्धे॒ चतु॑श्चत्वारिꣳशत॒म् चतु॑श्चत्वारिꣳशतꣳ रुन्धे रुन्धे॒ चतु॑श्चत्वारिꣳशतम् । \newline
3. चतु॑श्चत्वारिꣳशत॒ मन्वनु॒ चतु॑श्चत्वारिꣳशत॒म् चतु॑श्चत्वारिꣳशत॒ मनु॑ । \newline
4. चतु॑श्चत्वारिꣳशत॒मिति॒ चतुः॑ - च॒त्वा॒रिꣳ॒॒श॒त॒म् । \newline
5. अनु॑ ब्रूयाद् ब्रूया॒ दन्वनु॑ ब्रूयात् । \newline
6. ब्रू॒या॒ दि॒न्द्रि॒यका॑मस्ये न्द्रि॒यका॑मस्य ब्रूयाद् ब्रूया दिन्द्रि॒यका॑मस्य । \newline
7. इ॒न्द्रि॒यका॑मस्य॒ चतु॑श्चत्वारिꣳशदक्षरा॒ चतु॑श्चत्वारिꣳशदक्ष रेन्द्रि॒यका॑मस्ये न्द्रि॒यका॑मस्य॒ चतु॑श्चत्वारिꣳशदक्षरा । \newline
8. इ॒न्द्रि॒यका॑म॒स्येती᳚न्द्रि॒य - का॒म॒स्य॒ । \newline
9. चतु॑श्चत्वारिꣳशदक्षरा त्रि॒ष्टुक् त्रि॒ष्टुक् चतु॑श्चत्वारिꣳशदक्षरा॒ चतु॑श्चत्वारिꣳशदक्षरा त्रि॒ष्टुक् । \newline
10. चतु॑श्चत्वारिꣳशदक्ष॒रेति॒ चतु॑श्चत्वारिꣳशत् - अ॒क्ष॒रा॒ । \newline
11. त्रि॒ष्टु गि॑न्द्रि॒य मि॑न्द्रि॒यम् त्रि॒ष्टुक् त्रि॒ष्टु गि॑न्द्रि॒यम् । \newline
12. इ॒न्द्रि॒यम् त्रि॒ष्टुप् त्रि॒ष्टुबि॑न्द्रि॒य मि॑न्द्रि॒यम् त्रि॒ष्टुप् । \newline
13. त्रि॒ष्टुप् त्रि॒ष्टुभा᳚ त्रि॒ष्टुभा᳚ त्रि॒ष्टुप् त्रि॒ष्टुप् त्रि॒ष्टुभा᳚ । \newline
14. त्रि॒ष्टुभै॒वैव त्रि॒ष्टुभा᳚ त्रि॒ष्टुभै॒व । \newline
15. ए॒वास्मा॑ अस्मा ए॒वैवास्मै᳚ । \newline
16. अ॒स्मा॒ इ॒न्द्रि॒य मि॑न्द्रि॒य म॑स्मा अस्मा इन्द्रि॒यम् । \newline
17. इ॒न्द्रि॒य मवावे᳚ न्द्रि॒य मि॑न्द्रि॒य मव॑ । \newline
18. अव॑ रुन्धे रु॒न्धे ऽवाव॑ रुन्धे । \newline
19. रु॒न्धे॒ ऽष्टाच॑त्वारिꣳशत म॒ष्टाच॑त्वारिꣳशतꣳ रुन्धे रुन्धे॒ ऽष्टाच॑त्वारिꣳशतम् । \newline
20. अ॒ष्टाच॑त्वारिꣳशत॒ मन्वन्व॒ष्टाच॑त्वारिꣳशत म॒ष्टाच॑त्वारिꣳशत॒ मनु॑ । \newline
21. अ॒ष्टाच॑त्वारिꣳशत॒मित्य॒ष्टा - च॒त्वा॒रिꣳ॒॒श॒त॒म् । \newline
22. अनु॑ ब्रूयाद् ब्रूया॒दन्वनु॑ ब्रूयात् । \newline
23. ब्रू॒या॒त् प॒शुका॑मस्य प॒शुका॑मस्य ब्रूयाद् ब्रूयात् प॒शुका॑मस्य । \newline
24. प॒शुका॑मस्या॒ ष्टाच॑त्वारिꣳशदक्षरा॒ ऽष्टाच॑त्वारिꣳशदक्षरा प॒शुका॑मस्य प॒शुका॑मस्या॒ ष्टाच॑त्वारिꣳशदक्षरा । \newline
25. प॒शुका॑म॒स्येति॑ प॒शु - का॒म॒स्य॒ । \newline
26. अ॒ष्टाच॑त्वारिꣳशदक्षरा॒ जग॑ती॒ जग॑त्य॒ष्टाच॑त्वारिꣳशदक्षरा॒ ऽष्टाच॑त्वारिꣳशदक्षरा॒ जग॑ती । \newline
27. अ॒ष्टाच॑त्वारिꣳशदक्ष॒रेत्य॒ष्टाच॑त्वारिꣳशत् - अ॒क्ष॒रा॒ । \newline
28. जग॑ती॒ जाग॑ता॒ जाग॑ता॒ जग॑ती॒ जग॑ती॒ जाग॑ताः । \newline
29. जाग॑ताः प॒शवः॑ प॒शवो॒ जाग॑ता॒ जाग॑ताः प॒शवः॑ । \newline
30. प॒शवो॒ जग॑त्या॒ जग॑त्या प॒शवः॑ प॒शवो॒ जग॑त्या । \newline
31. जग॑त्यै॒वैव जग॑त्या॒ जग॑त्यै॒व । \newline
32. ए॒वास्मा॑ अस्मा ए॒वैवास्मै᳚ । \newline
33. अ॒स्मै॒ प॒शून् प॒शू न॑स्मा अस्मै प॒शून् । \newline
34. प॒शू नवाव॑ प॒शून् प॒शू नव॑ । \newline
35. अव॑ रुन्धे रु॒न्धे ऽवाव॑ रुन्धे । \newline
36. रु॒न्धे॒ सर्वा॑णि॒ सर्वा॑णि रुन्धे रुन्धे॒ सर्वा॑णि । \newline
37. सर्वा॑णि॒ छन्दाꣳ॑सि॒ छन्दाꣳ॑सि॒ सर्वा॑णि॒ सर्वा॑णि॒ छन्दाꣳ॑सि । \newline
38. छन्दाꣳ॒॒ स्यन्वनु॒ छन्दाꣳ॑सि॒ छन्दाꣳ॒॒ स्यनु॑ । \newline
39. अनु॑ ब्रूयाद् ब्रूया॒ दन्वनु॑ ब्रूयात् । \newline
40. ब्रू॒या॒द् ब॒हु॒या॒जिनो॑ बहुया॒जिनो᳚ ब्रूयाद् ब्रूयाद् बहुया॒जिनः॑ । \newline
41. ब॒हु॒या॒जिनः॒ सर्वा॑णि॒ सर्वा॑णि बहुया॒जिनो॑ बहुया॒जिनः॒ सर्वा॑णि । \newline
42. ब॒हु॒या॒जिन॒ इति॑ बहु - या॒जिनः॑ । \newline
43. सर्वा॑णि॒ वै वै सर्वा॑णि॒ सर्वा॑णि॒ वै । \newline
44. वा ए॒तस्यै॒तस्य॒ वै वा ए॒तस्य॑ । \newline
45. ए॒तस्य॒ छन्दाꣳ॑सि॒ छन्दाꣳ॑ स्ये॒त स्यै॒तस्य॒ छन्दाꣳ॑सि । \newline
46. छन्दाꣳ॒॒ स्यव॑रुद्धा॒ न्यव॑रुद्धानि॒ छन्दाꣳ॑सि॒ छन्दाꣳ॒॒ स्यव॑रुद्धानि । \newline
47. अव॑रुद्धानि॒ यो यो ऽव॑रुद्धा॒ न्यव॑रुद्धानि॒ यः । \newline
48. अव॑रुद्धा॒नीत्यव॑ - रु॒द्धा॒नि॒ । \newline
49. यो ब॑हुया॒जी ब॑हुया॒जी यो यो ब॑हुया॒जी । \newline
50. ब॒हु॒या॒ ज्यप॑रिमित॒ मप॑रिमितम् बहुया॒जी ब॑हुया॒ ज्यप॑रिमितम् । \newline
51. ब॒हु॒या॒जीति॑ बहु - या॒जी । \newline
52. अप॑रिमित॒ मन्व न्वप॑रिमित॒ मप॑रिमित॒ मनु॑ । \newline
53. अप॑रिमित॒मित्यप॑रि - मि॒त॒म् । \newline
54. अनु॑ ब्रूयाद् ब्रूया॒ दन्वनु॑ ब्रूयात् । \newline
55. ब्रू॒या॒ दप॑रिमित॒स्या प॑रिमितस्य ब्रूयाद् ब्रूया॒ दप॑रिमितस्य । \newline
56. अप॑रिमित॒स्या व॑रुद्ध्या॒ अव॑रुद्ध्या॒ अप॑रिमित॒स्या प॑रिमित॒स्या व॑रुद्ध्यै । \newline
57. अप॑रिमित॒स्येत्यप॑रि - मि॒त॒स्य॒ । \newline
58. अव॑रुद्ध्या॒ इत्यव॑ - रु॒द्ध्यै॒ । \newline

\textbf{Ghana Paata } \newline

1. अव॑ रुन्धे रु॒न्धे ऽवाव॑ रुन्धे॒ चतु॑श्चत्वारिꣳशत॒म् चतु॑श्चत्वारिꣳशतꣳ रु॒न्धे ऽवाव॑ रुन्धे॒ चतु॑श्चत्वारिꣳशतम् । \newline
2. रु॒न्धे॒ चतु॑श्चत्वारिꣳशत॒म् चतु॑श्चत्वारिꣳशतꣳ रुन्धे रुन्धे॒ चतु॑श्चत्वारिꣳशत॒ मन्वनु॒ चतु॑श्चत्वारिꣳशतꣳ रुन्धे रुन्धे॒ चतु॑श्चत्वारिꣳशत॒ मनु॑ । \newline
3. चतु॑श्चत्वारिꣳशत॒ मन्वनु॒ चतु॑श्चत्वारिꣳशत॒म् चतु॑श्चत्वारिꣳशत॒ मनु॑ ब्रूयाद् ब्रूया॒दनु॒ चतु॑श्चत्वारिꣳशत॒म् चतु॑श्चत्वारिꣳशत॒ मनु॑ ब्रूयात् । \newline
4. चतु॑श्चत्वारिꣳशत॒मिति॒ चतुः॑ - च॒त्वा॒रिꣳ॒॒श॒त॒म् । \newline
5. अनु॑ ब्रूयाद् ब्रूया॒ दन्वनु॑ ब्रूया दिन्द्रि॒यका॑मस्ये न्द्रि॒यका॑मस्य ब्रूया॒ दन्वनु॑ ब्रूया दिन्द्रि॒यका॑मस्य । \newline
6. ब्रू॒या॒ दि॒न्द्रि॒यका॑मस्ये न्द्रि॒यका॑मस्य ब्रूयाद् ब्रूया दिन्द्रि॒यका॑मस्य॒ चतु॑श्चत्वारिꣳशदक्षरा॒ चतु॑श्चत्वारिꣳशदक्षरेन्द्रि॒यका॑मस्य ब्रूयाद् ब्रूया दिन्द्रि॒यका॑मस्य॒ चतु॑श्चत्वारिꣳशदक्षरा । \newline
7. इ॒न्द्रि॒यका॑मस्य॒ चतु॑श्चत्वारिꣳशदक्षरा॒ चतु॑श्चत्वारिꣳशदक्ष रेन्द्रि॒यका॑मस्ये न्द्रि॒यका॑मस्य॒ चतु॑श्चत्वारिꣳशदक्षरा त्रि॒ष्टुक् त्रि॒ष्टुक् चतु॑श्चत्वारिꣳशदक्ष रेन्द्रि॒यका॑मस्ये न्द्रि॒यका॑मस्य॒ चतु॑श्चत्वारिꣳशदक्षरा त्रि॒ष्टुक् । \newline
8. इ॒न्द्रि॒यका॑म॒स्येती᳚न्द्रि॒य - का॒म॒स्य॒ । \newline
9. चतु॑श्चत्वारिꣳशदक्षरा त्रि॒ष्टुक् त्रि॒ष्टुक् चतु॑श्चत्वारिꣳशदक्षरा॒ चतु॑श्चत्वारिꣳशदक्षरा त्रि॒ष्टुगि॑न्द्रि॒य मि॑न्द्रि॒यम् त्रि॒ष्टुक् चतु॑श्चत्वारिꣳशदक्षरा॒ चतु॑श्चत्वारिꣳशदक्षरा त्रि॒ष्टुगि॑न्द्रि॒यम् । \newline
10. चतु॑श्चत्वारिꣳशदक्ष॒रेति॒ चतु॑श्चत्वारिꣳशत् - अ॒क्ष॒रा॒ । \newline
11. त्रि॒ष्टु गि॑न्द्रि॒य मि॑न्द्रि॒यम् त्रि॒ष्टुक् त्रि॒ष्टु गि॑न्द्रि॒यम् त्रि॒ष्टुप् त्रि॒ष्टु बि॑न्द्रि॒यम् त्रि॒ष्टुक् त्रि॒ष्टु गि॑न्द्रि॒यम् त्रि॒ष्टुप् । \newline
12. इ॒न्द्रि॒यम् त्रि॒ष्टुप् त्रि॒ष्टु बि॑न्द्रि॒य मि॑न्द्रि॒यम् त्रि॒ष्टुप् त्रि॒ष्टुभा᳚ त्रि॒ष्टुभा᳚ त्रि॒ष्टु बि॑न्द्रि॒य मि॑न्द्रि॒यम् त्रि॒ष्टुप् त्रि॒ष्टुभा᳚ । \newline
13. त्रि॒ष्टुप् त्रि॒ष्टुभा᳚ त्रि॒ष्टुभा᳚ त्रि॒ष्टुप् त्रि॒ष्टुप् त्रि॒ष्टुभै॒वैव त्रि॒ष्टुभा᳚ त्रि॒ष्टुप् त्रि॒ष्टुप् त्रि॒ष्टुभै॒व । \newline
14. त्रि॒ष्टुभै॒वैव त्रि॒ष्टुभा᳚ त्रि॒ष्टुभै॒वास्मा॑ अस्मा ए॒व त्रि॒ष्टुभा᳚ त्रि॒ष्टुभै॒वास्मै᳚ । \newline
15. ए॒वास्मा॑ अस्मा ए॒वैवास्मा॑ इन्द्रि॒य मि॑न्द्रि॒य म॑स्मा ए॒वैवास्मा॑ इन्द्रि॒यम् । \newline
16. अ॒स्मा॒ इ॒न्द्रि॒य मि॑न्द्रि॒य म॑स्मा अस्मा इन्द्रि॒य मवावे᳚ न्द्रि॒य म॑स्मा अस्मा इन्द्रि॒य मव॑ । \newline
17. इ॒न्द्रि॒य मवावे᳚ न्द्रि॒य मि॑न्द्रि॒य मव॑ रुन्धे रु॒न्धे ऽवे᳚ न्द्रि॒य मि॑न्द्रि॒य मव॑ रुन्धे । \newline
18. अव॑ रुन्धे रु॒न्धे ऽवाव॑ रुन्धे॒ ऽष्टाच॑त्वारिꣳशत म॒ष्टाच॑त्वारिꣳशतꣳ रु॒न्धे ऽवाव॑ रुन्धे॒ ऽष्टाच॑त्वारिꣳशतम् । \newline
19. रु॒न्धे॒ ऽष्टाच॑त्वारिꣳशत म॒ष्टाच॑त्वारिꣳशतꣳ रुन्धे रुन्धे॒ ऽष्टाच॑त्वारिꣳशत॒ मन्वन्व॒ष्टाच॑त्वारिꣳशतꣳ रुन्धे रुन्धे॒ ऽष्टाच॑त्वारिꣳशत॒ मनु॑ । \newline
20. अ॒ष्टाच॑त्वारिꣳशत॒ मन्वन्व॒ष्टाच॑त्वारिꣳशत म॒ष्टाच॑त्वारिꣳशत॒ मनु॑ ब्रूयाद् ब्रूया॒ दन्व॒ष्टाच॑त्वारिꣳशत म॒ष्टाच॑त्वारिꣳशत॒ मनु॑ ब्रूयात् । \newline
21. अ॒ष्टाच॑त्वारिꣳशत॒मित्य॒ष्टा - च॒त्वा॒रिꣳ॒॒श॒त॒म् । \newline
22. अनु॑ ब्रूयाद् ब्रूया॒दन्वनु॑ ब्रूयात् प॒शुका॑मस्य प॒शुका॑मस्य ब्रूया॒ दन्वनु॑ ब्रूयात् प॒शुका॑मस्य । \newline
23. ब्रू॒या॒त् प॒शुका॑मस्य प॒शुका॑मस्य ब्रूयाद् ब्रूयात् प॒शुका॑मस्या॒ ष्टाच॑त्वारिꣳशदक्षरा॒ ऽष्टाच॑त्वारिꣳशदक्षरा प॒शुका॑मस्य ब्रूयाद् ब्रूयात् प॒शुका॑मस्या॒ ष्टाच॑त्वारिꣳशदक्षरा । \newline
24. प॒शुका॑मस्या॒ष्टाच॑त्वारिꣳशदक्षरा॒ ऽष्टाच॑त्वारिꣳशदक्षरा प॒शुका॑मस्य प॒शुका॑मस्या॒ ष्टाच॑त्वारिꣳशदक्षरा॒ जग॑ती॒ जग॑त्य॒ष्टाच॑त्वारिꣳशदक्षरा प॒शुका॑मस्य प॒शुका॑मस्या॒ ष्टाच॑त्वारिꣳशदक्षरा॒ जग॑ती । \newline
25. प॒शुका॑म॒स्येति॑ प॒शु - का॒म॒स्य॒ । \newline
26. अ॒ष्टाच॑त्वारिꣳशदक्षरा॒ जग॑ती॒ जग॑ त्य॒ष्टाच॑त्वारिꣳशदक्षरा॒ ऽष्टाच॑त्वारिꣳशदक्षरा॒ जग॑ती॒ जाग॑ता॒ जाग॑ता॒ जग॑ त्य॒ष्टाच॑त्वारिꣳशदक्षरा॒ ऽष्टाच॑त्वारिꣳशदक्षरा॒ जग॑ती॒ जाग॑ताः । \newline
27. अ॒ष्टाच॑त्वारिꣳशदक्ष॒रेत्य॒ष्टाच॑त्वारिꣳशत् - अ॒क्ष॒रा॒ । \newline
28. जग॑ती॒ जाग॑ता॒ जाग॑ता॒ जग॑ती॒ जग॑ती॒ जाग॑ताः प॒शवः॑ प॒शवो॒ जाग॑ता॒ जग॑ती॒ जग॑ती॒ जाग॑ताः प॒शवः॑ । \newline
29. जाग॑ताः प॒शवः॑ प॒शवो॒ जाग॑ता॒ जाग॑ताः प॒शवो॒ जग॑त्या॒ जग॑त्या प॒शवो॒ जाग॑ता॒ जाग॑ताः प॒शवो॒ जग॑त्या । \newline
30. प॒शवो॒ जग॑त्या॒ जग॑त्या प॒शवः॑ प॒शवो॒ जग॑त्यै॒वैव जग॑त्या प॒शवः॑ प॒शवो॒ जग॑त्यै॒व । \newline
31. जग॑त्यै॒वैव जग॑त्या॒ जग॑त्यै॒वास्मा॑ अस्मा ए॒व जग॑त्या॒ जग॑त्यै॒वास्मै᳚ । \newline
32. ए॒वास्मा॑ अस्मा ए॒वैवास्मै॑ प॒शून् प॒शू न॑स्मा ए॒वैवास्मै॑ प॒शून् । \newline
33. अ॒स्मै॒ प॒शून् प॒शू न॑स्मा अस्मै प॒शू नवाव॑ प॒शू न॑स्मा अस्मै प॒शू नव॑ । \newline
34. प॒शू नवाव॑ प॒शून् प॒शू नव॑ रुन्धे रु॒न्धे ऽव॑ प॒शून् प॒शू नव॑ रुन्धे । \newline
35. अव॑ रुन्धे रु॒न्धे ऽवाव॑ रुन्धे॒ सर्वा॑णि॒ सर्वा॑णि रु॒न्धे ऽवाव॑ रुन्धे॒ सर्वा॑णि । \newline
36. रु॒न्धे॒ सर्वा॑णि॒ सर्वा॑णि रुन्धे रुन्धे॒ सर्वा॑णि॒ छन्दाꣳ॑सि॒ छन्दाꣳ॑सि॒ सर्वा॑णि रुन्धे रुन्धे॒ सर्वा॑णि॒ छन्दाꣳ॑सि । \newline
37. सर्वा॑णि॒ छन्दाꣳ॑सि॒ छन्दाꣳ॑सि॒ सर्वा॑णि॒ सर्वा॑णि॒ छन्दाꣳ॒॒स्यन्वनु॒ छन्दाꣳ॑सि॒ सर्वा॑णि॒ सर्वा॑णि॒ छन्दाꣳ॒॒स्यनु॑ । \newline
38. छन्दाꣳ॒॒स्यन्वनु॒ छन्दाꣳ॑सि॒ छन्दाꣳ॒॒स्यनु॑ ब्रूयाद् ब्रूया॒दनु॒ छन्दाꣳ॑सि॒ छन्दाꣳ॒॒स्यनु॑ ब्रूयात् । \newline
39. अनु॑ ब्रूयाद् ब्रूया॒दन्वनु॑ ब्रूयाद् बहुया॒जिनो॑ बहुया॒जिनो᳚ ब्रूया॒दन्वनु॑ ब्रूयाद् बहुया॒जिनः॑ । \newline
40. ब्रू॒या॒द् ब॒हु॒या॒जिनो॑ बहुया॒जिनो᳚ ब्रूयाद् ब्रूयाद् बहुया॒जिनः॒ सर्वा॑णि॒ सर्वा॑णि बहुया॒जिनो᳚ ब्रूयाद् ब्रूयाद् बहुया॒जिनः॒ सर्वा॑णि । \newline
41. ब॒हु॒या॒जिनः॒ सर्वा॑णि॒ सर्वा॑णि बहुया॒जिनो॑ बहुया॒जिनः॒ सर्वा॑णि॒ वै वै सर्वा॑णि बहुया॒जिनो॑ बहुया॒जिनः॒ सर्वा॑णि॒ वै । \newline
42. ब॒हु॒या॒जिन॒ इति॑ बहु - या॒जिनः॑ । \newline
43. सर्वा॑णि॒ वै वै सर्वा॑णि॒ सर्वा॑णि॒ वा ए॒तस्यै॒तस्य॒ वै सर्वा॑णि॒ सर्वा॑णि॒ वा ए॒तस्य॑ । \newline
44. वा ए॒तस्यै॒तस्य॒ वै वा ए॒तस्य॒ छन्दाꣳ॑सि॒ छन्दाꣳ॑स्ये॒तस्य॒ वै वा ए॒तस्य॒ छन्दाꣳ॑सि । \newline
45. ए॒तस्य॒ छन्दाꣳ॑सि॒ छन्दाꣳ॑ स्ये॒त स्यै॒तस्य॒ छन्दाꣳ॒॒ स्यव॑रुद्धा॒ न्यव॑रुद्धानि॒ छन्दाꣳ॑ स्ये॒त स्यै॒तस्य॒ छन्दाꣳ॒॒ स्यव॑रुद्धानि । \newline
46. छन्दाꣳ॒॒ स्यव॑रुद्धा॒ न्यव॑रुद्धानि॒ छन्दाꣳ॑सि॒ छन्दाꣳ॒॒ स्यव॑रुद्धानि॒ यो यो ऽव॑रुद्धानि॒ छन्दाꣳ॑सि॒ छन्दाꣳ॒॒ स्यव॑रुद्धानि॒ यः । \newline
47. अव॑रुद्धानि॒ यो यो ऽव॑रुद्धा॒ न्यव॑रुद्धानि॒ यो ब॑हुया॒जी ब॑हुया॒जी यो ऽव॑रुद्धा॒ न्यव॑रुद्धानि॒ यो ब॑हुया॒जी । \newline
48. अव॑रुद्धा॒नीत्यव॑ - रु॒द्धा॒नि॒ । \newline
49. यो ब॑हुया॒जी ब॑हुया॒जी यो यो ब॑हुया॒ ज्यप॑रिमित॒ मप॑रिमितम् बहुया॒जी यो यो ब॑हुया॒ ज्यप॑रिमितम् । \newline
50. ब॒हु॒या॒ ज्यप॑रिमित॒ मप॑रिमितम् बहुया॒जी ब॑हुया॒ ज्यप॑रिमित॒ मन्वन्वप॑रिमितम् बहुया॒जी ब॑हुया॒ ज्यप॑रिमित॒ मनु॑ । \newline
51. ब॒हु॒या॒जीति॑ बहु - या॒जी । \newline
52. अप॑रिमित॒ मन्वन्वप॑रिमित॒ मप॑रिमित॒ मनु॑ ब्रूयाद् ब्रूया॒ दन्वप॑रिमित॒ मप॑रिमित॒ मनु॑ ब्रूयात् । \newline
53. अप॑रिमित॒मित्यप॑रि - मि॒त॒म् । \newline
54. अनु॑ ब्रूयाद् ब्रूया॒दन्वनु॑ ब्रूया॒ दप॑रिमित॒स्या प॑रिमितस्य ब्रूया॒दन्वनु॑ ब्रूया॒ दप॑रिमितस्य । \newline
55. ब्रू॒या॒ दप॑रिमित॒स्या प॑रिमितस्य ब्रूयाद् ब्रूया॒ दप॑रिमित॒स्या व॑रुद्ध्या॒ अव॑रुद्ध्या॒ अप॑रिमितस्य ब्रूयाद् ब्रूया॒ दप॑रिमित॒स्या व॑रुद्ध्यै । \newline
56. अप॑रिमित॒स्या व॑रुद्ध्या॒ अव॑रुद्ध्या॒ अप॑रिमित॒स्या प॑रिमित॒स्या व॑रुद्ध्यै । \newline
57. अप॑रिमित॒स्येत्यप॑रि - मि॒त॒स्य॒ । \newline
58. अव॑रुद्ध्या॒ इत्यव॑ - रु॒द्ध्यै॒ । \newline
\pagebreak
\markright{ TS 2.5.11.1  \hfill https://www.vedavms.in \hfill}
\addcontentsline{toc}{section}{ TS 2.5.11.1 }
\section*{ TS 2.5.11.1 }

\textbf{TS 2.5.11.1 } \newline
\textbf{Samhita Paata} \newline

निवी॑तं मनु॒ष्या॑णां प्राचीनावी॒तं पि॑तृ॒णामुप॑वीतं दे॒वाना॒मुप॑ व्ययते देवल॒क्ष्ममे॒व तत् कु॑रुते॒ तिष्ठ॒न्नन्वा॑ह॒ तिष्ठ॒न॒. ह्याश्रु॑ततरं॒ ॅवद॑ति॒ तिष्ठ॒न्नन्वा॑ह सुव॒र्गस्य॑ लो॒कस्या॒भिजि॑त्या॒ आसी॑नो यजत्य॒स्मिन्ने॒व लो॒के प्रति॑तिष्ठति॒ यत् क्रौ॒ञ्चम॒न्वाहा॑ऽऽसु॒रं तद्-यन्म॒न्द्रं मा॑नु॒षं तद्यद॑न्त॒रा तथ् सदे॑वमन्त॒राऽनूच्यꣳ॑ सदेव॒त्वाय॑ वि॒द्वाꣳसो॒ वै - [  ] \newline

\textbf{Pada Paata} \newline

निवी॑त॒मिति॒ नि - वी॒त॒म् । म॒नु॒ष्या॑णाम् । प्रा॒ची॒ना॒वी॒तमिति॑ प्राचीन - आ॒वी॒तम् । पि॒तृ॒णाम् । उप॑वीत॒मित्युप॑ - वी॒त॒म् । दे॒वाना᳚म् । उपेति॑ । व्य॒य॒ते॒ । दे॒व॒ल॒क्ष्ममिति॑ देव - ल॒क्ष्मम् । ए॒व । तत् । कु॒रु॒ते॒ । तिष्ठन्न्॑ । अन्विति॑ । आ॒ह॒ । तिष्ठन्न्॑ । हि । आश्रु॑ततर॒मित्याश्रु॑त - त॒र॒म् । वद॑ति । तिष्ठन्न्॑ । अन्विति॑ । आ॒ह॒ । सु॒व॒र्गस्येति॑ सुवः - गस्य॑ । लो॒कस्य॑ । अ॒भिजि॑त्या॒ इत्य॒भि - जि॒त्यै॒ । आसी॑नः । य॒ज॒ति॒ । अ॒स्मिन्न् । ए॒व । लो॒के । प्रतीति॑ । ति॒ष्ठ॒ति॒ । यत् । क्रौ॒ञ्चम् । अ॒न्वाहेत्य॑नु-आह॑ । आ॒सु॒रम् । तत् । यत् । म॒न्द्रम् । मा॒नु॒षम् । तत् । यत् । अ॒न्त॒रा । तत् । सदे॑व॒मिति॒ स - दे॒व॒म् । अ॒न्त॒रा । अ॒नूच्य॒मित्य॑नु - उच्य᳚म् । स॒दे॒व॒त्वायेति॑ सदेव - त्वाय॑ । वि॒द्वाꣳसः॑ । वै ।  \newline


\textbf{Krama Paata} \newline

निवी॑तम् मनु॒ष्या॑णाम् । निवी॑त॒मिति॒ नि - वी॒त॒म् । म॒नु॒ष्या॑णाम् प्राचीनावी॒तम् । प्रा॒ची॒ना॒वी॒तम् पि॑तृ॒णाम् । प्रा॒ची॒ना॒वी॒तमिति॑ प्राचीन - आ॒वी॒तम् । पि॒तृ॒णामुप॑वीतम् । उप॑वीतम् दे॒वाना᳚म् । उप॑वीत॒मित्युप॑ - वी॒त॒म् । दे॒वाना॒मुप॑ । उप॑ व्ययते । व्य॒य॒ते॒ दे॒व॒ल॒क्ष्मम् । दे॒व॒ल॒क्ष्ममे॒व । दे॒व॒ल॒क्ष्ममिति॑ देव - ल॒क्ष्मम् । ए॒व तत् । तत् कु॑रुते । कु॒रु॒ते॒ तिष्ठन्न्॑ । तिष्ठ॒न्ननु॑ । अन्वा॑ह । आ॒ह॒ तिष्ठन्न्॑ । तिष्ठ॒न्॒. हि । ह्याश्रु॑ततरम् । आश्रु॑ततर॒म् ॅवद॑ति । आश्रु॑ततर॒मित्याश्रु॑त - त॒र॒म् । वद॑ति॒ तिष्ठन्न्॑ । तिष्ठ॒न्ननु॑ । अन्वा॑ह । आ॒ह॒ सु॒व॒र्गस्य॑ । सु॒व॒र्गस्य॑ लो॒कस्य॑ । सु॒र्व॒र्गस्येति॑ सुवः - गस्य॑ । लो॒कस्या॒भिजि॑त्ये । अ॒भिजि॑त्या॒ आसी॑नः । अ॒भिजि॑त्या॒ इत्य॒भि - जि॒त्यै॒ । आसी॑नो यजति । य॒ज॒त्य॒स्मिन्न् । अ॒स्मिन्ने॒व । ए॒व लो॒के । लो॒के प्रति॑ । प्रति॑ तिष्ठति । ति॒ष्ठ॒ति॒ यत् । यत् क्रौ॒ञ्चम् । क्रौ॒ञ्चम॒न्वाह॑ । अ॒न्वाहा॑सु॒रम् । अ॒न्वाहेत्य॑नु - आह॑ । आ॒सु॒रम् तत् । तद् यत् । यन् म॒न्द्रम् । म॒न्द्रम् मा॑नु॒षम् । मा॒नु॒षम् तत् । तद् यत् । यद॑न्त॒रा । अ॒न्त॒रा तत् । तथ् सदे॑वम् । सदे॑वमन्त॒रा । सदे॑व॒मिति॒ स - दे॒व॒म् । अ॒न्त॒राऽनूच्य᳚म् । अ॒नूच्यꣳ॑ सदेव॒त्वाय॑ । अ॒नूच्य॒मित्य॑नु - उच्य᳚म् । स॒दे॒व॒त्वाय॑ वि॒द्वाꣳसः॑ । स॒दे॒व॒त्वायेति॑ सदेव - त्वाय॑ । वि॒द्वाꣳसो॒ वै । वै पु॒रा \newline

\textbf{Jatai Paata} \newline

1. निवी॑तम् मनु॒ष्या॑णाम् मनु॒ष्या॑णा॒म् निवी॑त॒म् निवी॑तम् मनु॒ष्या॑णाम् । \newline
2. निवी॑त॒मिति॒ नि - वी॒त॒म् । \newline
3. म॒नु॒ष्या॑णाम् प्राचीनावी॒तम् प्रा॑चीनावी॒तम् म॑नु॒ष्या॑णाम् मनु॒ष्या॑णाम् प्राचीनावी॒तम् । \newline
4. प्रा॒ची॒ना॒वी॒तम् पि॑तृ॒णाम् पि॑तृ॒णाम् प्रा॑चीनावी॒तम् प्रा॑चीनावी॒तम् पि॑तृ॒णाम् । \newline
5. प्रा॒ची॒ना॒वी॒तमिति॑ प्राचीन - आ॒वी॒तम् । \newline
6. पि॒तृ॒णा मुप॑वीत॒ मुप॑वीतम् पितृ॒णाम् पि॑तृ॒णा मुप॑वीतम् । \newline
7. उप॑वीतम् दे॒वाना᳚म् दे॒वाना॒ मुप॑वीत॒ मुप॑वीतम् दे॒वाना᳚म् । \newline
8. उप॑वीत॒मित्युप॑ - वी॒त॒म् । \newline
9. दे॒वाना॒ मुपोप॑ दे॒वाना᳚म् दे॒वाना॒ मुप॑ । \newline
10. उप॑ व्ययते व्ययत॒ उपोप॑ व्ययते । \newline
11. व्य॒य॒ते॒ दे॒व॒ल॒क्ष्मम् दे॑वल॒क्ष्मं ॅव्य॑यते व्ययते देवल॒क्ष्मम् । \newline
12. दे॒व॒ल॒क्ष्म मे॒वैव दे॑वल॒क्ष्मम् दे॑वल॒क्ष्म मे॒व । \newline
13. दे॒व॒ल॒क्ष्ममिति॑ देव - ल॒क्ष्मम् । \newline
14. ए॒व तत् तदे॒वैव तत् । \newline
15. तत् कु॑रुते कुरुते॒ तत् तत् कु॑रुते । \newline
16. कु॒रु॒ते॒ तिष्ठꣳ॒॒ स्तिष्ठ॑न् कुरुते कुरुते॒ तिष्ठन्न्॑ । \newline
17. तिष्ठ॒न् नन्वनु॒ तिष्ठꣳ॒॒ स्तिष्ठ॒न् ननु॑ । \newline
18. अन्वा॑ हा॒हा न्वन्वा॑ह । \newline
19. आ॒ह॒ तिष्ठꣳ॒॒ स्तिष्ठ॑न् नाहाह॒ तिष्ठन्न्॑ । \newline
20. तिष्ठ॒न्॒. हि हि तिष्ठꣳ॒॒ स्तिष्ठ॒न्॒. हि । \newline
21. ह्याश्रु॑ततर॒ माश्रु॑ततरꣳ॒॒ हि ह्याश्रु॑ततरम् । \newline
22. आश्रु॑ततरं॒ ॅवद॑ति॒ वद॒ त्याश्रु॑ततर॒ माश्रु॑ततरं॒ ॅवद॑ति । \newline
23. आश्रु॑ततर॒मित्याश्रु॑त - त॒र॒म् । \newline
24. वद॑ति॒ तिष्ठꣳ॒॒ स्तिष्ठ॒न्॒. वद॑ति॒ वद॑ति॒ तिष्ठन्न्॑ । \newline
25. तिष्ठ॒न् नन्वनु॒ तिष्ठꣳ॒॒ स्तिष्ठ॒न् ननु॑ । \newline
26. अन्वा॑ हा॒हा न्वन्वा॑ह । \newline
27. आ॒ह॒ सु॒व॒र्गस्य॑ सुव॒र्गस्या॑ हाह सुव॒र्गस्य॑ । \newline
28. सु॒व॒र्गस्य॑ लो॒कस्य॑ लो॒कस्य॑ सुव॒र्गस्य॑ सुव॒र्गस्य॑ लो॒कस्य॑ । \newline
29. सु॒व॒र्गस्येति॑ सुवः - गस्य॑ । \newline
30. लो॒कस्या॒ भिजि॑त्या अ॒भिजि॑त्यै लो॒कस्य॑ लो॒कस्या॒ भिजि॑त्यै । \newline
31. अ॒भिजि॑त्या॒ आसी॑न॒ आसी॑नो॒ ऽभिजि॑त्या अ॒भिजि॑त्या॒ आसी॑नः । \newline
32. अ॒भिजि॑त्या॒ इत्य॒भि - जि॒त्यै॒ । \newline
33. आसी॑नो यजति यज॒ त्यासी॑न॒ आसी॑नो यजति । \newline
34. य॒ज॒ त्य॒स्मिन् न॒स्मिन्. य॑जति यज त्य॒स्मिन्न् । \newline
35. अ॒स्मिन् ने॒वैवास्मिन् न॒स्मिन् ने॒व । \newline
36. ए॒व लो॒के लो॒क ए॒वैव लो॒के । \newline
37. लो॒के प्रति॒ प्रति॑ लो॒के लो॒के प्रति॑ । \newline
38. प्रति॑ तिष्ठति तिष्ठति॒ प्रति॒ प्रति॑ तिष्ठति । \newline
39. ति॒ष्ठ॒ति॒ यद् यत् ति॑ष्ठति तिष्ठति॒ यत् । \newline
40. यत् क्रौ॒ञ्चम् क्रौ॒ञ्चं ॅयद् यत् क्रौ॒ञ्चम् । \newline
41. क्रौ॒ञ्च म॒न्वाहा॒ न्वाह॑ क्रौ॒ञ्चम् क्रौ॒ञ्च म॒न्वाह॑ । \newline
42. अ॒न्वाहा॑ सु॒र मा॑सु॒र म॒न्वाहा॒न्वाहा॑ सु॒रम् । \newline
43. अ॒न्वाहेत्य॑नु - आह॑ । \newline
44. आ॒सु॒रम् तत् तदा॑सु॒र मा॑सु॒रम् तत् । \newline
45. तद् यद् यत् तत् तद् यत् । \newline
46. यन् म॒न्द्रम् म॒न्द्रं ॅयद् यन् म॒न्द्रम् । \newline
47. म॒न्द्रम् मा॑नु॒षम् मा॑नु॒षम् म॒न्द्रम् म॒न्द्रम् मा॑नु॒षम् । \newline
48. मा॒नु॒षम् तत् तन् मा॑नु॒षम् मा॑नु॒षम् तत् । \newline
49. तद् यद् यत् तत् तद् यत् । \newline
50. यद॑न्त॒रा ऽन्त॒रा यद् यद॑न्त॒रा । \newline
51. अ॒न्त॒रा तत् तद॑न्त॒रा ऽन्त॒रा तत् । \newline
52. तथ् सदे॑वꣳ॒॒ सदे॑व॒म् तत् तथ् सदे॑वम् । \newline
53. सदे॑व मन्त॒रा ऽन्त॒रा सदे॑वꣳ॒॒ सदे॑व मन्त॒रा । \newline
54. सदे॑व॒मिति॒ स - दे॒व॒म् । \newline
55. अ॒न्त॒रा ऽनूच्य॑ म॒नूच्य॑ मन्त॒रा ऽन्त॒रा ऽनूच्य᳚म् । \newline
56. अ॒नूच्यꣳ॑ सदेव॒त्वाय॑ सदेव॒त्वाया॒ नूच्य॑ म॒नूच्यꣳ॑ सदेव॒त्वाय॑ । \newline
57. अ॒नूच्य॒मित्य॑नु - उच्य᳚म् । \newline
58. स॒दे॒व॒त्वाय॑ वि॒द्वाꣳसो॑ वि॒द्वाꣳसः॑ सदेव॒त्वाय॑ सदेव॒त्वाय॑ वि॒द्वाꣳसः॑ । \newline
59. स॒दे॒व॒त्वायेति॑ सदेव - त्वाय॑ । \newline
60. वि॒द्वाꣳसो॒ वै वै वि॒द्वाꣳसो॑ वि॒द्वाꣳसो॒ वै । \newline
61. वै पु॒रा पु॒रा वै वै पु॒रा । \newline

\textbf{Ghana Paata } \newline

1. निवी॑तम् मनु॒ष्या॑णाम् मनु॒ष्या॑णा॒म् निवी॑त॒म् निवी॑तम् मनु॒ष्या॑णाम् प्राचीनावी॒तम् प्रा॑चीनावी॒तम् म॑नु॒ष्या॑णा॒म् निवी॑त॒म् निवी॑तम् मनु॒ष्या॑णाम् प्राचीनावी॒तम् । \newline
2. निवी॑त॒मिति॒ नि - वी॒त॒म् । \newline
3. म॒नु॒ष्या॑णाम् प्राचीनावी॒तम् प्रा॑चीनावी॒तम् म॑नु॒ष्या॑णाम् मनु॒ष्या॑णाम् प्राचीनावी॒तम् पि॑तृ॒णाम् पि॑तृ॒णाम् प्रा॑चीनावी॒तम् म॑नु॒ष्या॑णाम् मनु॒ष्या॑णाम् प्राचीनावी॒तम् पि॑तृ॒णाम् । \newline
4. प्रा॒ची॒ना॒वी॒तम् पि॑तृ॒णाम् पि॑तृ॒णाम् प्रा॑चीनावी॒तम् प्रा॑चीनावी॒तम् पि॑तृ॒णा मुप॑वीत॒ मुप॑वीतम् पितृ॒णाम् प्रा॑चीनावी॒तम् प्रा॑चीनावी॒तम् पि॑तृ॒णा मुप॑वीतम् । \newline
5. प्रा॒ची॒ना॒वी॒तमिति॑ प्राचीन - आ॒वी॒तम् । \newline
6. पि॒तृ॒णा मुप॑वीत॒ मुप॑वीतम् पितृ॒णाम् पि॑तृ॒णा मुप॑वीतम् दे॒वाना᳚म् दे॒वाना॒ मुप॑वीतम् पितृ॒णाम् पि॑तृ॒णा मुप॑वीतम् दे॒वाना᳚म् । \newline
7. उप॑वीतम् दे॒वाना᳚म् दे॒वाना॒ मुप॑वीत॒ मुप॑वीतम् दे॒वाना॒ मुपोप॑ दे॒वाना॒ मुप॑वीत॒ मुप॑वीतम् दे॒वाना॒ मुप॑ । \newline
8. उप॑वीत॒मित्युप॑ - वी॒त॒म् । \newline
9. दे॒वाना॒ मुपोप॑ दे॒वाना᳚म् दे॒वाना॒ मुप॑ व्ययते व्ययत॒ उप॑ दे॒वाना᳚म् दे॒वाना॒ मुप॑ व्ययते । \newline
10. उप॑ व्ययते व्ययत॒ उपोप॑ व्ययते देवल॒क्ष्मम् दे॑वल॒क्ष्मं ॅव्य॑यत॒ उपोप॑ व्ययते देवल॒क्ष्मम् । \newline
11. व्य॒य॒ते॒ दे॒व॒ल॒क्ष्मम् दे॑वल॒क्ष्मं ॅव्य॑यते व्ययते देवल॒क्ष्म मे॒वैव दे॑वल॒क्ष्मं ॅव्य॑यते व्ययते देवल॒क्ष्म मे॒व । \newline
12. दे॒व॒ल॒क्ष्म मे॒वैव दे॑वल॒क्ष्मम् दे॑वल॒क्ष्म मे॒व तत् तदे॒व दे॑वल॒क्ष्मम् दे॑वल॒क्ष्म मे॒व तत् । \newline
13. दे॒व॒ल॒क्ष्ममिति॑ देव - ल॒क्ष्मम् । \newline
14. ए॒व तत् तदे॒वैव तत् कु॑रुते कुरुते॒ तदे॒वैव तत् कु॑रुते । \newline
15. तत् कु॑रुते कुरुते॒ तत् तत् कु॑रुते॒ तिष्ठꣳ॒॒ स्तिष्ठ॑न् कुरुते॒ तत् तत् कु॑रुते॒ तिष्ठन्न्॑ । \newline
16. कु॒रु॒ते॒ तिष्ठꣳ॒॒ स्तिष्ठ॑न् कुरुते कुरुते॒ तिष्ठ॒न् नन्वनु॒ तिष्ठ॑न् कुरुते कुरुते॒ तिष्ठ॒न् ननु॑ । \newline
17. तिष्ठ॒न् नन्वनु॒ तिष्ठꣳ॒॒ स्तिष्ठ॒न् नन्वा॑ हा॒हानु॒ तिष्ठꣳ॒॒ स्तिष्ठ॒न् नन्वा॑ह । \newline
18. अन्वा॑ हा॒हा न्वन्वा॑ह॒ तिष्ठꣳ॒॒ स्तिष्ठ॑न् ना॒हा न्वन्वा॑ह॒ तिष्ठन्न्॑ । \newline
19. आ॒ह॒ तिष्ठꣳ॒॒ स्तिष्ठ॑न् नाहाह॒ तिष्ठ॒न्॒. हि हि तिष्ठ॑न् नाहाह॒ तिष्ठ॒न्॒. हि । \newline
20. तिष्ठ॒न्॒. हि हि तिष्ठꣳ॒॒ स्तिष्ठ॒न् ह्याश्रु॑ततर॒ माश्रु॑ततरꣳ॒॒ हि तिष्ठꣳ॒॒ स्तिष्ठ॒न् ह्याश्रु॑ततरम् । \newline
21. ह्याश्रु॑ततर॒ माश्रु॑ततरꣳ॒॒ हि ह्याश्रु॑ततरं॒ ॅवद॑ति॒ वद॒ त्याश्रु॑ततरꣳ॒॒ हि ह्याश्रु॑ततरं॒ ॅवद॑ति । \newline
22. आश्रु॑ततरं॒ ॅवद॑ति॒ वद॒ त्याश्रु॑ततर॒ माश्रु॑ततरं॒ ॅवद॑ति॒ तिष्ठꣳ॒॒ स्तिष्ठ॒न्॒. वद॒ त्याश्रु॑ततर॒ माश्रु॑ततरं॒ ॅवद॑ति॒ तिष्ठन्न्॑ । \newline
23. आश्रु॑ततर॒मित्याश्रु॑त - त॒र॒म् । \newline
24. वद॑ति॒ तिष्ठꣳ॒॒ स्तिष्ठ॒न्॒. वद॑ति॒ वद॑ति॒ तिष्ठ॒न् नन्वनु॒ तिष्ठ॒न्॒. वद॑ति॒ वद॑ति॒ तिष्ठ॒न् ननु॑ । \newline
25. तिष्ठ॒न् नन्वनु॒ तिष्ठꣳ॒॒ स्तिष्ठ॒न् नन्वा॑ हा॒हानु॒ तिष्ठꣳ॒॒ स्तिष्ठ॒न् नन्वा॑ह । \newline
26. अन्वा॑ हा॒हा न्वन्वा॑ह सुव॒र्गस्य॑ सुव॒र्गस्या॒ हान्वन्वा॑ह सुव॒र्गस्य॑ । \newline
27. आ॒ह॒ सु॒व॒र्गस्य॑ सुव॒र्गस्या॑ हाह सुव॒र्गस्य॑ लो॒कस्य॑ लो॒कस्य॑ सुव॒र्गस्या॑ हाह सुव॒र्गस्य॑ लो॒कस्य॑ । \newline
28. सु॒व॒र्गस्य॑ लो॒कस्य॑ लो॒कस्य॑ सुव॒र्गस्य॑ सुव॒र्गस्य॑ लो॒कस्या॒ भिजि॑त्या अ॒भिजि॑त्यै लो॒कस्य॑ सुव॒र्गस्य॑ सुव॒र्गस्य॑ लो॒कस्या॒ भिजि॑त्यै । \newline
29. सु॒व॒र्गस्येति॑ सुवः - गस्य॑ । \newline
30. लो॒कस्या॒ भिजि॑त्या अ॒भिजि॑त्यै लो॒कस्य॑ लो॒कस्या॒ भिजि॑त्या॒ आसी॑न॒ आसी॑नो॒ ऽभिजि॑त्यै लो॒कस्य॑ लो॒कस्या॒ भिजि॑त्या॒ आसी॑नः । \newline
31. अ॒भिजि॑त्या॒ आसी॑न॒ आसी॑नो॒ ऽभिजि॑त्या अ॒भिजि॑त्या॒ आसी॑नो यजति यज॒त्यासी॑नो॒ ऽभिजि॑त्या अ॒भिजि॑त्या॒ आसी॑नो यजति । \newline
32. अ॒भिजि॑त्या॒ इत्य॒भि - जि॒त्यै॒ । \newline
33. आसी॑नो यजति यज॒त्यासी॑न॒ आसी॑नो यजत्य॒स्मिन् न॒स्मिन्. य॑ज॒त्यासी॑न॒ आसी॑नो यजत्य॒स्मिन्न् । \newline
34. य॒ज॒त्य॒स्मिन् न॒स्मिन्. य॑जति यजत्य॒स्मिन् ने॒वैवास्मिन्. य॑जति यजत्य॒स्मिन् ने॒व । \newline
35. अ॒स्मिन् ने॒वैवास्मिन् न॒स्मिन् ने॒व लो॒के लो॒क ए॒वास्मिन् न॒स्मिन् ने॒व लो॒के । \newline
36. ए॒व लो॒के लो॒क ए॒वैव लो॒के प्रति॒ प्रति॑ लो॒क ए॒वैव लो॒के प्रति॑ । \newline
37. लो॒के प्रति॒ प्रति॑ लो॒के लो॒के प्रति॑ तिष्ठति तिष्ठति॒ प्रति॑ लो॒के लो॒के प्रति॑ तिष्ठति । \newline
38. प्रति॑ तिष्ठति तिष्ठति॒ प्रति॒ प्रति॑ तिष्ठति॒ यद् यत् ति॑ष्ठति॒ प्रति॒ प्रति॑ तिष्ठति॒ यत् । \newline
39. ति॒ष्ठ॒ति॒ यद् यत् ति॑ष्ठति तिष्ठति॒ यत् क्रौ॒ञ्चम् क्रौ॒ञ्चं ॅयत् ति॑ष्ठति तिष्ठति॒ यत् क्रौ॒ञ्चम् । \newline
40. यत् क्रौ॒ञ्चम् क्रौ॒ञ्चं ॅयद् यत् क्रौ॒ञ्च म॒न्वाहा॒न्वाह॑ क्रौ॒ञ्चं ॅयद् यत् क्रौ॒ञ्च म॒न्वाह॑ । \newline
41. क्रौ॒ञ्च म॒न्वाहा॒न्वाह॑ क्रौ॒ञ्चम् क्रौ॒ञ्च म॒न्वाहा॑सु॒र मा॑सु॒र म॒न्वाह॑ क्रौ॒ञ्चम् क्रौ॒ञ्च म॒न्वाहा॑सु॒रम् । \newline
42. अ॒न्वाहा॑सु॒र मा॑सु॒र म॒न्वाहा॒ न्वाहा॑सु॒रम् तत् तदा॑सु॒र म॒न्वाहा॒ न्वाहा॑सु॒रम् तत् । \newline
43. अ॒न्वाहेत्य॑नु - आह॑ । \newline
44. आ॒सु॒रम् तत् तदा॑सु॒र मा॑सु॒रम् तद् यद् यत् तदा॑सु॒र मा॑सु॒रम् तद् यत् । \newline
45. तद् यद् यत् तत् तद् यन् म॒न्द्रम् म॒न्द्रं ॅयत् तत् तद् यन् म॒न्द्रम् । \newline
46. यन् म॒न्द्रम् म॒न्द्रं ॅयद् यन् म॒न्द्रम् मा॑नु॒षम् मा॑नु॒षम् म॒न्द्रं ॅयद् यन् म॒न्द्रम् मा॑नु॒षम् । \newline
47. म॒न्द्रम् मा॑नु॒षम् मा॑नु॒षम् म॒न्द्रम् म॒न्द्रम् मा॑नु॒षम् तत् तन् मा॑नु॒षम् म॒न्द्रम् म॒न्द्रम् मा॑नु॒षम् तत् । \newline
48. मा॒नु॒षम् तत् तन् मा॑नु॒षम् मा॑नु॒षम् तद् यद् यत् तन् मा॑नु॒षम् मा॑नु॒षम् तद् यत् । \newline
49. तद् यद् यत् तत् तद् यद॑न्त॒रा ऽन्त॒रा यत् तत् तद् यद॑न्त॒रा । \newline
50. यद॑न्त॒रा ऽन्त॒रा यद् यद॑न्त॒रा तत् तद॑न्त॒रा यद् यद॑न्त॒रा तत् । \newline
51. अ॒न्त॒रा तत् तद॑न्त॒रा ऽन्त॒रा तथ् सदे॑वꣳ॒॒ सदे॑व॒म् तद॑न्त॒रा ऽन्त॒रा तथ् सदे॑वम् । \newline
52. तथ् सदे॑वꣳ॒॒ सदे॑व॒म् तत् तथ् सदे॑व मन्त॒रा ऽन्त॒रा सदे॑व॒म् तत् तथ् सदे॑व मन्त॒रा । \newline
53. सदे॑व मन्त॒रा ऽन्त॒रा सदे॑वꣳ॒॒ सदे॑व मन्त॒रा ऽनूच्य॑ म॒नूच्य॑ मन्त॒रा सदे॑वꣳ॒॒ सदे॑व मन्त॒रा ऽनूच्य᳚म् । \newline
54. सदे॑व॒मिति॒ स - दे॒व॒म् । \newline
55. अ॒न्त॒रा ऽनूच्य॑ म॒नूच्य॑ मन्त॒रा ऽन्त॒रा ऽनूच्यꣳ॑ सदेव॒त्वाय॑ सदेव॒त्वाया॒ नूच्य॑ मन्त॒रा ऽन्त॒रा ऽनूच्यꣳ॑ सदेव॒त्वाय॑ । \newline
56. अ॒नूच्यꣳ॑ सदेव॒त्वाय॑ सदेव॒त्वाया॒ नूच्य॑ म॒नूच्यꣳ॑ सदेव॒त्वाय॑ वि॒द्वाꣳसो॑ वि॒द्वाꣳसः॑ सदेव॒त्वाया॒ नूच्य॑ म॒नूच्यꣳ॑ सदेव॒त्वाय॑ वि॒द्वाꣳसः॑ । \newline
57. अ॒नूच्य॒मित्य॑नु - उच्य᳚म् । \newline
58. स॒दे॒व॒त्वाय॑ वि॒द्वाꣳसो॑ वि॒द्वाꣳसः॑ सदेव॒त्वाय॑ सदेव॒त्वाय॑ वि॒द्वाꣳसो॒ वै वै वि॒द्वाꣳसः॑ सदेव॒त्वाय॑ सदेव॒त्वाय॑ वि॒द्वाꣳसो॒ वै । \newline
59. स॒दे॒व॒त्वायेति॑ सदेव - त्वाय॑ । \newline
60. वि॒द्वाꣳसो॒ वै वै वि॒द्वाꣳसो॑ वि॒द्वाꣳसो॒ वै पु॒रा पु॒रा वै वि॒द्वाꣳसो॑ वि॒द्वाꣳसो॒ वै पु॒रा । \newline
61. वै पु॒रा पु॒रा वै वै पु॒रा होता॑रो॒ होता॑रः पु॒रा वै वै पु॒रा होता॑रः । \newline
\pagebreak
\markright{ TS 2.5.11.2  \hfill https://www.vedavms.in \hfill}
\addcontentsline{toc}{section}{ TS 2.5.11.2 }
\section*{ TS 2.5.11.2 }

\textbf{TS 2.5.11.2 } \newline
\textbf{Samhita Paata} \newline

पु॒रा होता॑रोऽभूव॒न् तस्मा॒द्-विधृ॑ता॒ अद्ध्वा॒नोऽभू॑व॒न् न पन्था॑नः॒ सम॑रुक्षन्नन्तर्वे॒द्य॑न्यः पादो॒ भव॑ति बहिर्वे॒द्य॑न्योऽथान्वा॒हाद्ध्व॑नां॒ ॅविधृ॑त्यै प॒थामसꣳ॑ रोहा॒याथो॑ भू॒तंचै॒व भ॑वि॒ष्यच्चाव॑ रु॒न्धेऽथो॒ परि॑मितं चै॒वाप॑रिमितं॒ चाव॑ रु॒न्धेऽथो᳚ ग्रा॒म्याꣳश्चै॒व प॒शूना॑र॒ण्याꣳश्चाव॑ रु॒न्धेऽथो॑ - [  ] \newline

\textbf{Pada Paata} \newline

पु॒रा । होता॑रः । अ॒भू॒व॒न्न् । तस्मा᳚त् । विधृ॑ता॒ इति॒ वि - धृ॒ताः॒ । अद्ध्वा॑नः । अभू॑वन्न् । न । पन्था॑नः । समिति॑ । अ॒रु॒क्ष॒न्न् । अ॒न्त॒र्वे॒दीत्य॑न्तः - वे॒दि । अ॒न्यः । पादः॑ । भव॑ति । ब॒हि॒र्वे॒दीति॑ बहिः - वे॒दि । अ॒न्यः । अथ॑ । अन्विति॑ । आ॒ह॒ । अद्ध्व॑नाम् । विधृ॑त्या॒ इति॒ वि - धृ॒त्यै॒ । प॒थाम् । असꣳ॑रोहा॒येत्यसं᳚ - रो॒हा॒य॒ । अथो॒ इति॑ । भू॒तम् । च॒ । ए॒व । भ॒वि॒ष्यत् । च॒ । अवेति॑ । रु॒न्धे॒ । अथो॒ इति॑ । परि॑मित॒मिति॒ परि॑ - मि॒त॒म् । च॒ । ए॒व । अप॑रिमित॒मित्यप॑रि - मि॒त॒म् । च॒ । अवेति॑ । रु॒न्धे॒ । अथो॒ इति॑ । ग्रा॒म्यान् । च॒ । ए॒व । प॒शून् । आ॒र॒ण्यान् । च॒ । अवेति॑ । रु॒न्धे॒ । अथो॒ इति॑ ।  \newline


\textbf{Krama Paata} \newline

पु॒रा होता॑रः । होता॑रो ऽभूवन्न् । अ॒भू॒व॒न् तस्मा᳚त् । तस्मा॒द् विधृ॑ताः । विधृ॑ता॒ अद्ध्वा॑नः । विधृ॑ता॒ इति॒ वि - धृ॒ताः॒ । अद्ध्वा॒नो ऽभू॑वन्न् । अभू॑व॒न् न । न पन्था॑नः । पन्था॑नः॒ सम् । सम॑रुक्षन्न् । अ॒रु॒क्ष॒न्न॒न्त॒र्वे॒दि । अ॒न्त॒र्वे॒द्य॑न्यः । अ॒न्त॒र्वे॒दीत्य॑न्तः - वे॒दि । अ॒न्यः पादः॑ । पादो॒ भव॑ति । भव॑ति बहिर्वे॒दि । ब॒ही॒र्वे॒द्य॑न्यः । ब॒ही॒र्वे॒दीति॑ बहिः - वे॒दि । अ॒न्योऽथ॑ । अथानु॑ । अन्वा॑ह । आ॒हाद्ध्व॑नाम् । अद्ध्व॑ना॒म् ॅविधृ॑त्यै । विधृ॑त्यै प॒थाम् । विधृ॑त्या॒ इति॒ वि - धृ॒त्यै॒ । प॒थामसꣳ॑रोहाय । असꣳ॑रोहा॒याथो᳚ । असꣳ॑रोहा॒येत्यस᳚म् - रो॒हा॒य॒ । अथो॑ भू॒तम् । अथो॒ इत्यथो᳚ । भू॒तम् च॑ । चै॒व । ए॒व भ॑वि॒ष्यत् । भ॒वि॒ष्यच्च॑ । चाव॑ । अव॑ रुन्धे । रु॒न्धे ऽथो᳚ । अथो॒ परि॑मितम् । अथो॒ इत्यथो᳚ । परि॑मितम् च । परि॑मित॒मिति॒ परि॑ - मि॒त॒म् । चै॒व । ए॒वाप॑रिमितम् । अप॑रिमितम् च । अप॑रिमित॒मित्यप॑रि - मि॒त॒म् । चाव॑ । अव॑ रुन्धे । रु॒न्धे ऽथो᳚ । अथो᳚ ग्रा॒म्यान् । अथो॒ इत्यथो᳚ । ग्रा॒म्याꣳश्च॑ । चै॒व । ए॒व प॒शून् । प॒शूना॑र॒ण्यान् । आ॒र॒ण्याꣳश्च॑ । चाव॑ । अव॑ रुन्धे । रु॒न्धे ऽथो᳚ । अथो॑ देवलो॒कम् । अथो॒ इत्यथो᳚ \newline

\textbf{Jatai Paata} \newline

1. पु॒रा होता॑रो॒ होता॑रः पु॒रा पु॒रा होता॑रः । \newline
2. होता॑रो ऽभूवन् नभूव॒न्॒. होता॑रो॒ होता॑रो ऽभूवन्न् । \newline
3. अ॒भू॒व॒न् तस्मा॒त् तस्मा॑ दभूवन् नभूव॒न् तस्मा᳚त् । \newline
4. तस्मा॒द् विधृ॑ता॒ विधृ॑ता॒ स्तस्मा॒त् तस्मा॒द् विधृ॑ताः । \newline
5. विधृ॑ता॒ अद्ध्वा॒नो ऽद्ध्वा॑नो॒ विधृ॑ता॒ विधृ॑ता॒ अद्ध्वा॑नः । \newline
6. विधृ॑ता॒ इति॒ वि - धृ॒ताः॒ । \newline
7. अद्ध्वा॒नो ऽभू॑व॒न् नभू॑व॒न् नद्ध्वा॒नो ऽद्ध्वा॒नो ऽभू॑वन्न् । \newline
8. अभू॑व॒न् न नाभू॑व॒न् नभू॑व॒न् न । \newline
9. न पन्था॑नः॒ पन्था॑नो॒ न न पन्था॑नः । \newline
10. पन्था॑नः॒ सꣳ सम् पन्था॑नः॒ पन्था॑नः॒ सम् । \newline
11. स म॑रुक्षन् नरुक्ष॒न् थ्सꣳ स म॑रुक्षन्न् । \newline
12. अ॒रु॒क्ष॒न् न॒न्त॒र्वे॒ द्य॑न्तर्वे॒ द्य॑रुक्षन् नरुक्षन् नन्तर्वे॒दि । \newline
13. अ॒न्त॒र्वे॒द्या᳚(1॒)न्यो᳚(1॒) ऽन्यो᳚ ऽन्तर्वे॒ द्य॑न्तर्वे॒ द्य॑न्यः । \newline
14. अ॒न्त॒र्वे॒दीत्य॑न्तः - वे॒दि । \newline
15. अ॒न्यः पादः॒ पादो॒ ऽन्यो᳚ ऽन्यः पादः॑ । \newline
16. पादो॒ भव॑ति॒ भव॑ति॒ पादः॒ पादो॒ भव॑ति । \newline
17. भव॑ति बहिर्वे॒दि ब॑हिर्वे॒दि भव॑ति॒ भव॑ति बहिर्वे॒दि । \newline
18. ब॒हि॒र्वे॒द्या᳚(1॒)न्यो᳚ ऽन्यो ब॑हिर्वे॒दि ब॑हिर्वे॒द्य॑न्यः । \newline
19. ब॒हि॒र्वे॒दीति॑ बहिः - वे॒दि । \newline
20. अ॒न्यो ऽथाथा॒न्यो᳚ ऽन्यो ऽथ॑ । \newline
21. अथान्वन्वथा थानु॑ । \newline
22. अन्वा॑ हा॒हा न्वन्वा॑ह । \newline
23. आ॒हाद्ध्व॑ना॒ मद्ध्व॑ना माहा॒हा द्ध्व॑नाम् । \newline
24. अद्ध्व॑नां॒ ॅविधृ॑त्यै॒ विधृ॑त्या॒ अद्ध्व॑ना॒ मद्ध्व॑नां॒ ॅविधृ॑त्यै । \newline
25. विधृ॑त्यै प॒थाम् प॒थां ॅविधृ॑त्यै॒ विधृ॑त्यै प॒थाम् । \newline
26. विधृ॑त्या॒ इति॒ वि - धृ॒त्यै॒ । \newline
27. प॒था मसꣳ॑रोहा॒या सꣳ॑रोहाय प॒थाम् प॒था मसꣳ॑रोहाय । \newline
28. असꣳ॑रोहा॒या थो॒ अथो॒ असꣳ॑रोहा॒या सꣳ॑रोहा॒याथो᳚ । \newline
29. असꣳ॑रोहा॒येत्यसं᳚ - रो॒हा॒य॒ । \newline
30. अथो॑ भू॒तम् भू॒त मथो॒ अथो॑ भू॒तम् । \newline
31. अथो॒ इत्यथो᳚ । \newline
32. भू॒तम् च॑ च भू॒तम् भू॒तम् च॑ । \newline
33. चै॒वैव च॑ चै॒व । \newline
34. ए॒व भ॑वि॒ष्यद् भ॑वि॒ष्य दे॒वैव भ॑वि॒ष्यत् । \newline
35. भ॒वि॒ष्यच् च॑ च भवि॒ष्यद् भ॑वि॒ष्यच् च॑ । \newline
36. चावाव॑ च॒ चाव॑ । \newline
37. अव॑ रुन्धे रु॒न्धे ऽवाव॑ रुन्धे । \newline
38. रु॒न्धे ऽथो॒ अथो॑ रुन्धे रु॒न्धे ऽथो᳚ । \newline
39. अथो॒ परि॑मित॒म् परि॑मित॒ मथो॒ अथो॒ परि॑मितम् । \newline
40. अथो॒ इत्यथो᳚ । \newline
41. परि॑मितम् च च॒ परि॑मित॒म् परि॑मितम् च । \newline
42. परि॑मित॒मिति॒ परि॑ - मि॒त॒म् । \newline
43. चै॒वैव च॑ चै॒व । \newline
44. ए॒वा प॑रिमित॒ मप॑रिमित मे॒वैवा प॑रिमितम् । \newline
45. अप॑रिमितम् च॒ चाप॑रिमित॒ मप॑रिमितम् च । \newline
46. अप॑रिमित॒मित्यप॑रि - मि॒त॒म् । \newline
47. चावाव॑ च॒ चाव॑ । \newline
48. अव॑ रुन्धे रु॒न्धे ऽवाव॑ रुन्धे । \newline
49. रु॒न्धे ऽथो॒ अथो॑ रुन्धे रु॒न्धे ऽथो᳚ । \newline
50. अथो᳚ ग्रा॒म्यान् ग्रा॒म्या नथो॒ अथो᳚ ग्रा॒म्यान् । \newline
51. अथो॒ इत्यथो᳚ । \newline
52. ग्रा॒म्याꣳ श्च॑ च ग्रा॒म्यान् ग्रा॒म्याꣳ श्च॑ । \newline
53. चै॒वैव च॑ चै॒व । \newline
54. ए॒व प॒शून् प॒शू ने॒वैव प॒शून् । \newline
55. प॒शू ना॑र॒ण्या ना॑र॒ण्यान् प॒शून् प॒शू ना॑र॒ण्यान् । \newline
56. आ॒र॒ण्याꣳ श्च॑ चार॒ण्या ना॑र॒ण्याꣳ श्च॑ । \newline
57. चावाव॑ च॒ चाव॑ । \newline
58. अव॑ रुन्धे रु॒न्धे ऽवाव॑ रुन्धे । \newline
59. रु॒न्धे ऽथो॒ अथो॑ रुन्धे रु॒न्धे ऽथो᳚ । \newline
60. अथो॑ देवलो॒कम् दे॑वलो॒क मथो॒ अथो॑ देवलो॒कम् । \newline
61. अथो॒ इत्यथो᳚ । \newline

\textbf{Ghana Paata } \newline

1. पु॒रा होता॑रो॒ होता॑रः पु॒रा पु॒रा होता॑रो ऽभूवन् नभूव॒न्॒. होता॑रः पु॒रा पु॒रा होता॑रो ऽभूवन्न् । \newline
2. होता॑रो ऽभूवन् नभूव॒न्॒. होता॑रो॒ होता॑रो ऽभूव॒न् तस्मा॒त् तस्मा॑ दभूव॒न्॒. होता॑रो॒ होता॑रो ऽभूव॒न् तस्मा᳚त् । \newline
3. अ॒भू॒व॒न् तस्मा॒त् तस्मा॑ दभूवन् नभूव॒न् तस्मा॒द् विधृ॑ता॒ विधृ॑ता॒ स्तस्मा॑ दभूवन् नभूव॒न् तस्मा॒द् विधृ॑ताः । \newline
4. तस्मा॒द् विधृ॑ता॒ विधृ॑ता॒ स्तस्मा॒त् तस्मा॒द् विधृ॑ता॒ अद्ध्वा॒नो ऽद्ध्वा॑नो॒ विधृ॑ता॒ स्तस्मा॒त् तस्मा॒द् विधृ॑ता॒ अद्ध्वा॑नः । \newline
5. विधृ॑ता॒ अद्ध्वा॒नो ऽद्ध्वा॑नो॒ विधृ॑ता॒ विधृ॑ता॒ अद्ध्वा॒नो ऽभू॑व॒न् नभू॑व॒न् नद्ध्वा॑नो॒ विधृ॑ता॒ विधृ॑ता॒ अद्ध्वा॒नो ऽभू॑वन्न् । \newline
6. विधृ॑ता॒ इति॒ वि - धृ॒ताः॒ । \newline
7. अद्ध्वा॒नो ऽभू॑व॒न् नभू॑व॒न् नद्ध्वा॒नो ऽद्ध्वा॒नो ऽभू॑व॒न् न नाभू॑व॒न् नद्ध्वा॒नो ऽद्ध्वा॒नो ऽभू॑व॒न् न । \newline
8. अभू॑व॒न् न नाभू॑व॒न् नभू॑व॒न् न पन्था॑नः॒ पन्था॑नो॒ नाभू॑व॒न् नभू॑व॒न् न पन्था॑नः । \newline
9. न पन्था॑नः॒ पन्था॑नो॒ न न पन्था॑नः॒ सꣳ सम् पन्था॑नो॒ न न पन्था॑नः॒ सम् । \newline
10. पन्था॑नः॒ सꣳ सम् पन्था॑नः॒ पन्था॑नः॒ स म॑रुक्षन् नरुक्ष॒न् थ्सम् पन्था॑नः॒ पन्था॑नः॒ स म॑रुक्षन्न् । \newline
11. स म॑रुक्षन् नरुक्ष॒न् थ्सꣳ स म॑रुक्षन् नन्तर्वे॒ द्य॑न्तर्वे॒ द्य॑रुक्ष॒न् थ्सꣳ स म॑रुक्षन् नन्तर्वे॒दि । \newline
12. अ॒रु॒क्ष॒न् न॒न्त॒र्वे॒ द्य॑न्तर्वे॒ द्य॑रुक्षन् नरुक्षन् नन्तर्वे॒द्या᳚(1॒)न्यो᳚(ओ1॒) ऽन्यो᳚ ऽन्तर्वे॒ द्य॑रुक्षन् नरुक्षन् नन्तर्वे॒ द्य॑न्यः । \newline
13. अ॒न्त॒र्वे॒द्या᳚(1॒)न्यो᳚(ओ1॒) ऽन्यो᳚ ऽन्तर्वे॒ द्य॑न्तर्वे॒ द्य॑न्यः पादः॒ पादो॒ ऽन्यो᳚ ऽन्तर्वे॒ द्य॑न्तर्वे॒ द्य॑न्यः पादः॑ । \newline
14. अ॒न्त॒र्वे॒दीत्य॑न्तः - वे॒दि । \newline
15. अ॒न्यः पादः॒ पादो॒ ऽन्यो᳚ ऽन्यः पादो॒ भव॑ति॒ भव॑ति॒ पादो॒ ऽन्यो᳚ ऽन्यः पादो॒ भव॑ति । \newline
16. पादो॒ भव॑ति॒ भव॑ति॒ पादः॒ पादो॒ भव॑ति बहिर्वे॒दि ब॑हिर्वे॒दि भव॑ति॒ पादः॒ पादो॒ भव॑ति बहिर्वे॒दि । \newline
17. भव॑ति बहिर्वे॒दि ब॑हिर्वे॒दि भव॑ति॒ भव॑ति बहिर्वे॒द्या᳚(1॒)न्यो᳚ ऽन्यो ब॑हिर्वे॒दि भव॑ति॒ भव॑ति बहिर्वे॒द्य॑न्यः । \newline
18. ब॒हि॒र्वे॒द्या᳚(1॒)न्यो᳚ ऽन्यो ब॑हिर्वे॒दि ब॑हिर्वे॒द्य॑न्यो ऽथाथा॒न्यो ब॑हिर्वे॒दि ब॑हिर्वे॒द्य॑न्यो ऽथ॑ । \newline
19. ब॒हि॒र्वे॒दीति॑ बहिः - वे॒दि । \newline
20. अ॒न्यो ऽथाथा॒न्यो᳚ ऽन्यो ऽथा न्वन्व था॒न्यो᳚ ऽन्यो ऽथानु॑ । \newline
21. अथा न्वन्व थाथा न्वा॑हा॒हा न्वथाथा न्वा॑ह । \newline
22. अन्वा॑हा॒हा न्वन्वा॒हा द्ध्व॑ना॒ मद्ध्व॑ना मा॒हान्वन्वा॒हा द्ध्व॑नाम् । \newline
23. आ॒हाद्ध्व॑ना॒ मद्ध्व॑ना माहा॒हा द्ध्व॑नां॒ ॅविधृ॑त्यै॒ विधृ॑त्या॒ अद्ध्व॑ना माहा॒हा द्ध्व॑नां॒ ॅविधृ॑त्यै । \newline
24. अद्ध्व॑नां॒ ॅविधृ॑त्यै॒ विधृ॑त्या॒ अद्ध्व॑ना॒ मद्ध्व॑नां॒ ॅविधृ॑त्यै प॒थाम् प॒थां ॅविधृ॑त्या॒ अद्ध्व॑ना॒ मद्ध्व॑नां॒ ॅविधृ॑त्यै प॒थाम् । \newline
25. विधृ॑त्यै प॒थाम् प॒थां ॅविधृ॑त्यै॒ विधृ॑त्यै प॒था मसꣳ॑रोहा॒या सꣳ॑रोहाय प॒थां ॅविधृ॑त्यै॒ विधृ॑त्यै प॒था मसꣳ॑रोहाय । \newline
26. विधृ॑त्या॒ इति॒ वि - धृ॒त्यै॒ । \newline
27. प॒था मसꣳ॑रोहा॒या सꣳ॑रोहाय प॒थाम् प॒था मसꣳ॑रोहा॒याथो॒ अथो॒ असꣳ॑रोहाय प॒थाम् प॒था मसꣳ॑रोहा॒याथो᳚ । \newline
28. असꣳ॑रोहा॒याथो॒ अथो॒ असꣳ॑रोहा॒या सꣳ॑रोहा॒याथो॑ भू॒तम् भू॒त मथो॒ असꣳ॑रोहा॒या सꣳ॑रोहा॒याथो॑ भू॒तम् । \newline
29. असꣳ॑रोहा॒येत्यसं᳚ - रो॒हा॒य॒ । \newline
30. अथो॑ भू॒तम् भू॒त मथो॒ अथो॑ भू॒तम् च॑ च भू॒त मथो॒ अथो॑ भू॒तम् च॑ । \newline
31. अथो॒ इत्यथो᳚ । \newline
32. भू॒तम् च॑ च भू॒तम् भू॒तम् चै॒वैव च॑ भू॒तम् भू॒तम् चै॒व । \newline
33. चै॒वैव च॑ चै॒व भ॑वि॒ष्यद् भ॑वि॒ष्यदे॒व च॑ चै॒व भ॑वि॒ष्यत् । \newline
34. ए॒व भ॑वि॒ष्यद् भ॑वि॒ष्य दे॒वैव भ॑वि॒ष्यच् च॑ च भवि॒ष्य दे॒वैव भ॑वि॒ष्यच् च॑ । \newline
35. भ॒वि॒ष्यच् च॑ च भवि॒ष्यद् भ॑वि॒ष्यच् चावाव॑ च भवि॒ष्यद् भ॑वि॒ष्यच् चाव॑ । \newline
36. चावाव॑ च॒ चाव॑ रुन्धे रु॒न्धे ऽव॑ च॒ चाव॑ रुन्धे । \newline
37. अव॑ रुन्धे रु॒न्धे ऽवाव॑ रु॒न्धे ऽथो॒ अथो॑ रु॒न्धे ऽवाव॑ रु॒न्धे ऽथो᳚ । \newline
38. रु॒न्धे ऽथो॒ अथो॑ रुन्धे रु॒न्धे ऽथो॒ परि॑मित॒म् परि॑मित॒ मथो॑ रुन्धे रु॒न्धे ऽथो॒ परि॑मितम् । \newline
39. अथो॒ परि॑मित॒म् परि॑मित॒ मथो॒ अथो॒ परि॑मितम् च च॒ परि॑मित॒ मथो॒ अथो॒ परि॑मितम् च । \newline
40. अथो॒ इत्यथो᳚ । \newline
41. परि॑मितम् च च॒ परि॑मित॒म् परि॑मितम् चै॒वैव च॒ परि॑मित॒म् परि॑मितम् चै॒व । \newline
42. परि॑मित॒मिति॒ परि॑ - मि॒त॒म् । \newline
43. चै॒वैव च॑ चै॒वा प॑रिमित॒ मप॑रिमित मे॒व च॑ चै॒वा प॑रिमितम् । \newline
44. ए॒वाप॑रिमित॒ मप॑रिमित मे॒वैवा प॑रिमितम् च॒ चाप॑रिमित मे॒वैवा प॑रिमितम् च । \newline
45. अप॑रिमितम् च॒ चाप॑रिमित॒ मप॑रिमित॒म् चावाव॒ चाप॑रिमित॒ मप॑रिमित॒म् चाव॑ । \newline
46. अप॑रिमित॒मित्यप॑रि - मि॒त॒म् । \newline
47. चावाव॑ च॒ चाव॑ रुन्धे रु॒न्धे ऽव॑ च॒ चाव॑ रुन्धे । \newline
48. अव॑ रुन्धे रु॒न्धे ऽवाव॑ रु॒न्धे ऽथो॒ अथो॑ रु॒न्धे ऽवाव॑ रु॒न्धे ऽथो᳚ । \newline
49. रु॒न्धे ऽथो॒ अथो॑ रुन्धे रु॒न्धे ऽथो᳚ ग्रा॒म्यान् ग्रा॒म्या नथो॑ रुन्धे रु॒न्धे ऽथो᳚ ग्रा॒म्यान् । \newline
50. अथो᳚ ग्रा॒म्यान् ग्रा॒म्या नथो॒ अथो᳚ ग्रा॒म्याꣳ श्च॑ च ग्रा॒म्या नथो॒ अथो᳚ ग्रा॒म्याꣳ श्च॑ । \newline
51. अथो॒ इत्यथो᳚ । \newline
52. ग्रा॒म्याꣳ श्च॑ च ग्रा॒म्यान् ग्रा॒म्याꣳ श्चै॒वैव च॑ ग्रा॒म्यान् ग्रा॒म्याꣳ श्चै॒व । \newline
53. चै॒वैव च॑ चै॒व प॒शून् प॒शू ने॒व च॑ चै॒व प॒शून् । \newline
54. ए॒व प॒शून् प॒शू ने॒वैव प॒शू ना॑र॒ण्या ना॑र॒ण्यान् प॒शू ने॒वैव प॒शू ना॑र॒ण्यान् । \newline
55. प॒शू ना॑र॒ण्या ना॑र॒ण्यान् प॒शून् प॒शू ना॑र॒ण्याꣳ श्च॑ चार॒ण्यान् प॒शून् प॒शू ना॑र॒ण्याꣳ श्च॑ । \newline
56. आ॒र॒ण्याꣳ श्च॑ चार॒ण्या ना॑र॒ण्याꣳ श्चावाव॑ चार॒ण्या ना॑र॒ण्याꣳ श्चाव॑ । \newline
57. चावाव॑ च॒ चाव॑ रुन्धे रु॒न्धे ऽव॑ च॒ चाव॑ रुन्धे । \newline
58. अव॑ रुन्धे रु॒न्धे ऽवाव॑ रु॒न्धे ऽथो॒ अथो॑ रु॒न्धे ऽवाव॑ रु॒न्धे ऽथो᳚ । \newline
59. रु॒न्धे ऽथो॒ अथो॑ रुन्धे रु॒न्धे ऽथो॑ देवलो॒कम् दे॑वलो॒क मथो॑ रुन्धे रु॒न्धे ऽथो॑ देवलो॒कम् । \newline
60. अथो॑ देवलो॒कम् दे॑वलो॒क मथो॒ अथो॑ देवलो॒कम् च॑ च देवलो॒क मथो॒ अथो॑ देवलो॒कम् च॑ । \newline
61. अथो॒ इत्यथो᳚ । \newline
\pagebreak
\markright{ TS 2.5.11.3  \hfill https://www.vedavms.in \hfill}
\addcontentsline{toc}{section}{ TS 2.5.11.3 }
\section*{ TS 2.5.11.3 }

\textbf{TS 2.5.11.3 } \newline
\textbf{Samhita Paata} \newline

देवलो॒कं चै॒व म॑नुष्य लो॒कं चा॒भि ज॑यति दे॒वा वै सा॑मिधे॒नीर॒नूच्य॑ य॒ज्ञ्ं नान्व॑पश्य॒न्थ्स प्र॒जाप॑तिस्तू॒ष्णी-मा॑घा॒रमा ऽघा॑रय॒त् ततो॒ वै दे॒वा य॒ज्ञ्मन्व॑पश्य॒न्॒. यत् तू॒ष्णीमा॑घा॒र-मा॑घा॒रय॑ति य॒ज्ञ्स्यानु॑ख्यात्या॒ अथो॑ सामिधे॒नीरे॒वाभ्य॑-न॒क्त्यलू᳚क्षो भवति॒ य ए॒वं ॅवेदाथो॑ त॒र्पय॑त्ये॒वैना॒-स्तृप्य॑ति प्र॒जया॑ प॒शुभि॒ - [  ] \newline

\textbf{Pada Paata} \newline

दे॒व॒लो॒कमिति॑ देव - लो॒कम् । च॒ । ए॒व । म॒नु॒ष्य॒लो॒कमिति॑ मनुष्य - लो॒कम् । च॒ । अ॒भीति॑ । ज॒य॒ति॒ । दे॒वाः । वै । सा॒मि॒धे॒नीरिति॑ सां - इ॒धे॒नीः । अ॒नूच्येत्य॑नु - उच्य॑ । य॒ज्ञ्म् । न । अन्विति॑ । अ॒प॒श्य॒न्न् । सः । प्र॒जाप॑ति॒रिति॑ प्र॒जा - प॒तिः॒ । तू॒ष्णीम् । आ॒घा॒रमित्या᳚-घा॒रम् । एति॑ । अ॒घा॒रय॒त् । ततः॑ । वै । दे॒वाः । य॒ज्ञ्म् । अन्विति॑ । अ॒प॒श्य॒न्न् । यत् । तू॒ष्णीम् । आ॒घा॒रमित्या᳚ - घा॒रम् । आ॒घा॒रय॒तीत्या᳚ - घा॒रय॑ति । य॒ज्ञ्स्य॑ । अनु॑ख्यात्या॒ इत्यनु॑ - ख्या॒त्यै॒ । अथो॒ इति॑ । सा॒मि॒धे॒नीरिति॑ सां - इ॒धे॒नीः । ए॒व । अ॒भीति॑ । अ॒न॒क्ति॒ । अलू᳚क्षः । भ॒व॒ति॒ । यः । ए॒वम् । वेद॑ । अथो॒ इति॑ । त॒र्पय॑ति । ए॒व । ए॒नाः॒ । तृप्य॑ति । प्र॒जयेति॑ प्र - जया᳚ । प॒शुभि॒रिति॑ प॒शु - भिः॒ ।  \newline


\textbf{Krama Paata} \newline

दे॒व॒लो॒कम् च॑ । दे॒व॒लो॒कमिति॑ देव - लो॒कम् । चै॒व । ए॒व म॑नुष्यलो॒कम् । म॒नु॒ष्य॒लो॒कम् च॑ । म॒नु॒ष्य॒लो॒कमिति॑ मनुष्य - लो॒कम् । चा॒भि । अ॒भि ज॑यति । ज॒य॒ति॒ दे॒वाः । दे॒वा वै । वै सा॑मिधे॒नीः । सा॒मि॒धे॒नीर॒नूच्य॑ । सा॒मि॒धे॒नीरिति॑ साम् - इ॒धे॒नीः । अ॒नूच्य॑ य॒ज्ञ्म् । अ॒नूच्येत्य॑नु - उच्य॑ । य॒ज्ञ्म् न । नानु॑ । अन्व॑पश्यन्न् । अ॒प॒श्य॒न्थ् सः । स प्र॒जाप॑तिः । प्र॒जाप॑ति स्तू॒ष्णीम् । प्र॒जाप॑ति॒रिति॑ प्र॒जा - प॒तिः॒ । तू॒ष्णीमा॑घा॒रम् । आ॒घा॒रमा । आ॒घा॒रमित्या᳚ - घा॒रम् । आ ऽघा॑रयत् । अ॒घा॒र॒य॒त् ततः॑ । ततो॒ वै । वै दे॒वाः । दे॒वा य॒ज्ञ्म् । य॒ज्ञ्मनु॑ । अन्व॑पश्यन्न् । अ॒प॒श्य॒न्॒. यत् । यत् तू॒ष्णीम् । तू॒ष्णीमा॑घा॒रम् । आ॒घा॒रमा॑घा॒रय॑ति । आ॒घा॒रमित्या᳚ - घा॒रम् । आ॒घा॒रय॑ति य॒ज्ञ्स्य॑ । आ॒घा॒रय॒तीत्या᳚ - घा॒रय॑ति । य॒ज्ञ्स्यानु॑ख्यात्यै । अनु॑ख्यात्या॒ अथो᳚ । अनु॑ख्यात्या॒ इत्यनु॑ - ख्या॒त्यै॒ । अथो॑ सामिधे॒नीः । अथो॒ इत्यथो᳚ । सा॒मि॒धे॒नीरे॒व । सा॒मि॒धे॒नीरिति॑ साम् - इ॒धे॒नीः । ए॒वाभि । अ॒भ्य॑नक्ति । अ॒न॒क्त्यलू᳚क्षः । अलू᳚क्षो भवति । भ॒व॒ति॒ यः । य ए॒वम् । ए॒वम् ॅवेद॑ । वेदाथो᳚ । अथो॑ त॒र्पय॑ति । अथो॒ इत्यथो᳚ । त॒र्पय॑त्ये॒व । ए॒वैनाः᳚ । ए॒ना॒स्तृप्य॑ति । तृप्य॑ति प्र॒जया᳚ । प्र॒जया॑ प॒शुभिः॑ । प्र॒जयेति॑ प्र - जया᳚ । प॒शुभि॒र् यः । प॒शुभि॒रिति॑ प॒शु - भिः॒ \newline

\textbf{Jatai Paata} \newline

1. दे॒व॒लो॒कम् च॑ च देवलो॒कम् दे॑वलो॒कम् च॑ । \newline
2. दे॒व॒लो॒कमिति॑ देव - लो॒कम् । \newline
3. चै॒वैव च॑ चै॒व । \newline
4. ए॒व म॑नुष्यलो॒कम् म॑नुष्यलो॒क मे॒वैव म॑नुष्यलो॒कम् । \newline
5. म॒नु॒ष्य॒लो॒कम् च॑ च मनुष्यलो॒कम् म॑नुष्यलो॒कम् च॑ । \newline
6. म॒नु॒ष्य॒लो॒कमिति॑ मनुष्य - लो॒कम् । \newline
7. चा॒भ्य॑भि च॑ चा॒भि । \newline
8. अ॒भि ज॑यति जय त्य॒भ्य॑भि ज॑यति । \newline
9. ज॒य॒ति॒ दे॒वा दे॒वा ज॑यति जयति दे॒वाः । \newline
10. दे॒वा वै वै दे॒वा दे॒वा वै । \newline
11. वै सा॑मिधे॒नीः सा॑मिधे॒नीर् वै वै सा॑मिधे॒नीः । \newline
12. सा॒मि॒धे॒नी र॒नूच्या॒ नूच्य॑ सामिधे॒नीः सा॑मिधे॒नी र॒नूच्य॑ । \newline
13. सा॒मि॒धे॒नीरिति॑ सां - इ॒धे॒नीः । \newline
14. अ॒नूच्य॑ य॒ज्ञ्ं ॅय॒ज्ञ् म॒नूच्या॒ नूच्य॑ य॒ज्ञ्म् । \newline
15. अ॒नूच्येत्य॑नु - उच्य॑ । \newline
16. य॒ज्ञ्म् न न य॒ज्ञ्ं ॅय॒ज्ञ्म् न । \newline
17. नान्वनु॒ न नानु॑ । \newline
18. अन्व॑पश्यन् नपश्य॒न् नन्वन्व॑पश्यन्न् । \newline
19. अ॒प॒श्य॒न् थ्स सो॑ ऽपश्यन् नपश्य॒न् थ्सः । \newline
20. स प्र॒जाप॑तिः प्र॒जाप॑तिः॒ स स प्र॒जाप॑तिः । \newline
21. प्र॒जाप॑ति स्तू॒ष्णीम् तू॒ष्णीम् प्र॒जाप॑तिः प्र॒जाप॑ति स्तू॒ष्णीम् । \newline
22. प्र॒जाप॑ति॒रिति॑ प्र॒जा - प॒तिः॒ । \newline
23. तू॒ष्णी मा॑घा॒र मा॑घा॒रम् तू॒ष्णीम् तू॒ष्णी मा॑घा॒रम् । \newline
24. आ॒घा॒र मा ऽऽघा॒र मा॑घा॒र मा । \newline
25. आ॒घा॒रमित्या᳚ - घा॒रम् । \newline
26. आ ऽघा॑रय दघारय॒दा ऽघा॑रयत् । \newline
27. अ॒घा॒र॒य॒त् तत॒ स्ततो॑ ऽघारय दघारय॒त् ततः॑ । \newline
28. ततो॒ वै वै तत॒ स्ततो॒ वै । \newline
29. वै दे॒वा दे॒वा वै वै दे॒वाः । \newline
30. दे॒वा य॒ज्ञ्ं ॅय॒ज्ञ्म् दे॒वा दे॒वा य॒ज्ञ्म् । \newline
31. य॒ज्ञ् मन्वनु॑ य॒ज्ञ्ं ॅय॒ज्ञ् मनु॑ । \newline
32. अन्व॑पश्यन् नपश्य॒न् नन्वन्व॑पश्यन्न् । \newline
33. अ॒प॒श्य॒न्॒. यद् यद॑पश्यन् नपश्य॒न्॒. यत् । \newline
34. यत् तू॒ष्णीम् तू॒ष्णीं ॅयद् यत् तू॒ष्णीम् । \newline
35. तू॒ष्णी मा॑घा॒र मा॑घा॒रम् तू॒ष्णीम् तू॒ष्णी मा॑घा॒रम् । \newline
36. आ॒घा॒र मा॑घा॒रय॑ त्याघा॒रय॑ त्याघा॒र मा॑घा॒र मा॑घा॒रय॑ति । \newline
37. आ॒घा॒रमित्या᳚ - घा॒रम् । \newline
38. आ॒घा॒रय॑ति य॒ज्ञ्स्य॑ य॒ज्ञ्स्या॑ घा॒रय॑ त्याघा॒रय॑ति य॒ज्ञ्स्य॑ । \newline
39. आ॒घा॒रय॒तीत्या᳚ - घा॒रय॑ति । \newline
40. य॒ज्ञ्स्यानु॑ख्यात्या॒ अनु॑ख्यात्यै य॒ज्ञ्स्य॑ य॒ज्ञ्स्यानु॑ख्यात्यै । \newline
41. अनु॑ख्यात्या॒ अथो॒ अथो॒ अनु॑ख्यात्या॒ अनु॑ख्यात्या॒ अथो᳚ । \newline
42. अनु॑ख्यात्या॒ इत्यनु॑ - ख्या॒त्यै॒ । \newline
43. अथो॑ सामिधे॒नीः सा॑मिधे॒नी रथो॒ अथो॑ सामिधे॒नीः । \newline
44. अथो॒ इत्यथो᳚ । \newline
45. सा॒मि॒धे॒नी रे॒वैव सा॑मिधे॒नीः सा॑मिधे॒नी रे॒व । \newline
46. सा॒मि॒धे॒नीरिति॑ सां - इ॒धे॒नीः । \newline
47. ए॒वाभ्या᳚(1॒)भ्ये॑वैवाभि । \newline
48. अ॒भ्य॑नक् त्यनक् त्य॒भ्या᳚(1॒)भ्य॑नक्ति । \newline
49. अ॒न॒क् त्यलू॒क्षो ऽलू᳚क्षो ऽनक् त्यन॒क् त्यलू᳚क्षः । \newline
50. अलू᳚क्षो भवति भव॒त्यलू॒क्षो ऽलू᳚क्षो भवति । \newline
51. भ॒व॒ति॒ यो यो भ॑वति भवति॒ यः । \newline
52. य ए॒व मे॒वं ॅयो य ए॒वम् । \newline
53. ए॒वं ॅवेद॒ वेदै॒व मे॒वं ॅवेद॑ । \newline
54. वेदाथो॒ अथो॒ वेद॒ वेदाथो᳚ । \newline
55. अथो॑ त॒र्पय॑ति त॒र्पय॒ त्यथो॒ अथो॑ त॒र्पय॑ति । \newline
56. अथो॒ इत्यथो᳚ । \newline
57. त॒र्पय॑ त्ये॒वैव त॒र्पय॑ति त॒र्पय॑ त्ये॒व । \newline
58. ए॒वैना॑ एना ए॒वैवैनाः᳚ । \newline
59. ए॒ना॒ स्तृप्य॑ति॒ तृप्य॑त्येना एना॒ स्तृप्य॑ति । \newline
60. तृप्य॑ति प्र॒जया᳚ प्र॒जया॒ तृप्य॑ति॒ तृप्य॑ति प्र॒जया᳚ । \newline
61. प्र॒जया॑ प॒शुभिः॑ प॒शुभिः॑ प्र॒जया᳚ प्र॒जया॑ प॒शुभिः॑ । \newline
62. प्र॒जयेति॑ प्र - जया᳚ । \newline
63. प॒शुभि॒र् यो यः प॒शुभिः॑ प॒शुभि॒र् यः । \newline
64. प॒शुभि॒रिति॑ प॒शु - भिः॒ । \newline

\textbf{Ghana Paata } \newline

1. दे॒व॒लो॒कम् च॑ च देवलो॒कम् दे॑वलो॒कम् चै॒वैव च॑ देवलो॒कम् दे॑वलो॒कम् चै॒व । \newline
2. दे॒व॒लो॒कमिति॑ देव - लो॒कम् । \newline
3. चै॒वैव च॑ चै॒व म॑नुष्यलो॒कम् म॑नुष्यलो॒क मे॒व च॑ चै॒व म॑नुष्यलो॒कम् । \newline
4. ए॒व म॑नुष्यलो॒कम् म॑नुष्यलो॒क मे॒वैव म॑नुष्यलो॒कम् च॑ च मनुष्यलो॒क मे॒वैव म॑नुष्यलो॒कम् च॑ । \newline
5. म॒नु॒ष्य॒लो॒कम् च॑ च मनुष्यलो॒कम् म॑नुष्यलो॒कम् चा॒भ्य॑भि च॑ मनुष्यलो॒कम् म॑नुष्यलो॒कम् चा॒भि । \newline
6. म॒नु॒ष्य॒लो॒कमिति॑ मनुष्य - लो॒कम् । \newline
7. चा॒भ्य॑भि च॑ चा॒भि ज॑यति जयत्य॒भि च॑ चा॒भि ज॑यति । \newline
8. अ॒भि ज॑यति जय त्य॒भ्य॑भि ज॑यति दे॒वा दे॒वा ज॑य त्य॒भ्य॑भि ज॑यति दे॒वाः । \newline
9. ज॒य॒ति॒ दे॒वा दे॒वा ज॑यति जयति दे॒वा वै वै दे॒वा ज॑यति जयति दे॒वा वै । \newline
10. दे॒वा वै वै दे॒वा दे॒वा वै सा॑मिधे॒नीः सा॑मिधे॒नीर् वै दे॒वा दे॒वा वै सा॑मिधे॒नीः । \newline
11. वै सा॑मिधे॒नीः सा॑मिधे॒नीर् वै वै सा॑मिधे॒नी र॒नूच्या॒ नूच्य॑ सामिधे॒नीर् वै वै सा॑मिधे॒नी र॒नूच्य॑ । \newline
12. सा॒मि॒धे॒नी र॒नूच्या॒ नूच्य॑ सामिधे॒नीः सा॑मिधे॒नी र॒नूच्य॑ य॒ज्ञ्ं ॅय॒ज्ञ् म॒नूच्य॑ सामिधे॒नीः सा॑मिधे॒नी र॒नूच्य॑ य॒ज्ञ्म् । \newline
13. सा॒मि॒धे॒नीरिति॑ सां - इ॒धे॒नीः । \newline
14. अ॒नूच्य॑ य॒ज्ञ्ं ॅय॒ज्ञ् म॒नूच्या॒ नूच्य॑ य॒ज्ञ्म् न न य॒ज्ञ् म॒नूच्या॒ नूच्य॑ य॒ज्ञ्म् न । \newline
15. अ॒नूच्येत्य॑नु - उच्य॑ । \newline
16. य॒ज्ञ्म् न न य॒ज्ञ्ं ॅय॒ज्ञ्म् नान्वनु॒ न य॒ज्ञ्ं ॅय॒ज्ञ्म् नानु॑ । \newline
17. नान्वनु॒ न नान्व॑पश्यन् नपश्य॒न् ननु॒ न नान्व॑पश्यन्न् । \newline
18. अन्व॑पश्यन् नपश्य॒न् नन्वन्व॑ पश्य॒न् थ्स सो॑ ऽपश्य॒न् नन्वन्व॑ पश्य॒न् थ्सः । \newline
19. अ॒प॒श्य॒न् थ्स सो॑ ऽपश्यन् नपश्य॒न् थ्स प्र॒जाप॑तिः प्र॒जाप॑तिः॒ सो॑ ऽपश्यन् नपश्य॒न् थ्स प्र॒जाप॑तिः । \newline
20. स प्र॒जाप॑तिः प्र॒जाप॑तिः॒ स स प्र॒जाप॑ति स्तू॒ष्णीम् तू॒ष्णीम् प्र॒जाप॑तिः॒ स स प्र॒जाप॑ति स्तू॒ष्णीम् । \newline
21. प्र॒जाप॑ति स्तू॒ष्णीम् तू॒ष्णीम् प्र॒जाप॑तिः प्र॒जाप॑ति स्तू॒ष्णी मा॑घा॒र मा॑घा॒रम् तू॒ष्णीम् प्र॒जाप॑तिः प्र॒जाप॑ति स्तू॒ष्णी मा॑घा॒रम् । \newline
22. प्र॒जाप॑ति॒रिति॑ प्र॒जा - प॒तिः॒ । \newline
23. तू॒ष्णी मा॑घा॒र मा॑घा॒रम् तू॒ष्णीम् तू॒ष्णी मा॑घा॒र मा ऽऽघा॒रम् तू॒ष्णीम् तू॒ष्णी मा॑घा॒र मा । \newline
24. आ॒घा॒र मा ऽऽघा॒र मा॑घा॒र मा ऽघा॑रय दघारय॒ दा ऽऽघा॒र मा॑घा॒र मा ऽघा॑रयत् । \newline
25. आ॒घा॒रमित्या᳚ - घा॒रम् । \newline
26. आ ऽघा॑रय दघारय॒ दा ऽघा॑रय॒त् तत॒ स्ततो॑ ऽघारय॒ दा ऽघा॑रय॒त् ततः॑ । \newline
27. अ॒घा॒र॒य॒त् तत॒ स्ततो॑ ऽघारय दघारय॒त् ततो॒ वै वै ततो॑ ऽघारय दघारय॒त् ततो॒ वै । \newline
28. ततो॒ वै वै तत॒ स्ततो॒ वै दे॒वा दे॒वा वै तत॒ स्ततो॒ वै दे॒वाः । \newline
29. वै दे॒वा दे॒वा वै वै दे॒वा य॒ज्ञ्ं ॅय॒ज्ञ्म् दे॒वा वै वै दे॒वा य॒ज्ञ्म् । \newline
30. दे॒वा य॒ज्ञ्ं ॅय॒ज्ञ्म् दे॒वा दे॒वा य॒ज्ञ् मन्वनु॑ य॒ज्ञ्म् दे॒वा दे॒वा य॒ज्ञ् मनु॑ । \newline
31. य॒ज्ञ् मन्वनु॑ य॒ज्ञ्ं ॅय॒ज्ञ् मन्व॑पश्यन् नपश्य॒न् ननु॑ य॒ज्ञ्ं ॅय॒ज्ञ् मन्व॑पश्यन्न् । \newline
32. अन्व॑पश्यन् नपश्य॒न् नन्वन्व॑पश्य॒न्॒. यद् यद॑पश्य॒न् नन्वन्व॑पश्य॒न्॒. यत् । \newline
33. अ॒प॒श्य॒न्॒. यद् यद॑पश्यन् नपश्य॒न्॒. यत् तू॒ष्णीम् तू॒ष्णीं ॅयद॑पश्यन् नपश्य॒न्॒. यत् तू॒ष्णीम् । \newline
34. यत् तू॒ष्णीम् तू॒ष्णीं ॅयद् यत् तू॒ष्णी मा॑घा॒र मा॑घा॒रम् तू॒ष्णीं ॅयद् यत् तू॒ष्णी मा॑घा॒रम् । \newline
35. तू॒ष्णी मा॑घा॒र मा॑घा॒रम् तू॒ष्णीम् तू॒ष्णी मा॑घा॒र मा॑घा॒रय॑ त्याघा॒रय॑ त्याघा॒रम् तू॒ष्णीम् तू॒ष्णी मा॑घा॒र मा॑घा॒रय॑ति । \newline
36. आ॒घा॒र मा॑घा॒रय॑ त्याघा॒रय॑ त्याघा॒र मा॑घा॒र मा॑घा॒रय॑ति य॒ज्ञ्स्य॑ य॒ज्ञ्स्या॑ घा॒रय॑ त्याघा॒र मा॑घा॒र मा॑घा॒रय॑ति य॒ज्ञ्स्य॑ । \newline
37. आ॒घा॒रमित्या᳚ - घा॒रम् । \newline
38. आ॒घा॒रय॑ति य॒ज्ञ्स्य॑ य॒ज्ञ्स्या॑ घा॒रय॑ त्याघा॒रय॑ति य॒ज्ञ्स्या नु॑ख्यात्या॒ अनु॑ख्यात्यै य॒ज्ञ्स्या॑ घा॒रय॑ त्याघा॒रय॑ति य॒ज्ञ्स्या नु॑ख्यात्यै । \newline
39. आ॒घा॒रय॒तीत्या᳚ - घा॒रय॑ति । \newline
40. य॒ज्ञ्स्या नु॑ख्यात्या॒ अनु॑ख्यात्यै य॒ज्ञ्स्य॑ य॒ज्ञ्स्या नु॑ख्यात्या॒ अथो॒ अथो॒ अनु॑ख्यात्यै य॒ज्ञ्स्य॑ य॒ज्ञ्स्या नु॑ख्यात्या॒ अथो᳚ । \newline
41. अनु॑ख्यात्या॒ अथो॒ अथो॒ अनु॑ख्यात्या॒ अनु॑ख्यात्या॒ अथो॑ सामिधे॒नीः सा॑मिधे॒नी रथो॒ अनु॑ख्यात्या॒ अनु॑ख्यात्या॒ अथो॑ सामिधे॒नीः । \newline
42. अनु॑ख्यात्या॒ इत्यनु॑ - ख्या॒त्यै॒ । \newline
43. अथो॑ सामिधे॒नीः सा॑मिधे॒नी रथो॒ अथो॑ सामिधे॒नी रे॒वैव सा॑मिधे॒नी रथो॒ अथो॑ सामिधे॒नी रे॒व । \newline
44. अथो॒ इत्यथो᳚ । \newline
45. सा॒मि॒धे॒नी रे॒वैव सा॑मिधे॒नीः सा॑मिधे॒नी रे॒वाभ्या᳚(1॒)भ्ये॑व सा॑मिधे॒नीः सा॑मिधे॒नी रे॒वाभि । \newline
46. सा॒मि॒धे॒नीरिति॑ सां - इ॒धे॒नीः । \newline
47. ए॒वाभ्या᳚(1॒)भ्ये॑वैवा भ्य॑नक्त्य नक्त्य॒ भ्ये॑वैवा भ्य॑नक्ति । \newline
48. अ॒भ्य॑नक्त्य नक्त्य॒भ्या᳚(1॒)भ्य॑ न॒क्त्य लू॒क्षो ऽलू᳚क्षो ऽनक्त्य॒भ्या᳚(1॒)भ्य॑न॒क्त्य लू᳚क्षः । \newline
49. अ॒न॒क्त्य लू॒क्षो ऽलू᳚क्षो ऽनक्त्य न॒क्त्य लू᳚क्षो भवति भव॒ त्यलू᳚क्षो ऽनक्त्य न॒क्त्य लू᳚क्षो भवति । \newline
50. अलू᳚क्षो भवति भव॒ त्यलू॒क्षो ऽलू᳚क्षो भवति॒ यो यो भ॑व॒ त्यलू॒क्षो ऽलू᳚क्षो भवति॒ यः । \newline
51. भ॒व॒ति॒ यो यो भ॑वति भवति॒ य ए॒व मे॒वं ॅयो भ॑वति भवति॒ य ए॒वम् । \newline
52. य ए॒व मे॒वं ॅयो य ए॒वं ॅवेद॒ वेदै॒वं ॅयो य ए॒वं ॅवेद॑ । \newline
53. ए॒वं ॅवेद॒ वेदै॒व मे॒वं ॅवेदाथो॒ अथो॒ वेदै॒व मे॒वं ॅवेदाथो᳚ । \newline
54. वेदाथो॒ अथो॒ वेद॒ वेदाथो॑ त॒र्पय॑ति त॒र्पय॒ त्यथो॒ वेद॒ वेदाथो॑ त॒र्पय॑ति । \newline
55. अथो॑ त॒र्पय॑ति त॒र्पय॒ त्यथो॒ अथो॑ त॒र्पय॑ त्ये॒वैव त॒र्पय॒ त्यथो॒ अथो॑ त॒र्पय॑ त्ये॒व । \newline
56. अथो॒ इत्यथो᳚ । \newline
57. त॒र्पय॑ त्ये॒वैव त॒र्पय॑ति त॒र्पय॑ त्ये॒वैना॑ एना ए॒व त॒र्पय॑ति त॒र्पय॑ त्ये॒वैनाः᳚ । \newline
58. ए॒वैना॑ एना ए॒वैवैना॒ स्तृप्य॑ति॒ तृप्य॑त्येना ए॒वैवैना॒ स्तृप्य॑ति । \newline
59. ए॒ना॒ स्तृप्य॑ति॒ तृप्य॑त्येना एना॒ स्तृप्य॑ति प्र॒जया᳚ प्र॒जया॒ तृप्य॑त्येना एना॒ स्तृप्य॑ति प्र॒जया᳚ । \newline
60. तृप्य॑ति प्र॒जया᳚ प्र॒जया॒ तृप्य॑ति॒ तृप्य॑ति प्र॒जया॑ प॒शुभिः॑ प॒शुभिः॑ प्र॒जया॒ तृप्य॑ति॒ तृप्य॑ति प्र॒जया॑ प॒शुभिः॑ । \newline
61. प्र॒जया॑ प॒शुभिः॑ प॒शुभिः॑ प्र॒जया᳚ प्र॒जया॑ प॒शुभि॒र् यो यः प॒शुभिः॑ प्र॒जया᳚ प्र॒जया॑ प॒शुभि॒र् यः । \newline
62. प्र॒जयेति॑ प्र - जया᳚ । \newline
63. प॒शुभि॒र् यो यः प॒शुभिः॑ प॒शुभि॒र् य ए॒व मे॒वं ॅयः प॒शुभिः॑ प॒शुभि॒र् य ए॒वम् । \newline
64. प॒शुभि॒रिति॑ प॒शु - भिः॒ । \newline
\pagebreak
\markright{ TS 2.5.11.4  \hfill https://www.vedavms.in \hfill}
\addcontentsline{toc}{section}{ TS 2.5.11.4 }
\section*{ TS 2.5.11.4 }

\textbf{TS 2.5.11.4 } \newline
\textbf{Samhita Paata} \newline

-र्य ए॒वं ॅवेद॒ यदेक॑या ऽऽघा॒रये॒देकां᳚ प्रीणीया॒द्यद् द्वाभ्यां॒ द्वे प्री॑णीया॒द्यद् ति॒सृभि॒रति॒ तद्रे॑चये॒त्मन॒सा ऽऽघा॑रयति॒ मन॑सा॒ ह्यना᳚प्तमा॒प्यते॑ ति॒र्यञ्च॒मा घा॑रय॒त्यछ॑म्बट्कारं॒ ॅवाक् च॒ मन॑श्चा ऽऽर्तीयेताम॒हं दे॒वेभ्यो॑ ह॒व्यं ॅव॑हा॒मीति॒ वाग॑ब्रवीद॒हं दे॒वेभ्य॒ इति॒ मन॒स्तौ प्र॒जाप॑तिं प्र॒श्नमै॑ताꣳ॒॒ सो᳚ऽब्रवीत् - [  ] \newline

\textbf{Pada Paata} \newline

यः । ए॒वम् । वेद॑ । यत् । एक॑या । आ॒घा॒रये॒दित्या᳚ - घा॒रये᳚त् । एका᳚म् । प्री॒णी॒या॒त् । यत् । द्वाभ्या᳚म् । द्वे इति॑ । प्री॒णी॒या॒त् । यत् । ति॒सृभि॒रिति॑ ति॒सृ - भिः॒ । अतीति॑ । तत् । रे॒च॒ये॒त् । मन॑सा । एति॑ । घा॒र॒य॒ति॒ । मन॑सा । हि । अना᳚प्तम् । आ॒प्यते᳚ । ति॒र्यञ्च᳚म् । एति॑ । घा॒र॒य॒ति॒ । अछ॑बंट्कार॒मित्यछ॑बंट् - का॒र॒म् । वाक् । च॒ । मनः॑ । च॒ । आ॒र्ती॒ये॒ता॒म् । अ॒हम् । दे॒वेभ्यः॑ । ह॒व्यम् । व॒हा॒मि॒ । इति॑ । वाक् । अ॒ब्र॒वी॒त् । अ॒हम् । दे॒वेभ्यः॑ । इति॑ । मनः॑ । तौ । प्र॒जाप॑ति॒मिति॑ प्र॒जा - प॒ति॒म् । प्र॒श्नम् । ऐ॒ता॒म् । सः । अ॒ब्र॒वी॒त् ।  \newline


\textbf{Krama Paata} \newline

य ए॒वम् । ए॒वम् ॅवेद॑ । वेद॒ यत् । यदेक॑या । एक॑या ऽऽघा॒रये᳚त् । आ॒घा॒रये॒देका᳚म् । आ॒घा॒रये॒दित्या᳚ - घा॒रये᳚त् । एका᳚म् प्रीणीयात् । प्री॒णी॒या॒द् यत् । यद् द्वाभ्या᳚म् । द्वाभ्या॒म् द्वे । द्वे प्री॑णीयात् । द्वे इति॒ द्वे । प्री॒णी॒या॒द् यत् । यत् ति॒सृभिः॑ । ति॒सृभि॒रति॑ । ति॒सृभि॒रिति॑ ति॒सृ - भिः॒ । अति॒ तत् । तद् रे॑चयेत् । रे॒च॒ये॒न् मन॑सा । मन॒सा । आ घा॑रयति । घा॒र॒य॒ति॒ मन॑सा । मन॑सा॒ हि । ह्यना᳚प्तम् । अना᳚प्तमा॒प्यते᳚ । आ॒प्यते॑ ति॒र्यञ्च᳚म् । ति॒र्यञ्च॒मा । आ घा॑रयति । घा॒र॒य॒त्यछ॑म्बट्कारम् । अछ॑म्बट्कार॒म् ॅवाक् । अछ॑म्बट्कार॒मित्यछ॑म्बट् - का॒र॒म् । वाक् च॑ । च॒ मनः॑ । मन॑श्च । चा॒र्ती॒ये॒ता॒म् । आ॒र्ती॒ये॒ता॒म॒हम् । अ॒हम् दे॒वेभ्यः॑ । दे॒वेभ्यो॑ ह॒व्यम् । ह॒व्यम् ॅव॑हामि । व॒हा॒मीति॑ । इति॒ वाक् । वाग॑ब्रवीत् । अ॒ब्र॒वी॒द॒हम् । अ॒हम् दे॒वेभ्यः॑ । दे॒वेभ्य॒ इति॑ । इति॒ मनः॑ । मन॒ स्तौ । तौ प्र॒जाप॑तिम् । प्र॒जाप॑तिम् प्र॒श्ञम् । प्र॒जाप॑ति॒मिति॑ प्र॒जा - प॒ति॒म् । प्र॒श्ञमै॑ताम् । ऐ॒ताꣳ॒॒ सः । सो᳚ऽब्रवीत् । अ॒ब्र॒वी॒त् प्र॒जाप॑तिः \newline

\textbf{Jatai Paata} \newline

1. य ए॒व मे॒वं ॅयो य ए॒वम् । \newline
2. ए॒वं ॅवेद॒ वेदै॒व मे॒वं ॅवेद॑ । \newline
3. वेद॒ यद् यद् वेद॒ वेद॒ यत् । \newline
4. यदेक॒ यैक॑या॒ यद् यदेक॑या । \newline
5. एक॑या ऽऽघा॒रये॑ दाघा॒रये॒ देक॒यैक॑या ऽऽघा॒रये᳚त् । \newline
6. आ॒घा॒रये॒ देका॒ मेका॑ माघा॒रये॑ दाघा॒रये॒ देका᳚म् । \newline
7. आ॒घा॒रये॒दित्या᳚ - घा॒रये᳚त् । \newline
8. एका᳚म् प्रीणीयात् प्रीणीया॒देका॒ मेका᳚म् प्रीणीयात् । \newline
9. प्री॒णी॒या॒द् यद् यत् प्री॑णीयात् प्रीणीया॒द् यत् । \newline
10. यद् द्वाभ्या॒म् द्वाभ्यां॒ ॅयद् यद् द्वाभ्या᳚म् । \newline
11. द्वाभ्या॒म् द्वे द्वे द्वाभ्या॒म् द्वाभ्या॒म् द्वे । \newline
12. द्वे प्री॑णीयात् प्रीणीया॒द् द्वे द्वे प्री॑णीयात् । \newline
13. द्वे इति॒ द्वे । \newline
14. प्री॒णी॒या॒द् यद् यत् प्री॑णीयात् प्रीणीया॒द् यत् । \newline
15. यत् ति॒सृभि॑ स्ति॒सृभि॒र् यद् यत् ति॒सृभिः॑ । \newline
16. ति॒सृभि॒ रत्यति॑ ति॒सृभि॑ स्ति॒सृभि॒ रति॑ । \newline
17. ति॒सृभि॒रिति॑ ति॒सृ - भिः॒ । \newline
18. अति॒ तत् तदत्यति॒ तत् । \newline
19. तद् रे॑चयेद् रेचये॒त् तत् तद् रे॑चयेत् । \newline
20. रे॒च॒ये॒न् मन॑सा॒ मन॑सा रेचयेद् रेचये॒न् मन॑सा । \newline
21. मन॒सा ऽऽमन॑सा॒ मन॑सा । \newline
22. आ घा॑रयति घारय॒त्या घा॑रयति । \newline
23. घा॒र॒य॒ति॒ मन॑सा॒ मन॑सा घारयति घारयति॒ मन॑सा । \newline
24. मन॑सा॒ हि हि मन॑सा॒ मन॑सा॒ हि । \newline
25. ह्यना᳚प्त॒ मना᳚प्तꣳ॒॒ हि ह्यना᳚प्तम् । \newline
26. अना᳚प्त मा॒प्यत॑ आ॒प्यते ऽना᳚प्त॒ मना᳚प्त मा॒प्यते᳚ । \newline
27. आ॒प्यते॑ ति॒र्यञ्च॑म् ति॒र्यञ्च॑ मा॒प्यत॑ आ॒प्यते॑ ति॒र्यञ्च᳚म् । \newline
28. ति॒र्यञ्च॒ मा ति॒र्यञ्च॑म् ति॒र्यञ्च॒ मा । \newline
29. आ घा॑रयति घारय॒त्या घा॑रयति । \newline
30. घा॒र॒य॒ त्यछं॑बट्कार॒ मछं॑बट्कारम् घारयति घारय॒ त्यछं॑बट्कारम् । \newline
31. अछं॑बट्कारं॒ ॅवाग् वागछं॑बट्कार॒ मछं॑बट्कारं॒ ॅवाक् । \newline
32. अछं॑बट्कार॒मित्यछं॑बट् - का॒र॒म् । \newline
33. वाक् च॑ च॒ वाग् वाक् च॑ । \newline
34. च॒ मनो॒ मन॑श्च च॒ मनः॑ । \newline
35. मन॑श्च च॒ मनो॒ मन॑श्च । \newline
36. चा॒र्ती॒ये॒ता॒ मा॒र्ती॒ये॒ता॒म् च॒ चा॒र्ती॒ये॒ता॒म् । \newline
37. आ॒र्ती॒ये॒ता॒ म॒ह म॒ह मा᳚र्तीयेता मार्तीयेता म॒हम् । \newline
38. अ॒हम् दे॒वेभ्यो॑ दे॒वेभ्यो॒ ऽह म॒हम् दे॒वेभ्यः॑ । \newline
39. दे॒वेभ्यो॑ ह॒व्यꣳ ह॒व्यम् दे॒वेभ्यो॑ दे॒वेभ्यो॑ ह॒व्यम् । \newline
40. ह॒व्यं ॅव॑हामि वहामि ह॒व्यꣳ ह॒व्यं ॅव॑हामि । \newline
41. व॒हा॒मीतीति॑ वहामि वहा॒मीति॑ । \newline
42. इति॒ वाग् वागितीति॒ वाक् । \newline
43. वाग॑ब्रवी दब्रवी॒द् वाग् वाग॑ब्रवीत् । \newline
44. अ॒ब्र॒वी॒द॒ह म॒ह म॑ब्रवी दब्रवी द॒हम् । \newline
45. अ॒हम् दे॒वेभ्यो॑ दे॒वेभ्यो॒ ऽह म॒हम् दे॒वेभ्यः॑ । \newline
46. दे॒वेभ्य॒ इतीति॑ दे॒वेभ्यो॑ दे॒वेभ्य॒ इति॑ । \newline
47. इति॒ मनो॒ मन॒ इतीति॒ मनः॑ । \newline
48. मन॒स्तौ तौ मनो॒ मन॒ स्तौ । \newline
49. तौ प्र॒जाप॑तिम् प्र॒जाप॑ति॒म् तौ तौ प्र॒जाप॑तिम् । \newline
50. प्र॒जाप॑तिम् प्र॒श्ञम् प्र॒श्ञम् प्र॒जाप॑तिम् प्र॒जाप॑तिम् प्र॒श्ञम् । \newline
51. प्र॒जाप॑ति॒मिति॑ प्र॒जा - प॒ति॒म् । \newline
52. प्र॒श्ञ मै॑ता मैताम् प्र॒श्ञम् प्र॒श्ञ मै॑ताम् । \newline
53. ऐ॒ताꣳ॒॒ स स ऐ॑ता मैताꣳ॒॒ सः । \newline
54. सो᳚ ऽब्रवी दब्रवी॒थ् स सो᳚ ऽब्रवीत् । \newline
55. अ॒ब्र॒वी॒त् प्र॒जाप॑तिः प्र॒जाप॑ति रब्रवी दब्रवीत् प्र॒जाप॑तिः । \newline

\textbf{Ghana Paata } \newline

1. य ए॒व मे॒वं ॅयो य ए॒वं ॅवेद॒ वेदै॒वं ॅयो य ए॒वं ॅवेद॑ । \newline
2. ए॒वं ॅवेद॒ वेदै॒व मे॒वं ॅवेद॒ यद् यद् वेदै॒व मे॒वं ॅवेद॒ यत् । \newline
3. वेद॒ यद् यद् वेद॒ वेद॒ यदेक॒ यैक॑या॒ यद् वेद॒ वेद॒ यदेक॑या । \newline
4. यदेक॒ यैक॑या॒ यद् यदेक॑या ऽऽघा॒रये॑ दाघा॒रये॒ देक॑या॒ यद् यदेक॑या ऽऽघा॒रये᳚त् । \newline
5. एक॑या ऽऽघा॒रये॑ दाघा॒रये॒ देक॒यैक॑या ऽऽघा॒रये॒देका॒ मेका॑ माघा॒रये॒ देक॒यैक॑या ऽऽघा॒रये॒ देका᳚म् । \newline
6. आ॒घा॒रये॒ देका॒ मेका॑ माघा॒रये॑ दाघा॒रये॒ देका᳚म् प्रीणीयात् प्रीणीया॒ देका॑ माघा॒रये॑ दाघा॒रये॒ देका᳚म् प्रीणीयात् । \newline
7. आ॒घा॒रये॒दित्या᳚ - घा॒रये᳚त् । \newline
8. एका᳚म् प्रीणीयात् प्रीणीया॒ देका॒ मेका᳚म् प्रीणीया॒द् यद् यत् प्री॑णीया॒ देका॒ मेका᳚म् प्रीणीया॒द् यत् । \newline
9. प्री॒णी॒या॒द् यद् यत् प्री॑णीयात् प्रीणीया॒द् यद् द्वाभ्या॒म् द्वाभ्यां॒ ॅयत् प्री॑णीयात् प्रीणीया॒द् यद् द्वाभ्या᳚म् । \newline
10. यद् द्वाभ्या॒म् द्वाभ्यां॒ ॅयद् यद् द्वाभ्या॒म् द्वे द्वे द्वाभ्यां॒ ॅयद् यद् द्वाभ्या॒म् द्वे । \newline
11. द्वाभ्या॒म् द्वे द्वे द्वाभ्या॒म् द्वाभ्या॒म् द्वे प्री॑णीयात् प्रीणीया॒द् द्वे द्वाभ्या॒म् द्वाभ्या॒म् द्वे प्री॑णीयात् । \newline
12. द्वे प्री॑णीयात् प्रीणीया॒द् द्वे द्वे प्री॑णीया॒द् यद् यत् प्री॑णीया॒द् द्वे द्वे प्री॑णीया॒द् यत् । \newline
13. द्वे इति॒ द्वे । \newline
14. प्री॒णी॒या॒द् यद् यत् प्री॑णीयात् प्रीणीया॒द् यत् ति॒सृभि॑ स्ति॒सृभि॒र् यत् प्री॑णीयात् प्रीणीया॒द् यत् ति॒सृभिः॑ । \newline
15. यत् ति॒सृभि॑ स्ति॒सृभि॒र् यद् यत् ति॒सृभि॒ रत्यति॑ ति॒सृभि॒र् यद् यत् ति॒सृभि॒रति॑ । \newline
16. ति॒सृभि॒ रत्यति॑ ति॒सृभि॑ स्ति॒सृभि॒ रति॒ तत् तदति॑ ति॒सृभि॑ स्ति॒सृभि॒ रति॒ तत् । \newline
17. ति॒सृभि॒रिति॑ ति॒सृ - भिः॒ । \newline
18. अति॒ तत् तदत्यति॒ तद् रे॑चयेद् रेचये॒त् तदत्यति॒ तद् रे॑चयेत् । \newline
19. तद् रे॑चयेद् रेचये॒त् तत् तद् रे॑चये॒न् मन॑सा॒ मन॑सा रेचये॒त् तत् तद् रे॑चये॒न् मन॑सा । \newline
20. रे॒च॒ये॒न् मन॑सा॒ मन॑सा रेचयेद् रेचये॒न् मन॒सा ऽऽमन॑सा रेचयेद् रेचये॒न् मन॑सा । \newline
21. मन॒सा ऽऽमन॑सा॒ मन॒सा ऽऽघा॑रयति घारय॒त्या मन॑सा॒ मन॒सा ऽऽघा॑रयति । \newline
22. आ घा॑रयति घारय॒त्या घा॑रयति॒ मन॑सा॒ मन॑सा घारय॒त्या घा॑रयति॒ मन॑सा । \newline
23. घा॒र॒य॒ति॒ मन॑सा॒ मन॑सा घारयति घारयति॒ मन॑सा॒ हि हि मन॑सा घारयति घारयति॒ मन॑सा॒ हि । \newline
24. मन॑सा॒ हि हि मन॑सा॒ मन॑सा॒ ह्यना᳚प्त॒ मना᳚प्तꣳ॒॒ हि मन॑सा॒ मन॑सा॒ ह्यना᳚प्तम् । \newline
25. ह्यना᳚प्त॒ मना᳚प्तꣳ॒॒ हि ह्यना᳚प्त मा॒प्यत॑ आ॒प्यते ऽना᳚प्तꣳ॒॒ हि ह्यना᳚प्त मा॒प्यते᳚ । \newline
26. अना᳚प्त मा॒प्यत॑ आ॒प्यते ऽना᳚प्त॒ मना᳚प्त मा॒प्यते॑ ति॒र्यञ्च॑म् ति॒र्यञ्च॑ मा॒प्यते ऽना᳚प्त॒ मना᳚प्त मा॒प्यते॑ ति॒र्यञ्च᳚म् । \newline
27. आ॒प्यते॑ ति॒र्यञ्च॑म् ति॒र्यञ्च॑ मा॒प्यत॑ आ॒प्यते॑ ति॒र्यञ्च॒ मा ति॒र्यञ्च॑ मा॒प्यत॑ आ॒प्यते॑ ति॒र्यञ्च॒ मा । \newline
28. ति॒र्यञ्च॒ मा ति॒र्यञ्च॑म् ति॒र्यञ्च॒ मा घा॑रयति घारय॒त्या ति॒र्यञ्च॑म् ति॒र्यञ्च॒ मा घा॑रयति । \newline
29. आ घा॑रयति घारय॒त्या घा॑रय॒ त्यछं॑बट्कार॒ मछं॑बट्कारम् घारय॒त्या घा॑रय॒ त्यछं॑बट्कारम् । \newline
30. घा॒र॒य॒ त्यछं॑बट्कार॒ मछं॑बट्कारम् घारयति घारय॒ त्यछं॑बट्कारं॒ ॅवाग् वागछं॑बट्कारम् घारयति घारय॒ त्यछं॑बट्कारं॒ ॅवाक् । \newline
31. अछं॑बट्कारं॒ ॅवाग् वागछं॑बट्कार॒ मछं॑बट्कारं॒ ॅवाक् च॑ च॒ वागछं॑बट्कार॒ मछं॑बट्कारं॒ ॅवाक् च॑ । \newline
32. अछं॑बट्कार॒मित्यछं॑बट् - का॒र॒म् । \newline
33. वाक् च॑ च॒ वाग् वाक् च॒ मनो॒ मन॑श्च॒ वाग् वाक् च॒ मनः॑ । \newline
34. च॒ मनो॒ मन॑श्च च॒ मन॑श्च च॒ मन॑श्च च॒ मन॑श्च । \newline
35. मन॑श्च च॒ मनो॒ मन॑ श्चार्तीयेता मार्तीयेताम् च॒ मनो॒ मन॑ श्चार्तीयेताम् । \newline
36. चा॒र्ती॒ये॒ता॒ मा॒र्ती॒ये॒ता॒म् च॒ चा॒र्ती॒ये॒ता॒ म॒ह म॒ह मा᳚र्तीयेताम् च चार्तीयेता म॒हम् । \newline
37. आ॒र्ती॒ये॒ता॒ म॒ह म॒ह मा᳚र्तीयेता मार्तीयेता म॒हम् दे॒वेभ्यो॑ दे॒वेभ्यो॒ ऽह मा᳚र्तीयेता मार्तीयेता म॒हम् दे॒वेभ्यः॑ । \newline
38. अ॒हम् दे॒वेभ्यो॑ दे॒वेभ्यो॒ ऽह म॒हम् दे॒वेभ्यो॑ ह॒व्यꣳ ह॒व्यम् दे॒वेभ्यो॒ ऽह म॒हम् दे॒वेभ्यो॑ ह॒व्यम् । \newline
39. दे॒वेभ्यो॑ ह॒व्यꣳ ह॒व्यम् दे॒वेभ्यो॑ दे॒वेभ्यो॑ ह॒व्यं ॅव॑हामि वहामि ह॒व्यम् दे॒वेभ्यो॑ दे॒वेभ्यो॑ ह॒व्यं ॅव॑हामि । \newline
40. ह॒व्यं ॅव॑हामि वहामि ह॒व्यꣳ ह॒व्यं ॅव॑हा॒मीतीति॑ वहामि ह॒व्यꣳ ह॒व्यं ॅव॑हा॒मीति॑ । \newline
41. व॒हा॒मीतीति॑ वहामि वहा॒मीति॒ वाग् वागिति॑ वहामि वहा॒मीति॒ वाक् । \newline
42. इति॒ वाग् वागितीति॒ वाग॑ब्रवी दब्रवी॒द् वागितीति॒ वाग॑ब्रवीत् । \newline
43. वाग॑ब्रवी दब्रवी॒द् वाग् वाग॑ब्रवीद॒ह म॒ह म॑ब्रवी॒द् वाग् वाग॑ब्रवीद॒हम् । \newline
44. अ॒ब्र॒वी॒द॒ह म॒ह म॑ब्रवी दब्रवी द॒हम् दे॒वेभ्यो॑ दे॒वेभ्यो॒ ऽह म॑ब्रवी दब्रवी द॒हम् दे॒वेभ्यः॑ । \newline
45. अ॒हम् दे॒वेभ्यो॑ दे॒वेभ्यो॒ ऽह म॒हम् दे॒वेभ्य॒ इतीति॑ दे॒वेभ्यो॒ ऽह म॒हम् दे॒वेभ्य॒ इति॑ । \newline
46. दे॒वेभ्य॒ इतीति॑ दे॒वेभ्यो॑ दे॒वेभ्य॒ इति॒ मनो॒ मन॒ इति॑ दे॒वेभ्यो॑ दे॒वेभ्य॒ इति॒ मनः॑ । \newline
47. इति॒ मनो॒ मन॒ इतीति॒ मन॒ स्तौ तौ मन॒ इतीति॒ मन॒स्तौ । \newline
48. मन॒ स्तौ तौ मनो॒ मन॒ स्तौ प्र॒जाप॑तिम् प्र॒जाप॑ति॒म् तौ मनो॒ मन॒ स्तौ प्र॒जाप॑तिम् । \newline
49. तौ प्र॒जाप॑तिम् प्र॒जाप॑ति॒म् तौ तौ प्र॒जाप॑तिम् प्र॒श्ञम् प्र॒श्ञम् प्र॒जाप॑ति॒म् तौ तौ प्र॒जाप॑तिम् प्र॒श्ञम् । \newline
50. प्र॒जाप॑तिम् प्र॒श्ञम् प्र॒श्ञम् प्र॒जाप॑तिम् प्र॒जाप॑तिम् प्र॒श्ञ मै॑ता मैताम् प्र॒श्ञम् प्र॒जाप॑तिम् प्र॒जाप॑तिम् प्र॒श्ञ मै॑ताम् । \newline
51. प्र॒जाप॑ति॒मिति॑ प्र॒जा - प॒ति॒म् । \newline
52. प्र॒श्ञ मै॑ता मैताम् प्र॒श्ञम् प्र॒श्ञ मै॑ताꣳ॒॒ स स ऐ॑ताम् प्र॒श्ञम् प्र॒श्ञ मै॑ताꣳ॒॒ सः । \newline
53. ऐ॒ताꣳ॒॒ स स ऐ॑ता मैताꣳ॒॒ सो᳚ ऽब्रवी दब्रवी॒थ् स ऐ॑ता मैताꣳ॒॒ सो᳚ ऽब्रवीत् । \newline
54. सो᳚ ऽब्रवी दब्रवी॒थ् स सो᳚ ऽब्रवीत् प्र॒जाप॑तिः प्र॒जाप॑ति रब्रवी॒थ् स सो᳚ ऽब्रवीत् प्र॒जाप॑तिः । \newline
55. अ॒ब्र॒वी॒त् प्र॒जाप॑तिः प्र॒जाप॑ति रब्रवी दब्रवीत् प्र॒जाप॑तिर् दू॒तीर् दू॒तीः प्र॒जाप॑ति रब्रवी दब्रवीत् प्र॒जाप॑तिर् दू॒तीः । \newline
\pagebreak
\markright{ TS 2.5.11.5  \hfill https://www.vedavms.in \hfill}
\addcontentsline{toc}{section}{ TS 2.5.11.5 }
\section*{ TS 2.5.11.5 }

\textbf{TS 2.5.11.5 } \newline
\textbf{Samhita Paata} \newline

प्र॒जाप॑तिर्दू॒तीरे॒व त्वं मन॑सोऽसि॒ यद्धि मन॑सा॒ ध्याय॑ति॒ तद्वा॒चा वद॒तीति॒ तत् खलु॒ तुभ्यं॒ न वा॒चा जु॑हव॒न्नित्य॑ब्रवी॒त् तस्मा॒न्मन॑सा प्र॒जाप॑तये जुह्वति॒मन॑ इव॒ हि प्र॒जाप॑तिः प्र॒जाप॑ते॒राप्त्यै॑ परि॒धीन्थ्सं मा᳚र्ष्टि पु॒नात्ये॒वैना॒न्‌त्रिर्म॑द्ध्य॒मं त्रयो॒ वै प्रा॒णाः प्रा॒णाने॒वाभि ज॑यति॒ त्रिर्द॑क्षिणा॒र्द्ध्यं॑ त्रय॑ - [  ] \newline

\textbf{Pada Paata} \newline

प्र॒जाप॑ति॒रिति॑ प्र॒जा - प॒तिः॒ । दू॒तीः । ए॒व । त्वम् । मन॑सः । अ॒सि॒ । यत् । हि । मन॑सा । ध्याय॑ति । तत् । वा॒चा । वद॑ति । इति॑ । तत् । खलु॑ । तुभ्य᳚म् । न । वा॒चा । जु॒ह॒व॒न्न् । इति॑ । अ॒ब्र॒वी॒त् । तस्मा᳚त् । मन॑सा । प्र॒जाप॑तय॒ इति॑ प्र॒जा - प॒त॒ये॒ । जु॒ह्व॒ति॒ । मनः॑ । इ॒व॒ । हि । प्र॒जाप॑ति॒रिति॑ प्र॒जा - प॒तिः॒ । प्र॒जाप॑ते॒रिति॑ प्र॒जा - प॒तेः॒ । आप्त्यै᳚ । प॒रि॒धीनिति॑ परि - धीन् । समिति॑ । मा॒र्ष्टि॒ । पु॒नाति॑ । ए॒व । ए॒ना॒न् । त्रिः । म॒द्ध्य॒मम् । त्रयः॑ । वै । प्रा॒णा इति॑ प्र - अ॒नाः । प्रा॒णानिति॑ प्र - अ॒नान् । ए॒व । अ॒भिति॑ । ज॒य॒ति॒ । त्रिः । द॒क्षि॒णा॒द्‌र्ध्य॑मिति॑ दक्षिण - अ॒द्‌र्ध्य᳚म् । त्रयः॑ ।  \newline


\textbf{Krama Paata} \newline

प्र॒जाप॑तिर् दू॒तीः । प्र॒जाप॑ति॒रिति॑ प्र॒जा - प॒तिः॒ । दू॒तीरे॒व । ए॒व त्वम् । त्वम् मन॑सः । मन॑सो ऽसि । अ॒सि॒ यत् । यद्धि । हि मन॑सा । मन॑सा॒ ध्याय॑ति । ध्याय॑ति॒ तत् । तद् वा॒चा । वा॒चा व॑दति । व॒द॒तीति॑ । इति॒ तत् । तत् खलु॑ । खलु॒ तुभ्य᳚म् । तुभ्य॒म् न । न वा॒चा । वा॒चा जु॑हवन्न् । जु॒ह॒व॒न्निति॑ । इत्य॑ब्रवीत् । अ॒ब्र॒वी॒त् तस्मा᳚त् । तस्मा॒न् मन॑सा । मन॑सा प्र॒जाप॑तये । प्र॒जाप॑तये जुह्वति । प्र॒जापत॑य॒ इति॑ प्र॒जा - प॒त॒ये॒ । जु॒ह्व॒ति॒ मनः॑ । मन॑ इव । इ॒व॒ हि । हि प्र॒जाप॑तिः । प्र॒जाप॑तिः प्र॒जाप॑तेः । प्र॒जाप॑ति॒रिति॑ प्र॒जा - प॒तिः॒ । प्र॒जाप॑ते॒राप्त्यै᳚ । प्र॒जाप॑ते॒रिति॑ प्र॒जा - प॒तेः॒ । आप्त्यै॑ परि॒धीन् । प॒रि॒धीन्थ् सम् । प॒रि॒धीनिति॑ परि - धीन् । सम् मा᳚र्ष्टि । मा॒र्ष्टि॒ पु॒नाति॑ । पु॒नात्ये॒व । ए॒वैनान्॑ । ए॒ना॒न् त्रिः । त्रिर् म॑द्ध्य॒मम् । म॒द्ध्य॒मम् त्रयः॑ । त्रयो॒ वै । वै प्रा॒णाः । प्रा॒णाः प्रा॒णान् । प्रा॒णा इति॑ प्र - अ॒नाः । प्रा॒णाने॒व । प्रा॒णानिति॑ प्र - अ॒नान् । ए॒वाभि । अ॒भि ज॑यति । ज॒य॒ति॒ त्रिः । त्रिर् द॑क्षिणा॒र्द्ध्य᳚म् । द॒क्षि॒णा॒र्द्ध्य॑म् त्रयः॑ । द॒क्षि॒णा॒र्द्ध्य॑मिति॑ दक्षिण - अ॒र्द्ध्य᳚म् । त्रय॑ इ॒मे \newline

\textbf{Jatai Paata} \newline

1. प्र॒जाप॑तिर् दू॒तीर् दू॒तीः प्र॒जाप॑तिः प्र॒जाप॑तिर् दू॒तीः । \newline
2. प्र॒जाप॑ति॒रिति॑ प्र॒जा - प॒तिः॒ । \newline
3. दू॒ती रे॒वैव दू॒तीर् दू॒ती रे॒व । \newline
4. ए॒व त्वम् त्व मे॒वैव त्वम् । \newline
5. त्वम् मन॑सो॒ मन॑ स॒स्त्वम् त्वम् मन॑सः । \newline
6. मन॑सो ऽस्यसि॒ मन॑सो॒ मन॑सो ऽसि । \newline
7. अ॒सि॒ यद् यद॑स्यसि॒ यत् । \newline
8. यद्धि हि यद् यद्धि । \newline
9. हि मन॑सा॒ मन॑सा॒ हि हि मन॑सा । \newline
10. मन॑सा॒ ध्याय॑ति॒ ध्याय॑ति॒ मन॑सा॒ मन॑सा॒ ध्याय॑ति । \newline
11. ध्याय॑ति॒ तत् तद् ध्याय॑ति॒ ध्याय॑ति॒ तत् । \newline
12. तद् वा॒चा वा॒चा तत् तद् वा॒चा । \newline
13. वा॒चा वद॑ति॒ वद॑ति वा॒चा वा॒चा वद॑ति । \newline
14. वद॒तीतीति॒ वद॑ति॒ वद॒तीति॑ । \newline
15. इति॒ तत् तदितीति॒ तत् । \newline
16. तत् खलु॒ खलु॒ तत् तत् खलु॑ । \newline
17. खलु॒ तुभ्य॒म् तुभ्य॒म् खलु॒ खलु॒ तुभ्य᳚म् । \newline
18. तुभ्य॒म् न न तुभ्य॒म् तुभ्य॒म् न । \newline
19. न वा॒चा वा॒चा न न वा॒चा । \newline
20. वा॒चा जु॑हवन् जुहवन्. वा॒चा वा॒चा जु॑हवन्न् । \newline
21. जु॒ह॒व॒न् नितीति॑ जुहवन् जुहव॒न् निति॑ । \newline
22. इत्य॑ब्रवी दब्रवी॒ दिती त्य॑ब्रवीत् । \newline
23. अ॒ब्र॒वी॒त् तस्मा॒त् तस्मा॑ दब्रवी दब्रवी॒त् तस्मा᳚त् । \newline
24. तस्मा॒न् मन॑सा॒ मन॑सा॒ तस्मा॒त् तस्मा॒न् मन॑सा । \newline
25. मन॑सा प्र॒जाप॑तये प्र॒जाप॑तये॒ मन॑सा॒ मन॑सा प्र॒जाप॑तये । \newline
26. प्र॒जाप॑तये जुह्वति जुह्वति प्र॒जाप॑तये प्र॒जाप॑तये जुह्वति । \newline
27. प्र॒जाप॑तय॒ इति॑ प्र॒जा - प॒त॒ये॒ । \newline
28. जु॒ह्व॒ति॒ मनो॒ मनो॑ जुह्वति जुह्वति॒ मनः॑ । \newline
29. मन॑ इवे व॒ मनो॒ मन॑ इव । \newline
30. इ॒व॒ हि हीवे॑ व॒ हि । \newline
31. हि प्र॒जाप॑तिः प्र॒जाप॑ति॒र्॒. हि हि प्र॒जाप॑तिः । \newline
32. प्र॒जाप॑तिः प्र॒जाप॑तेः प्र॒जाप॑तेः प्र॒जाप॑तिः प्र॒जाप॑तिः प्र॒जाप॑तेः । \newline
33. प्र॒जाप॑ति॒रिति॑ प्र॒जा - प॒तिः॒ । \newline
34. प्र॒जाप॑ते॒ राप्त्या॒ आप्त्यै᳚ प्र॒जाप॑तेः प्र॒जाप॑ते॒ राप्त्यै᳚ । \newline
35. प्र॒जाप॑ते॒रिति॑ प्र॒जा - प॒तेः॒ । \newline
36. आप्त्यै॑ परि॒धीन् प॑रि॒धी नाप्त्या॒ आप्त्यै॑ परि॒धीन् । \newline
37. प॒रि॒धीन् थ्सꣳ सम् प॑रि॒धीन् प॑रि॒धीन् थ्सम् । \newline
38. प॒रि॒धीनिति॑ परि - धीन् । \newline
39. सम् मा᳚र्ष्टि मार्ष्टि॒ सꣳ सम् मा᳚र्ष्टि । \newline
40. मा॒र्ष्टि॒ पु॒नाति॑ पु॒नाति॑ मार्ष्टि मार्ष्टि पु॒नाति॑ । \newline
41. पु॒ना त्ये॒वैव पु॒नाति॑ पु॒ना त्ये॒व । \newline
42. ए॒वैना॑ नेना ने॒वैवैनान्॑ । \newline
43. ए॒ना॒न् त्रि स्त्रिरे॑ना नेना॒न् त्रिः । \newline
44. त्रिर् म॑द्ध्य॒मम् म॑द्ध्य॒मम् त्रिस्त्रिर् म॑द्ध्य॒मम् । \newline
45. म॒द्ध्य॒मम् त्रय॒स्त्रयो॑ मद्ध्य॒मम् म॑द्ध्य॒मम् त्रयः॑ । \newline
46. त्रयो॒ वै वै त्रय॒ स्त्रयो॒ वै । \newline
47. वै प्रा॒णाः प्रा॒णा वै वै प्रा॒णाः । \newline
48. प्रा॒णाः प्रा॒णान् प्रा॒णान् प्रा॒णाः प्रा॒णाः प्रा॒णान् । \newline
49. प्रा॒णा इति॑ प्र - अ॒नाः । \newline
50. प्रा॒णा ने॒वैव प्रा॒णान् प्रा॒णा ने॒व । \newline
51. प्रा॒णानिति॑ प्र - अ॒नान् । \newline
52. ए॒वाभ्या᳚(1॒)भ्ये॑वैवाभि । \newline
53. अ॒भि ज॑यति जय त्य॒भ्य॑भि ज॑यति । \newline
54. ज॒य॒ति॒ त्रि स्त्रिर् ज॑यति जयति॒ त्रिः । \newline
55. त्रिर् द॑क्षिणा॒र्द्ध्य॑म् दक्षिणा॒र्द्ध्य॑म् त्रि स्त्रिर् द॑क्षिणा॒र्द्ध्य᳚म् । \newline
56. द॒क्षि॒णा॒र्द्ध्य॑म् त्रय॒ स्त्रयो॑ दक्षिणा॒र्द्ध्य॑म् दक्षिणा॒र्द्ध्य॑म् त्रयः॑ । \newline
57. द॒क्षि॒णा॒र्द्ध्य॑मिति॑ दक्षिण - अ॒र्द्ध्य᳚म् । \newline
58. त्रय॑ इ॒म इ॒मे त्रय॒ स्त्रय॑ इ॒मे । \newline

\textbf{Ghana Paata } \newline

1. प्र॒जाप॑तिर् दू॒तीर् दू॒तीः प्र॒जाप॑तिः प्र॒जाप॑तिर् दू॒ती रे॒वैव दू॒तीः प्र॒जाप॑तिः प्र॒जाप॑तिर् दू॒तीरे॒व । \newline
2. प्र॒जाप॑ति॒रिति॑ प्र॒जा - प॒तिः॒ । \newline
3. दू॒ती रे॒वैव दू॒तीर् दू॒तीरे॒व त्वम् त्व मे॒व दू॒तीर् दू॒तीरे॒व त्वम् । \newline
4. ए॒व त्वम् त्व मे॒वैव त्वम् मन॑सो॒ मन॑स॒ स्त्व मे॒वैव त्वम् मन॑सः । \newline
5. त्वम् मन॑सो॒ मन॑स॒ स्त्वम् त्वम् मन॑सो ऽस्यसि॒ मन॑स॒ स्त्वम् त्वम् मन॑सो ऽसि । \newline
6. मन॑सो ऽस्यसि॒ मन॑सो॒ मन॑सो ऽसि॒ यद् यद॑सि॒ मन॑सो॒ मन॑सो ऽसि॒ यत् । \newline
7. अ॒सि॒ यद् यद॑स्यसि॒ यद्धि हि यद॑स्यसि॒ यद्धि । \newline
8. यद्धि हि यद् यद्धि मन॑सा॒ मन॑सा॒ हि यद् यद्धि मन॑सा । \newline
9. हि मन॑सा॒ मन॑सा॒ हि हि मन॑सा॒ ध्याय॑ति॒ ध्याय॑ति॒ मन॑सा॒ हि हि मन॑सा॒ ध्याय॑ति । \newline
10. मन॑सा॒ ध्याय॑ति॒ ध्याय॑ति॒ मन॑सा॒ मन॑सा॒ ध्याय॑ति॒ तत् तद् ध्याय॑ति॒ मन॑सा॒ मन॑सा॒ ध्याय॑ति॒ तत् । \newline
11. ध्याय॑ति॒ तत् तद् ध्याय॑ति॒ ध्याय॑ति॒ तद् वा॒चा वा॒चा तद् ध्याय॑ति॒ ध्याय॑ति॒ तद् वा॒चा । \newline
12. तद् वा॒चा वा॒चा तत् तद् वा॒चा वद॑ति॒ वद॑ति वा॒चा तत् तद् वा॒चा वद॑ति । \newline
13. वा॒चा वद॑ति॒ वद॑ति वा॒चा वा॒चा वद॒तीतीति॒ वद॑ति वा॒चा वा॒चा वद॒तीति॑ । \newline
14. वद॒तीतीति॒ वद॑ति॒ वद॒तीति॒ तत् तदिति॒ वद॑ति॒ वद॒तीति॒ तत् । \newline
15. इति॒ तत् तदितीति॒ तत् खलु॒ खलु॒ तदितीति॒ तत् खलु॑ । \newline
16. तत् खलु॒ खलु॒ तत् तत् खलु॒ तुभ्य॒म् तुभ्य॒म् खलु॒ तत् तत् खलु॒ तुभ्य᳚म् । \newline
17. खलु॒ तुभ्य॒म् तुभ्य॒म् खलु॒ खलु॒ तुभ्य॒म् न न तुभ्य॒म् खलु॒ खलु॒ तुभ्य॒म् न । \newline
18. तुभ्य॒म् न न तुभ्य॒म् तुभ्य॒म् न वा॒चा वा॒चा न तुभ्य॒म् तुभ्य॒म् न वा॒चा । \newline
19. न वा॒चा वा॒चा न न वा॒चा जु॑हवन् जुहवन्. वा॒चा न न वा॒चा जु॑हवन्न् । \newline
20. वा॒चा जु॑हवन् जुहवन्. वा॒चा वा॒चा जु॑हव॒न् नितीति॑ जुहवन्. वा॒चा वा॒चा जु॑हव॒न् निति॑ । \newline
21. जु॒ह॒व॒न् नितीति॑ जुहवन् जुहव॒न् नित्य॑ब्रवी दब्रवी॒ दिति॑ जुहवन् जुहव॒न् नित्य॑ब्रवीत् । \newline
22. इत्य॑ब्रवी दब्रवी॒ दिती त्य॑ब्रवी॒त् तस्मा॒त् तस्मा॑ दब्रवी॒ दिती त्य॑ब्रवी॒त् तस्मा᳚त् । \newline
23. अ॒ब्र॒वी॒त् तस्मा॒त् तस्मा॑ दब्रवी दब्रवी॒त् तस्मा॒न् मन॑सा॒ मन॑सा॒ तस्मा॑ दब्रवी दब्रवी॒त् तस्मा॒न् मन॑सा । \newline
24. तस्मा॒न् मन॑सा॒ मन॑सा॒ तस्मा॒त् तस्मा॒न् मन॑सा प्र॒जाप॑तये प्र॒जाप॑तये॒ मन॑सा॒ तस्मा॒त् तस्मा॒न् मन॑सा प्र॒जाप॑तये । \newline
25. मन॑सा प्र॒जाप॑तये प्र॒जाप॑तये॒ मन॑सा॒ मन॑सा प्र॒जाप॑तये जुह्वति जुह्वति प्र॒जाप॑तये॒ मन॑सा॒ मन॑सा प्र॒जाप॑तये जुह्वति । \newline
26. प्र॒जाप॑तये जुह्वति जुह्वति प्र॒जाप॑तये प्र॒जाप॑तये जुह्वति॒ मनो॒ मनो॑ जुह्वति प्र॒जाप॑तये प्र॒जाप॑तये जुह्वति॒ मनः॑ । \newline
27. प्र॒जाप॑तय॒ इति॑ प्र॒जा - प॒त॒ये॒ । \newline
28. जु॒ह्व॒ति॒ मनो॒ मनो॑ जुह्वति जुह्वति॒ मन॑ इवे व॒ मनो॑ जुह्वति जुह्वति॒ मन॑ इव । \newline
29. मन॑ इवे व॒ मनो॒ मन॑ इव॒ हि हीव॒ मनो॒ मन॑ इव॒ हि । \newline
30. इ॒व॒ हि हीवे॑ व॒ हि प्र॒जाप॑तिः प्र॒जाप॑ति॒र्॒. हीवे॑ व॒ हि प्र॒जाप॑तिः । \newline
31. हि प्र॒जाप॑तिः प्र॒जाप॑ति॒र्॒. हि हि प्र॒जाप॑तिः प्र॒जाप॑तेः प्र॒जाप॑तेः प्र॒जाप॑ति॒र्॒. हि हि प्र॒जाप॑तिः प्र॒जाप॑तेः । \newline
32. प्र॒जाप॑तिः प्र॒जाप॑तेः प्र॒जाप॑तेः प्र॒जाप॑तिः प्र॒जाप॑तिः प्र॒जाप॑ते॒ राप्त्या॒ आप्त्यै᳚ प्र॒जाप॑तेः प्र॒जाप॑तिः प्र॒जाप॑तिः प्र॒जाप॑ते॒ राप्त्यै᳚ । \newline
33. प्र॒जाप॑ति॒रिति॑ प्र॒जा - प॒तिः॒ । \newline
34. प्र॒जाप॑ते॒ राप्त्या॒ आप्त्यै᳚ प्र॒जाप॑तेः प्र॒जाप॑ते॒ राप्त्यै॑ परि॒धीन् प॑रि॒धी नाप्त्यै᳚ प्र॒जाप॑तेः प्र॒जाप॑ते॒ राप्त्यै॑ परि॒धीन् । \newline
35. प्र॒जाप॑ते॒रिति॑ प्र॒जा - प॒तेः॒ । \newline
36. आप्त्यै॑ परि॒धीन् प॑रि॒धी नाप्त्या॒ आप्त्यै॑ परि॒धीन् थ्सꣳ सम् प॑रि॒धी नाप्त्या॒ आप्त्यै॑ परि॒धीन् थ्सम् । \newline
37. प॒रि॒धीन् थ्सꣳ सम् प॑रि॒धीन् प॑रि॒धीन् थ्सम् मा᳚र्ष्टि मार्ष्टि॒ सम् प॑रि॒धीन् प॑रि॒धीन् थ्सम् मा᳚र्ष्टि । \newline
38. प॒रि॒धीनिति॑ परि - धीन् । \newline
39. सम् मा᳚र्ष्टि मार्ष्टि॒ सꣳ सम् मा᳚र्ष्टि पु॒नाति॑ पु॒नाति॑ मार्ष्टि॒ सꣳ सम् मा᳚र्ष्टि पु॒नाति॑ । \newline
40. मा॒र्ष्टि॒ पु॒नाति॑ पु॒नाति॑ मार्ष्टि मार्ष्टि पु॒ना त्ये॒वैव पु॒नाति॑ मार्ष्टि मार्ष्टि पु॒नात्ये॒व । \newline
41. पु॒नात्ये॒वैव पु॒नाति॑ पु॒ना त्ये॒वैना॑ नेना ने॒व पु॒नाति॑ पु॒ना त्ये॒वैनान्॑ । \newline
42. ए॒वैना॑ नेना ने॒वैवैना॒न् त्रि स्त्रिरे॑ना ने॒वैवैना॒न् त्रिः । \newline
43. ए॒ना॒न् त्रि स्त्रिरे॑ना नेना॒न् त्रिर् म॑द्ध्य॒मम् म॑द्ध्य॒मम् त्रिरे॑ना नेना॒न् त्रिर् म॑द्ध्य॒मम् । \newline
44. त्रिर् म॑द्ध्य॒मम् म॑द्ध्य॒मम् त्रि स्त्रिर् म॑द्ध्य॒मम् त्रय॒ स्त्रयो॑ मद्ध्य॒मम् त्रि स्त्रिर् म॑द्ध्य॒मम् त्रयः॑ । \newline
45. म॒द्ध्य॒मम् त्रय॒ स्त्रयो॑ मद्ध्य॒मम् म॑द्ध्य॒मम् त्रयो॒ वै वै त्रयो॑ मद्ध्य॒मम् म॑द्ध्य॒मम् त्रयो॒ वै । \newline
46. त्रयो॒ वै वै त्रय॒ स्त्रयो॒ वै प्रा॒णाः प्रा॒णा वै त्रय॒ स्त्रयो॒ वै प्रा॒णाः । \newline
47. वै प्रा॒णाः प्रा॒णा वै वै प्रा॒णाः प्रा॒णान् प्रा॒णान् प्रा॒णा वै वै प्रा॒णाः प्रा॒णान् । \newline
48. प्रा॒णाः प्रा॒णान् प्रा॒णान् प्रा॒णाः प्रा॒णाः प्रा॒णा ने॒वैव प्रा॒णान् प्रा॒णाः प्रा॒णाः प्रा॒णा ने॒व । \newline
49. प्रा॒णा इति॑ प्र - अ॒नाः । \newline
50. प्रा॒णा ने॒वैव प्रा॒णान् प्रा॒णा ने॒वाभ्या᳚(1॒)भ्ये॑व प्रा॒णान् प्रा॒णा ने॒वाभि । \newline
51. प्रा॒णानिति॑ प्र - अ॒नान् । \newline
52. ए॒वाभ्या᳚(1॒)भ्ये॑ वैवाभि ज॑यति जय त्य॒भ्ये॑ वैवाभि ज॑यति । \newline
53. अ॒भि ज॑यति जय त्य॒भ्य॑भि ज॑यति॒ त्रिस्त्रिर् ज॑य त्य॒भ्य॑भि ज॑यति॒ त्रिः । \newline
54. ज॒य॒ति॒ त्रिस्त्रिर् ज॑यति जयति॒ त्रिर् द॑क्षिणा॒र्द्ध्य॑म् दक्षिणा॒र्द्ध्य॑म् त्रिर् ज॑यति जयति॒ त्रिर् द॑क्षिणा॒र्द्ध्य᳚म् । \newline
55. त्रिर् द॑क्षिणा॒र्द्ध्य॑म् दक्षिणा॒र्द्ध्य॑म् त्रि स्त्रिर् द॑क्षिणा॒र्द्ध्य॑म् त्रय॒स्त्रयो॑ दक्षिणा॒र्द्ध्य॑म् त्रि स्त्रिर् द॑क्षिणा॒र्द्ध्य॑म् त्रयः॑ । \newline
56. द॒क्षि॒णा॒र्द्ध्य॑म् त्रय॒ स्त्रयो॑ दक्षिणा॒र्द्ध्य॑म् दक्षिणा॒र्द्ध्य॑म् त्रय॑ इ॒म इ॒मे त्रयो॑ दक्षिणा॒र्द्ध्य॑म् दक्षिणा॒र्द्ध्य॑म् त्रय॑ इ॒मे । \newline
57. द॒क्षि॒णा॒र्द्ध्य॑मिति॑ दक्षिण - अ॒र्द्ध्य᳚म् । \newline
58. त्रय॑ इ॒म इ॒मे त्रय॒ स्त्रय॑ इ॒मे लो॒का लो॒का इ॒मे त्रय॒ स्त्रय॑ इ॒मे लो॒काः । \newline
\pagebreak
\markright{ TS 2.5.11.6  \hfill https://www.vedavms.in \hfill}
\addcontentsline{toc}{section}{ TS 2.5.11.6 }
\section*{ TS 2.5.11.6 }

\textbf{TS 2.5.11.6 } \newline
\textbf{Samhita Paata} \newline

इ॒मे लो॒का इ॒माने॒व लो॒कान॒भि ज॑यति॒ त्रिरु॑त्तरा॒र्द्ध्यं॑ त्रयो॒ वै दे॑व॒यानाः॒ पन्था॑न॒स्ताने॒वाभि ज॑यति॒ त्रिरुप॑ वाजयति॒ त्रयो॒ वै दे॑वलो॒का दे॑वलो॒काने॒वाभि ज॑यति॒ द्वाद॑श॒ सं प॑द्यन्ते॒ द्वाद॑श॒ मासाः᳚ संॅवथ्स॒रःसं॑ॅवथ्स॒रमे॒व प्री॑णा॒त्यथो॑ संॅवथ्स॒रमे॒वास्मा॒उप॑ दधाति सुव॒र्गस्य॑ लो॒कस्य॒ सम॑ष्‌ट्या आघा॒रमा घा॑रयति ति॒र इ॑व॒ - [  ] \newline

\textbf{Pada Paata} \newline

इ॒मे । लो॒काः । इ॒मान् । ए॒व । लो॒कान् । अ॒भीति॑ । ज॒य॒ति॒ । त्रिः । उ॒त्त॒रा॒द्‌र्ध्य॑मित्यु॑त्तर - अ॒द्‌र्ध्य᳚म् । त्रयः॑ । वै । दे॒व॒याना॒ इति॑ देव - यानाः᳚ । पन्था॑नः । तान् । ए॒व । अ॒भीति॑ । ज॒य॒ति॒ । त्रिः । उपेति॑ । वा॒ज॒य॒ति॒ । त्रयः॑ । वै । दे॒व॒लो॒का इति॑ देव - लो॒काः । दे॒व॒लो॒कानिति॑ देव - लो॒कान् । ए॒व । अ॒भीति॑ । ज॒य॒ति॒ । द्वाद॑श । समिति॑ । प॒द्य॒न्ते॒ । द्वाद॑श । मासाः᳚ । सं॒ॅव॒थ्स॒र इति॑ सं - व॒थ्स॒रः । सं॒ॅव॒थ्स॒रमिति॑ सं - व॒थ्स॒रम् । ए॒व । प्री॒णा॒ति॒ । अथो॒ इति॑ । सं॒ॅव॒थ्स॒रमिति॑ सं - व॒थ्स॒रम् । ए॒व । अ॒स्मै॒ । उपेति॑ । द॒धा॒ति॒ । सु॒व॒र्गस्येति॑ सुवः - गस्य॑ । लो॒कस्य॑ । सम॑ष्ट्या॒ इति॒ सं - अ॒ष्ट्यै॒ । आ॒घा॒रमित्या᳚ - घा॒रम् । एति॑ । घा॒र॒य॒ति॒ । ति॒रः । इ॒व॒ ।  \newline


\textbf{Krama Paata} \newline

इ॒मे लो॒काः । लो॒का इ॒मान् । इ॒माने॒व । ए॒व लो॒कान् । लो॒कान॒भि । अ॒भि ज॑यति । ज॒य॒ति॒ त्रिः । त्रिरु॑त्तरा॒र्द्ध्य᳚म् । उ॒त्त॒रा॒र्द्ध्य॑म् त्रयः॑ । उ॒त्त॒रा॒र्द्ध्य॑मित्यु॑त्तर - अ॒र्द्ध्य᳚म् । त्रयो॒ वै । वै दे॑व॒यानाः᳚ । दे॒व॒यानाः॒ पन्था॑नः । दे॒व॒याना॒ इति॑ देव - यानाः᳚ । पन्था॑न॒ स्तान् । ताने॒व । ए॒वाभि । अ॒भि ज॑यति । ज॒य॒ति॒ त्रिः । त्रिरुप॑ । उप॑ वाजयति । वा॒ज॒य॒ति॒ त्रयः॑ । त्रयो॒ वै । वै दे॑वलो॒काः । दे॒व॒लो॒का दे॑वलो॒कान् । दे॒व॒लो॒का इति॑ देव - लो॒काः । दे॒व॒लो॒काने॒व । दे॒व॒लो॒कानिति॑ देव - लो॒कान् । ए॒वाभि । अ॒भि ज॑यति । ज॒य॒ति॒ द्वाद॑श । द्वाद॑श॒ सम् । सम् प॑द्यन्ते । प॒द्य॒न्ते॒ द्वाद॑श । द्वाद॑श॒ मासाः᳚ । मासाः᳚ सम्ॅवथ्स॒रः । स॒म्ॅव॒थ्स॒रः स॑म्ॅवथ्स॒रम् । स॒म्ॅव॒थ्स॒र इति॑ सम् - व॒थ्स॒रः । स॒म्ॅव॒थ्स॒रमे॒व । स॒म्ॅव॒थ्स॒रमिति॑ सम् - व॒थ्स॒रम् । ए॒व प्री॑णाति । प्री॒णा॒त्यथो᳚ । अथो॑ सम्ॅवथ्स॒रम् । अथो॒ इत्यथो᳚ । स॒म्ॅव॒थ्स॒रमे॒व । स॒म्ॅव॒थ्स॒रमिति॑ सम् - व॒थ्स॒रम् । ए॒वास्मै᳚ । अ॒स्मा॒ उप॑ । उप॑ दधाति । द॒धा॒ति॒ सु॒व॒र्गस्य॑ । सु॒व॒र्गस्य॑ लो॒कस्य॑ । सु॒व॒र्गस्येति॑ सुवः - गस्य॑ । लो॒कस्य॒ सम॑ष्ट्यै । सम॑ष्ट्या आघा॒रम् । सम॑ष्ट्या॒ इति॒ सम् - अ॒ष्ट्यै॒ । आ॒घा॒रमा । आ॒घा॒रमित्या᳚ - घा॒रम् । आ घा॑रयति । घा॒र॒य॒ति॒ ति॒रः । ति॒र इ॑व । इ॒व॒ वै \newline

\textbf{Jatai Paata} \newline

1. इ॒मे लो॒का लो॒का इ॒म इ॒मे लो॒काः । \newline
2. लो॒का इ॒मा नि॒मान् ॅलो॒का लो॒का इ॒मान् । \newline
3. इ॒मा ने॒वैवे मा नि॒मा ने॒व । \newline
4. ए॒व लो॒कान् ॅलो॒का ने॒वैव लो॒कान् । \newline
5. लो॒का न॒भ्य॑भि लो॒कान् ॅलो॒का न॒भि । \newline
6. अ॒भि ज॑यति जय त्य॒भ्य॑भि ज॑यति । \newline
7. ज॒य॒ति॒ त्रि स्त्रिर् ज॑यति जयति॒ त्रिः । \newline
8. त्रिरु॑त्तरा॒र्द्ध्य॑ मुत्तरा॒र्द्ध्य॑म् त्रि स्त्रिरु॑त्तरा॒र्द्ध्य᳚म् । \newline
9. उ॒त्त॒रा॒र्द्ध्य॑म् त्रय॒ स्त्रय॑ उत्तरा॒र्द्ध्य॑ मुत्तरा॒र्द्ध्य॑म् त्रयः॑ । \newline
10. उ॒त्त॒रा॒र्द्ध्य॑मित्यु॑त्तर - अ॒र्द्ध्य᳚म् । \newline
11. त्रयो॒ वै वै त्रय॒ स्त्रयो॒ वै । \newline
12. वै दे॑व॒याना॑ देव॒याना॒ वै वै दे॑व॒यानाः᳚ । \newline
13. दे॒व॒यानाः॒ पन्था॑नः॒ पन्था॑नो देव॒याना॑ देव॒यानाः॒ पन्था॑नः । \newline
14. दे॒व॒याना॒ इति॑ देव - यानाः᳚ । \newline
15. पन्था॑न॒ स्ताꣳ स्तान् पन्था॑नः॒ पन्था॑न॒ स्तान् । \newline
16. ता ने॒वैव ताꣳ स्ता ने॒व । \newline
17. ए॒वाभ्या᳚(1॒)भ्ये॑वैवाभि । \newline
18. अ॒भि ज॑यति जय त्य॒भ्य॑भि ज॑यति । \newline
19. ज॒य॒ति॒ त्रि स्त्रिर् ज॑यति जयति॒ त्रिः । \newline
20. त्रिरुपोप॒ त्रि स्त्रिरुप॑ । \newline
21. उप॑ वाजयति वाजय॒ त्युपोप॑ वाजयति । \newline
22. वा॒ज॒य॒ति॒ त्रय॒ स्त्रयो॑ वाजयति वाजयति॒ त्रयः॑ । \newline
23. त्रयो॒ वै वै त्रय॒ स्त्रयो॒ वै । \newline
24. वै दे॑वलो॒का दे॑वलो॒का वै वै दे॑वलो॒काः । \newline
25. दे॒व॒लो॒का दे॑वलो॒कान् दे॑वलो॒कान् दे॑वलो॒का दे॑वलो॒का दे॑वलो॒कान् । \newline
26. दे॒व॒लो॒का इति॑ देव - लो॒काः । \newline
27. दे॒व॒लो॒का ने॒वैव दे॑वलो॒कान् दे॑वलो॒का ने॒व । \newline
28. दे॒व॒लो॒कानिति॑ देव - लो॒कान् । \newline
29. ए॒वाभ्या᳚(1॒)भ्ये॑वैवाभि । \newline
30. अ॒भि ज॑यति जय त्य॒भ्य॑भि ज॑यति । \newline
31. ज॒य॒ति॒ द्वाद॑श॒ द्वाद॑श जयति जयति॒ द्वाद॑श । \newline
32. द्वाद॑श॒ सꣳ सम् द्वाद॑श॒ द्वाद॑श॒ सम् । \newline
33. सम् प॑द्यन्ते पद्यन्ते॒ सꣳ सम् प॑द्यन्ते । \newline
34. प॒द्य॒न्ते॒ द्वाद॑श॒ द्वाद॑श पद्यन्ते पद्यन्ते॒ द्वाद॑श । \newline
35. द्वाद॑श॒ मासा॒ मासा॒ द्वाद॑श॒ द्वाद॑श॒ मासाः᳚ । \newline
36. मासाः᳚ संॅवथ्स॒रः सं॑ॅवथ्स॒रो मासा॒ मासाः᳚ संॅवथ्स॒रः । \newline
37. सं॒ॅव॒थ्स॒रः सं॑ॅवथ्स॒रꣳ सं॑ॅवथ्स॒रꣳ सं॑ॅवथ्स॒रः सं॑ॅवथ्स॒रः सं॑ॅवथ्स॒रम् । \newline
38. सं॒ॅव॒थ्स॒र इति॑ सं - व॒थ्स॒रः । \newline
39. सं॒ॅव॒थ्स॒र मे॒वैव सं॑ॅवथ्स॒रꣳ सं॑ॅवथ्स॒र मे॒व । \newline
40. सं॒ॅव॒थ्स॒रमिति॑ सं - व॒थ्स॒रम् । \newline
41. ए॒व प्री॑णाति प्रीणा त्ये॒वैव प्री॑णाति । \newline
42. प्री॒णा॒ त्यथो॒ अथो᳚ प्रीणाति प्रीणा॒ त्यथो᳚ । \newline
43. अथो॑ संॅवथ्स॒रꣳ सं॑ॅवथ्स॒र मथो॒ अथो॑ संॅवथ्स॒रम् । \newline
44. अथो॒ इत्यथो᳚ । \newline
45. सं॒ॅव॒थ्स॒र मे॒वैव सं॑ॅवथ्स॒रꣳ सं॑ॅवथ्स॒र मे॒व । \newline
46. सं॒ॅव॒थ्स॒रमिति॑ सं - व॒थ्स॒रम् । \newline
47. ए॒वास्मा॑ अस्मा ए॒वैवास्मै᳚ । \newline
48. अ॒स्मा॒ उपोपा᳚स्मा अस्मा॒ उप॑ । \newline
49. उप॑ दधाति दधा॒ त्युपोप॑ दधाति । \newline
50. द॒धा॒ति॒ सु॒व॒र्गस्य॑ सुव॒र्गस्य॑ दधाति दधाति सुव॒र्गस्य॑ । \newline
51. सु॒व॒र्गस्य॑ लो॒कस्य॑ लो॒कस्य॑ सुव॒र्गस्य॑ सुव॒र्गस्य॑ लो॒कस्य॑ । \newline
52. सु॒व॒र्गस्येति॑ सुवः - गस्य॑ । \newline
53. लो॒कस्य॒ सम॑ष्ट्यै॒ सम॑ष्ट्यै लो॒कस्य॑ लो॒कस्य॒ सम॑ष्ट्यै । \newline
54. सम॑ष्ट्या आघा॒र मा॑घा॒रꣳ सम॑ष्ट्यै॒ सम॑ष्ट्या आघा॒रम् । \newline
55. सम॑ष्ट्या॒ इति॒ सं - अ॒ष्ट्यै॒ । \newline
56. आ॒घा॒र मा ऽऽघा॒र मा॑घा॒र मा । \newline
57. आ॒घा॒रमित्या᳚ - घा॒रम् । \newline
58. आ घा॑रयति घारय॒त्या घा॑रयति । \newline
59. घा॒र॒य॒ति॒ ति॒र स्ति॒रो घा॑रयति घारयति ति॒रः । \newline
60. ति॒र इ॑वे व ति॒र स्ति॒र इ॑व । \newline
61. इ॒व॒ वै वा इ॑वे व॒ वै । \newline

\textbf{Ghana Paata } \newline

1. इ॒मे लो॒का लो॒का इ॒म इ॒मे लो॒का इ॒मा नि॒मान् ॅलो॒का इ॒म इ॒मे लो॒का इ॒मान् । \newline
2. लो॒का इ॒मा नि॒मान् ॅलो॒का लो॒का इ॒मा ने॒वैवे मान् ॅलो॒का लो॒का इ॒मा ने॒व । \newline
3. इ॒मा ने॒वैवे मा नि॒मा ने॒व लो॒कान् ॅलो॒का ने॒वे मा नि॒मा ने॒व लो॒कान् । \newline
4. ए॒व लो॒कान् ॅलो॒का ने॒वैव लो॒का न॒भ्य॑भि लो॒का ने॒वैव लो॒का न॒भि । \newline
5. लो॒का न॒भ्य॑भि लो॒कान् ॅलो॒का न॒भि ज॑यति जयत्य॒भि लो॒कान् ॅलो॒का न॒भि ज॑यति । \newline
6. अ॒भि ज॑यति जय त्य॒भ्य॑भि ज॑यति॒ त्रिस्त्रिर् ज॑य त्य॒भ्य॑भि ज॑यति॒ त्रिः । \newline
7. ज॒य॒ति॒ त्रि स्त्रिर् ज॑यति जयति॒ त्रि रु॑त्तरा॒र्द्ध्य॑ मुत्तरा॒र्द्ध्य॑म् त्रिर् ज॑यति जयति॒ त्रि रु॑त्तरा॒र्द्ध्य᳚म् । \newline
8. त्रिरु॑त्तरा॒र्द्ध्य॑ मुत्तरा॒र्द्ध्य॑म् त्रि स्त्रि रु॑त्तरा॒र्द्ध्य॑म् त्रय॒स्त्रय॑ उत्तरा॒र्द्ध्य॑म् त्रि स्त्रि रु॑त्तरा॒र्द्ध्य॑म् त्रयः॑ । \newline
9. उ॒त्त॒रा॒र्द्ध्य॑म् त्रय॒ स्त्रय॑ उत्तरा॒र्द्ध्य॑ मुत्तरा॒र्द्ध्य॑म् त्रयो॒ वै वै त्रय॑ उत्तरा॒र्द्ध्य॑ मुत्तरा॒र्द्ध्य॑म् त्रयो॒ वै । \newline
10. उ॒त्त॒रा॒र्द्ध्य॑मित्यु॑त्तर - अ॒र्द्ध्य᳚म् । \newline
11. त्रयो॒ वै वै त्रय॒ स्त्रयो॒ वै दे॑व॒याना॑ देव॒याना॒ वै त्रय॒ स्त्रयो॒ वै दे॑व॒यानाः᳚ । \newline
12. वै दे॑व॒याना॑ देव॒याना॒ वै वै दे॑व॒यानाः॒ पन्था॑नः॒ पन्था॑नो देव॒याना॒ वै वै दे॑व॒यानाः॒ पन्था॑नः । \newline
13. दे॒व॒यानाः॒ पन्था॑नः॒ पन्था॑नो देव॒याना॑ देव॒यानाः॒ पन्था॑न॒ स्ताꣳ स्तान् पन्था॑नो देव॒याना॑ देव॒यानाः॒ पन्था॑न॒ स्तान् । \newline
14. दे॒व॒याना॒ इति॑ देव - यानाः᳚ । \newline
15. पन्था॑न॒ स्ताꣳ स्तान् पन्था॑नः॒ पन्था॑न॒ स्ता ने॒वैव तान् पन्था॑नः॒ पन्था॑न॒ स्ता ने॒व । \newline
16. ता ने॒वैव ताꣳ स्ता ने॒वाभ्या᳚(1॒)भ्ये॑व ताꣳ स्ता ने॒वाभि । \newline
17. ए॒वाभ्या᳚(1॒)भ्ये॑ वैवाभि ज॑यति जय त्य॒भ्ये॑ वैवाभि ज॑यति । \newline
18. अ॒भि ज॑यति जय त्य॒भ्य॑भि ज॑यति॒ त्रि स्त्रिर् ज॑य त्य॒भ्य॑भि ज॑यति॒ त्रिः । \newline
19. ज॒य॒ति॒ त्रि स्त्रिर् ज॑यति जयति॒ त्रिरुपोप॒ त्रिर् ज॑यति जयति॒ त्रिरुप॑ । \newline
20. त्रिरुपोप॒ त्रि स्त्रिरुप॑ वाजयति वाजय॒ त्युप॒ त्रि स्त्रिरुप॑ वाजयति । \newline
21. उप॑ वाजयति वाजय॒ त्युपोप॑ वाजयति॒ त्रय॒ स्त्रयो॑ वाजय॒ त्युपोप॑ वाजयति॒ त्रयः॑ । \newline
22. वा॒ज॒य॒ति॒ त्रय॒ स्त्रयो॑ वाजयति वाजयति॒ त्रयो॒ वै वै त्रयो॑ वाजयति वाजयति॒ त्रयो॒ वै । \newline
23. त्रयो॒ वै वै त्रय॒ स्त्रयो॒ वै दे॑वलो॒का दे॑वलो॒का वै त्रय॒ स्त्रयो॒ वै दे॑वलो॒काः । \newline
24. वै दे॑वलो॒का दे॑वलो॒का वै वै दे॑वलो॒का दे॑वलो॒कान् दे॑वलो॒कान् दे॑वलो॒का वै वै दे॑वलो॒का दे॑वलो॒कान् । \newline
25. दे॒व॒लो॒का दे॑वलो॒कान् दे॑वलो॒कान् दे॑वलो॒का दे॑वलो॒का दे॑वलो॒का ने॒वैव दे॑वलो॒कान् दे॑वलो॒का दे॑वलो॒का दे॑वलो॒का ने॒व । \newline
26. दे॒व॒लो॒का इति॑ देव - लो॒काः । \newline
27. दे॒व॒लो॒का ने॒वैव दे॑वलो॒कान् दे॑वलो॒का ने॒वाभ्या᳚(1॒)भ्ये॑व दे॑वलो॒कान् दे॑वलो॒का ने॒वाभि । \newline
28. दे॒व॒लो॒कानिति॑ देव - लो॒कान् । \newline
29. ए॒वाभ्या᳚(1॒)भ्ये॑ वैवाभि ज॑यति जय त्य॒भ्ये॑ वैवाभि ज॑यति । \newline
30. अ॒भि ज॑यति जय त्य॒भ्य॑भि ज॑यति॒ द्वाद॑श॒ द्वाद॑श जय त्य॒भ्य॑भि ज॑यति॒ द्वाद॑श । \newline
31. ज॒य॒ति॒ द्वाद॑श॒ द्वाद॑श जयति जयति॒ द्वाद॑श॒ सꣳ सम् द्वाद॑श जयति जयति॒ द्वाद॑श॒ सम् । \newline
32. द्वाद॑श॒ सꣳ सम् द्वाद॑श॒ द्वाद॑श॒ सम् प॑द्यन्ते पद्यन्ते॒ सम् द्वाद॑श॒ द्वाद॑श॒ सम् प॑द्यन्ते । \newline
33. सम् प॑द्यन्ते पद्यन्ते॒ सꣳ सम् प॑द्यन्ते॒ द्वाद॑श॒ द्वाद॑श पद्यन्ते॒ सꣳ सम् प॑द्यन्ते॒ द्वाद॑श । \newline
34. प॒द्य॒न्ते॒ द्वाद॑श॒ द्वाद॑श पद्यन्ते पद्यन्ते॒ द्वाद॑श॒ मासा॒ मासा॒ द्वाद॑श पद्यन्ते पद्यन्ते॒ द्वाद॑श॒ मासाः᳚ । \newline
35. द्वाद॑श॒ मासा॒ मासा॒ द्वाद॑श॒ द्वाद॑श॒ मासाः᳚ संॅवथ्स॒रः सं॑ॅवथ्स॒रो मासा॒ द्वाद॑श॒ द्वाद॑श॒ मासाः᳚ संॅवथ्स॒रः । \newline
36. मासाः᳚ संॅवथ्स॒रः सं॑ॅवथ्स॒रो मासा॒ मासाः᳚ संॅवथ्स॒रः सं॑ॅवथ्स॒रꣳ सं॑ॅवथ्स॒रꣳ सं॑ॅवथ्स॒रो मासा॒ मासाः᳚ संॅवथ्स॒रः सं॑ॅवथ्स॒रम् । \newline
37. सं॒ॅव॒थ्स॒रः सं॑ॅवथ्स॒रꣳ सं॑ॅवथ्स॒रꣳ सं॑ॅवथ्स॒रः सं॑ॅवथ्स॒रः सं॑ॅवथ्स॒र मे॒वैव सं॑ॅवथ्स॒रꣳ सं॑ॅवथ्स॒रः सं॑ॅवथ्स॒रः सं॑ॅवथ्स॒र मे॒व । \newline
38. सं॒ॅव॒थ्स॒र इति॑ सं - व॒थ्स॒रः । \newline
39. सं॒ॅव॒थ्स॒र मे॒वैव सं॑ॅवथ्स॒रꣳ सं॑ॅवथ्स॒र मे॒व प्री॑णाति प्रीणात्ये॒व सं॑ॅवथ्स॒रꣳ सं॑ॅवथ्स॒र मे॒व प्री॑णाति । \newline
40. सं॒ॅव॒थ्स॒रमिति॑ सं - व॒थ्स॒रम् । \newline
41. ए॒व प्री॑णाति प्रीणा त्ये॒वैव प्री॑णा॒त्यथो॒ अथो᳚ प्रीणा त्ये॒वैव प्री॑णा॒ त्यथो᳚ । \newline
42. प्री॒णा॒ त्यथो॒ अथो᳚ प्रीणाति प्रीणा॒ त्यथो॑ संॅवथ्स॒रꣳ सं॑ॅवथ्स॒र मथो᳚ प्रीणाति प्रीणा॒ त्यथो॑ संॅवथ्स॒रम् । \newline
43. अथो॑ संॅवथ्स॒रꣳ सं॑ॅवथ्स॒र मथो॒ अथो॑ संॅवथ्स॒र मे॒वैव सं॑ॅवथ्स॒र मथो॒ अथो॑ संॅवथ्स॒र मे॒व । \newline
44. अथो॒ इत्यथो᳚ । \newline
45. सं॒ॅव॒थ्स॒र मे॒वैव सं॑ॅवथ्स॒रꣳ सं॑ॅवथ्स॒र मे॒वास्मा॑ अस्मा ए॒व सं॑ॅवथ्स॒रꣳ सं॑ॅवथ्स॒र मे॒वास्मै᳚ । \newline
46. सं॒ॅव॒थ्स॒रमिति॑ सं - व॒थ्स॒रम् । \newline
47. ए॒वास्मा॑ अस्मा ए॒वैवास्मा॒ उपोपा᳚स्मा ए॒वैवास्मा॒ उप॑ । \newline
48. अ॒स्मा॒ उपोपा᳚स्मा अस्मा॒ उप॑ दधाति दधा॒ त्युपा᳚स्मा अस्मा॒ उप॑ दधाति । \newline
49. उप॑ दधाति दधा॒ त्युपोप॑ दधाति सुव॒र्गस्य॑ सुव॒र्गस्य॑ दधा॒ त्युपोप॑ दधाति सुव॒र्गस्य॑ । \newline
50. द॒धा॒ति॒ सु॒व॒र्गस्य॑ सुव॒र्गस्य॑ दधाति दधाति सुव॒र्गस्य॑ लो॒कस्य॑ लो॒कस्य॑ सुव॒र्गस्य॑ दधाति दधाति सुव॒र्गस्य॑ लो॒कस्य॑ । \newline
51. सु॒व॒र्गस्य॑ लो॒कस्य॑ लो॒कस्य॑ सुव॒र्गस्य॑ सुव॒र्गस्य॑ लो॒कस्य॒ सम॑ष्ट्यै॒ सम॑ष्ट्यै लो॒कस्य॑ सुव॒र्गस्य॑ सुव॒र्गस्य॑ लो॒कस्य॒ सम॑ष्ट्यै । \newline
52. सु॒व॒र्गस्येति॑ सुवः - गस्य॑ । \newline
53. लो॒कस्य॒ सम॑ष्ट्यै॒ सम॑ष्ट्यै लो॒कस्य॑ लो॒कस्य॒ सम॑ष्ट्या आघा॒र मा॑घा॒रꣳ सम॑ष्ट्यै लो॒कस्य॑ लो॒कस्य॒ सम॑ष्ट्या आघा॒रम् । \newline
54. सम॑ष्ट्या आघा॒र मा॑घा॒रꣳ सम॑ष्ट्यै॒ सम॑ष्ट्या आघा॒र मा ऽऽघा॒रꣳ सम॑ष्ट्यै॒ सम॑ष्ट्या आघा॒र मा । \newline
55. सम॑ष्ट्या॒ इति॒ सं - अ॒ष्ट्यै॒ । \newline
56. आ॒घा॒र मा ऽऽघा॒र मा॑घा॒र मा घा॑रयति घारय॒त्या ऽऽघा॒र मा॑घा॒र मा घा॑रयति । \newline
57. आ॒घा॒रमित्या᳚ - घा॒रम् । \newline
58. आ घा॑रयति घारय॒त्या घा॑रयति ति॒र स्ति॒रो घा॑रय॒त्या घा॑रयति ति॒रः । \newline
59. घा॒र॒य॒ति॒ ति॒र स्ति॒रो घा॑रयति घारयति ति॒र इ॑वे व ति॒रो घा॑रयति घारयति ति॒र इ॑व । \newline
60. ति॒र इ॑वे व ति॒र स्ति॒र इ॑व॒ वै वा इ॑व ति॒र स्ति॒र इ॑व॒ वै । \newline
61. इ॒व॒ वै वा इ॑वे व॒ वै सु॑व॒र्गः सु॑व॒र्गो वा इ॑वे व॒ वै सु॑व॒र्गः । \newline
\pagebreak
\markright{ TS 2.5.11.7  \hfill https://www.vedavms.in \hfill}
\addcontentsline{toc}{section}{ TS 2.5.11.7 }
\section*{ TS 2.5.11.7 }

\textbf{TS 2.5.11.7 } \newline
\textbf{Samhita Paata} \newline

वै सु॑व॒र्गो लो॒कः सु॑व॒र्गमे॒वास्मै॑ लो॒कं प्ररो॑चयत्यृ॒जुमा घा॑रयत्यृ॒जुरि॑व॒ हि प्रा॒णः संत॑त॒मा घा॑रयति प्रा॒णाना॑म॒न्नाद्य॑स्य॒ संत॑त्या॒ अथो॒ रक्ष॑सा॒मप॑हत्यै॒ यं का॒मये॑त प्र॒मायु॑कः स्या॒दिति॑ जि॒ह्मं तस्या ऽऽघा॑रयेत् प्रा॒णमे॒वास्मा᳚ज्जि॒ह्मं न॑यति ता॒जक् प्रमी॑यते॒शिरो॒ वा ए॒तद्-य॒ज्ञ्स्य॒ यदा॑घा॒र आ॒त्मा ध्रु॒वा - [  ] \newline

\textbf{Pada Paata} \newline

वै । सु॒व॒र्ग इति॑ सुवः - गः । लो॒कः । सु॒व॒र्गमिति॑ सुवः - गम् । ए॒व । अ॒स्मै॒ । लो॒कम् । प्रेति॑ । रो॒च॒य॒ति॒ । ऋ॒जुम् । एति॑ । घा॒र॒य॒ति॒ । ऋ॒जुः । इ॒व॒ । हि । प्रा॒ण इति॑ प्र - अ॒नः । संत॑त॒मिति॒ सं - त॒त॒म् । एति॑ । घा॒र॒य॒ति॒ । प्रा॒णाना॒मिति॑ प्र - अ॒नाना᳚म् । अ॒न्नाद्य॒स्येत्य॑न्न - अद्य॑स्य । संत॑त्या॒ इति॒ सं - त॒त्यै॒ । अथो॒ इति॑ । रक्ष॑साम् । अप॑हत्या॒ इत्यप॑ - ह॒त्यै॒ । यम् । का॒मये॑त । प्र॒मायु॑क॒ इति॑ प्र - मायु॑कः । स्या॒त् । इति॑ । जि॒ह्मम् । तस्य॑ । एति॑ । घा॒र॒ये॒त् । प्रा॒णमिति॑ प्र-अ॒नम् । ए॒व । अ॒स्मा॒त् । जि॒ह्मम् । न॒य॒ति॒ । ता॒जक् । प्रेति॑ । मी॒य॒ते॒ । शिरः॑ । वै । ए॒तत् । य॒ज्ञ्स्य॑ । यत् । आ॒घा॒र इत्या᳚ - घा॒रः । आ॒त्मा । ध्रु॒वा ।  \newline


\textbf{Krama Paata} \newline

वै सु॑व॒र्गः । सु॒व॒र्गो लो॒कः । सु॒व॒र्ग इति॑ सुवः - गः । लो॒कः सु॑व॒र्गम् । सु॒व॒र्गमे॒व । सु॒व॒र्गमिति॑ सुवः - गम् । ए॒वास्मै᳚ । अ॒स्मै॒ लो॒कम् । लो॒कम् प्र । प्र रो॑चयति । रो॒च॒य॒त्यृ॒जुम् । ऋ॒जुमा । आ घा॑रयति । घा॒र॒य॒त्यृ॒जुः । ऋ॒जुरि॑व । इ॒व॒ हि । हि प्रा॒णः । प्रा॒णः सन्त॑तम् । प्रा॒ण इति॑ प्र - अ॒नः । सन्त॑त॒मा । सन्त॑त॒मिति॒ सम् - त॒त॒म् । आ घा॑रयति । घा॒र॒य॒ति॒ प्रा॒णाना᳚म् । प्रा॒णाना॑म॒न्नाद्य॑स्य । प्रा॒णाना॒मिति॑ प्र - अ॒नाना᳚म् । अ॒न्नाद्य॑स्य॒ सन्त॑त्यै । अ॒न्नाद्य॒स्येत्य॑न्न - अद्य॑स्य । सन्त॑त्या॒ अथो᳚ । सन्त॑त्या॒ इति॒ सम् - त॒त्यै॒ । अथो॒ रक्ष॑साम् । अथो॒ इत्यथो᳚ । रक्ष॑सा॒मप॑हत्यै । अप॑हत्यै॒ यम् । अप॑हत्या॒ इत्यप॑ - ह॒त्यै॒ । यम् का॒मये॑त । का॒मये॑त प्र॒मायु॑कः । प्र॒मायु॑कः स्यात् । प्र॒मायु॑क॒ इति॑ प्र - मायु॑कः । स्या॒दिति॑ । इति॑ जि॒ह्मम् । जि॒ह्मम् तस्य॑ । तस्या । आ घा॑रयेत् । घा॒र॒ये॒त् प्रा॒णम् । प्रा॒णमे॒व । प्रा॒णमिति॑ प्र - अ॒नम् । ए॒वास्मा᳚त् । अ॒स्मा॒ज् जि॒ह्मम् । जि॒ह्मम् न॑यति । न॒य॒ति॒ ता॒जक् । ता॒जक् प्र । प्र मी॑यते । मी॒य॒ते॒ शिरः॑ । शिरो॒ वै । वा ए॒तत् । ए॒तद् य॒ज्ञ्स्य॑ । य॒ज्ञ्स्य॒ यत् । यदा॑घा॒रः । आ॒घा॒र आ॒त्मा । आ॒घा॒र इत्या᳚ - घा॒रः । आ॒त्मा ध्रु॒वा । ध्रु॒वा ऽऽघा॒रम् \newline

\textbf{Jatai Paata} \newline

1. वै सु॑व॒र्गः सु॑व॒र्गो वै वै सु॑व॒र्गः । \newline
2. सु॒व॒र्गो लो॒को लो॒कः सु॑व॒र्गः सु॑व॒र्गो लो॒कः । \newline
3. सु॒व॒र्ग इति॑ सुवः - गः । \newline
4. लो॒कः सु॑व॒र्गꣳ सु॑व॒र्गम् ॅलो॒को लो॒कः सु॑व॒र्गम् । \newline
5. सु॒व॒र्ग मे॒वैव सु॑व॒र्गꣳ सु॑व॒र्ग मे॒व । \newline
6. सु॒व॒र्गमिति॑ सुवः - गम् । \newline
7. ए॒वास्मा॑ अस्मा ए॒वैवास्मै᳚ । \newline
8. अ॒स्मै॒ लो॒कम् ॅलो॒क म॑स्मा अस्मै लो॒कम् । \newline
9. लो॒कम् प्र प्र लो॒कम् ॅलो॒कम् प्र । \newline
10. प्र रो॑चयति रोचयति॒ प्र प्र रो॑चयति । \newline
11. रो॒च॒य॒ त्यृ॒जु मृ॒जुꣳ रो॑चयति रोचय त्यृ॒जुम् । \newline
12. ऋ॒जु मार्जु मृ॒जु मा । \newline
13. आ घा॑रयति घारय॒त्या घा॑रयति । \newline
14. घा॒र॒य॒ त्यृ॒जुर्. ऋ॒जुर् घा॑रयति घारय त्यृ॒जुः । \newline
15. ऋ॒जुरि॑वे व॒ र्जुर्. ऋ॒जुरि॑व । \newline
16. इ॒व॒ हि हीवे॑ व॒ हि । \newline
17. हि प्रा॒णः प्रा॒णो हि हि प्रा॒णः । \newline
18. प्रा॒णः सन्त॑तꣳ॒॒ सन्त॑तम् प्रा॒णः प्रा॒णः सन्त॑तम् । \newline
19. प्रा॒ण इति॑ प्र - अ॒नः । \newline
20. सन्त॑त॒ मा सन्त॑तꣳ॒॒ सन्त॑त॒ मा । \newline
21. सन्त॑त॒मिति॒ सं - त॒त॒म् । \newline
22. आ घा॑रयति घारय॒त्या घा॑रयति । \newline
23. घा॒र॒य॒ति॒ प्रा॒णाना᳚म् प्रा॒णाना᳚म् घारयति घारयति प्रा॒णाना᳚म् । \newline
24. प्रा॒णाना॑ म॒न्नाद्य॑स्या॒ न्नाद्य॑स्य प्रा॒णाना᳚म् प्रा॒णाना॑ म॒न्नाद्य॑स्य । \newline
25. प्रा॒णाना॒मिति॑ प्र - अ॒नाना᳚म् । \newline
26. अ॒न्नाद्य॑स्य॒ सन्त॑त्यै॒ सन्त॑त्या अ॒न्नाद्य॑स्या॒ न्नाद्य॑स्य॒ सन्त॑त्यै । \newline
27. अ॒न्नाद्य॒स्येत्य॑न्न - अद्य॑स्य । \newline
28. सन्त॑त्या॒ अथो॒ अथो॒ सन्त॑त्यै॒ सन्त॑त्या॒ अथो᳚ । \newline
29. सन्त॑त्या॒ इति॒ सं - त॒त्यै॒ । \newline
30. अथो॒ रक्ष॑साꣳ॒॒ रक्ष॑सा॒ मथो॒ अथो॒ रक्ष॑साम् । \newline
31. अथो॒ इत्यथो᳚ । \newline
32. रक्ष॑सा॒ मप॑हत्या॒ अप॑हत्यै॒ रक्ष॑साꣳ॒॒ रक्ष॑सा॒ मप॑हत्यै । \newline
33. अप॑हत्यै॒ यं ॅय मप॑हत्या॒ अप॑हत्यै॒ यम् । \newline
34. अप॑हत्या॒ इत्यप॑ - ह॒त्यै॒ । \newline
35. यम् का॒मये॑त का॒मये॑त॒ यं ॅयम् का॒मये॑त । \newline
36. का॒मये॑त प्र॒मायु॑कः प्र॒मायु॑कः का॒मये॑त का॒मये॑त प्र॒मायु॑कः । \newline
37. प्र॒मायु॑कः स्याथ् स्यात् प्र॒मायु॑कः प्र॒मायु॑कः स्यात् । \newline
38. प्र॒मायु॑क॒ इति॑ प्र - मायु॑कः । \newline
39. स्या॒ दितीति॑ स्याथ् स्या॒दिति॑ । \newline
40. इति॑ जि॒ह्मम् जि॒ह्म मितीति॑ जि॒ह्मम् । \newline
41. जि॒ह्मम् तस्य॒ तस्य॑ जि॒ह्मम् जि॒ह्मम् तस्य॑ । \newline
42. तस्या तस्य॒ तस्या । \newline
43. आ घा॑रयेद् घारये॒दा घा॑रयेत् । \newline
44. घा॒र॒ये॒त् प्रा॒णम् प्रा॒णम् घा॑रयेद् घारयेत् प्रा॒णम् । \newline
45. प्रा॒ण मे॒वैव प्रा॒णम् प्रा॒ण मे॒व । \newline
46. प्रा॒णमिति॑ प्र - अ॒नम् । \newline
47. ए॒वास्मा॑ दस्मा दे॒वैवास्मा᳚त् । \newline
48. अ॒स्मा॒ज् जि॒ह्मम् जि॒ह्म म॑स्मा दस्माज् जि॒ह्मम् । \newline
49. जि॒ह्मम् न॑यति नयति जि॒ह्मम् जि॒ह्मम् न॑यति । \newline
50. न॒य॒ति॒ ता॒जक् ता॒जङ् न॑यति नयति ता॒जक् । \newline
51. ता॒जक् प्र प्र ता॒जक् ता॒जक् प्र । \newline
52. प्र मी॑यते मीयते॒ प्र प्र मी॑यते । \newline
53. मी॒य॒ते॒ शिरः॒ शिरो॑ मीयते मीयते॒ शिरः॑ । \newline
54. शिरो॒ वै वै शिरः॒ शिरो॒ वै । \newline
55. वा ए॒त दे॒तद् वै वा ए॒तत् । \newline
56. ए॒तद् य॒ज्ञ्स्य॑ य॒ज्ञ्स्यै॒त दे॒तद् य॒ज्ञ्स्य॑ । \newline
57. य॒ज्ञ्स्य॒ यद् यद् य॒ज्ञ्स्य॑ य॒ज्ञ्स्य॒ यत् । \newline
58. यदा॑घा॒र आ॑घा॒रो यद् यदा॑घा॒रः । \newline
59. आ॒घा॒र आ॒त्मा ऽऽत्मा ऽऽघा॒र आ॑घा॒र आ॒त्मा । \newline
60. आ॒घा॒र इत्या᳚ - घा॒रः । \newline
61. आ॒त्मा ध्रु॒वा ध्रु॒वा ऽऽत्मा ऽऽत्मा ध्रु॒वा । \newline
62. ध्रु॒वा ऽऽघा॒र मा॑घा॒रम् ध्रु॒वा ध्रु॒वा ऽऽघा॒रम् । \newline

\textbf{Ghana Paata } \newline

1. वै सु॑व॒र्गः सु॑व॒र्गो वै वै सु॑व॒र्गो लो॒को लो॒कः सु॑व॒र्गो वै वै सु॑व॒र्गो लो॒कः । \newline
2. सु॒व॒र्गो लो॒को लो॒कः सु॑व॒र्गः सु॑व॒र्गो लो॒कः सु॑व॒र्गꣳ सु॑व॒र्गम् ॅलो॒कः सु॑व॒र्गः सु॑व॒र्गो लो॒कः सु॑व॒र्गम् । \newline
3. सु॒व॒र्ग इति॑ सुवः - गः । \newline
4. लो॒कः सु॑व॒र्गꣳ सु॑व॒र्गम् ॅलो॒को लो॒कः सु॑व॒र्ग मे॒वैव सु॑व॒र्गम् ॅलो॒को लो॒कः सु॑व॒र्ग मे॒व । \newline
5. सु॒व॒र्ग मे॒वैव सु॑व॒र्गꣳ सु॑व॒र्ग मे॒वास्मा॑ अस्मा ए॒व सु॑व॒र्गꣳ सु॑व॒र्ग मे॒वास्मै᳚ । \newline
6. सु॒व॒र्गमिति॑ सुवः - गम् । \newline
7. ए॒वास्मा॑ अस्मा ए॒वैवास्मै॑ लो॒कम् ॅलो॒क म॑स्मा ए॒वैवास्मै॑ लो॒कम् । \newline
8. अ॒स्मै॒ लो॒कम् ॅलो॒क म॑स्मा अस्मै लो॒कम् प्र प्र लो॒क म॑स्मा अस्मै लो॒कम् प्र । \newline
9. लो॒कम् प्र प्र लो॒कम् ॅलो॒कम् प्र रो॑चयति रोचयति॒ प्र लो॒कम् ॅलो॒कम् प्र रो॑चयति । \newline
10. प्र रो॑चयति रोचयति॒ प्र प्र रो॑चय त्यृ॒जु मृ॒जुꣳ रो॑चयति॒ प्र प्र रो॑चय त्यृ॒जुम् । \newline
11. रो॒च॒य॒ त्यृ॒जु मृ॒जुꣳ रो॑चयति रोचय त्यृ॒जु मार्जुꣳ रो॑चयति रोचय त्यृ॒जु मा । \newline
12. ऋ॒जु मार्जु मृ॒जु मा घा॑रयति घारय त्या॒र्जु मृ॒जु मा घा॑रयति । \newline
13. आ घा॑रयति घारय॒त्या घा॑रय त्यृ॒जुर्. ऋ॒जुर् घा॑रय॒त्या घा॑रय त्यृ॒जुः । \newline
14. घा॒र॒य॒ त्यृ॒जुर्. ऋ॒जुर् घा॑रयति घारय त्यृ॒जुरि॑वे व॒ र्जुर् घा॑रयति घारय त्यृ॒जुरि॑व । \newline
15. ऋ॒जुरि॑वे व॒ र्जुर्. ऋ॒जुरि॑व॒ हि हीव॒ र्जुर्. ऋ॒जुरि॑व॒ हि । \newline
16. इ॒व॒ हि हीवे॑ व॒ हि प्रा॒णः प्रा॒णो हीवे॑ व॒ हि प्रा॒णः । \newline
17. हि प्रा॒णः प्रा॒णो हि हि प्रा॒णः सन्त॑तꣳ॒॒ सन्त॑तम् प्रा॒णो हि हि प्रा॒णः सन्त॑तम् । \newline
18. प्रा॒णः सन्त॑तꣳ॒॒ सन्त॑तम् प्रा॒णः प्रा॒णः सन्त॑त॒ मा सन्त॑तम् प्रा॒णः प्रा॒णः सन्त॑त॒ मा । \newline
19. प्रा॒ण इति॑ प्र - अ॒नः । \newline
20. सन्त॑त॒ मा सन्त॑तꣳ॒॒ सन्त॑त॒ मा घा॑रयति घारय॒त्या सन्त॑तꣳ॒॒ सन्त॑त॒ मा घा॑रयति । \newline
21. सन्त॑त॒मिति॒ सं - त॒त॒म् । \newline
22. आ घा॑रयति घारय॒त्या घा॑रयति प्रा॒णाना᳚म् प्रा॒णाना᳚म् घारय॒त्या घा॑रयति प्रा॒णाना᳚म् । \newline
23. घा॒र॒य॒ति॒ प्रा॒णाना᳚म् प्रा॒णाना᳚म् घारयति घारयति प्रा॒णाना॑ म॒न्नाद्य॑स्या॒ न्नाद्य॑स्य प्रा॒णाना᳚म् घारयति घारयति प्रा॒णाना॑ म॒न्नाद्य॑स्य । \newline
24. प्रा॒णाना॑ म॒न्नाद्य॑स्या॒ न्नाद्य॑स्य प्रा॒णाना᳚म् प्रा॒णाना॑ म॒न्नाद्य॑स्य॒ सन्त॑त्यै॒ सन्त॑त्या अ॒न्नाद्य॑स्य प्रा॒णाना᳚म् प्रा॒णाना॑ म॒न्नाद्य॑स्य॒ सन्त॑त्यै । \newline
25. प्रा॒णाना॒मिति॑ प्र - अ॒नाना᳚म् । \newline
26. अ॒न्नाद्य॑स्य॒ सन्त॑त्यै॒ सन्त॑त्या अ॒न्नाद्य॑स्या॒ न्नाद्य॑स्य॒ सन्त॑त्या॒ अथो॒ अथो॒ सन्त॑त्या अ॒न्नाद्य॑स्या॒ न्नाद्य॑स्य॒ सन्त॑त्या॒ अथो᳚ । \newline
27. अ॒न्नाद्य॒स्येत्य॑न्न - अद्य॑स्य । \newline
28. सन्त॑त्या॒ अथो॒ अथो॒ सन्त॑त्यै॒ सन्त॑त्या॒ अथो॒ रक्ष॑साꣳ॒॒ रक्ष॑सा॒ मथो॒ सन्त॑त्यै॒ सन्त॑त्या॒ अथो॒ रक्ष॑साम् । \newline
29. सन्त॑त्या॒ इति॒ सं - त॒त्यै॒ । \newline
30. अथो॒ रक्ष॑साꣳ॒॒ रक्ष॑सा॒ मथो॒ अथो॒ रक्ष॑सा॒ मप॑हत्या॒ अप॑हत्यै॒ रक्ष॑सा॒ मथो॒ अथो॒ रक्ष॑सा॒ मप॑हत्यै । \newline
31. अथो॒ इत्यथो᳚ । \newline
32. रक्ष॑सा॒ मप॑हत्या॒ अप॑हत्यै॒ रक्ष॑साꣳ॒॒ रक्ष॑सा॒ मप॑हत्यै॒ यं ॅय मप॑हत्यै॒ रक्ष॑साꣳ॒॒ रक्ष॑सा॒ मप॑हत्यै॒ यम् । \newline
33. अप॑हत्यै॒ यं ॅय मप॑हत्या॒ अप॑हत्यै॒ यम् का॒मये॑त का॒मये॑त॒ य मप॑हत्या॒ अप॑हत्यै॒ यम् का॒मये॑त । \newline
34. अप॑हत्या॒ इत्यप॑ - ह॒त्यै॒ । \newline
35. यम् का॒मये॑त का॒मये॑त॒ यं ॅयम् का॒मये॑त प्र॒मायु॑कः प्र॒मायु॑कः का॒मये॑त॒ यं ॅयम् का॒मये॑त प्र॒मायु॑कः । \newline
36. का॒मये॑त प्र॒मायु॑कः प्र॒मायु॑कः का॒मये॑त का॒मये॑त प्र॒मायु॑कः स्याथ् स्यात् प्र॒मायु॑कः का॒मये॑त का॒मये॑त प्र॒मायु॑कः स्यात् । \newline
37. प्र॒मायु॑कः स्याथ् स्यात् प्र॒मायु॑कः प्र॒मायु॑कः स्या॒ दितीति॑ स्यात् प्र॒मायु॑कः प्र॒मायु॑कः स्या॒दिति॑ । \newline
38. प्र॒मायु॑क॒ इति॑ प्र - मायु॑कः । \newline
39. स्या॒ दितीति॑ स्याथ् स्या॒दिति॑ जि॒ह्मम् जि॒ह्म मिति॑ स्याथ् स्या॒दिति॑ जि॒ह्मम् । \newline
40. इति॑ जि॒ह्मम् जि॒ह्म मितीति॑ जि॒ह्मम् तस्य॒ तस्य॑ जि॒ह्म मितीति॑ जि॒ह्मम् तस्य॑ । \newline
41. जि॒ह्मम् तस्य॒ तस्य॑ जि॒ह्मम् जि॒ह्मम् तस्या तस्य॑ जि॒ह्मम् जि॒ह्मम् तस्या । \newline
42. तस्या तस्य॒ तस्या घा॑रयेद् घारये॒दा तस्य॒ तस्या घा॑रयेत् । \newline
43. आ घा॑रयेद् घारये॒दा घा॑रयेत् प्रा॒णम् प्रा॒णम् घा॑रये॒दा घा॑रयेत् प्रा॒णम् । \newline
44. घा॒र॒ये॒त् प्रा॒णम् प्रा॒णम् घा॑रयेद् घारयेत् प्रा॒ण मे॒वैव प्रा॒णम् घा॑रयेद् घारयेत् प्रा॒ण मे॒व । \newline
45. प्रा॒ण मे॒वैव प्रा॒णम् प्रा॒ण मे॒वास्मा॑ दस्मादे॒व प्रा॒णम् प्रा॒ण मे॒वास्मा᳚त् । \newline
46. प्रा॒णमिति॑ प्र - अ॒नम् । \newline
47. ए॒वास्मा॑ दस्मा दे॒वैवास्मा᳚ज् जि॒ह्मम् जि॒ह्म म॑स्मा दे॒वैवास्मा᳚ज् जि॒ह्मम् । \newline
48. अ॒स्मा॒ज् जि॒ह्मम् जि॒ह्म म॑स्मा दस्माज् जि॒ह्मम् न॑यति नयति जि॒ह्म म॑स्मा दस्माज् जि॒ह्मम् न॑यति । \newline
49. जि॒ह्मम् न॑यति नयति जि॒ह्मम् जि॒ह्मम् न॑यति ता॒जक् ता॒जङ् न॑यति जि॒ह्मम् जि॒ह्मम् न॑यति ता॒जक् । \newline
50. न॒य॒ति॒ ता॒जक् ता॒जङ् न॑यति नयति ता॒जक् प्र प्र ता॒जङ् न॑यति नयति ता॒जक् प्र । \newline
51. ता॒जक् प्र प्र ता॒जक् ता॒जक् प्र मी॑यते मीयते॒ प्र ता॒जक् ता॒जक् प्र मी॑यते । \newline
52. प्र मी॑यते मीयते॒ प्र प्र मी॑यते॒ शिरः॒ शिरो॑ मीयते॒ प्र प्र मी॑यते॒ शिरः॑ । \newline
53. मी॒य॒ते॒ शिरः॒ शिरो॑ मीयते मीयते॒ शिरो॒ वै वै शिरो॑ मीयते मीयते॒ शिरो॒ वै । \newline
54. शिरो॒ वै वै शिरः॒ शिरो॒ वा ए॒त दे॒तद् वै शिरः॒ शिरो॒ वा ए॒तत् । \newline
55. वा ए॒त दे॒तद् वै वा ए॒तद् य॒ज्ञ्स्य॑ य॒ज्ञ्स्यै॒तद् वै वा ए॒तद् य॒ज्ञ्स्य॑ । \newline
56. ए॒तद् य॒ज्ञ्स्य॑ य॒ज्ञ्स्यै॒त दे॒तद् य॒ज्ञ्स्य॒ यद् यद् य॒ज्ञ्स्यै॒त दे॒तद् य॒ज्ञ्स्य॒ यत् । \newline
57. य॒ज्ञ्स्य॒ यद् यद् य॒ज्ञ्स्य॑ य॒ज्ञ्स्य॒ यदा॑घा॒र आ॑घा॒रो यद् य॒ज्ञ्स्य॑ य॒ज्ञ्स्य॒ यदा॑घा॒रः । \newline
58. यदा॑घा॒र आ॑घा॒रो यद् यदा॑घा॒र आ॒त्मा ऽऽत्मा ऽऽघा॒रो यद् यदा॑घा॒र आ॒त्मा । \newline
59. आ॒घा॒र आ॒त्मा ऽऽत्मा ऽऽघा॒र आ॑घा॒र आ॒त्मा ध्रु॒वा ध्रु॒वा ऽऽत्मा ऽऽघा॒र आ॑घा॒र आ॒त्मा ध्रु॒वा । \newline
60. आ॒घा॒र इत्या᳚ - घा॒रः । \newline
61. आ॒त्मा ध्रु॒वा ध्रु॒वा ऽऽत्मा ऽऽत्मा ध्रु॒वा ऽऽघा॒र मा॑घा॒रम् ध्रु॒वा ऽऽत्मा ऽऽत्मा ध्रु॒वा ऽऽघा॒रम् । \newline
62. ध्रु॒वा ऽऽघा॒र मा॑घा॒रम् ध्रु॒वा ध्रु॒वा ऽऽघा॒र मा॒घार्या॒ घार्या॑ घा॒रम् ध्रु॒वा ध्रु॒वा ऽऽघा॒र मा॒घार्य॑ । \newline
\pagebreak
\markright{ TS 2.5.11.8  \hfill https://www.vedavms.in \hfill}
\addcontentsline{toc}{section}{ TS 2.5.11.8 }
\section*{ TS 2.5.11.8 }

\textbf{TS 2.5.11.8 } \newline
\textbf{Samhita Paata} \newline

ऽऽघा॒रमा॒घार्य॑ ध्रु॒वाꣳ सम॑नक्त्या॒त्मन्ने॒व य॒ज्ञ्स्य॒ शिरः॒ प्रति॑ दधात्य॒ग्निर्दे॒वानां᳚ दू॒त आसी॒द् दैव्योऽसु॑राणां॒ तौ प्र॒जाप॑तिं प्र॒श्नमै॑ताꣳ॒॒ स प्र॒जाप॑ति र्ब्राह्म॒णम॑ब्रवीदे॒तद्वि ब्रू॒हीत्या श्रा॑व॒येती॒दं दे॑वाः शृणु॒तेति॒ वाव तद॑ब्रवीद॒ग्नि र्दे॒वो होतेति॒ य ए॒व दे॒वानां॒ तम॑वृणीत॒ ततो॑ दे॒वा - [  ] \newline

\textbf{Pada Paata} \newline

आ॒घा॒रमित्या᳚ - घा॒रम् । आ॒घार्येत्या᳚ - घार्य॑ । ध्रु॒वाम् । समिति॑ । अ॒न॒क्ति॒ । आ॒त्मन्न् । ए॒व । य॒ज्ञ्स्य॑ । शिरः॑ । प्रतीति॑ । द॒धा॒ति॒ । अ॒ग्निः । दे॒वाना᳚म् । दू॒तः । आसी᳚त् । दैव्यः॑ । असु॑राणाम् । तौ । प्र॒जाप॑ति॒मिति॑ प्र॒जा - प॒ति॒म् । प्र॒श्नम् । ऐ॒ता॒म् । सः । प्र॒जाप॑ति॒रिति॑ प्र॒जा - प॒तिः॒ । ब्रा॒ह्म॒णम् । अ॒ब्र॒वी॒त् । ए॒तत् । वीति॑ । ब्रू॒हि॒ । इति॑ । एति॑ । श्रा॒व॒य॒ । इति॑ । इ॒दम् । दे॒वाः॒ । शृ॒णु॒त॒ । इति॑ । वाव । तत् । अ॒ब्र॒वी॒त् । अ॒ग्निः । दे॒वः । होता᳚ । इति॑ । यः । ए॒व । दे॒वाना᳚म् । तम् । अ॒वृ॒णी॒त॒ । ततः॑ । दे॒वाः ।  \newline


\textbf{Krama Paata} \newline

आ॒घा॒रमा॒घार्य॑ । आ॒घा॒रमित्या᳚ - घा॒रम् । आ॒घार्य॑ ध्रु॒वाम् । आ॒घार्येत्या᳚ - घार्य॑ । ध्रु॒वाꣳ सम् । सम॑नक्ति । अ॒न॒क्त्या॒त्मन्न् । आ॒त्मन्ने॒व । ए॒व य॒ज्ञ्स्य॑ । य॒ज्ञ्स्य॒ शिरः॑ । शिरः॒ प्रति॑ । प्रति॑ दधाति । द॒धा॒त्य॒ग्निः । अ॒ग्निर् दे॒वाना᳚म् । दे॒वाना᳚म् दू॒तः । दू॒त आसी᳚त् । आसी॒द् दैव्यः॑ । दैव्यो ऽसु॑राणाम् । असु॑राणा॒म् तौ । तौ प्र॒जाप॑तिम् । प्र॒जाप॑तिम् प्र॒श्ञम् । प्र॒जाप॑ति॒मिति॑ प्र॒जा - प॒ति॒म् । प्र॒श्ञमै॑ताम् । ऐ॒ताꣳ॒॒ सः । स प्र॒जाप॑तिः । प्र॒जाप॑तिर् ब्राह्म॒णम् । प्र॒जाप॑ति॒रिति॑ प्र॒जा - प॒तिः॒ । ब्रा॒ह्म॒णम॑ब्रवीत् । अ॒ब्र॒वी॒दे॒तत् । ए॒तद् वि । वि ब्रू॑हि । ब्रू॒हीति॑ । इत्या । आ श्रा॑वय । श्रा॒व॒येति॑ । इती॒दम् । इ॒दम् दे॑वाः । दे॒वाः॒ शृ॒णु॒त॒ । शृ॒णु॒तेति॑ । इति॒ वाव । वाव तत् । तद॑ब्रवीत् । अ॒ब्र॒वी॒द॒ग्निः । अ॒ग्निर् दे॒वः । दे॒वो होता᳚ । होतेति॑ । इति॒ यः । य ए॒व । ए॒व दे॒वाना᳚म् । दे॒वाना॒म् तम् । तम॑वृणीत । अ॒वृ॒णी॒त॒ ततः॑ । ततो॑ दे॒वाः ( ) । दे॒वा अभ॑वन्न् \newline

\textbf{Jatai Paata} \newline

1. आ॒घा॒र मा॒घार्या॒ घार्या॑ घा॒र मा॑घा॒र मा॒घार्य॑ । \newline
2. आ॒घा॒रमित्या᳚ - घा॒रम् । \newline
3. आ॒घार्य॑ ध्रु॒वाम् ध्रु॒वा मा॒घार्या॒ घार्य॑ ध्रु॒वाम् । \newline
4. आ॒घार्येत्या᳚ - घार्य॑ । \newline
5. ध्रु॒वाꣳ सꣳ सम् ध्रु॒वाम् ध्रु॒वाꣳ सम् । \newline
6. स म॑नक्त्यनक्ति॒ सꣳ स म॑नक्ति । \newline
7. अ॒न॒क् त्या॒त्मन् ना॒त्मन् न॑नक् त्यनक् त्या॒त्मन्न् । \newline
8. आ॒त्मन् ने॒वैवात्मन् ना॒त्मन् ने॒व । \newline
9. ए॒व य॒ज्ञ्स्य॑ य॒ज्ञ्स्यै॒वैव य॒ज्ञ्स्य॑ । \newline
10. य॒ज्ञ्स्य॒ शिरः॒ शिरो॑ य॒ज्ञ्स्य॑ य॒ज्ञ्स्य॒ शिरः॑ । \newline
11. शिरः॒ प्रति॒ प्रति॒ शिरः॒ शिरः॒ प्रति॑ । \newline
12. प्रति॑ दधाति दधाति॒ प्रति॒ प्रति॑ दधाति । \newline
13. द॒धा॒ त्य॒ग्नि र॒ग्निर् द॑धाति दधा त्य॒ग्निः । \newline
14. अ॒ग्निर् दे॒वाना᳚म् दे॒वाना॑ म॒ग्नि र॒ग्निर् दे॒वाना᳚म् । \newline
15. दे॒वाना᳚म् दू॒तो दू॒तो दे॒वाना᳚म् दे॒वाना᳚म् दू॒तः । \newline
16. दू॒त आसी॒ दासी᳚द् दू॒तो दू॒त आसी᳚त् । \newline
17. आसी॒द् दैव्यो॒ दैव्य॒ आसी॒ दासी॒द् दैव्यः॑ । \newline
18. दैव्यो ऽसु॑राणा॒ मसु॑राणा॒म् दैव्यो॒ दैव्यो ऽसु॑राणाम् । \newline
19. असु॑राणा॒म् तौ ता वसु॑राणा॒ मसु॑राणा॒म् तौ । \newline
20. तौ प्र॒जाप॑तिम् प्र॒जाप॑ति॒म् तौ तौ प्र॒जाप॑तिम् । \newline
21. प्र॒जाप॑तिम् प्र॒श्ञम् प्र॒श्ञम् प्र॒जाप॑तिम् प्र॒जाप॑तिम् प्र॒श्ञम् । \newline
22. प्र॒जाप॑ति॒मिति॑ प्र॒जा - प॒ति॒म् । \newline
23. प्र॒श्ञ मै॑ता मैताम् प्र॒श्ञम् प्र॒श्ञ मै॑ताम् । \newline
24. ऐ॒ताꣳ॒॒ स स ऐ॑ता मैताꣳ॒॒ सः । \newline
25. स प्र॒जाप॑तिः प्र॒जाप॑तिः॒ स स प्र॒जाप॑तिः । \newline
26. प्र॒जाप॑तिर् ब्राह्म॒णम् ब्रा᳚ह्म॒णम् प्र॒जाप॑तिः प्र॒जाप॑तिर् ब्राह्म॒णम् । \newline
27. प्र॒जाप॑ति॒रिति॑ प्र॒जा - प॒तिः॒ । \newline
28. ब्रा॒ह्म॒ण म॑ब्रवी दब्रवीद् ब्राह्म॒णम् ब्रा᳚ह्म॒ण म॑ब्रवीत् । \newline
29. अ॒ब्र॒वी॒ दे॒त दे॒त द॑ब्रवी दब्रवी दे॒तत् । \newline
30. ए॒तद् वि व्ये॑त दे॒तद् वि । \newline
31. वि ब्रू॑हि ब्रूहि॒ वि वि ब्रू॑हि । \newline
32. ब्रू॒हीतीति॑ ब्रूहि ब्रू॒हीति॑ । \newline
33. इत्येतीत्या । \newline
34. आ श्रा॑वय श्राव॒या श्रा॑वय । \newline
35. श्रा॒व॒ये तीति॑ श्रावय श्राव॒ये ति॑ । \newline
36. इती॒द मि॒द मितीती॒दम् । \newline
37. इ॒दम् दे॑वा देवा इ॒द मि॒दम् दे॑वाः । \newline
38. दे॒वाः॒ शृ॒णु॒त॒ शृ॒णु॒त॒ दे॒वा॒ दे॒वाः॒ शृ॒णु॒त॒ । \newline
39. शृ॒णु॒ते तीति॑ शृणुत शृणु॒ते ति॑ । \newline
40. इति॒ वाव वावे तीति॒ वाव । \newline
41. वाव तत् तद् वाव वाव तत् । \newline
42. तद॑ब्रवी दब्रवी॒त् तत् तद॑ब्रवीत् । \newline
43. अ॒ब्र॒वी॒ द॒ग्नि र॒ग्नि र॑ब्रवी दब्रवी द॒ग्निः । \newline
44. अ॒ग्निर् दे॒वो दे॒वो᳚ ऽग्नि र॒ग्निर् दे॒वः । \newline
45. दे॒वो होता॒ होता॑ दे॒वो दे॒वो होता᳚ । \newline
46. होतेतीति॒ होता॒ होतेति॑ । \newline
47. इति॒ यो य इतीति॒ यः । \newline
48. य ए॒वैव यो य ए॒व । \newline
49. ए॒व दे॒वाना᳚म् दे॒वाना॑ मे॒वैव दे॒वाना᳚म् । \newline
50. दे॒वाना॒म् तम् तम् दे॒वाना᳚म् दे॒वाना॒म् तम् । \newline
51. त म॑वृणीता वृणीत॒ तम् त म॑वृणीत । \newline
52. अ॒वृ॒णी॒त॒ तत॒ स्ततो॑ ऽवृणीता वृणीत॒ ततः॑ । \newline
53. ततो॑ दे॒वा दे॒वा स्तत॒ स्ततो॑ दे॒वाः । \newline
54. दे॒वा अभ॑व॒न् नभ॑वन् दे॒वा दे॒वा अभ॑वन्न् । \newline

\textbf{Ghana Paata } \newline

1. आ॒घा॒र मा॒घार्या॒ घार्या॑ घा॒र मा॑घा॒र मा॒घार्य॑ ध्रु॒वाम् ध्रु॒वा मा॒घार्या॑ घा॒र मा॑घा॒र मा॒घार्य॑ ध्रु॒वाम् । \newline
2. आ॒घा॒रमित्या᳚ - घा॒रम् । \newline
3. आ॒घार्य॑ ध्रु॒वाम् ध्रु॒वा मा॒घार्या॒ घार्य॑ ध्रु॒वाꣳ सꣳ सम् ध्रु॒वा मा॒घार्या॒ घार्य॑ ध्रु॒वाꣳ सम् । \newline
4. आ॒घार्येत्या᳚ - घार्य॑ । \newline
5. ध्रु॒वाꣳ सꣳ सम् ध्रु॒वाम् ध्रु॒वाꣳ स म॑नक्त्यनक्ति॒ सम् ध्रु॒वाम् ध्रु॒वाꣳ स म॑नक्ति । \newline
6. स म॑नक्त्यनक्ति॒ सꣳ स म॑नक्त्या॒त्मन् ना॒त्मन् न॑नक्ति॒ सꣳ स म॑नक्त्या॒त्मन्न् । \newline
7. अ॒न॒क्त्या॒त्मन् ना॒त्मन् न॑नक् त्यनक् त्या॒त्मन् ने॒वैवात्मन् न॑नक् त्यनक् त्या॒त्मन् ने॒व । \newline
8. आ॒त्मन् ने॒वैवात्मन् ना॒त्मन् ने॒व य॒ज्ञ्स्य॑ य॒ज्ञ्स्यै॒वात्मन् ना॒त्मन् ने॒व य॒ज्ञ्स्य॑ । \newline
9. ए॒व य॒ज्ञ्स्य॑ य॒ज्ञ्स्यै॒वैव य॒ज्ञ्स्य॒ शिरः॒ शिरो॑ य॒ज्ञ्स्यै॒वैव य॒ज्ञ्स्य॒ शिरः॑ । \newline
10. य॒ज्ञ्स्य॒ शिरः॒ शिरो॑ य॒ज्ञ्स्य॑ य॒ज्ञ्स्य॒ शिरः॒ प्रति॒ प्रति॒ शिरो॑ य॒ज्ञ्स्य॑ य॒ज्ञ्स्य॒ शिरः॒ प्रति॑ । \newline
11. शिरः॒ प्रति॒ प्रति॒ शिरः॒ शिरः॒ प्रति॑ दधाति दधाति॒ प्रति॒ शिरः॒ शिरः॒ प्रति॑ दधाति । \newline
12. प्रति॑ दधाति दधाति॒ प्रति॒ प्रति॑ दधा त्य॒ग्नि र॒ग्निर् द॑धाति॒ प्रति॒ प्रति॑ दधा त्य॒ग्निः । \newline
13. द॒धा॒ त्य॒ग्नि र॒ग्निर् द॑धाति दधा त्य॒ग्निर् दे॒वाना᳚म् दे॒वाना॑ म॒ग्निर् द॑धाति दधा त्य॒ग्निर् दे॒वाना᳚म् । \newline
14. अ॒ग्निर् दे॒वाना᳚म् दे॒वाना॑ म॒ग्नि र॒ग्निर् दे॒वाना᳚म् दू॒तो दू॒तो दे॒वाना॑ म॒ग्नि र॒ग्निर् दे॒वाना᳚म् दू॒तः । \newline
15. दे॒वाना᳚म् दू॒तो दू॒तो दे॒वाना᳚म् दे॒वाना᳚म् दू॒त आसी॒ दासी᳚द् दू॒तो दे॒वाना᳚म् दे॒वाना᳚म् दू॒त आसी᳚त् । \newline
16. दू॒त आसी॒ दासी᳚द् दू॒तो दू॒त आसी॒द् दैव्यो॒ दैव्य॒ आसी᳚द् दू॒तो दू॒त आसी॒द् दैव्यः॑ । \newline
17. आसी॒द् दैव्यो॒ दैव्य॒ आसी॒ दासी॒द् दैव्यो ऽसु॑राणा॒ मसु॑राणा॒म् दैव्य॒ आसी॒ दासी॒द् दैव्यो ऽसु॑राणाम् । \newline
18. दैव्यो ऽसु॑राणा॒ मसु॑राणा॒म् दैव्यो॒ दैव्यो ऽसु॑राणा॒म् तौ ता वसु॑राणा॒म् दैव्यो॒ दैव्यो ऽसु॑राणा॒म् तौ । \newline
19. असु॑राणा॒म् तौ ता वसु॑राणा॒ मसु॑राणा॒म् तौ प्र॒जाप॑तिम् प्र॒जाप॑ति॒म् ता वसु॑राणा॒ मसु॑राणा॒म् तौ प्र॒जाप॑तिम् । \newline
20. तौ प्र॒जाप॑तिम् प्र॒जाप॑ति॒म् तौ तौ प्र॒जाप॑तिम् प्र॒श्ञम् प्र॒श्ञम् प्र॒जाप॑ति॒म् तौ तौ प्र॒जाप॑तिम् प्र॒श्ञम् । \newline
21. प्र॒जाप॑तिम् प्र॒श्ञम् प्र॒श्ञम् प्र॒जाप॑तिम् प्र॒जाप॑तिम् प्र॒श्ञ मै॑ता मैताम् प्र॒श्ञम् प्र॒जाप॑तिम् प्र॒जाप॑तिम् प्र॒श्ञ मै॑ताम् । \newline
22. प्र॒जाप॑ति॒मिति॑ प्र॒जा - प॒ति॒म् । \newline
23. प्र॒श्ञ मै॑ता मैताम् प्र॒श्ञम् प्र॒श्ञ मै॑ताꣳ॒॒ स स ऐ॑ताम् प्र॒श्ञम् प्र॒श्ञ मै॑ताꣳ॒॒ सः । \newline
24. ऐ॒ताꣳ॒॒ स स ऐ॑ता मैताꣳ॒॒ स प्र॒जाप॑तिः प्र॒जाप॑तिः॒ स ऐ॑ता मैताꣳ॒॒ स प्र॒जाप॑तिः । \newline
25. स प्र॒जाप॑तिः प्र॒जाप॑तिः॒ स स प्र॒जाप॑तिर् ब्राह्म॒णम् ब्रा᳚ह्म॒णम् प्र॒जाप॑तिः॒ स स प्र॒जाप॑तिर् ब्राह्म॒णम् । \newline
26. प्र॒जाप॑तिर् ब्राह्म॒णम् ब्रा᳚ह्म॒णम् प्र॒जाप॑तिः प्र॒जाप॑तिर् ब्राह्म॒ण म॑ब्रवी दब्रवीद् ब्राह्म॒णम् प्र॒जाप॑तिः प्र॒जाप॑तिर् ब्राह्म॒ण म॑ब्रवीत् । \newline
27. प्र॒जाप॑ति॒रिति॑ प्र॒जा - प॒तिः॒ । \newline
28. ब्रा॒ह्म॒ण म॑ब्रवी दब्रवीद् ब्राह्म॒णम् ब्रा᳚ह्म॒ण म॑ब्रवी दे॒त दे॒त द॑ब्रवीद् ब्राह्म॒णम् ब्रा᳚ह्म॒ण म॑ब्रवी दे॒तत् । \newline
29. अ॒ब्र॒वी॒ दे॒त दे॒त द॑ब्रवी दब्रवी दे॒तद् वि व्ये॑त द॑ब्रवी दब्रवी दे॒तद् वि । \newline
30. ए॒तद् वि व्ये॑त दे॒तद् वि ब्रू॑हि ब्रूहि॒ व्ये॑त दे॒तद् वि ब्रू॑हि । \newline
31. वि ब्रू॑हि ब्रूहि॒ वि वि ब्रू॒हीतीति॑ ब्रूहि॒ वि वि ब्रू॒हीति॑ । \newline
32. ब्रू॒हीतीति॑ ब्रूहि ब्रू॒हीत्येति॑ ब्रूहि ब्रू॒हीत्या । \newline
33. इत्येतीत्या श्रा॑वय श्राव॒येतीत्या श्रा॑वय । \newline
34. आ श्रा॑वय श्राव॒या श्रा॑व॒ये तीति॑ श्राव॒या श्रा॑व॒ये ति॑ । \newline
35. श्रा॒व॒ये तीति॑ श्रावय श्राव॒ये ती॒द मि॒द मिति॑ श्रावय श्राव॒ये ती॒दम् । \newline
36. इती॒द मि॒द मितीती॒दम् दे॑वा देवा इ॒द मितीती॒दम् दे॑वाः । \newline
37. इ॒दम् दे॑वा देवा इ॒द मि॒दम् दे॑वाः शृणुत शृणुत देवा इ॒द मि॒दम् दे॑वाः शृणुत । \newline
38. दे॒वाः॒ शृ॒णु॒त॒ शृ॒णु॒त॒ दे॒वा॒ दे॒वाः॒ शृ॒णु॒ते तीति॑ शृणुत देवा देवाः शृणु॒ते ति॑ । \newline
39. शृ॒णु॒ते तीति॑ शृणुत शृणु॒ते ति॒ वाव वावे ति॑ शृणुत शृणु॒ते ति॒ वाव । \newline
40. इति॒ वाव वावे तीति॒ वाव तत् तद् वावे तीति॒ वाव तत् । \newline
41. वाव तत् तद् वाव वाव तद॑ब्रवी दब्रवी॒त् तद् वाव वाव तद॑ब्रवीत् । \newline
42. तद॑ब्रवी दब्रवी॒त् तत् तद॑ब्रवी द॒ग्नि र॒ग्नि र॑ब्रवी॒त् तत् तद॑ब्रवी द॒ग्निः । \newline
43. अ॒ब्र॒वी॒ द॒ग्नि र॒ग्नि र॑ब्रवी दब्रवी द॒ग्निर् दे॒वो दे॒वो᳚ ऽग्नि र॑ब्रवी दब्रवी द॒ग्निर् दे॒वः । \newline
44. अ॒ग्निर् दे॒वो दे॒वो᳚ ऽग्नि र॒ग्निर् दे॒वो होता॒ होता॑ दे॒वो᳚ ऽग्नि र॒ग्निर् दे॒वो होता᳚ । \newline
45. दे॒वो होता॒ होता॑ दे॒वो दे॒वो होतेतीति॒ होता॑ दे॒वो दे॒वो होतेति॑ । \newline
46. होतेतीति॒ होता॒ होतेति॒ यो य इति॒ होता॒ होतेति॒ यः । \newline
47. इति॒ यो य इतीति॒ य ए॒वैव य इतीति॒ य ए॒व । \newline
48. य ए॒वैव यो य ए॒व दे॒वाना᳚म् दे॒वाना॑ मे॒व यो य ए॒व दे॒वाना᳚म् । \newline
49. ए॒व दे॒वाना᳚म् दे॒वाना॑ मे॒वैव दे॒वाना॒म् तम् तम् दे॒वाना॑ मे॒वैव दे॒वाना॒म् तम् । \newline
50. दे॒वाना॒म् तम् तम् दे॒वाना᳚म् दे॒वाना॒म् त म॑वृणीता वृणीत॒ तम् दे॒वाना᳚म् दे॒वाना॒म् त म॑वृणीत । \newline
51. त म॑वृणीता वृणीत॒ तम् त म॑वृणीत॒ तत॒ स्ततो॑ ऽवृणीत॒ तम् त म॑वृणीत॒ ततः॑ । \newline
52. अ॒वृ॒णी॒त॒ तत॒ स्ततो॑ ऽवृणीता वृणीत॒ ततो॑ दे॒वा दे॒वा स्ततो॑ ऽवृणीता वृणीत॒ ततो॑ दे॒वाः । \newline
53. ततो॑ दे॒वा दे॒वा स्तत॒ स्ततो॑ दे॒वा अभ॑व॒न् नभ॑वन् दे॒वा स्तत॒ स्ततो॑ दे॒वा अभ॑वन्न् । \newline
54. दे॒वा अभ॑व॒न् नभ॑वन् दे॒वा दे॒वा अभ॑व॒न् परा॒ परा ऽभ॑वन् दे॒वा दे॒वा अभ॑व॒न् परा᳚ । \newline
\pagebreak
\markright{ TS 2.5.11.9  \hfill https://www.vedavms.in \hfill}
\addcontentsline{toc}{section}{ TS 2.5.11.9 }
\section*{ TS 2.5.11.9 }

\textbf{TS 2.5.11.9 } \newline
\textbf{Samhita Paata} \newline

अभ॑व॒न् पराऽसु॑रा॒ यस्यै॒वं ॅवि॒दुषः॑ प्रव॒रं प्र॑वृ॒णते॒ भव॑त्या॒त्मना॒ परा᳚ऽस्य॒ भ्रातृ॑व्यो भवति॒ यद्ब्रा᳚ह्म॒णश्चा ब्रा᳚ह्मणश्च प्र॒श्नमे॒यातां᳚ ब्राह्म॒णायाधि॑ ब्रूया॒द् यद् ब्रा᳚ह्म॒णाया॒द्ध्याहा॒ ऽऽत्मनेऽद्ध्या॑ह॒ यद्ब्रा᳚ह्म॒णं प॒राहा॒ऽऽत्मानं॒ परा॑ऽऽह॒ तस्मा᳚द् ब्राह्म॒णो न प॒रोच्यः॑ ॥ \newline

\textbf{Pada Paata} \newline

अभ॑वन्न् । परेति॑ । असु॑राः । यस्य॑ । ए॒वम् । वि॒दुषः॑ । प्र॒व॒रमिति॑ प्र - व॒रम् । प्र॒वृ॒णत॒ इति॑ प्र - वृ॒णते᳚ । भव॑ति । आ॒त्मना᳚ । परेति॑ । अ॒स्य॒ । भ्रातृ॑व्यः । भ॒व॒ति॒ । यत् । ब्रा॒ह्म॒णः । च॒ । अब्रा᳚ह्मणः । च॒ । प्र॒श्नम् । ए॒याता॒मित्या᳚ - इ॒याता᳚म् । ब्रा॒ह्म॒णाय॑ । अधीति॑ । ब्रू॒या॒त् । यत् । ब्रा॒ह्म॒णाय॑ । अ॒द्ध्याहेत्य॑धि-आह॑ । आ॒त्मने᳚ । अधीति॑ । आ॒ह॒ । यत् । ब्रा॒ह्म॒णम् । प॒राहेति॑ परा - आह॑ । आ॒त्मान᳚म् । परेति॑ । आ॒ह॒ । तस्मा᳚त् । ब्रा॒ह्म॒णः । न । प॒रोच्य॒ इति॑ परा-उच्यः॑ ॥  \newline


\textbf{Krama Paata} \newline

अभ॑व॒न् परा᳚ । परा ऽसु॑राः । असु॑रा॒ यस्य॑ । यस्यै॒वम् । ए॒वम् ॅवि॒दुषः॑ । वि॒दुषः॑ प्रव॒रम् । प्र॒व॒रम् प्र॑वृ॒णते᳚ । प्र॒व॒रमिति॑ प्र - व॒रम् । प्र॒वृ॒णते॒ भव॑ति । प्र॒वृ॒णत॒ इति॑ प्र - वृ॒णते᳚ । भव॑त्या॒त्मना᳚ । आ॒त्मना॒ परा᳚ । परा᳚ऽस्य । अ॒स्य॒ भ्रातृ॑व्यः । भ्रातृ॑व्यो भवति । भ॒व॒ति॒ यत् । यद् ब्रा᳚ह्म॒णः । बा॒ह्म॒णश्च॑ । चाब्रा᳚ह्मणः । अब्रा᳚ह्मणश्च । च॒ प्र॒श्ञम् । प्र॒श्ञमे॒याता᳚म् । ए॒याता᳚म् ब्राह्म॒णाय॑ । ए॒याता॒मित्या᳚ - इ॒याता᳚म् । ब्रा॒ह्म॒णायाधि॑ । अधि॑ ब्रूयात् । ब्रू॒या॒द् यत् । यद् ब्रा᳚ह्म॒णाय॑ । बा॒ह्म॒णाया॒द्ध्याह॑ । अ॒द्ध्याहा॒त्मने᳚ । अ॒द्ध्याहेत्य॑धि - आह॑ । आ॒त्मनेऽधि॑ । अद्ध्या॑ह । आ॒ह॒ यत् । यद् ब्रा᳚ह्म॒णम् । ब्रा॒ह्म॒णम् प॒राह॑ । प॒राहा॒त्मान᳚म् । प॒राहेति॑ परा - आह॑ । आ॒त्मान॒म् परा᳚ । परा॑ ऽऽह । आ॒ह॒ तस्मा᳚त् । तस्मा᳚द् ब्राह॒णः । ब्रा॒ह्म॒णो न । न प॒रोच्यः॑ । प॒रोच्य॒ इति॑ परा - उच्यः॑ । \newline

\textbf{Jatai Paata} \newline

1. अभ॑व॒न् परा॒ परा ऽभ॑व॒न् नभ॑व॒न् परा᳚ । \newline
2. परा ऽसु॑रा॒ असु॑राः॒ परा॒ परा ऽसु॑राः । \newline
3. असु॑रा॒ यस्य॒ यस्यासु॑रा॒ असु॑रा॒ यस्य॑ । \newline
4. यस्यै॒व मे॒वं ॅयस्य॒ यस्यै॒वम् । \newline
5. ए॒वं ॅवि॒दुषो॑ वि॒दुष॑ ए॒व मे॒वं ॅवि॒दुषः॑ । \newline
6. वि॒दुषः॑ प्रव॒रम् प्र॑व॒रं ॅवि॒दुषो॑ वि॒दुषः॑ प्रव॒रम् । \newline
7. प्र॒व॒रम् प्र॑वृ॒णते᳚ प्रवृ॒णते᳚ प्रव॒रम् प्र॑व॒रम् प्र॑वृ॒णते᳚ । \newline
8. प्र॒व॒रमिति॑ प्र - व॒रम् । \newline
9. प्र॒वृ॒णते॒ भव॑ति॒ भव॑ति प्रवृ॒णते᳚ प्रवृ॒णते॒ भव॑ति । \newline
10. प्र॒वृ॒णत॒ इति॑ प्र - वृ॒णते᳚ । \newline
11. भव॑ त्या॒त्मना॒ ऽऽत्मना॒ भव॑ति॒ भव॑ त्या॒त्मना᳚ । \newline
12. आ॒त्मना॒ परा॒ परा॒ ऽऽत्मना॒ ऽऽत्मना॒ परा᳚ । \newline
13. परा᳚ ऽस्यास्य॒ परा॒ परा᳚ ऽस्य । \newline
14. अ॒स्य॒ भ्रातृ॑व्यो॒ भ्रातृ॑व्यो ऽस्यास्य॒ भ्रातृ॑व्यः । \newline
15. भ्रातृ॑व्यो भवति भवति॒ भ्रातृ॑व्यो॒ भ्रातृ॑व्यो भवति । \newline
16. भ॒व॒ति॒ यद् यद् भ॑वति भवति॒ यत् । \newline
17. यद् ब्रा᳚ह्म॒णो ब्रा᳚ह्म॒णो यद् यद् ब्रा᳚ह्म॒णः । \newline
18. ब्रा॒ह्म॒णश्च॑ च ब्राह्म॒णो ब्रा᳚ह्म॒णश्च॑ । \newline
19. चाब्रा᳚ह्म॒णो ऽब्रा᳚ह्मणश्च॒ चाब्रा᳚ह्मणः । \newline
20. अब्रा᳚ह्मणश्च॒ चाब्रा᳚ह्म॒णो ऽब्रा᳚ह्मणश्च । \newline
21. च॒ प्र॒श्ञम् प्र॒श्ञम् च॑ च प्र॒श्ञम् । \newline
22. प्र॒श्ञ मे॒याता॑ मे॒याता᳚म् प्र॒श्ञम् प्र॒श्ञ मे॒याता᳚म् । \newline
23. ए॒याता᳚म् ब्राह्म॒णाय॑ ब्राह्म॒णाये॒याता॑ मे॒याता᳚म् ब्राह्म॒णाय॑ । \newline
24. ए॒याता॒मित्या᳚ - इ॒याता᳚म् । \newline
25. ब्रा॒ह्म॒णाया ध्यधि॑ ब्राह्म॒णाय॑ ब्राह्म॒णायाधि॑ । \newline
26. अधि॑ ब्रूयाद् ब्रूया॒ दध्यधि॑ ब्रूयात् । \newline
27. ब्रू॒या॒द् यद् यद् ब्रू॑याद् ब्रूया॒द् यत् । \newline
28. यद् ब्रा᳚ह्म॒णाय॑ ब्राह्म॒णाय॒ यद् यद् ब्रा᳚ह्म॒णाय॑ । \newline
29. ब्रा॒ह्म॒णाया॒ द्ध्याहा॒ द्ध्याह॑ ब्राह्म॒णाय॑ ब्राह्म॒णाया॒ द्ध्याह॑ । \newline
30. अ॒द्ध्याहा॒त्मन॑ आ॒त्मने॒ ऽद्ध्याहा॒ द्ध्याहा॒त्मने᳚ । \newline
31. अ॒द्ध्याहेत्य॑धि - आह॑ । \newline
32. आ॒त्मने ऽध्यध्या॒त्मन॑ आ॒त्मने ऽधि॑ । \newline
33. अध्या॑ हा॒हा ध्यध्या॑ह । \newline
34. आ॒ह॒ यद् यदा॑हाह॒ यत् । \newline
35. यद् ब्रा᳚ह्म॒णम् ब्रा᳚ह्म॒णं ॅयद् यद् ब्रा᳚ह्म॒णम् । \newline
36. ब्रा॒ह्म॒णम् प॒राह॑ प॒राह॑ ब्राह्म॒णम् ब्रा᳚ह्म॒णम् प॒राह॑ । \newline
37. प॒राहा॒त्मान॑ मा॒त्मान॑म् प॒राह॑ प॒राहा॒त्मान᳚म् । \newline
38. प॒राहेति॑ परा - आह॑ । \newline
39. आ॒त्मान॒म् परा॒ परा॒ ऽऽत्मान॑ मा॒त्मान॒म् परा᳚ । \newline
40. परा॑ ऽऽहाह॒ परा॒ परा॑ ऽऽह । \newline
41. आ॒ह॒ तस्मा॒त् तस्मा॑ दाहाह॒ तस्मा᳚त् । \newline
42. तस्मा᳚द् ब्राह्म॒णो ब्रा᳚ह्म॒ण स्तस्मा॒त् तस्मा᳚द् ब्राह्म॒णः । \newline
43. ब्रा॒ह्म॒णो न न ब्रा᳚ह्म॒णो ब्रा᳚ह्म॒णो न । \newline
44. न प॒रोच्यः॑ प॒रोच्यो॒ न न प॒रोच्यः॑ । \newline
45. प॒रोच्य॒ इति॑ परा - उच्यः॑ । \newline

\textbf{Ghana Paata } \newline

1. अभ॑व॒न् परा॒ परा ऽभ॑व॒न् नभ॑व॒न् परा ऽसु॑रा॒ असु॑राः॒ परा ऽभ॑व॒न् नभ॑व॒न् परा ऽसु॑राः । \newline
2. परा ऽसु॑रा॒ असु॑राः॒ परा॒ परा ऽसु॑रा॒ यस्य॒ यस्यासु॑राः॒ परा॒ परा ऽसु॑रा॒ यस्य॑ । \newline
3. असु॑रा॒ यस्य॒ यस्यासु॑रा॒ असु॑रा॒ यस्यै॒व मे॒वं ॅयस्यासु॑रा॒ असु॑रा॒ यस्यै॒वम् । \newline
4. यस्यै॒व मे॒वं ॅयस्य॒ यस्यै॒वं ॅवि॒दुषो॑ वि॒दुष॑ ए॒वं ॅयस्य॒ यस्यै॒वं ॅवि॒दुषः॑ । \newline
5. ए॒वं ॅवि॒दुषो॑ वि॒दुष॑ ए॒व मे॒वं ॅवि॒दुषः॑ प्रव॒रम् प्र॑व॒रं ॅवि॒दुष॑ ए॒व मे॒वं ॅवि॒दुषः॑ प्रव॒रम् । \newline
6. वि॒दुषः॑ प्रव॒रम् प्र॑व॒रं ॅवि॒दुषो॑ वि॒दुषः॑ प्रव॒रम् प्र॑वृ॒णते᳚ प्रवृ॒णते᳚ प्रव॒रं ॅवि॒दुषो॑ वि॒दुषः॑ प्रव॒रम् प्र॑वृ॒णते᳚ । \newline
7. प्र॒व॒रम् प्र॑वृ॒णते᳚ प्रवृ॒णते᳚ प्रव॒रम् प्र॑व॒रम् प्र॑वृ॒णते॒ भव॑ति॒ भव॑ति प्रवृ॒णते᳚ प्रव॒रम् प्र॑व॒रम् प्र॑वृ॒णते॒ भव॑ति । \newline
8. प्र॒व॒रमिति॑ प्र - व॒रम् । \newline
9. प्र॒वृ॒णते॒ भव॑ति॒ भव॑ति प्रवृ॒णते᳚ प्रवृ॒णते॒ भव॑ त्या॒त्मना॒ ऽऽत्मना॒ भव॑ति प्रवृ॒णते᳚ प्रवृ॒णते॒ भव॑ त्या॒त्मना᳚ । \newline
10. प्र॒वृ॒णत॒ इति॑ प्र - वृ॒णते᳚ । \newline
11. भव॑ त्या॒त्मना॒ ऽऽत्मना॒ भव॑ति॒ भव॑ त्या॒त्मना॒ परा॒ परा॒ ऽऽत्मना॒ भव॑ति॒ भव॑ त्या॒त्मना॒ परा᳚ । \newline
12. आ॒त्मना॒ परा॒ परा॒ ऽऽत्मना॒ ऽऽत्मना॒ परा᳚ ऽस्यास्य॒ परा॒ ऽऽत्मना॒ ऽऽत्मना॒ परा᳚ ऽस्य । \newline
13. परा᳚ ऽस्यास्य॒ परा॒ परा᳚ ऽस्य॒ भ्रातृ॑व्यो॒ भ्रातृ॑व्यो ऽस्य॒ परा॒ परा᳚ ऽस्य॒ भ्रातृ॑व्यः । \newline
14. अ॒स्य॒ भ्रातृ॑व्यो॒ भ्रातृ॑व्यो ऽस्यास्य॒ भ्रातृ॑व्यो भवति भवति॒ भ्रातृ॑व्यो ऽस्यास्य॒ भ्रातृ॑व्यो भवति । \newline
15. भ्रातृ॑व्यो भवति भवति॒ भ्रातृ॑व्यो॒ भ्रातृ॑व्यो भवति॒ यद् यद् भ॑वति॒ भ्रातृ॑व्यो॒ भ्रातृ॑व्यो भवति॒ यत् । \newline
16. भ॒व॒ति॒ यद् यद् भ॑वति भवति॒ यद् ब्रा᳚ह्म॒णो ब्रा᳚ह्म॒णो यद् भ॑वति भवति॒ यद् ब्रा᳚ह्म॒णः । \newline
17. यद् ब्रा᳚ह्म॒णो ब्रा᳚ह्म॒णो यद् यद् ब्रा᳚ह्म॒णश्च॑ च ब्राह्म॒णो यद् यद् ब्रा᳚ह्म॒णश्च॑ । \newline
18. ब्रा॒ह्म॒णश्च॑ च ब्राह्म॒णो ब्रा᳚ह्म॒णश्चा ब्रा᳚ह्म॒णो ऽब्रा᳚ह्मणश्च ब्राह्म॒णो ब्रा᳚ह्म॒ण श्चाब्रा᳚ह्मणः । \newline
19. चाब्रा᳚ह्म॒णो ऽब्रा᳚ह्मणश्च॒ चाब्रा᳚ह्मणश्च॒ चाब्रा᳚ह्मणश्च॒ चाब्रा᳚ह्मणश्च । \newline
20. अब्रा᳚ह्मणश्च॒ चाब्रा᳚ह्म॒णो ऽब्रा᳚ह्मणश्च प्र॒श्ञम् प्र॒श्ञम् चाब्रा᳚ह्म॒णो ऽब्रा᳚ह्मणश्च प्र॒श्ञम् । \newline
21. च॒ प्र॒श्ञम् प्र॒श्ञम् च॑ च प्र॒श्ञ मे॒याता॑ मे॒याता᳚म् प्र॒श्ञम् च॑ च प्र॒श्ञ मे॒याता᳚म् । \newline
22. प्र॒श्ञ मे॒याता॑ मे॒याता᳚म् प्र॒श्ञम् प्र॒श्ञ मे॒याता᳚म् ब्राह्म॒णाय॑ ब्राह्म॒णा ये॒याता᳚म् प्र॒श्ञम् प्र॒श्ञ मे॒याता᳚म् ब्राह्म॒णाय॑ । \newline
23. ए॒याता᳚म् ब्राह्म॒णाय॑ ब्राह्म॒णा ये॒याता॑ मे॒याता᳚म् ब्राह्म॒णाया ध्यधि॑ ब्राह्म॒णा ये॒याता॑ मे॒याता᳚म् ब्राह्म॒णायाधि॑ । \newline
24. ए॒याता॒मित्या᳚ - इ॒याता᳚म् । \newline
25. ब्रा॒ह्म॒णाया ध्यधि॑ ब्राह्म॒णाय॑ ब्राह्म॒णायाधि॑ ब्रूयाद् ब्रूया॒दधि॑ ब्राह्म॒णाय॑ ब्राह्म॒णायाधि॑ ब्रूयात् । \newline
26. अधि॑ ब्रूयाद् ब्रूया॒ दध्यधि॑ ब्रूया॒द् यद् यद् ब्रू॑या॒ दध्यधि॑ ब्रूया॒द् यत् । \newline
27. ब्रू॒या॒द् यद् यद् ब्रू॑याद् ब्रूया॒द् यद् ब्रा᳚ह्म॒णाय॑ ब्राह्म॒णाय॒ यद् ब्रू॑याद् ब्रूया॒द् यद् ब्रा᳚ह्म॒णाय॑ । \newline
28. यद् ब्रा᳚ह्म॒णाय॑ ब्राह्म॒णाय॒ यद् यद् ब्रा᳚ह्म॒णाया॒ द्ध्याहा॒ द्ध्याह॑ ब्राह्म॒णाय॒ यद् यद् ब्रा᳚ह्म॒णाया॒ द्ध्याह॑ । \newline
29. ब्रा॒ह्म॒णाया॒ द्ध्याहा॒ द्ध्याह॑ ब्राह्म॒णाय॑ ब्राह्म॒णाया॒ द्ध्याहा॒त्मन॑ आ॒त्मने॒ ऽद्ध्याह॑ ब्राह्म॒णाय॑ ब्राह्म॒णाया॒ द्ध्याहा॒त्मने᳚ । \newline
30. अ॒द्ध्याहा॒त्मन॑ आ॒त्मने॒ ऽद्ध्याहा॒ द्ध्याहा॒त्मने ऽध्यध्या॒त्मने॒ ऽद्ध्याहा॒ द्ध्याहा॒त्मने ऽधि॑ । \newline
31. अ॒द्ध्याहेत्य॑धि - आह॑ । \newline
32. आ॒त्मने ऽध्यध्या॒त्मन॑ आ॒त्मने ऽध्या॑हा॒हा ध्या॒त्मन॑ आ॒त्मने ऽध्या॑ह । \newline
33. अध्या॑हा॒हा ध्यध्या॑ह॒ यद् यदा॒हा ध्यध्या॑ह॒ यत् । \newline
34. आ॒ह॒ यद् यदा॑हाह॒ यद् ब्रा᳚ह्म॒णम् ब्रा᳚ह्म॒णं ॅयदा॑हाह॒ यद् ब्रा᳚ह्म॒णम् । \newline
35. यद् ब्रा᳚ह्म॒णम् ब्रा᳚ह्म॒णं ॅयद् यद् ब्रा᳚ह्म॒णम् प॒राह॑ प॒राह॑ ब्राह्म॒णं ॅयद् यद् ब्रा᳚ह्म॒णम् प॒राह॑ । \newline
36. ब्रा॒ह्म॒णम् प॒राह॑ प॒राह॑ ब्राह्म॒णम् ब्रा᳚ह्म॒णम् प॒राहा॒त्मान॑ मा॒त्मान॑म् प॒राह॑ ब्राह्म॒णम् ब्रा᳚ह्म॒णम् प॒राहा॒त्मान᳚म् । \newline
37. प॒राहा॒त्मान॑ मा॒त्मान॑म् प॒राह॑ प॒राहा॒त्मान॒म् परा॒ परा॒ ऽऽत्मान॑म् प॒राह॑ प॒राहा॒त्मान॒म् परा᳚ । \newline
38. प॒राहेति॑ परा - आह॑ । \newline
39. आ॒त्मान॒म् परा॒ परा॒ ऽऽत्मान॑ मा॒त्मान॒म् परा॑ ऽऽहाह॒ परा॒ ऽऽत्मान॑ मा॒त्मान॒म् परा॑ ऽऽह । \newline
40. परा॑ ऽऽहाह॒ परा॒ परा॑ ऽऽह॒ तस्मा॒त् तस्मा॑ दाह॒ परा॒ परा॑ ऽऽह॒ तस्मा᳚त् । \newline
41. आ॒ह॒ तस्मा॒त् तस्मा॑ दाहाह॒ तस्मा᳚द् ब्राह्म॒णो ब्रा᳚ह्म॒ण स्तस्मा॑ दाहाह॒ तस्मा᳚द् ब्राह्म॒णः । \newline
42. तस्मा᳚द् ब्राह्म॒णो ब्रा᳚ह्म॒ण स्तस्मा॒त् तस्मा᳚द् ब्राह्म॒णो न न ब्रा᳚ह्म॒ण स्तस्मा॒त् तस्मा᳚द् ब्राह्म॒णो न । \newline
43. ब्रा॒ह्म॒णो न न ब्रा᳚ह्म॒णो ब्रा᳚ह्म॒णो न प॒रोच्यः॑ प॒रोच्यो॒ न ब्रा᳚ह्म॒णो ब्रा᳚ह्म॒णो न प॒रोच्यः॑ । \newline
44. न प॒रोच्यः॑ प॒रोच्यो॒ न न प॒रोच्यः॑ । \newline
45. प॒रोच्य॒ इति॑ परा - उच्यः॑ । \newline
\pagebreak
\markright{ TS 2.5.12.1  \hfill https://www.vedavms.in \hfill}
\addcontentsline{toc}{section}{ TS 2.5.12.1 }
\section*{ TS 2.5.12.1 }

\textbf{TS 2.5.12.1 } \newline
\textbf{Samhita Paata} \newline

आयु॑ष्ट >1, आयु॒र्दा अ॑ग्न॒ >2, आ प्या॑यस्व॒ >3, सं ते >4, ऽव॑ ते॒ हेड॒ >5, उदु॑त्त॒मं >6, प्रणो॑ दे॒व्या >7, नो॑ दि॒वो >8 , ऽग्ना॑ विष्णू॒ >9, अग्ना॑विष्णू >10,इ॒मं मे॑ वरुण॒ >11,तत्त्वा॑ या॒>12, म्यु दु॒त्यम् >13 , चि॒त्रम् >14 ॥अ॒पां नपा॒दा ह्यस्था॑दु॒पस्थं॑ जि॒ह्माना॑मू॒र्द्ध्वो वि॒द्युतं॒ ॅवसा॑नः । तस्य॒ ज्येष्ठं॑ महि॒मानं॒ ॅवह॑न्ती॒ र्.हिर॑ण्यवर्णाः॒ परि॑ यन्ति य॒ह्वीः ॥ स-  [  ] \newline

\textbf{Pada Paata} \newline

आयुः॑ । ते॒ । आ॒यु॒र्दा इत्या॑युः - दाः । अ॒ग्ने॒ । एति॑ । प्या॒य॒स्व॒ । समिति॑ । ते॒ । अवेति॑ । ते॒ । हेडः॑ । उदिति॑ । उ॒त्त॒ममित्यु॑त् - त॒मम् । प्रेति॑ । नः॒ । दे॒वी । एति॑ । नः॒ । दि॒वः । अग्ना॑विष्णू॒ इत्यग्ना᳚-वि॒ष्णू॒ । अग्ना॑विष्णू॒ इत्यग्ना᳚ - वि॒ष्णू॒ । इ॒मम् । मे॒ । व॒रु॒ण॒ । तत् । त्वा॒ । या॒मि॒ । उदिति॑ । उ॒ । त्यम् । चि॒त्रम् ॥ अ॒पाम् । नपा᳚त् । एति॑ । हि । अस्था᳚त् । उ॒पस्थ॒मित्यु॒प - स्थ॒म् । जि॒ह्माना᳚म् । ऊ॒द्‌र्ध्वः । वि॒द्युत॒मिति॑ वि - द्युत᳚म् । वसा॑नः ॥ तस्य॑ । ज्येष्ठ᳚म् । म॒हि॒मान᳚म् । वह॑न्तीः । हिर॑ण्यवर्णा॒ इति॒ हिर॑ण्य - व॒र्णाः॒ । परीति॑ । य॒न्ति॒ । य॒ह्वीः ॥ समिति॑ ।  \newline


\textbf{Krama Paata} \newline

आयु॑ष्टे । त॒ आ॒यु॒र्दाः । आ॒यु॒र्दा अ॑ग्ने । आ॒यु॒र्दा इत्या॑युः - दाः । अ॒ग्न॒ आ । आ प्या॑यस्व । प्या॒य॒स्व॒ सम् । सम् ते᳚ । ते ऽव॑ । अव॑ ते । ते॒ हेडः॑ । हेड॒ उत् । उदु॑त्त॒मम् । उ॒त्त॒मम् प्र । उ॒त्त॒ममित्यु॑त् - त॒मम् । प्र णः॑ । नो॒ दे॒वी । दे॒व्या । आ नः॑ । नो॒ दि॒वः । दि॒वो ऽग्ना॑विष्णू । अग्ना॑विष्णू॒ अग्ना॑विष्णू । अग्ना॑विष्णू॒ इत्यग्ना᳚ - वि॒ष्णू॒ । अग्ना॑विष्णू इ॒मम् । अग्ना॑विष्णू॒ इत्यग्ना᳚ - वि॒ष्णू॒ । इ॒मम् मे᳚ । मे॒ व॒रु॒ण॒ । व॒रु॒ण॒ तत् । तत् त्वा᳚ । त्वा॒ या॒मि॒ । या॒म्युत् । उद् उ॑ । उ॒ त्यम् । त्यम् चि॒त्रम् । चि॒त्रमिति॑ चि॒त्रम् ॥ अ॒पान्नपा᳚त् । नपा॒दा । आ हि । ह्यस्था᳚त् । अस्था॑दु॒पस्थ᳚म् । उ॒पस्थ॑म् जि॒ह्माना᳚म् । उ॒पस्थ॒मित्यु॒प - स्थ॒म् । जि॒ह्माना॑मू॒र्द्ध्वः । ऊ॒र्द्धो वि॒द्युत᳚म् । वि॒द्युत॒म् ॅवसा॑नः । वि॒द्युत॒मिति॑ वि - द्युत᳚म् । वसा॑न॒ इति॒ वसा॑नः ॥ तस्य॒ ज्येष्ठ᳚म् । ज्येष्ठ॑म् महि॒मान᳚म् । म॒हि॒मान॒म् ॅवह॑न्तीः । वह॑न्ती॒र्॒. हिर॑ण्यवर्णाः । हिर॑ण्यवर्णाः॒ परि॑ । हिर॑ण्यवर्णा॒ इति॒ हिर॑ण्य - व॒र्णाः॒ । परि॑ यन्ति । य॒न्ति॒ य॒ह्वीः । य॒ह्वीरिति॑ य॒ह्वीः ॥ सम॒न्याः \newline

\textbf{Jatai Paata} \newline

1. आयु॑ष्टे त॒ आयु॒ रायु॑ष्टे । \newline
2. त॒ आ॒यु॒र्दा आ॑यु॒र्दा स्ते॑ त आयु॒र्दाः । \newline
3. आ॒यु॒र्दा अ॑ग्ने अग्न आयु॒र्दा आ॑यु॒र्दा अ॑ग्ने । \newline
4. आ॒यु॒र्दा इत्या॑युः - दाः । \newline
5. अ॒ग्न॒ आ ऽग्ने॑ अग्न॒ आ । \newline
6. आ प्या॑यस्व प्याय॒स्वा प्या॑यस्व । \newline
7. प्या॒य॒स्व॒ सꣳ सम् प्या॑यस्व प्यायस्व॒ सम् । \newline
8. सम् ते॑ ते॒ सꣳ सम् ते᳚ । \newline
9. ते ऽवाव॑ ते॒ ते ऽव॑ । \newline
10. अव॑ ते॒ ते ऽवाव॑ ते । \newline
11. ते॒ हेडो॒ हेड॑ स्ते ते॒ हेडः॑ । \newline
12. हेड॒ उदुद्धेडो॒ हेड॒ उत् । \newline
13. उदु॑त्त॒म मु॑त्त॒म मुदुदु॑त्त॒मम् । \newline
14. उ॒त्त॒मम् प्र प्रोत्त॒म मु॑त्त॒मम् प्र । \newline
15. उ॒त्त॒ममित्यु॑त् - त॒मम् । \newline
16. प्र णो॑ नः॒ प्र प्र णः॑ । \newline
17. नो॒ दे॒वी दे॒वी नो॑ नो दे॒वी । \newline
18. दे॒व्या दे॒वी दे॒व्या । \newline
19. आ नो॑ न॒ आ नः॑ । \newline
20. नो॒ दि॒वो दि॒वो नो॑ नो दि॒वः । \newline
21. दि॒वो ऽग्ना॑विष्णू॒ अग्ना॑विष्णू दि॒वो दि॒वो ऽग्ना॑विष्णू । \newline
22. अग्ना॑विष्णू॒ अग्ना॑विष्णू । \newline
23. अग्ना॑विष्णू॒ इत्यग्ना᳚ - वि॒ष्णू॒ । \newline
24. अग्ना॑विष्णू इ॒म मि॒म मग्ना॑विष्णू॒ अग्ना॑विष्णू इ॒मम् । \newline
25. अग्ना॑विष्णू॒ इत्यग्ना᳚ - वि॒ष्णू॒ । \newline
26. इ॒मम् मे॑ म इ॒म मि॒मम् मे᳚ । \newline
27. मे॒ व॒रु॒ण॒ व॒रु॒ण॒ मे॒ मे॒ व॒रु॒ण॒ । \newline
28. व॒रु॒ण॒ तत् तद् व॑रुण वरुण॒ तत् । \newline
29. तत् त्वा᳚ त्वा॒ तत् तत् त्वा᳚ । \newline
30. त्वा॒ या॒मि॒ या॒मि॒ त्वा॒ त्वा॒ या॒मि॒ । \newline
31. या॒म्युदुद् या॑मि या॒म्युत् । \newline
32. उदु॑ वु॒ वुदुदु॑ । \newline
33. उ॒ त्यम् त्य मु॑ वु॒ त्यम् । \newline
34. त्यम् चि॒त्रम् चि॒त्रम् त्यम् त्यम् चि॒त्रम् । \newline
35. चि॒त्रमिति॑ चि॒त्रम् । \newline
36. अ॒पाम् नपा॒न् नपा॑द॒पा म॒पाम् नपा᳚त् । \newline
37. नपा॒दा नपा॒न् नपा॒दा । \newline
38. आ हि ह्या हि । \newline
39. ह्यस्था॒ दस्था॒द्धि ह्यस्था᳚त् । \newline
40. अस्था॑ दु॒पस्थ॑ मु॒पस्थ॒ मस्था॒ दस्था॑ दु॒पस्थ᳚म् । \newline
41. उ॒पस्थ॑म् जि॒ह्माना᳚म् जि॒ह्माना॑ मु॒पस्थ॑ मु॒पस्थ॑म् जि॒ह्माना᳚म् । \newline
42. उ॒पस्थ॒मित्यु॒प - स्थ॒म् । \newline
43. जि॒ह्माना॑ मू॒र्द्ध्व ऊ॒र्द्ध्वो जि॒ह्माना᳚म् जि॒ह्माना॑ मू॒र्द्ध्वः । \newline
44. ऊ॒र्द्ध्वो वि॒द्युतं॑ ॅवि॒द्युत॑ मू॒र्द्ध्व ऊ॒र्द्ध्वो वि॒द्युत᳚म् । \newline
45. वि॒द्युतं॒ ॅवसा॑नो॒ वसा॑नो वि॒द्युतं॑ ॅवि॒द्युतं॒ ॅवसा॑नः । \newline
46. वि॒द्युत॒मिति॑ वि - द्युत᳚म् । \newline
47. वसा॑न॒ इति॒ वसा॑नः । \newline
48. तस्य॒ ज्येष्ठ॒म् ज्येष्ठ॒म् तस्य॒ तस्य॒ ज्येष्ठ᳚म् । \newline
49. ज्येष्ठ॑म् महि॒मान॑म् महि॒मान॒म् ज्येष्ठ॒म् ज्येष्ठ॑म् महि॒मान᳚म् । \newline
50. म॒हि॒मानं॒ ॅवह॑न्ती॒र् वह॑न्तीर् महि॒मान॑म् महि॒मानं॒ ॅवह॑न्तीः । \newline
51. वह॑न्ती॒र्॒. हिर॑ण्यवर्णा॒ हिर॑ण्यवर्णा॒ वह॑न्ती॒र् वह॑न्ती॒र्॒. हिर॑ण्यवर्णाः । \newline
52. हिर॑ण्यवर्णाः॒ परि॒ परि॒ हिर॑ण्यवर्णा॒ हिर॑ण्यवर्णाः॒ परि॑ । \newline
53. हिर॑ण्यवर्णा॒ इति॒ हिर॑ण्य - व॒र्णाः॒ । \newline
54. परि॑ यन्ति यन्ति॒ परि॒ परि॑ यन्ति । \newline
55. य॒न्ति॒ य॒ह्वीर् य॒ह्वीर् य॑न्ति यन्ति य॒ह्वीः । \newline
56. य॒ह्वीरिति॑ य॒ह्वीः । \newline
57. स म॒न्या अ॒न्याः सꣳ स म॒न्याः । \newline

\textbf{Ghana Paata } \newline

1. आयु॑ष्टे त॒ आयु॒रायु॑ष्ट॒ आयु॒र्दा आ॑यु॒र्दा स्त॒ आयु॒रायु॑ष्ट॒ आयु॒र्दाः । \newline
2. त॒ आ॒यु॒र्दा आ॑यु॒र्दा स्ते॑ त आयु॒र्दा अ॑ग्ने अग्न आयु॒र्दा स्ते॑ त आयु॒र्दा अ॑ग्ने । \newline
3. आ॒यु॒र्दा अ॑ग्ने अग्न आयु॒र्दा आ॑यु॒र्दा अ॑ग्न॒ आ ऽग्न॑ आयु॒र्दा आ॑यु॒र्दा अ॑ग्न॒ आ । \newline
4. आ॒यु॒र्दा इत्या॑युः - दाः । \newline
5. अ॒ग्न॒ आ ऽग्ने॑ अग्न॒ आ प्या॑यस्व प्याय॒स्वा ऽग्ने॑ अग्न॒ आ प्या॑यस्व । \newline
6. आ प्या॑यस्व प्याय॒स्वा प्या॑यस्व॒ सꣳ सम् प्या॑य॒स्वा प्या॑यस्व॒ सम् । \newline
7. प्या॒य॒स्व॒ सꣳ सम् प्या॑यस्व प्यायस्व॒ सम् ते॑ ते॒ सम् प्या॑यस्व प्यायस्व॒ सम् ते᳚ । \newline
8. सम् ते॑ ते॒ सꣳ सम् ते ऽवाव॑ ते॒ सꣳ सम् ते ऽव॑ । \newline
9. ते ऽवाव॑ ते॒ ते ऽव॑ ते॒ ते ऽव॑ ते॒ ते ऽव॑ ते । \newline
10. अव॑ ते॒ ते ऽवाव॑ ते॒ हेडो॒ हेड॒ स्ते ऽवाव॑ ते॒ हेडः॑ । \newline
11. ते॒ हेडो॒ हेड॑ स्ते ते॒ हेड॒ उदुद्धेड॑ स्ते ते॒ हेड॒ उत् । \newline
12. हेड॒ उदुद्धेडो॒ हेड॒ उदु॑त्त॒म मु॑त्त॒म मुद्धेडो॒ हेड॒ उदु॑त्त॒मम् । \newline
13. उदु॑त्त॒म मु॑त्त॒म मुदु दु॑त्त॒मम् प्र प्रोत्त॒म मुदु दु॑त्त॒मम् प्र । \newline
14. उ॒त्त॒मम् प्र प्रोत्त॒म मु॑त्त॒मम् प्र णो॑ नः॒ प्रोत्त॒म मु॑त्त॒मम् प्र णः॑ । \newline
15. उ॒त्त॒ममित्यु॑त् - त॒मम् । \newline
16. प्र णो॑ नः॒ प्र प्र णो॑ दे॒वी दे॒वी नः॒ प्र प्र णो॑ दे॒वी । \newline
17. नो॒ दे॒वी दे॒वी नो॑ नो दे॒व्या दे॒वी नो॑ नो दे॒व्या । \newline
18. दे॒व्या दे॒वी दे॒व्या नो॑ न॒ आ दे॒वी दे॒व्या नः॑ । \newline
19. आ नो॑ न॒ आ नो॑ दि॒वो दि॒वो न॒ आ नो॑ दि॒वः । \newline
20. नो॒ दि॒वो दि॒वो नो॑ नो दि॒वो ऽग्ना॑विष्णू॒ अग्ना॑विष्णू दि॒वो नो॑ नो दि॒वो ऽग्ना॑विष्णू । \newline
21. दि॒वो ऽग्ना॑विष्णू॒ अग्ना॑विष्णू दि॒वो दि॒वो ऽग्ना॑विष्णू । \newline
22. अग्ना॑विष्णू॒ अग्ना॑विष्णू । \newline
23. अग्ना॑विष्णू॒ इत्यग्ना᳚ - वि॒ष्णू॒ । \newline
24. अग्ना॑विष्णू इ॒म मि॒म मग्ना॑विष्णू॒ अग्ना॑विष्णू इ॒मम् मे॑ म इ॒म मग्ना॑विष्णू॒ अग्ना॑विष्णू इ॒मम् मे᳚ । \newline
25. अग्ना॑विष्णू॒ इत्यग्ना᳚ - वि॒ष्णू॒ । \newline
26. इ॒मम् मे॑ म इ॒म मि॒मम् मे॑ वरुण वरुण म इ॒म मि॒मम् मे॑ वरुण । \newline
27. मे॒ व॒रु॒ण॒ व॒रु॒ण॒ मे॒ मे॒ व॒रु॒ण॒ तत् तद् व॑रुण मे मे वरुण॒ तत् । \newline
28. व॒रु॒ण॒ तत् तद् व॑रुण वरुण॒ तत् त्वा᳚ त्वा॒ तद् व॑रुण वरुण॒ तत् त्वा᳚ । \newline
29. तत् त्वा᳚ त्वा॒ तत् तत् त्वा॑ यामि यामि त्वा॒ तत् तत् त्वा॑ यामि । \newline
30. त्वा॒ या॒मि॒ या॒मि॒ त्वा॒ त्वा॒ या॒म्युदुद् या॑मि त्वा त्वा या॒म्युत् । \newline
31. या॒म्युदुद् या॑मि या॒म्युदु॑ वु॒ वुद् या॑मि या॒म्युदु॑ । \newline
32. उदु॑ वु॒ वुदुदु॒ त्यम् त्य मु॒ वुदुदु॒ त्यम् । \newline
33. उ॒ त्यम् त्य मु॑ वु॒ त्यम् चि॒त्रम् चि॒त्रम् त्य मु॑ वु॒ त्यम् चि॒त्रम् । \newline
34. त्यम् चि॒त्रम् चि॒त्रम् त्यम् त्यम् चि॒त्रम् । \newline
35. चि॒त्रमिति॑ चि॒त्रम् । \newline
36. अ॒पाम् नपा॒न् नपा॑द॒पा म॒पाम् नपा॒दा नपा॑द॒पा म॒पाम् नपा॒दा । \newline
37. नपा॒दा नपा॒न् नपा॒दा हि ह्या नपा॒न् नपा॒दा हि । \newline
38. आ हि ह्या  ह्यस्था॒  दस्था॒द् ध्या ह्यस्था᳚त् । \newline
39. ह्यस्था॒ दस्था॒द्धि ह्यस्था॑ दु॒पस्थ॑ मु॒पस्थ॒ मस्था॒द्धि ह्यस्था॑ दु॒पस्थ᳚म् । \newline
40. अस्था॑दु॒पस्थ॑ मु॒पस्थ॒ मस्था॒ दस्था॑ दु॒पस्थ॑म् जि॒ह्माना᳚म् जि॒ह्माना॑ मु॒पस्थ॒ मस्था॒ दस्था॑ दु॒पस्थ॑म् जि॒ह्माना᳚म् । \newline
41. उ॒पस्थ॑म् जि॒ह्माना᳚म् जि॒ह्माना॑ मु॒पस्थ॑ मु॒पस्थ॑म् जि॒ह्माना॑ मू॒र्द्ध्व ऊ॒र्द्ध्वो जि॒ह्माना॑ मु॒पस्थ॑ मु॒पस्थ॑म् जि॒ह्माना॑ मू॒र्द्ध्वः । \newline
42. उ॒पस्थ॒मित्यु॒प - स्थ॒म् । \newline
43. जि॒ह्माना॑ मू॒र्द्ध्व ऊ॒र्द्ध्वो जि॒ह्माना᳚म् जि॒ह्माना॑ मू॒र्द्ध्वो वि॒द्युतं॑ ॅवि॒द्युत॑ मू॒र्द्ध्वो जि॒ह्माना᳚म् जि॒ह्माना॑ मू॒र्द्ध्वो वि॒द्युत᳚म् । \newline
44. ऊ॒र्द्ध्वो वि॒द्युतं॑ ॅवि॒द्युत॑ मू॒र्द्ध्व ऊ॒र्द्ध्वो वि॒द्युतं॒ ॅवसा॑नो॒ वसा॑नो वि॒द्युत॑ मू॒र्द्ध्व ऊ॒र्द्ध्वो वि॒द्युतं॒ ॅवसा॑नः । \newline
45. वि॒द्युतं॒ ॅवसा॑नो॒ वसा॑नो वि॒द्युतं॑ ॅवि॒द्युतं॒ ॅवसा॑नः । \newline
46. वि॒द्युत॒मिति॑ वि - द्युत᳚म् । \newline
47. वसा॑न॒ इति॒ वसा॑नः । \newline
48. तस्य॒ ज्येष्ठ॒म् ज्येष्ठ॒म् तस्य॒ तस्य॒ ज्येष्ठ॑म् महि॒मान॑म् महि॒मान॒म् ज्येष्ठ॒म् तस्य॒ तस्य॒ ज्येष्ठ॑म् महि॒मान᳚म् । \newline
49. ज्येष्ठ॑म् महि॒मान॑म् महि॒मान॒म् ज्येष्ठ॒म् ज्येष्ठ॑म् महि॒मानं॒ ॅवह॑न्ती॒र् वह॑न्तीर् महि॒मान॒म् ज्येष्ठ॒म् ज्येष्ठ॑म् महि॒मानं॒ ॅवह॑न्तीः । \newline
50. म॒हि॒मानं॒ ॅवह॑न्ती॒र् वह॑न्तीर् महि॒मान॑म् महि॒मानं॒ ॅवह॑न्ती॒र्॒. हिर॑ण्यवर्णा॒ हिर॑ण्यवर्णा॒ वह॑न्तीर् महि॒मान॑म् महि॒मानं॒ ॅवह॑न्ती॒र्॒. हिर॑ण्यवर्णाः । \newline
51. वह॑न्ती॒र्॒. हिर॑ण्यवर्णा॒ हिर॑ण्यवर्णा॒ वह॑न्ती॒र् वह॑न्ती॒र्॒. हिर॑ण्यवर्णाः॒ परि॒ परि॒ हिर॑ण्यवर्णा॒ वह॑न्ती॒र् वह॑न्ती॒र्॒. हिर॑ण्यवर्णाः॒ परि॑ । \newline
52. हिर॑ण्यवर्णाः॒ परि॒ परि॒ हिर॑ण्यवर्णा॒ हिर॑ण्यवर्णाः॒ परि॑ यन्ति यन्ति॒ परि॒ हिर॑ण्यवर्णा॒ हिर॑ण्यवर्णाः॒ परि॑ यन्ति । \newline
53. हिर॑ण्यवर्णा॒ इति॒ हिर॑ण्य - व॒र्णाः॒ । \newline
54. परि॑ यन्ति यन्ति॒ परि॒ परि॑ यन्ति य॒ह्वीर् य॒ह्वीर् य॑न्ति॒ परि॒ परि॑ यन्ति य॒ह्वीः । \newline
55. य॒न्ति॒ य॒ह्वीर् य॒ह्वीर् य॑न्ति यन्ति य॒ह्वीः । \newline
56. य॒ह्वीरिति॑ य॒ह्वीः । \newline
57. स म॒न्या अ॒न्याः सꣳ स म॒न्या यन्ति॒ यन्त्य॒न्याः सꣳ स म॒न्या यन्ति॑ । \newline
\pagebreak
\markright{ TS 2.5.12.2  \hfill https://www.vedavms.in \hfill}
\addcontentsline{toc}{section}{ TS 2.5.12.2 }
\section*{ TS 2.5.12.2 }

\textbf{TS 2.5.12.2 } \newline
\textbf{Samhita Paata} \newline

-म॒न्या यन्त्युप॑ यन्त्य॒न्याः स॑मा॒नमू॒र्वं न॒द्यः॑ पृणन्ति । तमू॒ शुचिꣳ॒॒ शुच॑यो दीदि॒वाꣳ स॑म॒पां नपा॑तं॒ परि॑तस्थु॒रापः॑ । तमस्मे॑रा युव॒तयो॒ युवा॑नं मर्मृ॒ज्यमा॑नाः॒ परि॑ य॒न्त्यापः॑ ॥ स शु॒क्रेण॒ शिक्व॑ना रे॒वद॒ग्निर्दी॒दाया॑नि॒द्ध्मो घृ॒तनि॑र्णिग॒फ्सु ॥इन्द्रा॒वरु॑णयोर॒हꣳस॒म्राजो॒रव॒ आ वृ॑णे । ता नो॑ मृडात ई॒दृशे᳚ ॥ इन्द्रा॑वरुणा यु॒वम॑द्ध्व॒राय॑ नो - [  ] \newline

\textbf{Pada Paata} \newline

अ॒न्याः । यन्ति॑ । उपेति॑ । य॒न्ति॒ । अ॒न्याः । स॒मा॒नम् । ऊ॒र्वम् । न॒द्यः॑ । पृ॒ण॒न्ति॒ ॥ तम् । उ॒ । शुचि᳚म् । शुच॑यः । दी॒दि॒वाꣳस᳚म् । अ॒पाम् । नपा॑तम् । परीति॑ । त॒स्थुः॒ । आपः॑ ॥ तम् । अस्मे॑राः । यु॒व॒तयः॑ । युवा॑नम् । म॒र्मृ॒ज्यमा॑नाः । परीति॑ । य॒न्ति॒ । आपः॑ ॥ सः । शु॒क्रेण॑ । शिक्व॑ना । रे॒वत् । अ॒ग्निः । दी॒दाय॑ । अ॒नि॒द्ध्मः । घृ॒तनि॑र्णि॒गिति॑ घृ॒त - नि॒र्णि॒क् । अ॒फ्स्वित्य॑प् - सु ॥ इन्द्रा॒वरु॑णयो॒रितीन्द्रा᳚ - वरु॑णयोः । अ॒हम् । स॒म्राजो॒रिति॑ सं - राजोः᳚ । अवः॑ । एति॑ । वृ॒णे॒ ॥ ता । नः॒ । मृ॒डा॒तः॒ । ई॒दृशे᳚ ॥ इन्द्रा॑वरु॒णेतीन्द्रा᳚ - व॒रु॒णा॒ । यु॒वम् । अ॒द्ध्व॒राय॑ । नः॒ ।  \newline


\textbf{Krama Paata} \newline

अ॒न्या यन्ति॑ । यन्त्युप॑ । उप॑ यन्ति । य॒न्त्य॒न्याः । अ॒न्याः स॑मा॒नम् । स॒मा॒नमू॒र्वम् । ऊ॒र्वम् न॒द्यः॑ । न॒द्यः॑ पृणन्ति । पृ॒ण॒न्तीति॑ पृणन्ति ॥ तमु॑ । ऊ॒ शुचि᳚म् । शुचिꣳ॒॒ शुच॑यः । शुच॑यो दीदि॒वाꣳस᳚म् । दी॒दि॒वाꣳस॑म॒पाम् । अ॒पाम् नपा॑तम् । नपा॑त॒म् परि॑ । परि॑ तस्थुः । त॒स्थु॒रापः॑ । आप॒ इत्यापः॑ ॥ तमस्मे॑राः । अस्मे॑रा युव॒तयः॑ । यु॒व॒तयो॒ युवा॑नम् । युवा॑नम् मर्मृ॒ज्यमा॑नाः । म॒र्मृ॒ज्यमा॑नाः॒ परि॑ । परि॑ यन्ति । य॒न्त्यापः॑ । आप॒ इत्यापः॑ ॥ स शु॒क्रेण॑ । शु॒क्रेण॒ शिक्व॑ना । शिक्व॑ना रे॒वत् । रे॒वद॒ग्निः । अ॒ग्निर् दी॒दाय॑ । दी॒दाया॑नि॒द्ध्मः । अ॒नि॒द्ध्मो घृ॒तनि॑र्णिक् । घृ॒तनि॑र्णिग॒फ्सु । घृ॒तनि॑र्णि॒गिति॑ घृ॒त - नि॒र्णि॒क्॒ । अ॒फ्स्वित्य॑प् - सु ॥ इन्द्रा॒वरु॑णयोर॒हम् । इन्द्रा॒वरु॑णयो॒रितीन्द्रा᳚ - वरु॑णयोः । अ॒हꣳ स॒म्राजोः᳚ । स॒म्राजो॒रवः॑ । स॒म्राजो॒रिति॑ सम् - राजोः᳚ । अव॒ आ । आ वृ॑णे । वृ॒ण॒ इति॑ वृणे ॥ ता नः॑ । नो॒ मृ॒डा॒तः॒ । मृ॒डा॒त॒ ई॒दृशे᳚ । ई॒दृश॒ इती॒दृशे᳚ ॥ इन्द्रा॑वरुणा यु॒वम् । इन्द्रा॑वरु॒णेतीन्द्रा᳚ - व॒रु॒णा॒ । यु॒वम॑द्ध्व॒रायः॑ । अ॒द्ध्व॒राय॑ नः । नो॒ वि॒शे \newline

\textbf{Jatai Paata} \newline

1. अ॒न्या यन्ति॒ यन्त्य॒न्या अ॒न्या यन्ति॑ । \newline
2. यन्त्युपोप॒ यन्ति॒ यन्त्युप॑ । \newline
3. उप॑ यन्ति य॒न्त्युपोप॑ यन्ति । \newline
4. य॒न्त्य॒न्या अ॒न्या य॑न्ति यन्त्य॒न्याः । \newline
5. अ॒न्याः स॑मा॒नꣳ स॑मा॒न म॒न्या अ॒न्याः स॑मा॒नम् । \newline
6. स॒मा॒न मू॒र्व मू॒र्वꣳ स॑मा॒नꣳ स॑मा॒न मू॒र्वम् । \newline
7. ऊ॒र्वम् न॒द्यो॑ न॒द्य॑ ऊ॒र्व मू॒र्वम् न॒द्यः॑ । \newline
8. न॒द्यः॑ पृणन्ति पृणन्ति न॒द्यो॑ न॒द्यः॑ पृणन्ति । \newline
9. पृ॒ण॒न्तीति॑ पृणन्ति । \newline
10. त मु॑ वु॒ तम् त मु॑ । \newline
11. ऊ॒ शुचिꣳ॒॒ शुचि॑ मु वू॒ शुचि᳚म् । \newline
12. शुचिꣳ॒॒ शुच॑यः॒ शुच॑यः॒ शुचिꣳ॒॒ शुचिꣳ॒॒ शुच॑यः । \newline
13. शुच॑यो दीदि॒वाꣳस॑म् दीदि॒वाꣳसꣳ॒॒ शुच॑यः॒ शुच॑यो दीदि॒वाꣳस᳚म् । \newline
14. दी॒दि॒वाꣳस॑ म॒पा म॒पाम् दी॑दि॒वाꣳस॑म् दीदि॒वाꣳस॑ म॒पाम् । \newline
15. अ॒पाम् नपा॑त॒म् नपा॑त म॒पा म॒पाम् नपा॑तम् । \newline
16. नपा॑त॒म् परि॒ परि॒ णपा॑त॒म् नपा॑त॒म् परि॑ । \newline
17. परि॑ तस्थु स्तस्थुः॒ परि॒ परि॑ तस्थुः । \newline
18. त॒स्थु॒राप॒ आप॑ स्तस्थु स्तस्थु॒ रापः॑ । \newline
19. आप॒ इत्यापः॑ । \newline
20. त मस्मे॑रा॒ अस्मे॑रा॒ स्तम् त मस्मे॑राः । \newline
21. अस्मे॑रा युव॒तयो॑ युव॒तयो॒ अस्मे॑रा॒ अस्मे॑रा युव॒तयः॑ । \newline
22. यु॒व॒तयो॒ युवा॑नं॒ ॅयुवा॑नं ॅयुव॒तयो॑ युव॒तयो॒ युवा॑नम् । \newline
23. युवा॑नम् मर्मृ॒ज्यमा॑ना मर्मृ॒ज्यमा॑ना॒ युवा॑नं॒ ॅयुवा॑नम् मर्मृ॒ज्यमा॑नाः । \newline
24. म॒र्मृ॒ज्यमा॑नाः॒ परि॒ परि॑ मर्मृ॒ज्यमा॑ना मर्मृ॒ज्यमा॑नाः॒ परि॑ । \newline
25. परि॑ यन्ति यन्ति॒ परि॒ परि॑ यन्ति । \newline
26. य॒न्त्याप॒ आपो॑ यन्ति य॒न्त्यापः॑ । \newline
27. आप॒ इत्यापः॑ । \newline
28. स शु॒क्रेण॑ शु॒क्रेण॒ स स शु॒क्रेण॑ । \newline
29. शु॒क्रेण॒ शिक्व॑ना॒ शिक्व॑ना शु॒क्रेण॑ शु॒क्रेण॒ शिक्व॑ना । \newline
30. शिक्व॑ना रे॒वद् रे॒वच्छिक्व॑ना॒ शिक्व॑ना रे॒वत् । \newline
31. रे॒व द॒ग्नि र॒ग्नी रे॒वद् रे॒वद॒ग्निः । \newline
32. अ॒ग्निर् दी॒दाय॑ दी॒दाया॒ग्नि र॒ग्निर् दी॒दाय॑ । \newline
33. दी॒दाया॑नि॒द्ध्मो॑ ऽनि॒द्ध्मो दी॒दाय॑ दी॒दाया॑नि॒द्ध्मः । \newline
34. अ॒नि॒द्ध्मो घृ॒तनि॑र्णिग् घृ॒तनि॑र्णि गनि॒द्ध्मो॑ ऽनि॒द्ध्मो घृ॒तनि॑र्णिक् । \newline
35. घृ॒तनि॑र्णि ग॒फ्स्व॑फ्सु घृ॒तनि॑र्णिग् घृ॒तनि॑र्णि ग॒फ्सु । \newline
36. घृ॒तनि॑र्णि॒गिति॑ घृ॒त - नि॒र्णि॒क् । \newline
37. अ॒फ्स्वित्य॑प् - सु । \newline
38. इन्द्रा॒वरु॑णयोर॒ह म॒ह मिन्द्रा॒वरु॑णयो॒ रिन्द्रा॒वरु॑णयोर॒हम् । \newline
39. इन्द्रा॒वरु॑णयो॒रितीन्द्रा᳚ - वरु॑णयोः । \newline
40. अ॒हꣳ स॒म्राजोः᳚ स॒म्राजो॑र॒ह म॒हꣳ स॒म्राजोः᳚ । \newline
41. स॒म्राजो॒रवो ऽवः॑ स॒म्राजोः᳚ स॒म्राजो॒रवः॑ । \newline
42. स॒म्राजो॒रिति॑ सं - राजोः᳚ । \newline
43. अव॒ आ ऽवो ऽव॒ आ । \newline
44. आ वृ॑णे वृण॒ आ वृ॑णे । \newline
45. वृ॒ण॒ इति॑ वृणे । \newline
46. ता नो॑ न॒ स्ता ता नः॑ । \newline
47. नो॒ मृ॒डा॒तो॒ मृ॒डा॒तो॒ नो॒ नो॒ मृ॒डा॒तः॒ । \newline
48. मृ॒डा॒त॒ ई॒दृश॑ ई॒दृशे॑ मृडातो मृडात ई॒दृशे᳚ । \newline
49. ई॒दृश॒ इती॒दृशे᳚ । \newline
50. इन्द्रा॑वरुणा यु॒वं ॅयु॒व मिन्द्रा॑वरु॒ णेन्द्रा॑वरुणा यु॒वम् । \newline
51. इन्द्रा॑वरु॒णेतीन्द्रा᳚ - व॒रु॒णा॒ । \newline
52. यु॒व म॑द्ध्व॒राया᳚ द्ध्व॒राय॑ यु॒वं ॅयु॒व म॑द्ध्व॒राय॑ । \newline
53. अ॒द्ध्व॒राय॑ नो नो अद्ध्व॒राया᳚ द्ध्व॒राय॑ नः । \newline
54. नो॒ वि॒शे वि॒शे नो॑ नो वि॒शे । \newline

\textbf{Ghana Paata } \newline

1. अ॒न्या यन्ति॒ यन्त्य॒न्या अ॒न्या यन्त्युपोप॒ यन्त्य॒न्या अ॒न्या यन्त्युप॑ । \newline
2. यन्त्युपोप॒ यन्ति॒ यन्त्युप॑ यन्ति य॒न्त्युप॒ यन्ति॒ यन्त्युप॑ यन्ति । \newline
3. उप॑ यन्ति य॒न्त्युपोप॑ यन्त्य॒न्या अ॒न्या य॒न्त्युपोप॑ यन्त्य॒न्याः । \newline
4. य॒न्त्य॒न्या अ॒न्या य॑न्ति यन्त्य॒न्याः स॑मा॒नꣳ स॑मा॒न म॒न्या य॑न्ति यन्त्य॒न्याः स॑मा॒नम् । \newline
5. अ॒न्याः स॑मा॒नꣳ स॑मा॒न म॒न्या अ॒न्याः स॑मा॒न मू॒र्व मू॒र्वꣳ स॑मा॒न म॒न्या अ॒न्याः स॑मा॒न मू॒र्वम् । \newline
6. स॒मा॒न मू॒र्व मू॒र्वꣳ स॑मा॒नꣳ स॑मा॒न मू॒र्वम् न॒द्यो॑ न॒द्य॑ ऊ॒र्वꣳ स॑मा॒नꣳ स॑मा॒न मू॒र्वम् न॒द्यः॑ । \newline
7. ऊ॒र्वम् न॒द्यो॑ न॒द्य॑ ऊ॒र्व मू॒र्वम् न॒द्यः॑ पृणन्ति पृणन्ति न॒द्य॑ ऊ॒र्व मू॒र्वम् न॒द्यः॑ पृणन्ति । \newline
8. न॒द्यः॑ पृणन्ति पृणन्ति न॒द्यो॑ न॒द्यः॑ पृणन्ति । \newline
9. पृ॒ण॒न्तीति॑ पृणन्ति । \newline
10. त मु॑ वु॒ तम् त मू॒ शुचिꣳ॒॒ शुचि॑ मु॒ तम् त मू॒ शुचि᳚म् । \newline
11. ऊ॒ शुचिꣳ॒॒ शुचि॑ मु वू॒ शुचिꣳ॒॒ शुच॑यः॒ शुच॑यः॒ शुचि॑ मु वू॒ शुचिꣳ॒॒ शुच॑यः । \newline
12. शुचिꣳ॒॒ शुच॑यः॒ शुच॑यः॒ शुचिꣳ॒॒ शुचिꣳ॒॒ शुच॑यो दीदि॒वाꣳस॑म् दीदि॒वाꣳसꣳ॒॒ शुच॑यः॒ शुचिꣳ॒॒ शुचिꣳ॒॒ शुच॑यो दीदि॒वाꣳस᳚म् । \newline
13. शुच॑यो दीदि॒वाꣳस॑म् दीदि॒वाꣳसꣳ॒॒ शुच॑यः॒ शुच॑यो दीदि॒वाꣳस॑ म॒पा म॒पाम् दी॑दि॒वाꣳसꣳ॒॒ शुच॑यः॒ शुच॑यो दीदि॒वाꣳस॑ म॒पाम् । \newline
14. दी॒दि॒वाꣳस॑ म॒पा म॒पाम् दी॑दि॒वाꣳस॑म् दीदि॒वाꣳस॑ म॒पाम् नपा॑त॒म् नपा॑त म॒पाम् दी॑दि॒वाꣳस॑म् दीदि॒वाꣳस॑ म॒पाम् नपा॑तम् । \newline
15. अ॒पाम् नपा॑त॒म् नपा॑त म॒पा म॒पाम् नपा॑त॒म् परि॒ परि॒ णपा॑त म॒पा म॒पाम् नपा॑त॒म् परि॑ । \newline
16. नपा॑त॒म् परि॒ परि॒ णपा॑त॒म् नपा॑त॒म् परि॑ तस्थु स्तस्थुः॒ परि॒ णपा॑त॒म् नपा॑त॒म् परि॑ तस्थुः । \newline
17. परि॑ तस्थु स्तस्थुः॒ परि॒ परि॑ तस्थु॒राप॒ आप॑ स्तस्थुः॒ परि॒ परि॑ तस्थु॒रापः॑ । \newline
18. त॒स्थु॒ राप॒ आप॑ स्तस्थु स्तस्थु॒ रापः॑ । \newline
19. आप॒ इत्यापः॑ । \newline
20. त मस्मे॑रा॒ अस्मे॑रा॒ स्तम् त मस्मे॑रा युव॒तयो॑ युव॒तयो॒ अस्मे॑रा॒ स्तम् त मस्मे॑रा युव॒तयः॑ । \newline
21. अस्मे॑रा युव॒तयो॑ युव॒तयो॒ अस्मे॑रा॒ अस्मे॑रा युव॒तयो॒ युवा॑नं॒ ॅयुवा॑नं ॅयुव॒तयो॒ अस्मे॑रा॒ अस्मे॑रा युव॒तयो॒ युवा॑नम् । \newline
22. यु॒व॒तयो॒ युवा॑नं॒ ॅयुवा॑नं ॅयुव॒तयो॑ युव॒तयो॒ युवा॑नम् मर्मृ॒ज्यमा॑ना मर्मृ॒ज्यमा॑ना॒ युवा॑नं ॅयुव॒तयो॑ युव॒तयो॒ युवा॑नम् मर्मृ॒ज्यमा॑नाः । \newline
23. युवा॑नम् मर्मृ॒ज्यमा॑ना मर्मृ॒ज्यमा॑ना॒ युवा॑नं॒ ॅयुवा॑नम् मर्मृ॒ज्यमा॑नाः॒ परि॒ परि॑ मर्मृ॒ज्यमा॑ना॒ युवा॑नं॒ ॅयुवा॑नम् मर्मृ॒ज्यमा॑नाः॒ परि॑ । \newline
24. म॒र्मृ॒ज्यमा॑नाः॒ परि॒ परि॑ मर्मृ॒ज्यमा॑ना मर्मृ॒ज्यमा॑नाः॒ परि॑ यन्ति यन्ति॒ परि॑ मर्मृ॒ज्यमा॑ना मर्मृ॒ज्यमा॑नाः॒ परि॑ यन्ति । \newline
25. परि॑ यन्ति यन्ति॒ परि॒ परि॑ य॒न्त्याप॒ आपो॑ यन्ति॒ परि॒ परि॑ य॒न्त्यापः॑ । \newline
26. य॒न्त्याप॒ आपो॑ यन्ति य॒न्त्यापः॑ । \newline
27. आप॒ इत्यापः॑ । \newline
28. स शु॒क्रेण॑ शु॒क्रेण॒ स स शु॒क्रेण॒ शिक्व॑ना॒ शिक्व॑ना शु॒क्रेण॒ स स शु॒क्रेण॒ शिक्व॑ना । \newline
29. शु॒क्रेण॒ शिक्व॑ना॒ शिक्व॑ना शु॒क्रेण॑ शु॒क्रेण॒ शिक्व॑ना रे॒वद् रे॒वच्छिक्व॑ना शु॒क्रेण॑ शु॒क्रेण॒ शिक्व॑ना रे॒वत् । \newline
30. शिक्व॑ना रे॒वद् रे॒वच्छिक्व॑ना॒ शिक्व॑ना रे॒व द॒ग्नि र॒ग्नी रे॒वच्छिक्व॑ना॒ शिक्व॑ना रे॒व द॒ग्निः । \newline
31. रे॒व द॒ग्नि र॒ग्नी रे॒वद् रे॒व द॒ग्निर् दी॒दाय॑ दी॒दाया॒ग्नी रे॒वद् रे॒व द॒ग्निर् दी॒दाय॑ । \newline
32. अ॒ग्निर् दी॒दाय॑ दी॒दाया॒ ग्निर॒ग्निर् दी॒दाया॑ नि॒द्ध्मो॑ ऽनि॒द्ध्मो दी॒दाया॒ग्नि र॒ग्निर् दी॒दाया॑ नि॒द्ध्मः । \newline
33. दी॒दाया॑ नि॒द्ध्मो॑ ऽनि॒द्ध्मो दी॒दाय॑ दी॒दाया॑ नि॒द्ध्मो घृ॒तनि॑र्णिग् घृ॒तनि॑र्णि गनि॒द्ध्मो दी॒दाय॑ दी॒दाया॑ नि॒द्ध्मो घृ॒तनि॑र्णिक् । \newline
34. अ॒नि॒द्ध्मो घृ॒तनि॑र्णिग् घृ॒तनि॑र्णि गनि॒द्ध्मो॑ ऽनि॒द्ध्मो घृ॒तनि॑र्णि ग॒फ्स्व॑फ्सु घृ॒तनि॑र्णि गनि॒द्ध्मो॑ ऽनि॒द्ध्मो घृ॒तनि॑र्णि ग॒फ्सु । \newline
35. घृ॒तनि॑र्णि ग॒फ्स्व॑फ्सु घृ॒तनि॑र्णिग् घृ॒तनि॑र्णि ग॒फ्सु । \newline
36. घृ॒तनि॑र्णि॒गिति॑ घृ॒त - नि॒र्णि॒क् । \newline
37. अ॒फ्स्वित्य॑प् - सु । \newline
38. इन्द्रा॒वरु॑णयो र॒ह म॒ह मिन्द्रा॒वरु॑णयो॒ रिन्द्रा॒वरु॑णयो र॒हꣳ स॒म्राजोः᳚ स॒म्राजो॑र॒ह मिन्द्रा॒वरु॑णयो॒ रिन्द्रा॒वरु॑णयो र॒हꣳ स॒म्राजोः᳚ । \newline
39. इन्द्रा॒वरु॑णयो॒रितीन्द्रा᳚ - वरु॑णयोः । \newline
40. अ॒हꣳ स॒म्राजोः᳚ स॒म्राजो॑ र॒ह म॒हꣳ स॒म्राजो॒ रवो ऽवः॑ स॒म्राजो॑ र॒ह म॒हꣳ स॒म्राजो॒ रवः॑ । \newline
41. स॒म्राजो॒ रवो ऽवः॑ स॒म्राजोः᳚ स॒म्राजो॒ रव॒ आ ऽवः॑ स॒म्राजोः᳚ स॒म्राजो॒ रव॒ आ । \newline
42. स॒म्राजो॒रिति॑ सं - राजोः᳚ । \newline
43. अव॒ आ ऽवो ऽव॒ आ वृ॑णे वृण॒ आ ऽवो ऽव॒ आ वृ॑णे । \newline
44. आ वृ॑णे वृण॒ आ वृ॑णे । \newline
45. वृ॒ण॒ इति॑ वृणे । \newline
46. ता नो॑ न॒ स्ता ता नो॑ मृडातो मृडातो न॒ स्ता ता नो॑ मृडातः । \newline
47. नो॒ मृ॒डा॒तो॒ मृ॒डा॒तो॒ नो॒ नो॒ मृ॒डा॒त॒ ई॒दृश॑ ई॒दृशे॑ मृडातो नो नो मृडात ई॒दृशे᳚ । \newline
48. मृ॒डा॒त॒ ई॒दृश॑ ई॒दृशे॑ मृडातो मृडात ई॒दृशे᳚ । \newline
49. ई॒दृश॒ इती॒दृशे᳚ । \newline
50. इन्द्रा॑वरुणा यु॒वं ॅयु॒व मिन्द्रा॑वरु॒ णेन्द्रा॑वरुणा यु॒व म॑द्ध्व॒राया᳚ द्ध्व॒राय॑ यु॒व मिन्द्रा॑वरु॒ णेन्द्रा॑वरुणा यु॒व म॑द्ध्व॒राय॑ । \newline
51. इन्द्रा॑वरु॒णेतीन्द्रा᳚ - व॒रु॒णा॒ । \newline
52. यु॒व म॑द्ध्व॒राया᳚ द्ध्व॒राय॑ यु॒वं ॅयु॒व म॑द्ध्व॒राय॑ नो नो अद्ध्व॒राय॑ यु॒वं ॅयु॒व म॑द्ध्व॒राय॑ नः । \newline
53. अ॒द्ध्व॒राय॑ नो नो अद्ध्व॒राया᳚ द्ध्व॒राय॑ नो वि॒शे वि॒शे नो॑ अद्ध्व॒राया᳚ द्ध्व॒राय॑ नो वि॒शे । \newline
54. नो॒ वि॒शे वि॒शे नो॑ नो वि॒शे जना॑य॒ जना॑य वि॒शे नो॑ नो वि॒शे जना॑य । \newline
\pagebreak
\markright{ TS 2.5.12.3  \hfill https://www.vedavms.in \hfill}
\addcontentsline{toc}{section}{ TS 2.5.12.3 }
\section*{ TS 2.5.12.3 }

\textbf{TS 2.5.12.3 } \newline
\textbf{Samhita Paata} \newline

वि॒शे जना॑य॒ महि॒ शर्म॑ यच्छतं । दी॒र्घप्र॑यज्यु॒मति॒ यो व॑नु॒ष्यति॑ व॒यं ज॑येम॒ पृत॑नासु दू॒ढ्यः॑ ॥ आ नो॑मित्रावरुणा॒>15, प्रबा॒हवा᳚>16 ॥त्वं नो॑ अग्ने॒ वरु॑णस्य वि॒द्वान् दे॒वस्य॒ हेडोऽव॑ यासि सीष्ठाः । यजि॑ष्ठो॒ वह्नि॑ तमः॒ शोशु॑चानो॒ विश्वा॒ द्वेषाꣳ॑सि॒ प्रमु॑मुग्ध्य॒स्मत् ॥स त्वंनो॑ अग्नेऽव॒मो भ॑वो॒ती नेदि॑ष्ठो अ॒स्या उ॒षसो॒ व्यु॑ष्टौ । अव॑ यक्ष्व नो॒ वरु॑णꣳ॒॒ - [  ] \newline

\textbf{Pada Paata} \newline

वि॒शे । जना॑य । महि॑ । शर्म॑ । य॒च्छ॒त॒म् ॥ दी॒र्घप्र॑यज्यु॒मिति॑ दी॒र्घ - प्र॒य॒ज्यु॒म् । अतीति॑ । यः । व॒नु॒ष्यति॑ । व॒यम् । ज॒ये॒म॒ । पृत॑नासु । दू॒ढ्यः॑ ॥ एति॑ । नः॒ । मि॒त्रा॒व॒रु॒णेति॑ मित्रा - व॒रु॒णा॒ । प्रेति॑ । बा॒हवा᳚ ॥ त्वम् । नः॒ । अ॒ग्ने॒ । वरु॑णस्य । वि॒द्वान् । दे॒वस्य॑ । हेडः॑ । अवेति॑ । या॒सि॒सी॒ष्ठाः॒ ॥ यजि॑ष्ठः । वह्नि॑तम॒ इति॒ वह्नि॑-त॒मः॒ । शोशु॑चानः । विश्वा᳚ । द्वेषाꣳ॑सि । प्रेति॑ । मु॒मु॒ग्धि॒ । अ॒स्मत् ॥ सः । त्वम् । नः॒ । अ॒ग्ने॒ । अ॒व॒मः । भ॒व॒ । ऊ॒ती । नेदि॑ष्ठः । अ॒स्याः । उ॒षसः॑ । व्यु॑ष्टा॒विति॒ वि - उ॒ष्टौ॒ ॥ अवेति॑ । य॒क्ष्व॒ । नः॒ । वरु॑णम् ।  \newline


\textbf{Krama Paata} \newline

वि॒शे जना॑य । जना॑य॒ महि॑ । महि॒ शर्म॑ । शर्म॑ यच्छतम् । य॒च्छ॒त॒मिति॑ यच्छतम् ॥ दी॒र्घप्र॑यज्यु॒मति॑ । दी॒र्घप्र॑यज्यु॒मिति॑ दी॒र्घ - प्र॒य॒ज्यु॒॒म् । अति॒ यः । यो व॑नु॒ष्यति॑ । व॒नु॒ष्यति॑ व॒यम् । व॒यम् ज॑येम । ज॒ये॒म॒ पृत॑नासु । पृत॑नासु दू॒ढ्यः॑ । दू॒ढ्य॑ इति॑ दू॒ढ्यः॑ ॥ आ नः॑ । नो॒ मि॒त्रा॒व॒रु॒णा॒ । मि॒त्रा॒व॒रु॒णा॒ प्र । मि॒त्रा॒व॒रु॒णेति॑ मित्रा - व॒रु॒णा॒ । प्र बा॒हवा᳚ । बा॒हवेति॑ बा॒हवा᳚ ॥ त्वम् नः॑ । नो॒ अ॒ग्ने॒ । अ॒ग्ने॒ वरु॑णस्य । वरु॑णस्य वि॒द्वान् । वि॒द्वान् दे॒वस्य॑ । दे॒वस्य॒ हेडः॑ । हेडो ऽव॑ । अव॑ यासिसीष्ठाः । या॒सि॒सी॒ष्ठा॒ इति॑ यासिसीष्ठाः ॥ यजि॑ष्ठो॒ वह्नि॑तमः । वह्नि॑तमः॒ शोशु॑चानः । वह्नि॑तम॒ इति॒ वह्नि॑ - त॒मः॒ । शोशु॑चानो॒ विश्वा᳚ । विश्वा॒ द्वेषाꣳ॑सि । द्वेषाꣳ॑सि॒ प्र । प्र मु॑मुग्धि । मु॒मु॒ग्ध्य॒स्मत् । अ॒स्मदित्य॒स्मत् ॥ स त्वम् । त्वम् नः॑ । नो॒ अ॒ग्ने॒ । अ॒ग्ने॒ऽव॒मः । अ॒व॒मो भ॑व । भ॒वो॒ती । ऊ॒ती नेदि॑ष्ठः । नेदि॑ष्ठो अ॒स्याः । अ॒स्या उ॒षसः॑ । उ॒षसो॒ व्यु॑ष्टौ । व्यु॑ष्टा॒विति॒ वि - उ॒ष्टौ॒ ॥ अव॑ यक्ष्व । य॒क्ष्व॒ नः॒ । नो॒ वरु॑णम् । वरु॑णꣳ॒॒ ररा॑णः \newline

\textbf{Jatai Paata} \newline

1. वि॒शे जना॑य॒ जना॑य वि॒शे वि॒शे जना॑य । \newline
2. जना॑य॒ महि॒ महि॒ जना॑य॒ जना॑य॒ महि॑ । \newline
3. महि॒ शर्म॒ शर्म॒ महि॒ महि॒ शर्म॑ । \newline
4. शर्म॑ यच्छतं ॅयच्छतꣳ॒॒ शर्म॒ शर्म॑ यच्छतम् । \newline
5. य॒च्छ॒त॒मिति॑ यच्छतम् । \newline
6. दी॒र्घप्र॑यज्यु॒ मत्यति॑ दी॒र्घप्र॑यज्युम् दी॒र्घप्र॑यज्यु॒ मति॑ । \newline
7. दी॒र्घप्र॑यज्यु॒मिति॑ दी॒र्घ - प्र॒य॒ज्यु॒म् । \newline
8. अति॒ यो यो अत्यति॒ यः । \newline
9. यो व॑नु॒ष्यति॑ वनु॒ष्यति॒ यो यो व॑नु॒ष्यति॑ । \newline
10. व॒नु॒ष्यति॑ व॒यं ॅव॒यं ॅव॑नु॒ष्यति॑ वनु॒ष्यति॑ व॒यम् । \newline
11. व॒यम् ज॑येम जयेम व॒यं ॅव॒यम् ज॑येम । \newline
12. ज॒ये॒म॒ पृत॑नासु॒ पृत॑नासु जयेम जयेम॒ पृत॑नासु । \newline
13. पृत॑नासु दू॒ढ्यो॑ दू॒ढ्यः॑ पृत॑नासु॒ पृत॑नासु दू॒ढ्यः॑ । \newline
14. दू॒ढ्य॑ इति॑ दू॒ढ्यः॑ । \newline
15. आ नो॑ न॒ आ नः॑ । \newline
16. नो॒ मि॒त्रा॒व॒रु॒णा॒ मि॒त्रा॒व॒रु॒णा॒ नो॒ नो॒ मि॒त्रा॒व॒रु॒णा॒ । \newline
17. मि॒त्रा॒व॒रु॒णा॒ प्र प्र मि॑त्रावरुणा मित्रावरुणा॒ प्र । \newline
18. मि॒त्रा॒व॒रु॒णेति॑ मित्रा - व॒रु॒णा॒ । \newline
19. प्र बा॒हवा॑ बा॒हवा॒ प्र प्र बा॒हवा᳚ । \newline
20. बा॒हवेति॑ बा॒हवा᳚ । \newline
21. त्वम् नो॑ न॒ स्त्वम् त्वम् नः॑ । \newline
22. नो॒ अ॒ग्ने॒ अ॒ग्ने॒ नो॒ नो॒ अ॒ग्ने॒ । \newline
23. अ॒ग्ने॒ वरु॑णस्य॒ वरु॑णस्याग्ने अग्ने॒ वरु॑णस्य । \newline
24. वरु॑णस्य वि॒द्वान्. वि॒द्वान्. वरु॑णस्य॒ वरु॑णस्य वि॒द्वान् । \newline
25. वि॒द्वान् दे॒वस्य॑ दे॒वस्य॑ वि॒द्वान्. वि॒द्वान् दे॒वस्य॑ । \newline
26. दे॒वस्य॒ हेडो॒ हेडो॑ दे॒वस्य॑ दे॒वस्य॒ हेडः॑ । \newline
27. हेडो ऽवाव॒ हेडो॒ हेडो ऽव॑ । \newline
28. अव॑ यासिसीष्ठा यासिसीष्ठा॒ अवाव॑ यासिसीष्ठाः । \newline
29. या॒सि॒सी॒ष्ठा॒ इति॑ यासिसीष्ठाः । \newline
30. यजि॑ष्ठो॒ वह्नि॑तमो॒ वह्नि॑तमो॒ यजि॑ष्ठो॒ यजि॑ष्ठो॒ वह्नि॑तमः । \newline
31. वह्नि॑तमः॒ शोशु॑चानः॒ शोशु॑चानो॒ वह्नि॑तमो॒ वह्नि॑तमः॒ शोशु॑चानः । \newline
32. वह्नि॑तम॒ इति॒ वह्नि॑ - त॒मः॒ । \newline
33. शोशु॑चानो॒ विश्वा॒ विश्वा॒ शोशु॑चानः॒ शोशु॑चानो॒ विश्वा᳚ । \newline
34. विश्वा॒ द्वेषाꣳ॑सि॒ द्वेषाꣳ॑सि॒ विश्वा॒ विश्वा॒ द्वेषाꣳ॑सि । \newline
35. द्वेषाꣳ॑सि॒ प्र प्र द्वेषाꣳ॑सि॒ द्वेषाꣳ॑सि॒ प्र । \newline
36. प्र मु॑मुग्धि मुमुग्धि॒ प्र प्र मु॑मुग्धि । \newline
37. मु॒मु॒ग्ध्य॒स्म द॒स्मन् मु॑मुग्धि मुमुग्ध्य॒स्मत् । \newline
38. अ॒स्मदित्य॒स्मत् । \newline
39. स त्वम् त्वꣳ स स त्वम् । \newline
40. त्वम् नो॑ न॒स्त्वम् त्वम् नः॑ । \newline
41. नो॒ अ॒ग्ने॒ अ॒ग्ने॒ नो॒ नो॒ अ॒ग्ने॒ । \newline
42. अ॒ग्ने॒ ऽव॒मो॑ ऽव॒मो᳚ ऽग्ने अग्ने ऽव॒मः । \newline
43. अ॒व॒मो भ॑व भवाव॒मो॑ ऽव॒मो भ॑व । \newline
44. भ॒वो॒त्यू॑ती भ॑व भवो॒ती । \newline
45. ऊ॒ती नेदि॑ष्ठो॒ नेदि॑ष्ठ ऊ॒त्यू॑ती नेदि॑ष्ठः । \newline
46. नेदि॑ष्ठो अ॒स्या अ॒स्या नेदि॑ष्ठो॒ नेदि॑ष्ठो अ॒स्याः । \newline
47. अ॒स्या उ॒षस॑ उ॒षसो॑ अ॒स्या अ॒स्या उ॒षसः॑ । \newline
48. उ॒षसो॒ व्यु॑ष्टौ॒ व्यु॑ष्टा वु॒षस॑ उ॒षसो॒ व्यु॑ष्टौ । \newline
49. व्यु॑ष्टा॒विति॒ वि - उ॒ष्टौ॒ । \newline
50. अव॑ यक्ष्व य॒क्ष्वावाव॑ यक्ष्व । \newline
51. य॒क्ष्व॒ नो॒ नो॒ य॒क्ष्व॒ य॒क्ष्व॒ नः॒ । \newline
52. नो॒ वरु॑णं॒ ॅवरु॑णम् नो नो॒ वरु॑णम् । \newline
53. वरु॑णꣳ॒॒ ररा॑णो॒ ररा॑णो॒ वरु॑णं॒ ॅवरु॑णꣳ॒॒ ररा॑णः । \newline

\textbf{Ghana Paata } \newline

1. वि॒शे जना॑य॒ जना॑य वि॒शे वि॒शे जना॑य॒ महि॒ महि॒ जना॑य वि॒शे वि॒शे जना॑य॒ महि॑ । \newline
2. जना॑य॒ महि॒ महि॒ जना॑य॒ जना॑य॒ महि॒ शर्म॒ शर्म॒ महि॒ जना॑य॒ जना॑य॒ महि॒ शर्म॑ । \newline
3. महि॒ शर्म॒ शर्म॒ महि॒ महि॒ शर्म॑ यच्छतं ॅयच्छतꣳ॒॒ शर्म॒ महि॒ महि॒ शर्म॑ यच्छतम् । \newline
4. शर्म॑ यच्छतं ॅयच्छतꣳ॒॒ शर्म॒ शर्म॑ यच्छतम् । \newline
5. य॒च्छ॒त॒मिति॑ यच्छतम् । \newline
6. दी॒र्घप्र॑यज्यु॒ मत्यति॑ दी॒र्घप्र॑यज्युम् दी॒र्घप्र॑यज्यु॒ मति॒ यो यो अति॑ दी॒र्घप्र॑यज्युम् दी॒र्घप्र॑यज्यु॒ मति॒ यः । \newline
7. दी॒र्घप्र॑यज्यु॒मिति॑ दी॒र्घ - प्र॒य॒ज्यु॒म् । \newline
8. अति॒ यो यो अत्यति॒ यो व॑नु॒ष्यति॑ वनु॒ष्यति॒ यो अत्यति॒ यो व॑नु॒ष्यति॑ । \newline
9. यो व॑नु॒ष्यति॑ वनु॒ष्यति॒ यो यो व॑नु॒ष्यति॑ व॒यं ॅव॒यं ॅव॑नु॒ष्यति॒ यो यो व॑नु॒ष्यति॑ व॒यम् । \newline
10. व॒नु॒ष्यति॑ व॒यं ॅव॒यं ॅव॑नु॒ष्यति॑ वनु॒ष्यति॑ व॒यम् ज॑येम जयेम व॒यं ॅव॑नु॒ष्यति॑ वनु॒ष्यति॑ व॒यम् ज॑येम । \newline
11. व॒यम् ज॑येम जयेम व॒यं ॅव॒यम् ज॑येम॒ पृत॑नासु॒ पृत॑नासु जयेम व॒यं ॅव॒यम् ज॑येम॒ पृत॑नासु । \newline
12. ज॒ये॒म॒ पृत॑नासु॒ पृत॑नासु जयेम जयेम॒ पृत॑नासु दू॒ढ्यो॑ दू॒ढ्यः॑ पृत॑नासु जयेम जयेम॒ पृत॑नासु दू॒ढ्यः॑ । \newline
13. पृत॑नासु दू॒ढ्यो॑ दू॒ढ्यः॑ पृत॑नासु॒ पृत॑नासु दू॒ढ्यः॑ । \newline
14. दू॒ढ्य॑ इति॑ दू॒ढ्यः॑ । \newline
15. आ नो॑ न॒ आ नो॑ मित्रावरुणा मित्रावरुणा न॒ आ नो॑ मित्रावरुणा । \newline
16. नो॒ मि॒त्रा॒व॒रु॒णा॒ मि॒त्रा॒व॒रु॒णा॒ नो॒ नो॒ मि॒त्रा॒व॒रु॒णा॒ प्र प्र मि॑त्रावरुणा नो नो मित्रावरुणा॒ प्र । \newline
17. मि॒त्रा॒व॒रु॒णा॒ प्र प्र मि॑त्रावरुणा मित्रावरुणा॒ प्र बा॒हवा॑ बा॒हवा॒ प्र मि॑त्रावरुणा मित्रावरुणा॒ प्र बा॒हवा᳚ । \newline
18. मि॒त्रा॒व॒रु॒णेति॑ मित्रा - व॒रु॒णा॒ । \newline
19. प्र बा॒हवा॑ बा॒हवा॒ प्र प्र बा॒हवा᳚ । \newline
20. बा॒हवेति॑ बा॒हवा᳚ । \newline
21. त्वम् नो॑ न॒स्त्वम् त्वम् नो॑ अग्ने अग्ने न॒स्त्वम् त्वम् नो॑ अग्ने । \newline
22. नो॒ अ॒ग्ने॒ अ॒ग्ने॒ नो॒ नो॒ अ॒ग्ने॒ वरु॑णस्य॒ वरु॑णस्याग्ने नो नो अग्ने॒ वरु॑णस्य । \newline
23. अ॒ग्ने॒ वरु॑णस्य॒ वरु॑णस्याग्ने अग्ने॒ वरु॑णस्य वि॒द्वान्. वि॒द्वान्. वरु॑णस्याग्ने अग्ने॒ वरु॑णस्य वि॒द्वान् । \newline
24. वरु॑णस्य वि॒द्वान्. वि॒द्वान्. वरु॑णस्य॒ वरु॑णस्य वि॒द्वान् दे॒वस्य॑ दे॒वस्य॑ वि॒द्वान्. वरु॑णस्य॒ वरु॑णस्य वि॒द्वान् दे॒वस्य॑ । \newline
25. वि॒द्वान् दे॒वस्य॑ दे॒वस्य॑ वि॒द्वान्. वि॒द्वान् दे॒वस्य॒ हेडो॒ हेडो॑ दे॒वस्य॑ वि॒द्वान्. वि॒द्वान् दे॒वस्य॒ हेडः॑ । \newline
26. दे॒वस्य॒ हेडो॒ हेडो॑ दे॒वस्य॑ दे॒वस्य॒ हेडो ऽवाव॒ हेडो॑ दे॒वस्य॑ दे॒वस्य॒ हेडो ऽव॑ । \newline
27. हेडो ऽवाव॒ हेडो॒ हेडो ऽव॑ यासिसीष्ठा यासिसीष्ठा॒ अव॒ हेडो॒ हेडो ऽव॑ यासिसीष्ठाः । \newline
28. अव॑ यासिसीष्ठा यासिसीष्ठा॒ अवाव॑ यासिसीष्ठाः । \newline
29. या॒सि॒सी॒ष्ठा॒ इति॑ यासिसीष्ठाः । \newline
30. यजि॑ष्ठो॒ वह्नि॑तमो॒ वह्नि॑तमो॒ यजि॑ष्ठो॒ यजि॑ष्ठो॒ वह्नि॑तमः॒ शोशु॑चानः॒ शोशु॑चानो॒ वह्नि॑तमो॒ यजि॑ष्ठो॒ यजि॑ष्ठो॒ वह्नि॑तमः॒ शोशु॑चानः । \newline
31. वह्नि॑तमः॒ शोशु॑चानः॒ शोशु॑चानो॒ वह्नि॑तमो॒ वह्नि॑तमः॒ शोशु॑चानो॒ विश्वा॒ विश्वा॒ शोशु॑चानो॒ वह्नि॑तमो॒ वह्नि॑तमः॒ शोशु॑चानो॒ विश्वा᳚ । \newline
32. वह्नि॑तम॒ इति॒ वह्नि॑ - त॒मः॒ । \newline
33. शोशु॑चानो॒ विश्वा॒ विश्वा॒ शोशु॑चानः॒ शोशु॑चानो॒ विश्वा॒ द्वेषाꣳ॑सि॒ द्वेषाꣳ॑सि॒ विश्वा॒ शोशु॑चानः॒ शोशु॑चानो॒ विश्वा॒ द्वेषाꣳ॑सि । \newline
34. विश्वा॒ द्वेषाꣳ॑सि॒ द्वेषाꣳ॑सि॒ विश्वा॒ विश्वा॒ द्वेषाꣳ॑सि॒ प्र प्र द्वेषाꣳ॑सि॒ विश्वा॒ विश्वा॒ द्वेषाꣳ॑सि॒ प्र । \newline
35. द्वेषाꣳ॑सि॒ प्र प्र द्वेषाꣳ॑सि॒ द्वेषाꣳ॑सि॒ प्र मु॑मुग्धि मुमुग्धि॒ प्र द्वेषाꣳ॑सि॒ द्वेषाꣳ॑सि॒ प्र मु॑मुग्धि । \newline
36. प्र मु॑मुग्धि मुमुग्धि॒ प्र प्र मु॑मुग्ध्य॒स्म द॒स्मन् मु॑मुग्धि॒ प्र प्र मु॑मुग्ध्य॒स्मत् । \newline
37. मु॒मु॒ग्ध्य॒स्म द॒स्मन् मु॑मुग्धि मुमुग्ध्य॒स्मत् । \newline
38. अ॒स्मदित्य॒स्मत् । \newline
39. स त्वम् त्वꣳ स स त्वम् नो॑ न॒स्त्वꣳ स स त्वम् नः॑ । \newline
40. त्वम् नो॑ न॒ स्त्वम् त्वम् नो॑ अग्ने अग्ने न॒ स्त्वम् त्वम् नो॑ अग्ने । \newline
41. नो॒ अ॒ग्ने॒ अ॒ग्ने॒ नो॒ नो॒ अ॒ग्ने॒ ऽव॒मो॑ ऽव॒मो᳚ ऽग्ने नो नो अग्ने ऽव॒मः । \newline
42. अ॒ग्ने॒ ऽव॒मो॑ ऽव॒मो᳚ ऽग्ने अग्ने ऽव॒मो भ॑व भवाव॒मो᳚ ऽग्ने अग्ने ऽव॒मो भ॑व । \newline
43. अ॒व॒मो भ॑व भवाव॒मो॑ ऽव॒मो भ॑वो॒ त्यू॑ती भ॑वाव॒मो॑ ऽव॒मो भ॑वो॒ती । \newline
44. भ॒वो॒ त्यू॑ती भ॑व भवो॒ती नेदि॑ष्ठो॒ नेदि॑ष्ठ ऊ॒ती भ॑व भवो॒ती नेदि॑ष्ठः । \newline
45. ऊ॒ती नेदि॑ष्ठो॒ नेदि॑ष्ठ ऊ॒त्यू॑ती नेदि॑ष्ठो अ॒स्या अ॒स्या नेदि॑ष्ठ ऊ॒त्यू॑ती नेदि॑ष्ठो अ॒स्याः । \newline
46. नेदि॑ष्ठो अ॒स्या अ॒स्या नेदि॑ष्ठो॒ नेदि॑ष्ठो अ॒स्या उ॒षस॑ उ॒षसो॑ अ॒स्या नेदि॑ष्ठो॒ नेदि॑ष्ठो अ॒स्या उ॒षसः॑ । \newline
47. अ॒स्या उ॒षस॑ उ॒षसो॑ अ॒स्या अ॒स्या उ॒षसो॒ व्यु॑ष्टौ॒ व्यु॑ष्टा वु॒षसो॑ अ॒स्या अ॒स्या उ॒षसो॒ व्यु॑ष्टौ । \newline
48. उ॒षसो॒ व्यु॑ष्टौ॒ व्यु॑ष्टा वु॒षस॑ उ॒षसो॒ व्यु॑ष्टौ । \newline
49. व्यु॑ष्टा॒विति॒ वि - उ॒ष्टौ॒ । \newline
50. अव॑ यक्ष्व य॒क्ष्वावाव॑ यक्ष्व नो नो य॒क्ष्वावाव॑ यक्ष्व नः । \newline
51. य॒क्ष्व॒ नो॒ नो॒ य॒क्ष्व॒ य॒क्ष्व॒ नो॒ वरु॑णं॒ ॅवरु॑णम् नो यक्ष्व यक्ष्व नो॒ वरु॑णम् । \newline
52. नो॒ वरु॑णं॒ ॅवरु॑णम् नो नो॒ वरु॑णꣳ॒॒ ररा॑णो॒ ररा॑णो॒ वरु॑णम् नो नो॒ वरु॑णꣳ॒॒ ररा॑णः । \newline
53. वरु॑णꣳ॒॒ ररा॑णो॒ ररा॑णो॒ वरु॑णं॒ ॅवरु॑णꣳ॒॒ ररा॑णो वी॒हि वी॒हि ररा॑णो॒ वरु॑णं॒ ॅवरु॑णꣳ॒॒ ररा॑णो वी॒हि । \newline
\pagebreak
\markright{ TS 2.5.12.4  \hfill https://www.vedavms.in \hfill}
\addcontentsline{toc}{section}{ TS 2.5.12.4 }
\section*{ TS 2.5.12.4 }

\textbf{TS 2.5.12.4 } \newline
\textbf{Samhita Paata} \newline

ररा॑णो वी॒हि मृ॑डी॒कꣳ सु॒हवो॑ न एधि ॥प्रप्रा॒यम॒ग्निर्भ॑र॒तस्य॑ शृण्वे॒ वि यथ् सूर्यो॒ न रोच॑ते बृ॒हद्भाः । अ॒भि यः पू॒रुं पृत॑नासु त॒स्थौ दी॒दाय॒ दैव्यो॒ अति॑थिः शि॒वो नः॑ ॥प्र ते॑ यक्षि॒ प्र त॑ इयर्मि॒ मन्म॒ भुवो॒ यथा॒ वन्द्यो॑ नो॒ हवे॑षु । धन्व॑न्निव प्र॒पा अ॑सि॒ त्वम॑ग्न इय॒क्षवे॑ पू॒रवे᳚ प्रत्न राजन्न् ॥ \newline

\textbf{Pada Paata} \newline

ररा॑णः । वी॒हि । मृ॒डी॒कम् । सु॒हव॒ इति॑ सु - हवः॑ । नः॒ । ए॒धि॒ ॥ प्रप्रेति॒ प्र - प्र॒ । अ॒यम् । अ॒ग्निः । भ॒र॒तस्य॑ । शृ॒ण्वे॒ । वीति॑ । यत् । सूर्यः॑ । न । रोच॑ते । बृ॒हत् । भाः ॥ अ॒भीति॑ । यः । पू॒रुम् । पृत॑नासु । त॒स्थौ । दी॒दाय॑ । दैव्यः॑ । अति॑थिः । शि॒वः । नः॒ ॥ प्रेति॑ । ते॒ । य॒क्षि॒ । प्रेति॑ । ते॒ । इ॒य॒र्मि॒ । मन्म॑ । भुवः॑ । यथा᳚ । वन्द्यः॑ । नः॒ । हवे॑षु ॥ धन्वन्न्॑ । इ॒व॒ । प्र॒पेति॑ प्र - पा । अ॒सि॒ । त्वम् । अ॒ग्ने॒ । इ॒य॒क्षवे᳚ । पू॒रवे᳚ । प्र॒त्न॒ । रा॒ज॒न्न् ॥  \newline


\textbf{Krama Paata} \newline

ररा॑णो वी॒हि । वी॒हि मृ॑डी॒कम् । मृ॒डी॒कꣳ सु॒हवः॑ । सु॒हवो॑ नः । सु॒हव॒ इति॑ सु - हवः॑ । न॒ ए॒धि॒ । ए॒धीत्ये॑धि ॥ प्रप्रा॒यम् । प्रप्रेति॒ प्र - प्र॒ । अ॒यम॒ग्निः । अ॒ग्निर् भ॑र॒तस्य॑ । भ॒र॒तस्य॑ शृण्वे । शृ॒ण्वे॒ वि । वि यत् । यथ् सूर्यः॑ । सूर्यो॒ न । न रोच॑ते । रोच॑ते बृ॒हत् । बृ॒हद् भाः । भा इति॒ भाः ॥ अ॒भि यः । यः पू॒रुम् । पू॒रुम् पृत॑नासु । पृत॑नासु त॒स्थौ । त॒स्थौ दी॒दाय॑ । दी॒दाय॒ दैव्यः॑ । दैव्यो॒ अति॑थिः । अति॑थिः शि॒वः । शि॒वो नः॑ । न॒ इति॑ नः ॥ प्र ते᳚ । ते॒ य॒क्षि॒ । य॒क्षि॒ प्र । प्र ते᳚ । त॒ इ॒य॒र्मि॒ । इ॒य॒र्मि॒ मन्म॑ । मन्म॒ भुवः॑ । भुवो॒ यथा᳚ । यथा॒ वन्द्यः॑ । वन्द्यो॑ नः । नो॒ हवे॑षु । हवे॒ष्विति॒ हवे॑षु ॥ धन्व॑न्निव । इ॒व॒ प्र॒पा । प्र॒पा अ॑सि । प्र॒पेति॑ प्र - पा । अ॒सि॒ त्वम् । त्वम॑ग्ने । अ॒ग्न॒ इ॒य॒क्षवे᳚ । इ॒य॒क्षवे॑ पू॒रवे᳚ । पू॒रवे᳚ प्रत्न । प्र॒त्न॒ रा॒ज॒न्न्॒ ( ) । रा॒ज॒न्निति॑ राजन्न् । \newline

\textbf{Jatai Paata} \newline

1. ररा॑णो वी॒हि वी॒हि ररा॑णो॒ ररा॑णो वी॒हि । \newline
2. वी॒हि मृ॑डी॒कम् मृ॑डी॒कं ॅवी॒हि वी॒हि मृ॑डी॒कम् । \newline
3. मृ॒डी॒कꣳ सु॒हवः॑ सु॒हवो॑ मृडी॒कम् मृ॑डी॒कꣳ सु॒हवः॑ । \newline
4. सु॒हवो॑ नो नः सु॒हवः॑ सु॒हवो॑ नः । \newline
5. सु॒हव॒ इति॑ सु - हवः॑ । \newline
6. न॒ ए॒ध्ये॒धि॒ नो॒ न॒ ए॒धि॒ । \newline
7. ए॒धीत्ये॑धि । \newline
8. प्रप्रा॒य म॒यम् प्रप्र॒ प्रप्रा॒यम् । \newline
9. प्रप्रेति॒ प्र - प्र॒ । \newline
10. अ॒य म॒ग्नि र॒ग्नि र॒य म॒य म॒ग्निः । \newline
11. अ॒ग्निर् भ॑र॒तस्य॑ भर॒तस्या॒ग्नि र॒ग्निर् भ॑र॒तस्य॑ । \newline
12. भ॒र॒तस्य॑ शृण्वे शृण्वे भर॒तस्य॑ भर॒तस्य॑ शृण्वे । \newline
13. शृ॒ण्वे॒ वि वि शृ॑ण्वे शृण्वे॒ वि । \newline
14. वि यद् यद् वि वि यत् । \newline
15. यथ् सूर्यः॒ सूर्यो॒ यद् यथ् सूर्यः॑ । \newline
16. सूर्यो॒ न न सूर्यः॒ सूर्यो॒ न । \newline
17. न रोच॑ते॒ रोच॑ते॒ न न रोच॑ते । \newline
18. रोच॑ते बृ॒हद् बृ॒हद् रोच॑ते॒ रोच॑ते बृ॒हत् । \newline
19. बृ॒हद् भा भा बृ॒हद् बृ॒हद् भाः । \newline
20. भा इति॒ भाः । \newline
21. अ॒भि यो यो अ॒भ्य॑भि यः । \newline
22. यः पू॒रुम् पू॒रुं ॅयो यः पू॒रुम् । \newline
23. पू॒रुम् पृत॑नासु॒ पृत॑नासु पू॒रुम् पू॒रुम् पृत॑नासु । \newline
24. पृत॑नासु त॒स्थौ त॒स्थौ पृत॑नासु॒ पृत॑नासु त॒स्थौ । \newline
25. त॒स्थौ दी॒दाय॑ दी॒दाय॑ त॒स्थौ त॒स्थौ दी॒दाय॑ । \newline
26. दी॒दाय॒ दैव्यो॒ दैव्यो॑ दी॒दाय॑ दी॒दाय॒ दैव्यः॑ । \newline
27. दैव्यो॒ अति॑थि॒ रति॑थि॒र् दैव्यो॒ दैव्यो॒ अति॑थिः । \newline
28. अति॑थिः शि॒वः शि॒वो अति॑थि॒ रति॑थिः शि॒वः । \newline
29. शि॒वो नो॑ नः शि॒वः शि॒वो नः॑ । \newline
30. न॒ इति॑ नः । \newline
31. प्र ते॑ ते॒ प्र प्र ते᳚ । \newline
32. ते॒ य॒क्षि॒ य॒क्षि॒ ते॒ ते॒ य॒क्षि॒ । \newline
33. य॒क्षि॒ प्र प्र य॑क्षि यक्षि॒ प्र । \newline
34. प्र ते॑ ते॒ प्र प्र ते᳚ । \newline
35. त॒ इ॒य॒र्मी॒य॒र्मि॒ ते॒ त॒ इ॒य॒र्मि॒ । \newline
36. इ॒य॒र्मि॒ मन्म॒ मन्मे॑ यर्मीयर्मि॒ मन्म॑ । \newline
37. मन्म॒ भुवो॒ भुवो॒ मन्म॒ मन्म॒ भुवः॑ । \newline
38. भुवो॒ यथा॒ यथा॒ भुवो॒ भुवो॒ यथा᳚ । \newline
39. यथा॒ वन्द्यो॒ वन्द्यो॒ यथा॒ यथा॒ वन्द्यः॑ । \newline
40. वन्द्यो॑ नो नो॒ वन्द्यो॒ वन्द्यो॑ नः । \newline
41. नो॒ हवे॑षु॒ हवे॑षु नो नो॒ हवे॑षु । \newline
42. हवे॒ष्विति॒ हवे॑षु । \newline
43. धन्व॑न् निवे व॒ धन्व॒न् धन्व॑न् निव । \newline
44. इ॒व॒ प्रपा॒ प्रपे॑वे व॒ प्रपा᳚ । \newline
45. प्रपा॑ अस्यसि॒ प्रपा॒ प्रपा॑ असि । \newline
46. प्र॒पेति॑ प्र - पा । \newline
47. अ॒सि॒ त्वम् त्व म॑स्यसि॒ त्वम् । \newline
48. त्व म॑ग्ने अग्ने॒ त्वम् त्व म॑ग्ने । \newline
49. अ॒ग्न॒ इ॒य॒क्षव॑ इय॒क्षवे॑ अग्ने अग्न इय॒क्षवे᳚ । \newline
50. इ॒य॒क्षवे॑ पू॒रवे॑ पू॒रव॑ इय॒क्षव॑ इय॒क्षवे॑ पू॒रवे᳚ । \newline
51. पू॒रवे᳚ प्रत्न प्रत्न पू॒रवे॑ पू॒रवे᳚ प्रत्न । \newline
52. प्र॒त्न॒ रा॒ज॒न् रा॒ज॒न् प्र॒त्न॒ प्र॒त्न॒ रा॒ज॒न्न् । \newline
53. रा॒ज॒न्निति॑ राजन्न् । \newline

\textbf{Ghana Paata } \newline

1. ररा॑णो वी॒हि वी॒हि ररा॑णो॒ ररा॑णो वी॒हि मृ॑डी॒कम् मृ॑डी॒कं ॅवी॒हि ररा॑णो॒ ररा॑णो वी॒हि मृ॑डी॒कम् । \newline
2. वी॒हि मृ॑डी॒कम् मृ॑डी॒कं ॅवी॒हि वी॒हि मृ॑डी॒कꣳ सु॒हवः॑ सु॒हवो॑ मृडी॒कं ॅवी॒हि वी॒हि मृ॑डी॒कꣳ सु॒हवः॑ । \newline
3. मृ॒डी॒कꣳ सु॒हवः॑ सु॒हवो॑ मृडी॒कम् मृ॑डी॒कꣳ सु॒हवो॑ नो नः सु॒हवो॑ मृडी॒कम् मृ॑डी॒कꣳ सु॒हवो॑ नः । \newline
4. सु॒हवो॑ नो नः सु॒हवः॑ सु॒हवो॑ न एध्येधि नः सु॒हवः॑ सु॒हवो॑ न एधि । \newline
5. सु॒हव॒ इति॑ सु - हवः॑ । \newline
6. न॒ ए॒ध्ये॒धि॒ नो॒ न॒ ए॒धि॒ । \newline
7. ए॒धीत्ये॑धि । \newline
8. प्रप्रा॒य म॒यम् प्रप्र॒ प्रप्रा॒य म॒ग्नि र॒ग्नि र॒यम् प्रप्र॒ प्रप्रा॒य म॒ग्निः । \newline
9. प्रप्रेति॒ प्र - प्र॒ । \newline
10. अ॒य म॒ग्नि र॒ग्नि र॒य म॒य म॒ग्निर् भ॑र॒तस्य॑ भर॒तस्या॒ ग्निर॒य म॒य म॒ग्निर् भ॑र॒तस्य॑ । \newline
11. अ॒ग्निर् भ॑र॒तस्य॑ भर॒तस्या॒ ग्नि र॒ग्निर् भ॑र॒तस्य॑ शृण्वे शृण्वे भर॒तस्या॒ ग्नि र॒ग्निर् भ॑र॒तस्य॑ शृण्वे । \newline
12. भ॒र॒तस्य॑ शृण्वे शृण्वे भर॒तस्य॑ भर॒तस्य॑ शृण्वे॒ वि वि शृ॑ण्वे भर॒तस्य॑ भर॒तस्य॑ शृण्वे॒ वि । \newline
13. शृ॒ण्वे॒ वि वि शृ॑ण्वे शृण्वे॒ वि यद् यद् वि शृ॑ण्वे शृण्वे॒ वि यत् । \newline
14. वि यद् यद् वि वि यथ् सूर्यः॒ सूर्यो॒ यद् वि वि यथ् सूर्यः॑ । \newline
15. यथ् सूर्यः॒ सूर्यो॒ यद् यथ् सूर्यो॒ न न सूर्यो॒ यद् यथ् सूर्यो॒ न । \newline
16. सूर्यो॒ न न सूर्यः॒ सूर्यो॒ न रोच॑ते॒ रोच॑ते॒ न सूर्यः॒ सूर्यो॒ न रोच॑ते । \newline
17. न रोच॑ते॒ रोच॑ते॒ न न रोच॑ते बृ॒हद् बृ॒हद् रोच॑ते॒ न न रोच॑ते बृ॒हत् । \newline
18. रोच॑ते बृ॒हद् बृ॒हद् रोच॑ते॒ रोच॑ते बृ॒हद् भा भा बृ॒हद् रोच॑ते॒ रोच॑ते बृ॒हद् भाः । \newline
19. बृ॒हद् भा भा बृ॒हद् बृ॒हद् भाः । \newline
20. भा इति॒ भाः । \newline
21. अ॒भि यो यो अ॒भ्य॑भि यः पू॒रुम् पू॒रुं ॅयो अ॒भ्य॑भि यः पू॒रुम् । \newline
22. यः पू॒रुम् पू॒रुं ॅयो यः पू॒रुम् पृत॑नासु॒ पृत॑नासु पू॒रुं ॅयो यः पू॒रुम् पृत॑नासु । \newline
23. पू॒रुम् पृत॑नासु॒ पृत॑नासु पू॒रुम् पू॒रुम् पृत॑नासु त॒स्थौ त॒स्थौ पृत॑नासु पू॒रुम् पू॒रुम् पृत॑नासु त॒स्थौ । \newline
24. पृत॑नासु त॒स्थौ त॒स्थौ पृत॑नासु॒ पृत॑नासु त॒स्थौ दी॒दाय॑ दी॒दाय॑ त॒स्थौ पृत॑नासु॒ पृत॑नासु त॒स्थौ दी॒दाय॑ । \newline
25. त॒स्थौ दी॒दाय॑ दी॒दाय॑ त॒स्थौ त॒स्थौ दी॒दाय॒ दैव्यो॒ दैव्यो॑ दी॒दाय॑ त॒स्थौ त॒स्थौ दी॒दाय॒ दैव्यः॑ । \newline
26. दी॒दाय॒ दैव्यो॒ दैव्यो॑ दी॒दाय॑ दी॒दाय॒ दैव्यो॒ अति॑थि॒ रति॑थि॒र् दैव्यो॑ दी॒दाय॑ दी॒दाय॒ दैव्यो॒ अति॑थिः । \newline
27. दैव्यो॒ अति॑थि॒ रति॑थि॒र् दैव्यो॒ दैव्यो॒ अति॑थिः शि॒वः शि॒वो अति॑थि॒र् दैव्यो॒ दैव्यो॒ अति॑थिः शि॒वः । \newline
28. अति॑थिः शि॒वः शि॒वो अति॑थि॒ रति॑थिः शि॒वो नो॑ नः शि॒वो अति॑थि॒ रति॑थिः शि॒वो नः॑ । \newline
29. शि॒वो नो॑ नः शि॒वः शि॒वो नः॑ । \newline
30. न॒ इति॑ नः । \newline
31. प्र ते॑ ते॒ प्र प्र ते॑ यक्षि यक्षि ते॒ प्र प्र ते॑ यक्षि । \newline
32. ते॒ य॒क्षि॒ य॒क्षि॒ ते॒ ते॒ य॒क्षि॒ प्र प्र य॑क्षि ते ते यक्षि॒ प्र । \newline
33. य॒क्षि॒ प्र प्र य॑क्षि यक्षि॒ प्र ते॑ ते॒ प्र य॑क्षि यक्षि॒ प्र ते᳚ । \newline
34. प्र ते॑ ते॒ प्र प्र त॑ इयर्मी यर्मि ते॒ प्र प्र त॑ इयर्मि । \newline
35. त॒ इ॒य॒र्मी॒ य॒र्मि॒ ते॒ त॒ इ॒य॒र्मि॒ मन्म॒ मन्मे॑ यर्मि ते त इयर्मि॒ मन्म॑ । \newline
36. इ॒य॒र्मि॒ मन्म॒ मन्मे॑ यर्मी यर्मि॒ मन्म॒ भुवो॒ भुवो॒ मन्मे॑ यर्मी यर्मि॒ मन्म॒ भुवः॑ । \newline
37. मन्म॒ भुवो॒ भुवो॒ मन्म॒ मन्म॒ भुवो॒ यथा॒ यथा॒ भुवो॒ मन्म॒ मन्म॒ भुवो॒ यथा᳚ । \newline
38. भुवो॒ यथा॒ यथा॒ भुवो॒ भुवो॒ यथा॒ वन्द्यो॒ वन्द्यो॒ यथा॒ भुवो॒ भुवो॒ यथा॒ वन्द्यः॑ । \newline
39. यथा॒ वन्द्यो॒ वन्द्यो॒ यथा॒ यथा॒ वन्द्यो॑ नो नो॒ वन्द्यो॒ यथा॒ यथा॒ वन्द्यो॑ नः । \newline
40. वन्द्यो॑ नो नो॒ वन्द्यो॒ वन्द्यो॑ नो॒ हवे॑षु॒ हवे॑षु नो॒ वन्द्यो॒ वन्द्यो॑ नो॒ हवे॑षु । \newline
41. नो॒ हवे॑षु॒ हवे॑षु नो नो॒ हवे॑षु । \newline
42. हवे॒ष्विति॒ हवे॑षु । \newline
43. धन्व॑न् निवे व॒ धन्व॒न् धन्व॑न् निव॒ प्रपा॒ प्रपे॑व॒ धन्व॒न् धन्व॑न् निव॒ प्रपा᳚ । \newline
44. इ॒व॒ प्रपा॒ प्रपे॑वे व॒ प्रपा॑ अस्यसि॒ प्रपे॑वे व॒ प्रपा॑ असि । \newline
45. प्रपा॑ अस्यसि॒ प्रपा॒ प्रपा॑ असि॒ त्वम् त्व म॑सि॒ प्रपा॒ प्रपा॑ असि॒ त्वम् । \newline
46. प्र॒पेति॑ प्र - पा । \newline
47. अ॒सि॒ त्वम् त्व म॑स्यसि॒ त्व म॑ग्ने अग्ने॒ त्व म॑स्यसि॒ त्व म॑ग्ने । \newline
48. त्व म॑ग्ने अग्ने॒ त्वम् त्व म॑ग्न इय॒क्षव॑ इय॒क्षवे॑ अग्ने॒ त्वम् त्व म॑ग्न इय॒क्षवे᳚ । \newline
49. अ॒ग्न॒ इ॒य॒क्षव॑ इय॒क्षवे॑ अग्ने अग्न इय॒क्षवे॑ पू॒रवे॑ पू॒रव॑ इय॒क्षवे॑ अग्ने अग्न इय॒क्षवे॑ पू॒रवे᳚ । \newline
50. इ॒य॒क्षवे॑ पू॒रवे॑ पू॒रव॑ इय॒क्षव॑ इय॒क्षवे॑ पू॒रवे᳚ प्रत्न प्रत्न पू॒रव॑ इय॒क्षव॑ इय॒क्षवे॑ पू॒रवे᳚ प्रत्न । \newline
51. पू॒रवे᳚ प्रत्न प्रत्न पू॒रवे॑ पू॒रवे᳚ प्रत्न राजन् राजन् प्रत्न पू॒रवे॑ पू॒रवे᳚ प्रत्न राजन्न् । \newline
52. प्र॒त्न॒ रा॒ज॒न् रा॒ज॒न् प्र॒त्न॒ प्र॒त्न॒ रा॒ज॒न्न् । \newline
53. रा॒ज॒न्निति॑ राजन्न् । \newline
\pagebreak
\markright{ TS 2.5.12.5  \hfill https://www.vedavms.in \hfill}
\addcontentsline{toc}{section}{ TS 2.5.12.5 }
\section*{ TS 2.5.12.5 }

\textbf{TS 2.5.12.5 } \newline
\textbf{Samhita Paata} \newline

वि पाज॑सा॒ >17, वि ज्योति॑षा >18 ॥ स त्वम॑ग्ने॒ प्रती॑केन॒ प्रत्यो॑ष यातुधा॒न्यः॑ । उ॒रु॒क्षये॑षु॒ दीद्य॑त् ॥तꣳ सु॒प्रती॑कꣳ सु॒दृशꣳ॒॒ स्वञ्च॒मवि॑द्वाꣳसो वि॒दुष्ट॑रꣳ सपेम ।स य॑क्ष॒-द्विश्वा॑ व॒युना॑नि वि॒द्वान् प्र ह॒व्यम॒ग्निर॒मृते॑षु वोचत् ॥अꣳ॒॒हो॒मुचे॑ >19, वि॒वेष॒ यन्मा॒ >20, विन॑ इ॒न्द्रे >21, न्द्र॑ क्ष॒त्र >22मि॑न्द्रि॒याणि॑ शतक्र॒तो >23, ऽनु॑ ते दायि >24 ॥ \newline

\textbf{Pada Paata} \newline

वीति॑ । पाज॑सा । वीति॑ । ज्योति॑षा ॥ सः । त्वम् । अ॒ग्ने॒ । प्रती॑केन । प्रतीति॑ । ओ॒ष॒ । या॒तु॒धा॒न्य॑ इति॑ यातु - धा॒न्यः॑ ॥ उ॒रु॒क्षये॒ष्वित्यु॑रु - क्षये॑षु । दीद्य॑त् ॥ तम् । सु॒प्रती॑क॒मिति॑ सु - प्रती॑कम् । सु॒दृश॒मिति॑ सु - दृश᳚म् । स्वञ्च᳚म् । अवि॑द्वाꣳसः । वि॒दुष्ट॑र॒मिति॑ वि॒दुः - त॒र॒म् । स॒पे॒म॒ ॥ सः । य॒क्ष॒त् । विश्वा᳚ । व॒युना॑नि । वि॒द्वान् । प्रेति॑ । ह॒व्यम् । अ॒ग्निः । अ॒मृते॑षु । वो॒च॒त् ॥ अꣳ॒॒हो॒मुच॒ इत्यꣳ॑हः - मुचे᳚ । वि॒वेष॑ । यत् । मा॒ । वीति॑ । नः॒ । इ॒न्द्र॒ । इन्द्र॑ । क्ष॒त्रम् । इ॒न्द्रि॒याणि॑ । श॒त॒क्र॒तो॒ इति॑ शत-क्र॒तो॒ । अन्विति॑ । ते॒ । दा॒यि॒ ॥  \newline


\textbf{Krama Paata} \newline

वि पाज॑सा । पाज॑सा॒ वि । वि ज्योति॑षा । ज्योति॒षेति॒ ज्योति॑षा ॥ स त्वम् । त्वम॑ग्ने । अ॒ग्ने॒ प्रती॑केन । प्रती॑केन॒ प्रति॑ । प्रत्यो॑ष । ओ॒ष॒ या॒तु॒धा॒न्यः॑ । या॒तु॒धा॒न्य॑ इति॑ यातु - धा॒न्यः॑ ॥ उ॒रु॒क्षये॑षु॒ दीद्य॑त् । उ॒रु॒क्षये॒ष्वित्यु॑रु - क्षये॑षु । दीद्य॒दिति॒ दीद्य॑त् ॥ तꣳ सु॒प्रती॑कम् । सु॒प्रती॑कꣳ सु॒दृश᳚म् । सु॒प्रती॑क॒मिति॑ सु - प्रती॑कम् । सु॒दृशꣳ॒॒ स्वञ्च᳚म् । सु॒दृश॒मिति॑ सु - दृश᳚म् । स्वञ्च॒मवि॑द्वाꣳसः । अवि॑द्वाꣳसो वि॒दुष्ट॑रम् । वि॒दुष्ट॑रꣳ सपेम । वि॒दुष्ट॑र॒मिति॑ वि॒दुः - त॒र॒म् । स॒पे॒मेति॑ सपेम ॥ स य॑क्षत् । य॒क्ष॒द् विश्वा᳚ । विश्वा॑ व॒युना॑नि । व॒युना॑नि वि॒द्वान् । वि॒द्वान् प्र । प्र ह॒व्यम् । ह॒व्यम॒ग्निः । अ॒ग्निर॒मृते॑षु । अ॒मृते॑षु वोचत् । वो॒च॒दिति॑ वोचत् ॥ अꣳ॒॒हो॒मुचे॑ वि॒वेष॑ । अꣳ॒॒हो॒मुच॒ इत्यꣳ॑हः - मुचे᳚ । वि॒वेष॒ यत् । यन् मा᳚ । मा॒ वि । वि नः॑ । न॒ इ॒न्द्र॒ । इ॒न्द्रेन्द्र॑ । इन्द्र॑ क्ष॒त्रम् । क्ष॒त्रमि॑न्द्रि॒याणि॑ । इ॒न्द्रि॒याणि॑ शतक्रतो । श॒त॒क्र॒तो ऽनु॑ । श॒त॒क्र॒तो॒ इति॑ शत - क्र॒तो॒ । अनु॑ ते । ते॒ दा॒यि॒ । दा॒यीति॑ दायि । \newline

\textbf{Jatai Paata} \newline

1. वि पाज॑सा॒ पाज॑सा॒ वि वि पाज॑सा । \newline
2. पाज॑सा॒ वि वि पाज॑सा॒ पाज॑सा॒ वि । \newline
3. वि ज्योति॑षा॒ ज्योति॑षा॒ वि वि ज्योति॑षा । \newline
4. ज्योति॒षेति॒ ज्योति॑षा । \newline
5. स त्वम् त्वꣳ स स त्वम् । \newline
6. त्व म॑ग्ने अग्ने॒ त्वम् त्व म॑ग्ने । \newline
7. अ॒ग्ने॒ प्रती॑केन॒ प्रती॑केनाग्ने अग्ने॒ प्रती॑केन । \newline
8. प्रती॑केन॒ प्रति॒ प्रति॒ प्रती॑केन॒ प्रती॑केन॒ प्रति॑ । \newline
9. प्रत्यो॑षौष॒ प्रति॒ प्रत्यो॑ष । \newline
10. ओ॒ष॒ या॒तु॒धा॒न्यो॑ यातुधा॒न्य॑ ओषौष यातुधा॒न्यः॑ । \newline
11. या॒तु॒धा॒न्य॑ इति॑ यातु - धा॒न्यः॑ । \newline
12. उ॒रु॒क्षये॑षु॒ दीद्य॒द् दीद्य॑दुरु॒क्षये॑षू रु॒क्षये॑षु॒ दीद्य॑त् । \newline
13. उ॒रु॒क्षये॒ष्वित्यु॑रु - क्षये॑षु । \newline
14. दीद्य॒दिति॒ दीद्य॑त् । \newline
15. तꣳ सु॒प्रती॑कꣳ सु॒प्रती॑क॒म् तम् तꣳ सु॒प्रती॑कम् । \newline
16. सु॒प्रती॑कꣳ सु॒दृशꣳ॑ सु॒दृशꣳ॑ सु॒प्रती॑कꣳ सु॒प्रती॑कꣳ सु॒दृश᳚म् । \newline
17. सु॒प्रती॑क॒मिति॑ सु - प्रती॑कम् । \newline
18. सु॒दृशꣳ॒॒ स्वञ्चꣳ॒॒ स्वञ्चꣳ॑ सु॒दृशꣳ॑ सु॒दृशꣳ॒॒ स्वञ्च᳚म् । \newline
19. सु॒दृश॒मिति॑ सु - दृश᳚म् । \newline
20. स्वञ्च॒ मवि॑द्वाꣳ॒॒सो ऽवि॑द्वाꣳसः॒ स्वञ्चꣳ॒॒ स्वञ्च॒ मवि॑द्वाꣳसः । \newline
21. अवि॑द्वाꣳसो वि॒दुष्ट॑रं ॅवि॒दुष्ट॑र॒ मवि॑द्वाꣳ॒॒सो ऽवि॑द्वाꣳसो वि॒दुष्ट॑रम् । \newline
22. वि॒दुष्ट॑रꣳ सपेम सपेम वि॒दुष्ट॑रं ॅवि॒दुष्ट॑रꣳ सपेम । \newline
23. वि॒दुष्ट॑र॒मिति॑ वि॒दुः - त॒र॒म् । \newline
24. स॒पे॒मेति॑ सपेम । \newline
25. स य॑क्षद् यक्ष॒थ् स स य॑क्षत् । \newline
26. य॒क्ष॒द् विश्वा॒ विश्वा॑ यक्षद् यक्ष॒द् विश्वा᳚ । \newline
27. विश्वा॑ व॒युना॑नि व॒युना॑नि॒ विश्वा॒ विश्वा॑ व॒युना॑नि । \newline
28. व॒युना॑नि वि॒द्वान्. वि॒द्वान्. व॒युना॑नि व॒युना॑नि वि॒द्वान् । \newline
29. वि॒द्वान् प्र प्र वि॒द्वान्. वि॒द्वान् प्र । \newline
30. प्र ह॒व्यꣳ ह॒व्यम् प्र प्र ह॒व्यम् । \newline
31. ह॒व्य म॒ग्नि र॒ग्निर्. ह॒व्यꣳ ह॒व्य म॒ग्निः । \newline
32. अ॒ग्नि र॒मृते᳚ ष्व॒मृते᳚ ष्व॒ग्नि र॒ग्नि र॒मृते॑षु । \newline
33. अ॒मृते॑षु वोचद् वोचद॒मृते᳚ ष्व॒मृते॑षु वोचत् । \newline
34. वो॒च॒दिति॑ वोचत् । \newline
35. अꣳ॒॒हो॒मुचे॑ वि॒वेष॑ वि॒वेषाꣳ॑हो॒मुचे ऽꣳ॑हो॒मुचे॑ वि॒वेष॑ । \newline
36. अꣳ॒॒हो॒मुच॒ इत्यꣳ॑हः - मुचे᳚ । \newline
37. वि॒वेष॒ यद् यद् वि॒वेष॑ वि॒वेष॒ यत् । \newline
38. यन् मा॑ मा॒ यद् यन् मा᳚ । \newline
39. मा॒ वि वि मा॑ मा॒ वि । \newline
40. वि नो॑ नो॒ वि वि नः॑ । \newline
41. न॒ इ॒न्द्रे॒ न्द्र॒ नो॒ न॒ इ॒न्द्र॒ । \newline
42. इ॒न्द्रे न्द्रे न्द्रे᳚ न्द्रे॒ न्द्रे न्द्र॑ । \newline
43. इन्द्र॑ क्ष॒त्रम् क्ष॒त्र मिन्द्रे न्द्र॑ क्ष॒त्रम् । \newline
44. क्ष॒त्र मि॑न्द्रि॒याणी᳚ न्द्रि॒याणि॑ क्ष॒त्रम् क्ष॒त्र मि॑न्द्रि॒याणि॑ । \newline
45. इ॒न्द्रि॒याणि॑ शतक्रतो शतक्रतो इन्द्रि॒याणी᳚ न्द्रि॒याणि॑ शतक्रतो । \newline
46. श॒त॒क्र॒तोऽन्वनु॑ शतक्रतो शतक्र॒तोऽनु॑ । \newline
47. श॒त॒क्र॒तो॒ इति॑ शत - क्र॒तो॒ । \newline
48. अनु॑ ते ते॒ अन्वनु॑ ते । \newline
49. ते॒ दा॒यि॒ दा॒यि॒ ते॒ ते॒ दा॒यि॒ । \newline
50. दा॒यीति॑ दायि । \newline

\textbf{Ghana Paata } \newline

1. वि पाज॑सा॒ पाज॑सा॒ वि वि पाज॑सा॒ वि वि पाज॑सा॒ वि वि पाज॑सा॒ वि । \newline
2. पाज॑सा॒ वि वि पाज॑सा॒ पाज॑सा॒ वि ज्योति॑षा॒ ज्योति॑षा॒ वि पाज॑सा॒ पाज॑सा॒ वि ज्योति॑षा । \newline
3. वि ज्योति॑षा॒ ज्योति॑षा॒ वि वि ज्योति॑षा । \newline
4. ज्योति॒षेति॒ ज्योति॑षा । \newline
5. स त्वम् त्वꣳ स स त्व म॑ग्ने अग्ने॒ त्वꣳ स स त्व म॑ग्ने । \newline
6. त्व म॑ग्ने अग्ने॒ त्वम् त्व म॑ग्ने॒ प्रती॑केन॒ प्रती॑केनाग्ने॒ त्वम् त्व म॑ग्ने॒ प्रती॑केन । \newline
7. अ॒ग्ने॒ प्रती॑केन॒ प्रती॑केनाग्ने अग्ने॒ प्रती॑केन॒ प्रति॒ प्रति॒ प्रती॑केनाग्ने अग्ने॒ प्रती॑केन॒ प्रति॑ । \newline
8. प्रती॑केन॒ प्रति॒ प्रति॒ प्रती॑केन॒ प्रती॑केन॒ प्रत्यो॑षौष॒ प्रति॒ प्रती॑केन॒ प्रती॑केन॒ प्रत्यो॑ष । \newline
9. प्रत्यो॑षौष॒ प्रति॒ प्रत्यो॑ष यातुधा॒न्यो॑ यातुधा॒न्य॑ ओष॒ प्रति॒ प्रत्यो॑ष यातुधा॒न्यः॑ । \newline
10. ओ॒ष॒ या॒तु॒धा॒न्यो॑ यातुधा॒न्य॑ ओषौष यातुधा॒न्यः॑ । \newline
11. या॒तु॒धा॒न्य॑ इति॑ यातु - धा॒न्यः॑ । \newline
12. उ॒रु॒क्षये॑षु॒ दीद्य॒द् दीद्य॑ दुरु॒क्षये॑षू रु॒क्षये॑षु॒ दीद्य॑त् । \newline
13. उ॒रु॒क्षये॒ष्वित्यु॑रु - क्षये॑षु । \newline
14. दीद्य॒दिति॒ दीद्य॑त् । \newline
15. तꣳ सु॒प्रती॑कꣳ सु॒प्रती॑क॒म् तम् तꣳ सु॒प्रती॑कꣳ सु॒दृशꣳ॑ सु॒दृशꣳ॑ सु॒प्रती॑क॒म् तम् तꣳ सु॒प्रती॑कꣳ सु॒दृश᳚म् । \newline
16. सु॒प्रती॑कꣳ सु॒दृशꣳ॑ सु॒दृशꣳ॑ सु॒प्रती॑कꣳ सु॒प्रती॑कꣳ सु॒दृशꣳ॒॒ स्वञ्चꣳ॒॒ स्वञ्चꣳ॑ सु॒दृशꣳ॑ सु॒प्रती॑कꣳ सु॒प्रती॑कꣳ सु॒दृशꣳ॒॒ स्वञ्च᳚म् । \newline
17. सु॒प्रती॑क॒मिति॑ सु - प्रती॑कम् । \newline
18. सु॒दृशꣳ॒॒ स्वञ्चꣳ॒॒ स्वञ्चꣳ॑ सु॒दृशꣳ॑ सु॒दृशꣳ॒॒ स्वञ्च॒ मवि॑द्वाꣳ॒॒सो ऽवि॑द्वाꣳसः॒ स्वञ्चꣳ॑ सु॒दृशꣳ॑ सु॒दृशꣳ॒॒ स्वञ्च॒ मवि॑द्वाꣳसः । \newline
19. सु॒दृश॒मिति॑ सु - दृश᳚म् । \newline
20. स्वञ्च॒ मवि॑द्वाꣳ॒॒सो ऽवि॑द्वाꣳसः॒ स्वञ्चꣳ॒॒ स्वञ्च॒ मवि॑द्वाꣳसो वि॒दुष्ट॑रं ॅवि॒दुष्ट॑र॒ मवि॑द्वाꣳसः॒ स्वञ्चꣳ॒॒ स्वञ्च॒ मवि॑द्वाꣳसो वि॒दुष्ट॑रम् । \newline
21. अवि॑द्वाꣳसो वि॒दुष्ट॑रं ॅवि॒दुष्ट॑र॒ मवि॑द्वाꣳ॒॒सो ऽवि॑द्वाꣳसो वि॒दुष्ट॑रꣳ सपेम सपेम वि॒दुष्ट॑र॒ मवि॑द्वाꣳ॒॒सो ऽवि॑द्वाꣳसो वि॒दुष्ट॑रꣳ सपेम । \newline
22. वि॒दुष्ट॑रꣳ सपेम सपेम वि॒दुष्ट॑रं ॅवि॒दुष्ट॑रꣳ सपेम । \newline
23. वि॒दुष्ट॑र॒मिति॑ वि॒दुः - त॒र॒म् । \newline
24. स॒पे॒मेति॑ सपेम । \newline
25. स य॑क्षद् यक्ष॒थ् स स य॑क्ष॒द् विश्वा॒ विश्वा॑ यक्ष॒थ् स स य॑क्ष॒द् विश्वा᳚ । \newline
26. य॒क्ष॒द् विश्वा॒ विश्वा॑ यक्षद् यक्ष॒द् विश्वा॑ व॒युना॑नि व॒युना॑नि॒ विश्वा॑ यक्षद् यक्ष॒द् विश्वा॑ व॒युना॑नि । \newline
27. विश्वा॑ व॒युना॑नि व॒युना॑नि॒ विश्वा॒ विश्वा॑ व॒युना॑नि वि॒द्वान्. वि॒द्वान्. व॒युना॑नि॒ विश्वा॒ विश्वा॑ व॒युना॑नि वि॒द्वान् । \newline
28. व॒युना॑नि वि॒द्वान्. वि॒द्वान्. व॒युना॑नि व॒युना॑नि वि॒द्वान् प्र प्र वि॒द्वान्. व॒युना॑नि व॒युना॑नि वि॒द्वान् प्र । \newline
29. वि॒द्वान् प्र प्र वि॒द्वान्. वि॒द्वान् प्र ह॒व्यꣳ ह॒व्यम् प्र वि॒द्वान्. वि॒द्वान् प्र ह॒व्यम् । \newline
30. प्र ह॒व्यꣳ ह॒व्यम् प्र प्र ह॒व्य म॒ग्नि र॒ग्निर्. ह॒व्यम् प्र प्र ह॒व्य म॒ग्निः । \newline
31. ह॒व्य म॒ग्निर॒ग्निर्. ह॒व्यꣳ ह॒व्य म॒ग्नि र॒मृते᳚ ष्व॒मृते᳚ ष्व॒ग्निर्. ह॒व्यꣳ ह॒व्य म॒ग्नि र॒मृते॑षु । \newline
32. अ॒ग्नि र॒मृते᳚ ष्व॒मृते᳚ ष्व॒ग्नि र॒ग्नि र॒मृते॑षु वोचद् वोच द॒मृते᳚ ष्व॒ग्नि र॒ग्नि र॒मृते॑षु वोचत् । \newline
33. अ॒मृते॑षु वोचद् वोच द॒मृते᳚ ष्व॒मृते॑षु वोचत् । \newline
34. वो॒च॒दिति॑ वोचत् । \newline
35. अꣳ॒॒हो॒मुचे॑ वि॒वेष॑ वि॒वेषाꣳ॑हो॒मुचे ऽꣳ॑हो॒मुचे॑ वि॒वेष॒ यद् यद् वि॒वेषाꣳ॑हो॒मुचे 
ऽꣳ॑हो॒मुचे॑ वि॒वेष॒ यत् । \newline
36. अꣳ॒॒हो॒मुच॒ इत्यꣳ॑हः - मुचे᳚ । \newline
37. वि॒वेष॒ यद् यद् वि॒वेष॑ वि॒वेष॒ यन् मा॑ मा॒ यद् वि॒वेष॑ वि॒वेष॒ यन् मा᳚ । \newline
38. यन् मा॑ मा॒ यद् यन् मा॒ वि वि मा॒ यद् यन् मा॒ वि । \newline
39. मा॒ वि वि मा॑ मा॒ वि नो॑ नो॒ वि मा॑ मा॒ वि नः॑ । \newline
40. वि नो॑ नो॒ वि वि न॑ इन्द्रे न्द्र नो॒ वि वि न॑ इन्द्र । \newline
41. न॒ इ॒न्द्रे॒ न्द्र॒ नो॒ न॒ इ॒न्द्रे न्द्रे न्द्रे᳚ न्द्र नो न इ॒न्द्रे न्द्र॑ । \newline
42. इ॒न्द्रे न्द्रे न्द्रे᳚ न्द्रे॒ न्द्रे न्द्र॑ क्ष॒त्रम् क्ष॒त्र मिन्द्रे᳚ न्द्रे॒ न्द्रे न्द्र॑ क्ष॒त्रम् । \newline
43. इन्द्र॑ क्ष॒त्रम् क्ष॒त्र मिन्द्रे न्द्र॑ क्ष॒त्र मि॑न्द्रि॒या णी᳚न्द्रि॒याणि॑ क्ष॒त्र मिन्द्रे न्द्र॑ क्ष॒त्र मि॑न्द्रि॒याणि॑ । \newline
44. क्ष॒त्र मि॑न्द्रि॒या णी᳚न्द्रि॒याणि॑ क्ष॒त्रम् क्ष॒त्र मि॑न्द्रि॒याणि॑ शतक्रतो शतक्रतो इन्द्रि॒याणि॑ क्ष॒त्रम् क्ष॒त्र मि॑न्द्रि॒याणि॑ शतक्रतो । \newline
45. इ॒न्द्रि॒याणि॑ शतक्रतो शतक्रतो इन्द्रि॒या णी᳚न्द्रि॒याणि॑ शतक्र॒तोऽन्वनु॑ शतक्रतो इन्द्रि॒या णी᳚न्द्रि॒याणि॑ शतक्र॒तोऽनु॑ । \newline
46. श॒त॒क्र॒तोऽन्वनु॑ शतक्रतो शतक्र॒तोऽनु॑ ते॒ ते ऽनु॑ शतक्रतो शतक्र॒तोऽनु॑ ते । \newline
47. श॒त॒क्र॒तो॒ इति॑ शत - क्र॒तो॒ । \newline
48. अनु॑ ते ते॒ अन्वनु॑ ते दायि दायि ते॒ अन्वनु॑ ते दायि । \newline
49. ते॒ दा॒यि॒ दा॒यि॒ ते॒ ते॒ दा॒यि॒ । \newline
50. दा॒यीति॑ दायि । \newline
\pagebreak


\end{document}