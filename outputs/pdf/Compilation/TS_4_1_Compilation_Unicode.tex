\documentclass[17pt]{extarticle}
\usepackage{babel}
\usepackage{fontspec}
\usepackage{polyglossia}
\usepackage{extsizes}

\usepackage{color}   %May be necessary if you want to color links
\usepackage{hyperref}
\hypersetup{
    colorlinks=true, %set true if you want colored links
    linktoc=all,     %set to all if you want both sections and subsections linked
    linkcolor=black,  %choose some color if you want links to stand out
}

\setmainlanguage{sanskrit}
\setotherlanguages{english} %% or other languages
\setlength{\parindent}{0pt}
\pagestyle{myheadings}
\newfontfamily\devanagarifont[Script=Devanagari]{AdishilaVedic}
\renewcommand{\theHsection}{\thepart.section.\thesection}

\newcommand{\VAR}[1]{}
\newcommand{\BLOCK}[1]{}




\begin{document}
\begin{titlepage}
    \begin{center}
 
\begin{sanskrit}
    { \Large
    कृष्ण यजुर्वेदीय तैत्तिरीय संहिता,पद,जटा,घन पाठः 
    }
    \\
    \vspace{2.5cm}
    \mbox{ \Large
    4.1     चतुर्थकाण्डे प्रथमः प्रश्नः- अग्निचित्यङ्ग मन्त्रपाठाभिधानं   }
\end{sanskrit}
\end{center}

\end{titlepage}
\tableofcontents
\phantomsection
\pagebreak

\markright{ TS 4.1.1.1  \hfill https://www.vedavms.in \hfill}

\section{ TS 4.1.1.1 }

\textbf{TS 4.1.1.1 } \newline
\textbf{Samhita Paata} \newline

यु॒ञ्जा॒नः प्र॑थ॒मं मन॑स्त॒त्वाय॑ सवि॒ता धियः॑ । अ॒ग्निं ज्योति॑र्नि॒चाय्य॑ पृथि॒व्या अद्ध्या ऽभ॑रत् ॥ यु॒क्त्वाय॒ मन॑सा दे॒वान्थ् सुव॑र्य॒तो धि॒या दिवं᳚ । बृ॒हज्ज्योतिः॑ करिष्य॒तः स॑वि॒ता प्रसु॑वाति॒ तान् ॥ यु॒क्तेन॒ मन॑सा व॒यं दे॒वस्य॑ सवि॒तुः स॒वे । सु॒व॒र्गेया॑य॒ शक्त्यै᳚ ॥ यु॒ञ्जते॒ मन॑ उ॒त यु॑ञ्जते॒ धियो॒ विप्रा॒ विप्र॑स्य बृह॒तो वि॑प॒श्चितः॑ । वि होत्रा॑ दधे वयुना॒ विदेक॒ इन् - [  ] \newline

\textbf{Pada Paata} \newline

यु॒ञ्जा॒नः । प्र॒थ॒मम् । मनः॑ । त॒त्वाय॑ । स॒वि॒ता । धियः॑ ॥ अ॒ग्निम् । ज्योतिः॑ । नि॒चाय्येति॑ नि - चाय्य॑ । पृ॒थि॒व्याः । अधि॑ । एति॑ । अ॒भ॒र॒त् ॥ यु॒क्त्वाय॑ । मन॑सा । दे॒वान् । सुवः॑ । य॒तः । धि॒या । दिव᳚म् ॥ बृ॒हत् । ज्योतिः॑ । क॒रि॒ष्य॒तः । स॒वि॒ता । प्रेति॑ । सु॒वा॒ति॒ । तान् ॥ यु॒क्तेन॑ । मन॑सा । व॒यम् । दे॒वस्य॑ । स॒वि॒तुः । स॒वे ॥ सु॒व॒र्गेया॒येति॑ सुवः - गेया॑य । शक्त्यै᳚ ॥ यु॒ञ्जते᳚ । मनः॑ । उ॒त । यु॒ञ्ज॒ते॒ । धियः॑ । विप्राः᳚ । विप्र॑स्य । बृ॒ह॒तः । वि॒प॒श्चितः॑ ॥ वीति॑ । होत्राः᳚ । द॒धे॒ । व॒यु॒ना॒विदिति॑ वयुन - वित् । एकः॑ । इत् ।  \newline


\textbf{Krama Paata} \newline

यु॒ञ्जा॒नः प्र॑थ॒मम् । प्र॒थ॒मम् मनः॑ । मन॑स्त॒त्वाय॑ । त॒त्वाय॑ सवि॒ता । स॒वि॒ता धियः॑ । धिय॒ इति॒ धियः॑ ॥ अ॒ग्निम् ज्योतिः॑ । ज्योति॑र् नि॒चाय्य॑ । नि॒चाय्य॑ पृथि॒व्याः । नि॒चाय्येति॑ नि - चाय्य॑ । पृ॒थि॒व्या अधि॑ । अध्या । आ ऽभ॑रत् । अ॒भ॒र॒दित्य॑भरत् ॥ 
यु॒क्त्वाय॒ मन॑सा । मन॑सा दे॒वान् । दे॒वान्थ् सुवः॑ । सुव॑र् य॒तः । य॒तो धि॒या । धि॒या दिवम्᳚ । दिव॒मिति॒ दिव᳚म् ॥ बृ॒हज् ज्योतिः॑ । ज्योतिः॑ करिष्य॒तः । क॒रि॒ष्य॒तः स॑वि॒ता । स॒वि॒ता प्र । प्र सु॑वाति । सु॒वा॒ति॒ तान् । तानिति॒ तान् ॥ यु॒क्तेन॒ मन॑सा । मन॑सा व॒यम् । 
व॒यम् दे॒वस्य॑ । दे॒वस्य॑ सवि॒तुः । स॒वि॒तुः स॒वे । स॒व इति॑ स॒वे ॥ सु॒व॒र्गेया॑य॒ शक्त्यै᳚ । सु॒व॒र्गेया॒येति॑ सुवः - गेया॑य । शक्त्या॒ इति॒ शक्त्यै᳚ ॥ यु॒ञ्जते॒ मनः॑ । मन॑ उ॒त । उ॒त यु॑ञ्जते । यु॒ञ्ज॒ते॒ धियः॑ । धियो॒ विप्राः᳚ । विप्रा॒ विप्र॑स्य । विप्र॑स्य बृह॒तः । बृ॒ह॒तो वि॑प॒श्चितः॑ । वि॒प॒श्चित॒ इति॑ विप॒श्चितः॑ ॥ वि होत्राः᳚ । होत्रा॑ दधे । द॒धे॒ व॒यु॒ना॒वित् । व॒यु॒ना॒विदेकः॑ । व॒यु॒ना॒विदिति॑ वयुन - वित् । एक॒ इत् । इन् म॒ही \newline

\textbf{Jatai Paata} \newline

1. यु॒ञ्जा॒नः प्र॑थ॒मम् प्र॑थ॒मं ॅयु॑ञ्जा॒नो यु॑ञ्जा॒नः प्र॑थ॒मम् । \newline
2. प्र॒थ॒मम् मनो॒ मनः॑ प्रथ॒मम् प्र॑थ॒मम् मनः॑ । \newline
3. मन॑ स्त॒त्वाय॑ त॒त्वाय॒ मनो॒ मन॑ स्त॒त्वाय॑ । \newline
4. त॒त्वाय॑ सवि॒ता स॑वि॒ता त॒त्वाय॑ त॒त्वाय॑ सवि॒ता । \newline
5. स॒वि॒ता धियो॒ धियः॑ सवि॒ता स॑वि॒ता धियः॑ । \newline
6. धि॒य इति॒ धियः॑ । \newline
7. अ॒ग्निम् ज्योति॒र् ज्योति॑ र॒ग्नि म॒ग्निम् ज्योतिः॑ । \newline
8. ज्योति॑र् नि॒चाय्य॑ नि॒चाय्य॒ ज्योति॒र् ज्योति॑र् नि॒चाय्य॑ । \newline
9. नि॒चाय्य॑ पृथि॒व्याः पृ॑थि॒व्या नि॒चाय्य॑ नि॒चाय्य॑ पृथि॒व्याः । \newline
10. नि॒चाय्येति॑ नि - चाय्य॑ । \newline
11. पृ॒थि॒व्या अध्यधि॑ पृथि॒व्याः पृ॑थि॒व्या अधि॑ । \newline
12. अध्या ऽध्यध्या । \newline
13. आ ऽभ॑र दभर॒दा ऽभ॑रत् । \newline
14. अ॒भ॒र॒दित्य॑भरत् । \newline
15. यु॒क्त्वाय॒ मन॑सा॒ मन॑सा यु॒क्त्वाय॑ यु॒क्त्वाय॒ मन॑सा । \newline
16. मन॑सा दे॒वान् दे॒वान् मन॑सा॒ मन॑सा दे॒वान् । \newline
17. दे॒वान् थ्सुवः॒ सुव॑र् दे॒वान् दे॒वान् थ्सुवः॑ । \newline
18. सुव॑र् य॒तो य॒तः सुवः॒ सुव॑र् य॒तः । \newline
19. य॒तो धि॒या धि॒या य॒तो य॒तो धि॒या । \newline
20. धि॒या दिव॒म् दिव॑म् धि॒या धि॒या दिव᳚म् । \newline
21. दिव॒मिति॒ दिव᳚म् । \newline
22. बृ॒हज् ज्योति॒र् ज्योति॑र् बृ॒हद् बृ॒हज् ज्योतिः॑ । \newline
23. ज्योतिः॑ करिष्य॒तः क॑रिष्य॒तो ज्योति॒र् ज्योतिः॑ करिष्य॒तः । \newline
24. क॒रि॒ष्य॒तः स॑वि॒ता स॑वि॒ता क॑रिष्य॒तः क॑रिष्य॒तः स॑वि॒ता । \newline
25. स॒वि॒ता प्र प्र स॑वि॒ता स॑वि॒ता प्र । \newline
26. प्र सु॑वाति सुवाति॒ प्र प्र सु॑वाति । \newline
27. सु॒वा॒ति॒ ताꣳ स्तान् थ्सु॑वाति सुवाति॒ तान् । \newline
28. तानिति॒ तान् । \newline
29. यु॒क्तेन॒ मन॑सा॒ मन॑सा यु॒क्तेन॑ यु॒क्तेन॒ मन॑सा । \newline
30. मन॑सा व॒यं ॅव॒यम् मन॑सा॒ मन॑सा व॒यम् । \newline
31. व॒यम् दे॒वस्य॑ दे॒वस्य॑ व॒यं ॅव॒यम् दे॒वस्य॑ । \newline
32. दे॒वस्य॑ सवि॒तुः स॑वि॒तुर् दे॒वस्य॑ दे॒वस्य॑ सवि॒तुः । \newline
33. स॒वि॒तुः स॒वे स॒वे स॑वि॒तुः स॑वि॒तुः स॒वे । \newline
34. स॒व इति॑ स॒वे । \newline
35. सु॒व॒र्गेया॑य॒ शक्त्यै॒ शक्त्यै॑ सुव॒र्गेया॑य सुव॒र्गेया॑य॒ शक्त्यै᳚ । \newline
36. सु॒व॒र्गेया॒येति॑ सुवः - गेया॑य । \newline
37. शक्त्या॒ इति॒ शक्त्यै᳚ । \newline
38. यु॒ञ्जते॒ मनो॒ मनो॑ यु॒ञ्जते॑ यु॒ञ्जते॒ मनः॑ । \newline
39. मन॑ उ॒तोत मनो॒ मन॑ उ॒त । \newline
40. उ॒त यु॑ञ्जते युञ्जत उ॒तोत यु॑ञ्जते । \newline
41. यु॒ञ्ज॒ते॒ धियो॒ धियो॑ युञ्जते युञ्जते॒ धियः॑ । \newline
42. धियो॒ विप्रा॒ विप्रा॒ धियो॒ धियो॒ विप्राः᳚ । \newline
43. विप्रा॒ विप्र॑स्य॒ विप्र॑स्य॒ विप्रा॒ विप्रा॒ विप्र॑स्य । \newline
44. विप्र॑स्य बृह॒तो बृ॑ह॒तो विप्र॑स्य॒ विप्र॑स्य बृह॒तः । \newline
45. बृ॒ह॒तो वि॑प॒श्चितो॑ विप॒श्चितो॑ बृह॒तो बृ॑ह॒तो वि॑प॒श्चितः॑ । \newline
46. वि॒प॒श्चित॒ इति॑ विप॒श्चितः॑ । \newline
47. वि होत्रा॒ होत्रा॒ वि वि होत्राः᳚ । \newline
48. होत्रा॑ दधे दधे॒ होत्रा॒ होत्रा॑ दधे । \newline
49. द॒धे॒ व॒यु॒ना॒विद् व॑युना॒विद् द॑धे दधे वयुना॒वित् । \newline
50. व॒यु॒ना॒वि देक॒ एको॑ वयुना॒ विद्व॑युना॒वि देकः॑ । \newline
51. व॒यु॒ना॒विदिति॑ वयुन - वित् । \newline
52. एक॒ इदि देक॒ एक॒ इत् । \newline
53. इन् म॒ही म॒ही दिन् म॒ही । \newline

\textbf{Ghana Paata } \newline

1. यु॒ञ्जा॒नः प्र॑थ॒मम् प्र॑थ॒मं ॅयु॑ञ्जा॒नो यु॑ञ्जा॒नः प्र॑थ॒मम् मनो॒ मनः॑ प्रथ॒मं ॅयु॑ञ्जा॒नो यु॑ञ्जा॒नः प्र॑थ॒मम् मनः॑ । \newline
2. प्र॒थ॒मम् मनो॒ मनः॑ प्रथ॒मम् प्र॑थ॒मम् मन॑ स्त॒त्वाय॑ त॒त्वाय॒ मनः॑ प्रथ॒मम् प्र॑थ॒मम् मन॑ स्त॒त्वाय॑ । \newline
3. मन॑ स्त॒त्वाय॑ त॒त्वाय॒ मनो॒ मन॑ स्त॒त्वाय॑ सवि॒ता स॑वि॒ता त॒त्वाय॒ मनो॒ मन॑ स्त॒त्वाय॑ सवि॒ता । \newline
4. त॒त्वाय॑ सवि॒ता स॑वि॒ता त॒त्वाय॑ त॒त्वाय॑ सवि॒ता धियो॒ धियः॑ सवि॒ता त॒त्वाय॑ त॒त्वाय॑ सवि॒ता धियः॑ । \newline
5. स॒वि॒ता धियो॒ धियः॑ सवि॒ता स॑वि॒ता धियः॑ । \newline
6. धि॒य इति॒ धियः॑ । \newline
7. अ॒ग्निम् ज्योति॒र् ज्योति॑ र॒ग्नि म॒ग्निम् ज्योति॑र् नि॒चाय्य॑ नि॒चाय्य॒ ज्योति॑ र॒ग्नि म॒ग्निम् ज्योति॑र् नि॒चाय्य॑ । \newline
8. ज्योति॑र् नि॒चाय्य॑ नि॒चाय्य॒ ज्योति॒र् ज्योति॑र् नि॒चाय्य॑ पृथि॒व्याः पृ॑थि॒व्या नि॒चाय्य॒ ज्योति॒र् ज्योति॑र् नि॒चाय्य॑ पृथि॒व्याः । \newline
9. नि॒चाय्य॑ पृथि॒व्याः पृ॑थि॒व्या नि॒चाय्य॑ नि॒चाय्य॑ पृथि॒व्या अध्यधि॑ पृथि॒व्या नि॒चाय्य॑ नि॒चाय्य॑ पृथि॒व्या अधि॑ । \newline
10. नि॒चाय्येति॑ नि - चाय्य॑ । \newline
11. पृ॒थि॒व्या अध्यधि॑ पृथि॒व्याः पृ॑थि॒व्या अध्या ऽधि॑ पृथि॒व्याः पृ॑थि॒व्या अध्या । \newline
12. अध्या ऽध्यध्या ऽभ॑र दभर॒दा ऽध्यध्या ऽभ॑रत् । \newline
13. आ ऽभ॑र दभर॒दा ऽभ॑रत् । \newline
14. अ॒भ॒र॒दित्य॑भरत् । \newline
15. यु॒क्त्वाय॒ मन॑सा॒ मन॑सा यु॒क्त्वाय॑ यु॒क्त्वाय॒ मन॑सा दे॒वान् दे॒वान् मन॑सा यु॒क्त्वाय॑ यु॒क्त्वाय॒ मन॑सा दे॒वान् । \newline
16. मन॑सा दे॒वान् दे॒वान् मन॑सा॒ मन॑सा दे॒वान् थ्सुवः॒ सुव॑र् दे॒वान् मन॑सा॒ मन॑सा दे॒वान् थ्सुवः॑ । \newline
17. दे॒वान् थ्सुवः॒ सुव॑र् दे॒वान् दे॒वान् थ्सुव॑र् य॒तो य॒तः सुव॑र् दे॒वान् दे॒वान् थ्सुव॑र् य॒तः । \newline
18. सुव॑र् य॒तो य॒तः सुवः॒ सुव॑र् य॒तो धि॒या धि॒या य॒तः सुवः॒ सुव॑र् य॒तो धि॒या । \newline
19. य॒तो धि॒या धि॒या य॒तो य॒तो धि॒या दिव॒म् दिव॑म् धि॒या य॒तो य॒तो धि॒या दिव᳚म् । \newline
20. धि॒या दिव॒म् दिव॑म् धि॒या धि॒या दिव᳚म् । \newline
21. दिव॒मिति॒ दिव᳚म् । \newline
22. बृ॒हज् ज्योति॒र् ज्योति॑र् बृ॒हद् बृ॒हज् ज्योतिः॑ करिष्य॒तः क॑रिष्य॒तो ज्योति॑र् बृ॒हद् बृ॒हज् ज्योतिः॑ करिष्य॒तः । \newline
23. ज्योतिः॑ करिष्य॒तः क॑रिष्य॒तो ज्योति॒र् ज्योतिः॑ करिष्य॒तः स॑वि॒ता स॑वि॒ता क॑रिष्य॒तो ज्योति॒र् ज्योतिः॑ करिष्य॒तः स॑वि॒ता । \newline
24. क॒रि॒ष्य॒तः स॑वि॒ता स॑वि॒ता क॑रिष्य॒तः क॑रिष्य॒तः स॑वि॒ता प्र प्र स॑वि॒ता क॑रिष्य॒तः क॑रिष्य॒तः स॑वि॒ता प्र । \newline
25. स॒वि॒ता प्र प्र स॑वि॒ता स॑वि॒ता प्र सु॑वाति सुवाति॒ प्र स॑वि॒ता स॑वि॒ता प्र सु॑वाति । \newline
26. प्र सु॑वाति सुवाति॒ प्र प्र सु॑वाति॒ ताꣳ स्तान् थ्सु॑वाति॒ प्र प्र सु॑वाति॒ तान् । \newline
27. सु॒वा॒ति॒ ताꣳ स्तान् थ्सु॑वाति सुवाति॒ तान् । \newline
28. तानिति॒ तान् । \newline
29. यु॒क्तेन॒ मन॑सा॒ मन॑सा यु॒क्तेन॑ यु॒क्तेन॒ मन॑सा व॒यं ॅव॒यम् मन॑सा यु॒क्तेन॑ यु॒क्तेन॒ मन॑सा व॒यम् । \newline
30. मन॑सा व॒यं ॅव॒यम् मन॑सा॒ मन॑सा व॒यम् दे॒वस्य॑ दे॒वस्य॑ व॒यम् मन॑सा॒ मन॑सा व॒यम् दे॒वस्य॑ । \newline
31. व॒यम् दे॒वस्य॑ दे॒वस्य॑ व॒यं ॅव॒यम् दे॒वस्य॑ सवि॒तुः स॑वि॒तुर् दे॒वस्य॑ व॒यं ॅव॒यम् दे॒वस्य॑ सवि॒तुः । \newline
32. दे॒वस्य॑ सवि॒तुः स॑वि॒तुर् दे॒वस्य॑ दे॒वस्य॑ सवि॒तुः स॒वे स॒वे स॑वि॒तुर् दे॒वस्य॑ दे॒वस्य॑ सवि॒तुः स॒वे । \newline
33. स॒वि॒तुः स॒वे स॒वे स॑वि॒तुः स॑वि॒तुः स॒वे । \newline
34. स॒व इति॑ स॒वे । \newline
35. सु॒व॒र्गेया॑य॒ शक्त्यै॒ शक्त्यै॑ सुव॒र्गेया॑य सुव॒र्गेया॑य॒ शक्त्यै᳚ । \newline
36. सु॒व॒र्गेया॒येति॑ सुवः - गेया॑य । \newline
37. शक्त्या॒ इति॒ शक्त्यै᳚ । \newline
38. यु॒ञ्जते॒ मनो॒ मनो॑ यु॒ञ्जते॑ यु॒ञ्जते॒ मन॑ उ॒तोत मनो॑ यु॒ञ्जते॑ यु॒ञ्जते॒ मन॑ उ॒त । \newline
39. मन॑ उ॒तोत मनो॒ मन॑ उ॒त यु॑ञ्जते युञ्जत उ॒त मनो॒ मन॑ उ॒त यु॑ञ्जते । \newline
40. उ॒त यु॑ञ्जते युञ्जत उ॒तोत यु॑ञ्जते॒ धियो॒ धियो॑ युञ्जत उ॒तोत यु॑ञ्जते॒ धियः॑ । \newline
41. यु॒ञ्ज॒ते॒ धियो॒ धियो॑ युञ्जते युञ्जते॒ धियो॒ विप्रा॒ विप्रा॒ धियो॑ युञ्जते युञ्जते॒ धियो॒ विप्राः᳚ । \newline
42. धियो॒ विप्रा॒ विप्रा॒ धियो॒ धियो॒ विप्रा॒ विप्र॑स्य॒ विप्र॑स्य॒ विप्रा॒ धियो॒ धियो॒ विप्रा॒ विप्र॑स्य । \newline
43. विप्रा॒ विप्र॑स्य॒ विप्र॑स्य॒ विप्रा॒ विप्रा॒ विप्र॑स्य बृह॒तो बृ॑ह॒तो विप्र॑स्य॒ विप्रा॒ विप्रा॒ विप्र॑स्य बृह॒तः । \newline
44. विप्र॑स्य बृह॒तो बृ॑ह॒तो विप्र॑स्य॒ विप्र॑स्य बृह॒तो वि॑प॒श्चितो॑ विप॒श्चितो॑ बृह॒तो विप्र॑स्य॒ विप्र॑स्य बृह॒तो वि॑प॒श्चितः॑ । \newline
45. बृ॒ह॒तो वि॑प॒श्चितो॑ विप॒श्चितो॑ बृह॒तो बृ॑ह॒तो वि॑प॒श्चितः॑ । \newline
46. वि॒प॒श्चित॒ इति॑ विप॒श्चितः॑ । \newline
47. वि होत्रा॒ होत्रा॒ वि वि होत्रा॑ दधे दधे॒ होत्रा॒ वि वि होत्रा॑ दधे । \newline
48. होत्रा॑ दधे दधे॒ होत्रा॒ होत्रा॑ दधे वयुना॒विद् व॑युना॒विद् द॑धे॒ होत्रा॒ होत्रा॑ दधे वयुना॒वित् । \newline
49. द॒धे॒ व॒यु॒ना॒विद् व॑युना॒विद् द॑धे दधे वयुना॒वि देक॒ एको॑ वयुना॒विद् द॑धे दधे वयुना॒वि देकः॑ । \newline
50. व॒यु॒ना॒वि देक॒ एको॑ वयुना॒विद् व॑युना॒ विदेक॒ इदिदेको॑ वयुना॒विद् व॑युना॒ विदेक॒ इत् । \newline
51. व॒यु॒ना॒विदिति॑ वयुन - वित् । \newline
52. एक॒ इदिदेक॒ एक॒ इन् म॒ही म॒ही देक॒ एक॒ इन् म॒ही । \newline
53. इन् म॒ही म॒ही दिन् म॒ही दे॒वस्य॑ दे॒वस्य॑ म॒ही दिन् म॒ही दे॒वस्य॑ । \newline
\pagebreak
\markright{ TS 4.1.1.2  \hfill https://www.vedavms.in \hfill}

\section{ TS 4.1.1.2 }

\textbf{TS 4.1.1.2 } \newline
\textbf{Samhita Paata} \newline

म॒ही दे॒वस्य॑ सवि॒तुः परि॑ष्टुतिः ॥ यु॒जे वां॒ ब्रह्म॑ पू॒र्व्यं नमो॑भि॒र्वि श्लोका॑ यन्ति प॒थ्ये॑व॒ सूराः᳚ । शृ॒ण्वन्ति॒ विश्वे॑ अ॒मृत॑स्य पु॒त्रा आ ये धामा॑नि दि॒व्यानि॑ त॒स्थुः ॥ यस्य॑ प्र॒याण॒मन्व॒न्य इद्य॒युर्दे॒वा दे॒वस्य॑ महि॒मान॒मर्च॑तः । यः पार्थि॑वानि विम॒मे स एत॑शो॒ रजाꣳ॑सि दे॒वः स॑वि॒ता म॑हित्व॒ना ॥ देव॑ सवितः॒ प्रसु॑व य॒ज्ञ्ं प्रस॑व - [  ] \newline

\textbf{Pada Paata} \newline

म॒ही । दे॒वस्य॑ । स॒वि॒तुः । परि॑ष्टुति॒रिति॒ परि॑-स्तु॒तिः॒ ॥ यु॒जे । वा॒म् । ब्रह्म॑ । पू॒र्व्यम् । नमो॑भि॒रिति॒ नमः॑ - भिः॒ । वीति॑ । श्लोकाः᳚ । य॒न्ति॒ । प॒थ्या᳚ । इ॒व॒ । सूराः᳚ ॥ शृ॒ण्वन्ति॑ । विश्वे᳚ । अ॒मृत॑स्य । पु॒त्राः । एति॑ । ये । धामा॑नि । दि॒व्यानि॑ । त॒स्थुः ॥ यस्य॑ । प्र॒याण॒मिति॑ प्र - यान᳚म् । अन्विति॑ । अ॒न्ये । इत् । य॒युः । दे॒वाः । दे॒वस्य॑ । म॒हि॒मान᳚म् । अर्च॑तः ॥ यः । पार्थि॑वानि । वि॒म॒म इति॑ वि - म॒मे । सः । एत॑शः । रजाꣳ॑सि । दे॒वः । स॒वि॒ता । म॒हि॒त्व॒नेति॑ महि-त्व॒ना ॥ देव॑ । स॒वि॒तः॒ । प्रेति॑ । सु॒व॒ । य॒ज्ञ्म् । प्रेति॑ । सु॒व॒ ।  \newline


\textbf{Krama Paata} \newline

म॒ही दे॒वस्य॑ । दे॒वस्य॑ सवि॒तुः । स॒वि॒तुः परि॑ष्टुतिः । परि॑ष्टुति॒रिति॒ परि॑ - स्तु॒तिः॒ ॥ यु॒जे वा᳚म् । वा॒म् ब्रह्म॑ । ब्रह्म॑ पू॒र्व्यम् । पू॒र्व्यम् नमो॑भिः । नमो॑भि॒र् वि । नमो॑भि॒रिति॒ नमः॑ - भिः॒ । वि श्लोकाः᳚ । श्लोका॑ यन्ति । य॒न्ति॒ प॒थ्या᳚ । प॒थ्ये॑व । इ॒व॒ सूराः᳚ । सूरा॒ इति॒ सूराः᳚ ॥ शृ॒ण्वन्ति॒ विश्वे᳚ । विश्वे॑ अ॒मृत॑स्य । अ॒मृत॑स्य पु॒त्राः । पु॒त्रा आ । आ ये । ये धामा॑नि । धामा॑नि दि॒व्यानि॑ । दि॒व्यानि॑ त॒स्थुः । त॒स्थुरिति॑ त॒स्थुः ॥ यस्य॑ प्र॒याण᳚म् । प्र॒याण॒मनु॑ । प्र॒याण॒मिति॑ प्र - यान᳚म् । अन्व॒न्ये । अ॒न्य इत् । इद् य॒युः । य॒युर् दे॒वाः । दे॒वा दे॒वस्य॑ । दे॒वस्य॑ महि॒मान᳚म् । म॒हि॒मान॒मर्च॑तः । अर्च॑त॒ इत्यर्च॑तः ॥ यः पार्त्थि॑वानि । पार्त्थि॑वानि विम॒मे । वि॒म॒मे सः । वि॒म॒म इति॑ वि - म॒मे । 
स एत॑शः । एत॑शो॒ रजाꣳ॑सि । रजाꣳ॑सि दे॒वः । दे॒वः स॑वि॒ता । स॒वि॒ता म॑हित्व॒ना । म॒हि॒त्व॒नेति॑ महि - त्व॒ना ॥ देव॑ सवितः । स॒वि॒तः॒ प्र । प्र सु॑व । सु॒व॒ य॒ज्ञ्म् । य॒ज्ञ्म् प्र । प्र सु॑व । सु॒व॒ य॒ज्ञ्प॑तिम् \newline

\textbf{Jatai Paata} \newline

1. म॒ही दे॒वस्य॑ दे॒वस्य॑ म॒ही म॒ही दे॒वस्य॑ । \newline
2. दे॒वस्य॑ सवि॒तुः स॑वि॒तुर् दे॒वस्य॑ दे॒वस्य॑ सवि॒तुः । \newline
3. स॒वि॒तुः परि॑ष्टुतिः॒ परि॑ष्टुतिः सवि॒तुः स॑वि॒तुः परि॑ष्टुतिः । \newline
4. परि॑ष्टुति॒रिति॒ परि॑ - स्तु॒तिः॒ । \newline
5. यु॒जे वां᳚ ॅवां ॅयु॒जे यु॒जे वा᳚म् । \newline
6. वा॒म् ब्रह्म॒ ब्रह्म॑ वां ॅवा॒म् ब्रह्म॑ । \newline
7. ब्रह्म॑ पू॒र्व्यम् पू॒र्व्यम् ब्रह्म॒ ब्रह्म॑ पू॒र्व्यम् । \newline
8. पू॒र्व्यम् नमो॑भि॒र् नमो॑भिः पू॒र्व्यम् पू॒र्व्यम् नमो॑भिः । \newline
9. नमो॑भि॒र् वि वि नमो॑भि॒र् नमो॑भि॒र् वि । \newline
10. नमो॑भि॒रिति॒ नमः॑ - भिः॒ । \newline
11. वि श्लोकाः॒ श्लोका॒ वि वि श्लोकाः᳚ । \newline
12. श्लोका॑ यन्ति यन्ति॒ श्लोकाः॒ श्लोका॑ यन्ति । \newline
13. य॒न्ति॒ प॒थ्या॑ प॒थ्या॑ यन्ति यन्ति प॒थ्या᳚ । \newline
14. प॒थ्ये॑वे व प॒थ्या॑ प॒थ्ये॑व । \newline
15. इ॒व॒ सूराः॒ सूरा॑ इवे व॒ सूराः᳚ । \newline
16. सूरा॒ इति॒ सूराः᳚ । \newline
17. शृ॒ण्वन्ति॒ विश्वे॒ विश्वे॑ शृ॒ण्वन्ति॑ शृ॒ण्वन्ति॒ विश्वे᳚ । \newline
18. विश्वे॑ अ॒मृत॑स्या॒ मृत॑स्य॒ विश्वे॒ विश्वे॑ अ॒मृत॑स्य । \newline
19. अ॒मृत॑स्य पु॒त्राः पु॒त्रा अ॒मृत॑स्या॒ मृत॑स्य पु॒त्राः । \newline
20. पु॒त्रा आ पु॒त्राः पु॒त्रा आ । \newline
21. आ ये य आ ये । \newline
22. ये धामा॑नि॒ धामा॑नि॒ ये ये धामा॑नि । \newline
23. धामा॑नि दि॒व्यानि॑ दि॒व्यानि॒ धामा॑नि॒ धामा॑नि दि॒व्यानि॑ । \newline
24. दि॒व्यानि॑ त॒स्थु स्त॒स्थुर् दि॒व्यानि॑ दि॒व्यानि॑ त॒स्थुः । \newline
25. त॒स्थुरिति॑ त॒स्थुः । \newline
26. यस्य॑ प्र॒याण॑म् प्र॒याणं॒ ॅयस्य॒ यस्य॑ प्र॒याण᳚म् । \newline
27. प्र॒याण॒ मन्वनु॑ प्र॒याण॑म् प्र॒याण॒ मनु॑ । \newline
28. प्र॒याण॒मिति॑ प्र - यान᳚म् । \newline
29. अन् व॒न्ये अ॒न्ये अन् वन् व॒न्ये । \newline
30. अ॒न्य इदि द॒न्ये अ॒न्य इत् । \newline
31. इद् य॒युर् य॒ युरि दिद् य॒युः । \newline
32. य॒युर् दे॒वा दे॒वा य॒युर् य॒युर् दे॒वाः । \newline
33. दे॒वा दे॒वस्य॑ दे॒वस्य॑ दे॒वा दे॒वा दे॒वस्य॑ । \newline
34. दे॒वस्य॑ महि॒मान॑म् महि॒मान॑म् दे॒वस्य॑ दे॒वस्य॑ महि॒मान᳚म् । \newline
35. म॒हि॒मान॒ मर्च॑तो॒ अर्च॑तो महि॒मान॑म् महि॒मान॒ मर्च॑तः । \newline
36. अर्च॑त॒ इत्यर्च॑तः । \newline
37. यः पार्थि॑वानि॒ पार्थि॑वानि॒ यो यः पार्थि॑वानि । \newline
38. पार्थि॑वानि विम॒मे वि॑म॒मे पार्थि॑वानि॒ पार्थि॑वानि विम॒मे । \newline
39. वि॒म॒मे स स वि॑म॒मे वि॑म॒मे सः । \newline
40. वि॒म॒म इति॑ वि - म॒मे । \newline
41. स एत॑श॒ एत॑शः॒ स स एत॑शः । \newline
42. एत॑शो॒ रजाꣳ॑सि॒ रजाꣳ॒॒ स्येत॑श॒ एत॑शो॒ रजाꣳ॑सि । \newline
43. रजाꣳ॑सि दे॒वो दे॒वो रजाꣳ॑सि॒ रजाꣳ॑सि दे॒वः । \newline
44. दे॒वः स॑वि॒ता स॑वि॒ता दे॒वो दे॒वः स॑वि॒ता । \newline
45. स॒वि॒ता म॑हित्व॒ना म॑हित्व॒ना स॑वि॒ता स॑वि॒ता म॑हित्व॒ना । \newline
46. म॒हि॒त्व॒नेति॑ महि - त्व॒ना । \newline
47. देव॑ सवितः सवित॒र् देव॒ देव॑ सवितः । \newline
48. स॒वि॒तः॒ प्र प्र स॑वितः सवितः॒ प्र । \newline
49. प्र सु॑व सुव॒ प्र प्र सु॑व । \newline
50. सु॒व॒ य॒ज्ञ्ं ॅय॒ज्ञ्ꣳ सु॑व सुव य॒ज्ञ्म् । \newline
51. य॒ज्ञ्म् प्र प्र य॒ज्ञ्ं ॅय॒ज्ञ्म् प्र । \newline
52. प्र सु॑व सुव॒ प्र प्र सु॑व । \newline
53. सु॒व॒ य॒ज्ञ्प॑तिं ॅय॒ज्ञ्प॑तिꣳ सुव सुव य॒ज्ञ्प॑तिम् । \newline

\textbf{Ghana Paata } \newline

1. म॒ही दे॒वस्य॑ दे॒वस्य॑ म॒ही म॒ही दे॒वस्य॑ सवि॒तुः स॑वि॒तुर् दे॒वस्य॑ म॒ही म॒ही दे॒वस्य॑ सवि॒तुः । \newline
2. दे॒वस्य॑ सवि॒तुः स॑वि॒तुर् दे॒वस्य॑ दे॒वस्य॑ सवि॒तुः परि॑ष्टुतिः॒ परि॑ष्टुतिः सवि॒तुर् दे॒वस्य॑ दे॒वस्य॑ सवि॒तुः परि॑ष्टुतिः । \newline
3. स॒वि॒तुः परि॑ष्टुतिः॒ परि॑ष्टुतिः सवि॒तुः स॑वि॒तुः परि॑ष्टुतिः । \newline
4. परि॑ष्टुति॒रिति॒ परि॑ - स्तु॒तिः॒ । \newline
5. यु॒जे वां᳚ ॅवां ॅयु॒जे यु॒जे वा॒म् ब्रह्म॒ ब्रह्म॑ वां ॅयु॒जे यु॒जे वा॒म् ब्रह्म॑ । \newline
6. वा॒म् ब्रह्म॒ ब्रह्म॑ वां ॅवा॒म् ब्रह्म॑ पू॒र्व्यम् पू॒र्व्यम् ब्रह्म॑ वां ॅवा॒म् ब्रह्म॑ पू॒र्व्यम् । \newline
7. ब्रह्म॑ पू॒र्व्यम् पू॒र्व्यम् ब्रह्म॒ ब्रह्म॑ पू॒र्व्यम् नमो॑भि॒र् नमो॑भिः पू॒र्व्यम् ब्रह्म॒ ब्रह्म॑ पू॒र्व्यम् नमो॑भिः । \newline
8. पू॒र्व्यन् नमो॑भि॒र् नमो॑भिः पू॒र्व्यम् पू॒र्व्यम् नमो॑भि॒र् वि वि नमो॑भिः पू॒र्व्यम् पू॒र्व्यम् नमो॑भि॒र् वि । \newline
9. नमो॑भि॒र् वि वि नमो॑भि॒र् नमो॑भि॒र् वि श्लोकाः॒ श्लोका॒ वि नमो॑भि॒र् नमो॑भि॒र् वि श्लोकाः᳚ । \newline
10. नमो॑भि॒रिति॒ नमः॑ - भिः॒ । \newline
11. वि श्लोकाः॒ श्लोका॒ वि वि श्लोका॑ यन्ति यन्ति॒ श्लोका॒ वि वि श्लोका॑ यन्ति । \newline
12. श्लोका॑ यन्ति यन्ति॒ श्लोकाः॒ श्लोका॑ यन्ति प॒थ्या॑ प॒थ्या॑ यन्ति॒ श्लोकाः॒ श्लोका॑ यन्ति प॒थ्या᳚ । \newline
13. य॒न्ति॒ प॒थ्या॑ प॒थ्या॑ यन्ति यन्ति प॒थ्ये॑वेव प॒थ्या॑ यन्ति यन्ति प॒थ्ये॑व । \newline
14. प॒थ्ये॑वेव प॒थ्या॑ प॒थ्ये॑व॒ सूराः॒ सूरा॑ इव प॒थ्या॑ प॒थ्ये॑व॒ सूराः᳚ । \newline
15. इ॒व॒ सूराः॒ सूरा॑ इवेव॒ सूराः᳚ । \newline
16. सूरा॒ इति॒ सूराः᳚ । \newline
17. शृ॒ण्वन्ति॒ विश्वे॒ विश्वे॑ शृ॒ण्वन्ति॑ शृ॒ण्वन्ति॒ विश्वे॑ अ॒मृत॑स्या॒ मृत॑स्य॒ विश्वे॑ शृ॒ण्वन्ति॑ शृ॒ण्वन्ति॒ विश्वे॑ अ॒मृत॑स्य । \newline
18. विश्वे॑ अ॒मृत॑स्या॒ मृत॑स्य॒ विश्वे॒ विश्वे॑ अ॒मृत॑स्य पु॒त्राः पु॒त्रा अ॒मृत॑स्य॒ विश्वे॒ विश्वे॑ अ॒मृत॑स्य पु॒त्राः । \newline
19. अ॒मृत॑स्य पु॒त्राः पु॒त्रा अ॒मृत॑स्या॒ मृत॑स्य पु॒त्रा आ पु॒त्रा अ॒मृत॑स्या॒ मृत॑स्य पु॒त्रा आ । \newline
20. पु॒त्रा आ पु॒त्राः पु॒त्रा आ ये य आ पु॒त्राः पु॒त्रा आ ये । \newline
21. आ ये य आ ये धामा॑नि॒ धामा॑नि॒ य आ ये धामा॑नि । \newline
22. ये धामा॑नि॒ धामा॑नि॒ ये ये धामा॑नि दि॒व्यानि॑ दि॒व्यानि॒ धामा॑नि॒ ये ये धामा॑नि दि॒व्यानि॑ । \newline
23. धामा॑नि दि॒व्यानि॑ दि॒व्यानि॒ धामा॑नि॒ धामा॑नि दि॒व्यानि॑ त॒स्थु स्त॒स्थुर् दि॒व्यानि॒ धामा॑नि॒ धामा॑नि दि॒व्यानि॑ त॒स्थुः । \newline
24. दि॒व्यानि॑ त॒स्थु स्त॒स्थुर् दि॒व्यानि॑ दि॒व्यानि॑ त॒स्थुः । \newline
25. त॒स्थुरिति॑ त॒स्थुः । \newline
26. यस्य॑ प्र॒याण॑म् प्र॒याणं॒ ॅयस्य॒ यस्य॑ प्र॒याण॒ मन्वनु॑ प्र॒याणं॒ ॅयस्य॒ यस्य॑ प्र॒याण॒ मनु॑ । \newline
27. प्र॒याण॒ मन्वनु॑ प्र॒याण॑म् प्र॒याण॒ मन्व॒न्ये अ॒न्ये ऽनु॑ प्र॒याण॑म् प्र॒याण॒ मन्व॒न्ये । \newline
28. प्र॒याण॒मिति॑ प्र - यान᳚म् । \newline
29. अन्व॒न्ये अ॒न्ये अन्वन् व॒न्य इदिद॒न्ये अन्वन् व॒न्य इत् । \newline
30. अ॒न्य इदिद॒न्ये अ॒न्य इद् य॒युर् य॒यु रिद॒न्ये अ॒न्य इद् य॒युः । \newline
31. इद् य॒युर् य॒यु रिदिद् य॒युर् दे॒वा दे॒वा य॒यु रिदिद् य॒युर् दे॒वाः । \newline
32. य॒युर् दे॒वा दे॒वा य॒युर् य॒युर् दे॒वा दे॒वस्य॑ दे॒वस्य॑ दे॒वा य॒युर् य॒युर् दे॒वा दे॒वस्य॑ । \newline
33. दे॒वा दे॒वस्य॑ दे॒वस्य॑ दे॒वा दे॒वा दे॒वस्य॑ महि॒मान॑म् महि॒मान॑म् दे॒वस्य॑ दे॒वा दे॒वा दे॒वस्य॑ महि॒मान᳚म् । \newline
34. दे॒वस्य॑ महि॒मान॑म् महि॒मान॑म् दे॒वस्य॑ दे॒वस्य॑ महि॒मान॒ मर्च॑तो॒ अर्च॑तो महि॒मान॑म् दे॒वस्य॑ दे॒वस्य॑ महि॒मान॒ मर्च॑तः । \newline
35. म॒हि॒मान॒ मर्च॑तो॒ अर्च॑तो महि॒मान॑म् महि॒मान॒ मर्च॑तः । \newline
36. अर्च॑त॒ इत्यर्च॑तः । \newline
37. यः पार्थि॑वानि॒ पार्थि॑वानि॒ यो यः पार्थि॑वानि विम॒मे वि॑म॒मे पार्थि॑वानि॒ यो यः पार्थि॑वानि विम॒मे । \newline
38. पार्थि॑वानि विम॒मे वि॑म॒मे पार्थि॑वानि॒ पार्थि॑वानि विम॒मे स स वि॑म॒मे पार्थि॑वानि॒ पार्थि॑वानि विम॒मे सः । \newline
39. वि॒म॒मे स स वि॑म॒मे वि॑म॒मे स एत॑श॒ एत॑शः॒ स वि॑म॒मे वि॑म॒मे स एत॑शः । \newline
40. वि॒म॒म इति॑ वि - म॒मे । \newline
41. स एत॑श॒ एत॑शः॒ स स एत॑शो॒ रजाꣳ॑सि॒ रजाꣳ॒॒ स्येत॑शः॒ स स एत॑शो॒ रजाꣳ॑सि । \newline
42. एत॑शो॒ रजाꣳ॑सि॒ रजाꣳ॒॒ स्येत॑श॒ एत॑शो॒ रजाꣳ॑सि दे॒वो दे॒वो रजाꣳ॒॒ स्येत॑श॒ एत॑शो॒ रजाꣳ॑सि दे॒वः । \newline
43. रजाꣳ॑सि दे॒वो दे॒वो रजाꣳ॑सि॒ रजाꣳ॑सि दे॒वः स॑वि॒ता स॑वि॒ता दे॒वो रजाꣳ॑सि॒ रजाꣳ॑सि दे॒वः स॑वि॒ता । \newline
44. दे॒वः स॑वि॒ता स॑वि॒ता दे॒वो दे॒वः स॑वि॒ता म॑हित्व॒ना म॑हित्व॒ना स॑वि॒ता दे॒वो दे॒वः स॑वि॒ता म॑हित्व॒ना । \newline
45. स॒वि॒ता म॑हित्व॒ना म॑हित्व॒ना स॑वि॒ता स॑वि॒ता म॑हित्व॒ना । \newline
46. म॒हि॒त्व॒नेति॑ महि - त्व॒ना । \newline
47. देव॑ सवितः सवित॒र् देव॒ देव॑ सवितः॒ प्र प्र स॑वित॒र् देव॒ देव॑ सवितः॒ प्र । \newline
48. स॒वि॒तः॒ प्र प्र स॑वितः सवितः॒ प्र सु॑व सुव॒ प्र स॑वितः सवितः॒ प्र सु॑व । \newline
49. प्र सु॑व सुव॒ प्र प्र सु॑व य॒ज्ञ्ं ॅय॒ज्ञ्ꣳ सु॑व॒ प्र प्र सु॑व य॒ज्ञ्म् । \newline
50. सु॒व॒ य॒ज्ञ्ं ॅय॒ज्ञ्ꣳ सु॑व सुव य॒ज्ञ्म् प्र प्र य॒ज्ञ्ꣳ सु॑व सुव य॒ज्ञ्म् प्र । \newline
51. य॒ज्ञ्म् प्र प्र य॒ज्ञ्ं ॅय॒ज्ञ्म् प्र सु॑व सुव॒ प्र य॒ज्ञ्ं ॅय॒ज्ञ्म् प्र सु॑व । \newline
52. प्र सु॑व सुव॒ प्र प्र सु॑व य॒ज्ञ्प॑तिं ॅय॒ज्ञ्प॑तिꣳ सुव॒ प्र प्र सु॑व य॒ज्ञ्प॑तिम् । \newline
53. सु॒व॒ य॒ज्ञ्प॑तिं ॅय॒ज्ञ्प॑तिꣳ सुव सुव य॒ज्ञ्प॑ति॒म् भगा॑य॒ भगा॑य य॒ज्ञ्प॑तिꣳ सुव सुव य॒ज्ञ्प॑ति॒म् भगा॑य । \newline
\pagebreak
\markright{ TS 4.1.1.3  \hfill https://www.vedavms.in \hfill}

\section{ TS 4.1.1.3 }

\textbf{TS 4.1.1.3 } \newline
\textbf{Samhita Paata} \newline

य॒ज्ञ्प॑तिं॒ भगा॑य दि॒व्यो ग॑न्ध॒र्वः । के॒त॒पूः केत॑न्नः पुनातु वा॒चस्पति॒र्वाच॑म॒द्य स्व॑दाति नः ॥ इ॒मं नो॑ देव सवितर्य॒ज्ञ्ं प्रसु॑व देवा॒युवꣳ॑ सखि॒विदꣳ॑ सत्रा॒जितं॑ धन॒जितꣳ॑ सुव॒र्जितं᳚ ॥ ऋ॒चा स्तोमꣳ॒॒ सम॑र्द्धय गाय॒त्रेण॑ रथन्त॒रं । बृ॒हद्-गा॑य॒त्रव॑र्तनि ॥ दे॒वस्य॑ त्वा सवि॒तुः प्र॑स॒वे᳚ ऽश्विनो᳚र्बा॒हुभ्यां᳚ पू॒ष्णो हस्ता᳚भ्यां गाय॒त्रेण॒ छन्द॒सा ऽऽद॑देऽङ्गिर॒स्वदभ्रि॑रसि॒ नारि॑ - [  ] \newline

\textbf{Pada Paata} \newline

य॒ज्ञ्प॑ति॒मिति॑ य॒ज्ञ् - प॒ति॒म् । भगा॑य । दि॒व्यः । ग॒न्ध॒र्वः ॥ के॒त॒पूरिति॑ केत - पूः । केत᳚म् । नः॒ । पु॒ना॒तु॒ । वा॒चः । पतिः॑ । वाच᳚म् । अ॒द्य । स्व॒दा॒ति॒ । नः॒ ॥ इ॒मम् । नः॒ । दे॒व॒ । स॒वि॒तः॒ । य॒ज्ञ्म् । प्रेति॑ । सु॒व॒ । दे॒वा॒युव॒मिति॑ देव - युव᳚म् । स॒खि॒विद॒मिति॑ सखि - विद᳚म् । स॒त्रा॒जित॒मिति॑ सत्र - जित᳚म् । ध॒न॒जित॒मिति॑ धन - जित᳚म् । सु॒व॒र्जित॒मिति॑ सुवः-जित᳚म् ॥ ऋ॒चा । स्तोम᳚म् । समिति॑ । अ॒द्‌र्ध॒य॒ । गा॒य॒त्रेण॑ । र॒थ॒न्त॒रमिति॑ रथं - त॒रम् ॥ बृ॒हत् । गा॒य॒त्रव॑र्त॒नीति॑ गाय॒त्र - व॒र्त॒नि॒ ॥ दे॒वस्य॑ । त्वा॒ । स॒वि॒तुः । प्र॒स॒व इति॑ प्र - स॒वे । अ॒श्विनोः᳚ । बा॒हुभ्या॒मिति॑ बा॒हु - भ्या॒म् । पू॒ष्णः । हस्ता᳚भ्याम् । गा॒य॒त्रेण॑ । छन्द॑सा । एति॑ । द॒दे॒ । अ॒ङ्गि॒र॒स्वत् । अभ्रिः॑ । अ॒सि॒ । नारिः॑ ।  \newline


\textbf{Krama Paata} \newline

य॒ज्ञ्प॑ति॒म् भगा॑य । य॒ज्ञ्प॑ति॒मिति॑ य॒ज्ञ् - प॒ति॒म् । भगा॑य दि॒व्यः । दि॒व्यो ग॑न्ध॒र्वः । ग॒न्ध॒र्व इति॑ गन्ध॒र्वः ॥ के॒त॒पूः केत᳚म् । के॒त॒पूरिति॑ केत - पूः । केत॑म् नः । नः॒ पु॒ना॒तु॒ । पु॒ना॒तु॒ वा॒चः । वा॒चस्पतिः॑ । पति॒र् वाच᳚म् । वाच॑म॒द्य । अ॒द्य स्व॑दाति । स्व॒दा॒ति॒ नः॒ । न॒ इति॑ नः ॥ इ॒मम् नः॑ । नो॒ दे॒व॒ । दे॒व॒ स॒वि॒तः॒ । स॒वि॒त॒र् य॒ज्ञ्म् । य॒ज्ञ्म् प्र । प्र सु॑व । सु॒व॒ दे॒वा॒युव᳚म् । दे॒वा॒युवꣳ॑ सखि॒विद᳚म् । दे॒वा॒युव॒मिति॑ देव - युव᳚म् । स॒खि॒विदꣳ॑ सत्रा॒जित᳚म् । स॒खि॒विद॒मिति॑ सखि - विद᳚म् । स॒त्रा॒जित॑म् धन॒जित᳚म् । स॒त्रा॒जित॒मिति॑ सत्र - जित᳚म् । ध॒न॒जितꣳ॑ सुव॒र्जित᳚म् । ध॒न॒जित॒मिति॑ धन - जित᳚म् । सु॒व॒र्जित॒मिति॑ सुवः - 
जित᳚म् ॥ ऋ॒चा स्तोम᳚म् । स्तोमꣳ॒॒ सम् । सम॑र्द्धय । अ॒र्द्ध॒य॒ गा॒य॒त्रेण॑ । गा॒य॒त्रेण॑ रथन्त॒रम् । र॒थ॒न्त॒रमिति॑ रथम् - त॒रम् ॥ बृ॒हद् गा॑य॒त्रव॑र्तनि । गा॒य॒त्रव॑र्त॒नीति॑ गाय॒त्र - व॒र्त॒नि॒ ॥ दे॒वस्य॑ त्वा । त्वा॒ स॒वि॒तुः । स॒वि॒तुः प्र॑स॒वे । प्र॒स॒वे᳚ऽश्विनोः᳚ । प्र॒स॒व इति॑ प्र - स॒वे । अ॒श्विनो᳚र् बा॒हुभ्या᳚म् । बा॒हुभ्या᳚म् पू॒ष्णः । बा॒हुभ्या॒मिति॑ बा॒हु - भ्या॒म् । पू॒ष्णो हस्ता᳚भ्याम् । हस्ता᳚भ्याम् गाय॒त्रेण॑ । गा॒य॒त्रेण॒ छन्द॑सा । छन्द॒सा । आ द॑दे । द॒दे॒ ऽङ्गि॒र॒स्वत् । अ॒ङ्गि॒र॒स्वदभ्रिः॑ । अभ्रि॑रसि । अ॒सि॒ नारिः॑ । नारि॑रसि \newline

\textbf{Jatai Paata} \newline

1. य॒ज्ञ्प॑ति॒म् भगा॑य॒ भगा॑य य॒ज्ञ्प॑तिं ॅय॒ज्ञ्प॑ति॒म् भगा॑य । \newline
2. य॒ज्ञ्प॑ति॒मिति॑ य॒ज्ञ् - प॒ति॒म् । \newline
3. भगा॑य दि॒व्यो दि॒व्यो भगा॑य॒ भगा॑य दि॒व्यः । \newline
4. दि॒व्यो ग॑न्ध॒र्वो ग॑न्ध॒र्वो दि॒व्यो दि॒व्यो ग॑न्ध॒र्वः । \newline
5. ग॒न्ध॒र्व इति॑ गन्ध॒र्वः । \newline
6. के॒त॒पूः केत॒म् केत॑म् केत॒पूः के॑त॒पूः केत᳚म् । \newline
7. के॒त॒पूरिति॑ केत - पूः । \newline
8. केत॑म् नो नः॒ केत॒म् केत॑म् नः । \newline
9. नः॒ पु॒ना॒तु॒ पु॒ना॒तु॒ नो॒ नः॒ पु॒ना॒तु॒ । \newline
10. पु॒ना॒तु॒ वा॒चो वा॒चः पु॑नातु पुनातु वा॒चः । \newline
11. वा॒च स्पति॒ष् पति॑र् वा॒चो वा॒च स्पतिः॑ । \newline
12. पति॒र् वाचं॒ ॅवाच॒म् पति॒ष् पति॒र् वाच᳚म् । \newline
13. वाच॑ म॒द्याद्य वाचं॒ ॅवाच॑ म॒द्य । \newline
14. अ॒द्य स्व॑दाति स्वदा त्य॒द्याद्य स्व॑दाति । \newline
15. स्व॒दा॒ति॒ नो॒ नः॒ स्व॒दा॒ति॒ स्व॒दा॒ति॒ नः॒ । \newline
16. न॒ इति॑ नः । \newline
17. इ॒मम् नो॑ न इ॒म मि॒मम् नः॑ । \newline
18. नो॒ दे॒व॒ दे॒व॒ नो॒ नो॒ दे॒व॒ । \newline
19. दे॒व॒ स॒वि॒तः॒ स॒वि॒त॒र् दे॒व॒ दे॒व॒ स॒वि॒तः॒ । \newline
20. स॒वि॒त॒र् य॒ज्ञ्ं ॅय॒ज्ञ्ꣳ स॑वितः सवितर् य॒ज्ञ्म् । \newline
21. य॒ज्ञ्म् प्र प्र य॒ज्ञ्ं ॅय॒ज्ञ्म् प्र । \newline
22. प्र सु॑व सुव॒ प्र प्र सु॑व । \newline
23. सु॒व॒ दे॒वा॒युव॑म् देवा॒युवꣳ॑ सुव सुव देवा॒युव᳚म् । \newline
24. दे॒वा॒युवꣳ॑ सखि॒विदꣳ॑ सखि॒विद॑म् देवा॒युव॑म् देवा॒युवꣳ॑ सखि॒विद᳚म् । \newline
25. दे॒वा॒युव॒मिति॑ देव - युव᳚म् । \newline
26. स॒खि॒विदꣳ॑ सत्रा॒जितꣳ॑ सत्रा॒जितꣳ॑ सखि॒विदꣳ॑ सखि॒विदꣳ॑ सत्रा॒जित᳚म् । \newline
27. स॒खि॒विद॒मिति॑ सखि - विद᳚म् । \newline
28. स॒त्रा॒जित॑म् धन॒जित॑म् धन॒जितꣳ॑ सत्रा॒जितꣳ॑ सत्रा॒जित॑म् धन॒जित᳚म् । \newline
29. स॒त्रा॒जित॒मिति॑ सत्र - जित᳚म् । \newline
30. ध॒न॒जितꣳ॑ सुव॒र्जितꣳ॑ सुव॒र्जित॑म् धन॒जित॑म् धन॒जितꣳ॑ सुव॒र्जित᳚म् । \newline
31. ध॒न॒जित॒मिति॑ धन - जित᳚म् । \newline
32. सु॒व॒र्जित॒मिति॑ सुवः - जित᳚म् । \newline
33. ऋ॒चा स्तोमꣳ॒॒ स्तोम॑ मृ॒च र्‌चा स्तोम᳚म् । \newline
34. स्तोमꣳ॒॒ सꣳ सꣳ स्तोमꣳ॒॒ स्तोमꣳ॒॒ सम् । \newline
35. स म॑र्द्धया र्द्धय॒ सꣳ स म॑र्द्धय । \newline
36. अ॒र्द्ध॒य॒ गा॒य॒त्रेण॑ गाय॒त्रेणा᳚ र्द्धया र्द्धय गाय॒त्रेण॑ । \newline
37. गा॒य॒त्रेण॑ रथन्त॒रꣳ र॑थन्त॒रम् गा॑य॒त्रेण॑ गाय॒त्रेण॑ रथन्त॒रम् । \newline
38. र॒थ॒न्त॒रमिति॑ रथं - त॒रम् । \newline
39. बृ॒हद् गा॑य॒त्रव॑र्तनि गाय॒त्रव॑र्तनि बृ॒हद् बृ॒हद् गा॑य॒त्रव॑र्तनि । \newline
40. गा॒य॒त्रव॑र्त॒नीति॑ गाय॒त्र - व॒र्त॒नि॒ । \newline
41. दे॒वस्य॑ त्वा त्वा दे॒वस्य॑ दे॒वस्य॑ त्वा । \newline
42. त्वा॒ स॒वि॒तुः स॑वि॒तु स्त्वा᳚ त्वा सवि॒तुः । \newline
43. स॒वि॒तुः प्र॑स॒वे प्र॑स॒वे स॑वि॒तुः स॑वि॒तुः प्र॑स॒वे । \newline
44. प्र॒स॒वे᳚ ऽश्विनो॑ र॒श्विनोः᳚ प्रस॒वे प्र॑स॒वे᳚ ऽश्विनोः᳚ । \newline
45. प्र॒स॒व इति॑ प्र - स॒वे । \newline
46. अ॒श्विनो᳚र् बा॒हुभ्या᳚म् बा॒हुभ्या॑ म॒श्विनो॑ र॒श्विनो᳚र् बा॒हुभ्या᳚म् । \newline
47. बा॒हुभ्या᳚म् पू॒ष्णः पू॒ष्णो बा॒हुभ्या᳚म् बा॒हुभ्या᳚म् पू॒ष्णः । \newline
48. बा॒हुभ्या॒मिति॑ बा॒हु - भ्या॒म् । \newline
49. पू॒ष्णो हस्ता᳚भ्याꣳ॒॒ हस्ता᳚भ्याम् पू॒ष्णः पू॒ष्णो हस्ता᳚भ्याम् । \newline
50. हस्ता᳚भ्याम् गाय॒त्रेण॑ गाय॒त्रेण॒ हस्ता᳚भ्याꣳ॒॒ हस्ता᳚भ्याम् गाय॒त्रेण॑ । \newline
51. गा॒य॒त्रेण॒ छन्द॑सा॒ छन्द॑सा गाय॒त्रेण॑ गाय॒त्रेण॒ छन्द॑सा । \newline
52. छन्द॒साऽऽ च्छन्द॑सा॒ छन्द॒सा । \newline
53. आ द॑दे दद॒ आ द॑दे । \newline
54. द॒दे॒ ऽङ्गि॒र॒स्व द॑ङ्गिर॒स्वद् द॑दे ददे ऽङ्गिर॒स्वत् । \newline
55. अ॒ङ्गि॒र॒स्व दभ्रि॒ रभ्रि॑ रङ्गिर॒स्व द॑ङ्गिर॒स्व दभ्रिः॑ । \newline
56. अभ्रि॑ रस्य॒स्य भ्रि॒ रभ्रि॑ रसि । \newline
57. अ॒सि॒ नारि॒र् नारि॑ रस्यसि॒ नारिः॑ । \newline
58. नारि॑ रस्यसि॒ नारि॒र् नारि॑ रसि । \newline

\textbf{Ghana Paata } \newline

1. य॒ज्ञ्प॑ति॒म् भगा॑य॒ भगा॑य य॒ज्ञ्प॑तिं ॅय॒ज्ञ्प॑ति॒म् भगा॑य दि॒व्यो दि॒व्यो भगा॑य य॒ज्ञ्प॑तिं ॅय॒ज्ञ्प॑ति॒म् भगा॑य दि॒व्यः । \newline
2. य॒ज्ञ्प॑ति॒मिति॑ य॒ज्ञ् - प॒ति॒म् । \newline
3. भगा॑य दि॒व्यो दि॒व्यो भगा॑य॒ भगा॑य दि॒व्यो ग॑न्ध॒र्वो ग॑न्ध॒र्वो दि॒व्यो भगा॑य॒ भगा॑य दि॒व्यो ग॑न्ध॒र्वः । \newline
4. दि॒व्यो ग॑न्ध॒र्वो ग॑न्ध॒र्वो दि॒व्यो दि॒व्यो ग॑न्ध॒र्वः । \newline
5. ग॒न्ध॒र्व इति॑ गन्ध॒र्वः । \newline
6. के॒त॒पूः केत॒म् केत॑म् केत॒पूः के॑त॒पूः केत॑म् नो नः॒ केत॑म् केत॒पूः के॑त॒पूः केत॑न्नः । \newline
7. के॒त॒पूरिति॑ केत - पूः । \newline
8. केत॑न्नो नः॒ केत॒म् केत॑न्नः पुनातु पुनातु नः॒ केत॒म् केत॑म् नः पुनातु । \newline
9. नः॒ पु॒ना॒तु॒ पु॒ना॒तु॒ नो॒ नः॒ पु॒ना॒तु॒ वा॒चो वा॒चः पु॑नातु नो नः पुनातु वा॒चः । \newline
10. पु॒ना॒तु॒ वा॒चो वा॒चः पु॑नातु पुनातु वा॒च स्पति॒ष् पति॑र् वा॒चः पु॑नातु पुनातु वा॒च स्पतिः॑ । \newline
11. वा॒च स्पति॒ष् पति॑र् वा॒चो वा॒च स्पति॒र् वाचं॒ ॅवाच॒म् पति॑र् वा॒चो वा॒च स्पति॒र् वाच᳚म् । \newline
12. पति॒र् वाचं॒ ॅवाच॒म् पति॒ष् पति॒र् वाच॑ म॒द्याद्य वाच॒म् पति॒ष् पति॒र् वाच॑ म॒द्य । \newline
13. वाच॑ म॒द्याद्य वाचं॒ ॅवाच॑ म॒द्य स्व॑दाति स्वदा त्य॒द्य वाचं॒ ॅवाच॑ म॒द्य स्व॑दाति । \newline
14. अ॒द्य स्व॑दाति स्वदा त्य॒द्याद्य स्व॑दाति नो नः स्वदा त्य॒द्याद्य स्व॑दाति नः । \newline
15. स्व॒दा॒ति॒ नो॒ नः॒ स्व॒दा॒ति॒ स्व॒दा॒ति॒ नः॒ । \newline
16. न॒ इति॑ नः । \newline
17. इ॒मन् नो॑ न इ॒म मि॒मम् नो॑ देव देव न इ॒म मि॒मम् नो॑ देव । \newline
18. नो॒ दे॒व॒ दे॒व॒ नो॒ नो॒ दे॒व॒ स॒वि॒तः॒ स॒वि॒त॒र् दे॒व॒ नो॒ नो॒ दे॒व॒ स॒वि॒तः॒ । \newline
19. दे॒व॒ स॒वि॒तः॒ स॒वि॒त॒र् दे॒व॒ दे॒व॒ स॒वि॒त॒र् य॒ज्ञ्ं ॅय॒ज्ञ्ꣳ स॑वितर् देव देव सवितर् य॒ज्ञ्म् । \newline
20. स॒वि॒त॒र् य॒ज्ञ्ं ॅय॒ज्ञ्ꣳ स॑वितः सवितर् य॒ज्ञ्म् प्र प्र य॒ज्ञ्ꣳ स॑वितः सवितर् य॒ज्ञ्म् प्र । \newline
21. य॒ज्ञ्म् प्र प्र य॒ज्ञ्ं ॅय॒ज्ञ्म् प्र सु॑व सुव॒ प्र य॒ज्ञ्ं ॅय॒ज्ञ्म् प्र सु॑व । \newline
22. प्र सु॑व सुव॒ प्र प्र सु॑व देवा॒युव॑म् देवा॒युवꣳ॑ सुव॒ प्र प्र सु॑व देवा॒युव᳚म् । \newline
23. सु॒व॒ दे॒वा॒युव॑म् देवा॒युवꣳ॑ सुव सुव देवा॒युवꣳ॑ सखि॒विदꣳ॑ सखि॒विद॑म् देवा॒युवꣳ॑ सुव सुव देवा॒युवꣳ॑ सखि॒विद᳚म् । \newline
24. दे॒वा॒युवꣳ॑ सखि॒विदꣳ॑ सखि॒विद॑म् देवा॒युव॑म् देवा॒युवꣳ॑ सखि॒विदꣳ॑ सत्रा॒जितꣳ॑ सत्रा॒जितꣳ॑ सखि॒विद॑म् देवा॒युव॑म् देवा॒युवꣳ॑ सखि॒विदꣳ॑ सत्रा॒जित᳚म् । \newline
25. दे॒वा॒युव॒मिति॑ देव - युव᳚म् । \newline
26. स॒खि॒विदꣳ॑ सत्रा॒जितꣳ॑ सत्रा॒जितꣳ॑ सखि॒विदꣳ॑ सखि॒विदꣳ॑ सत्रा॒जित॑म् धन॒जित॑म् धन॒जितꣳ॑ सत्रा॒जितꣳ॑ सखि॒विदꣳ॑ सखि॒विदꣳ॑ सत्रा॒जित॑म् धन॒जित᳚म् । \newline
27. स॒खि॒विद॒मिति॑ सखि - विद᳚म् । \newline
28. स॒त्रा॒जित॑म् धन॒जित॑म् धन॒जितꣳ॑ सत्रा॒जितꣳ॑ सत्रा॒जित॑म् धन॒जितꣳ॑ सुव॒र्जितꣳ॑ सुव॒र्जित॑म् धन॒जितꣳ॑ सत्रा॒जितꣳ॑ सत्रा॒जित॑म् धन॒जितꣳ॑ सुव॒र्जित᳚म् । \newline
29. स॒त्रा॒जित॒मिति॑ सत्र - जित᳚म् । \newline
30. ध॒न॒जितꣳ॑ सुव॒र्जितꣳ॑ सुव॒र्जित॑म् धन॒जित॑म् धन॒जितꣳ॑ सुव॒र्जित᳚म् । \newline
31. ध॒न॒जित॒मिति॑ धन - जित᳚म् । \newline
32. सु॒व॒र्जित॒मिति॑ सुवः - जित᳚म् । \newline
33. ऋ॒चा स्तोमꣳ॒॒ स्तोम॑ मृ॒च र्‌चा स्तोमꣳ॒॒ सꣳ सꣳ स्तोम॑ मृ॒च र्‌चा 
स्तोमꣳ॒॒ सम् । \newline
34. स्तोमꣳ॒॒ सꣳ सꣳ स्तोमꣳ॒॒ स्तोमꣳ॒॒ सम॑र्द्ध यार्द्धय॒ सꣳ स्तोमꣳ॒॒ स्तोमꣳ॒॒ सम॑र्द्धय । \newline
35. सम॑र्द्ध यार्द्धय॒ सꣳ स म॑र्द्धय गाय॒त्रेण॑ गाय॒त्रेणा᳚ र्द्धय॒ सꣳ सम॑र्द्धय गाय॒त्रेण॑ । \newline
36. अ॒र्द्ध॒य॒ गा॒य॒त्रेण॑ गाय॒त्रेणा᳚ र्द्धया र्द्धय गाय॒त्रेण॑ रथन्त॒रꣳ र॑थन्त॒रम् गा॑य॒त्रेणा᳚ र्द्धया र्द्धय गाय॒त्रेण॑ रथन्त॒रम् । \newline
37. गा॒य॒त्रेण॑ रथन्त॒रꣳ र॑थन्त॒रम् गा॑य॒त्रेण॑ गाय॒त्रेण॑ रथन्त॒रम् । \newline
38. र॒थ॒न्त॒रमिति॑ रथं - त॒रम् । \newline
39. बृ॒हद् गा॑य॒त्रव॑र्तनि गाय॒त्रव॑र्तनि बृ॒हद् बृ॒हद् गा॑य॒त्रव॑र्तनि । \newline
40. गा॒य॒त्रव॑र्त॒नीति॑ गाय॒त्र - व॒र्त॒नि॒ । \newline
41. दे॒वस्य॑ त्वा त्वा दे॒वस्य॑ दे॒वस्य॑ त्वा सवि॒तुः स॑वि॒तु स्त्वा॑ दे॒वस्य॑ दे॒वस्य॑ त्वा सवि॒तुः । \newline
42. त्वा॒ स॒वि॒तुः स॑वि॒तु स्त्वा᳚ त्वा सवि॒तुः प्र॑स॒वे प्र॑स॒वे स॑वि॒तु स्त्वा᳚ त्वा सवि॒तुः प्र॑स॒वे । \newline
43. स॒वि॒तुः प्र॑स॒वे प्र॑स॒वे स॑वि॒तुः स॑वि॒तुः प्र॑स॒वे᳚ ऽश्विनो॑ र॒श्विनोः᳚ प्रस॒वे स॑वि॒तुः स॑वि॒तुः प्र॑स॒वे᳚ ऽश्विनोः᳚ । \newline
44. प्र॒स॒वे᳚ ऽश्विनो॑ र॒श्विनोः᳚ प्रस॒वे प्र॑स॒वे᳚ ऽश्विनो᳚र् बा॒हुभ्या᳚म् बा॒हुभ्या॑ म॒श्विनोः᳚ प्रस॒वे प्र॑स॒वे᳚ ऽश्विनो᳚र् बा॒हुभ्या᳚म् । \newline
45. प्र॒स॒व इति॑ प्र - स॒वे । \newline
46. अ॒श्विनो᳚र् बा॒हुभ्या᳚म् बा॒हुभ्या॑ म॒श्विनो॑ र॒श्विनो᳚र् बा॒हुभ्या᳚म् पू॒ष्णः पू॒ष्णो बा॒हुभ्या॑ म॒श्विनो॑ र॒श्विनो᳚र् बा॒हुभ्या᳚म् पू॒ष्णः । \newline
47. बा॒हुभ्या᳚म् पू॒ष्णः पू॒ष्णो बा॒हुभ्या᳚म् बा॒हुभ्या᳚म् पू॒ष्णो हस्ता᳚भ्याꣳ॒॒ हस्ता᳚भ्याम् पू॒ष्णो बा॒हुभ्या᳚म् बा॒हुभ्या᳚म् पू॒ष्णो हस्ता᳚भ्याम् । \newline
48. बा॒हुभ्या॒मिति॑ बा॒हु - भ्या॒म् । \newline
49. पू॒ष्णो हस्ता᳚भ्याꣳ॒॒ हस्ता᳚भ्याम् पू॒ष्णः पू॒ष्णो हस्ता᳚भ्याम् गाय॒त्रेण॑ गाय॒त्रेण॒ हस्ता᳚भ्याम् पू॒ष्णः पू॒ष्णो हस्ता᳚भ्याम् गाय॒त्रेण॑ । \newline
50. हस्ता᳚भ्याम् गाय॒त्रेण॑ गाय॒त्रेण॒ हस्ता᳚भ्याꣳ॒॒ हस्ता᳚भ्याम् गाय॒त्रेण॒ छन्द॑सा॒ छन्द॑सा गाय॒त्रेण॒ हस्ता᳚भ्याꣳ॒॒ हस्ता᳚भ्याम् गाय॒त्रेण॒ छन्द॑सा । \newline
51. गा॒य॒त्रेण॒ छन्द॑सा॒ छन्द॑सा गाय॒त्रेण॑ गाय॒त्रेण॒ छन्द॒सा ऽऽच्छन्द॑सा गाय॒त्रेण॑ गाय॒त्रेण॒ छन्द॒सा । \newline
52. छन्द॒सा ऽऽच्छन्द॑सा॒ छन्द॒सा ऽऽद॑दे दद॒ आ च्छन्द॑सा॒ छन्द॒सा ऽऽद॑दे । \newline
53. आ द॑दे दद॒ आ द॑दे ऽङ्गिर॒स्व द॑ङ्गिर॒स्वद् द॑द॒ आ द॑दे ऽङ्गिर॒स्वत् । \newline
54. द॒दे॒ ऽङ्गि॒र॒स्व द॑ङ्गिर॒स्वद् द॑दे ददे ऽङ्गिर॒स्व दभ्रि॒ रभ्रि॑ रङ्गिर॒स्वद् द॑दे ददे ऽङ्गिर॒स्व दभ्रिः॑ । \newline
55. अ॒ङ्गि॒र॒स्व दभ्रि॒ रभ्रि॑ रङ्गिर॒स्व द॑ङ्गिर॒स्व दभ्रि॑ रस्य॒ स्यभ्रि॑ रङ्गिर॒स्व द॑ङ्गिर॒स्वद् अभ्रि॑ रसि । \newline
56. अभ्रि॑ रस्य॒ स्यभ्रि॒ रभ्रि॑ रसि॒ नारि॒र् नारि॑ र॒स्यभ्रि॒ रभ्रि॑रसि॒ नारिः॑ । \newline
57. अ॒सि॒ नारि॒र् नारि॑ रस्यसि॒ नारि॑ रस्यसि॒ नारि॑ रस्यसि॒ नारि॑रसि । \newline
58. नारि॑ रस्यसि॒ नारि॒र् नारि॑ रसि पृथि॒व्याः पृ॑थि॒व्या अ॑सि॒ नारि॒र् नारि॑ रसि पृथि॒व्याः । \newline
\pagebreak
\markright{ TS 4.1.1.4  \hfill https://www.vedavms.in \hfill}

\section{ TS 4.1.1.4 }

\textbf{TS 4.1.1.4 } \newline
\textbf{Samhita Paata} \newline

-रसि पृथि॒व्याः स॒धस्था॑द॒ग्निं पु॑री॒ष्य॑मङ्गिर॒स्वदा भ॑र॒ त्रैष्टु॑भेन त्वा॒ छन्द॒सा ऽऽद॑देऽङ्गिर॒स्वद्-बभ्रि॑रसि॒ नारि॑रसि॒ त्वया॑ व॒यꣳ स॒धस्थ॒ आग्निꣳ श॑केम॒ खनि॑तुं पुरी॒ष्यं॑ जाग॑तेन त्वा॒ छन्द॒सा ऽऽद॑देऽङ्गिर॒स्वद्धस्त॑ आ॒धाय॑ सवि॒ता बिभ्र॒दभ्रिꣳ॑ हिर॒ण्ययीं᳚ । तया॒ ज्योति॒रज॑स्र॒-मिद॒ग्निं खा॒त्वी न॒ आ भ॒रानु॑ष्टुभेन त्वा॒ छन्द॒सा ( ) ऽऽद॑देऽङ्गिर॒स्वत् ॥ \newline

\textbf{Pada Paata} \newline

अ॒सि॒ । पृ॒थि॒व्याः । स॒धस्था॒दिति॑ स॒ध - स्था॒त् । अ॒ग्निम् । पु॒री॒ष्य᳚म् । अ॒ङ्गि॒र॒स्वत् । एति॑ । भ॒र॒ । त्रैष्टु॑भेन । त्वा॒ । छन्द॑सा । एति॑ । द॒दे॒ । अ॒ङ्गि॒र॒स्वत् । बभ्रिः॑ । अ॒सि॒ । नारिः॑ । अ॒सि॒ । त्वया᳚ । व॒यम् । स॒धस्थ॒ इति॑ स॒ध - स्थे॒ । एति॑ । अ॒ग्निम् । श॒के॒म॒ । खनि॑तुम् । पु॒री॒ष्य᳚म् । जाग॑तेन । त्वा॒ । छन्द॑सा । एति॑ । द॒दे॒ । अ॒ङ्गि॒र॒स्वत् । हस्ते᳚ । आ॒धायेत्या᳚ - धाय॑ । स॒वि॒ता । बिभ्र॑त् । अभ्रि᳚म् । हि॒र॒ण्ययी᳚म् ॥ तया᳚ । ज्योतिः॑ । अज॑स्रम् । इत् । अ॒ग्निम् । खा॒त्वी । नः॒ । एति॑ । भ॒र॒ । आनु॑ष्टुभे॒नेत्यानु॑ - स्तु॒भे॒न॒ । त्वा॒ । छन्द॑सा ( ) । एति॑ । द॒दे॒ । अ॒ङ्गि॒र॒स्वत् ॥  \newline


\textbf{Krama Paata} \newline

अ॒सि॒ पृ॒थि॒व्याः । पृ॒थि॒व्याः स॒धस्था᳚त् । स॒धस्था॑द॒ग्निम् । स॒धस्था॒दिति॑ स॒ध - स्था॒त्॒ । अ॒ग्निम् पु॑री॒ष्य᳚म् । पु॒री॒ष्य॑मङ्गिर॒स्वत् । अ॒ङ्गि॒र॒स्वदा । आ भ॑र । भ॒र॒ त्रैष्टु॑भेन । त्रैष्टु॑भेन त्वा । त्वा॒ छन्द॑सा । छन्द॒सा । आ द॑दे । द॒दे॒ ऽङ्गि॒र॒स्वत् । अ॒ङ्गि॒र॒स्वद् बभ्रिः॑ । बभ्रि॑रसि । अ॒सि॒ नारिः॑ । नारि॑रसि । अ॒सि॒ त्वया᳚ । त्वया॑ व॒यम् । व॒यꣳ स॒धस्थे᳚ । स॒धस्थ॒ आ । स॒धस्थ॒ इति॑ स॒ध - स्थे॒ । आऽग्निम् । अ॒ग्निꣳ श॑केम । श॒के॒म॒ खनि॑तुम् । खनि॑तुम् पुरी॒ष्य᳚म् । पु॒री॒ष्य॑म् जाग॑तेन । जाग॑तेन त्वा । त्वा॒ छन्द॑सा । छन्द॒सा । आ द॑दे । द॒दे॒ ऽङ्गि॒र॒स्वत् । अ॒ङ्गि॒र॒स्वद्धस्ते᳚ । हस्त॑ आ॒धाय॑ । आ॒धाय॑ सवि॒ता । आ॒धायेत्या᳚ - धाय॑ । स॒वि॒ता बिभ्र॑त् । बिभ्र॒दभ्रि᳚म् । अभ्रिꣳ॑ हिर॒ण्ययीम्᳚ । हि॒र॒ण्ययी॒मिति॑ हिर॒ण्ययी᳚म् ॥ तया॒ ज्योतिः॑ । ज्योति॒रज॑स्रम् । अज॑स्र॒मित् । इद॒ग्निम् । अ॒ग्निम् खा॒त्वी । खा॒त्वी नः॑ । न॒ आ । आ भ॑र । भ॒रानु॑ष्टुभेन । आनु॑ष्टुभेन त्वा ( ) । आनु॑ष्टुभे॒नेत्यानु॑ - स्तु॒भे॒न॒ । त्वा॒ छन्द॑सा । छन्द॒सा । आ द॑दे । द॒दे॒ ऽङ्गि॒र॒स्वत् । अ॒ङ्गि॒र॒स्वदित्य॑ङ्गिर॒स्वत् । \newline

\textbf{Jatai Paata} \newline

1. अ॒सि॒ पृ॒थि॒व्याः पृ॑थि॒व्या अ॑स्यसि पृथि॒व्याः । \newline
2. पृ॒थि॒व्याः स॒धस्था᳚थ् स॒धस्था᳚त् पृथि॒व्याः पृ॑थि॒व्याः स॒धस्था᳚त् । \newline
3. स॒धस्था॑ द॒ग्नि म॒ग्निꣳ स॒धस्था᳚थ् स॒धस्था॑ द॒ग्निम् । \newline
4. स॒धस्था॒दिति॑ स॒ध - स्था॒त् । \newline
5. अ॒ग्निम् पु॑री॒ष्य॑म् पुरी॒ष्य॑ म॒ग्नि म॒ग्निम् पु॑री॒ष्य᳚म् । \newline
6. पु॒री॒ष्य॑ मङ्गिर॒स्व द॑ङ्गिर॒स्वत् पु॑री॒ष्य॑म् पुरी॒ष्य॑ मङ्गिर॒स्वत् । \newline
7. अ॒ङ्गि॒र॒स्वदा ऽङ्गि॑र॒स्व द॑ङ्गिर॒स्वदा । \newline
8. आ भ॑र भ॒रा भ॑र । \newline
9. भ॒र॒ त्रैष्टु॑भेन॒ त्रैष्टु॑भेन भर भर॒ त्रैष्टु॑भेन । \newline
10. त्रैष्टु॑भेन त्वा त्वा॒ त्रैष्टु॑भेन॒ त्रैष्टु॑भेन त्वा । \newline
11. त्वा॒ छन्द॑सा॒ छन्द॑सा त्वा त्वा॒ छन्द॑सा । \newline
12. छन्द॒साऽऽ च्छन्द॑सा॒ छन्द॒सा । \newline
13. आ द॑दे दद॒ आ द॑दे । \newline
14. द॒दे॒ ऽङ्गि॒र॒स्व द॑ङ्गिर॒स्वद् द॑दे ददे ऽङ्गिर॒स्वत् । \newline
15. अ॒ङ्गि॒र॒स्वद् बभ्रि॒र् बभ्रि॑ रङ्गिर॒स्व द॑ङ्गिर॒स्वद् बभ्रिः॑ । \newline
16. बभ्रि॑ रस्यसि॒ बभ्रि॒र् बभ्रि॑ रसि । \newline
17. अ॒सि॒ नारि॒र् नारि॑ रस्यसि॒ नारिः॑ । \newline
18. नारि॑ रस्यसि॒ नारि॒र् नारि॑ रसि । \newline
19. अ॒सि॒ त्वया॒ त्वया᳚ ऽस्यसि॒ त्वया᳚ । \newline
20. त्वया॑ व॒यं ॅव॒यम् त्वया॒ त्वया॑ व॒यम् । \newline
21. व॒यꣳ स॒धस्थे॑ स॒धस्थे॑ व॒यं ॅव॒यꣳ स॒धस्थे᳚ । \newline
22. स॒धस्थ॒ आ स॒धस्थे॑ स॒धस्थ॒ आ । \newline
23. स॒धस्थ॒ इति॑ स॒ध - स्थे॒ । \newline
24. आ ऽग्नि म॒ग्नि मा ऽग्निम् । \newline
25. अ॒ग्निꣳ श॑केम शके मा॒ग्नि म॒ग्निꣳ श॑केम । \newline
26. श॒के॒म॒ खनि॑तु॒म् खनि॑तुꣳ शकेम शकेम॒ खनि॑तुम् । \newline
27. खनि॑तुम् पुरी॒ष्य॑म् पुरी॒ष्य॑म् खनि॑तु॒म् खनि॑तुम् पुरी॒ष्य᳚म् । \newline
28. पु॒री॒ष्य॑म् जाग॑तेन॒ जाग॑तेन पुरी॒ष्य॑म् पुरी॒ष्य॑म् जाग॑तेन । \newline
29. जाग॑तेन त्वा त्वा॒ जाग॑तेन॒ जाग॑तेन त्वा । \newline
30. त्वा॒ छन्द॑सा॒ छन्द॑सा त्वा त्वा॒ छन्द॑सा । \newline
31. छन्द॒साऽऽ च्छन्द॑सा॒ छन्द॒सा । \newline
32. आ द॑दे दद॒ आ द॑दे । \newline
33. द॒दे॒ ऽङ्गि॒र॒स्व द॑ङ्गिर॒स्वद् द॑दे ददे ऽङ्गिर॒स्वत् । \newline
34. अ॒ङ्गि॒र॒स्व द्धस्ते॒ हस्ते᳚ ऽङ्गिर॒स्व द॑ङ्गिर॒स्व द्धस्ते᳚ । \newline
35. हस्त॑ आ॒धाया॒ धाय॒ हस्ते॒ हस्त॑ आ॒धाय॑ । \newline
36. आ॒धाय॑ सवि॒ता स॑वि॒ता ऽऽधाया॒ धाय॑ सवि॒ता । \newline
37. आ॒धायेत्या᳚ - धाय॑ । \newline
38. स॒वि॒ता बिभ्र॒द् बिभ्र॑थ् सवि॒ता स॑वि॒ता बिभ्र॑त् । \newline
39. बिभ्र॒ दभ्रि॒ मभ्रि॒म् बिभ्र॒द् बिभ्र॒ दभ्रि᳚म् । \newline
40. अभ्रिꣳ॑ हिर॒ण्ययीꣳ॑ हिर॒ण्ययी॒ मभ्रि॒ मभ्रिꣳ॑ हिर॒ण्ययी᳚म् । \newline
41. हि॒र॒ण्ययी॒मिति॑ हिर॒ण्ययी᳚म् । \newline
42. तया॒ ज्योति॒र् ज्योति॒ स्तया॒ तया॒ ज्योतिः॑ । \newline
43. ज्योति॒ रज॑स्र॒ मज॑स्र॒म् ज्योति॒र् ज्योति॒ रज॑स्रम् । \newline
44. अज॑स्र॒ मिदि दज॑स्र॒ मज॑स्र॒ मित् । \newline
45. इद॒ग्नि म॒ग्नि मिदि द॒ग्निम् । \newline
46. अ॒ग्निम् खा॒त्वी खा॒त्व्य॑ग्नि म॒ग्निम् खा॒त्वी । \newline
47. खा॒त्वी नो॑ नः खा॒त्वी खा॒त्वी नः॑ । \newline
48. न॒ आ नो॑ न॒ आ । \newline
49. आ भ॑र भ॒रा भ॑र । \newline
50. भ॒रा नु॑ष्टुभे॒ना नु॑ष्टुभेन भर भ॒रानु॑ष्टुभेन । \newline
51. आनु॑ष्टुभेन त्वा॒ त्वा ऽऽनु॑ष्टुभे॒ना नु॑ष्टुभेन त्वा । \newline
52. आनु॑ष्टुभे॒नेत्यानु॑ - स्तु॒भे॒न॒ । \newline
53. त्वा॒ छन्द॑सा॒ छन्द॑सा त्वा त्वा॒ छन्द॑सा । \newline
54. छन्द॒साऽऽ च्छन्द॑सा॒ छन्द॒सा । \newline
55. आ द॑दे दद॒ आ द॑दे । \newline
56. द॒दे॒ ऽङ्गि॒र॒स्व द॑ङ्गिर॒स्वद् द॑दे ददे ऽङ्गिर॒स्वत् । \newline
57. अ॒ङ्गि॒र॒स्वदित्य॑ङ्गिर॒स्वत् । \newline

\textbf{Ghana Paata } \newline

1. अ॒सि॒ पृ॒थि॒व्याः पृ॑थि॒व्या अ॑स्यसि पृथि॒व्याः स॒धस्था᳚थ् स॒धस्था᳚त् पृथि॒व्या अ॑स्यसि पृथि॒व्याः स॒धस्था᳚त् । \newline
2. पृ॒थि॒व्याः स॒धस्था᳚थ् स॒धस्था᳚त् पृथि॒व्याः पृ॑थि॒व्याः स॒धस्था॑ द॒ग्नि म॒ग्निꣳ स॒धस्था᳚त् पृथि॒व्याः पृ॑थि॒व्याः स॒धस्था॑ द॒ग्निम् । \newline
3. स॒धस्था॑ द॒ग्नि म॒ग्निꣳ स॒धस्था᳚थ् स॒धस्था॑ द॒ग्निम् पु॑री॒ष्य॑म् पुरी॒ष्य॑ म॒ग्निꣳ स॒धस्था᳚थ् स॒धस्था॑ द॒ग्निम् पु॑री॒ष्य᳚म् । \newline
4. स॒धस्था॒दिति॑ स॒ध - स्था॒त् । \newline
5. अ॒ग्निम् पु॑री॒ष्य॑म् पुरी॒ष्य॑ म॒ग्नि म॒ग्निम् पु॑री॒ष्य॑ मङ्गिर॒स्व द॑ङ्गिर॒स्वत् पु॑री॒ष्य॑ म॒ग्नि म॒ग्निम् पु॑री॒ष्य॑ मङ्गिर॒स्वत् । \newline
6. पु॒री॒ष्य॑ मङ्गिर॒स्व द॑ङ्गिर॒स्वत् पु॑री॒ष्य॑म् पुरी॒ष्य॑ मङ्गिर॒स्वदा ऽङ्गि॑र॒स्वत् पु॑री॒ष्य॑म् पुरी॒ष्य॑ मङ्गिर॒स्वदा । \newline
7. अ॒ङ्गि॒र॒स्वदा ऽङ्गि॑र॒स्व द॑ङ्गिर॒स्वदा भ॑र भ॒रा ऽङ्गि॑र॒स्व द॑ङ्गिर॒स्वदा भ॑र । \newline
8. आ भ॑र भ॒रा भ॑र॒ त्रैष्टु॑भेन॒ त्रैष्टु॑भेन भ॒रा भ॑र॒ त्रैष्टु॑भेन । \newline
9. भ॒र॒ त्रैष्टु॑भेन॒ त्रैष्टु॑भेन भर भर॒ त्रैष्टु॑भेन त्वा त्वा॒ त्रैष्टु॑भेन भर भर॒ त्रैष्टु॑भेन त्वा । \newline
10. त्रैष्टु॑भेन त्वा त्वा॒ त्रैष्टु॑भेन॒ त्रैष्टु॑भेन त्वा॒ छन्द॑सा॒ छन्द॑सा त्वा॒ त्रैष्टु॑भेन॒ त्रैष्टु॑भेन त्वा॒ छन्द॑सा । \newline
11. त्वा॒ छन्द॑सा॒ छन्द॑सा त्वा त्वा॒ छन्द॒सा ऽऽच्छन्द॑सा त्वा त्वा॒ छन्द॒सा । \newline
12. छन्द॒सा ऽऽच्छन्द॑सा॒ छन्द॒सा ऽऽद॑दे दद॒ आ च्छन्द॑सा॒ छन्द॒सा ऽऽद॑दे । \newline
13. आ द॑दे दद॒ आ द॑दे ऽङ्गिर॒स्व द॑ङ्गिर॒स्वद् द॑द॒ आ द॑दे ऽङ्गिर॒स्वत् । \newline
14. द॒दे॒ ऽङ्गि॒र॒स्व द॑ङ्गिर॒स्वद् द॑दे ददे ऽङ्गिर॒स्वद् बभ्रि॒र् बभ्रि॑ रङ्गिर॒स्वद् द॑दे ददे ऽङ्गिर॒स्वद् बभ्रिः॑ । \newline
15. अ॒ङ्गि॒र॒स्वद् बभ्रि॒र् बभ्रि॑ रङ्गिर॒स्व द॑ङ्गिर॒स्वद् बभ्रि॑ रस्यसि॒ बभ्रि॑ रङ्गिर॒स्व द॑ङ्गिर॒स्वद् बभ्रि॑ रसि । \newline
16. बभ्रि॑ रस्यसि॒ बभ्रि॒र् बभ्रि॑ रसि॒ नारि॒र् नारि॑ रसि॒ बभ्रि॒र् बभ्रि॑ रसि॒ नारिः॑ । \newline
17. अ॒सि॒ नारि॒र् नारि॑ रस्यसि॒ नारि॑ रस्यसि॒ नारि॑ रस्यसि॒ नारि॑रसि । \newline
18. नारि॑ रस्यसि॒ नारि॒र् नारि॑रसि॒ त्वया॒ त्वया॑ ऽसि॒ नारि॒र् नारि॑र् असि॒ त्वया᳚ । \newline
19. अ॒सि॒ त्वया॒ त्वया᳚ ऽस्यसि॒ त्वया॑ व॒यं ॅव॒यम् त्वया᳚ ऽस्यसि॒ त्वया॑ व॒यम् । \newline
20. त्वया॑ व॒यं ॅव॒यम् त्वया॒ त्वया॑ व॒यꣳ स॒धस्थे॑ स॒धस्थे॑ व॒यम् त्वया॒ त्वया॑ व॒यꣳ स॒धस्थे᳚ । \newline
21. व॒यꣳ स॒धस्थे॑ स॒धस्थे॑ व॒यं ॅव॒यꣳ स॒धस्थ॒ आ स॒धस्थे॑ व॒यं ॅव॒यꣳ स॒धस्थ॒ आ । \newline
22. स॒धस्थ॒ आ स॒धस्थे॑ स॒धस्थ॒ आ ऽग्नि म॒ग्नि मा स॒धस्थे॑ स॒धस्थ॒ आ ऽग्निम् । \newline
23. स॒धस्थ॒ इति॑ स॒ध - स्थे॒ । \newline
24. आ ऽग्नि म॒ग्नि मा ऽग्निꣳ श॑केम शकेमा॒ग्नि मा ऽग्निꣳ श॑केम । \newline
25. अ॒ग्निꣳ श॑केम शके मा॒ग्नि म॒ग्निꣳ श॑केम॒ खनि॑तु॒म् खनि॑तुꣳ शके मा॒ग्नि म॒ग्निꣳ श॑केम॒ खनि॑तुम् । \newline
26. श॒के॒म॒ खनि॑तु॒म् खनि॑तुꣳ शकेम शकेम॒ खनि॑तुम् पुरी॒ष्य॑म् पुरी॒ष्य॑म् खनि॑तुꣳ शकेम शकेम॒ खनि॑तुम् पुरी॒ष्य᳚म् । \newline
27. खनि॑तुम् पुरी॒ष्य॑म् पुरी॒ष्य॑म् खनि॑तु॒म् खनि॑तुम् पुरी॒ष्य॑म् जाग॑तेन॒ जाग॑तेन पुरी॒ष्य॑म् खनि॑तु॒म् खनि॑तुम् पुरी॒ष्य॑म् जाग॑तेन । \newline
28. पु॒री॒ष्य॑म् जाग॑तेन॒ जाग॑तेन पुरी॒ष्य॑म् पुरी॒ष्य॑म् जाग॑तेन त्वा त्वा॒ जाग॑तेन पुरी॒ष्य॑म् पुरी॒ष्य॑म् जाग॑तेन त्वा । \newline
29. जाग॑तेन त्वा त्वा॒ जाग॑तेन॒ जाग॑तेन त्वा॒ छन्द॑सा॒ छन्द॑सा त्वा॒ जाग॑तेन॒ जाग॑तेन त्वा॒ छन्द॑सा । \newline
30. त्वा॒ छन्द॑सा॒ छन्द॑सा त्वा त्वा॒ छन्द॒सा ऽऽच्छन्द॑सा त्वा त्वा॒ छन्द॒सा । \newline
31. छन्द॒सा ऽऽच्छन्द॑सा॒ छन्द॒सा ऽऽद॑दे दद॒ आ च्छन्द॑सा॒ छन्द॒सा ऽऽद॑दे । \newline
32. आ द॑दे दद॒ आ द॑दे ऽङ्गिर॒स्व द॑ङ्गिर॒स्वद् द॑द॒ आ द॑दे ऽङ्गिर॒स्वत् । \newline
33. द॒दे॒ ऽङ्गि॒र॒स्व द॑ङ्गिर॒स्वद् द॑दे ददे ऽङ्गिर॒स्व द्धस्ते॒ हस्ते᳚ ऽङ्गिर॒स्वद् द॑दे ददे 
ऽङ्गिर॒स्व द्धस्ते᳚ । \newline
34. अ॒ङ्गि॒र॒स्व द्धस्ते॒ हस्ते᳚ ऽङ्गिर॒स्व द॑ङ्गिर॒स्व द्धस्त॑ आ॒धाया॒ धाय॒ हस्ते᳚ ऽङ्गिर॒स्व द॑ङ्गिर॒ स्वद्धस्त॑ आ॒धाय॑ । \newline
35. हस्त॑ आ॒धाया॒ धाय॒ हस्ते॒ हस्त॑ आ॒धाय॑ सवि॒ता स॑वि॒ता ऽऽधाय॒ हस्ते॒ हस्त॑ आ॒धाय॑ सवि॒ता । \newline
36. आ॒धाय॑ सवि॒ता स॑वि॒ता ऽऽधाया॒ धाय॑ सवि॒ता बिभ्र॒द् बिभ्र॑थ् सवि॒ता ऽऽधाया॒ धाय॑ सवि॒ता बिभ्र॑त् । \newline
37. आ॒धायेत्या᳚ - धाय॑ । \newline
38. स॒वि॒ता बिभ्र॒द् बिभ्र॑थ् सवि॒ता स॑वि॒ता बिभ्र॒ दभ्रि॒ मभ्रि॒म् बिभ्र॑थ् सवि॒ता स॑वि॒ता बिभ्र॒ दभ्रि᳚म् । \newline
39. बिभ्र॒ दभ्रि॒ मभ्रि॒म् बिभ्र॒द् बिभ्र॒ दभ्रिꣳ॑ हिर॒ण्ययीꣳ॑ हिर॒ण्ययी॒ मभ्रि॒म् बिभ्र॒द् बिभ्र॒ दभ्रिꣳ॑ हिर॒ण्ययी᳚म् । \newline
40. अभ्रिꣳ॑ हिर॒ण्ययीꣳ॑ हिर॒ण्ययी॒ मभ्रि॒ मभ्रिꣳ॑ हिर॒ण्ययी᳚म् । \newline
41. हि॒र॒ण्ययी॒मिति॑ हिर॒ण्ययी᳚म् । \newline
42. तया॒ ज्योति॒र् ज्योति॒ स्तया॒ तया॒ ज्योति॒ रज॑स्र॒ मज॑स्र॒म् ज्योति॒ स्तया॒ तया॒ ज्योति॒ रज॑स्रम् । \newline
43. ज्योति॒ रज॑स्र॒ मज॑स्र॒म् ज्योति॒र् ज्योति॒ रज॑स्र॒ मिदि दज॑स्र॒म् ज्योति॒र् ज्योति॒ रज॑स्र॒ मित् । \newline
44. अज॑स्र॒ मिदि दज॑स्र॒ मज॑स्र॒ मिद॒ग्नि म॒ग्नि मिदज॑स्र॒ मज॑स्र॒ मिद॒ग्निम् । \newline
45. इद॒ग्नि म॒ग्नि मिदिद॒ग्निम् खा॒त्वी खा॒त्व्य॑ग्नि मिदिद॒ग्निम् खा॒त्वी । \newline
46. अ॒ग्निम् खा॒त्वी खा॒त्व्य॑ग्नि म॒ग्निम् खा॒त्वी नो॑ नः खा॒त्व्य॑ग्नि म॒ग्निम् खा॒त्वी नः॑ । \newline
47. खा॒त्वी नो॑ नः खा॒त्वी खा॒त्वी न॒ आ नः॑ खा॒त्वी खा॒त्वी न॒ आ । \newline
48. न॒ आ नो॑ न॒ आ भ॑र भ॒रा नो॑ न॒ आ भ॑र । \newline
49. आ भ॑र भ॒रा भ॒रा नु॑ष्टुभे॒ना नु॑ष्टुभेन भ॒रा भ॒रा नु॑ष्टुभेन । \newline
50. भ॒रा नु॑ष्टुभे॒ना नु॑ष्टुभेन भर भ॒रा नु॑ष्टुभेन त्वा॒ त्वा ऽऽनु॑ष्टुभेन भर भ॒रा नु॑ष्टुभेन त्वा । \newline
51. आनु॑ष्टुभेन त्वा॒ त्वा ऽऽनु॑ष्टुभे॒ना नु॑ष्टुभेन त्वा॒ छन्द॑सा॒ छन्द॑सा॒ त्वा ऽऽनु॑ष्टुभे॒ना नु॑ष्टुभेन त्वा॒ छन्द॑सा । \newline
52. आनु॑ष्टुभे॒नेत्यानु॑ - स्तु॒भे॒न॒ । \newline
53. त्वा॒ छन्द॑सा॒ छन्द॑सा त्वा त्वा॒ छन्द॒सा ऽऽच्छन्द॑सा त्वा त्वा॒ छन्द॒सा । \newline
54. छन्द॒सा ऽऽच्छन्द॑सा॒ छन्द॒सा ऽऽद॑दे दद॒ आ च्छन्द॑सा॒ छन्द॒सा ऽऽद॑दे । \newline
55. आ द॑दे दद॒ आ द॑दे ऽङ्गिर॒स्व द॑ङ्गिर॒स्वद् द॑द॒ आ द॑दे ऽङ्गिर॒स्वत् । \newline
56. द॒दे॒ ऽङ्गि॒र॒स्व द॑ङ्गिर॒स्वद् द॑दे ददे ऽङ्गिर॒स्वत् । \newline
57. अ॒ङ्गि॒र॒स्वदित्य॑ङ्गिर॒स्वत् । \newline
\pagebreak
\markright{ TS 4.1.2.1  \hfill https://www.vedavms.in \hfill}

\section{ TS 4.1.2.1 }

\textbf{TS 4.1.2.1 } \newline
\textbf{Samhita Paata} \newline

इ॒माम॑गृभ्णन् रश॒नामृ॒तस्य॒ पूर्व॒ आयु॑षि वि॒दथे॑षु क॒व्या । तया॑ दे॒वाः सु॒तमा ब॑भूवुर्. ऋ॒तस्य॒ साम᳚न्थ् स॒रमा॒रप॑न्ती ॥ प्रतू᳚र्तं ॅवाजि॒न्ना द्र॑व॒ वरि॑ष्ठा॒मनु॑ सं॒ॅवतं᳚ । दि॒वि ते॒ जन्म॑ पर॒मम॒न्तरि॑क्षे॒ नाभिः॑ पृथि॒व्यामधि॒ योनिः॑ ॥ यु॒ञ्जाथाꣳ॒॒ रास॑भं ॅयु॒वम॒स्मिन्. यामे॑ वृषण्वसू । अ॒ग्निं भर॑न्तमस्म॒युं ॥ योगे॑योगे त॒वस्त॑रं॒ ॅवाजे॑वाजे हवामहे । सखा॑य॒ इन्द्र॑म॒तये᳚ ॥ प्र॒तूर्व॒- [  ] \newline

\textbf{Pada Paata} \newline

इ॒माम् । अ॒गृ॒भ्ण॒न्न् । र॒श॒नाम् । ऋ॒तस्य॑ । पूर्वे᳚ । आयु॑षि । वि॒दथे॑षु । क॒व्या ॥ तया᳚ । दे॒वाः । सु॒तम् । एति॑ । ब॒भू॒वुः॒ । ऋ॒तस्य॑ । सामन्न्॑ । स॒रम् । आ॒रप॒न्तीत्या᳚ - रप॑न्ती ॥ प्रतू᳚र्त॒मिति॑ प्र - तू॒र्त॒म् । वा॒जि॒न्न् । एति॑ । द्र॒व॒ । वरि॑ष्ठाम् । अन्विति॑ । सं॒ॅवत॒मिति॑ सं - वत᳚म् ॥ दि॒वि । ते॒ । जन्म॑ । प॒र॒मम् । अ॒न्तरि॑क्षे । नाभिः॑ । पृ॒थि॒व्याम् । अधीति॑ । योनिः॑ ॥ यु॒ञ्जाथा᳚म् । रास॑भम् । यु॒वम् । अ॒स्मिन्न् । यामे᳚ । वृ॒ष॒ण्व॒सू॒ इति॑ वृषण् - व॒सू॒ ॥ अ॒ग्निम् । भर॑न्तम् । अ॒स्म॒युमित्य॑स्म - युम् ॥ योगे॑योग॒ इति॒ योगे᳚ - यो॒गे॒ । त॒वस्त॑र॒मिति॑ त॒वः - त॒र॒म् । वाजे॑वाज॒ इति॒ वाजे᳚ - वा॒जे॒ । ह॒वा॒म॒हे॒ ॥ सखा॑यः । इन्द्र᳚म् । ऊ॒तये᳚ ॥ प्र॒तूर्व॒न्निति॑ प्र - तूर्वन्न्॑ ।  \newline


\textbf{Krama Paata} \newline

इ॒माम॑गृभ्णन्न् । अ॒गृ॒भ्ण॒न् र॒श॒नाम् । र॒श॒नामृ॒तस्य॑ । ऋ॒तस्य॒ पूर्वे᳚ । पूर्व॒ आयु॑षि । आयु॑षि वि॒दथे॑षु । वि॒दथे॑षु क॒व्या । क॒व्येति॑ क॒व्या ॥ तया॑ दे॒वाः । दे॒वाः सु॒तम् । सु॒तमा । आ ब॑भूवुः । ब॒भू॒वु॒र्॒. ऋ॒तस्य॑ । ऋ॒तस्य॒ सामन्न्॑ । साम᳚न्थ् स॒रम् । स॒रमा॒रप॑न्ती । आ॒रप॒न्तीत्या᳚ - रप॑न्ती ॥ प्रतू᳚र्तं ॅवाजिन्न् । प्रतू᳚र्त॒मिति॒ प्र - तू॒र्त॒म् । वा॒जि॒न्ना । आ द्र॑व । द्र॒व॒ वरि॑ष्ठाम् । वरि॑ष्ठा॒मनु॑ । अनु॑ स॒म्ॅवत᳚म् । स॒म्ॅवत॒मिति॑ सम् - वत᳚म् ॥ दि॒वि ते᳚ । ते॒ जन्म॑ । जन्म॑ पर॒मम् । प॒र॒मम॒न्तरि॑क्षे । अ॒न्तरि॑क्षे॒ नाभिः॑ । नाभिः॑ पृथि॒व्याम् । पृ॒थि॒व्यामधि॑ । अधि॒ योनिः॑ । योनि॒रिति॒ योनिः॑ ॥ यु॒ञ्जाथाꣳ॒॒ रास॑भम् । रास॑भं ॅयु॒वम् । यु॒वम॒स्मिन्न् । अ॒स्मिन्. यामे᳚ । यामे॑ वृषण्वसू । वृ॒ष॒ण्व॒सू॒ इति॑ वृषण्ण् - व॒सू॒ ॥ अ॒ग्निम् भर॑न्तम् । भर॑न्तमस्म॒युम् । अ॒स्म॒युमित्य॑स्म - युम् ॥ योगे॑योगे त॒वस्त॑रम् । योगे॑योग॒ इति॒ योगे᳚ - यो॒गे॒ । त॒वस्त॑रं॒ ॅवाजे॑वाजे । त॒वस्त॑र॒मिति॑ त॒वः - त॒र॒म् । वाजे॑वाजे हवामहे । वाजे॑वाज॒ इति॒ वाजे᳚ - वा॒जे॒ । ह॒वा॒म॒ह॒ इति॑ हवामहे ॥ सखा॑य॒ इन्द्र᳚म् । इन्द्र॑मू॒तये᳚ । ऊ॒तय॒ इत्यू॒तये᳚ ॥ प्र॒तूर्व॒न्ना । प्र॒तूर्व॒न्निति॑ प्र - तूर्वन्न्॑ \newline

\textbf{Jatai Paata} \newline

1. इ॒मा म॑गृभ्णन् नगृभ्णन् नि॒मा मि॒मा म॑गृभ्णन्न् । \newline
2. अ॒गृ॒भ्ण॒न् र॒श॒नाꣳ र॑श॒ना म॑गृभ्णन् नगृभ्णन् रश॒नाम् । \newline
3. र॒श॒ना मृ॒तस्य॒ र्‌तस्य॑ रश॒नाꣳ र॑श॒ना मृ॒तस्य॑ । \newline
4. ऋ॒तस्य॒ पूर्वे॒ पूर्व॑ ऋ॒तस्य॒ र्‌तस्य॒ पूर्वे᳚ । \newline
5. पूर्व॒ आयु॒ ष्यायु॑षि॒ पूर्वे॒ पूर्व॒ आयु॑षि । \newline
6. आयु॑षि वि॒दथे॑षु वि॒दथे॒ ष्वायु॒ ष्यायु॑षि वि॒दथे॑षु । \newline
7. वि॒दथे॑षु क॒व्या क॒व्या वि॒दथे॑षु वि॒दथे॑षु क॒व्या । \newline
8. क॒व्येति॑ क॒व्या । \newline
9. तया॑ दे॒वा दे॒वा स्तया॒ तया॑ दे॒वाः । \newline
10. दे॒वाः सु॒तꣳ सु॒तम् दे॒वा दे॒वाः सु॒तम् । \newline
11. सु॒त मा सु॒तꣳ सु॒त मा । \newline
12. आ ब॑भूवुर् बभूवु॒रा ब॑भूवुः । \newline
13. ब॒भू॒वु॒र्॒. ऋ॒तस्य॒ र्‌तस्य॑ बभूवुर् बभूवुर्. ऋ॒तस्य॑ । \newline
14. ऋ॒तस्य॒ साम॒न् थ्साम॑न् नृ॒तस्य॒ र्‌तस्य॒ सामन्न्॑ । \newline
15. सामन्᳚ थ्स॒रꣳ स॒रꣳ साम॒न् थ्सामन्᳚ थ्स॒रम् । \newline
16. स॒र मा॒रप॑ न्त्या॒रप॑न्ती स॒रꣳ स॒र मा॒रप॑न्ती । \newline
17. आ॒रप॒न्तीत्या᳚ - रप॑न्ती । \newline
18. प्रतू᳚र्तं ॅवाजिन्. वाजि॒न् प्रतू᳚र्त॒म् प्रतू᳚र्तं ॅवाजिन्न् । \newline
19. प्रतू᳚र्त॒मिति॒ प्र - तू॒र्त॒म् । \newline
20. वा॒जि॒न् ना वा॑जिन्. वाजि॒न् ना । \newline
21. आ द्र॑व द्र॒वा द्र॑व । \newline
22. द्र॒व॒ वरि॑ष्ठां॒ ॅवरि॑ष्ठाम् द्रव द्रव॒ वरि॑ष्ठाम् । \newline
23. वरि॑ष्ठा॒ मन्वनु॒ वरि॑ष्ठां॒ ॅवरि॑ष्ठा॒ मनु॑ । \newline
24. अनु॑ सं॒ॅवतꣳ॑ सं॒ॅवत॒ मन्वनु॑ सं॒ॅवत᳚म् । \newline
25. सं॒ॅवत॒मिति॑ सं - वत᳚म् । \newline
26. दि॒वि ते॑ ते दि॒वि दि॒वि ते᳚ । \newline
27. ते॒ जन्म॒ जन्म॑ ते ते॒ जन्म॑ । \newline
28. जन्म॑ पर॒मम् प॑र॒मम् जन्म॒ जन्म॑ पर॒मम् । \newline
29. प॒र॒म म॒न्तरि॑क्षे अ॒न्तरि॑क्षे पर॒मम् प॑र॒म म॒न्तरि॑क्षे । \newline
30. अ॒न्तरि॑क्षे॒ नाभि॒र् नाभि॑ र॒न्तरि॑क्षे अ॒न्तरि॑क्षे॒ नाभिः॑ । \newline
31. नाभिः॑ पृथि॒व्याम् पृ॑थि॒व्याम् नाभि॒र् नाभिः॑ पृथि॒व्याम् । \newline
32. पृ॒थि॒व्या मध्यधि॑ पृथि॒व्याम् पृ॑थि॒व्या मधि॑ । \newline
33. अधि॒ योनि॒र् योनि॒ रध्यधि॒ योनिः॑ । \newline
34. योनि॒रिति॒ योनिः॑ । \newline
35. यु॒ञ्जाथाꣳ॒॒ रास॑भꣳ॒॒ रास॑भं ॅयु॒ञ्जाथां᳚ ॅयु॒ञ्जाथाꣳ॒॒ रास॑भम् । \newline
36. रास॑भं ॅयु॒वं ॅयु॒वꣳ रास॑भꣳ॒॒ रास॑भं ॅयु॒वम् । \newline
37. यु॒व म॒स्मिन् न॒स्मिन्. यु॒वं ॅयु॒व म॒स्मिन्न् । \newline
38. अ॒स्मिन्. यामे॒ यामे॑ अ॒स्मिन् न॒स्मिन्. यामे᳚ । \newline
39. यामे॑ वृषण्व॒सू वृ॑षण्व॒सू यामे॒ यामे॑ वृषण्व॒सू । \newline
40. वृ॒ष॒ण्व॒सू॒ इति॑ वृषण् - व॒सू॒ । \newline
41. अ॒ग्निम् भर॑न्त॒म् भर॑न्त म॒ग्नि म॒ग्निम् भर॑न्तम् । \newline
42. भर॑न्त मस्म॒यु म॑स्म॒युम् भर॑न्त॒म् भर॑न्त मस्म॒युम् । \newline
43. अ॒स्म॒युमित्य॑स्म - युम् । \newline
44. योगे॑योगे त॒वस्त॑रम् त॒वस्त॑रं॒ ॅयोगे॑योगे॒ योगे॑योगे त॒वस्त॑रम् । \newline
45. योगे॑योग॒ इति॒ योगे᳚ - यो॒गे॒ । \newline
46. त॒वस्त॑रं॒ ॅवाजे॑वाजे॒ वाजे॑वाजे त॒वस्त॑रम् त॒वस्त॑रं॒ ॅवाजे॑वाजे । \newline
47. त॒वस्त॑र॒मिति॑ त॒वः - त॒र॒म् । \newline
48. वाजे॑वाजे हवामहे हवामहे॒ वाजे॑वाजे॒ वाजे॑वाजे हवामहे । \newline
49. वाजे॑वाज॒ इति॒ वाजे᳚ - वा॒जे॒ । \newline
50. ह॒वा॒म॒ह॒ इति॑ हवामहे । \newline
51. सखा॑य॒ इन्द्र॒ मिन्द्रꣳ॒॒ सखा॑यः॒ सखा॑य॒ इन्द्र᳚म् । \newline
52. इन्द्र॑ मू॒तय॑ ऊ॒तय॒ इन्द्र॒ मिन्द्र॑ मू॒तये᳚ । \newline
53. ऊ॒तय॒ इत्यू॒तये᳚ । \newline
54. प्र॒तूर्व॒न् ना प्र॒तूर्व॑न् प्र॒तूर्व॒न् ना । \newline
55. प्र॒तूर्व॒न्निति॑ प्र - तूर्वन्न्॑ । \newline

\textbf{Ghana Paata } \newline

1. इ॒मा म॑गृभ्णन् नगृभ्णन् नि॒मा मि॒मा म॑गृभ्णन् रश॒नाꣳ र॑श॒ना म॑गृभ्णन् नि॒मा मि॒मा म॑गृभ्णन् रश॒नाम् । \newline
2. अ॒गृ॒भ्ण॒न् र॒श॒नाꣳ र॑श॒ना म॑गृभ्णन् नगृभ्णन् रश॒ना मृ॒तस्य॒ र्‌तस्य॑ रश॒ना म॑गृभ्णन् नगृभ्णन् रश॒ना मृ॒तस्य॑ । \newline
3. र॒श॒ना मृ॒तस्य॒ र्‌तस्य॑ रश॒नाꣳ र॑श॒ना मृ॒तस्य॒ पूर्वे॒ पूर्व॑ ऋ॒तस्य॑ रश॒नाꣳ र॑श॒ना मृ॒तस्य॒ पूर्वे᳚ । \newline
4. ऋ॒तस्य॒ पूर्वे॒ पूर्व॑ ऋ॒तस्य॒ र्‌तस्य॒ पूर्व॒ आयु॒ष्या यु॑षि॒ पूर्व॑ ऋ॒तस्य॒ र्‌तस्य॒ पूर्व॒ आयु॑षि । \newline
5. पूर्व॒ आयु॒ष्या यु॑षि॒ पूर्वे॒ पूर्व॒ आयु॑षि वि॒दथे॑षु वि॒दथे॒ ष्वायु॑षि॒ पूर्वे॒ पूर्व॒ आयु॑षि वि॒दथे॑षु । \newline
6. आयु॑षि वि॒दथे॑षु वि॒दथे॒ ष्वायु॒ ष्यायु॑षि वि॒दथे॑षु क॒व्या क॒व्या वि॒दथे॒ ष्वायु॒ ष्यायु॑षि वि॒दथे॑षु क॒व्या । \newline
7. वि॒दथे॑षु क॒व्या क॒व्या वि॒दथे॑षु वि॒दथे॑षु क॒व्या । \newline
8. क॒व्येति॑ क॒व्या । \newline
9. तया॑ दे॒वा दे॒वा स्तया॒ तया॑ दे॒वाः सु॒तꣳ सु॒तम् दे॒वा स्तया॒ तया॑ दे॒वाः सु॒तम् । \newline
10. दे॒वाः सु॒तꣳ सु॒तम् दे॒वा दे॒वाः सु॒तमा सु॒तम् दे॒वा दे॒वाः सु॒तमा । \newline
11. सु॒तमा सु॒तꣳ सु॒तमा ब॑भूवुर् बभूवु॒रा सु॒तꣳ सु॒तमा ब॑भूवुः । \newline
12. आ ब॑भूवुर् बभूवु॒रा ब॑भूवुर्. ऋ॒तस्य॒ र्‌तस्य॑ बभूवु॒रा ब॑भूवुर्. ऋ॒तस्य॑ । \newline
13. ब॒भू॒वु॒र्॒. ऋ॒तस्य॒ र्‌तस्य॑ बभूवुर् बभूवुर्. ऋ॒तस्य॒ साम॒न् थ्साम॑न् नृ॒तस्य॑ बभूवुर् बभूवुर्. ऋ॒तस्य॒ सामन्न्॑ । \newline
14. ऋ॒तस्य॒ साम॒न् थ्साम॑न् नृ॒तस्य॒ र्‌तस्य॒ सामन्᳚ थ्स॒रꣳ स॒रꣳ साम॑न् नृ॒तस्य॒ र्‌तस्य॒ सामन्᳚ थ्स॒रम् । \newline
15. सामन्᳚ थ्स॒रꣳ स॒रꣳ साम॒न् थ्सामन्᳚ थ्स॒र मा॒रप॑ न्त्या॒रप॑न्ती स॒रꣳ साम॒न् थ्सामन्᳚ थ्स॒र मा॒रप॑न्ती । \newline
16. स॒र मा॒रप॑ न्त्या॒रप॑न्ती स॒रꣳ स॒र मा॒रप॑न्ती । \newline
17. आ॒रप॒न्तीत्या᳚ - रप॑न्ती । \newline
18. प्रतू᳚र्तं ॅवाजिन्. वाजि॒न् प्रतू᳚र्त॒म् प्रतू᳚र्तं ॅवाजि॒न्ना वा॑जि॒न् प्रतू᳚र्त॒म् प्रतू᳚र्तं ॅवाजि॒न्ना । \newline
19. प्रतू᳚र्त॒मिति॒ प्र - तू॒र्त॒म् । \newline
20. वा॒जि॒न्ना वा॑जिन्. वाजि॒न्ना द्र॑व द्र॒वा वा॑जिन्. वाजि॒न्ना द्र॑व । \newline
21. आ द्र॑व द्र॒वा द्र॑व॒ वरि॑ष्ठां॒ ॅवरि॑ष्ठाम् द्र॒वा द्र॑व॒ वरि॑ष्ठाम् । \newline
22. द्र॒व॒ वरि॑ष्ठां॒ ॅवरि॑ष्ठाम् द्रव द्रव॒ वरि॑ष्ठा॒ मन्वनु॒ वरि॑ष्ठाम् द्रव द्रव॒ वरि॑ष्ठा॒ मनु॑ । \newline
23. वरि॑ष्ठा॒ मन्वनु॒ वरि॑ष्ठां॒ ॅवरि॑ष्ठा॒ मनु॑ सं॒ॅवतꣳ॑ सं॒ॅवत॒ मनु॒ वरि॑ष्ठां॒ ॅवरि॑ष्ठा॒ मनु॑ सं॒ॅवत᳚म् । \newline
24. अनु॑ सं॒ॅवतꣳ॑ सं॒ॅवत॒ मन्वनु॑ सं॒ॅवत᳚म् । \newline
25. सं॒ॅवत॒मिति॑ सं - वत᳚म् । \newline
26. दि॒वि ते॑ ते दि॒वि दि॒वि ते॒ जन्म॒ जन्म॑ ते दि॒वि दि॒वि ते॒ जन्म॑ । \newline
27. ते॒ जन्म॒ जन्म॑ ते ते॒ जन्म॑ पर॒मम् प॑र॒मम् जन्म॑ ते ते॒ जन्म॑ पर॒मम् । \newline
28. जन्म॑ पर॒मम् प॑र॒मम् जन्म॒ जन्म॑ पर॒म म॒न्तरि॑क्षे अ॒न्तरि॑क्षे पर॒मम् जन्म॒ जन्म॑ पर॒म म॒न्तरि॑क्षे । \newline
29. प॒र॒म म॒न्तरि॑क्षे अ॒न्तरि॑क्षे पर॒मम् प॑र॒म म॒न्तरि॑क्षे॒ नाभि॒र् नाभि॑ र॒न्तरि॑क्षे पर॒मम् प॑र॒म म॒न्तरि॑क्षे॒ नाभिः॑ । \newline
30. अ॒न्तरि॑क्षे॒ नाभि॒र् नाभि॑ र॒न्तरि॑क्षे अ॒न्तरि॑क्षे॒ नाभिः॑ पृथि॒व्याम् पृ॑थि॒व्यान् नाभि॑ र॒न्तरि॑क्षे अ॒न्तरि॑क्षे॒ नाभिः॑ पृथि॒व्याम् । \newline
31. नाभिः॑ पृथि॒व्याम् पृ॑थि॒व्याम् नाभि॒र् नाभिः॑ पृथि॒व्या मध्यधि॑ पृथि॒व्याम् नाभि॒र् नाभिः॑ पृथि॒व्या मधि॑ । \newline
32. पृ॒थि॒व्या मध्यधि॑ पृथि॒व्याम् पृ॑थि॒व्या मधि॒ योनि॒र् योनि॒ रधि॑ पृथि॒व्याम् पृ॑थि॒व्या मधि॒ योनिः॑ । \newline
33. अधि॒ योनि॒र् योनि॒ रध्यधि॒ योनिः॑ । \newline
34. योनि॒रिति॒ योनिः॑ । \newline
35. यु॒ञ्जाथाꣳ॒॒ रास॑भꣳ॒॒ रास॑भं ॅयु॒ञ्जाथां᳚ ॅयु॒ञ्जाथाꣳ॒॒ रास॑भं ॅयु॒वं ॅयु॒वꣳ रास॑भं ॅयु॒ञ्जाथां᳚ ॅयु॒ञ्जाथाꣳ॒॒ रास॑भं ॅयु॒वम् । \newline
36. रास॑भं ॅयु॒वं ॅयु॒वꣳ रास॑भꣳ॒॒ रास॑भं ॅयु॒व म॒स्मिन् न॒स्मिन्. यु॒वꣳ रास॑भꣳ॒॒ रास॑भं ॅयु॒व म॒स्मिन्न् । \newline
37. यु॒व म॒स्मिन् न॒स्मिन्. यु॒वं ॅयु॒व म॒स्मिन्. यामे॒ यामे॑ अ॒स्मिन्. यु॒वं ॅयु॒व म॒स्मिन्. यामे᳚ । \newline
38. अ॒स्मिन्. यामे॒ यामे॑ अ॒स्मिन् न॒स्मिन्. यामे॑ वृषण्व॒सू वृ॑षण्व॒सू यामे॑ अ॒स्मिन् न॒स्मिन्. यामे॑ वृषण्व॒सू । \newline
39. यामे॑ वृषण्व॒सू वृ॑षण्व॒सू यामे॒ यामे॑ वृषण्व॒सू । \newline
40. वृ॒ष॒ण्व॒सू॒ इति॑ वृषण् - व॒सू॒ । \newline
41. अ॒ग्निम् भर॑न्त॒म् भर॑न्त म॒ग्नि म॒ग्निम् भर॑न्त मस्म॒यु म॑स्म॒युम् भर॑न्त म॒ग्नि म॒ग्निम् भर॑न्त मस्म॒युम् । \newline
42. भर॑न्त मस्म॒यु म॑स्म॒युम् भर॑न्त॒म् भर॑न्त मस्म॒युम् । \newline
43. अ॒स्म॒युमित्य॑स्म - युम् । \newline
44. योगे॑योगे त॒वस्त॑रम् त॒वस्त॑रं॒ ॅयोगे॑योगे॒ योगे॑योगे त॒वस्त॑रं॒ ॅवाजे॑वाजे॒ वाजे॑वाजे त॒वस्त॑रं॒ ॅयोगे॑योगे॒ योगे॑योगे त॒वस्त॑रं॒ ॅवाजे॑वाजे । \newline
45. योगे॑योग॒ इति॒ योगे᳚ - यो॒गे॒ । \newline
46. त॒वस्त॑रं॒ ॅवाजे॑वाजे॒ वाजे॑वाजे त॒वस्त॑रम् त॒वस्त॑रं॒ ॅवाजे॑वाजे हवामहे हवामहे॒ वाजे॑वाजे त॒वस्त॑रम् त॒वस्त॑रं॒ ॅवाजे॑वाजे हवामहे । \newline
47. त॒वस्त॑र॒मिति॑ त॒वः - त॒र॒म् । \newline
48. वाजे॑वाजे हवामहे हवामहे॒ वाजे॑वाजे॒ वाजे॑वाजे हवामहे । \newline
49. वाजे॑वाज॒ इति॒ वाजे᳚ - वा॒जे॒ । \newline
50. ह॒वा॒म॒ह॒ इति॑ हवामहे । \newline
51. सखा॑य॒ इन्द्र॒ मिन्द्रꣳ॒॒ सखा॑यः॒ सखा॑य॒ इन्द्र॑ मू॒तय॑ ऊ॒तय॒ इन्द्रꣳ॒॒ सखा॑यः॒ सखा॑य॒ इन्द्र॑ मू॒तये᳚ । \newline
52. इन्द्र॑ मू॒तय॑ ऊ॒तय॒ इन्द्र॒ मिन्द्र॑ मू॒तये᳚ । \newline
53. ऊ॒तय॒ इत्यू॒तये᳚ । \newline
54. प्र॒तूर्व॒न्ना प्र॒तूर्व॑न् प्र॒तूर्व॒न् नेही॒ह्या प्र॒तूर्व॑न् प्र॒तूर्व॒न् नेहि॑ । \newline
55. प्र॒तूर्व॒न्निति॑ प्र - तूर्वन्न्॑ । \newline
\pagebreak
\markright{ TS 4.1.2.2  \hfill https://www.vedavms.in \hfill}

\section{ TS 4.1.2.2 }

\textbf{TS 4.1.2.2 } \newline
\textbf{Samhita Paata} \newline

-न्नेह्य॑व॒क्राम॒न्नश॑स्ती रु॒द्रस्य॒ गाण॑पत्यान् मयो॒भूरेहि॑ । उ॒र्व॑न्तरि॑क्ष॒मन्वि॑हि स्व॒स्ति ग॑व्यूति॒रभ॑यानि कृ॒ण्वन्न् ॥ पू॒ष्णा स॒युजा॑ स॒ह । पृ॒थि॒व्याः स॒धस्था॑द॒ग्निं पु॑रि॒ष्य॑-मङ्गिर॒स्व-दच्छे᳚ह्य॒ग्निं पु॑री॒ष्य॑ -मङ्गिर॒स्वद-च्छे॑मो॒ऽग्निं पु॑री॒ष्य॑-मङ्गिर॒स्वद्-भ॑रिष्यामो॒ऽग्निं पु॑री॒ष्य॑-मङ्गिर॒स्वद्-भ॑रामः ॥ अन्व॒ग्निरु॒षसा॒-मग्र॑मख्य॒-दन्वहा॑नि प्रथ॒मो जा॒तवे॑दाः । अनु॒ सूर्य॑स्य - [  ] \newline

\textbf{Pada Paata} \newline

एति॑ । इ॒हि॒ । अ॒व॒क्राम॒न्नित्य॑व - क्रामन्न्॑ । अश॑स्तीः । रु॒द्रस्य॑ । गाण॑पत्या॒दिति॒ गाण॑ - प॒त्या॒त् । म॒यो॒भूरिति॑ मयः - भूः । एति॑ । इ॒हि॒ ॥ उ॒रु । अ॒न्तरि॑क्षम् । अन्विति॑ । इ॒हि॒ । स्व॒स्तिग॑व्यूति॒रिति॑ स्व॒स्ति - ग॒व्यू॒तिः॒ । अभ॑यानि । कृ॒ण्वन्न् ॥ पू॒ष्णा । स॒युजेति॑ स - युजा᳚ । स॒ह ॥ पृ॒थि॒व्याः । स॒धस्था॒दिति॑ स॒ध - स्था॒त् । अ॒ग्निम् । पु॒रि॒ष्य᳚म् । अ॒ङ्गि॒र॒स्वत् । अच्छ॑ । इ॒हि॒ । अ॒ग्निम् । पु॒री॒ष्य᳚म् । अ॒ङ्गि॒र॒स्वत् । अच्छ॑ । इ॒मः॒ । अ॒ग्निम् । पु॒री॒ष्य᳚म् । अ॒ङ्गि॒र॒स्वत् । भ॒रि॒ष्या॒मः॒ । अ॒ग्निम् । पु॒री॒ष्य᳚म् । अ॒ङ्गि॒र॒स्वत् । भ॒रा॒मः॒ ॥ अन्विति॑ । अ॒ग्निः । उ॒षसा᳚म् । अग्र᳚म् । अ॒ख्य॒त् । अन्विति॑ । अहा॑नि । प्र॒थ॒मः । जा॒तवे॑दा॒ इति॑ जा॒त - वे॒दाः॒ ॥ अन्विति॑ । सूर्य॑स्य ।  \newline


\textbf{Krama Paata} \newline

एहि॑ । इ॒ह्य॒व॒क्रामन्न्॑ । अ॒व॒क्राम॒न्नश॑स्तीः । अ॒व॒क्राम॒न्नित्य॑व - क्रामन्न्॑ । अश॑स्ती रु॒द्रस्य॑ । रु॒द्रस्य॒ गाण॑पत्यात् । गाण॑पत्यान् मयो॒भूः । गाण॑पत्या॒दिति॒ गाण॑ - प॒त्या॒त्॒ । म॒यो॒भूरा । म॒यो॒भूरिति॑ मयः - भूः । एहि॑ । इ॒हीती॑हि ॥ उ॒र्व॑न्तरि॑क्षम् । अ॒न्तरि॑क्ष॒मनु॑ । अन्वि॑हि । इ॒हि॒ स्व॒स्तिग॑व्यूतिः । स्व॒स्तिग॑व्यूति॒रभ॑यानि । स्व॒स्तिग॑व्यूति॒रिति॑ स्व॒स्ति - ग॒व्यू॒तिः॒ । अभ॑यानि कृ॒ण्वन्न् । कृ॒ण्वन्निति॑ कृ॒ण्वन्न् ॥ पू॒ष्णा स॒युजा᳚ । स॒युजा॑ स॒ह । स॒युजेति॑ स - युजा᳚ । स॒हेति॑ स॒ह ॥ पृ॒थि॒व्याः स॒धस्था᳚त् । स॒धस्था॑द॒ग्निम् । स॒धस्था॒दिति॑स॒ध - स्था॒त्॒ । अ॒ग्निम् पु॑री॒ष्य᳚म् । पु॒री॒ष्य॑मङ्गिर॒स्वत् । अ॒ङ्गि॒र॒स्वदच्छ॑ । अच्छे॑हि । इ॒ह्य॒ग्निम् । अ॒ग्निम् पु॑री॒ष्य᳚म् । पु॒री॒ष्य॑मङ्गिर॒स्वत् । अ॒ङ्गि॒र॒स्वदच्छ॑ । अच्छे॑मः । इ॒मो॒ऽग्निम् । अ॒ग्निम् पु॑री॒ष्य᳚म् । पु॒री॒ष्य॑मङ्गिर॒स्वत् । अ॒ङ्गि॒र॒स्वद् भ॑रिष्यामः । भ॒रि॒ष्या॒मो॒ऽग्निम् । अ॒ग्निम् पु॑री॒ष्य᳚म् । पु॒री॒ष्य॑मङ्गिर॒स्वत् । अ॒ङ्गि॒र॒स्वद् भ॑रामः । भ॒रा॒म॒ इति॑ भरामः ॥ अन्व॒ग्निः । अ॒ग्निरु॒षसा᳚म् । उ॒षसा॒मग्र᳚म् । अग्र॑मख्यत् । अ॒ख्य॒दनु॑ । अन्वहा॑नि । अहा॑नि प्रथ॒मः । प्र॒थ॒मो जा॒तवे॑दाः । जा॒तवे॑दा॒ इति॑ जा॒त - वे॒दाः॒ ॥ अनु॒ सूर्य॑स्य । सूर्य॑स्य पुरु॒त्रा \newline

\textbf{Jatai Paata} \newline

1. एही॒ह्येहि॑ । \newline
2. इ॒ह्य॒व॒क्राम॑न् नव॒क्राम॑न् निही ह्यव॒क्रामन्न्॑ । \newline
3. अ॒व॒क्राम॒न् नश॑स्ती॒ रश॑स्ती रव॒क्राम॑न् नव॒क्राम॒न् नश॑स्तीः । \newline
4. अ॒व॒क्राम॒न्नित्य॑व - क्रामन्न्॑ । \newline
5. अश॑स्ती रु॒द्रस्य॑ रु॒द्रस्या श॑स्ती॒ रश॑स्ती रु॒द्रस्य॑ । \newline
6. रु॒द्रस्य॒ गाण॑पत्या॒द् गाण॑पत्याद् रु॒द्रस्य॑ रु॒द्रस्य॒ गाण॑पत्यात् । \newline
7. गाण॑पत्यान् मयो॒भूर् म॑यो॒भूर् गाण॑पत्या॒द् गाण॑पत्यान् मयो॒भूः । \newline
8. गाण॑पत्या॒दिति॒ गाण॑ - प॒त्या॒त् । \newline
9. म॒यो॒भूरा म॑यो॒भूर् म॑यो॒भूरा । \newline
10. म॒यो॒भूरिति॑ मयः - भूः । \newline
11. एही॒ह्येहि॑ । \newline
12. इ॒हीती॑हि । \newline
13. उ॒र्व॑न्तरि॑क्ष म॒न्तरि॑क्ष मु॒रू᳚(1॒)र्व॑न्तरि॑क्षम् । \newline
14. अ॒न्तरि॑क्ष॒ मन्वन् व॒न्तरि॑क्ष म॒न्तरि॑क्ष॒ मनु॑ । \newline
15. अन्वि॑ही॒ ह्यन् वन् वि॑हि । \newline
16. इ॒हि॒ स्व॒स्तिग॑व्यूतिः स्व॒स्तिग॑व्यूति रिहीहि स्व॒स्तिग॑व्यूतिः । \newline
17. स्व॒स्तिग॑व्यूति॒ रभ॑या॒ न्यभ॑यानि स्व॒स्तिग॑व्यूतिः स्व॒स्तिग॑व्यूति॒ रभ॑यानि । \newline
18. स्व॒स्तिग॑व्यूति॒रिति॑ स्व॒स्ति - ग॒व्यू॒तिः॒ । \newline
19. अभ॑यानि कृ॒ण्वन् कृ॒ण्वन् नभ॑या॒ न्यभ॑यानि कृ॒ण्वन्न् । \newline
20. कृ॒ण्वन्निति॑ कृ॒ण्वन्न् । \newline
21. पू॒ष्णा स॒युजा॑ स॒युजा॑ पू॒ष्णा पू॒ष्णा स॒युजा᳚ । \newline
22. स॒युजा॑ स॒ह स॒ह स॒युजा॑ स॒युजा॑ स॒ह । \newline
23. स॒युजेति॑ स - युजा᳚ । \newline
24. स॒हेति॑ स॒ह । \newline
25. पृ॒थि॒व्याः स॒धस्था᳚थ् स॒धस्था᳚त् पृथि॒व्याः पृ॑थि॒व्याः स॒धस्था᳚त् । \newline
26. स॒धस्था॑ द॒ग्नि म॒ग्निꣳ स॒धस्था᳚थ् स॒धस्था॑ द॒ग्निम् । \newline
27. स॒धस्था॒दिति॑ स॒ध - स्था॒त् । \newline
28. अ॒ग्निम् पु॑री॒ष्य॑म् पुरी॒ष्य॑ म॒ग्नि म॒ग्निम् पु॑री॒ष्य᳚म् । \newline
29. पु॒री॒ष्य॑ मङ्गिर॒स्व द॑ङ्गिर॒स्वत् पु॑री॒ष्य॑म् पुरी॒ष्य॑ मङ्गिर॒स्वत् । \newline
30. अ॒ङ्गि॒ र॒स्व दच्छा च्छा᳚ङ्गिर॒स्व द॑ङ्गिर॒स्व दच्छ॑ । \newline
31. अच्छे॑ ही॒ह्य च्छाच्छे॑हि । \newline
32. इ॒ह्य॒ग्नि म॒ग्नि मि॑ही ह्य॒ग्निम् । \newline
33. अ॒ग्निम् पु॑री॒ष्य॑म् पुरी॒ष्य॑ म॒ग्नि म॒ग्निम् पु॑री॒ष्य᳚म् । \newline
34. पु॒री॒ष्य॑ मङ्गिर॒स्व द॑ङ्गिर॒स्वत् पु॑री॒ष्य॑म् पुरी॒ष्य॑ मङ्गिर॒स्वत् । \newline
35. अ॒ङ्गि॒र॒स्व दच्छा च्छा᳚ङ्गिर॒स्व द॑ङ्गिर॒स्व दच्छ॑ । \newline
36. अच्छे॑ म इमो॒ अच्छाच्छे॑ मः । \newline
37. इ॒मो॒ ऽग्नि म॒ग्नि मि॑म इमो॒ ऽग्निम् । \newline
38. अ॒ग्निम् पु॑री॒ष्य॑म् पुरी॒ष्य॑ म॒ग्नि म॒ग्निम् पु॑री॒ष्य᳚म् । \newline
39. पु॒री॒ष्य॑ मङ्गिर॒स्व द॑ङ्गिर॒स्वत् पु॑री॒ष्य॑म् पुरी॒ष्य॑ मङ्गिर॒स्वत् । \newline
40. अ॒ङ्गि॒र॒स्वद् भ॑रिष्यामो भरिष्यामो ऽङ्गिर॒स्व द॑ङ्गिर॒स्वद् भ॑रिष्यामः । \newline
41. भ॒रि॒ष्या॒मो॒ ऽग्नि म॒ग्निम् भ॑रिष्यामो भरिष्यामो॒ ऽग्निम् । \newline
42. अ॒ग्निम् पु॑री॒ष्य॑म् पुरी॒ष्य॑ म॒ग्नि म॒ग्निम् पु॑री॒ष्य᳚म् । \newline
43. पु॒री॒ष्य॑ मङ्गिर॒स्व द॑ङ्गिर॒स्वत् पु॑री॒ष्य॑म् पुरी॒ष्य॑ मङ्गिर॒स्वत् । \newline
44. अ॒ङ्गि॒र॒स्वद् भ॑रामो भरामो ऽङ्गिर॒स्व द॑ङ्गिर॒स्वद् भ॑रामः । \newline
45. भ॒रा॒म॒ इति॑ भरामः । \newline
46. अन्व॒ग्नि र॒ग्नि रन्वन्व॒ग्निः । \newline
47. अ॒ग्नि रु॒षसा॑ मु॒षसा॑ म॒ग्नि र॒ग्नि रु॒षसा᳚म् । \newline
48. उ॒षसा॒ मग्र॒ मग्र॑ मु॒षसा॑ मु॒षसा॒ मग्र᳚म् । \newline
49. अग्र॑ मख्य दख्य॒ दग्र॒ मग्र॑ मख्यत् । \newline
50. अ॒ख्य॒ दन् वन् व॑ख्य दख्य॒ दनु॑ । \newline
51. अन्वहा॒न् यहा॒न्यन्वन् वहा॑नि । \newline
52. अहा॑नि प्रथ॒मः प्र॑थ॒मो ऽहा॒ न्यहा॑नि प्रथ॒मः । \newline
53. प्र॒थ॒मो जा॒तवे॑दा जा॒तवे॑दाः प्रथ॒मः प्र॑थ॒मो जा॒तवे॑दाः । \newline
54. जा॒तवे॑दा॒ इति॑ जा॒त - वे॒दाः॒ । \newline
55. अनु॒ सूर्य॑स्य॒ सूर्य॒स्या न्वनु॒ सूर्य॑स्य । \newline
56. सूर्य॑स्य पुरु॒त्रा पु॑रु॒त्रा सूर्य॑स्य॒ सूर्य॑स्य पुरु॒त्रा । \newline

\textbf{Ghana Paata } \newline

1. एही॒ह्ये ह्य॑व॒क्राम॑न् नव॒क्राम॑न् नि॒ह्ये ह्य॑व॒क्रामन्न्॑ । \newline
2. इ॒ह्य॒व॒क्राम॑न् नव॒क्राम॑न् निही ह्यव॒क्राम॒न् नश॑स्ती॒ रश॑स्ती रव॒क्राम॑न् निही ह्यव॒क्राम॒न् नश॑स्तीः । \newline
3. अ॒व॒क्राम॒न् नश॑स्ती॒ रश॑स्ती रव॒क्राम॑न् नव॒क्राम॒न् नश॑स्ती रु॒द्रस्य॑ रु॒द्रस्या श॑स्ती रव॒क्राम॑न् नव॒क्राम॒न् नश॑स्ती रु॒द्रस्य॑ । \newline
4. अ॒व॒क्राम॒न्नित्य॑व - क्रामन्न्॑ । \newline
5. अश॑स्ती रु॒द्रस्य॑ रु॒द्रस्या श॑स्ती॒ रश॑स्ती रु॒द्रस्य॒ गाण॑पत्या॒द् गाण॑पत्याद् रु॒द्रस्या श॑स्ती॒ रश॑स्ती रु॒द्रस्य॒ गाण॑पत्यात् । \newline
6. रु॒द्रस्य॒ गाण॑पत्या॒द् गाण॑पत्याद् रु॒द्रस्य॑ रु॒द्रस्य॒ गाण॑पत्यान् मयो॒भूर् म॑यो॒भूर् गाण॑पत्याद् रु॒द्रस्य॑ रु॒द्रस्य॒ गाण॑पत्यान् मयो॒भूः । \newline
7. गाण॑पत्यान् मयो॒भूर् म॑यो॒भूर् गाण॑पत्या॒द् गाण॑पत्यान् मयो॒भूरा म॑यो॒भूर् गाण॑पत्या॒द् गाण॑पत्यान् मयो॒भूरा । \newline
8. गाण॑पत्या॒दिति॒ गाण॑ - प॒त्या॒त् । \newline
9. म॒यो॒भूरा म॑यो॒भूर् म॑यो॒भू रेही॒ह्या म॑यो॒भूर् म॑यो॒भू रेहि॑ । \newline
10. म॒यो॒भूरिति॑ मयः - भूः । \newline
11. एही॒ ह्येहि॑ । \newline
12. इ॒हीती॑हि । \newline
13. उ॒र्व॑न्तरि॑क्ष म॒न्तरि॑क्ष मु॒रू᳚(1॒)र्व॑न्तरि॑क्ष॒ मन्वन् व॒न्तरि॑क्ष मु॒रू᳚(1॒)र्व॑न्तरि॑क्ष॒ मनु॑ । \newline
14. अ॒न्तरि॑क्ष॒ मन्वन् व॒न्तरि॑क्ष म॒न्तरि॑क्ष॒ मन्वि॑ही॒ ह्यन्व॒न्तरि॑क्ष म॒न्तरि॑क्ष॒ मन्वि॑हि । \newline
15. अन्वि॑ही॒ ह्यन्वन्वि॑हि स्व॒स्तिग॑व्यूतिः स्व॒स्तिग॑व्यूति रि॒ह्यन् वन्वि॑हि स्व॒स्तिग॑व्यूतिः । \newline
16. इ॒हि॒ स्व॒स्तिग॑व्यूतिः स्व॒स्तिग॑व्यू तिरिहीहि स्व॒स्तिग॑व्यूति॒ रभ॑या॒ न्यभ॑यानि स्व॒स्तिग॑व्यू तिरिहीहि स्व॒स्तिग॑व्यूति॒ रभ॑यानि । \newline
17. स्व॒स्तिग॑व्यूति॒ रभ॑या॒ न्यभ॑यानि स्व॒स्तिग॑व्यूतिः स्व॒स्तिग॑व्यूति॒ रभ॑यानि कृ॒ण्वन् कृ॒ण्वन् नभ॑यानि स्व॒स्तिग॑व्यूतिः स्व॒स्तिग॑व्यूति॒ रभ॑यानि कृ॒ण्वन्न् । \newline
18. स्व॒स्तिग॑व्यूति॒रिति॑ स्व॒स्ति - ग॒व्यू॒तिः॒ । \newline
19. अभ॑यानि कृ॒ण्वन् कृ॒ण्वन् नभ॑या॒ न्यभ॑यानि कृ॒ण्वन्न् । \newline
20. कृ॒ण्वन्निति॑ कृ॒ण्वन्न् । \newline
21. पू॒ष्णा स॒युजा॑ स॒युजा॑ पू॒ष्णा पू॒ष्णा स॒युजा॑ स॒ह स॒ह स॒युजा॑ पू॒ष्णा पू॒ष्णा स॒युजा॑ स॒ह । \newline
22. स॒युजा॑ स॒ह स॒ह स॒युजा॑ स॒युजा॑ स॒ह । \newline
23. स॒युजेति॑ स - युजा᳚ । \newline
24. स॒हेति॑ स॒ह । \newline
25. पृ॒थि॒व्याः स॒धस्था᳚थ् स॒धस्था᳚त् पृथि॒व्याः पृ॑थि॒व्याः स॒धस्था॑ द॒ग्नि म॒ग्निꣳ स॒धस्था᳚त् पृथि॒व्याः पृ॑थि॒व्याः स॒धस्था॑ द॒ग्निम् । \newline
26. स॒धस्था॑ द॒ग्नि म॒ग्निꣳ स॒धस्था᳚थ् स॒धस्था॑ द॒ग्निम् पु॑री॒ष्य॑म् पुरी॒ष्य॑ म॒ग्निꣳ स॒धस्था᳚थ् स॒धस्था॑ द॒ग्निम् पु॑री॒ष्य᳚म् । \newline
27. स॒धस्था॒दिति॑ स॒ध - स्था॒त् । \newline
28. अ॒ग्निम् पु॑री॒ष्य॑म् पुरी॒ष्य॑ म॒ग्नि म॒ग्निम् पु॑री॒ष्य॑ मङ्गिर॒स्व द॑ङ्गिर॒स्वत् पु॑री॒ष्य॑ म॒ग्नि म॒ग्निम् पु॑री॒ष्य॑ मङ्गिर॒स्वत् । \newline
29. पु॒री॒ष्य॑ मङ्गिर॒स्व द॑ङ्गिर॒स्वत् पु॑री॒ष्य॑म् पुरी॒ष्य॑ मङ्गिर॒स्व दच्छा च्छा᳚ङ्गिर॒स्वत् पु॑री॒ष्य॑म् पुरी॒ष्य॑ मङ्गिर॒स्व दच्छ॑ । \newline
30. अ॒ङ्गि॒र॒स्व दच्छा च्छा᳚ङ्गिर॒स्व द॑ङ्गिर॒स्व दच्छे॑ ही॒ ह्यच्छा᳚ङ्गिर॒स्व द॑ङ्गिर॒स्व दच्छे॑हि । \newline
31. अच्छे॑ ही॒ह्यच्छाच्छे᳚ ह्य॒ग्नि म॒ग्नि मि॒ह्यच्छाच्छे᳚ ह्य॒ग्निम् । \newline
32. इ॒ह्य॒ग्नि म॒ग्नि मि॑ही ह्य॒ग्निम् पु॑री॒ष्य॑म् पुरी॒ष्य॑ म॒ग्नि मि॑ही ह्य॒ग्निम् पु॑री॒ष्य᳚म् । \newline
33. अ॒ग्निम् पु॑री॒ष्य॑म् पुरी॒ष्य॑ म॒ग्नि म॒ग्निम् पु॑री॒ष्य॑ मङ्गिर॒स्व द॑ङ्गिर॒स्वत् पु॑री॒ष्य॑ म॒ग्नि म॒ग्निम् पु॑री॒ष्य॑ मङ्गिर॒स्वत् । \newline
34. पु॒री॒ष्य॑ मङ्गिर॒स्व द॑ङ्गिर॒स्वत् पु॑री॒ष्य॑म् पुरी॒ष्य॑ मङ्गिर॒स्व दच्छा च्छा᳚ङ्गिर॒स्वत् पु॑री॒ष्य॑म् पुरी॒ष्य॑ मङ्गिर॒स्व दच्छ॑ । \newline
35. अ॒ङ्गि॒र॒स्व दच्छाच्छा᳚ ङ्गिर॒स्व द॑ङ्गिर॒स्व दच्छे॑ म इमो॒ अच्छा᳚ङ्गिर॒स्व द॑ङ्गिर॒स्व दच्छे॑ मः । \newline
36. अच्छे॑ म इमो॒ अच्छाच्छे॑ मो॒ ऽग्नि म॒ग्नि मि॑मो॒ अच्छाच्छे॑ मो॒ ऽग्निम् । \newline
37. इ॒मो॒ ऽग्नि म॒ग्नि मि॑म इमो॒ ऽग्निम् पु॑री॒ष्य॑म् पुरी॒ष्य॑ म॒ग्नि मि॑म इमो॒ ऽग्निम् पु॑री॒ष्य᳚म् । \newline
38. अ॒ग्निम् पु॑री॒ष्य॑म् पुरी॒ष्य॑ म॒ग्नि म॒ग्निम् पु॑री॒ष्य॑ मङ्गिर॒स्व द॑ङ्गिर॒स्वत् पु॑री॒ष्य॑ म॒ग्नि म॒ग्निम् पु॑री॒ष्य॑ मङ्गिर॒स्वत् । \newline
39. पु॒री॒ष्य॑ मङ्गिर॒स्व द॑ङ्गिर॒स्वत् पु॑री॒ष्य॑म् पुरी॒ष्य॑ मङ्गिर॒स्वद् भ॑रिष्यामो भरिष्यामो ऽङ्गिर॒स्वत् पु॑री॒ष्य॑म् पुरी॒ष्य॑ मङ्गिर॒स्वद् भ॑रिष्यामः । \newline
40. अ॒ङ्गि॒र॒स्वद् भ॑रिष्यामो भरिष्यामो ऽङ्गिर॒स्व द॑ङ्गिर॒स्वद् भ॑रिष्यामो॒ ऽग्नि म॒ग्निम् भ॑रिष्यामो ऽङ्गिर॒स्व द॑ङ्गिर॒स्वद् भ॑रिष्यामो॒ ऽग्निम् । \newline
41. भ॒रि॒ष्या॒मो॒ ऽग्नि म॒ग्निम् भ॑रिष्यामो भरिष्यामो॒ ऽग्निम् पु॑री॒ष्य॑म् पुरी॒ष्य॑ म॒ग्निम् भ॑रिष्यामो भरिष्यामो॒ ऽग्निम् पु॑री॒ष्य᳚म् । \newline
42. अ॒ग्निम् पु॑री॒ष्य॑म् पुरी॒ष्य॑ म॒ग्नि म॒ग्निम् पु॑री॒ष्य॑ मङ्गिर॒स्व द॑ङ्गिर॒स्वत् पु॑री॒ष्य॑ म॒ग्नि म॒ग्निम् पु॑री॒ष्य॑ मङ्गिर॒स्वत् । \newline
43. पु॒री॒ष्य॑ मङ्गिर॒स्व द॑ङ्गिर॒स्वत् पु॑री॒ष्य॑म् पुरी॒ष्य॑ मङ्गिर॒स्वद् भ॑रामो भरामो ऽङ्गिर॒स्वत् पु॑री॒ष्य॑म् पुरी॒ष्य॑ मङ्गिर॒स्वद् भ॑रामः । \newline
44. अ॒ङ्गि॒र॒स्वद् भ॑रामो भरामो ऽङ्गिर॒स्व द॑ङ्गिर॒स्वद् भ॑रामः । \newline
45. भ॒रा॒म॒ इति॑ भरामः । \newline
46. अन्व॒ग्नि र॒ग्नि रन्व न्व॒ग्नि रु॒षसा॑ मु॒षसा॑ म॒ग्नि रन्व न्व॒ग्नि रु॒षसा᳚म् । \newline
47. अ॒ग्नि रु॒षसा॑ मु॒षसा॑ म॒ग्नि र॒ग्नि रु॒षसा॒ मग्र॒ मग्र॑ मु॒षसा॑ म॒ग्नि र॒ग्नि रु॒षसा॒ मग्र᳚म् । \newline
48. उ॒षसा॒ मग्र॒ मग्र॑ मु॒षसा॑ मु॒षसा॒ मग्र॑ मख्य दख्य॒ दग्र॑ मु॒षसा॑ मु॒षसा॒ मग्र॑ मख्यत् । \newline
49. अग्र॑ मख्य दख्य॒ दग्र॒ मग्र॑ मख्य॒ दन्वन् व॑ख्य॒ दग्र॒ मग्र॑ मख्य॒ दनु॑ । \newline
50. अ॒ख्य॒ दन्वन् व॑ख्य दख्य॒ दन्वहा॒ न्यहा॒न्यन् व॑ख्य दख्य॒ दन्वहा॑नि । \newline
51. अन्वहा॒ न्यहा॒ न्यन्व न्वहा॑नि प्रथ॒मः प्र॑थ॒मो ऽहा॒न्यन्व न्वहा॑नि प्रथ॒मः । \newline
52. अहा॑नि प्रथ॒मः प्र॑थ॒मो ऽहा॒न्यहा॑नि प्रथ॒मो जा॒तवे॑दा जा॒तवे॑दाः प्रथ॒मो ऽहा॒ न्यहा॑नि प्रथ॒मो जा॒तवे॑दाः । \newline
53. प्र॒थ॒मो जा॒तवे॑दा जा॒तवे॑दाः प्रथ॒मः प्र॑थ॒मो जा॒तवे॑दाः । \newline
54. जा॒तवे॑दा॒ इति॑ जा॒त - वे॒दाः॒ । \newline
55. अनु॒ सूर्य॑स्य॒ सूर्य॒स्या न्वनु॒ सूर्य॑स्य पुरु॒त्रा पु॑रु॒त्रा सूर्य॒स्या न्वनु॒ सूर्य॑स्य पुरु॒त्रा । \newline
56. सूर्य॑स्य पुरु॒त्रा पु॑रु॒त्रा सूर्य॑स्य॒ सूर्य॑स्य पुरु॒त्रा च॑ च पुरु॒त्रा सूर्य॑स्य॒ सूर्य॑स्य पुरु॒त्रा च॑ । \newline
\pagebreak
\markright{ TS 4.1.2.3  \hfill https://www.vedavms.in \hfill}

\section{ TS 4.1.2.3 }

\textbf{TS 4.1.2.3 } \newline
\textbf{Samhita Paata} \newline

पुरु॒त्रा च॑ र॒श्मीननु॒ द्यावा॑पृथि॒वी आ त॑तान ॥ आ॒गत्य॑ वा॒ज्यद्ध्व॑नः॒ सर्वा॒ मृधो॒ विधू॑नुते । अ॒ग्निꣳ स॒धस्थे॑ मह॒ति चक्षु॑षा॒ नि चि॑कीषते ॥ आ॒क्रम्य॑ वाजिन् पृथि॒वीम॒ग्निमि॑च्छ रु॒चा त्वं । भूम्या॑ वृ॒त्वाय॑ नो ब्रूहि॒ यतः॒ खना॑म॒ तं ॅव॒यं ॥ द्यौस्ते॑ पृ॒ष्ठं पृ॑थि॒वी स॒धस्थ॑मा॒त्मा ऽन्तरि॑क्षꣳ समु॒द्रस्ते॒ योनिः॑ । वि॒ख्याय॒ चक्षु॑षा॒ त्वम॒भि ति॑ष्ठ- [  ] \newline

\textbf{Pada Paata} \newline

पु॒रु॒त्रेति॑ पुरु - त्रा । च॒ । र॒श्मीन् । अन्विति॑ । द्यावा॑पृथि॒वी इति॒ द्यावा᳚ - पृ॒थि॒वी । एति॑ । त॒ता॒न॒ ॥ आ॒गत्येत्या᳚ - गत्य॑ । वा॒जी । अद्ध्व॑नः । सर्वाः᳚ । मृधः॑ । वीति॑ । धू॒नु॒ते॒ ॥ अ॒ग्निम् । स॒धस्थ॒ इति॑ स॒ध - स्थे॒ । म॒ह॒ति । चक्षु॑षा । नीति॑ । चि॒की॒ष॒ते॒ ॥ आ॒क्रम्येत्या᳚ - क्रम्य॑ । वा॒जि॒न्न् । पृ॒थि॒वीम् । अ॒ग्निम् । इ॒च्छ॒ । रु॒चा । त्वम् ॥ भूम्याः᳚ । वृ॒त्वाय॑ । नः॒ । ब्रू॒हि॒ । यतः॑ । खना॑म । तम् । व॒यम् ॥ द्यौः । ते॒ । पृ॒ष्ठम् । पृ॒थि॒वी । स॒धस्थ॒मिति॑ स॒ध - स्थ॒म् । आ॒त्मा । अ॒न्तरि॑क्षम् । स॒मु॒द्रः । ते॒ । योनिः॑ ॥ वि॒ख्यायेति॑ वि - ख्याय॑ । चक्षु॑षा । त्वम् । अ॒भीति॑ । ति॒ष्ठ॒ ।  \newline


\textbf{Krama Paata} \newline

पु॒रु॒त्रा च॑ । पु॒रु॒त्रेति॑ पुरु - त्रा । च॒ र॒श्मीन् । र॒श्मीननु॑ । अनु॒ द्यावा॑पृथि॒वी । द्यावा॑पृथि॒वी आ । द्यावा॑पृथि॒वी इति॒ द्यावा᳚ - पृ॒थि॒वी । आ त॑तान । त॒ता॒नेति॑ ततान ॥ आ॒गत्य॑ वा॒जी । आ॒गत्येत्या᳚ - गत्य॑ । वा॒ज्यद्ध्व॑नः । अद्ध्व॑नः॒ सर्वाः᳚ । सर्वा॒ मृधः॑ । मृधो॒ वि । वि धू॑नुते । धू॒नु॒त॒ इति॑ धूनुते ॥ 
अ॒ग्निꣳ स॒धस्थे᳚ । स॒धस्थे॑ मह॒ति । स॒धस्थ॒ इति॑ स॒ध - स्थे॒ । म॒ह॒ति चक्षु॑षा । चक्षु॑षा॒ नि । नि चि॑कीषते । चि॒की॒ष॒त॒ इति॑ चिकीषते ॥ आ॒क्रम्य॑ वाजिन्न् । आ॒क्रम्येत्या᳚ - क्रम्य॑ । वा॒जि॒न् पृ॒थि॒वीम् । पृ॒थि॒वीम॒ग्निम् । अ॒ग्निमि॑च्छ । इ॒च्छ॒ रु॒चा । रु॒चा त्वम् । त्वमिति॒ त्वम् ॥ भूम्या॑ वृ॒त्वाय॑ । वृ॒त्वाय॑ नः । नो॒ ब्रू॒हि॒ । ब्रू॒हि॒ यतः॑ । यतः॒ खना॑म । खना॑म॒ तम् । तं ॅव॒यम् । व॒यमिति॑ व॒यम् ॥ द्यौस्ते᳚ । ते॒ पृ॒ष्ठम् । पृ॒ष्ठम् पृ॑थि॒वी । पृ॒थि॒वी स॒धस्थ᳚म् । स॒धस्थ॑मा॒त्मा । स॒धस्थ॒मिति॑ स॒ध - स्थ॒म् । आ॒त्माऽन्तरि॑क्षम् । अ॒न्तरि॑क्षꣳ समु॒द्रः । स॒मु॒द्रस्ते᳚ । ते॒ योनिः॑ । योनि॒रिति॒ योनिः॑ ॥ वि॒ख्याय॒ चक्षु॑षा । वि॒ख्यायेति॑ वि - ख्याय॑ । चक्षु॑षा॒ त्वम् । त्वम॒भि । अ॒भि ति॑ष्ठ । ति॒ष्ठ॒ पृ॒त॒न्य॒तः \newline

\textbf{Jatai Paata} \newline

1. पु॒रु॒त्रा च॑ च पुरु॒त्रा पु॑रु॒त्रा च॑ । \newline
2. पु॒रु॒त्रेति॑ पुरु - त्रा । \newline
3. च॒ र॒श्मीन् र॒श्मीꣳ श्च॑च र॒श्मीन् । \newline
4. र॒श्मी नन्वनु॑ र॒श्मीन् र॒श्मी ननु॑ । \newline
5. अनु॒ द्यावा॑पृथि॒वी द्यावा॑पृथि॒वी अन्वनु॒ द्यावा॑पृथि॒वी । \newline
6. द्यावा॑पृथि॒वी आ द्यावा॑पृथि॒वी द्यावा॑पृथि॒वी आ । \newline
7. द्यावा॑पृथि॒वी इति॒ द्यावा᳚ - पृ॒थि॒वी । \newline
8. आ त॑तान तता॒ना त॑तान । \newline
9. त॒ता॒नेति॑ ततान । \newline
10. आ॒गत्य॑ वा॒जी वा॒ज्या॑गत्या॒ गत्य॑ वा॒जी । \newline
11. आ॒गत्येत्या᳚ - गत्य॑ । \newline
12. वा॒ज्यद्ध्व॑नो॒ अद्ध्व॑नो वा॒जी वा॒ज्यद्ध्व॑नः । \newline
13. अद्ध्व॑नः॒ सर्वाः॒ सर्वा॒ अद्ध्व॑नो॒ अद्ध्व॑नः॒ सर्वाः᳚ । \newline
14. सर्वा॒ मृधो॒ मृधः॒ सर्वाः॒ सर्वा॒ मृधः॑ । \newline
15. मृधो॒ वि वि मृधो॒ मृधो॒ वि । \newline
16. वि धू॑नुते धूनुते॒ वि वि धू॑नुते । \newline
17. धू॒नु॒त॒ इति॑ धूनुते । \newline
18. अ॒ग्निꣳ स॒धस्थे॑ स॒धस्थे॒ ऽग्नि म॒ग्निꣳ स॒धस्थे᳚ । \newline
19. स॒धस्थे॑ मह॒ति म॑ह॒ति स॒धस्थे॑ स॒धस्थे॑ मह॒ति । \newline
20. स॒धस्थ॒ इति॑ स॒ध - स्थे॒ । \newline
21. म॒ह॒ति चक्षु॑षा॒ चक्षु॑षा मह॒ति म॑ह॒ति चक्षु॑षा । \newline
22. चक्षु॑षा॒ नि नि चक्षु॑षा॒ चक्षु॑षा॒ नि । \newline
23. नि चि॑कीषते चिकीषते॒ नि नि चि॑कीषते । \newline
24. चि॒की॒ष॒त॒ इति॑ चिकीषते । \newline
25. आ॒क्रम्य॑ वाजिन्. वाजिन् ना॒क्रम्या॒ क्रम्य॑ वाजिन्न् । \newline
26. आ॒क्रम्येत्या᳚ - क्रम्य॑ । \newline
27. वा॒जि॒न् पृ॒थि॒वीम् पृ॑थि॒वीं ॅवा॑जिन्. वाजिन् पृथि॒वीम् । \newline
28. पृ॒थि॒वी म॒ग्नि म॒ग्निम् पृ॑थि॒वीम् पृ॑थि॒वी म॒ग्निम् । \newline
29. अ॒ग्नि मि॑च्छे च्छा॒ग्नि म॒ग्नि मि॑च्छ । \newline
30. इ॒च्छ॒ रु॒चा रु॒चेच्छे᳚ च्छ रु॒चा । \newline
31. रु॒चा त्वम् त्वꣳ रु॒चा रु॒चा त्वम् । \newline
32. त्वमिति॒ त्वम् । \newline
33. भूम्या॑ वृ॒त्वाय॑ वृ॒त्वाय॒ भूम्या॒ भूम्या॑ वृ॒त्वाय॑ । \newline
34. वृ॒त्वाय॑ नो नो वृ॒त्वाय॑ वृ॒त्वाय॑ नः । \newline
35. नो॒ ब्रू॒हि॒ ब्रू॒हि॒ नो॒ नो॒ ब्रू॒हि॒ । \newline
36. ब्रू॒हि॒ यतो॒ यतो᳚ ब्रूहि ब्रूहि॒ यतः॑ । \newline
37. यतः॒ खना॑म॒ खना॑म॒ यतो॒ यतः॒ खना॑म । \newline
38. खना॑म॒ तम् तम् खना॑म॒ खना॑म॒ तम् । \newline
39. तं ॅव॒यं ॅव॒यम् तम् तं ॅव॒यम् । \newline
40. व॒यमिति॑ व॒यम् । \newline
41. द्यौ स्ते॑ ते॒ द्यौर् द्यौ स्ते᳚ । \newline
42. ते॒ पृ॒ष्ठम् पृ॒ष्ठम् ते॑ ते पृ॒ष्ठम् । \newline
43. पृ॒ष्ठम् पृ॑थि॒वी पृ॑थि॒वी पृ॒ष्ठम् पृ॒ष्ठम् पृ॑थि॒वी । \newline
44. पृ॒थि॒वी स॒धस्थꣳ॑ स॒धस्थ॑म् पृथि॒वी पृ॑थि॒वी स॒धस्थ᳚म् । \newline
45. स॒धस्थ॑ मा॒त्मा ऽऽत्मा स॒धस्थꣳ॑ स॒धस्थ॑ मा॒त्मा । \newline
46. स॒धस्थ॒मिति॑ स॒ध - स्थ॒म् । \newline
47. आ॒त्मा ऽन्तरि॑क्ष म॒न्तरि॑क्ष मा॒त्मा ऽऽत्मा ऽन्तरि॑क्षम् । \newline
48. अ॒न्तरि॑क्षꣳ समु॒द्रः स॑मु॒द्रो अ॒न्तरि॑क्ष म॒न्तरि॑क्षꣳ समु॒द्रः । \newline
49. स॒मु॒द्र स्ते॑ ते समु॒द्रः स॑मु॒द्र स्ते᳚ । \newline
50. ते॒ योनि॒र् योनि॑ स्ते ते॒ योनिः॑ । \newline
51. योनि॒रिति॒ योनिः॑ । \newline
52. वि॒ख्याय॒ चक्षु॑षा॒ चक्षु॑षा वि॒ख्याय॑ वि॒ख्याय॒ चक्षु॑षा । \newline
53. वि॒ख्यायेति॑ वि - ख्याय॑ । \newline
54. चक्षु॑षा॒ त्वम् त्वम् चक्षु॑षा॒ चक्षु॑षा॒ त्वम् । \newline
55. त्व म॒भ्य॑भि त्वम् त्व म॒भि । \newline
56. अ॒भि ति॑ष्ठ तिष्ठा॒भ्य॑भि ति॑ष्ठ । \newline
57. ति॒ष्ठ॒ पृ॒त॒न्य॒तः पृ॑तन्य॒त स्ति॑ष्ठ तिष्ठ पृतन्य॒तः । \newline

\textbf{Ghana Paata } \newline

1. पु॒रु॒त्रा च॑ च पुरु॒त्रा पु॑रु॒त्रा च॑ र॒श्मीन् र॒श्मीꣳश्च॑ पुरु॒त्रा पु॑रु॒त्रा च॑ र॒श्मीन् । \newline
2. पु॒रु॒त्रेति॑ पुरु - त्रा । \newline
3. च॒ र॒श्मीन् र॒श्मीꣳश्च॑ च र॒श्मी नन्वनु॑ र॒श्मीꣳश्च॑ च र॒श्मी ननु॑ । \newline
4. र॒श्मी नन्वनु॑ र॒श्मीन् र॒श्मी ननु॒ द्यावा॑पृथि॒वी द्यावा॑पृथि॒वी अनु॑ र॒श्मीन् र॒श्मी ननु॒ द्यावा॑पृथि॒वी । \newline
5. अनु॒ द्यावा॑पृथि॒वी द्यावा॑पृथि॒वी अन्वनु॒ द्यावा॑पृथि॒वी आ द्यावा॑पृथि॒वी अन्वनु॒ द्यावा॑पृथि॒वी आ । \newline
6. द्यावा॑पृथि॒वी आ द्यावा॑पृथि॒वी द्यावा॑पृथि॒वी आ त॑तान तता॒ना द्यावा॑पृथि॒वी द्यावा॑पृथि॒वी आ त॑तान । \newline
7. द्यावा॑पृथि॒वी इति॒ द्यावा᳚ - पृ॒थि॒वी । \newline
8. आ त॑तान तता॒ना त॑तान । \newline
9. त॒ता॒नेति॑ ततान । \newline
10. आ॒गत्य॑ वा॒जी वा॒ज्या॑ गत्या॒ गत्य॑ वा॒ज्यद्ध्व॑नो॒ अद्ध्व॑नो वा॒ज्या॑ग त्या॒गत्य॑ वा॒ज्यद्ध्व॑नः । \newline
11. आ॒गत्येत्या᳚ - गत्य॑ । \newline
12. वा॒ज्यद्ध्व॑नो॒ अद्ध्व॑नो वा॒जी वा॒ज्यद्ध्व॑नः॒ सर्वाः॒ सर्वा॒ अद्ध्व॑नो वा॒जी वा॒ज्यद्ध्व॑नः॒ सर्वाः᳚ । \newline
13. अद्ध्व॑नः॒ सर्वाः॒ सर्वा॒ अद्ध्व॑नो॒ अद्ध्व॑नः॒ सर्वा॒ मृधो॒ मृधः॒ सर्वा॒ अद्ध्व॑नो॒ अद्ध्व॑नः॒ सर्वा॒ मृधः॑ । \newline
14. सर्वा॒ मृधो॒ मृधः॒ सर्वाः॒ सर्वा॒ मृधो॒ वि वि मृधः॒ सर्वाः॒ सर्वा॒ मृधो॒ वि । \newline
15. मृधो॒ वि वि मृधो॒ मृधो॒ वि धू॑नुते धूनुते॒ वि मृधो॒ मृधो॒ वि धू॑नुते । \newline
16. वि धू॑नुते धूनुते॒ वि वि धू॑नुते । \newline
17. धू॒नु॒त॒ इति॑ धूनुते । \newline
18. अ॒ग्निꣳ स॒धस्थे॑ स॒धस्थे॒ ऽग्नि म॒ग्निꣳ स॒धस्थे॑ मह॒ति म॑ह॒ति स॒धस्थे॒ ऽग्नि म॒ग्निꣳ स॒धस्थे॑ मह॒ति । \newline
19. स॒धस्थे॑ मह॒ति म॑ह॒ति स॒धस्थे॑ स॒धस्थे॑ मह॒ति चक्षु॑षा॒ चक्षु॑षा मह॒ति स॒धस्थे॑ स॒धस्थे॑ मह॒ति चक्षु॑षा । \newline
20. स॒धस्थ॒ इति॑ स॒ध - स्थे॒ । \newline
21. म॒ह॒ति चक्षु॑षा॒ चक्षु॑षा मह॒ति म॑ह॒ति चक्षु॑षा॒ नि नि चक्षु॑षा मह॒ति म॑ह॒ति चक्षु॑षा॒ नि । \newline
22. चक्षु॑षा॒ नि नि चक्षु॑षा॒ चक्षु॑षा॒ नि चि॑कीषते चिकीषते॒ नि चक्षु॑षा॒ चक्षु॑षा॒ नि चि॑कीषते । \newline
23. नि चि॑कीषते चिकीषते॒ नि नि चि॑कीषते । \newline
24. चि॒की॒ष॒त॒ इति॑ चिकीषते । \newline
25. आ॒क्रम्य॑ वाजिन्. वाजिन् ना॒क्रम्या॒ क्रम्य॑ वाजिन् पृथि॒वीम् पृ॑थि॒वीं ॅवा॑जिन् ना॒क्रम्या॒ क्रम्य॑ वाजिन् पृथि॒वीम् । \newline
26. आ॒क्रम्येत्या᳚ - क्रम्य॑ । \newline
27. वा॒जि॒न् पृ॒थि॒वीम् पृ॑थि॒वीं ॅवा॑जिन्. वाजिन् पृथि॒वी म॒ग्नि म॒ग्निम् पृ॑थि॒वीं ॅवा॑जिन्. वाजिन् पृथि॒वी म॒ग्निम् । \newline
28. पृ॒थि॒वी म॒ग्नि म॒ग्निम् पृ॑थि॒वीम् पृ॑थि॒वी म॒ग्नि मि॑च्छे च्छा॒ग्निम् पृ॑थि॒वीम् पृ॑थि॒वी म॒ग्नि मि॑च्छ । \newline
29. अ॒ग्नि मि॑च्छे च्छा॒ग्नि म॒ग्नि मि॑च्छ रु॒चा रु॒चे च्छा॒ग्नि म॒ग्नि मि॑च्छ रु॒चा । \newline
30. इ॒च्छ॒ रु॒चा रु॒चेच्छे᳚ च्छ रु॒चा त्वम् त्वꣳ रु॒चेच्छे᳚ च्छ रु॒चा त्वम् । \newline
31. रु॒चा त्वम् त्वꣳ रु॒चा रु॒चा त्वम् । \newline
32. त्वमिति॒ त्वम् । \newline
33. भूम्या॑ वृ॒त्वाय॑ वृ॒त्वाय॒ भूम्या॒ भूम्या॑ वृ॒त्वाय॑ नो नो वृ॒त्वाय॒ भूम्या॒ भूम्या॑ वृ॒त्वाय॑ नः । \newline
34. वृ॒त्वाय॑ नो नो वृ॒त्वाय॑ वृ॒त्वाय॑ नो ब्रूहि ब्रूहि नो वृ॒त्वाय॑ वृ॒त्वाय॑ नो ब्रूहि । \newline
35. नो॒ ब्रू॒हि॒ ब्रू॒हि॒ नो॒ नो॒ ब्रू॒हि॒ यतो॒ यतो᳚ ब्रूहि नो नो ब्रूहि॒ यतः॑ । \newline
36. ब्रू॒हि॒ यतो॒ यतो᳚ ब्रूहि ब्रूहि॒ यतः॒ खना॑म॒ खना॑म॒ यतो᳚ ब्रूहि ब्रूहि॒ यतः॒ खना॑म । \newline
37. यतः॒ खना॑म॒ खना॑म॒ यतो॒ यतः॒ खना॑म॒ तम् तम् खना॑म॒ यतो॒ यतः॒ खना॑म॒ तम् । \newline
38. खना॑म॒ तम् तम् खना॑म॒ खना॑म॒ तं ॅव॒यं ॅव॒यम् तम् खना॑म॒ खना॑म॒ तं ॅव॒यम् । \newline
39. तं ॅव॒यं ॅव॒यम् तम् तं ॅव॒यम् । \newline
40. व॒यमिति॑ व॒यम् । \newline
41. द्यौ स्ते॑ ते॒ द्यौर् द्यौ स्ते॑ पृ॒ष्ठम् पृ॒ष्ठम् ते॒ द्यौर् द्यौ स्ते॑ पृ॒ष्ठम् । \newline
42. ते॒ पृ॒ष्ठम् पृ॒ष्ठम् ते॑ ते पृ॒ष्ठम् पृ॑थि॒वी पृ॑थि॒वी पृ॒ष्ठम् ते॑ ते पृ॒ष्ठम् पृ॑थि॒वी । \newline
43. पृ॒ष्ठम् पृ॑थि॒वी पृ॑थि॒वी पृ॒ष्ठम् पृ॒ष्ठम् पृ॑थि॒वी स॒धस्थꣳ॑ स॒धस्थ॑म् पृथि॒वी पृ॒ष्ठम् पृ॒ष्ठम् पृ॑थि॒वी स॒धस्थ᳚म् । \newline
44. पृ॒थि॒वी स॒धस्थꣳ॑ स॒धस्थ॑म् पृथि॒वी पृ॑थि॒वी स॒धस्थ॑ मा॒त्मा ऽऽत्मा स॒धस्थ॑म् पृथि॒वी पृ॑थि॒वी स॒धस्थ॑ मा॒त्मा । \newline
45. स॒धस्थ॑ मा॒त्मा ऽऽत्मा स॒धस्थꣳ॑ स॒धस्थ॑ मा॒त्मा ऽन्तरि॑क्ष म॒न्तरि॑क्ष मा॒त्मा स॒धस्थꣳ॑ स॒धस्थ॑ मा॒त्मा ऽन्तरि॑क्षम् । \newline
46. स॒धस्थ॒मिति॑ स॒ध - स्थ॒म् । \newline
47. आ॒त्मा ऽन्तरि॑क्ष म॒न्तरि॑क्ष मा॒त्मा ऽऽत्मा ऽन्तरि॑क्षꣳ समु॒द्रः स॑मु॒द्रो अ॒न्तरि॑क्ष मा॒त्मा ऽऽत्मा ऽन्तरि॑क्षꣳ समु॒द्रः । \newline
48. अ॒न्तरि॑क्षꣳ समु॒द्रः स॑मु॒द्रो अ॒न्तरि॑क्ष म॒न्तरि॑क्षꣳ समु॒द्र स्ते॑ ते समु॒द्रो अ॒न्तरि॑क्ष म॒न्तरि॑क्षꣳ समु॒द्र स्ते᳚ । \newline
49. स॒मु॒द्र स्ते॑ ते समु॒द्रः स॑मु॒द्र स्ते॒ योनि॒र् योनि॑ स्ते समु॒द्रः स॑मु॒द्र स्ते॒ योनिः॑ । \newline
50. ते॒ योनि॒र् योनि॑ स्ते ते॒ योनिः॑ । \newline
51. योनि॒रिति॒ योनिः॑ । \newline
52. वि॒ख्याय॒ चक्षु॑षा॒ चक्षु॑षा वि॒ख्याय॑ वि॒ख्याय॒ चक्षु॑षा॒ त्वम् त्वम् चक्षु॑षा वि॒ख्याय॑ वि॒ख्याय॒ चक्षु॑षा॒ त्वम् । \newline
53. वि॒ख्यायेति॑ वि - ख्याय॑ । \newline
54. चक्षु॑षा॒ त्वम् त्वम् चक्षु॑षा॒ चक्षु॑षा॒ त्व म॒भ्य॑भि त्वम् चक्षु॑षा॒ चक्षु॑षा॒ त्व म॒भि । \newline
55. त्वम॒भ्य॑भि त्वम् त्व म॒भि ति॑ष्ठ तिष्ठा॒भि त्वम् त्व म॒भि ति॑ष्ठ । \newline
56. अ॒भि ति॑ष्ठ तिष्ठा॒भ्य॑भि ति॑ष्ठ पृतन्य॒तः पृ॑तन्य॒त स्ति॑ष्ठा॒ भ्य॑भि ति॑ष्ठ पृतन्य॒तः । \newline
57. ति॒ष्ठ॒ पृ॒त॒न्य॒तः पृ॑तन्य॒त स्ति॑ष्ठ तिष्ठ पृतन्य॒तः । \newline
\pagebreak
\markright{ TS 4.1.2.4  \hfill https://www.vedavms.in \hfill}

\section{ TS 4.1.2.4 }

\textbf{TS 4.1.2.4 } \newline
\textbf{Samhita Paata} \newline

पृतन्य॒तः ॥ उत्क्रा॑म मह॒ते सौभ॑गाया॒-स्मादा॒स्थाना᳚द् द्रविणो॒दा वा॑जिन्न् । व॒यꣳ स्या॑म सुम॒तौ पृ॑थि॒व्या अ॒ग्निं ख॑नि॒ष्यन्त॑ उ॒पस्थे॑ अस्याः ॥ उद॑क्रमीद् द्रविणो॒दा वा॒ज्यर्वाऽकः॒ स लो॒कꣳ सुकृ॑तं पृथि॒व्याः । ततः॑ खनेम सु॒प्रती॑कम॒ग्निꣳ सुवो॒ रुहा॑णा॒ अधि॒ नाक॑ उत्त॒मे ॥ अ॒पो दे॒वीरुप॑ सृज॒ मधु॑मतीरय॒क्ष्माय॑ प्र॒जाभ्यः॑ । तासाꣳ॒॒ स्थाना॒दुज्जि॑हता॒-मोष॑धयः सुपिप्प॒लाः ॥ जिघ॑र्म्य॒- [  ] \newline

\textbf{Pada Paata} \newline

पृ॒त॒न्य॒तः ॥ उदिति॑ । क्रा॒म॒ । म॒ह॒ते । सौभ॑गाय । अ॒स्मात् । आ॒स्थाना॒दित्या᳚ - स्थाना᳚त् । द्र॒वि॒णो॒दा इति॑ द्रविणः - दाः । वा॒जि॒न्न् ॥ व॒यम् । स्या॒म॒ । सु॒म॒ताविति॑ सु - म॒तौ । पृ॒थि॒व्याः । अ॒ग्निम् । ख॒नि॒ष्यन्तः॑ । उ॒पस्थ॒ इत्यु॒प - स्थे॒ । अ॒स्याः॒ ॥ उदिति॑ । अ॒क्र॒मी॒त् । द्र॒वि॒णो॒दा इति॑ द्रविणः - दाः । वा॒जी । अर्वा᳚ । अकः॑ । सः । लो॒कम् । सुकृ॑त॒मिति॒ सु-कृ॒त॒म् । पृ॒थि॒व्याः ॥ ततः॑ । ख॒ने॒म॒ । सु॒प्रती॑क॒मिति॑ सु-प्रती॑कम् । अ॒ग्निम् । सुवः॑ । रुहा॑णाः । अधीति॑ । नाके᳚ । उ॒त्त॒म इत्यु॑त् - त॒मे ॥ अ॒पः । दे॒वीः । उपेति॑ । सृ॒ज॒ । मधु॑मती॒रिति॒ मधु॑ - म॒तीः॒ । अ॒य॒क्ष्माय॑ । प्र॒जाभ्य॒ इति॑ प्र-जाभ्यः॑ ॥ तासा᳚म् । स्थाना᳚त् । उदिति॑ । जि॒ह॒ता॒म् । ओष॑धयः । सु॒पि॒प्प॒ला इति॑ सु - पि॒प्प॒लाः ॥ जिघ॑र्मि ।  \newline


\textbf{Krama Paata} \newline

पृ॒त॒न्य॒त इति॑ पृतन्य॒तः ॥ उत् क्रा॑म । क्रा॒म॒ म॒ह॒ते । म॒ह॒ते सौभ॑गाय । सौभ॑गाया॒स्मात् । अ॒स्मादा॒स्थाना᳚त् । आ॒स्थाना᳚द् द्रविणो॒दाः । आ॒स्थाना॒दित्या᳚ - स्थाना᳚त् । द्र॒वि॒णो॒दा वा॑जिन्न् । द्र॒वि॒णो॒दा इति॑ द्रविणः - दाः । वा॒जि॒न्निति॑ वाजिन्न् ॥ व॒यꣳ स्या॑म । स्या॒म॒ सु॒म॒तौ । सु॒म॒तौ पृ॑थि॒व्याः । सु॒म॒ताविति॑ सु - म॒तौ । पृ॒थि॒व्या अ॒ग्निम् । अ॒ग्निम् ख॑नि॒ष्यन्तः॑ । ख॒नि॒ष्यन्त॑ उ॒पस्थे᳚ । उ॒पस्थे॑ अस्याः । उ॒पस्थ॒ इत्यु॒प - स्थे॒ । अ॒स्या॒ इत्य॑स्याः ॥ उद॑क्रमीत् । अ॒क्र॒मी॒द् द्र॒वि॒णो॒दाः । द्र॒वि॒णो॒दा वा॒जी । द्र॒वि॒णो॒दा इति॑ द्रविणः - दाः । वा॒ज्यर्वा᳚ । अर्वाऽकः॑ । अकः॒ सः । स लो॒कम् । लो॒कꣳ सुकृ॑तम् । सुकृ॑तम् पृथि॒व्याः । सुकृ॑त॒मिति॒ सु - कृ॒त॒म् । पृ॒थि॒व्या इति॑ पृथि॒व्याः ॥ ततः॑ खनेम । ख॒ने॒म॒ सु॒प्रती॑कम् । सु॒प्रती॑कम॒ग्निम् । सु॒प्रती॑क॒मिति॑ सु - प्रती॑कम् । अ॒ग्निꣳ सुवः॑ । सुवो॒ रुहा॑णाः । रुहा॑णा॒ अधि॑ । अधि॒ नाके᳚ । नाक॑ उत्त॒मे । उ॒त्त॒म इत्यु॑त् - त॒मे ॥ अ॒पो दे॒वीः । दे॒वीरुप॑ । उप॑ सृज । सृ॒ज॒ मधु॑मतीः । मधु॑मतीरय॒क्ष्माय॑ । मधु॑मती॒रिति॒ मधु॑ - म॒तीः॒ । अ॒य॒क्ष्माय॑ प्र॒जाभ्यः॑ । प्र॒जाभ्य॒ इति॑ प्र - जाभ्यः॑ ॥ तासाꣳ॒॒ स्थाना᳚त् । स्थाना॒दुत् । उज् जि॑हताम् । जि॒ह॒ता॒मोष॑धयः । ओष॑धयः सुपिप्प॒लाः । सु॒पि॒प्प॒ला इति॑ सु - पि॒प्प॒लाः ॥ जिघ॑र्म्य॒ग्निम् \newline

\textbf{Jatai Paata} \newline

1. पृ॒त॒न्य॒त इति॑ पृतन्य॒तः । \newline
2. उत् क्रा॑म क्रा॒मोदुत् क्रा॑म । \newline
3. क्रा॒म॒ म॒ह॒ते म॑ह॒ते क्रा॑म क्राम मह॒ते । \newline
4. म॒ह॒ते सौभ॑गाय॒ सौभ॑गाय मह॒ते म॑ह॒ते सौभ॑गाय । \newline
5. सौभ॑गाया॒ स्मा द॒स्माथ् सौभ॑गाय॒ सौभ॑गाया॒ स्मात् । \newline
6. अ॒स्मा दा॒स्थाना॑ दा॒स्थाना॑ द॒स्मा द॒स्मा दा॒स्थाना᳚त् । \newline
7. आ॒स्थाना᳚द् द्रविणो॒दा द्र॑विणो॒दा आ॒स्थाना॑ दा॒स्थाना᳚द् द्रविणो॒दाः । \newline
8. आ॒स्थाना॒दित्या᳚ - स्थाना᳚त् । \newline
9. द्र॒वि॒णो॒दा वा॑जिन्. वाजिन् द्रविणो॒दा द्र॑विणो॒दा वा॑जिन्न् । \newline
10. द्र॒वि॒णो॒दा इति॑ द्रविणः - दाः । \newline
11. वा॒जि॒न्निति॑ वाजिन्न् । \newline
12. व॒यꣳ स्या॑म स्याम व॒यं ॅव॒यꣳ स्या॑म । \newline
13. स्या॒म॒ सु॒म॒तौ सु॑म॒तौ स्या॑म स्याम सुम॒तौ । \newline
14. सु॒म॒तौ पृ॑थि॒व्याः पृ॑थि॒व्याः सु॑म॒तौ सु॑म॒तौ पृ॑थि॒व्याः । \newline
15. सु॒म॒ताविति॑ सु - म॒तौ । \newline
16. पृ॒थि॒व्या अ॒ग्नि म॒ग्निम् पृ॑थि॒व्याः पृ॑थि॒व्या अ॒ग्निम् । \newline
17. अ॒ग्निम् ख॑नि॒ष्यन्तः॑ खनि॒ष्यन्तो॑ अ॒ग्नि म॒ग्निम् ख॑नि॒ष्यन्तः॑ । \newline
18. ख॒नि॒ष्यन्त॑ उ॒पस्थ॑ उ॒पस्थे॑ खनि॒ष्यन्तः॑ खनि॒ष्यन्त॑ उ॒पस्थे᳚ । \newline
19. उ॒पस्थे॑ अस्या अस्या उ॒पस्थ॑ उ॒पस्थे॑ अस्याः । \newline
20. उ॒पस्थ॒ इत्यु॒प - स्थे॒ । \newline
21. अ॒स्या इत्य॑स्याः । \newline
22. उद॑क्रमी दक्रमी॒ दुदु द॑क्रमीत् । \newline
23. अ॒क्र॒मी॒द् द्र॒वि॒णो॒दा द्र॑विणो॒दा अ॑क्रमी दक्रमीद् द्रविणो॒दाः । \newline
24. द्र॒वि॒णो॒दा वा॒जी वा॒जी द्र॑विणो॒दा द्र॑विणो॒दा वा॒जी । \newline
25. द्र॒वि॒णो॒दा इति॑ द्रविणः - दाः । \newline
26. वा॒ज्यर्वा ऽर्वा॑ वा॒जी वा॒ज्यर्वा᳚ । \newline
27. अर्वा ऽक॒ रक॒ रर्वा ऽर्वा ऽकः॑ । \newline
28. अकः॒ स सो अक॒ रकः॒ सः । \newline
29. स लो॒कम् ॅलो॒कꣳ स स लो॒कम् । \newline
30. लो॒कꣳ सुकृ॑तꣳ॒॒ सुकृ॑तम् ॅलो॒कम् ॅलो॒कꣳ सुकृ॑तम् । \newline
31. सुकृ॑तम् पृथि॒व्याः पृ॑थि॒व्याः सुकृ॑तꣳ॒॒ सुकृ॑तम् पृथि॒व्याः । \newline
32. सुकृ॑त॒मिति॒ सु - कृ॒त॒म् । \newline
33. पृ॒थि॒व्या इति॑ पृथि॒व्याः । \newline
34. ततः॑ खनेम खनेम॒ तत॒ स्ततः॑ खनेम । \newline
35. ख॒ने॒म॒ सु॒प्रती॑कꣳ सु॒प्रती॑कम् खनेम खनेम सु॒प्रती॑कम् । \newline
36. सु॒प्रती॑क म॒ग्नि म॒ग्निꣳ सु॒प्रती॑कꣳ सु॒प्रती॑क म॒ग्निम् । \newline
37. सु॒प्रती॑क॒मिति॑ सु - प्रती॑कम् । \newline
38. अ॒ग्निꣳ सुवः॒ सुव॑ र॒ग्नि म॒ग्निꣳ सुवः॑ । \newline
39. सुवो॒ रुहा॑णा॒ रुहा॑णाः॒ सुवः॒ सुवो॒ रुहा॑णाः । \newline
40. रुहा॑णा॒ अध्यधि॒ रुहा॑णा॒ रुहा॑णा॒ अधि॑ । \newline
41. अधि॒ नाके॒ नाके॒ अध्यधि॒ नाके᳚ । \newline
42. नाक॑ उत्त॒म उ॑त्त॒मे नाके॒ नाक॑ उत्त॒मे । \newline
43. उ॒त्त॒म इत्यु॑त् - त॒मे । \newline
44. अ॒पो दे॒वीर् दे॒वी र॒पो अ॒पो दे॒वीः । \newline
45. दे॒वी रुपोप॑ दे॒वीर् दे॒वी रुप॑ । \newline
46. उप॑ सृज सृ॒जोपोप॑ सृज । \newline
47. सृ॒ज॒ मधु॑मती॒र् मधु॑मतीः सृज सृज॒ मधु॑मतीः । \newline
48. मधु॑मती रय॒क्ष्मा या॑य॒क्ष्माय॒ मधु॑मती॒र् मधु॑मती रय॒क्ष्माय॑ । \newline
49. मधु॑मती॒रिति॒ मधु॑ - म॒तीः॒ । \newline
50. अ॒य॒क्ष्माय॑ प्र॒जाभ्यः॑ प्र॒जाभ्यो॑ ऽय॒क्ष्माया॑ य॒क्ष्माय॑ प्र॒जाभ्यः॑ । \newline
51. प्र॒जाभ्य॒ इति॑ प्र - जाभ्यः॑ । \newline
52. तासाꣳ॒॒ स्थाना॒थ् स्थाना॒त् तासा॒म् तासाꣳ॒॒ स्थाना᳚त् । \newline
53. स्थाना॒ दुदुत् थाना॒थ् स्थाना॒ दुत् । \newline
54. उज् जि॑हताम् जिहता॒ मुदुज् जि॑हताम् । \newline
55. जि॒ह॒ता॒ मोष॑धय॒ ओष॑धयो जिहताम् जिहता॒ मोष॑धयः । \newline
56. ओष॑धयः सुपिप्प॒लाः सु॑पिप्प॒ला ओष॑धय॒ ओष॑धयः सुपिप्प॒लाः । \newline
57. सु॒पि॒प्प॒ला इति॑ सु - पि॒प्प॒लाः । \newline
58. जिघ॑र्म्य॒ग्नि म॒ग्निम् जिघ॑र्मि॒ जिघ॑र्म्य॒ग्निम् । \newline

\textbf{Ghana Paata } \newline

1. पृ॒त॒न्य॒त इति॑ पृतन्य॒तः । \newline
2. उत् क्रा॑म क्रा॒मोदुत् क्रा॑म मह॒ते म॑ह॒ते क्रा॒मोदुत् क्रा॑म मह॒ते । \newline
3. क्रा॒म॒ म॒ह॒ते म॑ह॒ते क्रा॑म क्राम मह॒ते सौभ॑गाय॒ सौभ॑गाय मह॒ते क्रा॑म क्राम मह॒ते सौभ॑गाय । \newline
4. म॒ह॒ते सौभ॑गाय॒ सौभ॑गाय मह॒ते म॑ह॒ते सौभ॑गाया॒ स्मा द॒स्माथ् सौभ॑गाय मह॒ते म॑ह॒ते सौभ॑गा या॒स्मात् । \newline
5. सौभ॑गाया॒ स्मा द॒स्माथ् सौभ॑गाय॒ सौभ॑गाया॒ स्मा दा॒स्थाना॑ दा॒स्थाना॑ द॒स्माथ् सौभ॑गाय॒ सौभ॑गाया॒ स्मा दा॒स्थाना᳚त् । \newline
6. अ॒स्मा दा॒स्थाना॑ दा॒स्थाना॑ द॒स्मा द॒स्मा दा॒स्थाना᳚द् द्रविणो॒दा द्र॑विणो॒दा आ॒स्थाना॑ द॒स्मा द॒स्मा दा॒स्थाना᳚द् द्रविणो॒दाः । \newline
7. आ॒स्थाना᳚द् द्रविणो॒दा द्र॑विणो॒दा आ॒स्थाना॑ दा॒स्थाना᳚द् द्रविणो॒दा वा॑जिन्. वाजिन् द्रविणो॒दा आ॒स्थाना॑ दा॒स्थाना᳚द् द्रविणो॒दा वा॑जिन्न् । \newline
8. आ॒स्थाना॒दित्या᳚ - स्थाना᳚त् । \newline
9. द्र॒वि॒णो॒दा वा॑जिन्. वाजिन् द्रविणो॒दा द्र॑विणो॒दा वा॑जिन्न् । \newline
10. द्र॒वि॒णो॒दा इति॑ द्रविणः - दाः । \newline
11. वा॒जि॒न्निति॑ वाजिन्न् । \newline
12. व॒यꣳ स्या॑म स्याम व॒यं ॅव॒यꣳ स्या॑म सुम॒तौ सु॑म॒तौ स्या॑म व॒यं ॅव॒यꣳ स्या॑म सुम॒तौ । \newline
13. स्या॒म॒ सु॒म॒तौ सु॑म॒तौ स्या॑म स्याम सुम॒तौ पृ॑थि॒व्याः पृ॑थि॒व्याः सु॑म॒तौ स्या॑म स्याम सुम॒तौ पृ॑थि॒व्याः । \newline
14. सु॒म॒तौ पृ॑थि॒व्याः पृ॑थि॒व्याः सु॑म॒तौ सु॑म॒तौ पृ॑थि॒व्या अ॒ग्नि म॒ग्निम् पृ॑थि॒व्याः सु॑म॒तौ सु॑म॒तौ पृ॑थि॒व्या अ॒ग्निम् । \newline
15. सु॒म॒ताविति॑ सु - म॒तौ । \newline
16. पृ॒थि॒व्या अ॒ग्नि म॒ग्निम् पृ॑थि॒व्याः पृ॑थि॒व्या अ॒ग्निम् ख॑नि॒ष्यन्तः॑ खनि॒ष्यन्तो॑ अ॒ग्निम् पृ॑थि॒व्याः पृ॑थि॒व्या अ॒ग्निम् ख॑नि॒ष्यन्तः॑ । \newline
17. अ॒ग्निम् ख॑नि॒ष्यन्तः॑ खनि॒ष्यन्तो॑ अ॒ग्नि म॒ग्निम् ख॑नि॒ष्यन्त॑ उ॒पस्थ॑ उ॒पस्थे॑ खनि॒ष्यन्तो॑ अ॒ग्नि म॒ग्निम् ख॑नि॒ष्यन्त॑ उ॒पस्थे᳚ । \newline
18. ख॒नि॒ष्यन्त॑ उ॒पस्थ॑ उ॒पस्थे॑ खनि॒ष्यन्तः॑ खनि॒ष्यन्त॑ उ॒पस्थे॑ अस्या अस्या उ॒पस्थे॑ खनि॒ष्यन्तः॑ खनि॒ष्यन्त॑ उ॒पस्थे॑ अस्याः । \newline
19. उ॒पस्थे॑ अस्या अस्या उ॒पस्थ॑ उ॒पस्थे॑ अस्याः । \newline
20. उ॒पस्थ॒ इत्यु॒प - स्थे॒ । \newline
21. अ॒स्या इत्य॑स्याः । \newline
22. उद॑क्रमी दक्रमी॒ दुदु द॑क्रमीद् द्रविणो॒दा द्र॑विणो॒दा अ॑क्रमी॒ दुदु द॑क्रमीद् द्रविणो॒दाः । \newline
23. अ॒क्र॒मी॒द् द्र॒वि॒णो॒दा द्र॑विणो॒दा अ॑क्रमी दक्रमीद् द्रविणो॒दा वा॒जी वा॒जी द्र॑विणो॒दा अ॑क्रमी दक्रमीद् द्रविणो॒दा वा॒जी । \newline
24. द्र॒वि॒णो॒दा वा॒जी वा॒जी द्र॑विणो॒दा द्र॑विणो॒दा वा॒ज्यर्वा ऽर्वा॑ वा॒जी द्र॑विणो॒दा द्र॑विणो॒दा वा॒ज्यर्वा᳚ । \newline
25. द्र॒वि॒णो॒दा इति॑ द्रविणः - दाः । \newline
26. वा॒ज्यर्वा ऽर्वा॑ वा॒जी वा॒ज्यर्वा ऽक॒ रक॒ रर्वा॑ वा॒जी वा॒ज्यर्वा ऽकः॑ । \newline
27. अर्वा ऽक॒ रक॒ रर्वा ऽर्वा ऽकः॒ स सो अक॒ रर्वा ऽर्वा ऽकः॒ सः । \newline
28. अकः॒ स सो अक॒ रकः॒ स लो॒कम् ॅलो॒कꣳ सो अक॒ रकः॒ स लो॒कम् । \newline
29. स लो॒कम् ॅलो॒कꣳ स स लो॒कꣳ सुकृ॑तꣳ॒॒ सुकृ॑तम् ॅलो॒कꣳ स स लो॒कꣳ सुकृ॑तम् । \newline
30. लो॒कꣳ सुकृ॑तꣳ॒॒ सुकृ॑तम् ॅलो॒कम् ॅलो॒कꣳ सुकृ॑तम् पृथि॒व्याः पृ॑थि॒व्याः सुकृ॑तम् ॅलो॒कम् ॅलो॒कꣳ सुकृ॑तम् पृथि॒व्याः । \newline
31. सुकृ॑तम् पृथि॒व्याः पृ॑थि॒व्याः सुकृ॑तꣳ॒॒ सुकृ॑तम् पृथि॒व्याः । \newline
32. सुकृ॑त॒मिति॒ सु - कृ॒त॒म् । \newline
33. पृ॒थि॒व्या इति॑ पृथि॒व्याः । \newline
34. ततः॑ खनेम खनेम॒ तत॒ स्ततः॑ खनेम सु॒प्रती॑कꣳ सु॒प्रती॑कम् खनेम॒ तत॒ स्ततः॑ खनेम सु॒प्रती॑कम् । \newline
35. ख॒ने॒म॒ सु॒प्रती॑कꣳ सु॒प्रती॑कम् खनेम खनेम सु॒प्रती॑क म॒ग्नि म॒ग्निꣳ सु॒प्रती॑कम् खनेम खनेम सु॒प्रती॑क म॒ग्निम् । \newline
36. सु॒प्रती॑क म॒ग्नि म॒ग्निꣳ सु॒प्रती॑कꣳ सु॒प्रती॑क म॒ग्निꣳ सुवः॒ सुव॑ र॒ग्निꣳ सु॒प्रती॑कꣳ सु॒प्रती॑क म॒ग्निꣳ सुवः॑ । \newline
37. सु॒प्रती॑क॒मिति॑ सु - प्रती॑कम् । \newline
38. अ॒ग्निꣳ सुवः॒ सुव॑ र॒ग्नि म॒ग्निꣳ सुवो॒ रुहा॑णा॒ रुहा॑णाः॒ सुव॑ र॒ग्नि म॒ग्निꣳ सुवो॒ रुहा॑णाः । \newline
39. सुवो॒ रुहा॑णा॒ रुहा॑णाः॒ सुवः॒ सुवो॒ रुहा॑णा॒ अध्यधि॒ रुहा॑णाः॒ सुवः॒ सुवो॒ रुहा॑णा॒ अधि॑ । \newline
40. रुहा॑णा॒ अध्यधि॒ रुहा॑णा॒ रुहा॑णा॒ अधि॒ नाके॒ नाके॒ अधि॒ रुहा॑णा॒ रुहा॑णा॒ अधि॒ नाके᳚ । \newline
41. अधि॒ नाके॒ नाके॒ अध्यधि॒ नाक॑ उत्त॒म उ॑त्त॒मे नाके॒ अध्यधि॒ नाक॑ उत्त॒मे । \newline
42. नाक॑ उत्त॒म उ॑त्त॒मे नाके॒ नाक॑ उत्त॒मे । \newline
43. उ॒त्त॒म इत्यु॑त् - त॒मे । \newline
44. अ॒पो दे॒वीर् दे॒वी र॒पो अ॒पो दे॒वी रुपोप॑ दे॒वी र॒पो अ॒पो दे॒वी रुप॑ । \newline
45. दे॒वी रुपोप॑ दे॒वीर् दे॒वी रुप॑ सृज सृ॒जोप॑ दे॒वीर् दे॒वी रुप॑ सृज । \newline
46. उप॑ सृज सृ॒जोपोप॑ सृज॒ मधु॑मती॒र् मधु॑मतीः सृ॒जोपोप॑ सृज॒ मधु॑मतीः । \newline
47. सृ॒ज॒ मधु॑मती॒र् मधु॑मतीः सृज सृज॒ मधु॑मती रय॒क्ष्माया॑ य॒क्ष्माय॒ मधु॑मतीः सृज सृज॒ मधु॑मती रय॒क्ष्माय॑ । \newline
48. मधु॑मती रय॒क्ष्माया॑ य॒क्ष्माय॒ मधु॑मती॒र् मधु॑मती रय॒क्ष्माय॑ प्र॒जाभ्यः॑ 
प्र॒जाभ्यो॑ ऽय॒क्ष्माय॒ मधु॑मती॒र् मधु॑मती रय॒क्ष्माय॑ प्र॒जाभ्यः॑ । \newline
49. मधु॑मती॒रिति॒ मधु॑ - म॒तीः॒ । \newline
50. अ॒य॒क्ष्माय॑ प्र॒जाभ्यः॑ प्र॒जाभ्यो॑ ऽय॒क्ष्माया॑ य॒क्ष्माय॑ प्र॒जाभ्यः॑ । \newline
51. प्र॒जाभ्य॒ इति॑ प्र - जाभ्यः॑ । \newline
52. तासाꣳ॒॒ स्थाना॒थ् स्थाना॒त् तासा॒म् तासाꣳ॒॒ स्थाना॒ दुदुत् थाना॒त् तासा॒म् तासाꣳ॒॒ स्थाना॒ दुत् । \newline
53. स्थाना॒ दुदुत् थाना॒थ् स्थाना॒ दुज् जि॑हताम् जिहता॒ मुत् थाना॒थ् स्थाना॒ दुज् जि॑हताम् । \newline
54. उज् जि॑हताम् जिहता॒ मुदुज् जि॑हता॒ मोष॑धय॒ ओष॑धयो जिहता॒ मुदुज् जि॑हता॒ मोष॑धयः । \newline
55. जि॒ह॒ता॒ मोष॑धय॒ ओष॑धयो जिहताम् जिहता॒ मोष॑धयः सुपिप्प॒लाः सु॑पिप्प॒ला ओष॑धयो जिहताम् जिहता॒ मोष॑धयः सुपिप्प॒लाः । \newline
56. ओष॑धयः सुपिप्प॒लाः सु॑पिप्प॒ला ओष॑धय॒ ओष॑धयः सुपिप्प॒लाः । \newline
57. सु॒पि॒प्प॒ला इति॑ सु - पि॒प्प॒लाः । \newline
58. जिघ॑र् म्य॒ग्नि म॒ग्निम् जिघ॑र्मि॒ जिघ॑र् म्य॒ग्निम् मन॑सा॒ मन॑सा॒ ऽग्निम् जिघ॑र्मि॒ जिघ॑र् म्य॒ग्निम् मन॑सा । \newline
\pagebreak
\markright{ TS 4.1.2.5  \hfill https://www.vedavms.in \hfill}

\section{ TS 4.1.2.5 }

\textbf{TS 4.1.2.5 } \newline
\textbf{Samhita Paata} \newline

-ग्निं मन॑सा घृ॒तेन॑ प्रति॒क्ष्यन्तं॒ भुव॑नानि॒ विश्वा᳚ । पृ॒थुं ति॑र॒श्चा वय॑सा बृ॒हन्तं॒ ॅव्यचि॑ष्ठ॒मन्नꣳ॑ रभ॒सं ॅविदा॑नं ॥ आ त्वा॑ जिघर्मि॒ वच॑सा घृ॒तेना॑ऽर॒क्षसा॒ मन॑सा॒ तज्जु॑षस्व । मर्य॑श्रीः स्पृह॒यद्-व॑र्णो अ॒ग्निर्नाऽभि॒मृशे॑ त॒नुवा॒ जर्.हृ॑षाणः ॥ परि॒ वाज॑पतिः क॒विर॒ग्निर्. ह॒व्यान्य॑क्रमीत् । दध॒द् रत्ना॑नि दा॒शुषे᳚ ॥ परि॑ त्वाऽग्ने॒ पुरं॑ ॅव॒यं ॅविप्रꣳ॑ सहस्य धीमहि । धृ॒षद् व॑र्णं दि॒वेदि॑वे भे॒त्तारं॑ ( ) भङ्गु॒राव॑तः ॥ त्वम॑ग्ने॒ द्युभि॒स्त्व-मा॑शुशु॒क्षणि॒स्त्व-म॒द्भ्यस्त्व-मश्म॑न॒स्परि॑ । त्वं ॅवने᳚भ्य॒ स्त्वमोष॑धीभ्य॒ स्त्वं नृ॒णां नृ॑पते जायसे॒ शुचिः॑ ॥ \newline

\textbf{Pada Paata} \newline

अ॒ग्निम् । मन॑सा । घृ॒तेन॑ । प्र॒ति॒क्ष्यन्त॒मिति॑ प्रति - क्ष्यन्त᳚म् । भुव॑नानि । विश्वा᳚ ॥ पृ॒थुम् । ति॒र॒श्चा । वय॑सा । बृ॒हन्त᳚म् । व्यचि॑ष्ठम् । अन्न᳚म् । र॒भ॒सम् । विदा॑नम् ॥ एति॑ । त्वा॒ । जि॒घ॒र्मि॒ । वच॑सा । घृ॒तेन॑ । अ॒र॒क्षसा᳚ । मन॑सा । तत् । जु॒ष॒स्व॒ ॥ मर्य॑श्री॒रिति॒ मर्य॑-श्रीः॒ । स्पृ॒ह॒यद्व॑र्ण॒ इति॑ स्पृह॒यत् - व॒र्णः॒ । अ॒ग्निः । न । अ॒भि॒मृश॒ इत्य॑भि - मृशे᳚ । त॒नुवा᳚ । जर्.हृ॑षाणः ॥ परीति॑ । वाज॑पति॒रिति॒ वाज॑ - प॒तिः॒ । क॒विः । अ॒ग्निः । ह॒व्यानि॑ । अ॒क्र॒मी॒त् ॥ दध॑त् । रत्ना॑नि । दा॒शुषे᳚ ॥ परीति॑ । त्वा॒ । अ॒ग्ने॒ । पुर᳚म् । व॒यम् । विप्र᳚म् । स॒ह॒स्य॒ । धी॒म॒हि॒ ॥ धृ॒षद्व॑र्ण॒मिति॑ धृ॒षत् - व॒र्ण॒म् । दि॒वेदि॑व॒ इति॑ दि॒वे - दि॒वे॒ । भे॒त्तार᳚म् ( ) । भ॒ङ्गु॒राव॑त॒ इति॑ भङ्गु॒र - व॒तः॒ ॥ त्वम् । अ॒ग्ने॒ । द्युभि॒रिति॒ द्यु - भिः॒ । त्वम् । आ॒शु॒शु॒क्षणिः॑ । त्वम् । अ॒द्भ्य इत्य॑त् - भ्यः । त्वम् । अश्म॑नः । परि॑ ॥ त्वम् । वने᳚भ्यः । त्वम् । ओष॑धीभ्य॒ इत्योष॑धि - भ्यः॒ । त्वम् । नृ॒णाम् । नृ॒प॒त॒ इति॑ नृ-प॒ते॒ । जा॒य॒से॒ । शुचिः॑ ॥  \newline


\textbf{Krama Paata} \newline

अ॒ग्निम् मन॑सा । मन॑सा घृ॒तेन॑ । घृ॒तेन॑ प्रति॒क्ष्यन्त᳚म् । प्र॒ति॒क्ष्यन्त॒म् भुव॑नानि । प्र॒ति॒क्ष्यन्त॒मिति॑ प्रति - क्ष्यन्त᳚म् । भुव॑नानि॒ विश्वा᳚ । विश्वेति॒ विश्वा᳚ ॥ पृ॒थुम् ति॑र॒श्चा । ति॒र॒श्चा वय॑सा । वय॑सा बृ॒हन्त᳚म् । बृ॒हन्तं॒ ॅव्यचि॑ष्ठम् । व्यचि॑ष्ठ॒मन्न᳚म् । अन्नꣳ॑ रभ॒सम् । र॒भ॒सं ॅविदा॑नम् । विदा॑न॒मिति॒ विदा॑नम् ॥ आ त्वा᳚ । त्वा॒ जि॒घ॒र्मि॒ । जि॒घ॒र्मि॒ वच॑सा । वच॑सा घृ॒तेन॑ । घृ॒तेना॑र॒क्षसा᳚ । अ॒र॒क्षसा॒ मन॑सा । मन॑सा॒ तत् । तज् जु॑षस्व । जु॒ष॒स्वेति॑ जुषस्व ॥ मर्य॑श्रीः स्पृह॒यद्व॑र्णः । मर्य॑श्री॒रिति॒ मर्य॑ - श्रीः॒ । स्पृ॒ह॒यद्व॑र्णो अ॒ग्निः । स्पृ॒ह॒यद्व॑र्ण॒ इति॑ स्पृह॒यत् - व॒र्णः॒ । अ॒ग्निर् न । नाभि॒मृशे᳚ । अ॒भि॒मृशे॑ त॒नुवा᳚ । अ॒भि॒मृश॒ इत्य॑भि - मृशे᳚ । त॒नुवा॒ जर्.हृ॑षाणः । जर्.हृ॑षाण॒ इति॒ जर्.हृ॑षाणः ॥ परि॒ वाज॑पतिः । वाज॑पतिः क॒विः । वाज॑पति॒रिति॒ वाज॑ - प॒तिः॒ । क॒विर॒ग्निः । अ॒ग्निर्. ह॒व्यानि॑ । ह॒व्यान्य॑क्रमीत् । अ॒क्र॒मी॒दित्य॑क्रमीत् ॥ दध॒द् रत्ना॑नि । रत्ना॑नि दा॒शुषे᳚ । दा॒शुष॒ इति॑ दा॒शुषे᳚ ॥ परि॑ त्वा । त्वा॒ऽग्ने॒ । अ॒ग्ने॒ पुर᳚म् । पुरं॑ ॅव॒यम् । व॒यं ॅविप्र᳚म् । विप्रꣳ॑ सहस्य । स॒ह॒स्य॒ धी॒म॒हि॒ । धी॒म॒हीति॑ धीमहि ॥ धृ॒षद्व॑र्णम् दि॒वेदि॑वे । धृ॒षद्व॑र्ण॒मिति॑ धृ॒षत् - व॒र्ण॒म् । दि॒वेदि॑वे भे॒त्तार᳚म् ( ) । दि॒वेदि॑व॒ इति॑ दि॒वे - दि॒वे॒ । भे॒त्तार॑म् भङ्गु॒राव॑तः । भ॒ङ्गु॒राव॑त॒ इति॑ भङ्गु॒र - व॒तः॒ ॥ त्वम॑ग्ने । अ॒ग्ने॒ द्युभिः॑ । द्युभि॒स्त्वम् । द्युभि॒रिति॒ द्यु - भिः॒ । त्वमा॑शुशु॒क्षणिः॑ । आ॒शु॒शु॒क्षणि॒ स्त्वम् । त्वम॒द्भ्यः । अ॒द्भ्यस्त्वम् । अ॒द्भ्य इत्य॑त् - भ्यः । त्वमश्म॑नः । अश्म॑न॒स्परि॑ । परीति॒ परि॑ ॥ त्वं ॅवने᳚भ्यः । वने᳚भ्य॒स्त्वम् । त्वमोष॑धीभ्यः । ओष॑धीभ्य॒स्त्वम् । ओष॑धीभ्य॒ इत्योष॑धि - भ्यः॒ । त्वम् नृ॒णाम् । नृ॒णाम् नृ॑पते । नृ॒प॒ते॒ जा॒य॒से॒ । नृ॒प॒त॒ इति॑ नृ - प॒ते॒ । जा॒य॒से॒ शुचिः॑ । शुचि॒रिति॒ शुचिः॑ । \newline

\textbf{Jatai Paata} \newline

1. अ॒ग्निम् मन॑सा॒ मन॑सा॒ ऽग्नि म॒ग्निम् मन॑सा । \newline
2. मन॑सा घृ॒तेन॑ घृ॒तेन॒ मन॑सा॒ मन॑सा घृ॒तेन॑ । \newline
3. घृ॒तेन॑ प्रति॒क्ष्यन्त॑म् प्रति॒क्ष्यन्त॑म् घृ॒तेन॑ घृ॒तेन॑ प्रति॒क्ष्यन्त᳚म् । \newline
4. प्र॒ति॒क्ष्यन्त॒म् भुव॑नानि॒ भुव॑नानि प्रति॒क्ष्यन्त॑म् प्रति॒क्ष्यन्त॒म् भुव॑नानि । \newline
5. प्र॒ति॒क्ष्यन्त॒मिति॑ प्रति - क्ष्यन्त᳚म् । \newline
6. भुव॑नानि॒ विश्वा॒ विश्वा॒ भुव॑नानि॒ भुव॑नानि॒ विश्वा᳚ । \newline
7. विश्वेति॒ विश्वा᳚ । \newline
8. पृ॒थुम् ति॑र॒श्चा ति॑र॒श्चा पृ॒थुम् पृ॒थुम् ति॑र॒श्चा । \newline
9. ति॒र॒श्चा वय॑सा॒ वय॑सा तिर॒श्चा ति॑र॒श्चा वय॑सा । \newline
10. वय॑सा बृ॒हन्त॑म् बृ॒हन्तं॒ ॅवय॑सा॒ वय॑सा बृ॒हन्त᳚म् । \newline
11. बृ॒हन्तं॒ ॅव्यचि॑ष्ठं॒ ॅव्यचि॑ष्ठम् बृ॒हन्त॑म् बृ॒हन्तं॒ ॅव्यचि॑ष्ठम् । \newline
12. व्यचि॑ष्ठ॒ मन्न॒ मन्नं॒ ॅव्यचि॑ष्ठं॒ ॅव्यचि॑ष्ठ॒ मन्न᳚म् । \newline
13. अन्नꣳ॑ रभ॒सꣳ र॑भ॒स मन्न॒ मन्नꣳ॑ रभ॒सम् । \newline
14. र॒भ॒सं ॅविदा॑नं॒ ॅविदा॑नꣳ रभ॒सꣳ र॑भ॒सं ॅविदा॑नम् । \newline
15. विदा॑न॒मिति॒ विदा॑नम् । \newline
16. आ त्वा॒ त्वा ऽऽत्वा᳚ । \newline
17. त्वा॒ जि॒घ॒र्मि॒ जि॒घ॒र्मि॒ त्वा॒ त्वा॒ जि॒घ॒र्मि॒ । \newline
18. जि॒घ॒र्मि॒ वच॑सा॒ वच॑सा जिघर्मि जिघर्मि॒ वच॑सा । \newline
19. वच॑सा घृ॒तेन॑ घृ॒तेन॒ वच॑सा॒ वच॑सा घृ॒तेन॑ । \newline
20. घृ॒तेना॑ र॒क्षसा॑ ऽर॒क्षसा॑ घृ॒तेन॑ घृ॒तेना॑ र॒क्षसा᳚ । \newline
21. अ॒र॒क्षसा॒ मन॑सा॒ मन॑सा ऽर॒क्षसा॑ ऽर॒क्षसा॒ मन॑सा । \newline
22. मन॑सा॒ तत् तन् मन॑सा॒ मन॑सा॒ तत् । \newline
23. तज् जु॑षस्व जुषस्व॒ तत् तज् जु॑षस्व । \newline
24. जु॒ष॒स्वेति॑ जुषस्व । \newline
25. मर्य॑श्रीः स्पृह॒यद्व॑र्णः स्पृह॒यद्व॑र्णो॒ मर्य॑श्री॒र् मर्य॑श्रीः स्पृह॒यद्व॑र्णः । \newline
26. मर्य॑श्री॒रिति॒ मर्य॑ - श्रीः॒ । \newline
27. स्पृ॒ह॒यद्व॑र्णो अ॒ग्निर॒ग्निः स्पृ॑ह॒यद्व॑र्णः स्पृह॒यद्व॑र्णो अ॒ग्निः । \newline
28. स्पृ॒ह॒यद्व॑र्ण॒ इति॑ स्पृह॒यत् - व॒र्णः॒ । \newline
29. अ॒ग्निर् न नाग्नि र॒ग्निर् न । \newline
30. नाभि॒मृशे॑ अभि॒मृशे॒ न नाभि॒मृशे᳚ । \newline
31. अ॒भि॒मृशे॑ त॒नुवा॑ त॒नुवा॑ ऽभि॒मृशे॑ अभि॒मृशे॑ त॒नुवा᳚ । \newline
32. अ॒भि॒मृश॒ इत्य॑भि - मृशे᳚ । \newline
33. त॒नुवा॒ जर्.हृ॑षाणो॒ जर्.हृ॑षाण स्त॒नुवा॑ त॒नुवा॒ जर्.हृ॑षाणः । \newline
34. जर्.हृ॑षाण॒ इति॒ जर्.हृ॑षाणः । \newline
35. परि॒ वाज॑पति॒र् वाज॑पतिः॒ परि॒ परि॒ वाज॑पतिः । \newline
36. वाज॑पतिः क॒विः क॒विर् वाज॑पति॒र् वाज॑पतिः क॒विः । \newline
37. वाज॑पति॒रिति॒ वाज॑ - प॒तिः॒ । \newline
38. क॒वि र॒ग्नि र॒ग्निः क॒विः क॒वि र॒ग्निः । \newline
39. अ॒ग्निर्. ह॒व्यानि॑ ह॒व्या न्य॒ग्नि र॒ग्निर्. ह॒व्यानि॑ । \newline
40. ह॒व्या न्य॑क्रमी दक्रमी द्ध॒व्यानि॑ ह॒व्या न्य॑क्रमीत् । \newline
41. अ॒क्र॒मी॒दित्य॑क्रमीत् । \newline
42. दध॒द् रत्ना॑नि॒ रत्ना॑नि॒ दध॒द् दध॒द् रत्ना॑नि । \newline
43. रत्ना॑नि दा॒शुषे॑ दा॒शुषे॒ रत्ना॑नि॒ रत्ना॑नि दा॒शुषे᳚ । \newline
44. दा॒शुष॒ इति॑ दा॒शुषे᳚ । \newline
45. परि॑ त्वा त्वा॒ परि॒ परि॑ त्वा । \newline
46. त्वा॒ ऽग्ने॒ ऽग्ने॒ त्वा॒ त्वा॒ ऽग्ने॒ । \newline
47. अ॒ग्ने॒ पुर॒म् पुर॑ मग्ने ऽग्ने॒ पुर᳚म् । \newline
48. पुरं॑ ॅव॒यं ॅव॒यम् पुर॒म् पुरं॑ ॅव॒यम् । \newline
49. व॒यं ॅविप्रं॒ ॅविप्रं॑ ॅव॒यं ॅव॒यं ॅविप्र᳚म् । \newline
50. विप्रꣳ॑ सहस्य सहस्य॒ विप्रं॒ ॅविप्रꣳ॑ सहस्य । \newline
51. स॒ह॒स्य॒ धी॒म॒हि॒ धी॒म॒हि॒ स॒ह॒स्य॒ स॒ह॒स्य॒ धी॒म॒हि॒ । \newline
52. धी॒म॒हीति॑ धीमहि । \newline
53. धृ॒षद्व॑र्णम् दि॒वेदि॑वे दि॒वेदि॑वे धृ॒षद्व॑र्णम् धृ॒षद्व॑र्णम् दि॒वेदि॑वे । \newline
54. धृ॒षद्व॑र्ण॒मिति॑ धृ॒षत् - व॒र्ण॒म् । \newline
55. दि॒वेदि॑वे भे॒त्तार॑म् भे॒त्तार॑म् दि॒वेदि॑वे दि॒वेदि॑वे भे॒त्तार᳚म् । \newline
56. दि॒वेदि॑व॒ इति॑ दि॒वे - दि॒वे॒ । \newline
57. भे॒त्तार॑म् भङ्गु॒राव॑तो भङ्गु॒राव॑तो भे॒त्तार॑म् भे॒त्तार॑म् भङ्गु॒राव॑तः । \newline
58. भ॒ङ्गु॒राव॑त॒ इति॑ भङ्गु॒र - व॒तः॒ । \newline
59. त्व म॑ग्ने अग्ने॒ त्वम् त्व म॑ग्ने । \newline
60. अ॒ग्ने॒ द्युभि॒र् द्युभि॑ रग्ने अग्ने॒ द्युभिः॑ । \newline
61. द्युभि॒ स्त्वम् त्वम् द्युभि॒र् द्युभि॒ स्त्वम् । \newline
62. द्युभि॒रिति॒ द्यु - भिः॒ । \newline
63. त्व मा॑शुशु॒क्षणि॑ राशुशु॒क्षणि॒ स्त्वम् त्व मा॑शुशु॒क्षणिः॑ । \newline
64. आ॒शु॒शु॒क्षणि॒ स्त्वम् त्व मा॑शुशु॒क्षणि॑ राशुशु॒क्षणि॒ स्त्वम् । \newline
65. त्व म॒द्भ्यो अ॒द्भ्य स्त्वम् त्व म॒द्भ्यः । \newline
66. अ॒द्भ्य स्त्वम् त्व म॒द्भ्यो अ॒द्भ्य स्त्वम् । \newline
67. अ॒द्भ्य इत्य॑त् - भ्यः । \newline
68. त्व मश्म॑नो॒ अश्म॑न॒ स्त्वम् त्व मश्म॑नः । \newline
69. अश्म॑न॒ स्परि॒ पर्यश्म॑नो॒ अश्म॑न॒ स्परि॑ । \newline
70. परीति॒ परि॑ । \newline
71. त्वं ॅवने᳚भ्यो॒ वने᳚भ्य॒ स्त्वम् त्वं ॅवने᳚भ्यः । \newline
72. वने᳚भ्य॒ स्त्वम् त्वं ॅवने᳚भ्यो॒ वने᳚भ्य॒ स्त्वम् । \newline
73. त्व मोष॑धीभ्य॒ ओष॑धीभ्य॒ स्त्वम् त्व मोष॑धीभ्यः । \newline
74. ओष॑धीभ्य॒ स्त्वम् त्व मोष॑धीभ्य॒ ओष॑धीभ्य॒ स्त्वम् । \newline
75. ओष॑धीभ्य॒ इत्योष॑धि - भ्यः॒ । \newline
76. त्वम् नृ॒णाम् नृ॒णाम् त्वम् त्वम् नृ॒णाम् । \newline
77. नृ॒णाम् नृ॑पते नृपते नृ॒णाम् नृ॒णाम् नृ॑पते । \newline
78. नृ॒प॒ते॒ जा॒य॒से॒ जा॒य॒से॒ नृ॒प॒ते॒ नृ॒प॒ते॒ जा॒य॒से॒ । \newline
79. नृ॒प॒त॒ इति॑ नृ - प॒ते॒ । \newline
80. जा॒य॒से॒ शुचिः॒ शुचि॑र् जायसे जायसे॒ शुचिः॑ । \newline
81. शुचि॒रिति॒ शुचिः॑ । \newline

\textbf{Ghana Paata } \newline

1. अ॒ग्निम् मन॑सा॒ मन॑सा॒ ऽग्नि म॒ग्निम् मन॑सा घृ॒तेन॑ घृ॒तेन॒ मन॑सा॒ ऽग्नि म॒ग्निम् मन॑सा घृ॒तेन॑ । \newline
2. मन॑सा घृ॒तेन॑ घृ॒तेन॒ मन॑सा॒ मन॑सा घृ॒तेन॑ प्रति॒क्ष्यन्त॑म् प्रति॒क्ष्यन्त॑म् घृ॒तेन॒ मन॑सा॒ मन॑सा घृ॒तेन॑ प्रति॒क्ष्यन्त᳚म् । \newline
3. घृ॒तेन॑ प्रति॒क्ष्यन्त॑म् प्रति॒क्ष्यन्त॑म् घृ॒तेन॑ घृ॒तेन॑ प्रति॒क्ष्यन्त॒म् भुव॑नानि॒ भुव॑नानि प्रति॒क्ष्यन्त॑म् घृ॒तेन॑ घृ॒तेन॑ प्रति॒क्ष्यन्त॒म् भुव॑नानि । \newline
4. प्र॒ति॒क्ष्यन्त॒म् भुव॑नानि॒ भुव॑नानि प्रति॒क्ष्यन्त॑म् प्रति॒क्ष्यन्त॒म् भुव॑नानि॒ विश्वा॒ विश्वा॒ भुव॑नानि प्रति॒क्ष्यन्त॑म् प्रति॒क्ष्यन्त॒म् भुव॑नानि॒ विश्वा᳚ । \newline
5. प्र॒ति॒क्ष्यन्त॒मिति॑ प्रति - क्ष्यन्त᳚म् । \newline
6. भुव॑नानि॒ विश्वा॒ विश्वा॒ भुव॑नानि॒ भुव॑नानि॒ विश्वा᳚ । \newline
7. विश्वेति॒ विश्वा᳚ । \newline
8. पृ॒थुम् ति॑र॒श्चा ति॑र॒श्चा पृ॒थुम् पृ॒थुम् ति॑र॒श्चा वय॑सा॒ वय॑सा तिर॒श्चा पृ॒थुम् पृ॒थुम् ति॑र॒श्चा वय॑सा । \newline
9. ति॒र॒श्चा वय॑सा॒ वय॑सा तिर॒श्चा ति॑र॒श्चा वय॑सा बृ॒हन्त॑म् बृ॒हन्तं॒ ॅवय॑सा तिर॒श्चा ति॑र॒श्चा वय॑सा बृ॒हन्त᳚म् । \newline
10. वय॑सा बृ॒हन्त॑म् बृ॒हन्तं॒ ॅवय॑सा॒ वय॑सा बृ॒हन्तं॒ ॅव्यचि॑ष्ठं॒ ॅव्यचि॑ष्ठम् बृ॒हन्तं॒ ॅवय॑सा॒ वय॑सा बृ॒हन्तं॒ ॅव्यचि॑ष्ठम् । \newline
11. बृ॒हन्तं॒ ॅव्यचि॑ष्ठं॒ ॅव्यचि॑ष्ठम् बृ॒हन्त॑म् बृ॒हन्तं॒ ॅव्यचि॑ष्ठ॒ मन्न॒ मन्नं॒ ॅव्यचि॑ष्ठम् बृ॒हन्त॑म् बृ॒हन्तं॒ ॅव्यचि॑ष्ठ॒ मन्न᳚म् । \newline
12. व्यचि॑ष्ठ॒ मन्न॒ मन्नं॒ ॅव्यचि॑ष्ठं॒ ॅव्यचि॑ष्ठ॒ मन्नꣳ॑ रभ॒सꣳ र॑भ॒स मन्नं॒ ॅव्यचि॑ष्ठं॒ ॅव्यचि॑ष्ठ॒ मन्नꣳ॑ रभ॒सम् । \newline
13. अन्नꣳ॑ रभ॒सꣳ र॑भ॒स मन्न॒ मन्नꣳ॑ रभ॒सं ॅविदा॑नं॒ ॅविदा॑नꣳ रभ॒स मन्न॒ मन्नꣳ॑ रभ॒सं ॅविदा॑नम् । \newline
14. र॒भ॒सं ॅविदा॑नं॒ ॅविदा॑नꣳ रभ॒सꣳ र॑भ॒सं ॅविदा॑नम् । \newline
15. विदा॑न॒मिति॒ विदा॑नम् । \newline
16. आ त्वा॒ त्वा ऽऽत्वा॑ जिघर्मि जिघर्मि॒ त्वा ऽऽत्वा॑ जिघर्मि । \newline
17. त्वा॒ जि॒घ॒र्मि॒ जि॒घ॒र्मि॒ त्वा॒ त्वा॒ जि॒घ॒र्मि॒ वच॑सा॒ वच॑सा जिघर्मि त्वा त्वा जिघर्मि॒ वच॑सा । \newline
18. जि॒घ॒र्मि॒ वच॑सा॒ वच॑सा जिघर्मि जिघर्मि॒ वच॑सा घृ॒तेन॑ घृ॒तेन॒ वच॑सा जिघर्मि जिघर्मि॒ वच॑सा घृ॒तेन॑ । \newline
19. वच॑सा घृ॒तेन॑ घृ॒तेन॒ वच॑सा॒ वच॑सा घृ॒तेना॑ र॒क्षसा॑ ऽर॒क्षसा॑ घृ॒तेन॒ वच॑सा॒ वच॑सा घृ॒तेना॑ र॒क्षसा᳚ । \newline
20. घृ॒तेना॑ र॒क्षसा॑ ऽर॒क्षसा॑ घृ॒तेन॑ घृ॒तेना॑ र॒क्षसा॒ मन॑सा॒ मन॑सा ऽर॒क्षसा॑ घृ॒तेन॑ घृ॒तेना॑ र॒क्षसा॒ मन॑सा । \newline
21. अ॒र॒क्षसा॒ मन॑सा॒ मन॑सा ऽर॒क्षसा॑ ऽर॒क्षसा॒ मन॑सा॒ तत् तन् मन॑सा ऽर॒क्षसा॑ ऽर॒क्षसा॒ मन॑सा॒ तत् । \newline
22. मन॑सा॒ तत् तन् मन॑सा॒ मन॑सा॒ तज् जु॑षस्व जुषस्व॒ तन् मन॑सा॒ मन॑सा॒ तज् जु॑षस्व । \newline
23. तज् जु॑षस्व जुषस्व॒ तत् तज् जु॑षस्व । \newline
24. जु॒ष॒स्वेति॑ जुषस्व । \newline
25. मर्य॑श्रीः स्पृह॒यद्व॑र्णः स्पृह॒यद्व॑र्णो॒ मर्य॑श्री॒र् मर्य॑श्रीः स्पृह॒यद्व॑र्णो अ॒ग्नि र॒ग्निः स्पृ॑ह॒यद्व॑र्णो॒ मर्य॑श्री॒र् मर्य॑श्रीः स्पृह॒यद्व॑र्णो अ॒ग्निः । \newline
26. मर्य॑श्री॒रिति॒ मर्य॑ - श्रीः॒ । \newline
27. स्पृ॒ह॒यद्व॑र्णो अ॒ग्नि र॒ग्निः स्पृ॑ह॒यद्व॑र्णः स्पृह॒यद्व॑र्णो अ॒ग्निर् न नाग्निः स्पृ॑ह॒यद्व॑र्णः स्पृह॒यद्व॑र्णो अ॒ग्निर् न । \newline
28. स्पृ॒ह॒यद्व॑र्ण॒ इति॑ स्पृह॒यत् - व॒र्णः॒ । \newline
29. अ॒ग्निर् न नाग्नि र॒ग्निर् नाभि॒मृशे॑ अभि॒मृशे॒ नाग्नि र॒ग्निर् नाभि॒मृशे᳚ । \newline
30. नाभि॒मृशे॑ अभि॒मृशे॒ न नाभि॒मृशे॑ त॒नुवा॑ त॒नुवा॑ ऽभि॒मृशे॒ न नाभि॒मृशे॑ त॒नुवा᳚ । \newline
31. अ॒भि॒मृशे॑ त॒नुवा॑ त॒नुवा॑ ऽभि॒मृशे॑ अभि॒मृशे॑ त॒नुवा॒ जर्.हृ॑षाणो॒ जर्.हृ॑षाण स्त॒नुवा॑ ऽभि॒मृशे॑ अभि॒मृशे॑ त॒नुवा॒ जर्.हृ॑षाणः । \newline
32. अ॒भि॒मृश॒ इत्य॑भि - मृशे᳚ । \newline
33. त॒नुवा॒ जर्.हृ॑षाणो॒ जर्.हृ॑षाण स्त॒नुवा॑ त॒नुवा॒ जर्.हृ॑षाणः । \newline
34. जर्.हृ॑षाण॒ इति॒ जर्.हृ॑षाणः । \newline
35. परि॒ वाज॑पति॒र् वाज॑पतिः॒ परि॒ परि॒ वाज॑पतिः क॒विः क॒विर् वाज॑पतिः॒ परि॒ परि॒ वाज॑पतिः क॒विः । \newline
36. वाज॑पतिः क॒विः क॒विर् वाज॑पति॒र् वाज॑पतिः क॒वि र॒ग्नि र॒ग्निः क॒विर् वाज॑पति॒र् वाज॑पतिः क॒वि र॒ग्निः । \newline
37. वाज॑पति॒रिति॒ वाज॑ - प॒तिः॒ । \newline
38. क॒वि र॒ग्नि र॒ग्निः क॒विः क॒वि र॒ग्निर्. ह॒व्यानि॑ ह॒व्या न्य॒ग्निः क॒विः क॒वि र॒ग्निर्. ह॒व्यानि॑ । \newline
39. अ॒ग्निर्. ह॒व्यानि॑ ह॒व्या न्य॒ग्नि र॒ग्निर्. ह॒व्या न्य॑क्रमी दक्रमी द्ध॒व्या न्य॒ग्नि र॒ग्निर्. ह॒व्या न्य॑क्रमीत् । \newline
40. ह॒व्या न्य॑क्रमी दक्रमी द्ध॒व्यानि॑ ह॒व्या न्य॑क्रमीत् । \newline
41. अ॒क्र॒मी॒दित्य॑क्रमीत् । \newline
42. दध॒द् रत्ना॑नि॒ रत्ना॑नि॒ दध॒द् दध॒द् रत्ना॑नि दा॒शुषे॑ दा॒शुषे॒ रत्ना॑नि॒ दध॒द् दध॒द् रत्ना॑नि दा॒शुषे᳚ । \newline
43. रत्ना॑नि दा॒शुषे॑ दा॒शुषे॒ रत्ना॑नि॒ रत्ना॑नि दा॒शुषे᳚ । \newline
44. दा॒शुष॒ इति॑ दा॒शुषे᳚ । \newline
45. परि॑ त्वा त्वा॒ परि॒ परि॑ त्वा ऽग्ने ऽग्ने त्वा॒ परि॒ परि॑ त्वा ऽग्ने । \newline
46. त्वा॒ ऽग्ने॒ ऽग्ने॒ त्वा॒ त्वा॒ ऽग्ने॒ पुर॒म् पुर॑ मग्ने त्वा त्वा ऽग्ने॒ पुर᳚म् । \newline
47. अ॒ग्ने॒ पुर॒म् पुर॑ मग्ने ऽग्ने॒ पुरं॑ ॅव॒यं ॅव॒यम् पुर॑ मग्ने ऽग्ने॒ पुरं॑ ॅव॒यम् । \newline
48. पुरं॑ ॅव॒यं ॅव॒यम् पुर॒म् पुरं॑ ॅव॒यं ॅविप्रं॒ ॅविप्रं॑ ॅव॒यम् पुर॒म् पुरं॑ ॅव॒यं ॅविप्र᳚म् । \newline
49. व॒यं ॅविप्रं॒ ॅविप्रं॑ ॅव॒यं ॅव॒यं ॅविप्रꣳ॑ सहस्य सहस्य॒ विप्रं॑ ॅव॒यं ॅव॒यं ॅविप्रꣳ॑ सहस्य । \newline
50. विप्रꣳ॑ सहस्य सहस्य॒ विप्रं॒ ॅविप्रꣳ॑ सहस्य धीमहि धीमहि सहस्य॒ विप्रं॒ ॅविप्रꣳ॑ सहस्य धीमहि । \newline
51. स॒ह॒स्य॒ धी॒म॒हि॒ धी॒म॒हि॒ स॒ह॒स्य॒ स॒ह॒स्य॒ धी॒म॒हि॒ । \newline
52. धी॒म॒हीति॑ धीमहि । \newline
53. धृ॒षद्व॑र्णम् दि॒वेदि॑वे दि॒वेदि॑वे धृ॒षद्व॑र्णम् धृ॒षद्व॑र्णम् दि॒वेदि॑वे भे॒त्तार॑म् भे॒त्तार॑म् दि॒वेदि॑वे धृ॒षद्व॑र्णम् धृ॒षद्व॑र्णम् दि॒वेदि॑वे भे॒त्तार᳚म् । \newline
54. धृ॒षद्व॑र्ण॒मिति॑ धृ॒षत् - व॒र्ण॒म् । \newline
55. दि॒वेदि॑वे भे॒त्तार॑म् भे॒त्तार॑म् दि॒वेदि॑वे दि॒वेदि॑वे भे॒त्तार॑म् भङ्गु॒राव॑तो भङ्गु॒राव॑तो भे॒त्तार॑म् दि॒वेदि॑वे दि॒वेदि॑वे भे॒त्तार॑म् भङ्गु॒राव॑तः । \newline
56. दि॒वेदि॑व॒ इति॑ दि॒वे - दि॒वे॒ । \newline
57. भे॒त्तार॑म् भङ्गु॒राव॑तो भङ्गु॒राव॑तो भे॒त्तार॑म् भे॒त्तार॑म् भङ्गु॒राव॑तः । \newline
58. भ॒ङ्गु॒राव॑त॒ इति॑ भङ्गु॒र - व॒तः॒ । \newline
59. त्व म॑ग्ने अग्ने॒ त्वम् त्व म॑ग्ने॒ द्युभि॒र् द्युभि॑ रग्ने॒ त्वम् त्वम॑ग्ने॒ द्युभिः॑ । \newline
60. अ॒ग्ने॒ द्युभि॒र् द्युभि॑ रग्ने अग्ने॒ द्युभि॒ स्त्वम् त्वम् द्युभि॑ रग्ने अग्ने॒ द्युभि॒ स्त्वम् । \newline
61. द्युभि॒स्त्वम् त्वम् द्युभि॒र् द्युभि॒ स्त्व मा॑शुशु॒क्षणि॑ राशुशु॒क्षणि॒ स्त्वम् द्युभि॒र् द्युभि॒ स्त्व मा॑शुशु॒क्षणिः॑ । \newline
62. द्युभि॒रिति॒ द्यु - भिः॒ । \newline
63. त्व मा॑शुशु॒क्षणि॑ राशुशु॒क्षणि॒ स्त्वम् त्व मा॑शुशु॒क्षणि॒ स्त्वम् त्व मा॑शुशु॒क्षणि॒ स्त्वम् त्व मा॑शुशु॒क्षणि॒ स्त्वम् । \newline
64. आ॒शु॒शु॒क्षणि॒ स्त्वम् त्व मा॑शुशु॒क्षणि॑ राशुशु॒क्षणि॒ स्त्व म॒द्भ्यो अ॒द्भ्य स्त्व मा॑शुशु॒क्षणि॑ राशुशु॒क्षणि॒ स्त्व म॒द्भ्यः । \newline
65. त्वम॒द्भ्यो अ॒द्भ्य स्त्वम् त्वम॒द्भ्य स्त्वम् त्वम॒द्भ्य स्त्वम् त्वम॒द्भ्य स्त्वम् । \newline
66. अ॒द्भ्य स्त्वम् त्वम॒द्भ्यो अ॒द्भ्य स्त्व मश्म॑नो॒ अश्म॑न॒ स्त्व म॒द्भ्यो अ॒द्भ्य स्त्व मश्म॑नः । \newline
67. अ॒द्भ्य इत्य॑त् - भ्यः । \newline
68. त्वमश्म॑नो॒ अश्म॑न॒ स्त्वम् त्वमश्म॑न॒ स्परि॒ पर्यश्म॑न॒ स्त्वम् त्वमश्म॑न॒ स्परि॑ । \newline
69. अश्म॑न॒ स्परि॒ पर्यश्म॑नो॒ अश्म॑न॒ स्परि॑ । \newline
70. परीति॒ परि॑ । \newline
71. त्वं ॅवने᳚भ्यो॒ वने᳚भ्य॒ स्त्वम् त्वं ॅवने᳚भ्य॒ स्त्वम् त्वं ॅवने᳚भ्य॒ स्त्वम् त्वं ॅवने᳚भ्य॒ स्त्वम् । \newline
72. वने᳚भ्य॒ स्त्वम् त्वं ॅवने᳚भ्यो॒ वने᳚भ्य॒ स्त्व मोष॑धीभ्य॒ ओष॑धीभ्य॒ स्त्वं ॅवने᳚भ्यो॒ वने᳚भ्य॒ स्त्व मोष॑धीभ्यः । \newline
73. त्वमोष॑धीभ्य॒ ओष॑धीभ्य॒ स्त्वम् त्वमोष॑धीभ्य॒ स्त्वम् त्वमोष॑धीभ्य॒ स्त्वम् त्वमोष॑धीभ्य॒ स्त्वम् । \newline
74. ओष॑धीभ्य॒ स्त्वम् त्वमोष॑धीभ्य॒ ओष॑धीभ्य॒ स्त्वन् नृ॒णाम् नृ॒णाम् त्वमोष॑धीभ्य॒ ओष॑धीभ्य॒ स्त्वन् नृ॒णाम् । \newline
75. ओष॑धीभ्य॒ इत्योष॑धि - भ्यः॒ । \newline
76. त्वम् नृ॒णाम् नृ॒णाम् त्वम् त्वम् नृ॒णाम् नृ॑पते नृपते नृ॒णाम् त्वम् त्वम् नृ॒णाम् नृ॑पते । \newline
77. नृ॒णाम् नृ॑पते नृपते नृ॒णाम् नृ॒णाम् नृ॑पते जायसे जायसे नृपते नृ॒णाम् नृ॒णाम् नृ॑पते जायसे । \newline
78. नृ॒प॒ते॒ जा॒य॒से॒ जा॒य॒से॒ नृ॒प॒ते॒ नृ॒प॒ते॒ जा॒य॒से॒ शुचिः॒ शुचि॑र् जायसे नृपते नृपते जायसे॒ शुचिः॑ । \newline
79. नृ॒प॒त॒ इति॑ नृ - प॒ते॒ । \newline
80. जा॒य॒से॒ शुचिः॒ शुचि॑र् जायसे जायसे॒ शुचिः॑ । \newline
81. शुचि॒रिति॒ शुचिः॑ । \newline
\pagebreak
\markright{ TS 4.1.3.1  \hfill https://www.vedavms.in \hfill}

\section{ TS 4.1.3.1 }

\textbf{TS 4.1.3.1 } \newline
\textbf{Samhita Paata} \newline

दे॒वस्य॑ त्वा सवि॒तुः प्र॑स॒वे᳚ऽश्विनो᳚ र्बा॒हुभ्यां᳚ पू॒ष्णो हस्ता᳚भ्यां पृथि॒व्याः स॒धस्थे॒ऽग्निं पु॑री॒ष्य॑-मङ्गिर॒स्वत् ख॑नामि ॥ ज्योति॑ष्मन्तं त्वाऽग्ने सु॒प्रती॑क॒मज॑स्रेण भा॒नुना॒ दीद्या॑नं । शि॒वं प्र॒जाभ्योऽहिꣳ॑ सन्तं पृथि॒व्याः स॒धस्थे॒ऽग्निं पु॑री॒ष्य॑ -मङ्गिर॒स्वत् ख॑नामि ॥ अ॒पां पृ॒ष्ठम॑सि स॒प्रथा॑ उ॒र्व॑ग्निं भ॑रि॒ष्यदप॑रावपिष्ठं । वर्द्ध॑मानं म॒ह आ च॒ पुष्क॑रं दि॒वो मात्र॑या वरि॒णा प्र॑थस्व ॥ शर्म॑ च स्थो॒- [  ] \newline

\textbf{Pada Paata} \newline

दे॒वस्य॑ । त्वा॒ । स॒वि॒तुः । प्र॒स॒व इति॑ प्र - स॒वे । अ॒श्विनोः᳚ । बा॒हुभ्या॒मिति॑ बा॒हु - भ्या॒म् । पू॒ष्णः । हस्ता᳚भ्याम् । पृ॒थि॒व्याः । स॒धस्थ॒ इति॑ स॒ध - स्थे॒ । अ॒ग्निम् । पु॒री॒ष्य᳚म् । अ॒ङ्गि॒र॒स्वत् । ख॒ना॒मि॒ ॥ ज्योति॑ष्मन्तम् । त्वा॒ । अ॒ग्ने॒ । सु॒प्रती॑क॒मिति॑ सु - प्रती॑कम् । अज॑स्रेण । भा॒नुना᳚ । दीद्या॑नम् ॥ शि॒वम् । प्र॒जाभ्य॒ इति॑ प्र - जाभ्यः॑ । अहिꣳ॑सन्तम् । पृ॒थि॒व्याः । स॒धस्थ॒ इति॑ स॒ध - स्थे॒ । अ॒ग्निम् । पु॒री॒ष्य᳚म् । अ॒ङ्गि॒र॒स्वत् । ख॒ना॒मि॒ ॥ अ॒पाम् । पृ॒ष्ठम् । अ॒सि॒ । स॒प्रथा॒ इति॑ स - प्रथाः᳚ । उ॒रु । अ॒ग्निम् । भ॒रि॒ष्यत् । प॑रावपिष्ठ॒मित्यप॑रा - व॒पि॒ष्ठ॒म् ॥ वद्‌र्ध॑मानम् । म॒हः । एति॑ । च॒ । पुष्क॑रम् । दि॒वः । मात्र॑या । व॒रि॒णा । प्र॒थ॒स्व॒ ॥ शर्म॑ । च॒ । स्थः॒ ।  \newline


\textbf{Krama Paata} \newline

दे॒वस्य॑ त्वा । त्वा॒ स॒वि॒तुः । स॒वि॒तुः प्र॑स॒वे । प्र॒स॒वे᳚ऽश्विनोः᳚ । प्र॒स॒व इति॑ प्र - स॒वे । अ॒श्विनो᳚र् बा॒हुभ्या᳚म् । बा॒हुभ्या᳚म् पू॒ष्णः । बा॒हुभ्या॒मिति॑ बा॒हु - भ्या॒म् । पू॒ष्णो हस्ता᳚भ्याम् । हस्ता᳚भ्याम् पृथि॒व्याः । पृ॒थि॒व्याः स॒धस्थे᳚ । स॒धस्थे॒ऽग्निम् । स॒धस्थ॒ इति॑ स॒ध - स्थे॒ । अ॒ग्निम् पु॑री॒ष्य᳚म् । पु॒री॒ष्य॑मङ्गिर॒स्वत् । अ॒ङ्गि॒र॒स्वत् ख॑नामि । ख॒ना॒मीति॑ खनामि ॥ ज्योति॑ष्मन्तम् त्वा । त्वा॒ऽग्ने॒ । अ॒ग्ने॒ सु॒प्रती॑कम् । सु॒प्रती॑क॒मज॑स्रेण । सु॒प्रती॑क॒मिति॑ सु - प्रती॑कम् । अज॑स्रेण भा॒नुना᳚ । भा॒नुना॒ दीद्या॑नम् । दीद्या॑न॒मिति॒ दीद्या॑नम् ॥ शि॒वम् प्र॒जाभ्यः॑ । प्र॒जाभ्योऽहिꣳ॑सन्तम् । प्र॒जाभ्य॒ इति॑ प्र - जाभ्यः॑ । अहिꣳ॑सन्तम् पृथि॒व्याः । पृ॒थि॒व्याः स॒धस्थे᳚ । स॒धस्थे॒ऽग्निम् । स॒धस्थ॒ इति॑ स॒ध - स्थे॒ । अ॒ग्निम् पु॑री॒ष्य᳚म् । पु॒री॒ष्य॑मङ्गिर॒स्वत् । अ॒ङ्गि॒र॒स्वत् ख॑नामि । ख॒ना॒मीति॑ खनामि ॥ अ॒पाम् पृ॒ष्ठम् । पृ॒ष्ठम॑सि । अ॒सि॒ स॒प्रथाः᳚ । स॒प्रथा॑ उ॒रु । स॒प्रथा॒ इति॑ स - प्रथाः᳚ । उ॒र्व॑ग्निम् । अ॒ग्निम् भ॑रि॒ष्यत् । भ॒रि॒ष्यदप॑रावपिष्ठम् । अप॑रावपिष्ठ॒मित्यप॑रा - व॒पि॒ष्ठ॒म् ॥ वर्द्ध॑मानम् म॒हः । म॒ह आ । आ च॑ । च॒ पुष्क॑रम् । पुष्क॑रम् दि॒वः । दि॒वो मात्र॑या । मात्र॑या वरि॒णा । व॒रि॒णा प्र॑थस्व । प्र॒थ॒स्वेति॑ प्रथस्व ॥ शर्म॑ च । च॒ स्थः॒ । स्थो॒ वर्म॑ \newline

\textbf{Jatai Paata} \newline

1. दे॒वस्य॑ त्वा त्वा दे॒वस्य॑ दे॒वस्य॑ त्वा । \newline
2. त्वा॒ स॒वि॒तुः स॑वि॒तु स्त्वा᳚ त्वा सवि॒तुः । \newline
3. स॒वि॒तुः प्र॑स॒वे प्र॑स॒वे स॑वि॒तुः स॑वि॒तुः प्र॑स॒वे । \newline
4. प्र॒स॒वे᳚ ऽश्विनो॑ र॒श्विनोः᳚ प्रस॒वे प्र॑स॒वे᳚ ऽश्विनोः᳚ । \newline
5. प्र॒स॒व इति॑ प्र - स॒वे । \newline
6. अ॒श्विनो᳚र् बा॒हुभ्या᳚म् बा॒हुभ्या॑ म॒श्विनो॑ र॒श्विनो᳚र् बा॒हुभ्या᳚म् । \newline
7. बा॒हुभ्या᳚म् पू॒ष्णः पू॒ष्णो बा॒हुभ्या᳚म् बा॒हुभ्या᳚म् पू॒ष्णः । \newline
8. बा॒हुभ्या॒मिति॑ बा॒हु - भ्या॒म् । \newline
9. पू॒ष्णो हस्ता᳚भ्याꣳ॒॒ हस्ता᳚भ्याम् पू॒ष्णः पू॒ष्णो हस्ता᳚भ्याम् । \newline
10. हस्ता᳚भ्याम् पृथि॒व्याः पृ॑थि॒व्या हस्ता᳚भ्याꣳ॒॒ हस्ता᳚भ्याम् पृथि॒व्याः । \newline
11. पृ॒थि॒व्याः स॒धस्थे॑ स॒धस्थे॑ पृथि॒व्याः पृ॑थि॒व्याः स॒धस्थे᳚ । \newline
12. स॒धस्थे॒ ऽग्नि म॒ग्निꣳ स॒धस्थे॑ स॒धस्थे॒ ऽग्निम् । \newline
13. स॒धस्थ॒ इति॑ स॒ध - स्थे॒ । \newline
14. अ॒ग्निम् पु॑री॒ष्य॑म् पुरी॒ष्य॑ म॒ग्नि म॒ग्निम् पु॑री॒ष्य᳚म् । \newline
15. पु॒री॒ष्य॑ मङ्गिर॒स्व द॑ङ्गिर॒स्वत् पु॑री॒ष्य॑म् पुरी॒ष्य॑ मङ्गिर॒स्वत् । \newline
16. अ॒ङ्गि॒र॒स्वत् ख॑नामि खना म्यङ्गिर॒स्व द॑ङ्गिर॒स्वत् ख॑नामि । \newline
17. ख॒ना॒मीति॑ खनामि । \newline
18. ज्योति॑ष्मन्तम् त्वा त्वा॒ ज्योति॑ष्मन्त॒म् ज्योति॑ष्मन्तम् त्वा । \newline
19. त्वा॒ ऽग्ने॒ अ॒ग्ने॒ त्वा॒ त्वा॒ ऽग्ने॒ । \newline
20. अ॒ग्ने॒ सु॒प्रती॑कꣳ सु॒प्रती॑क मग्ने अग्ने सु॒प्रती॑कम् । \newline
21. सु॒प्रती॑क॒ मज॑स्रे॒णाज॑स्रेण सु॒प्रती॑कꣳ सु॒प्रती॑क॒ मज॑स्रेण । \newline
22. सु॒प्रती॑क॒मिति॑ सु - प्रती॑कम् । \newline
23. अज॑स्रेण भा॒नुना॑ भा॒नुना ऽज॑स्रे॒णा ज॑स्रेण भा॒नुना᳚ । \newline
24. भा॒नुना॒ दीद्या॑न॒म् दीद्या॑नम् भा॒नुना॑ भा॒नुना॒ दीद्या॑नम् । \newline
25. दीद्या॑न॒मिति॒ दीद्या॑नम् । \newline
26. शि॒वम् प्र॒जाभ्यः॑ प्र॒जाभ्यः॑ शि॒वꣳ शि॒वम् प्र॒जाभ्यः॑ । \newline
27. प्र॒जाभ्यो ऽहिꣳ॑सन्त॒ महिꣳ॑सन्तम् प्र॒जाभ्यः॑ प्र॒जाभ्यो ऽहिꣳ॑सन्तम् । \newline
28. प्र॒जाभ्य॒ इति॑ प्र - जाभ्यः॑ । \newline
29. अहिꣳ॑सन्तम् पृथि॒व्याः पृ॑थि॒व्या अहिꣳ॑सन्त॒ महिꣳ॑सन्तम् पृथि॒व्याः । \newline
30. पृ॒थि॒व्याः स॒धस्थे॑ स॒धस्थे॑ पृथि॒व्याः पृ॑थि॒व्याः स॒धस्थे᳚ । \newline
31. स॒धस्थे॒ ऽग्नि म॒ग्निꣳ स॒धस्थे॑ स॒धस्थे॒ ऽग्निम् । \newline
32. स॒धस्थ॒ इति॑ स॒ध - स्थे॒ । \newline
33. अ॒ग्निम् पु॑री॒ष्य॑म् पुरी॒ष्य॑ म॒ग्नि म॒ग्निम् पु॑री॒ष्य᳚म् । \newline
34. पु॒री॒ष्य॑ मङ्गिर॒स्व द॑ङ्गिर॒स्वत् पु॑री॒ष्य॑म् पुरी॒ष्य॑ मङ्गिर॒स्वत् । \newline
35. अ॒ङ्गि॒र॒स्वत् ख॑नामि खना म्यङ्गिर॒स्व द॑ङ्गिर॒स्वत् ख॑नामि । \newline
36. ख॒ना॒मीति॑ खनामि । \newline
37. अ॒पाम् पृ॒ष्ठम् पृ॒ष्ठ म॒पा म॒पाम् पृ॒ष्ठम् । \newline
38. पृ॒ष्ठ म॑स्यसि पृ॒ष्ठम् पृ॒ष्ठ म॑सि । \newline
39. अ॒सि॒ स॒प्रथाः᳚ स॒प्रथा॑ अस्यसि स॒प्रथाः᳚ । \newline
40. स॒प्रथा॑ उ॒रू॑रु स॒प्रथाः᳚ स॒प्रथा॑ उ॒रु । \newline
41. स॒प्रथा॒ इति॑ स - प्रथाः᳚ । \newline
42. उ॒र्व॑ग्नि म॒ग्नि मु॒रू᳚(1॒)र्व॑ग्निम् । \newline
43. अ॒ग्निम् भ॑रि॒ष्यद् भ॑रि॒ष्य द॒ग्नि म॒ग्निम् भ॑रि॒ष्यत् । \newline
44. भ॒रि॒ष्य दप॑रावपि॒ष्ठ मप॑रावपिष्ठम् भरि॒ष्यद् भ॑रि॒ष्य दप॑रावपिष्ठम् । \newline
45. अप॑रावपिष्ठ॒मित्यप॑रा - व॒पि॒ष्ठ॒म् । \newline
46. वर्द्ध॑मानम् म॒हो म॒हो वर्द्ध॑मानं॒ ॅवर्द्ध॑मानम् म॒हः । \newline
47. म॒ह आ म॒हो म॒ह आ । \newline
48. आ च॒ चा च॑ । \newline
49. च॒ पुष्क॑र॒म् पुष्क॑रम् च च॒ पुष्क॑रम् । \newline
50. पुष्क॑रम् दि॒वो दि॒वः पुष्क॑र॒म् पुष्क॑रम् दि॒वः । \newline
51. दि॒वो मात्र॑या॒ मात्र॑या दि॒वो दि॒वो मात्र॑या । \newline
52. मात्र॑या वरि॒णा व॑रि॒णा मात्र॑या॒ मात्र॑या वरि॒णा । \newline
53. व॒रि॒णा प्र॑थस्व प्रथस्व वरि॒णा व॑रि॒णा प्र॑थस्व । \newline
54. प्र॒थ॒स्वेति॑ प्रथस्व । \newline
55. शर्म॑ च च॒ शर्म॒ शर्म॑ च । \newline
56. च॒ स्थः॒ स्थ॒ श्च॒ च॒ स्थः॒ । \newline
57. स्थो॒ वर्म॒ वर्म॑ स्थः स्थो॒ वर्म॑ । \newline

\textbf{Ghana Paata } \newline

1. दे॒वस्य॑ त्वा त्वा दे॒वस्य॑ दे॒वस्य॑ त्वा सवि॒तुः स॑वि॒तु स्त्वा॑ दे॒वस्य॑ दे॒वस्य॑ त्वा सवि॒तुः । \newline
2. त्वा॒ स॒वि॒तुः स॑वि॒तु स्त्वा᳚ त्वा सवि॒तुः प्र॑स॒वे प्र॑स॒वे स॑वि॒तु स्त्वा᳚ त्वा सवि॒तुः प्र॑स॒वे । \newline
3. स॒वि॒तुः प्र॑स॒वे प्र॑स॒वे स॑वि॒तुः स॑वि॒तुः प्र॑स॒वे᳚ ऽश्विनो॑ र॒श्विनोः᳚ प्रस॒वे स॑वि॒तुः स॑वि॒तुः प्र॑स॒वे᳚ ऽश्विनोः᳚ । \newline
4. प्र॒स॒वे᳚ ऽश्विनो॑ र॒श्विनोः᳚ प्रस॒वे प्र॑स॒वे᳚ ऽश्विनो᳚र् बा॒हुभ्या᳚म् बा॒हुभ्या॑ म॒श्विनोः᳚ प्रस॒वे प्र॑स॒वे᳚ ऽश्विनो᳚र् बा॒हुभ्या᳚म् । \newline
5. प्र॒स॒व इति॑ प्र - स॒वे । \newline
6. अ॒श्विनो᳚र् बा॒हुभ्या᳚म् बा॒हुभ्या॑ म॒श्विनो॑ र॒श्विनो᳚र् बा॒हुभ्या᳚म् पू॒ष्णः पू॒ष्णो बा॒हुभ्या॑ म॒श्विनो॑ र॒श्विनो᳚र् बा॒हुभ्या᳚म् पू॒ष्णः । \newline
7. बा॒हुभ्या᳚म् पू॒ष्णः पू॒ष्णो बा॒हुभ्या᳚म् बा॒हुभ्या᳚म् पू॒ष्णो हस्ता᳚भ्याꣳ॒॒ हस्ता᳚भ्याम् पू॒ष्णो बा॒हुभ्या᳚म् बा॒हुभ्या᳚म् पू॒ष्णो हस्ता᳚भ्याम् । \newline
8. बा॒हुभ्या॒मिति॑ बा॒हु - भ्या॒म् । \newline
9. पू॒ष्णो हस्ता᳚भ्याꣳ॒॒ हस्ता᳚भ्याम् पू॒ष्णः पू॒ष्णो हस्ता᳚भ्याम् पृथि॒व्याः पृ॑थि॒व्या हस्ता᳚भ्याम् पू॒ष्णः पू॒ष्णो हस्ता᳚भ्याम् पृथि॒व्याः । \newline
10. हस्ता᳚भ्याम् पृथि॒व्याः पृ॑थि॒व्या हस्ता᳚भ्याꣳ॒॒ हस्ता᳚भ्याम् पृथि॒व्याः स॒धस्थे॑ स॒धस्थे॑ पृथि॒व्या हस्ता᳚भ्याꣳ॒॒ हस्ता᳚भ्याम् पृथि॒व्याः स॒धस्थे᳚ । \newline
11. पृ॒थि॒व्याः स॒धस्थे॑ स॒धस्थे॑ पृथि॒व्याः पृ॑थि॒व्याः स॒धस्थे॒ ऽग्नि म॒ग्निꣳ स॒धस्थे॑ पृथि॒व्याः पृ॑थि॒व्याः स॒धस्थे॒ ऽग्निम् । \newline
12. स॒धस्थे॒ ऽग्नि म॒ग्निꣳ स॒धस्थे॑ स॒धस्थे॒ ऽग्निम् पु॑री॒ष्य॑म् पुरी॒ष्य॑ म॒ग्निꣳ स॒धस्थे॑ स॒धस्थे॒ ऽग्निम् पु॑री॒ष्य᳚म् । \newline
13. स॒धस्थ॒ इति॑ स॒ध - स्थे॒ । \newline
14. अ॒ग्निम् पु॑री॒ष्य॑म् पुरी॒ष्य॑ म॒ग्नि म॒ग्निम् पु॑री॒ष्य॑ मङ्गिर॒स्व द॑ङ्गिर॒स्वत् पु॑री॒ष्य॑ म॒ग्नि म॒ग्निम् पु॑री॒ष्य॑ मङ्गिर॒स्वत् । \newline
15. पु॒री॒ष्य॑ मङ्गिर॒स्व द॑ङ्गिर॒स्वत् पु॑री॒ष्य॑म् पुरी॒ष्य॑ मङ्गिर॒स्वत् ख॑नामि खना म्यङ्गिर॒स्वत् पु॑री॒ष्य॑म् पुरी॒ष्य॑ मङ्गिर॒स्वत् ख॑नामि । \newline
16. अ॒ङ्गि॒र॒स्वत् ख॑नामि खना म्यङ्गिर॒स्व द॑ङ्गिर॒स्वत् ख॑नामि । \newline
17. ख॒ना॒मीति॑ खनामि । \newline
18. ज्योति॑ष्मन्तम् त्वा त्वा॒ ज्योति॑ष्मन्त॒म् ज्योति॑ष्मन्तम् त्वा ऽग्ने अग्ने त्वा॒ ज्योति॑ष्मन्त॒म् ज्योति॑ष्मन्तम् त्वा ऽग्ने । \newline
19. त्वा॒ ऽग्ने॒ अ॒ग्ने॒ त्वा॒ त्वा॒ ऽग्ने॒ सु॒प्रती॑कꣳ सु॒प्रती॑क मग्ने त्वा त्वा ऽग्ने सु॒प्रती॑कम् । \newline
20. अ॒ग्ने॒ सु॒प्रती॑कꣳ सु॒प्रती॑क मग्ने अग्ने सु॒प्रती॑क॒ मज॑स्रे॒णा ज॑स्रेण सु॒प्रती॑क मग्ने अग्ने सु॒प्रती॑क॒ मज॑स्रेण । \newline
21. सु॒प्रती॑क॒ मज॑स्रे॒णा ज॑स्रेण सु॒प्रती॑कꣳ सु॒प्रती॑क॒ मज॑स्रेण भा॒नुना॑ भा॒नुना ऽज॑स्रेण सु॒प्रती॑कꣳ सु॒प्रती॑क॒ मज॑स्रेण भा॒नुना᳚ । \newline
22. सु॒प्रती॑क॒मिति॑ सु - प्रती॑कम् । \newline
23. अज॑स्रेण भा॒नुना॑ भा॒नुना ऽज॑स्रे॒णा ज॑स्रेण भा॒नुना॒ दीद्या॑न॒म् दीद्या॑नम् भा॒नुना ऽज॑स्रे॒णा ज॑स्रेण भा॒नुना॒ दीद्या॑नम् । \newline
24. भा॒नुना॒ दीद्या॑न॒म् दीद्या॑नम् भा॒नुना॑ भा॒नुना॒ दीद्या॑नम् । \newline
25. दीद्या॑न॒मिति॒ दीद्या॑नम् । \newline
26. शि॒वम् प्र॒जाभ्यः॑ प्र॒जाभ्यः॑ शि॒वꣳ शि॒वम् प्र॒जाभ्यो ऽहिꣳ॑सन्त॒ महिꣳ॑सन्तम् प्र॒जाभ्यः॑ शि॒वꣳ शि॒वम् प्र॒जाभ्यो ऽहिꣳ॑सन्तम् । \newline
27. प्र॒जाभ्यो ऽहिꣳ॑सन्त॒ महिꣳ॑सन्तम् प्र॒जाभ्यः॑ प्र॒जाभ्यो ऽहिꣳ॑सन्तम् पृथि॒व्याः पृ॑थि॒व्या अहिꣳ॑सन्तम् प्र॒जाभ्यः॑ प्र॒जाभ्यो ऽहिꣳ॑सन्तम् पृथि॒व्याः । \newline
28. प्र॒जाभ्य॒ इति॑ प्र - जाभ्यः॑ । \newline
29. अहिꣳ॑सन्तम् पृथि॒व्याः पृ॑थि॒व्या अहिꣳ॑सन्त॒ महिꣳ॑सन्तम् पृथि॒व्याः स॒धस्थे॑ स॒धस्थे॑ पृथि॒व्या अहिꣳ॑सन्त॒ महिꣳ॑सन्तम् पृथि॒व्याः स॒धस्थे᳚ । \newline
30. पृ॒थि॒व्याः स॒धस्थे॑ स॒धस्थे॑ पृथि॒व्याः पृ॑थि॒व्याः स॒धस्थे॒ ऽग्नि म॒ग्निꣳ स॒धस्थे॑ पृथि॒व्याः पृ॑थि॒व्याः स॒धस्थे॒ ऽग्निम् । \newline
31. स॒धस्थे॒ ऽग्नि म॒ग्निꣳ स॒धस्थे॑ स॒धस्थे॒ ऽग्निम् पु॑री॒ष्य॑म् पुरी॒ष्य॑ म॒ग्निꣳ स॒धस्थे॑ स॒धस्थे॒ ऽग्निम् पु॑री॒ष्य᳚म् । \newline
32. स॒धस्थ॒ इति॑ स॒ध - स्थे॒ । \newline
33. अ॒ग्निम् पु॑री॒ष्य॑म् पुरी॒ष्य॑ म॒ग्नि म॒ग्निम् पु॑री॒ष्य॑ मङ्गिर॒स्व द॑ङ्गिर॒स्वत् पु॑री॒ष्य॑ म॒ग्नि म॒ग्निम् पु॑री॒ष्य॑ मङ्गिर॒स्वत् । \newline
34. पु॒री॒ष्य॑ मङ्गिर॒स्व द॑ङ्गिर॒स्वत् पु॑री॒ष्य॑म् पुरी॒ष्य॑ मङ्गिर॒स्वत् ख॑नामि खना म्यङ्गिर॒स्वत् पु॑री॒ष्य॑म् पुरी॒ष्य॑ मङ्गिर॒स्वत् ख॑नामि । \newline
35. अ॒ङ्गि॒र॒स्वत् ख॑नामि खना म्यङ्गिर॒स्व द॑ङ्गिर॒स्वत् ख॑नामि । \newline
36. ख॒ना॒मीति॑ खनामि । \newline
37. अ॒पाम् पृ॒ष्ठम् पृ॒ष्ठ म॒पा म॒पाम् पृ॒ष्ठ म॑स्यसि पृ॒ष्ठ म॒पा म॒पाम् पृ॒ष्ठ म॑सि । \newline
38. पृ॒ष्ठ म॑स्यसि पृ॒ष्ठम् पृ॒ष्ठ म॑सि स॒प्रथाः᳚ स॒प्रथा॑ असि पृ॒ष्ठम् पृ॒ष्ठ म॑सि स॒प्रथाः᳚ । \newline
39. अ॒सि॒ स॒प्रथाः᳚ स॒प्रथा॑ अस्यसि स॒प्रथा॑ उ॒रू॑रु स॒प्रथा॑ अस्यसि स॒प्रथा॑ उ॒रु । \newline
40. स॒प्रथा॑ उ॒रू॑रु स॒प्रथाः᳚ स॒प्रथा॑ उ॒र्व॑ग्नि म॒ग्नि मु॒रु स॒प्रथाः᳚ स॒प्रथा॑ उ॒र्व॑ग्निम् । \newline
41. स॒प्रथा॒ इति॑ स - प्रथाः᳚ । \newline
42. उ॒र्व॑ग्नि म॒ग्नि मु॒रू᳚(1॒)र्व॑ग्निम् भ॑रि॒ष्यद् भ॑रि॒ष्य द॒ग्नि मु॒रू᳚(1॒)र्व॑ग्निम् भ॑रि॒ष्यत् । \newline
43. अ॒ग्निम् भ॑रि॒ष्यद् भ॑रि॒ष्य द॒ग्नि म॒ग्निम् भ॑रि॒ष्य दप॑रावपि॒ष्ठ मप॑रावपिष्ठम् भरि॒ष्य द॒ग्नि म॒ग्निम् भ॑रि॒ष्य दप॑रावपिष्ठम् । \newline
44. भ॒रि॒ष्य दप॑रावपि॒ष्ठ मप॑रावपिष्ठम् भरि॒ष्यद् भ॑रि॒ष्य दप॑रावपिष्ठम् । \newline
45. अप॑रावपिष्ठ॒मित्यप॑रा - व॒पि॒ष्ठ॒म् । \newline
46. वर्द्ध॑मानम् म॒हो म॒हो वर्द्ध॑मानं॒ ॅवर्द्ध॑मानम् म॒ह आ म॒हो वर्द्ध॑मानं॒ ॅवर्द्ध॑मानम् म॒ह आ । \newline
47. म॒ह आ म॒हो म॒ह आ च॒ चा म॒हो म॒ह आ च॑ । \newline
48. आ च॒ चा च॒ पुष्क॑र॒म् पुष्क॑र॒म् चा च॒ पुष्क॑रम् । \newline
49. च॒ पुष्क॑र॒म् पुष्क॑रम् च च॒ पुष्क॑रम् दि॒वो दि॒वः पुष्क॑रम् च च॒ पुष्क॑रम् दि॒वः । \newline
50. पुष्क॑रम् दि॒वो दि॒वः पुष्क॑र॒म् पुष्क॑रम् दि॒वो मात्र॑या॒ मात्र॑या दि॒वः पुष्क॑र॒म् पुष्क॑रम् दि॒वो मात्र॑या । \newline
51. दि॒वो मात्र॑या॒ मात्र॑या दि॒वो दि॒वो मात्र॑या वरि॒णा व॑रि॒णा मात्र॑या दि॒वो दि॒वो मात्र॑या वरि॒णा । \newline
52. मात्र॑या वरि॒णा व॑रि॒णा मात्र॑या॒ मात्र॑या वरि॒णा प्र॑थस्व प्रथस्व वरि॒णा मात्र॑या॒ मात्र॑या वरि॒णा प्र॑थस्व । \newline
53. व॒रि॒णा प्र॑थस्व प्रथस्व वरि॒णा व॑रि॒णा प्र॑थस्व । \newline
54. प्र॒थ॒स्वेति॑ प्रथस्व । \newline
55. शर्म॑ च च॒ शर्म॒ शर्म॑ च स्थः स्थश्च॒ शर्म॒ शर्म॑ च स्थः । \newline
56. च॒ स्थः॒ स्थ॒ श्च॒ च॒ स्थो॒ वर्म॒ वर्म॑ स्थ श्च च स्थो॒ वर्म॑ । \newline
57. स्थो॒ वर्म॒ वर्म॑ स्थः स्थो॒ वर्म॑ च च॒ वर्म॑ स्थः स्थो॒ वर्म॑ च । \newline
\pagebreak
\markright{ TS 4.1.3.2  \hfill https://www.vedavms.in \hfill}

\section{ TS 4.1.3.2 }

\textbf{TS 4.1.3.2 } \newline
\textbf{Samhita Paata} \newline

वर्म॑ च स्थो॒ अच्छि॑द्रे बहु॒ले उ॒भे । व्यच॑स्वती॒ सं ॅव॑साथां भ॒र्तम॒ग्निं पु॑री॒ष्यं᳚ ॥ संॅव॑साथाꣳ सुव॒र्विदा॑ स॒मीची॒ उर॑सा॒ त्मना᳚ । अ॒ग्निम॒न्त र्भ॑रि॒ष्यन्ती॒ ज्योति॑ष्मन्त॒ मज॑स्र॒मित् ॥ पु॒री॒ष्यो॑ऽसि वि॒श्वभ॑राः । अथ॑र्वा त्वा प्रथ॒मो निर॑मन्थदग्ने ॥ त्वाम॑ग्ने॒ पुष्क॑रा॒दद्ध्यथ॑र्वा॒ निर॑मन्थत । मू॒र्द्ध्नो विश्व॑स्य वा॒घतः॑ ॥ तमु॑ त्वा द॒द्ध्यङ्ङृषिः॑ पु॒त्र ई॑धे॒- [  ] \newline

\textbf{Pada Paata} \newline

वर्म॑ । च॒ । स्थः॒ । अच्छि॑द्रे॒ इति॑ । ब॒हु॒ले इति॑ । उ॒भे इति॑ ॥ व्यच॑स्वती॒ इति॑ । समिति॑ । व॒सा॒था॒म् । भ॒र्तम् । अ॒ग्निम् । पु॒री॒ष्य᳚म् ॥ समिति॑ । व॒सा॒था॒म् । सु॒व॒र्विदेति॑ सुवः - विदा᳚ । स॒मीची॒ इति॑ । उर॑सा । त्मना᳚ ॥ अ॒ग्निम् । अ॒न्तः । भ॒रि॒ष्यन्ती॒ इति॑ । ज्योति॑ष्मन्तम् । अज॑स्रम् । इत् ॥ पु॒री॒ष्यः॑ । अ॒सि॒ । वि॒श्वभ॑रा॒ इति॑ वि॒श्व - भ॒राः॒ ॥ अथ॑र्वा । त्वा॒ । प्र॒थ॒मः । निरिति॑ । अ॒म॒न्थ॒त् । अ॒ग्ने॒ ॥ त्वाम् । अ॒ग्ने॒ । पुष्क॑रात् । अधीति॑ । अथ॑र्वा । निरिति॑ । अ॒म॒न्थ॒त॒ ॥ मू॒द्‌र्ध्नः । विश्व॑स्य । वा॒घतः॑ ॥ तम् । उ॒ । त्वा॒ । द॒द्ध्यङ् । ऋषिः॑ । पु॒त्रः । ई॒धे॒ ।  \newline


\textbf{Krama Paata} \newline

वर्म॑ च । च॒ स्थः॒ । स्थो॒ अच्छि॑द्रे । अच्छि॑द्रे बहु॒ले । अच्छि॑द्रे॒ इत्यच्छि॑द्रे । ब॒हु॒ले उ॒भे । ब॒हु॒ले इति॑ बहु॒ले । उ॒भे इत्यु॒भे ॥ व्यच॑स्वती॒ सम् । व्यच॑स्वती॒ इति॒ व्यच॑स्वती । सं ॅव॑साथाम् । व॒सा॒था॒म् भ॒र्तम् । भ॒र्तम॒ग्निम् । अ॒ग्निम् पु॑री॒ष्य᳚म् । पु॒री॒ष्य॑मिति॑ पुरी॒ष्य᳚म् ॥ सं ॅव॑साथाम् । व॒सा॒थाꣳ॒॒ सु॒व॒र्विदा᳚ । सु॒व॒र्विदा॑ स॒मीची᳚ । सु॒व॒र्विदेति॑ सुवः - विदा᳚ । स॒मीची॒ उर॑सा । स॒मीची॒ इति॑ स॒मीची᳚ । उर॑सा॒ त्मना᳚ । त्मनेति॒त्मना᳚ ॥ अ॒ग्निम॒न्तः । अ॒न्तर् भ॑रि॒ष्यन्ती᳚ । भ॒रि॒ष्यन्ती॒ ज्योति॑ष्मन्तम् । भ॒रि॒ष्यन्ती॒ इति॑ भरि॒ष्यन्ती᳚ । ज्योति॑ष्मन्त॒मज॑स्रम् । अज॑स्र॒मित् । इदितीत् ॥ पु॒री॒ष्यो॑ऽसि । अ॒सि॒ वि॒श्वभ॑राः । वि॒श्वभ॑रा॒ इति॑ वि॒श्व - भ॒राः॒ ॥ अथ॑र्वा त्वा । त्वा॒ प्र॒थ॒मः । प्र॒थ॒मो निः । निर॑मन्थत् । अ॒म॒न्थ॒द॒ग्ने॒ । अ॒ग्न॒ इत्य॑ग्ने ॥ त्वाम॑ग्ने । अ॒ग्ने पुष्क॑रात् । पुष्क॑रा॒दधि॑ । अद्ध्यथ॑र्वा । अथ॑र्वा॒ निः । निर॑मन्थत । अ॒म॒न्थ॒तेत्य॑मन्थत ॥ मू॒र्द्ध्नो विश्व॑स्य । विश्व॑स्य वा॒घतः॑ । वा॒घत॒ इति॑ वा॒घतः॑ ॥ तमु॑ । उ॒ त्वा॒ । त्वा॒ द॒द्ध्यङ् । द॒ध्यङ्ङृषिः॑ । ऋषिः॑ पु॒त्रः । पु॒त्र ई॑धे । 
ई॒धे॒ अथ॑र्वणः \newline

\textbf{Jatai Paata} \newline

1. वर्म॑ च च॒ वर्म॒ वर्म॑ च । \newline
2. च॒ स्थः॒ स्थ॒ श्च॒ च॒ स्थः॒ । \newline
3. स्थो॒ अच्छि॑द्रे॒ अच्छि॑द्रे स्थः स्थो॒ अच्छि॑द्रे । \newline
4. अच्छि॑द्रे बहु॒ले ब॑हु॒ले अच्छि॑द्रे॒ अच्छि॑द्रे बहु॒ले । \newline
5. अच्छि॑द्रे॒ इत्यच्छि॑द्रे । \newline
6. ब॒हु॒ले उ॒भे उ॒भे ब॑हु॒ले ब॑हु॒ले उ॒भे । \newline
7. ब॒हु॒ले इति॑ बहु॒ले । \newline
8. उ॒भे इत्यु॒भे । \newline
9. व्यच॑स्वती॒ सꣳ सं ॅव्यच॑स्वती॒ व्यच॑स्वती॒ सम् । \newline
10. व्यच॑स्वती॒ इति॒ व्यच॑स्वती । \newline
11. सं ॅव॑साथां ॅवसाथाꣳ॒॒ सꣳ सं ॅव॑साथाम् । \newline
12. व॒सा॒था॒म् भ॒र्तम् भ॒र्तं ॅव॑साथां ॅवसाथाम् भ॒र्तम् । \newline
13. भ॒र्त म॒ग्नि म॒ग्निम् भ॒र्तम् भ॒र्त म॒ग्निम् । \newline
14. अ॒ग्निम् पु॑री॒ष्य॑म् पुरी॒ष्य॑ म॒ग्नि म॒ग्निम् पु॑री॒ष्य᳚म् । \newline
15. पु॒री॒ष्य॑मिति॑ पुरी॒ष्य᳚म् । \newline
16. सं ॅव॑साथां ॅवसाथाꣳ॒॒ सꣳ सं ॅव॑साथाम् । \newline
17. व॒सा॒थाꣳ॒॒ सु॒व॒र्विदा॑ सुव॒र्विदा॑ वसाथां ॅवसाथाꣳ सुव॒र्विदा᳚ । \newline
18. सु॒व॒र्विदा॑ स॒मीची॑ स॒मीची॑ सुव॒र्विदा॑ सुव॒र्विदा॑ स॒मीची᳚ । \newline
19. सु॒व॒र्विदेति॑ सुवः - विदा᳚ । \newline
20. स॒मीची॒ उर॒सोर॑सा स॒मीची॑ स॒मीची॒ उर॑सा । \newline
21. स॒मीची॒ इति॑ स॒मीची᳚ । \newline
22. उर॑सा॒ त्मना॒ त्मनो र॒सोर॑सा॒ त्मना᳚ । \newline
23. त्मनेति॒त्मना᳚ । \newline
24. अ॒ग्नि म॒न्त र॒न्त र॒ग्नि म॒ग्नि म॒न्तः । \newline
25. अ॒न्तर् भ॑रि॒ष्यन्ती॑ भरि॒ष्यन्ती॑ अ॒न्त र॒न्तर् भ॑रि॒ष्यन्ती᳚ । \newline
26. भ॒रि॒ष्यन्ती॒ ज्योति॑ष्मन्त॒म् ज्योति॑ष्मन्तम् भरि॒ष्यन्ती॑ भरि॒ष्यन्ती॒ ज्योति॑ष्मन्तम् । \newline
27. भ॒रि॒ष्यन्ती॒ इति॑ भरि॒ष्यन्ती᳚ । \newline
28. ज्योति॑ष्मन्त॒ मज॑स्र॒ मज॑स्र॒म् ज्योति॑ष्मन्त॒म् ज्योति॑ष्मन्त॒ मज॑स्रम् । \newline
29. अज॑स्र॒ मिदि दज॑स्र॒ मज॑स्र॒ मित् । \newline
30. इदितीत् । \newline
31. पु॒री॒ष्यो᳚ ऽस्यसि पुरी॒ष्यः॑ पुरी॒ष्यो॑ ऽसि । \newline
32. अ॒सि॒ वि॒श्वभ॑रा वि॒श्वभ॑रा अस्यसि वि॒श्वभ॑राः । \newline
33. वि॒श्वभ॑रा॒ इति॑ वि॒श्व - भ॒राः॒ । \newline
34. अथ॑र्वा त्वा॒ त्वा ऽथ॒र्वा ऽथ॑र्वा त्वा । \newline
35. त्वा॒ प्र॒थ॒मः प्र॑थ॒म स्त्वा᳚ त्वा प्रथ॒मः । \newline
36. प्र॒थ॒मो निर् णिष् प्र॑थ॒मः प्र॑थ॒मो निः । \newline
37. निर॑मन्थ दमन्थ॒न् निर् णिर॑मन्थत् । \newline
38. अ॒म॒न्थ॒ द॒ग्ने॒ अ॒ग्ने॒ ऽम॒न्थ॒ द॒म॒न्थ॒ द॒ग्ने॒ । \newline
39. अ॒ग्न॒ इत्य॑ग्ने । \newline
40. त्वा म॑ग्ने अग्ने॒ त्वाम् त्वा म॑ग्ने । \newline
41. अ॒ग्ने॒ पुष्क॑रा॒त् पुष्क॑रादग्ने अग्ने॒ पुष्क॑रात् । \newline
42. पुष्क॑रा॒ दध्यधि॒ पुष्क॑रा॒त् पुष्क॑रा॒ दधि॑ । \newline
43. अध्यथ॒र्वा ऽथ॒र्वा ऽध्यध्य थ॑र्वा । \newline
44. अथ॑र्वा॒ निर् णिरथ॒र्वा ऽथ॑र्वा॒ निः । \newline
45. निर॑मन्थता मन्थत॒ निर् णिर॑मन्थत । \newline
46. अ॒म॒न्थ॒तेत्य॑मन्थत । \newline
47. मू॒र्द्ध्नो विश्व॑स्य॒ विश्व॑स्य मू॒र्द्ध्नो मू॒र्द्ध्नो विश्व॑स्य । \newline
48. विश्व॑स्य वा॒घतो॑ वा॒घतो॒ विश्व॑स्य॒ विश्व॑स्य वा॒घतः॑ । \newline
49. वा॒घत॒ इति॑ वा॒घतः॑ । \newline
50. त मु॒ तम् त मु॑ । \newline
51. उ॒ त्वा॒ त्व॒ वु॒ त्वा॒ । \newline
52. त्वा॒ द॒द्ध्यङ् द॒द्ध्यङ् त्वा᳚ त्वा द॒द्ध्यङ् । \newline
53. द॒द्ध्यङ् ङृषि॒र्॒. ऋषि॑र् द॒द्ध्यङ् द॒द्ध्यङ् ङृषिः॑ । \newline
54. ऋषिः॑ पु॒त्रः पु॒त्र ऋषि॒र्॒. ऋषिः॑ पु॒त्रः । \newline
55. पु॒त्र ई॑ध ईधे पु॒त्रः पु॒त्र ई॑धे । \newline
56. ई॒धे॒ अथ॑र्वणो॒ अथ॑र्वण ईध ईधे॒ अथ॑र्वणः । \newline

\textbf{Ghana Paata } \newline

1. वर्म॑ च च॒ वर्म॒ वर्म॑ च स्थः स्थ श्च॒ वर्म॒ वर्म॑ च स्थः । \newline
2. च॒ स्थः॒ स्थ॒ श्च॒ च॒ स्थो॒ अच्छि॑द्रे॒ अच्छि॑द्रे स्थ श्च च स्थो॒ अच्छि॑द्रे । \newline
3. स्थो॒ अच्छि॑द्रे॒ अच्छि॑द्रे स्थः स्थो॒ अच्छि॑द्रे बहु॒ले ब॑हु॒ले अच्छि॑द्रे स्थः स्थो॒ अच्छि॑द्रे बहु॒ले । \newline
4. अच्छि॑द्रे बहु॒ले ब॑हु॒ले अच्छि॑द्रे॒ अच्छि॑द्रे बहु॒ले उ॒भे उ॒भे ब॑हु॒ले अच्छि॑द्रे॒ अच्छि॑द्रे बहु॒ले उ॒भे । \newline
5. अच्छि॑द्रे॒ इत्यच्छि॑द्रे । \newline
6. ब॒हु॒ले उ॒भे उ॒भे ब॑हु॒ले ब॑हु॒ले उ॒भे । \newline
7. ब॒हु॒ले इति॑ बहु॒ले । \newline
8. उ॒भे इत्यु॒भे । \newline
9. व्यच॑स्वती॒ सꣳ सं ॅव्यच॑स्वती॒ व्यच॑स्वती॒ सं ॅव॑साथां ॅवसाथाꣳ॒॒ सं ॅव्यच॑स्वती॒ व्यच॑स्वती॒ सं ॅव॑साथाम् । \newline
10. व्यच॑स्वती॒ इति॒ व्यच॑स्वती । \newline
11. सं ॅव॑साथां ॅवसाथाꣳ॒॒ सꣳ सं ॅव॑साथाम् भ॒र्तम् भ॒र्तं ॅव॑साथाꣳ॒॒ सꣳ सं ॅव॑साथाम् भ॒र्तम् । \newline
12. व॒सा॒था॒म् भ॒र्तम् भ॒र्तं ॅव॑साथां ॅवसाथाम् भ॒र्त म॒ग्नि म॒ग्निम् भ॒र्तं ॅव॑साथां ॅवसाथाम् भ॒र्त म॒ग्निम् । \newline
13. भ॒र्त म॒ग्नि म॒ग्निम् भ॒र्तम् भ॒र्त म॒ग्निम् पु॑री॒ष्य॑म् पुरी॒ष्य॑ म॒ग्निम् भ॒र्तम् भ॒र्त म॒ग्निम् पु॑री॒ष्य᳚म् । \newline
14. अ॒ग्निम् पु॑री॒ष्य॑म् पुरी॒ष्य॑ म॒ग्नि म॒ग्निम् पु॑री॒ष्य᳚म् । \newline
15. पु॒री॒ष्य॑मिति॑ पुरी॒ष्य᳚म् । \newline
16. सं ॅव॑साथां ॅवसाथाꣳ॒॒ सꣳ सं ॅव॑साथाꣳ सुव॒र्विदा॑ सुव॒र्विदा॑ वसाथाꣳ॒॒ सꣳ सं ॅव॑साथाꣳ सुव॒र्विदा᳚ । \newline
17. व॒सा॒थाꣳ॒॒ सु॒व॒र्विदा॑ सुव॒र्विदा॑ वसाथां ॅवसाथाꣳ सुव॒र्विदा॑ स॒मीची॑ स॒मीची॑ सुव॒र्विदा॑ वसाथां ॅवसाथाꣳ सुव॒र्विदा॑ स॒मीची᳚ । \newline
18. सु॒व॒र्विदा॑ स॒मीची॑ स॒मीची॑ सुव॒र्विदा॑ सुव॒र्विदा॑ स॒मीची॒ उर॒सोर॑सा स॒मीची॑ सुव॒र्विदा॑ सुव॒र्विदा॑ स॒मीची॒ उर॑सा । \newline
19. सु॒व॒र्विदेति॑ सुवः - विदा᳚ । \newline
20. स॒मीची॒ उर॒सोर॑सा स॒मीची॑ स॒मीची॒ उर॑सा॒ त्मना॒ त्मनोर॑सा स॒मीची॑ स॒मीची॒ उर॑सा॒ त्मना᳚ । \newline
21. स॒मीची॒ इति॑ स॒मीची᳚ । \newline
22. उर॑सा॒ त्मना॒ त्मनोर॒सो र॑सा॒ त्मना᳚ । \newline
23. त्मनेति॒त्मना᳚ । \newline
24. अ॒ग्नि म॒न्त र॒न्त र॒ग्नि म॒ग्नि म॒न्तर् भ॑रि॒ष्यन्ती॑ भरि॒ष्यन्ती॑ अ॒न्त र॒ग्नि म॒ग्नि म॒न्तर् भ॑रि॒ष्यन्ती᳚ । \newline
25. अ॒न्तर् भ॑रि॒ष्यन्ती॑ भरि॒ष्यन्ती॑ अ॒न्त र॒न्तर् भ॑रि॒ष्यन्ती॒ ज्योति॑ष्मन्त॒म् ज्योति॑ष्मन्तम् भरि॒ष्यन्ती॑ अ॒न्त र॒न्तर् भ॑रि॒ष्यन्ती॒ ज्योति॑ष्मन्तम् । \newline
26. भ॒रि॒ष्यन्ती॒ ज्योति॑ष्मन्त॒म् ज्योति॑ष्मन्तम् भरि॒ष्यन्ती॑ भरि॒ष्यन्ती॒ ज्योति॑ष्मन्त॒ मज॑स्र॒ मज॑स्र॒म् ज्योति॑ष्मन्तम् भरि॒ष्यन्ती॑ भरि॒ष्यन्ती॒ ज्योति॑ष्मन्त॒ मज॑स्रम् । \newline
27. भ॒रि॒ष्यन्ती॒ इति॑ भरि॒ष्यन्ती᳚ । \newline
28. ज्योति॑ष्मन्त॒ मज॑स्र॒ मज॑स्र॒म् ज्योति॑ष्मन्त॒म् ज्योति॑ष्मन्त॒ मज॑स्र॒ मिदि दज॑स्र॒म् ज्योति॑ष्मन्त॒म् ज्योति॑ष्मन्त॒ मज॑स्र॒ मित् । \newline
29. अज॑स्र॒ मिदि दज॑स्र॒ मज॑स्र॒ मित् । \newline
30. इदितीत् । \newline
31. पु॒री॒ष्यो᳚ ऽस्यसि पुरी॒ष्यः॑ पुरी॒ष्यो॑ ऽसि वि॒श्वभ॑रा वि॒श्वभ॑रा असि पुरी॒ष्यः॑ पुरी॒ष्यो॑ ऽसि वि॒श्वभ॑राः । \newline
32. अ॒सि॒ वि॒श्वभ॑रा वि॒श्वभ॑रा अस्यसि वि॒श्वभ॑राः । \newline
33. वि॒श्वभ॑रा॒ इति॑ वि॒श्व - भ॒राः॒ । \newline
34. अथ॑र्वा त्वा॒ त्वा ऽथ॒र्वा ऽथ॑र्वा त्वा प्रथ॒मः प्र॑थ॒म स्त्वा ऽथ॒र्वा ऽथ॑र्वा त्वा प्रथ॒मः । \newline
35. त्वा॒ प्र॒थ॒मः प्र॑थ॒म स्त्वा᳚ त्वा प्रथ॒मो निर् णिष् प्र॑थ॒म स्त्वा᳚ त्वा प्रथ॒मो निः । \newline
36. प्र॒थ॒मो निर् णिष् प्र॑थ॒मः प्र॑थ॒मो निर॑मन्थ दमन्थ॒न् निष् प्र॑थ॒मः प्र॑थ॒मो निर॑मन्थत् । \newline
37. निर॑मन्थ दमन्थ॒न् निर् णिर॑मन्थ दग्ने अग्ने ऽमन्थ॒न् निर् णिर॑मन्थ दग्ने । \newline
38. अ॒म॒न्थ॒ द॒ग्ने॒ अ॒ग्ने॒ ऽम॒न्थ॒ द॒म॒न्थ॒ द॒ग्ने॒ । \newline
39. अ॒ग्न॒ इत्य॑ग्ने । \newline
40. त्वा म॑ग्ने अग्ने॒ त्वाम् त्वा म॑ग्ने॒ पुष्क॑रा॒त् पुष्क॑रा दग्ने॒ त्वाम् त्वा म॑ग्ने॒ पुष्क॑रात् । \newline
41. अ॒ग्ने॒ पुष्क॑रा॒त् पुष्क॑रा दग्ने अग्ने॒ पुष्क॑रा॒ दध्यधि॒ पुष्क॑रा दग्ने अग्ने॒ पुष्क॑रा॒ दधि॑ । \newline
42. पुष्क॑रा॒ दध्यधि॒ पुष्क॑रा॒त् पुष्क॑रा॒ दध्यथ॒र्वा ऽथ॒र्वा ऽधि॒ पुष्क॑रा॒त् पुष्क॑रा॒ दध्यथ॑र्वा । \newline
43. अध्यथ॒र्वा ऽथ॒र्वा ऽध्यध्य थ॑र्वा॒ निर् णिरथ॒र्वा ऽध्यध्य थ॑र्वा॒ निः । \newline
44. अथ॑र्वा॒ निर् णिरथ॒र्वा ऽथ॑र्वा॒ निर॑मन्थता मन्थत॒ निरथ॒र्वा ऽथ॑र्वा॒ निर॑मन्थत । \newline
45. निर॑मन्थता मन्थत॒ निर् णिर॑मन्थत । \newline
46. अ॒म॒न्थ॒तेत्य॑मन्थत । \newline
47. मू॒र्द्ध्नो विश्व॑स्य॒ विश्व॑स्य मू॒र्द्ध्नो मू॒र्द्ध्नो विश्व॑स्य वा॒घतो॑ वा॒घतो॒ विश्व॑स्य मू॒र्द्ध्नो मू॒र्द्ध्नो विश्व॑स्य वा॒घतः॑ । \newline
48. विश्व॑स्य वा॒घतो॑ वा॒घतो॒ विश्व॑स्य॒ विश्व॑स्य वा॒घतः॑ । \newline
49. वा॒घत॒ इति॑ वा॒घतः॑ । \newline
50. त मु॑ वु॒ तम् त मु॑ त्वा त्वो॒ तम् तमु॑ त्वा । \newline
51. उ॒ त्वा॒ त्व॒ वु॒ त्वा॒ द॒द्ध्यङ् द॒द्ध्यङ् त्व॑ वु त्वा द॒द्ध्यङ् । \newline
52. त्वा॒ द॒द्ध्यङ् द॒द्ध्यङ् त्वा᳚ त्वा द॒द्ध्यङ् ङृषि॒र्॒. ऋषि॑र् द॒द्ध्यङ् त्वा᳚ त्वा द॒द्ध्यङ् ङृषिः॑ । \newline
53. द॒द्ध्यङ् ङृषि॒र्॒. ऋषि॑र् द॒द्ध्यङ् द॒द्ध्यङ् ङृषिः॑ पु॒त्रः पु॒त्र ऋषि॑र् द॒द्ध्यङ् 
द॒द्ध्यङ् ङृषिः॑ पु॒त्रः । \newline
54. ऋषिः॑ पु॒त्रः पु॒त्र ऋषि॒र्॒. ऋषिः॑ पु॒त्र ई॑ध ईधे पु॒त्र ऋषि॒र्॒. ऋषिः॑ पु॒त्र ई॑धे । \newline
55. पु॒त्र ई॑ध ईधे पु॒त्रः पु॒त्र ई॑धे॒ अथ॑र्वणो॒ अथ॑र्वण ईधे पु॒त्रः पु॒त्र ई॑धे॒ अथ॑र्वणः । \newline
56. ई॒धे॒ अथ॑र्वणो॒ अथ॑र्वण ईध ईधे॒ अथ॑र्वणः । \newline
\pagebreak
\markright{ TS 4.1.3.3  \hfill https://www.vedavms.in \hfill}

\section{ TS 4.1.3.3 }

\textbf{TS 4.1.3.3 } \newline
\textbf{Samhita Paata} \newline

अथ॑र्वणः । वृ॒त्र॒हणं॑ पुरन्द॒रं ॥ तमु॑ त्वा पा॒थ्यो वृषा॒ समी॑धे दस्यु॒हन्त॑मं । ध॒न॒ञ्ज॒यꣳ रणे॑रणे ॥ सीद॑ होतः॒ स्व उ॑ लो॒के चि॑कि॒त्वान्थ् सा॒दया॑ य॒ज्ञ्ꣳ सु॑कृ॒तस्य॒ योनौ᳚ । दे॒वा॒वीर्दे॒वान्. ह॒विषा॑ यजा॒स्यग्ने॑ बृ॒हद्-यज॑माने॒ वयो॑ धाः ॥ नि होता॑ होतृ॒षद॑ने॒ विदा॑नस्त्वे॒षो दी॑दि॒वाꣳ अ॑सदथ् सु॒दक्षः॑ । अद॑ब्धव्रत प्रमति॒र्वसि॑ष्ठः सहस्रं भ॒रः शुचि॑जिह्वो अ॒ग्निः ॥ सꣳ सी॑दस्व म॒हाꣳ अ॑सि॒ शोच॑स्व- [  ] \newline

\textbf{Pada Paata} \newline

अथ॑र्वणः ॥ वृ॒त्र॒हण॒मिति॑ वृत्र - हन᳚म् । पु॒र॒न्द॒रमिति॑ पुरं - द॒रम् ॥ तम् । उ॒ । त्वा॒ । पा॒थ्यः । वृषा᳚ । समिति॑ । ई॒धे॒ । द॒स्यु॒हन्त॑म॒मिति॑ दस्यु-हन्त॑मम् ॥ ध॒न॒ञ्ज॒यमिति॑ धनं - ज॒यम् । रणे॑रण॒ इति॒ रणे᳚-र॒णे॒ ॥ सीद॑ । हो॒तः॒ । स्वे । उ॒ । लो॒के । चि॒कि॒त्वान् । सा॒दय॑ । य॒ज्ञ्म् । सु॒कृ॒तस्येति॑ सु-कृ॒तस्य॑ । योनौ᳚ ॥ दे॒वा॒वीरिति॑ देव-अ॒वीः । दे॒वान् । ह॒विषा᳚ । य॒जा॒सि॒ । अग्ने᳚ । बृ॒हत् । यज॑माने । वयः॑ । धाः॒ ॥ नीति॑ । होता᳚ । हो॒तृ॒षद॑न॒ इति॑ होतृ-सद॑ने । विदा॑नः । त्वे॒षः । दी॒दि॒वान् । अ॒स॒द॒त् । सु॒दक्ष॒ इति॑ सु - दक्षः॑ ॥ अद॑ब्धव्रत प्रमति॒रित्यद॑ब्धव्रत-प्र॒म॒तिः॒ । वसि॑ष्ठः । स॒ह॒स्र॒भं॒र इति॑ सहस्रं-भ॒रः । शुचि॑जिह्व॒ इति॒ शुचि॑ - जि॒ह्वः॒ । अ॒ग्निः ॥ समिति॑ । सी॒द॒स्व॒ । म॒हान् । अ॒सि॒ । शोच॑स्व ।  \newline


\textbf{Krama Paata} \newline

अथ॑र्वण॒ इत्यथ॑र्वणः ॥ वृ॒त्र॒हण॑म् पुरन्द॒रम् । वृ॒त्र॒हण॒मिति॑ वृत्र - हन᳚म् । पु॒र॒न्द॒रमिति॑ पुरम् - द॒रम् ॥ तमु॑ । उ॒ त्वा॒ । त्वा॒ पा॒थ्यः । पा॒थ्यो वृषा᳚ । वृषा॒ सम् । समी॑धे । ई॒धे॒ द॒स्यु॒हन्त॑मम् । द॒स्यु॒हन्त॑म॒मिति॑ दस्यु - हन्त॑मम् ॥ ध॒न॒ञ्ज॒यꣳ रणे॑रणे । ध॒न॒ञ्ज॒यमिति॑ धनम् - ज॒यम् । रणे॑रण॒ इति॒ रणे᳚ - र॒णे॒ ॥ सीद॑ होतः । हो॒तः॒ स्वे । स्व उ॑ । उ॒ लो॒के । लो॒के चि॑कि॒त्वान् । चि॒कि॒त्वान्थ् सा॒दय॑ । सा॒दया॑ य॒ज्ञ्म् । य॒ज्ञ्ꣳ सु॑कृ॒तस्य॑ । सु॒कृ॒तस्य॒ योनौ᳚ । सु॒कृ॒तस्येति॑ सु - कृ॒तस्य॑ । योना॒विति॒ योनौ᳚ ॥ दे॒वा॒वीर् दे॒वान् । दे॒वा॒वीरिति॑ देव - अ॒वीः । दे॒वान्. ह॒विषा᳚ । ह॒विषा॑ यजासि । य॒जा॒स्यग्ने᳚ । अग्ने॑ बृ॒हत् । बृ॒हद् यज॑माने । यज॑माने॒ वयः॑ । वयो॑ धाः । धा॒ इति॑ धाः ॥ नि होता᳚ । होता॑ होतृ॒षद॑ने । हो॒तृ॒षद॑ने॒ विदा॑नः । हो॒तृ॒षद॑न॒ इति॑ होतृ - सद॑ने । विदा॑नस्त्वे॒षः । त्वे॒षो दी॑दि॒वान् । दी॒दि॒वाꣳ अ॑सदत् । अ॒स॒द॒थ् सु॒दक्षः॑ । 
सु॒दक्ष॒ इति॑ सु - दक्षः॑ ॥ अद॑ब्धव्रतप्रमति॒र् वसि॑ष्ठः । अद॑ब्धव्रतप्रमति॒रित्यद॑ब्धव्रत - प्र॒म॒तिः॒ । वसि॑ष्ठः सहस्रम्भ॒रः । स॒ह॒स्र॒म्भ॒रः शुचि॑जिह्वः । स॒ह॒स्र॒म्भ॒र इति॑ सहस्रम् - भ॒रः । शुचि॑जिह्वो अ॒ग्निः । शुचि॑जिह्व॒ इति॒ शुचि॑ - जि॒ह्वः॒ । अ॒ग्निरित्य॒ग्निः ॥ सꣳ सी॑दस्व । सी॒द॒स्व॒ म॒हान् । म॒हाꣳ अ॑सि । अ॒सि॒ शोच॑स्व ( ) । शोच॑स्व देव॒वीत॑मः \newline

\textbf{Jatai Paata} \newline

1. अथ॑र्वण॒ इत्यथ॑र्वणः । \newline
2. वृ॒त्र॒हण॑म् पुरन्द॒रम् पु॑रन्द॒रं ॅवृ॑त्र॒हणं॑ ॅवृत्र॒हण॑म् पुरन्द॒रम् । \newline
3. वृ॒त्र॒हण॒मिति॑ वृत्र - हन᳚म् । \newline
4. पु॒र॒न्द॒रमिति॑ पुरं - द॒रम् । \newline
5. त मु॒ तम् त मु॑ । \newline
6. उ॒ त्वा॒ त्व॒ वु॒ त्वा॒ । \newline
7. त्वा॒ पा॒थ्यः पा॒थ्य स्त्वा᳚ त्वा पा॒थ्यः । \newline
8. पा॒थ्यो वृषा॒ वृषा॑ पा॒थ्यः पा॒थ्यो वृषा᳚ । \newline
9. वृषा॒ सꣳ सं ॅवृषा॒ वृषा॒ सम् । \newline
10. स मी॑ध ईधे॒ सꣳ स मी॑धे । \newline
11. ई॒धे॒ द॒स्यु॒हन्त॑मम् दस्यु॒हन्त॑म मीध ईधे दस्यु॒हन्त॑मम् । \newline
12. द॒स्यु॒हन्त॑म॒मिति॑ दस्यु - हन्त॑मम् । \newline
13. ध॒न॒ञ्ज॒यꣳ रणे॑रणे॒ रणे॑रणे धनञ्ज॒यम् ध॑नञ्ज॒यꣳ रणे॑रणे । \newline
14. ध॒न॒ञ्ज॒यमिति॑ धनं - ज॒यम् । \newline
15. रणे॑रण॒ इति॒ रणे᳚ - र॒णे॒ । \newline
16. सीद॑ होतर् होतः॒ सीद॒ सीद॑ होतः । \newline
17. हो॒तः॒ स्वे स्वे हो॑तो होतः॒ स्वे । \newline
18. स्व उ॑ वु॒ स्वे स्व उ॑ । \newline
19. उ॒ लो॒के लो॒क उ॑ वु लो॒के । \newline
20. लो॒के चि॑कि॒त्वाꣳ श्चि॑कि॒त्वान् ॅलो॒के लो॒के चि॑कि॒त्वान् । \newline
21. चि॒कि॒त्वान् थ्सा॒दय॑ सा॒दय॑ चिकि॒त्वाꣳ श्चि॑कि॒त्वान् थ्सा॒दय॑ । \newline
22. सा॒दया॑ य॒ज्ञ्ं ॅय॒ज्ञ्ꣳ सा॒दय॑ सा॒दया॑ य॒ज्ञ्म् । \newline
23. य॒ज्ञ्ꣳ सु॑कृ॒तस्य॑ सुकृ॒तस्य॑ य॒ज्ञ्ं ॅय॒ज्ञ्ꣳ सु॑कृ॒तस्य॑ । \newline
24. सु॒कृ॒तस्य॒ योनौ॒ योनौ॑ सुकृ॒तस्य॑ सुकृ॒तस्य॒ योनौ᳚ । \newline
25. सु॒कृ॒तस्येति॑ सु - कृ॒तस्य॑ । \newline
26. योना॒विति॒ योनौ᳚ । \newline
27. दे॒वा॒वीर् दे॒वान् दे॒वान् दे॑वा॒वीर् दे॑वा॒वीर् दे॒वान् । \newline
28. दे॒वा॒वीरिति॑ देव - अ॒वीः । \newline
29. दे॒वान्. ह॒विषा॑ ह॒विषा॑ दे॒वान् दे॒वान्. ह॒विषा᳚ । \newline
30. ह॒विषा॑ यजासि यजासि ह॒विषा॑ ह॒विषा॑ यजासि । \newline
31. य॒जा॒स्यग्ने ऽग्ने॑ यजासि यजा॒स्यग्ने᳚ । \newline
32. अग्ने॑ बृ॒हद् बृ॒ह दग्ने ऽग्ने॑ बृ॒हत् । \newline
33. बृ॒हद् यज॑माने॒ यज॑माने बृ॒हद् बृ॒हद् यज॑माने । \newline
34. यज॑माने॒ वयो॒ वयो॒ यज॑माने॒ यज॑माने॒ वयः॑ । \newline
35. वयो॑ धा धा॒ वयो॒ वयो॑ धाः । \newline
36. धा॒ इति॑ धाः । \newline
37. नि होता॒ होता॒ नि नि होता᳚ । \newline
38. होता॑ होतृ॒षद॑ने होतृ॒षद॑ने॒ होता॒ होता॑ होतृ॒षद॑ने । \newline
39. हो॒तृ॒षद॑ने॒ विदा॑नो॒ विदा॑नो होतृ॒षद॑ने होतृ॒षद॑ने॒ विदा॑नः । \newline
40. हो॒तृ॒षद॑न॒ इति॑ होतृ - सद॑ने । \newline
41. विदा॑न स्त्वे॒ष स्त्वे॒षो विदा॑नो॒ विदा॑न स्त्वे॒षः । \newline
42. त्वे॒षो दी॑दि॒वान् दी॑दि॒वान् त्वे॒ष स्त्वे॒षो दी॑दि॒वान् । \newline
43. दी॒दि॒वाꣳ अ॑सद दसदद् दीदि॒वान् दी॑दि॒वाꣳ अ॑सदत् । \newline
44. अ॒स॒द॒थ् सु॒दक्षः॑ सु॒दक्षो॑ असद दसदथ् सु॒दक्षः॑ । \newline
45. सु॒दक्ष॒ इति॑ सु - दक्षः॑ । \newline
46. अद॑ब्धव्रतप्रमति॒र् वसि॑ष्ठो॒ वसि॑ष्ठो॒ अद॑ब्धव्रतप्रमति॒ रद॑ब्धव्रतप्रमति॒र् वसि॑ष्ठः । \newline
47. अद॑ब्धव्रत प्रमति॒रित्यद॑ब्धव्रत - प्र॒म॒तिः॒ । \newline
48. वसि॑ष्ठः सहस्रंभ॒रः स॑हस्रंभ॒रो वसि॑ष्ठो॒ वसि॑ष्ठः सहस्रंभ॒रः । \newline
49. स॒ह॒स्रं॒भ॒रः शुचि॑जिह्वः॒ शुचि॑जिह्वः सहस्रंभ॒रः स॑हस्रंभ॒रः शुचि॑जिह्वः । \newline
50. स॒ह॒स्रं॒भ॒र इति॑ सहस्रं - भ॒रः । \newline
51. शुचि॑जिह्वो अ॒ग्नि र॒ग्निः शुचि॑जिह्वः॒ शुचि॑जिह्वो अ॒ग्निः । \newline
52. शुचि॑जिह्व॒ इति॒ शुचि॑ - जि॒ह्वः॒ । \newline
53. अ॒ग्निरित्य॒ग्निः । \newline
54. सꣳ सी॑दस्व सीदस्व॒ सꣳ सꣳ सी॑दस्व । \newline
55. सी॒द॒स्व॒ म॒हान् म॒हान् थ्सी॑दस्व सीदस्व म॒हान् । \newline
56. म॒हाꣳ अ॑स्यसि म॒हान् म॒हाꣳ अ॑सि । \newline
57. अ॒सि॒ शोच॑स्व॒ शोच॑स्वा स्यसि॒ शोच॑स्व । \newline
58. शोच॑स्व देव॒वीत॑मो देव॒वीत॑मः॒ शोच॑स्व॒ शोच॑स्व देव॒वीत॑मः । \newline

\textbf{Ghana Paata } \newline

1. अथ॑र्वण॒ इत्यथ॑र्वणः । \newline
2. वृ॒त्र॒हण॑म् पुरन्द॒रम् पु॑रन्द॒रं ॅवृ॑त्र॒हणं॑ ॅवृत्र॒हण॑म् पुरन्द॒रम् । \newline
3. वृ॒त्र॒हण॒मिति॑ वृत्र - हन᳚म् । \newline
4. पु॒र॒न्द॒रमिति॑ पुरं - द॒रम् । \newline
5. त मु॑ वु॒ तम् त मु॑ त्वा त्वो॒ तम् तमु॑ त्वा । \newline
6. उ॒ त्वा॒ त्व॒ वु॒ त्वा॒ पा॒थ्यः पा॒थ्य स्त्व॑ वु त्वा पा॒थ्यः । \newline
7. त्वा॒ पा॒थ्यः पा॒थ्य स्त्वा᳚ त्वा पा॒थ्यो वृषा॒ वृषा॑ पा॒थ्य स्त्वा᳚ त्वा पा॒थ्यो वृषा᳚ । \newline
8. पा॒थ्यो वृषा॒ वृषा॑ पा॒थ्यः पा॒थ्यो वृषा॒ सꣳ सं ॅवृषा॑ पा॒थ्यः पा॒थ्यो वृषा॒ सम् । \newline
9. वृषा॒ सꣳ सं ॅवृषा॒ वृषा॒ स मी॑ध ईधे॒ सं ॅवृषा॒ वृषा॒ स मी॑धे । \newline
10. समी॑ध ईधे॒ सꣳ स मी॑धे दस्यु॒हन्त॑मम् दस्यु॒हन्त॑म मीधे॒ सꣳ स मी॑धे दस्यु॒हन्त॑मम् । \newline
11. ई॒धे॒ द॒स्यु॒हन्त॑मम् दस्यु॒हन्त॑म मीध ईधे दस्यु॒हन्त॑मम् । \newline
12. द॒स्यु॒हन्त॑म॒मिति॑ दस्यु - हन्त॑मम् । \newline
13. ध॒न॒ञ्ज॒यꣳ रणे॑रणे॒ रणे॑रणे धनञ्ज॒यम् ध॑नञ्ज॒यꣳ रणे॑रणे । \newline
14. ध॒न॒ञ्ज॒यमिति॑ धनं - ज॒यम् । \newline
15. रणे॑रण॒ इति॒ रणे᳚ - र॒णे॒ । \newline
16. सीद॑ होतर् होतः॒ सीद॒ सीद॑ होतः॒ स्वे स्वे हो॑तः॒ सीद॒ सीद॑ होतः॒ स्वे । \newline
17. हो॒तः॒ स्वे स्वे हो॑तर् होतः॒ स्व उ॑ वु॒ स्वे हो॑तर् होतः॒ स्व उ॑ । \newline
18. स्व उ॑ वु॒ स्वे स्व उ॑ लो॒के लो॒क उ॒ स्वे स्व उ॑ लो॒के । \newline
19. उ॒ लो॒के लो॒क उ॑ वु लो॒के चि॑कि॒त्वाꣳ श्चि॑कि॒त्वान् ॅलो॒क उ॑ वु लो॒के चि॑कि॒त्वान् । \newline
20. लो॒के चि॑कि॒त्वाꣳ श्चि॑कि॒त्वान् ॅलो॒के लो॒के चि॑कि॒त्वान् थ्सा॒दय॑ सा॒दय॑ चिकि॒त्वान् ॅलो॒के लो॒के चि॑कि॒त्वान् थ्सा॒दय॑ । \newline
21. चि॒कि॒त्वान् थ्सा॒दय॑ सा॒दय॑ चिकि॒त्वाꣳ श्चि॑कि॒त्वान् थ्सा॒दया॑ य॒ज्ञ्ं ॅय॒ज्ञ्ꣳ सा॒दय॑ चिकि॒त्वाꣳ श्चि॑कि॒त्वान् थ्सा॒दया॑ य॒ज्ञ्म् । \newline
22. सा॒दया॑ य॒ज्ञ्ं ॅय॒ज्ञ्ꣳ सा॒दय॑ सा॒दया॑ य॒ज्ञ्ꣳ सु॑कृ॒तस्य॑ सुकृ॒तस्य॑ य॒ज्ञ्ꣳ सा॒दय॑ सा॒दया॑ य॒ज्ञ्ꣳ सु॑कृ॒तस्य॑ । \newline
23. य॒ज्ञ्ꣳ सु॑कृ॒तस्य॑ सुकृ॒तस्य॑ य॒ज्ञ्ं ॅय॒ज्ञ्ꣳ सु॑कृ॒तस्य॒ योनौ॒ योनौ॑ सुकृ॒तस्य॑ य॒ज्ञ्ं ॅय॒ज्ञ्ꣳ सु॑कृ॒तस्य॒ योनौ᳚ । \newline
24. सु॒कृ॒तस्य॒ योनौ॒ योनौ॑ सुकृ॒तस्य॑ सुकृ॒तस्य॒ योनौ᳚ । \newline
25. सु॒कृ॒तस्येति॑ सु - कृ॒तस्य॑ । \newline
26. योना॒विति॒ योनौ᳚ । \newline
27. दे॒वा॒वीर् दे॒वान् दे॒वान् दे॑वा॒वीर् दे॑वा॒वीर् दे॒वान्. ह॒विषा॑ ह॒विषा॑ दे॒वान् दे॑वा॒वीर् दे॑वा॒वीर् दे॒वान्. ह॒विषा᳚ । \newline
28. दे॒वा॒वीरिति॑ देव - अ॒वीः । \newline
29. दे॒वान्. ह॒विषा॑ ह॒विषा॑ दे॒वान् दे॒वान्. ह॒विषा॑ यजासि यजासि ह॒विषा॑ दे॒वान् दे॒वान्. ह॒विषा॑ यजासि । \newline
30. ह॒विषा॑ यजासि यजासि ह॒विषा॑ ह॒विषा॑ यजा॒स्यग्ने ऽग्ने॑ यजासि ह॒विषा॑ ह॒विषा॑ यजा॒स्यग्ने᳚ । \newline
31. य॒जा॒स्यग्ने ऽग्ने॑ यजासि यजा॒स्यग्ने॑ बृ॒हद् बृ॒हदग्ने॑ यजासि यजा॒स्यग्ने॑ बृ॒हत् । \newline
32. अग्ने॑ बृ॒हद् बृ॒हदग्ने ऽग्ने॑ बृ॒हद् यज॑माने॒ यज॑माने बृ॒हदग्ने ऽग्ने॑ बृ॒हद् यज॑माने । \newline
33. बृ॒हद् यज॑माने॒ यज॑माने बृ॒हद् बृ॒हद् यज॑माने॒ वयो॒ वयो॒ यज॑माने बृ॒हद् बृ॒हद् यज॑माने॒ वयः॑ । \newline
34. यज॑माने॒ वयो॒ वयो॒ यज॑माने॒ यज॑माने॒ वयो॑ धा धा॒ वयो॒ यज॑माने॒ यज॑माने॒ वयो॑ धाः । \newline
35. वयो॑ धा धा॒ वयो॒ वयो॑ धाः । \newline
36. धा॒ इति॑ धाः । \newline
37. नि होता॒ होता॒ नि नि होता॑ होतृ॒षद॑ने होतृ॒षद॑ने॒ होता॒ नि नि होता॑ होतृ॒षद॑ने । \newline
38. होता॑ होतृ॒षद॑ने होतृ॒षद॑ने॒ होता॒ होता॑ होतृ॒षद॑ने॒ विदा॑नो॒ विदा॑नो होतृ॒षद॑ने॒ होता॒ होता॑ होतृ॒षद॑ने॒ विदा॑नः । \newline
39. हो॒तृ॒षद॑ने॒ विदा॑नो॒ विदा॑नो होतृ॒षद॑ने होतृ॒षद॑ने॒ विदा॑न स्त्वे॒ष स्त्वे॒षो विदा॑नो होतृ॒षद॑ने होतृ॒षद॑ने॒ विदा॑न स्त्वे॒षः । \newline
40. हो॒तृ॒षद॑न॒ इति॑ होतृ - सद॑ने । \newline
41. विदा॑न स्त्वे॒ष स्त्वे॒षो विदा॑नो॒ विदा॑न स्त्वे॒षो दी॑दि॒वान् दी॑दि॒वान् त्वे॒षो विदा॑नो॒ विदा॑न स्त्वे॒षो दी॑दि॒वान् । \newline
42. त्वे॒षो दी॑दि॒वान् दी॑दि॒वान् त्वे॒ष स्त्वे॒षो दी॑दि॒वाꣳ अ॑सद दसदद् दीदि॒वान् त्वे॒ष स्त्वे॒षो दी॑दि॒वाꣳ अ॑सदत् । \newline
43. दी॒दि॒वाꣳ अ॑सद दसदद् दीदि॒वान् दी॑दि॒वाꣳ अ॑सदथ् सु॒दक्षः॑ सु॒दक्षो॑ असदद् दीदि॒वान् 
दी॑दि॒वाꣳ अ॑सदथ् सु॒दक्षः॑ । \newline
44. अ॒स॒द॒थ् सु॒दक्षः॑ सु॒दक्षो॑ असद दसदथ् सु॒दक्षः॑ । \newline
45. सु॒दक्ष॒ इति॑ सु - दक्षः॑ । \newline
46. अद॑ब्धव्रतप्रमति॒र् वसि॑ष्ठो॒ वसि॑ष्ठो॒ अद॑ब्धव्रतप्रमति॒ रद॑ब्धव्रतप्रमति॒र् वसि॑ष्ठः सहस्रंभ॒रः स॑हस्रंभ॒रो वसि॑ष्ठो॒ अद॑ब्धव्रतप्रमति॒ रद॑ब्धव्रतप्रमति॒र् वसि॑ष्ठः सहस्रंभ॒रः । \newline
47. अद॑ब्धव्रतप्रमति॒रित्यद॑ब्धव्रत - प्र॒म॒तिः॒ । \newline
48. वसि॑ष्ठः सहस्रंभ॒रः स॑हस्रंभ॒रो वसि॑ष्ठो॒ वसि॑ष्ठः सहस्रंभ॒रः शुचि॑जिह्वः॒ शुचि॑जिह्वः सहस्रंभ॒रो वसि॑ष्ठो॒ वसि॑ष्ठः सहस्रंभ॒रः शुचि॑जिह्वः । \newline
49. स॒ह॒स्रं॒भ॒रः शुचि॑जिह्वः॒ शुचि॑जिह्वः सहस्रंभ॒रः स॑हस्रंभ॒रः शुचि॑जिह्वो अ॒ग्नि र॒ग्निः शुचि॑जिह्वः सहस्रंभ॒रः स॑हस्रंभ॒रः शुचि॑जिह्वो अ॒ग्निः । \newline
50. स॒ह॒स्रं॒भ॒र इति॑ सहस्रं - भ॒रः । \newline
51. शुचि॑जिह्वो अ॒ग्नि र॒ग्निः शुचि॑जिह्वः॒ शुचि॑जिह्वो अ॒ग्निः । \newline
52. शुचि॑जिह्व॒ इति॒ शुचि॑ - जि॒ह्वः॒ । \newline
53. अ॒ग्निरित्य॒ग्निः । \newline
54. सꣳ सी॑दस्व सीदस्व॒ सꣳ सꣳ सी॑दस्व म॒हान् म॒हान् थ्सी॑दस्व॒ सꣳ सꣳ सी॑दस्व म॒हान् । \newline
55. सी॒द॒स्व॒ म॒हान् म॒हान् थ्सी॑दस्व सीदस्व म॒हाꣳ अ॑स्यसि म॒हान् थ्सी॑दस्व सीदस्व म॒हाꣳ अ॑सि । \newline
56. म॒हाꣳ अ॑स्यसि म॒हान् म॒हाꣳ अ॑सि॒ शोच॑स्व॒ शोच॑स्वासि म॒हान् म॒हाꣳ अ॑सि॒ शोच॑स्व । \newline
57. अ॒सि॒ शोच॑स्व॒ शोच॑स्वा स्यसि॒ शोच॑स्व देव॒वीत॑मो देव॒वीत॑मः॒ शोच॑स्वा स्यसि॒ शोच॑स्व देव॒वीत॑मः । \newline
58. शोच॑स्व देव॒वीत॑मो देव॒वीत॑मः॒ शोच॑स्व॒ शोच॑स्व देव॒वीत॑मः । \newline
\pagebreak
\markright{ TS 4.1.3.4  \hfill https://www.vedavms.in \hfill}

\section{ TS 4.1.3.4 }

\textbf{TS 4.1.3.4 } \newline
\textbf{Samhita Paata} \newline

देव॒वीत॑मः । वि धू॒मम॑ग्ने अरु॒षं मि॑येद्ध्य सृ॒ज प्र॑शस्त दर्.श॒तं ॥ जनि॑ष्वा॒ हि जेन्यो॒ अग्रे॒ अह्नाꣳ॑ हि॒तो हि॒तेष्व॑रु॒षो वने॑षु । दमे॑दमे स॒प्त रत्ना॒ दधा॑नो॒ऽग्निर्.होता॒ नि ष॑सादा॒ यजी॑यान् ॥ \newline

\textbf{Pada Paata} \newline

दे॒व॒वीत॑म॒ इति॑ देव - वीत॑मः ॥ वीति॑ । धू॒मम् । अ॒ग्ने॒ । अ॒रु॒षम् । मि॒ये॒द्ध्य॒ । सृ॒ज । प्र॒श॒स्तेति॑ प्र - श॒स्त॒ । द॒र्॒.श॒तम् ॥ जनि॑ष्व । हि । जेन्यः॑ । अग्रे᳚ । अह्ना᳚म् । हि॒तः । हि॒तेषु॑ । अ॒रु॒षः । वने॑षु ॥ दमे॑दम॒ इति॒ दमे᳚ - द॒मे॒ । स॒प्त । रत्ना᳚ । दधा॑नः । अ॒ग्निः । होता᳚ । नीति॑ । स॒सा॒द॒ । यजी॑यान् ॥  \newline


\textbf{Krama Paata} \newline

दे॒व॒वीत॑म॒ इति॑ देव - वीत॑मः ॥ वि धू॒मम् । धू॒मम॑ग्ने । अ॒ग्ने॒ अ॒रु॒षम् । अ॒रु॒षम् मि॑येद्ध्य । मि॒ये॒द्ध्य॒ सृ॒ज । सृ॒ज प्र॑शस्त । प्र॒श॒स्त॒ द॒र्॒.श॒तम् । प्र॒श॒स्तेति॑ प्र - श॒स्त॒ । द॒र्॒.श॒तमिति॑ दर्.श॒तम् ॥ जनि॑ष्वा॒ हि । 
हि जेन्यः॑ । जेन्यो॒ अग्रे᳚ । अग्रे॒ अह्ना᳚म् । अह्नाꣳ॑ हि॒तः । हि॒तो हि॒तेषु॑ । हि॒तेष्व॑रु॒षः । अ॒रु॒षो वने॑षु । वने॒ष्विति॒ वने॑षु ॥ दमे॑दमे स॒प्त । दमे॑दम॒ इति॒ दमे᳚ - द॒मे॒ । स॒प्त रत्ना᳚ । रत्ना॒ दधा॑नः । दधा॑नो॒ऽग्निः । अ॒ग्निर्. होता᳚ । होता॒ नि । नि ष॑साद । स॒सा॒दा॒ यजी॑यान् । यजी॑या॒निति॒ यजी॑यान् । \newline

\textbf{Jatai Paata} \newline

1. दे॒व॒वीत॑म॒ इति॑ देव - वीत॑मः । \newline
2. वि धू॒मम् धू॒मं ॅवि वि धू॒मम् । \newline
3. धू॒म म॑ग्ने अग्ने धू॒मम् धू॒म म॑ग्ने । \newline
4. अ॒ग्ने॒ अ॒रु॒ष म॑रु॒ष म॑ग्ने अग्ने अरु॒षम् । \newline
5. अ॒रु॒षम् मि॑येद्ध्य मियेद्ध्यारु॒ष म॑रु॒षम् मि॑येद्ध्य । \newline
6. मि॒ये॒द्ध्य॒ सृ॒ज सृ॒ज मि॑येद्ध्य मियेद्ध्य सृ॒ज । \newline
7. सृ॒ज प्र॑शस्त प्रशस्त सृ॒ज सृ॒ज प्र॑शस्त । \newline
8. प्र॒श॒स्त॒ द॒र्॒.श॒तम् द॑र्.श॒तम् प्र॑शस्त प्रशस्त दर्.श॒तम् । \newline
9. प्र॒श॒स्तेति॑ प्र - श॒स्त॒ । \newline
10. द॒र्॒.श॒तमिति॑ दर्.श॒तम् । \newline
11. जनि॑ष्वा॒ हि हि जनि॑ष्व॒ जनि॑ष्वा॒ हि । \newline
12. हि जेन्यो॒ जेन्यो॒ हि हि जेन्यः॑ । \newline
13. जेन्यो॒ अग्रे॒ अग्रे॒ जेन्यो॒ जेन्यो॒ अग्रे᳚ । \newline
14. अग्रे॒ अह्ना॒ मह्ना॒ मग्रे॒ अग्रे॒ अह्ना᳚म् । \newline
15. अह्नाꣳ॑ हि॒तो हि॒तो अह्ना॒ मह्नाꣳ॑ हि॒तः । \newline
16. हि॒तो हि॒तेषु॑ हि॒तेषु॑ हि॒तो हि॒तो हि॒तेषु॑ । \newline
17. हि॒ते ष्व॑रु॒षो अ॑रु॒षो हि॒तेषु॑ हि॒ते ष्व॑रु॒षः । \newline
18. अ॒रु॒षो वने॑षु॒ वने᳚ ष्वरु॒षो अ॑रु॒षो वने॑षु । \newline
19. वने॒ष्विति॒ वने॑षु । \newline
20. दमे॑दमे स॒प्त स॒प्त दमे॑दमे॒ दमे॑दमे स॒प्त । \newline
21. दमे॑दम॒ इति॒ दमे᳚ - द॒मे॒ । \newline
22. स॒प्त रत्ना॒ रत्ना॑ स॒प्त स॒प्त रत्ना᳚ । \newline
23. रत्ना॒ दधा॑नो॒ दधा॑नो॒ रत्ना॒ रत्ना॒ दधा॑नः । \newline
24. दधा॑नो॒ ऽग्नि र॒ग्निर् दधा॑नो॒ दधा॑नो॒ ऽग्निः । \newline
25. अ॒ग्निर्. होता॒ होता॒ ऽग्नि र॒ग्निर्. होता᳚ । \newline
26. होता॒ नि नि होता॒ होता॒ नि । \newline
27. नि ष॑साद ससाद॒ नि नि ष॑साद । \newline
28. स॒सा॒दा॒ यजी॑या॒न्॒. यजी॑यान् थ्ससाद ससादा॒ यजी॑यान् । \newline
29. यजी॑या॒निति॒ यजी॑यान् । \newline

\textbf{Ghana Paata } \newline

1. दे॒व॒वीत॑म॒ इति॑ देव - वीत॑मः । \newline
2. वि धू॒मम् धू॒मं ॅवि वि धू॒म म॑ग्ने अग्ने धू॒मं ॅवि वि धू॒म म॑ग्ने । \newline
3. धू॒म म॑ग्ने अग्ने धू॒मम् धू॒म म॑ग्ने अरु॒ष म॑रु॒ष म॑ग्ने धू॒मम् धू॒म म॑ग्ने अरु॒षम् । \newline
4. अ॒ग्ने॒ अ॒रु॒ष म॑रु॒ष म॑ग्ने अग्ने अरु॒षम् मि॑येद्ध्य मियेद्ध्या रु॒ष म॑ग्ने अग्ने अरु॒षम् मि॑येद्ध्य । \newline
5. अ॒रु॒षम् मि॑येद्ध्य मियेद्ध्या रु॒ष म॑रु॒षम् मि॑येद्ध्य सृ॒ज सृ॒ज मि॑येद्ध्या रु॒ष म॑रु॒षम् मि॑येद्ध्य सृ॒ज । \newline
6. मि॒ये॒द्ध्य॒ सृ॒ज सृ॒ज मि॑येद्ध्य मियेद्ध्य सृ॒ज प्र॑शस्त प्रशस्त सृ॒ज मि॑येद्ध्य मियेद्ध्य सृ॒ज प्र॑शस्त । \newline
7. सृ॒ज प्र॑शस्त प्रशस्त सृ॒ज सृ॒ज प्र॑शस्त दर्.श॒तम् द॑र्.श॒तम् प्र॑शस्त सृ॒ज सृ॒ज प्र॑शस्त दर्.श॒तम् । \newline
8. प्र॒श॒स्त॒ द॒र्॒.श॒तम् द॑र्.श॒तम् प्र॑शस्त प्रशस्त दर्.श॒तम् । \newline
9. प्र॒श॒स्तेति॑ प्र - श॒स्त॒ । \newline
10. द॒र्॒.श॒तमिति॑ दर्.श॒तम् । \newline
11. जनि॑ष्वा॒ हि हि जनि॑ष्व॒ जनि॑ष्वा॒ हि जेन्यो॒ जेन्यो॒ हि जनि॑ष्व॒ जनि॑ष्वा॒ हि जेन्यः॑ । \newline
12. हि जेन्यो॒ जेन्यो॒ हि हि जेन्यो॒ अग्रे॒ अग्रे॒ जेन्यो॒ हि हि जेन्यो॒ अग्रे᳚ । \newline
13. जेन्यो॒ अग्रे॒ अग्रे॒ जेन्यो॒ जेन्यो॒ अग्रे॒ अह्ना॒ मह्ना॒ मग्रे॒ जेन्यो॒ जेन्यो॒ अग्रे॒ अह्ना᳚म् । \newline
14. अग्रे॒ अह्ना॒ मह्ना॒ मग्रे॒ अग्रे॒ अह्नाꣳ॑ हि॒तो हि॒तो अह्ना॒ मग्रे॒ अग्रे॒ अह्नाꣳ॑ हि॒तः । \newline
15. अह्नाꣳ॑ हि॒तो हि॒तो अह्ना॒ मह्नाꣳ॑ हि॒तो हि॒तेषु॑ हि॒तेषु॑ हि॒तो अह्ना॒ मह्नाꣳ॑ हि॒तो हि॒तेषु॑ । \newline
16. हि॒तो हि॒तेषु॑ हि॒तेषु॑ हि॒तो हि॒तो हि॒ते ष्व॑रु॒षो अ॑रु॒षो हि॒तेषु॑ हि॒तो हि॒तो हि॒ते ष्व॑रु॒षः । \newline
17. हि॒ते ष्व॑रु॒षो अ॑रु॒षो हि॒तेषु॑ हि॒ते ष्व॑रु॒षो वने॑षु॒ वने᳚ ष्वरु॒षो हि॒तेषु॑ हि॒ते ष्व॑रु॒षो वने॑षु । \newline
18. अ॒रु॒षो वने॑षु॒ वने᳚ ष्वरु॒षो अ॑रु॒षो वने॑षु । \newline
19. वने॒ष्विति॒ वने॑षु । \newline
20. दमे॑दमे स॒प्त स॒प्त दमे॑दमे॒ दमे॑दमे स॒प्त रत्ना॒ रत्ना॑ स॒प्त दमे॑दमे॒ दमे॑दमे स॒प्त रत्ना᳚ । \newline
21. दमे॑दम॒ इति॒ दमे᳚ - द॒मे॒ । \newline
22. स॒प्त रत्ना॒ रत्ना॑ स॒प्त स॒प्त रत्ना॒ दधा॑नो॒ दधा॑नो॒ रत्ना॑ स॒प्त स॒प्त रत्ना॒ दधा॑नः । \newline
23. रत्ना॒ दधा॑नो॒ दधा॑नो॒ रत्ना॒ रत्ना॒ दधा॑नो॒ ऽग्नि र॒ग्निर् दधा॑नो॒ रत्ना॒ रत्ना॒ दधा॑नो॒ ऽग्निः । \newline
24. दधा॑नो॒ ऽग्नि र॒ग्निर् दधा॑नो॒ दधा॑नो॒ ऽग्निर्. होता॒ होता॒ ऽग्निर् दधा॑नो॒ दधा॑नो॒ ऽग्निर्. होता᳚ । \newline
25. अ॒ग्निर्. होता॒ होता॒ ऽग्नि र॒ग्निर्. होता॒ नि नि होता॒ ऽग्नि र॒ग्निर्. होता॒ नि । \newline
26. होता॒ नि नि होता॒ होता॒ नि ष॑साद ससाद॒ नि होता॒ होता॒ नि ष॑साद । \newline
27. नि ष॑साद ससाद॒ नि नि ष॑सादा॒ यजी॑या॒न्॒. यजी॑यान् थ्ससाद॒ नि नि ष॑सादा॒ यजी॑यान् । \newline
28. स॒सा॒दा॒ यजी॑या॒न्॒. यजी॑यान् थ्ससाद ससादा॒ यजी॑यान् । \newline
29. यजी॑या॒निति॒ यजी॑यान् । \newline
\pagebreak
\markright{ TS 4.1.4.1  \hfill https://www.vedavms.in \hfill}

\section{ TS 4.1.4.1 }

\textbf{TS 4.1.4.1 } \newline
\textbf{Samhita Paata} \newline

सं ते॑ वा॒युर्मा॑त॒रिश्वा॑ दधातूत्ता॒नायै॒ हृद॑यं॒ ॅयद्विलि॑ष्टं । दे॒वानां॒ ॅयश्चर॑ति प्रा॒णथे॑न॒ तस्मै॑ च देवि॒ वष॑डस्तु॒ तुभ्यं᳚ ॥ सुजा॑तो॒ ज्योति॑षा स॒ह शर्म॒ वरू॑थ॒माऽस॑दः॒ सुवः॑ । वासो॑ अग्ने वि॒श्वरू॑पꣳ॒॒ संॅव्य॑यस्व विभावसो ॥ उदु॑ तिष्ठ स्वद्ध्व॒रावा॑ नो दे॒व्या कृ॒पा । दृ॒शे च॑ भा॒सा बृ॑ह॒ता सु॑शु॒क्वनि॒राऽग्ने॑ याहि सुश॒स्तिभिः॑ ॥ \newline

\textbf{Pada Paata} \newline

समिति॑ । ते॒ । वा॒युः । मा॒त॒रिश्वा᳚ । द॒धा॒तु॒ । उ॒त्ता॒नाया॒ इत्यु॑त् - ता॒नायै᳚ । हृद॑यम् । यत् । विलि॑ष्ट॒मिति॒ वि - लि॒ष्ट॒म् ॥ दे॒वाना᳚म् । यः । चर॑ति । प्रा॒णथे॒नेति॑ प्र - अ॒नथे॑न । तस्मै᳚ । च॒ । दे॒वि॒ । वष॑ट् । अ॒स्तु॒ । तुभ्य᳚म् ॥ सुजा॑त॒ इति॒ सु-जा॒तः॒ । ज्योति॑षा । स॒ह । शर्म॑ । वरू॑थम् । एति॑ । अ॒स॒दः॒ । सुवः॑ ॥ वासः॑ । अ॒ग्ने॒ । वि॒श्वरू॑प॒मिति॑ वि॒श्व - रू॒प॒म् । समिति॑ । व्य॒य॒स्व॒ । वि॒भा॒व॒सो॒ इति॑ विभा-व॒सो॒ ॥ उदिति॑ । उ॒ । ति॒ष्ठ॒ । स्व॒द्ध्व॒रेति॑ सु - अ॒द्ध्व॒र॒ । अव॑ । नः॒ । दे॒व्या । कृ॒पा ॥ दृ॒शे । च॒ । भा॒सा । बृ॒ह॒ता । सु॒शु॒क्वनि॒रिति॑ सु - शु॒क्वनिः॑ । एति॑ । अ॒ग्ने॒ । या॒हि॒ । सु॒श॒स्तिभि॒रिति॑ सुश॒स्ति - भिः॒ ॥  \newline


\textbf{Krama Paata} \newline

सम् ते᳚ । ते॒ वा॒युः । वा॒युर् मा॑त॒रिश्वा᳚ । मा॒त॒रिश्वा॑ दधातु । द॒धा॒तू॒त्ता॒नायै᳚ । उ॒त्ता॒नायै॒ हृद॑यम् । उ॒त्ता॒नाया॒ इत्यु॑त् - ता॒नायै᳚ । हृद॑यं॒ ॅयत् । यद् विलि॑ष्टम् । विलि॑ष्ट॒मिति॒ वि - लि॒ष्ट॒म् ॥ दे॒वानां॒ ॅयः । यश्चर॑ति । चर॑ति प्रा॒णथे॑न । प्रा॒णथे॑न॒ तस्मै᳚ । प्रा॒णथे॒नेति॑ प्र - अ॒नथे॑न । तस्मै॑ च । च॒ दे॒वि॒ । दे॒वि॒ वष॑ट् । वष॑डस्तु । अ॒स्तु॒ तुभ्य᳚म् । तुभ्य॒मिति॒ तुभ्य᳚म् ॥ सुजा॑तो॒ ज्योति॑षा । सुजा॑त॒ इति॒ सु - जा॒तः॒ । ज्योति॑षा स॒ह । स॒ह शर्म॑ । शर्म॒ वरू॑थम् । वरू॑थ॒मा । आऽस॑दः । अ॒स॒दः॒ सुवः॑ । सुव॒रिति॒ सुवः॑ ॥ वासो॑ अग्ने । अ॒ग्ने॒ वि॒श्वरू॑पम् । वि॒श्वरू॑पꣳ॒॒ सम् । वि॒श्वरू॑प॒मिति॑ वि॒श्व - रू॒प॒म् । सं ॅव्य॑यस्व । व्य॒य॒स्व॒ वि॒भा॒व॒सो॒ । वि॒भा॒व॒सो॒ इति॑ विभा - व॒सो॒ ॥ उदु॑ । उ॒ ति॒ष्ठ॒ । ति॒ष्ठ॒ स्व॒द्ध्व॒र॒ । स्व॒द्ध्व॒राव॑ । स्व॒द्ध्व॒रेति॑ सु - अ॒द्ध्व॒र॒ । अवा॑ नः । नो॒ दे॒व्या । दे॒व्या कृ॒पा । कृ॒पेति॑ कृ॒पा ॥ दृ॒शे च॑ । च॒ भा॒सा । भा॒सा बृ॑ह॒ता । बृ॒ह॒ता सु॑शु॒क्वनिः॑ । सु॒शु॒क्वनि॒रा । सु॒शु॒क्वनि॒रिति॑ सु - शु॒क्वनिः॑ ।? आऽग्ने᳚ । अ॒ग्ने॒ या॒हि॒ । या॒हि॒ सु॒श॒स्तिभिः॑ । सु॒श॒स्तिभि॒रिति॑ सुश॒स्ति - भिः॒ । \newline

\textbf{Jatai Paata} \newline

1. सम् ते॑ ते॒ सꣳ सम् ते᳚ । \newline
2. ते॒ वा॒युर् वा॒यु स्ते॑ ते वा॒युः । \newline
3. वा॒युर् मा॑त॒रिश्वा॑ मात॒रिश्वा॑ वा॒युर् वा॒युर् मा॑त॒रिश्वा᳚ । \newline
4. मा॒त॒रिश्वा॑ दधातु दधातु मात॒रिश्वा॑ मात॒रिश्वा॑ दधातु । \newline
5. द॒धा॒तू॒ त्ता॒नाया॑ उत्ता॒नायै॑ दधातु दधातू त्ता॒नायै᳚ । \newline
6. उ॒त्ता॒नायै॒ हृद॑यꣳ॒॒ हृद॑य मुत्ता॒नाया॑ उत्ता॒नायै॒ हृद॑यम् । \newline
7. उ॒त्ता॒नाया॒ इत्यु॑त् - ता॒नायै᳚ । \newline
8. हृद॑यं॒ ॅयद् यद्धृद॑यꣳ॒॒ हृद॑यं॒ ॅयत् । \newline
9. यद् विलि॑ष्टं॒ ॅविलि॑ष्टं॒ ॅयद् यद् विलि॑ष्टम् । \newline
10. विलि॑ष्ट॒मिति॒ वि - लि॒ष्ट॒म् । \newline
11. दे॒वानां॒ ॅयो यो दे॒वाना᳚म् दे॒वानां॒ ॅयः । \newline
12. यश्चर॑ति॒ चर॑ति॒ यो यश्चर॑ति । \newline
13. चर॑ति प्रा॒णथे॑न प्रा॒णथे॑न॒ चर॑ति॒ चर॑ति प्रा॒णथे॑न । \newline
14. प्रा॒णथे॑न॒ तस्मै॒ तस्मै᳚ प्रा॒णथे॑न प्रा॒णथे॑न॒ तस्मै᳚ । \newline
15. प्रा॒णथे॒नेति॑ प्र - अ॒नथे॑न । \newline
16. तस्मै॑ च च॒ तस्मै॒ तस्मै॑ च । \newline
17. च॒ दे॒वि॒ दे॒वि॒ च॒ च॒ दे॒वि॒ । \newline
18. दे॒वि॒ वष॒ड् वष॑ड् देवि देवि॒ वष॑ट् । \newline
19. वष॑ डस्त्वस्तु॒ वष॒ड् वष॑डस्तु । \newline
20. अ॒स्तु॒ तुभ्य॒म् तुभ्य॑ मस्त्वस्तु॒ तुभ्य᳚म् । \newline
21. तुभ्य॒मिति॒ तुभ्य᳚म् । \newline
22. सुजा॑तो॒ ज्योति॑षा॒ ज्योति॑षा॒ सुजा॑तः॒ सुजा॑तो॒ ज्योति॑षा । \newline
23. सुजा॑त॒ इति॒ सु - जा॒तः॒ । \newline
24. ज्योति॑षा स॒ह स॒ह ज्योति॑षा॒ ज्योति॑षा स॒ह । \newline
25. स॒ह शर्म॒ शर्म॑ स॒ह स॒ह शर्म॑ । \newline
26. शर्म॒ वरू॑थं॒ ॅवरू॑थꣳ॒॒ शर्म॒ शर्म॒ वरू॑थम् । \newline
27. वरू॑थ॒ मा वरू॑थं॒ ॅवरू॑थ॒ मा । \newline
28. आ ऽस॑दो असद॒ आ ऽस॑दः । \newline
29. अ॒स॒दः॒ सुवः॒ सुव॑ रसदो असदः॒ सुवः॑ । \newline
30. सुव॒रिति॒ सुवः॑ । \newline
31. वासो॑ अग्ने अग्ने॒ वासो॒ वासो॑ अग्ने । \newline
32. अ॒ग्ने॒ वि॒श्वरू॑पं ॅवि॒श्वरू॑प मग्ने अग्ने वि॒श्वरू॑पम् । \newline
33. वि॒श्वरू॑पꣳ॒॒ सꣳ सं ॅवि॒श्वरू॑पं ॅवि॒श्वरू॑पꣳ॒॒ सम् । \newline
34. वि॒श्वरू॑प॒मिति॑ वि॒श्व - रू॒प॒म् । \newline
35. सं ॅव्य॑यस्व व्ययस्व॒ सꣳ सं ॅव्य॑यस्व । \newline
36. व्य॒य॒स्व॒ वि॒भा॒व॒सो॒ वि॒भा॒व॒सो॒ व्य॒य॒स्व॒ व्य॒य॒स्व॒ वि॒भा॒व॒सो॒ । \newline
37. वि॒भा॒व॒सो॒ इति॑ विभा - व॒सो॒ । \newline
38. उदु॑ वु॒ वुदुदु॑ । \newline
39. उ॒ ति॒ष्ठ॒ ति॒ष्ठ॒ वु॒ ति॒ष्ठ॒ । \newline
40. ति॒ष्ठ॒ स्व॒द्ध्व॒र॒ स्व॒द्ध्व॒र॒ ति॒ष्ठ॒ ति॒ष्ठ॒ स्व॒द्ध्व॒र॒ । \newline
41. स्व॒द्ध्व॒ रावाव॑ स्वद्ध्वर स्वद्ध्व॒ राव॑ । \newline
42. स्व॒द्ध्व॒रेति॑ सु - अ॒द्ध्व॒र॒ । \newline
43. अवा॑ नो॒ नो ऽवावा॑ नः । \newline
44. नो॒ दे॒व्या दे॒व्या नो॑ नो दे॒व्या । \newline
45. दे॒व्या कृ॒पा कृ॒पा दे॒व्या दे॒व्या कृ॒पा । \newline
46. कृ॒पेति॑ कृ॒पा । \newline
47. दृ॒शे च॑ च दृ॒शे दृ॒शे च॑ । \newline
48. च॒ भा॒सा भा॒सा च॑ च भा॒सा । \newline
49. भा॒सा बृ॑ह॒ता बृ॑ह॒ता भा॒सा भा॒सा बृ॑ह॒ता । \newline
50. बृ॒ह॒ता सु॑शु॒क्वनिः॑ सुशु॒क्वनि॑र् बृह॒ता बृ॑ह॒ता सु॑शु॒क्वनिः॑ । \newline
51. सु॒शु॒क्वनि॒रा सु॑शु॒क्वनिः॑ सुशु॒क्वनि॒रा । \newline
52. सु॒शु॒क्वनि॒रिति॑ सु - शु॒क्वनिः॑ । \newline
53. आ ऽग्ने॑ अग्न॒ आ ऽग्ने᳚ । \newline
54. अ॒ग्ने॒ या॒हि॒ या॒ह्य॒ग्ने॒ अ॒ग्ने॒ या॒हि॒ । \newline
55. या॒हि॒ सु॒श॒स्तिभिः॑ सुश॒स्तिभि॑र् याहि याहि सुश॒स्तिभिः॑ । \newline
56. सु॒श॒स्तिभि॒रिति॑ सुश॒स्ति - भिः॒ । \newline

\textbf{Ghana Paata } \newline

1. सम् ते॑ ते॒ सꣳ सम् ते॑ वा॒युर् वा॒यु स्ते॒ सꣳ सम् ते॑ वा॒युः । \newline
2. ते॒ वा॒युर् वा॒यु स्ते॑ ते वा॒युर् मा॑त॒रिश्वा॑ मात॒रिश्वा॑ वा॒यु स्ते॑ ते वा॒युर् मा॑त॒रिश्वा᳚ । \newline
3. वा॒युर् मा॑त॒रिश्वा॑ मात॒रिश्वा॑ वा॒युर् वा॒युर् मा॑त॒रिश्वा॑ दधातु दधातु मात॒रिश्वा॑ वा॒युर् वा॒युर् मा॑त॒रिश्वा॑ दधातु । \newline
4. मा॒त॒रिश्वा॑ दधातु दधातु मात॒रिश्वा॑ मात॒रिश्वा॑ दधा तूत्ता॒नाया॑ उत्ता॒नायै॑ दधातु मात॒रिश्वा॑ मात॒रिश्वा॑ दधा तूत्ता॒नायै᳚ । \newline
5. द॒धा॒ तू॒त्ता॒नाया॑ उत्ता॒नायै॑ दधातु दधा तूत्ता॒नायै॒ हृद॑यꣳ॒॒ हृद॑य मुत्ता॒नायै॑ दधातु दधा तूत्ता॒नायै॒ हृद॑यम् । \newline
6. उ॒त्ता॒नायै॒ हृद॑यꣳ॒॒ हृद॑य मुत्ता॒नाया॑ उत्ता॒नायै॒ हृद॑यं॒ ॅयद् यद्धृद॑य मुत्ता॒नाया॑ उत्ता॒नायै॒ हृद॑यं॒ ॅयत् । \newline
7. उ॒त्ता॒नाया॒ इत्यु॑त् - ता॒नायै᳚ । \newline
8. हृद॑यं॒ ॅयद् यद्धृद॑यꣳ॒॒ हृद॑यं॒ ॅयद् विलि॑ष्टं॒ ॅविलि॑ष्टं॒ ॅयद्धृद॑यꣳ॒॒ हृद॑यं॒ ॅयद् विलि॑ष्टम् । \newline
9. यद् विलि॑ष्टं॒ ॅविलि॑ष्टं॒ ॅयद् यद् विलि॑ष्टम् । \newline
10. विलि॑ष्ट॒मिति॒ वि - लि॒ष्ट॒म् । \newline
11. दे॒वानां॒ ॅयो यो दे॒वाना᳚म् दे॒वानां॒ ॅयश्चर॑ति॒ चर॑ति॒ यो दे॒वाना᳚म् दे॒वानां॒ ॅयश्चर॑ति । \newline
12. यश्चर॑ति॒ चर॑ति॒ यो यश्चर॑ति प्रा॒णथे॑न प्रा॒णथे॑न॒ चर॑ति॒ यो यश्चर॑ति प्रा॒णथे॑न । \newline
13. चर॑ति प्रा॒णथे॑न प्रा॒णथे॑न॒ चर॑ति॒ चर॑ति प्रा॒णथे॑न॒ तस्मै॒ तस्मै᳚ प्रा॒णथे॑न॒ चर॑ति॒ चर॑ति प्रा॒णथे॑न॒ तस्मै᳚ । \newline
14. प्रा॒णथे॑न॒ तस्मै॒ तस्मै᳚ प्रा॒णथे॑न प्रा॒णथे॑न॒ तस्मै॑ च च॒ तस्मै᳚ प्रा॒णथे॑न प्रा॒णथे॑न॒ तस्मै॑ च । \newline
15. प्रा॒णथे॒नेति॑ प्र - अ॒नथे॑न । \newline
16. तस्मै॑ च च॒ तस्मै॒ तस्मै॑ च देवि देवि च॒ तस्मै॒ तस्मै॑ च देवि । \newline
17. च॒ दे॒वि॒ दे॒वि॒ च॒ च॒ दे॒वि॒ वष॒ड् वष॑ड् देवि च च देवि॒ वष॑ट् । \newline
18. दे॒वि॒ वष॒ड् वष॑ड् देवि देवि॒ वष॑ डस्त्वस्तु॒ वष॑ड् देवि देवि॒ वष॑डस्तु । \newline
19. वष॑ डस्त्वस्तु॒ वष॒ड् वष॑डस्तु॒ तुभ्य॒म् तुभ्य॑ मस्तु॒ वष॒ड् वष॑डस्तु॒ तुभ्य᳚म् । \newline
20. अ॒स्तु॒ तुभ्य॒म् तुभ्य॑ मस्त्वस्तु॒ तुभ्य᳚म् । \newline
21. तुभ्य॒मिति॒ तुभ्य᳚म् । \newline
22. सुजा॑तो॒ ज्योति॑षा॒ ज्योति॑षा॒ सुजा॑तः॒ सुजा॑तो॒ ज्योति॑षा स॒ह स॒ह ज्योति॑षा॒ सुजा॑तः॒ सुजा॑तो॒ ज्योति॑षा स॒ह । \newline
23. सुजा॑त॒ इति॒ सु - जा॒तः॒ । \newline
24. ज्योति॑षा स॒ह स॒ह ज्योति॑षा॒ ज्योति॑षा स॒ह शर्म॒ शर्म॑ स॒ह ज्योति॑षा॒ ज्योति॑षा स॒ह शर्म॑ । \newline
25. स॒ह शर्म॒ शर्म॑ स॒ह स॒ह शर्म॒ वरू॑थं॒ ॅवरू॑थꣳ॒॒ शर्म॑ स॒ह स॒ह शर्म॒ वरू॑थम् । \newline
26. शर्म॒ वरू॑थं॒ ॅवरू॑थꣳ॒॒ शर्म॒ शर्म॒ वरू॑थ॒ मा वरू॑थꣳ॒॒ शर्म॒ शर्म॒ वरू॑थ॒ मा । \newline
27. वरू॑थ॒ मा वरू॑थं॒ ॅवरू॑थ॒ मा ऽस॑दो असद॒ आ वरू॑थं॒ ॅवरू॑थ॒ मा ऽस॑दः । \newline
28. आ ऽस॑दो असद॒ आ ऽस॑दः॒ सुवः॒ सुव॑ रसद॒ आ ऽस॑दः॒ सुवः॑ । \newline
29. अ॒स॒दः॒ सुवः॒ सुव॑ रसदो असदः॒ सुवः॑ । \newline
30. सुव॒रिति॒ सुवः॑ । \newline
31. वासो॑ अग्ने अग्ने॒ वासो॒ वासो॑ अग्ने वि॒श्वरू॑पं ॅवि॒श्वरू॑प मग्ने॒ वासो॒ वासो॑ अग्ने वि॒श्वरू॑पम् । \newline
32. अ॒ग्ने॒ वि॒श्वरू॑पं ॅवि॒श्वरू॑प मग्ने अग्ने वि॒श्वरू॑पꣳ॒॒ सꣳ सं ॅवि॒श्वरू॑प मग्ने अग्ने वि॒श्वरू॑पꣳ॒॒ सम् । \newline
33. वि॒श्वरू॑पꣳ॒॒ सꣳ सं ॅवि॒श्वरू॑पं ॅवि॒श्वरू॑पꣳ॒॒ सं ॅव्य॑यस्व व्ययस्व॒ सं ॅवि॒श्वरू॑पं ॅवि॒श्वरू॑पꣳ॒॒ सं ॅव्य॑यस्व । \newline
34. वि॒श्वरू॑प॒मिति॑ वि॒श्व - रू॒प॒म् । \newline
35. सं ॅव्य॑यस्व व्ययस्व॒ सꣳ सं ॅव्य॑यस्व विभावसो विभावसो व्ययस्व॒ सꣳ सं ॅव्य॑यस्व विभावसो । \newline
36. व्य॒य॒स्व॒ वि॒भा॒व॒सो॒ वि॒भा॒व॒सो॒ व्य॒य॒स्व॒ व्य॒य॒स्व॒ वि॒भा॒व॒सो॒ । \newline
37. वि॒भा॒व॒सो॒ इति॑ विभा - व॒सो॒ । \newline
38. उदु॑ वु॒ वु दुदु॑ तिष्ठ तिष्ठ॒ वु दुदु॑ तिष्ठ । \newline
39. उ॒ ति॒ष्ठ॒ ति॒ष्ठ॒ वु॒ ति॒ष्ठ॒ स्व॒द्ध्व॒र॒ स्व॒द्ध्व॒र॒ ति॒ष्ठ॒ वु॒ ति॒ष्ठ॒ स्व॒द्ध्व॒र॒ । \newline
40. ति॒ष्ठ॒ स्व॒द्ध्व॒र॒ स्व॒द्ध्व॒र॒ ति॒ष्ठ॒ ति॒ष्ठ॒ स्व॒द्ध्व॒रा वाव॑ स्वद्ध्वर तिष्ठ तिष्ठ स्वद्ध्व॒राव॑ । \newline
41. स्व॒द्ध्व॒रा वाव॑ स्वद्ध्वर स्वद्ध्व॒रावा॑ नो॒ नो ऽव॑ स्वद्ध्वर स्वद्ध्व॒रावा॑ नः । \newline
42. स्व॒द्ध्व॒रेति॑ सु - अ॒द्ध्व॒र॒ । \newline
43. अवा॑ नो॒ नो ऽवावा॑ नो दे॒व्या दे॒व्या नो ऽवावा॑ नो दे॒व्या । \newline
44. नो॒ दे॒व्या दे॒व्या नो॑ नो दे॒व्या कृ॒पा कृ॒पा दे॒व्या नो॑ नो दे॒व्या कृ॒पा । \newline
45. दे॒व्या कृ॒पा कृ॒पा दे॒व्या दे॒व्या कृ॒पा । \newline
46. कृ॒पेति॑ कृ॒पा । \newline
47. दृ॒शे च॑ च दृ॒शे दृ॒शे च॑ भा॒सा भा॒सा च॑ दृ॒शे दृ॒शे च॑ भा॒सा । \newline
48. च॒ भा॒सा भा॒सा च॑ च भा॒सा बृ॑ह॒ता बृ॑ह॒ता भा॒सा च॑ च भा॒सा बृ॑ह॒ता । \newline
49. भा॒सा बृ॑ह॒ता बृ॑ह॒ता भा॒सा भा॒सा बृ॑ह॒ता सु॑शु॒क्वनिः॑ सुशु॒क्वनि॑र् बृह॒ता भा॒सा भा॒सा बृ॑ह॒ता सु॑शु॒क्वनिः॑ । \newline
50. बृ॒ह॒ता सु॑शु॒क्वनिः॑ सुशु॒क्वनि॑र् बृह॒ता बृ॑ह॒ता सु॑शु॒क्वनि॒रा सु॑शु॒क्वनि॑र् बृह॒ता बृ॑ह॒ता सु॑शु॒क्वनि॒रा । \newline
51. सु॒शु॒क्वनि॒रा सु॑शु॒क्वनिः॑ सुशु॒क्वनि॒रा ऽग्ने॑ अग्न॒ आ सु॑शु॒क्वनिः॑ सुशु॒क्वनि॒रा ऽग्ने᳚ । \newline
52. सु॒शु॒क्वनि॒रिति॑ सु - शु॒क्वनिः॑ । \newline
53. आ ऽग्ने॑ अग्न॒ आ ऽग्ने॑ याहि याह्यग्न॒ आ ऽग्ने॑ याहि । \newline
54. अ॒ग्ने॒ या॒हि॒ या॒ह्य॒ग्ने॒ अ॒ग्ने॒ या॒हि॒ सु॒श॒स्तिभिः॑ सुश॒स्तिभि॑र् याह्यग्ने अग्ने याहि सुश॒स्तिभिः॑ । \newline
55. या॒हि॒ सु॒श॒स्तिभिः॑ सुश॒स्तिभि॑र् याहि याहि सुश॒स्तिभिः॑ । \newline
56. सु॒श॒स्तिभि॒रिति॑ सुश॒स्ति - भिः॒ । \newline
\pagebreak
\markright{ TS 4.1.4.2  \hfill https://www.vedavms.in \hfill}

\section{ TS 4.1.4.2 }

\textbf{TS 4.1.4.2 } \newline
\textbf{Samhita Paata} \newline

ऊ॒र्द्ध्व ऊ॒ षु ण॑ ऊ॒तये॒ तिष्ठा॑ दे॒वो न स॑वि॒ता । ऊ॒र्द्ध्वो वाज॑स्य॒ सनि॑ता॒ यद॒ञ्जिभि॑-र्वा॒घद्भि॑-र्वि॒ह्वया॑महे ॥ स जा॒तो गर्भो॑ असि॒ रोद॑स्यो॒रग्ने॒ चारु॒र्विभृ॑त॒ ओष॑धीषु । चि॒त्रः शिशुः॒ परि॒ तमाꣳ॑स्य॒क्तः प्र मा॒तृभ्यो॒ अधि॒ कनि॑क्रदद्गाः ॥ स्थि॒रो भ॑व वी॒ड्व॑ङ्ग आ॒शुर्भ॑व वा॒ज्य॑र्वन्न् । पृ॒थुर्भ॑व सु॒षद॒स्त्वम॒ग्नेः पु॑रीष॒वाह॑नः ॥ शि॒वो भ॑व - [  ] \newline

\textbf{Pada Paata} \newline

ऊ॒द्‌र्ध्वः । उ॒ । स्विति॑ । नः॒ । ऊ॒तये᳚ । तिष्ठ॑ । दे॒वः । न । स॒वि॒ता ॥ ऊ॒द्‌र्ध्वः । वाज॑स्य । सनि॑ता । यत् । अ॒ञ्जिभि॒रित्य॒ञ्जि - भिः॒ । वा॒घद्भि॒रिति॑ वा॒घत् - भिः॒ । वि॒ह्वया॑मह॒ इति॑ वि - ह्वया॑महे ॥ सः । जा॒तः । गर्भः॑ । अ॒सि॒ । रोद॑स्योः । अग्ने᳚ । चारुः॑ । विभृ॑त॒ इति॒ वि - भृ॒तः॒ । ओष॑धीषु ॥ चि॒त्रः । शिशुः॑ । परीति॑ । तमाꣳ॑सी । अ॒क्तः । प्रेति॑ । मा॒तृभ्य॒ इति॑ मा॒तृ - भ्यः॒ । अधीति॑ । कनि॑क्रदत् । गाः॒ ॥ स्थि॒रः । भ॒व॒ । वी॒ड्व॑ङ्ग॒ इति॑ वी॒डु - अ॒ङ्गः॒ । आ॒शुः । भ॒व॒ । वा॒जी । अ॒र्व॒न्न् ॥ पृ॒थुः । भ॒व॒ । सु॒षद॒ इति॑ सु - सदः॑ । त्वम् । अ॒ग्नेः । पु॒री॒ष॒वाह॑न॒ इति॑ पुरीष - वाह॑नः ॥ शि॒वः । भ॒व॒ ।  \newline


\textbf{Krama Paata} \newline

ऊ॒र्द्ध्व उ॑ । ऊ॒ षु णः॑ । सु नः॑ । न॒ ऊ॒तये᳚ । ऊ॒तये॒ तिष्ठ॑ । तिष्ठा॑ दे॒वः । दे॒वो न । न स॑वि॒ता । स॒वि॒तेति॑ सवि॒ता ॥ ऊ॒र्द्धो वाज॑स्य । वाज॑स्य॒ सनि॑ता । सनि॑ता॒ यत् । यद॒ञ्जिभिः॑ । अ॒ञ्जिभि॑र् वा॒घद्भिः॑ । अ॒ञ्जिभि॒रित्य॒ञ्जि - भिः॒ । वा॒घद्भि॑र् वि॒ह्वया॑महे । वा॒घद्भि॒रिति॑ वा॒घत् - भिः॒ । वि॒ह्वया॑मह॒ इति॑ वि - ह्वया॑महे ॥ स जा॒तः । जा॒तो गर्भः॑ । गर्भो॑ असि । अ॒सि॒ रोद॑स्योः । रोद॑स्यो॒रग्ने᳚ । अग्ने॒ चारुः॑ । चारु॒र् विभृ॑तः । विभृ॑त॒ ओष॑धीषु । विभृ॑त॒ इति॒ वि - भृ॒तः॒ । ओष॑धी॒ष्वित्योष॑धीषु ॥ चि॒त्रः शिशुः॑ । शिशुः॒ परि॑ । परि॒ तमाꣳ॑सि । तमाꣳ॑स्य॒क्तः । अ॒क्तः प्र । प्र मा॒तृभ्यः॑ । मा॒तृभ्यो॒ अधि॑ । मा॒तृभ्य॒ इति॑ मा॒तृ - भ्यः॒ । अधि॒ कनि॑क्रदत् । कनि॑क्रदद् गाः । गा॒ इति॑ गाः ॥ स्थि॒रो भ॑व । भ॒व॒ वी॒ड्व॑ङ्गः । वी॒ड्व॑ङ्ग आ॒शुः । वी॒ड्व॑ङ्ग॒ इति॑ वी॒डु - अ॒ङ्गः॒ । आ॒शुर् भ॑व । भ॒व॒ वा॒जी । वा॒ज्य॑र्वन्न् । अ॒र्व॒न्नित्य॑र्वन्न् ॥ पृ॒थुर् भ॑व । भ॒व॒ सु॒षदः॑ । सु॒षद॒स्त्वम् । सु॒षद॒ इति॑ सु - सदः॑ । त्वम॒ग्नेः । अ॒ग्नेः पु॑रीष॒वाह॑नः । पु॒री॒ष॒वाह॑न॒ इति॑ पुरीष - वाह॑नः ॥ शि॒वो भ॑व । भ॒व॒ प्र॒जाभ्यः॑ \newline

\textbf{Jatai Paata} \newline

1. ऊ॒र्द्ध्व उ॑ वु वू॒र्द्ध्व ऊ॒र्द्ध्व उ॑ । \newline
2. ऊ॒ षु णो॑ नः॒ सू॑ षु णः॑ । \newline
3. सु नो॑ नः॒ सु सु नः॑ । \newline
4. न॒ ऊ॒तय॑ ऊ॒तये॑ नो न ऊ॒तये᳚ । \newline
5. ऊ॒तये॒ तिष्ठ॒ तिष्ठो॒तय॑ ऊ॒तये॒ तिष्ठ॑ । \newline
6. तिष्ठा॑ दे॒वो दे॒व स्तिष्ठ॒ तिष्ठा॑ दे॒वः । \newline
7. दे॒वो न न दे॒वो दे॒वो न । \newline
8. न स॑वि॒ता स॑वि॒ता न न स॑वि॒ता । \newline
9. स॒वि॒तेति॑ सवि॒ता । \newline
10. ऊ॒र्द्ध्वो वाज॑स्य॒ वाज॑ स्यो॒र्द्ध्व ऊ॒र्द्ध्वो वाज॑स्य । \newline
11. वाज॑स्य॒ सनि॑ता॒ सनि॑ता॒ वाज॑स्य॒ वाज॑स्य॒ सनि॑ता । \newline
12. सनि॑ता॒ यद् यथ् सनि॑ता॒ सनि॑ता॒ यत् । \newline
13. यद॒ञ्जिभि॑ र॒ञ्जिभि॒र् यद् यद॒ञ्जिभिः॑ । \newline
14. अ॒ञ्जिभि॑र् वा॒घद्भि॑र् वा॒घद्भि॑ र॒ञ्जिभि॑ र॒ञ्जिभि॑र् वा॒घद्भिः॑ । \newline
15. अ॒ञ्जिभि॒रित्य॒ञ्जि - भिः॒ । \newline
16. वा॒घद्भि॑र् वि॒ह्वया॑महे वि॒ह्वया॑महे वा॒घद्भि॑र् वा॒घद्भि॑र् वि॒ह्वया॑महे । \newline
17. वा॒घद्भि॒रिति॑ वा॒घत् - भिः॒ । \newline
18. वि॒ह्वया॑मह॒ इति॑ वि - ह्वया॑महे । \newline
19. स जा॒तो जा॒तः स स जा॒तः । \newline
20. जा॒तो गर्भो॒ गर्भो॑ जा॒तो जा॒तो गर्भः॑ । \newline
21. गर्भो॑ अस्यसि॒ गर्भो॒ गर्भो॑ असि । \newline
22. अ॒सि॒ रोद॑स्यो॒ रोद॑स्यो रस्यसि॒ रोद॑स्योः । \newline
23. रोद॑स्यो॒ रग्ने ऽग्ने॒ रोद॑स्यो॒ रोद॑स्यो॒ रग्ने᳚ । \newline
24. अग्ने॒ चारु॒ श्चारु॒ रग्ने ऽग्ने॒ चारुः॑ । \newline
25. चारु॒र् विभृ॑तो॒ विभृ॑त॒ श्चारु॒ श्चारु॒र् विभृ॑तः । \newline
26. विभृ॑त॒ ओष॑धी॒ ष्वोष॑धीषु॒ विभृ॑तो॒ विभृ॑त॒ ओष॑धीषु । \newline
27. विभृ॑त॒ इति॒ वि - भृ॒तः॒ । \newline
28. ओष॑धी॒ष्वित्योष॑धीषु । \newline
29. चि॒त्रः शिशुः॒ शिशु॑ श्चि॒त्र श्चि॒त्रः शिशुः॑ । \newline
30. शिशुः॒ परि॒ परि॒ शिशुः॒ शिशुः॒ परि॑ । \newline
31. परि॒ तमाꣳ॑सि॒ तमाꣳ॑सि॒ परि॒ परि॒ तमाꣳ॑सि । \newline
32. तमाꣳ॑ स्य॒क्तो अ॒क्त स्तमाꣳ॑सि॒ तमाꣳ॑ स्य॒क्तः । \newline
33. अ॒क्तः प्र प्राक्तो अ॒क्तः प्र । \newline
34. प्र मा॒तृभ्यो॑ मा॒तृभ्यः॒ प्र प्र मा॒तृभ्यः॑ । \newline
35. मा॒तृभ्यो॒ अध्यधि॑ मा॒तृभ्यो॑ मा॒तृभ्यो॒ अधि॑ । \newline
36. मा॒तृभ्य॒ इति॑ मा॒तृ - भ्यः॒ । \newline
37. अधि॒ कनि॑क्रद॒त् कनि॑क्र द॒दध्यधि॒ कनि॑क्रदत् । \newline
38. कनि॑क्रदद् गा गाः॒ कनि॑क्रद॒त् कनि॑क्रदद् गाः । \newline
39. गा॒ इति॑ गाः । \newline
40. स्थि॒रो भ॑व भव स्थि॒रः स्थि॒रो भ॑व । \newline
41. भ॒व॒ वी॒ड्व॑ङ्गो वी॒ड्व॑ङ्गो भव भव वी॒ड्व॑ङ्गः । \newline
42. वी॒ड्व॑ङ्ग आ॒शु रा॒शुर् वी॒ड्व॑ङ्गो वी॒ड्व॑ङ्ग आ॒शुः । \newline
43. वी॒ड्व॑ङ्ग॒ इति॑ वी॒डु - अ॒ङ्गः॒ । \newline
44. आ॒शुर् भ॑व भवा॒शु रा॒शुर् भ॑व । \newline
45. भ॒व॒ वा॒जी वा॒जी भ॑व भव वा॒जी । \newline
46. वा॒ज्य॑र्वन् नर्वन्. वा॒जी वा॒ज्य॑र्वन्न् । \newline
47. अ॒र्व॒न्नित्य॑र्वन्न् । \newline
48. पृ॒थुर् भ॑व भव पृ॒थुः पृ॒थुर् भ॑व । \newline
49. भ॒व॒ सु॒षदः॑ सु॒षदो॑ भव भव सु॒षदः॑ । \newline
50. सु॒षद॒ स्त्वम् त्वꣳ सु॒षदः॑ सु॒षद॒ स्त्वम् । \newline
51. सु॒षद॒ इति॑ सु - सदः॑ । \newline
52. त्व म॒ग्ने र॒ग्ने स्त्वम् त्व म॒ग्नेः । \newline
53. अ॒ग्नेः पु॑रीष॒वाह॑नः पुरीष॒वाह॑नो अ॒ग्ने र॒ग्नेः पु॑रीष॒वाह॑नः । \newline
54. पु॒री॒ष॒वाह॑न॒ इति॑ पुरीष - वाह॑नः । \newline
55. शि॒वो भ॑व भव शि॒वः शि॒वो भ॑व । \newline
56. भ॒व॒ प्र॒जाभ्यः॑ प्र॒जाभ्यो॑ भव भव प्र॒जाभ्यः॑ । \newline

\textbf{Ghana Paata } \newline

1. ऊ॒र्द्ध्व उ॑ वु वू॒र्द्ध्व ऊ॒र्द्ध्व ऊ॒ षु णो॑ नः॒ सू᳚ र्द्ध्व ऊ॒र्द्ध्व ऊ॒ षु णः॑ । \newline
2. ऊ॒ षु णो॑ नः॒ सू॑ षु ण॑ ऊ॒तय॑ ऊ॒तये॑ नः॒ सू॑ षु ण॑ ऊ॒तये᳚ । \newline
3. सु नो॑ नः॒ सु सु न॑ ऊ॒तय॑ ऊ॒तये॑ नः॒ सु सु न॑ ऊ॒तये᳚ । \newline
4. न॒ ऊ॒तय॑ ऊ॒तये॑ नो न ऊ॒तये॒ तिष्ठ॒ तिष्ठो॒तये॑ नो न ऊ॒तये॒ तिष्ठ॑ । \newline
5. ऊ॒तये॒ तिष्ठ॒ तिष्ठो॒तय॑ ऊ॒तये॒ तिष्ठा॑ दे॒वो दे॒व स्तिष्ठो॒तय॑ ऊ॒तये॒ तिष्ठा॑ दे॒वः । \newline
6. तिष्ठा॑ दे॒वो दे॒व स्तिष्ठ॒ तिष्ठा॑ दे॒वो न न दे॒व स्तिष्ठ॒ तिष्ठा॑ दे॒वो न । \newline
7. दे॒वो न न दे॒वो दे॒वो न स॑वि॒ता स॑वि॒ता न दे॒वो दे॒वो न स॑वि॒ता । \newline
8. न स॑वि॒ता स॑वि॒ता न न स॑वि॒ता । \newline
9. स॒वि॒तेति॑ सवि॒ता । \newline
10. ऊ॒र्द्ध्वो वाज॑स्य॒ वाज॑ स्यो॒र्द्ध्व ऊ॒र्द्ध्वो वाज॑स्य॒ सनि॑ता॒ सनि॑ता॒ वाज॑ स्यो॒र्द्ध्व ऊ॒र्द्ध्वो वाज॑स्य॒ सनि॑ता । \newline
11. वाज॑स्य॒ सनि॑ता॒ सनि॑ता॒ वाज॑स्य॒ वाज॑स्य॒ सनि॑ता॒ यद् यथ् सनि॑ता॒ वाज॑स्य॒ वाज॑स्य॒ सनि॑ता॒ यत् । \newline
12. सनि॑ता॒ यद् यथ् सनि॑ता॒ सनि॑ता॒ यद॒ञ्जिभि॑ र॒ञ्जिभि॒र् यथ् सनि॑ता॒ सनि॑ता॒ यद॒ञ्जिभिः॑ । \newline
13. यद॒ञ्जिभि॑ र॒ञ्जिभि॒र् यद् यद॒ञ्जिभि॑र् वा॒घद्भि॑र् वा॒घद्भि॑ र॒ञ्जिभि॒र् यद् यद॒ञ्जिभि॑र् वा॒घद्भिः॑ । \newline
14. अ॒ञ्जिभि॑र् वा॒घद्भि॑र् वा॒घद्भि॑ र॒ञ्जिभि॑ र॒ञ्जिभि॑र् वा॒घद्भि॑र् वि॒ह्वया॑महे वि॒ह्वया॑महे वा॒घद्भि॑ र॒ञ्जिभि॑ र॒ञ्जिभि॑र् वा॒घद्भि॑र् वि॒ह्वया॑महे । \newline
15. अ॒ञ्जिभि॒रित्य॒ञ्जि - भिः॒ । \newline
16. वा॒घद्भि॑र् वि॒ह्वया॑महे वि॒ह्वया॑महे वा॒घद्भि॑र् वा॒घद्भि॑र् वि॒ह्वया॑महे । \newline
17. वा॒घद्भि॒रिति॑ वा॒घत् - भिः॒ । \newline
18. वि॒ह्वया॑मह॒ इति॑ वि - ह्वया॑महे । \newline
19. स जा॒तो जा॒तः स स जा॒तो गर्भो॒ गर्भो॑ जा॒तः स स जा॒तो गर्भः॑ । \newline
20. जा॒तो गर्भो॒ गर्भो॑ जा॒तो जा॒तो गर्भो॑ अस्यसि॒ गर्भो॑ जा॒तो जा॒तो गर्भो॑ असि । \newline
21. गर्भो॑ अस्यसि॒ गर्भो॒ गर्भो॑ असि॒ रोद॑स्यो॒ रोद॑स्यो रसि॒ गर्भो॒ गर्भो॑ असि॒ रोद॑स्योः । \newline
22. अ॒सि॒ रोद॑स्यो॒ रोद॑स्यो रस्यसि॒ रोद॑स्यो॒ रग्ने ऽग्ने॒ रोद॑स्यो रस्यसि॒ रोद॑स्यो॒ रग्ने᳚ । \newline
23. रोद॑स्यो॒ रग्ने ऽग्ने॒ रोद॑स्यो॒ रोद॑स्यो॒ रग्ने॒ चारु॒ श्चारु॒ रग्ने॒ रोद॑स्यो॒ रोद॑स्यो॒ रग्ने॒ चारुः॑ । \newline
24. अग्ने॒ चारु॒ श्चारु॒ रग्ने ऽग्ने॒ चारु॒र् विभृ॑तो॒ विभृ॑त॒ श्चारु॒ रग्ने ऽग्ने॒ चारु॒र् विभृ॑तः । \newline
25. चारु॒र् विभृ॑तो॒ विभृ॑त॒ श्चारु॒ श्चारु॒र् विभृ॑त॒ ओष॑धी॒ ष्वोष॑धीषु॒ विभृ॑त॒ श्चारु॒ श्चारु॒र् विभृ॑त॒ ओष॑धीषु । \newline
26. विभृ॑त॒ ओष॑धी॒ ष्वोष॑धीषु॒ विभृ॑तो॒ विभृ॑त॒ ओष॑धीषु । \newline
27. विभृ॑त॒ इति॒ वि - भृ॒तः॒ । \newline
28. ओष॑धी॒ष्वित्योष॑धीषु । \newline
29. चि॒त्रः शिशुः॒ शिशु॑ श्चि॒त्र श्चि॒त्रः शिशुः॒ परि॒ परि॒ शिशु॑ श्चि॒त्र श्चि॒त्रः शिशुः॒ परि॑ । \newline
30. शिशुः॒ परि॒ परि॒ शिशुः॒ शिशुः॒ परि॒ तमाꣳ॑सि॒ तमाꣳ॑सि॒ परि॒ शिशुः॒ शिशुः॒ परि॒ तमाꣳ॑सि । \newline
31. परि॒ तमाꣳ॑सि॒ तमाꣳ॑सि॒ परि॒ परि॒ तमाꣳ॑ स्य॒क्तो अ॒क्त स्तमाꣳ॑सि॒ परि॒ परि॒ तमाꣳ॑ स्य॒क्तः । \newline
32. तमाꣳ॑ स्य॒क्तो अ॒क्त स्तमाꣳ॑सि॒ तमाꣳ॑ स्य॒क्तः प्र प्राक्त स्तमाꣳ॑सि॒ तमाꣳ॑ स्य॒क्तः प्र । \newline
33. अ॒क्तः प्र प्राक्तो अ॒क्तः प्र मा॒तृभ्यो॑ मा॒तृभ्यः॒ प्राक्तो अ॒क्तः प्र मा॒तृभ्यः॑ । \newline
34. प्र मा॒तृभ्यो॑ मा॒तृभ्यः॒ प्र प्र मा॒तृभ्यो॒ अध्यधि॑ मा॒तृभ्यः॒ प्र प्र मा॒तृभ्यो॒ अधि॑ । \newline
35. मा॒तृभ्यो॒ अध्यधि॑ मा॒तृभ्यो॑ मा॒तृभ्यो॒ अधि॒ कनि॑क्रद॒त् कनि॑क्र द॒दधि॑ मा॒तृभ्यो॑ मा॒तृभ्यो॒ अधि॒ कनि॑क्रदत् । \newline
36. मा॒तृभ्य॒ इति॑ मा॒तृ - भ्यः॒ । \newline
37. अधि॒ कनि॑क्रद॒त् कनि॑क्रद॒ दध्यधि॒ कनि॑क्रदद् गा गाः॒ कनि॑क्र द॒दध्यधि॒ कनि॑क्रदद् गाः । \newline
38. कनि॑क्रदद् गा गाः॒ कनि॑क्रद॒त् कनि॑क्रदद् गाः । \newline
39. गा॒ इति॑ गाः । \newline
40. स्थि॒रो भ॑व भव स्थि॒रः स्थि॒रो भ॑व वी॒ड्व॑ङ्गो वी॒ड्व॑ङ्गो भव स्थि॒रः स्थि॒रो भ॑व वी॒ड्व॑ङ्गः । \newline
41. भ॒व॒ वी॒ड्व॑ङ्गो वी॒ड्व॑ङ्गो भव भव वी॒ड्व॑ङ्ग आ॒शु रा॒शुर् वी॒ड्व॑ङ्गो भव भव वी॒ड्व॑ङ्ग आ॒शुः । \newline
42. वी॒ड्व॑ङ्ग आ॒शु रा॒शुर् वी॒ड्व॑ङ्गो वी॒ड्व॑ङ्ग आ॒शुर् भ॑व भवा॒शुर् वी॒ड्व॑ङ्गो वी॒ड्व॑ङ्ग आ॒शुर् भ॑व । \newline
43. वी॒ड्व॑ङ्ग॒ इति॑ वी॒डु - अ॒ङ्गः॒ । \newline
44. आ॒शुर् भ॑व भवा॒ शुरा॒शुर् भ॑व वा॒जी वा॒जी भ॑वा॒ शुरा॒शुर् भ॑व वा॒जी । \newline
45. भ॒व॒ वा॒जी वा॒जी भ॑व भव वा॒ज्य॑र्वन् नर्वन्. वा॒जी भ॑व भव वा॒ज्य॑र्वन्न् । \newline
46. वा॒ज्य॑र्वन् नर्वन्. वा॒जी वा॒ज्य॑र्वन्न् । \newline
47. अ॒र्व॒न्नित्य॑र्वन्न् । \newline
48. पृ॒थुर् भ॑व भव पृ॒थुः पृ॒थुर् भ॑व सु॒षदः॑ सु॒षदो॑ भव पृ॒थुः पृ॒थुर् भ॑व सु॒षदः॑ । \newline
49. भ॒व॒ सु॒षदः॑ सु॒षदो॑ भव भव सु॒षद॒ स्त्वम् त्वꣳ सु॒षदो॑ भव भव सु॒षद॒ स्त्वम् । \newline
50. सु॒षद॒ स्त्वम् त्वꣳ सु॒षदः॑ सु॒षद॒ स्त्व म॒ग्ने र॒ग्ने स्त्वꣳ सु॒षदः॑ सु॒षद॒ स्त्व म॒ग्नेः । \newline
51. सु॒षद॒ इति॑ सु - सदः॑ । \newline
52. त्व म॒ग्ने र॒ग्ने स्त्वम् त्व म॒ग्नेः पु॑रीष॒वाह॑नः पुरीष॒वाह॑नो अ॒ग्ने स्त्वम् त्व म॒ग्नेः पु॑रीष॒वाह॑नः । \newline
53. अ॒ग्नेः पु॑रीष॒वाह॑नः पुरीष॒वाह॑नो अ॒ग्ने र॒ग्नेः पु॑रीष॒वाह॑नः । \newline
54. पु॒री॒ष॒वाह॑न॒ इति॑ पुरीष - वाह॑नः । \newline
55. शि॒वो भ॑व भव शि॒वः शि॒वो भ॑व प्र॒जाभ्यः॑ प्र॒जाभ्यो॑ भव शि॒वः शि॒वो भ॑व प्र॒जाभ्यः॑ । \newline
56. भ॒व॒ प्र॒जाभ्यः॑ प्र॒जाभ्यो॑ भव भव प्र॒जाभ्यो॒ मानु॑षीभ्यो॒ मानु॑षीभ्यः प्र॒जाभ्यो॑ भव भव प्र॒जाभ्यो॒ मानु॑षीभ्यः । \newline
\pagebreak
\markright{ TS 4.1.4.3  \hfill https://www.vedavms.in \hfill}

\section{ TS 4.1.4.3 }

\textbf{TS 4.1.4.3 } \newline
\textbf{Samhita Paata} \newline

प्र॒जाभ्यो॒ मानु॑षीभ्य॒स्त्वम॑ङ्गिरः । मा द्यावा॑पृथि॒वी अ॒भि शू॑शुचो॒ माऽन्तरि॑क्षं॒ मा वन॒स्पतीन्॑ ॥ प्रैतु॑ वा॒जी कनि॑क्रद॒-न्नान॑द॒द्-रास॑भः॒ पत्वा᳚ । भर॑न्न॒ग्निं पु॑री॒ष्यं॑ मा पा॒द्यायु॑षः पु॒रा ॥ रास॑भो वां॒ कनि॑क्रद॒थ् सुयु॑क्तो वृषणा॒ रथे᳚ । स वा॑म॒ग्निं पु॑री॒ष्य॑मा॒शुर्दू॒तो व॑हादि॒तः ॥ वृषा॒ऽग्निं ॅवृष॑णं॒ भर॑न्न॒पां गर्भꣳ॑ समु॒द्रियं᳚ । अग्न॒ आ या॑हि- [  ] \newline

\textbf{Pada Paata} \newline

प्र॒जाभ्य॒ इति॑ प्र - जाभ्यः॑ । मानु॑षीभ्यः । त्वम् । अ॒ङ्गि॒रः॒ ॥ मा । द्यावा॑पृथि॒वी इति॒ द्यावा᳚ - पृ॒थि॒वी । अ॒भीति॑ । शू॒शु॒चः॒ । मा । अ॒न्तरि॑क्षम् । मा । वन॒स्पतीन्॑ ॥ प्रेति॑ । ए॒तु॒ । वा॒जी । कनि॑क्रदत् । नान॑दत् । रास॑भः । पत्वा᳚ ॥ भरन्न्॑ । अ॒ग्निम् । पु॒री॒ष्य᳚म् । मा । पा॒दि॒ । आयु॑षः । पु॒रा ॥ रास॑भः । वा॒म् । कनि॑क्रदत् । सुयु॑क्त॒ इति॒ सु - यु॒क्तः॒ । वृ॒ष॒णा॒ । रथे᳚ ॥ सः । वा॒म् । अ॒ग्निम् । पु॒री॒ष्य᳚म् । आ॒शुः । दू॒तः । व॒हा॒त् । इ॒तः ॥ वृषा᳚ । अ॒ग्निम् । वृष॑णम् । भरन्न्॑ । अ॒पाम् । गर्भ᳚म् । स॒मु॒द्रिय᳚म् ॥ अग्ने᳚ । एति॑ । या॒हि॒ ।  \newline


\textbf{Krama Paata} \newline

प्र॒जाभ्यो॒ मानु॑षीभ्यः । प्र॒जाभ्य॒ इति॑ प्र - जाभ्यः॑ । मानु॑षीभ्य॒स्त्वम् । त्वम॑ङ्गिरः । अ॒ङ्गि॒र॒ इत्य॑ङ्गिरः ॥ मा द्यावा॑पृथि॒वी । द्यावा॑पृथि॒वी अ॒भि । द्यावा॑पृथि॒वी इति॒ द्यावा᳚ - पृ॒थि॒वी । अ॒भि शू॑शुचः । शू॒शु॒चो॒ मा । माऽन्तरि॑क्षम् । अ॒न्तरि॑क्ष॒म् मा । मा वन॒स्पतीन्॑ । वन॒स्पती॒निति॑ वन॒स्पतीन्॑ ॥ प्रैतु॑ । ए॒तु॒ वा॒जी । वा॒जी कनि॑क्रदत् । कनि॑क्रद॒न् नान॑दत् । नान॑द॒द् रास॑भः । रास॑भः॒ पत्वा᳚ । पत्वेति॒ पत्वा᳚ ॥ भर॑न्न॒ग्निम् । अ॒ग्निम् पु॑री॒ष्य᳚म् । पु॒री॒ष्य॑म् मा । मा पा॑दि । पा॒द्यायु॑षः । आयु॑षः पु॒रा । पु॒रेति॑ पु॒रा ॥ रास॑भो वाम् । वा॒म् कनि॑क्रदत् । कनि॑क्रद॒थ् सुयु॑क्तः । सुयु॑क्तो वृषणा । सुयु॑क्त॒ इति॒ सु - यु॒क्तः॒ । वृ॒ष॒णा॒ रथे᳚ । रथ॒ इति॒ रथे᳚ ॥ स वा᳚म् । वा॒म॒ग्निम् । अ॒ग्निम् पु॑री॒ष्य᳚म् । पु॒री॒ष्य॑मा॒शुः । आ॒शुर् दू॒तः । दू॒तो व॑हात् । व॒हा॒दि॒तः । इ॒त इती॒तः ॥ वृषा॒ ऽग्निम् । अ॒ग्निं ॅवृष॑णम् । वृष॑ण॒म् भरन्न्॑ । भर॑न्न॒पाम् । अ॒पाम् गर्भ᳚म् । गर्भꣳ॑ समु॒द्रियम्᳚ । स॒मु॒द्रिय॒मिति॑ समु॒द्रिय᳚म् ॥ अग्न॒ आ । आ या॑हि ( ) । या॒हि॒ वी॒तये᳚ \newline

\textbf{Jatai Paata} \newline

1. प्र॒जाभ्यो॒ मानु॑षीभ्यो॒ मानु॑षीभ्यः प्र॒जाभ्यः॑ प्र॒जाभ्यो॒ मानु॑षीभ्यः । \newline
2. प्र॒जाभ्य॒ इति॑ प्र - जाभ्यः॑ । \newline
3. मानु॑षीभ्य॒ स्त्वम् त्वम् मानु॑षीभ्यो॒ मानु॑षीभ्य॒ स्त्वम् । \newline
4. त्व म॑ङ्गिरो अङ्गिर॒ स्त्वम् त्व म॑ङ्गिरः । \newline
5. अ॒ङ्गि॒र॒ इत्य॑ङ्गिरः । \newline
6. मा द्यावा॑पृथि॒वी द्यावा॑पृथि॒वी मा मा द्यावा॑पृथि॒वी । \newline
7. द्यावा॑पृथि॒वी अ॒भ्य॑भि द्यावा॑पृथि॒वी द्यावा॑पृथि॒वी अ॒भि । \newline
8. द्यावा॑पृथि॒वी इति॒ द्यावा᳚ - पृ॒थि॒वी । \newline
9. अ॒भि शू॑शुचः शूशुचो अ॒भ्य॑भि शू॑शुचः । \newline
10. शू॒शु॒चो॒ मा मा शू॑शुचः शूशुचो॒ मा । \newline
11. मा ऽन्तरि॑क्ष म॒न्तरि॑क्ष॒म् मा मा ऽन्तरि॑क्षम् । \newline
12. अ॒न्तरि॑क्ष॒म् मा मा ऽन्तरि॑क्ष म॒न्तरि॑क्ष॒म् मा । \newline
13. मा वन॒स्पती॒न्॒. वन॒स्पती॒न् मा मा वन॒स्पतीन्॑ । \newline
14. वन॒स्पती॒निति॑ वन॒स्पतीन्॑ । \newline
15. प्रैत्वे॑तु॒ प्र प्रैतु॑ । \newline
16. ए॒तु॒ वा॒जी वा॒ज्ये᳚त्वेतु वा॒जी । \newline
17. वा॒जी कनि॑क्रद॒त् कनि॑क्रदद् वा॒जी वा॒जी कनि॑क्रदत् । \newline
18. कनि॑क्रद॒न् नान॑द॒न् नान॑द॒त् कनि॑क्रद॒त् कनि॑क्रद॒न् नान॑दत् । \newline
19. नान॑द॒द् रास॑भो॒ रास॑भो॒ नान॑द॒न् नान॑द॒द् रास॑भः । \newline
20. रास॑भः॒ पत्वा॒ पत्वा॒ रास॑भो॒ रास॑भः॒ पत्वा᳚ । \newline
21. पत्वेति॒ पत्वा᳚ । \newline
22. भर॑न् न॒ग्नि म॒ग्निम् भर॒न् भर॑न् न॒ग्निम् । \newline
23. अ॒ग्निम् पु॑री॒ष्य॑म् पुरी॒ष्य॑ म॒ग्नि म॒ग्निम् पु॑री॒ष्य᳚म् । \newline
24. पु॒री॒ष्य॑म् मा मा पु॑री॒ष्य॑म् पुरी॒ष्य॑म् मा । \newline
25. मा पा॑दि पादि॒ मा मा पा॑दि । \newline
26. पा॒द्यायु॑ष॒ आयु॑षः पादि पा॒द्यायु॑षः । \newline
27. आयु॑षः पु॒रा पु॒रा ऽऽयु॑ष॒ आयु॑षः पु॒रा । \newline
28. पु॒रेति॑ पु॒रा । \newline
29. रास॑भो वां ॅवाꣳ॒॒ रास॑भो॒ रास॑भो वाम् । \newline
30. वा॒म् कनि॑क्रद॒त् कनि॑क्रदद् वां ॅवा॒म् कनि॑क्रदत् । \newline
31. कनि॑क्रद॒थ् सुयु॑क्तः॒ सुयु॑क्तः॒ कनि॑क्रद॒त् कनि॑क्रद॒थ् सुयु॑क्तः । \newline
32. सुयु॑क्तो वृषणा वृषणा॒ सुयु॑क्तः॒ सुयु॑क्तो वृषणा । \newline
33. सुयु॑क्त॒ इति॒ सु - यु॒क्तः॒ । \newline
34. वृ॒ष॒णा॒ रथे॒ रथे॑ वृषणा वृषणा॒ रथे᳚ । \newline
35. रथ॒ इति॒ रथे᳚ । \newline
36. स वां᳚ ॅवाꣳ॒॒ स स वा᳚म् । \newline
37. वा॒ म॒ग्नि म॒ग्निं ॅवां᳚ ॅवा म॒ग्निम् । \newline
38. अ॒ग्निम् पु॑री॒ष्य॑म् पुरी॒ष्य॑ म॒ग्नि म॒ग्निम् पु॑री॒ष्य᳚म् । \newline
39. पु॒री॒ष्य॑ मा॒शु रा॒शुः पु॑री॒ष्य॑म् पुरी॒ष्य॑ मा॒शुः । \newline
40. आ॒शुर् दू॒तो दू॒त आ॒शु रा॒शुर् दू॒तः । \newline
41. दू॒तो व॑हाद् वहाद् दू॒तो दू॒तो व॑हात् । \newline
42. व॒हा॒ दि॒त इ॒तो व॑हाद् वहा दि॒तः । \newline
43. इ॒त इती॒तः । \newline
44. वृषा॒ ऽग्नि म॒ग्निं ॅवृषा॒ वृषा॒ ऽग्निम् । \newline
45. अ॒ग्निं ॅवृष॑णं॒ ॅवृष॑ण म॒ग्नि म॒ग्निं ॅवृष॑णम् । \newline
46. वृष॑ण॒म् भर॒न् भर॒न् वृष॑णं॒ ॅवृष॑ण॒म् भरन्न्॑ । \newline
47. भर॑न् न॒पा म॒पाम् भर॒न् भर॑न् न॒पाम् । \newline
48. अ॒पाम् गर्भ॒म् गर्भ॑ म॒पा म॒पाम् गर्भ᳚म् । \newline
49. गर्भꣳ॑ समु॒द्रियꣳ॑ समु॒द्रिय॒म् गर्भ॒म् गर्भꣳ॑ समु॒द्रिय᳚म् । \newline
50. स॒मु॒द्रिय॒मिति॑ समु॒द्रिय᳚म् । \newline
51. अग्न॒ आ ऽग्ने॑ ऽग्न॒ आ । \newline
52. आ या॑हि या॒ह्या या॑हि । \newline
53. या॒हि॒ वी॒तये॑ वी॒तये॑ याहि याहि वी॒तये᳚ । \newline

\textbf{Ghana Paata } \newline

1. प्र॒जाभ्यो॒ मानु॑षीभ्यो॒ मानु॑षीभ्यः प्र॒जाभ्यः॑ प्र॒जाभ्यो॒ मानु॑षीभ्य॒ स्त्वम् त्वम् मानु॑षीभ्यः प्र॒जाभ्यः॑ प्र॒जाभ्यो॒ मानु॑षीभ्य॒ स्त्वम् । \newline
2. प्र॒जाभ्य॒ इति॑ प्र - जाभ्यः॑ । \newline
3. मानु॑षीभ्य॒ स्त्वम् त्वम् मानु॑षीभ्यो॒ मानु॑षीभ्य॒ स्त्व म॑ङ्गिरो अङ्गिर॒ स्त्वम् मानु॑षीभ्यो॒ मानु॑षीभ्य॒ स्त्व म॑ङ्गिरः । \newline
4. त्व म॑ङ्गिरो अङ्गिर॒ स्त्वम् त्व म॑ङ्गिरः । \newline
5. अ॒ङ्गि॒र॒ इत्य॑ङ्गिरः । \newline
6. मा द्यावा॑पृथि॒वी द्यावा॑पृथि॒वी मा मा द्यावा॑पृथि॒वी अ॒भ्य॑भि द्यावा॑पृथि॒वी मा मा द्यावा॑पृथि॒वी अ॒भि । \newline
7. द्यावा॑पृथि॒वी अ॒भ्य॑भि द्यावा॑पृथि॒वी द्यावा॑पृथि॒वी अ॒भि शू॑शुचः शूशुचो अ॒भि द्यावा॑पृथि॒वी द्यावा॑पृथि॒वी अ॒भि शू॑शुचः । \newline
8. द्यावा॑पृथि॒वी इति॒ द्यावा᳚ - पृ॒थि॒वी । \newline
9. अ॒भि शू॑शुचः शूशुचो अ॒भ्य॑भि शू॑शुचो॒ मा मा शू॑शुचो अ॒भ्य॑भि शू॑शुचो॒ मा । \newline
10. शू॒शु॒चो॒ मा मा शू॑शुचः शूशुचो॒ मा ऽन्तरि॑क्ष म॒न्तरि॑क्ष॒म् मा शू॑शुचः शूशुचो॒ मा ऽन्तरि॑क्षम् । \newline
11. मा ऽन्तरि॑क्ष म॒न्तरि॑क्ष॒म् मा मा ऽन्तरि॑क्ष॒म् मा मा ऽन्तरि॑क्ष॒म् मा मा ऽन्तरि॑क्ष॒म् मा । \newline
12. अ॒न्तरि॑क्ष॒म् मा मा ऽन्तरि॑क्ष म॒न्तरि॑क्ष॒म् मा वन॒स्पती॒न्॒. वन॒स्पती॒न् मा ऽन्तरि॑क्ष म॒न्तरि॑क्ष॒म् मा वन॒स्पतीन्॑ । \newline
13. मा वन॒स्पती॒न्॒. वन॒स्पती॒न् मा मा वन॒स्पतीन्॑ । \newline
14. वन॒स्पती॒निति॑ वन॒स्पतीन्॑ । \newline
15. प्रै त्वे॑तु॒ प्र प्रैतु॑ वा॒जी वा॒ज्ये॑तु॒ प्र प्रैतु॑ वा॒जी । \newline
16. ए॒तु॒ वा॒जी वा॒ज्ये᳚ त्वेतु वा॒जी कनि॑क्रद॒त् कनि॑क्रदद् वा॒ज्ये᳚ त्वेतु वा॒जी कनि॑क्रदत् । \newline
17. वा॒जी कनि॑क्रद॒त् कनि॑क्रदद् वा॒जी वा॒जी कनि॑क्रद॒न् नान॑द॒न् नान॑द॒त् कनि॑क्रदद् वा॒जी वा॒जी कनि॑क्रद॒न् नान॑दत् । \newline
18. कनि॑क्रद॒न् नान॑द॒न् नान॑द॒त् कनि॑क्रद॒त् कनि॑क्रद॒न् नान॑द॒द् रास॑भो॒ रास॑भो॒ नान॑द॒त् कनि॑क्रद॒त् कनि॑क्रद॒न् नान॑द॒द् रास॑भः । \newline
19. नान॑द॒द् रास॑भो॒ रास॑भो॒ नान॑द॒न् नान॑द॒द् रास॑भः॒ पत्वा॒ पत्वा॒ रास॑भो॒ नान॑द॒न् नान॑द॒द् रास॑भः॒ पत्वा᳚ । \newline
20. रास॑भः॒ पत्वा॒ पत्वा॒ रास॑भो॒ रास॑भः॒ पत्वा᳚ । \newline
21. पत्वेति॒ पत्वा᳚ । \newline
22. भर॑न् न॒ग्नि म॒ग्निम् भर॒न् भर॑न् न॒ग्निम् पु॑री॒ष्य॑म् पुरी॒ष्य॑ म॒ग्निम् भर॒न् भर॑न् न॒ग्निम् पु॑री॒ष्य᳚म् । \newline
23. अ॒ग्निम् पु॑री॒ष्य॑म् पुरी॒ष्य॑ म॒ग्नि म॒ग्निम् पु॑री॒ष्य॑म् मा मा पु॑री॒ष्य॑ म॒ग्नि म॒ग्निम् पु॑री॒ष्य॑म् मा । \newline
24. पु॒री॒ष्य॑म् मा मा पु॑री॒ष्य॑म् पुरी॒ष्य॑म् मा पा॑दि पादि॒ मा पु॑री॒ष्य॑म् पुरी॒ष्य॑म् मा पा॑दि । \newline
25. मा पा॑दि पादि॒ मा मा पा॒द्यायु॑ष॒ आयु॑षः पादि॒ मा मा पा॒द्यायु॑षः । \newline
26. पा॒द्यायु॑ष॒ आयु॑षः पादि पा॒द्यायु॑षः पु॒रा पु॒रा ऽऽयु॑षः पादि पा॒द्यायु॑षः पु॒रा । \newline
27. आयु॑षः पु॒रा पु॒रा ऽऽयु॑ष॒ आयु॑षः पु॒रा । \newline
28. पु॒रेति॑ पु॒रा । \newline
29. रास॑भो वां ॅवाꣳ॒॒ रास॑भो॒ रास॑भो वा॒म् कनि॑क्रद॒त् कनि॑क्रदद् वाꣳ॒॒ रास॑भो॒ रास॑भो वा॒म् कनि॑क्रदत् । \newline
30. वा॒म् कनि॑क्रद॒त् कनि॑क्रदद् वां ॅवा॒म् कनि॑क्रद॒थ् सुयु॑क्तः॒ सुयु॑क्तः॒ कनि॑क्रदद् वां ॅवा॒म् कनि॑क्रद॒थ् सुयु॑क्तः । \newline
31. कनि॑क्रद॒थ् सुयु॑क्तः॒ सुयु॑क्तः॒ कनि॑क्रद॒त् कनि॑क्रद॒थ् सुयु॑क्तो वृषणा वृषणा॒ सुयु॑क्तः॒ कनि॑क्रद॒त् कनि॑क्रद॒थ् सुयु॑क्तो वृषणा । \newline
32. सुयु॑क्तो वृषणा वृषणा॒ सुयु॑क्तः॒ सुयु॑क्तो वृषणा॒ रथे॒ रथे॑ वृषणा॒ सुयु॑क्तः॒ सुयु॑क्तो वृषणा॒ रथे᳚ । \newline
33. सुयु॑क्त॒ इति॒ सु - यु॒क्तः॒ । \newline
34. वृ॒ष॒णा॒ रथे॒ रथे॑ वृषणा वृषणा॒ रथे᳚ । \newline
35. रथ॒ इति॒ रथे᳚ । \newline
36. स वां᳚ ॅवाꣳ॒॒ स स वा॑ म॒ग्नि म॒ग्निं ॅवाꣳ॒॒ स स वा॑ म॒ग्निम् । \newline
37. वा॒ म॒ग्नि म॒ग्निं ॅवां᳚ ॅवा म॒ग्निम् पु॑री॒ष्य॑म् पुरी॒ष्य॑ म॒ग्निं ॅवां᳚ ॅवा म॒ग्निम् पु॑री॒ष्य᳚म् । \newline
38. अ॒ग्निम् पु॑री॒ष्य॑म् पुरी॒ष्य॑ म॒ग्नि म॒ग्निम् पु॑री॒ष्य॑ मा॒शु रा॒शुः पु॑री॒ष्य॑ म॒ग्नि म॒ग्निम् पु॑री॒ष्य॑ मा॒शुः । \newline
39. पु॒री॒ष्य॑ मा॒शु रा॒शुः पु॑री॒ष्य॑म् पुरी॒ष्य॑ मा॒शुर् दू॒तो दू॒त आ॒शुः पु॑री॒ष्य॑म् पुरी॒ष्य॑ मा॒शुर् दू॒तः । \newline
40. आ॒शुर् दू॒तो दू॒त आ॒शु रा॒शुर् दू॒तो व॑हाद् वहाद् दू॒त आ॒शु रा॒शुर् दू॒तो व॑हात् । \newline
41. दू॒तो व॑हाद् वहाद् दू॒तो दू॒तो व॑हा दि॒त इ॒तो व॑हाद् दू॒तो दू॒तो व॑हा दि॒तः । \newline
42. व॒हा॒ दि॒त इ॒तो व॑हाद् वहा दि॒तः । \newline
43. इ॒त इती॒तः । \newline
44. वृषा॒ ऽग्नि म॒ग्निं ॅवृषा॒ वृषा॒ ऽग्निं ॅवृष॑णं॒ ॅवृष॑ण म॒ग्निं ॅवृषा॒ वृषा॒ ऽग्निं ॅवृष॑णम् । \newline
45. अ॒ग्निं ॅवृष॑णं॒ ॅवृष॑ण म॒ग्नि म॒ग्निं ॅवृष॑ण॒म् भर॒न् भर॒न् वृष॑ण म॒ग्नि म॒ग्निं ॅवृष॑ण॒म् भरन्न्॑ । \newline
46. वृष॑ण॒म् भर॒न् भर॒न् वृष॑णं॒ ॅवृष॑ण॒म् भर॑न् न॒पा म॒पाम् भर॒न् वृष॑णं॒ ॅवृष॑ण॒म् भर॑न् न॒पाम् । \newline
47. भर॑न् न॒पा म॒पाम् भर॒न् भर॑न् न॒पाम् गर्भ॒म् गर्भ॑ म॒पाम् भर॒न् भर॑न् न॒पाम् गर्भ᳚म् । \newline
48. अ॒पाम् गर्भ॒म् गर्भ॑ म॒पा म॒पाम् गर्भꣳ॑ समु॒द्रियꣳ॑ समु॒द्रिय॒म् गर्भ॑ म॒पा म॒पाम् गर्भꣳ॑ समु॒द्रिय᳚म् । \newline
49. गर्भꣳ॑ समु॒द्रियꣳ॑ समु॒द्रिय॒म् गर्भ॒म् गर्भꣳ॑ समु॒द्रिय᳚म् । \newline
50. स॒मु॒द्रिय॒मिति॑ समु॒द्रिय᳚म् । \newline
51. अग्न॒ आ ऽग्ने॑ ऽन॒ आ या॑हि या॒ह्या ऽग्ने॑ ऽग्न॒ आ या॑हि । \newline
52. आ या॑हि या॒ह्या या॑हि वी॒तये॑ वी॒तये॑ या॒ह्या या॑हि वी॒तये᳚ । \newline
53. या॒हि॒ वी॒तये॑ वी॒तये॑ याहि याहि वी॒तय॑ ऋ॒त मृ॒तं ॅवी॒तये॑ याहि याहि वी॒तय॑ ऋ॒तम् । \newline
\pagebreak
\markright{ TS 4.1.4.4  \hfill https://www.vedavms.in \hfill}

\section{ TS 4.1.4.4 }

\textbf{TS 4.1.4.4 } \newline
\textbf{Samhita Paata} \newline

वी॒तय॑ ऋ॒तꣳ स॒त्यं ॥ ओष॑धयः॒ प्रति॑ गृह्णीता॒ऽग्निमे॒तꣳ शि॒वमा॒यन्त॑म॒भ्यत्र॑ यु॒ष्मान् । व्यस्य॒न् विश्वा॒ अम॑ती॒ररा॑ती-र्नि॒षीद॑न्नो॒ अप॑ दुर्म॒तिꣳ ह॑नत् ॥ ओष॑धयः॒ प्रति॑ मोदद्ध्वमेनं॒ पुष्पा॑वतीः सुपिप्प॒लाः । अ॒यं ॅवो॒ गर्भ॑ ऋ॒त्वियः॑ प्र॒त्नꣳ स॒धस्थ॒मा ऽस॑दत् ॥ \newline

\textbf{Pada Paata} \newline

वी॒तये᳚ । ऋ॒तम् । स॒त्यम् ॥ ओष॑धयः । प्रतीति॑ । गृ॒ह्णी॒त॒ । अ॒ग्निम् । ए॒तम् । शि॒वम् । आ॒यन्त॒मित्या᳚ - यन्त᳚म् । अ॒भीति॑ । अत्र॑ । यु॒ष्मान् ॥ व्यस्य॒न्निति॑ वि - अस्यन्न्॑ । विश्वाः᳚ । अम॑तीः । अरा॑तीः । नि॒षीद॒न्नि ति॑ नि - सीदन्न्॑ । नः॒ । अपेति॑ । दु॒र्म॒तिमिति॑ दुः- म॒तिम् । ह॒न॒त् ॥ ओष॑धयः । प्रतीति॑ । मो॒द॒द्ध्व॒म् । ए॒न॒म् । पुष्पा॑वती॒रिति॒ पुष्प॑ - व॒तीः॒ । सु॒पि॒प्प॒ला इति॑ सु - पि॒प्प॒लाः ॥ अ॒यम् । वः॒ । गर्भः॑ । ऋ॒त्वियः॑ । प्र॒त्नम् । स॒धस्थ॒मिति॑ स॒ध-स्थ॒म् । एति॑ । अ॒स॒द॒त् ॥  \newline


\textbf{Krama Paata} \newline

वी॒तय॑ ऋ॒तम् । ऋ॒तꣳ स॒त्यम् । स॒त्यमिति॑ स॒त्यम् ॥ ओष॑धयः॒ प्रति॑ । प्रति॑ गृह्णीत । गृ॒ह्णी॒ता॒ग्निम् । अ॒ग्निमे॒तम् । ए॒तꣳ शि॒वम् । शि॒वमा॒यन्त᳚म् । आ॒यन्त॑म॒भि । आ॒यन्त॒मित्या᳚ - यन्त᳚म् । अ॒भ्यत्र॑ । अत्र॑ यु॒ष्मान् । यु॒ष्मानिति॑ यु॒ष्मान् ॥ व्यस्य॒न् विश्वाः᳚ । व्यस्य॒न्निति॑ वि - अस्यन्न्॑ । विश्वा॒ अम॑तीः । अम॑ती॒ररा॑तीः । अरा॑तीर् नि॒षीदन्न्॑ । नि॒षीद॑न् नः । नि॒षीद॒न्निति॑ नि - सीदन्न्॑ । नो॒ अप॑ । अप॑ दुर्म॒तिम् । दु॒र्म॒तिꣳ ह॑नत् । दु॒र्म॒तिमिति॑ दुः - म॒तिम् । ह॒न॒दिति॑ हनत् ॥ ओष॑धयः॒ प्रति॑ । प्रति॑ मोदद्ध्वम् । मो॒द॒द्ध्व॒मे॒न॒म् । ए॒न॒म् पुष्पा॑वतीः । पुष्पा॑वतीः सुपिप्प॒लाः । पुष्पा॑वती॒रिति॒ पुष्प॑ - व॒तीः॒ । सु॒पि॒प्प॒ला इति॑ सु - पि॒प्प॒लाः ॥ अ॒यं ॅवः॑ । वो॒ गर्भः॑ । गर्भ॑ ऋ॒त्वियः॑ । ऋ॒त्वियः॑ प्र॒त्नम् । प्र॒त्नꣳ स॒धस्थ᳚म् । स॒धस्थ॒मा । स॒धस्थ॒मिति॑ स॒ध - स्थ॒म् । आऽस॑दत् । अ॒स॒द॒दित्य॑सदत् । \newline

\textbf{Jatai Paata} \newline

1. वी॒तय॑ ऋ॒त मृ॒तं ॅवी॒तये॑ वी॒तय॑ ऋ॒तम् । \newline
2. ऋ॒तꣳ स॒त्यꣳ स॒त्य मृ॒त मृ॒तꣳ स॒त्यम् । \newline
3. स॒त्यमिति॑ स॒त्यम् । \newline
4. ओष॑धयः॒ प्रति॒ प्रत्योष॑धय॒ ओष॑धयः॒ प्रति॑ । \newline
5. प्रति॑ गृह्णीत गृह्णीत॒ प्रति॒ प्रति॑ गृह्णीत । \newline
6. गृ॒ह्णी॒ता॒ग्नि म॒ग्निम् गृ॑ह्णीत गृह्णीता॒ग्निम् । \newline
7. अ॒ग्नि मे॒त मे॒त म॒ग्नि म॒ग्नि मे॒तम् । \newline
8. ए॒तꣳ शि॒वꣳ शि॒व मे॒त मे॒तꣳ शि॒वम् । \newline
9. शि॒व मा॒यन्त॑ मा॒यन्तꣳ॑ शि॒वꣳ शि॒व मा॒यन्त᳚म् । \newline
10. आ॒यन्त॑ म॒भ्या᳚(1॒)भ्या॑यन्त॑ मा॒यन्त॑ म॒भि । \newline
11. आ॒यन्त॒मित्या᳚ - यन्त᳚म् । \newline
12. अ॒भ्यत्रात्रा॒ भ्य॑भ्यत्र॑ । \newline
13. अत्र॑ यु॒ष्मान्. यु॒ष्मा नत्रात्र॑ यु॒ष्मान् । \newline
14. यु॒ष्मानिति॑ यु॒ष्मान् । \newline
15. व्यस्य॒न्॒. विश्वा॒ विश्वा॒ व्यस्य॒न् व्यस्य॒न्॒. विश्वाः᳚ । \newline
16. व्यस्य॒न्निति॑ वि - अस्यन्न्॑ । \newline
17. विश्वा॒ अम॑ती॒ रम॑ती॒र् विश्वा॒ विश्वा॒ अम॑तीः । \newline
18. अम॑ती॒ ररा॑ती॒ ररा॑ती॒ रम॑ती॒ रम॑ती॒ ररा॑तीः । \newline
19. अरा॑तीर् नि॒षीद॑न् नि॒षीद॒न् नरा॑ती॒ ररा॑तीर् नि॒षीदन्न्॑ । \newline
20. नि॒षीद॑न् नो नो नि॒षीद॑न् नि॒षीद॑न् नः । \newline
21. नि॒षीद॒न्नि ति॑ नि - सीदन्न्॑ । \newline
22. नो॒ अपाप॑ नो नो॒ अप॑ । \newline
23. अप॑ दुर्म॒तिम् दु॑र्म॒ति मपाप॑ दुर्म॒तिम् । \newline
24. दु॒र्म॒तिꣳ ह॑नद्धनद् दुर्म॒तिम् दु॑र्म॒तिꣳ ह॑नत् । \newline
25. दु॒र्म॒तिमिति॑ दुः - म॒तिम् । \newline
26. ह॒न॒दिति॑ हनत् । \newline
27. ओष॑धयः॒ प्रति॒ प्रत्योष॑धय॒ ओष॑धयः॒ प्रति॑ । \newline
28. प्रति॑ मोदद्ध्वम् मोदद्ध्व॒म् प्रति॒ प्रति॑ मोदद्ध्वम् । \newline
29. मो॒द॒द्ध्व॒ मे॒न॒ मे॒न॒म् मो॒द॒द्ध्व॒म् मो॒द॒द्ध्व॒ मे॒न॒म् । \newline
30. ए॒न॒म् पुष्पा॑वतीः॒ पुष्पा॑वतीरेन मेन॒म् पुष्पा॑वतीः । \newline
31. पुष्पा॑वतीः सुपिप्प॒लाः सु॑पिप्प॒लाः पुष्पा॑वतीः॒ पुष्पा॑वतीः सुपिप्प॒लाः । \newline
32. पुष्पा॑वती॒रिति॒ पुष्प॑ - व॒तीः॒ । \newline
33. सु॒पि॒प्प॒ला इति॑ सु - पि॒प्प॒लाः । \newline
34. अ॒यं ॅवो॑ वो॒ ऽय म॒यं ॅवः॑ । \newline
35. वो॒ गर्भो॒ गर्भो॑ वो वो॒ गर्भः॑ । \newline
36. गर्भ॑ ऋ॒त्विय॑ ऋ॒त्वियो॒ गर्भो॒ गर्भ॑ ऋ॒त्वियः॑ । \newline
37. ऋ॒त्वियः॑ प्र॒त्नम् प्र॒त्न मृ॒त्विय॑ ऋ॒त्वियः॑ प्र॒त्नम् । \newline
38. प्र॒त्नꣳ स॒धस्थꣳ॑ स॒धस्थ॑म् प्र॒त्नम् प्र॒त्नꣳ स॒धस्थ᳚म् । \newline
39. स॒धस्थ॒ मा स॒धस्थꣳ॑ स॒धस्थ॒ मा । \newline
40. स॒धस्थ॒मिति॑ स॒ध - स्थ॒म् । \newline
41. आ ऽस॑द दसद॒दा ऽस॑दत् । \newline
42. अ॒स॒द॒दित्य॑सदत् । \newline

\textbf{Ghana Paata } \newline

1. वी॒तय॑ ऋ॒त मृ॒तं ॅवी॒तये॑ वी॒तय॑ ऋ॒तꣳ स॒त्यꣳ स॒त्य मृ॒तं ॅवी॒तये॑ वी॒तय॑ ऋ॒तꣳ स॒त्यम् । \newline
2. ऋ॒तꣳ स॒त्यꣳ स॒त्य मृ॒त मृ॒तꣳ स॒त्यम् । \newline
3. स॒त्यमिति॑ स॒त्यम् । \newline
4. ओष॑धयः॒ प्रति॒ प्रत्योष॑धय॒ ओष॑धयः॒ प्रति॑ गृह्णीत गृह्णीत॒ प्रत्योष॑धय॒ ओष॑धयः॒ प्रति॑ गृह्णीत । \newline
5. प्रति॑ गृह्णीत गृह्णीत॒ प्रति॒ प्रति॑ गृह्णीता॒ग्नि म॒ग्निम् गृ॑ह्णीत॒ प्रति॒ प्रति॑ गृह्णीता॒ग्निम् । \newline
6. गृ॒ह्णी॒ता॒ग्नि म॒ग्निम् गृ॑ह्णीत गृह्णीता॒ग्नि मे॒त मे॒त म॒ग्निम् गृ॑ह्णीत गृह्णीता॒ग्नि मे॒तम् । \newline
7. अ॒ग्नि मे॒त मे॒त म॒ग्नि म॒ग्नि मे॒तꣳ शि॒वꣳ शि॒व मे॒त म॒ग्नि म॒ग्नि मे॒तꣳ शि॒वम् । \newline
8. ए॒तꣳ शि॒वꣳ शि॒व मे॒त मे॒तꣳ शि॒व मा॒यन्त॑ मा॒यन्तꣳ॑ शि॒व मे॒त मे॒तꣳ शि॒व मा॒यन्त᳚म् । \newline
9. शि॒व मा॒यन्त॑ मा॒यन्तꣳ॑ शि॒वꣳ शि॒व मा॒यन्त॑ म॒भ्या᳚(1॒)भ्या॑ यन्तꣳ॑ शि॒वꣳ शि॒व मा॒यन्त॑ म॒भि । \newline
10. आ॒यन्त॑ म॒भ्या᳚(1॒)भ्या॑ यन्त॑ मा॒यन्त॑ म॒भ्य त्रात्रा॒भ्या॑ यन्त॑ मा॒यन्त॑ म॒भ्यत्र॑ । \newline
11. आ॒यन्त॒मित्या᳚ - यन्त᳚म् । \newline
12. अ॒भ्य त्रात्रा॒ भ्य॑ भ्यत्र॑ यु॒ष्मान्. यु॒ष्मा नत्रा॒ भ्य॑ भ्यत्र॑ यु॒ष्मान् । \newline
13. अत्र॑ यु॒ष्मान्. यु॒ष्मा नत्रात्र॑ यु॒ष्मान् । \newline
14. यु॒ष्मानिति॑ यु॒ष्मान् । \newline
15. व्यस्य॒न्॒. विश्वा॒ विश्वा॒ व्यस्य॒न् व्यस्य॒न्॒. विश्वा॒ अम॑ती॒ रम॑ती॒र् विश्वा॒ व्यस्य॒न् व्यस्य॒न्॒. विश्वा॒ अम॑तीः । \newline
16. व्यस्य॒न्निति॑ वि - अस्यन्न्॑ । \newline
17. विश्वा॒ अम॑ती॒ रम॑ती॒र् विश्वा॒ विश्वा॒ अम॑ती॒ ररा॑ती॒ ररा॑ती॒ रम॑ती॒र् विश्वा॒ विश्वा॒ अम॑ती॒ ररा॑तीः । \newline
18. अम॑ती॒ ररा॑ती॒ ररा॑ती॒ रम॑ती॒ रम॑ती॒ ररा॑तीर् नि॒षीद॑न् नि॒षीद॒न् नरा॑ती॒ रम॑ती॒ रम॑ती॒ ररा॑तीर् नि॒षीदन्न्॑ । \newline
19. अरा॑तीर् नि॒षीद॑न् नि॒षीद॒न् नरा॑ती॒ ररा॑तीर् नि॒षीद॑न् नो नो नि॒षीद॒न् नरा॑ती॒ ररा॑तीर् नि॒षीद॑न् नः । \newline
20. नि॒षीद॑न् नो नो नि॒षीद॑न् नि॒षीद॑न् नो॒ अपाप॑ नो नि॒षीद॑न् नि॒षीद॑न् नो॒ अप॑ । \newline
21. नि॒षीद॒न्नि ति॑ नि - सीदन्न्॑ । \newline
22. नो॒ अपाप॑ नो नो॒ अप॑ दुर्म॒तिम् दु॑र्म॒ति मप॑ नो नो॒ अप॑ दुर्म॒तिम् । \newline
23. अप॑ दुर्म॒तिम् दु॑र्म॒ति मपाप॑ दुर्म॒तिꣳ ह॑न द्धनद् दुर्म॒ति मपाप॑ दुर्म॒तिꣳ ह॑नत् । \newline
24. दु॒र्म॒तिꣳ ह॑न द्धनद् दुर्म॒तिम् दु॑र्म॒तिꣳ ह॑नत् । \newline
25. दु॒र्म॒तिमिति॑ दुः - म॒तिम् । \newline
26. ह॒न॒दिति॑ हनत् । \newline
27. ओष॑धयः॒ प्रति॒ प्रत्योष॑धय॒ ओष॑धयः॒ प्रति॑ मोदद्ध्वम् मोदद्ध्व॒म् प्रत्योष॑धय॒ ओष॑धयः॒ प्रति॑ मोदद्ध्वम् । \newline
28. प्रति॑ मोदद्ध्वम् मोदद्ध्व॒म् प्रति॒ प्रति॑ मोदद्ध्व मेन मेनम् मोदद्ध्व॒म् प्रति॒ प्रति॑ मोदद्ध्व मेनम् । \newline
29. मो॒द॒द्ध्व॒ मे॒न॒ मे॒न॒म् मो॒द॒द्ध्व॒म् मो॒द॒द्ध्व॒ मे॒न॒म् पुष्पा॑वतीः॒ पुष्पा॑वती रेनम् मोदद्ध्वम् मोदद्ध्व मेन॒म् पुष्पा॑वतीः । \newline
30. ए॒न॒म् पुष्पा॑वतीः॒ पुष्पा॑वती रेन मेन॒म् पुष्पा॑वतीः सुपिप्प॒लाः सु॑पिप्प॒लाः पुष्पा॑वती रेन मेन॒म् पुष्पा॑वतीः सुपिप्प॒लाः । \newline
31. पुष्पा॑वतीः सुपिप्प॒लाः सु॑पिप्प॒लाः पुष्पा॑वतीः॒ पुष्पा॑वतीः सुपिप्प॒लाः । \newline
32. पुष्पा॑वती॒रिति॒ पुष्प॑ - व॒तीः॒ । \newline
33. सु॒पि॒प्प॒ला इति॑ सु - पि॒प्प॒लाः । \newline
34. अ॒यं ॅवो॑ वो॒ ऽय म॒यं ॅवो॒ गर्भो॒ गर्भो॑ वो॒ ऽय म॒यं ॅवो॒ गर्भः॑ । \newline
35. वो॒ गर्भो॒ गर्भो॑ वो वो॒ गर्भ॑ ऋ॒त्विय॑ ऋ॒त्वियो॒ गर्भो॑ वो वो॒ गर्भ॑ ऋ॒त्वियः॑ । \newline
36. गर्भ॑ ऋ॒त्विय॑ ऋ॒त्वियो॒ गर्भो॒ गर्भ॑ ऋ॒त्वियः॑ प्र॒त्नम् प्र॒त्न मृ॒त्वियो॒ गर्भो॒ गर्भ॑ ऋ॒त्वियः॑ प्र॒त्नम् । \newline
37. ऋ॒त्वियः॑ प्र॒त्नम् प्र॒त्न मृ॒त्विय॑ ऋ॒त्वियः॑ प्र॒त्नꣳ स॒धस्थꣳ॑ स॒धस्थ॑म् प्र॒त्न मृ॒त्विय॑ ऋ॒त्वियः॑ प्र॒त्नꣳ स॒धस्थ᳚म् । \newline
38. प्र॒त्नꣳ स॒धस्थꣳ॑ स॒धस्थ॑म् प्र॒त्नम् प्र॒त्नꣳ स॒धस्थ॒ मा स॒धस्थ॑म् प्र॒त्नम् प्र॒त्नꣳ स॒धस्थ॒ मा । \newline
39. स॒धस्थ॒ मा स॒धस्थꣳ॑ स॒धस्थ॒ मा ऽस॑द दसद॒दा स॒धस्थꣳ॑ स॒धस्थ॒ मा ऽस॑दत् । \newline
40. स॒धस्थ॒मिति॑ स॒ध - स्थ॒म् । \newline
41. आ ऽस॑द दसद॒ दा ऽस॑दत् । \newline
42. अ॒स॒द॒दित्य॑सदत् । \newline
\pagebreak
\markright{ TS 4.1.5.1  \hfill https://www.vedavms.in \hfill}

\section{ TS 4.1.5.1 }

\textbf{TS 4.1.5.1 } \newline
\textbf{Samhita Paata} \newline

वि पाज॑सा पृ॒थुना॒ शोशु॑चानो॒ बाध॑स्व द्वि॒षो र॒क्षसो॒ अमी॑वाः । सु॒शर्म॑णो बृह॒तः शर्म॑णि स्याम॒ग्नेर॒हꣳ सु॒हव॑स्य॒ प्रणी॑तौ ॥ आपो॒ हि ष्ठा म॑यो॒भुव॒स्ता न॑ ऊ॒र्जे द॑धातन । म॒हे रणा॑य॒ चक्ष॑से ॥ यो वः॑ शि॒वत॑मो॒ रस॒स्तस्य॑ भाजयते॒ह नः॑ । उ॒श॒तीरि॑व मा॒तरः॑ ॥ तस्मा॒ अरं॑ गमाम वो॒ यस्य॒ क्षया॑य॒ जिन्व॑थ । आपो॑ ज॒नय॑था च नः ॥ मि॒त्रः - [  ] \newline

\textbf{Pada Paata} \newline

वीति॑ । पाज॑सा । पृ॒थुना᳚ । शोशु॑चानः । बाध॑स्व । द्वि॒षः । र॒क्षसः॑ । अमी॑वाः ॥ सु॒शर्म॑ण॒ इति॑ सु - शर्म॑णः । बृ॒ह॒तः । शर्म॑णि । स्या॒म् । अ॒ग्नेः । अ॒हम् । सु॒हव॒स्येति॑ सु-हव॑स्य । प्रणी॑ता॒विति॒ प्र-नी॒तौ॒ ॥ आपः॑ । हि । स्थ । म॒यो॒भुव॒ इति॑ मयः-भुवः॑ । ताः । नः॒ । ऊ॒र्जे । द॒धा॒त॒न॒ ॥ म॒हे । रणा॑य । चक्ष॑से ॥ यः । वः॒ । शि॒वत॑म॒ इति॑ शि॒व-त॒मः॒ । रसः॑ । तस्य॑ । भा॒ज॒य॒त॒ । इ॒ह । नः॒ ॥ उ॒श॒तीः । इ॒व॒ । मा॒तरः॑ ॥ तस्मै᳚ । अर᳚म् । ग॒मा॒म॒ । वः॒ । यस्य॑ । क्षया॑य । जिन्व॑थ ॥ आपः॑ । ज॒नय॑थ । च॒ । नः॒ ॥ मि॒त्रः ।  \newline


\textbf{Krama Paata} \newline

वि पाज॑सा । पाज॑सा पृ॒थुना᳚ । पृ॒थुना॒ शोशु॑चानः । शोशु॑चानो॒ बाध॑स्व । बाध॑स्व द्वि॒षः । द्वि॒षो र॒क्षसः॑ । र॒क्षसो॒ अमी॑वाः । अमी॑वा॒ इत्यमी॑वाः ॥ सु॒शर्म॑णो बृह॒तः । सु॒शर्म॑ण॒ इति॑ सु - शर्म॑णः । बृ॒ह॒तः शर्म॑णि । शर्म॑णि स्याम् । स्या॒म॒ग्नेः । अ॒ग्नेर॒हम् । अ॒हꣳ सु॒हव॑स्य । सु॒हव॑स्य॒ प्रणी॑तौ । सु॒हव॒स्येति॑ सु - हव॑स्य । प्रणी॑ता॒विति॒ प्र - नी॒तौ॒ ॥ आपो॒ हि । हि ष्ठ । स्था म॑यो॒भुवः॑ । म॒यो॒भुव॒स्ताः । म॒यो॒भुव॒ इति॑ मयः - भुवः॑ । ता नः॑ । न॒ ऊ॒र्जे । ऊ॒र्जे द॑धातन । द॒धा॒त॒नेति॑ दधातन ॥ म॒हे रणा॑य । रणा॑य॒ चक्ष॑से । चक्ष॑स॒ इति॒ चक्ष॑से ॥ यो वः॑ । वः॒ शि॒वत॑मः । शि॒वत॑मो॒ रसः॑ । शि॒वत॑म॒ इति॑ शि॒व - त॒मः॒ । रस॒स्तस्य॑ । तस्य॑ भाजयत । भा॒ज॒य॒ते॒ह । इ॒ह नः॑ । न॒ इति॑ नः ॥ उ॒श॒तीरि॑व । इ॒व॒ मा॒तरः॑ । मा॒तर॒ इति॑ मा॒तरः॑ ॥ तस्मा॒ अर᳚म् । अर॑म् गमाम । ग॒मा॒म॒ वः॒ । वो॒ यस्य॑ । यस्य॒ क्षया॑य । क्षया॑य॒ जिन्व॑थ । जिन्व॒थेति॒ जिन्व॑थ ॥ आपो॑ ज॒नय॑थ । ज॒नय॑था च । च॒ नः॒ । न॒ इति॑ नः ॥ मि॒त्रः सꣳ॒॒सृज्य॑ \newline

\textbf{Jatai Paata} \newline

1. वि पाज॑सा॒ पाज॑सा॒ वि वि पाज॑सा । \newline
2. पाज॑सा पृ॒थुना॑ पृ॒थुना॒ पाज॑सा॒ पाज॑सा पृ॒थुना᳚ । \newline
3. पृ॒थुना॒ शोशु॑चानः॒ शोशु॑चानः पृ॒थुना॑ पृ॒थुना॒ शोशु॑चानः । \newline
4. शोशु॑चानो॒ बाध॑स्व॒ बाध॑स्व॒ शोशु॑चानः॒ शोशु॑चानो॒ बाध॑स्व । \newline
5. बाध॑स्व द्वि॒षो द्वि॒षो बाध॑स्व॒ बाध॑स्व द्वि॒षः । \newline
6. द्वि॒षो र॒क्षसो॑ र॒क्षसो᳚ द्वि॒षो द्वि॒षो र॒क्षसः॑ । \newline
7. र॒क्षसो॒ अमी॑वा॒ अमी॑वा र॒क्षसो॑ र॒क्षसो॒ अमी॑वाः । \newline
8. अमी॑वा॒ इत्यमी॑वाः । \newline
9. सु॒शर्म॑णो बृह॒तो बृ॑ह॒तः सु॒शर्म॑णः सु॒शर्म॑णो बृह॒तः । \newline
10. सु॒शर्म॑ण॒ इति॑ सु - शर्म॑णः । \newline
11. बृ॒ह॒तः शर्म॑णि॒ शर्म॑णि बृह॒तो बृ॑ह॒तः शर्म॑णि । \newline
12. शर्म॑णि स्याꣳ स्याꣳ॒॒ शर्म॑णि॒ शर्म॑णि स्याम् । \newline
13. स्या॒ म॒ग्ने र॒ग्नेः स्याꣳ॑ स्या म॒ग्नेः । \newline
14. अ॒ग्नेर॒ह म॒ह म॒ग्ने र॒ग्ने र॒हम् । \newline
15. अ॒हꣳ सु॒हव॑स्य सु॒हव॑स्या॒ह म॒हꣳ सु॒हव॑स्य । \newline
16. सु॒हव॑स्य॒ प्रणी॑तौ॒ प्रणी॑तौ सु॒हव॑स्य सु॒हव॑स्य॒ प्रणी॑तौ । \newline
17. सु॒हव॒स्येति॑ सु - हव॑स्य । \newline
18. प्रणी॑ता॒विति॒ प्र - नी॒तौ॒ । \newline
19. आपो॒ हि ह्याप॒ आपो॒ हि । \newline
20. हि ष्ठ स्थ हि हि ष्ठ । \newline
21. स्था म॑यो॒भुवो॑ मयो॒भुवः॒ स्थ स्था म॑यो॒भुवः॑ । \newline
22. म॒यो॒भुव॒ स्ता स्ता म॑यो॒भुवो॑ मयो॒भुव॒ स्ताः । \newline
23. म॒यो॒भुव॒ इति॑ मयः - भुवः॑ । \newline
24. ता नो॑ न॒ स्ता स्ता नः॑ । \newline
25. न॒ ऊ॒र्ज ऊ॒र्जे नो॑ न ऊ॒र्जे । \newline
26. ऊ॒र्जे द॑धातन दधातनो॒र्ज ऊ॒र्जे द॑धातन । \newline
27. द॒धा॒त॒नेति॑ दधातन । \newline
28. म॒हे रणा॑य॒ रणा॑य म॒हे म॒हे रणा॑य । \newline
29. रणा॑य॒ चक्ष॑से॒ चक्ष॑से॒ रणा॑य॒ रणा॑य॒ चक्ष॑से । \newline
30. चक्ष॑स॒ इति॒ चक्ष॑से । \newline
31. यो वो॑ वो॒ यो यो वः॑ । \newline
32. वः॒ शि॒वत॑मः शि॒वत॑मो वो वः शि॒वत॑मः । \newline
33. शि॒वत॑मो॒ रसो॒ रसः॑ शि॒वत॑मः शि॒वत॑मो॒ रसः॑ । \newline
34. शि॒वत॑म॒ इति॑ शि॒व - त॒मः॒ । \newline
35. रस॒ स्तस्य॒ तस्य॒ रसो॒ रस॒ स्तस्य॑ । \newline
36. तस्य॑ भाजयत भाजयत॒ तस्य॒ तस्य॑ भाजयत । \newline
37. भा॒ज॒य॒ते॒ हेह भा॑जयत भाजयते॒ह । \newline
38. इ॒ह नो॑ न इ॒हेह नः॑ । \newline
39. न॒ इति॑ नः । \newline
40. उ॒श॒ती रि॑वे वोश॒ती रु॑श॒ती रि॑व । \newline
41. इ॒व॒ मा॒तरो॑ मा॒तर॑ इवे व मा॒तरः॑ । \newline
42. मा॒तर॒ इति॑ मा॒तरः॑ । \newline
43. तस्मा॒ अर॒ मर॒म् तस्मै॒ तस्मा॒ अर᳚म् । \newline
44. अर॑म् गमाम गमा॒मा र॒ मर॑म् गमाम । \newline
45. ग॒मा॒म॒ वो॒ वो॒ ग॒मा॒म॒ ग॒मा॒म॒ वः॒ । \newline
46. वो॒ यस्य॒ यस्य॑ वो वो॒ यस्य॑ । \newline
47. यस्य॒ क्षया॑य॒ क्षया॑य॒ यस्य॒ यस्य॒ क्षया॑य । \newline
48. क्षया॑य॒ जिन्व॑थ॒ जिन्व॑थ॒ क्षया॑य॒ क्षया॑य॒ जिन्व॑थ । \newline
49. जिन्व॒थेति॒ जिन्व॑थ । \newline
50. आपो॑ ज॒नय॑थ ज॒नय॒थाप॒ आपो॑ ज॒नय॑थ । \newline
51. ज॒नय॑था च च ज॒नय॑थ ज॒नय॑था च । \newline
52. च॒ नो॒ न॒श्च॒ च॒ नः॒ । \newline
53. न॒ इति॑ नः । \newline
54. मि॒त्रः सꣳ॒॒सृज्य॑ सꣳ॒॒सृज्य॑ मि॒त्रो मि॒त्रः सꣳ॒॒सृज्य॑ । \newline

\textbf{Ghana Paata } \newline

1. वि पाज॑सा॒ पाज॑सा॒ वि वि पाज॑सा पृ॒थुना॑ पृ॒थुना॒ पाज॑सा॒ वि वि पाज॑सा पृ॒थुना᳚ । \newline
2. पाज॑सा पृ॒थुना॑ पृ॒थुना॒ पाज॑सा॒ पाज॑सा पृ॒थुना॒ शोशु॑चानः॒ शोशु॑चानः पृ॒थुना॒ पाज॑सा॒ पाज॑सा पृ॒थुना॒ शोशु॑चानः । \newline
3. पृ॒थुना॒ शोशु॑चानः॒ शोशु॑चानः पृ॒थुना॑ पृ॒थुना॒ शोशु॑चानो॒ बाध॑स्व॒ बाध॑स्व॒ शोशु॑चानः पृ॒थुना॑ पृ॒थुना॒ शोशु॑चानो॒ बाध॑स्व । \newline
4. शोशु॑चानो॒ बाध॑स्व॒ बाध॑स्व॒ शोशु॑चानः॒ शोशु॑चानो॒ बाध॑स्व द्वि॒षो द्वि॒षो बाध॑स्व॒ शोशु॑चानः॒ शोशु॑चानो॒ बाध॑स्व द्वि॒षः । \newline
5. बाध॑स्व द्वि॒षो द्वि॒षो बाध॑स्व॒ बाध॑स्व द्वि॒षो र॒क्षसो॑ र॒क्षसो᳚ द्वि॒षो बाध॑स्व॒ बाध॑स्व द्वि॒षो र॒क्षसः॑ । \newline
6. द्वि॒षो र॒क्षसो॑ र॒क्षसो᳚ द्वि॒षो द्वि॒षो र॒क्षसो॒ अमी॑वा॒ अमी॑वा र॒क्षसो᳚ द्वि॒षो द्वि॒षो र॒क्षसो॒ अमी॑वाः । \newline
7. र॒क्षसो॒ अमी॑वा॒ अमी॑वा र॒क्षसो॑ र॒क्षसो॒ अमी॑वाः । \newline
8. अमी॑वा॒ इत्यमी॑वाः । \newline
9. सु॒शर्म॑णो बृह॒तो बृ॑ह॒तः सु॒शर्म॑णः सु॒शर्म॑णो बृह॒तः शर्म॑णि॒ शर्म॑णि बृह॒तः सु॒शर्म॑णः सु॒शर्म॑णो बृह॒तः शर्म॑णि । \newline
10. सु॒शर्म॑ण॒ इति॑ सु - शर्म॑णः । \newline
11. बृ॒ह॒तः शर्म॑णि॒ शर्म॑णि बृह॒तो बृ॑ह॒तः शर्म॑णि स्याꣳ स्याꣳ॒॒ शर्म॑णि बृह॒तो बृ॑ह॒तः शर्म॑णि स्याम् । \newline
12. शर्म॑णि स्याꣳ स्याꣳ॒॒ शर्म॑णि॒ शर्म॑णि स्या म॒ग्ने र॒ग्नेः स्याꣳ॒॒ शर्म॑णि॒ शर्म॑णि स्या म॒ग्नेः । \newline
13. स्या॒ म॒ग्ने र॒ग्नेः स्याꣳ॑ स्या म॒ग्ने र॒ह म॒ह म॒ग्नेः स्याꣳ॑ स्या म॒ग्ने र॒हम् । \newline
14. अ॒ग्ने र॒ह म॒ह म॒ग्ने र॒ग्ने र॒हꣳ सु॒हव॑स्य सु॒हव॑ स्या॒ह म॒ग्ने र॒ग्ने र॒हꣳ सु॒हव॑स्य । \newline
15. अ॒हꣳ सु॒हव॑स्य सु॒हव॑ स्या॒ह म॒हꣳ सु॒हव॑स्य॒ प्रणी॑तौ॒ प्रणी॑तौ सु॒हव॑ स्या॒ह म॒हꣳ सु॒हव॑स्य॒ प्रणी॑तौ । \newline
16. सु॒हव॑स्य॒ प्रणी॑तौ॒ प्रणी॑तौ सु॒हव॑स्य सु॒हव॑स्य॒ प्रणी॑तौ । \newline
17. सु॒हव॒स्येति॑ सु - हव॑स्य । \newline
18. प्रणी॑ता॒विति॒ प्र - नी॒तौ॒ । \newline
19. आपो॒ हि ह्याप॒ आपो॒ हि ष्ठ स्थ ह्याप॒ आपो॒ हि ष्ठ । \newline
20. हि ष्ठ स्थ हि हि ष्ठा म॑यो॒भुवो॑ मयो॒भुवः॒ स्थ हि हि ष्ठा म॑यो॒भुवः॑ । \newline
21. स्था म॑यो॒भुवो॑ मयो॒भुवः॒ स्थ स्था म॑यो॒भुव॒ स्ता स्ता म॑यो॒भुवः॒ स्थ स्था म॑यो॒भुव॒ स्ताः । \newline
22. म॒यो॒भुव॒ स्ता स्ता म॑यो॒भुवो॑ मयो॒भुव॒ स्ता नो॑ न॒ स्ता म॑यो॒भुवो॑ मयो॒भुव॒ स्ता नः॑ । \newline
23. म॒यो॒भुव॒ इति॑ मयः - भुवः॑ । \newline
24. ता नो॑ न॒ स्ता स्ता न॑ ऊ॒र्ज ऊ॒र्जे न॒ स्ता स्ता न॑ ऊ॒र्जे । \newline
25. न॒ ऊ॒र्ज ऊ॒र्जे नो॑ न ऊ॒र्जे द॑धातन दधातनो॒र्जे नो॑ न ऊ॒र्जे द॑धातन । \newline
26. ऊ॒र्जे द॑धातन दधातनो॒र्ज ऊ॒र्जे द॑धातन । \newline
27. द॒धा॒त॒नेति॑ दधातन । \newline
28. म॒हे रणा॑य॒ रणा॑य म॒हे म॒हे रणा॑य॒ चक्ष॑से॒ चक्ष॑से॒ रणा॑य म॒हे म॒हे रणा॑य॒ चक्ष॑से । \newline
29. रणा॑य॒ चक्ष॑से॒ चक्ष॑से॒ रणा॑य॒ रणा॑य॒ चक्ष॑से । \newline
30. चक्ष॑स॒ इति॒ चक्ष॑से । \newline
31. यो वो॑ वो॒ यो यो वः॑ शि॒वत॑मः शि॒वत॑मो वो॒ यो यो वः॑ शि॒वत॑मः । \newline
32. वः॒ शि॒वत॑मः शि॒वत॑मो वो वः शि॒वत॑मो॒ रसो॒ रसः॑ शि॒वत॑मो वो वः शि॒वत॑मो॒ रसः॑ । \newline
33. शि॒वत॑मो॒ रसो॒ रसः॑ शि॒वत॑मः शि॒वत॑मो॒ रस॒ स्तस्य॒ तस्य॒ रसः॑ शि॒वत॑मः शि॒वत॑मो॒ रस॒ स्तस्य॑ । \newline
34. शि॒वत॑म॒ इति॑ शि॒व - त॒मः॒ । \newline
35. रस॒ स्तस्य॒ तस्य॒ रसो॒ रस॒ स्तस्य॑ भाजयत भाजयत॒ तस्य॒ रसो॒ रस॒ स्तस्य॑ भाजयत । \newline
36. तस्य॑ भाजयत भाजयत॒ तस्य॒ तस्य॑ भाजयते॒ हेह भा॑जयत॒ तस्य॒ तस्य॑ भाजयते॒ह । \newline
37. भा॒ज॒य॒ते॒ हेह भा॑जयत भाजयते॒ह नो॑ न इ॒ह भा॑जयत भाजयते॒ह नः॑ । \newline
38. इ॒ह नो॑ न इ॒हेह नः॑ । \newline
39. न॒ इति॑ नः । \newline
40. उ॒श॒ती रि॑वे वोश॒ती रु॑श॒ती रि॑व मा॒तरो॑ मा॒तर॑ इवोश॒ती रु॑श॒ती रि॑व मा॒तरः॑ । \newline
41. इ॒व॒ मा॒तरो॑ मा॒तर॑ इवेव मा॒तरः॑ । \newline
42. मा॒तर॒ इति॑ मा॒तरः॑ । \newline
43. तस्मा॒ अर॒ मर॒म् तस्मै॒ तस्मा॒ अर॑म् गमाम गमा॒मा र॒म् तस्मै॒ तस्मा॒ अर॑म् गमाम । \newline
44. अर॑म् गमाम गमा॒मा र॒ मर॑म् गमाम वो वो गमा॒मा र॒ मर॑म् गमाम वः । \newline
45. ग॒मा॒म॒ वो॒ वो॒ ग॒मा॒म॒ ग॒मा॒म॒ वो॒ यस्य॒ यस्य॑ वो गमाम गमाम वो॒ यस्य॑ । \newline
46. वो॒ यस्य॒ यस्य॑ वो वो॒ यस्य॒ क्षया॑य॒ क्षया॑य॒ यस्य॑ वो वो॒ यस्य॒ क्षया॑य । \newline
47. यस्य॒ क्षया॑य॒ क्षया॑य॒ यस्य॒ यस्य॒ क्षया॑य॒ जिन्व॑थ॒ जिन्व॑थ॒ क्षया॑य॒ यस्य॒ यस्य॒ क्षया॑य॒ जिन्व॑थ । \newline
48. क्षया॑य॒ जिन्व॑थ॒ जिन्व॑थ॒ क्षया॑य॒ क्षया॑य॒ जिन्व॑थ । \newline
49. जिन्व॒थेति॒ जिन्व॑थ । \newline
50. आपो॑ ज॒नय॑थ ज॒नय॒थाप॒ आपो॑ ज॒नय॑था च च ज॒नय॒थाप॒ आपो॑ ज॒नय॑था च । \newline
51. ज॒नय॑था च च ज॒नय॑थ ज॒नय॑था च नो नश्च ज॒नय॑थ ज॒नय॑था च नः । \newline
52. च॒ नो॒ न॒ श्च॒ च॒ नः॒ । \newline
53. न॒ इति॑ नः । \newline
54. मि॒त्रः सꣳ॒॒सृज्य॑ सꣳ॒॒सृज्य॑ मि॒त्रो मि॒त्रः सꣳ॒॒सृज्य॑ पृथि॒वीम् पृ॑थि॒वीꣳ सꣳ॒॒सृज्य॑ मि॒त्रो मि॒त्रः सꣳ॒॒सृज्य॑ पृथि॒वीम् । \newline
\pagebreak
\markright{ TS 4.1.5.2  \hfill https://www.vedavms.in \hfill}

\section{ TS 4.1.5.2 }

\textbf{TS 4.1.5.2 } \newline
\textbf{Samhita Paata} \newline

सꣳ॒॒सृज्य॑ पृथि॒वीं भूमिं॑ च॒ ज्योति॑षा स॒ह । सुजा॑तं जा॒तवे॑दसम॒ग्निं ॅवै᳚श्वान॒रं ॅवि॒भुं ॥ अ॒य॒क्ष्माय॑ त्वा॒ सꣳ सृ॑जामि प्र॒जाभ्यः॑ । विश्वे᳚ त्वा दे॒वा वै᳚श्वान॒राः सꣳ सृ॑ज॒न्त्वा-नु॑ष्टुभेन॒ छन्द॑साऽङ्गिर॒स्वत् ॥ रु॒द्राः स॒भृंत्य॑ पृथि॒वीं बृ॒हज्ज्योतिः॒ समी॑धिरे । तेषां᳚ भा॒नुरज॑स्र॒ इच्छु॒क्रो दे॒वेषु॑ रोचते ॥ सꣳ सृ॑ष्टां॒ ॅवसु॑भी रु॒द्रैर्द्धीरैः᳚ कर्म॒ण्यां᳚ मृदं᳚ । हस्ता᳚भ्यां मृ॒द्वीं कृ॒त्वा सि॑नीवा॒ली क॑रोतु॒ - [  ] \newline

\textbf{Pada Paata} \newline

सꣳ॒॒सृज्येति॑ सं - सृज्य॑ । पृ॒थि॒वीम् । भूमि᳚म् । च॒ । ज्योति॑षा । स॒ह ॥ सुजा॑त॒मिति॒ सु - जा॒त॒म् । जा॒तवे॑दस॒मिति॑ जा॒त-वे॒द॒स॒म् । अ॒ग्निम् । वै॒श्वा॒न॒रम् । वि॒भुमिति॑ वि - भुम् ॥ अ॒य॒क्ष्माय॑ । त्वा॒ । समिति॑ । सृ॒जा॒मि॒ । प्र॒जाभ्य॒ इति॑ प्र-जाभ्यः॑ ॥ विश्वे᳚ । त्वा॒ । दे॒वाः । वै॒श्वा॒न॒राः । समिति॑ । सृ॒ज॒न्तु॒ । आनु॑ष्टुभे॒नेत्यानु॑-स्तु॒भे॒न॒ । छन्द॑सा । अ॒ङ्गि॒र॒स्वत् ॥ रु॒द्राः । स॒भृंत्येति॑ सं - भृत्य॑ । पृ॒थि॒वीम् । बृ॒हत् । ज्योतिः॑ । समिति॑ । ई॒धि॒रे॒ ॥ तेषा᳚म् । भा॒नुः । अज॑स्रः । इत् । शु॒क्रः । दे॒वेषु॑ । रो॒च॒ते॒ ॥ सꣳसृ॑ष्टा॒मिति॒ सं - सृ॒ष्टा॒म् । वसु॑भि॒रिति॒ वसु॑ - भिः॒ । रु॒द्रैः । धीरैः᳚ । क॒र्म॒ण्या᳚म् । मृद᳚म् ॥ हस्ता᳚भ्याम् । मृ॒द्वीम् । कृ॒त्वा । सि॒नी॒वा॒ली । क॒रो॒तु॒ ।  \newline


\textbf{Krama Paata} \newline

सꣳ॒॒सृज्य॑ पृथि॒वीम् । सꣳ॒॒सृज्येति॑ सम् - सृज्य॑ । पृ॒थि॒वीम् भूमि᳚म् । भूमि॑म् च । च॒ ज्योति॑षा । ज्योति॑षा स॒ह । स॒हेति॑ स॒ह ॥ सुजा॑तम् जा॒तवे॑दसम् । सुजा॑त॒मिति॒ सु - जा॒त॒म् । जा॒तवे॑दसम॒ग्निम् । जा॒तवे॑दस॒मिति॑ जा॒त - वे॒द॒स॒म् । अ॒ग्निं ॅवै᳚श्वान॒रम् । वै॒श्वा॒न॒रं ॅवि॒भुम् । वि॒भुमिति॑ वि - भुम् ॥ अ॒य॒क्ष्माय॑ त्वा । त्वा॒ सम् । सꣳ सृ॑जामि । सृ॒जा॒मि॒ प्र॒जाभ्यः॑ । प्र॒जाभ्य॒ इति॑ प्र - जाभ्यः॑ ॥ विश्वे᳚ त्वा । त्वा॒ दे॒वाः । दे॒वा वै᳚श्वान॒राः । वै॒श्वा॒न॒राः सम् । सꣳ सृ॑जन्तु । सृ॒ज॒न्त्वानु॑ष्टुभेन । आनु॑ष्टुभेन॒ छन्द॑सा । आनु॑ष्टुभे॒नेत्यानु॑ - स्तु॒भे॒न॒ । छन्द॑सा ऽङ्गिर॒स्वत् । अ॒ङ्गि॒र॒स्वदित्य॑ङ्गिर॒स्वत् ॥ रु॒द्राः स॒म्भृत्य॑ । स॒म्भृत्य॑ पृथि॒वीम् । स॒म्भृत्येति॑ सम् - भृत्य॑ । पृ॒थि॒वीम् बृ॒हत् । बृ॒हज् ज्योतिः॑ । ज्योतिः॒ सम् । समी॑धिरे । ई॒धि॒र॒ इती॑धिरे ॥ तेषा᳚म् भा॒नुः । भा॒नुरज॑स्रः । अज॑स्र॒ इत् । इच्छु॒क्रः । शु॒क्रो दे॒वेषु॑ । दे॒वेषु॑ रोचते । रो॒च॒त॒ इति॑ रोचते ॥ सꣳसृ॑ष्टां॒ ॅवसु॑भिः । सꣳसृ॑ष्टा॒मिति॒ सम् - सृ॒ष्टा॒म् । वसु॑भी रु॒द्रैः । वसु॑भि॒रिति॒ वसु॑ - भिः॒ । रु॒द्रैर् धीरैः᳚ । धीरैः᳚ कर्म॒ण्या᳚म् । क॒र्म॒ण्या᳚म् मृद᳚म् । मृद॒मिति॒ मृद᳚म् ॥ हस्ता᳚भ्याम् मृ॒द्वीम् । मृ॒द्वीम् कृ॒त्वा । कृ॒त्वा सि॑नीवा॒ली । सि॒नी॒वा॒ली क॑रोतु । क॒रो॒तु॒ ताम् \newline

\textbf{Jatai Paata} \newline

1. सꣳ॒॒सृज्य॑ पृथि॒वीम् पृ॑थि॒वीꣳ सꣳ॒॒सृज्य॑ सꣳ॒॒सृज्य॑ पृथि॒वीम् । \newline
2. सꣳ॒॒सृज्येति॑ सं - सृज्य॑ । \newline
3. पृ॒थि॒वीम् भूमि॒म् भूमि॑म् पृथि॒वीम् पृ॑थि॒वीम् भूमि᳚म् । \newline
4. भूमि॑म् च च॒ भूमि॒म् भूमि॑म् च । \newline
5. च॒ ज्योति॑षा॒ ज्योति॑षा च च॒ ज्योति॑षा । \newline
6. ज्योति॑षा स॒ह स॒ह ज्योति॑षा॒ ज्योति॑षा स॒ह । \newline
7. स॒हेति॑ स॒ह । \newline
8. सुजा॑तम् जा॒तवे॑दसम् जा॒तवे॑दसꣳ॒॒ सुजा॑तꣳ॒॒ सुजा॑तम् जा॒तवे॑दसम् । \newline
9. सुजा॑त॒मिति॒ सु - जा॒त॒म् । \newline
10. जा॒तवे॑दस म॒ग्नि म॒ग्निम् जा॒तवे॑दसम् जा॒तवे॑दस म॒ग्निम् । \newline
11. जा॒तवे॑दस॒मिति॑ जा॒त - वे॒द॒स॒म् । \newline
12. अ॒ग्निं ॅवै᳚श्वान॒रं ॅवै᳚श्वान॒र म॒ग्नि म॒ग्निं ॅवै᳚श्वान॒रम् । \newline
13. वै॒श्वा॒न॒रं ॅवि॒भुं ॅवि॒भुं ॅवै᳚श्वान॒रं ॅवै᳚श्वान॒रं ॅवि॒भुम् । \newline
14. वि॒भुमिति॑ वि - भुम् । \newline
15. अ॒य॒क्ष्माय॑ त्वा त्वा ऽय॒क्ष्माया॑ य॒क्ष्माय॑ त्वा । \newline
16. त्वा॒ सꣳ सम् त्वा᳚ त्वा॒ सम् । \newline
17. सꣳ सृ॑जामि सृजामि॒ सꣳ सꣳ सृ॑जामि । \newline
18. सृ॒जा॒मि॒ प्र॒जाभ्यः॑ प्र॒जाभ्यः॑ सृजामि सृजामि प्र॒जाभ्यः॑ । \newline
19. प्र॒जाभ्य॒ इति॑ प्र - जाभ्यः॑ । \newline
20. विश्वे᳚ त्वा त्वा॒ विश्वे॒ विश्वे᳚ त्वा । \newline
21. त्वा॒ दे॒वा दे॒वा स्त्वा᳚ त्वा दे॒वाः । \newline
22. दे॒वा वै᳚श्वान॒रा वै᳚श्वान॒रा दे॒वा दे॒वा वै᳚श्वान॒राः । \newline
23. वै॒श्वा॒न॒राः सꣳ सं ॅवै᳚श्वान॒रा वै᳚श्वान॒राः सम् । \newline
24. सꣳ सृ॑जन्तु सृजन्तु॒ सꣳ सꣳ सृ॑जन्तु । \newline
25. सृ॒ज॒ न्त्वानु॑ष्टुभे॒ना नु॑ष्टुभेन सृजन्तु सृज॒ न्त्वानु॑ष्टुभेन । \newline
26. आनु॑ष्टुभेन॒ छन्द॑सा॒ छन्द॒सा ऽऽनु॑ष्टुभे॒ना नु॑ष्टुभेन॒ छन्द॑सा । \newline
27. आनु॑ष्टुभे॒नेत्यानु॑ - स्तु॒भे॒न॒ । \newline
28. छन्द॑सा ऽङ्गिर॒स्व द॑ङ्गिर॒स्वच् छन्द॑सा॒ छन्द॑सा ऽङ्गिर॒स्वत् । \newline
29. अ॒ङ्गि॒र॒स्वदित्य॑ङ्गिर॒स्वत् । \newline
30. रु॒द्राः सं॒भृत्य॑ सं॒भृत्य॑ रु॒द्रा रु॒द्राः सं॒भृत्य॑ । \newline
31. सं॒भृत्य॑ पृथि॒वीम् पृ॑थि॒वीꣳ सं॒भृत्य॑ सं॒भृत्य॑ पृथि॒वीम् । \newline
32. सं॒भृत्येति॑ सं - भृत्य॑ । \newline
33. पृ॒थि॒वीम् बृ॒हद् बृ॒हत् पृ॑थि॒वीम् पृ॑थि॒वीम् बृ॒हत् । \newline
34. बृ॒हज् ज्योति॒र् ज्योति॑र् बृ॒हद् बृ॒हज् ज्योतिः॑ । \newline
35. ज्योतिः॒ सꣳ सम् ज्योति॒र् ज्योतिः॒ सम् । \newline
36. स मी॑धिर ईधिरे॒ सꣳ स मी॑धिरे । \newline
37. ई॒धि॒र॒ इती॑धिरे । \newline
38. तेषा᳚म् भा॒नुर् भा॒नु स्तेषा॒म् तेषा᳚म् भा॒नुः । \newline
39. भा॒नु रज॒स्रो ऽज॑स्रो भा॒नुर् भा॒नु रज॑स्रः । \newline
40. अज॑स्र॒ इदि दज॒स्रो ऽज॑स्र॒ इत् । \newline
41. इच्छु॒क्रः शु॒क्र इदिच्छु॒क्रः । \newline
42. शु॒क्रो दे॒वेषु॑ दे॒वेषु॑ शु॒क्रः शु॒क्रो दे॒वेषु॑ । \newline
43. दे॒वेषु॑ रोचते रोचते दे॒वेषु॑ दे॒वेषु॑ रोचते । \newline
44. रो॒च॒त॒ इति॑ रोचते । \newline
45. सꣳसृ॑ष्टां॒ ॅवसु॑भि॒र् वसु॑भिः॒ सꣳसृ॑ष्टाꣳ॒॒ सꣳसृ॑ष्टां॒ ॅवसु॑भिः । \newline
46. सꣳसृ॑ष्टा॒मिति॒ सं - सृ॒ष्टा॒म् । \newline
47. वसु॑भी रु॒द्रै रु॒द्रैर् वसु॑भि॒र् वसु॑भी रु॒द्रैः । \newline
48. वसु॑भि॒रिति॒ वसु॑ - भिः॒ । \newline
49. रु॒द्रैर् धीरै॒र् धीरै॑ रु॒द्रै रु॒द्रैर् धीरैः᳚ । \newline
50. धीरैः᳚ कर्म॒ण्या᳚म् कर्म॒ण्या᳚म् धीरै॒र् धीरैः᳚ कर्म॒ण्या᳚म् । \newline
51. क॒र्म॒ण्या᳚म् मृद॒म् मृद॑म् कर्म॒ण्या᳚म् कर्म॒ण्या᳚म् मृद᳚म् । \newline
52. मृद॒मिति॒ मृद᳚म् । \newline
53. हस्ता᳚भ्याम् मृ॒द्वीम् मृ॒द्वीꣳ हस्ता᳚भ्याꣳ॒॒ हस्ता᳚भ्याम् मृ॒द्वीम् । \newline
54. मृ॒द्वीम् कृ॒त्वा कृ॒त्वा मृ॒द्वीम् मृ॒द्वीम् कृ॒त्वा । \newline
55. कृ॒त्वा सि॑नीवा॒ली सि॑नीवा॒ली कृ॒त्वा कृ॒त्वा सि॑नीवा॒ली । \newline
56. सि॒नी॒वा॒ली क॑रोतु करोतु सिनीवा॒ली सि॑नीवा॒ली क॑रोतु । \newline
57. क॒रो॒तु॒ ताम् ताम् क॑रोतु करोतु॒ ताम् । \newline

\textbf{Ghana Paata } \newline

1. सꣳ॒॒सृज्य॑ पृथि॒वीम् पृ॑थि॒वीꣳ सꣳ॒॒सृज्य॑ सꣳ॒॒सृज्य॑ पृथि॒वीम् भूमि॒म् भूमि॑म् पृथि॒वीꣳ सꣳ॒॒सृज्य॑ सꣳ॒॒सृज्य॑ पृथि॒वीम् भूमि᳚म् । \newline
2. सꣳ॒॒सृज्येति॑ सं - सृज्य॑ । \newline
3. पृ॒थि॒वीम् भूमि॒म् भूमि॑म् पृथि॒वीम् पृ॑थि॒वीम् भूमि॑म् च च॒ भूमि॑म् पृथि॒वीम् पृ॑थि॒वीम् भूमि॑म् च । \newline
4. भूमि॑म् च च॒ भूमि॒म् भूमि॑म् च॒ ज्योति॑षा॒ ज्योति॑षा च॒ भूमि॒म् भूमि॑म् च॒ ज्योति॑षा । \newline
5. च॒ ज्योति॑षा॒ ज्योति॑षा च च॒ ज्योति॑षा स॒ह स॒ह ज्योति॑षा च च॒ ज्योति॑षा स॒ह । \newline
6. ज्योति॑षा स॒ह स॒ह ज्योति॑षा॒ ज्योति॑षा स॒ह । \newline
7. स॒हेति॑ स॒ह । \newline
8. सुजा॑तम् जा॒तवे॑दसम् जा॒तवे॑दसꣳ॒॒ सुजा॑तꣳ॒॒ सुजा॑तम् जा॒तवे॑दस म॒ग्नि म॒ग्निम् जा॒तवे॑दसꣳ॒॒ सुजा॑तꣳ॒॒ सुजा॑तम् जा॒तवे॑दस म॒ग्निम् । \newline
9. सुजा॑त॒मिति॒ सु - जा॒त॒म् । \newline
10. जा॒तवे॑दस म॒ग्नि म॒ग्निम् जा॒तवे॑दसम् जा॒तवे॑दस म॒ग्निं ॅवै᳚श्वान॒रं ॅवै᳚श्वान॒र म॒ग्निम् जा॒तवे॑दसम् जा॒तवे॑दस म॒ग्निं ॅवै᳚श्वान॒रम् । \newline
11. जा॒तवे॑दस॒मिति॑ जा॒त - वे॒द॒स॒म् । \newline
12. अ॒ग्निं ॅवै᳚श्वान॒रं ॅवै᳚श्वान॒र म॒ग्नि म॒ग्निं ॅवै᳚श्वान॒रं ॅवि॒भुं ॅवि॒भुं ॅवै᳚श्वान॒र म॒ग्नि म॒ग्निं ॅवै᳚श्वान॒रं ॅवि॒भुम् । \newline
13. वै॒श्वा॒न॒रं ॅवि॒भुं ॅवि॒भुं ॅवै᳚श्वान॒रं ॅवै᳚श्वान॒रं ॅवि॒भुम् । \newline
14. वि॒भुमिति॑ वि - भुम् । \newline
15. अ॒य॒क्ष्माय॑ त्वा त्वा ऽय॒क्ष्माया॑ य॒क्ष्माय॑ त्वा॒ सꣳ सम् त्वा॑ ऽय॒क्ष्माया॑ य॒क्ष्माय॑ त्वा॒ सम् । \newline
16. त्वा॒ सꣳ सम् त्वा᳚ त्वा॒ सꣳ सृ॑जामि सृजामि॒ सम् त्वा᳚ त्वा॒ सꣳ सृ॑जामि । \newline
17. सꣳ सृ॑जामि सृजामि॒ सꣳ सꣳ सृ॑जामि प्र॒जाभ्यः॑ प्र॒जाभ्यः॑ सृजामि॒ सꣳ सꣳ सृ॑जामि प्र॒जाभ्यः॑ । \newline
18. सृ॒जा॒मि॒ प्र॒जाभ्यः॑ प्र॒जाभ्यः॑ सृजामि सृजामि प्र॒जाभ्यः॑ । \newline
19. प्र॒जाभ्य॒ इति॑ प्र - जाभ्यः॑ । \newline
20. विश्वे᳚ त्वा त्वा॒ विश्वे॒ विश्वे᳚ त्वा दे॒वा दे॒वा स्त्वा॒ विश्वे॒ विश्वे᳚ त्वा दे॒वाः । \newline
21. त्वा॒ दे॒वा दे॒वा स्त्वा᳚ त्वा दे॒वा वै᳚श्वान॒रा वै᳚श्वान॒रा दे॒वा स्त्वा᳚ त्वा दे॒वा वै᳚श्वान॒राः । \newline
22. दे॒वा वै᳚श्वान॒रा वै᳚श्वान॒रा दे॒वा दे॒वा वै᳚श्वान॒राः सꣳ सं ॅवै᳚श्वान॒रा दे॒वा दे॒वा वै᳚श्वान॒राः सम् । \newline
23. वै॒श्वा॒न॒राः सꣳ सं ॅवै᳚श्वान॒रा वै᳚श्वान॒राः सꣳ सृ॑जन्तु सृजन्तु॒ सं ॅवै᳚श्वान॒रा वै᳚श्वान॒राः सꣳ सृ॑जन्तु । \newline
24. सꣳ सृ॑जन्तु सृजन्तु॒ सꣳ सꣳ सृ॑ज॒न् त्वानु॑ष्टुभे॒ना नु॑ष्टुभेन सृजन्तु॒ सꣳ सꣳ सृ॑ज॒न् त्वानु॑ष्टुभेन । \newline
25. सृ॒ज॒न् त्वानु॑ष्टुभे॒ना नु॑ष्टुभेन सृजन्तु सृज॒न् त्वानु॑ष्टुभेन॒ छन्द॑सा॒ छन्द॒सा ऽऽनु॑ष्टुभेन सृजन्तु सृज॒न् त्वानु॑ष्टुभेन॒ छन्द॑सा । \newline
26. आनु॑ष्टुभेन॒ छन्द॑सा॒ छन्द॒सा ऽऽनु॑ष्टुभे॒ना नु॑ष्टुभेन॒ छन्द॑सा ऽङ्गिर॒स्व द॑ङ्गिर॒स्व च्छन्द॒सा ऽऽनु॑ष्टुभे॒ना नु॑ष्टुभेन॒ छन्द॑सा ऽङ्गिर॒स्वत् । \newline
27. आनु॑ष्टुभे॒नेत्यानु॑ - स्तु॒भे॒न॒ । \newline
28. छन्द॑सा ऽङ्गिर॒स्व द॑ङ्गिर॒स्व च्छन्द॑सा॒ छन्द॑सा ऽङ्गिर॒स्वत् । \newline
29. अ॒ङ्गि॒र॒स्वदित्य॑ङ्गिर॒स्वत् । \newline
30. रु॒द्राः सं॒भृत्य॑ सं॒भृत्य॑ रु॒द्रा रु॒द्राः सं॒भृत्य॑ पृथि॒वीम् पृ॑थि॒वीꣳ सं॒भृत्य॑ रु॒द्रा रु॒द्राः सं॒भृत्य॑ पृथि॒वीम् । \newline
31. सं॒भृत्य॑ पृथि॒वीम् पृ॑थि॒वीꣳ सं॒भृत्य॑ सं॒भृत्य॑ पृथि॒वीम् बृ॒हद् बृ॒हत् पृ॑थि॒वीꣳ सं॒भृत्य॑ सं॒भृत्य॑ पृथि॒वीम् बृ॒हत् । \newline
32. सं॒भृत्येति॑ सं - भृत्य॑ । \newline
33. पृ॒थि॒वीम् बृ॒हद् बृ॒हत् पृ॑थि॒वीम् पृ॑थि॒वीम् बृ॒हज् ज्योति॒र् ज्योति॑र् बृ॒हत् पृ॑थि॒वीम् पृ॑थि॒वीम् बृ॒हज् ज्योतिः॑ । \newline
34. बृ॒हज् ज्योति॒र् ज्योति॑र् बृ॒हद् बृ॒हज् ज्योतिः॒ सꣳ सम् ज्योति॑र् बृ॒हद् बृ॒हज् ज्योतिः॒ सम् । \newline
35. ज्योतिः॒ सꣳ सम् ज्योति॒र् ज्योतिः॒ समी॑धिर ईधिरे॒ सम् ज्योति॒र् ज्योतिः॒ समी॑धिरे । \newline
36. समी॑धिर ईधिरे॒ सꣳ समी॑धिरे । \newline
37. ई॒धि॒र॒ इती॑धिरे । \newline
38. तेषा᳚म् भा॒नुर् भा॒नु स्तेषा॒म् तेषा᳚म् भा॒नु रज॒स्रो ऽज॑स्रो भा॒नु स्तेषा॒म् तेषा᳚म् भा॒नु रज॑स्रः । \newline
39. भा॒नु रज॒स्रो ऽज॑स्रो भा॒नुर् भा॒नु रज॑स्र॒ इदि दज॑स्रो भा॒नुर् भा॒नु रज॑स्र॒ इत् । \newline
40. अज॑स्र॒ इदि दज॒स्रो ऽज॑स्र॒ इच्छु॒क्रः शु॒क्र इदज॒स्रो ऽज॑स्र॒ इच्छु॒क्रः । \newline
41. इच्छु॒क्रः शु॒क्र इदिच्छु॒क्रो दे॒वेषु॑ दे॒वेषु॑ शु॒क्र इदिच्छु॒क्रो दे॒वेषु॑ । \newline
42. शु॒क्रो दे॒वेषु॑ दे॒वेषु॑ शु॒क्रः शु॒क्रो दे॒वेषु॑ रोचते रोचते दे॒वेषु॑ शु॒क्रः शु॒क्रो दे॒वेषु॑ रोचते । \newline
43. दे॒वेषु॑ रोचते रोचते दे॒वेषु॑ दे॒वेषु॑ रोचते । \newline
44. रो॒च॒त॒ इति॑ रोचते । \newline
45. सꣳसृ॑ष्टां॒ ॅवसु॑भि॒र् वसु॑भिः॒ सꣳसृ॑ष्टाꣳ॒॒ सꣳसृ॑ष्टां॒ ॅवसु॑भी रु॒द्रै रु॒द्रैर् वसु॑भिः॒ सꣳसृ॑ष्टाꣳ॒॒ सꣳसृ॑ष्टां॒ ॅवसु॑भी रु॒द्रैः । \newline
46. सꣳसृ॑ष्टा॒मिति॒ सं - सृ॒ष्टा॒म् । \newline
47. वसु॑भी रु॒द्रै रु॒द्रैर् वसु॑भि॒र् वसु॑भी रु॒द्रैर् धीरै॒र् धीरै॑ रु॒द्रैर् वसु॑भि॒र् वसु॑भी रु॒द्रैर् धीरैः᳚ । \newline
48. वसु॑भि॒रिति॒ वसु॑ - भिः॒ । \newline
49. रु॒द्रैर् धीरै॒र् धीरै॑ रु॒द्रै रु॒द्रैर् धीरैः᳚ कर्म॒ण्या᳚म् कर्म॒ण्या᳚म् धीरै॑ रु॒द्रै रु॒द्रैर् धीरैः᳚ कर्म॒ण्या᳚म् । \newline
50. धीरैः᳚ कर्म॒ण्या᳚म् कर्म॒ण्या᳚म् धीरै॒र् धीरैः᳚ कर्म॒ण्या᳚म् मृद॒म् मृद॑म् कर्म॒ण्या᳚म् धीरै॒र् धीरैः᳚ कर्म॒ण्या᳚म् मृद᳚म् । \newline
51. क॒र्म॒ण्या᳚म् मृद॒म् मृद॑म् कर्म॒ण्या᳚म् कर्म॒ण्या᳚म् मृद᳚म् । \newline
52. मृद॒मिति॒ मृद᳚म् । \newline
53. हस्ता᳚भ्याम् मृ॒द्वीम् मृ॒द्वीꣳ हस्ता᳚भ्याꣳ॒॒ हस्ता᳚भ्याम् मृ॒द्वीम् कृ॒त्वा कृ॒त्वा मृ॒द्वीꣳ हस्ता᳚भ्याꣳ॒॒ हस्ता᳚भ्याम् मृ॒द्वीम् कृ॒त्वा । \newline
54. मृ॒द्वीम् कृ॒त्वा कृ॒त्वा मृ॒द्वीम् मृ॒द्वीम् कृ॒त्वा सि॑नीवा॒ली सि॑नीवा॒ली कृ॒त्वा मृ॒द्वीम् मृ॒द्वीम् कृ॒त्वा सि॑नीवा॒ली । \newline
55. कृ॒त्वा सि॑नीवा॒ली सि॑नीवा॒ली कृ॒त्वा कृ॒त्वा सि॑नीवा॒ली क॑रोतु करोतु सिनीवा॒ली कृ॒त्वा कृ॒त्वा सि॑नीवा॒ली क॑रोतु । \newline
56. सि॒नी॒वा॒ली क॑रोतु करोतु सिनीवा॒ली सि॑नीवा॒ली क॑रोतु॒ ताम् ताम् क॑रोतु सिनीवा॒ली सि॑नीवा॒ली क॑रोतु॒ ताम् । \newline
57. क॒रो॒तु॒ ताम् ताम् क॑रोतु करोतु॒ ताम् । \newline
\pagebreak
\markright{ TS 4.1.5.3  \hfill https://www.vedavms.in \hfill}

\section{ TS 4.1.5.3 }

\textbf{TS 4.1.5.3 } \newline
\textbf{Samhita Paata} \newline

तां ॥ सि॒नी॒वा॒ली सु॑कप॒र्दा सु॑कुरी॒रा स्वौ॑प॒शा । सा तुभ्य॑मदिते मह॒ ओखां द॑धातु॒ हस्त॑योः ॥ उ॒खां क॑रोतु॒ शक्त्या॑ बा॒हुभ्या॒-मदि॑तिर्द्धि॒या । मा॒ता पु॒त्रं ॅयथो॒पस्थे॒ साऽग्निं बि॑भर्तु॒ गर्भ॒ आ ॥ म॒खस्य॒ शिरो॑ऽसि य॒ज्ञ्स्य॑ प॒दे स्थः॑ । वस॑वस्त्वा कृण्वन्तु गाय॒त्रेण॒ छन्द॑सा ऽङ्गिर॒स्वत् पृ॑थि॒व्य॑सि रु॒द्रास्त्वा॑ कृण्वन्तु॒ त्रैष्टु॑भेन॒ छन्द॑सा ऽङ्गिर॒स्वद॒न्तरि॑क्षमस्या - [  ] \newline

\textbf{Pada Paata} \newline

ताम् ॥ सि॒नी॒वा॒ली । सु॒क॒प॒र्देति॑ सु - क॒प॒र्दा । सु॒कु॒री॒रेति॑ सु - कु॒री॒रा । स्वौ॒प॒शेति॑ सु - औ॒प॒शा ॥ सा । तुभ्य᳚म् । अ॒दि॒ते॒ । म॒हे॒ । एति॑ । उ॒खाम् । द॒धा॒तु॒ । हस्त॑योः ॥ उ॒खाम् । क॒रो॒तु॒ । शक्त्या᳚ । बा॒हुभ्या॒मिति॑ बा॒हु - भ्या॒म् । अदि॑तिः । धि॒या ॥ मा॒ता । पु॒त्रम् । यथा᳚ । उ॒पस्थ॒ इत्यु॒प - स्थे॒ । सा । अ॒ग्निम् । बि॒भ॒र्तु॒ । गर्भे᳚ । आ ॥ म॒खस्य॑ । शिरः॑ । अ॒सि॒ । य॒ज्ञ्स्य॑ । प॒दे इति॑ । स्थः॒ ॥ वस॑वः । त्वा॒ । कृ॒ण्व॒न्तु॒ । गा॒य॒त्रेण॑ । छन्द॑सा । अ॒ङ्गि॒र॒स्वत् । पृ॒थि॒वी । अ॒सि॒ । रु॒द्राः । त्वा॒ । कृ॒ण्व॒न्तु॒ । त्रैष्टु॑भेन । छन्द॑सा । अ॒ङ्गि॒र॒स्वत् । अ॒न्तरि॑क्षम् । अ॒सि॒ ।  \newline


\textbf{Krama Paata} \newline

तामिति॒ ताम् ॥ सि॒नी॒वा॒ली सु॑कप॒र्दा । सु॒क॒प॒र्दा सु॑कुरी॒रा । सु॒क॒प॒र्देति॑ सु - क॒प॒र्दा । सु॒कु॒री॒रा स्वौ॑प॒शा । सु॒कु॒री॒रेति॑ सु - कु॒री॒रा । स्वौ॒प॒शेति॑ सु - औ॒प॒शा ॥ सा तुभ्य᳚म् । तुभ्य॑मदिते । अ॒दि॒ते॒ म॒हे॒ । म॒ह॒ आ । ओखाम् । उ॒खाम् द॑धातु । द॒धा॒तु॒ हस्त॑योः । हस्त॑यो॒रिति॒ हस्त॑योः ॥ उ॒खाम् क॑रोतु । क॒रो॒तु॒ शक्त्या᳚ । शक्त्या॑ बा॒हुभ्या᳚म् । बा॒हुभ्या॒मदि॑तिः । बा॒हुभ्या॒मिति॑ बा॒हु - भ्या॒म् । अदि॑तिर् धि॒या । धि॒येति॑ धि॒या ॥ मा॒ता पु॒त्रम् । पु॒त्रं ॅयथा᳚ । यथो॒पस्थे᳚ । उ॒पस्थे॒ सा । उ॒पस्थ॒ इत्यु॒प - स्थे॒ । साऽग्निम् । अ॒ग्निम् बि॑भर्तु । बि॒भ॒र्तु॒ गर्भे᳚ । गर्भ॒ आ । एत्या ॥ म॒खस्य॒ शिरः॑ । शिरो॑ऽसि । अ॒सि॒ य॒ज्ञ्स्य॑ । य॒ज्ञ्स्य॑ प॒दे । प॒दे स्थः॑ । प॒दे इति॑ प॒दे । स्थ॒ इति॑ स्थः ॥ वस॑वस्त्वा । त्वा॒ कृ॒ण्व॒न्तु॒ । कृ॒ण्व॒न्तु॒ गा॒य॒त्रेण॑ । गा॒य॒त्रेण॒ छन्द॑सा । छन्द॑सा ऽङ्गिर॒स्वत् । अ॒ङ्गि॒र॒स्वत् पृ॑थि॒वी । पृ॒थि॒व्य॑सि । अ॒सि॒ रु॒द्राः । रु॒द्रास्त्वा᳚ । त्वा॒ कृ॒ण्व॒न्तु॒ । कृ॒ण्व॒न्तु॒ त्रैष्टु॑भेन । त्रैष्टु॑भेन॒ छन्द॑सा । छन्द॑सा ऽङ्गिर॒स्वत् । अ॒ङ्गि॒र॒स्वद॒न्तरि॑क्षम् । अ॒न्तरि॑क्षमसि । अ॒स्या॒दि॒त्याः \newline

\textbf{Jatai Paata} \newline

1. तामिति॒ ताम् । \newline
2. सि॒नी॒वा॒ली सु॑कप॒र्दा सु॑कप॒र्दा सि॑नीवा॒ली सि॑नीवा॒ली सु॑कप॒र्दा । \newline
3. सु॒क॒प॒र्दा सु॑कुरी॒रा सु॑कुरी॒रा सु॑कप॒र्दा सु॑कप॒र्दा सु॑कुरी॒रा । \newline
4. सु॒क॒प॒र्देति॑ सु - क॒प॒र्दा । \newline
5. सु॒कु॒री॒रा स्वौ॑प॒शा स्वौ॑प॒शा सु॑कुरी॒रा सु॑कुरी॒रा स्वौ॑प॒शा । \newline
6. सु॒कु॒री॒रेति॑ सु - कु॒री॒रा । \newline
7. स्वौ॒प॒शेति॑ सु - औ॒प॒शा । \newline
8. सा तुभ्य॒म् तुभ्यꣳ॒॒ सा सा तुभ्य᳚म् । \newline
9. तुभ्य॑ मदिते अदिते॒ तुभ्य॒म् तुभ्य॑ मदिते । \newline
10. अ॒दि॒ते॒ म॒हे॒ म॒हे॒ अ॒दि॒ते॒ अ॒दि॒ते॒ म॒हे॒ । \newline
11. म॒ह॒ आ म॑हे मह॒ आ । \newline
12. ओखा मु॒खा मोखाम् । \newline
13. उ॒खाम् द॑धातु दधातू॒खा मु॒खाम् द॑धातु । \newline
14. द॒धा॒तु॒ हस्त॑यो॒र्॒. हस्त॑योर् दधातु दधातु॒ हस्त॑योः । \newline
15. हस्त॑यो॒रिति॒ हस्त॑योः । \newline
16. उ॒खाम् क॑रोतु करोतू॒खा मु॒खाम् क॑रोतु । \newline
17. क॒रो॒तु॒ शक्त्या॒ शक्त्या॑ करोतु करोतु॒ शक्त्या᳚ । \newline
18. शक्त्या॑ बा॒हुभ्या᳚म् बा॒हुभ्याꣳ॒॒ शक्त्या॒ शक्त्या॑ बा॒हुभ्या᳚म् । \newline
19. बा॒हुभ्या॒ मदि॑ति॒ रदि॑तिर् बा॒हुभ्या᳚म् बा॒हुभ्या॒ मदि॑तिः । \newline
20. बा॒हुभ्या॒मिति॑ बा॒हु - भ्या॒म् । \newline
21. अदि॑तिर् धि॒या धि॒या ऽदि॑ति॒ रदि॑तिर् धि॒या । \newline
22. धि॒येति॑ धि॒या । \newline
23. मा॒ता पु॒त्रम् पु॒त्रम् मा॒ता मा॒ता पु॒त्रम् । \newline
24. पु॒त्रं ॅयथा॒ यथा॑ पु॒त्रम् पु॒त्रं ॅयथा᳚ । \newline
25. यथो॒पस्थ॑ उ॒पस्थे॒ यथा॒ यथो॒पस्थे᳚ । \newline
26. उ॒पस्थे॒ सा सोपस्थ॑ उ॒पस्थे॒ सा । \newline
27. उ॒पस्थ॒ इत्यु॒प - स्थे॒ । \newline
28. सा ऽग्नि म॒ग्निꣳ सा सा ऽग्निम् । \newline
29. अ॒ग्निम् बि॑भर्तु बिभर्त्व॒ग्नि म॒ग्निम् बि॑भर्तु । \newline
30. बि॒भ॒र्तु॒ गर्भे॒ गर्भे॑ बिभर्तु बिभर्तु॒ गर्भे᳚ । \newline
31. गर्भ॒ आ गर्भे॒ गर्भ॒ आ । \newline
32. एत्या । \newline
33. म॒खस्य॒ शिरः॒ शिरो॑ म॒खस्य॑ म॒खस्य॒ शिरः॑ । \newline
34. शिरो᳚ ऽस्यसि॒ शिरः॒ शिरो॑ ऽसि । \newline
35. अ॒सि॒ य॒ज्ञ्स्य॑ य॒ज्ञ्स्या᳚ स्यसि य॒ज्ञ्स्य॑ । \newline
36. य॒ज्ञ्स्य॑ प॒दे प॒दे य॒ज्ञ्स्य॑ य॒ज्ञ्स्य॑ प॒दे । \newline
37. प॒दे स्थः॑ स्थः प॒दे प॒दे स्थः॑ । \newline
38. प॒दे इति॑ प॒दे । \newline
39. स्थ॒ इति॑ स्थः । \newline
40. वस॑व स्त्वा त्वा॒ वस॑वो॒ वस॑व स्त्वा । \newline
41. त्वा॒ कृ॒ण्व॒न्तु॒ कृ॒ण्व॒न्तु॒ त्वा॒ त्वा॒ कृ॒ण्व॒न्तु॒ । \newline
42. कृ॒ण्व॒न्तु॒ गा॒य॒त्रेण॑ गाय॒त्रेण॑ कृण्वन्तु कृण्वन्तु गाय॒त्रेण॑ । \newline
43. गा॒य॒त्रेण॒ छन्द॑सा॒ छन्द॑सा गाय॒त्रेण॑ गाय॒त्रेण॒ छन्द॑सा । \newline
44. छन्द॑सा ऽङ्गिर॒स्व द॑ङ्गिर॒स्वच् छन्द॑सा॒ छन्द॑सा ऽङ्गिर॒स्वत् । \newline
45. अ॒ङ्गि॒र॒स्वत् पृ॑थि॒वी पृ॑थि॒ व्य॑ङ्गिर॒स्व द॑ङ्गिर॒स्वत् पृ॑थि॒वी । \newline
46. पृ॒थि॒ व्य॑स्यसि पृथि॒वी पृ॑थि॒व्य॑सि । \newline
47. अ॒सि॒ रु॒द्रा रु॒द्रा अ॑स्यसि रु॒द्राः । \newline
48. रु॒द्रा स्त्वा᳚ त्वा रु॒द्रा रु॒द्रा स्त्वा᳚ । \newline
49. त्वा॒ कृ॒ण्व॒न्तु॒ कृ॒ण्व॒न्तु॒ त्वा॒ त्वा॒ कृ॒ण्व॒न्तु॒ । \newline
50. कृ॒ण्व॒न्तु॒ त्रैष्टु॑भेन॒ त्रैष्टु॑भेन कृण्वन्तु कृण्वन्तु॒ त्रैष्टु॑भेन । \newline
51. त्रैष्टु॑भेन॒ छन्द॑सा॒ छन्द॑सा॒ त्रैष्टु॑भेन॒ त्रैष्टु॑भेन॒ छन्द॑सा । \newline
52. छन्द॑सा ऽङ्गिर॒स्व द॑ङ्गिर॒स्वच् छन्द॑सा॒ छन्द॑सा ऽङ्गिर॒स्वत् । \newline
53. अ॒ङ्गि॒र॒स्व द॒न्तरि॑क्ष म॒न्तरि॑क्ष मङ्गिर॒स्व द॑ङ्गिर॒स्व द॒न्तरि॑क्षम् । \newline
54. अ॒न्तरि॑क्ष मस्य स्य॒न्तरि॑क्ष म॒न्तरि॑क्ष मसि । \newline
55. अ॒स्या॒दि॒त्या आ॑दि॒त्या अ॑स्य स्यादि॒त्याः । \newline

\textbf{Ghana Paata } \newline

1. तामिति॒ ताम् । \newline
2. सि॒नी॒वा॒ली सु॑कप॒र्दा सु॑कप॒र्दा सि॑नीवा॒ली सि॑नीवा॒ली सु॑कप॒र्दा सु॑कुरी॒रा सु॑कुरी॒रा सु॑कप॒र्दा सि॑नीवा॒ली सि॑नीवा॒ली सु॑कप॒र्दा सु॑कुरी॒रा । \newline
3. सु॒क॒प॒र्दा सु॑कुरी॒रा सु॑कुरी॒रा सु॑कप॒र्दा सु॑कप॒र्दा सु॑कुरी॒रा स्वौ॑प॒शा स्वौ॑प॒शा सु॑कुरी॒रा सु॑कप॒र्दा सु॑कप॒र्दा सु॑कुरी॒रा स्वौ॑प॒शा । \newline
4. सु॒क॒प॒र्देति॑ सु - क॒प॒र्दा । \newline
5. सु॒कु॒री॒रा स्वौ॑प॒शा स्वौ॑प॒शा सु॑कुरी॒रा सु॑कुरी॒रा स्वौ॑प॒शा । \newline
6. सु॒कु॒री॒रेति॑ सु - कु॒री॒रा । \newline
7. स्वौ॒प॒शेति॑ सु - औ॒प॒शा । \newline
8. सा तुभ्य॒म् तुभ्यꣳ॒॒ सा सा तुभ्य॑ मदिते अदिते॒ तुभ्यꣳ॒॒ सा सा तुभ्य॑ मदिते । \newline
9. तुभ्य॑ मदिते अदिते॒ तुभ्य॒म् तुभ्य॑ मदिते महे महे अदिते॒ तुभ्य॒म् तुभ्य॑ मदिते महे । \newline
10. अ॒दि॒ते॒ म॒हे॒ म॒हे॒ अ॒दि॒ते॒ अ॒दि॒ते॒ म॒ह॒ आ म॑हे अदिते अदिते मह॒ आ । \newline
11. म॒ह॒ आ म॑हे मह॒ ओखा मु॒खा मा म॑हे मह॒ ओखाम् । \newline
12. ओखा मु॒खा मोखाम् द॑धातु दधातू॒खा मोखाम् द॑धातु । \newline
13. उ॒खाम् द॑धातु दधातू॒खा मु॒खाम् द॑धातु॒ हस्त॑यो॒र्॒. हस्त॑योर् दधातू॒खा मु॒खाम् द॑धातु॒ हस्त॑योः । \newline
14. द॒धा॒तु॒ हस्त॑यो॒र्॒. हस्त॑योर् दधातु दधातु॒ हस्त॑योः । \newline
15. हस्त॑यो॒रिति॒ हस्त॑योः । \newline
16. उ॒खाम् क॑रोतु करोतू॒खा मु॒खाम् क॑रोतु॒ शक्त्या॒ शक्त्या॑ करोतू॒खा मु॒खाम् क॑रोतु॒ शक्त्या᳚ । \newline
17. क॒रो॒तु॒ शक्त्या॒ शक्त्या॑ करोतु करोतु॒ शक्त्या॑ बा॒हुभ्या᳚म् बा॒हुभ्याꣳ॒॒ शक्त्या॑ करोतु करोतु॒ शक्त्या॑ बा॒हुभ्या᳚म् । \newline
18. शक्त्या॑ बा॒हुभ्या᳚म् बा॒हुभ्याꣳ॒॒ शक्त्या॒ शक्त्या॑ बा॒हुभ्या॒ मदि॑ति॒ रदि॑तिर् बा॒हुभ्याꣳ॒॒ शक्त्या॒ शक्त्या॑ बा॒हुभ्या॒ मदि॑तिः । \newline
19. बा॒हुभ्या॒ मदि॑ति॒ रदि॑तिर् बा॒हुभ्या᳚म् बा॒हुभ्या॒ मदि॑तिर् धि॒या धि॒या ऽदि॑तिर् बा॒हुभ्या᳚म् बा॒हुभ्या॒ मदि॑तिर् धि॒या । \newline
20. बा॒हुभ्या॒मिति॑ बा॒हु - भ्या॒म् । \newline
21. अदि॑तिर् धि॒या धि॒या ऽदि॑ति॒ रदि॑तिर् धि॒या । \newline
22. धि॒येति॑ धि॒या । \newline
23. मा॒ता पु॒त्रम् पु॒त्रम् मा॒ता मा॒ता पु॒त्रं ॅयथा॒ यथा॑ पु॒त्रम् मा॒ता मा॒ता पु॒त्रं ॅयथा᳚ । \newline
24. पु॒त्रं ॅयथा॒ यथा॑ पु॒त्रम् पु॒त्रं ॅयथो॒ पस्थ॑ उ॒पस्थे॒ यथा॑ पु॒त्रम् पु॒त्रं ॅयथो॒ पस्थे᳚ । \newline
25. यथो॒ पस्थ॑ उ॒पस्थे॒ यथा॒ यथो॒ पस्थे॒ सा सोपस्थे॒ यथा॒ यथो॒ पस्थे॒ सा । \newline
26. उ॒पस्थे॒ सा सोपस्थ॑ उ॒पस्थे॒ सा ऽग्नि म॒ग्निꣳ सोपस्थ॑ उ॒पस्थे॒ सा ऽग्निम् । \newline
27. उ॒पस्थ॒ इत्यु॒प - स्थे॒ । \newline
28. सा ऽग्नि म॒ग्निꣳ सा सा ऽग्निम् बि॑भर्तु बिभर्त्व॒ग्निꣳ सा सा ऽग्निम् बि॑भर्तु । \newline
29. अ॒ग्निम् बि॑भर्तु बिभर्त्व॒ग्नि म॒ग्निम् बि॑भर्तु॒ गर्भे॒ गर्भे॑ बिभर्त्व॒ग्नि म॒ग्निम् बि॑भर्तु॒ गर्भे᳚ । \newline
30. बि॒भ॒र्तु॒ गर्भे॒ गर्भे॑ बिभर्तु बिभर्तु॒ गर्भ॒ आ गर्भे॑ बिभर्तु बिभर्तु॒ गर्भ॒ आ । \newline
31. गर्भ॒ आ गर्भे॒ गर्भ॒ आ । \newline
32. एत्या । \newline
33. म॒खस्य॒ शिरः॒ शिरो॑ म॒खस्य॑ म॒खस्य॒ शिरो᳚ ऽस्यसि॒ शिरो॑ म॒खस्य॑ म॒खस्य॒ शिरो॑ ऽसि । \newline
34. शिरो᳚ ऽस्यसि॒ शिरः॒ शिरो॑ ऽसि य॒ज्ञ्स्य॑ य॒ज्ञ्स्या॑सि॒ शिरः॒ शिरो॑ ऽसि य॒ज्ञ्स्य॑ । \newline
35. अ॒सि॒ य॒ज्ञ्स्य॑ य॒ज्ञ्स्या᳚ स्यसि य॒ज्ञ्स्य॑ प॒दे प॒दे य॒ज्ञ्स्या᳚ स्यसि य॒ज्ञ्स्य॑ प॒दे । \newline
36. य॒ज्ञ्स्य॑ प॒दे प॒दे य॒ज्ञ्स्य॑ य॒ज्ञ्स्य॑ प॒दे स्थः॑ स्थः प॒दे य॒ज्ञ्स्य॑ य॒ज्ञ्स्य॑ प॒दे स्थः॑ । \newline
37. प॒दे स्थः॑ स्थः प॒दे प॒दे स्थः॑ । \newline
38. प॒दे इति॑ प॒दे । \newline
39. स्थ॒ इति॑ स्थः । \newline
40. वस॑व स्त्वा त्वा॒ वस॑वो॒ वस॑व स्त्वा कृण्वन्तु कृण्वन्तु त्वा॒ वस॑वो॒ वस॑व स्त्वा कृण्वन्तु । \newline
41. त्वा॒ कृ॒ण्व॒न्तु॒ कृ॒ण्व॒न्तु॒ त्वा॒ त्वा॒ कृ॒ण्व॒न्तु॒ गा॒य॒त्रेण॑ गाय॒त्रेण॑ कृण्वन्तु त्वा त्वा कृण्वन्तु गाय॒त्रेण॑ । \newline
42. कृ॒ण्व॒न्तु॒ गा॒य॒त्रेण॑ गाय॒त्रेण॑ कृण्वन्तु कृण्वन्तु गाय॒त्रेण॒ छन्द॑सा॒ छन्द॑सा गाय॒त्रेण॑ कृण्वन्तु कृण्वन्तु गाय॒त्रेण॒ छन्द॑सा । \newline
43. गा॒य॒त्रेण॒ छन्द॑सा॒ छन्द॑सा गाय॒त्रेण॑ गाय॒त्रेण॒ छन्द॑सा ऽङ्गिर॒स्व द॑ङ्गिर॒स्व च्छन्द॑सा गाय॒त्रेण॑ गाय॒त्रेण॒ छन्द॑सा ऽङ्गिर॒स्वत् । \newline
44. छन्द॑सा ऽङ्गिर॒स्व द॑ङ्गिर॒स्व च्छन्द॑सा॒ छन्द॑सा ऽङ्गिर॒स्वत् पृ॑थि॒वी पृ॑थि॒ व्य॑ङ्गिर॒स्व च्छन्द॑सा॒ छन्द॑सा ऽङ्गिर॒स्वत् पृ॑थि॒वी । \newline
45. अ॒ङ्गि॒र॒स्वत् पृ॑थि॒वी पृ॑थि॒ व्य॑ङ्गिर॒स्व द॑ङ्गिर॒स्वत् पृ॑थि॒ व्य॑स्यसि पृथि॒ व्य॑ङ्गिर॒स्व द॑ङ्गिर॒स्वत् पृ॑थि॒ व्य॑सि । \newline
46. पृ॒थि॒ व्य॑स्यसि पृथि॒वी पृ॑थि॒ व्य॑सि रु॒द्रा रु॒द्रा अ॑सि पृथि॒वी पृ॑थि॒ व्य॑सि रु॒द्राः । \newline
47. अ॒सि॒ रु॒द्रा रु॒द्रा अ॑स्यसि रु॒द्रा स्त्वा᳚ त्वा रु॒द्रा अ॑स्यसि रु॒द्रा स्त्वा᳚ । \newline
48. रु॒द्रा स्त्वा᳚ त्वा रु॒द्रा रु॒द्रा स्त्वा॑ कृण्वन्तु कृण्वन्तु त्वा रु॒द्रा रु॒द्रा स्त्वा॑ कृण्वन्तु । \newline
49. त्वा॒ कृ॒ण्व॒न्तु॒ कृ॒ण्व॒न्तु॒ त्वा॒ त्वा॒ कृ॒ण्व॒न्तु॒ त्रैष्टु॑भेन॒ त्रैष्टु॑भेन कृण्वन्तु त्वा त्वा कृण्वन्तु॒ त्रैष्टु॑भेन । \newline
50. कृ॒ण्व॒न्तु॒ त्रैष्टु॑भेन॒ त्रैष्टु॑भेन कृण्वन्तु कृण्वन्तु॒ त्रैष्टु॑भेन॒ छन्द॑सा॒ छन्द॑सा॒ त्रैष्टु॑भेन कृण्वन्तु कृण्वन्तु॒ त्रैष्टु॑भेन॒ छन्द॑सा । \newline
51. त्रैष्टु॑भेन॒ छन्द॑सा॒ छन्द॑सा॒ त्रैष्टु॑भेन॒ त्रैष्टु॑भेन॒ छन्द॑सा ऽङ्गिर॒स्व द॑ङ्गिर॒स्व च्छन्द॑सा॒ त्रैष्टु॑भेन॒ त्रैष्टु॑भेन॒ छन्द॑सा ऽङ्गिर॒स्वत् । \newline
52. छन्द॑सा ऽङ्गिर॒स्व द॑ङ्गिर॒स्व च्छन्द॑सा॒ छन्द॑सा ऽङ्गिर॒स्व द॒न्तरि॑क्ष म॒न्तरि॑क्ष मङ्गिर॒स्व च्छन्द॑सा॒ छन्द॑सा ऽङ्गिर॒स्व द॒न्तरि॑क्षम् । \newline
53. अ॒ङ्गि॒र॒स्व द॒न्तरि॑क्ष म॒न्तरि॑क्ष मङ्गिर॒स्व द॑ङ्गिर॒स्व द॒न्तरि॑क्ष मस्यस्य॒ न्तरि॑क्ष मङ्गिर॒स्व द॑ङ्गिर॒स्व द॒न्तरि॑क्ष मसि । \newline
54. अ॒न्तरि॑क्ष मस्यस्य॒ न्तरि॑क्ष म॒न्तरि॑क्ष मस्यादि॒त्या आ॑दि॒त्या अ॑स्य॒न्तरि॑क्ष म॒न्तरि॑क्ष मस्यादि॒त्याः । \newline
55. अ॒स्या॒दि॒त्या आ॑दि॒त्या अ॑स्य स्यादि॒त्या स्त्वा᳚ त्वा ऽऽदि॒त्या अ॑स्य स्यादि॒त्या स्त्वा᳚ । \newline
\pagebreak
\markright{ TS 4.1.5.4  \hfill https://www.vedavms.in \hfill}

\section{ TS 4.1.5.4 }

\textbf{TS 4.1.5.4 } \newline
\textbf{Samhita Paata} \newline

-दि॒त्यास्त्वा॑ कृण्वन्तु॒ जाग॑तेन॒ छन्द॑साऽङ्गिर॒स्वद् द्यौर॑सि॒ विश्वे᳚ त्वा दे॒वा वै᳚श्वान॒राः कृ॑ण्व॒न्त्वानु॑ष्टुभेन॒ छन्द॑सा-ऽङ्गिर॒स्वद्-दिशो॑ऽसि ध्रु॒वाऽसि॑ धा॒रया॒ मयि॑ प्र॒जाꣳ रा॒यस्पोषं॑ गौप॒त्यꣳ सु॒वीर्यꣳ॑ सजा॒तान्. यज॑माना॒याऽदि॑त्यै॒ रास्ना॒ऽस्य दि॑तिस्ते॒ बिलं॑ गृह्णातु॒ पाङ्क्ते॑न॒ छन्द॑सा ऽङ्गिर॒स्वत् ॥ कृ॒त्वाय॒ सा म॒हीमु॒खां मृ॒न्मयीं॒ ॅयोनि॑म॒ग्नये᳚ । तां पु॒त्रेभ्यः॒ सं प्रा ( ) य॑च्छ॒ददि॑तिः श्र॒पया॒निति॑ ॥ \newline

\textbf{Pada Paata} \newline

आ॒दि॒त्याः । त्वा॒ । कृ॒ण्व॒न्तु॒ । जाग॑तेन । छन्द॑सा । अ॒ङ्गि॒र॒स्वत् । द्यौः । अ॒सि॒ । विश्वे᳚ । त्वा॒ । दे॒वाः । वै॒श्वा॒न॒राः । कृ॒ण्व॒न्तु॒ । आनु॑ष्टुभे॒नेत्यानु॑ - स्तु॒भे॒न॒ । छन्द॑सा । अ॒ङ्गि॒र॒स्वत् । दिशः॑ । अ॒सि॒ । ध्रु॒वा । अ॒सि॒ । धा॒रय॑ । मयि॑ । प्र॒जामिति॑ प्र-जाम् । रा॒यः । पोष᳚म् । गौ॒प॒त्यम् । सु॒वीर्य॒मिति॑ सु - वीर्य᳚म् । स॒जा॒तानिति॑ स - जा॒तान् । यज॑मानाय । अदि॑त्यै । रास्ना᳚ । अ॒सि॒ । अदि॑तिः । ते॒ । बिल᳚म् । गृ॒ह्णा॒तु॒ । पाङ्क्ते॑न । छन्द॑सा । अ॒ङ्गि॒र॒स्वत् ॥ कृ॒त्वाय॑ । सा । म॒हीम् । उ॒खाम् । मृ॒न्मयी॒मिति॑ मृत्-मयी᳚म् । योनि᳚म् । अ॒ग्नये᳚ ॥ ताम् । पु॒त्रेभ्यः॑ । सम् । प्रेति॑ ( ) । अ॒य॒च्छ॒त् । अदि॑तिः । श्र॒पयान्॑ । इति॑ ॥  \newline


\textbf{Krama Paata} \newline

आ॒दि॒त्यास्त्वा᳚ । त्वा॒ कृ॒ण्व॒न्तु॒ । कृ॒ण्व॒न्तु॒ जाग॑तेन । जाग॑तेन॒ छन्द॑सा । छन्द॑सा ऽङ्गिर॒स्वत् । अ॒ङ्गि॒र॒स्वद् द्यौः । द्यौर॑सि । अ॒सि॒ विश्वे᳚ । विश्वे᳚ त्वा । त्वा॒ दे॒वाः । दे॒वा वै᳚श्वान॒राः । वै॒श्वा॒न॒राः कृ॑ण्वन्तु । कृ॒ण्व॒न्त्वानु॑ष्टुभेन । आनु॑ष्टुभेन॒ छन्द॑सा । आनु॑ष्टुभे॒नेत्यानु॑ - स्तु॒भे॒न॒ । छन्द॑सा ऽङ्गिर॒स्वत् । अ॒ङ्गि॒र॒स्वद् दिशः॑ । दिशो॑ऽसि । अ॒सि॒ ध्रु॒वा । ध्रु॒वाऽसि॑ । अ॒सि॒ धा॒रय॑ । धा॒रया॒ मयि॑ । मयि॑ प्र॒जाम् । प्र॒जाꣳ रा॒यः । प्र॒जामिति॑ प्र - जाम् । रा॒यस्पोष᳚म् । पोष॑म् गौप॒त्यम् । गौ॒प॒त्यꣳ सु॒वीर्य᳚म् । सु॒वीर्यꣳ॑ सजा॒तान् । सु॒वीर्य॒मिति॑ सु - वीर्य᳚म् । स॒जा॒तान्. यज॑मानाय । स॒जा॒तानिति॑ स - जा॒तान् । यज॑माना॒यादि॑त्यै । अदि॑त्यै॒ रास्ना᳚ । रास्ना॑ऽसि । अ॒स्यदि॑तिः । अदि॑तिस्ते । ते॒ बिल᳚म् । बिल॑म् गृह्णातु । गृ॒ह्णा॒तु॒ पाङ्क्ते॑न । पाङ्क्ते॑न॒ छन्द॑सा । छन्द॑सा ऽङ्गिर॒स्वत् । अ॒ङ्गि॒र॒स्वदित्य॑ङ्गिर॒स्वत् ॥ कृ॒त्वाय॒ सा । सा म॒हीम् । म॒हीमु॒खाम् । उ॒खाम् मृ॒न्मयी᳚म् । मृ॒न्मयीं॒ ॅयोनि᳚म् । मृ॒न्मयी॒मिति॑ मृत् - मयी᳚म् । योनि॑म॒ग्नये᳚ । अ॒ग्नय॒ इत्य॒ग्नये᳚ ॥ ताम् पु॒त्रेभ्यः॑ । पु॒त्रेभ्यः॒ सम् । सम् प्र ( ) । प्राय॑च्छत् । अ॒य॒च्छ॒ददि॑तिः । अदि॑तिः श्र॒पयान्॑ । श्र॒पया॒निति॑ । इतीतीति॑ । \newline

\textbf{Jatai Paata} \newline

1. आ॒दि॒त्या स्त्वा᳚ त्वा ऽऽदि॒त्या आ॑दि॒त्या स्त्वा᳚ । \newline
2. त्वा॒ कृ॒ण्व॒न्तु॒ कृ॒ण्व॒न्तु॒ त्वा॒ त्वा॒ कृ॒ण्व॒न्तु॒ । \newline
3. कृ॒ण्व॒न्तु॒ जाग॑तेन॒ जाग॑तेन कृण्वन्तु कृण्वन्तु॒ जाग॑तेन । \newline
4. जाग॑तेन॒ छन्द॑सा॒ छन्द॑सा॒ जाग॑तेन॒ जाग॑तेन॒ छन्द॑सा । \newline
5. छन्द॑सा ऽङ्गिर॒स्व द॑ङ्गिर॒स्वच् छन्द॑सा॒ छन्द॑सा ऽङ्गिर॒स्वत् । \newline
6. अ॒ङ्गि॒र॒स्वद् द्यौर् द्यौ र॑ङ्गिर॒स्व द॑ङ्गिर॒स्वद् द्यौः । \newline
7. द्यौ र॑स्यसि॒ द्यौर् द्यौ र॑सि । \newline
8. अ॒सि॒ विश्वे॒ विश्वे᳚ ऽस्यसि॒ विश्वे᳚ । \newline
9. विश्वे᳚ त्वा त्वा॒ विश्वे॒ विश्वे᳚ त्वा । \newline
10. त्वा॒ दे॒वा दे॒वा स्त्वा᳚ त्वा दे॒वाः । \newline
11. दे॒वा वै᳚श्वान॒रा वै᳚श्वान॒रा दे॒वा दे॒वा वै᳚श्वान॒राः । \newline
12. वै॒श्वा॒न॒राः कृ॑ण्वन्तु कृण्वन्तु वैश्वान॒रा वै᳚श्वान॒राः कृ॑ण्वन्तु । \newline
13. कृ॒ण्व॒ न्त्वानु॑ष्टुभे॒ना नु॑ष्टुभेन कृण्वन्तु कृण्व॒ न्त्वानु॑ष्टुभेन । \newline
14. आनु॑ष्टुभेन॒ छन्द॑सा॒ छन्द॒सा ऽऽनु॑ष्टुभे॒ना नु॑ष्टुभेन॒ छन्द॑सा । \newline
15. आनु॑ष्टुभे॒नेत्यानु॑ - स्तु॒भे॒न॒ । \newline
16. छन्द॑सा ऽङ्गिर॒स्व द॑ङ्गिर॒स्वच् छन्द॑सा॒ छन्द॑सा ऽङ्गिर॒स्वत् । \newline
17. अ॒ङ्गि॒र॒स्वद् दिशो॒ दिशो᳚ ऽङ्गिर॒स्व द॑ङ्गिर॒स्वद् दिशः॑ । \newline
18. दिशो᳚ ऽस्यसि॒ दिशो॒ दिशो॑ ऽसि । \newline
19. अ॒सि॒ ध्रु॒वा ध्रु॒वा ऽस्य॑सि ध्रु॒वा । \newline
20. ध्रु॒वा ऽस्य॑सि ध्रु॒वा ध्रु॒वा ऽसि॑ । \newline
21. अ॒सि॒ धा॒रय॑ धा॒रया᳚ स्यसि धा॒रय॑ । \newline
22. धा॒रया॒ मयि॒ मयि॑ धा॒रय॑ धा॒रया॒ मयि॑ । \newline
23. मयि॑ प्र॒जाम् प्र॒जाम् मयि॒ मयि॑ प्र॒जाम् । \newline
24. प्र॒जाꣳ रा॒यो रा॒यः प्र॒जाम् प्र॒जाꣳ रा॒यः । \newline
25. प्र॒जामिति॑ प्र - जाम् । \newline
26. रा॒य स्पोष॒म् पोषꣳ॑ रा॒यो रा॒य स्पोष᳚म् । \newline
27. पोष॑म् गौप॒त्यम् गौ॑प॒त्यम् पोष॒म् पोष॑म् गौप॒त्यम् । \newline
28. गौ॒प॒त्यꣳ सु॒वीर्यꣳ॑ सु॒वीर्य॑म् गौप॒त्यम् गौ॑प॒त्यꣳ सु॒वीर्य᳚म् । \newline
29. सु॒वीर्यꣳ॑ सजा॒तान् थ्स॑जा॒तान् थ्सु॒वीर्यꣳ॑ सु॒वीर्यꣳ॑ सजा॒तान् । \newline
30. सु॒वीर्य॒मिति॑ सु - वीर्य᳚म् । \newline
31. स॒जा॒तान्. यज॑मानाय॒ यज॑मानाय सजा॒तान् थ्स॑जा॒तान्. यज॑मानाय । \newline
32. स॒जा॒तानिति॑ स - जा॒तान् । \newline
33. यज॑माना॒या दि॑त्या॒ अदि॑त्यै॒ यज॑मानाय॒ यज॑माना॒या दि॑त्यै । \newline
34. अदि॑त्यै॒ रास्ना॒ रास्ना ऽदि॑त्या॒ अदि॑त्यै॒ रास्ना᳚ । \newline
35. रास्ना᳚ ऽस्यसि॒ रास्ना॒ रास्ना॑ ऽसि । \newline
36. अ॒स्यदि॑ति॒ रदि॑ति रस्य॒ स्यदि॑तिः । \newline
37. अदि॑ति स्ते ते॒ अदि॑ति॒ रदि॑ति स्ते । \newline
38. ते॒ बिल॒म् बिल॑म् ते ते॒ बिल᳚म् । \newline
39. बिल॑म् गृह्णातु गृह्णातु॒ बिल॒म् बिल॑म् गृह्णातु । \newline
40. गृ॒ह्णा॒तु॒ पाङ्क्ते॑न॒ पाङ्क्ते॑न गृह्णातु गृह्णातु॒ पाङ्क्ते॑न । \newline
41. पाङ्क्ते॑न॒ छन्द॑सा॒ छन्द॑सा॒ पाङ्क्ते॑न॒ पाङ्क्ते॑न॒ छन्द॑सा । \newline
42. छन्द॑सा ऽङ्गिर॒स्व द॑ङ्गिर॒स्वच् छन्द॑सा॒ छन्द॑सा ऽङ्गिर॒स्वत् । \newline
43. अ॒ङ्गि॒र॒स्व दित्य॑ङ्गिर॒स्वत् । \newline
44. कृ॒त्वाय॒ सा सा कृ॒त्वाय॑ कृ॒त्वाय॒ सा । \newline
45. सा म॒हीम् म॒हीꣳ सा सा म॒हीम् । \newline
46. म॒ही मु॒खा मु॒खाम् म॒हीम् म॒ही मु॒खाम् । \newline
47. उ॒खाम् मृ॒न्मयी᳚म् मृ॒न्मयी॑ मु॒खा मु॒खाम् मृ॒न्मयी᳚म् । \newline
48. मृ॒न्मयीं॒ ॅयोनिं॒ ॅयोनि॑म् मृ॒न्मयी᳚म् मृ॒न्मयीं॒ ॅयोनि᳚म् । \newline
49. मृ॒न्मयी॒मिति॑ मृत् - मयी᳚म् । \newline
50. योनि॑ म॒ग्नये॑ अ॒ग्नये॒ योनिं॒ ॅयोनि॑ म॒ग्नये᳚ । \newline
51. अ॒ग्नय॒ इत्य॒ग्नये᳚ । \newline
52. ताम् पु॒त्रेभ्यः॑ पु॒त्रेभ्य॒ स्ताम् ताम् पु॒त्रेभ्यः॑ । \newline
53. पु॒त्रेभ्यः॒ सꣳ सम् पु॒त्रेभ्यः॑ पु॒त्रेभ्यः॒ सम् । \newline
54. सम् प्र प्र सꣳ सम् प्र । \newline
55. प्राय॑च्छ दयच्छ॒त् प्र प्राय॑च्छत् । \newline
56. अ॒य॒च्छ॒ ददि॑ति॒ रदि॑ति रयच्छ दयच्छ॒ ददि॑तिः । \newline
57. अदि॑तिः श्र॒पया᳚ञ् छ्र॒पया॒ नदि॑ति॒ रदि॑तिः श्र॒पयान्॑ । \newline
58. श्र॒पया॒ नितीति॑ श्र॒पया᳚ञ् छ्र॒पया॒ निति॑ । \newline
59. इतीतीति॑ । \newline

\textbf{Ghana Paata } \newline

1. आ॒दि॒त्या स्त्वा᳚ त्वा ऽऽदि॒त्या आ॑दि॒त्या स्त्वा॑ कृण्वन्तु कृण्वन्तु त्वा ऽऽदि॒त्या आ॑दि॒त्या स्त्वा॑ कृण्वन्तु । \newline
2. त्वा॒ कृ॒ण्व॒न्तु॒ कृ॒ण्व॒न्तु॒ त्वा॒ त्वा॒ कृ॒ण्व॒न्तु॒ जाग॑तेन॒ जाग॑तेन कृण्वन्तु त्वा त्वा कृण्वन्तु॒ जाग॑तेन । \newline
3. कृ॒ण्व॒न्तु॒ जाग॑तेन॒ जाग॑तेन कृण्वन्तु कृण्वन्तु॒ जाग॑तेन॒ छन्द॑सा॒ छन्द॑सा॒ जाग॑तेन कृण्वन्तु कृण्वन्तु॒ जाग॑तेन॒ छन्द॑सा । \newline
4. जाग॑तेन॒ छन्द॑सा॒ छन्द॑सा॒ जाग॑तेन॒ जाग॑तेन॒ छन्द॑सा ऽङ्गिर॒स्व द॑ङ्गिर॒स्व च्छन्द॑सा॒ जाग॑तेन॒ जाग॑तेन॒ छन्द॑सा ऽङ्गिर॒स्वत् । \newline
5. छन्द॑सा ऽङ्गिर॒स्व द॑ङ्गिर॒स्व च्छन्द॑सा॒ छन्द॑सा ऽङ्गिर॒स्वद् द्यौर् द्यौ र॑ङ्गिर॒स्व च्छन्द॑सा॒ छन्द॑सा ऽङ्गिर॒स्वद् द्यौः । \newline
6. अ॒ङ्गि॒र॒स्वद् द्यौर् द्यौ र॑ङ्गिर॒स्व द॑ङ्गिर॒स्वद् द्यौ र॑स्यसि॒ द्यौ र॑ङ्गिर॒स्व द॑ङ्गिर॒स्वद् द्यौ र॑सि । \newline
7. द्यौ र॑स्यसि॒ द्यौर् द्यौर॑सि॒ विश्वे॒ विश्वे॑ ऽसि॒ द्यौर् द्यौ र॑सि॒ विश्वे᳚ । \newline
8. अ॒सि॒ विश्वे॒ विश्वे᳚ ऽस्यसि॒ विश्वे᳚ त्वा त्वा॒ विश्वे᳚ ऽस्यसि॒ विश्वे᳚ त्वा । \newline
9. विश्वे᳚ त्वा त्वा॒ विश्वे॒ विश्वे᳚ त्वा दे॒वा दे॒वा स्त्वा॒ विश्वे॒ विश्वे᳚ त्वा दे॒वाः । \newline
10. त्वा॒ दे॒वा दे॒वा स्त्वा᳚ त्वा दे॒वा वै᳚श्वान॒रा वै᳚श्वान॒रा दे॒वा स्त्वा᳚ त्वा दे॒वा वै᳚श्वान॒राः । \newline
11. दे॒वा वै᳚श्वान॒रा वै᳚श्वान॒रा दे॒वा दे॒वा वै᳚श्वान॒राः कृ॑ण्वन्तु कृण्वन्तु वैश्वान॒रा दे॒वा दे॒वा वै᳚श्वान॒राः कृ॑ण्वन्तु । \newline
12. वै॒श्वा॒न॒राः कृ॑ण्वन्तु कृण्वन्तु वैश्वान॒रा वै᳚श्वान॒राः कृ॑ण्व॒न् त्वानु॑ष्टुभे॒ना नु॑ष्टुभेन कृण्वन्तु वैश्वान॒रा वै᳚श्वान॒राः कृ॑ण्व॒न् त्वानु॑ष्टुभेन । \newline
13. कृ॒ण्व॒न् त्वानु॑ष्टुभे॒ना नु॑ष्टुभेन कृण्वन्तु कृण्व॒न् त्वानु॑ष्टुभेन॒ छन्द॑सा॒ छन्द॒सा ऽऽनु॑ष्टुभेन कृण्वन्तु कृण्व॒न् त्वानु॑ष्टुभेन॒ छन्द॑सा । \newline
14. आनु॑ष्टुभेन॒ छन्द॑सा॒ छन्द॒सा ऽऽनु॑ष्टुभे॒ना नु॑ष्टुभेन॒ छन्द॑सा ऽङ्गिर॒स्व द॑ङ्गिर॒स्व च्छन्द॒सा ऽऽनु॑ष्टुभे॒ना नु॑ष्टुभेन॒ छन्द॑सा ऽङ्गिर॒स्वत् । \newline
15. आनु॑ष्टुभे॒नेत्यानु॑ - स्तु॒भे॒न॒ । \newline
16. छन्द॑सा ऽङ्गिर॒स्व द॑ङ्गिर॒स्व च्छन्द॑सा॒ छन्द॑सा ऽङ्गिर॒स्वद् दिशो॒ दिशो᳚ ऽङ्गिर॒स्व च्छन्द॑सा॒ छन्द॑सा ऽङ्गिर॒स्वद् दिशः॑ । \newline
17. अ॒ङ्गि॒र॒स्वद् दिशो॒ दिशो᳚ ऽङ्गिर॒स्व द॑ङ्गिर॒स्वद् दिशो᳚ ऽस्यसि॒ दिशो᳚ ऽङ्गिर॒स्व द॑ङ्गिर॒स्वद् दिशो॑ ऽसि । \newline
18. दिशो᳚ ऽस्यसि॒ दिशो॒ दिशो॑ ऽसि ध्रु॒वा ध्रु॒वा ऽसि॒ दिशो॒ दिशो॑ ऽसि ध्रु॒वा । \newline
19. अ॒सि॒ ध्रु॒वा ध्रु॒वा ऽस्य॑सि ध्रु॒वा ऽस्य॑सि ध्रु॒वा ऽस्य॑सि ध्रु॒वा ऽसि॑ । \newline
20. ध्रु॒वा ऽस्य॑सि ध्रु॒वा ध्रु॒वा ऽसि॑ धा॒रय॑ धा॒रया॑सि ध्रु॒वा ध्रु॒वा ऽसि॑ धा॒रय॑ । \newline
21. अ॒सि॒ धा॒रय॑ धा॒रया᳚ स्यसि धा॒रया॒ मयि॒ मयि॑ धा॒रया᳚ स्यसि धा॒रया॒ मयि॑ । \newline
22. धा॒रया॒ मयि॒ मयि॑ धा॒रय॑ धा॒रया॒ मयि॑ प्र॒जाम् प्र॒जाम् मयि॑ धा॒रय॑ धा॒रया॒ मयि॑ प्र॒जाम् । \newline
23. मयि॑ प्र॒जाम् प्र॒जाम् मयि॒ मयि॑ प्र॒जाꣳ रा॒यो रा॒यः प्र॒जाम् मयि॒ मयि॑ प्र॒जाꣳ रा॒यः । \newline
24. प्र॒जाꣳ रा॒यो रा॒यः प्र॒जाम् प्र॒जाꣳ रा॒य स्पोष॒म् पोषꣳ॑ रा॒यः प्र॒जाम् प्र॒जाꣳ रा॒य स्पोष᳚म् । \newline
25. प्र॒जामिति॑ प्र - जाम् । \newline
26. रा॒य स्पोष॒म् पोषꣳ॑ रा॒यो रा॒य स्पोष॑म् गौप॒त्यम् गौ॑प॒त्यम् पोषꣳ॑ रा॒यो रा॒य स्पोष॑म् गौप॒त्यम् । \newline
27. पोष॑म् गौप॒त्यम् गौ॑प॒त्यम् पोष॒म् पोष॑म् गौप॒त्यꣳ सु॒वीर्यꣳ॑ सु॒वीर्य॑म् गौप॒त्यम् पोष॒म् पोष॑म् गौप॒त्यꣳ सु॒वीर्य᳚म् । \newline
28. गौ॒प॒त्यꣳ सु॒वीर्यꣳ॑ सु॒वीर्य॑म् गौप॒त्यम् गौ॑प॒त्यꣳ सु॒वीर्यꣳ॑ सजा॒तान् थ्स॑जा॒तान् थ्सु॒वीर्य॑म् गौप॒त्यम् गौ॑प॒त्यꣳ सु॒वीर्यꣳ॑ सजा॒तान् । \newline
29. सु॒वीर्यꣳ॑ सजा॒तान् थ्स॑जा॒तान् थ्सु॒वीर्यꣳ॑ सु॒वीर्यꣳ॑ सजा॒तान्. यज॑मानाय॒ यज॑मानाय सजा॒तान् थ्सु॒वीर्यꣳ॑ सु॒वीर्यꣳ॑ सजा॒तान्. यज॑मानाय । \newline
30. सु॒वीर्य॒मिति॑ सु - वीर्य᳚म् । \newline
31. स॒जा॒तान्. यज॑मानाय॒ यज॑मानाय सजा॒तान् थ्स॑जा॒तान्. यज॑माना॒या दि॑त्या॒ अदि॑त्यै॒ यज॑मानाय सजा॒तान् थ्स॑जा॒तान्. यज॑माना॒या दि॑त्यै । \newline
32. स॒जा॒तानिति॑ स - जा॒तान् । \newline
33. यज॑माना॒या दि॑त्या॒ अदि॑त्यै॒ यज॑मानाय॒ यज॑माना॒या दि॑त्यै॒ रास्ना॒ रास्ना ऽदि॑त्यै॒ यज॑मानाय॒ यज॑माना॒या दि॑त्यै॒ रास्ना᳚ । \newline
34. अदि॑त्यै॒ रास्ना॒ रास्ना ऽदि॑त्या॒ अदि॑त्यै॒ रास्ना᳚ ऽस्यसि॒ रास्ना ऽदि॑त्या॒ अदि॑त्यै॒ रास्ना॑ ऽसि । \newline
35. रास्ना᳚ ऽस्यसि॒ रास्ना॒ रास्ना॒ ऽस्यदि॑ति॒ रदि॑ति रसि॒ रास्ना॒ रास्ना॒ ऽस्यदि॑तिः । \newline
36. अ॒स्य दि॑ति॒ रदि॑ति रस्य॒ स्यदि॑ति स्ते ते॒ अदि॑ति रस्य॒ स्यदि॑ति स्ते । \newline
37. अदि॑ति स्ते ते॒ अदि॑ति॒ रदि॑ति स्ते॒ बिल॒म् बिल॑म् ते॒ अदि॑ति॒ रदि॑ति स्ते॒ बिल᳚म् । \newline
38. ते॒ बिल॒म् बिल॑म् ते ते॒ बिल॑म् गृह्णातु गृह्णातु॒ बिल॑म् ते ते॒ बिल॑म् गृह्णातु । \newline
39. बिल॑म् गृह्णातु गृह्णातु॒ बिल॒म् बिल॑म् गृह्णातु॒ पाङ्क्ते॑न॒ पाङ्क्ते॑न गृह्णातु॒ बिल॒म् बिल॑म् गृह्णातु॒ पाङ्क्ते॑न । \newline
40. गृ॒ह्णा॒तु॒ पाङ्क्ते॑न॒ पाङ्क्ते॑न गृह्णातु गृह्णातु॒ पाङ्क्ते॑न॒ छन्द॑सा॒ छन्द॑सा॒ पाङ्क्ते॑न गृह्णातु गृह्णातु॒ पाङ्क्ते॑न॒ छन्द॑सा । \newline
41. पाङ्क्ते॑न॒ छन्द॑सा॒ छन्द॑सा॒ पाङ्क्ते॑न॒ पाङ्क्ते॑न॒ छन्द॑सा ऽङ्गिर॒स्व द॑ङ्गिर॒स्व च्छन्द॑सा॒ पाङ्क्ते॑न॒ पाङ्क्ते॑न॒ छन्द॑सा ऽङ्गिर॒स्वत् । \newline
42. छन्द॑सा ऽङ्गिर॒स्व द॑ङ्गिर॒स्व च्छन्द॑सा॒ छन्द॑सा ऽङ्गिर॒स्वत् । \newline
43. अ॒ङ्गि॒र॒स्वदित्य॑ङ्गिर॒स्वत् । \newline
44. कृ॒त्वाय॒ सा सा कृ॒त्वाय॑ कृ॒त्वाय॒ सा म॒हीम् म॒हीꣳ सा कृ॒त्वाय॑ कृ॒त्वाय॒ सा म॒हीम् । \newline
45. सा म॒हीम् म॒हीꣳ सा सा म॒ही मु॒खा मु॒खाम् म॒हीꣳ सा सा म॒ही मु॒खाम् । \newline
46. म॒ही मु॒खा मु॒खाम् म॒हीम् म॒ही मु॒खाम् मृ॒न्मयी᳚म् मृ॒न्मयी॑ मु॒खाम् म॒हीम् म॒ही मु॒खाम् मृ॒न्मयी᳚म् । \newline
47. उ॒खाम् मृ॒न्मयी᳚म् मृ॒न्मयी॑ मु॒खा मु॒खाम् मृ॒न्मयीं॒ ॅयोनिं॒ ॅयोनि॑म् मृ॒न्मयी॑ मु॒खा मु॒खाम् मृ॒न्मयीं॒ ॅयोनि᳚म् । \newline
48. मृ॒न्मयीं॒ ॅयोनिं॒ ॅयोनि॑म् मृ॒न्मयी᳚म् मृ॒न्मयीं॒ ॅयोनि॑ म॒ग्नये॑ अ॒ग्नये॒ योनि॑म् मृ॒न्मयी᳚म् मृ॒न्मयीं॒ ॅयोनि॑ म॒ग्नये᳚ । \newline
49. मृ॒न्मयी॒मिति॑ मृत् - मयी᳚म् । \newline
50. योनि॑ म॒ग्नये॑ अ॒ग्नये॒ योनिं॒ ॅयोनि॑ म॒ग्नये᳚ । \newline
51. अ॒ग्नय॒ इत्य॒ग्नये᳚ । \newline
52. ताम् पु॒त्रेभ्यः॑ पु॒त्रेभ्य॒ स्ताम् ताम् पु॒त्रेभ्यः॒ सꣳ सम् पु॒त्रेभ्य॒ स्ताम् ताम् पु॒त्रेभ्यः॒ सम् । \newline
53. पु॒त्रेभ्यः॒ सꣳ सम् पु॒त्रेभ्यः॑ पु॒त्रेभ्यः॒ सम् प्र प्र सम् पु॒त्रेभ्यः॑ पु॒त्रेभ्यः॒ सम् प्र । \newline
54. सम् प्र प्र सꣳ सम् प्राय॑च्छ दयच्छ॒त् प्र सꣳ सम् प्राय॑च्छत् । \newline
55. प्राय॑च्छ दयच्छ॒त् प्र प्राय॑च्छ॒ ददि॑ति॒ रदि॑ति रयच्छ॒त् प्र प्रा य॑च्छ॒ ददि॑तिः । \newline
56. अ॒य॒च्छ॒ ददि॑ति॒ रदि॑ति रयच्छ दयच्छ॒ ददि॑तिः श्र॒पया᳚ञ् छ्र॒पया॒ नदि॑ति रयच्छ दयच्छ॒ ददि॑तिः श्र॒पयान्॑ । \newline
57. अदि॑तिः श्र॒पया᳚ञ् छ्र॒पया॒ नदि॑ति॒ रदि॑तिः श्र॒पया॒ नितीति॑ श्र॒पया॒ नदि॑ति॒ रदि॑तिः श्र॒पया॒ निति॑ । \newline
58. श्र॒पया॒ नितीति॑ श्र॒पया᳚ञ् छ्र॒पया॒ निति॑ । \newline
59. इतीतीति॑ । \newline
\pagebreak
\markright{ TS 4.1.6.1  \hfill https://www.vedavms.in \hfill}

\section{ TS 4.1.6.1 }

\textbf{TS 4.1.6.1 } \newline
\textbf{Samhita Paata} \newline

वस॑वस्त्वा धूपयन्तु गाय॒त्रेण॒ छन्द॑साऽङ्गिर॒स्वद्-रु॒द्रास्त्वा॑ धूपयन्तु॒ त्रैष्टु॑भेन॒ छन्द॑साऽङ्गिर॒स्व-दा॑दि॒त्यास्त्वा॑ धूपयन्तु॒ जाग॑तेन॒ छन्द॑साऽङ्गिर॒स्वद्- विश्वे᳚ त्वा दे॒वा वै᳚श्वान॒रा धू॑पय॒न्त्वानु॑ष्टुभेन॒ छन्द॑साऽङ्गिर॒स्व-दिन्द्र॑स्त्वा धूपयत्वङ्गिर॒स्वद् -विष्णु॑स्त्वा धूपयत्वङ्गिर॒स्वद्-वरु॑णस्त्वा धूपयत्वङ्गिर॒स्व-ददि॑तिस्त्वा दे॒वी वि॒श्वदे᳚व्यावती पृथि॒व्याः स॒धस्थे᳚ऽङ्गिर॒स्वत् ख॑नत्ववट दे॒वानां᳚ त्वा॒ पत्नी᳚ - [  ] \newline

\textbf{Pada Paata} \newline

वस॑वः । त्वा॒ । धू॒प॒य॒न्तु॒ । गा॒य॒त्रेण॑ । छन्द॑सा । अ॒ङ्गि॒र॒स्वत् । रु॒द्राः । त्वा॒ । धू॒प॒य॒न्तु॒ । त्रैष्टु॑भेन । छन्द॑सा । अ॒ङ्गि॒र॒स्वत् । आ॒दि॒त्याः । त्वा॒ । धू॒प॒य॒न्तु॒ । जाग॑तेन । छन्द॑सा । अ॒ङ्गि॒र॒स्वत् । विश्वे᳚ । त्वा॒ । दे॒वाः । वै॒श्वा॒न॒राः । धू॒प॒य॒न्तु॒ । आनु॑ष्टुभे॒नेत्यानु॑ - स्तु॒भे॒न॒ । छन्द॑सा । अ॒ङ्गि॒र॒स्वत् । इन्द्रः॑ । त्वा॒ । धू॒प॒य॒तु॒ । अ॒ङ्गि॒र॒स्वत् । विष्णुः॑ । त्वा॒ । धू॒प॒य॒तु॒ । अ॒ङ्गि॒र॒स्वत् । वरु॑णः । त्वा॒ । धू॒प॒य॒तु॒ । अ॒ङ्गि॒र॒स्वत् । अदि॑तिः । त्वा॒ । दे॒वी । वि॒श्वदे᳚व्याव॒तीति॑ वि॒श्वदे᳚व्य - व॒ती॒ । पृ॒थि॒व्याः । स॒धस्थ॒ इति॑ स॒ध - स्थे॒ । अ॒ङ्गि॒र॒स्वत् । ख॒न॒तु॒ । अ॒व॒ट॒ । दे॒वाना᳚म् । त्वा॒ । पत्नीः᳚ ।  \newline


\textbf{Krama Paata} \newline

वस॑वस्त्वा । त्वा॒ धू॒प॒य॒न्तु॒ । धू॒प॒य॒न्तु॒ गा॒य॒त्रेण॑ । गा॒य॒त्रेण॒ छन्द॑सा । छन्द॑सा ऽङ्गिर॒स्वत् । अ॒ङ्गि॒र॒स्वद् रु॒द्राः । रु॒द्रास्त्वा᳚ । त्वा॒ धू॒प॒य॒न्तु॒ । धू॒प॒य॒न्तु॒ त्रैष्टु॑भेन । त्रैष्टु॑भेन॒ छन्द॑सा । छन्द॑सा ऽङ्गिर॒स्वत् । अ॒ङ्गि॒र॒स्वदा॑दि॒त्याः । आ॒दि॒त्यास्त्वा᳚ । त्वा॒ धू॒प॒य॒न्तु॒ । धू॒प॒य॒न्तु॒ जाग॑तेन । जाग॑तेन॒ छन्द॑सा । छन्द॑सा ऽङ्गिर॒स्वत् । अ॒ङ्गि॒र॒स्वद् विश्वे᳚ । विश्वे᳚ त्वा । त्वा॒ दे॒वाः । दे॒वा वै᳚श्वान॒राः । वै॒श्वा॒न॒रा धू॑पयन्तु । धू॒प॒य॒न्त्वानु॑ष्टुभेन । आनु॑ष्टुभेन॒ छन्द॑सा । आनु॑ष्टुभे॒नेत्यानु॑ - स्तु॒भे॒न॒ । छन्द॑सा ऽङ्गिर॒स्वत् । अ॒ङ्गि॒र॒स्वदिन्द्रः॑ । इन्द्र॑स्त्वा । त्वा॒ धू॒प॒य॒तु॒ । धू॒प॒य॒त्व॒ङ्गि॒र॒स्वत् । अ॒ङ्गि॒र॒स्वद् विष्णुः॑ । विष्णु॑स्त्वा । त्वा॒ धू॒प॒य॒तु॒ । धू॒प॒य॒त्व॒ङ्गि॒र॒स्वत् । अ॒ङ्गि॒र॒स्वद् वरु॑णः । वरु॑णस्त्वा । त्वा॒ धू॒प॒य॒तु॒ । धू॒प॒य॒त्व॒ङ्गि॒र॒स्वत् । अ॒ङ्गि॒र॒स्वददि॑तिः । अदि॑तिस्त्वा । त्वा॒ दे॒वी । दे॒वी वि॒श्वदे᳚व्यावती । वि॒श्वदे᳚व्यावती पृथि॒व्याः । वि॒श्वदे᳚व्याव॒तीति॑वि॒श्वदे᳚व्य - व॒ती॒ । पृ॒थि॒व्याः स॒धस्थे᳚ । स॒धस्थे᳚ ऽङ्गिर॒स्वत् । स॒धस्थ॒ इति॑ स॒ध - स्थे॒ । अ॒ङ्गि॒र॒स्वत् ख॑नतु । ख॒न॒त्व॒व॒ट॒ । अ॒व॒ट॒ दे॒वाना᳚म् । दे॒वाना᳚म् त्वा । त्वा॒ पत्नीः᳚ । पत्नी᳚र् दे॒वीः \newline

\textbf{Jatai Paata} \newline

1. वस॑व स्त्वा त्वा॒ वस॑वो॒ वस॑व स्त्वा । \newline
2. त्वा॒ धू॒प॒य॒न्तु॒ धू॒प॒य॒न्तु॒ त्वा॒ त्वा॒ धू॒प॒य॒न्तु॒ । \newline
3. धू॒प॒य॒न्तु॒ गा॒य॒त्रेण॑ गाय॒त्रेण॑ धूपयन्तु धूपयन्तु गाय॒त्रेण॑ । \newline
4. गा॒य॒त्रेण॒ छन्द॑सा॒ छन्द॑सा गाय॒त्रेण॑ गाय॒त्रेण॒ छन्द॑सा । \newline
5. छन्द॑सा ऽङ्गिर॒स्व द॑ङ्गिर॒स्व च्छन्द॑सा॒ छन्द॑सा ऽङ्गिर॒स्वत् । \newline
6. अ॒ङ्गि॒र॒स्वद् रु॒द्रा रु॒द्रा अ॑ङ्गिर॒स्व द॑ङ्गिर॒स्वद् रु॒द्राः । \newline
7. रु॒द्रा स्त्वा᳚ त्वा रु॒द्रा रु॒द्रा स्त्वा᳚ । \newline
8. त्वा॒ धू॒प॒य॒न्तु॒ धू॒प॒य॒न्तु॒ त्वा॒ त्वा॒ धू॒प॒य॒न्तु॒ । \newline
9. धू॒प॒य॒न्तु॒ त्रैष्टु॑भेन॒ त्रैष्टु॑भेन धूपयन्तु धूपयन्तु॒ त्रैष्टु॑भेन । \newline
10. त्रैष्टु॑भेन॒ छन्द॑सा॒ छन्द॑सा॒ त्रैष्टु॑भेन॒ त्रैष्टु॑भेन॒ छन्द॑सा । \newline
11. छन्द॑सा ऽङ्गिर॒स्व द॑ङ्गिर॒स्वच् छन्द॑सा॒ छन्द॑सा ऽङ्गिर॒स्वत् । \newline
12. अ॒ङ्गि॒र॒स्व दा॑दि॒त्या आ॑दि॒त्या अ॑ङ्गिर॒स्व द॑ङ्गिर॒स्व दा॑दि॒त्याः । \newline
13. आ॒दि॒त्या स्त्वा᳚ त्वा ऽऽदि॒त्या आ॑दि॒त्या स्त्वा᳚ । \newline
14. त्वा॒ धू॒प॒य॒न्तु॒ धू॒प॒य॒न्तु॒ त्वा॒ त्वा॒ धू॒प॒य॒न्तु॒ । \newline
15. धू॒प॒य॒न्तु॒ जाग॑तेन॒ जाग॑तेन धूपयन्तु धूपयन्तु॒ जाग॑तेन । \newline
16. जाग॑तेन॒ छन्द॑सा॒ छन्द॑सा॒ जाग॑तेन॒ जाग॑तेन॒ छन्द॑सा । \newline
17. छन्द॑सा ऽङ्गिर॒स्व द॑ङ्गिर॒स्वच् छन्द॑सा॒ छन्द॑सा ऽङ्गिर॒स्वत् । \newline
18. अ॒ङ्गि॒र॒स्वद् विश्वे॒ विश्वे᳚ ऽङ्गिर॒स्व द॑ङ्गिर॒स्वद् विश्वे᳚ । \newline
19. विश्वे᳚ त्वा त्वा॒ विश्वे॒ विश्वे᳚ त्वा । \newline
20. त्वा॒ दे॒वा दे॒वा स्त्वा᳚ त्वा दे॒वाः । \newline
21. दे॒वा वै᳚श्वान॒रा वै᳚श्वान॒रा दे॒वा दे॒वा वै᳚श्वान॒राः । \newline
22. वै॒श्वा॒न॒रा धू॑पयन्तु धूपयन्तु वैश्वान॒रा वै᳚श्वान॒रा धू॑पयन्तु । \newline
23. धू॒प॒य॒ न्त्वानु॑ष्टुभे॒ना नु॑ष्टुभेन धूपयन्तु धूपय॒ न्त्वानु॑ष्टुभेन । \newline
24. आनु॑ष्टुभेन॒ छन्द॑सा॒ छन्द॒सा ऽऽनु॑ष्टुभे॒ना नु॑ष्टुभेन॒ छन्द॑सा । \newline
25. आनु॑ष्टुभे॒नेत्यानु॑ - स्तु॒भे॒न॒ । \newline
26. छन्द॑सा ऽङ्गिर॒स्व द॑ङ्गिर॒स्व च्छन्द॑सा॒ छन्द॑सा ऽङ्गिर॒स्वत् । \newline
27. अ॒ङ्गि॒र॒स्व दिन्द्र॒ इन्द्रो᳚ ऽङ्गिर॒स्व द॑ङ्गिर॒स्व दिन्द्रः॑ । \newline
28. इन्द्र॑ स्त्वा॒ त्वेन्द्र॒ इन्द्र॑ स्त्वा । \newline
29. त्वा॒ धू॒प॒य॒तु॒ धू॒प॒य॒तु॒ त्वा॒ त्वा॒ धू॒प॒य॒तु॒ । \newline
30. धू॒प॒य॒ त्व॒ङ्गि॒र॒स्व द॑ङ्गिर॒स्वद् धू॑पयतु धूपय त्वङ्गिर॒स्वत् । \newline
31. अ॒ङ्गि॒र॒स्वद् विष्णु॒र् विष्णु॑ रङ्गिर॒स्व द॑ङ्गिर॒स्वद् विष्णुः॑ । \newline
32. विष्णु॑ स्त्वा त्वा॒ विष्णु॒र् विष्णु॑ स्त्वा । \newline
33. त्वा॒ धू॒प॒य॒तु॒ धू॒प॒य॒तु॒ त्वा॒ त्वा॒ धू॒प॒य॒तु॒ । \newline
34. धू॒प॒य॒ त्व॒ङ्गि॒र॒स्व द॑ङ्गिर॒स्वद् धू॑पयतु धूपय त्वङ्गिर॒स्वत् । \newline
35. अ॒ङ्गि॒र॒स्वद् वरु॑णो॒ वरु॑णो ऽङ्गिर॒स्व द॑ङ्गिर॒स्वद् वरु॑णः । \newline
36. वरु॑ण स्त्वा त्वा॒ वरु॑णो॒ वरु॑ण स्त्वा । \newline
37. त्वा॒ धू॒प॒य॒तु॒ धू॒प॒य॒तु॒ त्वा॒ त्वा॒ धू॒प॒य॒तु॒ । \newline
38. धू॒प॒य॒ त्व॒ङ्गि॒र॒स्व द॑ङ्गिर॒स्वद् धू॑पयतु धूपय त्वङ्गिर॒स्वत् । \newline
39. अ॒ङ्गि॒र॒स्व ददि॑ति॒ रदि॑ति रङ्गिर॒स्व द॑ङ्गिर॒स्व ददि॑तिः । \newline
40. अदि॑ति स्त्वा॒ त्वा ऽदि॑ति॒ रदि॑ति स्त्वा । \newline
41. त्वा॒ दे॒वी दे॒वी त्वा᳚ त्वा दे॒वी । \newline
42. दे॒वी वि॒श्वदे᳚व्यावती वि॒श्वदे᳚व्यावती दे॒वी दे॒वी वि॒श्वदे᳚व्यावती । \newline
43. वि॒श्वदे᳚व्यावती पृथि॒व्याः पृ॑थि॒व्या वि॒श्वदे᳚व्यावती वि॒श्वदे᳚व्यावती पृथि॒व्याः । \newline
44. वि॒श्वदे᳚व्याव॒तीति॑ वि॒श्वदे᳚व्य - व॒ती॒ । \newline
45. पृ॒थि॒व्याः स॒धस्थे॑ स॒धस्थे॑ पृथि॒व्याः पृ॑थि॒व्याः स॒धस्थे᳚ । \newline
46. स॒धस्थे᳚ ऽङ्गिर॒स्व द॑ङ्गिर॒स्वथ् स॒धस्थे॑ स॒धस्थे᳚ ऽङ्गिर॒स्वत् । \newline
47. स॒धस्थ॒ इति॑ स॒ध - स्थे॒ । \newline
48. अ॒ङ्गि॒र॒स्वत् ख॑नतु खनत्वङ्गिर॒स्व द॑ङ्गिर॒स्वत् ख॑नतु । \newline
49. ख॒न॒ त्व॒व॒टा॒ व॒ट॒ ख॒न॒तु॒ ख॒न॒त्व॒ व॒ट॒ । \newline
50. अ॒व॒ट॒ दे॒वाना᳚म् दे॒वाना॑ मवटा वट दे॒वाना᳚म् । \newline
51. दे॒वाना᳚म् त्वा त्वा दे॒वाना᳚म् दे॒वाना᳚म् त्वा । \newline
52. त्वा॒ पत्नीः॒ पत्नी᳚ स्त्वा त्वा॒ पत्नीः᳚ । \newline
53. पत्नी᳚र् दे॒वीर् दे॒वीः पत्नीः॒ पत्नी᳚र् दे॒वीः । \newline

\textbf{Ghana Paata } \newline

1. वस॑व स्त्वा त्वा॒ वस॑वो॒ वस॑व स्त्वा धूपयन्तु धूपयन्तु त्वा॒ वस॑वो॒ वस॑व स्त्वा धूपयन्तु । \newline
2. त्वा॒ धू॒प॒य॒न्तु॒ धू॒प॒य॒न्तु॒ त्वा॒ त्वा॒ धू॒प॒य॒न्तु॒ गा॒य॒त्रेण॑ गाय॒त्रेण॑ धूपयन्तु त्वा त्वा धूपयन्तु गाय॒त्रेण॑ । \newline
3. धू॒प॒य॒न्तु॒ गा॒य॒त्रेण॑ गाय॒त्रेण॑ धूपयन्तु धूपयन्तु गाय॒त्रेण॒ छन्द॑सा॒ छन्द॑सा गाय॒त्रेण॑ धूपयन्तु धूपयन्तु गाय॒त्रेण॒ छन्द॑सा । \newline
4. गा॒य॒त्रेण॒ छन्द॑सा॒ छन्द॑सा गाय॒त्रेण॑ गाय॒त्रेण॒ छन्द॑सा ऽङ्गिर॒स्व द॑ङ्गिर॒स्व च्छन्द॑सा गाय॒त्रेण॑ गाय॒त्रेण॒ छन्द॑सा ऽङ्गिर॒स्वत् । \newline
5. छन्द॑सा ऽङ्गिर॒स्व द॑ङ्गिर॒स्व च्छन्द॑सा॒ छन्द॑सा ऽङ्गिर॒स्वद् रु॒द्रा रु॒द्रा अ॑ङ्गिर॒स्व च्छन्द॑सा॒ छन्द॑सा ऽङ्गिर॒स्वद् रु॒द्राः । \newline
6. अ॒ङ्गि॒र॒स्वद् रु॒द्रा रु॒द्रा अ॑ङ्गिर॒स्व द॑ङ्गिर॒स्वद् रु॒द्रा स्त्वा᳚ त्वा रु॒द्रा अ॑ङ्गिर॒स्व द॑ङ्गिर॒स्वद् रु॒द्रा स्त्वा᳚ । \newline
7. रु॒द्रा स्त्वा᳚ त्वा रु॒द्रा रु॒द्रा स्त्वा॑ धूपयन्तु धूपयन्तु त्वा रु॒द्रा रु॒द्रा स्त्वा॑ धूपयन्तु । \newline
8. त्वा॒ धू॒प॒य॒न्तु॒ धू॒प॒य॒न्तु॒ त्वा॒ त्वा॒ धू॒प॒य॒न्तु॒ त्रैष्टु॑भेन॒ त्रैष्टु॑भेन धूपयन्तु त्वा त्वा धूपयन्तु॒ त्रैष्टु॑भेन । \newline
9. धू॒प॒य॒न्तु॒ त्रैष्टु॑भेन॒ त्रैष्टु॑भेन धूपयन्तु धूपयन्तु॒ त्रैष्टु॑भेन॒ छन्द॑सा॒ छन्द॑सा॒ त्रैष्टु॑भेन धूपयन्तु धूपयन्तु॒ त्रैष्टु॑भेन॒ छन्द॑सा । \newline
10. त्रैष्टु॑भेन॒ छन्द॑सा॒ छन्द॑सा॒ त्रैष्टु॑भेन॒ त्रैष्टु॑भेन॒ छन्द॑सा ऽङ्गिर॒स्व द॑ङ्गिर॒स्व च्छन्द॑सा॒ त्रैष्टु॑भेन॒ त्रैष्टु॑भेन॒ छन्द॑सा ऽङ्गिर॒स्वत् । \newline
11. छन्द॑सा ऽङ्गिर॒स्व द॑ङ्गिर॒स्व च्छन्द॑सा॒ छन्द॑सा ऽङ्गिर॒स्व दा॑दि॒त्या आ॑दि॒त्या अ॑ङ्गिर॒स्व च्छन्द॑सा॒ छन्द॑सा ऽङ्गिर॒स्व दा॑दि॒त्याः । \newline
12. अ॒ङ्गि॒र॒स्व दा॑दि॒त्या आ॑दि॒त्या अ॑ङ्गिर॒स्व द॑ङ्गिर॒स्व दा॑दि॒त्या स्त्वा᳚ त्वा ऽऽदि॒त्या अ॑ङ्गिर॒स्व द॑ङ्गिर॒स्व दा॑दि॒त्या स्त्वा᳚ । \newline
13. आ॒दि॒त्या स्त्वा᳚ त्वा ऽऽदि॒त्या आ॑दि॒त्या स्त्वा॑ धूपयन्तु धूपयन्तु त्वा ऽऽदि॒त्या आ॑दि॒त्या स्त्वा॑ धूपयन्तु । \newline
14. त्वा॒ धू॒प॒य॒न्तु॒ धू॒प॒य॒न्तु॒ त्वा॒ त्वा॒ धू॒प॒य॒न्तु॒ जाग॑तेन॒ जाग॑तेन धूपयन्तु त्वा त्वा धूपयन्तु॒ जाग॑तेन । \newline
15. धू॒प॒य॒न्तु॒ जाग॑तेन॒ जाग॑तेन धूपयन्तु धूपयन्तु॒ जाग॑तेन॒ छन्द॑सा॒ छन्द॑सा॒ जाग॑तेन धूपयन्तु धूपयन्तु॒ जाग॑तेन॒ छन्द॑सा । \newline
16. जाग॑तेन॒ छन्द॑सा॒ छन्द॑सा॒ जाग॑तेन॒ जाग॑तेन॒ छन्द॑सा ऽङ्गिर॒स्व द॑ङ्गिर॒स्व च्छन्द॑सा॒ जाग॑तेन॒ जाग॑तेन॒ छन्द॑सा ऽङ्गिर॒स्वत् । \newline
17. छन्द॑सा ऽङ्गिर॒स्व द॑ङ्गिर॒स्व च्छन्द॑सा॒ छन्द॑सा ऽङ्गिर॒स्वद् विश्वे॒ विश्वे᳚ ऽङ्गिर॒स्व च्छन्द॑सा॒ छन्द॑सा ऽङ्गिर॒स्वद् विश्वे᳚ । \newline
18. अ॒ङ्गि॒र॒स्वद् विश्वे॒ विश्वे᳚ ऽङ्गिर॒स्व द॑ङ्गिर॒स्वद् विश्वे᳚ त्वा त्वा॒ विश्वे᳚ ऽङ्गिर॒स्व द॑ङ्गिर॒स्वद् विश्वे᳚ त्वा । \newline
19. विश्वे᳚ त्वा त्वा॒ विश्वे॒ विश्वे᳚ त्वा दे॒वा दे॒वा स्त्वा॒ विश्वे॒ विश्वे᳚ त्वा दे॒वाः । \newline
20. त्वा॒ दे॒वा दे॒वा स्त्वा᳚ त्वा दे॒वा वै᳚श्वान॒रा वै᳚श्वान॒रा दे॒वा स्त्वा᳚ त्वा दे॒वा वै᳚श्वान॒राः । \newline
21. दे॒वा वै᳚श्वान॒रा वै᳚श्वान॒रा दे॒वा दे॒वा वै᳚श्वान॒रा धू॑पयन्तु धूपयन्तु वैश्वान॒रा दे॒वा दे॒वा वै᳚श्वान॒रा धू॑पयन्तु । \newline
22. वै॒श्वा॒न॒रा धू॑पयन्तु धूपयन्तु वैश्वान॒रा वै᳚श्वान॒रा धू॑पय॒न् त्वानु॑ष्टुभे॒ नानु॑ष्टुभेन धूपयन्तु वैश्वान॒रा वै᳚श्वान॒रा धू॑पय॒न् त्वानु॑ष्टुभेन । \newline
23. धू॒प॒य॒न् त्वानु॑ष्टुभे॒ नानु॑ष्टुभेन धूपयन्तु धूपय॒न् त्वानु॑ष्टुभेन॒ छन्द॑सा॒ छन्द॒सा ऽऽनु॑ष्टुभेन धूपयन्तु धूपय॒न् त्वानु॑ष्टुभेन॒ छन्द॑सा । \newline
24. आनु॑ष्टुभेन॒ छन्द॑सा॒ छन्द॒सा ऽऽनु॑ष्टुभे॒ नानु॑ष्टुभेन॒ छन्द॑सा ऽङ्गिर॒स्व द॑ङ्गिर॒स्व च्छन्द॒सा ऽऽनु॑ष्टुभे॒ना नु॑ष्टुभेन॒ छन्द॑सा ऽङ्गिर॒स्वत् । \newline
25. आनु॑ष्टुभे॒नेत्यानु॑ - स्तु॒भे॒न॒ । \newline
26. छन्द॑सा ऽङ्गिर॒स्व द॑ङ्गिर॒स्व च्छन्द॑सा॒ छन्द॑सा ऽङ्गिर॒स्व दिन्द्र॒ इन्द्रो᳚ ऽङ्गिर॒स्व च्छन्द॑सा॒ छन्द॑सा ऽङ्गिर॒स्व दिन्द्रः॑ । \newline
27. अ॒ङ्गि॒र॒स्व दिन्द्र॒ इन्द्रो᳚ ऽङ्गिर॒स्व द॑ङ्गिर॒स्व दिन्द्र॑ स्त्वा॒ त्वेन्द्रो᳚ ऽङ्गिर॒स्व द॑ङ्गिर॒स्वद् इन्द्र॑ स्त्वा । \newline
28. इन्द्र॑ स्त्वा॒ त्वेन्द्र॒ इन्द्र॑ स्त्वा धूपयतु धूपयतु॒ त्वेन्द्र॒ इन्द्र॑ स्त्वा धूपयतु । \newline
29. त्वा॒ धू॒प॒य॒तु॒ धू॒प॒य॒तु॒ त्वा॒ त्वा॒ धू॒प॒य॒ त्व॒ङ्गि॒र॒स्व द॑ङ्गिर॒स्वद् धू॑पयतु त्वा त्वा 
धूपय  त्वङ्गिर॒स्वत् । \newline
30. धू॒प॒य॒ त्व॒ङ्गि॒र॒स्व द॑ङ्गिर॒स्वद् धू॑पयतु धूपय त्वङ्गिर॒स्वद् विष्णु॒र् विष्णु॑ रङ्गिर॒स्वद् धू॑पयतु धूपय त्वङ्गिर॒स्वद् विष्णुः॑ । \newline
31. अ॒ङ्गि॒र॒स्वद् विष्णु॒र् विष्णु॑ रङ्गिर॒स्व द॑ङ्गिर॒स्वद् विष्णु॑ स्त्वा त्वा॒ विष्णु॑ रङ्गिर॒स्व द॑ङ्गिर॒स्वद् विष्णु॑ स्त्वा । \newline
32. विष्णु॑ स्त्वा त्वा॒ विष्णु॒र् विष्णु॑ स्त्वा धूपयतु धूपयतु त्वा॒ विष्णु॒र् विष्णु॑ स्त्वा धूपयतु । \newline
33. त्वा॒ धू॒प॒य॒तु॒ धू॒प॒य॒तु॒ त्वा॒ त्वा॒ धू॒प॒य॒ त्व॒ङ्गि॒र॒स्व द॑ङ्गिर॒स्वद् धू॑पयतु त्वा त्वा धूपय त्वङ्गिर॒स्वत् । \newline
34. धू॒प॒य॒ त्व॒ङ्गि॒र॒स्व द॑ङ्गिर॒स्वद् धू॑पयतु धूपय त्वङ्गिर॒स्वद् वरु॑णो॒ वरु॑णो ऽङ्गिर॒स्वद् धू॑पयतु धूपय त्वङ्गिर॒स्वद् वरु॑णः । \newline
35. अ॒ङ्गि॒र॒स्वद् वरु॑णो॒ वरु॑णो ऽङ्गिर॒स्व द॑ङ्गिर॒स्वद् वरु॑ण स्त्वा त्वा॒ वरु॑णो ऽङ्गिर॒स्व द॑ङ्गिर॒स्वद् वरु॑ण स्त्वा । \newline
36. वरु॑ण स्त्वा त्वा॒ वरु॑णो॒ वरु॑ण स्त्वा धूपयतु धूपयतु त्वा॒ वरु॑णो॒ वरु॑ण स्त्वा धूपयतु । \newline
37. त्वा॒ धू॒प॒य॒तु॒ धू॒प॒य॒तु॒ त्वा॒ त्वा॒ धू॒प॒य॒ त्व॒ङ्गि॒र॒स्व द॑ङ्गिर॒स्वद् धू॑पयतु त्वा त्वा धूपय त्वङ्गिर॒स्वत् । \newline
38. धू॒प॒य॒ त्व॒ङ्गि॒र॒स्व द॑ङ्गिर॒स्वद् धू॑पयतु धूपय त्वङ्गिर॒स्व ददि॑ति॒ रदि॑ति रङ्गिर॒स्वद् धू॑पयतु धूपय त्वङ्गिर॒स्व ददि॑तिः । \newline
39. अ॒ङ्गि॒र॒स्व ददि॑ति॒ रदि॑ति रङ्गिर॒स्व द॑ङ्गिर॒स्व ददि॑ति स्त्वा॒ त्वा ऽदि॑ति रङ्गिर॒स्व द॑ङ्गिर॒स्व ददि॑ति स्त्वा । \newline
40. अदि॑ति स्त्वा॒ त्वा ऽदि॑ति॒ रदि॑ति स्त्वा दे॒वी दे॒वी त्वा ऽदि॑ति॒ रदि॑ति स्त्वा दे॒वी । \newline
41. त्वा॒ दे॒वी दे॒वी त्वा᳚ त्वा दे॒वी वि॒श्वदे᳚व्यावती वि॒श्वदे᳚व्यावती दे॒वी त्वा᳚ त्वा दे॒वी वि॒श्वदे᳚व्यावती । \newline
42. दे॒वी वि॒श्वदे᳚व्यावती वि॒श्वदे᳚व्यावती दे॒वी दे॒वी वि॒श्वदे᳚व्यावती पृथि॒व्याः पृ॑थि॒व्या वि॒श्वदे᳚व्यावती दे॒वी दे॒वी वि॒श्वदे᳚व्यावती पृथि॒व्याः । \newline
43. वि॒श्वदे᳚व्यावती पृथि॒व्याः पृ॑थि॒व्या वि॒श्वदे᳚व्यावती वि॒श्वदे᳚व्यावती पृथि॒व्याः स॒धस्थे॑ स॒धस्थे॑ पृथि॒व्या वि॒श्वदे᳚व्यावती वि॒श्वदे᳚व्यावती पृथि॒व्याः स॒धस्थे᳚ । \newline
44. वि॒श्वदे᳚व्याव॒तीति॑ वि॒श्वदे᳚व्य - व॒ती॒ । \newline
45. पृ॒थि॒व्याः स॒धस्थे॑ स॒धस्थे॑ पृथि॒व्याः पृ॑थि॒व्याः स॒धस्थे᳚ ऽङ्गिर॒स्व द॑ङ्गिर॒स्वथ् स॒धस्थे॑ पृथि॒व्याः पृ॑थि॒व्याः स॒धस्थे᳚ ऽङ्गिर॒स्वत् । \newline
46. स॒धस्थे᳚ ऽङ्गिर॒स्व द॑ङ्गिर॒स्वथ् स॒धस्थे॑ स॒धस्थे᳚ ऽङ्गिर॒स्वत् ख॑नतु खन त्वङ्गिर॒स्वथ् स॒धस्थे॑ स॒धस्थे᳚ ऽङ्गिर॒स्वत् ख॑नतु । \newline
47. स॒धस्थ॒ इति॑ स॒ध - स्थे॒ । \newline
48. अ॒ङ्गि॒र॒स्वत् ख॑नतु खन त्वङ्गिर॒स्व द॑ङ्गिर॒स्वत् ख॑न त्ववटा वट खन त्वङ्गिर॒स्व द॑ङ्गिर॒स्वत् ख॑न त्ववट । \newline
49. ख॒न॒ त्व॒व॒टा॒ व॒ट॒ ख॒न॒तु॒ ख॒न॒ त्व॒व॒ट॒ दे॒वाना᳚म् दे॒वाना॑ मवट खनतु खन त्ववट दे॒वाना᳚म् । \newline
50. अ॒व॒ट॒ दे॒वाना᳚म् दे॒वाना॑ मवटा वट दे॒वाना᳚म् त्वा त्वा दे॒वाना॑ मवटा वट दे॒वाना᳚म् त्वा । \newline
51. दे॒वाना᳚म् त्वा त्वा दे॒वाना᳚म् दे॒वाना᳚म् त्वा॒ पत्नीः॒ पत्नी᳚ स्त्वा दे॒वाना᳚म् दे॒वाना᳚म् त्वा॒ पत्नीः᳚ । \newline
52. त्वा॒ पत्नीः॒ पत्नी᳚ स्त्वा त्वा॒ पत्नी᳚र् दे॒वीर् दे॒वीः पत्नी᳚ स्त्वा त्वा॒ पत्नी᳚र् दे॒वीः । \newline
53. पत्नी᳚र् दे॒वीर् दे॒वीः पत्नीः॒ पत्नी᳚र् दे॒वीर् वि॒श्वदे᳚व्यावतीर् वि॒श्वदे᳚व्यावतीर् दे॒वीः पत्नीः॒ पत्नी᳚र् दे॒वीर् वि॒श्वदे᳚व्यावतीः । \newline
\pagebreak
\markright{ TS 4.1.6.2  \hfill https://www.vedavms.in \hfill}

\section{ TS 4.1.6.2 }

\textbf{TS 4.1.6.2 } \newline
\textbf{Samhita Paata} \newline

र्दे॒वी र्वि॒श्वदे᳚व्यावतीः पृथि॒व्याः स॒धस्थे᳚ऽङ्गिर॒स्वद्-द॑धतूखे धि॒षणा᳚स्त्वा दे॒वीर्वि॒श्वदे᳚व्यावतीः पृथि॒व्याः स॒धस्थे᳚-ऽङ्गिर॒स्व-द॒भीन्ध॑तामुखे॒ ग्नास्त्वा॑ दे॒वीर्वि॒श्वदे᳚व्यावतीः पृथि॒व्याः स॒धस्थे᳚ऽङ्गिर॒स्व-च्छ्र॑पयन्तूखे॒ वरू᳚त्रयो॒ जन॑यस्त्वा दे॒वीर्वि॒श्वदे᳚व्यावतीः पृथि॒व्याः स॒धस्थे᳚ऽङ्गिर॒स्वत् प॑चन्तूखे । मित्रै॒तामु॒खां प॑चै॒षा मा भे॑दि । ए॒तां ते॒ परि॑ ददा॒म्यभि॑त्त्यै ॥ अ॒भीमां - [  ] \newline

\textbf{Pada Paata} \newline

दे॒वीः । वि॒श्वदे᳚व्यावती॒रिति॑ वि॒श्वदे᳚व्य - व॒तीः॒ । पृ॒थि॒व्याः । स॒धस्थ॒ इति॑ स॒ध - स्थे॒ । अ॒ङ्गि॒र॒स्वत् । द॒ध॒तु॒ । उ॒खे॒ । धि॒षणाः᳚ । त्वा॒ । दे॒वीः । वि॒श्वदे᳚व्यावती॒रिति॑ वि॒श्वदे᳚व्य - व॒तीः॒ । पृ॒थि॒व्याः । स॒धस्थ॒ इति॑ स॒ध - स्थे॒ । अ॒ङ्गि॒र॒स्वत् । अ॒भीति॑ । इ॒न्ध॒ता॒म् । उ॒खे॒ । ग्नाः । त्वा॒ । दे॒वीः । वि॒श्वदे᳚व्यावती॒रिति॑ वि॒श्वदे᳚व्य - व॒तीः॒ । पृ॒थि॒व्याः । स॒धस्थ॒ इति॑ स॒ध - स्थे॒ । अ॒ङ्गि॒र॒स्वत् । श्र॒प॒य॒न्तु॒ । उ॒खे॒ । वरू᳚त्रयः । जन॑यः । त्वा॒ । दे॒वीः । वि॒श्वदे᳚व्यावती॒रिति॑ वि॒श्वदे᳚व्य - व॒तीः॒ । पृ॒थि॒व्याः । स॒धस्थ॒ इति॑ स॒ध - स्थे॒ । अ॒ङ्गि॒र॒स्वत् । प॒च॒न्तु॒ । उ॒खे॒ ॥ मित्र॑ । ए॒ताम् । उ॒खाम् । प॒च॒ । ए॒षा । मा । भे॒दि॒ ॥ ए॒ताम् । ते॒ । परीति॑ । द॒दा॒मि॒ । अभि॑त्त्यै ॥ अ॒भीति॑ । इ॒माम् ।  \newline


\textbf{Krama Paata} \newline

दे॒वीर् वि॒श्वदे᳚व्यावतीः । वि॒श्वदे᳚व्यावतीः पृथि॒व्याः । वि॒श्वदे᳚व्यावती॒रिति॑ वि॒श्वदे᳚व्य - व॒तीः॒ । पृ॒थि॒व्याः स॒धस्थे᳚ । स॒धस्थे᳚ ऽङ्गिर॒स्वत् । स॒धस्थ॒ इति॑ स॒ध - स्थे॒ । अ॒ङ्गि॒र॒स्वद् द॑धतु । द॒ध॒तू॒खे॒ । उ॒खे॒ धि॒षणाः᳚ । धि॒षणा᳚स्त्वा । त्वा॒ दे॒वीः । दे॒वीर् वि॒श्वदे᳚व्यावतीः । वि॒श्वदे᳚व्यावतीः पृथि॒व्याः । वि॒श्वदे᳚व्यावती॒रिति॑ वि॒श्वदे᳚व्य - व॒तीः॒ । पृ॒थि॒व्याः स॒धस्थे᳚ । स॒धस्थे᳚ ऽङ्गिर॒स्वत् । स॒धस्थ॒ इति॑ स॒ध - स्थे॒ । अ॒ङ्गि॒र॒स्वद॒भि । अ॒भीन्ध॑ताम् । इ॒न्ध॒ता॒मु॒खे॒ । उ॒खे॒ ग्नाः । ग्नास्त्वा᳚ । त्वा॒ दे॒वीः । दे॒वीर् वि॒श्वदे᳚व्यावतीः । वि॒श्वदे᳚व्यावतीः पृथि॒व्याः । वि॒श्वदे᳚व्यावती॒रिति॑ वि॒श्वदे᳚व्य - व॒तीः॒ । पृ॒थि॒व्याः स॒धस्थे᳚ । स॒धस्थे᳚ ऽङ्गिर॒स्वत् । स॒धस्थ॒ इति॑ स॒ध - स्थे॒ । अ॒ङ्गि॒र॒स्वच्छ्र॑पयन्तु । श्र॒प॒य॒न्तू॒खे॒ । उ॒खे॒ वरू᳚त्रयः । वरू᳚त्रयो॒ जन॑यः । जन॑यस्त्वा । त्वा॒ दे॒वीः । दे॒वीर् वि॒श्वदे᳚व्यावतीः । वि॒श्वदे᳚व्यावतीः पृथि॒व्याः । वि॒श्वदे᳚व्यावती॒रिति॑ वि॒श्वदे᳚व्य - व॒तीः॒ । पृ॒थि॒व्याः स॒धस्थे᳚ । स॒धस्थे᳚ ऽङ्गिर॒स्वत् । स॒धस्थ॒ इति॑ स॒ध - स्थे॒ । अ॒ङ्गि॒र॒स्वत् प॑चन्तु । प॒च॒न्तू॒खे॒ । उ॒खे॒ इत्यु॑खे ॥ मित्रै॒ताम् । ए॒तामु॒खाम् । उ॒खाम् प॑च । प॒चै॒षा । ए॒षा मा । मा भे॑दि । भे॒दीति॑ भेदि ॥ ए॒ताम् ते᳚ । ते॒ परि॑ । परि॑ ददामि । द॒दा॒म्यभि॑त्यै । अभि॑त्या॒ इत्यभि॑त्यै ॥ अ॒भीमाम् । इ॒माम् म॑हि॒ना \newline

\textbf{Jatai Paata} \newline

1. दे॒वीर् वि॒श्वदे᳚व्यावतीर् वि॒श्वदे᳚व्यावतीर् दे॒वीर् दे॒वीर् वि॒श्वदे᳚व्यावतीः । \newline
2. वि॒श्वदे᳚व्यावतीः पृथि॒व्याः पृ॑थि॒व्या वि॒श्वदे᳚व्यावतीर् वि॒श्वदे᳚व्यावतीः पृथि॒व्याः । \newline
3. वि॒श्वदे᳚व्यावती॒रिति॑ वि॒श्वदे᳚व्य - व॒तीः॒ । \newline
4. पृ॒थि॒व्याः स॒धस्थे॑ स॒धस्थे॑ पृथि॒व्याः पृ॑थि॒व्याः स॒धस्थे᳚ । \newline
5. स॒धस्थे᳚ ऽङ्गिर॒स्व द॑ङ्गिर॒स्वथ् स॒धस्थे॑ स॒धस्थे᳚ ऽङ्गिर॒स्वत् । \newline
6. स॒धस्थ॒ इति॑ स॒ध - स्थे॒ । \newline
7. अ॒ङ्गि॒र॒स्वद् द॑धतु दधत्वङ्गिर॒स्व द॑ङ्गिर॒स्वद् द॑धतु । \newline
8. द॒ध॒तू॒ख॒ उ॒खे॒ द॒ध॒तु॒ द॒ध॒तू॒खे॒ । \newline
9. उ॒खे॒ धि॒षणा॑ धि॒षणा॑ उख उखे धि॒षणाः᳚ । \newline
10. धि॒षणा᳚ स्त्वा त्वा धि॒षणा॑ धि॒षणा᳚ स्त्वा । \newline
11. त्वा॒ दे॒वीर् दे॒वी स्त्वा᳚ त्वा दे॒वीः । \newline
12. दे॒वीर् वि॒श्वदे᳚व्यावतीर् वि॒श्वदे᳚व्यावतीर् दे॒वीर् दे॒वीर् वि॒श्वदे᳚व्यावतीः । \newline
13. वि॒श्वदे᳚व्यावतीः पृथि॒व्याः पृ॑थि॒व्या वि॒श्वदे᳚व्यावतीर् वि॒श्वदे᳚व्यावतीः पृथि॒व्याः । \newline
14. वि॒श्वदे᳚व्यावती॒रिति॑ वि॒श्वदे᳚व्य - व॒तीः॒ । \newline
15. पृ॒थि॒व्याः स॒धस्थे॑ स॒धस्थे॑ पृथि॒व्याः पृ॑थि॒व्याः स॒धस्थे᳚ । \newline
16. स॒धस्थे᳚ ऽङ्गिर॒स्व द॑ङ्गिर॒स्वथ् स॒धस्थे॑ स॒धस्थे᳚ ऽङ्गिर॒स्वत् । \newline
17. स॒धस्थ॒ इति॑ स॒ध - स्थे॒ । \newline
18. अ॒ङ्गि॒र॒स्व द॒भ्या᳚(1॒)भ्य॑ङ्गिर॒स्व द॑ङ्गिर॒स्व द॒भि । \newline
19. अ॒भीन्ध॑ता मिन्धता म॒भ्य॑भी न्ध॑ताम् । \newline
20. इ॒न्ध॒ता॒ मु॒ख॒ उ॒ख॒ इ॒न्ध॒ता॒ मि॒न्ध॒ता॒ मु॒खे॒ । \newline
21. उ॒खे॒ ग्ना ग्ना उ॑ख उखे॒ ग्नाः । \newline
22. ग्ना स्त्वा᳚ त्वा॒ ग्ना ग्ना स्त्वा᳚ । \newline
23. त्वा॒ दे॒वीर् दे॒वी स्त्वा᳚ त्वा दे॒वीः । \newline
24. दे॒वीर् वि॒श्वदे᳚व्यावतीर् वि॒श्वदे᳚व्यावतीर् दे॒वीर् दे॒वीर् वि॒श्वदे᳚व्यावतीः । \newline
25. वि॒श्वदे᳚व्यावतीः पृथि॒व्याः पृ॑थि॒व्या वि॒श्वदे᳚व्यावतीर् वि॒श्वदे᳚व्यावतीः पृथि॒व्याः । \newline
26. वि॒श्वदे᳚व्यावती॒रिति॑ वि॒श्वदे᳚व्य - व॒तीः॒ । \newline
27. पृ॒थि॒व्याः स॒धस्थे॑ स॒धस्थे॑ पृथि॒व्याः पृ॑थि॒व्याः स॒धस्थे᳚ । \newline
28. स॒धस्थे᳚ ऽङ्गिर॒स्व द॑ङ्गिर॒स्वथ् स॒धस्थे॑ स॒धस्थे᳚ ऽङ्गिर॒स्वत् । \newline
29. स॒धस्थ॒ इति॑ स॒ध - स्थे॒ । \newline
30. अ॒ङ्गि॒र॒स्व च्छ्र॑पयन्तु श्रपय न्त्वङ्गिर॒स्व द॑ङ्गिर॒स्व च्छ्र॑पयन्तु । \newline
31. श्र॒प॒य॒न्तू॒ख॒ उ॒खे॒ श्र॒प॒य॒न्तु॒ श्र॒प॒य॒न्तू॒खे॒ । \newline
32. उ॒खे॒ वरू᳚त्रयो॒ वरू᳚त्रय उख उखे॒ वरू᳚त्रयः । \newline
33. वरू᳚त्रयो॒ जन॑यो॒ जन॑यो॒ वरू᳚त्रयो॒ वरू᳚त्रयो॒ जन॑यः । \newline
34. जन॑य स्त्वा त्वा॒ जन॑यो॒ जन॑य स्त्वा । \newline
35. त्वा॒ दे॒वीर् दे॒वी स्त्वा᳚ त्वा दे॒वीः । \newline
36. दे॒वीर् वि॒श्वदे᳚व्यावतीर् वि॒श्वदे᳚व्यावतीर् दे॒वीर् दे॒वीर् वि॒श्वदे᳚व्यावतीः । \newline
37. वि॒श्वदे᳚व्यावतीः पृथि॒व्याः पृ॑थि॒व्या वि॒श्वदे᳚व्यावतीर् वि॒श्वदे᳚व्यावतीः पृथि॒व्याः । \newline
38. वि॒श्वदे᳚व्यावती॒रिति॑ वि॒श्वदे᳚व्य - व॒तीः॒ । \newline
39. पृ॒थि॒व्याः स॒धस्थे॑ स॒धस्थे॑ पृथि॒व्याः पृ॑थि॒व्याः स॒धस्थे᳚ । \newline
40. स॒धस्थे᳚ ऽङ्गिर॒स्व द॑ङ्गिर॒स्वथ् स॒धस्थे॑ स॒धस्थे᳚ ऽङ्गिर॒स्वत् । \newline
41. स॒धस्थ॒ इति॑ स॒ध - स्थे॒ । \newline
42. अ॒ङ्गि॒र॒स्वत् प॑चन्तु पचन्त्वङ्गिर॒स्व द॑ङ्गिर॒स्वत् प॑चन्तु । \newline
43. प॒च॒न्तू॒खे॒ उ॒खे॒ प॒च॒न्तु॒ प॒च॒न्तू॒खे॒ । \newline
44. उ॒ख॒ इत्यु॑खे । \newline
45. मित्रै॒ता मे॒ताम् मित्र॒ मित्रै॒ताम् । \newline
46. ए॒ता मु॒खा मु॒खा मे॒ता मे॒ता मु॒खाम् । \newline
47. उ॒खाम् प॑च पचो॒खा मु॒खाम् प॑च । \newline
48. प॒चै॒षैषा प॑च पचै॒षा । \newline
49. ए॒षा मा मैषैषा मा । \newline
50. मा भे॑दि भेदि॒ मा मा भे॑दि । \newline
51. भे॒दीति॑ भेदि । \newline
52. ए॒ताम् ते॑ त ए॒ता मे॒ताम् ते᳚ । \newline
53. ते॒ परि॒ परि॑ ते ते॒ परि॑ । \newline
54. परि॑ ददामि ददामि॒ परि॒ परि॑ ददामि । \newline
55. द॒दा॒ म्यभि॑त्त्या॒ अभि॑त्त्यै ददामि ददा॒ म्यभि॑त्त्यै । \newline
56. अभि॑त्या॒ इत्यभि॑त्यै । \newline
57. अ॒भीमा मि॒मा म॒भ्य॑भीमाम् । \newline
58. इ॒माम् म॑हि॒ना म॑हि॒नेमा मि॒माम् म॑हि॒ना । \newline

\textbf{Ghana Paata } \newline

1. दे॒वीर् वि॒श्वदे᳚व्यावतीर् वि॒श्वदे᳚व्यावतीर् दे॒वीर् दे॒वीर् वि॒श्वदे᳚व्यावतीः पृथि॒व्याः पृ॑थि॒व्या वि॒श्वदे᳚व्यावतीर् दे॒वीर् दे॒वीर् वि॒श्वदे᳚व्यावतीः पृथि॒व्याः । \newline
2. वि॒श्वदे᳚व्यावतीः पृथि॒व्याः पृ॑थि॒व्या वि॒श्वदे᳚व्यावतीर् वि॒श्वदे᳚व्यावतीः पृथि॒व्याः स॒धस्थे॑ स॒धस्थे॑ पृथि॒व्या वि॒श्वदे᳚व्यावतीर् वि॒श्वदे᳚व्यावतीः पृथि॒व्याः स॒धस्थे᳚ । \newline
3. वि॒श्वदे᳚व्यावती॒रिति॑ वि॒श्वदे᳚व्य - व॒तीः॒ । \newline
4. पृ॒थि॒व्याः स॒धस्थे॑ स॒धस्थे॑ पृथि॒व्याः पृ॑थि॒व्याः स॒धस्थे᳚ ऽङ्गिर॒स्व द॑ङ्गिर॒स्वथ् स॒धस्थे॑ पृथि॒व्याः पृ॑थि॒व्याः स॒धस्थे᳚ ऽङ्गिर॒स्वत् । \newline
5. स॒धस्थे᳚ ऽङ्गिर॒स्व द॑ङ्गिर॒स्वथ् स॒धस्थे॑ स॒धस्थे᳚ ऽङ्गिर॒स्वद् द॑धतु दध त्वङ्गिर॒स्वथ् स॒धस्थे॑ स॒धस्थे᳚ ऽङ्गिर॒स्वद् द॑धतु । \newline
6. स॒धस्थ॒ इति॑ स॒ध - स्थे॒ । \newline
7. अ॒ङ्गि॒र॒स्वद् द॑धतु दध त्वङ्गिर॒स्व द॑ङ्गिर॒स्वद् द॑ध तूख उखे दध त्वङ्गिर॒स्व द॑ङ्गिर॒स्वद् द॑ध तूखे । \newline
8. द॒ध॒ तू॒ख॒ उ॒खे॒ द॒ध॒तु॒ द॒ध॒ तू॒खे॒ धि॒षणा॑ धि॒षणा॑ उखे दधतु दध तूखे धि॒षणाः᳚ । \newline
9. उ॒खे॒ धि॒षणा॑ धि॒षणा॑ उख उखे धि॒षणा᳚ स्त्वा त्वा धि॒षणा॑ उख उखे धि॒षणा᳚ स्त्वा । \newline
10. धि॒षणा᳚ स्त्वा त्वा धि॒षणा॑ धि॒षणा᳚ स्त्वा दे॒वीर् दे॒वी स्त्वा॑ धि॒षणा॑ धि॒षणा᳚ स्त्वा दे॒वीः । \newline
11. त्वा॒ दे॒वीर् दे॒वी स्त्वा᳚ त्वा दे॒वीर् वि॒श्वदे᳚व्यावतीर् वि॒श्वदे᳚व्यावतीर् दे॒वी स्त्वा᳚ त्वा दे॒वीर् वि॒श्वदे᳚व्यावतीः । \newline
12. दे॒वीर् वि॒श्वदे᳚व्यावतीर् वि॒श्वदे᳚व्यावतीर् दे॒वीर् दे॒वीर् वि॒श्वदे᳚व्यावतीः पृथि॒व्याः पृ॑थि॒व्या वि॒श्वदे᳚व्यावतीर् दे॒वीर् दे॒वीर् वि॒श्वदे᳚व्यावतीः पृथि॒व्याः । \newline
13. वि॒श्वदे᳚व्यावतीः पृथि॒व्याः पृ॑थि॒व्या वि॒श्वदे᳚व्यावतीर् वि॒श्वदे᳚व्यावतीः पृथि॒व्याः स॒धस्थे॑ स॒धस्थे॑ पृथि॒व्या वि॒श्वदे᳚व्यावतीर् वि॒श्वदे᳚व्यावतीः पृथि॒व्याः स॒धस्थे᳚ । \newline
14. वि॒श्वदे᳚व्यावती॒रिति॑ वि॒श्वदे᳚व्य - व॒तीः॒ । \newline
15. पृ॒थि॒व्याः स॒धस्थे॑ स॒धस्थे॑ पृथि॒व्याः पृ॑थि॒व्याः स॒धस्थे᳚ ऽङ्गिर॒स्व द॑ङ्गिर॒स्वथ् स॒धस्थे॑ पृथि॒व्याः पृ॑थि॒व्याः स॒धस्थे᳚ ऽङ्गिर॒स्वत् । \newline
16. स॒धस्थे᳚ ऽङ्गिर॒स्व द॑ङ्गिर॒स्वथ् स॒धस्थे॑ स॒धस्थे᳚ ऽङ्गिर॒स्व द॒भ्या᳚(1॒)भ्य॑ङ्गिर॒स्वथ् स॒धस्थे॑ स॒धस्थे᳚ ऽङ्गिर॒स्व द॒भि । \newline
17. स॒धस्थ॒ इति॑ स॒ध - स्थे॒ । \newline
18. अ॒ङ्गि॒र॒स्व द॒भ्या᳚(1॒)भ्य॑ङ्गिर॒स्व द॑ङ्गिर॒स्व द॒भीन्ध॑ता मिन्धता म॒भ्य॑ङ्गिर॒स्व द॑ङ्गिर॒स्व द॒भीन्ध॑ताम् । \newline
19. अ॒भीन्ध॑ता मिन्धता म॒भ्य॑ भीन्ध॑ता मुख उख इन्धता म॒भ्य॑ भीन्ध॑ता मुखे । \newline
20. इ॒न्ध॒ता॒ मु॒ख॒ उ॒ख॒ इ॒न्ध॒ता॒ मि॒न्ध॒ता॒ मु॒खे॒ ग्ना ग्ना उ॑ख इन्धता मिन्धता मुखे॒ ग्नाः । \newline
21. उ॒खे॒ ग्ना ग्ना उ॑ख उखे॒ ग्ना स्त्वा᳚ त्वा॒ ग्ना उ॑ख उखे॒ ग्ना स्त्वा᳚ । \newline
22. ग्ना स्त्वा᳚ त्वा॒ ग्ना ग्ना स्त्वा॑ दे॒वीर् दे॒वी स्त्वा॒ ग्ना ग्ना स्त्वा॑ दे॒वीः । \newline
23. त्वा॒ दे॒वीर् दे॒वी स्त्वा᳚ त्वा दे॒वीर् वि॒श्वदे᳚व्यावतीर् वि॒श्वदे᳚व्यावतीर् दे॒वी स्त्वा᳚ त्वा दे॒वीर् वि॒श्वदे᳚व्यावतीः । \newline
24. दे॒वीर् वि॒श्वदे᳚व्यावतीर् वि॒श्वदे᳚व्यावतीर् दे॒वीर् दे॒वीर् वि॒श्वदे᳚व्यावतीः पृथि॒व्याः पृ॑थि॒व्या वि॒श्वदे᳚व्यावतीर् दे॒वीर् दे॒वीर् वि॒श्वदे᳚व्यावतीः पृथि॒व्याः । \newline
25. वि॒श्वदे᳚व्यावतीः पृथि॒व्याः पृ॑थि॒व्या वि॒श्वदे᳚व्यावतीर् वि॒श्वदे᳚व्यावतीः पृथि॒व्याः स॒धस्थे॑ स॒धस्थे॑ पृथि॒व्या वि॒श्वदे᳚व्यावतीर् वि॒श्वदे᳚व्यावतीः पृथि॒व्याः स॒धस्थे᳚ । \newline
26. वि॒श्वदे᳚व्यावती॒रिति॑ वि॒श्वदे᳚व्य - व॒तीः॒ । \newline
27. पृ॒थि॒व्याः स॒धस्थे॑ स॒धस्थे॑ पृथि॒व्याः पृ॑थि॒व्याः स॒धस्थे᳚ ऽङ्गिर॒स्व द॑ङ्गिर॒स्वथ् स॒धस्थे॑ पृथि॒व्याः पृ॑थि॒व्याः स॒धस्थे᳚ ऽङ्गिर॒स्वत् । \newline
28. स॒धस्थे᳚ ऽङ्गिर॒स्व द॑ङ्गिर॒स्वथ् स॒धस्थे॑ स॒धस्थे᳚ ऽङ्गिर॒स्व च्छ्र॑पयन्तु श्रपयन् त्वङ्गिर॒स्वथ् स॒धस्थे॑ स॒धस्थे᳚ ऽङ्गिर॒स्व च्छ्र॑पयन्तु । \newline
29. स॒धस्थ॒ इति॑ स॒ध - स्थे॒ । \newline
30. अ॒ङ्गि॒र॒स्व च्छ्र॑पयन्तु श्रपयन् त्वङ्गिर॒स्व द॑ङ्गिर॒स्व च्छ्र॑पयन् तूख उखे श्रपयन् त्वङ्गिर॒स्व द॑ङ्गिर॒स्व च्छ्र॑पयन् तूखे । \newline
31. श्र॒प॒य॒न् तू॒ख॒ उ॒खे॒ श्र॒प॒य॒न्तु॒ श्र॒प॒य॒न् तू॒खे॒ वरू᳚त्रयो॒ वरू᳚त्रय उखे श्रपयन्तु श्रपयन् तूखे॒ वरू᳚त्रयः । \newline
32. उ॒खे॒ वरू᳚त्रयो॒ वरू᳚त्रय उख उखे॒ वरू᳚त्रयो॒ जन॑यो॒ जन॑यो॒ वरू᳚त्रय उख उखे॒ वरू᳚त्रयो॒ जन॑यः । \newline
33. वरू᳚त्रयो॒ जन॑यो॒ जन॑यो॒ वरू᳚त्रयो॒ वरू᳚त्रयो॒ जन॑य स्त्वा त्वा॒ जन॑यो॒ वरू᳚त्रयो॒ वरू᳚त्रयो॒ जन॑य स्त्वा । \newline
34. जन॑य स्त्वा त्वा॒ जन॑यो॒ जन॑य स्त्वा दे॒वीर् दे॒वी स्त्वा॒ जन॑यो॒ जन॑य स्त्वा दे॒वीः । \newline
35. त्वा॒ दे॒वीर् दे॒वी स्त्वा᳚ त्वा दे॒वीर् वि॒श्वदे᳚व्यावतीर् वि॒श्वदे᳚व्यावतीर् दे॒वी स्त्वा᳚ त्वा दे॒वीर् वि॒श्वदे᳚व्यावतीः । \newline
36. दे॒वीर् वि॒श्वदे᳚व्यावतीर् वि॒श्वदे᳚व्यावतीर् दे॒वीर् दे॒वीर् वि॒श्वदे᳚व्यावतीः पृथि॒व्याः पृ॑थि॒व्या वि॒श्वदे᳚व्यावतीर् दे॒वीर् दे॒वीर् वि॒श्वदे᳚व्यावतीः पृथि॒व्याः । \newline
37. वि॒श्वदे᳚व्यावतीः पृथि॒व्याः पृ॑थि॒व्या वि॒श्वदे᳚व्यावतीर् वि॒श्वदे᳚व्यावतीः पृथि॒व्याः स॒धस्थे॑ स॒धस्थे॑ पृथि॒व्या वि॒श्वदे᳚व्यावतीर् वि॒श्वदे᳚व्यावतीः पृथि॒व्याः स॒धस्थे᳚ । \newline
38. वि॒श्वदे᳚व्यावती॒रिति॑ वि॒श्वदे᳚व्य - व॒तीः॒ । \newline
39. पृ॒थि॒व्याः स॒धस्थे॑ स॒धस्थे॑ पृथि॒व्याः पृ॑थि॒व्याः स॒धस्थे᳚ ऽङ्गिर॒स्व द॑ङ्गिर॒स्वथ् स॒धस्थे॑ पृथि॒व्याः पृ॑थि॒व्याः स॒धस्थे᳚ ऽङ्गिर॒स्वत् । \newline
40. स॒धस्थे᳚ ऽङ्गिर॒स्व द॑ङ्गिर॒स्वथ् स॒धस्थे॑ स॒धस्थे᳚ ऽङ्गिर॒स्वत् प॑चन्तु पचन् त्वङ्गिर॒स्वथ् स॒धस्थे॑ स॒धस्थे᳚ ऽङ्गिर॒स्वत् प॑चन्तु । \newline
41. स॒धस्थ॒ इति॑ स॒ध - स्थे॒ । \newline
42. अ॒ङ्गि॒र॒स्वत् प॑चन्तु पचन् त्वङ्गिर॒स्व द॑ङ्गिर॒स्वत् प॑चन्तूखे उखे पचन् त्वङ्गिर॒स्व द॑ङ्गिर॒स्वत् प॑चन्तूखे । \newline
43. प॒च॒न्तू॒खे॒ उ॒खे॒ प॒च॒न्तु॒ प॒च॒न्तू॒खे॒ । \newline
44. उ॒ख॒ इत्यु॑खे । \newline
45. मित्रै॒ता मे॒ताम् मित्र॒ मित्रै॒ता मु॒खा मु॒खा मे॒ताम् मित्र॒ मित्रै॒ता मु॒खाम् । \newline
46. ए॒ता मु॒खा मु॒खा मे॒ता मे॒ता मु॒खाम् प॑च पचो॒खा मे॒ता मे॒ता मु॒खाम् प॑च । \newline
47. उ॒खाम् प॑च पचो॒खा मु॒खाम् प॑चै॒ षैषा प॑चो॒खा मु॒खाम् प॑चै॒षा । \newline
48. प॒चै॒ षैषा प॑च पचै॒षा मा मैषा प॑च पचै॒षा मा । \newline
49. ए॒षा मा मैषैषा मा भे॑दि भेदि॒ मैषैषा मा भे॑दि । \newline
50. मा भे॑दि भेदि॒ मा मा भे॑दि । \newline
51. भे॒दीति॑ भेदि । \newline
52. ए॒ताम् ते॑ त ए॒ता मे॒ताम् ते॒ परि॒ परि॑ त ए॒ता मे॒ताम् ते॒ परि॑ । \newline
53. ते॒ परि॒ परि॑ ते ते॒ परि॑ ददामि ददामि॒ परि॑ ते ते॒ परि॑ ददामि । \newline
54. परि॑ ददामि ददामि॒ परि॒ परि॑ ददा॒ म्यभि॑त्त्या॒ अभि॑त्त्यै ददामि॒ परि॒ परि॑ ददा॒ म्यभि॑त्त्यै । \newline
55. द॒दा॒ म्यभि॑त्त्या॒ अभि॑त्त्यै ददामि ददा॒ म्यभि॑त्त्यै । \newline
56. अभि॑त्या॒ इत्यभि॑त्यै । \newline
57. अ॒भीमा मि॒मा म॒भ्य॑ भीमाम् म॑हि॒ना म॑हि॒नेमा म॒भ्य॑ भीमाम् म॑हि॒ना । \newline
58. इ॒माम् म॑हि॒ना म॑हि॒नेमा मि॒माम् म॑हि॒ना दिव॒म् दिव॑म् महि॒नेमा मि॒माम् म॑हि॒ना दिव᳚म् । \newline
\pagebreak
\markright{ TS 4.1.6.3  \hfill https://www.vedavms.in \hfill}

\section{ TS 4.1.6.3 }

\textbf{TS 4.1.6.3 } \newline
\textbf{Samhita Paata} \newline

म॑हि॒ना दिवं॑ मि॒त्रो ब॑भूव स॒प्रथाः᳚ । उ॒त श्रव॑सा पृथि॒वीं ॥ मि॒त्रस्य॑ चर्.षणी॒धृतः॒ श्रवो॑ दे॒वस्य॑ सान॒सिं । द्यु॒म्नं चि॒त्रश्र॑वस्तमं ॥ दे॒वस्त्वा॑ सवि॒तोद्व॑पतु सुपा॒णिः स्व॑ङ्गु॒रिः । सु॒बा॒हुरु॒त शक्त्या᳚ ॥ अप॑द्यमाना पृथि॒व्याशा॒ दिश॒ आ पृ॑ण । उत्ति॑ष्ठ बृह॒ती भ॑वो॒र्द्ध्वा ति॑ष्ठ ध्रु॒वा त्वं ॥ वस॑व॒स्त्वा ऽऽच्छृ॑न्दन्तु गाय॒त्रेण॒ छन्द॑साऽङ्गिर॒स्वद् रु॒द्रास्त्वाऽऽ च्छृ॑न्दन्तु॒ ( ) त्रैष्टु॑भेन॒ छन्द॑साऽङ्गिर॒स्व-दा॑दि॒त्यास्त्वा ऽऽच्छृ॑न्दन्तु॒ जाग॑तेन॒ छन्द॑साऽङ्गिर॒स्वद्-विश्वे᳚ त्वा दे॒वा वै᳚श्वान॒रा आ च्छृ॑न्द॒न्त्वानु॑ष्टुभेन॒ छन्द॑साऽङ्गिर॒स्वत् ॥ \newline

\textbf{Pada Paata} \newline

म॒हि॒ना । दिव᳚म् । मि॒त्रः । ब॒भू॒व॒ । स॒प्रथा॒ इति॑ स - प्रथाः᳚ ॥ उ॒त । श्रव॑सा । पृ॒थि॒वीम् ॥ मि॒त्रस्य॑ । च॒र्॒.ष॒णी॒धृत॒ इति॑ चर्.षणि - धृतः॑ । श्रवः॑ । दे॒वस्य॑ । सा॒न॒सिम् ॥ द्यु॒म्नम् । चि॒त्रश्र॑वस्तम॒मिति॑ चि॒त्रश्र॑वः - त॒म॒म् ॥ दे॒वः । त्वा॒ । स॒वि॒ता । उदिति॑ । व॒प॒तु॒ । सु॒पा॒णिरिति॑ सु - पा॒णिः । स्व॒ङ्गु॒रिरिति॑ सु - अ॒ङ्गु॒रिः ॥ सु॒बा॒हुरिति॑ सु - बा॒हुः । उ॒त । शक्त्या᳚ ॥ अप॑द्यमाना । पृ॒थि॒वि॒ । आशाः᳚ । दिशः॑ । एति॑ । पृ॒ण॒ ॥ उदिति॑ । ति॒ष्ठ॒ । बृ॒ह॒ती । भ॒व॒ । ऊ॒द्‌र्ध्वा । ति॒ष्ठ॒ । ध्रु॒वा । त्वम् ॥ वस॑वः । त्वा॒ । एति॑ । छृ॒न्द॒न्तु॒ । गा॒य॒त्रेण॑ । छन्द॑सा । अ॒ङ्गि॒र॒स्वत् । रु॒द्राः । त्वा॒ । एति॑ । छृ॒न्द॒न्तु॒ ( ) । त्रैष्टु॑भेन । छन्द॑सा । अ॒ङ्गि॒र॒स्वत् । आ॒दि॒त्याः । त्वा॒ । एति॑ । छृ॒न्द॒न्तु॒ । जाग॑तेन । छन्द॑सा । अ॒ङ्गि॒र॒स्वत् । विश्वे᳚ । त्वा॒ । दे॒वाः । वै॒श्वा॒न॒राः । एति॑ । छृ॒न्द॒न्तु॒ । आनु॑ष्टुभे॒नेत्यानु॑ - स्तु॒भे॒न॒ । छन्द॑सा । अ॒ङ्गि॒र॒स्वत् ॥  \newline


\textbf{Krama Paata} \newline

म॒हि॒ना दिव᳚म् । दिव॑म् मि॒त्रः । मि॒त्रो ब॑भूव । ब॒भू॒व॒ स॒प्रथाः᳚ । स॒प्रथा॒ इति॑ स - प्रथाः᳚ ॥ उ॒त श्रव॑सा । श्रव॑सा पृथि॒वीम् । पृ॒थि॒वीमिति॑ पृथि॒वीम् ॥ मि॒त्रस्य॑ चर्.षणी॒धृतः॑ । च॒र्॒.ष॒णी॒धृतः॒ श्रवः॑ । च॒र्॒.ष॒णी॒धृत॒ इति॑ चर्.षणि - धृतः॑ । श्रवो॑ दे॒वस्य॑ । दे॒वस्य॑ सान॒सिम् । सा॒न॒सिमिति॑ सान॒सिम् ॥ द्यु॒म्नम् चि॒त्रश्र॑वस्तमम् । चि॒त्रश्र॑वस्तम॒मिति॑ चि॒त्रश्र॑वः - त॒म॒म् ॥ दे॒वस्त्वा᳚ । त्वा॒ स॒वि॒ता । स॒वि॒तोत् । उद् व॑पतु । व॒प॒तु॒ सु॒पा॒णिः । सु॒पा॒णिः स्व॑ङ्गु॒रिः । सु॒पा॒णिरिति॑ सु - पा॒णिः । स्व॒ङ्गु॒रिरिति॑ सु - अ॒ङ्गु॒रिः ॥ सु॒बा॒हुरु॒त । सु॒बा॒हुरिति॑ सु - बा॒हुः । उ॒त शक्त्या᳚ । शक्त्येति॒ शक्त्या᳚ ॥ अप॑द्यमाना पृथिवि । पृ॒थि॒व्याशाः᳚ । आशा॒ दिशः॑ । दिश॒ आ । आ पृ॑ण । पृ॒णेति॑ पृण ॥ उत् ति॑ष्ठ । ति॒ष्ठ॒ बृ॒ह॒ती । बृ॒ह॒ती भ॑व । भ॒वो॒र्द्ध्वा । ऊ॒र्द्ध्वा ति॑ष्ठ । ति॒ष्ठ॒ ध्रु॒वा । धु॒वा त्वम् । त्वमिति॒ त्वम् ॥ वस॑वस्त्वा । त्वा । आ छृ॑न्दन्तु ( ) । छृ॒न्द॒न्तु॒ गा॒य॒त्रेण॑ । गा॒य॒त्रेण॒ छन्द॑सा । छन्द॑सा ऽङ्गिर॒स्वत् । अ॒ङ्गि॒र॒स्वद् रु॒द्राः । रु॒द्रास्त्वा᳚ । त्वा । आ छृ॑न्दन्तु । छृ॒न्द॒न्तु॒ त्रैष्टु॑भेन । त्रैष्टु॑भेन॒ छन्द॑सा । छन्द॑सा ऽङ्गिर॒स्वत् । अ॒ङ्गि॒र॒स्वदा॑दि॒त्याः । आ॒दि॒त्यास्त्वा᳚ । त्वा । आ छृ॑न्दन्तु । छृ॒न्द॒न्तु॒ जाग॑तेन । जाग॑तेन॒ छन्द॑सा । छन्द॑सा ऽङ्गिर॒स्वत् । अ॒ङ्गि॒र॒स्वद् विश्वे᳚ । विश्वे᳚ त्वा । त्वा॒ दे॒वाः । दे॒वा वै᳚श्वान॒राः । वै॒श्वा॒न॒रा आ । आ छृ॑न्दन्तु । छृ॒न्द॒न्त्वानु॑ष्टुभेन । आनु॑ष्टुभेन॒ छन्द॑सा । आनु॑ष्टुभे॒नेत्यानु॑ - स्तु॒भे॒न॒ । छन्द॑सा ऽङ्गिर॒स्वत् । अ॒ङ्गि॒र॒स्वदित्य॑ङ्गिर॒स्वत् । \newline

\textbf{Jatai Paata} \newline

1. म॒हि॒ना दिव॒म् दिव॑म् महि॒ना म॑हि॒ना दिव᳚म् । \newline
2. दिव॑म् मि॒त्रो मि॒त्रो दिव॒म् दिव॑म् मि॒त्रः । \newline
3. मि॒त्रो ब॑भूव बभूव मि॒त्रो मि॒त्रो ब॑भूव । \newline
4. ब॒भू॒व॒ स॒प्रथाः᳚ स॒प्रथा॑ बभूव बभूव स॒प्रथाः᳚ । \newline
5. स॒प्रथा॒ इति॑ स - प्रथाः᳚ । \newline
6. उ॒त श्रव॑सा॒ श्रव॑सो॒तोत श्रव॑सा । \newline
7. श्रव॑सा पृथि॒वीम् पृ॑थि॒वीꣳ श्रव॑सा॒ श्रव॑सा पृथि॒वीम् । \newline
8. पृ॒थि॒वीमिति॑ पृथि॒वीम् । \newline
9. मि॒त्रस्य॑ चर्.षणी॒धृत॑ श्चर्.षणी॒धृतो॑ मि॒त्रस्य॑ मि॒त्रस्य॑ चर्.षणी॒धृतः॑ । \newline
10. च॒र्॒.ष॒णी॒धृतः॒ श्रवः॒ श्रव॑श्चर्.षणी॒धृत॑ श्चर्.षणी॒धृतः॒ श्रवः॑ । \newline
11. च॒र्॒.ष॒णी॒धृत॒ इति॑ चर्.षणि - धृतः॑ । \newline
12. श्रवो॑ दे॒वस्य॑ दे॒वस्य॒ श्रवः॒ श्रवो॑ दे॒वस्य॑ । \newline
13. दे॒वस्य॑ सान॒सिꣳ सा॑न॒सिम् दे॒वस्य॑ दे॒वस्य॑ सान॒सिम् । \newline
14. सा॒न॒सिमिति॑ सान॒सिम् । \newline
15. द्यु॒म्नम् चि॒त्रश्र॑वस्तमम् चि॒त्रश्र॑वस्तमम् द्यु॒म्नम् द्यु॒म्नम् चि॒त्रश्र॑वस्तमम् । \newline
16. चि॒त्रश्र॑वस्तम॒मिति॑ चि॒त्रश्र॑वः - त॒म॒म् । \newline
17. दे॒व स्त्वा᳚ त्वा दे॒वो दे॒व स्त्वा᳚ । \newline
18. त्वा॒ स॒वि॒ता स॑वि॒ता त्वा᳚ त्वा सवि॒ता । \newline
19. स॒वि॒तोदुथ् स॑वि॒ता स॑वि॒तोत् । \newline
20. उद् व॑पतु वप॒तू दुद् व॑पतु । \newline
21. व॒प॒तु॒ सु॒पा॒णिः सु॑पा॒णिर् व॑पतु वपतु सुपा॒णिः । \newline
22. सु॒पा॒णिः स्व॑ङ्गु॒रिः स्व॑ङ्गु॒रिः सु॑पा॒णिः सु॑पा॒णिः स्व॑ङ्गु॒रिः । \newline
23. सु॒पा॒णिरिति॑ सु - पा॒णिः । \newline
24. स्व॒ङ्गु॒रिरिति॑ सु - अ॒ङ्गु॒रिः । \newline
25. सु॒बा॒हु रु॒तोत सु॑बा॒हुः सु॑बा॒हु रु॒त । \newline
26. सु॒बा॒हुरिति॑ सु - बा॒हुः । \newline
27. उ॒त शक्त्या॒ शक्त्यो॒तोत शक्त्या᳚ । \newline
28. शक्त्येति॒ शक्त्या᳚ । \newline
29. अप॑द्यमाना पृथिवि पृथि॒व्यप॑द्यमा॒ना ऽप॑द्यमाना पृथिवि । \newline
30. पृ॒थि॒व्याशा॒ आशाः᳚ पृथिवि पृथि॒व्याशाः᳚ । \newline
31. आशा॒ दिशो॒ दिश॒ आशा॒ आशा॒ दिशः॑ । \newline
32. दिश॒ आ दिशो॒ दिश॒ आ । \newline
33. आ पृ॑ण पृ॒णा पृ॑ण । \newline
34. पृ॒णेति॑ पृण । \newline
35. उत् ति॑ष्ठ ति॒ष्ठोदुत् ति॑ष्ठ । \newline
36. ति॒ष्ठ॒ बृ॒ह॒ती बृ॑ह॒ती ति॑ष्ठ तिष्ठ बृह॒ती । \newline
37. बृ॒ह॒ती भ॑व भव बृह॒ती बृ॑ह॒ती भ॑व । \newline
38. भ॒वो॒र्द्ध्वो र्द्ध्वा भ॑व भवो॒र्द्ध्वा । \newline
39. ऊ॒र्द्ध्वा ति॑ष्ठ तिष्ठो॒र्द्ध्वो र्द्ध्वा ति॑ष्ठ । \newline
40. ति॒ष्ठ॒ ध्रु॒वा ध्रु॒वा ति॑ष्ठ तिष्ठ ध्रु॒वा । \newline
41. ध्रु॒वा त्वम् त्वम् ध्रु॒वा ध्रु॒वा त्वम् । \newline
42. त्वमिति॒ त्वम् । \newline
43. वस॑व स्त्वा त्वा॒ वस॑वो॒ वस॑व स्त्वा । \newline
44. त्वा ऽऽत्वा॒ त्वा । \newline
45. आ च्छृ॑न्दन्तु छृन्द॒न् त्वा छृ॑न्दन्तु । \newline
46. छृ॒न्द॒न्तु॒ गा॒य॒त्रेण॑ गाय॒त्रेण॑ छृन्दन्तु छृन्दन्तु गाय॒त्रेण॑ । \newline
47. गा॒य॒त्रेण॒ छन्द॑सा॒ छन्द॑सा गाय॒त्रेण॑ गाय॒त्रेण॒ छन्द॑सा । \newline
48. छन्द॑सा ऽङ्गिर॒स्व द॑ङ्गिर॒स्व च्छन्द॑सा॒ छन्द॑सा ऽङ्गिर॒स्वत् । \newline
49. अ॒ङ्गि॒र॒स्वद् रु॒द्रा रु॒द्रा अ॑ङ्गिर॒स्व द॑ङ्गिर॒स्वद् रु॒द्राः । \newline
50. रु॒द्रा स्त्वा᳚ त्वा रु॒द्रा रु॒द्रा स्त्वा᳚ । \newline
51. त्वा ऽऽत्वा॒ त्वा । \newline
52. आ च्छृ॑न्दन्तु छृन्द॒न् त्वा छृ॑न्दन्तु । \newline
53. छृ॒न्द॒न्तु॒ त्रैष्टु॑भेन॒ त्रैष्टु॑भेन छृन्दन्तु छृन्दन्तु॒ त्रैष्टु॑भेन । \newline
54. त्रैष्टु॑भेन॒ छन्द॑सा॒ छन्द॑सा॒ त्रैष्टु॑भेन॒ त्रैष्टु॑भेन॒ छन्द॑सा । \newline
55. छन्द॑सा ऽङ्गिर॒स्व द॑ङ्गिर॒स्वच् छन्द॑सा॒ छन्द॑सा ऽङ्गिर॒स्वत् । \newline
56. अ॒ङ्गि॒र॒स्व दा॑दि॒त्या आ॑दि॒त्या अ॑ङ्गिर॒स्व द॑ङ्गिर॒स्व दा॑दि॒त्याः । \newline
57. आ॒दि॒त्या स्त्वा᳚ त्वा ऽऽदि॒त्या आ॑दि॒त्या स्त्वा᳚ । \newline
58. त्वा ऽऽत्वा॒ त्वा । \newline
59. आ च्छृ॑न्दन्तु छृन्द॒न्त्वा च्छृ॑न्दन्तु । \newline
60. छृ॒न्द॒न्तु॒ जाग॑तेन॒ जाग॑तेन छृन्दन्तु छृन्दन्तु॒ जाग॑तेन । \newline
61. जाग॑तेन॒ छन्द॑सा॒ छन्द॑सा॒ जाग॑तेन॒ जाग॑तेन॒ छन्द॑सा । \newline
62. छन्द॑सा ऽङ्गिर॒स्व द॑ङ्गिर॒स्वच् छन्द॑सा॒ छन्द॑सा ऽङ्गिर॒स्वत् । \newline
63. अ॒ङ्गि॒र॒स्वद् विश्वे॒ विश्वे᳚ ऽङ्गिर॒स्व द॑ङ्गिर॒स्वद् विश्वे᳚ । \newline
64. विश्वे᳚ त्वा त्वा॒ विश्वे॒ विश्वे᳚ त्वा । \newline
65. त्वा॒ दे॒वा दे॒वा स्त्वा᳚ त्वा दे॒वाः । \newline
66. दे॒वा वै᳚श्वान॒रा वै᳚श्वान॒रा दे॒वा दे॒वा वै᳚श्वान॒राः । \newline
67. वै॒श्वा॒न॒रा आ वै᳚श्वान॒रा वै᳚श्वान॒रा आ । \newline
68. आ च्छृ॑न्दन्तु छृन्द॒न् त्वा छृ॑न्दन्तु । \newline
69. छृ॒न्द॒न् त्वानु॑ष्टुभे॒ना नु॑ष्टुभेन छृन्दन्तु छृन्द॒न् त्वानु॑ष्टुभेन । \newline
70. आनु॑ष्टुभेन॒ छन्द॑सा॒ छन्द॒सा ऽऽनु॑ष्टुभे॒ना नु॑ष्टुभेन॒ छन्द॑सा । \newline
71. आनु॑ष्टुभे॒नेत्यानु॑ - स्तु॒भे॒न॒ । \newline
72. छन्द॑सा ऽङ्गिर॒स्व द॑ङ्गिर॒स्वच् छन्द॑सा॒ छन्द॑सा ऽङ्गिर॒स्वत् । \newline
73. अ॒ङ्गि॒र॒स्व दित्य॑ङ्गिर॒स्वत् । \newline

\textbf{Ghana Paata } \newline

1. म॒हि॒ना दिव॒म् दिव॑म् महि॒ना म॑हि॒ना दिव॑म् मि॒त्रो मि॒त्रो दिव॑म् महि॒ना म॑हि॒ना दिव॑म् मि॒त्रः । \newline
2. दिव॑म् मि॒त्रो मि॒त्रो दिव॒म् दिव॑म् मि॒त्रो ब॑भूव बभूव मि॒त्रो दिव॒म् दिव॑म् मि॒त्रो ब॑भूव । \newline
3. मि॒त्रो ब॑भूव बभूव मि॒त्रो मि॒त्रो ब॑भूव स॒प्रथाः᳚ स॒प्रथा॑ बभूव मि॒त्रो मि॒त्रो ब॑भूव स॒प्रथाः᳚ । \newline
4. ब॒भू॒व॒ स॒प्रथाः᳚ स॒प्रथा॑ बभूव बभूव स॒प्रथाः᳚ । \newline
5. स॒प्रथा॒ इति॑ स - प्रथाः᳚ । \newline
6. उ॒त श्रव॑सा॒ श्रव॑सो॒ तोत श्रव॑सा पृथि॒वीम् पृ॑थि॒वीꣳ श्रव॑सो॒ तोत श्रव॑सा पृथि॒वीम् । \newline
7. श्रव॑सा पृथि॒वीम् पृ॑थि॒वीꣳ श्रव॑सा॒ श्रव॑सा पृथि॒वीम् । \newline
8. पृ॒थि॒वीमिति॑ पृथि॒वीम् । \newline
9. मि॒त्रस्य॑ चर्.षणी॒धृत॑ श्चर्.षणी॒धृतो॑ मि॒त्रस्य॑ मि॒त्रस्य॑ चर्.षणी॒धृतः॒ श्रवः॒ श्रव॑ श्चर्.षणी॒धृतो॑ मि॒त्रस्य॑ मि॒त्रस्य॑ चर्.षणी॒धृतः॒ श्रवः॑ । \newline
10. च॒र्॒.ष॒णी॒धृतः॒ श्रवः॒ श्रव॑ श्चर्.षणी॒धृत॑ श्चर्.षणी॒धृतः॒ श्रवो॑ दे॒वस्य॑ दे॒वस्य॒ श्रव॑ श्चर्.षणी॒धृत॑ श्चर्.षणी॒धृतः॒ श्रवो॑ दे॒वस्य॑ । \newline
11. च॒र्॒.ष॒णी॒धृत॒ इति॑ चर्.षणि - धृतः॑ । \newline
12. श्रवो॑ दे॒वस्य॑ दे॒वस्य॒ श्रवः॒ श्रवो॑ दे॒वस्य॑ सान॒सिꣳ सा॑न॒सिम् दे॒वस्य॒ श्रवः॒ श्रवो॑ दे॒वस्य॑ सान॒सिम् । \newline
13. दे॒वस्य॑ सान॒सिꣳ सा॑न॒सिम् दे॒वस्य॑ दे॒वस्य॑ सान॒सिम् । \newline
14. सा॒न॒सिमिति॑ सान॒सिम् । \newline
15. द्यु॒म्नम् चि॒त्रश्र॑वस्तमम् चि॒त्रश्र॑वस्तमम् द्यु॒म्नम् द्यु॒म्नम् चि॒त्रश्र॑वस्तमम् । \newline
16. चि॒त्रश्र॑वस्तम॒मिति॑ चि॒त्रश्र॑वः - त॒म॒म् । \newline
17. दे॒व स्त्वा᳚ त्वा दे॒वो दे॒व स्त्वा॑ सवि॒ता स॑वि॒ता त्वा॑ दे॒वो दे॒व स्त्वा॑ सवि॒ता । \newline
18. त्वा॒ स॒वि॒ता स॑वि॒ता त्वा᳚ त्वा सवि॒तो दुथ् स॑वि॒ता त्वा᳚ त्वा सवि॒तोत् । \newline
19. स॒वि॒तो दुथ् स॑वि॒ता स॑वि॒तोद् व॑पतु वप॒तूथ् स॑वि॒ता स॑वि॒तोद् व॑पतु । \newline
20. उद् व॑पतु वप॒ तूदुद् व॑पतु सुपा॒णिः सु॑पा॒णिर् व॑प॒ तूदुद् व॑पतु सुपा॒णिः । \newline
21. व॒प॒तु॒ सु॒पा॒णिः सु॑पा॒णिर् व॑पतु वपतु सुपा॒णिः स्व॑ङ्गु॒रिः स्व॑ङ्गु॒रिः सु॑पा॒णिर् व॑पतु वपतु सुपा॒णिः स्व॑ङ्गु॒रिः । \newline
22. सु॒पा॒णिः स्व॑ङ्गु॒रिः स्व॑ङ्गु॒रिः सु॑पा॒णिः सु॑पा॒णिः स्व॑ङ्गु॒रिः । \newline
23. सु॒पा॒णिरिति॑ सु - पा॒णिः । \newline
24. स्व॒ङ्गु॒रिरिति॑ सु - अ॒ङ्गु॒रिः । \newline
25. सु॒बा॒हु रु॒तोत सु॑बा॒हुः सु॑बा॒हु रु॒त शक्त्या॒ शक्त्यो॒त सु॑बा॒हुः सु॑बा॒हु रु॒त शक्त्या᳚ । \newline
26. सु॒बा॒हुरिति॑ सु - बा॒हुः । \newline
27. उ॒त शक्त्या॒ शक्त्यो॒ तोत शक्त्या᳚ । \newline
28. शक्त्येति॒ शक्त्या᳚ । \newline
29. अप॑द्यमाना पृथिवि पृथि॒ व्यप॑द्यमा॒ना ऽप॑द्यमाना पृथि॒व्याशा॒ आशाः᳚ पृथि॒ व्यप॑द्यमा॒ना ऽप॑द्यमाना पृथि॒व्याशाः᳚ । \newline
30. पृ॒थि॒व्याशा॒ आशाः᳚ पृथिवि पृथि॒ व्याशा॒ दिशो॒ दिश॒ आशाः᳚ पृथिवि पृथि॒ व्याशा॒ दिशः॑ । \newline
31. आशा॒ दिशो॒ दिश॒ आशा॒ आशा॒ दिश॒ आ दिश॒ आशा॒ आशा॒ दिश॒ आ । \newline
32. दिश॒ आ दिशो॒ दिश॒ आ पृ॑ण पृ॒णा दिशो॒ दिश॒ आ पृ॑ण । \newline
33. आ पृ॑ण पृ॒णा पृ॑ण । \newline
34. पृ॒णेति॑ पृण । \newline
35. उत् ति॑ष्ठ ति॒ष्ठोदुत् ति॑ष्ठ बृह॒ती बृ॑ह॒ती ति॒ष्ठोदुत् ति॑ष्ठ बृह॒ती । \newline
36. ति॒ष्ठ॒ बृ॒ह॒ती बृ॑ह॒ती ति॑ष्ठ तिष्ठ बृह॒ती भ॑व भव बृह॒ती ति॑ष्ठ तिष्ठ बृह॒ती भ॑व । \newline
37. बृ॒ह॒ती भ॑व भव बृह॒ती बृ॑ह॒ती भ॑वो॒र् द्ध्वोर् द्ध्वा भ॑व बृह॒ती बृ॑ह॒ती भ॑वो॒ र्द्ध्वा । \newline
38. भ॒वो॒र् द्ध्वो र्द्ध्वा भ॑व भवो॒र्द्ध्वा ति॑ष्ठ तिष्ठो॒ र्द्ध्वा भ॑व भवो॒ र्द्ध्वा ति॑ष्ठ । \newline
39. ऊ॒र्द्ध्वा ति॑ष्ठ तिष्ठो॒र् द्ध्वो र्द्ध्वा ति॑ष्ठ ध्रु॒वा ध्रु॒वा ति॑ष्ठो॒र् द्ध्वो र्द्ध्वा ति॑ष्ठ ध्रु॒वा । \newline
40. ति॒ष्ठ॒ ध्रु॒वा ध्रु॒वा ति॑ष्ठ तिष्ठ ध्रु॒वा त्वम् त्वम् ध्रु॒वा ति॑ष्ठ तिष्ठ ध्रु॒वा त्वम् । \newline
41. ध्रु॒वा त्वम् त्वम् ध्रु॒वा ध्रु॒वा त्वम् । \newline
42. त्वमिति॒ त्वम् । \newline
43. वस॑व स्त्वा त्वा॒ वस॑वो॒ वस॑व॒ स्त्वा ऽऽत्वा॒ वस॑वो॒ वस॑व॒ स्त्वा । \newline
44. त्वा ऽऽत्वा॒ त्वा ऽऽच्छृ॑न्दन्तु छृन्द॒न् त्वा त्वा॒ त्वा ऽऽच्छृ॑न्दन्तु । \newline
45. आ छृ॑न्दन्तु छृन्द॒न् त्वा छृ॑न्दन्तु गाय॒त्रेण॑ गाय॒त्रेण॑ छृन्द॒न् त्वा छृ॑न्दन्तु गाय॒त्रेण॑ । \newline
46. छृ॒न्द॒न्तु॒ गा॒य॒त्रेण॑ गाय॒त्रेण॑ छृन्दन्तु छृन्दन्तु गाय॒त्रेण॒ छन्द॑सा॒ छन्द॑सा गाय॒त्रेण॑ छृन्दन्तु छृन्दन्तु गाय॒त्रेण॒ छन्द॑सा । \newline
47. गा॒य॒त्रेण॒ छन्द॑सा॒ छन्द॑सा गाय॒त्रेण॑ गाय॒त्रेण॒ छन्द॑सा ऽङ्गिर॒स्व द॑ङ्गिर॒स्व च्छन्द॑सा गाय॒त्रेण॑ गाय॒त्रेण॒ छन्द॑सा ऽङ्गिर॒स्वत् । \newline
48. छन्द॑सा ऽङ्गिर॒स्व द॑ङ्गिर॒स्व च्छन्द॑सा॒ छन्द॑सा ऽङ्गिर॒स्वद् रु॒द्रा रु॒द्रा अ॑ङ्गिर॒स्व च्छन्द॑सा॒ छन्द॑सा ऽङ्गिर॒स्वद् रु॒द्राः । \newline
49. अ॒ङ्गि॒र॒स्वद् रु॒द्रा रु॒द्रा अ॑ङ्गिर॒स्व द॑ङ्गिर॒स्वद् रु॒द्रा स्त्वा᳚ त्वा रु॒द्रा अ॑ङ्गिर॒स्व द॑ङ्गिर॒स्वद् रु॒द्रा स्त्वा᳚ । \newline
50. रु॒द्रा स्त्वा᳚ त्वा रु॒द्रा रु॒द्रा स्त्वा ऽऽत्वा॑ रु॒द्रा रु॒द्रा स्त्वा । \newline
51. त्वा ऽऽत्वा॒ त्वा ऽऽच्छृ॑न्दन्तु छृन्द॒न् त्वा त्वा॒ त्वा ऽऽच्छृ॑न्दन्तु । \newline
52. आ च्छृ॑न्दन्तु छृन्द॒न् त्वा छृ॑न्दन्तु॒ त्रैष्टु॑भेन॒ त्रैष्टु॑भेन छृन्द॒न् त्वा छृ॑न्दन्तु॒ त्रैष्टु॑भेन । \newline
53. छृ॒न्द॒न्तु॒ त्रैष्टु॑भेन॒ त्रैष्टु॑भेन छृन्दन्तु छृन्दन्तु॒ त्रैष्टु॑भेन॒ छन्द॑सा॒ छन्द॑सा॒ त्रैष्टु॑भेन छृन्दन्तु छृन्दन्तु॒ त्रैष्टु॑भेन॒ छन्द॑सा । \newline
54. त्रैष्टु॑भेन॒ छन्द॑सा॒ छन्द॑सा॒ त्रैष्टु॑भेन॒ त्रैष्टु॑भेन॒ छन्द॑सा ऽङ्गिर॒स्व द॑ङ्गिर॒स्व च्छन्द॑सा॒ त्रैष्टु॑भेन॒ त्रैष्टु॑भेन॒ छन्द॑सा ऽङ्गिर॒स्वत् । \newline
55. छन्द॑सा ऽङ्गिर॒स्व द॑ङ्गिर॒स्व च्छन्द॑सा॒ छन्द॑सा ऽङ्गिर॒स्व दा॑दि॒त्या आ॑दि॒त्या अ॑ङ्गिर॒स्व च्छन्द॑सा॒ छन्द॑सा ऽङ्गिर॒स्व दा॑दि॒त्याः । \newline
56. अ॒ङ्गि॒र॒स्व दा॑दि॒त्या आ॑दि॒त्या अ॑ङ्गिर॒स्व द॑ङ्गिर॒स्व दा॑दि॒त्या स्त्वा᳚ त्वा ऽऽदि॒त्या अ॑ङ्गिर॒स्व द॑ङ्गिर॒स्व दा॑दि॒त्या स्त्वा᳚ । \newline
57. आ॒दि॒त्या स्त्वा᳚ त्वा ऽऽदि॒त्या आ॑दि॒त्या स्त्वा ऽऽत्वा॑ ऽऽदि॒त्या आ॑दि॒त्या स्त्वा । \newline
58. त्वा ऽऽत्वा॒ त्वा ऽऽच्छृ॑न्दन्तु छृन्द॒न् त्वा त्वा॒ त्वा ऽऽच्छृ॑न्दन्तु । \newline
59. आ च्छृ॑न्दन्तु छृन्द॒न् त्वा छृ॑न्दन्तु॒ जाग॑तेन॒ जाग॑तेन छृन्द॒न् त्वा छृ॑न्दन्तु॒ जाग॑तेन । \newline
60. छृ॒न्द॒न्तु॒ जाग॑तेन॒ जाग॑तेन छृन्दन्तु छृन्दन्तु॒ जाग॑तेन॒ छन्द॑सा॒ छन्द॑सा॒ जाग॑तेन छृन्दन्तु छृन्दन्तु॒ जाग॑तेन॒ छन्द॑सा । \newline
61. जाग॑तेन॒ छन्द॑सा॒ छन्द॑सा॒ जाग॑तेन॒ जाग॑तेन॒ छन्द॑सा ऽङ्गिर॒स्व द॑ङ्गिर॒स्व च्छन्द॑सा॒ जाग॑तेन॒ जाग॑तेन॒ छन्द॑सा ऽङ्गिर॒स्वत् । \newline
62. छन्द॑सा ऽङ्गिर॒स्व द॑ङ्गिर॒स्व च्छन्द॑सा॒ छन्द॑सा ऽङ्गिर॒स्वद् विश्वे॒ विश्वे᳚ ऽङ्गिर॒स्व च्छन्द॑सा॒ छन्द॑सा ऽङ्गिर॒स्वद् विश्वे᳚ । \newline
63. अ॒ङ्गि॒र॒स्वद् विश्वे॒ विश्वे᳚ ऽङ्गिर॒स्व द॑ङ्गिर॒स्वद् विश्वे᳚ त्वा त्वा॒ विश्वे᳚ ऽङ्गिर॒स्व द॑ङ्गिर॒स्वद् विश्वे᳚ त्वा । \newline
64. विश्वे᳚ त्वा त्वा॒ विश्वे॒ विश्वे᳚ त्वा दे॒वा दे॒वा स्त्वा॒ विश्वे॒ विश्वे᳚ त्वा दे॒वाः । \newline
65. त्वा॒ दे॒वा दे॒वा स्त्वा᳚ त्वा दे॒वा वै᳚श्वान॒रा वै᳚श्वान॒रा दे॒वा स्त्वा᳚ त्वा दे॒वा वै᳚श्वान॒राः । \newline
66. दे॒वा वै᳚श्वान॒रा वै᳚श्वान॒रा दे॒वा दे॒वा वै᳚श्वान॒रा आ वै᳚श्वान॒रा दे॒वा दे॒वा वै᳚श्वान॒रा आ । \newline
67. वै॒श्वा॒न॒रा आ वै᳚श्वान॒रा वै᳚श्वान॒रा आ च्छृ॑न्दन्तु छृन्दन्तु॒दा वै᳚श्वान॒रा वै᳚श्वान॒रा आ च्छृ॑न्दन्तु । \newline
68. आ छृ॑न्दन्तु छृन्द॒न् त्वा छृ॑न्द॒ न्त्वानु॑ष्टुभे॒ना नु॑ष्टुभेन छृन्द॒न् त्वा छृ॑न्द॒न्त्वा नु॑ष्टुभेन । \newline
69. छृ॒न्द॒न् त्वानु॑ष्टुभे॒ना नु॑ष्टुभेन छृन्दन्तु छृन्द॒न् त्वानु॑ष्टुभेन॒ छन्द॑सा॒ छन्द॒सा ऽऽनु॑ष्टुभेन छृन्दन्तु छृन्द॒न् त्वानु॑ष्टुभेन॒ छन्द॑सा । \newline
70. आनु॑ष्टुभेन॒ छन्द॑सा॒ छन्द॒सा ऽऽनु॑ष्टुभे॒ना नु॑ष्टुभेन॒ छन्द॑सा ऽङ्गिर॒स्व द॑ङ्गिर॒स्व च्छन्द॒सा ऽऽनु॑ष्टुभे॒ना नु॑ष्टुभेन॒ छन्द॑सा ऽङ्गिर॒स्वत् । \newline
71. आनु॑ष्टुभे॒नेत्यानु॑ - स्तु॒भे॒न॒ । \newline
72. छन्द॑सा ऽङ्गिर॒स्व द॑ङ्गिर॒स्व च्छन्द॑सा॒ छन्द॑सा ऽङ्गिर॒स्वत् । \newline
73. अ॒ङ्गि॒र॒स्वदित्य॑ङ्गिर॒स्वत् । \newline
\pagebreak
\markright{ TS 4.1.7.1  \hfill https://www.vedavms.in \hfill}

\section{ TS 4.1.7.1 }

\textbf{TS 4.1.7.1 } \newline
\textbf{Samhita Paata} \newline

समा᳚स्त्वाऽग्न ऋ॒तवो॑ वर्द्धयन्तु संॅवथ्स॒रा ऋष॑यो॒ यानि॑ स॒त्या । सं दि॒व्येन॑ दीदिहि रोच॒नेन॒ विश्वा॒ आ भा॑हि प्र॒दिशः॑ पृथि॒व्याः ॥ सं चे॒द्ध्यस्वा᳚ऽग्ने॒ प्र च॑ बोधयैन॒मुच्च॑ तिष्ठ मह॒ते सौभ॑गाय । मा च॑ रिषदुपस॒त्ता ते॑ अग्ने ब्र॒ह्माण॑स्ते य॒शसः॑ सन्तु॒ माऽन्ये ॥ त्वाम॑ग्ने वृणते ब्राह्म॒णा इ॒मे शि॒वो अ॑ग्ने - [  ] \newline

\textbf{Pada Paata} \newline

समाः᳚ । त्वा॒ । अ॒ग्ने॒ । ऋ॒तवः॑ । व॒द्‌र्ध॒य॒न्तु॒ । स॒ॅवं॒थ्स॒रा इति॑ सं-व॒थ्स॒राः । ऋष॑यः । यानि॑ । स॒त्या ॥ समिति॑ । दि॒व्येन॑ । दी॒दि॒हि॒ । रो॒च॒नेन॑ । विश्वाः᳚ । एति॑ । भा॒हि॒ । प्र॒दिश॒ इति॑ प्र - दिशः॑ । पृ॒थि॒व्याः ॥ समिति॑ । च॒ । इ॒द्ध्यस्व॑ । अ॒ग्ने॒ । प्रेति॑ । च॒ । बो॒ध॒य॒ । ए॒न॒म् । उदिति॑ । च॒ । ति॒ष्ठ॒ । म॒ह॒ते । सौभ॑गाय ॥ मा । च॒ । रि॒ष॒त् । उ॒प॒स॒त्तेत्यु॑प - स॒त्ता । ते॒ । अ॒ग्ने॒ । ब्र॒ह्माणः॑ । ते॒ । य॒शसः॑ । स॒न्तु॒ । मा । अ॒न्ये ॥ त्वाम् । अ॒ग्ने॒ । वृ॒ण॒ते॒ । ब्रा॒ह्म॒णाः । इ॒मे । शि॒वः । अ॒ग्ने॒ ।  \newline


\textbf{Krama Paata} \newline

समा᳚स्त्वा । त्वा॒ऽग्ने॒ । अ॒ग्न॒ ऋ॒तवः॑ । ऋ॒तवो॑ वर्द्धयन्तु । व॒र्द्ध॒य॒न्तु॒ स॒म्ॅव॒थ्स॒राः । स॒म्ॅव॒थ्स॒रा ऋष॑यः । स॒म्ॅव॒थ्स॒रा इति॑ सं - व॒थ्स॒राः । ऋष॑यो॒ यानि॑ । यानि॑ स॒त्या । स॒त्येति॑ स॒त्या ॥ सम् दि॒व्येन॑ । दि॒व्येन॑ दीदिहि । दी॒दि॒हि॒ रो॒च॒नेन॑ । रो॒च॒नेन॒ विश्वाः᳚ । विश्वा॒ आ । आ भा॑हि । भा॒हि॒ प्र॒दिशः॑ । प्र॒दिशः॑ पृथि॒व्याः । प्र॒दिश॒ इति॑ प्र - दिशः॑ । पृ॒थि॒व्या इति॑ पृथि॒व्याः ॥ सञ्च॑ । चे॒द्ध्यस्व॑ । इ॒द्ध्यस्वा᳚ग्ने । अ॒ग्ने॒ प्र । प्र च॑ । च॒ बो॒ध॒य॒ । बो॒ध॒यै॒न॒म् । ए॒न॒मुत् । उच्च॑ । च॒ ति॒ष्ठ॒ । ति॒ष्ठ॒ म॒ह॒ते । म॒ह॒ते सौभ॑गाय । सौभ॑गा॒येति॒ सौभ॑गाय ॥ मा च॑ । च॒ रि॒ष॒त्॒ । रि॒ष॒दु॒प॒स॒त्ता । उ॒प॒स॒त्ता ते᳚ । उ॒प॒स॒त्तेत्यु॑प - स॒त्ता । ते॒ अ॒ग्ने॒ । अ॒ग्ने॒ ब्र॒ह्माणः॑ । ब्र॒ह्माण॑स्ते । ते॒ य॒शसः॑ । य॒शसः॑ सन्तु । स॒न्तु॒ मा । माऽन्ये । अ॒न्य इत्य॒न्ये ॥ त्वाम॑ग्ने । अ॒ग्ने॒ वृ॒ण॒ते॒ । वृ॒ण॒ते॒ ब्रा॒ह्म॒णाः । ब्रा॒ह्म॒णा इ॒मे । इ॒मे शि॒वः । शि॒वो अ॑ग्ने । अ॒ग्ने॒ स॒म्ॅवर॑णे \newline

\textbf{Jatai Paata} \newline

1. समा᳚ स्त्वा त्वा॒ समाः॒ समा᳚ स्त्वा । \newline
2. त्वा॒ ऽग्ने॒ अ॒ग्ने॒ त्वा॒ त्वा॒ ऽग्ने॒ । \newline
3. अ॒ग्न॒ ऋ॒तव॑ ऋ॒तवो॑ अग्ने अग्न ऋ॒तवः॑ । \newline
4. ऋ॒तवो॑ वर्द्धयन्तु वर्द्धयन् त्वृ॒तव॑ ऋ॒तवो॑ वर्द्धयन्तु । \newline
5. व॒र्द्ध॒य॒न्तु॒ सं॒ॅव॒थ्स॒राः सं॑ॅवथ्स॒रा व॑र्द्धयन्तु वर्द्धयन्तु संॅवथ्स॒राः । \newline
6. सं॒ॅव॒थ्स॒रा ऋष॑य॒ ऋष॑यः संॅवथ्स॒राः सं॑ॅवथ्स॒रा ऋष॑यः । \newline
7. सं॒ॅव॒थ्स॒रा इति॑ सं - व॒थ्स॒राः । \newline
8. ऋष॑यो॒ यानि॒ यान्यृष॑य॒ ऋष॑यो॒ यानि॑ । \newline
9. यानि॑ स॒त्या स॒त्या यानि॒ यानि॑ स॒त्या । \newline
10. स॒त्येति॑ स॒त्या । \newline
11. सम् दि॒व्येन॑ दि॒व्येन॒ सꣳ सम् दि॒व्येन॑ । \newline
12. दि॒व्येन॑ दीदिहि दीदिहि दि॒व्येन॑ दि॒व्येन॑ दीदिहि । \newline
13. दी॒दि॒हि॒ रो॒च॒नेन॑ रोच॒नेन॑ दीदिहि दीदिहि रोच॒नेन॑ । \newline
14. रो॒च॒नेन॒ विश्वा॒ विश्वा॑ रोच॒नेन॑ रोच॒नेन॒ विश्वाः᳚ । \newline
15. विश्वा॒ आ विश्वा॒ विश्वा॒ आ । \newline
16. आ भा॑हि भा॒ह्या भा॑हि । \newline
17. भा॒हि॒ प्र॒दिशः॑ प्र॒दिशो॑ भाहि भाहि प्र॒दिशः॑ । \newline
18. प्र॒दिशः॑ पृथि॒व्याः पृ॑थि॒व्याः प्र॒दिशः॑ प्र॒दिशः॑ पृथि॒व्याः । \newline
19. प्र॒दिश॒ इति॑ प्र - दिशः॑ । \newline
20. पृ॒थि॒व्या इति॑ पृथि॒व्याः । \newline
21. सम् च॑ च॒ सꣳ सम् च॑ । \newline
22. चे॒ द्ध्यस्वे॒ द्ध्यस्व॑ च चे॒ द्ध्यस्व॑ । \newline
23. इ॒द्ध्यस्वा᳚ग्ने अग्न इ॒द्ध्यस्वे॒ द्ध्यस्वा᳚ग्ने । \newline
24. अ॒ग्ने॒ प्र प्राग्ने॑ अग्ने॒ प्र । \newline
25. प्र च॑ च॒ प्र प्र च॑ । \newline
26. च॒ बो॒ध॒य॒ बो॒ध॒य॒ च॒ च॒ बो॒ध॒य॒ । \newline
27. बो॒ध॒यै॒न॒ मे॒न॒म् बो॒ध॒य॒ बो॒ध॒यै॒न॒म् । \newline
28. ए॒न॒ मुदुदे॑न मेन॒ मुत् । \newline
29. उच् च॒ चोदुच् च॑ । \newline
30. च॒ ति॒ष्ठ॒ ति॒ष्ठ॒ च॒ च॒ ति॒ष्ठ॒ । \newline
31. ति॒ष्ठ॒ म॒ह॒ते म॑ह॒ते ति॑ष्ठ तिष्ठ मह॒ते । \newline
32. म॒ह॒ते सौभ॑गाय॒ सौभ॑गाय मह॒ते म॑ह॒ते सौभ॑गाय । \newline
33. सौभ॑गा॒येति॒ सौभ॑गाय । \newline
34. मा च॑ च॒ मा मा च॑ । \newline
35. च॒ रि॒ष॒द् रि॒ष॒च् च॒ च॒ रि॒ष॒त् । \newline
36. रि॒ष॒ दु॒प॒स॒त्तो प॑स॒त्ता रि॑षद् रिष दुपस॒त्ता । \newline
37. उ॒प॒स॒त्ता ते॑ त उपस॒त्तो प॑स॒त्ता ते᳚ । \newline
38. उ॒प॒स॒त्तेत्यु॑प - स॒त्ता । \newline
39. ते॒ अ॒ग्ने॒ अ॒ग्ने॒ ते॒ ते॒ अ॒ग्ने॒ । \newline
40. अ॒ग्ने॒ ब्र॒ह्माणो᳚ ब्र॒ह्माणो॑ अग्ने अग्ने ब्र॒ह्माणः॑ । \newline
41. ब्र॒ह्माण॑ स्ते ते ब्र॒ह्माणो᳚ ब्र॒ह्माण॑ स्ते । \newline
42. ते॒ य॒शसो॑ य॒शस॑ स्ते ते य॒शसः॑ । \newline
43. य॒शसः॑ सन्तु सन्तु य॒शसो॑ य॒शसः॑ सन्तु । \newline
44. स॒न्तु॒ मा मा स॑न्तु सन्तु॒ मा । \newline
45. मा ऽन्ये अ॒न्ये मा मा ऽन्ये । \newline
46. अ॒न्य इत्य॒न्ये । \newline
47. त्वा म॑ग्ने अग्ने॒ त्वाम् त्वा म॑ग्ने । \newline
48. अ॒ग्ने॒ वृ॒ण॒ते॒ वृ॒ण॒ते॒ अ॒ग्ने॒ अ॒ग्ने॒ वृ॒ण॒ते॒ । \newline
49. वृ॒ण॒ते॒ ब्रा॒ह्म॒णा ब्रा᳚ह्म॒णा वृ॑णते वृणते ब्राह्म॒णाः । \newline
50. ब्रा॒ह्म॒णा इ॒म इ॒मे ब्रा᳚ह्म॒णा ब्रा᳚ह्म॒णा इ॒मे । \newline
51. इ॒मे शि॒वः शि॒व इ॒म इ॒मे शि॒वः । \newline
52. शि॒वो अ॑ग्ने अग्ने शि॒वः शि॒वो अ॑ग्ने । \newline
53. अ॒ग्ने॒ सं॒ॅवर॑णे सं॒ॅवर॑णे अग्ने अग्ने सं॒ॅवर॑णे । \newline

\textbf{Ghana Paata } \newline

1. समा᳚ स्त्वा त्वा॒ समाः॒ समा᳚ स्त्वा ऽग्ने अग्ने त्वा॒ समाः॒ समा᳚ स्त्वा ऽग्ने । \newline
2. त्वा॒ ऽग्ने॒ अ॒ग्ने॒ त्वा॒ त्वा॒ ऽग्न॒ ऋ॒तव॑ ऋ॒तवो॑ अग्ने त्वा त्वा ऽग्न ऋ॒तवः॑ । \newline
3. अ॒ग्न॒ ऋ॒तव॑ ऋ॒तवो॑ अग्ने अग्न ऋ॒तवो॑ वर्द्धयन्तु वर्द्धयन् त्वृ॒तवो॑ अग्ने अग्न ऋ॒तवो॑ वर्द्धयन्तु । \newline
4. ऋ॒तवो॑ वर्द्धयन्तु वर्द्धयन् त्वृ॒तव॑ ऋ॒तवो॑ वर्द्धयन्तु सम्ॅवंथ्स॒राः स॑म्ॅवंथ्स॒रा 
व॑र्द्धयन् त्वृ॒तव॑ ऋ॒तवो॑ वर्द्धयन्तु सम्ॅवथ्स॒राः । \newline
5. व॒र्द्ध॒य॒न्तु॒ स॒म्ॅव॒थ्स॒राः स॑म्ॅवथ्स॒रा व॑र्द्धयन्तु वर्द्धयन्तु सम्ॅवथ्स॒रा ऋष॑य॒ ऋष॑यः सम्ॅवथ्स॒रा व॑र्द्धयन्तु वर्द्धयन्तु स॑म्ॅवथ्स॒रा ऋष॑यः । \newline
6. सं॒ॅव॒थ्स॒रा ऋष॑य॒ ऋष॑यः संॅवथ्स॒राः सं॑ॅवथ्स॒रा ऋष॑यो॒ यानि॒ यान् यृष॑यः संॅवथ्स॒राः सं॑ॅवथ्स॒रा ऋष॑यो॒ यानि॑ । \newline
7. सं॒ॅव॒थ्स॒रा इति॑ सं - व॒थ्स॒राः । \newline
8. ऋष॑यो॒ यानि॒ यान् यृष॑य॒ ऋष॑यो॒ यानि॑ स॒त्या स॒त्या यान् यृष॑य॒ ऋष॑यो॒ यानि॑ स॒त्या । \newline
9. यानि॑ स॒त्या स॒त्या यानि॒ यानि॑ स॒त्या । \newline
10. स॒त्येति॑ स॒त्या । \newline
11. सम् दि॒व्येन॑ दि॒व्येन॒ सꣳ सम् दि॒व्येन॑ दीदिहि दीदिहि दि॒व्येन॒ सꣳ सम् दि॒व्येन॑ दीदिहि । \newline
12. दि॒व्येन॑ दीदिहि दीदिहि दि॒व्येन॑ दि॒व्येन॑ दीदिहि रोच॒नेन॑ रोच॒नेन॑ दीदिहि दि॒व्येन॑ दि॒व्येन॑ दीदिहि रोच॒नेन॑ । \newline
13. दी॒दि॒हि॒ रो॒च॒नेन॑ रोच॒नेन॑ दीदिहि दीदिहि रोच॒नेन॒ विश्वा॒ विश्वा॑ रोच॒नेन॑ दीदिहि दीदिहि रोच॒नेन॒ विश्वाः᳚ । \newline
14. रो॒च॒नेन॒ विश्वा॒ विश्वा॑ रोच॒नेन॑ रोच॒नेन॒ विश्वा॒ आ विश्वा॑ रोच॒नेन॑ रोच॒नेन॒ विश्वा॒ आ । \newline
15. विश्वा॒ आ विश्वा॒ विश्वा॒ आ भा॑हि भा॒ह्या विश्वा॒ विश्वा॒ आ भा॑हि । \newline
16. आ भा॑हि भा॒ह्या भा॑हि प्र॒दिशः॑ प्र॒दिशो॑ भा॒ह्या भा॑हि प्र॒दिशः॑ । \newline
17. भा॒हि॒ प्र॒दिशः॑ प्र॒दिशो॑ भाहि भाहि प्र॒दिशः॑ पृथि॒व्याः पृ॑थि॒व्याः प्र॒दिशो॑ भाहि भाहि प्र॒दिशः॑ पृथि॒व्याः । \newline
18. प्र॒दिशः॑ पृथि॒व्याः पृ॑थि॒व्याः प्र॒दिशः॑ प्र॒दिशः॑ पृथि॒व्याः । \newline
19. प्र॒दिश॒ इति॑ प्र - दिशः॑ । \newline
20. पृ॒थि॒व्या इति॑ पृथि॒व्याः । \newline
21. सम् च॑ च॒ सꣳ सम् चे॒द्ध्य स्वे॒द्ध्यस्व॑ च॒ सꣳ सम् चे॒द्ध्यस्व॑ । \newline
22. चे॒द्ध्य स्वे॒द्ध्यस्व॑ च चे॒ द्ध्यस्वा᳚ग्ने अग्न इ॒द्ध्यस्व॑ च चे॒द्ध्यस्वा᳚ग्ने । \newline
23. इ॒द्ध्य स्वा᳚ग्ने अग्न इ॒द्ध्य स्वे॒द्ध्य स्वा᳚ग्ने॒ प्र प्राग्न॑ इ॒द्ध्य स्वे॒द्ध्य स्वा᳚ग्ने॒ प्र । \newline
24. अ॒ग्ने॒ प्र प्राग्ने॑ अग्ने॒ प्र च॑ च॒ प्राग्ने॑ अग्ने॒ प्र च॑ । \newline
25. प्र च॑ च॒ प्र प्र च॑ बोधय बोधय च॒ प्र प्र च॑ बोधय । \newline
26. च॒ बो॒ध॒य॒ बो॒ध॒य॒ च॒ च॒ बो॒ध॒ यै॒न॒ मे॒न॒म् बो॒ध॒य॒ च॒ च॒ बो॒ध॒ यै॒न॒म् । \newline
27. बो॒ध॒ यै॒न॒ मे॒न॒म् बो॒ध॒य॒ बो॒ध॒ यै॒न॒ मुदु दे॑नम् बोधय बोध यैन॒ मुत् । \newline
28. ए॒न॒ मुदु दे॑न मेन॒ मुच्च॒ चोदे॑न मेन॒ मुच्च॑ । \newline
29. उच् च॒ चोदुच्च॑ तिष्ठ तिष्ठ॒ चोदुच्च॑ तिष्ठ । \newline
30. च॒ ति॒ष्ठ॒ ति॒ष्ठ॒ च॒ च॒ ति॒ष्ठ॒ म॒ह॒ते म॑ह॒ते ति॑ष्ठ च च तिष्ठ मह॒ते । \newline
31. ति॒ष्ठ॒ म॒ह॒ते म॑ह॒ते ति॑ष्ठ तिष्ठ मह॒ते सौभ॑गाय॒ सौभ॑गाय मह॒ते ति॑ष्ठ तिष्ठ मह॒ते सौभ॑गाय । \newline
32. म॒ह॒ते सौभ॑गाय॒ सौभ॑गाय मह॒ते म॑ह॒ते सौभ॑गाय । \newline
33. सौभ॑गा॒येति॒ सौभ॑गाय । \newline
34. मा च॑ च॒ मा मा च॑ रिषद् रिषच्च॒ मा मा च॑ रिषत् । \newline
35. च॒ रि॒ष॒द् रि॒ष॒च् च॒ च॒ रि॒ष॒ दु॒प॒स॒त्तो प॑स॒त्ता रि॑षच् च च रिष दुपस॒त्ता । \newline
36. रि॒ष॒ दु॒प॒स ॒त्तोप॑स॒त्ता रि॑षद् रिष दुपस॒त्ता ते॑ त उपस॒त्ता रि॑षद् रिषदु पस॒त्ता ते᳚ । \newline
37. उ॒प॒स॒त्ता ते॑ त उपस॒ त्तोप॑स॒त्ता ते॑ अग्ने अग्ने त उपस॒ त्तोप॑स॒त्ता ते॑ अग्ने । \newline
38. उ॒प॒स॒त्तेत्यु॑प - स॒त्ता । \newline
39. ते॒ अ॒ग्ने॒ अ॒ग्ने॒ ते॒ ते॒ अ॒ग्ने॒ ब्र॒ह्माणो᳚ ब्र॒ह्माणो॑ अग्ने ते ते अग्ने ब्र॒ह्माणः॑ । \newline
40. अ॒ग्ने॒ ब्र॒ह्माणो᳚ ब्र॒ह्माणो॑ अग्ने अग्ने ब्र॒ह्माण॑ स्ते ते ब्र॒ह्माणो॑ अग्ने अग्ने ब्र॒ह्माण॑ स्ते । \newline
41. ब्र॒ह्माण॑ स्ते ते ब्र॒ह्माणो᳚ ब्र॒ह्माण॑ स्ते य॒शसो॑ य॒शस॑ स्ते ब्र॒ह्माणो᳚ ब्र॒ह्माण॑ स्ते य॒शसः॑ । \newline
42. ते॒ य॒शसो॑ य॒शस॑ स्ते ते य॒शसः॑ सन्तु सन्तु य॒शस॑ स्ते ते य॒शसः॑ सन्तु । \newline
43. य॒शसः॑ सन्तु सन्तु य॒शसो॑ य॒शसः॑ सन्तु॒ मा मा स॑न्तु य॒शसो॑ य॒शसः॑ सन्तु॒ मा । \newline
44. स॒न्तु॒ मा मा स॑न्तु सन्तु॒ मा ऽन्ये अ॒न्ये मा स॑न्तु सन्तु॒ मा ऽन्ये । \newline
45. मा ऽन्ये अ॒न्ये मा मा ऽन्ये । \newline
46. अ॒न्य इत्य॒न्ये । \newline
47. त्वा म॑ग्ने अग्ने॒ त्वाम् त्वा म॑ग्ने वृणते वृणते अग्ने॒ त्वाम् त्वा म॑ग्ने वृणते । \newline
48. अ॒ग्ने॒ वृ॒ण॒ते॒ वृ॒ण॒ते॒ अ॒ग्ने॒ अ॒ग्ने॒ वृ॒ण॒ते॒ ब्रा॒ह्म॒णा ब्रा᳚ह्म॒णा वृ॑णते अग्ने अग्ने वृणते ब्राह्म॒णाः । \newline
49. वृ॒ण॒ते॒ ब्रा॒ह्म॒णा ब्रा᳚ह्म॒णा वृ॑णते वृणते ब्राह्म॒णा इ॒म इ॒मे ब्रा᳚ह्म॒णा वृ॑णते वृणते ब्राह्म॒णा इ॒मे । \newline
50. ब्रा॒ह्म॒णा इ॒म इ॒मे ब्रा᳚ह्म॒णा ब्रा᳚ह्म॒णा इ॒मे शि॒वः शि॒व इ॒मे ब्रा᳚ह्म॒णा ब्रा᳚ह्म॒णा इ॒मे शि॒वः । \newline
51. इ॒मे शि॒वः शि॒व इ॒म इ॒मे शि॒वो अ॑ग्ने अग्ने शि॒व इ॒म इ॒मे शि॒वो अ॑ग्ने । \newline
52. शि॒वो अ॑ग्ने अग्ने शि॒वः शि॒वो अ॑ग्ने सं॒ॅवर॑णे सं॒ॅवर॑णे अग्ने शि॒वः शि॒वो अ॑ग्ने सं॒ॅवर॑णे । \newline
53. अ॒ग्ने॒ सं॒ॅवर॑णे सं॒ॅवर॑णे अग्ने अग्ने सं॒ॅवर॑णे भव भव सं॒ॅवर॑णे अग्ने अग्ने सं॒ॅवर॑णे भव । \newline
\pagebreak
\markright{ TS 4.1.7.2  \hfill https://www.vedavms.in \hfill}

\section{ TS 4.1.7.2 }

\textbf{TS 4.1.7.2 } \newline
\textbf{Samhita Paata} \newline

सं॒ ॅवर॑णे भवा नः । स॒प॒त्न॒हा नो॑ अभिमाति॒जिच्च॒ स्वे गये॑ जागृ॒ह्य प्र॑युच्छन्न् ॥ इ॒हैवाग्ने॒ अधि॑ धारया र॒यिं मा त्वा॒ निक्र॑न् पूर्व॒चितो॑ निका॒रिणः॑ । क्ष॒त्रम॑ग्ने सु॒यम॑मस्तु॒ तुभ्य॑मुपस॒त्ता व॑र्द्धतां ते॒ अनि॑ष्टृतः ॥ क्ष॒त्रेणा᳚ऽग्ने॒ स्वायुः॒ सꣳ र॑भस्व मि॒त्रेणा᳚ऽग्ने मित्र॒धेये॑ यतस्व । स॒जा॒तानां᳚ मद्ध्यम॒स्था ए॑धि॒ राज्ञा॑मग्ने विह॒व्यो॑ दीदिही॒ह ॥ अति॒ - [  ] \newline

\textbf{Pada Paata} \newline

सं॒ॅवर॑ण॒ इति॑ सं - वर॑णे । भ॒व॒ । नः॒ ॥ स॒प॒त्न॒हेति॑ सपत्न-हा । नः॒ । अ॒भि॒मा॒ति॒जिदित्य॑भिमाति - जित् । च॒ । स्वे । गये᳚ । जा॒गृ॒हि॒ । अप्र॑युच्छ॒न्नित्यप्र॑ - यु॒च्छ॒न्न् ॥ इ॒ह । ए॒व । अ॒ग्ने॒ । अधीति॑ । धा॒र॒य॒ । र॒यिम् । मा । त्वा॒ । नीति॑ । क्र॒न्न् । पू॒र्व॒चित॒ इति॑ पूर्व - चितः॑ । नि॒का॒रिण॒ इति॑ नि - का॒रिणः॑ ॥ क्ष॒त्रम् । अ॒ग्ने॒ । सु॒यम॒मिति॑ सु - यम᳚म् । अ॒स्तु॒ । तुभ्य᳚म् । उ॒प॒स॒त्तेत्यु॑प-स॒त्ता । व॒द्‌र्ध॒ता॒म् । ते॒ । अनि॑ष्टृतः ॥ क्ष॒त्रेण॑ । अ॒ग्ने॒ । स्वायु॒रिति॑ सु - आयुः॑ । समिति॑ । र॒भ॒स्व॒ । मि॒त्रेण॑ । अ॒ग्ने॒ । मि॒त्र॒धेय॒ इति॑ मित्र - धेये᳚ । य॒त॒स्व॒ ॥ स॒जा॒ताना॒मिति॑ स - जा॒ताना᳚म् । म॒द्ध्य॒म॒स्था इति॑ मद्ध्यम-स्थाः । ए॒धि॒ । राज्ञा᳚म् । अ॒ग्ने॒ । वि॒ह॒व्य॑ इति॑ वि - ह॒व्यः॑ । दी॒दि॒हि॒ । इ॒ह ॥ अतीति॑ ।  \newline


\textbf{Krama Paata} \newline

स॒म्ॅवर॑णे भव । स॒म्ॅवर॑ण॒ इति॑ सं - वर॑णे । भ॒वा॒ नः॒ । न॒ इति॑ नः ॥ स॒प॒त्न॒हा नः॑ । स॒प॒त्न॒हेति॑ सपत्न - हा । नो॒ अ॒भि॒मा॒ति॒जित् । अ॒भि॒मा॒ति॒जिच्च॑ । अ॒भि॒मा॒ति॒,जिदित्य॑भिमाति - जित् । च॒ स्वे । स्वे गये᳚ । गये॑ जागृहि । जा॒गृ॒ह्यप्र॑युच्छन्न् । अप्र॑युच्छ॒न्नित्यप्र॑ - यु॒च्छ॒न्न्॒ ॥ इ॒हैव । ए॒वाग्ने᳚ । अ॒ग्ने॒ अधि॑ । अधि॑ धारय । धा॒र॒या॒ र॒यिम् । र॒यिम् मा । मा त्वा᳚ । त्वा॒ नि । नि क्रन्न्॑ । क्र॒न् पू॒र्व॒चितः॑ । पू॒र्व॒चितो॑ निका॒रिणः॑ । पू॒र्व॒चित॒ इति॑ पूर्व - चितः॑ । नि॒का॒रिण॒ इति॑ नि - का॒रिणः॑ ॥ क्ष॒त्रम॑ग्ने । अ॒ग्ने॒ सु॒यम᳚म् । सु॒यम॑मस्तु । सु॒यम॒मिति॑ सु - यम᳚म् । अ॒स्तु॒ तुभ्य᳚म् । तुभ्य॑मुपस॒त्ता । उ॒प॒स॒त्ता व॑र्द्धताम् । उ॒प॒स॒त्तेत्यु॑प - स॒त्ता । व॒र्द्ध॒ता॒म् ते॒ । ते॒ अनि॑ष्टृतः । अनि॑ष्टृत॒ इत्यनि॑ष्टृतः ॥ क्ष॒त्रेणा᳚ग्ने । अ॒ग्ने॒ स्वायुः॑ । स्वायुः॒ सम् । स्वायु॒रिति॑ सु - आयुः॑ । सꣳ र॑भस्व । र॒भ॒स्व॒ मि॒त्रेण॑ । मि॒त्रेणा᳚ग्ने । अ॒ग्ने॒ मि॒त्र॒धेये᳚ । मि॒त्र॒धेये॑ यतस्व । मि॒त्र॒धेय॒ इति॑ मित्र - धेये᳚ । य॒त॒स्वेति॑ यतस्व ॥ स॒जा॒ताना᳚म् मद्ध्यम॒स्थाः । स॒जा॒ताना॒मिति॑ स - जा॒ताना᳚म् । म॒द्ध्य॒म॒स्था ए॑धि । म॒द्ध्य॒म॒स्था इति॑ मद्ध्यम - स्थाः । ए॒धि॒ राज्ञा᳚म् । राज्ञा॑मग्ने । अ॒ग्ने॒ वि॒ह॒व्यः॑ । वि॒ह॒व्यो॑ दीदिहि । वि॒ह॒व्य॑ इति॑ वि - ह॒व्यः॑ । दी॒दि॒ही॒ह । इ॒हेती॒ह ॥ अति॒ निहः॑ \newline

\textbf{Jatai Paata} \newline

1. सं॒ॅवर॑णे भव भव सं॒ॅवर॑णे सं॒ॅवर॑णे भव । \newline
2. सं॒ॅवर॑ण॒ इति॑ सं - वर॑णे । \newline
3. भ॒वा॒ नो॒ नो॒ भ॒व॒ भ॒वा॒ नः॒ । \newline
4. न॒ इति॑ नः । \newline
5. स॒प॒त्न॒हा नो॑ नः सपत्न॒हा स॑पत्न॒हा नः॑ । \newline
6. स॒प॒त्न॒हेति॑ सपत्न - हा । \newline
7. नो॒ अ॒भि॒मा॒ति॒जि द॑भिमाति॒जिन् नो॑ नो अभिमाति॒जित् । \newline
8. अ॒भि॒मा॒ति॒जिच् च॑ चाभिमाति॒जि द॑भिमाति॒जिच् च॑ । \newline
9. अ॒भि॒मा॒ति॒जिदित्य॑भिमाति - जित् । \newline
10. च॒ स्वे स्वे च॑ च॒ स्वे । \newline
11. स्वे गये॒ गये॒ स्वे स्वे गये᳚ । \newline
12. गये॑ जागृहि जागृहि॒ गये॒ गये॑ जागृहि । \newline
13. जा॒गृ॒ ह्यप्र॑युच्छ॒न् नप्र॑युच्छन् जागृहि जागृ॒ ह्यप्र॑युच्छन्न् । \newline
14. अप्र॑युच्छ॒न्नित्यप्र॑ - यु॒च्छ॒न्न् । \newline
15. इ॒हैवैवेहे हैव । \newline
16. ए॒वाग्ने॑ अग्न ए॒वैवाग्ने᳚ । \newline
17. अ॒ग्ने॒ अध्यध्य॑ग्ने अग्ने॒ अधि॑ । \newline
18. अधि॑ धारय धार॒या ध्यधि॑ धारय । \newline
19. धा॒र॒या॒ र॒यिꣳ र॒यिम् धा॑रय धारया र॒यिम् । \newline
20. र॒यिम् मा मा र॒यिꣳ र॒यिम् मा । \newline
21. मा त्वा᳚ त्वा॒ मा मा त्वा᳚ । \newline
22. त्वा॒ नि नि त्वा᳚ त्वा॒ नि । \newline
23. नि क्र॑न् क्र॒न् नि नि क्रन्न्॑ । \newline
24. क्र॒न् पू॒र्व॒चितः॑ पूर्व॒चितः॑ क्रन् क्रन् पूर्व॒चितः॑ । \newline
25. पू॒र्व॒चितो॑ निका॒रिणो॑ निका॒रिणः॑ पूर्व॒चितः॑ पूर्व॒चितो॑ निका॒रिणः॑ । \newline
26. पू॒र्व॒चित॒ इति॑ पूर्व - चितः॑ । \newline
27. नि॒का॒रिण॒ इति॑ नि - का॒रिणः॑ । \newline
28. क्ष॒त्र म॑ग्ने अग्ने क्ष॒त्रम् क्ष॒त्र म॑ग्ने । \newline
29. अ॒ग्ने॒ सु॒यमꣳ॑ सु॒यम॑ मग्ने अग्ने सु॒यम᳚म् । \newline
30. सु॒यम॑ मस्त्वस्तु सु॒यमꣳ॑ सु॒यम॑ मस्तु । \newline
31. सु॒यम॒मिति॑ सु - यम᳚म् । \newline
32. अ॒स्तु॒ तुभ्य॒म् तुभ्य॑ मस्त्वस्तु॒ तुभ्य᳚म् । \newline
33. तुभ्य॑ मुपस॒त्तो प॑स॒त्ता तुभ्य॒म् तुभ्य॑ मुपस॒त्ता । \newline
34. उ॒प॒स॒त्ता व॑र्द्धतां ॅवर्द्धता मुपस॒त्तो प॑स॒त्ता व॑र्द्धताम् । \newline
35. उ॒प॒स॒त्तेत्यु॑प - स॒त्ता । \newline
36. व॒र्द्ध॒ता॒म् ते॒ ते॒ व॒र्द्ध॒तां॒ ॅव॒र्द्ध॒ता॒म् ते॒ । \newline
37. ते॒ अनि॑ष्टृतो॒ अनि॑ष्टृत स्ते ते॒ अनि॑ष्टृतः । \newline
38. अनि॑ष्टृत॒ इत्यनि॑ष्टृतः । \newline
39. क्ष॒त्रेणा᳚ग्ने अग्ने क्ष॒त्रेण॑ क्ष॒त्रेणा᳚ग्ने । \newline
40. अ॒ग्ने॒ स्वायुः॒ स्वायु॑ रग्ने अग्ने॒ स्वायुः॑ । \newline
41. स्वायुः॒ सꣳ सꣳ स्वायुः॒ स्वायुः॒ सम् । \newline
42. स्वायु॒रिति॑ सु - आयुः॑ । \newline
43. सꣳ र॑भस्व रभस्व॒ सꣳ सꣳ र॑भस्व । \newline
44. र॒भ॒स्व॒ मि॒त्रेण॑ मि॒त्रेण॑ रभस्व रभस्व मि॒त्रेण॑ । \newline
45. मि॒त्रेणा᳚ग्ने अग्ने मि॒त्रेण॑ मि॒त्रेणा᳚ग्ने । \newline
46. अ॒ग्ने॒ मि॒त्र॒धेये॑ मित्र॒धेये॑ अग्ने अग्ने मित्र॒धेये᳚ । \newline
47. मि॒त्र॒धेये॑ यतस्व यतस्व मित्र॒धेये॑ मित्र॒धेये॑ यतस्व । \newline
48. मि॒त्र॒धेय॒ इति॑ मित्र - धेये᳚ । \newline
49. य॒त॒स्वेति॑ यतस्व । \newline
50. स॒जा॒ताना᳚म् मद्ध्यम॒स्था म॑द्ध्यम॒स्थाः स॑जा॒तानाꣳ॑ सजा॒ताना᳚म् मद्ध्यम॒स्थाः । \newline
51. स॒जा॒ताना॒मिति॑ स - जा॒ताना᳚म् । \newline
52. म॒द्ध्य॒म॒स्था ए᳚ध्येधि मद्ध्यम॒स्था म॑द्ध्यम॒स्था ए॑धि । \newline
53. म॒द्ध्य॒म॒स्था इति॑ मद्ध्यम - स्थाः । \newline
54. ए॒धि॒ राज्ञाꣳ॒॒ राज्ञा॑ मेध्येधि॒ राज्ञा᳚म् । \newline
55. राज्ञा॑ मग्ने अग्ने॒ राज्ञाꣳ॒॒ राज्ञा॑ मग्ने । \newline
56. अ॒ग्ने॒ वि॒ह॒व्यो॑ विह॒व्यो॑ अग्ने अग्ने विह॒व्यः॑ । \newline
57. वि॒ह॒व्यो॑ दीदिहि दीदिहि विह॒व्यो॑ विह॒व्यो॑ दीदिहि । \newline
58. वि॒ह॒व्य॑ इति॑ वि - ह॒व्यः॑ । \newline
59. दी॒दि॒ही॒हे ह दी॑दिहि दीदिही॒ह । \newline
60. इ॒हेती॒ह । \newline
61. अति॒ निहो॒ निहो॒ अत्यति॒ निहः॑ । \newline

\textbf{Ghana Paata } \newline

1. सं॒ॅवर॑णे भव भव सं॒ॅवर॑णे सं॒ॅवर॑णे भवा नो नो भव सं॒ॅवर॑णे सं॒ॅवर॑णे भवा नः । \newline
2. सं॒ॅवर॑ण॒ इति॑ सं - वर॑णे । \newline
3. भ॒वा॒ नो॒ नो॒ भ॒व॒ भ॒वा॒ नः॒ । \newline
4. न॒ इति॑ नः । \newline
5. स॒प॒त्न॒हा नो॑ नः सपत्न॒हा स॑पत्न॒हा नो॑ अभिमाति॒जि द॑भिमाति॒जिन् नः॑ सपत्न॒हा स॑पत्न॒हा नो॑ अभिमाति॒जित् । \newline
6. स॒प॒त्न॒हेति॑ सपत्न - हा । \newline
7. नो॒ अ॒भि॒मा॒ति॒जि द॑भिमाति॒जिन् नो॑ नो अभिमाति॒जिच् च॑ चाभिमाति॒जिन् नो॑ नो अभिमाति॒जिच् च॑ । \newline
8. अ॒भि॒मा॒ति॒जिच् च॑ चाभिमाति॒जि द॑भिमाति॒जिच् च॒ स्वे स्वे चा॑भिमाति॒जि द॑भिमाति॒जिच् च॒ स्वे । \newline
9. अ॒भि॒मा॒ति॒जिदित्य॑भिमाति - जित् । \newline
10. च॒ स्वे स्वे च॑ च॒ स्वे गये॒ गये॒ स्वे च॑ च॒ स्वे गये᳚ । \newline
11. स्वे गये॒ गये॒ स्वे स्वे गये॑ जागृहि जागृहि॒ गये॒ स्वे स्वे गये॑ जागृहि । \newline
12. गये॑ जागृहि जागृहि॒ गये॒ गये॑ जागृ॒ ह्यप्र॑युच्छ॒न् नप्र॑युच्छन् जागृहि॒ गये॒ गये॑ जागृ॒ ह्यप्र॑युच्छन्न् । \newline
13. जा॒गृ॒ ह्यप्र॑युच्छ॒न् नप्र॑युच्छन् जागृहि जागृ॒ ह्यप्र॑युच्छन्न् । \newline
14. अप्र॑युच्छ॒न्नित्यप्र॑ - यु॒च्छ॒न्न् । \newline
15. इ॒है वैवेहे हैवाग्ने॑ अग्न ए॒वेहे हैवाग्ने᳚ । \newline
16. ए॒वाग्ने॑ अग्न ए॒वै वाग्ने॒ अध्यध्य॑ग्न ए॒वैवाग्ने॒ अधि॑ । \newline
17. अ॒ग्ने॒ अध्यध्य॑ग्ने अग्ने॒ अधि॑ धारय धार॒या ध्य॑ग्ने अग्ने॒ अधि॑ धारय । \newline
18. अधि॑ धारय धार॒या ध्यधि॑ धारया र॒यिꣳ र॒यिम् धा॑र॒या ध्यधि॑ धारया र॒यिम् । \newline
19. धा॒र॒या॒ र॒यिꣳ र॒यिम् धा॑रय धारया र॒यिम् मा मा र॒यिम् धा॑रय धारया र॒यिम् मा । \newline
20. र॒यिम् मा मा र॒यिꣳ र॒यिम् मा त्वा᳚ त्वा॒ मा र॒यिꣳ र॒यिम् मा त्वा᳚ । \newline
21. मा त्वा᳚ त्वा॒ मा मा त्वा॒ नि नि त्वा॒ मा मा त्वा॒ नि । \newline
22. त्वा॒ नि नि त्वा᳚ त्वा॒ नि क्र॑न् क्र॒न् नि त्वा᳚ त्वा॒ नि क्रन्न्॑ । \newline
23. नि क्र॑न् क्र॒न् नि नि क्र॑न् पूर्व॒चितः॑ पूर्व॒चितः॑ क्र॒न् नि नि क्र॑न् पूर्व॒चितः॑ । \newline
24. क्र॒न् पू॒र्व॒चितः॑ पूर्व॒चितः॑ क्रन् क्रन् पूर्व॒चितो॑ निका॒रिणो॑ निका॒रिणः॑ पूर्व॒चितः॑ क्रन् क्रन् पूर्व॒चितो॑ निका॒रिणः॑ । \newline
25. पू॒र्व॒चितो॑ निका॒रिणो॑ निका॒रिणः॑ पूर्व॒चितः॑ पूर्व॒चितो॑ निका॒रिणः॑ । \newline
26. पू॒र्व॒चित॒ इति॑ पूर्व - चितः॑ । \newline
27. नि॒का॒रिण॒ इति॑ नि - का॒रिणः॑ । \newline
28. क्ष॒त्र म॑ग्ने अग्ने क्ष॒त्रम् क्ष॒त्र म॑ग्ने सु॒यमꣳ॑ सु॒यम॑ मग्ने क्ष॒त्रम् क्ष॒त्र म॑ग्ने सु॒यम᳚म् । \newline
29. अ॒ग्ने॒ सु॒यमꣳ॑ सु॒यम॑ मग्ने अग्ने सु॒यम॑ मस्त्वस्तु सु॒यम॑ मग्ने अग्ने सु॒यम॑ मस्तु । \newline
30. सु॒यम॑ मस्त्वस्तु सु॒यमꣳ॑ सु॒यम॑ मस्तु॒ तुभ्य॒म् तुभ्य॑ मस्तु सु॒यमꣳ॑ सु॒यम॑ मस्तु॒ तुभ्य᳚म् । \newline
31. सु॒यम॒मिति॑ सु - यम᳚म् । \newline
32. अ॒स्तु॒ तुभ्य॒म् तुभ्य॑ मस्त्वस्तु॒ तुभ्य॑ मुपस॒त्तो प॑स॒त्ता तुभ्य॑ मस्त्वस्तु॒ तुभ्य॑ मुपस॒त्ता । \newline
33. तुभ्य॑ मुपस॒त्तो प॑स॒त्ता तुभ्य॒म् तुभ्य॑ मुपस॒त्ता व॑र्द्धतां ॅवर्द्धता मुपस॒त्ता तुभ्य॒म् तुभ्य॑ मुपस॒त्ता व॑र्द्धताम् । \newline
34. उ॒प॒स॒त्ता व॑र्द्धतां ॅवर्द्धता मुपस॒त्तो प॑स॒त्ता व॑र्द्धताम् ते ते वर्द्धता मुपस॒त्तो प॑स॒त्ता व॑र्द्धताम् ते । \newline
35. उ॒प॒स॒त्तेत्यु॑प - स॒त्ता । \newline
36. व॒र्द्ध॒ता॒म् ते॒ ते॒ व॒र्द्ध॒तां॒ ॅव॒र्द्ध॒ता॒म् ते॒ अनि॑ष्टृतो॒ अनि॑ष्टृत स्ते वर्द्धतां ॅवर्द्धताम् ते॒ अनि॑ष्टृतः । \newline
37. ते॒ अनि॑ष्टृतो॒ अनि॑ष्टृत स्ते ते॒ अनि॑ष्टृतः । \newline
38. अनि॑ष्टृत॒ इत्यनि॑ष्टृतः । \newline
39. क्ष॒त्रेणा᳚ग्ने अग्ने क्ष॒त्रेण॑ क्ष॒त्रेणा᳚ग्ने॒ स्वायुः॒ स्वायु॑ रग्ने क्ष॒त्रेण॑ क्ष॒त्रेणा᳚ग्ने॒ स्वायुः॑ । \newline
40. अ॒ग्ने॒ स्वायुः॒ स्वायु॑रग्ने अग्ने॒ स्वायुः॒ सꣳ सꣳ स्वायु॑ रग्ने अग्ने॒ स्वायुः॒ सम् । \newline
41. स्वायुः॒ सꣳ सꣳ स्वायुः॒ स्वायुः॒ सꣳ र॑भस्व रभस्व॒ सꣳ स्वायुः॒ स्वायुः॒ सꣳ र॑भस्व । \newline
42. स्वायु॒रिति॑ सु - आयुः॑ । \newline
43. सꣳ र॑भस्व रभस्व॒ सꣳ सꣳ र॑भस्व मि॒त्रेण॑ मि॒त्रेण॑ रभस्व॒ सꣳ सꣳ र॑भस्व मि॒त्रेण॑ । \newline
44. र॒भ॒स्व॒ मि॒त्रेण॑ मि॒त्रेण॑ रभस्व रभस्व मि॒त्रेणा᳚ग्ने अग्ने मि॒त्रेण॑ रभस्व रभस्व मि॒त्रेणा᳚ग्ने । \newline
45. मि॒त्रेणा᳚ग्ने अग्ने मि॒त्रेण॑ मि॒त्रेणा᳚ग्ने मित्र॒धेये॑ मित्र॒धेये॑ अग्ने मि॒त्रेण॑ मि॒त्रेणा᳚ग्ने मित्र॒धेये᳚ । \newline
46. अ॒ग्ने॒ मि॒त्र॒धेये॑ मित्र॒धेये॑ अग्ने अग्ने मित्र॒धेये॑ यतस्व यतस्व मित्र॒धेये॑ अग्ने अग्ने मित्र॒धेये॑ यतस्व । \newline
47. मि॒त्र॒धेये॑ यतस्व यतस्व मित्र॒धेये॑ मित्र॒धेये॑ यतस्व । \newline
48. मि॒त्र॒धेय॒ इति॑ मित्र - धेये᳚ । \newline
49. य॒त॒स्वेति॑ यतस्व । \newline
50. स॒जा॒ताना᳚म् मद्ध्यम॒स्था म॑द्ध्यम॒स्थाः स॑जा॒तानाꣳ॑ सजा॒ताना᳚म् मद्ध्यम॒स्था 
ए᳚ध्येधि मद्ध्यम॒स्थाः स॑जा॒तानाꣳ॑ सजा॒ताना᳚म् मद्ध्यम॒स्था ए॑धि । \newline
51. स॒जा॒ताना॒मिति॑ स - जा॒ताना᳚म् । \newline
52. म॒द्ध्य॒म॒स्था ए᳚ध्येधि मद्ध्यम॒स्था म॑द्ध्यम॒स्था ए॑धि॒ राज्ञाꣳ॒॒ राज्ञा॑ मेधि मद्ध्यम॒स्था म॑द्ध्यम॒स्था ए॑धि॒ राज्ञा᳚म् । \newline
53. म॒द्ध्य॒म॒स्था इति॑ मद्ध्यम - स्थाः । \newline
54. ए॒धि॒ राज्ञाꣳ॒॒ राज्ञा॑ मेध्येधि॒ राज्ञा॑ मग्ने अग्ने॒ राज्ञा॑ मेध्येधि॒ राज्ञा॑ मग्ने । \newline
55. राज्ञा॑ मग्ने अग्ने॒ राज्ञाꣳ॒॒ राज्ञा॑ मग्ने विह॒व्यो॑ विह॒व्यो॑ अग्ने॒ राज्ञाꣳ॒॒ राज्ञा॑ मग्ने विह॒व्यः॑ । \newline
56. अ॒ग्ने॒ वि॒ह॒व्यो॑ विह॒व्यो॑ अग्ने अग्ने विह॒व्यो॑ दीदिहि दीदिहि विह॒व्यो॑ अग्ने अग्ने विह॒व्यो॑ दीदिहि । \newline
57. वि॒ह॒व्यो॑ दीदिहि दीदिहि विह॒व्यो॑ विह॒व्यो॑ दीदिही॒हेह दी॑दिहि विह॒व्यो॑ विह॒व्यो॑ दीदिही॒ह । \newline
58. वि॒ह॒व्य॑ इति॑ वि - ह॒व्यः॑ । \newline
59. दी॒दि॒ही॒ हेह दी॑दिहि दीदिही॒ह । \newline
60. इ॒हेती॒ह । \newline
61. अति॒ निहो॒ निहो॒ अत्यति॒ निहो॒ अत्यति॒ निहो॒ अत्यति॒ निहो॒ अति॑ । \newline
\pagebreak
\markright{ TS 4.1.7.3  \hfill https://www.vedavms.in \hfill}

\section{ TS 4.1.7.3 }

\textbf{TS 4.1.7.3 } \newline
\textbf{Samhita Paata} \newline

निहो॒ अति॒ स्रिधो ऽत्यचि॑त्ति॒-मत्यरा॑तिमग्ने । विश्वा॒ ह्य॑ग्ने दुरि॒ता सह॒स्वाथा॒स्मभ्यꣳ॑ स॒हवी॑राꣳ र॒यिन्दाः᳚ ॥ अ॒ना॒धृ॒ष्यो जा॒तव॑दा॒ अनि॑ष्टृतो वि॒राड॑ग्ने क्षत्र॒भृद्-दी॑दिही॒ह । विश्वा॒ आशाः᳚ प्रमु॒ञ्चन् मानु॑षीर्भि॒यः शि॒वाभि॑र॒द्य परि॑ पाहि नो वृ॒धे ॥ बृह॑स्पते सवितर्बो॒धयै॑नꣳ॒॒ सꣳशि॑तं चिथ्सं त॒राꣳ सꣳ शि॑शाधि । व॒र्द्धयै॑नं मह॒ते सौभ॑गाय॒- [  ] \newline

\textbf{Pada Paata} \newline

निहः॑ । अतीति॑ । स्रिधः॑ । अतीति॑ । अचि॑त्तिम् । अतीति॑ । अरा॑तिम् । अ॒ग्ने॒ ॥ विश्वा᳚ । हि । अ॒ग्ने॒ । दु॒रि॒तेति॑ दुः - इ॒ता । सह॑स्व । अथ॑ । अ॒स्मभ्य॒मित्य॒स्म - भ्य॒म् । स॒हवी॑रा॒मिति॑ स॒ह - वी॒रा॒म् । र॒यिम् । दाः॒ ॥ अ॒ना॒धृ॒ष्य इत्य॑ना - धृ॒ष्यः । जा॒तवे॑दा॒ इति॑ जा॒त - वे॒दाः॒ । अनि॑ष्टृतः । वि॒राडिति॑ वि - राट् । अ॒ग्ने॒ । क्ष॒त्र॒भृदिति॑ क्षत्र - भृत् । दी॒दि॒हि॒ । इ॒ह ॥ विश्वाः᳚ । आशाः᳚ । प्र॒मु॒ञ्चन्निति॑ प्र - मु॒ञ्चन्न् । मानु॑षीः । भि॒यः । शि॒वाभिः॑ । अ॒द्य । परीति॑ । पा॒हि॒ । नः॒ । वृ॒धे ॥ बृह॑स्पते । स॒वि॒तः॒ । बो॒धय॑ । ए॒न॒म् । सꣳशि॑त॒मिति॒ सं - शि॒त॒म् । चि॒त् । स॒तं॒रामिति॑ सं - त॒राम् । समिति॑ । शि॒शा॒धि॒ ॥ व॒द्‌र्धय॑ । ए॒न॒म् । म॒ह॒ते । सौभ॑गाय ।  \newline


\textbf{Krama Paata} \newline

निहो॒ अति॑ । अति॒ स्रिधः॑ । स्रिधोऽति॑ । अत्यचि॑त्तिम् । अचि॑त्ति॒मति॑ । अत्यरा॑तिम् । अरा॑तिमग्ने । अ॒ग्न॒ इत्य॑ग्ने ॥ विश्वा॒ हि । ह्य॑ग्ने । अ॒ग्ने॒ दु॒रि॒ता । दु॒रि॒ता सह॑स्व । दु॒रि॒तेति॑ दुः - इ॒ता । सह॒स्वाथ॑ । अथा॒स्मभ्य᳚म् । अ॒स्मभ्यꣳ॑ स॒हवी॑राम् । अ॒स्मभ्य॒मित्य॒स्म - भ्य॒म् । स॒हवी॑राꣳ र॒यिम् । स॒हवी॑रा॒मिति॑ स॒ह - वी॒रा॒म् । र॒यिम् दाः᳚ । दा॒ इति॑ दाः ॥ अ॒ना॒धृ॒ष्यो जा॒तवे॑दाः । अ॒ना॒धृ॒ष्य इत्य॑ना - धृ॒ष्यः । जा॒तवे॑दा॒ अनि॑ष्टृतः । जा॒तवे॑दा॒ इति॑ जा॒त - वे॒दाः॒ । अनि॑ष्टृतो वि॒राट् । वि॒राड॑ग्ने । वि॒राडिति॑ वि - राट् । अ॒ग्ने॒ क्ष॒त्र॒भृत् । क्ष॒त्र॒भृद् दी॑दिहि । क्ष॒त्र॒भृदिति॑ क्षत्र - भृत् । दी॒दि॒ही॒ह । इ॒हेती॒ह ॥ विश्वा॒ आशाः᳚ । आशाः᳚ प्रमु॒ञ्चन्न् । प्र॒मु॒ञ्चन् मानु॑षीः । प्र॒मु॒ञ्चन्निति॑ प्र - मु॒ञ्चन्न् । मानु॑षीर् भि॒यः । भि॒यः शि॒वाभिः॑ । शि॒वाभि॑र॒द्य । अ॒द्य परि॑ । परि॑ पाहि । पा॒हि॒ नः॒ । नो॒ वृ॒धे । वृ॒ध इति॑ वृ॒धे ॥ बृह॑स्पते सवितः । स॒वि॒त॒र् बो॒धय॑ । बो॒धयै॑नम् । ए॒नꣳ॒॒ सꣳशि॑तम् । सꣳशि॑तम् चित् । सꣳशि॑त॒मिति॒ सम् - शि॒त॒म् । चि॒थ् स॒न्त॒राम् । स॒न्त॒राꣳ सम् । स॒न्त॒रामिति॑ सम् - त॒राम् । सꣳ शि॑शाधि । शि॒शा॒धीति॑ शिशाधि ॥ व॒र्द्धयै॑नम् । ए॒न॒म् म॒ह॒ते । म॒ह॒ते सौभ॑गाय ( ) । सौभ॑गाय॒ विश्वे᳚ \newline

\textbf{Jatai Paata} \newline

1. निहो॒ अत्यति॒ निहो॒ निहो॒ अति॑ । \newline
2. अति॒ स्रिधः॒ स्रिधो ऽत्यति॒ स्रिधः॑ । \newline
3. स्रिधो ऽत्यति॒ स्रिधः॒ स्रिधो ऽति॑ । \newline
4. अत्यचि॑त्ति॒ मचि॑त्ति॒ मत्य त्यचि॑त्तिम् । \newline
5. अचि॑त्ति॒ मत्य त्यचि॑त्ति॒ मचि॑त्ति॒ मति॑ । \newline
6. अत्यरा॑ति॒ मरा॑ति॒ मत्य त्यरा॑तिम् । \newline
7. अरा॑ति मग्ने अग्ने॒ अरा॑ति॒ मरा॑ति मग्ने । \newline
8. अ॒ग्न॒ इत्य॑ग्ने । \newline
9. विश्वा॒ हि हि विश्वा॒ विश्वा॒ हि । \newline
10. ह्य॑ग्ने अग्ने॒ हि ह्य॑ग्ने । \newline
11. अ॒ग्ने॒ दु॒रि॒ता दु॑रि॒ता ऽग्ने॑ अग्ने दुरि॒ता । \newline
12. दु॒रि॒ता सह॑स्व॒ सह॑स्व दुरि॒ता दु॑रि॒ता सह॑स्व । \newline
13. दु॒रि॒तेति॑ दुः - इ॒ता । \newline
14. सह॒स्वा थाथ॒ सह॑स्व॒ सह॒स्वाथ॑ । \newline
15. अथा॒ स्मभ्य॑ म॒स्मभ्य॒ मथाथा॒ स्मभ्य᳚म् । \newline
16. अ॒स्मभ्यꣳ॑ स॒हवी॑राꣳ स॒हवी॑रा म॒स्मभ्य॑ म॒स्मभ्यꣳ॑ स॒हवी॑राम् । \newline
17. अ॒स्मभ्य॒मित्य॒स्म - भ्य॒म् । \newline
18. स॒हवी॑राꣳ र॒यिꣳ र॒यिꣳ स॒हवी॑राꣳ स॒हवी॑राꣳ र॒यिम् । \newline
19. स॒हवी॑रा॒मिति॑ स॒ह - वी॒रा॒म् । \newline
20. र॒यिम् दा॑ दा र॒यिꣳ र॒यिम् दाः᳚ । \newline
21. दा॒ इति॑ दाः । \newline
22. अ॒ना॒धृ॒ष्यो जा॒तवे॑दा जा॒तवे॑दा अनाधृ॒ष्यो॑ ऽनाधृ॒ष्यो जा॒तवे॑दाः । \newline
23. अ॒ना॒धृ॒ष्य इत्य॑ना - धृ॒ष्यः । \newline
24. जा॒तवे॑दा॒ अनि॑ष्टृतो॒ अनि॑ष्टृतो जा॒तवे॑दा जा॒तवे॑दा॒ अनि॑ष्टृतः । \newline
25. जा॒तवे॑दा॒ इति॑ जा॒त - वे॒दाः॒ । \newline
26. अनि॑ष्टृतो वि॒राड् वि॒रा डनि॑ष्टृतो॒ अनि॑ष्टृतो वि॒राट् । \newline
27. वि॒रा ड॑ग्ने अग्ने वि॒राड् वि॒रा ड॑ग्ने । \newline
28. वि॒राडिति॑ वि - राट् । \newline
29. अ॒ग्ने॒ क्ष॒त्र॒भृत् क्ष॑त्र॒भृ द॑ग्ने अग्ने क्षत्र॒भृत् । \newline
30. क्ष॒त्र॒भृद् दी॑दिहि दीदिहि क्षत्र॒भृत् क्ष॑त्र॒भृद् दी॑दिहि । \newline
31. क्ष॒त्र॒भृदिति॑ क्षत्र - भृत् । \newline
32. दी॒दि॒ही॒हेह दी॑दिहि दीदिही॒ह । \newline
33. इ॒हेती॒ह । \newline
34. विश्वा॒ आशा॒ आशा॒ विश्वा॒ विश्वा॒ आशाः᳚ । \newline
35. आशाः᳚ प्रमु॒ञ्चन् प्र॑मु॒ञ्चन् नाशा॒ आशाः᳚ प्रमु॒ञ्चन्न् । \newline
36. प्र॒मु॒ञ्चन् मानु॑षी॒र् मानु॑षीः प्रमु॒ञ्चन् प्र॑मु॒ञ्चन् मानु॑षीः । \newline
37. प्र॒मु॒ञ्चन्निति॑ प्र - मु॒ञ्चन्न् । \newline
38. मानु॑षीर् भि॒यो भि॒यो मानु॑षी॒र् मानु॑षीर् भि॒यः । \newline
39. भि॒यः शि॒वाभिः॑ शि॒वाभि॑र् भि॒यो भि॒यः शि॒वाभिः॑ । \newline
40. शि॒वाभि॑ र॒द्याद्य शि॒वाभिः॑ शि॒वाभि॑ र॒द्य । \newline
41. अ॒द्य परि॒ पर्य॒द्याद्य परि॑ । \newline
42. परि॑ पाहि पाहि॒ परि॒ परि॑ पाहि । \newline
43. पा॒हि॒ नो॒ नः॒ पा॒हि॒ पा॒हि॒ नः॒ । \newline
44. नो॒ वृ॒धे वृ॒धे नो॑ नो वृ॒धे । \newline
45. वृ॒ध इति॑ वृ॒धे । \newline
46. बृह॑स्पते सवितः सवित॒र् बृह॑स्पते॒ बृह॑स्पते सवितः । \newline
47. स॒वि॒त॒र् बो॒धय॑ बो॒धय॑ सवितः सवितर् बो॒धय॑ । \newline
48. बो॒ध यै॑न मेनम् बो॒धय॑ बो॒ध यै॑नम् । \newline
49. ए॒नꣳ॒॒ सꣳशि॑तꣳ॒॒ सꣳशि॑त मेन मेनꣳ॒॒ सꣳशि॑तम् । \newline
50. सꣳशि॑तम् चिच् चि॒थ् सꣳशि॑तꣳ॒॒ सꣳशि॑तम् चित् । \newline
51. सꣳशि॑त॒मिति॒ सं - शि॒त॒म् । \newline
52. चि॒थ् स॒न्त॒राꣳ स॑न्त॒राम् चि॑च् चिथ् सन्त॒राम् । \newline
53. स॒न्त॒राꣳ सꣳ सꣳ स॑न्त॒राꣳ स॑न्त॒राꣳ सम् । \newline
54. स॒न्त॒रामिति॑ सं - त॒राम् । \newline
55. सꣳ शि॑शाधि शिशाधि॒ सꣳ सꣳ शि॑शाधि । \newline
56. शि॒शा॒धीति॑ शिशाधि । \newline
57. व॒र्द्ध यै॑न मेनं ॅव॒र्द्धय॑ व॒र्द्ध यै॑नम् । \newline
58. ए॒न॒म् म॒ह॒ते म॑ह॒त ए॑न मेनम् मह॒ते । \newline
59. म॒ह॒ते सौभ॑गाय॒ सौभ॑गाय मह॒ते म॑ह॒ते सौभ॑गाय । \newline
60. सौभ॑गाय॒ विश्वे॒ विश्वे॒ सौभ॑गाय॒ सौभ॑गाय॒ विश्वे᳚ । \newline

\textbf{Ghana Paata } \newline

1. निहो॒ अत्यति॒ निहो॒ निहो॒ अति॒ स्रिधः॒ स्रिधो ऽति॒ निहो॒ निहो॒ अति॒ स्रिधः॑ । \newline
2. अति॒ स्रिधः॒ स्रिधो ऽत्यति॒ स्रिधो ऽत्यति॒ स्रिधो ऽत्यति॒ स्रिधो ऽति॑ । \newline
3. स्रिधो ऽत्यति॒ स्रिधः॒ स्रिधो ऽत्यचि॑त्ति॒ मचि॑त्ति॒ मति॒ स्रिधः॒ स्रिधो ऽत्यचि॑त्तिम् । \newline
4. अत्यचि॑त्ति॒ मचि॑त्ति॒ मत्य त्यचि॑त्ति॒ मत्य त्यचि॑त्ति॒ मत्य त्यचि॑त्ति॒ मति॑ । \newline
5. अचि॑त्ति॒ मत्य त्यचि॑त्ति॒ मचि॑त्ति॒ मत्यरा॑ति॒ मरा॑ति॒ मत्यचि॑त्ति॒ मचि॑त्ति॒ मत्यरा॑तिम् । \newline
6. अत्यरा॑ति॒ मरा॑ति॒ मत्य त्यरा॑ति मग्ने अग्ने॒ अरा॑ति॒ मत्यत्यरा॑ति मग्ने । \newline
7. अरा॑ति मग्ने अग्ने॒ अरा॑ति॒ मरा॑ति मग्ने । \newline
8. अ॒ग्न॒ इत्य॑ग्ने । \newline
9. विश्वा॒ हि हि विश्वा॒ विश्वा॒ ह्य॑ग्ने अग्ने॒ हि विश्वा॒ विश्वा॒ ह्य॑ग्ने । \newline
10. ह्य॑ग्ने अग्ने॒ हि ह्य॑ग्ने दुरि॒ता दु॑रि॒ता ऽग्ने॒ हि ह्य॑ग्ने दुरि॒ता । \newline
11. अ॒ग्ने॒ दु॒रि॒ता दु॑रि॒ता ऽग्ने॑ अग्ने दुरि॒ता सह॑स्व॒ सह॑स्व दुरि॒ता ऽग्ने॑ अग्ने दुरि॒ता सह॑स्व । \newline
12. दु॒रि॒ता सह॑स्व॒ सह॑स्व दुरि॒ता दु॑रि॒ता सह॒स्वा थाथ॒ सह॑स्व दुरि॒ता दु॑रि॒ता सह॒स्वाथ॑ । \newline
13. दु॒रि॒तेति॑ दुः - इ॒ता । \newline
14. सह॒स्वाथाथ॒ सह॑स्व॒ सह॒स्वाथा॒ स्मभ्य॑ म॒स्मभ्य॒ मथ॒ सह॑स्व॒ सह॒स्वाथा॒ स्मभ्य᳚म् । \newline
15. अथा॒स्मभ्य॑ म॒स्मभ्य॒ मथाथा॒ स्मभ्यꣳ॑ स॒हवी॑राꣳ स॒हवी॑रा म॒स्मभ्य॒ मथाथा॒ स्मभ्यꣳ॑ स॒हवी॑राम् । \newline
16. अ॒स्मभ्यꣳ॑ स॒हवी॑राꣳ स॒हवी॑रा म॒स्मभ्य॑ म॒स्मभ्यꣳ॑ स॒हवी॑राꣳ र॒यिꣳ र॒यिꣳ स॒हवी॑रा म॒स्मभ्य॑ म॒स्मभ्यꣳ॑ स॒हवी॑राꣳ र॒यिम् । \newline
17. अ॒स्मभ्य॒मित्य॒स्म - भ्य॒म् । \newline
18. स॒हवी॑राꣳ र॒यिꣳ र॒यिꣳ स॒हवी॑राꣳ स॒हवी॑राꣳ र॒यिम् दा॑ दा र॒यिꣳ स॒हवी॑राꣳ स॒हवी॑राꣳ र॒यिम् दाः᳚ । \newline
19. स॒हवी॑रा॒मिति॑ स॒ह - वी॒रा॒म् । \newline
20. र॒यिम् दा॑ दा र॒यिꣳ र॒यिम् दाः᳚ । \newline
21. दा॒ इति॑ दाः । \newline
22. अ॒ना॒धृ॒ष्यो जा॒तवे॑दा जा॒तवे॑दा अनाधृ॒ष्यो॑ ऽनाधृ॒ष्यो जा॒तवे॑दा॒ अनि॑ष्टृतो॒ अनि॑ष्टृतो जा॒तवे॑दा अनाधृ॒ष्यो॑ ऽनाधृ॒ष्यो जा॒तवे॑दा॒ अनि॑ष्टृतः । \newline
23. अ॒ना॒धृ॒ष्य इत्य॑ना - धृ॒ष्यः । \newline
24. जा॒तवे॑दा॒ अनि॑ष्टृतो॒ अनि॑ष्टृतो जा॒तवे॑दा जा॒तवे॑दा॒ अनि॑ष्टृतो वि॒राड् वि॒रा डनि॑ष्टृतो जा॒तवे॑दा जा॒तवे॑दा॒ अनि॑ष्टृतो वि॒राट् । \newline
25. जा॒तवे॑दा॒ इति॑ जा॒त - वे॒दाः॒ । \newline
26. अनि॑ष्टृतो वि॒राड् वि॒राड् अनि॑ष्टृतो॒ अनि॑ष्टृतो वि॒रा ड॑ग्ने अग्ने वि॒रा डनि॑ष्टृतो॒ अनि॑ष्टृतो वि॒रा ड॑ग्ने । \newline
27. वि॒रा ड॑ग्ने अग्ने वि॒राड् वि॒रा ड॑ग्ने क्षत्र॒भृत् क्ष॑त्र॒भृ द॑ग्ने वि॒राड् वि॒रा ड॑ग्ने क्षत्र॒भृत् । \newline
28. वि॒राडिति॑ वि - राट् । \newline
29. अ॒ग्ने॒ क्ष॒त्र॒भृत् क्ष॑त्र॒भृ द॑ग्ने अग्ने क्षत्र॒भृद् दी॑दिहि दीदिहि क्षत्र॒भृ द॑ग्ने अग्ने क्षत्र॒भृद् दी॑दिहि । \newline
30. क्ष॒त्र॒भृद् दी॑दिहि दीदिहि क्षत्र॒भृत् क्ष॑त्र॒भृद् दी॑दिही॒ हेह दी॑दिहि क्षत्र॒भृत् क्ष॑त्र॒भृद् दी॑दिही॒ह । \newline
31. क्ष॒त्र॒भृदिति॑ क्षत्र - भृत् । \newline
32. दी॒दि॒ही॒ हेह दी॑दिहि दीदिही॒ह । \newline
33. इ॒हेती॒ह । \newline
34. विश्वा॒ आशा॒ आशा॒ विश्वा॒ विश्वा॒ आशाः᳚ प्रमु॒ञ्चन् प्र॑मु॒ञ्चन् नाशा॒ विश्वा॒ विश्वा॒ आशाः᳚ प्रमु॒ञ्चन्न् । \newline
35. आशाः᳚ प्रमु॒ञ्चन् प्र॑मु॒ञ्चन् नाशा॒ आशाः᳚ प्रमु॒ञ्चन् मानु॑षी॒र् मानु॑षीः प्रमु॒ञ्चन् नाशा॒ आशाः᳚ प्रमु॒ञ्चन् मानु॑षीः । \newline
36. प्र॒मु॒ञ्चन् मानु॑षी॒र् मानु॑षीः प्रमु॒ञ्चन् प्र॑मु॒ञ्चन् मानु॑षीर् भि॒यो भि॒यो मानु॑षीः प्रमु॒ञ्चन् प्र॑मु॒ञ्चन् मानु॑षीर् भि॒यः । \newline
37. प्र॒मु॒ञ्चन्निति॑ प्र - मु॒ञ्चन्न् । \newline
38. मानु॑षीर् भि॒यो भि॒यो मानु॑षी॒र् मानु॑षीर् भि॒यः शि॒वाभिः॑ शि॒वाभि॑र् भि॒यो मानु॑षी॒र् मानु॑षीर् भि॒यः शि॒वाभिः॑ । \newline
39. भि॒यः शि॒वाभिः॑ शि॒वाभि॑र् भि॒यो भि॒यः शि॒वाभि॑ र॒द्याद्य शि॒वाभि॑र् भि॒यो भि॒यः शि॒वाभि॑ र॒द्य । \newline
40. शि॒वाभि॑ र॒द्याद्य शि॒वाभिः॑ शि॒वाभि॑ र॒द्य परि॒ पर्य॒द्य शि॒वाभिः॑ शि॒वाभि॑ र॒द्य परि॑ । \newline
41. अ॒द्य परि॒ पर्य॒द्याद्य परि॑ पाहि पाहि॒ पर्य॒द्याद्य परि॑ पाहि । \newline
42. परि॑ पाहि पाहि॒ परि॒ परि॑ पाहि नो नः पाहि॒ परि॒ परि॑ पाहि नः । \newline
43. पा॒हि॒ नो॒ नः॒ पा॒हि॒ पा॒हि॒ नो॒ वृ॒धे वृ॒धे नः॑ पाहि पाहि नो वृ॒धे । \newline
44. नो॒ वृ॒धे वृ॒धे नो॑ नो वृ॒धे । \newline
45. वृ॒ध इति॑ वृ॒धे । \newline
46. बृह॑स्पते सवितः सवित॒र् बृह॑स्पते॒ बृह॑स्पते सवितर् बो॒धय॑ बो॒धय॑ सवित॒र् बृह॑स्पते॒ बृह॑स्पते सवितर् बो॒धय॑ । \newline
47. स॒वि॒त॒र् बो॒धय॑ बो॒धय॑ सवितः सवितर् बो॒ध यै॑न मेनम् बो॒धय॑ सवितः सवितर् बो॒ध यै॑नम् । \newline
48. बो॒ध यै॑न मेनम् बो॒धय॑ बो॒ध यै॑नꣳ॒॒ सꣳशि॑तꣳ॒॒ सꣳशि॑त मेनम् बो॒धय॑ बो॒ध यै॑नꣳ॒॒ सꣳशि॑तम् । \newline
49. ए॒नꣳ॒॒ सꣳशि॑तꣳ॒॒ सꣳशि॑त मेन मेनꣳ॒॒ सꣳशि॑तम् चिच् चि॒थ् सꣳशि॑त मेन मेनꣳ॒॒ सꣳशि॑तम् चित् । \newline
50. सꣳशि॑तम् चिच् चि॒थ् सꣳशि॑तꣳ॒॒ सꣳशि॑तम् चिथ् सन्त॒राꣳ स॑न्त॒राम् चि॒थ् सꣳशि॑तꣳ॒॒ सꣳशि॑तम् चिथ् सन्त॒राम् । \newline
51. सꣳशि॑त॒मिति॒ सं - शि॒त॒म् । \newline
52. चि॒थ् स॒न्त॒राꣳ स॑न्त॒राम् चि॑च् चिथ् सन्त॒राꣳ सꣳ सꣳ स॑न्त॒राम् चि॑च् चिथ् सन्त॒राꣳ सम् । \newline
53. स॒न्त॒राꣳ सꣳ सꣳ स॑न्त॒राꣳ स॑न्त॒राꣳ सꣳ शि॑शाधि शिशाधि॒ सꣳ स॑न्त॒राꣳ स॑न्त॒राꣳ सꣳ शि॑शाधि । \newline
54. स॒न्त॒रामिति॑ सं - त॒राम् । \newline
55. सꣳ शि॑शाधि शिशाधि॒ सꣳ सꣳ शि॑शाधि । \newline
56. शि॒शा॒धीति॑ शिशाधि । \newline
57. व॒र्द्धयै॑न मेनं ॅव॒र्द्धय॑ व॒र्द्ध यै॑नम् मह॒ते म॑ह॒त ए॑नं ॅव॒र्द्धय॑ व॒र्द्ध यै॑नम् मह॒ते । \newline
58. ए॒न॒म् म॒ह॒ते म॑ह॒त ए॑न मेनम् मह॒ते सौभ॑गाय॒ सौभ॑गाय मह॒त ए॑न मेनम् मह॒ते सौभ॑गाय । \newline
59. म॒ह॒ते सौभ॑गाय॒ सौभ॑गाय मह॒ते म॑ह॒ते सौभ॑गाय॒ विश्वे॒ विश्वे॒ सौभ॑गाय मह॒ते म॑ह॒ते सौभ॑गाय॒ विश्वे᳚ । \newline
60. सौभ॑गाय॒ विश्वे॒ विश्वे॒ सौभ॑गाय॒ सौभ॑गाय॒ विश्व॑ एन मेनं॒ ॅविश्वे॒ सौभ॑गाय॒ सौभ॑गाय॒ विश्व॑ एनम् । \newline
\pagebreak
\markright{ TS 4.1.7.4  \hfill https://www.vedavms.in \hfill}

\section{ TS 4.1.7.4 }

\textbf{TS 4.1.7.4 } \newline
\textbf{Samhita Paata} \newline

विश्व॑ एन॒मनु॑ मदन्तु दे॒वाः ॥ अ॒मु॒त्र॒भूया॒दध॒ यद्य॒मस्य॒ बृह॑स्पते अ॒भिश॑स्ते॒र मु॑ञ्चः । प्रत्यौ॑हता-म॒श्विना॑ मृ॒त्युम॑स्माद् दे॒वाना॑-मग्ने भि॒षजा॒ शची॑भिः ॥ उद्व॒यं तम॑स॒स्परि॒ पश्य॑न्तो॒ ज्योति॒रुत्त॑रं । दे॒वं दे॑व॒त्रा सूर्य॒मग॑न्म॒ ज्योति॑रुत्त॒मं ॥ \newline

\textbf{Pada Paata} \newline

विश्वे᳚ । ए॒न॒म् । अन्विति॑ । म॒द॒न्तु॒ । दे॒वाः ॥ अ॒मु॒त्र॒भूया॒दित्य॑मुत्र - भूया᳚त् । अध॑ । यत् । य॒मस्य॑ । बृह॑स्पते । अ॒भिश॑स्ते॒रित्य॒भि - श॒स्तेः॒ । अमु॑ञ्चः ॥ प्रतीति॑ । औ॒ह॒ता॒म् । अ॒श्विना᳚ । मृ॒त्युम् । अ॒स्मा॒त् । दे॒वाना᳚म् । अ॒ग्ने॒ । भि॒षजा᳚ । शची॑भि॒रिति॒ शचि॑ - भिः॒ ॥ उदिति॑ । व॒यम् । तम॑सः । परीति॑ । पश्य॑न्तः । ज्योतिः॑ । उत्त॑र॒मित्युत् - त॒र॒म् ॥ दे॒वम् । दे॒व॒त्रेति॑ देव - त्रा । सूर्य᳚म् । अग॑न्म । ज्योतिः॑ । उ॒त्त॒ममित्यु॑त् - त॒मम् ॥  \newline


\textbf{Krama Paata} \newline

विश्व॑ एनम् । ए॒न॒मनु॑ । अनु॑ मदन्तु । म॒द॒न्तु॒ दे॒वाः । दे॒वा इति॑ दे॒वाः ॥ अ॒मु॒त्र॒भूया॒दध॑ । अ॒मु॒त्र॒भूया॒दित्य॑मुत्र - भूया᳚त् । अध॒ यत् । यद् य॒मस्य॑ । य॒मस्य॒ बृह॑स्पते । बृह॑स्पते अ॒भिश॑स्तेः । अ॒भिश॑स्ते॒रमु॑ञ्चः । अ॒भिश॑स्ते॒रित्य॒भि - श॒स्तेः॒ । अमु॑ञ्च॒ इत्यमु॑ञ्चः ॥ प्रत्यौ॑हताम् । औ॒ह॒ता॒म॒श्विना᳚ । अ॒श्विना॑मृ॒त्युम् । मृ॒त्युम॑स्मात् । अ॒स्मा॒द् दे॒वाना᳚म् । दे॒वाना॑मग्ने । अ॒ग्ने॒ भि॒षजा᳚ । भि॒षजा॒ शची॑भिः । शची॑भि॒रिति॒ शचि॑ - भिः॒ ॥ उद् व॒यम् । व॒यम् तम॑सः । तम॑स॒स्परि॑ । परि॒ पश्य॑न्तः । पश्य॑न्तो॒ ज्योतिः॑ । ज्योति॒रुत्त॑रम् । उत्त॑र॒मित्युत् - त॒र॒म् ॥ दे॒वम् दे॑व॒त्रा । दे॒व॒त्रा सूर्य᳚म् । दे॒व॒त्रेति॑ देव - त्रा । सूर्य॒मग॑न्म । अग॑न्म॒ ज्योतिः॑ । ज्योति॑रुत्त॒मम् । उ॒त्त॒ममित्यु॑त् - त॒मम् । \newline

\textbf{Jatai Paata} \newline

1. विश्व॑ एन मेनं॒ ॅविश्वे॒ विश्व॑ एनम् । \newline
2. ए॒न॒ मन्वन्वे॑न मेन॒ मनु॑ । \newline
3. अनु॑ मदन्तु मद॒न् त्वन्वनु॑ मदन्तु । \newline
4. म॒द॒न्तु॒ दे॒वा दे॒वा म॑दन्तु मदन्तु दे॒वाः । \newline
5. दे॒वा इति॑ दे॒वाः । \newline
6. अ॒मु॒त्र॒भूया॒ दधाधा॑ मुत्र॒भूया॑ दमुत्र॒भूया॒ दध॑ । \newline
7. अ॒मु॒त्र॒भूया॒दित्य॑मुत्र - भूया᳚त् । \newline
8. अध॒ यद् यदधाध॒ यत् । \newline
9. यद् य॒मस्य॑ य॒मस्य॒ यद् यद् य॒मस्य॑ । \newline
10. य॒मस्य॒ बृह॑स्पते॒ बृह॑स्पते य॒मस्य॑ य॒मस्य॒ बृह॑स्पते । \newline
11. बृह॑स्पते अ॒भिश॑स्ते र॒भिश॑स्ते॒र् बृह॑स्पते॒ बृह॑स्पते अ॒भिश॑स्तेः । \newline
12. अ॒भिश॑स्ते॒ रमु॑ञ्चो॒ अमु॑ञ्चो अ॒भिश॑स्ते र॒भिश॑स्ते॒ रमु॑ञ्चः । \newline
13. अ॒भिश॑स्ते॒रित्य॒भि - श॒स्तेः॒ । \newline
14. अमु॑ञ्च॒ इत्यमु॑ञ्चः । \newline
15. प्रत्यौ॑हता मौहता॒म् प्रति॒ प्रत्यौ॑हताम् । \newline
16. औ॒ह॒ता॒ म॒श्विना॒ ऽश्विनौ॑हता मौहता म॒श्विना᳚ । \newline
17. अ॒श्विना॑ मृ॒त्युम् मृ॒त्यु म॒श्विना॒ ऽश्विना॑ मृ॒त्युम् । \newline
18. मृ॒त्यु म॑स्मा दस्मान् मृ॒त्युम् मृ॒त्यु म॑स्मात् । \newline
19. अ॒स्मा॒द् दे॒वाना᳚म् दे॒वाना॑ मस्मा दस्माद् दे॒वाना᳚म् । \newline
20. दे॒वाना॑ मग्ने अग्ने दे॒वाना᳚म् दे॒वाना॑ मग्ने । \newline
21. अ॒ग्ने॒ भि॒षजा॑ भि॒षजा᳚ ऽग्ने अग्ने भि॒षजा᳚ । \newline
22. भि॒षजा॒ शची॑भिः॒ शची॑भिर् भि॒षजा॑ भि॒षजा॒ शची॑भिः । \newline
23. शची॑भि॒रिति॒ शचि॑ - भिः॒ । \newline
24. उद् व॒यं ॅव॒य मुदुद् व॒यम् । \newline
25. व॒यम् तम॑स॒ स्तम॑सो व॒यं ॅव॒यम् तम॑सः । \newline
26. तम॑स॒ स्परि॒ परि॒ तम॑स॒ स्तम॑स॒ स्परि॑ । \newline
27. परि॒ पश्य॑न्तः॒ पश्य॑न्तः॒ परि॒ परि॒ पश्य॑न्तः । \newline
28. पश्य॑न्तो॒ ज्योति॒र् ज्योतिः॒ पश्य॑न्तः॒ पश्य॑न्तो॒ ज्योतिः॑ । \newline
29. ज्योति॒ रुत्त॑र॒ मुत्त॑र॒म् ज्योति॒र् ज्योति॒ रुत्त॑रम् । \newline
30. उत्त॑र॒मित्युत् - त॒र॒म् । \newline
31. दे॒वम् दे॑व॒त्रा दे॑व॒त्रा दे॒वम् दे॒वम् दे॑व॒त्रा । \newline
32. दे॒व॒त्रा सूर्यꣳ॒॒ सूर्य॑म् देव॒त्रा दे॑व॒त्रा सूर्य᳚म् । \newline
33. दे॒व॒त्रेति॑ देव - त्रा । \newline
34. सूर्य॒ मग॒न्माग॑न्म॒ सूर्यꣳ॒॒ सूर्य॒ मग॑न्म । \newline
35. अग॑न्म॒ ज्योति॒र् ज्योति॒ रग॒न्मा ग॑न्म॒ ज्योतिः॑ । \newline
36. ज्योति॑ रुत्त॒म मु॑त्त॒मम् ज्योति॒र् ज्योति॑ रुत्त॒मम् । \newline
37. उ॒त्त॒ममित्यु॑त् - त॒मम् । \newline

\textbf{Ghana Paata } \newline

1. विश्व॑ एन मेनं॒ ॅविश्वे॒ विश्व॑ एन॒ मन्वन्वे॑नं॒ ॅविश्वे॒ विश्व॑ एन॒ मनु॑ । \newline
2. ए॒न॒ मन्वन्वे॑न मेन॒ मनु॑ मदन्तु मद॒न् त्वन्वे॑न मेन॒ मनु॑ मदन्तु । \newline
3. अनु॑ मदन्तु मद॒न् त्वन्वनु॑ मदन्तु दे॒वा दे॒वा म॑द॒न् त्वन्वनु॑ मदन्तु दे॒वाः । \newline
4. म॒द॒न्तु॒ दे॒वा दे॒वा म॑दन्तु मदन्तु दे॒वाः । \newline
5. दे॒वा इति॑ दे॒वाः । \newline
6. अ॒मु॒त्र॒भूया॒ दधाधा॑ मुत्र॒भूया॑ दमुत्र॒भूया॒ दध॒ यद् यदधा॑ मुत्र॒भूया॑ दमुत्र॒भूया॒ दध॒ यत् । \newline
7. अ॒मु॒त्र॒भूया॒दित्य॑मुत्र - भूया᳚त् । \newline
8. अध॒ यद् यदधाध॒ यद् य॒मस्य॑ य॒मस्य॒ यदधाध॒ यद् य॒मस्य॑ । \newline
9. यद् य॒मस्य॑ य॒मस्य॒ यद् यद् य॒मस्य॒ बृह॑स्पते॒ बृह॑स्पते य॒मस्य॒ यद् यद् य॒मस्य॒ बृह॑स्पते । \newline
10. य॒मस्य॒ बृह॑स्पते॒ बृह॑स्पते य॒मस्य॑ य॒मस्य॒ बृह॑स्पते अ॒भिश॑स्ते र॒भिश॑स्ते॒र् बृह॑स्पते य॒मस्य॑ य॒मस्य॒ बृह॑स्पते अ॒भिश॑स्तेः । \newline
11. बृह॑स्पते अ॒भिश॑स्ते र॒भिश॑स्ते॒र् बृह॑स्पते॒ बृह॑स्पते अ॒भिश॑स्ते॒ रमु॑ञ्चो॒ अमु॑ञ्चो अ॒भिश॑स्ते॒र् बृह॑स्पते॒ बृह॑स्पते अ॒भिश॑स्ते॒ रमु॑ञ्चः । \newline
12. अ॒भिश॑स्ते॒ रमु॑ञ्चो॒ अमु॑ञ्चो अ॒भिश॑स्ते र॒भिश॑स्ते॒ रमु॑ञ्चः । \newline
13. अ॒भिश॑स्ते॒रित्य॒भि - श॒स्तेः॒ । \newline
14. अमु॑ञ्च॒ इत्यमु॑ञ्चः । \newline
15. प्रत्यौ॑हता मौहता॒म् प्रति॒ प्रत्यौ॑हता म॒श्विना॒ ऽश्विनौ॑हता॒म् प्रति॒ प्रत्यौ॑हता म॒श्विना᳚ । \newline
16. औ॒ह॒ता॒ म॒श्विना॒ ऽश्विनौ॑हता मौहता म॒श्विना॑ मृ॒त्युम् मृ॒त्यु म॒श्विनौ॑हता मौहता म॒श्विना॑ मृ॒त्युम् । \newline
17. अ॒श्विना॑ मृ॒त्युम् मृ॒त्यु म॒श्विना॒ ऽश्विना॑ मृ॒त्यु म॑स्मा दस्मान् मृ॒त्यु म॒श्विना॒ ऽश्विना॑ मृ॒त्यु म॑स्मात् । \newline
18. मृ॒त्यु म॑स्मा दस्मान् मृ॒त्युम् मृ॒त्यु म॑स्माद् दे॒वाना᳚म् दे॒वाना॑ मस्मान् मृ॒त्युम् मृ॒त्यु म॑स्माद् दे॒वाना᳚म् । \newline
19. अ॒स्मा॒द् दे॒वाना᳚म् दे॒वाना॑ मस्मा दस्माद् दे॒वाना॑ मग्ने अग्ने दे॒वाना॑ मस्मा दस्माद् दे॒वाना॑ मग्ने । \newline
20. दे॒वाना॑ मग्ने अग्ने दे॒वाना᳚म् दे॒वाना॑ मग्ने भि॒षजा॑ भि॒षजा᳚ ऽग्ने दे॒वाना᳚म् दे॒वाना॑ मग्ने भि॒षजा᳚ । \newline
21. अ॒ग्ने॒ भि॒षजा॑ भि॒षजा᳚ ऽग्ने अग्ने भि॒षजा॒ शची॑भिः॒ शची॑भिर् भि॒षजा᳚ ऽग्ने अग्ने भि॒षजा॒ शची॑भिः । \newline
22. भि॒षजा॒ शची॑भिः॒ शची॑भिर् भि॒षजा॑ भि॒षजा॒ शची॑भिः । \newline
23. शची॑भि॒रिति॒ शचि॑ - भिः॒ । \newline
24. उद् व॒यं ॅव॒य मुदुद् व॒यम् तम॑स॒ स्तम॑सो व॒य मुदुद् व॒यम् तम॑सः । \newline
25. व॒यम् तम॑स॒ स्तम॑सो व॒यं ॅव॒यम् तम॑स॒ स्परि॒ परि॒ तम॑सो व॒यं ॅव॒यम् तम॑स॒ स्परि॑ । \newline
26. तम॑स॒ स्परि॒ परि॒ तम॑स॒ स्तम॑स॒ स्परि॒ पश्य॑न्तः॒ पश्य॑न्तः॒ परि॒ तम॑स॒ स्तम॑स॒ स्परि॒ पश्य॑न्तः । \newline
27. परि॒ पश्य॑न्तः॒ पश्य॑न्तः॒ परि॒ परि॒ पश्य॑न्तो॒ ज्योति॒र् ज्योतिः॒ पश्य॑न्तः॒ परि॒ परि॒ पश्य॑न्तो॒ ज्योतिः॑ । \newline
28. पश्य॑न्तो॒ ज्योति॒र् ज्योतिः॒ पश्य॑न्तः॒ पश्य॑न्तो॒ ज्योति॒ रुत्त॑र॒ मुत्त॑र॒म् ज्योतिः॒ पश्य॑न्तः॒ पश्य॑न्तो॒ ज्योति॒ रुत्त॑रम् । \newline
29. ज्योति॒ रुत्त॑र॒ मुत्त॑र॒म् ज्योति॒र् ज्योति॒ रुत्त॑रम् । \newline
30. उत्त॑र॒मित्युत् - त॒र॒म् । \newline
31. दे॒वम् दे॑व॒त्रा दे॑व॒त्रा दे॒वम् दे॒वम् दे॑व॒त्रा सूर्यꣳ॒॒ सूर्य॑म् देव॒त्रा दे॒वम् दे॒वम् दे॑व॒त्रा सूर्य᳚म् । \newline
32. दे॒व॒त्रा सूर्यꣳ॒॒ सूर्य॑म् देव॒त्रा दे॑व॒त्रा सूर्य॒ मग॒न्मा ग॑न्म॒ सूर्य॑म् देव॒त्रा दे॑व॒त्रा सूर्य॒ मग॑न्म । \newline
33. दे॒व॒त्रेति॑ देव - त्रा । \newline
34. सूर्य॒ मग॒न्मा ग॑न्म॒ सूर्यꣳ॒॒ सूर्य॒ मग॑न्म॒ ज्योति॒र् ज्योति॒ रग॑न्म॒ सूर्यꣳ॒॒ सूर्य॒ मग॑न्म॒ ज्योतिः॑ । \newline
35. अग॑न्म॒ ज्योति॒र् ज्योति॒ रग॒न्मा ग॑न्म॒ ज्योति॑ रुत्त॒म मु॑त्त॒मम् ज्योति॒ रग॒न्मा ग॑न्म॒ ज्योति॑ रुत्त॒मम् । \newline
36. ज्योति॑ रुत्त॒म मु॑त्त॒मम् ज्योति॒र् ज्योति॑ रुत्त॒मम् । \newline
37. उ॒त्त॒ममित्यु॑त् - त॒मम् । \newline
\pagebreak
\markright{ TS 4.1.8.1  \hfill https://www.vedavms.in \hfill}

\section{ TS 4.1.8.1 }

\textbf{TS 4.1.8.1 } \newline
\textbf{Samhita Paata} \newline

ऊ॒र्द्ध्वा अ॑स्य स॒मिधो॑ भवन्त्यू॒र्द्ध्वा शु॒क्रा शो॒चीꣳष्य॒ग्नेः । द्यु॒मत्त॑मा सु॒प्रती॑कस्य सू॒नोः ॥ तनू॒नपा॒दसु॑रो वि॒श्ववे॑दा दे॒वो दे॒वेषु॑ दे॒वः । प॒थ आऽन॑क्ति॒ मद्ध्वा॑ घृ॒तेन॑ ॥ मद्ध्वा॑ य॒ज्ञ्ं न॑क्षसे प्रीणा॒नो नरा॒शꣳसो॑ अग्ने । सु॒कृद्दे॒वः स॑वि॒ता वि॒श्ववा॑रः ॥ अच्छा॒यमे॑ति॒ शव॑सा घृ॒तेने॑डा॒नो वह्नि॒र्नम॑सा । अ॒ग्निꣳ स्रुचो॑ अद्ध्व॒रेषु॑ प्र॒यथ्सु॑ ॥ स य॑क्षदस्य महि॒मान॑म॒ग्नेः स - [  ] \newline

\textbf{Pada Paata} \newline

ऊ॒द्‌र्ध्वाः । अ॒स्य॒ । स॒मिध॒ इति॑ सं-इधः॑ । भ॒व॒न्ति॒ । ऊ॒द्‌र्ध्वा । शु॒क्रा । शो॒चीꣳषि॑ । अ॒ग्नेः ॥ द्यु॒मत्त॒मेति॑ द्यु॒मत् - त॒मा॒ । सु॒प्रती॑क॒स्येति॑ सु - प्रती॑कस्य । सू॒नोः ॥ तनू॒नपा॒दिति॒ तनू᳚-नपा᳚त् । असु॑रः । वि॒श्ववे॑दा॒ इति॑ वि॒श्व-वे॒दाः॒ । दे॒वः । दे॒वेषु॑ । दे॒वः ॥ प॒थः । एति॑ । अ॒न॒क्ति॒ । मद्ध्वा᳚ । घृ॒तेन॑ ॥ मद्ध्वा᳚ । य॒ज्ञ्म् । न॒क्ष॒से॒ । प्री॒णा॒नः । नरा॒शꣳसः॑ । अ॒ग्ने॒ ॥ सु॒कृदिति॑ सु - कृत् । दे॒वः । स॒वि॒ता । वि॒श्ववा॑र॒ इति॑ वि॒श्व - वा॒रः॒ ॥ अच्छ॑ । अ॒यम् । ए॒ति॒ । शव॑सा । घृ॒तेन॑ । ई॒डा॒नः । वह्निः॑ । नम॑सा ॥ अ॒ग्निम् । स्रुचः॑ । अ॒द्ध्व॒रेषु॑ । प्र॒यथ्स्विति॑ प्र॒यत् - सु॒ ॥ सः । य॒क्ष॒त् । अ॒स्य॒ । म॒हि॒मान᳚म् । अ॒ग्नेः । सः ।  \newline


\textbf{Krama Paata} \newline

ऊ॒र्द्ध्वा अ॑स्य । अ॒स्य॒ स॒मिधः॑ । स॒मिधो॑ भवन्ति । स॒मिध॒ इति॑ सम् - इधः॑ । भ॒व॒न्त्यू॒र्द्ध्वा । ऊ॒र्द्ध्वा शु॒क्रा । शु॒क्रा शो॒चीꣳषि॑ । शो॒चीꣳष्य॒ग्नेः । अ॒ग्नेरित्य॒ग्नेः  द्यु॒मत्त॑मा सु॒प्रती॑कस्य । द्यु॒मत्त॒मेति॑ द्यु॒मत् - त॒मा॒ । सु॒प्रती॑कस्य सू॒नोः । सू॒प्रती॑क॒स्येति॑ सु - प्रती॑कस्य । सू॒नोरिति॑ सू॒नोः ॥ तनू॒नपा॒दसु॑रः । तनू॒नपा॒दिति॒ तनू᳚ - नपा᳚त् । असु॑रो वि॒श्ववे॑दाः । वि॒श्ववे॑दा दे॒वः । वि॒श्ववे॑दा॒ इति॑ वि॒श्व - वे॒दाः॒ । दे॒वो दे॒वेषु॑ । दे॒वेषु॑ दे॒वः । दे॒व इति॑ दे॒वः ॥ प॒थ आ । आ ऽन॑क्ति । अ॒न॒क्ति॒ मद्ध्वा᳚ । मद्ध्वा॑ घृ॒तेन॑ । घृ॒तेनेति॑ घृ॒तेन॑ ॥ मद्ध्वा॑ य॒ज्ञ्म् । य॒ज्ञ्म् न॑क्षसे । न॒क्ष॒से॒ प्री॒णा॒नः । प्री॒णा॒नो नरा॒शꣳसः॑ । नरा॒शꣳसो॑ अग्ने । अ॒ग्न॒ इत्य॑ग्ने ॥ सु॒कृद् दे॒वः । सु॒कृदिति॑ सु - कृत् । दे॒वः स॑वि॒ता । स॒वि॒ता वि॒श्ववा॑रः । वि॒श्ववा॑र॒ इति॑ वि॒श्व - वा॒रः॒ ॥ अच्छा॒ऽयम् । अ॒यमे॑ति । ए॒ति॒ शव॑सा । शव॑सा घृ॒तेन॑ । घृ॒तेने॑डा॒नः । ई॒डा॒नो वह्निः॑ । वह्नि॒र् नम॑सा । नम॒सेति॒ नम॑सा ॥ अ॒ग्निꣳ स्रुचः॑ । स्रुचो॑ अद्ध्व॒रेषु॑ । अ॒द्ध्व॒रेषु॑ प्र॒यथ्सु॑ । प्र॒यथ्स्विति॑ प्र॒यत् - सु॒ ॥ स य॑क्षत् । य॒क्ष॒द॒स्य॒ । अ॒स्य॒ म॒हि॒मान᳚म् । म॒हि॒मान॑म॒ग्नेः । अ॒ग्नेः सः । स ई᳚म् \newline

\textbf{Jatai Paata} \newline

1. ऊ॒र्द्ध्वा अ॑स्या स्यो॒र्द्ध्वा ऊ॒र्द्ध्वा अ॑स्य । \newline
2. अ॒स्य॒ स॒मिधः॑ स॒मिधो॑ अस्यास्य स॒मिधः॑ । \newline
3. स॒मिधो॑ भवन्ति भवन्ति स॒मिधः॑ स॒मिधो॑ भवन्ति । \newline
4. स॒मिध॒ इति॑ सं - इधः॑ । \newline
5. भ॒व॒ न्त्यू॒र्द्ध्वो र्द्ध्वा भ॑वन्ति भव न्त्यू॒र्द्ध्वा । \newline
6. ऊ॒र्द्ध्वा शु॒क्रा शु॒क्रोर्द्ध्वो र्द्ध्वा शु॒क्रा । \newline
7. शु॒क्रा शो॒चीꣳषि॑ शो॒चीꣳषि॑ शु॒क्रा शु॒क्रा शो॒चीꣳषि॑ । \newline
8. शो॒चीꣳ ष्य॒ग्ने र॒ग्नेः शो॒चीꣳषि॑ शो॒चीꣳ ष्य॒ग्नेः । \newline
9. अ॒ग्नेरित्य॒ग्नेः । \newline
10. द्यु॒मत्त॑मा सु॒प्रती॑कस्य सु॒प्रती॑कस्य द्यु॒मत्त॑मा द्यु॒मत्त॑मा सु॒प्रती॑कस्य । \newline
11. द्यु॒मत्त॒मेति॑ द्यु॒मत् - त॒मा॒ । \newline
12. सु॒प्रती॑कस्य सू॒नोः सू॒नोः सु॒प्रती॑कस्य सु॒प्रती॑कस्य सू॒नोः । \newline
13. सु॒प्रती॑क॒स्येति॑ सु - प्रती॑कस्य । \newline
14. सू॒नोरिति॑ सू॒नोः । \newline
15. तनू॒नपा॒ दसु॑रो॒ असु॑र॒ स्तनू॒नपा॒त् तनू॒नपा॒ दसु॑रः । \newline
16. तनू॒नपा॒दिति॒ तनू᳚ - नपा᳚त् । \newline
17. असु॑रो वि॒श्ववे॑दा वि॒श्ववे॑दा॒ असु॑रो॒ असु॑रो वि॒श्ववे॑दाः । \newline
18. वि॒श्ववे॑दा दे॒वो दे॒वो वि॒श्ववे॑दा वि॒श्ववे॑दा दे॒वः । \newline
19. वि॒श्ववे॑दा॒ इति॑ वि॒श्व - वे॒दाः॒ । \newline
20. दे॒वो दे॒वेषु॑ दे॒वेषु॑ दे॒वो दे॒वो दे॒वेषु॑ । \newline
21. दे॒वेषु॑ दे॒वो दे॒वो दे॒वेषु॑ दे॒वेषु॑ दे॒वः । \newline
22. दे॒व इति॑ दे॒वः । \newline
23. प॒थ आ प॒थः प॒थ आ । \newline
24. आ ऽन॑क्त्य न॒क्त्या ऽन॑क्ति । \newline
25. अ॒न॒क्ति॒ मद्ध्वा॒ मद्ध्वा॑ ऽनक्त्यनक्ति॒ मद्ध्वा᳚ । \newline
26. मद्ध्वा॑ घृ॒तेन॑ घृ॒तेन॒ मद्ध्वा॒ मद्ध्वा॑ घृ॒तेन॑ । \newline
27. घृ॒तेनेति॑ घृ॒तेन॑ । \newline
28. मद्ध्वा॑ य॒ज्ञ्ं ॅय॒ज्ञ्म् मद्ध्वा॒ मद्ध्वा॑ य॒ज्ञ्म् । \newline
29. य॒ज्ञ्म् न॑क्षसे नक्षसे य॒ज्ञ्ं ॅय॒ज्ञ्म् न॑क्षसे । \newline
30. न॒क्ष॒से॒ प्री॒णा॒नः प्री॑णा॒नो न॑क्षसे नक्षसे प्रीणा॒नः । \newline
31. प्री॒णा॒नो नरा॒शꣳसो॒ नरा॒शꣳसः॑ प्रीणा॒नः प्री॑णा॒नो नरा॒शꣳसः॑ । \newline
32. नरा॒शꣳसो॑ अग्ने अग्ने॒ नरा॒शꣳसो॒ नरा॒शꣳसो॑ अग्ने । \newline
33. अ॒ग्न॒ इत्य॑ग्ने । \newline
34. सु॒कृद् दे॒वो दे॒वः सु॒कृथ् सु॒कृद् दे॒वः । \newline
35. सु॒कृदिति॑ सु - कृत् । \newline
36. दे॒वः स॑वि॒ता स॑वि॒ता दे॒वो दे॒वः स॑वि॒ता । \newline
37. स॒वि॒ता वि॒श्ववा॑रो वि॒श्ववा॑रः सवि॒ता स॑वि॒ता वि॒श्ववा॑रः । \newline
38. वि॒श्ववा॑र॒ इति॑ वि॒श्व - वा॒रः॒ । \newline
39. अच्छा॒य म॒य मच्छा च्छा॒यम् । \newline
40. अ॒य मे᳚त्ये त्य॒य म॒य मे॑ति । \newline
41. ए॒ति॒ शव॑सा॒ शव॑ सैत्येति॒ शव॑सा । \newline
42. शव॑सा घृ॒तेन॑ घृ॒तेन॒ शव॑सा॒ शव॑सा घृ॒तेन॑ । \newline
43. घृ॒तेने॑डा॒न ई॑डा॒नो घृ॒तेन॑ घृ॒तेने॑डा॒नः । \newline
44. ई॒डा॒नो वह्नि॒र् वह्नि॑ रीडा॒न ई॑डा॒नो वह्निः॑ । \newline
45. वह्नि॒र् नम॑सा॒ नम॑सा॒ वह्नि॒र् वह्नि॒र् नम॑सा । \newline
46. नम॒सेति॒ नम॑सा । \newline
47. अ॒ग्निꣳ स्रुचः॒ स्रुचो॑ अ॒ग्नि म॒ग्निꣳ स्रुचः॑ । \newline
48. स्रुचो॑ अद्ध्व॒रे ष्व॑द्ध्व॒रेषु॒ स्रुचः॒ स्रुचो॑ अद्ध्व॒रेषु॑ । \newline
49. अ॒द्ध्व॒रेषु॑ प्र॒यथ्सु॑ प्र॒य थ्स्व॑द्ध्व॒रे ष्व॑द्ध्व॒रेषु॑ प्र॒यथ्सु॑ । \newline
50. प्र॒यथ्स्विति॑ प्र॒यत् - सु॒ । \newline
51. स य॑क्षद् यक्ष॒थ् स स य॑क्षत् । \newline
52. य॒क्ष॒ द॒स्या॒स्य॒ य॒क्ष॒द् य॒क्ष॒ द॒स्य॒ । \newline
53. अ॒स्य॒ म॒हि॒मान॑म् महि॒मान॑ मस्यास्य महि॒मान᳚म् । \newline
54. म॒हि॒मान॑ म॒ग्ने र॒ग्नेर् म॑हि॒मान॑म् महि॒मान॑ म॒ग्नेः । \newline
55. अ॒ग्नेः स सो अ॒ग्ने र॒ग्नेः सः । \newline
56. स ई॑ मीꣳ॒॒ स स ई᳚म् । \newline

\textbf{Ghana Paata } \newline

1. ऊ॒र्द्ध्वा अ॑स्या स्यो॒र्द्ध्वा ऊ॒र्द्ध्वा अ॑स्य स॒मिधः॑ स॒मिधो॑ अस्यो॒र्द्ध्वा ऊ॒र्द्ध्वा अ॑स्य स॒मिधः॑ । \newline
2. अ॒स्य॒ स॒मिधः॑ स॒मिधो॑ अस्यास्य स॒मिधो॑ भवन्ति भवन्ति स॒मिधो॑ अस्यास्य स॒मिधो॑ भवन्ति । \newline
3. स॒मिधो॑ भवन्ति भवन्ति स॒मिधः॑ स॒मिधो॑ भवन् त्यू॒र्द्ध्वोर्द्ध्वा भ॑वन्ति स॒मिधः॑ स॒मिधो॑ भवन् त्यू॒र्द्ध्वा । \newline
4. स॒मिध॒ इति॑ सं - इधः॑ । \newline
5. भ॒व॒न् त्यू॒र्द्ध्वोर्द्ध्वा भ॑वन्ति भवन् त्यू॒र्द्ध्वा शु॒क्रा शु॒क्रोर्द्ध्वा भ॑वन्ति भवन् त्यू॒र्द्ध्वा शु॒क्रा । \newline
6. ऊ॒र्द्ध्वा शु॒क्रा शु॒क्रो-र्द्ध्वोर्द्ध्वा शु॒क्रा शो॒चीꣳषि॑ शो॒चीꣳषि॑ शु॒क्रो-र्द्ध्वोर्द्ध्वा शु॒क्रा शो॒चीꣳषि॑ । \newline
7. शु॒क्रा शो॒चीꣳषि॑ शो॒चीꣳषि॑ शु॒क्रा शु॒क्रा शो॒चीꣳष्य॒ग्ने र॒ग्नेः शो॒चीꣳषि॑ शु॒क्रा शु॒क्रा शो॒चीꣳष्य॒ग्नेः । \newline
8. शो॒चीꣳष्य॒ग्ने र॒ग्नेः शो॒चीꣳषि॑ शो॒चीꣳष्य॒ग्नेः । \newline
9. अ॒ग्नेरित्य॒ग्नेः । \newline
10. द्यु॒मत्त॑मा सु॒प्रती॑कस्य सु॒प्रती॑कस्य द्यु॒मत्त॑मा द्यु॒मत्त॑मा सु॒प्रती॑कस्य सू॒नोः सू॒नोः सु॒प्रती॑कस्य द्यु॒मत्त॑मा द्यु॒मत्त॑मा सु॒प्रती॑कस्य सू॒नोः । \newline
11. द्यु॒मत्त॒मेति॑ द्यु॒मत् - त॒मा॒ । \newline
12. सु॒प्रती॑कस्य सू॒नोः सू॒नोः सु॒प्रती॑कस्य सु॒प्रती॑कस्य सू॒नोः । \newline
13. सु॒प्रती॑क॒स्येति॑ सु - प्रती॑कस्य । \newline
14. सू॒नोरिति॑ सू॒नोः । \newline
15. तनू॒नपा॒ दसु॑रो॒ असु॑र॒ स्तनू॒नपा॒त् तनू॒नपा॒ दसु॑रो वि॒श्ववे॑दा वि॒श्ववे॑दा॒ असु॑र॒ स्तनू॒नपा॒त् तनू॒नपा॒ दसु॑रो वि॒श्ववे॑दाः । \newline
16. तनू॒नपा॒दिति॒ तनू᳚ - नपा᳚त् । \newline
17. असु॑रो वि॒श्ववे॑दा वि॒श्ववे॑दा॒ असु॑रो॒ असु॑रो वि॒श्ववे॑दा दे॒वो दे॒वो वि॒श्ववे॑दा॒ असु॑रो॒ असु॑रो वि॒श्ववे॑दा दे॒वः । \newline
18. वि॒श्ववे॑दा दे॒वो दे॒वो वि॒श्ववे॑दा वि॒श्ववे॑दा दे॒वो दे॒वेषु॑ दे॒वेषु॑ दे॒वो वि॒श्ववे॑दा वि॒श्ववे॑दा दे॒वो दे॒वेषु॑ । \newline
19. वि॒श्ववे॑दा॒ इति॑ वि॒श्व - वे॒दाः॒ । \newline
20. दे॒वो दे॒वेषु॑ दे॒वेषु॑ दे॒वो दे॒वो दे॒वेषु॑ दे॒वो दे॒वो दे॒वेषु॑ दे॒वो दे॒वो दे॒वेषु॑ दे॒वः । \newline
21. दे॒वेषु॑ दे॒वो दे॒वो दे॒वेषु॑ दे॒वेषु॑ दे॒वः । \newline
22. दे॒व इति॑ दे॒वः । \newline
23. प॒थ आ प॒थः प॒थ आ ऽन॑क्त्य न॒क्त्या प॒थः प॒थ आ ऽन॑क्ति । \newline
24. आ ऽन॑क्त्य न॒क्त्या ऽन॑क्ति॒ मद्ध्वा॒ मद्ध्वा॑ ऽन॒क्त्या ऽन॑क्ति॒ मद्ध्वा᳚ । \newline
25. अ॒न॒क्ति॒ मद्ध्वा॒ मद्ध्वा॑ ऽनक्त्य नक्ति॒ मद्ध्वा॑ घृ॒तेन॑ घृ॒तेन॒ मद्ध्वा॑ ऽनक्त्य नक्ति॒ मद्ध्वा॑ घृ॒तेन॑ । \newline
26. मद्ध्वा॑ घृ॒तेन॑ घृ॒तेन॒ मद्ध्वा॒ मद्ध्वा॑ घृ॒तेन॑ । \newline
27. घृ॒तेनेति॑ घृ॒तेन॑ । \newline
28. मद्ध्वा॑ य॒ज्ञ्ं ॅय॒ज्ञ्म् मद्ध्वा॒ मद्ध्वा॑ य॒ज्ञ्म् न॑क्षसे नक्षसे य॒ज्ञ्म् मद्ध्वा॒ मद्ध्वा॑ य॒ज्ञ्म् न॑क्षसे । \newline
29. य॒ज्ञ्म् न॑क्षसे नक्षसे य॒ज्ञ्ं ॅय॒ज्ञ्म् न॑क्षसे प्रीणा॒नः प्री॑णा॒नो न॑क्षसे य॒ज्ञ्ं ॅय॒ज्ञ्म् न॑क्षसे प्रीणा॒नः । \newline
30. न॒क्ष॒से॒ प्री॒णा॒नः प्री॑णा॒नो न॑क्षसे नक्षसे प्रीणा॒नो नरा॒शꣳसो॒ नरा॒शꣳसः॑ प्रीणा॒नो न॑क्षसे नक्षसे प्रीणा॒नो नरा॒शꣳसः॑ । \newline
31. प्री॒णा॒नो नरा॒शꣳसो॒ नरा॒शꣳसः॑ प्रीणा॒नः प्री॑णा॒नो नरा॒शꣳसो॑ अग्ने अग्ने॒ नरा॒शꣳसः॑ प्रीणा॒नः प्री॑णा॒नो नरा॒शꣳसो॑ अग्ने । \newline
32. नरा॒शꣳसो॑ अग्ने अग्ने॒ नरा॒शꣳसो॒ नरा॒शꣳसो॑ अग्ने । \newline
33. अ॒ग्न॒ इत्य॑ग्ने । \newline
34. सु॒कृद् दे॒वो दे॒वः सु॒कृथ् सु॒कृद् दे॒वः स॑वि॒ता स॑वि॒ता दे॒वः सु॒कृथ् सु॒कृद् दे॒वः स॑वि॒ता । \newline
35. सु॒कृदिति॑ सु - कृत् । \newline
36. दे॒वः स॑वि॒ता स॑वि॒ता दे॒वो दे॒वः स॑वि॒ता वि॒श्ववा॑रो वि॒श्ववा॑रः सवि॒ता दे॒वो दे॒वः स॑वि॒ता वि॒श्ववा॑रः । \newline
37. स॒वि॒ता वि॒श्ववा॑रो वि॒श्ववा॑रः सवि॒ता स॑वि॒ता वि॒श्ववा॑रः । \newline
38. वि॒श्ववा॑र॒ इति॑ वि॒श्व - वा॒रः॒ । \newline
39. अच्छा॒य म॒य मच्छा च्छा॒य मे᳚त्ये त्य॒य मच्छा च्छा॒य मे॑ति । \newline
40. अ॒य मे᳚त्येत्य॒य म॒य मे॑ति॒ शव॑सा॒ शव॑सै त्य॒य म॒य मे॑ति॒ शव॑सा । \newline
41. ए॒ति॒ शव॑सा॒ शव॑सै त्येति॒ शव॑सा घृ॒तेन॑ घृ॒तेन॒ शव॑ सैत्येति॒ शव॑सा घृ॒तेन॑ । \newline
42. शव॑सा घृ॒तेन॑ घृ॒तेन॒ शव॑सा॒ शव॑सा घृ॒तेने॑ डा॒न ई॑डा॒नो घृ॒तेन॒ शव॑सा॒ शव॑सा घृ॒तेने॑ डा॒नः । \newline
43. घृ॒तेने॑डा॒न ई॑डा॒नो घृ॒तेन॑ घृ॒तेने॑ डा॒नो वह्नि॒र् वह्नि॑ रीडा॒नो घृ॒तेन॑ घृ॒तेने॑ डा॒नो वह्निः॑ । \newline
44. ई॒डा॒नो वह्नि॒र् वह्नि॑ रीडा॒न ई॑डा॒नो वह्नि॒र् नम॑सा॒ नम॑सा॒ वह्नि॑ रीडा॒न ई॑डा॒नो वह्नि॒र् नम॑सा । \newline
45. वह्नि॒र् नम॑सा॒ नम॑सा॒ वह्नि॒र् वह्नि॒र् नम॑सा । \newline
46. नम॒सेति॒ नम॑सा । \newline
47. अ॒ग्निꣳ स्रुचः॒ स्रुचो॑ अ॒ग्नि म॒ग्निꣳ स्रुचो॑ अद्ध्व॒रे ष्व॑द्ध्व॒रेषु॒ स्रुचो॑ अ॒ग्नि म॒ग्निꣳ स्रुचो॑ अद्ध्व॒रेषु॑ । \newline
48. स्रुचो॑ अद्ध्व॒रे ष्व॑द्ध्व॒रेषु॒ स्रुचः॒ स्रुचो॑ अद्ध्व॒रेषु॑ प्र॒यथ्सु॑ प्र॒य थ्स्व॑द्ध्व॒रेषु॒ स्रुचः॒ स्रुचो॑ अद्ध्व॒रेषु॑ प्र॒यथ्सु॑ । \newline
49. अ॒द्ध्व॒रेषु॑ प्र॒यथ्सु॑ प्र॒यथ् स्व॑द्ध्व॒रे ष्व॑द्ध्व॒रेषु॑ प्र॒यथ्सु॑ । \newline
50. प्र॒यथ्स्विति॑ प्र॒यत् - सु॒ । \newline
51. स य॑क्षद् यक्ष॒थ् स स य॑क्ष दस्यास्य यक्ष॒थ् स स य॑क्ष दस्य । \newline
52. य॒क्ष॒ द॒स्या॒स्य॒ य॒क्ष॒द् य॒क्ष॒ द॒स्य॒ म॒हि॒मान॑म् महि॒मान॑ मस्य यक्षद् यक्ष दस्य महि॒मान᳚म् । \newline
53. अ॒स्य॒ म॒हि॒मान॑म् महि॒मान॑ मस्यास्य महि॒मान॑ म॒ग्ने र॒ग्नेर् म॑हि॒मान॑ मस्यास्य महि॒मान॑ म॒ग्नेः । \newline
54. म॒हि॒मान॑ म॒ग्ने र॒ग्नेर् म॑हि॒मान॑म् महि॒मान॑ म॒ग्नेः स सो अ॒ग्नेर् म॑हि॒मान॑म् महि॒मान॑ म॒ग्नेः सः । \newline
55. अ॒ग्नेः स सो अ॒ग्ने र॒ग्नेः स ई॑मीꣳ॒॒ सो अ॒ग्ने र॒ग्नेः स ई᳚म् । \newline
56. स ई॑ मीꣳ॒॒ स स ई॑ म॒न्द्रासु॑ म॒न्द्रास्वीꣳ॒॒ स स ई॑ म॒न्द्रासु॑ । \newline
\pagebreak
\markright{ TS 4.1.8.2  \hfill https://www.vedavms.in \hfill}

\section{ TS 4.1.8.2 }

\textbf{TS 4.1.8.2 } \newline
\textbf{Samhita Paata} \newline

ई॑ म॒न्द्रासु॑ प्र॒यसः॑ । वसु॒श्चेति॑ष्ठो वसु॒धात॑मश्च ॥ द्वारो॑ दे॒वीरन्व॑स्य॒ विश्वे᳚ व्र॒ता द॑दन्ते अ॒ग्नेः । उ॒रु॒व्यच॑सो॒ धाम्ना॒ पत्य॑मानाः ॥ ते अ॑स्य॒ योष॑णे दि॒व्ये न योना॑वु॒षासा॒नक्ता᳚ । इ॒मं ॅय॒ज्ञ्म॑वता मद्ध्व॒रं नः॑ ॥ दैव्या॑ होतारावू॒र्द्ध्व-म॑द्ध्व॒रं नो॒ऽग्नेर्जि॒ह्वाम॒भि गृ॑णीतं । कृ॒णु॒तं नः॒ स्वि॑ष्टिं ॥ ति॒स्रो दे॒वीर्ब॒र्॒.हिरेदꣳ स॑द॒न्त्विडा॒ सर॑स्वती॒- [  ] \newline

\textbf{Pada Paata} \newline

ई॒म् । म॒न्द्रासु॑ । प्र॒यसः॑ ॥ वसुः॑ । चेति॑ष्ठः । व॒सु॒धात॑म॒ इति॑ वसु - धात॑मः । च॒ ॥ द्वारः॑ । दे॒वीः । अन्विति॑ । अ॒स्य॒ । विश्वे᳚ । व्र॒ता । द॒द॒न्ते॒ । अ॒ग्नेः ॥ उ॒रु॒व्यच॑स॒ इत्यु॑रु - व्यच॑सः । धाम्ना᳚ । पत्य॑मानाः ॥ ते इति॑ । अ॒स्य॒ । योष॑णे॒ इति॑ । दि॒व्ये इति॑ । न । योनौ᳚ । उ॒षासा॒नक्ता᳚ ॥ इ॒मम् । य॒ज्ञ्म् । अ॒व॒ता॒म् । अ॒द्ध्व॒रम् । नः॒ ॥ दैव्या᳚ । हो॒ता॒रौ॒ । ऊ॒द्‌र्ध्वम् । अ॒द्ध्व॒रम् । नः॒ । अ॒ग्नेः । जि॒ह्वाम् । अ॒भीति॑ । गृ॒णी॒त॒म् ॥ कृ॒णु॒तम् । नः॒ । स्वि॑ष्टि॒मिति॒ सु-इ॒ष्टि॒म् ॥ ति॒स्रः । दे॒वीः । ब॒र्॒.हिः । एति॑ । इ॒दम् । स॒द॒न्तु॒ । इडा᳚ । सर॑स्वती ।  \newline


\textbf{Krama Paata} \newline

ई॒ म॒न्द्रासु॑ । म॒न्द्रासु॑ प्र॒यसः॑ । प्र॒यस॒ इति॑ प्र॒यसः॑ ॥ वसु॒श्चेति॑ष्ठः । चेति॑ष्ठो वसु॒धात॑मः । व॒सु॒धात॑मश्च । व॒सु॒धात॑म॒ इति॑ वसु - धात॑मः । चेति॑ च ॥ द्वारो॑ दे॒वीः । दे॒वीरनु॑ । अन्व॑स्य । अ॒स्य॒ विश्वे᳚ । विश्वे᳚ व्र॒ता । व्र॒ता द॑दन्ते । द॒द॒न्ते॒ अ॒ग्नेः । अ॒ग्नेरित्य॒ग्नेः ॥ उ॒रु॒व्यच॑सो॒ धाम्ना᳚ । उ॒रु॒व्यच॑स॒ इत्यु॑रु - व्यच॑सः । धाम्ना॒ पत्य॑मानाः । पत्य॑माना॒ इति॒ पत्य॑मानाः ॥ ते अ॑स्य । ते इति॒ ते । अ॒स्य॒ योष॑णे । योष॑णे दि॒व्ये । योष॑णे॒ इति॒ योष॑णे । दि॒व्ये न । दि॒व्ये इति॑ दि॒व्ये । न योनौ᳚ । योना॑वु॒षासा॒नक्ता᳚ । उ॒षासा॒नक्तेत्यु॒षासा॒नक्ता᳚ ॥ इ॒मं ॅय॒ज्ञ्म् । य॒ज्ञ्म॑वताम् । अ॒व॒ता॒म॒द्ध्व॒रम् । अ॒द्ध्व॒रम् नः॑ । न॒ इति॑ नः ॥ दैव्या॑ होतारौ । हो॒ता॒रा॒वू॒र्द्ध्वम् । ऊ॒र्द्ध्वम॑द्ध्व॒रम् । अ॒द्ध्व॒रम् नः॑ । नो॒ऽग्नेः । अ॒ग्नेर् जि॒ह्वाम् । जि॒ह्वाम॒भि । अ॒भि गृ॑णीतम् । गृ॒णी॒त॒मिति॑ गृणीतम् ॥ कृ॒णु॒तम् नः॑ । नः॒ स्वि॑ष्टिम् । स्वि॑ष्टि॒मिति॒ सु - इ॒ष्टि॒म् ॥ ति॒स्रो दे॒वीः । दे॒वीर् ब॒र्॒.हिः । ब॒र्॒.हिरा । एदम् । इ॒दꣳ स॑दन्तु । स॒द॒न्त्विडा᳚ । इडा॒ सर॑स्वती । सर॑स्वती॒ भार॑ती \newline

\textbf{Jatai Paata} \newline

1. ई॒ म॒न्द्रासु॑ म॒न्द्रास्वी॑ मीम॒न्द्रासु॑ । \newline
2. म॒न्द्रासु॑ प्र॒यसः॑ प्र॒यसो॑ म॒न्द्रासु॑ म॒न्द्रासु॑ प्र॒यसः॑ । \newline
3. प्र॒यस॒ इति॑ प्र॒यसः॑ । \newline
4. वसु॒ श्चेति॑ष्ठ॒ श्चेति॑ष्ठो॒ वसु॒र् वसु॒ श्चेति॑ष्ठः । \newline
5. चेति॑ष्ठो वसु॒धात॑मो वसु॒धात॑म॒ श्चेति॑ष्ठ॒ श्चेति॑ष्ठो वसु॒धात॑मः । \newline
6. व॒सु॒धात॑मश्च च वसु॒धात॑मो वसु॒धात॑मश्च । \newline
7. व॒सु॒धात॑म॒ इति॑ वसु - धात॑मः । \newline
8. चेति॑ च । \newline
9. द्वारो॑ दे॒वीर् दे॒वीर् द्वारो॒ द्वारो॑ दे॒वीः । \newline
10. दे॒वी रन्वनु॑ दे॒वीर् दे॒वी रनु॑ । \newline
11. अन्व॑स्या॒ स्या न्वन् व॑स्य । \newline
12. अ॒स्य॒ विश्वे॒ विश्वे॑ अस्यास्य॒ विश्वे᳚ । \newline
13. विश्वे᳚ व्र॒ता व्र॒ता विश्वे॒ विश्वे᳚ व्र॒ता । \newline
14. व्र॒ता द॑दन्ते ददन्ते व्र॒ता व्र॒ता द॑दन्ते । \newline
15. द॒द॒न्ते॒ अ॒ग्ने र॒ग्नेर् द॑दन्ते ददन्ते अ॒ग्नेः । \newline
16. अ॒ग्नेरित्य॒ग्नेः । \newline
17. उ॒रु॒व्यच॑सो॒ धाम्ना॒ धाम्नो॑ रु॒व्यच॑स उरु॒व्यच॑सो॒ धाम्ना᳚ । \newline
18. उ॒रु॒व्यच॑स॒ इत्यु॑रु - व्यच॑सः । \newline
19. धाम्ना॒ पत्य॑मानाः॒ पत्य॑माना॒ धाम्ना॒ धाम्ना॒ पत्य॑मानाः । \newline
20. पत्य॑माना॒ इति॒ पत्य॑मानाः । \newline
21. ते अ॑स्यास्य॒ ते ते अ॑स्य । \newline
22. ते इति॒ ते । \newline
23. अ॒स्य॒ योष॑णे॒ योष॑णे अस्यास्य॒ योष॑णे । \newline
24. योष॑णे दि॒व्ये दि॒व्ये योष॑णे॒ योष॑णे दि॒व्ये । \newline
25. योष॑णे॒ इति॒ योष॑णे । \newline
26. दि॒व्ये न न दि॒व्ये दि॒व्ये न । \newline
27. दि॒व्ये इति॑ दि॒व्ये । \newline
28. न योनौ॒ योनौ॒ न न योनौ᳚ । \newline
29. योना॑ वु॒षासा॒नक्तो॒ षासा॒नक्ता॒ योनौ॒ योना॑ वु॒षासा॒नक्ता᳚ । \newline
30. उ॒षासा॒नक्ते त्यु॒षासा॒नक्ता᳚ । \newline
31. इ॒मं ॅय॒ज्ञ्ं ॅय॒ज्ञ् मि॒म मि॒मं ॅय॒ज्ञ्म् । \newline
32. य॒ज्ञ् म॑वता मवतां ॅय॒ज्ञ्ं ॅय॒ज्ञ् म॑वताम् । \newline
33. अ॒व॒ता॒ म॒द्ध्व॒र म॑द्ध्व॒र म॑वता मवता मद्ध्व॒रम् । \newline
34. अ॒द्ध्व॒रम् नो॑ नो अद्ध्व॒र म॑द्ध्व॒रम् नः॑ । \newline
35. न॒ इति॑ नः । \newline
36. दैव्या॑ होतारौ होतारौ॒ दैव्या॒ दैव्या॑ होतारौ । \newline
37. हो॒ता॒रा॒ वू॒र्द्ध्व मू॒र्द्ध्वꣳ हो॑तारौ होतारा वू॒र्द्ध्वम् । \newline
38. ऊ॒र्द्ध्व म॑द्ध्व॒र म॑द्ध्व॒र मू॒र्द्ध्व मू॒र्द्ध्व म॑द्ध्व॒रम् । \newline
39. अ॒द्ध्व॒रम् नो॑ नो अद्ध्व॒र म॑द्ध्व॒रम् नः॑ । \newline
40. नो॒ ऽग्ने र॒ग्नेर् नो॑ नो॒ ऽग्नेः । \newline
41. अ॒ग्नेर् जि॒ह्वाम् जि॒ह्वा म॒ग्ने र॒ग्नेर् जि॒ह्वाम् । \newline
42. जि॒ह्वा म॒भ्य॑भि जि॒ह्वाम् जि॒ह्वा म॒भि । \newline
43. अ॒भि गृ॑णीतम् गृणीत म॒भ्य॑भि गृ॑णीतम् । \newline
44. गृ॒णी॒त॒मिति॑ गृणीतम् । \newline
45. कृ॒णु॒तम् नो॑ नः कृणु॒तम् कृ॑णु॒तम् नः॑ । \newline
46. नः॒ स्वि॑ष्टिꣳ॒॒ स्वि॑ष्टिम् नो नः॒ स्वि॑ष्टिम् । \newline
47. स्वि॑ष्टि॒मिति॒ सु - इ॒ष्टि॒म् । \newline
48. ति॒स्रो दे॒वीर् दे॒वी स्ति॒स्र स्ति॒स्रो दे॒वीः । \newline
49. दे॒वीर् ब॒र्॒.हिर् ब॒र्॒.हिर् दे॒वीर् दे॒वीर् ब॒र्॒.हिः । \newline
50. ब॒र्॒.हिरा ब॒र्॒.हिर् ब॒र्॒.हिरा । \newline
51. एद मि॒द मेदम् । \newline
52. इ॒दꣳ स॑दन्तु सदन्त्वि॒द मि॒दꣳ स॑दन्तु । \newline
53. स॒द॒न्त्विडेडा॑ सदन्तु सद॒न्त्विडा᳚ । \newline
54. इडा॒ सर॑स्वती॒ सर॑स्व॒ती डेडा॒ सर॑स्वती । \newline
55. सर॑स्वती॒ भार॑ती॒ भार॑ती॒ सर॑स्वती॒ सर॑स्वती॒ भार॑ती । \newline

\textbf{Ghana Paata } \newline

1. ई॒ म॒न्द्रासु॑ म॒न्द्रास्वी॑ मी म॒न्द्रासु॑ प्र॒यसः॑ प्र॒यसो॑ म॒न्द्रास्वी॑ मी म॒न्द्रासु॑ प्र॒यसः॑ । \newline
2. म॒न्द्रासु॑ प्र॒यसः॑ प्र॒यसो॑ म॒न्द्रासु॑ म॒न्द्रासु॑ प्र॒यसः॑ । \newline
3. प्र॒यस॒ इति॑ प्र॒यसः॑ । \newline
4. वसु॒ श्चेति॑ष्ठ॒ श्चेति॑ष्ठो॒ वसु॒र् वसु॒ श्चेति॑ष्ठो वसु॒धात॑मो वसु॒धात॑म॒ श्चेति॑ष्ठो॒ वसु॒र् वसु॒ श्चेति॑ष्ठो वसु॒धात॑मः । \newline
5. चेति॑ष्ठो वसु॒धात॑मो वसु॒धात॑म॒ श्चेति॑ष्ठ॒ श्चेति॑ष्ठो वसु॒धात॑मश्च च वसु॒धात॑म॒ श्चेति॑ष्ठ॒ श्चेति॑ष्ठो वसु॒धात॑मश्च । \newline
6. व॒सु॒धात॑मश्च च वसु॒धात॑मो वसु॒धात॑मश्च । \newline
7. व॒सु॒धात॑म॒ इति॑ वसु - धात॑मः । \newline
8. चेति॑ च । \newline
9. द्वारो॑ दे॒वीर् दे॒वीर् द्वारो॒ द्वारो॑ दे॒वी रन्वनु॑ दे॒वीर् द्वारो॒ द्वारो॑ दे॒वी रनु॑ । \newline
10. दे॒वी रन्वनु॑ दे॒वीर् दे॒वी रन्व॑स्या॒ स्यानु॑ दे॒वीर् दे॒वी रन्व॑स्य । \newline
11. अन्व॑स्या॒स्या न्वन्व॑स्य॒ विश्वे॒ विश्वे॑ अ॒स्या न्वन्व॑स्य॒ विश्वे᳚ । \newline
12. अ॒स्य॒ विश्वे॒ विश्वे॑ अस्यास्य॒ विश्वे᳚ व्र॒ता व्र॒ता विश्वे॑ अस्यास्य॒ विश्वे᳚ व्र॒ता । \newline
13. विश्वे᳚ व्र॒ता व्र॒ता विश्वे॒ विश्वे᳚ व्र॒ता द॑दन्ते ददन्ते व्र॒ता विश्वे॒ विश्वे᳚ व्र॒ता द॑दन्ते । \newline
14. व्र॒ता द॑दन्ते ददन्ते व्र॒ता व्र॒ता द॑दन्ते अ॒ग्ने र॒ग्नेर् द॑दन्ते व्र॒ता व्र॒ता द॑दन्ते अ॒ग्नेः । \newline
15. द॒द॒न्ते॒ अ॒ग्ने र॒ग्नेर् द॑दन्ते ददन्ते अ॒ग्नेः । \newline
16. अ॒ग्नेरित्य॒ग्नेः । \newline
17. उ॒रु॒व्यच॑सो॒ धाम्ना॒ धाम्नो॑ रु॒व्यच॑स उरु॒व्यच॑सो॒ धाम्ना॒ पत्य॑मानाः॒ पत्य॑माना॒ धाम्नो॑ रु॒व्यच॑स उरु॒व्यच॑सो॒ धाम्ना॒ पत्य॑मानाः । \newline
18. उ॒रु॒व्यच॑स॒ इत्यु॑रु - व्यच॑सः । \newline
19. धाम्ना॒ पत्य॑मानाः॒ पत्य॑माना॒ धाम्ना॒ धाम्ना॒ पत्य॑मानाः । \newline
20. पत्य॑माना॒ इति॒ पत्य॑मानाः । \newline
21. ते अ॑स्यास्य॒ ते ते अ॑स्य॒ योष॑णे॒ योष॑णे अस्य॒ ते ते अ॑स्य॒ योष॑णे । \newline
22. ते इति॒ ते । \newline
23. अ॒स्य॒ योष॑णे॒ योष॑णे अस्यास्य॒ योष॑णे दि॒व्ये दि॒व्ये योष॑णे अस्यास्य॒ योष॑णे दि॒व्ये । \newline
24. योष॑णे दि॒व्ये दि॒व्ये योष॑णे॒ योष॑णे दि॒व्ये न न दि॒व्ये योष॑णे॒ योष॑णे दि॒व्ये न । \newline
25. योष॑णे॒ इति॒ योष॑णे । \newline
26. दि॒व्ये न न दि॒व्ये दि॒व्ये न योनौ॒ योनौ॒ न दि॒व्ये दि॒व्ये न योनौ᳚ । \newline
27. दि॒व्ये इति॑ दि॒व्ये । \newline
28. न योनौ॒ योनौ॒ न न योना॑ वु॒षासा॒नक्तो॒ षासा॒नक्ता॒ योनौ॒ न न योना॑ वु॒षासा॒नक्ता᳚ । \newline
29. योना॑ वु॒षासा॒नक्तो॒ षासा॒नक्ता॒ योनौ॒ योना॑ वु॒षासा॒नक्ता᳚ । \newline
30. उ॒षासा॒नक्तेत्यु॒षासा॒नक्ता᳚ । \newline
31. इ॒मं ॅय॒ज्ञ्ं ॅय॒ज्ञ् मि॒म मि॒मं ॅय॒ज्ञ् म॑वता मवतां ॅय॒ज्ञ् मि॒म मि॒मं ॅय॒ज्ञ् म॑वताम् । \newline
32. य॒ज्ञ् म॑वता मवतां ॅय॒ज्ञ्ं ॅय॒ज्ञ् म॑वता मद्ध्व॒र म॑द्ध्व॒र म॑वतां ॅय॒ज्ञ्ं ॅय॒ज्ञ् म॑वता मद्ध्व॒रम् । \newline
33. अ॒व॒ता॒ म॒द्ध्व॒र म॑द्ध्व॒र म॑वता मवता मद्ध्व॒रम् नो॑ नो अद्ध्व॒र म॑वता मवता मद्ध्व॒रम् नः॑ । \newline
34. अ॒द्ध्व॒रम् नो॑ नो अद्ध्व॒र म॑द्ध्व॒रम् नः॑ । \newline
35. न॒ इति॑ नः । \newline
36. दैव्या॑ होतारौ होतारौ॒ दैव्या॒ दैव्या॑ होतारा वू॒र्द्ध्व मू॒र्द्ध्वꣳ हो॑तारौ॒ दैव्या॒ दैव्या॑ होतारा वू॒र्द्ध्वम् । \newline
37. हो॒ता॒रा॒ वू॒र्द्ध्व मू॒र्द्ध्वꣳ हो॑तारौ होतारा वू॒र्द्ध्व म॑द्ध्व॒र म॑द्ध्व॒र मू॒र्द्ध्वꣳ हो॑तारौ होतारा वू॒र्द्ध्व म॑द्ध्व॒रम् । \newline
38. ऊ॒र्द्ध्व म॑द्ध्व॒र म॑द्ध्व॒र मू॒र्द्ध्व मू॒र्द्ध्व म॑द्ध्व॒रम् नो॑ नो अद्ध्व॒र मू॒र्द्ध्व मू॒र्द्ध्व म॑द्ध्व॒रम् नः॑ । \newline
39. अ॒द्ध्व॒रम् नो॑ नो अद्ध्व॒र म॑द्ध्व॒रम् नो॒ ऽग्ने र॒ग्नेर् नो॑ अद्ध्व॒र म॑द्ध्व॒रम् नो॒ ऽग्नेः । \newline
40. नो॒ ऽग्ने र॒ग्नेर् नो॑ नो॒ ऽग्नेर् जि॒ह्वाम् जि॒ह्वा म॒ग्नेर् नो॑ नो॒ ऽग्नेर् जि॒ह्वाम् । \newline
41. अ॒ग्नेर् जि॒ह्वाम् जि॒ह्वा म॒ग्ने र॒ग्नेर् जि॒ह्वा म॒भ्य॑भि जि॒ह्वा म॒ग्ने र॒ग्नेर् जि॒ह्वा म॒भि । \newline
42. जि॒ह्वा म॒भ्य॑भि जि॒ह्वाम् जि॒ह्वा म॒भि गृ॑णीतम् गृणीत म॒भि जि॒ह्वाम् जि॒ह्वा म॒भि गृ॑णीतम् । \newline
43. अ॒भि गृ॑णीतम् गृणीत म॒भ्य॑भि गृ॑णीतम् । \newline
44. गृ॒णी॒त॒मिति॑ गृणीतम् । \newline
45. कृ॒णु॒तम् नो॑ नः कृणु॒तम् कृ॑णु॒तम् नः॒ स्वि॑ष्टिꣳ॒॒ स्वि॑ष्टिम् नः कृणु॒तम् कृ॑णु॒तम् नः॒ स्वि॑ष्टिम् । \newline
46. नः॒ स्वि॑ष्टिꣳ॒॒ स्वि॑ष्टिम् नो नः॒ स्वि॑ष्टिम् । \newline
47. स्वि॑ष्टि॒मिति॒ सु - इ॒ष्टि॒म् । \newline
48. ति॒स्रो दे॒वीर् दे॒वी स्ति॒स्र स्ति॒स्रो दे॒वीर् ब॒र्॒.हिर् ब॒र्॒.हिर् दे॒वी स्ति॒स्र स्ति॒स्रो दे॒वीर् ब॒र्॒.हिः । \newline
49. दे॒वीर् ब॒र्॒.हिर् ब॒र्॒.हिर् दे॒वीर् दे॒वीर् ब॒र्॒.हिरा ब॒र्॒.हिर् दे॒वीर् दे॒वीर् ब॒र्॒.हिरा । \newline
50. ब॒र्॒.हिरा ब॒र्॒.हिर् ब॒र्॒.हि रेद मि॒द मा ब॒र्॒.हिर् ब॒र्॒.हि रेदम् । \newline
51. एद मि॒द मेदꣳ स॑दन्तु सदन्त्वि॒द मेदꣳ स॑दन्तु । \newline
52. इ॒दꣳ स॑दन्तु सदन् त्वि॒द मि॒दꣳ स॑द॒न् त्विडेडा॑ सदन् त्वि॒द मि॒दꣳ स॑द॒न् त्विडा᳚ । \newline
53. स॒द॒न् त्विडेडा॑ सदन्तु सद॒न् त्विडा॒ सर॑स्वती॒ सर॑स्व॒तीडा॑ सदन्तु सद॒न्त्विडा॒ सर॑स्वती । \newline
54. इडा॒ सर॑स्वती॒ सर॑स्व॒ तीडेडा॒ सर॑स्वती॒ भार॑ती॒ भार॑ती॒ सर॑स्व॒ तीडेडा॒ सर॑स्वती॒ भार॑ती । \newline
55. सर॑स्वती॒ भार॑ती॒ भार॑ती॒ सर॑स्वती॒ सर॑स्वती॒ भार॑ती । \newline
\pagebreak
\markright{ TS 4.1.8.3  \hfill https://www.vedavms.in \hfill}

\section{ TS 4.1.8.3 }

\textbf{TS 4.1.8.3 } \newline
\textbf{Samhita Paata} \newline

भार॑ती । म॒ही गृ॑णा॒ना ॥ तन्न॑स्तु॒रीप॒मद्भु॑तं पुरु॒क्षु त्वष्टा॑ सु॒वीरं᳚ । रा॒यस्पोषं॒ ॅवि ष्य॑तु॒ नाभि॑म॒स्मे ॥ वन॑स्प॒तेऽव॑ सृजा॒ ररा॑ण॒स्त्मना॑ दे॒वेषु॑ । अ॒ग्निर्.ह॒व्यꣳ श॑मि॒ता सू॑दयाति ॥ अग्ने॒ स्वाहा॑ कृणुहि जातवेद॒ इन्द्रा॑य ह॒व्यं । विश्वे॑ दे॒वा ह॒विरि॒दं जु॑षन्तां ॥ हि॒र॒ण्य॒ग॒र्भः सम॑वर्त॒ताग्रे॑ भू॒तस्य॑ जा॒तः पति॒रेक॑ आसीत् । स दा॑धार पृथि॒वीं द्या - [  ] \newline

\textbf{Pada Paata} \newline

भार॑ती ॥ म॒ही । गृ॒णा॒ना ॥ तत् । नः॒ । तु॒रीप᳚म् । अद्भु॑तम् । पु॒रु॒क्षु । त्वष्टा᳚ । सु॒वीर॒मिति॑ सु - वीर᳚म् ॥ रा॒यः । पोष᳚म् । वीति॑ । स्य॒तु॒ । नाभि᳚म् । अ॒स्मे इति॑ ॥ वन॑स्पते । अवेति॑ । सृ॒ज॒ । ररा॑णः । त्मना᳚ । दे॒वेषु॑ ॥ अ॒ग्निः । ह॒व्यम् । श॒मि॒ता । सू॒द॒या॒ति॒ ॥ अग्ने᳚ । स्वाहा᳚ । कृ॒णु॒हि॒ । जा॒त॒वे॒द॒ इति॑ जात - वे॒दः॒ । इन्द्रा॑य । ह॒व्यम् ॥ विश्वे᳚ । दे॒वाः । ह॒विः । इ॒दम् । जु॒ष॒न्ता॒म् ॥ हि॒र॒ण्य॒ग॒र्भ इति॑ हिरण्य-ग॒र्भः । समिति॑ । अ॒व॒र्त॒त॒ । अग्रे᳚ । भू॒तस्य॑ । जा॒तः । पतिः॑ । एकः॑ । आ॒सी॒त् ॥ सः । दा॒धा॒र॒ । पृ॒थि॒वीम् । द्याम् ।  \newline


\textbf{Krama Paata} \newline

भार॒तीति॒ भार॑ती ॥ म॒ही गृ॑णा॒ना । गृ॒णा॒नेति॑ गृणा॒ना ॥ तन्नः॑ । न॒स्तु॒रीप᳚म् । तु॒रीप॒मद्भु॑तम् । अद्भु॑तम् पुरु॒क्षु । पु॒रु॒क्षु त्वष्टा᳚ । त्वष्टा॑ सु॒वीर᳚म् । सु॒वीर॒मिति॑ सु - वीर᳚म् ॥ रा॒यस्पोष᳚म् । पोषं॒ ॅवि । वि ष्य॑तु । स्य॒तु॒ नाभि᳚म् । नाभि॑म॒स्मे । अ॒स्मे इत्य॒स्मे ॥ वन॑स्प॒तेऽव॑ । अव॑ सृज । सृ॒जा॒ ररा॑णः । ररा॑ण॒स्त्मना᳚ । त्मना॑ दे॒वेषु॑ । दे॒वेष्विति॑ दे॒वेषु॑ ॥ अ॒ग्निर्. ह॒व्यम् । ह॒व्यꣳ श॑मि॒ता । श॒मि॒ता सू॑दयाति । सू॒द॒या॒तीति॑ सूदयाति ॥ अग्ने॒ स्वाहा᳚ । स्वाहा॑ कृणुहि । कृ॒णु॒हि॒ जा॒त॒वे॒दः॒ । जा॒त॒वे॒द॒ इन्द्रा॑य । जा॒त॒वे॒द॒ इति॑ जात - वे॒दः॒ । इन्द्रा॑य ह॒व्यम् । ह॒व्यमिति॑ ह॒व्यम् ॥ विश्वे॑ दे॒वाः । दे॒वा ह॒विः । ह॒विरि॒दम् । इ॒दम् जु॑षन्ताम् । जु॒ष॒न्ता॒मिति॑ जुषन्ताम् ॥ हि॒र॒ण्य॒ग॒र्भः सम् । हि॒र॒ण्य॒ग॒र्भ इति॑ हिरण्य - ग॒र्भः । सम॑वर्तत । अ॒व॒र्त॒ताग्रे᳚ । अग्रे॑ भू॒तस्य॑ । भू॒तस्य॑ जा॒तः । जा॒तः पतिः॑ । पति॒रेकः॑ । एक॑ आसीत् । आ॒सी॒दित्या॑सीत् ॥ स दा॑धार । दा॒धा॒र॒ पृ॒थि॒वीम् । पृ॒थि॒वीम् द्याम् । द्यामु॒त \newline

\textbf{Jatai Paata} \newline

1. भार॒तीति॒ भार॑ती । \newline
2. म॒ही गृ॑णा॒ना गृ॑णा॒ना म॒ही म॒ही गृ॑णा॒ना । \newline
3. गृ॒णा॒नेति॑ गृणा॒ना । \newline
4. तन् नो॑ न॒ स्तत् तन् नः॑ । \newline
5. न॒ स्तु॒रीप॑म् तु॒रीप॑म् नो न स्तु॒रीप᳚म् । \newline
6. तु॒रीप॒ मद्भु॑त॒ मद्भु॑तम् तु॒रीप॑म् तु॒रीप॒ मद्भु॑तम् । \newline
7. अद्भु॑तम् पुरु॒क्षु पु॑रु॒ क्ष्वद्भु॑त॒ मद्भु॑तम् पुरु॒क्षु । \newline
8. पु॒रु॒क्षु त्वष्टा॒ त्वष्टा॑ पुरु॒क्षु पु॑रु॒क्षु त्वष्टा᳚ । \newline
9. त्वष्टा॑ सु॒वीरꣳ॑ सु॒वीर॒म् त्वष्टा॒ त्वष्टा॑ सु॒वीर᳚म् । \newline
10. सु॒वीर॒मिति॑ सु - वीर᳚म् । \newline
11. रा॒य स्पोष॒म् पोषꣳ॑ रा॒यो रा॒य स्पोष᳚म् । \newline
12. पोषं॒ ॅवि वि पोष॒म् पोषं॒ ॅवि । \newline
13. वि ष्य॑तु स्यतु॒ वि वि ष्य॑तु । \newline
14. स्य॒तु॒ नाभि॒म् नाभिꣳ॑ स्यतु स्यतु॒ नाभि᳚म् । \newline
15. नाभि॑ म॒स्मे अ॒स्मे नाभि॒म् नाभि॑ म॒स्मे । \newline
16. अ॒स्मे इत्य॒स्मे । \newline
17. वन॑स्प॒ते ऽवाव॒ वन॑स्पते॒ वन॑स्प॒ते ऽव॑ । \newline
18. अव॑ सृज सृ॒जा वाव॑ सृज । \newline
19. सृ॒जा॒ ररा॑णो॒ ररा॑णः सृज सृजा॒ ररा॑णः । \newline
20. ररा॑ण॒ स्त्मना॒ त्मना॒ ररा॑णो॒ ररा॑ण॒ स्त्मना᳚ । \newline
21. त्मना॑ दे॒वेषु॑ दे॒वेषु॒ त्मना॒ त्मना॑ दे॒वेषु॑ । \newline
22. दे॒वेष्विति॑ दे॒वेषु॑ । \newline
23. अ॒ग्निर्. ह॒व्यꣳ ह॒व्य म॒ग्नि र॒ग्निर्. ह॒व्यम् । \newline
24. ह॒व्यꣳ श॑मि॒ता श॑मि॒ता ह॒व्यꣳ ह॒व्यꣳ श॑मि॒ता । \newline
25. श॒मि॒ता सू॑दयाति सूदयाति शमि॒ता श॑मि॒ता सू॑दयाति । \newline
26. सू॒द॒या॒तीति॑ सूदयाति । \newline
27. अग्ने॒ स्वाहा॒ स्वाहा ऽग्ने ऽग्ने॒ स्वाहा᳚ । \newline
28. स्वाहा॑ कृणुहि कृणुहि॒ स्वाहा॒ स्वाहा॑ कृणुहि । \newline
29. कृ॒णु॒हि॒ जा॒त॒वे॒दो॒ जा॒त॒वे॒दः॒ कृ॒णु॒हि॒ कृ॒णु॒हि॒ जा॒त॒वे॒दः॒ । \newline
30. जा॒त॒वे॒द॒ इन्द्रा॒ येन्द्रा॑य जातवेदो जातवेद॒ इन्द्रा॑य । \newline
31. जा॒त॒वे॒द॒ इति॑ जात - वे॒दः॒ । \newline
32. इन्द्रा॑य ह॒व्यꣳ ह॒व्य मिन्द्रा॒ येन्द्रा॑य ह॒व्यम् । \newline
33. ह॒व्यमिति॑ ह॒व्यम् । \newline
34. विश्वे॑ दे॒वा दे॒वा विश्वे॒ विश्वे॑ दे॒वाः । \newline
35. दे॒वा ह॒विर्. ह॒विर् दे॒वा दे॒वा ह॒विः । \newline
36. ह॒वि रि॒द मि॒दꣳ ह॒विर्. ह॒वि रि॒दम् । \newline
37. इ॒दम् जु॑षन्ताम् जुषन्ता मि॒द मि॒दम् जु॑षन्ताम् । \newline
38. जु॒ष॒न्ता॒मिति॑ जुषन्ताम् । \newline
39. हि॒र॒ण्य॒ग॒र्भः सꣳ सꣳ हि॑रण्यग॒र्भो हि॑रण्यग॒र्भः सम् । \newline
40. हि॒र॒ण्य॒ग॒र्भ इति॑ हिरण्य - ग॒र्भः । \newline
41. स म॑वर्तता वर्तत॒ सꣳ स म॑वर्तत । \newline
42. अ॒व॒र्त॒ताग्रे॒ अग्रे॑ ऽवर्तता वर्त॒ताग्रे᳚ । \newline
43. अग्रे॑ भू॒तस्य॑ भू॒तस्याग्रे॒ अग्रे॑ भू॒तस्य॑ । \newline
44. भू॒तस्य॑ जा॒तो जा॒तो भू॒तस्य॑ भू॒तस्य॑ जा॒तः । \newline
45. जा॒तः पति॒ष् पति॑र् जा॒तो जा॒तः पतिः॑ । \newline
46. पति॒ रेक॒ एक॒ स्पति॒ष् पति॒ रेकः॑ । \newline
47. एक॑ आसी दासी॒ देक॒ एक॑ आसीत् । \newline
48. आ॒सी॒दित्या॑सीत् । \newline
49. स दा॑धार दाधार॒ स स दा॑धार । \newline
50. दा॒धा॒र॒ पृ॒थि॒वीम् पृ॑थि॒वीम् दा॑धार दाधार पृथि॒वीम् । \newline
51. पृ॒थि॒वीम् द्याम् द्याम् पृ॑थि॒वीम् पृ॑थि॒वीम् द्याम् । \newline
52. द्या मु॒तोत द्याम् द्या मु॒त । \newline

\textbf{Ghana Paata } \newline

1. भार॒तीति॒ भार॑ती । \newline
2. म॒ही गृ॑णा॒ना गृ॑णा॒ना म॒ही म॒ही गृ॑णा॒ना । \newline
3. गृ॒णा॒नेति॑ गृणा॒ना । \newline
4. तन् नो॑ न॒ स्तत् तन् न॑ स्तु॒रीप॑म् तु॒रीप॑न् न॒ स्तत् तन् न॑ स्तु॒रीप᳚म् । \newline
5. न॒ स्तु॒रीप॑म् तु॒रीप॑न् नो न स्तु॒रीप॒ मद्भु॑त॒ मद्भु॑तम् तु॒रीप॑म् नो न स्तु॒रीप॒ मद्भु॑तम् । \newline
6. तु॒रीप॒ मद्भु॑त॒ मद्भु॑तम् तु॒रीप॑म् तु॒रीप॒ मद्भु॑तम् पुरु॒क्षु पु॑रु॒ क्ष्वद्भु॑तम् तु॒रीप॑म् तु॒रीप॒ मद्भु॑तम् पुरु॒क्षु । \newline
7. अद्भु॑तम् पुरु॒क्षु पु॑रु॒ क्ष्वद्भु॑त॒ मद्भु॑तम् पुरु॒क्षु त्वष्टा॒ त्वष्टा॑ पुरु॒ क्ष्वद्भु॑त॒ मद्भु॑तम् पुरु॒क्षु त्वष्टा᳚ । \newline
8. पु॒रु॒क्षु त्वष्टा॒ त्वष्टा॑ पुरु॒क्षु पु॑रु॒क्षु त्वष्टा॑ सु॒वीरꣳ॑ सु॒वीर॒म् त्वष्टा॑ पुरु॒क्षु पु॑रु॒क्षु त्वष्टा॑ सु॒वीर᳚म् । \newline
9. त्वष्टा॑ सु॒वीरꣳ॑ सु॒वीर॒म् त्वष्टा॒ त्वष्टा॑ सु॒वीर᳚म् । \newline
10. सु॒वीर॒मिति॑ सु - वीर᳚म् । \newline
11. रा॒य स्पोष॒म् पोषꣳ॑ रा॒यो रा॒य स्पोषं॒ ॅवि वि पोषꣳ॑ रा॒यो रा॒य स्पोषं॒ ॅवि । \newline
12. पोषं॒ ॅवि वि पोष॒म् पोषं॒ ॅवि ष्य॑तु स्यतु॒ वि पोष॒म् पोषं॒ ॅवि ष्य॑तु । \newline
13. वि ष्य॑तु स्यतु॒ वि वि ष्य॑तु॒ नाभि॒न् नाभिꣳ॑ स्यतु॒ वि वि ष्य॑तु॒ नाभि᳚म् । \newline
14. स्य॒तु॒ नाभि॒न् नाभिꣳ॑ स्यतु स्यतु॒ नाभि॑ म॒स्मे अ॒स्मे नाभिꣳ॑ स्यतु स्यतु॒ नाभि॑ म॒स्मे । \newline
15. नाभि॑ म॒स्मे अ॒स्मे नाभि॒म् नाभि॑ म॒स्मे । \newline
16. अ॒स्मे इत्य॒स्मे । \newline
17. वन॑स्प॒ते ऽवाव॒ वन॑स्पते॒ वन॑स्प॒ते ऽव॑ सृज सृ॒जाव॒ वन॑स्पते॒ वन॑स्प॒ते ऽव॑ सृज । \newline
18. अव॑ सृज सृ॒जा वाव॑ सृजा॒ ररा॑णो॒ ररा॑णः सृ॒जा वाव॑ सृजा॒ ररा॑णः । \newline
19. सृ॒जा॒ ररा॑णो॒ ररा॑णः सृज सृजा॒ ररा॑ण॒ स्त्मना॒ त्मना॒ ररा॑णः सृज सृजा॒ ररा॑ण॒ स्त्मना᳚ । \newline
20. ररा॑ण॒ स्त्मना॒ त्मना॒ ररा॑णो॒ ररा॑ण॒ स्त्मना॑ दे॒वेषु॑ दे॒वेषु॒ त्मना॒ ररा॑णो॒ ररा॑ण॒ स्त्मना॑ दे॒वेषु॑ । \newline
21. त्मना॑ दे॒वेषु॑ दे॒वेषु॒ त्मना॒ त्मना॑ दे॒वेषु॑ । \newline
22. दे॒वेष्विति॑ दे॒वेषु॑ । \newline
23. अ॒ग्निर्. ह॒व्यꣳ ह॒व्य म॒ग्नि र॒ग्निर्. ह॒व्यꣳ श॑मि॒ता श॑मि॒ता ह॒व्य म॒ग्नि र॒ग्निर्. ह॒व्यꣳ श॑मि॒ता । \newline
24. ह॒व्यꣳ श॑मि॒ता श॑मि॒ता ह॒व्यꣳ ह॒व्यꣳ श॑मि॒ता सू॑दयाति सूदयाति शमि॒ता ह॒व्यꣳ ह॒व्यꣳ श॑मि॒ता सू॑दयाति । \newline
25. श॒मि॒ता सू॑दयाति सूदयाति शमि॒ता श॑मि॒ता सू॑दयाति । \newline
26. सू॒द॒या॒तीति॑ सूदयाति । \newline
27. अग्ने॒ स्वाहा॒ स्वाहा ऽग्ने ऽग्ने॒ स्वाहा॑ कृणुहि कृणुहि॒ स्वाहा ऽग्ने ऽग्ने॒ स्वाहा॑ कृणुहि । \newline
28. स्वाहा॑ कृणुहि कृणुहि॒ स्वाहा॒ स्वाहा॑ कृणुहि जातवेदो जातवेदः कृणुहि॒ स्वाहा॒ स्वाहा॑ कृणुहि जातवेदः । \newline
29. कृ॒णु॒हि॒ जा॒त॒वे॒दो॒ जा॒त॒वे॒दः॒ कृ॒णु॒हि॒ कृ॒णु॒हि॒ जा॒त॒वे॒द॒ इन्द्रा॒ येन्द्रा॑य जातवेदः कृणुहि कृणुहि जातवेद॒ इन्द्रा॑य । \newline
30. जा॒त॒वे॒द॒ इन्द्रा॒ येन्द्रा॑य जातवेदो जातवेद॒ इन्द्रा॑य ह॒व्यꣳ ह॒व्य मिन्द्रा॑य जातवेदो जातवेद॒ इन्द्रा॑य ह॒व्यम् । \newline
31. जा॒त॒वे॒द॒ इति॑ जात - वे॒दः॒ । \newline
32. इन्द्रा॑य ह॒व्यꣳ ह॒व्य मिन्द्रा॒ये न्द्रा॑य ह॒व्यम् । \newline
33. ह॒व्यमिति॑ ह॒व्यम् । \newline
34. विश्वे॑ दे॒वा दे॒वा विश्वे॒ विश्वे॑ दे॒वा ह॒विर्. ह॒विर् दे॒वा विश्वे॒ विश्वे॑ दे॒वा ह॒विः । \newline
35. दे॒वा ह॒विर्. ह॒विर् दे॒वा दे॒वा ह॒वि रि॒द मि॒दꣳ ह॒विर् दे॒वा दे॒वा ह॒वि रि॒दम् । \newline
36. ह॒वि रि॒द मि॒दꣳ ह॒विर्. ह॒वि रि॒दम् जु॑षन्ताम् जुषन्ता मि॒दꣳ ह॒विर्. ह॒वि रि॒दम् जु॑षन्ताम् । \newline
37. इ॒दम् जु॑षन्ताम् जुषन्ता मि॒द मि॒दम् जु॑षन्ताम् । \newline
38. जु॒ष॒न्ता॒मिति॑ जुषन्ताम् । \newline
39. हि॒र॒ण्य॒ग॒र्भः सꣳ सꣳ हि॑रण्यग॒र्भो हि॑रण्यग॒र्भः सम॑वर्तता वर्तत॒ सꣳ हि॑रण्यग॒र्भो हि॑रण्यग॒र्भः स म॑वर्तत । \newline
40. हि॒र॒ण्य॒ग॒र्भ इति॑ हिरण्य - ग॒र्भः । \newline
41. सम॑वर्तता वर्तत॒ सꣳ सम॑वर्त॒ताग्रे॒ अग्रे॑ ऽवर्तत॒ सꣳ स म॑वर्त॒ताग्रे᳚ । \newline
42. अ॒व॒र्त॒ताग्रे॒ अग्रे॑ ऽवर्तता वर्त॒ताग्रे॑ भू॒तस्य॑ भू॒तस्याग्रे॑ ऽवर्तता वर्त॒ताग्रे॑ भू॒तस्य॑ । \newline
43. अग्रे॑ भू॒तस्य॑ भू॒तस्याग्रे॒ अग्रे॑ भू॒तस्य॑ जा॒तो जा॒तो भू॒तस्याग्रे॒ अग्रे॑ भू॒तस्य॑ जा॒तः । \newline
44. भू॒तस्य॑ जा॒तो जा॒तो भू॒तस्य॑ भू॒तस्य॑ जा॒तः पति॒ष् पति॑र् जा॒तो भू॒तस्य॑ भू॒तस्य॑ जा॒तः पतिः॑ । \newline
45. जा॒तः पति॒ष् पति॑र् जा॒तो जा॒तः पति॒ रेक॒ एक॒ स्पति॑र् जा॒तो जा॒तः पति॒ रेकः॑ । \newline
46. पति॒ रेक॒ एक॒ स्पति॒ष् पति॒ रेक॑ आसी दासी॒ देक॒ स्पति॒ष् पति॒ रेक॑ आसीत् । \newline
47. एक॑ आसी दासी॒ देक॒ एक॑ आसीत् । \newline
48. आ॒सी॒दित्या॑सीत् । \newline
49. स दा॑धार दाधार॒ स स दा॑धार पृथि॒वीम् पृ॑थि॒वीम् दा॑धार॒ स स दा॑धार पृथि॒वीम् । \newline
50. दा॒धा॒र॒ पृ॒थि॒वीम् पृ॑थि॒वीम् दा॑धार दाधार पृथि॒वीम् द्याम् द्याम् पृ॑थि॒वीम् दा॑धार दाधार पृथि॒वीम् द्याम् । \newline
51. पृ॒थि॒वीम् द्याम् द्याम् पृ॑थि॒वीम् पृ॑थि॒वीम् द्या मु॒तोत द्याम् पृ॑थि॒वीम् पृ॑थि॒वीम् द्या मु॒त । \newline
52. द्या मु॒तोत द्याम् द्या मु॒तेमा मि॒मा मु॒त द्याम् द्या मु॒ते माम् । \newline
\pagebreak
\markright{ TS 4.1.8.4  \hfill https://www.vedavms.in \hfill}

\section{ TS 4.1.8.4 }

\textbf{TS 4.1.8.4 } \newline
\textbf{Samhita Paata} \newline

-मु॒तेमां कस्मै॑ दे॒वाय॑ ह॒विषा॑ विधेम ॥ यः प्रा॑ण॒तो नि॑मिष॒तो म॑हि॒त्वैक॒ इद्राजा॒ जग॑तो ब॒भूव॑ । य ईशे॑ अ॒स्य द्वि॒पद॒श्चतु॑ष्पदः᳡कस्मै॑ दे॒वाय॑ ह॒विषा॑ विधेम ॥ य आ᳚त्म॒दा ब॑ल॒दा यस्य॒ विश्व॑ उ॒पास॑ते प्र॒शिषं॒ ॅयस्य॑ दे॒वाः । यस्य॑ छा॒याऽमृतं॒ ॅयस्य॑ मृ॒त्युः कस्मै॑ दे॒वाय॑ ह॒विषा॑ विधेम ॥ यस्ये॒मे हि॒मव॑न्तो महि॒त्वा यस्य॑ समु॒द्रꣳ र॒सया॑ स॒हा - [  ] \newline

\textbf{Pada Paata} \newline

उ॒त । इ॒माम् । कस्मै᳚ । दे॒वाय॑ । ह॒विषा᳚ । वि॒धे॒म॒ ॥ यः । प्रा॒ण॒त इति॑ प्र - अ॒न॒तः । नि॒मि॒ष॒त इति॑ नि - मि॒ष॒तः । म॒हि॒त्वेति॑ महि - त्वा । एकः॑ । इत् । राजा᳚ । जग॑तः । ब॒भूव॑ ॥ यः । ईशे᳚ । अ॒स्य । द्वि॒पद॒ इति॑ द्वि - पदः॑ । चतु॑ष्पद॒ इति॒ चतुः॑ - प॒दः॒ । कस्मै᳚ । दे॒वाय॑ । ह॒विषा᳚ । वि॒धे॒म॒ ॥ यः । आ॒त्म॒दा इत्या᳚त्म-दाः । ब॒ल॒दा इति॑ बल-दाः । यस्य॑ । विश्वे᳚ । उ॒पास॑त॒ इत्यु॑प - आस॑ते । प्र॒शिष॒मिति॑ प्र - शिष᳚म् । यस्य॑ । दे॒वाः ॥ यस्य॑ । छा॒या । अ॒मृत᳚म् । यस्य॑ । मृ॒त्युः । कस्मै᳚ । दे॒वाय॑ । ह॒विषा᳚ । वि॒धे॒म॒ ॥ यस्य॑ । इ॒मे । हि॒मव॑न्त॒ इति॑ हि॒म - व॒न्तः॒ । म॒हि॒त्वेति॑ महि - त्वा । यस्य॑ । स॒मु॒द्रम् । र॒सया᳚ । स॒ह ।  \newline


\textbf{Krama Paata} \newline

उ॒तेमाम् । इ॒माम् कस्मै᳚ । कस्मै॑ दे॒वाय॑ । दे॒वाय॑ ह॒विषा᳚ । ह॒विषा॑ विधेम । वि॒धे॒मति॑ विधेम ॥ यः प्रा॑ण॒तः । प्रा॒ण॒तो नि॑मिष॒तः । प्रा॒ण॒त इति॑ प्र - अ॒न॒तः । नि॒मि॒ष॒तो म॑हि॒त्वा । नि॒मि॒ष॒त इति॑ नि - मि॒ष॒तः । म॒हि॒त्वैकः॑ । म॒हि॒त्वेति॑ महि - त्वा । एक॒ इत् । इद् राजा᳚ । राजा॒ जग॑तः । जग॑तो ब॒भूव॑ । ब॒भूवेति॑ ब॒भूव॑ ॥ य ईशे᳚ । ईशे॑ अ॒स्य । अ॒स्य द्वि॒पदः॑ । द्वि॒पद॒श्चतु॑ष्पदः । द्वि॒पद॒ इति॑ द्वि - पदः॑ । चतु॑ष्पदः॒ कस्मै᳚ । चतु॑ष्पद॒ इति॒ चतुः॑ - प॒दः॒ । कस्मै॑ दे॒वाय॑ । दे॒वाय॑ ह॒विषा᳚ । ह॒विषा॑ विधेम । वि॒धे॒मेति॑ विधेम ॥ य आ᳚त्म॒दाः । आ॒त्म॒दा ब॑ल॒दाः । आ॒त्म॒दा इत्या᳚त्म - दाः । ब॒ल॒दा यस्य॑ । ब॒ल॒दा इति॑ बल - दाः । यस्य॒ विश्वे᳚ । विश्व॑ उ॒पास॑ते । उ॒पास॑ते प्र॒शिष᳚म् । उ॒पास॑त॒ इत्यु॑प - आस॑ते । प्र॒शिषं॒ ॅयस्य॑ । प्र॒शिष॒मिति॑ प्र - शिष᳚म् । यस्य॑ दे॒वाः । दे॒वा इति॑ दे॒वाः ॥ यस्य॑ छा॒या । छा॒याऽमृत᳚म् । अ॒मृतं॒ ॅयस्य॑ । यस्य॑ मृ॒त्युः । मृ॒त्युः कस्मै᳚ । कस्मै॑ दे॒वाय॑ । दे॒वाय॑ ह॒विषा᳚ । ह॒विषा॑ विधेम । वि॒धे॒मेति॑ विधेम ॥ यस्ये॒मे । इ॒मे हि॒मव॑न्तः । हि॒मव॑न्तो महि॒त्वा । हि॒मव॑न्त॒ इति॑ हि॒म - व॒न्तः॒ । म॒हि॒त्वा यस्य॑ । म॒हि॒त्वेति॑ महि - त्वा । यस्य॑ समु॒द्रम् । स॒मु॒द्रꣳ र॒सया᳚ । र॒सया॑ स॒ह । स॒हाहुः \newline

\textbf{Jatai Paata} \newline

1. उ॒ते मा मि॒मा मु॒तोते माम् । \newline
2. इ॒माम् कस्मै॒ कस्मा॑ इ॒मा मि॒माम् कस्मै᳚ । \newline
3. कस्मै॑ दे॒वाय॑ दे॒वाय॒ कस्मै॒ कस्मै॑ दे॒वाय॑ । \newline
4. दे॒वाय॑ ह॒विषा॑ ह॒विषा॑ दे॒वाय॑ दे॒वाय॑ ह॒विषा᳚ । \newline
5. ह॒विषा॑ विधेम विधेम ह॒विषा॑ ह॒विषा॑ विधेम । \newline
6. वि॒धे॒मेति॑ विधेम । \newline
7. यः प्रा॑ण॒तः प्रा॑ण॒तो यो यः प्रा॑ण॒तः । \newline
8. प्रा॒ण॒तो नि॑मिष॒तो नि॑मिष॒तः प्रा॑ण॒तः प्रा॑ण॒तो नि॑मिष॒तः । \newline
9. प्रा॒ण॒त इति॑ प्र - अ॒न॒तः । \newline
10. नि॒मि॒ष॒तो म॑हि॒त्वा म॑हि॒त्वा नि॑मिष॒तो नि॑मिष॒तो म॑हि॒त्वा । \newline
11. नि॒मि॒ष॒त इति॑ नि - मि॒ष॒तः । \newline
12. म॒हि॒त्वैक॒ एको॑ महि॒त्वा म॑हि॒त्वैकः॑ । \newline
13. म॒हि॒त्वेति॑ महि - त्वा । \newline
14. एक॒ इदिदेक॒ एक॒ इत् । \newline
15. इद् राजा॒ राजेदिद् राजा᳚ । \newline
16. राजा॒ जग॑तो॒ जग॑तो॒ राजा॒ राजा॒ जग॑तः । \newline
17. जग॑तो ब॒भूव॑ ब॒भूव॒ जग॑तो॒ जग॑तो ब॒भूव॑ । \newline
18. ब॒भूवेति॑ ब॒भूव॑ । \newline
19. य ईश॒ ईशे॒ यो य ईशे᳚ । \newline
20. ईशे॑ अ॒स्या स्येश॒ ईशे॑ अ॒स्य । \newline
21. अ॒स्य द्वि॒पदो᳚ द्वि॒पदो॑ अ॒स्यास्य द्वि॒पदः॑ । \newline
22. द्वि॒पद॒ श्चतु॑ष्पद॒ श्चतु॑ष्पदो द्वि॒पदो᳚ द्वि॒पद॒ श्चतु॑ष्पदः । \newline
23. द्वि॒पद॒ इति॑ द्वि - पदः॑ । \newline
24. चतु॑ष्पदः॒ कस्मै॒ कस्मै॒ चतु॑ष्पद॒ श्चतु॑ष्पदः॒ कस्मै᳚ । \newline
25. चतु॑ष्पद॒ इति॒ चतुः॑ - प॒दः॒ । \newline
26. कस्मै॑ दे॒वाय॑ दे॒वाय॒ कस्मै॒ कस्मै॑ दे॒वाय॑ । \newline
27. दे॒वाय॑ ह॒विषा॑ ह॒विषा॑ दे॒वाय॑ दे॒वाय॑ ह॒विषा᳚ । \newline
28. ह॒विषा॑ विधेम विधेम ह॒विषा॑ ह॒विषा॑ विधेम । \newline
29. वि॒धे॒मेति॑ विधेम । \newline
30. य आ᳚त्म॒दा आ᳚त्म॒दा यो य आ᳚त्म॒दाः । \newline
31. आ॒त्म॒दा ब॑ल॒दा ब॑ल॒दा आ᳚त्म॒दा आ᳚त्म॒दा ब॑ल॒दाः । \newline
32. आ॒त्म॒दा इत्या᳚त्म - दाः । \newline
33. ब॒ल॒दा यस्य॒ यस्य॑ बल॒दा ब॑ल॒दा यस्य॑ । \newline
34. ब॒ल॒दा इति॑ बल - दाः । \newline
35. यस्य॒ विश्वे॒ विश्वे॒ यस्य॒ यस्य॒ विश्वे᳚ । \newline
36. विश्व॑ उ॒पास॑त उ॒पास॑ते॒ विश्वे॒ विश्व॑ उ॒पास॑ते । \newline
37. उ॒पास॑ते प्र॒शिष॑म् प्र॒शिष॑ मु॒पास॑त उ॒पास॑ते प्र॒शिष᳚म् । \newline
38. उ॒पास॑त॒ इत्यु॑प - आस॑ते । \newline
39. प्र॒शिषं॒ ॅयस्य॒ यस्य॑ प्र॒शिष॑म् प्र॒शिषं॒ ॅयस्य॑ । \newline
40. प्र॒शिष॒मिति॑ प्र - शिष᳚म् । \newline
41. यस्य॑ दे॒वा दे॒वा यस्य॒ यस्य॑ दे॒वाः । \newline
42. दे॒वा इति॑ दे॒वाः । \newline
43. यस्य॑ छा॒या छा॒या यस्य॒ यस्य॑ छा॒या । \newline
44. छा॒या ऽमृत॑ म॒मृत॑म् छा॒या छा॒या ऽमृत᳚म् । \newline
45. अ॒मृतं॒ ॅयस्य॒ यस्या॒ मृत॑ म॒मृतं॒ ॅयस्य॑ । \newline
46. यस्य॑ मृ॒त्युर् मृ॒त्युर् यस्य॒ यस्य॑ मृ॒त्युः । \newline
47. मृ॒त्युः कस्मै॒ कस्मै॑ मृ॒त्युर् मृ॒त्युः कस्मै᳚ । \newline
48. कस्मै॑ दे॒वाय॑ दे॒वाय॒ कस्मै॒ कस्मै॑ दे॒वाय॑ । \newline
49. दे॒वाय॑ ह॒विषा॑ ह॒विषा॑ दे॒वाय॑ दे॒वाय॑ ह॒विषा᳚ । \newline
50. ह॒विषा॑ विधेम विधेम ह॒विषा॑ ह॒विषा॑ विधेम । \newline
51. वि॒धे॒मेति॑ विधेम । \newline
52. यस्ये॒म इ॒मे यस्य॒ यस्ये॒मे । \newline
53. इ॒मे हि॒मव॑न्तो हि॒मव॑न्त इ॒म इ॒मे हि॒मव॑न्तः । \newline
54. हि॒मव॑न्तो महि॒त्वा म॑हि॒त्वा हि॒मव॑न्तो हि॒मव॑न्तो महि॒त्वा । \newline
55. हि॒मव॑न्त॒ इति॑ हि॒म - व॒न्तः॒ । \newline
56. म॒हि॒त्वा यस्य॒ यस्य॑ महि॒त्वा म॑हि॒त्वा यस्य॑ । \newline
57. म॒हि॒त्वेति॑ महि - त्वा । \newline
58. यस्य॑ समु॒द्रꣳ स॑मु॒द्रं ॅयस्य॒ यस्य॑ समु॒द्रम् । \newline
59. स॒मु॒द्रꣳ र॒सया॑ र॒सया॑ समु॒द्रꣳ स॑मु॒द्रꣳ र॒सया᳚ । \newline
60. र॒सया॑ स॒ह स॒ह र॒सया॑ र॒सया॑ स॒ह । \newline
61. स॒हाहु रा॒हुः स॒ह स॒हाहुः । \newline

\textbf{Ghana Paata } \newline

1. उ॒तेमा मि॒मा मु॒तोतेमाम् कस्मै॒ कस्मा॑ इ॒मा मु॒तोते माम् कस्मै᳚ । \newline
2. इ॒माम् कस्मै॒ कस्मा॑ इ॒मा मि॒माम् कस्मै॑ दे॒वाय॑ दे॒वाय॒ कस्मा॑ इ॒मा मि॒माम् कस्मै॑ दे॒वाय॑ । \newline
3. कस्मै॑ दे॒वाय॑ दे॒वाय॒ कस्मै॒ कस्मै॑ दे॒वाय॑ ह॒विषा॑ ह॒विषा॑ दे॒वाय॒ कस्मै॒ कस्मै॑ दे॒वाय॑ ह॒विषा᳚ । \newline
4. दे॒वाय॑ ह॒विषा॑ ह॒विषा॑ दे॒वाय॑ दे॒वाय॑ ह॒विषा॑ विधेम विधेम ह॒विषा॑ दे॒वाय॑ दे॒वाय॑ ह॒विषा॑ विधेम । \newline
5. ह॒विषा॑ विधेम विधेम ह॒विषा॑ ह॒विषा॑ विधेम । \newline
6. वि॒धे॒मेति॑ विधेम । \newline
7. यः प्रा॑ण॒तः प्रा॑ण॒तो यो यः प्रा॑ण॒तो नि॑मिष॒तो नि॑मिष॒तः प्रा॑ण॒तो यो यः प्रा॑ण॒तो नि॑मिष॒तः । \newline
8. प्रा॒ण॒तो नि॑मिष॒तो नि॑मिष॒तः प्रा॑ण॒तः प्रा॑ण॒तो नि॑मिष॒तो म॑हि॒त्वा म॑हि॒त्वा नि॑मिष॒तः प्रा॑ण॒तः प्रा॑ण॒तो नि॑मिष॒तो म॑हि॒त्वा । \newline
9. प्रा॒ण॒त इति॑ प्र - अ॒न॒तः । \newline
10. नि॒मि॒ष॒तो म॑हि॒त्वा म॑हि॒त्वा नि॑मिष॒तो नि॑मिष॒तो म॑हि॒ त्वैक॒ एको॑ महि॒त्वा नि॑मिष॒तो नि॑मिष॒तो म॑हि॒ त्वैकः॑ । \newline
11. नि॒मि॒ष॒त इति॑ नि - मि॒ष॒तः । \newline
12. म॒हि॒ त्वैक॒ एको॑ महि॒त्वा म॑हि॒ त्वैक॒ इदि देको॑ महि॒त्वा म॑हि॒ त्वैक॒ इत् । \newline
13. म॒हि॒त्वेति॑ महि - त्वा । \newline
14. एक॒ इदि देक॒ एक॒ इद् राजा॒ राजे देक॒ एक॒ इद् राजा᳚ । \newline
15. इद् राजा॒ राजे दिद् राजा॒ जग॑तो॒ जग॑तो॒ राजे दिद् राजा॒ जग॑तः । \newline
16. राजा॒ जग॑तो॒ जग॑तो॒ राजा॒ राजा॒ जग॑तो ब॒भूव॑ ब॒भूव॒ जग॑तो॒ राजा॒ राजा॒ जग॑तो ब॒भूव॑ । \newline
17. जग॑तो ब॒भूव॑ ब॒भूव॒ जग॑तो॒ जग॑तो ब॒भूव॑ । \newline
18. ब॒भूवेति॑ ब॒भूव॑ । \newline
19. य ईश॒ ईशे॒ यो य ईशे॑ अ॒स्या स्येशे॒ यो य ईशे॑ अ॒स्य । \newline
20. ईशे॑ अ॒स्या स्येश॒ ईशे॑ अ॒स्य द्वि॒पदो᳚ द्वि॒पदो॑ अ॒स्येश॒ ईशे॑ अ॒स्य द्वि॒पदः॑ । \newline
21. अ॒स्य द्वि॒पदो᳚ द्वि॒पदो॑ अ॒स्यास्य द्वि॒पद॒ श्चतु॑ष्पद॒ श्चतु॑ष्पदो द्वि॒पदो॑ अ॒स्यास्य द्वि॒पद॒ श्चतु॑ष्पदः । \newline
22. द्वि॒पद॒ श्चतु॑ष्पद॒ श्चतु॑ष्पदो द्वि॒पदो᳚ द्वि॒पद॒ श्चतु॑ष्पदः॒ कस्मै॒ कस्मै॒ चतु॑ष्पदो द्वि॒पदो᳚ द्वि॒पद॒ श्चतु॑ष्पदः॒ कस्मै᳚ । \newline
23. द्वि॒पद॒ इति॑ द्वि - पदः॑ । \newline
24. चतु॑ष्पदः॒ कस्मै॒ कस्मै॒ चतु॑ष्पद॒ श्चतु॑ष्पदः॒ कस्मै॑ दे॒वाय॑ दे॒वाय॒ कस्मै॒ चतु॑ष्पद॒ श्चतु॑ष्पदः॒ कस्मै॑ दे॒वाय॑ । \newline
25. चतु॑ष्पद॒ इति॒ चतुः॑ - प॒दः॒ । \newline
26. कस्मै॑ दे॒वाय॑ दे॒वाय॒ कस्मै॒ कस्मै॑ दे॒वाय॑ ह॒विषा॑ ह॒विषा॑ दे॒वाय॒ कस्मै॒ कस्मै॑ दे॒वाय॑ ह॒विषा᳚ । \newline
27. दे॒वाय॑ ह॒विषा॑ ह॒विषा॑ दे॒वाय॑ दे॒वाय॑ ह॒विषा॑ विधेम विधेम ह॒विषा॑ दे॒वाय॑ दे॒वाय॑ ह॒विषा॑ विधेम । \newline
28. ह॒विषा॑ विधेम विधेम ह॒विषा॑ ह॒विषा॑ विधेम । \newline
29. वि॒धे॒मेति॑ विधेम । \newline
30. य आ᳚त्म॒दा आ᳚त्म॒दा यो य आ᳚त्म॒दा ब॑ल॒दा ब॑ल॒दा आ᳚त्म॒दा यो य आ᳚त्म॒दा ब॑ल॒दाः । \newline
31. आ॒त्म॒दा ब॑ल॒दा ब॑ल॒दा आ᳚त्म॒दा आ᳚त्म॒दा ब॑ल॒दा यस्य॒ यस्य॑ बल॒दा आ᳚त्म॒दा आ᳚त्म॒दा ब॑ल॒दा यस्य॑ । \newline
32. आ॒त्म॒दा इत्या᳚त्म - दाः । \newline
33. ब॒ल॒दा यस्य॒ यस्य॑ बल॒दा ब॑ल॒दा यस्य॒ विश्वे॒ विश्वे॒ यस्य॑ बल॒दा ब॑ल॒दा यस्य॒ विश्वे᳚ । \newline
34. ब॒ल॒दा इति॑ बल - दाः । \newline
35. यस्य॒ विश्वे॒ विश्वे॒ यस्य॒ यस्य॒ विश्व॑ उ॒पास॑त उ॒पास॑ते॒ विश्वे॒ यस्य॒ यस्य॒ विश्व॑ उ॒पास॑ते । \newline
36. विश्व॑ उ॒पास॑त उ॒पास॑ते॒ विश्वे॒ विश्व॑ उ॒पास॑ते प्र॒शिष॑म् प्र॒शिष॑ मु॒पास॑ते॒ विश्वे॒ विश्व॑ उ॒पास॑ते प्र॒शिष᳚म् । \newline
37. उ॒पास॑ते प्र॒शिष॑म् प्र॒शिष॑ मु॒पास॑त उ॒पास॑ते प्र॒शिषं॒ ॅयस्य॒ यस्य॑ प्र॒शिष॑ मु॒पास॑त उ॒पास॑ते प्र॒शिषं॒ ॅयस्य॑ । \newline
38. उ॒पास॑त॒ इत्यु॑प - आस॑ते । \newline
39. प्र॒शिषं॒ ॅयस्य॒ यस्य॑ प्र॒शिष॑म् प्र॒शिषं॒ ॅयस्य॑ दे॒वा दे॒वा यस्य॑ प्र॒शिष॑म् प्र॒शिषं॒ ॅयस्य॑ दे॒वाः । \newline
40. प्र॒शिष॒मिति॑ प्र - शिष᳚म् । \newline
41. यस्य॑ दे॒वा दे॒वा यस्य॒ यस्य॑ दे॒वाः । \newline
42. दे॒वा इति॑ दे॒वाः । \newline
43. यस्य॑ छा॒या छा॒या यस्य॒ यस्य॑ छा॒या ऽमृत॑ म॒मृत॑म् छा॒या यस्य॒ यस्य॑ छा॒या ऽमृत᳚म् । \newline
44. छा॒या ऽमृत॑ म॒मृत॑म् छा॒या छा॒या ऽमृतं॒ ॅयस्य॒ यस्या॒मृत॑म् छा॒या छा॒या ऽमृतं॒ ॅयस्य॑ । \newline
45. अ॒मृतं॒ ॅयस्य॒ यस्या॒मृत॑ म॒मृतं॒ ॅयस्य॑ मृ॒त्युर् मृ॒त्युर् यस्या॒मृत॑ म॒मृतं॒ ॅयस्य॑ मृ॒त्युः । \newline
46. यस्य॑ मृ॒त्युर् मृ॒त्युर् यस्य॒ यस्य॑ मृ॒त्युः कस्मै॒ कस्मै॑ मृ॒त्युर् यस्य॒ यस्य॑ मृ॒त्युः कस्मै᳚ । \newline
47. मृ॒त्युः कस्मै॒ कस्मै॑ मृ॒त्युर् मृ॒त्युः कस्मै॑ दे॒वाय॑ दे॒वाय॒ कस्मै॑ मृ॒त्युर् मृ॒त्युः कस्मै॑ दे॒वाय॑ । \newline
48. कस्मै॑ दे॒वाय॑ दे॒वाय॒ कस्मै॒ कस्मै॑ दे॒वाय॑ ह॒विषा॑ ह॒विषा॑ दे॒वाय॒ कस्मै॒ कस्मै॑ दे॒वाय॑ ह॒विषा᳚ । \newline
49. दे॒वाय॑ ह॒विषा॑ ह॒विषा॑ दे॒वाय॑ दे॒वाय॑ ह॒विषा॑ विधेम विधेम ह॒विषा॑ दे॒वाय॑ दे॒वाय॑ ह॒विषा॑ विधेम । \newline
50. ह॒विषा॑ विधेम विधेम ह॒विषा॑ ह॒विषा॑ विधेम । \newline
51. वि॒धे॒मेति॑ विधेम । \newline
52. यस्ये॒म इ॒मे यस्य॒ यस्ये॒मे हि॒मव॑न्तो हि॒मव॑न्त इ॒मे यस्य॒ यस्ये॒मे हि॒मव॑न्तः । \newline
53. इ॒मे हि॒मव॑न्तो हि॒मव॑न्त इ॒म इ॒मे हि॒मव॑न्तो महि॒त्वा म॑हि॒त्वा हि॒मव॑न्त इ॒म इ॒मे हि॒मव॑न्तो महि॒त्वा । \newline
54. हि॒मव॑न्तो महि॒त्वा म॑हि॒त्वा हि॒मव॑न्तो हि॒मव॑न्तो महि॒त्वा यस्य॒ यस्य॑ महि॒त्वा हि॒मव॑न्तो हि॒मव॑न्तो महि॒त्वा यस्य॑ । \newline
55. हि॒मव॑न्त॒ इति॑ हि॒म - व॒न्तः॒ । \newline
56. म॒हि॒त्वा यस्य॒ यस्य॑ महि॒त्वा म॑हि॒त्वा यस्य॑ समु॒द्रꣳ स॑मु॒द्रं ॅयस्य॑ महि॒त्वा म॑हि॒त्वा यस्य॑ समु॒द्रम् । \newline
57. म॒हि॒त्वेति॑ महि - त्वा । \newline
58. यस्य॑ समु॒द्रꣳ स॑मु॒द्रं ॅयस्य॒ यस्य॑ समु॒द्रꣳ र॒सया॑ र॒सया॑ समु॒द्रं ॅयस्य॒ यस्य॑ समु॒द्रꣳ र॒सया᳚ । \newline
59. स॒मु॒द्रꣳ र॒सया॑ र॒सया॑ समु॒द्रꣳ स॑मु॒द्रꣳ र॒सया॑ स॒ह स॒ह र॒सया॑ समु॒द्रꣳ स॑मु॒द्रꣳ र॒सया॑ स॒ह । \newline
60. र॒सया॑ स॒ह स॒ह र॒सया॑ र॒सया॑ स॒हाहु रा॒हुः स॒ह र॒सया॑ र॒सया॑ स॒हाहुः । \newline
61. स॒हाहु रा॒हुः स॒ह स॒हाहुः । \newline
\pagebreak
\markright{ TS 4.1.8.5  \hfill https://www.vedavms.in \hfill}

\section{ TS 4.1.8.5 }

\textbf{TS 4.1.8.5 } \newline
\textbf{Samhita Paata} \newline

-ऽऽहुः । यस्ये॒माः प्र॒दिशो॒ यस्य॑ बा॒हू कस्मै॑ दे॒वाय॑ ह॒विषा॑ विधेम ॥ यं क्रन्द॑सी॒ अव॑सा तस्तभा॒ने अ॒भ्यैक्षे॑तां॒ मन॑सा॒ रेज॑माने ।यत्राधि॒ सूर॒ उदि॑तौ॒ व्येति॒ कस्मै॑ दे॒वाय॑ ह॒विषा॑ विधेम ॥येन॒ द्यौरु॒ग्रा पृ॑थि॒वी च॑ दृ॒ढे येन॒ सु॒वः॑ स्तभि॒तं ॅयेन॒ नाकः॑ ।यो अ॒न्तरि॑क्षे॒ रज॑सो वि॒मानः᳡कस्मै॑ दे॒वाय॑ ह॒विषा॑ विधेम ॥ आपो॑ ह॒ यन्म॑ह॒ती र्विश्व॒ - [  ] \newline

\textbf{Pada Paata} \newline

आ॒हुः ॥ यस्य॑ । इ॒माः । प्र॒दिश॒ इति॑ प्र-दिशः॑ । यस्य॑ । बा॒हू इति॑ । कस्मै᳚ । दे॒वाय॑ । ह॒विषा᳚ । वि॒धे॒म॒ ॥ यम् । क्रन्द॑सी॒ इति॑ । अव॑सा । त॒स्त॒भा॒ने इति॑ । अ॒भ्यैक्षे॑ता॒मित्य॑भि - ऐक्षे॑तां । मन॑सा । रेज॑माने॒ इति॑ ॥ यत्र॑ । अधीति॑ । सूरः॑ । उदि॑ता॒वित्युत् - इ॒तौ॒ । व्येतीति॑ वि - एति॑ । कस्मै᳚ । दे॒वाय॑ । ह॒विषा᳚ । वि॒धे॒म॒ ॥ येन॑ । द्यौः । उ॒ग्रा । पृ॒थि॒वी । च॒ । दृ॒ढे इति॑ । येन॑ । सुवः॑ । स्त॒भि॒तम् । येन॑ । नाकः॑ ॥ यः । अ॒न्तरि॑क्षे । रज॑सः । वि॒मान॒ इति॑ वि-मानः॑ । कस्मै᳚ । दे॒वाय॑ । ह॒विषा᳚ । वि॒धे॒म॒ ॥ आपः॑ । ह॒ । यत् । म॒ह॒तीः । विश्व᳚म् ।  \newline


\textbf{Krama Paata} \newline

आ॒हुरित्या॒हुः ॥ यस्ये॒माः । इ॒माः प्र॒दिशः॑ । प्र॒दिशो॒ यस्य॑ । प्र॒दिश॒ इति॑ प्र - दिशः॑ । यस्य॑ बा॒हू । बा॒हू कस्मै᳚ । बा॒हू इति॑ बा॒हू । कस्मै॑ दे॒वाय॑ । दे॒वाय॑ ह॒विषा᳚ । ह॒विषा॑ विधेम । वि॒धे॒मेति॑ विधेम ॥ यम् क्रन्द॑सी । क्रन्द॑सी॒ अव॑सा । क्रन्द॑सी॒ इति॒ क्रन्द॑सी । अव॑सा तस्तभा॒ने । त॒स्त॒भा॒ने अ॒भ्यैक्षे॑ताम् । त॒स्त॒भा॒ने इति॑ तस्तभा॒ने । अ॒भ्यैक्षे॑ता॒म् मन॑सा । अ॒भ्यैक्षे॑ता॒मित्य॑भि - ऐक्षे॑ताम् । मन॑सा॒ रेज॑माने । रेज॑माने॒ इति॒ रेज॑माने ॥ यत्राऽधि॑ । अधि॒ सूरः॑ । सूर॒ उदि॑तौ । उदि॑तौ॒ व्येति॑ । उदि॑ता॒वित्युत् - इ॒तौ॒ । व्येति॒ कस्मै᳚ । व्येतीति॑ वि - एति॑ । कस्मै॑ दे॒वाय॑ । दे॒वाय॑ ह॒विषा᳚ । ह॒विषा॑ विधेम । वि॒धे॒मेति॑ विधेम ॥ येन॒ द्यौः । द्यौरु॒ग्रा । उ॒ग्रा पृ॑थि॒वी । पृ॒थि॒वी च॑ । च॒ दृ॒ढे । दृ॒ढे येन॑ । दृ॒ढे इति॑ दृ॒ढे । येन॒ सुवः॑ । सुवः॑ स्तभि॒तम् । स्त॒भि॒तं ॅयेन॑ । येन॒ नाकः॑ । नाक॒ इति॒ नाकः॑ ॥ यो अ॒न्तरि॑क्षे । अ॒न्तरि॑क्षे॒ रज॑सः । रज॑सो वि॒मानः॑ । वि॒मानः॒ कस्मै᳚ । वि॒मान॒ इति॑ वि - मानः॑ । कस्मै॑ दे॒वाय॑ । दे॒वाय॑ ह॒विषा᳚ । ह॒विषा॑ विधेम । वि॒धे॒मेति॑ विधेम ॥ आपो॑ ह । ह॒ यत् । यन् म॑ह॒तीः । म॒ह॒तीर् विश्व᳚म् ( ) । विश्व॒मायन्न्॑ \newline

\textbf{Jatai Paata} \newline

1. आ॒हुरित्या॒हुः । \newline
2. यस्ये॒ मा इ॒मा यस्य॒ यस्ये॒ माः । \newline
3. इ॒माः प्र॒दिशः॑ प्र॒दिश॑ इ॒मा इ॒माः प्र॒दिशः॑ । \newline
4. प्र॒दिशो॒ यस्य॒ यस्य॑ प्र॒दिशः॑ प्र॒दिशो॒ यस्य॑ । \newline
5. प्र॒दिश॒ इति॑ प्र - दिशः॑ । \newline
6. यस्य॑ बा॒हू बा॒हू यस्य॒ यस्य॑ बा॒हू । \newline
7. बा॒हू कस्मै॒ कस्मै॑ बा॒हू बा॒हू कस्मै᳚ । \newline
8. बा॒हू इति॑ बा॒हू । \newline
9. कस्मै॑ दे॒वाय॑ दे॒वाय॒ कस्मै॒ कस्मै॑ दे॒वाय॑ । \newline
10. दे॒वाय॑ ह॒विषा॑ ह॒विषा॑ दे॒वाय॑ दे॒वाय॑ ह॒विषा᳚ । \newline
11. ह॒विषा॑ विधेम विधेम ह॒विषा॑ ह॒विषा॑ विधेम । \newline
12. वि॒धे॒मेति॑ विधेम । \newline
13. यम् क्रन्द॑सी॒ क्रन्द॑सी॒ यं ॅयम् क्रन्द॑सी । \newline
14. क्रन्द॑सी॒ अव॒सा ऽव॑सा॒ क्रन्द॑सी॒ क्रन्द॑सी॒ अव॑सा । \newline
15. क्रन्द॑सी॒ इति॒ क्रन्द॑सी । \newline
16. अव॑सा तस्तभा॒ने त॑स्तभा॒ने अव॒सा ऽव॑सा तस्तभा॒ने । \newline
17. त॒स्त॒भा॒ने अ॒भ्यैक्षे॑ता म॒भ्यैक्षे॑तां तस्तभा॒ने त॑स्तभा॒ने अ॒भ्यैक्षे॑तां । \newline
18. त॒स्त॒भा॒ने इति॑ तस्तभा॒ने । \newline
19. अ॒भ्यैक्षे॑तां॒ मन॑सा॒ मन॑सा॒ ऽभ्यैक्षे॑ता म॒भ्यैक्षे॑तां॒ मन॑सा । \newline
20. अ॒भ्यैक्षे॑ता॒मित्य॑भि - ऐक्षे॑तां । \newline
21. मन॑सा॒ रेज॑माने॒ रेज॑माने॒ मन॑सा॒ मन॑सा॒ रेज॑माने । \newline
22. रेज॑माने॒ इति॒ रेज॑माने । \newline
23. यत्रा ऽध्यधि॒ यत्र॒ यत्रा ऽधि॑ । \newline
24. अधि॒ सूरः॒ सूरो॒ अध्यधि॒ सूरः॑ । \newline
25. सूर॒ उदि॑ता॒ वुदि॑तौ॒ सूरः॒ सूर॒ उदि॑तौ । \newline
26. उदि॑तौ॒ व्येति॒ व्येत्युदि॑ता॒ वुदि॑तौ॒ व्येति॑ । \newline
27. उदि॑ता॒वित्युत् - इ॒तौ॒ । \newline
28. व्येति॒ कस्मै॒ कस्मै॒ व्येति॒ व्येति॒ कस्मै᳚ । \newline
29. व्येतीति॑ वि - एति॑ । \newline
30. कस्मै॑ दे॒वाय॑ दे॒वाय॒ कस्मै॒ कस्मै॑ दे॒वाय॑ । \newline
31. दे॒वाय॑ ह॒विषा॑ ह॒विषा॑ दे॒वाय॑ दे॒वाय॑ ह॒विषा᳚ । \newline
32. ह॒विषा॑ विधेम विधेम ह॒विषा॑ ह॒विषा॑ विधेम । \newline
33. वि॒धे॒मेति॑ विधेम । \newline
34. येन॒ द्यौर् द्यौर् येन॒ येन॒ द्यौः । \newline
35. द्यौ रु॒ग्रोग्रा द्यौर् द्यौ रु॒ग्रा । \newline
36. उ॒ग्रा पृ॑थि॒वी पृ॑थि॒ व्यु॑ग्रोग्रा पृ॑थि॒वी । \newline
37. पृ॒थि॒वी च॑ च पृथि॒वी पृ॑थि॒वी च॑ । \newline
38. च॒ दृ॒ढे दृ॒ढे च॑ च दृ॒ढे । \newline
39. दृ॒ढे येन॒ येन॑ दृ॒ढे दृ॒ढे येन॑ । \newline
40. दृ॒ढे इति॑ दृ॒ढे । \newline
41. येन॒ सुवः॒ सुव॒र् येन॒ येन॒ सुवः॑ । \newline
42. सुवः॑ स्तभि॒तꣳ स्त॑भि॒तꣳ सुवः॒ सुवः॑ स्तभि॒तम् । \newline
43. स्त॒भि॒तं ॅयेन॒ येन॑ स्तभि॒तꣳ स्त॑भि॒तं ॅयेन॑ । \newline
44. येन॒ नाको॒ नाको॒ येन॒ येन॒ नाकः॑ । \newline
45. नाक॒ इति॒ नाकः॑ । \newline
46. यो अ॒न्तरि॑क्षे अ॒न्तरि॑क्षे॒ यो यो अ॒न्तरि॑क्षे । \newline
47. अ॒न्तरि॑क्षे॒ रज॑सो॒ रज॑सो अ॒न्तरि॑क्षे अ॒न्तरि॑क्षे॒ रज॑सः । \newline
48. रज॑सो वि॒मानो॑ वि॒मानो॒ रज॑सो॒ रज॑सो वि॒मानः॑ । \newline
49. वि॒मानः॒ कस्मै॒ कस्मै॑ वि॒मानो॑ वि॒मानः॒ कस्मै᳚ । \newline
50. वि॒मान॒ इति॑ वि - मानः॑ । \newline
51. कस्मै॑ दे॒वाय॑ दे॒वाय॒ कस्मै॒ कस्मै॑ दे॒वाय॑ । \newline
52. दे॒वाय॑ ह॒विषा॑ ह॒विषा॑ दे॒वाय॑ दे॒वाय॑ ह॒विषा᳚ । \newline
53. ह॒विषा॑ विधेम विधेम ह॒विषा॑ ह॒विषा॑ विधेम । \newline
54. वि॒धे॒मेति॑ विधेम । \newline
55. आपो॑ ह॒ हा प॒ आपो॑ ह । \newline
56. ह॒ यद् यद्ध॑ ह॒ यत् । \newline
57. यन् म॑ह॒तीर् म॑ह॒तीर् यद् यन् म॑ह॒तीः । \newline
58. म॒ह॒तीर् विश्वं॒ ॅविश्व॑म् मह॒तीर् म॑ह॒तीर् विश्व᳚म् । \newline
59. विश्व॒ माय॒न् नाय॒न्॒. विश्वं॒ ॅविश्व॒ मायन्न्॑ । \newline

\textbf{Ghana Paata } \newline

1. आ॒हुरित्या॒हुः । \newline
2. यस्ये॒मा इ॒मा यस्य॒ यस्ये॒माः प्र॒दिशः॑ प्र॒दिश॑ इ॒मा यस्य॒ यस्ये॒माः प्र॒दिशः॑ । \newline
3. इ॒माः प्र॒दिशः॑ प्र॒दिश॑ इ॒मा इ॒माः प्र॒दिशो॒ यस्य॒ यस्य॑ प्र॒दिश॑ इ॒मा इ॒माः प्र॒दिशो॒ यस्य॑ । \newline
4. प्र॒दिशो॒ यस्य॒ यस्य॑ प्र॒दिशः॑ प्र॒दिशो॒ यस्य॑ बा॒हू बा॒हू यस्य॑ प्र॒दिशः॑ प्र॒दिशो॒ यस्य॑ बा॒हू । \newline
5. प्र॒दिश॒ इति॑ प्र - दिशः॑ । \newline
6. यस्य॑ बा॒हू बा॒हू यस्य॒ यस्य॑ बा॒हू कस्मै॒ कस्मै॑ बा॒हू यस्य॒ यस्य॑ बा॒हू कस्मै᳚ । \newline
7. बा॒हू कस्मै॒ कस्मै॑ बा॒हू बा॒हू कस्मै॑ दे॒वाय॑ दे॒वाय॒ कस्मै॑ बा॒हू बा॒हू कस्मै॑ दे॒वाय॑ । \newline
8. बा॒हू इति॑ बा॒हू । \newline
9. कस्मै॑ दे॒वाय॑ दे॒वाय॒ कस्मै॒ कस्मै॑ दे॒वाय॑ ह॒विषा॑ ह॒विषा॑ दे॒वाय॒ कस्मै॒ कस्मै॑ दे॒वाय॑ ह॒विषा᳚ । \newline
10. दे॒वाय॑ ह॒विषा॑ ह॒विषा॑ दे॒वाय॑ दे॒वाय॑ ह॒विषा॑ विधेम विधेम ह॒विषा॑ दे॒वाय॑ दे॒वाय॑ ह॒विषा॑ विधेम । \newline
11. ह॒विषा॑ विधेम विधेम ह॒विषा॑ ह॒विषा॑ विधेम । \newline
12. वि॒धे॒मेति॑ विधेम । \newline
13. यम् क्रन्द॑सी॒ क्रन्द॑सी॒ यं ॅयम् क्रन्द॑सी॒ अव॒सा ऽव॑सा॒ क्रन्द॑सी॒ यं ॅयम् क्रन्द॑सी॒ अव॑सा । \newline
14. क्रन्द॑सी॒ अव॒सा ऽव॑सा॒ क्रन्द॑सी॒ क्रन्द॑सी॒ अव॑सा तस्तभा॒ने त॑स्तभा॒ने अव॑सा॒ क्रन्द॑सी॒ क्रन्द॑सी॒ अव॑सा तस्तभा॒ने । \newline
15. क्रन्द॑सी॒ इति॒ क्रन्द॑सी । \newline
16. अव॑सा तस्तभा॒ने त॑स्तभा॒ने अव॒सा ऽव॑सा तस्तभा॒ने अ॒भ्यैक्षे॑ता म॒भ्यैक्षे॑तां तस्तभा॒ने अव॒सा ऽव॑सा तस्तभा॒ने अ॒भ्यैक्षे॑तां । \newline
17. त॒स्त॒भा॒ने अ॒भ्यैक्षे॑ता म॒भ्यैक्षे॑तां तस्तभा॒ने त॑स्तभा॒ने अ॒भ्यैक्षे॑तां॒ मन॑सा॒ 
मन॑सा॒ ऽभ्यैक्षे॑तां तस्तभा॒ने त॑स्तभा॒ने अ॒भ्यैक्षे॑तां॒ मन॑सा । \newline
18. त॒स्त॒भा॒ने इति॑ तस्तभा॒ने । \newline
19. अ॒भ्यैक्षे॑तां॒ मन॑सा॒ मन॑सा॒ ऽभ्यैक्षे॑ता म॒भ्यैक्षे॑तां॒ मन॑सा॒ रेज॑माने॒ रेज॑माने॒ 
मन॑सा॒ ऽभ्यैक्षे॑ता म॒भ्यैक्षे॑तां॒ मन॑सा॒ रेज॑माने । \newline
20. अ॒भ्यैक्षे॑ता॒मित्य॑भि - ऐक्षे॑तां । \newline
21. मन॑सा॒ रेज॑माने॒ रेज॑माने॒ मन॑सा॒ मन॑सा॒ रेज॑माने । \newline
22. रेज॑माने॒ इति॒ रेज॑माने । \newline
23. यत्रा ऽध्यधि॒ यत्र॒ यत्रा ऽधि॒ सूरः॒ सूरो॒ अधि॒ यत्र॒ यत्रा ऽधि॒ सूरः॑ । \newline
24. अधि॒ सूरः॒ सूरो॒ अध्यधि॒ सूर॒ उदि॑ता॒ वुदि॑तौ॒ सूरो॒ अध्यधि॒ सूर॒ उदि॑तौ । \newline
25. सूर॒ उदि॑ता॒ वुदि॑तौ॒ सूरः॒ सूर॒ उदि॑तौ॒ व्येति॒ व्येत्युदि॑तौ॒ सूरः॒ सूर॒ उदि॑तौ॒ व्येति॑ । \newline
26. उदि॑तौ॒ व्येति॒ व्येत्युदि॑ता॒ वुदि॑तौ॒ व्येति॒ कस्मै॒ कस्मै॒ व्येत्युदि॑ता॒ वुदि॑तौ॒ व्येति॒ कस्मै᳚ । \newline
27. उदि॑ता॒वित्युत् - इ॒तौ॒ । \newline
28. व्येति॒ कस्मै॒ कस्मै॒ व्येति॒ व्येति॒ कस्मै॑ दे॒वाय॑ दे॒वाय॒ कस्मै॒ व्येति॒ व्येति॒ कस्मै॑ दे॒वाय॑ । \newline
29. व्येतीति॑ वि - एति॑ । \newline
30. कस्मै॑ दे॒वाय॑ दे॒वाय॒ कस्मै॒ कस्मै॑ दे॒वाय॑ ह॒विषा॑ ह॒विषा॑ दे॒वाय॒ कस्मै॒ कस्मै॑ दे॒वाय॑ ह॒विषा᳚ । \newline
31. दे॒वाय॑ ह॒विषा॑ ह॒विषा॑ दे॒वाय॑ दे॒वाय॑ ह॒विषा॑ विधेम विधेम ह॒विषा॑ दे॒वाय॑ दे॒वाय॑ ह॒विषा॑ विधेम । \newline
32. ह॒विषा॑ विधेम विधेम ह॒विषा॑ ह॒विषा॑ विधेम । \newline
33. वि॒धे॒मेति॑ विधेम । \newline
34. येन॒ द्यौर् द्यौर् येन॒ येन॒ द्यौ रु॒ग्रोग्रा द्यौर् येन॒ येन॒ द्यौ रु॒ग्रा । \newline
35. द्यौ रु॒ग्रोग्रा द्यौर् द्यौ रु॒ग्रा पृ॑थि॒वी पृ॑थि॒ व्यु॑ग्रा द्यौर् द्यौ रु॒ग्रा पृ॑थि॒वी । \newline
36. उ॒ग्रा पृ॑थि॒वी पृ॑थि॒ व्यु॑ग्रोग्रा पृ॑थि॒वी च॑ च पृथि॒ व्यु॑ग्रोग्रा पृ॑थि॒वी च॑ । \newline
37. पृ॒थि॒वी च॑ च पृथि॒वी पृ॑थि॒वी च॑ दृ॒ढे दृ॒ढे च॑ पृथि॒वी पृ॑थि॒वी च॑ दृ॒ढे । \newline
38. च॒ दृ॒ढे दृ॒ढे च॑ च दृ॒ढे येन॒ येन॑ दृ॒ढे च॑ च दृ॒ढे येन॑ । \newline
39. दृ॒ढे येन॒ येन॑ दृ॒ढे दृ॒ढे येन॒ सुवः॒ सुव॒र् येन॑ दृ॒ढे दृ॒ढे येन॒ सुवः॑ । \newline
40. दृ॒ढे इति॑ दृ॒ढे । \newline
41. येन॒ सुवः॒ सुव॒र् येन॒ येन॒ सुवः॑ स्तभि॒तꣳ स्त॑भि॒तꣳ सुव॒र् येन॒ येन॒ सुवः॑ स्तभि॒तम् । \newline
42. सुवः॑ स्तभि॒तꣳ स्त॑भि॒तꣳ सुवः॒ सुवः॑ स्तभि॒तं ॅयेन॒ येन॑ स्तभि॒तꣳ सुवः॒ सुवः॑ स्तभि॒तं ॅयेन॑ । \newline
43. स्त॒भि॒तं ॅयेन॒ येन॑ स्तभि॒तꣳ स्त॑भि॒तं ॅयेन॒ नाको॒ नाको॒ येन॑ स्तभि॒तꣳ स्त॑भि॒तं ॅयेन॒ नाकः॑ । \newline
44. येन॒ नाको॒ नाको॒ येन॒ येन॒ नाकः॑ । \newline
45. नाक॒ इति॒ नाकः॑ । \newline
46. यो अ॒न्तरि॑क्षे अ॒न्तरि॑क्षे॒ यो यो अ॒न्तरि॑क्षे॒ रज॑सो॒ रज॑सो अ॒न्तरि॑क्षे॒ यो यो अ॒न्तरि॑क्षे॒ रज॑सः । \newline
47. अ॒न्तरि॑क्षे॒ रज॑सो॒ रज॑सो अ॒न्तरि॑क्षे अ॒न्तरि॑क्षे॒ रज॑सो वि॒मानो॑ वि॒मानो॒ रज॑सो अ॒न्तरि॑क्षे अ॒न्तरि॑क्षे॒ रज॑सो वि॒मानः॑ । \newline
48. रज॑सो वि॒मानो॑ वि॒मानो॒ रज॑सो॒ रज॑सो वि॒मानः॒ कस्मै॒ कस्मै॑ वि॒मानो॒ रज॑सो॒ रज॑सो वि॒मानः॒ कस्मै᳚ । \newline
49. वि॒मानः॒ कस्मै॒ कस्मै॑ वि॒मानो॑ वि॒मानः॒ कस्मै॑ दे॒वाय॑ दे॒वाय॒ कस्मै॑ वि॒मानो॑ वि॒मानः॒ कस्मै॑ दे॒वाय॑ । \newline
50. वि॒मान॒ इति॑ वि - मानः॑ । \newline
51. कस्मै॑ दे॒वाय॑ दे॒वाय॒ कस्मै॒ कस्मै॑ दे॒वाय॑ ह॒विषा॑ ह॒विषा॑ दे॒वाय॒ कस्मै॒ कस्मै॑ दे॒वाय॑ ह॒विषा᳚ । \newline
52. दे॒वाय॑ ह॒विषा॑ ह॒विषा॑ दे॒वाय॑ दे॒वाय॑ ह॒विषा॑ विधेम विधेम ह॒विषा॑ दे॒वाय॑ दे॒वाय॑ ह॒विषा॑ विधेम । \newline
53. ह॒विषा॑ विधेम विधेम ह॒विषा॑ ह॒विषा॑ विधेम । \newline
54. वि॒धे॒मेति॑ विधेम । \newline
55. आपो॑ ह॒ हाप॒ आपो॑ ह॒ यद् य द्धाप॒ आपो॑ ह॒ यत् । \newline
56. ह॒ यद् यद्ध॑ ह॒ यन् म॑ह॒तीर् म॑ह॒तीर् यद्ध॑ ह॒ यन् म॑ह॒तीः । \newline
57. यन् म॑ह॒तीर् म॑ह॒तीर् यद् यन् म॑ह॒तीर् विश्वं॒ ॅविश्व॑म् मह॒तीर् यद् यन् म॑ह॒तीर् विश्व᳚म् । \newline
58. म॒ह॒तीर् विश्वं॒ ॅविश्व॑म् मह॒तीर् म॑ह॒तीर् विश्व॒ माय॒न् नाय॒न्॒. विश्व॑म् मह॒तीर् म॑ह॒तीर् विश्व॒ मायन्न्॑ । \newline
59. विश्व॒ माय॒न् नाय॒न्॒. विश्वं॒ ॅविश्व॒ माय॒न् दक्ष॒म् दक्ष॒ माय॒न्॒. विश्वं॒ ॅविश्व॒ माय॒न् दक्ष᳚म् । \newline
\pagebreak
\markright{ TS 4.1.8.6  \hfill https://www.vedavms.in \hfill}

\section{ TS 4.1.8.6 }

\textbf{TS 4.1.8.6 } \newline
\textbf{Samhita Paata} \newline

-माय॒न् दक्षं॒ दधा॑ना ज॒नय॑न्तीर॒ग्निं ।ततो॑ दे॒वानां॒ निर॑वर्त॒तासु॒रेकः᳡कस्मै॑ दे॒वाय॑ ह॒विषा॑ विधेम ॥यश्चि॒दापो॑ महि॒ना प॒र्यप॑श्य॒द्-दक्षं॒ दधा॑ना ज॒नय॑न्तीर॒ग्निं ।यो दे॒वेष्वधि॑ दे॒व एक॒ आसी॒त् कस्मै॑ दे॒वाय॑ ह॒विषा॑ विधेम ॥ \newline

\textbf{Pada Paata} \newline

आयन्न्॑ । दक्ष᳚म् । दधा॑नाः । ज॒नय॑न्तीः । अ॒ग्निम् ॥ ततः॑ । दे॒वाना᳚म् । निरिति॑ । अ॒व॒र्त॒त॒ । असुः॑ । एकः॑ । कस्मै᳚ । दे॒वाय॑ । ह॒विषा᳚ । वि॒धे॒म॒ ॥ यः । चि॒त् । आपः॑ । म॒हि॒ना । प॒र्यप॑श्य॒दिति॑ परि-अप॑श्यत् । दक्ष᳚म् । दधा॑नाः । ज॒नय॑न्तीः । अ॒ग्निम् ॥ यः । दे॒वेषु॑ । अधीति॑ । दे॒वः । एकः॑ । आसी᳚त् । कस्मै᳚ । दे॒वाय॑ । ह॒विषा᳚ । वि॒धे॒म॒ ॥  \newline


\textbf{Krama Paata} \newline

आय॒न् दक्ष᳚म् । दक्ष॒म् दधा॑नाः । दधा॑ना ज॒नय॑न्तीः । ज॒नय॑न्तीर॒ग्निम् । अ॒ग्निमित्य॒ग्निम् ॥ ततो॑ दे॒वाना᳚म् । दे॒वाना॒म् निः । निर॑वर्तत । अ॒व॒र्त॒तासुः॑ । असु॒रेकः॑ । एकः॒ कस्मै᳚ । कस्मै॑ दे॒वाय॑ । दे॒वाय॑ ह॒विषा᳚ । ह॒विषा॑ विधेम । वि॒धे॒मेति॑ विधेम ॥ यश्चि॑त् । चि॒दापः॑ । आपो॑ महि॒ना । म॒हि॒ना प॒र्यप॑श्यत् । प॒र्यप॑श्य॒द् दक्ष᳚म् । प॒र्यप॑श्य॒दिति॑ परि - अप॑श्यत् । दक्ष॒म् दधा॑नाः । दधा॑ना ज॒नय॑न्तीः । ज॒नय॑न्तीर॒ग्निम् । अ॒ग्निमित्य॒ग्निम् ॥ यो दे॒वेषु॑ । दे॒वेष्वधि॑ । अधि॑ दे॒वः । दे॒व एकः॑ । एक॒ आसी᳚त् । आसी॒त् कस्मै᳚ । कस्मै॑ दे॒वाय॑ । दे॒वाय॑ ह॒विषा᳚ । ह॒विषा॑ विधेम । वि॒धे॒मेति॑ विधेम । \newline

\textbf{Jatai Paata} \newline

1. आय॒न् दक्ष॒म् दक्ष॒ माय॒न् नाय॒न् दक्ष᳚म् । \newline
2. दक्ष॒म् दधा॑ना॒ दधा॑ना॒ दक्ष॒म् दक्ष॒म् दधा॑नाः । \newline
3. दधा॑ना ज॒नय॑न्तीर् ज॒नय॑न्ती॒र् दधा॑ना॒ दधा॑ना ज॒नय॑न्तीः । \newline
4. ज॒नय॑न्ती र॒ग्नि म॒ग्निम् ज॒नय॑न्तीर् ज॒नय॑न्ती र॒ग्निम् । \newline
5. अ॒ग्निमित्य॒ग्निम् । \newline
6. ततो॑ दे॒वाना᳚म् दे॒वाना॒म् तत॒ स्ततो॑ दे॒वाना᳚म् । \newline
7. दे॒वाना॒म् निर् णिर् दे॒वाना᳚म् दे॒वाना॒म् निः । \newline
8. निर॑वर्तता वर्तत॒ निर् णि र॑वर्तत । \newline
9. अ॒व॒र्त॒ता सु॒ रसु॑रवर्तता वर्त॒ता सुः॑ । \newline
10. असु॒ रेक॒ एको॒ असु॒ रसु॒ रेकः॑ । \newline
11. एकः॒ कस्मै॒ कस्मा॒ एक॒ एकः॒ कस्मै᳚ । \newline
12. कस्मै॑ दे॒वाय॑ दे॒वाय॒ कस्मै॒ कस्मै॑ दे॒वाय॑ । \newline
13. दे॒वाय॑ ह॒विषा॑ ह॒विषा॑ दे॒वाय॑ दे॒वाय॑ ह॒विषा᳚ । \newline
14. ह॒विषा॑ विधेम विधेम ह॒विषा॑ ह॒विषा॑ विधेम । \newline
15. वि॒धे॒मेति॑ विधेम । \newline
16. यश्चि॑च् चि॒द् यो यश्चि॑त् । \newline
17. चि॒दाप॒ आप॑ श्चिच् चि॒दापः॑ । \newline
18. आपो॑ महि॒ना म॑हि॒ना ऽऽप॒ आपो॑ महि॒ना । \newline
19. म॒हि॒ना प॒र्यप॑श्यत् प॒र्यप॑श्यन् महि॒ना म॑हि॒ना प॒र्यप॑श्यत् । \newline
20. प॒र्यप॑श्य॒द् दक्ष॒म् दक्ष॑म् प॒र्यप॑श्यत् प॒र्यप॑श्य॒द् दक्ष᳚म् । \newline
21. प॒र्यप॑श्य॒दिति॑ परि - अप॑श्यत् । \newline
22. दक्ष॒म् दधा॑ना॒ दधा॑ना॒ दक्ष॒म् दक्ष॒म् दधा॑नाः । \newline
23. दधा॑ना ज॒नय॑न्तीर् ज॒नय॑न्ती॒र् दधा॑ना॒ दधा॑ना ज॒नय॑न्तीः । \newline
24. ज॒नय॑न्ती र॒ग्नि म॒ग्निम् ज॒नय॑न्तीर् ज॒नय॑न्ती र॒ग्निम् । \newline
25. अ॒ग्निमित्य॒ग्निम् । \newline
26. यो दे॒वेषु॑ दे॒वेषु॒ यो यो दे॒वेषु॑ । \newline
27. दे॒वे ष्वध्यधि॑ दे॒वेषु॑ दे॒वे ष्वधि॑ । \newline
28. अधि॑ दे॒वो दे॒वो अध्यधि॑ दे॒वः । \newline
29. दे॒व एक॒ एको॑ दे॒वो दे॒व एकः॑ । \newline
30. एक॒ आसी॒ दासी॒ देक॒ एक॒ आसी᳚त् । \newline
31. आसी॒त् कस्मै॒ कस्मा॒ आसी॒ दासी॒त् कस्मै᳚ । \newline
32. कस्मै॑ दे॒वाय॑ दे॒वाय॒ कस्मै॒ कस्मै॑ दे॒वाय॑ । \newline
33. दे॒वाय॑ ह॒विषा॑ ह॒विषा॑ दे॒वाय॑ दे॒वाय॑ ह॒विषा᳚ । \newline
34. ह॒विषा॑ विधेम विधेम ह॒विषा॑ ह॒विषा॑ विधेम । \newline
35. वि॒धे॒मेति॑ विधेम । \newline

\textbf{Ghana Paata } \newline

1. आय॒न् दक्ष॒म् दक्ष॒ माय॒न् नाय॒न् दक्ष॒म् दधा॑ना॒ दधा॑ना॒ दक्ष॒ माय॒न् नाय॒न् दक्ष॒म् दधा॑नाः । \newline
2. दक्ष॒म् दधा॑ना॒ दधा॑ना॒ दक्ष॒म् दक्ष॒म् दधा॑ना ज॒नय॑न्तीर् ज॒नय॑न्ती॒र् दधा॑ना॒ दक्ष॒म् दक्ष॒म् दधा॑ना ज॒नय॑न्तीः । \newline
3. दधा॑ना ज॒नय॑न्तीर् ज॒नय॑न्ती॒र् दधा॑ना॒ दधा॑ना ज॒नय॑न्ती र॒ग्नि म॒ग्निम् ज॒नय॑न्ती॒र् दधा॑ना॒ दधा॑ना ज॒नय॑न्ती र॒ग्निम् । \newline
4. ज॒नय॑न्ती र॒ग्नि म॒ग्निम् ज॒नय॑न्तीर् ज॒नय॑न्ती र॒ग्निम् । \newline
5. अ॒ग्निमित्य॒ग्निम् । \newline
6. ततो॑ दे॒वाना᳚म् दे॒वाना॒म् तत॒ स्ततो॑ दे॒वाना॒म् निर् णिर् दे॒वाना॒म् तत॒ स्ततो॑ दे॒वाना॒म् निः । \newline
7. दे॒वाना॒म् निर् णिर् दे॒वाना᳚म् दे॒वाना॒म् निर॑वर्तता वर्तत॒ निर् दे॒वाना᳚म् दे॒वाना॒म् निर॑वर्तत । \newline
8. निर॑वर्तता वर्तत॒ निर् णिर॑वर्त॒ता सु॒रसु॑ रवर्तत॒ निर् णिर॑वर्त॒तासुः॑ । \newline
9. अ॒व॒र्त॒ता सु॒रसु॑ रवर्तता वर्त॒ता सु॒रेक॒ एको॒ असु॑ रवर्तता वर्त॒तासु॒ रेकः॑ । \newline
10. असु॒रेक॒ एको॒ असु॒ रसु॒ रेकः॒ कस्मै॒ कस्मा॒ एको॒ असु॒ रसु॒ रेकः॒ कस्मै᳚ । \newline
11. एकः॒ कस्मै॒ कस्मा॒ एक॒ एकः॒ कस्मै॑ दे॒वाय॑ दे॒वाय॒ कस्मा॒ एक॒ एकः॒ कस्मै॑ दे॒वाय॑ । \newline
12. कस्मै॑ दे॒वाय॑ दे॒वाय॒ कस्मै॒ कस्मै॑ दे॒वाय॑ ह॒विषा॑ ह॒विषा॑ दे॒वाय॒ कस्मै॒ कस्मै॑ दे॒वाय॑ ह॒विषा᳚ । \newline
13. दे॒वाय॑ ह॒विषा॑ ह॒विषा॑ दे॒वाय॑ दे॒वाय॑ ह॒विषा॑ विधेम विधेम ह॒विषा॑ दे॒वाय॑ दे॒वाय॑ ह॒विषा॑ विधेम । \newline
14. ह॒विषा॑ विधेम विधेम ह॒विषा॑ ह॒विषा॑ विधेम । \newline
15. वि॒धे॒मेति॑ विधेम । \newline
16. यश्चि॑च् चि॒द् यो यश्चि॒ दाप॒ आप॑ श्चि॒द् यो यश्चि॒ दापः॑ । \newline
17. चि॒दाप॒ आप॑ श्चिच् चि॒दापो॑ महि॒ना म॑हि॒ना ऽऽप॑श्चिच् चि॒दापो॑ महि॒ना । \newline
18. आपो॑ महि॒ना म॑हि॒ना ऽऽप॒ आपो॑ महि॒ना प॒र्यप॑श्यत् प॒र्यप॑श्यन् महि॒ना ऽऽप॒ आपो॑ महि॒ना प॒र्यप॑श्यत् । \newline
19. म॒हि॒ना प॒र्यप॑श्यत् प॒र्यप॑श्यन् महि॒ना म॑हि॒ना प॒र्यप॑श्य॒द् दक्ष॒म् दक्ष॑म् प॒र्यप॑श्यन् महि॒ना म॑हि॒ना प॒र्यप॑श्य॒द् दक्ष᳚म् । \newline
20. प॒र्यप॑श्य॒द् दक्ष॒म् दक्ष॑म् प॒र्यप॑श्यत् प॒र्यप॑श्य॒द् दक्ष॒म् दधा॑ना॒ दधा॑ना॒ दक्ष॑म् प॒र्यप॑श्यत् प॒र्यप॑श्य॒द् दक्ष॒म् दधा॑नाः । \newline
21. प॒र्यप॑श्य॒दिति॑ परि - अप॑श्यत् । \newline
22. दक्ष॒म् दधा॑ना॒ दधा॑ना॒ दक्ष॒म् दक्ष॒म् दधा॑ना ज॒नय॑न्तीर् ज॒नय॑न्ती॒र् दधा॑ना॒ दक्ष॒म् दक्ष॒म् दधा॑ना ज॒नय॑न्तीः । \newline
23. दधा॑ना ज॒नय॑न्तीर् ज॒नय॑न्ती॒र् दधा॑ना॒ दधा॑ना ज॒नय॑न्ती र॒ग्नि म॒ग्निम् ज॒नय॑न्ती॒र् दधा॑ना॒ दधा॑ना ज॒नय॑न्ती र॒ग्निम् । \newline
24. ज॒नय॑न्ती र॒ग्नि म॒ग्निम् ज॒नय॑न्तीर् ज॒नय॑न्ती र॒ग्निम् । \newline
25. अ॒ग्निमित्य॒ग्निम् । \newline
26. यो दे॒वेषु॑ दे॒वेषु॒ यो यो दे॒वे ष्वध्यधि॑ दे॒वेषु॒ यो यो दे॒वेष्वधि॑ । \newline
27. दे॒वे ष्वध्यधि॑ दे॒वेषु॑ दे॒वेष्वधि॑ दे॒वो दे॒वो अधि॑ दे॒वेषु॑ दे॒वेष्वधि॑ दे॒वः । \newline
28. अधि॑ दे॒वो दे॒वो अध्यधि॑ दे॒व एक॒ एको॑ दे॒वो अध्यधि॑ दे॒व एकः॑ । \newline
29. दे॒व एक॒ एको॑ दे॒वो दे॒व एक॒ आसी॒ दासी॒ देको॑ दे॒वो दे॒व एक॒ आसी᳚त् । \newline
30. एक॒ आसी॒ दासी॒ देक॒ एक॒ आसी॒त् कस्मै॒ कस्मा॒ आसी॒ देक॒ एक॒ आसी॒त् कस्मै᳚ । \newline
31. आसी॒त् कस्मै॒ कस्मा॒ आसी॒ दासी॒त् कस्मै॑ दे॒वाय॑ दे॒वाय॒ कस्मा॒ आसी॒ दासी॒त् कस्मै॑ दे॒वाय॑ । \newline
32. कस्मै॑ दे॒वाय॑ दे॒वाय॒ कस्मै॒ कस्मै॑ दे॒वाय॑ ह॒विषा॑ ह॒विषा॑ दे॒वाय॒ कस्मै॒ कस्मै॑ दे॒वाय॑ ह॒विषा᳚ । \newline
33. दे॒वाय॑ ह॒विषा॑ ह॒विषा॑ दे॒वाय॑ दे॒वाय॑ ह॒विषा॑ विधेम विधेम ह॒विषा॑ दे॒वाय॑ दे॒वाय॑ ह॒विषा॑ विधेम । \newline
34. ह॒विषा॑ विधेम विधेम ह॒विषा॑ ह॒विषा॑ विधेम । \newline
35. वि॒धे॒मेति॑ विधेम । \newline
\pagebreak
\markright{ TS 4.1.9.1  \hfill https://www.vedavms.in \hfill}

\section{ TS 4.1.9.1 }

\textbf{TS 4.1.9.1 } \newline
\textbf{Samhita Paata} \newline

आकू॑तिम॒ग्निं प्र॒युजꣳ॒॒ स्वाहा॒ मनो॑ मे॒धाम॒ग्निं प्र॒युजꣳ॒॒ स्वाहा॑ चि॒त्तं ॅविज्ञा॑तम॒ग्निं प्र॒युजꣳ॒॒ स्वाहा॑ वा॒चो विधृ॑तिम॒ग्निं प्र॒युजꣳ॒॒ स्वाहा᳚ प्र॒जाप॑तये॒ मन॑वे॒ स्वाहा॒ऽग्नये॑ वैश्वान॒राय॒ स्वाहा॒ विश्वे॑ दे॒वस्य॑ ने॒तुर्मर्तो॑ वृणीत स॒ख्यं ॅविश्वे॑ रा॒य इ॑षुद्ध्यसि द्यु॒म्नं ॅवृ॑णीत पु॒ष्यसे॒ स्वाहा॒ मा सु भि॑त्था॒ मा सु रि॑षो॒ दृꣳह॑स्व वी॒डय॑स्व॒ सु । अबं॑ धृष्णु वी॒रय॑स्वा॒ - [  ] \newline

\textbf{Pada Paata} \newline

आकू॑ति॒मित्या - कू॒ति॒म् । अ॒ग्निम् । प्र॒युज॒मिति॑ प्र - युज᳚म् । स्वाहा᳚ । मनः॑ । मे॒धाम् । अ॒ग्निम् । प्र॒युज॒मिति॑ प्र-युज᳚म् । स्वाहा᳚ । चि॒त्तम् । विज्ञा॑त॒मिति॑ वि - ज्ञा॒त॒म् । अ॒ग्निम् । प्र॒युज॒मिति॑ प्र - युज᳚म् । स्वाहा᳚ । वा॒चः । विधृ॑ति॒मिति॒ वि-धृ॒ति॒म् । अ॒ग्निम् । प्र॒युज॒मिति॑ प्र - युज᳚म् । स्वाहा᳚ । प्र॒जाप॑तय॒ इति॑ प्र॒जा - प॒त॒ये॒ । मन॑वे । स्वाहा᳚ । अ॒ग्नये᳚ । वै॒श्वा॒न॒राय॑ । स्वाहा᳚ । विश्वे᳚ । दे॒वस्य॑ । ने॒तुः । मर्तः॑ । वृ॒णी॒त॒ । स॒ख्यम् । विश्वे᳚ । रा॒यः । इ॒षु॒द्ध्य॒सि॒ । द्यु॒म्नम् । वृ॒णी॒त॒ । पु॒ष्यसे᳚ । स्वाहा᳚ । मा । स्विति॑ । भि॒त्थाः॒ । मा । स्विति॑ । रि॒षः॒ । दृꣳह॑स्व । वी॒डय॑स्व । सु ॥ अबं॑ । धृ॒ष्णु॒ । वी॒रय॑स्व ।  \newline


\textbf{Krama Paata} \newline

आकू॑तिम॒ग्निम् । आकू॑ति॒मित्या - कू॒ति॒म् । अ॒ग्निम् प्र॒युज᳚म् । प्र॒युजꣳ॒॒ स्वाहा᳚ । प्र॒युज॒मिति॑ प्र - युज᳚म् । स्वाहा॒ मनः॑ । मनो॑ मे॒धाम् । मे॒धाम॒ग्निम् । अ॒ग्निम् प्र॒युज᳚म् । प्र॒युजꣳ॒॒ स्वाहा᳚ । प्र॒युज॒मिति॑ प्र - युज᳚म् । स्वाहा॑ चि॒त्तम् । चि॒त्तं ॅविज्ञा॑तम् । विज्ञा॑तम॒ग्निम् । विज्ञा॑त॒मिति॒ वि - ज्ञा॒त॒म् । अ॒ग्निम् प्र॒युज᳚म् । प्र॒युजꣳ॒॒ स्वाहा᳚ । प्र॒युज॒मिति॑ प्र - युज᳚म् । स्वाहा॑ वा॒चः । वा॒चो विधृ॑तिम् । विधृ॑तिम॒ग्निम् । विधृ॑ति॒मिति॒ वि - धृ॒ति॒म् । अ॒ग्निम् प्र॒युज᳚म् । प्र॒युजꣳ॒॒ स्वाहा᳚ । प्र॒युज॒मिति॑ प्र - युज᳚म् । स्वाहा᳚ प्र॒जाप॑तये । प्र॒जाप॑तये॒ मन॑वे । प्र॒जाप॑तय॒ इति॑ प्र॒जा - प॒त॒ये॒ । मन॑वे॒ स्वाहा᳚ । स्वाहा॒ऽग्नये᳚ । अ॒ग्नये॑ वैश्वान॒राय॑ । वै॒श्वा॒न॒राय॒ स्वाहा᳚ । स्वाहा॒ विश्वे᳚ । विश्वे॑ दे॒वस्य॑ । दे॒वस्य॑ ने॒तुः । ने॒तुर् मर्तः॑ । मर्तो॑ वृणीत । वृ॒णी॒त॒ स॒ख्यम् । स॒ख्यं ॅविश्वे᳚ । विश्वे॑ रा॒यः । रा॒य इ॑षुद्ध्यसि । इ॒षु॒द्ध्य॒सि॒ द्यु॒म्नम् । द्यु॒म्नं ॅवृ॑णीत । वृ॒णी॒त॒ पु॒ष्यसे᳚ । पु॒ष्यसे॒ स्वाहा᳚ । स्वाहा॒ मा । मा सु । सु भि॑त्थाः । भि॒त्था॒ मा । मा सु । सु रि॑षः । रि॒षो॒ दृꣳह॑स्व । दृꣳह॑स्व वी॒डय॑स्व । वी॒डय॑स्व॒ सु । स्विति॒ सु ॥ अम्ब॑ धृष्णु । धृ॒ष्णु॒ वी॒रय॑स्व । वी॒रय॑स्वा॒ग्निः \newline

\textbf{Jatai Paata} \newline

1. आकू॑ति म॒ग्नि म॒ग्नि माकू॑ति॒ माकू॑ति म॒ग्निम् । \newline
2. आकू॑ति॒मित्या - कू॒ति॒म् । \newline
3. अ॒ग्निम् प्र॒युज॑म् प्र॒युज॑ म॒ग्नि म॒ग्निम् प्र॒युज᳚म् । \newline
4. प्र॒युजꣳ॒॒ स्वाहा॒ स्वाहा᳚ प्र॒युज॑म् प्र॒युजꣳ॒॒ स्वाहा᳚ । \newline
5. प्र॒युज॒मिति॑ प्र - युज᳚म् । \newline
6. स्वाहा॒ मनो॒ मनः॒ स्वाहा॒ स्वाहा॒ मनः॑ । \newline
7. मनो॑ मे॒धाम् मे॒धाम् मनो॒ मनो॑ मे॒धाम् । \newline
8. मे॒धा म॒ग्नि म॒ग्निम् मे॒धाम् मे॒धा म॒ग्निम् । \newline
9. अ॒ग्निम् प्र॒युज॑म् प्र॒युज॑ म॒ग्नि म॒ग्निम् प्र॒युज᳚म् । \newline
10. प्र॒युजꣳ॒॒ स्वाहा॒ स्वाहा᳚ प्र॒युज॑म् प्र॒युजꣳ॒॒ स्वाहा᳚ । \newline
11. प्र॒युज॒मिति॑ प्र - युज᳚म् । \newline
12. स्वाहा॑ चि॒त्तम् चि॒त्तꣳ स्वाहा॒ स्वाहा॑ चि॒त्तम् । \newline
13. चि॒त्तं ॅविज्ञा॑तं॒ ॅविज्ञा॑तम् चि॒त्तम् चि॒त्तं ॅविज्ञा॑तम् । \newline
14. विज्ञा॑त म॒ग्नि म॒ग्निं ॅविज्ञा॑तं॒ ॅविज्ञा॑त म॒ग्निम् । \newline
15. विज्ञा॑त॒मिति॒ वि - ज्ञा॒त॒म् । \newline
16. अ॒ग्निम् प्र॒युज॑म् प्र॒युज॑ म॒ग्नि म॒ग्निम् प्र॒युज᳚म् । \newline
17. प्र॒युजꣳ॒॒ स्वाहा॒ स्वाहा᳚ प्र॒युज॑म् प्र॒युजꣳ॒॒ स्वाहा᳚ । \newline
18. प्र॒युज॒मिति॑ प्र - युज᳚म् । \newline
19. स्वाहा॑ वा॒चो वा॒चः स्वाहा॒ स्वाहा॑ वा॒चः । \newline
20. वा॒चो विधृ॑तिं॒ ॅविधृ॑तिं ॅवा॒चो वा॒चो विधृ॑तिम् । \newline
21. विधृ॑ति म॒ग्नि म॒ग्निं ॅविधृ॑तिं॒ ॅविधृ॑ति म॒ग्निम् । \newline
22. विधृ॑ति॒मिति॒ वि - धृ॒ति॒म् । \newline
23. अ॒ग्निम् प्र॒युज॑म् प्र॒युज॑ म॒ग्नि म॒ग्निम् प्र॒युज᳚म् । \newline
24. प्र॒युजꣳ॒॒ स्वाहा॒ स्वाहा᳚ प्र॒युज॑म् प्र॒युजꣳ॒॒ स्वाहा᳚ । \newline
25. प्र॒युज॒मिति॑ प्र - युज᳚म् । \newline
26. स्वाहा᳚ प्र॒जाप॑तये प्र॒जाप॑तये॒ स्वाहा॒ स्वाहा᳚ प्र॒जाप॑तये । \newline
27. प्र॒जाप॑तये॒ मन॑वे॒ मन॑वे प्र॒जाप॑तये प्र॒जाप॑तये॒ मन॑वे । \newline
28. प्र॒जाप॑तय॒ इति॑ प्र॒जा - प॒त॒ये॒ । \newline
29. मन॑वे॒ स्वाहा॒ स्वाहा॒ मन॑वे॒ मन॑वे॒ स्वाहा᳚ । \newline
30. स्वाहा॒ ऽग्नये॑ अ॒ग्नये॒ स्वाहा॒ स्वाहा॒ ऽग्नये᳚ । \newline
31. अ॒ग्नये॑ वैश्वान॒राय॑ वैश्वान॒राया॒ ग्नये॑ अ॒ग्नये॑ वैश्वान॒राय॑ । \newline
32. वै॒श्वा॒न॒राय॒ स्वाहा॒ स्वाहा॑ वैश्वान॒राय॑ वैश्वान॒राय॒ स्वाहा᳚ । \newline
33. स्वाहा॒ विश्वे॒ विश्वे॒ स्वाहा॒ स्वाहा॒ विश्वे᳚ । \newline
34. विश्वे॑ दे॒वस्य॑ दे॒वस्य॒ विश्वे॒ विश्वे॑ दे॒वस्य॑ । \newline
35. दे॒वस्य॑ ने॒तुर् ने॒तुर् दे॒वस्य॑ दे॒वस्य॑ ने॒तुः । \newline
36. ने॒तुर् मर्तो॒ मर्तो॑ ने॒तुर् ने॒तुर् मर्तः॑ । \newline
37. मर्तो॑ वृणीत वृणीत॒ मर्तो॒ मर्तो॑ वृणीत । \newline
38. वृ॒णी॒त॒ स॒ख्यꣳ स॒ख्यं ॅवृ॑णीत वृणीत स॒ख्यम् । \newline
39. स॒ख्यं ॅविश्वे॒ विश्वे॑ स॒ख्यꣳ स॒ख्यं ॅविश्वे᳚ । \newline
40. विश्वे॑ रा॒यो रा॒यो विश्वे॒ विश्वे॑ रा॒यः । \newline
41. रा॒य इ॑षुद्ध्यसी षुद्ध्यसि रा॒यो रा॒य इ॑षुद्ध्यसि । \newline
42. इ॒षु॒द्ध्य॒सि॒ द्यु॒म्नम् द्यु॒म्न मि॑षुद्ध्यसी षुद्ध्यसि द्यु॒म्नम् । \newline
43. द्यु॒म्नं ॅवृ॑णीत वृणीत द्यु॒म्नम् द्यु॒म्नं ॅवृ॑णीत । \newline
44. वृ॒णी॒त॒ पु॒ष्यसे॑ पु॒ष्यसे॑ वृणीत वृणीत पु॒ष्यसे᳚ । \newline
45. पु॒ष्यसे॒ स्वाहा॒ स्वाहा॑ पु॒ष्यसे॑ पु॒ष्यसे॒ स्वाहा᳚ । \newline
46. स्वाहा॒ मा मा स्वाहा॒ स्वाहा॒ मा । \newline
47. मा सु सु मा मा सु । \newline
48. सु भि॑त्था भित्थाः॒ सु सु भि॑त्थाः । \newline
49. भि॒त्था॒ मा मा भि॑त्था भित्था॒ मा । \newline
50. मा सु सु मा मा सु । \newline
51. सु रि॑षो रिषः॒ सु सु रि॑षः । \newline
52. रि॒षो॒ दृꣳह॑स्व॒ दृꣳह॑स्व रिषो रिषो॒ दृꣳह॑स्व । \newline
53. दृꣳह॑स्व वी॒डय॑स्व वी॒डय॑स्व॒ दृꣳह॑स्व॒ दृꣳह॑स्व वी॒डय॑स्व । \newline
54. वी॒डय॑स्व॒ सु सु वी॒डय॑स्व वी॒डय॑स्व॒ सु । \newline
55. स्विति॒ सु । \newline
56. अंब॑ धृष्णु धृ॒ष्ण्वंबांब॑ धृष्णु । \newline
57. धृ॒ष्णु॒ वी॒रय॑स्व वी॒रय॑स्व धृष्णु धृष्णु वी॒रय॑स्व । \newline
58. वी॒रय॑स्वा॒ ग्नि र॒ग्निर् वी॒रय॑स्व वी॒रय॑स्वा॒ग्निः । \newline

\textbf{Ghana Paata } \newline

1. आकू॑ति म॒ग्नि म॒ग्नि माकू॑ति॒ माकू॑ति म॒ग्निम् प्र॒युज॑म् प्र॒युज॑ म॒ग्नि माकू॑ति॒ माकू॑ति म॒ग्निम् प्र॒युज᳚म् । \newline
2. आकू॑ति॒मित्या - कू॒ति॒म् । \newline
3. अ॒ग्निम् प्र॒युज॑म् प्र॒युज॑ म॒ग्नि म॒ग्निम् प्र॒युजꣳ॒॒ स्वाहा॒ स्वाहा᳚ प्र॒युज॑ म॒ग्नि म॒ग्निम् प्र॒युजꣳ॒॒ स्वाहा᳚ । \newline
4. प्र॒युजꣳ॒॒ स्वाहा॒ स्वाहा᳚ प्र॒युज॑म् प्र॒युजꣳ॒॒ स्वाहा॒ मनो॒ मनः॒ स्वाहा᳚ प्र॒युज॑म् प्र॒युजꣳ॒॒ स्वाहा॒ मनः॑ । \newline
5. प्र॒युज॒मिति॑ प्र - युज᳚म् । \newline
6. स्वाहा॒ मनो॒ मनः॒ स्वाहा॒ स्वाहा॒ मनो॑ मे॒धाम् मे॒धाम् मनः॒ स्वाहा॒ स्वाहा॒ मनो॑ मे॒धाम् । \newline
7. मनो॑ मे॒धाम् मे॒धाम् मनो॒ मनो॑ मे॒धा म॒ग्नि म॒ग्निम् मे॒धाम् मनो॒ मनो॑ मे॒धा म॒ग्निम् । \newline
8. मे॒धा म॒ग्नि म॒ग्निम् मे॒धाम् मे॒धा म॒ग्निम् प्र॒युज॑म् प्र॒युज॑ म॒ग्निम् मे॒धाम् मे॒धा म॒ग्निम् प्र॒युज᳚म् । \newline
9. अ॒ग्निम् प्र॒युज॑म् प्र॒युज॑ म॒ग्नि म॒ग्निम् प्र॒युजꣳ॒॒ स्वाहा॒ स्वाहा᳚ प्र॒युज॑ म॒ग्नि म॒ग्निम् प्र॒युजꣳ॒॒ स्वाहा᳚ । \newline
10. प्र॒युजꣳ॒॒ स्वाहा॒ स्वाहा᳚ प्र॒युज॑म् प्र॒युजꣳ॒॒ स्वाहा॑ चि॒त्तम् चि॒त्तꣳ स्वाहा᳚ प्र॒युज॑म् प्र॒युजꣳ॒॒ स्वाहा॑ चि॒त्तम् । \newline
11. प्र॒युज॒मिति॑ प्र - युज᳚म् । \newline
12. स्वाहा॑ चि॒त्तम् चि॒त्तꣳ स्वाहा॒ स्वाहा॑ चि॒त्तं ॅविज्ञा॑तं॒ ॅविज्ञा॑तम् चि॒त्तꣳ स्वाहा॒ स्वाहा॑ चि॒त्तं ॅविज्ञा॑तम् । \newline
13. चि॒त्तं ॅविज्ञा॑तं॒ ॅविज्ञा॑तम् चि॒त्तम् चि॒त्तं ॅविज्ञा॑त म॒ग्नि म॒ग्निं ॅविज्ञा॑तम् चि॒त्तम् चि॒त्तं ॅविज्ञा॑त म॒ग्निम् । \newline
14. विज्ञा॑त म॒ग्नि म॒ग्निं ॅविज्ञा॑तं॒ ॅविज्ञा॑त म॒ग्निम् प्र॒युज॑म् प्र॒युज॑ म॒ग्निं ॅविज्ञा॑तं॒ ॅविज्ञा॑त म॒ग्निम् प्र॒युज᳚म् । \newline
15. विज्ञा॑त॒मिति॒ वि - ज्ञा॒त॒म् । \newline
16. अ॒ग्निम् प्र॒युज॑म् प्र॒युज॑ म॒ग्नि म॒ग्निम् प्र॒युजꣳ॒॒ स्वाहा॒ स्वाहा᳚ प्र॒युज॑ म॒ग्नि म॒ग्निम् प्र॒युजꣳ॒॒ स्वाहा᳚ । \newline
17. प्र॒युजꣳ॒॒ स्वाहा॒ स्वाहा᳚ प्र॒युज॑म् प्र॒युजꣳ॒॒ स्वाहा॑ वा॒चो वा॒चः स्वाहा᳚ प्र॒युज॑म् प्र॒युजꣳ॒॒ स्वाहा॑ वा॒चः । \newline
18. प्र॒युज॒मिति॑ प्र - युज᳚म् । \newline
19. स्वाहा॑ वा॒चो वा॒चः स्वाहा॒ स्वाहा॑ वा॒चो विधृ॑तिं॒ ॅविधृ॑तिं ॅवा॒चः स्वाहा॒ स्वाहा॑ वा॒चो विधृ॑तिम् । \newline
20. वा॒चो विधृ॑तिं॒ ॅविधृ॑तिं ॅवा॒चो वा॒चो विधृ॑ति म॒ग्नि म॒ग्निं ॅविधृ॑तिं ॅवा॒चो वा॒चो विधृ॑ति म॒ग्निम् । \newline
21. विधृ॑ति म॒ग्नि म॒ग्निं ॅविधृ॑तिं॒ ॅविधृ॑ति म॒ग्निम् प्र॒युज॑म् प्र॒युज॑ म॒ग्निं ॅविधृ॑तिं॒ ॅविधृ॑ति म॒ग्निम् प्र॒युज᳚म् । \newline
22. विधृ॑ति॒मिति॒ वि - धृ॒ति॒म् । \newline
23. अ॒ग्निम् प्र॒युज॑म् प्र॒युज॑ म॒ग्नि म॒ग्निम् प्र॒युजꣳ॒॒ स्वाहा॒ स्वाहा᳚ प्र॒युज॑ म॒ग्नि म॒ग्निम् प्र॒युजꣳ॒॒ स्वाहा᳚ । \newline
24. प्र॒युजꣳ॒॒ स्वाहा॒ स्वाहा᳚ प्र॒युज॑म् प्र॒युजꣳ॒॒ स्वाहा᳚ प्र॒जाप॑तये प्र॒जाप॑तये॒ स्वाहा᳚ प्र॒युज॑म् प्र॒युजꣳ॒॒ स्वाहा᳚ प्र॒जाप॑तये । \newline
25. प्र॒युज॒मिति॑ प्र - युज᳚म् । \newline
26. स्वाहा᳚ प्र॒जाप॑तये प्र॒जाप॑तये॒ स्वाहा॒ स्वाहा᳚ प्र॒जाप॑तये॒ मन॑वे॒ मन॑वे प्र॒जाप॑तये॒ स्वाहा॒ स्वाहा᳚ प्र॒जाप॑तये॒ मन॑वे । \newline
27. प्र॒जाप॑तये॒ मन॑वे॒ मन॑वे प्र॒जाप॑तये प्र॒जाप॑तये॒ मन॑वे॒ स्वाहा॒ स्वाहा॒ मन॑वे प्र॒जाप॑तये प्र॒जाप॑तये॒ मन॑वे॒ स्वाहा᳚ । \newline
28. प्र॒जाप॑तय॒ इति॑ प्र॒जा - प॒त॒ये॒ । \newline
29. मन॑वे॒ स्वाहा॒ स्वाहा॒ मन॑वे॒ मन॑वे॒ स्वाहा॒ ऽग्नये॑ अ॒ग्नये॒ स्वाहा॒ मन॑वे॒ मन॑वे॒ स्वाहा॒ ऽग्नये᳚ । \newline
30. स्वाहा॒ ऽग्नये॑ अ॒ग्नये॒ स्वाहा॒ स्वाहा॒ ऽग्नये॑ वैश्वान॒राय॑ वैश्वान॒राया॒ ग्नये॒ स्वाहा॒ स्वाहा॒ ऽग्नये॑ वैश्वान॒राय॑ । \newline
31. अ॒ग्नये॑ वैश्वान॒राय॑ वैश्वान॒राया॒ग्नये॑ अ॒ग्नये॑ वैश्वान॒राय॒ स्वाहा॒ स्वाहा॑ वैश्वान॒राया॒ ग्नये॑ अ॒ग्नये॑ वैश्वान॒राय॒ स्वाहा᳚ । \newline
32. वै॒श्वा॒न॒राय॒ स्वाहा॒ स्वाहा॑ वैश्वान॒राय॑ वैश्वान॒राय॒ स्वाहा॒ विश्वे॒ विश्वे॒ स्वाहा॑ वैश्वान॒राय॑ वैश्वान॒राय॒ स्वाहा॒ विश्वे᳚ । \newline
33. स्वाहा॒ विश्वे॒ विश्वे॒ स्वाहा॒ स्वाहा॒ विश्वे॑ दे॒वस्य॑ दे॒वस्य॒ विश्वे॒ स्वाहा॒ स्वाहा॒ विश्वे॑ दे॒वस्य॑ । \newline
34. विश्वे॑ दे॒वस्य॑ दे॒वस्य॒ विश्वे॒ विश्वे॑ दे॒वस्य॑ ने॒तुर् ने॒तुर् दे॒वस्य॒ विश्वे॒ विश्वे॑ दे॒वस्य॑ ने॒तुः । \newline
35. दे॒वस्य॑ ने॒तुर् ने॒तुर् दे॒वस्य॑ दे॒वस्य॑ ने॒तुर् मर्तो॒ मर्तो॑ ने॒तुर् दे॒वस्य॑ दे॒वस्य॑ ने॒तुर् मर्तः॑ । \newline
36. ने॒तुर् मर्तो॒ मर्तो॑ ने॒तुर् ने॒तुर् मर्तो॑ वृणीत वृणीत॒ मर्तो॑ ने॒तुर् ने॒तुर् मर्तो॑ वृणीत । \newline
37. मर्तो॑ वृणीत वृणीत॒ मर्तो॒ मर्तो॑ वृणीत स॒ख्यꣳ स॒ख्यं ॅवृ॑णीत॒ मर्तो॒ मर्तो॑ वृणीत स॒ख्यम् । \newline
38. वृ॒णी॒त॒ स॒ख्यꣳ स॒ख्यं ॅवृ॑णीत वृणीत स॒ख्यं ॅविश्वे॒ विश्वे॑ स॒ख्यं ॅवृ॑णीत वृणीत स॒ख्यं ॅविश्वे᳚ । \newline
39. स॒ख्यं ॅविश्वे॒ विश्वे॑ स॒ख्यꣳ स॒ख्यं ॅविश्वे॑ रा॒यो रा॒यो विश्वे॑ स॒ख्यꣳ स॒ख्यं ॅविश्वे॑ रा॒यः । \newline
40. विश्वे॑ रा॒यो रा॒यो विश्वे॒ विश्वे॑ रा॒य इ॑षुद्ध्यसी षुद्ध्यसि रा॒यो विश्वे॒ विश्वे॑ रा॒य इ॑षुद्ध्यसि । \newline
41. रा॒य इ॑षुद्ध्यसी षुद्ध्यसि रा॒यो रा॒य इ॑षुद्ध्यसि द्यु॒म्नम् द्यु॒म्न मि॑षुद्ध्यसि रा॒यो रा॒य इ॑षुद्ध्यसि द्यु॒म्नम् । \newline
42. इ॒षु॒द्ध्य॒सि॒ द्यु॒म्नम् द्यु॒म्न मि॑षुद्ध्यसी षुद्ध्यसि द्यु॒म्नं ॅवृ॑णीत वृणीत द्यु॒म्न मि॑षुद्ध्यसी षुद्ध्यसि द्यु॒म्नं ॅवृ॑णीत । \newline
43. द्यु॒म्नं ॅवृ॑णीत वृणीत द्यु॒म्नम् द्यु॒म्नं ॅवृ॑णीत पु॒ष्यसे॑ पु॒ष्यसे॑ वृणीत द्यु॒म्नम् द्यु॒म्नं ॅवृ॑णीत पु॒ष्यसे᳚ । \newline
44. वृ॒णी॒त॒ पु॒ष्यसे॑ पु॒ष्यसे॑ वृणीत वृणीत पु॒ष्यसे॒ स्वाहा॒ स्वाहा॑ पु॒ष्यसे॑ वृणीत वृणीत पु॒ष्यसे॒ स्वाहा᳚ । \newline
45. पु॒ष्यसे॒ स्वाहा॒ स्वाहा॑ पु॒ष्यसे॑ पु॒ष्यसे॒ स्वाहा॒ मा मा स्वाहा॑ पु॒ष्यसे॑ पु॒ष्यसे॒ स्वाहा॒ मा । \newline
46. स्वाहा॒ मा मा स्वाहा॒ स्वाहा॒ मा सु सु मा स्वाहा॒ स्वाहा॒ मा सु । \newline
47. मा सु सु मा मा सु भि॑त्था भित्थाः॒ सु मा मा सु भि॑त्थाः । \newline
48. सु भि॑त्था भित्थाः॒ सु सु भि॑त्था॒ मा मा भि॑त्थाः॒ सु सु भि॑त्था॒ मा । \newline
49. भि॒त्था॒ मा मा भि॑त्था भित्था॒ मा सु सु मा भि॑त्था भित्था॒ मा सु । \newline
50. मा सु सु मा मा सु रि॑षो रिषः॒ सु मा मा सु रि॑षः । \newline
51. सु रि॑षो रिषः॒ सु सु रि॑षो॒ दृꣳह॑स्व॒ दृꣳह॑स्व रिषः॒ सु सु रि॑षो॒ दृꣳह॑स्व । \newline
52. रि॒षो॒ दृꣳह॑स्व॒ दृꣳह॑स्व रिषो रिषो॒ दृꣳह॑स्व वी॒डय॑स्व वी॒डय॑स्व॒ दृꣳह॑स्व रिषो रिषो॒ दृꣳह॑स्व वी॒डय॑स्व । \newline
53. दृꣳह॑स्व वी॒डय॑स्व वी॒डय॑स्व॒ दृꣳह॑स्व॒ दृꣳह॑स्व वी॒डय॑स्व॒ सु सु वी॒डय॑स्व॒ दृꣳह॑स्व॒ दृꣳह॑स्व वी॒डय॑स्व॒ सु । \newline
54. वी॒डय॑स्व॒ सु सु वी॒डय॑स्व वी॒डय॑स्व॒ सु । \newline
55. स्विति॒ सु । \newline
56. अंब॑ धृष्णु धृ॒ष्ण्वंबांब॑ धृष्णु वी॒रय॑स्व वी॒रय॑स्व धृ॒ष्ण्वंबांब॑ धृष्णु वी॒रय॑स्व । \newline
57. धृ॒ष्णु॒ वी॒रय॑स्व वी॒रय॑स्व धृष्णु धृष्णु वी॒रय॑स्वा॒ ग्नि र॒ग्निर् वी॒रय॑स्व धृष्णु धृष्णु वी॒रय॑स्वा॒ग्निः । \newline
58. वी॒रय॑स्वा॒ ग्निर॒ग्निर् वी॒रय॑स्व वी॒रय॑स्वा॒ ग्निश्च॑ चा॒ग्निर् वी॒रय॑स्व वी॒रय॑स्वा॒ ग्निश्च॑ । \newline
\pagebreak
\markright{ TS 4.1.9.2  \hfill https://www.vedavms.in \hfill}

\section{ TS 4.1.9.2 }

\textbf{TS 4.1.9.2 } \newline
\textbf{Samhita Paata} \newline

-ऽग्निश्चे॒दं क॑रिष्यथः ॥ दृꣳह॑स्व देवि पृथिवि स्व॒स्तय॑ आसु॒री मा॒या स्व॒धया॑ कृ॒ताऽसि॑ । जुष्टं॑ दे॒वाना॑मि॒दम॑स्तु ह॒व्यमरि॑ष्टा॒ त्वमुदि॑हि य॒ज्ञे अ॒स्मिन्न् ॥ मित्रै॒तामु॒खां त॑पै॒षा मा भे॑दि । ए॒तां ते॒ परि॑ ददा॒म्यभि॑त्त्यै ॥ द्र्व॑न्नः स॒र्पिरा॑सुतिः प्र॒त्नो होता॒ वरे᳚ण्यः । सह॑सस्पु॒त्रो अद्भु॑तः ॥ पर॑स्या॒ अधि॑ सं॒ॅवतोऽव॑राꣳ अ॒भ्या - [  ] \newline

\textbf{Pada Paata} \newline

अ॒ग्निः । च॒ । इ॒दम् । क॒रि॒ष्य॒थः॒ ॥ दृꣳह॑स्व । दे॒वि॒ । पृ॒थि॒वि॒ । स्व॒स्तये᳚ । आ॒सु॒री । मा॒या । स्व॒धयेति॑ स्व - धया᳚ । कृ॒ता । अ॒सि॒ ॥ जुष्ट᳚म् । दे॒वाना᳚म् । इ॒दम् । अ॒स्तु॒ । ह॒व्यम् । अरि॑ष्टा । त्वम् । उदिति॑ । इ॒हि॒ । य॒ज्ञे । अ॒स्मिन्न् ॥ मित्र॑ । ए॒ताम् । उ॒खाम् । त॒प॒ । ए॒षा । मा । भे॒दि॒ ॥ ए॒ताम् । ते॒ । परीति॑ । द॒दा॒मि॒ । अभि॑त्त्यै ॥ द्र्‌व॑न्न॒ इति॒ द्रु - अ॒न्नः॒ । स॒र्पिरा॑सुति॒रिति॑ स॒र्पिः - आ॒सु॒तिः॒ । प्र॒त्नः । होता᳚ । वरे᳚ण्यः ॥ सह॑सः । पु॒त्रः । अद्भु॑तः ॥ पर॑स्याः । अधीति॑ । सं॒ॅवत॒ इति॑ सं - वतः॑ । अव॑रान् । अ॒भि । एति॑ ।  \newline


\textbf{Krama Paata} \newline

अ॒ग्निश्च॑ । चे॒दम् । इ॒दम् क॑रिष्यथः । क॒रि॒ष्य॒थ॒ इति॑ करिष्यथः ॥ दृꣳह॑स्व देवि । दे॒वि॒ पृ॒थि॒वि॒ । पृ॒थि॒वि॒ स्व॒स्तये᳚ । स्व॒स्तय॑ आसु॒री । आ॒सु॒री मा॒या । मा॒या स्व॒धया᳚ । स्व॒धया॑ कृ॒ता । स्व॒धयेति॑ स्व - धया᳚ । कृ॒ताऽसि॑ । अ॒सीत्य॑सि ॥ जुष्ट॑म् दे॒वाना᳚म् । दे॒वाना॑मि॒दम् । इ॒दम॑स्तु । अ॒स्तु॒ ह॒व्यम् । ह॒व्यमरि॑ष्टा । अरि॑ष्टा॒ त्वम् । त्वमुत् । उदि॑हि । इ॒हि॒ य॒ज्ञे । य॒ज्ञे अ॒स्मिन्न् । अ॒स्मिन्नित्य॒स्मिन्न् ॥ मित्रै॒ताम् । ए॒तामु॒खाम् । उ॒खाम् त॑प । त॒पै॒षा । ए॒षा मा । मा भे॑दि । भे॒दीति॑ भेदि ॥ ए॒ताम् ते᳚ । ते॒ परि॑ । परि॑ ददामि । द॒दा॒म्यभि॑त्त्यै । अभि॑त्त्या॒ इत्यभि॑त्त्यै ॥ द्र्व॑न्नः स॒र्पिरा॑सुतिः । द्र्व॑न्न॒ इति॒ द्रु - अ॒न्नः॒ । स॒र्पिरा॑सुतिः प्र॒त्नः । स॒र्पिरा॑सुति॒रिति॑ स॒र्पिः - आ॒सु॒तिः॒ । प्र॒त्नो होता᳚ । होता॒ वरे᳚ण्यः । वरे᳚ण्य॒ इति॒ वरे᳚ण्यः ॥ सह॑सस्पु॒त्रः । पु॒त्रो अद्भु॑तः । अद्भु॑त॒ इत्यद्भु॑तः ॥ पर॑स्या॒ अधि॑ । अधि॑ स॒म्ॅवतः॑ । स॒म्ॅवतोऽव॑रान् । स॒म्ॅवत॒ इति॑ सम् - वतः॑ । अव॑राꣳ अ॒भि । अ॒भ्या । आ त॑र \newline

\textbf{Jatai Paata} \newline

1. अ॒ग्निश्च॑ चा॒ग्नि र॒ग्निश्च॑ । \newline
2. चे॒ द मि॒दम् च॑ चे॒ दम् । \newline
3. इ॒दम् क॑रिष्यथः करिष्यथ इ॒द मि॒दम् क॑रिष्यथः । \newline
4. क॒रि॒ष्य॒थ॒ इति॑ करिष्यथः । \newline
5. दृꣳह॑स्व देवि देवि॒ दृꣳह॑स्व॒ दृꣳह॑स्व देवि । \newline
6. दे॒वि॒ पृ॒थि॒वि॒ पृ॒थि॒वि॒ दे॒वि॒ दे॒वि॒ पृ॒थि॒वि॒ । \newline
7. पृ॒थि॒वि॒ स्व॒स्तये᳚ स्व॒स्तये॑ पृथिवि पृथिवि स्व॒स्तये᳚ । \newline
8. स्व॒स्तय॑ आसु॒र्या॑सु॒री स्व॒स्तये᳚ स्व॒स्तय॑ आसु॒री । \newline
9. आ॒सु॒री मा॒या मा॒या ऽऽसु॒र्या॑सु॒री मा॒या । \newline
10. मा॒या स्व॒धया᳚ स्व॒धया॑ मा॒या मा॒या स्व॒धया᳚ । \newline
11. स्व॒धया॑ कृ॒ता कृ॒ता स्व॒धया᳚ स्व॒धया॑ कृ॒ता । \newline
12. स्व॒धयेति॑ स्व - धया᳚ । \newline
13. कृ॒ता ऽस्य॑सि कृ॒ता कृ॒ता ऽसि॑ । \newline
14. अ॒सीत्य॑सि । \newline
15. जुष्ट॑म् दे॒वाना᳚म् दे॒वाना॒म् जुष्ट॒म् जुष्ट॑म् दे॒वाना᳚म् । \newline
16. दे॒वाना॑ मि॒द मि॒दम् दे॒वाना᳚म् दे॒वाना॑ मि॒दम् । \newline
17. इ॒द म॑स्त्व स्त्वि॒द मि॒द म॑स्तु । \newline
18. अ॒स्तु॒ ह॒व्यꣳ ह॒व्य म॑स्त्वस्तु ह॒व्यम् । \newline
19. ह॒व्य मरि॒ष्टा ऽरि॑ष्टा ह॒व्यꣳ ह॒व्य मरि॑ष्टा । \newline
20. अरि॑ष्टा॒ त्वम् त्व मरि॒ष्टा ऽरि॑ष्टा॒ त्वम् । \newline
21. त्व मुदुत् त्वम् त्व मुत् । \newline
22. उदि॑ही॒ ह्युदु दि॑हि । \newline
23. इ॒हि॒ य॒ज्ञे य॒ज्ञ् इ॑हीहि य॒ज्ञे । \newline
24. य॒ज्ञे अ॒स्मिन् न॒स्मिन्. य॒ज्ञे य॒ज्ञे अ॒स्मिन्न् । \newline
25. अ॒स्मिन्नित्य॒स्मिन्न् । \newline
26. मित्रै॒ता मे॒ताम् मित्र॒ मित्रै॒ताम् । \newline
27. ए॒ता मु॒खा मु॒खा मे॒ता मे॒ता मु॒खाम् । \newline
28. उ॒खाम् त॑प तपो॒खा मु॒खाम् त॑प । \newline
29. त॒पै॒षैषा त॑प तपै॒षा । \newline
30. ए॒षा मा मैषैषा मा । \newline
31. मा भे॑दि भेदि॒ मा मा भे॑दि । \newline
32. भे॒दीति॑ भेदि । \newline
33. ए॒ताम् ते॑ त ए॒ता मे॒ताम् ते᳚ । \newline
34. ते॒ परि॒ परि॑ ते ते॒ परि॑ । \newline
35. परि॑ ददामि ददामि॒ परि॒ परि॑ ददामि । \newline
36. द॒दा॒ म्यभि॑त्त्या॒ अभि॑त्त्यै ददामि ददा॒ म्यभि॑त्त्यै । \newline
37. अभि॑त्त्या॒ इत्यभि॑त्त्यै । \newline
38. द्र्व॑न्नः स॒र्पिरा॑सुतिः स॒र्पिरा॑सुति॒र् द्र्व॑न्नो॒ द्र्व॑न्नः स॒र्पिरा॑सुतिः । \newline
39. द्र्व॑न्न॒ इति॒ द्रु - अ॒न्नः॒ । \newline
40. स॒र्पिरा॑सुतिः प्र॒त्नः प्र॒त्नः स॒र्पिरा॑सुतिः स॒र्पिरा॑सुतिः प्र॒त्नः । \newline
41. स॒र्पिरा॑सुति॒रिति॑ स॒र्पिः - आ॒सु॒तिः॒ । \newline
42. प्र॒त्नो होता॒ होता᳚ प्र॒त्नः प्र॒त्नो होता᳚ । \newline
43. होता॒ वरे᳚ण्यो॒ वरे᳚ण्यो॒ होता॒ होता॒ वरे᳚ण्यः । \newline
44. वरे᳚ण्य॒ इति॒ वरे᳚ण्यः । \newline
45. सह॑स स्पु॒त्रः पु॒त्रः सह॑सः॒ सह॑स स्पु॒त्रः । \newline
46. पु॒त्रो अद्भु॑तो॒ अद्भु॑तः पु॒त्रः पु॒त्रो अद्भु॑तः । \newline
47. अद्भु॑त॒ इत्यद्भु॑तः । \newline
48. पर॑स्या॒ अध्यधि॒ पर॑स्याः॒ पर॑स्या॒ अधि॑ । \newline
49. अधि॑ सं॒ॅवतः॑ सं॒ॅवतो॒ अध्यधि॑ सं॒ॅवतः॑ । \newline
50. सं॒ॅवतो ऽव॑राꣳ॒॒ अव॑रान् थ्सं॒ॅवतः॑ सं॒ॅवतो ऽव॑रान् । \newline
51. सं॒ॅवत॒ इति॑ सं - वतः॑ । \newline
52. अव॑राꣳ अ॒भ्य॑भ्यव॑राꣳ॒॒ अव॑राꣳ अ॒भि । \newline
53. अ॒भ्या ऽभ्य॑भ्या । \newline
54. आ त॑र त॒रा त॑र । \newline

\textbf{Ghana Paata } \newline

1. अ॒ग्निश्च॑ चा॒ग्नि र॒ग्नि श्चे॒द मि॒दम् चा॒ग्नि र॒ग्नि श्चे॒दम् । \newline
2. चे॒द मि॒दम् च॑ चे॒दम् क॑रिष्यथः करिष्यथ इ॒दम् च॑ चे॒दम् क॑रिष्यथः । \newline
3. इ॒दम् क॑रिष्यथः करिष्यथ इ॒द मि॒दम् क॑रिष्यथः । \newline
4. क॒रि॒ष्य॒थ॒ इति॑ करिष्यथः । \newline
5. दृꣳह॑स्व देवि देवि॒ दृꣳह॑स्व॒ दृꣳह॑स्व देवि पृथिवि पृथिवि देवि॒ दृꣳह॑स्व॒ दृꣳह॑स्व देवि पृथिवि । \newline
6. दे॒वि॒ पृ॒थि॒वि॒ पृ॒थि॒वि॒ दे॒वि॒ दे॒वि॒ पृ॒थि॒वि॒ स्व॒स्तये᳚ स्व॒स्तये॑ पृथिवि देवि देवि पृथिवि स्व॒स्तये᳚ । \newline
7. पृ॒थि॒वि॒ स्व॒स्तये᳚ स्व॒स्तये॑ पृथिवि पृथिवि स्व॒स्तय॑ आसु॒र्या॑सु॒री स्व॒स्तये॑ पृथिवि पृथिवि स्व॒स्तय॑ आसु॒री । \newline
8. स्व॒स्तय॑ आसु॒र्या॑सु॒री स्व॒स्तये᳚ स्व॒स्तय॑ आसु॒री मा॒या मा॒या ऽऽसु॒री स्व॒स्तये᳚ स्व॒स्तय॑ आसु॒री मा॒या । \newline
9. आ॒सु॒री मा॒या मा॒या ऽऽसु॒र्या॑सु॒री मा॒या स्व॒धया᳚ स्व॒धया॑ मा॒या ऽऽसु॒र्या॑सु॒री मा॒या स्व॒धया᳚ । \newline
10. मा॒या स्व॒धया᳚ स्व॒धया॑ मा॒या मा॒या स्व॒धया॑ कृ॒ता कृ॒ता स्व॒धया॑ मा॒या मा॒या स्व॒धया॑ कृ॒ता । \newline
11. स्व॒धया॑ कृ॒ता कृ॒ता स्व॒धया᳚ स्व॒धया॑ कृ॒ता ऽस्य॑सि कृ॒ता स्व॒धया᳚ स्व॒धया॑ कृ॒ता ऽसि॑ । \newline
12. स्व॒धयेति॑ स्व - धया᳚ । \newline
13. कृ॒ता ऽस्य॑सि कृ॒ता कृ॒ता ऽसि॑ । \newline
14. अ॒सीत्य॑सि । \newline
15. जुष्ट॑म् दे॒वाना᳚म् दे॒वाना॒म् जुष्ट॒म् जुष्ट॑म् दे॒वाना॑ मि॒द मि॒दम् दे॒वाना॒म् जुष्ट॒म् जुष्ट॑म् दे॒वाना॑ मि॒दम् । \newline
16. दे॒वाना॑ मि॒द मि॒दम् दे॒वाना᳚म् दे॒वाना॑ मि॒द म॑स्त्व स्त्वि॒दम् दे॒वाना᳚म् दे॒वाना॑ मि॒द म॑स्तु । \newline
17. इ॒द म॑स्त्व स्त्वि॒द मि॒द म॑स्तु ह॒व्यꣳ ह॒व्य म॑स्त्वि॒द मि॒द म॑स्तु ह॒व्यम् । \newline
18. अ॒स्तु॒ ह॒व्यꣳ ह॒व्य म॑स्त्वस्तु ह॒व्य मरि॒ष्टा ऽरि॑ष्टा ह॒व्य म॑स्त्वस्तु ह॒व्य मरि॑ष्टा । \newline
19. ह॒व्य मरि॒ष्टा ऽरि॑ष्टा ह॒व्यꣳ ह॒व्य मरि॑ष्टा॒ त्वम् त्व मरि॑ष्टा ह॒व्यꣳ ह॒व्य मरि॑ष्टा॒ त्वम् । \newline
20. अरि॑ष्टा॒ त्वम् त्व मरि॒ष्टा ऽरि॑ष्टा॒ त्व मुदुत् त्व मरि॒ष्टा ऽरि॑ष्टा॒ त्व मुत् । \newline
21. त्व मुदुत् त्वम् त्व मुदि॑ ही॒ह्युत् त्वम् त्व मुदि॑हि । \newline
22. उदि॑ही॒ ह्युदुदि॑हि य॒ज्ञे य॒ज्ञ् इ॒ह्यु दुदि॑हि य॒ज्ञे । \newline
23. इ॒हि॒ य॒ज्ञे य॒ज्ञ् इ॑हीहि य॒ज्ञे अ॒स्मिन् न॒स्मिन्. य॒ज्ञ् इ॑हीहि य॒ज्ञे अ॒स्मिन्न् । \newline
24. य॒ज्ञे अ॒स्मिन् न॒स्मिन्. य॒ज्ञे य॒ज्ञे अ॒स्मिन्न् । \newline
25. अ॒स्मिन्नित्य॒स्मिन्न् । \newline
26. मित्रै॒ता मे॒ताम् मित्र॒ मित्रै॒ता मु॒खा मु॒खा मे॒ताम् मित्र॒ मित्रै॒ता मु॒खाम् । \newline
27. ए॒ता मु॒खा मु॒खा मे॒ता मे॒ता मु॒खाम् त॑प तपो॒खा मे॒ता मे॒ता मु॒खाम् त॑प । \newline
28. उ॒खाम् त॑प तपो॒खा मु॒खाम् त॑पै॒ षैषा त॑पो॒खा मु॒खाम् त॑पै॒षा । \newline
29. त॒पै॒ षैषा त॑प तपै॒षा मा मैषा त॑प तपै॒षा मा । \newline
30. ए॒षा मा मैषैषा मा भे॑दि भेदि॒ मैषैषा मा भे॑दि । \newline
31. मा भे॑दि भेदि॒ मा मा भे॑दि । \newline
32. भे॒दीति॑ भेदि । \newline
33. ए॒ताम् ते॑ त ए॒ता मे॒ताम् ते॒ परि॒ परि॑ त ए॒ता मे॒ताम् ते॒ परि॑ । \newline
34. ते॒ परि॒ परि॑ ते ते॒ परि॑ ददामि ददामि॒ परि॑ ते ते॒ परि॑ ददामि । \newline
35. परि॑ ददामि ददामि॒ परि॒ परि॑ ददा॒ म्यभि॑त्त्या॒ अभि॑त्त्यै ददामि॒ परि॒ परि॑ ददा॒ म्यभि॑त्त्यै । \newline
36. द॒दा॒ म्यभि॑त्त्या॒ अभि॑त्त्यै ददामि ददा॒ म्यभि॑त्त्यै । \newline
37. अभि॑त्त्या॒ इत्यभि॑त्त्यै । \newline
38. द्र्व॑न्नः स॒र्पिरा॑सुतिः स॒र्पिरा॑सुति॒र् द्र्व॑न्नो॒ द्र्व॑न्नः स॒र्पिरा॑सुतिः प्र॒त्नः प्र॒त्नः स॒र्पिरा॑सुति॒र् द्र्व॑न्नो॒ द्र्व॑न्नः स॒र्पिरा॑सुतिः प्र॒त्नः । \newline
39. द्र्व॑न्न॒ इति॒ द्रु - अ॒न्नः॒ । \newline
40. स॒र्पिरा॑सुतिः प्र॒त्नः प्र॒त्नः स॒र्पिरा॑सुतिः स॒र्पिरा॑सुतिः प्र॒त्नो होता॒ होता᳚ प्र॒त्नः स॒र्पिरा॑सुतिः स॒र्पिरा॑सुतिः प्र॒त्नो होता᳚ । \newline
41. स॒र्पिरा॑सुति॒रिति॑ स॒र्पिः - आ॒सु॒तिः॒ । \newline
42. प्र॒त्नो होता॒ होता᳚ प्र॒त्नः प्र॒त्नो होता॒ वरे᳚ण्यो॒ वरे᳚ण्यो॒ होता᳚ प्र॒त्नः प्र॒त्नो होता॒ वरे᳚ण्यः । \newline
43. होता॒ वरे᳚ण्यो॒ वरे᳚ण्यो॒ होता॒ होता॒ वरे᳚ण्यः । \newline
44. वरे᳚ण्य॒ इति॒ वरे᳚ण्यः । \newline
45. सह॑स स्पु॒त्रः पु॒त्रः सह॑सः॒ सह॑स स्पु॒त्रो अद्भु॑तो॒ अद्भु॑तः पु॒त्रः सह॑सः॒ सह॑स स्पु॒त्रो अद्भु॑तः । \newline
46. पु॒त्रो अद्भु॑तो॒ अद्भु॑तः पु॒त्रः पु॒त्रो अद्भु॑तः । \newline
47. अद्भु॑त॒ इत्यद्भु॑तः । \newline
48. पर॑स्या॒ अध्यधि॒ पर॑स्याः॒ पर॑स्या॒ अधि॑ सं॒ॅवतः॑ सं॒ॅवतो॒ अधि॒ पर॑स्याः॒ पर॑स्या॒ अधि॑ सं॒ॅवतः॑ । \newline
49. अधि॑ सं॒ॅवतः॑ सं॒ॅवतो॒ अध्यधि॑ सं॒ॅवतो ऽव॑राꣳ॒॒ अव॑रान् थ्सं॒ॅवतो॒ अध्यधि॑ सं॒ॅवतो ऽव॑रान् । \newline
50. सं॒ॅवतो ऽव॑राꣳ॒॒ अव॑रान् थ्सं॒ॅवतः॑ सं॒ॅवतो ऽव॑राꣳ अ॒भ्य॑भ्य व॑रान् थ्सं॒ॅवतः॑ सं॒ॅवतो ऽव॑राꣳ अ॒भि । \newline
51. सं॒ॅवत॒ इति॑ सं - वतः॑ । \newline
52. अव॑राꣳ अ॒भ्य॑भ्य व॑राꣳ॒॒ अव॑राꣳ अ॒भ्याऽभ्य व॑राꣳ॒॒ अव॑राꣳ अ॒भ्या । \newline
53. अ॒भ्या ऽभ्य॑भ्या त॑र त॒रा ऽभ्य॑भ्या त॑र । \newline
54. आ त॑र त॒रा त॑र । \newline
\pagebreak
\markright{ TS 4.1.9.3  \hfill https://www.vedavms.in \hfill}

\section{ TS 4.1.9.3 }

\textbf{TS 4.1.9.3 } \newline
\textbf{Samhita Paata} \newline

त॑र । यत्रा॒हमस्मि॒ ताꣳ अ॑व ॥ प॒र॒मस्याः᳚ परा॒वतो॑ रो॒हिद॑श्व इ॒हाऽ*ग॑हि । पु॒री॒ष्यः॑ पुरुप्रि॒योऽग्ने॒ त्वं त॑रा॒ मृधः॑ ॥ सीद॒ त्वं मा॒तुर॒स्या उ॒पस्थे॒ विश्वा᳚न्यग्ने व॒युना॑नि वि॒द्वान् । मैना॑म॒र्चिषा॒ मा तप॑सा॒ऽभि शू॑शुचो॒ऽन्तर॑स्याꣳ शु॒क्र ज्यो॑ति॒र्वि भा॑हि ॥ अ॒न्तर॑ग्ने रु॒चा त्वमु॒खायै॒ सद॑ने॒ स्वे । तस्या॒स्त्वꣳ हर॑सा॒ तप॒न् ( ) जात॑वेदः शि॒वो भ॑व ॥ शि॒वो भू॒त्वा मह्य॑म॒ग्नेऽथो॑ सीद शि॒वस्त्वं । शि॒वाः कृ॒त्वा दिशः॒ सर्वाः॒ स्वां ॅयोनि॑मि॒हाऽऽस॑दः ॥ \newline

\textbf{Pada Paata} \newline

त॒र॒ ॥ यत्र॑ । अ॒हम् । अस्मि॑ । तान् । अ॒व॒ ॥ प॒र॒मस्याः᳚ । प॒रा॒वत॒ इति॑ परा - वतः॑ । रो॒हिद॑श्व॒ इति॑ रो॒हित् - अ॒श्वः॒ । इ॒ह । एति॑ । ग॒हि॒ ॥ पु॒री॒ष्यः॑ । पु॒रु॒प्रि॒य इति॑ पुरु - प्रि॒यः । अग्ने᳚ । त्वम् । त॒र॒ । मृधः॑ ॥ सीद॑ । त्वम् । मा॒तुः । अ॒स्याः । उ॒पस्थ॒ इत्यु॒प - स्थे॒ । विश्वा॑नि । अ॒ग्ने॒ । व॒युना॑नि । वि॒द्वान् ॥ मा । ए॒ना॒म् । अ॒र्चिषा᳚ । मा । तप॑सा । अ॒भीति॑ । शू॒शु॒चः॒ । अ॒न्तः । अ॒स्या॒म् । शु॒क्रज्यो॑ति॒रिति॑ शु॒क्र - ज्यो॒तिः॒ । वीति॑ । भा॒हि॒ ॥ अ॒न्तः । अ॒ग्ने॒ । रु॒चा । त्वम् । उ॒खायै᳚ । सद॑ने । स्वे ॥ तस्याः᳚ । त्वम् । हर॑सा । तपन्न्॑ ( ) । जात॑वेद॒ इति॒ जात॑-वे॒दः॒ । शि॒वः । भ॒व॒ ॥ शि॒वः । भू॒त्वा । मह्य᳚म् । अ॒ग्ने॒ । अथो॒ इति॑ । सी॒द॒ । शि॒वः । त्वम् ॥ शि॒वाः । कृ॒त्वा । दिशः॑ । सर्वाः᳚ । स्वाम् । योनि᳚म् । इ॒ह । एति॑ । अ॒स॒दः॒ ॥  \newline


\textbf{Krama Paata} \newline

त॒रेति॑ तर ॥ यत्रा॒ऽहम् । अ॒हमस्मि॑ । अस्मि॒ तान् । ताꣳ अ॑व । अ॒वेत्य॑व ॥ प॒र॒मस्याः᳚ परा॒वतः॑ । प॒रा॒वतो॑ रो॒हिद॑श्वः । प॒रा॒वत॒ इति॑ परा - वतः॑ । रो॒हिद॑श्व इ॒ह । रो॒हिद॑श्व॒ इति॑ रो॒हित् - अ॒श्वः॒ । इ॒हा । आ ग॑हि । ग॒हीति॑ गहि ॥ पु॒री॒ष्यः॑ पुरुप्रि॒यः । पु॒रु॒प्रि॒योऽग्ने᳚ । पु॒रु॒प्रि॒य इति॑ पुरु - प्रि॒यः । अग्ने॒ त्वम् । त्वम् त॑र । त॒रा॒ मृधः॑ । मृध॒ इति॒ मृधः॑ ॥ सीद॒ त्वम् । त्वम् मा॒तुः । मा॒तुर॒स्याः । अ॒स्या उ॒पस्थे᳚ । उ॒पस्थे॒ विश्वा॑नि । उ॒पस्थ॒ इत्यु॒प - स्थे॒ । विश्वा᳚न्यग्ने । अ॒ग्ने॒ व॒युना॑नि । व॒युना॑नि वि॒द्वान् । वि॒द्वानिति॑ वि॒द्वान् ॥ मैना᳚म् । ए॒ना॒म॒र्चिषा᳚ । अ॒र्चिषा॒ मा । मा तप॑सा । तप॑सा॒ऽभि । अ॒भि शू॑शुचः । शू॒शु॒चो॒ऽन्तः । अ॒न्तर॑स्याम् । अ॒स्याꣳ॒॒ शु॒क्रज्यो॑तिः । शु॒क्रज्यो॑ति॒र् वि । शु॒क्रज्यो॑ति॒रिति॑ शु॒क्र - ज्यो॒तिः॒ । वि भा॑हि । भा॒हीति॑ भाहि ॥ अ॒न्तर॑ग्ने । अ॒ग्ने॒ रु॒चा । रु॒चा त्वम् । त्वमु॒खायै᳚ । उ॒खायै॒ सद॑ने । सद॑ने॒ स्वे । स्व इति॒ स्वे ॥ तस्या॒स्त्वम् । त्वꣳ हर॑सा । हर॑सा॒ तपन्न्॑ ( ) । तप॒न् जात॑वेदः । जात॑वेदः शि॒वः । जात॑वेद॒ इति॒ जात॑ - वे॒दः॒ । शि॒वो भ॑व । भ॒वेति॑ भव ॥ शि॒वो भू॒त्वा । भू॒त्वा मह्य᳚म् । मह्य॑मग्ने । अ॒ग्नेऽथो᳚ । अथो॑ सीद । अथो॒ इत्यथो᳚ । सी॒द॒ शि॒वः । शि॒वस्त्वम् । त्वमिति॒ त्वम् ॥ शि॒वाः कृ॒त्वा । कृ॒त्वा दिशः॑ । दिशः॒ सर्वाः᳚ । सर्वाः॒ स्वाम् । स्वां ॅयोनि᳚म् । योनि॑मि॒ह । इ॒हा । आऽस॑दः । अ॒स॒द॒ इत्य॑सदः । \newline

\textbf{Jatai Paata} \newline

1. त॒रेति॑ तर । \newline
2. यत्रा॒ ऽह म॒हं ॅयत्र॒ यत्रा॒ ऽहम् । \newline
3. अ॒ह मस्म्य स्म्य॒ह म॒ह मस्मि॑ । \newline
4. अस्मि॒ ताꣳ स्ताꣳ अस्म्यस्मि॒ तान् । \newline
5. ताꣳ अ॑वाव॒ ताꣳ स्ताꣳ अ॑व । \newline
6. अ॒वेत्य॑व । \newline
7. प॒र॒मस्याः᳚ परा॒वतः॑ परा॒वतः॑ पर॒मस्याः᳚ पर॒मस्याः᳚ परा॒वतः॑ । \newline
8. प॒रा॒वतो॑ रो॒हिद॑श्वो रो॒हिद॑श्वः परा॒वतः॑ परा॒वतो॑ रो॒हिद॑श्वः । \newline
9. प॒रा॒वत॒ इति॑ परा - वतः॑ । \newline
10. रो॒हिद॑श्व इ॒हे ह रो॒हिद॑श्वो रो॒हिद॑श्व इ॒ह । \newline
11. रो॒हिद॑श्व॒ इति॑ रो॒हित् - अ॒श्वः॒ । \newline
12. इ॒हेहे हा । \newline
13. आ ग॑हि ग॒ह्या ग॑हि । \newline
14. ग॒हीति॑ गहि । \newline
15. पु॒री॒ष्यः॑ पुरुप्रि॒यः पु॑रुप्रि॒यः पु॑री॒ष्यः॑ पुरी॒ष्यः॑ पुरुप्रि॒यः । \newline
16. पु॒रु॒प्रि॒यो ऽग्ने ऽग्ने॑ पुरुप्रि॒यः पु॑रुप्रि॒यो ऽग्ने᳚ । \newline
17. पु॒रु॒प्रि॒य इति॑ पुरु - प्रि॒यः । \newline
18. अग्ने॒ त्वम् त्व मग्ने ऽग्ने॒ त्वम् । \newline
19. त्वम् त॑र तर॒ त्वम् त्वम् त॑र । \newline
20. त॒रा॒ मृधो॒ मृध॑ स्तर तरा॒ मृधः॑ । \newline
21. मृध॒ इति॒ मृधः॑ । \newline
22. सीद॒ त्वम् त्वꣳ सीद॒ सीद॒ त्वम् । \newline
23. त्वम् मा॒तुर् मा॒तु स्त्वम् त्वम् मा॒तुः । \newline
24. मा॒तु र॒स्या अ॒स्या मा॒तुर् मा॒तु र॒स्याः । \newline
25. अ॒स्या उ॒पस्थ॑ उ॒पस्थे॑ अ॒स्या अ॒स्या उ॒पस्थे᳚ । \newline
26. उ॒पस्थे॒ विश्वा॑नि॒ विश्वा᳚ न्यु॒पस्थ॑ उ॒पस्थे॒ विश्वा॑नि । \newline
27. उ॒पस्थ॒ इत्यु॒प - स्थे॒ । \newline
28. विश्वा᳚ न्यग्ने अग्ने॒ विश्वा॑नि॒ विश्वा᳚ न्यग्ने । \newline
29. अ॒ग्ने॒ व॒युना॑नि व॒युना᳚ न्यग्ने अग्ने व॒युना॑नि । \newline
30. व॒युना॑नि वि॒द्वान्. वि॒द्वान्. व॒युना॑नि व॒युना॑नि वि॒द्वान् । \newline
31. वि॒द्वानिति॑ वि॒द्वान् । \newline
32. मैना॑ मेना॒म् मा मैना᳚म् । \newline
33. ए॒ना॒ म॒र्चिषा॒ ऽर्चिषै॑ना मेना म॒र्चिषा᳚ । \newline
34. अ॒र्चिषा॒ मा मा ऽर्चिषा॒ ऽर्चिषा॒ मा । \newline
35. मा तप॑सा॒ तप॑सा॒ मा मा तप॑सा । \newline
36. तप॑सा॒ ऽभ्य॑भि तप॑सा॒ तप॑सा॒ ऽभि । \newline
37. अ॒भि शू॑शुचः शूशुचो अ॒भ्य॑भि शू॑शुचः । \newline
38. शू॒शु॒चो॒ ऽन्त र॒न्तः शू॑शुचः शूशुचो॒ ऽन्तः । \newline
39. अ॒न्त र॑स्या मस्या म॒न्त र॒न्त र॑स्याम् । \newline
40. अ॒स्याꣳ॒॒ शु॒क्रज्यो॑तिः शु॒क्रज्यो॑ति रस्या मस्याꣳ शु॒क्रज्यो॑तिः । \newline
41. शु॒क्रज्यो॑ति॒र् वि वि शु॒क्रज्यो॑तिः शु॒क्रज्यो॑ति॒र् वि । \newline
42. शु॒क्रज्यो॑ति॒रिति॑ शु॒क्र - ज्यो॒तिः॒ । \newline
43. वि भा॑हि भाहि॒ वि वि भा॑हि । \newline
44. भा॒हीति॑ भाहि । \newline
45. अ॒न्त र॑ग्ने अग्ने॑ अ॒न्त र॒न्त र॑ग्ने । \newline
46. अ॒ग्ने॒ रु॒चा रु॒चा ऽग्ने॑ अग्ने रु॒चा । \newline
47. रु॒चा त्वम् त्वꣳ रु॒चा रु॒चा त्वम् । \newline
48. त्व मु॒खाया॑ उ॒खायै॒ त्वम् त्व मु॒खायै᳚ । \newline
49. उ॒खायै॒ सद॑ने॒ सद॑न उ॒खाया॑ उ॒खायै॒ सद॑ने । \newline
50. सद॑ने॒ स्वे स्वे सद॑ने॒ सद॑ने॒ स्वे । \newline
51. स्व इति॒ स्वे । \newline
52. तस्या॒ स्त्वम् त्वम् तस्या॒ स्तस्या॒ स्त्वम् । \newline
53. त्वꣳ हर॑सा॒ हर॑सा॒ त्वम् त्वꣳ हर॑सा । \newline
54. हर॑सा॒ तप॒न् तप॒न्॒. हर॑सा॒ हर॑सा॒ तपन्न्॑ । \newline
55. तप॒न् जात॑वेदो॒ जात॑वेद॒ स्तप॒न् तप॒न् जात॑वेदः । \newline
56. जात॑वेदः शि॒वः शि॒वो जात॑वेदो॒ जात॑वेदः शि॒वः । \newline
57. जात॑वेद॒ इति॒ जात॑ - वे॒दः॒ । \newline
58. शि॒वो भ॑व भव शि॒वः शि॒वो भ॑व । \newline
59. भ॒वेति॑ भव । \newline
60. शि॒वो भू॒त्वा भू॒त्वा शि॒वः शि॒वो भू॒त्वा । \newline
61. भू॒त्वा मह्य॒म् मह्य॑म् भू॒त्वा भू॒त्वा मह्य᳚म् । \newline
62. मह्य॑ मग्ने अग्ने॒ मह्य॒म् मह्य॑ मग्ने । \newline
63. अ॒ग्ने ऽथो॒ अथो॑ अग्ने अ॒ग्ने ऽथो᳚ । \newline
64. अथो॑ सीद सी॒दाथो॒ अथो॑ सीद । \newline
65. अथो॒ इत्यथो᳚ । \newline
66. सी॒द॒ शि॒वः शि॒वः सी॑द सीद शि॒वः । \newline
67. शि॒व स्त्वम् त्वꣳ शि॒वः शि॒व स्त्वम् । \newline
68. त्वमिति॒ त्वम् । \newline
69. शि॒वाः कृ॒त्वा कृ॒त्वा शि॒वाः शि॒वाः कृ॒त्वा । \newline
70. कृ॒त्वा दिशो॒ दिशः॑ कृ॒त्वा कृ॒त्वा दिशः॑ । \newline
71. दिशः॒ सर्वाः॒ सर्वा॒ दिशो॒ दिशः॒ सर्वाः᳚ । \newline
72. सर्वाः॒ स्वाꣳ स्वाꣳ सर्वाः॒ सर्वाः॒ स्वाम् । \newline
73. स्वां ॅयोनिं॒ ॅयोनिꣳ॒॒ स्वाꣳ स्वां ॅयोनि᳚म् । \newline
74. योनि॑ मि॒हेह योनिं॒ ॅयोनि॑ मि॒ह । \newline
75. इ॒हेहे हा । \newline
76. आ ऽस॑दो असद॒ आ ऽस॑दः । \newline
77. अ॒स॒द॒ इत्य॑सदः । \newline

\textbf{Ghana Paata } \newline

1. त॒रेति॑ तर । \newline
2. यत्रा॒ ऽह म॒हं ॅयत्र॒ यत्रा॒ ऽह मस्म्य स्म्य॒हं ॅयत्र॒ यत्रा॒ ऽह मस्मि॑ । \newline
3. अ॒ह मस्म्य स्म्य॒ह म॒ह मस्मि॒ ताꣳ स्ताꣳ अस्म्य॒ह म॒ह मस्मि॒ तान् । \newline
4. अस्मि॒ ताꣳ स्ताꣳ अस्म्यस्मि॒ ताꣳ अ॑वाव॒ ताꣳ अस्म्यस्मि॒ ताꣳ अ॑व । \newline
5. ताꣳ अ॑वाव॒ ताꣳ स्ताꣳ अ॑व । \newline
6. अ॒वेत्य॑व । \newline
7. प॒र॒मस्याः᳚ परा॒वतः॑ परा॒वतः॑ पर॒मस्याः᳚ पर॒मस्याः᳚ परा॒वतो॑ रो॒हिद॑श्वो रो॒हिद॑श्वः परा॒वतः॑ पर॒मस्याः᳚ पर॒मस्याः᳚ परा॒वतो॑ रो॒हिद॑श्वः । \newline
8. प॒रा॒वतो॑ रो॒हिद॑श्वो रो॒हिद॑श्वः परा॒वतः॑ परा॒वतो॑ रो॒हिद॑श्व इ॒हेह रो॒हिद॑श्वः परा॒वतः॑ परा॒वतो॑ रो॒हिद॑श्व इ॒ह । \newline
9. प॒रा॒वत॒ इति॑ परा - वतः॑ । \newline
10. रो॒हिद॑श्व इ॒हेह रो॒हिद॑श्वो रो॒हिद॑श्व इ॒हेह रो॒हिद॑श्वो रो॒हिद॑श्व इ॒हा । \newline
11. रो॒हिद॑श्व॒ इति॑ रो॒हित् - अ॒श्वः॒ । \newline
12. इ॒हेहे हा ग॑हि ग॒ह्येहे हा ग॑हि । \newline
13. आ ग॑हि ग॒ह्या ग॑हि । \newline
14. ग॒हीति॑ गहि । \newline
15. पु॒री॒ष्यः॑ पुरुप्रि॒यः पु॑रुप्रि॒यः पु॑री॒ष्यः॑ पुरी॒ष्यः॑ पुरुप्रि॒यो ऽग्ने ऽग्ने॑ पुरुप्रि॒यः पु॑री॒ष्यः॑ पुरी॒ष्यः॑ पुरुप्रि॒यो ऽग्ने᳚ । \newline
16. पु॒रु॒प्रि॒यो ऽग्ने ऽग्ने॑ पुरुप्रि॒यः पु॑रुप्रि॒यो ऽग्ने॒ त्वम् त्व मग्ने॑ पुरुप्रि॒यः पु॑रुप्रि॒यो ऽग्ने॒ त्वम् । \newline
17. पु॒रु॒प्रि॒य इति॑ पुरु - प्रि॒यः । \newline
18. अग्ने॒ त्वम् त्व मग्ने ऽग्ने॒ त्वम् त॑र तर॒ त्व मग्ने ऽग्ने॒ त्वम् त॑र । \newline
19. त्वम् त॑र तर॒ त्वम् त्वम् त॑रा॒ मृधो॒ मृध॑ स्तर॒ त्वम् त्वम् त॑रा॒ मृधः॑ । \newline
20. त॒रा॒ मृधो॒ मृध॑ स्तर तरा॒ मृधः॑ । \newline
21. मृध॒ इति॒ मृधः॑ । \newline
22. सीद॒ त्वम् त्वꣳ सीद॒ सीद॒ त्वम् मा॒तुर् मा॒तु स्त्वꣳ सीद॒ सीद॒ त्वम् मा॒तुः । \newline
23. त्वम् मा॒तुर् मा॒तु स्त्वम् त्वम् मा॒तु र॒स्या अ॒स्या मा॒तु स्त्वम् त्वम् मा॒तु र॒स्याः । \newline
24. मा॒तु र॒स्या अ॒स्या मा॒तुर् मा॒तु र॒स्या उ॒पस्थ॑ उ॒पस्थे॑ अ॒स्या मा॒तुर् मा॒तु र॒स्या उ॒पस्थे᳚ । \newline
25. अ॒स्या उ॒पस्थ॑ उ॒पस्थे॑ अ॒स्या अ॒स्या उ॒पस्थे॒ विश्वा॑नि॒ विश्वा᳚ न्यु॒पस्थे॑ अ॒स्या अ॒स्या उ॒पस्थे॒ विश्वा॑नि । \newline
26. उ॒पस्थे॒ विश्वा॑नि॒ विश्वा᳚ न्यु॒पस्थ॑ उ॒पस्थे॒ विश्वा᳚न्यग्ने अग्ने॒ विश्वा᳚ न्यु॒पस्थ॑ उ॒पस्थे॒ विश्वा᳚न्यग्ने । \newline
27. उ॒पस्थ॒ इत्यु॒प - स्थे॒ । \newline
28. विश्वा᳚न्यग्ने अग्ने॒ विश्वा॑नि॒ विश्वा᳚न्यग्ने व॒युना॑नि व॒युना᳚न्यग्ने॒ विश्वा॑नि॒ विश्वा᳚न्यग्ने व॒युना॑नि । \newline
29. अ॒ग्ने॒ व॒युना॑नि व॒युना᳚न्यग्ने अग्ने व॒युना॑नि वि॒द्वान्. वि॒द्वान्. व॒युना᳚न्यग्ने अग्ने व॒युना॑नि वि॒द्वान् । \newline
30. व॒युना॑नि वि॒द्वान्. वि॒द्वान्. व॒युना॑नि व॒युना॑नि वि॒द्वान् । \newline
31. वि॒द्वानिति॑ वि॒द्वान् । \newline
32. मैना॑ मेना॒म् मा मैना॑ म॒र्चिषा॒ ऽर्चिषै॑ना॒म् मा मैना॑ म॒र्चिषा᳚ । \newline
33. ए॒ना॒ म॒र्चिषा॒ ऽर्चिषै॑ना मेना म॒र्चिषा॒ मा मा ऽर्चिषै॑ना मेना म॒र्चिषा॒ मा । \newline
34. अ॒र्चिषा॒ मा मा ऽर्चिषा॒ ऽर्चिषा॒ मा तप॑सा॒ तप॑सा॒ मा ऽर्चिषा॒ ऽर्चिषा॒ मा तप॑सा । \newline
35. मा तप॑सा॒ तप॑सा॒ मा मा तप॑सा॒ ऽभ्य॑भि तप॑सा॒ मा मा तप॑सा॒ ऽभि । \newline
36. तप॑सा॒ ऽभ्य॑भि तप॑सा॒ तप॑सा॒ ऽभि शू॑शुचः शूशुचो अ॒भि तप॑सा॒ तप॑सा॒ ऽभि शू॑शुचः । \newline
37. अ॒भि शू॑शुचः शूशुचो अ॒भ्य॑भि शू॑शुचो॒ ऽन्त र॒न्तः शू॑शुचो अ॒भ्य॑भि शू॑शुचो॒ ऽन्तः । \newline
38. शू॒शु॒चो॒ ऽन्त र॒न्तः शू॑शुचः शूशुचो॒ ऽन्त र॑स्या मस्या म॒न्तः शू॑शुचः शूशुचो॒ ऽन्त र॑स्याम् । \newline
39. अ॒न्त र॑स्या मस्या म॒न्त र॒न्त र॑स्याꣳ शु॒क्रज्यो॑तिः शु॒क्रज्यो॑ति रस्या म॒न्त र॒न्त र॑स्याꣳ शु॒क्रज्यो॑तिः । \newline
40. अ॒स्याꣳ॒॒ शु॒क्रज्यो॑तिः शु॒क्रज्यो॑ति रस्या मस्याꣳ शु॒क्रज्यो॑ति॒र् वि वि शु॒क्रज्यो॑ति रस्या मस्याꣳ शु॒क्रज्यो॑ति॒र् वि । \newline
41. शु॒क्रज्यो॑ति॒र् वि वि शु॒क्रज्यो॑तिः शु॒क्रज्यो॑ति॒र् वि भा॑हि भाहि॒ वि शु॒क्रज्यो॑तिः शु॒क्रज्यो॑ति॒र् वि भा॑हि । \newline
42. शु॒क्रज्यो॑ति॒रिति॑ शु॒क्र - ज्यो॒तिः॒ । \newline
43. वि भा॑हि भाहि॒ वि वि भा॑हि । \newline
44. भा॒हीति॑ भाहि । \newline
45. अ॒न्त र॑ग्ने अग्ने॑ अ॒न्त र॒न्त र॑ग्ने रु॒चा रु॒चा ऽग्ने॑ अ॒न्तर॒न्त र॑ग्ने रु॒चा । \newline
46. अ॒ग्ने॒ रु॒चा रु॒चा ऽग्ने॑ अग्ने रु॒चा त्वम् त्वꣳ रु॒चा ऽग्ने॑ अग्ने रु॒चा त्वम् । \newline
47. रु॒चा त्वम् त्वꣳ रु॒चा रु॒चा त्व मु॒खाया॑ उ॒खायै॒ त्वꣳ रु॒चा रु॒चा त्व मु॒खायै᳚ । \newline
48. त्व मु॒खाया॑ उ॒खायै॒ त्वम् त्व मु॒खायै॒ सद॑ने॒ सद॑न उ॒खायै॒ त्वम् त्व मु॒खायै॒ सद॑ने । \newline
49. उ॒खायै॒ सद॑ने॒ सद॑न उ॒खाया॑ उ॒खायै॒ सद॑ने॒ स्वे स्वे सद॑न उ॒खाया॑ उ॒खायै॒ सद॑ने॒ स्वे । \newline
50. सद॑ने॒ स्वे स्वे सद॑ने॒ सद॑ने॒ स्वे । \newline
51. स्व इति॒ स्वे । \newline
52. तस्या॒ स्त्वम् त्वम् तस्या॒ स्तस्या॒ स्त्वꣳ हर॑सा॒ हर॑सा॒ त्वम् तस्या॒ स्तस्या॒ स्त्वꣳ हर॑सा । \newline
53. त्वꣳ हर॑सा॒ हर॑सा॒ त्वम् त्वꣳ हर॑सा॒ तप॒न् तप॒न्॒. हर॑सा॒ त्वम् त्वꣳ हर॑सा॒ तपन्न्॑ । \newline
54. हर॑सा॒ तप॒न् तप॒न्॒. हर॑सा॒ हर॑सा॒ तप॒न् जात॑वेदो॒ जात॑वेद॒ स्तप॒न्॒. हर॑सा॒ हर॑सा॒ तप॒न् जात॑वेदः । \newline
55. तप॒न् जात॑वेदो॒ जात॑वेद॒ स्तप॒न् तप॒न् जात॑वेदः शि॒वः शि॒वो जात॑वेद॒ स्तप॒न् तप॒न् जात॑वेदः शि॒वः । \newline
56. जात॑वेदः शि॒वः शि॒वो जात॑वेदो॒ जात॑वेदः शि॒वो भ॑व भव शि॒वो जात॑वेदो॒ जात॑वेदः शि॒वो भ॑व । \newline
57. जात॑वेद॒ इति॒ जात॑ - वे॒दः॒ । \newline
58. शि॒वो भ॑व भव शि॒वः शि॒वो भ॑व । \newline
59. भ॒वेति॑ भव । \newline
60. शि॒वो भू॒त्वा भू॒त्वा शि॒वः शि॒वो भू॒त्वा मह्य॒म् मह्य॑म् भू॒त्वा शि॒वः शि॒वो भू॒त्वा मह्य᳚म् । \newline
61. भू॒त्वा मह्य॒म् मह्य॑म् भू॒त्वा भू॒त्वा मह्य॑ मग्ने अग्ने॒ मह्य॑म् भू॒त्वा भू॒त्वा मह्य॑ मग्ने । \newline
62. मह्य॑ मग्ने अग्ने॒ मह्य॒म् मह्य॑ म॒ग्ने ऽथो॒ अथो॑ अग्ने॒ मह्य॒म् मह्य॑ म॒ग्ने ऽथो᳚ । \newline
63. अ॒ग्ने ऽथो॒ अथो॑ अग्ने अ॒ग्ने ऽथो॑ सीद सी॒दाथो॑ अग्ने अ॒ग्ने ऽथो॑ सीद । \newline
64. अथो॑ सीद सी॒दाथो॒ अथो॑ सीद शि॒वः शि॒वः सी॒दाथो॒ अथो॑ सीद शि॒वः । \newline
65. अथो॒ इत्यथो᳚ । \newline
66. सी॒द॒ शि॒वः शि॒वः सी॑द सीद शि॒व स्त्वम् त्वꣳ शि॒वः सी॑द सीद शि॒व स्त्वम् । \newline
67. शि॒व स्त्वम् त्वꣳ शि॒वः शि॒व स्त्वम् । \newline
68. त्वमिति॒ त्वम् । \newline
69. शि॒वाः कृ॒त्वा कृ॒त्वा शि॒वाः शि॒वाः कृ॒त्वा दिशो॒ दिशः॑ कृ॒त्वा शि॒वाः शि॒वाः कृ॒त्वा दिशः॑ । \newline
70. कृ॒त्वा दिशो॒ दिशः॑ कृ॒त्वा कृ॒त्वा दिशः॒ सर्वाः॒ सर्वा॒ दिशः॑ कृ॒त्वा कृ॒त्वा दिशः॒ सर्वाः᳚ । \newline
71. दिशः॒ सर्वाः॒ सर्वा॒ दिशो॒ दिशः॒ सर्वाः॒ स्वाꣳ स्वाꣳ सर्वा॒ दिशो॒ दिशः॒ सर्वाः॒ स्वाम् । \newline
72. सर्वाः॒ स्वाꣳ स्वाꣳ सर्वाः॒ सर्वाः॒ स्वां ॅयोनिं॒ ॅयोनिꣳ॒॒ स्वाꣳ सर्वाः॒ सर्वाः॒ स्वां ॅयोनि᳚म् । \newline
73. स्वां ॅयोनिं॒ ॅयोनिꣳ॒॒ स्वाꣳ स्वां ॅयोनि॑ मि॒हेह योनिꣳ॒॒ स्वाꣳ स्वां ॅयोनि॑ मि॒ह । \newline
74. योनि॑ मि॒हेह योनिं॒ ॅयोनि॑ मि॒हेह योनिं॒ ॅयोनि॑ मि॒हा । \newline
75. इ॒हेहेहा ऽस॑दो असद॒ एहेहा ऽस॑दः । \newline
76. आ ऽस॑दो असद॒ आ ऽस॑दः । \newline
77. अ॒स॒द॒ इत्य॑सदः । \newline
\pagebreak
\markright{ TS 4.1.10.1  \hfill https://www.vedavms.in \hfill}

\section{ TS 4.1.10.1 }

\textbf{TS 4.1.10.1 } \newline
\textbf{Samhita Paata} \newline

यद॑ग्ने॒ यानि॒ कानि॒ चाऽऽते॒ दारू॑णि द॒द्ध्मसि॑ । तद॑स्तु॒ तुभ्य॒मिद्-घृ॒तं तज्जु॑षस्व यविष्ठ्य ॥ यदत्त्यु॑प॒जिह्वि॑का॒ यद्व॒म्रो अ॑ति॒सर्प॑ति । सर्वं॒ तद॑स्तु ते घृ॒तं तज्जु॑षस्व यविष्ठ्य ॥ रात्रिꣳ॑ रात्रि॒मप्र॑यावं॒ भर॒न्तोऽश्वा॑येव॒ तिष्ठ॑ते घा॒सम॑स्मै । रा॒यस्पोषे॑ण॒ समि॒षा मद॒न्तोऽग्ने॒ मा ते॒ प्रति॑वेशा रिषाम ॥ नाभा॑ - [  ] \newline

\textbf{Pada Paata} \newline

यत् । अ॒ग्ने॒ । यानि॑ । कानि॑ । च॒ । एति॑ । ते॒ । दारू॑णि । द॒द्ध्मसि॑ ॥ तत् । अ॒स्तु॒ । तुभ्य᳚म् । इत् । घृ॒तम् । तत् । जु॒ष॒स्व॒ । य॒वि॒ष्ठ्य॒ ॥ यत् । अत्ति॑ । उ॒प॒जिह्वि॒केत्यु॑प - जिह्वि॑का । यत् । व॒म्रः । अ॒ति॒सर्प॒तीत्य॑ति-सर्प॑ति ॥ सर्व᳚म् । तत् । अ॒स्तु॒ । ते॒ । घृ॒तम् । तत् । जु॒ष॒स्व॒ । य॒वि॒ष्ठ्य॒ ॥ रात्रिꣳ॑रात्रि॒मिति॒ रात्रि᳚म् - रा॒त्रि॒म् । अप्र॑याव॒मित्यप्र॑-या॒व॒म् । भर॑न्तः । अश्वा॑य । इ॒व॒ । तिष्ठ॑ते । घा॒सम् । अ॒स्मै॒ ॥ रा॒यः । पोषे॑ण । समिति॑ । इ॒षा । मद॑न्तः । अग्ने᳚ । मा । ते॒ । प्रति॑वेशा॒ इति॒ प्रति॑ - वे॒शाः॒ । रि॒षा॒म॒ ॥ नाभा᳚ ।  \newline


\textbf{Krama Paata} \newline

यद॑ग्ने । अ॒ग्ने॒ यानि॑ । यानि॒ कानि॑ । कानि॑ च । चा । आ ते᳚ । ते॒ दारू॑णि । दारू॑णि द॒द्ध्मसि॑ । द॒द्ध्मसीति॑ द॒द्ध्मसि॑ ॥ तद॑स्तु । अ॒स्तु॒ तुभ्य᳚म् । तुभ्य॒मित् । इद् घृ॒तम् । घृ॒तम् तत् । तज् जु॑षस्व । जु॒ष॒स्व॒ य॒वि॒ष्ठ्य॒ । य॒वि॒ष्ठ्येति॑ यविष्ठ्य ॥ यदत्ति॑ । अत्त्यु॑प॒जिह्वि॑का । उ॒प॒जिह्वि॑का॒ यत् । उ॒प॒जिह्वि॒केत्यु॑प - जिह्वि॑का । यद् व॒म्रः । व॒म्रो अ॑ति॒सर्प॑ति । अ॒ति॒सर्प॒तीत्य॑ति - सर्प॑ति ॥ सर्व॒म् तत् । तद॑स्तु । अ॒स्तु॒ ते॒ । ते॒ घृ॒तम् । घृ॒तम् तत् । तज् जु॑षस्व । जु॒ष॒स्व॒ य॒वि॒ष्ठ्य॒ । य॒वि॒ष्ठ्येति॑ यविष्ठ्य ॥ रात्रिꣳ॑रात्रि॒मप्र॑यावम् । रात्रिꣳ॑रात्रि॒मिति॒ रात्रि᳚म् - रा॒त्रि॒म् । अप्र॑याव॒म् भर॑न्तः । अप्र॑याव॒मित्यप्र॑ - या॒व॒म् । भर॒न्तोऽश्वा॑य । अश्वा॑येव । इ॒व॒ तिष्ठ॑ते । तिष्ठ॑ते घा॒सम् । घा॒सम॑स्मै । अ॒स्मा॒ इत्य॑स्मै ॥ रा॒यस्पोषे॑ण । पोषे॑ण॒ सम् । समि॒षा । इ॒षा मद॑न्तः । मद॒न्तोऽग्ने᳚ । अग्ने॒ मा । मा ते᳚ । ते॒ प्रति॑वेशाः । प्रति॑वेशा रिषाम । प्रति॑वेशा॒ इति॒ प्रति॑ - वे॒शाः॒ । रि॒षा॒मेति॑ रिषाम ॥ नाभा॑ पृथि॒व्याः \newline

\textbf{Jatai Paata} \newline

1. यद॑ग्ने अग्ने॒ यद् यद॑ग्ने । \newline
2. अ॒ग्ने॒ यानि॒ यान्य॑ग्ने अग्ने॒ यानि॑ । \newline
3. यानि॒ कानि॒ कानि॒ यानि॒ यानि॒ कानि॑ । \newline
4. कानि॑ च च॒ कानि॒ कानि॑ च । \newline
5. चा च॒ चा । \newline
6. आ ते॑ त॒ आ ते᳚ । \newline
7. ते॒ दारू॑णि॒ दारू॑णि ते ते॒ दारू॑णि । \newline
8. दारू॑णि द॒द्ध्मसि॑ द॒द्ध्मसि॒ दारू॑णि॒ दारू॑णि द॒द्ध्मसि॑ । \newline
9. द॒द्ध्मसीति॑ द॒द्ध्मसि॑ । \newline
10. तद॑स्त्वस्तु॒ तत् तद॑स्तु । \newline
11. अ॒स्तु॒ तुभ्य॒म् तुभ्य॑ मस्त्वस्तु॒ तुभ्य᳚म् । \newline
12. तुभ्य॒ मिदित् तुभ्य॒म् तुभ्य॒ मित् । \newline
13. इद् घृ॒तम् घृ॒त मिदिद् घृ॒तम् । \newline
14. घृ॒तम् तत् तद् घृ॒तम् घृ॒तम् तत् । \newline
15. तज् जु॑षस्व जुषस्व॒ तत् तज् जु॑षस्व । \newline
16. जु॒ष॒स्व॒ य॒वि॒ष्ठ्य॒ य॒वि॒ष्ठ्य॒ जु॒ष॒स्व॒ जु॒ष॒स्व॒ य॒वि॒ष्ठ्य॒ । \newline
17. य॒वि॒ष्ठ्येति॑ यविष्ठ्य । \newline
18. यदत्त्यत्ति॒ यद् यदत्ति॑ । \newline
19. अत्त्यु॑प॒जिह्वि॑ कोप॒जिह्वि॒का ऽत्त्य त्त्यु॑प॒जिह्वि॑का । \newline
20. उ॒प॒जिह्वि॑का॒ यद् यदु॑प॒जिह्वि॑ कोप॒जिह्वि॑का॒ यत् । \newline
21. उ॒प॒जिह्वि॒केत्यु॑प - जिह्वि॑का । \newline
22. यद् व॒म्रो व॒म्रो यद् यद् व॒म्रः । \newline
23. व॒म्रो अ॑ति॒सर्प॑ त्यति॒सर्प॑ति व॒म्रो व॒म्रो अ॑ति॒सर्प॑ति । \newline
24. अ॒ति॒सर्प॒तीत्य॑ति - सर्प॑ति । \newline
25. सर्व॒म् तत् तथ् सर्वꣳ॒॒ सर्व॒म् तत् । \newline
26. तद॑स्त्वस्तु॒ तत् तद॑स्तु । \newline
27. अ॒स्तु॒ ते॒ ते॒ अ॒स्त्व॒स्तु॒ ते॒ । \newline
28. ते॒ घृ॒तम् घृ॒तम् ते॑ ते घृ॒तम् । \newline
29. घृ॒तम् तत् तद् घृ॒तम् घृ॒तम् तत् । \newline
30. तज् जु॑षस्व जुषस्व॒ तत् तज् जु॑षस्व । \newline
31. जु॒ष॒स्व॒ य॒वि॒ष्ठ्य॒ य॒वि॒ष्ठ्य॒ जु॒ष॒स्व॒ जु॒ष॒स्व॒ य॒वि॒ष्ठ्य॒ । \newline
32. य॒वि॒ष्ठ्येति॑ यविष्ठ्य । \newline
33. रात्रिꣳ॑रात्रि॒ मप्र॑याव॒ मप्र॑यावꣳ॒॒ रात्रिꣳ॑रात्रिꣳ॒॒ रात्रिꣳ॑रात्रि॒ मप्र॑यावम् । \newline
34. रात्रिꣳ॑रात्रि॒मिति॒ रात्रि᳚म् - रा॒त्रि॒म् । \newline
35. अप्र॑याव॒म् भर॑न्तो॒ भर॑न्तो॒ अप्र॑याव॒ मप्र॑याव॒म् भर॑न्तः । \newline
36. अप्र॑याव॒मित्यप्र॑ - या॒व॒म् । \newline
37. भर॒न्तो ऽश्वा॒या श्वा॑य॒ भर॑न्तो॒ भर॒न्तो ऽश्वा॑य । \newline
38. अश्वा॑ये वे॒ वाश्वा॒या श्वा॑ये व । \newline
39. इ॒व॒ तिष्ठ॑ते॒ तिष्ठ॑त इवे व॒ तिष्ठ॑ते । \newline
40. तिष्ठ॑ते घा॒सम् घा॒सम् तिष्ठ॑ते॒ तिष्ठ॑ते घा॒सम् । \newline
41. घा॒स म॑स्मा अस्मै घा॒सम् घा॒स म॑स्मै । \newline
42. अ॒स्मा॒ इत्य॑स्मै । \newline
43. रा॒य स्पोषे॑ण॒ पोषे॑ण रा॒यो रा॒य स्पोषे॑ण । \newline
44. पोषे॑ण॒ सꣳ सम् पोषे॑ण॒ पोषे॑ण॒ सम् । \newline
45. स मि॒षेषा सꣳ स मि॒षा । \newline
46. इ॒षा मद॑न्तो॒ मद॑न्त इ॒षेषा मद॑न्तः । \newline
47. मद॒न्तो ऽग्ने ऽग्ने॒ मद॑न्तो॒ मद॒न्तो ऽग्ने᳚ । \newline
48. अग्ने॒ मा मा ऽग्ने ऽग्ने॒ मा । \newline
49. मा ते॑ ते॒ मा मा ते᳚ । \newline
50. ते॒ प्रति॑वेशाः॒ प्रति॑वेशा स्ते ते॒ प्रति॑वेशाः । \newline
51. प्रति॑वेशा रिषाम रिषाम॒ प्रति॑वेशाः॒ प्रति॑वेशा रिषाम । \newline
52. प्रति॑वेशा॒ इति॒ प्रति॑ - वे॒शाः॒ । \newline
53. रि॒षा॒मेति॑ रिषाम । \newline
54. नाभा॑ पृथि॒व्याः पृ॑थि॒व्या नाभा॒ नाभा॑ पृथि॒व्याः । \newline

\textbf{Ghana Paata } \newline

1. यद॑ग्ने अग्ने॒ यद् यद॑ग्ने॒ यानि॒ यान्य॑ग्ने॒ यद् यद॑ग्ने॒ यानि॑ । \newline
2. अ॒ग्ने॒ यानि॒ यान्य॑ग्ने अग्ने॒ यानि॒ कानि॒ कानि॒ यान्य॑ग्ने अग्ने॒ यानि॒ कानि॑ । \newline
3. यानि॒ कानि॒ कानि॒ यानि॒ यानि॒ कानि॑ च च॒ कानि॒ यानि॒ यानि॒ कानि॑ च । \newline
4. कानि॑ च च॒ कानि॒ कानि॒ चा च॒ कानि॒ कानि॒ चा । \newline
5. चा च॒ चा ते॑ त॒ आ च॒ चा ते᳚ । \newline
6. आ ते॑ त॒ आ ते॒ दारू॑णि॒ दारू॑णि त॒ आ ते॒ दारू॑णि । \newline
7. ते॒ दारू॑णि॒ दारू॑णि ते ते॒ दारू॑णि द॒द्ध्मसि॑ द॒द्ध्मसि॒ दारू॑णि ते ते॒ दारू॑णि द॒द्ध्मसि॑ । \newline
8. दारू॑णि द॒द्ध्मसि॑ द॒द्ध्मसि॒ दारू॑णि॒ दारू॑णि द॒द्ध्मसि॑ । \newline
9. द॒द्ध्मसीति॑ द॒द्ध्मसि॑ । \newline
10. तद॑ स्त्वस्तु॒ तत् तद॑स्तु॒ तुभ्य॒म् तुभ्य॑ मस्तु॒ तत् तद॑स्तु॒ तुभ्य᳚म् । \newline
11. अ॒स्तु॒ तुभ्य॒म् तुभ्य॑ मस्त्वस्तु॒ तुभ्य॒ मिदित् तुभ्य॑ मस्त्वस्तु॒ तुभ्य॒ मित् । \newline
12. तुभ्य॒ मिदित् तुभ्य॒म् तुभ्य॒ मिद् घृ॒तम् घृ॒त मित् तुभ्य॒म् तुभ्य॒ मिद् घृ॒तम् । \newline
13. इद् घृ॒तम् घृ॒त मिदिद् घृ॒तम् तत् तद् घृ॒त मिदिद् घृ॒तम् तत् । \newline
14. घृ॒तम् तत् तद् घृ॒तम् घृ॒तम् तज् जु॑षस्व जुषस्व॒ तद् घृ॒तम् घृ॒तम् तज् जु॑षस्व । \newline
15. तज् जु॑षस्व जुषस्व॒ तत् तज् जु॑षस्व यविष्ठ्य यविष्ठ्य जुषस्व॒ तत् तज् जु॑षस्व यविष्ठ्य । \newline
16. जु॒ष॒स्व॒ य॒वि॒ष्ठ्य॒ य॒वि॒ष्ठ्य॒ जु॒ष॒स्व॒ जु॒ष॒स्व॒ य॒वि॒ष्ठ्य॒ । \newline
17. य॒वि॒ष्ठ्येति॑ यविष्ठ्य । \newline
18. यदत् त्यत्ति॒ यद् यदत् त्यु॑प॒जिह्वि॑को प॒जिह्वि॒का ऽत्ति॒ यद् यदत् त्यु॑प॒जिह्वि॑का । \newline
19. अत्त्यु॑प॒जिह्वि॑को प॒जिह्वि॒का ऽत्त्यत् त्यु॑प॒जिह्वि॑का॒ यद् यदु॑प॒जिह्वि॒का ऽत्त्यत्त्यु॑ प॒जिह्वि॑का॒ यत् । \newline
20. उ॒प॒जिह्वि॑का॒ यद् यदु॑प॒जिह्वि॑को प॒जिह्वि॑का॒ यद् व॒म्रो व॒म्रो यदु॑प॒जिह्वि॑को प॒जिह्वि॑का॒ यद् व॒म्रः । \newline
21. उ॒प॒जिह्वि॒केत्यु॑प - जिह्वि॑का । \newline
22. यद् व॒म्रो व॒म्रो यद् यद् व॒म्रो अ॑ति॒सर्प॑ त्यति॒सर्प॑ति व॒म्रो यद् यद् व॒म्रो अ॑ति॒सर्प॑ति । \newline
23. व॒म्रो अ॑ति॒सर्प॑ त्यति॒सर्प॑ति व॒म्रो व॒म्रो अ॑ति॒सर्प॑ति । \newline
24. अ॒ति॒सर्प॒तीत्य॑ति - सर्प॑ति । \newline
25. सर्व॒म् तत् तथ् सर्वꣳ॒॒ सर्व॒म् तद॑स्त्वस्तु॒ तथ् सर्वꣳ॒॒ सर्व॒म् तद॑स्तु । \newline
26. तद॑स्त्वस्तु॒ तत् तद॑स्तु ते ते अस्तु॒ तत् तद॑स्तु ते । \newline
27. अ॒स्तु॒ ते॒ ते॒ अ॒स्त्व॒स्तु॒ ते॒ घृ॒तम् घृ॒तम् ते॑ अस्त्वस्तु ते घृ॒तम् । \newline
28. ते॒ घृ॒तम् घृ॒तम् ते॑ ते घृ॒तम् तत् तद् घृ॒तम् ते॑ ते घृ॒तम् तत् । \newline
29. घृ॒तम् तत् तद् घृ॒तम् घृ॒तम् तज् जु॑षस्व जुषस्व॒ तद् घृ॒तम् घृ॒तम् तज् जु॑षस्व । \newline
30. तज् जु॑षस्व जुषस्व॒ तत् तज् जु॑षस्व यविष्ठ्य यविष्ठ्य जुषस्व॒ तत् तज् जु॑षस्व यविष्ठ्य । \newline
31. जु॒ष॒स्व॒ य॒वि॒ष्ठ्य॒ य॒वि॒ष्ठ्य॒ जु॒ष॒स्व॒ जु॒ष॒स्व॒ य॒वि॒ष्ठ्य॒ । \newline
32. य॒वि॒ष्ठ्येति॑ यविष्ठ्य । \newline
33. रात्रिꣳ॑रात्रि॒ मप्र॑याव॒ मप्र॑यावꣳ॒॒ रात्रिꣳ॑रात्रिꣳ॒॒ रात्रिꣳ॑रात्रि॒ मप्र॑याव॒म् भर॑न्तो॒ भर॑न्तो॒ अप्र॑यावꣳ॒॒ रात्रिꣳ॑रात्रिꣳ॒॒ रात्रिꣳ॑रात्रि॒ मप्र॑याव॒म् भर॑न्तः । \newline
34. रात्रिꣳ॑रात्रि॒मिति॒ रात्रि᳚म् - रा॒त्रि॒म् । \newline
35. अप्र॑याव॒म् भर॑न्तो॒ भर॑न्तो॒ अप्र॑याव॒ मप्र॑याव॒म् भर॒न्तो ऽश्वा॒या श्वा॑य॒ भर॑न्तो॒ अप्र॑याव॒ मप्र॑याव॒म् भर॒न्तो ऽश्वा॑य । \newline
36. अप्र॑याव॒मित्यप्र॑ - या॒व॒म् । \newline
37. भर॒न्तो ऽश्वा॒या श्वा॑य॒ भर॑न्तो॒ भर॒न्तो ऽश्वा॑येवे॒ वाश्वा॑य॒ भर॑न्तो॒ भर॒न्तो ऽश्वा॑येव । \newline
38. अश्वा॑येवे॒ वाश्वा॒ याश्वा॑येव॒ तिष्ठ॑ते॒ तिष्ठ॑त इ॒वाश्वा॒ याश्वा॑येव॒ तिष्ठ॑ते । \newline
39. इ॒व॒ तिष्ठ॑ते॒ तिष्ठ॑त इवेव॒ तिष्ठ॑ते घा॒सम् घा॒सम् तिष्ठ॑त इवेव॒ तिष्ठ॑ते घा॒सम् । \newline
40. तिष्ठ॑ते घा॒सम् घा॒सम् तिष्ठ॑ते॒ तिष्ठ॑ते घा॒स म॑स्मा अस्मै घा॒सम् तिष्ठ॑ते॒ तिष्ठ॑ते घा॒स म॑स्मै । \newline
41. घा॒स म॑स्मा अस्मै घा॒सम् घा॒स म॑स्मै । \newline
42. अ॒स्मा॒ इत्य॑स्मै । \newline
43. रा॒य स्पोषे॑ण॒ पोषे॑ण रा॒यो रा॒य स्पोषे॑ण॒ सꣳ सम् पोषे॑ण रा॒यो रा॒य स्पोषे॑ण॒ सम् । \newline
44. पोषे॑ण॒ सꣳ सम् पोषे॑ण॒ पोषे॑ण॒ स मि॒षेषा सम् पोषे॑ण॒ पोषे॑ण॒ स मि॒षा । \newline
45. स मि॒षेषा सꣳ स मि॒षा मद॑न्तो॒ मद॑न्त इ॒षा सꣳ स मि॒षा मद॑न्तः । \newline
46. इ॒षा मद॑न्तो॒ मद॑न्त इ॒षेषा मद॒न्तो ऽग्ने ऽग्ने॒ मद॑न्त इ॒षेषा मद॒न्तो ऽग्ने᳚ । \newline
47. मद॒न्तो ऽग्ने ऽग्ने॒ मद॑न्तो॒ मद॒न्तो ऽग्ने॒ मा मा ऽग्ने॒ मद॑न्तो॒ मद॒न्तो ऽग्ने॒ मा । \newline
48. अग्ने॒ मा मा ऽग्ने ऽग्ने॒ मा ते॑ ते॒ मा ऽग्ने ऽग्ने॒ मा ते᳚ । \newline
49. मा ते॑ ते॒ मा मा ते॒ प्रति॑वेशाः॒ प्रति॑वेशा स्ते॒ मा मा ते॒ प्रति॑वेशाः । \newline
50. ते॒ प्रति॑वेशाः॒ प्रति॑वेशा स्ते ते॒ प्रति॑वेशा रिषाम रिषाम॒ प्रति॑वेशा स्ते ते॒ प्रति॑वेशा रिषाम । \newline
51. प्रति॑वेशा रिषाम रिषाम॒ प्रति॑वेशाः॒ प्रति॑वेशा रिषाम । \newline
52. प्रति॑वेशा॒ इति॒ प्रति॑ - वे॒शाः॒ । \newline
53. रि॒षा॒मेति॑ रिषाम । \newline
54. नाभा॑ पृथि॒व्याः पृ॑थि॒व्या नाभा॒ नाभा॑ पृथि॒व्याः स॑मिधा॒नꣳ स॑मिधा॒नम् पृ॑थि॒व्या नाभा॒ नाभा॑ पृथि॒व्याः स॑मिधा॒नम् । \newline
\pagebreak
\markright{ TS 4.1.10.2  \hfill https://www.vedavms.in \hfill}

\section{ TS 4.1.10.2 }

\textbf{TS 4.1.10.2 } \newline
\textbf{Samhita Paata} \newline

पृथि॒व्याः स॑मिधा॒-नम॒ग्निꣳ रा॒यस्पोषा॑य बृह॒ते ह॑वामहे । इ॒र॒म्म॒दं बृ॒हदु॑क्थं॒ ॅयज॑त्रं॒ जेता॑रम॒ग्निं पृत॑नासु सास॒हिं ॥ याः सेना॑ अ॒भीत्व॑रीरा-व्या॒धिनी॒-रुग॑णा उ॒त । ये स्ते॒ना ये च॒ तस्क॑रा॒स्ताꣳस्ते॑ अ॒ग्नेऽपि॑ दधाम्या॒स्ये᳚ ॥ दꣳष्ट्रा᳚भ्यां म॒लिम्लू॒न् जंभ्यै॒-स्तस्क॑राꣳ उ॒त । हनू᳚भ्याꣳ स्ते॒नान्-भ॑गव॒स्ताꣳ-स्त्वं खा॑द॒ सुखा॑दितान् ॥ ये जने॑षु म॒लिम्ल॑वः स्ते॒नास॒-स्तस्क॑रा॒ वने᳚ । ये - [  ] \newline

\textbf{Pada Paata} \newline

पृ॒थि॒व्याः । स॒मि॒धा॒नमिति॑ सं-इ॒धा॒नम् । अ॒ग्निम् । रा॒यः । पोषा॑य । बृ॒ह॒ते । ह॒वा॒म॒हे॒ ॥ इ॒र॒मं॒दमिती॑रं - म॒दम् । बृ॒हदु॑क्थ॒मिति॑ बृ॒हत् - उ॒क्थ॒म् । यज॑त्रम् । जेता॑रम् । अ॒ग्निम् । पृत॑नासु । सा॒स॒हिम् ॥ याः । सेनाः᳚ । अ॒भीत्व॑री॒रित्य॑भि - इत्व॑रीः । आ॒व्या॒धिनी॒रित्या᳚ - व्या॒धिनीः᳚ । उग॑णाः । उ॒त ॥ ये । स्ते॒नाः । ये । च॒ । तस्क॑राः । तान् । ते॒ । अ॒ग्ने॒ । अपीति॑ । द॒धा॒मि॒ । आ॒स्ये᳚ ॥ दꣳष्ट्रा᳚भ्याम् । म॒लिम्लून्न्॑ । जंभ्यैः᳚ । तस्क॑रान् । उ॒त ॥ हनू᳚भ्या॒मिति॒ हनु॑ - भ्या॒म् । स्ते॒नान् । भ॒ग॒व॒ इति॑ भग - वः॒ । तान् । त्वम् । खा॒द॒ । सुखा॑दिता॒निति॒ सु - खा॒दि॒ता॒न् ॥ ये । जने॑षु । म॒लिम्ल॑वः । स्ते॒नासः॑ । तस्क॑राः । वने᳚ ॥ ये ।  \newline


\textbf{Krama Paata} \newline

पृ॒थि॒व्याः स॑मिधा॒नम् । स॒मि॒धा॒नम॒ग्निम् । स॒मि॒धा॒नमिति॑ सम् - इ॒धा॒नम् । अ॒ग्निꣳ रा॒यः । रा॒यस्पोषा॑य । पोषा॑य बृह॒ते । बृ॒ह॒ते ह॑वामहे । ह॒वा॒म॒ह॒ इति॑ हवामहे ॥ इ॒र॒म्म॒दम् बृ॒हदु॑क्थम् । इ॒र॒म्म॒दमिती॑रम् - म॒दम् । बृ॒हदु॑क्थं॒ ॅयज॑त्रम् । बृ॒हदु॑क्थ॒मिति॑ बृ॒हत् - उ॒क्थ॒म् । यज॑त्र॒म् जेता॑रम् । जेता॑रम॒ग्निम् । अ॒ग्निम् पृत॑नासु । पृत॑नासु सास॒हिम् । सा॒स॒हिमिति॑ सास॒हिम् ॥ याः सेनाः᳚ । सेना॑ अ॒भीत्व॑रीः । अ॒भीत्व॑रीराव्या॒धिनीः᳚ । अ॒भीत्व॑री॒रित्य॑भि - इत्व॑रीः । आ॒व्या॒धिनी॒रुग॑णाः । आ॒व्या॒धिनी॒रित्या᳚ - व्या॒धिनीः᳚ । उग॑णा उ॒त । उ॒तेत्यु॒त ॥ ये स्ते॒नाः । स्ते॒ना ये । ये च॑ । च॒ तस्क॑राः । तस्क॑रा॒ स्तान् । ताꣳस्ते᳚ । ते॒ अ॒ग्ने॒ । अ॒ग्नेऽपि॑ । अपि॑ दधामि । द॒धा॒म्या॒स्ये᳚ । आ॒स्य॑ इत्या॒स्ये᳚ ॥ दꣳष्ट्रा᳚भ्याम्म॒लिम्लून्॑ । म॒लिम्लू॒न् जम्भ्यैः᳚ । जम्भ्यै॒स्तस्क॑रान् । तस्क॑राꣳ उ॒त । उ॒तेत्यु॒त ॥ हनू᳚भ्याꣳ स्ते॒नान् । हनू᳚भ्या॒मिति॒ हनु॑ - भ्या॒म् । स्ते॒नान् भ॑गवः । भ॒ग॒व॒स्तान् । भ॒ग॒व॒ इति॑ भग - वः॒ । ताꣳस्त्वम् । त्वम् खा॑द । खा॒द॒ सुखा॑दितान् । सुखा॑दिता॒निति॒ सु - खा॒दि॒ता॒न्॒ ॥ ये जने॑षु । जने॑षु म॒लिम्ल॑वः । म॒लिम्ल॑वः स्ते॒नासः॑ । स्ते॒नास॒स्तस्क॑राः । तस्क॑रा॒ वने᳚ । वन॒ इति॒ वने᳚ ॥ ये कक्षे॑षु \newline

\textbf{Jatai Paata} \newline

1. पृ॒थि॒व्याः स॑मिधा॒नꣳ स॑मिधा॒नम् पृ॑थि॒व्याः पृ॑थि॒व्याः स॑मिधा॒नम् । \newline
2. स॒मि॒धा॒न म॒ग्नि म॒ग्निꣳ स॑मिधा॒नꣳ स॑मिधा॒न म॒ग्निम् । \newline
3. स॒मि॒धा॒नमिति॑ सं - इ॒धा॒नम् । \newline
4. अ॒ग्निꣳ रा॒यो रा॒यो अ॒ग्नि म॒ग्निꣳ रा॒यः । \newline
5. रा॒य स्पोषा॑य॒ पोषा॑य रा॒यो रा॒य स्पोषा॑य । \newline
6. पोषा॑य बृह॒ते बृ॑ह॒ते पोषा॑य॒ पोषा॑य बृह॒ते । \newline
7. बृ॒ह॒ते ह॑वामहे हवामहे बृह॒ते बृ॑ह॒ते ह॑वामहे । \newline
8. ह॒वा॒म॒ह॒ इति॑ हवामहे । \newline
9. इ॒र॒म्म॒दम् बृ॒हदु॑क्थम् बृ॒हदु॑क्थ मिरम्म॒द मि॑रम्म॒दम् बृ॒हदु॑क्थम् । \newline
10. इ॒र॒म्म॒दमिती॑रं - म॒दम् । \newline
11. बृ॒हदु॑क्थं॒ ॅयज॑त्रं॒ ॅयज॑त्रम् बृ॒हदु॑क्थम् बृ॒हदु॑क्थं॒ ॅयज॑त्रम् । \newline
12. बृ॒हदु॑क्थ॒मिति॑ बृ॒हत् - उ॒क्थ॒म् । \newline
13. यज॑त्र॒म् जेता॑र॒म् जेता॑रं॒ ॅयज॑त्रं॒ ॅयज॑त्र॒म् जेता॑रम् । \newline
14. जेता॑र म॒ग्नि म॒ग्निम् जेता॑र॒म् जेता॑र म॒ग्निम् । \newline
15. अ॒ग्निम् पृत॑नासु॒ पृत॑ना स्व॒ग्नि म॒ग्निम् पृत॑नासु । \newline
16. पृत॑नासु सास॒हिꣳ सा॑स॒हिम् पृत॑नासु॒ पृत॑नासु सास॒हिम् । \newline
17. सा॒स॒हिमिति॑ सास॒हिम् । \newline
18. याः सेनाः॒ सेना॒ या याः सेनाः᳚ । \newline
19. सेना॑ अ॒भीत्व॑री र॒भीत्व॑रीः॒ सेनाः॒ सेना॑ अ॒भीत्व॑रीः । \newline
20. अ॒भीत्व॑री राव्या॒धिनी॑ राव्या॒धिनी॑ र॒भीत्व॑री र॒भीत्व॑री राव्या॒धिनीः᳚ । \newline
21. अ॒भीत्व॑री॒रित्य॑भि - इत्व॑रीः । \newline
22. आ॒व्या॒धिनी॒ रुग॑णा॒ उग॑णा आव्या॒धिनी॑ राव्या॒धिनी॒ रुग॑णाः । \newline
23. आ॒व्या॒धिनी॒रित्या᳚ - व्या॒धिनीः᳚ । \newline
24. उग॑णा उ॒तोतोग॑णा॒ उग॑णा उ॒त । \newline
25. उ॒तेत्यु॒त । \newline
26. ये स्ते॒नाः स्ते॒ना ये ये स्ते॒नाः । \newline
27. स्ते॒ना ये ये स्ते॒नाः स्ते॒ना ये । \newline
28. ये च॑ च॒ ये ये च॑ । \newline
29. च॒ तस्क॑रा॒ स्तस्क॑राश्च च॒ तस्क॑राः । \newline
30. तस्क॑रा॒ स्ताꣳ स्ताꣳ स्तस्क॑रा॒ स्तस्क॑रा॒ स्तान् । \newline
31. ताꣳ स्ते॑ ते॒ ताꣳ स्ताꣳ स्ते᳚ । \newline
32. ते॒ अ॒ग्ने॒ अ॒ग्ने॒ ते॒ ते॒ अ॒ग्ने॒ । \newline
33. अ॒ग्ने ऽप्यप्य॑ग्ने अ॒ग्ने ऽपि॑ । \newline
34. अपि॑ दधामि दधा॒ म्यप्यपि॑ दधामि । \newline
35. द॒धा॒ म्या॒स्य॑ आ॒स्ये॑ दधामि दधा म्या॒स्ये᳚ । \newline
36. आ॒स्य॑ इत्या॒स्ये᳚ । \newline
37. दꣳष्ट्रा᳚भ्याम् म॒लिम्लू᳚न् म॒लिम्लू॒न् दꣳष्ट्रा᳚भ्या॒म् दꣳष्ट्रा᳚भ्याम् म॒लिम्लून्॑ । \newline
38. म॒लिम्लू॒न् जंभ्यै॒र् जंभ्यै᳚र् म॒लिम्लू᳚न् म॒लिम्लू॒न् जंभ्यैः᳚ । \newline
39. जंभ्यै॒ स्तस्क॑रा॒न् तस्क॑रा॒न् जंभ्यै॒र् जंभ्यै॒ स्तस्क॑रान् । \newline
40. तस्क॑राꣳ उ॒तोत तस्क॑रा॒न् तस्क॑राꣳ उ॒त । \newline
41. उ॒तेत्यु॒त । \newline
42. हनू᳚भ्याꣳ स्ते॒नान् थ्स्ते॒नान्. हनू᳚भ्याꣳ॒॒ हनू᳚भ्याꣳ स्ते॒नान् । \newline
43. हनू᳚भ्या॒मिति॒ हनु॑ - भ्या॒म् । \newline
44. स्ते॒नान् भ॑गवो भगवः॒ स्ते॒नान् थ्स्ते॒नान् भ॑गवः । \newline
45. भ॒ग॒व॒ स्ताꣳ स्तान् भ॑गवो भगव॒ स्तान् । \newline
46. भ॒ग॒व॒ इति॑ भग - वः॒ । \newline
47. ताꣳ स्त्वम् त्वम् ताꣳ स्ताꣳ स्त्वम् । \newline
48. त्वम् खा॑द खाद॒ त्वम् त्वम् खा॑द । \newline
49. खा॒द॒ सुखा॑दिता॒न् थ्सुखा॑दितान् खाद खाद॒ सुखा॑दितान् । \newline
50. सुखा॑दिता॒निति॒ सु - खा॒दि॒ता॒न् । \newline
51. ये जने॑षु॒ जने॑षु॒ ये ये जने॑षु । \newline
52. जने॑षु म॒लिम्ल॑वो म॒लिम्ल॑वो॒ जने॑षु॒ जने॑षु म॒लिम्ल॑वः । \newline
53. म॒लिम्ल॑वः स्ते॒नासः॑ स्ते॒नासो॑ म॒लिम्ल॑वो म॒लिम्ल॑वः स्ते॒नासः॑ । \newline
54. स्ते॒नास॒ स्तस्क॑रा॒ स्तस्क॑राः स्ते॒नासः॑ स्ते॒नास॒ स्तस्क॑राः । \newline
55. तस्क॑रा॒ वने॒ वने॒ तस्क॑रा॒ स्तस्क॑रा॒ वने᳚ । \newline
56. वन॒ इति॒ वने᳚ । \newline
57. ये कक्षे॑षु॒ कक्षे॑षु॒ ये ये कक्षे॑षु । \newline

\textbf{Ghana Paata } \newline

1. पृ॒थि॒व्याः स॑मिधा॒नꣳ स॑मिधा॒नम् पृ॑थि॒व्याः पृ॑थि॒व्याः स॑मिधा॒न म॒ग्नि म॒ग्निꣳ स॑मिधा॒नम् पृ॑थि॒व्याः पृ॑थि॒व्याः स॑मिधा॒न म॒ग्निम् । \newline
2. स॒मि॒धा॒न म॒ग्नि म॒ग्निꣳ स॑मिधा॒नꣳ स॑मिधा॒न म॒ग्निꣳ रा॒यो रा॒यो अ॒ग्निꣳ स॑मिधा॒नꣳ स॑मिधा॒न म॒ग्निꣳ रा॒यः । \newline
3. स॒मि॒धा॒नमिति॑ सं - इ॒धा॒नम् । \newline
4. अ॒ग्निꣳ रा॒यो रा॒यो अ॒ग्नि म॒ग्निꣳ रा॒य स्पोषा॑य॒ पोषा॑य रा॒यो अ॒ग्नि म॒ग्निꣳ रा॒य स्पोषा॑य । \newline
5. रा॒य स्पोषा॑य॒ पोषा॑य रा॒यो रा॒य स्पोषा॑य बृह॒ते बृ॑ह॒ते पोषा॑य रा॒यो रा॒य स्पोषा॑य बृह॒ते । \newline
6. पोषा॑य बृह॒ते बृ॑ह॒ते पोषा॑य॒ पोषा॑य बृह॒ते ह॑वामहे हवामहे बृह॒ते पोषा॑य॒ पोषा॑य बृह॒ते ह॑वामहे । \newline
7. बृ॒ह॒ते ह॑वामहे हवामहे बृह॒ते बृ॑ह॒ते ह॑वामहे । \newline
8. ह॒वा॒म॒ह॒ इति॑ हवामहे । \newline
9. इ॒र॒म्म॒दम् बृ॒हदु॑क्थम् बृ॒हदु॑क्थ मिरम्म॒द मि॑रम्म॒दम् बृ॒हदु॑क्थं॒ ॅयज॑त्रं॒ ॅयज॑त्रम् बृ॒हदु॑क्थ मिरम्म॒द मि॑रम्म॒दम् बृ॒हदु॑क्थं॒ ॅयज॑त्रम् । \newline
10. इ॒र॒म्म॒दमिती॑रं - म॒दम् । \newline
11. बृ॒हदु॑क्थं॒ ॅयज॑त्रं॒ ॅयज॑त्रम् बृ॒हदु॑क्थम् बृ॒हदु॑क्थं॒ ॅयज॑त्र॒म् जेता॑र॒म् जेता॑रं॒ ॅयज॑त्रम् बृ॒हदु॑क्थम् बृ॒हदु॑क्थं॒ ॅयज॑त्र॒म् जेता॑रम् । \newline
12. बृ॒हदु॑क्थ॒मिति॑ बृ॒हत् - उ॒क्थ॒म् । \newline
13. यज॑त्र॒म् जेता॑र॒म् जेता॑रं॒ ॅयज॑त्रं॒ ॅयज॑त्र॒म् जेता॑र म॒ग्नि म॒ग्निम् जेता॑रं॒ ॅयज॑त्रं॒ ॅयज॑त्र॒म् जेता॑र म॒ग्निम् । \newline
14. जेता॑र म॒ग्नि म॒ग्निम् जेता॑र॒म् जेता॑र म॒ग्निम् पृत॑नासु॒ पृत॑ना स्व॒ग्निम् जेता॑र॒म् जेता॑र म॒ग्निम् पृत॑नासु । \newline
15. अ॒ग्निम् पृत॑नासु॒ पृत॑ना स्व॒ग्नि म॒ग्निम् पृत॑नासु सास॒हिꣳ सा॑स॒हिम् पृत॑ना स्व॒ग्नि म॒ग्निम् पृत॑नासु सास॒हिम् । \newline
16. पृत॑नासु सास॒हिꣳ सा॑स॒हिम् पृत॑नासु॒ पृत॑नासु सास॒हिम् । \newline
17. सा॒स॒हिमिति॑ सास॒हिम् । \newline
18. याः सेनाः॒ सेना॒ या याः सेना॑ अ॒भीत्व॑री र॒भीत्व॑रीः॒ सेना॒ या याः सेना॑ अ॒भीत्व॑रीः । \newline
19. सेना॑ अ॒भीत्व॑री र॒भीत्व॑रीः॒ सेनाः॒ सेना॑ अ॒भीत्व॑री राव्या॒धिनी॑ राव्या॒धिनी॑ र॒भीत्व॑रीः॒ सेनाः॒ सेना॑ अ॒भीत्व॑री राव्या॒धिनीः᳚ । \newline
20. अ॒भीत्व॑री राव्या॒धिनी॑ राव्या॒धिनी॑ र॒भीत्व॑री र॒भीत्व॑री राव्या॒धिनी॒ रुग॑णा॒ उग॑णा आव्या॒धिनी॑ र॒भीत्व॑री र॒भीत्व॑री राव्या॒धिनी॒ रुग॑णाः । \newline
21. अ॒भीत्व॑री॒रित्य॑भि - इत्व॑रीः । \newline
22. आ॒व्या॒धिनी॒ रुग॑णा॒ उग॑णा आव्या॒धिनी॑ राव्या॒धिनी॒ रुग॑णा उ॒तोतो ग॑णा आव्या॒धिनी॑ राव्या॒धिनी॒ रुग॑णा उ॒त । \newline
23. आ॒व्या॒धिनी॒रित्या᳚ - व्या॒धिनीः᳚ । \newline
24. उग॑णा उ॒तोतो ग॑णा॒ उग॑णा उ॒त । \newline
25. उ॒तेत्यु॒त । \newline
26. ये स्ते॒नाः स्ते॒ना ये ये स्ते॒ना ये ये स्ते॒ना ये ये स्ते॒ना ये । \newline
27. स्ते॒ना ये ये स्ते॒नाः स्ते॒ना ये च॑ च॒ ये स्ते॒नाः स्ते॒ना ये च॑ । \newline
28. ये च॑ च॒ ये ये च॒ तस्क॑रा॒ स्तस्क॑राश्च॒ ये ये च॒ तस्क॑राः । \newline
29. च॒ तस्क॑रा॒ स्तस्क॑राश्च च॒ तस्क॑रा॒ स्ताꣳ स्ताꣳ स्तस्क॑राश्च च॒ तस्क॑रा॒ स्तान् । \newline
30. तस्क॑रा॒ स्ताꣳ स्ताꣳ स्तस्क॑रा॒ स्तस्क॑रा॒ स्ताꣳ स्ते॑ ते॒ ताꣳ स्तस्क॑रा॒ स्तस्क॑रा॒ स्ताꣳ स्ते᳚ । \newline
31. ताꣳ स्ते॑ ते॒ ताꣳ स्ताꣳ स्ते॑ अग्ने अग्ने ते॒ ताꣳ स्ताꣳ स्ते॑ अग्ने । \newline
32. ते॒ अ॒ग्ने॒ अ॒ग्ने॒ ते॒ ते॒ अ॒ग्ने ऽप्यप्य॑ग्ने ते ते अ॒ग्ने ऽपि॑ । \newline
33. अ॒ग्ने ऽप्यप्य॑ग्ने अ॒ग्ने ऽपि॑ दधामि दधा॒ म्यप्य॑ग्ने अ॒ग्ने ऽपि॑ दधामि । \newline
34. अपि॑ दधामि दधा॒ म्यप्यपि॑ दधा म्या॒स्य॑ आ॒स्ये॑ दधा॒ म्यप्यपि॑ दधा म्या॒स्ये᳚ । \newline
35. द॒धा॒ म्या॒स्य॑ आ॒स्ये॑ दधामि दधा म्या॒स्ये᳚ । \newline
36. आ॒स्य॑ इत्या॒स्ये᳚ । \newline
37. दꣳष्ट्रा᳚भ्याम् म॒लिम्लू᳚न् म॒लिम्लू॒न् दꣳष्ट्रा᳚भ्या॒म् दꣳष्ट्रा᳚भ्याम् म॒लिम्लू॒न् जंभ्यै॒र् जंभ्यै᳚र् म॒लिम्लू॒न् दꣳष्ट्रा᳚भ्या॒म् दꣳष्ट्रा᳚भ्याम् म॒लिम्लू॒न् जंभ्यैः᳚ । \newline
38. म॒लिम्लू॒न् जंभ्यै॒र् जंभ्यै᳚र् म॒लिम्लू᳚न् म॒लिम्लू॒न् जंभ्यै॒ स्तस्क॑रा॒न् तस्क॑रा॒न् जंभ्यै᳚र् म॒लिम्लू᳚न् म॒लिम्लू॒न् जंभ्यै॒ स्तस्क॑रान् । \newline
39. जंभ्यै॒ स्तस्क॑रा॒न् तस्क॑रा॒न् जंभ्यै॒र् जंभ्यै॒ स्तस्क॑राꣳ उ॒तोत तस्क॑रा॒न् जंभ्यै॒र् जंभ्यै॒ स्तस्क॑राꣳ उ॒त । \newline
40. तस्क॑राꣳ उ॒तोत तस्क॑रा॒न् तस्क॑राꣳ उ॒त । \newline
41. उ॒तेत्यु॒त । \newline
42. हनू᳚भ्याꣳ स्ते॒नान् थ्स्ते॒नान्. हनू᳚भ्याꣳ॒॒ हनू᳚भ्याꣳ स्ते॒नान् भ॑गवो भगवः स्ते॒नान्. हनू᳚भ्याꣳ॒॒ हनू᳚भ्याꣳ स्ते॒नान् भ॑गवः । \newline
43. हनू᳚भ्या॒मिति॒ हनु॑ - भ्या॒म् । \newline
44. स्ते॒नान् भ॑गवो भगवः स्ते॒नान् थ्स्ते॒नान् भ॑गव॒ स्ताꣳ स्तान् भ॑गवः स्ते॒नान् थ्स्ते॒नान् भ॑गव॒ स्तान् । \newline
45. भ॒ग॒व॒ स्ताꣳ स्तान् भ॑गवो भगव॒ स्ताꣳ स्त्वम् त्वम् तान् भ॑गवो भगव॒ स्ताꣳ स्त्वम् । \newline
46. भ॒ग॒व॒ इति॑ भग - वः॒ । \newline
47. ताꣳ स्त्वम् त्वम् ताꣳ स्ताꣳ स्त्वम् खा॑द खाद॒ त्वम् ताꣳ स्ताꣳ स्त्वम् खा॑द । \newline
48. त्वम् खा॑द खाद॒ त्वम् त्वम् खा॑द॒ सुखा॑दिता॒न् थ्सुखा॑दितान् खाद॒ त्वम् त्वम् खा॑द॒ सुखा॑दितान् । \newline
49. खा॒द॒ सुखा॑दिता॒न् थ्सुखा॑दितान् खाद खाद॒ सुखा॑दितान् । \newline
50. सुखा॑दिता॒निति॒ सु - खा॒दि॒ता॒न् । \newline
51. ये जने॑षु॒ जने॑षु॒ ये ये जने॑षु म॒लिम्ल॑वो म॒लिम्ल॑वो॒ जने॑षु॒ ये ये जने॑षु म॒लिम्ल॑वः । \newline
52. जने॑षु म॒लिम्ल॑वो म॒लिम्ल॑वो॒ जने॑षु॒ जने॑षु म॒लिम्ल॑वः स्ते॒नासः॑ स्ते॒नासो॑ म॒लिम्ल॑वो॒ जने॑षु॒ जने॑षु म॒लिम्ल॑वः स्ते॒नासः॑ । \newline
53. म॒लिम्ल॑वः स्ते॒नासः॑ स्ते॒नासो॑ म॒लिम्ल॑वो म॒लिम्ल॑वः स्ते॒नास॒ स्तस्क॑रा॒ स्तस्क॑राः स्ते॒नासो॑ म॒लिम्ल॑वो म॒लिम्ल॑वः स्ते॒नास॒ स्तस्क॑राः । \newline
54. स्ते॒नास॒ स्तस्क॑रा॒ स्तस्क॑राः स्ते॒नासः॑ स्ते॒नास॒ स्तस्क॑रा॒ वने॒ वने॒ तस्क॑राः स्ते॒नासः॑ स्ते॒नास॒ स्तस्क॑रा॒ वने᳚ । \newline
55. तस्क॑रा॒ वने॒ वने॒ तस्क॑रा॒ स्तस्क॑रा॒ वने᳚ । \newline
56. वन॒ इति॒ वने᳚ । \newline
57. ये कक्षे॑षु॒ कक्षे॑षु॒ ये ये कक्षे᳚ ष्वघा॒यवो॑ अघा॒यवः॒ कक्षे॑षु॒ ये ये कक्षे᳚ ष्वघा॒यवः॑ । \newline
\pagebreak
\markright{ TS 4.1.10.3  \hfill https://www.vedavms.in \hfill}

\section{ TS 4.1.10.3 }

\textbf{TS 4.1.10.3 } \newline
\textbf{Samhita Paata} \newline

कक्षे᳚ष्वघा॒ यव॒स्ताꣳस्ते॑ दधामि॒ जंभ॑योः ॥ यो अ॒स्मभ्य॑मराती॒याद्-यश्च॑ नो॒ द्वेष॑ते॒ जनः॑ । निन्दा॒द्यो अ॒स्मान् दिफ्सा᳚च्च॒ सर्वं॒ तं म॑स्म॒सा कु॑रु ॥ सꣳशि॑तं मे॒ ब्रह्म॒ सꣳशि॑तं ॅवी॒र्यं॑ बलं᳚ । सꣳशि॑तं क्ष॒त्रं जि॒ष्णु यस्या॒ऽहमस्मि॑ पु॒रोहि॑तः ॥ उदे॑षां बा॒हू अ॑तिर॒मुद्वर्च॒ उदू॒ बलं᳚ । क्षि॒णोमि॒ ब्रह्म॑णा॒-ऽमित्रा॒नुन्न॑यामि॒ - [  ] \newline

\textbf{Pada Paata} \newline

कक्षे॑षु । अ॒घा॒यव॒ इत्य॑घ - यवः॑ । तान् । ते॒ । द॒धा॒मि॒ । जंभ॑योः ॥ यः । अ॒स्मभ्य॒मित्य॒स्म-भ्य॒म् । अ॒रा॒ती॒यात् । यः । च॒ । नः॒ । द्वेष॑ते । जनः॑ ॥ निन्दा᳚त् । यः । अ॒स्मान् । दिफ्सा᳚त् । च॒ । सर्व᳚म् । तम् । म॒स्म॒सा । कु॒रु॒ ॥ सꣳशि॑त॒मिति॒ सं - शि॒त॒म् । मे॒ । ब्रह्म॑ । सꣳशि॑त॒मिति॒ सं - शि॒त॒म् । वी॒र्य᳚म् । बल᳚म् ॥ सꣳशि॑त॒मिति॒ सं - शि॒त॒म् । क्ष॒त्रम् । जि॒ष्णु । यस्य॑ । अ॒हम् । अस्मि॑ । पु॒रोहि॑त॒ इति॑ पु॒रः - हि॒तः॒ ॥ उदिति॑ । ए॒षा॒म् । बा॒हू इति॑ । अ॒ति॒र॒म् । उदिति॑ । वर्चः॑ । उदिति॑ । उ॒ । बल᳚म् ॥ क्षि॒णोमि॑ । ब्रह्म॑णा । अ॒मित्रान्॑ । उदिति॑ । न॒या॒मि॒ ।  \newline


\textbf{Krama Paata} \newline

कक्षे᳚ष्वघा॒यवः॑ । अ॒घा॒यव॒स्तान् । अ॒घा॒यव॒ इत्य॑घ - यवः॑ । ताꣳस्ते᳚ । ते॒ द॒धा॒मि॒ । द॒धा॒मि॒ जम्भ॑योः । जम्भ॑यो॒रिति॒ जम्भ॑योः ॥ यो अ॒स्मभ्य᳚म् । अ॒स्मभ्य॑मराती॒यात् । अ॒स्मभ्य॒मित्य॒स्म - भ्य॒म् । अ॒रा॒ती॒याद् यः । यश्च॑ । च॒ नः॒ । नो॒ द्वेष॑ते । द्वेष॑ते॒ जनः॑ । जन॒ इति॒ जनः॑ ॥ निन्दा॒द् यः । यो अ॒स्मान् । अ॒स्मान् दिफ्सा᳚त् । दिफ्सा᳚च् च । च॒ सर्व᳚म् । सर्व॒म् तम् । तम् म॑स्म॒सा । म॒स्म॒सा कु॑रु । कु॒र्विति॑ कुरु ॥ सꣳशि॑तम् मे । सꣳशि॑त॒मिति॒ सम् - शि॒त॒म् । मे॒ ब्रह्म॑ । ब्रह्म॒ सꣳशि॑तम् । सꣳशि॑तम् ॅवी॒र्य᳚म् । सꣳशि॑त॒मिति॒ सम् - शि॒त॒म् । वी॒र्य॑म् बल᳚म् । बल॒मिति॒ बल᳚म् ॥ सꣳशि॑तम् क्ष॒त्रम् । सꣳशि॑त॒मिति॒ सम् - शि॒त॒म् । क्ष॒त्रम् जि॒ष्णु । जि॒ष्णु यस्य॑ । यस्या॒हम् । अ॒हमस्मि॑ । अस्मि॑ पु॒रोहि॑तः । पु॒रोहि॑त॒ इति॑ पु॒रः - हि॒तः॒ ॥ उदे॑षाम् । ए॒षा॒म् बा॒हू । बा॒हू अ॑तिरम् । बा॒हू इति॑ बा॒हू । अ॒ति॒र॒मुत् । उद् वर्चः॑ । वर्च॒ उत् । उदु॑ । ऊ॒ बल᳚म् । बल॒मिति॒ बल᳚म् ॥ क्षि॒णोमि॒ ब्रह्म॑णा । ब्रह्म॑णा॒ऽमित्रान्॑ । अ॒मित्रा॒नुत् । उन्न॑यामि । न॒या॒मि॒ स्वान् \newline

\textbf{Jatai Paata} \newline

1. कक्षे᳚ ष्वघा॒यवो॑ अघा॒यवः॒ कक्षे॑षु॒ कक्षे᳚ ष्वघा॒यवः॑ । \newline
2. अ॒घा॒यव॒ स्ताꣳ स्ता न॑घा॒यवो॑ अघा॒यव॒ स्तान् । \newline
3. अ॒घा॒यव॒ इत्य॑घ - यवः॑ । \newline
4. ताꣳ स्ते॑ ते॒ ताꣳ स्ताꣳ स्ते᳚ । \newline
5. ते॒ द॒धा॒मि॒ द॒धा॒मि॒ ते॒ ते॒ द॒धा॒मि॒ । \newline
6. द॒धा॒मि॒ जंभ॑यो॒र् जंभ॑योर् दधामि दधामि॒ जंभ॑योः । \newline
7. जम्भ॑यो॒रिति॒ जम्भ॑योः । \newline
8. यो अ॒स्मभ्य॑ म॒स्मभ्यं॒ ॅयो यो अ॒स्मभ्य᳚म् । \newline
9. अ॒स्मभ्य॑ मराती॒या द॑राती॒या द॒स्मभ्य॑ म॒स्मभ्य॑ मराती॒यात् । \newline
10. अ॒स्मभ्य॒मित्य॒स्म - भ्य॒म् । \newline
11. अ॒रा॒ती॒याद् यो यो अ॑राती॒या द॑राती॒याद् यः । \newline
12. यश्च॑ च॒ यो यश्च॑ । \newline
13. च॒ नो॒ न॒श्च॒ च॒ नः॒ । \newline
14. नो॒ द्वेष॑ते॒ द्वेष॑ते नो नो॒ द्वेष॑ते । \newline
15. द्वेष॑ते॒ जनो॒ जनो॒ द्वेष॑ते॒ द्वेष॑ते॒ जनः॑ । \newline
16. जन॒ इति॒ जनः॑ । \newline
17. निन्दा॒द् यो यो निन्दा॒न् निन्दा॒द् यः । \newline
18. यो अ॒स्मा न॒स्मान्. यो यो अ॒स्मान् । \newline
19. अ॒स्मान् दिफ्सा॒द् दिफ्सा॑ द॒स्मा न॒स्मान् दिफ्सा᳚त् । \newline
20. दिफ्सा᳚च् च च॒ दिफ्सा॒द् दिफ्सा᳚च् च । \newline
21. च॒ सर्वꣳ॒॒ सर्व॑म् च च॒ सर्व᳚म् । \newline
22. सर्व॒म् तम् तꣳ सर्वꣳ॒॒ सर्व॒म् तम् । \newline
23. तम् म॑स्म॒सा म॑स्म॒सा तम् तम् म॑स्म॒सा । \newline
24. म॒स्म॒सा कु॑रु कुरु मस्म॒सा म॑स्म॒सा कु॑रु । \newline
25. कु॒र्विति॑ कुरु । \newline
26. सꣳशि॑तम् मे मे॒ सꣳशि॑तꣳ॒॒ सꣳशि॑तम् मे । \newline
27. सꣳशि॑त॒मिति॒ सं - शि॒त॒म् । \newline
28. मे॒ ब्रह्म॒ ब्रह्म॑ मे मे॒ ब्रह्म॑ । \newline
29. ब्रह्म॒ सꣳशि॑तꣳ॒॒ सꣳशि॑त॒म् ब्रह्म॒ ब्रह्म॒ सꣳशि॑तम् । \newline
30. सꣳशि॑तं ॅवी॒र्यं॑ ॅवी॒र्यꣳ॑ सꣳशि॑तꣳ॒॒ सꣳशि॑तं ॅवी॒र्य᳚म् । \newline
31. सꣳशि॑त॒मिति॒ सं - शि॒त॒म् । \newline
32. वी॒र्य॑म् बल॒म् बलं॑ ॅवी॒र्यं॑ ॅवी॒र्य॑म् बल᳚म् । \newline
33. बल॒मिति॒ बल᳚म् । \newline
34. सꣳशि॑तम् क्ष॒त्रम् क्ष॒त्रꣳ सꣳशि॑तꣳ॒॒ सꣳशि॑तम् क्ष॒त्रम् । \newline
35. सꣳशि॑त॒मिति॒ सं - शि॒त॒म् । \newline
36. क्ष॒त्रम् जि॒ष्णु जि॒ष्णु क्ष॒त्रम् क्ष॒त्रम् जि॒ष्णु । \newline
37. जि॒ष्णु यस्य॒ यस्य॑ जि॒ष्णु जि॒ष्णु यस्य॑ । \newline
38. यस्या॒ह म॒हं ॅयस्य॒ यस्या॒हम् । \newline
39. अ॒ह मस्म्य स्म्य॒ह म॒ह मस्मि॑ । \newline
40. अस्मि॑ पु॒रोहि॑तः पु॒रोहि॑तो॒ अस्म्यस्मि॑ पु॒रोहि॑तः । \newline
41. पु॒रोहि॑त॒ इति॑ पु॒रः - हि॒तः॒ । \newline
42. उदे॑षा मेषा॒ मुदु दे॑षाम् । \newline
43. ए॒षा॒म् बा॒हू बा॒हू ए॑षा मेषाम् बा॒हू । \newline
44. बा॒हू अ॑तिर मतिरम् बा॒हू बा॒हू अ॑तिरम् । \newline
45. बा॒हू इति॑ बा॒हू । \newline
46. अ॒ति॒र॒ मुदु द॑तिर मतिर॒ मुत् । \newline
47. उद् वर्चो॒ वर्च॒ उदुद् वर्चः॑ । \newline
48. वर्च॒ उदुद् वर्चो॒ वर्च॒ उत् । \newline
49. उदु॑ वु॒ वुदुदु॑ । \newline
50. ऊ॒ बल॒म् बल॑ मु वू॒ बल᳚म् । \newline
51. बल॒मिति॒ बल᳚म् । \newline
52. क्षि॒णोमि॒ ब्रह्म॑णा॒ ब्रह्म॑णा क्षि॒णोमि॑ क्षि॒णोमि॒ ब्रह्म॑णा । \newline
53. ब्रह्म॑णा॒ ऽमित्रा॑ न॒मित्रा॒न् ब्रह्म॑णा॒ ब्रह्म॑णा॒ ऽमित्रान्॑ । \newline
54. अ॒मित्रा॒ नुदु द॒मित्रा॑ न॒मित्रा॒ नुत् । \newline
55. उन् न॑यामि नया॒ म्युदुन् न॑यामि । \newline
56. न॒या॒मि॒ स्वान् थ्स्वान् न॑यामि नयामि॒ स्वान् । \newline

\textbf{Ghana Paata } \newline

1. कक्षे᳚ ष्वघा॒यवो॑ अघा॒यवः॒ कक्षे॑षु॒ कक्षे᳚ ष्वघा॒यव॒ स्ताꣳ स्तान॑घा॒यवः॒ कक्षे॑षु॒ कक्षे᳚ ष्वघा॒यव॒ स्तान् । \newline
2. अ॒घा॒यव॒ स्ताꣳ स्तान॑घा॒यवो॑ अघा॒यव॒ स्ताꣳ स्ते॑ ते॒ ता न॑घा॒यवो॑ अघा॒यव॒ स्ताꣳ स्ते᳚ । \newline
3. अ॒घा॒यव॒ इत्य॑घ - यवः॑ । \newline
4. ताꣳ स्ते॑ ते॒ ताꣳ स्ताꣳ स्ते॑ दधामि दधामि ते॒ ताꣳ स्ताꣳ स्ते॑ दधामि । \newline
5. ते॒ द॒धा॒मि॒ द॒धा॒मि॒ ते॒ ते॒ द॒धा॒मि॒ जंभ॑यो॒र् जंभ॑योर् दधामि ते ते दधामि॒ जंभ॑योः । \newline
6. द॒धा॒मि॒ जंभ॑यो॒र् जंभ॑योर् दधामि दधामि॒ जंभ॑योः । \newline
7. जम्भ॑यो॒रिति॒ जम्भ॑योः । \newline
8. यो अ॒स्मभ्य॑ म॒स्मभ्यं॒ ॅयो यो अ॒स्मभ्य॑ मराती॒या द॑राती॒या द॒स्मभ्यं॒ ॅयो यो अ॒स्मभ्य॑ मराती॒यात् । \newline
9. अ॒स्मभ्य॑ मराती॒या द॑राती॒या द॒स्मभ्य॑ म॒स्मभ्य॑ मराती॒याद् यो यो अ॑राती॒या द॒स्मभ्य॑ म॒स्मभ्य॑ मराती॒याद् यः । \newline
10. अ॒स्मभ्य॒मित्य॒स्म - भ्य॒म् । \newline
11. अ॒रा॒ती॒याद् यो यो अ॑राती॒या द॑राती॒याद् यश्च॑ च॒ यो अ॑राती॒या द॑राती॒याद् यश्च॑ । \newline
12. यश्च॑ च॒ यो यश्च॑ नो नश्च॒ यो यश्च॑ नः । \newline
13. च॒ नो॒ न॒श्च॒ च॒ नो॒ द्वेष॑ते॒ द्वेष॑ते नश्च च नो॒ द्वेष॑ते । \newline
14. नो॒ द्वेष॑ते॒ द्वेष॑ते नो नो॒ द्वेष॑ते॒ जनो॒ जनो॒ द्वेष॑ते नो नो॒ द्वेष॑ते॒ जनः॑ । \newline
15. द्वेष॑ते॒ जनो॒ जनो॒ द्वेष॑ते॒ द्वेष॑ते॒ जनः॑ । \newline
16. जन॒ इति॒ जनः॑ । \newline
17. निन्दा॒द् यो यो निन्दा॒न् निन्दा॒द् यो अ॒स्मा न॒स्मान्. यो निन्दा॒न् निन्दा॒द् यो अ॒स्मान् । \newline
18. यो अ॒स्मा न॒स्मान्. यो यो अ॒स्मान् दिफ्सा॒द् दिफ्सा॑ द॒स्मान्. यो यो अ॒स्मान् दिफ्सा᳚त् । \newline
19. अ॒स्मान् दिफ्सा॒द् दिफ्सा॑ द॒स्मा न॒स्मान् दिफ्सा᳚च् च च॒ दिफ्सा॑ द॒स्मा न॒स्मान् दिफ्सा᳚च् च । \newline
20. दिफ्सा᳚च् च च॒ दिफ्सा॒द् दिफ्सा᳚च् च॒ सर्वꣳ॒॒ सर्व॑म् च॒ दिफ्सा॒द् दिफ्सा᳚च् च॒ सर्व᳚म् । \newline
21. च॒ सर्वꣳ॒॒ सर्व॑म् च च॒ सर्व॒म् तम् तꣳ सर्व॑म् च च॒ सर्व॒म् तम् । \newline
22. सर्व॒म् तम् तꣳ सर्वꣳ॒॒ सर्व॒म् तम् म॑स्म॒सा म॑स्म॒सा तꣳ सर्वꣳ॒॒ सर्व॒म् तम् म॑स्म॒सा । \newline
23. तम् म॑स्म॒सा म॑स्म॒सा तम् तम् म॑स्म॒सा कु॑रु कुरु मस्म॒सा तम् तम् म॑स्म॒सा कु॑रु । \newline
24. म॒स्म॒सा कु॑रु कुरु मस्म॒सा म॑स्म॒सा कु॑रु । \newline
25. कु॒र्विति॑ कुरु । \newline
26. सꣳशि॑तम् मे मे॒ सꣳशि॑तꣳ॒॒ सꣳशि॑तम् मे॒ ब्रह्म॒ ब्रह्म॑ मे॒ सꣳशि॑तꣳ॒॒ सꣳशि॑तम् मे॒ ब्रह्म॑ । \newline
27. सꣳशि॑त॒मिति॒ सं - शि॒त॒म् । \newline
28. मे॒ ब्रह्म॒ ब्रह्म॑ मे मे॒ ब्रह्म॒ सꣳशि॑तꣳ॒॒ सꣳशि॑त॒म् ब्रह्म॑ मे मे॒ ब्रह्म॒ सꣳशि॑तम् । \newline
29. ब्रह्म॒ सꣳशि॑तꣳ॒॒ सꣳशि॑त॒म् ब्रह्म॒ ब्रह्म॒ सꣳशि॑तं ॅवी॒र्यं॑ ॅवी॒र्यꣳ॑ सꣳशि॑त॒म् ब्रह्म॒ ब्रह्म॒ सꣳशि॑तं ॅवी॒र्य᳚म् । \newline
30. सꣳशि॑तं ॅवी॒र्यं॑ ॅवी॒र्यꣳ॑ सꣳशि॑तꣳ॒॒ सꣳशि॑तं ॅवी॒र्य॑म् बल॒म् बलं॑ ॅवी॒र्यꣳ॑ सꣳशि॑तꣳ॒॒ सꣳशि॑तं ॅवी॒र्य॑म् बल᳚म् । \newline
31. सꣳशि॑त॒मिति॒ सं - शि॒त॒म् । \newline
32. वी॒र्य॑म् बल॒म् बलं॑ ॅवी॒र्यं॑ ॅवी॒र्य॑म् बल᳚म् । \newline
33. बल॒मिति॒ बल᳚म् । \newline
34. सꣳशि॑तम् क्ष॒त्रम् क्ष॒त्रꣳ सꣳशि॑तꣳ॒॒ सꣳशि॑तम् क्ष॒त्रम् जि॒ष्णु जि॒ष्णु क्ष॒त्रꣳ सꣳशि॑तꣳ॒॒ सꣳशि॑तम् क्ष॒त्रम् जि॒ष्णु । \newline
35. सꣳशि॑त॒मिति॒ सं - शि॒त॒म् । \newline
36. क्ष॒त्रम् जि॒ष्णु जि॒ष्णु क्ष॒त्रम् क्ष॒त्रम् जि॒ष्णु यस्य॒ यस्य॑ जि॒ष्णु क्ष॒त्रम् क्ष॒त्रम् जि॒ष्णु यस्य॑ । \newline
37. जि॒ष्णु यस्य॒ यस्य॑ जि॒ष्णु जि॒ष्णु यस्या॒ह म॒हं ॅयस्य॑ जि॒ष्णु जि॒ष्णु यस्या॒हम् । \newline
38. यस्या॒ह म॒हं ॅयस्य॒ यस्या॒ह मस्म्य स्म्य॒हं ॅयस्य॒ यस्या॒ह मस्मि॑ । \newline
39. अ॒ह मस्म्यस्म्य॒ह म॒ह मस्मि॑ पु॒रोहि॑तः पु॒रोहि॑तो॒ अस्म्य॒ह म॒ह मस्मि॑ पु॒रोहि॑तः । \newline
40. अस्मि॑ पु॒रोहि॑तः पु॒रोहि॑तो॒ अस्म्यस्मि॑ पु॒रोहि॑तः । \newline
41. पु॒रोहि॑त॒ इति॑ पु॒रः - हि॒तः॒ । \newline
42. उदे॑षा मेषा॒ मुदुदे॑षाम् बा॒हू बा॒हू ए॑षा॒ मुदुदे॑षाम् बा॒हू । \newline
43. ए॒षा॒म् बा॒हू बा॒हू ए॑षा मेषाम् बा॒हू अ॑तिर मतिरम् बा॒हू ए॑षा मेषाम् बा॒हू अ॑तिरम् । \newline
44. बा॒हू अ॑तिर मतिरम् बा॒हू बा॒हू अ॑तिर॒ मुदु द॑तिरम् बा॒हू बा॒हू अ॑तिर॒ मुत् । \newline
45. बा॒हू इति॑ बा॒हू । \newline
46. अ॒ति॒र॒ मुदु द॑तिर मतिर॒ मुद् वर्चो॒ वर्च॒ उद॑तिर मतिर॒ मुद् वर्चः॑ । \newline
47. उद् वर्चो॒ वर्च॒ उदुद् वर्च॒ उदुद् वर्च॒ उदुद् वर्च॒ उत् । \newline
48. वर्च॒ उदुद् वर्चो॒ वर्च॒ उदु॑ वु॒ वुद् वर्चो॒ वर्च॒ उदु॑ । \newline
49. उदु॑ वु॒ वुदु दू॒ बल॒म् बल॑ मु॒ वुदु दू॒ बल᳚म् । \newline
50. ऊ॒ बल॒म् बल॑ मु वू॒ बल᳚म् । \newline
51. बल॒मिति॒ बल᳚म् । \newline
52. क्षि॒णोमि॒ ब्रह्म॑णा॒ ब्रह्म॑णा क्षि॒णोमि॑ क्षि॒णोमि॒ ब्रह्म॑णा॒ ऽमित्रा॑ न॒मित्रा॒न् ब्रह्म॑णा क्षि॒णोमि॑ क्षि॒णोमि॒ ब्रह्म॑णा॒ ऽमित्रान्॑ । \newline
53. ब्रह्म॑णा॒ ऽमित्रा॑ न॒मित्रा॒न् ब्रह्म॑णा॒ ब्रह्म॑णा॒ ऽमित्रा॒नुदु द॒मित्रा॒न् ब्रह्म॑णा॒ ब्रह्म॑णा॒ ऽमित्रा॒ नुत् । \newline
54. अ॒मित्रा॒ नुदु द॒मित्रा॑ न॒मित्रा॒ नुन् न॑यामि नया॒ म्युद॒मित्रा॑ न॒मित्रा॒नुन् न॑यामि । \newline
55. उन् न॑यामि नया॒म्युदुन् न॑यामि॒ स्वान् थ्स्वान् न॑या॒ म्युदुन् न॑यामि॒ स्वान् । \newline
56. न॒या॒मि॒ स्वान् थ्स्वान् न॑यामि नयामि॒ स्वाꣳ अ॒ह म॒हꣳ स्वान् न॑यामि नयामि॒ स्वाꣳ अ॒हम् । \newline
\pagebreak
\markright{ TS 4.1.10.4  \hfill https://www.vedavms.in \hfill}

\section{ TS 4.1.10.4 }

\textbf{TS 4.1.10.4 } \newline
\textbf{Samhita Paata} \newline

स्वाꣳ अ॒हं ॥ दृ॒शा॒नो रु॒क्म उ॒र्व्या व्य॑द्यौद्दु॒र्मर्.ष॒मायुः॑ श्रि॒ये रु॑चा॒नः । अ॒ग्निर॒मृतो॑ अभव॒द्वयो॑-भि॒र्यदे॑नं॒ द्यौरज॑नयथ् सु॒रेताः᳚ ॥ विश्वा॑ रू॒पाणि॒ प्रति॑ मुञ्चते क॒विः प्राऽसा॑वीद्भ॒द्रं द्वि॒पदे॒ चतु॑ष्पदे । वि नाक॑मख्यथ् सवि॒ता वरे॒ण्योऽनु॑ प्र॒याण॑मु॒षसो॒ वि॑राजति ॥ नक्तो॒षासा॒ सम॑नसा॒ विरू॑पे धा॒पये॑ते॒ शिशु॒मेकꣳ॑ समी॒ची । द्यावा॒ क्षामा॑ रु॒क्मो - [  ] \newline

\textbf{Pada Paata} \newline

स्वान् । अ॒हम् ॥ दृ॒शा॒नः । रु॒क्मः । उ॒र्व्या । वीति॑ । अ॒द्यौ॒त् । दु॒र्मर्.ष॒मिति॑ दुः - मर्.ष᳚म् । आयुः॑ । श्रि॒ये । रु॒चा॒नः ॥ अ॒ग्निः । अ॒मृतः॑ । अ॒भ॒व॒त् । वयो॑भि॒रिति॒ वयः॑ - भिः॒ । यत् । ए॒न॒म् । द्यौः । अज॑नयत् । सु॒रेता॒ इति॑ सु - रेताः᳚ ॥ विश्वा᳚ । रू॒पाणि॑ । प्रतीति॑ । मु॒ञ्च॒ते॒ । क॒विः । प्रेति॑ । अ॒सा॒वी॒त् । भ॒द्रम् । द्वि॒पद॒ इति॑ द्वि-पदे᳚ । चतु॑ष्पद॒ इति॒ चतुः॑ - प॒दे॒ ॥ वीति॑ । नाक᳚म् । अ॒ख्य॒त् । स॒वि॒ता । वरे᳚ण्यः । अन्विति॑ । प्र॒याण॒मिति॑ प्र - यान᳚म् । उ॒षसः॑ । वीति॑ । रा॒ज॒ति॒ ॥ नक्तो॒षासा᳚ । सम॑न॒सेति॒ स - म॒न॒सा॒ । विरू॑पे॒ इति॒ वि-रू॒पे॒ । धा॒पये॑ते॒ इति॑ । शिशु᳚म् । एक᳚म् । स॒मी॒ची इति॑ ॥ द्यावा᳚ । क्षाम॑ । रु॒क्मः ।  \newline


\textbf{Krama Paata} \newline

स्वाꣳ अ॒हम् । अ॒हमित्य॒हम् ॥ दृ॒शा॒नो रु॒क्मः । रु॒क्म उ॒र्व्या । उ॒र्व्या वि । व्य॑द्यौत् । अ॒द्यौ॒द् दु॒र्मर्.ष᳚म् । दु॒र्मर्.ष॒मायुः॑ । दु॒र्मर्.ष॒मिति॑ दुः - मर्.ष᳚म् । आयुः॑ श्रि॒ये । श्रि॒ये रु॑चा॒नः । रु॒चा॒न इति॑ रुचा॒नः ॥ अ॒ग्निर॒मृतः॑ । अ॒मृतो॑ अभवत् । अ॒भ॒व॒द् वयो॑भिः । वयो॑भि॒र् यत् । वयो॑भि॒रिति॒ वयः॑ - भिः॒ । यदे॑नम् । ए॒न॒म् द्यौः । द्यौरज॑नयत् । अज॑नयथ् सु॒रेताः᳚ । सु॒रेता॒ इति॑ सु - रेताः᳚ ॥ विश्वा॑ रू॒पाणि॑ । रू॒पाणि॒ प्रति॑ । प्रति॑ मुञ्चते । मु॒ञ्च॒ते॒ क॒विः । क॒विः प्र । प्रासा॑वीत् । अ॒सा॒वी॒द् भ॒द्रम् । भ॒द्रम् द्वि॒पदे᳚ । द्वि॒पदे॒ चतु॑ष्पदे । द्वि॒पद॒ इति॑ द्वि - पदे᳚ । चतु॑ष्पद॒ इति॒ चतुः॑ - प॒दे॒ ॥ वि नाक᳚म् । नाक॑मख्यत् । अ॒ख्य॒थ् स॒वि॒ता । स॒वि॒ता वरे᳚ण्यः । वरे॒ण्योऽनु॑ । अनु॑ प्र॒याण᳚म् । प्र॒याण॑मु॒षसः॑ । प्र॒याण॒मिति॑ प्र - यान᳚म् । उ॒षसो॒ वि । वि रा॑जति । रा॒ज॒तीति॑ राजति ॥ नक्तो॒षासा॒ सम॑नसा । सम॑नसा॒ विरू॑पे । सम॑न॒सेति॒ स - म॒न॒सा॒ । विरू॑पे धा॒पये॑ते । विरू॑पे॒ इति॒ वि - रू॒पे॒ । धा॒पये॑ते॒ शिशु᳚म् । धा॒पये॑ते॒ इति॑ धा॒पये॑ते । शिशु॒मेक᳚म् । एकꣳ॑ समी॒ची । स॒मी॒ची इति॑ समी॒ची ॥ द्यावा॒ क्षाम॑ । क्षामा॑ रु॒क्मः ( ) । रु॒क्मो अ॒न्तः \newline

\textbf{Jatai Paata} \newline

1. स्वाꣳ अ॒ह म॒हꣳ स्वान् थ्स्वाꣳ अ॒हम् । \newline
2. अ॒हमित्य॒हम् । \newline
3. दृ॒शा॒नो रु॒क्मो रु॒क्मो दृ॑शा॒नो दृ॑शा॒नो रु॒क्मः । \newline
4. रु॒क्म उ॒र्व्योर्व्या रु॒क्मो रु॒क्म उ॒र्व्या । \newline
5. उ॒र्व्या वि व्यु॑र्व्यो र्व्या वि । \newline
6. व्य॑द्यौ दद्यौ॒द् वि व्य॑द्यौत् । \newline
7. अ॒द्यौ॒द् दु॒र्मर्.ष॑म् दु॒र्मर्.ष॑ मद्यौ दद्यौद् दु॒र्मर्.ष᳚म् । \newline
8. दु॒र्मर्.ष॒ मायु॒ रायु॑र् दु॒र्मर्.ष॑म् दु॒र्मर्.ष॒ मायुः॑ । \newline
9. दु॒र्मर्.ष॒मिति॑ दुः - मर्.ष᳚म् । \newline
10. आयुः॑ श्रि॒ये श्रि॒य आयु॒ रायुः॑ श्रि॒ये । \newline
11. श्रि॒ये रु॑चा॒नो रु॑चा॒नः श्रि॒ये श्रि॒ये रु॑चा॒नः । \newline
12. रु॒चा॒न इति॑ रुचा॒नः । \newline
13. अ॒ग्नि र॒मृतो॑ अ॒मृतो॑ अ॒ग्नि र॒ग्नि र॒मृतः॑ । \newline
14. अ॒मृतो॑ अभव दभव द॒मृतो॑ अ॒मृतो॑ अभवत् । \newline
15. अ॒भ॒व॒द् वयो॑भि॒र् वयो॑भि रभव दभव॒द् वयो॑भिः । \newline
16. वयो॑भि॒र् यद् यद् वयो॑भि॒र् वयो॑भि॒र् यत् । \newline
17. वयो॑भि॒रिति॒ वयः॑ - भिः॒ । \newline
18. यदे॑न मेनं॒ ॅयद् यदे॑नम् । \newline
19. ए॒न॒म् द्यौर् द्यौ रे॑न मेन॒म् द्यौः । \newline
20. द्यौ रज॑नय॒ दज॑नय॒द् द्यौर् द्यौ रज॑नयत् । \newline
21. अज॑नयथ् सु॒रेताः᳚ सु॒रेता॒ अज॑नय॒ दज॑नयथ् सु॒रेताः᳚ । \newline
22. सु॒रेता॒ इति॑ सु - रेताः᳚ । \newline
23. विश्वा॑ रू॒पाणि॑ रू॒पाणि॒ विश्वा॒ विश्वा॑ रू॒पाणि॑ । \newline
24. रू॒पाणि॒ प्रति॒ प्रति॑ रू॒पाणि॑ रू॒पाणि॒ प्रति॑ । \newline
25. प्रति॑ मुञ्चते मुञ्चते॒ प्रति॒ प्रति॑ मुञ्चते । \newline
26. मु॒ञ्च॒ते॒ क॒विः क॒विर् मु॑ञ्चते मुञ्चते क॒विः । \newline
27. क॒विः प्र प्र क॒विः क॒विः प्र । \newline
28. प्रासा॑वी दसावी॒त् प्र प्रासा॑वीत् । \newline
29. अ॒सा॒वी॒द् भ॒द्रम् भ॒द्र म॑सावी दसावीद् भ॒द्रम् । \newline
30. भ॒द्रम् द्वि॒पदे᳚ द्वि॒पदे॑ भ॒द्रम् भ॒द्रम् द्वि॒पदे᳚ । \newline
31. द्वि॒पदे॒ चतु॑ष्पदे॒ चतु॑ष्पदे द्वि॒पदे᳚ द्वि॒पदे॒ चतु॑ष्पदे । \newline
32. द्वि॒पद॒ इति॑ द्वि - पदे᳚ । \newline
33. चतु॑ष्पद॒ इति॒ चतुः॑ - प॒दे॒ । \newline
34. वि नाक॒म् नाकं॒ ॅवि वि नाक᳚म् । \newline
35. नाक॑ मख्य दख्य॒न् नाक॒म् नाक॑ मख्यत् । \newline
36. अ॒ख्य॒थ् स॒वि॒ता स॑वि॒ता ऽख्य॑ दख्यथ् सवि॒ता । \newline
37. स॒वि॒ता वरे᳚ण्यो॒ वरे᳚ण्यः सवि॒ता स॑वि॒ता वरे᳚ण्यः । \newline
38. वरे॒ण्यो ऽन्वनु॒ वरे᳚ण्यो॒ वरे॒ण्यो ऽनु॑ । \newline
39. अनु॑ प्र॒याण॑म् प्र॒याण॒ मन्वनु॑ प्र॒याण᳚म् । \newline
40. प्र॒याण॑ मु॒षस॑ उ॒षसः॑ प्र॒याण॑म् प्र॒याण॑ मु॒षसः॑ । \newline
41. प्र॒याण॒मिति॑ प्र - यान᳚म् । \newline
42. उ॒षसो॒ वि व्यु॑षस॑ उ॒षसो॒ वि । \newline
43. वि रा॑जति राजति॒ वि वि रा॑जति । \newline
44. रा॒ज॒तीति॑ राजति । \newline
45. नक्तो॒षासा॒ सम॑नसा॒ सम॑नसा॒ नक्तो॒षासा॒ नक्तो॒षासा॒ सम॑नसा । \newline
46. सम॑नसा॒ विरू॑पे॒ विरू॑पे॒ सम॑नसा॒ सम॑नसा॒ विरू॑पे । \newline
47. सम॑न॒सेति॒ स - म॒न॒सा॒ । \newline
48. विरू॑पे धा॒पये॑ते धा॒पये॑ते॒ विरू॑पे॒ विरू॑पे धा॒पये॑ते । \newline
49. विरू॑पे॒ इति॒ वि - रू॒पे॒ । \newline
50. धा॒पये॑ते॒ शिशुꣳ॒॒ शिशु॑म् धा॒पये॑ते धा॒पये॑ते॒ शिशु᳚म् । \newline
51. धा॒पये॑ते॒ इति॑ धा॒पये॑ते । \newline
52. शिशु॒ मेक॒ मेकꣳ॒॒ शिशुꣳ॒॒ शिशु॒ मेक᳚म् । \newline
53. एकꣳ॑ स॒मीची॑ स॒मीची॒ एक॒ मेकꣳ॑ स॒मीची᳚ । \newline
54. स॒मीची॒ इति॑ स॒मीची᳚ । \newline
55. द्यावा॒ क्षाम॒ क्षाम॒ द्यावा॒ द्यावा॒ क्षाम॑ । \newline
56. क्षामा॑ रु॒क्मो रु॒क्मः क्षाम॒ क्षामा॑ रु॒क्मः । \newline
57. रु॒क्मो अ॒न्त र॒न्ता रु॒क्मो रु॒क्मो अ॒न्तः । \newline

\textbf{Ghana Paata } \newline

1. स्वाꣳ अ॒ह म॒हꣳ स्वान् थ्स्वाꣳ अ॒हम् । \newline
2. अ॒हमित्य॒हम् । \newline
3. दृ॒शा॒नो रु॒क्मो रु॒क्मो दृ॑शा॒नो दृ॑शा॒नो रु॒क्म उ॒र्व्योर्व्या रु॒क्मो दृ॑शा॒नो दृ॑शा॒नो रु॒क्म उ॒र्व्या । \newline
4. रु॒क्म उ॒र्व्योर्व्या रु॒क्मो रु॒क्म उ॒र्व्या वि व्यु॑र्व्या रु॒क्मो रु॒क्म उ॒र्व्या वि । \newline
5. उ॒र्व्या वि व्यु॑र्व्योर्व्या व्य॑द्यौ दद्यौ॒द् व्यु॑र्व्योर्व्या व्य॑द्यौत् । \newline
6. व्य॑द्यौ दद्यौ॒द् वि व्य॑द्यौद् दु॒र्मर्.ष॑म् दु॒र्मर्.ष॑ मद्यौ॒द् वि व्य॑द्यौद् दु॒र्मर्.ष᳚म् । \newline
7. अ॒द्यौ॒द् दु॒र्मर्.ष॑म् दु॒र्मर्.ष॑ मद्यौ दद्यौद् दु॒र्मर्.ष॒ मायु॒ रायु॑र् दु॒र्मर्.ष॑ मद्यौ दद्यौद् दु॒र्मर्.ष॒ मायुः॑ । \newline
8. दु॒र्मर्.ष॒ मायु॒ रायु॑र् दु॒र्मर्.ष॑म् दु॒र्मर्.ष॒ मायुः॑ श्रि॒ये श्रि॒य आयु॑र् दु॒र्मर्.ष॑म् दु॒र्मर्.ष॒ मायुः॑ श्रि॒ये । \newline
9. दु॒र्मर्.ष॒मिति॑ दुः - मर्.ष᳚म् । \newline
10. आयुः॑ श्रि॒ये श्रि॒य आयु॒ रायुः॑ श्रि॒ये रु॑चा॒नो रु॑चा॒नः श्रि॒य आयु॒ रायुः॑ श्रि॒ये रु॑चा॒नः । \newline
11. श्रि॒ये रु॑चा॒नो रु॑चा॒नः श्रि॒ये श्रि॒ये रु॑चा॒नः । \newline
12. रु॒चा॒न इति॑ रुचा॒नः । \newline
13. अ॒ग्नि र॒मृतो॑ अ॒मृतो॑ अ॒ग्नि र॒ग्नि र॒मृतो॑ अभव दभव द॒मृतो॑ अ॒ग्नि र॒ग्नि र॒मृतो॑ अभवत् । \newline
14. अ॒मृतो॑ अभव दभव द॒मृतो॑ अ॒मृतो॑ अभव॒द् वयो॑भि॒र् वयो॑भि रभव द॒मृतो॑ अ॒मृतो॑ अभव॒द् वयो॑भिः । \newline
15. अ॒भ॒व॒द् वयो॑भि॒र् वयो॑भि रभव दभव॒द् वयो॑भि॒र् यद् यद् वयो॑भि रभव दभव॒द् वयो॑भि॒र् यत् । \newline
16. वयो॑भि॒र् यद् यद् वयो॑भि॒र् वयो॑भि॒र् यदे॑न मेनं॒ ॅयद् वयो॑भि॒र् वयो॑भि॒र् यदे॑नम् । \newline
17. वयो॑भि॒रिति॒ वयः॑ - भिः॒ । \newline
18. यदे॑न मेनं॒ ॅयद् यदे॑न॒म् द्यौर् द्यौ रे॑नं॒ ॅयद् यदे॑न॒म् द्यौः । \newline
19. ए॒न॒म् द्यौर् द्यौ रे॑न मेन॒म् द्यौ रज॑नय॒ दज॑नय॒द् द्यौरे॑न मेन॒म् द्यौ रज॑नयत् । \newline
20. द्यौ रज॑नय॒ दज॑नय॒द् द्यौर् द्यौ रज॑नयथ् सु॒रेताः᳚ सु॒रेता॒ अज॑नय॒द् द्यौर् द्यौ रज॑नयथ् सु॒रेताः᳚ । \newline
21. अज॑नयथ् सु॒रेताः᳚ सु॒रेता॒ अज॑नय॒ दज॑नयथ् सु॒रेताः᳚ । \newline
22. सु॒रेता॒ इति॑ सु - रेताः᳚ । \newline
23. विश्वा॑ रू॒पाणि॑ रू॒पाणि॒ विश्वा॒ विश्वा॑ रू॒पाणि॒ प्रति॒ प्रति॑ रू॒पाणि॒ विश्वा॒ विश्वा॑ रू॒पाणि॒ प्रति॑ । \newline
24. रू॒पाणि॒ प्रति॒ प्रति॑ रू॒पाणि॑ रू॒पाणि॒ प्रति॑ मुञ्चते मुञ्चते॒ प्रति॑ रू॒पाणि॑ रू॒पाणि॒ प्रति॑ मुञ्चते । \newline
25. प्रति॑ मुञ्चते मुञ्चते॒ प्रति॒ प्रति॑ मुञ्चते क॒विः क॒विर् मु॑ञ्चते॒ प्रति॒ प्रति॑ मुञ्चते क॒विः । \newline
26. मु॒ञ्च॒ते॒ क॒विः क॒विर् मु॑ञ्चते मुञ्चते क॒विः प्र प्र क॒विर् मु॑ञ्चते मुञ्चते क॒विः प्र । \newline
27. क॒विः प्र प्र क॒विः क॒विः प्रासा॑वी दसावी॒त् प्र क॒विः क॒विः प्रासा॑वीत् । \newline
28. प्रासा॑वी दसावी॒त् प्र प्रासा॑वीद् भ॒द्रम् भ॒द्र म॑सावी॒त् प्र प्रासा॑वीद् भ॒द्रम् । \newline
29. अ॒सा॒वी॒द् भ॒द्रम् भ॒द्र म॑सावी दसावीद् भ॒द्रम् द्वि॒पदे᳚ द्वि॒पदे॑ भ॒द्र म॑सावी दसावीद् भ॒द्रम् द्वि॒पदे᳚ । \newline
30. भ॒द्रम् द्वि॒पदे᳚ द्वि॒पदे॑ भ॒द्रम् भ॒द्रम् द्वि॒पदे॒ चतु॑ष्पदे॒ चतु॑ष्पदे द्वि॒पदे॑ भ॒द्रम् भ॒द्रम् द्वि॒पदे॒ चतु॑ष्पदे । \newline
31. द्वि॒पदे॒ चतु॑ष्पदे॒ चतु॑ष्पदे द्वि॒पदे᳚ द्वि॒पदे॒ चतु॑ष्पदे । \newline
32. द्वि॒पद॒ इति॑ द्वि - पदे᳚ । \newline
33. चतु॑ष्पद॒ इति॒ चतुः॑ - प॒दे॒ । \newline
34. वि नाक॒न्नाकं॒ ॅवि वि नाक॑ मख्य दख्य॒न् नाकं॒ ॅवि वि नाक॑ मख्यत् । \newline
35. नाक॑ मख्य दख्य॒न् नाक॒न् नाक॑ मख्यथ् सवि॒ता स॑वि॒ता ऽख्य॒न् नाक॒न् नाक॑ मख्यथ् सवि॒ता । \newline
36. अ॒ख्य॒थ् स॒वि॒ता स॑वि॒ता ऽख्य॑ दख्यथ् सवि॒ता वरे᳚ण्यो॒ वरे᳚ण्यः सवि॒ता ऽख्य॑ दख्यथ् सवि॒ता वरे᳚ण्यः । \newline
37. स॒वि॒ता वरे᳚ण्यो॒ वरे᳚ण्यः सवि॒ता स॑वि॒ता वरे॒ण्यो ऽन्वनु॒ वरे᳚ण्यः सवि॒ता स॑वि॒ता वरे॒ण्यो ऽनु॑ । \newline
38. वरे॒ण्यो ऽन्वनु॒ वरे᳚ण्यो॒ वरे॒ण्यो ऽनु॑ प्र॒याण॑म् प्र॒याण॒ मनु॒ वरे᳚ण्यो॒ वरे॒ण्यो ऽनु॑ प्र॒याण᳚म् । \newline
39. अनु॑ प्र॒याण॑म् प्र॒याण॒ मन्वनु॑ प्र॒याण॑ मु॒षस॑ उ॒षसः॑ प्र॒याण॒ मन्वनु॑ प्र॒याण॑ मु॒षसः॑ । \newline
40. प्र॒याण॑ मु॒षस॑ उ॒षसः॑ प्र॒याण॑म् प्र॒याण॑ मु॒षसो॒ वि व्यु॑षसः॑ प्र॒याण॑म् प्र॒याण॑ मु॒षसो॒ वि । \newline
41. प्र॒याण॒मिति॑ प्र - यान᳚म् । \newline
42. उ॒षसो॒ वि व्यु॑षस॑ उ॒षसो॒ वि रा॑जति राजति॒ व्यु॑षस॑ उ॒षसो॒ वि रा॑जति । \newline
43. वि रा॑जति राजति॒ वि वि रा॑जति । \newline
44. रा॒ज॒तीति॑ राजति । \newline
45. नक्तो॒षासा॒ सम॑नसा॒ सम॑नसा॒ नक्तो॒षासा॒ नक्तो॒षासा॒ सम॑नसा॒ विरू॑पे॒ विरू॑पे॒ सम॑नसा॒ नक्तो॒षासा॒ नक्तो॒षासा॒ सम॑नसा॒ विरू॑पे । \newline
46. सम॑नसा॒ विरू॑पे॒ विरू॑पे॒ सम॑नसा॒ सम॑नसा॒ विरू॑पे धा॒पये॑ते धा॒पये॑ते॒ विरू॑पे॒ सम॑नसा॒ सम॑नसा॒ विरू॑पे धा॒पये॑ते । \newline
47. सम॑न॒सेति॒ स - म॒न॒सा॒ । \newline
48. विरू॑पे धा॒पये॑ते धा॒पये॑ते॒ विरू॑पे॒ विरू॑पे धा॒पये॑ते॒ शिशुꣳ॒॒ शिशु॑म् धा॒पये॑ते॒ विरू॑पे॒ विरू॑पे धा॒पये॑ते॒ शिशु᳚म् । \newline
49. विरू॑पे॒ इति॒ वि - रू॒पे॒ । \newline
50. धा॒पये॑ते॒ शिशुꣳ॒॒ शिशु॑म् धा॒पये॑ते धा॒पये॑ते॒ शिशु॒ मेक॒ मेकꣳ॒॒ शिशु॑म् धा॒पये॑ते धा॒पये॑ते॒ शिशु॒ मेक᳚म् । \newline
51. धा॒पये॑ते॒ इति॑ धा॒पये॑ते । \newline
52. शिशु॒ मेक॒ मेकꣳ॒॒ शिशुꣳ॒॒ शिशु॒ मेकꣳ॑ स॒मीची॑ स॒मीची॒ एकꣳ॒॒ शिशुꣳ॒॒ शिशु॒ मेकꣳ॑ स॒मीची᳚ । \newline
53. एकꣳ॑ स॒मीची॑ स॒मीची॒ एक॒ मेकꣳ॑ स॒मीची᳚ । \newline
54. स॒मीची॒ इति॑ स॒मीची᳚ । \newline
55. द्यावा॒ क्षाम॒ क्षाम॒ द्यावा॒ द्यावा॒ क्षामा॑ रु॒क्मो रु॒क्मः क्षाम॒ द्यावा॒ द्यावा॒ क्षामा॑ रु॒क्मः । \newline
56. क्षामा॑ रु॒क्मो रु॒क्मः क्षाम॒ क्षामा॑ रु॒क्मो अ॒न्त र॒न्ता रु॒क्मः क्षाम॒ क्षामा॑ रु॒क्मो अ॒न्तः । \newline
57. रु॒क्मो अ॒न्त र॒न्ता रु॒क्मो रु॒क्मो अ॒न्तर् वि व्य॑न्ता रु॒क्मो रु॒क्मो अ॒न्तर् वि । \newline
\pagebreak
\markright{ TS 4.1.10.5  \hfill https://www.vedavms.in \hfill}

\section{ TS 4.1.10.5 }

\textbf{TS 4.1.10.5 } \newline
\textbf{Samhita Paata} \newline

अ॒न्तर्वि भा॑ति दे॒वा अ॒ग्निं धा॑रयन् द्रविणो॒दाः ॥ सु॒प॒र्णो॑ऽसि ग॒रुत्मा᳚न् त्रि॒वृत्ते॒ शिरो॑ गाय॒त्रं चक्षुः॒ स्तोम॑ आ॒त्मा साम॑ ते त॒नूर्वा॑मदे॒व्यं बृ॑हद्-रथन्त॒रे प॒क्षौ य॑ज्ञाय॒ज्ञियं॒ पुच्छं॒ छन्दाꣳ॒॒स्यङ्गा॑नि॒ धिष्णि॑याः श॒फा यजूꣳ॑षि॒ नाम॑ ॥ सु॒प॒र्णो॑ऽसि ग॒रुत्मा॒न् दिवं॑ गच्छ॒ सुवः॑ पत ॥ \newline

\textbf{Pada Paata} \newline

अ॒न्तः । वीति॑ । भा॒ति॒ । दे॒वाः । अ॒ग्निम् । धा॒र॒य॒न्न् । द्र॒वि॒णो॒दा इति॑ द्रविणः - दाः ॥ सु॒प॒र्ण इति॑ सु - प॒र्णः । अ॒सि॒ । ग॒रुत्मान्॑ । त्रि॒वृदिति॑ त्रि - वृत् । ते॒ । शिरः॑ । गा॒य॒त्रम् । चक्षुः॑ । स्तोमः॑ । आ॒त्मा । साम॑ । ते॒ । त॒नूः । वा॒म॒दे॒व्यमिति॑ वाम - दे॒व्यम् । बृ॒ह॒द्र॒थ॒न्त॒रे इति॑ बृहत् - र॒थ॒न्त॒रे । प॒क्षौ । य॒ज्ञा॒य॒ज्ञिय᳚म् । पुच्छ᳚म् । छन्दाꣳ॑सि । अङ्गा॑नि । धिष्णि॑याः । श॒फाः । यजूꣳ॑षि । नाम॑ ॥ सु॒प॒र्ण इति॑ सु - प॒र्णः । अ॒सि॒ । ग॒रुत्मान्॑ । दिव᳚म् । ग॒च्छ॒ । सुवः॑ । प॒त॒ ॥  \newline


\textbf{Krama Paata} \newline

अ॒न्तर् वि । वि भा॑ति । भा॒ति॒ दे॒वाः । दे॒वा अ॒ग्निम् । अ॒ग्निम् धा॑रयन्न् । धा॒र॒य॒न् द्र॒वि॒णो॒दाः । द्र॒वि॒णो॒दा इति॑ द्रविणः - दाः ॥ सु॒प॒र्णो॑ऽसि । सु॒प॒र्ण इति॑ सु - प॒र्णः । अ॒सि॒ ग॒रुत्मान्॑ । ग॒रुत्मा᳚न् त्रि॒वृत् । त्रि॒वृत् ते᳚ । त्रि॒वृदिति॑ त्रि - वृत् । ते॒ शिरः॑ । शिरो॑ गाय॒त्रम् । गा॒य॒त्रम् चक्षुः॑ । चक्षुः॒ स्तोमः॑ । स्तोम॑ आ॒त्मा । आ॒त्मा साम॑ । साम॑ ते । ते॒ त॒नूः । त॒नूर् वा॑मदे॒व्यम् । वा॒म॒दे॒व्यम् बृ॑हद्रथन्त॒रे । वा॒म॒दे॒व्यमिति॑ वाम - दे॒व्यम् । बृ॒ह॒द्र॒थ॒न्त॒रे प॒क्षौ । बृ॒ह॒द्र॒थ॒न्त॒रे इति॑ बृहत् - र॒थ॒न्त॒रे । प॒क्षौ य॑ज्ञाय॒ज्ञिय᳚म् । य॒ज्ञा॒य॒ज्ञिय॒म् पुच्छ᳚म् । पुच्छ॒म् छन्दाꣳ॑सि । छन्दाꣳ॒॒स्यङ्गा॑नि । अङ्गा॑नि॒ धिष्णि॑याः । धिष्णि॑याः श॒फाः । श॒फा यजूꣳ॑षि । यजूꣳ॑षि॒ नाम॑ । नामेति॒ नाम॑ ॥ सु॒प॒र्णो॑ऽसि । सु॒प॒र्ण इति॑ सु - प॒र्णः । अ॒सि॒ ग॒रुत्मान्॑ । ग॒रुत्मा॒न् दिव᳚म् । दिव॑म् गच्छ । ग॒च्छ॒ सुवः॑ । सुवः॑ पत । प॒तेति॑ पत । \newline

\textbf{Jatai Paata} \newline

1. अ॒न्तर् वि व्य॑न्त र॒न्तर् वि । \newline
2. वि भा॑ति भाति॒ वि वि भा॑ति । \newline
3. भा॒ति॒ दे॒वा दे॒वा भा॑ति भाति दे॒वाः । \newline
4. दे॒वा अ॒ग्नि म॒ग्निम् दे॒वा दे॒वा अ॒ग्निम् । \newline
5. अ॒ग्निम् धा॑रयन् धारयन् न॒ग्नि म॒ग्निम् धा॑रयन्न् । \newline
6. धा॒र॒य॒न् द्र॒वि॒णो॒दा द्र॑विणो॒दा धा॑रयन् धारयन् द्रविणो॒दाः । \newline
7. द्र॒वि॒णो॒दा इति॑ द्रविणः - दाः । \newline
8. सु॒प॒र्णो᳚ ऽस्यसि सुप॒र्णः सु॑प॒र्णो॑ ऽसि । \newline
9. सु॒प॒र्ण इति॑ सु - प॒र्णः । \newline
10. अ॒सि॒ ग॒रुत्मा᳚न् ग॒रुत्मा॑ नस्यसि ग॒रुत्मान्॑ । \newline
11. ग॒रुत्मा᳚न् त्रि॒वृत् त्रि॒वृद् ग॒रुत्मा᳚न् ग॒रुत्मा᳚न् त्रि॒वृत् । \newline
12. त्रि॒वृत् ते॑ ते त्रि॒वृत् त्रि॒वृत् ते᳚ । \newline
13. त्रि॒वृदिति॑ त्रि - वृत् । \newline
14. ते॒ शिरः॒ शिर॑ स्ते ते॒ शिरः॑ । \newline
15. शिरो॑ गाय॒त्रम् गा॑य॒त्रꣳ शिरः॒ शिरो॑ गाय॒त्रम् । \newline
16. गा॒य॒त्रम् चक्षु॒ श्चक्षु॑र् गाय॒त्रम् गा॑य॒त्रम् चक्षुः॑ । \newline
17. चक्षुः॒ स्तोमः॒ स्तोम॒ श्चक्षु॒ श्चक्षुः॒ स्तोमः॑ । \newline
18. स्तोम॑ आ॒त्मा ऽऽत्मा स्तोमः॒ स्तोम॑ आ॒त्मा । \newline
19. आ॒त्मा साम॒ सामा॒त्मा ऽऽत्मा साम॑ । \newline
20. साम॑ ते ते॒ साम॒ साम॑ ते । \newline
21. ते॒ त॒नू स्त॒नू स्ते॑ ते त॒नूः । \newline
22. त॒नूर् वा॑मदे॒व्यं ॅवा॑मदे॒व्यम् त॒नू स्त॒नूर् वा॑मदे॒व्यम् । \newline
23. वा॒म॒दे॒व्यम् बृ॑हद्रथन्त॒रे बृ॑हद्रथन्त॒रे वा॑मदे॒व्यं ॅवा॑मदे॒व्यम् बृ॑हद्रथन्त॒रे । \newline
24. वा॒म॒दे॒व्यमिति॑ वाम - दे॒व्यम् । \newline
25. बृ॒ह॒द्र॒थ॒न्त॒रे प॒क्षौ प॒क्षौ बृ॑हद्रथन्त॒रे बृ॑हद्रथन्त॒रे प॒क्षौ । \newline
26. बृ॒ह॒द्र॒थ॒न्त॒रे इति॑ बृहत् - र॒थ॒न्त॒रे । \newline
27. प॒क्षौ य॑ज्ञाय॒ज्ञियं॑ ॅयज्ञाय॒ज्ञिय॑म् प॒क्षौ प॒क्षौ य॑ज्ञाय॒ज्ञिय᳚म् । \newline
28. य॒ज्ञा॒य॒ज्ञिय॒म् पुच्छ॒म् पुच्छं॑ ॅयज्ञाय॒ज्ञियं॑ ॅयज्ञाय॒ज्ञिय॒म् पुच्छ᳚म् । \newline
29. पुच्छ॒म् छन्दाꣳ॑सि॒ छन्दाꣳ॑सि॒ पुच्छ॒म् पुच्छ॒म् छन्दाꣳ॑सि । \newline
30. छन्दाꣳ॒॒ स्यङ्गा॒ न्यङ्गा॑नि॒ छन्दाꣳ॑सि॒ छन्दाꣳ॒॒ स्यङ्गा॑नि । \newline
31. अङ्गा॑नि॒ धिष्णि॑या॒ धिष्णि॑या॒ अङ्गा॒ न्यङ्गा॑नि॒ धिष्णि॑याः । \newline
32. धिष्णि॑याः श॒फाः श॒फा धिष्णि॑या॒ धिष्णि॑याः श॒फाः । \newline
33. श॒फा यजूꣳ॑षि॒ यजूꣳ॑षि श॒फाः श॒फा यजूꣳ॑षि । \newline
34. यजूꣳ॑षि॒ नाम॒ नाम॒ यजूꣳ॑षि॒ यजूꣳ॑षि॒ नाम॑ । \newline
35. नामेति॒ नाम॑ । \newline
36. सु॒प॒र्णो᳚ ऽस्यसि सुप॒र्णः सु॑प॒र्णो॑ ऽसि । \newline
37. सु॒प॒र्ण इति॑ सु - प॒र्णः । \newline
38. अ॒सि॒ ग॒रुत्मा᳚न् ग॒रुत्मा॑ नस्यसि ग॒रुत्मान्॑ । \newline
39. ग॒रुत्मा॒न् दिव॒म् दिव॑म् ग॒रुत्मा᳚न् ग॒रुत्मा॒न् दिव᳚म् । \newline
40. दिव॑म् गच्छ गच्छ॒ दिव॒म् दिव॑म् गच्छ । \newline
41. ग॒च्छ॒ सुवः॒ सुव॑र् गच्छ गच्छ॒ सुवः॑ । \newline
42. सुवः॑ पत पत॒ सुवः॒ सुवः॑ पत । \newline
43. प॒तेति॑ पत । \newline

\textbf{Ghana Paata } \newline

1. अ॒न्तर् वि व्य॑न्त र॒न्तर् वि भा॑ति भाति॒ व्य॑न्त र॒न्तर् वि भा॑ति । \newline
2. वि भा॑ति भाति॒ वि वि भा॑ति दे॒वा दे॒वा भा॑ति॒ वि वि भा॑ति दे॒वाः । \newline
3. भा॒ति॒ दे॒वा दे॒वा भा॑ति भाति दे॒वा अ॒ग्नि म॒ग्निम् दे॒वा भा॑ति भाति दे॒वा अ॒ग्निम् । \newline
4. दे॒वा अ॒ग्नि म॒ग्निम् दे॒वा दे॒वा अ॒ग्निम् धा॑रयन् धारयन् न॒ग्निम् दे॒वा दे॒वा अ॒ग्निम् धा॑रयन्न् । \newline
5. अ॒ग्निम् धा॑रयन् धारयन् न॒ग्नि म॒ग्निम् धा॑रयन् द्रविणो॒दा द्र॑विणो॒दा धा॑रयन् न॒ग्नि म॒ग्निम् धा॑रयन् द्रविणो॒दाः । \newline
6. धा॒र॒य॒न् द्र॒वि॒णो॒दा द्र॑विणो॒दा धा॑रयन् धारयन् द्रविणो॒दाः । \newline
7. द्र॒वि॒णो॒दा इति॑ द्रविणः - दाः । \newline
8. सु॒प॒र्णो᳚ ऽस्यसि सुप॒र्णः सु॑प॒र्णो॑ ऽसि ग॒रुत्मा᳚न् ग॒रुत्मा॑ नसि सुप॒र्णः सु॑प॒र्णो॑ ऽसि ग॒रुत्मान्॑ । \newline
9. सु॒प॒र्ण इति॑ सु - प॒र्णः । \newline
10. अ॒सि॒ ग॒रुत्मा᳚न् ग॒रुत्मा॑ नस्यसि ग॒रुत्मा᳚न् त्रि॒वृत् त्रि॒वृद् ग॒रुत्मा॑ नस्यसि ग॒रुत्मा᳚न् त्रि॒वृत् । \newline
11. ग॒रुत्मा᳚न् त्रि॒वृत् त्रि॒वृद् ग॒रुत्मा᳚न् ग॒रुत्मा᳚न् त्रि॒वृत् ते॑ ते त्रि॒वृद् ग॒रुत्मा᳚न् ग॒रुत्मा᳚न् त्रि॒वृत् ते᳚ । \newline
12. त्रि॒वृत् ते॑ ते त्रि॒वृत् त्रि॒वृत् ते॒ शिरः॒ शिर॑ स्ते त्रि॒वृत् त्रि॒वृत् ते॒ शिरः॑ । \newline
13. त्रि॒वृदिति॑ त्रि - वृत् । \newline
14. ते॒ शिरः॒ शिर॑ स्ते ते॒ शिरो॑ गाय॒त्रम् गा॑य॒त्रꣳ शिर॑ स्ते ते॒ शिरो॑ गाय॒त्रम् । \newline
15. शिरो॑ गाय॒त्रम् गा॑य॒त्रꣳ शिरः॒ शिरो॑ गाय॒त्रम् चक्षु॒ श्चक्षु॑र् गाय॒त्रꣳ शिरः॒ शिरो॑ गाय॒त्रम् चक्षुः॑ । \newline
16. गा॒य॒त्रम् चक्षु॒ श्चक्षु॑र् गाय॒त्रम् गा॑य॒त्रम् चक्षुः॒ स्तोमः॒ स्तोम॒ श्चक्षु॑र् गाय॒त्रम् गा॑य॒त्रम् चक्षुः॒ स्तोमः॑ । \newline
17. चक्षुः॒ स्तोमः॒ स्तोम॒ श्चक्षु॒ श्चक्षुः॒ स्तोम॑ आ॒त्मा ऽऽत्मा स्तोम॒ श्चक्षु॒ श्चक्षुः॒ स्तोम॑ आ॒त्मा । \newline
18. स्तोम॑ आ॒त्मा ऽऽत्मा स्तोमः॒ स्तोम॑ आ॒त्मा साम॒ सामा॒त्मा स्तोमः॒ स्तोम॑ आ॒त्मा साम॑ । \newline
19. आ॒त्मा साम॒ सामा॒त्मा ऽऽत्मा साम॑ ते ते॒ सामा॒त्मा ऽऽत्मा साम॑ ते । \newline
20. साम॑ ते ते॒ साम॒ साम॑ ते त॒नू स्त॒नू स्ते॒ साम॒ साम॑ ते त॒नूः । \newline
21. ते॒ त॒नू स्त॒नू स्ते॑ ते त॒नूर् वा॑मदे॒व्यं ॅवा॑मदे॒व्यम् त॒नू स्ते॑ ते त॒नूर् वा॑मदे॒व्यम् । \newline
22. त॒नूर् वा॑मदे॒व्यं ॅवा॑मदे॒व्यम् त॒नू स्त॒नूर् वा॑मदे॒व्यम् बृ॑हद्रथन्त॒रे बृ॑हद्रथन्त॒रे वा॑मदे॒व्यम् त॒नू स्त॒नूर् वा॑मदे॒व्यम् बृ॑हद्रथन्त॒रे । \newline
23. वा॒म॒दे॒व्यम् बृ॑हद्रथन्त॒रे बृ॑हद्रथन्त॒रे वा॑मदे॒व्यं ॅवा॑मदे॒व्यम् बृ॑हद्रथन्त॒रे प॒क्षौ प॒क्षौ बृ॑हद्रथन्त॒रे वा॑मदे॒व्यं ॅवा॑मदे॒व्यम् बृ॑हद्रथन्त॒रे प॒क्षौ । \newline
24. वा॒म॒दे॒व्यमिति॑ वाम - दे॒व्यम् । \newline
25. बृ॒ह॒द्र॒थ॒न्त॒रे प॒क्षौ प॒क्षौ बृ॑हद्रथन्त॒रे बृ॑हद्रथन्त॒रे प॒क्षौ य॑ज्ञाय॒ज्ञियं॑ ॅयज्ञाय॒ज्ञिय॑म् प॒क्षौ बृ॑हद्रथन्त॒रे बृ॑हद्रथन्त॒रे प॒क्षौ य॑ज्ञाय॒ज्ञिय᳚म् । \newline
26. बृ॒ह॒द्र॒थ॒न्त॒रे इति॑ बृहत् - र॒थ॒न्त॒रे । \newline
27. प॒क्षौ य॑ज्ञाय॒ज्ञियं॑ ॅयज्ञाय॒ज्ञिय॑म् प॒क्षौ प॒क्षौ य॑ज्ञाय॒ज्ञिय॒म् पुच्छ॒म् पुच्छं॑ ॅयज्ञाय॒ज्ञिय॑म् प॒क्षौ प॒क्षौ य॑ज्ञाय॒ज्ञिय॒म् पुच्छ᳚म् । \newline
28. य॒ज्ञा॒य॒ज्ञिय॒म् पुच्छ॒म् पुच्छं॑ ॅयज्ञाय॒ज्ञियं॑ ॅयज्ञाय॒ज्ञिय॒म् पुच्छ॒म् छन्दाꣳ॑सि॒ छन्दाꣳ॑सि॒ पुच्छं॑ ॅयज्ञाय॒ज्ञियं॑ ॅयज्ञाय॒ज्ञिय॒म् पुच्छ॒म् छन्दाꣳ॑सि । \newline
29. पुच्छ॒म् छन्दाꣳ॑सि॒ छन्दाꣳ॑सि॒ पुच्छ॒म् पुच्छ॒म् छन्दाꣳ॒॒ स्यङ्गा॒ न्यङ्गा॑नि॒ छन्दाꣳ॑सि॒ पुच्छ॒म् पुच्छ॒म् छन्दाꣳ॒॒ स्यङ्गा॑नि । \newline
30. छन्दाꣳ॒॒ स्यङ्गा॒ न्यङ्गा॑नि॒ छन्दाꣳ॑सि॒ छन्दाꣳ॒॒ स्यङ्गा॑नि॒ धिष्णि॑या॒ धिष्णि॑या॒ अङ्गा॑नि॒ छन्दाꣳ॑सि॒ छन्दाꣳ॒॒ स्यङ्गा॑नि॒ धिष्णि॑याः । \newline
31. अङ्गा॑नि॒ धिष्णि॑या॒ धिष्णि॑या॒ अङ्गा॒ न्यङ्गा॑नि॒ धिष्णि॑याः श॒फाः श॒फा धिष्णि॑या॒ अङ्गा॒ न्यङ्गा॑नि॒ धिष्णि॑याः श॒फाः । \newline
32. धिष्णि॑याः श॒फाः श॒फा धिष्णि॑या॒ धिष्णि॑याः श॒फा यजूꣳ॑षि॒ यजूꣳ॑षि श॒फा धिष्णि॑या॒ धिष्णि॑याः श॒फा यजूꣳ॑षि । \newline
33. श॒फा यजूꣳ॑षि॒ यजूꣳ॑षि श॒फाः श॒फा यजूꣳ॑षि॒ नाम॒ नाम॒ यजूꣳ॑षि श॒फाः श॒फा यजूꣳ॑षि॒ नाम॑ । \newline
34. यजूꣳ॑षि॒ नाम॒ नाम॒ यजूꣳ॑षि॒ यजूꣳ॑षि॒ नाम॑ । \newline
35. नामेति॒ नाम॑ । \newline
36. सु॒प॒र्णो᳚ ऽस्यसि सुप॒र्णः सु॑प॒र्णो॑ ऽसि ग॒रुत्मा᳚न् ग॒रुत्मा॑ नसि सुप॒र्णः सु॑प॒र्णो॑ ऽसि ग॒रुत्मान्॑ । \newline
37. सु॒प॒र्ण इति॑ सु - प॒र्णः । \newline
38. अ॒सि॒ ग॒रुत्मा᳚न् ग॒रुत्मा॑ नस्यसि ग॒रुत्मा॒न् दिव॒म् दिव॑म् ग॒रुत्मा॑ नस्यसि ग॒रुत्मा॒न् दिव᳚म् । \newline
39. ग॒रुत्मा॒न् दिव॒म् दिव॑म् ग॒रुत्मा᳚न् ग॒रुत्मा॒न् दिव॑म् गच्छ गच्छ॒ दिव॑म् ग॒रुत्मा᳚न् ग॒रुत्मा॒न् दिव॑म् गच्छ । \newline
40. दिव॑म् गच्छ गच्छ॒ दिव॒म् दिव॑म् गच्छ॒ सुवः॒ सुव॑र् गच्छ॒ दिव॒म् दिव॑म् गच्छ॒ सुवः॑ । \newline
41. ग॒च्छ॒ सुवः॒ सुव॑र् गच्छ गच्छ॒ सुवः॑ पत पत॒ सुव॑र् गच्छ गच्छ॒ सुवः॑ पत । \newline
42. सुवः॑ पत पत॒ सुवः॒ सुवः॑ पत । \newline
43. प॒तेति॑ पत । \newline
\pagebreak
\markright{ TS 4.1.11.1  \hfill https://www.vedavms.in \hfill}

\section{ TS 4.1.11.1 }

\textbf{TS 4.1.11.1 } \newline
\textbf{Samhita Paata} \newline

अग्ने॒ यं ॅय॒ज्ञ्म॑द्ध्व॒रं ॅवि॒श्वतः॑ परि॒भूरसि॑ । स इद्दे॒वेषु॑ गच्छति ॥ सोम॒ यास्ते॑ मयो॒भुव॑ ऊ॒तयः॒ सन्ति॑ दा॒शुषे᳚ । ताभि॑र्नोऽवि॒ता भ॑व ॥ अ॒ग्निर्मू॒र्धा >1 , भुवः॑ >त्वन्नः॑ सोम॒ >3 , या ते॒ धामा॑नि >4 ॥ तथ् स॑वि॒तुर्वरे᳚ण्यं॒ भर्गो॑ दे॒वस्य॑ धीमहि । धियो॒ यो नः॑ प्रचो॒दया᳚त् ॥ अचि॑त्ती॒ यच्च॑कृ॒मा दैव्ये॒ जने॑ दी॒नैर्दक्षैः॒ प्रभू॑ती पूरुष॒त्वता᳚ । \newline

\textbf{Pada Paata} \newline

अग्ने᳚ । यम् । य॒ज्ञ्म् । अ॒द्ध्व॒रम् । वि॒श्वतः॑ । प॒रि॒भूरिति॑ परि - भूः । असि॑ ॥ सः । इत् । दे॒वेषु॑ । ग॒च्छ॒ति॒ ॥ सोम॑ । याः । ते॒ । म॒यो॒भुव॒ इति॑ मयः - भुवः॑ । ऊ॒तयः॑ । सन्ति॑ । दा॒शुषे᳚ ॥ ताभिः॑ । नः॒ । अ॒वि॒ता । भ॒व॒ ॥ अ॒ग्निः । मू॒द्‌र्धा । भुवः॑ ॥ त्वम् । नः॒ । सो॒म॒ । या । ते॒ । धामा॑नि ॥ तत् । स॒वि॒तुः । वरे᳚ण्यम् । भर्गः॑ । दे॒वस्य॑ । धी॒म॒हि॒ ॥ धियः॑ । यः । नः॒ । प्र॒चो॒दया॒दिति॑ प्र-चो॒दया᳚त् ॥ अचि॑त्ती । यत् । च॒कृ॒म । दैव्ये᳚ । जने᳚ । दी॒नैः । दक्षैः᳚ । प्रभू॒तीति॒ प्र - भू॒ती॒ । पू॒रु॒ष॒त्वतेति॑ पूरुष - त्वता᳚ ॥  \newline


\textbf{Krama Paata} \newline

अग्ने॒ यम् । यं ॅय॒ज्ञ्म् । य॒ज्ञ्म॑द्ध्व॒रम् । अ॒द्ध्व॒रं ॅवि॒श्वतः॑ । वि॒श्वतः॑ परि॒भूः । प॒रि॒भूरसि॑ । प॒रि॒भूरिति॑ परि - भूः । असीत्यसि॑ ॥ स इत् । इद् दे॒वेषु॑ । दे॒वेषु॑ गच्छति । ग॒च्छ॒तीति॑ गच्छति ॥ सोम॒ याः । यास्ते᳚ । ते॒ म॒यो॒भुवः॑ । म॒यो॒भुव॑ ऊ॒तयः॑ । म॒यो॒भुव॒ इति॑ मयः - भुवः॑ । ऊ॒तयः॒ सन्ति॑ । सन्ति॑ दा॒शुषे᳚ । दा॒शुष॒ इति॑ दा॒शुषे᳚ ॥ ताभि॑र् नः । नो॒ऽवि॒ता । अ॒वि॒ता भ॑व । भ॒वेति॑ भव ॥ अ॒ग्निर् मू॒र्द्धा । मू॒र्द्धा भुवः॑ । भुव॒ इति॒ भुवः॑ ॥ त्वम् नः॑ । नः॒ सो॒म॒ । सो॒म॒ या । या ते᳚ । ते॒ धामा॑नि । धामा॒नीति॒ धामा॑नि ॥ तथ् स॑वि॒तुः । स॒वि॒तुर् वरे᳚ण्यम् । वरे᳚ण्य॒म् भर्गः॑ । भर्गो॑ दे॒वस्य॑ । दे॒वस्य॑ धीमहि । धी॒म॒हीति॑ धीमहि ॥ धियो॒ यः । यो नः॑ । नः॒ प्र॒चो॒दया᳚त् । प्र॒चो॒दया॒दिति॑ प्र - चो॒दया᳚त् । अचि॑त्ती॒ यत् । यच् च॑कृ॒म । च॒कृ॒मा दैव्ये᳚ । दैव्ये॒ जने᳚ । जने॑ दी॒नैः । दी॒नैर् दक्षैः᳚ । दक्षैः॒ प्रभू॑ती । प्रभू॑ती पूरुष॒त्वता᳚ । प्रभू॒तीति॒ प्र - भू॒ती॒ । पू॒रु॒ष॒त्वतेति॑ पूरुष - त्वता᳚ । \newline

\textbf{Jatai Paata} \newline

1. अग्ने॒ यं ॅय मग्ने ऽग्ने॒ यम् । \newline
2. यं ॅय॒ज्ञ्ं ॅय॒ज्ञ्ं ॅयं ॅयं ॅय॒ज्ञ्म् । \newline
3. य॒ज्ञ् म॑द्ध्व॒र म॑द्ध्व॒रं ॅय॒ज्ञ्ं ॅय॒ज्ञ् म॑द्ध्व॒रम् । \newline
4. अ॒द्ध्व॒रं ॅवि॒श्वतो॑ वि॒श्वतो॑ अद्ध्व॒र म॑द्ध्व॒रं ॅवि॒श्वतः॑ । \newline
5. वि॒श्वतः॑ परि॒भूः प॑रि॒भूर् वि॒श्वतो॑ वि॒श्वतः॑ परि॒भूः । \newline
6. प॒रि॒भू रस्यसि॑ परि॒भूः प॑रि॒भू रसि॑ । \newline
7. प॒रि॒भूरिति॑ परि - भूः । \newline
8. असीत्यसि॑ । \newline
9. स इदिथ् स स इत् । \newline
10. इद् दे॒वेषु॑ दे॒वे ष्विदिद् दे॒वेषु॑ । \newline
11. दे॒वेषु॑ गच्छति गच्छति दे॒वेषु॑ दे॒वेषु॑ गच्छति । \newline
12. ग॒च्छ॒तीति॑ गच्छति । \newline
13. सोम॒ या याः सोम॒ सोम॒ याः । \newline
14. या स्ते॑ ते॒ या या स्ते᳚ । \newline
15. ते॒ म॒यो॒भुवो॑ मयो॒भुव॑ स्ते ते मयो॒भुवः॑ । \newline
16. म॒यो॒भुव॑ ऊ॒तय॑ ऊ॒तयो॑ मयो॒भुवो॑ मयो॒भुव॑ ऊ॒तयः॑ । \newline
17. म॒यो॒भुव॒ इति॑ मयः - भुवः॑ । \newline
18. ऊ॒तयः॒ सन्ति॒ सन्त्यू॒तय॑ ऊ॒तयः॒ सन्ति॑ । \newline
19. सन्ति॑ दा॒शुषे॑ दा॒शुषे॒ सन्ति॒ सन्ति॑ दा॒शुषे᳚ । \newline
20. दा॒शुष॒ इति॑ दा॒शुषे᳚ । \newline
21. ताभि॑र् नो न॒ स्ताभि॒ स्ताभि॑र् नः । \newline
22. नो॒ ऽवि॒ता ऽवि॒ता नो॑ नो ऽवि॒ता । \newline
23. अ॒वि॒ता भ॑व भवावि॒ता ऽवि॒ता भ॑व । \newline
24. भ॒वेति॑ भव । \newline
25. अ॒ग्निर् मू॒र्द्धा मू॒र्द्धा ऽग्नि र॒ग्निर् मू॒र्द्धा । \newline
26. मू॒र्द्धा भुवो॒ भुवो॑ मू॒र्द्धा मू॒र्द्धा भुवः॑ । \newline
27. भुव॒ इति॒ भुवः॑ । \newline
28. त्वम् नो॑ न॒ स्त्वम् त्वम् नः॑ । \newline
29. नः॒ सो॒म॒ सो॒म॒ नो॒ नः॒ सो॒म॒ । \newline
30. सो॒म॒ या या सो॑म सोम॒ या । \newline
31. या ते॑ ते॒ या या ते᳚ । \newline
32. ते॒ धामा॑नि॒ धामा॑नि ते ते॒ धामा॑नि । \newline
33. धामा॒नीति॒ धामा॑नि । \newline
34. तथ् स॑वि॒तुः स॑वि॒तु स्तत् तथ् स॑वि॒तुः । \newline
35. स॒वि॒तुर् वरे᳚ण्यं॒ ॅवरे᳚ण्यꣳ सवि॒तुः स॑वि॒तुर् वरे᳚ण्यम् । \newline
36. वरे᳚ण्य॒म् भर्गो॒ भर्गो॒ वरे᳚ण्यं॒ ॅवरे᳚ण्य॒म् भर्गः॑ । \newline
37. भर्गो॑ दे॒वस्य॑ दे॒वस्य॒ भर्गो॒ भर्गो॑ दे॒वस्य॑ । \newline
38. दे॒वस्य॑ धीमहि धीमहि दे॒वस्य॑ दे॒वस्य॑ धीमहि । \newline
39. धी॒म॒हीति॑ धीमहि । \newline
40. धियो॒ यो यो धियो॒ धियो॒ यः । \newline
41. यो नो॑ नो॒ यो यो नः॑ । \newline
42. नः॒ प्र॒चो॒दया᳚त् प्रचो॒दया᳚न् नो नः प्रचो॒दया᳚त् । \newline
43. प्र॒चो॒दया॒दिति॑ प्र - चो॒दया᳚त् । \newline
44. अचि॑त्ती॒ यद् यदचि॒ त्त्यचि॑त्ती॒ यत् । \newline
45. यच् च॑कृ॒म च॑कृ॒म यद् यच् च॑कृ॒म । \newline
46. च॒कृ॒मा दैव्ये॒ दैव्ये॑ चकृ॒म च॑कृ॒मा दैव्ये᳚ । \newline
47. दैव्ये॒ जने॒ जने॒ दैव्ये॒ दैव्ये॒ जने᳚ । \newline
48. जने॑ दी॒नैर् दी॒नैर् जने॒ जने॑ दी॒नैः । \newline
49. दी॒नैर् दक्षै॒र् दक्षै᳚र् दी॒नैर् दी॒नैर् दक्षैः᳚ । \newline
50. दक्षैः॒ प्रभू॑ती॒ प्रभू॑ती॒ दक्षै॒र् दक्षैः॒ प्रभू॑ती । \newline
51. प्रभू॑ती पूरुष॒त्वता॑ पूरुष॒त्वता॒ प्रभू॑ती॒ प्रभू॑ती पूरुष॒त्वता᳚ । \newline
52. प्रभू॒तीति॒ प्र - भू॒ती॒ । \newline
53. पू॒रु॒ष॒त्वतेति॑ पूरुष - त्वता᳚ । \newline

\textbf{Ghana Paata } \newline

1. अग्ने॒ यं ॅय मग्ने ऽग्ने॒ यं ॅय॒ज्ञ्ं ॅय॒ज्ञ्ं ॅय मग्ने ऽग्ने॒ यं ॅय॒ज्ञ्म् । \newline
2. यं ॅय॒ज्ञ्ं ॅय॒ज्ञ्ं ॅयं ॅयं ॅय॒ज्ञ् म॑द्ध्व॒र म॑द्ध्व॒रं ॅय॒ज्ञ्ं ॅयं ॅयं ॅय॒ज्ञ् म॑द्ध्व॒रम् । \newline
3. य॒ज्ञ् म॑द्ध्व॒र म॑द्ध्व॒रं ॅय॒ज्ञ्ं ॅय॒ज्ञ् म॑द्ध्व॒रं ॅवि॒श्वतो॑ वि॒श्वतो॑ अद्ध्व॒रं ॅय॒ज्ञ्ं ॅय॒ज्ञ् म॑द्ध्व॒रं ॅवि॒श्वतः॑ । \newline
4. अ॒द्ध्व॒रं ॅवि॒श्वतो॑ वि॒श्वतो॑ अद्ध्व॒र म॑द्ध्व॒रं ॅवि॒श्वतः॑ परि॒भूः प॑रि॒भूर् वि॒श्वतो॑ अद्ध्व॒र म॑द्ध्व॒रं ॅवि॒श्वतः॑ परि॒भूः । \newline
5. वि॒श्वतः॑ परि॒भूः प॑रि॒भूर् वि॒श्वतो॑ वि॒श्वतः॑ परि॒भू रस्यसि॑ परि॒भूर् वि॒श्वतो॑ वि॒श्वतः॑ परि॒भू रसि॑ । \newline
6. प॒रि॒भू रस्यसि॑ परि॒भूः प॑रि॒भू रसि॑ । \newline
7. प॒रि॒भूरिति॑ परि - भूः । \newline
8. असीत्यसि॑ । \newline
9. स इदिथ् स स इद् दे॒वेषु॑ दे॒वेष्विथ् स स इद् दे॒वेषु॑ । \newline
10. इद् दे॒वेषु॑ दे॒वे ष्विदिद् दे॒वेषु॑ गच्छति गच्छति दे॒वे ष्विदिद् दे॒वेषु॑ गच्छति । \newline
11. दे॒वेषु॑ गच्छति गच्छति दे॒वेषु॑ दे॒वेषु॑ गच्छति । \newline
12. ग॒च्छ॒तीति॑ गच्छति । \newline
13. सोम॒ या याः सोम॒ सोम॒ या स्ते॑ ते॒ याः सोम॒ सोम॒ या स्ते᳚ । \newline
14. या स्ते॑ ते॒ या या स्ते॑ मयो॒भुवो॑ मयो॒भुव॑ स्ते॒ या या स्ते॑ मयो॒भुवः॑ । \newline
15. ते॒ म॒यो॒भुवो॑ मयो॒भुव॑ स्ते ते मयो॒भुव॑ ऊ॒तय॑ ऊ॒तयो॑ मयो॒भुव॑ स्ते ते मयो॒भुव॑ ऊ॒तयः॑ । \newline
16. म॒यो॒भुव॑ ऊ॒तय॑ ऊ॒तयो॑ मयो॒भुवो॑ मयो॒भुव॑ ऊ॒तयः॒ सन्ति॒ सन्त्यू॒तयो॑ मयो॒भुवो॑ मयो॒भुव॑ ऊ॒तयः॒ सन्ति॑ । \newline
17. म॒यो॒भुव॒ इति॑ मयः - भुवः॑ । \newline
18. ऊ॒तयः॒ सन्ति॒ सन्त्यू॒तय॑ ऊ॒तयः॒ सन्ति॑ दा॒शुषे॑ दा॒शुषे॒ सन्त्यू॒तय॑ ऊ॒तयः॒ सन्ति॑ दा॒शुषे᳚ । \newline
19. सन्ति॑ दा॒शुषे॑ दा॒शुषे॒ सन्ति॒ सन्ति॑ दा॒शुषे᳚ । \newline
20. दा॒शुष॒ इति॑ दा॒शुषे᳚ । \newline
21. ताभि॑र् नो न॒ स्ताभि॒ स्ताभि॑र् नो ऽवि॒ता ऽवि॒ता न॒ स्ताभि॒ स्ताभि॑र् नो ऽवि॒ता । \newline
22. नो॒ ऽवि॒ता ऽवि॒ता नो॑ नो ऽवि॒ता भ॑व भवा वि॒ता नो॑ नो ऽवि॒ता भ॑व । \newline
23. अ॒वि॒ता भ॑व भवा वि॒ता ऽवि॒ता भ॑व । \newline
24. भ॒वेति॑ भव । \newline
25. अ॒ग्निर् मू॒र्द्धा मू॒र्द्धा ऽग्नि र॒ग्निर् मू॒र्द्धा भुवो॒ भुवो॑ मू॒र्द्धा ऽग्नि र॒ग्निर् मू॒र्द्धा भुवः॑ । \newline
26. मू॒र्द्धा भुवो॒ भुवो॑ मू॒र्द्धा मू॒र्द्धा भुवः॑ । \newline
27. भुव॒ इति॒ भुवः॑ । \newline
28. त्वम् नो॑ न॒ स्त्वम् त्वम् नः॑ सोम सोम न॒ स्त्वम् त्वम् नः॑ सोम । \newline
29. नः॒ सो॒म॒ सो॒म॒ नो॒ नः॒ सो॒म॒ या या सो॑म नो नः सोम॒ या । \newline
30. सो॒म॒ या या सो॑म सोम॒ या ते॑ ते॒ या सो॑म सोम॒ या ते᳚ । \newline
31. या ते॑ ते॒ या या ते॒ धामा॑नि॒ धामा॑नि ते॒ या या ते॒ धामा॑नि । \newline
32. ते॒ धामा॑नि॒ धामा॑नि ते ते॒ धामा॑नि । \newline
33. धामा॒नीति॒ धामा॑नि । \newline
34. तथ् स॑वि॒तुः स॑वि॒तु स्तत् तथ् स॑वि॒तुर् वरे᳚ण्यं॒ ॅवरे᳚ण्यꣳ सवि॒तु स्तत् तथ् स॑वि॒तुर् वरे᳚ण्यम् । \newline
35. स॒वि॒तुर् वरे᳚ण्यं॒ ॅवरे᳚ण्यꣳ सवि॒तुः स॑वि॒तुर् वरे᳚ण्य॒म् भर्गो॒ भर्गो॒ वरे᳚ण्यꣳ सवि॒तुः स॑वि॒तुर् वरे᳚ण्य॒म् भर्गः॑ । \newline
36. वरे᳚ण्य॒म् भर्गो॒ भर्गो॒ वरे᳚ण्यं॒ ॅवरे᳚ण्य॒म् भर्गो॑ दे॒वस्य॑ दे॒वस्य॒ भर्गो॒ वरे᳚ण्यं॒ ॅवरे᳚ण्य॒म् भर्गो॑ दे॒वस्य॑ । \newline
37. भर्गो॑ दे॒वस्य॑ दे॒वस्य॒ भर्गो॒ भर्गो॑ दे॒वस्य॑ धीमहि धीमहि दे॒वस्य॒ भर्गो॒ भर्गो॑ दे॒वस्य॑ धीमहि । \newline
38. दे॒वस्य॑ धीमहि धीमहि दे॒वस्य॑ दे॒वस्य॑ धीमहि । \newline
39. धी॒म॒हीति॑ धीमहि । \newline
40. धियो॒ यो यो धियो॒ धियो॒ यो नो॑ नो॒ यो धियो॒ धियो॒ यो नः॑ । \newline
41. यो नो॑ नो॒ यो यो नः॑ प्रचो॒दया᳚त् प्रचो॒दया᳚न् नो॒ यो यो नः॑ प्रचो॒दया᳚त् । \newline
42. नः॒ प्र॒चो॒दया᳚त् प्रचो॒दया᳚न् नो नः प्रचो॒दया᳚त् । \newline
43. प्र॒चो॒दया॒दिति॑ प्र - चो॒दया᳚त् । \newline
44. अचि॑त्ती॒ यद् यदचि॒ त्त्यचि॑त्ती॒ यच् च॑कृ॒म च॑कृ॒म यदचि॒ त्त्यचि॑त्ती॒ यच् च॑कृ॒म । \newline
45. यच् च॑कृ॒म च॑कृ॒म यद् यच् च॑कृ॒मा दैव्ये॒ दैव्ये॑ चकृ॒म यद् यच् च॑कृ॒मा दैव्ये᳚ । \newline
46. च॒कृ॒मा दैव्ये॒ दैव्ये॑ चकृ॒म च॑कृ॒मा दैव्ये॒ जने॒ जने॒ दैव्ये॑ चकृ॒म च॑कृ॒मा दैव्ये॒ जने᳚ । \newline
47. दैव्ये॒ जने॒ जने॒ दैव्ये॒ दैव्ये॒ जने॑ दी॒नैर् दी॒नैर् जने॒ दैव्ये॒ दैव्ये॒ जने॑ दी॒नैः । \newline
48. जने॑ दी॒नैर् दी॒नैर् जने॒ जने॑ दी॒नैर् दक्षै॒र् दक्षै᳚र् दी॒नैर् जने॒ जने॑ दी॒नैर् दक्षैः᳚ । \newline
49. दी॒नैर् दक्षै॒र् दक्षै᳚र् दी॒नैर् दी॒नैर् दक्षैः॒ प्रभू॑ती॒ प्रभू॑ती॒ दक्षै᳚र् दी॒नैर् दी॒नैर् दक्षैः॒ प्रभू॑ती । \newline
50. दक्षैः॒ प्रभू॑ती॒ प्रभू॑ती॒ दक्षै॒र् दक्षैः॒ प्रभू॑ती पूरुष॒त्वता॑ पूरुष॒त्वता॒ प्रभू॑ती॒ दक्षै॒र् दक्षैः॒ प्रभू॑ती पूरुष॒त्वता᳚ । \newline
51. प्रभू॑ती पूरुष॒त्वता॑ पूरुष॒त्वता॒ प्रभू॑ती॒ प्रभू॑ती पूरुष॒त्वता᳚ । \newline
52. प्रभू॒तीति॒ प्र - भू॒ती॒ । \newline
53. पू॒रु॒ष॒त्वतेति॑ पूरुष - त्वता᳚ । \newline
\pagebreak
\markright{ TS 4.1.11.2  \hfill https://www.vedavms.in \hfill}

\section{ TS 4.1.11.2 }

\textbf{TS 4.1.11.2 } \newline
\textbf{Samhita Paata} \newline

दे॒वेषु॑ च सवित॒र्मानु॑षेषु च॒ त्वन्नो॒ अत्र॑ सुवता॒दना॑गसः ॥ चो॒द॒यि॒त्री सू॒नृता॑नां॒ चेत॑न्ती सुमती॒नां । य॒ज्ञ्ं द॑धे॒ सर॑स्वती ॥ पावी॑रवी क॒न्या॑ चि॒त्रायुः॒ सर॑स्वती वी॒रप॑त्नी॒ धियं॑ धात् । ग्नाभि॒रच्छि॑द्रꣳ शर॒णꣳ स॒जोषा॑ दुरा॒धर्.षं॑ गृण॒ते शर्म॑ यꣳसत् ॥ पू॒षा गा अन्वे॑तु नः पू॒षा र॑क्ष॒त्वर्व॑तः । पू॒षा वाजꣳ॑ सनोतु नः ॥ शु॒क्रं ते॑ अ॒न्यद्य॑ज॒तं ते॑ अ॒न्य - [  ] \newline

\textbf{Pada Paata} \newline

दे॒वेषु । च॒ । स॒वि॒तः॒ । मानु॑षेषु । च॒ । त्वम् । नः॒ । अत्र॑ । सु॒व॒ता॒त् । अना॑गसः ॥ चो॒द॒यि॒त्री । सू॒नृता॑नाम् । चेत॑न्ती । सु॒म॒ती॒नामिति॑ सु - म॒ती॒नाम् ॥ य॒ज्ञ्म् । द॒धे॒ । सर॑स्वती ॥ पावी॑रवी । क॒न्या᳚ । चि॒त्रायु॒रिति॑ चि॒त्र - आ॒युः॒ । सर॑स्वती । वी॒रप॒त्नीति॑ वी॒र - प॒त्नी॒ । धिय᳚म् । धा॒त् ॥ ग्नाभिः॑ । अच्छि॑द्रम् । श॒र॒णम् । स॒जोषा॒ इति॑ स-जोषाः᳚ । दु॒रा॒धर्.ष॒मिति॑ दुः - आ॒धर्.ष᳚म् । गृ॒ण॒ते । शर्म॑ । यꣳ॒॒स॒त् ॥ पू॒षा । गाः । अन्विति॑ । ए॒तु॒ । नः॒ । पू॒षा । र॒क्ष॒तु॒ । अर्व॑तः ॥ पू॒षा । वाज᳚म् । स॒नो॒तु॒ । नः॒ ॥ शु॒क्रम् । ते॒ । अ॒न्यत् । य॒ज॒तम् । ते॒ । अ॒न्यत् ।  \newline


\textbf{Krama Paata} \newline

दे॒वेषु॑ च । च॒ स॒वि॒तः॒ । स॒वि॒त॒र् मानु॑षेषु । मानु॑षेषु च । च॒ त्वम् । त्वम् नः॑ । नो॒ अत्र॑ । अत्र॑ सुवतात् । सु॒व॒ता॒दना॑गसः । अना॑गस॒ इत्यना॑गसः ॥ चो॒द॒यि॒त्री सू॒नृता॑नाम् । सू॒नृता॑ना॒म् चेत॑न्ती । चेत॑न्ती सुमती॒नाम् । सु॒म॒ती॒नामिति॑ सु - म॒ती॒नाम् ॥ य॒ज्ञ्म् द॑धे । द॒धे॒ सर॑स्वती । सर॑स्व॒तीति॒ सर॑स्वती ॥ पावी॑रवी क॒न्या᳚ । क॒न्या॑ चि॒त्रायुः॑ । चि॒त्रायुः॒ सर॑स्वती । चि॒त्रायु॒रिति॑ चि॒त्र - आ॒युः॒ । सर॑स्वती वी॒रप॑त्नी । वी॒रप॑त्नी॒ धिय᳚म् । वी॒रप॒त्नीति॑ वी॒र - प॒त्नी॒ । धिय॑म् धात् । धा॒दिति॑ धात् ॥ ग्नाभि॒रच्छि॑द्रम् । अच्छि॑द्रꣳ शर॒णम् । श॒र॒णꣳ स॒जोषाः᳚ । स॒जोषा॑ दुरा॒धर्.ष᳚म् । स॒जोषा॒ इति॑ स - जोषाः᳚ । दु॒रा॒धर्.ष॑म् गृण॒ते । दु॒रा॒धर्.ष॒मिति॑ दुः - आ॒धर्.ष᳚म् । गृ॒ण॒ते शर्म॑ । शर्म॑ यꣳसत् । यꣳ॒॒स॒दिति॑ यꣳसत् ॥ पू॒षा गाः । गा अनु॑ । अन्वे॑तु । ए॒तु॒ नः॒ । नः॒ पू॒षा । पू॒षा र॑क्षतु । र॒क्ष॒त्वर्व॑तः । अर्व॑त॒ इत्यर्व॑तः ॥ पू॒षा वाज᳚म् । वाजꣳ॑ सनोतु । स॒नो॒तु॒ नः॒ । न॒ इति॑ नः ॥ शु॒क्रम् ते᳚ । ते॒ अ॒न्यत् । अ॒न्यद् य॑ज॒तम् । य॒ज॒तम् ते᳚ । ते॒ अ॒न्यत् । अ॒न्यद् विषु॑रूपे \newline

\textbf{Jatai Paata} \newline

1. दे॒वेषु॑ च च दे॒वेषु॑ दे॒वेषु॑ च । \newline
2. च॒ स॒वि॒तः॒ स॒वि॒त॒श्च॒ च॒ स॒वि॒तः॒ । \newline
3. स॒वि॒त॒र् मानु॑षेषु॒ मानु॑षेषु सवितः सवित॒र् मानु॑षेषु । \newline
4. मानु॑षेषु च च॒ मानु॑षेषु॒ मानु॑षेषु च । \newline
5. च॒ त्वम् त्वम् च॑ च॒ त्वम् । \newline
6. त्वम् नो॑ न॒ स्त्वम् त्वम् नः॑ । \newline
7. नो॒ अत्रात्र॑ नो नो॒ अत्र॑ । \newline
8. अत्र॑ सुवताथ् सुवता॒ दत्रात्र॑ सुवतात् । \newline
9. सु॒व॒ता॒ दना॑ग॒सो ऽना॑गसः सुवताथ् सुवता॒ दना॑गसः । \newline
10. अना॑गस॒ इत्यना॑गसः । \newline
11. चो॒द॒यि॒त्री सू॒नृता॑नाꣳ सू॒नृता॑नाम् चोदयि॒त्री चो॑दयि॒त्री सू॒नृता॑नाम् । \newline
12. सू॒नृता॑ना॒म् चेत॑न्ती॒ चेत॑न्ती सू॒नृता॑नाꣳ सू॒नृता॑ना॒म् चेत॑न्ती । \newline
13. चेत॑न्ती सुमती॒नाꣳ सु॑मती॒नाम् चेत॑न्ती॒ चेत॑न्ती सुमती॒नाम् । \newline
14. सु॒म॒ती॒नामिति॑ सु - म॒ती॒नाम् । \newline
15. य॒ज्ञ्म् द॑धे दधे य॒ज्ञ्ं ॅय॒ज्ञ्म् द॑धे । \newline
16. द॒धे॒ सर॑स्वती॒ सर॑स्वती दधे दधे॒ सर॑स्वती । \newline
17. सर॑स्व॒तीति॒ सर॑स्वती । \newline
18. पावी॑रवी क॒न्या॑ क॒न्या॑ पावी॑रवी॒ पावी॑रवी क॒न्या᳚ । \newline
19. क॒न्या॑ चि॒त्रायु॑ श्चि॒त्रायुः॑ क॒न्या॑ क॒न्या॑ चि॒त्रायुः॑ । \newline
20. चि॒त्रायुः॒ सर॑स्वती॒ सर॑स्वती चि॒त्रायु॑ श्चि॒त्रायुः॒ सर॑स्वती । \newline
21. चि॒त्रायु॒रिति॑ चि॒त्र - आ॒युः॒ । \newline
22. सर॑स्वती वी॒रप॑त्नी वी॒रप॑त्नी॒ सर॑स्वती॒ सर॑स्वती वी॒रप॑त्नी । \newline
23. वी॒रप॑त्नी॒ धिय॒म् धियं॑ ॅवी॒रप॑त्नी वी॒रप॑त्नी॒ धिय᳚म् । \newline
24. वी॒रप॒त्नीति॑ वी॒र - प॒त्नी॒ । \newline
25. धिय॑म् धाद् धा॒द् धिय॒म् धिय॑म् धात् । \newline
26. धा॒दिति॑ धात् । \newline
27. ग्नाभि॒ रच्छि॑द्र॒ मच्छि॑द्र॒म् ग्नाभि॒र् ग्नाभि॒ रच्छि॑द्रम् । \newline
28. अच्छि॑द्रꣳ शर॒णꣳ श॑र॒ण मच्छि॑द्र॒ मच्छि॑द्रꣳ शर॒णम् । \newline
29. श॒र॒णꣳ स॒जोषाः᳚ स॒जोषाः᳚ शर॒णꣳ श॑र॒णꣳ स॒जोषाः᳚ । \newline
30. स॒जोषा॑ दुरा॒धर्.ष॑म् दुरा॒धर्.षꣳ॑ स॒जोषाः᳚ स॒जोषा॑ दुरा॒धर्.ष᳚म् । \newline
31. स॒जोषा॒ इति॑ स - जोषाः᳚ । \newline
32. दु॒रा॒धर्.ष॑म् गृण॒ते गृ॑ण॒ते दु॑रा॒धर्.ष॑म् दुरा॒धर्.ष॑म् गृण॒ते । \newline
33. दु॒रा॒धर्.ष॒मिति॑ दुः - आ॒धर्.ष᳚म् । \newline
34. गृ॒ण॒ते शर्म॒ शर्म॑ गृण॒ते गृ॑ण॒ते शर्म॑ । \newline
35. शर्म॑ यꣳसद् यꣳस॒ च्छर्म॒ शर्म॑ यꣳसत् । \newline
36. यꣳ॒॒स॒दिति॑ यꣳसत् । \newline
37. पू॒षा गा गाः पू॒षा पू॒षा गाः । \newline
38. गा अन्वनु॒ गा गा अनु॑ । \newline
39. अन्वे᳚ त्वे॒त्वन् वन् वे॑तु । \newline
40. ए॒तु॒ नो॒ न॒ ए॒त्वे॒तु॒ नः॒ । \newline
41. नः॒ पू॒षा पू॒षा नो॑ नः पू॒षा । \newline
42. पू॒षा र॑क्षतु रक्षतु पू॒षा पू॒षा र॑क्षतु । \newline
43. र॒क्ष॒त्वर्व॑तो॒ अर्व॑तो रक्षतु रक्ष॒त्वर्व॑तः । \newline
44. अर्व॑त॒ इत्यर्व॑तः । \newline
45. पू॒षा वाजं॒ ॅवाज॑म् पू॒षा पू॒षा वाज᳚म् । \newline
46. वाजꣳ॑ सनोतु सनोतु॒ वाजं॒ ॅवाजꣳ॑ सनोतु । \newline
47. स॒नो॒तु॒ नो॒ नः॒ स॒नो॒तु॒ स॒नो॒तु॒ नः॒ । \newline
48. न॒ इति॑ नः । \newline
49. शु॒क्रम् ते॑ ते शु॒क्रꣳ शु॒क्रम् ते᳚ । \newline
50. ते॒ अ॒न्य द॒न्यत् ते॑ ते अ॒न्यत् । \newline
51. अ॒न्यद् य॑ज॒तं ॅय॑ज॒त म॒न्य द॒न्यद् य॑ज॒तम् । \newline
52. य॒ज॒तम् ते॑ ते यज॒तं ॅय॑ज॒तम् ते᳚ । \newline
53. ते॒ अ॒न्य द॒न्यत् ते॑ ते अ॒न्यत् । \newline
54. अ॒न्यद् विषु॑रूपे॒ विषु॑रूपे अ॒न्य द॒न्यद् विषु॑रूपे । \newline

\textbf{Ghana Paata } \newline

1. दे॒वेषु॑ च च दे॒वेषु॑ दे॒वेषु॑ च सवितः सवित श्च दे॒वेषु॑ दे॒वेषु॑ च सवितः । \newline
2. च॒ स॒वि॒तः॒ स॒वि॒त॒ श्च॒ च॒ स॒वि॒त॒र् मानु॑षेषु॒ मानु॑षेषु सवित श्च च सवित॒र् मानु॑षेषु । \newline
3. स॒वि॒त॒र् मानु॑षेषु॒ मानु॑षेषु सवितः सवित॒र् मानु॑षेषु च च॒ मानु॑षेषु सवितः सवित॒र् मानु॑षेषु च । \newline
4. मानु॑षेषु च च॒ मानु॑षेषु॒ मानु॑षेषु च॒ त्वम् त्वम् च॒ मानु॑षेषु॒ मानु॑षेषु च॒ त्वम् । \newline
5. च॒ त्वम् त्वम् च॑ च॒ त्वम् नो॑ न॒ स्त्वम् च॑ च॒ त्वम् नः॑ । \newline
6. त्वम् नो॑ न॒ स्त्वम् त्वम् नो॒ अत्रात्र॑ न॒ स्त्वम् त्वम् नो॒ अत्र॑ । \newline
7. नो॒ अत्रात्र॑ नो नो॒ अत्र॑ सुवताथ् सुवता॒ दत्र॑ नो नो॒ अत्र॑ सुवतात् । \newline
8. अत्र॑ सुवताथ् सुवता॒ दत्रात्र॑ सुवता॒ दना॑ग॒सो ऽना॑गसः सुवता॒ दत्रात्र॑ सुवता॒ दना॑गसः । \newline
9. सु॒व॒ता॒ दना॑ग॒सो ऽना॑गसः सुवताथ् सुवता॒ दना॑गसः । \newline
10. अना॑गस॒ इत्यना॑गसः । \newline
11. चो॒द॒यि॒त्री सू॒नृता॑नाꣳ सू॒नृता॑नाम् चोदयि॒त्री चो॑दयि॒त्री सू॒नृता॑ना॒म् चेत॑न्ती॒ चेत॑न्ती सू॒नृता॑नाम् चोदयि॒त्री चो॑दयि॒त्री सू॒नृता॑ना॒म् चेत॑न्ती । \newline
12. सू॒नृता॑ना॒म् चेत॑न्ती॒ चेत॑न्ती सू॒नृता॑नाꣳ सू॒नृता॑ना॒म् चेत॑न्ती सुमती॒नाꣳ सु॑मती॒नाम् चेत॑न्ती सू॒नृता॑नाꣳ सू॒नृता॑ना॒म् चेत॑न्ती सुमती॒नाम् । \newline
13. चेत॑न्ती सुमती॒नाꣳ सु॑मती॒नाम् चेत॑न्ती॒ चेत॑न्ती सुमती॒नाम् । \newline
14. सु॒म॒ती॒नामिति॑ सु - म॒ती॒नाम् । \newline
15. य॒ज्ञ्म् द॑धे दधे य॒ज्ञ्ं ॅय॒ज्ञ्म् द॑धे॒ सर॑स्वती॒ सर॑स्वती दधे य॒ज्ञ्ं ॅय॒ज्ञ्म् द॑धे॒ सर॑स्वती । \newline
16. द॒धे॒ सर॑स्वती॒ सर॑स्वती दधे दधे॒ सर॑स्वती । \newline
17. सर॑स्व॒तीति॒ सर॑स्वती । \newline
18. पावी॑रवी क॒न्या॑ क॒न्या॑ पावी॑रवी॒ पावी॑रवी क॒न्या॑ चि॒त्रायु॑ श्चि॒त्रायुः॑ क॒न्या॑ पावी॑रवी॒ पावी॑रवी क॒न्या॑ चि॒त्रायुः॑ । \newline
19. क॒न्या॑ चि॒त्रायु॑ श्चि॒त्रायुः॑ क॒न्या॑ क॒न्या॑ चि॒त्रायुः॒ सर॑स्वती॒ सर॑स्वती चि॒त्रायुः॑ क॒न्या॑ क॒न्या॑ 
चि॒त्रायुः॒ सर॑स्वती । \newline
20. चि॒त्रायुः॒ सर॑स्वती॒ सर॑स्वती चि॒त्रायु॑ श्चि॒त्रायुः॒ सर॑स्वती वी॒रप॑त्नी वी॒रप॑त्नी॒ सर॑स्वती चि॒त्रायु॑ श्चि॒त्रायुः॒ सर॑स्वती वी॒रप॑त्नी । \newline
21. चि॒त्रायु॒रिति॑ चि॒त्र - आ॒युः॒ । \newline
22. सर॑स्वती वी॒रप॑त्नी वी॒रप॑त्नी॒ सर॑स्वती॒ सर॑स्वती वी॒रप॑त्नी॒ धिय॒म् धियं॑ ॅवी॒रप॑त्नी॒ सर॑स्वती॒ सर॑स्वती वी॒रप॑त्नी॒ धिय᳚म् । \newline
23. वी॒रप॑त्नी॒ धिय॒म् धियं॑ ॅवी॒रप॑त्नी वी॒रप॑त्नी॒ धिय॑म् धाद् धा॒द् धियं॑ ॅवी॒रप॑त्नी वी॒रप॑त्नी॒ धिय॑म् धात् । \newline
24. वी॒रप॒त्नीति॑ वी॒र - प॒त्नी॒ । \newline
25. धिय॑म् धाद् धा॒द् धिय॒म् धिय॑म् धात् । \newline
26. धा॒दिति॑ धात् । \newline
27. ग्नाभि॒ रच्छि॑द्र॒ मच्छि॑द्र॒म् ग्नाभि॒र् ग्नाभि॒ रच्छि॑द्रꣳ शर॒णꣳ श॑र॒ण मच्छि॑द्र॒म् ग्नाभि॒र् ग्नाभि॒ रच्छि॑द्रꣳ शर॒णम् । \newline
28. अच्छि॑द्रꣳ शर॒णꣳ श॑र॒ण मच्छि॑द्र॒ मच्छि॑द्रꣳ शर॒णꣳ स॒जोषाः᳚ स॒जोषाः᳚ शर॒ण मच्छि॑द्र॒ मच्छि॑द्रꣳ शर॒णꣳ स॒जोषाः᳚ । \newline
29. श॒र॒णꣳ स॒जोषाः᳚ स॒जोषाः᳚ शर॒णꣳ श॑र॒णꣳ स॒जोषा॑ दुरा॒धर्.ष॑म् दुरा॒धर्.षꣳ॑ स॒जोषाः᳚ शर॒णꣳ श॑र॒णꣳ स॒जोषा॑ दुरा॒धर्.ष᳚म् । \newline
30. स॒जोषा॑ दुरा॒धर्.ष॑म् दुरा॒धर्.षꣳ॑ स॒जोषाः᳚ स॒जोषा॑ दुरा॒धर्.ष॑म् गृण॒ते गृ॑ण॒ते दु॑रा॒धर्.षꣳ॑ स॒जोषाः᳚ स॒जोषा॑ दुरा॒धर्.ष॑म् गृण॒ते । \newline
31. स॒जोषा॒ इति॑ स - जोषाः᳚ । \newline
32. दु॒रा॒धर्.ष॑म् गृण॒ते गृ॑ण॒ते दु॑रा॒धर्.ष॑म् दुरा॒धर्.ष॑म् गृण॒ते शर्म॒ शर्म॑ गृण॒ते दु॑रा॒धर्.ष॑म् दुरा॒धर्.ष॑म् गृण॒ते शर्म॑ । \newline
33. दु॒रा॒धर्.ष॒मिति॑ दुः - आ॒धर्.ष᳚म् । \newline
34. गृ॒ण॒ते शर्म॒ शर्म॑ गृण॒ते गृ॑ण॒ते शर्म॑ यꣳसद् यꣳस॒ च्छर्म॑ गृण॒ते गृ॑ण॒ते शर्म॑ यꣳसत् । \newline
35. शर्म॑ यꣳसद् यꣳस॒ च्छर्म॒ शर्म॑ यꣳसत् । \newline
36. यꣳ॒॒स॒दिति॑ यꣳसत् । \newline
37. पू॒षा गा गाः पू॒षा पू॒षा गा अन्वनु॒ गाः पू॒षा पू॒षा गा अनु॑ । \newline
38. गा अन्वनु॒ गा गा अन्वे᳚ त्वे॒त्वनु॒ गा गा अन्वे॑तु । \newline
39. अन्वे᳚त्वे॒ त्वन्वन्वे॑तु नो न ए॒त्वन्वन्वे॑तु नः । \newline
40. ए॒तु॒ नो॒ न॒ ए॒त्वे॒तु॒ नः॒ पू॒षा पू॒षा न॑ एत्वेतु नः पू॒षा । \newline
41. नः॒ पू॒षा पू॒षा नो॑ नः पू॒षा र॑क्षतु रक्षतु पू॒षा नो॑ नः पू॒षा र॑क्षतु । \newline
42. पू॒षा र॑क्षतु रक्षतु पू॒षा पू॒षा र॑क्ष॒ त्वर्व॑तो॒ अर्व॑तो रक्षतु पू॒षा पू॒षा र॑क्ष॒ त्वर्व॑तः । \newline
43. र॒क्ष॒ त्वर्व॑तो॒ अर्व॑तो रक्षतु रक्ष॒ त्वर्व॑तः । \newline
44. अर्व॑त॒ इत्यर्व॑तः । \newline
45. पू॒षा वाजं॒ ॅवाज॑म् पू॒षा पू॒षा वाजꣳ॑ सनोतु सनोतु॒ वाज॑म् पू॒षा पू॒षा वाजꣳ॑ सनोतु । \newline
46. वाजꣳ॑ सनोतु सनोतु॒ वाजं॒ ॅवाजꣳ॑ सनोतु नो नः सनोतु॒ वाजं॒ ॅवाजꣳ॑ सनोतु नः । \newline
47. स॒नो॒तु॒ नो॒ नः॒ स॒नो॒तु॒ स॒नो॒तु॒ नः॒ । \newline
48. न॒ इति॑ नः । \newline
49. शु॒क्रम् ते॑ ते शु॒क्रꣳ शु॒क्रम् ते॑ अ॒न्य द॒न्यत् ते॑ शु॒क्रꣳ शु॒क्रम् ते॑ अ॒न्यत् । \newline
50. ते॒ अ॒न्य द॒न्यत् ते॑ ते अ॒न्यद् य॑ज॒तं ॅय॑ज॒त म॒न्यत् ते॑ ते अ॒न्यद् य॑ज॒तम् । \newline
51. अ॒न्यद् य॑ज॒तं ॅय॑ज॒त म॒न्य द॒न्यद् य॑ज॒तम् ते॑ ते यज॒त म॒न्य द॒न्यद् य॑ज॒तम् ते᳚ । \newline
52. य॒ज॒तम् ते॑ ते यज॒तं ॅय॑ज॒तम् ते॑ अ॒न्य द॒न्यत् ते॑ यज॒तं ॅय॑ज॒तम् ते॑ अ॒न्यत् । \newline
53. ते॒ अ॒न्य द॒न्यत् ते॑ ते अ॒न्यद् विषु॑रूपे॒ विषु॑रूपे अ॒न्यत् ते॑ ते अ॒न्यद् विषु॑रूपे । \newline
54. अ॒न्यद् विषु॑रूपे॒ विषु॑रूपे अ॒न्य द॒न्यद् विषु॑रूपे॒ अह॑नी॒ अह॑नी॒ विषु॑रूपे अ॒न्य द॒न्यद् विषु॑रूपे॒ अह॑नी । \newline
\pagebreak
\markright{ TS 4.1.11.3  \hfill https://www.vedavms.in \hfill}

\section{ TS 4.1.11.3 }

\textbf{TS 4.1.11.3 } \newline
\textbf{Samhita Paata} \newline

-द्विषु॑रूपे॒ अह॑नी॒ द्यौरि॑वासि । विश्वा॒ हि मा॒या अव॑सि स्वधावो भ॒द्रा ते॑ पूषन्नि॒ह रा॒तिर॑स्तु ॥ ते॑ऽवर्द्धन्त॒ स्वत॑वसो महित्व॒ना ऽऽनाकं॑ त॒स्थुरु॒रु च॑क्रिरे॒ सदः॑ । विष्णु॒ र्यद्धाऽऽव॒द्-वृष॑णं मद॒च्युतं॒ ॅवयो॒ न सी॑द॒न्नधि॑ ब॒र्॒.हिषि॑ प्रि॒ये ॥ प्रचि॒त्रम॒र्कं गृ॑ण॒ते तु॒राय॒ मारु॑ताय॒ स्वत॑वसे भरद्ध्वं । ये सहाꣳ॑सि॒ सह॑सा॒ सह॑न्ते॒ - [  ] \newline

\textbf{Pada Paata} \newline

विषु॑रूपे॒ इति॒ विषु॑ - रू॒पे॒ । अह॑नी॒ इति॑ । द्यौः । इ॒व॒ । अ॒सि॒ ॥ विश्वाः᳚ । हि । मा॒याः । अव॑सि । स्व॒धा॒व॒ इति॑ स्वधा - वः॒ । भ॒द्रा । ते॒ । पू॒ष॒न्न् । इ॒ह । रा॒तिः । अ॒स्तु॒ ॥ ते । अ॒व॒द्‌र्ध॒न्त॒ । स्वत॑वस॒ इति॒ स्व - त॒व॒सः॒ । म॒हि॒त्व॒नेति॑ महि - त्व॒ना । एति॑ । नाक᳚म् । त॒स्थुः । उ॒रु । च॒क्रि॒रे॒ । सदः॑ ॥ विष्णुः॑ । यत् । ह॒ । आव॑त् । वृष॑णम् । म॒द॒च्युत॒मिति॑ मद - च्युत᳚म् । वयः॑ । न । सी॒द॒न्न् । अधीति॑ । ब॒र्॒.हिषि॑ । प्रि॒ये ॥ प्रेति॑ । चि॒त्रम् । अ॒र्कम् । गृ॒ण॒ते । तु॒राय॑ । मारु॑ताय । स्वत॑वस॒ इति॒ स्व - त॒व॒से॒ । भ॒र॒द्ध्व॒म् ॥ ये । सहाꣳ॑सि । सह॑सा । सह॑न्ते ।  \newline


\textbf{Krama Paata} \newline

विषु॑रूपे॒ अह॑नी । विषु॑रूपे॒ इति॒ विषु॑ - रू॒पे॒ । अह॑नी॒ द्यौः । अह॑नी॒ इत्यह॑नी । द्यौरि॑व । इ॒वा॒सि॒ । अ॒सीत्य॑सि ॥ विश्वा॒ हि । हि मा॒याः । मा॒या अव॑सि । अव॑सि स्वधावः । स्व॒धा॒वो॒ भ॒द्राः । स्व॒धा॒व॒ इति॑ स्वधा - वः॒ । भ॒द्रा ते᳚ । ते॒ पू॒ष॒न्न्॒ । पू॒ष॒न्नि॒ह । इ॒ह रा॒तिः । रा॒तिर॑स्तु । अ॒स्त्वित्य॑स्तु ॥ ते॑ ऽवर्द्धन्त । अ॒व॒र्द्ध॒न्त॒ स्वत॑वसः । स्वत॑वसो महित्व॒ना । स्वत॑वस॒ इति॒ स्व - त॒व॒सः॒ । म॒हि॒त्व॒ना ऽऽ नाक᳚म् । म॒हि॒त्व॒नेति॑ महि - त्व॒ना । आ नाक᳚म् । नाक॑म् त॒स्थुः । त॒स्थुरु॒रु । उ॒रु च॑क्रिरे । च॒क्रि॒रे॒ सदः॑ । सद॒ इति॒ सदः॑ ॥ विष्णु॒र् यत् । यद्ध॑ । हाव॑त् । आव॒द् वृष॑णम् । वृष॑णम् मद॒च्युत᳚म् । म॒द॒च्युतं॒ ॅवयः॑ । म॒द॒च्युत॒मिति॑ मद - च्युत᳚म् । वयो॒ न । न सी॑दन्न् । सी॒द॒न्नधि॑ । अधि॑ ब॒र्॒.हिषि॑ । ब॒र्.॒हिषि॑ प्रि॒ये । प्रि॒य इति॑ प्रि॒ये ॥ प्र चि॒त्रम् । चि॒त्रम॒र्कम् । अ॒र्कम् गृ॑ण॒ते । गृ॒ण॒ते तु॒राय॑ । तु॒राय॒ मारु॑ताय । मारु॑ताय॒ स्वत॑वसे । स्वत॑वसे भरद्ध्वम् । स्वत॑वस॒ इति॒ स्व - त॒व॒से॒ । भ॒र॒द्ध्व॒मिति॑ भरद्ध्वम् ॥ ये सहाꣳ॑सि । सहाꣳ॑सि॒ सह॑सा । सह॑सा॒ सह॑न्ते । सह॑न्ते॒ रेज॑ते \newline

\textbf{Jatai Paata} \newline

1. विषु॑रूपे॒ अह॑नी॒ अह॑नी॒ विषु॑रूपे॒ विषु॑रूपे॒ अह॑नी । \newline
2. विषु॑रूपे॒ इति॒ विषु॑ - रू॒पे॒ । \newline
3. अह॑नी॒ द्यौर् द्यौ रह॑नी॒ अह॑नी॒ द्यौः । \newline
4. अह॑नी॒ इत्यह॑नी । \newline
5. द्यौ रि॑वे व॒ द्यौर् द्यौ रि॑व । \newline
6. इ॒वा॒स्य॒ सी॒वे॒ वा॒सि॒ । \newline
7. अ॒सीत्य॑सि । \newline
8. विश्वा॒ हि हि विश्वा॒ विश्वा॒ हि । \newline
9. हि मा॒या मा॒या हि हि मा॒याः । \newline
10. मा॒या अव॒स्य व॑सि मा॒या मा॒या अव॑सि । \newline
11. अव॑सि स्वधावः स्वधा॒वो ऽव॒ स्यव॑सि स्वधावः । \newline
12. स्व॒धा॒वो॒ भ॒द्रा भ॒द्रा स्व॑धावः स्वधावो भ॒द्रा । \newline
13. स्व॒धा॒व॒ इति॑ स्वधा - वः॒ । \newline
14. भ॒द्रा ते॑ ते भ॒द्रा भ॒द्रा ते᳚ । \newline
15. ते॒ पू॒ष॒न् पू॒ष॒न् ते॒ ते॒ पू॒ष॒न्न् । \newline
16. पू॒ष॒न् नि॒हे ह पू॑षन् पूषन् नि॒ह । \newline
17. इ॒ह रा॒ती रा॒ति रि॒हे ह रा॒तिः । \newline
18. रा॒ति र॑स्त्वस्तु रा॒ती रा॒ति र॑स्तु । \newline
19. अ॒स्त्वित्य॑स्तु । \newline
20. ते॑ ऽवर्द्धन्ता वर्द्धन्त॒ ते ते॑ ऽवर्द्धन्त । \newline
21. अ॒व॒र्द्ध॒न्त॒ स्वत॑वसः॒ स्वत॑वसो ऽवर्द्धन्ता वर्द्धन्त॒ स्वत॑वसः । \newline
22. स्वत॑वसो महित्व॒ना म॑हित्व॒ना स्वत॑वसः॒ स्वत॑वसो महित्व॒ना । \newline
23. स्वत॑वस॒ इति॒ स्व - त॒व॒सः॒ । \newline
24. म॒हि॒त्व॒ना ऽऽ नाक॒म् नाक॒ मा म॑हित्व॒ना म॑हित्व॒ना ऽऽ नाक᳚म् । \newline
25. म॒हि॒त्व॒नेति॑ महि - त्व॒ना । \newline
26. आ नाक॒म् नाक॒ मा नाक᳚म् । \newline
27. नाक॑म् त॒स्थु स्त॒स्थुर् नाक॒म् नाक॑म् त॒स्थुः । \newline
28. त॒स्थु रु॒रू॑रु त॒स्थु स्त॒स्थु रु॒रु । \newline
29. उ॒रु च॑क्रिरे चक्रिर उ॒रू॑रु च॑क्रिरे । \newline
30. च॒क्रि॒रे॒ सदः॒ सद॑ श्चक्रिरे चक्रिरे॒ सदः॑ । \newline
31. सद॒ इति॒ सदः॑ । \newline
32. विष्णु॒र् यद् यद् विष्णु॒र् विष्णु॒र् यत् । \newline
33. यद्ध॑ ह॒ यद् यद्ध॑ । \newline
34. हाव॒ दाव॑ द्ध॒ हाव॑त् । \newline
35. आव॒द् वृष॑णं॒ ॅवृष॑ण॒ माव॒ दाव॒द् वृष॑णम् । \newline
36. वृष॑णम् मद॒च्युत॑म् मद॒च्युतं॒ ॅवृष॑णं॒ ॅवृष॑णम् मद॒च्युत᳚म् । \newline
37. म॒द॒च्युतं॒ ॅवयो॒ वयो॑ मद॒च्युत॑म् मद॒च्युतं॒ ॅवयः॑ । \newline
38. म॒द॒च्युत॒मिति॑ मद - च्युत᳚म् । \newline
39. वयो॒ न न वयो॒ वयो॒ न । \newline
40. न सी॑दन् थ्सीद॒न् न न सी॑दन्न् । \newline
41. सी॒द॒न् नध्यधि॑ षीदन् थ्सीद॒न् नधि॑ । \newline
42. अधि॑ ब॒र्॒.हिषि॑ ब॒र्॒.हि ष्यध्यधि॑ ब॒र्॒.हिषि॑ । \newline
43. ब॒र्॒.हिषि॑ प्रि॒ये प्रि॒ये ब॒र्॒.हिषि॑ ब॒र्॒.हिषि॑ प्रि॒ये । \newline
44. प्रि॒य इति॑ प्रि॒ये । \newline
45. प्र चि॒त्रम् चि॒त्रम् प्र प्र चि॒त्रम् । \newline
46. चि॒त्र म॒र्क म॒र्कम् चि॒त्रम् चि॒त्र म॒र्कम् । \newline
47. अ॒र्कम् गृ॑ण॒ते गृ॑ण॒ते अ॒र्क म॒र्कम् गृ॑ण॒ते । \newline
48. गृ॒ण॒ते तु॒राय॑ तु॒राय॑ गृण॒ते गृ॑ण॒ते तु॒राय॑ । \newline
49. तु॒राय॒ मारु॑ताय॒ मारु॑ताय तु॒राय॑ तु॒राय॒ मारु॑ताय । \newline
50. मारु॑ताय॒ स्वत॑वसे॒ स्वत॑वसे॒ मारु॑ताय॒ मारु॑ताय॒ स्वत॑वसे । \newline
51. स्वत॑वसे भरद्ध्वम् भरद्ध्वꣳ॒॒ स्वत॑वसे॒ स्वत॑वसे भरद्ध्वम् । \newline
52. स्वत॑वस॒ इति॒ स्व - त॒व॒से॒ । \newline
53. भ॒र॒द्ध्व॒मिति॑ भरद्ध्वम् । \newline
54. ये सहाꣳ॑सि॒ सहाꣳ॑सि॒ ये ये सहाꣳ॑सि । \newline
55. सहाꣳ॑सि॒ सह॑सा॒ सह॑सा॒ सहाꣳ॑सि॒ सहाꣳ॑सि॒ सह॑सा । \newline
56. सह॑सा॒ सह॑न्ते॒ सह॑न्ते॒ सह॑सा॒ सह॑सा॒ सह॑न्ते । \newline
57. सह॑न्ते॒ रेज॑ते॒ रेज॑ते॒ सह॑न्ते॒ सह॑न्ते॒ रेज॑ते । \newline

\textbf{Ghana Paata } \newline

1. विषु॑रूपे॒ अह॑नी॒ अह॑नी॒ विषु॑रूपे॒ विषु॑रूपे॒ अह॑नी॒ द्यौर् द्यौ रह॑नी॒ विषु॑रूपे॒ विषु॑रूपे॒ अह॑नी॒ द्यौः । \newline
2. विषु॑रूपे॒ इति॒ विषु॑ - रू॒पे॒ । \newline
3. अह॑नी॒ द्यौर् द्यौ रह॑नी॒ अह॑नी॒ द्यौ रि॑वेव॒ द्यौ रह॑नी॒ अह॑नी॒ द्यौ रि॑व । \newline
4. अह॑नी॒ इत्यह॑नी । \newline
5. द्यौ रि॑वेव॒ द्यौर् द्यौ रि॑वा स्य सीव॒ द्यौर् द्यौ रि॑वासि । \newline
6. इ॒वा॒ स्य॒ सी॒वे॒ वा॒सि॒ । \newline
7. अ॒सीत्य॑सि । \newline
8. विश्वा॒ हि हि विश्वा॒ विश्वा॒ हि मा॒या मा॒या हि विश्वा॒ विश्वा॒ हि मा॒याः । \newline
9. हि मा॒या मा॒या हि हि मा॒या अव॒ स्यव॑सि मा॒या हि हि मा॒या अव॑सि । \newline
10. मा॒या अव॒ स्यव॑सि मा॒या मा॒या अव॑सि स्वधावः स्वधा॒वो ऽव॑सि मा॒या मा॒या अव॑सि स्वधावः । \newline
11. अव॑सि स्वधावः स्वधा॒वो ऽव॒स्यव॑सि स्वधावो भ॒द्रा भ॒द्रा स्व॑धा॒वो ऽव॒स्यव॑सि स्वधावो भ॒द्रा । \newline
12. स्व॒धा॒वो॒ भ॒द्रा भ॒द्रा स्व॑धावः स्वधावो भ॒द्रा ते॑ ते भ॒द्रा स्व॑धावः स्वधावो भ॒द्रा ते᳚ । \newline
13. स्व॒धा॒व॒ इति॑ स्वधा - वः॒ । \newline
14. भ॒द्रा ते॑ ते भ॒द्रा भ॒द्रा ते॑ पूषन् पूषन् ते भ॒द्रा भ॒द्रा ते॑ पूषन्न् । \newline
15. ते॒ पू॒ष॒न् पू॒ष॒न् ते॒ ते॒ पू॒ष॒न् नि॒हेह पू॑षन् ते ते पूषन् नि॒ह । \newline
16. पू॒ष॒न् नि॒हेह पू॑षन् पूषन् नि॒ह रा॒ती रा॒ति रि॒ह पू॑षन् पूषन् नि॒ह रा॒तिः । \newline
17. इ॒ह रा॒ती रा॒ति रि॒हेह रा॒ति र॑स्त्वस्तु रा॒ति रि॒हेह रा॒ति र॑स्तु । \newline
18. रा॒ति र॑स्त्वस्तु रा॒ती रा॒ति र॑स्तु । \newline
19. अ॒स्त्वित्य॑स्तु । \newline
20. ते॑ ऽवर्द्धन्ता वर्द्धन्त॒ ते ते॑ ऽवर्द्धन्त॒ स्वत॑वसः॒ स्वत॑वसो ऽवर्द्धन्त॒ ते ते॑ ऽवर्द्धन्त॒ स्वत॑वसः । \newline
21. अ॒व॒र्द्ध॒न्त॒ स्वत॑वसः॒ स्वत॑वसो ऽवर्द्धन्ता वर्द्धन्त॒ स्वत॑वसो महित्व॒ना म॑हित्व॒ना स्वत॑वसो ऽवर्द्धन्ता वर्द्धन्त॒ स्वत॑वसो महित्व॒ना । \newline
22. स्वत॑वसो महित्व॒ना म॑हित्व॒ना स्वत॑वसः॒ स्वत॑वसो महित्व॒ना ऽऽ नाक॒म् नाक॒ मा म॑हित्व॒ना स्वत॑वसः॒ स्वत॑वसो महित्व॒ना ऽऽ नाक᳚म् । \newline
23. स्वत॑वस॒ इति॒ स्व - त॒व॒सः॒ । \newline
24. म॒हि॒त्व॒ना ऽऽ नाक॒म् नाक॒ मा म॑हित्व॒ना म॑हित्व॒ना ऽऽ नाक॒म् त॒स्थु स्त॒स्थुर् नाक॒ मा म॑हित्व॒ना म॑हित्व॒ना ऽऽ नाक॒म् त॒स्थुः । \newline
25. म॒हि॒त्व॒नेति॑ महि - त्व॒ना । \newline
26. आ नाक॒म् नाक॒ मा नाक॑म् त॒स्थु स्त॒स्थुर् नाक॒ मा नाक॑म् त॒स्थुः । \newline
27. नाक॑म् त॒स्थु स्त॒स्थुर् नाक॒म् नाक॑म् त॒स्थु रु॒रू॑रु त॒स्थुर् नाक॒म् नाक॑म् त॒स्थु रु॒रु । \newline
28. त॒स्थु रु॒रू॑रु त॒स्थु स्त॒स्थु रु॒रु च॑क्रिरे चक्रिर उ॒रु त॒स्थु स्त॒स्थु रु॒रु च॑क्रिरे । \newline
29. उ॒रु च॑क्रिरे चक्रिर उ॒रू॑रु च॑क्रिरे॒ सदः॒ सद॑ श्चक्रिर उ॒रू॑रु च॑क्रिरे॒ सदः॑ । \newline
30. च॒क्रि॒रे॒ सदः॒ सद॑ श्चक्रिरे चक्रिरे॒ सदः॑ । \newline
31. सद॒ इति॒ सदः॑ । \newline
32. विष्णु॒र् यद् यद् विष्णु॒र् विष्णु॒र् यद्ध॑ह॒ यद् विष्णु॒र् विष्णु॒र् यद्ध॑ । \newline
33. यद्ध॑ह॒ यद् यद्धा व॒दा व॑द्ध॒ यद् यद्धाव॑त् । \newline
34. हाव॒दा व॑द्ध॒ हाव॒द् वृष॑णं॒ ॅवृष॑ण॒ माव॑द्ध॒ हाव॒द् वृष॑णम् । \newline
35. आव॒द् वृष॑णं॒ ॅवृष॑ण॒ माव॒ दाव॒द् वृष॑णम् मद॒च्युत॑म् मद॒च्युतं॒ ॅवृष॑ण॒ माव॒ दाव॒द् वृष॑णम् मद॒च्युत᳚म् । \newline
36. वृष॑णम् मद॒च्युत॑म् मद॒च्युतं॒ ॅवृष॑णं॒ ॅवृष॑णम् मद॒च्युतं॒ ॅवयो॒ वयो॑ मद॒च्युतं॒ ॅवृष॑णं॒ ॅवृष॑णम् मद॒च्युतं॒ ॅवयः॑ । \newline
37. म॒द॒च्युतं॒ ॅवयो॒ वयो॑ मद॒च्युत॑म् मद॒च्युतं॒ ॅवयो॒ न न वयो॑ मद॒च्युत॑म् मद॒च्युतं॒ ॅवयो॒ न । \newline
38. म॒द॒च्युत॒मिति॑ मद - च्युत᳚म् । \newline
39. वयो॒ न न वयो॒ वयो॒ न सी॑दन् थ्सीद॒न् न वयो॒ वयो॒ न सी॑दन्न् । \newline
40. न सी॑दन् थ्सीद॒न् न न सी॑द॒न् नध्यधि॑ षीद॒न् न न सी॑द॒न् नधि॑ । \newline
41. सी॒द॒न् नध्यधि॑ षीदन् थ्सीद॒न् नधि॑ ब॒र्॒.हिषि॑ ब॒र्॒.हि ष्यधि॑ षीदन् थ्सीद॒न् नधि॑ ब॒र्॒.हिषि॑ । \newline
42. अधि॑ ब॒र्॒.हिषि॑ ब॒र्॒.हि ष्यध्यधि॑ ब॒र्॒.हिषि॑ प्रि॒ये प्रि॒ये ब॒र्॒.हि ष्यध्यधि॑ ब॒र्॒.हिषि॑ प्रि॒ये । \newline
43. ब॒र्॒.हिषि॑ प्रि॒ये प्रि॒ये ब॒र्॒.हिषि॑ ब॒र्॒.हिषि॑ प्रि॒ये । \newline
44. प्रि॒य इति॑ प्रि॒ये । \newline
45. प्र चि॒त्रम् चि॒त्रम् प्र प्र चि॒त्र म॒र्क म॒र्कम् चि॒त्रम् प्र प्र चि॒त्र म॒र्कम् । \newline
46. चि॒त्र म॒र्क म॒र्कम् चि॒त्रम् चि॒त्र म॒र्कम् गृ॑ण॒ते गृ॑ण॒ते अ॒र्कम् चि॒त्रम् चि॒त्र म॒र्कम् गृ॑ण॒ते । \newline
47. अ॒र्कम् गृ॑ण॒ते गृ॑ण॒ते अ॒र्क म॒र्कम् गृ॑ण॒ते तु॒राय॑ तु॒राय॑ गृण॒ते अ॒र्क म॒र्कम् गृ॑ण॒ते तु॒राय॑ । \newline
48. गृ॒ण॒ते तु॒राय॑ तु॒राय॑ गृण॒ते गृ॑ण॒ते तु॒राय॒ मारु॑ताय॒ मारु॑ताय तु॒राय॑ गृण॒ते गृ॑ण॒ते तु॒राय॒ मारु॑ताय । \newline
49. तु॒राय॒ मारु॑ताय॒ मारु॑ताय तु॒राय॑ तु॒राय॒ मारु॑ताय॒ स्वत॑वसे॒ स्वत॑वसे॒ मारु॑ताय तु॒राय॑ तु॒राय॒ मारु॑ताय॒ स्वत॑वसे । \newline
50. मारु॑ताय॒ स्वत॑वसे॒ स्वत॑वसे॒ मारु॑ताय॒ मारु॑ताय॒ स्वत॑वसे भरद्ध्वम् भरद्ध्वꣳ॒॒ स्वत॑वसे॒ मारु॑ताय॒ मारु॑ताय॒ स्वत॑वसे भरद्ध्वम् । \newline
51. स्वत॑वसे भरद्ध्वम् भरद्ध्वꣳ॒॒ स्वत॑वसे॒ स्वत॑वसे भरद्ध्वम् । \newline
52. स्वत॑वस॒ इति॒ स्व - त॒व॒से॒ । \newline
53. भ॒र॒द्ध्व॒मिति॑ भरद्ध्वम् । \newline
54. ये सहाꣳ॑सि॒ सहाꣳ॑सि॒ ये ये सहाꣳ॑सि॒ सह॑सा॒ सह॑सा॒ सहाꣳ॑सि॒ ये ये सहाꣳ॑सि॒ सह॑सा । \newline
55. सहाꣳ॑सि॒ सह॑सा॒ सह॑सा॒ सहाꣳ॑सि॒ सहाꣳ॑सि॒ सह॑सा॒ सह॑न्ते॒ सह॑न्ते॒ सह॑सा॒ सहाꣳ॑सि॒ सहाꣳ॑सि॒ सह॑सा॒ सह॑न्ते । \newline
56. सह॑सा॒ सह॑न्ते॒ सह॑न्ते॒ सह॑सा॒ सह॑सा॒ सह॑न्ते॒ रेज॑ते॒ रेज॑ते॒ सह॑न्ते॒ सह॑सा॒ सह॑सा॒ सह॑न्ते॒ रेज॑ते । \newline
57. सह॑न्ते॒ रेज॑ते॒ रेज॑ते॒ सह॑न्ते॒ सह॑न्ते॒ रेज॑ते अग्ने अग्ने॒ रेज॑ते॒ सह॑न्ते॒ सह॑न्ते॒ रेज॑ते अग्ने । \newline
\pagebreak
\markright{ TS 4.1.11.4  \hfill https://www.vedavms.in \hfill}

\section{ TS 4.1.11.4 }

\textbf{TS 4.1.11.4 } \newline
\textbf{Samhita Paata} \newline

रेज॑ते अग्ने पृथि॒वी म॒खेभ्यः॑ ॥ विश्वे॑ दे॒वा >5, विश्वे॑ देवाः >6 ॥ द्यावा॑ नः पृथि॒वी इ॒मꣳ सि॒द्ध्रम॒द्य दि॑वि॒स्पृशं᳚ । य॒ज्ञ्ं दे॒वेषु॒ यच्छतां ॥ प्र पू᳚र्व॒जे पि॒तरा॒ नव्य॑सीभिर्गी॒र्भिः कृ॑णुद्ध्वꣳ॒॒ सद॑ने ऋ॒तस्य॑ । आ नो᳚ द्यावापृथिवी॒ दैव्ये॑न॒ जने॑न यातं॒ महि॑ वां॒ ॅवरू॑थं ॥ अ॒ग्निꣳ स्तोमे॑न बोधय समिधा॒नो अम॑र्त्यं । ह॒व्या दे॒वेषु॑ नो दधत् ॥ स ह॑व्य॒वाडम॑र्त्य उ॒शिग्दू॒तश्चनो॑हितः ( ) । अ॒ग्निर्द्धि॒या समृ॑ण्वति ॥ शन्नो॑ भवन्तु॒>7 , वाजे॑वाजे> 8 ॥ \newline

\textbf{Pada Paata} \newline

रेज॑ते । अ॒ग्ने॒ । पृ॒थि॒वी । म॒खेभ्यः॑ ॥ विश्वे᳚ । दे॒वाः । विश्वे᳚ । दे॒वाः॒ ॥ द्यावा᳚ । नः॒ । पृ॒थि॒वी इति॑ । इ॒मम् । सि॒द्ध्रम् । अ॒द्य । दि॒वि॒स्पृश॒मिति॑ दिवि - स्पृश᳚म् ॥ य॒ज्ञ्म् । दे॒वेषु॑ । य॒च्छ॒ता॒म् ॥ प्रेति॑ । पू॒र्व॒जे इति॑ पूर्व-जे । पि॒तरा᳚ । नव्य॑सीभिः । गी॒र्भिः । कृ॒णु॒द्ध्व॒म् । सद॑ने॒ इति॑ । ऋ॒तस्य॑ ॥ एति॑ । नः॒ । द्या॒वा॒पृ॒थि॒वी॒ इति॑ द्यावा-पृ॒थि॒वी । दैव्ये॑न । जने॑न । या॒त॒म् । महि॑ । वा॒म् । वरू॑थम् ॥ अ॒ग्निम् । स्तोमे॑न । बो॒ध॒य॒ । स॒मि॒धा॒न इति॑ सं - इ॒धा॒नः । अम॑र्त्यम् ॥ ह॒व्या । दे॒वेषु॑ । नः॒ । द॒ध॒त् ॥ सः । ह॒व्य॒वाडिति॑ हव्य - वाट् । अम॑र्त्यः । उ॒शिक् । दू॒तः । चनो॑हितः ( ) ॥ अ॒ग्निः । धि॒या । समिति॑ । ऋ॒ण्व॒ति॒ ॥ शम् । नः॒ । भ॒व॒न्तु॒ । वाजे॑वाज॒ इति॒ वाजे᳚ - वा॒जे॒ ॥  \newline


\textbf{Krama Paata} \newline

रेज॑ते अग्ने । अ॒ग्ने॒ पृ॒थि॒वी । पृ॒थि॒वी म॒खेभ्यः॑ । म॒खेभ्य॒ इति॑ म॒खेभ्यः॑ ॥ विश्वे॑ दे॒वाः । दे॒वा विश्वे᳚ । विश्वे॑ देवाः । दे॒वा॒ इति॑ देवाः ॥ द्यावा॑ नः । नः॒ पृ॒थि॒वी । पृ॒थि॒वी इ॒मम् । पृ॒थि॒वी इति॑ पृथि॒वी । इ॒मꣳ सि॒ध्रम् । सि॒ध्रम॒द्य । अ॒द्य दि॑वि॒स्पृश᳚म् । दि॒वि॒स्पृश॒मिति॑ दिवि - स्पृश᳚म् ॥ य॒ज्ञ्म् दे॒वेषु॑ । दे॒वेषु॑ यच्छताम् । य॒च्छ॒ता॒मिति॑ यच्छताम् ॥ प्र पू᳚र्व॒जे । पू॒र्व॒जे पि॒तरा᳚ । पू॒र्व॒जे इति॑ पूर्व - जे । पि॒तरा॒ नव्य॑सीभिः । नव्य॑सीभिर् गी॒र्भिः । गी॒र्भिः कृ॑णुद्ध्वम् । कृ॒णु॒द्ध्वꣳ॒॒ सद॑ने । सद॑ने ऋ॒तस्य॑ । सद॑ने॒ इति॒ सद॑ने । ऋ॒तस्ये,त्यृ॒तस्य॑ ॥ आ नः॑ । नो॒ द्या॒वा॒पृ॒थि॒वी॒ । द्या॒वा॒पृ॒थि॒वी॒ दैव्ये॑न । द्या॒वा॒पृ॒थि॒वी॒ इति॑ द्यावा - पृ॒थि॒वी॒ । दैव्ये॑न॒ जने॑न । जने॑न यातम् । या॒त॒म् महि॑ । महि॑ वाम् । वां॒ ॅवरू॑थम् । वरू॑थ॒मिति॒ वरू॑थम् ॥ अ॒ग्निꣳ स्तोमे॑न । स्तोमे॑न बोधय । बो॒ध॒य॒ स॒मि॒धा॒नः । स॒मि॒धा॒नो अम॑र्त्यम् । स॒मि॒धा॒न इति॑ सम् - इ॒धा॒नः । अम॑र्त्य॒मित्यम॑र्त्यम् ॥ ह॒व्या दे॒वेषु॑ । दे॒वेषु॑ नः । नो॒ द॒ध॒त्॒ । द॒ध॒दिति॑ दधत् ॥ स ह॑व्य॒वाट् । ह॒व्य॒वाडम॑र्त्यः । ह॒व्य॒वाडिति॑ हव्य - वाट् । अम॑र्त्य उ॒शिक् । उ॒शिग् दू॒तः । दू॒तश्चनो॑हितः । चनो॑हित॒ इति॒ चनो॑हितः ( ) ॥ अ॒ग्निर् धि॒या । धि॒या सम् । समृ॑ण्वति । ऋ॒ण्व॒तीत्यृ॑ण्वति ॥ शम् नः॑ । नो॒ भ॒व॒न्तु॒ । भ॒व॒न्तु॒ वाजे॑वाजे । वाजे॑वाज॒ इति॒ वाजे᳚ - वा॒जे॒ । \newline

\textbf{Jatai Paata} \newline

1. रेज॑ते अग्ने अग्ने॒ रेज॑ते॒ रेज॑ते अग्ने । \newline
2. अ॒ग्ने॒ पृ॒थि॒वी पृ॑थि॒ व्य॑ग्ने अग्ने पृथि॒वी । \newline
3. पृ॒थि॒वी म॒खेभ्यो॑ म॒खेभ्यः॑ पृथि॒वी पृ॑थि॒वी म॒खेभ्यः॑ । \newline
4. म॒खेभ्य॒ इति॑ म॒खेभ्यः॑ । \newline
5. विश्वे॑ दे॒वा दे॒वा विश्वे॒ विश्वे॑ दे॒वाः । \newline
6. दे॒वा विश्वे॒ विश्वे॑ दे॒वा दे॒वा विश्वे᳚ । \newline
7. विश्वे॑ देवा देवा॒ विश्वे॒ विश्वे॑ देवाः । \newline
8. दे॒वा॒ इति॑ देवाः । \newline
9. द्यावा॑ नो नो॒ द्यावा॒ द्यावा॑ नः । \newline
10. नः॒ पृ॒थि॒वी पृ॑थि॒वी नो॑ नः पृथि॒वी । \newline
11. पृ॒थि॒वी इ॒म मि॒मम् पृ॑थि॒वी पृ॑थि॒वी इ॒मम् । \newline
12. पृ॒थि॒वी इति॑ पृथि॒वी । \newline
13. इ॒मꣳ सि॒द्ध्रꣳ सि॒द्ध्र मि॒म मि॒मꣳ सि॒द्ध्रम् । \newline
14. सि॒द्ध्र म॒द्याद्य सि॒द्ध्रꣳ सि॒द्ध्र म॒द्य । \newline
15. अ॒द्य दि॑वि॒स्पृश॑म् दिवि॒स्पृश॑ म॒द्याद्य दि॑वि॒स्पृश᳚म् । \newline
16. दि॒वि॒स्पृश॒मिति॑ दिवि - स्पृश᳚म् । \newline
17. य॒ज्ञ्म् दे॒वेषु॑ दे॒वेषु॑ य॒ज्ञ्ं ॅय॒ज्ञ्म् दे॒वेषु॑ । \newline
18. दे॒वेषु॑ यच्छतां ॅयच्छताम् दे॒वेषु॑ दे॒वेषु॑ यच्छताम् । \newline
19. य॒च्छ॒ता॒मिति॑ यच्छताम् । \newline
20. प्र पू᳚र्व॒जे पू᳚र्व॒जे प्र प्र पू᳚र्व॒जे । \newline
21. पू॒र्व॒जे पि॒तरा॑ पि॒तरा॑ पूर्व॒जे पू᳚र्व॒जे पि॒तरा᳚ । \newline
22. पू॒र्व॒जे इति॑ पूर्व - जे । \newline
23. पि॒तरा॒ नव्य॑सीभि॒र् नव्य॑सीभिः पि॒तरा॑ पि॒तरा॒ नव्य॑सीभिः । \newline
24. नव्य॑सीभिर् गी॒र्भिर् गी॒र्भिर् नव्य॑सीभि॒र् नव्य॑सीभिर् गी॒र्भिः । \newline
25. गी॒र्भिः कृ॑णुद्ध्वम् कृणुद्ध्वम् गी॒र्भिर् गी॒र्भिः कृ॑णुद्ध्वम् । \newline
26. कृ॒णु॒द्ध्वꣳ॒॒ सद॑ने॒ सद॑ने कृणुद्ध्वम् कृणुद्ध्वꣳ॒॒ सद॑ने । \newline
27. सद॑ने ऋ॒तस्य॒ र्‌तस्य॒ सद॑ने॒ सद॑ने ऋ॒तस्य॑ । \newline
28. सद॑ने॒ इति॒ सद॑ने । \newline
29. ऋ॒तस्ये त्यृ॒तस्य॑ । \newline
30. आ नो॑ न॒ आ नः॑ । \newline
31. नो॒ द्या॒वा॒पृ॒थि॒वी॒ द्या॒वा॒पृ॒थि॒वी॒ नो॒ नो॒ द्या॒वा॒पृ॒थि॒वी॒ । \newline
32. द्या॒वा॒पृ॒थि॒वी॒ दैव्ये॑न॒ दैव्ये॑न द्यावापृथिवी द्यावापृथिवी॒ दैव्ये॑न । \newline
33. द्या॒वा॒पृ॒थि॒वी॒ इति॑ द्यावा - पृ॒थि॒वी॒ । \newline
34. दैव्ये॑न॒ जने॑न॒ जने॑न॒ दैव्ये॑न॒ दैव्ये॑न॒ जने॑न । \newline
35. जने॑न यातं ॅयात॒म् जने॑न॒ जने॑न यातम् । \newline
36. या॒त॒म् महि॒ महि॑ यातं ॅयात॒म् महि॑ । \newline
37. महि॑ वां ॅवा॒म् महि॒ महि॑ वाम् । \newline
38. वां॒ ॅवरू॑थं॒ ॅवरू॑थं ॅवां ॅवां॒ ॅवरू॑थम् । \newline
39. वरू॑थ॒मिति॒ वरू॑थम् । \newline
40. अ॒ग्निꣳ स्तोमे॑न॒ स्तोमे॑ना॒ग्नि म॒ग्निꣳ स्तोमे॑न । \newline
41. स्तोमे॑न बोधय बोधय॒ स्तोमे॑न॒ स्तोमे॑न बोधय । \newline
42. बो॒ध॒य॒ स॒मि॒धा॒नः स॑मिधा॒नो बो॑धय बोधय समिधा॒नः । \newline
43. स॒मि॒धा॒नो अम॑र्त्य॒ मम॑र्त्यꣳ समिधा॒नः स॑मिधा॒नो अम॑र्त्यम् । \newline
44. स॒मि॒धा॒न इति॑ सं - इ॒धा॒नः । \newline
45. अम॑र्त्य॒ मित्यम॑र्त्यम् । \newline
46. ह॒व्या दे॒वेषु॑ दे॒वेषु॑ ह॒व्या ह॒व्या दे॒वेषु॑ । \newline
47. दे॒वेषु॑ नो नो दे॒वेषु॑ दे॒वेषु॑ नः । \newline
48. नो॒ द॒ध॒द् द॒ध॒न् नो॒ नो॒ द॒ध॒त् । \newline
49. द॒ध॒दिति॑ दधत् । \newline
50. स ह॑व्य॒वा ड्ढ॑व्य॒वाट् थ्स स ह॑व्य॒वाट् । \newline
51. ह॒व्य॒वा डम॑र्त्यो॒ अम॑र्त्यो हव्य॒वा ड्ढ॑व्य॒वा डम॑र्त्यः । \newline
52. ह॒व्य॒वाडिति॑ हव्य - वाट् । \newline
53. अम॑र्त्य उ॒शि गु॒शि गम॑र्त्यो॒ अम॑र्त्य उ॒शिक् । \newline
54. उ॒शिग् दू॒तो दू॒त उ॒शि गु॒शिग् दू॒तः । \newline
55. दू॒त श्चनो॑हित॒ श्चनो॑हितो दू॒तो दू॒त श्चनो॑हितः । \newline
56. चनो॑हित॒ इति॒ चनो॑हितः । \newline
57. अ॒ग्निर् धि॒या धि॒या ऽग्नि र॒ग्निर् धि॒या । \newline
58. धि॒या सꣳ सम् धि॒या धि॒या सम् । \newline
59. स मृ॑ण्व त्यृण्वति॒ सꣳ स मृ॑ण्वति । \newline
60. ऋ॒ण्व॒ती त्यृ॑ण्वति । \newline
61. शम् नो॑ नः॒ शꣳ शम् नः॑ । \newline
62. नो॒ भ॒व॒न्तु॒ भ॒व॒न्तु॒ नो॒ नो॒ भ॒व॒न्तु॒ । \newline
63. भ॒व॒न्तु॒ वाजे॑वाजे॒ वाजे॑वाजे भवन्तु भवन्तु॒ वाजे॑वाजे । \newline
64. वाजे॑वाज॒ इति॒ वाजे᳚ - वा॒जे॒ । \newline

\textbf{Ghana Paata } \newline

1. रेज॑ते अग्ने अग्ने॒ रेज॑ते॒ रेज॑ते अग्ने पृथि॒वी पृ॑थि॒ व्य॑ग्ने॒ रेज॑ते॒ रेज॑ते अग्ने पृथि॒वी । \newline
2. अ॒ग्ने॒ पृ॒थि॒वी पृ॑थि॒ व्य॑ग्ने अग्ने पृथि॒वी म॒खेभ्यो॑ म॒खेभ्यः॑ पृथि॒व्य॑ग्ने अग्ने पृथि॒वी म॒खेभ्यः॑ । \newline
3. पृ॒थि॒वी म॒खेभ्यो॑ म॒खेभ्यः॑ पृथि॒वी पृ॑थि॒वी म॒खेभ्यः॑ । \newline
4. म॒खेभ्य॒ इति॑ म॒खेभ्यः॑ । \newline
5. विश्वे॑ दे॒वा दे॒वा विश्वे॒ विश्वे॑ दे॒वा विश्वे॒ विश्वे॑ दे॒वा विश्वे॒ विश्वे॑ दे॒वा विश्वे᳚ । \newline
6. दे॒वा विश्वे॒ विश्वे॑ दे॒वा दे॒वा विश्वे॑ देवा देवा॒ विश्वे॑ दे॒वा दे॒वा विश्वे॑ देवाः । \newline
7. विश्वे॑ देवा देवा॒ विश्वे॒ विश्वे॑ देवाः । \newline
8. दे॒वा॒ इति॑ देवाः । \newline
9. द्यावा॑ नो नो॒ द्यावा॒ द्यावा॑ नः पृथि॒वी पृ॑थि॒वी नो॒ द्यावा॒ द्यावा॑ नः पृथि॒वी । \newline
10. नः॒ पृ॒थि॒वी पृ॑थि॒वी नो॑ नः पृथि॒वी इ॒म मि॒मम् पृ॑थि॒वी नो॑ नः पृथि॒वी इ॒मम् । \newline
11. पृ॒थि॒वी इ॒म मि॒मम् पृ॑थि॒वी पृ॑थि॒वी इ॒मꣳ सि॒द्ध्रꣳ सि॒द्ध्र मि॒मम् पृ॑थि॒वी पृ॑थि॒वी इ॒मꣳ सि॒द्ध्रम् । \newline
12. पृ॒थि॒वी इति॑ पृथि॒वी । \newline
13. इ॒मꣳ सि॒द्ध्रꣳ सि॒द्ध्र मि॒म मि॒मꣳ सि॒द्ध्र म॒द्याद्य सि॒द्ध्र मि॒म मि॒मꣳ सि॒द्ध्र म॒द्य । \newline
14. सि॒द्ध्र म॒द्याद्य सि॒द्ध्रꣳ सि॒द्ध्र म॒द्य दि॑वि॒स्पृश॑म् दिवि॒स्पृश॑ म॒द्य सि॒द्ध्रꣳ सि॒द्ध्र म॒द्य दि॑वि॒स्पृश᳚म् । \newline
15. अ॒द्य दि॑वि॒स्पृश॑म् दिवि॒स्पृश॑ म॒द्याद्य दि॑वि॒स्पृश᳚म् । \newline
16. दि॒वि॒स्पृश॒मिति॑ दिवि - स्पृश᳚म् । \newline
17. य॒ज्ञ्म् दे॒वेषु॑ दे॒वेषु॑ य॒ज्ञ्ं ॅय॒ज्ञ्म् दे॒वेषु॑ यच्छतां ॅयच्छताम् दे॒वेषु॑ य॒ज्ञ्ं ॅय॒ज्ञ्म् दे॒वेषु॑ यच्छताम् । \newline
18. दे॒वेषु॑ यच्छतां ॅयच्छताम् दे॒वेषु॑ दे॒वेषु॑ यच्छताम् । \newline
19. य॒च्छ॒ता॒मिति॑ यच्छताम् । \newline
20. प्र पू᳚र्व॒जे पू᳚र्व॒जे प्र प्र पू᳚र्व॒जे पि॒तरा॑ पि॒तरा॑ पूर्व॒जे प्र प्र पू᳚र्व॒जे पि॒तरा᳚ । \newline
21. पू॒र्व॒जे पि॒तरा॑ पि॒तरा॑ पूर्व॒जे पू᳚र्व॒जे पि॒तरा॒ नव्य॑सीभि॒र् नव्य॑सीभिः पि॒तरा॑ पूर्व॒जे पू᳚र्व॒जे पि॒तरा॒ नव्य॑सीभिः । \newline
22. पू॒र्व॒जे इति॑ पूर्व - जे । \newline
23. पि॒तरा॒ नव्य॑सीभि॒र् नव्य॑सीभिः पि॒तरा॑ पि॒तरा॒ नव्य॑सीभिर् गी॒र्भिर् गी॒र्भिर् नव्य॑सीभिः पि॒तरा॑ पि॒तरा॒ नव्य॑सीभिर् गी॒र्भिः । \newline
24. नव्य॑सीभिर् गी॒र्भिर् गी॒र्भिर् नव्य॑सीभि॒र् नव्य॑सीभिर् गी॒र्भिः कृ॑णुद्ध्वम् कृणुद्ध्वम् गी॒र्भिर् नव्य॑सीभि॒र् नव्य॑सीभिर् गी॒र्भिः कृ॑णुद्ध्वम् । \newline
25. गी॒र्भिः कृ॑णुद्ध्वम् कृणुद्ध्वम् गी॒र्भिर् गी॒र्भिः कृ॑णुद्ध्वꣳ॒॒ सद॑ने॒ सद॑ने कृणुद्ध्वम् गी॒र्भिर् गी॒र्भिः कृ॑णुद्ध्वꣳ॒॒ सद॑ने । \newline
26. कृ॒णु॒द्ध्वꣳ॒॒ सद॑ने॒ सद॑ने कृणुद्ध्वम् कृणुद्ध्वꣳ॒॒ सद॑ने ऋ॒तस्य॒ र्‌तस्य॒ सद॑ने कृणुद्ध्वम् कृणुद्ध्वꣳ॒॒ सद॑ने ऋ॒तस्य॑ । \newline
27. सद॑ने ऋ॒तस्य॒ र्‌तस्य॒ सद॑ने॒ सद॑ने ऋ॒तस्य॑ । \newline
28. सद॑ने॒ इति॒ सद॑ने । \newline
29. ऋ॒तस्येत्यृ॒तस्य॑ । \newline
30. आ नो॑ न॒ आ नो᳚ द्यावापृथिवी द्यावापृथिवी न॒ आ नो᳚ द्यावापृथिवी । \newline
31. नो॒ द्या॒वा॒पृ॒थि॒वी॒ द्या॒वा॒पृ॒थि॒वी॒ नो॒ नो॒ द्या॒वा॒पृ॒थि॒वी॒ दैव्ये॑न॒ दैव्ये॑न द्यावापृथिवी नो नो द्यावापृथिवी॒ दैव्ये॑न । \newline
32. द्या॒वा॒पृ॒थि॒वी॒ दैव्ये॑न॒ दैव्ये॑न द्यावापृथिवी द्यावापृथिवी॒ दैव्ये॑न॒ जने॑न॒ जने॑न॒ दैव्ये॑न द्यावापृथिवी द्यावापृथिवी॒ दैव्ये॑न॒ जने॑न । \newline
33. द्या॒वा॒पृ॒थि॒वी॒ इति॑ द्यावा - पृ॒थि॒वी॒ । \newline
34. दैव्ये॑न॒ जने॑न॒ जने॑न॒ दैव्ये॑न॒ दैव्ये॑न॒ जने॑न यातं ॅयात॒म् जने॑न॒ दैव्ये॑न॒ दैव्ये॑न॒ जने॑न यातम् । \newline
35. जने॑न यातं ॅयात॒म् जने॑न॒ जने॑न यात॒म् महि॒ महि॑ यात॒म् जने॑न॒ जने॑न यात॒म् महि॑ । \newline
36. या॒त॒म् महि॒ महि॑ यातं ॅयात॒म् महि॑ वां ॅवा॒म् महि॑ यातं ॅयात॒म् महि॑ वाम् । \newline
37. महि॑ वां ॅवा॒म् महि॒ महि॑ वां॒ ॅवरू॑थं॒ ॅवरू॑थं ॅवा॒म् महि॒ महि॑ वां॒ ॅवरू॑थम् । \newline
38. वां॒ ॅवरू॑थं॒ ॅवरू॑थं ॅवां ॅवां॒ ॅवरू॑थम् । \newline
39. वरू॑थ॒मिति॒ वरू॑थम् । \newline
40. अ॒ग्निꣳ स्तोमे॑न॒ स्तोमे॑ना॒ग्नि म॒ग्निꣳ स्तोमे॑न बोधय बोधय॒ स्तोमे॑ना॒ग्नि म॒ग्निꣳ स्तोमे॑न बोधय । \newline
41. स्तोमे॑न बोधय बोधय॒ स्तोमे॑न॒ स्तोमे॑न बोधय समिधा॒नः स॑मिधा॒नो बो॑धय॒ स्तोमे॑न॒ स्तोमे॑न बोधय समिधा॒नः । \newline
42. बो॒ध॒य॒ स॒मि॒धा॒नः स॑मिधा॒नो बो॑धय बोधय समिधा॒नो अम॑र्त्य॒ मम॑र्त्यꣳ समिधा॒नो बो॑धय बोधय समिधा॒नो अम॑र्त्यम् । \newline
43. स॒मि॒धा॒नो अम॑र्त्य॒ मम॑र्त्यꣳ समिधा॒नः स॑मिधा॒नो अम॑र्त्यम् । \newline
44. स॒मि॒धा॒न इति॑ सं - इ॒धा॒नः । \newline
45. अम॑र्त्य॒मित्यम॑र्त्यम् । \newline
46. ह॒व्या दे॒वेषु॑ दे॒वेषु॑ ह॒व्या ह॒व्या दे॒वेषु॑ नो नो दे॒वेषु॑ ह॒व्या ह॒व्या दे॒वेषु॑ नः । \newline
47. दे॒वेषु॑ नो नो दे॒वेषु॑ दे॒वेषु॑ नो दधद् दधन् नो दे॒वेषु॑ दे॒वेषु॑ नो दधत् । \newline
48. नो॒ द॒ध॒द् द॒ध॒न् नो॒ नो॒ द॒ध॒त् । \newline
49. द॒ध॒दिति॑ दधत् । \newline
50. स ह॑व्य॒ वाड्ढ॑व्य॒वाट् थ्स स ह॑व्य॒वा डम॑र्त्यो॒ अम॑र्त्यो हव्य॒वाट् थ्स स ह॑व्य॒वा डम॑र्त्यः । \newline
51. ह॒व्य॒वा डम॑र्त्यो॒ अम॑र्त्यो हव्य॒वा ड्ढ॑व्य॒वा डम॑र्त्य उ॒शि गु॒शि गम॑र्त्यो हव्य॒वा ड्ढ॑व्य॒वा डम॑र्त्य उ॒शिक् । \newline
52. ह॒व्य॒वाडिति॑ हव्य - वाट् । \newline
53. अम॑र्त्य उ॒शि गु॒शि गम॑र्त्यो॒ अम॑र्त्य उ॒शिग् दू॒तो दू॒त उ॒शि गम॑र्त्यो॒ अम॑र्त्य उ॒शिग् दू॒तः । \newline
54. उ॒शिग् दू॒तो दू॒त उ॒शि गु॒शिग् दू॒त श्चनो॑हित॒ श्चनो॑हितो दू॒त उ॒शि गु॒शिग् दू॒त श्चनो॑हितः । \newline
55. दू॒त श्चनो॑हित॒ श्चनो॑हितो दू॒तो दू॒त श्चनो॑हितः । \newline
56. चनो॑हित॒ इति॒ चनो॑हितः । \newline
57. अ॒ग्निर् धि॒या धि॒या ऽग्नि र॒ग्निर् धि॒या सꣳ सम् धि॒या ऽग्नि र॒ग्निर् धि॒या सम् । \newline
58. धि॒या सꣳ सम् धि॒या धि॒या समृ॑ण्व त्यृण्वति॒ सम् धि॒या धि॒या समृ॑ण्वति । \newline
59. समृ॑ण्व त्यृण्वति॒ सꣳ समृ॑ण्वति । \newline
60. ऋ॒ण्व॒तीत्यृ॑ण्वति । \newline
61. शन्नो॑ नः॒ शꣳ शन्नो॑ भवन्तु भवन्तु नः॒ शꣳ शन्नो॑ भवन्तु । \newline
62. नो॒ भ॒व॒न्तु॒ भ॒व॒न्तु॒ नो॒ नो॒ भ॒व॒न्तु॒ वाजे॑वाजे॒ वाजे॑वाजे भवन्तु नो नो भवन्तु॒ वाजे॑वाजे । \newline
63. भ॒व॒न्तु॒ वाजे॑वाजे॒ वाजे॑वाजे भवन्तु भवन्तु॒ वाजे॑वाजे । \newline
64. वाजे॑वाज॒ इति॒ वाजे᳚ - वा॒जे॒ । \newline
\pagebreak


\end{document}