\documentclass[17pt]{extarticle}
\usepackage{babel}
\usepackage{fontspec}
\usepackage{polyglossia}
\usepackage{extsizes}

\usepackage{color}   %May be necessary if you want to color links
\usepackage{hyperref}
\hypersetup{
    colorlinks=true, %set true if you want colored links
    linktoc=all,     %set to all if you want both sections and subsections linked
    linkcolor=black,  %choose some color if you want links to stand out
}

\setmainlanguage{sanskrit}
\setotherlanguages{english} %% or other languages
\setlength{\parindent}{0pt}
\pagestyle{myheadings}
\newfontfamily\devanagarifont[Script=Devanagari]{AdishilaVedic}
\renewcommand{\theHsection}{\thepart.section.\thesection}

\newcommand{\VAR}[1]{}
\newcommand{\BLOCK}[1]{}




\begin{document}
\begin{titlepage}
    \begin{center}
 
\begin{sanskrit}
    { \Large
    कृष्ण यजुर्वेदीय तैत्तिरीय संहिता,पद,जटा,घन पाठः 
    }
    \\
    \vspace{2.5cm}
    \mbox{ \Large
    4.1     चतुर्थकाण्डे प्रथमः प्रश्नः- अग्निचित्यङ्ग मन्त्रपाठाभिधानं   }
\end{sanskrit}
\end{center}

\end{titlepage}
\tableofcontents
\phantomsection
\pagebreak

\markright{ TS 4.1.1.1  \hfill https://www.vedavms.in \hfill}

\section{ TS 4.1.1.1 }

\textbf{TS 4.1.1.1 } \newline
\textbf{Samhita Paata} \newline

यु॒ञ्जा॒नः प्र॑थ॒मं मन॑स्त॒त्वाय॑ सवि॒ता धियः॑ । अ॒ग्निं ज्योति॑र्नि॒चाय्य॑ पृथि॒व्या अद्ध्या ऽभ॑रत् ॥ यु॒क्त्वाय॒ मन॑सा दे॒वान्थ् सुव॑र्य॒तो धि॒या दिवं᳚ । बृ॒हज्ज्योतिः॑ करिष्य॒तः स॑वि॒ता प्रसु॑वाति॒ तान् ॥ यु॒क्तेन॒ मन॑सा व॒यं दे॒वस्य॑ सवि॒तुः स॒वे । सु॒व॒र्गेया॑य॒ शक्त्यै᳚ ॥ यु॒ञ्जते॒ मन॑ उ॒त यु॑ञ्जते॒ धियो॒ विप्रा॒ विप्र॑स्य बृह॒तो वि॑प॒श्चितः॑ । वि होत्रा॑ दधे वयुना॒ विदेक॒ इन् - [  ] \newline

\textbf{Pada Paata} \newline

यु॒ञ्जा॒नः । प्र॒थ॒मम् । मनः॑ । त॒त्वाय॑ । स॒वि॒ता । धियः॑ ॥ अ॒ग्निम् । ज्योतिः॑ । नि॒चाय्येति॑ नि - चाय्य॑ । पृ॒थि॒व्याः । अधि॑ । एति॑ । अ॒भ॒र॒त् ॥ यु॒क्त्वाय॑ । मन॑सा । दे॒वान् । सुवः॑ । य॒तः । धि॒या । दिव᳚म् ॥ बृ॒हत् । ज्योतिः॑ । क॒रि॒ष्य॒तः । स॒वि॒ता । प्रेति॑ । सु॒वा॒ति॒ । तान् ॥ यु॒क्तेन॑ । मन॑सा । व॒यम् । दे॒वस्य॑ । स॒वि॒तुः । स॒वे ॥ सु॒व॒र्गेया॒येति॑ सुवः - गेया॑य । शक्त्यै᳚ ॥ यु॒ञ्जते᳚ । मनः॑ । उ॒त । यु॒ञ्ज॒ते॒ । धियः॑ । विप्राः᳚ । विप्र॑स्य । बृ॒ह॒तः । वि॒प॒श्चितः॑ ॥ वीति॑ । होत्राः᳚ । द॒धे॒ । व॒यु॒ना॒विदिति॑ वयुन - वित् । एकः॑ । इत् ।  \newline




\markright{ TS 4.1.1.2  \hfill https://www.vedavms.in \hfill}

\section{ TS 4.1.1.2 }

\textbf{TS 4.1.1.2 } \newline
\textbf{Samhita Paata} \newline

म॒ही दे॒वस्य॑ सवि॒तुः परि॑ष्टुतिः ॥ यु॒जे वां॒ ब्रह्म॑ पू॒र्व्यं नमो॑भि॒र्वि श्लोका॑ यन्ति प॒थ्ये॑व॒ सूराः᳚ । शृ॒ण्वन्ति॒ विश्वे॑ अ॒मृत॑स्य पु॒त्रा आ ये धामा॑नि दि॒व्यानि॑ त॒स्थुः ॥ यस्य॑ प्र॒याण॒मन्व॒न्य इद्य॒युर्दे॒वा दे॒वस्य॑ महि॒मान॒मर्च॑तः । यः पार्थि॑वानि विम॒मे स एत॑शो॒ रजाꣳ॑सि दे॒वः स॑वि॒ता म॑हित्व॒ना ॥ देव॑ सवितः॒ प्रसु॑व य॒ज्ञ्ं प्रस॑व - [  ] \newline

\textbf{Pada Paata} \newline

म॒ही । दे॒वस्य॑ । स॒वि॒तुः । परि॑ष्टुति॒रिति॒ परि॑-स्तु॒तिः॒ ॥ यु॒जे । वा॒म् । ब्रह्म॑ । पू॒र्व्यम् । नमो॑भि॒रिति॒ नमः॑ - भिः॒ । वीति॑ । श्लोकाः᳚ । य॒न्ति॒ । प॒थ्या᳚ । इ॒व॒ । सूराः᳚ ॥ शृ॒ण्वन्ति॑ । विश्वे᳚ । अ॒मृत॑स्य । पु॒त्राः । एति॑ । ये । धामा॑नि । दि॒व्यानि॑ । त॒स्थुः ॥ यस्य॑ । प्र॒याण॒मिति॑ प्र - यान᳚म् । अन्विति॑ । अ॒न्ये । इत् । य॒युः । दे॒वाः । दे॒वस्य॑ । म॒हि॒मान᳚म् । अर्च॑तः ॥ यः । पार्थि॑वानि । वि॒म॒म इति॑ वि - म॒मे । सः । एत॑शः । रजाꣳ॑सि । दे॒वः । स॒वि॒ता । म॒हि॒त्व॒नेति॑ महि-त्व॒ना ॥ देव॑ । स॒वि॒तः॒ । प्रेति॑ । सु॒व॒ । य॒ज्ञ्म् । प्रेति॑ । सु॒व॒ ।  \newline




\markright{ TS 4.1.1.3  \hfill https://www.vedavms.in \hfill}

\section{ TS 4.1.1.3 }

\textbf{TS 4.1.1.3 } \newline
\textbf{Samhita Paata} \newline

य॒ज्ञ्प॑तिं॒ भगा॑य दि॒व्यो ग॑न्ध॒र्वः । के॒त॒पूः केत॑न्नः पुनातु वा॒चस्पति॒र्वाच॑म॒द्य स्व॑दाति नः ॥ इ॒मं नो॑ देव सवितर्य॒ज्ञ्ं प्रसु॑व देवा॒युवꣳ॑ सखि॒विदꣳ॑ सत्रा॒जितं॑ धन॒जितꣳ॑ सुव॒र्जितं᳚ ॥ ऋ॒चा स्तोमꣳ॒॒ सम॑र्द्धय गाय॒त्रेण॑ रथन्त॒रं । बृ॒हद्-गा॑य॒त्रव॑र्तनि ॥ दे॒वस्य॑ त्वा सवि॒तुः प्र॑स॒वे᳚ ऽश्विनो᳚र्बा॒हुभ्यां᳚ पू॒ष्णो हस्ता᳚भ्यां गाय॒त्रेण॒ छन्द॒सा ऽऽद॑देऽङ्गिर॒स्वदभ्रि॑रसि॒ नारि॑ - [  ] \newline

\textbf{Pada Paata} \newline

य॒ज्ञ्प॑ति॒मिति॑ य॒ज्ञ् - प॒ति॒म् । भगा॑य । दि॒व्यः । ग॒न्ध॒र्वः ॥ के॒त॒पूरिति॑ केत - पूः । केत᳚म् । नः॒ । पु॒ना॒तु॒ । वा॒चः । पतिः॑ । वाच᳚म् । अ॒द्य । स्व॒दा॒ति॒ । नः॒ ॥ इ॒मम् । नः॒ । दे॒व॒ । स॒वि॒तः॒ । य॒ज्ञ्म् । प्रेति॑ । सु॒व॒ । दे॒वा॒युव॒मिति॑ देव - युव᳚म् । स॒खि॒विद॒मिति॑ सखि - विद᳚म् । स॒त्रा॒जित॒मिति॑ सत्र - जित᳚म् । ध॒न॒जित॒मिति॑ धन - जित᳚म् । सु॒व॒र्जित॒मिति॑ सुवः-जित᳚म् ॥ ऋ॒चा । स्तोम᳚म् । समिति॑ । अ॒द्‌र्ध॒य॒ । गा॒य॒त्रेण॑ । र॒थ॒न्त॒रमिति॑ रथं - त॒रम् ॥ बृ॒हत् । गा॒य॒त्रव॑र्त॒नीति॑ गाय॒त्र - व॒र्त॒नि॒ ॥ दे॒वस्य॑ । त्वा॒ । स॒वि॒तुः । प्र॒स॒व इति॑ प्र - स॒वे । अ॒श्विनोः᳚ । बा॒हुभ्या॒मिति॑ बा॒हु - भ्या॒म् । पू॒ष्णः । हस्ता᳚भ्याम् । गा॒य॒त्रेण॑ । छन्द॑सा । एति॑ । द॒दे॒ । अ॒ङ्गि॒र॒स्वत् । अभ्रिः॑ । अ॒सि॒ । नारिः॑ ।  \newline




\markright{ TS 4.1.1.4  \hfill https://www.vedavms.in \hfill}

\section{ TS 4.1.1.4 }

\textbf{TS 4.1.1.4 } \newline
\textbf{Samhita Paata} \newline

-रसि पृथि॒व्याः स॒धस्था॑द॒ग्निं पु॑री॒ष्य॑मङ्गिर॒स्वदा भ॑र॒ त्रैष्टु॑भेन त्वा॒ छन्द॒सा ऽऽद॑देऽङ्गिर॒स्वद्-बभ्रि॑रसि॒ नारि॑रसि॒ त्वया॑ व॒यꣳ स॒धस्थ॒ आग्निꣳ श॑केम॒ खनि॑तुं पुरी॒ष्यं॑ जाग॑तेन त्वा॒ छन्द॒सा ऽऽद॑देऽङ्गिर॒स्वद्धस्त॑ आ॒धाय॑ सवि॒ता बिभ्र॒दभ्रिꣳ॑ हिर॒ण्ययीं᳚ । तया॒ ज्योति॒रज॑स्र॒-मिद॒ग्निं खा॒त्वी न॒ आ भ॒रानु॑ष्टुभेन त्वा॒ छन्द॒सा ( ) ऽऽद॑देऽङ्गिर॒स्वत् ॥ \newline

\textbf{Pada Paata} \newline

अ॒सि॒ । पृ॒थि॒व्याः । स॒धस्था॒दिति॑ स॒ध - स्था॒त् । अ॒ग्निम् । पु॒री॒ष्य᳚म् । अ॒ङ्गि॒र॒स्वत् । एति॑ । भ॒र॒ । त्रैष्टु॑भेन । त्वा॒ । छन्द॑सा । एति॑ । द॒दे॒ । अ॒ङ्गि॒र॒स्वत् । बभ्रिः॑ । अ॒सि॒ । नारिः॑ । अ॒सि॒ । त्वया᳚ । व॒यम् । स॒धस्थ॒ इति॑ स॒ध - स्थे॒ । एति॑ । अ॒ग्निम् । श॒के॒म॒ । खनि॑तुम् । पु॒री॒ष्य᳚म् । जाग॑तेन । त्वा॒ । छन्द॑सा । एति॑ । द॒दे॒ । अ॒ङ्गि॒र॒स्वत् । हस्ते᳚ । आ॒धायेत्या᳚ - धाय॑ । स॒वि॒ता । बिभ्र॑त् । अभ्रि᳚म् । हि॒र॒ण्ययी᳚म् ॥ तया᳚ । ज्योतिः॑ । अज॑स्रम् । इत् । अ॒ग्निम् । खा॒त्वी । नः॒ । एति॑ । भ॒र॒ । आनु॑ष्टुभे॒नेत्यानु॑ - स्तु॒भे॒न॒ । त्वा॒ । छन्द॑सा ( ) । एति॑ । द॒दे॒ । अ॒ङ्गि॒र॒स्वत् ॥  \newline




\markright{ TS 4.1.2.1  \hfill https://www.vedavms.in \hfill}

\section{ TS 4.1.2.1 }

\textbf{TS 4.1.2.1 } \newline
\textbf{Samhita Paata} \newline

इ॒माम॑गृभ्णन् रश॒नामृ॒तस्य॒ पूर्व॒ आयु॑षि वि॒दथे॑षु क॒व्या । तया॑ दे॒वाः सु॒तमा ब॑भूवुर्. ऋ॒तस्य॒ साम᳚न्थ् स॒रमा॒रप॑न्ती ॥ प्रतू᳚र्तं ॅवाजि॒न्ना द्र॑व॒ वरि॑ष्ठा॒मनु॑ सं॒ॅवतं᳚ । दि॒वि ते॒ जन्म॑ पर॒मम॒न्तरि॑क्षे॒ नाभिः॑ पृथि॒व्यामधि॒ योनिः॑ ॥ यु॒ञ्जाथाꣳ॒॒ रास॑भं ॅयु॒वम॒स्मिन्. यामे॑ वृषण्वसू । अ॒ग्निं भर॑न्तमस्म॒युं ॥ योगे॑योगे त॒वस्त॑रं॒ ॅवाजे॑वाजे हवामहे । सखा॑य॒ इन्द्र॑म॒तये᳚ ॥ प्र॒तूर्व॒- [  ] \newline

\textbf{Pada Paata} \newline

इ॒माम् । अ॒गृ॒भ्ण॒न्न् । र॒श॒नाम् । ऋ॒तस्य॑ । पूर्वे᳚ । आयु॑षि । वि॒दथे॑षु । क॒व्या ॥ तया᳚ । दे॒वाः । सु॒तम् । एति॑ । ब॒भू॒वुः॒ । ऋ॒तस्य॑ । सामन्न्॑ । स॒रम् । आ॒रप॒न्तीत्या᳚ - रप॑न्ती ॥ प्रतू᳚र्त॒मिति॑ प्र - तू॒र्त॒म् । वा॒जि॒न्न् । एति॑ । द्र॒व॒ । वरि॑ष्ठाम् । अन्विति॑ । सं॒ॅवत॒मिति॑ सं - वत᳚म् ॥ दि॒वि । ते॒ । जन्म॑ । प॒र॒मम् । अ॒न्तरि॑क्षे । नाभिः॑ । पृ॒थि॒व्याम् । अधीति॑ । योनिः॑ ॥ यु॒ञ्जाथा᳚म् । रास॑भम् । यु॒वम् । अ॒स्मिन्न् । यामे᳚ । वृ॒ष॒ण्व॒सू॒ इति॑ वृषण् - व॒सू॒ ॥ अ॒ग्निम् । भर॑न्तम् । अ॒स्म॒युमित्य॑स्म - युम् ॥ योगे॑योग॒ इति॒ योगे᳚ - यो॒गे॒ । त॒वस्त॑र॒मिति॑ त॒वः - त॒र॒म् । वाजे॑वाज॒ इति॒ वाजे᳚ - वा॒जे॒ । ह॒वा॒म॒हे॒ ॥ सखा॑यः । इन्द्र᳚म् । ऊ॒तये᳚ ॥ प्र॒तूर्व॒न्निति॑ प्र - तूर्वन्न्॑ ।  \newline




\markright{ TS 4.1.2.2  \hfill https://www.vedavms.in \hfill}

\section{ TS 4.1.2.2 }

\textbf{TS 4.1.2.2 } \newline
\textbf{Samhita Paata} \newline

-न्नेह्य॑व॒क्राम॒न्नश॑स्ती रु॒द्रस्य॒ गाण॑पत्यान् मयो॒भूरेहि॑ । उ॒र्व॑न्तरि॑क्ष॒मन्वि॑हि स्व॒स्ति ग॑व्यूति॒रभ॑यानि कृ॒ण्वन्न् ॥ पू॒ष्णा स॒युजा॑ स॒ह । पृ॒थि॒व्याः स॒धस्था॑द॒ग्निं पु॑रि॒ष्य॑-मङ्गिर॒स्व-दच्छे᳚ह्य॒ग्निं पु॑री॒ष्य॑ -मङ्गिर॒स्वद-च्छे॑मो॒ऽग्निं पु॑री॒ष्य॑-मङ्गिर॒स्वद्-भ॑रिष्यामो॒ऽग्निं पु॑री॒ष्य॑-मङ्गिर॒स्वद्-भ॑रामः ॥ अन्व॒ग्निरु॒षसा॒-मग्र॑मख्य॒-दन्वहा॑नि प्रथ॒मो जा॒तवे॑दाः । अनु॒ सूर्य॑स्य - [  ] \newline

\textbf{Pada Paata} \newline

एति॑ । इ॒हि॒ । अ॒व॒क्राम॒न्नित्य॑व - क्रामन्न्॑ । अश॑स्तीः । रु॒द्रस्य॑ । गाण॑पत्या॒दिति॒ गाण॑ - प॒त्या॒त् । म॒यो॒भूरिति॑ मयः - भूः । एति॑ । इ॒हि॒ ॥ उ॒रु । अ॒न्तरि॑क्षम् । अन्विति॑ । इ॒हि॒ । स्व॒स्तिग॑व्यूति॒रिति॑ स्व॒स्ति - ग॒व्यू॒तिः॒ । अभ॑यानि । कृ॒ण्वन्न् ॥ पू॒ष्णा । स॒युजेति॑ स - युजा᳚ । स॒ह ॥ पृ॒थि॒व्याः । स॒धस्था॒दिति॑ स॒ध - स्था॒त् । अ॒ग्निम् । पु॒रि॒ष्य᳚म् । अ॒ङ्गि॒र॒स्वत् । अच्छ॑ । इ॒हि॒ । अ॒ग्निम् । पु॒री॒ष्य᳚म् । अ॒ङ्गि॒र॒स्वत् । अच्छ॑ । इ॒मः॒ । अ॒ग्निम् । पु॒री॒ष्य᳚म् । अ॒ङ्गि॒र॒स्वत् । भ॒रि॒ष्या॒मः॒ । अ॒ग्निम् । पु॒री॒ष्य᳚म् । अ॒ङ्गि॒र॒स्वत् । भ॒रा॒मः॒ ॥ अन्विति॑ । अ॒ग्निः । उ॒षसा᳚म् । अग्र᳚म् । अ॒ख्य॒त् । अन्विति॑ । अहा॑नि । प्र॒थ॒मः । जा॒तवे॑दा॒ इति॑ जा॒त - वे॒दाः॒ ॥ अन्विति॑ । सूर्य॑स्य ।  \newline




\markright{ TS 4.1.2.3  \hfill https://www.vedavms.in \hfill}

\section{ TS 4.1.2.3 }

\textbf{TS 4.1.2.3 } \newline
\textbf{Samhita Paata} \newline

पुरु॒त्रा च॑ र॒श्मीननु॒ द्यावा॑पृथि॒वी आ त॑तान ॥ आ॒गत्य॑ वा॒ज्यद्ध्व॑नः॒ सर्वा॒ मृधो॒ विधू॑नुते । अ॒ग्निꣳ स॒धस्थे॑ मह॒ति चक्षु॑षा॒ नि चि॑कीषते ॥ आ॒क्रम्य॑ वाजिन् पृथि॒वीम॒ग्निमि॑च्छ रु॒चा त्वं । भूम्या॑ वृ॒त्वाय॑ नो ब्रूहि॒ यतः॒ खना॑म॒ तं ॅव॒यं ॥ द्यौस्ते॑ पृ॒ष्ठं पृ॑थि॒वी स॒धस्थ॑मा॒त्मा ऽन्तरि॑क्षꣳ समु॒द्रस्ते॒ योनिः॑ । वि॒ख्याय॒ चक्षु॑षा॒ त्वम॒भि ति॑ष्ठ- [  ] \newline

\textbf{Pada Paata} \newline

पु॒रु॒त्रेति॑ पुरु - त्रा । च॒ । र॒श्मीन् । अन्विति॑ । द्यावा॑पृथि॒वी इति॒ द्यावा᳚ - पृ॒थि॒वी । एति॑ । त॒ता॒न॒ ॥ आ॒गत्येत्या᳚ - गत्य॑ । वा॒जी । अद्ध्व॑नः । सर्वाः᳚ । मृधः॑ । वीति॑ । धू॒नु॒ते॒ ॥ अ॒ग्निम् । स॒धस्थ॒ इति॑ स॒ध - स्थे॒ । म॒ह॒ति । चक्षु॑षा । नीति॑ । चि॒की॒ष॒ते॒ ॥ आ॒क्रम्येत्या᳚ - क्रम्य॑ । वा॒जि॒न्न् । पृ॒थि॒वीम् । अ॒ग्निम् । इ॒च्छ॒ । रु॒चा । त्वम् ॥ भूम्याः᳚ । वृ॒त्वाय॑ । नः॒ । ब्रू॒हि॒ । यतः॑ । खना॑म । तम् । व॒यम् ॥ द्यौः । ते॒ । पृ॒ष्ठम् । पृ॒थि॒वी । स॒धस्थ॒मिति॑ स॒ध - स्थ॒म् । आ॒त्मा । अ॒न्तरि॑क्षम् । स॒मु॒द्रः । ते॒ । योनिः॑ ॥ वि॒ख्यायेति॑ वि - ख्याय॑ । चक्षु॑षा । त्वम् । अ॒भीति॑ । ति॒ष्ठ॒ ।  \newline




\markright{ TS 4.1.2.4  \hfill https://www.vedavms.in \hfill}

\section{ TS 4.1.2.4 }

\textbf{TS 4.1.2.4 } \newline
\textbf{Samhita Paata} \newline

पृतन्य॒तः ॥ उत्क्रा॑म मह॒ते सौभ॑गाया॒-स्मादा॒स्थाना᳚द् द्रविणो॒दा वा॑जिन्न् । व॒यꣳ स्या॑म सुम॒तौ पृ॑थि॒व्या अ॒ग्निं ख॑नि॒ष्यन्त॑ उ॒पस्थे॑ अस्याः ॥ उद॑क्रमीद् द्रविणो॒दा वा॒ज्यर्वाऽकः॒ स लो॒कꣳ सुकृ॑तं पृथि॒व्याः । ततः॑ खनेम सु॒प्रती॑कम॒ग्निꣳ सुवो॒ रुहा॑णा॒ अधि॒ नाक॑ उत्त॒मे ॥ अ॒पो दे॒वीरुप॑ सृज॒ मधु॑मतीरय॒क्ष्माय॑ प्र॒जाभ्यः॑ । तासाꣳ॒॒ स्थाना॒दुज्जि॑हता॒-मोष॑धयः सुपिप्प॒लाः ॥ जिघ॑र्म्य॒- [  ] \newline

\textbf{Pada Paata} \newline

पृ॒त॒न्य॒तः ॥ उदिति॑ । क्रा॒म॒ । म॒ह॒ते । सौभ॑गाय । अ॒स्मात् । आ॒स्थाना॒दित्या᳚ - स्थाना᳚त् । द्र॒वि॒णो॒दा इति॑ द्रविणः - दाः । वा॒जि॒न्न् ॥ व॒यम् । स्या॒म॒ । सु॒म॒ताविति॑ सु - म॒तौ । पृ॒थि॒व्याः । अ॒ग्निम् । ख॒नि॒ष्यन्तः॑ । उ॒पस्थ॒ इत्यु॒प - स्थे॒ । अ॒स्याः॒ ॥ उदिति॑ । अ॒क्र॒मी॒त् । द्र॒वि॒णो॒दा इति॑ द्रविणः - दाः । वा॒जी । अर्वा᳚ । अकः॑ । सः । लो॒कम् । सुकृ॑त॒मिति॒ सु-कृ॒त॒म् । पृ॒थि॒व्याः ॥ ततः॑ । ख॒ने॒म॒ । सु॒प्रती॑क॒मिति॑ सु-प्रती॑कम् । अ॒ग्निम् । सुवः॑ । रुहा॑णाः । अधीति॑ । नाके᳚ । उ॒त्त॒म इत्यु॑त् - त॒मे ॥ अ॒पः । दे॒वीः । उपेति॑ । सृ॒ज॒ । मधु॑मती॒रिति॒ मधु॑ - म॒तीः॒ । अ॒य॒क्ष्माय॑ । प्र॒जाभ्य॒ इति॑ प्र-जाभ्यः॑ ॥ तासा᳚म् । स्थाना᳚त् । उदिति॑ । जि॒ह॒ता॒म् । ओष॑धयः । सु॒पि॒प्प॒ला इति॑ सु - पि॒प्प॒लाः ॥ जिघ॑र्मि ।  \newline




\markright{ TS 4.1.2.5  \hfill https://www.vedavms.in \hfill}

\section{ TS 4.1.2.5 }

\textbf{TS 4.1.2.5 } \newline
\textbf{Samhita Paata} \newline

-ग्निं मन॑सा घृ॒तेन॑ प्रति॒क्ष्यन्तं॒ भुव॑नानि॒ विश्वा᳚ । पृ॒थुं ति॑र॒श्चा वय॑सा बृ॒हन्तं॒ ॅव्यचि॑ष्ठ॒मन्नꣳ॑ रभ॒सं ॅविदा॑नं ॥ आ त्वा॑ जिघर्मि॒ वच॑सा घृ॒तेना॑ऽर॒क्षसा॒ मन॑सा॒ तज्जु॑षस्व । मर्य॑श्रीः स्पृह॒यद्-व॑र्णो अ॒ग्निर्नाऽभि॒मृशे॑ त॒नुवा॒ जर्.हृ॑षाणः ॥ परि॒ वाज॑पतिः क॒विर॒ग्निर्. ह॒व्यान्य॑क्रमीत् । दध॒द् रत्ना॑नि दा॒शुषे᳚ ॥ परि॑ त्वाऽग्ने॒ पुरं॑ ॅव॒यं ॅविप्रꣳ॑ सहस्य धीमहि । धृ॒षद् व॑र्णं दि॒वेदि॑वे भे॒त्तारं॑ ( ) भङ्गु॒राव॑तः ॥ त्वम॑ग्ने॒ द्युभि॒स्त्व-मा॑शुशु॒क्षणि॒स्त्व-म॒द्भ्यस्त्व-मश्म॑न॒स्परि॑ । त्वं ॅवने᳚भ्य॒ स्त्वमोष॑धीभ्य॒ स्त्वं नृ॒णां नृ॑पते जायसे॒ शुचिः॑ ॥ \newline

\textbf{Pada Paata} \newline

अ॒ग्निम् । मन॑सा । घृ॒तेन॑ । प्र॒ति॒क्ष्यन्त॒मिति॑ प्रति - क्ष्यन्त᳚म् । भुव॑नानि । विश्वा᳚ ॥ पृ॒थुम् । ति॒र॒श्चा । वय॑सा । बृ॒हन्त᳚म् । व्यचि॑ष्ठम् । अन्न᳚म् । र॒भ॒सम् । विदा॑नम् ॥ एति॑ । त्वा॒ । जि॒घ॒र्मि॒ । वच॑सा । घृ॒तेन॑ । अ॒र॒क्षसा᳚ । मन॑सा । तत् । जु॒ष॒स्व॒ ॥ मर्य॑श्री॒रिति॒ मर्य॑-श्रीः॒ । स्पृ॒ह॒यद्व॑र्ण॒ इति॑ स्पृह॒यत् - व॒र्णः॒ । अ॒ग्निः । न । अ॒भि॒मृश॒ इत्य॑भि - मृशे᳚ । त॒नुवा᳚ । जर्.हृ॑षाणः ॥ परीति॑ । वाज॑पति॒रिति॒ वाज॑ - प॒तिः॒ । क॒विः । अ॒ग्निः । ह॒व्यानि॑ । अ॒क्र॒मी॒त् ॥ दध॑त् । रत्ना॑नि । दा॒शुषे᳚ ॥ परीति॑ । त्वा॒ । अ॒ग्ने॒ । पुर᳚म् । व॒यम् । विप्र᳚म् । स॒ह॒स्य॒ । धी॒म॒हि॒ ॥ धृ॒षद्व॑र्ण॒मिति॑ धृ॒षत् - व॒र्ण॒म् । दि॒वेदि॑व॒ इति॑ दि॒वे - दि॒वे॒ । भे॒त्तार᳚म् ( ) । भ॒ङ्गु॒राव॑त॒ इति॑ भङ्गु॒र - व॒तः॒ ॥ त्वम् । अ॒ग्ने॒ । द्युभि॒रिति॒ द्यु - भिः॒ । त्वम् । आ॒शु॒शु॒क्षणिः॑ । त्वम् । अ॒द्भ्य इत्य॑त् - भ्यः । त्वम् । अश्म॑नः । परि॑ ॥ त्वम् । वने᳚भ्यः । त्वम् । ओष॑धीभ्य॒ इत्योष॑धि - भ्यः॒ । त्वम् । नृ॒णाम् । नृ॒प॒त॒ इति॑ नृ-प॒ते॒ । जा॒य॒से॒ । शुचिः॑ ॥  \newline




\markright{ TS 4.1.3.1  \hfill https://www.vedavms.in \hfill}

\section{ TS 4.1.3.1 }

\textbf{TS 4.1.3.1 } \newline
\textbf{Samhita Paata} \newline

दे॒वस्य॑ त्वा सवि॒तुः प्र॑स॒वे᳚ऽश्विनो᳚ र्बा॒हुभ्यां᳚ पू॒ष्णो हस्ता᳚भ्यां पृथि॒व्याः स॒धस्थे॒ऽग्निं पु॑री॒ष्य॑-मङ्गिर॒स्वत् ख॑नामि ॥ ज्योति॑ष्मन्तं त्वाऽग्ने सु॒प्रती॑क॒मज॑स्रेण भा॒नुना॒ दीद्या॑नं । शि॒वं प्र॒जाभ्योऽहिꣳ॑ सन्तं पृथि॒व्याः स॒धस्थे॒ऽग्निं पु॑री॒ष्य॑ -मङ्गिर॒स्वत् ख॑नामि ॥ अ॒पां पृ॒ष्ठम॑सि स॒प्रथा॑ उ॒र्व॑ग्निं भ॑रि॒ष्यदप॑रावपिष्ठं । वर्द्ध॑मानं म॒ह आ च॒ पुष्क॑रं दि॒वो मात्र॑या वरि॒णा प्र॑थस्व ॥ शर्म॑ च स्थो॒- [  ] \newline

\textbf{Pada Paata} \newline

दे॒वस्य॑ । त्वा॒ । स॒वि॒तुः । प्र॒स॒व इति॑ प्र - स॒वे । अ॒श्विनोः᳚ । बा॒हुभ्या॒मिति॑ बा॒हु - भ्या॒म् । पू॒ष्णः । हस्ता᳚भ्याम् । पृ॒थि॒व्याः । स॒धस्थ॒ इति॑ स॒ध - स्थे॒ । अ॒ग्निम् । पु॒री॒ष्य᳚म् । अ॒ङ्गि॒र॒स्वत् । ख॒ना॒मि॒ ॥ ज्योति॑ष्मन्तम् । त्वा॒ । अ॒ग्ने॒ । सु॒प्रती॑क॒मिति॑ सु - प्रती॑कम् । अज॑स्रेण । भा॒नुना᳚ । दीद्या॑नम् ॥ शि॒वम् । प्र॒जाभ्य॒ इति॑ प्र - जाभ्यः॑ । अहिꣳ॑सन्तम् । पृ॒थि॒व्याः । स॒धस्थ॒ इति॑ स॒ध - स्थे॒ । अ॒ग्निम् । पु॒री॒ष्य᳚म् । अ॒ङ्गि॒र॒स्वत् । ख॒ना॒मि॒ ॥ अ॒पाम् । पृ॒ष्ठम् । अ॒सि॒ । स॒प्रथा॒ इति॑ स - प्रथाः᳚ । उ॒रु । अ॒ग्निम् । भ॒रि॒ष्यत् । प॑रावपिष्ठ॒मित्यप॑रा - व॒पि॒ष्ठ॒म् ॥ वद्‌र्ध॑मानम् । म॒हः । एति॑ । च॒ । पुष्क॑रम् । दि॒वः । मात्र॑या । व॒रि॒णा । प्र॒थ॒स्व॒ ॥ शर्म॑ । च॒ । स्थः॒ ।  \newline




\markright{ TS 4.1.3.2  \hfill https://www.vedavms.in \hfill}

\section{ TS 4.1.3.2 }

\textbf{TS 4.1.3.2 } \newline
\textbf{Samhita Paata} \newline

वर्म॑ च स्थो॒ अच्छि॑द्रे बहु॒ले उ॒भे । व्यच॑स्वती॒ सं ॅव॑साथां भ॒र्तम॒ग्निं पु॑री॒ष्यं᳚ ॥ संॅव॑साथाꣳ सुव॒र्विदा॑ स॒मीची॒ उर॑सा॒ त्मना᳚ । अ॒ग्निम॒न्त र्भ॑रि॒ष्यन्ती॒ ज्योति॑ष्मन्त॒ मज॑स्र॒मित् ॥ पु॒री॒ष्यो॑ऽसि वि॒श्वभ॑राः । अथ॑र्वा त्वा प्रथ॒मो निर॑मन्थदग्ने ॥ त्वाम॑ग्ने॒ पुष्क॑रा॒दद्ध्यथ॑र्वा॒ निर॑मन्थत । मू॒र्द्ध्नो विश्व॑स्य वा॒घतः॑ ॥ तमु॑ त्वा द॒द्ध्यङ्ङृषिः॑ पु॒त्र ई॑धे॒- [  ] \newline

\textbf{Pada Paata} \newline

वर्म॑ । च॒ । स्थः॒ । अच्छि॑द्रे॒ इति॑ । ब॒हु॒ले इति॑ । उ॒भे इति॑ ॥ व्यच॑स्वती॒ इति॑ । समिति॑ । व॒सा॒था॒म् । भ॒र्तम् । अ॒ग्निम् । पु॒री॒ष्य᳚म् ॥ समिति॑ । व॒सा॒था॒म् । सु॒व॒र्विदेति॑ सुवः - विदा᳚ । स॒मीची॒ इति॑ । उर॑सा । त्मना᳚ ॥ अ॒ग्निम् । अ॒न्तः । भ॒रि॒ष्यन्ती॒ इति॑ । ज्योति॑ष्मन्तम् । अज॑स्रम् । इत् ॥ पु॒री॒ष्यः॑ । अ॒सि॒ । वि॒श्वभ॑रा॒ इति॑ वि॒श्व - भ॒राः॒ ॥ अथ॑र्वा । त्वा॒ । प्र॒थ॒मः । निरिति॑ । अ॒म॒न्थ॒त् । अ॒ग्ने॒ ॥ त्वाम् । अ॒ग्ने॒ । पुष्क॑रात् । अधीति॑ । अथ॑र्वा । निरिति॑ । अ॒म॒न्थ॒त॒ ॥ मू॒द्‌र्ध्नः । विश्व॑स्य । वा॒घतः॑ ॥ तम् । उ॒ । त्वा॒ । द॒द्ध्यङ् । ऋषिः॑ । पु॒त्रः । ई॒धे॒ ।  \newline




\markright{ TS 4.1.3.3  \hfill https://www.vedavms.in \hfill}

\section{ TS 4.1.3.3 }

\textbf{TS 4.1.3.3 } \newline
\textbf{Samhita Paata} \newline

अथ॑र्वणः । वृ॒त्र॒हणं॑ पुरन्द॒रं ॥ तमु॑ त्वा पा॒थ्यो वृषा॒ समी॑धे दस्यु॒हन्त॑मं । ध॒न॒ञ्ज॒यꣳ रणे॑रणे ॥ सीद॑ होतः॒ स्व उ॑ लो॒के चि॑कि॒त्वान्थ् सा॒दया॑ य॒ज्ञ्ꣳ सु॑कृ॒तस्य॒ योनौ᳚ । दे॒वा॒वीर्दे॒वान्. ह॒विषा॑ यजा॒स्यग्ने॑ बृ॒हद्-यज॑माने॒ वयो॑ धाः ॥ नि होता॑ होतृ॒षद॑ने॒ विदा॑नस्त्वे॒षो दी॑दि॒वाꣳ अ॑सदथ् सु॒दक्षः॑ । अद॑ब्धव्रत प्रमति॒र्वसि॑ष्ठः सहस्रं भ॒रः शुचि॑जिह्वो अ॒ग्निः ॥ सꣳ सी॑दस्व म॒हाꣳ अ॑सि॒ शोच॑स्व- [  ] \newline

\textbf{Pada Paata} \newline

अथ॑र्वणः ॥ वृ॒त्र॒हण॒मिति॑ वृत्र - हन᳚म् । पु॒र॒न्द॒रमिति॑ पुरं - द॒रम् ॥ तम् । उ॒ । त्वा॒ । पा॒थ्यः । वृषा᳚ । समिति॑ । ई॒धे॒ । द॒स्यु॒हन्त॑म॒मिति॑ दस्यु-हन्त॑मम् ॥ ध॒न॒ञ्ज॒यमिति॑ धनं - ज॒यम् । रणे॑रण॒ इति॒ रणे᳚-र॒णे॒ ॥ सीद॑ । हो॒तः॒ । स्वे । उ॒ । लो॒के । चि॒कि॒त्वान् । सा॒दय॑ । य॒ज्ञ्म् । सु॒कृ॒तस्येति॑ सु-कृ॒तस्य॑ । योनौ᳚ ॥ दे॒वा॒वीरिति॑ देव-अ॒वीः । दे॒वान् । ह॒विषा᳚ । य॒जा॒सि॒ । अग्ने᳚ । बृ॒हत् । यज॑माने । वयः॑ । धाः॒ ॥ नीति॑ । होता᳚ । हो॒तृ॒षद॑न॒ इति॑ होतृ-सद॑ने । विदा॑नः । त्वे॒षः । दी॒दि॒वान् । अ॒स॒द॒त् । सु॒दक्ष॒ इति॑ सु - दक्षः॑ ॥ अद॑ब्धव्रत प्रमति॒रित्यद॑ब्धव्रत-प्र॒म॒तिः॒ । वसि॑ष्ठः । स॒ह॒स्र॒भं॒र इति॑ सहस्रं-भ॒रः । शुचि॑जिह्व॒ इति॒ शुचि॑ - जि॒ह्वः॒ । अ॒ग्निः ॥ समिति॑ । सी॒द॒स्व॒ । म॒हान् । अ॒सि॒ । शोच॑स्व ।  \newline




\markright{ TS 4.1.3.4  \hfill https://www.vedavms.in \hfill}

\section{ TS 4.1.3.4 }

\textbf{TS 4.1.3.4 } \newline
\textbf{Samhita Paata} \newline

देव॒वीत॑मः । वि धू॒मम॑ग्ने अरु॒षं मि॑येद्ध्य सृ॒ज प्र॑शस्त दर्.श॒तं ॥ जनि॑ष्वा॒ हि जेन्यो॒ अग्रे॒ अह्नाꣳ॑ हि॒तो हि॒तेष्व॑रु॒षो वने॑षु । दमे॑दमे स॒प्त रत्ना॒ दधा॑नो॒ऽग्निर्.होता॒ नि ष॑सादा॒ यजी॑यान् ॥ \newline

\textbf{Pada Paata} \newline

दे॒व॒वीत॑म॒ इति॑ देव - वीत॑मः ॥ वीति॑ । धू॒मम् । अ॒ग्ने॒ । अ॒रु॒षम् । मि॒ये॒द्ध्य॒ । सृ॒ज । प्र॒श॒स्तेति॑ प्र - श॒स्त॒ । द॒र्॒.श॒तम् ॥ जनि॑ष्व । हि । जेन्यः॑ । अग्रे᳚ । अह्ना᳚म् । हि॒तः । हि॒तेषु॑ । अ॒रु॒षः । वने॑षु ॥ दमे॑दम॒ इति॒ दमे᳚ - द॒मे॒ । स॒प्त । रत्ना᳚ । दधा॑नः । अ॒ग्निः । होता᳚ । नीति॑ । स॒सा॒द॒ । यजी॑यान् ॥  \newline




\markright{ TS 4.1.4.1  \hfill https://www.vedavms.in \hfill}

\section{ TS 4.1.4.1 }

\textbf{TS 4.1.4.1 } \newline
\textbf{Samhita Paata} \newline

सं ते॑ वा॒युर्मा॑त॒रिश्वा॑ दधातूत्ता॒नायै॒ हृद॑यं॒ ॅयद्विलि॑ष्टं । दे॒वानां॒ ॅयश्चर॑ति प्रा॒णथे॑न॒ तस्मै॑ च देवि॒ वष॑डस्तु॒ तुभ्यं᳚ ॥ सुजा॑तो॒ ज्योति॑षा स॒ह शर्म॒ वरू॑थ॒माऽस॑दः॒ सुवः॑ । वासो॑ अग्ने वि॒श्वरू॑पꣳ॒॒ संॅव्य॑यस्व विभावसो ॥ उदु॑ तिष्ठ स्वद्ध्व॒रावा॑ नो दे॒व्या कृ॒पा । दृ॒शे च॑ भा॒सा बृ॑ह॒ता सु॑शु॒क्वनि॒राऽग्ने॑ याहि सुश॒स्तिभिः॑ ॥ \newline

\textbf{Pada Paata} \newline

समिति॑ । ते॒ । वा॒युः । मा॒त॒रिश्वा᳚ । द॒धा॒तु॒ । उ॒त्ता॒नाया॒ इत्यु॑त् - ता॒नायै᳚ । हृद॑यम् । यत् । विलि॑ष्ट॒मिति॒ वि - लि॒ष्ट॒म् ॥ दे॒वाना᳚म् । यः । चर॑ति । प्रा॒णथे॒नेति॑ प्र - अ॒नथे॑न । तस्मै᳚ । च॒ । दे॒वि॒ । वष॑ट् । अ॒स्तु॒ । तुभ्य᳚म् ॥ सुजा॑त॒ इति॒ सु-जा॒तः॒ । ज्योति॑षा । स॒ह । शर्म॑ । वरू॑थम् । एति॑ । अ॒स॒दः॒ । सुवः॑ ॥ वासः॑ । अ॒ग्ने॒ । वि॒श्वरू॑प॒मिति॑ वि॒श्व - रू॒प॒म् । समिति॑ । व्य॒य॒स्व॒ । वि॒भा॒व॒सो॒ इति॑ विभा-व॒सो॒ ॥ उदिति॑ । उ॒ । ति॒ष्ठ॒ । स्व॒द्ध्व॒रेति॑ सु - अ॒द्ध्व॒र॒ । अव॑ । नः॒ । दे॒व्या । कृ॒पा ॥ दृ॒शे । च॒ । भा॒सा । बृ॒ह॒ता । सु॒शु॒क्वनि॒रिति॑ सु - शु॒क्वनिः॑ । एति॑ । अ॒ग्ने॒ । या॒हि॒ । सु॒श॒स्तिभि॒रिति॑ सुश॒स्ति - भिः॒ ॥  \newline




\markright{ TS 4.1.4.2  \hfill https://www.vedavms.in \hfill}

\section{ TS 4.1.4.2 }

\textbf{TS 4.1.4.2 } \newline
\textbf{Samhita Paata} \newline

ऊ॒र्द्ध्व ऊ॒ षु ण॑ ऊ॒तये॒ तिष्ठा॑ दे॒वो न स॑वि॒ता । ऊ॒र्द्ध्वो वाज॑स्य॒ सनि॑ता॒ यद॒ञ्जिभि॑-र्वा॒घद्भि॑-र्वि॒ह्वया॑महे ॥ स जा॒तो गर्भो॑ असि॒ रोद॑स्यो॒रग्ने॒ चारु॒र्विभृ॑त॒ ओष॑धीषु । चि॒त्रः शिशुः॒ परि॒ तमाꣳ॑स्य॒क्तः प्र मा॒तृभ्यो॒ अधि॒ कनि॑क्रदद्गाः ॥ स्थि॒रो भ॑व वी॒ड्व॑ङ्ग आ॒शुर्भ॑व वा॒ज्य॑र्वन्न् । पृ॒थुर्भ॑व सु॒षद॒स्त्वम॒ग्नेः पु॑रीष॒वाह॑नः ॥ शि॒वो भ॑व - [  ] \newline

\textbf{Pada Paata} \newline

ऊ॒द्‌र्ध्वः । उ॒ । स्विति॑ । नः॒ । ऊ॒तये᳚ । तिष्ठ॑ । दे॒वः । न । स॒वि॒ता ॥ ऊ॒द्‌र्ध्वः । वाज॑स्य । सनि॑ता । यत् । अ॒ञ्जिभि॒रित्य॒ञ्जि - भिः॒ । वा॒घद्भि॒रिति॑ वा॒घत् - भिः॒ । वि॒ह्वया॑मह॒ इति॑ वि - ह्वया॑महे ॥ सः । जा॒तः । गर्भः॑ । अ॒सि॒ । रोद॑स्योः । अग्ने᳚ । चारुः॑ । विभृ॑त॒ इति॒ वि - भृ॒तः॒ । ओष॑धीषु ॥ चि॒त्रः । शिशुः॑ । परीति॑ । तमाꣳ॑सी । अ॒क्तः । प्रेति॑ । मा॒तृभ्य॒ इति॑ मा॒तृ - भ्यः॒ । अधीति॑ । कनि॑क्रदत् । गाः॒ ॥ स्थि॒रः । भ॒व॒ । वी॒ड्व॑ङ्ग॒ इति॑ वी॒डु - अ॒ङ्गः॒ । आ॒शुः । भ॒व॒ । वा॒जी । अ॒र्व॒न्न् ॥ पृ॒थुः । भ॒व॒ । सु॒षद॒ इति॑ सु - सदः॑ । त्वम् । अ॒ग्नेः । पु॒री॒ष॒वाह॑न॒ इति॑ पुरीष - वाह॑नः ॥ शि॒वः । भ॒व॒ ।  \newline




\markright{ TS 4.1.4.3  \hfill https://www.vedavms.in \hfill}

\section{ TS 4.1.4.3 }

\textbf{TS 4.1.4.3 } \newline
\textbf{Samhita Paata} \newline

प्र॒जाभ्यो॒ मानु॑षीभ्य॒स्त्वम॑ङ्गिरः । मा द्यावा॑पृथि॒वी अ॒भि शू॑शुचो॒ माऽन्तरि॑क्षं॒ मा वन॒स्पतीन्॑ ॥ प्रैतु॑ वा॒जी कनि॑क्रद॒-न्नान॑द॒द्-रास॑भः॒ पत्वा᳚ । भर॑न्न॒ग्निं पु॑री॒ष्यं॑ मा पा॒द्यायु॑षः पु॒रा ॥ रास॑भो वां॒ कनि॑क्रद॒थ् सुयु॑क्तो वृषणा॒ रथे᳚ । स वा॑म॒ग्निं पु॑री॒ष्य॑मा॒शुर्दू॒तो व॑हादि॒तः ॥ वृषा॒ऽग्निं ॅवृष॑णं॒ भर॑न्न॒पां गर्भꣳ॑ समु॒द्रियं᳚ । अग्न॒ आ या॑हि- [  ] \newline

\textbf{Pada Paata} \newline

प्र॒जाभ्य॒ इति॑ प्र - जाभ्यः॑ । मानु॑षीभ्यः । त्वम् । अ॒ङ्गि॒रः॒ ॥ मा । द्यावा॑पृथि॒वी इति॒ द्यावा᳚ - पृ॒थि॒वी । अ॒भीति॑ । शू॒शु॒चः॒ । मा । अ॒न्तरि॑क्षम् । मा । वन॒स्पतीन्॑ ॥ प्रेति॑ । ए॒तु॒ । वा॒जी । कनि॑क्रदत् । नान॑दत् । रास॑भः । पत्वा᳚ ॥ भरन्न्॑ । अ॒ग्निम् । पु॒री॒ष्य᳚म् । मा । पा॒दि॒ । आयु॑षः । पु॒रा ॥ रास॑भः । वा॒म् । कनि॑क्रदत् । सुयु॑क्त॒ इति॒ सु - यु॒क्तः॒ । वृ॒ष॒णा॒ । रथे᳚ ॥ सः । वा॒म् । अ॒ग्निम् । पु॒री॒ष्य᳚म् । आ॒शुः । दू॒तः । व॒हा॒त् । इ॒तः ॥ वृषा᳚ । अ॒ग्निम् । वृष॑णम् । भरन्न्॑ । अ॒पाम् । गर्भ᳚म् । स॒मु॒द्रिय᳚म् ॥ अग्ने᳚ । एति॑ । या॒हि॒ ।  \newline




\markright{ TS 4.1.4.4  \hfill https://www.vedavms.in \hfill}

\section{ TS 4.1.4.4 }

\textbf{TS 4.1.4.4 } \newline
\textbf{Samhita Paata} \newline

वी॒तय॑ ऋ॒तꣳ स॒त्यं ॥ ओष॑धयः॒ प्रति॑ गृह्णीता॒ऽग्निमे॒तꣳ शि॒वमा॒यन्त॑म॒भ्यत्र॑ यु॒ष्मान् । व्यस्य॒न् विश्वा॒ अम॑ती॒ररा॑ती-र्नि॒षीद॑न्नो॒ अप॑ दुर्म॒तिꣳ ह॑नत् ॥ ओष॑धयः॒ प्रति॑ मोदद्ध्वमेनं॒ पुष्पा॑वतीः सुपिप्प॒लाः । अ॒यं ॅवो॒ गर्भ॑ ऋ॒त्वियः॑ प्र॒त्नꣳ स॒धस्थ॒मा ऽस॑दत् ॥ \newline

\textbf{Pada Paata} \newline

वी॒तये᳚ । ऋ॒तम् । स॒त्यम् ॥ ओष॑धयः । प्रतीति॑ । गृ॒ह्णी॒त॒ । अ॒ग्निम् । ए॒तम् । शि॒वम् । आ॒यन्त॒मित्या᳚ - यन्त᳚म् । अ॒भीति॑ । अत्र॑ । यु॒ष्मान् ॥ व्यस्य॒न्निति॑ वि - अस्यन्न्॑ । विश्वाः᳚ । अम॑तीः । अरा॑तीः । नि॒षीद॒न्नि ति॑ नि - सीदन्न्॑ । नः॒ । अपेति॑ । दु॒र्म॒तिमिति॑ दुः- म॒तिम् । ह॒न॒त् ॥ ओष॑धयः । प्रतीति॑ । मो॒द॒द्ध्व॒म् । ए॒न॒म् । पुष्पा॑वती॒रिति॒ पुष्प॑ - व॒तीः॒ । सु॒पि॒प्प॒ला इति॑ सु - पि॒प्प॒लाः ॥ अ॒यम् । वः॒ । गर्भः॑ । ऋ॒त्वियः॑ । प्र॒त्नम् । स॒धस्थ॒मिति॑ स॒ध-स्थ॒म् । एति॑ । अ॒स॒द॒त् ॥  \newline




\markright{ TS 4.1.5.1  \hfill https://www.vedavms.in \hfill}

\section{ TS 4.1.5.1 }

\textbf{TS 4.1.5.1 } \newline
\textbf{Samhita Paata} \newline

वि पाज॑सा पृ॒थुना॒ शोशु॑चानो॒ बाध॑स्व द्वि॒षो र॒क्षसो॒ अमी॑वाः । सु॒शर्म॑णो बृह॒तः शर्म॑णि स्याम॒ग्नेर॒हꣳ सु॒हव॑स्य॒ प्रणी॑तौ ॥ आपो॒ हि ष्ठा म॑यो॒भुव॒स्ता न॑ ऊ॒र्जे द॑धातन । म॒हे रणा॑य॒ चक्ष॑से ॥ यो वः॑ शि॒वत॑मो॒ रस॒स्तस्य॑ भाजयते॒ह नः॑ । उ॒श॒तीरि॑व मा॒तरः॑ ॥ तस्मा॒ अरं॑ गमाम वो॒ यस्य॒ क्षया॑य॒ जिन्व॑थ । आपो॑ ज॒नय॑था च नः ॥ मि॒त्रः - [  ] \newline

\textbf{Pada Paata} \newline

वीति॑ । पाज॑सा । पृ॒थुना᳚ । शोशु॑चानः । बाध॑स्व । द्वि॒षः । र॒क्षसः॑ । अमी॑वाः ॥ सु॒शर्म॑ण॒ इति॑ सु - शर्म॑णः । बृ॒ह॒तः । शर्म॑णि । स्या॒म् । अ॒ग्नेः । अ॒हम् । सु॒हव॒स्येति॑ सु-हव॑स्य । प्रणी॑ता॒विति॒ प्र-नी॒तौ॒ ॥ आपः॑ । हि । स्थ । म॒यो॒भुव॒ इति॑ मयः-भुवः॑ । ताः । नः॒ । ऊ॒र्जे । द॒धा॒त॒न॒ ॥ म॒हे । रणा॑य । चक्ष॑से ॥ यः । वः॒ । शि॒वत॑म॒ इति॑ शि॒व-त॒मः॒ । रसः॑ । तस्य॑ । भा॒ज॒य॒त॒ । इ॒ह । नः॒ ॥ उ॒श॒तीः । इ॒व॒ । मा॒तरः॑ ॥ तस्मै᳚ । अर᳚म् । ग॒मा॒म॒ । वः॒ । यस्य॑ । क्षया॑य । जिन्व॑थ ॥ आपः॑ । ज॒नय॑थ । च॒ । नः॒ ॥ मि॒त्रः ।  \newline




\markright{ TS 4.1.5.2  \hfill https://www.vedavms.in \hfill}

\section{ TS 4.1.5.2 }

\textbf{TS 4.1.5.2 } \newline
\textbf{Samhita Paata} \newline

सꣳ॒॒सृज्य॑ पृथि॒वीं भूमिं॑ च॒ ज्योति॑षा स॒ह । सुजा॑तं जा॒तवे॑दसम॒ग्निं ॅवै᳚श्वान॒रं ॅवि॒भुं ॥ अ॒य॒क्ष्माय॑ त्वा॒ सꣳ सृ॑जामि प्र॒जाभ्यः॑ । विश्वे᳚ त्वा दे॒वा वै᳚श्वान॒राः सꣳ सृ॑ज॒न्त्वा-नु॑ष्टुभेन॒ छन्द॑साऽङ्गिर॒स्वत् ॥ रु॒द्राः स॒भृंत्य॑ पृथि॒वीं बृ॒हज्ज्योतिः॒ समी॑धिरे । तेषां᳚ भा॒नुरज॑स्र॒ इच्छु॒क्रो दे॒वेषु॑ रोचते ॥ सꣳ सृ॑ष्टां॒ ॅवसु॑भी रु॒द्रैर्द्धीरैः᳚ कर्म॒ण्यां᳚ मृदं᳚ । हस्ता᳚भ्यां मृ॒द्वीं कृ॒त्वा सि॑नीवा॒ली क॑रोतु॒ - [  ] \newline

\textbf{Pada Paata} \newline

सꣳ॒॒सृज्येति॑ सं - सृज्य॑ । पृ॒थि॒वीम् । भूमि᳚म् । च॒ । ज्योति॑षा । स॒ह ॥ सुजा॑त॒मिति॒ सु - जा॒त॒म् । जा॒तवे॑दस॒मिति॑ जा॒त-वे॒द॒स॒म् । अ॒ग्निम् । वै॒श्वा॒न॒रम् । वि॒भुमिति॑ वि - भुम् ॥ अ॒य॒क्ष्माय॑ । त्वा॒ । समिति॑ । सृ॒जा॒मि॒ । प्र॒जाभ्य॒ इति॑ प्र-जाभ्यः॑ ॥ विश्वे᳚ । त्वा॒ । दे॒वाः । वै॒श्वा॒न॒राः । समिति॑ । सृ॒ज॒न्तु॒ । आनु॑ष्टुभे॒नेत्यानु॑-स्तु॒भे॒न॒ । छन्द॑सा । अ॒ङ्गि॒र॒स्वत् ॥ रु॒द्राः । स॒भृंत्येति॑ सं - भृत्य॑ । पृ॒थि॒वीम् । बृ॒हत् । ज्योतिः॑ । समिति॑ । ई॒धि॒रे॒ ॥ तेषा᳚म् । भा॒नुः । अज॑स्रः । इत् । शु॒क्रः । दे॒वेषु॑ । रो॒च॒ते॒ ॥ सꣳसृ॑ष्टा॒मिति॒ सं - सृ॒ष्टा॒म् । वसु॑भि॒रिति॒ वसु॑ - भिः॒ । रु॒द्रैः । धीरैः᳚ । क॒र्म॒ण्या᳚म् । मृद᳚म् ॥ हस्ता᳚भ्याम् । मृ॒द्वीम् । कृ॒त्वा । सि॒नी॒वा॒ली । क॒रो॒तु॒ ।  \newline




\markright{ TS 4.1.5.3  \hfill https://www.vedavms.in \hfill}

\section{ TS 4.1.5.3 }

\textbf{TS 4.1.5.3 } \newline
\textbf{Samhita Paata} \newline

तां ॥ सि॒नी॒वा॒ली सु॑कप॒र्दा सु॑कुरी॒रा स्वौ॑प॒शा । सा तुभ्य॑मदिते मह॒ ओखां द॑धातु॒ हस्त॑योः ॥ उ॒खां क॑रोतु॒ शक्त्या॑ बा॒हुभ्या॒-मदि॑तिर्द्धि॒या । मा॒ता पु॒त्रं ॅयथो॒पस्थे॒ साऽग्निं बि॑भर्तु॒ गर्भ॒ आ ॥ म॒खस्य॒ शिरो॑ऽसि य॒ज्ञ्स्य॑ प॒दे स्थः॑ । वस॑वस्त्वा कृण्वन्तु गाय॒त्रेण॒ छन्द॑सा ऽङ्गिर॒स्वत् पृ॑थि॒व्य॑सि रु॒द्रास्त्वा॑ कृण्वन्तु॒ त्रैष्टु॑भेन॒ छन्द॑सा ऽङ्गिर॒स्वद॒न्तरि॑क्षमस्या - [  ] \newline

\textbf{Pada Paata} \newline

ताम् ॥ सि॒नी॒वा॒ली । सु॒क॒प॒र्देति॑ सु - क॒प॒र्दा । सु॒कु॒री॒रेति॑ सु - कु॒री॒रा । स्वौ॒प॒शेति॑ सु - औ॒प॒शा ॥ सा । तुभ्य᳚म् । अ॒दि॒ते॒ । म॒हे॒ । एति॑ । उ॒खाम् । द॒धा॒तु॒ । हस्त॑योः ॥ उ॒खाम् । क॒रो॒तु॒ । शक्त्या᳚ । बा॒हुभ्या॒मिति॑ बा॒हु - भ्या॒म् । अदि॑तिः । धि॒या ॥ मा॒ता । पु॒त्रम् । यथा᳚ । उ॒पस्थ॒ इत्यु॒प - स्थे॒ । सा । अ॒ग्निम् । बि॒भ॒र्तु॒ । गर्भे᳚ । आ ॥ म॒खस्य॑ । शिरः॑ । अ॒सि॒ । य॒ज्ञ्स्य॑ । प॒दे इति॑ । स्थः॒ ॥ वस॑वः । त्वा॒ । कृ॒ण्व॒न्तु॒ । गा॒य॒त्रेण॑ । छन्द॑सा । अ॒ङ्गि॒र॒स्वत् । पृ॒थि॒वी । अ॒सि॒ । रु॒द्राः । त्वा॒ । कृ॒ण्व॒न्तु॒ । त्रैष्टु॑भेन । छन्द॑सा । अ॒ङ्गि॒र॒स्वत् । अ॒न्तरि॑क्षम् । अ॒सि॒ ।  \newline




\markright{ TS 4.1.5.4  \hfill https://www.vedavms.in \hfill}

\section{ TS 4.1.5.4 }

\textbf{TS 4.1.5.4 } \newline
\textbf{Samhita Paata} \newline

-दि॒त्यास्त्वा॑ कृण्वन्तु॒ जाग॑तेन॒ छन्द॑साऽङ्गिर॒स्वद् द्यौर॑सि॒ विश्वे᳚ त्वा दे॒वा वै᳚श्वान॒राः कृ॑ण्व॒न्त्वानु॑ष्टुभेन॒ छन्द॑सा-ऽङ्गिर॒स्वद्-दिशो॑ऽसि ध्रु॒वाऽसि॑ धा॒रया॒ मयि॑ प्र॒जाꣳ रा॒यस्पोषं॑ गौप॒त्यꣳ सु॒वीर्यꣳ॑ सजा॒तान्. यज॑माना॒याऽदि॑त्यै॒ रास्ना॒ऽस्य दि॑तिस्ते॒ बिलं॑ गृह्णातु॒ पाङ्क्ते॑न॒ छन्द॑सा ऽङ्गिर॒स्वत् ॥ कृ॒त्वाय॒ सा म॒हीमु॒खां मृ॒न्मयीं॒ ॅयोनि॑म॒ग्नये᳚ । तां पु॒त्रेभ्यः॒ सं प्रा ( ) य॑च्छ॒ददि॑तिः श्र॒पया॒निति॑ ॥ \newline

\textbf{Pada Paata} \newline

आ॒दि॒त्याः । त्वा॒ । कृ॒ण्व॒न्तु॒ । जाग॑तेन । छन्द॑सा । अ॒ङ्गि॒र॒स्वत् । द्यौः । अ॒सि॒ । विश्वे᳚ । त्वा॒ । दे॒वाः । वै॒श्वा॒न॒राः । कृ॒ण्व॒न्तु॒ । आनु॑ष्टुभे॒नेत्यानु॑ - स्तु॒भे॒न॒ । छन्द॑सा । अ॒ङ्गि॒र॒स्वत् । दिशः॑ । अ॒सि॒ । ध्रु॒वा । अ॒सि॒ । धा॒रय॑ । मयि॑ । प्र॒जामिति॑ प्र-जाम् । रा॒यः । पोष᳚म् । गौ॒प॒त्यम् । सु॒वीर्य॒मिति॑ सु - वीर्य᳚म् । स॒जा॒तानिति॑ स - जा॒तान् । यज॑मानाय । अदि॑त्यै । रास्ना᳚ । अ॒सि॒ । अदि॑तिः । ते॒ । बिल᳚म् । गृ॒ह्णा॒तु॒ । पाङ्क्ते॑न । छन्द॑सा । अ॒ङ्गि॒र॒स्वत् ॥ कृ॒त्वाय॑ । सा । म॒हीम् । उ॒खाम् । मृ॒न्मयी॒मिति॑ मृत्-मयी᳚म् । योनि᳚म् । अ॒ग्नये᳚ ॥ ताम् । पु॒त्रेभ्यः॑ । सम् । प्रेति॑ ( ) । अ॒य॒च्छ॒त् । अदि॑तिः । श्र॒पयान्॑ । इति॑ ॥  \newline




\markright{ TS 4.1.6.1  \hfill https://www.vedavms.in \hfill}

\section{ TS 4.1.6.1 }

\textbf{TS 4.1.6.1 } \newline
\textbf{Samhita Paata} \newline

वस॑वस्त्वा धूपयन्तु गाय॒त्रेण॒ छन्द॑साऽङ्गिर॒स्वद्-रु॒द्रास्त्वा॑ धूपयन्तु॒ त्रैष्टु॑भेन॒ छन्द॑साऽङ्गिर॒स्व-दा॑दि॒त्यास्त्वा॑ धूपयन्तु॒ जाग॑तेन॒ छन्द॑साऽङ्गिर॒स्वद्- विश्वे᳚ त्वा दे॒वा वै᳚श्वान॒रा धू॑पय॒न्त्वानु॑ष्टुभेन॒ छन्द॑साऽङ्गिर॒स्व-दिन्द्र॑स्त्वा धूपयत्वङ्गिर॒स्वद् -विष्णु॑स्त्वा धूपयत्वङ्गिर॒स्वद्-वरु॑णस्त्वा धूपयत्वङ्गिर॒स्व-ददि॑तिस्त्वा दे॒वी वि॒श्वदे᳚व्यावती पृथि॒व्याः स॒धस्थे᳚ऽङ्गिर॒स्वत् ख॑नत्ववट दे॒वानां᳚ त्वा॒ पत्नी᳚ - [  ] \newline

\textbf{Pada Paata} \newline

वस॑वः । त्वा॒ । धू॒प॒य॒न्तु॒ । गा॒य॒त्रेण॑ । छन्द॑सा । अ॒ङ्गि॒र॒स्वत् । रु॒द्राः । त्वा॒ । धू॒प॒य॒न्तु॒ । त्रैष्टु॑भेन । छन्द॑सा । अ॒ङ्गि॒र॒स्वत् । आ॒दि॒त्याः । त्वा॒ । धू॒प॒य॒न्तु॒ । जाग॑तेन । छन्द॑सा । अ॒ङ्गि॒र॒स्वत् । विश्वे᳚ । त्वा॒ । दे॒वाः । वै॒श्वा॒न॒राः । धू॒प॒य॒न्तु॒ । आनु॑ष्टुभे॒नेत्यानु॑ - स्तु॒भे॒न॒ । छन्द॑सा । अ॒ङ्गि॒र॒स्वत् । इन्द्रः॑ । त्वा॒ । धू॒प॒य॒तु॒ । अ॒ङ्गि॒र॒स्वत् । विष्णुः॑ । त्वा॒ । धू॒प॒य॒तु॒ । अ॒ङ्गि॒र॒स्वत् । वरु॑णः । त्वा॒ । धू॒प॒य॒तु॒ । अ॒ङ्गि॒र॒स्वत् । अदि॑तिः । त्वा॒ । दे॒वी । वि॒श्वदे᳚व्याव॒तीति॑ वि॒श्वदे᳚व्य - व॒ती॒ । पृ॒थि॒व्याः । स॒धस्थ॒ इति॑ स॒ध - स्थे॒ । अ॒ङ्गि॒र॒स्वत् । ख॒न॒तु॒ । अ॒व॒ट॒ । दे॒वाना᳚म् । त्वा॒ । पत्नीः᳚ ।  \newline




\markright{ TS 4.1.6.2  \hfill https://www.vedavms.in \hfill}

\section{ TS 4.1.6.2 }

\textbf{TS 4.1.6.2 } \newline
\textbf{Samhita Paata} \newline

र्दे॒वी र्वि॒श्वदे᳚व्यावतीः पृथि॒व्याः स॒धस्थे᳚ऽङ्गिर॒स्वद्-द॑धतूखे धि॒षणा᳚स्त्वा दे॒वीर्वि॒श्वदे᳚व्यावतीः पृथि॒व्याः स॒धस्थे᳚-ऽङ्गिर॒स्व-द॒भीन्ध॑तामुखे॒ ग्नास्त्वा॑ दे॒वीर्वि॒श्वदे᳚व्यावतीः पृथि॒व्याः स॒धस्थे᳚ऽङ्गिर॒स्व-च्छ्र॑पयन्तूखे॒ वरू᳚त्रयो॒ जन॑यस्त्वा दे॒वीर्वि॒श्वदे᳚व्यावतीः पृथि॒व्याः स॒धस्थे᳚ऽङ्गिर॒स्वत् प॑चन्तूखे । मित्रै॒तामु॒खां प॑चै॒षा मा भे॑दि । ए॒तां ते॒ परि॑ ददा॒म्यभि॑त्त्यै ॥ अ॒भीमां - [  ] \newline

\textbf{Pada Paata} \newline

दे॒वीः । वि॒श्वदे᳚व्यावती॒रिति॑ वि॒श्वदे᳚व्य - व॒तीः॒ । पृ॒थि॒व्याः । स॒धस्थ॒ इति॑ स॒ध - स्थे॒ । अ॒ङ्गि॒र॒स्वत् । द॒ध॒तु॒ । उ॒खे॒ । धि॒षणाः᳚ । त्वा॒ । दे॒वीः । वि॒श्वदे᳚व्यावती॒रिति॑ वि॒श्वदे᳚व्य - व॒तीः॒ । पृ॒थि॒व्याः । स॒धस्थ॒ इति॑ स॒ध - स्थे॒ । अ॒ङ्गि॒र॒स्वत् । अ॒भीति॑ । इ॒न्ध॒ता॒म् । उ॒खे॒ । ग्नाः । त्वा॒ । दे॒वीः । वि॒श्वदे᳚व्यावती॒रिति॑ वि॒श्वदे᳚व्य - व॒तीः॒ । पृ॒थि॒व्याः । स॒धस्थ॒ इति॑ स॒ध - स्थे॒ । अ॒ङ्गि॒र॒स्वत् । श्र॒प॒य॒न्तु॒ । उ॒खे॒ । वरू᳚त्रयः । जन॑यः । त्वा॒ । दे॒वीः । वि॒श्वदे᳚व्यावती॒रिति॑ वि॒श्वदे᳚व्य - व॒तीः॒ । पृ॒थि॒व्याः । स॒धस्थ॒ इति॑ स॒ध - स्थे॒ । अ॒ङ्गि॒र॒स्वत् । प॒च॒न्तु॒ । उ॒खे॒ ॥ मित्र॑ । ए॒ताम् । उ॒खाम् । प॒च॒ । ए॒षा । मा । भे॒दि॒ ॥ ए॒ताम् । ते॒ । परीति॑ । द॒दा॒मि॒ । अभि॑त्त्यै ॥ अ॒भीति॑ । इ॒माम् ।  \newline




\markright{ TS 4.1.6.3  \hfill https://www.vedavms.in \hfill}

\section{ TS 4.1.6.3 }

\textbf{TS 4.1.6.3 } \newline
\textbf{Samhita Paata} \newline

म॑हि॒ना दिवं॑ मि॒त्रो ब॑भूव स॒प्रथाः᳚ । उ॒त श्रव॑सा पृथि॒वीं ॥ मि॒त्रस्य॑ चर्.षणी॒धृतः॒ श्रवो॑ दे॒वस्य॑ सान॒सिं । द्यु॒म्नं चि॒त्रश्र॑वस्तमं ॥ दे॒वस्त्वा॑ सवि॒तोद्व॑पतु सुपा॒णिः स्व॑ङ्गु॒रिः । सु॒बा॒हुरु॒त शक्त्या᳚ ॥ अप॑द्यमाना पृथि॒व्याशा॒ दिश॒ आ पृ॑ण । उत्ति॑ष्ठ बृह॒ती भ॑वो॒र्द्ध्वा ति॑ष्ठ ध्रु॒वा त्वं ॥ वस॑व॒स्त्वा ऽऽच्छृ॑न्दन्तु गाय॒त्रेण॒ छन्द॑साऽङ्गिर॒स्वद् रु॒द्रास्त्वाऽऽ च्छृ॑न्दन्तु॒ ( ) त्रैष्टु॑भेन॒ छन्द॑साऽङ्गिर॒स्व-दा॑दि॒त्यास्त्वा ऽऽच्छृ॑न्दन्तु॒ जाग॑तेन॒ छन्द॑साऽङ्गिर॒स्वद्-विश्वे᳚ त्वा दे॒वा वै᳚श्वान॒रा आ च्छृ॑न्द॒न्त्वानु॑ष्टुभेन॒ छन्द॑साऽङ्गिर॒स्वत् ॥ \newline

\textbf{Pada Paata} \newline

म॒हि॒ना । दिव᳚म् । मि॒त्रः । ब॒भू॒व॒ । स॒प्रथा॒ इति॑ स - प्रथाः᳚ ॥ उ॒त । श्रव॑सा । पृ॒थि॒वीम् ॥ मि॒त्रस्य॑ । च॒र्॒.ष॒णी॒धृत॒ इति॑ चर्.षणि - धृतः॑ । श्रवः॑ । दे॒वस्य॑ । सा॒न॒सिम् ॥ द्यु॒म्नम् । चि॒त्रश्र॑वस्तम॒मिति॑ चि॒त्रश्र॑वः - त॒म॒म् ॥ दे॒वः । त्वा॒ । स॒वि॒ता । उदिति॑ । व॒प॒तु॒ । सु॒पा॒णिरिति॑ सु - पा॒णिः । स्व॒ङ्गु॒रिरिति॑ सु - अ॒ङ्गु॒रिः ॥ सु॒बा॒हुरिति॑ सु - बा॒हुः । उ॒त । शक्त्या᳚ ॥ अप॑द्यमाना । पृ॒थि॒वि॒ । आशाः᳚ । दिशः॑ । एति॑ । पृ॒ण॒ ॥ उदिति॑ । ति॒ष्ठ॒ । बृ॒ह॒ती । भ॒व॒ । ऊ॒द्‌र्ध्वा । ति॒ष्ठ॒ । ध्रु॒वा । त्वम् ॥ वस॑वः । त्वा॒ । एति॑ । छृ॒न्द॒न्तु॒ । गा॒य॒त्रेण॑ । छन्द॑सा । अ॒ङ्गि॒र॒स्वत् । रु॒द्राः । त्वा॒ । एति॑ । छृ॒न्द॒न्तु॒ ( ) । त्रैष्टु॑भेन । छन्द॑सा । अ॒ङ्गि॒र॒स्वत् । आ॒दि॒त्याः । त्वा॒ । एति॑ । छृ॒न्द॒न्तु॒ । जाग॑तेन । छन्द॑सा । अ॒ङ्गि॒र॒स्वत् । विश्वे᳚ । त्वा॒ । दे॒वाः । वै॒श्वा॒न॒राः । एति॑ । छृ॒न्द॒न्तु॒ । आनु॑ष्टुभे॒नेत्यानु॑ - स्तु॒भे॒न॒ । छन्द॑सा । अ॒ङ्गि॒र॒स्वत् ॥  \newline




\markright{ TS 4.1.7.1  \hfill https://www.vedavms.in \hfill}

\section{ TS 4.1.7.1 }

\textbf{TS 4.1.7.1 } \newline
\textbf{Samhita Paata} \newline

समा᳚स्त्वाऽग्न ऋ॒तवो॑ वर्द्धयन्तु संॅवथ्स॒रा ऋष॑यो॒ यानि॑ स॒त्या । सं दि॒व्येन॑ दीदिहि रोच॒नेन॒ विश्वा॒ आ भा॑हि प्र॒दिशः॑ पृथि॒व्याः ॥ सं चे॒द्ध्यस्वा᳚ऽग्ने॒ प्र च॑ बोधयैन॒मुच्च॑ तिष्ठ मह॒ते सौभ॑गाय । मा च॑ रिषदुपस॒त्ता ते॑ अग्ने ब्र॒ह्माण॑स्ते य॒शसः॑ सन्तु॒ माऽन्ये ॥ त्वाम॑ग्ने वृणते ब्राह्म॒णा इ॒मे शि॒वो अ॑ग्ने - [  ] \newline

\textbf{Pada Paata} \newline

समाः᳚ । त्वा॒ । अ॒ग्ने॒ । ऋ॒तवः॑ । व॒द्‌र्ध॒य॒न्तु॒ । स॒ॅवं॒थ्स॒रा इति॑ सं-व॒थ्स॒राः । ऋष॑यः । यानि॑ । स॒त्या ॥ समिति॑ । दि॒व्येन॑ । दी॒दि॒हि॒ । रो॒च॒नेन॑ । विश्वाः᳚ । एति॑ । भा॒हि॒ । प्र॒दिश॒ इति॑ प्र - दिशः॑ । पृ॒थि॒व्याः ॥ समिति॑ । च॒ । इ॒द्ध्यस्व॑ । अ॒ग्ने॒ । प्रेति॑ । च॒ । बो॒ध॒य॒ । ए॒न॒म् । उदिति॑ । च॒ । ति॒ष्ठ॒ । म॒ह॒ते । सौभ॑गाय ॥ मा । च॒ । रि॒ष॒त् । उ॒प॒स॒त्तेत्यु॑प - स॒त्ता । ते॒ । अ॒ग्ने॒ । ब्र॒ह्माणः॑ । ते॒ । य॒शसः॑ । स॒न्तु॒ । मा । अ॒न्ये ॥ त्वाम् । अ॒ग्ने॒ । वृ॒ण॒ते॒ । ब्रा॒ह्म॒णाः । इ॒मे । शि॒वः । अ॒ग्ने॒ ।  \newline




\markright{ TS 4.1.7.2  \hfill https://www.vedavms.in \hfill}

\section{ TS 4.1.7.2 }

\textbf{TS 4.1.7.2 } \newline
\textbf{Samhita Paata} \newline

सं॒ ॅवर॑णे भवा नः । स॒प॒त्न॒हा नो॑ अभिमाति॒जिच्च॒ स्वे गये॑ जागृ॒ह्य प्र॑युच्छन्न् ॥ इ॒हैवाग्ने॒ अधि॑ धारया र॒यिं मा त्वा॒ निक्र॑न् पूर्व॒चितो॑ निका॒रिणः॑ । क्ष॒त्रम॑ग्ने सु॒यम॑मस्तु॒ तुभ्य॑मुपस॒त्ता व॑र्द्धतां ते॒ अनि॑ष्टृतः ॥ क्ष॒त्रेणा᳚ऽग्ने॒ स्वायुः॒ सꣳ र॑भस्व मि॒त्रेणा᳚ऽग्ने मित्र॒धेये॑ यतस्व । स॒जा॒तानां᳚ मद्ध्यम॒स्था ए॑धि॒ राज्ञा॑मग्ने विह॒व्यो॑ दीदिही॒ह ॥ अति॒ - [  ] \newline

\textbf{Pada Paata} \newline

सं॒ॅवर॑ण॒ इति॑ सं - वर॑णे । भ॒व॒ । नः॒ ॥ स॒प॒त्न॒हेति॑ सपत्न-हा । नः॒ । अ॒भि॒मा॒ति॒जिदित्य॑भिमाति - जित् । च॒ । स्वे । गये᳚ । जा॒गृ॒हि॒ । अप्र॑युच्छ॒न्नित्यप्र॑ - यु॒च्छ॒न्न् ॥ इ॒ह । ए॒व । अ॒ग्ने॒ । अधीति॑ । धा॒र॒य॒ । र॒यिम् । मा । त्वा॒ । नीति॑ । क्र॒न्न् । पू॒र्व॒चित॒ इति॑ पूर्व - चितः॑ । नि॒का॒रिण॒ इति॑ नि - का॒रिणः॑ ॥ क्ष॒त्रम् । अ॒ग्ने॒ । सु॒यम॒मिति॑ सु - यम᳚म् । अ॒स्तु॒ । तुभ्य᳚म् । उ॒प॒स॒त्तेत्यु॑प-स॒त्ता । व॒द्‌र्ध॒ता॒म् । ते॒ । अनि॑ष्टृतः ॥ क्ष॒त्रेण॑ । अ॒ग्ने॒ । स्वायु॒रिति॑ सु - आयुः॑ । समिति॑ । र॒भ॒स्व॒ । मि॒त्रेण॑ । अ॒ग्ने॒ । मि॒त्र॒धेय॒ इति॑ मित्र - धेये᳚ । य॒त॒स्व॒ ॥ स॒जा॒ताना॒मिति॑ स - जा॒ताना᳚म् । म॒द्ध्य॒म॒स्था इति॑ मद्ध्यम-स्थाः । ए॒धि॒ । राज्ञा᳚म् । अ॒ग्ने॒ । वि॒ह॒व्य॑ इति॑ वि - ह॒व्यः॑ । दी॒दि॒हि॒ । इ॒ह ॥ अतीति॑ ।  \newline




\markright{ TS 4.1.7.3  \hfill https://www.vedavms.in \hfill}

\section{ TS 4.1.7.3 }

\textbf{TS 4.1.7.3 } \newline
\textbf{Samhita Paata} \newline

निहो॒ अति॒ स्रिधो ऽत्यचि॑त्ति॒-मत्यरा॑तिमग्ने । विश्वा॒ ह्य॑ग्ने दुरि॒ता सह॒स्वाथा॒स्मभ्यꣳ॑ स॒हवी॑राꣳ र॒यिन्दाः᳚ ॥ अ॒ना॒धृ॒ष्यो जा॒तव॑दा॒ अनि॑ष्टृतो वि॒राड॑ग्ने क्षत्र॒भृद्-दी॑दिही॒ह । विश्वा॒ आशाः᳚ प्रमु॒ञ्चन् मानु॑षीर्भि॒यः शि॒वाभि॑र॒द्य परि॑ पाहि नो वृ॒धे ॥ बृह॑स्पते सवितर्बो॒धयै॑नꣳ॒॒ सꣳशि॑तं चिथ्सं त॒राꣳ सꣳ शि॑शाधि । व॒र्द्धयै॑नं मह॒ते सौभ॑गाय॒- [  ] \newline

\textbf{Pada Paata} \newline

निहः॑ । अतीति॑ । स्रिधः॑ । अतीति॑ । अचि॑त्तिम् । अतीति॑ । अरा॑तिम् । अ॒ग्ने॒ ॥ विश्वा᳚ । हि । अ॒ग्ने॒ । दु॒रि॒तेति॑ दुः - इ॒ता । सह॑स्व । अथ॑ । अ॒स्मभ्य॒मित्य॒स्म - भ्य॒म् । स॒हवी॑रा॒मिति॑ स॒ह - वी॒रा॒म् । र॒यिम् । दाः॒ ॥ अ॒ना॒धृ॒ष्य इत्य॑ना - धृ॒ष्यः । जा॒तवे॑दा॒ इति॑ जा॒त - वे॒दाः॒ । अनि॑ष्टृतः । वि॒राडिति॑ वि - राट् । अ॒ग्ने॒ । क्ष॒त्र॒भृदिति॑ क्षत्र - भृत् । दी॒दि॒हि॒ । इ॒ह ॥ विश्वाः᳚ । आशाः᳚ । प्र॒मु॒ञ्चन्निति॑ प्र - मु॒ञ्चन्न् । मानु॑षीः । भि॒यः । शि॒वाभिः॑ । अ॒द्य । परीति॑ । पा॒हि॒ । नः॒ । वृ॒धे ॥ बृह॑स्पते । स॒वि॒तः॒ । बो॒धय॑ । ए॒न॒म् । सꣳशि॑त॒मिति॒ सं - शि॒त॒म् । चि॒त् । स॒तं॒रामिति॑ सं - त॒राम् । समिति॑ । शि॒शा॒धि॒ ॥ व॒द्‌र्धय॑ । ए॒न॒म् । म॒ह॒ते । सौभ॑गाय ।  \newline




\markright{ TS 4.1.7.4  \hfill https://www.vedavms.in \hfill}

\section{ TS 4.1.7.4 }

\textbf{TS 4.1.7.4 } \newline
\textbf{Samhita Paata} \newline

विश्व॑ एन॒मनु॑ मदन्तु दे॒वाः ॥ अ॒मु॒त्र॒भूया॒दध॒ यद्य॒मस्य॒ बृह॑स्पते अ॒भिश॑स्ते॒र मु॑ञ्चः । प्रत्यौ॑हता-म॒श्विना॑ मृ॒त्युम॑स्माद् दे॒वाना॑-मग्ने भि॒षजा॒ शची॑भिः ॥ उद्व॒यं तम॑स॒स्परि॒ पश्य॑न्तो॒ ज्योति॒रुत्त॑रं । दे॒वं दे॑व॒त्रा सूर्य॒मग॑न्म॒ ज्योति॑रुत्त॒मं ॥ \newline

\textbf{Pada Paata} \newline

विश्वे᳚ । ए॒न॒म् । अन्विति॑ । म॒द॒न्तु॒ । दे॒वाः ॥ अ॒मु॒त्र॒भूया॒दित्य॑मुत्र - भूया᳚त् । अध॑ । यत् । य॒मस्य॑ । बृह॑स्पते । अ॒भिश॑स्ते॒रित्य॒भि - श॒स्तेः॒ । अमु॑ञ्चः ॥ प्रतीति॑ । औ॒ह॒ता॒म् । अ॒श्विना᳚ । मृ॒त्युम् । अ॒स्मा॒त् । दे॒वाना᳚म् । अ॒ग्ने॒ । भि॒षजा᳚ । शची॑भि॒रिति॒ शचि॑ - भिः॒ ॥ उदिति॑ । व॒यम् । तम॑सः । परीति॑ । पश्य॑न्तः । ज्योतिः॑ । उत्त॑र॒मित्युत् - त॒र॒म् ॥ दे॒वम् । दे॒व॒त्रेति॑ देव - त्रा । सूर्य᳚म् । अग॑न्म । ज्योतिः॑ । उ॒त्त॒ममित्यु॑त् - त॒मम् ॥  \newline




\markright{ TS 4.1.8.1  \hfill https://www.vedavms.in \hfill}

\section{ TS 4.1.8.1 }

\textbf{TS 4.1.8.1 } \newline
\textbf{Samhita Paata} \newline

ऊ॒र्द्ध्वा अ॑स्य स॒मिधो॑ भवन्त्यू॒र्द्ध्वा शु॒क्रा शो॒चीꣳष्य॒ग्नेः । द्यु॒मत्त॑मा सु॒प्रती॑कस्य सू॒नोः ॥ तनू॒नपा॒दसु॑रो वि॒श्ववे॑दा दे॒वो दे॒वेषु॑ दे॒वः । प॒थ आऽन॑क्ति॒ मद्ध्वा॑ घृ॒तेन॑ ॥ मद्ध्वा॑ य॒ज्ञ्ं न॑क्षसे प्रीणा॒नो नरा॒शꣳसो॑ अग्ने । सु॒कृद्दे॒वः स॑वि॒ता वि॒श्ववा॑रः ॥ अच्छा॒यमे॑ति॒ शव॑सा घृ॒तेने॑डा॒नो वह्नि॒र्नम॑सा । अ॒ग्निꣳ स्रुचो॑ अद्ध्व॒रेषु॑ प्र॒यथ्सु॑ ॥ स य॑क्षदस्य महि॒मान॑म॒ग्नेः स - [  ] \newline

\textbf{Pada Paata} \newline

ऊ॒द्‌र्ध्वाः । अ॒स्य॒ । स॒मिध॒ इति॑ सं-इधः॑ । भ॒व॒न्ति॒ । ऊ॒द्‌र्ध्वा । शु॒क्रा । शो॒चीꣳषि॑ । अ॒ग्नेः ॥ द्यु॒मत्त॒मेति॑ द्यु॒मत् - त॒मा॒ । सु॒प्रती॑क॒स्येति॑ सु - प्रती॑कस्य । सू॒नोः ॥ तनू॒नपा॒दिति॒ तनू᳚-नपा᳚त् । असु॑रः । वि॒श्ववे॑दा॒ इति॑ वि॒श्व-वे॒दाः॒ । दे॒वः । दे॒वेषु॑ । दे॒वः ॥ प॒थः । एति॑ । अ॒न॒क्ति॒ । मद्ध्वा᳚ । घृ॒तेन॑ ॥ मद्ध्वा᳚ । य॒ज्ञ्म् । न॒क्ष॒से॒ । प्री॒णा॒नः । नरा॒शꣳसः॑ । अ॒ग्ने॒ ॥ सु॒कृदिति॑ सु - कृत् । दे॒वः । स॒वि॒ता । वि॒श्ववा॑र॒ इति॑ वि॒श्व - वा॒रः॒ ॥ अच्छ॑ । अ॒यम् । ए॒ति॒ । शव॑सा । घृ॒तेन॑ । ई॒डा॒नः । वह्निः॑ । नम॑सा ॥ अ॒ग्निम् । स्रुचः॑ । अ॒द्ध्व॒रेषु॑ । प्र॒यथ्स्विति॑ प्र॒यत् - सु॒ ॥ सः । य॒क्ष॒त् । अ॒स्य॒ । म॒हि॒मान᳚म् । अ॒ग्नेः । सः ।  \newline




\markright{ TS 4.1.8.2  \hfill https://www.vedavms.in \hfill}

\section{ TS 4.1.8.2 }

\textbf{TS 4.1.8.2 } \newline
\textbf{Samhita Paata} \newline

ई॑ म॒न्द्रासु॑ प्र॒यसः॑ । वसु॒श्चेति॑ष्ठो वसु॒धात॑मश्च ॥ द्वारो॑ दे॒वीरन्व॑स्य॒ विश्वे᳚ व्र॒ता द॑दन्ते अ॒ग्नेः । उ॒रु॒व्यच॑सो॒ धाम्ना॒ पत्य॑मानाः ॥ ते अ॑स्य॒ योष॑णे दि॒व्ये न योना॑वु॒षासा॒नक्ता᳚ । इ॒मं ॅय॒ज्ञ्म॑वता मद्ध्व॒रं नः॑ ॥ दैव्या॑ होतारावू॒र्द्ध्व-म॑द्ध्व॒रं नो॒ऽग्नेर्जि॒ह्वाम॒भि गृ॑णीतं । कृ॒णु॒तं नः॒ स्वि॑ष्टिं ॥ ति॒स्रो दे॒वीर्ब॒र्॒.हिरेदꣳ स॑द॒न्त्विडा॒ सर॑स्वती॒- [  ] \newline

\textbf{Pada Paata} \newline

ई॒म् । म॒न्द्रासु॑ । प्र॒यसः॑ ॥ वसुः॑ । चेति॑ष्ठः । व॒सु॒धात॑म॒ इति॑ वसु - धात॑मः । च॒ ॥ द्वारः॑ । दे॒वीः । अन्विति॑ । अ॒स्य॒ । विश्वे᳚ । व्र॒ता । द॒द॒न्ते॒ । अ॒ग्नेः ॥ उ॒रु॒व्यच॑स॒ इत्यु॑रु - व्यच॑सः । धाम्ना᳚ । पत्य॑मानाः ॥ ते इति॑ । अ॒स्य॒ । योष॑णे॒ इति॑ । दि॒व्ये इति॑ । न । योनौ᳚ । उ॒षासा॒नक्ता᳚ ॥ इ॒मम् । य॒ज्ञ्म् । अ॒व॒ता॒म् । अ॒द्ध्व॒रम् । नः॒ ॥ दैव्या᳚ । हो॒ता॒रौ॒ । ऊ॒द्‌र्ध्वम् । अ॒द्ध्व॒रम् । नः॒ । अ॒ग्नेः । जि॒ह्वाम् । अ॒भीति॑ । गृ॒णी॒त॒म् ॥ कृ॒णु॒तम् । नः॒ । स्वि॑ष्टि॒मिति॒ सु-इ॒ष्टि॒म् ॥ ति॒स्रः । दे॒वीः । ब॒र्॒.हिः । एति॑ । इ॒दम् । स॒द॒न्तु॒ । इडा᳚ । सर॑स्वती ।  \newline




\markright{ TS 4.1.8.3  \hfill https://www.vedavms.in \hfill}

\section{ TS 4.1.8.3 }

\textbf{TS 4.1.8.3 } \newline
\textbf{Samhita Paata} \newline

भार॑ती । म॒ही गृ॑णा॒ना ॥ तन्न॑स्तु॒रीप॒मद्भु॑तं पुरु॒क्षु त्वष्टा॑ सु॒वीरं᳚ । रा॒यस्पोषं॒ ॅवि ष्य॑तु॒ नाभि॑म॒स्मे ॥ वन॑स्प॒तेऽव॑ सृजा॒ ररा॑ण॒स्त्मना॑ दे॒वेषु॑ । अ॒ग्निर्.ह॒व्यꣳ श॑मि॒ता सू॑दयाति ॥ अग्ने॒ स्वाहा॑ कृणुहि जातवेद॒ इन्द्रा॑य ह॒व्यं । विश्वे॑ दे॒वा ह॒विरि॒दं जु॑षन्तां ॥ हि॒र॒ण्य॒ग॒र्भः सम॑वर्त॒ताग्रे॑ भू॒तस्य॑ जा॒तः पति॒रेक॑ आसीत् । स दा॑धार पृथि॒वीं द्या - [  ] \newline

\textbf{Pada Paata} \newline

भार॑ती ॥ म॒ही । गृ॒णा॒ना ॥ तत् । नः॒ । तु॒रीप᳚म् । अद्भु॑तम् । पु॒रु॒क्षु । त्वष्टा᳚ । सु॒वीर॒मिति॑ सु - वीर᳚म् ॥ रा॒यः । पोष᳚म् । वीति॑ । स्य॒तु॒ । नाभि᳚म् । अ॒स्मे इति॑ ॥ वन॑स्पते । अवेति॑ । सृ॒ज॒ । ररा॑णः । त्मना᳚ । दे॒वेषु॑ ॥ अ॒ग्निः । ह॒व्यम् । श॒मि॒ता । सू॒द॒या॒ति॒ ॥ अग्ने᳚ । स्वाहा᳚ । कृ॒णु॒हि॒ । जा॒त॒वे॒द॒ इति॑ जात - वे॒दः॒ । इन्द्रा॑य । ह॒व्यम् ॥ विश्वे᳚ । दे॒वाः । ह॒विः । इ॒दम् । जु॒ष॒न्ता॒म् ॥ हि॒र॒ण्य॒ग॒र्भ इति॑ हिरण्य-ग॒र्भः । समिति॑ । अ॒व॒र्त॒त॒ । अग्रे᳚ । भू॒तस्य॑ । जा॒तः । पतिः॑ । एकः॑ । आ॒सी॒त् ॥ सः । दा॒धा॒र॒ । पृ॒थि॒वीम् । द्याम् ।  \newline




\markright{ TS 4.1.8.4  \hfill https://www.vedavms.in \hfill}

\section{ TS 4.1.8.4 }

\textbf{TS 4.1.8.4 } \newline
\textbf{Samhita Paata} \newline

-मु॒तेमां कस्मै॑ दे॒वाय॑ ह॒विषा॑ विधेम ॥ यः प्रा॑ण॒तो नि॑मिष॒तो म॑हि॒त्वैक॒ इद्राजा॒ जग॑तो ब॒भूव॑ । य ईशे॑ अ॒स्य द्वि॒पद॒श्चतु॑ष्पदः᳡कस्मै॑ दे॒वाय॑ ह॒विषा॑ विधेम ॥ य आ᳚त्म॒दा ब॑ल॒दा यस्य॒ विश्व॑ उ॒पास॑ते प्र॒शिषं॒ ॅयस्य॑ दे॒वाः । यस्य॑ छा॒याऽमृतं॒ ॅयस्य॑ मृ॒त्युः कस्मै॑ दे॒वाय॑ ह॒विषा॑ विधेम ॥ यस्ये॒मे हि॒मव॑न्तो महि॒त्वा यस्य॑ समु॒द्रꣳ र॒सया॑ स॒हा - [  ] \newline

\textbf{Pada Paata} \newline

उ॒त । इ॒माम् । कस्मै᳚ । दे॒वाय॑ । ह॒विषा᳚ । वि॒धे॒म॒ ॥ यः । प्रा॒ण॒त इति॑ प्र - अ॒न॒तः । नि॒मि॒ष॒त इति॑ नि - मि॒ष॒तः । म॒हि॒त्वेति॑ महि - त्वा । एकः॑ । इत् । राजा᳚ । जग॑तः । ब॒भूव॑ ॥ यः । ईशे᳚ । अ॒स्य । द्वि॒पद॒ इति॑ द्वि - पदः॑ । चतु॑ष्पद॒ इति॒ चतुः॑ - प॒दः॒ । कस्मै᳚ । दे॒वाय॑ । ह॒विषा᳚ । वि॒धे॒म॒ ॥ यः । आ॒त्म॒दा इत्या᳚त्म-दाः । ब॒ल॒दा इति॑ बल-दाः । यस्य॑ । विश्वे᳚ । उ॒पास॑त॒ इत्यु॑प - आस॑ते । प्र॒शिष॒मिति॑ प्र - शिष᳚म् । यस्य॑ । दे॒वाः ॥ यस्य॑ । छा॒या । अ॒मृत᳚म् । यस्य॑ । मृ॒त्युः । कस्मै᳚ । दे॒वाय॑ । ह॒विषा᳚ । वि॒धे॒म॒ ॥ यस्य॑ । इ॒मे । हि॒मव॑न्त॒ इति॑ हि॒म - व॒न्तः॒ । म॒हि॒त्वेति॑ महि - त्वा । यस्य॑ । स॒मु॒द्रम् । र॒सया᳚ । स॒ह ।  \newline




\markright{ TS 4.1.8.5  \hfill https://www.vedavms.in \hfill}

\section{ TS 4.1.8.5 }

\textbf{TS 4.1.8.5 } \newline
\textbf{Samhita Paata} \newline

-ऽऽहुः । यस्ये॒माः प्र॒दिशो॒ यस्य॑ बा॒हू कस्मै॑ दे॒वाय॑ ह॒विषा॑ विधेम ॥ यं क्रन्द॑सी॒ अव॑सा तस्तभा॒ने अ॒भ्यैक्षे॑तां॒ मन॑सा॒ रेज॑माने ।यत्राधि॒ सूर॒ उदि॑तौ॒ व्येति॒ कस्मै॑ दे॒वाय॑ ह॒विषा॑ विधेम ॥येन॒ द्यौरु॒ग्रा पृ॑थि॒वी च॑ दृ॒ढे येन॒ सु॒वः॑ स्तभि॒तं ॅयेन॒ नाकः॑ ।यो अ॒न्तरि॑क्षे॒ रज॑सो वि॒मानः᳡कस्मै॑ दे॒वाय॑ ह॒विषा॑ विधेम ॥ आपो॑ ह॒ यन्म॑ह॒ती र्विश्व॒ - [  ] \newline

\textbf{Pada Paata} \newline

आ॒हुः ॥ यस्य॑ । इ॒माः । प्र॒दिश॒ इति॑ प्र-दिशः॑ । यस्य॑ । बा॒हू इति॑ । कस्मै᳚ । दे॒वाय॑ । ह॒विषा᳚ । वि॒धे॒म॒ ॥ यम् । क्रन्द॑सी॒ इति॑ । अव॑सा । त॒स्त॒भा॒ने इति॑ । अ॒भ्यैक्षे॑ता॒मित्य॑भि - ऐक्षे॑तां । मन॑सा । रेज॑माने॒ इति॑ ॥ यत्र॑ । अधीति॑ । सूरः॑ । उदि॑ता॒वित्युत् - इ॒तौ॒ । व्येतीति॑ वि - एति॑ । कस्मै᳚ । दे॒वाय॑ । ह॒विषा᳚ । वि॒धे॒म॒ ॥ येन॑ । द्यौः । उ॒ग्रा । पृ॒थि॒वी । च॒ । दृ॒ढे इति॑ । येन॑ । सुवः॑ । स्त॒भि॒तम् । येन॑ । नाकः॑ ॥ यः । अ॒न्तरि॑क्षे । रज॑सः । वि॒मान॒ इति॑ वि-मानः॑ । कस्मै᳚ । दे॒वाय॑ । ह॒विषा᳚ । वि॒धे॒म॒ ॥ आपः॑ । ह॒ । यत् । म॒ह॒तीः । विश्व᳚म् ।  \newline




\markright{ TS 4.1.8.6  \hfill https://www.vedavms.in \hfill}

\section{ TS 4.1.8.6 }

\textbf{TS 4.1.8.6 } \newline
\textbf{Samhita Paata} \newline

-माय॒न् दक्षं॒ दधा॑ना ज॒नय॑न्तीर॒ग्निं ।ततो॑ दे॒वानां॒ निर॑वर्त॒तासु॒रेकः᳡कस्मै॑ दे॒वाय॑ ह॒विषा॑ विधेम ॥यश्चि॒दापो॑ महि॒ना प॒र्यप॑श्य॒द्-दक्षं॒ दधा॑ना ज॒नय॑न्तीर॒ग्निं ।यो दे॒वेष्वधि॑ दे॒व एक॒ आसी॒त् कस्मै॑ दे॒वाय॑ ह॒विषा॑ विधेम ॥ \newline

\textbf{Pada Paata} \newline

आयन्न्॑ । दक्ष᳚म् । दधा॑नाः । ज॒नय॑न्तीः । अ॒ग्निम् ॥ ततः॑ । दे॒वाना᳚म् । निरिति॑ । अ॒व॒र्त॒त॒ । असुः॑ । एकः॑ । कस्मै᳚ । दे॒वाय॑ । ह॒विषा᳚ । वि॒धे॒म॒ ॥ यः । चि॒त् । आपः॑ । म॒हि॒ना । प॒र्यप॑श्य॒दिति॑ परि-अप॑श्यत् । दक्ष᳚म् । दधा॑नाः । ज॒नय॑न्तीः । अ॒ग्निम् ॥ यः । दे॒वेषु॑ । अधीति॑ । दे॒वः । एकः॑ । आसी᳚त् । कस्मै᳚ । दे॒वाय॑ । ह॒विषा᳚ । वि॒धे॒म॒ ॥  \newline




\markright{ TS 4.1.9.1  \hfill https://www.vedavms.in \hfill}

\section{ TS 4.1.9.1 }

\textbf{TS 4.1.9.1 } \newline
\textbf{Samhita Paata} \newline

आकू॑तिम॒ग्निं प्र॒युजꣳ॒॒ स्वाहा॒ मनो॑ मे॒धाम॒ग्निं प्र॒युजꣳ॒॒ स्वाहा॑ चि॒त्तं ॅविज्ञा॑तम॒ग्निं प्र॒युजꣳ॒॒ स्वाहा॑ वा॒चो विधृ॑तिम॒ग्निं प्र॒युजꣳ॒॒ स्वाहा᳚ प्र॒जाप॑तये॒ मन॑वे॒ स्वाहा॒ऽग्नये॑ वैश्वान॒राय॒ स्वाहा॒ विश्वे॑ दे॒वस्य॑ ने॒तुर्मर्तो॑ वृणीत स॒ख्यं ॅविश्वे॑ रा॒य इ॑षुद्ध्यसि द्यु॒म्नं ॅवृ॑णीत पु॒ष्यसे॒ स्वाहा॒ मा सु भि॑त्था॒ मा सु रि॑षो॒ दृꣳह॑स्व वी॒डय॑स्व॒ सु । अबं॑ धृष्णु वी॒रय॑स्वा॒ - [  ] \newline

\textbf{Pada Paata} \newline

आकू॑ति॒मित्या - कू॒ति॒म् । अ॒ग्निम् । प्र॒युज॒मिति॑ प्र - युज᳚म् । स्वाहा᳚ । मनः॑ । मे॒धाम् । अ॒ग्निम् । प्र॒युज॒मिति॑ प्र-युज᳚म् । स्वाहा᳚ । चि॒त्तम् । विज्ञा॑त॒मिति॑ वि - ज्ञा॒त॒म् । अ॒ग्निम् । प्र॒युज॒मिति॑ प्र - युज᳚म् । स्वाहा᳚ । वा॒चः । विधृ॑ति॒मिति॒ वि-धृ॒ति॒म् । अ॒ग्निम् । प्र॒युज॒मिति॑ प्र - युज᳚म् । स्वाहा᳚ । प्र॒जाप॑तय॒ इति॑ प्र॒जा - प॒त॒ये॒ । मन॑वे । स्वाहा᳚ । अ॒ग्नये᳚ । वै॒श्वा॒न॒राय॑ । स्वाहा᳚ । विश्वे᳚ । दे॒वस्य॑ । ने॒तुः । मर्तः॑ । वृ॒णी॒त॒ । स॒ख्यम् । विश्वे᳚ । रा॒यः । इ॒षु॒द्ध्य॒सि॒ । द्यु॒म्नम् । वृ॒णी॒त॒ । पु॒ष्यसे᳚ । स्वाहा᳚ । मा । स्विति॑ । भि॒त्थाः॒ । मा । स्विति॑ । रि॒षः॒ । दृꣳह॑स्व । वी॒डय॑स्व । सु ॥ अबं॑ । धृ॒ष्णु॒ । वी॒रय॑स्व ।  \newline




\markright{ TS 4.1.9.2  \hfill https://www.vedavms.in \hfill}

\section{ TS 4.1.9.2 }

\textbf{TS 4.1.9.2 } \newline
\textbf{Samhita Paata} \newline

-ऽग्निश्चे॒दं क॑रिष्यथः ॥ दृꣳह॑स्व देवि पृथिवि स्व॒स्तय॑ आसु॒री मा॒या स्व॒धया॑ कृ॒ताऽसि॑ । जुष्टं॑ दे॒वाना॑मि॒दम॑स्तु ह॒व्यमरि॑ष्टा॒ त्वमुदि॑हि य॒ज्ञे अ॒स्मिन्न् ॥ मित्रै॒तामु॒खां त॑पै॒षा मा भे॑दि । ए॒तां ते॒ परि॑ ददा॒म्यभि॑त्त्यै ॥ द्र्व॑न्नः स॒र्पिरा॑सुतिः प्र॒त्नो होता॒ वरे᳚ण्यः । सह॑सस्पु॒त्रो अद्भु॑तः ॥ पर॑स्या॒ अधि॑ सं॒ॅवतोऽव॑राꣳ अ॒भ्या - [  ] \newline

\textbf{Pada Paata} \newline

अ॒ग्निः । च॒ । इ॒दम् । क॒रि॒ष्य॒थः॒ ॥ दृꣳह॑स्व । दे॒वि॒ । पृ॒थि॒वि॒ । स्व॒स्तये᳚ । आ॒सु॒री । मा॒या । स्व॒धयेति॑ स्व - धया᳚ । कृ॒ता । अ॒सि॒ ॥ जुष्ट᳚म् । दे॒वाना᳚म् । इ॒दम् । अ॒स्तु॒ । ह॒व्यम् । अरि॑ष्टा । त्वम् । उदिति॑ । इ॒हि॒ । य॒ज्ञे । अ॒स्मिन्न् ॥ मित्र॑ । ए॒ताम् । उ॒खाम् । त॒प॒ । ए॒षा । मा । भे॒दि॒ ॥ ए॒ताम् । ते॒ । परीति॑ । द॒दा॒मि॒ । अभि॑त्त्यै ॥ द्र्‌व॑न्न॒ इति॒ द्रु - अ॒न्नः॒ । स॒र्पिरा॑सुति॒रिति॑ स॒र्पिः - आ॒सु॒तिः॒ । प्र॒त्नः । होता᳚ । वरे᳚ण्यः ॥ सह॑सः । पु॒त्रः । अद्भु॑तः ॥ पर॑स्याः । अधीति॑ । सं॒ॅवत॒ इति॑ सं - वतः॑ । अव॑रान् । अ॒भि । एति॑ ।  \newline




\markright{ TS 4.1.9.3  \hfill https://www.vedavms.in \hfill}

\section{ TS 4.1.9.3 }

\textbf{TS 4.1.9.3 } \newline
\textbf{Samhita Paata} \newline

त॑र । यत्रा॒हमस्मि॒ ताꣳ अ॑व ॥ प॒र॒मस्याः᳚ परा॒वतो॑ रो॒हिद॑श्व इ॒हाऽ*ग॑हि । पु॒री॒ष्यः॑ पुरुप्रि॒योऽग्ने॒ त्वं त॑रा॒ मृधः॑ ॥ सीद॒ त्वं मा॒तुर॒स्या उ॒पस्थे॒ विश्वा᳚न्यग्ने व॒युना॑नि वि॒द्वान् । मैना॑म॒र्चिषा॒ मा तप॑सा॒ऽभि शू॑शुचो॒ऽन्तर॑स्याꣳ शु॒क्र ज्यो॑ति॒र्वि भा॑हि ॥ अ॒न्तर॑ग्ने रु॒चा त्वमु॒खायै॒ सद॑ने॒ स्वे । तस्या॒स्त्वꣳ हर॑सा॒ तप॒न् ( ) जात॑वेदः शि॒वो भ॑व ॥ शि॒वो भू॒त्वा मह्य॑म॒ग्नेऽथो॑ सीद शि॒वस्त्वं । शि॒वाः कृ॒त्वा दिशः॒ सर्वाः॒ स्वां ॅयोनि॑मि॒हाऽऽस॑दः ॥ \newline

\textbf{Pada Paata} \newline

त॒र॒ ॥ यत्र॑ । अ॒हम् । अस्मि॑ । तान् । अ॒व॒ ॥ प॒र॒मस्याः᳚ । प॒रा॒वत॒ इति॑ परा - वतः॑ । रो॒हिद॑श्व॒ इति॑ रो॒हित् - अ॒श्वः॒ । इ॒ह । एति॑ । ग॒हि॒ ॥ पु॒री॒ष्यः॑ । पु॒रु॒प्रि॒य इति॑ पुरु - प्रि॒यः । अग्ने᳚ । त्वम् । त॒र॒ । मृधः॑ ॥ सीद॑ । त्वम् । मा॒तुः । अ॒स्याः । उ॒पस्थ॒ इत्यु॒प - स्थे॒ । विश्वा॑नि । अ॒ग्ने॒ । व॒युना॑नि । वि॒द्वान् ॥ मा । ए॒ना॒म् । अ॒र्चिषा᳚ । मा । तप॑सा । अ॒भीति॑ । शू॒शु॒चः॒ । अ॒न्तः । अ॒स्या॒म् । शु॒क्रज्यो॑ति॒रिति॑ शु॒क्र - ज्यो॒तिः॒ । वीति॑ । भा॒हि॒ ॥ अ॒न्तः । अ॒ग्ने॒ । रु॒चा । त्वम् । उ॒खायै᳚ । सद॑ने । स्वे ॥ तस्याः᳚ । त्वम् । हर॑सा । तपन्न्॑ ( ) । जात॑वेद॒ इति॒ जात॑-वे॒दः॒ । शि॒वः । भ॒व॒ ॥ शि॒वः । भू॒त्वा । मह्य᳚म् । अ॒ग्ने॒ । अथो॒ इति॑ । सी॒द॒ । शि॒वः । त्वम् ॥ शि॒वाः । कृ॒त्वा । दिशः॑ । सर्वाः᳚ । स्वाम् । योनि᳚म् । इ॒ह । एति॑ । अ॒स॒दः॒ ॥  \newline




\markright{ TS 4.1.10.1  \hfill https://www.vedavms.in \hfill}

\section{ TS 4.1.10.1 }

\textbf{TS 4.1.10.1 } \newline
\textbf{Samhita Paata} \newline

यद॑ग्ने॒ यानि॒ कानि॒ चाऽऽते॒ दारू॑णि द॒द्ध्मसि॑ । तद॑स्तु॒ तुभ्य॒मिद्-घृ॒तं तज्जु॑षस्व यविष्ठ्य ॥ यदत्त्यु॑प॒जिह्वि॑का॒ यद्व॒म्रो अ॑ति॒सर्प॑ति । सर्वं॒ तद॑स्तु ते घृ॒तं तज्जु॑षस्व यविष्ठ्य ॥ रात्रिꣳ॑ रात्रि॒मप्र॑यावं॒ भर॒न्तोऽश्वा॑येव॒ तिष्ठ॑ते घा॒सम॑स्मै । रा॒यस्पोषे॑ण॒ समि॒षा मद॒न्तोऽग्ने॒ मा ते॒ प्रति॑वेशा रिषाम ॥ नाभा॑ - [  ] \newline

\textbf{Pada Paata} \newline

यत् । अ॒ग्ने॒ । यानि॑ । कानि॑ । च॒ । एति॑ । ते॒ । दारू॑णि । द॒द्ध्मसि॑ ॥ तत् । अ॒स्तु॒ । तुभ्य᳚म् । इत् । घृ॒तम् । तत् । जु॒ष॒स्व॒ । य॒वि॒ष्ठ्य॒ ॥ यत् । अत्ति॑ । उ॒प॒जिह्वि॒केत्यु॑प - जिह्वि॑का । यत् । व॒म्रः । अ॒ति॒सर्प॒तीत्य॑ति-सर्प॑ति ॥ सर्व᳚म् । तत् । अ॒स्तु॒ । ते॒ । घृ॒तम् । तत् । जु॒ष॒स्व॒ । य॒वि॒ष्ठ्य॒ ॥ रात्रिꣳ॑रात्रि॒मिति॒ रात्रि᳚म् - रा॒त्रि॒म् । अप्र॑याव॒मित्यप्र॑-या॒व॒म् । भर॑न्तः । अश्वा॑य । इ॒व॒ । तिष्ठ॑ते । घा॒सम् । अ॒स्मै॒ ॥ रा॒यः । पोषे॑ण । समिति॑ । इ॒षा । मद॑न्तः । अग्ने᳚ । मा । ते॒ । प्रति॑वेशा॒ इति॒ प्रति॑ - वे॒शाः॒ । रि॒षा॒म॒ ॥ नाभा᳚ ।  \newline




\markright{ TS 4.1.10.2  \hfill https://www.vedavms.in \hfill}

\section{ TS 4.1.10.2 }

\textbf{TS 4.1.10.2 } \newline
\textbf{Samhita Paata} \newline

पृथि॒व्याः स॑मिधा॒-नम॒ग्निꣳ रा॒यस्पोषा॑य बृह॒ते ह॑वामहे । इ॒र॒म्म॒दं बृ॒हदु॑क्थं॒ ॅयज॑त्रं॒ जेता॑रम॒ग्निं पृत॑नासु सास॒हिं ॥ याः सेना॑ अ॒भीत्व॑रीरा-व्या॒धिनी॒-रुग॑णा उ॒त । ये स्ते॒ना ये च॒ तस्क॑रा॒स्ताꣳस्ते॑ अ॒ग्नेऽपि॑ दधाम्या॒स्ये᳚ ॥ दꣳष्ट्रा᳚भ्यां म॒लिम्लू॒न् जंभ्यै॒-स्तस्क॑राꣳ उ॒त । हनू᳚भ्याꣳ स्ते॒नान्-भ॑गव॒स्ताꣳ-स्त्वं खा॑द॒ सुखा॑दितान् ॥ ये जने॑षु म॒लिम्ल॑वः स्ते॒नास॒-स्तस्क॑रा॒ वने᳚ । ये - [  ] \newline

\textbf{Pada Paata} \newline

पृ॒थि॒व्याः । स॒मि॒धा॒नमिति॑ सं-इ॒धा॒नम् । अ॒ग्निम् । रा॒यः । पोषा॑य । बृ॒ह॒ते । ह॒वा॒म॒हे॒ ॥ इ॒र॒मं॒दमिती॑रं - म॒दम् । बृ॒हदु॑क्थ॒मिति॑ बृ॒हत् - उ॒क्थ॒म् । यज॑त्रम् । जेता॑रम् । अ॒ग्निम् । पृत॑नासु । सा॒स॒हिम् ॥ याः । सेनाः᳚ । अ॒भीत्व॑री॒रित्य॑भि - इत्व॑रीः । आ॒व्या॒धिनी॒रित्या᳚ - व्या॒धिनीः᳚ । उग॑णाः । उ॒त ॥ ये । स्ते॒नाः । ये । च॒ । तस्क॑राः । तान् । ते॒ । अ॒ग्ने॒ । अपीति॑ । द॒धा॒मि॒ । आ॒स्ये᳚ ॥ दꣳष्ट्रा᳚भ्याम् । म॒लिम्लून्न्॑ । जंभ्यैः᳚ । तस्क॑रान् । उ॒त ॥ हनू᳚भ्या॒मिति॒ हनु॑ - भ्या॒म् । स्ते॒नान् । भ॒ग॒व॒ इति॑ भग - वः॒ । तान् । त्वम् । खा॒द॒ । सुखा॑दिता॒निति॒ सु - खा॒दि॒ता॒न् ॥ ये । जने॑षु । म॒लिम्ल॑वः । स्ते॒नासः॑ । तस्क॑राः । वने᳚ ॥ ये ।  \newline




\markright{ TS 4.1.10.3  \hfill https://www.vedavms.in \hfill}

\section{ TS 4.1.10.3 }

\textbf{TS 4.1.10.3 } \newline
\textbf{Samhita Paata} \newline

कक्षे᳚ष्वघा॒ यव॒स्ताꣳस्ते॑ दधामि॒ जंभ॑योः ॥ यो अ॒स्मभ्य॑मराती॒याद्-यश्च॑ नो॒ द्वेष॑ते॒ जनः॑ । निन्दा॒द्यो अ॒स्मान् दिफ्सा᳚च्च॒ सर्वं॒ तं म॑स्म॒सा कु॑रु ॥ सꣳशि॑तं मे॒ ब्रह्म॒ सꣳशि॑तं ॅवी॒र्यं॑ बलं᳚ । सꣳशि॑तं क्ष॒त्रं जि॒ष्णु यस्या॒ऽहमस्मि॑ पु॒रोहि॑तः ॥ उदे॑षां बा॒हू अ॑तिर॒मुद्वर्च॒ उदू॒ बलं᳚ । क्षि॒णोमि॒ ब्रह्म॑णा॒-ऽमित्रा॒नुन्न॑यामि॒ - [  ] \newline

\textbf{Pada Paata} \newline

कक्षे॑षु । अ॒घा॒यव॒ इत्य॑घ - यवः॑ । तान् । ते॒ । द॒धा॒मि॒ । जंभ॑योः ॥ यः । अ॒स्मभ्य॒मित्य॒स्म-भ्य॒म् । अ॒रा॒ती॒यात् । यः । च॒ । नः॒ । द्वेष॑ते । जनः॑ ॥ निन्दा᳚त् । यः । अ॒स्मान् । दिफ्सा᳚त् । च॒ । सर्व᳚म् । तम् । म॒स्म॒सा । कु॒रु॒ ॥ सꣳशि॑त॒मिति॒ सं - शि॒त॒म् । मे॒ । ब्रह्म॑ । सꣳशि॑त॒मिति॒ सं - शि॒त॒म् । वी॒र्य᳚म् । बल᳚म् ॥ सꣳशि॑त॒मिति॒ सं - शि॒त॒म् । क्ष॒त्रम् । जि॒ष्णु । यस्य॑ । अ॒हम् । अस्मि॑ । पु॒रोहि॑त॒ इति॑ पु॒रः - हि॒तः॒ ॥ उदिति॑ । ए॒षा॒म् । बा॒हू इति॑ । अ॒ति॒र॒म् । उदिति॑ । वर्चः॑ । उदिति॑ । उ॒ । बल᳚म् ॥ क्षि॒णोमि॑ । ब्रह्म॑णा । अ॒मित्रान्॑ । उदिति॑ । न॒या॒मि॒ ।  \newline




\markright{ TS 4.1.10.4  \hfill https://www.vedavms.in \hfill}

\section{ TS 4.1.10.4 }

\textbf{TS 4.1.10.4 } \newline
\textbf{Samhita Paata} \newline

स्वाꣳ अ॒हं ॥ दृ॒शा॒नो रु॒क्म उ॒र्व्या व्य॑द्यौद्दु॒र्मर्.ष॒मायुः॑ श्रि॒ये रु॑चा॒नः । अ॒ग्निर॒मृतो॑ अभव॒द्वयो॑-भि॒र्यदे॑नं॒ द्यौरज॑नयथ् सु॒रेताः᳚ ॥ विश्वा॑ रू॒पाणि॒ प्रति॑ मुञ्चते क॒विः प्राऽसा॑वीद्भ॒द्रं द्वि॒पदे॒ चतु॑ष्पदे । वि नाक॑मख्यथ् सवि॒ता वरे॒ण्योऽनु॑ प्र॒याण॑मु॒षसो॒ वि॑राजति ॥ नक्तो॒षासा॒ सम॑नसा॒ विरू॑पे धा॒पये॑ते॒ शिशु॒मेकꣳ॑ समी॒ची । द्यावा॒ क्षामा॑ रु॒क्मो - [  ] \newline

\textbf{Pada Paata} \newline

स्वान् । अ॒हम् ॥ दृ॒शा॒नः । रु॒क्मः । उ॒र्व्या । वीति॑ । अ॒द्यौ॒त् । दु॒र्मर्.ष॒मिति॑ दुः - मर्.ष᳚म् । आयुः॑ । श्रि॒ये । रु॒चा॒नः ॥ अ॒ग्निः । अ॒मृतः॑ । अ॒भ॒व॒त् । वयो॑भि॒रिति॒ वयः॑ - भिः॒ । यत् । ए॒न॒म् । द्यौः । अज॑नयत् । सु॒रेता॒ इति॑ सु - रेताः᳚ ॥ विश्वा᳚ । रू॒पाणि॑ । प्रतीति॑ । मु॒ञ्च॒ते॒ । क॒विः । प्रेति॑ । अ॒सा॒वी॒त् । भ॒द्रम् । द्वि॒पद॒ इति॑ द्वि-पदे᳚ । चतु॑ष्पद॒ इति॒ चतुः॑ - प॒दे॒ ॥ वीति॑ । नाक᳚म् । अ॒ख्य॒त् । स॒वि॒ता । वरे᳚ण्यः । अन्विति॑ । प्र॒याण॒मिति॑ प्र - यान᳚म् । उ॒षसः॑ । वीति॑ । रा॒ज॒ति॒ ॥ नक्तो॒षासा᳚ । सम॑न॒सेति॒ स - म॒न॒सा॒ । विरू॑पे॒ इति॒ वि-रू॒पे॒ । धा॒पये॑ते॒ इति॑ । शिशु᳚म् । एक᳚म् । स॒मी॒ची इति॑ ॥ द्यावा᳚ । क्षाम॑ । रु॒क्मः ।  \newline




\markright{ TS 4.1.10.5  \hfill https://www.vedavms.in \hfill}

\section{ TS 4.1.10.5 }

\textbf{TS 4.1.10.5 } \newline
\textbf{Samhita Paata} \newline

अ॒न्तर्वि भा॑ति दे॒वा अ॒ग्निं धा॑रयन् द्रविणो॒दाः ॥ सु॒प॒र्णो॑ऽसि ग॒रुत्मा᳚न् त्रि॒वृत्ते॒ शिरो॑ गाय॒त्रं चक्षुः॒ स्तोम॑ आ॒त्मा साम॑ ते त॒नूर्वा॑मदे॒व्यं बृ॑हद्-रथन्त॒रे प॒क्षौ य॑ज्ञाय॒ज्ञियं॒ पुच्छं॒ छन्दाꣳ॒॒स्यङ्गा॑नि॒ धिष्णि॑याः श॒फा यजूꣳ॑षि॒ नाम॑ ॥ सु॒प॒र्णो॑ऽसि ग॒रुत्मा॒न् दिवं॑ गच्छ॒ सुवः॑ पत ॥ \newline

\textbf{Pada Paata} \newline

अ॒न्तः । वीति॑ । भा॒ति॒ । दे॒वाः । अ॒ग्निम् । धा॒र॒य॒न्न् । द्र॒वि॒णो॒दा इति॑ द्रविणः - दाः ॥ सु॒प॒र्ण इति॑ सु - प॒र्णः । अ॒सि॒ । ग॒रुत्मान्॑ । त्रि॒वृदिति॑ त्रि - वृत् । ते॒ । शिरः॑ । गा॒य॒त्रम् । चक्षुः॑ । स्तोमः॑ । आ॒त्मा । साम॑ । ते॒ । त॒नूः । वा॒म॒दे॒व्यमिति॑ वाम - दे॒व्यम् । बृ॒ह॒द्र॒थ॒न्त॒रे इति॑ बृहत् - र॒थ॒न्त॒रे । प॒क्षौ । य॒ज्ञा॒य॒ज्ञिय᳚म् । पुच्छ᳚म् । छन्दाꣳ॑सि । अङ्गा॑नि । धिष्णि॑याः । श॒फाः । यजूꣳ॑षि । नाम॑ ॥ सु॒प॒र्ण इति॑ सु - प॒र्णः । अ॒सि॒ । ग॒रुत्मान्॑ । दिव᳚म् । ग॒च्छ॒ । सुवः॑ । प॒त॒ ॥  \newline




\markright{ TS 4.1.11.1  \hfill https://www.vedavms.in \hfill}

\section{ TS 4.1.11.1 }

\textbf{TS 4.1.11.1 } \newline
\textbf{Samhita Paata} \newline

अग्ने॒ यं ॅय॒ज्ञ्म॑द्ध्व॒रं ॅवि॒श्वतः॑ परि॒भूरसि॑ । स इद्दे॒वेषु॑ गच्छति ॥ सोम॒ यास्ते॑ मयो॒भुव॑ ऊ॒तयः॒ सन्ति॑ दा॒शुषे᳚ । ताभि॑र्नोऽवि॒ता भ॑व ॥ अ॒ग्निर्मू॒र्धा >1 , भुवः॑ >त्वन्नः॑ सोम॒ >3 , या ते॒ धामा॑नि >4 ॥ तथ् स॑वि॒तुर्वरे᳚ण्यं॒ भर्गो॑ दे॒वस्य॑ धीमहि । धियो॒ यो नः॑ प्रचो॒दया᳚त् ॥ अचि॑त्ती॒ यच्च॑कृ॒मा दैव्ये॒ जने॑ दी॒नैर्दक्षैः॒ प्रभू॑ती पूरुष॒त्वता᳚ । \newline

\textbf{Pada Paata} \newline

अग्ने᳚ । यम् । य॒ज्ञ्म् । अ॒द्ध्व॒रम् । वि॒श्वतः॑ । प॒रि॒भूरिति॑ परि - भूः । असि॑ ॥ सः । इत् । दे॒वेषु॑ । ग॒च्छ॒ति॒ ॥ सोम॑ । याः । ते॒ । म॒यो॒भुव॒ इति॑ मयः - भुवः॑ । ऊ॒तयः॑ । सन्ति॑ । दा॒शुषे᳚ ॥ ताभिः॑ । नः॒ । अ॒वि॒ता । भ॒व॒ ॥ अ॒ग्निः । मू॒द्‌र्धा । भुवः॑ ॥ त्वम् । नः॒ । सो॒म॒ । या । ते॒ । धामा॑नि ॥ तत् । स॒वि॒तुः । वरे᳚ण्यम् । भर्गः॑ । दे॒वस्य॑ । धी॒म॒हि॒ ॥ धियः॑ । यः । नः॒ । प्र॒चो॒दया॒दिति॑ प्र-चो॒दया᳚त् ॥ अचि॑त्ती । यत् । च॒कृ॒म । दैव्ये᳚ । जने᳚ । दी॒नैः । दक्षैः᳚ । प्रभू॒तीति॒ प्र - भू॒ती॒ । पू॒रु॒ष॒त्वतेति॑ पूरुष - त्वता᳚ ॥  \newline




\markright{ TS 4.1.11.2  \hfill https://www.vedavms.in \hfill}

\section{ TS 4.1.11.2 }

\textbf{TS 4.1.11.2 } \newline
\textbf{Samhita Paata} \newline

दे॒वेषु॑ च सवित॒र्मानु॑षेषु च॒ त्वन्नो॒ अत्र॑ सुवता॒दना॑गसः ॥ चो॒द॒यि॒त्री सू॒नृता॑नां॒ चेत॑न्ती सुमती॒नां । य॒ज्ञ्ं द॑धे॒ सर॑स्वती ॥ पावी॑रवी क॒न्या॑ चि॒त्रायुः॒ सर॑स्वती वी॒रप॑त्नी॒ धियं॑ धात् । ग्नाभि॒रच्छि॑द्रꣳ शर॒णꣳ स॒जोषा॑ दुरा॒धर्.षं॑ गृण॒ते शर्म॑ यꣳसत् ॥ पू॒षा गा अन्वे॑तु नः पू॒षा र॑क्ष॒त्वर्व॑तः । पू॒षा वाजꣳ॑ सनोतु नः ॥ शु॒क्रं ते॑ अ॒न्यद्य॑ज॒तं ते॑ अ॒न्य - [  ] \newline

\textbf{Pada Paata} \newline

दे॒वेषु । च॒ । स॒वि॒तः॒ । मानु॑षेषु । च॒ । त्वम् । नः॒ । अत्र॑ । सु॒व॒ता॒त् । अना॑गसः ॥ चो॒द॒यि॒त्री । सू॒नृता॑नाम् । चेत॑न्ती । सु॒म॒ती॒नामिति॑ सु - म॒ती॒नाम् ॥ य॒ज्ञ्म् । द॒धे॒ । सर॑स्वती ॥ पावी॑रवी । क॒न्या᳚ । चि॒त्रायु॒रिति॑ चि॒त्र - आ॒युः॒ । सर॑स्वती । वी॒रप॒त्नीति॑ वी॒र - प॒त्नी॒ । धिय᳚म् । धा॒त् ॥ ग्नाभिः॑ । अच्छि॑द्रम् । श॒र॒णम् । स॒जोषा॒ इति॑ स-जोषाः᳚ । दु॒रा॒धर्.ष॒मिति॑ दुः - आ॒धर्.ष᳚म् । गृ॒ण॒ते । शर्म॑ । यꣳ॒॒स॒त् ॥ पू॒षा । गाः । अन्विति॑ । ए॒तु॒ । नः॒ । पू॒षा । र॒क्ष॒तु॒ । अर्व॑तः ॥ पू॒षा । वाज᳚म् । स॒नो॒तु॒ । नः॒ ॥ शु॒क्रम् । ते॒ । अ॒न्यत् । य॒ज॒तम् । ते॒ । अ॒न्यत् ।  \newline




\markright{ TS 4.1.11.3  \hfill https://www.vedavms.in \hfill}

\section{ TS 4.1.11.3 }

\textbf{TS 4.1.11.3 } \newline
\textbf{Samhita Paata} \newline

-द्विषु॑रूपे॒ अह॑नी॒ द्यौरि॑वासि । विश्वा॒ हि मा॒या अव॑सि स्वधावो भ॒द्रा ते॑ पूषन्नि॒ह रा॒तिर॑स्तु ॥ ते॑ऽवर्द्धन्त॒ स्वत॑वसो महित्व॒ना ऽऽनाकं॑ त॒स्थुरु॒रु च॑क्रिरे॒ सदः॑ । विष्णु॒ र्यद्धाऽऽव॒द्-वृष॑णं मद॒च्युतं॒ ॅवयो॒ न सी॑द॒न्नधि॑ ब॒र्॒.हिषि॑ प्रि॒ये ॥ प्रचि॒त्रम॒र्कं गृ॑ण॒ते तु॒राय॒ मारु॑ताय॒ स्वत॑वसे भरद्ध्वं । ये सहाꣳ॑सि॒ सह॑सा॒ सह॑न्ते॒ - [  ] \newline

\textbf{Pada Paata} \newline

विषु॑रूपे॒ इति॒ विषु॑ - रू॒पे॒ । अह॑नी॒ इति॑ । द्यौः । इ॒व॒ । अ॒सि॒ ॥ विश्वाः᳚ । हि । मा॒याः । अव॑सि । स्व॒धा॒व॒ इति॑ स्वधा - वः॒ । भ॒द्रा । ते॒ । पू॒ष॒न्न् । इ॒ह । रा॒तिः । अ॒स्तु॒ ॥ ते । अ॒व॒द्‌र्ध॒न्त॒ । स्वत॑वस॒ इति॒ स्व - त॒व॒सः॒ । म॒हि॒त्व॒नेति॑ महि - त्व॒ना । एति॑ । नाक᳚म् । त॒स्थुः । उ॒रु । च॒क्रि॒रे॒ । सदः॑ ॥ विष्णुः॑ । यत् । ह॒ । आव॑त् । वृष॑णम् । म॒द॒च्युत॒मिति॑ मद - च्युत᳚म् । वयः॑ । न । सी॒द॒न्न् । अधीति॑ । ब॒र्॒.हिषि॑ । प्रि॒ये ॥ प्रेति॑ । चि॒त्रम् । अ॒र्कम् । गृ॒ण॒ते । तु॒राय॑ । मारु॑ताय । स्वत॑वस॒ इति॒ स्व - त॒व॒से॒ । भ॒र॒द्ध्व॒म् ॥ ये । सहाꣳ॑सि । सह॑सा । सह॑न्ते ।  \newline




\markright{ TS 4.1.11.4  \hfill https://www.vedavms.in \hfill}

\section{ TS 4.1.11.4 }

\textbf{TS 4.1.11.4 } \newline
\textbf{Samhita Paata} \newline

रेज॑ते अग्ने पृथि॒वी म॒खेभ्यः॑ ॥ विश्वे॑ दे॒वा >5, विश्वे॑ देवाः >6 ॥ द्यावा॑ नः पृथि॒वी इ॒मꣳ सि॒द्ध्रम॒द्य दि॑वि॒स्पृशं᳚ । य॒ज्ञ्ं दे॒वेषु॒ यच्छतां ॥ प्र पू᳚र्व॒जे पि॒तरा॒ नव्य॑सीभिर्गी॒र्भिः कृ॑णुद्ध्वꣳ॒॒ सद॑ने ऋ॒तस्य॑ । आ नो᳚ द्यावापृथिवी॒ दैव्ये॑न॒ जने॑न यातं॒ महि॑ वां॒ ॅवरू॑थं ॥ अ॒ग्निꣳ स्तोमे॑न बोधय समिधा॒नो अम॑र्त्यं । ह॒व्या दे॒वेषु॑ नो दधत् ॥ स ह॑व्य॒वाडम॑र्त्य उ॒शिग्दू॒तश्चनो॑हितः ( ) । अ॒ग्निर्द्धि॒या समृ॑ण्वति ॥ शन्नो॑ भवन्तु॒>7 , वाजे॑वाजे> 8 ॥ \newline

\textbf{Pada Paata} \newline

रेज॑ते । अ॒ग्ने॒ । पृ॒थि॒वी । म॒खेभ्यः॑ ॥ विश्वे᳚ । दे॒वाः । विश्वे᳚ । दे॒वाः॒ ॥ द्यावा᳚ । नः॒ । पृ॒थि॒वी इति॑ । इ॒मम् । सि॒द्ध्रम् । अ॒द्य । दि॒वि॒स्पृश॒मिति॑ दिवि - स्पृश᳚म् ॥ य॒ज्ञ्म् । दे॒वेषु॑ । य॒च्छ॒ता॒म् ॥ प्रेति॑ । पू॒र्व॒जे इति॑ पूर्व-जे । पि॒तरा᳚ । नव्य॑सीभिः । गी॒र्भिः । कृ॒णु॒द्ध्व॒म् । सद॑ने॒ इति॑ । ऋ॒तस्य॑ ॥ एति॑ । नः॒ । द्या॒वा॒पृ॒थि॒वी॒ इति॑ द्यावा-पृ॒थि॒वी । दैव्ये॑न । जने॑न । या॒त॒म् । महि॑ । वा॒म् । वरू॑थम् ॥ अ॒ग्निम् । स्तोमे॑न । बो॒ध॒य॒ । स॒मि॒धा॒न इति॑ सं - इ॒धा॒नः । अम॑र्त्यम् ॥ ह॒व्या । दे॒वेषु॑ । नः॒ । द॒ध॒त् ॥ सः । ह॒व्य॒वाडिति॑ हव्य - वाट् । अम॑र्त्यः । उ॒शिक् । दू॒तः । चनो॑हितः ( ) ॥ अ॒ग्निः । धि॒या । समिति॑ । ऋ॒ण्व॒ति॒ ॥ शम् । नः॒ । भ॒व॒न्तु॒ । वाजे॑वाज॒ इति॒ वाजे᳚ - वा॒जे॒ ॥  \newline






\end{document}