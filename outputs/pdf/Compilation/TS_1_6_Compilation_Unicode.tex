\documentclass[17pt]{extarticle}
\usepackage{babel}
\usepackage{fontspec}
\usepackage{polyglossia}
\usepackage{extsizes}

\usepackage{color}   %May be necessary if you want to color links
\usepackage{hyperref}
\hypersetup{
    colorlinks=true, %set true if you want colored links
    linktoc=all,     %set to all if you want both sections and subsections linked
    linkcolor=black,  %choose some color if you want links to stand out
}

\setmainlanguage{sanskrit}
\setotherlanguages{english} %% or other languages
\setlength{\parindent}{0pt}
\pagestyle{myheadings}
\newfontfamily\devanagarifont[Script=Devanagari]{AdishilaVedic}
\renewcommand{\theHsection}{\thepart.section.\thesection}

\newcommand{\VAR}[1]{}
\newcommand{\BLOCK}[1]{}




\begin{document}
\begin{titlepage}
    \begin{center}
 
\begin{sanskrit}
    { \Large
    कृष्ण यजुर्वेदीय तैत्तिरीय संहिता,पद,जटा,घन पाठः 
    }
    \\
    \vspace{2.5cm}
    \mbox{ \Large
    1.6     प्रथमकाण्डे षष्ठः प्रश्नः - (याजमानकाण्डं)   }
\end{sanskrit}
\end{center}

\end{titlepage}
\tableofcontents
\phantomsection
\pagebreak

\markright{ TS 1.6.1.1  \hfill https://www.vedavms.in \hfill}

\section{ TS 1.6.1.1 }

\textbf{TS 1.6.1.1 } \newline
\textbf{Samhita Paata} \newline

सं त्वा॑ सिञ्चामि॒ यजु॑षा प्र॒जामायु॒र्द्धनं॑ च । बृह॒स्पति॑प्रसूतो॒ यज॑मान इ॒ह मा रि॑षत् ॥ आज्य॑मसि स॒त्यम॑सि स॒त्यस्याद्ध्य॑क्षमसि ह॒विर॑सि वैश्वान॒रं ॅवै᳚श्वदे॒व-मुत्पू॑तशुष्मꣳ स॒त्यौजाः॒ सहो॑ऽसि॒ सह॑मानमसि॒ सह॒स्वारा॑तीः॒ सह॑स्वारातीय॒तः सह॑स्व॒ पृत॑नाः॒ सह॑स्व पृतन्य॒तः । स॒हस्र॑वीर्यमसि॒ तन्मा॑ जि॒न्वाज्य॒स्याज्य॑मसि स॒त्यस्य॑ स॒त्यम॑सि स॒त्यायु॑ - [ ] \newline

\textbf{Pada Paata} \newline

समिति॑ । त्वा॒ । सि॒ञ्चा॒मि॒ । यजु॑षा । प्र॒जामिति॑ प्र-जाम् । आयुः॑ । धन᳚म् । च॒ ॥ बृह॒स्पति॑ प्रसूत॒ इति॒ बृह॒स्पति॑ - प्र॒सू॒तः॒ । यज॑मानः । इ॒ह । मा । रि॒ष॒त् ॥ आज्य᳚म् । अ॒सि॒ । स॒त्यम् । अ॒सि॒ । स॒त्यस्य॑ । अद्ध्य॑क्ष॒मित्यधि॑ - अ॒क्ष॒म् । अ॒सि॒ । ह॒विः । अ॒सि॒ । वै॒श्वा॒न॒रम् । वै॒श्व॒दे॒वमिति॑ वैश्व - दे॒वम् । उत्पू॑तशुष्म॒मित्युत्पू॑त - शु॒ष्म॒म् । स॒त्यौजा॒ इति॑ स॒त्य-ओ॒जाः॒ । सहः॑ । अ॒सि॒ । सह॑मानम् । अ॒सि॒ । सह॑स्व । अरा॑तीः । सह॑स्व । अ॒रा॒ती॒य॒तः । सह॑स्व । पृत॑नाः । सह॑स्व । पृ॒त॒न्य॒तः । स॒हस्र॑वीर्य॒मिति॑ स॒हस्र॑-वि॒र्य॒म् । अ॒सि॒ । तत् । मा॒ । जि॒न्व॒ । आज्य॑स्य । आज्य᳚म् । अ॒सि॒ । स॒त्यस्य॑ । स॒त्यम् । अ॒सि॒ । स॒त्यायु॒रिति॑ स॒त्य - आ॒युः॒ ।  \newline


\textbf{Krama Paata} \newline

सम् त्वा᳚ । त्वा॒ सि॒ञ्चा॒मि॒ । सि॒ञ्चा॒मि॒ यजु॑षा । यजु॑षा प्र॒जाम् । प्र॒जामायुः॑ । प्र॒जामिति॑ प्र - जाम् । आयु॒र् धन᳚म् । धन॑म् च । चेति॑ च ॥ बृह॒स्पति॑प्रसूतो॒ यज॑मानः । बृह॒स्पति॑प्रसूत॒ इति॒ बृह॒स्पति॑ - प्र॒सू॒तः॒ । यज॑मान इ॒ह । इ॒ह मा । मा रि॑षत् । रि॒ष॒दिति॑ रिषत् ॥ आज्य॑मसि । अ॒सि॒ स॒त्यम् । स॒त्यम॑सि । अ॒सि॒ स॒त्यस्य॑ । स॒त्यस्याद्ध्य॑क्षम् । अद्ध्य॑क्षमसि । अद्ध्य॑क्ष॒मित्यधि॑ - अ॒क्ष॒म् । अ॒सि॒ ह॒विः । ह॒विर॑सि । अ॒सि॒ वै॒श्वा॒न॒रम् । वै॒श्वा॒न॒रं ॅवै᳚श्वदे॒वम् । वै॒श्व॒दे॒वमुत्पू॑तशुष्मम् । वै॒श्व॒दे॒वमिति॑ वैश्व - दे॒वम् । उत्पू॑तशुष्मꣳ स॒त्यौजाः᳚ । उत्पू॑तशुष्म॒मित्युत्पू॑त - शु॒ष्म॒म् । स॒त्यौजाः॒ सहः॑ । स॒त्यौजा॒ इति॑ स॒त्य - ओ॒जाः॒ । सहो॑ऽसि । अ॒सि॒ सह॑मानम् । सह॑मानमसि । अ॒सि॒ सह॑स्व । सह॒स्वारा॑तीः । अरा॑तीः॒ सह॑स्व । सह॑स्वारातीय॒तः । अ॒रा॒ती॒य॒तः सह॑स्व । सह॑स्व॒ पृत॑नाः । पृत॑नाः॒ सह॑स्व । सह॑स्व पृतन्य॒तः । पृ॒त॒न्य॒त इति॑ पृतन्य॒तः ॥ स॒हस्र॑वीर्यमसि । स॒हस्र॑वीर्य॒मिति॑ स॒हस्र॑ - वी॒र्य॒म् । अ॒सि॒ तत् । तन्मा᳚ । मा॒ जि॒न्व॒ । जि॒न्वाज्य॑स्य । आज्य॒स्याज्य᳚म् । आज्य॑मसि । अ॒सि॒ स॒त्यस्य॑ । स॒त्यस्य॑ स॒त्यम् । स॒त्यम॑सि । अ॒सि॒ स॒त्यायुः॑ । स॒त्यायु॑रसि । स॒त्यायु॒रिति॑ स॒त्य - आ॒युः॒ \newline

\textbf{Jatai Paata} \newline

1. सम् त्वा᳚ त्वा॒ सꣳ सम् त्वा᳚ । \newline
2. त्वा॒ सि॒ञ्चा॒मि॒ सि॒ञ्चा॒मि॒ त्वा॒ त्वा॒ सि॒ञ्चा॒मि॒ । \newline
3. सि॒ञ्चा॒मि॒ यजु॑षा॒ यजु॑षा सिञ्चामि सिञ्चामि॒ यजु॑षा । \newline
4. यजु॑षा प्र॒जाम् प्र॒जां ॅयजु॑षा॒ यजु॑षा प्र॒जाम् । \newline
5. प्र॒जा मायु॒रायुः॑ प्र॒जाम् प्र॒जा मायुः॑ । \newline
6. प्र॒जामिति॑ प्र - जाम् । \newline
7. आयु॒र् धन॒म् धन॒ मायु॒रायु॒र् धन᳚म् । \newline
8. धन॑म् च च॒ धन॒म् धन॑म् च । \newline
9. चेति॑ च । \newline
10. बृह॒स्पति॑प्रसूतो॒ यज॑मानो॒ यज॑मानो॒ बृह॒स्पति॑प्रसूतो॒ बृह॒स्पति॑प्रसूतो॒ यज॑मानः । \newline
11. बृह॒स्पति॑प्रसूत॒ इति॒ बृह॒स्पति॑ - प्र॒सू॒तः॒ । \newline
12. यज॑मान इ॒हे ह यज॑मानो॒ यज॑मान इ॒ह । \newline
13. इ॒ह मा मेहे ह मा । \newline
14. मा रि॑षद् रिष॒न् मा मा रि॑षत् । \newline
15. रि॒ष॒दिति॑ रिषत् । \newline
16. आज्य॑ मस्य॒ स्याज्य॒ माज्य॑ मसि । \newline
17. अ॒सि॒ स॒त्यꣳ स॒त्य म॑स्यसि स॒त्यम् । \newline
18. स॒त्य म॑स्यसि स॒त्यꣳ स॒त्य म॑सि । \newline
19. अ॒सि॒ स॒त्यस्य॑ स॒त्यस्या᳚ स्यसि स॒त्यस्य॑ । \newline
20. स॒त्यस्या द्ध्य॑क्ष॒ मद्ध्य॑क्षꣳ स॒त्यस्य॑ स॒त्यस्या द्ध्य॑क्षम् । \newline
21. अद्ध्य॑क्ष मस्य॒स्यद् ध्य॑क्ष॒ मद्ध्य॑क्ष मसि । \newline
22. अद्ध्य॑क्ष॒मित्यधि॑ - अ॒क्ष॒म् । \newline
23. अ॒सि॒ ह॒विर्. ह॒विर॑स्यसि ह॒विः । \newline
24. ह॒वि र॑स्यसि ह॒विर्. ह॒विर॑सि । \newline
25. अ॒सि॒ वै॒श्वा॒न॒रं ॅवै᳚श्वान॒र म॑स्यसि वैश्वान॒रम् । \newline
26. वै॒श्वा॒न॒रं ॅवै᳚श्वदे॒वं ॅवै᳚श्वदे॒वं ॅवै᳚श्वान॒रं ॅवै᳚श्वान॒रं ॅवै᳚श्वदे॒वम् । \newline
27. वै॒श्व॒दे॒व मुत्पू॑तशुष्म॒ मुत्पू॑तशुष्मं ॅवैश्वदे॒वं ॅवै᳚श्वदे॒व मुत्पू॑तशुष्मम् । \newline
28. वै॒श्व॒दे॒वमिति॑ वैश्व - दे॒वम् । \newline
29. उत्पू॑तशुष्मꣳ स॒त्यौजाः᳚ स॒त्यौजा॒ उत्पू॑तशुष्म॒ मुत्पू॑तशुष्मꣳ स॒त्यौजाः᳚ । \newline
30. उत्पू॑तशुष्म॒मित्युत्पू॑त - शु॒ष्म॒म् । \newline
31. स॒त्यौजाः॒ सहः॒ सहः॑ स॒त्यौजाः᳚ स॒त्यौजाः॒ सहः॑ । \newline
32. स॒त्यौजा॒ इति॑ स॒त्य - ओ॒जाः॒ । \newline
33. सहो᳚ ऽस्यसि॒ सहः॒ सहो॑ ऽसि । \newline
34. अ॒सि॒ सह॑मान॒(ग्म्॒) सह॑मान मस्यसि॒ सह॑मानम् । \newline
35. सह॑मान मस्यसि॒ सह॑मान॒(ग्म्॒) सह॑मान मसि । \newline
36. अ॒सि॒ सह॑स्व॒ सह॑स्वा स्यसि॒ सह॑स्व । \newline
37. सह॒स्वा रा॑ती॒ररा॑तीः॒ सह॑स्व॒ सह॒स्वारा॑तीः । \newline
38. अरा॑तीः॒ सह॑स्व॒ सह॒स्वा रा॑ती॒ररा॑तीः॒ सह॑स्व । \newline
39. सह॑स्वा रातीय॒तो॑ ऽरातीय॒तः सह॑स्व॒ सह॑स्वा रातीय॒तः । \newline
40. अ॒रा॒ती॒य॒तः सह॑स्व॒ सह॑स्वा रातीय॒तो॑ ऽरातीय॒तः सह॑स्व । \newline
41. सह॑स्व॒ पृत॑नाः॒ पृत॑नाः॒ सह॑स्व॒ सह॑स्व॒ पृत॑नाः । \newline
42. पृत॑नाः॒ सह॑स्व॒ सह॑स्व॒ पृत॑नाः॒ पृत॑नाः॒ सह॑स्व । \newline
43. सह॑स्व पृतन्य॒तः पृ॑तन्य॒तः सह॑स्व॒ सह॑स्व पृतन्य॒तः । \newline
44. पृ॒त॒न्य॒त इति॑ पृतन्य॒तः । \newline
45. स॒हस्र॑वीर्य मस्यसि स॒हस्र॑वीर्यꣳ स॒हस्र॑वीर्य मसि । \newline
46. स॒हस्र॑वीर्य॒मिति॑ स॒हस्र॑ - वी॒र्य॒म् । \newline
47. अ॒सि॒ तत् तद॑स्यसि॒ तत् । \newline
48. तन् मा॑ मा॒ तत् तन् मा᳚ । \newline
49. मा॒ जि॒न्व॒ जि॒न्व॒ मा॒ मा॒ जि॒न्व॒ । \newline
50. जि॒न्वाज्य॒ स्याज्य॑स्य जिन्व जि॒न्वाज्य॑स्य । \newline
51. आज्य॒स्याज्य॒ माज्य॒ माज्य॒ स्याज्य॒ स्याज्य᳚म् । \newline
52. आज्य॑ मस्य॒ स्याज्य॒ माज्य॑ मसि । \newline
53. अ॒सि॒ स॒त्यस्य॑ स॒त्यस्या᳚ स्यसि स॒त्यस्य॑ । \newline
54. स॒त्यस्य॑ स॒त्यꣳ स॒त्यꣳ स॒त्यस्य॑ स॒त्यस्य॑ स॒त्यम् । \newline
55. स॒त्य म॑स्यसि स॒त्यꣳ स॒त्य म॑सि । \newline
56. अ॒सि॒ स॒त्यायुः॑ स॒त्यायु॑ रस्यसि स॒त्यायुः॑ । \newline
57. स॒त्यायु॑ रस्यसि स॒त्यायुः॑ स॒त्यायु॑रसि । \newline
58. स॒त्यायु॒रिति॑ स॒त्य - आ॒युः॒ । \newline

\textbf{Ghana Paata } \newline

1. सम् त्वा᳚ त्वा॒ सꣳ सम् त्वा॑ सिञ्चामि सिञ्चामि त्वा॒ सꣳ सम् त्वा॑ सिञ्चामि । \newline
2. त्वा॒ सि॒ञ्चा॒मि॒ सि॒ञ्चा॒मि॒ त्वा॒ त्वा॒ सि॒ञ्चा॒मि॒ यजु॑षा॒ यजु॑षा सिञ्चामि त्वा त्वा सिञ्चामि॒ यजु॑षा । \newline
3. सि॒ञ्चा॒मि॒ यजु॑षा॒ यजु॑षा सिञ्चामि सिञ्चामि॒ यजु॑षा प्र॒जाम् प्र॒जां ॅयजु॑षा सिञ्चामि सिञ्चामि॒ यजु॑षा प्र॒जाम् । \newline
4. यजु॑षा प्र॒जाम् प्र॒जां ॅयजु॑षा॒ यजु॑षा प्र॒जा मायु॒रायुः॑ प्र॒जां ॅयजु॑षा॒ यजु॑षा प्र॒जा मायुः॑ । \newline
5. प्र॒जा मायु॒रायुः॑ प्र॒जाम् प्र॒जा मायु॒र् धन॒म् धन॒ मायुः॑ प्र॒जाम् प्र॒जा मायु॒र् धन᳚म् । \newline
6. प्र॒जामिति॑ प्र - जाम् । \newline
7. आयु॒र् धन॒म् धन॒ मायु॒रायु॒र् धन॑म् च च॒ धन॒ मायु॒रायु॒र् धन॑म् च । \newline
8. धन॑म् च च॒ धन॒म् धन॑म् च । \newline
9. चेति॑ च । \newline
10. बृह॒स्पति॑प्रसूतो॒ यज॑मानो॒ यज॑मानो॒ बृह॒स्पति॑प्रसूतो॒ बृह॒स्पति॑प्रसूतो॒ यज॑मान इ॒हे ह यज॑मानो॒ बृह॒स्पति॑प्रसूतो॒ बृह॒स्पति॑प्रसूतो॒ यज॑मान इ॒ह । \newline
11. बृह॒स्पति॑प्रसूत॒ इति॒ बृह॒स्पति॑ - प्र॒सू॒तः॒ । \newline
12. यज॑मान इ॒हे ह यज॑मानो॒ यज॑मान इ॒ह मा मेह यज॑मानो॒ यज॑मान इ॒ह मा । \newline
13. इ॒ह मा मेहे ह मा रि॑षद् रिष॒न् मेहे ह मा रि॑षत् । \newline
14. मा रि॑षद् रिष॒न् मा मा रि॑षत् । \newline
15. रि॒ष॒दिति॑ रिषत् । \newline
16. आज्य॑ मस्य॒स्याज्य॒ माज्य॑ मसि स॒त्यꣳ स॒त्य म॒स्याज्य॒ माज्य॑ मसि स॒त्यम् । \newline
17. अ॒सि॒ स॒त्यꣳ स॒त्य म॑स्यसि स॒त्य म॑स्यसि स॒त्य म॑स्यसि स॒त्य म॑सि । \newline
18. स॒त्य म॑स्यसि स॒त्यꣳ स॒त्य म॑सि स॒त्यस्य॑ स॒त्यस्या॑सि स॒त्यꣳ स॒त्य म॑सि स॒त्यस्य॑ । \newline
19. अ॒सि॒ स॒त्यस्य॑ स॒त्यस्या᳚स्यसि स॒त्यस्याद्ध्य॑क्ष॒ मद्ध्य॑क्षꣳ स॒त्यस्या᳚स्यसि स॒त्यस्याद्ध्य॑क्षम् । \newline
20. स॒त्यस्याद्ध्य॑क्ष॒ मद्ध्य॑क्षꣳ स॒त्यस्य॑ स॒त्यस्याद्ध्य॑क्ष मस्य॒स्यद्ध्य॑क्षꣳ स॒त्यस्य॑ स॒त्यस्याद्ध्य॑क्ष मसि । \newline
21. अद्ध्य॑क्ष मस्य॒स्यद्ध्य॑क्ष॒ मद्ध्य॑क्ष मसि ह॒विर्. ह॒विर॒स्यद्ध्य॑क्ष॒ मद्ध्य॑क्ष मसि ह॒विः । \newline
22. अद्ध्य॑क्ष॒मित्यधि॑ - अ॒क्ष॒म् । \newline
23. अ॒सि॒ ह॒विर्. ह॒विर॑स्यसि ह॒विर॑स्यसि ह॒विर॑स्यसि ह॒विर॑सि । \newline
24. ह॒विर॑स्यसि ह॒विर्. ह॒विर॑सि वैश्वान॒रं ॅवै᳚श्वान॒र म॑सि ह॒विर्. ह॒विर॑सि वैश्वान॒रम् । \newline
25. अ॒सि॒ वै॒श्वा॒न॒रं ॅवै᳚श्वान॒र म॑स्यसि वैश्वान॒रं ॅवै᳚श्वदे॒वं ॅवै᳚श्वदे॒वं ॅवै᳚श्वान॒र म॑स्यसि वैश्वान॒रं ॅवै᳚श्वदे॒वम् । \newline
26. वै॒श्वा॒न॒रं ॅवै᳚श्वदे॒वं ॅवै᳚श्वदे॒वं ॅवै᳚श्वान॒रं ॅवै᳚श्वान॒रं ॅवै᳚श्वदे॒व मुत्पू॑तशुष्म॒ मुत्पू॑तशुष्मं ॅवैश्वदे॒वं ॅवै᳚श्वान॒रं ॅवै᳚श्वान॒रं ॅवै᳚श्वदे॒व मुत्पू॑तशुष्मम् । \newline
27. वै॒श्व॒दे॒व मुत्पू॑तशुष्म॒ मुत्पू॑तशुष्मं ॅवैश्वदे॒वं ॅवै᳚श्वदे॒व मुत्पू॑तशुष्मꣳ स॒त्यौजाः᳚ स॒त्यौजा॒ उत्पू॑तशुष्मं ॅवैश्वदे॒वं ॅवै᳚श्वदे॒व मुत्पू॑तशुष्मꣳ स॒त्यौजाः᳚ । \newline
28. वै॒श्व॒दे॒वमिति॑ वैश्व - दे॒वम् । \newline
29. उत्पू॑तशुष्मꣳ स॒त्यौजाः᳚ स॒त्यौजा॒ उत्पू॑तशुष्म॒ मुत्पू॑तशुष्मꣳ स॒त्यौजाः॒ सहः॒ सहः॑ स॒त्यौजा॒ उत्पू॑तशुष्म॒ मुत्पू॑तशुष्मꣳ स॒त्यौजाः॒ सहः॑ । \newline
30. उत्पू॑तशुष्म॒मित्युत्पू॑त - शु॒ष्म॒म् । \newline
31. स॒त्यौजाः॒ सहः॒ सहः॑ स॒त्यौजाः᳚ स॒त्यौजाः॒ सहो᳚ ऽस्यसि॒ सहः॑ स॒त्यौजाः᳚ स॒त्यौजाः॒ सहो॑ ऽसि । \newline
32. स॒त्यौजा॒ इति॑ स॒त्य - ओ॒जाः॒ । \newline
33. सहो᳚ ऽस्यसि॒ सहः॒ सहो॑ ऽसि॒ सह॑मान॒(ग्म्॒) सह॑मान मसि॒ सहः॒ सहो॑ ऽसि॒ सह॑मानम् । \newline
34. अ॒सि॒ सह॑मान॒(ग्म्॒) सह॑मान मस्यसि॒ सह॑मान मस्यसि॒ सह॑मान मस्यसि॒ सह॑मान मसि । \newline
35. सह॑मान मस्यसि॒ सह॑मान॒(ग्म्॒) सह॑मान मसि॒ सह॑स्व॒ सह॑स्वासि॒ सह॑मान॒(ग्म्॒) सह॑मान मसि॒ सह॑स्व । \newline
36. अ॒सि॒ सह॑स्व॒ सह॑स्वास्यसि॒ सह॒स्वा रा॑ती॒ररा॑तीः॒ सह॑स्वास्यसि॒ सह॒स्वारा॑तीः । \newline
37. सह॒स्वारा॑ती॒ररा॑तीः॒ सह॑स्व॒ सह॒स्वारा॑तीः॒ सह॑स्व॒ सह॒स्वारा॑तीः॒ सह॑स्व॒ सह॒स्वारा॑तीः॒ सह॑स्व । \newline
38. अरा॑तीः॒ सह॑स्व॒ सह॒स्वारा॑ती॒ररा॑तीः॒ सह॑स्वारातीय॒तो॑ ऽरातीय॒तः सह॒स्वारा॑ती॒ररा॑तीः॒ सह॑स्वारातीय॒तः । \newline
39. सह॑स्वारातीय॒तो॑ ऽरातीय॒तः सह॑स्व॒ सह॑स्वारातीय॒तः सह॑स्व॒ सह॑स्वारातीय॒तः सह॑स्व॒ सह॑स्वारातीय॒तः सह॑स्व । \newline
40. अ॒रा॒ती॒य॒तः सह॑स्व॒ सह॑स्वारातीय॒तो॑ ऽरातीय॒तः सह॑स्व॒ पृत॑नाः॒ पृत॑नाः॒ सह॑स्वारातीय॒तो॑ ऽरातीय॒तः सह॑स्व॒ पृत॑नाः । \newline
41. सह॑स्व॒ पृत॑नाः॒ पृत॑नाः॒ सह॑स्व॒ सह॑स्व॒ पृत॑नाः॒ सह॑स्व॒ सह॑स्व॒ पृत॑नाः॒ सह॑स्व॒ सह॑स्व॒ पृत॑नाः॒ सह॑स्व । \newline
42. पृत॑नाः॒ सह॑स्व॒ सह॑स्व॒ पृत॑नाः॒ पृत॑नाः॒ सह॑स्व पृतन्य॒तः पृ॑तन्य॒तः सह॑स्व॒ पृत॑नाः॒ पृत॑नाः॒ सह॑स्व पृतन्य॒तः । \newline
43. सह॑स्व पृतन्य॒तः पृ॑तन्य॒तः सह॑स्व॒ सह॑स्व पृतन्य॒तः । \newline
44. पृ॒त॒न्य॒त इति॑ पृतन्य॒तः । \newline
45. स॒हस्र॑वीर्य मस्यसि स॒हस्र॑वीर्यꣳ स॒हस्र॑वीर्य मसि॒ तत् तद॑सि स॒हस्र॑वीर्यꣳ स॒हस्र॑वीर्य मसि॒ तत् । \newline
46. स॒हस्र॑वीर्य॒मिति॑ स॒हस्र॑ - वी॒र्य॒म् । \newline
47. अ॒सि॒ तत् तद॑स्यसि॒ तन् मा॑ मा॒ तद॑स्यसि॒ तन् मा᳚ । \newline
48. तन् मा॑ मा॒ तत् तन् मा॑ जिन्व जिन्व मा॒ तत् तन् मा॑ जिन्व । \newline
49. मा॒ जि॒न्व॒ जि॒न्व॒ मा॒ मा॒ जि॒न्वाज्य॒स्याज्य॑स्य जिन्व मा मा जि॒न्वाज्य॑स्य । \newline
50. जि॒न्वाज्य॒स्याज्य॑स्य जिन्व जि॒न्वाज्य॒स्याज्य॒ माज्य॒ माज्य॑स्य जिन्व जि॒न्वाज्य॒स्याज्य᳚म् । \newline
51. आज्य॒स्याज्य॒ माज्य॒ माज्य॒स्याज्य॒स्याज्य॑ मस्य॒स्याज्य॒ माज्य॒स्याज्य॒स्याज्य॑ मसि । \newline
52. आज्य॑ मस्य॒स्याज्य॒ माज्य॑ मसि स॒त्यस्य॑ स॒त्यस्या॒स्याज्य॒ माज्य॑ मसि स॒त्यस्य॑ । \newline
53. अ॒सि॒ स॒त्यस्य॑ स॒त्यस्या᳚स्यसि स॒त्यस्य॑ स॒त्यꣳ स॒त्यꣳ स॒त्यस्या᳚स्यसि स॒त्यस्य॑ स॒त्यम् । \newline
54. स॒त्यस्य॑ स॒त्यꣳ स॒त्यꣳ स॒त्यस्य॑ स॒त्यस्य॑ स॒त्य म॑स्यसि स॒त्यꣳ स॒त्यस्य॑ स॒त्यस्य॑ स॒त्य म॑सि । \newline
55. स॒त्य म॑स्यसि स॒त्यꣳ स॒त्य म॑सि स॒त्यायुः॑ स॒त्यायु॑रसि स॒त्यꣳ स॒त्य म॑सि स॒त्यायुः॑ । \newline
56. अ॒सि॒ स॒त्यायुः॑ स॒त्यायु॑रस्यसि स॒त्यायु॑रस्यसि स॒त्यायु॑रस्यसि स॒त्यायु॑रसि । \newline
57. स॒त्यायु॑रस्यसि स॒त्यायुः॑ स॒त्यायु॑रसि स॒त्यशु॑ष्मꣳ स॒त्यशु॑ष्म मसि स॒त्यायुः॑ स॒त्यायु॑रसि स॒त्यशु॑ष्मम् । \newline
58. स॒त्यायु॒रिति॑ स॒त्य - आ॒युः॒ । \newline
\pagebreak
\markright{ TS 1.6.1.2  \hfill https://www.vedavms.in \hfill}

\section{ TS 1.6.1.2 }

\textbf{TS 1.6.1.2 } \newline
\textbf{Samhita Paata} \newline

रसि स॒त्यशु॑ष्ममसि स॒त्येन॑ त्वा॒ऽभि घा॑रयामि॒ तस्य॑ ते भक्षीय पञ्चा॒नां त्वा॒ वाता॑नां ॅय॒न्त्राय॑ ध॒र्त्राय॑ गृह्णामि पञ्चा॒नां त्व॑र्तू॒नां ॅय॒न्त्राय॑ ध॒र्त्राय॑ गृह्णामि पञ्चा॒नां त्वा॑ दि॒शां ॅय॒न्त्राय॑ ध॒र्त्राय॑ गृह्णामि पञ्चा॒नां त्वा॑ पञ्चज॒नानां᳚ ॅय॒न्त्राय॑ ध॒र्त्राय॑ गृह्णामि च॒रोस्त्वा॒ पञ्च॑बिलस्य य॒न्त्राय॑ ध॒र्त्राय॑ गृह्णामि॒ ब्रह्म॑णस्त्वा॒ तेज॑से य॒न्त्राय॑ ध॒र्त्राय॑ गृह्णामि क्ष॒त्रस्य॒ त्वौज॑से य॒न्त्राय॑ - [ ] \newline

\textbf{Pada Paata} \newline

अ॒सि॒ । स॒त्यशु॑ष्म॒मिति॑ स॒त्य - शु॒ष्म॒म् । अ॒सि॒ । स॒त्येन॑ । त्वा॒ । अ॒भीति॑ । घा॒र॒या॒मि॒ । तस्य॑ । ते॒ । भ॒क्षी॒य॒ । प॒ञ्चा॒नाम् । त्वा॒ । वाता॑नाम् । य॒न्त्राय॑ । ध॒र्त्राय॑ । गृ॒ह्णा॒मि॒ । प॒ञ्चा॒नाम् । त्वा॒ । ऋ॒तू॒नाम् । य॒न्त्राय॑ । ध॒र्त्राय॑ । गृ॒ह्णा॒मि॒ । प॒ञ्चा॒नाम् । त्वा॒ । दि॒शाम् । य॒न्त्राय॑ । ध॒र्त्राय॑ । गृ॒ह्णा॒मि॒ । प॒ञ्चा॒नाम् । त्वा॒ । प॒ञ्च॒ज॒नाना॒मिति॑ पञ्च - ज॒नाना᳚म् । य॒न्त्राय॑ । ध॒र्त्राय॑ । गृ॒ह्णा॒मि॒ । च॒रोः । त्वा॒ । पञ्च॑बिल॒स्येति॒ पञ्च॑ - बि॒ल॒स्य॒ । य॒न्त्राय॑ । ध॒र्त्राय॑ । गृ॒ह्णा॒मि॒ । ब्रह्म॑णः । त्वा॒ । तेज॑से । य॒न्त्राय॑ । ध॒र्त्राय॑ । गृ॒ह्णा॒मि॒ । क्ष॒त्रस्य॑ । त्वा॒ । ओज॑से । य॒न्त्राय॑ ।  \newline


\textbf{Krama Paata} \newline

अ॒सि॒ स॒त्यशु॑ष्मम् । स॒त्यशु॑ष्ममसि । स॒त्यशु॑ष्म॒मिति॑ स॒त्य - शु॒ष्म॒म् । अ॒सि॒ स॒त्येन॑ । स॒त्येन॑ त्वा । त्वा॒ऽभि । अ॒भि घा॑रयामि । घा॒र॒या॒मि॒ तस्य॑ । तस्य॑ ते । ते॒ भ॒क्षी॒य॒ । भ॒क्षी॒य॒ प॒ञ्चा॒नाम् । प॒ञ्चा॒नाम् त्वा᳚ । त्वा॒ वाता॑नाम् । वाता॑नां ॅय॒न्त्राय॑ । य॒न्त्राय॑ ध॒र्त्राय॑ । ध॒र्त्राय॑ गृह्णामि । गृ॒ह्णा॒मि॒ प॒ञ्चा॒नाम् । प॒ञ्चा॒नाम् त्वा᳚ । त्व॒र्तू॒नाम् । ऋ॒तू॒नां ॅय॒न्त्राय॑ । य॒न्त्राय॑ ध॒र्त्राय॑ । ध॒र्त्राय॑ गृह्णामि । गृ॒ह्णा॒मि॒ प॒ञ्चा॒नाम् । प॒ञ्चा॒नाम् त्वा᳚ । त्वा॒ दि॒शाम् । दि॒शां ॅय॒न्त्राय॑ । य॒न्त्राय॑ ध॒र्त्राय॑ । ध॒र्त्राय॑ गृह्णामि । गृ॒ह्णा॒मि॒ प॒ञ्चा॒नाम् । प॒ञ्चा॒नाम् त्वा᳚ । त्वा॒ प॒ञ्च॒ज॒नाना᳚म् । प॒ञ्च॒ज॒नानां᳚ ॅय॒न्त्राय॑ । प॒ञ्च॒ज॒नाना॒मिति॑ पञ्च - ज॒नाना᳚म् । य॒न्त्राय॑ ध॒र्त्राय॑ । ध॒र्त्राय॑ गृह्णामि । गृ॒ह्णा॒मि॒ च॒रोः । च॒रोस्त्वा᳚ । त्वा॒ पञ्च॑बिलस्य । पञ्च॑बिलस्य य॒न्त्राय॑ । पञ्च॑बिल॒स्येति॒ पञ्च॑ - बि॒ल॒स्य॒ । य॒न्त्राय॑ ध॒र्त्राय॑ । ध॒र्त्राय॑ गृह्णामि । गृ॒ह्णा॒मि॒ ब्रह्म॑णः । ब्रह्म॑णस्त्वा । त्वा॒ तेज॑से । तेज॑से य॒न्त्राय॑ । य॒न्त्राय॑ ध॒र्त्राय॑ । ध॒र्त्राय॑ गृह्णामि । गृ॒ह्णा॒मि॒ क्ष॒त्रस्य॑ । क्ष॒त्रस्य॑ त्वा । त्वौज॑से । ओज॑से य॒न्त्राय॑ ( ) । य॒न्त्राय॑ ध॒र्त्राय॑ \newline

\textbf{Jatai Paata} \newline

1. अ॒सि॒ स॒त्यशु॑ष्मꣳ स॒त्यशु॑ष्म मस्यसि स॒त्यशु॑ष्मम् । \newline
2. स॒त्यशु॑ष्म मस्यसि स॒त्यशु॑ष्मꣳ स॒त्यशु॑ष्म मसि । \newline
3. स॒त्यशु॑ष्म॒मिति॑ स॒त्य - शु॒ष्म॒म् । \newline
4. अ॒सि॒ स॒त्येन॑ स॒त्येना᳚स्यसि स॒त्येन॑ । \newline
5. स॒त्येन॑ त्वा त्वा स॒त्येन॑ स॒त्येन॑ त्वा । \newline
6. त्वा॒ ऽभ्य॑भि त्वा᳚ त्वा॒ ऽभि । \newline
7. अ॒भि घा॑रयामि घारया म्य॒भ्य॑भि घा॑रयामि । \newline
8. घा॒र॒या॒मि॒ तस्य॒ तस्य॑ घारयामि घारयामि॒ तस्य॑ । \newline
9. तस्य॑ ते ते॒ तस्य॒ तस्य॑ ते । \newline
10. ते॒ भ॒क्षी॒य॒ भ॒क्षी॒य॒ ते॒ ते॒ भ॒क्षी॒य॒ । \newline
11. भ॒क्षी॒य॒ प॒ञ्चा॒नाम् प॑ञ्चा॒नाम् भ॑क्षीय भक्षीय पञ्चा॒नाम् । \newline
12. प॒ञ्चा॒नाम् त्वा᳚ त्वा पञ्चा॒नाम् प॑ञ्चा॒नाम् त्वा᳚ । \newline
13. त्वा॒ वाता॑नां॒ ॅवाता॑नाम् त्वा त्वा॒ वाता॑नाम् । \newline
14. वाता॑नां ॅय॒न्त्राय॑ य॒न्त्राय॒ वाता॑नां॒ ॅवाता॑नां ॅय॒न्त्राय॑ । \newline
15. य॒न्त्राय॑ ध॒र्त्राय॑ ध॒र्त्राय॑ य॒न्त्राय॑ य॒न्त्राय॑ ध॒र्त्राय॑ । \newline
16. ध॒र्त्राय॑ गृह्णामि गृह्णामि ध॒र्त्राय॑ ध॒र्त्राय॑ गृह्णामि । \newline
17. गृ॒ह्णा॒मि॒ प॒ञ्चा॒नाम् प॑ञ्चा॒नाम् गृ॑ह्णामि गृह्णामि पञ्चा॒नाम् । \newline
18. प॒ञ्चा॒नाम् त्वा᳚ त्वा पञ्चा॒नाम् प॑ञ्चा॒नाम् त्वा᳚ । \newline
19. त्व॒र्तू॒ना मृ॑तू॒नाम् त्वा᳚ त्वर्तू॒नाम् । \newline
20. ऋ॒तू॒नां ॅय॒न्त्राय॑ य॒न्त्राय॑ र्तू॒ना मृ॑तू॒नां ॅय॒न्त्राय॑ । \newline
21. य॒न्त्राय॑ ध॒र्त्राय॑ ध॒र्त्राय॑ य॒न्त्राय॑ य॒न्त्राय॑ ध॒र्त्राय॑ । \newline
22. ध॒र्त्राय॑ गृह्णामि गृह्णामि ध॒र्त्राय॑ ध॒र्त्राय॑ गृह्णामि । \newline
23. गृ॒ह्णा॒मि॒ प॒ञ्चा॒नाम् प॑ञ्चा॒नाम् गृ॑ह्णामि गृह्णामि पञ्चा॒नाम् । \newline
24. प॒ञ्चा॒नाम् त्वा᳚ त्वा पञ्चा॒नाम् प॑ञ्चा॒नाम् त्वा᳚ । \newline
25. त्वा॒ दि॒शाम् दि॒शाम् त्वा᳚ त्वा दि॒शाम् । \newline
26. दि॒शां ॅय॒न्त्राय॑ य॒न्त्राय॑ दि॒शाम् दि॒शां ॅय॒न्त्राय॑ । \newline
27. य॒न्त्राय॑ ध॒र्त्राय॑ ध॒र्त्राय॑ य॒न्त्राय॑ य॒न्त्राय॑ ध॒र्त्राय॑ । \newline
28. ध॒र्त्राय॑ गृह्णामि गृह्णामि ध॒र्त्राय॑ ध॒र्त्राय॑ गृह्णामि । \newline
29. गृ॒ह्णा॒मि॒ प॒ञ्चा॒नाम् प॑ञ्चा॒नाम् गृ॑ह्णामि गृह्णामि पञ्चा॒नाम् । \newline
30. प॒ञ्चा॒नाम् त्वा᳚ त्वा पञ्चा॒नाम् प॑ञ्चा॒नाम् त्वा᳚ । \newline
31. त्वा॒ प॒ञ्च॒ज॒नाना᳚म् पञ्चज॒नाना᳚म् त्वा त्वा पञ्चज॒नाना᳚म् । \newline
32. प॒ञ्च॒ज॒नानां᳚ ॅय॒न्त्राय॑ य॒न्त्राय॑ पञ्चज॒नाना᳚म् पञ्चज॒नानां᳚ ॅय॒न्त्राय॑ । \newline
33. प॒ञ्च॒ज॒नाना॒मिति॑ पञ्च - ज॒नाना᳚म् । \newline
34. य॒न्त्राय॑ ध॒र्त्राय॑ ध॒र्त्राय॑ य॒न्त्राय॑ य॒न्त्राय॑ ध॒र्त्राय॑ । \newline
35. ध॒र्त्राय॑ गृह्णामि गृह्णामि ध॒र्त्राय॑ ध॒र्त्राय॑ गृह्णामि । \newline
36. गृ॒ह्णा॒मि॒ च॒रो श्च॒रोर् गृ॑ह्णामि गृह्णामि च॒रोः । \newline
37. च॒रो स्त्वा᳚ त्वा च॒रो श्च॒रो स्त्वा᳚ । \newline
38. त्वा॒ पञ्च॑बिलस्य॒ पञ्च॑बिलस्य त्वा त्वा॒ पञ्च॑बिलस्य । \newline
39. पञ्च॑बिलस्य य॒न्त्राय॑ य॒न्त्राय॒ पञ्च॑बिलस्य॒ पञ्च॑बिलस्य य॒न्त्राय॑ । \newline
40. पञ्च॑बिल॒स्येति॒ पञ्च॑ - बि॒ल॒स्य॒ । \newline
41. य॒न्त्राय॑ ध॒र्त्राय॑ ध॒र्त्राय॑ य॒न्त्राय॑ य॒न्त्राय॑ ध॒र्त्राय॑ । \newline
42. ध॒र्त्राय॑ गृह्णामि गृह्णामि ध॒र्त्राय॑ ध॒र्त्राय॑ गृह्णामि । \newline
43. गृ॒ह्णा॒मि॒ ब्रह्म॑णो॒ ब्रह्म॑णो गृह्णामि गृह्णामि॒ ब्रह्म॑णः । \newline
44. ब्रह्म॑ण स्त्वा त्वा॒ ब्रह्म॑णो॒ ब्रह्म॑ण स्त्वा । \newline
45. त्वा॒ तेज॑से॒ तेज॑से त्वा त्वा॒ तेज॑से । \newline
46. तेज॑से य॒न्त्राय॑ य॒न्त्राय॒ तेज॑से॒ तेज॑से य॒न्त्राय॑ । \newline
47. य॒न्त्राय॑ ध॒र्त्राय॑ ध॒र्त्राय॑ य॒न्त्राय॑ य॒न्त्राय॑ ध॒र्त्राय॑ । \newline
48. ध॒र्त्राय॑ गृह्णामि गृह्णामि ध॒र्त्राय॑ ध॒र्त्राय॑ गृह्णामि । \newline
49. गृ॒ह्णा॒मि॒ क्ष॒त्रस्य॑ क्ष॒त्रस्य॑ गृह्णामि गृह्णामि क्ष॒त्रस्य॑ । \newline
50. क्ष॒त्रस्य॑ त्वा त्वा क्ष॒त्रस्य॑ क्ष॒त्रस्य॑ त्वा । \newline
51. त्वौज॑स॒ ओज॑से त्वा॒ त्वौज॑से । \newline
52. ओज॑से य॒न्त्राय॑ य॒न्त्रा यौज॑स॒ ओज॑से य॒न्त्राय॑ । \newline
53. य॒न्त्राय॑ ध॒र्त्राय॑ ध॒र्त्राय॑ य॒न्त्राय॑ य॒न्त्राय॑ ध॒र्त्राय॑ । \newline

\textbf{Ghana Paata } \newline

1. अ॒सि॒ स॒त्यशु॑ष्मꣳ स॒त्यशु॑ष्म मस्यसि स॒त्यशु॑ष्म मस्यसि स॒त्यशु॑ष्म मस्यसि स॒त्यशु॑ष्म मसि । \newline
2. स॒त्यशु॑ष्म मस्यसि स॒त्यशु॑ष्मꣳ स॒त्यशु॑ष्म मसि स॒त्येन॑ स॒त्येना॑सि स॒त्यशु॑ष्मꣳ स॒त्यशु॑ष्म मसि स॒त्येन॑ । \newline
3. स॒त्यशु॑ष्म॒मिति॑ स॒त्य - शु॒ष्म॒म् । \newline
4. अ॒सि॒ स॒त्येन॑ स॒त्येना᳚स्यसि स॒त्येन॑ त्वा त्वा स॒त्येना᳚स्यसि स॒त्येन॑ त्वा । \newline
5. स॒त्येन॑ त्वा त्वा स॒त्येन॑ स॒त्येन॑ त्वा॒ ऽभ्य॑भि त्वा॑ स॒त्येन॑ स॒त्येन॑ त्वा॒ ऽभि । \newline
6. त्वा॒ ऽभ्य॑भि त्वा᳚ त्वा॒ ऽभि घा॑रयामि घारयाम्य॒भि त्वा᳚ त्वा॒ ऽभि घा॑रयामि । \newline
7. अ॒भि घा॑रयामि घारयाम्य॒भ्य॑भि घा॑रयामि॒ तस्य॒ तस्य॑ घारयाम्य॒भ्य॑भि घा॑रयामि॒ तस्य॑ । \newline
8. घा॒र॒या॒मि॒ तस्य॒ तस्य॑ घारयामि घारयामि॒ तस्य॑ ते ते॒ तस्य॑ घारयामि घारयामि॒ तस्य॑ ते । \newline
9. तस्य॑ ते ते॒ तस्य॒ तस्य॑ ते भक्षीय भक्षीय ते॒ तस्य॒ तस्य॑ ते भक्षीय । \newline
10. ते॒ भ॒क्षी॒य॒ भ॒क्षी॒य॒ ते॒ ते॒ भ॒क्षी॒य॒ प॒ञ्चा॒नाम् प॑ञ्चा॒नाम् भ॑क्षीय ते ते भक्षीय पञ्चा॒नाम् । \newline
11. भ॒क्षी॒य॒ प॒ञ्चा॒नाम् प॑ञ्चा॒नाम् भ॑क्षीय भक्षीय पञ्चा॒नाम् त्वा᳚ त्वा पञ्चा॒नाम् भ॑क्षीय भक्षीय पञ्चा॒नाम् त्वा᳚ । \newline
12. प॒ञ्चा॒नाम् त्वा᳚ त्वा पञ्चा॒नाम् प॑ञ्चा॒नाम् त्वा॒ वाता॑नां॒ ॅवाता॑नाम् त्वा पञ्चा॒नाम् प॑ञ्चा॒नाम् त्वा॒ वाता॑नाम् । \newline
13. त्वा॒ वाता॑नां॒ ॅवाता॑नाम् त्वा त्वा॒ वाता॑नां ॅय॒न्त्राय॑ य॒न्त्राय॒ वाता॑नाम् त्वा त्वा॒ वाता॑नां ॅय॒न्त्राय॑ । \newline
14. वाता॑नां ॅय॒न्त्राय॑ य॒न्त्राय॒ वाता॑नां॒ ॅवाता॑नां ॅय॒न्त्राय॑ ध॒र्त्राय॑ ध॒र्त्राय॑ य॒न्त्राय॒ वाता॑नां॒ ॅवाता॑नां ॅय॒न्त्राय॑ ध॒र्त्राय॑ । \newline
15. य॒न्त्राय॑ ध॒र्त्राय॑ ध॒र्त्राय॑ य॒न्त्राय॑ य॒न्त्राय॑ ध॒र्त्राय॑ गृह्णामि गृह्णामि ध॒र्त्राय॑ य॒न्त्राय॑ य॒न्त्राय॑ ध॒र्त्राय॑ गृह्णामि । \newline
16. ध॒र्त्राय॑ गृह्णामि गृह्णामि ध॒र्त्राय॑ ध॒र्त्राय॑ गृह्णामि पञ्चा॒नाम् प॑ञ्चा॒नाम् गृ॑ह्णामि ध॒र्त्राय॑ ध॒र्त्राय॑ गृह्णामि पञ्चा॒नाम् । \newline
17. गृ॒ह्णा॒मि॒ प॒ञ्चा॒नाम् प॑ञ्चा॒नाम् गृ॑ह्णामि गृह्णामि पञ्चा॒नाम् त्वा᳚ त्वा पञ्चा॒नाम् गृ॑ह्णामि गृह्णामि पञ्चा॒नाम् त्वा᳚ । \newline
18. प॒ञ्चा॒नाम् त्वा᳚ त्वा पञ्चा॒नाम् प॑ञ्चा॒नाम् त्व॑र्तू॒ना मृ॑तू॒नाम् त्वा॑ पञ्चा॒नाम् प॑ञ्चा॒नाम् त्व॑र्तू॒नाम् । \newline
19. त्व॒र्तू॒ना मृ॑तू॒नाम् त्वा᳚ त्वर्तू॒नां ॅय॒न्त्राय॑ य॒न्त्राय॑ र्तू॒नाम् त्वा᳚ त्वर्तू॒नां ॅय॒न्त्राय॑ । \newline
20. ऋ॒तू॒नां ॅय॒न्त्राय॑ य॒न्त्राय॑ र्तू॒ना मृ॑तू॒नां ॅय॒न्त्राय॑ ध॒र्त्राय॑ ध॒र्त्राय॑ य॒न्त्राय॑ र्तू॒ना मृ॑तू॒नां ॅय॒न्त्राय॑ ध॒र्त्राय॑ । \newline
21. य॒न्त्राय॑ ध॒र्त्राय॑ ध॒र्त्राय॑ य॒न्त्राय॑ य॒न्त्राय॑ ध॒र्त्राय॑ गृह्णामि गृह्णामि ध॒र्त्राय॑ य॒न्त्राय॑ य॒न्त्राय॑ ध॒र्त्राय॑ गृह्णामि । \newline
22. ध॒र्त्राय॑ गृह्णामि गृह्णामि ध॒र्त्राय॑ ध॒र्त्राय॑ गृह्णामि पञ्चा॒नाम् प॑ञ्चा॒नाम् गृ॑ह्णामि ध॒र्त्राय॑ ध॒र्त्राय॑ गृह्णामि पञ्चा॒नाम् । \newline
23. गृ॒ह्णा॒मि॒ प॒ञ्चा॒नाम् प॑ञ्चा॒नाम् गृ॑ह्णामि गृह्णामि पञ्चा॒नाम् त्वा᳚ त्वा पञ्चा॒नाम् गृ॑ह्णामि गृह्णामि पञ्चा॒नाम् त्वा᳚ । \newline
24. प॒ञ्चा॒नाम् त्वा᳚ त्वा पञ्चा॒नाम् प॑ञ्चा॒नाम् त्वा॑ दि॒शाम् दि॒शाम् त्वा॑ पञ्चा॒नाम् प॑ञ्चा॒नाम् त्वा॑ दि॒शाम् । \newline
25. त्वा॒ दि॒शाम् दि॒शाम् त्वा᳚ त्वा दि॒शां ॅय॒न्त्राय॑ य॒न्त्राय॑ दि॒शाम् त्वा᳚ त्वा दि॒शां ॅय॒न्त्राय॑ । \newline
26. दि॒शां ॅय॒न्त्राय॑ य॒न्त्राय॑ दि॒शाम् दि॒शां ॅय॒न्त्राय॑ ध॒र्त्राय॑ ध॒र्त्राय॑ य॒न्त्राय॑ दि॒शाम् दि॒शां ॅय॒न्त्राय॑ ध॒र्त्राय॑ । \newline
27. य॒न्त्राय॑ ध॒र्त्राय॑ ध॒र्त्राय॑ य॒न्त्राय॑ य॒न्त्राय॑ ध॒र्त्राय॑ गृह्णामि गृह्णामि ध॒र्त्राय॑ य॒न्त्राय॑ य॒न्त्राय॑ ध॒र्त्राय॑ गृह्णामि । \newline
28. ध॒र्त्राय॑ गृह्णामि गृह्णामि ध॒र्त्राय॑ ध॒र्त्राय॑ गृह्णामि पञ्चा॒नाम् प॑ञ्चा॒नाम् गृ॑ह्णामि ध॒र्त्राय॑ ध॒र्त्राय॑ गृह्णामि पञ्चा॒नाम् । \newline
29. गृ॒ह्णा॒मि॒ प॒ञ्चा॒नाम् प॑ञ्चा॒नाम् गृ॑ह्णामि गृह्णामि पञ्चा॒नाम् त्वा᳚ त्वा पञ्चा॒नाम् गृ॑ह्णामि गृह्णामि पञ्चा॒नाम् त्वा᳚ । \newline
30. प॒ञ्चा॒नाम् त्वा᳚ त्वा पञ्चा॒नाम् प॑ञ्चा॒नाम् त्वा॑ पञ्चज॒नाना᳚म् पञ्चज॒नाना᳚म् त्वा पञ्चा॒नाम् प॑ञ्चा॒नाम् त्वा॑ पञ्चज॒नाना᳚म् । \newline
31. त्वा॒ प॒ञ्च॒ज॒नाना᳚म् पञ्चज॒नाना᳚म् त्वा त्वा पञ्चज॒नानां᳚ ॅय॒न्त्राय॑ य॒न्त्राय॑ पञ्चज॒नाना᳚म् त्वा त्वा पञ्चज॒नानां᳚ ॅय॒न्त्राय॑ । \newline
32. प॒ञ्च॒ज॒नानां᳚ ॅय॒न्त्राय॑ य॒न्त्राय॑ पञ्चज॒नाना᳚म् पञ्चज॒नानां᳚ ॅय॒न्त्राय॑ ध॒र्त्राय॑ ध॒र्त्राय॑ य॒न्त्राय॑ पञ्चज॒नाना᳚म् पञ्चज॒नानां᳚ ॅय॒न्त्राय॑ ध॒र्त्राय॑ । \newline
33. प॒ञ्च॒ज॒नाना॒मिति॑ पञ्च - ज॒नाना᳚म् । \newline
34. य॒न्त्राय॑ ध॒र्त्राय॑ ध॒र्त्राय॑ य॒न्त्राय॑ य॒न्त्राय॑ ध॒र्त्राय॑ गृह्णामि गृह्णामि ध॒र्त्राय॑ य॒न्त्राय॑ य॒न्त्राय॑ ध॒र्त्राय॑ गृह्णामि । \newline
35. ध॒र्त्राय॑ गृह्णामि गृह्णामि ध॒र्त्राय॑ ध॒र्त्राय॑ गृह्णामि च॒रो श्च॒रोर् गृ॑ह्णामि ध॒र्त्राय॑ ध॒र्त्राय॑ गृह्णामि च॒रोः । \newline
36. गृ॒ह्णा॒मि॒ च॒रो श्च॒रोर् गृ॑ह्णामि गृह्णामि च॒रोस्त्वा᳚ त्वा च॒रोर् गृ॑ह्णामि गृह्णामि च॒रोस्त्वा᳚ । \newline
37. च॒रोस्त्वा᳚ त्वा च॒रोश्च॒रोस्त्वा॒ पञ्च॑बिलस्य॒ पञ्च॑बिलस्य त्वा च॒रोश्च॒रोस्त्वा॒ पञ्च॑बिलय । \newline
38. त्वा॒ पञ्च॑बिलस्य॒ पञ्च॑बिलस्य त्वा त्वा॒ पञ्च॑बिलस्य य॒न्त्राय॑ य॒न्त्राय॒ पञ्च॑बिलस्य त्वा त्वा॒ पञ्च॑बिलस्य य॒न्त्राय॑ । \newline
39. पञ्च॑बिलस्य य॒न्त्राय॑ य॒न्त्राय॒ पञ्च॑बिलस्य॒ पञ्च॑बिलस्य य॒न्त्राय॑ ध॒र्त्राय॑ ध॒र्त्राय॑ य॒न्त्राय॒ पञ्च॑बिलस्य॒ पञ्च॑बिलस्य य॒न्त्राय॑ ध॒र्त्राय॑ । \newline
40. पञ्च॑बिल॒स्येति॒ पञ्च॑ - बि॒ल॒स्य॒ । \newline
41. य॒न्त्राय॑ ध॒र्त्राय॑ ध॒र्त्राय॑ य॒न्त्राय॑ य॒न्त्राय॑ ध॒र्त्राय॑ गृह्णामि गृह्णामि ध॒र्त्राय॑ य॒न्त्राय॑ य॒न्त्राय॑ ध॒र्त्राय॑ गृह्णामि । \newline
42. ध॒र्त्राय॑ गृह्णामि गृह्णामि ध॒र्त्राय॑ ध॒र्त्राय॑ गृह्णामि॒ ब्रह्म॑णो॒ ब्रह्म॑णो गृह्णामि ध॒र्त्राय॑ ध॒र्त्राय॑ गृह्णामि॒ ब्रह्म॑णः । \newline
43. गृ॒ह्णा॒मि॒ ब्रह्म॑णो॒ ब्रह्म॑णो गृह्णामि गृह्णामि॒ ब्रह्म॑णस्त्वा त्वा॒ ब्रह्म॑णो गृह्णामि गृह्णामि॒ ब्रह्म॑णस्त्वा । \newline
44. ब्रह्म॑णस्त्वा त्वा॒ ब्रह्म॑णो॒ ब्रह्म॑णस्त्वा॒ तेज॑से॒ तेज॑से त्वा॒ ब्रह्म॑णो॒ ब्रह्म॑णस्त्वा॒ तेज॑से । \newline
45. त्वा॒ तेज॑से॒ तेज॑से त्वा त्वा॒ तेज॑से य॒न्त्राय॑ य॒न्त्राय॒ तेज॑से त्वा त्वा॒ तेज॑से य॒न्त्राय॑ । \newline
46. तेज॑से य॒न्त्राय॑ य॒न्त्राय॒ तेज॑से॒ तेज॑से य॒न्त्राय॑ ध॒र्त्राय॑ ध॒र्त्राय॑ य॒न्त्राय॒ तेज॑से॒ तेज॑से य॒न्त्राय॑ ध॒र्त्राय॑ । \newline
47. य॒न्त्राय॑ ध॒र्त्राय॑ ध॒र्त्राय॑ य॒न्त्राय॑ य॒न्त्राय॑ ध॒र्त्राय॑ गृह्णामि गृह्णामि ध॒र्त्राय॑ य॒न्त्राय॑ य॒न्त्राय॑ ध॒र्त्राय॑ गृह्णामि । \newline
48. ध॒र्त्राय॑ गृह्णामि गृह्णामि ध॒र्त्राय॑ ध॒र्त्राय॑ गृह्णामि क्ष॒त्रस्य॑ क्ष॒त्रस्य॑ गृह्णामि ध॒र्त्राय॑ ध॒र्त्राय॑ गृह्णामि क्ष॒त्रस्य॑ । \newline
49. गृ॒ह्णा॒मि॒ क्ष॒त्रस्य॑ क्ष॒त्रस्य॑ गृह्णामि गृह्णामि क्ष॒त्रस्य॑ त्वा त्वा क्ष॒त्रस्य॑ गृह्णामि गृह्णामि क्ष॒त्रस्य॑ त्वा । \newline
50. क्ष॒त्रस्य॑ त्वा त्वा क्ष॒त्रस्य॑ क्ष॒त्रस्य॒ त्वौज॑स॒ ओज॑से त्वा क्ष॒त्रस्य॑ क्ष॒त्रस्य॒ त्वौज॑से । \newline
51. त्वौज॑स॒ ओज॑से त्वा॒ त्वौज॑से य॒न्त्राय॑ य॒न्त्रायौज॑से त्वा॒ त्वौज॑से य॒न्त्राय॑ । \newline
52. ओज॑से य॒न्त्राय॑ य॒न्त्रायौज॑स॒ ओज॑से य॒न्त्राय॑ ध॒र्त्राय॑ ध॒र्त्राय॑ य॒न्त्रायौज॑स॒ ओज॑से य॒न्त्राय॑ ध॒र्त्राय॑ । \newline
53. य॒न्त्राय॑ ध॒र्त्राय॑ ध॒र्त्राय॑ य॒न्त्राय॑ य॒न्त्राय॑ ध॒र्त्राय॑ गृह्णामि गृह्णामि ध॒र्त्राय॑ य॒न्त्राय॑ य॒न्त्राय॑ ध॒र्त्राय॑ गृह्णामि । \newline
\pagebreak
\markright{ TS 1.6.1.3  \hfill https://www.vedavms.in \hfill}

\section{ TS 1.6.1.3 }

\textbf{TS 1.6.1.3 } \newline
\textbf{Samhita Paata} \newline

ध॒र्त्राय॑ गृह्णामि वि॒शे त्वा॑ य॒न्त्राय॑ ध॒र्त्राय॑ गृह्णामि सु॒वीर्या॑य त्वा गृह्णामि सुप्रजा॒स्त्वाय॑ त्वा गृह्णामि रा॒यस्पोषा॑य त्वा गृह्णामि ब्रह्मवर्च॒साय॑ त्वा गृह्णामि॒ भूर॒स्माकꣳ॑ ह॒विर्दे॒वाना॑-मा॒शिषो॒ यज॑मानस्य दे॒वानां᳚ त्वा दे॒वता᳚भ्यो गृह्णामि॒ कामा॑य त्वा गृह्णामि ॥ \newline

\textbf{Pada Paata} \newline

ध॒र्त्राय॑ । गृ॒ह्णा॒मि॒ । वि॒शे । त्वा॒ । य॒न्त्राय॑ । ध॒र्त्राय॑ । गृ॒ह्णा॒मि॒ । सु॒वीर्या॒येति॑ सु - वीर्या॑य । त्वा॒ । गृ॒ह्णा॒मि॒ । सु॒प्र॒जा॒स्त्वायेति॑ सुप्रजाः - त्वाय॑ । त्वा॒ । गृ॒ह्णा॒मि॒ । रा॒यः । पोषा॑य । त्वा॒ । गृ॒ह्णा॒मि॒ । ब्र॒ह्म॒व॒र्च॒सायेति॑ ब्रह्म-व॒र्च॒साय॑ । त्वा॒ । गृ॒ह्णा॒मि॒ । भूः । अ॒स्माक᳚म् । ह॒विः । दे॒वाना᳚म् । आ॒शिष॒ इत्या᳚-शिषः॑ । यज॑मानस्य । दे॒वाना᳚म् । त्वा॒ । दे॒वता᳚भ्यः । गृ॒ह्णा॒मि॒ । कामा॑य । त्वा॒ । गृ॒ह्णा॒मि॒ ॥  \newline


\textbf{Krama Paata} \newline

ध॒र्त्राय॑ गृह्णामि । गृ॒ह्णा॒मि॒ वि॒शे । वि॒शे त्वा᳚ । त्वा॒ य॒न्त्राय॑ । य॒न्त्राय॑ ध॒र्त्राय॑ । ध॒र्त्राय॑ गृह्णामि । गृ॒ह्णा॒मि॒ सु॒वीर्या॑य । सु॒वीर्या॑य त्वा । सु॒वीर्या॒येति॑ सु - वीर्या॑य । त्वा॒ गृ॒ह्णा॒मि॒ । गृ॒ह्णा॒मि॒ सु॒प्र॒जा॒स्त्वाय॑ । सु॒प्र॒जा॒स्त्वाय॑ त्वा । सु॒प्र॒जा॒स्त्वायेति॑ सुप्रजाः - त्वाय॑ । त्वा॒ गृ॒ह्णा॒मि॒ । गृ॒ह्णा॒मि॒ रा॒यः । रा॒यस्पोषा॑य । पोषा॑य त्वा । त्वा॒ गृ॒ह्णा॒मि॒ । गृ॒ह्णा॒मि॒ ब्र॒ह्म॒व॒र्च॒साय॑ । ब्र॒ह्म॒व॒र्च॒साय॑ त्वा । ब्र॒ह्म॒व॒र्च॒सायेति॑ ब्रह्म - व॒र्च॒साय॑ । त्वा॒ गृ॒ह्णा॒मि॒ । गृ॒ह्णा॒मि॒ भूः । भूर॒स्माक᳚म् । अ॒स्माकꣳ॑ ह॒विः । ह॒विर्,दे॒वाना᳚म् । दे॒वाना॑मा॒शिषः॑ । आ॒शिषो॒ यज॑मानस्य । आ॒शिष॒ इत्या᳚ - शिषः॑ । यज॑मानस्य दे॒वाना᳚म् । दे॒वाना᳚म् त्वा । त्वा॒ दे॒वता᳚भ्यः । दे॒वता᳚भ्यो गृह्णामि । गृ॒ह्णा॒मि॒ कामा॑य । कामा॑य त्वा । त्वा॒ गृ॒ह्णा॒मि॒ । गृ॒ह्णा॒मीति॑ गृह्णामि । \newline

\textbf{Jatai Paata} \newline

1. ध॒र्त्राय॑ गृह्णामि गृह्णामि ध॒र्त्राय॑ ध॒र्त्राय॑ गृह्णामि । \newline
2. गृ॒ह्णा॒मि॒ वि॒शे वि॒शे गृ॑ह्णामि गृह्णामि वि॒शे । \newline
3. वि॒शे त्वा᳚ त्वा वि॒शे वि॒शे त्वा᳚ । \newline
4. त्वा॒ य॒न्त्राय॑ य॒न्त्राय॑ त्वा त्वा य॒न्त्राय॑ । \newline
5. य॒न्त्राय॑ ध॒र्त्राय॑ ध॒र्त्राय॑ य॒न्त्राय॑ य॒न्त्राय॑ ध॒र्त्राय॑ । \newline
6. ध॒र्त्राय॑ गृह्णामि गृह्णामि ध॒र्त्राय॑ ध॒र्त्राय॑ गृह्णामि । \newline
7. गृ॒ह्णा॒मि॒ सु॒वीर्या॑य सु॒वीर्या॑य गृह्णामि गृह्णामि सु॒वीर्या॑य । \newline
8. सु॒वीर्या॑य त्वा त्वा सु॒वीर्या॑य सु॒वीर्या॑य त्वा । \newline
9. सु॒वीर्या॒येति॑ सु - वीर्या॑य । \newline
10. त्वा॒ गृ॒ह्णा॒मि॒ गृ॒ह्णा॒मि॒ त्वा॒ त्वा॒ गृ॒ह्णा॒मि॒ । \newline
11. गृ॒ह्णा॒मि॒ सु॒प्र॒जा॒स्त्वाय॑ सुप्रजा॒स्त्वाय॑ गृह्णामि गृह्णामि सुप्रजा॒स्त्वाय॑ । \newline
12. सु॒प्र॒जा॒स्त्वाय॑ त्वा त्वा सुप्रजा॒स्त्वाय॑ सुप्रजा॒स्त्वाय॑ त्वा । \newline
13. सु॒प्र॒जा॒स्त्वायेति॑ सुप्रजाः - त्वाय॑ । \newline
14. त्वा॒ गृ॒ह्णा॒मि॒ गृ॒ह्णा॒मि॒ त्वा॒ त्वा॒ गृ॒ह्णा॒मि॒ । \newline
15. गृ॒ह्णा॒मि॒ रा॒यो रा॒यो गृ॑ह्णामि गृह्णामि रा॒यः । \newline
16. रा॒य स्पोषा॑य॒ पोषा॑य रा॒यो रा॒य स्पोषा॑य । \newline
17. पोषा॑य त्वा त्वा॒ पोषा॑य॒ पोषा॑य त्वा । \newline
18. त्वा॒ गृ॒ह्णा॒मि॒ गृ॒ह्णा॒मि॒ त्वा॒ त्वा॒ गृ॒ह्णा॒मि॒ । \newline
19. गृ॒ह्णा॒मि॒ ब्र॒ह्म॒व॒र्च॒साय॑ ब्रह्मवर्च॒साय॑ गृह्णामि गृह्णामि ब्रह्मवर्च॒साय॑ । \newline
20. ब्र॒ह्म॒व॒र्च॒साय॑ त्वा त्वा ब्रह्मवर्च॒साय॑ ब्रह्मवर्च॒साय॑ त्वा । \newline
21. ब्र॒ह्म॒व॒र्च॒सायेति॑ ब्रह्म - व॒र्च॒साय॑ । \newline
22. त्वा॒ गृ॒ह्णा॒मि॒ गृ॒ह्णा॒मि॒ त्वा॒ त्वा॒ गृ॒ह्णा॒मि॒ । \newline
23. गृ॒ह्णा॒मि॒ भूर् भूर् गृ॑ह्णामि गृह्णामि॒ भूः । \newline
24. भू र॒स्माक॑ म॒स्माक॒म् भूर् भू र॒स्माक᳚म् । \newline
25. अ॒स्माक(ग्म्॑) ह॒विर्. ह॒वि र॒स्माक॑ म॒स्माक(ग्म्॑) ह॒विः । \newline
26. ह॒विर् दे॒वाना᳚म् दे॒वाना(ग्म्॑) ह॒विर्. ह॒विर् दे॒वाना᳚म् । \newline
27. दे॒वाना॑ मा॒शिष॑ आ॒शिषो॑ दे॒वाना᳚म् दे॒वाना॑ मा॒शिषः॑ । \newline
28. आ॒शिषो॒ यज॑मानस्य॒ यज॑मा नस्या॒शिष॑ आ॒शिषो॒ यज॑मानस्य । \newline
29. आ॒शिष॒ इत्या᳚ - शिषः॑ । \newline
30. यज॑मानस्य दे॒वाना᳚म् दे॒वानां॒ ॅयज॑मानस्य॒ यज॑मानस्य दे॒वाना᳚म् । \newline
31. दे॒वाना᳚म् त्वा त्वा दे॒वाना᳚म् दे॒वाना᳚म् त्वा । \newline
32. त्वा॒ दे॒वता᳚भ्यो दे॒वता᳚भ्य स्त्वा त्वा दे॒वता᳚भ्यः । \newline
33. दे॒वता᳚भ्यो गृह्णामि गृह्णामि दे॒वता᳚भ्यो दे॒वता᳚भ्यो गृह्णामि । \newline
34. गृ॒ह्णा॒मि॒ कामा॑य॒ कामा॑य गृह्णामि गृह्णामि॒ कामा॑य । \newline
35. कामा॑य त्वा त्वा॒ कामा॑य॒ कामा॑य त्वा । \newline
36. त्वा॒ गृ॒ह्णा॒मि॒ गृ॒ह्णा॒मि॒ त्वा॒ त्वा॒ गृ॒ह्णा॒मि॒ । \newline
37. गृ॒ह्णा॒मीति॑ गृह्णामि । \newline

\textbf{Ghana Paata } \newline

1. ध॒र्त्राय॑ गृह्णामि गृह्णामि ध॒र्त्राय॑ ध॒र्त्राय॑ गृह्णामि वि॒शे वि॒शे गृ॑ह्णामि ध॒र्त्राय॑ ध॒र्त्राय॑ गृह्णामि वि॒शे । \newline
2. गृ॒ह्णा॒मि॒ वि॒शे वि॒शे गृ॑ह्णामि गृह्णामि वि॒शे त्वा᳚ त्वा वि॒शे गृ॑ह्णामि गृह्णामि वि॒शे त्वा᳚ । \newline
3. वि॒शे त्वा᳚ त्वा वि॒शे वि॒शे त्वा॑ य॒न्त्राय॑ य॒न्त्राय॑ त्वा वि॒शे वि॒शे त्वा॑ य॒न्त्राय॑ । \newline
4. त्वा॒ य॒न्त्राय॑ य॒न्त्राय॑ त्वा त्वा य॒न्त्राय॑ ध॒र्त्राय॑ ध॒र्त्राय॑ य॒न्त्राय॑ त्वा त्वा य॒न्त्राय॑ ध॒र्त्राय॑ । \newline
5. य॒न्त्राय॑ ध॒र्त्राय॑ ध॒र्त्राय॑ य॒न्त्राय॑ य॒न्त्राय॑ ध॒र्त्राय॑ गृह्णामि गृह्णामि ध॒र्त्राय॑ य॒न्त्राय॑ य॒न्त्राय॑ ध॒र्त्राय॑ गृह्णामि । \newline
6. ध॒र्त्राय॑ गृह्णामि गृह्णामि ध॒र्त्राय॑ ध॒र्त्राय॑ गृह्णामि सु॒वीर्या॑य सु॒वीर्या॑य गृह्णामि ध॒र्त्राय॑ ध॒र्त्राय॑ गृह्णामि सु॒वीर्या॑य । \newline
7. गृ॒ह्णा॒मि॒ सु॒वीर्या॑य सु॒वीर्या॑य गृह्णामि गृह्णामि सु॒वीर्या॑य त्वा त्वा सु॒वीर्या॑य गृह्णामि गृह्णामि सु॒वीर्या॑य त्वा । \newline
8. सु॒वीर्या॑य त्वा त्वा सु॒वीर्या॑य सु॒वीर्या॑य त्वा गृह्णामि गृह्णामि त्वा सु॒वीर्या॑य सु॒वीर्या॑य त्वा गृह्णामि । \newline
9. सु॒वीर्या॒येति॑ सु - वीर्या॑य । \newline
10. त्वा॒ गृ॒ह्णा॒मि॒ गृ॒ह्णा॒मि॒ त्वा॒ त्वा॒ गृ॒ह्णा॒मि॒ सु॒प्र॒जा॒स्त्वाय॑ सुप्रजा॒स्त्वाय॑ गृह्णामि त्वा त्वा गृह्णामि सुप्रजा॒स्त्वाय॑ । \newline
11. गृ॒ह्णा॒मि॒ सु॒प्र॒जा॒स्त्वाय॑ सुप्रजा॒स्त्वाय॑ गृह्णामि गृह्णामि सुप्रजा॒स्त्वाय॑ त्वा त्वा सुप्रजा॒स्त्वाय॑ गृह्णामि गृह्णामि सुप्रजा॒स्त्वाय॑ त्वा । \newline
12. सु॒प्र॒जा॒स्त्वाय॑ त्वा त्वा सुप्रजा॒स्त्वाय॑ सुप्रजा॒स्त्वाय॑ त्वा गृह्णामि गृह्णामि त्वा सुप्रजा॒स्त्वाय॑ सुप्रजा॒स्त्वाय॑ त्वा गृह्णामि । \newline
13. सु॒प्र॒जा॒स्त्वायेति॑ सुप्रजाः - त्वाय॑ । \newline
14. त्वा॒ गृ॒ह्णा॒मि॒ गृ॒ह्णा॒मि॒ त्वा॒ त्वा॒ गृ॒ह्णा॒मि॒ रा॒यो रा॒यो गृ॑ह्णामि त्वा त्वा गृह्णामि रा॒यः । \newline
15. गृ॒ह्णा॒मि॒ रा॒यो रा॒यो गृ॑ह्णामि गृह्णामि रा॒य स्पोषा॑य॒ पोषा॑य रा॒यो गृ॑ह्णामि गृह्णामि रा॒य स्पोषा॑य । \newline
16. रा॒य स्पोषा॑य॒ पोषा॑य रा॒यो रा॒य स्पोषा॑य त्वा त्वा॒ पोषा॑य रा॒यो रा॒य स्पोषा॑य त्वा । \newline
17. पोषा॑य त्वा त्वा॒ पोषा॑य॒ पोषा॑य त्वा गृह्णामि गृह्णामि त्वा॒ पोषा॑य॒ पोषा॑य त्वा गृह्णामि । \newline
18. त्वा॒ गृ॒ह्णा॒मि॒ गृ॒ह्णा॒मि॒ त्वा॒ त्वा॒ गृ॒ह्णा॒मि॒ ब्र॒ह्म॒व॒र्च॒साय॑ ब्रह्मवर्च॒साय॑ गृह्णामि त्वा त्वा गृह्णामि ब्रह्मवर्च॒साय॑ । \newline
19. गृ॒ह्णा॒मि॒ ब्र॒ह्म॒व॒र्च॒साय॑ ब्रह्मवर्च॒साय॑ गृह्णामि गृह्णामि ब्रह्मवर्च॒साय॑ त्वा त्वा ब्रह्मवर्च॒साय॑ गृह्णामि गृह्णामि ब्रह्मवर्च॒साय॑ त्वा । \newline
20. ब्र॒ह्म॒व॒र्च॒साय॑ त्वा त्वा ब्रह्मवर्च॒साय॑ ब्रह्मवर्च॒साय॑ त्वा गृह्णामि गृह्णामि त्वा ब्रह्मवर्च॒साय॑ ब्रह्मवर्च॒साय॑ त्वा गृह्णामि । \newline
21. ब्र॒ह्म॒व॒र्च॒सायेति॑ ब्रह्म - व॒र्च॒साय॑ । \newline
22. त्वा॒ गृ॒ह्णा॒मि॒ गृ॒ह्णा॒मि॒ त्वा॒ त्वा॒ गृ॒ह्णा॒मि॒ भूर् भूर् गृ॑ह्णामि त्वा त्वा गृह्णामि॒ भूः । \newline
23. गृ॒ह्णा॒मि॒ भूर् भूर् गृ॑ह्णामि गृह्णामि॒ भूर॒स्माक॑ म॒स्माक॒म् भूर् गृ॑ह्णामि गृह्णामि॒ भूर॒स्माक᳚म् । \newline
24. भूर॒स्माक॑ म॒स्माक॒म् भूर् भूर॒स्माक(ग्म्॑) ह॒विर्. ह॒विर॒स्माक॒म् भूर् भूर॒स्माक(ग्म्॑) ह॒विः । \newline
25. अ॒स्माक(ग्म्॑) ह॒विर्. ह॒विर॒स्माक॑ म॒स्माक(ग्म्॑) ह॒विर् दे॒वाना᳚म् दे॒वाना(ग्म्॑) ह॒विर॒स्माक॑ म॒स्माक(ग्म्॑) ह॒विर् दे॒वाना᳚म् । \newline
26. ह॒विर् दे॒वाना᳚म् दे॒वाना(ग्म्॑) ह॒विर्. ह॒विर् दे॒वाना॑ मा॒शिष॑ आ॒शिषो॑ दे॒वाना(ग्म्॑) ह॒विर्. ह॒विर् दे॒वाना॑ मा॒शिषः॑ । \newline
27. दे॒वाना॑ मा॒शिष॑ आ॒शिषो॑ दे॒वाना᳚म् दे॒वाना॑ मा॒शिषो॒ यज॑मानस्य॒ यज॑मानस्या॒शिषो॑ दे॒वाना᳚म् दे॒वाना॑ मा॒शिषो॒ यज॑मानस्य । \newline
28. आ॒शिषो॒ यज॑मानस्य॒ यज॑मानस्या॒शिष॑ आ॒शिषो॒ यज॑मानस्य दे॒वाना᳚म् दे॒वानां॒ ॅयज॑मानस्या॒शिष॑ आ॒शिषो॒ यज॑मानस्य दे॒वाना᳚म् । \newline
29. आ॒शिष॒ इत्या᳚ - शिषः॑ । \newline
30. यज॑मानस्य दे॒वाना᳚म् दे॒वानां॒ ॅयज॑मानस्य॒ यज॑मानस्य दे॒वाना᳚म् त्वा त्वा दे॒वानां॒ ॅयज॑मानस्य॒ यज॑मानस्य दे॒वाना᳚म् त्वा । \newline
31. दे॒वाना᳚म् त्वा त्वा दे॒वाना᳚म् दे॒वाना᳚म् त्वा दे॒वता᳚भ्यो दे॒वता᳚भ्यस्त्वा दे॒वाना᳚म् दे॒वाना᳚म् त्वा दे॒वता᳚भ्यः । \newline
32. त्वा॒ दे॒वता᳚भ्यो दे॒वता᳚भ्यस्त्वा त्वा दे॒वता᳚भ्यो गृह्णामि गृह्णामि दे॒वता᳚भ्यस्त्वा त्वा दे॒वता᳚भ्यो गृह्णामि । \newline
33. दे॒वता᳚भ्यो गृह्णामि गृह्णामि दे॒वता᳚भ्यो दे॒वता᳚भ्यो गृह्णामि॒ कामा॑य॒ कामा॑य गृह्णामि दे॒वता᳚भ्यो दे॒वता᳚भ्यो गृह्णामि॒ कामा॑य । \newline
34. गृ॒ह्णा॒मि॒ कामा॑य॒ कामा॑य गृह्णामि गृह्णामि॒ कामा॑य त्वा त्वा॒ कामा॑य गृह्णामि गृह्णामि॒ कामा॑य त्वा । \newline
35. कामा॑य त्वा त्वा॒ कामा॑य॒ कामा॑य त्वा गृह्णामि गृह्णामि त्वा॒ कामा॑य॒ कामा॑य त्वा गृह्णामि । \newline
36. त्वा॒ गृ॒ह्णा॒मि॒ गृ॒ह्णा॒मि॒ त्वा॒ त्वा॒ गृ॒ह्णा॒मि॒ । \newline
37. गृ॒ह्णा॒मीति॑ गृह्णामि । \newline
\pagebreak
\markright{ TS 1.6.2.1  \hfill https://www.vedavms.in \hfill}

\section{ TS 1.6.2.1 }

\textbf{TS 1.6.2.1 } \newline
\textbf{Samhita Paata} \newline

ध्रु॒वो॑ऽसि ध्रु॒वो॑ऽहꣳ स॑जा॒तेषु॑ भूयासं॒ धीर॒श्चेत्ता॑ वसु॒विदु॒ग्रो᳚ऽस्यु॒ग्रो॑ऽहꣳ स॑जा॒तेषु॑ भूयास-मु॒ग्रश्चेत्ता॑ वसु॒विद॑भि॒-भूर॑स्यभि॒भूर॒हꣳ स॑जा॒तेषु॑ भूयासमभि॒भूश्चेत्ता॑ वसु॒विद् यु॒॒नज्मि॑ त्वा॒ ब्रह्म॑णा॒ दैव्ये॑न ह॒व्याया॒स्मै वोढ॒वे जा॑तवेदः ॥ इन्धा॑नास्त्वा सुप्र॒जसः॑ सु॒वीरा॒ ज्योग्जी॑वेम बलि॒हृतो॑ व॒यं ते᳚ ॥ यन्मे॑ अग्ने अ॒स्य य॒ज्ञ्स्य॒ रिष्या॒ - [ ] \newline

\textbf{Pada Paata} \newline

ध्रु॒वः । अ॒सि॒ । ध्रु॒वः । अ॒हम् । स॒जा॒तेष्विति॑ स - जा॒तेषु॑ । भू॒या॒स॒म् । धीरः॑ । चेत्ता᳚ । व॒सु॒विदिति॑ वसु - वित् । उ॒ग्रः । अ॒सि॒ । उ॒ग्रः । अ॒हम् । स॒जा॒तेष्विति॑ स-जा॒तेषु॑ । भू॒या॒स॒म् । उ॒ग्रः । चेत्ता᳚ । व॒सु॒विदिति॑ वसु - वित् । अ॒भि॒भूरित्य॑भि - भूः । अ॒सि॒ । अ॒भि॒भूरित्य॑भि - भूः । अ॒हम् । स॒जा॒तेष्विति॑ स - जा॒तेषु॑ । भू॒या॒स॒म् । अ॒भि॒भूरित्य॑भि - भूः । चेत्ता᳚ । व॒सु॒विदिति॑ वसु-वित् । यु॒नज्मि॑ । त्वा॒ । ब्रह्म॑णा । दैव्ये॑न । ह॒व्याय॑ । अ॒स्मै । वो॒ढ॒वे । जा॒त॒वे॒द॒ इति॑ जात - वे॒दः॒ ॥ इन्धा॑नाः । त्वा॒ । सु॒प्र॒जस॒ इति॑ सु - प्र॒जसः॑ । सु॒वीरा॒ इति॑ सु - वीराः᳚ । ज्योक् । जी॒वे॒म॒ । ब॒लि॒हृत॒ इति॑ बलि - हृतः॑ । व॒यम् । ते॒ ॥ यत् । मे॒ । अ॒ग्ने॒ । अ॒स्य । य॒ज्ञ्स्य॑ । रिष्या᳚त् ।  \newline


\textbf{Krama Paata} \newline

ध्रु॒वो॑ऽसि । अ॒सि॒ ध्रु॒वः । ध्रु॒वो॑ऽहम् । अ॒हꣳ स॑जा॒तेषु॑ । स॒जा॒तेषु॑ भूयासम् । स॒जा॒तेष्विति॑ स - जा॒तेषु॑ । भू॒या॒स॒म् धीरः॑ । धीर॒श्चेत्ता᳚ । चेत्ता॑ वसु॒वित् । व॒सु॒विदु॒ग्रः । व॒सु॒विदिति॑ वसु - वित् । उ॒ग्रो॑ऽसि । अ॒स्यु॒ग्रः । उ॒ग्रो॑ऽहम् । अ॒हꣳ स॑जा॒तेषु॑ । स॒जा॒तेषु॑ भूयासम् । स॒जा॒तेष्विति॑ स - जा॒तेषु॑ । भू॒या॒स॒मु॒ग्रः । उ॒ग्रश्चेत्ता᳚ । चेत्ता॑ वसु॒वित् । व॒सु॒विद॑भि॒भूः । व॒सु॒विदिति॑ वसु - वित् । अ॒भि॒भूर॑सि । अ॒भि॒भूरित्य॑भि - भूः । अ॒स्य॒भि॒भूः । अ॒भि॒भूर॒हम् । अ॒भि॒भूरित्य॑भि - भूः । अ॒हꣳ स॑जा॒तेषु॑ । स॒जा॒तेषु॑ भूयासम् । स॒जा॒तेष्विति॑ स - जा॒तेषु॑ । भू॒या॒स॒म॒भि॒भूः । अ॒भि॒भूश्चेत्ता᳚ । अ॒भि॒भूरित्य॑भि - भूः । चेत्ता॑ वसु॒वित् । व॒सु॒विद् यु॒नज्मि॑ । व॒सु॒विदिति॑ वसु - वित् । यु॒नज्मि॑ त्वा । त्वा॒ ब्रह्म॑णा । ब्रह्म॑णा॒ दैव्ये॑न । दैव्ये॑न ह॒व्याय॑ । ह॒व्याया॒स्मै । अ॒स्मै वोढ॒वे । वोढ॒वे जा॑तवेदः । जा॒त॒वे॒द॒ इति॑ जात - वे॒दः॒ ॥ इन्धा॑नास्त्वा । त्वा॒ सु॒प्र॒जसः॑ । सु॒प्र॒जसः॑ सु॒वीराः᳚ । सु॒प्र॒जस॒ इति॑ सु - प्र॒जसः॑ । सु॒वीरा॒ ज्योक् । सु॒वीरा॒ इति॑ सु - वीराः᳚ । ज्योग् जी॑वेम । जी॒वे॒म॒ ब॒लि॒हृतः॑ । ब॒लि॒हृतो॑ व॒यम् । ब॒लि॒हृत॒ इति॑ बलि - हृतः॑ । व॒यम् ते᳚ । त॒ इति॑ ते ॥ यन्मे᳚ । मे॒ अ॒ग्ने॒ । अ॒ग्ने॒ अ॒स्य । अ॒स्य य॒ज्ञ्स्य॑ । य॒ज्ञ्स्य॒ रिष्या᳚त् । रिष्या॒द् यत् \newline

\textbf{Jatai Paata} \newline

1. ध्रु॒वो᳚ ऽस्यसि ध्रु॒वो ध्रु॒वो॑ ऽसि । \newline
2. अ॒सि॒ ध्रु॒वो ध्रु॒वो᳚ ऽस्यसि ध्रु॒वः । \newline
3. ध्रु॒वो॑ ऽह म॒हम् ध्रु॒वो ध्रु॒वो॑ ऽहम् । \newline
4. अ॒हꣳ स॑जा॒तेषु॑ सजा॒तेष्व॒ह म॒हꣳ स॑जा॒तेषु॑ । \newline
5. स॒जा॒तेषु॑ भूयासम् भूयासꣳ सजा॒तेषु॑ सजा॒तेषु॑ भूयासम् । \newline
6. स॒जा॒तेष्विति॑ स - जा॒तेषु॑ । \newline
7. भू॒या॒स॒म् धीरो॒ धीरो॑ भूयासम् भूयास॒म् धीरः॑ । \newline
8. धीर॒ श्चेत्ता॒ चेत्ता॒ धीरो॒ धीर॒ श्चेत्ता᳚ । \newline
9. चेत्ता॑ वसु॒विद् व॑सु॒विच् चेत्ता॒ चेत्ता॑ वसु॒वित् । \newline
10. व॒सु॒विदु॒ग्र उ॒ग्रो व॑सु॒विद् व॑सु॒विदु॒ग्रः । \newline
11. व॒सु॒विदिति॑ वसु - वित् । \newline
12. उ॒ग्रो᳚ ऽस्यस्यु॒ग्र उ॒ग्रो॑ ऽसि । \newline
13. अ॒स्यु॒ग्र उ॒ग्रो᳚ ऽस्यस्यु॒ग्रः । \newline
14. उ॒ग्रो॑ ऽह म॒ह मु॒ग्र उ॒ग्रो॑ ऽहम् । \newline
15. अ॒हꣳ स॑जा॒तेषु॑ सजा॒ते ष्व॒ह म॒हꣳ स॑जा॒तेषु॑ । \newline
16. स॒जा॒तेषु॑ भूयासम् भूयासꣳ सजा॒तेषु॑ सजा॒तेषु॑ भूयासम् । \newline
17. स॒जा॒तेष्विति॑ स - जा॒तेषु॑ । \newline
18. भू॒या॒स॒ मु॒ग्र उ॒ग्रो भू॑यासम् भूयास मु॒ग्रः । \newline
19. उ॒ग्र श्चेत्ता॒ चेत्तो॒ग्र उ॒ग्र श्चेत्ता᳚ । \newline
20. चेत्ता॑ वसु॒विद् व॑सु॒विच् चेत्ता॒ चेत्ता॑ वसु॒वित् । \newline
21. व॒सु॒वि द॑भि॒भू र॑भि॒भूर् व॑सु॒विद् व॑सु॒वि द॑भि॒भूः । \newline
22. व॒सु॒विदिति॑ वसु - वित् । \newline
23. अ॒भि॒भू र॑स्य स्यभि॒भू र॑भि॒भूर॑सि । \newline
24. अ॒भि॒भूरित्य॑भि - भूः । \newline
25. अ॒स्य॒भि॒भू र॑भि॒भू र॑स्य स्यभि॒भूः । \newline
26. अ॒भि॒भू र॒ह म॒ह म॑भि॒भू र॑भि॒भूर॒हम् । \newline
27. अ॒भि॒भूरित्य॑भि - भूः । \newline
28. अ॒हꣳ स॑जा॒तेषु॑ सजा॒तेष्व॒ह म॒हꣳ स॑जा॒तेषु॑ । \newline
29. स॒जा॒तेषु॑ भूयासम् भूयासꣳ सजा॒तेषु॑ सजा॒तेषु॑ भूयासम् । \newline
30. स॒जा॒तेष्विति॑ स - जा॒तेषु॑ । \newline
31. भू॒या॒स॒ म॒भि॒भू र॑भि॒भूर् भू॑यासम् भूयास मभि॒भूः । \newline
32. अ॒भि॒भू श्चेत्ता॒ चेत्ता॑ ऽभि॒भू र॑भि॒भू श्चेत्ता᳚ । \newline
33. अ॒भि॒भूरित्य॑भि - भूः । \newline
34. चेत्ता॑ वसु॒विद् व॑सु॒विच् चेत्ता॒ चेत्ता॑ वसु॒वित् । \newline
35. व॒सु॒विद् यु॒नज्मि॑ यु॒नज्मि॑ वसु॒विद् व॑सु॒विद् यु॒नज्मि॑ । \newline
36. व॒सु॒विदिति॑ वसु - वित् । \newline
37. यु॒नज्मि॑ त्वा त्वा यु॒नज्मि॑ यु॒नज्मि॑ त्वा । \newline
38. त्वा॒ ब्रह्म॑णा॒ ब्रह्म॑णा त्वा त्वा॒ ब्रह्म॑णा । \newline
39. ब्रह्म॑णा॒ दैव्ये॑न॒ दैव्ये॑न॒ ब्रह्म॑णा॒ ब्रह्म॑णा॒ दैव्ये॑न । \newline
40. दैव्ये॑न ह॒व्याय॑ ह॒व्याय॒ दैव्ये॑न॒ दैव्ये॑न ह॒व्याय॑ । \newline
41. ह॒व्याया॒स्मा अ॒स्मै ह॒व्याय॑ ह॒व्याया॒स्मै । \newline
42. अ॒स्मै वोढ॒वे वोढ॒वे᳚ ऽस्मा अ॒स्मै वोढ॒वे । \newline
43. वोढ॒वे जा॑तवेदो जातवे॒दो वोढ॒वे वोढ॒वे जा॑तवेदः । \newline
44. जा॒त॒वे॒द॒ इति॑ जात - वे॒दः॒ । \newline
45. इन्धा॑ना स्त्वा॒ त्वेन्धा॑ना॒ इन्धा॑ना स्त्वा । \newline
46. त्वा॒ सु॒प्र॒जसः॑ सुप्र॒जस॑ स्त्वा त्वा सुप्र॒जसः॑ । \newline
47. सु॒प्र॒जसः॑ सु॒वीराः᳚ सु॒वीराः᳚ सुप्र॒जसः॑ सुप्र॒जसः॑ सु॒वीराः᳚ । \newline
48. सु॒प्र॒जस॒ इति॑ सु - प्र॒जसः॑ । \newline
49. सु॒वीरा॒ ज्योग् ज्योख् सु॒वीराः᳚ सु॒वीरा॒ ज्योक् । \newline
50. सु॒वीरा॒ इति॑ सु - वीराः᳚ । \newline
51. ज्योग् जी॑वेम जीवेम॒ ज्योग् ज्योग् जी॑वेम । \newline
52. जी॒वे॒म॒ ब॒लि॒हृतो॑ बलि॒हृतो॑ जीवेम जीवेम बलि॒हृतः॑ । \newline
53. ब॒लि॒हृतो॑ व॒यं ॅव॒यम् ब॑लि॒हृतो॑ बलि॒हृतो॑ व॒यम् । \newline
54. ब॒लि॒हृत॒ इति॑ बलि - हृतः॑ । \newline
55. व॒यम् ते॑ ते व॒यं ॅव॒यम् ते᳚ । \newline
56. त॒ इति॑ ते । \newline
57. यन् मे॑ मे॒ यद् यन् मे᳚ । \newline
58. मे॒ अ॒ग्ने॒ ऽग्ने॒ मे॒ मे॒ अ॒ग्ने॒ । \newline
59. अ॒ग्ने॒ अ॒स्यास्याग्ने᳚ ऽग्ने अ॒स्य । \newline
60. अ॒स्य य॒ज्ञ्स्य॑ य॒ज्ञ् स्या॒स्यास्य य॒ज्ञ्स्य॑ । \newline
61. य॒ज्ञ्स्य॒ रिष्या॒द् रिष्या᳚द् य॒ज्ञ्स्य॑ य॒ज्ञ्स्य॒ रिष्या᳚त् । \newline
62. रिष्या॒द् यद् यद् रिष्या॒द् रिष्या॒द् यत् । \newline

\textbf{Ghana Paata } \newline

1. ध्रु॒वो᳚ ऽस्यसि ध्रु॒वो ध्रु॒वो॑ ऽसि ध्रु॒वो ध्रु॒वो॑ ऽसि ध्रु॒वो ध्रु॒वो॑ ऽसि ध्रु॒वः । \newline
2. अ॒सि॒ ध्रु॒वो ध्रु॒वो᳚ ऽस्यसि ध्रु॒वो॑ ऽह म॒हम् ध्रु॒वो᳚ ऽस्यसि ध्रु॒वो॑ ऽहम् । \newline
3. ध्रु॒वो॑ ऽह म॒हम् ध्रु॒वो ध्रु॒वो॑ ऽहꣳ स॑जा॒तेषु॑ सजा॒तेष्व॒हम् ध्रु॒वो ध्रु॒वो॑ ऽहꣳ स॑जा॒तेषु॑ । \newline
4. अ॒हꣳ स॑जा॒तेषु॑ सजा॒तेष्व॒ह म॒हꣳ स॑जा॒तेषु॑ भूयासम् भूयासꣳ सजा॒तेष्व॒ह म॒हꣳ स॑जा॒तेषु॑ भूयासम् । \newline
5. स॒जा॒तेषु॑ भूयासम् भूयासꣳ सजा॒तेषु॑ सजा॒तेषु॑ भूयास॒म् धीरो॒ धीरो॑ भूयासꣳ सजा॒तेषु॑ सजा॒तेषु॑ भूयास॒म् धीरः॑ । \newline
6. स॒जा॒तेष्विति॑ स - जा॒तेषु॑ । \newline
7. भू॒या॒स॒म् धीरो॒ धीरो॑ भूयासम् भूयास॒म् धीर॒श्चेत्ता॒ चेत्ता॒ धीरो॑ भूयासम् भूयास॒म् धीर॒श्चेत्ता᳚ । \newline
8. धीर॒श्चेत्ता॒ चेत्ता॒ धीरो॒ धीर॒श्चेत्ता॑ वसु॒विद् व॑सु॒विच् चेत्ता॒ धीरो॒ धीर॒श्चेत्ता॑ वसु॒वित् । \newline
9. चेत्ता॑ वसु॒विद् व॑सु॒विच् चेत्ता॒ चेत्ता॑ वसु॒विदु॒ग्र उ॒ग्रो व॑सु॒विच् चेत्ता॒ चेत्ता॑ वसु॒विदु॒ग्रः । \newline
10. व॒सु॒विदु॒ग्र उ॒ग्रो व॑सु॒विद् व॑सु॒विदु॒ग्रो᳚ ऽस्यस्यु॒ग्रो व॑सु॒विद् व॑सु॒विदु॒ग्रो॑ ऽसि । \newline
11. व॒सु॒विदिति॑ वसु - वित् । \newline
12. उ॒ग्रो᳚ ऽस्यस्यु॒ग्र उ॒ग्रो᳚ ऽस्यु॒ग्र उ॒ग्रो᳚ ऽस्यु॒ग्र उ॒ग्रो᳚ ऽस्यु॒ग्रः । \newline
13. अ॒स्यु॒ग्र उ॒ग्रो᳚ ऽस्यस्यु॒ग्रो॑ ऽह म॒ह मु॒ग्रो᳚ ऽस्यस्यु॒ग्रो॑ ऽहम् । \newline
14. उ॒ग्रो॑ ऽह म॒ह मु॒ग्र उ॒ग्रो॑ ऽहꣳ स॑जा॒तेषु॑ सजा॒तेष्व॒ह मु॒ग्र उ॒ग्रो॑ ऽहꣳ स॑जा॒तेषु॑ । \newline
15. अ॒हꣳ स॑जा॒तेषु॑ सजा॒तेष्व॒ह म॒हꣳ स॑जा॒तेषु॑ भूयासम् भूयासꣳ सजा॒तेष्व॒ह म॒हꣳ स॑जा॒तेषु॑ भूयासम् । \newline
16. स॒जा॒तेषु॑ भूयासम् भूयासꣳ सजा॒तेषु॑ सजा॒तेषु॑ भूयास मु॒ग्र उ॒ग्रो भू॑यासꣳ सजा॒तेषु॑ सजा॒तेषु॑ भूयास मु॒ग्रः । \newline
17. स॒जा॒तेष्विति॑ स - जा॒तेषु॑ । \newline
18. भू॒या॒स॒ मु॒ग्र उ॒ग्रो भू॑यासम् भूयास मु॒ग्रश्चेत्ता॒ चेत्तो॒ग्रो भू॑यासम् भूयास मु॒ग्रश्चेत्ता᳚ । \newline
19. उ॒ग्रश्चेत्ता॒ चेत्तो॒ग्र उ॒ग्रश्चेत्ता॑ वसु॒विद् व॑सु॒विच् चेत्तो॒ग्र उ॒ग्रश्चेत्ता॑ वसु॒वित् । \newline
20. चेत्ता॑ वसु॒विद् व॑सु॒विच् चेत्ता॒ चेत्ता॑ वसु॒विद॑भि॒भू र॑भि॒भूर् व॑सु॒विच् चेत्ता॒ चेत्ता॑ वसु॒विद॑भि॒भूः । \newline
21. व॒सु॒विद॑भि॒भू र॑भि॒भूर् व॑सु॒विद् व॑सु॒विद॑भि॒भू र॑स्यस्यभि॒भूर् व॑सु॒विद् व॑सु॒विद॑भि॒भूर॑सि । \newline
22. व॒सु॒विदिति॑ वसु - वित् । \newline
23. अ॒भि॒भू र॑स्यस्यभि॒भू र॑भि॒भू र॑स्यभि॒भू र॑भि॒भू र॑स्यभि॒भू र॑भि॒भू र॑स्यभि॒भूः । \newline
24. अ॒भि॒भूरित्य॑भि - भूः । \newline
25. अ॒स्य॒भि॒भू र॑भि॒भू र॑स्यस्यभि॒भूर॒ह म॒ह म॑भि॒भू र॑स्यस्यभि॒भू र॒हम् । \newline
26. अ॒भि॒भूर॒ह म॒ह म॑भि॒भू र॑भि॒भू र॒हꣳ स॑जा॒तेषु॑ सजा॒तेष्व॒ह म॑भि॒भू र॑भि॒भू र॒हꣳ स॑जा॒तेषु॑ । \newline
27. अ॒भि॒भूरित्य॑भि - भूः । \newline
28. अ॒हꣳ स॑जा॒तेषु॑ सजा॒तेष्व॒ह म॒हꣳ स॑जा॒तेषु॑ भूयासम् भूयासꣳ सजा॒तेष्व॒ह म॒हꣳ स॑जा॒तेषु॑ भूयासम् । \newline
29. स॒जा॒तेषु॑ भूयासम् भूयासꣳ सजा॒तेषु॑ सजा॒तेषु॑ भूयास मभि॒भूर॑भि॒भूर् भू॑यासꣳ सजा॒तेषु॑ सजा॒तेषु॑ भूयास मभि॒भूः । \newline
30. स॒जा॒तेष्विति॑ स - जा॒तेषु॑ । \newline
31. भू॒या॒स॒ म॒भि॒भूर॑भि॒भूर् भू॑यासम् भूयास मभि॒भूश्चेत्ता॒ चेत्ता॑ ऽभि॒भूर् भू॑यासम् भूयास मभि॒भूश्चेत्ता᳚ । \newline
32. अ॒भि॒भू श्चेत्ता॒ चेत्ता॑ ऽभि॒भूर॑भि॒भू श्चेत्ता॑ वसु॒विद् व॑सु॒विच् चेत्ता॑ ऽभि॒भूर॑भि॒भू श्चेत्ता॑ वसु॒वित् । \newline
33. अ॒भि॒भूरित्य॑भि - भूः । \newline
34. चेत्ता॑ वसु॒विद् व॑सु॒विच् चेत्ता॒ चेत्ता॑ वसु॒विद् यु॒नज्मि॑ यु॒नज्मि॑ वसु॒विच् चेत्ता॒ चेत्ता॑ वसु॒विद् यु॒नज्मि॑ । \newline
35. व॒सु॒विद् यु॒नज्मि॑ यु॒नज्मि॑ वसु॒विद् व॑सु॒विद् यु॒नज्मि॑ त्वा त्वा यु॒नज्मि॑ वसु॒विद् व॑सु॒विद् यु॒नज्मि॑ त्वा । \newline
36. व॒सु॒विदिति॑ वसु - वित् । \newline
37. यु॒नज्मि॑ त्वा त्वा यु॒नज्मि॑ यु॒नज्मि॑ त्वा॒ ब्रह्म॑णा॒ ब्रह्म॑णा त्वा यु॒नज्मि॑ यु॒नज्मि॑ त्वा॒ ब्रह्म॑णा । \newline
38. त्वा॒ ब्रह्म॑णा॒ ब्रह्म॑णा त्वा त्वा॒ ब्रह्म॑णा॒ दैव्ये॑न॒ दैव्ये॑न॒ ब्रह्म॑णा त्वा त्वा॒ ब्रह्म॑णा॒ दैव्ये॑न । \newline
39. ब्रह्म॑णा॒ दैव्ये॑न॒ दैव्ये॑न॒ ब्रह्म॑णा॒ ब्रह्म॑णा॒ दैव्ये॑न ह॒व्याय॑ ह॒व्याय॒ दैव्ये॑न॒ ब्रह्म॑णा॒ ब्रह्म॑णा॒ दैव्ये॑न ह॒व्याय॑ । \newline
40. दैव्ये॑न ह॒व्याय॑ ह॒व्याय॒ दैव्ये॑न॒ दैव्ये॑न ह॒व्याया॒स्मा अ॒स्मै ह॒व्याय॒ दैव्ये॑न॒ दैव्ये॑न ह॒व्याया॒स्मै । \newline
41. ह॒व्याया॒स्मा अ॒स्मै ह॒व्याय॑ ह॒व्याया॒स्मै वोढ॒वे वोढ॒वे᳚ ऽस्मै ह॒व्याय॑ ह॒व्याया॒स्मै वोढ॒वे । \newline
42. अ॒स्मै वोढ॒वे वोढ॒वे᳚ ऽस्मा अ॒स्मै वोढ॒वे जा॑तवेदो जातवेदो॒ वोढ॒वे᳚ ऽस्मा अ॒स्मै वोढ॒वे जा॑तवेदः । \newline
43. वोढ॒वे जा॑तवेदो जातवेदो॒ वोढ॒वे वोढ॒वे जा॑तवेदः । \newline
44. जा॒त॒वे॒द॒ इति॑ जात - वे॒दः॒ । \newline
45. इन्धा॑नास्त्वा॒ त्वेन्धा॑ना॒ इन्धा॑नास्त्वा सुप्र॒जसः॑ सुप्र॒जस॒ स्त्वेन्धा॑ना॒ इन्धा॑नास्त्वा सुप्र॒जसः॑ । \newline
46. त्वा॒ सु॒प्र॒जसः॑ सुप्र॒जस॑स्त्वा त्वा सुप्र॒जसः॑ सु॒वीराः᳚ सु॒वीराः᳚ सुप्र॒जस॑स्त्वा त्वा सुप्र॒जसः॑ सु॒वीराः᳚ । \newline
47. सु॒प्र॒जसः॑ सु॒वीराः᳚ सु॒वीराः᳚ सुप्र॒जसः॑ सुप्र॒जसः॑ सु॒वीरा॒ ज्योग् ज्योख् सु॒वीराः᳚ सुप्र॒जसः॑ सुप्र॒जसः॑ सु॒वीरा॒ ज्योक् । \newline
48. सु॒प्र॒जस॒ इति॑ सु - प्र॒जसः॑ । \newline
49. सु॒वीरा॒ ज्योग् ज्योख् सु॒वीराः᳚ सु॒वीरा॒ ज्योग् जी॑वेम जीवेम॒ ज्योख् सु॒वीराः᳚ सु॒वीरा॒ ज्योग् जी॑वेम । \newline
50. सु॒वीरा॒ इति॑ सु - वीराः᳚ । \newline
51. ज्योग् जी॑वेम जीवेम॒ ज्योग् ज्योग् जी॑वेम बलि॒हृतो॑ बलि॒हृतो॑ जीवेम॒ ज्योग् ज्योग् जी॑वेम बलि॒हृतः॑ । \newline
52. जी॒वे॒म॒ ब॒लि॒हृतो॑ बलि॒हृतो॑ जीवेम जीवेम बलि॒हृतो॑ व॒यं ॅव॒यम् ब॑लि॒हृतो॑ जीवेम जीवेम बलि॒हृतो॑ व॒यम् । \newline
53. ब॒लि॒हृतो॑ व॒यं ॅव॒यम् ब॑लि॒हृतो॑ बलि॒हृतो॑ व॒यम् ते॑ ते व॒यम् ब॑लि॒हृतो॑ बलि॒हृतो॑ व॒यम् ते᳚ । \newline
54. ब॒लि॒हृत॒ इति॑ बलि - हृतः॑ । \newline
55. व॒यम् ते॑ ते व॒यं ॅव॒यम् ते᳚ । \newline
56. त॒ इति॑ ते । \newline
57. यन् मे॑ मे॒ यद् यन् मे॑ अग्ने ऽग्ने मे॒ यद् यन् मे॑ अग्ने । \newline
58. मे॒ अ॒ग्ने॒ ऽग्ने॒ मे॒ मे॒ अ॒ग्ने॒ अ॒स्यास्याग्ने॑ मे मे अग्ने अ॒स्य । \newline
59. अ॒ग्ने॒ अ॒स्यास्याग्ने᳚ ऽग्ने अ॒स्य य॒ज्ञ्स्य॑ य॒ज्ञ्स्या॒स्याग्ने᳚ ऽग्ने अ॒स्य य॒ज्ञ्स्य॑ । \newline
60. अ॒स्य य॒ज्ञ्स्य॑ य॒ज्ञ्स्या॒स्यास्य य॒ज्ञ्स्य॒ रिष्या॒द् रिष्या᳚द् य॒ज्ञ्स्या॒स्यास्य य॒ज्ञ्स्य॒ रिष्या᳚त् । \newline
61. य॒ज्ञ्स्य॒ रिष्या॒द् रिष्या᳚द् य॒ज्ञ्स्य॑ य॒ज्ञ्स्य॒ रिष्या॒द् यद् यद् रिष्या᳚द् य॒ज्ञ्स्य॑ य॒ज्ञ्स्य॒ रिष्या॒द् यत् । \newline
62. रिष्या॒द् यद् यद् रिष्या॒द् रिष्या॒द् यद् वा॑ वा॒ यद् रिष्या॒द् रिष्या॒द् यद् वा᳚ । \newline
\pagebreak
\markright{ TS 1.6.2.2  \hfill https://www.vedavms.in \hfill}

\section{ TS 1.6.2.2 }

\textbf{TS 1.6.2.2 } \newline
\textbf{Samhita Paata} \newline

द् यद्वा॒ स्कन्दा॒-दाज्य॑स्यो॒त वि॑ष्णो । तेन॑ हन्मि स॒पत्नं॑ दुर्मरा॒युमैनं॑ दधामि॒ निर्.ऋ॑त्या उ॒पस्थे᳚ । भूर् भुवः॒ सुव॒रुच्छु॑ष्मो अग्ने॒ यज॑मानायैधि॒ निशु॑ष्मो अभि॒दास॑ते । अग्ने॒ देवे᳚द्ध॒ मन्वि॑द्ध॒ मन्द्र॑जि॒ह्वा-म॑र्त्यस्य ते होतर्मू॒र्द्धन्ना जि॑घर्मि रा॒यस्पोषा॑य सुप्रजा॒स्त्वाय॑ सु॒वीर्या॑य॒ मनो॑ऽसि प्राजाप॒त्यं मन॑सा मा भू॒तेना *वि॑श॒ वाग॑स्यै॒न्द्री स॑पत्न॒क्षय॑णी - [ ] \newline

\textbf{Pada Paata} \newline

यत् । वा॒ । स्कन्दा᳚त् । आज्य॑स्य । उ॒त । वि॒ष्णो॒ इति॑ ॥ तेन॑ । ह॒न्मि॒ । स॒पत्न᳚म् । दु॒र्म॒रा॒युमिति॑ दुः - म॒रा॒युम् । एति॑ । ए॒न॒म् । द॒धा॒मि॒ । निर्.ऋ॑त्या॒ इति॒ निः-ऋ॒त्याः॒ । उ॒पस्थ॒ इत्यु॒प - स्थे॒ ॥ भूः । भुवः॑ । सुवः॑ । उच्छु॑ष्म॒ इत्युत् - शु॒ष्मः॒ । अ॒ग्ने॒ । यज॑मानाय । ए॒धि॒ । निशु॑ष्म॒ इति॒ नि-शु॒ष्मः॒ । अ॒भि॒दास॑त॒ इत्य॑भि - दास॑ते ॥ अग्ने᳚ । देवे॒द्धेति॒ देव॑ - इ॒द्ध॒ । मन्वि॒द्धेति॒ मनु॑ - इ॒द्ध॒ । मन्द्र॑जि॒ह्वेति॒ मन्द्र॑-जि॒ह्व॒ । अम॑र्त्यस्य । ते॒ । हो॒तः॒ । मू॒र्द्धन्न् । एति॑ । जि॒घ॒र्मि॒ । रा॒यः । पोषा॑य । सु॒प्र॒जा॒स्त्वायेति॑ सुप्रजाः - त्वाय॑ । सु॒वीर्या॒येति॑ सु - वीर्या॑य । मनः॑ । अ॒सि॒ । प्रा॒जा॒प॒त्यमिति॑ प्राजा-प॒त्यम् । मन॑सा । मा॒ । भू॒तेन॑ । एति॑ । वि॒श॒ । वाक् । अ॒सि॒ । ऐ॒न्द्री । स॒प॒त्न॒क्षय॒णीति॑ सपत्न - क्षय॑णी ।  \newline


\textbf{Krama Paata} \newline

यद् वा᳚ । वा॒ स्कन्दा᳚त् । स्कन्दा॒दाज्य॑स्य । आज्य॑स्यो॒त । उ॒त वि॑ष्णो । वि॒ष्णो॒ इति॑ विष्णो ॥ तेन॑ हन्मि । ह॒न्मि॒ स॒पत्न᳚म् । स॒पत्न॑म् दुर्मरा॒युम् । दु॒र्म॒रा॒युमा । दु॒र्म॒रा॒युमिति॑ दुः - म॒रा॒युम् । ऐन᳚म् । ए॒न॒म् द॒धा॒मि॒ । द॒धा॒मि॒ निर्.ऋ॑त्याः । निर्.ऋ॑त्या उ॒पस्थे᳚ । निर्.ऋ॑त्या॒ इति॒ निः - ऋ॒त्याः॒ । उ॒पस्थ॒ इत्यु॒प - स्थे॒ ॥ भूर्भुवः॑ । भुवः॒ सुवः॑ । सुव॒रुच्छु॑ष्मः । उच्छु॑ष्मो अग्ने । उच्छु॑ष्म॒ इत्युत् - शु॒ष्मः॒ । अ॒ग्ने॒ यज॑मानाय । यज॑मानायैधि । ए॒धि॒ निशु॑ष्मः । निशु॑ष्मो अभि॒दास॑ते । निशु॑ष्म॒ इति॒ नि - शु॒ष्मः॒ । अ॒भि॒दास॑त॒ इत्य॑भि दास॑ते ॥ अग्ने॒ देवे᳚द्ध । देवे᳚द्ध॒ मन्वि॑द्ध । देवे॒द्धेति॒ देव॑ - इ॒द्ध॒ । मन्वि॑द्ध॒ मन्द्र॑जिह्व । मन्वि॒द्धेति॒ मनु॑ - इ॒द्ध॒ । 
मन्द्र॑जि॒ह्वाम॑र्त्यस्य । मन्द्र॑जि॒ह्वेति॒ मन्द्र॑ - जि॒ह्व॒ । अम॑र्त्यस्य ते । ते॒ हो॒तः॒ । हो॒त॒र्,मू॒र्द्धन्न् । मू॒र्द्धन्ना । आ जि॑घर्मि । जि॒घ॒र्मि॒ रा॒यः । रा॒यस्पोषा॑य । पोषा॑य सुप्रजा॒स्त्वाय॑ । सु॒प्र॒जा॒स्त्वाय॑ सु॒वीर्या॑य । सु॒प्र॒जा॒स्त्वायेति॑ सुप्रजाः - त्वाय॑ । सु॒वीर्या॑य॒ मनः॑ । सु॒वीर्या॒येति॑ सु - वीर्या॑य । मनो॑ऽसि । अ॒सि॒ प्रा॒जा॒प॒त्यम् । प्रा॒जा॒प॒त्यम् मन॑सा । प्रा॒जा॒प॒त्यमिति॑ प्राजा - प॒त्यम् । मन॑सा मा । मा॒ भू॒तेन॑ । भू॒तेना । आ वि॑श । वि॒श॒ वाक् । वाग॑सि । अ॒स्यै॒न्द्री । ऐ॒न्द्री स॑पत्न॒क्षय॑णी । स॒प॒त्न॒क्षय॑णी वा॒चा । स॒प॒त्न॒क्षय॒णीति॑ सपत्न - क्षय॑णी \newline

\textbf{Jatai Paata} \newline

1. यद् वा॑ वा॒ यद् यद् वा᳚ । \newline
2. वा॒ स्कन्दा॒थ् स्कन्दा᳚द् वा वा॒ स्कन्दा᳚त् । \newline
3. स्कन्दा॒ दाज्य॒ स्याज्य॑स्य॒ स्कन्दा॒थ् स्कन्दा॒ दाज्य॑स्य । \newline
4. आज्य॑ स्यो॒तो ताज्य॒ स्याज्य॑स्यो॒त । \newline
5. उ॒त वि॑ष्णो विष्णो उ॒तोत वि॑ष्णो । \newline
6. वि॒ष्णो॒ इति॑ विष्णो । \newline
7. तेन॑ हन्मि हन्मि॒ तेन॒ तेन॑ हन्मि । \newline
8. ह॒न्मि॒ स॒पत्न(ग्म्॑) स॒पत्न(ग्म्॑) हन्मि हन्मि स॒पत्न᳚म् । \newline
9. स॒पत्न॑म् दुर्मरा॒युम् दु॑र्मरा॒युꣳ स॒पत्न(ग्म्॑) स॒पत्न॑म् दुर्मरा॒युम् । \newline
10. दु॒र्म॒रा॒यु मा दु॑र्मरा॒युम् दु॑र्मरा॒यु मा । \newline
11. दु॒र्म॒रा॒युमिति॑ दुः - म॒रा॒युम् । \newline
12. ऐन॑ मेन॒ मैन᳚म् । \newline
13. ए॒न॒म् द॒धा॒मि॒ द॒धा॒म्ये॒न॒ मे॒न॒म् द॒धा॒मि॒ । \newline
14. द॒धा॒मि॒ निर्.ऋ॑त्या॒ निर्.ऋ॑त्या दधामि दधामि॒ निर्.ऋ॑त्याः । \newline
15. निर्.ऋ॑त्या उ॒पस्थ॑ उ॒पस्थे॒ निर्.ऋ॑त्या॒ निर्.ऋ॑त्या उ॒पस्थे᳚ । \newline
16. निर्.ऋ॑त्या॒ इति॒ निः - ऋ॒त्याः॒ । \newline
17. उ॒पस्थ॒ इत्यु॒प - स्थे॒ । \newline
18. भूर् भुवो॒ भुवो॒ भूर् भूर् भुवः॑ । \newline
19. भुवः॒ सुवः॒ सुव॒र् भुवो॒ भुवः॒ सुवः॑ । \newline
20. सुव॒ रुच्छु॑ष्म॒ उच्छु॑ष्मः॒ सुवः॒ सुव॒ रुच्छु॑ष्मः । \newline
21. उच्छु॑ष्मो अग्ने ऽग्न॒ उच्छु॑ष्म॒ उच्छु॑ष्मो अग्ने । \newline
22. उच्छु॑ष्म॒ इत्युत् - शु॒ष्मः॒ । \newline
23. अ॒ग्ने॒ यज॑मानाय॒ यज॑मानायाग्ने ऽग्ने॒ यज॑मानाय । \newline
24. यज॑माना यैध्येधि॒ यज॑मानाय॒ यज॑माना यैधि । \newline
25. ए॒धि॒ निशु॑ष्मो॒ निशु॑ष्म एध्येधि॒ निशु॑ष्मः । \newline
26. निशु॑ष्मो अभि॒दास॑ते ऽभि॒दास॑ते॒ निशु॑ष्मो॒ निशु॑ष्मो अभि॒दास॑ते । \newline
27. निशु॑ष्म॒ इति॒ नि - शु॒ष्मः॒ । \newline
28. अ॒भि॒दास॑त॒ इत्य॑भि - दास॑ते । \newline
29. अग्ने॒ देवे᳚द्ध॒ देवे॒द्धाग्ने ऽग्ने॒ देवे᳚द्ध । \newline
30. देवे᳚द्ध॒ मन्वि॑द्ध॒ मन्वि॑द्ध॒ देवे᳚द्ध॒ देवे᳚द्ध॒ मन्वि॑द्ध । \newline
31. देवे॒द्धेति॒ देव॑ - इ॒द्ध॒ । \newline
32. मन्वि॑द्ध॒ मन्द्र॑जिह्व॒ मन्द्र॑जिह्व॒ मन्वि॑द्ध॒ मन्वि॑द्ध॒ मन्द्र॑जिह्व । \newline
33. मन्वि॒द्धेति॒ मनु॑ - इ॒द्ध॒ । \newline
34. मन्द्र॑जि॒ह्वा म॑र्त्य॒स्याम॑र्त्यस्य॒ मन्द्र॑जिह्व॒ मन्द्र॑जि॒ह्वा म॑र्त्यस्य । \newline
35. मन्द्र॑जि॒ह्वेति॒ मन्द्र॑ - जि॒ह्व॒ । \newline
36. अम॑र्त्यस्य ते॒ ते ऽम॑र्त्य॒स्या म॑र्त्यस्य ते । \newline
37. ते॒ हो॒त॒र्॒. हो॒त॒स्ते॒ ते॒ हो॒तः॒ । \newline
38. हो॒त॒र् मू॒र्द्धन् मू॒र्द्धन्. हो॑तर्. होतर् मू॒र्द्धन्न् । \newline
39. मू॒र्द्धन् ना मू॒र्द्धन् मू॒र्द्धन् ना । \newline
40. आ जि॑घर्मि जिघ॒र्म्या जि॑घर्मि । \newline
41. जि॒घ॒र्मि॒ रा॒यो रा॒यो जि॑घर्मि जिघर्मि रा॒यः । \newline
42. रा॒य स्पोषा॑य॒ पोषा॑य रा॒यो रा॒य स्पोषा॑य । \newline
43. पोषा॑य सुप्रजा॒स्त्वाय॑ सुप्रजा॒स्त्वाय॒ पोषा॑य॒ पोषा॑य सुप्रजा॒स्त्वाय॑ । \newline
44. सु॒प्र॒जा॒स्त्वाय॑ सु॒वीर्या॑य सु॒वीर्या॑य सुप्रजा॒स्त्वाय॑ सुप्रजा॒स्त्वाय॑ सु॒वीर्या॑य । \newline
45. सु॒प्र॒जा॒स्त्वायेति॑ सुप्रजाः - त्वाय॑ । \newline
46. सु॒वीर्या॑य॒ मनो॒ मनः॑ सु॒वीर्या॑य सु॒वीर्या॑य॒ मनः॑ । \newline
47. सु॒वीर्या॒येति॑ सु - वीर्या॑य । \newline
48. मनो᳚ ऽस्यसि॒ मनो॒ मनो॑ ऽसि । \newline
49. अ॒सि॒ प्रा॒जा॒प॒त्यम् प्रा॑जाप॒त्य म॑स्यसि प्राजाप॒त्यम् । \newline
50. प्रा॒जा॒प॒त्यम् मन॑सा॒ मन॑सा प्राजाप॒त्यम् प्रा॑जाप॒त्यम् मन॑सा । \newline
51. प्रा॒जा॒प॒त्यमिति॑ प्राजा - प॒त्यम् । \newline
52. मन॑सा मा मा॒ मन॑सा॒ मन॑सा मा । \newline
53. मा॒ भू॒तेन॑ भू॒तेन॑ मा मा भू॒तेन॑ । \newline
54. भू॒तेना भू॒तेन॑ भू॒तेना । \newline
55. आ वि॑श वि॒शा वि॑श । \newline
56. वि॒श॒ वाग् वाग् वि॑श विश॒ वाक् । \newline
57. वाग॑स्यसि॒ वाग् वाग॑सि । \newline
58. अ॒स्यै॒न्द्र्यै᳚(1॒)न्द्र्य॑स्य स्यै॒न्द्री । \newline
59. ऐ॒न्द्री स॑पत्न॒क्षय॑णी सपत्न॒क्षय॑ण्यै॒न्द्र्यै᳚न्द्री स॑पत्न॒क्षय॑णी । \newline
60. स॒प॒त्न॒क्षय॑णी वा॒चा वा॒चा स॑पत्न॒क्षय॑णी सपत्न॒क्षय॑णी वा॒चा । \newline
61. स॒प॒त्न॒क्षय॒णीति॑ सपत्न - क्षय॑णी । \newline

\textbf{Ghana Paata } \newline

1. यद् वा॑ वा॒ यद् यद् वा॒ स्कन्दा॒थ् स्कन्दा᳚द् वा॒ यद् यद् वा॒ स्कन्दा᳚त् । \newline
2. वा॒ स्कन्दा॒थ् स्कन्दा᳚द् वा वा॒ स्कन्दा॒ दाज्य॒स्याज्य॑स्य॒ स्कन्दा᳚द् वा वा॒ स्कन्दा॒ दाज्य॑स्य । \newline
3. स्कन्दा॒ दाज्य॒स्याज्य॑स्य॒ स्कन्दा॒थ् स्कन्दा॒ दाज्य॑स्यो॒तोताज्य॑स्य॒ स्कन्दा॒थ् स्कन्दा॒ दाज्य॑स्यो॒त । \newline
4. आज्य॑स्यो॒तो ताज्य॒स्याज्य॑स्यो॒त वि॑ष्णो विष्णो उ॒ताज्य॒ स्याज्य॑स्यो॒त वि॑ष्णो । \newline
5. उ॒त वि॑ष्णो विष्णो उ॒तोत वि॑ष्णो । \newline
6. वि॒ष्णो॒ इति॑ विष्णो । \newline
7. तेन॑ हन्मि हन्मि॒ तेन॒ तेन॑ हन्मि स॒पत्न(ग्म्॑) स॒पत्न(ग्म्॑) हन्मि॒ तेन॒ तेन॑ हन्मि स॒पत्न᳚म् । \newline
8. ह॒न्मि॒ स॒पत्न(ग्म्॑) स॒पत्न(ग्म्॑) हन्मि हन्मि स॒पत्न॑म् दुर्मरा॒युम् दु॑र्मरा॒युꣳ स॒पत्न(ग्म्॑) हन्मि हन्मि स॒पत्न॑म् दुर्मरा॒युम् । \newline
9. स॒पत्न॑म् दुर्मरा॒युम् दु॑र्मरा॒युꣳ स॒पत्न(ग्म्॑) स॒पत्न॑म् दुर्मरा॒यु मा दु॑र्मरा॒युꣳ स॒पत्न(ग्म्॑) स॒पत्न॑म् दुर्मरा॒यु मा । \newline
10. दु॒र्म॒रा॒यु मा दु॑र्मरा॒युम् दु॑र्मरा॒यु मैन॑ मेन॒ मा दु॑र्मरा॒युम् दु॑र्मरा॒यु मैन᳚म् । \newline
11. दु॒र्म॒रा॒युमिति॑ दुः - म॒रा॒युम् । \newline
12. ऐन॑ मेन॒ मैन॑म् दधामि दधाम्येन॒ मैन॑म् दधामि । \newline
13. ए॒न॒म् द॒धा॒मि॒ द॒धा॒म्ये॒न॒ मे॒न॒म् द॒धा॒मि॒ निर्.ऋ॑त्या॒ निर्.ऋ॑त्या दधाम्येन मेनम् दधामि॒ निर्.ऋ॑त्याः । \newline
14. द॒धा॒मि॒ निर्.ऋ॑त्या॒ निर्.ऋ॑त्या दधामि दधामि॒ निर्.ऋ॑त्या उ॒पस्थ॑ उ॒पस्थे॒ निर्.ऋ॑त्या दधामि दधामि॒ निर्.ऋ॑त्या उ॒पस्थे᳚ । \newline
15. निर्.ऋ॑त्या उ॒पस्थ॑ उ॒पस्थे॒ निर्.ऋ॑त्या॒ निर्.ऋ॑त्या उ॒पस्थे᳚ । \newline
16. निर्.ऋ॑त्या॒ इति॒ निः - ऋ॒त्याः॒ । \newline
17. उ॒पस्थ॒ इत्यु॒प - स्थे॒ । \newline
18. भूर् भुवो॒ भुवो॒ भूर् भूर् भुवः॒ सुवः॒ सुव॒र् भुवो॒ भूर् भूर् भुवः॒ सुवः॑ । \newline
19. भुवः॒ सुवः॒ सुव॒र् भुवो॒ भुवः॒ सुव॒ रुच्छु॑ष्म॒ उच्छु॑ष्मः॒ सुव॒र् भुवो॒ भुवः॒ सुव॒ रुच्छु॑ष्मः । \newline
20. सुव॒ रुच्छु॑ष्म॒ उच्छु॑ष्मः॒ सुवः॒ सुव॒ रुच्छु॑ष्मो अग्ने ऽग्न॒ उच्छु॑ष्मः॒ सुवः॒ सुव॒ रुच्छु॑ष्मो अग्ने । \newline
21. उच्छु॑ष्मो अग्ने ऽग्न॒ उच्छु॑ष्म॒ उच्छु॑ष्मो अग्ने॒ यज॑मानाय॒ यज॑मानायाग्न॒ उच्छु॑ष्म॒ उच्छु॑ष्मो अग्ने॒ यज॑मानाय । \newline
22. उच्छु॑ष्म॒ इत्युत् - शु॒ष्मः॒ । \newline
23. अ॒ग्ने॒ यज॑मानाय॒ यज॑मानायाग्ने ऽग्ने॒ यज॑मानायैध्येधि॒ यज॑मानायाग्ने ऽग्ने॒ यज॑मानायैधि । \newline
24. यज॑मानायैध्येधि॒ यज॑मानाय॒ यज॑मानायैधि॒ निशु॑ष्मो॒ निशु॑ष्म एधि॒ यज॑मानाय॒ यज॑मानायैधि॒ निशु॑ष्मः । \newline
25. ए॒धि॒ निशु॑ष्मो॒ निशु॑ष्म एध्येधि॒ निशु॑ष्मो अभि॒दास॑ते ऽभि॒दास॑ते॒ निशु॑ष्म एध्येधि॒ निशु॑ष्मो अभि॒दास॑ते । \newline
26. निशु॑ष्मो अभि॒दास॑ते ऽभि॒दास॑ते॒ निशु॑ष्मो॒ निशु॑ष्मो अभि॒दास॑ते । \newline
27. निशु॑ष्म॒ इति॒ नि - शु॒ष्मः॒ । \newline
28. अ॒भि॒दास॑त॒ इत्य॑भि - दास॑ते । \newline
29. अग्ने॒ देवे᳚द्ध॒ देवे॒द्धाग्ने ऽग्ने॒ देवे᳚द्ध॒ मन्वि॑द्ध॒ मन्वि॑द्ध॒ देवे॒द्धाग्ने ऽग्ने॒ देवे᳚द्ध॒ मन्वि॑द्ध । \newline
30. देवे᳚द्ध॒ मन्वि॑द्ध॒ मन्वि॑द्ध॒ देवे᳚द्ध॒ देवे᳚द्ध॒ मन्वि॑द्ध॒ मन्द्र॑जिह्व॒ मन्द्र॑जिह्व॒ मन्वि॑द्ध॒ देवे᳚द्ध॒ देवे᳚द्ध॒ मन्वि॑द्ध॒ मन्द्र॑जिह्व । \newline
31. देवे॒द्धेति॒ देव॑ - इ॒द्ध॒ । \newline
32. मन्वि॑द्ध॒ मन्द्र॑जिह्व॒ मन्द्र॑जिह्व॒ मन्वि॑द्ध॒ मन्वि॑द्ध॒ मन्द्र॑जि॒ह्वाम॑र्त्य॒स्याम॑र्त्यस्य॒ मन्द्र॑जिह्व॒ मन्वि॑द्ध॒ मन्वि॑द्ध॒ मन्द्र॑जि॒ह्वाम॑र्त्यस्य । \newline
33. मन्वि॒द्धेति॒ मनु॑ - इ॒द्ध॒ । \newline
34. मन्द्र॑जि॒ह्वाम॑र्त्य॒स्याम॑र्त्यस्य॒ मन्द्र॑जिह्व॒ मन्द्र॑जि॒ह्वाम॑र्त्यस्य ते॒ ते ऽम॑र्त्यस्य॒ मन्द्र॑जिह्व॒ मन्द्र॑जि॒ह्वाम॑र्त्यस्य ते । \newline
35. मन्द्र॑जि॒ह्वेति॒ मन्द्र॑ - जि॒ह्व॒ । \newline
36. अम॑र्त्यस्य ते॒ ते ऽम॑र्त्य॒स्याम॑र्त्यस्य ते होतर्. होत॒स्ते ऽम॑र्त्य॒स्याम॑र्त्यस्य ते होतः । \newline
37. ते॒ हो॒त॒र्॒. हो॒त॒स्ते॒ ते॒ हो॒त॒र् मू॒र्द्धन् मू॒र्द्धन्. हो॑तस्ते ते होतर् मू॒र्द्धन्न् । \newline
38. हो॒त॒र् मू॒र्द्धन् मू॒र्द्धन्. हो॑तर्. होतर् मू॒र्द्धन् ना मू॒र्द्धन्. हो॑तर्. होतर् मू॒र्द्धन् ना । \newline
39. मू॒र्द्धन् ना मू॒र्द्धन् मू॒र्द्धन् ना जि॑घर्मि जिघ॒र्म्या मू॒र्द्धन् मू॒र्द्धन् ना जि॑घर्मि । \newline
40. आ जि॑घर्मि जिघ॒र्म्या जि॑घर्मि रा॒यो रा॒यो जि॑घ॒र्म्या जि॑घर्मि रा॒यः । \newline
41. जि॒घ॒र्मि॒ रा॒यो रा॒यो जि॑घर्मि जिघर्मि रा॒य स्पोषा॑य॒ पोषा॑य रा॒यो जि॑घर्मि जिघर्मि रा॒य स्पोषा॑य । \newline
42. रा॒य स्पोषा॑य॒ पोषा॑य रा॒यो रा॒य स्पोषा॑य सुप्रजा॒स्त्वाय॑ सुप्रजा॒स्त्वाय॒ पोषा॑य रा॒यो रा॒य स्पोषा॑य सुप्रजा॒स्त्वाय॑ । \newline
43. पोषा॑य सुप्रजा॒स्त्वाय॑ सुप्रजा॒स्त्वाय॒ पोषा॑य॒ पोषा॑य सुप्रजा॒स्त्वाय॑ सु॒वीर्या॑य सु॒वीर्या॑य सुप्रजा॒स्त्वाय॒ पोषा॑य॒ पोषा॑य सुप्रजा॒स्त्वाय॑ सु॒वीर्या॑य । \newline
44. सु॒प्र॒जा॒स्त्वाय॑ सु॒वीर्या॑य सु॒वीर्या॑य सुप्रजा॒स्त्वाय॑ सुप्रजा॒स्त्वाय॑ सु॒वीर्या॑य॒ मनो॒ मनः॑ सु॒वीर्या॑य सुप्रजा॒स्त्वाय॑ सुप्रजा॒स्त्वाय॑ सु॒वीर्या॑य॒ मनः॑ । \newline
45. सु॒प्र॒जा॒स्त्वायेति॑ सुप्रजाः - त्वाय॑ । \newline
46. सु॒वीर्या॑य॒ मनो॒ मनः॑ सु॒वीर्या॑य सु॒वीर्या॑य॒ मनो᳚ ऽस्यसि॒ मनः॑ सु॒वीर्या॑य सु॒वीर्या॑य॒ मनो॑ ऽसि । \newline
47. सु॒वीर्या॒येति॑ सु - वीर्या॑य । \newline
48. मनो᳚ ऽस्यसि॒ मनो॒ मनो॑ ऽसि प्राजाप॒त्यम् प्रा॑जाप॒त्य म॑सि॒ मनो॒ मनो॑ ऽसि प्राजाप॒त्यम् । \newline
49. अ॒सि॒ प्रा॒जा॒प॒त्यम् प्रा॑जाप॒त्य म॑स्यसि प्राजाप॒त्यम् मन॑सा॒ मन॑सा प्राजाप॒त्य म॑स्यसि प्राजाप॒त्यम् मन॑सा । \newline
50. प्रा॒जा॒प॒त्यम् मन॑सा॒ मन॑सा प्राजाप॒त्यम् प्रा॑जाप॒त्यम् मन॑सा मा मा॒ मन॑सा प्राजाप॒त्यम् प्रा॑जाप॒त्यम् मन॑सा मा । \newline
51. प्रा॒जा॒प॒त्यमिति॑ प्राजा - प॒त्यम् । \newline
52. मन॑सा मा मा॒ मन॑सा॒ मन॑सा मा भू॒तेन॑ भू॒तेन॑ मा॒ मन॑सा॒ मन॑सा मा भू॒तेन॑ । \newline
53. मा॒ भू॒तेन॑ भू॒तेन॑ मा मा भू॒तेना भू॒तेन॑ मा मा भू॒तेना । \newline
54. भू॒तेना भू॒तेन॑ भू॒तेना वि॑श वि॒शा भू॒तेन॑ भू॒तेना वि॑श । \newline
55. आ वि॑श वि॒शा वि॑श॒ वाग् वाग् वि॒शा वि॑श॒ वाक् । \newline
56. वि॒श॒ वाग् वाग् वि॑श विश॒ वाग॑स्यसि॒ वाग् वि॑श विश॒ वाग॑सि । \newline
57. वाग॑स्यसि॒ वाग् वाग॑स्यै॒न्द्र्यै᳚(1॒)न्द्र्य॑सि॒ वाग् वाग॑स्यै॒न्द्री । \newline
58. अ॒स्यै॒न्द्र्यै᳚(1॒)न्द्र्य॑स्यस्यै॒न्द्री स॑पत्न॒क्षय॑णी सपत्न॒क्षय॑ण्यै॒न्द्र्य॑स्यस्यै॒न्द्री स॑पत्न॒क्षय॑णी । \newline
59. ऐ॒न्द्री स॑पत्न॒क्षय॑णी सपत्न॒क्षय॑ण्यै॒न्द्र्यै᳚न्द्री स॑पत्न॒क्षय॑णी वा॒चा वा॒चा स॑पत्न॒क्षय॑ण्यै॒न्द्र्यै᳚न्द्री वा॒चा । \newline
60. स॒प॒त्न॒क्षय॑णी वा॒चा वा॒चा स॑पत्न॒क्षय॑णी सपत्न॒क्षय॑णी वा॒चा मा॑ मा वा॒चा स॑पत्न॒क्षय॑णी सपत्न॒क्षय॑णी वा॒चा मा᳚ । \newline
61. स॒प॒त्न॒क्षय॒णीति॑ सपत्न - क्षय॑णी । \newline
\pagebreak
\markright{ TS 1.6.2.3  \hfill https://www.vedavms.in \hfill}

\section{ TS 1.6.2.3 }

\textbf{TS 1.6.2.3 } \newline
\textbf{Samhita Paata} \newline

वा॒चा मे᳚न्द्रि॒येणा *वि॑श वस॒न्तमृ॑तू॒नां प्री॑णामि॒ स मा᳚ प्री॒तः प्री॑णातु ग्री॒ष्ममृ॑तू॒नां प्री॑णामि॒ स मा᳚ प्री॒तः प्री॑णातु व॒र्॒.षा ऋ॑तू॒नां प्री॑णामि॒ ता मा᳚ प्री॒ताः प्री॑णन्तु श॒रद॑मृतू॒नां प्री॑णामि॒ सा मा᳚ प्री॒ता प्री॑णातु हेमन्तशिशि॒रावृ॑तू॒नां प्री॑णामि॒ तौ मा᳚ प्री॒तौ प्री॑णीता-म॒ग्नीषोम॑यो-र॒हं दे॑वय॒ज्यया॒ चक्षु॑ष्मान् भूयासम॒ग्नेर॒हं दे॑वय॒ज्यया᳚ऽन्ना॒दो भू॑यासं॒ - [ ] \newline

\textbf{Pada Paata} \newline

वा॒चा । मा॒ । इ॒न्द्रि॒येण॑ । एति॑ । वि॒श॒ । व॒स॒न्तम् । ऋ॒तू॒नाम् । प्री॒णा॒मि॒ । सः । मा॒ । प्री॒तः । प्री॒णा॒तु॒ । ग्री॒ष्मम् । ऋ॒तू॒नाम् । प्री॒णा॒मि॒ । सः । मा॒ । प्री॒तः । प्री॒णा॒तु॒ । व॒र्॒.षा ः । ऋ॒तू॒नाम् । प्री॒णा॒मि॒ । ताः । मा॒ । प्री॒ताः । प्री॒ण॒न्तु॒ । श॒रद᳚म् । ऋ॒तू॒नाम् । प्री॒णा॒मि॒ । सा । मा॒ । प्री॒ता । प्री॒णा॒तु॒ । हे॒म॒न्त॒शि॒शि॒राविति॑ हेमन्त - शि॒शि॒रौ । ऋ॒तू॒नाम् । प्री॒णा॒मि॒ । तौ । मा॒ । प्री॒तौ । प्री॒णी॒ता॒म् । अ॒ग्नीषोम॑यो॒रित्य॒ग्नी - सोम॑योः । अ॒हम् । दे॒व॒य॒ज्ययेति॑ देव- य॒ज्यया᳚ । चक्षु॑ष्मान् । भू॒या॒स॒म् । अ॒ग्नेः । अ॒हम् । दे॒व॒य॒ज्ययेति॑ देव - य॒ज्यया᳚ । अ॒न्ना॒द इत्य॑न्न - अ॒दः । भू॒या॒स॒म् ।  \newline


\textbf{Krama Paata} \newline

वा॒चा मा᳚ । मे॒न्द्रि॒येण॑ । इ॒न्द्रि॒येणा । आ वि॑श । वि॒श॒ व॒स॒न्तम् । व॒स॒न्तमृ॑तू॒नाम् । ऋ॒तू॒नाम् प्री॑णामि । प्री॒णा॒मि॒ सः । स मा᳚ । मा॒ प्री॒तः । प्री॒तः प्री॑णातु । प्री॒णा॒तु॒ ग्री॒ष्मम् । ग्री॒ष्ममृ॑तू॒नाम् । ऋ॒तू॒नाम् प्री॑णामि । प्री॒णा॒मि॒ सः । स मा᳚ । मा॒ प्री॒तः । प्री॒तः प्री॑णातु । प्री॒णा॒तु॒ व॒र्॒.षाः । व॒र्॒.षा ऋ॑तू॒नाम् । ऋ॒तू॒नाम् प्री॑णामि । प्री॒णा॒मि॒ ताः । ता मा᳚ । मा॒ प्री॒ताः । प्री॒ताः प्री॑णन्तु । प्री॒ण॒न्तु॒ श॒रद᳚म् । श॒रद॑मृतू॒नाम् । ऋ॒तू॒नाम् प्री॑णामि । प्री॒णा॒मि॒ सा । सा मा᳚ । मा॒ प्री॒ता । प्री॒ता प्री॑णातु । प्री॒णा॒तु॒ हे॒म॒न्त॒शि॒शि॒रौ । हे॒म॒न्त॒शि॒शि॒रावृ॑तू॒नाम् । हे॒म॒न्त॒शि॒शि॒राविति॑ हेमन्त - शि॒शि॒रौ । ऋ॒तू॒नाम् प्री॑णामि । प्री॒णा॒मि॒ तौ । तौ मा᳚ । मा॒ प्री॒तौ । प्री॒तौ प्री॑णीताम् । प्री॒णी॒ता॒म॒ग्नीषोम॑योः । अ॒ग्नीषोम॑योर॒हम् । अ॒ग्नीषोम॑यो॒रित्य॒ग्नी - सोम॑योः । अ॒हम् दे॑वय॒ज्यया᳚ । दे॒व॒य॒ज्यया॒ चक्षु॑ष्मान् । दे॒व॒य॒ज्ययेति॑ देव - य॒ज्यया᳚ । चक्षु॑ष्मान् भूयासम् । भू॒या॒स॒म॒ग्नेः । अ॒ग्नेर॒हम् । अ॒हम् दे॑वय॒ज्यया᳚ । दे॒व॒य॒ज्यया᳚ऽन्ना॒दः । दे॒व॒य॒ज्ययेति॑ देव - य॒ज्यया᳚ । अ॒न्ना॒दो भू॑यासम् ( ) । अ॒न्ना॒द इत्य॑न्न - अ॒दः । भू॒या॒स॒म् दब्धिः॑ \newline

\textbf{Jatai Paata} \newline

1. वा॒चा मा॑ मा वा॒चा वा॒चा मा᳚ । \newline
2. मे॒न्द्रि॒येणे᳚ न्द्रि॒येण॑ मा मेन्द्रि॒येण॑ । \newline
3. इ॒न्द्रि॒येणेन्द्रि॒येणे᳚ न्द्रि॒येणा । \newline
4. आ वि॑श वि॒शा वि॑श । \newline
5. वि॒श॒ व॒स॒न्तं ॅव॑स॒न्तं ॅवि॑श विश वस॒न्तम् । \newline
6. व॒स॒न्त मृ॑तू॒ना मृ॑तू॒नां ॅव॑स॒न्तं ॅव॑स॒न्त मृ॑तू॒नाम् । \newline
7. ऋ॒तू॒नाम् प्री॑णामि प्रीणाम्यृतू॒ना मृ॑तू॒नाम् प्री॑णामि । \newline
8. प्री॒णा॒मि॒ स स प्री॑णामि प्रीणामि॒ सः । \newline
9. स मा॑ मा॒ स स मा᳚ । \newline
10. मा॒ प्री॒तः प्री॒तो मा॑ मा प्री॒तः । \newline
11. प्री॒तः प्री॑णातु प्रीणातु प्री॒तः प्री॒तः प्री॑णातु । \newline
12. प्री॒णा॒तु॒ ग्री॒ष्मम् ग्री॒ष्मम् प्री॑णातु प्रीणातु ग्री॒ष्मम् । \newline
13. ग्री॒ष्म मृ॑तू॒ना मृ॑तू॒नाम् ग्री॒ष्मम् ग्री॒ष्म मृ॑तू॒नाम् । \newline
14. ऋ॒तू॒नाम् प्री॑णामि प्रीणा म्यृतू॒ना मृ॑तू॒नाम् प्री॑णामि । \newline
15. प्री॒णा॒मि॒ स स प्री॑णामि प्रीणामि॒ सः । \newline
16. स मा॑ मा॒ स स मा᳚ । \newline
17. मा॒ प्री॒तः प्री॒तो मा॑ मा प्री॒तः । \newline
18. प्री॒तः प्री॑णातु प्रीणातु प्री॒तः प्री॒तः प्री॑णातु । \newline
19. प्री॒णा॒तु॒ व॒र्॒.षा व॒र्॒.षाः प्री॑णातु प्रीणातु व॒र्॒.षाः । \newline
20. व॒र्॒.षा ऋ॑तू॒ना मृ॑तू॒नां ॅव॒र्॒.षा व॒र्॒.षा ऋ॑तू॒नाम् । \newline
21. ऋ॒तू॒नाम् प्री॑णामि प्रीणा म्यृतू॒ना मृ॑तू॒नाम् प्री॑णामि । \newline
22. प्री॒णा॒मि॒ तास्ताः प्री॑णामि प्रीणामि॒ ताः । \newline
23. ता मा॑ मा॒ तास्ता मा᳚ । \newline
24. मा॒ प्री॒ताः प्री॒ता मा॑ मा प्री॒ताः । \newline
25. प्री॒ताः प्री॑णन्तु प्रीणन्तु प्री॒ताः प्री॒ताः प्री॑णन्तु । \newline
26. प्री॒ण॒न्तु॒ श॒रद(ग्म्॑) श॒रद॑म् प्रीणन्तु प्रीणन्तु श॒रद᳚म् । \newline
27. श॒रद॑ मृतू॒ना मृ॑तू॒नाꣳ श॒रद(ग्म्॑) श॒रद॑ मृतू॒नाम् । \newline
28. ऋ॒तू॒नाम् प्री॑णामि प्रीणा म्यृतू॒ना मृ॑तू॒नाम् प्री॑णामि । \newline
29. प्री॒णा॒मि॒ सा सा प्री॑णामि प्रीणामि॒ सा । \newline
30. सा मा॑ मा॒ सा सा मा᳚ । \newline
31. मा॒ प्री॒ता प्री॒ता मा॑ मा प्री॒ता । \newline
32. प्री॒ता प्री॑णातु प्रीणातु प्री॒ता प्री॒ता प्री॑णातु । \newline
33. प्री॒णा॒तु॒ हे॒म॒न्त॒शि॒शि॒रौ हे॑मन्तशिशि॒रौ प्री॑णातु प्रीणातु हेमन्तशिशि॒रौ । \newline
34. हे॒म॒न्त॒शि॒शि॒रा वृ॑तू॒ना मृ॑तू॒नाꣳ हे॑मन्तशिशि॒रौ हे॑मन्तशिशि॒रा वृ॑तू॒नाम् । \newline
35. हे॒म॒न्त॒शि॒शि॒राविति॑ हेमन्त - शि॒शि॒रौ । \newline
36. ऋ॒तू॒नाम् प्री॑णामि प्रीणा म्यृतू॒ना मृ॑तू॒नाम् प्री॑णामि । \newline
37. प्री॒णा॒मि॒ तौ तौ प्री॑णामि प्रीणामि॒ तौ । \newline
38. तौ मा॑ मा॒ तौ तौ मा᳚ । \newline
39. मा॒ प्री॒तौ प्री॒तौ मा॑ मा प्री॒तौ । \newline
40. प्री॒तौ प्री॑णीताम् प्रीणीताम् प्री॒तौ प्री॒तौ प्री॑णीताम् । \newline
41. प्री॒णी॒ता॒ म॒ग्नीषोम॑यो र॒ग्नीषोम॑योः प्रीणीताम् प्रीणीता म॒ग्नीषोम॑योः । \newline
42. अ॒ग्नीषोम॑यो र॒ह म॒ह म॒ग्नीषोम॑यो र॒ग्नीषोम॑यो र॒हम् । \newline
43. अ॒ग्नीषोम॑यो॒रित्य॒ग्नी - सोम॑योः । \newline
44. अ॒हम् दे॑वय॒ज्यया॑ देवय॒ज्यया॒ ऽह म॒हम् दे॑वय॒ज्यया᳚ । \newline
45. दे॒व॒य॒ज्यया॒ चक्षु॑ष्मा॒(ग्ग्॒) श्चक्षु॑ष्मान् देवय॒ज्यया॑ देवय॒ज्यया॒ चक्षु॑ष्मान् । \newline
46. दे॒व॒य॒ज्ययेति॑ देव - य॒ज्यया᳚ । \newline
47. चक्षु॑ष्मान् भूयासम् भूयास॒म् चक्षु॑ष्मा॒(ग्ग्॒) श्चक्षु॑ष्मान् भूयासम् । \newline
48. भू॒या॒स॒ म॒ग्ने र॒ग्नेर् भू॑यासम् भूयास म॒ग्नेः । \newline
49. अ॒ग्ने र॒ह म॒ह म॒ग्ने र॒ग्ने र॒हम् । \newline
50. अ॒हम् दे॑वय॒ज्यया॑ देवय॒ज्यया॒ ऽह म॒हम् दे॑वय॒ज्यया᳚ । \newline
51. दे॒व॒य॒ज्यया᳚ ऽन्ना॒दो᳚ ऽन्ना॒दो दे॑वय॒ज्यया॑ देवय॒ज्यया᳚ ऽन्ना॒दः । \newline
52. दे॒व॒य॒ज्ययेति॑ देव - य॒ज्यया᳚ । \newline
53. अ॒न्ना॒दो भू॑यासम् भूयास मन्ना॒दो᳚ ऽन्ना॒दो भू॑यासम् । \newline
54. अ॒न्ना॒द इत्य॑न्न - अ॒दः । \newline
55. भू॒या॒स॒म् दब्धि॒र् दब्धि॑र् भूयासम् भूयास॒म् दब्धिः॑ । \newline

\textbf{Ghana Paata } \newline

1. वा॒चा मा॑ मा वा॒चा वा॒चा मे᳚न्द्रि॒येणे᳚ न्द्रि॒येण॑ मा वा॒चा वा॒चा मे᳚न्द्रि॒येण॑ । \newline
2. मे॒न्द्रि॒येणे᳚ न्द्रि॒येण॑ मा मेन्द्रि॒येणेन्द्रि॒येण॑ मा मेन्द्रि॒येणा । \newline
3. इ॒न्द्रि॒येणेन्द्रि॒येणे᳚ न्द्रि॒येणा वि॑श वि॒शेन्द्रि॒येणे᳚ न्द्रि॒येणा वि॑श । \newline
4. आ वि॑श वि॒शा वि॑श वस॒न्तं ॅव॑स॒न्तं ॅवि॒शा वि॑श वस॒न्तम् । \newline
5. वि॒श॒ व॒स॒न्तं ॅव॑स॒न्तं ॅवि॑श विश वस॒न्त मृ॑तू॒ना मृ॑तू॒नां ॅव॑स॒न्तं ॅवि॑श विश वस॒न्त मृ॑तू॒नाम् । \newline
6. व॒स॒न्त मृ॑तू॒ना मृ॑तू॒नां ॅव॑स॒न्तं ॅव॑स॒न्त मृ॑तू॒नाम् प्री॑णामि प्रीणाम्यृतू॒नां ॅव॑स॒न्तं ॅव॑स॒न्त मृ॑तू॒नाम् प्री॑णामि । \newline
7. ऋ॒तू॒नाम् प्री॑णामि प्रीणाम्यृतू॒ना मृ॑तू॒नाम् प्री॑णामि॒ स स प्री॑णाम्यृतू॒ना मृ॑तू॒नाम् प्री॑णामि॒ सः । \newline
8. प्री॒णा॒मि॒ स स प्री॑णामि प्रीणामि॒ स मा॑ मा॒ स प्री॑णामि प्रीणामि॒ स मा᳚ । \newline
9. स मा॑ मा॒ स स मा᳚ प्री॒तः प्री॒तो मा॒ स स मा᳚ प्री॒तः । \newline
10. मा॒ प्री॒तः प्री॒तो मा॑ मा प्री॒तः प्री॑णातु प्रीणातु प्री॒तो मा॑ मा प्री॒तः प्री॑णातु । \newline
11. प्री॒तः प्री॑णातु प्रीणातु प्री॒तः प्री॒तः प्री॑णातु ग्री॒ष्मम् ग्री॒ष्मम् प्री॑णातु प्री॒तः प्री॒तः प्री॑णातु ग्री॒ष्मम् । \newline
12. प्री॒णा॒तु॒ ग्री॒ष्मम् ग्री॒ष्मम् प्री॑णातु प्रीणातु ग्री॒ष्म मृ॑तू॒ना मृ॑तू॒नाम् ग्री॒ष्मम् प्री॑णातु प्रीणातु ग्री॒ष्म मृ॑तू॒नाम् । \newline
13. ग्री॒ष्म मृ॑तू॒ना मृ॑तू॒नाम् ग्री॒ष्मम् ग्री॒ष्म मृ॑तू॒नाम् प्री॑णामि प्रीणाम्यृतू॒नाम् ग्री॒ष्मम् ग्री॒ष्म मृ॑तू॒नाम् प्री॑णामि । \newline
14. ऋ॒तू॒नाम् प्री॑णामि प्रीणाम्यृतू॒ना मृ॑तू॒नाम् प्री॑णामि॒ स स प्री॑णाम्यृतू॒ना मृ॑तू॒नाम् प्री॑णामि॒ सः । \newline
15. प्री॒णा॒मि॒ स स प्री॑णामि प्रीणामि॒ स मा॑ मा॒ स प्री॑णामि प्रीणामि॒ स मा᳚ । \newline
16. सो मा॑ मा॒ स स मा᳚ प्री॒तः प्री॒तो मा॒ स स मा᳚ प्री॒तः । \newline
17. मा॒ प्री॒तः प्री॒तो मा॑ मा प्री॒तः प्री॑णातु प्रीणातु प्री॒तो मा॑ मा प्री॒तः प्री॑णातु । \newline
18. प्री॒तः प्री॑णातु प्रीणातु प्री॒तः प्री॒तः प्री॑णातु व॒र्॒.षा व॒र्॒.षाः प्री॑णातु प्री॒तः प्री॒तः प्री॑णातु व॒र्॒.षाः । \newline
19. प्री॒णा॒तु॒ व॒र्॒.षा व॒र्॒.षाः प्री॑णातु प्रीणातु व॒र्॒.षा ऋ॑तू॒ना मृ॑तू॒नां ॅव॒र्॒.षाः प्री॑णातु प्रीणातु व॒र्॒.षा ऋ॑तू॒नाम् । \newline
20. व॒र्॒.षा ऋ॑तू॒ना मृ॑तू॒नां ॅव॒र्॒.षा व॒र्॒.षा ऋ॑तू॒नाम् प्री॑णामि प्रीणाम्यृतू॒नां ॅव॒र्॒.षा व॒र्॒.षा ऋ॑तू॒नाम् प्री॑णामि । \newline
21. ऋ॒तू॒नाम् प्री॑णामि प्रीणाम्यृतू॒ना मृ॑तू॒नाम् प्री॑णामि॒ तास्ताः प्री॑णाम्यृतू॒ना मृ॑तू॒नाम् प्री॑णामि॒ ताः । \newline
22. प्री॒णा॒मि॒ तास्ताः प्री॑णामि प्रीणामि॒ ता मा॑ मा॒ ताः प्री॑णामि प्रीणामि॒ ता मा᳚ । \newline
23. ता मा॑ मा॒ तास्ता मा᳚ प्री॒ताः प्री॒ता मा॒ तास्ता मा᳚ प्री॒ताः । \newline
24. मा॒ प्री॒ताः प्री॒ता मा॑ मा प्री॒ताः प्री॑णन्तु प्रीणन्तु प्री॒ता मा॑ मा प्री॒ताः प्री॑णन्तु । \newline
25. प्री॒ताः प्री॑णन्तु प्रीणन्तु प्री॒ताः प्री॒ताः प्री॑णन्तु श॒रद(ग्म्॑) श॒रद॑म् प्रीणन्तु प्री॒ताः प्री॒ताः प्री॑णन्तु श॒रद᳚म् । \newline
26. प्री॒ण॒न्तु॒ श॒रद(ग्म्॑) श॒रद॑म् प्रीणन्तु प्रीणन्तु श॒रद॑ मृतू॒ना मृ॑तू॒नाꣳ श॒रद॑म् प्रीणन्तु प्रीणन्तु श॒रद॑ मृतू॒नाम् । \newline
27. श॒रद॑ मृतू॒ना मृ॑तू॒नाꣳ श॒रद(ग्म्॑) श॒रद॑ मृतू॒नाम् प्री॑णामि प्रीणाम्यृतू॒नाꣳ श॒रद(ग्म्॑) श॒रद॑ मृतू॒नाम् प्री॑णामि । \newline
28. ऋ॒तू॒नाम् प्री॑णामि प्रीणाम्यृतू॒ना मृ॑तू॒नाम् प्री॑णामि॒ सा सा प्री॑णाम्यृतू॒ना मृ॑तू॒नाम् प्री॑णामि॒ सा । \newline
29. प्री॒णा॒मि॒ सा सा प्री॑णामि प्रीणामि॒ सा मा॑ मा॒ सा प्री॑णामि प्रीणामि॒ सा मा᳚ । \newline
30. सा मा॑ मा॒ सा सा मा᳚ प्री॒ता प्री॒ता मा॒ सा सा मा᳚ प्री॒ता । \newline
31. मा॒ प्री॒ता प्री॒ता मा॑ मा प्री॒ता प्री॑णातु प्रीणातु प्री॒ता मा॑ मा प्री॒ता प्री॑णातु । \newline
32. प्री॒ता प्री॑णातु प्रीणातु प्री॒ता प्री॒ता प्री॑णातु हेमन्तशिशि॒रौ हे॑मन्तशिशि॒रौ प्री॑णातु प्री॒ता प्री॒ता प्री॑णातु हेमन्तशिशि॒रौ । \newline
33. प्री॒णा॒तु॒ हे॒म॒न्त॒शि॒शि॒रौ हे॑मन्तशिशि॒रौ प्री॑णातु प्रीणातु हेमन्तशिशि॒रा वृ॑तू॒ना मृ॑तू॒नाꣳ हे॑मन्तशिशि॒रौ प्री॑णातु प्रीणातु हेमन्तशिशि॒रा वृ॑तू॒नाम् । \newline
34. हे॒म॒न्त॒शि॒शि॒रा वृ॑तू॒ना मृ॑तू॒नाꣳ हे॑मन्तशिशि॒रौ हे॑मन्तशिशि॒रा वृ॑तू॒नाम् प्री॑णामि प्रीणाम्यृतू॒नाꣳ हे॑मन्तशिशि॒रौ हे॑मन्तशिशि॒रा वृ॑तू॒नाम् प्री॑णामि । \newline
35. हे॒म॒न्त॒शि॒शि॒राविति॑ हेमन्त - शि॒शि॒रौ । \newline
36. ऋ॒तू॒नाम् प्री॑णामि प्रीणाम्यृतू॒ना मृ॑तू॒नाम् प्री॑णामि॒ तौ तौ प्री॑णाम्यृतू॒ना मृ॑तू॒नाम् प्री॑णामि॒ तौ । \newline
37. प्री॒णा॒मि॒ तौ तौ प्री॑णामि प्रीणामि॒ तौ मा॑ मा॒ तौ प्री॑णामि प्रीणामि॒ तौ मा᳚ । \newline
38. तौ मा॑ मा॒ तौ तौ मा᳚ प्री॒तौ प्री॒तौ मा॒ तौ तौ मा᳚ प्री॒तौ । \newline
39. मा॒ प्री॒तौ प्री॒तौ मा॑ मा प्री॒तौ प्री॑णीताम् प्रीणीताम् प्री॒तौ मा॑ मा प्री॒तौ प्री॑णीताम् । \newline
40. प्री॒तौ प्री॑णीताम् प्रीणीताम् प्री॒तौ प्री॒तौ प्री॑णीता म॒ग्नीषोम॑यो र॒ग्नीषोम॑योः प्रीणीताम् प्री॒तौ प्री॒तौ प्री॑णीता म॒ग्नीषोम॑योः । \newline
41. प्री॒णी॒ता॒ म॒ग्नीषोम॑यो र॒ग्नीषोम॑योः प्रीणीताम् प्रीणीता म॒ग्नीषोम॑योर॒ह म॒ह म॒ग्नीषोम॑योः प्रीणीताम् प्रीणीता म॒ग्नीषोम॑योर॒हम् । \newline
42. अ॒ग्नीषोम॑योर॒ह म॒ह म॒ग्नीषोम॑यो र॒ग्नीषोम॑योर॒हम् दे॑वय॒ज्यया॑ देवय॒ज्यया॒ ऽह म॒ग्नीषोम॑यो र॒ग्नीषोम॑योर॒हम् दे॑वय॒ज्यया᳚ । \newline
43. अ॒ग्नीषोम॑यो॒रित्य॒ग्नी - सोम॑योः । \newline
44. अ॒हम् दे॑वय॒ज्यया॑ देवय॒ज्यया॒ ऽह म॒हम् दे॑वय॒ज्यया॒ चक्षु॑ष्मा॒(ग्ग्॒) श्चक्षु॑ष्मान् देवय॒ज्यया॒ ऽह म॒हम् दे॑वय॒ज्यया॒ चक्षु॑ष्मान् । \newline
45. दे॒व॒य॒ज्यया॒ चक्षु॑ष्मा॒(ग्ग्॒) श्चक्षु॑ष्मान् देवय॒ज्यया॑ देवय॒ज्यया॒ चक्षु॑ष्मान् भूयासम् भूयास॒म् चक्षु॑ष्मान् देवय॒ज्यया॑ देवय॒ज्यया॒ चक्षु॑ष्मान् भूयासम् । \newline
46. दे॒व॒य॒ज्ययेति॑ देव - य॒ज्यया᳚ । \newline
47. चक्षु॑ष्मान् भूयासम् भूयास॒म् चक्षु॑ष्मा॒(ग्ग्॒) श्चक्षु॑ष्मान् भूयास म॒ग्नेर॒ग्नेर् भू॑यास॒म् 
चक्षु॑ष्मा॒(ग्ग्॒) श्चक्षु॑ष्मान् भूयास म॒ग्नेः । \newline
48. भू॒या॒स॒ म॒ग्नेर॒ग्नेर् भू॑यासम् भूयास म॒ग्नेर॒ह म॒ह म॒ग्नेर् भू॑यासम् भूयास म॒ग्नेर॒हम् । \newline
49. अ॒ग्नेर॒ह म॒ह म॒ग्नेर॒ग्नेर॒हम् दे॑वय॒ज्यया॑ देवय॒ज्यया॒ ऽह म॒ग्नेर॒ग्नेर॒हम् दे॑वय॒ज्यया᳚ । \newline
50. अ॒हम् दे॑वय॒ज्यया॑ देवय॒ज्यया॒ ऽह म॒हम् दे॑वय॒ज्यया᳚ ऽन्ना॒दो᳚ ऽन्ना॒दो दे॑वय॒ज्यया॒ ऽह म॒हम् दे॑वय॒ज्यया᳚ ऽन्ना॒दः । \newline
51. दे॒व॒य॒ज्यया᳚ ऽन्ना॒दो᳚ ऽन्ना॒दो दे॑वय॒ज्यया॑ देवय॒ज्यया᳚ ऽन्ना॒दो भू॑यासम् भूयास मन्ना॒दो दे॑वय॒ज्यया॑ देवय॒ज्यया᳚ ऽन्ना॒दो भू॑यासम् । \newline
52. दे॒व॒य॒ज्ययेति॑ देव - य॒ज्यया᳚ । \newline
53. अ॒न्ना॒दो भू॑यासम् भूयास मन्ना॒दो᳚ ऽन्ना॒दो भू॑यास॒म् दब्धि॒र् दब्धि॑र् भूयास मन्ना॒दो᳚ ऽन्ना॒दो भू॑यास॒म् दब्धिः॑ । \newline
54. अ॒न्ना॒द इत्य॑न्न - अ॒दः । \newline
55. भू॒या॒स॒म् दब्धि॒र् दब्धि॑र् भूयासम् भूयास॒म् दब्धि॑रस्यसि॒ दब्धि॑र् भूयासम् भूयास॒म् दब्धि॑रसि । \newline
\pagebreak
\markright{ TS 1.6.2.4  \hfill https://www.vedavms.in \hfill}

\section{ TS 1.6.2.4 }

\textbf{TS 1.6.2.4 } \newline
\textbf{Samhita Paata} \newline

दब्धि॑र॒स्यद॑ब्धो भूयासम॒मुं द॑भेय-म॒ग्नीषोम॑यो-र॒हं दे॑वय॒ज्यया॑ वृत्र॒हा भू॑यासमिन्द्राग्नि॒योर॒हं दे॑वय॒ज्यये᳚न्द्रिया॒व्य॑न्ना॒दो भू॑यास॒मिन्द्र॑स्या॒ऽहं दे॑वय॒ज्यये᳚न्द्रिया॒वी भू॑यासं महे॒न्द्रस्या॒ऽहं दे॑वय॒ज्यया॑ जे॒मानं॑ महि॒मानं॑ गमेयम॒ग्नेः स्वि॑ष्ट॒कृतो॒ऽहं दे॑वय॒ज्यया ऽऽयु॑ष्मान्. य॒ज्ञेन॑ प्रति॒ष्ठां ग॑मेयं ॥ \newline

\textbf{Pada Paata} \newline

दब्धिः॑ । अ॒सि॒ । अद॑ब्धः । भू॒या॒स॒म् । अ॒मुम् । द॒भे॒य॒म् । अ॒ग्नीषोम॑यो॒रित्य॒ग्नी - सोम॑योः । अ॒हम् । दे॒व॒य॒ज्ययेति॑ देव- य॒ज्यया᳚ । वृ॒त्र॒हेति॑ वृत्र - हा । भू॒या॒स॒म् । इ॒न्द्रा॒ग्नि॒योरिती᳚न्द्र - अ॒ग्नि॒योः । अ॒हम् । दे॒व॒य॒ज्ययेति॑ देव - य॒ज्यया᳚ । इ॒न्द्रि॒या॒वी । अ॒न्ना॒द इत्य॑न्न - अ॒दः । भू॒या॒स॒म् । इन्द्र॑स्य । अ॒हम् । दे॒व॒य॒ज्ययेति॑ देव - य॒ज्यया᳚ । इ॒न्द्रि॒या॒वी । भू॒या॒स॒म् । म॒हे॒न्द्रस्येति॑ महा - इ॒न्द्रस्य॑ । अ॒हम् । दे॒व॒य॒ज्ययेति॑ देव - य॒ज्यया᳚ । जे॒मान᳚म् । म॒हि॒मान᳚म् । ग॒मे॒य॒म् । अ॒ग्नेः । स्वि॒ष्ट॒कृत॒ इति॑ स्विष्ट - कृतः॑ । अ॒हम् । दे॒व॒य॒ज्ययेति॑ देव - य॒ज्यया᳚ । आयु॑ष्मान् । य॒ज्ञेन॑ । प्र॒ति॒ष्ठामिति॑ प्रति - स्थाम् । ग॒मे॒य॒म् ॥  \newline


\textbf{Krama Paata} \newline

दब्धि॑रसि । अ॒स्यद॑ब्धः । अद॑ब्धो भूयासम् । भू॒या॒स॒म॒मुम् । अ॒मुम् द॑भेयम् । द॒भे॒य॒म॒ग्नीषोम॑योः । अ॒ग्नीषोम॑योर॒हम् । अ॒ग्नीषोम॑यो॒रित्य॒ग्नी - सोम॑योः । अ॒हम् दे॑वय॒ज्यया᳚ । दे॒व॒य॒ज्यया॑ वृत्र॒हा । दे॒व॒य॒ज्ययेति॑ देव - य॒ज्यया᳚ । वृ॒त्र॒हा भू॑यासम् । वृ॒त्र॒हेति॑ वृत्र - हा । भू॒या॒स॒मि॒न्द्रा॒ग्नि॒योः । इ॒न्द्रा॒ग्नि॒योर॒हम् । इ॒न्द्रा॒ग्नि॒योरिती᳚न्द्र - अ॒ग्नि॒योः । अ॒हम् दे॑वय॒ज्यया᳚ । दे॒व॒य॒ज्यये᳚न्द्रिया॒वी । दे॒व॒य॒ज्ययेति॑ देव - य॒ज्यया᳚ । इ॒न्द्रि॒या॒व्य॑न्ना॒दः । अ॒न्ना॒दो भू॑यासम् । अ॒न्ना॒द इत्य॑न्न - अ॒दः । भू॒या॒स॒मिन्द्र॑स्य । इन्द्र॑स्या॒हम् । अ॒हम् दे॑वय॒ज्यया᳚ । दे॒व॒य॒ज्यये᳚न्द्रिया॒वी । दे॒व॒य॒ज्ययेति॑ देव - य॒ज्यया᳚ । इ॒न्द्रि॒या॒वी भू॑यासम् । भू॒या॒स॒म् म॒हे॒न्द्रस्य॑ । म॒हे॒न्द्रस्या॒हम् । म॒हे॒न्द्रस्येति॑ महा - इ॒न्द्रस्य॑ । अ॒हम् दे॑वय॒ज्यया᳚ । दे॒व॒य॒ज्यया॑ जे॒मान᳚म् । दे॒व॒य॒ज्ययेति॑ देव - य॒ज्यया᳚ । जे॒मान॑म् महि॒मान᳚म् । म॒हि॒मान॑म् गमेयम् । ग॒मे॒य॒म॒ग्नेः । अ॒ग्नेः स्वि॑ष्ट॒कृतः॑ । स्वि॒ष्ट॒कृतो॒ऽहम् । स्वि॒ष्ट॒कृत॒ इति॑ स्विष्ट - कृतः॑ । अ॒हम् दे॑वय॒ज्यया᳚ । दे॒व॒य॒ज्यया ऽऽयु॑ष्मान् । दे॒व॒य॒ज्ययेति॑ देव - य॒ज्यया᳚ । आयु॑ष्मान्. य॒ज्ञेन॑ । य॒ज्ञेन॑ प्रति॒ष्ठाम् । प्र॒ति॒ष्ठाम् ग॑मेयम् । प्र॒ति॒ष्ठामिति॑ प्रति - स्थाम् । ग॒मे॒य॒मिति॑ गमेयम् । \newline

\textbf{Jatai Paata} \newline

1. दब्धि॑ रस्यसि॒ दब्धि॒र् दब्धि॑रसि । \newline
2. अ॒स्यद॒ब्धो ऽद॑ब्धो ऽस्य॒स्यद॑ब्धः । \newline
3. अद॑ब्धो भूयासम् भूयास॒ मद॒ब्धो ऽद॑ब्धो भूयासम् । \newline
4. भू॒या॒स॒ म॒मु म॒मुम् भू॑यासम् भूयास म॒मुम् । \newline
5. अ॒मुम् द॑भेयम् दभेय म॒मु म॒मुम् द॑भेयम् । \newline
6. द॒भे॒य॒ म॒ग्नीषोम॑यो र॒ग्नीषोम॑योर् दभेयम् दभेय म॒ग्नीषोम॑योः । \newline
7. अ॒ग्नीषोम॑योर॒ह म॒ह म॒ग्नीषोम॑यो र॒ग्नीषोम॑योर॒हम् । \newline
8. अ॒ग्नीषोम॑यो॒रित्य॒ग्नी - सोम॑योः । \newline
9. अ॒हम् दे॑वय॒ज्यया॑ देवय॒ज्यया॒ ऽह म॒हम् दे॑वय॒ज्यया᳚ । \newline
10. दे॒व॒य॒ज्यया॑ वृत्र॒हा वृ॑त्र॒हा दे॑वय॒ज्यया॑ देवय॒ज्यया॑ वृत्र॒हा । \newline
11. दे॒व॒य॒ज्ययेति॑ देव - य॒ज्यया᳚ । \newline
12. वृ॒त्र॒हा भू॑यासम् भूयासं ॅवृत्र॒हा वृ॑त्र॒हा भू॑यासम् । \newline
13. वृ॒त्र॒हेति॑ वृत्र - हा । \newline
14. भू॒या॒स॒ मि॒न्द्रा॒ग्नि॒यो रि॑न्द्राग्नि॒योर् भू॑यासम् भूयास मिन्द्राग्नि॒योः । \newline
15. इ॒न्द्रा॒ग्नि॒योर॒ह म॒ह मि॑न्द्राग्नि॒यो रि॑न्द्राग्नि॒योर॒हम् । \newline
16. इ॒न्द्रा॒ग्नि॒योरिती᳚न्द्र - अ॒ग्नि॒योः । \newline
17. अ॒हम् दे॑वय॒ज्यया॑ देवय॒ज्यया॒ ऽह म॒हम् दे॑वय॒ज्यया᳚ । \newline
18. दे॒व॒य॒ज्य ये᳚न्द्रिया॒वीन्द्रि॑या॒वी दे॑वय॒ज्यया॑ देवय॒ज्य ये᳚न्द्रिया॒वी । \newline
19. दे॒व॒य॒ज्ययेति॑ देव - य॒ज्यया᳚ । \newline
20. इ॒न्द्रि॒या॒ व्य॑न्ना॒दो᳚ ऽन्ना॒द इ॑न्द्रिया॒ वीन्द्रि॑या॒ व्य॑न्ना॒दः । \newline
21. अ॒न्ना॒दो भू॑यासम् भूयास मन्ना॒दो᳚ ऽन्ना॒दो भू॑यासम् । \newline
22. अ॒न्ना॒द इत्य॑न्न - अ॒दः । \newline
23. भू॒या॒स॒ मिन्द्र॒स्ये न्द्र॑स्य भूयासम् भूयास॒ मिन्द्र॑स्य । \newline
24. इन्द्र॑स्या॒ह म॒ह मिन्द्र॒स्ये न्द्र॑स्या॒हम् । \newline
25. अ॒हम् दे॑वय॒ज्यया॑ देवय॒ज्यया॒ ऽह म॒हम् दे॑वय॒ज्यया᳚ । \newline
26. दे॒व॒य॒ज्य ये᳚न्द्रिया॒ वीन्द्रि॑या॒वी दे॑वय॒ज्यया॑ देवय॒ज्य ये᳚न्द्रिया॒वी । \newline
27. दे॒व॒य॒ज्ययेति॑ देव - य॒ज्यया᳚ । \newline
28. इ॒न्द्रि॒या॒वी भू॑यासम् भूयास मिन्द्रिया॒ वीन्द्रि॑या॒वी भू॑यासम् । \newline
29. भू॒या॒स॒म् म॒हे॒न्द्रस्य॑ महे॒न्द्रस्य॑ भूयासम् भूयासम् महे॒न्द्रस्य॑ । \newline
30. म॒हे॒न्द्रस्या॒ह म॒हम् म॑हे॒न्द्रस्य॑ महे॒न्द्रस्या॒हम् । \newline
31. म॒हे॒न्द्रस्येति॑ महा - इ॒न्द्रस्य॑ । \newline
32. अ॒हम् दे॑वय॒ज्यया॑ देवय॒ज्यया॒ ऽह म॒हम् दे॑वय॒ज्यया᳚ । \newline
33. दे॒व॒य॒ज्यया॑ जे॒मान॑म् जे॒मान॑म् देवय॒ज्यया॑ देवय॒ज्यया॑ जे॒मान᳚म् । \newline
34. दे॒व॒य॒ज्ययेति॑ देव - य॒ज्यया᳚ । \newline
35. जे॒मान॑म् महि॒मान॑म् महि॒मान॑म् जे॒मान॑म् जे॒मान॑म् महि॒मान᳚म् । \newline
36. म॒हि॒मान॑म् गमेयम् गमेयम् महि॒मान॑म् महि॒मान॑म् गमेयम् । \newline
37. ग॒मे॒य॒ म॒ग्ने र॒ग्नेर् ग॑मेयम् गमेय म॒ग्नेः । \newline
38. अ॒ग्नेः स्वि॑ष्ट॒कृतः॑ स्विष्ट॒कृतो॒ ऽग्नेर॒ग्नेः स्वि॑ष्ट॒कृतः॑ । \newline
39. स्वि॒ष्ट॒कृतो॒ ऽह म॒हꣳ स्वि॑ष्ट॒कृतः॑ स्विष्ट॒कृतो॒ ऽहम् । \newline
40. स्वि॒ष्ट॒कृत॒ इति॑ स्विष्ट - कृतः॑ । \newline
41. अ॒हम् दे॑वय॒ज्यया॑ देवय॒ज्यया॒ ऽह म॒हम् दे॑वय॒ज्यया᳚ । \newline
42. दे॒व॒य॒ज्यया ऽऽयु॑ष्मा॒ नायु॑ष्मान् देवय॒ज्यया॑ देवय॒ज्यया ऽऽयु॑ष्मान् । \newline
43. दे॒व॒य॒ज्ययेति॑ देव - य॒ज्यया᳚ । \newline
44. आयु॑ष्मान्. य॒ज्ञेन॑ य॒ज्ञेनायु॑ष्मा॒ नायु॑ष्मान्. य॒ज्ञेन॑ । \newline
45. य॒ज्ञेन॑ प्रति॒ष्ठाम् प्र॑ति॒ष्ठां ॅय॒ज्ञेन॑ य॒ज्ञेन॑ प्रति॒ष्ठाम् । \newline
46. प्र॒ति॒ष्ठाम् ग॑मेयम् गमेयम् प्रति॒ष्ठाम् प्र॑ति॒ष्ठाम् ग॑मेयम् । \newline
47. प्र॒ति॒ष्ठामिति॑ प्रति - स्थाम् । \newline
48. ग॒मे॒य॒मिति॑ गमेयम् । \newline

\textbf{Ghana Paata } \newline

1. दब्धि॑ रस्यसि॒ दब्धि॒र् दब्धि॑ र॒स्यद॒ब्धो ऽद॑ब्धो ऽसि॒ दब्धि॒र् दब्धि॑ र॒स्यद॑ब्धः । \newline
2. अ॒स्यद॒ब्धो ऽद॑ब्धो ऽस्य॒स्यद॑ब्धो भूयासम् भूयास॒ मद॑ब्धो ऽस्य॒स्यद॑ब्धो भूयासम् । \newline
3. अद॑ब्धो भूयासम् भूयास॒ मद॒ब्धो ऽद॑ब्धो भूयास म॒मु म॒मुम् भू॑यास॒ मद॒ब्धो ऽद॑ब्धो भूयास म॒मुम् । \newline
4. भू॒या॒स॒ म॒मु म॒मुम् भू॑यासम् भूयास म॒मुम् द॑भेयम् दभेय म॒मुम् भू॑यासम् भूयास म॒मुम् द॑भेयम् । \newline
5. अ॒मुम् द॑भेयम् दभेय म॒मु म॒मुम् द॑भेय म॒ग्नीषोम॑यो र॒ग्नीषोम॑योर् दभेय म॒मु म॒मुम् द॑भेय म॒ग्नीषोम॑योः । \newline
6. द॒भे॒य॒ म॒ग्नीषोम॑यो र॒ग्नीषोम॑योर् दभेयम् दभेय म॒ग्नीषोम॑योर॒ह म॒ह म॒ग्नीषोम॑योर् दभेयम् दभेय म॒ग्नीषोम॑योर॒हम् । \newline
7. अ॒ग्नीषोम॑योर॒ह म॒ह म॒ग्नीषोम॑यो र॒ग्नीषोम॑योर॒हम् दे॑वय॒ज्यया॑ देवय॒ज्यया॒ ऽह म॒ग्नीषोम॑यो र॒ग्नीषोम॑योर॒हम् दे॑वय॒ज्यया᳚ । \newline
8. अ॒ग्नीषोम॑यो॒रित्य॒ग्नी - सोम॑योः । \newline
9. अ॒हम् दे॑वय॒ज्यया॑ देवय॒ज्यया॒ ऽह म॒हम् दे॑वय॒ज्यया॑ वृत्र॒हा वृ॑त्र॒हा दे॑वय॒ज्यया॒ ऽह म॒हम् दे॑वय॒ज्यया॑ वृत्र॒हा । \newline
10. दे॒व॒य॒ज्यया॑ वृत्र॒हा वृ॑त्र॒हा दे॑वय॒ज्यया॑ देवय॒ज्यया॑ वृत्र॒हा भू॑यासम् भूयासं ॅवृत्र॒हा दे॑वय॒ज्यया॑ देवय॒ज्यया॑ वृत्र॒हा भू॑यासम् । \newline
11. दे॒व॒य॒ज्ययेति॑ देव - य॒ज्यया᳚ । \newline
12. वृ॒त्र॒हा भू॑यासम् भूयासं ॅवृत्र॒हा वृ॑त्र॒हा भू॑यास मिन्द्राग्नि॒यो रि॑न्द्राग्नि॒योर् भू॑यासं ॅवृत्र॒हा वृ॑त्र॒हा भू॑यास मिन्द्राग्नि॒योः । \newline
13. वृ॒त्र॒हेति॑ वृत्र - हा । \newline
14. भू॒या॒स॒ मि॒न्द्रा॒ग्नि॒यो रि॑न्द्राग्नि॒योर् भू॑यासम् भूयास मिन्द्राग्नि॒योर॒ह म॒ह मि॑न्द्राग्नि॒योर् भू॑यासम् भूयास मिन्द्राग्नि॒योर॒हम् । \newline
15. इ॒न्द्रा॒ग्नि॒योर॒ह म॒ह मि॑न्द्राग्नि॒यो रि॑न्द्राग्नि॒योर॒हम् दे॑वय॒ज्यया॑ देवय॒ज्यया॒ ऽह मि॑न्द्राग्नि॒यो रि॑न्द्राग्नि॒योर॒हम् दे॑वय॒ज्यया᳚ । \newline
16. इ॒न्द्रा॒ग्नि॒योरिती᳚न्द्र - अ॒ग्नि॒योः । \newline
17. अ॒हम् दे॑वय॒ज्यया॑ देवय॒ज्यया॒ ऽह म॒हम् दे॑वय॒ज्य ये᳚न्द्रिया॒वीन्द्रि॑या॒वी दे॑वय॒ज्यया॒ ऽह म॒हम् दे॑वय॒ज्यये᳚न्द्रिया॒वी । \newline
18. दे॒व॒य॒ज्य ये᳚न्द्रिया॒वीन्द्रि॑या॒वी दे॑वय॒ज्यया॑ देवय॒ज्य ये᳚न्द्रिया॒व्य॑न्ना॒दो᳚ ऽन्ना॒द इ॑न्द्रिया॒वी दे॑वय॒ज्यया॑ देवय॒ज्य ये᳚न्द्रिया॒व्य॑न्ना॒दः । \newline
19. दे॒व॒य॒ज्ययेति॑ देव - य॒ज्यया᳚ । \newline
20. इ॒न्द्रि॒या॒व्य॑न्ना॒दो᳚ ऽन्ना॒द इ॑न्द्रिया॒वी न्द्रि॑या॒व्य॑न्ना॒दो भू॑यासम् भूयास मन्ना॒द इ॑न्द्रिया॒वी न्द्रि॑या॒व्य॑न्ना॒दो भू॑यासम् । \newline
21. अ॒न्ना॒दो भू॑यासम् भूयास मन्ना॒दो᳚ ऽन्ना॒दो भू॑यास॒ मिन्द्र॒स्ये न्द्र॑स्य भूयास मन्ना॒दो᳚ ऽन्ना॒दो भू॑यास॒ मिन्द्र॑स्य । \newline
22. अ॒न्ना॒द इत्य॑न्न - अ॒दः । \newline
23. भू॒या॒स॒ मिन्द्र॒स्ये न्द्र॑स्य भूयासम् भूयास॒ मिन्द्र॑स्या॒ह म॒ह मिन्द्र॑स्य भूयासम् भूयास॒ मिन्द्र॑स्या॒हम् । \newline
24. इन्द्र॑स्या॒ह म॒ह मिन्द्र॒स्ये न्द्र॑स्या॒हम् दे॑वय॒ज्यया॑ देवय॒ज्यया॒ ऽह मिन्द्र॒स्ये न्द्र॑स्या॒हम् दे॑वय॒ज्यया᳚ । \newline
25. अ॒हम् दे॑वय॒ज्यया॑ देवय॒ज्यया॒ ऽह म॒हम् दे॑वय॒ज्य ये᳚न्द्रिया॒वीन्द्रि॑या॒वी दे॑वय॒ज्यया॒ ऽह म॒हम् दे॑वय॒ज्यये᳚न्द्रिया॒वी । \newline
26. दे॒व॒य॒ज्य ये᳚न्द्रिया॒वीन्द्रि॑या॒वी दे॑वय॒ज्यया॑ देवय॒ज्य ये᳚न्द्रिया॒वी भू॑यासम् भूयास मिन्द्रिया॒वी दे॑वय॒ज्यया॑ देवय॒ज्य ये᳚न्द्रिया॒वी भू॑यासम् । \newline
27. दे॒व॒य॒ज्ययेति॑ देव - य॒ज्यया᳚ । \newline
28. इ॒न्द्रि॒या॒वी भू॑यासम् भूयास मिन्द्रिया॒वी न्द्रि॑या॒वी भू॑यासम् महे॒न्द्रस्य॑ महे॒न्द्रस्य॑ भूयास मिन्द्रिया॒वी न्द्रि॑या॒वी भू॑यासम् महे॒न्द्रस्य॑ । \newline
29. भू॒या॒स॒म् म॒हे॒न्द्रस्य॑ महे॒न्द्रस्य॑ भूयासम् भूयासम् महे॒न्द्रस्या॒ह म॒हम् म॑हे॒न्द्रस्य॑ भूयासम् भूयासम् महे॒न्द्रस्या॒हम् । \newline
30. म॒हे॒न्द्रस्या॒ह म॒हम् म॑हे॒न्द्रस्य॑ महे॒न्द्रस्या॒हम् दे॑वय॒ज्यया॑ देवय॒ज्यया॒ ऽहम् म॑हे॒न्द्रस्य॑ महे॒न्द्रस्या॒हम् दे॑वय॒ज्यया᳚ । \newline
31. म॒हे॒न्द्रस्येति॑ महा - इ॒न्द्रस्य॑ । \newline
32. अ॒हम् दे॑वय॒ज्यया॑ देवय॒ज्यया॒ ऽह म॒हम् दे॑वय॒ज्यया॑ जे॒मान॑म् जे॒मान॑म् देवय॒ज्यया॒ ऽह म॒हम् दे॑वय॒ज्यया॑ जे॒मान᳚म् । \newline
33. दे॒व॒य॒ज्यया॑ जे॒मान॑म् जे॒मान॑म् देवय॒ज्यया॑ देवय॒ज्यया॑ जे॒मान॑म् महि॒मान॑म् महि॒मान॑म् जे॒मान॑म् देवय॒ज्यया॑ देवय॒ज्यया॑ जे॒मान॑म् महि॒मान᳚म् । \newline
34. दे॒व॒य॒ज्ययेति॑ देव - य॒ज्यया᳚ । \newline
35. जे॒मान॑म् महि॒मान॑म् महि॒मान॑म् जे॒मान॑म् जे॒मान॑म् महि॒मान॑म् गमेयम् गमेयम् महि॒मान॑म् जे॒मान॑म् जे॒मान॑म् महि॒मान॑म् गमेयम् । \newline
36. म॒हि॒मान॑म् गमेयम् गमेयम् महि॒मान॑म् महि॒मान॑म् गमेय म॒ग्नेर॒ग्नेर् ग॑मेयम् महि॒मान॑म् महि॒मान॑म् गमेय म॒ग्नेः । \newline
37. ग॒मे॒य॒ म॒ग्नेर॒ग्नेर् ग॑मेयम् गमेय म॒ग्नेः स्वि॑ष्ट॒कृतः॑ स्विष्ट॒कृतो॒ ऽग्नेर् ग॑मेयम् गमेय म॒ग्नेः स्वि॑ष्ट॒कृतः॑ । \newline
38. अ॒ग्नेः स्वि॑ष्ट॒कृतः॑ स्विष्ट॒कृतो॒ ऽग्नेर॒ग्नेः स्वि॑ष्ट॒कृतो॒ ऽह म॒हꣳ स्वि॑ष्ट॒कृतो॒ ऽग्नेर॒ग्नेः स्वि॑ष्ट॒कृतो॒ ऽहम् । \newline
39. स्वि॒ष्ट॒कृतो॒ ऽह म॒हꣳ स्वि॑ष्ट॒कृतः॑ स्विष्ट॒कृतो॒ ऽहम् दे॑वय॒ज्यया॑ देवय॒ज्यया॒ ऽहꣳ स्वि॑ष्ट॒कृतः॑ स्विष्ट॒कृतो॒ ऽहम् दे॑वय॒ज्यया᳚ । \newline
40. स्वि॒ष्ट॒कृत॒ इति॑ स्विष्ट - कृतः॑ । \newline
41. अ॒हम् दे॑वय॒ज्यया॑ देवय॒ज्यया॒ ऽह म॒हम् दे॑वय॒ज्यया ऽऽयु॑ष्मा॒ नायु॑ष्मान् देवय॒ज्यया॒ ऽह म॒हम् दे॑वय॒ज्यया ऽऽयु॑ष्मान् । \newline
42. दे॒व॒य॒ज्यया ऽऽयु॑ष्मा॒ नायु॑ष्मान् देवय॒ज्यया॑ देवय॒ज्यया ऽऽयु॑ष्मान्. य॒ज्ञेन॑ य॒ज्ञेनायु॑ष्मान् देवय॒ज्यया॑ देवय॒ज्यया ऽऽयु॑ष्मान्. य॒ज्ञेन॑ । \newline
43. दे॒व॒य॒ज्ययेति॑ देव - य॒ज्यया᳚ । \newline
44. आयु॑ष्मान्. य॒ज्ञेन॑ य॒ज्ञेनायु॑ष्मा॒ नायु॑ष्मान्. य॒ज्ञेन॑ प्रति॒ष्ठाम् प्र॑ति॒ष्ठां ॅय॒ज्ञेनायु॑ष्मा॒ नायु॑ष्मान्. य॒ज्ञेन॑ प्रति॒ष्ठाम् । \newline
45. य॒ज्ञेन॑ प्रति॒ष्ठाम् प्र॑ति॒ष्ठां ॅय॒ज्ञेन॑ य॒ज्ञेन॑ प्रति॒ष्ठाम् ग॑मेयम् गमेयम् प्रति॒ष्ठां ॅय॒ज्ञेन॑ य॒ज्ञेन॑ प्रति॒ष्ठाम् ग॑मेयम् । \newline
46. प्र॒ति॒ष्ठाम् ग॑मेयम् गमेयम् प्रति॒ष्ठाम् प्र॑ति॒ष्ठाम् ग॑मेयम् । \newline
47. प्र॒ति॒ष्ठामिति॑ प्रति - स्थाम् । \newline
48. ग॒मे॒य॒मिति॑ गमेयम् । \newline
\pagebreak
\markright{ TS 1.6.3.1  \hfill https://www.vedavms.in \hfill}

\section{ TS 1.6.3.1 }

\textbf{TS 1.6.3.1 } \newline
\textbf{Samhita Paata} \newline

अ॒ग्निर्मा॒ दुरि॑ष्टात् पातु सवि॒ताऽघशꣳ॑सा॒द्यो मेऽन्ति॑ दू॒रे॑ऽराती॒यति॒ तमे॒तेन॑ जेषꣳ॒॒ सुरू॑पवर्.षवर्ण॒ एही॒मान् भ॒द्रान् दुर्याꣳ॑ अ॒भ्येहि॒ मामनु॑व्रता॒ न्यु॑ शी॒र्॒.षाणि॑ मृढ्व॒मिड॒ एह्यदि॑त॒ एहि॒ सर॑स्व॒त्येहि॒ रन्ति॑रसि॒ रम॑तिरसि सू॒नर्य॑सि॒ जुष्टे॒ जुष्टिं॑ तेऽशी॒योप॑हूत उपह॒वं - [ ] \newline

\textbf{Pada Paata} \newline

अ॒ग्निः । मा॒ । दुरि॑ष्टा॒दिति॒ दुः - इ॒ष्टा॒त् । पा॒तु॒ । स॒वि॒ता । अ॒घशꣳ॑सा॒दित्य॒घ-शꣳ॒॒सा॒त् । यः । मे॒ । अन्ति॑ । दू॒रे । अ॒रा॒ती॒यति॑ । तम् । ए॒तेन॑ । जे॒ष॒म् । सुरू॑पवर्.षवर्ण॒ इति॒ सुरू॑प - व॒र्॒.ष॒व॒र्णे॒ । एति॑ । इ॒हि॒ । इ॒मान् । भ॒द्रान् । दुर्यान्॑ । अ॒भि । एति॑ । इ॒हि॒ । माम् । अनु॑व्र॒तेत्यनु॑-व्र॒ता॒ । नीति॑ । उ॒ । शी॒र्.॒षाणि॑ । मृ॒ढ्व॒म् । इडे᳚ । एति॑ । इ॒हि॒ । अदि॑ते । एति॑ । इ॒हि॒ । सर॑स्वति । एति॑ । इ॒हि॒ । रन्तिः॑ । अ॒सि॒ । रम॑तिः । अ॒सि॒ । सू॒नरी᳚ । अ॒सि॒ । जुष्टे᳚ । जुष्टि᳚म् । ते॒ । अ॒शी॒य॒ । उप॑हूत॒ इत्युप॑ - हू॒ते॒ । उ॒प॒ह॒वमित्यु॑प - ह॒वम् ।  \newline


\textbf{Krama Paata} \newline

अ॒ग्निर् मा᳚ । मा॒ दुरि॑ष्टात् । दुरि॑ष्टात् पातु । दुरि॑ष्टा॒दिति॒ दुः - इ॒ष्टा॒त्॒ । पा॒तु॒ स॒वि॒ता । स॒वि॒ताऽघशꣳ॑सात् । अ॒घशꣳ॑सा॒द् यः । अ॒घशꣳ॑सा॒दित्य॒घ - शꣳ॒॒सा॒त्॒ । यो मे᳚ । मेऽन्ति॑ । अन्ति॑ दू॒रे । दू॒रे॑ ऽराती॒यति॑ । अ॒रा॒ती॒यति॒ तम् । तमे॒तेन॑ । ए॒तेन॑ जेषम् । जे॒षꣳ॒॒ सुरू॑पवर्.षवर्णे । सुरू॑पवर्.षवर्ण॒ आ । सुरू॑पवर्.षवर्ण॒ इति॒ सुरू॑प - व॒र्.ष॒व॒र्णे॒ । एहि॑ । इ॒ही॒मान् । इ॒मान्,भ॒द्रान् । भ॒द्रान्,दुर्यान्॑ । दुर्याꣳ॑ अ॒भि । अ॒भ्या । एहि॑ । इ॒हि॒ माम् । मामनु॑व्रता । अनु॑व्रता॒ नि । अनु॑व्र॒तेत्यनु॑ - व्र॒ता॒ । न्यु॑ । उ॒ शी॒र्॒.षाणि॑ । शी॒र्॒.षाणि॑ मृढ्वम् । मृ॒ढ्व॒मीडे᳚ । इड॒ आ । एहि॑ । इ॒ह्यदि॑ते । अदि॑त॒ आ । एहि॑ । इ॒हि॒ सर॑स्वति । सर॑स्व॒त्या । एहि॑ । इ॒हि॒ रन्तिः॑ । रन्ति॑रसि । अ॒सि॒ रम॑तिः । रम॑तिरसि । अ॒सि॒ सू॒नरी᳚ । सू॒नर्य॑सि । अ॒सि॒ जुष्टे᳚ । जुष्टे॒ जुष्टि᳚म् । जुष्टि॑म् ते । ते॒ऽशी॒य॒ । अ॒शी॒योप॑हूते । उप॑होत उपह॒वम् । उप॑हूत॒ इत्युप॑ - हू॒ते॒ । उ॒प॒ह॒वम् ते᳚ । उ॒प॒ह॒वमित्यु॑प - ह॒वम् \newline

\textbf{Jatai Paata} \newline

1. अ॒ग्निर् मा॑ मा॒ ऽग्नि र॒ग्निर् मा᳚ । \newline
2. मा॒ दुरि॑ष्टा॒द् दुरि॑ष्टान् मा मा॒ दुरि॑ष्टात् । \newline
3. दुरि॑ष्टात् पातु पातु॒ दुरि॑ष्टा॒द् दुरि॑ष्टात् पातु । \newline
4. दुरि॑ष्टा॒दिति॒ दुः - इ॒ष्टा॒त् । \newline
5. पा॒तु॒ स॒वि॒ता स॑वि॒ता पा॑तु पातु सवि॒ता । \newline
6. स॒वि॒ता ऽघश(ग्म्॑)सा द॒घश(ग्म्॑)साथ् सवि॒ता स॑वि॒ता ऽघश(ग्म्॑)सात् । \newline
7. अ॒घश(ग्म्॑)सा॒द् यो यो॑ ऽघश(ग्म्॑)सा द॒घश(ग्म्॑)सा॒द् यः । \newline
8. अ॒घश(ग्म्॑)सा॒दित्य॒घ - श॒(ग्म्॒)सा॒त् । \newline
9. यो मे॑ मे॒ यो यो मे᳚ । \newline
10. मे ऽन्त्यन्ति॑ मे॒ मे ऽन्ति॑ । \newline
11. अन्ति॑ दू॒रे दू॒रे ऽन्त्यन्ति॑ दू॒रे । \newline
12. दू॒रे॑ ऽराती॒यत्य॑ राती॒यति॑ दू॒रे दू॒रे॑ ऽराती॒यति॑ । \newline
13. अ॒रा॒ती॒यति॒ तम् त म॑राती॒य त्य॑राती॒यति॒ तम् । \newline
14. त मे॒ते नै॒तेन॒ तम् त मे॒तेन॑ । \newline
15. ए॒तेन॑ जेषम् जेष मे॒ते नै॒तेन॑ जेषम् । \newline
16. जे॒ष॒(ग्म्॒) सुरू॑पवर्.षवर्णे॒ सुरू॑पवर्.षवर्णे जेषम् जेष॒(ग्म्॒) सुरू॑पवर्.षवर्णे । \newline
17. सुरू॑पवर्.षवर्ण॒ आ सुरू॑पवर्.षवर्णे॒ सुरू॑पवर्.षवर्ण॒ आ । \newline
18. सुरू॑पवर्.षवर्ण॒ इति॒ सुरू॑प - व॒र्॒.ष॒व॒र्णे॒ । \newline
19. एही॒ह्येहि॑ । \newline
20. इ॒ही॒मा नि॒मा नि॑हीही॒मान् । \newline
21. इ॒मान् भ॒द्रान् भ॒द्रा नि॒मा नि॒मान् भ॒द्रान् । \newline
22. भ॒द्रान् दुर्या॒न् दुर्या᳚न् भ॒द्रान् भ॒द्रान् दुर्यान्॑ । \newline
23. दुर्या(ग्म्॑) अ॒भ्य॑भि दुर्या॒न् दुर्या(ग्म्॑) अ॒भि । \newline
24. अ॒भ्या ऽभ्य॑भ्या । \newline
25. एही॒ह्येहि॑ । \newline
26. इ॒हि॒ माम् मा मि॑हीहि॒ माम् । \newline
27. मा मनु॑व्र॒ता ऽनु॑व्रता॒ माम् मा मनु॑व्रता । \newline
28. अनु॑व्रता॒ नि न्यनु॑व्र॒ता ऽनु॑व्रता॒ नि । \newline
29. अनु॑व्र॒तेत्यनु॑ - व्र॒ता॒ । \newline
30. न्यु॑ वु॒ नि न्यु॑ । \newline
31. उ॒ शी॒र्॒.षाणि॑ शी॒र्॒.षाण्यु॑ वु शी॒र्॒.षाणि॑ । \newline
32. शी॒र्॒.षाणि॑ मृढ्वम् मृढ्वꣳ शी॒र्॒.षाणि॑ शी॒र्॒.षाणि॑ मृढ्वम् । \newline
33. मृ॒ढ्व॒ मिड॒ इडे॑ मृढ्वम् मृढ्व॒ मिडे᳚ । \newline
34. इड॒ एड॒ इड॒ आ । \newline
35. एही॒ह्येहि॑ । \newline
36. इ॒ह्यदि॒ते ऽदि॑त इही॒ह्यदि॑ते । \newline
37. अदि॑त॒ आ ऽदि॒ते ऽदि॑त॒ आ । \newline
38. एही॒ह्येहि॑ । \newline
39. इ॒हि॒ सर॑स्वति॒ सर॑स्वतीहीहि॒ सर॑स्वति । \newline
40. सर॑स्व॒त्या सर॑स्वति॒ सर॑स्व॒त्या । \newline
41. एही॒ह्येहि॑ । \newline
42. इ॒हि॒ रन्ती॒ रन्ति॑ रिहीहि॒ रन्तिः॑ । \newline
43. रन्ति॑ रस्यसि॒ रन्ती॒ रन्ति॑रसि । \newline
44. अ॒सि॒ रम॑ती॒ रम॑ति रस्यसि॒ रम॑तिः । \newline
45. रम॑ति रस्यसि॒ रम॑ती॒ रम॑तिरसि । \newline
46. अ॒सि॒ सू॒नरी॑ सू॒नर्य॑स्यसि सू॒नरी᳚ । \newline
47. सू॒नर्य॑स्यसि सू॒नरी॑ सू॒नर्य॑सि । \newline
48. अ॒सि॒ जुष्टे॒ जुष्टे᳚ ऽस्यसि॒ जुष्टे᳚ । \newline
49. जुष्टे॒ जुष्टि॒म् जुष्टि॒म् जुष्टे॒ जुष्टे॒ जुष्टि᳚म् । \newline
50. जुष्टि॑म् ते ते॒ जुष्टि॒म् जुष्टि॑म् ते । \newline
51. ते॒ ऽशी॒या॒शी॒य॒ ते॒ ते॒ ऽशी॒य॒ । \newline
52. अ॒शी॒योप॑हूत॒ उप॑हूते ऽशीया शी॒योप॑हूते । \newline
53. उप॑हूत उपह॒व मु॑पह॒व मुप॑हूत॒ उप॑हूत उपह॒वम् । \newline
54. उप॑हूत॒ इत्युप॑ - हू॒ते॒ । \newline
55. उ॒प॒ह॒वम् ते॑ त उपह॒व मु॑पह॒वम् ते᳚ । \newline
56. उ॒प॒ह॒वमित्यु॑प - ह॒वम् । \newline

\textbf{Ghana Paata } \newline

1. अ॒ग्निर् मा॑ मा॒ ऽग्निर॒ग्निर् मा॒ दुरि॑ष्टा॒द् दुरि॑ष्टान् मा॒ ऽग्निर॒ग्निर् मा॒ दुरि॑ष्टात् । \newline
2. मा॒ दुरि॑ष्टा॒द् दुरि॑ष्टान् मा मा॒ दुरि॑ष्टात् पातु पातु॒ दुरि॑ष्टान् मा मा॒ दुरि॑ष्टात् पातु । \newline
3. दुरि॑ष्टात् पातु पातु॒ दुरि॑ष्टा॒द् दुरि॑ष्टात् पातु सवि॒ता स॑वि॒ता पा॑तु॒ दुरि॑ष्टा॒द् दुरि॑ष्टात् पातु सवि॒ता । \newline
4. दुरि॑ष्टा॒दिति॒ दुः - इ॒ष्टा॒त् । \newline
5. पा॒तु॒ स॒वि॒ता स॑वि॒ता पा॑तु पातु सवि॒ता ऽघश(ग्म्॑)सा द॒घश(ग्म्॑)साथ् सवि॒ता पा॑तु पातु सवि॒ता ऽघश(ग्म्॑)सात् । \newline
6. स॒वि॒ता ऽघश(ग्म्॑)सा द॒घश(ग्म्॑)साथ् सवि॒ता स॑वि॒ता ऽघश(ग्म्॑)सा॒द् यो यो॑ ऽघश(ग्म्॑)साथ् सवि॒ता स॑वि॒ता ऽघश(ग्म्॑)सा॒द् यः । \newline
7. अ॒घश(ग्म्॑)सा॒द् यो यो॑ ऽघश(ग्म्॑)सा द॒घश(ग्म्॑)सा॒द् यो मे॑ मे यो॒ ऽघश(ग्म्॑)सा द॒घश(ग्म्॑)सा॒द् यो मे᳚ । \newline
8. अ॒घश(ग्म्॑)सा॒दित्य॒घ - श॒(ग्म्॒)सा॒त् । \newline
9. यो मे॑ मे॒ यो यो मे ऽन्त्यन्ति॑ मे॒ यो यो मे ऽन्ति॑ । \newline
10. मे ऽन्त्यन्ति॑ मे॒ मे ऽन्ति॑ दू॒रे दू॒रे ऽन्ति॑ मे॒ मे ऽन्ति॑ दू॒रे । \newline
11. अन्ति॑ दू॒रे दू॒रे ऽन्त्यन्ति॑ दू॒रे॑ ऽराती॒यत्य॑राती॒यति॑ दू॒रे ऽन्त्यन्ति॑ दू॒रे॑ ऽराती॒यति॑ । \newline
12. दू॒रे॑ ऽराती॒यत्य॑राती॒यति॑ दू॒रे दू॒रे॑ ऽराती॒यति॒ तम् त म॑राती॒यति॑ दू॒रे दू॒रे॑ ऽराती॒यति॒ तम् । \newline
13. अ॒रा॒ती॒यति॒ तम् त म॑राती॒यत्य॑राती॒यति॒ त मे॒तेनै॒तेन॒ त म॑राती॒यत्य॑राती॒यति॒ त मे॒तेन॑ । \newline
14. त मे॒तेनै॒तेन॒ तम् त मे॒तेन॑ जेषम् जेष मे॒तेन॒ तम् त मे॒तेन॑ जेषम् । \newline
15. ए॒तेन॑ जेषम् जेष मे॒तेनै॒तेन॑ जेष॒(ग्म्॒) सुरू॑पवर्.षवर्णे॒ सुरू॑पवर्.षवर्णे जेष मे॒तेनै॒तेन॑ जेष॒(ग्म्॒) सुरू॑पवर्.षवर्णे । \newline
16. जे॒ष॒(ग्म्॒) सुरू॑पवर्.षवर्णे॒ सुरू॑पवर्.षवर्णे जेषम् जेष॒(ग्म्॒) सुरू॑पवर्.षवर्ण॒ आ सुरू॑पवर्.षवर्णे जेषम् जेष॒(ग्म्॒) सुरू॑पवर्.षवर्ण॒ आ । \newline
17. सुरू॑पवर्.षवर्ण॒ आ सुरू॑पवर्.षवर्णे॒ सुरू॑पवर्.षवर्ण॒ एही॒ह्या सुरू॑पवर्.षवर्णे॒ सुरू॑पवर्.षवर्ण॒ एहि॑ । \newline
18. सुरू॑पवर्.षवर्ण॒ इति॒ सुरू॑प - व॒र्॒.ष॒व॒र्णे॒ । \newline
19. एही॒ह्ये ही॒मा नि॒मा नि॒ह्ये ही॒मान् । \newline
20. इ॒ही॒मा नि॒मा नि॑हीही॒मान् भ॒द्रान् भ॒द्रा नि॒मा नि॑हीही॒मान् भ॒द्रान् । \newline
21. इ॒मान् भ॒द्रान् भ॒द्रा नि॒मा नि॒मान् भ॒द्रान् दुर्या॒न् दुर्या᳚न् भ॒द्रा नि॒मा नि॒मान् भ॒द्रान् दुर्यान्॑ । \newline
22. भ॒द्रान् दुर्या॒न् दुर्या᳚न् भ॒द्रान् भ॒द्रान् दुर्या(ग्म्॑) अ॒भ्य॑भि दुर्या᳚न् भ॒द्रान् भ॒द्रान् दुर्या(ग्म्॑) अ॒भि । \newline
23. दुर्या(ग्म्॑) अ॒भ्य॑भि दुर्या॒न् दुर्या(ग्म्॑) अ॒भ्या ऽभि दुर्या॒न् दुर्या(ग्म्॑) अ॒भ्या । \newline
24. अ॒भ्या ऽभ्य॑भ्येही॒ह्या ऽभ्य॑भ्येहि॑ । \newline
25. एही॒ह्येहि॒ माम् मा मि॒ह्येहि॒ माम् । \newline
26. इ॒हि॒ माम् मा मि॑हीहि॒ मा मनु॑व्र॒ता ऽनु॑व्रता॒ मा मि॑हीहि॒ मा मनु॑व्रता । \newline
27. मा मनु॑व्र॒ता ऽनु॑व्रता॒ माम् मा मनु॑व्रता॒ नि न्यनु॑व्रता॒ माम् मा मनु॑व्रता॒ नि । \newline
28. अनु॑व्रता॒ नि न्यनु॑व्र॒ता ऽनु॑व्रता॒ न्यु॑ वु॒ न्यनु॑व्र॒ता ऽनु॑व्रता॒ न्यु॑ । \newline
29. अनु॑व्र॒तेत्यनु॑ - व्र॒ता॒ । \newline
30. न्यु॑ वु॒ नि न्यु॑ शी॒र्॒.षाणि॑ शी॒र्॒.षाण्यु॒ नि न्यु॑ शी॒र्॒.षाणि॑ । \newline
31. उ॒ शी॒र्॒.षाणि॑ शी॒र्॒.षाण्यु॑ वु शी॒र्॒.षाणि॑ मृढ्वम् मृढ्वꣳ शी॒र्॒.षाण्यु॑ वु शी॒र्॒.षाणि॑ मृढ्वम् । \newline
32. शी॒र्॒.षाणि॑ मृढ्वम् मृढ्वꣳ शी॒र्॒.षाणि॑ शी॒र्॒.षाणि॑ मृढ्व॒ मिड॒ इडे॑ मृढ्वꣳ शी॒र्॒.षाणि॑ शी॒र्॒.षाणि॑ मृढ्व॒ मिडे᳚ । \newline
33. मृ॒ढ्व॒ मिड॒ इडे॑ मृढ्वम् मृढ्व॒ मिड॒ एडे॑ मृढ्वम् मृढ्व॒ मिड॒ आ । \newline
34. इड॒ एड॒ इड॒ एही॒ह्येड॒ इड॒ एहि॑ । \newline
35. एही॒ह्येह्यदि॒ते ऽदि॑त इ॒ह्येह्यदि॑ते । \newline
36. इ॒ह्यदि॒ते ऽदि॑त इही॒ह्यदि॑त॒ आ ऽदि॑त इही॒ह्यदि॑त॒ आ । \newline
37. अदि॑त॒ आ ऽदि॒ते ऽदि॑त॒ एही॒ह्या ऽदि॒ते ऽदि॑त॒ एहि॑ । \newline
38. एही॒ह्येहि॒ सर॑स्वति॒ सर॑स्वती॒ह्येहि॒ सर॑स्वति । \newline
39. इ॒हि॒ सर॑स्वति॒ सर॑स्वतीहीहि॒ सर॑स्व॒त्या सर॑स्वतीहीहि॒ सर॑स्व॒त्या । \newline
40. सर॑स्व॒त्या सर॑स्वति॒ सर॑स्व॒त्येही॒ह्या सर॑स्वति॒ सर॑स्व॒त्येहि॑ । \newline
41. एही॒ह्येहि॒ रन्ती॒ रन्ति॑ रि॒ह्येहि॒ रन्तिः॑ । \newline
42. इ॒हि॒ रन्ती॒ रन्ति॑ रिहीहि॒ रन्ति॑रस्यसि॒ रन्ति॑ रिहीहि॒ रन्ति॑रसि । \newline
43. रन्ति॑रस्यसि॒ रन्ती॒ रन्ति॑रसि॒ रम॑ती॒ रम॑तिरसि॒ रन्ती॒ रन्ति॑रसि॒ रम॑तिः । \newline
44. अ॒सि॒ रम॑ती॒ रम॑तिरस्यसि॒ रम॑तिरस्यसि॒ रम॑तिरस्यसि॒ रम॑तिरसि । \newline
45. रम॑तिरस्यसि॒ रम॑ती॒ रम॑तिरसि सू॒नरी॑ सू॒नर्य॑सि॒ रम॑ती॒ रम॑तिरसि सू॒नरी᳚ । \newline
46. अ॒सि॒ सू॒नरी॑ सू॒नर्य॑स्यसि सू॒नर्य॑स्यसि सू॒नर्य॑स्यसि सू॒नर्य॑सि । \newline
47. सू॒नर्य॑स्यसि सू॒नरी॑ सू॒नर्य॑सि॒ जुष्टे॒ जुष्टे॑ ऽसि सू॒नरी॑ सू॒नर्य॑सि॒ जुष्टे᳚ । \newline
48. अ॒सि॒ जुष्टे॒ जुष्टे᳚ ऽस्यसि॒ जुष्टे॒ जुष्टि॒म् जुष्टि॒म् जुष्टे᳚ ऽस्यसि॒ जुष्टे॒ जुष्टि᳚म् । \newline
49. जुष्टे॒ जुष्टि॒म् जुष्टि॒म् जुष्टे॒ जुष्टे॒ जुष्टि॑म् ते ते॒ जुष्टि॒म् जुष्टे॒ जुष्टे॒ जुष्टि॑म् ते । \newline
50. जुष्टि॑म् ते ते॒ जुष्टि॒म् जुष्टि॑म् ते ऽशीयाशीय ते॒ जुष्टि॒म् जुष्टि॑म् ते ऽशीय । \newline
51. ते॒ ऽशी॒या॒शी॒य॒ ते॒ ते॒ ऽशी॒योप॑हूत॒ उप॑हूते ऽशीय ते ते ऽशी॒योप॑हूते । \newline
52. अ॒शी॒योप॑हूत॒ उप॑हूते ऽशीयाशी॒योप॑हूत उपह॒व मु॑पह॒व मुप॑हूते ऽशीयाशी॒योप॑हूत उपह॒वम् । \newline
53. उप॑हूत उपह॒व मु॑पह॒व मुप॑हूत॒ उप॑हूत उपह॒वम् ते॑ त उपह॒व मुप॑हूत॒ उप॑हूत उपह॒वम् ते᳚ । \newline
54. उप॑हूत॒ इत्युप॑ - हू॒ते॒ । \newline
55. उ॒प॒ह॒वम् ते॑ त उपह॒व मु॑पह॒वम् ते॑ ऽशीयाशीय त उपह॒व मु॑पह॒वम् ते॑ ऽशीय । \newline
56. उ॒प॒ह॒वमित्यु॑प - ह॒वम् । \newline
\pagebreak
\markright{ TS 1.6.3.2  \hfill https://www.vedavms.in \hfill}

\section{ TS 1.6.3.2 }

\textbf{TS 1.6.3.2 } \newline
\textbf{Samhita Paata} \newline

ते॑ऽशीय॒ सा मे॑ स॒त्याऽऽशीर॒स्य य॒ज्ञ्स्य॑ भूया॒दरे॑डता॒ मन॑सा॒ तच्छ॑केयं ॅय॒ज्ञो दिवꣳ॑ रोहतु य॒ज्ञो दिवं॑ गच्छतु॒ यो दे॑व॒यानः॒ पन्था॒स्तेन॑ य॒ज्ञो दे॒वाꣳ अप्ये᳚त्व॒स्मास्विन्द्र॑ इन्द्रि॒यं द॑धात्व॒स्मान्राय॑ उ॒त य॒ज्ञाः स॑चन्ताम॒स्मासु॑ सन्त्वा॒शिषः॒ सा नः॑ प्रि॒या सु॒प्रतू᳚र्तिर्म॒घोनी॒ जुष्टि॑रसि जु॒षस्व॑ नो॒ जुष्टा॑ नो - [ ] \newline

\textbf{Pada Paata} \newline

ते॒ । अ॒शी॒य॒ । सा । मे॒ । स॒त्या । आ॒शीरित्या᳚ - शीः । अ॒स्य । य॒ज्ञ्स्य॑ । भू॒या॒त् । अरे॑डता । मन॑सा । तत् । श॒के॒य॒म् । य॒ज्ञ्ः । दिव᳚म् । रो॒ह॒तु॒ । य॒ज्ञ्ः । दिव᳚म् । ग॒च्छ॒तु॒ । यः । दे॒व॒यान॒ इति॑ देव-यानः॑ । पन्थाः᳚ । तेन॑ । य॒ज्ञ्ः । दे॒वान् । अपीति॑ । ए॒तु॒ । अ॒स्मासु॑ । इन्द्रः॑ । इ॒न्द्रि॒यम् । द॒धा॒तु॒ । अ॒स्मान् । रायः॑ । उ॒त । य॒ज्ञाः । स॒च॒न्ता॒म् । अ॒स्मासु॑ । स॒न्तु॒ । आ॒शिष॒ इत्या᳚ - शिषः॑ । सा । नः॒ । प्रि॒या । सु॒प्रतू᳚र्ति॒रिति॑ सु - प्रतू᳚र्तिः । म॒घोनी᳚ । जुष्टिः॑ । अ॒सि॒ । जु॒षस्व॑ । नः॒ । जुष्टा᳚ । नः॒ ।  \newline


\textbf{Krama Paata} \newline

ते॒ऽशी॒य॒ । अ॒शी॒य॒ सा । सा मे᳚ । मे॒ स॒त्या । स॒त्याऽऽशीः । आ॒शीर॒स्य । आ॒शीरित्या᳚ - शीः । अ॒स्य य॒ज्ञ्स्य॑ । य॒ज्ञ्स्य॑ भूयात् । भू॒या॒दरे॑डता । अरे॑डता॒ मन॑सा । मन॑सा॒ तत् । तच्छ॑केयम् । श॒के॒यं॒ ॅय॒ज्ञ्ः । य॒ज्ञो दिव᳚म् । दिवꣳ॑ रोहतु । रो॒ह॒तु॒ य॒ज्ञ्ः । य॒ज्ञो दिव᳚म् । दिव॑म् गच्छतु । ग॒च्छ॒तु॒ यः । यो दे॑व॒यानः॑ । दे॒व॒यानः॒ पन्थाः᳚ । दे॒व॒यान॒ इति॑ देव - यानः॑ । पन्था॒स्तेन॑ । तेन॑ य॒ज्ञ्ः । य॒ज्ञो दे॒वान् । दे॒वाꣳ अपि॑ । अप्ये॑तु । ए॒त्व॒स्मासु॑ । अ॒स्मास्विन्द्रः॑ । इन्द्र॑ इन्द्रि॒यम् । इ॒न्द्रि॒यम् द॑धातु । द॒धा॒त्व॒स्मान् । अ॒स्मान् रायः॑ । राय॑ उ॒त । उ॒त य॒ज्ञाः । य॒ज्ञाः स॑चन्ताम् । स॒च॒न्ता॒म॒स्मासु॑ । अ॒स्मासु॑ सन्तु । स॒न्त्वा॒शिषः॑ । आ॒शिषः॒ सा । आ॒शिष॒ इत्या᳚ - शिषः॑ । सा नः॑ । नः॒ प्रि॒या । प्रि॒या सु॒प्रतू᳚र्तिः । सु॒प्रतू᳚र्तिर्,म॒घोनी᳚ । सु॒प्रतू᳚र्ति॒रिति॑ सु - प्रतू᳚र्तिः । म॒घोनी॒ जुष्टिः॑ । जुष्टि॑रसि । अ॒सि॒ जु॒षस्व॑ । जु॒षस्व॑ नः । नो॒ जुष्टा᳚ । जुष्टा॑ नः । नो॒ऽसि॒ \newline

\textbf{Jatai Paata} \newline

1. ते॒ ऽशी॒या॒शी॒य॒ ते॒ ते॒ ऽशी॒य॒ । \newline
2. अ॒शी॒य॒ सा सा ऽशी॑याशीय॒ सा । \newline
3. सा मे॑ मे॒ सा सा मे᳚ । \newline
4. मे॒ स॒त्या स॒त्या मे॑ मे स॒त्या । \newline
5. स॒त्या ऽऽशीरा॒शीः स॒त्या स॒त्या ऽऽशीः । \newline
6. आ॒शी र॒स्यास्या शीरा॒शी र॒स्य । \newline
7. आ॒शीरित्या᳚ - शीः । \newline
8. अ॒स्य य॒ज्ञ्स्य॑ य॒ज्ञ् स्या॒स्यास्य य॒ज्ञ्स्य॑ । \newline
9. य॒ज्ञ्स्य॑ भूयाद् भूयाद् य॒ज्ञ्स्य॑ य॒ज्ञ्स्य॑ भूयात् । \newline
10. भू॒या॒द रे॑ड॒ता ऽरे॑डता भूयाद् भूया॒द रे॑डता । \newline
11. अरे॑डता॒ मन॑सा॒ मन॒सा ऽरे॑ड॒ता ऽरे॑डता॒ मन॑सा । \newline
12. मन॑सा॒ तत् तन् मन॑सा॒ मन॑सा॒ तत् । \newline
13. तच्छ॑केयꣳ शकेय॒म् तत् तच्छ॑केयम् । \newline
14. श॒के॒यं॒ ॅय॒ज्ञो य॒ज्ञ्ः श॑केयꣳ शकेयं ॅय॒ज्ञ्ः । \newline
15. य॒ज्ञो दिव॒म् दिवं॑ ॅय॒ज्ञो य॒ज्ञो दिव᳚म् । \newline
16. दिव(ग्म्॑) रोहतु रोहतु॒ दिव॒म् दिव(ग्म्॑) रोहतु । \newline
17. रो॒ह॒तु॒ य॒ज्ञो य॒ज्ञो रो॑हतु रोहतु य॒ज्ञ्ः । \newline
18. य॒ज्ञो दिव॒म् दिवं॑ ॅय॒ज्ञो य॒ज्ञो दिव᳚म् । \newline
19. दिव॑म् गच्छतु गच्छतु॒ दिव॒म् दिव॑म् गच्छतु । \newline
20. ग॒च्छ॒तु॒ यो यो ग॑च्छतु गच्छतु॒ यः । \newline
21. यो दे॑व॒यानो॑ देव॒यानो॒ यो यो दे॑व॒यानः॑ । \newline
22. दे॒व॒यानः॒ पन्थाः॒ पन्था॑ देव॒यानो॑ देव॒यानः॒ पन्थाः᳚ । \newline
23. दे॒व॒यान॒ इति॑ देव - यानः॑ । \newline
24. पन्था॒ स्तेन॒ तेन॒ पन्थाः॒ पन्था॒ स्तेन॑ । \newline
25. तेन॑ य॒ज्ञो य॒ज्ञ् स्तेन॒ तेन॑ य॒ज्ञ्ः । \newline
26. य॒ज्ञो दे॒वान् दे॒वान्. य॒ज्ञो य॒ज्ञो दे॒वान् । \newline
27. दे॒वाꣳ अप्यपि॑ दे॒वान् दे॒वाꣳ अपि॑ । \newline
28. अप्ये᳚ त्वे॒त्व प्यप्ये॑तु । \newline
29. ए॒त्व॒ स्मास्व॒स्मा स्वे᳚त्वे त्व॒स्मासु॑ । \newline
30. अ॒स्मा स्विन्द्र॒ इन्द्रो॒ ऽस्मास्व॒स्मा स्विन्द्रः॑ । \newline
31. इन्द्र॑ इन्द्रि॒य मि॑न्द्रि॒य मिन्द्र॒ इन्द्र॑ इन्द्रि॒यम् । \newline
32. इ॒न्द्रि॒यम् द॑धातु दधात्विन्द्रि॒य मि॑न्द्रि॒यम् द॑धातु । \newline
33. द॒धा॒त्व॒स्मा न॒स्मान् द॑धातु दधात्व॒स्मान् । \newline
34. अ॒स्मान् रायो॒ रायो॒ ऽस्मा न॒स्मान् रायः॑ । \newline
35. राय॑ उ॒तोत रायो॒ राय॑ उ॒त । \newline
36. उ॒त य॒ज्ञा य॒ज्ञा उ॒तोत य॒ज्ञाः । \newline
37. य॒ज्ञाः स॑चन्ताꣳ सचन्तां ॅय॒ज्ञा य॒ज्ञाः स॑चन्ताम् । \newline
38. स॒च॒न्ता॒ म॒स्मा स्व॒स्मासु॑ सचन्ताꣳ सचन्ता म॒स्मासु॑ । \newline
39. अ॒स्मासु॑ सन्तु सन्त्व॒ स्मास्व॒स्मासु॑ सन्तु । \newline
40. स॒न्त्वा॒शिष॑ आ॒शिषः॑ सन्तु सन्त्वा॒शिषः॑ । \newline
41. आ॒शिषः॒ सा सा ऽऽशिष॑ आ॒शिषः॒ सा । \newline
42. आ॒शिष॒ इत्या᳚ - शिषः॑ । \newline
43. सा नो॑ नः॒ सा सा नः॑ । \newline
44. नः॒ प्रि॒या प्रि॒या नो॑ नः प्रि॒या । \newline
45. प्रि॒या सु॒प्रतू᳚र्तिः सु॒प्रतू᳚र्तिः प्रि॒या प्रि॒या सु॒प्रतू᳚र्तिः । \newline
46. सु॒प्रतू᳚र्तिर् म॒घोनी॑ म॒घोनी॑ सु॒प्रतू᳚र्तिः सु॒प्रतू᳚र्तिर् म॒घोनी᳚ । \newline
47. सु॒प्रतू᳚र्ति॒रिति॑ सु - प्रतू᳚र्तिः । \newline
48. म॒घोनी॒ जुष्टि॒र् जुष्टि॑र् म॒घोनी॑ म॒घोनी॒ जुष्टिः॑ । \newline
49. जुष्टि॑रस्यसि॒ जुष्टि॒र् जुष्टि॑रसि । \newline
50. अ॒सि॒ जु॒षस्व॑ जु॒षस्वा᳚स्यसि जु॒षस्व॑ । \newline
51. जु॒षस्व॑ नो नो जु॒षस्व॑ जु॒षस्व॑ नः । \newline
52. नो॒ जुष्टा॒ जुष्टा॑ नो नो॒ जुष्टा᳚ । \newline
53. जुष्टा॑ नो नो॒ जुष्टा॒ जुष्टा॑ नः । \newline
54. नो॒ ऽस्य॒सि॒ नो॒ नो॒ ऽसि॒ । \newline

\textbf{Ghana Paata } \newline

1. ते॒ ऽशी॒या॒शी॒य॒ ते॒ ते॒ ऽशी॒य॒ सा सा ऽशी॑य ते ते ऽशीय॒ सा । \newline
2. अ॒शी॒य॒ सा सा ऽशी॑याशीय॒ सा मे॑ मे॒ सा ऽशी॑याशीय॒ सा मे᳚ । \newline
3. सा मे॑ मे॒ सा सा मे॑ स॒त्या स॒त्या मे॒ सा सा मे॑ स॒त्या । \newline
4. मे॒ स॒त्या स॒त्या मे॑ मे स॒त्या ऽऽशीरा॒शीः स॒त्या मे॑ मे स॒त्या ऽऽशीः । \newline
5. स॒त्या ऽऽशीरा॒शीः स॒त्या स॒त्या ऽऽशी र॒स्यास्याशीः स॒त्या स॒त्या ऽऽशीर॒स्य । \newline
6. आ॒शी र॒स्यास्याशी रा॒शीर॒स्य य॒ज्ञ्स्य॑ य॒ज्ञ्स्या॒स्याशी रा॒शीर॒स्य य॒ज्ञ्स्य॑ । \newline
7. आ॒शीरित्या᳚ - शीः । \newline
8. अ॒स्य य॒ज्ञ्स्य॑ य॒ज्ञ्स्या॒स्यास्य य॒ज्ञ्स्य॑ भूयाद् भूयाद् य॒ज्ञ्स्या॒स्यास्य य॒ज्ञ्स्य॑ भूयात् । \newline
9. य॒ज्ञ्स्य॑ भूयाद् भूयाद् य॒ज्ञ्स्य॑ य॒ज्ञ्स्य॑ भूया॒दरे॑ड॒ता ऽरे॑डता भूयाद् य॒ज्ञ्स्य॑ य॒ज्ञ्स्य॑ भूया॒दरे॑डता । \newline
10. भू॒या॒दरे॑ड॒ता ऽरे॑डता भूयाद् भूया॒दरे॑डता॒ मन॑सा॒ मन॒सा ऽरे॑डता भूयाद् भूया॒दरे॑डता॒ मन॑सा । \newline
11. अरे॑डता॒ मन॑सा॒ मन॒सा ऽरे॑ड॒ता ऽरे॑डता॒ मन॑सा॒ तत् तन् मन॒सा ऽरे॑ड॒ता ऽरे॑डता॒ मन॑सा॒ तत् । \newline
12. मन॑सा॒ तत् तन् मन॑सा॒ मन॑सा॒ तच्छ॑केयꣳ शकेय॒म् तन् मन॑सा॒ मन॑सा॒ तच्छ॑केयम् । \newline
13. तच्छ॑केयꣳ शकेय॒म् तत् तच्छ॑केयं ॅय॒ज्ञो य॒ज्ञ्ः श॑केय॒म् तत् तच्छ॑केयं ॅय॒ज्ञ्ः । \newline
14. श॒के॒यं॒ ॅय॒ज्ञो य॒ज्ञ्ः श॑केयꣳ शकेयं ॅय॒ज्ञो दिव॒म् दिवं॑ ॅय॒ज्ञ्ः श॑केयꣳ शकेयं ॅय॒ज्ञो दिव᳚म् । \newline
15. य॒ज्ञो दिव॒म् दिवं॑ ॅय॒ज्ञो य॒ज्ञो दिव(ग्म्॑) रोहतु रोहतु॒ दिवं॑ ॅय॒ज्ञो य॒ज्ञो दिव(ग्म्॑) रोहतु । \newline
16. दिव(ग्म्॑) रोहतु रोहतु॒ दिव॒म् दिव(ग्म्॑) रोहतु य॒ज्ञो य॒ज्ञो रो॑हतु॒ दिव॒म् दिव(ग्म्॑) रोहतु य॒ज्ञ्ः । \newline
17. रो॒ह॒तु॒ य॒ज्ञो य॒ज्ञो रो॑हतु रोहतु य॒ज्ञो दिव॒म् दिवं॑ ॅय॒ज्ञो रो॑हतु रोहतु य॒ज्ञो दिव᳚म् । \newline
18. य॒ज्ञो दिव॒म् दिवं॑ ॅय॒ज्ञो य॒ज्ञो दिव॑म् गच्छतु गच्छतु॒ दिवं॑ ॅय॒ज्ञो य॒ज्ञो दिव॑म् गच्छतु । \newline
19. दिव॑म् गच्छतु गच्छतु॒ दिव॒म् दिव॑म् गच्छतु॒ यो यो ग॑च्छतु॒ दिव॒म् दिव॑म् गच्छतु॒ यः । \newline
20. ग॒च्छ॒तु॒ यो यो ग॑च्छतु गच्छतु॒ यो दे॑व॒यानो॑ देव॒यानो॒ यो ग॑च्छतु गच्छतु॒ यो दे॑व॒यानः॑ । \newline
21. यो दे॑व॒यानो॑ देव॒यानो॒ यो यो दे॑व॒यानः॒ पन्थाः॒ पन्था॑ देव॒यानो॒ यो यो दे॑व॒यानः॒ पन्थाः᳚ । \newline
22. दे॒व॒यानः॒ पन्थाः॒ पन्था॑ देव॒यानो॑ देव॒यानः॒ पन्था॒स्तेन॒ तेन॒ पन्था॑ देव॒यानो॑ देव॒यानः॒ पन्था॒स्तेन॑ । \newline
23. दे॒व॒यान॒ इति॑ देव - यानः॑ । \newline
24. पन्था॒स्तेन॒ तेन॒ पन्थाः॒ पन्था॒स्तेन॑ य॒ज्ञो य॒ज्ञ्स्तेन॒ पन्थाः॒ पन्था॒स्तेन॑ य॒ज्ञ्ः । \newline
25. तेन॑ य॒ज्ञो य॒ज्ञ्स्तेन॒ तेन॑ य॒ज्ञो दे॒वान् दे॒वान्. य॒ज्ञ्स्तेन॒ तेन॑ य॒ज्ञो दे॒वान् । \newline
26. य॒ज्ञो दे॒वान् दे॒वान्. य॒ज्ञो य॒ज्ञो दे॒वाꣳ अप्यपि॑ दे॒वान्. य॒ज्ञो य॒ज्ञो दे॒वाꣳ अपि॑ । \newline
27. दे॒वाꣳ अप्यपि॑ दे॒वान् दे॒वाꣳ अप्ये᳚त्वे॒त्वपि॑ दे॒वान् दे॒वाꣳ अप्ये॑तु । \newline
28. अप्ये᳚ त्वे॒त्वप्यप्ये᳚ त्व॒स्मा स्व॒स्मा स्वे॒त्वप्यप्ये᳚ त्व॒स्मासु॑ । \newline
29. ए॒त्व॒स्मा स्व॒स्मास्वे᳚ त्वेत्व॒स्मास्विन्द्र॒ इन्द्रो॒ ऽस्मास्वे᳚ त्वेत्व॒स्मास्विन्द्रः॑ । \newline
30. अ॒स्मास्विन्द्र॒ इन्द्रो॒ ऽस्मास्व॒स्मास्विन्द्र॑ इन्द्रि॒य मि॑न्द्रि॒य मिन्द्रो॒ ऽस्मास्व॒स्मास्विन्द्र॑ इन्द्रि॒यम् । \newline
31. इन्द्र॑ इन्द्रि॒य मि॑न्द्रि॒य मिन्द्र॒ इन्द्र॑ इन्द्रि॒यम् द॑धातु दधात्विन्द्रि॒य मिन्द्र॒ इन्द्र॑ इन्द्रि॒यम् द॑धातु । \newline
32. इ॒न्द्रि॒यम् द॑धातु दधात्विन्द्रि॒य मि॑न्द्रि॒यम् द॑धात्व॒स्मा न॒स्मान् द॑धात्विन्द्रि॒य मि॑न्द्रि॒यम् द॑धात्व॒स्मान् । \newline
33. द॒धा॒त्व॒स्मा न॒स्मान् द॑धातु दधात्व॒स्मान् रायो॒ रायो॒ ऽस्मान् द॑धातु दधात्व॒स्मान् रायः॑ । \newline
34. अ॒स्मान् रायो॒ रायो॒ ऽस्मा न॒स्मान् राय॑ उ॒तोत रायो॒ ऽस्मा न॒स्मान् राय॑ उ॒त । \newline
35. राय॑ उ॒तोत रायो॒ राय॑ उ॒त य॒ज्ञा य॒ज्ञा उ॒त रायो॒ राय॑ उ॒त य॒ज्ञाः । \newline
36. उ॒त य॒ज्ञा य॒ज्ञा उ॒तोत य॒ज्ञाः स॑चन्ताꣳ सचन्तां ॅय॒ज्ञा उ॒तोत य॒ज्ञाः स॑चन्ताम् । \newline
37. य॒ज्ञाः स॑चन्ताꣳ सचन्तां ॅय॒ज्ञा य॒ज्ञाः स॑चन्ता म॒स्मास्व॒स्मासु॑ सचन्तां ॅय॒ज्ञा य॒ज्ञाः स॑चन्ता म॒स्मासु॑ । \newline
38. स॒च॒न्ता॒ म॒स्मास्व॒स्मासु॑ सचन्ताꣳ सचन्ता म॒स्मासु॑ सन्तु सन्त्व॒स्मासु॑ सचन्ताꣳ सचन्ता म॒स्मासु॑ सन्तु । \newline
39. अ॒स्मासु॑ सन्तु सन्त्व॒स्मा स्व॒स्मासु॑ सन्त्वा॒शिष॑ आ॒शिषः॑ सन्त्व॒स्मा स्व॒स्मासु॑ सन्त्वा॒शिषः॑ । \newline
40. स॒न्त्वा॒शिष॑ आ॒शिषः॑ सन्तु सन्त्वा॒शिषः॒ सा सा ऽऽशिषः॑ सन्तु सन्त्वा॒शिषः॒ सा । \newline
41. आ॒शिषः॒ सा सा ऽऽशिष॑ आ॒शिषः॒ सा नो॑ नः॒ सा ऽऽशिष॑ आ॒शिषः॒ सा नः॑ । \newline
42. आ॒शिष॒ इत्या᳚ - शिषः॑ । \newline
43. सा नो॑ नः॒ सा सा नः॑ प्रि॒या प्रि॒या नः॒ सा सा नः॑ प्रि॒या । \newline
44. नः॒ प्रि॒या प्रि॒या नो॑ नः प्रि॒या सु॒प्रतू᳚र्तिः सु॒प्रतू᳚र्तिः प्रि॒या नो॑ नः प्रि॒या सु॒प्रतू᳚र्तिः । \newline
45. प्रि॒या सु॒प्रतू᳚र्तिः सु॒प्रतू᳚र्तिः प्रि॒या प्रि॒या सु॒प्रतू᳚र्तिर् म॒घोनी॑ म॒घोनी॑ सु॒प्रतू᳚र्तिः प्रि॒या प्रि॒या सु॒प्रतू᳚र्तिर् म॒घोनी᳚ । \newline
46. सु॒प्रतू᳚र्तिर् म॒घोनी॑ म॒घोनी॑ सु॒प्रतू᳚र्तिः सु॒प्रतू᳚र्तिर् म॒घोनी॒ जुष्टि॒र् जुष्टि॑र् म॒घोनी॑ सु॒प्रतू᳚र्तिः सु॒प्रतू᳚र्तिर् म॒घोनी॒ जुष्टिः॑ । \newline
47. सु॒प्रतू᳚र्ति॒रिति॑ सु - प्रतू᳚र्तिः । \newline
48. म॒घोनी॒ जुष्टि॒र् जुष्टि॑र् म॒घोनी॑ म॒घोनी॒ जुष्टि॑रस्यसि॒ जुष्टि॑र् म॒घोनी॑ म॒घोनी॒ जुष्टि॑रसि । \newline
49. जुष्टि॑रस्यसि॒ जुष्टि॒र् जुष्टि॑रसि जु॒षस्व॑ जु॒षस्वा॑सि॒ जुष्टि॒र् जुष्टि॑रसि जु॒षस्व॑ । \newline
50. अ॒सि॒ जु॒षस्व॑ जु॒षस्वा᳚स्यसि जु॒षस्व॑ नो नो जु॒षस्वा᳚स्यसि जु॒षस्व॑ नः । \newline
51. जु॒षस्व॑ नो नो जु॒षस्व॑ जु॒षस्व॑ नो॒ जुष्टा॒ जुष्टा॑ नो जु॒षस्व॑ जु॒षस्व॑ नो॒ जुष्टा᳚ । \newline
52. नो॒ जुष्टा॒ जुष्टा॑ नो नो॒ जुष्टा॑ नो नो॒ जुष्टा॑ नो नो॒ जुष्टा॑ नः । \newline
53. जुष्टा॑ नो नो॒ जुष्टा॒ जुष्टा॑ नो ऽस्यसि नो॒ जुष्टा॒ जुष्टा॑ नो ऽसि । \newline
54. नो॒ ऽस्य॒सि॒ नो॒ नो॒ ऽसि॒ जुष्टि॒म् जुष्टि॑ मसि नो नो ऽसि॒ जुष्टि᳚म् । \newline
\pagebreak
\markright{ TS 1.6.3.3  \hfill https://www.vedavms.in \hfill}

\section{ TS 1.6.3.3 }

\textbf{TS 1.6.3.3 } \newline
\textbf{Samhita Paata} \newline

ऽसि॒ जुष्टिं॑ ते गमेयं॒ मनो॒ ज्योति॑र् जुषता॒माज्यं॒ ॅविच्छि॑न्नं ॅय॒ज्ञ्ꣳ समि॒मं द॑धातु । बृह॒स्पति॑-स्तनुतामि॒मन्नो॒ विश्वे॑ दे॒वा इ॒ह मा॑दयन्तां ॥ ब्रद्ध्न॒ पिन्व॑स्व॒ दद॑तो मे॒ मा क्षा॑यि कुर्व॒तो मे॒ मोप॑ दसत् प्र॒जाप॑तेर् भा॒गो᳚ऽस्यूर्ज॑स्वा॒न् पय॑स्वान् प्राणापा॒नौ मे॑ पाहि समानव्या॒नौ मे॑ पाह्युदानव्या॒नौ मे॑ पा॒ह्यक्षि॑तो॒ऽस्यक्षि॑त्यै त्वा॒ ( ) मा मे᳚ क्षेष्ठा अ॒मुत्रा॒मुष्मि॑न् ॅलो॒के ॥ \newline

\textbf{Pada Paata} \newline

अ॒सि॒ । जुष्टि᳚म् । ते॒ । ग॒मे॒य॒म् । मनः॑ । ज्योतिः॑ । जु॒ष॒ता॒म् । आज्य᳚म् । विच्छि॑न्न॒मिति॒ वि - छि॒न्न॒म् । य॒ज्ञ्म् । समिति॑ । इ॒मम् । द॒धा॒तु॒ । बृह॒स्पतिः॑ । त॒नु॒ता॒म् । इ॒मम् । नः॒ । विश्वे᳚ । दे॒वाः । इ॒ह । मा॒द॒य॒न्ता॒म् । ब्रद्ध्न॑ । पिन्व॑स्व । दद॑तः । मे॒ । मा । क्षा॒यि॒ । कु॒र्व॒तः । मे॒ । मा । उपेति॑ । द॒स॒त् । प्र॒जाप॑ते॒रिति॑ प्र॒जा - प॒तेः॒ । भा॒गः । अ॒सि॒ । ऊर्ज॑स्वान् । पय॑स्वान् । प्रा॒णा॒पा॒नाविति॑ प्राण-अ॒पा॒नौ । मे॒ । पा॒हि॒ । स॒मा॒न॒व्या॒नाविति॑ समान - व्या॒नौ । मे॒ । पा॒हि॒ । उ॒दा॒न॒व्या॒नावित्यु॑दान - व्या॒नौ । मे॒ । पा॒हि॒ । अक्षि॑तः । अ॒सि॒ । अक्षि॑त्यै । त्वा॒ ( ) । मा । मे॒ । क्षे॒ष्ठाः॒ । अ॒मुत्र॑ । अ॒मुष्मिन्न्॑ । लो॒के ॥  \newline


\textbf{Krama Paata} \newline

अ॒सि॒ जुष्टि᳚म् । जुष्टि॑म् ते । ते॒ ग॒मे॒य॒म् । ग॒मे॒य॒म् मनः॑ । मनो॒ ज्योतिः॑ । ज्योति॑र्,जुषताम् । जु॒ष॒ता॒माज्य᳚म् । आज्यं॒ ॅविच्छि॑न्नम् । विच्छि॑न्नं ॅय॒ज्ञ्म् । विच्छि॑न्न॒मिति॒ वि - छि॒न्न॒म् । य॒ज्ञ्ꣳ सम् । समि॒मम् । इ॒मम् द॑धातु । द॒धा॒त्विति॑ दधातु ॥ बृह॒स्पति॑स्तनुताम् । त॒नु॒ता॒मि॒मम् । इ॒मम् नः॑ । नो॒ विश्वे᳚ । विश्वे॑ दे॒वाः । दे॒वा इ॒ह । इ॒ह मा॑दयन्ताम् । मा॒द॒य॒न्ता॒मिति मा॑दयन्ताम् ॥ ब्रद्ध्न॒ पिन्व॑स्व । पिन्व॑स्व॒ दद॑तः । दद॑तो मे । मे॒ मा । मा क्षा॑यि । क्षा॒यि॒ कु॒र्व॒तः । कु॒र्व॒तो मे᳚ । मे॒ मा । मोप॑ । उप॑ दसत् । द॒स॒त्,प्र॒जाप॑तेः । प्र॒जाप॑तेर्,भा॒गः । प्र॒जाप॑ते॒रिति॑ प्र॒जा - प॒तेः॒ । भा॒गो॑ऽसि । अ॒स्यूर्ज॑स्वान् । ऊर्ज॑स्वा॒न् पय॑स्वान् । पय॑स्वान् प्राणापा॒नौ । प्रा॒णा॒पा॒नौ मे᳚ । प्रा॒णा॒पा॒नाविति॑ प्राण - अ॒पा॒नौ । मे॒ पा॒हि॒ । पा॒हि॒ स॒मा॒न॒व्या॒नौ । स॒मा॒न॒व्या॒नौ मे᳚ । स॒मा॒न॒व्या॒नाविति॑ समान - व्या॒नौ । मे॒ पा॒हि॒ । पा॒ह्यु॒दा॒न॒व्या॒नौ । उ॒दा॒न॒व्या॒नौ मे᳚ । उ॒दा॒न॒व्या॒नावित्यु॑दान - व्या॒नौ । मे॒ पा॒हि॒ । पा॒ह्यक्षि॑तः । अक्षि॑तोऽसि । अ॒स्यक्षि॑त्यै । अक्षि॑त्यै त्वा ( ) । त्वा॒ मा । मा मे᳚ । मे॒ क्षे॒ष्ठाः॒ । क्षे॒ष्ठा॒ अ॒मुत्र॑ । अ॒मुत्रा॒मुष्मिन्न्॑ । अ॒मुष्मि॑न् ॅलो॒के । लो॒क इति॑ लो॒के । \newline

\textbf{Jatai Paata} \newline

1. अ॒सि॒ जुष्टि॒म् जुष्टि॑ मस्यसि॒ जुष्टि᳚म् । \newline
2. जुष्टि॑म् ते ते॒ जुष्टि॒म् जुष्टि॑म् ते । \newline
3. ते॒ ग॒मे॒य॒म् ग॒मे॒य॒म् ते॒ ते॒ ग॒मे॒य॒म् । \newline
4. ग॒मे॒य॒म् मनो॒ मनो॑ गमेयम् गमेय॒म् मनः॑ । \newline
5. मनो॒ ज्योति॒र् ज्योति॒र् मनो॒ मनो॒ ज्योतिः॑ । \newline
6. ज्योति॑र् जुषताम् जुषता॒म् ज्योति॒र् ज्योति॑र् जुषताम् । \newline
7. जु॒ष॒ता॒ माज्य॒ माज्य॑म् जुषताम् जुषता॒ माज्य᳚म् । \newline
8. आज्यं॒ ॅविच्छि॑न्नं॒ ॅविच्छि॑न्न॒ माज्य॒ माज्यं॒ ॅविच्छि॑न्नम् । \newline
9. विच्छि॑न्नं ॅय॒ज्ञ्ं ॅय॒ज्ञ्ं ॅविच्छि॑न्नं॒ ॅविच्छि॑न्नं ॅय॒ज्ञ्म् । \newline
10. विच्छि॑न्न॒मिति॒ वि - छि॒न्न॒म् । \newline
11. य॒ज्ञ्ꣳ सꣳ सं ॅय॒ज्ञ्ं ॅय॒ज्ञ्ꣳ सम् । \newline
12. स मि॒म मि॒मꣳ सꣳ स मि॒मम् । \newline
13. इ॒मम् द॑धातु दधात्वि॒म मि॒मम् द॑धातु । \newline
14. द॒धा॒त्विति॑ दधातु । \newline
15. बृह॒स्पति॑ स्तनुताम् तनुता॒म् बृह॒स्पति॒र् बृह॒स्पति॑ स्तनुताम् । \newline
16. त॒नु॒ता॒ मि॒म मि॒मम् त॑नुताम् तनुता मि॒मम् । \newline
17. इ॒मम् नो॑ न इ॒म मि॒मम् नः॑ । \newline
18. नो॒ विश्वे॒ विश्वे॑ नो नो॒ विश्वे᳚ । \newline
19. विश्वे॑ दे॒वा दे॒वा विश्वे॒ विश्वे॑ दे॒वाः । \newline
20. दे॒वा इ॒हे ह दे॒वा दे॒वा इ॒ह । \newline
21. इ॒ह मा॑दयन्ताम् मादयन्ता मि॒हे ह मा॑दयन्ताम् । \newline
22. मा॒द॒य॒न्ता॒मिति॑ मादयन्ताम् । \newline
23. ब्रद्ध्न॒ पिन्व॑स्व॒ पिन्व॑स्व॒ ब्रद्ध्न॒ ब्रद्ध्न॒ पिन्व॑स्व । \newline
24. पिन्व॑स्व॒ दद॑तो॒ दद॑तः॒ पिन्व॑स्व॒ पिन्व॑स्व॒ दद॑तः । \newline
25. दद॑तो मे मे॒ दद॑तो॒ दद॑तो मे । \newline
26. मे॒ मा मा मे॑ मे॒ मा । \newline
27. मा क्षा॑यि क्षायि॒ मा मा क्षा॑यि । \newline
28. क्षा॒यि॒ कु॒र्व॒तः कु॑र्व॒तः क्षा॑यि क्षायि कुर्व॒तः । \newline
29. कु॒र्व॒तो मे॑ मे कुर्व॒तः कु॑र्व॒तो मे᳚ । \newline
30. मे॒ मा मा मे॑ मे॒ मा । \newline
31. मोपोप॒ मा मोप॑ । \newline
32. उप॑ दसद् दस॒ दुपोप॑ दसत् । \newline
33. द॒स॒त् प्र॒जाप॑तेः प्र॒जाप॑तेर् दसद् दसत् प्र॒जाप॑तेः । \newline
34. प्र॒जाप॑तेर् भा॒गो भा॒गः प्र॒जाप॑तेः प्र॒जाप॑तेर् भा॒गः । \newline
35. प्र॒जाप॑ते॒रिति॑ प्र॒जा - प॒तेः॒ । \newline
36. भा॒गो᳚ ऽस्यसि भा॒गो भा॒गो॑ ऽसि । \newline
37. अ॒स्यूर्ज॑स्वा॒ नूर्ज॑स्वा नस्य॒ स्यूर्ज॑स्वान् । \newline
38. ऊर्ज॑स्वा॒न् पय॑स्वा॒न् पय॑स्वा॒ नूर्ज॑स्वा॒ नूर्ज॑स्वा॒न् पय॑स्वान् । \newline
39. पय॑स्वान् प्राणापा॒नौ प्रा॑णापा॒नौ पय॑स्वा॒न् पय॑स्वान् प्राणापा॒नौ । \newline
40. प्रा॒णा॒पा॒नौ मे॑ मे प्राणापा॒नौ प्रा॑णापा॒नौ मे᳚ । \newline
41. प्रा॒णा॒पा॒नाविति॑ प्राण - अ॒पा॒नौ । \newline
42. मे॒ पा॒हि॒ पा॒हि॒ मे॒ मे॒ पा॒हि॒ । \newline
43. पा॒हि॒ स॒मा॒न॒व्या॒नौ स॑मानव्या॒नौ पा॑हि पाहि समानव्या॒नौ । \newline
44. स॒मा॒न॒व्या॒नौ मे॑ मे समानव्या॒नौ स॑मानव्या॒नौ मे᳚ । \newline
45. स॒मा॒न॒व्या॒नाविति॑ समान - व्या॒नौ । \newline
46. मे॒ पा॒हि॒ पा॒हि॒ मे॒ मे॒ पा॒हि॒ । \newline
47. पा॒ ह्यु॒दा॒न॒व्या॒ना वु॑दानव्या॒नौ पा॑हि पा ह्युदानव्या॒नौ । \newline
48. उ॒दा॒न॒व्या॒नौ मे॑ म उदानव्या॒ना वु॑दानव्या॒नौ मे᳚ । \newline
49. उ॒दा॒न॒व्या॒नावित्यु॑दान - व्या॒नौ । \newline
50. मे॒ पा॒हि॒ पा॒हि॒ मे॒ मे॒ पा॒हि॒ । \newline
51. पा॒ह्यक्षि॒तो ऽक्षि॑तः पाहि पा॒ह्यक्षि॑तः । \newline
52. अक्षि॑तो ऽस्य॒स्यक्षि॒तो ऽक्षि॑तो ऽसि । \newline
53. अ॒स्यक्षि॑त्या॒ अक्षि॑त्या अस्य॒स्यक्षि॑त्यै । \newline
54. अक्षि॑त्यै त्वा॒ त्वा ऽक्षि॑त्या॒ अक्षि॑त्यै त्वा । \newline
55. त्वा॒ मा मा त्वा᳚ त्वा॒ मा । \newline
56. मा मे॑ मे॒ मा मा मे᳚ । \newline
57. मे॒ क्षे॒ष्ठाः॒ क्षे॒ष्ठा॒ मे॒ मे॒ क्षे॒ष्ठाः॒ । \newline
58. क्षे॒ष्ठा॒ अ॒मुत्रा॒मुत्र॑ क्षेष्ठाः क्षेष्ठा अ॒मुत्र॑ । \newline
59. अ॒मुत्रा॒मुष्मि॑न् न॒मुष्मि॑न् न॒मुत्रा॒ मुत्रा॒मुष्मिन्न्॑ । \newline
60. अ॒मुष्मि॑न् ॅलो॒के लो॒के॑ ऽमुष्मि॑न् न॒मुष्मि॑न् ॅलो॒के । \newline
61. लो॒क इति॑ लो॒के । \newline

\textbf{Ghana Paata } \newline

1. अ॒सि॒ जुष्टि॒म् जुष्टि॑ मस्यसि॒ जुष्टि॑म् ते ते॒ जुष्टि॑ मस्यसि॒ जुष्टि॑म् ते । \newline
2. जुष्टि॑म् ते ते॒ जुष्टि॒म् जुष्टि॑म् ते गमेयम् गमेयम् ते॒ जुष्टि॒म् जुष्टि॑म् ते गमेयम् । \newline
3. ते॒ ग॒मे॒य॒म् ग॒मे॒य॒म् ते॒ ते॒ ग॒मे॒य॒म् मनो॒ मनो॑ गमेयम् ते ते गमेय॒म् मनः॑ । \newline
4. ग॒मे॒य॒म् मनो॒ मनो॑ गमेयम् गमेय॒म् मनो॒ ज्योति॒र् ज्योति॒र् मनो॑ गमेयम् गमेय॒म् मनो॒ ज्योतिः॑ । \newline
5. मनो॒ ज्योति॒र् ज्योति॒र् मनो॒ मनो॒ ज्योति॑र् जुषताम् जुषता॒म् ज्योति॒र् मनो॒ मनो॒ ज्योति॑र् जुषताम् । \newline
6. ज्योति॑र् जुषताम् जुषता॒म् ज्योति॒र् ज्योति॑र् जुषता॒ माज्य॒ माज्य॑म् जुषता॒म् ज्योति॒र् ज्योति॑र् जुषता॒ माज्य᳚म् । \newline
7. जु॒ष॒ता॒ माज्य॒ माज्य॑म् जुषताम् जुषता॒ माज्यं॒ ॅविच्छि॑न्नं॒ ॅविच्छि॑न्न॒ माज्य॑म् जुषताम् जुषता॒ माज्यं॒ ॅविच्छि॑न्नम् । \newline
8. आज्यं॒ ॅविच्छि॑न्नं॒ ॅविच्छि॑न्न॒ माज्य॒ माज्यं॒ ॅविच्छि॑न्नं ॅय॒ज्ञ्ं ॅय॒ज्ञ्ं ॅविच्छि॑न्न॒ माज्य॒ माज्यं॒ ॅविच्छि॑न्नं ॅय॒ज्ञ्म् । \newline
9. विच्छि॑न्नं ॅय॒ज्ञ्ं ॅय॒ज्ञ्ं ॅविच्छि॑न्नं॒ ॅविच्छि॑न्नं ॅय॒ज्ञ्ꣳ सꣳ सं ॅय॒ज्ञ्ं ॅविच्छि॑न्नं॒ ॅविच्छि॑न्नं ॅय॒ज्ञ्ꣳ सम् । \newline
10. विच्छि॑न्न॒मिति॒ वि - छि॒न्न॒म् । \newline
11. य॒ज्ञ्ꣳ सꣳ सं ॅय॒ज्ञ्ं ॅय॒ज्ञ्ꣳ स मि॒म मि॒मꣳ सं ॅय॒ज्ञ्ं ॅय॒ज्ञ्ꣳ स मि॒मम् । \newline
12. स मि॒म मि॒मꣳ सꣳ स मि॒मम् द॑धातु दधात्वि॒मꣳ सꣳ स मि॒मम् द॑धातु । \newline
13. इ॒मम् द॑धातु दधात्वि॒म मि॒मम् द॑धातु । \newline
14. द॒धा॒त्विति॑ दधातु । \newline
15. बृह॒स्पति॑ स्तनुताम् तनुता॒म् बृह॒स्पति॒र् बृह॒स्पति॑ स्तनुता मि॒म मि॒मम् त॑नुता॒म् बृह॒स्पति॒र् बृह॒स्पति॑ स्तनुता मि॒मम् । \newline
16. त॒नु॒ता॒ मि॒म मि॒मम् त॑नुताम् तनुता मि॒मम् नो॑ न इ॒मम् त॑नुताम् तनुता मि॒मम् नः॑ । \newline
17. इ॒मम् नो॑ न इ॒म मि॒मम् नो॒ विश्वे॒ विश्वे॑ न इ॒म मि॒मम् नो॒ विश्वे᳚ । \newline
18. नो॒ विश्वे॒ विश्वे॑ नो नो॒ विश्वे॑ दे॒वा दे॒वा विश्वे॑ नो नो॒ विश्वे॑ दे॒वाः । \newline
19. विश्वे॑ दे॒वा दे॒वा विश्वे॒ विश्वे॑ दे॒वा इ॒हे ह दे॒वा विश्वे॒ विश्वे॑ दे॒वा इ॒ह । \newline
20. दे॒वा इ॒हे ह दे॒वा दे॒वा इ॒ह मा॑दयन्ताम् मादयन्ता मि॒ह दे॒वा दे॒वा इ॒ह मा॑दयन्ताम् । \newline
21. इ॒ह मा॑दयन्ताम् मादयन्ता मि॒हे ह मा॑दयन्ताम् । \newline
22. मा॒द॒य॒न्ता॒मिति॑ मादयन्ताम् । \newline
23. ब्रद्ध्न॒ पिन्व॑स्व॒ पिन्व॑स्व॒ ब्रद्ध्न॒ ब्रद्ध्न॒ पिन्व॑स्व॒ दद॑तो॒ दद॑तः॒ पिन्व॑स्व॒ ब्रद्ध्न॒ ब्रद्ध्न॒ पिन्व॑स्व॒ दद॑तः । \newline
24. पिन्व॑स्व॒ दद॑तो॒ दद॑तः॒ पिन्व॑स्व॒ पिन्व॑स्व॒ दद॑तो मे मे॒ दद॑तः॒ पिन्व॑स्व॒ पिन्व॑स्व॒ दद॑तो मे । \newline
25. दद॑तो मे मे॒ दद॑तो॒ दद॑तो मे॒ मा मा मे॒ दद॑तो॒ दद॑तो मे॒ मा । \newline
26. मे॒ मा मा मे॑ मे॒ मा क्षा॑यि क्षायि॒ मा मे॑ मे॒ मा क्षा॑यि । \newline
27. मा क्षा॑यि क्षायि॒ मा मा क्षा॑यि कुर्व॒तः कु॑र्व॒तः क्षा॑यि॒ मा मा क्षा॑यि कुर्व॒तः । \newline
28. क्षा॒यि॒ कु॒र्व॒तः कु॑र्व॒तः क्षा॑यि क्षायि कुर्व॒तो मे॑ मे कुर्व॒तः क्षा॑यि क्षायि कुर्व॒तो मे᳚ । \newline
29. कु॒र्व॒तो मे॑ मे कुर्व॒तः कु॑र्व॒तो मे॒ मा मा मे॑ कुर्व॒तः कु॑र्व॒तो मे॒ मा । \newline
30. मे॒ मा मा मे॑ मे॒ मोपोप॒ मा मे॑ मे॒ मोप॑ । \newline
31. मोपोप॒ मा मोप॑ दसद् दस॒दुप॒ मा मोप॑ दसत् । \newline
32. उप॑ दसद् दस॒दुपोप॑ दसत् प्र॒जाप॑तेः प्र॒जाप॑तेर् दस॒दुपोप॑ दसत् प्र॒जाप॑तेः । \newline
33. द॒स॒त् प्र॒जाप॑तेः प्र॒जाप॑तेर् दसद् दसत् प्र॒जाप॑तेर् भा॒गो भा॒गः प्र॒जाप॑तेर् दसद् दसत् प्र॒जाप॑तेर् भा॒गः । \newline
34. प्र॒जाप॑तेर् भा॒गो भा॒गः प्र॒जाप॑तेः प्र॒जाप॑तेर् भा॒गो᳚ ऽस्यसि भा॒गः प्र॒जाप॑तेः प्र॒जाप॑तेर् भा॒गो॑ ऽसि । \newline
35. प्र॒जाप॑ते॒रिति॑ प्र॒जा - प॒तेः॒ । \newline
36. भा॒गो᳚ ऽस्यसि भा॒गो भा॒गो᳚ ऽस्यूर्ज॑स्वा॒ नूर्ज॑स्वा नसि भा॒गो भा॒गो᳚ ऽस्यूर्ज॑स्वान् । \newline
37. अ॒स्यूर्ज॑स्वा॒ नूर्ज॑स्वा नस्य॒स्यूर्ज॑स्वा॒न् पय॑स्वा॒न् पय॑स्वा॒ नूर्ज॑स्वा नस्य॒स्यूर्ज॑स्वा॒न् पय॑स्वान् । \newline
38. ऊर्ज॑स्वा॒न् पय॑स्वा॒न् पय॑स्वा॒ नूर्ज॑स्वा॒ नूर्ज॑स्वा॒न् पय॑स्वान् प्राणापा॒नौ प्रा॑णापा॒नौ पय॑स्वा॒ नूर्ज॑स्वा॒ नूर्ज॑स्वा॒न् पय॑स्वान् प्राणापा॒नौ । \newline
39. पय॑स्वान् प्राणापा॒नौ प्रा॑णापा॒नौ पय॑स्वा॒न् पय॑स्वान् प्राणापा॒नौ मे॑ मे प्राणापा॒नौ पय॑स्वा॒न् पय॑स्वान् प्राणापा॒नौ मे᳚ । \newline
40. प्रा॒णा॒पा॒नौ मे॑ मे प्राणापा॒नौ प्रा॑णापा॒नौ मे॑ पाहि पाहि मे प्राणापा॒नौ प्रा॑णापा॒नौ मे॑ पाहि । \newline
41. प्रा॒णा॒पा॒नाविति॑ प्राण - अ॒पा॒नौ । \newline
42. मे॒ पा॒हि॒ पा॒हि॒ मे॒ मे॒ पा॒हि॒ स॒मा॒न॒व्या॒नौ स॑मानव्या॒नौ पा॑हि मे मे पाहि समानव्या॒नौ । \newline
43. पा॒हि॒ स॒मा॒न॒व्या॒नौ स॑मानव्या॒नौ पा॑हि पाहि समानव्या॒नौ मे॑ मे समानव्या॒नौ पा॑हि पाहि समानव्या॒नौ मे᳚ । \newline
44. स॒मा॒न॒व्या॒नौ मे॑ मे समानव्या॒नौ स॑मानव्या॒नौ मे॑ पाहि पाहि मे समानव्या॒नौ स॑मानव्या॒नौ मे॑ पाहि । \newline
45. स॒मा॒न॒व्या॒नाविति॑ समान - व्या॒नौ । \newline
46. मे॒ पा॒हि॒ पा॒हि॒ मे॒ मे॒ पा॒ह्यु॒दा॒न॒व्या॒ना वु॑दानव्या॒नौ पा॑हि मे मे पाह्युदानव्या॒नौ । \newline
47. पा॒ह्यु॒दा॒न॒व्या॒ना वु॑दानव्या॒नौ पा॑हि पाह्युदानव्या॒नौ मे॑ म उदानव्या॒नौ पा॑हि पाह्युदानव्या॒नौ मे᳚ । \newline
48. उ॒दा॒न॒व्या॒नौ मे॑ म उदानव्या॒ना वु॑दानव्या॒नौ मे॑ पाहि पाहि म उदानव्या॒ना वु॑दानव्या॒नौ मे॑ पाहि । \newline
49. उ॒दा॒न॒व्या॒नावित्यु॑दान - व्या॒नौ । \newline
50. मे॒ पा॒हि॒ पा॒हि॒ मे॒ मे॒ पा॒ह्यक्षि॒तो ऽक्षि॑तः पाहि मे मे पा॒ह्यक्षि॑तः । \newline
51. पा॒ह्यक्षि॒तो ऽक्षि॑तः पाहि पा॒ह्यक्षि॑तो ऽस्य॒स्यक्षि॑तः पाहि पा॒ह्यक्षि॑तो ऽसि । \newline
52. अक्षि॑तो ऽस्य॒स्यक्षि॒तो ऽक्षि॑तो॒ ऽस्यक्षि॑त्या॒ अक्षि॑त्या अ॒स्यक्षि॒तो ऽक्षि॑तो॒ ऽस्यक्षि॑त्यै । \newline
53. अ॒स्यक्षि॑त्या॒ अक्षि॑त्या अस्य॒स्यक्षि॑त्यै त्वा॒ त्वा ऽक्षि॑त्या अस्य॒स्यक्षि॑त्यै त्वा । \newline
54. अक्षि॑त्यै त्वा॒ त्वा ऽक्षि॑त्या॒ अक्षि॑त्यै त्वा॒ मा मा त्वा ऽक्षि॑त्या॒ अक्षि॑त्यै त्वा॒ मा । \newline
55. त्वा॒ मा मा त्वा᳚ त्वा॒ मा मे॑ मे॒ मा त्वा᳚ त्वा॒ मा मे᳚ । \newline
56. मा मे॑ मे॒ मा मा मे᳚ क्षेष्ठाः क्षेष्ठा मे॒ मा मा मे᳚ क्षेष्ठाः । \newline
57. मे॒ क्षे॒ष्ठाः॒ क्षे॒ष्ठा॒ मे॒ मे॒ क्षे॒ष्ठा॒ अ॒मुत्रा॒मुत्र॑ क्षेष्ठा मे मे क्षेष्ठा अ॒मुत्र॑ । \newline
58. क्षे॒ष्ठा॒ अ॒मुत्रा॒मुत्र॑ क्षेष्ठाः क्षेष्ठा अ॒मुत्रा॒मुष्मि॑न् न॒मुष्मि॑न् न॒मुत्र॑ क्षेष्ठाः क्षेष्ठा अ॒मुत्रा॒मुष्मिन्न्॑ । \newline
59. अ॒मुत्रा॒मुष्मि॑न् न॒मुष्मि॑न् न॒मुत्रा॒मुत्रा॒मुष्मि॑न्न् ॅलो॒के लो॒के॑ ऽमुष्मि॑न् न॒मुत्रा॒मुत्रा॒मुष्मि॑न्न् ॅलो॒के । \newline
60. अ॒मुष्मि॑न्न् ॅलो॒के लो॒के॑ ऽमुष्मि॑न् न॒मुष्मि॑न् ॅलो॒के । \newline
61. लो॒क इति॑ लो॒के । \newline
\pagebreak
\markright{ TS 1.6.4.1  \hfill https://www.vedavms.in \hfill}

\section{ TS 1.6.4.1 }

\textbf{TS 1.6.4.1 } \newline
\textbf{Samhita Paata} \newline

ब॒र्॒.हिषो॒ऽहं दे॑वय॒ज्यया᳚ प्र॒जावा᳚न् भूयासं॒ नरा॒शꣳस॑स्या॒हं दे॑वय॒ज्यया॑ पशु॒मान् भू॑यासम॒ग्नेः स्वि॑ष्ट॒कृतो॒ऽहं दे॑वय॒ज्ययाऽऽयु॑ष्मान्. य॒ज्ञेन॑ प्रति॒ष्ठां ग॑मेयम॒ग्नेर॒ह-मुज्जि॑ति॒-मनूज्जे॑षꣳ॒॒ सोम॑स्या॒ह - मुज्जि॑ति॒-मनूज्जे॑षम॒ग्नेर॒ह-मुज्जि॑ति॒-मनूज्जे॑ष-म॒ग्नीषोम॑योर॒ह-मुज्जि॑ति॒-मनूज्जे॑ष-मिन्द्राग्नि॒योर॒ह-मुज्जि॑ति॒-मनूज्जे॑ष॒-मिन्द्र॑स्या॒ह- [ ] \newline

\textbf{Pada Paata} \newline

ब॒र्॒.हिषः॑ । अ॒हम् । दे॒व॒य॒ज्ययेति॑ देव - य॒ज्यया᳚ । प्र॒जावा॒निति॑ प्र॒जा - वा॒न् । भू॒या॒स॒म् । नरा॒शꣳस॑स्य । अ॒हम् । दे॒व॒य॒ज्ययेति॑ देव - य॒ज्यया᳚ । प॒शु॒मानिति॑ पशु - मान् । भू॒या॒स॒म् । अ॒ग्नेः । स्वि॒ष्ट॒कृत॒ इति॑ स्विष्ट - कृतः॑ । अ॒हम् । दे॒व॒य॒ज्ययेति॑ देव - य॒ज्यया᳚ । आयु॑ष्मान् । य॒ज्ञेन॑ । प्र॒ति॒ष्ठामिति॑ प्रति - स्थाम् । ग॒मे॒य॒म् । अ॒ग्नेः । अ॒हम् । उज्जि॑ति॒मित्युत् - जि॒ति॒म् । अनु॑ । उदिति॑ । जे॒ष॒म् । सोम॑स्य । अ॒हम् । उज्जि॑ति॒मित्युत् - जि॒ति॒म् । अनु॑ । उदिति॑ । जे॒ष॒म् । अ॒ग्नेः । अ॒हम् । उज्जि॑ति॒मित्युत्-जि॒ति॒म् । अनु॑ । उदिति॑ । जे॒ष॒म् । अ॒ग्नीषोम॑यो॒रित्य॒ग्नी - सोम॑योः । अ॒हम् । उज्जि॑ति॒मित्युत् - जि॒ति॒म् । अनु॑ । उदिति॑ । जे॒ष॒म् । इ॒न्द्रा॒ग्नि॒योरिती᳚न्द्र - अ॒ग्नि॒योः । अ॒हम् । उज्जि॑ति॒मित्युत्-जि॒ति॒म् । अनु॑ । उदिति॑ । जे॒ष॒म् । इन्द्र॑स्य । अ॒हम् ।  \newline


\textbf{Krama Paata} \newline

ब॒र्.॒हिषो॒ऽहम् । अ॒हम् दे॑वय॒ज्यया᳚ । दे॒व॒य॒ज्यया᳚ प्र॒जावान्॑ । दे॒व॒य॒ज्ययेति॑ देव - य॒ज्यया᳚ । प्र॒जावा᳚न्,भूयासम् । प्र॒जावा॒निति॑ प्र॒जा - वा॒न्॒ । भू॒या॒स॒म् नरा॒शꣳस॑स्य । नरा॒शꣳस॑स्या॒हम् । अ॒हम् दे॑वय॒ज्यया᳚ । दे॒व॒य॒ज्यया॑ पशु॒मान् । दे॒व॒य॒ज्ययेति॑ देव - य॒ज्यया᳚ । प॒शु॒मान् भू॑यासम् । प॒शु॒मानिति॑ पशु - मान् । भू॒या॒स॒म॒ग्नेः । अ॒ग्नेः स्वि॑ष्ट॒कृतः॑ । स्वि॒ष्ट॒कृतो॒ऽहम् । स्वि॒ष्ट॒कृत॒ इति॑ स्विष्ट - कृतः॑ । अ॒हम् दे॑वय॒ज्यया᳚ । दे॒व॒य॒ज्यया ऽऽयु॑ष्मान् । दे॒व॒य॒ज्ययेति॑ देव - य॒ज्यया᳚ । आयु॑ष्मान्. य॒ज्ञेन॑ । य॒ज्ञेन॑ प्रति॒ष्ठाम् । प्र॒ति॒ष्ठाम् ग॑मेयम् । प्र॒ति॒ष्ठामिति॑ प्रति - स्थाम् । ग॒मे॒य॒म॒ग्नेः । अ॒ग्नेर॒हम् । अ॒हमुज्जि॑तिम् । उज्जि॑ति॒मनु॑ । उज्जि॑ति॒मित्युत् - जि॒ति॒म् । अनूत् । उज्जे॑षम् । जे॒षꣳ॒॒ सोम॑स्य । सोम॑स्या॒हम् । अ॒हमुज्जि॑तिम् । उज्जि॑ति॒मनु॑ । उज्जि॑ति॒मित्युत् - जि॒ति॒म् । अनूत् । उज्जे॑षम् । जे॒ष॒म॒ग्नेः । अ॒ग्नेर॒हम् । अ॒हमुज्जि॑तिम् । उज्जि॑ति॒मनु॑ । उज्जि॑ति॒मित्युत् - जि॒ति॒म् । अनूत् । उज्जे॑षम् । जे॒ष॒म॒ग्नीषोम॑योः । अ॒ग्नीषोम॑योर॒हम् । अ॒ग्नीषोम॑यो॒रित्य॒ग्नी - सोम॑योः । अ॒हमुज्जि॑तिम् । उज्जि॑ति॒मनु॑ । उज्जि॑ति॒मित्युत् - जि॒ति॒म् । अनूत् । उज्जे॑षम् । जे॒ष॒मि॒न्द्रा॒ग्नि॒योः । इ॒न्द्रा॒ग्नि॒योर॒हम् । इ॒न्द्रा॒ग्नि॒योरिती᳚न्द्र - अ॒ग्नि॒योः । अ॒हमुज्जि॑तिम् । उज्जि॑ति॒मनु॑ । उज्जि॑ति॒मित्युत् - जि॒ति॒म् । अनूत् । उज्जे॑षम् । जे॒ष॒मिन्द्र॑स्य । इन्द्र॑स्या॒हम् । अ॒हमुज्जि॑तिम् \newline

\textbf{Jatai Paata} \newline

1. ब॒र्॒.हिषो॒ ऽह म॒हम् ब॒र्॒.हिषो॑ ब॒र्॒.हिषो॒ ऽहम् । \newline
2. अ॒हम् दे॑वय॒ज्यया॑ देवय॒ज्यया॒ ऽह म॒हम् दे॑वय॒ज्यया᳚ । \newline
3. दे॒व॒य॒ज्यया᳚ प्र॒जावा᳚न् प्र॒जावा᳚न् देवय॒ज्यया॑ देवय॒ज्यया᳚ प्र॒जावान्॑ । \newline
4. दे॒व॒य॒ज्ययेति॑ देव - य॒ज्यया᳚ । \newline
5. प्र॒जावा᳚न् भूयासम् भूयासम् प्र॒जावा᳚न् प्र॒जावा᳚न् भूयासम् । \newline
6. प्र॒जावा॒निति॑ प्र॒जा - वा॒न् । \newline
7. भू॒या॒स॒म् नरा॒शꣳस॑स्य॒ नरा॒शꣳस॑स्य भूयासम् भूयास॒म् नरा॒शꣳस॑स्य । \newline
8. नरा॒शꣳस॑स्या॒ह म॒हम् नरा॒शꣳस॑स्य॒ नरा॒शꣳस॑स्या॒हम् । \newline
9. अ॒हम् दे॑वय॒ज्यया॑ देवय॒ज्यया॒ ऽह म॒हम् दे॑वय॒ज्यया᳚ । \newline
10. दे॒व॒य॒ज्यया॑ पशु॒मान् प॑शु॒मान् दे॑वय॒ज्यया॑ देवय॒ज्यया॑ पशु॒मान् । \newline
11. दे॒व॒य॒ज्ययेति॑ देव - य॒ज्यया᳚ । \newline
12. प॒शु॒मान् भू॑यासम् भूयासम् पशु॒मान् प॑शु॒मान् भू॑यासम् । \newline
13. प॒शु॒मानिति॑ पशु - मान् । \newline
14. भू॒या॒स॒ म॒ग्ने र॒ग्नेर् भू॑यासम् भूयास म॒ग्नेः । \newline
15. अ॒ग्नेः स्वि॑ष्ट॒कृतः॑ स्विष्ट॒कृतो॒ ऽग्ने र॒ग्नेः स्वि॑ष्ट॒कृतः॑ । \newline
16. स्वि॒ष्ट॒कृतो॒ ऽह म॒हꣳ स्वि॑ष्ट॒कृतः॑ स्विष्ट॒कृतो॒ ऽहम् । \newline
17. स्वि॒ष्ट॒कृत॒ इति॑ स्विष्ट - कृतः॑ । \newline
18. अ॒हम् दे॑वय॒ज्यया॑ देवय॒ज्यया॒ ऽह म॒हम् दे॑वय॒ज्यया᳚ । \newline
19. दे॒व॒य॒ज्यया ऽऽयु॑ष्मा॒ नायु॑ष्मान् देवय॒ज्यया॑ देवय॒ज्यया ऽऽयु॑ष्मान् । \newline
20. दे॒व॒य॒ज्ययेति॑ देव - य॒ज्यया᳚ । \newline
21. आयु॑ष्मान्. य॒ज्ञेन॑ य॒ज्ञेनायु॑ष्मा॒ नायु॑ष्मान्. य॒ज्ञेन॑ । \newline
22. य॒ज्ञेन॑ प्रति॒ष्ठाम् प्र॑ति॒ष्ठां ॅय॒ज्ञेन॑ य॒ज्ञेन॑ प्रति॒ष्ठाम् । \newline
23. प्र॒ति॒ष्ठाम् ग॑मेयम् गमेयम् प्रति॒ष्ठाम् प्र॑ति॒ष्ठाम् ग॑मेयम् । \newline
24. प्र॒ति॒ष्ठामिति॑ प्रति - स्थाम् । \newline
25. ग॒मे॒य॒ म॒ग्ने र॒ग्नेर् ग॑मेयम् गमेय म॒ग्नेः । \newline
26. अ॒ग्ने र॒ह म॒ह म॒ग्ने र॒ग्नेर॒हम् । \newline
27. अ॒ह मुज्जि॑ति॒ मुज्जि॑ति म॒ह म॒ह मुज्जि॑तिम् । \newline
28. उज्जि॑ति॒ मन्वनूज्जि॑ति॒ मुज्जि॑ति॒ मनु॑ । \newline
29. उज्जि॑ति॒मित्युत् - जि॒ति॒म् । \newline
30. अनू दुदन्वनूत् । \newline
31. उज् जे॑षम् जेष॒ मुदुज् जे॑षम् । \newline
32. जे॒ष॒(ग्म्॒) सोम॑स्य॒ सोम॑स्य जेषम् जेष॒(ग्म्॒) सोम॑स्य । \newline
33. सोम॑स्या॒ह म॒हꣳ सोम॑स्य॒ सोम॑स्या॒हम् । \newline
34. अ॒ह मुज्जि॑ति॒ मुज्जि॑ति म॒ह म॒ह मुज्जि॑तिम् । \newline
35. उज्जि॑ति॒ मन्वनूज्जि॑ति॒ मुज्जि॑ति॒ मनु॑ । \newline
36. उज्जि॑ति॒मित्युत् - जि॒ति॒म् । \newline
37. अनू दुदन्वनूत् । \newline
38. उज् जे॑षम् जेष॒ मुदुज् जे॑षम् । \newline
39. जे॒ष॒ म॒ग्ने र॒ग्नेर् जे॑षम् जेष म॒ग्नेः । \newline
40. अ॒ग्ने र॒ह म॒ह म॒ग्ने र॒ग्नेर॒हम् । \newline
41. अ॒ह मुज्जि॑ति॒ मुज्जि॑ति म॒ह म॒ह मुज्जि॑तिम् । \newline
42. उज्जि॑ति॒ मन्वनूज्जि॑ति॒ मुज्जि॑ति॒ मनु॑ । \newline
43. उज्जि॑ति॒मित्युत् - जि॒ति॒म् । \newline
44. अनू दुदन्वनूत् । \newline
45. उज् जे॑षम् जेष॒ मुदुज् जे॑षम् । \newline
46. जे॒ष॒ म॒ग्नीषोम॑यो र॒ग्नीषोम॑योर् जेषम् जेष म॒ग्नीषोम॑योः । \newline
47. अ॒ग्नीषोम॑यो र॒ह म॒ह म॒ग्नीषोम॑यो र॒ग्नीषोम॑यो र॒हम् । \newline
48. अ॒ग्नीषोम॑यो॒रित्य॒ग्नी - सोम॑योः । \newline
49. अ॒ह मुज्जि॑ति॒ मुज्जि॑ति म॒ह म॒ह मुज्जि॑तिम् । \newline
50. उज्जि॑ति॒ मन्वनूज्जि॑ति॒ मुज्जि॑ति॒ मनु॑ । \newline
51. उज्जि॑ति॒मित्युत् - जि॒ति॒म् । \newline
52. अनू दुदन्वनूत् । \newline
53. उज् जे॑षम् जेष॒ मुदुज् जे॑षम् । \newline
54. जे॒ष॒ मि॒न्द्रा॒ग्नि॒यो रि॑न्द्राग्नि॒योर् जे॑षम् जेष मिन्द्राग्नि॒योः । \newline
55. इ॒न्द्रा॒ग्नि॒यो र॒ह म॒ह मि॑न्द्राग्नि॒यो रि॑न्द्राग्नि॒यो र॒हम् । \newline
56. इ॒न्द्रा॒ग्नि॒योरिती᳚न्द्र - अ॒ग्नि॒योः । \newline
57. अ॒ह मुज्जि॑ति॒ मुज्जि॑ति म॒ह म॒ह मुज्जि॑तिम् । \newline
58. उज्जि॑ति॒ मन्वनूज्जि॑ति॒ मुज्जि॑ति॒ मनु॑ । \newline
59. उज्जि॑ति॒मित्युत् - जि॒ति॒म् । \newline
60. अनू दुदन्वनूत् । \newline
61. उज् जे॑षम् जेष॒ मुदुज् जे॑षम् । \newline
62. जे॒ष॒ मिन्द्र॒स्ये न्द्र॑स्य जेषम् जेष॒ मिन्द्र॑स्य । \newline
63. इन्द्र॑स्या॒ह म॒ह मिन्द्र॒स्ये न्द्र॑स्या॒हम् । \newline
64. अ॒ह मुज्जि॑ति॒ मुज्जि॑ति म॒ह म॒ह मुज्जि॑तिम् । \newline

\textbf{Ghana Paata } \newline

1. ब॒र्॒.हिषो॒ ऽह म॒हम् ब॒र्॒.हिषो॑ ब॒र्॒.हिषो॒ ऽहम् दे॑वय॒ज्यया॑ देवय॒ज्यया॒ ऽहम् ब॒र्॒.हिषो॑ ब॒र्॒.हिषो॒ ऽहम् दे॑वय॒ज्यया᳚ । \newline
2. अ॒हम् दे॑वय॒ज्यया॑ देवय॒ज्यया॒ ऽह म॒हम् दे॑वय॒ज्यया᳚ प्र॒जावा᳚न् प्र॒जावा᳚न् देवय॒ज्यया॒ ऽह म॒हम् दे॑वय॒ज्यया᳚ प्र॒जावान्॑ । \newline
3. दे॒व॒य॒ज्यया᳚ प्र॒जावा᳚न् प्र॒जावा᳚न् देवय॒ज्यया॑ देवय॒ज्यया᳚ प्र॒जावा᳚न् भूयासम् भूयासम् प्र॒जावा᳚न् देवय॒ज्यया॑ देवय॒ज्यया᳚ प्र॒जावा᳚न् भूयासम् । \newline
4. दे॒व॒य॒ज्ययेति॑ देव - य॒ज्यया᳚ । \newline
5. प्र॒जावा᳚न् भूयासम् भूयासम् प्र॒जावा᳚न् प्र॒जावा᳚न् भूयास॒म् नरा॒शꣳस॑स्य॒ नरा॒शꣳस॑स्य भूयासम् प्र॒जावा᳚न् प्र॒जावा᳚न् भूयास॒म् नरा॒शꣳस॑स्य । \newline
6. प्र॒जावा॒निति॑ प्र॒जा - वा॒न् । \newline
7. भू॒या॒स॒म् नरा॒शꣳस॑स्य॒ नरा॒शꣳस॑स्य भूयासम् भूयास॒म् नरा॒शꣳस॑स्या॒ह म॒हम् नरा॒शꣳस॑स्य भूयासम् भूयास॒म् नरा॒शꣳस॑स्या॒हम् । \newline
8. नरा॒शꣳस॑स्या॒ह म॒हन्नरा॒शꣳस॑स्य॒ नरा॒शꣳस॑स्या॒हम् दे॑वय॒ज्यया॑ देवय॒ज्यया॒ ऽहन्नरा॒शꣳस॑स्य॒ नरा॒शꣳस॑स्या॒हम् दे॑वय॒ज्यया᳚ । \newline
9. अ॒हम् दे॑वय॒ज्यया॑ देवय॒ज्यया॒ ऽह म॒हम् दे॑वय॒ज्यया॑ पशु॒मान् प॑शु॒मान् दे॑वय॒ज्यया॒ ऽह म॒हम् दे॑वय॒ज्यया॑ पशु॒मान् । \newline
10. दे॒व॒य॒ज्यया॑ पशु॒मान् प॑शु॒मान् दे॑वय॒ज्यया॑ देवय॒ज्यया॑ पशु॒मान् भू॑यासम् भूयासम् पशु॒मान् दे॑वय॒ज्यया॑ देवय॒ज्यया॑ पशु॒मान् भू॑यासम् । \newline
11. दे॒व॒य॒ज्ययेति॑ देव - य॒ज्यया᳚ । \newline
12. प॒शु॒मान् भू॑यासम् भूयासम् पशु॒मान् प॑शु॒मान् भू॑यास म॒ग्नेर॒ग्नेर् भू॑यासम् पशु॒मान् प॑शु॒मान् भू॑यास म॒ग्नेः । \newline
13. प॒शु॒मानिति॑ पशु - मान् । \newline
14. भू॒या॒स॒ म॒ग्नेर॒ग्नेर् भू॑यासम् भूयास म॒ग्नेः स्वि॑ष्ट॒कृतः॑ स्विष्ट॒कृतो॒ ऽग्नेर् भू॑यासम् भूयास म॒ग्नेः स्वि॑ष्ट॒कृतः॑ । \newline
15. अ॒ग्नेः स्वि॑ष्ट॒कृतः॑ स्विष्ट॒कृतो॒ ऽग्नेर॒ग्नेः स्वि॑ष्ट॒कृतो॒ ऽह म॒हꣳ स्वि॑ष्ट॒कृतो॒ ऽग्नेर॒ग्नेः स्वि॑ष्ट॒कृतो॒ ऽहम् । \newline
16. स्वि॒ष्ट॒कृतो॒ ऽह म॒हꣳ स्वि॑ष्ट॒कृतः॑ स्विष्ट॒कृतो॒ ऽहम् दे॑वय॒ज्यया॑ देवय॒ज्यया॒ ऽहꣳ स्वि॑ष्ट॒कृतः॑ स्विष्ट॒कृतो॒ ऽहम् दे॑वय॒ज्यया᳚ । \newline
17. स्वि॒ष्ट॒कृत॒ इति॑ स्विष्ट - कृतः॑ । \newline
18. अ॒हम् दे॑वय॒ज्यया॑ देवय॒ज्यया॒ ऽह म॒हम् दे॑वय॒ज्यया ऽऽयु॑ष्मा॒ नायु॑ष्मान् देवय॒ज्यया॒ ऽह म॒हम् दे॑वय॒ज्यया ऽऽयु॑ष्मान् । \newline
19. दे॒व॒य॒ज्यया ऽऽयु॑ष्मा॒ नायु॑ष्मान् देवय॒ज्यया॑ देवय॒ज्यया ऽऽयु॑ष्मान्. य॒ज्ञेन॑ य॒ज्ञेनायु॑ष्मान् देवय॒ज्यया॑ देवय॒ज्यया ऽऽयु॑ष्मान्. य॒ज्ञेन॑ । \newline
20. दे॒व॒य॒ज्ययेति॑ देव - य॒ज्यया᳚ । \newline
21. आयु॑ष्मान्. य॒ज्ञेन॑ य॒ज्ञेनायु॑ष्मा॒ नायु॑ष्मान्. य॒ज्ञेन॑ प्रति॒ष्ठाम् प्र॑ति॒ष्ठां ॅय॒ज्ञेनायु॑ष्मा॒ नायु॑ष्मान्. य॒ज्ञेन॑ प्रति॒ष्ठाम् । \newline
22. य॒ज्ञेन॑ प्रति॒ष्ठाम् प्र॑ति॒ष्ठां ॅय॒ज्ञेन॑ य॒ज्ञेन॑ प्रति॒ष्ठाम् ग॑मेयम् गमेयम् प्रति॒ष्ठां ॅय॒ज्ञेन॑ य॒ज्ञेन॑ प्रति॒ष्ठाम् ग॑मेयम् । \newline
23. प्र॒ति॒ष्ठाम् ग॑मेयम् गमेयम् प्रति॒ष्ठाम् प्र॑ति॒ष्ठाम् ग॑मेय म॒ग्नेर॒ग्नेर् ग॑मेयम् प्रति॒ष्ठाम् प्र॑ति॒ष्ठाम् ग॑मेय म॒ग्नेः । \newline
24. प्र॒ति॒ष्ठामिति॑ प्रति - स्थाम् । \newline
25. ग॒मे॒य॒ म॒ग्नेर॒ग्नेर् ग॑मेयम् गमेय म॒ग्नेर॒ह म॒ह म॒ग्नेर् ग॑मेयम् गमेय म॒ग्नेर॒हम् । \newline
26. अ॒ग्नेर॒ह म॒ह म॒ग्ने र॒ग्नेर॒ह मुज्जि॑ति॒ मुज्जि॑ति म॒ह म॒ग्ने र॒ग्नेर॒ह मुज्जि॑तिम् । \newline
27. अ॒ह मुज्जि॑ति॒ मुज्जि॑ति म॒ह म॒ह मुज्जि॑ति॒ मन्वनूज्जि॑ति म॒ह म॒ह मुज्जि॑ति॒ मनु॑ । \newline
28. उज्जि॑ति॒ मन्वनूज्जि॑ति॒ मुज्जि॑ति॒ मनूदुदनूज्जि॑ति॒ मुज्जि॑ति॒ मनूत् । \newline
29. उज्जि॑ति॒मित्युत् - जि॒ति॒म् । \newline
30. अनूदुदन्वनूज् जे॑षम् जेष॒ मुदन्वनूज् जे॑षम् । \newline
31. उज् जे॑षम् जेष॒ मुदुज् जे॑ष॒(ग्म्॒) सोम॑स्य॒ सोम॑स्य जेष॒ मुदुज् जे॑ष॒(ग्म्॒) सोम॑स्य । \newline
32. जे॒ष॒(ग्म्॒) सोम॑स्य॒ सोम॑स्य जेषम् जेष॒(ग्म्॒) सोम॑स्या॒ह म॒हꣳ सोम॑स्य जेषम् जेष॒(ग्म्॒) सोम॑स्या॒हम् । \newline
33. सोम॑स्या॒ह म॒हꣳ सोम॑स्य॒ सोम॑स्या॒ह मुज्जि॑ति॒ मुज्जि॑ति म॒हꣳ सोम॑स्य॒ सोम॑स्या॒ह मुज्जि॑तिम् । \newline
34. अ॒ह मुज्जि॑ति॒ मुज्जि॑ति म॒ह म॒ह मुज्जि॑ति॒ मन्वनूज्जि॑ति म॒ह म॒ह मुज्जि॑ति॒ मनु॑ । \newline
35. उज्जि॑ति॒ मन्वनूज्जि॑ति॒ मुज्जि॑ति॒ मनूदुदनूज्जि॑ति॒ मुज्जि॑ति॒ मनूत् । \newline
36. उज्जि॑ति॒मित्युत् - जि॒ति॒म् । \newline
37. अनूदुदन्वनूज् जे॑षम् जेष॒ मुदन्वनूज् जे॑षम् । \newline
38. उज् जे॑षम् जेष॒ मुदुज् जे॑ष म॒ग्नेर॒ग्नेर् जे॑ष॒ मुदुज् जे॑ष म॒ग्नेः । \newline
39. जे॒ष॒ म॒ग्ने र॒ग्नेर् जे॑षम् जेष म॒ग्नेर॒ह म॒ह म॒ग्नेर् जे॑षम् जेष म॒ग्नेर॒हम् । \newline
40. अ॒ग्नेर॒ह म॒ह म॒ग्ने र॒ग्नेर॒ह मुज्जि॑ति॒ मुज्जि॑ति म॒ह म॒ग्ने र॒ग्नेर॒ह मुज्जि॑तिम् । \newline
41. अ॒ह मुज्जि॑ति॒ मुज्जि॑ति म॒ह म॒ह मुज्जि॑ति॒ मन्वनूज्जि॑ति म॒ह म॒ह मुज्जि॑ति॒ मनु॑ । \newline
42. उज्जि॑ति॒ मन्वनूज्जि॑ति॒ मुज्जि॑ति॒ मनूदुदनूज्जि॑ति॒ मुज्जि॑ति॒ मनूत् । \newline
43. उज्जि॑ति॒मित्युत् - जि॒ति॒म् । \newline
44. अनूदुदन्वनूज् जे॑षम् जेष॒ मुदन्वनूज् जे॑षम् । \newline
45. उज् जे॑षम् जेष॒ मुदुज् जे॑ष म॒ग्नीषोम॑यो र॒ग्नीषोम॑योर् जेष॒ मुदुज् जे॑ष म॒ग्नीषोम॑योः । \newline
46. जे॒ष॒ म॒ग्नीषोम॑यो र॒ग्नीषोम॑योर् जेषम् जेष म॒ग्नीषोम॑योर॒ह म॒ह म॒ग्नीषोम॑योर् जेषम् जेष म॒ग्नीषोम॑योर॒हम् । \newline
47. अ॒ग्नीषोम॑योर॒ह म॒ह म॒ग्नीषोम॑यो र॒ग्नीषोम॑योर॒ह मुज्जि॑ति॒ मुज्जि॑ति म॒ह म॒ग्नीषोम॑यो र॒ग्नीषोम॑योर॒ह मुज्जि॑तिम् । \newline
48. अ॒ग्नीषोम॑यो॒रित्य॒ग्नी - सोम॑योः । \newline
49. अ॒ह मुज्जि॑ति॒ मुज्जि॑ति म॒ह म॒ह मुज्जि॑ति॒ मन्वनूज्जि॑ति म॒ह म॒ह मुज्जि॑ति॒ मनु॑ । \newline
50. उज्जि॑ति॒ मन्वनूज्जि॑ति॒ मुज्जि॑ति॒ मनू॑दुदनूज्जि॑ति॒ मुज्जि॑ति॒ मनूत् । \newline
51. उज्जि॑ति॒मित्युत् - जि॒ति॒म् । \newline
52. अनूदुदन्वनूज् जे॑षम् जेष॒ मुदन्वनूज् जे॑षम् । \newline
53. उज् जे॑षम् जेष॒ मुदुज् जे॑ष मिन्द्राग्नि॒यो रि॑न्द्राग्नि॒योर् जे॑ष॒ मुदुज् जे॑ष मिन्द्राग्नि॒योः । \newline
54. जे॒ष॒ मि॒न्द्रा॒ग्नि॒यो रि॑न्द्राग्नि॒योर् जे॑षम् जेष मिन्द्राग्नि॒योर॒ह म॒ह मि॑न्द्राग्नि॒योर् जे॑षम् जेष मिन्द्राग्नि॒योर॒हम् । \newline
55. इ॒न्द्रा॒ग्नि॒योर॒ह म॒ह मि॑न्द्राग्नि॒यो रि॑न्द्राग्नि॒योर॒ह मुज्जि॑ति॒ मुज्जि॑ति म॒ह मि॑न्द्राग्नि॒यो रि॑न्द्राग्नि॒योर॒ह मुज्जि॑तिम् । \newline
56. इ॒न्द्रा॒ग्नि॒योरिती᳚न्द्र - अ॒ग्नि॒योः । \newline
57. अ॒ह मुज्जि॑ति॒ मुज्जि॑ति म॒ह म॒ह मुज्जि॑ति॒ मन्वनूज्जि॑ति म॒ह म॒ह मुज्जि॑ति॒ मनु॑ । \newline
58. उज्जि॑ति॒ मन्वनूज्जि॑ति॒ मुज्जि॑ति॒ मनूदुदनूज्जि॑ति॒ मुज्जि॑ति॒ मनूत् । \newline
59. उज्जि॑ति॒मित्युत् - जि॒ति॒म् । \newline
60. अनूदुदन् वनूज् जे॑षम् जेष॒ मुदन्वनूज् जे॑षम् । \newline
61. उज् जे॑षम् जेष॒ मुदुज् जे॑ष॒ मिन्द्र॒स्ये न्द्र॑स्य जेष॒ मुदुज् जे॑ष॒ मिन्द्र॑स्य । \newline
62. जे॒ष॒ मिन्द्र॒स्ये न्द्र॑स्य जेषम् जेष॒ मिन्द्र॑स्या॒ह म॒ह मिन्द्र॑स्य जेषम् जेष॒ मिन्द्र॑स्या॒हम् । \newline
63. इन्द्र॑स्या॒ह म॒ह मिन्द्र॒स्ये न्द्र॑स्या॒ह मुज्जि॑ति॒ मुज्जि॑ति म॒ह मिन्द्र॒स्ये न्द्र॑स्या॒ह मुज्जि॑तिम् । \newline
64. अ॒ह मुज्जि॑ति॒ मुज्जि॑ति म॒ह म॒ह मुज्जि॑ति॒ मन्वनूज्जि॑ति म॒ह म॒ह मुज्जि॑ति॒ मनु॑ । \newline
\pagebreak
\markright{ TS 1.6.4.2  \hfill https://www.vedavms.in \hfill}

\section{ TS 1.6.4.2 }

\textbf{TS 1.6.4.2 } \newline
\textbf{Samhita Paata} \newline

मुज्जि॑ति॒मनूज्जे॑षं महे॒न्द्रस्या॒हमुज्जि॑ति॒- मनूज्जे॑षम॒ग्नेः स्वि॑ष्ट॒कृतो॒ऽह मुज्जि॑ति॒-मनूज्जे॑षं॒ ॅवाज॑स्य मा प्रस॒वेनो᳚द् ग्रा॒भेणोद॑ग्रभीत् । अथा॑ स॒पत्नाꣳ॒॒ इन्द्रो॑ मे निग्रा॒भेणाध॑राꣳ अकः ॥उ॒द्ग्रा॒भं च॑ निग्रा॒भं च॒ ब्रह्म॑ दे॒वा अ॑वीवृधन्न् । अथा॑ स॒पत्ना॑निन्द्रा॒ग्नी मे॑ विषू॒चीना॒न् व्य॑स्यतां ॥ एमा अ॑ग्मन्ना॒शिषो॒ दोह॑कामा॒ इन्द्र॑वन्तो - [ ] \newline

\textbf{Pada Paata} \newline

उज्जि॑ति॒मित्युत् - जि॒ति॒म् । अनु॑ । उदिति॑ । जे॒ष॒म् । म॒हे॒न्द्रस्येति॑ महा - इ॒न्द्रस्य॑ । अ॒हम् । उज्जि॑ति॒मित्युत्-जि॒ति॒म् । अनु॑ । उदिति॑ । जे॒ष॒म् । अ॒ग्नेः । स्वि॒ष्ट॒कृत॒ इति॑ स्विष्ट - कृतः॑ । अ॒हम् । उज्जि॑ति॒मित्युत् - जि॒ति॒म् । अनु॑ । उदिति॑ । जे॒ष॒म् । वाज॑स्य । मा॒ । प्र॒स॒वेनेति॑ प्र - स॒वेन॑ । उ॒द्ग्रा॒भेणेत्यु॑त् - ग्रा॒भेण॑ । उदिति॑ । अ॒ग्र॒भी॒त् ॥ अथ॑ । स॒पत्नान्॑ । इन्द्रः॑ । मे॒ । नि॒ग्रा॒भेणेति॑ नि - ग्रा॒भेण॑ । अध॑रान् । अ॒कः॒ ॥ उ॒द्ग्रा॒भमित्यु॑त् - ग्रा॒भम् । च॒ । नि॒ग्रा॒भमिति॑ नि - ग्रा॒भम् । च॒ । ब्रह्म॑ । दे॒वाः । अ॒वी॒वृ॒ध॒न्न् ॥ अथ॑ । स॒पत्नान्॑ । इ॒न्द्रा॒ग्नी इती᳚न्द्र - अ॒ग्नी । मे॒ । वि॒षू॒चीनान्॑ । वीति॑ । अ॒स्य॒ता॒म् ॥ एति॑ । इ॒माः । अ॒ग्म॒न्न् । आ॒शिष॒ इत्या᳚ - शिषः॑ । दोह॑कामा॒ इति॒ दोह॑ - का॒माः॒ । इन्द्र॑वन्त॒ इतीन्द्र॑- व॒न्तः॒ ।  \newline


\textbf{Krama Paata} \newline

उज्जि॑ति॒मनु॑ । उज्जि॑ति॒मित्युत् - जि॒ति॒म् । अनूत् । उज्जे॑षम् । जे॒ष॒म् म॒हे॒न्द्रस्य॑ । म॒हे॒न्द्रस्या॒हम् । म॒हे॒न्द्रस्येति॑ महा - इ॒न्द्रस्य॑ । अ॒हमुज्जि॑तिम् । उज्जि॑ति॒मनु॑ । उज्जि॑ति॒मित्युत् - जि॒ति॒म् । अनूत् । उज्जे॑षम् । जे॒ष॒म॒ग्नेः । अ॒ग्नेः स्वि॑ष्ट॒कृतः॑ । स्वि॒ष्ट॒कृतो॒ऽहम् । स्वि॒ष्ट॒कृत॒ इति॑ स्विष्ट - कृतः॑ । अ॒हमुज्जि॑तिम् । उज्जि॑ति॒मनु॑ । उज्जि॑ति॒मित्युत् - जि॒ति॒म् । अनूत् । उज्जे॑षम् । जे॒षं॒ ॅवाज॑स्य । वाज॑स्य मा । मा॒ प्र॒स॒वेन॑ । प्र॒स॒वेनो᳚द्ग्रा॒भेण॑ । प्र॒स॒वेनेति॑ प्र - स॒वेन॑ । उ॒द्ग्रा॒भेणोत् । उ॒द्ग्रा॒भेणेत्यु॑त् - ग्रा॒भेण॑ । उद॑ग्रभीत् । अ॒ग्र॒भी॒दित्य॑ग्रभीत् ॥ अथा॑ स॒पत्नान्॑ । स॒पत्नाꣳ॒॒ इन्द्रः॑ । इन्द्रो॑ मे । मे॒ नि॒ग्रा॒भेण॑ । नि॒ग्रा॒भेणाध॑रान् । नि॒ग्रा॒भेणेति॑ नि - ग्रा॒भेण॑ । अध॑राꣳ अकः । अ॒क॒रित्य॑कः ॥ उ॒द्ग्रा॒भम् च॑ । उ॒द्ग्रा॒भमित्यु॑त् - ग्रा॒भम् । च॒ नि॒ग्रा॒भम् । नि॒ग्रा॒भम् च॑ । नि॒ग्रा॒भमिति॑ नि - ग्रा॒भम् । च॒ ब्रह्म॑ । ब्रह्म॑ दे॒वाः । दे॒वा अ॑वीवृधन्न् । अ॒वी॒वृ॒ध॒न्नित्य॑वीवृधन्न् ॥ अथा॑ स॒पत्नान्॑ । स॒पत्ना॑निन्द्रा॒ग्नी । इ॒न्द्रा॒ग्नी मे᳚ । इ॒न्द्रा॒ग्नी इती᳚न्द्र - अ॒ग्नी । मे॒ वि॒षू॒चीनान्॑ । वि॒षू॒चीना॒न्॒. वि । व्य॑स्यताम् । अ॒स्य॒ता॒मित्य॑स्यताम् ॥ एमाः । इ॒मा अ॑ग्मन्न् । अ॒ग्म॒न्ना॒शिषः॑ । आ॒शिषो॒ दोह॑कामाः । आ॒शिष॒ इत्या᳚ - शिषः॑ । दोह॑कामा॒ इन्द्र॑वन्तः । दोह॑कामा॒ इति॒ दोह॑ - का॒माः॒ । इन्द्र॑वन्तो वनामहे । इन्द्र॑वन्त॒ इतीन्द्र॑ - व॒न्तः॒ \newline

\textbf{Jatai Paata} \newline

1. उज्जि॑ति॒ मन्वनूज्जि॑ति॒ मुज्जि॑ति॒ मनु॑ । \newline
2. उज्जि॑ति॒मित्युत् - जि॒ति॒म् । \newline
3. अनू दुदन्वनूत् । \newline
4. उज् जे॑षम् जेष॒ मुदुज् जे॑षम् । \newline
5. जे॒ष॒म् म॒हे॒न्द्रस्य॑ महे॒न्द्रस्य॑ जेषम् जेषम् महे॒न्द्रस्य॑ । \newline
6. म॒हे॒न्द्रस्या॒ह म॒हम् म॑हे॒न्द्रस्य॑ महे॒न्द्रस्या॒हम् । \newline
7. म॒हे॒न्द्रस्येति॑ महा - इ॒न्द्रस्य॑ । \newline
8. अ॒ह मुज्जि॑ति॒ मुज्जि॑ति म॒ह म॒ह मुज्जि॑तिम् । \newline
9. उज्जि॑ति॒ मन्वनूज्जि॑ति॒ मुज्जि॑ति॒ मनु॑ । \newline
10. उज्जि॑ति॒मित्युत् - जि॒ति॒म् । \newline
11. अनू दुदन्वनूत् । \newline
12. उज् जे॑षम् जेष॒ मुदुज् जे॑षम् । \newline
13. जे॒ष॒ म॒ग्ने र॒ग्नेर् जे॑षम् जेष म॒ग्नेः । \newline
14. अ॒ग्नेः स्वि॑ष्ट॒कृतः॑ स्विष्ट॒कृतो॒ ऽग्नेर॒ग्नेः स्वि॑ष्ट॒कृतः॑ । \newline
15. स्वि॒ष्ट॒कृतो॒ ऽह म॒हꣳ स्वि॑ष्ट॒कृतः॑ स्विष्ट॒कृतो॒ ऽहम् । \newline
16. स्वि॒ष्ट॒कृत॒ इति॑ स्विष्ट - कृतः॑ । \newline
17. अ॒ह मुज्जि॑ति॒ मुज्जि॑ति म॒ह म॒ह मुज्जि॑तिम् । \newline
18. उज्जि॑ति॒ मन्वनूज्जि॑ति॒ मुज्जि॑ति॒ मनु॑ । \newline
19. उज्जि॑ति॒मित्युत् - जि॒ति॒म् । \newline
20. अनू दुदन्वनूत् । \newline
21. उज् जे॑षम् जेष॒ मुदुज् जे॑षम् । \newline
22. जे॒षं॒ ॅवाज॑स्य॒ वाज॑स्य जेषम् जेषं॒ ॅवाज॑स्य । \newline
23. वाज॑स्य मा मा॒ वाज॑स्य॒ वाज॑स्य मा । \newline
24. मा॒ प्र॒स॒वेन॑ प्रस॒वेन॑ मा मा प्रस॒वेन॑ । \newline
25. प्र॒स॒वे नो᳚द्ग्रा॒भे णो᳚द्ग्रा॒भेण॑ प्रस॒वेन॑ प्रस॒वे नो᳚द्ग्रा॒भेण॑ । \newline
26. प्र॒स॒वेनेति॑ प्र - स॒वेन॑ । \newline
27. उ॒द्ग्रा॒भे णोदुदु॑द्ग्रा॒भे णो᳚द्ग्रा॒भेणोत् । \newline
28. उ॒द्ग्रा॒भेणेत्यु॑त् - ग्रा॒भेण॑ । \newline
29. उद॑ग्रभी दग्रभी॒ दुदु द॑ग्रभीत् । \newline
30. अ॒ग्र॒भी॒दित्य॑ग्रभीत् । \newline
31. अथा॑ स॒पत्ना᳚न् थ्स॒पत्ना॒(ग्म्॒) अथाथा॑ स॒पत्नान्॑ । \newline
32. स॒पत्ना॒(ग्म्॒) इन्द्र॒ इन्द्रः॑ स॒पत्ना᳚न् थ्स॒पत्ना॒(ग्म्॒) इन्द्रः॑ । \newline
33. इन्द्रो॑ मे म॒ इन्द्र॒ इन्द्रो॑ मे । \newline
34. मे॒ नि॒ग्रा॒भेण॑ निग्रा॒भेण॑ मे मे निग्रा॒भेण॑ । \newline
35. नि॒ग्रा॒भेणाध॑रा॒(ग्म्॒) अध॑रान् निग्रा॒भेण॑ निग्रा॒भेणाध॑रान् । \newline
36. नि॒ग्रा॒भेणेति॑ नि - ग्रा॒भेण॑ । \newline
37. अध॑राꣳ अक रक॒ रध॑ रा॒(ग्म्॒) अध॑राꣳ अकः । \newline
38. अ॒क॒रित्य॑कः । \newline
39. उ॒द्ग्रा॒भम् च॑ चोद्ग्रा॒भ मु॑द्ग्रा॒भम् च॑ । \newline
40. उ॒द्ग्रा॒भमित्यु॑त् - ग्रा॒भम् । \newline
41. च॒ नि॒ग्रा॒भम् नि॑ग्रा॒भम् च॑ च निग्रा॒भम् । \newline
42. नि॒ग्रा॒भम् च॑ च निग्रा॒भम् नि॑ग्रा॒भम् च॑ । \newline
43. नि॒ग्रा॒भमिति॑ नि - ग्रा॒भम् । \newline
44. च॒ ब्रह्म॒ ब्रह्म॑ च च॒ ब्रह्म॑ । \newline
45. ब्रह्म॑ दे॒वा दे॒वा ब्रह्म॒ ब्रह्म॑ दे॒वाः । \newline
46. दे॒वा अ॑वीवृधन् नवीवृधन् दे॒वा दे॒वा अ॑वीवृधन्न् । \newline
47. अ॒वी॒वृ॒ध॒न्नित्य॑वीवृधन्न् । \newline
48. अथा॑ स॒पत्ना᳚न् थ्स॒पत्ना॒ नथाथा॑ स॒पत्नान्॑ । \newline
49. स॒पत्ना॑ निन्द्रा॒ग्नी इ॑न्द्रा॒ग्नी स॒पत्ना᳚न् थ्स॒पत्ना॑ निन्द्रा॒ग्नी । \newline
50. इ॒न्द्रा॒ग्नी मे॑ म इन्द्रा॒ग्नी इ॑न्द्रा॒ग्नी मे᳚ । \newline
51. इ॒न्द्रा॒ग्नी इती᳚न्द्र - अ॒ग्नी । \newline
52. मे॒ वि॒षू॒चीनान्॑. विषू॒चीना᳚न् मे मे विषू॒चीनान्॑ । \newline
53. वि॒षू॒चीना॒न्॒. वि वि वि॑षू॒चीनान्॑. विषू॒चीना॒न्॒. वि । \newline
54. व्य॑स्यता मस्यतां॒ ॅवि व्य॑स्यताम् । \newline
55. अ॒स्य॒ता॒मित्य॑स्यताम् । \newline
56. एमा इ॒मा एमाः । \newline
57. इ॒मा अ॑ग्मन् नग्मन् नि॒मा इ॒मा अ॑ग्मन्न् । \newline
58. अ॒ग्म॒न् ना॒शिष॑ आ॒शिषो᳚ ऽग्मन् नग्मन् ना॒शिषः॑ । \newline
59. आ॒शिषो॒ दोह॑कामा॒ दोह॑कामा आ॒शिष॑ आ॒शिषो॒ दोह॑कामाः । \newline
60. आ॒शिष॒ इत्या᳚ - शिषः॑ । \newline
61. दोह॑कामा॒ इन्द्र॑वन्त॒ इन्द्र॑वन्तो॒ दोह॑कामा॒ दोह॑कामा॒ इन्द्र॑वन्तः । \newline
62. दोह॑कामा॒ इति॒ दोह॑ - का॒माः॒ । \newline
63. इन्द्र॑वन्तो वनामहे वनामह॒ इन्द्र॑वन्त॒ इन्द्र॑वन्तो वनामहे । \newline
64. इन्द्र॑वन्त॒ इतीन्द्र॑ - व॒न्तः॒ । \newline

\textbf{Ghana Paata } \newline

1. उज्जि॑ति॒ मन्वनूज्जि॑ति॒ मुज्जि॑ति॒ मनूदुदनूज्जि॑ति॒ मुज्जि॑ति॒ मनूत् । \newline
2. उज्जि॑ति॒मित्युत् - जि॒ति॒म् । \newline
3. अनूदुदन् वनूज् जे॑षम् जेष॒ मुदन्वनूज् जे॑षम् । \newline
4. उज् जे॑षम् जेष॒ मुदुज् जे॑षम् महे॒न्द्रस्य॑ महे॒न्द्रस्य॑ जेष॒ मुदुज् जे॑षम् महे॒न्द्रस्य॑ । \newline
5. जे॒ष॒म् म॒हे॒न्द्रस्य॑ महे॒न्द्रस्य॑ जेषम् जेषम् महे॒न्द्रस्या॒ह म॒हम् म॑हे॒न्द्रस्य॑ जेषम् जेषम् महे॒न्द्रस्या॒हम् । \newline
6. म॒हे॒न्द्रस्या॒ह म॒हम् म॑हे॒न्द्रस्य॑ महे॒न्द्रस्या॒ह मुज्जि॑ति॒ मुज्जि॑ति म॒हम् म॑हे॒न्द्रस्य॑ महे॒न्द्रस्या॒ह मुज्जि॑तिम् । \newline
7. म॒हे॒न्द्रस्येति॑ महा - इ॒न्द्रस्य॑ । \newline
8. अ॒ह मुज्जि॑ति॒ मुज्जि॑ति म॒ह म॒ह मुज्जि॑ति॒ मन्वनूज्जि॑ति म॒ह म॒ह मुज्जि॑ति॒ मनु॑ । \newline
9. उज्जि॑ति॒ मन्वनूज्जि॑ति॒ मुज्जि॑ति॒ मनूदुदनूज्जि॑ति॒ मुज्जि॑ति॒ मनूत् । \newline
10. उज्जि॑ति॒मित्युत् - जि॒ति॒म् । \newline
11. अनूदुदन् वनू᳚ज् जे॑षम् जेष॒ मुदन्वनूज् जे॑षम् । \newline
12. उज् जे॑षम् जेष॒ मुदुज् जे॑ष म॒ग्नेर॒ग्नेर् जे॑ष॒ मुदुज् जे॑ष म॒ग्नेः । \newline
13. जे॒ष॒ म॒ग्नेर॒ग्नेर् जे॑षम् जेष म॒ग्नेः स्वि॑ष्ट॒कृतः॑ स्विष्ट॒कृतो॒ ऽग्नेर् जे॑षम् जेष म॒ग्नेः स्वि॑ष्ट॒कृतः॑ । \newline
14. अ॒ग्नेः स्वि॑ष्ट॒कृतः॑ स्विष्ट॒कृतो॒ ऽग्नेर॒ग्नेः स्वि॑ष्ट॒कृतो॒ ऽह म॒हꣳ स्वि॑ष्ट॒कृतो॒ ऽग्नेर॒ग्नेः स्वि॑ष्ट॒कृतो॒ ऽहम् । \newline
15. स्वि॒ष्ट॒कृतो॒ ऽह म॒हꣳ स्वि॑ष्ट॒कृतः॑ स्विष्ट॒कृतो॒ ऽह मुज्जि॑ति॒ मुज्जि॑ति म॒हꣳ स्वि॑ष्ट॒कृतः॑ स्विष्ट॒कृतो॒ ऽह मुज्जि॑तिम् । \newline
16. स्वि॒ष्ट॒कृत॒ इति॑ स्विष्ट - कृतः॑ । \newline
17. अ॒ह मुज्जि॑ति॒ मुज्जि॑ति म॒ह म॒ह मुज्जि॑ति॒ मन्वनूज्जि॑ति म॒ह म॒ह मुज्जि॑ति॒ मनु॑ । \newline
18. उज्जि॑ति॒ मन्वनूज्जि॑ति॒ मुज्जि॑ति॒ मनूदुदनूज्जि॑ति॒ मुज्जि॑ति॒ मनूत् । \newline
19. उज्जि॑ति॒मित्युत् - जि॒ति॒म् । \newline
20. अनूदुदन् वनूज् जे॑षम् जेष॒ मुदन्वनूज् जे॑षम् । \newline
21. उज् जे॑षम् जेष॒ मुदुज् जे॑षं॒ ॅवाज॑स्य॒ वाज॑स्य जेष॒ मुदुज् जे॑षं॒ ॅवाज॑स्य । \newline
22. जे॒षं॒ ॅवाज॑स्य॒ वाज॑स्य जेषम् जेषं॒ ॅवाज॑स्य मा मा॒ वाज॑स्य जेषम् जेषं॒ ॅवाज॑स्य मा । \newline
23. वाज॑स्य मा मा॒ वाज॑स्य॒ वाज॑स्य मा प्रस॒वेन॑ प्रस॒वेन॑ मा॒ वाज॑स्य॒ वाज॑स्य मा प्रस॒वेन॑ । \newline
24. मा॒ प्र॒स॒वेन॑ प्रस॒वेन॑ मा मा प्रस॒वेनो᳚द्ग्रा॒भे णो᳚द्ग्रा॒भेण॑ प्रस॒वेन॑ मा मा प्रस॒वेनो᳚द्ग्रा॒भेण॑ । \newline
25. प्र॒स॒वेनो᳚द्ग्रा॒भे णो᳚द्ग्रा॒भेण॑ प्रस॒वेन॑ प्रस॒वेनो᳚द्ग्रा॒भे णोदुदु॑द्ग्रा॒भेण॑ प्रस॒वेन॑ प्रस॒वेनो᳚द्ग्रा॒भेणोत् । \newline
26. प्र॒स॒वेनेति॑ प्र - स॒वेन॑ । \newline
27. उ॒द्ग्रा॒भेणो दुदु॑द्ग्रा॒भे णो᳚द्ग्रा॒भे णो द॑ग्रभी दग्रभी॒ दुदु॑द्ग्रा॒भे णो᳚द्ग्रा॒भेणोद॑ग्रभीत् । \newline
28. उ॒द्ग्रा॒भेणेत्यु॑त् - ग्रा॒भेण॑ । \newline
29. उद॑ग्रभी दग्रभी॒ दुदुद॑ग्रभीत् । \newline
30. अ॒ग्र॒भी॒दित्य॑ग्रभीत् । \newline
31. अथा॑ स॒पत्ना᳚न् थ्स॒पत्ना॒(ग्म्॒) अथाथा॑ स॒पत्ना॒(ग्म्॒) इन्द्र॒ इन्द्रः॑ स॒पत्ना॒(ग्म्॒) अथाथा॑ स॒पत्ना॒(ग्म्॒) इन्द्रः॑ । \newline
32. स॒पत्ना॒(ग्म्॒) इन्द्र॒ इन्द्रः॑ स॒पत्ना᳚न् थ्स॒पत्ना॒(ग्म्॒) इन्द्रो॑ मे म॒ इन्द्रः॑ स॒पत्ना᳚न् थ्स॒पत्ना॒(ग्म्॒) इन्द्रो॑ मे । \newline
33. इन्द्रो॑ मे म॒ इन्द्र॒ इन्द्रो॑ मे निग्रा॒भेण॑ निग्रा॒भेण॑ म॒ इन्द्र॒ इन्द्रो॑ मे निग्रा॒भेण॑ । \newline
34. मे॒ नि॒ग्रा॒भेण॑ निग्रा॒भेण॑ मे मे निग्रा॒भेणाध॑रा॒(ग्म्॒) अध॑रान् निग्रा॒भेण॑ मे मे निग्रा॒भेणाध॑रान् । \newline
35. नि॒ग्रा॒भेणाध॑रा॒(ग्म्॒) अध॑रान् निग्रा॒भेण॑ निग्रा॒भेणाध॑राꣳ अक रक॒ रध॑रान् निग्रा॒भेण॑ निग्रा॒भेणाध॑राꣳ अकः । \newline
36. नि॒ग्रा॒भेणेति॑ नि - ग्रा॒भेण॑ । \newline
37. अध॑राꣳ अक रक॒ रध॑रा॒(ग्म्॒) अध॑राꣳ अकः । \newline
38. अ॒क॒रित्य॑कः । \newline
39. उ॒द्ग्रा॒भम् च॑ चोद्ग्रा॒भ मु॑द्ग्रा॒भम् च॑ निग्रा॒भम् नि॑ग्रा॒भम् चो᳚द्ग्रा॒भ मु॑द्ग्रा॒भम् च॑ निग्रा॒भम् । \newline
40. उ॒द्ग्रा॒भमित्यु॑त् - ग्रा॒भम् । \newline
41. च॒ नि॒ग्रा॒भम् नि॑ग्रा॒भम् च॑ च निग्रा॒भम् च॑ च निग्रा॒भम् च॑ च निग्रा॒भम् च॑ । \newline
42. नि॒ग्रा॒भम् च॑ च निग्रा॒भ न्नि॑ग्रा॒भम् च॒ ब्रह्म॒ ब्रह्म॑ च निग्रा॒भ न्नि॑ग्रा॒भम् च॒ ब्रह्म॑ । \newline
43. नि॒ग्रा॒भमिति॑ नि - ग्रा॒भम् । \newline
44. च॒ ब्रह्म॒ ब्रह्म॑ च च॒ ब्रह्म॑ दे॒वा दे॒वा ब्रह्म॑ च च॒ ब्रह्म॑ दे॒वाः । \newline
45. ब्रह्म॑ दे॒वा दे॒वा ब्रह्म॒ ब्रह्म॑ दे॒वा अ॑वीवृधन् नवीवृधन् दे॒वा ब्रह्म॒ ब्रह्म॑ दे॒वा अ॑वीवृधन्न् । \newline
46. दे॒वा अ॑वीवृधन् नवीवृधन् दे॒वा दे॒वा अ॑वीवृधन्न् । \newline
47. अ॒वी॒वृ॒ध॒न्नित्य॑वीवृधन्न् । \newline
48. अथा॑ स॒पत्ना᳚न् थ्स॒पत्ना॒ नथाथा॑ स॒पत्ना॑ निन्द्रा॒ग्नी इ॑न्द्रा॒ग्नी स॒पत्ना॒ नथाथा॑ स॒पत्ना॑ निन्द्रा॒ग्नी । \newline
49. स॒पत्ना॑ निन्द्रा॒ग्नी इ॑न्द्रा॒ग्नी स॒पत्ना᳚न् थ्स॒पत्ना॑ निन्द्रा॒ग्नी मे॑ म इन्द्रा॒ग्नी स॒पत्ना᳚न् थ्स॒पत्ना॑ निन्द्रा॒ग्नी मे᳚ । \newline
50. इ॒न्द्रा॒ग्नी मे॑ म इन्द्रा॒ग्नी इ॑न्द्रा॒ग्नी मे॑ विषू॒चीनान्॑. विषू॒चीना᳚न् म इन्द्रा॒ग्नी इ॑न्द्रा॒ग्नी मे॑ विषू॒चीनान्॑ । \newline
51. इ॒न्द्रा॒ग्नी इती᳚न्द्र - अ॒ग्नी । \newline
52. मे॒ वि॒षू॒चीनान्॑. विषू॒चीना᳚न् मे मे विषू॒चीना॒न्॒. वि वि वि॑षू॒चीना᳚न् मे मे विषू॒चीना॒न्॒. वि । \newline
53. वि॒षू॒चीना॒न्॒. वि वि वि॑षू॒चीनान्॑. विषू॒चीना॒न् व्य॑स्यता मस्यतां॒ ॅवि वि॑षू॒चीनान्॑. विषू॒चीना॒न् व्य॑स्यताम् । \newline
54. व्य॑स्यता मस्यतां॒ ॅवि व्य॑स्यताम् । \newline
55. अ॒स्य॒ता॒मित्य॑स्यताम् । \newline
56. एमा इ॒मा एमा अ॑ग्मन् नग्मन् नि॒मा एमा अ॑ग्मन्न् । \newline
57. इ॒मा अ॑ग्मन् नग्मन् नि॒मा इ॒मा अ॑ग्मन् ना॒शिष॑ आ॒शिषो᳚ ऽग्मन् नि॒मा इ॒मा अ॑ग्मन् ना॒शिषः॑ । \newline
58. अ॒ग्म॒न् ना॒शिष॑ आ॒शिषो᳚ ऽग्मन् नग्मन् ना॒शिषो॒ दोह॑कामा॒ दोह॑कामा आ॒शिषो᳚ ऽग्मन् नग्मन् ना॒शिषो॒ दोह॑कामाः । \newline
59. आ॒शिषो॒ दोह॑कामा॒ दोह॑कामा आ॒शिष॑ आ॒शिषो॒ दोह॑कामा॒ इन्द्र॑वन्त॒ इन्द्र॑वन्तो॒ दोह॑कामा आ॒शिष॑ आ॒शिषो॒ दोह॑कामा॒ इन्द्र॑वन्तः । \newline
60. आ॒शिष॒ इत्या᳚ - शिषः॑ । \newline
61. दोह॑कामा॒ इन्द्र॑वन्त॒ इन्द्र॑वन्तो॒ दोह॑कामा॒ दोह॑कामा॒ इन्द्र॑वन्तो वनामहे वनामह॒ इन्द्र॑वन्तो॒ दोह॑कामा॒ दोह॑कामा॒ इन्द्र॑वन्तो वनामहे । \newline
62. दोह॑कामा॒ इति॒ दोह॑ - का॒माः॒ । \newline
63. इन्द्र॑वन्तो वनामहे वनामह॒ इन्द्र॑वन्त॒ इन्द्र॑वन्तो वनामहे धुक्षी॒महि॑ धुक्षी॒महि॑ वनामह॒ इन्द्र॑वन्त॒ इन्द्र॑वन्तो वनामहे धुक्षी॒महि॑ । \newline
64. इन्द्र॑वन्त॒ इतीन्द्र॑ - व॒न्तः॒ । \newline
\pagebreak
\markright{ TS 1.6.4.3  \hfill https://www.vedavms.in \hfill}

\section{ TS 1.6.4.3 }

\textbf{TS 1.6.4.3 } \newline
\textbf{Samhita Paata} \newline

वनामहे धुक्षी॒महि॑ प्र॒जामिषं᳚ ॥ रोहि॑तेन त्वा॒ऽग्निर् दे॒वतां᳚ गमयतु॒ हरि॑भ्यां॒ त्वेन्द्रो॑ दे॒वतां᳚ गमय॒त्वेत॑शेन त्वा॒ सूर्यो॑ दे॒वतां᳚ गमयतु॒ वि ते॑ मुञ्चामि रश॒ना वि र॒श्मीन् वि योक्त्रा॒ यानि॑ परि॒चर्त॑नानि ध॒त्ताद॒स्मासु॒ द्रवि॑णं॒ ॅयच्च॑ भ॒द्रं प्र णो᳚ ब्रूताद्-भाग॒धान् दे॒वता॑सु ॥ विष्णोः᳚ शं॒ॅयोर॒हं दे॑वय॒ज्यया॑ य॒ज्ञेन॑ प्रति॒ष्ठां ग॑मेयꣳ॒॒ सोम॑स्या॒हं दे॑वय॒ज्यया॑ - [ ] \newline

\textbf{Pada Paata} \newline

व॒ना॒म॒हे॒ । धु॒क्षी॒महि॑ । प्र॒जामिति॑ प्र-जाम् । इष᳚म् ॥ रोहि॑तेन । त्वा॒ । अ॒ग्निः । दे॒वता᳚म् । ग॒म॒य॒तु॒ । हरि॑भ्या॒मिति॒ हरि॑ - भ्या॒म् । त्वा॒ । इन्द्रः॑ । दे॒वता᳚म् । ग॒म॒य॒तु॒ । एत॑शेन । त्वा॒ । सूर्यः॑ । दे॒वता᳚म् । ग॒म॒य॒तु॒ । वीति॑ । ते॒ । मु॒ञ्चा॒मि॒ । र॒श॒नाः । वीति॑ । र॒श्मीन् । वीति॑ । योक्त्रा᳚ । यानि॑ । प॒रि॒चर्त॑ना॒नीति॑ परि - चर्त॑नानि । ध॒त्तात् । अ॒स्मासु॑ । द्रवि॑णम् । यत् । च॒ । भ॒द्रम् । प्रेति॑ । नः॒ । ब्रू॒ता॒त् । भा॒ग॒धानिति॑ भाग - धान् । दे॒वता॑सु ॥ विष्णोः᳚ । श॒म्ॅयोरिति॑ शं - योः । अ॒हम् । दे॒व॒य॒ज्ययेति॑ देव-य॒ज्यया᳚ । य॒ज्ञेन॑ । प्र॒ति॒ष्ठामिति॑ प्रति - स्थाम् । ग॒मे॒य॒म् । सोम॑स्य । अ॒हम् । दे॒व॒य॒ज्ययेति॑ देव - य॒ज्यया᳚ ।  \newline


\textbf{Krama Paata} \newline

व॒ना॒म॒हे॒ धु॒क्षी॒महि॑ । धु॒क्षी॒महि॑ प्र॒जाम् । प्र॒जामिष᳚म् । प्र॒जामिति॑ प्र - जाम् । इष॒मितीष᳚म् ॥ रोहि॑तेन त्वा । त्वा॒ऽग्निः । अ॒ग्निर् दे॒वता᳚म् । दे॒वता᳚म् गमयतु । ग॒म॒य॒तु॒ हरि॑भ्याम् । हरि॑भ्याम् त्वा । हरि॑भ्या॒मिति॒ हरि॑ - भ्या॒म् । त्वेन्द्रः॑ । इन्द्रो॑ दे॒वता᳚म् । दे॒वता᳚म् गमयतु । ग॒म॒य॒त्वेत॑शेन । एत॑शेन त्वा । त्वा॒ सूर्यः॑ । सूर्यो॑ दे॒वता᳚म् । दे॒वता᳚म् गमयतु । ग॒म॒य॒तु॒ वि । वि ते᳚ । ते॒ मु॒ञ्चा॒मि॒ । मु॒ञ्चा॒मि॒ र॒श॒नाः । र॒श॒ना वि । वि र॒श्मीन् । र॒श्मीन्. वि । वि योक्त्रा᳚ । योक्त्रा॒ यानि॑ । यानि॑ परि॒चर्त॑नानि । प॒रि॒चर्त॑नानि ध॒त्तात् । प॒रि॒चर्त॑ना॒नीति॑ परि - चर्त॑नानि । ध॒त्ताद॒स्मासु॑ । अ॒स्मासु॒ द्रवि॑णम् । द्रवि॑णं॒ ॅयत् । यच्च॑ । च॒ भ॒द्रम् । भ॒द्रम् प्र । प्र णः॑ । नो॒ ब्रू॒ता॒त्॒ । ब्रू॒ता॒द् भा॒ग॒धान् । भा॒ग॒धान् दे॒वता॑सु । भा॒ग॒धानिति॑ भाग - धान् । दे॒वता॒स्विति॑ दे॒वता॑सु ॥ विष्णोः᳚ श॒म्ॅयोः । श॒म्ॅयोर॒हम् । श॒म्ॅयोरिति॑ शम् - योः । अ॒हम् दे॑वय॒ज्यया᳚ । दे॒व॒य॒ज्यया॑ य॒ज्ञेन॑ । दे॒व॒य॒ज्ययेति॑ देव - य॒ज्यया᳚ । य॒ज्ञेन॑ प्रति॒ष्ठाम् । प्र॒ति॒ष्ठाम् ग॑मेयम् । प्र॒ति॒ष्ठामिति॑ प्रति - स्थाम् । ग॒मे॒यꣳ॒॒ सोम॑स्य । सोम॑स्या॒हम् । अ॒हम् दे॑वय॒ज्यया᳚ । दे॒व॒य॒ज्यया॑ सु॒रेताः᳚ । दे॒व॒य॒ज्ययेति॑ देव - य॒ज्यया᳚ \newline

\textbf{Jatai Paata} \newline

1. व॒ना॒म॒हे॒ धु॒क्षी॒महि॑ धुक्षी॒महि॑ वनामहे वनामहे धुक्षी॒महि॑ । \newline
2. धु॒क्षी॒महि॑ प्र॒जाम् प्र॒जाम् धु॑क्षी॒महि॑ धुक्षी॒महि॑ प्र॒जाम् । \newline
3. प्र॒जा मिष॒ मिष॑म् प्र॒जाम् प्र॒जा मिष᳚म् । \newline
4. प्र॒जामिति॑ प्र - जाम् । \newline
5. इष॒मितीष᳚म् । \newline
6. रोहि॑तेन त्वा त्वा॒ रोहि॑तेन॒ रोहि॑तेन त्वा । \newline
7. त्वा॒ ऽग्नि र॒ग्नि स्त्वा᳚ त्वा॒ ऽग्निः । \newline
8. अ॒ग्निर् दे॒वता᳚म् दे॒वता॑ म॒ग्नि र॒ग्निर् दे॒वता᳚म् । \newline
9. दे॒वता᳚म् गमयतु गमयतु दे॒वता᳚म् दे॒वता᳚म् गमयतु । \newline
10. ग॒म॒य॒तु॒ हरि॑भ्या॒(ग्म्॒) हरि॑भ्याम् गमयतु गमयतु॒ हरि॑भ्याम् । \newline
11. हरि॑भ्याम् त्वा त्वा॒ हरि॑भ्या॒(ग्म्॒) हरि॑भ्याम् त्वा । \newline
12. हरि॑भ्या॒मिति॒ हरि॑ - भ्या॒म् । \newline
13. त्वेन्द्र॒ इन्द्र॑ स्त्वा॒ त्वेन्द्रः॑ । \newline
14. इन्द्रो॑ दे॒वता᳚म् दे॒वता॒ मिन्द्र॒ इन्द्रो॑ दे॒वता᳚म् । \newline
15. दे॒वता᳚म् गमयतु गमयतु दे॒वता᳚म् दे॒वता᳚म् गमयतु । \newline
16. ग॒म॒य॒ त्वेत॑शे॒ नैत॑शेन गमयतु गमय॒ त्वेत॑शेन । \newline
17. एत॑शेन त्वा॒ त्वैत॑शे॒ नैत॑शेन त्वा । \newline
18. त्वा॒ सूर्यः॒ सूर्य॑ स्त्वा त्वा॒ सूर्यः॑ । \newline
19. सूर्यो॑ दे॒वता᳚म् दे॒वता॒(ग्म्॒) सूर्यः॒ सूर्यो॑ दे॒वता᳚म् । \newline
20. दे॒वता᳚म् गमयतु गमयतु दे॒वता᳚म् दे॒वता᳚म् गमयतु । \newline
21. ग॒म॒य॒तु॒ वि वि ग॑मयतु गमयतु॒ वि । \newline
22. वि ते॑ ते॒ वि वि ते᳚ । \newline
23. ते॒ मु॒ञ्चा॒मि॒ मु॒ञ्चा॒मि॒ ते॒ ते॒ मु॒ञ्चा॒मि॒ । \newline
24. मु॒ञ्चा॒मि॒ र॒श॒ना र॑श॒ना मु॑ञ्चामि मुञ्चामि रश॒नाः । \newline
25. र॒श॒ना वि वि र॑श॒ना र॑श॒ना वि । \newline
26. वि र॒श्मीन् र॒श्मीन्. वि वि र॒श्मीन् । \newline
27. र॒श्मीन्. वि वि र॒श्मीन् र॒श्मीन्. वि । \newline
28. वि योक्त्रा॒ योक्त्रा॒ वि वि योक्त्रा᳚ । \newline
29. योक्त्रा॒ यानि॒ यानि॒ योक्त्रा॒ योक्त्रा॒ यानि॑ । \newline
30. यानि॑ परि॒चर्त॑नानि परि॒चर्त॑नानि॒ यानि॒ यानि॑ परि॒चर्त॑नानि । \newline
31. प॒रि॒चर्त॑नानि ध॒त्ताद् ध॒त्तात् प॑रि॒चर्त॑नानि परि॒चर्त॑नानि ध॒त्तात् । \newline
32. प॒रि॒चर्त॑ना॒नीति॑ परि - चर्त॑नानि । \newline
33. ध॒त्ता द॒स्मा स्व॒स्मासु॑ ध॒त्ताद् ध॒त्ता द॒स्मासु॑ । \newline
34. अ॒स्मासु॒ द्रवि॑ण॒म् द्रवि॑ण म॒स्मा स्व॒स्मासु॒ द्रवि॑णम् । \newline
35. द्रवि॑णं॒ ॅयद् यद् द्रवि॑ण॒म् द्रवि॑णं॒ ॅयत् । \newline
36. यच् च॑ च॒ यद् यच् च॑ । \newline
37. च॒ भ॒द्रम् भ॒द्रम् च॑ च भ॒द्रम् । \newline
38. भ॒द्रम् प्र प्र भ॒द्रम् भ॒द्रम् प्र । \newline
39. प्र णो॑ नः॒ प्र प्र णः॑ । \newline
40. नो॒ ब्रू॒ता॒द् ब्रू॒ता॒न् नो॒ नो॒ ब्रू॒ता॒त् । \newline
41. ब्रू॒ता॒द् भा॒ग॒धान् भा॑ग॒धान् ब्रू॑ताद् ब्रूताद् भाग॒धान् । \newline
42. भा॒ग॒धान् दे॒वता॑सु दे॒वता॑सु भाग॒धान् भा॑ग॒धान् दे॒वता॑सु । \newline
43. भा॒ग॒धानिति॑ भाग - धान् । \newline
44. दे॒वता॒स्विति॑ दे॒वता॑सु । \newline
45. विष्णोः᳚ श॒म्ॅयोः श॒म्ॅयोर् विष्णो॒र् विष्णोः᳚ श॒म्ॅयोः । \newline
46. श॒म्ॅयो र॒ह म॒हꣳ श॒म्ॅयोः श॒म्ॅयो र॒हम् । \newline
47. श॒म्ॅयोरिति॑ शं - योः । \newline
48. अ॒हम् दे॑वय॒ज्यया॑ देवय॒ज्यया॒ ऽह म॒हम् दे॑वय॒ज्यया᳚ । \newline
49. दे॒व॒य॒ज्यया॑ य॒ज्ञेन॑ य॒ज्ञेन॑ देवय॒ज्यया॑ देवय॒ज्यया॑ य॒ज्ञेन॑ । \newline
50. दे॒व॒य॒ज्ययेति॑ देव - य॒ज्यया᳚ । \newline
51. य॒ज्ञेन॑ प्रति॒ष्ठाम् प्र॑ति॒ष्ठां ॅय॒ज्ञेन॑ य॒ज्ञेन॑ प्रति॒ष्ठाम् । \newline
52. प्र॒ति॒ष्ठाम् ग॑मेयम् गमेयम् प्रति॒ष्ठाम् प्र॑ति॒ष्ठाम् ग॑मेयम् । \newline
53. प्र॒ति॒ष्ठामिति॑ प्रति - स्थाम् । \newline
54. ग॒मे॒य॒(ग्म्॒) सोम॑स्य॒ सोम॑स्य गमेयम् गमेय॒(ग्म्॒) सोम॑स्य । \newline
55. सोम॑स्या॒ह म॒हꣳ सोम॑स्य॒ सोम॑स्या॒हम् । \newline
56. अ॒हम् दे॑वय॒ज्यया॑ देवय॒ज्यया॒ ऽह म॒हम् दे॑वय॒ज्यया᳚ । \newline
57. दे॒व॒य॒ज्यया॑ सु॒रेताः᳚ सु॒रेता॑ देवय॒ज्यया॑ देवय॒ज्यया॑ सु॒रेताः᳚ । \newline
58. दे॒व॒य॒ज्ययेति॑ देव - य॒ज्यया᳚ । \newline

\textbf{Ghana Paata } \newline

1. व॒ना॒म॒हे॒ धु॒क्षी॒महि॑ धुक्षी॒महि॑ वनामहे वनामहे धुक्षी॒महि॑ प्र॒जाम् प्र॒जाम् धु॑क्षी॒महि॑ वनामहे वनामहे धुक्षी॒महि॑ प्र॒जाम् । \newline
2. धु॒क्षी॒महि॑ प्र॒जाम् प्र॒जाम् धु॑क्षी॒महि॑ धुक्षी॒महि॑ प्र॒जा मिष॒ मिष॑म् प्र॒जाम् धु॑क्षी॒महि॑ धुक्षी॒महि॑ प्र॒जा मिष᳚म् । \newline
3. प्र॒जा मिष॒ मिष॑म् प्र॒जाम् प्र॒जा मिष᳚म् । \newline
4. प्र॒जामिति॑ प्र - जाम् । \newline
5. इष॒मितीष᳚म् । \newline
6. रोहि॑तेन त्वा त्वा॒ रोहि॑तेन॒ रोहि॑तेन त्वा॒ ऽग्निर॒ग्निस्त्वा॒ रोहि॑तेन॒ रोहि॑तेन त्वा॒ ऽग्निः । \newline
7. त्वा॒ ऽग्निर॒ग्निस्त्वा᳚ त्वा॒ ऽग्निर् दे॒वता᳚म् दे॒वता॑ म॒ग्निस्त्वा᳚ त्वा॒ ऽग्निर् दे॒वता᳚म् । \newline
8. अ॒ग्निर् दे॒वता᳚म् दे॒वता॑ म॒ग्निर॒ग्निर् दे॒वता᳚म् गमयतु गमयतु दे॒वता॑ म॒ग्निर॒ग्निर् दे॒वता᳚म् गमयतु । \newline
9. दे॒वता᳚म् गमयतु गमयतु दे॒वता᳚म् दे॒वता᳚म् गमयतु॒ हरि॑भ्या॒(ग्म्॒) हरि॑भ्याम् गमयतु दे॒वता᳚म् दे॒वता᳚म् गमयतु॒ हरि॑भ्याम् । \newline
10. ग॒म॒य॒तु॒ हरि॑भ्या॒(ग्म्॒) हरि॑भ्याम् गमयतु गमयतु॒ हरि॑भ्याम् त्वा त्वा॒ हरि॑भ्याम् गमयतु गमयतु॒ हरि॑भ्याम् त्वा । \newline
11. हरि॑भ्याम् त्वा त्वा॒ हरि॑भ्या॒(ग्म्॒) हरि॑भ्या॒म् त्वेन्द्र॒ इन्द्र॑स्त्वा॒ हरि॑भ्या॒(ग्म्॒) हरि॑भ्या॒म् त्वेन्द्रः॑ । \newline
12. हरि॑भ्या॒मिति॒ हरि॑ - भ्या॒म् । \newline
13. त्वेन्द्र॒ इन्द्र॑स्त्वा॒ त्वेन्द्रो॑ दे॒वता᳚म् दे॒वता॒ मिन्द्र॑स्त्वा॒ त्वेन्द्रो॑ दे॒वता᳚म् । \newline
14. इन्द्रो॑ दे॒वता᳚म् दे॒वता॒ मिन्द्र॒ इन्द्रो॑ दे॒वता᳚म् गमयतु गमयतु दे॒वता॒ मिन्द्र॒ इन्द्रो॑ दे॒वता᳚म् गमयतु । \newline
15. दे॒वता᳚म् गमयतु गमयतु दे॒वता᳚म् दे॒वता᳚म् गमय॒ त्वेत॑शे॒नैत॑शेन गमयतु दे॒वता᳚म् दे॒वता᳚म् गमय॒त्वेत॑शेन । \newline
16. ग॒म॒य॒ त्वेत॑शे॒नैत॑शेन गमयतु गमय॒त्वेत॑शेन त्वा॒ त्वैत॑शेन गमयतु गमय॒त्वेत॑शेन त्वा । \newline
17. एत॑शेन त्वा॒ त्वैत॑शे॒नैत॑शेन त्वा॒ सूर्यः॒ सूर्य॒ स्त्वैत॑शे॒नैत॑शेन त्वा॒ सूर्यः॑ । \newline
18. त्वा॒ सूर्यः॒ सूर्य॑स्त्वा त्वा॒ सूर्यो॑ दे॒वता᳚म् दे॒वता॒(ग्म्॒) सूर्य॑स्त्वा त्वा॒ सूर्यो॑ दे॒वता᳚म् । \newline
19. सूर्यो॑ दे॒वता᳚म् दे॒वता॒(ग्म्॒) सूर्यः॒ सूर्यो॑ दे॒वता᳚म् गमयतु गमयतु दे॒वता॒(ग्म्॒) सूर्यः॒ सूर्यो॑ दे॒वता᳚म् गमयतु । \newline
20. दे॒वता᳚म् गमयतु गमयतु दे॒वता᳚म् दे॒वता᳚म् गमयतु॒ वि वि ग॑मयतु दे॒वता᳚म् दे॒वता᳚म् गमयतु॒ वि । \newline
21. ग॒म॒य॒तु॒ वि वि ग॑मयतु गमयतु॒ वि ते॑ ते॒ वि ग॑मयतु गमयतु॒ वि ते᳚ । \newline
22. वि ते॑ ते॒ वि वि ते॑ मुञ्चामि मुञ्चामि ते॒ वि वि ते॑ मुञ्चामि । \newline
23. ते॒ मु॒ञ्चा॒मि॒ मु॒ञ्चा॒मि॒ ते॒ ते॒ मु॒ञ्चा॒मि॒ र॒श॒ना र॑श॒ना मु॑ञ्चामि ते ते मुञ्चामि रश॒नाः । \newline
24. मु॒ञ्चा॒मि॒ र॒श॒ना र॑श॒ना मु॑ञ्चामि मुञ्चामि रश॒ना वि वि र॑श॒ना मु॑ञ्चामि मुञ्चामि रश॒ना वि । \newline
25. र॒श॒ना वि वि र॑श॒ना र॑श॒ना वि र॒श्मीन् र॒श्मीन्. वि र॑श॒ना र॑श॒ना वि र॒श्मीन् । \newline
26. वि र॒श्मीन् र॒श्मीन्. वि वि र॒श्मीन्. वि वि र॒श्मीन्. वि वि र॒श्मीन्. वि । \newline
27. र॒श्मीन्. वि वि र॒श्मीन् र॒श्मीन्. वि योक्त्रा॒ योक्त्रा॒ वि र॒श्मीन् र॒श्मीन्. वि योक्त्रा᳚ । \newline
28. वि योक्त्रा॒ योक्त्रा॒ वि वि योक्त्रा॒ यानि॒ यानि॒ योक्त्रा॒ वि वि योक्त्रा॒ यानि॑ । \newline
29. योक्त्रा॒ यानि॒ यानि॒ योक्त्रा॒ योक्त्रा॒ यानि॑ परि॒चर्त॑नानि परि॒चर्त॑नानि॒ यानि॒ योक्त्रा॒ योक्त्रा॒ यानि॑ परि॒चर्त॑नानि । \newline
30. यानि॑ परि॒चर्त॑नानि परि॒चर्त॑नानि॒ यानि॒ यानि॑ परि॒चर्त॑नानि ध॒त्ताद् ध॒त्तात् प॑रि॒चर्त॑नानि॒ यानि॒ यानि॑ परि॒चर्त॑नानि ध॒त्तात् । \newline
31. प॒रि॒चर्त॑नानि ध॒त्ताद् ध॒त्तात् प॑रि॒चर्त॑नानि परि॒चर्त॑नानि ध॒त्ताद॒स्मा स्व॒स्मासु॑ ध॒त्तात् प॑रि॒चर्त॑नानि परि॒चर्त॑नानि ध॒त्ताद॒स्मासु॑ । \newline
32. प॒रि॒चर्त॑ना॒नीति॑ परि - चर्त॑नानि । \newline
33. ध॒त्ताद॒स्मा स्व॒स्मासु॑ ध॒त्ताद् ध॒त्ताद॒स्मासु॒ द्रवि॑ण॒म् द्रवि॑ण म॒स्मासु॑ ध॒त्ताद् ध॒त्ताद॒स्मासु॒ द्रवि॑णम् । \newline
34. अ॒स्मासु॒ द्रवि॑ण॒म् द्रवि॑ण म॒स्मा स्व॒स्मासु॒ द्रवि॑णं॒ ॅयद् यद् द्रवि॑ण म॒स्मा स्व॒स्मासु॒ द्रवि॑णं॒ ॅयत् । \newline
35. द्रवि॑णं॒ ॅयद् यद् द्रवि॑ण॒म् द्रवि॑णं॒ ॅयच् च॑ च॒ यद् द्रवि॑ण॒म् द्रवि॑णं॒ ॅयच् च॑ । \newline
36. यच् च॑ च॒ यद् यच् च॑ भ॒द्रम् भ॒द्रम् च॒ यद् यच् च॑ भ॒द्रम् । \newline
37. च॒ भ॒द्रम् भ॒द्रम् च॑ च भ॒द्रम् प्र प्र भ॒द्रम् च॑ च भ॒द्रम् प्र । \newline
38. भ॒द्रम् प्र प्र भ॒द्रम् भ॒द्रम् प्र णो॑ नः॒ प्र भ॒द्रम् भ॒द्रम् प्र णः॑ । \newline
39. प्र णो॑ नः॒ प्र प्र णो᳚ ब्रूताद् ब्रूतान् नः॒ प्र प्र णो᳚ ब्रूतात् । \newline
40. नो॒ ब्रू॒ता॒द् ब्रू॒ता॒न् नो॒ नो॒ ब्रू॒ता॒द् भा॒ग॒धान् भा॑ग॒धान् ब्रू॑तान् नो नो ब्रूताद् भाग॒धान् । \newline
41. ब्रू॒ता॒द् भा॒ग॒धान् भा॑ग॒धान् ब्रू॑ताद् ब्रूताद् भाग॒धान् दे॒वता॑सु दे॒वता॑सु भाग॒धान् ब्रू॑ताद् ब्रूताद् भाग॒धान् दे॒वता॑सु । \newline
42. भा॒ग॒धान् दे॒वता॑सु दे॒वता॑सु भाग॒धान् भा॑ग॒धान् दे॒वता॑सु । \newline
43. भा॒ग॒धानिति॑ भाग - धान् । \newline
44. दे॒वता॒स्विति॑ दे॒वता॑सु । \newline
45. विष्णोः᳚ श॒म्ॅयोः श॒म्ॅयोर् विष्णो॒र् विष्णोः᳚ श॒म्ॅयोर॒ह म॒हꣳ श॒म्ॅयोर् विष्णो॒र् विष्णोः᳚ 
श॒म्ॅयोर॒हम् । \newline
46. श॒म्ॅयोर॒ह म॒हꣳ श॒म्ॅयोः श॒म्ॅयोर॒हम् दे॑वय॒ज्यया॑ देवय॒ज्यया॒ ऽहꣳ श॒म्ॅयोः श॒म्ॅयोर॒हम् दे॑वय॒ज्यया᳚ । \newline
47. श॒म्ॅयोरिति॑ शं - योः । \newline
48. अ॒हम् दे॑वय॒ज्यया॑ देवय॒ज्यया॒ ऽह म॒हम् दे॑वय॒ज्यया॑ य॒ज्ञेन॑ य॒ज्ञेन॑ देवय॒ज्यया॒ ऽह म॒हम् दे॑वय॒ज्यया॑ य॒ज्ञेन॑ । \newline
49. दे॒व॒य॒ज्यया॑ य॒ज्ञेन॑ य॒ज्ञेन॑ देवय॒ज्यया॑ देवय॒ज्यया॑ य॒ज्ञेन॑ प्रति॒ष्ठाम् प्र॑ति॒ष्ठां ॅय॒ज्ञेन॑ देवय॒ज्यया॑ देवय॒ज्यया॑ य॒ज्ञेन॑ प्रति॒ष्ठाम् । \newline
50. दे॒व॒य॒ज्ययेति॑ देव - य॒ज्यया᳚ । \newline
51. य॒ज्ञेन॑ प्रति॒ष्ठाम् प्र॑ति॒ष्ठां ॅय॒ज्ञेन॑ य॒ज्ञेन॑ प्रति॒ष्ठाम् ग॑मेयम् गमेयम् प्रति॒ष्ठां ॅय॒ज्ञेन॑ य॒ज्ञेन॑ प्रति॒ष्ठाम् ग॑मेयम् । \newline
52. प्र॒ति॒ष्ठाम् ग॑मेयम् गमेयम् प्रति॒ष्ठाम् प्र॑ति॒ष्ठाम् ग॑मेय॒(ग्म्॒) सोम॑स्य॒ सोम॑स्य गमेयम् प्रति॒ष्ठाम् प्र॑ति॒ष्ठाम् ग॑मेय॒(ग्म्॒) सोम॑स्य । \newline
53. प्र॒ति॒ष्ठामिति॑ प्रति - स्थाम् । \newline
54. ग॒मे॒य॒(ग्म्॒) सोम॑स्य॒ सोम॑स्य गमेयम् गमेय॒(ग्म्॒) सोम॑स्या॒ह म॒हꣳ सोम॑स्य गमेयम् गमेय॒(ग्म्॒) सोम॑स्या॒हम् । \newline
55. सोम॑स्या॒ह म॒हꣳ सोम॑स्य॒ सोम॑स्या॒हम् दे॑वय॒ज्यया॑ देवय॒ज्यया॒ ऽहꣳ सोम॑स्य॒ सोम॑स्या॒हम् दे॑वय॒ज्यया᳚ । \newline
56. अ॒हम् दे॑वय॒ज्यया॑ देवय॒ज्यया॒ ऽह म॒हम् दे॑वय॒ज्यया॑ सु॒रेताः᳚ सु॒रेता॑ देवय॒ज्यया॒ ऽह म॒हम् दे॑वय॒ज्यया॑ सु॒रेताः᳚ । \newline
57. दे॒व॒य॒ज्यया॑ सु॒रेताः᳚ सु॒रेता॑ देवय॒ज्यया॑ देवय॒ज्यया॑ सु॒रेता॒ रेतो॒ रेतः॑ सु॒रेता॑ देवय॒ज्यया॑ देवय॒ज्यया॑ सु॒रेता॒ रेतः॑ । \newline
58. दे॒व॒य॒ज्ययेति॑ देव - य॒ज्यया᳚ । \newline
\pagebreak
\markright{ TS 1.6.4.4  \hfill https://www.vedavms.in \hfill}

\section{ TS 1.6.4.4 }

\textbf{TS 1.6.4.4 } \newline
\textbf{Samhita Paata} \newline

सु॒रेता॒ रेतो॑ धिषीय॒ त्वष्टु॑र॒हं दे॑वय॒ज्यया॑ पशू॒नाꣳ रू॒पं पु॑षेयं दे॒वानां॒ पत्नी॑र॒ग्निर् गृ॒हप॑तिर् य॒ज्ञ्स्य॑ मिथु॒नं तयो॑र॒हं दे॑वय॒ज्यया॑ मिथु॒नेन॒ प्रभू॑यासं ॅवे॒दो॑ऽसि॒ वित्ति॑रसि वि॒देय॒ कर्मा॑ऽसि क॒रुण॑मसि क्रि॒यासꣳ॑ स॒निर॑सि सनि॒ताऽसि॑ स॒नेयं॑ घृ॒तव॑न्तं कुला॒यिनꣳ॑ रा॒यस्पोषꣳ॑ सह॒स्रिणं॑ ॅवे॒दो द॑दातु वा॒जिनं᳚ ॥ \newline

\textbf{Pada Paata} \newline

सु॒रेता॒ इति॑ सु - रेताः᳚ । रेतः॑ । धि॒षी॒य॒ । त्वष्टुः॑ । अ॒हम् । दे॒व॒य॒ज्ययेति॑ देव-य॒ज्यया᳚ । प॒शू॒नाम् । रू॒पम् । पु॒षे॒य॒म् । दे॒वाना᳚म् । पत्नीः᳚ । अ॒ग्निः । गृ॒हप॑ति॒रिति॑ गृ॒ह - प॒तिः॒ । य॒ज्ञ्स्य॑ । मि॒थु॒नम् । तयोः᳚ । अ॒हम् । दे॒व॒य॒ज्ययेति॑ देव - य॒ज्यया᳚ । मि॒थु॒नेन॑ । प्रेति॑ । भू॒या॒स॒म् । वे॒दः । अ॒सि॒ । वित्तिः॑ । अ॒सि॒ । वि॒देय॑ । कर्म॑ । अ॒सि॒ । क॒रुण᳚म् । अ॒सि॒ । क्रि॒यास᳚म् । स॒निः । अ॒सि॒ । स॒नि॒ता । अ॒सि॒ । स॒नेय᳚म् । घृ॒तव॑न्त॒मिति॑ घृ॒त-व॒न्त॒म् । कु॒ला॒यिन᳚म् । रा॒यः । पोष᳚म् । स॒ह॒स्रिण᳚म् । वे॒दः । द॒दा॒तु॒ । वा॒जिन᳚म् ॥  \newline


\textbf{Krama Paata} \newline

सु॒रेता॒ रेतः॑ । सु॒रेता॒ इति॑ सु - रेताः᳚ । रेतो॑ धिषीय । धी॒षी॒य॒ त्वष्टुः॑ । त्वष्टु॑र॒हम् । अ॒हम् दे॑वय॒ज्यया᳚ । दे॒व॒य॒ज्यया॑ पशू॒नाम् । दे॒व॒य॒ज्ययेति॑ देव - य॒ज्यया᳚ । प॒शू॒नाꣳ रू॒पम् । रू॒पम् पु॑षेयम् । पु॒षे॒य॒म् दे॒वाना᳚म् । दे॒वाना॒म् पत्नीः᳚ । पत्नी॑र॒ग्निः । अ॒ग्निर्,गृ॒हप॑तिः । गृ॒हप॑तिर्,य॒ज्ञ्स्य॑ । गृ॒हप॑ति॒रिति॑ गृ॒ह - प॒तिः॒ । य॒ज्ञ्स्य॑ मिथु॒नम् । मि॒थु॒नम् तयोः᳚ । तयो॑र॒हम् । अ॒हम् दे॑वय॒ज्यया᳚ । दे॒व॒य॒ज्यया॑ मिथु॒नेन॑ । दे॒व॒य॒ज्ययेति॑ देव - य॒ज्यया᳚ । मि॒थु॒नेन॒ प्र । प्र भू॑यासम् । भू॒या॒सं॒ ॅवे॒दः । वे॒दो॑ऽसि । अ॒सि॒ वित्तिः॑ । वित्ति॑रसि । अ॒सि॒ वि॒देय॑ । वि॒देय॒ कर्म॑ । कर्मा॑सि । अ॒सि॒ क॒रुण᳚म् । क॒रुण॑मसि । अ॒सि॒ क्रि॒यास᳚म् । क्रि॒यासꣳ॑ स॒निः । स॒निर॑सि । अ॒सि॒ स॒नि॒ता । स॒नि॒ताऽसि॑ । अ॒सि॒ स॒नेय᳚म् । स॒नेय॑म् घृ॒तव॑न्तम् । घृ॒तव॑न्तम् कुला॒यिन᳚म् । घृ॒तव॑न्त॒मिति॑ घृ॒त - व॒न्त॒म् । कु॒ला॒यिनꣳ॑ रा॒यः । रा॒यस्पोष᳚म् । पोषꣳ॑ सह॒स्रिण᳚म् । स॒ह॒स्रिणं॑ ॅवे॒दः । वे॒दो द॑दातु । द॒दा॒तु॒ वा॒जिन᳚म् । वा॒जिन॒मिति॑ वा॒जिन᳚म् । \newline

\textbf{Jatai Paata} \newline

1. सु॒रेता॒ रेतो॒ रेतः॑ सु॒रेताः᳚ सु॒रेता॒ रेतः॑ । \newline
2. सु॒रेता॒ इति॑ सु - रेताः᳚ । \newline
3. रेतो॑ धिषीय धिषीय॒ रेतो॒ रेतो॑ धिषीय । \newline
4. धि॒षी॒य॒ त्वष्टु॒ स्त्वष्टु॑र् धिषीय धिषीय॒ त्वष्टुः॑ । \newline
5. त्वष्टु॑र॒ह म॒हम् त्वष्टु॒ स्त्वष्टु॑ र॒हम् । \newline
6. अ॒हम् दे॑वय॒ज्यया॑ देवय॒ज्यया॒ ऽह म॒हम् दे॑वय॒ज्यया᳚ । \newline
7. दे॒व॒य॒ज्यया॑ पशू॒नाम् प॑शू॒नाम् दे॑वय॒ज्यया॑ देवय॒ज्यया॑ पशू॒नाम् । \newline
8. दे॒व॒य॒ज्ययेति॑ देव - य॒ज्यया᳚ । \newline
9. प॒शू॒नाꣳ रू॒पꣳ रू॒पम् प॑शू॒नाम् प॑शू॒नाꣳ रू॒पम् । \newline
10. रू॒पम् पु॑षेयम् पुषेयꣳ रू॒पꣳ रू॒पम् पु॑षेयम् । \newline
11. पु॒षे॒य॒म् दे॒वाना᳚म् दे॒वाना᳚म् पुषेयम् पुषेयम् दे॒वाना᳚म् । \newline
12. दे॒वाना॒म् पत्नीः॒ पत्नी᳚र् दे॒वाना᳚म् दे॒वाना॒म् पत्नीः᳚ । \newline
13. पत्नी॑ र॒ग्निर॒ग्निः पत्नीः॒ पत्नी॑र॒ग्निः । \newline
14. अ॒ग्निर् गृ॒हप॑तिर् गृ॒हप॑ति र॒ग्नि र॒ग्निर् गृ॒हप॑तिः । \newline
15. गृ॒हप॑तिर् य॒ज्ञ्स्य॑ य॒ज्ञ्स्य॑ गृ॒हप॑तिर् गृ॒हप॑तिर् य॒ज्ञ्स्य॑ । \newline
16. गृ॒हप॑ति॒रिति॑ गृ॒ह - प॒तिः॒ । \newline
17. य॒ज्ञ्स्य॑ मिथु॒नम् मि॑थु॒नं ॅय॒ज्ञ्स्य॑ य॒ज्ञ्स्य॑ मिथु॒नम् । \newline
18. मि॒थु॒नम् तयो॒स्तयो᳚र् मिथु॒नम् मि॑थु॒नम् तयोः᳚ । \newline
19. तयो॑र॒ह म॒हम् तयो॒ स्तयो॑ र॒हम् । \newline
20. अ॒हम् दे॑वय॒ज्यया॑ देवय॒ज्यया॒ ऽह म॒हम् दे॑वय॒ज्यया᳚ । \newline
21. दे॒व॒य॒ज्यया॑ मिथु॒नेन॑ मिथु॒नेन॑ देवय॒ज्यया॑ देवय॒ज्यया॑ मिथु॒नेन॑ । \newline
22. दे॒व॒य॒ज्ययेति॑ देव - य॒ज्यया᳚ । \newline
23. मि॒थु॒नेन॒ प्र प्र मि॑थु॒नेन॑ मिथु॒नेन॒ प्र । \newline
24. प्र भू॑यासम् भूयास॒म् प्र प्र भू॑यासम् । \newline
25. भू॒या॒सं॒ ॅवे॒दो वे॒दो भू॑यासम् भूयासं ॅवे॒दः । \newline
26. वे॒दो᳚ ऽस्यसि वे॒दो वे॒दो॑ ऽसि । \newline
27. अ॒सि॒ वित्ति॒र् वित्ति॑ रस्यसि॒ वित्तिः॑ । \newline
28. वित्ति॑ रस्यसि॒ वित्ति॒र् वित्ति॑ रसि । \newline
29. अ॒सि॒ वि॒देय॑ वि॒देया᳚ स्यसि वि॒देय॑ । \newline
30. वि॒देय॒ कर्म॒ कर्म॑ वि॒देय॑ वि॒देय॒ कर्म॑ । \newline
31. कर्मा᳚स्यसि॒ कर्म॒ कर्मा॑सि । \newline
32. अ॒सि॒ क॒रुण॑म् क॒रुण॑ मस्यसि क॒रुण᳚म् । \newline
33. क॒रुण॑ मस्यसि क॒रुण॑म् क॒रुण॑ मसि । \newline
34. अ॒सि॒ क्रि॒यास॑म् क्रि॒यास॑ मस्यसि क्रि॒यास᳚म् । \newline
35. क्रि॒यास(ग्म्॑) स॒निः स॒निः क्रि॒यास॑म् क्रि॒यास(ग्म्॑) स॒निः । \newline
36. स॒नि र॑स्यसि स॒निः स॒नि र॑सि । \newline
37. अ॒सि॒ स॒नि॒ता स॑नि॒ता ऽस्य॑सि सनि॒ता । \newline
38. स॒नि॒ता ऽस्य॑सि सनि॒ता स॑नि॒ता ऽसि॑ । \newline
39. अ॒सि॒ स॒नेय(ग्म्॑) स॒नेय॑ मस्यसि स॒नेय᳚म् । \newline
40. स॒नेय॑म् घृ॒तव॑न्तम् घृ॒तव॑न्तꣳ स॒नेय(ग्म्॑) स॒नेय॑म् घृ॒तव॑न्तम् । \newline
41. घृ॒तव॑न्तम् कुला॒यिन॑म् कुला॒यिन॑म् घृ॒तव॑न्तम् घृ॒तव॑न्तम् कुला॒यिन᳚म् । \newline
42. घृ॒तव॑न्त॒मिति॑ घृ॒त - व॒न्त॒म् । \newline
43. कु॒ला॒यिन(ग्म्॑) रा॒यो रा॒यः कु॑ला॒यिन॑म् कुला॒यिन(ग्म्॑) रा॒यः । \newline
44. रा॒य स्पोष॒म् पोष(ग्म्॑) रा॒यो रा॒य स्पोष᳚म् । \newline
45. पोष(ग्म्॑) सह॒स्रिण(ग्म्॑) सह॒स्रिण॒म् पोष॒म् पोष(ग्म्॑) सह॒स्रिण᳚म् । \newline
46. स॒ह॒स्रिणं॑ ॅवे॒दो वे॒दः स॑ह॒स्रिण(ग्म्॑) सह॒स्रिणं॑ ॅवे॒दः । \newline
47. वे॒दो द॑दातु ददातु वे॒दो वे॒दो द॑दातु । \newline
48. द॒दा॒तु॒ वा॒जिनं॑ ॅवा॒जिन॑म् ददातु ददातु वा॒जिन᳚म् । \newline
49. वा॒जिन॒मिति॑ वा॒जिन᳚म् । \newline

\textbf{Ghana Paata } \newline

1. सु॒रेता॒ रेतो॒ रेतः॑ सु॒रेताः᳚ सु॒रेता॒ रेतो॑ धिषीय धिषीय॒ रेतः॑ सु॒रेताः᳚ सु॒रेता॒ रेतो॑ धिषीय । \newline
2. सु॒रेता॒ इति॑ सु - रेताः᳚ । \newline
3. रेतो॑ धिषीय धिषीय॒ रेतो॒ रेतो॑ धिषीय॒ त्वष्टु॒ स्त्वष्टु॑र् धिषीय॒ रेतो॒ रेतो॑ धिषीय॒ त्वष्टुः॑ । \newline
4. धि॒षी॒य॒ त्वष्टु॒स्त्वष्टु॑र् धिषीय धिषीय॒ त्वष्टु॑र॒ह म॒हम् त्वष्टु॑र् धिषीय धिषीय॒ त्वष्टु॑र॒हम् । \newline
5. त्वष्टु॑र॒ह म॒हम् त्वष्टु॒स्त्वष्टु॑र॒हम् दे॑वय॒ज्यया॑ देवय॒ज्यया॒ ऽहम् त्वष्टु॒ स्त्वष्टु॑र॒हम् दे॑वय॒ज्यया᳚ । \newline
6. अ॒हम् दे॑वय॒ज्यया॑ देवय॒ज्यया॒ ऽह म॒हम् दे॑वय॒ज्यया॑ पशू॒नाम् प॑शू॒नाम् दे॑वय॒ज्यया॒ ऽह म॒हम् दे॑वय॒ज्यया॑ पशू॒नाम् । \newline
7. दे॒व॒य॒ज्यया॑ पशू॒नाम् प॑शू॒नाम् दे॑वय॒ज्यया॑ देवय॒ज्यया॑ पशू॒नाꣳ रू॒पꣳ रू॒पम् प॑शू॒नाम् दे॑वय॒ज्यया॑ देवय॒ज्यया॑ पशू॒नाꣳ रू॒पम् । \newline
8. दे॒व॒य॒ज्ययेति॑ देव - य॒ज्यया᳚ । \newline
9. प॒शू॒नाꣳ रू॒पꣳ रू॒पम् प॑शू॒नाम् प॑शू॒नाꣳ रू॒पम् पु॑षेयम् पुषेयꣳ रू॒पम् प॑शू॒नाम् प॑शू॒नाꣳ रू॒पम् पु॑षेयम् । \newline
10. रू॒पम् पु॑षेयम् पुषेयꣳ रू॒पꣳ रू॒पम् पु॑षेयम् दे॒वाना᳚म् दे॒वाना᳚म् पुषेयꣳ रू॒पꣳ रू॒पम् पु॑षेयम् दे॒वाना᳚म् । \newline
11. पु॒षे॒य॒म् दे॒वाना᳚म् दे॒वाना᳚म् पुषेयम् पुषेयम् दे॒वाना॒म् पत्नीः॒ पत्नी᳚र् दे॒वाना᳚म् पुषेयम् पुषेयम् दे॒वाना॒म् पत्नीः᳚ । \newline
12. दे॒वाना॒म् पत्नीः॒ पत्नी᳚र् दे॒वाना᳚म् दे॒वाना॒म् पत्नी॑ र॒ग्निर॒ग्निः पत्नी᳚र् दे॒वाना᳚म् दे॒वाना॒म् पत्नी॑र॒ग्निः । \newline
13. पत्नी॑ र॒ग्निर॒ग्निः पत्नीः॒ पत्नी॑र॒ग्निर् गृ॒हप॑तिर् गृ॒हप॑तिर॒ग्निः पत्नीः॒ पत्नी॑र॒ग्निर् गृ॒हप॑तिः । \newline
14. अ॒ग्निर् गृ॒हप॑तिर् गृ॒हप॑ति र॒ग्निर॒ग्निर् गृ॒हप॑तिर् य॒ज्ञ्स्य॑ य॒ज्ञ्स्य॑ गृ॒हप॑तिर॒ग्निर॒ग्निर् गृ॒हप॑तिर् य॒ज्ञ्स्य॑ । \newline
15. गृ॒हप॑तिर् य॒ज्ञ्स्य॑ य॒ज्ञ्स्य॑ गृ॒हप॑तिर् गृ॒हप॑तिर् य॒ज्ञ्स्य॑ मिथु॒नम् मि॑थु॒नं ॅय॒ज्ञ्स्य॑ गृ॒हप॑तिर् गृ॒हप॑तिर् य॒ज्ञ्स्य॑ मिथु॒नम् । \newline
16. गृ॒हप॑ति॒रिति॑ गृ॒ह - प॒तिः॒ । \newline
17. य॒ज्ञ्स्य॑ मिथु॒नम् मि॑थु॒नं ॅय॒ज्ञ्स्य॑ य॒ज्ञ्स्य॑ मिथु॒नम् तयो॒स्तयो᳚र् मिथु॒नं ॅय॒ज्ञ्स्य॑ य॒ज्ञ्स्य॑ मिथु॒नम् तयोः᳚ । \newline
18. मि॒थु॒नम् तयो॒स्तयो᳚र् मिथु॒नम् मि॑थु॒नम् तयो॑र॒ह म॒हम् तयो᳚र् मिथु॒नम् मि॑थु॒नम् तयो॑र॒हम् । \newline
19. तयो॑र॒ह म॒हम् तयो॒स्तयो॑र॒हम् दे॑वय॒ज्यया॑ देवय॒ज्यया॒ ऽहम् तयो॒स्तयो॑र॒हम् दे॑वय॒ज्यया᳚ । \newline
20. अ॒हम् दे॑वय॒ज्यया॑ देवय॒ज्यया॒ ऽह म॒हम् दे॑वय॒ज्यया॑ मिथु॒नेन॑ मिथु॒नेन॑ देवय॒ज्यया॒ ऽह म॒हम् दे॑वय॒ज्यया॑ मिथु॒नेन॑ । \newline
21. दे॒व॒य॒ज्यया॑ मिथु॒नेन॑ मिथु॒नेन॑ देवय॒ज्यया॑ देवय॒ज्यया॑ मिथु॒नेन॒ प्र प्र मि॑थु॒नेन॑ देवय॒ज्यया॑ देवय॒ज्यया॑ मिथु॒नेन॒ प्र । \newline
22. दे॒व॒य॒ज्ययेति॑ देव - य॒ज्यया᳚ । \newline
23. मि॒थु॒नेन॒ प्र प्र मि॑थु॒नेन॑ मिथु॒नेन॒ प्र भू॑यासम् भूयास॒म् प्र मि॑थु॒नेन॑ मिथु॒नेन॒ प्र भू॑यासम् । \newline
24. प्र भू॑यासम् भूयास॒म् प्र प्र भू॑यासं ॅवे॒दो वे॒दो भू॑यास॒म् प्र प्र भू॑यासं ॅवे॒दः । \newline
25. भू॒या॒सं॒ ॅवे॒दो वे॒दो भू॑यासम् भूयासं ॅवे॒दो᳚ ऽस्यसि वे॒दो भू॑यासम् भूयासं ॅवे॒दो॑ ऽसि । \newline
26. वे॒दो᳚ ऽस्यसि वे॒दो वे॒दो॑ ऽसि॒ वित्ति॒र् वित्ति॑रसि वे॒दो वे॒दो॑ ऽसि॒ वित्तिः॑ । \newline
27. अ॒सि॒ वित्ति॒र् वित्ति॑ रस्यसि॒ वित्ति॑ रस्यसि॒ वित्ति॑ रस्यसि॒ वित्ति॑रसि । \newline
28. वित्ति॑ रस्यसि॒ वित्ति॒र् वित्ति॑रसि वि॒देय॑ वि॒देया॑सि॒ वित्ति॒र् वित्ति॑रसि वि॒देय॑ । \newline
29. अ॒सि॒ वि॒देय॑ वि॒देया᳚स्यसि वि॒देय॒ कर्म॒ कर्म॑ वि॒देया᳚स्यसि वि॒देय॒ कर्म॑ । \newline
30. वि॒देय॒ कर्म॒ कर्म॑ वि॒देय॑ वि॒देय॒ कर्मा᳚स्यसि॒ कर्म॑ वि॒देय॑ वि॒देय॒ कर्मा॑सि । \newline
31. कर्मा᳚स्यसि॒ कर्म॒ कर्मा॑सि क॒रुण॑म् क॒रुण॑ मसि॒ कर्म॒ कर्मा॑सि क॒रुण᳚म् । \newline
32. अ॒सि॒ क॒रुण॑म् क॒रुण॑ मस्यसि क॒रुण॑ मस्यसि क॒रुण॑ मस्यसि क॒रुण॑ मसि । \newline
33. क॒रुण॑ मस्यसि क॒रुण॑म् क॒रुण॑ मसि क्रि॒यास॑म् क्रि॒यास॑ मसि क॒रुण॑म् क॒रुण॑ मसि क्रि॒यास᳚म् । \newline
34. अ॒सि॒ क्रि॒यास॑म् क्रि॒यास॑ मस्यसि क्रि॒यास(ग्म्॑) स॒निः स॒निः क्रि॒यास॑ मस्यसि क्रि॒यास(ग्म्॑) स॒निः । \newline
35. क्रि॒यास(ग्म्॑) स॒निः स॒निः क्रि॒यास॑म् क्रि॒यास(ग्म्॑) स॒निर॑स्यसि स॒निः क्रि॒यास॑म् क्रि॒यास(ग्म्॑) स॒निर॑सि । \newline
36. स॒निर॑स्यसि स॒निः स॒निर॑सि सनि॒ता स॑नि॒ता ऽसि॑ स॒निः स॒निर॑सि सनि॒ता । \newline
37. अ॒सि॒ स॒नि॒ता स॑नि॒ता ऽस्य॑सि सनि॒ता ऽस्य॑सि सनि॒ता ऽस्य॑सि सनि॒ता ऽसि॑ । \newline
38. स॒नि॒ता ऽस्य॑सि सनि॒ता स॑नि॒ता ऽसि॑ स॒नेय(ग्म्॑) स॒नेय॑ मसि सनि॒ता स॑नि॒ता ऽसि॑ स॒नेय᳚म् । \newline
39. अ॒सि॒ स॒नेय(ग्म्॑) स॒नेय॑ मस्यसि स॒नेय॑म् घृ॒तव॑न्तम् घृ॒तव॑न्तꣳ स॒नेय॑ मस्यसि स॒नेय॑म् घृ॒तव॑न्तम् । \newline
40. स॒नेय॑म् घृ॒तव॑न्तम् घृ॒तव॑न्तꣳ स॒नेय(ग्म्॑) स॒नेय॑म् घृ॒तव॑न्तम् कुला॒यिन॑म् कुला॒यिन॑म् घृ॒तव॑न्तꣳ स॒नेय(ग्म्॑) स॒नेय॑म् घृ॒तव॑न्तम् कुला॒यिन᳚म् । \newline
41. घृ॒तव॑न्तम् कुला॒यिन॑म् कुला॒यिन॑म् घृ॒तव॑न्तम् घृ॒तव॑न्तम् कुला॒यिन(ग्म्॑) रा॒यो रा॒यः कु॑ला॒यिन॑म् घृ॒तव॑न्तम् घृ॒तव॑न्तम् कुला॒यिन(ग्म्॑) रा॒यः । \newline
42. घृ॒तव॑न्त॒मिति॑ घृ॒त - व॒न्त॒म् । \newline
43. कु॒ला॒यिन(ग्म्॑) रा॒यो रा॒यः कु॑ला॒यिन॑म् कुला॒यिन(ग्म्॑) रा॒य स्पोष॒म् पोष(ग्म्॑) रा॒यः कु॑ला॒यिन॑म् कुला॒यिन(ग्म्॑) रा॒य स्पोष᳚म् । \newline
44. रा॒य स्पोष॒म् पोष(ग्म्॑) रा॒यो रा॒य स्पोष(ग्म्॑) सह॒स्रिण(ग्म्॑) सह॒स्रिण॒म् पोष(ग्म्॑) रा॒यो रा॒य स्पोष(ग्म्॑) सह॒स्रिण᳚म् । \newline
45. पोष(ग्म्॑) सह॒स्रिण(ग्म्॑) सह॒स्रिण॒म् पोष॒म् पोष(ग्म्॑) सह॒स्रिणं॑ ॅवे॒दो वे॒दः स॑ह॒स्रिण॒म् पोष॒म् पोष(ग्म्॑) सह॒स्रिणं॑ ॅवे॒दः । \newline
46. स॒ह॒स्रिणं॑ ॅवे॒दो वे॒दः स॑ह॒स्रिण(ग्म्॑) सह॒स्रिणं॑ ॅवे॒दो द॑दातु ददातु वे॒दः स॑ह॒स्रिण(ग्म्॑) सह॒स्रिणं॑ ॅवे॒दो द॑दातु । \newline
47. वे॒दो द॑दातु ददातु वे॒दो वे॒दो द॑दातु वा॒जिनं॑ ॅवा॒जिन॑म् ददातु वे॒दो वे॒दो द॑दातु वा॒जिन᳚म् । \newline
48. द॒दा॒तु॒ वा॒जिनं॑ ॅवा॒जिन॑म् ददातु ददातु वा॒जिन᳚म् । \newline
49. वा॒जिन॒मिति॑ वा॒जिन᳚म् । \newline
\pagebreak
\markright{ TS 1.6.5.1  \hfill https://www.vedavms.in \hfill}

\section{ TS 1.6.5.1 }

\textbf{TS 1.6.5.1 } \newline
\textbf{Samhita Paata} \newline

आ प्या॑यतां ध्रु॒वा घृ॒तेन॑ य॒ज्ञ्ंॅय॑ज्ञ्ं॒ प्रति॑ देव॒यद्भ्यः॑ । सू॒र्याया॒ ऊधोऽदि॑त्या उ॒पस्थ॑ उ॒रुधा॑रा पृथि॒वी य॒ज्ञे अ॒स्मिन्न् ॥ प्र॒जाप॑तेर् वि॒भान्नाम॑ लो॒कस्तस्मिꣳ॑स्त्वा दधामि स॒ह यज॑मानेन॒ सद॑सि॒ सन्मे॑ भूयाः॒ सर्व॑मसि॒ सर्वं॑ मे भूयाः पू॒र्णम॑सि पू॒र्णं मे॑ भूया॒ अक्षि॑तमसि॒ मा मे᳚ क्षेष्ठाः॒ प्राच्यां᳚ दि॒शि दे॒वा ऋ॒त्विजो॑ मार्जयन्तां॒ दक्षि॑णायां- [ ] \newline

\textbf{Pada Paata} \newline

एति॑ । प्या॒य॒ता॒म् । ध्रु॒वा । घृ॒तेन॑ । य॒ज्ञ्ं ॅय॑ज्ञ्॒मिति॑ य॒ज्ञ्म् - य॒ज्ञ्॒म् । प्रतीति॑ । दे॒व॒यद्भ्य॒ इति॑ देव॒यत् - भ्यः॒ ॥ सू॒र्यायाः᳚ । ऊधः॑ । अदि॑त्याः । उ॒पस्थ॒ इत्यु॒प - स्थे॒ । उ॒रुधा॒रेत्यु॒रु - धा॒रा॒ । पृ॒थि॒वी । य॒ज्ञे । अ॒स्मिन्न् ॥ प्र॒जाप॑ते॒रिति॑ प्र॒जा-प॒तेः॒ । वि॒भानिति॑ वि-भान् । नाम॑ । लो॒कः । तस्मिन्न्॑ । त्वा॒ । द॒धा॒मि॒ । स॒ह । यज॑मानेन । सत् । अ॒सि॒ । सत् । मे॒ । भू॒याः॒ । सर्व᳚म् । अ॒सि॒ । सर्व᳚म् । मे॒ । भू॒याः॒ । पू॒र्णम् । अ॒सि॒ । पू॒र्णम् । मे॒ । भू॒याः॒ । अक्षि॑तम् । अ॒सि॒ । मा । मे॒ । क्ष॒ष्ठाः॒ । प्राच्या᳚म् । दि॒शि । दे॒वाः । ऋ॒त्विजः॑ । मा॒र्ज॒य॒न्ता॒म् । दक्षि॑णायाम् ।  \newline


\textbf{Krama Paata} \newline

आ प्या॑यताम् । प्या॒य॒ता॒म् ध्रु॒वा । ध्रु॒वा घृ॒तेन॑ । घृ॒तेन॑ य॒ज्ञ्म्ॅय॑ज्ञ्म् । य॒ज्ञ्म्ॅय॑ज्ञ्ं॒ प्रति॑ । य॒ज्ञ्म्ॅय॑ज्ञ्॒मिति॑ य॒ज्ञ्ं - य॒ज्ञ्॒म् । प्रति॑ देव॒यद्भ्यः॑ । दे॒व॒यद्भ्य॒ इति॑ देव॒यत् - भ्यः॒ ॥ सू॒र्याया॒ ऊधः॑ । ऊधोऽदि॑त्याः । अदि॑त्या उ॒पस्थे᳚ । उ॒पस्थ॑ उ॒रुधा॑रा । उ॒पस्थ॒ इत्यु॒प - स्थे॒ । उ॒रुधा॑रा पृथि॒वी । उ॒रुधा॒रेत्यु॒रु - धा॒रा॒ । पृ॒थि॒वी य॒ज्ञे । य॒ज्ञे अ॒स्मिन्न् । अ॒स्मिन्नित्य॒स्मिन्न् ॥ प्र॒जाप॑तेर्,वि॒भान् । प्र॒जाप॑ते॒रिति॑ प्र॒जा - प॒तेः॒ । वि॒भान्नाम॑ । वि॒भानिति॑ वि - भान् । नाम॑ लो॒कः । लो॒कस्तस्मिन्न्॑ । तस्मिꣳ॑ स्त्वा । त्वा॒ द॒धा॒मि॒ । द॒धा॒मि॒ स॒ह । स॒ह यज॑मानेन । यज॑मानेन॒ सत् । सद॑सि । अ॒सि॒ सत् । सन्मे᳚ । मे॒ भू॒याः॒ । भू॒याः॒ सर्व᳚म् । सर्व॑मसि । अ॒सि॒ सर्व᳚म् । सर्व॑म् मे । मे॒ भू॒याः॒ । भू॒याः॒ पू॒र्णम् । पू॒र्णम॑सि । अ॒सि॒ पू॒र्णम् । पू॒र्णम् मे᳚ । मे॒ भू॒याः॒ । भू॒या॒ अक्षि॑तम् । अक्षि॑तमसि । अ॒सि॒ मा । मा मे᳚ । मे॒ क्षे॒ष्ठाः॒ । क्षे॒ष्ठाः॒ प्राच्या᳚म् । प्राच्या᳚म् दि॒शि । दि॒शि दे॒वाः । दे॒वा ऋ॒त्विजः॑ । ऋ॒त्विजो॑ मार्जयन्ताम् । मा॒र्ज॒य॒न्ता॒म् दक्षि॑णायाम् । दक्षि॑णायाम् दि॒शि \newline

\textbf{Jatai Paata} \newline

1. आ प्या॑यताम् प्यायता॒ मा प्या॑यताम् । \newline
2. प्या॒य॒ता॒म् ध्रु॒वा ध्रु॒वा प्या॑यताम् प्यायताम् ध्रु॒वा । \newline
3. ध्रु॒वा घृ॒तेन॑ घृ॒तेन॑ ध्रु॒वा ध्रु॒वा घृ॒तेन॑ । \newline
4. घृ॒तेन॑ य॒ज्ञ्ंॅय॑ज्ञ्ं ॅय॒ज्ञ्ंॅय॑ज्ञ्म् घृ॒तेन॑ घृ॒तेन॑ य॒ज्ञ्ंॅय॑ज्ञ्म् । \newline
5. य॒ज्ञ्ंॅय॑ज्ञ्॒म् प्रति॒ प्रति॑ य॒ज्ञ्ंॅय॑ज्ञ्ं ॅय॒ज्ञ्ंॅय॑ज्ञ्॒म् प्रति॑ । \newline
6. य॒ज्ञ्ंॅय॑ज्ञ्॒मिति॑ य॒ज्ञ्म् - य॒ज्ञ्॒म् । \newline
7. प्रति॑ देव॒यद्भ्यो॑ देव॒यद्भ्यः॒ प्रति॒ प्रति॑ देव॒यद्भ्यः॑ । \newline
8. दे॒व॒यद्भ्य॒ इति॑ देव॒यत् - भ्यः॒ । \newline
9. सू॒र्याया॒ ऊध॒ ऊधः॑ सू॒र्यायाः᳚ सू॒र्याया॒ ऊधः॑ । \newline
10. ऊधो ऽदि॑त्या॒ अदि॑त्या॒ ऊध॒ ऊधो ऽदि॑त्याः । \newline
11. अदि॑त्या उ॒पस्थ॑ उ॒पस्थे ऽदि॑त्या॒ अदि॑त्या उ॒पस्थे᳚ । \newline
12. उ॒पस्थ॑ उ॒रुधा॑रो॒ रुधा॑रो॒ पस्थ॑ उ॒पस्थ॑ उ॒रुधा॑रा । \newline
13. उ॒पस्थ॒ इत्यु॒प - स्थे॒ । \newline
14. उ॒रुधा॑रा पृथि॒वी पृ॑थि॒ व्यु॑रुधा॑रो॒ रुधा॑रा पृथि॒वी । \newline
15. उ॒रुधा॒रेत्यु॒रु - धा॒रा॒ । \newline
16. पृ॒थि॒वी य॒ज्ञे य॒ज्ञे पृ॑थि॒वी पृ॑थि॒वी य॒ज्ञे । \newline
17. य॒ज्ञे अ॒स्मिन् न॒स्मिन्. य॒ज्ञे य॒ज्ञे अ॒स्मिन्न् । \newline
18. अ॒स्मिन्नित्य॒स्मिन्न् । \newline
19. प्र॒जाप॑तेर् वि॒भान्. वि॒भान् प्र॒जाप॑तेः प्र॒जाप॑तेर् वि॒भान् । \newline
20. प्र॒जाप॑ते॒रिति॑ प्र॒जा - प॒तेः॒ । \newline
21. वि॒भान् नाम॒ नाम॑ वि॒भान्. वि॒भान् नाम॑ । \newline
22. वि॒भानिति॑ वि - भान् । \newline
23. नाम॑ लो॒को लो॒को नाम॒ नाम॑ लो॒कः । \newline
24. लो॒क स्तस्मि॒(ग्ग्॒) स्तस्मि॑न्न् ॅलो॒को लो॒क स्तस्मिन्न्॑ । \newline
25. तस्मि(ग्ग्॑) स्त्वा त्वा॒ तस्मि॒(ग्ग्॒) स्तस्मि(ग्ग्॑) स्त्वा । \newline
26. त्वा॒ द॒धा॒मि॒ द॒धा॒मि॒ त्वा॒ त्वा॒ द॒धा॒मि॒ । \newline
27. द॒धा॒मि॒ स॒ह स॒ह द॑धामि दधामि स॒ह । \newline
28. स॒ह यज॑मानेन॒ यज॑मानेन स॒ह स॒ह यज॑मानेन । \newline
29. यज॑मानेन॒ सथ् सद् यज॑मानेन॒ यज॑मानेन॒ सत् । \newline
30. सद॑स्यसि॒ सथ् सद॑सि । \newline
31. अ॒सि॒ सथ् सद॑स्यसि॒ सत् । \newline
32. सन् मे॑ मे॒ सथ् सन् मे᳚ । \newline
33. मे॒ भू॒या॒ भू॒या॒ मे॒ मे॒ भू॒याः॒ । \newline
34. भू॒याः॒ सर्व॒(ग्म्॒) सर्व॑म् भूया भूयाः॒ सर्व᳚म् । \newline
35. सर्व॑ मस्यसि॒ सर्व॒(ग्म्॒) सर्व॑ मसि । \newline
36. अ॒सि॒ सर्व॒(ग्म्॒) सर्व॑ मस्यसि॒ सर्व᳚म् । \newline
37. सर्व॑म् मे मे॒ सर्व॒(ग्म्॒) सर्व॑म् मे । \newline
38. मे॒ भू॒या॒ भू॒या॒ मे॒ मे॒ भू॒याः॒ । \newline
39. भू॒याः॒ पू॒र्णम् पू॒र्णम् भू॑या भूयाः पू॒र्णम् । \newline
40. पू॒र्ण म॑स्यसि पू॒र्णम् पू॒र्ण म॑सि । \newline
41. अ॒सि॒ पू॒र्णम् पू॒र्ण म॑स्यसि पू॒र्णम् । \newline
42. पू॒र्णम् मे॑ मे पू॒र्णम् पू॒र्णम् मे᳚ । \newline
43. मे॒ भू॒या॒ भू॒या॒ मे॒ मे॒ भू॒याः॒ । \newline
44. भू॒या॒ अक्षि॑त॒ मक्षि॑तम् भूया भूया॒ अक्षि॑तम् । \newline
45. अक्षि॑त मस्य॒ स्यक्षि॑त॒ मक्षि॑त मसि । \newline
46. अ॒सि॒ मा मा ऽस्य॑सि॒ मा । \newline
47. मा मे॑ मे॒ मा मा मे᳚ । \newline
48. मे॒ क्षे॒ष्ठाः॒ क्षे॒ष्ठा॒ मे॒ मे॒ क्षे॒ष्ठाः॒ । \newline
49. क्षे॒ष्ठाः॒ प्राच्या॒म् प्राच्या᳚म् क्षेष्ठाः क्षेष्ठाः॒ प्राच्या᳚म् । \newline
50. प्राच्या᳚म् दि॒शि दि॒शि प्राच्या॒म् प्राच्या᳚म् दि॒शि । \newline
51. दि॒शि दे॒वा दे॒वा दि॒शि दि॒शि दे॒वाः । \newline
52. दे॒वा ऋ॒त्विज॑ ऋ॒त्विजो॑ दे॒वा दे॒वा ऋ॒त्विजः॑ । \newline
53. ऋ॒त्विजो॑ मार्जयन्ताम् मार्जयन्ता मृ॒त्विज॑ ऋ॒त्विजो॑ मार्जयन्ताम् । \newline
54. मा॒र्ज॒य॒न्ता॒म् दक्षि॑णाया॒म् दक्षि॑णायाम् मार्जयन्ताम् मार्जयन्ता॒म् दक्षि॑णायाम् । \newline
55. दक्षि॑णायाम् दि॒शि दि॒शि दक्षि॑णाया॒म् दक्षि॑णायाम् दि॒शि । \newline

\textbf{Ghana Paata } \newline

1. आ प्या॑यताम् प्यायता॒ मा प्या॑यताम् ध्रु॒वा ध्रु॒वा प्या॑यता॒ मा प्या॑यताम् ध्रु॒वा । \newline
2. प्या॒य॒ता॒म् ध्रु॒वा ध्रु॒वा प्या॑यताम् प्यायताम् ध्रु॒वा घृ॒तेन॑ घृ॒तेन॑ ध्रु॒वा प्या॑यताम् प्यायताम् ध्रु॒वा घृ॒तेन॑ । \newline
3. ध्रु॒वा घृ॒तेन॑ घृ॒तेन॑ ध्रु॒वा ध्रु॒वा घृ॒तेन॑ य॒ज्ञ्ंॅय॑ज्ञ्ं ॅय॒ज्ञ्ंॅय॑ज्ञ्म् घृ॒तेन॑ ध्रु॒वा ध्रु॒वा घृ॒तेन॑ य॒ज्ञ्ंॅय॑ज्ञ्म् । \newline
4. घृ॒तेन॑ य॒ज्ञ्ंॅय॑ज्ञ्ं ॅय॒ज्ञ्ंॅय॑ज्ञ्म् घृ॒तेन॑ घृ॒तेन॑ य॒ज्ञ्ंॅय॑ज्ञ्॒म् प्रति॒ प्रति॑ य॒ज्ञ्ंॅय॑ज्ञ्म् घृ॒तेन॑ घृ॒तेन॑ य॒ज्ञ्ंॅय॑ज्ञ्॒म् प्रति॑ । \newline
5. य॒ज्ञ्ंॅय॑ज्ञ्॒म् प्रति॒ प्रति॑ य॒ज्ञ्ंॅय॑ज्ञ्ं ॅय॒ज्ञ्ंॅय॑ज्ञ्॒म् प्रति॑ देव॒यद्भ्यो॑ देव॒यद्भ्यः॒ प्रति॑ य॒ज्ञ्ंॅय॑ज्ञ्ं ॅय॒ज्ञ्ंॅय॑ज्ञ्॒म् प्रति॑ देव॒यद्भ्यः॑ । \newline
6. य॒ज्ञ्ंॅय॑ज्ञ्॒मिति॑ य॒ज्ञ्म् - य॒ज्ञ्॒म् । \newline
7. प्रति॑ देव॒यद्भ्यो॑ देव॒यद्भ्यः॒ प्रति॒ प्रति॑ देव॒यद्भ्यः॑ । \newline
8. दे॒व॒यद्भ्य॒ इति॑ देव॒यत् - भ्यः॒ । \newline
9. सू॒र्याया॒ ऊध॒ ऊधः॑ सू॒र्यायाः᳚ सू॒र्याया॒ ऊधो ऽदि॑त्या॒ अदि॑त्या॒ ऊधः॑ सू॒र्यायाः᳚ सू॒र्याया॒ ऊधो ऽदि॑त्याः । \newline
10. ऊधो ऽदि॑त्या॒ अदि॑त्या॒ ऊध॒ ऊधो ऽदि॑त्या उ॒पस्थ॑ उ॒पस्थे ऽदि॑त्या॒ ऊध॒ ऊधो ऽदि॑त्या उ॒पस्थे᳚ । \newline
11. अदि॑त्या उ॒पस्थ॑ उ॒पस्थे ऽदि॑त्या॒ अदि॑त्या उ॒पस्थ॑ उ॒रुधा॑रो॒रुधा॑रो॒पस्थे ऽदि॑त्या॒ अदि॑त्या उ॒पस्थ॑ उ॒रुधा॑रा । \newline
12. उ॒पस्थ॑ उ॒रुधा॑ रो॒रुधा॑रो॒पस्थ॑ उ॒पस्थ॑ उ॒रुधा॑रा पृथि॒वी पृ॑थि॒व्यु॑ रुधा॑रो॒पस्थ॑ उ॒पस्थ॑ उ॒रुधा॑रा पृथि॒वी । \newline
13. उ॒पस्थ॒ इत्यु॒प - स्थे॒ । \newline
14. उ॒रुधा॑रा पृथि॒वी पृ॑थि॒व्यु॑ रुधा॑रो॒रुधा॑रा पृथि॒वी य॒ज्ञे य॒ज्ञे पृ॑थि॒व्यु॑ रुधा॑रो॒रुधा॑रा पृथि॒वी य॒ज्ञे । \newline
15. उ॒रुधा॒रेत्यु॒रु - धा॒रा॒ । \newline
16. पृ॒थि॒वी य॒ज्ञे य॒ज्ञे पृ॑थि॒वी पृ॑थि॒वी य॒ज्ञे अ॒स्मिन् न॒स्मिन्. य॒ज्ञे पृ॑थि॒वी पृ॑थि॒वी य॒ज्ञे अ॒स्मिन्न् । \newline
17. य॒ज्ञे अ॒स्मिन् न॒स्मिन्. य॒ज्ञे य॒ज्ञे अ॒स्मिन्न् । \newline
18. अ॒स्मिन्नित्य॒स्मिन्न् । \newline
19. प्र॒जाप॑तेर् वि॒भान्. वि॒भान् प्र॒जाप॑तेः प्र॒जाप॑तेर् वि॒भान् नाम॒ नाम॑ वि॒भान् प्र॒जाप॑तेः प्र॒जाप॑तेर् वि॒भान् नाम॑ । \newline
20. प्र॒जाप॑ते॒रिति॑ प्र॒जा - प॒तेः॒ । \newline
21. वि॒भान् नाम॒ नाम॑ वि॒भान्. वि॒भान् नाम॑ लो॒को लो॒को नाम॑ वि॒भान्. वि॒भान् नाम॑ लो॒कः । \newline
22. वि॒भानिति॑ वि - भान् । \newline
23. नाम॑ लो॒को लो॒को नाम॒ नाम॑ लो॒कस्तस्मि॒(ग्ग्॒) स्तस्मि॑न्न् ॅलो॒को नाम॒ नाम॑ लो॒कस्तस्मिन्न्॑ । \newline
24. लो॒कस्तस्मि॒(ग्ग्॒) स्तस्मि॑न्न् ॅलो॒को लो॒कस्तस्मि(ग्ग्॑) स्त्वा त्वा॒ तस्मि॑न्न् ॅलो॒को लो॒कस्तस्मि(ग्ग्॑) स्त्वा । \newline
25. तस्मि(ग्ग्॑) स्त्वा त्वा॒ तस्मि॒(ग्ग्॒) स्तस्मि(ग्ग्॑) स्त्वा दधामि दधामि त्वा॒ तस्मि॒(ग्ग्॒) स्तस्मि(ग्ग्॑) स्त्वा दधामि । \newline
26. त्वा॒ द॒धा॒मि॒ द॒धा॒मि॒ त्वा॒ त्वा॒ द॒धा॒मि॒ स॒ह स॒ह द॑धामि त्वा त्वा दधामि स॒ह । \newline
27. द॒धा॒मि॒ स॒ह स॒ह द॑धामि दधामि स॒ह यज॑मानेन॒ यज॑मानेन स॒ह द॑धामि दधामि स॒ह यज॑मानेन । \newline
28. स॒ह यज॑मानेन॒ यज॑मानेन स॒ह स॒ह यज॑मानेन॒ सथ् सद् यज॑मानेन स॒ह स॒ह यज॑मानेन॒ सत् । \newline
29. यज॑मानेन॒ सथ् सद् यज॑मानेन॒ यज॑मानेन॒ सद॑स्यसि॒ सद् यज॑मानेन॒ यज॑मानेन॒ सद॑सि । \newline
30. सद॑स्यसि॒ सथ् सद॑सि॒ सथ् सद॑सि॒ सथ् सद॑सि॒ सत् । \newline
31. अ॒सि॒ सथ् सद॑स्यसि॒ सन् मे॑ मे॒ सद॑स्यसि॒ सन् मे᳚ । \newline
32. सन् मे॑ मे॒ सथ् सन् मे॑ भूया भूया मे॒ सथ् सन् मे॑ भूयाः । \newline
33. मे॒ भू॒या॒ भू॒या॒ मे॒ मे॒ भू॒याः॒ सर्व॒(ग्म्॒) सर्व॑म् भूया मे मे भूयाः॒ सर्व᳚म् । \newline
34. भू॒याः॒ सर्व॒(ग्म्॒) सर्व॑म् भूया भूयाः॒ सर्व॑ मस्यसि॒ सर्व॑म् भूया भूयाः॒ सर्व॑ मसि । \newline
35. सर्व॑ मस्यसि॒ सर्व॒(ग्म्॒) सर्व॑ मसि॒ सर्व॒(ग्म्॒) सर्व॑ मसि॒ सर्व॒(ग्म्॒) सर्व॑ मसि॒ सर्व᳚म् । \newline
36. अ॒सि॒ सर्व॒(ग्म्॒) सर्व॑ मस्यसि॒ सर्व॑म् मे मे॒ सर्व॑ मस्यसि॒ सर्व॑म् मे । \newline
37. सर्व॑म् मे मे॒ सर्व॒(ग्म्॒) सर्व॑म् मे भूया भूया मे॒ सर्व॒(ग्म्॒) सर्व॑म् मे भूयाः । \newline
38. मे॒ भू॒या॒ भू॒या॒ मे॒ मे॒ भू॒याः॒ पू॒र्णम् पू॒र्णम् भू॑या मे मे भूयाः पू॒र्णम् । \newline
39. भू॒याः॒ पू॒र्णम् पू॒र्णम् भू॑या भूयाः पू॒र्ण म॑स्यसि पू॒र्णम् भू॑या भूयाः पू॒र्ण म॑सि । \newline
40. पू॒र्ण म॑स्यसि पू॒र्णम् पू॒र्ण म॑सि पू॒र्णम् पू॒र्ण म॑सि पू॒र्णम् पू॒र्ण म॑सि पू॒र्णम् । \newline
41. अ॒सि॒ पू॒र्णम् पू॒र्ण म॑स्यसि पू॒र्णम् मे॑ मे पू॒र्ण म॑स्यसि पू॒र्णम् मे᳚ । \newline
42. पू॒र्णम् मे॑ मे पू॒र्णम् पू॒र्णम् मे॑ भूया भूया मे पू॒र्णम् पू॒र्णम् मे॑ भूयाः । \newline
43. मे॒ भू॒या॒ भू॒या॒ मे॒ मे॒ भू॒या॒ अक्षि॑त॒ मक्षि॑तम् भूया मे मे भूया॒ अक्षि॑तम् । \newline
44. भू॒या॒ अक्षि॑त॒ मक्षि॑तम् भूया भूया॒ अक्षि॑त मस्य॒स्यक्षि॑तम् भूया भूया॒ अक्षि॑त मसि । \newline
45. अक्षि॑त मस्य॒स्यक्षि॑त॒ मक्षि॑त मसि॒ मा मा ऽस्यक्षि॑त॒ मक्षि॑त मसि॒ मा । \newline
46. अ॒सि॒ मा मा ऽस्य॑सि॒ मा मे॑ मे॒ मा ऽस्य॑सि॒ मा मे᳚ । \newline
47. मा मे॑ मे॒ मा मा मे᳚ क्षेष्ठाः क्षेष्ठा मे॒ मा मा मे᳚ क्षेष्ठाः । \newline
48. मे॒ क्षे॒ष्ठाः॒ क्षे॒ष्ठा॒ मे॒ मे॒ क्षे॒ष्ठाः॒ प्राच्या॒म् प्राच्या᳚म् क्षेष्ठा मे मे क्षेष्ठाः॒ प्राच्या᳚म् । \newline
49. क्षे॒ष्ठाः॒ प्राच्या॒म् प्राच्या᳚म् क्षेष्ठाः क्षेष्ठाः॒ प्राच्या᳚म् दि॒शि दि॒शि प्राच्या᳚म् क्षेष्ठाः क्षेष्ठाः॒ प्राच्या᳚म् दि॒शि । \newline
50. प्राच्या᳚म् दि॒शि दि॒शि प्राच्या॒म् प्राच्या᳚म् दि॒शि दे॒वा दे॒वा दि॒शि प्राच्या॒म् प्राच्या᳚म् दि॒शि दे॒वाः । \newline
51. दि॒शि दे॒वा दे॒वा दि॒शि दि॒शि दे॒वा ऋ॒त्विज॑ ऋ॒त्विजो॑ दे॒वा दि॒शि दि॒शि दे॒वा ऋ॒त्विजः॑ । \newline
52. दे॒वा ऋ॒त्विज॑ ऋ॒त्विजो॑ दे॒वा दे॒वा ऋ॒त्विजो॑ मार्जयन्ताम् मार्जयन्ता मृ॒त्विजो॑ दे॒वा दे॒वा ऋ॒त्विजो॑ मार्जयन्ताम् । \newline
53. ऋ॒त्विजो॑ मार्जयन्ताम् मार्जयन्ता मृ॒त्विज॑ ऋ॒त्विजो॑ मार्जयन्ता॒म् दक्षि॑णाया॒म् दक्षि॑णायाम् मार्जयन्ता मृ॒त्विज॑ ऋ॒त्विजो॑ मार्जयन्ता॒म् दक्षि॑णायाम् । \newline
54. मा॒र्ज॒य॒न्ता॒म् दक्षि॑णाया॒म् दक्षि॑णायाम् मार्जयन्ताम् मार्जयन्ता॒म् दक्षि॑णायाम् दि॒शि दि॒शि दक्षि॑णायाम् मार्जयन्ताम् मार्जयन्ता॒म् दक्षि॑णायाम् दि॒शि । \newline
55. दक्षि॑णायाम् दि॒शि दि॒शि दक्षि॑णाया॒म् दक्षि॑णायाम् दि॒शि मासा॒ मासा॑ दि॒शि दक्षि॑णाया॒म् दक्षि॑णायाम् दि॒शि मासाः᳚ । \newline
\pagebreak
\markright{ TS 1.6.5.2  \hfill https://www.vedavms.in \hfill}

\section{ TS 1.6.5.2 }

\textbf{TS 1.6.5.2 } \newline
\textbf{Samhita Paata} \newline

दि॒शि मासाः᳚ पि॒तरो॑ मार्जयन्तां प्र॒तीच्यां᳚ दि॒शि गृ॒हाः प॒शवो॑ मार्जयन्ता॒मुदी᳚च्यां दि॒श्याप॒ ओष॑धयो॒ वन॒स्पत॑यो मार्जयन्तामू॒र्द्ध्वायां᳚ दि॒शि य॒ज्ञ्ः सं॑ॅवथ्स॒रो य॒ज्ञ्प॑तिर् मार्जयन्तां॒ ॅविष्णोः॒ क्रमो᳚ऽस्यभिमाति॒हा गा॑य॒त्रेण॒ छन्द॑सा पृथि॒वीमनु॒ वि क्र॑मे॒ निर्भ॑क्तः॒ स यं द्वि॒ष्मो विष्णोः॒ क्रमो᳚ऽस्यभिशस्ति॒हा त्रैष्टु॑भेन॒ छन्द॑सा॒ ऽन्तरि॑क्ष॒मनु॒ वि क्र॑मे॒ निर्भ॑क्तः॒ स यं द्वि॒ष्मो विष्णोः॒ क्रमो᳚ऽस्यरातीय॒तो ह॒न्ता जाग॑तेन॒ छन्द॑सा॒ दिव॒मनु॒ वि क्र॑मे॒ निर्भ॑क्तः॒ स यं द्वि॒ष्मो विष्णोः॒ क्रमो॑ऽसि शत्रूय॒तो ह॒न्ताऽऽनु॑ष्टुभेन॒ छन्द॑सा॒ दिशोऽनु॒ वि क्र॑मे॒ निर्भ॑क्तः॒ स यं द्वि॒ष्मः ॥ \newline

\textbf{Pada Paata} \newline

दि॒शि । मासाः᳚ । पि॒तरः॑ । मा॒र्ज॒य॒न्ता॒म् । प्र॒तीच्या᳚म् । दि॒शि । गृ॒हाः । प॒शवः॑ । मा॒र्ज॒य॒न्ता॒म् । उदी᳚च्याम् । दि॒शि । आपः॑ । ओष॑धयः । वन॒स्पत॑यः । मा॒र्ज॒य॒न्ता॒म् । ऊ॒द्‌र्ध्वाया᳚म् । दि॒शि । य॒ज्ञ्ः । सम्ॅव॒थ्स॒र इति॑ सं - व॒थ्स॒रः । य॒ज्ञ्प॑ति॒रिति॑ य॒ज्ञ् - प॒तिः॒ । मा॒र्ज॒य॒न्ता॒म् । विष्णोः᳚ । क्रमः॑ । अ॒सि॒ । अ॒भि॒मा॒ति॒हेत्य॑भिमाति - हा । गा॒य॒त्रेण॑ । छन्द॑सा । पृ॒थि॒वीम् । अनु॑ । वीति॑ । क्र॒मे॒ । निर्भ॑क्त॒ इति॒ निः - भ॒क्तः॒ । सः । यम् । द्वि॒ष्मः । विष्णोः᳚ । क्रमः॑ । अ॒सि॒ । अ॒भि॒श॒स्ति॒हेत्य॑भिशस्ति - हा । त्रैष्टु॑भेन । छन्द॑सा । अ॒न्तरि॑क्षम् । अनु॑ । वीति॑ । क्र॒मे॒ । निर्भ॑क्त॒ इति॒ निः-भ॒क्तः॒ । सः । यम् । द्वि॒ष्मः । विष्णोः᳚ ( ) । क्रमः॑ । अ॒सि॒ । अ॒रा॒ती॒य॒तः । ह॒न्ता । जाग॑तेन । छन्द॑सा । दिव᳚म् । अनु॑ । वीति॑ । क्र॒मे॒ । निर्भ॑क्त॒ इति॒ निः - भ॒क्तः॒ । सः । यम् । द्वि॒ष्मः । विष्णाः᳚ । क्रमः॑ । अ॒सि॒ । श॒त्रू॒य॒त इति॑ शत्रु - य॒तः । ह॒न्ता । आनु॑ष्टुभे॒नेत्यानु॑ - स्थु॒भे॒न॒ । छन्द॑सा । दिशः॑ । अनु॑ । वीति॑ । क्र॒मे॒ । निर्भ॑क्त॒ इति॒ निः - भ॒क्तः॒ । सः । यम् । द्वि॒ष्मः ॥  \newline


\textbf{Krama Paata} \newline

दि॒शि मासाः᳚ । मासाः᳚ पि॒तरः॑ । पि॒तरो॑ मार्जयन्ताम् । मा॒र्ज॒य॒न्ता॒म् प्र॒तीच्या᳚म् । प्र॒तीच्या᳚म् दि॒शि । दि॒शि गृ॒हाः । गृ॒हाः प॒शवः॑ । प॒शवो॑ मार्जयन्ताम् । मा॒र्ज॒य॒न्ता॒मुदी᳚च्याम् । उदी᳚च्याम् दि॒शि । दि॒श्यापः॑ । आप॒ ओष॑धयः । ओष॑धयो॒ वन॒स्पत॑यः । वन॒स्पत॑यो मार्जयन्ताम् । मा॒र्ज॒य॒न्ता॒मू॒र्द्ध्वाया᳚म् । ऊ॒र्द्ध्वाया᳚म् दि॒शि । दि॒शि य॒ज्ञ्ः । य॒ज्ञ्ः स॑म्ॅवथ्स॒रः । स॒म्ॅव॒थ्स॒रो य॒ज्ञ्प॑तिः । स॒म्ॅव॒थ्स॒र इति॑ सम् - व॒थ्स॒रः । य॒ज्ञ्प॑तिर्,मार्जयन्ताम् । य॒ज्ञ्प॑ति॒रिति॑ य॒ज्ञ् - प॒तिः॒ । मा॒र्ज॒य॒न्ता॒म् ॅविष्णोः᳚ । विष्णोः॒ क्रमः॑ । क्रमो॑ऽसि । अ॒स्य॒भि॒मा॒ति॒हा । अ॒भि॒मा॒ति॒हा गा॑य॒त्रेण॑ । अ॒भि॒मा॒ति॒हेत्य॑भिमाति - हा । गा॒य॒त्रेण॒ छन्द॑सा । छन्द॑सा पृथि॒वीम् । पृ॒थि॒वीमनु॑ । अनु॒ वि । वि क्र॑मे । क्र॒मे॒ निर्भ॑क्तः । निर्भ॑क्तः॒ सः । निर्भ॑क्त॒ इति॒ निः - भ॒क्तः॒ । स यम् । यम् द्वि॒ष्मः । द्वि॒ष्मो विष्णोः᳚ । विष्णोः॒ क्रमः॑ । क्रमो॑ऽसि । अ॒स्य॒भि॒श॒स्ति॒हा । अ॒भि॒श॒स्ति॒हा त्रैष्टु॑भेन । अ॒भि॒श॒स्ति॒हेत्य॑भिशस्ति - हा । त्रैष्टु॑भेन॒ छन्द॑सा । छन्द॑सा॒ ऽन्तरि॑क्षम् । अ॒न्तरि॑क्ष॒मनु॑ । अनु॒ वि । वि क्र॑मे । क्र॒मे॒ निर्भ॑क्तः । निर्भ॑क्तः॒ सः । निर्भ॑क्त॒ इति॒ निः - भ॒क्तः॒ । स यम् । यम् द्वि॒ष्मः । द्वि॒ष्मो विष्णोः᳚ । विष्णोः॒ क्रमः॑ । क्रमो॑ऽसि । अ॒स्य॒रा॒ती॒य॒तः । अ॒रा॒ती॒य॒तो ह॒न्ता । ह॒न्ता जाग॑तेन । जाग॑तेन॒ छन्द॑सा । छन्द॑सा॒ दिव᳚म् । दिव॒मनु॑ । अनु॒ वि । विक्र॑मे । क्र॒मे॒ निर्भ॑क्तः । निर्भ॑क्तः॒ सः । निर्भ॑क्त॒ इति॒ निः - भ॒क्तः॒ । स यम् । यम् द्वि॒ष्मः । द्वि॒ष्मो विष्णोः᳚ ( ) । विष्णोः॒ क्रमः॑ । क्रमो॑ऽसि । अ॒सि॒ श॒त्रू॒य॒तः । श॒त्रू॒य॒तो ह॒न्ता । श॒त्रू॒य॒त इति॑ शत्रु - य॒तः । ह॒न्ताऽऽनु॑ष्टुभेन । आनु॑ष्टुभेन॒ छन्द॑सा । आनु॑ष्टुभे॒नेत्यानु॑ - स्तु॒भे॒न॒ । छन्द॑सा॒ दिशः॑ । दिशोऽनु॑ । अनु॒ वि । वि क्र॑मे । क्र॒मे॒ निर्भ॑क्तः । निर्भ॑क्तः॒ सः । निर्भ॑क्त॒ इति॒ निः - भ॒क्तः॒ । स यम् । यम् द्वि॒ष्मः । द्वि॒ष्म इति॑ द्वि॒ष्मः । \newline

\textbf{Jatai Paata} \newline

1. दि॒शि मासा॒ मासा॑ दि॒शि दि॒शि मासाः᳚ । \newline
2. मासाः᳚ पि॒तरः॑ पि॒तरो॒ मासा॒ मासाः᳚ पि॒तरः॑ । \newline
3. पि॒तरो॑ मार्जयन्ताम् मार्जयन्ताम् पि॒तरः॑ पि॒तरो॑ मार्जयन्ताम् । \newline
4. मा॒र्ज॒य॒न्ता॒म् प्र॒तीच्या᳚म् प्र॒तीच्या᳚म् मार्जयन्ताम् मार्जयन्ताम् प्र॒तीच्या᳚म् । \newline
5. प्र॒तीच्या᳚म् दि॒शि दि॒शि प्र॒तीच्या᳚म् प्र॒तीच्या᳚म् दि॒शि । \newline
6. दि॒शि गृ॒हा गृ॒हा दि॒शि दि॒शि गृ॒हाः । \newline
7. गृ॒हाः प॒शवः॑ प॒शवो॑ गृ॒हा गृ॒हाः प॒शवः॑ । \newline
8. प॒शवो॑ मार्जयन्ताम् मार्जयन्ताम् प॒शवः॑ प॒शवो॑ मार्जयन्ताम् । \newline
9. मा॒र्ज॒य॒न्ता॒ मुदी᳚च्या॒ मुदी᳚च्याम् मार्जयन्ताम् मार्जयन्ता॒ मुदी᳚च्याम् । \newline
10. उदी᳚च्याम् दि॒शि दि॒श्युदी᳚च्या॒ मुदी᳚च्याम् दि॒शि । \newline
11. दि॒श्याप॒ आपो॑ दि॒शि दि॒श्यापः॑ । \newline
12. आप॒ ओष॑धय॒ ओष॑धय॒ आप॒ आप॒ ओष॑धयः । \newline
13. ओष॑धयो॒ वन॒स्पत॑यो॒ वन॒स्पत॑य॒ ओष॑धय॒ ओष॑धयो॒ वन॒स्पत॑यः । \newline
14. वन॒स्पत॑यो मार्जयन्ताम् मार्जयन्तां॒ ॅवन॒स्पत॑यो॒ वन॒स्पत॑यो मार्जयन्ताम् । \newline
15. मा॒र्ज॒य॒न्ता॒ मू॒र्द्ध्वाया॑ मू॒र्द्ध्वाया᳚म् मार्जयन्ताम् मार्जयन्ता मू॒र्द्ध्वाया᳚म् । \newline
16. ऊ॒र्द्ध्वाया᳚म् दि॒शि दि॒श्यू᳚र्द्ध्वाया॑ मू॒र्द्ध्वाया᳚म् दि॒शि । \newline
17. दि॒शि य॒ज्ञो य॒ज्ञो दि॒शि दि॒शि य॒ज्ञ्ः । \newline
18. य॒ज्ञ्ः स॑म्ॅवथ्स॒रः स॑म्ॅवथ्स॒रो य॒ज्ञो य॒ज्ञ्ः सम्ॅव॑थ्सरः । \newline
19. स॒म्ॅव॒थ्स॒रो य॒ज्ञ्प॑तिर् य॒ज्ञ्प॑तिः॒ स॑म्ॅवथ्स॒रः स॑म्ॅवथ्स॒रो य॒ज्ञ्प॑तिः । \newline
20. स॒म्ॅव॒थ्स॒र इति॑ सं - व॒थ्स॒रः । \newline
21. य॒ज्ञ्प॑तिर् मार्जयन्ताम् मार्जयन्तां ॅय॒ज्ञ्प॑तिर् य॒ज्ञ्प॑तिर् मार्जयन्ताम् । \newline
22. य॒ज्ञ्प॑ति॒रिति॑ य॒ज्ञ् - प॒तिः॒ । \newline
23. मा॒र्ज॒य॒न्तां॒ ॅविष्णो॒र् विष्णो᳚र् मार्जयन्ताम् मार्जयन्तां॒ ॅविष्णोः᳚ । \newline
24. विष्णोः॒ क्रमः॒ क्रमो॒ विष्णो॒र् विष्णोः॒ क्रमः॑ । \newline
25. क्रमो᳚ ऽस्यसि॒ क्रमः॒ क्रमो॑ ऽसि । \newline
26. अ॒स्य॒भि॒मा॒ति॒हा ऽभि॑माति॒हा ऽस्य॑ स्यभिमाति॒हा । \newline
27. अ॒भि॒मा॒ति॒हा गा॑य॒त्रेण॑ गाय॒त्रे णा॑भिमाति॒हा ऽभि॑माति॒हा गा॑य॒त्रेण॑ । \newline
28. अ॒भि॒मा॒ति॒हेत्य॑भिमाति - हा । \newline
29. गा॒य॒त्रेण॒ छन्द॑सा॒ छन्द॑सा गाय॒त्रेण॑ गाय॒त्रेण॒ छन्द॑सा । \newline
30. छन्द॑सा पृथि॒वीम् पृ॑थि॒वीम् छन्द॑सा॒ छन्द॑सा पृथि॒वीम् । \newline
31. पृ॒थि॒वी मन्वनु॑ पृथि॒वीम् पृ॑थि॒वी मनु॑ । \newline
32. अनु॒ वि व्यन्वनु॒ वि । \newline
33. वि क्र॑मे क्रमे॒ वि वि क्र॑मे । \newline
34. क्र॒मे॒ निर्भ॑क्तो॒ निर्भ॑क्तः क्रमे क्रमे॒ निर्भ॑क्तः । \newline
35. निर्भ॑क्तः॒ स स निर्भ॑क्तो॒ निर्भ॑क्तः॒ सः । \newline
36. निर्भ॑क्त॒ इति॒ निः - भ॒क्तः॒ । \newline
37. स यं ॅयꣳ स स यम् । \newline
38. यम् द्वि॒ष्मो द्वि॒ष्मो यं ॅयम् द्वि॒ष्मः । \newline
39. द्वि॒ष्मो विष्णो॒र् विष्णो᳚र् द्वि॒ष्मो द्वि॒ष्मो विष्णोः᳚ । \newline
40. विष्णोः॒ क्रमः॒ क्रमो॒ विष्णो॒र् विष्णोः॒ क्रमः॑ । \newline
41. क्रमो᳚ ऽस्यसि॒ क्रमः॒ क्रमो॑ ऽसि । \newline
42. अ॒स्य॒भि॒श॒स्ति॒हा ऽभि॑शस्ति॒हा ऽस्य॑ स्यभिशस्ति॒हा । \newline
43. अ॒भि॒श॒स्ति॒हा त्रैष्टु॑भेन॒ त्रैष्टु॑भे नाभिशस्ति॒हा ऽभि॑शस्ति॒हा त्रैष्टु॑भेन । \newline
44. अ॒भि॒श॒स्ति॒हेत्य॑भिशस्ति - हा । \newline
45. त्रैष्टु॑भेन॒ छन्द॑सा॒ छन्द॑सा॒ त्रैष्टु॑भेन॒ त्रैष्टु॑भेन॒ छन्द॑सा । \newline
46. छन्द॑सा॒ ऽन्तरि॑क्ष म॒न्तरि॑क्ष॒म् छन्द॑सा॒ छन्द॑सा॒ ऽन्तरि॑क्षम् । \newline
47. अ॒न्तरि॑क्ष॒ मन्व न्व॒न्तरि॑क्ष म॒न्तरि॑क्ष॒ मनु॑ । \newline
48. अनु॒ वि व्यन्वनु॒ वि । \newline
49. वि क्र॑मे क्रमे॒ वि वि क्र॑मे । \newline
50. क्र॒मे॒ निर्भ॑क्तो॒ निर्भ॑क्तः क्रमे क्रमे॒ निर्भ॑क्तः । \newline
51. निर्भ॑क्तः॒ स स निर्भ॑क्तो॒ निर्भ॑क्तः॒ सः । \newline
52. निर्भ॑क्त॒ इति॒ निः - भ॒क्तः॒ । \newline
53. स यं ॅयꣳ स स यम् । \newline
54. यम् द्वि॒ष्मो द्वि॒ष्मो यं ॅयम् द्वि॒ष्मः । \newline
55. द्वि॒ष्मो विष्णो॒र् विष्णो᳚र् द्वि॒ष्मो द्वि॒ष्मो विष्णोः᳚ । \newline
56. विष्णोः॒ क्रमः॒ क्रमो॒ विष्णो॒र् विष्णोः॒ क्रमः॑ । \newline
57. क्रमो᳚ ऽस्यसि॒ क्रमः॒ क्रमो॑ ऽसि । \newline
58. अ॒स्य॒रा॒ती॒य॒तो॑ ऽरातीय॒तो᳚ ऽस्य स्यरातीय॒तः । \newline
59. अ॒रा॒ती॒य॒तो ह॒न्ता ह॒न्ता ऽरा॑तीय॒तो॑ ऽरातीय॒तो ह॒न्ता । \newline
60. ह॒न्ता जाग॑तेन॒ जाग॑तेन ह॒न्ता ह॒न्ता जाग॑तेन । \newline
61. जाग॑तेन॒ छन्द॑सा॒ छन्द॑सा॒ जाग॑तेन॒ जाग॑तेन॒ छन्द॑सा । \newline
62. छन्द॑सा॒ दिव॒म् दिव॒म् छन्द॑सा॒ छन्द॑सा॒ दिव᳚म् । \newline
63. दिव॒ मन्वनु॒ दिव॒म् दिव॒ मनु॑ । \newline
64. अनु॒ वि व्यन्वनु॒ वि । \newline
65. वि क्र॑मे क्रमे॒ वि वि क्र॑मे । \newline
66. क्र॒मे॒ निर्भ॑क्तो॒ निर्भ॑क्तः क्रमे क्रमे॒ निर्भ॑क्तः । \newline
67. निर्भ॑क्तः॒ स स निर्भ॑क्तो॒ निर्भ॑क्तः॒ सः । \newline
68. निर्भ॑क्त॒ इति॒ निः - भ॒क्तः॒ । \newline
69. स यं ॅयꣳ स स यम् । \newline
70. यम् द्वि॒ष्मो द्वि॒ष्मो यं ॅयम् द्वि॒ष्मः । \newline
71. द्वि॒ष्मो विष्णो॒र् विष्णो᳚र् द्वि॒ष्मो द्वि॒ष्मो विष्णोः᳚ । \newline
72. विष्णोः॒ क्रमः॒ क्रमो॒ विष्णो॒र् विष्णोः॒ क्रमः॑ । \newline
73. क्रमो᳚ ऽस्यसि॒ क्रमः॒ क्रमो॑ ऽसि । \newline
74. अ॒सि॒ श॒त्रू॒य॒तः श॑त्रूय॒तो᳚ ऽस्यसि शत्रूय॒तः । \newline
75. श॒त्रू॒य॒तो ह॒न्ता ह॒न्ता श॑त्रूय॒तः श॑त्रूय॒तो ह॒न्ता । \newline
76. श॒त्रू॒य॒त इति॑ शत्रु - य॒तः । \newline
77. ह॒न्ता ऽऽनु॑ष्टुभे॒ नानु॑ष्टुभेन ह॒न्ता ह॒न्ता ऽऽनु॑ष्टुभेन । \newline
78. आनु॑ष्टुभेन॒ छन्द॑सा॒ छन्द॒सा ऽऽनु॑ष्टुभे॒ नानु॑ष्टुभेन॒ छन्द॑सा । \newline
79. आनु॑ष्टुभे॒नेत्यानु॑ - स्तु॒भे॒न॒ । \newline
80. छन्द॑सा॒ दिशो॒ दिश॒ श्छन्द॑सा॒ छन्द॑सा॒ दिशः॑ । \newline
81. दिशो ऽन्वनु॒ दिशो॒ दिशो ऽनु॑ । \newline
82. अनु॒ वि व्यन्वनु॒ वि । \newline
83. वि क्र॑मे क्रमे॒ वि वि क्र॑मे । \newline
84. क्र॒मे॒ निर्भ॑क्तो॒ निर्भ॑क्तः क्रमे क्रमे॒ निर्भ॑क्तः । \newline
85. निर्भ॑क्तः॒ स स निर्भ॑क्तो॒ निर्भ॑क्तः॒ सः । \newline
86. निर्भ॑क्त॒ इति॒ निः - भ॒क्तः॒ । \newline
87. स यं ॅयꣳ स स यम् । \newline
88. यम् द्वि॒ष्मो द्वि॒ष्मो यं ॅयम् द्वि॒ष्मः । \newline
89. द्वि॒ष्म इति॑ द्वि॒ष्मः । \newline

\textbf{Ghana Paata } \newline

1. दि॒शि मासा॒ मासा॑ दि॒शि दि॒शि मासाः᳚ पि॒तरः॑ पि॒तरो॒ मासा॑ दि॒शि दि॒शि मासाः᳚ पि॒तरः॑ । \newline
2. मासाः᳚ पि॒तरः॑ पि॒तरो॒ मासा॒ मासाः᳚ पि॒तरो॑ मार्जयन्ताम् मार्जयन्ताम् पि॒तरो॒ मासा॒ मासाः᳚ पि॒तरो॑ मार्जयन्ताम् । \newline
3. पि॒तरो॑ मार्जयन्ताम् मार्जयन्ताम् पि॒तरः॑ पि॒तरो॑ मार्जयन्ताम् प्र॒तीच्या᳚म् प्र॒तीच्या᳚म् मार्जयन्ताम् पि॒तरः॑ पि॒तरो॑ मार्जयन्ताम् प्र॒तीच्या᳚म् । \newline
4. मा॒र्ज॒य॒न्ता॒म् प्र॒तीच्या᳚म् प्र॒तीच्या᳚म् मार्जयन्ताम् मार्जयन्ताम् प्र॒तीच्या᳚म् दि॒शि दि॒शि प्र॒तीच्या᳚म् मार्जयन्ताम् मार्जयन्ताम् प्र॒तीच्या᳚म् दि॒शि । \newline
5. प्र॒तीच्या᳚म् दि॒शि दि॒शि प्र॒तीच्या᳚म् प्र॒तीच्या᳚म् दि॒शि गृ॒हा गृ॒हा दि॒शि प्र॒तीच्या᳚म् प्र॒तीच्या᳚म् दि॒शि गृ॒हाः । \newline
6. दि॒शि गृ॒हा गृ॒हा दि॒शि दि॒शि गृ॒हाः प॒शवः॑ प॒शवो॑ गृ॒हा दि॒शि दि॒शि गृ॒हाः प॒शवः॑ । \newline
7. गृ॒हाः प॒शवः॑ प॒शवो॑ गृ॒हा गृ॒हाः प॒शवो॑ मार्जयन्ताम् मार्जयन्ताम् प॒शवो॑ गृ॒हा गृ॒हाः प॒शवो॑ मार्जयन्ताम् । \newline
8. प॒शवो॑ मार्जयन्ताम् मार्जयन्ताम् प॒शवः॑ प॒शवो॑ मार्जयन्ता॒ मुदी᳚च्या॒ मुदी᳚च्याम् मार्जयन्ताम् प॒शवः॑ प॒शवो॑ मार्जयन्ता॒ मुदी᳚च्याम् । \newline
9. मा॒र्ज॒य॒न्ता॒ मुदी᳚च्या॒ मुदी᳚च्याम् मार्जयन्ताम् मार्जयन्ता॒ मुदी᳚च्याम् दि॒शि दि॒श्युदी᳚च्याम् मार्जयन्ताम् मार्जयन्ता॒ मुदी᳚च्याम् दि॒शि । \newline
10. उदी᳚च्याम् दि॒शि दि॒श्युदी᳚च्या॒ मुदी᳚च्याम् दि॒श्याप॒ आपो॑ दि॒श्युदी᳚च्या॒ मुदी᳚च्याम् दि॒श्यापः॑ । \newline
11. दि॒श्याप॒ आपो॑ दि॒शि दि॒श्याप॒ ओष॑धय॒ ओष॑धय॒ आपो॑ दि॒शि दि॒श्याप॒ ओष॑धयः । \newline
12. आप॒ ओष॑धय॒ ओष॑धय॒ आप॒ आप॒ ओष॑धयो॒ वन॒स्पत॑यो॒ वन॒स्पत॑य॒ ओष॑धय॒ आप॒ आप॒ ओष॑धयो॒ वन॒स्पत॑यः । \newline
13. ओष॑धयो॒ वन॒स्पत॑यो॒ वन॒स्पत॑य॒ ओष॑धय॒ ओष॑धयो॒ वन॒स्पत॑यो मार्जयन्ताम् मार्जयन्तां॒ ॅवन॒स्पत॑य॒ ओष॑धय॒ ओष॑धयो॒ वन॒स्पत॑यो मार्जयन्ताम् । \newline
14. वन॒स्पत॑यो मार्जयन्ताम् मार्जयन्तां॒ ॅवन॒स्पत॑यो॒ वन॒स्पत॑यो मार्जयन्ता मू॒र्द्ध्वाया॑ मू॒र्द्ध्वाया᳚म् मार्जयन्तां॒ ॅवन॒स्पत॑यो॒ वन॒स्पत॑यो मार्जयन्ता मू॒र्द्ध्वाया᳚म् । \newline
15. मा॒र्ज॒य॒न्ता॒ मू॒र्द्ध्वाया॑ मू॒र्द्ध्वाया᳚म् मार्जयन्ताम् मार्जयन्ता मू॒र्द्ध्वाया᳚म् दि॒शि दि॒श्यू᳚र्द्ध्वाया᳚म् मार्जयन्ताम् मार्जयन्ता मू॒र्द्ध्वाया᳚म् दि॒शि । \newline
16. ऊ॒र्द्ध्वाया᳚म् दि॒शि दि॒श्यू᳚र्द्ध्वाया॑ मू॒र्द्ध्वाया᳚म् दि॒शि य॒ज्ञो य॒ज्ञो दि॒श्यू᳚र्द्ध्वाया॑ मू॒र्द्ध्वाया᳚म् दि॒शि य॒ज्ञ्ः । \newline
17. दि॒शि य॒ज्ञो य॒ज्ञो दि॒शि दि॒शि य॒ज्ञ्ः सम्ॅव॑थ्स॒रः स॑म्ॅवथ्स॒रो य॒ज्ञो दि॒शि दि॒शि य॒ज्ञ्ः स॑म्ॅवथ्स॒रः । \newline
18. य॒ज्ञ्ः स॑म्ॅवथ्स॒रः स॑म्ॅवथ्स॒रो य॒ज्ञो य॒ज्ञ्ः स॑म्ॅवथ्स॒रो य॒ज्ञ्प॑तिर् य॒ज्ञ्प॑तिः॒ स॑म्ॅवथ्स॒रो य॒ज्ञो य॒ज्ञ्ः स॑म्ॅवथ्स॒रो य॒ज्ञ्प॑तिः । \newline
19. स॒म्ॅव॒थ्स॒रो य॒ज्ञ्प॑तिर् य॒ज्ञ्प॑तिः॒ स॑म्ॅवथ्स॒रः स॑म्ॅवथ्स॒रो य॒ज्ञ्प॑तिर् मार्जयन्ताम् मार्जयन्तां ॅय॒ज्ञ्प॑तिः॒ स॑म्ॅवथ्स॒रः स॑म्ॅवथ्स॒रो य॒ज्ञ्प॑तिर् मार्जयन्ताम् । \newline
20. स॒म्ॅव॒थ्स॒र इति॑ सं - व॒थ्स॒रः । \newline
21. य॒ज्ञ्प॑तिर् मार्जयन्ताम् मार्जयन्तां ॅय॒ज्ञ्प॑तिर् य॒ज्ञ्प॑तिर् मार्जयन्तां॒ ॅविष्णो॒र् विष्णो᳚र् मार्जयन्तां ॅय॒ज्ञ्प॑तिर् य॒ज्ञ्प॑तिर् मार्जयन्तां॒ ॅविष्णोः᳚ । \newline
22. य॒ज्ञ्प॑ति॒रिति॑ य॒ज्ञ् - प॒तिः॒ । \newline
23. मा॒र्ज॒य॒न्तां॒ ॅविष्णो॒र् विष्णो᳚र् मार्जयन्ताम् मार्जयन्तां॒ ॅविष्णोः॒ क्रमः॒ क्रमो॒ विष्णो᳚र् मार्जयन्ताम् मार्जयन्तां॒ ॅविष्णोः॒ क्रमः॑ । \newline
24. विष्णोः॒ क्रमः॒ क्रमो॒ विष्णो॒र् विष्णोः॒ क्रमो᳚ ऽस्यसि॒ क्रमो॒ विष्णो॒र् विष्णोः॒ क्रमो॑ ऽसि । \newline
25. क्रमो᳚ ऽस्यसि॒ क्रमः॒ क्रमो᳚ ऽस्यभिमाति॒हा ऽभि॑माति॒हा ऽसि॒ क्रमः॒ क्रमो᳚ ऽस्यभिमाति॒हा । \newline
26. अ॒स्य॒भि॒मा॒ति॒हा ऽभि॑माति॒हा ऽस्य॑स्यभिमाति॒हा गा॑य॒त्रेण॑ गाय॒त्रेणा॑भिमाति॒हा ऽस्य॑स्यभिमाति॒हा गा॑य॒त्रेण॑ । \newline
27. अ॒भि॒मा॒ति॒हा गा॑य॒त्रेण॑ गाय॒त्रेणा॑भिमाति॒हा ऽभि॑माति॒हा गा॑य॒त्रेण॒ छन्द॑सा॒ छन्द॑सा गाय॒त्रेणा॑भिमाति॒हा ऽभि॑माति॒हा गा॑य॒त्रेण॒ छन्द॑सा । \newline
28. अ॒भि॒मा॒ति॒हेत्य॑भिमाति - हा । \newline
29. गा॒य॒त्रेण॒ छन्द॑सा॒ छन्द॑सा गाय॒त्रेण॑ गाय॒त्रेण॒ छन्द॑सा पृथि॒वीम् पृ॑थि॒वीम् छन्द॑सा गाय॒त्रेण॑ गाय॒त्रेण॒ छन्द॑सा पृथि॒वीम् । \newline
30. छन्द॑सा पृथि॒वीम् पृ॑थि॒वीम् छन्द॑सा॒ छन्द॑सा पृथि॒वी मन्वनु॑ पृथि॒वीम् छन्द॑सा॒ छन्द॑सा पृथि॒वी मनु॑ । \newline
31. पृ॒थि॒वी मन्वनु॑ पृथि॒वीम् पृ॑थि॒वी मनु॒ वि व्यनु॑ पृथि॒वीम् पृ॑थि॒वी मनु॒ वि । \newline
32. अनु॒ वि व्यन्वनु॒ वि क्र॑मे क्रमे॒ व्यन्वनु॒ वि क्र॑मे । \newline
33. वि क्र॑मे क्रमे॒ वि वि क्र॑मे॒ निर्भ॑क्तो॒ निर्भ॑क्तः क्रमे॒ वि वि क्र॑मे॒ निर्भ॑क्तः । \newline
34. क्र॒मे॒ निर्भ॑क्तो॒ निर्भ॑क्तः क्रमे क्रमे॒ निर्भ॑क्तः॒ स स निर्भ॑क्तः क्रमे क्रमे॒ निर्भ॑क्तः॒ सः । \newline
35. निर्भ॑क्तः॒ स स निर्भ॑क्तो॒ निर्भ॑क्तः॒ स यं ॅयꣳ स निर्भ॑क्तो॒ निर्भ॑क्तः॒ स यम् । \newline
36. निर्भ॑क्त॒ इति॒ निः - भ॒क्तः॒ । \newline
37. स यं ॅयꣳ स स यम् द्वि॒ष्मो द्वि॒ष्मो यꣳ स स यम् द्वि॒ष्मः । \newline
38. यम् द्वि॒ष्मो द्वि॒ष्मो यं ॅयम् द्वि॒ष्मो विष्णो॒र् विष्णो᳚र् द्वि॒ष्मो यं ॅयम् द्वि॒ष्मो विष्णोः᳚ । \newline
39. द्वि॒ष्मो विष्णो॒र् विष्णो᳚र् द्वि॒ष्मो द्वि॒ष्मो विष्णोः॒ क्रमः॒ क्रमो॒ विष्णो᳚र् द्वि॒ष्मो द्वि॒ष्मो विष्णोः॒ क्रमः॑ । \newline
40. विष्णोः॒ क्रमः॒ क्रमो॒ विष्णो॒र् विष्णोः॒ क्रमो᳚ ऽस्यसि॒ क्रमो॒ विष्णो॒र् विष्णोः॒ क्रमो॑ ऽसि । \newline
41. क्रमो᳚ ऽस्यसि॒ क्रमः॒ क्रमो᳚ ऽस्यभिशस्ति॒हा ऽभि॑शस्ति॒हा ऽसि॒ क्रमः॒ क्रमो᳚ ऽस्यभिशस्ति॒हा । \newline
42. अ॒स्य॒भि॒श॒स्ति॒हा ऽभि॑शस्ति॒हा ऽस्य॑स्यभिशस्ति॒हा त्रैष्टु॑भेन॒ त्रैष्टु॑भेनाभिशस्ति॒हा ऽस्य॑स्यभिशस्ति॒हा त्रैष्टु॑भेन । \newline
43. अ॒भि॒श॒स्ति॒हा त्रैष्टु॑भेन॒ त्रैष्टु॑भेनाभिशस्ति॒हा ऽभि॑शस्ति॒हा त्रैष्टु॑भेन॒ छन्द॑सा॒ छन्द॑सा॒ त्रैष्टु॑भेनाभिशस्ति॒हा ऽभि॑शस्ति॒हा त्रैष्टु॑भेन॒ छन्द॑सा । \newline
44. अ॒भि॒श॒स्ति॒हेत्य॑भिशस्ति - हा । \newline
45. त्रैष्टु॑भेन॒ छन्द॑सा॒ छन्द॑सा॒ त्रैष्टु॑भेन॒ त्रैष्टु॑भेन॒ छन्द॑सा॒ ऽन्तरि॑क्ष म॒न्तरि॑क्ष॒म् छन्द॑सा॒ त्रैष्टु॑भेन॒ त्रैष्टु॑भेन॒ छन्द॑सा॒ ऽन्तरि॑क्षम् । \newline
46. छन्द॑सा॒ ऽन्तरि॑क्ष म॒न्तरि॑क्ष॒म् छन्द॑सा॒ छन्द॑सा॒ ऽन्तरि॑क्ष॒ मन्वन्व॒न्तरि॑क्ष॒म् छन्द॑सा॒ छन्द॑सा॒ ऽन्तरि॑क्ष॒ मनु॑ । \newline
47. अ॒न्तरि॑क्ष॒ मन्वन्व॒न्तरि॑क्ष म॒न्तरि॑क्ष॒ मनु॒ वि व्यन्व॒न्तरि॑क्ष म॒न्तरि॑क्ष॒ मनु॒ वि । \newline
48. अनु॒ वि व्यन्वनु॒ वि क्र॑मे क्रमे॒ व्यन्वनु॒ वि क्र॑मे । \newline
49. वि क्र॑मे क्रमे॒ वि वि क्र॑मे॒ निर्भ॑क्तो॒ निर्भ॑क्तः क्रमे॒ वि वि क्र॑मे॒ निर्भ॑क्तः । \newline
50. क्र॒मे॒ निर्भ॑क्तो॒ निर्भ॑क्तः क्रमे क्रमे॒ निर्भ॑क्तः॒ स स निर्भ॑क्तः क्रमे क्रमे॒ निर्भ॑क्तः॒ सः । \newline
51. निर्भ॑क्तः॒ स स निर्भ॑क्तो॒ निर्भ॑क्तः॒ स यं ॅयꣳ स निर्भ॑क्तो॒ निर्भ॑क्तः॒ स यम् । \newline
52. निर्भ॑क्त॒ इति॒ निः - भ॒क्तः॒ । \newline
53. स यं ॅयꣳ स स यम् द्वि॒ष्मो द्वि॒ष्मो यꣳ स स यम् द्वि॒ष्मः । \newline
54. यम् द्वि॒ष्मो द्वि॒ष्मो यं ॅयम् द्वि॒ष्मो विष्णो॒र् विष्णो᳚र् द्वि॒ष्मो यं ॅयम् द्वि॒ष्मो विष्णोः᳚ । \newline
55. द्वि॒ष्मो विष्णो॒र् विष्णो᳚र् द्वि॒ष्मो द्वि॒ष्मो विष्णोः॒ क्रमः॒ क्रमो॒ विष्णो᳚र् द्वि॒ष्मो द्वि॒ष्मो विष्णोः॒ क्रमः॑ । \newline
56. विष्णोः॒ क्रमः॒ क्रमो॒ विष्णो॒र् विष्णोः॒ क्रमो᳚ ऽस्यसि॒ क्रमो॒ विष्णो॒र् विष्णोः॒ क्रमो॑ ऽसि । \newline
57. क्रमो᳚ ऽस्यसि॒ क्रमः॒ क्रमो᳚ ऽस्यरातीय॒तो॑ ऽरातीय॒तो॑ ऽसि॒ क्रमः॒ क्रमो᳚ ऽस्यरातीय॒तः । \newline
58. अ॒स्य॒रा॒ती॒य॒तो॑ ऽरातीय॒तो᳚ ऽस्यस्यरातीय॒तो ह॒न्ता ह॒न्ता ऽरा॑तीय॒तो᳚ ऽस्यस्यरातीय॒तो ह॒न्ता । \newline
59. अ॒रा॒ती॒य॒तो ह॒न्ता ह॒न्ता ऽरा॑तीय॒तो॑ ऽरातीय॒तो ह॒न्ता जाग॑तेन॒ जाग॑तेन ह॒न्ता ऽरा॑तीय॒तो॑ ऽरातीय॒तो ह॒न्ता जाग॑तेन । \newline
60. ह॒न्ता जाग॑तेन॒ जाग॑तेन ह॒न्ता ह॒न्ता जाग॑तेन॒ छन्द॑सा॒ छन्द॑सा॒ जाग॑तेन ह॒न्ता ह॒न्ता जाग॑तेन॒ छन्द॑सा । \newline
61. जाग॑तेन॒ छन्द॑सा॒ छन्द॑सा॒ जाग॑तेन॒ जाग॑तेन॒ छन्द॑सा॒ दिव॒म् दिव॒म् छन्द॑सा॒ जाग॑तेन॒ जाग॑तेन॒ छन्द॑सा॒ दिव᳚म् । \newline
62. छन्द॑सा॒ दिव॒म् दिव॒म् छन्द॑सा॒ छन्द॑सा॒ दिव॒ मन्वनु॒ दिव॒म् छन्द॑सा॒ छन्द॑सा॒ दिव॒ मनु॑ । \newline
63. दिव॒ मन्वनु॒ दिव॒म् दिव॒ मनु॒ वि व्यनु॒ दिव॒म् दिव॒ मनु॒ वि । \newline
64. अनु॒ वि व्यन्वनु॒ वि क्र॑मे क्रमे॒ व्यन्वनु॒ वि क्र॑मे । \newline
65. वि क्र॑मे क्रमे॒ वि वि क्र॑मे॒ निर्भ॑क्तो॒ निर्भ॑क्तः क्रमे॒ वि वि क्र॑मे॒ निर्भ॑क्तः । \newline
66. क्र॒मे॒ निर्भ॑क्तो॒ निर्भ॑क्तः क्रमे क्रमे॒ निर्भ॑क्तः॒ स स निर्भ॑क्तः क्रमे क्रमे॒ निर्भ॑क्तः॒ सः । \newline
67. निर्भ॑क्तः॒ स स निर्भ॑क्तो॒ निर्भ॑क्तः॒ स यं ॅयꣳ स निर्भ॑क्तो॒ निर्भ॑क्तः॒ स यम् । \newline
68. निर्भ॑क्त॒ इति॒ निः - भ॒क्तः॒ । \newline
69. स यं ॅयꣳ स स यम् द्वि॒ष्मो द्वि॒ष्मो यꣳ स स यम् द्वि॒ष्मः । \newline
70. यम् द्वि॒ष्मो द्वि॒ष्मो यं ॅयम् द्वि॒ष्मो विष्णो॒र् विष्णो᳚र् द्वि॒ष्मो यं ॅयम् द्वि॒ष्मो विष्णोः᳚ । \newline
71. द्वि॒ष्मो विष्णो॒र् विष्णो᳚र् द्वि॒ष्मो द्वि॒ष्मो विष्णोः॒ क्रमः॒ क्रमो॒ विष्णो᳚र् द्वि॒ष्मो द्वि॒ष्मो विष्णोः॒ क्रमः॑ । \newline
72. विष्णोः॒ क्रमः॒ क्रमो॒ विष्णो॒र् विष्णोः॒ क्रमो᳚ ऽस्यसि॒ क्रमो॒ विष्णो॒र् विष्णोः॒ क्रमो॑ ऽसि । \newline
73. क्रमो᳚ ऽस्यसि॒ क्रमः॒ क्रमो॑ ऽसि शत्रूय॒तः श॑त्रूय॒तो॑ ऽसि॒ क्रमः॒ क्रमो॑ ऽसि शत्रूय॒तः । \newline
74. अ॒सि॒ श॒त्रू॒य॒तः श॑त्रूय॒तो᳚ ऽस्यसि शत्रूय॒तो ह॒न्ता ह॒न्ता श॑त्रूय॒तो᳚ ऽस्यसि शत्रूय॒तो ह॒न्ता । \newline
75. श॒त्रू॒य॒तो ह॒न्ता ह॒न्ता श॑त्रूय॒तः श॑त्रूय॒तो ह॒न्ता ऽऽनु॑ष्टुभे॒नानु॑ष्टुभेन ह॒न्ता श॑त्रूय॒तः श॑त्रूय॒तो ह॒न्ता ऽऽनु॑ष्टुभेन । \newline
76. श॒त्रू॒य॒त इति॑ शत्रु - य॒तः । \newline
77. ह॒न्ता ऽऽनु॑ष्टुभे॒नानु॑ष्टुभेन ह॒न्ता ह॒न्ता ऽऽनु॑ष्टुभेन॒ छन्द॑सा॒ छन्द॒सा ऽऽनु॑ष्टुभेन ह॒न्ता ह॒न्ता ऽऽनु॑ष्टुभेन॒ छन्द॑सा । \newline
78. आनु॑ष्टुभेन॒ छन्द॑सा॒ छन्द॒सा ऽऽनु॑ष्टुभे॒नानु॑ष्टुभेन॒ छन्द॑सा॒ दिशो॒ दिश॒ श्छन्द॒सा ऽऽनु॑ष्टुभे॒नानु॑ष्टुभेन॒ छन्द॑सा॒ दिशः॑ । \newline
79. आनु॑ष्टुभे॒नेत्यानु॑ - स्तु॒भे॒न॒ । \newline
80. छन्द॑सा॒ दिशो॒ दिश॒ श्छन्द॑सा॒ छन्द॑सा॒ दिशो ऽन्वनु॒ दिश॒ श्छन्द॑सा॒ छन्द॑सा॒ दिशो ऽनु॑ । \newline
81. दिशो ऽन्वनु॒ दिशो॒ दिशो ऽनु॒ वि व्यनु॒ दिशो॒ दिशो ऽनु॒ वि । \newline
82. अनु॒ वि व्यन्वनु॒ वि क्र॑मे क्रमे॒ व्यन्वनु॒ वि क्र॑मे । \newline
83. वि क्र॑मे क्रमे॒ वि वि क्र॑मे॒ निर्भ॑क्तो॒ निर्भ॑क्तः क्रमे॒ वि वि क्र॑मे॒ निर्भ॑क्तः । \newline
84. क्र॒मे॒ निर्भ॑क्तो॒ निर्भ॑क्तः क्रमे क्रमे॒ निर्भ॑क्तः॒ स स निर्भ॑क्तः क्रमे क्रमे॒ निर्भ॑क्तः॒ सः । \newline
85. निर्भ॑क्तः॒ स स निर्भ॑क्तो॒ निर्भ॑क्तः॒ स यं ॅयꣳ स निर्भ॑क्तो॒ निर्भ॑क्तः॒ स यम् । \newline
86. निर्भ॑क्त॒ इति॒ निः - भ॒क्तः॒ । \newline
87. स यं ॅयꣳ स स यम् द्वि॒ष्मो द्वि॒ष्मो यꣳ स स यम् द्वि॒ष्मः । \newline
88. यम् द्वि॒ष्मो द्वि॒ष्मो यं ॅयम् द्वि॒ष्मः । \newline
89. द्वि॒ष्म इति॑ द्वि॒ष्मः । \newline
\pagebreak
\markright{ TS 1.6.6.1  \hfill https://www.vedavms.in \hfill}

\section{ TS 1.6.6.1 }

\textbf{TS 1.6.6.1 } \newline
\textbf{Samhita Paata} \newline

अग॑न्म॒ सुवः॒ सुव॑रगन्म स॒न्दृश॑स्ते॒ मा छि॑थ्सि॒ यत्ते॒ तप॒स्तस्मै॑ ते॒ माऽऽ वृ॑क्षि सु॒भूर॑सि॒ श्रेष्ठो॑ रश्मी॒नामा॑यु॒र्द्धा अ॒स्यायु॑र्मे धेहि वर्चो॒धा अ॑सि॒ वर्चो॒ मयि॑ धेही॒दम॒हम॒मुं भ्रातृ॑व्यमा॒भ्यो दि॒ग्भ्यो᳚ऽस्यै दि॒वो᳚ऽस्माद॒न्तरि॑क्षाद॒स्यै पृ॑थि॒व्या अ॒स्माद॒न्नाद्या॒न्निर्भ॑जामि॒ निर्भ॑क्तः॒ स यं द्वि॒ष्मः ॥ \newline

\textbf{Pada Paata} \newline

अग॑न्म । सुवः॑ । सुवः॑ । अ॒ग॒न्म॒ । स॒दृंश॒ इति॑ सं-दृशः॑ । ते॒ । मा । छि॒थ्सि॒ । यत् । ते॒ । तपः॑ । तस्मै᳚ । ते॒ । मा । एति॑ । वृ॒क्षि॒ । सु॒भूरिति॑ सु - भूः । अ॒सि॒ । श्रेष्ठः॑ । र॒श्मी॒नाम् । आ॒यु॒द्‌र्धा इत्या॑युः - धाः । अ॒सि॒ । आयुः॑ । मे॒ । धे॒हि॒ । व॒र्चो॒धा इति॑ वर्चः -धाः । अ॒सि॒ । वर्चः॑ । मयि॑ । ध॒हि॒ । इ॒दम् । अ॒हम् । अ॒मुम् । भ्रातृ॑व्यम् । आ॒भ्यः । दि॒ग्भ्य इति॑ दिक् - भ्यः । अ॒स्यै । दि॒वः । अ॒स्मात् । अ॒न्तरि॑क्षात् । अ॒स्यै । पृ॒थि॒व्याः । अ॒स्मात् । अ॒न्नाद्या॒दित्य॑न्न -अद्या᳚त् । निरिति॑ । भ॒जा॒मि॒ । निर्भ॑क्त॒ इति॒ निः - भ॒क्तः॒ । सः । यम् । द्वि॒ष्मः ॥  \newline


\textbf{Krama Paata} \newline

अग॑न्म॒ सुवः॑ । सुवः॒ सुवः॑ । सुव॑रगन्म । अ॒ग॒न्म॒ स॒न्दृशः॑ । स॒न्दृश॑स्ते । स॒न्दृश॒ इति॑ सम् - दृशः॑ । ते॒ मा । मा छि॑थ्सि । छि॒थ्सि॒ यत् । यत्ते᳚ । ते॒ तपः॑ । तप॒स्तस्मै᳚ । तस्मै॑ ते । ते॒ मा । मा ऽऽ वृ॑क्षि । आ वृ॑क्षि । वृ॒क्षि॒ सु॒भूः । सु॒भूर॑सि । सु॒भूरिति॑ सु - भूः । अ॒सि॒ श्रेष्ठः॑ । श्रेष्ठो॑ रश्मी॒नाम् । र॒श्मी॒नामा॑यु॒र्द्धाः । आ॒यु॒र्द्धा अ॑सि । आ॒यु॒र्द्धा इत्या॑युः - धाः । अ॒स्यायुः॑ । आयु॑र्,मे । मे॒ धे॒हि॒ । धे॒हि॒ व॒र्चो॒धाः । व॒र्चो॒धा अ॑सि । व॒र्चो॒धा इति॑ वर्चः - धाः । अ॒सि॒ वर्चः॑ । वर्चो॒ मयि॑ । मयि॑ धेहि । धे॒ही॒दम् । इ॒दम॒हम् । अ॒हम॒मुम् । अ॒मुम् भ्रातृ॑व्यम् । भ्रातृ॑व्यमा॒भ्यः । आ॒भ्यो दि॒ग्भ्यः । दि॒ग्भ्यो᳚ऽस्यै । दि॒ग्भ्य इति॑ दिक् - भ्यः । अ॒स्यै दि॒वः । दि॒वो᳚ऽस्मात् । अ॒स्माद॒न्तरि॑क्षात् । अ॒न्तरि॑क्षाद॒स्यै । अ॒स्यै पृ॑थि॒व्याः । पृ॒थि॒व्या अ॒स्मात् । अ॒स्माद॒न्नाद्या᳚त् । अ॒न्नाद्या॒न्निः । अ॒न्नाद्या॒दित्य॑न्न - अद्या᳚त् । निर् भ॑जामि । भ॒जा॒मि॒ निर्भ॑क्तः । निर्भ॑क्तः॒ सः । निर्भ॑क्त॒ इति॒ निः - भ॒क्तः॒ । स यम् । यम् द्वि॒ष्मः । द्वि॒ष्म इति॑ द्वि॒ष्मः । \newline

\textbf{Jatai Paata} \newline

1. अग॑न्म॒ सुवः॒ सुव॒ रग॒न्मा ग॑न्म॒ सुवः॑ । \newline
2. सुवः॒ सुवः॑ । \newline
3. सुव॑ रगन्मा गन्म॒ सुवः॒ सुव॑ रगन्म । \newline
4. अ॒ग॒न्म॒ स॒न्दृशः॑ स॒न्दृशो॑ ऽगन्मा गन्म स॒न्दृशः॑ । \newline
5. स॒न्दृश॑ स्ते ते स॒न्दृशः॑ स॒न्दृश॑ स्ते । \newline
6. स॒न्दृश॒ इति॑ सं - दृशः॑ । \newline
7. ते॒ मा मा ते॑ ते॒ मा । \newline
8. मा छि॑थ्सि छिथ्सि॒ मा मा छि॑थ्सि । \newline
9. छि॒थ्सि॒ यद् यछ् चि॑थ्सि छिथ्सि॒ यत् । \newline
10. यत् ते॑ ते॒ यद् यत् ते᳚ । \newline
11. ते॒ तप॒ स्तप॑ स्ते ते॒ तपः॑ । \newline
12. तप॒ स्तस्मै॒ तस्मै॒ तप॒ स्तप॒ स्तस्मै᳚ । \newline
13. तस्मै॑ ते ते॒ तस्मै॒ तस्मै॑ ते । \newline
14. ते॒ मा मा ते॑ ते॒ मा । \newline
15. मा ऽऽवृ॑क्षि वृ॒क्ष्या मा मा ऽऽवृ॑क्षि । \newline
16. आ वृ॑क्षि वृ॒क्ष्या वृ॑क्षि । \newline
17. वृ॒क्षि॒ सु॒भूः सु॒भूर् वृ॑क्षि वृक्षि सु॒भूः । \newline
18. सु॒भू र॑स्यसि सु॒भूः सु॒भू र॑सि । \newline
19. सु॒भूरिति॑ सु - भूः । \newline
20. अ॒सि॒ श्रेष्ठः॒ श्रेष्ठो᳚ ऽस्यसि॒ श्रेष्ठः॑ । \newline
21. श्रेष्ठो॑ रश्मी॒नाꣳ र॑श्मी॒नाꣳ श्रेष्ठः॒ श्रेष्ठो॑ रश्मी॒नाम् । \newline
22. र॒श्मी॒ना मा॑यु॒र्द्धा आ॑यु॒र्द्धा र॑श्मी॒नाꣳ र॑श्मी॒ना मा॑यु॒र्द्धाः । \newline
23. आ॒यु॒र्द्धा अ॑स्यस्यायु॒र्द्धा आ॑यु॒र्द्धा अ॑सि । \newline
24. आ॒यु॒र्द्धा इत्या॑युः - धाः । \newline
25. अ॒स्यायु॒ रायु॑ रस्य॒ स्यायुः॑ । \newline
26. आयु॑र् मे म॒ आयु॒ रायु॑र् मे । \newline
27. मे॒ धे॒हि॒ धे॒हि॒ मे॒ मे॒ धे॒हि॒ । \newline
28. धे॒हि॒ व॒र्चो॒धा व॑र्चो॒धा धे॑हि धेहि वर्चो॒धाः । \newline
29. व॒र्चो॒धा अ॑स्यसि वर्चो॒धा व॑र्चो॒धा अ॑सि । \newline
30. व॒र्चो॒धा इति॑ वर्चः - धाः । \newline
31. अ॒सि॒ वर्चो॒ वर्चो᳚ ऽस्यसि॒ वर्चः॑ । \newline
32. वर्चो॒ मयि॒ मयि॒ वर्चो॒ वर्चो॒ मयि॑ । \newline
33. मयि॑ धेहि धेहि॒ मयि॒ मयि॑ धेहि । \newline
34. धे॒ही॒द मि॒दम् धे॑हि धेही॒दम् । \newline
35. इ॒द म॒ह म॒ह मि॒द मि॒द म॒हम् । \newline
36. अ॒ह म॒मु म॒मु म॒ह म॒ह म॒मुम् । \newline
37. अ॒मुम् भ्रातृ॑व्य॒म् भ्रातृ॑व्य म॒मु म॒मुम् भ्रातृ॑व्यम् । \newline
38. भ्रातृ॑व्य मा॒भ्य आ॒भ्यो भ्रातृ॑व्य॒म् भ्रातृ॑व्य मा॒भ्यः । \newline
39. आ॒भ्यो दि॒ग्भ्यो दि॒ग्भ्य आ॒भ्य आ॒भ्यो दि॒ग्भ्यः । \newline
40. दि॒ग्भ्यो᳚ ऽस्या अ॒स्यै दि॒ग्भ्यो दि॒ग्भ्यो᳚ ऽस्यै । \newline
41. दि॒ग्भ्य इति॑ दिक् - भ्यः । \newline
42. अ॒स्यै दि॒वो दि॒वो᳚ ऽस्या अ॒स्यै दि॒वः । \newline
43. दि॒वो᳚ ऽस्मा द॒स्माद् दि॒वो दि॒वो᳚ ऽस्मात् । \newline
44. अ॒स्मा द॒न्तरि॑क्षा द॒न्तरि॑क्षा द॒स्मा द॒स्मा द॒न्तरि॑क्षात् । \newline
45. अ॒न्तरि॑क्षा द॒स्या अ॒स्या अ॒न्तरि॑क्षा द॒न्तरि॑क्षा द॒स्यै । \newline
46. अ॒स्यै पृ॑थि॒व्याः पृ॑थि॒व्या अ॒स्या अ॒स्यै पृ॑थि॒व्याः । \newline
47. पृ॒थि॒व्या अ॒स्मा द॒स्मात् पृ॑थि॒व्याः पृ॑थि॒व्या अ॒स्मात् । \newline
48. अ॒स्मा द॒न्नाद्या॑ द॒न्नाद्या॑ द॒स्मा द॒स्मा द॒न्नाद्या᳚त् । \newline
49. अ॒न्नाद्या॒न् निर् णिर॒ न्नाद्या॑ द॒न्नाद्या॒न् निः । \newline
50. अ॒न्नाद्या॒दित्य॑न्न - अद्या᳚त् । \newline
51. निर् भ॑जामि भजामि॒ निर् णिर् भ॑जामि । \newline
52. भ॒जा॒मि॒ निर्भ॑क्तो॒ निर्भ॑क्तो भजामि भजामि॒ निर्भ॑क्तः । \newline
53. निर्भ॑क्तः॒ स स निर्भ॑क्तो॒ निर्भ॑क्तः॒ सः । \newline
54. निर्भ॑क्त॒ इति॒ निः - भ॒क्तः॒ । \newline
55. स यं ॅयꣳ स स यम् । \newline
56. यम् द्वि॒ष्मो द्वि॒ष्मो यं ॅयम् द्वि॒ष्मः । \newline
57. द्वि॒ष्म इति॑ द्वि॒ष्मः । \newline

\textbf{Ghana Paata } \newline

1. अग॑न्म॒ सुवः॒ सुव॒ रग॒न्माग॑न्म॒ सुवः॑ । \newline
2. सुवः॒ सुवः॑ । \newline
3. सुव॑ रगन्मागन्म॒ सुवः॒ सुव॑ रगन्म स॒न्दृशः॑ स॒न्दृशो॑ ऽगन्म॒ सुवः॒ सुव॑ रगन्म स॒न्दृशः॑ । \newline
4. अ॒ग॒न्म॒ स॒न्दृशः॑ स॒न्दृशो॑ ऽगन्मागन्म स॒न्दृश॑स्ते ते स॒न्दृशो॑ ऽगन्मागन्म स॒न्दृश॑स्ते । \newline
5. स॒न्दृश॑स्ते ते स॒न्दृशः॑ स॒न्दृश॑स्ते॒ मा मा ते॑ स॒न्दृशः॑ स॒न्दृश॑स्ते॒ मा । \newline
6. स॒न्दृश॒ इति॑ सं - दृशः॑ । \newline
7. ते॒ मा मा ते॑ ते॒ मा छि॑थ्सि छिथ्सि॒ मा ते॑ ते॒ मा छि॑थ्सि । \newline
8. मा छि॑थ्सि छिथ्सि॒ मा मा छि॑थ्सि॒ यद् यछ् चि॑थ्सि॒ मा मा छि॑थ्सि॒ यत् । \newline
9. छि॒थ्सि॒ यद् यछ् चि॑थ्सि छिथ्सि॒ यत् ते॑ ते॒ यछ् चि॑थ्सि छिथ्सि॒ यत् ते᳚ । \newline
10. यत् ते॑ ते॒ यद् यत् ते॒ तप॒ स्तप॑स्ते॒ यद् यत् ते॒ तपः॑ । \newline
11. ते॒ तप॒ स्तप॑स्ते ते॒ तप॒ स्तस्मै॒ तस्मै॒ तप॑स्ते ते॒ तप॒स्तस्मै᳚ । \newline
12. तप॒ स्तस्मै॒ तस्मै॒ तप॒ स्तप॒ स्तस्मै॑ ते ते॒ तस्मै॒ तप॒ स्तप॒ स्तस्मै॑ ते । \newline
13. तस्मै॑ ते ते॒ तस्मै॒ तस्मै॑ ते॒ मा मा ते॒ तस्मै॒ तस्मै॑ ते॒ मा । \newline
14. ते॒ मा मा ते॑ ते॒ मा ऽऽवृ॑क्षि वृ॒क्ष्या मा ते॑ ते॒ मा ऽऽवृ॑क्षि । \newline
15. मा ऽऽवृ॑क्षि वृ॒क्ष्या मा मा ऽऽवृ॑क्षि सु॒भूः सु॒भूर् वृ॒क्ष्या मा मा ऽऽवृ॑क्षि सु॒भूः । \newline
16. आ वृ॑क्षि वृ॒क्ष्या वृ॑क्षि सु॒भूः सु॒भूर् वृ॒क्ष्या वृ॑क्षि सु॒भूः । \newline
17. वृ॒क्षि॒ सु॒भूः सु॒भूर् वृ॑क्षि वृक्षि सु॒भूर॑स्यसि सु॒भूर् वृ॑क्षि वृक्षि सु॒भूर॑सि । \newline
18. सु॒भू र॑स्यसि सु॒भूः सु॒भूर॑सि॒ श्रेष्ठः॒ श्रेष्ठो॑ ऽसि सु॒भूः सु॒भूर॑सि॒ श्रेष्ठः॑ । \newline
19. सु॒भूरिति॑ सु - भूः । \newline
20. अ॒सि॒ श्रेष्ठः॒ श्रेष्ठो᳚ ऽस्यसि॒ श्रेष्ठो॑ रश्मी॒नाꣳ र॑श्मी॒नाꣳ श्रेष्ठो᳚ ऽस्यसि॒ श्रेष्ठो॑ रश्मी॒नाम् । \newline
21. श्रेष्ठो॑ रश्मी॒नाꣳ र॑श्मी॒नाꣳ श्रेष्ठः॒ श्रेष्ठो॑ रश्मी॒ना मा॑यु॒र्द्धा आ॑यु॒र्द्धा र॑श्मी॒नाꣳ श्रेष्ठः॒ श्रेष्ठो॑ रश्मी॒ना मा॑यु॒र्द्धाः । \newline
22. र॒श्मी॒ना मा॑यु॒र्द्धा आ॑यु॒र्द्धा र॑श्मी॒नाꣳ र॑श्मी॒ना मा॑यु॒र्द्धा अ॑स्यस्यायु॒र्द्धा र॑श्मी॒नाꣳ र॑श्मी॒ना मा॑यु॒र्द्धा अ॑सि । \newline
23. आ॒यु॒र्द्धा अ॑स्यस्यायु॒र्द्धा आ॑यु॒र्द्धा अ॒स्यायु॒ रायु॑ रस्यायु॒र्द्धा आ॑यु॒र्द्धा अ॒स्यायुः॑ । \newline
24. आ॒यु॒र्द्धा इत्या॑युः - धाः । \newline
25. अ॒स्यायु॒ रायु॑ रस्य॒स्यायु॑र् मे म॒ आयु॑ रस्य॒स्यायु॑र् मे । \newline
26. आयु॑र् मे म॒ आयु॒रायु॑र् मे धेहि धेहि म॒ आयु॒रायु॑र् मे धेहि । \newline
27. मे॒ धे॒हि॒ धे॒हि॒ मे॒ मे॒ धे॒हि॒ व॒र्चो॒धा व॑र्चो॒धा धे॑हि मे मे धेहि वर्चो॒धाः । \newline
28. धे॒हि॒ व॒र्चो॒धा व॑र्चो॒धा धे॑हि धेहि वर्चो॒धा अ॑स्यसि वर्चो॒धा धे॑हि धेहि वर्चो॒धा अ॑सि । \newline
29. व॒र्चो॒धा अ॑स्यसि वर्चो॒धा व॑र्चो॒धा अ॑सि॒ वर्चो॒ वर्चो॑ ऽसि वर्चो॒धा व॑र्चो॒धा अ॑सि॒ वर्चः॑ । \newline
30. व॒र्चो॒धा इति॑ वर्चः - धाः । \newline
31. अ॒सि॒ वर्चो॒ वर्चो᳚ ऽस्यसि॒ वर्चो॒ मयि॒ मयि॒ वर्चो᳚ ऽस्यसि॒ वर्चो॒ मयि॑ । \newline
32. वर्चो॒ मयि॒ मयि॒ वर्चो॒ वर्चो॒ मयि॑ धेहि धेहि॒ मयि॒ वर्चो॒ वर्चो॒ मयि॑ धेहि । \newline
33. मयि॑ धेहि धेहि॒ मयि॒ मयि॑ धेही॒द मि॒दम् धे॑हि॒ मयि॒ मयि॑ धेही॒दम् । \newline
34. धे॒ही॒द मि॒दम् धे॑हि धेही॒द म॒ह म॒ह मि॒दम् धे॑हि धेही॒द म॒हम् । \newline
35. इ॒द म॒ह म॒ह मि॒द मि॒द म॒ह म॒मु म॒मु म॒ह मि॒द मि॒द म॒ह म॒मुम् । \newline
36. अ॒ह म॒मु म॒मु म॒ह म॒ह म॒मुम् भ्रातृ॑व्य॒म् भ्रातृ॑व्य म॒मु म॒ह म॒ह म॒मुम् भ्रातृ॑व्यम् । \newline
37. अ॒मुम् भ्रातृ॑व्य॒म् भ्रातृ॑व्य म॒मु म॒मुम् भ्रातृ॑व्य मा॒भ्य आ॒भ्यो भ्रातृ॑व्य म॒मु म॒मुम् भ्रातृ॑व्य मा॒भ्यः । \newline
38. भ्रातृ॑व्य मा॒भ्य आ॒भ्यो भ्रातृ॑व्य॒म् भ्रातृ॑व्य मा॒भ्यो दि॒ग्भ्यो दि॒ग्भ्य आ॒भ्यो भ्रातृ॑व्य॒म् भ्रातृ॑व्य मा॒भ्यो दि॒ग्भ्यः । \newline
39. आ॒भ्यो दि॒ग्भ्यो दि॒ग्भ्य आ॒भ्य आ॒भ्यो दि॒ग्भ्यो᳚ ऽस्या अ॒स्यै दि॒ग्भ्य आ॒भ्य आ॒भ्यो दि॒ग्भ्यो᳚ ऽस्यै । \newline
40. दि॒ग्भ्यो᳚ ऽस्या अ॒स्यै दि॒ग्भ्यो दि॒ग्भ्यो᳚ ऽस्यै दि॒वो दि॒वो᳚ ऽस्यै दि॒ग्भ्यो दि॒ग्भ्यो᳚ ऽस्यै दि॒वः । \newline
41. दि॒ग्भ्य इति॑ दिक् - भ्यः । \newline
42. अ॒स्यै दि॒वो दि॒वो᳚ ऽस्या अ॒स्यै दि॒वो᳚ ऽस्माद॒स्माद् दि॒वो᳚ ऽस्या अ॒स्यै दि॒वो᳚ ऽस्मात् । \newline
43. दि॒वो᳚ ऽस्माद॒स्माद् दि॒वो दि॒वो᳚ ऽस्माद॒न्तरि॑क्षा द॒न्तरि॑क्षा द॒स्माद् दि॒वो दि॒वो᳚ ऽस्माद॒न्तरि॑क्षात् । \newline
44. अ॒स्मा द॒न्तरि॑क्षा द॒न्तरि॑क्षा द॒स्मा द॒स्मा द॒न्तरि॑क्षाद॒स्या अ॒स्या अ॒न्तरि॑क्षा द॒स्मा द॒स्मा द॒न्तरि॑क्षाद॒स्यै । \newline
45. अ॒न्तरि॑क्षाद॒स्या अ॒स्या अ॒न्तरि॑क्षा द॒न्तरि॑क्षाद॒स्यै पृ॑थि॒व्याः पृ॑थि॒व्या अ॒स्या अ॒न्तरि॑क्षा द॒न्तरि॑क्षाद॒स्यै पृ॑थि॒व्याः । \newline
46. अ॒स्यै पृ॑थि॒व्याः पृ॑थि॒व्या अ॒स्या अ॒स्यै पृ॑थि॒व्या अ॒स्माद॒स्मात् पृ॑थि॒व्या अ॒स्या अ॒स्यै पृ॑थि॒व्या अ॒स्मात् । \newline
47. पृ॒थि॒व्या अ॒स्माद॒स्मात् पृ॑थि॒व्याः पृ॑थि॒व्या अ॒स्माद॒ न्नाद्या॑ द॒न्नाद्या॑ द॒स्मात् पृ॑थि॒व्याः पृ॑थि॒व्या अ॒स्माद॒न्नाद्या᳚त् । \newline
48. अ॒स्मा द॒न्नाद्या॑ द॒न्नाद्या॑ द॒स्मा द॒स्मा द॒न्नाद्या॒न् निर् णिर॒न्नाद्या॑ द॒स्मा द॒स्मा द॒न्नाद्या॒न् निः । \newline
49. अ॒न्नाद्या॒न् निर् णिर॒न्नाद्या॑ द॒न्नाद्या॒न् निर् भ॑जामि भजामि॒ निर॒न्नाद्या॑ द॒न्नाद्या॒न् निर् भ॑जामि । \newline
50. अ॒न्नाद्या॒दित्य॑न्न - अद्या᳚त् । \newline
51. निर् भ॑जामि भजामि॒ निर् णिर् भ॑जामि॒ निर्भ॑क्तो॒ निर्भ॑क्तो भजामि॒ निर् णिर् भ॑जामि॒ निर्भ॑क्तः । \newline
52. भ॒जा॒मि॒ निर्भ॑क्तो॒ निर्भ॑क्तो भजामि भजामि॒ निर्भ॑क्तः॒ स स निर्भ॑क्तो भजामि भजामि॒ निर्भ॑क्तः॒ सः । \newline
53. निर्भ॑क्तः॒ स स निर्भ॑क्तो॒ निर्भ॑क्तः॒ स यं ॅयꣳ स निर्भ॑क्तो॒ निर्भ॑क्तः॒ स यम् । \newline
54. निर्भ॑क्त॒ इति॒ निः - भ॒क्तः॒ । \newline
55. स यं ॅयꣳ स स यम् द्वि॒ष्मो द्वि॒ष्मो यꣳ स स यम् द्वि॒ष्मः । \newline
56. यम् द्वि॒ष्मो द्वि॒ष्मो यं ॅयम् द्वि॒ष्मः । \newline
57. द्वि॒ष्म इति॑ द्वि॒ष्मः । \newline
\pagebreak
\markright{ TS 1.6.6.2  \hfill https://www.vedavms.in \hfill}

\section{ TS 1.6.6.2 }

\textbf{TS 1.6.6.2 } \newline
\textbf{Samhita Paata} \newline

सं ज्योति॑षाऽभूवमै॒न्द्री-मा॒वृत॑-म॒न्वाव॑र्ते॒ सम॒हं प्र॒जया॒ सं मया᳚ प्र॒जा सम॒हꣳ रा॒यस्पोषे॑ण॒ सं मया॑ रा॒यस्पोषः॒ समि॑द्धो अग्ने मे दीदिहि समे॒द्धा ते॑ अग्ने दीद्यासं॒ ॅवसु॑मान्. य॒ज्ञो वसी॑यान् भूयास॒मग्न॒ आयूꣳ॑षि पवस॒ आ सु॒वोर्ज॒मिषं॑ च नः । आ॒रे बा॑धस्व दु॒च्छुनां᳚ ॥ अग्ने॒ पव॑स्व॒ स्वपा॑ अ॒स्मे वर्चः॑ सु॒वीर्यं᳚ । \newline

\textbf{Pada Paata} \newline

समिति॑ । ज्योति॑षा । अ॒भू॒व॒म् । ऐ॒न्द्रीम् । आ॒वृत॒मित्या᳚ - वृत᳚म् । अ॒न्वाव॑र्त॒ इत्य॑नु - आव॑र्ते । समिति॑ । अ॒हम् । प्र॒जयेति॑ प्र-जया᳚ । समिति॑ । मया᳚ । प्र॒जेति॑ प्र - जा । समिति॑ । अ॒हम् । रा॒यः । पोषे॑ण । समिति॑ । मया᳚ । रा॒यः । पोषः॑ । समि॑द्ध॒ इति॒ सम् - इ॒द्धः॒ । अ॒ग्ने॒ । मे॒ । दी॒दि॒हि॒ । स॒मे॒द्धेति॑ सम् - ए॒द्धा । ते॒ । अ॒ग्ने॒ । दी॒द्या॒स॒म् । वसु॑मा॒निति॒ वसु॑ - मा॒न् । य॒ज्ञ्ः । वसी॑यान् । भू॒या॒स॒म् । अग्ने᳚ । आयूꣳ॑षि । प॒व॒से॒ । एति॑ । सु॒व॒ । ऊर्ज᳚म् । इष᳚म् । च॒ । नः॒ ॥ आ॒रे । बा॒ध॒स्व॒ । दु॒च्छुना᳚म् ॥ अग्ने᳚ । पव॑स्व । स्वपा॒ इति॑ सु-अपाः᳚ । अ॒स्मे इति॑ । वर्चः॑ । सु॒वीर्य॒मिति॑ सु-वीर्य᳚म् ॥  \newline


\textbf{Krama Paata} \newline

सम् ज्योति॑षा । ज्योति॑षा ऽभूवम् । अ॒भू॒व॒मै॒न्द्रीम् । ऐ॒न्द्रीमा॒वृत᳚म् । आ॒वृत॑म॒न्वाव॑र्ते । आ॒वृत॒मित्या᳚ - वृत᳚म् । अ॒न्वाव॑र्ते॒ सम् । अ॒न्वाव॑र्त॒ इत्यु॑नु - आव॑र्ते । सम॒हम् । अ॒हम् प्र॒जया᳚ । प्र॒जया॒ सम् । प्र॒जयेति॑ प्र - जया᳚ । सम् मया᳚ । मया᳚ प्र॒जा । प्र॒जा सम् । प्र॒जेति॑ प्र - जा । सम॒हम् । अ॒हꣳ रा॒यः । रा॒यस्पोषे॑ण । पोषे॑ण॒ सम् । सम् मया᳚ । मया॑ रा॒यः । रा॒यस्पोषः॑ । पोषः॒ समि॑द्धः । समि॑द्धो अग्ने । समि॑द्ध॒ इति॒ सम् - इ॒द्धः॒ । अ॒ग्ने॒ मे॒ । मे॒ दी॒दि॒हि॒ । दी॒दि॒हि॒ स॒मे॒द्धा । स॒मे॒द्धा ते᳚ । स॒मे॒द्धेति॑ सम् - ए॒द्धा । ते॒ अ॒ग्ने॒ । अ॒ग्ने॒ दी॒द्या॒स॒म् । दी॒द्या॒स॒म् ॅवसु॑मान् । वसु॑मान्. य॒ज्ञ्ः । वसु॑मा॒निति॒ वसु॑ - मा॒न्॒ । य॒ज्ञो वसी॑यान् । वसी॑यान् भूयासम् । भू॒या॒स॒मग्ने᳚ । अग्न॒ आयूꣳ॑षि । आयूꣳ॑षि पवसे । प॒व॒स॒ आ । आ सु॑व । सु॒वोर्ज᳚म् । ऊर्ज॒मिष᳚म् । इष॑म् च । च॒ नः॒ । न॒ इति॑ नः ॥ आ॒रे बा॑धस्व । बा॒ध॒स्व॒ दु॒च्छुना᳚म् । दु॒च्छुना॒मिति॑ दु॒च्छुना᳚म् ॥ अग्ने॒ पव॑स्व । पव॑स्व॒ स्वपाः᳚ । स्वपा॑ अ॒स्मे । स्वपा॒ इति॑ सु - अपाः᳚ । अ॒स्मे वर्चः॑ । अ॒स्मे इत्य॒स्मे । वर्चः॑ सु॒वीर्य᳚म् । सु॒वीर्य॒मिति॑ सु - वीर्य᳚म् । \newline

\textbf{Jatai Paata} \newline

1. सम् ज्योति॑षा॒ ज्योति॑षा॒ सꣳ सम् ज्योति॑षा । \newline
2. ज्योति॑षा ऽभूव मभूव॒म् ज्योति॑षा॒ ज्योति॑षा ऽभूवम् । \newline
3. अ॒भू॒व॒ मै॒न्द्री मै॒न्द्री म॑भूव मभूव मै॒न्द्रीम् । \newline
4. ऐ॒न्द्री मा॒वृत॑ मा॒वृत॑ मै॒न्द्री मै॒न्द्री मा॒वृत᳚म् । \newline
5. आ॒वृत॑ म॒न्वाव॑र्ते॒ ऽन्वाव॑र्त आ॒वृत॑ मा॒वृत॑ म॒न्वाव॑र्ते । \newline
6. आ॒वृत॒मित्या᳚ - वृत᳚म् । \newline
7. अ॒न्वाव॑र्ते॒ सꣳ स म॒न्वाव॑र्ते॒ ऽन्वाव॑र्ते॒ सम् । \newline
8. अ॒न्वाव॑र्त॒ इत्य॑नु - आव॑र्ते । \newline
9. स म॒ह म॒हꣳ सꣳ स म॒हम् । \newline
10. अ॒हम् प्र॒जया᳚ प्र॒जया॒ ऽह म॒हम् प्र॒जया᳚ । \newline
11. प्र॒जया॒ सꣳ सम् प्र॒जया᳚ प्र॒जया॒ सम् । \newline
12. प्र॒जयेति॑ प्र - जया᳚ । \newline
13. सम् मया॒ मया॒ सꣳ सम् मया᳚ । \newline
14. मया᳚ प्र॒जा प्र॒जा मया॒ मया᳚ प्र॒जा । \newline
15. प्र॒जा सꣳ सम् प्र॒जा प्र॒जा सम् । \newline
16. प्र॒जेति॑ प्र - जा । \newline
17. स म॒ह म॒हꣳ सꣳ स म॒हम् । \newline
18. अ॒हꣳ रा॒यो रा॒यो॑ ऽह म॒हꣳ रा॒यः । \newline
19. रा॒य स्पोषे॑ण॒ पोषे॑ण रा॒यो रा॒य स्पोषे॑ण । \newline
20. पोषे॑ण॒ सꣳ सम् पोषे॑ण॒ पोषे॑ण॒ सम् । \newline
21. सम् मया॒ मया॒ सꣳ सम् मया᳚ । \newline
22. मया॑ रा॒यो रा॒यो मया॒ मया॑ रा॒यः । \newline
23. रा॒य स्पोषः॒ पोषो॑ रा॒यो रा॒य स्पोषः॑ । \newline
24. पोषः॒ समि॑द्धः॒ समि॑द्धः॒ पोषः॒ पोषः॒ समि॑द्धः । \newline
25. समि॑द्धो अग्ने ऽग्ने॒ समि॑द्धः॒ समि॑द्धो अग्ने । \newline
26. समि॑द्ध॒ इति॒ सम् - इ॒द्धः॒ । \newline
27. अ॒ग्ने॒ मे॒ मे॒ अ॒ग्ने॒ ऽग्ने॒ मे॒ । \newline
28. मे॒ दी॒दि॒हि॒ दी॒दि॒हि॒ मे॒ मे॒ दी॒दि॒हि॒ । \newline
29. दी॒दि॒हि॒ स॒मे॒द्धा स॑मे॒द्धा दी॑दिहि दीदिहि समे॒द्धा । \newline
30. स॒मे॒द्धा ते॑ ते समे॒द्धा स॑मे॒द्धा ते᳚ । \newline
31. स॒मे॒द्धेति॑ सम् - ए॒द्धा । \newline
32. ते॒ अ॒ग्ने॒ ऽग्ने॒ ते॒ ते॒ अ॒ग्ने॒ । \newline
33. अ॒ग्ने॒ दी॒द्या॒स॒म् दी॒द्या॒स॒ म॒ग्ने॒ ऽग्ने॒ दी॒द्या॒स॒म् । \newline
34. दी॒द्या॒सं॒ ॅवसु॑मा॒न्॒. वसु॑मान् दीद्यासम् दीद्यासं॒ ॅवसु॑मान् । \newline
35. वसु॑मान्. य॒ज्ञो य॒ज्ञो वसु॑मा॒न्॒. वसु॑मान्. य॒ज्ञ्ः । \newline
36. वसु॑मा॒निति॒ वसु॑ - मा॒न् । \newline
37. य॒ज्ञो वसी॑या॒न्॒. वसी॑यान्. य॒ज्ञो य॒ज्ञो वसी॑यान् । \newline
38. वसी॑यान् भूयासम् भूयासं॒ ॅवसी॑या॒न्॒. वसी॑यान् भूयासम् । \newline
39. भू॒या॒स॒ मग्ने ऽग्ने॑ भूयासम् भूयास॒ मग्ने᳚ । \newline
40. अग्न॒ आयू॒(ग्ग्॒) ष्यायू॒(ग्ग्॒) ष्यग्ने ऽग्न॒ आयू(ग्म्॑)षि । \newline
41. आयू(ग्म्॑)षि पवसे पवस॒ आयू॒(ग्ग्॒) ष्यायू(ग्म्॑)षि पवसे । \newline
42. प॒व॒स॒ आ प॑वसे पवस॒ आ । \newline
43. आ सु॑व सु॒वा सु॑व । \newline
44. सु॒वोर्ज॒ मूर्ज(ग्म्॑) सुव सु॒वोर्ज᳚म् । \newline
45. ऊर्ज॒ मिष॒ मिष॒ मूर्ज॒ मूर्ज॒ मिष᳚म् । \newline
46. इष॑म् च॒ चे ष॒ मिष॑म् च । \newline
47. च॒ नो॒ न॒श्च॒ च॒ नः॒ । \newline
48. न॒ इति॑ नः॒ । \newline
49. आ॒रे बा॑धस्व बाधस्वा॒र आ॒रे बा॑धस्व । \newline
50. बा॒ध॒स्व॒ दु॒च्छुना᳚म् दु॒च्छुना᳚म् बाधस्व बाधस्व दु॒च्छुना᳚म् । \newline
51. दु॒च्छुना॒मिति॑ दु॒च्छुना᳚म् । \newline
52. अग्ने॒ पव॑स्व॒ पव॒स्वाग्ने ऽग्ने॒ पव॑स्व । \newline
53. पव॑स्व॒ स्वपाः॒ स्वपाः॒ पव॑स्व॒ पव॑स्व॒ स्वपाः᳚ । \newline
54. स्वपा॑ अ॒स्मे अ॒स्मे स्वपाः॒ स्वपा॑ अ॒स्मे । \newline
55. स्वपा॒ इति॑ सु - अपाः᳚ । \newline
56. अ॒स्मे वर्चो॒ वर्चो॑ अ॒स्मे अ॒स्मे वर्चः॑ । \newline
57. अ॒स्मे इत्य॒स्मे । \newline
58. वर्चः॑ सु॒वीर्य(ग्म्॑) सु॒वीर्यं॒ ॅवर्चो॒ वर्चः॑ सु॒वीर्य᳚म् । \newline
59. सु॒वीर्य॒मिति॑ सु - वीर्य᳚म् । \newline

\textbf{Ghana Paata } \newline

1. सम् ज्योति॑षा॒ ज्योति॑षा॒ सꣳ सम् ज्योति॑षा ऽभूव मभूव॒म् ज्योति॑षा॒ सꣳ सम् ज्योति॑षा ऽभूवम् । \newline
2. ज्योति॑षा ऽभूव मभूव॒म् ज्योति॑षा॒ ज्योति॑षा ऽभूव मै॒न्द्री मै॒न्द्री म॑भूव॒म् ज्योति॑षा॒ ज्योति॑षा ऽभूव मै॒न्द्रीम् । \newline
3. अ॒भू॒व॒ मै॒न्द्री मै॒न्द्री म॑भूव मभूव मै॒न्द्री मा॒वृत॑ मा॒वृत॑ मै॒न्द्री म॑भूव मभूव मै॒न्द्री मा॒वृत᳚म् । \newline
4. ऐ॒न्द्री मा॒वृत॑ मा॒वृत॑ मै॒न्द्री मै॒न्द्री मा॒वृत॑ म॒न्वाव॑र्ते॒ ऽन्वाव॑र्त आ॒वृत॑ मै॒न्द्री मै॒न्द्री मा॒वृत॑ म॒न्वाव॑र्ते । \newline
5. आ॒वृत॑ म॒न्वाव॑र्ते॒ ऽन्वाव॑र्त आ॒वृत॑ मा॒वृत॑ म॒न्वाव॑र्ते॒ सꣳ स म॒न्वाव॑र्त आ॒वृत॑ मा॒वृत॑ म॒न्वाव॑र्ते॒ सम् । \newline
6. आ॒वृत॒मित्या᳚ - वृत᳚म् । \newline
7. अ॒न्वाव॑र्ते॒ सꣳ स म॒न्वाव॑र्ते॒ ऽन्वाव॑र्ते॒ स म॒ह म॒हꣳ स म॒न्वाव॑र्ते॒ ऽन्वाव॑र्ते॒ स म॒हम् । \newline
8. अ॒न्वाव॑र्त॒ इत्य॑नु - आव॑र्ते । \newline
9. स म॒ह म॒हꣳ सꣳ स म॒हम् प्र॒जया᳚ प्र॒जया॒ ऽहꣳ सꣳ स म॒हम् प्र॒जया᳚ । \newline
10. अ॒हम् प्र॒जया᳚ प्र॒जया॒ ऽह म॒हम् प्र॒जया॒ सꣳ सम् प्र॒जया॒ ऽह म॒हम् प्र॒जया॒ सम् । \newline
11. प्र॒जया॒ सꣳ सम् प्र॒जया᳚ प्र॒जया॒ सम् मया॒ मया॒ सम् प्र॒जया᳚ प्र॒जया॒ सम् मया᳚ । \newline
12. प्र॒जयेति॑ प्र - जया᳚ । \newline
13. सम् मया॒ मया॒ सꣳ सम् मया᳚ प्र॒जा प्र॒जा मया॒ सꣳ सम् मया᳚ प्र॒जा । \newline
14. मया᳚ प्र॒जा प्र॒जा मया॒ मया᳚ प्र॒जा सꣳ सम् प्र॒जा मया॒ मया᳚ प्र॒जा सम् । \newline
15. प्र॒जा सꣳ सम् प्र॒जा प्र॒जा स म॒ह म॒हꣳ सम् प्र॒जा प्र॒जा स म॒हम् । \newline
16. प्र॒जेति॑ प्र - जा । \newline
17. स म॒ह म॒हꣳ सꣳ स म॒हꣳ रा॒यो रा॒यो॑ ऽहꣳ सꣳ स म॒हꣳ रा॒यः । \newline
18. अ॒हꣳ रा॒यो रा॒यो॑ ऽह म॒हꣳ रा॒य स्पोषे॑ण॒ पोषे॑ण रा॒यो॑ ऽह म॒हꣳ रा॒य स्पोषे॑ण । \newline
19. रा॒य स्पोषे॑ण॒ पोषे॑ण रा॒यो रा॒य स्पोषे॑ण॒ सꣳ सम् पोषे॑ण रा॒यो रा॒य स्पोषे॑ण॒ सम् । \newline
20. पोषे॑ण॒ सꣳ सम् पोषे॑ण॒ पोषे॑ण॒ सम् मया॒ मया॒ सम् पोषे॑ण॒ पोषे॑ण॒ सम् मया᳚ । \newline
21. सम् मया॒ मया॒ सꣳ सम् मया॑ रा॒यो रा॒यो मया॒ सꣳ सम् मया॑ रा॒यः । \newline
22. मया॑ रा॒यो रा॒यो मया॒ मया॑ रा॒य स्पोषः॒ पोषो॑ रा॒यो मया॒ मया॑ रा॒य स्पोषः॑ । \newline
23. रा॒य स्पोषः॒ पोषो॑ रा॒यो रा॒य स्पोषः॒ समि॑द्धः॒ समि॑द्धः॒ पोषो॑ रा॒यो रा॒य स्पोषः॒ समि॑द्धः । \newline
24. पोषः॒ समि॑द्धः॒ समि॑द्धः॒ पोषः॒ पोषः॒ समि॑द्धो अग्ने ऽग्ने॒ समि॑द्धः॒ पोषः॒ पोषः॒ समि॑द्धो अग्ने । \newline
25. समि॑द्धो अग्ने ऽग्ने॒ समि॑द्धः॒ समि॑द्धो अग्ने मे मे अग्ने॒ समि॑द्धः॒ समि॑द्धो अग्ने मे । \newline
26. समि॑द्ध॒ इति॒ सम् - इ॒द्धः॒ । \newline
27. अ॒ग्ने॒ मे॒ मे॒ अ॒ग्ने॒ ऽग्ने॒ मे॒ दी॒दि॒हि॒ दी॒दि॒हि॒ मे॒ अ॒ग्ने॒ ऽग्ने॒ मे॒ दी॒दि॒हि॒ । \newline
28. मे॒ दी॒दि॒हि॒ दी॒दि॒हि॒ मे॒ मे॒ दी॒दि॒हि॒ स॒मे॒द्धा स॑मे॒द्धा दी॑दिहि मे मे दीदिहि समे॒द्धा । \newline
29. दी॒दि॒हि॒ स॒मे॒द्धा स॑मे॒द्धा दी॑दिहि दीदिहि समे॒द्धा ते॑ ते समे॒द्धा दी॑दिहि दीदिहि समे॒द्धा ते᳚ । \newline
30. स॒मे॒द्धा ते॑ ते समे॒द्धा स॑मे॒द्धा ते॑ अग्ने ऽग्ने ते समे॒द्धा स॑मे॒द्धा ते॑ अग्ने । \newline
31. स॒मे॒द्धेति॑ सम् - ए॒द्धा । \newline
32. ते॒ अ॒ग्ने॒ ऽग्ने॒ ते॒ ते॒ अ॒ग्ने॒ दी॒द्या॒स॒म् दी॒द्या॒स॒ म॒ग्ने॒ ते॒ ते॒ अ॒ग्ने॒ दी॒द्या॒स॒म् । \newline
33. अ॒ग्ने॒ दी॒द्या॒स॒म् दी॒द्या॒स॒ म॒ग्ने॒ ऽग्ने॒ दी॒द्या॒सं॒ ॅवसु॑मा॒न्॒. वसु॑मान् दीद्यास मग्ने ऽग्ने दीद्यासं॒ ॅवसु॑मान् । \newline
34. दी॒द्या॒सं॒ ॅवसु॑मा॒न्॒. वसु॑मान् दीद्यासम् दीद्यासं॒ ॅवसु॑मान्. य॒ज्ञो य॒ज्ञो वसु॑मान् दीद्यासम् दीद्यासं॒ ॅवसु॑मान्. य॒ज्ञ्ः । \newline
35. वसु॑मान्. य॒ज्ञो य॒ज्ञो वसु॑मा॒न्॒. वसु॑मान्. य॒ज्ञो वसी॑या॒न्॒. वसी॑यान्. य॒ज्ञो वसु॑मा॒न्॒. वसु॑मान्. य॒ज्ञो वसी॑यान् । \newline
36. वसु॑मा॒निति॒ वसु॑ - मा॒न् । \newline
37. य॒ज्ञो वसी॑या॒न्॒. वसी॑यान्. य॒ज्ञो य॒ज्ञो वसी॑यान् भूयासम् भूयासं॒ ॅवसी॑यान्. य॒ज्ञो य॒ज्ञो वसी॑यान् भूयासम् । \newline
38. वसी॑यान् भूयासम् भूयासं॒ ॅवसी॑या॒न्॒. वसी॑यान् भूयास॒ मग्ने ऽग्ने॑ भूयासं॒ ॅवसी॑या॒न्॒. वसी॑यान् भूयास॒ मग्ने᳚ । \newline
39. भू॒या॒स॒ मग्ने ऽग्ने॑ भूयासम् भूयास॒ मग्न॒ आयू॒(ग्ग्॒) ष्यायू॒(ग्ग्॒) ष्यग्ने॑ भूयासम् भूयास॒ मग्न॒ आयू(ग्म्॑)षि । \newline
40. अग्न॒ आयू॒(ग्ग्॒) ष्यायू॒(ग्ग्॒) ष्यग्ने ऽग्न॒ आयू(ग्म्॑)षि पवसे पवस॒ आयू॒(ग्ग्॒)ष्यग्ने ऽग्न॒ आयू(ग्म्॑)षि पवसे । \newline
41. आयू(ग्म्॑)षि पवसे पवस॒ आयू॒(ग्ग्॒) ष्यायू(ग्म्॑)षि पवस॒ आ प॑वस॒ आयू॒(ग्ग्॒) ष्यायू(ग्म्॑)षि पवस॒ आ । \newline
42. प॒व॒स॒ आ प॑वसे पवस॒ आ सु॑व सु॒वा प॑वसे पवस॒ आ सु॑व । \newline
43. आ सु॑व सु॒वा सु॒वोर्ज॒ मूर्ज(ग्म्॑) सु॒वा सु॒वोर्ज᳚म् । \newline
44. सु॒वोर्ज॒ मूर्ज(ग्म्॑) सुव सु॒वोर्ज॒ मिष॒ मिष॒ मूर्ज(ग्म्॑) सुव सु॒वोर्ज॒ मिष᳚म् । \newline
45. ऊर्ज॒ मिष॒ मिष॒ मूर्ज॒ मूर्ज॒ मिष॑म् च॒ चे ष॒ मूर्ज॒ मूर्ज॒ मिष॑म् च । \newline
46. इष॑म् च॒ चे ष॒ मिष॑म् च नो न॒श्चे ष॒ मिष॑म् च नः । \newline
47. च॒ नो॒ न॒श्च॒ च॒ नः॒ । \newline
48. न॒ इति॑ नः॒ । \newline
49. आ॒रे बा॑धस्व बाधस्वा॒र आ॒रे बा॑धस्व दु॒च्छुना᳚म् दु॒च्छुना᳚म् बाधस्वा॒र आ॒रे बा॑धस्व दु॒च्छुना᳚म् । \newline
50. बा॒ध॒स्व॒ दु॒च्छुना᳚म् दु॒च्छुना᳚म् बाधस्व बाधस्व दु॒च्छुना᳚म् । \newline
51. दु॒च्छुना॒मिति॑ दु॒च्छुना᳚म् । \newline
52. अग्ने॒ पव॑स्व॒ पव॒स्वाग्ने ऽग्ने॒ पव॑स्व॒ स्वपाः॒ स्वपाः॒ पव॒स्वाग्ने ऽग्ने॒ पव॑स्व॒ स्वपाः᳚ । \newline
53. पव॑स्व॒ स्वपाः॒ स्वपाः॒ पव॑स्व॒ पव॑स्व॒ स्वपा॑ अ॒स्मे अ॒स्मे स्वपाः॒ पव॑स्व॒ पव॑स्व॒ स्वपा॑ अ॒स्मे । \newline
54. स्वपा॑ अ॒स्मे अ॒स्मे स्वपाः॒ स्वपा॑ अ॒स्मे वर्चो॒ वर्चो॑ अ॒स्मे स्वपाः॒ स्वपा॑ अ॒स्मे वर्चः॑ । \newline
55. स्वपा॒ इति॑ सु - अपाः᳚ । \newline
56. अ॒स्मे वर्चो॒ वर्चो॑ अ॒स्मे अ॒स्मे वर्चः॑ सु॒वीर्य(ग्म्॑) सु॒वीर्यं॒ ॅवर्चो॑ अ॒स्मे अ॒स्मे वर्चः॑ सु॒वीर्य᳚म् । \newline
57. अ॒स्मित्य॒स्मे । \newline
58. वर्चः॑ सु॒वीर्य(ग्म्॑) सु॒वीर्यं॒ ॅवर्चो॒ वर्चः॑ सु॒वीर्य᳚म् । \newline
59. सु॒वीर्य॒मिति॑ सु - वीर्य᳚म् । \newline
\pagebreak
\markright{ TS 1.6.6.3  \hfill https://www.vedavms.in \hfill}

\section{ TS 1.6.6.3 }

\textbf{TS 1.6.6.3 } \newline
\textbf{Samhita Paata} \newline

दध॒त्पोषꣳ॑ र॒यिं मयि॑ । अग्ने॑ गृहपते सुगृहप॒तिर॒हं त्वया॑ गृ॒हप॑तिना भूयासꣳ सुगृहप॒तिर्मया॒ त्वं गृ॒हप॑तिना भूयाः श॒तꣳ हिमा॒स्तामा॒शिष॒मा शा॑से॒ तन्त॑वे॒ ज्योति॑ष्मतीं॒ तामा॒शिष॒मा शा॑से॒ऽमुष्मै॒ ज्योति॑ष्मतीं॒ कस्त्वा॑ युनक्ति॒ स त्वा॒ विमु॑ञ्च॒त्वग्ने᳚ व्रतपते व्र॒तम॑चारिषं॒ तद॑शकं॒ तन्मे॑ऽराधि य॒ज्ञो ब॑भूव॒ स आ - [ ] \newline

\textbf{Pada Paata} \newline

दध॑त् । पोष᳚म् । र॒यिम् । मयि॑ ॥ अग्ने᳚ । गृ॒ह॒प॒त॒ इति॑ गृह - प॒ते॒ । सु॒गृ॒ह॒प॒तिरिति॑ सु - गृ॒ह॒प॒तिः । अ॒हम् । त्वया᳚ । गृ॒हप॑ति॒नेति॑ गृ॒ह - प॒ति॒ना॒ । भू॒या॒स॒म् । सु॒गृ॒ह॒प॒तिरिति॑ सु – गृ॒ह॒प॒तिः  । मया᳚ । त्वम् । गृ॒हप॑ति॒नेति॑ गृ॒ह - प॒ति॒ना॒ । भू॒याः॒ । श॒तम् । हिमाः᳚ । ताम् । आ॒शिष॒मित्या᳚ - शिष᳚म् । एति॑ । शा॒से॒ । तन्त॑वे । ज्योति॑ष्मतीम् । ताम् । आ॒शिष॒मित्या᳚ - शिष᳚म् । एति॑ । शा॒से॒ । अ॒मुष्मै᳚ । ज्योति॑ष्मतीम् । कः । त्वा॒ । यु॒न॒क्ति॒ । सः । त्वा॒ । वीति॑ । मु॒ञ्च॒तु॒ । अग्ने᳚ । व्र॒त॒प॒त॒ इति॑ व्रत - प॒ते॒ । व्र॒तम् । अ॒चा॒रि॒ष॒म् । तत् । अ॒श॒क॒म् । तत् । मे॒ । अ॒रा॒धि॒ । य॒ज्ञ्ः । ब॒भू॒व॒ । सः । एति॑ ।  \newline


\textbf{Krama Paata} \newline

दध॒त्,पोष᳚म् । पोषꣳ॑ र॒यिम् । र॒यिम् मयि॑ । मयीति॒ मयि॑ ॥ अग्ने॑ गृहपते । गृ॒ह॒प॒ते॒ सु॒गृ॒ह॒प॒तिः । गृ॒ह॒प॒त॒ इति॑ गृह - प॒ते॒ । सु॒गृ॒ह॒प॒तिर॒हम् । सु॒गृ॒ह॒प॒तिरिति॑ सु - गृ॒ह॒प॒तिः । अ॒हम् त्वया᳚ । त्वया॑ गृ॒हप॑तिना । गृ॒हप॑तिना भूयासम् । गृ॒हप॑ति॒नेति॑ गृ॒ह - प॒ति॒ना॒ । भू॒या॒सꣳ॒॒ सु॒गृ॒ह॒प॒तिः । सु॒गृ॒ह॒प॒तिर्,मया᳚ । सु॒गृ॒ह॒प॒तिरिति॑ सु - गृ॒ह॒प॒तिः । मया॒ त्वम् । त्वम् गृ॒हप॑तिना । गृ॒हप॑तिना भूयाः । गृ॒हप॑ति॒नेति॑ गृ॒ह - प॒ति॒ना॒ । भू॒याः॒ श॒तम् । श॒तꣳ हिमाः᳚ । हिमा॒स्ताम् । तामा॒शिष᳚म् । आ॒शिष॒मा । आ॒शिष॒मित्या᳚ - शिष᳚म् । आ शा॑से । शा॒से॒ तन्त॑वे । तन्त॑वे॒ ज्योति॑ष्मतीम् । ज्योति॑ष्मती॒म् ताम् । तामा॒शिष᳚म् । आ॒शिष॒मा । आ॒शिष॒मित्या᳚ - शिष᳚म् । आ शा॑से । शा॒से॒ऽमुष्मै᳚ । अ॒मुष्मै॒ ज्योति॑ष्मतीम् । ज्योति॑ष्मती॒म् कः । कस्त्वा᳚ । त्वा॒ यु॒न॒क्ति॒ । यु॒न॒क्ति॒ सः । स त्वा᳚ । त्वा॒ वि । वि मु॑ञ्चतु । मु॒ञ्च॒त्वग्ने᳚ । अग्ने᳚ व्रतपते । व्र॒त॒प॒ते॒ व्र॒तम् । व्र॒त॒प॒त॒ इति॑ व्रत - प॒ते॒ । व्र॒तम॑चारिषम् । अ॒चा॒रि॒ष॒म् तत् । तद॑शकम् । अ॒श॒क॒म् तत् । तन्मे᳚ । मे॒ऽरा॒धि॒ । अ॒रा॒धि॒ य॒ज्ञ्ः । य॒ज्ञो ब॑भूव । ब॒भू॒व॒ सः । स आ । आ ब॑भूव \newline

\textbf{Jatai Paata} \newline

1. दध॒त् पोष॒म् पोष॒म् दध॒द् दध॒त् पोष᳚म् । \newline
2. पोष(ग्म्॑) र॒यिꣳ र॒यिम् पोष॒म् पोष(ग्म्॑) र॒यिम् । \newline
3. र॒यिम् मयि॒ मयि॑ र॒यिꣳ र॒यिम् मयि॑ । \newline
4. मयीति॒ मयि॑ । \newline
5. अग्ने॑ गृहपते गृहप॒ते ऽग्ने ऽग्ने॑ गृहपते । \newline
6. गृ॒ह॒प॒ते॒ सु॒गृ॒ह॒प॒तिः सु॑गृहप॒तिर् गृ॑हपते गृहपते सुगृहप॒तिः । \newline
7. गृ॒ह॒प॒त॒ इति॑ गृह - प॒ते॒ । \newline
8. सु॒गृ॒ह॒प॒ति र॒ह म॒हꣳ सु॑गृहप॒तिः सु॑गृहप॒ति र॒हम् । \newline
9. सु॒गृ॒ह॒प॒तिरिति॑ सु - गृ॒ह॒प॒तिः । \newline
10. अ॒हम् त्वया॒ त्वया॒ ऽह म॒हम् त्वया᳚ । \newline
11. त्वया॑ गृ॒हप॑तिना गृ॒हप॑तिना॒ त्वया॒ त्वया॑ गृ॒हप॑तिना । \newline
12. गृ॒हप॑तिना भूयासम् भूयासम् गृ॒हप॑तिना गृ॒हप॑तिना भूयासम् । \newline
13. गृ॒हप॑ति॒नेति॑ गृ॒ह - प॒ति॒ना॒ । \newline
14. भू॒या॒स॒(ग्म्॒) सु॒गृ॒ह॒प॒तिः सु॑गृहप॒तिर् भू॑यासम् भूयासꣳ सुगृहप॒तिः । \newline
15. सु॒गृ॒ह॒प॒तिर् मया॒ मया॑ सुगृहप॒तिः सु॑गृहप॒तिर् मया᳚ । \newline
16. सु॒गृ॒ह॒प॒तिरिति॑ सु - गृ॒ह॒प॒तिः । \newline
17. मया॒ त्वम् त्वम् मया॒ मया॒ त्वम् । \newline
18. त्वम् गृ॒हप॑तिना गृ॒हप॑तिना॒ त्वम् त्वम् गृ॒हप॑तिना । \newline
19. गृ॒हप॑तिना भूया भूया गृ॒हप॑तिना गृ॒हप॑तिना भूयाः । \newline
20. गृ॒हप॑ति॒नेति॑ गृ॒ह - प॒ति॒ना॒ । \newline
21. भू॒याः॒ श॒तꣳ श॒तम् भू॑या भूयाः श॒तम् । \newline
22. श॒तꣳ हिमा॒ हिमाः᳚ श॒तꣳ श॒तꣳ हिमाः᳚ । \newline
23. हिमा॒ स्ताम् ताꣳ हिमा॒ हिमा॒ स्ताम् । \newline
24. ता मा॒शिष॑ मा॒शिष॒म् ताम् ता मा॒शिष᳚म् । \newline
25. आ॒शिष॒ मा ऽऽशिष॑ मा॒शिष॒ मा । \newline
26. आ॒शिष॒मित्या᳚ - शिष᳚म् । \newline
27. आ शा॑से शास॒ आ शा॑से । \newline
28. शा॒से॒ तन्त॑वे॒ तन्त॑वे शासे शासे॒ तन्त॑वे । \newline
29. तन्त॑वे॒ ज्योति॑ष्मती॒म् ज्योति॑ष्मती॒म् तन्त॑वे॒ तन्त॑वे॒ ज्योति॑ष्मतीम् । \newline
30. ज्योति॑ष्मती॒म् ताम् ताम् ज्योति॑ष्मती॒म् ज्योति॑ष्मती॒म् ताम् । \newline
31. ता मा॒शिष॑ मा॒शिष॒म् ताम् ता मा॒शिष᳚म् । \newline
32. आ॒शिष॒ मा ऽऽशिष॑ मा॒शिष॒ मा । \newline
33. आ॒शिष॒मित्या᳚ - शिष᳚म् । \newline
34. आ शा॑से शास॒ आ शा॑से । \newline
35. शा॒से॒ ऽमुष्मा॑ अ॒मुष्मै॑ शासे शासे॒ ऽमुष्मै᳚ । \newline
36. अ॒मुष्मै॒ ज्योति॑ष्मती॒म् ज्योति॑ष्मती म॒मुष्मा॑ अ॒मुष्मै॒ ज्योति॑ष्मतीम् । \newline
37. ज्योति॑ष्मती॒म् कः को ज्योति॑ष्मती॒म् ज्योति॑ष्मती॒म् कः । \newline
38. क स्त्वा᳚ त्वा॒ कः क स्त्वा᳚ । \newline
39. त्वा॒ यु॒न॒क्ति॒ यु॒न॒क्ति॒ त्वा॒ त्वा॒ यु॒न॒क्ति॒ । \newline
40. यु॒न॒क्ति॒ स स यु॑नक्ति युनक्ति॒ सः । \newline
41. स त्वा᳚ त्वा॒ स स त्वा᳚ । \newline
42. त्वा॒ वि वि त्वा᳚ त्वा॒ वि । \newline
43. वि मु॑ञ्चतु मुञ्चतु॒ वि वि मु॑ञ्चतु । \newline
44. मु॒ञ्च॒ त्वग्ने ऽग्ने॑ मुञ्चतु मुञ्च॒ त्वग्ने᳚ । \newline
45. अग्ने᳚ व्रतपते व्रतप॒ते ऽग्ने ऽग्ने᳚ व्रतपते । \newline
46. व्र॒त॒प॒ते॒ व्र॒तं ॅव्र॒तं ॅव्र॑तपते व्रतपते व्र॒तम् । \newline
47. व्र॒त॒प॒त॒ इति॑ व्रत - प॒ते॒ । \newline
48. व्र॒त म॑चारिष मचारिषं ॅव्र॒तं ॅव्र॒त म॑चारिषम् । \newline
49. अ॒चा॒रि॒ष॒म् तत् तद॑चारिष मचारिष॒म् तत् । \newline
50. तद॑शक मशक॒म् तत् तद॑शकम् । \newline
51. अ॒श॒क॒म् तत् तद॑शक मशक॒म् तत् । \newline
52. तन् मे॑ मे॒ तत् तन् मे᳚ । \newline
53. मे॒ ऽरा॒ध्य॒ रा॒धि॒ मे॒ मे॒ ऽरा॒धि॒ । \newline
54. अ॒रा॒धि॒ य॒ज्ञो य॒ज्ञो॑ ऽराध्य राधि य॒ज्ञ्ः । \newline
55. य॒ज्ञो ब॑भूव बभूव य॒ज्ञो य॒ज्ञो ब॑भूव । \newline
56. ब॒भू॒व॒ स स ब॑भूव बभूव॒ सः । \newline
57. स आ स स आ । \newline
58. आ ब॑भूव बभू॒वा ब॑भूव । \newline

\textbf{Ghana Paata } \newline

1. दध॒त् पोष॒म् पोष॒म् दध॒द् दध॒त् पोष(ग्म्॑) र॒यिꣳ र॒यिम् पोष॒म् दध॒द् दध॒त् पोष(ग्म्॑) र॒यिम् । \newline
2. पोष(ग्म्॑) र॒यिꣳ र॒यिम् पोष॒म् पोष(ग्म्॑) र॒यिम् मयि॒ मयि॑ र॒यिम् पोष॒म् पोष(ग्म्॑) र॒यिम् मयि॑ । \newline
3. र॒यिम् मयि॒ मयि॑ र॒यिꣳ र॒यिम् मयि॑ । \newline
4. मयीति॒ मयि॑ । \newline
5. अग्ने॑ गृहपते गृहप॒ते ऽग्ने ऽग्ने॑ गृहपते सुगृहप॒तिः सु॑गृहप॒तिर् गृ॑हप॒ते ऽग्ने ऽग्ने॑ गृहपते सुगृहप॒तिः । \newline
6. गृ॒ह॒प॒ते॒ सु॒गृ॒ह॒प॒तिः सु॑गृहप॒तिर् गृ॑हपते गृहपते सुगृहप॒तिर॒ह म॒हꣳ सु॑गृहप॒तिर् गृ॑हपते गृहपते सुगृहप॒तिर॒हम् । \newline
7. गृ॒ह॒प॒त॒ इति॑ गृह - प॒ते॒ । \newline
8. सु॒गृ॒ह॒प॒तिर॒ह म॒हꣳ सु॑गृहप॒तिः सु॑गृहप॒तिर॒हम् त्वया॒ त्वया॒ ऽहꣳ सु॑गृहप॒तिः सु॑गृहप॒तिर॒हम् त्वया᳚ । \newline
9. सु॒गृ॒ह॒प॒तिरिति॑ सु - गृ॒ह॒प॒तिः । \newline
10. अ॒हम् त्वया॒ त्वया॒ ऽह म॒हम् त्वया॑ गृ॒हप॑तिना गृ॒हप॑तिना॒ त्वया॒ ऽह म॒हम् त्वया॑ गृ॒हप॑तिना । \newline
11. त्वया॑ गृ॒हप॑तिना गृ॒हप॑तिना॒ त्वया॒ त्वया॑ गृ॒हप॑तिना भूयासम् भूयासम् गृ॒हप॑तिना॒ त्वया॒ त्वया॑ गृ॒हप॑तिना भूयासम् । \newline
12. गृ॒हप॑तिना भूयासम् भूयासम् गृ॒हप॑तिना गृ॒हप॑तिना भूयासꣳ सुगृहप॒तिः सु॑गृहप॒तिर् भू॑यासम् गृ॒हप॑तिना गृ॒हप॑तिना भूयासꣳ सुगृहप॒तिः । \newline
13. गृ॒हप॑ति॒नेति॑ गृ॒ह - प॒ति॒ना॒ । \newline
14. भू॒या॒स॒(ग्म्॒) सु॒गृ॒ह॒प॒तिः सु॑गृहप॒तिर् भू॑यासम् भूयासꣳ सुगृहप॒तिर् मया॒ मया॑ सुगृहप॒तिर् भू॑यासम् भूयासꣳ सुगृहप॒तिर् मया᳚ । \newline
15. सु॒गृ॒ह॒प॒तिर् मया॒ मया॑ सुगृहप॒तिः सु॑गृहप॒तिर् मया॒ त्वम् त्वम् मया॑ सुगृहप॒तिः सु॑गृहप॒तिर् मया॒ त्वम् । \newline
16. सु॒गृ॒ह॒प॒तिरिति॑ सु - गृ॒ह॒प॒तिः । \newline
17. मया॒ त्वम् त्वम् मया॒ मया॒ त्वम् गृ॒हप॑तिना गृ॒हप॑तिना॒ त्वम् मया॒ मया॒ त्वम् गृ॒हप॑तिना । \newline
18. त्वम् गृ॒हप॑तिना गृ॒हप॑तिना॒ त्वम् त्वम् गृ॒हप॑तिना भूया भूया गृ॒हप॑तिना॒ त्वम् त्वम् गृ॒हप॑तिना भूयाः । \newline
19. गृ॒हप॑तिना भूया भूया गृ॒हप॑तिना गृ॒हप॑तिना भूयाः श॒तꣳ श॒तम् भू॑या गृ॒हप॑तिना गृ॒हप॑तिना भूयाः श॒तम् । \newline
20. गृ॒हप॑ति॒नेति॑ गृ॒ह - प॒ति॒ना॒ । \newline
21. भू॒याः॒ श॒तꣳ श॒तम् भू॑या भूयाः श॒तꣳ हिमा॒ हिमाः᳚ श॒तम् भू॑या भूयाः श॒तꣳ हिमाः᳚ । \newline
22. श॒तꣳ हिमा॒ हिमाः᳚ श॒तꣳ श॒तꣳ हिमा॒स्ताम् ताꣳ हिमाः᳚ श॒तꣳ श॒तꣳ हिमा॒स्ताम् । \newline
23. हिमा॒स्ताम् ताꣳ हिमा॒ हिमा॒स्ता मा॒शिष॑ मा॒शिष॒म् ताꣳ हिमा॒ हिमा॒स्ता मा॒शिष᳚म् । \newline
24. ता मा॒शिष॑ मा॒शिष॒म् ताम् ता मा॒शिष॒ मा ऽऽशिष॒म् ताम् ता मा॒शिष॒ मा । \newline
25. आ॒शिष॒ मा ऽऽशिष॑ मा॒शिष॒ मा शा॑से शास॒ आ ऽऽशिष॑ मा॒शिष॒ मा शा॑से । \newline
26. आ॒शिष॒मित्या᳚ - शिष᳚म् । \newline
27. आ शा॑से शास॒ आ शा॑से॒ तन्त॑वे॒ तन्त॑वे शास॒ आ शा॑से॒ तन्त॑वे । \newline
28. शा॒से॒ तन्त॑वे॒ तन्त॑वे शासे शासे॒ तन्त॑वे॒ ज्योति॑ष्मती॒म् ज्योति॑ष्मती॒म् तन्त॑वे शासे शासे॒ तन्त॑वे॒ ज्योति॑ष्मतीम् । \newline
29. तन्त॑वे॒ ज्योति॑ष्मती॒म् ज्योति॑ष्मती॒म् तन्त॑वे॒ तन्त॑वे॒ ज्योति॑ष्मती॒म् ताम् ताम् ज्योति॑ष्मती॒म् तन्त॑वे॒ तन्त॑वे॒ ज्योति॑ष्मती॒म् ताम् । \newline
30. ज्योति॑ष्मती॒म् ताम् ताम् ज्योति॑ष्मती॒म् ज्योति॑ष्मती॒म् ता मा॒शिष॑ मा॒शिष॒म् ताम् ज्योति॑ष्मती॒म् ज्योति॑ष्मती॒म् ता मा॒शिष᳚म् । \newline
31. ता मा॒शिष॑ मा॒शिष॒म् ताम् ता मा॒शिष॒ मा ऽऽशिष॒म् ताम् ता मा॒शिष॒ मा । \newline
32. आ॒शिष॒ मा ऽऽशिष॑ मा॒शिष॒ मा शा॑से शास॒ आ ऽऽशिष॑ मा॒शिष॒ मा शा॑से । \newline
33. आ॒शिष॒मित्या᳚ - शिष᳚म् । \newline
34. आ शा॑से शास॒ आ शा॑से॒ ऽमुष्मा॑ अ॒मुष्मै॑ शास॒ आ शा॑से॒ ऽमुष्मै᳚ । \newline
35. शा॒से॒ ऽमुष्मा॑ अ॒मुष्मै॑ शासे शासे॒ ऽमुष्मै॒ ज्योति॑ष्मती॒म् ज्योति॑ष्मती म॒मुष्मै॑ शासे शासे॒ ऽमुष्मै॒ ज्योति॑ष्मतीम् । \newline
36. अ॒मुष्मै॒ ज्योति॑ष्मती॒म् ज्योति॑ष्मती म॒मुष्मा॑ अ॒मुष्मै॒ ज्योति॑ष्मती॒म् कः को ज्योति॑ष्मती म॒मुष्मा॑ अ॒मुष्मै॒ ज्योति॑ष्मती॒म् कः । \newline
37. ज्योति॑ष्मती॒म् कः को ज्योति॑ष्मती॒म् ज्योति॑ष्मती॒म् कस्त्वा᳚ त्वा॒ को ज्योति॑ष्मती॒म् ज्योति॑ष्मती॒म् कस्त्वा᳚ । \newline
38. कस्त्वा᳚ त्वा॒ कः कस्त्वा॑ युनक्ति युनक्ति त्वा॒ कः कस्त्वा॑ युनक्ति । \newline
39. त्वा॒ यु॒न॒क्ति॒ यु॒न॒क्ति॒ त्वा॒ त्वा॒ यु॒न॒क्ति॒ स स यु॑नक्ति त्वा त्वा युनक्ति॒ सः । \newline
40. यु॒न॒क्ति॒ स स यु॑नक्ति युनक्ति॒ स त्वा᳚ त्वा॒ स यु॑नक्ति युनक्ति॒ स त्वा᳚ । \newline
41. स त्वा᳚ त्वा॒ स स त्वा॒ वि वि त्वा॒ स स त्वा॒ वि । \newline
42. त्वा॒ वि वि त्वा᳚ त्वा॒ वि मु॑ञ्चतु मुञ्चतु॒ वि त्वा᳚ त्वा॒ वि मु॑ञ्चतु । \newline
43. वि मु॑ञ्चतु मुञ्चतु॒ वि वि मु॑ञ्च॒त्वग्ने ऽग्ने॑ मुञ्चतु॒ वि वि मु॑ञ्च॒त्वग्ने᳚ । \newline
44. मु॒ञ्च॒त्वग्ने ऽग्ने॑ मुञ्चतु मुञ्च॒त्वग्ने᳚ व्रतपते व्रतप॒ते ऽग्ने॑ मुञ्चतु मुञ्च॒त्वग्ने᳚ व्रतपते । \newline
45. अग्ने᳚ व्रतपते व्रतप॒ते ऽग्ने ऽग्ने᳚ व्रतपते व्र॒तं ॅव्र॒तं ॅव्र॑तप॒ते ऽग्ने ऽग्ने᳚ व्रतपते व्र॒तम् । \newline
46. व्र॒त॒प॒ते॒ व्र॒तं ॅव्र॒तं ॅव्र॑तपते व्रतपते व्र॒त म॑चारिष मचारिषं ॅव्र॒तं ॅव्र॑तपते व्रतपते व्र॒त म॑चारिषम् । \newline
47. व्र॒त॒प॒त॒ इति॑ व्रत - प॒ते॒ । \newline
48. व्र॒त म॑चारिष मचारिषं ॅव्र॒तं ॅव्र॒त म॑चारिष॒म् तत् तद॑चारिषं ॅव्र॒तं ॅव्र॒त म॑चारिष॒म् तत् । \newline
49. अ॒चा॒रि॒ष॒म् तत् तद॑चारिष मचारिष॒म् तद॑शक मशक॒म् तद॑चारिष मचारिष॒म् तद॑शकम् । \newline
50. तद॑शक मशक॒म् तत् तद॑शक॒म् तत् तद॑शक॒म् तत् तद॑शक॒म् तत् । \newline
51. अ॒श॒क॒म् तत् तद॑शक मशक॒म् तन् मे॑ मे॒ तद॑शक मशक॒म् तन् मे᳚ । \newline
52. तन् मे॑ मे॒ तत् तन् मे॑ ऽराध्यराधि मे॒ तत् तन् मे॑ ऽराधि । \newline
53. मे॒ ऽरा॒ध्य॒रा॒धि॒ मे॒ मे॒ ऽरा॒धि॒ य॒ज्ञो य॒ज्ञो॑ ऽराधि मे मे ऽराधि य॒ज्ञ्ः । \newline
54. अ॒रा॒धि॒ य॒ज्ञो य॒ज्ञो॑ ऽराध्यराधि य॒ज्ञो ब॑भूव बभूव य॒ज्ञो॑ ऽराध्यराधि य॒ज्ञो ब॑भूव । \newline
55. य॒ज्ञो ब॑भूव बभूव य॒ज्ञो य॒ज्ञो ब॑भूव॒ स स ब॑भूव य॒ज्ञो य॒ज्ञो ब॑भूव॒ सः । \newline
56. ब॒भू॒व॒ स स ब॑भूव बभूव॒ स आ स ब॑भूव बभूव॒ स आ । \newline
57. स आ स स आ ब॑भूव बभू॒वा स स आ ब॑भूव । \newline
58. आ ब॑भूव बभू॒वा ब॑भूव॒ स स ब॑भू॒वा ब॑भूव॒ सः । \newline
\pagebreak
\markright{ TS 1.6.6.4  \hfill https://www.vedavms.in \hfill}

\section{ TS 1.6.6.4 }

\textbf{TS 1.6.6.4 } \newline
\textbf{Samhita Paata} \newline

ब॑भूव॒ स प्रज॑ज्ञे॒ स वा॑वृधे । स दे॒वाना॒मधि॑पतिर् बभूव॒ सो अ॒स्माꣳ अधि॑पतीन् करोतु व॒यꣳ स्या॑म॒ पत॑यो रयी॒णां ॥ गोमाꣳ॑ अ॒ग्नेऽवि॑माꣳ अ॒श्वी य॒ज्ञो नृ॒वथ्स॑खा॒ सद॒मिद॑प्रमृ॒ष्यः । इडा॑वाꣳ ए॒षो अ॑सुर प्र॒जावा᳚न् दी॒र्घो र॒यिः पृ॑थुबु॒द्ध्नः स॒भावान्॑ ॥ \newline

\textbf{Pada Paata} \newline

ब॒भू॒व॒ । सः । प्रेति॑ । ज॒ज्ञे॒ । सः । वा॒वृ॒धे॒ ॥ सः । दे॒वाना᳚म् । अधि॑पति॒रित्यधि॑ - प॒तिः॒ । ब॒भू॒व॒ । सः । अ॒स्मान् । अधि॑पती॒नित्यधि॑ - प॒ती॒न् । क॒रो॒तु॒ । व॒यम् । स्या॒म॒ । पत॑यः । र॒यी॒णाम् ॥ गोमा॒निति॒ गो - मा॒न् । अ॒ग्ने॒ । अवि॑मा॒नित्यवि॑ - मा॒न् । अ॒श्वी । य॒ज्ञ्ः । नृ॒वथ्स॒खेति॑ नृ॒वत् - स॒खा॒ । सद᳚म् । इत् । अ॒प्र॒मृ॒ष्य इत्य॑प्र - मृ॒ष्यः ॥ इडा॑वा॒नितीडा᳚ - वा॒न् । ए॒षः । अ॒सु॒र॒ । प्र॒जावा॒निति॑ प्र॒जा - वा॒न् । दी॒र्घः । र॒यिः । पृ॒थु॒बु॒द्ध्न इति॑ पृथु - बु॒द्ध्नः । स॒भावा॒निति॑ स॒भा - वा॒न् ॥  \newline


\textbf{Krama Paata} \newline

ब॒भू॒व॒ सः । स प्र । प्र ज॑ज्ञे । ज॒ज्ञे॒ सः । स वा॑वृधे । वा॒वृ॒ध॒ इति॑ वावृधे ॥ स दे॒वाना᳚म् । दे॒वाना॒मधि॑पतिः । अधि॑पतिर्,बभूव । अधि॑पति॒रित्यधि॑ - प॒तिः॒ । ब॒भू॒व॒ सः । सो अ॒स्मान् । अ॒स्माꣳ अधि॑पतीन् । अधि॑पतीन् करोतु । अधि॑पती॒नित्यधि॑ - प॒ती॒न्॒ । क॒रो॒तु॒ व॒यम् । व॒यꣳ स्या॑म । स्या॒म॒ पत॑यः । पत॑यो रयी॒णाम् । र॒यी॒णामिति॑ रयी॒णाम् ॥ गोमाꣳ॑ अग्ने । गोमा॒निति॒ गो - मा॒न्॒ । अ॒ग्नेऽवि॑मान् । अवि॑माꣳ अ॒श्वी । अवि॑मा॒नित्यवि॑ - मा॒न्॒ । अ॒श्वी य॒ज्ञ्ः । य॒ज्ञो नृ॒वथ्स॑खा । नृ॒वथ्स॑खा॒ सद᳚म् । नृ॒वथ्स॒खेति॑ नृ॒वत् - स॒खा॒ । सद॒मित् । इद॑प्रमृ॒ष्यः । अ॒प्र॒मृ॒ष्य इत्य॑प्र - मृ॒ष्यः ॥ इडा॑वाꣳ ए॒षः । इडा॑वा॒नितीडा᳚ - वा॒न्॒ । ए॒षो अ॑सुर । अ॒सु॒र॒ प्र॒जावान्॑ । प्र॒जावा᳚न् दी॒र्घः । प्र॒जावा॒निति॑ प्र॒जा - वा॒न्॒ । दी॒र्घो र॒यिः । र॒यिः पृ॑थुबु॒द्ध्नः । पृ॒थु॒बु॒द्ध्नः स॒भावान्॑ । पृ॒थु॒बु॒द्ध्न इति॑ पृथु - बु॒द्ध्नः । स॒भावा॒निति॑ स॒भा - वा॒न्॒ । \newline

\textbf{Jatai Paata} \newline

1. ब॒भू॒व॒ स स ब॑भूव बभूव॒ सः । \newline
2. स प्र प्र स स प्र । \newline
3. प्र ज॑ज्ञे जज्ञे॒ प्र प्र ज॑ज्ञे । \newline
4. ज॒ज्ञे॒ स स ज॑ज्ञे जज्ञे॒ सः । \newline
5. स वा॑वृधे वावृधे॒ स स वा॑वृधे । \newline
6. वा॒वृ॒ध॒ इति॑ वावृधे । \newline
7. स दे॒वाना᳚म् दे॒वाना॒(ग्म्॒) स स दे॒वाना᳚म् । \newline
8. दे॒वाना॒ मधि॑पति॒ रधि॑पतिर् दे॒वाना᳚म् दे॒वाना॒ मधि॑पतिः । \newline
9. अधि॑पतिर् बभूव बभू॒वाधि॑पति॒ रधि॑पतिर् बभूव । \newline
10. अधि॑पति॒रित्यधि॑ - प॒तिः॒ । \newline
11. ब॒भू॒व॒ स स ब॑भूव बभूव॒ सः । \newline
12. सो अ॒स्माꣳ अ॒स्मान् थ्स सो अ॒स्मान् । \newline
13. अ॒स्माꣳ अधि॑पती॒ नधि॑पती न॒स्माꣳ अ॒स्माꣳ अधि॑पतीन् । \newline
14. अधि॑पतीन् करोतु करो॒ त्वधि॑पती॒ नधि॑पतीन् करोतु । \newline
15. अधि॑पती॒नित्यधि॑ - प॒ती॒न् । \newline
16. क॒रो॒तु॒ व॒यं ॅव॒यम् क॑रोतु करोतु व॒यम् । \newline
17. व॒यꣳ स्या॑म स्याम व॒यं ॅव॒यꣳ स्या॑म । \newline
18. स्या॒म॒ पत॑यः॒ पत॑यः स्याम स्याम॒ पत॑यः । \newline
19. पत॑यो रयी॒णाꣳ र॑यी॒णाम् पत॑यः॒ पत॑यो रयी॒णाम् । \newline
20. र॒यी॒णामिति॑ रयी॒णाम् । \newline
21. गोमा(ग्म्॑) अग्ने ऽग्ने॒ गोमा॒न् गोमा(ग्म्॑) अग्ने । \newline
22. गोमा॒निति॒ गो - मा॒न् । \newline
23. अ॒ग्ने ऽवि॑मा॒(ग्म्॒) अवि॑माꣳ अग्ने॒ ऽग्ने ऽवि॑मान् । \newline
24. अवि॑माꣳ अ॒श्व्य॑ श्व्यवि॑मा॒(ग्म्॒) अवि॑माꣳ अ॒श्वी । \newline
25. अवि॑मा॒नित्यवि॑ - मा॒न् । \newline
26. अ॒श्वी य॒ज्ञो य॒ज्ञो᳚(1॒) ऽश्व्य॑श्वी य॒ज्ञ्ः । \newline
27. य॒ज्ञो नृ॒वथ्स॑खा नृ॒वथ्स॑खा य॒ज्ञो य॒ज्ञो नृ॒वथ्स॑खा । \newline
28. नृ॒वथ्स॑खा॒ सद॒(ग्म्॒) सद॑म् नृ॒वथ्स॑खा नृ॒वथ्स॑खा॒ सद᳚म् । \newline
29. नृ॒वथ्स॒खेति॑ नृ॒वत् - स॒खा॒ । \newline
30. सद॒ मिदिथ् सद॒(ग्म्॒) सद॒ मित् । \newline
31. इद॑प्रमृ॒ष्यो᳚ ऽप्रमृ॒ष्य इदि द॑प्रमृ॒ष्यः । \newline
32. अ॒प्र॒मृ॒ष्य इत्य॑प्र - मृ॒ष्यः । \newline
33. इडा॑वाꣳ ए॒ष ए॒ष इडा॑वा॒(ग्म्॒) इडा॑वाꣳ ए॒षः । \newline
34. इडा॑वा॒नितीडा᳚ - वा॒न् । \newline
35. ए॒षो असु॑रा सुरै॒ष ए॒षो असु॑र । \newline
36. अ॒सु॒र॒ प्र॒जावा᳚न् प्र॒जावा॑ नसुरासुर प्र॒जावान्॑ । \newline
37. प्र॒जावा᳚न् दी॒र्घो दी॒र्घः प्र॒जावा᳚न् प्र॒जावा᳚न् दी॒र्घः । \newline
38. प्र॒जावा॒निति॑ प्र॒जा - वा॒न् । \newline
39. दी॒र्घो र॒यी र॒यिर् दी॒र्घो दी॒र्घो र॒यिः । \newline
40. र॒यिः पृ॑थुबु॒द्ध्नः पृ॑थुबु॒द्ध्नो र॒यी र॒यिः पृ॑थुबु॒द्ध्नः । \newline
41. पृ॒थु॒बु॒द्ध्नः स॒भावा᳚न् थ्स॒भावा᳚न् पृथुबु॒द्ध्नः पृ॑थुबु॒द्ध्नः स॒भावान्॑ । \newline
42. पृ॒थु॒बु॒द्ध्न इति॑ पृथु - बु॒द्ध्नः । \newline
43. स॒भावा॒निति॑ स॒भा - वा॒न् । \newline

\textbf{Ghana Paata } \newline

1. ब॒भू॒व॒ स स ब॑भूव बभूव॒ स प्र प्र स ब॑भूव बभूव॒ स प्र । \newline
2. स प्र प्र स स प्र ज॑ज्ञे जज्ञे॒ प्र स स प्र ज॑ज्ञे । \newline
3. प्र ज॑ज्ञे जज्ञे॒ प्र प्र ज॑ज्ञे॒ स स ज॑ज्ञे॒ प्र प्र ज॑ज्ञे॒ सः । \newline
4. ज॒ज्ञे॒ स स ज॑ज्ञे जज्ञे॒ स वा॑वृधे वावृधे॒ स ज॑ज्ञे जज्ञे॒ स वा॑वृधे । \newline
5. स वा॑वृधे वावृधे॒ स स वा॑वृधे । \newline
6. वा॒वृ॒ध॒ इति॑ वावृधे । \newline
7. स दे॒वाना᳚म् दे॒वाना॒(ग्म्॒) स स दे॒वाना॒ मधि॑पति॒ रधि॑पतिर् दे॒वाना॒(ग्म्॒) स स दे॒वाना॒ मधि॑पतिः । \newline
8. दे॒वाना॒ मधि॑पति॒ रधि॑पतिर् दे॒वाना᳚म् दे॒वाना॒ मधि॑पतिर् बभूव बभू॒वाधि॑पतिर् दे॒वाना᳚म् दे॒वाना॒ मधि॑पतिर् बभूव । \newline
9. अधि॑पतिर् बभूव बभू॒वाधि॑पति॒ रधि॑पतिर् बभूव॒ स स ब॑भू॒वाधि॑पति॒ रधि॑पतिर् बभूव॒ सः । \newline
10. अधि॑पति॒रित्यधि॑ - प॒तिः॒ । \newline
11. ब॒भू॒व॒ स स ब॑भूव बभूव॒ सो अ॒स्माꣳ अ॒स्मान् थ्स ब॑भूव बभूव॒ सो अ॒स्मान् । \newline
12. सो अ॒स्माꣳ अ॒स्मान् थ्स सो अ॒स्माꣳ अधि॑पती॒ नधि॑पती न॒स्मान् थ्स सो अ॒स्माꣳ अधि॑पतीन् । \newline
13. अ॒स्माꣳ अधि॑पती॒ नधि॑पती न॒स्माꣳ अ॒स्माꣳ अधि॑पतीन् करोतु करो॒त्वधि॑पती न॒स्माꣳ अ॒स्माꣳ अधि॑पतीन् करोतु । \newline
14. अधि॑पतीन् करोतु करो॒त्वधि॑पती॒ नधि॑पतीन् करोतु व॒यं ॅव॒यम् क॑रो॒त्वधि॑पती॒ नधि॑पतीन् करोतु व॒यम् । \newline
15. अधि॑पती॒नित्यधि॑ - प॒ती॒न् । \newline
16. क॒रो॒तु॒ व॒यं ॅव॒यम् क॑रोतु करोतु व॒यꣳ स्या॑म स्याम व॒यम् क॑रोतु करोतु व॒यꣳ स्या॑म । \newline
17. व॒यꣳ स्या॑म स्याम व॒यं ॅव॒यꣳ स्या॑म॒ पत॑यः॒ पत॑यः स्याम व॒यं ॅव॒यꣳ स्या॑म॒ पत॑यः । \newline
18. स्या॒म॒ पत॑यः॒ पत॑यः स्याम स्याम॒ पत॑यो रयी॒णाꣳ र॑यी॒णाम् पत॑यः स्याम स्याम॒ पत॑यो रयी॒णाम् । \newline
19. पत॑यो रयी॒णाꣳ र॑यी॒णाम् पत॑यः॒ पत॑यो रयी॒णाम् । \newline
20. र॒यी॒णामिति॑ रयी॒णाम् । \newline
21. गोमा(ग्म्॑) अग्ने ऽग्ने॒ गोमा॒न् गोमा(ग्म्॑) अ॒ग्ने ऽवि॑मा॒(ग्म्॒) अवि॑माꣳ अग्ने॒ गोमा॒न् गोमा(ग्म्॑) अ॒ग्ने ऽवि॑मान् । \newline
22. गोमा॒निति॒ गो - मा॒न् । \newline
23. अ॒ग्ने ऽवि॑मा॒(ग्म्॒) अवि॑माꣳ अग्ने॒ ऽग्ने ऽवि॑माꣳ अ॒श्व्य॑श्व्यवि॑माꣳ अग्ने॒ ऽग्ने ऽवि॑माꣳ अ॒श्वी । \newline
24. अवि॑माꣳ अ॒श्व्य॑श्व्यवि॑मा॒(ग्म्॒) अवि॑माꣳ अ॒श्वी य॒ज्ञो य॒ज्ञो᳚ ऽश्व्यवि॑मा॒(ग्म्॒) अवि॑माꣳ अ॒श्वी य॒ज्ञ्ः । \newline
25. अवि॑मा॒नित्यवि॑ - मा॒न् । \newline
26. अ॒श्वी य॒ज्ञो य॒ज्ञो᳚(1॒) ऽश्व्य॑श्वी य॒ज्ञो नृ॒वथ्स॑खा नृ॒वथ्स॑खा य॒ज्ञो᳚(1॒) ऽश्व्य॑श्वी य॒ज्ञो नृ॒वथ्स॑खा । \newline
27. य॒ज्ञो नृ॒वथ्स॑खा नृ॒वथ्स॑खा य॒ज्ञो य॒ज्ञो नृ॒वथ्स॑खा॒ सद॒(ग्म्॒) सद॑न्नृ॒वथ्स॑खा य॒ज्ञो य॒ज्ञो नृ॒वथ्स॑खा॒ सद᳚म् । \newline
28. नृ॒वथ्स॑खा॒ सद॒(ग्म्॒) सद॑न्नृ॒वथ्स॑खा नृ॒वथ्स॑खा॒ सद॒ मिदिथ् सद॑न्नृ॒वथ्स॑खा नृ॒वथ्स॑खा॒ सद॒ मित् । \newline
29. नृ॒वथ्स॒खेति॑ नृ॒वत् - स॒खा॒ । \newline
30. सद॒ मिदिथ् सद॒(ग्म्॒) सद॒ मिद॑प्रमृ॒ष्यो᳚ ऽप्रमृ॒ष्य इथ् सद॒(ग्म्॒) सद॒ मिद॑प्रमृ॒ष्यः । \newline
31. इद॑प्रमृ॒ष्यो᳚ ऽप्रमृ॒ष्य इदिद॑प्रमृ॒ष्यः । \newline
32. अ॒प्र॒मृ॒ष्य इत्य॑प्र - मृ॒ष्यः । \newline
33. इडा॑वाꣳ ए॒ष ए॒ष इडा॑वा॒(ग्म्॒) इडा॑वाꣳ ए॒षो अ॑सुरासुरै॒ष इडा॑वा॒(ग्म्॒) इडा॑वाꣳ ए॒षो अ॑सुर । \newline
34. इडा॑वा॒नितीडा᳚ - वा॒न् । \newline
35. ए॒षो अ॑सुरासुरै॒ष ए॒षो अ॑सुर प्र॒जावा᳚न् प्र॒जावा॑ नसुरै॒ष ए॒षो अ॑सुर प्र॒जावान्॑ । \newline
36. अ॒सु॒र॒ प्र॒जावा᳚न् प्र॒जावा॑ नसुरासुर प्र॒जावा᳚न् दी॒र्घो दी॒र्घः प्र॒जावा॑ नसुरासुर प्र॒जावा᳚न् दी॒र्घः । \newline
37. प्र॒जावा᳚न् दी॒र्घो दी॒र्घः प्र॒जावा᳚न् प्र॒जावा᳚न् दी॒र्घो र॒यी र॒यिर् दी॒र्घः प्र॒जावा᳚न् प्र॒जावा᳚न् दी॒र्घो र॒यिः । \newline
38. प्र॒जावा॒निति॑ प्र॒जा - वा॒न् । \newline
39. दी॒र्घो र॒यी र॒यिर् दी॒र्घो दी॒र्घो र॒यिः पृ॑थुबु॒द्ध्नः पृ॑थुबु॒द्ध्नो र॒यिर् दी॒र्घो दी॒र्घो र॒यिः पृ॑थुबु॒द्ध्नः । \newline
40. र॒यिः पृ॑थुबु॒द्ध्नः पृ॑थुबु॒द्ध्नो र॒यी र॒यिः पृ॑थुबु॒द्ध्नः स॒भावा᳚न् थ्स॒भावा᳚न् पृथुबु॒द्ध्नो र॒यी र॒यिः पृ॑थुबु॒द्ध्नः स॒भावान्॑ । \newline
41. पृ॒थु॒बु॒द्ध्नः स॒भावा᳚न् थ्स॒भावा᳚न् पृथुबु॒द्ध्नः पृ॑थुबु॒द्ध्नः स॒भावान्॑ । \newline
42. पृ॒थु॒बु॒द्ध्न इति॑ पृथु - बु॒द्ध्नः । \newline
43. स॒भावा॒निति॑ स॒भा - वा॒न् । \newline
\pagebreak
\markright{ TS 1.6.7.1  \hfill https://www.vedavms.in \hfill}

\section{ TS 1.6.7.1 }

\textbf{TS 1.6.7.1 } \newline
\textbf{Samhita Paata} \newline

यथा॒ वै स॑मृतसो॒मा ए॒वं ॅवा ए॒ते स॑मृतय॒ज्ञा यद्द॑र्.शपूर्णमा॒सौ कस्य॒ वाऽह॑ दे॒वा य॒ज्ञ्मा॒गच्छ॑न्ति॒ कस्य॑ वा॒ न ब॑हू॒नां ॅयज॑मानानां॒ ॅयो वै दे॒वताः॒ पूर्वः॑ परिगृ॒ह्णाति॒ स ए॑नाः॒ श्वो भू॒ते य॑जत ए॒तद्वै दे॒वाना॑-मा॒यत॑नं॒ ॅयदा॑हव॒नीयो᳚ऽन्त॒राऽग्नी प॑शू॒नां गार्.ह॑पत्यो मनु॒ष्या॑णा-मन्वाहार्य॒पच॑नः पितृ॒णाम॒ग्निं गृ॑ह्णाति॒ स्व ए॒वायत॑ने दे॒वताः॒ परि॑ - [ ] \newline

\textbf{Pada Paata} \newline

यथा᳚ । वै । स॒मृ॒त॒सो॒मा इति॑ समृत - सो॒माः । ए॒वम् । वै । ए॒ते । स॒मृ॒त॒य॒ज्ञा इति॑ समृत - य॒ज्ञाः । यत् । द॒र्॒.श॒पू॒र्ण॒मा॒साविति॑ दर्.श - पू॒र्ण॒मा॒सौ । कस्य॑ । वा॒ । अह॑ । दे॒वाः । य॒ज्ञ्म् । आ॒गच्छ॒न्तीत्या᳚ - गच्छ॑न्ति । कस्य॑ । वा॒ । न । ब॒हू॒नाम् । यज॑मानानाम् । यः । वै । दे॒वताः᳚ । पूर्वः॑ । प॒रि॒गृ॒ह्णातीति॑ परि - गृ॒ह्णाति॑ । सः । ए॒नाः॒ । श्वः । भू॒ते । य॒ज॒ते॒ । ए॒तत् । वै । दे॒वाना᳚म् । आ॒यत॑न॒मित्या᳚ - यत॑नम् । यत् । आ॒ह॒व॒नीय॒ इत्या᳚ - ह॒व॒नीयः॑ । अ॒न्त॒रा । अ॒ग्नी इति॑ । प॒शू॒नाम् । गार्.ह॑पत्य॒ इति॒ गार्.ह॑ - प॒त्यः॒ । म॒नु॒ष्या॑णाम् । अ॒न्वा॒हा॒र्य॒पच॑न॒ इत्य॑न्वाहार्य - पच॑नः । पि॒तृ॒णाम् । अ॒ग्निम् । गृ॒ह्णा॒ति॒ । स्वे । ए॒व । आ॒यत॑न॒ इत्या᳚ - यत॑ने । दे॒वताः᳚ । परीति॑ ।  \newline


\textbf{Krama Paata} \newline

यथा॒ वै । वै स॑मृतसो॒माः । स॒मृ॒त॒सो॒मा ए॒वम् । स॒मृ॒त॒सो॒मा इति॑ समृत - सो॒माः । ए॒वं ॅवै । वा ए॒ते । ए॒ते स॑मृतय॒ज्ञाः । स॒मृ॒त॒य॒ज्ञा यत् । स॒मृ॒त॒य॒ज्ञा इति॑ समृत - य॒ज्ञाः । यद् द॑र्.शपूर्णमा॒सौ । द॒र्.॒श॒पू॒र्ण॒मा॒सौ कस्य॑ । द॒र्.॒श॒पू॒र्ण॒मा॒साविति॑ दर्.श - पू॒र्ण॒मा॒सौ । कस्य॑ वा । वाऽह॑ । अह॑ दे॒वाः । दे॒वा य॒ज्ञ्म् । य॒ज्ञ्मा॒गच्छ॑न्ति । आ॒गच्छ॑न्ति॒ कस्य॑ । आ॒गच्छ॒न्तीत्या᳚ - गच्छ॑न्ति । कस्य॑ वा । वा॒ न । न ब॑हू॒नाम् । ब॒हू॒नां ॅयज॑मानानाम् । यज॑मानानां॒ ॅयः । यो वै । वै दे॒वताः᳚ । दे॒वताः॒ पूर्वः॑ । पूर्वः॑ परिगृ॒ह्णाति॑ । प॒रि॒गृ॒ह्णाति॒ सः । प॒रि॒गृ॒ह्णातीति॑ परि - गृ॒ह्णाति॑ ॥ स ए॑नाः । ए॒नाः॒ श्वः । श्वो भू॒ते । भू॒ते य॑जते । य॒ज॒त॒ ए॒तत् । ए॒तद् वै । वै दे॒वाना᳚म् । दे॒वाना॑मा॒यत॑नम् । आ॒यत॑नं॒ ॅयत् । आ॒यत॑न॒मित्या᳚ - यत॑नम् । यदा॑हव॒नीयः॑ । आ॒ह॒व॒नीयो᳚न्त॒रा । आ॒ह॒व॒नीय॒ इत्या᳚ - ह॒व॒नीयः॑ । अ॒न्त॒राऽग्नी । अ॒ग्नी प॑शू॒नाम् । अ॒ग्नी इत्य॒ग्नी । प॒शू॒नाम् गार्.ह॑पत्यः । गार्.ह॑पत्यो मनु॒ष्या॑णाम् । गार्.ह॑पत्य॒ इति॒ गार्.ह॑ - प॒त्यः॒ । म॒नु॒ष्या॑णामन्वाहार्य॒पच॑नः । अ॒न्वा॒हा॒र्य॒पच॑नः पितृ॒णाम् । अ॒न्वा॒हा॒र्य॒पच॑न॒ इत्य॑न्वाहार्य - पच॑नः । पि॒तृ॒णाम॒ग्निम् । अ॒ग्निम् गृ॑ह्णाति । गृ॒ह्णा॒ति॒ स्वे । स्व ए॒व । ए॒वायत॑ने । आ॒यत॑ने दे॒वताः᳚ । आ॒यत॑न॒ इत्या᳚ - यत॑ने । दे॒वताः॒ परि॑ । परि॑ गृह्णाति \newline

\textbf{Jatai Paata} \newline

1. यथा॒ वै वै यथा॒ यथा॒ वै । \newline
2. वै स॑मृतसो॒माः स॑मृतसो॒मा वै वै स॑मृतसो॒माः । \newline
3. स॒मृ॒त॒सो॒मा ए॒व मे॒वꣳ स॑मृतसो॒माः स॑मृतसो॒मा ए॒वम् । \newline
4. स॒मृ॒त॒सो॒मा इति॑ समृत - सो॒माः । \newline
5. ए॒वं ॅवै वा ए॒व मे॒वं ॅवै । \newline
6. वा ए॒त ए॒ते वै वा ए॒ते । \newline
7. ए॒ते स॑मृतय॒ज्ञाः स॑मृतय॒ज्ञा ए॒त ए॒ते स॑मृतय॒ज्ञाः । \newline
8. स॒मृ॒त॒य॒ज्ञा यद् यथ् स॑मृतय॒ज्ञाः स॑मृतय॒ज्ञा यत् । \newline
9. स॒मृ॒त॒य॒ज्ञा इति॑ समृत - य॒ज्ञाः । \newline
10. यद् द॑र्.शपूर्णमा॒सौ द॑र्.शपूर्णमा॒सौ यद् यद् द॑र्.शपूर्णमा॒सौ । \newline
11. द॒र्॒.श॒पू॒र्ण॒मा॒सौ कस्य॒ कस्य॑ दर्.शपूर्णमा॒सौ द॑र्.शपूर्णमा॒सौ कस्य॑ । \newline
12. द॒र्॒.श॒पू॒र्ण॒मा॒साविति॑ दर्.श - पू॒र्ण॒मा॒सौ । \newline
13. कस्य॑ वा वा॒ कस्य॒ कस्य॑ वा । \newline
14. वा ऽहाह॑ वा॒ वा ऽह॑ । \newline
15. अह॑ दे॒वा दे॒वा अहाह॑ दे॒वाः । \newline
16. दे॒वा य॒ज्ञ्ं ॅय॒ज्ञ्म् दे॒वा दे॒वा य॒ज्ञ्म् । \newline
17. य॒ज्ञ् मा॒गच्छ॑न् त्या॒गच्छ॑न्ति य॒ज्ञ्ं ॅय॒ज्ञ् मा॒गच्छ॑न्ति । \newline
18. आ॒गच्छ॑न्ति॒ कस्य॒ कस्या॒गच्छ॑न् त्या॒गच्छ॑न्ति॒ कस्य॑ । \newline
19. आ॒गच्छ॒न्तीत्या᳚ - गच्छ॑न्ति । \newline
20. कस्य॑ वा वा॒ कस्य॒ कस्य॑ वा । \newline
21. वा॒ न न वा॑ वा॒ न । \newline
22. न ब॑हू॒नाम् ब॑हू॒नाम् न न ब॑हू॒नाम् । \newline
23. ब॒हू॒नां ॅयज॑मानानां॒ ॅयज॑मानानाम् बहू॒नाम् ब॑हू॒नां ॅयज॑मानानाम् । \newline
24. यज॑मानानां॒ ॅयो यो यज॑मानानां॒ ॅयज॑मानानां॒ ॅयः । \newline
25. यो वै वै यो यो वै । \newline
26. वै दे॒वता॑ दे॒वता॒ वै वै दे॒वताः᳚ । \newline
27. दे॒वताः॒ पूर्वः॒ पूर्वो॑ दे॒वता॑ दे॒वताः॒ पूर्वः॑ । \newline
28. पूर्वः॑ परिगृ॒ह्णाति॑ परिगृ॒ह्णाति॒ पूर्वः॒ पूर्वः॑ परिगृ॒ह्णाति॑ । \newline
29. प॒रि॒गृ॒ह्णाति॒ स स प॑रिगृ॒ह्णाति॑ परिगृ॒ह्णाति॒ सः । \newline
30. प॒रि॒गृ॒ह्णातीति॑ परि - गृ॒ह्णाति॑ । \newline
31. स ए॑ना एनाः॒ स स ए॑नाः । \newline
32. ए॒नाः॒ श्वः श्व ए॑ना एनाः॒ श्वः । \newline
33. श्वो भू॒ते भू॒ते श्वः श्वो भू॒ते । \newline
34. भू॒ते य॑जते यजते भू॒ते भू॒ते य॑जते । \newline
35. य॒ज॒त॒ ए॒त दे॒तद् य॑जते यजत ए॒तत् । \newline
36. ए॒तद् वै वा ए॒त दे॒तद् वै । \newline
37. वै दे॒वाना᳚म् दे॒वानां॒ ॅवै वै दे॒वाना᳚म् । \newline
38. दे॒वाना॑ मा॒यत॑न मा॒यत॑नम् दे॒वाना᳚म् दे॒वाना॑ मा॒यत॑नम् । \newline
39. आ॒यत॑नं॒ ॅयद् यदा॒यत॑न मा॒यत॑नं॒ ॅयत् । \newline
40. आ॒यत॑न॒मित्या᳚ - यत॑नम् । \newline
41. यदा॑हव॒नीय॑ आहव॒नीयो॒ यद् यदा॑हव॒नीयः॑ । \newline
42. आ॒ह॒व॒नीयो᳚ ऽन्त॒रा ऽन्त॒रा ऽऽह॑व॒नीय॑ आहव॒नीयो᳚ ऽन्त॒रा । \newline
43. आ॒ह॒व॒नीय॒ इत्या᳚ - ह॒व॒नीयः॑ । \newline
44. अ॒न्त॒रा ऽग्नी अ॒ग्नी अ॑न्त॒रा ऽन्त॒रा ऽग्नी । \newline
45. अ॒ग्नी प॑शू॒नाम् प॑शू॒ना म॒ग्नी अ॒ग्नी प॑शू॒नाम् । \newline
46. अ॒ग्नी इत्य॒ग्नी । \newline
47. प॒शू॒नाम् गार्.ह॑पत्यो॒ गार्.ह॑पत्यः पशू॒नाम् प॑शू॒नाम् गार्.ह॑पत्यः । \newline
48. गार्.ह॑पत्यो मनु॒ष्या॑णाम् मनु॒ष्या॑णा॒म् गार्.ह॑पत्यो॒ गार्.ह॑पत्यो मनु॒ष्या॑णाम् । \newline
49. गार्.ह॑पत्य॒ इति॒ गार्.ह॑ - प॒त्यः॒ । \newline
50. म॒नु॒ष्या॑णा मन्वाहार्य॒पच॑नो ऽन्वाहार्य॒पच॑नो मनु॒ष्या॑णाम् मनु॒ष्या॑णा मन्वाहार्य॒पच॑नः । \newline
51. अ॒न्वा॒हा॒र्य॒पच॑नः पितृ॒णाम् पि॑तृ॒णा म॑न्वाहार्य॒पच॑नो ऽन्वाहार्य॒पच॑नः पितृ॒णाम् । \newline
52. अ॒न्वा॒हा॒र्य॒पच॑न॒ इत्य॑न्वाहार्य - पच॑नः । \newline
53. पि॒तृ॒णा म॒ग्नि म॒ग्निम् पि॑तृ॒णाम् पि॑तृ॒णा म॒ग्निम् । \newline
54. अ॒ग्निम् गृ॑ह्णाति गृह्णा त्य॒ग्नि म॒ग्निम् गृ॑ह्णाति । \newline
55. गृ॒ह्णा॒ति॒ स्वे स्वे गृ॑ह्णाति गृह्णाति॒ स्वे । \newline
56. स्व ए॒वैव स्वे स्व ए॒व । \newline
57. ए॒वायत॑न आ॒यत॑न ए॒वैवायत॑ने । \newline
58. आ॒यत॑ने दे॒वता॑ दे॒वता॑ आ॒यत॑न आ॒यत॑ने दे॒वताः᳚ । \newline
59. आ॒यत॑न॒ इत्या᳚ - यत॑ने । \newline
60. दे॒वताः॒ परि॒ परि॑ दे॒वता॑ दे॒वताः॒ परि॑ । \newline
61. परि॑ गृह्णाति गृह्णाति॒ परि॒ परि॑ गृह्णाति । \newline

\textbf{Ghana Paata } \newline

1. यथा॒ वै वै यथा॒ यथा॒ वै स॑मृतसो॒माः स॑मृतसो॒मा वै यथा॒ यथा॒ वै स॑मृतसो॒माः । \newline
2. वै स॑मृतसो॒माः स॑मृतसो॒मा वै वै स॑मृतसो॒मा ए॒व मे॒वꣳ स॑मृतसो॒मा वै वै स॑मृतसो॒मा ए॒वम् । \newline
3. स॒मृ॒त॒सो॒मा ए॒व मे॒वꣳ स॑मृतसो॒माः स॑मृतसो॒मा ए॒वं ॅवै वा ए॒वꣳ स॑मृतसो॒माः स॑मृतसो॒मा ए॒वं ॅवै । \newline
4. स॒मृ॒त॒सो॒मा इति॑ समृत - सो॒माः । \newline
5. ए॒वं ॅवै वा ए॒व मे॒वं ॅवा ए॒त ए॒ते वा ए॒व मे॒वं ॅवा ए॒ते । \newline
6. वा ए॒त ए॒ते वै वा ए॒ते स॑मृतय॒ज्ञाः स॑मृतय॒ज्ञा ए॒ते वै वा ए॒ते स॑मृतय॒ज्ञाः । \newline
7. ए॒ते स॑मृतय॒ज्ञाः स॑मृतय॒ज्ञा ए॒त ए॒ते स॑मृतय॒ज्ञा यद् यथ् स॑मृतय॒ज्ञा ए॒त ए॒ते स॑मृतय॒ज्ञा यत् । \newline
8. स॒मृ॒त॒य॒ज्ञा यद् यथ् स॑मृतय॒ज्ञाः स॑मृतय॒ज्ञा यद् द॑र्.शपूर्णमा॒सौ द॑र्.शपूर्णमा॒सौ यथ् स॑मृतय॒ज्ञाः स॑मृतय॒ज्ञा यद् द॑र्.शपूर्णमा॒सौ । \newline
9. स॒मृ॒त॒य॒ज्ञा इति॑ समृत - य॒ज्ञाः । \newline
10. यद् द॑र्.शपूर्णमा॒सौ द॑र्.शपूर्णमा॒सौ यद् यद् द॑र्.शपूर्णमा॒सौ कस्य॒ कस्य॑ दर्.शपूर्णमा॒सौ यद् यद् द॑र्.शपूर्णमा॒सौ कस्य॑ । \newline
11. द॒र्॒.श॒पू॒र्ण॒मा॒सौ कस्य॒ कस्य॑ दर्.शपूर्णमा॒सौ द॑र्.शपूर्णमा॒सौ कस्य॑ वा वा॒ कस्य॑ दर्.शपूर्णमा॒सौ द॑र्.शपूर्णमा॒सौ कस्य॑ वा । \newline
12. द॒र्॒.श॒पू॒र्ण॒मा॒साविति॑ दर्.श - पू॒र्ण॒मा॒सौ । \newline
13. कस्य॑ वा वा॒ कस्य॒ कस्य॒ वा ऽहाह॑ वा॒ कस्य॒ कस्य॒ वा ऽह॑ । \newline
14. वा ऽहाह॑ वा॒ वा ऽह॑ दे॒वा दे॒वा अह॑ वा॒ वा ऽह॑ दे॒वाः । \newline
15. अह॑ दे॒वा दे॒वा अहाह॑ दे॒वा य॒ज्ञ्ं ॅय॒ज्ञ्म् दे॒वा अहाह॑ दे॒वा य॒ज्ञ्म् । \newline
16. दे॒वा य॒ज्ञ्ं ॅय॒ज्ञ्म् दे॒वा दे॒वा य॒ज्ञ् मा॒गच्छ॑न्त्या॒गच्छ॑न्ति य॒ज्ञ्म् दे॒वा दे॒वा य॒ज्ञ् मा॒गच्छ॑न्ति । \newline
17. य॒ज्ञ् मा॒गच्छ॑न्त्या॒गच्छ॑न्ति य॒ज्ञ्ं ॅय॒ज्ञ् मा॒गच्छ॑न्ति॒ कस्य॒ कस्या॒गच्छ॑न्ति य॒ज्ञ्ं ॅय॒ज्ञ् मा॒गच्छ॑न्ति॒ कस्य॑ । \newline
18. आ॒गच्छ॑न्ति॒ कस्य॒ कस्या॒गच्छ॑ न्त्या॒गच्छ॑न्ति॒ कस्य॑ वा वा॒ कस्या॒गच्छ॑ न्त्या॒गच्छ॑न्ति॒ कस्य॑ वा । \newline
19. आ॒गच्छ॒न्तीत्या᳚ - गच्छ॑न्ति । \newline
20. कस्य॑ वा वा॒ कस्य॒ कस्य॑ वा॒ न न वा॒ कस्य॒ कस्य॑ वा॒ न । \newline
21. वा॒ न न वा॑ वा॒ न ब॑हू॒नाम् ब॑हू॒नान्न वा॑ वा॒ न ब॑हू॒नाम् । \newline
22. न ब॑हू॒नाम् ब॑हू॒नान्न न ब॑हू॒नां ॅयज॑मानानां॒ ॅयज॑मानानाम् बहू॒नान्न न ब॑हू॒नां ॅयज॑मानानाम् । \newline
23. ब॒हू॒नां ॅयज॑मानानां॒ ॅयज॑मानानाम् बहू॒नाम् ब॑हू॒नां ॅयज॑मानानां॒ ॅयो यो यज॑मानानाम् बहू॒नाम् ब॑हू॒नां ॅयज॑मानानां॒ ॅयः । \newline
24. यज॑मानानां॒ ॅयो यो यज॑मानानां॒ ॅयज॑मानानां॒ ॅयो वै वै यो यज॑मानानां॒ ॅयज॑मानानां॒ ॅयो वै । \newline
25. यो वै वै यो यो वै दे॒वता॑ दे॒वता॒ वै यो यो वै दे॒वताः᳚ । \newline
26. वै दे॒वता॑ दे॒वता॒ वै वै दे॒वताः॒ पूर्वः॒ पूर्वो॑ दे॒वता॒ वै वै दे॒वताः॒ पूर्वः॑ । \newline
27. दे॒वताः॒ पूर्वः॒ पूर्वो॑ दे॒वता॑ दे॒वताः॒ पूर्वः॑ परिगृ॒ह्णाति॑ परिगृ॒ह्णाति॒ पूर्वो॑ दे॒वता॑ दे॒वताः॒ पूर्वः॑ परिगृ॒ह्णाति॑ । \newline
28. पूर्वः॑ परिगृ॒ह्णाति॑ परिगृ॒ह्णाति॒ पूर्वः॒ पूर्वः॑ परिगृ॒ह्णाति॒ स स प॑रिगृ॒ह्णाति॒ पूर्वः॒ पूर्वः॑ परिगृ॒ह्णाति॒ सः । \newline
29. प॒रि॒गृ॒ह्णाति॒ स स प॑रिगृ॒ह्णाति॑ परिगृ॒ह्णाति॒ स ए॑ना एनाः॒ स प॑रिगृ॒ह्णाति॑ परिगृ॒ह्णाति॒ स ए॑नाः । \newline
30. प॒रि॒गृ॒ह्णातीति॑ परि - गृ॒ह्णाति॑ । \newline
31. स ए॑ना एनाः॒ स स ए॑नाः॒ श्वः श्व ए॑नाः॒ स स ए॑नाः॒ श्वः । \newline
32. ए॒नाः॒ श्वः श्व ए॑ना एनाः॒ श्वो भू॒ते भू॒ते श्व ए॑ना एनाः॒ श्वो भू॒ते । \newline
33. श्वो भू॒ते भू॒ते श्वः श्वो भू॒ते य॑जते यजते भू॒ते श्वः श्वो भू॒ते य॑जते । \newline
34. भू॒ते य॑जते यजते भू॒ते भू॒ते य॑जत ए॒तदे॒तद् य॑जते भू॒ते भू॒ते य॑जत ए॒तत् । \newline
35. य॒ज॒त॒ ए॒तदे॒तद् य॑जते यजत ए॒तद् वै वा ए॒तद् य॑जते यजत ए॒तद् वै । \newline
36. ए॒तद् वै वा ए॒तदे॒तद् वै दे॒वाना᳚म् दे॒वानां॒ ॅवा ए॒तदे॒तद् वै दे॒वाना᳚म् । \newline
37. वै दे॒वाना᳚म् दे॒वानां॒ ॅवै वै दे॒वाना॑ मा॒यत॑न मा॒यत॑नम् दे॒वानां॒ ॅवै वै दे॒वाना॑ मा॒यत॑नम् । \newline
38. दे॒वाना॑ मा॒यत॑न मा॒यत॑नम् दे॒वाना᳚म् दे॒वाना॑ मा॒यत॑नं॒ ॅयद् यदा॒यत॑नम् दे॒वाना᳚म् दे॒वाना॑ मा॒यत॑नं॒ ॅयत् । \newline
39. आ॒यत॑नं॒ ॅयद् यदा॒यत॑न मा॒यत॑नं॒ ॅयदा॑हव॒नीय॑ आहव॒नीयो॒ यदा॒यत॑न मा॒यत॑नं॒ ॅयदा॑हव॒नीयः॑ । \newline
40. आ॒यत॑न॒मित्या᳚ - यत॑नम् । \newline
41. यदा॑हव॒नीय॑ आहव॒नीयो॒ यद् यदा॑हव॒नीयो᳚ ऽन्त॒रा ऽन्त॒रा ऽऽह॑व॒नीयो॒ यद् यदा॑हव॒नीयो᳚ ऽन्त॒रा । \newline
42. आ॒ह॒व॒नीयो᳚ ऽन्त॒रा ऽन्त॒रा ऽऽह॑व॒नीय॑ आहव॒नीयो᳚ ऽन्त॒रा ऽग्नी अ॒ग्नी अ॑न्त॒रा ऽऽह॑व॒नीय॑ आहव॒नीयो᳚ ऽन्त॒रा ऽग्नी । \newline
43. आ॒ह॒व॒नीय॒ इत्या᳚ - ह॒व॒नीयः॑ । \newline
44. अ॒न्त॒रा ऽग्नी अ॒ग्नी अ॑न्त॒रा ऽन्त॒रा ऽग्नी प॑शू॒नाम् प॑शू॒ना म॒ग्नी अ॑न्त॒रा ऽन्त॒रा ऽग्नी प॑शू॒नाम् । \newline
45. अ॒ग्नी प॑शू॒नाम् प॑शू॒ना म॒ग्नी अ॒ग्नी प॑शू॒नाम् गार्.ह॑पत्यो॒ गार्.ह॑पत्यः पशू॒ना म॒ग्नी अ॒ग्नी प॑शू॒नाम् गार्.ह॑पत्यः । \newline
46. अ॒ग्नी इत्य॒ग्नी । \newline
47. प॒शू॒नाम् गार्.ह॑पत्यो॒ गार्.ह॑पत्यः पशू॒नाम् प॑शू॒नाम् गार्.ह॑पत्यो मनु॒ष्या॑णाम् मनु॒ष्या॑णा॒म् गार्.ह॑पत्यः पशू॒नाम् प॑शू॒नाम् गार्.ह॑पत्यो मनु॒ष्या॑णाम् । \newline
48. गार्.ह॑पत्यो मनु॒ष्या॑णाम् मनु॒ष्या॑णा॒म् गार्.ह॑पत्यो॒ गार्.ह॑पत्यो मनु॒ष्या॑णा मन्वाहार्य॒पच॑नो ऽन्वाहार्य॒पच॑नो मनु॒ष्या॑णा॒म् गार्.ह॑पत्यो॒ गार्.ह॑पत्यो मनु॒ष्या॑णा मन्वाहार्य॒पच॑नः । \newline
49. गार्.ह॑पत्य॒ इति॒ गार्.ह॑ - प॒त्यः॒ । \newline
50. म॒नु॒ष्या॑णा मन्वाहार्य॒पच॑नो ऽन्वाहार्य॒पच॑नो मनु॒ष्या॑णाम् मनु॒ष्या॑णा मन्वाहार्य॒पच॑नः पितृ॒णाम् पि॑तृ॒णा म॑न्वाहार्य॒पच॑नो मनु॒ष्या॑णाम् मनु॒ष्या॑णा मन्वाहार्य॒पच॑नः पितृ॒णाम् । \newline
51. अ॒न्वा॒हा॒र्य॒पच॑नः पितृ॒णाम् पि॑तृ॒णा म॑न्वाहार्य॒पच॑नो ऽन्वाहार्य॒पच॑नः पितृ॒णा म॒ग्नि म॒ग्निम् पि॑तृ॒णा म॑न्वाहार्य॒पच॑नो ऽन्वाहार्य॒पच॑नः पितृ॒णा म॒ग्निम् । \newline
52. अ॒न्वा॒हा॒र्य॒पच॑न॒ इत्य॑न्वाहार्य - पच॑नः । \newline
53. पि॒तृ॒णा म॒ग्नि म॒ग्निम् पि॑तृ॒णाम् पि॑तृ॒णा म॒ग्निम् गृ॑ह्णाति गृह्णात्य॒ग्निम् पि॑तृ॒णाम् पि॑तृ॒णा म॒ग्निम् गृ॑ह्णाति । \newline
54. अ॒ग्निम् गृ॑ह्णाति गृह्णात्य॒ग्नि म॒ग्निम् गृ॑ह्णाति॒ स्वे स्वे गृ॑ह्णात्य॒ग्नि म॒ग्निम् गृ॑ह्णाति॒ स्वे । \newline
55. गृ॒ह्णा॒ति॒ स्वे स्वे गृ॑ह्णाति गृह्णाति॒ स्व ए॒वैव स्वे गृ॑ह्णाति गृह्णाति॒ स्व ए॒व । \newline
56. स्व ए॒वैव स्वे स्व ए॒वायत॑न आ॒यत॑न ए॒व स्वे स्व ए॒वायत॑ने । \newline
57. ए॒वायत॑न आ॒यत॑न ए॒वैवायत॑ने दे॒वता॑ दे॒वता॑ आ॒यत॑न ए॒वैवायत॑ने दे॒वताः᳚ । \newline
58. आ॒यत॑ने दे॒वता॑ दे॒वता॑ आ॒यत॑न आ॒यत॑ने दे॒वताः॒ परि॒ परि॑ दे॒वता॑ आ॒यत॑न आ॒यत॑ने दे॒वताः॒ परि॑ । \newline
59. आ॒यत॑न॒ इत्या᳚ - यत॑ने । \newline
60. दे॒वताः॒ परि॒ परि॑ दे॒वता॑ दे॒वताः॒ परि॑ गृह्णाति गृह्णाति॒ परि॑ दे॒वता॑ दे॒वताः॒ परि॑ गृह्णाति । \newline
61. परि॑ गृह्णाति गृह्णाति॒ परि॒ परि॑ गृह्णाति॒ तास्ता गृ॑ह्णाति॒ परि॒ परि॑ गृह्णाति॒ ताः । \newline
\pagebreak
\markright{ TS 1.6.7.2  \hfill https://www.vedavms.in \hfill}

\section{ TS 1.6.7.2 }

\textbf{TS 1.6.7.2 } \newline
\textbf{Samhita Paata} \newline

गृह्णाति॒ ताः श्वो भू॒ते य॑जते व्र॒तेन॒ वै मेद्ध्यो॒ -ऽग्निर् व्र॒तप॑तिर् ब्राह्म॒णो व्र॑त॒भृद्-व्र॒त-मु॑पै॒ष्यन् ब्रू॑या॒दग्ने᳚ व्रतपते व्र॒तं च॑रिष्या॒मीत्य॒ग्निर् वै दे॒वानां᳚ ॅव्र॒तप॑ति॒स्तस्मा॑ ए॒व प्र॑ति॒प्रोच्य॑ व्र॒तमा ल॑भते ब॒र्॒.हिषा॑ पू॒र्णमा॑से व्र॒तमुपै॑ति व॒थ्सैर॑मावा॒स्या॑यामे॒तद्ध्ये॑तयो॑-रा॒यत॑नमुप॒स्तीर्यः॒ पूर्व॑श्चा॒ग्निरप॑र॒श्चेत्या॑हुर् मनु॒ष्या॑ - [ ] \newline

\textbf{Pada Paata} \newline

गृ॒ह्णा॒ति॒ । ताः । श्वः । भू॒ते । य॒ज॒ते॒ । व्र॒तेन॑ । वै । मेद्ध्यः॑ । अ॒ग्निः । व्र॒तप॑ति॒रिति॑ व्र॒त - प॒तिः॒ । ब्रा॒ह्म॒णः । व्र॒त॒भृदिति॑ व्रत - भृत् । व्र॒तम् । उ॒पै॒ष्यन्नित्यु॑प - ए॒ष्यन्न् । ब्रू॒या॒त् । अग्ने᳚ । व्र॒त॒प॒त॒ इति॑ व्रत - प॒ते॒ । व्र॒तम् । च॒रि॒ष्या॒मि॒ । इति॑ । अ॒ग्निः । वै । दे॒वाना᳚म् । व्र॒तप॑ति॒रिति॑ व्र॒त - प॒तिः॒ । तस्मै᳚ । ए॒व । प्र॒ति॒प्रोच्येति॑ प्रति - प्रोच्य॑ । व्र॒तम् । एति॑ । ल॒भ॒ते॒ । ब॒र्॒.हिषा᳚ । पू॒र्णमा॑स॒ इति॑ पू॒र्ण - मा॒से॒ । व्र॒तम् । उपेति॑ । ए॒ति॒ । व॒थ्सैः । अ॒मा॒वा॒स्या॑या॒मित्य॑मा - वा॒स्या॑याम् । ए॒तत् । हि । ए॒तयोः᳚ । आ॒यत॑न॒मित्या᳚ - यत॑नम् । उ॒प॒स्तीर्य॒ इत्यु॑प - स्तीर्यः॑ । पूर्वः॑ । च॒ । अ॒ग्निः । अप॑रः । च॒ । इति॑ । आ॒हुः॒ । म॒नु॒ष्याः᳚ ।  \newline


\textbf{Krama Paata} \newline

गृ॒ह्णा॒ति॒ ताः । ताः श्वः । श्वो भू॒ते । भू॒ते य॑जते । य॒ज॒ते॒ व्र॒तेन॑ । व्र॒तेन॒ वै । वै मेद्ध्यः॑ । मेद्ध्यो॒ऽग्निः । अ॒ग्निर् व्र॒तप॑तिः । व्र॒तप॑तिर्,ब्राह्म॒णः । व्र॒तप॑ति॒रिति॑ व्र॒त - प॒तिः॒ । ब्रा॒ह्म॒णो व्र॑त॒भृत् । व्र॒त॒भृद् व्र॒तम् । व्र॒त॒भृदिति॑ व्रत - भृत् । व्र॒तमु॑पै॒ष्यन्न् । उ॒पै॒ष्यन्,ब्रू॑यात् । उ॒पै॒ष्यन्नित्यु॑प - ए॒ष्यन्न् । ब्रू॒या॒दग्ने᳚ । अग्ने᳚ व्रतपते । व्र॒त॒प॒ते॒ व्र॒तम् । व्र॒त॒प॒त॒ इति॑ व्रत - प॒ते॒ । व्र॒तम् च॑रिष्यामि । च॒रि॒ष्या॒मीति॑ । इत्य॒ग्निः । अ॒ग्निर् वै । वै दे॒वाना᳚म् । दे॒वानां᳚ ॅव्र॒तप॑तिः । व्र॒तप॑ति॒स्तस्मै᳚ । व्र॒तप॑ति॒रिति॑ व्र॒त - प॒तिः॒ । तस्मा॑ ए॒व । ए॒व प्र॑ति॒प्रोच्य॑ । प्र॒ति॒प्रोच्य॑ व्र॒तम् । प्र॒ति॒प्रोच्येति॑ प्रति - प्रोच्य॑ । व्र॒त मा । आ ल॑भते । ल॒भ॒ते॒ ब॒र्.॒हिषा᳚ । ब॒र्.॒हिषा॑ पू॒र्णमा॑से । पू॒र्णमा॑से व्र॒तम् । पू॒र्णमा॑स॒ इति॑ पू॒र्ण - मा॒से॒ । व्र॒तमुप॑ । उपै॑ति । ए॒ति॒ व॒थ्सैः । व॒थ्सैर॑मावा॒स्या॑याम् । अ॒मा॒वा॒स्या॑यामे॒तत् । अ॒मा॒वा॒स्या॑या॒मित्य॑मा - वा॒स्या॑याम् । ए॒तद्धि । ह्ये॑तयोः᳚ । ए॒तयो॑रा॒यत॑नम् । आ॒यतन॑मुप॒स्तीर्यः॑ । आ॒यत॑न॒मित्या᳚ - यत॑नम् । उ॒प॒स्तीर्यः॒ पूर्वः॑ । उ॒प॒स्तीर्य॒ इत्यु॑प - स्तीर्यः॑ । पूर्व॑श्च । चा॒ग्निः । अ॒ग्निरप॑रः । अप॑रश्च । चेति॑ । इत्या॑हुः । आ॒हु॒र्,म॒नु॒ष्याः᳚ । म॒नु॒ष्या॑ इत् \newline

\textbf{Jatai Paata} \newline

1. गृ॒ह्णा॒ति॒ तास्ता गृ॑ह्णाति गृह्णाति॒ ताः । \newline
2. ताः श्वः श्व स्ता स्ताः श्वः । \newline
3. श्वो भू॒ते भू॒ते श्वः श्वो भू॒ते । \newline
4. भू॒ते य॑जते यजते भू॒ते भू॒ते य॑जते । \newline
5. य॒ज॒ते॒ व्र॒तेन॑ व्र॒तेन॑ यजते यजते व्र॒तेन॑ । \newline
6. व्र॒तेन॒ वै वै व्र॒तेन॑ व्र॒तेन॒ वै । \newline
7. वै मेद्ध्यो॒ मेद्ध्यो॒ वै वै मेद्ध्यः॑ । \newline
8. मेद्ध्यो॒ ऽग्नि र॒ग्निर् मेद्ध्यो॒ मेद्ध्यो॒ ऽग्निः । \newline
9. अ॒ग्निर् व्र॒तप॑तिर् व्र॒तप॑ति र॒ग्नि र॒ग्निर् व्र॒तप॑तिः । \newline
10. व्र॒तप॑तिर् ब्राह्म॒णो ब्रा᳚ह्म॒णो व्र॒तप॑तिर् व्र॒तप॑तिर् ब्राह्म॒णः । \newline
11. व्र॒तप॑ति॒रिति॑ व्र॒त - प॒तिः॒ । \newline
12. ब्रा॒ह्म॒णो व्र॑त॒भृद् व्र॑त॒भृद् ब्रा᳚ह्म॒णो ब्रा᳚ह्म॒णो व्र॑त॒भृत् । \newline
13. व्र॒त॒भृद् व्र॒तं ॅव्र॒तं ॅव्र॑त॒भृद् व्र॑त॒भृद् व्र॒तम् । \newline
14. व्र॒त॒भृदिति॑ व्रत - भृत् । \newline
15. व्र॒त मु॑पै॒ष्यन् नु॑पै॒ष्यन् व्र॒तं ॅव्र॒त मु॑पै॒ष्यन्न् । \newline
16. उ॒पै॒ष्यन् ब्रू॑याद् ब्रूया दुपै॒ष्यन् नु॑पै॒ष्यन् ब्रू॑यात् । \newline
17. उ॒पै॒ष्यन्नित्यु॑प - ए॒ष्यन्न् । \newline
18. ब्रू॒या॒दग्ने ऽग्ने᳚ ब्रूयाद् ब्रूया॒दग्ने᳚ । \newline
19. अग्ने᳚ व्रतपते व्रतप॒ते ऽग्ने ऽग्ने᳚ व्रतपते । \newline
20. व्र॒त॒प॒ते॒ व्र॒तं ॅव्र॒तं ॅव्र॑तपते व्रतपते व्र॒तम् । \newline
21. व्र॒त॒प॒त॒ इति॑ व्रत - प॒ते॒ । \newline
22. व्र॒तम् च॑रिष्यामि चरिष्यामि व्र॒तं ॅव्र॒तम् च॑रिष्यामि । \newline
23. च॒रि॒ष्या॒ मीतीति॑ चरिष्यामि चरिष्या॒ मीति॑ । \newline
24. इत्य॒ग्नि र॒ग्नि रिती त्य॒ग्निः । \newline
25. अ॒ग्निर् वै वा अ॒ग्नि र॒ग्निर् वै । \newline
26. वै दे॒वाना᳚म् दे॒वानां॒ ॅवै वै दे॒वाना᳚म् । \newline
27. दे॒वानां᳚ ॅव्र॒तप॑तिर् व्र॒तप॑तिर् दे॒वाना᳚म् दे॒वानां᳚ ॅव्र॒तप॑तिः । \newline
28. व्र॒तप॑ति॒ स्तस्मै॒ तस्मै᳚ व्र॒तप॑तिर् व्र॒तप॑ति॒ स्तस्मै᳚ । \newline
29. व्र॒तप॑ति॒रिति॑ व्र॒त - प॒तिः॒ । \newline
30. तस्मा॑ ए॒वैव तस्मै॒ तस्मा॑ ए॒व । \newline
31. ए॒व प्र॑ति॒प्रोच्य॑ प्रति॒प्रो च्यै॒वैव प्र॑ति॒प्रोच्य॑ । \newline
32. प्र॒ति॒प्रोच्य॑ व्र॒तं ॅव्र॒तम् प्र॑ति॒प्रोच्य॑ प्रति॒प्रोच्य॑ व्र॒तम् । \newline
33. प्र॒ति॒प्रोच्येति॑ प्रति - प्रोच्य॑ । \newline
34. व्र॒त मा व्र॒तं ॅव्र॒त मा । \newline
35. आ ल॑भते लभत॒ आ ल॑भते । \newline
36. ल॒भ॒ते॒ ब॒र्॒.हिषा॑ ब॒र्॒.हिषा॑ लभते लभते ब॒र्॒.हिषा᳚ । \newline
37. ब॒र्॒.हिषा॑ पू॒र्णमा॑से पू॒र्णमा॑से ब॒र्॒.हिषा॑ ब॒र्॒.हिषा॑ पू॒र्णमा॑से । \newline
38. पू॒र्णमा॑से व्र॒तं ॅव्र॒तम् पू॒र्णमा॑से पू॒र्णमा॑से व्र॒तम् । \newline
39. पू॒र्णमा॑स॒ इति॑ पू॒र्ण - मा॒से॒ । \newline
40. व्र॒त मुपोप॑ व्र॒तं ॅव्र॒त मुप॑ । \newline
41. उपै᳚त्ये॒त्युपोपै॑ति । \newline
42. ए॒ति॒ व॒थ्सैर् व॒थ्सै रे᳚त्येति व॒थ्सैः । \newline
43. व॒थ्सै र॑मावा॒स्या॑या ममावा॒स्या॑यां ॅव॒थ्सैर् व॒थ्सै र॑मावा॒स्या॑याम् । \newline
44. अ॒मा॒वा॒स्या॑या मे॒त दे॒त द॑मावा॒स्या॑या ममावा॒स्या॑या मे॒तत् । \newline
45. अ॒मा॒वा॒स्या॑या॒मित्य॑मा - वा॒स्या॑याम् । \newline
46. ए॒तद्धि ह्ये॑त दे॒तद्धि । \newline
47. ह्ये॑तयो॑ रे॒तयो॒र्॒. हि ह्ये॑तयोः᳚ । \newline
48. ए॒तयो॑ रा॒यत॑न मा॒यत॑न मे॒तयो॑ रे॒तयो॑ रा॒यत॑नम् । \newline
49. आ॒यत॑न मुप॒स्तीर्य॑ उप॒स्तीर्य॑ आ॒यत॑न मा॒यत॑न मुप॒स्तीर्यः॑ । \newline
50. आ॒यत॑न॒मित्या᳚ - यत॑नम् । \newline
51. उ॒प॒स्तीर्यः॒ पूर्वः॒ पूर्व॑ उप॒स्तीर्य॑ उप॒स्तीर्यः॒ पूर्वः॑ । \newline
52. उ॒प॒स्तीर्य॒ इत्यु॑प - स्तीर्यः॑ । \newline
53. पूर्व॑ श्च च॒ पूर्वः॒ पूर्व॑ श्च । \newline
54. चा॒ग्नि र॒ग्निश्च॑ चा॒ग्निः । \newline
55. अ॒ग्नि रप॒रो ऽप॑रो॒ ऽग्नि र॒ग्नि रप॑रः । \newline
56. अप॑रश्च॒ चाप॒रो ऽप॑रश्च । \newline
57. चे तीति॑ च॒ चे ति॑ । \newline
58. इत्या॑हु राहु॒ रिती त्या॑हुः । \newline
59. आ॒हु॒र् म॒नु॒ष्या॑ मनु॒ष्या॑ आहु राहुर् मनु॒ष्याः᳚ । \newline
60. म॒नु॒ष्या॑ इदिन् म॑नु॒ष्या॑ मनु॒ष्या॑ इत् । \newline

\textbf{Ghana Paata } \newline

1. गृ॒ह्णा॒ति॒ तास्ता गृ॑ह्णाति गृह्णाति॒ ताः श्वः श्वस्ता गृ॑ह्णाति गृह्णाति॒ ताः श्वः । \newline
2. ताः श्वः श्वस्तास्ताः श्वो भू॒ते भू॒ते श्वस्तास्ताः श्वो भू॒ते । \newline
3. श्वो भू॒ते भू॒ते श्वः श्वो भू॒ते य॑जते यजते भू॒ते श्वः श्वो भू॒ते य॑जते । \newline
4. भू॒ते य॑जते यजते भू॒ते भू॒ते य॑जते व्र॒तेन॑ व्र॒तेन॑ यजते भू॒ते भू॒ते य॑जते व्र॒तेन॑ । \newline
5. य॒ज॒ते॒ व्र॒तेन॑ व्र॒तेन॑ यजते यजते व्र॒तेन॒ वै वै व्र॒तेन॑ यजते यजते व्र॒तेन॒ वै । \newline
6. व्र॒तेन॒ वै वै व्र॒तेन॑ व्र॒तेन॒ वै मेद्ध्यो॒ मेद्ध्यो॒ वै व्र॒तेन॑ व्र॒तेन॒ वै मेद्ध्यः॑ । \newline
7. वै मेद्ध्यो॒ मेद्ध्यो॒ वै वै मेद्ध्यो॒ ऽग्निर॒ग्निर् मेद्ध्यो॒ वै वै मेद्ध्यो॒ ऽग्निः । \newline
8. मेद्ध्यो॒ ऽग्निर॒ग्निर् मेद्ध्यो॒ मेद्ध्यो॒ ऽग्निर् व्र॒तप॑तिर् व्र॒तप॑ति र॒ग्निर् मेद्ध्यो॒ मेद्ध्यो॒ ऽग्निर् व्र॒तप॑तिः । \newline
9. अ॒ग्निर् व्र॒तप॑तिर् व्र॒तप॑ति र॒ग्निर॒ग्निर् व्र॒तप॑तिर् ब्राह्म॒णो ब्रा᳚ह्म॒णो व्र॒तप॑ति र॒ग्निर॒ग्निर् व्र॒तप॑तिर् ब्राह्म॒णः । \newline
10. व्र॒तप॑तिर् ब्राह्म॒णो ब्रा᳚ह्म॒णो व्र॒तप॑तिर् व्र॒तप॑तिर् ब्राह्म॒णो व्र॑त॒भृद् व्र॑त॒भृद् ब्रा᳚ह्म॒णो व्र॒तप॑तिर् व्र॒तप॑तिर् ब्राह्म॒णो व्र॑त॒भृत् । \newline
11. व्र॒तप॑ति॒रिति॑ व्र॒त - प॒तिः॒ । \newline
12. ब्रा॒ह्म॒णो व्र॑त॒भृद् व्र॑त॒भृद् ब्रा᳚ह्म॒णो ब्रा᳚ह्म॒णो व्र॑त॒भृद् व्र॒तं ॅव्र॒तं ॅव्र॑त॒भृद् ब्रा᳚ह्म॒णो ब्रा᳚ह्म॒णो व्र॑त॒भृद् व्र॒तम् । \newline
13. व्र॒त॒भृद् व्र॒तं ॅव्र॒तं ॅव्र॑त॒भृद् व्र॑त॒भृद् व्र॒त मु॑पै॒ष्यन् नु॑पै॒ष्यन् व्र॒तं ॅव्र॑त॒भृद् व्र॑त॒भृद् व्र॒त मु॑पै॒ष्यन्न् । \newline
14. व्र॒त॒भृदिति॑ व्रत - भृत् । \newline
15. व्र॒त मु॑पै॒ष्यन् नु॑पै॒ष्यन् व्र॒तं ॅव्र॒त मु॑पै॒ष्यन् ब्रू॑याद् ब्रूयादुपै॒ष्यन् व्र॒तं ॅव्र॒त मु॑पै॒ष्यन् ब्रू॑यात् । \newline
16. उ॒पै॒ष्यन् ब्रू॑याद् ब्रूयादुपै॒ष्यन् नु॑पै॒ष्यन् ब्रू॑या॒दग्ने ऽग्ने᳚ ब्रूयादुपै॒ष्यन् नु॑पै॒ष्यन् ब्रू॑या॒दग्ने᳚ । \newline
17. उ॒पै॒ष्यन्नित्यु॑प - ए॒ष्यन्न् । \newline
18. ब्रू॒या॒दग्ने ऽग्ने᳚ ब्रूयाद् ब्रूया॒दग्ने᳚ व्रतपते व्रतप॒ते ऽग्ने᳚ ब्रूयाद् ब्रूया॒दग्ने᳚ व्रतपते । \newline
19. अग्ने᳚ व्रतपते व्रतप॒ते ऽग्ने ऽग्ने᳚ व्रतपते व्र॒तं ॅव्र॒तं ॅव्र॑तप॒ते ऽग्ने ऽग्ने᳚ व्रतपते व्र॒तम् । \newline
20. व्र॒त॒प॒ते॒ व्र॒तं ॅव्र॒तं ॅव्र॑तपते व्रतपते व्र॒तम् च॑रिष्यामि चरिष्यामि व्र॒तं ॅव्र॑तपते व्रतपते व्र॒तम् च॑रिष्यामि । \newline
21. व्र॒त॒प॒त॒ इति॑ व्रत - प॒ते॒ । \newline
22. व्र॒तम् च॑रिष्यामि चरिष्यामि व्र॒तं ॅव्र॒तम् च॑रिष्या॒मीतीति॑ चरिष्यामि व्र॒तं ॅव्र॒तम् च॑रिष्या॒मीति॑ । \newline
23. च॒रि॒ष्या॒मीतीति॑ चरिष्यामि चरिष्या॒मी त्य॒ग्नि र॒ग्निरिति॑ चरिष्यामि चरिष्या॒मीत्य॒ग्निः । \newline
24. इत्य॒ग्नि र॒ग्नि रितीत्य॒ग्निर् वै वा अ॒ग्नि रितीत्य॒ग्निर् वै । \newline
25. अ॒ग्निर् वै वा अ॒ग्निर॒ग्निर् वै दे॒वाना᳚म् दे॒वानां॒ ॅवा अ॒ग्निर॒ग्निर् वै दे॒वाना᳚म् । \newline
26. वै दे॒वाना᳚म् दे॒वानां॒ ॅवै वै दे॒वानां᳚ ॅव्र॒तप॑तिर् व्र॒तप॑तिर् दे॒वानां॒ ॅवै वै दे॒वानां᳚ ॅव्र॒तप॑तिः । \newline
27. दे॒वानां᳚ ॅव्र॒तप॑तिर् व्र॒तप॑तिर् दे॒वाना᳚म् दे॒वानां᳚ ॅव्र॒तप॑ति॒ स्तस्मै॒ तस्मै᳚ व्र॒तप॑तिर् दे॒वाना᳚म् दे॒वानां᳚ ॅव्र॒तप॑ति॒ स्तस्मै᳚ । \newline
28. व्र॒तप॑ति॒ स्तस्मै॒ तस्मै᳚ व्र॒तप॑तिर् व्र॒तप॑ति॒ स्तस्मा॑ ए॒वैव तस्मै᳚ व्र॒तप॑तिर् व्र॒तप॑ति॒ स्तस्मा॑ ए॒व । \newline
29. व्र॒तप॑ति॒रिति॑ व्र॒त - प॒तिः॒ । \newline
30. तस्मा॑ ए॒वैव तस्मै॒ तस्मा॑ ए॒व प्र॑ति॒प्रोच्य॑ प्रति॒प्रोच्यै॒व तस्मै॒ तस्मा॑ ए॒व प्र॑ति॒प्रोच्य॑ । \newline
31. ए॒व प्र॑ति॒प्रोच्य॑ प्रति॒प्रोच्यै॒वैव प्र॑ति॒प्रोच्य॑ व्र॒तं ॅव्र॒तम् प्र॑ति॒प्रोच्यै॒वैव प्र॑ति॒प्रोच्य॑ व्र॒तम् । \newline
32. प्र॒ति॒प्रोच्य॑ व्र॒तं ॅव्र॒तम् प्र॑ति॒प्रोच्य॑ प्रति॒प्रोच्य॑ व्र॒त मा व्र॒तम् प्र॑ति॒प्रोच्य॑ प्रति॒प्रोच्य॑ व्र॒त मा । \newline
33. प्र॒ति॒प्रोच्येति॑ प्रति - प्रोच्य॑ । \newline
34. व्र॒त मा व्र॒तं ॅव्र॒त मा ल॑भते लभत॒ आ व्र॒तं ॅव्र॒त मा ल॑भते । \newline
35. आ ल॑भते लभत॒ आ ल॑भते ब॒र्॒.हिषा॑ ब॒र्॒.हिषा॑ लभत॒ आ ल॑भते ब॒र्॒.हिषा᳚ । \newline
36. ल॒भ॒ते॒ ब॒र्॒.हिषा॑ ब॒र्॒.हिषा॑ लभते लभते ब॒र्॒.हिषा॑ पू॒र्णमा॑से पू॒र्णमा॑से ब॒र्॒.हिषा॑ लभते लभते ब॒र्॒.हिषा॑ पू॒र्णमा॑से । \newline
37. ब॒र्॒.हिषा॑ पू॒र्णमा॑से पू॒र्णमा॑से ब॒र्॒.हिषा॑ ब॒र्॒.हिषा॑ पू॒र्णमा॑से व्र॒तं ॅव्र॒तम् पू॒र्णमा॑से ब॒र्॒.हिषा॑ ब॒र्॒.हिषा॑ पू॒र्णमा॑से व्र॒तम् । \newline
38. पू॒र्णमा॑से व्र॒तं ॅव्र॒तम् पू॒र्णमा॑से पू॒र्णमा॑से व्र॒त मुपोप॑ व्र॒तम् पू॒र्णमा॑से पू॒र्णमा॑से व्र॒त मुप॑ । \newline
39. पू॒र्णमा॑स॒ इति॑ पू॒र्ण - मा॒से॒ । \newline
40. व्र॒त मुपोप॑ व्र॒तं ॅव्र॒त मुपै᳚त्ये॒त्युप॑ व्र॒तं ॅव्र॒त मुपै॑ति । \newline
41. उपै᳚त्ये॒त्युपोपै॑ति व॒थ्सैर् व॒थ्सै रे॒त्युपोपै॑ति व॒थ्सैः । \newline
42. ए॒ति॒ व॒थ्सैर् व॒थ्सै रे᳚त्येति व॒थ्सै र॑मावा॒स्या॑या ममावा॒स्या॑यां ॅव॒थ्सैरे᳚त्येति व॒थ्सै र॑मावा॒स्या॑याम् । \newline
43. व॒थ्सै र॑मावा॒स्या॑या ममावा॒स्या॑यां ॅव॒थ्सैर् व॒थ्सै र॑मावा॒स्या॑या मे॒तदे॒त द॑मावा॒स्या॑यां ॅव॒थ्सैर् व॒थ्सै र॑मावा॒स्या॑या मे॒तत् । \newline
44. अ॒मा॒वा॒स्या॑या मे॒त दे॒तद॑मावा॒स्या॑या ममावा॒स्या॑या मे॒तद्धि ह्ये॑तद॑मावा॒स्या॑या ममावा॒स्या॑या मे॒तद्धि । \newline
45. अ॒मा॒वा॒स्या॑या॒मित्य॑मा - वा॒स्या॑याम् । \newline
46. ए॒तद्धि ह्ये॑त दे॒तद्ध्ये॑तयो॑ रे॒तयो॒र् ह्ये॑तदे॒तद्ध्ये॑तयोः᳚ । \newline
47. ह्ये॑तयो॑ रे॒तयो॒र्॒. हि ह्ये॑तयो॑ रा॒यत॑न मा॒यत॑न मे॒तयो॒र्॒. हि ह्ये॑तयो॑ रा॒यत॑नम् । \newline
48. ए॒तयो॑ रा॒यत॑न मा॒यत॑न मे॒तयो॑ रे॒तयो॑ रा॒यत॑न मुप॒स्तीर्य॑ उप॒स्तीर्य॑ आ॒यत॑न मे॒तयो॑ रे॒तयो॑ रा॒यत॑न मुप॒स्तीर्यः॑ । \newline
49. आ॒यत॑न मुप॒स्तीर्य॑ उप॒स्तीर्य॑ आ॒यत॑न मा॒यत॑न मुप॒स्तीर्यः॒ पूर्वः॒ पूर्व॑ उप॒स्तीर्य॑ आ॒यत॑न मा॒यत॑न मुप॒स्तीर्यः॒ पूर्वः॑ । \newline
50. आ॒यत॑न॒मित्या᳚ - यत॑नम् । \newline
51. उ॒प॒स्तीर्यः॒ पूर्वः॒ पूर्व॑ उप॒स्तीर्य॑ उप॒स्तीर्यः॒ पूर्व॑श्च च॒ पूर्व॑ उप॒स्तीर्य॑ उप॒स्तीर्यः॒ पूर्व॑श्च । \newline
52. उ॒प॒स्तीर्य॒ इत्यु॑प - स्तीर्यः॑ । \newline
53. पूर्व॑श्च च॒ पूर्वः॒ पूर्व॑श्चा॒ग्नि र॒ग्निश्च॒ पूर्वः॒ पूर्व॑श्चा॒ग्निः । \newline
54. चा॒ग्नि र॒ग्निश्च॑ चा॒ग्नि रप॒रो ऽप॑रो॒ ऽग्निश्च॑ चा॒ग्नि रप॑रः । \newline
55. अ॒ग्नि रप॒रो ऽप॑रो॒ ऽग्नि र॒ग्नि रप॑रश्च॒ चाप॑रो॒ ऽग्नि र॒ग्नि रप॑रश्च । \newline
56. अप॑रश्च॒ चाप॒रो ऽप॑र॒श्चे तीति॒ चाप॒रो ऽप॑र॒श्चे ति॑ । \newline
57. चे तीति॑ च॒ चे त्या॑हु राहु॒रिति॑ च॒ चे त्या॑हुः । \newline
58. इत्या॑हु राहु॒ रितीत्या॑हुर् मनु॒ष्या॑ मनु॒ष्या॑ आहु॒ रितीत्या॑हुर् मनु॒ष्याः᳚ । \newline
59. आ॒हु॒र् म॒नु॒ष्या॑ मनु॒ष्या॑ आहुराहुर् मनु॒ष्या॑ इदिन् म॑नु॒ष्या॑ आहुराहुर् मनु॒ष्या॑ इत् । \newline
60. म॒नु॒ष्या॑ इदिन् म॑नु॒ष्या॑ मनु॒ष्या॑ इन् नु न्विन् म॑नु॒ष्या॑ मनु॒ष्या॑ इन् नु । \newline
\pagebreak
\markright{ TS 1.6.7.3  \hfill https://www.vedavms.in \hfill}

\section{ TS 1.6.7.3 }

\textbf{TS 1.6.7.3 } \newline
\textbf{Samhita Paata} \newline

इन्न्वा उप॑स्तीर्ण-मि॒च्छन्ति॒ किमु॑ दे॒वा येषां॒ नवा॑वसान॒-मुपा᳚स्मि॒ञ्छ्वो य॒क्ष्यमा॑णे दे॒वता॑ वसन्ति॒ य ए॒वं ॅवि॒द्वान॒ग्नि-मु॑पस्तृ॒णाति॒ यज॑मानेन ग्रा॒म्याश्च॑ प॒शवो॑ऽव॒रुद्ध्या॑ आर॒ण्याश्चेत्या॑हु॒र् यद् ग्रा॒म्यानु॑प॒वस॑ति॒ तेन॑ ग्रा॒म्यानव॑ रुन्धे॒ यदा॑र॒ण्यस्या॒श्ञ॑0078;ति॒ तेना॑र॒ण्यान्. यदना᳚श्वानुप॒वसे᳚त् पितृदेव॒त्यः॑ स्यादार॒ण्यस्या᳚-श्ञातीन्द्रि॒यं - [ ] \newline

\textbf{Pada Paata} \newline

इत् । नु । वा । उप॑स्तीर्ण॒मियुप॑ - स्ती॒र्ण॒म् । इ॒च्छन्ति॑ । किम् । उ॒ । दे॒वाः । येषा᳚म् । नवा॑वसान॒मिति॒ नव॑ - अ॒व॒सा॒न॒म् । उपेति॑ । अ॒स्मि॒न्न् । श्वः । य॒क्ष्यमा॑णे । दे॒वताः᳚ । व॒स॒न्ति॒ । यः । ए॒वम् । वि॒द्वान् । अ॒ग्निम् । उ॒प॒स्तृ॒णातीत्यु॑प - स्तृ॒णाति॑ । यज॑मानेन । ग्रा॒म्याः । च॒ । प॒शवः॑ । अ॒व॒रुद्ध्या॒ इत्य॑व-रुद्ध्याः᳚ । आ॒र॒ण्याः । च॒ । इति॑ । आ॒हुः॒ । यत् । ग्रा॒म्यान् । उ॒प॒वस॒तीत्यु॑प - वस॑ति । तेन॑ । ग्रा॒म्यान् । अवेति॑ । रु॒न्धे॒ । यत् । आ॒र॒ण्यस्य॑ । अ॒श्नाति॑ । तेन॑ । आ॒र॒ण्यान् । यत् । अना᳚श्वान् । उ॒प॒वसे॒दित्यु॑प - वसे᳚त् । पि॒तृ॒दे॒व॒त्य॑ इति॑ पितृ - दे॒व॒त्यः॑ । स्या॒त् । आ॒र॒ण्यस्य॑ । अ॒श्ना॒ति॒ । इ॒न्द्रि॒यम् ।  \newline


\textbf{Krama Paata} \newline

इन्नु । न्वै । वा उप॑स्तीर्णम् । उप॑स्तीर्णमि॒च्छन्ति॑ । उप॑स्तीर्ण॒मित्युप॑ - स्ती॒र्ण॒म् । इ॒च्छन्ति॒ किम् । किमु॑ । उ॒ दे॒वाः । दे॒वा येषा᳚म् । येषा॒म् नवा॑वसानम् । नवा॑वसान॒मुप॑ । नवा॑वसान॒मिति॒नव॑ - अ॒व॒सा॒न॒म् । उपा᳚स्मिन्न् । अ॒स्मि॒ञ्छ्वः । श्वो य॒क्ष्यमा॑णे । य॒क्ष्यमा॑णे दे॒वताः᳚ । दे॒वता॑ वसन्ति । व॒स॒न्ति॒ यः । य ए॒वम् । ए॒वं ॅवि॒द्वान् । वि॒द्वान॒ग्निम् । अ॒ग्निमु॑पस्तृ॒णाति॑ । उ॒प॒स्तृ॒णाति॒ यज॑मानेन । उ॒प॒स्तृ॒णातीत्यु॑प - स्तृ॒णाति॑ । यज॑मानेन ग्रा॒म्याः । ग्रा॒म्याश्च॑ । च॒ प॒शवः॑ । प॒शवो॑ऽव॒रुद्ध्याः᳚ । अ॒व॒रुद्ध्या॑ आर॒ण्याः । अ॒व॒रुद्ध्या॒ इत्य॑व - रुद्ध्याः᳚ । आ॒र॒ण्याश्च॑ । चेति॑ । इत्या॑हुः । आ॒हु॒र् यत् । यद् ग्रा॒म्यान् । ग्रा॒म्यानु॑प॒वस॑ति । उ॒प॒वस॑ति॒ तेन॑ । उ॒प॒वस॒तीत्यु॑प - वस॑ति । तेन॑ ग्रा॒म्यान् । ग्रा॒म्यानव॑ । अव॑ रुन्धे । रु॒न्धे॒ यत् । यदा॑र॒ण्यस्य॑ । आ॒र॒ण्यस्या॒श्ञाति॑ । अ॒श्ञाति॒ तेन॑ । तेना॑र॒ण्यान् । आ॒र॒ण्यान्. यत् । यदना᳚श्वान् । अना᳚श्वानुप॒वसे᳚त् । उ॒प॒वसे᳚त् पितृदेव॒त्यः॑ । उ॒प॒वसे॒दित्यु॑प - वसे᳚त् । पि॒तृ॒दे॒व॒त्यः॑ स्यात् । पि॒तृ॒दे॒व॒त्य॑ इति॑ पितृ - दे॒व॒त्यः॑ । स्या॒दा॒र॒ण्यस्य॑ । आ॒र॒ण्यस्या᳚श्ञाति । अ॒श्ञा॒ती॒न्द्रि॒यम् । इ॒न्द्रि॒यं ॅवै \newline

\textbf{Jatai Paata} \newline

1. इन् नु न्विदिन् नु । \newline
2. न्वै वै नु न्वै । \newline
3. वा उप॑स्तीर्ण॒ मुप॑स्तीर्णं॒ ॅवै वा उप॑स्तीर्णम् । \newline
4. उप॑स्तीर्ण मि॒च्छन्ती॒ च्छन्त्युप॑स्तीर्ण॒ मुप॑स्तीर्ण मि॒च्छन्ति॑ । \newline
5. उप॑स्तीर्ण॒मियुप॑ - स्ती॒र्ण॒म् । \newline
6. इ॒च्छन्ति॒ किम् कि मि॒च्छन्ती॒ च्छन्ति॒ किम् । \newline
7. कि मु॑ वु॒ किम् कि मु॑ । \newline
8. उ॒ दे॒वा दे॒वा उ॑ वु दे॒वाः । \newline
9. दे॒वा येषां॒ ॅयेषा᳚म् दे॒वा दे॒वा येषा᳚म् । \newline
10. येषा॒म् नवा॑वसान॒म् नवा॑वसानं॒ ॅयेषां॒ ॅयेषा॒म् नवा॑वसानम् । \newline
11. नवा॑वसान॒ मुपोप॒ नवा॑वसान॒म् नवा॑वसान॒ मुप॑ । \newline
12. नवा॑वसान॒मिति॒ नव॑ - अ॒व॒सा॒न॒म् । \newline
13. उपा᳚स्मिन् नस्मि॒न् नुपोपा᳚स्मिन्न् । \newline
14. अ॒स्मि॒ञ् छ्वः श्वो᳚ ऽस्मिन् नस्मि॒ञ् छ्वः । \newline
15. श्वो य॒क्ष्यमा॑णे य॒क्ष्यमा॑णे॒ श्वः श्वो य॒क्ष्यमा॑णे । \newline
16. य॒क्ष्यमा॑णे दे॒वता॑ दे॒वता॑ य॒क्ष्यमा॑णे य॒क्ष्यमा॑णे दे॒वताः᳚ । \newline
17. दे॒वता॑ वसन्ति वसन्ति दे॒वता॑ दे॒वता॑ वसन्ति । \newline
18. व॒स॒न्ति॒ यो यो व॑सन्ति वसन्ति॒ यः । \newline
19. य ए॒व मे॒वं ॅयो य ए॒वम् । \newline
20. ए॒वं ॅवि॒द्वान्. वि॒द्वा ने॒व मे॒वं ॅवि॒द्वान् । \newline
21. वि॒द्वा न॒ग्नि म॒ग्निं ॅवि॒द्वान्. वि॒द्वा न॒ग्निम् । \newline
22. अ॒ग्नि मु॑पस्तृ॒णा त्यु॑पस्तृ॒णा त्य॒ग्नि म॒ग्नि मु॑पस्तृ॒णाति॑ । \newline
23. उ॒प॒स्तृ॒णाति॒ यज॑मानेन॒ यज॑माने नोपस्तृ॒णा त्यु॑पस्तृ॒णाति॒ यज॑मानेन । \newline
24. उ॒प॒स्तृ॒णातीत्यु॑प - स्तृ॒णाति॑ । \newline
25. यज॑मानेन ग्रा॒म्या ग्रा॒म्या यज॑मानेन॒ यज॑मानेन ग्रा॒म्याः । \newline
26. ग्रा॒म्याश्च॑ च ग्रा॒म्या ग्रा॒म्याश्च॑ । \newline
27. च॒ प॒शवः॑ प॒शव॑श्च च प॒शवः॑ । \newline
28. प॒शवो॑ ऽव॒रुद्ध्या॑ अव॒रुद्ध्याः᳚ प॒शवः॑ प॒शवो॑ ऽव॒रुद्ध्याः᳚ । \newline
29. अ॒व॒रुद्ध्या॑ आर॒ण्या आ॑र॒ण्या अ॑व॒रुद्ध्या॑ अव॒रुद्ध्या॑ आर॒ण्याः । \newline
30. अ॒व॒रुद्ध्या॒ इत्य॑व - रुद्ध्याः᳚ । \newline
31. आ॒र॒ण्याश्च॑ चार॒ण्या आ॑र॒ण्याश्च॑ । \newline
32. चे तीति॑ च॒ चे ति॑ । \newline
33. इत्या॑हु राहु॒ रिती त्या॑हुः । \newline
34. आ॒हु॒र् यद् यदा॑हु राहु॒र् यत् । \newline
35. यद् ग्रा॒म्यान् ग्रा॒म्यान्. यद् यद् ग्रा॒म्यान् । \newline
36. ग्रा॒म्या नु॑प॒वस॑ त्युप॒वस॑ति ग्रा॒म्यान् ग्रा॒म्या नु॑प॒वस॑ति । \newline
37. उ॒प॒वस॑ति॒ तेन॒ तेनो॑प॒वस॑ त्युप॒वस॑ति॒ तेन॑ । \newline
38. उ॒प॒वस॒तीत्यु॑प - वस॑ति । \newline
39. तेन॑ ग्रा॒म्यान् ग्रा॒म्यान् तेन॒ तेन॑ ग्रा॒म्यान् । \newline
40. ग्रा॒म्या नवाव॑ ग्रा॒म्यान् ग्रा॒म्या नव॑ । \newline
41. अव॑ रुन्धे रु॒न्धे ऽवाव॑ रुन्धे । \newline
42. रु॒न्धे॒ यद् यद् रु॑न्धे रुन्धे॒ यत् । \newline
43. यदा॑र॒ण्यस्या॑ र॒ण्यस्य॒ यद् यदा॑र॒ण्यस्य॑ । \newline
44. आ॒र॒ण्यस्या॒ श्ञात्य॒श्ञा त्या॑र॒ण्यस्या॑ र॒ण्यस्या॒ श्ञाति॑ । \newline
45. अ॒श्ञाति॒ तेन॒ तेना॒श्ञा त्य॒श्ञाति॒ तेन॑ । \newline
46. तेना॑र॒ण्या ना॑र॒ण्यान् तेन॒ तेना॑र॒ण्यान् । \newline
47. आ॒र॒ण्यान्. यद् यदा॑र॒ण्या ना॑र॒ण्यान्. यत् । \newline
48. यदना᳚श्वा॒ नना᳚श्वा॒न्॒. यद् यदना᳚श्वान् । \newline
49. अना᳚श्वा नुप॒वसे॑ दुप॒वसे॒ दना᳚श्वा॒ नना᳚श्वा नुप॒वसे᳚त् । \newline
50. उ॒प॒वसे᳚त् पितृदेव॒त्यः॑ पितृदेव॒त्य॑ उप॒वसे॑ दुप॒वसे᳚त् पितृदेव॒त्यः॑ । \newline
51. उ॒प॒वसे॒दित्यु॑प - वसे᳚त् । \newline
52. पि॒तृ॒दे॒व॒त्यः॑ स्याथ् स्यात् पितृदेव॒त्यः॑ पितृदेव॒त्यः॑ स्यात् । \newline
53. पि॒तृ॒दे॒व॒त्य॑ इति॑ पितृ - दे॒व॒त्यः॑ । \newline
54. स्या॒दा॒र॒ण्य स्या॑र॒ण्यस्य॑ स्याथ् स्यादार॒ण्यस्य॑ । \newline
55. आ॒र॒ण्यस्या᳚ श्ञा त्यश्ञा त्यार॒ण्यस्या॑ र॒ण्यस्या᳚ श्ञाति । \newline
56. अ॒श्ञा॒ती॒ न्द्रि॒य मि॑न्द्रि॒य म॑श्ञा त्यश्ञाती न्द्रि॒यम् । \newline
57. इ॒न्द्रि॒यं ॅवै वा इ॑न्द्रि॒य मि॑न्द्रि॒यं ॅवै । \newline

\textbf{Ghana Paata } \newline

1. इन् नु न्विदिन् न्वै वै न्विदिन् न्वै । \newline
2. न्वै वै नु न्वा उप॑स्तीर्ण॒ मुप॑स्तीर्णं॒ ॅवै नु न्वा उप॑स्तीर्णम् । \newline
3. वा उप॑स्तीर्ण॒ मुप॑स्तीर्णं॒ ॅवै वा उप॑स्तीर्ण मि॒च्छन्ती॒ च्छन्त्युप॑स्तीर्णं॒ ॅवै वा उप॑स्तीर्ण मि॒च्छन्ति॑ । \newline
4. उप॑स्तीर्ण मि॒च्छन्ती॒ च्छन्त्युप॑स्तीर्ण॒ मुप॑स्तीर्ण मि॒च्छन्ति॒ किम् कि मि॒च्छन्त्युप॑स्तीर्ण॒ मुप॑स्तीर्ण मि॒च्छन्ति॒ किम् । \newline
5. उप॑स्तीर्ण॒मियुप॑ - स्ती॒र्ण॒म् । \newline
6. इ॒च्छन्ति॒ किम् कि मि॒च्छन्ती॒च्छन्ति॒ कि मु॑ वु॒ कि मि॒च्छन्ती॒च्छन्ति॒ कि मु॑ । \newline
7. कि मु॑ वु॒ किम् कि मु॑ दे॒वा दे॒वा उ॒ किम् कि मु॑ दे॒वाः । \newline
8. उ॒ दे॒वा दे॒वा उ॑ वु दे॒वा येषां॒ ॅयेषा᳚म् दे॒वा उ॑ वु दे॒वा येषा᳚म् । \newline
9. दे॒वा येषां॒ ॅयेषा᳚म् दे॒वा दे॒वा येषा॒म् नवा॑वसान॒म् नवा॑वसानं॒ ॅयेषा᳚म् दे॒वा दे॒वा 
येषा॒म् नवा॑वसानम् । \newline
10. येषा॒म् नवा॑वसान॒म् नवा॑वसानं॒ ॅयेषां॒ ॅयेषा॒म् नवा॑वसान॒ मुपोप॒ नवा॑वसानं॒ ॅयेषां॒ ॅयेषा॒म् नवा॑वसान॒ मुप॑ । \newline
11. नवा॑वसान॒ मुपोप॒ नवा॑वसान॒ न्नवा॑वसान॒ मुपा᳚स्मिन् नस्मि॒न् नुप॒ नवा॑वसान॒ न्नवा॑वसान॒ मुपा᳚स्मिन्न् । \newline
12. नवा॑वसान॒मिति॒ नव॑ - अ॒व॒सा॒न॒म् । \newline
13. उपा᳚स्मिन् नस्मि॒न् नुपोपा᳚स्मि॒ञ् छ्वः श्वो᳚ ऽस्मि॒न् नुपोपा᳚स्मि॒ञ् छ्वः । \newline
14. अ॒स्मि॒ञ् छ्वः श्वो᳚ ऽस्मिन् नस्मि॒ञ् छ्वो य॒क्ष्यमा॑णे य॒क्ष्यमा॑णे॒ श्वो᳚ ऽस्मिन् नस्मि॒ञ् छ्वो य॒क्ष्यमा॑णे । \newline
15. श्वो य॒क्ष्यमा॑णे य॒क्ष्यमा॑णे॒ श्वः श्वो य॒क्ष्यमा॑णे दे॒वता॑ दे॒वता॑ य॒क्ष्यमा॑णे॒ श्वः श्वो य॒क्ष्यमा॑णे दे॒वताः᳚ । \newline
16. य॒क्ष्यमा॑णे दे॒वता॑ दे॒वता॑ य॒क्ष्यमा॑णे य॒क्ष्यमा॑णे दे॒वता॑ वसन्ति वसन्ति दे॒वता॑ य॒क्ष्यमा॑णे य॒क्ष्यमा॑णे दे॒वता॑ वसन्ति । \newline
17. दे॒वता॑ वसन्ति वसन्ति दे॒वता॑ दे॒वता॑ वसन्ति॒ यो यो व॑सन्ति दे॒वता॑ दे॒वता॑ वसन्ति॒ यः । \newline
18. व॒स॒न्ति॒ यो यो व॑सन्ति वसन्ति॒ य ए॒व मे॒वं ॅयो व॑सन्ति वसन्ति॒ य ए॒वम् । \newline
19. य ए॒व मे॒वं ॅयो य ए॒वं ॅवि॒द्वान्. वि॒द्वा ने॒वं ॅयो य ए॒वं ॅवि॒द्वान् । \newline
20. ए॒वं ॅवि॒द्वान्. वि॒द्वा ने॒व मे॒वं ॅवि॒द्वा न॒ग्नि म॒ग्निं ॅवि॒द्वा ने॒व मे॒वं ॅवि॒द्वा न॒ग्निम् । \newline
21. वि॒द्वा न॒ग्नि म॒ग्निं ॅवि॒द्वान्. वि॒द्वा न॒ग्नि मु॑पस्तृ॒णा त्यु॑पस्तृ॒णात्य॒ग्निं ॅवि॒द्वान्. वि॒द्वा न॒ग्नि मु॑पस्तृ॒णाति॑ । \newline
22. अ॒ग्नि मु॑पस्तृ॒णा त्यु॑पस्तृ॒णात्य॒ग्नि म॒ग्नि मु॑पस्तृ॒णाति॒ यज॑मानेन॒ यज॑माने नोपस्तृ॒णात्य॒ग्नि म॒ग्नि मु॑पस्तृ॒णाति॒ यज॑मानेन । \newline
23. उ॒प॒स्तृ॒णाति॒ यज॑मानेन॒ यज॑माने नोपस्तृ॒णा त्यु॑पस्तृ॒णाति॒ यज॑मानेन ग्रा॒म्या ग्रा॒म्या यज॑माने नोपस्तृ॒णा त्यु॑पस्तृ॒णाति॒ यज॑मानेन ग्रा॒म्याः । \newline
24. उ॒प॒स्तृ॒णातीत्यु॑प - स्तृ॒णाति॑ । \newline
25. यज॑मानेन ग्रा॒म्या ग्रा॒म्या यज॑मानेन॒ यज॑मानेन ग्रा॒म्याश्च॑ च ग्रा॒म्या यज॑मानेन॒ यज॑मानेन ग्रा॒म्याश्च॑ । \newline
26. ग्रा॒म्याश्च॑ च ग्रा॒म्या ग्रा॒म्याश्च॑ प॒शवः॑ प॒शव॑श्च ग्रा॒म्या ग्रा॒म्याश्च॑ प॒शवः॑ । \newline
27. च॒ प॒शवः॑ प॒शव॑श्च च प॒शवो॑ ऽव॒रुद्ध्या॑ अव॒रुद्ध्याः᳚ प॒शव॑श्च च प॒शवो॑ ऽव॒रुद्ध्याः᳚ । \newline
28. प॒शवो॑ ऽव॒रुद्ध्या॑ अव॒रुद्ध्याः᳚ प॒शवः॑ प॒शवो॑ ऽव॒रुद्ध्या॑ आर॒ण्या आ॑र॒ण्या अ॑व॒रुद्ध्याः᳚ प॒शवः॑ प॒शवो॑ ऽव॒रुद्ध्या॑ आर॒ण्याः । \newline
29. अ॒व॒रुद्ध्या॑ आर॒ण्या आ॑र॒ण्या अ॑व॒रुद्ध्या॑ अव॒रुद्ध्या॑ आर॒ण्याश्च॑ चार॒ण्या अ॑व॒रुद्ध्या॑ अव॒रुद्ध्या॑ आर॒ण्याश्च॑ । \newline
30. अ॒व॒रुद्ध्या॒ इत्य॑व - रुद्ध्याः᳚ । \newline
31. आ॒र॒ण्याश्च॑ चार॒ण्या आ॑र॒ण्याश्चे तीति॑ चार॒ण्या आ॑र॒ण्याश्चे ति॑ । \newline
32. चे तीति॑ च॒ चे त्या॑हु राहु॒रिति॑ च॒ चे त्या॑हुः । \newline
33. इत्या॑हु राहु॒ रितीत्या॑हु॒र् यद् यदा॑हु॒ रितीत्या॑हु॒र् यत् । \newline
34. आ॒हु॒र् यद् यदा॑हुराहु॒र् यद् ग्रा॒म्यान् ग्रा॒म्यान्. यदा॑हुराहु॒र् यद् ग्रा॒म्यान् । \newline
35. यद् ग्रा॒म्यान् ग्रा॒म्यान्. यद् यद् ग्रा॒म्या नु॑प॒वस॑त्युप॒वस॑ति ग्रा॒म्यान्. यद् यद् ग्रा॒म्या नु॑प॒वस॑ति । \newline
36. ग्रा॒म्या नु॑प॒वस॑त्युप॒वस॑ति ग्रा॒म्यान् ग्रा॒म्या नु॑प॒वस॑ति॒ तेन॒ तेनो॑प॒वस॑ति ग्रा॒म्यान् ग्रा॒म्या नु॑प॒वस॑ति॒ तेन॑ । \newline
37. उ॒प॒वस॑ति॒ तेन॒ तेनो॑प॒वस॑ त्युप॒वस॑ति॒ तेन॑ ग्रा॒म्यान् ग्रा॒म्यान् तेनो॑प॒वस॑ त्युप॒वस॑ति॒ तेन॑ ग्रा॒म्यान् । \newline
38. उ॒प॒वस॒तीत्यु॑प - वस॑ति । \newline
39. तेन॑ ग्रा॒म्यान् ग्रा॒म्यान् तेन॒ तेन॑ ग्रा॒म्या नवाव॑ ग्रा॒म्यान् तेन॒ तेन॑ ग्रा॒म्या नव॑ । \newline
40. ग्रा॒म्या नवाव॑ ग्रा॒म्यान् ग्रा॒म्या नव॑ रुन्धे रु॒न्धे ऽव॑ ग्रा॒म्यान् ग्रा॒म्या नव॑ रुन्धे । \newline
41. अव॑ रुन्धे रु॒न्धे ऽवाव॑ रुन्धे॒ यद् यद् रु॒न्धे ऽवाव॑ रुन्धे॒ यत् । \newline
42. रु॒न्धे॒ यद् यद् रु॑न्धे रुन्धे॒ यदा॑र॒ण्यस्या॑र॒ण्यस्य॒ यद् रु॑न्धे रुन्धे॒ यदा॑र॒ण्यस्य॑ । \newline
43. यदा॑र॒ण्य स्या॑र॒ण्यस्य॒ यद् यदा॑र॒ण्य स्या॒श्ञा त्य॒श्ञात्या॑र॒ण्यस्य॒ यद् यदा॑र॒ण्य स्या॒श्ञाति॑ । \newline
44. आ॒र॒ण्य स्या॒श्ञा त्य॒श्ञा त्या॑र॒ण्यस्या॑ र॒ण्यस्या॒श्ञाति॒ तेन॒ तेना॒श्ञा त्या॑र॒ण्यस्या॑ र॒ण्यस्या॒श्ञाति॒ तेन॑ । \newline
45. अ॒श्ञाति॒ तेन॒ तेना॒श्ञा त्य॒श्ञाति॒ तेना॑र॒ण्या ना॑र॒ण्यान् तेना॒श्ञा त्य॒श्ञाति॒ तेना॑र॒ण्यान् । \newline
46. तेना॑र॒ण्या ना॑र॒ण्यान् तेन॒ तेना॑र॒ण्यान्. यद् यदा॑र॒ण्यान् तेन॒ तेना॑र॒ण्यान्. यत् । \newline
47. आ॒र॒ण्यान्. यद् यदा॑र॒ण्या ना॑र॒ण्यान्. यदना᳚श्वा॒ नना᳚श्वा॒न्.॒ यदा॑र॒ण्या ना॑र॒ण्यान्. यदना᳚श्वान् । \newline
48. यदना᳚श्वा॒ नना᳚श्वा॒न्॒. यद् यदना᳚श्वा नुप॒वसे॑ दुप॒वसे॒ दना᳚श्वा॒न्॒. यद् यदना᳚श्वा नुप॒वसे᳚त् । \newline
49. अना᳚श्वा नुप॒वसे॑ दुप॒वसे॒ दना᳚श्वा॒ नना᳚श्वा नुप॒वसे᳚त् पितृदेव॒त्यः॑ पितृदेव॒त्य॑ उप॒वसे॒ दना᳚श्वा॒ नना᳚श्वा नुप॒वसे᳚त् पितृदेव॒त्यः॑ । \newline
50. उ॒प॒वसे᳚त् पितृदेव॒त्यः॑ पितृदेव॒त्य॑ उप॒वसे॑दुप॒वसे᳚त् पितृदेव॒त्यः॑ स्याथ् स्यात् पितृदेव॒त्य॑ उप॒वसे॑दुप॒वसे᳚त् पितृदेव॒त्यः॑ स्यात् । \newline
51. उ॒प॒वसे॒दित्यु॑प - वसे᳚त् । \newline
52. पि॒तृ॒दे॒व॒त्यः॑ स्याथ् स्यात् पितृदेव॒त्यः॑ पितृदेव॒त्यः॑ स्यादार॒ण्य स्या॑र॒ण्यस्य॑ स्यात् पितृदेव॒त्यः॑ पितृदेव॒त्यः॑ स्यादार॒ण्यस्य॑ । \newline
53. पि॒तृ॒दे॒व॒त्य॑ इति॑ पितृ - दे॒व॒त्यः॑ । \newline
54. स्या॒दा॒र॒ण्य स्या॑र॒ण्यस्य॑ स्याथ् स्यादार॒ण्य स्या᳚श्ञात्यश्ञा त्यार॒ण्यस्य॑ स्याथ् स्यादार॒ण्य स्या᳚श्ञाति । \newline
55. आ॒र॒ण्य स्या᳚श्ञा त्यश्ञा त्यार॒ण्यस्या॑ र॒ण्यस्या᳚श्ञा तीन्द्रि॒य मि॑न्द्रि॒य म॑श्ञा त्यार॒ण्यस्या॑ र॒ण्यस्या᳚श्ञा तीन्द्रि॒यम् । \newline
56. अ॒श्ञा॒ती॒न्द्रि॒य मि॑न्द्रि॒य म॑श्ञात्यश्ञा तीन्द्रि॒यं ॅवै वा इ॑न्द्रि॒य म॑श्ञात्यश्ञा तीन्द्रि॒यं ॅवै । \newline
57. इ॒न्द्रि॒यं ॅवै वा इ॑न्द्रि॒य मि॑न्द्रि॒यं ॅवा आ॑र॒ण्य मा॑र॒ण्यं ॅवा इ॑न्द्रि॒य मि॑न्द्रि॒यं ॅवा आ॑र॒ण्यम् । \newline
\pagebreak
\markright{ TS 1.6.7.4  \hfill https://www.vedavms.in \hfill}

\section{ TS 1.6.7.4 }

\textbf{TS 1.6.7.4 } \newline
\textbf{Samhita Paata} \newline

ॅवा आ॑र॒ण्यमि॑न्द्रि॒यमे॒वाऽऽत्मन् ध॑त्ते॒ यदना᳚श्वानुप॒वसे॒त् क्षोधु॑कः स्या॒द्यद॑श्ञी॒याद्रु॒-द्रो᳚ऽस्य प॒शून॒भि म॑न्येता॒ऽपो᳚ऽश्ञाति॒ तन्नेवा॑शि॒तं नेवाऽन॑शितं॒ न क्षोधु॑को॒ भव॑ति॒ नास्य॑ रु॒द्रः प॒शून॒भि म॑न्यते॒ वज्रो॒ वै य॒ज्ञ्ः क्षुत् खलु॒ वै म॑नु॒ष्य॑स्य॒ भ्रातृ॑व्यो॒ यदना᳚ऽश्वानुप॒वस॑ति॒ वज्रे॑णै॒व सा॒क्षात् ( ) क्षुधं॒ भ्रातृ॑व्यꣳ हन्ति ॥ \newline

\textbf{Pada Paata} \newline

वै । आ॒र॒ण्यम् । इ॒न्द्रि॒यम् । ए॒व । आ॒त्मन्न् । ध॒त्ते॒ । यत् । अना᳚श्वान् । उ॒प॒वसे॒दित्यु॑प - वसे᳚त् । क्षोधु॑कः । स्या॒त् । यत् । अ॒श्नी॒यात् । रु॒द्रः । अ॒स्य॒ । प॒शून् । अ॒भीति॑ । म॒न्ये॒त॒ । अ॒पः । अ॒श्ना॒ति॒ । तत् । न । इ॒व॒ । अ॒शि॒तम् । न । इ॒व॒ । अन॑शितम् । न । क्षोधु॑कः । भव॑ति । न । अ॒स्य॒ । रु॒द्रः । प॒शून् । अ॒भीति॑ । म॒न्य॒ते॒ । वज्रः॑ । वै । य॒ज्ञ्ः । क्षुत् । खलु॑ । वै । म॒नु॒ष्य॑स्य । भ्रातृ॑व्यः । यत् । अना᳚श्वान् । उ॒प॒वस॒तीत्यु॑प-वस॑ति । वज्रे॑ण । ए॒व । सा॒क्षादिति॑ स-अ॒क्षात् ( ) । क्षुध᳚म् । भ्रातृ॑व्यम् । ह॒न्ति॒ ॥  \newline


\textbf{Krama Paata} \newline

वा आ॑र॒ण्यम् । आ॒र॒ण्यमि॑न्द्रि॒यम् । इ॒न्द्रि॒यमे॒व । ए॒वात्मन्न् । आ॒त्मन् ध॑त्ते । ध॒त्ते॒ यत् । यदना᳚श्वान् । अना᳚श्वानुप॒वसे᳚त् । उ॒प॒वसे॒त्,क्षोधु॑कः । उ॒प॒वसे॒दित्यु॑प - वसे᳚त् । क्षोधु॑कः स्यात् । स्या॒द् यत् । यद॑श्ञी॒यात् । अ॒श्ञी॒याद् रु॒द्रः । रु॒द्रो᳚ऽस्य । अ॒स्य॒ प॒शून् । प॒शून॒भि । अ॒भि म॑न्येत । म॒न्ये॒ता॒पः । अ॒पो᳚ऽश्ञाति । अ॒श्ञा॒ति॒ तत् । तन्न । नेव॑ । इ॒वा॒शि॒तम् । अ॒शि॒तम् न । नेव॑ । इ॒वान॑शितम् । अन॑शित॒म् न । न क्षोधु॑कः । क्षोधु॑को॒ भव॑ति । भव॑ति॒ न । नास्य॑ । अ॒स्य॒ रु॒द्रः । रु॒द्रः प॒शून् । प॒शून॒भि । अ॒भि म॑न्यते । म॒न्य॒ते॒ वज्रः॑ । वज्रो॒ वै । वै य॒ज्ञ्ः । य॒ज्ञ्ः क्षुत् । क्षुत् खलु॑ । खलु॒ वै । वै म॑नु॒ष्य॑स्य । म॒नु॒ष्य॑स्य॒ भ्रातृ॑व्यः । भ्रातृ॑व्यो॒ यत् । यदना᳚श्वान् । अना᳚श्वानुप॒वस॑ति । उ॒प॒वस॑ति॒ वज्रे॑ण । उ॒प॒वस॒तीत्यु॑प - वस॑ति । वज्रे॑णै॒व । ए॒व सा॒क्षात् ( ) । सा॒क्षात्,क्षुध᳚म् । सा॒क्षादिति॑ स - अ॒क्षात् । क्षुध॒म् भ्रातृ॑व्यम् । भ्रातृ॑व्यꣳ हन्ति । ह॒न्तीति॑ हन्ति । \newline

\textbf{Jatai Paata} \newline

1. वा आ॑र॒ण्य मा॑र॒ण्यं ॅवै वा आ॑र॒ण्यम् । \newline
2. आ॒र॒ण्य मि॑न्द्रि॒य मि॑न्द्रि॒य मा॑र॒ण्य मा॑र॒ण्य मि॑न्द्रि॒यम् । \newline
3. इ॒न्द्रि॒य मे॒वैवे न्द्रि॒य मि॑न्द्रि॒य मे॒व । \newline
4. ए॒वात्मन् ना॒त्मन् ने॒वैवात्मन्न् । \newline
5. आ॒त्मन् ध॑त्ते धत्त आ॒त्मन् ना॒त्मन् ध॑त्ते । \newline
6. ध॒त्ते॒ यद् यद् ध॑त्ते धत्ते॒ यत् । \newline
7. यदना᳚श्वा॒ नना᳚श्वा॒न्॒. यद् यदना᳚श्वान् । \newline
8. अना᳚श्वा नुप॒वसे॑ दुप॒वसे॒ दना᳚श्वा॒ नना᳚श्वा नुप॒वसे᳚त् । \newline
9. उ॒प॒वसे॒त् क्षोधु॑कः॒ क्षोधु॑क उप॒वसे॑ दुप॒वसे॒त् क्षोधु॑कः । \newline
10. उ॒प॒वसे॒दित्यु॑प - वसे᳚त् । \newline
11. क्षोधु॑कः स्याथ् स्या॒त् क्षोधु॑कः॒ क्षोधु॑कः स्यात् । \newline
12. स्या॒द् यद् यथ् स्या᳚थ् स्या॒द् यत् । \newline
13. यद॑श्ञी॒या द॑श्ञी॒याद् यद् यद॑श्ञी॒यात् । \newline
14. अ॒श्ञी॒याद् रु॒द्रो रु॒द्रो᳚ ऽश्ञी॒या द॑श्ञी॒याद् रु॒द्रः । \newline
15. रु॒द्रो᳚ ऽस्यास्य रु॒द्रो रु॒द्रो᳚ ऽस्य । \newline
16. अ॒स्य॒ प॒शून् प॒शू न॑स्यास्य प॒शून् । \newline
17. प॒शू न॒भ्य॑भि प॒शून् प॒शू न॒भि । \newline
18. अ॒भि म॑न्येत मन्येता॒भ्य॑भि म॑न्येत । \newline
19. म॒न्ये॒ता॒पो॑ ऽपो म॑न्येत मन्येता॒पः । \newline
20. अ॒पो᳚ ऽश्ञा त्यश्ञात्य॒पो᳚(1॒) ऽपो᳚ ऽश्ञाति । \newline
21. अ॒श्ञा॒ति॒ तत् तद॑श्ञा त्यश्ञाति॒ तत् । \newline
22. तन् न न तत् तन् न । \newline
23. ने वे॑ व॒ न ने व॑ । \newline
24. इ॒वा॒शि॒त म॑शि॒त मि॑वे वाशि॒तम् । \newline
25. अ॒शि॒तम् न नाशि॒त म॑शि॒तम् न । \newline
26. ने वे॑ व॒ न ने व॑ । \newline
27. इ॒वान॑शित॒ मन॑शित मिवे॒ वान॑शितम् । \newline
28. अन॑शित॒म् न नान॑शित॒ मन॑शित॒म् न । \newline
29. न क्षोधु॑कः॒ क्षोधु॑को॒ न न क्षोधु॑कः । \newline
30. क्षोधु॑को॒ भव॑ति॒ भव॑ति॒ क्षोधु॑कः॒ क्षोधु॑को॒ भव॑ति । \newline
31. भव॑ति॒ न न भव॑ति॒ भव॑ति॒ न । \newline
32. नास्या᳚स्य॒ न नास्य॑ । \newline
33. अ॒स्य॒ रु॒द्रो रु॒द्रो᳚ ऽस्यास्य रु॒द्रः । \newline
34. रु॒द्रः प॒शून् प॒शून् रु॒द्रो रु॒द्रः प॒शून् । \newline
35. प॒शू न॒भ्य॑भि प॒शून् प॒शू न॒भि । \newline
36. अ॒भि म॑न्यते मन्यते॒ ऽभ्य॑भि म॑न्यते । \newline
37. म॒न्य॒ते॒ वज्रो॒ वज्रो॑ मन्यते मन्यते॒ वज्रः॑ । \newline
38. वज्रो॒ वै वै वज्रो॒ वज्रो॒ वै । \newline
39. वै य॒ज्ञो य॒ज्ञो वै वै य॒ज्ञ्ः । \newline
40. य॒ज्ञ्ः क्षुत् क्षुद् य॒ज्ञो य॒ज्ञ्ः क्षुत् । \newline
41. क्षुत् खलु॒ खलु॒ क्षुत् क्षुत् खलु॑ । \newline
42. खलु॒ वै वै खलु॒ खलु॒ वै । \newline
43. वै म॑नु॒ष्य॑स्य मनु॒ष्य॑स्य॒ वै वै म॑नु॒ष्य॑स्य । \newline
44. म॒नु॒ष्य॑स्य॒ भ्रातृ॑व्यो॒ भ्रातृ॑व्यो मनु॒ष्य॑स्य मनु॒ष्य॑स्य॒ भ्रातृ॑व्यः । \newline
45. भ्रातृ॑व्यो॒ यद् यद् भ्रातृ॑व्यो॒ भ्रातृ॑व्यो॒ यत् । \newline
46. यदना᳚श्वा॒ नना᳚श्वा॒न्॒. यद् यदना᳚श्वान् । \newline
47. अना᳚श्वा नुप॒वस॑ त्युप॒वस॒ त्यना᳚श्वा॒ नना᳚श्वा नुप॒वस॑ति । \newline
48. उ॒प॒वस॑ति॒ वज्रे॑ण॒ वज्रे॑णो प॒वस॑ त्युप॒वस॑ति॒ वज्रे॑ण । \newline
49. उ॒प॒वस॒तीत्यु॑प - वस॑ति । \newline
50. वज्रे॑णै॒ वैव वज्रे॑ण॒ वज्रे॑ णै॒व । \newline
51. ए॒व सा॒क्षाथ् सा॒क्षा दे॒वैव सा॒क्षात् । \newline
52. सा॒क्षात् क्षुध॒म् क्षुध(ग्म्॑) सा॒क्षाथ् सा॒क्षात् क्षुध᳚म् । \newline
53. सा॒क्षादिति॑ स - अ॒क्षात् । \newline
54. क्षुध॒म् भ्रातृ॑व्य॒म् भ्रातृ॑व्य॒म् क्षुध॒म् क्षुध॒म् भ्रातृ॑व्यम् । \newline
55. भ्रातृ॑व्यꣳ हन्ति हन्ति॒ भ्रातृ॑व्य॒म् भ्रातृ॑व्यꣳ हन्ति । \newline
56. ह॒न्तीति॑ हन्ति । \newline

\textbf{Ghana Paata } \newline

1. वा आ॑र॒ण्य मा॑र॒ण्यं ॅवै वा आ॑र॒ण्य मि॑न्द्रि॒य मि॑न्द्रि॒य मा॑र॒ण्यं ॅवै वा आ॑र॒ण्य मि॑न्द्रि॒यम् । \newline
2. आ॒र॒ण्य मि॑न्द्रि॒य मि॑न्द्रि॒य मा॑र॒ण्य मा॑र॒ण्य मि॑न्द्रि॒य मे॒वैवे न्द्रि॒य मा॑र॒ण्य मा॑र॒ण्य मि॑न्द्रि॒य मे॒व । \newline
3. इ॒न्द्रि॒य मे॒वैवे न्द्रि॒य मि॑न्द्रि॒य मे॒वात्मन् ना॒त्मन् ने॒वे न्द्रि॒य मि॑न्द्रि॒य मे॒वात्मन्न् । \newline
4. ए॒वात्मन् ना॒त्मन् ने॒वैवात्मन् ध॑त्ते धत्त आ॒त्मन् ने॒वैवात्मन् ध॑त्ते । \newline
5. आ॒त्मन् ध॑त्ते धत्त आ॒त्मन् ना॒त्मन् ध॑त्ते॒ यद् यद् ध॑त्त आ॒त्मन् ना॒त्मन् ध॑त्ते॒ यत् । \newline
6. ध॒त्ते॒ यद् यद् ध॑त्ते धत्ते॒ यदना᳚श्वा॒ नना᳚श्वा॒न्॒. यद् ध॑त्ते धत्ते॒ यदना᳚श्वान् । \newline
7. यदना᳚श्वा॒ नना᳚श्वा॒न्॒. यद् यदना᳚श्वा नुप॒वसे॑ दुप॒वसे॒ दना᳚श्वा॒न्॒. यद् यदना᳚श्वा नुप॒वसे᳚त् । \newline
8. अना᳚श्वा नुप॒वसे॑ दुप॒वसे॒ दना᳚श्वा॒ नना᳚श्वा नुप॒वसे॒त् क्षोधु॑कः॒ क्षोधु॑क उप॒वसे॒ दना᳚श्वा॒ नना᳚श्वा नुप॒वसे॒त् क्षोधु॑कः । \newline
9. उ॒प॒वसे॒त् क्षोधु॑कः॒ क्षोधु॑क उप॒वसे॑ दुप॒वसे॒त् क्षोधु॑कः स्याथ् स्या॒त् क्षोधु॑क उप॒वसे॑ दुप॒वसे॒त् क्षोधु॑कः स्यात् । \newline
10. उ॒प॒वसे॒दित्यु॑प - वसे᳚त् । \newline
11. क्षोधु॑कः स्याथ् स्या॒त् क्षोधु॑कः॒ क्षोधु॑कः स्या॒द् यद् यथ् स्या॒त् क्षोधु॑कः॒ क्षोधु॑कः स्या॒द् यत् । \newline
12. स्या॒द् यद् यथ् स्या᳚थ् स्या॒द् यद॑श्ञी॒या द॑श्ञी॒याद् यथ् स्या᳚थ् स्या॒द् यद॑श्ञी॒यात् । \newline
13. यद॑श्ञी॒या द॑श्ञी॒याद् यद् यद॑श्ञी॒याद् रु॒द्रो रु॒द्रो᳚ ऽश्ञी॒याद् यद् यद॑श्ञी॒याद् रु॒द्रः । \newline
14. अ॒श्ञी॒याद् रु॒द्रो रु॒द्रो᳚ ऽश्ञी॒या द॑श्ञी॒याद् रु॒द्रो᳚ ऽस्यास्य रु॒द्रो᳚ ऽश्ञी॒या द॑श्ञी॒याद् रु॒द्रो᳚ ऽस्य । \newline
15. रु॒द्रो᳚ ऽस्यास्य रु॒द्रो रु॒द्रो᳚ ऽस्य प॒शून् प॒शू न॑स्य रु॒द्रो रु॒द्रो᳚ ऽस्य प॒शून् । \newline
16. अ॒स्य॒ प॒शून् प॒शू न॑स्यास्य प॒शू न॒भ्य॑भि प॒शू न॑स्यास्य प॒शू न॒भि । \newline
17. प॒शू न॒भ्य॑भि प॒शून् प॒शू न॒भि म॑न्येत मन्येता॒भि प॒शून् प॒शू न॒भि म॑न्येत । \newline
18. अ॒भि म॑न्येत मन्येता॒भ्य॑भि म॑न्येता॒पो॑ ऽपो म॑न्येता॒भ्य॑भि म॑न्येता॒पः । \newline
19. म॒न्ये॒ता॒पो॑ ऽपो म॑न्येत मन्येता॒पो᳚ ऽश्ञात्यश्ञात्य॒पो म॑न्येत मन्येता॒पो᳚ ऽश्ञाति । \newline
20. अ॒पो᳚ ऽश्ञात्यश्ञात्य॒पो᳚(1॒) ऽपो᳚ ऽश्ञाति॒ तत् तद॑श्ञात्य॒पो᳚(1॒) ऽपो᳚ ऽश्ञाति॒ तत् । \newline
21. अ॒श्ञा॒ति॒ तत् तद॑श्ञा त्यश्ञाति॒ तन् न न तद॑श्ञा त्यश्ञाति॒ तन् न । \newline
22. तन् न न तत् तन् ने वे॑ व॒ न तत् तन् ने व॑ । \newline
23. ने वे॑ व॒ न ने वा॑शि॒त म॑शि॒त मि॑व॒ न ने वा॑शि॒तम् । \newline
24. इ॒वा॒शि॒त म॑शि॒त मि॑वे वाशि॒तम् न नाशि॒त मि॑वे वाशि॒तम् न । \newline
25. अ॒शि॒तम् न नाशि॒त म॑शि॒तम् ने वे॑ व॒ नाशि॒त म॑शि॒तम् ने व॑ । \newline
26. ने वे॑ व॒ न ने वान॑शित॒ मन॑शित मिव॒ न ने वान॑शितम् । \newline
27. इ॒वान॑शित॒ मन॑शित मिवे॒ वान॑शित॒म् न नान॑शित मिवे॒ वान॑शित॒म् न । \newline
28. अन॑शित॒म् न नान॑शित॒ मन॑शित॒म् न क्षोधु॑कः॒ क्षोधु॑को॒ नान॑शित॒ मन॑शित॒न्न क्षोधु॑कः । \newline
29. न क्षोधु॑कः॒ क्षोधु॑को॒ न न क्षोधु॑को॒ भव॑ति॒ भव॑ति॒ क्षोधु॑को॒ न न क्षोधु॑को॒ भव॑ति । \newline
30. क्षोधु॑को॒ भव॑ति॒ भव॑ति॒ क्षोधु॑कः॒ क्षोधु॑को॒ भव॑ति॒ न न भव॑ति॒ क्षोधु॑कः॒ क्षोधु॑को॒ भव॑ति॒ न । \newline
31. भव॑ति॒ न न भव॑ति॒ भव॑ति॒ नास्या᳚स्य॒ न भव॑ति॒ भव॑ति॒ नास्य॑ । \newline
32. नास्या᳚स्य॒ न नास्य॑ रु॒द्रो रु॒द्रो᳚ ऽस्य॒ न नास्य॑ रु॒द्रः । \newline
33. अ॒स्य॒ रु॒द्रो रु॒द्रो᳚ ऽस्यास्य रु॒द्रः प॒शून् प॒शून् रु॒द्रो᳚ ऽस्यास्य रु॒द्रः प॒शून् । \newline
34. रु॒द्रः प॒शून् प॒शून् रु॒द्रो रु॒द्रः प॒शू न॒भ्य॑भि प॒शून् रु॒द्रो रु॒द्रः प॒शू न॒भि । \newline
35. प॒शू न॒भ्य॑भि प॒शून् प॒शू न॒भि म॑न्यते मन्यते॒ ऽभि प॒शून् प॒शू न॒भि म॑न्यते । \newline
36. अ॒भि म॑न्यते मन्यते॒ ऽभ्य॑भि म॑न्यते॒ वज्रो॒ वज्रो॑ मन्यते॒ ऽभ्य॑भि म॑न्यते॒ वज्रः॑ । \newline
37. म॒न्य॒ते॒ वज्रो॒ वज्रो॑ मन्यते मन्यते॒ वज्रो॒ वै वै वज्रो॑ मन्यते मन्यते॒ वज्रो॒ वै । \newline
38. वज्रो॒ वै वै वज्रो॒ वज्रो॒ वै य॒ज्ञो य॒ज्ञो वै वज्रो॒ वज्रो॒ वै य॒ज्ञ्ः । \newline
39. वै य॒ज्ञो य॒ज्ञो वै वै य॒ज्ञ्ः क्षुत् क्षुद् य॒ज्ञो वै वै य॒ज्ञ्ः क्षुत् । \newline
40. य॒ज्ञ्ः क्षुत् क्षुद् य॒ज्ञो य॒ज्ञ्ः क्षुत् खलु॒ खलु॒ क्षुद् य॒ज्ञो य॒ज्ञ्ः क्षुत् खलु॑ । \newline
41. क्षुत् खलु॒ खलु॒ क्षुत् क्षुत् खलु॒ वै वै खलु॒ क्षुत् क्षुत् खलु॒ वै । \newline
42. खलु॒ वै वै खलु॒ खलु॒ वै म॑नु॒ष्य॑स्य मनु॒ष्य॑स्य॒ वै खलु॒ खलु॒ वै म॑नु॒ष्य॑स्य । \newline
43. वै म॑नु॒ष्य॑स्य मनु॒ष्य॑स्य॒ वै वै म॑नु॒ष्य॑स्य॒ भ्रातृ॑व्यो॒ भ्रातृ॑व्यो मनु॒ष्य॑स्य॒ वै वै म॑नु॒ष्य॑स्य॒ भ्रातृ॑व्यः । \newline
44. म॒नु॒ष्य॑स्य॒ भ्रातृ॑व्यो॒ भ्रातृ॑व्यो मनु॒ष्य॑स्य मनु॒ष्य॑स्य॒ भ्रातृ॑व्यो॒ यद् यद् भ्रातृ॑व्यो मनु॒ष्य॑स्य मनु॒ष्य॑स्य॒ भ्रातृ॑व्यो॒ यत् । \newline
45. भ्रातृ॑व्यो॒ यद् यद् भ्रातृ॑व्यो॒ भ्रातृ॑व्यो॒ यदना᳚श्वा॒ नना᳚श्वा॒न्॒. यद् भ्रातृ॑व्यो॒ भ्रातृ॑व्यो॒ यदना᳚श्वान् । \newline
46. यदना᳚श्वा॒ नना᳚श्वा॒न्॒. यद् यदना᳚श्वा नुप॒वस॑ त्युप॒वस॒ त्यना᳚श्वा॒न्॒. यद् यदना᳚श्वा नुप॒वस॑ति । \newline
47. अना᳚श्वा नुप॒वस॑ त्युप॒वस॒त्यना᳚श्वा॒ नना᳚श्वा नुप॒वस॑ति॒ वज्रे॑ण॒ वज्रे॑णोप॒वस॒ त्यना᳚श्वा॒ नना᳚श्वा नुप॒वस॑ति॒ वज्रे॑ण । \newline
48. उ॒प॒वस॑ति॒ वज्रे॑ण॒ वज्रे॑णोप॒वस॑ त्युप॒वस॑ति॒ वज्रे॑णै॒वैव वज्रे॑णोप॒वस॑ त्युप॒वस॑ति॒ वज्रे॑णै॒व । \newline
49. उ॒प॒वस॒तीत्यु॑प - वस॑ति । \newline
50. वज्रे॑णै॒वैव वज्रे॑ण॒ वज्रे॑णै॒व सा॒क्षाथ् सा॒क्षादे॒व वज्रे॑ण॒ वज्रे॑णै॒व सा॒क्षात् । \newline
51. ए॒व सा॒क्षाथ् सा॒क्षादे॒वैव सा॒क्षात् क्षुध॒म् क्षुध(ग्म्॑) सा॒क्षादे॒वैव सा॒क्षात् क्षुध᳚म् । \newline
52. सा॒क्षात् क्षुध॒म् क्षुध(ग्म्॑) सा॒क्षाथ् सा॒क्षात् क्षुध॒म् भ्रातृ॑व्य॒म् भ्रातृ॑व्य॒म् क्षुध(ग्म्॑) सा॒क्षाथ् सा॒क्षात् क्षुध॒म् भ्रातृ॑व्यम् । \newline
53. सा॒क्षादिति॑ स - अ॒क्षात् । \newline
54. क्षुध॒म् भ्रातृ॑व्य॒म् भ्रातृ॑व्य॒म् क्षुध॒म् क्षुध॒म् भ्रातृ॑व्यꣳ हन्ति हन्ति॒ भ्रातृ॑व्य॒म् क्षुध॒म् क्षुध॒म् भ्रातृ॑व्यꣳ हन्ति । \newline
55. भ्रातृ॑व्यꣳ हन्ति हन्ति॒ भ्रातृ॑व्य॒म् भ्रातृ॑व्यꣳ हन्ति । \newline
56. ह॒न्तीति॑ हन्ति । \newline
\pagebreak
\markright{ TS 1.6.8.1  \hfill https://www.vedavms.in \hfill}

\section{ TS 1.6.8.1 }

\textbf{TS 1.6.8.1 } \newline
\textbf{Samhita Paata} \newline

यो वै श्र॒द्धामना॑रभ्य य॒ज्ञेन॒ यज॑ते॒ नास्ये॒ष्टाय॒ श्रद्द॑धते॒ऽपः प्र ण॑यति श्र॒द्धा वा आपः॑ श्र॒द्धामे॒वाऽऽरभ्य॑ य॒ज्ञेन॑ यजत उ॒भये᳚ऽस्य देवमनु॒ष्या इ॒ष्टाय॒ श्रद्द॑धते॒ तदा॑हु॒रति॒ वा ए॒ता वर्त्रं॑ नेद॒न्त्यति॒ वाचं॒ मनो॒ वावैता नाति॑ नेद॒न्तीति॒ मन॑सा॒ प्र ण॑यती॒यं ॅवै मनो॒ - [ ] \newline

\textbf{Pada Paata} \newline

यः । वै । श्र॒द्धामिति॑ श्रत् - धाम् । अना॑र॒भ्येत्यना᳚ - र॒भ्य॒ । य॒ज्ञेन॑ । यज॑ते । न । अ॒स्य॒ । इ॒ष्टाय॑ । श्रत् । द॒ध॒ते॒ । अ॒पः । प्रेति॑ । न॒य॒ति॒ । श्र॒द्धेति॑ श्रत् - धा । वै । आपः॑ । श्र॒द्धामिति॑ श्रत् - धाम् । ए॒व । आ॒रभ्येत्या᳚ - रभ्य॑ । य॒ज्ञेन॑ । य॒ज॒ते॒ । उ॒भये᳚ । अ॒स्य॒ । दे॒व॒म॒नु॒ष्या इति॑ देव - म॒नु॒ष्याः । इ॒ष्टाय॑ । श्रत् । द॒ध॒ते॒ । तत् । आ॒हुः॒ । अतीति॑ । वै । ए॒ताः । वर्त्र᳚म् । ने॒द॒न्ति॒ । अतीति॑ । वाच᳚म् । मनः॑ । वाव । ए॒ताः । न । अतीति॑ । ने॒द॒न्ति॒ । इति॑ । मन॑सा । प्रेति॑ । न॒य॒ति॒ । इ॒यम् । वै । मनः॑ ।  \newline


\textbf{Krama Paata} \newline

यो वै । वै श्र॒द्धाम् । श्र॒द्धामना॑रभ्य । श्र॒द्धामिति॑ श्रत् - धाम् । अना॑रभ्य य॒ज्ञेन॑ । अना॑र॒भ्येत्यना᳚ - र॒भ्य॒ । य॒ज्ञेन॒ यज॑ते । यज॑ते॒ न । नास्य॑ । अ॒स्ये॒ष्टाय॑ । इ॒ष्टाय॒ श्रत् । श्रद् द॑धते । द॒ध॒ते॒ऽपः । अ॒पः प्र । प्र ण॑यति । न॒य॒ति॒ श्र॒द्धा । श्र॒द्धा वै । श्र॒द्धेति॑ श्रत् - धा । वा आपः॑ । आपः॑ श्र॒द्धाम् । श्र॒द्धामे॒व । श्र॒द्धामिति॑ श्रत् - धाम् । ए॒वारभ्य॑ । आ॒रभ्य॑ य॒ज्ञेन॑ । आ॒रभ्येत्या᳚ - रभ्य॑ । य॒ज्ञेन॑ यजते । य॒ज॒त॒ उ॒भये᳚ । उ॒भये᳚ऽस्य । अ॒स्य॒ दे॒व॒म॒नु॒ष्याः । दे॒व॒म॒नु॒ष्या इ॒ष्टाय॑ । दे॒व॒म॒नु॒ष्या इति॑ देव - म॒नु॒ष्याः । इ॒ष्टाय॒ श्रत् । श्रद् द॑धते । द॒ध॒ते॒ तत् । तदा॑हुः । आ॒हु॒रति॑ । अति॒ वै । वा ए॒ताः । ए॒ता वर्त्र᳚म् । वर्त्र॑म् नेदन्ति । ने॒द॒न्त्यति॑ । अति॒ वाच᳚म् । वाच॒म् मनः॑ । मनो॒ वाव । वावैताः । ए॒ता न । नाति॑ । अति॑ नेदन्ति । ने॒द॒न्तीति॑ । इति॒ मन॑सा । मन॑सा॒ प्र । प्र ण॑यति । न॒य॒ती॒यम् । इ॒यं ॅवै । वै मनः॑ । मनो॒ऽनया᳚ \newline

\textbf{Jatai Paata} \newline

1. यो वै वै यो यो वै । \newline
2. वै श्र॒द्धाꣳ श्र॒द्धां ॅवै वै श्र॒द्धाम् । \newline
3. श्र॒द्धा मना॑र॒भ्या ना॑रभ्य श्र॒द्धाꣳ श्र॒द्धा मना॑रभ्य । \newline
4. श्र॒द्धामिति॑ श्रत् - धाम् । \newline
5. अना॑रभ्य य॒ज्ञेन॑ य॒ज्ञेना ना॑र॒भ्या ना॑रभ्य य॒ज्ञेन॑ । \newline
6. अना॑र॒भ्येत्यना᳚ - र॒भ्य॒ । \newline
7. य॒ज्ञेन॒ यज॑ते॒ यज॑ते य॒ज्ञेन॑ य॒ज्ञेन॒ यज॑ते । \newline
8. यज॑ते॒ न न यज॑ते॒ यज॑ते॒ न । \newline
9. नास्या᳚स्य॒ न नास्य॑ । \newline
10. अ॒स्ये॒ ष्टाये॒ ष्टाया᳚स्यास्ये॒ ष्टाय॑ । \newline
11. इ॒ष्टाय॒ श्रच्छ्रदि॒ष्टाये॒ ष्टाय॒ श्रत् । \newline
12. श्रद् द॑धते दधते॒ श्रच्छ्रद् द॑धते । \newline
13. द॒ध॒ते॒ ऽपो॑ ऽपो द॑धते दधते॒ ऽपः । \newline
14. अ॒पः प्र प्रापो॑ ऽपः प्र । \newline
15. प्र ण॑यति नयति॒ प्र प्र ण॑यति । \newline
16. न॒य॒ति॒ श्र॒द्धा श्र॒द्धा न॑यति नयति श्र॒द्धा । \newline
17. श्र॒द्धा वै वै श्र॒द्धा श्र॒द्धा वै । \newline
18. श्र॒द्धेति॑ श्रत् - धा । \newline
19. वा आप॒ आपो॒ वै वा आपः॑ । \newline
20. आपः॑ श्र॒द्धाꣳ श्र॒द्धा माप॒ आपः॑ श्र॒द्धाम् । \newline
21. श्र॒द्धा मे॒वैव श्र॒द्धाꣳ श्र॒द्धा मे॒व । \newline
22. श्र॒द्धामिति॑ श्रत् - धाम् । \newline
23. ए॒वा रभ्या॒ रभ्यै॒ वैवारभ्य॑ । \newline
24. आ॒रभ्य॑ य॒ज्ञेन॑ य॒ज्ञेना॒ रभ्या॒रभ्य॑ य॒ज्ञेन॑ । \newline
25. आ॒रभ्येत्या᳚ - रभ्य॑ । \newline
26. य॒ज्ञेन॑ यजते यजते य॒ज्ञेन॑ य॒ज्ञेन॑ यजते । \newline
27. य॒ज॒त॒ उ॒भय॑ उ॒भये॑ यजते यजत उ॒भये᳚ । \newline
28. उ॒भये᳚ ऽस्यास्यो॒भय॑ उ॒भये᳚ ऽस्य । \newline
29. अ॒स्य॒ दे॒व॒म॒नु॒ष्या दे॑वमनु॒ष्या अ॑स्यास्य देवमनु॒ष्याः । \newline
30. दे॒व॒म॒नु॒ष्या इ॒ष्टाये॒ ष्टाय॑ देवमनु॒ष्या दे॑वमनु॒ष्या इ॒ष्टाय॑ । \newline
31. दे॒व॒म॒नु॒ष्या इति॑ देव - म॒नु॒ष्याः । \newline
32. इ॒ष्टाय॒ श्रच्छ्रदि॒ष्टाये॒ ष्टाय॒ श्रत् । \newline
33. श्रद् द॑धते दधते॒ श्रच्छ्रद् द॑धते । \newline
34. द॒ध॒ते॒ तत् तद् द॑धते दधते॒ तत् । \newline
35. तदा॑हु राहु॒ स्तत् तदा॑हुः । \newline
36. आ॒हु॒ रत्यत्या॑हु राहु॒रति॑ । \newline
37. अति॒ वै वा अत्यति॒ वै । \newline
38. वा ए॒ता ए॒ता वै वा ए॒ताः । \newline
39. ए॒ता वर्त्रं॒ ॅवर्त्र॑ मे॒ता ए॒ता वर्त्र᳚म् । \newline
40. वर्त्र॑म् नेदन्ति नेदन्ति॒ वर्त्रं॒ ॅवर्त्र॑म् नेदन्ति । \newline
41. ने॒द॒ न्त्यत्यति॑ नेदन्ति नेद॒न्त्यति॑ । \newline
42. अति॒ वाचं॒ ॅवाच॒ मत्यति॒ वाच᳚म् । \newline
43. वाच॒म् मनो॒ मनो॒ वाचं॒ ॅवाच॒म् मनः॑ । \newline
44. मनो॒ वाव वाव मनो॒ मनो॒ वाव । \newline
45. वावैता ए॒ता वाव वावैताः । \newline
46. ए॒ता न नैता ए॒ता न । \newline
47. नात्यति॒ न नाति॑ । \newline
48. अति॑ नेदन्ति नेद॒ न्त्यत्यति॑ नेदन्ति । \newline
49. ने॒द॒ न्तीतीति॑ नेदन्ति नेद॒न्तीति॑ । \newline
50. इति॒ मन॑सा॒ मन॒सेतीति॒ मन॑सा । \newline
51. मन॑सा॒ प्र प्र मन॑सा॒ मन॑सा॒ प्र । \newline
52. प्र ण॑यति नयति॒ प्र प्र ण॑यति । \newline
53. न॒य॒ती॒य मि॒यम् न॑यति नयती॒यम् । \newline
54. इ॒यं ॅवै वा इ॒य मि॒यं ॅवै । \newline
55. वै मनो॒ मनो॒ वै वै मनः॑ । \newline
56. मनो॒ ऽनया॒ ऽनया॒ मनो॒ मनो॒ ऽनया᳚ । \newline

\textbf{Ghana Paata } \newline

1. यो वै वै यो यो वै श्र॒द्धाꣳ श्र॒द्धां ॅवै यो यो वै श्र॒द्धाम् । \newline
2. वै श्र॒द्धाꣳ श्र॒द्धां ॅवै वै श्र॒द्धा मना॑र॒भ्या ना॑रभ्य श्र॒द्धां ॅवै वै श्र॒द्धा मना॑रभ्य । \newline
3. श्र॒द्धा मना॑र॒भ्या ना॑रभ्य श्र॒द्धाꣳ श्र॒द्धा मना॑रभ्य य॒ज्ञेन॑ य॒ज्ञेना ना॑रभ्य श्र॒द्धाꣳ श्र॒द्धा मना॑रभ्य य॒ज्ञेन॑ । \newline
4. श्र॒द्धामिति॑ श्रत् - धाम् । \newline
5. अना॑रभ्य य॒ज्ञेन॑ य॒ज्ञेनाना॑ र॒भ्याना॑रभ्य य॒ज्ञेन॒ यज॑ते॒ यज॑ते य॒ज्ञेनाना॑ र॒भ्याना॑रभ्य य॒ज्ञेन॒ यज॑ते । \newline
6. अना॑र॒भ्येत्यना᳚ - र॒भ्य॒ । \newline
7. य॒ज्ञेन॒ यज॑ते॒ यज॑ते य॒ज्ञेन॑ य॒ज्ञेन॒ यज॑ते॒ न न यज॑ते य॒ज्ञेन॑ य॒ज्ञेन॒ यज॑ते॒ न । \newline
8. यज॑ते॒ न न यज॑ते॒ यज॑ते॒ नास्या᳚स्य॒ न यज॑ते॒ यज॑ते॒ नास्य॑ । \newline
9. नास्या᳚स्य॒ न नास्ये॒ ष्टाये॒ ष्टाया᳚स्य॒ न नास्ये॒ ष्टाय॑ । \newline
10. अ॒स्ये॒ ष्टाये॒ ष्टाया᳚स्यास्ये॒ ष्टाय॒ श्रच्छ्र दि॒ष्टाया᳚स्यास्ये॒ ष्टाय॒ श्रत् । \newline
11. इ॒ष्टाय॒ श्रच्छ्रदि॒ष्टाये॒ ष्टाय॒ श्रद् द॑धते दधते॒ श्रदि॒ष्टाये॒ ष्टाय॒ श्रद् द॑धते । \newline
12. श्रद् द॑धते दधते॒ श्रच्छ्रद् द॑धते॒ ऽपो॑ ऽपो द॑धते॒ श्रच्छ्रद् द॑धते॒ ऽपः । \newline
13. द॒ध॒ते॒ ऽपो॑ ऽपो द॑धते दधते॒ ऽपः प्र प्रापो द॑धते दधते॒ ऽपः प्र । \newline
14. अ॒पः प्र प्रापो॑ ऽपः प्र ण॑यति नयति॒ प्रापो॑ ऽपः प्र ण॑यति । \newline
15. प्र ण॑यति नयति॒ प्र प्र ण॑यति श्र॒द्धा श्र॒द्धा न॑यति॒ प्र प्र ण॑यति श्र॒द्धा । \newline
16. न॒य॒ति॒ श्र॒द्धा श्र॒द्धा न॑यति नयति श्र॒द्धा वै वै श्र॒द्धा न॑यति नयति श्र॒द्धा वै । \newline
17. श्र॒द्धा वै वै श्र॒द्धा श्र॒द्धा वा आप॒ आपो॒ वै श्र॒द्धा श्र॒द्धा वा आपः॑ । \newline
18. श्र॒द्धेति॑ श्रत् - धा । \newline
19. वा आप॒ आपो॒ वै वा आपः॑ श्र॒द्धाꣳ श्र॒द्धा मापो॒ वै वा आपः॑ श्र॒द्धाम् । \newline
20. आपः॑ श्र॒द्धाꣳ श्र॒द्धा माप॒ आपः॑ श्र॒द्धा मे॒वैव श्र॒द्धा माप॒ आपः॑ श्र॒द्धा मे॒व । \newline
21. श्र॒द्धा मे॒वैव श्र॒द्धाꣳ श्र॒द्धा मे॒वा रभ्या॒रभ्यै॒व श्र॒द्धाꣳ श्र॒द्धा मे॒वारभ्य॑ । \newline
22. श्र॒द्धामिति॑ श्रत् - धाम् । \newline
23. ए॒वारभ्या॒ रभ्यै॒वैवारभ्य॑ य॒ज्ञेन॑ य॒ज्ञेना॒ रभ्यै॒वैवारभ्य॑ य॒ज्ञेन॑ । \newline
24. आ॒रभ्य॑ य॒ज्ञेन॑ य॒ज्ञे ना॒रभ्या॒रभ्य॑ य॒ज्ञेन॑ यजते यजते य॒ज्ञेना॒रभ्या॒रभ्य॑ य॒ज्ञेन॑ यजते । \newline
25. आ॒रभ्येत्या᳚ - रभ्य॑ । \newline
26. य॒ज्ञेन॑ यजते यजते य॒ज्ञेन॑ य॒ज्ञेन॑ यजत उ॒भय॑ उ॒भये॑ यजते य॒ज्ञेन॑ य॒ज्ञेन॑ यजत उ॒भये᳚ । \newline
27. य॒ज॒त॒ उ॒भय॑ उ॒भये॑ यजते यजत उ॒भये᳚ ऽस्यास्यो॒भये॑ यजते यजत उ॒भये᳚ ऽस्य । \newline
28. उ॒भये᳚ ऽस्यास्यो॒भय॑ उ॒भये᳚ ऽस्य देवमनु॒ष्या दे॑वमनु॒ष्या अ॑स्यो॒भय॑ उ॒भये᳚ ऽस्य देवमनु॒ष्याः । \newline
29. अ॒स्य॒ दे॒व॒म॒नु॒ष्या दे॑वमनु॒ष्या अ॑स्यास्य देवमनु॒ष्या इ॒ष्टाये॒ ष्टाय॑ देवमनु॒ष्या अ॑स्यास्य देवमनु॒ष्या इ॒ष्टाय॑ । \newline
30. दे॒व॒म॒नु॒ष्या इ॒ष्टाये॒ ष्टाय॑ देवमनु॒ष्या दे॑वमनु॒ष्या इ॒ष्टाय॒ श्रच्छ्रदि॒ष्टाय॑ देवमनु॒ष्या दे॑वमनु॒ष्या इ॒ष्टाय॒ श्रत् । \newline
31. दे॒व॒म॒नु॒ष्या इति॑ देव - म॒नु॒ष्याः । \newline
32. इ॒ष्टाय॒ श्रच्छ्रदि॒ष्टाये॒ ष्टाय॒ श्रद् द॑धते दधते॒ श्रदि॒ष्टाये॒ ष्टाय॒ श्रद् द॑धते । \newline
33. श्रद् द॑धते दधते॒ श्रच्छ्रद् द॑धते॒ तत् तद् द॑धते॒ श्रच्छ्रद् द॑धते॒ तत् । \newline
34. द॒ध॒ते॒ तत् तद् द॑धते दधते॒ तदा॑हु राहु॒स्तद् द॑धते दधते॒ तदा॑हुः । \newline
35. तदा॑हु राहु॒स्तत् तदा॑हु॒ रत्यत्या॑हु॒स्तत् तदा॑हु॒रति॑ । \newline
36. आ॒हु॒ रत्यत्या॑हु राहु॒रति॒ वै वा अत्या॑हु राहु॒रति॒ वै । \newline
37. अति॒ वै वा अत्यति॒ वा ए॒ता ए॒ता वा अत्यति॒ वा ए॒ताः । \newline
38. वा ए॒ता ए॒ता वै वा ए॒ता वर्त्रं॒ ॅवर्त्र॑ मे॒ता वै वा ए॒ता वर्त्र᳚म् । \newline
39. ए॒ता वर्त्रं॒ ॅवर्त्र॑ मे॒ता ए॒ता वर्त्र॑म् नेदन्ति नेदन्ति॒ वर्त्र॑ मे॒ता ए॒ता वर्त्र॑म् नेदन्ति । \newline
40. वर्त्र॑म् नेदन्ति नेदन्ति॒ वर्त्रं॒ ॅवर्त्र॑म् नेद॒न्त्यत्यति॑ नेदन्ति॒ वर्त्रं॒ ॅवर्त्र॑म् नेद॒न्त्यति॑ । \newline
41. ने॒द॒न्त्यत्यति॑ नेदन्ति नेद॒न्त्यति॒ वाचं॒ ॅवाच॒ मति॑ नेदन्ति नेद॒न्त्यति॒ वाच᳚म् । \newline
42. अति॒ वाचं॒ ॅवाच॒ मत्यति॒ वाच॒म् मनो॒ मनो॒ वाच॒ मत्यति॒ वाच॒म् मनः॑ । \newline
43. वाच॒म् मनो॒ मनो॒ वाचं॒ ॅवाच॒म् मनो॒ वाव वाव मनो॒ वाचं॒ ॅवाच॒म् मनो॒ वाव । \newline
44. मनो॒ वाव वाव मनो॒ मनो॒ वावैता ए॒ता वाव मनो॒ मनो॒ वावैताः । \newline
45. वावैता ए॒ता वाव वावैता न नैता वाव वावैता न । \newline
46. ए॒ता न नैता ए॒ता नात्यति॒ नैता ए॒ता नाति॑ । \newline
47. नात्यति॒ न नाति॑ नेदन्ति नेद॒न्त्यति॒ न नाति॑ नेदन्ति । \newline
48. अति॑ नेदन्ति नेद॒न्त्यत्यति॑ नेद॒न्तीतीति॑ नेद॒न्त्यत्यति॑ नेद॒न्तीति॑ । \newline
49. ने॒द॒न्तीतीति॑ नेदन्ति नेद॒न्तीति॒ मन॑सा॒ मन॒सेति॑ नेदन्ति नेद॒न्तीति॒ मन॑सा । \newline
50. इति॒ मन॑सा॒ मन॒सेतीति॒ मन॑सा॒ प्र प्र मन॒सेतीति॒ मन॑सा॒ प्र । \newline
51. मन॑सा॒ प्र प्र मन॑सा॒ मन॑सा॒ प्र ण॑यति नयति॒ प्र मन॑सा॒ मन॑सा॒ प्र ण॑यति । \newline
52. प्र ण॑यति नयति॒ प्र प्र ण॑यती॒य मि॒यन्न॑यति॒ प्र प्र ण॑यती॒यम् । \newline
53. न॒य॒ती॒य मि॒यम् न॑यति नयती॒यं ॅवै वा इ॒यम् न॑यति नयती॒यं ॅवै । \newline
54. इ॒यं ॅवै वा इ॒य मि॒यं ॅवै मनो॒ मनो॒ वा इ॒य मि॒यं ॅवै मनः॑ । \newline
55. वै मनो॒ मनो॒ वै वै मनो॒ ऽनया॒ ऽनया॒ मनो॒ वै वै मनो॒ ऽनया᳚ । \newline
56. मनो॒ ऽनया॒ ऽनया॒ मनो॒ मनो॒ ऽनयै॒वैवानया॒ मनो॒ मनो॒ ऽनयै॒व । \newline
\pagebreak
\markright{ TS 1.6.8.2  \hfill https://www.vedavms.in \hfill}

\section{ TS 1.6.8.2 }

\textbf{TS 1.6.8.2 } \newline
\textbf{Samhita Paata} \newline

ऽनयै॒वैनाः॒ प्र ण॑य॒त्य-स्क॑न्नहविर् भवति॒ य ए॒वं ॅवेद॑ यज्ञायु॒धानि॒ सं भ॑रति य॒ज्ञो वै य॑ज्ञायु॒धानि॑ य॒ज्ञ्मे॒व तथ्सं भ॑रति॒ यदेक॑मेकꣳ सं॒भरे᳚त्-पितृदेव॒त्या॑नि स्यु॒र्यथ् स॒ह सर्वा॑णि मानु॒षाणि॒ द्वेद्वे॒ संभ॑रति याज्यानुवा॒क्य॑योरे॒व रू॒पं क॑रो॒त्यथो॑ मिथु॒नमे॒वयो वै दश॑ यज्ञायु॒धानि॒ वेद॑ मुख॒तो᳚ऽस्य य॒ज्ञ्ः क॑ल्पते॒ स्फ्यः- [ ] \newline

\textbf{Pada Paata} \newline

अ॒नया᳚ । ए॒व । ए॒नाः॒ । प्रेति॑ । न॒य॒ति॒ । अस्क॑न्नहवि॒रित्यस्क॑न्न - ह॒विः॒ । भ॒व॒ति॒ । यः । ए॒वम् । वेद॑ । य॒ज्ञा॒यु॒धानीति॑ यज्ञ् - आ॒यु॒धानि॑ । समिति॑ । भ॒र॒ति॒ । य॒ज्ञ्ः । वै । य॒ज्ञा॒यु॒धानीति॑ यज्ञ् - आ॒यु॒धानि॑ । य॒ज्ञ्म् । ए॒व । तत् । समिति॑ । भ॒र॒ति॒ । यत् । एक॑मेक॒मित्येक᳚म् - ए॒क॒म् । स॒भंरे॒दिति॑ सं-भरे᳚त् । पि॒तृ॒दे॒व॒त्या॑नीति॑ पितृ - दे॒व॒त्या॑नि । स्युः॒ । यत् । स॒ह । सर्वा॑णि । मा॒नु॒षाणि॑ । द्वेद्वे॒ इति॒ द्वे - द्वे॒ । समिति॑ । भ॒र॒ति॒ । या॒ज्या॒नु॒वा॒क्य॑यो॒रिति॑ याज्या - अ॒नु॒वा॒क्य॑योः । ए॒व । रू॒पम् । क॒रो॒ति॒ । अथो॒ इति॑ । मि॒थु॒नम् । ए॒व । यः । वै । दश॑ । य॒ज्ञा॒यु॒धानीति॑ यज्ञ् - आ॒यु॒धानि॑ । वेद॑ । मु॒ख॒तः । अ॒स्य॒ । य॒ज्ञ्ः । क॒ल्प॒ते॒ । स्फ्यः ।  \newline


\textbf{Krama Paata} \newline

अ॒नयै॒व । ए॒वैनाः᳚ । ए॒नाः॒ प्र । प्र ण॑यति । न॒य॒त्यस्क॑न्नहविः । अस्क॑न्नहविर् भवति । अस्क॑न्नहवि॒रित्यस्क॑न्न - ह॒विः॒ । भ॒व॒ति॒ यः । य ए॒वम् । ए॒वं ॅवेद॑ । वेद॑ यज्ञायु॒धानि॑ । य॒ज्ञा॒यु॒धानि॒ सम् । य॒ज्ञा॒यु॒धानीति॑ यज्ञ् - आ॒यु॒धानि॑ । सम् भ॑रति । भ॒र॒ति॒ य॒ज्ञ्ः । य॒ज्ञो वै । वै य॑ज्ञायु॒धानि॑ । य॒ज्ञा॒यु॒धानि॑ य॒ज्ञ्म् । य॒ज्ञा॒यु॒धानीति॑ यज्ञ् - आ॒यु॒धानि॑ । य॒ज्ञ्मे॒व । ए॒व तत् । तथ् सम् । सम् भ॑रति । भ॒र॒ति॒ यत् । यदेक॑मेकम् । एक॑मेकꣳ स॒म्भरे᳚त् । एक॑मेक॒मित्येकं᳚ - ए॒क॒म् । स॒म्भरे᳚त्,पितृदेव॒त्या॑नि । स॒म्भरे॒दिति॑ सं - भरे᳚त् । पि॒तृ॒दे॒व॒त्या॑नि स्युः । पि॒तृ॒दे॒व॒त्या॑नीति॑ पितृ - दे॒व॒त्या॑नि । स्यु॒र् यत् । यथ् स॒ह । स॒ह सर्वा॑णि । सर्वा॑णि मानु॒षाणि॑ । मा॒नु॒षाणि॒ द्वेद्वे᳚ । द्वेद्वे॒ सम् । द्वेद्वे॒ इति॒ द्वे - द्वे॒ । सम् भ॑रति । भ॒र॒ति॒ या॒ज्या॒नु॒वा॒क्य॑योः । या॒ज्या॒नु॒वा॒क्य॑योरे॒व । या॒ज्या॒नु॒वा॒क्य॑यो॒रिति॑ याज्या - अ॒नु॒वा॒क्य॑योः । ए॒व रू॒पम् । रू॒पम् क॑रोति । क॒रो॒त्यथो᳚ । अथो॑ मिथु॒नम् । अथो॒ इत्यथो᳚ । मि॒थु॒नमे॒व । ए॒व यः । यो वै । वै दश॑ । दश॑ यज्ञायु॒धानि॑ । य॒ज्ञा॒यु॒धानि॒ वेद॑ । य॒ज्ञा॒यु॒धानीति॑ यज्ञ् - आ॒यु॒धानि॑ । वेद॑ मुख॒तः । मु॒ख॒तो᳚ऽस्य । अ॒स्य॒ य॒ज्ञ्ः । य॒ज्ञ्ः क॑ल्पते । क॒ल्प॒ते॒ स्फ्यः । स्फ्यश्च॑ \newline

\textbf{Jatai Paata} \newline

1. अ॒नयै॒ वैवानया॒ ऽनयै॒व । \newline
2. ए॒वैना॑ एना ए॒वैवैनाः᳚ । \newline
3. ए॒नाः॒ प्र प्रैना॑ एनाः॒ प्र । \newline
4. प्र ण॑यति नयति॒ प्र प्र ण॑यति । \newline
5. न॒य॒ त्यस्क॑न्नहवि॒ रस्क॑न्नहविर् नयति नय॒ त्यस्क॑न्नहविः । \newline
6. अस्क॑न्नहविर् भवति भव॒ त्यस्क॑न्नहवि॒ रस्क॑न्नहविर् भवति । \newline
7. अस्क॑न्नहवि॒रित्यस्क॑न्न - ह॒विः॒ । \newline
8. भ॒व॒ति॒ यो यो भ॑वति भवति॒ यः । \newline
9. य ए॒व मे॒वं ॅयो य ए॒वम् । \newline
10. ए॒वं ॅवेद॒ वेदै॒व मे॒वं ॅवेद॑ । \newline
11. वेद॑ यज्ञायु॒धानि॑ यज्ञायु॒धानि॒ वेद॒ वेद॑ यज्ञायु॒धानि॑ । \newline
12. य॒ज्ञा॒यु॒धानि॒ सꣳ सं ॅय॑ज्ञायु॒धानि॑ यज्ञायु॒धानि॒ सम् । \newline
13. य॒ज्ञा॒यु॒धानीति॑ यज्ञ् - आ॒यु॒धानि॑ । \newline
14. सम् भ॑रति भरति॒ सꣳ सम् भ॑रति । \newline
15. भ॒र॒ति॒ य॒ज्ञो य॒ज्ञो भ॑रति भरति य॒ज्ञ्ः । \newline
16. य॒ज्ञो वै वै य॒ज्ञो य॒ज्ञो वै । \newline
17. वै य॑ज्ञायु॒धानि॑ यज्ञायु॒धानि॒ वै वै य॑ज्ञायु॒धानि॑ । \newline
18. य॒ज्ञा॒यु॒धानि॑ य॒ज्ञ्ं ॅय॒ज्ञ्ं ॅय॑ज्ञायु॒धानि॑ यज्ञायु॒धानि॑ य॒ज्ञ्म् । \newline
19. य॒ज्ञा॒यु॒धानीति॑ यज्ञ् - आ॒यु॒धानि॑ । \newline
20. य॒ज्ञ् मे॒वैव य॒ज्ञ्ं ॅय॒ज्ञ् मे॒व । \newline
21. ए॒व तत् तदे॒वैव तत् । \newline
22. तथ् सꣳ सम् तत् तथ् सम् । \newline
23. सम् भ॑रति भरति॒ सꣳ सम् भ॑रति । \newline
24. भ॒र॒ति॒ यद् यद् भ॑रति भरति॒ यत् । \newline
25. यदेक॑ मेक॒ मेक॑मेकं॒ ॅयद् यदेक॑ मेकम् । \newline
26. एक॑मेकꣳ सं॒भरे᳚थ् सं॒भरे॒ देक॑मेक॒ मेक॑मेकꣳ सं॒भरे᳚त् । \newline
27. एक॑मेक॒मित्येक᳚म् - ए॒क॒म् । \newline
28. सं॒भरे᳚त् पितृदेव॒त्या॑नि पितृदेव॒त्या॑नि सं॒भरे᳚थ् सं॒भरे᳚त् पितृदेव॒त्या॑नि । \newline
29. सं॒भरे॒दिति॑ सं - भरे᳚त् । \newline
30. पि॒तृ॒दे॒व॒त्या॑नि स्युः स्युः पितृदेव॒त्या॑नि पितृदेव॒त्या॑नि स्युः । \newline
31. पि॒तृ॒दे॒व॒त्या॑नीति॑ पितृ - दे॒व॒त्या॑नि । \newline
32. स्यु॒र् यद् यथ् स्युः॑ स्यु॒र् यत् । \newline
33. यथ् स॒ह स॒ह यद् यथ् स॒ह । \newline
34. स॒ह सर्वा॑णि॒ सर्वा॑णि स॒ह स॒ह सर्वा॑णि । \newline
35. सर्वा॑णि मानु॒षाणि॑ मानु॒षाणि॒ सर्वा॑णि॒ सर्वा॑णि मानु॒षाणि॑ । \newline
36. मा॒नु॒षाणि॒ द्वेद्वे॒ द्वेद्वे॑ मानु॒षाणि॑ मानु॒षाणि॒ द्वेद्वे᳚ । \newline
37. द्वेद्वे॒ सꣳ सम् द्वेद्वे॒ द्वेद्वे॒ सम् । \newline
38. द्वेद्वे॒ इति॒ द्वे - द्वे॒ । \newline
39. सम् भ॑रति भरति॒ सꣳ सम् भ॑रति । \newline
40. भ॒र॒ति॒ या॒ज्या॒नु॒वा॒क्य॑योर् याज्यानुवा॒क्य॑योर् भरति भरति याज्यानुवा॒क्य॑योः । \newline
41. या॒ज्या॒नु॒वा॒क्य॑यो रे॒वैव या᳚ज्यानुवा॒क्य॑योर् याज्यानुवा॒क्य॑यो रे॒व । \newline
42. या॒ज्या॒नु॒वा॒क्य॑यो॒रिति॑ याज्या - अ॒नु॒वा॒क्य॑योः । \newline
43. ए॒व रू॒पꣳ रू॒प मे॒वैव रू॒पम् । \newline
44. रू॒पम् क॑रोति करोति रू॒पꣳ रू॒पम् क॑रोति । \newline
45. क॒रो॒त्यथो॒ अथो॑ करोति करो॒त्यथो᳚ । \newline
46. अथो॑ मिथु॒नम् मि॑थु॒न मथो॒ अथो॑ मिथु॒नम् । \newline
47. अथो॒ इत्यथो᳚ । \newline
48. मि॒थु॒न मे॒वैव मि॑थु॒नम् मि॑थु॒न मे॒व । \newline
49. ए॒व यो य ए॒वैव यः । \newline
50. यो वै वै यो यो वै । \newline
51. वै दश॒ दश॒ वै वै दश॑ । \newline
52. दश॑ यज्ञायु॒धानि॑ यज्ञायु॒धानि॒ दश॒ दश॑ यज्ञायु॒धानि॑ । \newline
53. य॒ज्ञा॒यु॒धानि॒ वेद॒ वेद॑ यज्ञायु॒धानि॑ यज्ञायु॒धानि॒ वेद॑ । \newline
54. य॒ज्ञा॒यु॒धानीति॑ यज्ञ् - आ॒यु॒धानि॑ । \newline
55. वेद॑ मुख॒तो मु॑ख॒तो वेद॒ वेद॑ मुख॒तः । \newline
56. मु॒ख॒तो᳚ ऽस्यास्य मुख॒तो मु॑ख॒तो᳚ ऽस्य । \newline
57. अ॒स्य॒ य॒ज्ञो य॒ज्ञो᳚ ऽस्यास्य य॒ज्ञ्ः । \newline
58. य॒ज्ञ्ः क॑ल्पते कल्पते य॒ज्ञो य॒ज्ञ्ः क॑ल्पते । \newline
59. क॒ल्प॒ते॒ स्फ्यः स्फ्यः क॑ल्पते कल्पते॒ स्फ्यः । \newline
60. स्फ्यश्च॑ च॒ स्फ्यः स्फ्यश्च॑ । \newline

\textbf{Ghana Paata } \newline

1. अ॒नयै॒वैवानया॒ ऽनयै॒वैना॑ एना ए॒वानया॒ ऽनयै॒वैनाः᳚ । \newline
2. ए॒वैना॑ एना ए॒वैवैनाः॒ प्र प्रैना॑ ए॒वैवैनाः॒ प्र । \newline
3. ए॒नाः॒ प्र प्रैना॑ एनाः॒ प्र ण॑यति नयति॒ प्रैना॑ एनाः॒ प्र ण॑यति । \newline
4. प्र ण॑यति नयति॒ प्र प्र ण॑य॒त्यस्क॑न्नहवि॒ रस्क॑न्नहविर् नयति॒ प्र प्र ण॑य॒त्यस्क॑न्नहविः । \newline
5. न॒य॒त्यस्क॑न्नहवि॒ रस्क॑न्नहविर् नयति नय॒त्यस्क॑न्नहविर् भवति भव॒त्यस्क॑न्नहविर् नयति नय॒त्यस्क॑न्नहविर् भवति । \newline
6. अस्क॑न्नहविर् भवति भव॒त्यस्क॑न्नहवि॒ रस्क॑न्नहविर् भवति॒ यो यो भ॑व॒त्यस्क॑न्नहवि॒ रस्क॑न्नहविर् भवति॒ यः । \newline
7. अस्क॑न्नहवि॒रित्यस्क॑न्न - ह॒विः॒ । \newline
8. भ॒व॒ति॒ यो यो भ॑वति भवति॒ य ए॒व मे॒वं ॅयो भ॑वति भवति॒ य ए॒वम् । \newline
9. य ए॒व मे॒वं ॅयो य ए॒वं ॅवेद॒ वेदै॒वं ॅयो य ए॒वं ॅवेद॑ । \newline
10. ए॒वं ॅवेद॒ वेदै॒व मे॒वं ॅवेद॑ यज्ञायु॒धानि॑ यज्ञायु॒धानि॒ वेदै॒व मे॒वं ॅवेद॑ यज्ञायु॒धानि॑ । \newline
11. वेद॑ यज्ञायु॒धानि॑ यज्ञायु॒धानि॒ वेद॒ वेद॑ यज्ञायु॒धानि॒ सꣳ सं ॅय॑ज्ञायु॒धानि॒ वेद॒ वेद॑ यज्ञायु॒धानि॒ सम् । \newline
12. य॒ज्ञा॒यु॒धानि॒ सꣳ सं ॅय॑ज्ञायु॒धानि॑ यज्ञायु॒धानि॒ सम् भ॑रति भरति॒ सं ॅय॑ज्ञायु॒धानि॑ यज्ञायु॒धानि॒ सम् भ॑रति । \newline
13. य॒ज्ञा॒यु॒धानीति॑ यज्ञ् - आ॒यु॒धानि॑ । \newline
14. सम् भ॑रति भरति॒ सꣳ सम् भ॑रति य॒ज्ञो य॒ज्ञो भ॑रति॒ सꣳ सम् भ॑रति य॒ज्ञ्ः । \newline
15. भ॒र॒ति॒ य॒ज्ञो य॒ज्ञो भ॑रति भरति य॒ज्ञो वै वै य॒ज्ञो भ॑रति भरति य॒ज्ञो वै । \newline
16. य॒ज्ञो वै वै य॒ज्ञो य॒ज्ञो वै य॑ज्ञायु॒धानि॑ यज्ञायु॒धानि॒ वै य॒ज्ञो य॒ज्ञो वै य॑ज्ञायु॒धानि॑ । \newline
17. वै य॑ज्ञायु॒धानि॑ यज्ञायु॒धानि॒ वै वै य॑ज्ञायु॒धानि॑ य॒ज्ञ्ं ॅय॒ज्ञ्ं ॅय॑ज्ञायु॒धानि॒ वै वै य॑ज्ञायु॒धानि॑ य॒ज्ञ्म् । \newline
18. य॒ज्ञा॒यु॒धानि॑ य॒ज्ञ्ं ॅय॒ज्ञ्ं ॅय॑ज्ञायु॒धानि॑ यज्ञायु॒धानि॑ य॒ज्ञ् मे॒वैव य॒ज्ञ्ं ॅय॑ज्ञायु॒धानि॑ यज्ञायु॒धानि॑ य॒ज्ञ् मे॒व । \newline
19. य॒ज्ञा॒यु॒धानीति॑ यज्ञ् - आ॒यु॒धानि॑ । \newline
20. य॒ज्ञ् मे॒वैव य॒ज्ञ्ं ॅय॒ज्ञ् मे॒व तत् तदे॒व य॒ज्ञ्ं ॅय॒ज्ञ् मे॒व तत् । \newline
21. ए॒व तत् तदे॒वैव तथ् सꣳ सम् तदे॒वैव तथ् सम् । \newline
22. तथ् सꣳ सम् तत् तथ् सम् भ॑रति भरति॒ सम् तत् तथ् सम् भ॑रति । \newline
23. सम् भ॑रति भरति॒ सꣳ सम् भ॑रति॒ यद् यद् भ॑रति॒ सꣳ सम् भ॑रति॒ यत् । \newline
24. भ॒र॒ति॒ यद् यद् भ॑रति भरति॒ यदेक॑मेक॒ मेक॑मेकं॒ ॅयद् भ॑रति भरति॒ यदेक॑मेकम् । \newline
25. यदेक॑मेक॒ मेक॑मेकं॒ ॅयद् यदेक॑मेकꣳ सं॒भरे᳚थ् सं॒भरे॒ देक॑मेकं॒ ॅयद् यदेक॑मेकꣳ सं॒भरे᳚त् । \newline
26. एक॑मेकꣳ सं॒भरे᳚थ् सं॒भरे॒ देक॑मेक॒ मेक॑मेकꣳ सं॒भरे᳚त् पितृदेव॒त्या॑नि पितृदेव॒त्या॑नि सं॒भरे॒देक॑मेक॒ मेक॑मेकꣳ सं॒भरे᳚त् पितृदेव॒त्या॑नि । \newline
27. एक॑मेक॒मित्येक᳚म् - ए॒क॒म् । \newline
28. सं॒भरे᳚त् पितृदेव॒त्या॑नि पितृदेव॒त्या॑नि सं॒भरे᳚थ् सं॒भरे᳚त् पितृदेव॒त्या॑नि स्युः स्युः पितृदेव॒त्या॑नि सं॒भरे᳚थ् सं॒भरे᳚त् पितृदेव॒त्या॑नि स्युः । \newline
29. सं॒भरे॒दिति॑ सं - भरे᳚त् । \newline
30. पि॒तृ॒दे॒व॒त्या॑नि स्युः स्युः पितृदेव॒त्या॑नि पितृदेव॒त्या॑नि स्यु॒र् यद् यथ् स्युः॑ पितृदेव॒त्या॑नि पितृदेव॒त्या॑नि स्यु॒र् यत् । \newline
31. पि॒तृ॒दे॒व॒त्या॑नीति॑ पितृ - दे॒व॒त्या॑नि । \newline
32. स्यु॒र् यद् यथ् स्युः॑ स्यु॒र् यथ् स॒ह स॒ह यथ् स्युः॑ स्यु॒र् यथ् स॒ह । \newline
33. यथ् स॒ह स॒ह यद् यथ् स॒ह सर्वा॑णि॒ सर्वा॑णि स॒ह यद् यथ् स॒ह सर्वा॑णि । \newline
34. स॒ह सर्वा॑णि॒ सर्वा॑णि स॒ह स॒ह सर्वा॑णि मानु॒षाणि॑ मानु॒षाणि॒ सर्वा॑णि स॒ह स॒ह सर्वा॑णि मानु॒षाणि॑ । \newline
35. सर्वा॑णि मानु॒षाणि॑ मानु॒षाणि॒ सर्वा॑णि॒ सर्वा॑णि मानु॒षाणि॒ द्वेद्वे॒ द्वेद्वे॑ मानु॒षाणि॒ सर्वा॑णि॒ सर्वा॑णि मानु॒षाणि॒ द्वेद्वे᳚ । \newline
36. मा॒नु॒षाणि॒ द्वेद्वे॒ द्वेद्वे॑ मानु॒षाणि॑ मानु॒षाणि॒ द्वेद्वे॒ सꣳ सम् द्वेद्वे॑ मानु॒षाणि॑ मानु॒षाणि॒ द्वेद्वे॒ सम् । \newline
37. द्वेद्वे॒ सꣳ सम् द्वेद्वे॒ द्वेद्वे॒ सम् भ॑रति भरति॒ सम् द्वेद्वे॒ द्वेद्वे॒ सम् भ॑रति । \newline
38. द्वेद्वे॒ इति॒ द्वे - द्वे॒ । \newline
39. सम् भ॑रति भरति॒ सꣳ सम् भ॑रति याज्यानुवा॒क्य॑योर् याज्यानुवा॒क्य॑योर् भरति॒ सꣳ सम् भ॑रति याज्यानुवा॒क्य॑योः । \newline
40. भ॒र॒ति॒ या॒ज्या॒नु॒वा॒क्य॑योर् याज्यानुवा॒क्य॑योर् भरति भरति याज्यानुवा॒क्य॑यो रे॒वैव या᳚ज्यानुवा॒क्य॑योर् भरति भरति याज्यानुवा॒क्य॑यो रे॒व । \newline
41. या॒ज्या॒नु॒वा॒क्य॑यो रे॒वैव या᳚ज्यानुवा॒क्य॑योर् याज्यानुवा॒क्य॑यो रे॒व रू॒पꣳ रू॒प मे॒व या᳚ज्यानुवा॒क्य॑योर् याज्यानुवा॒क्य॑यो रे॒व रू॒पम् । \newline
42. या॒ज्या॒नु॒वा॒क्य॑यो॒रिति॑ याज्या - अ॒नु॒वा॒क्य॑योः । \newline
43. ए॒व रू॒पꣳ रू॒प मे॒वैव रू॒पम् क॑रोति करोति रू॒प मे॒वैव रू॒पम् क॑रोति । \newline
44. रू॒पम् क॑रोति करोति रू॒पꣳ रू॒पम् क॑रो॒त्यथो॒ अथो॑ करोति रू॒पꣳ रू॒पम् क॑रो॒त्यथो᳚ । \newline
45. क॒रो॒त्यथो॒ अथो॑ करोति करो॒त्यथो॑ मिथु॒नम् मि॑थु॒न मथो॑ करोति करो॒त्यथो॑ मिथु॒नम् । \newline
46. अथो॑ मिथु॒नम् मि॑थु॒न मथो॒ अथो॑ मिथु॒न मे॒वैव मि॑थु॒न मथो॒ अथो॑ मिथु॒न मे॒व । \newline
47. अथो॒ इत्यथो᳚ । \newline
48. मि॒थु॒न मे॒वैव मि॑थु॒नम् मि॑थु॒न मे॒व यो य ए॒व मि॑थु॒नम् मि॑थु॒न मे॒व यः । \newline
49. ए॒व यो य ए॒वैव यो वै वै य ए॒वैव यो वै । \newline
50. यो वै वै यो यो वै दश॒ दश॒ वै यो यो वै दश॑ । \newline
51. वै दश॒ दश॒ वै वै दश॑ यज्ञायु॒धानि॑ यज्ञायु॒धानि॒ दश॒ वै वै दश॑ यज्ञायु॒धानि॑ । \newline
52. दश॑ यज्ञायु॒धानि॑ यज्ञायु॒धानि॒ दश॒ दश॑ यज्ञायु॒धानि॒ वेद॒ वेद॑ यज्ञायु॒धानि॒ दश॒ दश॑ यज्ञायु॒धानि॒ वेद॑ । \newline
53. य॒ज्ञा॒यु॒धानि॒ वेद॒ वेद॑ यज्ञायु॒धानि॑ यज्ञायु॒धानि॒ वेद॑ मुख॒तो मु॑ख॒तो वेद॑ यज्ञायु॒धानि॑ यज्ञायु॒धानि॒ वेद॑ मुख॒तः । \newline
54. य॒ज्ञा॒यु॒धानीति॑ यज्ञ् - आ॒यु॒धानि॑ । \newline
55. वेद॑ मुख॒तो मु॑ख॒तो वेद॒ वेद॑ मुख॒तो᳚ ऽस्यास्य मुख॒तो वेद॒ वेद॑ मुख॒तो᳚ ऽस्य । \newline
56. मु॒ख॒तो᳚ ऽस्यास्य मुख॒तो मु॑ख॒तो᳚ ऽस्य य॒ज्ञो य॒ज्ञो᳚ ऽस्य मुख॒तो मु॑ख॒तो᳚ ऽस्य य॒ज्ञ्ः । \newline
57. अ॒स्य॒ य॒ज्ञो य॒ज्ञो᳚ ऽस्यास्य य॒ज्ञ्ः क॑ल्पते कल्पते य॒ज्ञो᳚ ऽस्यास्य य॒ज्ञ्ः क॑ल्पते । \newline
58. य॒ज्ञ्ः क॑ल्पते कल्पते य॒ज्ञो य॒ज्ञ्ः क॑ल्पते॒ स्फ्यः स्फ्यः क॑ल्पते य॒ज्ञो य॒ज्ञ्ः क॑ल्पते॒ स्फ्यः । \newline
59. क॒ल्प॒ते॒ स्फ्यः स्फ्यः क॑ल्पते कल्पते॒ स्फ्यश्च॑ च॒ स्फ्यः क॑ल्पते कल्पते॒ स्फ्यश्च॑ । \newline
60. स्फ्यश्च॑ च॒ स्फ्यः स्फ्यश्च॑ क॒पाला॑नि क॒पाला॑नि च॒ स्फ्यः स्फ्यश्च॑ क॒पाला॑नि । \newline
\pagebreak
\markright{ TS 1.6.8.3  \hfill https://www.vedavms.in \hfill}

\section{ TS 1.6.8.3 }

\textbf{TS 1.6.8.3 } \newline
\textbf{Samhita Paata} \newline

च॑ क॒पाला॑नि चाग्निहोत्र॒हव॑णी च॒ शूर्पं॑ च कृष्णाजि॒नं च॒ शम्या॑ चो॒लूख॑लं च॒ मुस॑लं च दृ॒षच्चोप॑ला चै॒तानि॒ वै दश॑ यज्ञायु॒धानि॒ य ए॒वं ॅवेद॑ मुख॒तो᳚ऽस्य य॒ज्ञ्ः क॑ल्पते॒ यो वै दे॒वेभ्यः॑ प्रति॒प्रोच्य॑ य॒ज्ञेन॒ यज॑ते जु॒षन्ते᳚ऽस्य दे॒वा ह॒व्यꣳ ह॒विर् नि॑रु॒प्यमा॑णम॒भि म॑न्त्रयेता॒ऽग्निꣳ होता॑रमि॒ह तꣳ हु॑व॒ इति॑ - [ ] \newline

\textbf{Pada Paata} \newline

च॒ । क॒पाला॑नि । च॒ । अ॒ग्नि॒हो॒त्र॒हव॒णीत्य॑ग्निहोत्र - हव॑नी । च॒ । शूर्प᳚म् । च॒ । कृ॒ष्णा॒जि॒नमिति॑ कृष्ण - अ॒जि॒नम् । च॒ । शम्या᳚ । च॒ । उ॒लूख॑लम् । च॒ । मुस॑लम् । च॒ । दृ॒षत् । च॒ । उप॑ला । च॒ । ए॒तानि॑ । वै । दश॑ । य॒ज्ञा॒यु॒धानीति॑ यज्ञ् - आ॒यु॒धानि॑ । यः । ए॒वम् । वेद॑ । मु॒ख॒तः । अ॒स्य॒ । य॒ज्ञ्ः । क॒ल्प॒ते॒ । यः । वै । दे॒वेभ्यः॑ । प्र॒ति॒प्रोच्येति॑ प्रति - प्रोच्य॑ । य॒ज्ञेन॑ । यज॑ते । जु॒षन्ते᳚ । अ॒स्य॒ । दे॒वाः । ह॒व्यम् । ह॒विः । नि॒रु॒प्यमा॑ण॒मिति॑ निः - उ॒प्यमा॑नम् । अ॒भीति॑ । म॒न्त्र॒ये॒त॒ । अ॒ग्निम् । होता॑रम् । इ॒ह । तम् । हु॒वे॒ । इति॑ ।  \newline


\textbf{Krama Paata} \newline

च॒ क॒पाला॑नि । क॒पाला॑नि च । चा॒ग्नि॒हो॒त्र॒हव॑णी । अ॒ग्नि॒हो॒त्र॒हव॑णी च । अ॒ग्नि॒हो॒त्र॒हव॒णीत्य॑ग्निहोत्र - हव॑नी । च॒ शूर्प᳚म् । शूर्प॑म् च । च॒ कृ॒ष्णा॒जि॒नम् । कृ॒ष्णा॒जि॒नम् च॑ । कृ॒ष्णा॒जि॒नमिति॑ कृष्ण - अ॒जि॒नम् । च॒ शम्या᳚ । शम्या॑ च । चो॒लूख॑लम् । उ॒लूख॑लम् च । च॒ मुस॑लम् । मुस॑लम् च । च॒ दृ॒षत् । दृ॒षच्च॑ । चोप॑ला । उप॑ला च । चै॒तानि॑ । ए॒तानि॒ वै । वै दश॑ । दश॑,यज्ञायु॒धानि॑ । य॒ज्ञा॒यु॒धानि॒ यः । य॒ज्ञा॒यु॒धानीति॑ यज्ञ् - आ॒यु॒धानि॑ । य ए॒वम् । ए॒वं ॅवेद॑ । वेद॑ मुख॒तः । मु॒ख॒तो᳚ऽस्य । अ॒स्य॒ य॒ज्ञ्ः । य॒ज्ञ्ः क॑ल्पते । क॒ल्प॒ते॒ यः । यो वै । वै दे॒वेभ्यः॑ । दे॒वेभ्यः॑ प्रति॒प्रोच्य॑ । प्र॒ति॒प्रोच्य॑ य॒ज्ञेन॑ । प्र॒ति॒प्रोच्येति॑ प्रति - प्रोच्य॑ । य॒ज्ञेन॒ यज॑ते । यज॑ते जु॒षन्ते᳚ । जु॒षन्ते᳚ऽस्य । अ॒स्य॒ दे॒वाः । दे॒वा ह॒व्यम् । ह॒व्यꣳ ह॒विः । ह॒विर्,नि॑रु॒प्यमा॑णम् । नि॒रु॒प्यमा॑णम॒भि । नि॒रु॒प्यमा॑ण॒मिति॑ निः - उ॒प्यमा॑नम् । अ॒भि म॑न्त्रयेत । म॒न्त्र॒ये॒ता॒ग्निम् । अ॒ग्निꣳ होता॑रम् । होता॑रमि॒ह । इ॒ह तम् । तꣳ हु॑वे । हु॒व॒ इति॑ । इति॑ दे॒वेभ्यः॑ \newline

\textbf{Jatai Paata} \newline

1. च॒ क॒पाला॑नि क॒पाला॑नि च च क॒पाला॑नि । \newline
2. क॒पाला॑नि च च क॒पाला॑नि क॒पाला॑नि च । \newline
3. चा॒ग्नि॒हो॒त्र॒हव॑ ण्यग्निहोत्र॒हव॑णी च चाग्निहोत्र॒हव॑णी । \newline
4. अ॒ग्नि॒हो॒त्र॒हव॑णी च चाग्निहोत्र॒हव॑ ण्यग्निहोत्र॒हव॑णी च । \newline
5. अ॒ग्नि॒हो॒त्र॒हव॒णीत्य॑ग्निहोत्र - हव॑नी । \newline
6. च॒ शूर्प॒(ग्म्॒) शूर्प॑म् च च॒ शूर्प᳚म् । \newline
7. शूर्प॑म् च च॒ शूर्प॒(ग्म्॒) शूर्प॑म् च । \newline
8. च॒ कृ॒ष्णा॒जि॒नम् कृ॑ष्णाजि॒नम् च॑ च कृष्णाजि॒नम् । \newline
9. कृ॒ष्णा॒जि॒नम् च॑ च कृष्णाजि॒नम् कृ॑ष्णाजि॒नम् च॑ । \newline
10. कृ॒ष्णा॒जि॒नमिति॑ कृष्ण - अ॒जि॒नम् । \newline
11. च॒ शम्या॒ शम्या॑ च च॒ शम्या᳚ । \newline
12. शम्या॑ च च॒ शम्या॒ शम्या॑ च । \newline
13. चो॒लूख॑ल मु॒लूख॑लम् च चो॒लूख॑लम् । \newline
14. उ॒लूख॑लम् च चो॒लूख॑ल मु॒लूख॑लम् च । \newline
15. च॒ मुस॑ल॒म् मुस॑लम् च च॒ मुस॑लम् । \newline
16. मुस॑लम् च च॒ मुस॑ल॒म् मुस॑लम् च । \newline
17. च॒ दृ॒षद् दृ॒षच् च॑ च दृ॒षत् । \newline
18. दृ॒षच् च॑ च दृ॒षद् दृ॒षच् च॑ । \newline
19. चोप॒लोप॑ला च॒ चोप॑ला । \newline
20. उप॑ला च॒ चोप॒लोप॑ला च । \newline
21. चै॒तान्ये॒तानि॑ च चै॒तानि॑ । \newline
22. ए॒तानि॒ वै वा ए॒तान्ये॒तानि॒ वै । \newline
23. वै दश॒ दश॒ वै वै दश॑ । \newline
24. दश॑ यज्ञायु॒धानि॑ यज्ञायु॒धानि॒ दश॒ दश॑ यज्ञायु॒धानि॑ । \newline
25. य॒ज्ञा॒यु॒धानि॒ यो यो य॑ज्ञायु॒धानि॑ यज्ञायु॒धानि॒ यः । \newline
26. य॒ज्ञा॒यु॒धानीति॑ यज्ञ् - आ॒यु॒धानि॑ । \newline
27. य ए॒व मे॒वं ॅयो य ए॒वम् । \newline
28. ए॒वं ॅवेद॒ वेदै॒व मे॒वं ॅवेद॑ । \newline
29. वेद॑ मुख॒तो मु॑ख॒तो वेद॒ वेद॑ मुख॒तः । \newline
30. मु॒ख॒तो᳚ ऽस्यास्य मुख॒तो मु॑ख॒तो᳚ ऽस्य । \newline
31. अ॒स्य॒ य॒ज्ञो य॒ज्ञो᳚ ऽस्यास्य य॒ज्ञ्ः । \newline
32. य॒ज्ञ्ः क॑ल्पते कल्पते य॒ज्ञो य॒ज्ञ्ः क॑ल्पते । \newline
33. क॒ल्प॒ते॒ यो यः क॑ल्पते कल्पते॒ यः । \newline
34. यो वै वै यो यो वै । \newline
35. वै दे॒वेभ्यो॑ दे॒वेभ्यो॒ वै वै दे॒वेभ्यः॑ । \newline
36. दे॒वेभ्यः॑ प्रति॒प्रोच्य॑ प्रति॒प्रोच्य॑ दे॒वेभ्यो॑ दे॒वेभ्यः॑ प्रति॒प्रोच्य॑ । \newline
37. प्र॒ति॒प्रोच्य॑ य॒ज्ञेन॑ य॒ज्ञेन॑ प्रति॒प्रोच्य॑ प्रति॒प्रोच्य॑ य॒ज्ञेन॑ । \newline
38. प्र॒ति॒प्रोच्येति॑ प्रति - प्रोच्य॑ । \newline
39. य॒ज्ञेन॒ यज॑ते॒ यज॑ते य॒ज्ञेन॑ य॒ज्ञेन॒ यज॑ते । \newline
40. यज॑ते जु॒षन्ते॑ जु॒षन्ते॒ यज॑ते॒ यज॑ते जु॒षन्ते᳚ । \newline
41. जु॒षन्ते᳚ ऽस्यास्य जु॒षन्ते॑ जु॒षन्ते᳚ ऽस्य । \newline
42. अ॒स्य॒ दे॒वा दे॒वा अ॑स्यास्य दे॒वाः । \newline
43. दे॒वा ह॒व्यꣳ ह॒व्यम् दे॒वा दे॒वा ह॒व्यम् । \newline
44. ह॒व्यꣳ ह॒विर्. ह॒विर्. ह॒व्यꣳ ह॒व्यꣳ ह॒विः । \newline
45. ह॒विर् नि॑रु॒प्यमा॑णम् निरु॒प्यमा॑णꣳ ह॒विर्. ह॒विर् नि॑रु॒प्यमा॑णम् । \newline
46. नि॒रु॒प्यमा॑ण म॒भ्य॑भि नि॑रु॒प्यमा॑णम् निरु॒प्यमा॑ण म॒भि । \newline
47. नि॒रु॒प्यमा॑ण॒मिति॑ निः - उ॒प्यमा॑नम् । \newline
48. अ॒भि म॑न्त्रयेत मन्त्रयेता॒भ्य॑भि म॑न्त्रयेत । \newline
49. म॒न्त्र॒ये॒ता॒ग्नि म॒ग्निम् म॑न्त्रयेत मन्त्रयेता॒ग्निम् । \newline
50. अ॒ग्निꣳ होता॑र॒(ग्म्॒) होता॑र म॒ग्नि म॒ग्निꣳ होता॑रम् । \newline
51. होता॑र मि॒हे ह होता॑र॒(ग्म्॒) होता॑र मि॒ह । \newline
52. इ॒ह तम् त मि॒हे ह तम् । \newline
53. तꣳ हु॑वे हुवे॒ तम् तꣳ हु॑वे । \newline
54. हु॒व॒ इतीति॑ हुवे हुव॒ इति॑ । \newline
55. इति॑ दे॒वेभ्यो॑ दे॒वेभ्य॒ इतीति॑ दे॒वेभ्यः॑ । \newline

\textbf{Ghana Paata } \newline

1. च॒ क॒पाला॑नि क॒पाला॑नि च च क॒पाला॑नि च च क॒पाला॑नि च च क॒पाला॑नि च । \newline
2. क॒पाला॑नि च च क॒पाला॑नि क॒पाला॑नि चाग्निहोत्र॒हव॑ ण्यग्निहोत्र॒हव॑णी च क॒पाला॑नि क॒पाला॑नि चाग्निहोत्र॒हव॑णी । \newline
3. चा॒ग्नि॒हो॒त्र॒हव॑ ण्यग्निहोत्र॒हव॑णी च चाग्निहोत्र॒हव॑णी च चाग्निहोत्र॒हव॑णी च चाग्निहोत्र॒हव॑णी च । \newline
4. अ॒ग्नि॒हो॒त्र॒हव॑णी च चाग्निहोत्र॒हव॑ ण्यग्निहोत्र॒हव॑णी च॒ शूर्प॒(ग्म्॒) शूर्प॑म् चाग्निहोत्र॒हव॑ ण्यग्निहोत्र॒हव॑णी च॒ शूर्प᳚म् । \newline
5. अ॒ग्नि॒हो॒त्र॒हव॒णीत्य॑ग्निहोत्र - हव॑नी । \newline
6. च॒ शूर्प॒(ग्म्॒) शूर्प॑म् च च॒ शूर्प॑म् च च॒ शूर्प॑म् च च॒ शूर्प॑म् च । \newline
7. शूर्प॑म् च च॒ शूर्प॒(ग्म्॒) शूर्प॑म् च कृष्णाजि॒नम् कृ॑ष्णाजि॒नम् च॒ शूर्प॒(ग्म्॒) शूर्प॑म् च कृष्णाजि॒नम् । \newline
8. च॒ कृ॒ष्णा॒जि॒नम् कृ॑ष्णाजि॒नम् च॑ च कृष्णाजि॒नम् च॑ च कृष्णाजि॒नम् च॑ च कृष्णाजि॒नम् च॑ । \newline
9. कृ॒ष्णा॒जि॒नम् च॑ च कृष्णाजि॒नम् कृ॑ष्णाजि॒नम् च॒ शम्या॒ शम्या॑ च कृष्णाजि॒नम् कृ॑ष्णाजि॒नम् च॒ शम्या᳚ । \newline
10. कृ॒ष्णा॒जि॒नमिति॑ कृष्ण - अ॒जि॒नम् । \newline
11. च॒ शम्या॒ शम्या॑ च च॒ शम्या॑ च च॒ शम्या॑ च च॒ शम्या॑ च । \newline
12. शम्या॑ च च॒ शम्या॒ शम्या॑ चो॒लूख॑ल मु॒लूख॑लम् च॒ शम्या॒ शम्या॑ चो॒लूख॑लम् । \newline
13. चो॒लूख॑ल मु॒लूख॑लम् च चो॒लूख॑लम् च चो॒लूख॑लम् च चो॒लूख॑लम् च । \newline
14. उ॒लूख॑लम् च चो॒लूख॑ल मु॒लूख॑लम् च॒ मुस॑ल॒म् मुस॑लम् चो॒लूख॑ल मु॒लूख॑लम् च॒ मुस॑लम् । \newline
15. च॒ मुस॑ल॒म् मुस॑लम् च च॒ मुस॑लम् च च॒ मुस॑लम् च च॒ मुस॑लम् च । \newline
16. मुस॑लम् च च॒ मुस॑ल॒म् मुस॑लम् च दृ॒षद् दृ॒षच् च॒ मुस॑ल॒म् मुस॑लम् च दृ॒षत् । \newline
17. च॒ दृ॒षद् दृ॒षच् च॑ च दृ॒षच् च॑ च दृ॒षच् च॑ च दृ॒षच् च॑ । \newline
18. दृ॒षच् च॑ च दृ॒षद् दृ॒षच् चोप॒लोप॑ला च दृ॒षद् दृ॒षच् चोप॑ला । \newline
19. चोप॒लोप॑ला च॒ चोप॑ला च॒ चोप॑ला च॒ चोप॑ला च । \newline
20. उप॑ला च॒ चोप॒लोप॑ला चै॒तान्ये॒तानि॒ चोप॒लोप॑ला चै॒तानि॑ । \newline
21. चै॒तान्ये॒तानि॑ च चै॒तानि॒ वै वा ए॒तानि॑ च चै॒तानि॒ वै । \newline
22. ए॒तानि॒ वै वा ए॒तान्ये॒तानि॒ वै दश॒ दश॒ वा ए॒तान्ये॒तानि॒ वै दश॑ । \newline
23. वै दश॒ दश॒ वै वै दश॑ यज्ञायु॒धानि॑ यज्ञायु॒धानि॒ दश॒ वै वै दश॑ यज्ञायु॒धानि॑ । \newline
24. दश॑ यज्ञायु॒धानि॑ यज्ञायु॒धानि॒ दश॒ दश॑ यज्ञायु॒धानि॒ यो यो य॑ज्ञायु॒धानि॒ दश॒ दश॑ यज्ञायु॒धानि॒ यः । \newline
25. य॒ज्ञा॒यु॒धानि॒ यो यो य॑ज्ञायु॒धानि॑ यज्ञायु॒धानि॒ य ए॒व मे॒वं ॅयो य॑ज्ञायु॒धानि॑ यज्ञायु॒धानि॒ य ए॒वम् । \newline
26. य॒ज्ञा॒यु॒धानीति॑ यज्ञ् - आ॒यु॒धानि॑ । \newline
27. य ए॒व मे॒वं ॅयो य ए॒वं ॅवेद॒ वेदै॒वं ॅयो य ए॒वं ॅवेद॑ । \newline
28. ए॒वं ॅवेद॒ वेदै॒व मे॒वं ॅवेद॑ मुख॒तो मु॑ख॒तो वेदै॒व मे॒वं ॅवेद॑ मुख॒तः । \newline
29. वेद॑ मुख॒तो मु॑ख॒तो वेद॒ वेद॑ मुख॒तो᳚ ऽस्यास्य मुख॒तो वेद॒ वेद॑ मुख॒तो᳚ ऽस्य । \newline
30. मु॒ख॒तो᳚ ऽस्यास्य मुख॒तो मु॑ख॒तो᳚ ऽस्य य॒ज्ञो य॒ज्ञो᳚ ऽस्य मुख॒तो मु॑ख॒तो᳚ ऽस्य य॒ज्ञ्ः । \newline
31. अ॒स्य॒ य॒ज्ञो य॒ज्ञो᳚ ऽस्यास्य य॒ज्ञ्ः क॑ल्पते कल्पते य॒ज्ञो᳚ ऽस्यास्य य॒ज्ञ्ः क॑ल्पते । \newline
32. य॒ज्ञ्ः क॑ल्पते कल्पते य॒ज्ञो य॒ज्ञ्ः क॑ल्पते॒ यो यः क॑ल्पते य॒ज्ञो य॒ज्ञ्ः क॑ल्पते॒ यः । \newline
33. क॒ल्प॒ते॒ यो यः क॑ल्पते कल्पते॒ यो वै वै यः क॑ल्पते कल्पते॒ यो वै । \newline
34. यो वै वै यो यो वै दे॒वेभ्यो॑ दे॒वेभ्यो॒ वै यो यो वै दे॒वेभ्यः॑ । \newline
35. वै दे॒वेभ्यो॑ दे॒वेभ्यो॒ वै वै दे॒वेभ्यः॑ प्रति॒प्रोच्य॑ प्रति॒प्रोच्य॑ दे॒वेभ्यो॒ वै वै दे॒वेभ्यः॑ प्रति॒प्रोच्य॑ । \newline
36. दे॒वेभ्यः॑ प्रति॒प्रोच्य॑ प्रति॒प्रोच्य॑ दे॒वेभ्यो॑ दे॒वेभ्यः॑ प्रति॒प्रोच्य॑ य॒ज्ञेन॑ य॒ज्ञेन॑ प्रति॒प्रोच्य॑ दे॒वेभ्यो॑ दे॒वेभ्यः॑ प्रति॒प्रोच्य॑ य॒ज्ञेन॑ । \newline
37. प्र॒ति॒प्रोच्य॑ य॒ज्ञेन॑ य॒ज्ञेन॑ प्रति॒प्रोच्य॑ प्रति॒प्रोच्य॑ य॒ज्ञेन॒ यज॑ते॒ यज॑ते य॒ज्ञेन॑ प्रति॒प्रोच्य॑ प्रति॒प्रोच्य॑ य॒ज्ञेन॒ यज॑ते । \newline
38. प्र॒ति॒प्रोच्येति॑ प्रति - प्रोच्य॑ । \newline
39. य॒ज्ञेन॒ यज॑ते॒ यज॑ते य॒ज्ञेन॑ य॒ज्ञेन॒ यज॑ते जु॒षन्ते॑ जु॒षन्ते॒ यज॑ते य॒ज्ञेन॑ य॒ज्ञेन॒ यज॑ते जु॒षन्ते᳚ । \newline
40. यज॑ते जु॒षन्ते॑ जु॒षन्ते॒ यज॑ते॒ यज॑ते जु॒षन्ते᳚ ऽस्यास्य जु॒षन्ते॒ यज॑ते॒ यज॑ते जु॒षन्ते᳚ ऽस्य । \newline
41. जु॒षन्ते᳚ ऽस्यास्य जु॒षन्ते॑ जु॒षन्ते᳚ ऽस्य दे॒वा दे॒वा अ॑स्य जु॒षन्ते॑ जु॒षन्ते᳚ ऽस्य दे॒वाः । \newline
42. अ॒स्य॒ दे॒वा दे॒वा अ॑स्यास्य दे॒वा ह॒व्यꣳ ह॒व्यम् दे॒वा अ॑स्यास्य दे॒वा ह॒व्यम् । \newline
43. दे॒वा ह॒व्यꣳ ह॒व्यम् दे॒वा दे॒वा ह॒व्यꣳ ह॒विर्. ह॒विर्. ह॒व्यम् दे॒वा दे॒वा ह॒व्यꣳ ह॒विः । \newline
44. ह॒व्यꣳ ह॒विर्. ह॒विर्. ह॒व्यꣳ ह॒व्यꣳ ह॒विर् नि॑रु॒प्यमा॑णम् निरु॒प्यमा॑णꣳ ह॒विर्. ह॒व्यꣳ ह॒व्यꣳ ह॒विर् नि॑रु॒प्यमा॑णम् । \newline
45. ह॒विर् नि॑रु॒प्यमा॑णम् निरु॒प्यमा॑णꣳ ह॒विर्. ह॒विर् नि॑रु॒प्यमा॑ण म॒भ्य॑भि नि॑रु॒प्यमा॑णꣳ ह॒विर्. ह॒विर् नि॑रु॒प्यमा॑ण म॒भि । \newline
46. नि॒रु॒प्यमा॑ण म॒भ्य॑भि नि॑रु॒प्यमा॑णम् निरु॒प्यमा॑ण म॒भि म॑न्त्रयेत मन्त्रयेता॒भि नि॑रु॒प्यमा॑णम् निरु॒प्यमा॑ण म॒भि म॑न्त्रयेत । \newline
47. नि॒रु॒प्यमा॑ण॒मिति॑ निः - उ॒प्यमा॑नम् । \newline
48. अ॒भि म॑न्त्रयेत मन्त्रयेता॒भ्य॑भि म॑न्त्रयेता॒ग्नि म॒ग्निम् म॑न्त्रयेता॒भ्य॑भि म॑न्त्रयेता॒ग्निम् । \newline
49. म॒न्त्र॒ये॒ता॒ग्नि म॒ग्निम् म॑न्त्रयेत मन्त्रयेता॒ग्निꣳ होता॑र॒(ग्म्॒) होता॑र म॒ग्निम् म॑न्त्रयेत मन्त्रयेता॒ग्निꣳ होता॑रम् । \newline
50. अ॒ग्निꣳ होता॑र॒(ग्म्॒) होता॑र म॒ग्नि म॒ग्निꣳ होता॑र मि॒हे ह होता॑र म॒ग्नि म॒ग्निꣳ होता॑र मि॒ह । \newline
51. होता॑र मि॒हे ह होता॑र॒(ग्म्॒) होता॑र मि॒ह तम् त मि॒ह होता॑र॒(ग्म्॒) होता॑र मि॒ह तम् । \newline
52. इ॒ह तम् त मि॒हे ह तꣳ हु॑वे हुवे॒ त मि॒हे ह तꣳ हु॑वे । \newline
53. तꣳ हु॑वे हुवे॒ तम् तꣳ हु॑व॒ इतीति॑ हुवे॒ तम् तꣳ हु॑व॒ इति॑ । \newline
54. हु॒व॒ इतीति॑ हुवे हुव॒ इति॑ दे॒वेभ्यो॑ दे॒वेभ्य॒ इति॑ हुवे हुव॒ इति॑ दे॒वेभ्यः॑ । \newline
55. इति॑ दे॒वेभ्यो॑ दे॒वेभ्य॒ इतीति॑ दे॒वेभ्य॑ ए॒वैव दे॒वेभ्य॒ इतीति॑ दे॒वेभ्य॑ ए॒व । \newline
\pagebreak
\markright{ TS 1.6.8.4  \hfill https://www.vedavms.in \hfill}

\section{ TS 1.6.8.4 }

\textbf{TS 1.6.8.4 } \newline
\textbf{Samhita Paata} \newline

दे॒वेभ्य॑ ए॒व प्र॑ति॒प्रोच्य॑ य॒ज्ञेन॑ यजते जु॒षन्ते᳚ऽस्य दे॒वा ह॒व्यमे॒ष वै य॒ज्ञ्स्य॒ ग्रहो॑ गृही॒त्वैव य॒ज्ञेन॑ यजते॒ तदु॑दि॒त्वा वाचं॑ ॅयच्छति य॒ज्ञ्स्य॒ धृत्या॒ अथो॒ मन॑सा॒ वै प्र॒जाप॑तिर् य॒ज्ञ्म॑तनुत॒ मन॑सै॒व तद्-य॒ज्ञ्ं त॑नुते॒ रक्ष॑सा॒-मन॑न्ववचाराय॒ यो वै य॒ज्ञ्ं ॅयोग॒ आग॑ते यु॒नक्ति॑ यु॒ङ्क्ते यु॑ञ्जा॒नेषु॒ कस्त्वा॑ युनक्ति॒ स त्वा॑ युन॒क्त्वि-( ) -त्या॑ह प्र॒जाप॑ति॒र् वै कः प्र॒जाप॑तिनै॒वैनं॑ ॅयुनक्ति यु॒ङ्क्ते यु॑ञ्जा॒नेषु॑ ॥ \newline

\textbf{Pada Paata} \newline

दे॒वेभ्यः॑ । ए॒व । प्र॒ति॒प्रोच्येति॑ प्रति - प्रोच्य॑ । य॒ज्ञेन॑ । य॒ज॒ते॒ । जु॒षन्ते᳚ । अ॒स्य॒ । दे॒वाः । ह॒व्यम् । ए॒षः । वै । य॒ज्ञ्स्य॑ । ग्रहः॑ । गृ॒ही॒त्वा । ए॒व । य॒ज्ञेन॑ । य॒ज॒ते॒ । तत् । उ॒दि॒त्वा । वाच᳚म् । य॒च्छ॒ति॒ । य॒ज्ञ्स्य॑ । धृत्यै᳚ । अथो॒ इति॑ । मन॑सा । वै । प्र॒जाप॑ति॒रिति॑ प्र॒जा - प॒तिः॒ । य॒ज्ञ्म् । अ॒त॒नु॒त॒ । मन॑सा । ए॒व । तत् । य॒ज्ञ्म् । त॒नु॒ते॒ । रक्ष॑साम् । अन॑न्ववचारा॒येत्यन॑नु-अ॒व॒चा॒रा॒य॒ । यः । वै । य॒ज्ञ्म् । योगे᳚ । आग॑त॒ इत्या - ग॒ते॒ । यु॒नक्ति॑ । यु॒ङ्क्ते । यु॒ञ्जा॒नेषु॑ । कः । त्वा॒ । यु॒न॒क्ति॒ । सः । त्वा॒ । यु॒न॒क्तु॒ ( ) । इति॑ । आ॒ह॒ । प्र॒जाप॑ति॒रिति॑ प्र॒जा - प॒तिः॒ । वै । कः । प्र॒जाप॑ति॒नेति॑ प्र॒जा - प॒ति॒ना॒ । ए॒व । ए॒न॒म् । यु॒न॒क्ति॒ । यु॒ङ्क्ते । यु॒ञ्जा॒नेषु॑ ॥  \newline


\textbf{Krama Paata} \newline

दे॒वेभ्य॑ ए॒व । ए॒व प्र॑ति॒प्रोच्य॑ । प्र॒ति॒प्रोच्य॑ य॒ज्ञेन॑ । प्र॒ति॒प्रोच्येति॑ प्रति - प्रोच्य॑ । य॒ज्ञेन॑ यजते । य॒ज॒ते॒ जु॒षन्ते᳚ । जु॒षन्ते᳚ऽस्य । अ॒स्य॒ दे॒वाः । दे॒वा ह॒व्यम् । ह॒व्यमे॒षः । ए॒ष वै । वै य॒ज्ञ्स्य॑ । य॒ज्ञ्स्य॒ ग्रहः॑ । ग्रहो॑ गृही॒त्वा । गृ॒ही॒त्वैव । ए॒व य॒ज्ञेन॑ । य॒ज्ञेन॑ यजते । य॒ज॒ते॒ तत् । तदु॑दि॒त्वा । उ॒दि॒त्वा वाच᳚म् । वाचं॑ ॅयच्छति । य॒च्छ॒ति॒ य॒ज्ञ्स्य॑ । य॒ज्ञ्स्य॒ धृत्यै᳚ । धृत्या॒ अथो᳚ । अथो॒ मन॑सा । अथो॒ इत्यथो᳚ । मन॑सा॒ वै । वै प्र॒जाप॑तिः । प्र॒जाप॑तिर् य॒ज्ञ्म् । प्र॒जाप॑ति॒रिति॑ प्र॒जा - प॒तिः॒ । य॒ज्ञ्म॑तनुत । अ॒त॒नु॒त॒ मन॑सा । मन॑सै॒व । ए॒व तत् । तद् य॒ज्ञ्म् । य॒ज्ञ्म् त॑नुते । त॒नु॒ते॒ रक्ष॑साम् । रक्ष॑सा॒मन॑न्ववचाराय । अन॑न्ववचाराय॒ यः । अन॑न्ववचारा॒येत्यन॑नु - अ॒व॒चा॒रा॒य॒ । यो वै । वै य॒ज्ञ्म् । य॒ज्ञ्ं ॅयोगे᳚ । योग॒ आग॑ते । आग॑ते यु॒नक्ति॑ । आग॑त॒ इत्या - ग॒ते॒ । यु॒नक्ति॑ यु॒ङ्क्ते । यु॒ङ्क्ते यु॑ञ्जा॒नेषु॑ । यु॒ञ्जा॒नेषु॒ कः । कस्त्वा᳚ । त्वा॒ यु॒न॒क्ति॒ । यु॒न॒क्ति॒ सः । स त्वा᳚ । त्वा॒ यु॒न॒क्तु॒ ( ) । यु॒न॒क्त्विति॑ । इत्या॑ह । आ॒ह॒ प्र॒जाप॑तिः । प्र॒जाप॑ति॒र्,वै । प्र॒जाप॑ति॒रिति॑ प्र॒जा - प॒तिः॒ । वै कः । कः प्र॒जाप॑तिना । प्र॒जाप॑तिनै॒व । प्र॒जाप॑ति॒नेति॑ प्र॒जा - प॒ति॒ना॒ । ए॒वैन᳚म् । ए॒नं॒ ॅयु॒न॒क्ति॒ । यु॒न॒क्ति॒ यु॒ङ्क्ते । यु॒ङ्क्ते यु॑ञ्जा॒नेषु॑ । यु॒ञ्जा॒नेष्विति॑ युञ्जा॒नेषु॑ । \newline

\textbf{Jatai Paata} \newline

1. दे॒वेभ्य॑ ए॒वैव दे॒वेभ्यो॑ दे॒वेभ्य॑ ए॒व । \newline
2. ए॒व प्र॑ति॒प्रोच्य॑ प्रति॒प्रो च्यै॒वैव प्र॑ति॒प्रोच्य॑ । \newline
3. प्र॒ति॒प्रोच्य॑ य॒ज्ञेन॑ य॒ज्ञेन॑ प्रति॒प्रोच्य॑ प्रति॒प्रोच्य॑ य॒ज्ञेन॑ । \newline
4. प्र॒ति॒प्रोच्येति॑ प्रति - प्रोच्य॑ । \newline
5. य॒ज्ञेन॑ यजते यजते य॒ज्ञेन॑ य॒ज्ञेन॑ यजते । \newline
6. य॒ज॒ते॒ जु॒षन्ते॑ जु॒षन्ते॑ यजते यजते जु॒षन्ते᳚ । \newline
7. जु॒षन्ते᳚ ऽस्यास्य जु॒षन्ते॑ जु॒षन्ते᳚ ऽस्य । \newline
8. अ॒स्य॒ दे॒वा दे॒वा अ॑स्यास्य दे॒वाः । \newline
9. दे॒वा ह॒व्यꣳ ह॒व्यम् दे॒वा दे॒वा ह॒व्यम् । \newline
10. ह॒व्य मे॒ष ए॒ष ह॒व्यꣳ ह॒व्य मे॒षः । \newline
11. ए॒ष वै वा ए॒ष ए॒ष वै । \newline
12. वै य॒ज्ञ्स्य॑ य॒ज्ञ्स्य॒ वै वै य॒ज्ञ्स्य॑ । \newline
13. य॒ज्ञ्स्य॒ ग्रहो॒ ग्रहो॑ य॒ज्ञ्स्य॑ य॒ज्ञ्स्य॒ ग्रहः॑ । \newline
14. ग्रहो॑ गृही॒त्वा गृ॑ही॒त्वा ग्रहो॒ ग्रहो॑ गृही॒त्वा । \newline
15. गृ॒ही॒त्वैवैव गृ॑ही॒त्वा गृ॑ही॒त्वैव । \newline
16. ए॒व य॒ज्ञेन॑ य॒ज्ञेनै॒वैव य॒ज्ञेन॑ । \newline
17. य॒ज्ञेन॑ यजते यजते य॒ज्ञेन॑ य॒ज्ञेन॑ यजते । \newline
18. य॒ज॒ते॒ तत् तद् य॑जते यजते॒ तत् । \newline
19. तदु॑दि॒ त्वोदि॒त्वा तत् तदु॑दि॒त्वा । \newline
20. उ॒दि॒त्वा वाचं॒ ॅवाच॑ मुदि॒ त्वोदि॒त्वा वाच᳚म् । \newline
21. वाचं॑ ॅयच्छति यच्छति॒ वाचं॒ ॅवाचं॑ ॅयच्छति । \newline
22. य॒च्छ॒ति॒ य॒ज्ञ्स्य॑ य॒ज्ञ्स्य॑ यच्छति यच्छति य॒ज्ञ्स्य॑ । \newline
23. य॒ज्ञ्स्य॒ धृत्यै॒ धृत्यै॑ य॒ज्ञ्स्य॑ य॒ज्ञ्स्य॒ धृत्यै᳚ । \newline
24. धृत्या॒ अथो॒ अथो॒ धृत्यै॒ धृत्या॒ अथो᳚ । \newline
25. अथो॒ मन॑सा॒ मन॒सा ऽथो॒ अथो॒ मन॑सा । \newline
26. अथो॒ इत्य॒थो᳚ । \newline
27. मन॑सा॒ वै वै मन॑सा॒ मन॑सा॒ वै । \newline
28. वै प्र॒जाप॑तिः प्र॒जाप॑ति॒र् वै वै प्र॒जाप॑तिः । \newline
29. प्र॒जाप॑तिर् य॒ज्ञ्ं ॅय॒ज्ञ्म् प्र॒जाप॑तिः प्र॒जाप॑तिर् य॒ज्ञ्म् । \newline
30. प्र॒जाप॑ति॒रिति॑ प्र॒जा - प॒तिः॒ । \newline
31. य॒ज्ञ् म॑तनु तातनुत य॒ज्ञ्ं ॅय॒ज्ञ् म॑तनुत । \newline
32. अ॒त॒नु॒त॒ मन॑सा॒ मन॑सा ऽतनुतातनुत॒ मन॑सा । \newline
33. मन॑सै॒वैव मन॑सा॒ मन॑सै॒व । \newline
34. ए॒व तत् तदे॒वैव तत् । \newline
35. तद् य॒ज्ञ्ं ॅय॒ज्ञ्म् तत् तद् य॒ज्ञ्म् । \newline
36. य॒ज्ञ्म् त॑नुते तनुते य॒ज्ञ्ं ॅय॒ज्ञ्म् त॑नुते । \newline
37. त॒नु॒ते॒ रक्ष॑सा॒(ग्म्॒) रक्ष॑साम् तनुते तनुते॒ रक्ष॑साम् । \newline
38. रक्ष॑सा॒ मन॑न्ववचारा॒ यान॑न्ववचाराय॒ रक्ष॑सा॒(ग्म्॒) रक्ष॑सा॒ मन॑न्ववचाराय । \newline
39. अन॑न्ववचाराय॒ यो यो ऽन॑न्ववचारा॒ यान॑न्ववचाराय॒ यः । \newline
40. अन॑न्ववचारा॒येत्यन॑नु - अ॒व॒चा॒रा॒य॒ । \newline
41. यो वै वै यो यो वै । \newline
42. वै य॒ज्ञ्ं ॅय॒ज्ञ्ं ॅवै वै य॒ज्ञ्म् । \newline
43. य॒ज्ञ्ं ॅयोगे॒ योगे॑ य॒ज्ञ्ं ॅय॒ज्ञ्ं ॅयोगे᳚ । \newline
44. योग॒ आग॑त॒ आग॑ते॒ योगे॒ योग॒ आग॑ते । \newline
45. आग॑ते यु॒नक्ति॑ यु॒नक्त्याग॑त॒ आग॑ते यु॒नक्ति॑ । \newline
46. आग॑त॒ इत्या - ग॒ते॒ । \newline
47. यु॒नक्ति॑ यु॒ङ्क्ते यु॒ङ्क्ते यु॒नक्ति॑ यु॒नक्ति॑ यु॒ङ्क्ते । \newline
48. यु॒ङ्क्ते यु॑ञ्जा॒नेषु॑ युञ्जा॒नेषु॑ यु॒ङ्क्ते यु॒ङ्क्ते यु॑ञ्जा॒नेषु॑ । \newline
49. यु॒ञ्जा॒नेषु॒ कः को यु॑ञ्जा॒नेषु॑ युञ्जा॒नेषु॒ कः । \newline
50. कस्त्वा᳚ त्वा॒ कः कस्त्वा᳚ । \newline
51. त्वा॒ यु॒न॒क्ति॒ यु॒न॒क्ति॒ त्वा॒ त्वा॒ यु॒न॒क्ति॒ । \newline
52. यु॒न॒क्ति॒ स स यु॑नक्ति युनक्ति॒ सः । \newline
53. स त्वा᳚ त्वा॒ स स त्वा᳚ । \newline
54. त्वा॒ यु॒न॒क्तु॒ यु॒न॒क्तु॒ त्वा॒ त्वा॒ यु॒न॒क्तु॒ । \newline
55. यु॒न॒क्त्वितीति॑ युनक्तु युन॒क्त्विति॑ । \newline
56. इत्या॑हा॒हे तीत्या॑ह । \newline
57. आ॒ह॒ प्र॒जाप॑तिः प्र॒जाप॑ति राहाह प्र॒जाप॑तिः । \newline
58. प्र॒जाप॑ति॒र् वै वै प्र॒जाप॑तिः प्र॒जाप॑ति॒र् वै । \newline
59. प्र॒जाप॑ति॒रिति॑ प्र॒जा - प॒तिः॒ । \newline
60. वै कः को वै वै कः । \newline
61. कः प्र॒जाप॑तिना प्र॒जाप॑तिना॒ कः कः प्र॒जाप॑तिना । \newline
62. प्र॒जाप॑ति नै॒वैव प्र॒जाप॑तिना प्र॒जाप॑ति नै॒व । \newline
63. प्र॒जाप॑ति॒नेति॑ प्र॒जा - प॒ति॒ना॒ । \newline
64. ए॒वैन॑ मेन मे॒ वैवैन᳚म् । \newline
65. ए॒नं॒ ॅयु॒न॒क्ति॒ यु॒न॒क्त्ये॒न॒ मे॒नं॒ ॅयु॒न॒क्ति॒ । \newline
66. यु॒न॒क्ति॒ यु॒ङ्क्ते यु॒ङ्क्ते यु॑नक्ति युनक्ति यु॒ङ्क्ते । \newline
67. यु॒ङ्क्ते यु॑ञ्जा॒नेषु॑ युञ्जा॒नेषु॑ यु॒ङ्क्ते यु॒ङ्क्ते यु॑ञ्जा॒नेषु॑ । \newline
68. यु॒ञ्जा॒नेष्विति॑ युञ्जा॒नेषु॑ । \newline

\textbf{Ghana Paata } \newline

1. दे॒वेभ्य॑ ए॒वैव दे॒वेभ्यो॑ दे॒वेभ्य॑ ए॒व प्र॑ति॒प्रोच्य॑ प्रति॒प्रोच्यै॒व दे॒वेभ्यो॑ दे॒वेभ्य॑ ए॒व प्र॑ति॒प्रोच्य॑ । \newline
2. ए॒व प्र॑ति॒प्रोच्य॑ प्रति॒प्रोच्यै॒वैव प्र॑ति॒प्रोच्य॑ य॒ज्ञेन॑ य॒ज्ञेन॑ प्रति॒प्रोच्यै॒वैव प्र॑ति॒प्रोच्य॑ य॒ज्ञेन॑ । \newline
3. प्र॒ति॒प्रोच्य॑ य॒ज्ञेन॑ य॒ज्ञेन॑ प्रति॒प्रोच्य॑ प्रति॒प्रोच्य॑ य॒ज्ञेन॑ यजते यजते य॒ज्ञेन॑ प्रति॒प्रोच्य॑ प्रति॒प्रोच्य॑ य॒ज्ञेन॑ यजते । \newline
4. प्र॒ति॒प्रोच्येति॑ प्रति - प्रोच्य॑ । \newline
5. य॒ज्ञेन॑ यजते यजते य॒ज्ञेन॑ य॒ज्ञेन॑ यजते जु॒षन्ते॑ जु॒षन्ते॑ यजते य॒ज्ञेन॑ य॒ज्ञेन॑ यजते जु॒षन्ते᳚ । \newline
6. य॒ज॒ते॒ जु॒षन्ते॑ जु॒षन्ते॑ यजते यजते जु॒षन्ते᳚ ऽस्यास्य जु॒षन्ते॑ यजते यजते जु॒षन्ते᳚ ऽस्य । \newline
7. जु॒षन्ते᳚ ऽस्यास्य जु॒षन्ते॑ जु॒षन्ते᳚ ऽस्य दे॒वा दे॒वा अ॑स्य जु॒षन्ते॑ जु॒षन्ते᳚ ऽस्य दे॒वाः । \newline
8. अ॒स्य॒ दे॒वा दे॒वा अ॑स्यास्य दे॒वा ह॒व्यꣳ ह॒व्यम् दे॒वा अ॑स्यास्य दे॒वा ह॒व्यम् । \newline
9. दे॒वा ह॒व्यꣳ ह॒व्यम् दे॒वा दे॒वा ह॒व्य मे॒ष ए॒ष ह॒व्यम् दे॒वा दे॒वा ह॒व्य मे॒षः । \newline
10. ह॒व्य मे॒ष ए॒ष ह॒व्यꣳ ह॒व्य मे॒ष वै वा ए॒ष ह॒व्यꣳ ह॒व्य मे॒ष वै । \newline
11. ए॒ष वै वा ए॒ष ए॒ष वै य॒ज्ञ्स्य॑ य॒ज्ञ्स्य॒ वा ए॒ष ए॒ष वै य॒ज्ञ्स्य॑ । \newline
12. वै य॒ज्ञ्स्य॑ य॒ज्ञ्स्य॒ वै वै य॒ज्ञ्स्य॒ ग्रहो॒ ग्रहो॑ य॒ज्ञ्स्य॒ वै वै य॒ज्ञ्स्य॒ ग्रहः॑ । \newline
13. य॒ज्ञ्स्य॒ ग्रहो॒ ग्रहो॑ य॒ज्ञ्स्य॑ य॒ज्ञ्स्य॒ ग्रहो॑ गृही॒त्वा गृ॑ही॒त्वा ग्रहो॑ य॒ज्ञ्स्य॑ य॒ज्ञ्स्य॒ ग्रहो॑ गृही॒त्वा । \newline
14. ग्रहो॑ गृही॒त्वा गृ॑ही॒त्वा ग्रहो॒ ग्रहो॑ गृही॒त्वैवैव गृ॑ही॒त्वा ग्रहो॒ ग्रहो॑ गृही॒त्वैव । \newline
15. गृ॒ही॒त्वैवैव गृ॑ही॒त्वा गृ॑ही॒त्वैव य॒ज्ञेन॑ य॒ज्ञेनै॒व गृ॑ही॒त्वा गृ॑ही॒त्वैव य॒ज्ञेन॑ । \newline
16. ए॒व य॒ज्ञेन॑ य॒ज्ञेनै॒वैव य॒ज्ञेन॑ यजते यजते य॒ज्ञेनै॒वैव य॒ज्ञेन॑ यजते । \newline
17. य॒ज्ञेन॑ यजते यजते य॒ज्ञेन॑ य॒ज्ञेन॑ यजते॒ तत् तद् य॑जते य॒ज्ञेन॑ य॒ज्ञेन॑ यजते॒ तत् । \newline
18. य॒ज॒ते॒ तत् तद् य॑जते यजते॒ तदु॑दि॒त्वोदि॒त्वा तद् य॑जते यजते॒ तदु॑दि॒त्वा । \newline
19. तदु॑दि॒त्वोदि॒त्वा तत् तदु॑दि॒त्वा वाचं॒ ॅवाच॑ मुदि॒त्वा तत् तदु॑दि॒त्वा वाच᳚म् । \newline
20. उ॒दि॒त्वा वाचं॒ ॅवाच॑ मुदि॒त्वोदि॒त्वा वाचं॑ ॅयच्छति यच्छति॒ वाच॑ मुदि॒त्वोदि॒त्वा वाचं॑ ॅयच्छति । \newline
21. वाचं॑ ॅयच्छति यच्छति॒ वाचं॒ ॅवाचं॑ ॅयच्छति य॒ज्ञ्स्य॑ य॒ज्ञ्स्य॑ यच्छति॒ वाचं॒ ॅवाचं॑ ॅयच्छति य॒ज्ञ्स्य॑ । \newline
22. य॒च्छ॒ति॒ य॒ज्ञ्स्य॑ य॒ज्ञ्स्य॑ यच्छति यच्छति य॒ज्ञ्स्य॒ धृत्यै॒ धृत्यै॑ य॒ज्ञ्स्य॑ यच्छति यच्छति य॒ज्ञ्स्य॒ धृत्यै᳚ । \newline
23. य॒ज्ञ्स्य॒ धृत्यै॒ धृत्यै॑ य॒ज्ञ्स्य॑ य॒ज्ञ्स्य॒ धृत्या॒ अथो॒ अथो॒ धृत्यै॑ य॒ज्ञ्स्य॑ य॒ज्ञ्स्य॒ धृत्या॒ अथो᳚ । \newline
24. धृत्या॒ अथो॒ अथो॒ धृत्यै॒ धृत्या॒ अथो॒ मन॑सा॒ मन॒सा ऽथो॒ धृत्यै॒ धृत्या॒ अथो॒ मन॑सा । \newline
25. अथो॒ मन॑सा॒ मन॒सा ऽथो॒ अथो॒ मन॑सा॒ वै वै मन॒सा ऽथो॒ अथो॒ मन॑सा॒ वै । \newline
26. अथो॒ इत्य॒थो᳚ । \newline
27. मन॑सा॒ वै वै मन॑सा॒ मन॑सा॒ वै प्र॒जाप॑तिः प्र॒जाप॑ति॒र् वै मन॑सा॒ मन॑सा॒ वै प्र॒जाप॑तिः । \newline
28. वै प्र॒जाप॑तिः प्र॒जाप॑ति॒र् वै वै प्र॒जाप॑तिर् य॒ज्ञ्ं ॅय॒ज्ञ्म् प्र॒जाप॑ति॒र् वै वै प्र॒जाप॑तिर् य॒ज्ञ्म् । \newline
29. प्र॒जाप॑तिर् य॒ज्ञ्ं ॅय॒ज्ञ्म् प्र॒जाप॑तिः प्र॒जाप॑तिर् य॒ज्ञ् म॑तनुतातनुत य॒ज्ञ्म् प्र॒जाप॑तिः प्र॒जाप॑तिर् य॒ज्ञ् म॑तनुत । \newline
30. प्र॒जाप॑ति॒रिति॑ प्र॒जा - प॒तिः॒ । \newline
31. य॒ज्ञ् म॑तनुतातनुत य॒ज्ञ्ं ॅय॒ज्ञ् म॑तनुत॒ मन॑सा॒ मन॑सा ऽतनुत य॒ज्ञ्ं ॅय॒ज्ञ् म॑तनुत॒ मन॑सा । \newline
32. अ॒त॒नु॒त॒ मन॑सा॒ मन॑सा ऽतनुतातनुत॒ मन॑सै॒वैव मन॑सा ऽतनुतातनुत॒ मन॑सै॒व । \newline
33. मन॑सै॒वैव मन॑सा॒ मन॑सै॒व तत् तदे॒व मन॑सा॒ मन॑सै॒व तत् । \newline
34. ए॒व तत् तदे॒वैव तद् य॒ज्ञ्ं ॅय॒ज्ञ्म् तदे॒वैव तद् य॒ज्ञ्म् । \newline
35. तद् य॒ज्ञ्ं ॅय॒ज्ञ्म् तत् तद् य॒ज्ञ्म् त॑नुते तनुते य॒ज्ञ्म् तत् तद् य॒ज्ञ्म् त॑नुते । \newline
36. य॒ज्ञ्म् त॑नुते तनुते य॒ज्ञ्ं ॅय॒ज्ञ्म् त॑नुते॒ रक्ष॑सा॒(ग्म्॒) रक्ष॑साम् तनुते य॒ज्ञ्ं ॅय॒ज्ञ्म् त॑नुते॒ रक्ष॑साम् । \newline
37. त॒नु॒ते॒ रक्ष॑सा॒(ग्म्॒) रक्ष॑साम् तनुते तनुते॒ रक्ष॑सा॒ मन॑न्ववचारा॒या न॑न्ववचाराय॒ रक्ष॑साम् तनुते तनुते॒ रक्ष॑सा॒ मन॑न्ववचाराय । \newline
38. रक्ष॑सा॒ मन॑न्ववचारा॒या न॑न्ववचाराय॒ रक्ष॑सा॒(ग्म्॒) रक्ष॑सा॒ मन॑न्ववचाराय॒ यो यो ऽन॑न्ववचाराय॒ रक्ष॑सा॒(ग्म्॒) रक्ष॑सा॒ मन॑न्ववचाराय॒ यः । \newline
39. अन॑न्ववचाराय॒ यो यो ऽन॑न्ववचारा॒या न॑न्ववचाराय॒ यो वै वै यो ऽन॑न्ववचारा॒या न॑न्ववचाराय॒ यो वै । \newline
40. अन॑न्ववचारा॒येत्यन॑नु - अ॒व॒चा॒रा॒य॒ । \newline
41. यो वै वै यो यो वै य॒ज्ञ्ं ॅय॒ज्ञ्ं ॅवै यो यो वै य॒ज्ञ्म् । \newline
42. वै य॒ज्ञ्ं ॅय॒ज्ञ्ं ॅवै वै य॒ज्ञ्ं ॅयोगे॒ योगे॑ य॒ज्ञ्ं ॅवै वै य॒ज्ञ्ं ॅयोगे᳚ । \newline
43. य॒ज्ञ्ं ॅयोगे॒ योगे॑ य॒ज्ञ्ं ॅय॒ज्ञ्ं ॅयोग॒ आग॑त॒ आग॑ते॒ योगे॑ य॒ज्ञ्ं ॅय॒ज्ञ्ं ॅयोग॒ आग॑ते । \newline
44. योग॒ आग॑त॒ आग॑ते॒ योगे॒ योग॒ आग॑ते यु॒नक्ति॑ यु॒नक्त्याग॑ते॒ योगे॒ योग॒ आग॑ते यु॒नक्ति॑ । \newline
45. आग॑ते यु॒नक्ति॑ यु॒नक्त्याग॑त॒ आग॑ते यु॒नक्ति॑ यु॒ङ्क्ते यु॒ङ्क्ते यु॒नक्त्याग॑त॒ आग॑ते यु॒नक्ति॑ यु॒ङ्क्ते । \newline
46. आग॑त॒ इत्या - ग॒ते॒ । \newline
47. यु॒नक्ति॑ यु॒ङ्क्ते यु॒ङ्क्ते यु॒नक्ति॑ यु॒नक्ति॑ यु॒ङ्क्ते यु॑ञ्जा॒नेषु॑ युञ्जा॒नेषु॑ यु॒ङ्क्ते यु॒नक्ति॑ यु॒नक्ति॑ यु॒ङ्क्ते यु॑ञ्जा॒नेषु॑ । \newline
48. यु॒ङ्क्ते यु॑ञ्जा॒नेषु॑ युञ्जा॒नेषु॑ यु॒ङ्क्ते यु॒ङ्क्ते यु॑ञ्जा॒नेषु॒ कः को यु॑ञ्जा॒नेषु॑ यु॒ङ्क्ते यु॒ङ्क्ते यु॑ञ्जा॒नेषु॒ कः । \newline
49. यु॒ञ्जा॒नेषु॒ कः को यु॑ञ्जा॒नेषु॑ युञ्जा॒नेषु॒ कस्त्वा᳚ त्वा॒ को यु॑ञ्जा॒नेषु॑ युञ्जा॒नेषु॒ कस्त्वा᳚ । \newline
50. कस्त्वा᳚ त्वा॒ कः कस्त्वा॑ युनक्ति युनक्ति त्वा॒ कः कस्त्वा॑ युनक्ति । \newline
51. त्वा॒ यु॒न॒क्ति॒ यु॒न॒क्ति॒ त्वा॒ त्वा॒ यु॒न॒क्ति॒ स स यु॑नक्ति त्वा त्वा युनक्ति॒ सः । \newline
52. यु॒न॒क्ति॒ स स यु॑नक्ति युनक्ति॒ स त्वा᳚ त्वा॒ स यु॑नक्ति युनक्ति॒ स त्वा᳚ । \newline
53. स त्वा᳚ त्वा॒ स स त्वा॑ युनक्तु युनक्तु त्वा॒ स स त्वा॑ युनक्तु । \newline
54. त्वा॒ यु॒न॒क्तु॒ यु॒न॒क्तु॒ त्वा॒ त्वा॒ यु॒न॒क्त्वितीति॑ युनक्तु त्वा त्वा युन॒क्त्विति॑ । \newline
55. यु॒न॒क्त्वितीति॑ युनक्तु युन॒क्त्वित्या॑हा॒हे ति॑ युनक्तु युन॒क्त्वित्या॑ह । \newline
56. इत्या॑हा॒हे तीत्या॑ह प्र॒जाप॑तिः प्र॒जाप॑तिरा॒हे तीत्या॑ह प्र॒जाप॑तिः । \newline
57. आ॒ह॒ प्र॒जाप॑तिः प्र॒जाप॑तिराहाह प्र॒जाप॑ति॒र् वै वै प्र॒जाप॑तिराहाह प्र॒जाप॑ति॒र् वै । \newline
58. प्र॒जाप॑ति॒र् वै वै प्र॒जाप॑तिः प्र॒जाप॑ति॒र् वै कः को वै प्र॒जाप॑तिः प्र॒जाप॑ति॒र् वै कः । \newline
59. प्र॒जाप॑ति॒रिति॑ प्र॒जा - प॒तिः॒ । \newline
60. वै कः को वै वै कः प्र॒जाप॑तिना प्र॒जाप॑तिना॒ को वै वै कः प्र॒जाप॑तिना । \newline
61. कः प्र॒जाप॑तिना प्र॒जाप॑तिना॒ कः कः प्र॒जाप॑तिनै॒वैव प्र॒जाप॑तिना॒ कः कः प्र॒जाप॑तिनै॒व । \newline
62. प्र॒जाप॑तिनै॒वैव प्र॒जाप॑तिना प्र॒जाप॑तिनै॒वैन॑ मेन मे॒व प्र॒जाप॑तिना प्र॒जाप॑तिनै॒वैन᳚म् । \newline
63. प्र॒जाप॑ति॒नेति॑ प्र॒जा - प॒ति॒ना॒ । \newline
64. ए॒वैन॑ मेन मे॒वैवैनं॑ ॅयुनक्ति युनक्त्येन मे॒वैवैनं॑ ॅयुनक्ति । \newline
65. ए॒नं॒ ॅयु॒न॒क्ति॒ यु॒न॒क्त्ये॒न॒ मे॒नं॒ ॅयु॒न॒क्ति॒ यु॒ङ्क्ते यु॒ङ्क्ते यु॑नक्त्येन मेनं ॅयुनक्ति यु॒ङ्क्ते । \newline
66. यु॒न॒क्ति॒ यु॒ङ्क्ते यु॒ङ्क्ते यु॑नक्ति युनक्ति यु॒ङ्क्ते यु॑ञ्जा॒नेषु॑ युञ्जा॒नेषु॑ यु॒ङ्क्ते यु॑नक्ति युनक्ति यु॒ङ्क्ते यु॑ञ्जा॒नेषु॑ । \newline
67. यु॒ङ्क्ते यु॑ञ्जा॒नेषु॑ युञ्जा॒नेषु॑ यु॒ङ्क्ते यु॒ङ्क्ते यु॑ञ्जा॒नेषु॑ । \newline
68. यु॒ञ्जा॒नेष्विति॑ युञ्जा॒नेषु॑ । \newline
\pagebreak
\markright{ TS 1.6.9.1  \hfill https://www.vedavms.in \hfill}

\section{ TS 1.6.9.1 }

\textbf{TS 1.6.9.1 } \newline
\textbf{Samhita Paata} \newline

प्र॒जाप॑तिर् य॒ज्ञान॑सृजता-ग्निहो॒त्रं चा᳚ग्निष्टो॒मं च॑ पौर्णमा॒सीं चो॒क्थ्यं॑ चामावा॒स्यां᳚ चातिरा॒त्रं च॒ तानुद॑मिमीत॒ याव॑दग्निहो॒त्र-मासी॒त् तावा॑नग्निष्टो॒मो याव॑ती पौर्णमा॒सी तावा॑नु॒क्थ्यो॑ याव॑त्यमावा॒स्या॑ तावा॑नतिरा॒त्रो य ए॒वं ॅवि॒द्वान॑ग्निहो॒त्रं जु॒होति॒ याव॑दग्निष्टो॒मेनो॑ पा॒प्नोति॒ ताव॒दुपा᳚ऽऽप्नोति॒ य ए॒वं ॅवि॒द्वान् पौ᳚र्णमा॒सीं ॅयज॑ते॒ याव॑दु॒क्थ्ये॑नोपा॒प्नोति॒ - [ ] \newline

\textbf{Pada Paata} \newline

प्र॒जाप॑ति॒रिति॑ प्र॒जा - प॒तिः॒ । य॒ज्ञान् । अ॒सृ॒ज॒त॒ । अ॒ग्नि॒हो॒त्रमित्य॑ग्नि - हो॒त्रम् । च॒ । अ॒ग्नि॒ष्टो॒ममित्य॑ग्नि - स्तो॒मम् । च॒ । पौ॒र्ण॒मा॒सीमिति॑ पौर्ण - मा॒सीम् । च॒ । उ॒क्थ्य᳚म् । च॒ । अ॒मा॒वा॒स्या॑मित्य॑मा-वा॒स्या᳚म् । च॒ । अ॒ति॒रा॒त्रमित्य॑ति - रा॒त्रम् । च॒ । तान् । उदिति॑ । अ॒मि॒मी॒त॒ । याव॑त् । अ॒ग्नि॒हो॒त्रमित्य॑ग्नि - हो॒त्रम् । आसी᳚त् । तावान्॑ । अ॒ग्नि॒ष्टो॒म इत्य॑ग्नि - स्तो॒मः । याव॑ती । पौ॒र्ण॒मा॒सीति॑ पौर्ण - मा॒सी । तावान्॑ । उ॒क्थ्यः॑ । याव॑ती । अ॒मा॒वा॒स्येत्य॑मा - वा॒स्या᳚ । तावान्॑ । अ॒ति॒रा॒त्र इत्य॑ति-रा॒त्रः । यः । ए॒वम् । वि॒द्वान् । अ॒ग्नि॒हो॒त्रमित्य॑ग्नि - हो॒त्रम् । जु॒होति॑ । याव॑त् । अ॒ग्नि॒ष्टो॒मेनेत्य॑ग्नि - स्तो॒मेन॑ । उ॒पा॒प्नोतीत्यु॑प - आ॒प्नोति॑ । ताव॑त् । उपेति॑ । आ॒प्नो॒ति॒ । यः । ए॒वम् । वि॒द्वान् । पौ॒र्ण॒मा॒सीमिति॑ पौर्ण - मा॒सीम् । यज॑ते । याव॑त् । उ॒क्थ्ये॑न । उ॒पा॒प्नोतीत्यु॑प - आ॒प्नोति॑ ।  \newline


\textbf{Krama Paata} \newline

प्र॒जाप॑तिर्,य॒ज्ञान् । प्र॒जाप॑ति॒रिति॑ प्र॒जा - प॒तिः॒ । य॒ज्ञान॑सृजत । अ॒सृ॒ज॒ता॒ग्नि॒हो॒त्रम् । अ॒ग्नि॒हो॒त्रम् च॑ । अ॒ग्नि॒हो॒त्रमित्य॑ग्नि - हो॒त्रम् । चा॒ग्नि॒ष्टो॒मम् । अ॒ग्नि॒ष्टो॒मम् च॑ । अ॒ग्नि॒ष्टो॒ममित्य॑ग्नि - स्तो॒मम् । च॒ पौ॒र्ण॒मा॒सीम् । पौ॒र्ण॒मा॒सीम् च॑ । पौ॒र्ण॒मा॒सीमिति॑ पौर्ण - मा॒सीम् । चो॒क्थ्य᳚म् । उ॒क्थ्य॑म् च । चा॒मा॒वा॒स्या᳚म् । अ॒मा॒वा॒स्या᳚म् च । अ॒मा॒वा॒स्या॑मित्य॑मा - वा॒स्या᳚म् । चा॒ति॒रा॒त्रम् । अ॒ति॒रा॒त्रम् च॑ । अ॒ति॒रा॒त्रमित्य॑ति - रा॒त्रम् । च॒ तान् । तानुत् । उद॑मिमीत । अ॒मि॒मी॒त॒ याव॑त् । याव॑दग्निहो॒त्रम् । अ॒ग्नि॒हो॒त्रमासी᳚त् । अ॒ग्नि॒हो॒त्रमित्य॑ग्नि - हो॒त्रम् । आसी॒त् तावान्॑ । तावा॑नग्निष्टो॒मः । अ॒ग्नि॒ष्टो॒मो याव॑ती । अ॒ग्नि॒ष्टो॒म इत्य॑ग्नि - स्तो॒मः । याव॑ती पौर्णमा॒सी । पौ॒र्ण॒मा॒सी तावान्॑ । पौ॒र्ण॒मा॒सीति॑ पौर्ण - मा॒सी । तावा॑नु॒क्थ्यः॑ । उ॒क्थ्यो॑ याव॑ती । याव॑त्यमावा॒स्या᳚ । अ॒मा॒वा॒स्या॑ तावान्॑ । अ॒मा॒वा॒स्येत्य॑मा - वा॒स्या᳚ । तावा॑नतिरा॒त्रः । अ॒ति॒रा॒त्रो यः । अ॒ति॒रा॒त्र इत्य॑ति - रा॒त्रः । य ए॒वम् । ए॒वं ॅवि॒द्वान् । वि॒द्वान॑ग्निहो॒त्रम् । अ॒ग्नि॒हो॒त्रम् जु॒होति॑ । अ॒ग्नि॒हो॒त्रमित्य॑ग्नि - हो॒त्रम् । जु॒होति॒ याव॑त् । याव॑दग्निष्टो॒मेन॑ । अ॒ग्नि॒ष्टो॒मेनो॑पा॒प्नोति॑ । अ॒ग्नि॒ष्टो॒मेनेत्य॑ग्नि - स्तो॒मेन॑ । उ॒पा॒प्नोति॒ ताव॑त् । उ॒पा॒प्नोतीत्यु॑प - आ॒प्नोति॑ । ताव॒दुप॑ । उपा᳚प्नोति । आ॒प्नो॒ति॒ यः । य ए॒वम् । ए॒वं ॅवि॒द्वान् । वि॒द्वान्,पौ᳚र्णमा॒सीम् । पौ॒र्ण॒मा॒सीं ॅयज॑ते । पौ॒र्ण॒मा॒सीमिति॑ पौर्ण - मा॒सीम् । यज॑ते॒ याव॑त् । याव॑दु॒क्थ्ये॑न । उ॒क्थ्ये॑नोपा॒प्नोति॑ । उ॒पा॒प्नोति॒ ताव॑त् । उ॒पा॒प्नोतीत्यु॑प - आ॒प्नोति॑ \newline

\textbf{Jatai Paata} \newline

1. प्र॒जाप॑तिर् य॒ज्ञान्. य॒ज्ञान् प्र॒जाप॑तिः प्र॒जाप॑तिर् य॒ज्ञान् । \newline
2. प्र॒जाप॑ति॒रिति॑ प्र॒जा - प॒तिः॒ । \newline
3. य॒ज्ञा न॑सृज तासृजत य॒ज्ञान्. य॒ज्ञा न॑सृजत । \newline
4. अ॒सृ॒ज॒ ता॒ग्नि॒हो॒त्र म॑ग्निहो॒त्र म॑सृज तासृजता ग्निहो॒त्रम् । \newline
5. अ॒ग्नि॒हो॒त्रम् च॑ चाग्निहो॒त्र म॑ग्निहो॒त्रम् च॑ । \newline
6. अ॒ग्नि॒हो॒त्रमित्य॑ग्नि - हो॒त्रम् । \newline
7. चा॒ग्नि॒ष्टो॒म म॑ग्निष्टो॒मम् च॑ चाग्निष्टो॒मम् । \newline
8. अ॒ग्नि॒ष्टो॒मम् च॑ चाग्निष्टो॒म म॑ग्निष्टो॒मम् च॑ । \newline
9. अ॒ग्नि॒ष्टो॒ममित्य॑ग्नि - स्तो॒मम् । \newline
10. च॒ पौ॒र्ण॒मा॒सीम् पौ᳚र्णमा॒सीम् च॑ च पौर्णमा॒सीम् । \newline
11. पौ॒र्ण॒मा॒सीम् च॑ च पौर्णमा॒सीम् पौ᳚र्णमा॒सीम् च॑ । \newline
12. पौ॒र्ण॒मा॒सीमिति॑ पौर्ण - मा॒सीम् । \newline
13. चो॒क्थ्य॑ मु॒क्थ्य॑म् च चो॒क्थ्य᳚म् । \newline
14. उ॒क्थ्य॑म् च चो॒क्थ्य॑ मु॒क्थ्य॑म् च । \newline
15. चा॒मा॒वा॒स्या॑ ममावा॒स्या᳚म् च चामावा॒स्या᳚म् । \newline
16. अ॒मा॒वा॒स्या᳚म् च चामावा॒स्या॑ ममावा॒स्या᳚म् च । \newline
17. अ॒मा॒वा॒स्या॑मित्य॑मा - वा॒स्या᳚म् । \newline
18. चा॒ति॒रा॒त्र म॑तिरा॒त्रम् च॑ चातिरा॒त्रम् । \newline
19. अ॒ति॒रा॒त्रम् च॑ चातिरा॒त्र म॑तिरा॒त्रम् च॑ । \newline
20. अ॒ति॒रा॒त्रमित्य॑ति - रा॒त्रम् । \newline
21. च॒ ताꣳ स्ताꣳश्च॑ च॒ तान् । \newline
22. ता नुदुत् ताꣳस्ता नुत् । \newline
23. उद॑मिमी तामिमी॒ तोदुद॑ मिमीत । \newline
24. अ॒मि॒मी॒त॒ याव॒द् याव॑ दमिमी तामिमीत॒ याव॑त् । \newline
25. याव॑दग्निहो॒त्र म॑ग्निहो॒त्रं ॅयाव॒द् याव॑दग्निहो॒त्रम् । \newline
26. अ॒ग्नि॒हो॒त्र मासी॒दासी॑ दग्निहो॒त्र म॑ग्निहो॒त्र मासी᳚त् । \newline
27. अ॒ग्नि॒हो॒त्रमित्य॑ग्नि - हो॒त्रम् । \newline
28. आसी॒त् तावा॒न् तावा॒ नासी॒ दासी॒त् तावान्॑ । \newline
29. तावा॑ नग्निष्टो॒मो᳚ ऽग्निष्टो॒म स्तावा॒न् तावा॑ नग्निष्टो॒मः । \newline
30. अ॒ग्नि॒ष्टो॒मो याव॑ती॒ याव॑ त्यग्निष्टो॒मो᳚ ऽग्निष्टो॒मो याव॑ती । \newline
31. अ॒ग्नि॒ष्टो॒म इत्य॑ग्नि - स्तो॒मः । \newline
32. याव॑ती पौर्णमा॒सी पौ᳚र्णमा॒सी याव॑ती॒ याव॑ती पौर्णमा॒सी । \newline
33. पौ॒र्ण॒मा॒सी तावा॒न् तावा᳚न् पौर्णमा॒सी पौ᳚र्णमा॒सी तावान्॑ । \newline
34. पौ॒र्ण॒मा॒सीति॑ पौर्ण - मा॒सी । \newline
35. तावा॑ नु॒क्थ्य॑ उ॒क्थ्य॑ स्तावा॒न् तावा॑ नु॒क्थ्यः॑ । \newline
36. उ॒क्थ्यो॑ याव॑ती॒ याव॑त्यु॒क्थ्य॑ उ॒क्थ्यो॑ याव॑ती । \newline
37. याव॑त्यमावा॒स्या॑ ऽमावा॒स्या॑ याव॑ती॒ याव॑त्यमावा॒स्या᳚ । \newline
38. अ॒मा॒वा॒स्या॑ तावा॒न् तावा॑ नमावा॒स्या॑ ऽमावा॒स्या॑ तावान्॑ । \newline
39. अ॒मा॒वा॒स्येत्य॑मा - वा॒स्या᳚ । \newline
40. तावा॑ नतिरा॒त्रो॑ ऽतिरा॒त्र स्तावा॒न् तावा॑ नतिरा॒त्रः । \newline
41. अ॒ति॒रा॒त्रो यो यो॑ ऽतिरा॒त्रो॑ ऽतिरा॒त्रो यः । \newline
42. अ॒ति॒रा॒त्र इत्य॑ति - रा॒त्रः । \newline
43. य ए॒व मे॒वं ॅयो य ए॒वम् । \newline
44. ए॒वं ॅवि॒द्वान्. वि॒द्वा ने॒व मे॒वं ॅवि॒द्वान् । \newline
45. वि॒द्वा न॑ग्निहो॒त्र म॑ग्निहो॒त्रं ॅवि॒द्वान्. वि॒द्वा न॑ग्निहो॒त्रम् । \newline
46. अ॒ग्नि॒हो॒त्रम् जु॒होति॑ जु॒हो त्य॑ग्निहो॒त्र म॑ग्निहो॒त्रम् जु॒होति॑ । \newline
47. अ॒ग्नि॒हो॒त्रमित्य॑ग्नि - हो॒त्रम् । \newline
48. जु॒होति॒ याव॒द् याव॑ज् जु॒होति॑ जु॒होति॒ याव॑त् । \newline
49. याव॑ दग्निष्टो॒मे ना᳚ग्निष्टो॒मेन॒ याव॒द् याव॑ दग्निष्टो॒मेन॑ । \newline
50. अ॒ग्नि॒ष्टो॒मे नो॑पा॒प्नो त्यु॑पा॒प्नो त्य॑ग्निष्टो॒मे ना᳚ग्निष्टो॒मे नो॑पा॒प्नोति॑ । \newline
51. अ॒ग्नि॒ष्टो॒मेनेत्य॑ग्नि - स्तो॒मेन॑ । \newline
52. उ॒पा॒प्नोति॒ ताव॒त् ताव॑दुपा॒प्नो त्यु॑पा॒प्नोति॒ ताव॑त् । \newline
53. उ॒पा॒प्नोतीत्यु॑प - आ॒प्नोति॑ । \newline
54. ताव॒ दुपोप॒ ताव॒त् ताव॒ दुप॑ । \newline
55. उपा᳚प्नो त्याप्नो॒ त्यु पोपा᳚प्नोति । \newline
56. आ॒प्नो॒ति॒ यो य आ᳚प्नो त्याप्नोति॒ यः । \newline
57. य ए॒व मे॒वं ॅयो य ए॒वम् । \newline
58. ए॒वं ॅवि॒द्वान्. वि॒द्वा ने॒व मे॒वं ॅवि॒द्वान् । \newline
59. वि॒द्वान् पौ᳚र्णमा॒सीम् पौ᳚र्णमा॒सीं ॅवि॒द्वान्. वि॒द्वान् पौ᳚र्णमा॒सीम् । \newline
60. पौ॒र्ण॒मा॒सीं ॅयज॑ते॒ यज॑ते पौर्णमा॒सीम् पौ᳚र्णमा॒सीं ॅयज॑ते । \newline
61. पौ॒र्ण॒मा॒सीमिति॑ पौर्ण - मा॒सीम् । \newline
62. यज॑ते॒ याव॒द् याव॒द् यज॑ते॒ यज॑ते॒ याव॑त् । \newline
63. याव॑ दु॒क्थ्ये॑ नो॒क्थ्ये॑न॒ याव॒द् याव॑ दु॒क्थ्ये॑न । \newline
64. उ॒क्थ्ये॑ नोपा॒प्नो त्यु॑पा॒प्नो त्यु॒क्थ्ये॑ नो॒क्थ्ये॑नोपा॒प्नोति॑ । \newline
65. उ॒पा॒प्नोति॒ ताव॒त् ताव॑दुपा॒प्नो त्यु॑पा॒प्नोति॒ ताव॑त् । \newline
66. उ॒पा॒प्नोतीत्यु॑प - आ॒प्नोति॑ । \newline

\textbf{Ghana Paata } \newline

1. प्र॒जाप॑तिर् य॒ज्ञान्. य॒ज्ञान् प्र॒जाप॑तिः प्र॒जाप॑तिर् य॒ज्ञा न॑सृजतासृजत य॒ज्ञान् प्र॒जाप॑तिः प्र॒जाप॑तिर् य॒ज्ञा न॑सृजत । \newline
2. प्र॒जाप॑ति॒रिति॑ प्र॒जा - प॒तिः॒ । \newline
3. य॒ज्ञा न॑सृजतासृजत य॒ज्ञान्. य॒ज्ञा न॑सृजताग्निहो॒त्र म॑ग्निहो॒त्र म॑सृजत य॒ज्ञान्. य॒ज्ञा न॑सृजताग्निहो॒त्रम् । \newline
4. अ॒सृ॒ज॒ता॒ग्नि॒हो॒त्र म॑ग्निहो॒त्र म॑सृजता सृजताग्निहो॒त्रम् च॑ चाग्निहो॒त्र म॑सृजता सृजताग्निहो॒त्रम् च॑ । \newline
5. अ॒ग्नि॒हो॒त्रम् च॑ चाग्निहो॒त्र म॑ग्निहो॒त्रम् चा᳚ग्निष्टो॒म म॑ग्निष्टो॒मम् चा᳚ग्निहो॒त्र म॑ग्निहो॒त्रम् चा᳚ग्निष्टो॒मम् । \newline
6. अ॒ग्नि॒हो॒त्रमित्य॑ग्नि - हो॒त्रम् । \newline
7. चा॒ग्नि॒ष्टो॒म म॑ग्निष्टो॒मम् च॑ चाग्निष्टो॒मम् च॑ चाग्निष्टो॒मम् च॑ चाग्निष्टो॒मम् च॑ । \newline
8. अ॒ग्नि॒ष्टो॒मम् च॑ चाग्निष्टो॒म म॑ग्निष्टो॒मम् च॑ पौर्णमा॒सीम् पौ᳚र्णमा॒सीम् चा᳚ग्निष्टो॒म म॑ग्निष्टो॒मम् च॑ पौर्णमा॒सीम् । \newline
9. अ॒ग्नि॒ष्टो॒ममित्य॑ग्नि - स्तो॒मम् । \newline
10. च॒ पौ॒र्ण॒मा॒सीम् पौ᳚र्णमा॒सीम् च॑ च पौर्णमा॒सीम् च॑ च पौर्णमा॒सीम् च॑ च पौर्णमा॒सीम् च॑ । \newline
11. पौ॒र्ण॒मा॒सीम् च॑ च पौर्णमा॒सीम् पौ᳚र्णमा॒सीम् चो॒क्थ्य॑ मु॒क्थ्य॑म् च पौर्णमा॒सीम् पौ᳚र्णमा॒सीम् चो॒क्थ्य᳚म् । \newline
12. पौ॒र्ण॒मा॒सीमिति॑ पौर्ण - मा॒सीम् । \newline
13. चो॒क्थ्य॑ मु॒क्थ्य॑म् च चो॒क्थ्य॑म् च चो॒क्थ्य॑म् च चो॒क्थ्य॑म् च । \newline
14. उ॒क्थ्य॑म् च चो॒क्थ्य॑ मु॒क्थ्य॑म् चामावा॒स्या॑ ममावा॒स्या᳚म् चो॒क्थ्य॑ मु॒क्थ्य॑म् चामावा॒स्या᳚म् । \newline
15. चा॒मा॒वा॒स्या॑ ममावा॒स्या᳚म् च चामावा॒स्या᳚म् च चामावा॒स्या᳚म् च चामावा॒स्या᳚म् च । \newline
16. अ॒मा॒वा॒स्या᳚म् च चामावा॒स्या॑ ममावा॒स्या᳚म् चातिरा॒त्र म॑तिरा॒त्रम् चा॑मावा॒स्या॑ ममावा॒स्या᳚म् चातिरा॒त्रम् । \newline
17. अ॒मा॒वा॒स्या॑मित्य॑मा - वा॒स्या᳚म् । \newline
18. चा॒ति॒रा॒त्र म॑तिरा॒त्रम् च॑ चातिरा॒त्रम् च॑ चातिरा॒त्रम् च॑ चातिरा॒त्रम् च॑ । \newline
19. अ॒ति॒रा॒त्रम् च॑ चातिरा॒त्र म॑तिरा॒त्रम् च॒ ताꣳ स्ताꣳ श्चा॑तिरा॒त्र म॑तिरा॒त्रम् च॒ तान् । \newline
20. अ॒ति॒रा॒त्रमित्य॑ति - रा॒त्रम् । \newline
21. च॒ ताꣳ स्ताꣳ श्च॑ च॒ ता नुदुत् ताꣳ श्च॑ च॒ ता नुत् । \newline
22. ता नुदुत् ताꣳ स्ता नुद॑मिमीता मिमी॒तोत् ताꣳ स्ता नुद॑मिमीत । \newline
23. उद॑मिमीता मिमी॒तोदु द॑मिमीत॒ याव॒द् याव॑दमिमी॒तोदु द॑मिमीत॒ याव॑त् । \newline
24. अ॒मि॒मी॒त॒ याव॒द् याव॑ दमिमीतामिमीत॒ याव॑दग्निहो॒त्र म॑ग्निहो॒त्रं ॅयाव॑ दमिमीतामिमीत॒ याव॑दग्निहो॒त्रम् । \newline
25. याव॑दग्निहो॒त्र म॑ग्निहो॒त्रं ॅयाव॒द् याव॑दग्निहो॒त्र मासी॒ दासी॑ दग्निहो॒त्रं ॅयाव॒द् याव॑दग्निहो॒त्र मासी᳚त् । \newline
26. अ॒ग्नि॒हो॒त्र मासी॒ दासी॑दग्निहो॒त्र म॑ग्निहो॒त्र मासी॒त् तावा॒न् तावा॒ नासी॑दग्निहो॒त्र म॑ग्निहो॒त्र मासी॒त् तावान्॑ । \newline
27. अ॒ग्नि॒हो॒त्रमित्य॑ग्नि - हो॒त्रम् । \newline
28. आसी॒त् तावा॒न् तावा॒ नासी॒दासी॒त् तावा॑ नग्निष्टो॒मो᳚ ऽग्निष्टो॒मस्तावा॒ नासी॒दासी॒त् तावा॑ नग्निष्टो॒मः । \newline
29. तावा॑ नग्निष्टो॒मो᳚ ऽग्निष्टो॒म स्तावा॒न् तावा॑ नग्निष्टो॒मो याव॑ती॒ याव॑त्यग्निष्टो॒म स्तावा॒न् तावा॑ नग्निष्टो॒मो याव॑ती । \newline
30. अ॒ग्नि॒ष्टो॒मो याव॑ती॒ याव॑त्यग्निष्टो॒मो᳚ ऽग्निष्टो॒मो याव॑ती पौर्णमा॒सी पौ᳚र्णमा॒सी याव॑त्यग्निष्टो॒मो᳚ ऽग्निष्टो॒मो याव॑ती पौर्णमा॒सी । \newline
31. अ॒ग्नि॒ष्टो॒म इत्य॑ग्नि - स्तो॒मः । \newline
32. याव॑ती पौर्णमा॒सी पौ᳚र्णमा॒सी याव॑ती॒ याव॑ती पौर्णमा॒सी तावा॒न् तावा᳚न् पौर्णमा॒सी याव॑ती॒ याव॑ती पौर्णमा॒सी तावान्॑ । \newline
33. पौ॒र्ण॒मा॒सी तावा॒न् तावा᳚न् पौर्णमा॒सी पौ᳚र्णमा॒सी तावा॑ नु॒क्थ्य॑ उ॒क्थ्य॑स्तावा᳚न् पौर्णमा॒सी पौ᳚र्णमा॒सी तावा॑ नु॒क्थ्यः॑ । \newline
34. पौ॒र्ण॒मा॒सीति॑ पौर्ण - मा॒सी । \newline
35. तावा॑ नु॒क्थ्य॑ उ॒क्थ्य॑ स्तावा॒न् तावा॑ नु॒क्थ्यो॑ याव॑ती॒ याव॑त्यु॒क्थ्य॑ स्तावा॒न् तावा॑ नु॒क्थ्यो॑ याव॑ती । \newline
36. उ॒क्थ्यो॑ याव॑ती॒ याव॑त्यु॒क्थ्य॑ उ॒क्थ्यो॑ याव॑त्यमावा॒स्या॑ ऽमावा॒स्या॑ याव॑त्यु॒क्थ्य॑ उ॒क्थ्यो॑ याव॑त्यमावा॒स्या᳚ । \newline
37. याव॑त्यमावा॒स्या॑ ऽमावा॒स्या॑ याव॑ती॒ याव॑त्यमावा॒स्या॑ तावा॒न् तावा॑ नमावा॒स्या॑ याव॑ती॒ याव॑त्यमावा॒स्या॑ तावान्॑ । \newline
38. अ॒मा॒वा॒स्या॑ तावा॒न् तावा॑ नमावा॒स्या॑ ऽमावा॒स्या॑ तावा॑ नतिरा॒त्रो॑ ऽतिरा॒त्रस्तावा॑ नमावा॒स्या॑ ऽमावा॒स्या॑ तावा॑ नतिरा॒त्रः । \newline
39. अ॒मा॒वा॒स्येत्य॑मा - वा॒स्या᳚ । \newline
40. तावा॑ नतिरा॒त्रो॑ ऽतिरा॒त्र स्तावा॒न् तावा॑ नतिरा॒त्रो यो यो॑ ऽतिरा॒त्र स्तावा॒न् तावा॑ नतिरा॒त्रो यः । \newline
41. अ॒ति॒रा॒त्रो यो यो॑ ऽतिरा॒त्रो॑ ऽतिरा॒त्रो य ए॒व मे॒वं ॅयो॑ ऽतिरा॒त्रो॑ ऽतिरा॒त्रो य ए॒वम् । \newline
42. अ॒ति॒रा॒त्र इत्य॑ति - रा॒त्रः । \newline
43. य ए॒व मे॒वं ॅयो य ए॒वं ॅवि॒द्वान्. वि॒द्वा ने॒वं ॅयो य ए॒वं ॅवि॒द्वान् । \newline
44. ए॒वं ॅवि॒द्वान्. वि॒द्वा ने॒व मे॒वं ॅवि॒द्वा न॑ग्निहो॒त्र म॑ग्निहो॒त्रं ॅवि॒द्वा ने॒व मे॒वं ॅवि॒द्वा न॑ग्निहो॒त्रम् । \newline
45. वि॒द्वा न॑ग्निहो॒त्र म॑ग्निहो॒त्रं ॅवि॒द्वान्. वि॒द्वा न॑ग्निहो॒त्रम् जु॒होति॑ जु॒होत्य॑ग्निहो॒त्रं ॅवि॒द्वान्. वि॒द्वा न॑ग्निहो॒त्रम् जु॒होति॑ । \newline
46. अ॒ग्नि॒हो॒त्रम् जु॒होति॑ जु॒होत्य॑ग्निहो॒त्र म॑ग्निहो॒त्रम् जु॒होति॒ याव॒द् याव॑ज् जु॒होत्य॑ग्निहो॒त्र म॑ग्निहो॒त्रम् जु॒होति॒ याव॑त् । \newline
47. अ॒ग्नि॒हो॒त्रमित्य॑ग्नि - हो॒त्रम् । \newline
48. जु॒होति॒ याव॒द् याव॑ज् जु॒होति॑ जु॒होति॒ याव॑ दग्निष्टो॒मेना᳚ ग्निष्टो॒मेन॒ याव॑ज् जु॒होति॑ जु॒होति॒ याव॑दग्निष्टो॒मेन॑ । \newline
49. याव॑ दग्निष्टो॒मेना᳚ग्निष्टो॒मेन॒ याव॒द् याव॑ दग्निष्टो॒मे नो॑पा॒प्नो त्यु॑पा॒प्नो त्य॑ग्निष्टो॒मेन॒ याव॒द् याव॑ दग्निष्टो॒मेनो॑पा॒प्नोति॑ । \newline
50. अ॒ग्नि॒ष्टो॒मेनो॑पा॒प्नो त्यु॑पा॒प्नो त्य॑ग्निष्टो॒मेना᳚ ग्निष्टो॒मेनो॑पा॒प्नोति॒ ताव॒त् ताव॑दुपा॒प्नो त्य॑ग्निष्टो॒मेना᳚ग्निष्टो॒मेनो॑पा॒प्नोति॒ ताव॑त् । \newline
51. अ॒ग्नि॒ष्टो॒मेनेत्य॑ग्नि - स्तो॒मेन॑ । \newline
52. उ॒पा॒प्नोति॒ ताव॒त् ताव॑दुपा॒प्नो त्यु॑पा॒प्नोति॒ ताव॒दुपोप॒ ताव॑दुपा॒प्नो त्यु॑पा॒प्नोति॒ ताव॒दुप॑ । \newline
53. उ॒पा॒प्नोतीत्यु॑प - आ॒प्नोति॑ । \newline
54. ताव॒ दुपोप॒ ताव॒त् ताव॒ दुपा᳚प्नो त्याप्नो॒त्युप॒ ताव॒त् ताव॒दुपा᳚प्नोति । \newline
55. उपा᳚प्नो त्याप्नो॒ त्युपोपा᳚प्नोति॒ यो य आ᳚प्नो॒ त्युपोपा᳚प्नोति॒ यः । \newline
56. आ॒प्नो॒ति॒ यो य आ᳚प्नोत्याप्नोति॒ य ए॒व मे॒वं ॅय आ᳚प्नोत्याप्नोति॒ य ए॒वम् । \newline
57. य ए॒व मे॒वं ॅयो य ए॒वं ॅवि॒द्वान्. वि॒द्वा ने॒वं ॅयो य ए॒वं ॅवि॒द्वान् । \newline
58. ए॒वं ॅवि॒द्वान्. वि॒द्वा ने॒व मे॒वं ॅवि॒द्वान् पौ᳚र्णमा॒सीम् पौ᳚र्णमा॒सीं ॅवि॒द्वा ने॒व मे॒वं ॅवि॒द्वान् पौ᳚र्णमा॒सीम् । \newline
59. वि॒द्वान् पौ᳚र्णमा॒सीम् पौ᳚र्णमा॒सीं ॅवि॒द्वान्. वि॒द्वान् पौ᳚र्णमा॒सीं ॅयज॑ते॒ यज॑ते पौर्णमा॒सीं ॅवि॒द्वान्. वि॒द्वान् पौ᳚र्णमा॒सीं ॅयज॑ते । \newline
60. पौ॒र्ण॒मा॒सीं ॅयज॑ते॒ यज॑ते पौर्णमा॒सीम् पौ᳚र्णमा॒सीं ॅयज॑ते॒ याव॒द् याव॒द् यज॑ते पौर्णमा॒सीम् पौ᳚र्णमा॒सीं ॅयज॑ते॒ याव॑त् । \newline
61. पौ॒र्ण॒मा॒सीमिति॑ पौर्ण - मा॒सीम् । \newline
62. यज॑ते॒ याव॒द् याव॒द् यज॑ते॒ यज॑ते॒ याव॑ दु॒क्थ्ये॑ नो॒क्थ्ये॑न॒ याव॒द् यज॑ते॒ यज॑ते॒ याव॑ दु॒क्थ्ये॑न । \newline
63. याव॑ दु॒क्थ्ये॑ नो॒क्थ्ये॑न॒ याव॒द् याव॑ दु॒क्थ्ये॑नोपा॒प्नो त्यु॑पा॒प्नोत्यु॒क्थ्ये॑न॒ याव॒द् याव॑ दु॒क्थ्ये॑नोपा॒प्नोति॑ । \newline
64. उ॒क्थ्ये॑नोपा॒प्नो त्यु॑पा॒प्नो त्यु॒क्थ्ये॑नो॒क्थ्ये॑ नोपा॒प्नोति॒ ताव॒त् ताव॑ दुपा॒प्नो त्यु॒क्थ्ये॑नो॒क्थ्ये॑ नोपा॒प्नोति॒ ताव॑त् । \newline
65. उ॒पा॒प्नोति॒ ताव॒त् ताव॑दुपा॒प्नो त्यु॑पा॒प्नोति॒ ताव॒दुपोप॒ ताव॑दुपा॒प्नो त्यु॑पा॒प्नोति॒ ताव॒दुप॑ । \newline
66. उ॒पा॒प्नोतीत्यु॑प - आ॒प्नोति॑ । \newline
\pagebreak
\markright{ TS 1.6.9.2  \hfill https://www.vedavms.in \hfill}

\section{ TS 1.6.9.2 }

\textbf{TS 1.6.9.2 } \newline
\textbf{Samhita Paata} \newline

ताव॒दुपा᳚ऽऽप्नोति॒ य ए॒वं ॅवि॒द्वान॑मावा॒स्यां᳚ ॅयज॑ते॒ याव॑दतिरा॒त्रेणो॑पा॒प्नोति॒ ताव॒दुपा᳚ऽऽप्नोति परमे॒ष्ठिनो॒ वा ए॒ष य॒ज्ञोऽग्र॑ आसी॒त् तेन॒ स प॑र॒मां काष्ठा॑मगच्छ॒त् तेन॑ प्र॒जाप॑तिं नि॒रवा॑सायय॒त् तेन॑ प्र॒जाप॑तिः पर॒मां काष्ठा॑मगच्छ॒त् तेनेन्द्रं॑ नि॒रवा॑सायय॒त् तेनेन्द्रः॑ पर॒मां काष्ठा॑मगच्छ॒त् तेना॒ऽग्नीषोमौ॑ नि॒रवा॑सायय॒त् तेना॒ग्नीषोमौ॑ प॒रमां काष्ठा॑मगच्छतां॒ ॅय - [ ] \newline

\textbf{Pada Paata} \newline

ताव॑त् । उपेति॑ । आ॒प्नो॒ति॒ । यः । ए॒वम् । वि॒द्वान् । अ॒मा॒वा॒स्या॑मित्य॑मा - वा॒स्या᳚म् । यज॑ते । याव॑त् । अ॒ति॒रा॒त्रेणेत्य॑ति - रा॒त्रेण॑ । उ॒पा॒प्नोतीत्यु॑प - आ॒प्नोति॑ । ताव॑त् । उपेति॑ । आ॒प्नो॒ति॒ । प॒र॒मे॒ष्ठिनः॑ । वै । ए॒षः । य॒ज्ञ्ः । अग्रे᳚ । आ॒सी॒त् । तेन॑ । सः । प॒र॒माम् । काष्ठा᳚म् । अ॒ग॒च्छ॒त् । तेन॑ । प्र॒जाप॑ति॒मिति॑ प्र॒जा - प॒ति॒म् । नि॒रवा॑सायय॒दिति॑ निः - अवा॑साययत् । तेन॑ । प्र॒जाप॑ति॒रिति॑ प्र॒जा - प॒तिः॒ । प॒र॒माम् । काष्ठा᳚म् । अ॒ग॒च्छ॒त् । तेन॑ । इन्द्र᳚म् । नि॒रवा॑सायय॒दिति॑ निः-अवा॑साययत् । तेन॑ । इन्द्रः॑ । प॒र॒माम् । काष्ठा᳚म् । अ॒ग॒च्छ॒त् । तेन॑ । अ॒ग्नीषोमा॒वित्य॒ग्नी - सोमौ᳚ । नि॒रवा॑सायय॒दिति॑ निः - अवा॑साययत् । तेन॑ । अ॒ग्नीषोमा॒वित्य॒ग्नी - सोमौ᳚ । प॒र॒माम् । काष्ठा᳚म् । अ॒ग॒च्छ॒ता॒म् । यः ।  \newline


\textbf{Krama Paata} \newline

ताव॒दुप॑ । उपा᳚प्नोति । आ॒प्नो॒ति॒ यः । य ए॒वम् । ए॒वं ॅवि॒द्वान् । वि॒द्वान॑मावा॒स्या᳚म् । अ॒मा॒वा॒स्यां᳚ ॅयज॑ते । अ॒मा॒वा॒स्या॑मित्य॑मा - वा॒स्या᳚म् । यज॑ते॒ याव॑त् । याव॑दतिरा॒त्रेण॑ । अ॒ति॒रा॒त्रेणो॑पा॒प्नोति॑ । अ॒ति॒रा॒त्रेणेत्य॑ति - रा॒त्रेण॑ । उ॒पा॒प्नोति॒ ताव॑त् । उ॒पा॒प्नोतीत्यु॑प - आ॒प्नोति॑ । ताव॒दुप॑ । उपा᳚प्नोति । आ॒प्नो॒ति॒ प॒र॒मे॒ष्ठिनः॑ । प॒र॒मे॒ष्ठिनो॒ वै । वा ए॒षः । ए॒ष य॒ज्ञ्ः । य॒ज्ञोऽग्रे᳚ । अग्र॑ आसीत् । आ॒सी॒त्,तेन॑ । तेन॒ सः । स प॑र॒माम् । प॒र॒माम् काष्ठा᳚म् । काष्ठा॑मगच्छत् । अ॒ग॒च्छ॒त् तेन॑ । तेन॑ प्र॒जाप॑तिम् । प्र॒जाप॑तिम् नि॒रवा॑साययत् । प्र॒जाप॑ति॒मिति॑ प्र॒जा - प॒ति॒म् । नि॒रवा॑सायय॒त्,तेन॑ । नि॒रवा॑सायय॒दिति॑ निः - अवा॑साययत् । तेन॑ प्र॒जाप॑तिः । प्र॒जाप॑तिः पर॒माम् । प्र॒जाप॑ति॒रिति॑ प्र॒जा - प॒तिः॒ । प॒र॒माम् काष्ठा᳚म् । काष्ठा॑मगच्छत् । अ॒ग॒च्छ॒त् तेन॑ । तेनेन्द्र᳚म् । इन्द्र॑म् नि॒रवा॑साययत् । नि॒रवा॑सायय॒त्,तेन॑ । नि॒रवा॑सायय॒दिति॑ निः - अवा॑साययत् । तेनेन्द्रः॑ । इन्द्रः॑ पर॒माम् । प॒र॒माम् काष्ठा᳚म् । काष्ठा॑मगच्छत् । अ॒ग॒च्छ॒त्,तेन॑ । तेना॒ग्नीषोमौ᳚ । अ॒ग्नीषोमौ॑ नि॒रवा॑साययत् । अ॒ग्नीषोमा॒वित्य॒ग्नी - सोमौ᳚ । नि॒रवा॑सायय॒त्,तेन॑ । नि॒रवा॑सायय॒दिति॑ निः - अवा॑साययत् । तेना॒ग्नीषोमौ᳚ । अ॒ग्नीषोमौ॑ पर॒माम् । अ॒ग्नीषोमा॒वित्य॒ग्नी - सोमौ᳚ । प॒र॒माम् काष्ठा᳚म् । काष्ठा॑मगच्छताम् । अ॒ग॒च्छ॒तां॒ ॅयः । य ए॒वम् \newline

\textbf{Jatai Paata} \newline

1. ताव॒ दुपोप॒ ताव॒त् ताव॒ दुप॑ । \newline
2. उपा᳚प्नो त्याप्नो॒ त्युपोपा᳚प्नोति । \newline
3. आ॒प्नो॒ति॒ यो य आ᳚प्नो त्याप्नोति॒ यः । \newline
4. य ए॒व मे॒वं ॅयो य ए॒वम् । \newline
5. ए॒वं ॅवि॒द्वान्. वि॒द्वा ने॒व मे॒वं ॅवि॒द्वान् । \newline
6. वि॒द्वा न॑मावा॒स्या॑ ममावा॒स्यां᳚ ॅवि॒द्वान्. वि॒द्वा न॑मावा॒स्या᳚म् । \newline
7. अ॒मा॒वा॒स्यां᳚ ॅयज॑ते॒ यज॑ते ऽमावा॒स्या॑ ममावा॒स्यां᳚ ॅयज॑ते । \newline
8. अ॒मा॒वा॒स्या॑मित्य॑मा - वा॒स्या᳚म् । \newline
9. यज॑ते॒ याव॒द् याव॒द् यज॑ते॒ यज॑ते॒ याव॑त् । \newline
10. याव॑ दतिरा॒त्रे णा॑तिरा॒त्रेण॒ याव॒द् याव॑ दतिरा॒त्रेण॑ । \newline
11. अ॒ति॒रा॒त्रे णो॑पा॒प्नो त्यु॑पा॒प्नो त्य॑तिरा॒त्रे णा॑तिरा॒त्रे णो॑पा॒प्नोति॑ । \newline
12. अ॒ति॒रा॒त्रेणेत्य॑ति - रा॒त्रेण॑ । \newline
13. उ॒पा॒प्नोति॒ ताव॒त् ताव॑दुपा॒प्नो त्यु॑पा॒प्नोति॒ ताव॑त् । \newline
14. उ॒पा॒प्नोतीत्यु॑प - आ॒प्नोति॑ । \newline
15. ताव॒ दुपोप॒ ताव॒त् ताव॒ दुप॑ । \newline
16. उपा᳚प्नो त्याप्नो॒ त्युपोपा᳚प्नोति । \newline
17. आ॒प्नो॒ति॒ प॒र॒मे॒ष्ठिनः॑ परमे॒ष्ठिन॑ आप्नो त्याप्नोति परमे॒ष्ठिनः॑ । \newline
18. प॒र॒मे॒ष्ठिनो॒ वै वै प॑रमे॒ष्ठिनः॑ परमे॒ष्ठिनो॒ वै । \newline
19. वा ए॒ष ए॒ष वै वा ए॒षः । \newline
20. ए॒ष य॒ज्ञो य॒ज्ञ् ए॒ष ए॒ष य॒ज्ञ्ः । \newline
21. य॒ज्ञो ऽग्रे ऽग्रे॑ य॒ज्ञो य॒ज्ञो ऽग्रे᳚ । \newline
22. अग्र॑ आसी दासी॒ दग्रे ऽग्र॑ आसीत् । \newline
23. आ॒सी॒त् तेन॒ तेना॑सी दासी॒त् तेन॑ । \newline
24. तेन॒ स स तेन॒ तेन॒ सः । \newline
25. स प॑र॒माम् प॑र॒माꣳ स स प॑र॒माम् । \newline
26. प॒र॒माम् काष्ठा॒म् काष्ठा᳚म् पर॒माम् प॑र॒माम् काष्ठा᳚म् । \newline
27. काष्ठा॑ मगच्छ दगच्छ॒त् काष्ठा॒म् काष्ठा॑ मगच्छत् । \newline
28. अ॒ग॒च्छ॒त् तेन॒ तेना॑गच्छ दगच्छ॒त् तेन॑ । \newline
29. तेन॑ प्र॒जाप॑तिम् प्र॒जाप॑ति॒म् तेन॒ तेन॑ प्र॒जाप॑तिम् । \newline
30. प्र॒जाप॑तिम् नि॒रवा॑साययन् नि॒रवा॑साययत् प्र॒जाप॑तिम् प्र॒जाप॑तिम् नि॒रवा॑साययत् । \newline
31. प्र॒जाप॑ति॒मिति॑ प्र॒जा - प॒ति॒म् । \newline
32. नि॒रवा॑सायय॒त् तेन॒ तेन॑ नि॒रवा॑साययन् नि॒रवा॑सायय॒त् तेन॑ । \newline
33. नि॒रवा॑सायय॒दिति॑ निः - अवा॑साययत् । \newline
34. तेन॑ प्र॒जाप॑तिः प्र॒जाप॑ति॒ स्तेन॒ तेन॑ प्र॒जाप॑तिः । \newline
35. प्र॒जाप॑तिः पर॒माम् प॑र॒माम् प्र॒जाप॑तिः प्र॒जाप॑तिः पर॒माम् । \newline
36. प्र॒जाप॑ति॒रिति॑ प्र॒जा - प॒तिः॒ । \newline
37. प॒र॒माम् काष्ठा॒म् काष्ठा᳚म् पर॒माम् प॑र॒माम् काष्ठा᳚म् । \newline
38. काष्ठा॑ मगच्छ दगच्छ॒त् काष्ठा॒म् काष्ठा॑ मगच्छत् । \newline
39. अ॒ग॒च्छ॒त् तेन॒ तेना॑गच्छ दगच्छ॒त् तेन॑ । \newline
40. तेने न्द्र॒ मिन्द्र॒म् तेन॒ तेने न्द्र᳚म् । \newline
41. इन्द्र॑म् नि॒रवा॑साययन् नि॒रवा॑सायय॒ दिन्द्र॒ मिन्द्र॑म् नि॒रवा॑साययत् । \newline
42. नि॒रवा॑सायय॒त् तेन॒ तेन॑ नि॒रवा॑साययन् नि॒रवा॑सायय॒त् तेन॑ । \newline
43. नि॒रवा॑सायय॒दिति॑ निः - अवा॑साययत् । \newline
44. तेने न्द्र॒ इन्द्र॒ स्तेन॒ तेने न्द्रः॑ । \newline
45. इन्द्रः॑ पर॒माम् प॑र॒मा मिन्द्र॒ इन्द्रः॑ पर॒माम् । \newline
46. प॒र॒माम् काष्ठा॒म् काष्ठा᳚म् पर॒माम् प॑र॒माम् काष्ठा᳚म् । \newline
47. काष्ठा॑ मगच्छ दगच्छ॒त् काष्ठा॒म् काष्ठा॑ मगच्छत् । \newline
48. अ॒ग॒च्छ॒त् तेन॒ तेना॑गच्छ दगच्छ॒त् तेन॑ । \newline
49. तेना॒ग्नीषोमा॑ व॒ग्नीषोमौ॒ तेन॒ तेना॒ग्नीषोमौ᳚ । \newline
50. अ॒ग्नीषोमौ॑ नि॒रवा॑साययन् नि॒रवा॑सायय द॒ग्नीषोमा॑ व॒ग्नीषोमौ॑ नि॒रवा॑साययत् । \newline
51. अ॒ग्नीषोमा॒वित्य॒ग्नी - सोमौ᳚ । \newline
52. नि॒रवा॑सायय॒त् तेन॒ तेन॑ नि॒रवा॑साययन् नि॒रवा॑सायय॒त् तेन॑ । \newline
53. नि॒रवा॑सायय॒दिति॑ निः - अवा॑साययत् । \newline
54. तेना॒ग्नीषोमा॑ व॒ग्नीषोमौ॒ तेन॒ तेना॒ग्नीषोमौ᳚ । \newline
55. अ॒ग्नीषोमौ॑ पर॒माम् प॑र॒मा म॒ग्नीषोमा॑ व॒ग्नीषोमौ॑ पर॒माम् । \newline
56. अ॒ग्नीषोमा॒वित्य॒ग्नी - सोमौ᳚ । \newline
57. प॒र॒माम् काष्ठा॒म् काष्ठा᳚म् पर॒माम् प॑र॒माम् काष्ठा᳚म् । \newline
58. काष्ठा॑ मगच्छता मगच्छता॒म् काष्ठा॒म् काष्ठा॑ मगच्छताम् । \newline
59. अ॒ग॒च्छ॒तां॒ ॅयो यो॑ ऽगच्छता मगच्छतां॒ ॅयः । \newline
60. य ए॒व मे॒वं ॅयो य ए॒वम् । \newline

\textbf{Ghana Paata } \newline

1. ताव॒दुपोप॒ ताव॒त् ताव॒ दुपा᳚प्नो त्याप्नो॒त्युप॒ ताव॒त् ताव॒दुपा᳚प्नोति । \newline
2. उपा᳚प्नो त्याप्नो॒ त्युपोपा᳚प्नोति॒ यो य आ᳚प्नो॒ त्युपोपा᳚प्नोति॒ यः । \newline
3. आ॒प्नो॒ति॒ यो य आ᳚प्नोत्याप्नोति॒ य ए॒व मे॒वं ॅय आ᳚प्नोत्याप्नोति॒ य ए॒वम् । \newline
4. य ए॒व मे॒वं ॅयो य ए॒वं ॅवि॒द्वान्. वि॒द्वा ने॒वं ॅयो य ए॒वं ॅवि॒द्वान् । \newline
5. ए॒वं ॅवि॒द्वान्. वि॒द्वा ने॒व मे॒वं ॅवि॒द्वा न॑मावा॒स्या॑ ममावा॒स्यां᳚ ॅवि॒द्वा ने॒व मे॒वं ॅवि॒द्वा न॑मावा॒स्या᳚म् । \newline
6. वि॒द्वा न॑मावा॒स्या॑ ममावा॒स्यां᳚ ॅवि॒द्वान्. वि॒द्वा न॑मावा॒स्यां᳚ ॅयज॑ते॒ यज॑ते ऽमावा॒स्यां᳚ ॅवि॒द्वान्. वि॒द्वा न॑मावा॒स्यां᳚ ॅयज॑ते । \newline
7. अ॒मा॒वा॒स्यां᳚ ॅयज॑ते॒ यज॑ते ऽमावा॒स्या॑ ममावा॒स्यां᳚ ॅयज॑ते॒ याव॒द् याव॒द् यज॑ते ऽमावा॒स्या॑ ममावा॒स्यां᳚ ॅयज॑ते॒ याव॑त् । \newline
8. अ॒मा॒वा॒स्या॑मित्य॑मा - वा॒स्या᳚म् । \newline
9. यज॑ते॒ याव॒द् याव॒द् यज॑ते॒ यज॑ते॒ याव॑ दतिरा॒त्रे णा॑तिरा॒त्रेण॒ याव॒द् यज॑ते॒ यज॑ते॒ याव॑ दतिरा॒त्रेण॑ । \newline
10. याव॑ दतिरा॒त्रे णा॑तिरा॒त्रेण॒ याव॒द् याव॑ दतिरा॒त्रेणो॑पा॒प्नो त्यु॑पा॒प्नोत्य॑तिरा॒त्रेण॒ याव॒द् याव॑दतिरा॒त्रेणो॑पा॒प्नोति॑ । \newline
11. अ॒ति॒रा॒त्रेणो॑पा॒प्नो त्यु॑पा॒प्नो त्य॑तिरा॒त्रेणा॑ तिरा॒त्रेणो॑पा॒प्नोति॒ ताव॒त् ताव॑ दुपा॒प्नो त्य॑तिरा॒त्रेणा॑ तिरा॒त्रेणो॑पा॒प्नोति॒ ताव॑त् । \newline
12. अ॒ति॒रा॒त्रेणेत्य॑ति - रा॒त्रेण॑ । \newline
13. उ॒पा॒प्नोति॒ ताव॒त् ताव॑दुपा॒प्नो त्यु॑पा॒प्नोति॒ ताव॒दुपोप॒ ताव॑दुपा॒प्नो त्यु॑पा॒प्नोति॒ ताव॒दुप॑ । \newline
14. उ॒पा॒प्नोतीत्यु॑प - आ॒प्नोति॑ । \newline
15. ताव॒दुपोप॒ ताव॒त् ताव॒दुपा᳚प्नो त्याप्नो॒त्युप॒ ताव॒त् ताव॒दुपा᳚प्नोति । \newline
16. उपा᳚प्नो त्याप्नो॒ त्युपोपा᳚प्नोति परमे॒ष्ठिनः॑ परमे॒ष्ठिन॑ आप्नो॒ त्युपोपा᳚प्नोति परमे॒ष्ठिनः॑ । \newline
17. आ॒प्नो॒ति॒ प॒र॒मे॒ष्ठिनः॑ परमे॒ष्ठिन॑ आप्नोत्याप्नोति परमे॒ष्ठिनो॒ वै वै प॑रमे॒ष्ठिन॑ आप्नोत्याप्नोति परमे॒ष्ठिनो॒ वै । \newline
18. प॒र॒मे॒ष्ठिनो॒ वै वै प॑रमे॒ष्ठिनः॑ परमे॒ष्ठिनो॒ वा ए॒ष ए॒ष वै प॑रमे॒ष्ठिनः॑ परमे॒ष्ठिनो॒ वा ए॒षः । \newline
19. वा ए॒ष ए॒ष वै वा ए॒ष य॒ज्ञो य॒ज्ञ् ए॒ष वै वा ए॒ष य॒ज्ञ्ः । \newline
20. ए॒षो य॒ज्ञो य॒ज्ञ् ए॒ष ए॒ष य॒ज्ञो ऽग्रे ऽग्रे॑ य॒ज्ञ् ए॒ष ए॒ष य॒ज्ञो ऽग्रे᳚ । \newline
21. य॒ज्ञो ऽग्रे ऽग्रे॑ य॒ज्ञो य॒ज्ञो ऽग्र॑ आसी दासी॒दग्रे॑ य॒ज्ञो य॒ज्ञो ऽग्र॑ आसीत् । \newline
22. अग्र॑ आसी दासी॒दग्रे ऽग्र॑ आसी॒त् तेन॒ तेना॑सी॒दग्रे ऽग्र॑ आसी॒त् तेन॑ । \newline
23. आ॒सी॒त् तेन॒ तेना॑सी दासी॒त् तेन॒ स स तेना॑सी दासी॒त् तेन॒ सः । \newline
24. तेन॒ स स तेन॒ तेन॒ स प॑र॒माम् प॑र॒माꣳ स तेन॒ तेन॒ स प॑र॒माम् । \newline
25. स प॑र॒माम् प॑र॒माꣳ स स प॑र॒माम् काष्ठा॒म् काष्ठा᳚म् पर॒माꣳ स स प॑र॒माम् काष्ठा᳚म् । \newline
26. प॒र॒माम् काष्ठा॒म् काष्ठा᳚म् पर॒माम् प॑र॒माम् काष्ठा॑ मगच्छ दगच्छ॒त् काष्ठा᳚म् पर॒माम् प॑र॒माम् काष्ठा॑ मगच्छत् । \newline
27. काष्ठा॑ मगच्छ दगच्छ॒त् काष्ठा॒म् काष्ठा॑ मगच्छ॒त् तेन॒ तेना॑गच्छ॒त् काष्ठा॒म् काष्ठा॑ मगच्छ॒त् तेन॑ । \newline
28. अ॒ग॒च्छ॒त् तेन॒ तेना॑गच्छ दगच्छ॒त् तेन॑ प्र॒जाप॑तिम् प्र॒जाप॑ति॒म् तेना॑गच्छ दगच्छ॒त् तेन॑ प्र॒जाप॑तिम् । \newline
29. तेन॑ प्र॒जाप॑तिम् प्र॒जाप॑ति॒म् तेन॒ तेन॑ प्र॒जाप॑तिम् नि॒रवा॑साययन् नि॒रवा॑साययत् प्र॒जाप॑ति॒म् तेन॒ तेन॑ प्र॒जाप॑तिम् नि॒रवा॑साययत् । \newline
30. प्र॒जाप॑तिम् नि॒रवा॑साययन् नि॒रवा॑साययत् प्र॒जाप॑तिम् प्र॒जाप॑तिम् नि॒रवा॑सायय॒त् तेन॒ तेन॑ नि॒रवा॑साययत् प्र॒जाप॑तिम् प्र॒जाप॑तिम् नि॒रवा॑सायय॒त् तेन॑ । \newline
31. प्र॒जाप॑ति॒मिति॑ प्र॒जा - प॒ति॒म् । \newline
32. नि॒रवा॑सायय॒त् तेन॒ तेन॑ नि॒रवा॑साययन् नि॒रवा॑सायय॒त् तेन॑ प्र॒जाप॑तिः प्र॒जाप॑ति॒स्तेन॑ नि॒रवा॑साययन् नि॒रवा॑सायय॒त् तेन॑ प्र॒जाप॑तिः । \newline
33. नि॒रवा॑सायय॒दिति॑ निः - अवा॑साययत् । \newline
34. तेन॑ प्र॒जाप॑तिः प्र॒जाप॑ति॒ स्तेन॒ तेन॑ प्र॒जाप॑तिः पर॒माम् प॑र॒माम् प्र॒जाप॑ति॒ स्तेन॒ तेन॑ प्र॒जाप॑तिः पर॒माम् । \newline
35. प्र॒जाप॑तिः पर॒माम् प॑र॒माम् प्र॒जाप॑तिः प्र॒जाप॑तिः पर॒माम् काष्ठा॒म् काष्ठा᳚म् पर॒माम् प्र॒जाप॑तिः प्र॒जाप॑तिः पर॒माम् काष्ठा᳚म् । \newline
36. प्र॒जाप॑ति॒रिति॑ प्र॒जा - प॒तिः॒ । \newline
37. प॒र॒माम् काष्ठा॒म् काष्ठा᳚म् पर॒माम् प॑र॒माम् काष्ठा॑ मगच्छ दगच्छ॒त् काष्ठा᳚म् पर॒माम् प॑र॒माम् काष्ठा॑ मगच्छत् । \newline
38. काष्ठा॑ मगच्छ दगच्छ॒त् काष्ठा॒म् काष्ठा॑ मगच्छ॒त् तेन॒ तेना॑गच्छ॒त् काष्ठा॒म् काष्ठा॑ मगच्छ॒त् तेन॑ । \newline
39. अ॒ग॒च्छ॒त् तेन॒ तेना॑गच्छ दगच्छ॒त् तेने न्द्र॒ मिन्द्र॒म् तेना॑गच्छ दगच्छ॒त् तेने न्द्र᳚म् । \newline
40. तेने न्द्र॒ मिन्द्र॒म् तेन॒ तेने न्द्र॑म् नि॒रवा॑साययन् नि॒रवा॑सायय॒ दिन्द्र॒म् तेन॒ तेने न्द्र॑म् नि॒रवा॑साययत् । \newline
41. इन्द्र॑म् नि॒रवा॑साययन् नि॒रवा॑सायय॒ दिन्द्र॒ मिन्द्र॑म् नि॒रवा॑सायय॒त् तेन॒ तेन॑ नि॒रवा॑सायय॒ दिन्द्र॒ मिन्द्र॑म् नि॒रवा॑सायय॒त् तेन॑ । \newline
42. नि॒रवा॑सायय॒त् तेन॒ तेन॑ नि॒रवा॑साययन् नि॒रवा॑सायय॒त् तेने न्द्र॒ इन्द्र॒स्तेन॑ नि॒रवा॑साययन् नि॒रवा॑सायय॒त् तेने न्द्रः॑ । \newline
43. नि॒रवा॑सायय॒दिति॑ निः - अवा॑साययत् । \newline
44. तेने न्द्र॒ इन्द्र॒ स्तेन॒ तेने न्द्रः॑ पर॒माम् प॑र॒मा मिन्द्र॒स्तेन॒ तेने न्द्रः॑ पर॒माम् । \newline
45. इन्द्रः॑ पर॒माम् प॑र॒मा मिन्द्र॒ इन्द्रः॑ पर॒माम् काष्ठा॒म् काष्ठा᳚म् पर॒मा मिन्द्र॒ इन्द्रः॑ पर॒माम् काष्ठा᳚म् । \newline
46. प॒र॒माम् काष्ठा॒म् काष्ठा᳚म् पर॒माम् प॑र॒माम् काष्ठा॑ मगच्छ दगच्छ॒त् काष्ठा᳚म् पर॒माम् प॑र॒माम् काष्ठा॑ मगच्छत् । \newline
47. काष्ठा॑ मगच्छ दगच्छ॒त् काष्ठा॒म् काष्ठा॑ मगच्छ॒त् तेन॒ तेना॑गच्छ॒त् काष्ठा॒म् काष्ठा॑ मगच्छ॒त् तेन॑ । \newline
48. अ॒ग॒च्छ॒त् तेन॒ तेना॑गच्छ दगच्छ॒त् तेना॒ग्नीषोमा॑ व॒ग्नीषोमौ॒ तेना॑गच्छ दगच्छ॒त् तेना॒ग्नीषोमौ᳚ । \newline
49. तेना॒ग्नीषोमा॑ व॒ग्नीषोमौ॒ तेन॒ तेना॒ग्नीषोमौ॑ नि॒रवा॑साययन् नि॒रवा॑सायय द॒ग्नीषोमौ॒ तेन॒ तेना॒ग्नीषोमौ॑ नि॒रवा॑साययत् । \newline
50. अ॒ग्नीषोमौ॑ नि॒रवा॑साययन् नि॒रवा॑सायय द॒ग्नीषोमा॑ व॒ग्नीषोमौ॑ नि॒रवा॑सायय॒त् तेन॒ तेन॑ नि॒रवा॑सायय द॒ग्नीषोमा॑ व॒ग्नीषोमौ॑ नि॒रवा॑सायय॒त् तेन॑ । \newline
51. अ॒ग्नीषोमा॒वित्य॒ग्नी - सोमौ᳚ । \newline
52. नि॒रवा॑सायय॒त् तेन॒ तेन॑ नि॒रवा॑साययन् नि॒रवा॑सायय॒त् तेना॒ग्नीषोमा॑ व॒ग्नीषोमौ॒ तेन॑ नि॒रवा॑साययन् नि॒रवा॑सायय॒त् तेना॒ग्नीषोमौ᳚ । \newline
53. नि॒रवा॑सायय॒दिति॑ निः - अवा॑साययत् । \newline
54. तेना॒ग्नीषोमा॑ व॒ग्नीषोमौ॒ तेन॒ तेना॒ग्नीषोमौ॑ पर॒माम् प॑र॒मा म॒ग्नीषोमौ॒ तेन॒ तेना॒ग्नीषोमौ॑ पर॒माम् । \newline
55. अ॒ग्नीषोमौ॑ पर॒माम् प॑र॒मा म॒ग्नीषोमा॑ व॒ग्नीषोमौ॑ पर॒माम् काष्ठा॒म् काष्ठा᳚म् पर॒मा म॒ग्नीषोमा॑ व॒ग्नीषोमौ॑ पर॒माम् काष्ठा᳚म् । \newline
56. अ॒ग्नीषोमा॒वित्य॒ग्नी - सोमौ᳚ । \newline
57. प॒र॒माम् काष्ठा॒म् काष्ठा᳚म् पर॒माम् प॑र॒माम् काष्ठा॑ मगच्छता मगच्छता॒म् काष्ठा᳚म् पर॒माम् प॑र॒माम् काष्ठा॑ मगच्छताम् । \newline
58. काष्ठा॑ मगच्छता मगच्छता॒म् काष्ठा॒म् काष्ठा॑ मगच्छतां॒ ॅयो यो॑ ऽगच्छता॒म् काष्ठा॒म् काष्ठा॑ मगच्छतां॒ ॅयः । \newline
59. अ॒ग॒च्छ॒तां॒ ॅयो यो॑ ऽगच्छता मगच्छतां॒ ॅय ए॒व मे॒वं ॅयो॑ ऽगच्छता मगच्छतां॒ ॅय ए॒वम् । \newline
60. य ए॒व मे॒वं ॅयो य ए॒वं ॅवि॒द्वान्. वि॒द्वा ने॒वं ॅयो य ए॒वं ॅवि॒द्वान् । \newline
\pagebreak
\markright{ TS 1.6.9.3  \hfill https://www.vedavms.in \hfill}

\section{ TS 1.6.9.3 }

\textbf{TS 1.6.9.3 } \newline
\textbf{Samhita Paata} \newline

ए॒वं ॅवि॒द्वान् द॑र्.शपूर्णमा॒सौ यज॑ते पर॒मामे॒व काष्ठां᳚ गच्छति॒ यो वै प्रजा॑तेन य॒ज्ञेन॒ यज॑ते॒ प्र प्र॒जया॑ प॒शुभि॑र् मिथु॒नैर् जा॑यते॒ द्वाद॑श॒ मासाः᳚ संॅवथ्स॒रो द्वाद॑श द्व॒न्द्वानि॑ दर्.शपूर्णमा॒सयो॒स्तानि॑ सं॒पाद्या॒नीत्या॑हुर् व॒थ्सं चो॑पावसृ॒जत्यु॒खां चाधि॑ श्रय॒त्यव॑ च॒ हन्ति॑ दृ॒षदौ॑ च स॒माह॒न्त्यधि॑ च॒ वप॑ते क॒पाला॑नि॒ चोप॑ दधाति पुरो॒डाशं॑ - [ ] \newline

\textbf{Pada Paata} \newline

ए॒वम् । वि॒द्वान् । द॒र्॒.श॒पू॒र्ण॒मा॒साविति॑ दर्.श - पू॒र्ण॒मा॒सौ । यज॑ते । प॒र॒माम् । ए॒व । काष्ठा᳚म् । ग॒च्छ॒ति॒ । यः । वै । प्रजा॑ते॒नेति॒ प्र - जा॒ते॒न॒ । य॒ज्ञेन॑ । यज॑ते । प्रेति॑ । प्र॒जयेति॑ प्र - जया᳚ । प॒शुभि॒रिति॑ प॒शु - भिः॒ । मि॒थु॒नैः । जा॒य॒ते॒ । द्वाद॑श । मासाः᳚ । सं॒ॅव॒थ्स॒र इति॑ सं - व॒थ्स॒रः । द्वाद॑श । द्व॒न्द्वानीति॑ द्वं - द्वानि॑ । द॒र्॒.श॒पू॒र्ण॒मा॒सयो॒रिति॑ दर्.श - पू॒र्ण॒मा॒सयोः᳚ । तानि॑ । स॒पांद्या॒नीति॑ सं- पाद्या॑नि । इति॑ । आ॒हुः॒ । व॒थ्सम् । च॒ । उ॒पा॒व॒सृ॒जतीत्यु॑प - अ॒व॒सृ॒जति॑ । उ॒खाम् । च॒ । अधीति॑ । श्र॒य॒ति॒ । अवेति॑ । च॒ । हन्ति॑ । दृ॒षदौ᳚ । च॒ । स॒माह॒न्तीति॑ सं - आह॑न्ति । अधीति॑ । च॒ । वप॑ते । क॒पाला॑नि । च॒ । उपेति॑ । द॒धा॒ति॒ । पु॒रो॒डाश᳚म् । च॒ ।  \newline


\textbf{Krama Paata} \newline

ए॒वं ॅवि॒द्वान् । वि॒द्वान्,द॑र्.शपूर्णमा॒सौ । द॒र्॒.श॒पू॒र्ण॒मा॒सौ यज॑ते । द॒र्॒.श॒पू॒र्ण॒मा॒साविति॑ दर्.श - पू॒र्ण॒मा॒सौ । यज॑ते पर॒माम् । प॒र॒मामे॒व । ए॒व काष्ठा᳚म् । काष्ठा᳚म् गच्छति । ग॒च्छ॒ति॒ यः । यो वै । वै प्रजा॑तेन । प्रजा॑तेन य॒ज्ञेन॑ । प्रजा॑ते॒नेति॒ प्र - जा॒ते॒न॒ । य॒ज्ञेन॒ यज॑ते । यज॑ते॒ प्र । प्र प्र॒जया᳚ । प्र॒जया॑ प॒शुभिः॑ । प्र॒जयेति॑ प्र - जया᳚ । प॒शुभि॑र् मिथु॒नैः । प॒शुभि॒रिति॑ प॒शु - भिः॒ । मि॒थु॒नैर् जा॑यते । जा॒य॒ते॒ द्वाद॑श । द्वाद॑श॒ मासाः᳚ । मासाः᳚ सम्ॅवथ्स॒रः । स॒म्ॅव॒थ्स॒रो द्वाद॑श । स॒म्ॅव॒थ्स॒र इति॑ सं - व॒थ्स॒रः । द्वाद॑श द्व॒न्द्वानि॑ । द्व॒न्द्वानि॑ दर्.शपूर्णमा॒सयोः᳚ । द्व॒न्द्वानीति॑ द्वम् - द्वानि॑ । द॒र्.॒श॒पू॒र्ण॒मा॒सयो॒स्तानि॑ । द॒र्॒.श॒पू॒र्ण॒मा॒सयो॒रिति॑ दर्.श - पू॒र्ण॒मा॒सयोः᳚ । तानि॑ स॒म्पाद्या॑नि । स॒म्पाद्या॒नीति॑ । स॒म्पाद्या॒नीति॑ सम् - पाद्या॑नि । इत्या॑हुः । आ॒हु॒र्,व॒थ्सम् । व॒थ्सम् च॑ । चो॒पा॒व॒सृ॒जति॑ । उ॒पा॒व॒सृ॒जत्यु॒खाम् । उ॒पा॒व॒सृ॒जतीत्यु॑प - अ॒व॒सृ॒जति॑ । उ॒खाम् च॑ । चाधि॑ । अधि॑ श्रयति । श्र॒य॒त्यव॑ । अव॑ च । च॒ हन्ति॑ । हन्ति॑ दृ॒षदौ᳚ । दृ॒षदौ॑ च । च॒ स॒माह॑न्ति । स॒माह॒न्त्यधि॑ । स॒माह॒न्तीति॑ सम् - आह॑न्ति । अधि॑ च । च॒ वप॑ते । वप॑ते क॒पाला॑नि । क॒पाला॑नि च । चोप॑ । उप॑ दधाति । द॒धा॒ति॒ पु॒रो॒डाश᳚म् । पु॒रो॒डाश॑म् च । चा॒धि॒श्रय॑ति \newline

\textbf{Jatai Paata} \newline

1. ए॒वं ॅवि॒द्वान्. वि॒द्वा ने॒व मे॒वं ॅवि॒द्वान् । \newline
2. वि॒द्वान् द॑र्.शपूर्णमा॒सौ द॑र्.शपूर्णमा॒सौ वि॒द्वान्. वि॒द्वान् द॑र्.शपूर्णमा॒सौ । \newline
3. द॒र्॒.श॒पू॒र्ण॒मा॒सौ यज॑ते॒ यज॑ते दर्.शपूर्णमा॒सौ द॑र्.शपूर्णमा॒सौ यज॑ते । \newline
4. द॒र्॒.श॒पू॒र्ण॒मा॒साविति॑ दर्.श - पू॒र्ण॒मा॒सौ । \newline
5. यज॑ते पर॒माम् प॑र॒मां ॅयज॑ते॒ यज॑ते पर॒माम् । \newline
6. प॒र॒मा मे॒वैव प॑र॒माम् प॑र॒मा मे॒व । \newline
7. ए॒व काष्ठा॒म् काष्ठा॑ मे॒वैव काष्ठा᳚म् । \newline
8. काष्ठा᳚म् गच्छति गच्छति॒ काष्ठा॒म् काष्ठा᳚म् गच्छति । \newline
9. ग॒च्छ॒ति॒ यो यो ग॑च्छति गच्छति॒ यः । \newline
10. यो वै वै यो यो वै । \newline
11. वै प्रजा॑तेन॒ प्रजा॑तेन॒ वै वै प्रजा॑तेन । \newline
12. प्रजा॑तेन य॒ज्ञेन॑ य॒ज्ञेन॒ प्रजा॑तेन॒ प्रजा॑तेन य॒ज्ञेन॑ । \newline
13. प्रजा॑ते॒नेति॒ प्र - जा॒ते॒न॒ । \newline
14. य॒ज्ञेन॒ यज॑ते॒ यज॑ते य॒ज्ञेन॑ य॒ज्ञेन॒ यज॑ते । \newline
15. यज॑ते॒ प्र प्र यज॑ते॒ यज॑ते॒ प्र । \newline
16. प्र प्र॒जया᳚ प्र॒जया॒ प्र प्र प्र॒जया᳚ । \newline
17. प्र॒जया॑ प॒शुभिः॑ प॒शुभिः॑ प्र॒जया᳚ प्र॒जया॑ प॒शुभिः॑ । \newline
18. प्र॒जयेति॑ प्र - जया᳚ । \newline
19. प॒शुभि॑र् मिथु॒नैर् मि॑थु॒नैः प॒शुभिः॑ प॒शुभि॑र् मिथु॒नैः । \newline
20. प॒शुभि॒रिति॑ प॒शु - भिः॒ । \newline
21. मि॒थु॒नैर् जा॑यते जायते मिथु॒नैर् मि॑थु॒नैर् जा॑यते । \newline
22. जा॒य॒ते॒ द्वाद॑श॒ द्वाद॑श जायते जायते॒ द्वाद॑श । \newline
23. द्वाद॑श॒ मासा॒ मासा॒ द्वाद॑श॒ द्वाद॑श॒ मासाः᳚ । \newline
24. मासाः᳚ संॅवथ्स॒रः सं॑ॅवथ्स॒रो मासा॒ मासाः᳚ संॅवथ्स॒रः । \newline
25. सं॒ॅव॒थ्स॒रो द्वाद॑श॒ द्वाद॑श संॅवथ्स॒रः सं॑ॅवथ्स॒रो द्वाद॑श । \newline
26. सं॒ॅव॒थ्स॒र इति॑ सं - व॒थ्स॒रः । \newline
27. द्वाद॑श द्व॒न्द्वानि॑ द्व॒न्द्वानि॒ द्वाद॑श॒ द्वाद॑श द्व॒न्द्वानि॑ । \newline
28. द्व॒न्द्वानि॑ दर्.शपूर्णमा॒सयो᳚र् दर्.शपूर्णमा॒सयो᳚र् द्व॒न्द्वानि॑ द्व॒न्द्वानि॑ दर्.शपूर्णमा॒सयोः᳚ । \newline
29. द्व॒न्द्वानीति॑ द्वं - द्वानि॑ । \newline
30. द॒र्॒.श॒पू॒र्ण॒मा॒सयो॒ स्तानि॒ तानि॑ दर्.शपूर्णमा॒सयो᳚र् दर्.शपूर्णमा॒सयो॒ स्तानि॑ । \newline
31. द॒र्॒.श॒पू॒र्ण॒मा॒सयो॒रिति॑ दर्.श - पू॒र्ण॒मा॒सयोः᳚ । \newline
32. तानि॑ सं॒पाद्या॑नि सं॒पाद्या॑नि॒ तानि॒ तानि॑ सं॒पाद्या॑नि । \newline
33. सं॒पाद्या॒नी तीति॑ सं॒पाद्या॑नि सं॒पाद्या॒नी ति॑ । \newline
34. सं॒पाद्या॒नीति॑ सं - पाद्या॑नि । \newline
35. इत्या॑हु राहु॒ रिती त्या॑हुः । \newline
36. आ॒हु॒र् व॒थ्सं ॅव॒थ्स मा॑हु राहुर् व॒थ्सम् । \newline
37. व॒थ्सम् च॑ च व॒थ्सं ॅव॒थ्सम् च॑ । \newline
38. चो॒पा॒व॒सृ॒ज त्यु॑पावसृ॒जति॑ च चोपावसृ॒जति॑ । \newline
39. उ॒पा॒व॒सृ॒ज त्यु॒खा मु॒खा मु॑पावसृ॒ज त्यु॑पावसृ॒ज त्यु॒खाम् । \newline
40. उ॒पा॒व॒सृ॒जतीत्यु॑प - अ॒व॒सृ॒जति॑ । \newline
41. उ॒खाम् च॑ चो॒खा मु॒खाम् च॑ । \newline
42. चाध्यधि॑ च॒ चाधि॑ । \newline
43. अधि॑ श्रयति श्रय॒ त्यध्यधि॑ श्रयति । \newline
44. श्र॒य॒त्यवाव॑ श्रयति श्रय॒त्यव॑ । \newline
45. अव॑ च॒ चावाव॑ च । \newline
46. च॒ हन्ति॒ हन्ति॑ च च॒ हन्ति॑ । \newline
47. हन्ति॑ दृ॒षदौ॑ दृ॒षदौ॒ हन्ति॒ हन्ति॑ दृ॒षदौ᳚ । \newline
48. दृ॒षदौ॑ च च दृ॒षदौ॑ दृ॒षदौ॑ च । \newline
49. च॒ स॒माह॑न्ति स॒माह॑न्ति च च स॒माह॑न्ति । \newline
50. स॒माह॒ न्त्यध्यधि॑ स॒माह॑न्ति स॒माह॒ न्त्यधि॑ । \newline
51. स॒माह॒न्तीति॑ सं - आह॑न्ति । \newline
52. अधि॑ च॒ चाध्यधि॑ च । \newline
53. च॒ वप॑ते॒ वप॑ते च च॒ वप॑ते । \newline
54. वप॑ते क॒पाला॑नि क॒पाला॑नि॒ वप॑ते॒ वप॑ते क॒पाला॑नि । \newline
55. क॒पाला॑नि च च क॒पाला॑नि क॒पाला॑नि च । \newline
56. चोपोप॑ च॒ चोप॑ । \newline
57. उप॑ दधाति दधा॒ त्युपोप॑ दधाति । \newline
58. द॒धा॒ति॒ पु॒रो॒डाश॑म् पुरो॒डाश॑म् दधाति दधाति पुरो॒डाश᳚म् । \newline
59. पु॒रो॒डाश॑म् च च पुरो॒डाश॑म् पुरो॒डाश॑म् च । \newline
60. चा॒धि॒श्रय॑ त्यधि॒श्रय॑ति च चाधि॒श्रय॑ति । \newline

\textbf{Ghana Paata } \newline

1. ए॒वं ॅवि॒द्वान्. वि॒द्वा ने॒व मे॒वं ॅवि॒द्वान् द॑र्.शपूर्णमा॒सौ द॑र्.शपूर्णमा॒सौ वि॒द्वा ने॒व मे॒वं ॅवि॒द्वान् द॑र्.शपूर्णमा॒सौ । \newline
2. वि॒द्वान् द॑र्.शपूर्णमा॒सौ द॑र्.शपूर्णमा॒सौ वि॒द्वान्. वि॒द्वान् द॑र्.शपूर्णमा॒सौ यज॑ते॒ यज॑ते दर्.शपूर्णमा॒सौ वि॒द्वान्. वि॒द्वान् द॑र्.शपूर्णमा॒सौ यज॑ते । \newline
3. द॒र्॒.श॒पू॒र्ण॒मा॒सौ यज॑ते॒ यज॑ते दर्.शपूर्णमा॒सौ द॑र्.शपूर्णमा॒सौ यज॑ते पर॒माम् प॑र॒मां ॅयज॑ते दर्.शपूर्णमा॒सौ द॑र्.शपूर्णमा॒सौ यज॑ते पर॒माम् । \newline
4. द॒र्॒.श॒पू॒र्ण॒मा॒साविति॑ दर्.श - पू॒र्ण॒मा॒सौ । \newline
5. यज॑ते पर॒माम् प॑र॒मां ॅयज॑ते॒ यज॑ते पर॒मा मे॒वैव प॑र॒मां ॅयज॑ते॒ यज॑ते पर॒मा मे॒व । \newline
6. प॒र॒मा मे॒वैव प॑र॒माम् प॑र॒मा मे॒व काष्ठा॒म् काष्ठा॑ मे॒व प॑र॒माम् प॑र॒मा मे॒व काष्ठा᳚म् । \newline
7. ए॒व काष्ठा॒म् काष्ठा॑ मे॒वैव काष्ठा᳚म् गच्छति गच्छति॒ काष्ठा॑ मे॒वैव काष्ठा᳚म् गच्छति । \newline
8. काष्ठा᳚म् गच्छति गच्छति॒ काष्ठा॒म् काष्ठा᳚म् गच्छति॒ यो यो ग॑च्छति॒ काष्ठा॒म् काष्ठा᳚म् गच्छति॒ यः । \newline
9. ग॒च्छ॒ति॒ यो यो ग॑च्छति गच्छति॒ यो वै वै यो ग॑च्छति गच्छति॒ यो वै । \newline
10. यो वै वै यो यो वै प्रजा॑तेन॒ प्रजा॑तेन॒ वै यो यो वै प्रजा॑तेन । \newline
11. वै प्रजा॑तेन॒ प्रजा॑तेन॒ वै वै प्रजा॑तेन य॒ज्ञेन॑ य॒ज्ञेन॒ प्रजा॑तेन॒ वै वै प्रजा॑तेन य॒ज्ञेन॑ । \newline
12. प्रजा॑तेन य॒ज्ञेन॑ य॒ज्ञेन॒ प्रजा॑तेन॒ प्रजा॑तेन य॒ज्ञेन॒ यज॑ते॒ यज॑ते य॒ज्ञेन॒ प्रजा॑तेन॒ प्रजा॑तेन य॒ज्ञेन॒ यज॑ते । \newline
13. प्रजा॑ते॒नेति॒ प्र - जा॒ते॒न॒ । \newline
14. य॒ज्ञेन॒ यज॑ते॒ यज॑ते य॒ज्ञेन॑ य॒ज्ञेन॒ यज॑ते॒ प्र प्र यज॑ते य॒ज्ञेन॑ य॒ज्ञेन॒ यज॑ते॒ प्र । \newline
15. यज॑ते॒ प्र प्र यज॑ते॒ यज॑ते॒ प्र प्र॒जया᳚ प्र॒जया॒ प्र यज॑ते॒ यज॑ते॒ प्र प्र॒जया᳚ । \newline
16. प्र प्र॒जया᳚ प्र॒जया॒ प्र प्र प्र॒जया॑ प॒शुभिः॑ प॒शुभिः॑ प्र॒जया॒ प्र प्र प्र॒जया॑ प॒शुभिः॑ । \newline
17. प्र॒जया॑ प॒शुभिः॑ प॒शुभिः॑ प्र॒जया᳚ प्र॒जया॑ प॒शुभि॑र् मिथु॒नैर् मि॑थु॒नैः प॒शुभिः॑ प्र॒जया᳚ प्र॒जया॑ प॒शुभि॑र् मिथु॒नैः । \newline
18. प्र॒जयेति॑ प्र - जया᳚ । \newline
19. प॒शुभि॑र् मिथु॒नैर् मि॑थु॒नैः प॒शुभिः॑ प॒शुभि॑र् मिथु॒नैर् जा॑यते जायते मिथु॒नैः प॒शुभिः॑ प॒शुभि॑र् मिथु॒नैर् जा॑यते । \newline
20. प॒शुभि॒रिति॑ प॒शु - भिः॒ । \newline
21. मि॒थु॒नैर् जा॑यते जायते मिथु॒नैर् मि॑थु॒नैर् जा॑यते॒ द्वाद॑श॒ द्वाद॑श जायते मिथु॒नैर् मि॑थु॒नैर् जा॑यते॒ द्वाद॑श । \newline
22. जा॒य॒ते॒ द्वाद॑श॒ द्वाद॑श जायते जायते॒ द्वाद॑श॒ मासा॒ मासा॒ द्वाद॑श जायते जायते॒ द्वाद॑श॒ मासाः᳚ । \newline
23. द्वाद॑श॒ मासा॒ मासा॒ द्वाद॑श॒ द्वाद॑श॒ मासाः᳚ संॅवथ्स॒रः सं॑ॅवथ्स॒रो मासा॒ द्वाद॑श॒ द्वाद॑श॒ मासाः᳚ संॅवथ्स॒रः । \newline
24. मासाः᳚ संॅवथ्स॒रः सं॑ॅवथ्स॒रो मासा॒ मासाः᳚ संॅवथ्स॒रो द्वाद॑श॒ द्वाद॑श संॅवथ्स॒रो मासा॒ मासाः᳚ संॅवथ्स॒रो द्वाद॑श । \newline
25. सं॒ॅव॒थ्स॒रो द्वाद॑श॒ द्वाद॑श संॅवथ्स॒रः सं॑ॅवथ्स॒रो द्वाद॑श द्व॒न्द्वानि॑ द्व॒न्द्वानि॒ द्वाद॑श संॅवथ्स॒रः सं॑ॅवथ्स॒रो द्वाद॑श द्व॒न्द्वानि॑ । \newline
26. सं॒ॅव॒थ्स॒र इति॑ सं - व॒थ्स॒रः । \newline
27. द्वाद॑श द्व॒न्द्वानि॑ द्व॒न्द्वानि॒ द्वाद॑श॒ द्वाद॑श द्व॒न्द्वानि॑ दर्.शपूर्णमा॒सयो᳚र् दर्.शपूर्णमा॒सयो᳚र् द्व॒न्द्वानि॒ द्वाद॑श॒ द्वाद॑श द्व॒न्द्वानि॑ दर्.शपूर्णमा॒सयोः᳚ । \newline
28. द्व॒न्द्वानि॑ दर्.शपूर्णमा॒सयो᳚र् दर्.शपूर्णमा॒सयो᳚र् द्व॒न्द्वानि॑ द्व॒न्द्वानि॑ दर्.शपूर्णमा॒सयो॒ स्तानि॒ तानि॑ दर्.शपूर्णमा॒सयो᳚र् द्व॒न्द्वानि॑ द्व॒न्द्वानि॑ दर्.शपूर्णमा॒सयो॒ स्तानि॑ । \newline
29. द्व॒न्द्वानीति॑ द्वं - द्वानि॑ । \newline
30. द॒र्॒.श॒पू॒र्ण॒मा॒सयो॒ स्तानि॒ तानि॑ दर्.शपूर्णमा॒सयो᳚र् दर्.शपूर्णमा॒सयो॒ स्तानि॑ सं॒पाद्या॑नि सं॒पाद्या॑नि॒ तानि॑ दर्.शपूर्णमा॒सयो᳚र् दर्.शपूर्णमा॒सयो॒ स्तानि॑ सं॒पाद्या॑नि । \newline
31. द॒र्॒.श॒पू॒र्ण॒मा॒सयो॒रिति॑ दर्.श - पू॒र्ण॒मा॒सयोः᳚ । \newline
32. तानि॑ सं॒पाद्या॑नि सं॒पाद्या॑नि॒ तानि॒ तानि॑ सं॒पाद्या॒नीतीति॑ सं॒पाद्या॑नि॒ तानि॒ तानि॑ सं॒पाद्या॒नीति॑ । \newline
33. सं॒पाद्या॒नीतीति॑ सं॒पाद्या॑नि सं॒पाद्या॒नी त्या॑हु राहु॒रिति॑ सं॒पाद्या॑नि सं॒पाद्या॒नीत्या॑हुः । \newline
34. सं॒पाद्या॒नीति॑ सं - पाद्या॑नि । \newline
35. इत्या॑हु राहु॒रितीत्या॑हुर् व॒थ्सं ॅव॒थ्स मा॑हु॒रितीत्या॑हुर् व॒थ्सम् । \newline
36. आ॒हु॒र् व॒थ्सं ॅव॒थ्स मा॑हु राहुर् व॒थ्सम् च॑ च व॒थ्स मा॑हुराहुर् व॒थ्सम् च॑ । \newline
37. व॒थ्सम् च॑ च व॒थ्सं ॅव॒थ्सम् चो॑पावसृ॒ज त्यु॑पावसृ॒जति॑ च व॒थ्सं ॅव॒थ्सम् चो॑पावसृ॒जति॑ । \newline
38. चो॒पा॒व॒सृ॒ज त्यु॑पावसृ॒जति॑ च चोपावसृ॒जत्यु॒खा मु॒खा मु॑पावसृ॒जति॑ च चोपावसृ॒जत्यु॒खाम् । \newline
39. उ॒पा॒व॒सृ॒जत्यु॒खा मु॒खा मु॑पावसृ॒ज त्यु॑पावसृ॒जत्यु॒खाम् च॑ चो॒खा मु॑पावसृ॒ज त्यु॑पावसृ॒जत्यु॒खाम् च॑ । \newline
40. उ॒पा॒व॒सृ॒जतीत्यु॑प - अ॒व॒सृ॒जति॑ । \newline
41. उ॒खाम् च॑ चो॒खा मु॒खाम् चाध्यधि॑ चो॒खा मु॒खाम् चाधि॑ । \newline
42. चाध्यधि॑ च॒ चाधि॑ श्रयति श्रय॒त्यधि॑ च॒ चाधि॑ श्रयति । \newline
43. अधि॑ श्रयति श्रय॒त्यध्यधि॑ श्रय॒त्यवाव॑ श्रय॒त्यध्यधि॑ श्रय॒त्यव॑ । \newline
44. श्र॒य॒त्यवाव॑ श्रयति श्रय॒त्यव॑ च॒ चाव॑ श्रयति श्रय॒त्यव॑ च । \newline
45. अव॑ च॒ चावाव॑ च॒ हन्ति॒ हन्ति॒ चावाव॑ च॒ हन्ति॑ । \newline
46. च॒ हन्ति॒ हन्ति॑ च च॒ हन्ति॑ दृ॒षदौ॑ दृ॒षदौ॒ हन्ति॑ च च॒ हन्ति॑ दृ॒षदौ᳚ । \newline
47. हन्ति॑ दृ॒षदौ॑ दृ॒षदौ॒ हन्ति॒ हन्ति॑ दृ॒षदौ॑ च च दृ॒षदौ॒ हन्ति॒ हन्ति॑ दृ॒षदौ॑ च । \newline
48. दृ॒षदौ॑ च च दृ॒षदौ॑ दृ॒षदौ॑ च स॒माह॑न्ति स॒माह॑न्ति च दृ॒षदौ॑ दृ॒षदौ॑ च स॒माह॑न्ति । \newline
49. च॒ स॒माह॑न्ति स॒माह॑न्ति च च स॒माह॒न्त्यध्यधि॑ स॒माह॑न्ति च च स॒माह॒न्त्यधि॑ । \newline
50. स॒माह॒न्त्यध्यधि॑ स॒माह॑न्ति स॒माह॒न्त्यधि॑ च॒ चाधि॑ स॒माह॑न्ति स॒माह॒न्त्यधि॑ च । \newline
51. स॒माह॒न्तीति॑ सं - आह॑न्ति । \newline
52. अधि॑ च॒ चाध्यधि॑ च॒ वप॑ते॒ वप॑ते॒ चाध्यधि॑ च॒ वप॑ते । \newline
53. च॒ वप॑ते॒ वप॑ते च च॒ वप॑ते क॒पाला॑नि क॒पाला॑नि॒ वप॑ते च च॒ वप॑ते क॒पाला॑नि । \newline
54. वप॑ते क॒पाला॑नि क॒पाला॑नि॒ वप॑ते॒ वप॑ते क॒पाला॑नि च च क॒पाला॑नि॒ वप॑ते॒ वप॑ते क॒पाला॑नि च । \newline
55. क॒पाला॑नि च च क॒पाला॑नि क॒पाला॑नि॒ चोपोप॑ च क॒पाला॑नि क॒पाला॑नि॒ चोप॑ । \newline
56. चोपोप॑ च॒ चोप॑ दधाति दधा॒त्युप॑ च॒ चोप॑ दधाति । \newline
57. उप॑ दधाति दधा॒त्युपोप॑ दधाति पुरो॒डाश॑म् पुरो॒डाश॑म् दधा॒त्युपोप॑ दधाति पुरो॒डाश᳚म् । \newline
58. द॒धा॒ति॒ पु॒रो॒डाश॑म् पुरो॒डाश॑म् दधाति दधाति पुरो॒डाश॑म् च च पुरो॒डाश॑म् दधाति दधाति पुरो॒डाश॑म् च । \newline
59. पु॒रो॒डाश॑म् च च पुरो॒डाश॑म् पुरो॒डाश॑म् चाधि॒श्रय॑ त्यधि॒श्रय॑ति च पुरो॒डाश॑म् पुरो॒डाश॑म् चाधि॒श्रय॑ति । \newline
60. चा॒धि॒श्रय॑ त्यधि॒श्रय॑ति च चाधि॒श्रय॒त्याज्य॒ माज्य॑ मधि॒श्रय॑ति च चाधि॒श्रय॒त्याज्य᳚म् । \newline
\pagebreak
\markright{ TS 1.6.9.4  \hfill https://www.vedavms.in \hfill}

\section{ TS 1.6.9.4 }

\textbf{TS 1.6.9.4 } \newline
\textbf{Samhita Paata} \newline

चा ऽधि॒श्रय॒त्याज्यं॑ च स्तंबय॒जुश्च॒ हर॑त्य॒भि च॑ गृह्णाति॒ वेदिं॑ च परि गृ॒ह्णाति॒ पत्नीं᳚ च॒ संन॑ह्यति॒ प्रोक्ष॑णीश्चा ऽऽसा॒दय॒त्याज्यं॑ चै॒तानि॒ वै द्वाद॑श द्व॒न्द्वानि॑ दर्.शपूर्णमा॒सयो॒स्तानि॒ य ए॒वꣳ सं॒पाद्य॒ यज॑ते॒ प्रजा॑तेनै॒व य॒ज्ञेन॑ यजते॒ प्र प्र॒जया॑ प॒शुभि॑र् मिथु॒नैर् जा॑यते ॥ \newline

\textbf{Pada Paata} \newline

अ॒धि॒श्रय॒तीत्य॑धि - श्रय॑ति । आज्य᳚म् । च॒ । स्त॒बं॒य॒जुरिति॑ स्तंब - य॒जुः । च॒ । हर॑ति । अ॒भीति॑ । च॒ । गृ॒ह्णा॒ति॒ । वेदि᳚म् । च॒ । प॒रि॒गृ॒ह्णातीति॑ परि - गृ॒ह्णाति॑ । पत्नी᳚म् । च॒ । समिति॑ । न॒ह्य॒ति॒ । प्रोक्ष॑णी॒रिति॑ प्र - उक्ष॑णीः । च॒ । आ॒सा॒दय॒तीत्या᳚ - सा॒दय॑ति । आज्य᳚म् । च॒ । ए॒तानि॑ । वै । द्वाद॑श । द्व॒न्द्वानीति॑ द्वं - द्वानि॑ । द॒र्॒.श॒पू॒र्ण॒मा॒सया॒रिति॑ दर्.श - पू॒र्ण॒मा॒सयोः᳚ । तानि॑ । यः । ए॒वम् । स॒पांद्येति॑ सं - पाद्य॑ । यज॑ते । प्रजा॑ते॒नेति॒ प्र - जा॒ते॒न॒ । ए॒व । य॒ज्ञेन॑ । य॒ज॒ते॒ । प्रेति॑ । प्र॒जयेति॑ प्र - जया᳚ । प॒शुभि॒रिति॑ प॒शु - भिः॒ । मि॒थु॒नैः । जा॒य॒ते॒ ॥  \newline


\textbf{Krama Paata} \newline

अ॒धि॒श्रय॒त्याज्य᳚म् । अ॒धि॒श्रय॒तीत्य॑धि - श्रय॑ति । आज्य॑म् च । च॒ स्त॒म्ब॒य॒जुः । स्त॒म्ब॒य॒जुश्च॑ । स्त॒म्ब॒य॒जुरिति॑ स्तम्ब - य॒जुः । च॒ हर॑ति । हर॑त्य॒भि । अ॒भि च॑ । च॒ गृ॒ह्णा॒ति॒ । गृ॒ह्णा॒ति॒ वेदि᳚म् । वेदि॑म् च । च॒ प॒रि॒गृ॒ह्णाति॑ । प॒रि॒गृ॒ह्णाति॒ पत्नी᳚म् । प॒रि॒गृ॒ह्णातीति॑ परि - गृ॒ह्णाति॑ । पत्नी᳚म् च । च॒ सम् । सम् न॑ह्यति । न॒ह्य॒ति॒ प्रोक्ष॑णीः । प्रोक्ष॑णीश्च । प्रोक्ष॑णी॒रिति॑ प्र - उक्ष॑णीः । चा॒सा॒दय॑ति । आ॒सा॒दय॒त्याज्य᳚म् । आ॒सा॒दय॒तीत्या᳚ - सा॒दय॑ति । आज्य॑म् च । चै॒तानि॑ । ए॒तानि॒ वै । वै द्वाद॑श । द्वाद॑श द्व॒न्द्वानि॑ । द्व॒न्द्वानि॑ दर्.शपूर्णमा॒सयोः᳚ । द्व॒न्द्वानीति॑ द्वम् - द्वानि॑ । द॒र्.॒श॒पू॒र्ण॒मा॒सयो॒स्तानि॑ । द॒र्॒.श॒पू॒र्ण॒मा॒सयो॒रिति॑ दर्.श - पू॒र्ण॒मा॒सयोः᳚ । तानि॒ यः । य ए॒वम् । ए॒वꣳ स॒म्पाद्य॑ । स॒म्पाद्य॒ यज॑ते । स॒म्पाद्येति॑ सम् - पाद्य॑ । यज॑ते॒ प्रजा॑तेन । प्रजा॑तेनै॒व । प्रजा॑ते॒नेति॒ प्र - जा॒ते॒न॒ । ए॒व य॒ज्ञेन॑ । य॒ज्ञेन॑ यजते । य॒ज॒ते॒ प्र । प्र प्र॒जया᳚ । प्र॒जया॑ प॒शुभिः॑ । प्र॒जेयेति॑ प्र - जया᳚ । प॒शुभि॑र् मिथु॒नैः । प॒शुभि॒रिति॑ प॒शु - भिः॒ । मि॒थु॒नैर्,जा॑यते । जा॒य॒त॒ इति॑ जायते । \newline

\textbf{Jatai Paata} \newline

1. अ॒धि॒श्रय॒ त्याज्य॒ माज्य॑ मधि॒श्रय॑ त्यधि॒श्रय॒ त्याज्य᳚म् । \newline
2. अ॒धि॒श्रय॒तीत्य॑धि - श्रय॑ति । \newline
3. आज्य॑म् च॒ चाज्य॒ माज्य॑म् च । \newline
4. च॒ स्तं॒ब॒य॒जुः स्तं॑बय॒जुश्च॑ च स्तंबय॒जुः । \newline
5. स्तं॒ब॒य॒जुश्च॑ च स्तंबय॒जुः स्तं॑बय॒जुश्च॑ । \newline
6. स्तं॒ब॒य॒जुरिति॑ स्तंब - य॒जुः । \newline
7. च॒ हर॑ति॒ हर॑ति च च॒ हर॑ति । \newline
8. हर॑त्य॒ भ्य॑भि हर॑ति॒ हर॑ त्य॒भि । \newline
9. अ॒भि च॑ चा॒भ्य॑भि च॑ । \newline
10. च॒ गृ॒ह्णा॒ति॒ गृ॒ह्णा॒ति॒ च॒ च॒ गृ॒ह्णा॒ति॒ । \newline
11. गृ॒ह्णा॒ति॒ वेदिं॒ ॅवेदि॑म् गृह्णाति गृह्णाति॒ वेदि᳚म् । \newline
12. वेदि॑म् च च॒ वेदिं॒ ॅवेदि॑म् च । \newline
13. च॒ प॒रि॒गृ॒ह्णाति॑ परिगृ॒ह्णाति॑ च च परिगृ॒ह्णाति॑ । \newline
14. प॒रि॒गृ॒ह्णाति॒ पत्नी॒म् पत्नी᳚म् परिगृ॒ह्णाति॑ परिगृ॒ह्णाति॒ पत्नी᳚म् । \newline
15. प॒रि॒गृ॒ह्णातीति॑ परि - गृ॒ह्णाति॑ । \newline
16. पत्नी᳚म् च च॒ पत्नी॒म् पत्नी᳚म् च । \newline
17. च॒ सꣳ सम् च॑ च॒ सम् । \newline
18. सम् न॑ह्यति नह्यति॒ सꣳ सम् न॑ह्यति । \newline
19. न॒ह्य॒ति॒ प्रोक्ष॑णीः॒ प्रोक्ष॑णीर् नह्यति नह्यति॒ प्रोक्ष॑णीः । \newline
20. प्रोक्ष॑णीश्च च॒ प्रोक्ष॑णीः॒ प्रोक्ष॑णीश्च । \newline
21. प्रोक्ष॑णी॒रिति॑ प्र - उक्ष॑णीः । \newline
22. चा॒सा॒दय॑ त्यासा॒दय॑ति च चासा॒दय॑ति । \newline
23. आ॒सा॒दय॒ त्याज्य॒ माज्य॑ मासा॒दय॑ त्यासा॒दय॒ त्याज्य᳚म् । \newline
24. आ॒सा॒दय॒तीत्या᳚ - सा॒दय॑ति । \newline
25. आज्य॑म् च॒ चाज्य॒ माज्य॑म् च । \newline
26. चै॒ता न्ये॒तानि॑ च चै॒तानि॑ । \newline
27. ए॒तानि॒ वै वा ए॒ता न्ये॒तानि॒ वै । \newline
28. वै द्वाद॑श॒ द्वाद॑श॒ वै वै द्वाद॑श । \newline
29. द्वाद॑श द्व॒न्द्वानि॑ द्व॒न्द्वानि॒ द्वाद॑श॒ द्वाद॑श द्व॒न्द्वानि॑ । \newline
30. द्व॒न्द्वानि॑ दर्.शपूर्णमा॒सयो᳚र् दर्.शपूर्णमा॒सयो᳚र् द्व॒न्द्वानि॑ द्व॒न्द्वानि॑ दर्.शपूर्णमा॒सयोः᳚ । \newline
31. द्व॒न्द्वानीति॑ द्वं - द्वानि॑ । \newline
32. द॒र्॒.श॒पू॒र्ण॒मा॒सयो॒ स्तानि॒ तानि॑ दर्.शपूर्णमा॒सयो᳚र् दर्.शपूर्णमा॒सयो॒ स्तानि॑ । \newline
33. द॒र्॒.श॒पू॒र्ण॒मा॒सयो॒रिति॑ दर्.श - पू॒र्ण॒मा॒सयोः᳚ । \newline
34. तानि॒ यो य स्तानि॒ तानि॒ यः । \newline
35. य ए॒व मे॒वं ॅयो य ए॒वम् । \newline
36. ए॒वꣳ सं॒पाद्य॑ सं॒पाद्यै॒व मे॒वꣳ सं॒पाद्य॑ । \newline
37. सं॒पाद्य॒ यज॑ते॒ यज॑ते सं॒पाद्य॑ सं॒पाद्य॒ यज॑ते । \newline
38. सं॒पाद्येति॑ सं - पाद्य॑ । \newline
39. यज॑ते॒ प्रजा॑तेन॒ प्रजा॑तेन॒ यज॑ते॒ यज॑ते॒ प्रजा॑तेन । \newline
40. प्रजा॑ते नै॒वैव प्रजा॑तेन॒ प्रजा॑ते नै॒व । \newline
41. प्रजा॑ते॒नेति॒ प्र - जा॒ते॒न॒ । \newline
42. ए॒व य॒ज्ञेन॑ य॒ज्ञे नै॒वैव य॒ज्ञेन॑ । \newline
43. य॒ज्ञेन॑ यजते यजते य॒ज्ञेन॑ य॒ज्ञेन॑ यजते । \newline
44. य॒ज॒ते॒ प्र प्र य॑जते यजते॒ प्र । \newline
45. प्र प्र॒जया᳚ प्र॒जया॒ प्र प्र प्र॒जया᳚ । \newline
46. प्र॒जया॑ प॒शुभिः॑ प॒शुभिः॑ प्र॒जया᳚ प्र॒जया॑ प॒शुभिः॑ । \newline
47. प्र॒जयेति॑ प्र - जया᳚ । \newline
48. प॒शुभि॑र् मिथु॒नैर् मि॑थु॒नैः प॒शुभिः॑ प॒शुभि॑र् मिथु॒नैः । \newline
49. प॒शुभि॒रिति॑ प॒शु - भिः॒ । \newline
50. मि॒थु॒नैर् जा॑यते जायते मिथु॒नैर् मि॑थु॒नैर् जा॑यते । \newline
51. जा॒य॒त॒ इति॑ जायते । \newline

\textbf{Ghana Paata } \newline

1. अ॒धि॒श्रय॒त्याज्य॒ माज्य॑ मधि॒श्रय॑ त्यधि॒श्रय॒त्याज्य॑म् च॒ चाज्य॑ मधि॒श्रय॑ त्यधि॒श्रय॒त्याज्य॑म् च । \newline
2. अ॒धि॒श्रय॒तीत्य॑धि - श्रय॑ति । \newline
3. आज्य॑म् च॒ चाज्य॒ माज्य॑म् च स्तंबय॒जुः स्तं॑बय॒जु श्चाज्य॒ माज्य॑म् च स्तंबय॒जुः । \newline
4. च॒ स्तं॒ब॒य॒जुः स्तं॑बय॒जुश्च॑ च स्तंबय॒जुश्च॑ च स्तंबय॒जुश्च॑ च स्तंबय॒जुश्च॑ । \newline
5. स्तं॒ब॒य॒जुश्च॑ च स्तंबय॒जुः स्तं॑बय॒जुश्च॒ हर॑ति॒ हर॑ति च स्तंबय॒जुः स्तं॑बय॒जुश्च॒ हर॑ति । \newline
6. स्तं॒ब॒य॒जुरिति॑ स्तंब - य॒जुः । \newline
7. च॒ हर॑ति॒ हर॑ति च च॒ हर॑त्य॒भ्य॑भि हर॑ति च च॒ हर॑त्य॒भि । \newline
8. हर॑त्य॒भ्य॑भि हर॑ति॒ हर॑त्य॒भि च॑ चा॒भि हर॑ति॒ हर॑त्य॒भि च॑ । \newline
9. अ॒भि च॑ चा॒भ्य॑भि च॑ गृह्णाति गृह्णाति चा॒भ्य॑भि च॑ गृह्णाति । \newline
10. च॒ गृ॒ह्णा॒ति॒ गृ॒ह्णा॒ति॒ च॒ च॒ गृ॒ह्णा॒ति॒ वेदिं॒ ॅवेदि॑म् गृह्णाति च च गृह्णाति॒ वेदि᳚म् । \newline
11. गृ॒ह्णा॒ति॒ वेदिं॒ ॅवेदि॑म् गृह्णाति गृह्णाति॒ वेदि॑म् च च॒ वेदि॑म् गृह्णाति गृह्णाति॒ वेदि॑म् च । \newline
12. वेदि॑म् च च॒ वेदिं॒ ॅवेदि॑म् च परिगृ॒ह्णाति॑ परिगृ॒ह्णाति॑ च॒ वेदिं॒ ॅवेदि॑म् च परिगृ॒ह्णाति॑ । \newline
13. च॒ प॒रि॒गृ॒ह्णाति॑ परिगृ॒ह्णाति॑ च च परिगृ॒ह्णाति॒ पत्नी॒म् पत्नी᳚म् परिगृ॒ह्णाति॑ च च परिगृ॒ह्णाति॒ पत्नी᳚म् । \newline
14. प॒रि॒गृ॒ह्णाति॒ पत्नी॒म् पत्नी᳚म् परिगृ॒ह्णाति॑ परिगृ॒ह्णाति॒ पत्नी᳚म् च च॒ पत्नी᳚म् परिगृ॒ह्णाति॑ परिगृ॒ह्णाति॒ पत्नी᳚म् च । \newline
15. प॒रि॒गृ॒ह्णातीति॑ परि - गृ॒ह्णाति॑ । \newline
16. पत्नी᳚म् च च॒ पत्नी॒म् पत्नी᳚म् च॒ सꣳ सम् च॒ पत्नी॒म् पत्नी᳚म् च॒ सम् । \newline
17. च॒ सꣳ सम् च॑ च॒ सम् न॑ह्यति नह्यति॒ सम् च॑ च॒ सम् न॑ह्यति । \newline
18. सम् न॑ह्यति नह्यति॒ सꣳ सम् न॑ह्यति॒ प्रोक्ष॑णीः॒ प्रोक्ष॑णीर् नह्यति॒ सꣳ सम् न॑ह्यति॒ प्रोक्ष॑णीः । \newline
19. न॒ह्य॒ति॒ प्रोक्ष॑णीः॒ प्रोक्ष॑णीर् नह्यति नह्यति॒ प्रोक्ष॑णीश्च च॒ प्रोक्ष॑णीर् नह्यति नह्यति॒ प्रोक्ष॑णीश्च । \newline
20. प्रोक्ष॑णीश्च च॒ प्रोक्ष॑णीः॒ प्रोक्ष॑णी श्चासा॒दय॑ त्यासा॒दय॑ति च॒ प्रोक्ष॑णीः॒ प्रोक्ष॑णी श्चासा॒दय॑ति । \newline
21. प्रोक्ष॑णी॒रिति॑ प्र - उक्ष॑णीः । \newline
22. चा॒सा॒दय॑ त्यासा॒दय॑ति च चासा॒दय॒त्याज्य॒ माज्य॑ मासा॒दय॑ति च चासा॒दय॒त्याज्य᳚म् । \newline
23. आ॒सा॒दय॒त्याज्य॒ माज्य॑ मासा॒दय॑ त्यासा॒दय॒त्याज्य॑म् च॒ चाज्य॑ मासा॒दय॑ त्यासा॒दय॒त्याज्य॑म् च । \newline
24. आ॒सा॒दय॒तीत्या᳚ - सा॒दय॑ति । \newline
25. आज्य॑म् च॒ चाज्य॒ माज्य॑म् चै॒तान्ये॒तानि॒ चाज्य॒ माज्य॑म् चै॒तानि॑ । \newline
26. चै॒तान्ये॒तानि॑ च चै॒तानि॒ वै वा ए॒तानि॑ च चै॒तानि॒ वै । \newline
27. ए॒तानि॒ वै वा ए॒तान्ये॒तानि॒ वै द्वाद॑श॒ द्वाद॑श॒ वा ए॒तान्ये॒तानि॒ वै द्वाद॑श । \newline
28. वै द्वाद॑श॒ द्वाद॑श॒ वै वै द्वाद॑श द्व॒न्द्वानि॑ द्व॒न्द्वानि॒ द्वाद॑श॒ वै वै द्वाद॑श द्व॒न्द्वानि॑ । \newline
29. द्वाद॑श द्व॒न्द्वानि॑ द्व॒न्द्वानि॒ द्वाद॑श॒ द्वाद॑श द्व॒न्द्वानि॑ दर्.शपूर्णमा॒सयो᳚र् दर्.शपूर्णमा॒सयो᳚र् द्व॒न्द्वानि॒ द्वाद॑श॒ द्वाद॑श द्व॒न्द्वानि॑ दर्.शपूर्णमा॒सयोः᳚ । \newline
30. द्व॒न्द्वानि॑ दर्.शपूर्णमा॒सयो᳚र् दर्.शपूर्णमा॒सयो᳚र् द्व॒न्द्वानि॑ द्व॒न्द्वानि॑ दर्.शपूर्णमा॒सयो॒ स्तानि॒ तानि॑ दर्.शपूर्णमा॒सयो᳚र् द्व॒न्द्वानि॑ द्व॒न्द्वानि॑ दर्.शपूर्णमा॒सयो॒ स्तानि॑ । \newline
31. द्व॒न्द्वानीति॑ द्वं - द्वानि॑ । \newline
32. द॒र्॒.श॒पू॒र्ण॒मा॒सयो॒ स्तानि॒ तानि॑ दर्.शपूर्णमा॒सयो᳚र् दर्.शपूर्णमा॒सयो॒ स्तानि॒ यो यस्तानि॑ दर्.शपूर्णमा॒सयो᳚र् दर्.शपूर्णमा॒सयो॒ स्तानि॒ यः । \newline
33. द॒र्॒.श॒पू॒र्ण॒मा॒सयो॒रिति॑ दर्.श - पू॒र्ण॒मा॒सयोः᳚ । \newline
34. तानि॒ यो यस्तानि॒ तानि॒ य ए॒व मे॒वं ॅयस्तानि॒ तानि॒ य ए॒वम् । \newline
35. य ए॒व मे॒वं ॅयो य ए॒वꣳ सं॒पाद्य॑ सं॒पाद्यै॒वं ॅयो य ए॒वꣳ सं॒पाद्य॑ । \newline
36. ए॒वꣳ सं॒पाद्य॑ सं॒पाद्यै॒व मे॒वꣳ सं॒पाद्य॒ यज॑ते॒ यज॑ते सं॒पाद्यै॒व मे॒वꣳ सं॒पाद्य॒ यज॑ते । \newline
37. सं॒पाद्य॒ यज॑ते॒ यज॑ते सं॒पाद्य॑ सं॒पाद्य॒ यज॑ते॒ प्रजा॑तेन॒ प्रजा॑तेन॒ यज॑ते सं॒पाद्य॑ सं॒पाद्य॒ यज॑ते॒ प्रजा॑तेन । \newline
38. सं॒पाद्येति॑ सं - पाद्य॑ । \newline
39. यज॑ते॒ प्रजा॑तेन॒ प्रजा॑तेन॒ यज॑ते॒ यज॑ते॒ प्रजा॑तेनै॒वैव प्रजा॑तेन॒ यज॑ते॒ यज॑ते॒ प्रजा॑तेनै॒व । \newline
40. प्रजा॑तेनै॒वैव प्रजा॑तेन॒ प्रजा॑तेनै॒व य॒ज्ञेन॑ य॒ज्ञेनै॒व प्रजा॑तेन॒ प्रजा॑तेनै॒व य॒ज्ञेन॑ । \newline
41. प्रजा॑ते॒नेति॒ प्र - जा॒ते॒न॒ । \newline
42. ए॒व य॒ज्ञेन॑ य॒ज्ञेनै॒वैव य॒ज्ञेन॑ यजते यजते य॒ज्ञेनै॒वैव य॒ज्ञेन॑ यजते । \newline
43. य॒ज्ञेन॑ यजते यजते य॒ज्ञेन॑ य॒ज्ञेन॑ यजते॒ प्र प्र य॑जते य॒ज्ञेन॑ य॒ज्ञेन॑ यजते॒ प्र । \newline
44. य॒ज॒ते॒ प्र प्र य॑जते यजते॒ प्र प्र॒जया᳚ प्र॒जया॒ प्र य॑जते यजते॒ प्र प्र॒जया᳚ । \newline
45. प्र प्र॒जया᳚ प्र॒जया॒ प्र प्र प्र॒जया॑ प॒शुभिः॑ प॒शुभिः॑ प्र॒जया॒ प्र प्र प्र॒जया॑ प॒शुभिः॑ । \newline
46. प्र॒जया॑ प॒शुभिः॑ प॒शुभिः॑ प्र॒जया᳚ प्र॒जया॑ प॒शुभि॑र् मिथु॒नैर् मि॑थु॒नैः प॒शुभिः॑ प्र॒जया᳚ प्र॒जया॑ प॒शुभि॑र् मिथु॒नैः । \newline
47. प्र॒जयेति॑ प्र - जया᳚ । \newline
48. प॒शुभि॑र् मिथु॒नैर् मि॑थु॒नैः प॒शुभिः॑ प॒शुभि॑र् मिथु॒नैर् जा॑यते जायते मिथु॒नैः प॒शुभिः॑ प॒शुभि॑र् मिथु॒नैर् जा॑यते । \newline
49. प॒शुभि॒रिति॑ प॒शु - भिः॒ । \newline
50. मि॒थु॒नैर् जा॑यते जायते मिथु॒नैर् मि॑थु॒नैर् जा॑यते । \newline
51. जा॒य॒त॒ इति॑ जायते । \newline
\pagebreak
\markright{ TS 1.6.10.1  \hfill https://www.vedavms.in \hfill}

\section{ TS 1.6.10.1 }

\textbf{TS 1.6.10.1 } \newline
\textbf{Samhita Paata} \newline

ध्रु॒वो॑ऽसि ध्रु॒वो॑ऽहꣳ स॑जा॒तेषु॑ भूयास॒मित्या॑ह ध्रु॒वाने॒वैना᳚न् कुरुत उ॒ग्रो᳚ऽस्यु॒ग्रो॑ऽहꣳ स॑जा॒तेषु॑ भूयास॒-मित्या॒हाप्र॑तिवादिन ए॒वैना᳚न् कुरुते-ऽभि॒भूर॑स्यभि॒भूर॒हꣳ स॑जा॒तेषु॑ भूयास॒मित्या॑ह॒ य ए॒वैनं॑ प्रत्यु॒त्पिपी॑ते॒ तमुपा᳚स्यते यु॒नज्मि॑ त्वा॒ ब्रह्म॑णा॒ दैव्ये॒नेत्या॑है॒ष वाअ॒ग्नेर्योग॒स्तेनै॒ - [ ] \newline

\textbf{Pada Paata} \newline

ध्रु॒वः । अ॒सि॒ । ध्रु॒वः । अ॒हम् । स॒जा॒तेष्विति॑ स - जा॒तेषु॑ । भू॒या॒स॒म् । इति॑ । आ॒ह॒ । ध्रु॒वान् । ए॒व । ए॒ना॒न् । कु॒रु॒ते॒ । उ॒ग्रः । अ॒सि॒ । उ॒ग्रः । अ॒हम् । स॒जा॒तेष्विति॑ स-जा॒तेषु॑ । भू॒या॒स॒म् । इति॑ । आ॒ह॒ । अप्र॑तिवादिन॒ इत्यप्र॑ति - वा॒दि॒नः॒ । ए॒व । ए॒ना॒न् । कु॒रु॒ते॒ । अ॒भि॒भूरित्य॑भि - भूः । अ॒सि॒ । अ॒भि॒भूरित्य॑भि - भूः । अ॒हम् । स॒जा॒तेष्विति॑ स - जा॒तेषु॑ । भू॒या॒स॒म् । इति॑ । आ॒ह॒ । यः । ए॒व । ए॒न॒म् । प्र॒त्यु॒त्पिपी॑त॒ इति॑ प्रति-उ॒त्पिपी॑ते । तम् । उपेति॑ । अ॒स्य॒ते॒ । यु॒नज्मि॑ । त्वा॒ । ब्रह्म॑णा । दैव्ये॑न । इति॑ । आ॒ह॒ । ए॒षः । वै । अ॒ग्नेः । योगः॑ । तेन॑ ।  \newline


\textbf{Krama Paata} \newline

ध्रु॒वो॑ऽसि । अ॒सि॒ ध्रु॒वः । ध्रु॒वो॑ऽहम् । अ॒हꣳ स॑जा॒तेषु॑ । स॒जा॒तेषु॑ भूयासम् । स॒जा॒तेष्विति॑ स - जा॒तेषु॑ । भू॒या॒स॒मिति॑ । इत्या॑ह । आ॒ह॒ ध्रु॒वान् । ध्रु॒वाने॒व । ए॒वैनान्॑ । ए॒ना॒न् कु॒रु॒ते॒ । कु॒रु॒त॒ उ॒ग्रः । उ॒ग्रो॑ऽसि । अ॒स्यु॒ग्रः । उ॒ग्रो॑ऽहम् । अ॒हꣳ स॑जा॒तेषु॑ । स॒जा॒तेषु॑ भूयासम् । स॒जा॒तेष्विति॑ स - जा॒तेषु॑ । भू॒या॒स॒मिति॑ । इत्या॑ह । आ॒हाप्र॑तिवादिनः । अप्र॑तिवादिन ए॒व । अप्र॑तिवादिन॒ इत्यप्र॑ति - वा॒दि॒नः॒ । ए॒वैनान्॑ । ए॒ना॒न् कु॒रु॒ते॒ । कु॒रु॒ते॒ऽभि॒भूः । अ॒भि॒भूर॑सि । अ॒भि॒भूरित्य॑भि - भूः । अ॒स्य॒भि॒भूः । अ॒भि॒भूर॒हम् । अ॒भि॒भूरित्य॑भि - भूः । अ॒हꣳ स॑जा॒तेषु॑ । स॒जा॒तेषु॑ भूयासम् । स॒जा॒तेष्विति॑ स - जा॒तेषु॑ । भू॒या॒स॒मिति॑ । इत्या॑ह । आ॒ह॒ यः । य ए॒व । ए॒वैन᳚म् । ए॒न॒म् प्र॒त्यु॒त्पिपी॑ते । प्र॒त्यु॒त्पिपी॑ते॒ तम् । प्र॒त्यु॒त्पिपी॑त॒ इति॑ प्रति - उ॒त्पिपी॑ते । तमुप॑ । उपा᳚स्यते । अ॒स्य॒ते॒ यु॒नज्मि॑ । यु॒नज्मि॑ त्वा । त्वा॒ ब्रह्म॑णा । ब्रह्म॑णा॒ दैव्ये॑न । दैव्ये॒नेति॑ । इत्या॑ह । आ॒है॒षः । ए॒ष वै । वा अ॒ग्नेः । अ॒ग्नेर्,योगः॑ । योग॒स्तेन॑ । तेनै॒व \newline

\textbf{Jatai Paata} \newline

1. ध्रु॒वो᳚ ऽस्यसि ध्रु॒वो ध्रु॒वो॑ ऽसि । \newline
2. अ॒सि॒ ध्रु॒वो ध्रु॒वो᳚ ऽस्यसि ध्रु॒वः । \newline
3. ध्रु॒वो॑ ऽह म॒हम् ध्रु॒वो ध्रु॒वो॑ ऽहम् । \newline
4. अ॒हꣳ स॑जा॒तेषु॑ सजा॒ते ष्व॒ह म॒हꣳ स॑जा॒तेषु॑ । \newline
5. स॒जा॒तेषु॑ भूयासम् भूयासꣳ सजा॒तेषु॑ सजा॒तेषु॑ भूयासम् । \newline
6. स॒जा॒तेष्विति॑ स - जा॒तेषु॑ । \newline
7. भू॒या॒स॒ मितीति॑ भूयासम् भूयास॒ मिति॑ । \newline
8. इत्या॑हा॒हे तीत्या॑ह । \newline
9. आ॒ह॒ ध्रु॒वान् ध्रु॒वा ना॑हाह ध्रु॒वान् । \newline
10. ध्रु॒वा ने॒वैव ध्रु॒वान् ध्रु॒वा ने॒व । \newline
11. ए॒वैना॑ नेना ने॒वै वैनान्॑ । \newline
12. ए॒ना॒न् कु॒रु॒ते॒ कु॒रु॒त॒ ए॒ना॒ ने॒ना॒न् कु॒रु॒ते॒ । \newline
13. कु॒रु॒त॒ उ॒ग्र उ॒ग्रः कु॑रुते कुरुत उ॒ग्रः । \newline
14. उ॒ग्रो᳚ ऽस्यस्यु॒ग्र उ॒ग्रो॑ ऽसि । \newline
15. अ॒स्यु॒ग्र उ॒ग्रो᳚ ऽस्यस्यु॒ग्रः । \newline
16. उ॒ग्रो॑ ऽह म॒ह मु॒ग्र उ॒ग्रो॑ ऽहम् । \newline
17. अ॒हꣳ स॑जा॒तेषु॑ सजा॒तेष्व॒ह म॒हꣳ स॑जा॒तेषु॑ । \newline
18. स॒जा॒तेषु॑ भूयासम् भूयासꣳ सजा॒तेषु॑ सजा॒तेषु॑ भूयासम् । \newline
19. स॒जा॒तेष्विति॑ स - जा॒तेषु॑ । \newline
20. भू॒या॒स॒ मितीति॑ भूयासम् भूयास॒ मिति॑ । \newline
21. इत्या॑हा॒हे तीत्या॑ह । \newline
22. आ॒हा प्र॑तिवादि॒नो ऽप्र॑तिवादिन आहा॒हा प्र॑तिवादिनः । \newline
23. अप्र॑तिवादिन ए॒वै वाप्र॑तिवादि॒नो ऽप्र॑तिवादिन ए॒व । \newline
24. अप्र॑तिवादिन॒ इत्यप्र॑ति - वा॒दि॒नः॒ । \newline
25. ए॒वैना॑ नेना ने॒वै वैनान्॑ । \newline
26. ए॒ना॒न् कु॒रु॒ते॒ कु॒रु॒त॒ ए॒ना॒ ने॒ना॒न् कु॒रु॒ते॒ । \newline
27. कु॒रु॒ते॒ ऽभि॒भू र॑भि॒भूः कु॑रुते कुरुते ऽभि॒भूः । \newline
28. अ॒भि॒भू र॑स्यस्यभि॒भू र॑भि॒भू र॑सि । \newline
29. अ॒भि॒भूरित्य॑भि - भूः । \newline
30. अ॒स्य॒भि॒भू र॑भि॒भू र॑स्यस्यभि॒भूः । \newline
31. अ॒भि॒भू र॒ह म॒ह म॑भि॒भू र॑भि॒भू र॒हम् । \newline
32. अ॒भि॒भूरित्य॑भि - भूः । \newline
33. अ॒हꣳ स॑जा॒तेषु॑ सजा॒तेष्व॒ह म॒हꣳ स॑जा॒तेषु॑ । \newline
34. स॒जा॒तेषु॑ भूयासम् भूयासꣳ सजा॒तेषु॑ सजा॒तेषु॑ भूयासम् । \newline
35. स॒जा॒तेष्विति॑ स - जा॒तेषु॑ । \newline
36. भू॒या॒स॒ मितीति॑ भूयासम् भूयास॒ मिति॑ । \newline
37. इत्या॑हा॒हे तीत्या॑ह । \newline
38. आ॒ह॒ यो य आ॑हाह॒ यः । \newline
39. य ए॒वैव यो य ए॒व । \newline
40. ए॒वैन॑ मेन मे॒वैवैन᳚म् । \newline
41. ए॒न॒म् प्र॒त्यु॒त्पिपी॑ते प्रत्यु॒त्पिपी॑त एन मेनम् प्रत्यु॒त्पिपी॑ते । \newline
42. प्र॒त्यु॒त्पिपी॑ते॒ तम् तम् प्र॑त्यु॒त्पिपी॑ते प्रत्यु॒त्पिपी॑ते॒ तम् । \newline
43. प्र॒त्यु॒त्पिपी॑त॒ इति॑ प्रति - उ॒त्पिपी॑ते । \newline
44. त मुपोप॒ तम् त मुप॑ । \newline
45. उपा᳚स्यते ऽस्यत॒ उपोपा᳚स्यते । \newline
46. अ॒स्य॒ते॒ यु॒नज्मि॑ यु॒न ज्म्य॑स्यते ऽस्यते यु॒नज्मि॑ । \newline
47. यु॒नज्मि॑ त्वा त्वा यु॒नज्मि॑ यु॒नज्मि॑ त्वा । \newline
48. त्वा॒ ब्रह्म॑णा॒ ब्रह्म॑णा त्वा त्वा॒ ब्रह्म॑णा । \newline
49. ब्रह्म॑णा॒ दैव्ये॑न॒ दैव्ये॑न॒ ब्रह्म॑णा॒ ब्रह्म॑णा॒ दैव्ये॑न । \newline
50. दैव्ये॒ने तीति॒ दैव्ये॑न॒ दैव्ये॒ने ति॑ । \newline
51. इत्या॑हा॒हे तीत्या॑ह । \newline
52. आ॒है॒ष ए॒ष आ॑हा है॒षः । \newline
53. ए॒ष वै वा ए॒ष ए॒ष वै । \newline
54. वा अ॒ग्ने र॒ग्नेर् वै वा अ॒ग्नेः । \newline
55. अ॒ग्नेर् योगो॒ योगो॒ ऽग्ने र॒ग्नेर् योगः॑ । \newline
56. योग॒ स्तेन॒ तेन॒ योगो॒ योग॒ स्तेन॑ । \newline
57. तेनै॒ वैव तेन॒ तेनै॒व । \newline

\textbf{Ghana Paata } \newline

1. ध्रु॒वो᳚ ऽस्यसि ध्रु॒वो ध्रु॒वो॑ ऽसि ध्रु॒वो ध्रु॒वो॑ ऽसि ध्रु॒वो ध्रु॒वो॑ ऽसि ध्रु॒वः । \newline
2. अ॒सि॒ ध्रु॒वो ध्रु॒वो᳚ ऽस्यसि ध्रु॒वो॑ ऽह म॒हम् ध्रु॒वो᳚ ऽस्यसि ध्रु॒वो॑ ऽहम् । \newline
3. ध्रु॒वो॑ ऽह म॒हम् ध्रु॒वो ध्रु॒वो॑ ऽहꣳ स॑जा॒तेषु॑ सजा॒तेष्व॒हम् ध्रु॒वो ध्रु॒वो॑ ऽहꣳ स॑जा॒तेषु॑ । \newline
4. अ॒हꣳ स॑जा॒तेषु॑ सजा॒तेष्व॒ह म॒हꣳ स॑जा॒तेषु॑ भूयासम् भूयासꣳ सजा॒तेष्व॒ह म॒हꣳ स॑जा॒तेषु॑ भूयासम् । \newline
5. स॒जा॒तेषु॑ भूयासम् भूयासꣳ सजा॒तेषु॑ सजा॒तेषु॑ भूयास॒ मितीति॑ भूयासꣳ सजा॒तेषु॑ सजा॒तेषु॑ भूयास॒ मिति॑ । \newline
6. स॒जा॒तेष्विति॑ स - जा॒तेषु॑ । \newline
7. भू॒या॒स॒ मितीति॑ भूयासम् भूयास॒ मित्या॑हा॒हे ति॑ भूयासम् भूयास॒ मित्या॑ह । \newline
8. इत्या॑हा॒हे तीत्या॑ह ध्रु॒वान् ध्रु॒वा ना॒हे तीत्या॑ह ध्रु॒वान् । \newline
9. आ॒ह॒ ध्रु॒वान् ध्रु॒वा ना॑हाह ध्रु॒वा ने॒वैव ध्रु॒वा ना॑हाह ध्रु॒वा ने॒व । \newline
10. ध्रु॒वा ने॒वैव ध्रु॒वान् ध्रु॒वा ने॒वैना॑ नेना ने॒व ध्रु॒वान् ध्रु॒वा ने॒वैनान्॑ । \newline
11. ए॒वैना॑ नेना ने॒वैवैना᳚न् कुरुते कुरुत एना ने॒वैवैना᳚न् कुरुते । \newline
12. ए॒ना॒न् कु॒रु॒ते॒ कु॒रु॒त॒ ए॒ना॒ ने॒ना॒न् कु॒रु॒त॒ उ॒ग्र उ॒ग्रः कु॑रुत एना नेनान् कुरुत उ॒ग्रः । \newline
13. कु॒रु॒त॒ उ॒ग्र उ॒ग्रः कु॑रुते कुरुत उ॒ग्रो᳚ ऽस्यस्यु॒ग्रः कु॑रुते कुरुत उ॒ग्रो॑ ऽसि । \newline
14. उ॒ग्रो᳚ ऽस्यस्यु॒ग्र उ॒ग्रो᳚ ऽस्यु॒ग्र उ॒ग्रो᳚ ऽस्यु॒ग्र उ॒ग्रो᳚ ऽस्यु॒ग्रः । \newline
15. अ॒स्यु॒ग्र उ॒ग्रो᳚ ऽस्यस्यु॒ग्रो॑ ऽह म॒ह मु॒ग्रो᳚ ऽस्यस्यु॒ग्रो॑ ऽहम् । \newline
16. उ॒ग्रो॑ ऽह म॒ह मु॒ग्र उ॒ग्रो॑ ऽहꣳ स॑जा॒तेषु॑ सजा॒तेष्व॒ह मु॒ग्र उ॒ग्रो॑ ऽहꣳ स॑जा॒तेषु॑ । \newline
17. अ॒हꣳ स॑जा॒तेषु॑ सजा॒तेष्व॒ह म॒हꣳ स॑जा॒तेषु॑ भूयासम् भूयासꣳ सजा॒तेष्व॒ह म॒हꣳ स॑जा॒तेषु॑ भूयासम् । \newline
18. स॒जा॒तेषु॑ भूयासम् भूयासꣳ सजा॒तेषु॑ सजा॒तेषु॑ भूयास॒ मितीति॑ भूयासꣳ सजा॒तेषु॑ सजा॒तेषु॑ भूयास॒ मिति॑ । \newline
19. स॒जा॒तेष्विति॑ स - जा॒तेषु॑ । \newline
20. भू॒या॒स॒ मितीति॑ भूयासम् भूयास॒ मित्या॑हा॒हे ति॑ भूयासम् भूयास॒ मित्या॑ह । \newline
21. इत्या॑हा॒हे तीत्या॒हाप्र॑तिवादि॒नो ऽप्र॑तिवादिन आ॒हे तीत्या॒हाप्र॑तिवादिनः । \newline
22. आ॒हाप्र॑तिवादि॒नो ऽप्र॑तिवादिन आहा॒हाप्र॑तिवादिन ए॒वैवाप्र॑तिवादिन आहा॒हाप्र॑तिवादिन ए॒व । \newline
23. अप्र॑तिवादिन ए॒वैवाप्र॑तिवादि॒नो ऽप्र॑तिवादिन ए॒वैना॑ नेना ने॒वाप्र॑तिवादि॒नो ऽप्र॑तिवादिन ए॒वैनान्॑ । \newline
24. अप्र॑तिवादिन॒ इत्यप्र॑ति - वा॒दि॒नः॒ । \newline
25. ए॒वैना॑ नेना ने॒वैवैना᳚न् कुरुते कुरुत एना ने॒वैवैना᳚न् कुरुते । \newline
26. ए॒ना॒न् कु॒रु॒ते॒ कु॒रु॒त॒ ए॒ना॒ ने॒ना॒न् कु॒रु॒ते॒ ऽभि॒भू र॑भि॒भूः कु॑रुत एना नेनान् कुरुते ऽभि॒भूः । \newline
27. कु॒रु॒ते॒ ऽभि॒भू र॑भि॒भूः कु॑रुते कुरुते ऽभि॒भू र॑स्यस्यभि॒भूः कु॑रुते कुरुते ऽभि॒भूर॑सि । \newline
28. अ॒भि॒भू र॑स्यस्यभि॒भू र॑भि॒भू र॑स्यभि॒भू र॑भि॒भू र॑स्यभि॒भू र॑भि॒भू र॑स्यभि॒भूः । \newline
29. अ॒भि॒भूरित्य॑भि - भूः । \newline
30. अ॒स्य॒भि॒भू र॑भि॒भू र॑स्यस्यभि॒भू र॒ह म॒ह म॑भि॒भू र॑स्यस्यभि॒भू र॒हम् । \newline
31. अ॒भि॒भूर॒ह म॒ह म॑भि॒भू र॑भि॒भू र॒हꣳ स॑जा॒तेषु॑ सजा॒तेष्व॒ह म॑भि॒भू र॑भि॒भू र॒हꣳ स॑जा॒तेषु॑ । \newline
32. अ॒भि॒भूरित्य॑भि - भूः । \newline
33. अ॒हꣳ स॑जा॒तेषु॑ सजा॒तेष्व॒ह म॒हꣳ स॑जा॒तेषु॑ भूयासम् भूयासꣳ सजा॒तेष्व॒ह म॒हꣳ स॑जा॒तेषु॑ भूयासम् । \newline
34. स॒जा॒तेषु॑ भूयासम् भूयासꣳ सजा॒तेषु॑ सजा॒तेषु॑ भूयास॒ मितीति॑ भूयासꣳ सजा॒तेषु॑ सजा॒तेषु॑ भूयास॒ मिति॑ । \newline
35. स॒जा॒तेष्विति॑ स - जा॒तेषु॑ । \newline
36. भू॒या॒स॒ मितीति॑ भूयासम् भूयास॒ मित्या॑हा॒हे ति॑ भूयासम् भूयास॒ मित्या॑ह । \newline
37. इत्या॑हा॒हे तीत्या॑ह॒ यो य आ॒हे तीत्या॑ह॒ यः । \newline
38. आ॒ह॒ यो य आ॑हाह॒ य ए॒वैव य आ॑हाह॒ य ए॒व । \newline
39. य ए॒वैव यो य ए॒वैन॑ मेन मे॒व यो य ए॒वैन᳚म् । \newline
40. ए॒वैन॑ मेन मे॒वैवैन॑म् प्रत्यु॒त्पिपी॑ते प्रत्यु॒त्पिपी॑त एन मे॒वैवैन॑म् प्रत्यु॒त्पिपी॑ते । \newline
41. ए॒न॒म् प्र॒त्यु॒त्पिपी॑ते प्रत्यु॒त्पिपी॑त एन मेनम् प्रत्यु॒त्पिपी॑ते॒ तम् तम् प्र॑त्यु॒त्पिपी॑त एन मेनम् प्रत्यु॒त्पिपी॑ते॒ तम् । \newline
42. प्र॒त्यु॒त्पिपी॑ते॒ तम् तम् प्र॑त्यु॒त्पिपी॑ते प्रत्यु॒त्पिपी॑ते॒ त मुपोप॒ तम् प्र॑त्यु॒त्पिपी॑ते प्रत्यु॒त्पिपी॑ते॒ त मुप॑ । \newline
43. प्र॒त्यु॒त्पिपी॑त॒ इति॑ प्रति - उ॒त्पिपी॑ते । \newline
44. त मुपोप॒ तम् त मुपा᳚स्यते ऽस्यत॒ उप॒ तम् त मुपा᳚स्यते । \newline
45. उपा᳚स्यते ऽस्यत॒ उपोपा᳚स्यते यु॒नज्मि॑ यु॒नज्म्य॑स्यत॒ उपोपा᳚स्यते यु॒नज्मि॑ । \newline
46. अ॒स्य॒ते॒ यु॒नज्मि॑ यु॒नज्म्य॑स्यते ऽस्यते यु॒नज्मि॑ त्वा त्वा यु॒नज्म्य॑स्यते ऽस्यते यु॒नज्मि॑ त्वा । \newline
47. यु॒नज्मि॑ त्वा त्वा यु॒नज्मि॑ यु॒नज्मि॑ त्वा॒ ब्रह्म॑णा॒ ब्रह्म॑णा त्वा यु॒नज्मि॑ यु॒नज्मि॑ त्वा॒ ब्रह्म॑णा । \newline
48. त्वा॒ ब्रह्म॑णा॒ ब्रह्म॑णा त्वा त्वा॒ ब्रह्म॑णा॒ दैव्ये॑न॒ दैव्ये॑न॒ ब्रह्म॑णा त्वा त्वा॒ ब्रह्म॑णा॒ दैव्ये॑न । \newline
49. ब्रह्म॑णा॒ दैव्ये॑न॒ दैव्ये॑न॒ ब्रह्म॑णा॒ ब्रह्म॑णा॒ दैव्ये॒ने तीति॒ दैव्ये॑न॒ ब्रह्म॑णा॒ ब्रह्म॑णा॒ दैव्ये॒ने ति॑ । \newline
50. दैव्ये॒ने तीति॒ दैव्ये॑न॒ दैव्ये॒ने त्या॑हा॒हे ति॒ दैव्ये॑न॒ दैव्ये॒ने त्या॑ह । \newline
51. इत्या॑हा॒हे तीत्या॑है॒ष ए॒ष आ॒हे तीत्या॑है॒षः । \newline
52. आ॒है॒ष ए॒ष आ॑हाहै॒ष वै वा ए॒ष आ॑हाहै॒ष वै । \newline
53. ए॒ष वै वा ए॒ष ए॒ष वा अ॒ग्ने र॒ग्नेर् वा ए॒ष ए॒ष वा अ॒ग्नेः । \newline
54. वा अ॒ग्ने र॒ग्नेर् वै वा अ॒ग्नेर् योगो॒ योगो॒ ऽग्नेर् वै वा अ॒ग्नेर् योगः॑ । \newline
55. अ॒ग्नेर् योगो॒ योगो॒ ऽग्ने र॒ग्नेर् योग॒ स्तेन॒ तेन॒ योगो॒ ऽग्ने र॒ग्नेर् योग॒स्तेन॑ । \newline
56. योग॒स्तेन॒ तेन॒ योगो॒ योग॒ स्तेनै॒वैव तेन॒ योगो॒ योग॒ स्तेनै॒व । \newline
57. तेनै॒वैव तेन॒ तेनै॒वैन॑ मेन मे॒व तेन॒ तेनै॒वैन᳚म् । \newline
\pagebreak
\markright{ TS 1.6.10.2  \hfill https://www.vedavms.in \hfill}

\section{ TS 1.6.10.2 }

\textbf{TS 1.6.10.2 } \newline
\textbf{Samhita Paata} \newline

वैनं॑ ॅयुनक्ति य॒ज्ञ्स्य॒ वै समृ॑द्धेन दे॒वाः सु॑व॒र्गं ॅलो॒कमा॑यन्. य॒ज्ञ्स्य॒ व्यृ॑द्धे॒नासु॑रा॒न् परा॑भावय॒न्. यन्मे॑ अग्ने अ॒स्य य॒ज्ञ्स्य॒ रिष्या॒दित्या॑ह य॒ज्ञ्स्यै॒व तथ्समृ॑द्धेन॒ यज॑मानः सुव॒र्गं ॅलो॒कमे॑ति य॒ज्ञ्स्य॒ व्यृ॑द्धेन॒ भ्रातृ॑व्या॒न् परा॑ भावयत्यग्निहो॒त्र-मे॒ताभि॒र् व्याहृ॑तीभि॒रुप॑ सादयेद्यज्ञ्मु॒खं ॅवा अ॑ग्निहो॒त्रं ब्रह्मै॒ता व्याहृ॑तयो यज्ञ्मु॒ख ए॒व ब्रह्म॑ - [ ] \newline

\textbf{Pada Paata} \newline

ए॒व । ए॒न॒म् । यु॒न॒क्ति॒ । य॒ज्ञ्स्य॑ । वै । समृ॑द्धे॒नेति॒ सं - ऋ॒द्धे॒न॒ । दे॒वाः । सु॒व॒र्गमिति॑ सुवः - गम् । लो॒कम् । आ॒य॒न्न् । य॒ज्ञ्स्य॑ । व्यृ॑द्धे॒नेति॒ वि - ऋ॒द्धे॒न॒ । असु॑रान् । परेति॑ । अ॒भा॒व॒य॒न्न् । यत् । मे॒ । अ॒ग्ने॒ । अ॒स्य । य॒ज्ञ्स्य॑ । रिष्या᳚त् । इति॑ । आ॒ह॒ । य॒ज्ञ्स्य॑ । ए॒व । तत् । समृ॑द्धे॒नेति॒ सं - ऋ॒द्धे॒न॒ । यज॑मानः । सु॒व॒र्गमिति॑ सुवः - गम् । लो॒कम् । ए॒ति॒ । य॒ज्ञ्स्य॑ । व्यृ॑द्धे॒नेति॒ वि - ऋ॒द्धे॒न॒ । भ्रातृ॑व्यान् । परेति॑ । भा॒व॒य॒ति॒ । अ॒ग्नि॒हो॒त्रमित्य॑ग्नि - हो॒त्रम् । ए॒ताभिः॑ । व्याहृ॑तीभि॒रिति॒ व्याहृ॑ति - भिः॒ । उपेति॑ । सा॒द॒ये॒त् । य॒ज्ञ्॒मु॒खमिति॑ यज्ञ् - म॒खम् । वै । अ॒ग्नि॒हो॒त्रमित्य॑ग्नि - हो॒त्रम् । ब्रह्म॑ । ए॒ताः । व्याहृ॑तय॒ इति॑ वि - आहृ॑तयः । य॒ज्ञ्॒मु॒ख इति॑ यज्ञ् - मु॒खे । ए॒व । ब्रह्म॑ ।  \newline


\textbf{Krama Paata} \newline

ए॒वैन᳚म् । ए॒नं॒ ॅयु॒न॒क्ति॒ । यु॒न॒क्ति॒ य॒ज्ञ्स्य॑ । य॒ज्ञ्स्य॒ वै । वै समृ॑द्धेन । समृ॑द्धेन दे॒वाः । समृ॑द्धे॒नेति॒ सं - ऋ॒द्धे॒न॒ । दे॒वाः सु॑व॒र्गम् । सु॒व॒र्गं ॅलो॒कम् । सु॒व॒र्गमिति॑ सुवः - गम् । लो॒कमा॑यन्न् । आ॒य॒न्॒. य॒ज्ञ्स्य॑ । य॒ज्ञ्स्य॒ व्यृ॑द्धेन । व्यृ॑द्धे॒नासु॑रान् । व्यृ॑द्धे॒नेति॒ वि - ऋ॒द्धे॒न॒ । असु॑रा॒न् परा᳚ । परा॑ ऽभावयन्न् । अ॒भा॒व॒य॒न्. यत् । यन्मे᳚ । मे॒ अ॒ग्ने॒ । अ॒ग्ने॒ अ॒स्य । अ॒स्य य॒ज्ञ्स्य॑ । य॒ज्ञ्स्य॒ रिष्या᳚त् । रिष्या॒दिति॑ । इत्या॑ह । आ॒ह॒ य॒ज्ञ्स्य॑ । य॒ज्ञ्स्यै॒व । ए॒व तत् । तथ् समृ॑द्धेन । समृ॑द्धेन॒ यज॑मानः । समृ॑द्धे॒नेति॒ सम् - ऋ॒द्धे॒न॒ । यज॑मानः सुव॒र्गम् । सु॒व॒र्गं ॅलो॒कम् । सु॒व॒र्गमिति॑ सुवः - गम् । लो॒कमे॑ति । ए॒ति॒ य॒ज्ञ्स्य॑ । य॒ज्ञ्स्य॒ व्यृ॑द्धेन । व्यृ॑द्धेन॒ भ्रातृ॑व्यान् । व्यृ॑द्धे॒नेति॒ वि - ऋ॒द्धे॒न॒ । भ्रातृ॑व्या॒न् परा᳚ । परा॑ भावयति । भा॒व॒य॒त्य॒ग्नि॒हो॒त्रम् । अ॒ग्नि॒हो॒त्रमे॒ताभिः॑ । अ॒ग्नि॒हो॒त्रमित्य॑ग्नि - हो॒त्रम् । ए॒ताभि॒र्,व्याहृ॑तीभिः । व्याहृ॑तीभि॒रुप॑ । व्याहृ॑तीभि॒रिति॒ व्याहृ॑ति - भिः॒ । उप॑ सादयेत् । सा॒द॒ये॒द्,य॒ज्ञ्॒मु॒खम् । य॒ज्ञ्॒मु॒खं ॅवै । य॒ज्ञ्॒मु॒खमिति॑ यज्ञ् - मु॒खम् । वा अ॑ग्निहो॒त्रम् । अ॒ग्नि॒हो॒त्रम् ब्रह्म॑ । अ॒ग्नि॒हो॒त्रमित्य॑ग्नि - हो॒त्रम् । ब्रह्मै॒ताः । ए॒ता व्याहृ॑तयः । व्याहृ॑तयो यज्ञ्मु॒खे । व्याहृ॑तय॒ इति॑ वि - आहृ॑तयः । य॒ज्ञ्॒मु॒ख ए॒व । य॒ज्ञ्॒मु॒ख इति॑ यज्ञ् - मु॒खे । ए॒व ब्रह्म॑ । ब्रह्म॑ कुरुते \newline

\textbf{Jatai Paata} \newline

1. ए॒वैन॑ मेन मे॒वै वैन᳚म् । \newline
2. ए॒नं॒ ॅयु॒न॒क्ति॒ यु॒न॒क् त्ये॒न॒ मे॒नं॒ ॅयु॒न॒क्ति॒ । \newline
3. यु॒न॒क्ति॒ य॒ज्ञ्स्य॑ य॒ज्ञ्स्य॑ युनक्ति युनक्ति य॒ज्ञ्स्य॑ । \newline
4. य॒ज्ञ्स्य॒ वै वै य॒ज्ञ्स्य॑ य॒ज्ञ्स्य॒ वै । \newline
5. वै समृ॑द्धेन॒ समृ॑द्धेन॒ वै वै समृ॑द्धेन । \newline
6. समृ॑द्धेन दे॒वा दे॒वाः समृ॑द्धेन॒ समृ॑द्धेन दे॒वाः । \newline
7. समृ॑द्धे॒नेति॒ सं - ऋ॒द्धे॒न॒ । \newline
8. दे॒वाः सु॑व॒र्गꣳ सु॑व॒र्गम् दे॒वा दे॒वाः सु॑व॒र्गम् । \newline
9. सु॒व॒र्गम् ॅलो॒कम् ॅलो॒कꣳ सु॑व॒र्गꣳ सु॑व॒र्गम् ॅलो॒कम् । \newline
10. सु॒व॒र्गमिति॑ सुवः - गम् । \newline
11. लो॒क मा॑यन् नायन्न् ॅलो॒कम् ॅलो॒क मा॑यन्न् । \newline
12. आ॒य॒न्॒. य॒ज्ञ्स्य॑ य॒ज्ञ्स्या॑यन् नायन्. य॒ज्ञ्स्य॑ । \newline
13. य॒ज्ञ्स्य॒ व्यृ॑द्धेन॒ व्यृ॑द्धेन य॒ज्ञ्स्य॑ य॒ज्ञ्स्य॒ व्यृ॑द्धेन । \newline
14. व्यृ॑द्धे॒नासु॑रा॒ नसु॑रा॒न् व्यृ॑द्धेन॒ व्यृ॑द्धे॒नासु॑रान् । \newline
15. व्यृ॑द्धे॒नेति॒ वि - ऋ॒द्धे॒न॒ । \newline
16. असु॑रा॒न् परा॒ परा ऽसु॑रा॒ नसु॑रा॒न् परा᳚ । \newline
17. परा॑ ऽभावयन् नभावय॒न् परा॒ परा॑ ऽभावयन्न् । \newline
18. अ॒भा॒व॒य॒न्॒. यद् यद॑भावयन् नभावय॒न्॒. यत् । \newline
19. यन् मे॑ मे॒ यद् यन् मे᳚ । \newline
20. मे॒ अ॒ग्ने॒ ऽग्ने॒ मे॒ मे॒ अ॒ग्ने॒ । \newline
21. अ॒ग्ने॒ अ॒स्यास्याग्ने᳚ ऽग्ने अ॒स्य । \newline
22. अ॒स्य य॒ज्ञ्स्य॑ य॒ज्ञ् स्या॒स्यास्य य॒ज्ञ्स्य॑ । \newline
23. य॒ज्ञ्स्य॒ रिष्या॒द् रिष्या᳚द् य॒ज्ञ्स्य॑ य॒ज्ञ्स्य॒ रिष्या᳚त् । \newline
24. रिष्या॒ दितीति॒ रिष्या॒द् रिष्या॒ दिति॑ । \newline
25. इत्या॑हा॒हे तीत्या॑ह । \newline
26. आ॒ह॒ य॒ज्ञ्स्य॑ य॒ज्ञ् स्या॑हाह य॒ज्ञ्स्य॑ । \newline
27. य॒ज्ञ्स्यै॒ वैव य॒ज्ञ्स्य॑ य॒ज्ञ्स्यै॒व । \newline
28. ए॒व तत् तदे॒ वैव तत् । \newline
29. तथ् समृ॑द्धेन॒ समृ॑द्धेन॒ तत् तथ् समृ॑द्धेन । \newline
30. समृ॑द्धेन॒ यज॑मानो॒ यज॑मानः॒ समृ॑द्धेन॒ समृ॑द्धेन॒ यज॑मानः । \newline
31. समृ॑द्धे॒नेति॒ सं - ऋ॒द्धे॒न॒ । \newline
32. यज॑मानः सुव॒र्गꣳ सु॑व॒र्गं ॅयज॑मानो॒ यज॑मानः सुव॒र्गम् । \newline
33. सु॒व॒र्गम् ॅलो॒कम् ॅलो॒कꣳ सु॑व॒र्गꣳ सु॑व॒र्गम् ॅलो॒कम् । \newline
34. सु॒व॒र्गमिति॑ सुवः - गम् । \newline
35. लो॒क मे᳚त्येति लो॒कम् ॅलो॒क मे॑ति । \newline
36. ए॒ति॒ य॒ज्ञ्स्य॑ य॒ज्ञ् स्यै᳚त्येति य॒ज्ञ्स्य॑ । \newline
37. य॒ज्ञ्स्य॒ व्यृ॑द्धेन॒ व्यृ॑द्धेन य॒ज्ञ्स्य॑ य॒ज्ञ्स्य॒ व्यृ॑द्धेन । \newline
38. व्यृ॑द्धेन॒ भ्रातृ॑व्या॒न् भ्रातृ॑व्या॒न् व्यृ॑द्धेन॒ व्यृ॑द्धेन॒ भ्रातृ॑व्यान् । \newline
39. व्यृ॑द्धे॒नेति॒ वि - ऋ॒द्धे॒न॒ । \newline
40. भ्रातृ॑व्या॒न् परा॒ परा॒ भ्रातृ॑व्या॒न् भ्रातृ॑व्या॒न् परा᳚ । \newline
41. परा॑ भावयति भावयति॒ परा॒ परा॑ भावयति । \newline
42. भा॒व॒य॒ त्य॒ग्नि॒हो॒त्र म॑ग्निहो॒त्रम् भा॑वयति भावय त्यग्निहो॒त्रम् । \newline
43. अ॒ग्नि॒हो॒त्र मे॒ताभि॑ रे॒ताभि॑ रग्निहो॒त्र म॑ग्निहो॒त्र मे॒ताभिः॑ । \newline
44. अ॒ग्नि॒हो॒त्रमित्य॑ग्नि - हो॒त्रम् । \newline
45. ए॒ताभि॒र् व्याहृ॑तीभि॒र् व्याहृ॑तीभि रे॒ताभि॑ रे॒ताभि॒र् व्याहृ॑तीभिः । \newline
46. व्याहृ॑तीभि॒ रुपोप॒ व्याहृ॑तीभि॒र् व्याहृ॑तीभि॒ रुप॑ । \newline
47. व्याहृ॑तीभि॒रिति॒ व्याहृ॑ति - भिः॒ । \newline
48. उप॑ सादयेथ् सादये॒ दुपोप॑ सादयेत् । \newline
49. सा॒द॒ये॒द् य॒ज्ञ्॒मु॒खं ॅय॑ज्ञ्मु॒खꣳ सा॑दयेथ् सादयेद् यज्ञ्मु॒खम् । \newline
50. य॒ज्ञ्॒मु॒खं ॅवै वै य॑ज्ञ्मु॒खं ॅय॑ज्ञ्मु॒खं ॅवै । \newline
51. य॒ज्ञ्॒मु॒खमिति॑ यज्ञ् - मु॒खम् । \newline
52. वा अ॑ग्निहो॒त्र म॑ग्निहो॒त्रं ॅवै वा अ॑ग्निहो॒त्रम् । \newline
53. अ॒ग्नि॒हो॒त्रम् ब्रह्म॒ ब्रह्मा᳚ग्निहो॒त्र म॑ग्निहो॒त्रम् ब्रह्म॑ । \newline
54. अ॒ग्नि॒हो॒त्रमित्य॑ग्नि - हो॒त्रम् । \newline
55. ब्रह्मै॒ता ए॒ता ब्रह्म॒ ब्रह्मै॒ताः । \newline
56. ए॒ता व्याहृ॑तयो॒ व्याहृ॑तय ए॒ता ए॒ता व्याहृ॑तयः । \newline
57. व्याहृ॑तयो यज्ञ्मु॒खे य॑ज्ञ्मु॒खे व्याहृ॑तयो॒ व्याहृ॑तयो यज्ञ्मु॒खे । \newline
58. व्याहृ॑तय॒ इति॑ वि - आहृ॑तयः । \newline
59. य॒ज्ञ्॒मु॒ख ए॒वैव य॑ज्ञ्मु॒खे य॑ज्ञ्मु॒ख ए॒व । \newline
60. य॒ज्ञ्॒मु॒ख इति॑ यज्ञ् - मु॒खे । \newline
61. ए॒व ब्रह्म॒ ब्रह्मै॒वैव ब्रह्म॑ । \newline
62. ब्रह्म॑ कुरुते कुरुते॒ ब्रह्म॒ ब्रह्म॑ कुरुते । \newline

\textbf{Ghana Paata } \newline

1. ए॒वैन॑ मेन मे॒वैवैनं॑ ॅयुनक्ति युनक्त्येन मे॒वैवैनं॑ ॅयुनक्ति । \newline
2. ए॒नं॒ ॅयु॒न॒क्ति॒ यु॒न॒क्त्ये॒न॒ मे॒नं॒ ॅयु॒न॒क्ति॒ य॒ज्ञ्स्य॑ य॒ज्ञ्स्य॑ युनक्त्येन मेनं ॅयुनक्ति य॒ज्ञ्स्य॑ । \newline
3. यु॒न॒क्ति॒ य॒ज्ञ्स्य॑ य॒ज्ञ्स्य॑ युनक्ति युनक्ति य॒ज्ञ्स्य॒ वै वै य॒ज्ञ्स्य॑ युनक्ति युनक्ति य॒ज्ञ्स्य॒ वै । \newline
4. य॒ज्ञ्स्य॒ वै वै य॒ज्ञ्स्य॑ य॒ज्ञ्स्य॒ वै समृ॑द्धेन॒ समृ॑द्धेन॒ वै य॒ज्ञ्स्य॑ य॒ज्ञ्स्य॒ वै समृ॑द्धेन । \newline
5. वै समृ॑द्धेन॒ समृ॑द्धेन॒ वै वै समृ॑द्धेन दे॒वा दे॒वाः समृ॑द्धेन॒ वै वै समृ॑द्धेन दे॒वाः । \newline
6. समृ॑द्धेन दे॒वा दे॒वाः समृ॑द्धेन॒ समृ॑द्धेन दे॒वाः सु॑व॒र्गꣳ सु॑व॒र्गम् दे॒वाः समृ॑द्धेन॒ समृ॑द्धेन दे॒वाः सु॑व॒र्गम् । \newline
7. समृ॑द्धे॒नेति॒ सं - ऋ॒द्धे॒न॒ । \newline
8. दे॒वाः सु॑व॒र्गꣳ सु॑व॒र्गम् दे॒वा दे॒वाः सु॑व॒र्गम् ॅलो॒कम् ॅलो॒कꣳ सु॑व॒र्गम् दे॒वा दे॒वाः सु॑व॒र्गम् ॅलो॒कम् । \newline
9. सु॒व॒र्गम् ॅलो॒कम् ॅलो॒कꣳ सु॑व॒र्गꣳ सु॑व॒र्गम् ॅलो॒क मा॑यन् नायन् ॅलो॒कꣳ सु॑व॒र्गꣳ सु॑व॒र्गम् ॅलो॒क मा॑यन्न् । \newline
10. सु॒व॒र्गमिति॑ सुवः - गम् । \newline
11. लो॒क मा॑यन् नायन् ॅलो॒कम् ॅलो॒क मा॑यन्. य॒ज्ञ्स्य॑ य॒ज्ञ्स्या॑यन् ॅलो॒कम् ॅलो॒क मा॑यन्. य॒ज्ञ्स्य॑ । \newline
12. आ॒य॒न्॒. य॒ज्ञ्स्य॑ य॒ज्ञ्स्या॑यन् नायन्. य॒ज्ञ्स्य॒ व्यृ॑द्धेन॒ व्यृ॑द्धेन य॒ज्ञ्स्या॑यन् नायन्. य॒ज्ञ्स्य॒ व्यृ॑द्धेन । \newline
13. य॒ज्ञ्स्य॒ व्यृ॑द्धेन॒ व्यृ॑द्धेन य॒ज्ञ्स्य॑ य॒ज्ञ्स्य॒ व्यृ॑द्धे॒नासु॑रा॒ नसु॑रा॒न् व्यृ॑द्धेन य॒ज्ञ्स्य॑ य॒ज्ञ्स्य॒ व्यृ॑द्धे॒नासु॑रान् । \newline
14. व्यृ॑द्धे॒नासु॑रा॒ नसु॑रा॒न् व्यृ॑द्धेन॒ व्यृ॑द्धे॒नासु॑रा॒न् परा॒ परा ऽसु॑रा॒न् व्यृ॑द्धेन॒ व्यृ॑द्धे॒नासु॑रा॒न् परा᳚ । \newline
15. व्यृ॑द्धे॒नेति॒ वि - ऋ॒द्धे॒न॒ । \newline
16. असु॑रा॒न् परा॒ परा ऽसु॑रा॒ नसु॑रा॒न् परा॑ ऽभावयन् नभावय॒न् परा ऽसु॑रा॒ नसु॑रा॒न् परा॑ ऽभावयन्न् । \newline
17. परा॑ ऽभावयन् नभावय॒न् परा॒ परा॑ ऽभावय॒न्॒. यद् यद॑भावय॒न् परा॒ परा॑ ऽभावय॒न्॒. यत् । \newline
18. अ॒भा॒व॒य॒न्॒. यद् यद॑भावयन् नभावय॒न्॒. यन् मे॑ मे॒ यद॑भावयन् नभावय॒न्॒. यन् मे᳚ । \newline
19. यन् मे॑ मे॒ यद् यन् मे॑ अग्ने ऽग्ने मे॒ यद् यन् मे॑ अग्ने । \newline
20. मे॒ अ॒ग्ने॒ ऽग्ने॒ मे॒ मे॒ अ॒ग्ने॒ अ॒स्यास्याग्ने॑ मे मे अग्ने अ॒स्य । \newline
21. अ॒ग्ने॒ अ॒स्यास्याग्ने᳚ ऽग्ने अ॒स्य य॒ज्ञ्स्य॑ य॒ज्ञ्स्या॒स्याग्ने᳚ ऽग्ने अ॒स्य य॒ज्ञ्स्य॑ । \newline
22. अ॒स्य य॒ज्ञ्स्य॑ य॒ज्ञ्स्या॒स्यास्य य॒ज्ञ्स्य॒ रिष्या॒द् रिष्या᳚द् य॒ज्ञ्स्या॒स्यास्य य॒ज्ञ्स्य॒ रिष्या᳚त् । \newline
23. य॒ज्ञ्स्य॒ रिष्या॒द् रिष्या᳚द् य॒ज्ञ्स्य॑ य॒ज्ञ्स्य॒ रिष्या॒दितीति॒ रिष्या᳚द् य॒ज्ञ्स्य॑ य॒ज्ञ्स्य॒ रिष्या॒दिति॑ । \newline
24. रिष्या॒दितीति॒ रिष्या॒द् रिष्या॒ दित्या॑हा॒हे ति॒ रिष्या॒द् रिष्या॒दित्या॑ह । \newline
25. इत्या॑हा॒हे तीत्या॑ह य॒ज्ञ्स्य॑ य॒ज्ञ्स्या॒हे तीत्या॑ह य॒ज्ञ्स्य॑ । \newline
26. आ॒ह॒ य॒ज्ञ्स्य॑ य॒ज्ञ्स्या॑हाह य॒ज्ञ्स्यै॒वैव य॒ज्ञ्स्या॑हाह य॒ज्ञ्स्यै॒व । \newline
27. य॒ज्ञ्स्यै॒वैव य॒ज्ञ्स्य॑ य॒ज्ञ्स्यै॒व तत् तदे॒व य॒ज्ञ्स्य॑ य॒ज्ञ्स्यै॒व तत् । \newline
28. ए॒व तत् तदे॒वैव तथ् समृ॑द्धेन॒ समृ॑द्धेन॒ तदे॒वैव तथ् समृ॑द्धेन । \newline
29. तथ् समृ॑द्धेन॒ समृ॑द्धेन॒ तत् तथ् समृ॑द्धेन॒ यज॑मानो॒ यज॑मानः॒ समृ॑द्धेन॒ तत् तथ् समृ॑द्धेन॒ यज॑मानः । \newline
30. समृ॑द्धेन॒ यज॑मानो॒ यज॑मानः॒ समृ॑द्धेन॒ समृ॑द्धेन॒ यज॑मानः सुव॒र्गꣳ सु॑व॒र्गं ॅयज॑मानः॒ समृ॑द्धेन॒ समृ॑द्धेन॒ यज॑मानः सुव॒र्गम् । \newline
31. समृ॑द्धे॒नेति॒ सं - ऋ॒द्धे॒न॒ । \newline
32. यज॑मानः सुव॒र्गꣳ सु॑व॒र्गं ॅयज॑मानो॒ यज॑मानः सुव॒र्गम् ॅलो॒कम् ॅलो॒कꣳ सु॑व॒र्गं ॅयज॑मानो॒ यज॑मानः सुव॒र्गम् ॅलो॒कम् । \newline
33. सु॒व॒र्गम् ॅलो॒कम् ॅलो॒कꣳ सु॑व॒र्गꣳ सु॑व॒र्गम् ॅलो॒क मे᳚त्येति लो॒कꣳ सु॑व॒र्गꣳ सु॑व॒र्गम् ॅलो॒क मे॑ति । \newline
34. सु॒व॒र्गमिति॑ सुवः - गम् । \newline
35. लो॒क मे᳚त्येति लो॒कम् ॅलो॒क मे॑ति य॒ज्ञ्स्य॑ य॒ज्ञ्स्यै॑ति लो॒कम् ॅलो॒क मे॑ति य॒ज्ञ्स्य॑ । \newline
36. ए॒ति॒ य॒ज्ञ्स्य॑ य॒ज्ञ्स्यै᳚त्येति य॒ज्ञ्स्य॒ व्यृ॑द्धेन॒ व्यृ॑द्धेन य॒ज्ञ्स्यै᳚त्येति य॒ज्ञ्स्य॒ व्यृ॑द्धेन । \newline
37. य॒ज्ञ्स्य॒ व्यृ॑द्धेन॒ व्यृ॑द्धेन य॒ज्ञ्स्य॑ य॒ज्ञ्स्य॒ व्यृ॑द्धेन॒ भ्रातृ॑व्या॒न् भ्रातृ॑व्या॒न् व्यृ॑द्धेन य॒ज्ञ्स्य॑ य॒ज्ञ्स्य॒ व्यृ॑द्धेन॒ भ्रातृ॑व्यान् । \newline
38. व्यृ॑द्धेन॒ भ्रातृ॑व्या॒न् भ्रातृ॑व्या॒न् व्यृ॑द्धेन॒ व्यृ॑द्धेन॒ भ्रातृ॑व्या॒न् परा॒ परा॒ भ्रातृ॑व्या॒न् व्यृ॑द्धेन॒ व्यृ॑द्धेन॒ भ्रातृ॑व्या॒न् परा᳚ । \newline
39. व्यृ॑द्धे॒नेति॒ वि - ऋ॒द्धे॒न॒ । \newline
40. भ्रातृ॑व्या॒न् परा॒ परा॒ भ्रातृ॑व्या॒न् भ्रातृ॑व्या॒न् परा॑ भावयति भावयति॒ परा॒ भ्रातृ॑व्या॒न् भ्रातृ॑व्या॒न् परा॑ भावयति । \newline
41. परा॑ भावयति भावयति॒ परा॒ परा॑ भावयत्यग्निहो॒त्र म॑ग्निहो॒त्रम् भा॑वयति॒ परा॒ परा॑ भावयत्यग्निहो॒त्रम् । \newline
42. भा॒व॒य॒त्य॒ग्नि॒हो॒त्र म॑ग्निहो॒त्रम् भा॑वयति भावयत्यग्निहो॒त्र मे॒ताभि॑ रे॒ताभि॑ रग्निहो॒त्रम् भा॑वयति भावयत्यग्निहो॒त्र मे॒ताभिः॑ । \newline
43. अ॒ग्नि॒हो॒त्र मे॒ताभि॑ रे॒ताभि॑ रग्निहो॒त्र म॑ग्निहो॒त्र मे॒ताभि॒र् व्याहृ॑तीभि॒र् व्याहृ॑तीभि रे॒ताभि॑ रग्निहो॒त्र म॑ग्निहो॒त्र मे॒ताभि॒र् व्याहृ॑तीभिः । \newline
44. अ॒ग्नि॒हो॒त्रमित्य॑ग्नि - हो॒त्रम् । \newline
45. ए॒ताभि॒र् व्याहृ॑तीभि॒र् व्याहृ॑तीभि रे॒ताभि॑ रे॒ताभि॒र् व्याहृ॑तीभि॒रुपोप॒ व्याहृ॑तीभि रे॒ताभि॑ रे॒ताभि॒र् व्याहृ॑तीभि॒रुप॑ । \newline
46. व्याहृ॑तीभि॒रुपोप॒ व्याहृ॑तीभि॒र् व्याहृ॑तीभि॒रुप॑ सादयेथ् सादये॒दुप॒ व्याहृ॑तीभि॒र् व्याहृ॑तीभि॒रुप॑ सादयेत् । \newline
47. व्याहृ॑तीभि॒रिति॒ व्याहृ॑ति - भिः॒ । \newline
48. उप॑ सादयेथ् सादये॒दुपोप॑ सादयेद् यज्ञ्मु॒खं ॅय॑ज्ञ्मु॒खꣳ सा॑दये॒दुपोप॑ सादयेद् यज्ञ्मु॒खम् । \newline
49. सा॒द॒ये॒द् य॒ज्ञ्॒मु॒खं ॅय॑ज्ञ्मु॒खꣳ सा॑दयेथ् सादयेद् यज्ञ्मु॒खं ॅवै वै य॑ज्ञ्मु॒खꣳ सा॑दयेथ् सादयेद् यज्ञ्मु॒खं ॅवै । \newline
50. य॒ज्ञ्॒मु॒खं ॅवै वै य॑ज्ञ्मु॒खं ॅय॑ज्ञ्मु॒खं ॅवा अ॑ग्निहो॒त्र म॑ग्निहो॒त्रं ॅवै य॑ज्ञ्मु॒खं ॅय॑ज्ञ्मु॒खं ॅवा अ॑ग्निहो॒त्रम् । \newline
51. य॒ज्ञ्॒मु॒खमिति॑ यज्ञ् - मु॒खम् । \newline
52. वा अ॑ग्निहो॒त्र म॑ग्निहो॒त्रं ॅवै वा अ॑ग्निहो॒त्रम् ब्रह्म॒ ब्रह्मा᳚ग्निहो॒त्रं ॅवै वा अ॑ग्निहो॒त्रम् ब्रह्म॑ । \newline
53. अ॒ग्नि॒हो॒त्रम् ब्रह्म॒ ब्रह्मा᳚ग्निहो॒त्र म॑ग्निहो॒त्रम् ब्रह्मै॒ता ए॒ता ब्रह्मा᳚ग्निहो॒त्र म॑ग्निहो॒त्रम् ब्रह्मै॒ताः । \newline
54. अ॒ग्नि॒हो॒त्रमित्य॑ग्नि - हो॒त्रम् । \newline
55. ब्रह्मै॒ता ए॒ता ब्रह्म॒ ब्रह्मै॒ता व्याहृ॑तयो॒ व्याहृ॑तय ए॒ता ब्रह्म॒ ब्रह्मै॒ता व्याहृ॑तयः । \newline
56. ए॒ता व्याहृ॑तयो॒ व्याहृ॑तय ए॒ता ए॒ता व्याहृ॑तयो यज्ञ्मु॒खे य॑ज्ञ्मु॒खे व्याहृ॑तय ए॒ता ए॒ता व्याहृ॑तयो यज्ञ्मु॒खे । \newline
57. व्याहृ॑तयो यज्ञ्मु॒खे य॑ज्ञ्मु॒खे व्याहृ॑तयो॒ व्याहृ॑तयो यज्ञ्मु॒ख ए॒वैव य॑ज्ञ्मु॒खे व्याहृ॑तयो॒ व्याहृ॑तयो यज्ञ्मु॒ख ए॒व । \newline
58. व्याहृ॑तय॒ इति॑ वि - आहृ॑तयः । \newline
59. य॒ज्ञ्॒मु॒ख ए॒वैव य॑ज्ञ्मु॒खे य॑ज्ञ्मु॒ख ए॒व ब्रह्म॒ ब्रह्मै॒व य॑ज्ञ्मु॒खे य॑ज्ञ्मु॒ख ए॒व ब्रह्म॑ । \newline
60. य॒ज्ञ्॒मु॒ख इति॑ यज्ञ् - मु॒खे । \newline
61. ए॒व ब्रह्म॒ ब्रह्मै॒वैव ब्रह्म॑ कुरुते कुरुते॒ ब्रह्मै॒वैव ब्रह्म॑ कुरुते । \newline
62. ब्रह्म॑ कुरुते कुरुते॒ ब्रह्म॒ ब्रह्म॑ कुरुते संॅवथ्स॒रे सं॑ॅवथ्स॒रे कु॑रुते॒ ब्रह्म॒ ब्रह्म॑ कुरुते संॅवथ्स॒रे । \newline
\pagebreak
\markright{ TS 1.6.10.3  \hfill https://www.vedavms.in \hfill}

\section{ TS 1.6.10.3 }

\textbf{TS 1.6.10.3 } \newline
\textbf{Samhita Paata} \newline

कुरुते संॅवथ्स॒रे प॒र्याग॑त ए॒ताभि॑रे॒वोप॑ सादये॒द्-ब्रह्म॑णै॒वोभ॒यतः॑ संॅवथ्स॒रं परि॑ गृह्णाति दर्.शपूर्णमा॒सौ चा॑तुर्मा॒स्यान्या॒लभ॑मान ए॒ताभि॒र् व्याहृ॑तीभिर् ह॒वीꣳष्यासा॑दयेद् यज्ञ्मु॒खं ॅवै द॑र्.शपूर्णमा॒सौ चा॑तुर्मा॒स्यानि॒ ब्रह्मै॒ता व्याहृ॑तयो यज्ञ्मु॒ख ए॒व ब्रह्म॑ कुरुते संॅवथ्स॒रे प॒र्याग॑त ए॒ताभि॑रे॒वासा॑दये॒द् ब्रह्म॑णै॒वोभ॒यतः॑ संॅवथ्स॒रं परि॑गृह्णाति॒ यद्वै य॒ज्ञ्स्य॒ साम्ना᳚ क्रि॒यते॑ रा॒ष्ट्रं - [ ] \newline

\textbf{Pada Paata} \newline

कु॒रु॒ते॒ । सं॒ॅव॒थ्स॒र इति॑ सं - व॒थ्स॒रे । प॒र्याग॑त॒ इति॑ परि - आग॑ते । ए॒ताभिः॑ । ए॒व । उपेति॑ । सा॒द॒ये॒त् । ब्रह्म॑णा । ए॒व । उ॒भ॒यतः॑ । सं॒ॅव॒थ्स॒रमिति॑ सं - व॒थ्स॒रम् । परीति॑ । गृ॒ह्णा॒ति॒ । द॒र्॒.श॒पू॒र्ण॒मा॒साविति॑ दर्.श - पू॒र्ण॒मा॒सौ । चा॒तु॒र्मा॒स्यानीति॑ चातुः - मा॒स्यानि॑ । आ॒लभ॑मान॒ इत्या᳚ - लभ॑मानः । ए॒ताभिः॑ । व्याहृ॑तीभि॒रिति॒ व्याहृ॑ति - भिः॒ । ह॒वीꣳषि॑ । एति॑ । सा॒द॒ये॒त् । य॒ज्ञ्॒मु॒खमिति॑ यज्ञ् - मु॒खम् । वै । द॒र्॒.श॒पू॒र्ण॒मा॒साविति॑ दर्.श - पू॒र्ण॒मा॒सौ । चा॒तु॒र्मा॒स्यानीति॑ चातुः - मा॒स्यानि॑ । ब्रह्म॑ । ए॒ताः । व्याहृ॑तय॒ इति॑ वि - आहृ॑तयः । य॒ज्ञ्॒मु॒ख इति॑ यज्ञ् - मु॒खे । ए॒व । ब्रह्म॑ । कु॒रु॒ते॒ । सं॒ॅव॒थ्स॒र इति॑ सं - व॒थ्स॒रे । प॒र्याग॑त॒ इति॑ परि - आग॑ते । ए॒ताभिः॑ । ए॒व । एति॑ । सा॒द॒ये॒त् । ब्रह्म॑णा । ए॒व । उ॒भ॒यतः॑ । सं॒ॅव॒थ्स॒रमिति॑ सं - व॒थ्स॒रम् । परीति॑ । गृ॒ह्णा॒ति॒ । यत् । वै । य॒ज्ञ्स्य॑ । साम्ना᳚ । क्रि॒यते᳚ । रा॒ष्ट्रम् ।  \newline


\textbf{Krama Paata} \newline

कु॒रु॒ते॒ स॒म्ॅव॒थ्स॒रे । स॒म्ॅव॒थ्स॒रे प॒र्याग॑ते । स॒म्ॅव॒थ्स॒र इति॑ सं - व॒थ्स॒रे । प॒र्याग॑त ए॒ताभिः॑ । प॒र्याग॑त॒ इति॑ परि - आग॑ते । ए॒ताभि॑रे॒व । ए॒वोप॑ । उप॑ सादयेत् । सा॒द॒ये॒द् बह्म॑णा । ब्रह्म॑णै॒व । ए॒वोभ॒यतः॑ । उ॒भ॒यतः॑ सम्ॅवथ्स॒रम् । स॒म्ॅव॒थ्स॒रम् परि॑ । स॒म्ॅव॒थ्स॒रमिति॑ सं - व॒थ्स॒रम् । परि॑ गृह्णाति । गृ॒ह्णा॒ति॒ द॒र्॒.श॒पू॒र्ण॒मा॒सौ । द॒र्.॒श॒पू॒र्ण॒मा॒सौ चा॑तुर्मा॒स्यानि॑ । द॒र्॒.श॒पू॒र्ण॒मा॒साविति॑ दर्.श - पू॒र्ण॒मा॒सौ । चा॒तु॒र्मा॒स्यान्या॒लभ॑मानः । चा॒तु॒र्मा॒स्यानीति॑ चातुः - मा॒स्यानि॑ । आ॒लभ॑मान ए॒ताभिः॑ । आ॒लभ॑मान॒ इत्या᳚ - लभ॑मानः । ए॒ताभि॒र् व्याहृ॑तीभिः । व्याहृ॑तीभिर्. ह॒वीꣳषि॑ । व्याहृ॑तीभि॒रिति॒ व्याहृ॑ति - भिः॒ । ह॒वीꣳष्या । 
आ सा॑दयेत् । सा॒द॒ये॒द् य॒ज्ञ्॒मु॒खम् । य॒ज्ञ्॒मु॒खं ॅवै । य॒ज्ञ्॒मु॒खमिति॑ यज्ञ् - मु॒खम् । वै द॑र्.शपूर्णमा॒सौ । द॒र्.॒श॒पू॒र्ण॒मा॒सौ चा॑तुर्मा॒स्यानि॑ । द॒र्.॒श॒पू॒र्ण॒मा॒साविति॑ दर्.श - पू॒र्ण॒मा॒सौ । चा॒तु॒र्मा॒स्यानि॒ ब्रह्म॑ । चा॒तु॒र्मा॒स्यानीति॑ चातुः - मा॒स्यानि॑ । ब्रह्मै॒ताः । ए॒ता व्याहृ॑तयः । व्याहृ॑तयो यज्ञ्मु॒खे । व्याहृ॑तय॒ इति॑ वि - आहृ॑तयः । य॒ज्ञ्॒मु॒ख ए॒व । य॒ज्ञ्॒मु॒ख इति॑ यज्ञ् - मु॒खे । ए॒व ब्रह्म॑ । ब्रह्म॑ कुरुते । कु॒रु॒ते॒ स॒म्ॅव॒थ्स॒रे । स॒म्ॅव॒थ्स॒रे प॒र्याग॑ते । स॒म्ॅव॒थ्स॒र इति॑ सम् - व॒थ्स॒रे । प॒र्याग॑त ए॒ताभिः॑ । प॒र्याग॑त॒ इति॑ परि - आग॑ते । ए॒ताभि॑रे॒व । ए॒वा । आ सा॑दयेत् । सा॒द॒ये॒द् ब्रह्म॑णा । ब्रह्म॑णै॒व । ए॒वोभ॒यतः॑ । उ॒भ॒यतः॑ सम्ॅवथ्स॒रम् । स॒म्ॅव॒थ्स॒रम् परि॑ । स॒म्ॅव॒थ्स॒रमिति॑ सम् - व॒थ्स॒रम् । परि॑ गृह्णाति । गृ॒ह्णा॒ति॒ यत् । यद् वै । वै य॒ज्ञ्स्य॑ । य॒ज्ञ्स्य॒ साम्ना᳚ । साम्ना᳚ क्रि॒यते᳚ । क्रि॒यते॑ रा॒ष्ट्रम् । रा॒ष्ट्रं ॅय॒ज्ञ्स्य॑ \newline

\textbf{Jatai Paata} \newline

1. कु॒रु॒ते॒ सं॒ॅव॒थ्स॒रे सं॑ॅवथ्स॒रे कु॑रुते कुरुते संॅवथ्स॒रे । \newline
2. सं॒ॅव॒थ्स॒रे प॒र्याग॑ते प॒र्याग॑ते संॅवथ्स॒रे सं॑ॅवथ्स॒रे प॒र्याग॑ते । \newline
3. सं॒ॅव॒थ्स॒र इति॑ सं - व॒थ्स॒रे । \newline
4. प॒र्याग॑त ए॒ताभि॑ रे॒ताभिः॑ प॒र्याग॑ते प॒र्याग॑त ए॒ताभिः॑ । \newline
5. प॒र्याग॑त॒ इति॑ परि - आग॑ते । \newline
6. ए॒ताभि॑ रे॒ वैवैताभि॑ रे॒ताभि॑ रे॒व । \newline
7. ए॒वो पोपै॒वै वोप॑ । \newline
8. उप॑ सादयेथ् सादये॒ दुपोप॑ सादयेत् । \newline
9. सा॒द॒ये॒द् ब्रह्म॑णा॒ ब्रह्म॑णा सादयेथ् सादये॒द् ब्रह्म॑णा । \newline
10. ब्रह्म॑णै॒वैव ब्रह्म॑णा॒ ब्रह्म॑णै॒व । \newline
11. ए॒वोभ॒यत॑ उभ॒यत॑ ए॒वै वोभ॒यतः॑ । \newline
12. उ॒भ॒यतः॑ संॅवथ्स॒रꣳ सं॑ॅवथ्स॒र मु॑भ॒यत॑ उभ॒यतः॑ संॅवथ्स॒रम् । \newline
13. सं॒ॅव॒थ्स॒रम् परि॒ परि॑ संॅवथ्स॒रꣳ सं॑ॅवथ्स॒रम् परि॑ । \newline
14. सं॒ॅव॒थ्स॒रमिति॑ सं - व॒थ्स॒रम् । \newline
15. परि॑ गृह्णाति गृह्णाति॒ परि॒ परि॑ गृह्णाति । \newline
16. गृ॒ह्णा॒ति॒ द॒र्॒.श॒पू॒र्ण॒मा॒सौ द॑र्.शपूर्णमा॒सौ गृ॑ह्णाति गृह्णाति दर्.शपूर्णमा॒सौ । \newline
17. द॒र्॒.श॒पू॒र्ण॒मा॒सौ चा॑तुर्मा॒स्यानि॑ चातुर्मा॒स्यानि॑ दर्.शपूर्णमा॒सौ द॑र्.शपूर्णमा॒सौ चा॑तुर्मा॒स्यानि॑ । \newline
18. द॒र्॒.श॒पू॒र्ण॒मा॒साविति॑ दर्.श - पू॒र्ण॒मा॒सौ । \newline
19. चा॒तु॒र्मा॒स्या न्या॒लभ॑मान आ॒लभ॑मान श्चातुर्मा॒स्यानि॑ चातुर्मा॒स्या न्या॒लभ॑मानः । \newline
20. चा॒तु॒र्मा॒स्यानीति॑ चातुः - मा॒स्यानि॑ । \newline
21. आ॒लभ॑मान ए॒ताभि॑ रे॒ताभि॑ रा॒लभ॑मान आ॒लभ॑मान ए॒ताभिः॑ । \newline
22. आ॒लभ॑मान॒ इत्या᳚ - लभ॑मानः । \newline
23. ए॒ताभि॒र् व्याहृ॑तीभि॒र् व्याहृ॑तीभि रे॒ताभि॑ रे॒ताभि॒र् व्याहृ॑तीभिः । \newline
24. व्याहृ॑तीभिर्. ह॒वीꣳषि॑ ह॒वीꣳषि॒ व्याहृ॑तीभि॒र् व्याहृ॑तीभिर्. ह॒वीꣳषि॑ । \newline
25. व्याहृ॑तीभि॒रिति॒ व्याहृ॑ति - भिः॒ । \newline
26. ह॒वीꣳष्या ह॒वीꣳषि॑ ह॒वीꣳष्या । \newline
27. आ सा॑दयेथ् सादये॒दा सा॑दयेत् । \newline
28. सा॒द॒ये॒द् य॒ज्ञ्॒मु॒खं ॅय॑ज्ञ्मु॒खꣳ सा॑दयेथ् सादयेद् यज्ञ्मु॒खम् । \newline
29. य॒ज्ञ्॒मु॒खं ॅवै वै य॑ज्ञ्मु॒खं ॅय॑ज्ञ्मु॒खं ॅवै । \newline
30. य॒ज्ञ्॒मु॒खमिति॑ यज्ञ् - मु॒खम् । \newline
31. वै द॑र्.शपूर्णमा॒सौ द॑र्.शपूर्णमा॒सौ वै वै द॑र्.शपूर्णमा॒सौ । \newline
32. द॒र्॒.श॒पू॒र्ण॒मा॒सौ चा॑तुर्मा॒स्यानि॑ चातुर्मा॒स्यानि॑ दर्.शपूर्णमा॒सौ द॑र्.शपूर्णमा॒सौ चा॑तुर्मा॒स्यानि॑ । \newline
33. द॒र्॒.श॒पू॒र्ण॒मा॒साविति॑ दर्.श - पू॒र्ण॒मा॒सौ । \newline
34. चा॒तु॒र्मा॒स्यानि॒ ब्रह्म॒ ब्रह्म॑ चातुर्मा॒स्यानि॑ चातुर्मा॒स्यानि॒ ब्रह्म॑ । \newline
35. चा॒तु॒र्मा॒स्यानीति॑ चातुः - मा॒स्यानि॑ । \newline
36. ब्रह्मै॒ता ए॒ता ब्रह्म॒ ब्रह्मै॒ताः । \newline
37. ए॒ता व्याहृ॑तयो॒ व्याहृ॑तय ए॒ता ए॒ता व्याहृ॑तयः । \newline
38. व्याहृ॑तयो यज्ञ्मु॒खे य॑ज्ञ्मु॒खे व्याहृ॑तयो॒ व्याहृ॑तयो यज्ञ्मु॒खे । \newline
39. व्याहृ॑तय॒ इति॑ वि - आहृ॑तयः । \newline
40. य॒ज्ञ्॒मु॒ख ए॒वैव य॑ज्ञ्मु॒खे य॑ज्ञ्मु॒ख ए॒व । \newline
41. य॒ज्ञ्॒मु॒ख इति॑ यज्ञ् - मु॒खे । \newline
42. ए॒व ब्रह्म॒ ब्रह्मै॒वैव ब्रह्म॑ । \newline
43. ब्रह्म॑ कुरुते कुरुते॒ ब्रह्म॒ ब्रह्म॑ कुरुते । \newline
44. कु॒रु॒ते॒ सं॒ॅव॒थ्स॒रे सं॑ॅवथ्स॒रे कु॑रुते कुरुते संॅवथ्स॒रे । \newline
45. सं॒ॅव॒थ्स॒रे प॒र्याग॑ते प॒र्याग॑ते संॅवथ्स॒रे सं॑ॅवथ्स॒रे प॒र्याग॑ते । \newline
46. सं॒ॅव॒थ्स॒र इति॑ सं - व॒थ्स॒रे । \newline
47. प॒र्याग॑त ए॒ताभि॑ रे॒ताभिः॑ प॒र्याग॑ते प॒र्याग॑त ए॒ताभिः॑ । \newline
48. प॒र्याग॑त॒ इति॑ परि - आग॑ते । \newline
49. ए॒ताभि॑ रे॒ वैवैताभि॑ रे॒ताभि॑ रे॒व । \newline
50. ए॒वै वैवा । \newline
51. आ सा॑दयेथ् सादये॒दा सा॑दयेत् । \newline
52. सा॒द॒ये॒द् ब्रह्म॑णा॒ ब्रह्म॑णा सादयेथ् सादये॒द् ब्रह्म॑णा । \newline
53. ब्रह्म॑णै॒ वैव ब्रह्म॑णा॒ ब्रह्म॑णै॒व । \newline
54. ए॒वोभ॒यत॑ उभ॒यत॑ ए॒वै वोभ॒यतः॑ । \newline
55. उ॒भ॒यतः॑ संॅवथ्स॒रꣳ सं॑ॅवथ्स॒र मु॑भ॒यत॑ उभ॒यतः॑ संॅवथ्स॒रम् । \newline
56. सं॒ॅव॒थ्स॒रम् परि॒ परि॑ संॅवथ्स॒रꣳ सं॑ॅवथ्स॒रम् परि॑ । \newline
57. सं॒ॅव॒थ्स॒रमिति॑ सं - व॒थ्स॒रम् । \newline
58. परि॑ गृह्णाति गृह्णाति॒ परि॒ परि॑ गृह्णाति । \newline
59. गृ॒ह्णा॒ति॒ यद् यद् गृ॑ह्णाति गृह्णाति॒ यत् । \newline
60. यद् वै वै यद् यद् वै । \newline
61. वै य॒ज्ञ्स्य॑ य॒ज्ञ्स्य॒ वै वै य॒ज्ञ्स्य॑ । \newline
62. य॒ज्ञ्स्य॒ साम्ना॒ साम्ना॑ य॒ज्ञ्स्य॑ य॒ज्ञ्स्य॒ साम्ना᳚ । \newline
63. साम्ना᳚ क्रि॒यते᳚ क्रि॒यते॒ साम्ना॒ साम्ना᳚ क्रि॒यते᳚ । \newline
64. क्रि॒यते॑ रा॒ष्ट्रꣳ रा॒ष्ट्रम् क्रि॒यते᳚ क्रि॒यते॑ रा॒ष्ट्रम् । \newline
65. रा॒ष्ट्रं ॅय॒ज्ञ्स्य॑ य॒ज्ञ्स्य॑ रा॒ष्ट्रꣳ रा॒ष्ट्रं ॅय॒ज्ञ्स्य॑ । \newline

\textbf{Ghana Paata } \newline

1. कु॒रु॒ते॒ सं॒ॅव॒थ्स॒रे सं॑ॅवथ्स॒रे कु॑रुते कुरुते संॅवथ्स॒रे प॒र्याग॑ते प॒र्याग॑ते संॅवथ्स॒रे कु॑रुते कुरुते संॅवथ्स॒रे प॒र्याग॑ते । \newline
2. सं॒ॅव॒थ्स॒रे प॒र्याग॑ते प॒र्याग॑ते संॅवथ्स॒रे सं॑ॅवथ्स॒रे प॒र्याग॑त ए॒ताभि॑ रे॒ताभिः॑ प॒र्याग॑ते संॅवथ्स॒रे सं॑ॅवथ्स॒रे प॒र्याग॑त ए॒ताभिः॑ । \newline
3. सं॒ॅव॒थ्स॒र इति॑ सं - व॒थ्स॒रे । \newline
4. प॒र्याग॑त ए॒ताभि॑ रे॒ताभिः॑ प॒र्याग॑ते प॒र्याग॑त ए॒ताभि॑ रे॒वैवैताभिः॑ प॒र्याग॑ते प॒र्याग॑त ए॒ताभि॑रे॒व । \newline
5. प॒र्याग॑त॒ इति॑ परि - आग॑ते । \newline
6. ए॒ताभि॑ रे॒वैवैताभि॑ रे॒ताभि॑ रे॒वो पोपै॒ वैताभि॑ रे॒ताभि॑ रे॒वोप॑ । \newline
7. ए॒वो पोपै॒ वैवोप॑ सादयेथ् सादये॒ दुपै॒वैवोप॑ सादयेत् । \newline
8. उप॑ सादयेथ् सादये॒दुपोप॑ सादये॒द् ब्रह्म॑णा॒ ब्रह्म॑णा सादये॒दुपोप॑ सादये॒द् ब्रह्म॑णा । \newline
9. सा॒द॒ये॒द् ब्रह्म॑णा॒ ब्रह्म॑णा सादयेथ् सादये॒द् ब्रह्म॑णै॒वैव ब्रह्म॑णा सादयेथ् सादये॒द् ब्रह्म॑णै॒व । \newline
10. ब्रह्म॑णै॒वैव ब्रह्म॑णा॒ ब्रह्म॑णै॒वोभ॒यत॑ उभ॒यत॑ ए॒व ब्रह्म॑णा॒ ब्रह्म॑णै॒वोभ॒यतः॑ । \newline
11. ए॒वोभ॒यत॑ उभ॒यत॑ ए॒वैवोभ॒यतः॑ संॅवथ्स॒रꣳ सं॑ॅवथ्स॒र मु॑भ॒यत॑ ए॒वैवोभ॒यतः॑ संॅवथ्स॒रम् । \newline
12. उ॒भ॒यतः॑ संॅवथ्स॒रꣳ सं॑ॅवथ्स॒र मु॑भ॒यत॑ उभ॒यतः॑ संॅवथ्स॒रम् परि॒ परि॑ संॅवथ्स॒र मु॑भ॒यत॑ उभ॒यतः॑ संॅवथ्स॒रम् परि॑ । \newline
13. सं॒ॅव॒थ्स॒रम् परि॒ परि॑ संॅवथ्स॒रꣳ सं॑ॅवथ्स॒रम् परि॑ गृह्णाति गृह्णाति॒ परि॑ संॅवथ्स॒रꣳ सं॑ॅवथ्स॒रम् परि॑ गृह्णाति । \newline
14. सं॒ॅव॒थ्स॒रमिति॑ सं - व॒थ्स॒रम् । \newline
15. परि॑ गृह्णाति गृह्णाति॒ परि॒ परि॑ गृह्णाति दर्.शपूर्णमा॒सौ द॑र्.शपूर्णमा॒सौ गृ॑ह्णाति॒ परि॒ परि॑ गृह्णाति दर्.शपूर्णमा॒सौ । \newline
16. गृ॒ह्णा॒ति॒ द॒र्॒.श॒पू॒र्ण॒मा॒सौ द॑र्.शपूर्णमा॒सौ गृ॑ह्णाति गृह्णाति दर्.शपूर्णमा॒सौ चा॑तुर्मा॒स्यानि॑ चातुर्मा॒स्यानि॑ दर्.शपूर्णमा॒सौ गृ॑ह्णाति गृह्णाति दर्.शपूर्णमा॒सौ चा॑तुर्मा॒स्यानि॑ । \newline
17. द॒र्॒.श॒पू॒र्ण॒मा॒सौ चा॑तुर्मा॒स्यानि॑ चातुर्मा॒स्यानि॑ दर्.शपूर्णमा॒सौ द॑र्.शपूर्णमा॒सौ चा॑तुर्मा॒स्या न्या॒लभ॑मान आ॒लभ॑मान श्चातुर्मा॒स्यानि॑ दर्.शपूर्णमा॒सौ द॑र्.शपूर्णमा॒सौ चा॑तुर्मा॒स्या न्या॒लभ॑मानः । \newline
18. द॒र्॒.श॒पू॒र्ण॒मा॒साविति॑ दर्.श - पू॒र्ण॒मा॒सौ । \newline
19. चा॒तु॒र्मा॒स्या न्या॒लभ॑मान आ॒लभ॑मान श्चातुर्मा॒स्यानि॑ चातुर्मा॒स्या न्या॒लभ॑मान ए॒ताभि॑ रे॒ताभि॑ रा॒लभ॑मान श्चातुर्मा॒स्यानि॑ चातुर्मा॒स्या न्या॒लभ॑मान ए॒ताभिः॑ । \newline
20. चा॒तु॒र्मा॒स्यानीति॑ चातुः - मा॒स्यानि॑ । \newline
21. आ॒लभ॑मान ए॒ताभि॑ रे॒ताभि॑ रा॒लभ॑मान आ॒लभ॑मान ए॒ताभि॒र् व्याहृ॑तीभि॒र् व्याहृ॑तीभि रे॒ताभि॑ रा॒लभ॑मान आ॒लभ॑मान ए॒ताभि॒र् व्याहृ॑तीभिः । \newline
22. आ॒लभ॑मान॒ इत्या᳚ - लभ॑मानः । \newline
23. ए॒ताभि॒र् व्याहृ॑तीभि॒र् व्याहृ॑तीभि रे॒ताभि॑ रे॒ताभि॒र् व्याहृ॑तीभिर्. ह॒वीꣳषि॑ ह॒वीꣳषि॒ व्याहृ॑तीभि रे॒ताभि॑ रे॒ताभि॒र् व्याहृ॑तीभिर्. ह॒वीꣳषि॑ । \newline
24. व्याहृ॑तीभिर्. ह॒वीꣳषि॑ ह॒वीꣳषि॒ व्याहृ॑तीभि॒र् व्याहृ॑तीभिर्. ह॒वीꣳष्या ह॒वीꣳषि॒ व्याहृ॑तीभि॒र् व्याहृ॑तीभिर्. ह॒वीꣳष्या । \newline
25. व्याहृ॑तीभि॒रिति॒ व्याहृ॑ति - भिः॒ । \newline
26. ह॒वीꣳष्या ह॒वीꣳषि॑ ह॒वीꣳष्या सा॑दयेथ् सादये॒दा ह॒वीꣳषि॑ ह॒वीꣳष्या सा॑दयेत् । \newline
27. आ सा॑दयेथ् सादये॒दा सा॑दयेद् यज्ञ्मु॒खं ॅय॑ज्ञ्मु॒खꣳ सा॑दये॒दा सा॑दयेद् यज्ञ्मु॒खम् । \newline
28. सा॒द॒ये॒द् य॒ज्ञ्॒मु॒खं ॅय॑ज्ञ्मु॒खꣳ सा॑दयेथ् सादयेद् यज्ञ्मु॒खं ॅवै वै य॑ज्ञ्मु॒खꣳ सा॑दयेथ् सादयेद् यज्ञ्मु॒खं ॅवै । \newline
29. य॒ज्ञ्॒मु॒खं ॅवै वै य॑ज्ञ्मु॒खं ॅय॑ज्ञ्मु॒खं ॅवै द॑र्.शपूर्णमा॒सौ द॑र्.शपूर्णमा॒सौ वै य॑ज्ञ्मु॒खं ॅय॑ज्ञ्मु॒खं ॅवै द॑र्.शपूर्णमा॒सौ । \newline
30. य॒ज्ञ्॒मु॒खमिति॑ यज्ञ् - मु॒खम् । \newline
31. वै द॑र्.शपूर्णमा॒सौ द॑र्.शपूर्णमा॒सौ वै वै द॑र्.शपूर्णमा॒सौ चा॑तुर्मा॒स्यानि॑ चातुर्मा॒स्यानि॑ दर्.शपूर्णमा॒सौ वै वै द॑र्.शपूर्णमा॒सौ चा॑तुर्मा॒स्यानि॑ । \newline
32. द॒र्॒.श॒पू॒र्ण॒मा॒सौ चा॑तुर्मा॒स्यानि॑ चातुर्मा॒स्यानि॑ दर्.शपूर्णमा॒सौ द॑र्.शपूर्णमा॒सौ चा॑तुर्मा॒स्यानि॒ ब्रह्म॒ ब्रह्म॑ चातुर्मा॒स्यानि॑ दर्.शपूर्णमा॒सौ द॑र्.शपूर्णमा॒सौ चा॑तुर्मा॒स्यानि॒ ब्रह्म॑ । \newline
33. द॒र्॒.श॒पू॒र्ण॒मा॒साविति॑ दर्.श - पू॒र्ण॒मा॒सौ । \newline
34. चा॒तु॒र्मा॒स्यानि॒ ब्रह्म॒ ब्रह्म॑ चातुर्मा॒स्यानि॑ चातुर्मा॒स्यानि॒ ब्रह्मै॒ता ए॒ता ब्रह्म॑ चातुर्मा॒स्यानि॑ चातुर्मा॒स्यानि॒ ब्रह्मै॒ताः । \newline
35. चा॒तु॒र्मा॒स्यानीति॑ चातुः - मा॒स्यानि॑ । \newline
36. ब्रह्मै॒ता ए॒ता ब्रह्म॒ ब्रह्मै॒ता व्याहृ॑तयो॒ व्याहृ॑तय ए॒ता ब्रह्म॒ ब्रह्मै॒ता व्याहृ॑तयः । \newline
37. ए॒ता व्याहृ॑तयो॒ व्याहृ॑तय ए॒ता ए॒ता व्याहृ॑तयो यज्ञ्मु॒खे य॑ज्ञ्मु॒खे व्याहृ॑तय ए॒ता ए॒ता व्याहृ॑तयो यज्ञ्मु॒खे । \newline
38. व्याहृ॑तयो यज्ञ्मु॒खे य॑ज्ञ्मु॒खे व्याहृ॑तयो॒ व्याहृ॑तयो यज्ञ्मु॒ख ए॒वैव य॑ज्ञ्मु॒खे व्याहृ॑तयो॒ व्याहृ॑तयो यज्ञ्मु॒ख ए॒व । \newline
39. व्याहृ॑तय॒ इति॑ वि - आहृ॑तयः । \newline
40. य॒ज्ञ्॒मु॒ख ए॒वैव य॑ज्ञ्मु॒खे य॑ज्ञ्मु॒ख ए॒व ब्रह्म॒ ब्रह्मै॒व य॑ज्ञ्मु॒खे य॑ज्ञ्मु॒ख ए॒व ब्रह्म॑ । \newline
41. य॒ज्ञ्॒मु॒ख इति॑ यज्ञ् - मु॒खे । \newline
42. ए॒व ब्रह्म॒ ब्रह्मै॒वैव ब्रह्म॑ कुरुते कुरुते॒ ब्रह्मै॒वैव ब्रह्म॑ कुरुते । \newline
43. ब्रह्म॑ कुरुते कुरुते॒ ब्रह्म॒ ब्रह्म॑ कुरुते संॅवथ्स॒रे सं॑ॅवथ्स॒रे कु॑रुते॒ ब्रह्म॒ ब्रह्म॑ कुरुते संॅवथ्स॒रे । \newline
44. कु॒रु॒ते॒ सं॒ॅव॒थ्स॒रे सं॑ॅवथ्स॒रे कु॑रुते कुरुते संॅवथ्स॒रे प॒र्याग॑ते प॒र्याग॑ते संॅवथ्स॒रे कु॑रुते कुरुते संॅवथ्स॒रे प॒र्याग॑ते । \newline
45. सं॒ॅव॒थ्स॒रे प॒र्याग॑ते प॒र्याग॑ते संॅवथ्स॒रे सं॑ॅवथ्स॒रे प॒र्याग॑त ए॒ताभि॑ रे॒ताभिः॑ प॒र्याग॑ते संॅवथ्स॒रे सं॑ॅवथ्स॒रे प॒र्याग॑त ए॒ताभिः॑ । \newline
46. सं॒ॅव॒थ्स॒र इति॑ सं - व॒थ्स॒रे । \newline
47. प॒र्याग॑त ए॒ताभि॑ रे॒ताभिः॑ प॒र्याग॑ते प॒र्याग॑त ए॒ताभि॑ रे॒वैवैताभिः॑ प॒र्याग॑ते प॒र्याग॑त ए॒ताभि॑रे॒व । \newline
48. प॒र्याग॑त॒ इति॑ परि - आग॑ते । \newline
49. ए॒ताभि॑ रे॒वैवैताभि॑ रे॒ताभि॑ रे॒वैवैताभि॑ रे॒ताभि॑ रे॒वा । \newline
50. ए॒वैवैवा सा॑दयेथ् सादये॒दैवैवा सा॑दयेत् । \newline
51. आ सा॑दयेथ् सादये॒दा सा॑दये॒द् ब्रह्म॑णा॒ ब्रह्म॑णा सादये॒दा सा॑दये॒द् ब्रह्म॑णा । \newline
52. सा॒द॒ये॒द् ब्रह्म॑णा॒ ब्रह्म॑णा सादयेथ् सादये॒द् ब्रह्म॑णै॒वैव ब्रह्म॑णा सादयेथ् सादये॒द् ब्रह्म॑णै॒व । \newline
53. ब्रह्म॑णै॒वैव ब्रह्म॑णा॒ ब्रह्म॑णै॒वोभ॒यत॑ उभ॒यत॑ ए॒व ब्रह्म॑णा॒ ब्रह्म॑णै॒वोभ॒यतः॑ । \newline
54. ए॒वोभ॒यत॑ उभ॒यत॑ ए॒वैवोभ॒यतः॑ संॅवथ्स॒रꣳ सं॑ॅवथ्स॒र मु॑भ॒यत॑ ए॒वैवोभ॒यतः॑ संॅवथ्स॒रम् । \newline
55. उ॒भ॒यतः॑ संॅवथ्स॒रꣳ सं॑ॅवथ्स॒र मु॑भ॒यत॑ उभ॒यतः॑ संॅवथ्स॒रम् परि॒ परि॑ संॅवथ्स॒र मु॑भ॒यत॑ उभ॒यतः॑ संॅवथ्स॒रम् परि॑ । \newline
56. सं॒ॅव॒थ्स॒रम् परि॒ परि॑ संॅवथ्स॒रꣳ सं॑ॅवथ्स॒रम् परि॑ गृह्णाति गृह्णाति॒ परि॑ संॅवथ्स॒रꣳ सं॑ॅवथ्स॒रम् परि॑ गृह्णाति । \newline
57. सं॒ॅव॒थ्स॒रमिति॑ सं - व॒थ्स॒रम् । \newline
58. परि॑ गृह्णाति गृह्णाति॒ परि॒ परि॑ गृह्णाति॒ यद् यद् गृ॑ह्णाति॒ परि॒ परि॑ गृह्णाति॒ यत् । \newline
59. गृ॒ह्णा॒ति॒ यद् यद् गृ॑ह्णाति गृह्णाति॒ यद् वै वै यद् गृ॑ह्णाति गृह्णाति॒ यद् वै । \newline
60. यद् वै वै यद् यद् वै य॒ज्ञ्स्य॑ य॒ज्ञ्स्य॒ वै यद् यद् वै य॒ज्ञ्स्य॑ । \newline
61. वै य॒ज्ञ्स्य॑ य॒ज्ञ्स्य॒ वै वै य॒ज्ञ्स्य॒ साम्ना॒ साम्ना॑ य॒ज्ञ्स्य॒ वै वै य॒ज्ञ्स्य॒ साम्ना᳚ । \newline
62. य॒ज्ञ्स्य॒ साम्ना॒ साम्ना॑ य॒ज्ञ्स्य॑ य॒ज्ञ्स्य॒ साम्ना᳚ क्रि॒यते᳚ क्रि॒यते॒ साम्ना॑ य॒ज्ञ्स्य॑ य॒ज्ञ्स्य॒ साम्ना᳚ क्रि॒यते᳚ । \newline
63. साम्ना᳚ क्रि॒यते᳚ क्रि॒यते॒ साम्ना॒ साम्ना᳚ क्रि॒यते॑ रा॒ष्ट्रꣳ रा॒ष्ट्रम् क्रि॒यते॒ साम्ना॒ साम्ना᳚ क्रि॒यते॑ रा॒ष्ट्रम् । \newline
64. क्रि॒यते॑ रा॒ष्ट्रꣳ रा॒ष्ट्रम् क्रि॒यते᳚ क्रि॒यते॑ रा॒ष्ट्रं ॅय॒ज्ञ्स्य॑ य॒ज्ञ्स्य॑ रा॒ष्ट्रम् क्रि॒यते᳚ क्रि॒यते॑ रा॒ष्ट्रं ॅय॒ज्ञ्स्य॑ । \newline
65. रा॒ष्ट्रं ॅय॒ज्ञ्स्य॑ य॒ज्ञ्स्य॑ रा॒ष्ट्रꣳ रा॒ष्ट्रं ॅय॒ज्ञ्स्या॒शी रा॒शीर् य॒ज्ञ्स्य॑ रा॒ष्ट्रꣳ रा॒ष्ट्रं ॅय॒ज्ञ्स्या॒शीः । \newline
\pagebreak
\markright{ TS 1.6.10.4  \hfill https://www.vedavms.in \hfill}

\section{ TS 1.6.10.4 }

\textbf{TS 1.6.10.4 } \newline
\textbf{Samhita Paata} \newline

ॅय॒ज्ञ्स्या॒-*शीर्ग॑च्छति॒ यदृ॒चा विशं॑ ॅय॒ज्ञ्स्या॒- *शीर्ग॑च्छ॒त्यथ॑ ब्राह्म॒णो॑ऽना॒शीर्के॑ण य॒ज्ञेन॑ यजते सामिधे॒नी-र॑नुव॒क्ष्यन्ने॒ता व्याहृ॑तीः पु॒रस्ता᳚द्दद्ध्या॒द् ब्रह्मै॒व प्र॑ति॒पदं॑ कुरुते॒ तथा᳚ ब्राह्म॒णः साशी᳚र्केण य॒ज्ञेन॑ यजते॒ यं का॒मये॑त॒ यज॑मानं॒ भ्रातृ॑व्यमस्य य॒ज्ञ्स्या॒*शीर्ग॑च्छे॒दिति॒ तस्यै॒ता व्याहृ॑तीः पुरोनुवा॒क्या॑यां दद्ध्याद् भ्रातृव्यदेव॒त्या॑ वै पु॑रोनुवा॒क्या᳚ भ्रातृ॑व्यमे॒वास्य॑ य॒ज्ञ्स्या॒-[ ] \newline

\textbf{Pada Paata} \newline

य॒ज्ञ्स्य॑ । आ॒शीरित्या᳚ - शीः । ग॒च्छ॒ति॒ । यत् । ऋ॒चा । विश᳚म् । य॒ज्ञ्स्य॑ । आ॒शीरित्या᳚ - शीः । ग॒च्छ॒ति॒ । अथ॑ । ब्रा॒ह्म॒णः । अ॒ना॒शीर्के॑ण । य॒ज्ञेन॑ । य॒ज॒ते॒ । सा॒मि॒धे॒नीरिति॑ साम् - इ॒धे॒नीः । अ॒नु॒व॒क्ष्यन्नित्य॑नु - व॒क्ष्यन्न् । ए॒ताः । व्याहृ॑ती॒रिति॑ वि - आहृ॑तीः । पु॒रस्ता᳚त् । द॒द्ध्या॒त् । ब्रह्म॑ । ए॒व । प्र॒ति॒पद॒मिति॑ प्रति-पद᳚म् । कु॒रु॒ते॒ । तथा᳚ । ब्रा॒ह्म॒णः । साशी᳚र्के॒णेति॒ स -आ॒शी॒र्के॒ण॒ । य॒ज्ञेन॑ । य॒ज॒ते॒ । यम् । का॒मये॑त । यज॑मानम् । भ्रातृ॑व्यम् । अ॒स्य॒ । य॒ज्ञ्स्य॑ । आ॒शीरित्या᳚ - शीः । ग॒च्छे॒त् । इति॑ । तस्य॑ । ए॒ताः । व्याहृ॑ती॒रिति॑ वि - आहृ॑तीः । पु॒रो॒नु॒वा॒क्या॑या॒मिति॑ पुरः - अ॒नु॒वा॒क्या॑याम् । द॒द्ध्या॒त् । भ्रा॒तृ॒व्य॒दे॒व॒त्येति॑ भ्रातृव्य - दे॒व॒त्या᳚ । वै । पु॒रो॒नु॒वा॒क्येति॑ पुरः - अ॒नु॒वा॒क्या᳚ । भ्रातृ॑व्यम् । ए॒व । अ॒स्य॒ । य॒ज्ञ्स्य॑ ।  \newline


\textbf{Krama Paata} \newline

य॒ज्ञ्स्या॒शीः । आ॒शीर्,ग॑च्छति । आ॒शीरित्या᳚ - शीः । ग॒च्छ॒ति॒ यत् । यदृ॒चा । ऋ॒चा विश᳚म् । विशं॑ ॅय॒ज्ञ्स्य॑ । य॒ज्ञ्स्या॒शीः । आ॒शीर्,ग॑च्छति । आ॒शीरित्या᳚ - शीः । ग॒च्छ॒त्यथ॑ । अथ॑ ब्राह्म॒णः । ब्रा॒ह्म॒णो॑ ऽना॒शीर्के॑ण । अ॒ना॒शीर्के॑ण य॒ज्ञेन॑ । य॒ज्ञेन॑ यजते । य॒ज॒ते॒ सा॒मि॒धे॒नीः । सा॒मि॒धे॒नीर॑नुव॒क्ष्यन्न् । सा॒मि॒धे॒नीरिति॑ साम् - इ॒धे॒नीः । अ॒नु॒व॒क्ष्यन्ने॒ताः । अ॒नु॒व॒क्ष्यन्नित्य॑नु - व॒क्ष्यन्न् । ए॒ता व्याहृ॑तीः । व्याहृ॑तीः पु॒रस्ता᳚त् । व्याहृ॑ती॒रिति॑ वि - आहृ॑तीः । पु॒रस्ता᳚द् दद्ध्यात् । द॒द्ध्या॒द्,ब्रह्म॑ । ब्रह्मै॒व । ए॒व प्र॑ति॒पद᳚म् । प्र॒ति॒पद॑म् कुरुते । प्र॒ति॒पद॒मिति॑ प्रति - पद᳚म् । कु॒रु॒ते॒ तथा᳚ । तथा᳚ ब्राह्म॒णः । ब्रा॒ह्म॒णः साशी᳚र्केण । साशी᳚र्केण य॒ज्ञेन॑ । साशी᳚र्के॒णेति॒ स - आ॒शी॒र्के॒ण॒ । य॒ज्ञेन॑ यजते । य॒ज॒ते॒ यम् । यम् का॒मये॑त । का॒मये॑त॒ यज॑मानम् । यज॑मान॒म् भातृ॑व्यम् । भ्रातृ॑व्यमस्य । अ॒स्य॒ य॒ज्ञ्स्य॑ । य॒ज्ञ्स्या॒शीः । आ॒शीर् ग॑च्छेत् । आ॒शीरित्या᳚ - शीः । ग॒च्छे॒दिति॑ । इति॒ तस्य॑ । तस्यै॒ताः । ए॒ता व्याहृ॑तीः । व्याहृ॑तीः पुरोनुवा॒क्या॑याम् । व्याहृ॑ती॒रिति॑ वि - आहृ॑तीः । पु॒रो॒नु॒वा॒क्या॑याम् दध्यात् । पु॒रो॒नु॒वा॒क्या॑या॒मिति॑ पुरः - अ॒नु॒वा॒क्या॑याम् । द॒ध्या॒द् भ्रा॒तृ॒व्य॒दे॒व॒त्या᳚ । भ्रा॒तृ॒व्य॒दे॒व॒त्या॑ वै । भ्रा॒तृ॒व्य॒दे॒व॒त्येति॑ भ्रातृव्य - दे॒व॒त्या᳚ । वै पु॑रोनुवा॒क्या᳚ । पु॒रो॒नु॒वा॒क्या᳚ भ्रातृ॑व्यम् । पु॒रो॒नु॒वा॒क्येति॑ पुरः - अ॒नु॒वा॒क्या᳚ । भ्रातृ॑व्यमे॒व । ए॒वास्य॑ । अ॒स्य॒ य॒ज्ञ्स्य॑ । य॒ज्ञ्स्या॒शीः \newline

\textbf{Jatai Paata} \newline

1. य॒ज्ञ्स्या॒शी रा॒शीर् य॒ज्ञ्स्य॑ य॒ज्ञ्स्या॒शीः । \newline
2. आ॒शीर् ग॑च्छति गच्छ त्या॒शी रा॒शीर् ग॑च्छति । \newline
3. आ॒शीरित्या᳚ - शीः । \newline
4. ग॒च्छ॒ति॒ यद् यद् ग॑च्छति गच्छति॒ यत् । \newline
5. यदृ॒चर्चा यद् यदृ॒चा । \newline
6. ऋ॒चा विशं॒ ॅविश॑ मृ॒चर्चा विश᳚म् । \newline
7. विशं॑ ॅय॒ज्ञ्स्य॑ य॒ज्ञ्स्य॒ विशं॒ ॅविशं॑ ॅय॒ज्ञ्स्य॑ । \newline
8. य॒ज्ञ्स्या॒शी रा॒शीर् य॒ज्ञ्स्य॑ य॒ज्ञ्स्या॒शीः । \newline
9. आ॒शीर् ग॑च्छति गच्छ त्या॒शी रा॒शीर् ग॑च्छति । \newline
10. आ॒शीरित्या᳚ - शीः । \newline
11. ग॒च्छ॒ त्यथाथ॑ गच्छति गच्छ॒ त्यथ॑ । \newline
12. अथ॑ ब्राह्म॒णो ब्रा᳚ह्म॒णो ऽथाथ॑ ब्राह्म॒णः । \newline
13. ब्रा॒ह्म॒णो॑ ऽना॒शीर्के॑णा ना॒शीर्के॑ण ब्राह्म॒णो ब्रा᳚ह्म॒णो॑ ऽना॒शीर्के॑ण । \newline
14. अ॒ना॒शीर्के॑ण य॒ज्ञेन॑ य॒ज्ञेना॑ ना॒शीर्के॑णा ना॒शीर्के॑ण य॒ज्ञेन॑ । \newline
15. य॒ज्ञेन॑ यजते यजते य॒ज्ञेन॑ य॒ज्ञेन॑ यजते । \newline
16. य॒ज॒ते॒ सा॒मि॒धे॒नीः सा॑मिधे॒नीर् य॑जते यजते सामिधे॒नीः । \newline
17. सा॒मि॒धे॒नी र॑नुव॒क्ष्यन् न॑नुव॒क्ष्यन् थ्सा॑मिधे॒नीः सा॑मिधे॒नी र॑नुव॒क्ष्यन्न् । \newline
18. सा॒मि॒धे॒नीरिति॑ साम् - इ॒धे॒नीः । \newline
19. अ॒नु॒व॒क्ष्यन् ने॒ता ए॒ता अ॑नुव॒क्ष्यन् न॑नुव॒क्ष्यन् ने॒ताः । \newline
20. अ॒नु॒व॒क्ष्यन्नित्य॑नु - व॒क्ष्यन्न् । \newline
21. ए॒ता व्याहृ॑ती॒र् व्याहृ॑तीरे॒ता ए॒ता व्याहृ॑तीः । \newline
22. व्याहृ॑तीः पु॒रस्ता᳚त् पु॒रस्ता॒द् व्याहृ॑ती॒र् व्याहृ॑तीः पु॒रस्ता᳚त् । \newline
23. व्याहृ॑ती॒रिति॑ वि - आहृ॑तीः । \newline
24. पु॒रस्ता᳚द् दद्ध्याद् दद्ध्यात् पु॒रस्ता᳚त् पु॒रस्ता᳚द् दद्ध्यात् । \newline
25. द॒द्ध्या॒द् ब्रह्म॒ ब्रह्म॑ दद्ध्याद् दद्ध्या॒द् ब्रह्म॑ । \newline
26. ब्रह्मै॒ वैव ब्रह्म॒ ब्रह्मै॒व । \newline
27. ए॒व प्र॑ति॒पद॑म् प्रति॒पद॑ मे॒वैव प्र॑ति॒पद᳚म् । \newline
28. प्र॒ति॒पद॑म् कुरुते कुरुते प्रति॒पद॑म् प्रति॒पद॑म् कुरुते । \newline
29. प्र॒ति॒पद॒मिति॑ प्रति - पद᳚म् । \newline
30. कु॒रु॒ते॒ तथा॒ तथा॑ कुरुते कुरुते॒ तथा᳚ । \newline
31. तथा᳚ ब्राह्म॒णो ब्रा᳚ह्म॒ण स्तथा॒ तथा᳚ ब्राह्म॒णः । \newline
32. ब्रा॒ह्म॒णः साशी᳚र्केण॒ साशी᳚र्केण ब्राह्म॒णो ब्रा᳚ह्म॒णः साशी᳚र्केण । \newline
33. साशी᳚र्केण य॒ज्ञेन॑ य॒ज्ञेन॒ साशी᳚र्केण॒ साशी᳚र्केण य॒ज्ञेन॑ । \newline
34. साशी᳚र्के॒णेति॒ स - आ॒शी॒र्के॒ण॒ । \newline
35. य॒ज्ञेन॑ यजते यजते य॒ज्ञेन॑ य॒ज्ञेन॑ यजते । \newline
36. य॒ज॒ते॒ यं ॅयं ॅय॑जते यजते॒ यम् । \newline
37. यम् का॒मये॑त का॒मये॑त॒ यं ॅयम् का॒मये॑त । \newline
38. का॒मये॑त॒ यज॑मानं॒ ॅयज॑मानम् का॒मये॑त का॒मये॑त॒ यज॑मानम् । \newline
39. यज॑मान॒म् भ्रातृ॑व्य॒म् भ्रातृ॑व्यं॒ ॅयज॑मानं॒ ॅयज॑मान॒म् भ्रातृ॑व्यम् । \newline
40. भ्रातृ॑व्य मस्यास्य॒ भ्रातृ॑व्य॒म् भ्रातृ॑व्य मस्य । \newline
41. अ॒स्य॒ य॒ज्ञ्स्य॑ य॒ज्ञ् स्या᳚स्यास्य य॒ज्ञ्स्य॑ । \newline
42. य॒ज्ञ्स्या॒शी रा॒शीर् य॒ज्ञ्स्य॑ य॒ज्ञ्स्या॒शीः । \newline
43. आ॒शीर् ग॑च्छेद् गच्छे दा॒शी रा॒शीर् ग॑च्छेत् । \newline
44. आ॒शीरित्या᳚ - शीः । \newline
45. ग॒च्छे॒ दितीति॑ गच्छेद् गच्छे॒ दिति॑ । \newline
46. इति॒ तस्य॒ तस्ये तीति॒ तस्य॑ । \newline
47. तस्यै॒ता ए॒ता स्तस्य॒ तस्यै॒ताः । \newline
48. ए॒ता व्याहृ॑ती॒र् व्याहृ॑तीरे॒ता ए॒ता व्याहृ॑तीः । \newline
49. व्याहृ॑तीः पुरोनुवा॒क्या॑याम् पुरोनुवा॒क्या॑यां॒ ॅव्याहृ॑ती॒र् व्याहृ॑तीः पुरोनुवा॒क्या॑याम् । \newline
50. व्याहृ॑ती॒रिति॑ वि - आहृ॑तीः । \newline
51. पु॒रो॒नु॒वा॒क्या॑याम् दद्ध्याद् दद्ध्यात् पुरोनुवा॒क्या॑याम् पुरोनुवा॒क्या॑याम् दद्ध्यात् । \newline
52. पु॒रो॒नु॒वा॒क्या॑या॒मिति॑ पुरः - अ॒नु॒वा॒क्या॑याम् । \newline
53. द॒द्ध्या॒द् भ्रा॒तृ॒व्य॒दे॒व॒त्या᳚ भ्रातृव्यदेव॒त्या॑ दद्ध्याद् दद्ध्याद् भ्रातृव्यदेव॒त्या᳚ । \newline
54. भ्रा॒तृ॒व्य॒दे॒व॒त्या॑ वै वै भ्रा॑तृव्यदेव॒त्या᳚ भ्रातृव्यदेव॒त्या॑ वै । \newline
55. भ्रा॒तृ॒व्य॒दे॒व॒त्येति॑ भ्रातृव्य - दे॒व॒त्या᳚ । \newline
56. वै पु॑रोनुवा॒क्या॑ पुरोनुवा॒क्या॑ वै वै पु॑रोनुवा॒क्या᳚ । \newline
57. पु॒रो॒नु॒वा॒क्या᳚ भ्रातृ॑व्य॒म् भ्रातृ॑व्यम् पुरोनुवा॒क्या॑ पुरोनुवा॒क्या᳚ भ्रातृ॑व्यम् । \newline
58. पु॒रो॒नु॒वा॒क्येति॑ पुरः - अ॒नु॒वा॒क्या᳚ । \newline
59. भ्रातृ॑व्य मे॒वैव भ्रातृ॑व्य॒म् भ्रातृ॑व्य मे॒व । \newline
60. ए॒वास्या᳚ स्यै॒ वैवास्य॑ । \newline
61. अ॒स्य॒ य॒ज्ञ्स्य॑ य॒ज्ञ् स्या᳚स्यास्य य॒ज्ञ्स्य॑ । \newline
62. य॒ज्ञ्स्या॒शी रा॒शीर् य॒ज्ञ्स्य॑ य॒ज्ञ्स्या॒शीः । \newline

\textbf{Ghana Paata } \newline

1. य॒ज्ञ्स्या॒शी रा॒शीर् य॒ज्ञ्स्य॑ य॒ज्ञ्स्या॒शीर् ग॑च्छति गच्छत्या॒शीर् य॒ज्ञ्स्य॑ य॒ज्ञ्स्या॒शीर् ग॑च्छति । \newline
2. आ॒शीर् ग॑च्छति गच्छत्या॒शी रा॒शीर् ग॑च्छति॒ यद् यद् ग॑च्छत्या॒शी रा॒शीर् ग॑च्छति॒ यत् । \newline
3. आ॒शीरित्या᳚ - शीः । \newline
4. ग॒च्छ॒ति॒ यद् यद् ग॑च्छति गच्छति॒ यदृ॒चर्चा यद् ग॑च्छति गच्छति॒ यदृ॒चा । \newline
5. यदृ॒चर्चा यद् यदृ॒चा विशं॒ ॅविश॑ मृ॒चा यद् यदृ॒चा विश᳚म् । \newline
6. ऋ॒चा विशं॒ ॅविश॑ मृ॒चर्चा विशं॑ ॅय॒ज्ञ्स्य॑ य॒ज्ञ्स्य॒ विश॑ मृ॒चर्चा विशं॑ ॅय॒ज्ञ्स्य॑ । \newline
7. विशं॑ ॅय॒ज्ञ्स्य॑ य॒ज्ञ्स्य॒ विशं॒ ॅविशं॑ ॅय॒ज्ञ्स्या॒शी रा॒शीर् य॒ज्ञ्स्य॒ विशं॒ ॅविशं॑ ॅय॒ज्ञ्स्या॒शीः । \newline
8. य॒ज्ञ्स्या॒शी रा॒शीर् य॒ज्ञ्स्य॑ य॒ज्ञ्स्या॒शीर् ग॑च्छति गच्छत्या॒शीर् य॒ज्ञ्स्य॑ य॒ज्ञ्स्या॒शीर् ग॑च्छति । \newline
9. आ॒शीर् ग॑च्छति गच्छत्या॒शी रा॒शीर् ग॑च्छ॒त्यथाथ॑ गच्छत्या॒शी रा॒शीर् ग॑च्छ॒त्यथ॑ । \newline
10. आ॒शीरित्या᳚ - शीः । \newline
11. ग॒च्छ॒त्यथाथ॑ गच्छति गच्छ॒त्यथ॑ ब्राह्म॒णो ब्रा᳚ह्म॒णो ऽथ॑ गच्छति गच्छ॒त्यथ॑ ब्राह्म॒णः । \newline
12. अथ॑ ब्राह्म॒णो ब्रा᳚ह्म॒णो ऽथाथ॑ ब्राह्म॒णो॑ ऽना॒शीर्के॑णाना॒शीर्के॑ण ब्राह्म॒णो ऽथाथ॑ ब्राह्म॒णो॑ ऽना॒शीर्के॑ण । \newline
13. ब्रा॒ह्म॒णो॑ ऽना॒शीर्के॑णाना॒शीर्के॑ण ब्राह्म॒णो ब्रा᳚ह्म॒णो॑ ऽना॒शीर्के॑ण य॒ज्ञेन॑ य॒ज्ञेना॑ना॒शीर्के॑ण ब्राह्म॒णो ब्रा᳚ह्म॒णो॑ ऽना॒शीर्के॑ण य॒ज्ञेन॑ । \newline
14. अ॒ना॒शीर्के॑ण य॒ज्ञेन॑ य॒ज्ञे ना॑ना॒शीर्के॑णा ना॒शीर्के॑ण य॒ज्ञेन॑ यजते यजते य॒ज्ञे ना॑ना॒शीर्के॑णा ना॒शीर्के॑ण य॒ज्ञेन॑ यजते । \newline
15. य॒ज्ञेन॑ यजते यजते य॒ज्ञेन॑ य॒ज्ञेन॑ यजते सामिधे॒नीः सा॑मिधे॒नीर् य॑जते य॒ज्ञेन॑ य॒ज्ञेन॑ यजते सामिधे॒नीः । \newline
16. य॒ज॒ते॒ सा॒मि॒धे॒नीः सा॑मिधे॒नीर् य॑जते यजते सामिधे॒नी र॑नुव॒क्ष्यन् न॑नुव॒क्ष्यन् थ्सा॑मिधे॒नीर् य॑जते यजते सामिधे॒नी र॑नुव॒क्ष्यन्न् । \newline
17. सा॒मि॒धे॒नी र॑नुव॒क्ष्यन् न॑नुव॒क्ष्यन् थ्सा॑मिधे॒नीः सा॑मिधे॒नी र॑नुव॒क्ष्यन् ने॒ता ए॒ता अ॑नुव॒क्ष्यन् थ्सा॑मिधे॒नीः सा॑मिधे॒नी र॑नुव॒क्ष्यन् ने॒ताः । \newline
18. सा॒मि॒धे॒नीरिति॑ साम् - इ॒धे॒नीः । \newline
19. अ॒नु॒व॒क्ष्यन् ने॒ता ए॒ता अ॑नुव॒क्ष्यन् न॑नुव॒क्ष्यन् ने॒ता व्याहृ॑ती॒र् व्याहृ॑तीरे॒ता अ॑नुव॒क्ष्यन् न॑नुव॒क्ष्यन् ने॒ता व्याहृ॑तीः । \newline
20. अ॒नु॒व॒क्ष्यन्नित्य॑नु - व॒क्ष्यन्न् । \newline
21. ए॒ता व्याहृ॑ती॒र् व्याहृ॑तीरे॒ता ए॒ता व्याहृ॑तीः पु॒रस्ता᳚त् पु॒रस्ता॒द् व्याहृ॑तीरे॒ता ए॒ता व्याहृ॑तीः पु॒रस्ता᳚त् । \newline
22. व्याहृ॑तीः पु॒रस्ता᳚त् पु॒रस्ता॒द् व्याहृ॑ती॒र् व्याहृ॑तीः पु॒रस्ता᳚द् दद्ध्याद् दद्ध्यात् पु॒रस्ता॒द् व्याहृ॑ती॒र् व्याहृ॑तीः पु॒रस्ता᳚द् दद्ध्यात् । \newline
23. व्याहृ॑ती॒रिति॑ वि - आहृ॑तीः । \newline
24. पु॒रस्ता᳚द् दद्ध्याद् दद्ध्यात् पु॒रस्ता᳚त् पु॒रस्ता᳚द् दद्ध्या॒द् ब्रह्म॒ ब्रह्म॑ दद्ध्यात् पु॒रस्ता᳚त् पु॒रस्ता᳚द् दद्ध्या॒द् ब्रह्म॑ । \newline
25. द॒द्ध्या॒द् ब्रह्म॒ ब्रह्म॑ दद्ध्याद् दद्ध्या॒द् ब्रह्मै॒वैव ब्रह्म॑ दद्ध्याद् दद्ध्या॒द् ब्रह्मै॒व । \newline
26. ब्रह्मै॒वैव ब्रह्म॒ ब्रह्मै॒व प्र॑ति॒पद॑म् प्रति॒पद॑ मे॒व ब्रह्म॒ ब्रह्मै॒व प्र॑ति॒पद᳚म् । \newline
27. ए॒व प्र॑ति॒पद॑म् प्रति॒पद॑ मे॒वैव प्र॑ति॒पद॑म् कुरुते कुरुते प्रति॒पद॑ मे॒वैव प्र॑ति॒पद॑म् कुरुते । \newline
28. प्र॒ति॒पद॑म् कुरुते कुरुते प्रति॒पद॑म् प्रति॒पद॑म् कुरुते॒ तथा॒ तथा॑ कुरुते प्रति॒पद॑म् प्रति॒पद॑म् कुरुते॒ तथा᳚ । \newline
29. प्र॒ति॒पद॒मिति॑ प्रति - पद᳚म् । \newline
30. कु॒रु॒ते॒ तथा॒ तथा॑ कुरुते कुरुते॒ तथा᳚ ब्राह्म॒णो ब्रा᳚ह्म॒ण स्तथा॑ कुरुते कुरुते॒ तथा᳚ ब्राह्म॒णः । \newline
31. तथा᳚ ब्राह्म॒णो ब्रा᳚ह्म॒ण स्तथा॒ तथा᳚ ब्राह्म॒णः साशी᳚र्केण॒ साशी᳚र्केण ब्राह्म॒ण स्तथा॒ तथा᳚ ब्राह्म॒णः साशी᳚र्केण । \newline
32. ब्रा॒ह्म॒णः साशी᳚र्केण॒ साशी᳚र्केण ब्राह्म॒णो ब्रा᳚ह्म॒णः साशी᳚र्केण य॒ज्ञेन॑ य॒ज्ञेन॒ साशी᳚र्केण 
ब्राह्म॒णो ब्रा᳚ह्म॒णः साशी᳚र्केण य॒ज्ञेन॑ । \newline
33. साशी᳚र्केण य॒ज्ञेन॑ य॒ज्ञेन॒ साशी᳚र्केण॒ साशी᳚र्केण य॒ज्ञेन॑ यजते यजते य॒ज्ञेन॒ साशी᳚र्केण॒ साशी᳚र्केण य॒ज्ञेन॑ यजते । \newline
34. साशी᳚र्के॒णेति॒ स - आ॒शी॒र्के॒ण॒ । \newline
35. य॒ज्ञेन॑ यजते यजते य॒ज्ञेन॑ य॒ज्ञेन॑ यजते॒ यं ॅयं ॅय॑जते य॒ज्ञेन॑ य॒ज्ञेन॑ यजते॒ यम् । \newline
36. य॒ज॒ते॒ यं ॅयं ॅय॑जते यजते॒ यम् का॒मये॑त का॒मये॑त॒ यं ॅय॑जते यजते॒ यम् का॒मये॑त । \newline
37. यम् का॒मये॑त का॒मये॑त॒ यं ॅयम् का॒मये॑त॒ यज॑मानं॒ ॅयज॑मानम् का॒मये॑त॒ यं ॅयम् का॒मये॑त॒ यज॑मानम् । \newline
38. का॒मये॑त॒ यज॑मानं॒ ॅयज॑मानम् का॒मये॑त का॒मये॑त॒ यज॑मान॒म् भ्रातृ॑व्य॒म् भ्रातृ॑व्यं॒ ॅयज॑मानम् का॒मये॑त का॒मये॑त॒ यज॑मान॒म् भ्रातृ॑व्यम् । \newline
39. यज॑मान॒म् भ्रातृ॑व्य॒म् भ्रातृ॑व्यं॒ ॅयज॑मानं॒ ॅयज॑मान॒म् भ्रातृ॑व्य मस्यास्य॒ भ्रातृ॑व्यं॒ ॅयज॑मानं॒ ॅयज॑मान॒म् भ्रातृ॑व्य मस्य । \newline
40. भ्रातृ॑व्य मस्यास्य॒ भ्रातृ॑व्य॒म् भ्रातृ॑व्य मस्य य॒ज्ञ्स्य॑ य॒ज्ञ्स्या᳚स्य॒ भ्रातृ॑व्य॒म् भ्रातृ॑व्य मस्य य॒ज्ञ्स्य॑ । \newline
41. अ॒स्य॒ य॒ज्ञ्स्य॑ य॒ज्ञ्स्या᳚स्यास्य य॒ज्ञ्स्या॒शी रा॒शीर् य॒ज्ञ्स्या᳚स्यास्य य॒ज्ञ्स्या॒शीः । \newline
42. य॒ज्ञ्स्या॒शी रा॒शीर् य॒ज्ञ्स्य॑ य॒ज्ञ्स्या॒शीर् ग॑च्छेद् गच्छेदा॒शीर् य॒ज्ञ्स्य॑ य॒ज्ञ्स्या॒शीर् ग॑च्छेत् । \newline
43. आ॒शीर् ग॑च्छेद् गच्छेदा॒शी रा॒शीर् ग॑च्छे॒दितीति॑ गच्छेदा॒शी रा॒शीर् ग॑च्छे॒दिति॑ । \newline
44. आ॒शीरित्या᳚ - शीः । \newline
45. ग॒च्छे॒दितीति॑ गच्छेद् गच्छे॒दिति॒ तस्य॒ तस्ये ति॑ गच्छेद् गच्छे॒दिति॒ तस्य॑ । \newline
46. इति॒ तस्य॒ तस्ये तीति॒ तस्यै॒ता ए॒तास्तस्ये तीति॒ तस्यै॒ताः । \newline
47. तस्यै॒ता ए॒तास्तस्य॒ तस्यै॒ता व्याहृ॑ती॒र् व्याहृ॑ती रे॒तास्तस्य॒ तस्यै॒ता व्याहृ॑तीः । \newline
48. ए॒ता व्याहृ॑ती॒र् व्याहृ॑तीरे॒ता ए॒ता व्याहृ॑तीः पुरोनुवा॒क्या॑याम् पुरोनुवा॒क्या॑यां॒ ॅव्याहृ॑तीरे॒ता ए॒ता व्याहृ॑तीः पुरोनुवा॒क्या॑याम् । \newline
49. व्याहृ॑तीः पुरोनुवा॒क्या॑याम् पुरोनुवा॒क्या॑यां॒ ॅव्याहृ॑ती॒र् व्याहृ॑तीः पुरोनुवा॒क्या॑याम् दद्ध्याद् दद्ध्यात् पुरोनुवा॒क्या॑यां॒ ॅव्याहृ॑ती॒र् व्याहृ॑तीः पुरोनुवा॒क्या॑याम् दद्ध्यात् । \newline
50. व्याहृ॑ती॒रिति॑ वि - आहृ॑तीः । \newline
51. पु॒रो॒नु॒वा॒क्या॑याम् दद्ध्याद् दद्ध्यात् पुरोनुवा॒क्या॑याम् पुरोनुवा॒क्या॑याम् दद्ध्याद् भ्रातृव्यदेव॒त्या᳚ भ्रातृव्यदेव॒त्या॑ दद्ध्यात् पुरोनुवा॒क्या॑याम् पुरोनुवा॒क्या॑याम् दद्ध्याद् भ्रातृव्यदेव॒त्या᳚ । \newline
52. पु॒रो॒नु॒वा॒क्या॑या॒मिति॑ पुरः - अ॒नु॒वा॒क्या॑याम् । \newline
53. द॒द्ध्या॒द् भ्रा॒तृ॒व्य॒दे॒व॒त्या᳚ भ्रातृव्यदेव॒त्या॑ दद्ध्याद् दद्ध्याद् भ्रातृव्यदेव॒त्या॑ वै वै भ्रा॑तृव्यदेव॒त्या॑ दद्ध्याद् दद्ध्याद् भ्रातृव्यदेव॒त्या॑ वै । \newline
54. भ्रा॒तृ॒व्य॒दे॒व॒त्या॑ वै वै भ्रा॑तृव्यदेव॒त्या᳚ भ्रातृव्यदेव॒त्या॑ वै पु॑रोनुवा॒क्या॑ पुरोनुवा॒क्या॑ वै भ्रा॑तृव्यदेव॒त्या᳚ भ्रातृव्यदेव॒त्या॑ वै पु॑रोनुवा॒क्या᳚ । \newline
55. भ्रा॒तृ॒व्य॒दे॒व॒त्येति॑ भ्रातृव्य - दे॒व॒त्या᳚ । \newline
56. वै पु॑रोनुवा॒क्या॑ पुरोनुवा॒क्या॑ वै वै पु॑रोनुवा॒क्या᳚ भ्रातृ॑व्य॒म् भ्रातृ॑व्यम् पुरोनुवा॒क्या॑ वै वै पु॑रोनुवा॒क्या᳚ भ्रातृ॑व्यम् । \newline
57. पु॒रो॒नु॒वा॒क्या᳚ भ्रातृ॑व्य॒म् भ्रातृ॑व्यम् पुरोनुवा॒क्या॑ पुरोनुवा॒क्या᳚ भ्रातृ॑व्य मे॒वैव भ्रातृ॑व्यम् पुरोनुवा॒क्या॑ पुरोनुवा॒क्या᳚ भ्रातृ॑व्य मे॒व । \newline
58. पु॒रो॒नु॒वा॒क्येति॑ पुरः - अ॒नु॒वा॒क्या᳚ । \newline
59. भ्रातृ॑व्य मे॒वैव भ्रातृ॑व्य॒म् भ्रातृ॑व्य मे॒वास्या᳚स्यै॒व भ्रातृ॑व्य॒म् भ्रातृ॑व्य मे॒वास्य॑ । \newline
60. ए॒वास्या᳚स्यै॒वैवास्य॑ य॒ज्ञ्स्य॑ य॒ज्ञ्स्या᳚स्यै॒वैवास्य॑ य॒ज्ञ्स्य॑ । \newline
61. अ॒स्य॒ य॒ज्ञ्स्य॑ य॒ज्ञ्स्या᳚स्यास्य य॒ज्ञ्स्या॒शी रा॒शीर् य॒ज्ञ्स्या᳚स्यास्य य॒ज्ञ्स्या॒शीः । \newline
62. य॒ज्ञ्स्या॒शी रा॒शीर् य॒ज्ञ्स्य॑ य॒ज्ञ्स्या॒शीर् ग॑च्छति गच्छत्या॒शीर् य॒ज्ञ्स्य॑ य॒ज्ञ्स्या॒शीर् ग॑च्छति । \newline
\pagebreak
\markright{ TS 1.6.10.5  \hfill https://www.vedavms.in \hfill}

\section{ TS 1.6.10.5 }

\textbf{TS 1.6.10.5 } \newline
\textbf{Samhita Paata} \newline

ऽऽशीर्ग॑च्छति॒ यान् का॒मये॑त॒ यज॑मानान्थ् स॒माव॑त्येनान् य॒ज्ञ्स्या॒ ऽऽशीर्ग॑च्छे॒दिति॒ तेषा॑मे॒ता व्याहृ॑तीः पुरोनुवा॒क्या॑या अर्द्ध॒र्च एकां᳚ दद्ध्याद्-या॒ज्या॑यै पु॒रस्ता॒देकां᳚ ॅया॒ज्या॑या अर्द्ध॒र्च एकां॒ तथै॑नान्थ् स॒माव॑ती य॒ज्ञ्स्या॒ ऽऽशीर्ग॑च्छति॒ यथा॒ वै प॒र्जन्यः॒ सुवृ॑ष्टं॒ ॅवर्.ष॑त्ये॒वं ॅय॒ज्ञो यज॑मानाय वर्.षति॒ स्थल॑योद॒कं प॑रिगृ॒ह्णन्त्या॒शिषा॑ य॒ज्ञ्ं ॅयज॑मानः॒ परि॑ गृह्णाति॒ मनो॑ऽसि प्राजाप॒त्यं - [ ] \newline

\textbf{Pada Paata} \newline

आ॒शीरित्या᳚ - शीः । ग॒च्छ॒ति॒ । यान् । का॒मये॑त । यज॑मानान् । स॒माव॑ती । ए॒ना॒न् । य॒ज्ञ्स्य॑ । आ॒शीरित्या᳚ - शीः । ग॒च्छे॒त् । इति॑ । तेषा᳚म् । ए॒ताः । व्याहृ॑ती॒रिति॑ वि - आहृ॑तीः । पु॒रो॒नु॒वा॒क्या॑या॒ इति॑ पुरः-अ॒नु॒वा॒क्या॑याः । अ॒द्‌र्ध॒र्च इत्य॒॑द्‌र्ध-ऋ॒चे । एका᳚म् । द॒द्ध्या॒त् । या॒ज्या॑यै । पु॒रस्ता᳚त् । एका᳚म् । या॒ज्या॑याः । अ॒द्‌र्ध॒र्च इत्य॑द्‌र्ध - ऋ॒चे । एका᳚म् । तथा᳚ । ए॒ना॒न् । स॒माव॑ती । य॒ज्ञ्स्य॑ । आ॒शीरित्या᳚ - शीः । ग॒च्छ॒ति॒ । यथा᳚ । वै । प॒र्जन्यः॑ । सुवृ॑ष्ट॒मिति॒ सु - वृ॒ष्ट॒म् । वर्.ष॑ति । ए॒वम् । य॒ज्ञ्ः । यज॑मानाय । व॒र्॒.ष॒ति॒ । स्थल॑या । उ॒द॒कम् । प॒रि॒गृ॒ह्णन्तीति॑ परि - गृ॒ह्णन्ति॑ । आ॒शिषेत्या᳚ - शिषा᳚ । य॒ज्ञ्म् । यज॑मानः । परीति॑ । गृ॒ह्णा॒ति॒ । मनः॑ । अ॒सि॒ । प्रा॒जा॒प॒त्यमिति॑ प्राजा - प॒त्यम् ।  \newline


\textbf{Krama Paata} \newline

आ॒शीर्,ग॑च्छति । आ॒शीरित्या᳚ - शीः । ग॒च्छ॒ति॒ यान् । यान् का॒मये॑त । का॒मये॑त॒ यज॑मानान् । यज॑मानान्थ् स॒माव॑ती । स॒माव॑त्येनान् । ए॒ना॒न्.॒ य॒ज्ञ्स्य॑ । य॒ज्ञ्स्या॒शीः । आ॒शीर्,ग॑च्छेत् । आ॒शीरित्या᳚ - शीः । ग॒च्छे॒दिति॑ । इति॒ तेषा᳚म् । तेषा॑मे॒ताः । ए॒ता व्याहृ॑तीः । व्याहृ॑तीः पुरोनुवा॒क्या॑याः । व्याहृ॑ती॒रिति॑ वि - आहृ॑तीः । पु॒रो॒नु॒वा॒क्या॑या अर्द्ध॒र्चे । पु॒रो॒नु॒वा॒क्या॑या॒ इति॑ पुरः - अ॒नु॒वा॒क्या॑याः । अ॒र्द्ध॒र्च एका᳚म् । अ॒र्द्ध॒र्च इत्य॑र्द्ध - ऋ॒चे । एका᳚म् दद्ध्यात् । द॒द्ध्या॒द् या॒ज्या॑यै । या॒ज्या॑यै पु॒रस्ता᳚त् । पु॒रस्ता॒देका᳚म् । एकां᳚ ॅया॒ज्या॑याः । या॒ज्या॑या अर्द्ध॒र्चे । अ॒र्द्ध॒र्च एका᳚म् । अ॒र्द्ध॒र्च इत्य॑र्द्ध - ऋ॒चे । एका॒म् तथा᳚ । तथै॑नान् । ए॒ना॒न्थ् स॒माव॑ती । स॒माव॑ती य॒ज्ञ्स्य॑ । य॒ज्ञ्स्या॒शीः । आ॒शीर्,ग॑च्छति । आ॒शीरित्या᳚ - शीः । ग॒च्छ॒ति॒ यथा᳚ । यथा॒ वै । वै प॒र्जन्यः॑ । प॒र्जन्यः॒ सुवृ॑ष्टम् । सुवृ॑ष्टं॒ ॅवर्.ष॑ति । सुवृ॑ष्ट॒मिति॒ सु - वृ॒ष्ट॒म् । वर्.ष॑त्ये॒वम् । ए॒वं ॅय॒ज्ञ्ः । य॒ज्ञो यज॑मानाय । यज॑मानाय वर्.षति । व॒र्॒.ष॒ति॒ स्थल॑या । स्थल॑योद॒कम् । उ॒द॒कम् प॑रिगृ॒ह्णन्ति॑ । प॒रि॒गृ॒ह्णन्त्या॒शिषा᳚ । प॒रि॒गृ॒ह्णन्तीति॑ परि - गृ॒ह्णन्ति॑ । आ॒शिषा॑ य॒ज्ञ्म् । आ॒शिषेत्या᳚ - शिषा᳚ । य॒ज्ञ्ं ॅयज॑मानः । यज॑मानः॒ परि॑ । परि॑ गृह्णाति । गृ॒ह्णा॒ति॒ मनः॑ । मनो॑ऽसि । अ॒सि॒ प्रा॒जा॒प॒त्यम् । प्रा॒जा॒प॒त्यम् मन॑सा । प्रा॒जा॒प॒त्यमिति॑ प्राजा - प॒त्यम् \newline

\textbf{Jatai Paata} \newline

1. आ॒शीर् ग॑च्छति गच्छ त्या॒शी रा॒शीर् ग॑च्छति । \newline
2. आ॒शीरित्या᳚ - शीः । \newline
3. ग॒च्छ॒ति॒ यान्. यान् ग॑च्छति गच्छति॒ यान् । \newline
4. यान् का॒मये॑त का॒मये॑त॒ यान्. यान् का॒मये॑त । \newline
5. का॒मये॑त॒ यज॑माना॒न्॒. यज॑मानान् का॒मये॑त का॒मये॑त॒ यज॑मानान् । \newline
6. यज॑मानान् थ्स॒माव॑ती स॒माव॑ती॒ यज॑माना॒न्॒. यज॑मानान् थ्स॒माव॑ती । \newline
7. स॒माव॑ त्येना नेनान् थ्स॒माव॑ती स॒माव॑ त्येनान् । \newline
8. ए॒ना॒न्॒. य॒ज्ञ्स्य॑ य॒ज्ञ् स्यै॑ना नेनान्. य॒ज्ञ्स्य॑ । \newline
9. य॒ज्ञ्स्या॒शी रा॒शीर् य॒ज्ञ्स्य॑ य॒ज्ञ्स्या॒शीः । \newline
10. आ॒शीर् ग॑च्छेद् गच्छे दा॒शी रा॒शीर् ग॑च्छेत् । \newline
11. आ॒शीरित्या᳚ - शीः । \newline
12. ग॒च्छे॒ दितीति॑ गच्छेद् गच्छे॒ दिति॑ । \newline
13. इति॒ तेषा॒म् तेषा॒ मितीति॒ तेषा᳚म् । \newline
14. तेषा॑ मे॒ता ए॒ता स्तेषा॒म् तेषा॑ मे॒ताः । \newline
15. ए॒ता व्याहृ॑ती॒र् व्याहृ॑तीरे॒ता ए॒ता व्याहृ॑तीः । \newline
16. व्याहृ॑तीः पुरोनुवा॒क्या॑याः पुरोनुवा॒क्या॑या॒ व्याहृ॑ती॒र् व्याहृ॑तीः पुरोनुवा॒क्या॑याः । \newline
17. व्याहृ॑ती॒रिति॑ वि - आहृ॑तीः । \newline
18. पु॒रो॒नु॒वा॒क्या॑या अर्द्ध॒र्चे᳚ ऽर्द्ध॒र्चे पु॑रोनुवा॒क्या॑याः पुरोनुवा॒क्या॑या अर्द्ध॒र्चे । \newline
19. पु॒रो॒नु॒वा॒क्या॑या॒ इति॑ पुरः - अ॒नु॒वा॒क्या॑याः । \newline
20. अ॒र्द्ध॒र्च एका॒ मेका॑ मर्द्ध॒र्चे᳚ ऽर्द्ध॒र्च एका᳚म् । \newline
21. अ॒र्द्ध॒र्च इत्य॑र्द्ध - ऋ॒चे । \newline
22. एका᳚म् दद्ध्याद् दद्ध्या॒ देका॒ मेका᳚म् दद्ध्यात् । \newline
23. द॒द्ध्या॒द् या॒ज्या॑यै या॒ज्या॑यै दद्ध्याद् दद्ध्याद् या॒ज्या॑यै । \newline
24. या॒ज्या॑यै पु॒रस्ता᳚त् पु॒रस्ता᳚द् या॒ज्या॑यै या॒ज्या॑यै पु॒रस्ता᳚त् । \newline
25. पु॒रस्ता॒ देका॒ मेका᳚म् पु॒रस्ता᳚त् पु॒रस्ता॒ देका᳚म् । \newline
26. एकां᳚ ॅया॒ज्या॑या या॒ज्या॑या॒ एका॒ मेकां᳚ ॅया॒ज्या॑याः । \newline
27. या॒ज्या॑या अर्द्ध॒र्चे᳚ ऽर्द्ध॒र्चे या॒ज्या॑या या॒ज्या॑या अर्द्ध॒र्चे । \newline
28. अ॒र्द्ध॒र्च एका॒ मेका॑ मर्द्ध॒र्चे᳚ ऽर्द्ध॒र्च एका᳚म् । \newline
29. अ॒र्द्ध॒र्च इत्य॑र्द्ध - ऋ॒चे । \newline
30. एका॒म् तथा॒ तथैका॒ मेका॒म् तथा᳚ । \newline
31. तथै॑ना नेना॒न् तथा॒ तथै॑नान् । \newline
32. ए॒ना॒न् थ्स॒माव॑ती स॒माव॑ त्येना नेनान् थ्स॒माव॑ती । \newline
33. स॒माव॑ती य॒ज्ञ्स्य॑ य॒ज्ञ्स्य॑ स॒माव॑ती स॒माव॑ती य॒ज्ञ्स्य॑ । \newline
34. य॒ज्ञ्स्या॒शी रा॒शीर् य॒ज्ञ्स्य॑ य॒ज्ञ्स्या॒शीः । \newline
35. आ॒शीर् ग॑च्छति गच्छ त्या॒शी रा॒शीर् ग॑च्छति । \newline
36. आ॒शीरित्या᳚ - शीः । \newline
37. ग॒च्छ॒ति॒ यथा॒ यथा॑ गच्छति गच्छति॒ यथा᳚ । \newline
38. यथा॒ वै वै यथा॒ यथा॒ वै । \newline
39. वै प॒र्जन्यः॑ प॒र्जन्यो॒ वै वै प॒र्जन्यः॑ । \newline
40. प॒र्जन्यः॒ सुवृ॑ष्ट॒(ग्म्॒) सुवृ॑ष्टम् प॒र्जन्यः॑ प॒र्जन्यः॒ सुवृ॑ष्टम् । \newline
41. सुवृ॑ष्टं॒ ॅवर्.ष॑ति॒ वर्.ष॑ति॒ सुवृ॑ष्ट॒(ग्म्॒) सुवृ॑ष्टं॒ ॅवर्.ष॑ति । \newline
42. सुवृ॑ष्ट॒मिति॒ सु - वृ॒ष्ट॒म् । \newline
43. वर्.ष॑ त्ये॒व मे॒वं ॅवर्.ष॑ति॒ वर्.ष॑ त्ये॒वम् । \newline
44. ए॒वं ॅय॒ज्ञो य॒ज्ञ् ए॒व मे॒वं ॅय॒ज्ञ्ः । \newline
45. य॒ज्ञो यज॑मानाय॒ यज॑मानाय य॒ज्ञो य॒ज्ञो यज॑मानाय । \newline
46. यज॑मानाय वर्.षति वर्.षति॒ यज॑मानाय॒ यज॑मानाय वर्.षति । \newline
47. व॒र्॒.ष॒ति॒ स्थल॑या॒ स्थल॑या वर्.षति वर्.षति॒ स्थल॑या । \newline
48. स्थल॑यो द॒क मु॑द॒कꣳ स्थल॑या॒ स्थल॑यो द॒कम् । \newline
49. उ॒द॒कम् प॑रिगृ॒ह्णन्ति॑ परिगृ॒ह्ण न्त्यु॑द॒क मु॑द॒कम् प॑रिगृ॒ह्णन्ति॑ । \newline
50. प॒रि॒गृ॒ह्ण न्त्या॒शिषा॒ ऽऽशिषा॑ परिगृ॒ह्णन्ति॑ परिगृ॒ह्ण न्त्या॒शिषा᳚ । \newline
51. प॒रि॒गृ॒ह्णन्तीति॑ परि - गृ॒ह्णन्ति॑ । \newline
52. आ॒शिषा॑ य॒ज्ञ्ं ॅय॒ज्ञ् मा॒शिषा॒ ऽऽशिषा॑ य॒ज्ञ्म् । \newline
53. आ॒शिषेत्या᳚ - शिषा᳚ । \newline
54. य॒ज्ञ्ं ॅयज॑मानो॒ यज॑मानो य॒ज्ञ्ं ॅय॒ज्ञ्ं ॅयज॑मानः । \newline
55. यज॑मानः॒ परि॒ परि॒ यज॑मानो॒ यज॑मानः॒ परि॑ । \newline
56. परि॑ गृह्णाति गृह्णाति॒ परि॒ परि॑ गृह्णाति । \newline
57. गृ॒ह्णा॒ति॒ मनो॒ मनो॑ गृह्णाति गृह्णाति॒ मनः॑ । \newline
58. मनो᳚ ऽस्यसि॒ मनो॒ मनो॑ ऽसि । \newline
59. अ॒सि॒ प्रा॒जा॒प॒त्यम् प्रा॑जाप॒त्य म॑स्यसि प्राजाप॒त्यम् । \newline
60. प्रा॒जा॒प॒त्यम् मन॑सा॒ मन॑सा प्राजाप॒त्यम् प्रा॑जाप॒त्यम् मन॑सा । \newline
61. प्रा॒जा॒प॒त्यमिति॑ प्राजा - प॒त्यम् । \newline

\textbf{Ghana Paata } \newline

1. आ॒शीर् ग॑च्छति गच्छत्या॒शी रा॒शीर् ग॑च्छति॒ यान्. यान् ग॑च्छत्या॒शी रा॒शीर् ग॑च्छति॒ यान् । \newline
2. आ॒शीरित्या᳚ - शीः । \newline
3. ग॒च्छ॒ति॒ यान्. यान् ग॑च्छति गच्छति॒ यान् का॒मये॑त का॒मये॑त॒ यान् ग॑च्छति गच्छति॒ यान् का॒मये॑त । \newline
4. यान् का॒मये॑त का॒मये॑त॒ यान्. यान् का॒मये॑त॒ यज॑माना॒न्॒. यज॑मानान् का॒मये॑त॒ यान्. यान् का॒मये॑त॒ यज॑मानान् । \newline
5. का॒मये॑त॒ यज॑माना॒न्॒. यज॑मानान् का॒मये॑त का॒मये॑त॒ यज॑मानान् थ्स॒माव॑ती स॒माव॑ती॒ यज॑मानान् का॒मये॑त का॒मये॑त॒ यज॑मानान् थ्स॒माव॑ती । \newline
6. यज॑मानान् थ्स॒माव॑ती स॒माव॑ती॒ यज॑माना॒न्॒. यज॑मानान् थ्स॒माव॑त्येना नेनान् थ्स॒माव॑ती॒ यज॑माना॒न्॒. यज॑मानान् थ्स॒माव॑त्येनान् । \newline
7. स॒माव॑त्येना नेनान् थ्स॒माव॑ती स॒माव॑त्येनान्. य॒ज्ञ्स्य॑ य॒ज्ञ्स्यै॑नान् थ्स॒माव॑ती स॒माव॑त्येनान्. य॒ज्ञ्स्य॑ । \newline
8. ए॒ना॒न्॒. य॒ज्ञ्स्य॑ य॒ज्ञ्स्यै॑ना नेनान्. य॒ज्ञ्स्या॒शी रा॒शीर् य॒ज्ञ्स्यै॑ना नेनान्. य॒ज्ञ्स्या॒शीः । \newline
9. य॒ज्ञ्स्या॒शी रा॒शीर् य॒ज्ञ्स्य॑ य॒ज्ञ्स्या॒शीर् ग॑च्छेद् गच्छेदा॒शीर् य॒ज्ञ्स्य॑ य॒ज्ञ्स्या॒शीर् ग॑च्छेत् । \newline
10. आ॒शीर् ग॑च्छेद् गच्छेदा॒शी रा॒शीर् ग॑च्छे॒दितीति॑ गच्छेदा॒शी रा॒शीर् ग॑च्छे॒दिति॑ । \newline
11. आ॒शीरित्या᳚ - शीः । \newline
12. ग॒च्छे॒दितीति॑ गच्छेद् गच्छे॒दिति॒ तेषा॒म् तेषा॒ मिति॑ गच्छेद् गच्छे॒दिति॒ तेषा᳚म् । \newline
13. इति॒ तेषा॒म् तेषा॒ मितीति॒ तेषा॑ मे॒ता ए॒तास्तेषा॒ मितीति॒ तेषा॑ मे॒ताः । \newline
14. तेषा॑ मे॒ता ए॒ता स्तेषा॒म् तेषा॑ मे॒ता व्याहृ॑ती॒र् व्याहृ॑ती रे॒ता स्तेषा॒म् तेषा॑ मे॒ता व्याहृ॑तीः । \newline
15. ए॒ता व्याहृ॑ती॒र् व्याहृ॑तीरे॒ता ए॒ता व्याहृ॑तीः पुरोनुवा॒क्या॑याः पुरोनुवा॒क्या॑या॒ व्याहृ॑तीरे॒ता ए॒ता व्याहृ॑तीः पुरोनुवा॒क्या॑याः । \newline
16. व्याहृ॑तीः पुरोनुवा॒क्या॑याः पुरोनुवा॒क्या॑या॒ व्याहृ॑ती॒र् व्याहृ॑तीः पुरोनुवा॒क्या॑या अर्द्ध॒र्चे᳚ ऽर्द्ध॒र्चे पु॑रोनुवा॒क्या॑या॒ व्याहृ॑ती॒र् व्याहृ॑तीः पुरोनुवा॒क्या॑या अर्द्ध॒र्चे । \newline
17. व्याहृ॑ती॒रिति॑ वि - आहृ॑तीः । \newline
18. पु॒रो॒नु॒वा॒क्या॑या अर्द्ध॒र्चे᳚ ऽर्द्ध॒र्चे पु॑रोनुवा॒क्या॑याः पुरोनुवा॒क्या॑या अर्द्ध॒र्च एका॒ मेका॑ मर्द्ध॒र्चे पु॑रोनुवा॒क्या॑याः पुरोनुवा॒क्या॑या अर्द्ध॒र्च एका᳚म् । \newline
19. पु॒रो॒नु॒वा॒क्या॑या॒ इति॑ पुरः - अ॒नु॒वा॒क्या॑याः । \newline
20. अ॒र्द्ध॒र्च एका॒ मेका॑ मर्द्ध॒र्चे᳚ ऽर्द्ध॒र्च एका᳚म् दद्ध्याद् दद्ध्या॒देका॑ मर्द्ध॒र्चे᳚ ऽर्द्ध॒र्च एका᳚म् दद्ध्यात् । \newline
21. अ॒र्द्ध॒र्च इत्य॑र्द्ध - ऋ॒चे । \newline
22. एका᳚म् दद्ध्याद् दद्ध्या॒देका॒ मेका᳚म् दद्ध्याद् या॒ज्या॑यै या॒ज्या॑यै दद्ध्या॒देका॒ मेका᳚म् दद्ध्याद् या॒ज्या॑यै । \newline
23. द॒द्ध्या॒द् या॒ज्या॑यै या॒ज्या॑यै दद्ध्याद् दद्ध्याद् या॒ज्या॑यै पु॒रस्ता᳚त् पु॒रस्ता᳚द् या॒ज्या॑यै दद्ध्याद् दद्ध्याद् या॒ज्या॑यै पु॒रस्ता᳚त् । \newline
24. या॒ज्या॑यै पु॒रस्ता᳚त् पु॒रस्ता᳚द् या॒ज्या॑यै या॒ज्या॑यै पु॒रस्ता॒देका॒ मेका᳚म् पु॒रस्ता᳚द् या॒ज्या॑यै या॒ज्या॑यै पु॒रस्ता॒देका᳚म् । \newline
25. पु॒रस्ता॒देका॒ मेका᳚म् पु॒रस्ता᳚त् पु॒रस्ता॒देकां᳚ ॅया॒ज्या॑या या॒ज्या॑या॒ एका᳚म् पु॒रस्ता᳚त् पु॒रस्ता॒देकां᳚ ॅया॒ज्या॑याः । \newline
26. एकां᳚ ॅया॒ज्या॑या या॒ज्या॑या॒ एका॒ मेकां᳚ ॅया॒ज्या॑या अर्द्ध॒र्चे᳚ ऽर्द्ध॒र्चे या॒ज्या॑या॒ एका॒ मेकां᳚ ॅया॒ज्या॑या अर्द्ध॒र्चे । \newline
27. या॒ज्या॑या अर्द्ध॒र्चे᳚ ऽर्द्ध॒र्चे या॒ज्या॑या या॒ज्या॑या अर्द्ध॒र्च एका॒ मेका॑ मर्द्ध॒र्चे या॒ज्या॑या या॒ज्या॑या अर्द्ध॒र्च एका᳚म् । \newline
28. अ॒र्द्ध॒र्च एका॒ मेका॑ मर्द्ध॒र्चे᳚ ऽर्द्ध॒र्च एका॒म् तथा॒ तथैका॑ मर्द्ध॒र्चे᳚ ऽर्द्ध॒र्च एका॒म् तथा᳚ । \newline
29. अ॒र्द्ध॒र्च इत्य॑र्द्ध - ऋ॒चे । \newline
30. एका॒म् तथा॒ तथैका॒ मेका॒म् तथै॑ना नेना॒न् तथैका॒ मेका॒म् तथै॑नान् । \newline
31. तथै॑ना नेना॒न् तथा॒ तथै॑नान् थ्स॒माव॑ती स॒माव॑त्येना॒न् तथा॒ तथै॑नान् थ्स॒माव॑ती । \newline
32. ए॒ना॒न् थ्स॒माव॑ती स॒माव॑त्येना नेनान् थ्स॒माव॑ती य॒ज्ञ्स्य॑ य॒ज्ञ्स्य॑ स॒माव॑त्येना नेनान् थ्स॒माव॑ती य॒ज्ञ्स्य॑ । \newline
33. स॒माव॑ती य॒ज्ञ्स्य॑ य॒ज्ञ्स्य॑ स॒माव॑ती स॒माव॑ती य॒ज्ञ्स्या॒शी रा॒शीर् य॒ज्ञ्स्य॑ स॒माव॑ती स॒माव॑ती य॒ज्ञ्स्या॒शीः । \newline
34. य॒ज्ञ्स्या॒शी रा॒शीर् य॒ज्ञ्स्य॑ य॒ज्ञ्स्या॒शीर् ग॑च्छति गच्छत्या॒शीर् य॒ज्ञ्स्य॑ य॒ज्ञ्स्या॒शीर् ग॑च्छति । \newline
35. आ॒शीर् ग॑च्छति गच्छत्या॒शी रा॒शीर् ग॑च्छति॒ यथा॒ यथा॑ गच्छत्या॒शी रा॒शीर् ग॑च्छति॒ यथा᳚ । \newline
36. आ॒शीरित्या᳚ - शीः । \newline
37. ग॒च्छ॒ति॒ यथा॒ यथा॑ गच्छति गच्छति॒ यथा॒ वै वै यथा॑ गच्छति गच्छति॒ यथा॒ वै । \newline
38. यथा॒ वै वै यथा॒ यथा॒ वै प॒र्जन्यः॑ प॒र्जन्यो॒ वै यथा॒ यथा॒ वै प॒र्जन्यः॑ । \newline
39. वै प॒र्जन्यः॑ प॒र्जन्यो॒ वै वै प॒र्जन्यः॒ सुवृ॑ष्ट॒(ग्म्॒) सुवृ॑ष्टम् प॒र्जन्यो॒ वै वै प॒र्जन्यः॒ सुवृ॑ष्टम् । \newline
40. प॒र्जन्यः॒ सुवृ॑ष्ट॒(ग्म्॒) सुवृ॑ष्टम् प॒र्जन्यः॑ प॒र्जन्यः॒ सुवृ॑ष्टं॒ ॅवर्.ष॑ति॒ वर्.ष॑ति॒ सुवृ॑ष्टम् प॒र्जन्यः॑ प॒र्जन्यः॒ सुवृ॑ष्टं॒ ॅवर्.ष॑ति । \newline
41. सुवृ॑ष्टं॒ ॅवर्.ष॑ति॒ वर्.ष॑ति॒ सुवृ॑ष्ट॒(ग्म्॒) सुवृ॑ष्टं॒ ॅवर्.ष॑त्ये॒व मे॒वं ॅवर्.ष॑ति॒ सुवृ॑ष्ट॒(ग्म्॒) सुवृ॑ष्टं॒ ॅवर्.ष॑त्ये॒वम् । \newline
42. सुवृ॑ष्ट॒मिति॒ सु - वृ॒ष्ट॒म् । \newline
43. वर्.ष॑त्ये॒व मे॒वं ॅवर्.ष॑ति॒ वर्.ष॑त्ये॒वं ॅय॒ज्ञो य॒ज्ञ् ए॒वं ॅवर्.ष॑ति॒ वर्.ष॑त्ये॒वं ॅय॒ज्ञ्ः । \newline
44. ए॒वं ॅय॒ज्ञो य॒ज्ञ् ए॒व मे॒वं ॅय॒ज्ञो यज॑मानाय॒ यज॑मानाय य॒ज्ञ् ए॒व मे॒वं ॅय॒ज्ञो यज॑मानाय । \newline
45. य॒ज्ञो यज॑मानाय॒ यज॑मानाय य॒ज्ञो य॒ज्ञो यज॑मानाय वर्.षति वर्.षति॒ यज॑मानाय य॒ज्ञो य॒ज्ञो यज॑मानाय वर्.षति । \newline
46. यज॑मानाय वर्.षति वर्.षति॒ यज॑मानाय॒ यज॑मानाय वर्.षति॒ स्थल॑या॒ स्थल॑या वर्.षति॒ यज॑मानाय॒ यज॑मानाय वर्.षति॒ स्थल॑या । \newline
47. व॒र्॒.ष॒ति॒ स्थल॑या॒ स्थल॑या वर्.षति वर्.षति॒ स्थल॑योद॒क मु॑द॒कꣳ स्थल॑या वर्.षति वर्.षति॒ स्थल॑योद॒कम् । \newline
48. स्थल॑योद॒क मु॑द॒कꣳ स्थल॑या॒ स्थल॑योद॒कम् प॑रिगृ॒ह्णन्ति॑ परिगृ॒ह्णन्त्यु॑द॒कꣳ स्थल॑या॒ स्थल॑योद॒कम् प॑रिगृ॒ह्णन्ति॑ । \newline
49. उ॒द॒कम् प॑रिगृ॒ह्णन्ति॑ परिगृ॒ह्णन्त्यु॑द॒क मु॑द॒कम् प॑रिगृ॒ह्णन्त्या॒शिषा॒ ऽऽशिषा॑ परिगृ॒ह्णन्त्यु॑द॒क मु॑द॒कम् प॑रिगृ॒ह्णन्त्या॒शिषा᳚ । \newline
50. प॒रि॒गृ॒ह्णन्त्या॒शिषा॒ ऽऽशिषा॑ परिगृ॒ह्णन्ति॑ परिगृ॒ह्णन्त्या॒शिषा॑ य॒ज्ञ्ं ॅय॒ज्ञ् मा॒शिषा॑ परिगृ॒ह्णन्ति॑ परिगृ॒ह्णन्त्या॒शिषा॑ य॒ज्ञ्म् । \newline
51. प॒रि॒गृ॒ह्णन्तीति॑ परि - गृ॒ह्णन्ति॑ । \newline
52. आ॒शिषा॑ य॒ज्ञ्ं ॅय॒ज्ञ् मा॒शिषा॒ ऽऽशिषा॑ य॒ज्ञ्ं ॅयज॑मानो॒ यज॑मानो य॒ज्ञ् मा॒शिषा॒ ऽऽशिषा॑ य॒ज्ञ्ं ॅयज॑मानः । \newline
53. आ॒शिषेत्या᳚ - शिषा᳚ । \newline
54. य॒ज्ञ्ं ॅयज॑मानो॒ यज॑मानो य॒ज्ञ्ं ॅय॒ज्ञ्ं ॅयज॑मानः॒ परि॒ परि॒ यज॑मानो य॒ज्ञ्ं ॅय॒ज्ञ्ं ॅयज॑मानः॒ परि॑ । \newline
55. यज॑मानः॒ परि॒ परि॒ यज॑मानो॒ यज॑मानः॒ परि॑ गृह्णाति गृह्णाति॒ परि॒ यज॑मानो॒ यज॑मानः॒ परि॑ गृह्णाति । \newline
56. परि॑ गृह्णाति गृह्णाति॒ परि॒ परि॑ गृह्णाति॒ मनो॒ मनो॑ गृह्णाति॒ परि॒ परि॑ गृह्णाति॒ मनः॑ । \newline
57. गृ॒ह्णा॒ति॒ मनो॒ मनो॑ गृह्णाति गृह्णाति॒ मनो᳚ ऽस्यसि॒ मनो॑ गृह्णाति गृह्णाति॒ मनो॑ ऽसि । \newline
58. मनो᳚ ऽस्यसि॒ मनो॒ मनो॑ ऽसि प्राजाप॒त्यम् प्रा॑जाप॒त्य म॑सि॒ मनो॒ मनो॑ ऽसि प्राजाप॒त्यम् । \newline
59. अ॒सि॒ प्रा॒जा॒प॒त्यम् प्रा॑जाप॒त्य म॑स्यसि प्राजाप॒त्यम् मन॑सा॒ मन॑सा प्राजाप॒त्य म॑स्यसि प्राजाप॒त्यम् मन॑सा । \newline
60. प्रा॒जा॒प॒त्यम् मन॑सा॒ मन॑सा प्राजाप॒त्यम् प्रा॑जाप॒त्यम् मन॑सा मा मा॒ मन॑सा प्राजाप॒त्यम् प्रा॑जाप॒त्यम् मन॑सा मा । \newline
61. प्रा॒जा॒प॒त्यमिति॑ प्राजा - प॒त्यम् । \newline
\pagebreak
\markright{ TS 1.6.10.6  \hfill https://www.vedavms.in \hfill}

\section{ TS 1.6.10.6 }

\textbf{TS 1.6.10.6 } \newline
\textbf{Samhita Paata} \newline

मन॑सा मा भू॒तेनाऽऽवि॒शेत्या॑ह॒ मनो॒ वै प्रा॑जाप॒त्यं प्रा॑जाप॒त्यो य॒ज्ञो मन॑ ए॒व य॒ज्ञ्मा॒त्मन् ध॑त्ते॒ वाग॑स्यै॒न्द्री स॑पत्न॒क्षय॑णी वा॒चा मे᳚न्द्रि॒येणा-ऽऽवि॒शेत्या॑है॒न्द्री वै वाग्वाच॑-मे॒वैन्द्री- मा॒त्मन् ध॑त्ते ॥ \newline

\textbf{Pada Paata} \newline

मन॑सा । मा॒ । भू॒तेन॑ । एति॑ । वि॒श॒ । इति॑ । आ॒ह॒ । मनः॑ । वै । प्रा॒जा॒प॒त्यमिति॑ प्राजा - प॒त्यम् । प्रा॒जा॒प॒त्य इति॑ प्राजा - प॒त्यः । य॒ज्ञ्ः । मनः॑ । ए॒व । य॒ज्ञ्म् । आ॒त्मन्न् । ध॒त्ते॒ । वाक् । अ॒सि॒ । ऐ॒न्द्री । स॒प॒त्न॒क्षय॒णीति॑ सपत्न - क्षय॑णी । वा॒चा । मा॒ । इ॒न्द्रि॒येण॑ । एति॑ । वि॒श॒ । इति॑ । आ॒ह॒ । ऐ॒न्द्री । वै । वाक् । वाच᳚म् । ए॒व । ऐ॒न्द्रीम् । आ॒त्मन्न् । ध॒त्ते॒ ॥  \newline


\textbf{Krama Paata} \newline

मन॑सा मा । मा॒ भू॒तेन॑ । भू॒तेना । आ वि॑श । वि॒शेति॑ । इत्या॑ह । आ॒ह॒ मनः॑ । मनो॒ वै । वै प्रा॑जाप॒त्यम् । प्रा॒जा॒प॒त्यम् प्रा॑जाप॒त्यः । प्रा॒जा॒प॒त्यमिति॑ प्राजा - प॒त्यम् । प्रा॒जा॒प॒त्यो य॒ज्ञ्ः । प्रा॒जा॒प॒त्य इति॑ प्राजा - प॒त्यः । य॒ज्ञो मनः॑ । मन॑ ए॒व । ए॒व य॒ज्ञ्म् । य॒ज्ञ्मा॒त्मन्न् । आ॒त्मन् ध॑त्ते । ध॒त्ते॒ वाक् । वाग॑सि । अ॒स्यै॒न्द्री । ऐ॒न्द्री स॑पत्न॒क्षय॑णी । स॒प॒त्न॒क्षय॑णी वा॒चा । स॒प॒त्न॒क्षय॒णीति॑ सपत्न - क्षय॑णी । वा॒चा मा᳚ । मे॒न्द्रि॒येण॑ । इ॒न्द्रि॒येणा । आ वि॑श । वि॒शेति॑ । इत्या॑ह । आ॒है॒न्द्री । ऐ॒न्द्री वै । वै वाक् । वाग् वाच᳚म् । वाच॑मे॒व । ए॒वैन्द्रीम् । ऐ॒न्द्रीमा॒त्मन्न् । आ॒त्मन् ध॑त्ते । ध॒त्त॒ इति॑ धत्ते । \newline

\textbf{Jatai Paata} \newline

1. मन॑सा मा मा॒ मन॑सा॒ मन॑सा मा । \newline
2. मा॒ भू॒तेन॑ भू॒तेन॑ मा मा भू॒तेन॑ । \newline
3. भू॒तेना भू॒तेन॑ भू॒तेना । \newline
4. आ वि॑श वि॒शा वि॑श । \newline
5. वि॒शे तीति॑ विश वि॒शे ति॑ । \newline
6. इत्या॑ हा॒हे तीत्या॑ह । \newline
7. आ॒ह॒ मनो॒ मन॑ आहाह॒ मनः॑ । \newline
8. मनो॒ वै वै मनो॒ मनो॒ वै । \newline
9. वै प्रा॑जाप॒त्यम् प्रा॑जाप॒त्यं ॅवै वै प्रा॑जाप॒त्यम् । \newline
10. प्रा॒जा॒प॒त्यम् प्रा॑जाप॒त्यः प्रा॑जाप॒त्यः प्रा॑जाप॒त्यम् प्रा॑जाप॒त्यम् प्रा॑जाप॒त्यः । \newline
11. प्रा॒जा॒प॒त्यमिति॑ प्राजा - प॒त्यम् । \newline
12. प्रा॒जा॒प॒त्यो य॒ज्ञो य॒ज्ञ्ः प्रा॑जाप॒त्यः प्रा॑जाप॒त्यो य॒ज्ञ्ः । \newline
13. प्रा॒जा॒प॒त्य इति॑ प्राजा - प॒त्यः । \newline
14. य॒ज्ञो मनो॒ मनो॑ य॒ज्ञो य॒ज्ञो मनः॑ । \newline
15. मन॑ ए॒वैव मनो॒ मन॑ ए॒व । \newline
16. ए॒व य॒ज्ञ्ं ॅय॒ज्ञ् मे॒वैव य॒ज्ञ्म् । \newline
17. य॒ज्ञ् मा॒त्मन् ना॒त्मन्. य॒ज्ञ्ं ॅय॒ज्ञ् मा॒त्मन्न् । \newline
18. आ॒त्मन् ध॑त्ते धत्त आ॒त्मन् ना॒त्मन् ध॑त्ते । \newline
19. ध॒त्ते॒ वाग् वाग् ध॑त्ते धत्ते॒ वाक् । \newline
20. वाग॑स्यसि॒ वाग् वाग॑सि । \newline
21. अ॒स्यै॒ न्द्र्यै᳚(1॒)न्द्र्य॑स्य स्यै॒न्द्री । \newline
22. ऐ॒न्द्री स॑पत्न॒क्षय॑णी सपत्न॒क्षय॑ण् यै॒न्द्र्यै᳚न्द्री स॑पत्न॒क्षय॑णी । \newline
23. स॒प॒त्न॒क्षय॑णी वा॒चा वा॒चा स॑पत्न॒क्षय॑णी सपत्न॒क्षय॑णी वा॒चा । \newline
24. स॒प॒त्न॒क्षय॒णीति॑ सपत्न - क्षय॑णी । \newline
25. वा॒चा मा॑ मा वा॒चा वा॒चा मा᳚ । \newline
26. मे॒न्द्रि॒येणे᳚ न्द्रि॒येण॑ मा मेन्द्रि॒येण॑ । \newline
27. इ॒न्द्रि॒ये णेन्द्रि॒येणे᳚ न्द्रि॒येणा । \newline
28. आ वि॑श वि॒शा वि॑श । \newline
29. वि॒शे तीति॑ विश वि॒शे ति॑ । \newline
30. इत्या॑हा॒हे तीत्या॑ह । \newline
31. आ॒है॒न्द्र्यै᳚(1॒)न्द्र्या॑हा है॒न्द्री । \newline
32. ऐ॒न्द्री वै वा ऐ॒न्द्र्यै᳚न्द्री वै । \newline
33. वै वाग् वाग् वै वै वाक् । \newline
34. वाग् वाचं॒ ॅवाचं॒ ॅवाग् वाग् वाच᳚म् । \newline
35. वाच॑ मे॒वैव वाचं॒ ॅवाच॑ मे॒व । \newline
36. ए॒वैन्द्री मै॒न्द्री मे॒वै वैन्द्रीम् । \newline
37. ऐ॒न्द्री मा॒त्मन् ना॒त्मन् नै॒न्द्री मै॒न्द्री मा॒त्मन्न् । \newline
38. आ॒त्मन् ध॑त्ते धत्त आ॒त्मन् ना॒त्मन् ध॑त्ते । \newline
39. ध॒त्त॒ इति॑ धत्ते । \newline

\textbf{Ghana Paata } \newline

1. मन॑सा मा मा॒ मन॑सा॒ मन॑सा मा भू॒तेन॑ भू॒तेन॑ मा॒ मन॑सा॒ मन॑सा मा भू॒तेन॑ । \newline
2. मा॒ भू॒तेन॑ भू॒तेन॑ मा मा भू॒तेना भू॒तेन॑ मा मा भू॒तेना । \newline
3. भू॒तेना भू॒तेन॑ भू॒तेना वि॑श वि॒शा भू॒तेन॑ भू॒तेना वि॑श । \newline
4. आ वि॑श वि॒शा वि॒शे तीति॑ वि॒शा वि॒शे ति॑ । \newline
5. वि॒शे तीति॑ विश वि॒शे त्या॑हा॒हे ति॑ विश वि॒शे त्या॑ह । \newline
6. इत्या॑हा॒हे तीत्या॑ह॒ मनो॒ मन॑ आ॒हे तीत्या॑ह॒ मनः॑ । \newline
7. आ॒ह॒ मनो॒ मन॑ आहाह॒ मनो॒ वै वै मन॑ आहाह॒ मनो॒ वै । \newline
8. मनो॒ वै वै मनो॒ मनो॒ वै प्रा॑जाप॒त्यम् प्रा॑जाप॒त्यं ॅवै मनो॒ मनो॒ वै प्रा॑जाप॒त्यम् । \newline
9. वै प्रा॑जाप॒त्यम् प्रा॑जाप॒त्यं ॅवै वै प्रा॑जाप॒त्यम् प्रा॑जाप॒त्यः प्रा॑जाप॒त्यः प्रा॑जाप॒त्यं ॅवै वै प्रा॑जाप॒त्यम् प्रा॑जाप॒त्यः । \newline
10. प्रा॒जा॒प॒त्यम् प्रा॑जाप॒त्यः प्रा॑जाप॒त्यः प्रा॑जाप॒त्यम् प्रा॑जाप॒त्यम् प्रा॑जाप॒त्यो य॒ज्ञो य॒ज्ञ्ः प्रा॑जाप॒त्यः प्रा॑जाप॒त्यम् प्रा॑जाप॒त्यम् प्रा॑जाप॒त्यो य॒ज्ञ्ः । \newline
11. प्रा॒जा॒प॒त्यमिति॑ प्राजा - प॒त्यम् । \newline
12. प्रा॒जा॒प॒त्यो य॒ज्ञो य॒ज्ञ्ः प्रा॑जाप॒त्यः प्रा॑जाप॒त्यो य॒ज्ञो मनो॒ मनो॑ य॒ज्ञ्ः प्रा॑जाप॒त्यः प्रा॑जाप॒त्यो य॒ज्ञो मनः॑ । \newline
13. प्रा॒जा॒प॒त्य इति॑ प्राजा - प॒त्यः । \newline
14. य॒ज्ञो मनो॒ मनो॑ य॒ज्ञो य॒ज्ञो मन॑ ए॒वैव मनो॑ य॒ज्ञो य॒ज्ञो मन॑ ए॒व । \newline
15. मन॑ ए॒वैव मनो॒ मन॑ ए॒व य॒ज्ञ्ं ॅय॒ज्ञ् मे॒व मनो॒ मन॑ ए॒व य॒ज्ञ्म् । \newline
16. ए॒व य॒ज्ञ्ं ॅय॒ज्ञ् मे॒वैव य॒ज्ञ् मा॒त्मन् ना॒त्मन्. य॒ज्ञ् मे॒वैव य॒ज्ञ् मा॒त्मन्न् । \newline
17. य॒ज्ञ् मा॒त्मन् ना॒त्मन्. य॒ज्ञ्ं ॅय॒ज्ञ् मा॒त्मन् ध॑त्ते धत्त आ॒त्मन्. य॒ज्ञ्ं ॅय॒ज्ञ् मा॒त्मन् ध॑त्ते । \newline
18. आ॒त्मन् ध॑त्ते धत्त आ॒त्मन् ना॒त्मन् ध॑त्ते॒ वाग् वाग् ध॑त्त आ॒त्मन् ना॒त्मन् ध॑त्ते॒ वाक् । \newline
19. ध॒त्ते॒ वाग् वाग् ध॑त्ते धत्ते॒ वाग॑स्यसि॒ वाग् ध॑त्ते धत्ते॒ वाग॑सि । \newline
20. वाग॑स्यसि॒ वाग् वाग॑स्यै॒ न्द्र्यै᳚(1॒)न्द्र्य॑सि॒ वाग् वाग॑स्यै॒न्द्री । \newline
21. अ॒स्यै॒ न्द्र्यै᳚(1॒)न्द्र्य॑ स्यस्यै॒न्द्री स॑पत्न॒क्षय॑णी सपत्न॒क्षय॑ण्यै॒ न्द्र्य॑स्यस्यै॒न्द्री स॑पत्न॒क्षय॑णी । \newline
22. ऐ॒न्द्री स॑पत्न॒क्षय॑णी सपत्न॒क्षय॑ण्यै॒ न्द्र्यै᳚न्द्री स॑पत्न॒क्षय॑णी वा॒चा वा॒चा स॑पत्न॒क्षय॑ण्यै॒ न्द्र्यै᳚न्द्री स॑पत्न॒क्षय॑णी वा॒चा । \newline
23. स॒प॒त्न॒क्षय॑णी वा॒चा वा॒चा स॑पत्न॒क्षय॑णी सपत्न॒क्षय॑णी वा॒चा मा॑ मा वा॒चा स॑पत्न॒क्षय॑णी सपत्न॒क्षय॑णी वा॒चा मा᳚ । \newline
24. स॒प॒त्न॒क्षय॒णीति॑ सपत्न - क्षय॑णी । \newline
25. वा॒चा मा॑ मा वा॒चा वा॒चा मे᳚न्द्रि॒येणे᳚ न्द्रि॒येण॑ मा वा॒चा वा॒चा मे᳚न्द्रि॒येण॑ । \newline
26. मे॒न्द्रि॒येणे᳚ न्द्रि॒येण॑ मा मेन्द्रि॒ये णेन्द्रि॒येण॑ मा मेन्द्रि॒येणा । \newline
27. इ॒न्द्रि॒येणे न्द्रि॒येणे᳚ न्द्रि॒येणा वि॑श वि॒शेन्द्रि॒येणे᳚ न्द्रि॒येणा वि॑श । \newline
28. आ वि॑श वि॒शा वि॒शे तीति॑ वि॒शा वि॒शे ति॑ । \newline
29. वि॒शे तीति॑ विश वि॒शे त्या॑हा॒हे ति॑ विश वि॒शे त्या॑ह । \newline
30. इत्या॑हा॒हे तीत्या॑ है॒न्द्र्यै᳚(1॒)न्द्र्या॑हे तीत्या॑है॒न्द्री । \newline
31. आ॒है॒न्द्र्यै᳚(1॒)न्द्र्या॑हा है॒न्द्री वै वा ऐ॒न्द्र्या॑ हाहै॒न्द्री वै । \newline
32. ऐ॒न्द्री वै वा ऐ॒न्द्र्यै᳚न्द्री वै वाग् वाग् वा ऐ॒न्द्र्यै᳚न्द्री वै वाक् । \newline
33. वै वाग् वाग् वै वै वाग् वाचं॒ ॅवाचं॒ ॅवाग् वै वै वाग् वाच᳚म् । \newline
34. वाग् वाचं॒ ॅवाचं॒ ॅवाग् वाग् वाच॑ मे॒वैव वाचं॒ ॅवाग् वाग् वाच॑ मे॒व । \newline
35. वाच॑ मे॒वैव वाचं॒ ॅवाच॑ मे॒वैन्द्री मै॒न्द्री मे॒व वाचं॒ ॅवाच॑ मे॒वैन्द्रीम् । \newline
36. ए॒वैन्द्री मै॒न्द्री मे॒वैवैन्द्री मा॒त्मन् ना॒त्मन् नै॒न्द्री मे॒वैवैन्द्री मा॒त्मन्न् । \newline
37. ऐ॒न्द्री मा॒त्मन् ना॒त्मन् नै॒न्द्री मै॒न्द्री मा॒त्मन् ध॑त्ते धत्त आ॒त्मन् नै॒न्द्री मै॒न्द्री मा॒त्मन् ध॑त्ते । \newline
38. आ॒त्मन् ध॑त्ते धत्त आ॒त्मन् ना॒त्मन् ध॑त्ते । \newline
39. ध॒त्त॒ इति॑ धत्ते । \newline
\pagebreak
\markright{ TS 1.6.11.1  \hfill https://www.vedavms.in \hfill}

\section{ TS 1.6.11.1 }

\textbf{TS 1.6.11.1 } \newline
\textbf{Samhita Paata} \newline

यो वै स॑प्तद॒शं प्र॒जाप॑तिं ॅय॒ज्ञ्म॒न्वाय॑त्तं॒ ॅवेद॒ प्रति॑ य॒ज्ञेन॑ तिष्ठति॒ न य॒ज्ञाद् भ्रꣳ॑शत॒ आ श्रा॑व॒येति॒ चतु॑रक्षर॒मस्तु॒ श्रौष॒डिति॒ चतु॑रक्षरं॒ ॅयजेति॒ द्व्य॑क्षरं॒ ॅये यजा॑मह॒ इति॒ पञ्चा᳚क्षरं द्व्यक्ष॒रो व॑षट्का॒र ए॒ष वै स॑प्तद॒शः प्र॒जाप॑तिर् य॒ज्ञ्म॒न्वाय॑त्तो॒ य ए॒वं ॅवेद॒ प्रति॑ य॒ज्ञेन॑ तिष्ठति॒ न य॒ज्ञाद्- भ्रꣳ॑शते॒ यो वै य॒ज्ञ्स्य॒ प्राय॑णं प्रति॒ष्ठा -[ ] \newline

\textbf{Pada Paata} \newline

यः । वै । स॒प्त॒द॒शमिति॑ सप्त - द॒शम् । प्र॒जाप॑ति॒मिति॑ प्र॒जा - प॒ति॒म् । य॒ज्ञ्म् । अ॒न्वाय॑त्त॒मित्य॑नु - आय॑त्तम् । वेद॑ । प्रतीति॑ । य॒ज्ञेन॑ । ति॒ष्ठ॒ति॒ । न । य॒ज्ञात् । भ्रꣳ॒॒श॒ते॒ । एति॑ । श्रा॒व॒य॒ । इति॑ । चतु॑रक्षर॒मिति॒ चतुः॑ - अ॒क्ष॒र॒म् । अस्तु॑ । श्रौष॑ट् । इति॑ । चतु॑रक्षर॒मिति॒ चतुः॑ - अ॒क्ष॒र॒म् । यज॑ । इति॑ । द्व्य॑क्षर॒मिति॒ द्वि - अ॒क्ष॒र॒म् । ये । यजा॑महे । इति॑ । पञ्चा᳚क्षर॒मिति॒ पञ्च॑ - अ॒क्ष॒र॒म् । द्व्य॒क्ष॒र इति॑ द्वि - अ॒क्ष॒रः । व॒ष॒ट्का॒र इति॑ वषट् - का॒रः । ए॒षः । वै । स॒प्त॒द॒श इति॑ सप्त - द॒शः । प्र॒जाप॑ति॒रिति॑ प्र॒जा - प॒तिः॒ । य॒ज्ञ्म् । अ॒न्वाय॑त्त॒ इत्य॑नु-आय॑त्तः । यः । ए॒वम् । वेद॑ । प्रतीति॑ । य॒ज्ञेन॑ । ति॒ष्ठ॒ति॒ । न । य॒ज्ञात् । भ्रꣳ॒॒श॒ते॒ । यः । वै । य॒ज्ञ्स्य॑ । प्राय॑ण॒मिति॑ प्र - अय॑नम् । प्र॒ति॒ष्ठामिति॑ प्रति - स्थाम् ।  \newline


\textbf{Krama Paata} \newline

यो वै । वै स॑प्तद॒शम् । स॒प्त॒द॒शम् प्र॒जाप॑तिम् । स॒प्त॒द॒शमिति॑ सप्त - द॒शम् । प्र॒जाप॑तिं ॅय॒ज्ञ्म् । प्र॒जाप॑ति॒मिति॑ प्र॒जा - प॒ति॒म् । य॒ज्ञ्म॒न्वाय॑त्तम् । अ॒न्वाय॑त्तं॒ ॅवेद॑ । अ॒न्वाय॑त्त॒मित्य॑नु - आय॑त्तम् । वेद॒ प्रति॑ । प्रति॑ य॒ज्ञेन॑ । य॒ज्ञेन॑ तिष्ठति । ति॒ष्ठ॒ति॒ न । न य॒ज्ञात् । य॒ज्ञाद् भ्रꣳ॑शते । भ्रꣳ॒॒श॒त॒ आ । आ श्रा॑वय । श्रा॒व॒येति॑ । इति॒ चतु॑रक्षरम् । चतु॑रक्षर॒मस्तु॑ । चतु॑रक्षर॒मिति॒ चतुः॑ - अ॒क्ष॒र॒म् । अस्तु॒ श्रौष॑ट् । श्रौष॒डिति॑ । इति॒ चतु॑रक्षरम् । चतु॑रक्षरं॒ ॅयज॑ । चतु॑रक्षर॒मिति॒ चतुः॑ - अ॒क्ष॒र॒म् । यजेति॑ । इति॒ द्व्य॑क्षरम् । द्व्य॑क्षरं॒ ॅये । द्व्य॑क्षर॒मिति॒ द्वि - अ॒क्ष॒र॒म् । ये यजा॑महे । यजा॑मह॒ इति॑ । इति॒ पञ्चा᳚क्षरम् । पञ्चा᳚क्षरम् द्व्यक्ष॒रः । पञ्चा᳚क्षर॒मिति॒ पञ्च॑ - अ॒क्ष॒र॒म् । द्व्य॒क्ष॒रो व॑षट्का॒रः । द्व्य॒क्ष॒र इति॑ द्वि - अ॒क्ष॒रः । व॒ष॒ट्का॒र ए॒षः । व॒ष॒ट्का॒र इति॑ वषट् - का॒रः । ए॒ष वै । वै स॑प्तद॒शः । स॒प्त॒द॒शः प्र॒जाप॑तिः । स॒प्त॒द॒श इति॑ सप्त - द॒शः । प्र॒जाप॑तिर्,य॒ज्ञ्म् । प्र॒जाप॑ति॒रिति॑ प्र॒जा - प॒तिः॒ । य॒ज्ञ्म॒न्वाय॑त्तः । अ॒न्वाय॑त्तो॒ यः । अ॒न्वाय॑त्त॒ इत्य॑नु - आय॑त्तः । य ए॒वम् । ए॒वं ॅवेद॑ । वेद॒ प्रति॑ । प्रति॑ य॒ज्ञेन॑ । य॒ज्ञेन॑ तिष्ठति । ति॒ष्ठ॒ति॒ न । न य॒ज्ञात् । य॒ज्ञाद् भ्रꣳ॑शते । भ्रꣳ॒॒श॒ते॒ यः । यो वै । वै य॒ज्ञ्स्य॑ । य॒ज्ञ्स्य॒ प्राय॑णम् । प्राय॑णम् प्रति॒ष्ठाम् । प्राय॑ण॒मिति॑ प्र - अय॑नम् । प्र॒ति॒ष्ठामु॒दय॑नम् । प्र॒ति॒ष्ठामिति॑ प्रति - स्थाम् \newline

\textbf{Jatai Paata} \newline

1. यो वै वै यो यो वै । \newline
2. वै स॑प्तद॒शꣳ स॑प्तद॒शं ॅवै वै स॑प्तद॒शम् । \newline
3. स॒प्त॒द॒शम् प्र॒जाप॑तिम् प्र॒जाप॑तिꣳ सप्तद॒शꣳ स॑प्तद॒शम् प्र॒जाप॑तिम् । \newline
4. स॒प्त॒द॒शमिति॑ सप्त - द॒शम् । \newline
5. प्र॒जाप॑तिं ॅय॒ज्ञ्ं ॅय॒ज्ञ्म् प्र॒जाप॑तिम् प्र॒जाप॑तिं ॅय॒ज्ञ्म् । \newline
6. प्र॒जाप॑ति॒मिति॑ प्र॒जा - प॒ति॒म् । \newline
7. य॒ज्ञ् म॒न्वाय॑त्त म॒न्वाय॑त्तं ॅय॒ज्ञ्ं ॅय॒ज्ञ् म॒न्वाय॑त्तम् । \newline
8. अ॒न्वाय॑त्तं॒ ॅवेद॒ वेदा॒न्वाय॑त्त म॒न्वाय॑त्तं॒ ॅवेद॑ । \newline
9. अ॒न्वाय॑त्त॒मित्य॑नु - आय॑त्तम् । \newline
10. वेद॒ प्रति॒ प्रति॒ वेद॒ वेद॒ प्रति॑ । \newline
11. प्रति॑ य॒ज्ञेन॑ य॒ज्ञेन॒ प्रति॒ प्रति॑ य॒ज्ञेन॑ । \newline
12. य॒ज्ञेन॑ तिष्ठति तिष्ठति य॒ज्ञेन॑ य॒ज्ञेन॑ तिष्ठति । \newline
13. ति॒ष्ठ॒ति॒ न न ति॑ष्ठति तिष्ठति॒ न । \newline
14. न य॒ज्ञाद् य॒ज्ञान् न न य॒ज्ञात् । \newline
15. य॒ज्ञाद् भ्र(ग्म्॑)शते भ्रꣳशते य॒ज्ञाद् य॒ज्ञाद् भ्र(ग्म्॑)शते । \newline
16. भ्र॒(ग्म्॒)श॒त॒ आ भ्र(ग्म्॑)शते भ्रꣳशत॒ आ । \newline
17. आ श्रा॑वय श्राव॒या श्रा॑वय । \newline
18. श्रा॒व॒ये तीति॑ श्रावय श्राव॒ये ति॑ । \newline
19. इति॒ चतु॑रक्षर॒म् चतु॑रक्षर॒ मितीति॒ चतु॑रक्षरम् । \newline
20. चतु॑रक्षर॒ मस्त्वस्तु॒ चतु॑रक्षर॒म् चतु॑रक्षर॒ मस्तु॑ । \newline
21. चतु॑रक्षर॒मिति॒ चतुः॑ - अ॒क्ष॒र॒म् । \newline
22. अस्तु॒ श्रौष॒ट् छ्रौष॒ डस्त्वस्तु॒ श्रौष॑ट् । \newline
23. श्रौष॒ डितीति॒ श्रौष॒ट् छ्रौष॒ डिति॑ । \newline
24. इति॒ चतु॑रक्षर॒म् चतु॑रक्षर॒ मितीति॒ चतु॑रक्षरम् । \newline
25. चतु॑रक्षरं॒ ॅयज॒ यज॒ चतु॑रक्षर॒म् चतु॑रक्षरं॒ ॅयज॑ । \newline
26. चतु॑रक्षर॒मिति॒ चतुः॑ - अ॒क्ष॒र॒म् । \newline
27. यजे तीति॒ यज॒ यजे ति॑ । \newline
28. इति॒ द्व्य॑क्षर॒म् द्व्य॑क्षर॒ मितीति॒ द्व्य॑क्षरम् । \newline
29. द्व्य॑क्षरं॒ ॅये ये द्व्य॑क्षर॒म् द्व्य॑क्षरं॒ ॅये । \newline
30. द्व्य॑क्षर॒मिति॒ द्वि - अ॒क्ष॒र॒म् । \newline
31. ये यजा॑महे॒ यजा॑महे॒ ये ये यजा॑महे । \newline
32. यजा॑मह॒ इतीति॒ यजा॑महे॒ यजा॑मह॒ इति॑ । \newline
33. इति॒ पञ्चा᳚क्षर॒म् पञ्चा᳚क्षर॒ मितीति॒ पञ्चा᳚क्षरम् । \newline
34. पञ्चा᳚क्षरम् द्व्यक्ष॒रो द्व्य॑क्ष॒रः पञ्चा᳚क्षर॒म् पञ्चा᳚क्षरम् द्व्यक्ष॒रः । \newline
35. पञ्चा᳚क्षर॒मिति॒ पञ्च॑ - अ॒क्ष॒र॒म् । \newline
36. द्व्य॒क्ष॒रो व॑षट्का॒रो व॑षट्का॒रो द्व्य॑क्ष॒रो द्व्य॑क्ष॒रो व॑षट्का॒रः । \newline
37. द्व्य॒क्ष॒र इति॑ द्वि - अ॒क्ष॒रः । \newline
38. व॒ष॒ट्का॒र ए॒ष ए॒ष व॑षट्का॒रो व॑षट्का॒र ए॒षः । \newline
39. व॒ष॒ट्का॒र इति॑ वषट् - का॒रः । \newline
40. ए॒ष वै वा ए॒ष ए॒ष वै । \newline
41. वै स॑प्तद॒शः स॑प्तद॒शो वै वै स॑प्तद॒शः । \newline
42. स॒प्त॒द॒शः प्र॒जाप॑तिः प्र॒जाप॑तिः सप्तद॒शः स॑प्तद॒शः प्र॒जाप॑तिः । \newline
43. स॒प्त॒द॒श इति॑ सप्त - द॒शः । \newline
44. प्र॒जाप॑तिर् य॒ज्ञ्ं ॅय॒ज्ञ्म् प्र॒जाप॑तिः प्र॒जाप॑तिर् य॒ज्ञ्म् । \newline
45. प्र॒जाप॑ति॒रिति॑ प्र॒जा - प॒तिः॒ । \newline
46. य॒ज्ञ् म॒न्वाय॑त्तो॒ ऽन्वाय॑त्तो य॒ज्ञ्ं ॅय॒ज्ञ् म॒न्वाय॑त्तः । \newline
47. अ॒न्वाय॑त्तो॒ यो यो᳚ ऽन्वाय॑त्तो॒ ऽन्वाय॑त्तो॒ यः । \newline
48. अ॒न्वाय॑त्त॒ इत्य॑नु - आय॑त्तः । \newline
49. य ए॒व मे॒वं ॅयो य ए॒वम् । \newline
50. ए॒वं ॅवेद॒ वेदै॒व मे॒वं ॅवेद॑ । \newline
51. वेद॒ प्रति॒ प्रति॒ वेद॒ वेद॒ प्रति॑ । \newline
52. प्रति॑ य॒ज्ञेन॑ य॒ज्ञेन॒ प्रति॒ प्रति॑ य॒ज्ञेन॑ । \newline
53. य॒ज्ञेन॑ तिष्ठति तिष्ठति य॒ज्ञेन॑ य॒ज्ञेन॑ तिष्ठति । \newline
54. ति॒ष्ठ॒ति॒ न न ति॑ष्ठति तिष्ठति॒ न । \newline
55. न य॒ज्ञाद् य॒ज्ञान् न न य॒ज्ञात् । \newline
56. य॒ज्ञाद् भ्र(ग्म्॑)शते भ्रꣳशते य॒ज्ञाद् य॒ज्ञाद् भ्र(ग्म्॑)शते । \newline
57. भ्र॒(ग्म्॒)श॒ते॒ यो यो भ्र(ग्म्॑)शते भ्रꣳशते॒ यः । \newline
58. यो वै वै यो यो वै । \newline
59. वै य॒ज्ञ्स्य॑ य॒ज्ञ्स्य॒ वै वै य॒ज्ञ्स्य॑ । \newline
60. य॒ज्ञ्स्य॒ प्राय॑ण॒म् प्राय॑णं ॅय॒ज्ञ्स्य॑ य॒ज्ञ्स्य॒ प्राय॑णम् । \newline
61. प्राय॑ण॒म् प्रति॒ष्ठाम् प्र॑ति॒ष्ठाम् प्राय॑ण॒म् प्राय॑ण॒म् प्रति॒ष्ठाम् । \newline
62. प्राय॑ण॒मिति॑ प्र - अय॑नम् । \newline
63. प्र॒ति॒ष्ठा मु॒दय॑न मु॒दय॑नम् प्रति॒ष्ठाम् प्र॑ति॒ष्ठा मु॒दय॑नम् । \newline
64. प्र॒ति॒ष्ठामिति॑ प्रति - स्थाम् । \newline

\textbf{Ghana Paata } \newline

1. यो वै वै यो यो वै स॑प्तद॒शꣳ स॑प्तद॒शं ॅवै यो यो वै स॑प्तद॒शम् । \newline
2. वै स॑प्तद॒शꣳ स॑प्तद॒शं ॅवै वै स॑प्तद॒शम् प्र॒जाप॑तिम् प्र॒जाप॑तिꣳ सप्तद॒शं ॅवै वै स॑प्तद॒शम् प्र॒जाप॑तिम् । \newline
3. स॒प्त॒द॒शम् प्र॒जाप॑तिम् प्र॒जाप॑तिꣳ सप्तद॒शꣳ स॑प्तद॒शम् प्र॒जाप॑तिं ॅय॒ज्ञ्ं ॅय॒ज्ञ्म् प्र॒जाप॑तिꣳ सप्तद॒शꣳ स॑प्तद॒शम् प्र॒जाप॑तिं ॅय॒ज्ञ्म् । \newline
4. स॒प्त॒द॒शमिति॑ सप्त - द॒शम् । \newline
5. प्र॒जाप॑तिं ॅय॒ज्ञ्ं ॅय॒ज्ञ्म् प्र॒जाप॑तिम् प्र॒जाप॑तिं ॅय॒ज्ञ् म॒न्वाय॑त्त म॒न्वाय॑त्तं ॅय॒ज्ञ्म् प्र॒जाप॑तिम् प्र॒जाप॑तिं ॅय॒ज्ञ् म॒न्वाय॑त्तम् । \newline
6. प्र॒जाप॑ति॒मिति॑ प्र॒जा - प॒ति॒म् । \newline
7. य॒ज्ञ् म॒न्वाय॑त्त म॒न्वाय॑त्तं ॅय॒ज्ञ्ं ॅय॒ज्ञ् म॒न्वाय॑त्तं॒ ॅवेद॒ वेदा॒न्वाय॑त्तं ॅय॒ज्ञ्ं ॅय॒ज्ञ् म॒न्वाय॑त्तं॒ ॅवेद॑ । \newline
8. अ॒न्वाय॑त्तं॒ ॅवेद॒ वेदा॒न्वाय॑त्त म॒न्वाय॑त्तं॒ ॅवेद॒ प्रति॒ प्रति॒ वेदा॒न्वाय॑त्त म॒न्वाय॑त्तं॒ ॅवेद॒ प्रति॑ । \newline
9. अ॒न्वाय॑त्त॒मित्य॑नु - आय॑त्तम् । \newline
10. वेद॒ प्रति॒ प्रति॒ वेद॒ वेद॒ प्रति॑ य॒ज्ञेन॑ य॒ज्ञेन॒ प्रति॒ वेद॒ वेद॒ प्रति॑ य॒ज्ञेन॑ । \newline
11. प्रति॑ य॒ज्ञेन॑ य॒ज्ञेन॒ प्रति॒ प्रति॑ य॒ज्ञेन॑ तिष्ठति तिष्ठति य॒ज्ञेन॒ प्रति॒ प्रति॑ य॒ज्ञेन॑ तिष्ठति । \newline
12. य॒ज्ञेन॑ तिष्ठति तिष्ठति य॒ज्ञेन॑ य॒ज्ञेन॑ तिष्ठति॒ न न ति॑ष्ठति य॒ज्ञेन॑ य॒ज्ञेन॑ तिष्ठति॒ न । \newline
13. ति॒ष्ठ॒ति॒ न न ति॑ष्ठति तिष्ठति॒ न य॒ज्ञाद् य॒ज्ञान् न ति॑ष्ठति तिष्ठति॒ न य॒ज्ञात् । \newline
14. न य॒ज्ञाद् य॒ज्ञान् न न य॒ज्ञाद् भ्र(ग्म्॑)शते भ्रꣳशते य॒ज्ञान् न न य॒ज्ञाद् भ्र(ग्म्॑)शते । \newline
15. य॒ज्ञाद् भ्र(ग्म्॑)शते भ्रꣳशते य॒ज्ञाद् य॒ज्ञाद् भ्र(ग्म्॑)शत॒ आ भ्र(ग्म्॑)शते य॒ज्ञाद् य॒ज्ञाद् भ्र(ग्म्॑)शत॒ आ । \newline
16. भ्र॒(ग्म्॒)श॒त॒ आ भ्र(ग्म्॑)शते भ्रꣳशत॒ आ श्रा॑वय श्राव॒या भ्र(ग्म्॑)शते भ्रꣳशत॒ आ श्रा॑वय । \newline
17. आ श्रा॑वय श्राव॒या श्रा॑व॒ये तीति॑ श्राव॒या श्रा॑व॒ये ति॑ । \newline
18. श्रा॒व॒ये तीति॑ श्रावय श्राव॒ये ति॒ चतु॑रक्षर॒म् चतु॑रक्षर॒ मिति॑ श्रावय श्राव॒ये ति॒ चतु॑रक्षरम् । \newline
19. इति॒ चतु॑रक्षर॒म् चतु॑रक्षर॒ मितीति॒ चतु॑रक्षर॒ मस्त्वस्तु॒ चतु॑रक्षर॒ मितीति॒ चतु॑रक्षर॒ मस्तु॑ । \newline
20. चतु॑रक्षर॒ मस्त्वस्तु॒ चतु॑रक्षर॒म् चतु॑रक्षर॒ मस्तु॒ श्रौष॒ट् छ्रौष॒डस्तु॒ चतु॑रक्षर॒म् चतु॑रक्षर॒ मस्तु॒ श्रौष॑ट् । \newline
21. चतु॑रक्षर॒मिति॒ चतुः॑ - अ॒क्ष॒र॒म् । \newline
22. अस्तु॒ श्रौष॒ट् छ्रौष॒ड स्त्वस्तु॒ श्रौष॒डितीति॒ श्रौष॒ड स्त्वस्तु॒ श्रौष॒डिति॑ । \newline
23. श्रौष॒डितीति॒ श्रौष॒ट् छ्रौष॒डिति॒ चतु॑रक्षर॒म् चतु॑रक्षर॒ मिति॒ श्रौष॒ट् छ्रौष॒डिति॒ चतु॑रक्षरम् । \newline
24. इति॒ चतु॑रक्षर॒म् चतु॑रक्षर॒ मितीति॒ चतु॑रक्षरं॒ ॅयज॒ यज॒ चतु॑रक्षर॒ मितीति॒ चतु॑रक्षरं॒ ॅयज॑ । \newline
25. चतु॑रक्षरं॒ ॅयज॒ यज॒ चतु॑रक्षर॒म् चतु॑रक्षरं॒ ॅयजे तीति॒ यज॒ चतु॑रक्षर॒म् चतु॑रक्षरं॒ ॅयजे ति॑ । \newline
26. चतु॑रक्षर॒मिति॒ चतुः॑ - अ॒क्ष॒र॒म् । \newline
27. यजे तीति॒ यज॒ यजे ति॒ द्व्य॑क्षर॒म् द्व्य॑क्षर॒ मिति॒ यज॒ यजे ति॒ द्व्य॑क्षरम् । \newline
28. इति॒ द्व्य॑क्षर॒म् द्व्य॑क्षर॒ मितीति॒ द्व्य॑क्षरं॒ ॅये ये द्व्य॑क्षर॒ मितीति॒ द्व्य॑क्षरं॒ ॅये । \newline
29. द्व्य॑क्षरं॒ ॅये ये द्व्य॑क्षर॒म् द्व्य॑क्षरं॒ ॅये यजा॑महे॒ यजा॑महे॒ ये द्व्य॑क्षर॒म् द्व्य॑क्षरं॒ ॅये यजा॑महे । \newline
30. द्व्य॑क्षर॒मिति॒ द्वि - अ॒क्ष॒र॒म् । \newline
31. ये यजा॑महे॒ यजा॑महे॒ ये ये यजा॑मह॒ इतीति॒ यजा॑महे॒ ये ये यजा॑मह॒ इति॑ । \newline
32. यजा॑मह॒ इतीति॒ यजा॑महे॒ यजा॑मह॒ इति॒ पञ्चा᳚क्षर॒म् पञ्चा᳚क्षर॒ मिति॒ यजा॑महे॒ यजा॑मह॒ इति॒ पञ्चा᳚क्षरम् । \newline
33. इति॒ पञ्चा᳚क्षर॒म् पञ्चा᳚क्षर॒ मितीति॒ पञ्चा᳚क्षरम् द्व्यक्ष॒रो द्व्य॑क्ष॒रः पञ्चा᳚क्षर॒ मितीति॒ पञ्चा᳚क्षरम् द्व्यक्ष॒रः । \newline
34. पञ्चा᳚क्षरम् द्व्यक्ष॒रो द्व्य॑क्ष॒रः पञ्चा᳚क्षर॒म् पञ्चा᳚क्षरम् द्व्यक्ष॒रो व॑षट्का॒रो व॑षट्का॒रो द्व्य॑क्ष॒रः पञ्चा᳚क्षर॒म् पञ्चा᳚क्षरम् द्व्यक्ष॒रो व॑षट्का॒रः । \newline
35. पञ्चा᳚क्षर॒मिति॒ पञ्च॑ - अ॒क्ष॒र॒म् । \newline
36. द्व्य॒क्ष॒रो व॑षट्का॒रो व॑षट्का॒रो द्व्य॑क्ष॒रो द्व्य॑क्ष॒रो व॑षट्का॒र ए॒ष ए॒ष व॑षट्का॒रो द्व्य॑क्ष॒रो द्व्य॑क्ष॒रो व॑षट्का॒र ए॒षः । \newline
37. द्व्य॒क्ष॒र इति॑ द्वि - अ॒क्ष॒रः । \newline
38. व॒ष॒ट्का॒र ए॒ष ए॒ष व॑षट्का॒रो व॑षट्का॒र ए॒ष वै वा ए॒ष व॑षट्का॒रो व॑षट्का॒र ए॒ष वै । \newline
39. व॒ष॒ट्का॒र इति॑ वषट् - का॒रः । \newline
40. ए॒ष वै वा ए॒ष ए॒ष वै स॑प्तद॒शः स॑प्तद॒शो वा ए॒ष ए॒ष वै स॑प्तद॒शः । \newline
41. वै स॑प्तद॒शः स॑प्तद॒शो वै वै स॑प्तद॒शः प्र॒जाप॑तिः प्र॒जाप॑तिः सप्तद॒शो वै वै स॑प्तद॒शः प्र॒जाप॑तिः । \newline
42. स॒प्त॒द॒शः प्र॒जाप॑तिः प्र॒जाप॑तिः सप्तद॒शः स॑प्तद॒शः प्र॒जाप॑तिर् य॒ज्ञ्ं ॅय॒ज्ञ्म् प्र॒जाप॑तिः सप्तद॒शः स॑प्तद॒शः प्र॒जाप॑तिर् य॒ज्ञ्म् । \newline
43. स॒प्त॒द॒श इति॑ सप्त - द॒शः । \newline
44. प्र॒जाप॑तिर् य॒ज्ञ्ं ॅय॒ज्ञ्म् प्र॒जाप॑तिः प्र॒जाप॑तिर् य॒ज्ञ् म॒न्वाय॑त्तो॒ ऽन्वाय॑त्तो य॒ज्ञ्म् प्र॒जाप॑तिः प्र॒जाप॑तिर् य॒ज्ञ् म॒न्वाय॑त्तः । \newline
45. प्र॒जाप॑ति॒रिति॑ प्र॒जा - प॒तिः॒ । \newline
46. य॒ज्ञ् म॒न्वाय॑त्तो॒ ऽन्वाय॑त्तो य॒ज्ञ्ं ॅय॒ज्ञ् म॒न्वाय॑त्तो॒ यो यो᳚ ऽन्वाय॑त्तो य॒ज्ञ्ं ॅय॒ज्ञ् म॒न्वाय॑त्तो॒ यः । \newline
47. अ॒न्वाय॑त्तो॒ यो यो᳚ ऽन्वाय॑त्तो॒ ऽन्वाय॑त्तो॒ य ए॒व मे॒वं ॅयो᳚ ऽन्वाय॑त्तो॒ ऽन्वाय॑त्तो॒ य ए॒वम् । \newline
48. अ॒न्वाय॑त्त॒ इत्य॑नु - आय॑त्तः । \newline
49. य ए॒व मे॒वं ॅयो य ए॒वं ॅवेद॒ वेदै॒वं ॅयो य ए॒वं ॅवेद॑ । \newline
50. ए॒वं ॅवेद॒ वेदै॒व मे॒वं ॅवेद॒ प्रति॒ प्रति॒ वेदै॒व मे॒वं ॅवेद॒ प्रति॑ । \newline
51. वेद॒ प्रति॒ प्रति॒ वेद॒ वेद॒ प्रति॑ य॒ज्ञेन॑ य॒ज्ञेन॒ प्रति॒ वेद॒ वेद॒ प्रति॑ य॒ज्ञेन॑ । \newline
52. प्रति॑ य॒ज्ञेन॑ य॒ज्ञेन॒ प्रति॒ प्रति॑ य॒ज्ञेन॑ तिष्ठति तिष्ठति य॒ज्ञेन॒ प्रति॒ प्रति॑ य॒ज्ञेन॑ तिष्ठति । \newline
53. य॒ज्ञेन॑ तिष्ठति तिष्ठति य॒ज्ञेन॑ य॒ज्ञेन॑ तिष्ठति॒ न न ति॑ष्ठति य॒ज्ञेन॑ य॒ज्ञेन॑ तिष्ठति॒ न । \newline
54. ति॒ष्ठ॒ति॒ न न ति॑ष्ठति तिष्ठति॒ न य॒ज्ञाद् य॒ज्ञान् न ति॑ष्ठति तिष्ठति॒ न य॒ज्ञात् । \newline
55. न य॒ज्ञाद् य॒ज्ञान् न न य॒ज्ञाद् भ्र(ग्म्॑)शते भ्रꣳशते य॒ज्ञान् न न य॒ज्ञाद् भ्र(ग्म्॑)शते । \newline
56. य॒ज्ञाद् भ्र(ग्म्॑)शते भ्रꣳशते य॒ज्ञाद् य॒ज्ञाद् भ्र(ग्म्॑)शते॒ यो यो भ्र(ग्म्॑)शते य॒ज्ञाद् य॒ज्ञाद् भ्र(ग्म्॑)शते॒ यः । \newline
57. भ्र॒(ग्म्॒)श॒ते॒ यो यो भ्र(ग्म्॑)शते भ्रꣳशते॒ यो वै वै यो भ्र(ग्म्॑)शते भ्रꣳशते॒ यो वै । \newline
58. यो वै वै यो यो वै य॒ज्ञ्स्य॑ य॒ज्ञ्स्य॒ वै यो यो वै य॒ज्ञ्स्य॑ । \newline
59. वै य॒ज्ञ्स्य॑ य॒ज्ञ्स्य॒ वै वै य॒ज्ञ्स्य॒ प्राय॑ण॒म् प्राय॑णं ॅय॒ज्ञ्स्य॒ वै वै य॒ज्ञ्स्य॒ प्राय॑णम् । \newline
60. य॒ज्ञ्स्य॒ प्राय॑ण॒म् प्राय॑णं ॅय॒ज्ञ्स्य॑ य॒ज्ञ्स्य॒ प्राय॑ण॒म् प्रति॒ष्ठाम् प्र॑ति॒ष्ठाम् 
प्राय॑णं ॅय॒ज्ञ्स्य॑ य॒ज्ञ्स्य॒ प्राय॑ण॒म् प्रति॒ष्ठाम् । \newline
61. प्राय॑ण॒म् प्रति॒ष्ठाम् प्र॑ति॒ष्ठाम् प्राय॑ण॒म् प्राय॑ण॒म् प्रति॒ष्ठा मु॒दय॑न मु॒दय॑नम् प्रति॒ष्ठाम् प्राय॑ण॒म् प्राय॑ण॒म् प्रति॒ष्ठा मु॒दय॑नम् । \newline
62. प्राय॑ण॒मिति॑ प्र - अय॑नम् । \newline
63. प्र॒ति॒ष्ठा मु॒दय॑न मु॒दय॑नम् प्रति॒ष्ठाम् प्र॑ति॒ष्ठा मु॒दय॑नं॒ ॅवेद॒ वेदो॒दय॑नम् प्रति॒ष्ठाम् प्र॑ति॒ष्ठा मु॒दय॑नं॒ ॅवेद॑ । \newline
64. प्र॒ति॒ष्ठामिति॑ प्रति - स्थाम् । \newline
\pagebreak
\markright{ TS 1.6.11.2  \hfill https://www.vedavms.in \hfill}

\section{ TS 1.6.11.2 }

\textbf{TS 1.6.11.2 } \newline
\textbf{Samhita Paata} \newline

मु॒दय॑नं॒ ॅवेद॒ प्रति॑ष्ठिते॒नारि॑ष्टेन य॒ज्ञेन॑ सꣳ॒॒स्थां ग॑च्छ॒त्याश्रा॑व॒यास्तु॒ श्रौष॒ड्यज॒ ये यजा॑महे वषट्का॒र ए॒तद्वै य॒ज्ञ्स्य॒ प्राय॑णमे॒षा प्र॑ति॒ष्ठैतदु॒दय॑नं॒ ॅय ए॒वं ॅवेद॒ प्रति॑ष्ठिते॒नाऽरि॑ष्टेन य॒ज्ञेन॑ सꣳ॒॒स्थां ग॑च्छति॒ यो वै सू॒नृता॑यै॒ दोहं॒ ॅवेद॑ दु॒ह ए॒वैनां᳚ ॅय॒ज्ञो वै सू॒नृता ऽऽ श्रा॑व॒येत्यैवैना॑-मह्व॒दस्तु॒ - [ ] \newline

\textbf{Pada Paata} \newline

उ॒दय॑न॒मित्यु॑त् - अय॑नम् । वेद॑ । प्रति॑ष्ठिते॒नेति॒ प्रति॑ - स्थि॒ते॒न॒ । अरि॑ष्टेन । य॒ज्ञेन॑ । सꣳ॒॒स्थामिति॑ सं - स्थाम् । ग॒च्छ॒ति॒ । एति॑ । श्रा॒व॒य॒ । अस्तु॑ । श्रौष॑ट् । यज॑ । ये । यजा॑महे । व॒ष॒ट्का॒र इति॑ वषट् - का॒रः । ए॒तत् । वै । य॒ज्ञ्स्य॑ । प्राय॑ण॒मिति॑ प्र - अय॑नम् । ए॒षा । प्र॒ति॒ष्ठेति॑ प्रति - स्था । ए॒तत् । उ॒दय॑न॒मित्यु॑त् - अय॑नम् । यः । ए॒वम् । वेद॑ । प्रति॑ष्ठिते॒नेति॒ प्रति॑-स्थि॒ते॒न॒ । अरि॑ष्टेन । य॒ज्ञेन॑ । सꣳ॒॒स्थामिति॑ सं - स्थाम् । ग॒च्छ॒ति॒ । यः । वै । सू॒नृता॑यै । दोह᳚म् । वेद॑ । दु॒हे । ए॒व । ए॒ना॒म् । य॒ज्ञ्ः । वै । सू॒नृता᳚ । एति॑ । श्रा॒व॒य॒ । इति॑ । एति॑ । ए॒व । ए॒ना॒म् । अ॒ह्व॒त् । अस्तु॑ ।  \newline


\textbf{Krama Paata} \newline

उ॒दय॑नं॒ ॅवेद॑ । उ॒दय॑न॒मित्यु॑त् - अय॑नम् । वेद॒ प्रति॑ष्ठितेन । प्रति॑ष्ठिते॒नारि॑ष्टेन । प्रति॑ष्ठिते॒नेति॒ प्रति॑ - स्थि॒ते॒न॒ । अरि॑ष्टेन य॒ज्ञेन॑ । य॒ज्ञेन॑ सꣳ॒॒स्थाम् । सꣳ॒॒स्थाम् ग॑च्छति । सꣳ॒॒स्थामिति॑ सम् - स्थाम् । ग॒च्छ॒त्या । आ श्रा॑वय । श्रा॒व॒यास्तु॑ । अस्तु॒ श्रौष॑ट् । श्रौष॒ड् यज॑ । यज॒ ये । ये यजा॑महे । यजा॑महे वषट्का॒रः । व॒ष॒ट्का॒र ए॒तत् । व॒ष॒ट्का॒र इति॑ वषट् - का॒रः । ए॒तद् वै । वै य॒ज्ञ्स्य॑ । य॒ज्ञ्स्य॒ प्राय॑णम् । प्राय॑णमे॒षा । प्राय॑ण॒मिति॑ प्र - अय॑नम् । ए॒षा प्र॑ति॒ष्ठा । प्र॒ति॒ष्ठैतत् । प्र॒ति॒ष्ठेति॑ प्रति - स्था । ए॒तदु॒दय॑नम् । उ॒दय॑नं॒ ॅयः । उ॒दय॑न॒मित्यु॑त् - अय॑नम् । य ए॒वम् । ए॒वं ॅवेद॑ । वेद॒ प्रति॑ष्ठितेन । प्रति॑ष्ठिते॒नारि॑ष्टेन । प्रति॑ष्ठिते॒नेति॒ प्रति॑ - स्थि॒ते॒न॒ । अरि॑ष्टेन य॒ज्ञेन॑ । य॒ज्ञेन॑ सꣳ॒॒स्थाम् । सꣳ॒॒स्थाम् ग॑च्छति । सꣳ॒॒स्थामिति॑ सम् - स्थाम् । ग॒च्छ॒ति॒ यः । यो वै । वै सू॒नृता॑यै । सू॒नृता॑यै॒ दोह᳚म् । दोहं॒ ॅवेद॑ । वेद॑ दु॒हे । दु॒ह ए॒व । ए॒वैना᳚म् । ए॒नां॒ ॅय॒ज्ञ्ः । य॒ज्ञो वै । वै सू॒नृता᳚ । सू॒नृता । आ श्रा॑वय । श्रा॒व॒येति॑ । इत्या । ऐव । ए॒वैना᳚म् । ए॒ना॒म॒ह्व॒त्॒ । अ॒ह्व॒दस्तु॑ । अस्तु॒ श्रौष॑ट् \newline

\textbf{Jatai Paata} \newline

1. उ॒दय॑नं॒ ॅवेद॒ वेदो॒दय॑न मु॒दय॑नं॒ ॅवेद॑ । \newline
2. उ॒दय॑न॒मित्यु॑त् - अय॑नम् । \newline
3. वेद॒ प्रति॑ष्ठितेन॒ प्रति॑ष्ठितेन॒ वेद॒ वेद॒ प्रति॑ष्ठितेन । \newline
4. प्रति॑ष्ठिते॒ नारि॑ष्टे॒ नारि॑ष्टेन॒ प्रति॑ष्ठितेन॒ प्रति॑ष्ठिते॒ नारि॑ष्टेन । \newline
5. प्रति॑ष्ठिते॒नेति॒ प्रति॑ - स्थि॒ते॒न॒ । \newline
6. अरि॑ष्टेन य॒ज्ञेन॑ य॒ज्ञे नारि॑ष्टे॒ नारि॑ष्टेन य॒ज्ञेन॑ । \newline
7. य॒ज्ञेन॑ स॒(ग्ग्॒)स्थाꣳ स॒(ग्ग्॒)स्थां ॅय॒ज्ञेन॑ य॒ज्ञेन॑ स॒(ग्ग्॒)स्थाम् । \newline
8. स॒(ग्ग्॒)स्थाम् ग॑च्छति गच्छति स॒(ग्ग्॒)स्थाꣳ स॒(ग्ग्॒)स्थाम् ग॑च्छति । \newline
9. स॒(ग्ग्॒)स्थामिति॑ सं - स्थाम् । \newline
10. ग॒च्छ॒त्या ग॑च्छति गच्छ॒त्या । \newline
11. आ श्रा॑वय श्राव॒या श्रा॑वय । \newline
12. श्रा॒व॒या स्त्वस्तु॑ श्रावय श्राव॒यास्तु॑ । \newline
13. अस्तु॒ श्रौष॒ट् छ्रौष॒ डस्त्वस्तु॒ श्रौष॑ट् । \newline
14. श्रौष॒ड् यज॒ यज॒ श्रौष॒ट् छ्रौष॒ड् यज॑ । \newline
15. यज॒ ये ये यज॒ यज॒ ये । \newline
16. ये यजा॑महे॒ यजा॑महे॒ ये ये यजा॑महे । \newline
17. यजा॑महे वषट्का॒रो व॑षट्का॒रो यजा॑महे॒ यजा॑महे वषट्का॒रः । \newline
18. व॒ष॒ट्का॒र ए॒तदे॒तद् व॑षट्का॒रो व॑षट्का॒र ए॒तत् । \newline
19. व॒ष॒ट्का॒र इति॑ वषट् - का॒रः । \newline
20. ए॒तद् वै वा ए॒तदे॒तद् वै । \newline
21. वै य॒ज्ञ्स्य॑ य॒ज्ञ्स्य॒ वै वै य॒ज्ञ्स्य॑ । \newline
22. य॒ज्ञ्स्य॒ प्राय॑ण॒म् प्राय॑णं ॅय॒ज्ञ्स्य॑ य॒ज्ञ्स्य॒ प्राय॑णम् । \newline
23. प्राय॑ण मे॒षैषा प्राय॑ण॒म् प्राय॑ण मे॒षा । \newline
24. प्राय॑ण॒मिति॑ प्र - अय॑नम् । \newline
25. ए॒षा प्र॑ति॒ष्ठा प्र॑ति॒ष्ठैषैषा प्र॑ति॒ष्ठा । \newline
26. प्र॒ति॒ष्ठै तदे॒तत् प्र॑ति॒ष्ठा प्र॑ति॒ष्ठैतत् । \newline
27. प्र॒ति॒ष्ठेति॑ प्रति - स्था । \newline
28. ए॒त दु॒दय॑न मु॒दय॑न मे॒त दे॒त दु॒दय॑नम् । \newline
29. उ॒दय॑नं॒ ॅयो य उ॒दय॑न मु॒दय॑नं॒ ॅयः । \newline
30. उ॒दय॑न॒मित्यु॑त् - अय॑नम् । \newline
31. य ए॒व मे॒वं ॅयो य ए॒वम् । \newline
32. ए॒वं ॅवेद॒ वेदै॒व मे॒वं ॅवेद॑ । \newline
33. वेद॒ प्रति॑ष्ठितेन॒ प्रति॑ष्ठितेन॒ वेद॒ वेद॒ प्रति॑ष्ठितेन । \newline
34. प्रति॑ष्ठिते॒ नारि॑ष्टे॒ नारि॑ष्टेन॒ प्रति॑ष्ठितेन॒ प्रति॑ष्ठिते॒ नारि॑ष्टेन । \newline
35. प्रति॑ष्ठिते॒नेति॒ प्रति॑ - स्थि॒ते॒न॒ । \newline
36. अरि॑ष्टेन य॒ज्ञेन॑ य॒ज्ञे नारि॑ष्टे॒ नारि॑ष्टेन य॒ज्ञेन॑ । \newline
37. य॒ज्ञेन॑ स॒(ग्ग्॒)स्थाꣳ स॒(ग्ग्॒)स्थां ॅय॒ज्ञेन॑ य॒ज्ञेन॑ स॒(ग्ग्॒)स्थाम् । \newline
38. स॒(ग्ग्॒)स्थाम् ग॑च्छति गच्छति स॒(ग्ग्॒)स्थाꣳ स॒(ग्ग्॒)स्थाम् ग॑च्छति । \newline
39. स॒(ग्ग्॒)स्थामिति॑ सं - स्थाम् । \newline
40. ग॒च्छ॒ति॒ यो यो ग॑च्छति गच्छति॒ यः । \newline
41. यो वै वै यो यो वै । \newline
42. वै सू॒नृता॑यै सू॒नृता॑यै॒ वै वै सू॒नृता॑यै । \newline
43. सू॒नृता॑यै॒ दोह॒म् दोह(ग्म्॑) सू॒नृता॑यै सू॒नृता॑यै॒ दोह᳚म् । \newline
44. दोहं॒ ॅवेद॒ वेद॒ दोह॒म् दोहं॒ ॅवेद॑ । \newline
45. वेद॑ दु॒हे दु॒हे वेद॒ वेद॑ दु॒हे । \newline
46. दु॒ह ए॒वैव दु॒हे दु॒ह ए॒व । \newline
47. ए॒वैना॑ मेना मे॒वै वैना᳚म् । \newline
48. ए॒नां॒ ॅय॒ज्ञो य॒ज्ञ् ए॑ना मेनां ॅय॒ज्ञ्ः । \newline
49. य॒ज्ञो वै वै य॒ज्ञो य॒ज्ञो वै । \newline
50. वै सू॒नृता॑ सू॒नृता॒ वै वै सू॒नृता᳚ । \newline
51. सू॒नृता ऽऽसू॒नृता॑ सू॒नृता । \newline
52. आ श्रा॑वय श्राव॒या श्रा॑वय । \newline
53. श्रा॒व॒ये तीति॑ श्रावय श्राव॒ये ति॑ । \newline
54. इत्येतीत्या । \newline
55. ऐवैवैव । \newline
56. ए॒वैना॑ मेना मे॒वै वैना᳚म् । \newline
57. ए॒ना॒ म॒ह्व॒ द॒ह्व॒ दे॒ना॒ मे॒ना॒ म॒ह्व॒त् । \newline
58. अ॒ह्व॒ दस्त्व स्त्व॑ह्व दह्व॒ दस्तु॑ । \newline
59. अस्तु॒ श्रौष॒ट् छ्रौष॒ डस्त्वस्तु॒ श्रौष॑ट् । \newline

\textbf{Ghana Paata } \newline

1. उ॒दय॑नं॒ ॅवेद॒ वेदो॒दय॑न मु॒दय॑नं॒ ॅवेद॒ प्रति॑ष्ठितेन॒ प्रति॑ष्ठितेन॒ वेदो॒दय॑न मु॒दय॑नं॒ ॅवेद॒ प्रति॑ष्ठितेन । \newline
2. उ॒दय॑न॒मित्यु॑त् - अय॑नम् । \newline
3. वेद॒ प्रति॑ष्ठितेन॒ प्रति॑ष्ठितेन॒ वेद॒ वेद॒ प्रति॑ष्ठिते॒ नारि॑ष्टे॒नारि॑ष्टेन॒ प्रति॑ष्ठितेन॒ वेद॒ वेद॒ प्रति॑ष्ठिते॒ नारि॑ष्टेन । \newline
4. प्रति॑ष्ठिते॒ नारि॑ष्टे॒नारि॑ष्टेन॒ प्रति॑ष्ठितेन॒ प्रति॑ष्ठिते॒ नारि॑ष्टेन य॒ज्ञेन॑ य॒ज्ञेनारि॑ष्टेन॒ प्रति॑ष्ठितेन॒ प्रति॑ष्ठिते॒नारि॑ष्टेन य॒ज्ञेन॑ । \newline
5. प्रति॑ष्ठिते॒नेति॒ प्रति॑ - स्थि॒ते॒न॒ । \newline
6. अरि॑ष्टेन य॒ज्ञेन॑ य॒ज्ञेनारि॑ष्टे॒नारि॑ष्टेन य॒ज्ञेन॑ स॒(ग्ग्॒)स्थाꣳ स॒(ग्ग्॒)स्थां ॅय॒ज्ञेनारि॑ष्टे॒नारि॑ष्टेन य॒ज्ञेन॑ स॒(ग्ग्॒)स्थाम् । \newline
7. य॒ज्ञेन॑ स॒(ग्ग्॒)स्थाꣳ स॒(ग्ग्॒)स्थां ॅय॒ज्ञेन॑ य॒ज्ञेन॑ स॒(ग्ग्॒)स्थाम् ग॑च्छति गच्छति स॒(ग्ग्॒)स्थां ॅय॒ज्ञेन॑ य॒ज्ञेन॑ स॒(ग्ग्॒)स्थाम् ग॑च्छति । \newline
8. स॒(ग्ग्॒)स्थाम् ग॑च्छति गच्छति स॒(ग्ग्॒)स्थाꣳ स॒(ग्ग्॒)स्थाम् ग॑च्छ॒त्या ग॑च्छति स॒(ग्ग्॒)स्थाꣳ स॒(ग्ग्॒)स्थाम् ग॑च्छ॒त्या । \newline
9. स॒(ग्ग्॒)स्थामिति॑ सं - स्थाम् । \newline
10. ग॒च्छ॒त्या ग॑च्छति गच्छ॒त्या श्रा॑वय श्राव॒या ग॑च्छति गच्छ॒त्या श्रा॑वय । \newline
11. आ श्रा॑वय श्राव॒या श्रा॑व॒यास्त्वस्तु॑ श्राव॒या श्रा॑व॒यास्तु॑ । \newline
12. श्रा॒व॒यास्त्वस्तु॑ श्रावय श्राव॒यास्तु॒ श्रौष॒ट् छ्रौष॒डस्तु॑ श्रावय श्राव॒यास्तु॒ श्रौष॑ट् । \newline
13. अस्तु॒ श्रौष॒ट् छ्रौष॒डस्त्वस्तु॒ श्रौष॒ड् यज॒ यज॒ श्रौष॒डस्त्वस्तु॒ श्रौष॒ड् यज॑ । \newline
14. श्रौष॒ड् यज॒ यज॒ श्रौष॒ट् छ्रौष॒ड् यज॒ ये ये यज॒ श्रौष॒ट् छ्रौष॒ड् यज॒ ये । \newline
15. यज॒ ये ये यज॒ यज॒ ये यजा॑महे॒ यजा॑महे॒ ये यज॒ यज॒ ये यजा॑महे । \newline
16. ये यजा॑महे॒ यजा॑महे॒ ये ये यजा॑महे वषट्का॒रो व॑षट्का॒रो यजा॑महे॒ ये ये यजा॑महे वषट्का॒रः । \newline
17. यजा॑महे वषट्का॒रो व॑षट्का॒रो यजा॑महे॒ यजा॑महे वषट्का॒र ए॒तदे॒तद् व॑षट्का॒रो यजा॑महे॒ यजा॑महे वषट्का॒र ए॒तत् । \newline
18. व॒ष॒ट्का॒र ए॒तदे॒तद् व॑षट्का॒रो व॑षट्का॒र ए॒तद् वै वा ए॒तद् व॑षट्का॒रो व॑षट्का॒र ए॒तद् वै । \newline
19. व॒ष॒ट्का॒र इति॑ वषट् - का॒रः । \newline
20. ए॒तद् वै वा ए॒तदे॒तद् वै य॒ज्ञ्स्य॑ य॒ज्ञ्स्य॒ वा ए॒तदे॒तद् वै य॒ज्ञ्स्य॑ । \newline
21. वै य॒ज्ञ्स्य॑ य॒ज्ञ्स्य॒ वै वै य॒ज्ञ्स्य॒ प्राय॑ण॒म् प्राय॑णं ॅय॒ज्ञ्स्य॒ वै वै य॒ज्ञ्स्य॒ प्राय॑णम् । \newline
22. य॒ज्ञ्स्य॒ प्राय॑ण॒म् प्राय॑णं ॅय॒ज्ञ्स्य॑ य॒ज्ञ्स्य॒ प्राय॑ण मे॒षैषा प्राय॑णं ॅय॒ज्ञ्स्य॑ य॒ज्ञ्स्य॒ प्राय॑ण मे॒षा । \newline
23. प्राय॑ण मे॒षैषा प्राय॑ण॒म् प्राय॑ण मे॒षा प्र॑ति॒ष्ठा प्र॑ति॒ष्ठैषा प्राय॑ण॒म् प्राय॑ण मे॒षा प्र॑ति॒ष्ठा । \newline
24. प्राय॑ण॒मिति॑ प्र - अय॑नम् । \newline
25. ए॒षा प्र॑ति॒ष्ठा प्र॑ति॒ष्ठैषैषा प्र॑ति॒ष्ठैतदे॒तत् प्र॑ति॒ष्ठैषैषा प्र॑ति॒ष्ठैतत् । \newline
26. प्र॒ति॒ष्ठैतदे॒तत् प्र॑ति॒ष्ठा प्र॑ति॒ष्ठैतदु॒दय॑न मु॒दय॑न मे॒तत् प्र॑ति॒ष्ठा प्र॑ति॒ष्ठैतदु॒दय॑नम् । \newline
27. प्र॒ति॒ष्ठेति॑ प्रति - स्था । \newline
28. ए॒तदु॒दय॑न मु॒दय॑न मे॒तदे॒त दु॒दय॑नं॒ ॅयो य उ॒दय॑न मे॒तदे॒त दु॒दय॑नं॒ ॅयः । \newline
29. उ॒दय॑नं॒ ॅयो य उ॒दय॑न मु॒दय॑नं॒ ॅय ए॒व मे॒वं ॅय उ॒दय॑न मु॒दय॑नं॒ ॅय ए॒वम् । \newline
30. उ॒दय॑न॒मित्यु॑त् - अय॑नम् । \newline
31. य ए॒व मे॒वं ॅयो य ए॒वं ॅवेद॒ वेदै॒वं ॅयो य ए॒वं ॅवेद॑ । \newline
32. ए॒वं ॅवेद॒ वेदै॒व मे॒वं ॅवेद॒ प्रति॑ष्ठितेन॒ प्रति॑ष्ठितेन॒ वेदै॒व मे॒वं ॅवेद॒ प्रति॑ष्ठितेन । \newline
33. वेद॒ प्रति॑ष्ठितेन॒ प्रति॑ष्ठितेन॒ वेद॒ वेद॒ प्रति॑ष्ठिते॒ नारि॑ष्टे॒नारि॑ष्टेन॒ प्रति॑ष्ठितेन॒ वेद॒ वेद॒ प्रति॑ष्ठिते॒नारि॑ष्टेन । \newline
34. प्रति॑ष्ठिते॒ नारि॑ष्टे॒नारि॑ष्टेन॒ प्रति॑ष्ठितेन॒ प्रति॑ष्ठिते॒नारि॑ष्टेन य॒ज्ञेन॑ य॒ज्ञेनारि॑ष्टेन॒ प्रति॑ष्ठितेन॒ प्रति॑ष्ठिते॒नारि॑ष्टेन य॒ज्ञेन॑ । \newline
35. प्रति॑ष्ठिते॒नेति॒ प्रति॑ - स्थि॒ते॒न॒ । \newline
36. अरि॑ष्टेन य॒ज्ञेन॑ य॒ज्ञे नारि॑ष्टे॒नारि॑ष्टेन य॒ज्ञेन॑ स॒(ग्ग्॒)स्थाꣳ स॒(ग्ग्॒)स्थां ॅय॒ज्ञे नारि॑ष्टे॒नारि॑ष्टेन य॒ज्ञेन॑ स॒(ग्ग्॒)स्थाम् । \newline
37. य॒ज्ञेन॑ स॒(ग्ग्॒)स्थाꣳ स॒(ग्ग्॒)स्थां ॅय॒ज्ञेन॑ य॒ज्ञेन॑ स॒(ग्ग्॒)स्थाम् ग॑च्छति गच्छति स॒(ग्ग्॒)स्थां ॅय॒ज्ञेन॑ य॒ज्ञेन॑ स॒(ग्ग्॒)स्थाम् ग॑च्छति । \newline
38. स॒(ग्ग्॒)स्थाम् ग॑च्छति गच्छति स॒(ग्ग्॒)स्थाꣳ स॒(ग्ग्॒)स्थाम् ग॑च्छति॒ यो यो ग॑च्छति स॒(ग्ग्॒)स्थाꣳ स॒(ग्ग्॒)स्थाम् ग॑च्छति॒ यः । \newline
39. स॒(ग्ग्॒)स्थामिति॑ सं - स्थाम् । \newline
40. ग॒च्छ॒ति॒ यो यो ग॑च्छति गच्छति॒ यो वै वै यो ग॑च्छति गच्छति॒ यो वै । \newline
41. यो वै वै यो यो वै सू॒नृता॑यै सू॒नृता॑यै॒ वै यो यो वै सू॒नृता॑यै । \newline
42. वै सू॒नृता॑यै सू॒नृता॑यै॒ वै वै सू॒नृता॑यै॒ दोह॒म् दोह(ग्म्॑) सू॒नृता॑यै॒ वै वै सू॒नृता॑यै॒ दोह᳚म् । \newline
43. सू॒नृता॑यै॒ दोह॒म् दोह(ग्म्॑) सू॒नृता॑यै सू॒नृता॑यै॒ दोहं॒ ॅवेद॒ वेद॒ दोह(ग्म्॑) सू॒नृता॑यै सू॒नृता॑यै॒ दोहं॒ ॅवेद॑ । \newline
44. दोहं॒ ॅवेद॒ वेद॒ दोह॒म् दोहं॒ ॅवेद॑ दु॒हे दु॒हे वेद॒ दोह॒म् दोहं॒ ॅवेद॑ दु॒हे । \newline
45. वेद॑ दु॒हे दु॒हे वेद॒ वेद॑ दु॒ह ए॒वैव दु॒हे वेद॒ वेद॑ दु॒ह ए॒व । \newline
46. दु॒ह ए॒वैव दु॒हे दु॒ह ए॒वैना॑ मेना मे॒व दु॒हे दु॒ह ए॒वैना᳚म् । \newline
47. ए॒वैना॑ मेना मे॒वैवैनां᳚ ॅय॒ज्ञो य॒ज्ञ् ए॑ना मे॒वैवैनां᳚ ॅय॒ज्ञ्ः । \newline
48. ए॒नां॒ ॅय॒ज्ञो य॒ज्ञ् ए॑ना मेनां ॅय॒ज्ञो वै वै य॒ज्ञ् ए॑ना मेनां ॅय॒ज्ञो वै । \newline
49. य॒ज्ञो वै वै य॒ज्ञो य॒ज्ञो वै सू॒नृता॑ सू॒नृता॒ वै य॒ज्ञो य॒ज्ञो वै सू॒नृता᳚ । \newline
50. वै सू॒नृता॑ सू॒नृता॒ वै वै सू॒नृता ऽऽसू॒नृता॒ वै वै सू॒नृता । \newline
51. सू॒नृता ऽऽसू॒नृता॑ सू॒नृता ऽऽश्रा॑वय श्राव॒या सू॒नृता॑ सू॒नृता ऽऽश्रा॑वय । \newline
52. आ श्रा॑वय श्राव॒या श्रा॑व॒ये तीति॑ श्राव॒या श्रा॑व॒ये ति॑ । \newline
53. श्रा॒व॒ये तीति॑ श्रावय श्राव॒ये त्येति॑ श्रावय श्राव॒ये त्या । \newline
54. इत्येतीत्यैवैवेतीत्यैव । \newline
55. ऐवैवैवैना॑ मेना मे॒वैवैना᳚म् । \newline
56. ए॒वैना॑ मेना मे॒वैवैना॑ मह्वदह्वदेना मे॒वैवैना॑ मह्वत् । \newline
57. ए॒ना॒ म॒ह्व॒ द॒ह्व॒ दे॒ना॒ मे॒ना॒ म॒ह्व॒ दस्त्वस्त्व॑ह्व देना मेना मह्व॒दस्तु॑ । \newline
58. अ॒ह्व॒ दस्त्वस्त्व॑ह्व दह्व॒दस्तु॒ श्रौष॒ट् छ्रौष॒ड स्त्व॑ह्व दह्व॒दस्तु॒ श्रौष॑ट् । \newline
59. अस्तु॒ श्रौष॒ट् छ्रौष॒ डस्त्वस्तु॒ श्रौष॒डितीति॒ श्रौष॒ डस्त्वस्तु॒ श्रौष॒डिति॑ । \newline
\pagebreak
\markright{ TS 1.6.11.3  \hfill https://www.vedavms.in \hfill}

\section{ TS 1.6.11.3 }

\textbf{TS 1.6.11.3 } \newline
\textbf{Samhita Paata} \newline

श्रौष॒डित्यु॒पावा᳚स्रा॒ग्यजेत्युद॑नैषी॒द्ये यजा॑मह॒ इत्युपा॑ऽसदद् वषट्का॒रेण॑ दोग्द्ध्ये॒ष वै सू॒नृता॑यै॒ दोहो॒ य ए॒वं ॅवेद॑ दु॒ह ए॒वैनां᳚ दे॒वा वै स॒त्रमा॑सत॒ तेषां॒ दिशो॑ऽदस्य॒न्त ए॒तामा॒र्द्रां प॒ङ्क्तिम॑पश्य॒न्ना श्रा॑व॒येति॑ पुरोवा॒त-म॑जनय॒न्नस्तु॒ श्रौष॒डित्य॒भ्रꣳ सम॑प्लावय॒न्॒. यजेति॑ वि॒द्युत॑ - [ ] \newline

\textbf{Pada Paata} \newline

श्रौष॑ट् । इति॑ । उ॒पावा᳚स्रा॒गित्यु॑प-अवा᳚स्राक् । यज॑ । इति॑ । उदिति॑ । अ॒नै॒षी॒त् । ये । यजा॑महे । इति॑ । उपेति॑ । अ॒स॒द॒त् । व॒ष॒ट्का॒रेणेति॑ वषट्-का॒रेण॑ । दो॒ग्धि॒ । ए॒षः । वै । सू॒नृता॑यै । दोहः॑ । यः । ए॒वम् । वेद॑ । दु॒हे । ए॒व । ए॒ना॒म् । दे॒वाः । वै । स॒त्रम् । आ॒स॒त॒ । तेषा᳚म् । दिशः॑ । अ॒द॒स्य॒न्न् । ते । ए॒ताम् । आ॒र्द्राम् । प॒ङ्क्तिम् । अ॒प॒श्य॒न्न् । एति॑ । श्रा॒व॒य॒ । इति॑ । पु॒रो॒वा॒तमिति॑ पुरः - वा॒तम् । अ॒ज॒न॒य॒न्न् । अस्तु॑ । श्रौष॑ट् । इति॑ । अ॒भ्रम् । समिति॑ । अ॒प्ला॒व॒य॒न्न् । यज॑ । इति॑ । वि॒द्युत॒मिति॑ वि - द्युत᳚म् ।  \newline


\textbf{Krama Paata} \newline

श्रौष॒डिति॑ । इत्यु॒पावा᳚स्राक् । उ॒पावा᳚स्रा॒ग् यज॑ । उ॒पावा᳚स्रा॒गित्यु॑प - अवा᳚स्राक् । यजेति॑ । इत्युत् । उद॑नैषीत् । अ॒नै॒षी॒द् ये । ये यजा॑महे । यजा॑मह॒ इति॑ । इत्युप॑ । उपा॑सदत् । अ॒स॒द॒द् व॒ष॒ट्का॒रेण॑ । व॒ष॒ट्का॒रेण॑ दोग्धि । व॒ष॒ट्का॒रे॒णेति॑ वषट् - का॒रेण॑ । दो॒ग्ध्ये॒षः । ए॒ष वै । वै सू॒नृता॑यै । सू॒नृता॑यै॒ दोहः॑ । दोहो॒ यः । य ए॒वम् । ए॒वं ॅवेद॑ । वेद॑ दु॒हे । दु॒ह ए॒व । ए॒वैना᳚म् । ए॒ना॒म् दे॒वाः । दे॒वा वै । वै स॒त्रम् । स॒त्रमा॑सत । आ॒स॒त॒ तेषा᳚म् । तेषा॒म् दिशः॑ । दिशो॑ऽदस्यन्न् । अ॒द॒स्य॒न् ते । त ए॒ताम् । ए॒तामा॒र्द्राम् । आ॒र्द्राम् प॒ङ्क्तिम् । प॒ङ्क्तिम॑पश्यन्न् । अ॒प॒श्य॒न्ना । आ श्रा॑वय । श्रा॒व॒येति॑ । इति॑ पुरोवा॒तम् । पु॒रो॒वा॒तम॑जनयन्न् । पु॒रो॒वा॒तमिति॑ पुरः - वा॒तम् । अ॒ज॒न॒य॒न्नस्तु॑ । अस्तु॒ श्रौष॑ट् । श्रौष॒डिति॑ । इत्य॒भ्रम् । अ॒भ्रꣳ सम् । सम॑प्लावयन्न् । अ॒प्ला॒व॒य॒न्॒. यज॑ । यजेति॑ । इति॑ वि॒द्युत᳚म् । वि॒द्युत॑मजनयन्न् । वि॒द्युत॒मिति॑ वि - द्युत᳚म् \newline

\textbf{Jatai Paata} \newline

1. श्रौष॒ डितीति॒ श्रौष॒ट् छ्रौष॒ डिति॑ । \newline
2. इत्यु॒पावा᳚स्रा गु॒पावा᳚स्रा॒ गिती त्यु॒पावा᳚स्राक् । \newline
3. उ॒पावा᳚स्रा॒ग् यज॒ यजो॒पावा᳚ स्रागु॒पावा᳚स्रा॒ग् यज॑ । \newline
4. उ॒पावा᳚स्रा॒गित्यु॑प - अवा᳚स्राक् । \newline
5. यजे तीति॒ यज॒ यजे ति॑ । \newline
6. इत्यु दुदि तीत्युत् । \newline
7. उद॑नैषी दनैषी॒ दुदु द॑नैषीत् । \newline
8. अ॒नै॒षी॒द् ये ये॑ ऽनैषी दनैषी॒द् ये । \newline
9. ये यजा॑महे॒ यजा॑महे॒ ये ये यजा॑महे । \newline
10. यजा॑मह॒ इतीति॒ यजा॑महे॒ यजा॑मह॒ इति॑ । \newline
11. इत्युपोपे तीत्युप॑ । \newline
12. उपा॑स दद सद॒ दुपो पा॑सदत् । \newline
13. अ॒स॒द॒द् व॒ष॒ट्का॒रेण॑ वषट्का॒रेणा॑ सद दसदद् वषट्का॒रेण॑ । \newline
14. व॒ष॒ट्का॒रेण॑ दोग्धि दोग्धि वषट्का॒रेण॑ वषट्का॒रेण॑ दोग्धि । \newline
15. व॒ष॒ट्का॒रेणेति॑ वषट् - का॒रेण॑ । \newline
16. दो॒ग्ध्ये॒ष ए॒ष दो᳚ग्धि दोग्ध्ये॒षः । \newline
17. ए॒ष वै वा ए॒ष ए॒ष वै । \newline
18. वै सू॒नृता॑यै सू॒नृता॑यै॒ वै वै सू॒नृता॑यै । \newline
19. सू॒नृता॑यै॒ दोहो॒ दोहः॑ सू॒नृता॑यै सू॒नृता॑यै॒ दोहः॑ । \newline
20. दोहो॒ यो यो दोहो॒ दोहो॒ यः । \newline
21. य ए॒व मे॒वं ॅयो य ए॒वम् । \newline
22. ए॒वं ॅवेद॒ वेदै॒व मे॒वं ॅवेद॑ । \newline
23. वेद॑ दु॒हे दु॒हे वेद॒ वेद॑ दु॒हे । \newline
24. दु॒ह ए॒वैव दु॒हे दु॒ह ए॒व । \newline
25. ए॒वैना॑ मेना मे॒वै वैना᳚म् । \newline
26. ए॒ना॒म् दे॒वा दे॒वा ए॑ना मेनाम् दे॒वाः । \newline
27. दे॒वा वै वै दे॒वा दे॒वा वै । \newline
28. वै स॒त्रꣳ स॒त्रं ॅवै वै स॒त्रम् । \newline
29. स॒त्र मा॑सतासत स॒त्रꣳ स॒त्र मा॑सत । \newline
30. आ॒स॒त॒ तेषा॒म् तेषा॑ मास तासत॒ तेषा᳚म् । \newline
31. तेषा॒म् दिशो॒ दिश॒ स्तेषा॒म् तेषा॒म् दिशः॑ । \newline
32. दिशो॑ ऽदस्यन् नदस्य॒न् दिशो॒ दिशो॑ ऽदस्यन्न् । \newline
33. अ॒द॒स्य॒न् ते ते॑ ऽदस्यन् नदस्य॒न् ते । \newline
34. त ए॒ता मे॒ताम् ते त ए॒ताम् । \newline
35. ए॒ता मा॒र्द्रा मा॒र्द्रा मे॒ता मे॒ता मा॒र्द्राम् । \newline
36. आ॒र्द्राम् प॒ङ्क्तिम् प॒ङ्क्ति मा॒र्द्रा मा॒र्द्राम् प॒ङ्क्तिम् । \newline
37. प॒ङ्क्ति म॑पश्यन् नपश्यन् प॒ङ्क्तिम् प॒ङ्क्ति म॑पश्यन्न् । \newline
38. अ॒प॒श्य॒न् ना ऽप॑श्यन् नपश्य॒न् ना । \newline
39. आ श्रा॑वय श्राव॒या श्रा॑वय । \newline
40. श्रा॒व॒ये तीति॑ श्रावय श्राव॒ये ति॑ । \newline
41. इति॑ पुरोवा॒तम् पु॑रोवा॒त मितीति॑ पुरोवा॒तम् । \newline
42. पु॒रो॒वा॒त म॑जनयन् नजनयन् पुरोवा॒तम् पु॑रोवा॒त म॑जनयन्न् । \newline
43. पु॒रो॒वा॒तमिति॑ पुरः - वा॒तम् । \newline
44. अ॒ज॒न॒य॒न् नस्त्व स्त्व॑जनयन् नजनय॒न् नस्तु॑ । \newline
45. अस्तु॒ श्रौष॒ट् छ्रौष॒ डस्त्वस्तु॒ श्रौष॑ट् । \newline
46. श्रौष॒ डितीति॒ श्रौष॒ट् छ्रौष॒ डिति॑ । \newline
47. इत्य॒भ्र म॒भ्र मिती त्य॒भ्रम् । \newline
48. अ॒भ्रꣳ सꣳ स म॒भ्र म॒भ्रꣳ सम् । \newline
49. स म॑प्लावयन् नप्लावय॒न् थ्सꣳ स म॑प्लावयन्न् । \newline
50. अ॒प्ला॒व॒य॒न्॒. यज॒ यजा᳚प्लावयन् नप्लावय॒न्॒. यज॑ । \newline
51. यजे तीति॒ यज॒ यजे ति॑ । \newline
52. इति॑ वि॒द्युतं॑ ॅवि॒द्युत॒ मितीति॑ वि॒द्युत᳚म् । \newline
53. वि॒द्युत॑ मजनयन् नजनयन्. वि॒द्युतं॑ ॅवि॒द्युत॑ मजनयन्न् । \newline
54. वि॒द्युत॒मिति॑ वि - द्युत᳚म् । \newline

\textbf{Ghana Paata } \newline

1. श्रौष॒डितीति॒ श्रौष॒ट् छ्रौष॒ डित्यु॒पावा᳚स्रा गु॒पावा᳚स्रा॒गिति॒ श्रौष॒ट् छ्रौष॒ डित्यु॒पावा᳚स्राक् । \newline
2. इत्यु॒पावा᳚स्रा गु॒पावा᳚स्रा॒ गितीत्यु॒पावा᳚स्रा॒ग् यज॒ यजो॒पावा᳚स्रा॒ गितीत्यु॒पावा᳚स्रा॒ग् यज॑ । \newline
3. उ॒पावा᳚स्रा॒ग् यज॒ यजो॒पावा᳚स्रा गु॒पावा᳚स्रा॒ग् यजे तीति॒ यजो॒पावा᳚स्रा गु॒पावा᳚स्रा॒ग् यजे ति॑ । \newline
4. उ॒पावा᳚स्रा॒गित्यु॑प - अवा᳚स्राक् । \newline
5. यजे तीति॒ यज॒ यजे त्युदुदिति॒ यज॒ यजे त्युत् । \newline
6. इत्यु दुदिती त्युद॑नैषी दनैषी॒दुदिती त्युद॑नैषीत् । \newline
7. उद॑नैषीदनैषी॒ दुदुद॑नैषी॒द् ये ये॑ ऽनैषी॒ दुदुद॑नैषी॒द् ये । \newline
8. अ॒नै॒षी॒द् ये ये॑ ऽनैषीदनैषी॒द् ये यजा॑महे॒ यजा॑महे ये॑ ऽनैषीदनैषी॒द् ये यजा॑महे । \newline
9. ये यजा॑महे॒ यजा॑महे॒ ये ये यजा॑मह॒ इतीति॒ यजा॑महे॒ ये ये यजा॑मह॒ इति॑ । \newline
10. यजा॑मह॒ इतीति॒ यजा॑महे॒ यजा॑मह॒ इत्युपोपे ति॒ यजा॑महे॒ यजा॑मह॒ इत्युप॑ । \newline
11. इत्युपोपे तीत्युपा॑स ददसद॒दुपे तीत्युपा॑सदत् । \newline
12. उपा॑स ददसद॒ दुपोपा॑सदद् वषट्का॒रेण॑ वषट्का॒रे णा॑सद॒ दुपोपा॑सदद् वषट्का॒रेण॑ । \newline
13. अ॒स॒द॒द् व॒ष॒ट्का॒रेण॑ वषट्का॒रे णा॑सददसदद् वषट्का॒रेण॑ दोग्धि दोग्धि वषट्का॒रे णा॑सददसदद् वषट्का॒रेण॑ दोग्धि । \newline
14. व॒ष॒ट्का॒रेण॑ दोग्धि दोग्धि वषट्का॒रेण॑ वषट्का॒रेण॑ दोग्ध्ये॒ष ए॒ष दो᳚ग्धि वषट्का॒रेण॑ वषट्का॒रेण॑ दोग्ध्ये॒षः । \newline
15. व॒ष॒ट्का॒रेणेति॑ वषट् - का॒रेण॑ । \newline
16. दो॒ग्ध्ये॒ष ए॒ष दो᳚ग्धि दोग्ध्ये॒ष वै वा ए॒ष दो᳚ग्धि दोग्ध्ये॒ष वै । \newline
17. ए॒ष वै वा ए॒ष ए॒ष वै सू॒नृता॑यै सू॒नृता॑यै॒ वा ए॒ष ए॒ष वै सू॒नृता॑यै । \newline
18. वै सू॒नृता॑यै सू॒नृता॑यै॒ वै वै सू॒नृता॑यै॒ दोहो॒ दोहः॑ सू॒नृता॑यै॒ वै वै सू॒नृता॑यै॒ दोहः॑ । \newline
19. सू॒नृता॑यै॒ दोहो॒ दोहः॑ सू॒नृता॑यै सू॒नृता॑यै॒ दोहो॒ यो यो दोहः॑ सू॒नृता॑यै सू॒नृता॑यै॒ दोहो॒ यः । \newline
20. दोहो॒ यो यो दोहो॒ दोहो॒ य ए॒व मे॒वं ॅयो दोहो॒ दोहो॒ य ए॒वम् । \newline
21. य ए॒व मे॒वं ॅयो य ए॒वं ॅवेद॒ वेदै॒वं ॅयो य ए॒वं ॅवेद॑ । \newline
22. ए॒वं ॅवेद॒ वेदै॒व मे॒वं ॅवेद॑ दु॒हे दु॒हे वेदै॒व मे॒वं ॅवेद॑ दु॒हे । \newline
23. वेद॑ दु॒हे दु॒हे वेद॒ वेद॑ दु॒ह ए॒वैव दु॒हे वेद॒ वेद॑ दु॒ह ए॒व । \newline
24. दु॒ह ए॒वैव दु॒हे दु॒ह ए॒वैना॑ मेना मे॒व दु॒हे दु॒ह ए॒वैना᳚म् । \newline
25. ए॒वैना॑ मेना मे॒वैवैना᳚म् दे॒वा दे॒वा ए॑ना मे॒वैवैना᳚म् दे॒वाः । \newline
26. ए॒ना॒म् दे॒वा दे॒वा ए॑ना मेनाम् दे॒वा वै वै दे॒वा ए॑ना मेनाम् दे॒वा वै । \newline
27. दे॒वा वै वै दे॒वा दे॒वा वै स॒त्रꣳ स॒त्रं ॅवै दे॒वा दे॒वा वै स॒त्रम् । \newline
28. वै स॒त्रꣳ स॒त्रं ॅवै वै स॒त्र मा॑सतासत स॒त्रं ॅवै वै स॒त्र मा॑सत । \newline
29. स॒त्र मा॑सतासत स॒त्रꣳ स॒त्र मा॑सत॒ तेषा॒म् तेषा॑ मासत स॒त्रꣳ स॒त्र मा॑सत॒ तेषा᳚म् । \newline
30. आ॒स॒त॒ तेषा॒म् तेषा॑ मासतासत॒ तेषा॒म् दिशो॒ दिश॒स्तेषा॑ मासतासत॒ तेषा॒म् दिशः॑ । \newline
31. तेषा॒म् दिशो॒ दिश॒स्तेषा॒म् तेषा॒म् दिशो॑ ऽदस्यन् नदस्य॒न् दिश॒स्तेषा॒म् तेषा॒म् दिशो॑ ऽदस्यन्न् । \newline
32. दिशो॑ ऽदस्यन् नदस्य॒न् दिशो॒ दिशो॑ ऽदस्य॒न् ते ते॑ ऽदस्य॒न् दिशो॒ दिशो॑ ऽदस्य॒न् ते । \newline
33. अ॒द॒स्य॒न् ते ते॑ ऽदस्यन् नदस्य॒न् त ए॒ता मे॒ताम् ते॑ ऽदस्यन् नदस्य॒न् त ए॒ताम् । \newline
34. त ए॒ता मे॒ताम् ते त ए॒ता मा॒र्द्रा मा॒र्द्रा मे॒ताम् ते त ए॒ता मा॒र्द्राम् । \newline
35. ए॒ता मा॒र्द्रा मा॒र्द्रा मे॒ता मे॒ता मा॒र्द्राम् प॒ङ्क्तिम् प॒ङ्क्ति मा॒र्द्रा मे॒ता मे॒ता मा॒र्द्राम् प॒ङ्क्तिम् । \newline
36. आ॒र्द्राम् प॒ङ्क्तिम् प॒ङ्क्ति मा॒र्द्रा मा॒र्द्राम् प॒ङ्क्ति म॑पश्यन् नपश्यन् प॒ङ्क्ति मा॒र्द्रा मा॒र्द्राम् प॒ङ्क्ति म॑पश्यन्न् । \newline
37. प॒ङ्क्ति म॑पश्यन् नपश्यन् प॒ङ्क्तिम् प॒ङ्क्ति म॑पश्य॒न् ना ऽप॑श्यन् प॒ङ्क्तिम् प॒ङ्क्ति म॑पश्य॒न् ना । \newline
38. अ॒प॒श्य॒न् ना ऽप॑श्यन् नपश्य॒न् ना श्रा॑वय श्राव॒या ऽप॑श्यन् नपश्य॒न् ना श्रा॑वय । \newline
39. आ श्रा॑वय श्राव॒या श्रा॑व॒ये तीति॑ श्राव॒या श्रा॑व॒ये ति॑ । \newline
40. श्रा॒व॒ये तीति॑ श्रावय श्राव॒ये ति॑ पुरोवा॒तम् पु॑रोवा॒त मिति॑ श्रावय श्राव॒ये ति॑ पुरोवा॒तम् । \newline
41. इति॑ पुरोवा॒तम् पु॑रोवा॒त मितीति॑ पुरोवा॒त म॑जनयन् नजनयन् पुरोवा॒त मितीति॑ पुरोवा॒त म॑जनयन्न् । \newline
42. पु॒रो॒वा॒त म॑जनयन् नजनयन् पुरोवा॒तम् पु॑रोवा॒त म॑जनय॒न् नस्त्वस्त्व॑जनयन् पुरोवा॒तम् पु॑रोवा॒त म॑जनय॒न् नस्तु॑ । \newline
43. पु॒रो॒वा॒तमिति॑ पुरः - वा॒तम् । \newline
44. अ॒ज॒न॒य॒न् नस्त्व स्त्व॑जनयन् नजनय॒न् नस्तु॒ श्रौष॒ट् छ्रौष॒ डस्त्व॑जनयन् नजनय॒न् नस्तु॒ श्रौष॑ट् । \newline
45. अस्तु॒ श्रौष॒ट् छ्रौष॒ डस्त्वस्तु॒ श्रौष॒डितीति॒ श्रौष॒ डस्त्वस्तु॒ श्रौष॒डिति॑ । \newline
46. श्रौष॒ डितीति॒ श्रौष॒ट् छ्रौष॒ डित्य॒भ्र म॒भ्र मिति॒ श्रौष॒ट् छ्रौष॒ डित्य॒भ्रम् । \newline
47. इत्य॒भ्र म॒भ्र मितीत्य॒भ्रꣳ सꣳ स म॒भ्र मितीत्य॒भ्रꣳ सम् । \newline
48. अ॒भ्रꣳ सꣳ स म॒भ्र म॒भ्रꣳ स म॑प्लावयन् नप्लावय॒न् थ्स म॒भ्र म॒भ्रꣳ स म॑प्लावयन्न् । \newline
49. स म॑प्लावयन् नप्लावय॒न् थ्सꣳ स म॑प्लावय॒न्॒. यज॒ यजा᳚प्लावय॒न् थ्सꣳ स म॑प्लावय॒न्॒. यज॑ । \newline
50. अ॒प्ला॒व॒य॒न्॒. यज॒ यजा᳚प्लावयन् नप्लावय॒न्॒. यजे तीति॒ यजा᳚प्लावयन् नप्लावय॒न्॒. यजे ति॑ । \newline
51. यजे तीति॒ यज॒ यजे ति॑ वि॒द्युतं॑ ॅवि॒द्युत॒ मिति॒ यज॒ यजे ति॑ वि॒द्युत᳚म् । \newline
52. इति॑ वि॒द्युतं॑ ॅवि॒द्युत॒ मितीति॑ वि॒द्युत॑ मजनयन् नजनयन्. वि॒द्युत॒ मितीति॑ वि॒द्युत॑ मजनयन्न् । \newline
53. वि॒द्युत॑ मजनयन् नजनयन्. वि॒द्युतं॑ ॅवि॒द्युत॑ मजनय॒न्॒. ये ये॑ ऽजनयन्. वि॒द्युतं॑ ॅवि॒द्युत॑ मजनय॒न्॒. ये । \newline
54. वि॒द्युत॒मिति॑ वि - द्युत᳚म् । \newline
\pagebreak
\markright{ TS 1.6.11.4  \hfill https://www.vedavms.in \hfill}

\section{ TS 1.6.11.4 }

\textbf{TS 1.6.11.4 } \newline
\textbf{Samhita Paata} \newline

मजनय॒न्॒. ये यजा॑मह॒ इति॒ प्राव॑र्.षयन्न॒भ्य॑स्तनयन् वषट्का॒रेण॒ ततो॒ वै तेभ्यो॒ दिशः॒ प्राप्या॑यन्त॒ य ए॒वं ॅवेद॒ प्रास्मै॒ दिशः॑ प्यायन्ते प्र॒जाप॑तिं त्वो॒वेद॑ प्र॒जाप॑ति स्त्वंॅवेद॒ यं प्र॒जाप॑ति॒र् वेद॒ स पुण्यो॑ भवत्ये॒ष वै छ॑न्द॒स्यः॑ प्र॒जाप॑ति॒रा श्रा॑व॒याऽस्तु॒ श्रौष॒ड्यज॒ ये यजा॑महे वषट्का॒रो य ए॒वं ॅवेद॒ पुण्यो॑ भवति वस॒न्त - [ ] \newline

\textbf{Pada Paata} \newline

अ॒ज॒नय॒न्न् । ये । यजा॑महे । इति॑ । प्रेति॑ । अ॒व॒र्॒.ष॒य॒न्न् । अ॒भीति॑ । अ॒स्त॒न॒य॒न्न् । व॒ष॒ट्का॒रेणेति॑ वषट् - का॒रेण॑ । ततः॑ । वै । तेभ्यः॑ । दिशः॑ । प्रेति॑ । अ॒प्या॒य॒न्त॒ । यः । ए॒वम् । वेद॑ । प्रेति॑ । अ॒स्मै॒ । दिशः॑ । प्या॒य॒न्ते॒ । प्र॒जाप॑ति॒मिति॑ प्र॒जा - प॒ति॒म् । त्वो॒वेदेति॑ त्वः - वेद॑ । प्र॒जाप॑ति॒रिति॑ प्र॒जा - प॒तिः॒ । त्वं॒ॅवे॒देति॑  त्वं - वे॒द॒ । यम् । प्र॒जाप॑ति॒रिति॑ प्र॒जा-प॒तिः॒ । वेद॑ । सः । पुण्यः॑ । भ॒व॒ति॒ । ए॒षः । वै । छ॒न्द॒स्यः॑ । प्र॒जाप॑ति॒रिति॑ प्र॒जा - प॒तिः॒ । एति॑ । श्रा॒व॒य॒ । अस्तु॑ । श्रौष॑ट् । यज॑ । ये । यजा॑महे । व॒ष॒ट्का॒र इति॑ वषट् - का॒रः । यः । ए॒वम् । वेद॑ । पुण्यः॑ । भ॒व॒ति॒ । व॒स॒न्तम् ।  \newline


\textbf{Krama Paata} \newline

अ॒ज॒न॒य॒न्॒. ये । ये यजा॑महे । यजा॑मह॒ इति॑ । इति॒ प्र । प्राव॑र्.षयन्न् । अ॒व॒र्.ष॒य॒न्न॒भि । अ॒भ्य॑स्तनयन्न् । अ॒स्त॒न॒य॒न्॒. व॒ष॒ट्का॒रेण॑ । व॒ष॒ट्का॒रेण॒ ततः॑ । व॒ष॒ट्का॒रेणेति॑ वषट् - का॒रेण॑ । ततो॒ वै । वै तेभ्यः॑ । तेभ्यो॒ दिशः॑ । दिशः॒ प्र । प्राप्या॑यन्त । अ॒प्या॒य॒न्त॒ यः । य ए॒वम् । ए॒वं ॅवेद॑ । वेद॒ प्र । प्रास्मै᳚ । अ॒स्मै॒ दिशः॑ । दिशः॑ प्यायन्ते । प्या॒य॒न्ते॒ प्र॒जाप॑तिम् । प्र॒जाप॑तिम् त्वो॒वेद॑ । प्र॒जाप॑ति॒मिति॑ प्र॒जा - प॒ति॒म् । त्वो॒वेद॑ प्र॒जाप॑तिः । त्वो॒वेदेति॑ त्वः - वेद॑ । प्र॒जाप॑ति,स्त्वम्ॅवेद । प्र॒जाप॑ति॒रिति॑ प्र॒जा - प॒तिः॒ । त्व॒म्ॅवे॒द॒ यम् । त्व॒म्ॅवे॒देति॑ त्वम् - वे॒द॒ । यम् प्र॒जाप॑तिः । प्र॒जाप॑ति॒र् वेद॑ । प्र॒जाप॑ति॒रिति॑ प्र॒जा - प॒तिः॒ । वेद॒ सः । स पुण्यः॑ । पुण्यो॑ भवति । भ॒व॒त्ये॒षः । ए॒ष वै । वै छ॑न्द॒स्यः॑ । छ॒न्द॒स्यः॑ प्र॒जाप॑तिः । प्र॒जाप॑ति॒रा । प्र॒जाप॑ति॒रिति॑ प्र॒जा - प॒तिः॒ । आ श्रा॑वय । श्रा॒व॒यास्तु॑ । अस्तु॒ श्रौष॑ट् । श्रौष॒ड् यज॑ । यज॒ ये । ये यजा॑महे । यजा॑महे वषट्का॒रः । व॒ष॒ट्का॒रो यः । व॒ष॒ट्का॒र इति॑ वषट् - का॒रः । य ए॒वम् । ए॒वं ॅवेद॑ । वेद॒ पुण्यः॑ । पुण्यो॑ भवति । भ॒व॒ति॒ व॒स॒न्तम् । व॒स॒न्तमृ॑तू॒नाम् \newline

\textbf{Jatai Paata} \newline

1. अ॒ज॒न॒य॒न्॒. ये ये॑ ऽजनयन् नजनय॒न्॒. ये । \newline
2. ये यजा॑महे॒ यजा॑महे॒ ये ये यजा॑महे । \newline
3. यजा॑मह॒ इतीति॒ यजा॑महे॒ यजा॑मह॒ इति॑ । \newline
4. इति॒ प्र प्रे तीति॒ प्र । \newline
5. प्राव॑र्.षयन् नवर्.षय॒न् प्र प्राव॑र्.षयन्न् । \newline
6. अ॒व॒र्॒.ष॒य॒न् न॒भ्या᳚(1॒)भ्य॑वर्.षयन् नवर्.षयन् न॒भि । \newline
7. अ॒भ्य॑ स्तनयन् नस्तनयन् न॒भ्या᳚(1॒)भ्य॑ स्तनयन्न् । \newline
8. अ॒स्त॒न॒य॒न्॒. व॒ष॒ट्का॒रेण॑ वषट्का॒रेणा᳚ स्तनयन् नस्तनयन्. वषट्का॒रेण॑ । \newline
9. व॒ष॒ट्का॒रेण॒ तत॒स्ततो॑ वषट्का॒रेण॑ वषट्का॒रेण॒ ततः॑ । \newline
10. व॒ष॒ट्का॒रेणेति॑ वषट् - का॒रेण॑ । \newline
11. ततो॒ वै वै तत॒ स्ततो॒ वै । \newline
12. वै तेभ्य॒ स्तेभ्यो॒ वै वै तेभ्यः॑ । \newline
13. तेभ्यो॒ दिशो॒ दिश॒ स्तेभ्य॒ स्तेभ्यो॒ दिशः॑ । \newline
14. दिशः॒ प्र प्र दिशो॒ दिशः॒ प्र । \newline
15. प्राप्या॑यन्ता प्यायन्त॒ प्र प्राप्या॑यन्त । \newline
16. अ॒प्या॒य॒न्त॒ यो यो᳚ ऽप्यायन्ता प्यायन्त॒ यः । \newline
17. य ए॒व मे॒वं ॅयो य ए॒वम् । \newline
18. ए॒वं ॅवेद॒ वेदै॒व मे॒वं ॅवेद॑ । \newline
19. वेद॒ प्र प्र वेद॒ वेद॒ प्र । \newline
20. प्रास्मा॑ अस्मै॒ प्र प्रास्मै᳚ । \newline
21. अ॒स्मै॒ दिशो॒ दिशो᳚ ऽस्मा अस्मै॒ दिशः॑ । \newline
22. दिशः॑ प्यायन्ते प्यायन्ते॒ दिशो॒ दिशः॑ प्यायन्ते । \newline
23. प्या॒य॒न्ते॒ प्र॒जाप॑तिम् प्र॒जाप॑तिम् प्यायन्ते प्यायन्ते प्र॒जाप॑तिम् । \newline
24. प्र॒जाप॑तिम् त्वो॒वेद॑ त्वो॒वेद॑ प्र॒जाप॑तिम् प्र॒जाप॑तिम् त्वो॒वेद॑ । \newline
25. प्र॒जाप॑ति॒मिति॑ प्र॒जा - प॒ति॒म् । \newline
26. त्वो॒वेद॑ प्र॒जाप॑तिः प्र॒जाप॑ति स्त्वो॒वेद॑ त्वो॒वेद॑ प्र॒जाप॑तिः । \newline
27. त्वो॒वेदेति॑ त्वः - वेद॑ । \newline
28. प्र॒जाप॑ति स्त्वंॅवेद त्वंॅवेद प्र॒जाप॑तिः प्र॒जाप॑ति स्त्वंॅवेद । \newline
29. प्र॒जाप॑ति॒रिति॑ प्र॒जा - प॒तिः॒ । \newline
30. त्वं॒ॅवे॒द॒ यं ॅयम् त्वं॑ॅवेद त्वंॅवेद॒ यम् । \newline
31. त्वं॒ॅवे॒देति॑ त्वं - वे॒द॒ । \newline
32. यम् प्र॒जाप॑तिः प्र॒जाप॑ति॒र् यं ॅयम् प्र॒जाप॑तिः । \newline
33. प्र॒जाप॑ति॒र् वेद॒ वेद॑ प्र॒जाप॑तिः प्र॒जाप॑ति॒र् वेद॑ । \newline
34. प्र॒जाप॑ति॒रिति॑ प्र॒जा - प॒तिः॒ । \newline
35. वेद॒ स स वेद॒ वेद॒ सः । \newline
36. स पुण्यः॒ पुण्यः॒ स स पुण्यः॑ । \newline
37. पुण्यो॑ भवति भवति॒ पुण्यः॒ पुण्यो॑ भवति । \newline
38. भ॒व॒ त्ये॒ष ए॒ष भ॑वति भव त्ये॒षः । \newline
39. ए॒ष वै वा ए॒ष ए॒ष वै । \newline
40. वै छ॑न्द॒स्य॑ श्छन्द॒स्यो॑ वै वै छ॑न्द॒स्यः॑ । \newline
41. छ॒न्द॒स्यः॑ प्र॒जाप॑तिः प्र॒जाप॑ति श्छन्द॒स्य॑ श्छन्द॒स्यः॑ प्र॒जाप॑तिः । \newline
42. प्र॒जाप॑ति॒रा प्र॒जाप॑तिः प्र॒जाप॑ति॒रा । \newline
43. प्र॒जाप॑ति॒रिति॑ प्र॒जा - प॒तिः॒ । \newline
44. आ श्रा॑वय श्राव॒या श्रा॑वय । \newline
45. श्रा॒व॒या स्त्वस्तु॑ श्रावय श्राव॒यास्तु॑ । \newline
46. अस्तु॒ श्रौष॒ट् छ्रौष॒ डस्त्वस्तु॒ श्रौष॑ट् । \newline
47. श्रौष॒ड् यज॒ यज॒ श्रौष॒ट् छ्रौष॒ड् यज॑ । \newline
48. यज॒ ये ये यज॒ यज॒ ये । \newline
49. ये यजा॑महे॒ यजा॑महे॒ ये ये यजा॑महे । \newline
50. यजा॑महे वषट्का॒रो व॑षट्का॒रो यजा॑महे॒ यजा॑महे वषट्का॒रः । \newline
51. व॒ष॒ट्का॒रो यो यो व॑षट्का॒रो व॑षट्का॒रो यः । \newline
52. व॒ष॒ट्का॒र इति॑ वषट् - का॒रः । \newline
53. य ए॒व मे॒वं ॅयो य ए॒वम् । \newline
54. ए॒वं ॅवेद॒ वेदै॒व मे॒वं ॅवेद॑ । \newline
55. वेद॒ पुण्यः॒ पुण्यो॒ वेद॒ वेद॒ पुण्यः॑ । \newline
56. पुण्यो॑ भवति भवति॒ पुण्यः॒ पुण्यो॑ भवति । \newline
57. भ॒व॒ति॒ व॒स॒न्तं ॅव॑स॒न्तम् भ॑वति भवति वस॒न्तम् । \newline
58. व॒स॒न्त मृ॑तू॒ना मृ॑तू॒नां ॅव॑स॒न्तं ॅव॑स॒न्त मृ॑तू॒नाम् । \newline

\textbf{Ghana Paata } \newline

1. अ॒ज॒नय॒न्॒. ये ये॑ ऽज॒नय॒न् नज॒नय॒न्॒. ये यजा॑महे॒ यजा॑महे ये॑ ऽज॒नय॒न् नज॒नय॒न्॒. ये यजा॑महे । \newline
2. ये यजा॑महे॒ यजा॑महे॒ ये ये यजा॑मह॒ इतीति॒ यजा॑महे॒ ये ये यजा॑मह॒ इति॑ । \newline
3. यजा॑मह॒ इतीति॒ यजा॑महे॒ यजा॑मह॒ इति॒ प्र प्रे ति॒ यजा॑महे॒ यजा॑मह॒ इति॒ प्र । \newline
4. इति॒ प्र प्रे तीति॒ प्राव॑र्.षयन् नवर्.षय॒न् प्रे तीति॒ प्राव॑र्.षयन्न् । \newline
5. प्राव॑र्.षयन् नवर्.षय॒न् प्र प्राव॑र्.षयन् न॒भ्या᳚(1॒)भ्य॑वर्.षय॒न् प्र प्राव॑र्.षयन् न॒भि । \newline
6. अ॒व॒र्॒.ष॒य॒न् न॒भ्या᳚(1॒)भ्य॑वर्.षयन् नवर्.षयन् न॒भ्य॑स्तनयन् नस्तनयन् न॒भ्य॑वर्.षयन् नवर्.षयन् न॒भ्य॑स्तनयन्न् । \newline
7. अ॒भ्य॑स्तनयन् नस्तनयन् न॒भ्या᳚(1॒)भ्य॑स्तनयन्. वषट्का॒रेण॑ वषट्का॒रेणा᳚स्तनयन् न॒भ्या᳚(1॒)भ्य॑स्तनयन्. वषट्का॒रेण॑ । \newline
8. अ॒स्त॒न॒य॒न्॒. व॒ष॒ट्का॒रेण॑ वषट्का॒रेणा᳚ स्तनयन् नस्तनयन्. वषट्का॒रेण॒ तत॒स्ततो॑ वषट्का॒रेणा᳚ स्तनयन् नस्तनयन्. वषट्का॒रेण॒ ततः॑ । \newline
9. व॒ष॒ट्का॒रेण॒ तत॒स्ततो॑ वषट्का॒रेण॑ वषट्का॒रेण॒ ततो॒ वै वै ततो॑ वषट्का॒रेण॑ वषट्का॒रेण॒ ततो॒ वै । \newline
10. व॒ष॒ट्का॒रेणेति॑ वषट् - का॒रेण॑ । \newline
11. ततो॒ वै वै तत॒स्ततो॒ वै तेभ्य॒ स्तेभ्यो॒ वै तत॒स्ततो॒ वै तेभ्यः॑ । \newline
12. वै तेभ्य॒ स्तेभ्यो॒ वै वै तेभ्यो॒ दिशो॒ दिश॒ स्तेभ्यो॒ वै वै तेभ्यो॒ दिशः॑ । \newline
13. तेभ्यो॒ दिशो॒ दिश॒ स्तेभ्य॒ स्तेभ्यो॒ दिशः॒ प्र प्र दिश॒ स्तेभ्य॒ स्तेभ्यो॒ दिशः॒ प्र । \newline
14. दिशः॒ प्र प्र दिशो॒ दिशः॒ प्राप्या॑यन्ता प्यायन्त॒ प्र दिशो॒ दिशः॒ प्राप्या॑यन्त । \newline
15. प्राप्या॑यन्ता प्यायन्त॒ प्र प्राप्या॑यन्त॒ यो यो᳚ ऽप्यायन्त॒ प्र प्राप्या॑यन्त॒ यः । \newline
16. अ॒प्या॒य॒न्त॒ यो यो᳚ ऽप्यायन्ताप्यायन्त॒ य ए॒व मे॒वं ॅयो᳚ ऽप्यायन्ताप्यायन्त॒ य ए॒वम् । \newline
17. य ए॒व मे॒वं ॅयो य ए॒वं ॅवेद॒ वेदै॒वं ॅयो य ए॒वं ॅवेद॑ । \newline
18. ए॒वं ॅवेद॒ वेदै॒व मे॒वं ॅवेद॒ प्र प्र वेदै॒व मे॒वं ॅवेद॒ प्र । \newline
19. वेद॒ प्र प्र वेद॒ वेद॒ प्रास्मा॑ अस्मै॒ प्र वेद॒ वेद॒ प्रास्मै᳚ । \newline
20. प्रास्मा॑ अस्मै॒ प्र प्रास्मै॒ दिशो॒ दिशो᳚ ऽस्मै॒ प्र प्रास्मै॒ दिशः॑ । \newline
21. अ॒स्मै॒ दिशो॒ दिशो᳚ ऽस्मा अस्मै॒ दिशः॑ प्यायन्ते प्यायन्ते॒ दिशो᳚ ऽस्मा अस्मै॒ दिशः॑ प्यायन्ते । \newline
22. दिशः॑ प्यायन्ते प्यायन्ते॒ दिशो॒ दिशः॑ प्यायन्ते प्र॒जाप॑तिम् प्र॒जाप॑तिम् प्यायन्ते॒ दिशो॒ दिशः॑ प्यायन्ते प्र॒जाप॑तिम् । \newline
23. प्या॒य॒न्ते॒ प्र॒जाप॑तिम् प्र॒जाप॑तिम् प्यायन्ते प्यायन्ते प्र॒जाप॑तिम् त्वो॒वेद॑ त्वो॒वेद॑ प्र॒जाप॑तिम् प्यायन्ते प्यायन्ते प्र॒जाप॑तिम् त्वो॒वेद॑ । \newline
24. प्र॒जाप॑तिम् त्वो॒वेद॑ त्वो॒वेद॑ प्र॒जाप॑तिम् प्र॒जाप॑तिम् त्वो॒वेद॑ प्र॒जाप॑तिः प्र॒जाप॑ति स्त्वो॒वेद॑ प्र॒जाप॑तिम् प्र॒जाप॑तिम् त्वो॒वेद॑ प्र॒जाप॑तिः । \newline
25. प्र॒जाप॑ति॒मिति॑ प्र॒जा - प॒ति॒म् । \newline
26. त्वो॒वेद॑ प्र॒जाप॑तिः प्र॒जाप॑ति स्त्वो॒वेद॑ त्वो॒वेद॑ प्र॒जाप॑ति स्त्वंॅवेद त्वंॅवेद प्र॒जाप॑ति स्त्वो॒वेद॑ त्वो॒वेद॑ प्र॒जाप॑ति स्त्वंॅवेद । \newline
27. त्वो॒वेदेति॑ त्वः - वेद॑ । \newline
28. प्र॒जाप॑ति स्त्वंॅवेद त्वंॅवेद प्र॒जाप॑तिः प्र॒जाप॑ति स्त्वंॅवेद॒ यं ॅयम् त्वं॑ॅवेद प्र॒जाप॑तिः प्र॒जाप॑ति स्त्वंॅवेद॒ यम् । \newline
29. प्र॒जाप॑ति॒रिति॑ प्र॒जा - प॒तिः॒ । \newline
30. त्वं॒ॅवे॒द॒ यं ॅयम् त्वं॑ॅवेद त्वंॅवेद॒ यम् प्र॒जाप॑तिः प्र॒जाप॑ति॒र् यम् त्वं॑ॅवेद त्वंॅवेद॒ यम् प्र॒जाप॑तिः । \newline
31. त्वं॒ॅवे॒देति॑ त्वं - वे॒द॒ । \newline
32. यम् प्र॒जाप॑तिः प्र॒जाप॑ति॒र् यं ॅयम् प्र॒जाप॑ति॒र् वेद॒ वेद॑ प्र॒जाप॑ति॒र् यं ॅयम् प्र॒जाप॑ति॒र् वेद॑ । \newline
33. प्र॒जाप॑ति॒र् वेद॒ वेद॑ प्र॒जाप॑तिः प्र॒जाप॑ति॒र् वेद॒ स स वेद॑ प्र॒जाप॑तिः प्र॒जाप॑ति॒र् वेद॒ सः । \newline
34. प्र॒जाप॑ति॒रिति॑ प्र॒जा - प॒तिः॒ । \newline
35. वेद॒ स स वेद॒ वेद॒ स पुण्यः॒ पुण्यः॒ स वेद॒ वेद॒ स पुण्यः॑ । \newline
36. स पुण्यः॒ पुण्यः॒ स स पुण्यो॑ भवति भवति॒ पुण्यः॒ स स पुण्यो॑ भवति । \newline
37. पुण्यो॑ भवति भवति॒ पुण्यः॒ पुण्यो॑ भवत्ये॒ष ए॒ष भ॑वति॒ पुण्यः॒ पुण्यो॑ भवत्ये॒षः । \newline
38. भ॒व॒त्ये॒ष ए॒ष भ॑वति भवत्ये॒ष वै वा ए॒ष भ॑वति भवत्ये॒ष वै । \newline
39. ए॒ष वै वा ए॒ष ए॒ष वै छ॑न्द॒स्य॑ श्छन्द॒स्यो॑ वा ए॒ष ए॒ष वै छ॑न्द॒स्यः॑ । \newline
40. वै छ॑न्द॒स्य॑ श्छन्द॒स्यो॑ वै वै छ॑न्द॒स्यः॑ प्र॒जाप॑तिः प्र॒जाप॑ति श्छन्द॒स्यो॑ वै वै छ॑न्द॒स्यः॑ प्र॒जाप॑तिः । \newline
41. छ॒न्द॒स्यः॑ प्र॒जाप॑तिः प्र॒जाप॑ति श्छन्द॒स्य॑ श्छन्द॒स्यः॑ प्र॒जाप॑ति॒रा प्र॒जाप॑ति श्छन्द॒स्य॑ श्छन्द॒स्यः॑ प्र॒जाप॑ति॒रा । \newline
42. प्र॒जाप॑ति॒रा प्र॒जाप॑तिः प्र॒जाप॑ति॒रा श्रा॑वय श्राव॒या प्र॒जाप॑तिः प्र॒जाप॑ति॒रा श्रा॑वय । \newline
43. प्र॒जाप॑ति॒रिति॑ प्र॒जा - प॒तिः॒ । \newline
44. आ श्रा॑वय श्राव॒या श्रा॑व॒या स्त्वस्तु॑ श्राव॒या श्रा॑व॒यास्तु॑ । \newline
45. श्रा॒व॒या स्त्वस्तु॑ श्रावय श्राव॒यास्तु॒ श्रौष॒ट् छ्रौष॒डस्तु॑ श्रावय श्राव॒यास्तु॒ श्रौष॑ट् । \newline
46. अस्तु॒ श्रौष॒ट् छ्रौष॒ड स्त्वस्तु॒ श्रौष॒ड् यज॒ यज॒ श्रौष॒ डस्त्वस्तु॒ श्रौष॒ड् यज॑ । \newline
47. श्रौष॒ड् यज॒ यज॒ श्रौष॒ट् छ्रौष॒ड् यज॒ ये ये यज॒ श्रौष॒ट् छ्रौष॒ड् यज॒ ये । \newline
48. यज॒ ये ये यज॒ यज॒ ये यजा॑महे॒ यजा॑महे॒ ये यज॒ यज॒ ये यजा॑महे । \newline
49. ये यजा॑महे॒ यजा॑महे॒ ये ये यजा॑महे वषट्का॒रो व॑षट्का॒रो यजा॑महे॒ ये ये यजा॑महे वषट्का॒रः । \newline
50. यजा॑महे वषट्का॒रो व॑षट्का॒रो यजा॑महे॒ यजा॑महे वषट्का॒रो यो यो व॑षट्का॒रो यजा॑महे॒ यजा॑महे वषट्का॒रो यः । \newline
51. व॒ष॒ट्का॒रो यो यो व॑षट्का॒रो व॑षट्का॒रो य ए॒व मे॒वं ॅयो व॑षट्का॒रो व॑षट्का॒रो य ए॒वम् । \newline
52. व॒ष॒ट्का॒र इति॑ वषट् - का॒रः । \newline
53. य ए॒व मे॒वं ॅयो य ए॒वं ॅवेद॒ वेदै॒वं ॅयो य ए॒वं ॅवेद॑ । \newline
54. ए॒वं ॅवेद॒ वेदै॒व मे॒वं ॅवेद॒ पुण्यः॒ पुण्यो॒ वेदै॒व मे॒वं ॅवेद॒ पुण्यः॑ । \newline
55. वेद॒ पुण्यः॒ पुण्यो॒ वेद॒ वेद॒ पुण्यो॑ भवति भवति॒ पुण्यो॒ वेद॒ वेद॒ पुण्यो॑ भवति । \newline
56. पुण्यो॑ भवति भवति॒ पुण्यः॒ पुण्यो॑ भवति वस॒न्तं ॅव॑स॒न्तम् भ॑वति॒ पुण्यः॒ पुण्यो॑ भवति वस॒न्तम् । \newline
57. भ॒व॒ति॒ व॒स॒न्तं ॅव॑स॒न्तम् भ॑वति भवति वस॒न्त मृ॑तू॒ना मृ॑तू॒नां ॅव॑स॒न्तम् भ॑वति भवति वस॒न्त मृ॑तू॒नाम् । \newline
58. व॒स॒न्त मृ॑तू॒ना मृ॑तू॒नां ॅव॑स॒न्तं ॅव॑स॒न्त मृ॑तू॒नाम् प्री॑णामि प्रीणाम्यृतू॒नां ॅव॑स॒न्तं ॅव॑स॒न्त मृ॑तू॒नाम् प्री॑णामि । \newline
\pagebreak
\markright{ TS 1.6.11.5  \hfill https://www.vedavms.in \hfill}

\section{ TS 1.6.11.5 }

\textbf{TS 1.6.11.5 } \newline
\textbf{Samhita Paata} \newline

मृ॑तू॒नां प्री॑णा॒मीत्या॑ह॒र्तवो॒ वै प्र॑या॒जा ऋ॒तूने॒व प्री॑णाति॒ ते᳚ऽस्मै प्री॒ता य॑थापू॒र्वं क॑ल्पन्ते॒ कल्प॑न्तेऽस्मा ऋ॒तवो॒ य ए॒वं ॅवेदा॒ग्नीषोम॑योर॒हं दे॑वय॒ज्यया॒ चक्षु॑ष्मान् भूयास॒मित्या॑-हा॒ग्नीषोमा᳚भ्यां॒ वै य॒ज्ञ्श्चक्षु॑ष्मा॒न् ताभ्या॑मे॒व चक्षु॑रा॒त्मन् ध॑त्ते॒ ऽग्नेर॒हं दे॑वय॒ज्यया᳚ऽन्ना॒दो भू॑यास॒मित्या॑हा॒ग्निर्वै दे॒वाना॑मन्ना॒दस्ते नै॒ वा - [ ] \newline

\textbf{Pada Paata} \newline

ऋ॒तू॒नाम् । प्री॒णा॒मि॒ । इति॑ । आ॒ह॒ । ऋ॒तवः॑ । वै । प्र॒या॒जा इति॑ प्र - या॒जाः । ऋ॒तून् । ए॒व । प्री॒णा॒ति॒ । ते । अ॒स्मै॒ । प्री॒ताः । य॒था॒पू॒र्वमिति॑ यथा - पू॒र्वम् । क॒ल्प॒न्ते॒ । कल्प॑न्ते । अ॒स्मै॒ । ऋ॒तवः॑ । यः । ए॒वम् । वेद॑ । अ॒ग्नीषोम॑यो॒रित्य॒ग्नी - सोम॑योः । अ॒हम् । दे॒व॒य॒ज्ययेति॑ देव-य॒ज्यया᳚ । चक्षु॑ष्मान् । भू॒या॒स॒म् । इति॑ । आ॒ह॒ । अ॒ग्नीषोमा᳚भ्या॒मित्य॒ग्नी - सोमा᳚भ्याम् । वै । य॒ज्ञ्ः । चक्षु॑ष्मान् । ताभ्या᳚म् । ए॒व । चक्षुः॑ । आ॒त्मन्न् । ध॒त्ते॒ । अ॒ग्नेः । अ॒हम् । दे॒व॒य॒ज्ययेति॑ देव - य॒ज्यया᳚ । अ॒न्ना॒द इत्य॑न्न - अ॒दः । भू॒या॒स॒म् । इति॑ । आ॒ह॒ । अ॒ग्निः । वै । दे॒वाना᳚म् । अ॒न्ना॒द इत्य॑न्न - अ॒दः । तेन॑ । ए॒व ।  \newline


\textbf{Krama Paata} \newline

ऋ॒तू॒नाम् प्री॑णामि । प्री॒णा॒मीति॑ । इत्या॑ह । आ॒ह॒र्तवः॑ । ऋ॒तवो॒ वै । वै प्र॑या॒जाः । प्र॒या॒जा ऋ॒तून् । प्र॒या॒जा इति॑ प्र - या॒जाः । ऋ॒तूने॒व । ए॒व प्री॑णाति । प्री॒णा॒ति॒ ते । ते᳚ऽस्मै । अ॒स्मै॒ प्री॒ताः । प्री॒ता य॑थापू॒र्वम् । य॒था॒पू॒र्वम् क॑ल्पन्ते । य॒था॒पू॒र्वमिति॑ यथा - पू॒र्वम् । क॒ल्प॒न्ते॒ कल्प॑न्ते । कल्प॑न्तेऽस्मै । अ॒स्मा॒ ऋ॒तवः॑ । ऋ॒तवो॒ यः । य ए॒वम् । ए॒वं ॅवेद॑ । वेदा॒ग्नीषोम॑योः । अ॒ग्नीषोम॑योर॒हम् । अ॒ग्नीषोम॑यो॒रित्य॒ग्नी - सोम॑योः । अ॒हम् दे॑वय॒ज्यया᳚ । दे॒व॒य॒ज्यया॒ चक्षु॑ष्मान् । दे॒व॒य॒ज्ययेति॑ देव - य॒ज्यया᳚ । चक्षु॑ष्मान् भूयासम् । भू॒या॒स॒मिति॑ । इत्या॑ह । आ॒हा॒ग्नीषोमा᳚भ्याम् । अ॒ग्नीषोमा᳚भ्यां॒ ॅवै । अ॒ग्नीषोमा᳚भ्या॒मित्य॒ग्नी - सोमा᳚भ्याम् । वै य॒ज्ञ्ः । य॒ज्ञ्,श्चक्षु॑ष्मान् । चक्षु॑ष्मा॒न् ताभ्या᳚म् । ताभ्या॑मे॒व । ए॒व चक्षुः॑ । चक्षु॑रा॒त्मन्न् । आ॒त्मन् ध॑त्ते । ध॒त्ते॒ऽग्नेः । अ॒ग्नेर॒हम् । अ॒हम् दे॑वय॒ज्यया᳚ । दे॒व॒य॒ज्यया᳚ ऽन्ना॒दः । दे॒व॒य॒ज्ययेति॑ देव - य॒ज्यया᳚ । अ॒न्ना॒दो भू॑यासम् । अ॒न्ना॒द इत्य॑न्न - अ॒दः । भू॒या॒स॒मिति॑ । इत्या॑ह । आ॒हा॒ग्निः । अ॒ग्निर्,वै । वै दे॒वाना᳚म् । दे॒वाना॑मन्ना॒दः । अ॒न्ना॒दस्तेन॑ । अ॒न्ना॒द इत्य॑न्न - अ॒दः । तेनै॒व । ए॒वान्नाद्य᳚म् \newline

\textbf{Jatai Paata} \newline

1. ऋ॒तू॒नाम् प्री॑णामि प्रीणा म्यृतू॒ना मृ॑तू॒नाम् प्री॑णामि । \newline
2. प्री॒णा॒ मीतीति॑ प्रीणामि प्रीणा॒ मीति॑ । \newline
3. इत्या॑हा॒हे तीत्या॑ह । \newline
4. आ॒ह॒ र्तव॑ ऋ॒तव॑ आहाह॒ र्तवः॑ । \newline
5. ऋ॒तवो॒ वै वा ऋ॒तव॑ ऋ॒तवो॒ वै । \newline
6. वै प्र॑या॒जाः प्र॑या॒जा वै वै प्र॑या॒जाः । \newline
7. प्र॒या॒जा ऋ॒तू नृ॒तून् प्र॑या॒जाः प्र॑या॒जा ऋ॒तून् । \newline
8. प्र॒या॒जा इति॑ प्र - या॒जाः । \newline
9. ऋ॒तू ने॒वैव र्तू नृ॒तू ने॒व । \newline
10. ए॒व प्री॑णाति प्रीणा त्ये॒वैव प्री॑णाति । \newline
11. प्री॒णा॒ति॒ ते ते प्री॑णाति प्रीणाति॒ ते । \newline
12. ते᳚ ऽस्मा अस्मै॒ ते ते᳚ ऽस्मै । \newline
13. अ॒स्मै॒ प्री॒ताः प्री॒ता अ॑स्मा अस्मै प्री॒ताः । \newline
14. प्री॒ता य॑थापू॒र्वं ॅय॑थापू॒र्वम् प्री॒ताः प्री॒ता य॑थापू॒र्वम् । \newline
15. य॒था॒पू॒र्वम् क॑ल्पन्ते कल्पन्ते यथापू॒र्वं ॅय॑थापू॒र्वम् क॑ल्पन्ते । \newline
16. य॒था॒पू॒र्वमिति॑ यथा - पू॒र्वम् । \newline
17. क॒ल्प॒न्ते॒ कल्प॑न्ते॒ कल्प॑न्ते कल्पन्ते कल्पन्ते॒ कल्प॑न्ते । \newline
18. कल्प॑न्ते ऽस्मा अस्मै॒ कल्प॑न्ते॒ कल्प॑न्ते ऽस्मै । \newline
19. अ॒स्मा॒ ऋ॒तव॑ ऋ॒तवो᳚ ऽस्मा अस्मा ऋ॒तवः॑ । \newline
20. ऋ॒तवो॒ यो य ऋ॒तव॑ ऋ॒तवो॒ यः । \newline
21. य ए॒व मे॒वं ॅयो य ए॒वम् । \newline
22. ए॒वं ॅवेद॒ वेदै॒व मे॒वं ॅवेद॑ । \newline
23. वेदा॒ग्नीषोम॑यो र॒ग्नीषोम॑यो॒र् वेद॒ वेदा॒ग्नीषोम॑योः । \newline
24. अ॒ग्नीषोम॑यो र॒ह म॒ह म॒ग्नीषोम॑यो र॒ग्नीषोम॑यो र॒हम् । \newline
25. अ॒ग्नीषोम॑यो॒रित्य॒ग्नी - सोम॑योः । \newline
26. अ॒हम् दे॑वय॒ज्यया॑ देवय॒ज्यया॒ ऽह म॒हम् दे॑वय॒ज्यया᳚ । \newline
27. दे॒व॒य॒ज्यया॒ चक्षु॑ष्मा॒(ग्ग्॒) श्चक्षु॑ष्मान् देवय॒ज्यया॑ देवय॒ज्यया॒ चक्षु॑ष्मान् । \newline
28. दे॒व॒य॒ज्ययेति॑ देव - य॒ज्यया᳚ । \newline
29. चक्षु॑ष्मान् भूयासम् भूयास॒म् चक्षु॑ष्मा॒(ग्ग्॒) श्चक्षु॑ष्मान् भूयासम् । \newline
30. भू॒या॒स॒ मितीति॑ भूयासम् भूयास॒ मिति॑ । \newline
31. इत्या॑हा॒हे तीत्या॑ह । \newline
32. आ॒हा॒ग्नीषोमा᳚भ्या म॒ग्नीषोमा᳚भ्या माहाहा॒ ग्नीषोमा᳚भ्याम् । \newline
33. अ॒ग्नीषोमा᳚भ्यां॒ ॅवै वा अ॒ग्नीषोमा᳚भ्या म॒ग्नीषोमा᳚भ्यां॒ ॅवै । \newline
34. अ॒ग्नीषोमा᳚भ्या॒मित्य॒ग्नी - सोमा᳚भ्याम् । \newline
35. वै य॒ज्ञो य॒ज्ञो वै वै य॒ज्ञ्ः । \newline
36. य॒ज्ञ् श्चक्षु॑ष्मा॒(ग्ग्॒) श्चक्षु॑ष्मान्. य॒ज्ञो य॒ज्ञ् श्चक्षु॑ष्मान् । \newline
37. चक्षु॑ष्मा॒न् ताभ्या॒म् ताभ्या॒म् चक्षु॑ष्मा॒(ग्ग्॒) श्चक्षु॑ष्मा॒न् ताभ्या᳚म् । \newline
38. ताभ्या॑ मे॒वैव ताभ्या॒म् ताभ्या॑ मे॒व । \newline
39. ए॒व चक्षु॒ श्चक्षु॑ रे॒वैव चक्षुः॑ । \newline
40. चक्षु॑ रा॒त्मन् ना॒त्मꣳ श्चक्षु॒ श्चक्षु॑ रा॒त्मन्न् । \newline
41. आ॒त्मन् ध॑त्ते धत्त आ॒त्मन् ना॒त्मन् ध॑त्ते । \newline
42. ध॒त्ते॒ ऽग्ने र॒ग्नेर् ध॑त्ते धत्ते॒ ऽग्नेः । \newline
43. अ॒ग्ने र॒ह म॒ह म॒ग्ने र॒ग्ने र॒हम् । \newline
44. अ॒हम् दे॑वय॒ज्यया॑ देवय॒ज्यया॒ ऽह म॒हम् दे॑वय॒ज्यया᳚ । \newline
45. दे॒व॒य॒ज्यया᳚ ऽन्ना॒दो᳚ ऽन्ना॒दो दे॑वय॒ज्यया॑ देवय॒ज्यया᳚ ऽन्ना॒दः । \newline
46. दे॒व॒य॒ज्ययेति॑ देव - य॒ज्यया᳚ । \newline
47. अ॒न्ना॒दो भू॑यासम् भूयास मन्ना॒दो᳚ ऽन्ना॒दो भू॑यासम् । \newline
48. अ॒न्ना॒द इत्य॑न्न - अ॒दः । \newline
49. भू॒या॒स॒ मितीति॑ भूयासम् भूयास॒ मिति॑ । \newline
50. इत्या॑हा॒हे तीत्या॑ह । \newline
51. आ॒हा॒ग्नि र॒ग्नि रा॑हा हा॒ग्निः । \newline
52. अ॒ग्निर् वै वा अ॒ग्नि र॒ग्निर् वै । \newline
53. वै दे॒वाना᳚म् दे॒वानां॒ ॅवै वै दे॒वाना᳚म् । \newline
54. दे॒वाना॑ मन्ना॒दो᳚ ऽन्ना॒दो दे॒वाना᳚म् दे॒वाना॑ मन्ना॒दः । \newline
55. अ॒न्ना॒द स्तेन॒ तेना᳚न्ना॒दो᳚ ऽन्ना॒द स्तेन॑ । \newline
56. अ॒न्ना॒द इत्य॑न्न - अ॒दः । \newline
57. तेनै॒ वैव तेन॒ तेनै॒व । \newline
58. ए॒वान्नाद्य॑ म॒न्नाद्य॑ मे॒वै वान्नाद्य᳚म् । \newline

\textbf{Ghana Paata } \newline

1. ऋ॒तू॒नाम् प्री॑णामि प्रीणाम्यृतू॒ना मृ॑तू॒नाम् प्री॑णा॒मीतीति॑ प्रीणाम्यृतू॒ना मृ॑तू॒नाम् प्री॑णा॒मीति॑ । \newline
2. प्री॒णा॒मीतीति॑ प्रीणामि प्रीणा॒मीत्या॑हा॒हे ति॑ प्रीणामि प्रीणा॒मीत्या॑ह । \newline
3. इत्या॑हा॒हे तीत्या॑ह॒ र्तव॑ ऋ॒तव॑ आ॒हे तीत्या॑ह॒ र्तवः॑ । \newline
4. आ॒ह॒ र्तव॑ ऋ॒तव॑ आहाह॒ र्तवो॒ वै वा ऋ॒तव॑ आहाह॒ र्तवो॒ वै । \newline
5. ऋ॒तवो॒ वै वा ऋ॒तव॑ ऋ॒तवो॒ वै प्र॑या॒जाः प्र॑या॒जा वा ऋ॒तव॑ ऋ॒तवो॒ वै प्र॑या॒जाः । \newline
6. वै प्र॑या॒जाः प्र॑या॒जा वै वै प्र॑या॒जा ऋ॒तू नृ॒तून् प्र॑या॒जा वै वै प्र॑या॒जा ऋ॒तून् । \newline
7. प्र॒या॒जा ऋ॒तू नृ॒तून् प्र॑या॒जाः प्र॑या॒जा ऋ॒तू ने॒वैव र्तून् प्र॑या॒जाः प्र॑या॒जा ऋ॒तू ने॒व । \newline
8. प्र॒या॒जा इति॑ प्र - या॒जाः । \newline
9. ऋ॒तू ने॒वैव र्तू नृ॒तू ने॒व प्री॑णाति प्रीणात्ये॒व र्तू नृ॒तू ने॒व प्री॑णाति । \newline
10. ए॒व प्री॑णाति प्रीणात्ये॒वैव प्री॑णाति॒ ते ते प्री॑णात्ये॒वैव प्री॑णाति॒ ते । \newline
11. प्री॒णा॒ति॒ ते ते प्री॑णाति प्रीणाति ते᳚ ऽस्मा अस्मै॒ ते प्री॑णाति प्रीणाति ते᳚ ऽस्मै । \newline
12. ते᳚ ऽस्मा अस्मै॒ ते ते᳚ ऽस्मै प्री॒ताः प्री॒ता अ॑स्मै॒ ते ते᳚ ऽस्मै प्री॒ताः । \newline
13. अ॒स्मै॒ प्री॒ताः प्री॒ता अ॑स्मा अस्मै प्री॒ता य॑थापू॒र्वं ॅय॑थापू॒र्वम् प्री॒ता अ॑स्मा अस्मै प्री॒ता य॑थापू॒र्वम् । \newline
14. प्री॒ता य॑थापू॒र्वं ॅय॑थापू॒र्वम् प्री॒ताः प्री॒ता य॑थापू॒र्वम् क॑ल्पन्ते कल्पन्ते यथापू॒र्वम् प्री॒ताः प्री॒ता य॑थापू॒र्वम् क॑ल्पन्ते । \newline
15. य॒था॒पू॒र्वम् क॑ल्पन्ते कल्पन्ते यथापू॒र्वं ॅय॑थापू॒र्वम् क॑ल्पन्ते॒ कल्प॑न्ते॒ कल्प॑न्ते कल्पन्ते यथापू॒र्वं ॅय॑थापू॒र्वम् क॑ल्पन्ते॒ कल्प॑न्ते । \newline
16. य॒था॒पू॒र्वमिति॑ यथा - पू॒र्वम् । \newline
17. क॒ल्प॒न्ते॒ कल्प॑न्ते॒ कल्प॑न्ते कल्पन्ते कल्पन्ते॒ कल्प॑न्ते ऽस्मा अस्मै॒ कल्प॑न्ते कल्पन्ते कल्पन्ते॒ कल्प॑न्ते ऽस्मै । \newline
18. कल्प॑न्ते ऽस्मा अस्मै॒ कल्प॑न्ते॒ कल्प॑न्ते ऽस्मा ऋ॒तव॑ ऋ॒तवो᳚ ऽस्मै॒ कल्प॑न्ते॒ कल्प॑न्ते ऽस्मा ऋ॒तवः॑ । \newline
19. अ॒स्मा॒ ऋ॒तव॑ ऋ॒तवो᳚ ऽस्मा अस्मा ऋ॒तवो॒ यो य ऋ॒तवो᳚ ऽस्मा अस्मा ऋ॒तवो॒ यः । \newline
20. ऋ॒तवो॒ यो य ऋ॒तव॑ ऋ॒तवो॒ य ए॒व मे॒वं ॅय ऋ॒तव॑ ऋ॒तवो॒ य ए॒वम् । \newline
21. य ए॒व मे॒वं ॅयो य ए॒वं ॅवेद॒ वेदै॒वं ॅयो य ए॒वं ॅवेद॑ । \newline
22. ए॒वं ॅवेद॒ वेदै॒व मे॒वं ॅवेदा॒ग्नीषोम॑यो र॒ग्नीषोम॑यो॒र् वेदै॒व मे॒वं ॅवेदा॒ग्नीषोम॑योः । \newline
23. वेदा॒ग्नीषोम॑यो र॒ग्नीषोम॑यो॒र् वेद॒ वेदा॒ग्नीषोम॑यो र॒ह म॒ह म॒ग्नीषोम॑यो॒र् वेद॒ वेदा॒ग्नीषोम॑यो र॒हम् । \newline
24. अ॒ग्नीषोम॑यो र॒ह म॒ह म॒ग्नीषोम॑यो र॒ग्नीषोम॑यो र॒हम् दे॑वय॒ज्यया॑ देवय॒ज्यया॒ ऽह म॒ग्नीषोम॑यो र॒ग्नीषोम॑यो र॒हम् दे॑वय॒ज्यया᳚ । \newline
25. अ॒ग्नीषोम॑यो॒रित्य॒ग्नी - सोम॑योः । \newline
26. अ॒हम् दे॑वय॒ज्यया॑ देवय॒ज्यया॒ ऽह म॒हम् दे॑वय॒ज्यया॒ चक्षु॑ष्मा॒(ग्ग्॒) श्चक्षु॑ष्मान् देवय॒ज्यया॒ ऽह म॒हम् दे॑वय॒ज्यया॒ चक्षु॑ष्मान् । \newline
27. दे॒व॒य॒ज्यया॒ चक्षु॑ष्मा॒(ग्ग्॒) श्चक्षु॑ष्मान् देवय॒ज्यया॑ देवय॒ज्यया॒ चक्षु॑ष्मान् भूयासम् भूयास॒म् चक्षु॑ष्मान् देवय॒ज्यया॑ देवय॒ज्यया॒ चक्षु॑ष्मान् भूयासम् । \newline
28. दे॒व॒य॒ज्ययेति॑ देव - य॒ज्यया᳚ । \newline
29. चक्षु॑ष्मान् भूयासम् भूयास॒म् चक्षु॑ष्मा॒(ग्ग्॒) श्चक्षु॑ष्मान् भूयास॒ मितीति॑ भूयास॒म् चक्षु॑ष्मा॒(ग्ग्॒) श्चक्षु॑ष्मान् भूयास॒ मिति॑ । \newline
30. भू॒या॒स॒ मितीति॑ भूयासम् भूयास॒ मित्या॑हा॒हे ति॑ भूयासम् भूयास॒ मित्या॑ह । \newline
31. इत्या॑हा॒हे तीत्या॑हा॒ ग्नीषोमा᳚भ्या म॒ग्नीषोमा᳚भ्या मा॒हे तीत्या॑हा॒ ग्नीषोमा᳚भ्याम् । \newline
32. आ॒हा॒ग्नीषोमा᳚भ्या म॒ग्नीषोमा᳚भ्या माहाहा॒ग्नीषोमा᳚भ्यां॒ ॅवै वा अ॒ग्नीषोमा᳚भ्या माहाहा॒ग्नीषोमा᳚भ्यां॒ ॅवै । \newline
33. अ॒ग्नीषोमा᳚भ्यां॒ ॅवै वा अ॒ग्नीषोमा᳚भ्या म॒ग्नीषोमा᳚भ्यां॒ ॅवै य॒ज्ञो य॒ज्ञो वा अ॒ग्नीषोमा᳚भ्या म॒ग्नीषोमा᳚भ्यां॒ ॅवै य॒ज्ञ्ः । \newline
34. अ॒ग्नीषोमा᳚भ्या॒मित्य॒ग्नी - सोमा᳚भ्याम् । \newline
35. वै य॒ज्ञो य॒ज्ञो वै वै य॒ज्ञ् श्चक्षु॑ष्मा॒(ग्ग्॒) श्चक्षु॑ष्मान्. य॒ज्ञो वै वै य॒ज्ञ् श्चक्षु॑ष्मान् । \newline
36. य॒ज्ञ् श्चक्षु॑ष्मा॒(ग्ग्॒) श्चक्षु॑ष्मान्. य॒ज्ञो य॒ज्ञ् श्चक्षु॑ष्मा॒न् ताभ्या॒म् ताभ्या॒म् चक्षु॑ष्मान्. य॒ज्ञो य॒ज्ञ् श्चक्षु॑ष्मा॒न् ताभ्या᳚म् । \newline
37. चक्षु॑ष्मा॒न् ताभ्या॒म् ताभ्या॒म् चक्षु॑ष्मा॒(ग्ग्॒) श्चक्षु॑ष्मा॒न् ताभ्या॑ मे॒वैव ताभ्या॒म् चक्षु॑ष्मा॒(ग्ग्॒) श्चक्षु॑ष्मा॒न् ताभ्या॑ मे॒व । \newline
38. ताभ्या॑ मे॒वैव ताभ्या॒म् ताभ्या॑ मे॒व चक्षु॒ श्चक्षु॑रे॒व ताभ्या॒म् ताभ्या॑ मे॒व चक्षुः॑ । \newline
39. ए॒व चक्षु॒ श्चक्षु॑रे॒वैव चक्षु॑रा॒त्मन् ना॒त्मꣳ श्चक्षु॑रे॒वैव चक्षु॑रा॒त्मन्न् । \newline
40. चक्षु॑रा॒त्मन् ना॒त्मꣳ श्चक्षु॒ श्चक्षु॑रा॒त्मन् ध॑त्ते धत्त आ॒त्मꣳ श्चक्षु॒ श्चक्षु॑रा॒त्मन् ध॑त्ते । \newline
41. आ॒त्मन् ध॑त्ते धत्त आ॒त्मन् ना॒त्मन् ध॑त्ते॒ ऽग्नेर॒ग्नेर् ध॑त्त आ॒त्मन् ना॒त्मन् ध॑त्ते॒ ऽग्नेः । \newline
42. ध॒त्ते॒ ऽग्नेर॒ग्नेर् ध॑त्ते धत्ते॒ ऽग्नेर॒ह म॒ह म॒ग्नेर् ध॑त्ते धत्ते॒ ऽग्नेर॒हम् । \newline
43. अ॒ग्नेर॒ह म॒ह म॒ग्ने र॒ग्नेर॒हम् दे॑वय॒ज्यया॑ देवय॒ज्यया॒ ऽह म॒ग्ने र॒ग्नेर॒हम् दे॑वय॒ज्यया᳚ । \newline
44. अ॒हम् दे॑वय॒ज्यया॑ देवय॒ज्यया॒ ऽह म॒हम् दे॑वय॒ज्यया᳚ ऽन्ना॒दो᳚ ऽन्ना॒दो दे॑वय॒ज्यया॒ ऽह म॒हम् दे॑वय॒ज्यया᳚ ऽन्ना॒दः । \newline
45. दे॒व॒य॒ज्यया᳚ ऽन्ना॒दो᳚ ऽन्ना॒दो दे॑वय॒ज्यया॑ देवय॒ज्यया᳚ ऽन्ना॒दो भू॑यासम् भूयास मन्ना॒दो दे॑वय॒ज्यया॑ देवय॒ज्यया᳚ ऽन्ना॒दो भू॑यासम् । \newline
46. दे॒व॒य॒ज्ययेति॑ देव - य॒ज्यया᳚ । \newline
47. अ॒न्ना॒दो भू॑यासम् भूयास मन्ना॒दो᳚ ऽन्ना॒दो भू॑यास॒ मितीति॑ भूयास मन्ना॒दो᳚ ऽन्ना॒दो भू॑यास॒ मिति॑ । \newline
48. अ॒न्ना॒द इत्य॑न्न - अ॒दः । \newline
49. भू॒या॒स॒ मितीति॑ भूयासम् भूयास॒ मित्या॑हा॒हे ति॑ भूयासम् भूयास॒ मित्या॑ह । \newline
50. इत्या॑हा॒हे तीत्या॑हा॒ग्नि र॒ग्निरा॒हे तीत्या॑हा॒ग्निः । \newline
51. आ॒हा॒ग्नि र॒ग्नि रा॑हाहा॒ग्निर् वै वा अ॒ग्नि रा॑हाहा॒ग्निर् वै । \newline
52. अ॒ग्निर् वै वा अ॒ग्नि र॒ग्निर् वै दे॒वाना᳚म् दे॒वानां॒ ॅवा अ॒ग्नि र॒ग्निर् वै दे॒वाना᳚म् । \newline
53. वै दे॒वाना᳚म् दे॒वानां॒ ॅवै वै दे॒वाना॑ मन्ना॒दो᳚ ऽन्ना॒दो दे॒वानां॒ ॅवै वै दे॒वाना॑ मन्ना॒दः । \newline
54. दे॒वाना॑ मन्ना॒दो᳚ ऽन्ना॒दो दे॒वाना᳚म् दे॒वाना॑ मन्ना॒द स्तेन॒ तेना᳚न्ना॒दो दे॒वाना᳚म् दे॒वाना॑ मन्ना॒द स्तेन॑ । \newline
55. अ॒न्ना॒द स्तेन॒ तेना᳚न्ना॒दो᳚ ऽन्ना॒द स्तेनै॒वैव तेना᳚न्ना॒दो᳚ ऽन्ना॒द स्तेनै॒व । \newline
56. अ॒न्ना॒द इत्य॑न्न - अ॒दः । \newline
57. तेनै॒वैव तेन॒ तेनै॒वान्नाद्य॑ म॒न्नाद्य॑ मे॒व तेन॒ तेनै॒वान्नाद्य᳚म् । \newline
58. ए॒वान्नाद्य॑ म॒न्नाद्य॑ मे॒वैवान्नाद्य॑ मा॒त्मन् ना॒त्मन् न॒न्नाद्य॑ मे॒वैवान्नाद्य॑ मा॒त्मन्न् । \newline
\pagebreak
\markright{ TS 1.6.11.6  \hfill https://www.vedavms.in \hfill}

\section{ TS 1.6.11.6 }

\textbf{TS 1.6.11.6 } \newline
\textbf{Samhita Paata} \newline

ऽन्नाद्य॑-मा॒त्मन् ध॑त्ते॒ दब्धि॑र॒स्यद॑ब्धो भूयासम॒मुं द॑भेय॒मित्या॑है॒तया॒ वै दब्द्ध्या॑ दे॒वा असु॑रानदभ्नुव॒न्तयै॒व भ्रातृ॑व्यं दभ्नोत्य॒ग्नीषोम॑योर॒हं दे॑वय॒ज्यया॑ वृत्र॒हा भू॑यास॒मित्या॑हा॒ऽग्नीषोमा᳚भ्यां॒ ॅवा इन्द्रो॑ वृ॒त्रम॑ह॒न्ताभ्या॑मे॒व भ्रातृ॑व्यꣳ स्तृणुत इन्द्राग्नि॒योर॒हं दे॑वय॒ज्यये᳚न्द्रिया॒व्य॑न्ना॒दो भू॑यास॒मित्या॑हेन्द्रिया॒व्ये॑वान्ना॒दो भ॑व॒तीन्द्र॑स्या॒ - [ ] \newline

\textbf{Pada Paata} \newline

अ॒न्नाद्य॒मित्य॑न्न - अद्य᳚म् । आ॒त्मन्न् । ध॒त्ते॒ । दब्धिः॑ । अ॒सि॒ । अद॑ब्धः । भू॒या॒स॒म् । अ॒मुम् । द॒भे॒य॒म् । इति॑ । आ॒ह॒ । ए॒तया᳚ । वै । दब्ध्या᳚ । दे॒वाः । असु॑रान् । अ॒द॒भ्नु॒व॒॒न्न् । तया᳚ । ए॒व । भ्रातृ॑व्यम् । द॒भ्नो॒ति॒ । अ॒ग्नीषोम॑यो॒रित्य॒ग्नी - सोम॑योः । अ॒हम् । दे॒व॒य॒ज्ययेति॑ देव - य॒ज्यया᳚ । वृ॒त्र॒हेति॑ वृत्र - हा । भू॒या॒स॒म् । इति॑ । आ॒ह॒ । अ॒ग्नीषोमा᳚भ्या॒मित्य॒ग्नी - सोमा᳚भ्याम् । वै । इन्द्रः॑ । वृ॒त्रम् । अ॒ह॒न्न् । ताभ्या᳚म् । ए॒व । भ्रातृ॑व्यम् । स्तृ॒णु॒ते॒ । इ॒न्द्रा॒ग्नि॒योरिती᳚न्द्र - अ॒ग्नि॒योः । अ॒हम् । दे॒व॒य॒ज्ययेति॑ देव - य॒ज्यया᳚ । इ॒न्द्रि॒या॒वी । अ॒न्ना॒द इत्य॑न्न - अ॒दः । भू॒या॒स॒म् । इति॑ । आ॒ह॒ । इ॒न्द्रि॒या॒वी । ए॒व । अ॒न्ना॒द इत्य॑न्न - अ॒दः । भ॒व॒ति॒ । इन्द्र॑स्य ।  \newline


\textbf{Krama Paata} \newline

अ॒न्नाद्य॑मा॒त्मन्न् । अ॒न्नाद्य॒मित्य॑न्न - अद्य᳚म् । आ॒त्मन्,ध॑त्ते । ध॒त्ते॒ दब्धिः॑ । दब्धि॑रसि । अ॒स्यद॑ब्धः । अद॑ब्धो भूयासम् । भू॒या॒स॒म॒मुम् । अ॒मुम् द॑भेयम् । द॒भे॒य॒मिति॑ । इत्या॑ह । आ॒है॒तया᳚ । ए॒तया॒ वै । वै दब्ध्या᳚ । दब्ध्या॑ दे॒वाः । दे॒वा असु॑रान् । असु॑रानदभ्नुवन्न् । अ॒द॒भ्नु॒व॒न्,तया᳚ । तयै॒व । ए॒व भ्रातृ॑व्यम् । भ्रातृ॑व्यम् दभ्नोति । द॒भ्नो॒त्य॒ग्नीषोम॑योः । अ॒ग्नीषोम॑योर॒हम् । अ॒ग्नीषोम॑यो॒रित्य॒ग्नी - सोम॑योः । अ॒हम् दे॑वय॒ज्यया᳚ । दे॒व॒य॒ज्यया॑ वृत्र॒हा । दे॒व॒य॒ज्ययेति॑ देव - य॒ज्यया᳚ । वृ॒त्र॒हा भू॑यासम् । वृ॒त्र॒हेति॑ वृत्र - हा । भू॒या॒स॒मिति॑ । इत्या॑ह । आ॒हा॒ग्नीषोमा᳚भ्याम् । अ॒ग्नीषोमा᳚भ्यां॒ ॅवै । अ॒ग्नीषोमा᳚भ्या॒मित्य॒ग्नी - सोमा᳚भ्याम् । वा इन्द्रः॑ । इन्द्रो॑ वृ॒त्रम् । वृ॒त्रम॑हन्न् । अ॒ह॒न् ताभ्या᳚म् । ताभ्या॑मे॒व । ए॒व भ्रातृ॑व्यम् । भ्रातृ॑व्यꣳ स्तृणुते । स्तृ॒णु॒त॒ इ॒न्द्रा॒ग्नि॒योः । इ॒न्द्रा॒ग्नि॒योर॒हम् । इ॒न्द्रा॒ग्नि॒योरिती᳚न्द्र - अ॒ग्नि॒योः । अ॒हम् दे॑वय॒ज्यया᳚ । दे॒व॒य॒ज्यये᳚न्द्रिया॒वी । दे॒व॒य॒ज्ययेति॑ देव - य॒ज्यया᳚ । इ॒न्द्रि॒या॒व्य॑न्ना॒दः । अ॒न्ना॒दो भू॑यासम् । अ॒न्ना॒द इत्य॑न्न - अ॒दः । भू॒या॒स॒मिति॑ । इत्या॑ह । आ॒हे॒न्द्रि॒या॒वी । इ॒न्द्रि॒या॒व्ये॑व । ए॒वान्ना॒दः । अ॒न्ना॒दो भ॑वति । अ॒न्ना॒द इत्य॑न्न - अ॒दः । भ॒व॒तीन्द्र॑स्य । इन्द्र॑स्या॒हम् \newline

\textbf{Jatai Paata} \newline

1. अ॒न्नाद्य॑ मा॒त्मन् ना॒त्मन् न॒न्नाद्य॑ म॒न्नाद्य॑ मा॒त्मन्न् । \newline
2. अ॒न्नाद्य॒मित्य॑न्न - अद्य᳚म् । \newline
3. आ॒त्मन् ध॑त्ते धत्त आ॒त्मन् ना॒त्मन् ध॑त्ते । \newline
4. ध॒त्ते॒ दब्धि॒र् दब्धि॑र् धत्ते धत्ते॒ दब्धिः॑ । \newline
5. दब्धि॑ रस्यसि॒ दब्धि॒र् दब्धि॑ रसि । \newline
6. अ॒स्य द॒ब्धो ऽद॑ब्धो ऽस्य॒स्यद॑ब्धः । \newline
7. अद॑ब्धो भूयासम् भूयास॒ मद॒ब्धो ऽद॑ब्धो भूयासम् । \newline
8. भू॒या॒स॒ म॒मु म॒मुम् भू॑यासम् भूयास म॒मुम् । \newline
9. अ॒मुम् द॑भेयम् दभेय म॒मु म॒मुम् द॑भेयम् । \newline
10. द॒भे॒य॒ मितीति॑ दभेयम् दभेय॒ मिति॑ । \newline
11. इत्या॑हा॒हे तीत्या॑ह । \newline
12. आ॒है॒त यै॒तया॑ ऽऽहा है॒तया᳚ । \newline
13. ए॒तया॒ वै वा ए॒त यै॒तया॒ वै । \newline
14. वै दब्ध्या॒ दब्ध्या॒ वै वै दब्ध्या᳚ । \newline
15. दब्ध्या॑ दे॒वा दे॒वा दब्ध्या॒ दब्ध्या॑ दे॒वाः । \newline
16. दे॒वा असु॑रा॒ नसु॑रान् दे॒वा दे॒वा असु॑रान् । \newline
17. असु॑रा नदभ्नुवन् नदभ्नुव॒न् नसु॑रा॒ नसु॑रा नदभ्नुवन्न् । \newline
18. अ॒द॒भ्नु॒व॒न् तया॒ तया॑ ऽदभ्नुवन् नदभ्नुव॒न् तया᳚ । \newline
19. तयै॒वैव तया॒ तयै॒व । \newline
20. ए॒व भ्रातृ॑व्य॒म् भ्रातृ॑व्य मे॒वैव भ्रातृ॑व्यम् । \newline
21. भ्रातृ॑व्यम् दभ्नोति दभ्नोति॒ भ्रातृ॑व्य॒म् भ्रातृ॑व्यम् दभ्नोति । \newline
22. द॒भ्नो॒ त्य॒ग्नीषोम॑यो र॒ग्नीषोम॑योर् दभ्नोति दभ्नो त्य॒ग्नीषोम॑योः । \newline
23. अ॒ग्नीषोम॑यो र॒ह म॒ह म॒ग्नीषोम॑यो र॒ग्नीषोम॑यो र॒हम् । \newline
24. अ॒ग्नीषोम॑यो॒रित्य॒ग्नी - सोम॑योः । \newline
25. अ॒हम् दे॑वय॒ज्यया॑ देवय॒ज्यया॒ ऽह म॒हम् दे॑वय॒ज्यया᳚ । \newline
26. दे॒व॒य॒ज्यया॑ वृत्र॒हा वृ॑त्र॒हा दे॑वय॒ज्यया॑ देवय॒ज्यया॑ वृत्र॒हा । \newline
27. दे॒व॒य॒ज्ययेति॑ देव - य॒ज्यया᳚ । \newline
28. वृ॒त्र॒हा भू॑यासम् भूयासं ॅवृत्र॒हा वृ॑त्र॒हा भू॑यासम् । \newline
29. वृ॒त्र॒हेति॑ वृत्र - हा । \newline
30. भू॒या॒स॒ मितीति॑ भूयासम् भूयास॒ मिति॑ । \newline
31. इत्या॑हा॒हे तीत्या॑ह । \newline
32. आ॒हा॒ग्नीषोमा᳚भ्या म॒ग्नीषोमा᳚भ्या माहा हा॒ग्नीषोमा᳚भ्याम् । \newline
33. अ॒ग्नीषोमा᳚भ्यां॒ ॅवै वा अ॒ग्नीषोमा᳚भ्या म॒ग्नीषोमा᳚भ्यां॒ ॅवै । \newline
34. अ॒ग्नीषोमा᳚भ्या॒मित्य॒ग्नी - सोमा᳚भ्याम् । \newline
35. वा इन्द्र॒ इन्द्रो॒ वै वा इन्द्रः॑ । \newline
36. इन्द्रो॑ वृ॒त्रं ॅवृ॒त्र मिन्द्र॒ इन्द्रो॑ वृ॒त्रम् । \newline
37. वृ॒त्र म॑हन् नहन् वृ॒त्रं ॅवृ॒त्र म॑हन्न् । \newline
38. अ॒ह॒न् ताभ्या॒म् ताभ्या॑ महन् नह॒न् ताभ्या᳚म् । \newline
39. ताभ्या॑ मे॒वैव ताभ्या॒म् ताभ्या॑ मे॒व । \newline
40. ए॒व भ्रातृ॑व्य॒म् भ्रातृ॑व्य मे॒वैव भ्रातृ॑व्यम् । \newline
41. भ्रातृ॑व्यꣳ स्तृणुते स्तृणुते॒ भ्रातृ॑व्य॒म् भ्रातृ॑व्यꣳ स्तृणुते । \newline
42. स्तृ॒णु॒त॒ इ॒न्द्रा॒ग्नि॒यो रि॑न्द्राग्नि॒योः स्तृ॑णुते स्तृणुत इन्द्राग्नि॒योः । \newline
43. इ॒न्द्रा॒ग्नि॒यो र॒ह म॒ह मि॑न्द्राग्नि॒यो रि॑न्द्राग्नि॒यो र॒हम् । \newline
44. इ॒न्द्रा॒ग्नि॒योरिती᳚न्द्र - अ॒ग्नि॒योः । \newline
45. अ॒हम् दे॑वय॒ज्यया॑ देवय॒ज्यया॒ ऽह म॒हम् दे॑वय॒ज्यया᳚ । \newline
46. दे॒व॒य॒ज्य ये᳚न्द्रिया॒वी न्द्रि॑या॒वी दे॑वय॒ज्यया॑ देवय॒ज्यये᳚ न्द्रिया॒वी । \newline
47. दे॒व॒य॒ज्ययेति॑ देव - य॒ज्यया᳚ । \newline
48. इ॒न्द्रि॒या॒ व्य॑न्ना॒दो᳚ ऽन्ना॒द इ॑न्द्रिया॒वी न्द्रि॑या॒ व्य॑न्ना॒दः । \newline
49. अ॒न्ना॒दो भू॑यासम् भूयास मन्ना॒दो᳚ ऽन्ना॒दो भू॑यासम् । \newline
50. अ॒न्ना॒द इत्य॑न्न - अ॒दः । \newline
51. भू॒या॒स॒ मितीति॑ भूयासम् भूयास॒ मिति॑ । \newline
52. इत्या॑हा॒हे तीत्या॑ह । \newline
53. आ॒हे॒ न्द्रि॒या॒वी न्द्रि॑या॒ व्या॑हाहे न्द्रिया॒वी । \newline
54. इ॒न्द्रि॒या॒ व्ये॑वैवे न्द्रि॑या॒वी न्द्रि॑या॒ व्ये॑व । \newline
55. ए॒वान्ना॒दो᳚ ऽन्ना॒द ए॒वै वान्ना॒दः । \newline
56. अ॒न्ना॒दो भ॑वति भवत्यन्ना॒दो᳚ ऽन्ना॒दो भ॑वति । \newline
57. अ॒न्ना॒द इत्य॑न्न - अ॒दः । \newline
58. भ॒व॒ती न्द्र॒स्ये न्द्र॑स्य भवति भव॒ती न्द्र॑स्य । \newline
59. इन्द्र॑स्या॒ह म॒ह मिन्द्र॒स्ये न्द्र॑स्या॒हम् । \newline

\textbf{Ghana Paata } \newline

1. अ॒न्नाद्य॑ मा॒त्मन् ना॒त्मन् न॒न्नाद्य॑ म॒न्नाद्य॑ मा॒त्मन् ध॑त्ते धत्त आ॒त्मन् न॒न्नाद्य॑ म॒न्नाद्य॑ मा॒त्मन् ध॑त्ते । \newline
2. अ॒न्नाद्य॒मित्य॑न्न - अद्य᳚म् । \newline
3. आ॒त्मन् ध॑त्ते धत्त आ॒त्मन् ना॒त्मन् ध॑त्ते॒ दब्धि॒र् दब्धि॑र् धत्त आ॒त्मन् ना॒त्मन् ध॑त्ते॒ दब्धिः॑ । \newline
4. ध॒त्ते॒ दब्धि॒र् दब्धि॑र् धत्ते धत्ते॒ दब्धि॑ रस्यसि॒ दब्धि॑र् धत्ते धत्ते॒ दब्धि॑रसि । \newline
5. दब्धि॑ रस्यसि॒ दब्धि॒र् दब्धि॑ र॒स्यद॒ब्धो ऽद॑ब्धो ऽसि॒ दब्धि॒र् दब्धि॑ र॒स्यद॑ब्धः । \newline
6. अ॒स्यद॒ब्धो ऽद॑ब्धो ऽस्य॒स्यद॑ब्धो भूयासम् भूयास॒ मद॑ब्धो ऽस्य॒स्यद॑ब्धो भूयासम् । \newline
7. अद॑ब्धो भूयासम् भूयास॒ मद॒ब्धो ऽद॑ब्धो भूयास म॒मु म॒मुम् भू॑यास॒ मद॒ब्धो ऽद॑ब्धो भूयास म॒मुम् । \newline
8. भू॒या॒स॒ म॒मु म॒मुम् भू॑यासम् भूयास म॒मुम् द॑भेयम् दभेय म॒मुम् भू॑यासम् भूयास म॒मुम् द॑भेयम् । \newline
9. अ॒मुम् द॑भेयम् दभेय म॒मु म॒मुम् द॑भेय॒ मितीति॑ दभेय म॒मु म॒मुम् द॑भेय॒ मिति॑ । \newline
10. द॒भे॒य॒ मितीति॑ दभेयम् दभेय॒ मित्या॑हा॒हे ति॑ दभेयम् दभेय॒ मित्या॑ह । \newline
11. इत्या॑हा॒हे तीत्या॑है॒तयै॒तया॒ ऽऽहे तीत्या॑है॒तया᳚ । \newline
12. आ॒है॒तयै॒तया॑ ऽऽहाहै॒तया॒ वै वा ए॒तया॑ ऽऽहाहै॒तया॒ वै । \newline
13. ए॒तया॒ वै वा ए॒तयै॒तया॒ वै दब्ध्या॒ दब्ध्या॒ वा ए॒तयै॒तया॒ वै दब्ध्या᳚ । \newline
14. वै दब्ध्या॒ दब्ध्या॒ वै वै दब्ध्या॑ दे॒वा दे॒वा दब्ध्या॒ वै वै दब्ध्या॑ दे॒वाः । \newline
15. दब्ध्या॑ दे॒वा दे॒वा दब्ध्या॒ दब्ध्या॑ दे॒वा असु॑रा॒ नसु॑रान् दे॒वा दब्ध्या॒ दब्ध्या॑ दे॒वा असु॑रान् । \newline
16. दे॒वा असु॑रा॒ नसु॑रान् दे॒वा दे॒वा असु॑रा नदभ्नुवन् नदभ्नुव॒न् नसु॑रान् दे॒वा दे॒वा असु॑रा नदभ्नुवन्न् । \newline
17. असु॑रा नदभ्नुवन् नदभ्नुव॒न् नसु॑रा॒ नसु॑रा नदभ्नुव॒न् तया॒ तया॑ ऽदभ्नुव॒न् नसु॑रा॒ नसु॑रा नदभ्नुव॒न् तया᳚ । \newline
18. अ॒द॒भ्नु॒व॒न् तया॒ तया॑ ऽदभ्नुवन् नदभ्नुव॒न् तयै॒वैव तया॑ ऽदभ्नुवन् नदभ्नुव॒न् तयै॒व । \newline
19. तयै॒वैव तया॒ तयै॒व भ्रातृ॑व्य॒म् भ्रातृ॑व्य मे॒व तया॒ तयै॒व भ्रातृ॑व्यम् । \newline
20. ए॒व भ्रातृ॑व्य॒म् भ्रातृ॑व्य मे॒वैव भ्रातृ॑व्यम् दभ्नोति दभ्नोति॒ भ्रातृ॑व्य मे॒वैव भ्रातृ॑व्यम् दभ्नोति । \newline
21. भ्रातृ॑व्यम् दभ्नोति दभ्नोति॒ भ्रातृ॑व्य॒म् भ्रातृ॑व्यम् दभ्नोत्य॒ग्नीषोम॑यो र॒ग्नीषोम॑योर् दभ्नोति॒ भ्रातृ॑व्य॒म् भ्रातृ॑व्यम् दभ्नोत्य॒ग्नीषोम॑योः । \newline
22. द॒भ्नो॒ त्य॒ग्नीषोम॑यो र॒ग्नीषोम॑योर् दभ्नोति दभ्नो त्य॒ग्नीषोम॑योर॒ह म॒ह म॒ग्नीषोम॑योर् दभ्नोति दभ्नो त्य॒ग्नीषोम॑योर॒हम् । \newline
23. अ॒ग्नीषोम॑योर॒ह म॒ह म॒ग्नीषोम॑यो र॒ग्नीषोम॑योर॒हम् दे॑वय॒ज्यया॑ देवय॒ज्यया॒ ऽह म॒ग्नीषोम॑यो र॒ग्नीषोम॑योर॒हम् दे॑वय॒ज्यया᳚ । \newline
24. अ॒ग्नीषोम॑यो॒रित्य॒ग्नी - सोम॑योः । \newline
25. अ॒हम् दे॑वय॒ज्यया॑ देवय॒ज्यया॒ ऽह म॒हम् दे॑वय॒ज्यया॑ वृत्र॒हा वृ॑त्र॒हा दे॑वय॒ज्यया॒ ऽह म॒हम् दे॑वय॒ज्यया॑ वृत्र॒हा । \newline
26. दे॒व॒य॒ज्यया॑ वृत्र॒हा वृ॑त्र॒हा दे॑वय॒ज्यया॑ देवय॒ज्यया॑ वृत्र॒हा भू॑यासम् भूयासं ॅवृत्र॒हा दे॑वय॒ज्यया॑ देवय॒ज्यया॑ वृत्र॒हा भू॑यासम् । \newline
27. दे॒व॒य॒ज्ययेति॑ देव - य॒ज्यया᳚ । \newline
28. वृ॒त्र॒हा भू॑यासम् भूयासं ॅवृत्र॒हा वृ॑त्र॒हा भू॑यास॒ मितीति॑ भूयासं ॅवृत्र॒हा वृ॑त्र॒हा भू॑यास॒ मिति॑ । \newline
29. वृ॒त्र॒हेति॑ वृत्र - हा । \newline
30. भू॒या॒स॒ मितीति॑ भूयासम् भूयास॒ मित्या॑हा॒हे ति॑ भूयासम् भूयास॒ मित्या॑ह । \newline
31. इत्या॑हा॒हे तीत्या॑हा॒ग्नीषोमा᳚भ्या म॒ग्नीषोमा᳚भ्या मा॒हे तीत्या॑हा॒ग्नीषोमा᳚भ्याम् । \newline
32. आ॒हा॒ग्नीषोमा᳚भ्या म॒ग्नीषोमा᳚भ्या माहाहा॒ग्नीषोमा᳚भ्यां॒ ॅवै वा अ॒ग्नीषोमा᳚भ्या माहाहा॒ग्नीषोमा᳚भ्यां॒ ॅवै । \newline
33. अ॒ग्नीषोमा᳚भ्यां॒ ॅवै वा अ॒ग्नीषोमा᳚भ्या म॒ग्नीषोमा᳚भ्यां॒ ॅवा इन्द्र॒ इन्द्रो॒ वा अ॒ग्नीषोमा᳚भ्या म॒ग्नीषोमा᳚भ्यां॒ ॅवा इन्द्रः॑ । \newline
34. अ॒ग्नीषोमा᳚भ्या॒मित्य॒ग्नी - सोमा᳚भ्याम् । \newline
35. वा इन्द्र॒ इन्द्रो॒ वै वा इन्द्रो॑ वृ॒त्रं ॅवृ॒त्र मिन्द्रो॒ वै वा इन्द्रो॑ वृ॒त्रम् । \newline
36. इन्द्रो॑ वृ॒त्रं ॅवृ॒त्र मिन्द्र॒ इन्द्रो॑ वृ॒त्र म॑हन् नहन् वृ॒त्र मिन्द्र॒ इन्द्रो॑ वृ॒त्र म॑हन्न् । \newline
37. वृ॒त्र म॑हन् नहन् वृ॒त्रं ॅवृ॒त्र म॑ह॒न् ताभ्या॒म् ताभ्या॑ महन् वृ॒त्रं ॅवृ॒त्र म॑ह॒न् ताभ्या᳚म् । \newline
38. अ॒ह॒न् ताभ्या॒म् ताभ्या॑ महन् नह॒न् ताभ्या॑ मे॒वैव ताभ्या॑ महन् नह॒न् ताभ्या॑ मे॒व । \newline
39. ताभ्या॑ मे॒वैव ताभ्या॒म् ताभ्या॑ मे॒व भ्रातृ॑व्य॒म् भ्रातृ॑व्य मे॒व ताभ्या॒म् ताभ्या॑ मे॒व भ्रातृ॑व्यम् । \newline
40. ए॒व भ्रातृ॑व्य॒म् भ्रातृ॑व्य मे॒वैव भ्रातृ॑व्यꣳ स्तृणुते स्तृणुते॒ भ्रातृ॑व्य मे॒वैव भ्रातृ॑व्यꣳ स्तृणुते । \newline
41. भ्रातृ॑व्यꣳ स्तृणुते स्तृणुते॒ भ्रातृ॑व्य॒म् भ्रातृ॑व्यꣳ स्तृणुत इन्द्राग्नि॒यो रि॑न्द्राग्नि॒योः स्तृ॑णुते॒ भ्रातृ॑व्य॒म् भ्रातृ॑व्यꣳ स्तृणुत इन्द्राग्नि॒योः । \newline
42. स्तृ॒णु॒त॒ इ॒न्द्रा॒ग्नि॒यो रि॑न्द्राग्नि॒योः स्तृ॑णुते स्तृणुत इन्द्राग्नि॒योर॒ह म॒ह मि॑न्द्राग्नि॒योः स्तृ॑णुते स्तृणुत इन्द्राग्नि॒योर॒हम् । \newline
43. इ॒न्द्रा॒ग्नि॒योर॒ह म॒ह मि॑न्द्राग्नि॒यो रि॑न्द्राग्नि॒योर॒हम् दे॑वय॒ज्यया॑ देवय॒ज्यया॒ ऽह मि॑न्द्राग्नि॒यो रि॑न्द्राग्नि॒योर॒हम् दे॑वय॒ज्यया᳚ । \newline
44. इ॒न्द्रा॒ग्नि॒योरिती᳚न्द्र - अ॒ग्नि॒योः । \newline
45. अ॒हम् दे॑वय॒ज्यया॑ देवय॒ज्यया॒ ऽह म॒हम् दे॑वय॒ज्य ये᳚न्द्रिया॒वीन्द्रि॑या॒वी दे॑वय॒ज्यया॒ ऽह म॒हम् दे॑वय॒ज्यये᳚न्द्रिया॒वी । \newline
46. दे॒व॒य॒ज्य ये᳚न्द्रिया॒वीन्द्रि॑या॒वी दे॑वय॒ज्यया॑ देवय॒ज्य ये᳚न्द्रिया॒व्य॑न्ना॒दो᳚ ऽन्ना॒द इ॑न्द्रिया॒वी दे॑वय॒ज्यया॑ देवय॒ज्य ये᳚न्द्रिया॒व्य॑न्ना॒दः । \newline
47. दे॒व॒य॒ज्ययेति॑ देव - य॒ज्यया᳚ । \newline
48. इ॒न्द्रि॒या॒व्य॑न्ना॒दो᳚ ऽन्ना॒द इ॑न्द्रिया॒वी न्द्रि॑या॒व्य॑न्ना॒दो भू॑यासम् भूयास मन्ना॒द इ॑न्द्रिया॒वी न्द्रि॑या॒व्य॑न्ना॒दो भू॑यासम् । \newline
49. अ॒न्ना॒दो भू॑यासम् भूयास मन्ना॒दो᳚ ऽन्ना॒दो भू॑यास॒ मितीति॑ भूयास मन्ना॒दो᳚ ऽन्ना॒दो भू॑यास॒ मिति॑ । \newline
50. अ॒न्ना॒द इत्य॑न्न - अ॒दः । \newline
51. भू॒या॒स॒ मितीति॑ भूयासम् भूयास॒ मित्या॑हा॒हे ति॑ भूयासम् भूयास॒ मित्या॑ह । \newline
52. इत्या॑हा॒हे तीत्या॑हे न्द्रिया॒वी न्द्रि॑या॒व्या॑हे तीत्या॑हे न्द्रिया॒वी । \newline
53. आ॒हे॒ न्द्रि॒या॒वी न्द्रि॑या॒व्या॑हाहे न्द्रिया॒व्ये॑वैवे न्द्रि॑या॒व्या॑हाहे न्द्रिया॒व्ये॑व । \newline
54. इ॒न्द्रि॒या॒व्ये॑वैवे न्द्रि॑या॒वी न्द्रि॑या॒व्ये॑वान्ना॒दो᳚ ऽन्ना॒द ए॒वे न्द्रि॑या॒वी न्द्रि॑या॒व्ये॑वान्ना॒दः । \newline
55. ए॒वान्ना॒दो᳚ ऽन्ना॒द ए॒वैवान्ना॒दो भ॑वति भवत्यन्ना॒द ए॒वैवान्ना॒दो भ॑वति । \newline
56. अ॒न्ना॒दो भ॑वति भवत्यन्ना॒दो᳚ ऽन्ना॒दो भ॑व॒तीन्द्र॒स्ये न्द्र॑स्य भवत्यन्ना॒दो᳚ ऽन्ना॒दो भ॑व॒तीन्द्र॑स्य । \newline
57. अ॒न्ना॒द इत्य॑न्न - अ॒दः । \newline
58. भ॒व॒तीन्द्र॒स्ये न्द्र॑स्य भवति भव॒तीन्द्र॑स्या॒ह म॒ह मिन्द्र॑स्य भवति भव॒तीन्द्र॑स्या॒हम् । \newline
59. इन्द्र॑स्या॒ह म॒ह मिन्द्र॒स्ये न्द्र॑स्या॒हम् दे॑वय॒ज्यया॑ देवय॒ज्यया॒ ऽह मिन्द्र॒स्ये न्द्र॑स्या॒हम् दे॑वय॒ज्यया᳚ । \newline
\pagebreak
\markright{ TS 1.6.11.7  \hfill https://www.vedavms.in \hfill}

\section{ TS 1.6.11.7 }

\textbf{TS 1.6.11.7 } \newline
\textbf{Samhita Paata} \newline

ऽहं दे॑वय॒ज्यये᳚न्द्रिया॒वी भू॑यास॒मित्या॑हेन्द्रिया॒व्ये॑व भ॑वति महे॒न्द्रस्या॒ऽहं दे॑वय॒ज्यया॑ जे॒मानं॑ महि॒मानं॑ गमेय॒मित्या॑ह जे॒मान॑मे॒व म॑हि॒मानं॑ गच्छत्य॒ग्नेः स्वि॑ष्ट॒कृतो॒ऽहं दे॑वय॒ज्यया ऽऽयु॑ष्मान्. य॒ज्ञेन॑ प्रति॒ष्ठां ग॑मेय॒मित्या॒-हायु॑रे॒वात्मन् ध॑त्ते॒ प्रति॑ य॒ज्ञेन॑ -तिष्ठति ॥( प्र॒ति॒ष्ठा-म॑ह्व॒दस्तु॑-वि॒द्युतं॑-ॅवस॒न्तं- \newline

\textbf{Pada Paata} \newline

अ॒हम् । दे॒व॒य॒ज्ययेति॑ देव-य॒ज्यया᳚ । इ॒न्द्रि॒या॒वी । भू॒या॒स॒म् । इति॑ । आ॒ह॒ । इ॒न्द्रि॒या॒वी । ए॒व । भ॒व॒ति॒ । म॒हे॒न्द्रस्येति॑ महा - इ॒न्द्रस्य॑ । अ॒हम् । दे॒व॒य॒ज्ययेति॑ देव - य॒ज्यया᳚ । जे॒मान᳚म् । म॒हि॒मान᳚म् । ग॒मे॒य॒म् । इति॑ । आ॒ह॒ । जे॒मान᳚म् । ए॒व । म॒हि॒मान᳚म् । ग॒च्छ॒ति॒ । अ॒ग्नेः । स्वि॒ष्ट॒कृत॒ इति॑ स्विष्ट - कृतः॑ । अ॒हम् । दे॒व॒य॒ज्ययेति॑ देव - य॒ज्यया᳚ । आयु॑ष्मान् । य॒ज्ञेन॑ । प्र॒ति॒ष्ठामिति॑ प्रति- स्थाम् । ग॒मे॒य॒म् । इति॑ । आ॒ह॒ । आयुः॑ । ए॒व । आ॒त्मन्न् । ध॒त्ते॒ । प्रतीति॑ । य॒ज्ञेन॑ । ति॒ष्ठ॒ति॒ ॥  \newline


\textbf{Krama Paata} \newline

अ॒हम् दे॑वय॒ज्यया᳚ । दे॒व॒य॒ज्यये᳚न्द्रिया॒वी । दे॒व॒य॒ज्ययेति॑ देव - य॒ज्यया᳚ । इ॒न्द्रि॒या॒वी भू॑यासम् । भू॒या॒स॒मिति॑ । इत्या॑ह । आ॒हे॒न्द्रि॒या॒वी । इ॒न्द्रि॒या॒व्ये॑व । ए॒व भ॑वति । भ॒व॒ति॒ म॒हे॒न्द्रस्य॑ । म॒हे॒न्द्रस्या॒हम् । म॒हे॒न्द्रस्येति॑ महा - इ॒न्द्रस्य॑ । अ॒हम् दे॑वय॒ज्यया᳚ । दे॒व॒य॒ज्यया॑ जे॒मान᳚म् । दे॒व॒य॒ज्ययेति॑ देव - य॒ज्यया᳚ । जे॒मान॑म् महि॒मान॑म् । म॒हि॒मान॑म् गमेयम् । ग॒मे॒य॒मिति॑ । इत्या॑ह । आ॒ह॒ जे॒मान᳚म् । जे॒मान॑मे॒व । ए॒व म॑हि॒मान᳚म् । म॒हि॒मान॑म् गच्छति । ग॒च्छ॒त्य॒ग्नेः । अ॒ग्नेः स्वि॑ष्ट॒कृतः॑ । स्वि॒ष्ट॒कृतो॒ऽहम् ॥ स्वि॒ष्ट॒कृत॒ इति॑ स्विष्ट - कृतः॑ । अ॒हम् दे॑वय॒ज्यया᳚ । दे॒व॒य॒ज्यया ऽऽयु॑ष्मान् । दे॒व॒य॒ज्ययेति॑ देव - य॒ज्यया᳚ । आयु॑ष्मान्. य॒ज्ञेन॑ । य॒ज्ञेन॑ प्रति॒ष्ठाम् । प्र॒ति॒ष्ठाम् ग॑मेयम् । प्र॒ति॒ष्ठामिति॑ प्रति - स्थाम् । ग॒मे॒य॒मिति॑ । इत्या॑ह । आ॒हायुः॑ । आयु॑रे॒व । ए॒वात्मन्न् । आ॒त्मन् ध॑त्ते । ध॒त्ते॒ प्रति॑ । प्रति॑ य॒ज्ञेन॑ । य॒ज्ञेन॑ तिष्ठति । ति॒ष्ठ॒तीति॑ तिष्ठति । \newline

\textbf{Jatai Paata} \newline

1. अ॒हम् दे॑वय॒ज्यया॑ देवय॒ज्यया॒ ऽह म॒हम् दे॑वय॒ज्यया᳚ । \newline
2. दे॒व॒य॒ज्य ये᳚न्द्रिया॒वी न्द्रि॑या॒वी दे॑वय॒ज्यया॑ देवय॒ज्य ये᳚न्द्रिया॒वी । \newline
3. दे॒व॒य॒ज्ययेति॑ देव - य॒ज्यया᳚ । \newline
4. इ॒न्द्रि॒या॒वी भू॑यासम् भूयास मिन्द्रिया॒वी न्द्रि॑या॒वी भू॑यासम् । \newline
5. भू॒या॒स॒ मितीति॑ भूयासम् भूयास॒ मिति॑ । \newline
6. इत्या॑हा॒हे तीत्या॑ह । \newline
7. आ॒हे॒ न्द्रि॒या॒वी न्द्रि॑या॒ व्या॑हाहे न्द्रिया॒वी । \newline
8. इ॒न्द्रि॒या॒ व्ये॑वैवे न्द्रि॑या॒वी न्द्रि॑या॒ व्ये॑व । \newline
9. ए॒व भ॑वति भव त्ये॒वैव भ॑वति । \newline
10. भ॒व॒ति॒ म॒हे॒न्द्रस्य॑ महे॒न्द्रस्य॑ भवति भवति महे॒न्द्रस्य॑ । \newline
11. म॒हे॒ न्द्रस्या॒ह म॒हम् म॑हे॒न्द्रस्य॑ महे॒ न्द्रस्या॒हम् । \newline
12. म॒हे॒न्द्रस्येति॑ महा - इ॒न्द्रस्य॑ । \newline
13. अ॒हम् दे॑वय॒ज्यया॑ देवय॒ज्यया॒ ऽह म॒हम् दे॑वय॒ज्यया᳚ । \newline
14. दे॒व॒य॒ज्यया॑ जे॒मान॑म् जे॒मान॑म् देवय॒ज्यया॑ देवय॒ज्यया॑ जे॒मान᳚म् । \newline
15. दे॒व॒य॒ज्ययेति॑ देव - य॒ज्यया᳚ । \newline
16. जे॒मान॑म् महि॒मान॑म् महि॒मान॑म् जे॒मान॑म् जे॒मान॑म् महि॒मान᳚म् । \newline
17. म॒हि॒मान॑म् गमेयम् गमेयम् महि॒मान॑म् महि॒मान॑म् गमेयम् । \newline
18. ग॒मे॒य॒ मितीति॑ गमेयम् गमेय॒ मिति॑ । \newline
19. इत्या॑हा॒हे तीत्या॑ह । \newline
20. आ॒ह॒ जे॒मान॑म् जे॒मान॑ माहाह जे॒मान᳚म् । \newline
21. जे॒मान॑ मे॒वैव जे॒मान॑म् जे॒मान॑ मे॒व । \newline
22. ए॒व म॑हि॒मान॑म् महि॒मान॑ मे॒वैव म॑हि॒मान᳚म् । \newline
23. म॒हि॒मान॑म् गच्छति गच्छति महि॒मान॑म् महि॒मान॑म् गच्छति । \newline
24. ग॒च्छ॒त्य॒ग्ने र॒ग्नेर् ग॑च्छति गच्छत्य॒ग्नेः । \newline
25. अ॒ग्नेः स्वि॑ष्ट॒कृतः॑ स्विष्ट॒कृतो॒ ऽग्नेर॒ग्नेः स्वि॑ष्ट॒कृतः॑ । \newline
26. स्वि॒ष्ट॒कृतो॒ ऽह म॒हꣳ स्वि॑ष्ट॒कृतः॑ स्विष्ट॒कृतो॒ ऽहम् । \newline
27. स्वि॒ष्ट॒कृत॒ इति॑ स्विष्ट - कृतः॑ । \newline
28. अ॒हम् दे॑वय॒ज्यया॑ देवय॒ज्यया॒ ऽह म॒हम् दे॑वय॒ज्यया᳚ । \newline
29. दे॒व॒य॒ज्यया ऽऽयु॑ष्मा॒ नायु॑ष्मान् देवय॒ज्यया॑ देवय॒ज्यया ऽऽयु॑ष्मान् । \newline
30. दे॒व॒य॒ज्ययेति॑ देव - य॒ज्यया᳚ । \newline
31. आयु॑ष्मान्. य॒ज्ञेन॑ य॒ज्ञेनायु॑ष्मा॒ नायु॑ष्मान्. य॒ज्ञेन॑ । \newline
32. य॒ज्ञेन॑ प्रति॒ष्ठाम् प्र॑ति॒ष्ठां ॅय॒ज्ञेन॑ य॒ज्ञेन॑ प्रति॒ष्ठाम् । \newline
33. प्र॒ति॒ष्ठाम् ग॑मेयम् गमेयम् प्रति॒ष्ठाम् प्र॑ति॒ष्ठाम् ग॑मेयम् । \newline
34. प्र॒ति॒ष्ठामिति॑ प्रति - स्थाम् । \newline
35. ग॒मे॒य॒ मितीति॑ गमेयम् गमेय॒ मिति॑ । \newline
36. इत्या॑हा॒हे तीत्या॑ह । \newline
37. आ॒हायु॒ रायु॑ राहा॒हायुः॑ । \newline
38. आयु॑ रे॒वैवायु॒ रायु॑रे॒व । \newline
39. ए॒वात्मन् ना॒त्मन् ने॒वैवात्मन्न् । \newline
40. आ॒त्मन् ध॑त्ते धत्त आ॒त्मन् ना॒त्मन् ध॑त्ते । \newline
41. ध॒त्ते॒ प्रति॒ प्रति॑ धत्ते धत्ते॒ प्रति॑ । \newline
42. प्रति॑ य॒ज्ञेन॑ य॒ज्ञेन॒ प्रति॒ प्रति॑ य॒ज्ञेन॑ । \newline
43. य॒ज्ञेन॑ तिष्ठति तिष्ठति य॒ज्ञेन॑ य॒ज्ञेन॑ तिष्ठति । \newline
44. ति॒ष्ठ॒तीति॑ तिष्ठति । \newline

\textbf{Ghana Paata } \newline

1. अ॒हम् दे॑वय॒ज्यया॑ देवय॒ज्यया॒ ऽह म॒हम् दे॑वय॒ज्य ये᳚न्द्रिया॒वीन्द्रि॑या॒वी दे॑वय॒ज्यया॒ ऽह म॒हम् दे॑वय॒ज्यये᳚न्द्रिया॒वी । \newline
2. दे॒व॒य॒ज्य ये᳚न्द्रिया॒वीन्द्रि॑या॒वी दे॑वय॒ज्यया॑ देवय॒ज्य ये᳚न्द्रिया॒वी भू॑यासम् भूयास मिन्द्रिया॒वी दे॑वय॒ज्यया॑ देवय॒ज्य ये᳚न्द्रिया॒वी भू॑यासम् । \newline
3. दे॒व॒य॒ज्ययेति॑ देव - य॒ज्यया᳚ । \newline
4. इ॒न्द्रि॒या॒वी भू॑यासम् भूयास मिन्द्रिया॒वी न्द्रि॑या॒वी भू॑यास॒ मितीति॑ भूयास मिन्द्रिया॒वी न्द्रि॑या॒वी भू॑यास॒ मिति॑ । \newline
5. भू॒या॒स॒ मितीति॑ भूयासम् भूयास॒ मित्या॑हा॒हे ति॑ भूयासम् भूयास॒ मित्या॑ह । \newline
6. इत्या॑हा॒हे तीत्या॑हे न्द्रिया॒वी न्द्रि॑या॒व्या॑हे तीत्या॑हे न्द्रिया॒वी । \newline
7. आ॒हे॒ न्द्रि॒या॒वी न्द्रि॑या॒व्या॑हाहे न्द्रिया॒व्ये॑वैवे न्द्रि॑या॒व्या॑हाहे न्द्रिया॒व्ये॑व । \newline
8. इ॒न्द्रि॒या॒व्ये॑वैवे न्द्रि॑या॒वी न्द्रि॑या॒व्ये॑व भ॑वति भवत्ये॒वे न्द्रि॑या॒वी न्द्रि॑या॒व्ये॑व भ॑वति । \newline
9. ए॒व भ॑वति भवत्ये॒वैव भ॑वति महे॒न्द्रस्य॑ महे॒न्द्रस्य॑ भवत्ये॒वैव भ॑वति महे॒न्द्रस्य॑ । \newline
10. भ॒व॒ति॒ म॒हे॒न्द्रस्य॑ महे॒न्द्रस्य॑ भवति भवति महे॒न्द्रस्या॒ह म॒हम् म॑हे॒न्द्रस्य॑ भवति भवति महे॒न्द्रस्या॒हम् । \newline
11. म॒हे॒न्द्रस्या॒ह म॒हम् म॑हे॒न्द्रस्य॑ महे॒न्द्रस्या॒हम् दे॑वय॒ज्यया॑ देवय॒ज्यया॒ ऽहम् म॑हे॒न्द्रस्य॑ महे॒न्द्रस्या॒हम् दे॑वय॒ज्यया᳚ । \newline
12. म॒हे॒न्द्रस्येति॑ महा - इ॒न्द्रस्य॑ । \newline
13. अ॒हम् दे॑वय॒ज्यया॑ देवय॒ज्यया॒ ऽह म॒हम् दे॑वय॒ज्यया॑ जे॒मान॑म् जे॒मान॑म् देवय॒ज्यया॒ ऽह म॒हम् दे॑वय॒ज्यया॑ जे॒मान᳚म् । \newline
14. दे॒व॒य॒ज्यया॑ जे॒मान॑म् जे॒मान॑म् देवय॒ज्यया॑ देवय॒ज्यया॑ जे॒मान॑म् महि॒मान॑म् महि॒मान॑म् जे॒मान॑म् देवय॒ज्यया॑ देवय॒ज्यया॑ जे॒मान॑म् महि॒मान᳚म् । \newline
15. दे॒व॒य॒ज्ययेति॑ देव - य॒ज्यया᳚ । \newline
16. जे॒मान॑म् महि॒मान॑म् महि॒मान॑म् जे॒मान॑म् जे॒मान॑म् महि॒मान॑म् गमेयम् गमेयम् महि॒मान॑म् जे॒मान॑म् जे॒मान॑म् महि॒मान॑म् गमेयम् । \newline
17. म॒हि॒मान॑म् गमेयम् गमेयम् महि॒मान॑म् महि॒मान॑म् गमेय॒ मितीति॑ गमेयम् महि॒मान॑म् महि॒मान॑म् गमेय॒ मिति॑ । \newline
18. ग॒मे॒य॒ मितीति॑ गमेयम् गमेय॒ मित्या॑हा॒हे ति॑ गमेयम् गमेय॒ मित्या॑ह । \newline
19. इत्या॑हा॒हे तीत्या॑ह जे॒मान॑म् जे॒मान॑ मा॒हे तीत्या॑ह जे॒मान᳚म् । \newline
20. आ॒ह॒ जे॒मान॑म् जे॒मान॑ माहाह जे॒मान॑ मे॒वैव जे॒मान॑ माहाह जे॒मान॑ मे॒व । \newline
21. जे॒मान॑ मे॒वैव जे॒मान॑म् जे॒मान॑ मे॒व म॑हि॒मान॑म् महि॒मान॑ मे॒व जे॒मान॑म् जे॒मान॑ मे॒व म॑हि॒मान᳚म् । \newline
22. ए॒व म॑हि॒मान॑म् महि॒मान॑ मे॒वैव म॑हि॒मान॑म् गच्छति गच्छति महि॒मान॑ मे॒वैव म॑हि॒मान॑म् गच्छति । \newline
23. म॒हि॒मान॑म् गच्छति गच्छति महि॒मान॑म् महि॒मान॑म् गच्छत्य॒ग्ने र॒ग्नेर् ग॑च्छति महि॒मान॑म् महि॒मान॑म् गच्छत्य॒ग्नेः । \newline
24. ग॒च्छ॒त्य॒ग्ने र॒ग्नेर् ग॑च्छति गच्छत्य॒ग्नेः स्वि॑ष्ट॒कृतः॑ स्विष्ट॒कृतो॒ ऽग्नेर् ग॑च्छति गच्छत्य॒ग्नेः स्वि॑ष्ट॒कृतः॑ । \newline
25. अ॒ग्नेः स्वि॑ष्ट॒कृतः॑ स्विष्ट॒कृतो॒ ऽग्नेर॒ग्नेः स्वि॑ष्ट॒कृतो॒ ऽह म॒हꣳ स्वि॑ष्ट॒कृतो॒ ऽग्नेर॒ग्नेः स्वि॑ष्ट॒कृतो॒ ऽहम् । \newline
26. स्वि॒ष्ट॒कृतो॒ ऽह म॒हꣳ स्वि॑ष्ट॒कृतः॑ स्विष्ट॒कृतो॒ ऽहम् दे॑वय॒ज्यया॑ देवय॒ज्यया॒ ऽहꣳ स्वि॑ष्ट॒कृतः॑ स्विष्ट॒कृतो॒ ऽहम् दे॑वय॒ज्यया᳚ । \newline
27. स्वि॒ष्ट॒कृत॒ इति॑ स्विष्ट - कृतः॑ । \newline
28. अ॒हम् दे॑वय॒ज्यया॑ देवय॒ज्यया॒ ऽह म॒हम् दे॑वय॒ज्यया ऽऽयु॑ष्मा॒ नायु॑ष्मान् देवय॒ज्यया॒ ऽह म॒हम् दे॑वय॒ज्यया ऽऽयु॑ष्मान् । \newline
29. दे॒व॒य॒ज्यया ऽऽयु॑ष्मा॒ नायु॑ष्मान् देवय॒ज्यया॑ देवय॒ज्यया ऽऽयु॑ष्मान्. य॒ज्ञेन॑ य॒ज्ञेनायु॑ष्मान् देवय॒ज्यया॑ देवय॒ज्यया ऽऽयु॑ष्मान्. य॒ज्ञेन॑ । \newline
30. दे॒व॒य॒ज्ययेति॑ देव - य॒ज्यया᳚ । \newline
31. आयु॑ष्मान्. य॒ज्ञेन॑ य॒ज्ञेनायु॑ष्मा॒ नायु॑ष्मान्. य॒ज्ञेन॑ प्रति॒ष्ठाम् प्र॑ति॒ष्ठां ॅय॒ज्ञेनायु॑ष्मा॒ नायु॑ष्मान्. य॒ज्ञेन॑ प्रति॒ष्ठाम् । \newline
32. य॒ज्ञेन॑ प्रति॒ष्ठाम् प्र॑ति॒ष्ठां ॅय॒ज्ञेन॑ य॒ज्ञेन॑ प्रति॒ष्ठाम् ग॑मेयम् गमेयम् प्रति॒ष्ठां ॅय॒ज्ञेन॑ य॒ज्ञेन॑ प्रति॒ष्ठाम् ग॑मेयम् । \newline
33. प्र॒ति॒ष्ठाम् ग॑मेयम् गमेयम् प्रति॒ष्ठाम् प्र॑ति॒ष्ठाम् ग॑मेय॒ मितीति॑ गमेयम् प्रति॒ष्ठाम् प्र॑ति॒ष्ठाम् ग॑मेय॒ मिति॑ । \newline
34. प्र॒ति॒ष्ठामिति॑ प्रति - स्थाम् । \newline
35. ग॒मे॒य॒ मितीति॑ गमेयम् गमेय॒ मित्या॑हा॒हे ति॑ गमेयम् गमेय॒ मित्या॑ह । \newline
36. इत्या॑हा॒हे तीत्या॒हायु॒ रायु॑रा॒हे तीत्या॒हायुः॑ । \newline
37. आ॒हायु॒ रायु॑ राहा॒हायु॑ रे॒वैवायु॑ राहा॒हायु॑ रे॒व । \newline
38. आयु॑ रे॒वैवायु॒ रायु॑रे॒वात्मन् ना॒त्मन् ने॒वायु॒ रायु॑ रे॒वात्मन्न् । \newline
39. ए॒वात्मन् ना॒त्मन् ने॒वैवात्मन् ध॑त्ते धत्त आ॒त्मन् ने॒वैवात्मन् ध॑त्ते । \newline
40. आ॒त्मन् ध॑त्ते धत्त आ॒त्मन् ना॒त्मन् ध॑त्ते॒ प्रति॒ प्रति॑ धत्त आ॒त्मन् ना॒त्मन् ध॑त्ते॒ प्रति॑ । \newline
41. ध॒त्ते॒ प्रति॒ प्रति॑ धत्ते धत्ते॒ प्रति॑ य॒ज्ञेन॑ य॒ज्ञेन॒ प्रति॑ धत्ते धत्ते॒ प्रति॑ य॒ज्ञेन॑ । \newline
42. प्रति॑ य॒ज्ञेन॑ य॒ज्ञेन॒ प्रति॒ प्रति॑ य॒ज्ञेन॑ तिष्ठति तिष्ठति य॒ज्ञेन॒ प्रति॒ प्रति॑ य॒ज्ञेन॑ तिष्ठति । \newline
43. य॒ज्ञेन॑ तिष्ठति तिष्ठति य॒ज्ञेन॑ य॒ज्ञेन॑ तिष्ठति । \newline
44. ति॒ष्ठ॒तीति॑ तिष्ठति । \newline
\pagebreak
\markright{ TS 1.6.12.1  \hfill https://www.vedavms.in \hfill}

\section{ TS 1.6.12.1 }

\textbf{TS 1.6.12.1 } \newline
\textbf{Samhita Paata} \newline

इन्द्रं॑ ॅवो वि॒श्वत॒स्परि॒ हवा॑महे॒ जने᳚भ्यः । अ॒स्माक॑मस्तु॒ केव॑लः ॥ इन्द्रं॒ नरो॑ ने॒मधि॑ता हवन्ते॒ यत्पार्या॑ यु॒नज॑ते॒ धिय॒स्ताः । शूरो॒ नृषा॑ता॒ शव॑सश्चका॒न आ गोम॑ति व्र॒जे भ॑जा॒ त्वन्नः॑ ॥ इ॒न्द्रि॒याणि॑ शतक्रतो॒ या ते॒ जने॑षु प॒ञ्चसु॑ ॥ इन्द्र॒ तानि॑ त॒ आ वृ॑णे ॥ अनु॑ ते दायि म॒ह इ॑न्द्रि॒याय॑ स॒त्रा ते॒ विश्व॒मनु॑ वृत्र॒हत्ये᳚ । अनु॑ - [ ] \newline

\textbf{Pada Paata} \newline

इन्द्र᳚म् । वः॒ । वि॒श्वतः॑ । परीति॑ । हवा॑महे । जने᳚भ्यः ॥ अ॒स्माक᳚म् । अ॒स्तु॒ । केव॑लः ॥ इन्द्र᳚म् । नरः॑ । ने॒मधि॒तेति॑ ने॒म - धि॒ता॒ । ह॒व॒न्ते॒ । यत् । पार्याः᳚ । यु॒नज॑ते । धियः॑ । ताः ॥ शूरः॑ । नृषा॒तेति॒ नृ - सा॒ता॒ । शव॑सः । च॒का॒नः । एति॑ । गोम॒तीति॒ गो - म॒ति॒ । व्र॒जे । भ॒ज॒ । त्वम् । नः॒ ॥ इ॒न्द्रि॒याणि॑ । श॒त॒क्र॒तो॒ इति॑ शत - क्र॒तो॒ । या । ते॒ । जने॑षु । प॒ञ्चस्विति॑॑ प॒ञ्च - सु॒ ॥ इन्द्र॑ । तानि॑ । ते॒ । एति॑ । वृ॒णे॒ ॥ अन्विति॑ । ते॒ । दा॒यि॒ । म॒हे । इ॒न्द्रि॒याय॑ । स॒त्रा । ते॒ । विश्व᳚म् । अन्विति॑ । वृ॒त्र॒हत्य॒ इति॑ वृत्र - हत्ये᳚ ॥ अन्विति॑ ।  \newline


\textbf{Krama Paata} \newline

इन्द्रं॑ ॅवः । वो॒ वि॒श्वतः॑ । वि॒श्वत॒स्परि॑ । परि॒ हवा॑महे । हवा॑महे॒ जने᳚भ्यः । जने᳚भ्य॒ इति॒ जने᳚भ्यः ॥ अ॒स्माक॑मस्तु । अ॒स्तु॒ केव॑लः । केव॑ल॒ इति॒ केव॑लः ॥ इन्द्र॒म् नरः॑ । नरो॑ ने॒मधि॑ता । ने॒मधि॑ता हवन्ते । ने॒मधि॒तेति॑ ने॒म - धि॒ता॒ । ह॒व॒न्ते॒ यत् । यत्,पार्याः᳚ । पार्या॑ यु॒नज॑ते । यु॒नज॑ते॒ धियः॑ । धिय॒स्ताः । ता इति॒ ताः ॥ शूरो॒ नृषा॑ता । नृषा॑ता॒ शव॑सः । नृषा॒तेति॒ नृ - सा॒ता॒ । शव॑सश्चका॒नः । च॒का॒न आ । आ गोम॑ति । गोम॑ति व्र॒जे । गोम॒तीति॒ गो - म॒ति॒ । व्र॒जे भ॑ज । भ॒जा॒ त्वम् । त्वम् नः॑ । न॒ इति॑ नः ॥ इ॒न्द्रि॒याणि॑ शतक्रतो । श॒त॒क्र॒तो॒ या । श॒त॒क्र॒तो॒ इति॑ शत - क्र॒तो॒ । या ते᳚ । ते॒ जने॑षु । जने॑षु प॒ञ्चसु॑ । प॒ञ्चस्विति॑ प॒ञ्च - सु॒ ॥ इन्द्र॒ तानि॑ । तानि॑ ते । त॒ आ । आ वृ॑णे । वृ॒ण॒ इति॑ वृणे ॥ अनु॑ ते । ते॒ दा॒यि॒ । दा॒यि॒ म॒हे । म॒ह इ॑न्द्रि॒याय॑ । इ॒न्द्रि॒याय॑ स॒त्रा । स॒त्रा ते᳚ । ते॒ विश्व᳚म् । विश्व॒मनु॑ । अनु॑ वृत्र॒हत्ये᳚ । वृ॒त्र॒हत्य॒ इति॑ वृत्र - हत्ये᳚ ॥ अनु॑ क्ष॒त्रम् \newline

\textbf{Jatai Paata} \newline

1. इन्द्रं॑ ॅवो व॒ इन्द्र॒ मिन्द्रं॑ ॅवः । \newline
2. वो॒ वि॒श्वतो॑ वि॒श्वतो॑ वो वो वि॒श्वतः॑ । \newline
3. वि॒श्वत॒ स्परि॒ परि॑ वि॒श्वतो॑ वि॒श्वत॒ स्परि॑ । \newline
4. परि॒ हवा॑महे॒ हवा॑महे॒ परि॒ परि॒ हवा॑महे । \newline
5. हवा॑महे॒ जने᳚भ्यो॒ जने᳚भ्यो॒ हवा॑महे॒ हवा॑महे॒ जने᳚भ्यः । \newline
6. जने᳚भ्य॒ इति॒ जने᳚भ्यः । \newline
7. अ॒स्माक॑ मस्त्व स्त्व॒स्माक॑ म॒स्माक॑ मस्तु । \newline
8. अ॒स्तु॒ केव॑लः॒ केव॑लो अस्त्वस्तु॒ केव॑लः । \newline
9. केव॑ल॒ इति॒ केव॑लः । \newline
10. इन्द्र॒म् नरो॒ नर॒ इन्द्र॒ मिन्द्र॒म् नरः॑ । \newline
11. नरो॑ ने॒मधि॑ता ने॒मधि॑ता॒ नरो॒ नरो॑ ने॒मधि॑ता । \newline
12. ने॒मधि॑ता हवन्ते हवन्ते ने॒मधि॑ता ने॒मधि॑ता हवन्ते । \newline
13. ने॒मधि॒तेति॑ ने॒म - धि॒ता॒ । \newline
14. ह॒व॒न्ते॒ यद् यद्ध॑वन्ते हवन्ते॒ यत् । \newline
15. यत् पार्याः॒ पार्या॒ यद् यत् पार्याः᳚ । \newline
16. पार्या॑ यु॒नज॑ते यु॒नज॑ते॒ पार्याः॒ पार्या॑ यु॒नज॑ते । \newline
17. यु॒नज॑ते॒ धियो॒ धियो॑ यु॒नज॑ते यु॒नज॑ते॒ धियः॑ । \newline
18. धिय॒ स्ता स्ता धियो॒ धिय॒ स्ताः । \newline
19. ता इति॒ ताः । \newline
20. शूरो॒ नृषा॑ता॒ नृषा॑ता॒ शूरः॒ शूरो॒ नृषा॑ता । \newline
21. नृषा॑ता॒ शव॑सः॒ शव॑सो॒ नृषा॑ता॒ नृषा॑ता॒ शव॑सः । \newline
22. नृषा॒तेति॒ नृ - सा॒ता॒ । \newline
23. शव॑स श्चका॒न श्च॑का॒नः शव॑सः॒ शव॑स श्चका॒नः । \newline
24. च॒का॒न आ च॑का॒न श्च॑का॒न आ । \newline
25. आ गोम॑ति॒ गोम॒त्या गोम॑ति । \newline
26. गोम॑ति व्र॒जे व्र॒जे गोम॑ति॒ गोम॑ति व्र॒जे । \newline
27. गोम॒तीति॒ गो - म॒ति॒ । \newline
28. व्र॒जे भ॑ज भज व्र॒जे व्र॒जे भ॑ज । \newline
29. भ॒जा॒ त्वम् त्वम् भ॑ज भजा॒ त्वम् । \newline
30. त्वम् नो॑ न॒स्त्वम् त्वम् नः॑ । \newline
31. न॒ इति॑ नः । \newline
32. इ॒न्द्रि॒याणि॑ शतक्रतो शतक्रतो इन्द्रि॒याणी᳚ न्द्रि॒याणि॑ शतक्रतो । \newline
33. श॒त॒क्र॒तो॒ या या श॑तक्रतो शतक्रतो॒ या । \newline
34. श॒त॒क्र॒तो॒ इति॑ शत - क्र॒तो॒ । \newline
35. या ते॑ ते॒ या या ते᳚ । \newline
36. ते॒ जने॑षु॒ जने॑षु ते ते॒ जने॑षु । \newline
37. जने॑षु प॒ञ्चसु॑ प॒ञ्चसु॒ जने॑षु॒ जने॑षु प॒ञ्चसु॑ । \newline
38. प॒ञ्चस्विति॑ प॒ञ्च - सु॒ । \newline
39. इन्द्र॒ तानि॒ तानीन्द्रे न्द्र॒ तानि॑ । \newline
40. तानि॑ ते ते॒ तानि॒ तानि॑ ते । \newline
41. त॒ आ ते॑ त॒ आ । \newline
42. आ वृ॑णे वृण॒ आ वृ॑णे । \newline
43. वृ॒ण॒ इति॑ वृणे । \newline
44. अनु॑ ते ते॒ अन्वनु॑ ते । \newline
45. ते॒ दा॒यि॒ दा॒यि॒ ते॒ ते॒ दा॒यि॒ । \newline
46. दा॒यि॒ म॒हे म॒हे दा॑यि दायि म॒हे । \newline
47. म॒ह इ॑न्द्रि॒याये᳚ न्द्रि॒याय॑ म॒हे म॒ह इ॑न्द्रि॒याय॑ । \newline
48. इ॒न्द्रि॒याय॑ स॒त्रा स॒त्रेन्द्रि॒याये᳚ न्द्रि॒याय॑ स॒त्रा । \newline
49. स॒त्रा ते॑ ते स॒त्रा स॒त्रा ते᳚ । \newline
50. ते॒ विश्वं॒ ॅविश्व॑म् ते ते॒ विश्व᳚म् । \newline
51. विश्व॒ मन्वनु॒ विश्वं॒ ॅविश्व॒ मनु॑ । \newline
52. अनु॑ वृत्र॒हत्ये॑ वृत्र॒हत्ये॒ अन्वनु॑ वृत्र॒हत्ये᳚ । \newline
53. वृ॒त्र॒हत्य॒ इति॑ वृत्र - हत्ये᳚ । \newline
54. अनु॑ क्ष॒त्रम् क्ष॒त्र मन्वनु॑ क्ष॒त्रम् । \newline

\textbf{Ghana Paata } \newline

1. इन्द्रं॑ ॅवो व॒ इन्द्र॒ मिन्द्रं॑ ॅवो वि॒श्वतो॑ वि॒श्वतो॑ व॒ इन्द्र॒ मिन्द्रं॑ ॅवो वि॒श्वतः॑ । \newline
2. वो॒ वि॒श्वतो॑ वि॒श्वतो॑ वो वो वि॒श्वत॒ स्परि॒ परि॑ वि॒श्वतो॑ वो वो वि॒श्वत॒ स्परि॑ । \newline
3. वि॒श्वत॒ स्परि॒ परि॑ वि॒श्वतो॑ वि॒श्वत॒ स्परि॒ हवा॑महे॒ हवा॑महे॒ परि॑ वि॒श्वतो॑ वि॒श्वत॒ स्परि॒ हवा॑महे । \newline
4. परि॒ हवा॑महे॒ हवा॑महे॒ परि॒ परि॒ हवा॑महे॒ जने᳚भ्यो॒ जने᳚भ्यो॒ हवा॑महे॒ परि॒ परि॒ हवा॑महे॒ जने᳚भ्यः । \newline
5. हवा॑महे॒ जने᳚भ्यो॒ जने᳚भ्यो॒ हवा॑महे॒ हवा॑महे॒ जने᳚भ्यः । \newline
6. जने᳚भ्य॒ इति॒ जने᳚भ्यः । \newline
7. अ॒स्माक॑ मस्त्वस्त्व॒स्माक॑ म॒स्माक॑ मस्तु॒ केव॑लः॒ केव॑लो अस्त्व॒स्माक॑ म॒स्माक॑ मस्तु॒ केव॑लः । \newline
8. अ॒स्तु॒ केव॑लः॒ केव॑लो अस्त्वस्तु॒ केव॑लः । \newline
9. केव॑ल॒ इति॒ केव॑लः । \newline
10. इन्द्र॒म् नरो॒ नर॒ इन्द्र॒ मिन्द्र॒म् नरो॑ ने॒मधि॑ता ने॒मधि॑ता॒ नर॒ इन्द्र॒ मिन्द्र॒म् नरो॑ ने॒मधि॑ता । \newline
11. नरो॑ ने॒मधि॑ता ने॒मधि॑ता॒ नरो॒ नरो॑ ने॒मधि॑ता हवन्ते हवन्ते ने॒मधि॑ता॒ नरो॒ नरो॑ ने॒मधि॑ता हवन्ते । \newline
12. ने॒मधि॑ता हवन्ते हवन्ते ने॒मधि॑ता ने॒मधि॑ता हवन्ते॒ यद् यद्ध॑वन्ते ने॒मधि॑ता ने॒मधि॑ता हवन्ते॒ यत् । \newline
13. ने॒मधि॒तेति॑ ने॒म - धि॒ता॒ । \newline
14. ह॒व॒न्ते॒ यद् यद्ध॑वन्ते हवन्ते॒ यत् पार्याः॒ पार्या॒ यद्ध॑वन्ते हवन्ते॒ यत् पार्याः᳚ । \newline
15. यत् पार्याः॒ पार्या॒ यद् यत् पार्या॑ यु॒नज॑ते यु॒नज॑ते॒ पार्या॒ यद् यत् पार्या॑ यु॒नज॑ते । \newline
16. पार्या॑ यु॒नज॑ते यु॒नज॑ते॒ पार्याः॒ पार्या॑ यु॒नज॑ते॒ धियो॒ धियो॑ यु॒नज॑ते॒ पार्याः॒ पार्या॑ यु॒नज॑ते॒ धियः॑ । \newline
17. यु॒नज॑ते॒ धियो॒ धियो॑ यु॒नज॑ते यु॒नज॑ते॒ धिय॒स्ता स्ता धियो॑ यु॒नज॑ते यु॒नज॑ते॒ धिय॒स्ताः । \newline
18. धिय॒स्ता स्ता धियो॒ धिय॒स्ताः । \newline
19. ता इति॒ ताः । \newline
20. शूरो॒ नृषा॑ता॒ नृषा॑ता॒ शूरः॒ शूरो॒ नृषा॑ता॒ शव॑सः॒ शव॑सो॒ नृषा॑ता॒ शूरः॒ शूरो॒ नृषा॑ता॒ शव॑सः । \newline
21. नृषा॑ता॒ शव॑सः॒ शव॑सो॒ नृषा॑ता॒ नृषा॑ता॒ शव॑स श्चका॒न श्च॑का॒नः शव॑सो॒ नृषा॑ता॒ नृषा॑ता॒ शव॑स श्चका॒नः । \newline
22. नृषा॒तेति॒ नृ - सा॒ता॒ । \newline
23. शव॑ सश्चका॒न श्च॑का॒नः शव॑सः॒ शव॑स श्चका॒न आ च॑का॒नः शव॑सः॒ शव॑स श्चका॒न आ । \newline
24. च॒का॒न आ च॑का॒न श्च॑का॒न आ गोम॑ति॒ गोम॒त्या च॑का॒न श्च॑का॒न आ गोम॑ति । \newline
25. आ गोम॑ति॒ गोम॒त्या गोम॑ति व्र॒जे व्र॒जे गोम॒त्या गोम॑ति व्र॒जे । \newline
26. गोम॑ति व्र॒जे व्र॒जे गोम॑ति॒ गोम॑ति व्र॒जे भ॑ज भज व्र॒जे गोम॑ति॒ गोम॑ति व्र॒जे भ॑ज । \newline
27. गोम॒तीति॒ गो - म॒ति॒ । \newline
28. व्र॒जे भ॑ज भज व्र॒जे व्र॒जे भ॑जा॒ त्वम् त्वम् भ॑ज व्र॒जे व्र॒जे भ॑जा॒ त्वम् । \newline
29. भ॒जा॒ त्वम् त्वम् भ॑ज भजा॒ त्वम् नो॑ न॒स्त्वम् भ॑ज भजा॒ त्वम् नः॑ । \newline
30. त्वम् नो॑ न॒स्त्वम् त्वम् नः॑ । \newline
31. न॒ इति॑ नः । \newline
32. इ॒न्द्रि॒याणि॑ शतक्रतो शतक्रतो इन्द्रि॒याणी᳚न्द्रि॒याणि॑ शतक्रतो॒ या या श॑तक्रतो इन्द्रि॒याणी᳚न्द्रि॒याणि॑ शतक्रतो॒ या । \newline
33. श॒त॒क्र॒तो॒ या या श॑तक्रतो शतक्रतो॒ या ते॑ ते॒ या श॑तक्रतो शतक्रतो॒ या ते᳚ । \newline
34. श॒त॒क्र॒तो॒ इति॑ शत - क्र॒तो॒ । \newline
35. या ते॑ ते॒ या या ते॒ जने॑षु॒ जने॑षु ते॒ या या ते॒ जने॑षु । \newline
36. ते॒ जने॑षु॒ जने॑षु ते ते॒ जने॑षु प॒ञ्चसु॑ प॒ञ्चसु॒ जने॑षु ते ते॒ जने॑षु प॒ञ्चसु॑ । \newline
37. जने॑षु प॒ञ्चसु॑ प॒ञ्चसु॒ जने॑षु॒ जने॑षु प॒ञ्चसु॑ । \newline
38. प॒ञ्चस्विति॑ प॒ञ्च - सु॒ । \newline
39. इन्द्र॒ तानि॒ तानीन्द्रे न्द्र॒ तानि॑ ते ते॒ तानीन्द्रे न्द्र॒ तानि॑ ते । \newline
40. तानि॑ ते ते॒ तानि॒ तानि॑ त॒ आ ते॒ तानि॒ तानि॑ त॒ आ । \newline
41. त॒ आ ते॑ त॒ आ वृ॑णे वृण॒ आ ते॑ त॒ आ वृ॑णे । \newline
42. आ वृ॑णे वृण॒ आ वृ॑णे । \newline
43. वृ॒ण॒ इति॑ वृणे । \newline
44. अनु॑ ते ते॒ अन्वनु॑ ते दायि दायि ते॒ अन्वनु॑ ते दायि । \newline
45. ते॒ दा॒यि॒ दा॒यि॒ ते॒ ते॒ दा॒यि॒ म॒हे म॒हे दा॑यि ते ते दायि म॒हे । \newline
46. दा॒यि॒ म॒हे म॒हे दा॑यि दायि म॒ह इ॑न्द्रि॒याये᳚ न्द्रि॒याय॑ म॒हे दा॑यि दायि म॒ह इ॑न्द्रि॒याय॑ । \newline
47. म॒ह इ॑न्द्रि॒याये᳚ न्द्रि॒याय॑ म॒हे म॒ह इ॑न्द्रि॒याय॑ स॒त्रा स॒त्रेन्द्रि॒याय॑ म॒हे म॒ह इ॑न्द्रि॒याय॑ स॒त्रा । \newline
48. इ॒न्द्रि॒याय॑ स॒त्रा स॒त्रेन्द्रि॒याये᳚ न्द्रि॒याय॑ स॒त्रा ते॑ ते स॒त्रेन्द्रि॒याये᳚ न्द्रि॒याय॑ स॒त्रा ते᳚ । \newline
49. स॒त्रा ते॑ ते स॒त्रा स॒त्रा ते॒ विश्वं॒ ॅविश्व॑म् ते स॒त्रा स॒त्रा ते॒ विश्व᳚म् । \newline
50. ते॒ विश्वं॒ ॅविश्व॑म् ते ते॒ विश्व॒ मन्वनु॒ विश्व॑म् ते ते॒ विश्व॒ मनु॑ । \newline
51. विश्व॒ मन्वनु॒ विश्वं॒ ॅविश्व॒ मनु॑ वृत्र॒हत्ये॑ वृत्र॒हत्येऽनु॒ विश्वं॒ ॅविश्व॒ मनु॑ वृत्र॒हत्ये᳚ । \newline
52. अनु॑ वृत्र॒हत्ये॑ वृत्र॒हत्ये॒ अन्वनु॑ वृत्र॒हत्ये᳚ । \newline
53. वृ॒त्र॒हत्य॒ इति॑ वृत्र - हत्ये᳚ । \newline
54. अनु॑ क्ष॒त्रम् क्ष॒त्र मन्वनु॑ क्ष॒त्र मन्वनु॑ क्ष॒त्र मन्वनु॑ क्ष॒त्र मनु॑ । \newline
\pagebreak
\markright{ TS 1.6.12.2  \hfill https://www.vedavms.in \hfill}

\section{ TS 1.6.12.2 }

\textbf{TS 1.6.12.2 } \newline
\textbf{Samhita Paata} \newline

क्ष॒त्रमनु॒ सहो॑ यज॒त्रेन्द्र॑ दे॒वेभि॒रनु॑ ते नृ॒षह्ये᳚ ॥ आयस्मि᳚न्थ् स॒प्तवा॑स॒वा स्तिष्ठ॑न्ति स्वा॒रुहो॑ यथा । ऋषि॑र्.ह दीर्घ॒श्रुत्त॑म॒ इन्द्र॑स्य घ॒र्मो अति॑थिः ॥ आ॒मासु॑ प॒क्वमैर॑य॒ आ सूर्यꣳ॑ रोहयो दि॒वि । घ॒र्मं न सामं॑ तपता सुवृ॒क्तिभि॒र् जुष्टं॒ गिर्व॑णसे॒ गिरः॑ ॥ इन्द्र॒मिद् गा॒थिनो॑ बृ॒हदिन्द्र॑-म॒र्केभि॑-र॒र्किणः॑ । इन्द्रं॒ ॅवाणी॑रनूषत ॥ गाय॑न्ति त्वा गाय॒त्रिणो- [ ] \newline

\textbf{Pada Paata} \newline

क्ष॒त्रम् । अन्विति॑ । सहः॑ । य॒ज॒त्र॒ । इन्द्र॑ । दे॒वेभिः॑ । अन्विति॑ । ते॒ । नृ॒षह्य॒ इति॑ नृ-सह्ये᳚ ॥ एति॑ । यस्मिन्न्॑ । स॒प्त । वा॒स॒वाः । तिष्ठ॑न्ति । स्वा॒रुह॒ इति॑ स्व - रुहः॑ । य॒था॒ ॥ ऋषिः॑ । ह॒ । दी॒र्घ॒श्रुत्त॑म॒ इति॑ दीर्घ॒श्रुत् - त॒मः॒ । इन्द्र॑स्य । घ॒र्मः । अति॑थिः ॥ आ॒मासु॑ । प॒क्वम् । ऐर॑यः । एति॑ । सूर्य᳚म् । रो॒ह॒यः॒ । दि॒वि ॥ घ॒र्मम् । न । सामन्न्॑ । त॒प॒त॒ । सु॒वृ॒क्तिभि॒रिति॑ सुवृ॒क्ति - भिः॒ । जुष्ट᳚म् । गिर्व॑णसे । गिरः॑ ॥ इन्द्र᳚म् । इत् । गा॒थिनः॑ । बृ॒हत् । इन्द्र᳚म् । अ॒र्केभिः॑ । अ॒र्किणः॑ ॥ इन्द्र᳚म् । वाणीः᳚ । अ॒नू॒ष॒त॒ ॥ गाय॑न्ति । त्वा॒ । गा॒य॒त्रिणः॑ ।  \newline


\textbf{Krama Paata} \newline

क्ष॒त्रमनु॑ । अनु॒ सहः॑ । सहो॑ यजत्र । य॒ज॒त्रेन्द्र॑ । इन्द्र॑ दे॒वेभिः॑ । दे॒वेभि॒रनु॑ । अनु॑ ते । ते॒ नृ॒षह्ये᳚ । नृ॒षह्य॒ इति॑ नृ - सह्ये᳚ ॥ आ यस्मिन्न्॑ । यस्मि᳚न्थ् स॒प्त । स॒प्त वा॑स॒वाः । वा॒स॒वास्तिष्ठ॑न्ति । तिष्ठ॑न्ति स्वा॒रुहः॑ । स्वा॒रुहो॑ यथा । स्वा॒रुह॒ इति॑ स्व - रुहः॑ । य॒थेति॑ यथा ॥ ऋषि॑र्. ह । ह॒ दी॒र्घ॒श्रुत्त॑मः । दी॒र्घ॒श्रुत्त॑म॒ इन्द्र॑स्य । दी॒र्घ॒श्रुत्त॑म॒ इति॑ दीर्घ॒श्रुत् - त॒मः॒ । इन्द्र॑स्य घ॒र्मः । घ॒र्मो अति॑थिः । अति॑थि॒रित्यति॑थिः ॥ आ॒मासु॑ प॒क्वम् । प॒क्वमैर॑यः । ऐर॑य॒ आ । आ सूर्य᳚म् । सूर्यꣳ॑ रोहयः । रो॒ह॒यो॒ दि॒वि । दि॒वीति॑ दि॒वि ॥ घ॒र्मम् न । न सामन्न्॑ । साम॑न् तपत । त॒प॒ता॒ सु॒वृ॒क्तिभिः॑ । सु॒वृ॒क्तिभि॒र् जुष्ट᳚म् । सु॒वृ॒क्तिभि॒रिति॑ सुवृ॒क्ति - भिः॒ । जुष्ट॒म् गिर्व॑णसे । गिर्व॑णसे॒ गिरः॑ । गिर॒ इति॒ गिरः॑ ॥ इन्द्र॒मित् । इद् गा॒थिनः॑ । गा॒थिनो॑ बृ॒हत् । बृ॒हदिन्द्र᳚म् । इन्द्र॑म॒र्केभिः॑ । अ॒र्केभि॑र॒र्किणः॑ । अ॒र्किण॒ इत्य॒र्किणः॑ ॥ इन्द्रं॒ ॅवाणीः᳚ । वाणी॑रनूषत । अ॒नू॒ष॒तेत्य॑नूषत ॥ गाय॑न्ति त्वा । त्वा॒ गा॒य॒त्रिणः॑ । गा॒य॒त्रिणो ऽर्च॑न्ति \newline

\textbf{Jatai Paata} \newline

1. क्ष॒त्र मन्वनु॑ क्ष॒त्रम् क्ष॒त्र मनु॑ । \newline
2. अनु॒ सहः॒ सहो॒ अन्वनु॒ सहः॑ । \newline
3. सहो॑ यजत्र यजत्र॒ सहः॒ सहो॑ यजत्र । \newline
4. य॒ज॒त्रे न्द्रे न्द्र॑ यजत्र यज॒त्रे न्द्र॑ । \newline
5. इन्द्र॑ दे॒वेभि॑र् दे॒वेभि॒ रिन्द्रे न्द्र॑ दे॒वेभिः॑ । \newline
6. दे॒वेभि॒ रन्वनु॑ दे॒वेभि॑र् दे॒वेभि॒ रनु॑ । \newline
7. अनु॑ ते ते॒ अन्वनु॑ ते । \newline
8. ते॒ नृ॒षह्ये॑ नृ॒षह्ये॑ ते ते नृ॒षह्ये᳚ । \newline
9. नृ॒षह्य॒ इति॑ नृ - सह्ये᳚ । \newline
10. आ यस्मि॒न्॒. यस्मि॒न् ना यस्मिन्न्॑ । \newline
11. यस्मि᳚न् थ्स॒प्त स॒प्त यस्मि॒न्॒. यस्मि᳚न् थ्स॒प्त । \newline
12. स॒प्त वा॑स॒वा वा॑स॒वाः स॒प्त स॒प्त वा॑स॒वाः । \newline
13. वा॒स॒वा स्तिष्ठ॑न्ति॒ तिष्ठ॑न्ति वास॒वा वा॑स॒वा स्तिष्ठ॑न्ति । \newline
14. तिष्ठ॑न्ति स्वा॒रुहः॑ स्वा॒रुह॒ स्तिष्ठ॑न्ति॒ तिष्ठ॑न्ति स्वा॒रुहः॑ । \newline
15. स्वा॒रुहो॑ यथा यथा स्वा॒रुहः॑ स्वा॒रुहो॑ यथा । \newline
16. स्वा॒रुह॒ इति॑ स्व - रुहः॑ । \newline
17. य॒थेति॑ यथा । \newline
18. ऋषि॑र्. ह॒ हर्.षि॒र्॒.ऋषि॑र्. ह । \newline
19. ह॒ दी॒र्घ॒श्रुत्त॑मो दीर्घ॒श्रुत्त॑मो ह ह दीर्घ॒श्रुत्त॑मः । \newline
20. दी॒र्घ॒श्रुत्त॑म॒ इन्द्र॒स्ये न्द्र॑स्य दीर्घ॒श्रुत्त॑मो दीर्घ॒श्रुत्त॑म॒ इन्द्र॑स्य । \newline
21. दी॒र्घ॒श्रुत्त॑म॒ इति॑ दीर्घ॒श्रुत् - त॒मः॒ । \newline
22. इन्द्र॑स्य घ॒र्मो घ॒र्म इन्द्र॒स्ये न्द्र॑स्य घ॒र्मः । \newline
23. घ॒र्मो अति॑थि॒ रति॑थिर् घ॒र्मो घ॒र्मो अति॑थिः । \newline
24. अति॒थिरित्यति॑थिः । \newline
25. आ॒मासु॑ प॒क्वम् प॒क्व मा॒मा स्वा॒मासु॑ प॒क्वम् । \newline
26. प॒क्व मैर॑य॒ ऐर॑यः प॒क्वम् प॒क्व मैर॑यः । \newline
27. ऐर॑य॒ ऐर॑य॒ ऐर॑य॒ आ । \newline
28. आ सूर्य॒(ग्म्॒) सूर्य॒ मा सूर्य᳚म् । \newline
29. सूर्य(ग्म्॑) रोहयो रोहयः॒ सूर्य॒(ग्म्॒) सूर्य(ग्म्॑) रोहयः । \newline
30. रो॒ह॒यो॒ दि॒वि दि॒वि रो॑हयो रोहयो दि॒वि । \newline
31. दि॒वीति॑ दि॒वि । \newline
32. घ॒र्मम् न न घ॒र्मम् घ॒र्मम् न । \newline
33. न साम॒न् थ्साम॒न् न न सामन्न्॑ । \newline
34. साम॑न् तपत तपत॒ साम॒न् थ्साम॑न् तपत । \newline
35. त॒प॒ता॒ सु॒वृ॒क्तिभिः॑ सुवृ॒क्तिभि॑ स्तपत तपता सुवृ॒क्तिभिः॑ । \newline
36. सु॒वृ॒क्तिभि॒र् जुष्ट॒म् जुष्ट(ग्म्॑) सुवृ॒क्तिभिः॑ सुवृ॒क्तिभि॒र् जुष्ट᳚म् । \newline
37. सु॒वृ॒क्तिभि॒रिति॑ सुवृ॒क्ति - भिः॒ । \newline
38. जुष्ट॒म् गिर्व॑णसे॒ गिर्व॑णसे॒ जुष्ट॒म् जुष्ट॒म् गिर्व॑णसे । \newline
39. गिर्व॑णसे॒ गिरो॒ गिरो॒ गिर्व॑णसे॒ गिर्व॑णसे॒ गिरः॑ । \newline
40. गिर॒ इति॒ गिरः॑ । \newline
41. इन्द्र॒ मिदिदिन्द्र॒ मिन्द्र॒ मित् । \newline
42. इद् गा॒थिनो॑ गा॒थिन॒ इदिद् गा॒थिनः॑ । \newline
43. गा॒थिनो॑ बृ॒हद् बृ॒हद् गा॒थिनो॑ गा॒थिनो॑ बृ॒हत् । \newline
44. बृ॒हदिन्द्र॒ मिन्द्र॑म् बृ॒हद् बृ॒हदिन्द्र᳚म् । \newline
45. इन्द्र॑ म॒र्केभि॑ र॒र्केभि॒ रिन्द्र॒ मिन्द्र॑ म॒र्केभिः॑ । \newline
46. अ॒र्केभि॑ र॒र्किणो॑ अ॒र्किणो॑ अ॒र्केभि॑ र॒र्केभि॑ र॒र्किणः॑ । \newline
47. अ॒र्किण॒ इत्य॒र्किणः॑ । \newline
48. इन्द्रं॒ ॅवाणी॒र् वाणी॒ रिन्द्र॒ मिन्द्रं॒ ॅवाणीः᳚ । \newline
49. वाणी॑ रनूष तानूषत॒ वाणी॒र् वाणी॑ रनूषत । \newline
50. अ॒नू॒ष॒तेत्ये॑नूषत । \newline
51. गाय॑न्ति त्वा त्वा॒ गाय॑न्ति॒ गाय॑न्ति त्वा । \newline
52. त्वा॒ गा॒य॒त्रिणो॑ गाय॒त्रिण॑ स्त्वा त्वा गाय॒त्रिणः॑ । \newline
53. गा॒य॒त्रिणो ऽर्च॒न्त्यर्च॑न्ति गाय॒त्रिणो॑ गाय॒त्रिणो ऽर्च॑न्ति । \newline

\textbf{Ghana Paata } \newline

1. क्ष॒त्र मन्वनु॑ क्ष॒त्रम् क्ष॒त्र मनु॒ सहः॒ सहो ऽनु॑ क्ष॒त्रम् क्ष॒त्र मनु॒ सहः॑ । \newline
2. अनु॒ सहः॒ सहो॒ अन्वनु॒ सहो॑ यजत्र यजत्र॒ सहो॒ अन्वनु॒ सहो॑ यजत्र । \newline
3. सहो॑ यजत्र यजत्र॒ सहः॒ सहो॑ यज॒त्रे न्द्रे न्द्र॑ यजत्र॒ सहः॒ सहो॑ यज॒त्रे न्द्र॑ । \newline
4. य॒ज॒त्रे न्द्रे न्द्र॑ यजत्र यज॒त्रे न्द्र॑ दे॒वेभि॑र् दे॒वेभि॒रिन्द्र॑ यजत्र यज॒त्रे न्द्र॑ दे॒वेभिः॑ । \newline
5. इन्द्र॑ दे॒वेभि॑र् दे॒वेभि॒रिन्द्रे न्द्र॑ दे॒वेभि॒ रन्वनु॑ दे॒वेभि॒रिन्द्रे न्द्र॑ दे॒वेभि॒रनु॑ । \newline
6. दे॒वेभि॒ रन्वनु॑ दे॒वेभि॑र् दे॒वेभि॒रनु॑ ते ते ऽनु॑ दे॒वेभि॑र् दे॒वेभि॒रनु॑ ते । \newline
7. अनु॑ ते ते॒ अन्वनु॑ ते नृ॒षह्ये॑ नृ॒षह्ये॑ ते॒ अन्वनु॑ ते नृ॒षह्ये᳚ । \newline
8. ते॒ नृ॒षह्ये॑ नृ॒षह्ये॑ ते ते नृ॒षह्ये᳚ । \newline
9. नृ॒षह्य॒ इति॑ नृ - सह्ये᳚ । \newline
10. आ यस्मि॒न्॒. यस्मि॒न् ना यस्मि᳚न् थ्स॒प्त स॒प्त यस्मि॒न् ना यस्मि᳚न् थ्स॒प्त । \newline
11. यस्मि᳚न् थ्स॒प्त स॒प्त यस्मि॒न्॒. यस्मि᳚न् थ्स॒प्त वा॑स॒वा वा॑स॒वाः स॒प्त यस्मि॒न्॒. यस्मि᳚न् थ्स॒प्त वा॑स॒वाः । \newline
12. स॒प्त वा॑स॒वा वा॑स॒वाः स॒प्त स॒प्त वा॑स॒वा स्तिष्ठ॑न्ति॒ तिष्ठ॑न्ति वास॒वाः स॒प्त स॒प्त वा॑स॒वा स्तिष्ठ॑न्ति । \newline
13. वा॒स॒वा स्तिष्ठ॑न्ति॒ तिष्ठ॑न्ति वास॒वा वा॑स॒वा स्तिष्ठ॑न्ति स्वा॒रुहः॑ स्वा॒रुह॒ स्तिष्ठ॑न्ति वास॒वा वा॑स॒वा स्तिष्ठ॑न्ति स्वा॒रुहः॑ । \newline
14. तिष्ठ॑न्ति स्वा॒रुहः॑ स्वा॒रुह॒ स्तिष्ठ॑न्ति॒ तिष्ठ॑न्ति स्वा॒रुहो॑ यथा यथा स्वा॒रुह॒ स्तिष्ठ॑न्ति॒ तिष्ठ॑न्ति स्वा॒रुहो॑ यथा । \newline
15. स्वा॒रुहो॑ यथा यथा स्वा॒रुहः॑ स्वा॒रुहो॑ यथा । \newline
16. स्वा॒रुह॒ इति॑ स्व - रुहः॑ । \newline
17. य॒थेति॑ यथा । \newline
18. ऋषि॑र्. ह॒ ह र्.षि॒र्॒. ऋषि॑र्. ह दीर्घ॒श्रुत्त॑मो दीर्घ॒श्रुत्त॑मो॒ ह र्.षि॒र्॒. ऋषि॑र्. ह दीर्घ॒श्रुत्त॑मः । \newline
19. ह॒ दी॒र्घ॒श्रुत्त॑मो दीर्घ॒श्रुत्त॑मो ह ह दीर्घ॒श्रुत्त॑म॒ इन्द्र॒स्ये न्द्र॑स्य दीर्घ॒श्रुत्त॑मो ह ह दीर्घ॒श्रुत्त॑म॒ इन्द्र॑स्य । \newline
20. दी॒र्घ॒श्रुत्त॑म॒ इन्द्र॒स्ये न्द्र॑स्य दीर्घ॒श्रुत्त॑मो दीर्घ॒श्रुत्त॑म॒ इन्द्र॑स्य घ॒र्मो घ॒र्म इन्द्र॑स्य दीर्घ॒श्रुत्त॑मो दीर्घ॒श्रुत्त॑म॒ इन्द्र॑स्य घ॒र्मः । \newline
21. दी॒र्घ॒श्रुत्त॑म॒ इति॑ दीर्घ॒श्रुत् - त॒मः॒ । \newline
22. इन्द्र॑स्य घ॒र्मो घ॒र्म इन्द्र॒स्ये न्द्र॑स्य घ॒र्मो अति॑थि॒ रति॑थिर् घ॒र्म इन्द्र॒स्ये न्द्र॑स्य घ॒र्मो अति॑थिः । \newline
23. घ॒र्मो अति॑थि॒ रति॑थिर् घ॒र्मो घ॒र्मो अति॑थिः । \newline
24. अति॒थिरित्यति॑थिः । \newline
25. आ॒मासु॑ प॒क्वम् प॒क्व मा॒मास्वा॒मासु॑ प॒क्व मैर॑य॒ ऐर॑यः प॒क्व मा॒मास्वा॒मासु॑ प॒क्व मैर॑यः । \newline
26. प॒क्व मैर॑य॒ ऐर॑यः प॒क्वम् प॒क्व मैर॑य॒ ऐर॑यः प॒क्वम् प॒क्व मैर॑य॒ आ । \newline
27. ऐर॑य॒ ऐर॑य॒ ऐर॑य॒ आ सूर्य॒(ग्म्॒) सूर्य॒ मैर॑य॒ ऐर॑य॒ आ सूर्य᳚म् । \newline
28. आ सूर्य॒(ग्म्॒) सूर्य॒ मा सूर्य(ग्म्॑) रोहयो रोहयः॒ सूर्य॒ मा सूर्य(ग्म्॑) रोहयः । \newline
29. सूर्य(ग्म्॑) रोहयो रोहयः॒ सूर्य॒(ग्म्॒) सूर्य(ग्म्॑) रोहयो दि॒वि दि॒वि रो॑हयः॒ सूर्य॒(ग्म्॒) सूर्य(ग्म्॑) रोहयो दि॒वि । \newline
30. रो॒ह॒यो॒ दि॒वि दि॒वि रो॑हयो रोहयो दि॒वि । \newline
31. दि॒वीति॑ दि॒वि । \newline
32. घ॒र्मम् न न घ॒र्मम् घ॒र्मम् न साम॒न् थ्साम॒न् न घ॒र्मम् घ॒र्मम् न सामन्न्॑ । \newline
33. न साम॒न् थ्साम॒न् न न साम॑न् तपत तपत॒ साम॒न् न न साम॑न् तपत । \newline
34. साम॑न् तपत तपत॒ साम॒न् थ्साम॑न् तपता सुवृ॒क्तिभिः॑ सुवृ॒क्तिभि॑ स्तपत॒ साम॒न् थ्साम॑न् तपता सुवृ॒क्तिभिः॑ । \newline
35. त॒प॒ता॒ सु॒वृ॒क्तिभिः॑ सुवृ॒क्तिभि॑ स्तपत तपता सुवृ॒क्तिभि॒र् जुष्ट॒म् जुष्ट(ग्म्॑) सुवृ॒क्तिभि॑ स्तपत तपता सुवृ॒क्तिभि॒र् जुष्ट᳚म् । \newline
36. सु॒वृ॒क्तिभि॒र् जुष्ट॒म् जुष्ट(ग्म्॑) सुवृ॒क्तिभिः॑ सुवृ॒क्तिभि॒र् जुष्ट॒म् गिर्व॑णसे॒ गिर्व॑णसे॒ जुष्ट(ग्म्॑) सुवृ॒क्तिभिः॑ सुवृ॒क्तिभि॒र् जुष्ट॒म् गिर्व॑णसे । \newline
37. सु॒वृ॒क्तिभि॒रिति॑ सुवृ॒क्ति - भिः॒ । \newline
38. जुष्ट॒म् गिर्व॑णसे॒ गिर्व॑णसे॒ जुष्ट॒म् जुष्ट॒म् गिर्व॑णसे॒ गिरो॒ गिरो॒ गिर्व॑णसे॒ जुष्ट॒म् जुष्ट॒म् गिर्व॑णसे॒ गिरः॑ । \newline
39. गिर्व॑णसे॒ गिरो॒ गिरो॒ गिर्व॑णसे॒ गिर्व॑णसे॒ गिरः॑ । \newline
40. गिर॒ इति॒ गिरः॑ । \newline
41. इन्द्र॒ मिदि दिन्द्र॒ मिन्द्र॒ मिद् गा॒थिनो॑ गा॒थिन॒ इदिन्द्र॒ मिन्द्र॒ मिद् गा॒थिनः॑ । \newline
42. इद् गा॒थिनो॑ गा॒थिन॒ इदिद् गा॒थिनो॑ बृ॒हद् बृ॒हद् गा॒थिन॒ इदिद् गा॒थिनो॑ बृ॒हत् । \newline
43. गा॒थिनो॑ बृ॒हद् बृ॒हद् गा॒थिनो॑ गा॒थिनो॑ बृ॒हदिन्द्र॒ मिन्द्र॑म् बृ॒हद् गा॒थिनो॑ गा॒थिनो॑ बृ॒हदिन्द्र᳚म् । \newline
44. बृ॒हदिन्द्र॒ मिन्द्र॑म् बृ॒हद् बृ॒हदिन्द्र॑ म॒र्केभि॑ र॒र्केभि॒ रिन्द्र॑म् बृ॒हद् बृ॒हदिन्द्र॑ म॒र्केभिः॑ । \newline
45. इन्द्र॑ म॒र्केभि॑ र॒र्केभि॒ रिन्द्र॒ मिन्द्र॑ म॒र्केभि॑ र॒र्किणो॑ अ॒र्किणो॑ अ॒र्केभि॒रिन्द्र॒ मिन्द्र॑ म॒र्केभि॑ र॒र्किणः॑ । \newline
46. अ॒र्केभि॑ र॒र्किणो॑ अ॒र्किणो॑ अ॒र्केभि॑ र॒र्केभि॑ र॒र्किणः॑ । \newline
47. अ॒र्किण॒ इत्य॒र्किणः॑ । \newline
48. इन्द्रं॒ ॅवाणी॒र् वाणी॒रिन्द्र॒ मिन्द्रं॒ ॅवाणी॑ रनूषतानूषत॒ वाणी॒रिन्द्र॒ मिन्द्रं॒ ॅवाणी॑रनूषत । \newline
49. वाणी॑ रनूषतानूषत॒ वाणी॒र् वाणी॑ रनूषत । \newline
50. अ॒नू॒ष॒तेत्य॑नूषत । \newline
51. गाय॑न्ति त्वा त्वा॒ गाय॑न्ति॒ गाय॑न्ति त्वा गाय॒त्रिणो॑ गाय॒त्रिण॑ स्त्वा॒ गाय॑न्ति॒ गाय॑न्ति त्वा गाय॒त्रिणः॑ । \newline
52. त्वा॒ गा॒य॒त्रिणो॑ गाय॒त्रिण॑ स्त्वा त्वा गाय॒त्रिणो ऽर्च॒न्त्यर्च॑न्ति गाय॒त्रिण॑ स्त्वा त्वा गाय॒त्रिणो ऽर्च॑न्ति । \newline
53. गा॒य॒त्रिणो ऽर्च॒न्त्यर्च॑न्ति गाय॒त्रिणो॑ गाय॒त्रिणो ऽर्च॑न्त्य॒र्क म॒र्क मर्च॑न्ति गाय॒त्रिणो॑ गाय॒त्रिणो ऽर्च॑न्त्य॒र्कम् । \newline
\pagebreak
\markright{ TS 1.6.12.3  \hfill https://www.vedavms.in \hfill}

\section{ TS 1.6.12.3 }

\textbf{TS 1.6.12.3 } \newline
\textbf{Samhita Paata} \newline

ऽर्च॑न्त्य॒र्क-म॒र्किणः॑ । ब्र॒ह्माण॑स्त्वा शतक्रत॒वुद्-वꣳ॒॒श-मि॑व येमिरे ॥ अꣳ॒॒हो॒मुचे॒ प्र भ॑रेमा मनी॒षामो॑षिष्ठ॒दाव्.न्ने॑ सुम॒तिं गृ॑णा॒नाः । इ॒दमि॑न्द्र॒ प्रति॑ ह॒व्यं गृ॑भाय स॒त्याः स॑न्तु॒ यज॑मानस्य॒ कामाः᳚ ॥ वि॒वेष॒ यन्मा॑ धि॒षणा॑ ज॒जान॒ स्तवै॑ पु॒रा पार्या॒दिन्द्र॒-मह्नः॑ । अꣳह॑सो॒ यत्र॑ पी॒पर॒द्यथा॑ नो ना॒वेव॒ यान्त॑ मु॒भये॑ हवन्ते ॥ प्र स॒म्राजं॑ प्रथ॒म-म॑द्ध्व॒राणा॑ - [ ] \newline

\textbf{Pada Paata} \newline

अर्च॑न्ति । अ॒र्कम् । अ॒र्किणः॑ ॥ ब्र॒ह्माणः॑ । त्वा॒ । श॒त॒क्र॒त॒विति॑ शत - क्र॒तो॒ । उदिति॑ । वꣳ॒॒शम् । इ॒व॒ । ये॒मि॒रे॒ ॥ अꣳ॒॒हो॒मुच॒ इत्यꣳ॑हः - मुचे᳚ । प्रेति॑ । भ॒रे॒म॒ । म॒नी॒षाम् । ओ॒षि॒ष्ठ॒दाव्.न्न॒ इत्यो॑षिष्ठ - दाव्.न्ने᳚ । सु॒म॒तिमिति॑ सु - म॒तिम् । गृ॒णा॒नाः ॥ इ॒दम् । इ॒न्द्र॒ । प्रतीति॑ । ह॒व्यम् । गृ॒भा॒य॒ । स॒त्याः । स॒न्तु॒ । यज॑मानस्य । कामाः᳚ ॥ वि॒वेष॑ । यत् । मा॒ । धि॒षणा᳚ । ज॒जान॑ । स्तवै᳚ । पु॒रा । पार्या᳚त् । इन्द्र᳚म् । अह्नः॑ ॥ अꣳह॑सः । यत्र॑ । पी॒पर॑त् । यथा᳚ । नः॒ । ना॒वा । इ॒व॒ । यान्त᳚म् । उ॒भये᳚ । ह॒व॒न्ते॒ ॥ प्रेति॑ । स॒म्राज॒मिति॑ सं - राज᳚म् । प्र॒थ॒मम् । अ॒द्ध्व॒राणा᳚म् ।  \newline


\textbf{Krama Paata} \newline

अर्च॑न्त्य॒र्कम् । अ॒र्कम॒र्किणः॑ । अ॒र्किण॒ इत्य॒र्किणः॑ ॥ ब्र॒ह्माण॑स्त्वा । त्वा॒ श॒त॒क्र॒तो॒ । श॒त॒क्र॒त॒ उत् । श॒त॒क्र॒त॒विति॑ शत - क्र॒तो॒ । उद् वꣳ॒॒शम् । वꣳ॒॒शमि॑व । इ॒व॒ ये॒मि॒रे॒ । ये॒मि॒र॒ इति॑ येमिरे ॥ अꣳ॒॒हो॒मुचे॒ प्र । अꣳ॒॒हो॒मुच॒ इत्यꣳ॑हः - मुचे᳚ । प्र भ॑रेम । भ॒रे॒मा॒ म॒नी॒षाम् । म॒नी॒षामो॑षिष्ठ॒दाव्.न्ने᳚ । ओ॒षि॒ष्ठ॒दाव्.न्ने॑ सुम॒तिम् । ओ॒षि॒ष्ठ॒दाव्.न्न॒ इत्यो॑षिष्ठ - दाव्.न्ने᳚ । सु॒म॒तिं गृ॑णा॒नाः । सु॒म॒तिमिति॑ सु - म॒तिम् । गृ॒णा॒ना इति॑ गृणा॒नाः । इ॒दमि॑न्द्र । इ॒न्द्र॒ प्रति॑ । प्रति॑ ह॒व्यम् । ह॒व्यम् गृ॑भाय । गृ॒भा॒य॒ स॒त्याः । स॒त्याः स॑न्तु । स॒न्तु॒ यज॑मानस्य । यज॑मानस्य॒ कामाः᳚ । कामा॒ इति॒ कामाः᳚ ॥ वि॒वेष॒ यत् । यन्मा᳚ । मा॒ धि॒षणा᳚ । धि॒षणा॑ ज॒जान॑ । ज॒जान॒ स्तवै᳚ । स्तवै॑ पु॒रा । पु॒रा पार्या᳚त् । पार्या॒दिन्द्र᳚म् । इन्द्र॒मह्नः॑ । अह्न॒ इत्यह्नः॑ ॥ अꣳह॑सो॒ यत्र॑ । यत्र॑ पी॒पर॑त् । पी॒पर॒द् यथा᳚ । यथा॑ नः । नो॒ ना॒वा । ना॒वेव॑ । इ॒व॒ यान्त᳚म् । यान्त॑मु॒भये᳚ । उ॒भये॑ हवन्ते । ह॒व॒न्त॒ इति॑ हवन्ते ॥ प्र स॒म्राज᳚म् । स॒म्राज॑म् प्रथ॒मम् । स॒म्राज॒मिति॑ सम् - राज᳚म् । प्र॒थ॒मम॑द्ध्व॒राणा᳚म् । अ॒द्ध्व॒राणा॑मꣳहो॒मुच᳚म् \newline

\textbf{Jatai Paata} \newline

1. अर्च॑न्त्य॒र्क म॒र्क मर्च॒ न्त्यर्च॑ न्त्य॒र्कम् । \newline
2. अ॒र्क म॒र्किणो॑ अ॒र्किणो॑ अ॒र्क म॒र्क म॒र्किणः॑ । \newline
3. अ॒र्किण॒ इत्य॒र्किणः॑ । \newline
4. ब्र॒ह्माण॑ स्त्वा त्वा ब्र॒ह्माणो᳚ ब्र॒ह्माण॑ स्त्वा । \newline
5. त्वा॒ श॒त॒क्र॒तो॒ श॒त॒क्र॒तो॒ त्वा॒ त्वा॒ श॒त॒क्र॒तो॒ । \newline
6. श॒त॒क्र॒त वुदु च्छ॑तक्रतो शतक्र॒त वुत् । \newline
7. श॒त॒क्र॒त॒विति॑ शत - क्र॒तो॒ । \newline
8. उद् व॒(ग्म्॒)शं ॅव॒(ग्म्॒)श मुदुद् व॒(ग्म्॒)शम् । \newline
9. व॒(ग्म्॒)श मि॑वे व व॒(ग्म्॒)शं ॅव॒(ग्म्॒)श मि॑व । \newline
10. इ॒व॒ ये॒मि॒रे॒ ये॒मि॒र॒ इ॒वे॒ व॒ ये॒मि॒रे॒ । \newline
11. ये॒मि॒र॒ इति॑ येमिरे । \newline
12. अ॒(ग्म्॒)हो॒मुचे॒ प्र प्रा(ग्म्॑)हो॒मुचे ऽꣳ॑हो॒मुचे॒ प्र । \newline
13. अ॒(ग्म्॒)हो॒मुच॒ इत्य(ग्म्॑)हः - मुचे᳚ । \newline
14. प्र भ॑रेम भरेम॒ प्र प्र भ॑रेम । \newline
15. भ॒रे॒मा॒ म॒नी॒षाम् म॑नी॒षाम् भ॑रेम भरेमा मनी॒षाम् । \newline
16. म॒नी॒षा मो॑षिष्ठ॒दाव्.न्न॑ ओषिष्ठ॒दाव्.न्ने॑ मनी॒षाम् म॑नी॒षा मो॑षिष्ठ॒दाव्.न्ने᳚ । \newline
17. ओ॒षि॒ष्ठ॒दाव्.न्ने॑ सुम॒तिꣳ सु॑म॒ति मो॑षिष्ठ॒दाव्.न्न॑ ओषिष्ठ॒दाव्.न्ने॑ सुम॒तिम् । \newline
18. ओ॒षि॒ष्ठ॒दाव्.न्न॒ इत्यो॑षिष्ठ - दाव्.न्ने᳚ । \newline
19. सु॒म॒तिम् गृ॑णा॒ना गृ॑णा॒नाः सु॑म॒तिꣳ सु॑म॒तिम् गृ॑णा॒नाः । \newline
20. सु॒म॒तिमिति॑ सु - म॒तिम् । \newline
21. गृ॒णा॒ना इति॑ गृणा॒नाः । \newline
22. इ॒द मि॑न्द्रे न्द्रे॒ द मि॒द मि॑न्द्र । \newline
23. इ॒न्द्र॒ प्रति॒ प्रती᳚न्द्रे न्द्र॒ प्रति॑ । \newline
24. प्रति॑ ह॒व्यꣳ ह॒व्यम् प्रति॒ प्रति॑ ह॒व्यम् । \newline
25. ह॒व्यम् गृ॑भाय गृभाय ह॒व्यꣳ ह॒व्यम् गृ॑भाय । \newline
26. गृ॒भा॒य॒ स॒त्याः स॒त्या गृ॑भाय गृभाय स॒त्याः । \newline
27. स॒त्याः स॑न्तु सन्तु स॒त्याः स॒त्याः स॑न्तु । \newline
28. स॒न्तु॒ यज॑मानस्य॒ यज॑मानस्य सन्तु सन्तु॒ यज॑मानस्य । \newline
29. यज॑मानस्य॒ कामाः॒ कामा॒ यज॑मानस्य॒ यज॑मानस्य॒ कामाः᳚ । \newline
30. कामा॒ इति॒ कामाः᳚ । \newline
31. वि॒वेष॒ यद् यद् वि॒वेष॑ वि॒वेष॒ यत् । \newline
32. यन् मा॑ मा॒ यद् यन् मा᳚ । \newline
33. मा॒ धि॒षणा॑ धि॒षणा॑ मा मा धि॒षणा᳚ । \newline
34. धि॒षणा॑ ज॒जान॑ ज॒जान॑ धि॒षणा॑ धि॒षणा॑ ज॒जान॑ । \newline
35. ज॒जान॒ स्तवै॒ स्तवै॑ ज॒जान॑ ज॒जान॒ स्तवै᳚ । \newline
36. स्तवै॑ पु॒रा पु॒रा स्तवै॒ स्तवै॑ पु॒रा । \newline
37. पु॒रा पार्या॒त् पार्या᳚त् पु॒रा पु॒रा पार्या᳚त् । \newline
38. पार्या॒ दिन्द्र॒ मिन्द्र॒म् पार्या॒त् पार्या॒ दिन्द्र᳚म् । \newline
39. इन्द्र॒ मह्नो॒ अह्न॒ इन्द्र॒ मिन्द्र॒ मह्नः॑ । \newline
40. अह्न॒ इत्यह्नः॑ । \newline
41. अꣳह॑सो॒ यत्र॒ यत्राꣳह॑सो॒ अꣳह॑सो॒ यत्र॑ । \newline
42. यत्र॑ पी॒पर॑त् पी॒पर॒द् यत्र॒ यत्र॑ पी॒पर॑त् । \newline
43. पी॒पर॒द् यथा॒ यथा॑ पी॒पर॑त् पी॒पर॒द् यथा᳚ । \newline
44. यथा॑ नो नो॒ यथा॒ यथा॑ नः । \newline
45. नो॒ ना॒वा ना॒वा नो॑ नो ना॒वा । \newline
46. ना॒वेवे॑ व ना॒वा ना॒वेव॑ । \newline
47. इ॒व॒ यान्तं॒ ॅयान्त॑ मिवे व॒ यान्त᳚म् । \newline
48. यान्त॑ मु॒भय॑ उ॒भये॒ यान्तं॒ ॅयान्त॑ मु॒भये᳚ । \newline
49. उ॒भये॑ हवन्ते हवन्त उ॒भय॑ उ॒भये॑ हवन्ते । \newline
50. ह॒व॒न्त॒ इति॑ हवन्ते । \newline
51. प्र स॒म्राज(ग्म्॑) स॒म्राज॒म् प्र प्र स॒म्राज᳚म् । \newline
52. स॒म्राज॑म् प्रथ॒मम् प्र॑थ॒मꣳ स॒म्राज(ग्म्॑) स॒म्राज॑म् प्रथ॒मम् । \newline
53. स॒म्राज॒मिति॑ सं - राज᳚म् । \newline
54. प्र॒थ॒म म॑द्ध्व॒राणा॑ मद्ध्व॒राणा᳚म् प्रथ॒मम् प्र॑थ॒म म॑द्ध्व॒राणा᳚म् । \newline
55. अ॒द्ध्व॒राणा॑ मꣳहो॒मुच॑ मꣳहो॒मुच॑ मद्ध्व॒राणा॑ मद्ध्व॒राणा॑ मꣳहो॒मुच᳚म् । \newline

\textbf{Ghana Paata } \newline

1. अर्च॑न्त्य॒र्क म॒र्क मर्च॒न्त्यर्च॑न्त्य॒र्क म॒र्किणो॑ अ॒र्किणो॑ अ॒र्क मर्च॒न्त्यर्च॑न्त्य॒र्क म॒र्किणः॑ । \newline
2. अ॒र्क म॒र्किणो॑ अ॒र्किणो॑ अ॒र्क म॒र्क म॒र्किणः॑ । \newline
3. अ॒र्किण॒ इत्य॒र्किणः॑ । \newline
4. ब्र॒ह्माण॑ स्त्वा त्वा ब्र॒ह्माणो᳚ ब्र॒ह्माण॑ स्त्वा शतक्रतो शतक्रतो त्वा ब्र॒ह्माणो᳚ ब्र॒ह्माण॑ स्त्वा शतक्रतो । \newline
5. त्वा॒ श॒त॒क्र॒तो॒ श॒त॒क्र॒तो॒ त्वा॒ त्वा॒ श॒त॒क्र॒त वुदुच्छ॑तक्रतो त्वा त्वा शतक्र॒तवुत् । \newline
6. श॒त॒क्र॒त वुदुच्छ॑तक्रतो शतक्र॒तवुद् व॒(ग्म्॒)शं ॅव॒(ग्म्॒)श मुच्छ॑तक्रतो शतक्र॒तवुद् व॒(ग्म्॒)शम् । \newline
7. श॒त॒क्र॒त॒विति॑ शत - क्र॒तो॒ । \newline
8. उद् व॒(ग्म्॒)शं ॅव॒(ग्म्॒)श मुदुद् व॒(ग्म्॒)श मि॑वे व व॒(ग्म्॒)श मुदुद् व॒(ग्म्॒)श मि॑व । \newline
9. व॒(ग्म्॒)श मि॑वे व व॒(ग्म्॒)शं ॅव॒(ग्म्॒)श मि॑व येमिरे येमिर इव व॒(ग्म्॒)शं ॅव॒(ग्म्॒)श मि॑व येमिरे । \newline
10. इ॒व॒ ये॒मि॒रे॒ ये॒मि॒र॒ इ॒वे॒ व॒ ये॒मि॒रे॒ । \newline
11. ये॒मि॒र॒ इति॑ येमिरे । \newline
12. अ॒(ग्म्॒)हो॒मुचे॒ प्र प्रा(ग्म्॑)हो॒मुचे ऽ(ग्म्॑)हो॒मुचे॒ प्र भ॑रेम भरेम प्रा(ग्म्॑)हो॒मुचे ऽ(ग्म्॑)हो॒मुचे॒ प्र भ॑रेम । \newline
13. अ॒(ग्म्॒)हो॒मुच॒ इत्य(ग्म्॑)हः - मुचे᳚ । \newline
14. प्र भ॑रेम भरेम॒ प्र प्र भ॑रेमा मनी॒षाम् म॑नी॒षाम् भ॑रेम॒ प्र प्र भ॑रेमा मनी॒षाम् । \newline
15. भ॒रे॒मा॒ म॒नी॒षाम् म॑नी॒षाम् भ॑रेम भरेमा मनी॒षा मो॑षिष्ठ॒दाव्.न्न॑ ओषिष्ठ॒दाव्.न्ने॑ मनी॒षाम् भ॑रेम भरेमा मनी॒षा मो॑षिष्ठ॒दाव्.न्ने᳚ । \newline
16. म॒नी॒षा मो॑षिष्ठ॒दाव्.न्न॑ ओषिष्ठ॒दाव्.न्ने॑ मनी॒षाम् म॑नी॒षा मो॑षिष्ठ॒दाव्.न्ने॑ सुम॒तिꣳ सु॑म॒ति मो॑षिष्ठ॒दाव्.न्ने॑ मनी॒षाम् म॑नी॒षा मो॑षिष्ठ॒दाव्.न्ने॑ सुम॒तिम् । \newline
17. ओ॒षि॒ष्ठ॒दाव्.न्ने॑ सुम॒तिꣳ सु॑म॒ति मो॑षिष्ठ॒दाव्.न्न॑ ओषिष्ठ॒दाव्.न्ने॑ सुम॒तिम् गृ॑णा॒ना गृ॑णा॒नाः सु॑म॒ति मो॑षिष्ठ॒दाव्.न्न॑ ओषिष्ठ॒दाव्.न्ने॑ सुम॒तिम् गृ॑णा॒नाः । \newline
18. ओ॒षि॒ष्ठ॒दाव्.न्न॒ इत्यो॑षिष्ठ - दाव्.न्ने᳚ । \newline
19. सु॒म॒तिम् गृ॑णा॒ना गृ॑णा॒नाः सु॑म॒तिꣳ सु॑म॒तिम् गृ॑णा॒नाः । \newline
20. सु॒म॒तिमिति॑ सु - म॒तिम् । \newline
21. गृ॒णा॒ना इति॑ गृणा॒नाः । \newline
22. इ॒द मि॑न्द्रे न्द्रे॒ द मि॒द मि॑न्द्र॒ प्रति॒ प्रती᳚न्द्रे॒ द मि॒द मि॑न्द्र॒ प्रति॑ । \newline
23. इ॒न्द्र॒ प्रति॒ प्रती᳚न्द्रे न्द्र॒ प्रति॑ ह॒व्यꣳ ह॒व्यम् प्रती᳚न्द्रे न्द्र॒ प्रति॑ ह॒व्यम् । \newline
24. प्रति॑ ह॒व्यꣳ ह॒व्यम् प्रति॒ प्रति॑ ह॒व्यम् गृ॑भाय गृभाय ह॒व्यम् प्रति॒ प्रति॑ ह॒व्यम् गृ॑भाय । \newline
25. ह॒व्यम् गृ॑भाय गृभाय ह॒व्यꣳ ह॒व्यम् गृ॑भाय स॒त्याः स॒त्या गृ॑भाय ह॒व्यꣳ ह॒व्यम् गृ॑भाय स॒त्याः । \newline
26. गृ॒भा॒य॒ स॒त्याः स॒त्या गृ॑भाय गृभाय स॒त्याः स॑न्तु सन्तु स॒त्या गृ॑भाय गृभाय स॒त्याः स॑न्तु । \newline
27. स॒त्याः स॑न्तु सन्तु स॒त्याः स॒त्याः स॑न्तु॒ यज॑मानस्य॒ यज॑मानस्य सन्तु स॒त्याः स॒त्याः स॑न्तु॒ यज॑मानस्य । \newline
28. स॒न्तु॒ यज॑मानस्य॒ यज॑मानस्य सन्तु सन्तु॒ यज॑मानस्य॒ कामाः॒ कामा॒ यज॑मानस्य सन्तु सन्तु॒ यज॑मानस्य॒ कामाः᳚ । \newline
29. यज॑मानस्य॒ कामाः॒ कामा॒ यज॑मानस्य॒ यज॑मानस्य॒ कामाः᳚ । \newline
30. कामा॒ इति॒ कामाः᳚ । \newline
31. वि॒वेष॒ यद् यद् वि॒वेष॑ वि॒वेष॒ यन् मा॑ मा॒ यद् वि॒वेष॑ वि॒वेष॒ यन् मा᳚ । \newline
32. यन् मा॑ मा॒ यद् यन् मा॑ धि॒षणा॑ धि॒षणा॑ मा॒ यद् यन् मा॑ धि॒षणा᳚ । \newline
33. मा॒ धि॒षणा॑ धि॒षणा॑ मा मा धि॒षणा॑ ज॒जान॑ ज॒जान॑ धि॒षणा॑ मा मा धि॒षणा॑ ज॒जान॑ । \newline
34. धि॒षणा॑ ज॒जान॑ ज॒जान॑ धि॒षणा॑ धि॒षणा॑ ज॒जान॒ स्तवै॒ स्तवै॑ ज॒जान॑ धि॒षणा॑ धि॒षणा॑ ज॒जान॒ स्तवै᳚ । \newline
35. ज॒जान॒ स्तवै॒ स्तवै॑ ज॒जान॑ ज॒जान॒ स्तवै॑ पु॒रा पु॒रा स्तवै॑ ज॒जान॑ ज॒जान॒ स्तवै॑ पु॒रा । \newline
36. स्तवै॑ पु॒रा पु॒रा स्तवै॒ स्तवै॑ पु॒रा पार्या॒त् पार्या᳚त् पु॒रा स्तवै॒ स्तवै॑ पु॒रा पार्या᳚त् । \newline
37. पु॒रा पार्या॒त् पार्या᳚त् पु॒रा पु॒रा पार्या॒दिन्द्र॒ मिन्द्र॒म् पार्या᳚त् पु॒रा पु॒रा पार्या॒दिन्द्र᳚म् । \newline
38. पार्या॒दिन्द्र॒ मिन्द्र॒म् पार्या॒त् पार्या॒दिन्द्र॒ मह्नो॒ अह्न॒ इन्द्र॒म् पार्या॒त् पार्या॒दिन्द्र॒ मह्नः॑ । \newline
39. इन्द्र॒ मह्नो॒ अह्न॒ इन्द्र॒ मिन्द्र॒ मह्नः॑ । \newline
40. अह्न॒ इत्यह्नः॑ । \newline
41. अꣳह॑सो॒ यत्र॒ यत्राꣳह॑सो॒ अꣳह॑सो॒ यत्र॑ पी॒पर॑त् पी॒पर॒द् यत्राꣳह॑सो॒ अꣳह॑सो॒ यत्र॑ पी॒पर॑त् । \newline
42. यत्र॑ पी॒पर॑त् पी॒पर॒द् यत्र॒ यत्र॑ पी॒पर॒द् यथा॒ यथा॑ पी॒पर॒द् यत्र॒ यत्र॑ पी॒पर॒द् यथा᳚ । \newline
43. पी॒पर॒द् यथा॒ यथा॑ पी॒पर॑त् पी॒पर॒द् यथा॑ नो नो॒ यथा॑ पी॒पर॑त् पी॒पर॒द् यथा॑ नः । \newline
44. यथा॑ नो नो॒ यथा॒ यथा॑ नो ना॒वा ना॒वा नो॒ यथा॒ यथा॑ नो ना॒वा । \newline
45. नो॒ ना॒वा ना॒वा नो॑ नो ना॒वेवे॑ व ना॒वा नो॑ नो ना॒वेव॑ । \newline
46. ना॒वेवे॑ व ना॒वा ना॒वेव॒ यान्तं॒ ॅयान्त॑ मिव ना॒वा ना॒वेव॒ यान्त᳚म् । \newline
47. इ॒व॒ यान्तं॒ ॅयान्त॑ मिवे व॒ यान्त॑ मु॒भय॑ उ॒भये॒ यान्त॑ मिवे व॒ यान्त॑ मु॒भये᳚ । \newline
48. यान्त॑ मु॒भय॑ उ॒भये॒ यान्तं॒ ॅयान्त॑ मु॒भये॑ हवन्ते हवन्त उ॒भये॒ यान्तं॒ ॅयान्त॑ मु॒भये॑ हवन्ते । \newline
49. उ॒भये॑ हवन्ते हवन्त उ॒भय॑ उ॒भये॑ हवन्ते । \newline
50. ह॒व॒न्त॒ इति॑ हवन्ते । \newline
51. प्र स॒म्राज(ग्म्॑) स॒म्राज॒म् प्र प्र स॒म्राज॑म् प्रथ॒मम् प्र॑थ॒मꣳ स॒म्राज॒म् प्र प्र स॒म्राज॑म् प्रथ॒मम् । \newline
52. स॒म्राज॑म् प्रथ॒मम् प्र॑थ॒मꣳ स॒म्राज(ग्म्॑) स॒म्राज॑म् प्रथ॒म म॑द्ध्व॒राणा॑ मद्ध्व॒राणा᳚म् प्रथ॒मꣳ स॒म्राज(ग्म्॑) स॒म्राज॑म् प्रथ॒म म॑द्ध्व॒राणा᳚म् । \newline
53. स॒म्राज॒मिति॑ सं - राज᳚म् । \newline
54. प्र॒थ॒म म॑द्ध्व॒राणा॑ मद्ध्व॒राणा᳚म् प्रथ॒मम् प्र॑थ॒म म॑द्ध्व॒राणा॑ मꣳहो॒मुच॑ मꣳहो॒मुच॑ मद्ध्व॒राणा᳚म् प्रथ॒मम् प्र॑थ॒म म॑द्ध्व॒राणा॑ मꣳहो॒मुच᳚म् । \newline
55. अ॒द्ध्व॒राणा॑ मꣳहो॒मुच॑ मꣳहो॒मुच॑ मद्ध्व॒राणा॑ मद्ध्व॒राणा॑ मꣳहो॒मुचं॑ ॅवृष॒भं ॅवृ॑ष॒भ म(ग्म्॑)हो॒मुच॑ मद्ध्व॒राणा॑ मद्ध्व॒राणा॑ मꣳहो॒मुचं॑ ॅवृष॒भम् । \newline
\pagebreak
\markright{ TS 1.6.12.4  \hfill https://www.vedavms.in \hfill}

\section{ TS 1.6.12.4 }

\textbf{TS 1.6.12.4 } \newline
\textbf{Samhita Paata} \newline

मꣳहो॒मुचं॑ ॅवृष॒भं ॅय॒ज्ञिया॑नां । अ॒पां नपा॑तमश्विना॒ हय॑न्त-म॒स्मिन्न॑र इन्द्रि॒यं ध॑त्त॒मोजः॑ ॥ वि न॑ इन्द्र॒ मृधो॑ जहि नी॒चा य॑च्छ पृतन्य॒तः । अ॒ध॒स्प॒दं तमीं᳚ कृधि॒ यो अ॒स्माꣳ अ॑भि॒दास॑ति ॥इन्द्र॑ क्ष॒त्रम॒भि वा॒ममोजो ऽजा॑यथा वृषभ चर्.षणी॒नां । अपा॑नुदो॒ जन॑-ममित्र॒यन्त॑-मु॒रुं दे॒वेभ्यो॑ अकृणो-रु लो॒कं ॥ मृ॒गो न भी॒मः कु॑च॒रो गि॑रि॒ष्ठाः प॑रा॒वत - [ ] \newline

\textbf{Pada Paata} \newline

अꣳ॒॒हो॒मुच॒मित्यꣳ॑हः - मुच᳚म् । वृ॒ष॒भम् । य॒ज्ञिया॑नाम् ॥ अ॒पाम् । नपा॑तम् । अ॒श्वि॒ना॒ । हय॑न्तम् । अ॒स्मिन्न् । न॒रः॒ । इ॒न्द्रि॒यम् । ध॒त्त॒म् । ओजः॑ ॥ वीति॑ । नः॒ । इ॒न्द्र॒ । मृधः॑ । ज॒हि॒ । नी॒चा । य॒च्छ॒ । पृ॒त॒न्य॒तः ॥ अ॒ध॒स्प॒दमित्य॑धः - प॒दम् । तम् । ई॒म् । कृ॒धि॒ । यः । अ॒स्मान् । अ॒भि॒दास॒तीत्य॑भि - दास॑ति ॥ इन्द्र॑ । क्ष॒त्रम् । अ॒भीति॑ । वा॒मम् । ओजः॑ । अजा॑यथाः । वृ॒ष॒भ॒ । च॒र्॒.ष॒णी॒नाम् ॥ अपेति॑ । अ॒नु॒दः॒ । जन᳚म् । अ॒मि॒त्र॒यन्त॒मित्य॑मित्र - यन्त᳚म् । उ॒रुम् । दे॒वेभ्यः॑ । अ॒कृ॒णोः॒ । उ॒ । लो॒कम् ॥ मृ॒गः । न । भी॒मः । कु॒च॒रः । गि॒रि॒ष्ठा इति॑ गिरि - स्थाः । प॒रा॒वत॒ इति॑ परा - वतः॑ ।  \newline


\textbf{Krama Paata} \newline

अꣳ॒॒हो॒मुचं॑ ॅवृष॒भम् । अꣳ॒॒हो॒मुच॒मित्यꣳ॑हः - मुच᳚म् । वृ॒ष॒भं ॅय॒ज्ञिया॑नाम् । य॒ज्ञिया॑ना॒मिति॑ य॒ज्ञिया॑नाम् ॥ अ॒पाम् नपा॑तम् । नपा॑तमश्विना । अ॒श्वि॒ना॒ हय॑न्तम् । हय॑न्तम॒स्मिन्न् । अ॒स्मिन्न॑रः । न॒र॒ इ॒न्द्रि॒यम् । इ॒न्द्रि॒यम् ध॑त्तम् । ध॒त्त॒मोजः॑ । ओज॒ इत्योजः॑ ॥ वि नः॑ । न॒ इ॒न्द्र॒ । इ॒न्द्र॒ मृधः॑ । मृधो॑ जहि । ज॒हि॒ नी॒चा । नी॒चा य॑च्छ । य॒च्छ॒ पृ॒त॒न्य॒तः । पृ॒त॒न्य॒त इति॑ पृतन्य॒तः ॥ अ॒ध॒स्प॒दम् तम् । अ॒ध॒स्प॒दमित्य॑धः - प॒दम् । तमी᳚म् । ई॒म् कृ॒धि॒ । कृ॒धि॒ यः । यो अ॒स्मान् । अ॒स्माꣳ अ॑भि॒दास॑ति । अ॒भि॒दास॒तीत्य॑भि - दास॑ति ॥ इन्द्र॑ क्ष॒त्रम् । क्ष॒त्रम॒भि । अ॒भि वा॒मम् । वा॒ममोजः॑ । ओजोऽजा॑यथाः । अजा॑यथा वृषभ । वृ॒ष॒भ॒ च॒र्.॒ष॒णी॒नाम् । च॒र्.॒ष॒णी॒नामिति॑ चर्.षणी॒नाम् ॥ अपा॑नुदः । अ॒नु॒दो॒ जन᳚म् । जन॑म,मित्र॒यन्त᳚म् । अ॒मि॒त्र॒यन्त॑मु॒रुम् । अ॒मि॒त्र॒यन्त॒मित्य॑मित्र - यन्त᳚म् । उ॒रुम् दे॒वेभ्यः॑ । दे॒वेभ्यो॑ अकृणोः । अ॒कृ॒णो॒रु॒ । उ॒ लो॒कम् । लो॒कमिति॑ लो॒कम् ॥ मृ॒गो न । न भी॒मः । भी॒मः कु॑च॒रः । कु॒च॒रो गि॑रि॒ष्ठाः । गि॒रि॒ष्ठाः प॑रा॒वतः॑ । गि॒रि॒ष्ठा इति॑ गिरि - स्थाः । प॒रा॒वत॒ आ । प॒रा॒वत॒ इति॑ परा - वतः॑ \newline

\textbf{Jatai Paata} \newline

1. अ॒(ग्म्॒)हो॒मुचं॑ ॅवृष॒भं ॅवृ॑ष॒भ म(ग्म्॑)हो॒मुच॑ मꣳहो॒मुचं॑ ॅवृष॒भम् । \newline
2. अ॒(ग्म्॒)हो॒मुच॒मित्य(ग्म्॑)हः - मुच᳚म् । \newline
3. वृ॒ष॒भं ॅय॒ज्ञिया॑नां ॅय॒ज्ञिया॑नां ॅवृष॒भं ॅवृ॑ष॒भं ॅय॒ज्ञिया॑नाम् । \newline
4. य॒ज्ञिया॑ना॒मिति॑ य॒ज्ञिया॑नाम् । \newline
5. अ॒पाम् नपा॑त॒म् नपा॑त म॒पा म॒पाम् नपा॑तम् । \newline
6. नपा॑त मश्विना ऽश्विना॒ नपा॑त॒म् नपा॑त मश्विना । \newline
7. अ॒श्वि॒ना॒ हय॑न्त॒(ग्म्॒) हय॑न्त मश्विना ऽश्विना॒ हय॑न्तम् । \newline
8. हय॑न्त म॒स्मिन् न॒स्मिन्. हय॑न्त॒(ग्म्॒) हय॑न्त म॒स्मिन्न् । \newline
9. अ॒स्मिन् न॑रो नरो अ॒स्मिन् न॒स्मिन् न॑रः । \newline
10. न॒र॒ इ॒न्द्रि॒य मि॑न्द्रि॒यम् न॑रो नर इन्द्रि॒यम् । \newline
11. इ॒न्द्रि॒यम् ध॑त्तम् धत्त मिन्द्रि॒य मि॑न्द्रि॒यम् ध॑त्तम् । \newline
12. ध॒त्त॒ मोज॒ ओजो॑ धत्तम् धत्त॒ मोजः॑ । \newline
13. ओज॒ इत्योजः॑ । \newline
14. वि नो॑ नो॒ वि वि नः॑ । \newline
15. न॒ इ॒न्द्रे॒ न्द्र॒ नो॒ न॒ इ॒न्द्र॒ । \newline
16. इ॒न्द्र॒ मृधो॒ मृध॑ इन्द्रे न्द्र॒ मृधः॑ । \newline
17. मृधो॑ जहि जहि॒ मृधो॒ मृधो॑ जहि । \newline
18. ज॒हि॒ नी॒चा नी॒चा ज॑हि जहि नी॒चा । \newline
19. नी॒चा य॑च्छ यच्छ नी॒चा नी॒चा य॑च्छ । \newline
20. य॒च्छ॒ पृ॒त॒न्य॒तः पृ॑तन्य॒तो य॑च्छ यच्छ पृतन्य॒तः । \newline
21. पृ॒त॒न्य॒त इति॑ पृतन्य॒तः । \newline
22. अ॒ध॒स्प॒दम् तम् त म॑धस्प॒द म॑धस्प॒दम् तम् । \newline
23. अ॒ध॒स्प॒दमित्य॑धः - प॒दम् । \newline
24. त मी॑ मी॒म् तम् त मी᳚म् । \newline
25. ई॒म् कृ॒धि॒ कृ॒धी॒ मी॒म् कृ॒धि॒ । \newline
26. कृ॒धि॒ यो यस्कृ॑धि कृधि॒ यः । \newline
27. यो अ॒स्माꣳ अ॒स्मान्. यो यो अ॒स्मान् । \newline
28. अ॒स्माꣳ अ॑भि॒दास॑ त्यभि॒दास॑ त्य॒स्माꣳ अ॒स्माꣳ अ॑भि॒दास॑ति । \newline
29. अ॒भि॒दास॒तीत्य॑भि - दास॑ति । \newline
30. इन्द्र॑ क्ष॒त्रम् क्ष॒त्र मिन्द्रे न्द्र॑ क्ष॒त्रम् । \newline
31. क्ष॒त्र म॒भ्य॑भि क्ष॒त्रम् क्ष॒त्र म॒भि । \newline
32. अ॒भि वा॒मं ॅवा॒म म॒भ्य॑भि वा॒मम् । \newline
33. वा॒म मोज॒ ओजो॑ वा॒मं ॅवा॒म मोजः॑ । \newline
34. ओजो ऽजा॑यथा॒ अजा॑यथा॒ ओज॒ ओजो ऽजा॑यथाः । \newline
35. अजा॑यथा वृषभ वृष॒भाजा॑यथा॒ अजा॑यथा वृषभ । \newline
36. वृ॒ष॒भ॒ च॒र्॒.ष॒णी॒नाम् च॑र्.षणी॒नां ॅवृ॑षभ वृषभ चर्.षणी॒नाम् । \newline
37. च॒र्॒.ष॒णी॒नामिति॑ चर्.षणी॒नाम् । \newline
38. अपा॑नुदो ऽनुदो॒ अपापा॑नुदः । \newline
39. अ॒नु॒दो॒ जन॒म् जन॑ मनुदो ऽनुदो॒ जन᳚म् । \newline
40. जन॑ ममित्र॒यन्त॑ ममित्र॒यन्त॒म् जन॒म् जन॑ ममित्र॒यन्त᳚म् । \newline
41. अ॒मि॒त्र॒यन्त॑ मु॒रु मु॒रु म॑मित्र॒यन्त॑ ममित्र॒यन्त॑ मु॒रुम् । \newline
42. अ॒मि॒त्र॒यन्त॒मित्य॑मित्र - यन्त᳚म् । \newline
43. उ॒रुम् दे॒वेभ्यो॑ दे॒वेभ्य॑ उ॒रु मु॒रुम् दे॒वेभ्यः॑ । \newline
44. दे॒वेभ्यो॑ अकृणो रकृणोर् दे॒वेभ्यो॑ दे॒वेभ्यो॑ अकृणोः । \newline
45. अ॒कृ॒णो॒रु॒ वु॒ व॒कृ॒णो॒ र॒कृ॒णो॒रु॒ । \newline
46. उ॒ लो॒कम् ॅलो॒क मु॑ वु लो॒कम् । \newline
47. लो॒कमिति॑ लो॒कम् । \newline
48. मृ॒गो न न मृ॒गो मृ॒गो न । \newline
49. न भी॒मो भी॒मो न न भी॒मः । \newline
50. भी॒मः कु॑च॒रः कु॑च॒रो भी॒मो भी॒मः कु॑च॒रः । \newline
51. कु॒च॒रो गि॑रि॒ष्ठा गि॑रि॒ष्ठाः कु॑च॒रः कु॑च॒रो गि॑रि॒ष्ठाः । \newline
52. गि॒रि॒ष्ठाः प॑रा॒वतः॑ परा॒वतो॑ गिरि॒ष्ठा गि॑रि॒ष्ठाः प॑रा॒वतः॑ । \newline
53. गि॒रि॒ष्ठा इति॑ गिरि - स्थाः । \newline
54. प॒रा॒वत॒ आ प॑रा॒वतः॑ परा॒वत॒ आ । \newline
55. प॒रा॒वत॒ इति॑ परा - वतः॑ । \newline

\textbf{Ghana Paata } \newline

1. अ॒(ग्म्॒)हो॒मुचं॑ ॅवृष॒भं ॅवृ॑ष॒भ म(ग्म्॑)हो॒मुच॑ मꣳहो॒मुचं॑ ॅवृष॒भं ॅय॒ज्ञिया॑नां ॅय॒ज्ञिया॑नां ॅवृष॒भ म(ग्म्॑)हो॒मुच॑ मꣳहो॒मुचं॑ ॅवृष॒भं ॅय॒ज्ञिया॑नाम् । \newline
2. अ॒(ग्म्॒)हो॒मुच॒मित्य(ग्म्॑)हः - मुच᳚म् । \newline
3. वृ॒ष॒भं ॅय॒ज्ञिया॑नां ॅय॒ज्ञिया॑नां ॅवृष॒भं ॅवृ॑ष॒भं ॅय॒ज्ञिया॑नाम् । \newline
4. य॒ज्ञिया॑ना॒मिति॑ य॒ज्ञिया॑नाम् । \newline
5. अ॒पाम् नपा॑त॒म् नपा॑त म॒पा म॒पाम् नपा॑त मश्विना ऽश्विना॒ नपा॑त म॒पा म॒पाम् नपा॑त मश्विना । \newline
6. नपा॑त मश्विना ऽश्विना॒ नपा॑त॒म् नपा॑त मश्विना॒ हय॑न्त॒(ग्म्॒) हय॑न्त मश्विना॒ नपा॑त॒म् नपा॑त मश्विना॒ हय॑न्तम् । \newline
7. अ॒श्वि॒ना॒ हय॑न्त॒(ग्म्॒) हय॑न्त मश्विना ऽश्विना॒ हय॑न्त म॒स्मिन् न॒स्मिन्. हय॑न्त मश्विना ऽश्विना॒ हय॑न्त म॒स्मिन्न् । \newline
8. हय॑न्त म॒स्मिन् न॒स्मिन्. हय॑न्त॒(ग्म्॒) हय॑न्त म॒स्मिन् न॑रो नरो अ॒स्मिन्. हय॑न्त॒(ग्म्॒) हय॑न्त म॒स्मिन् न॑रः । \newline
9. अ॒स्मिन् न॑रो नरो अ॒स्मिन् न॒स्मिन् न॑र इन्द्रि॒य मि॑न्द्रि॒यम् न॑रो अ॒स्मिन् न॒स्मिन् न॑र इन्द्रि॒यम् । \newline
10. न॒र॒ इ॒न्द्रि॒य मि॑न्द्रि॒यम् न॑रो नर इन्द्रि॒यम् ध॑त्तम् धत्त मिन्द्रि॒यम् न॑रो नर इन्द्रि॒यम् ध॑त्तम् । \newline
11. इ॒न्द्रि॒यम् ध॑त्तम् धत्त मिन्द्रि॒य मि॑न्द्रि॒यम् ध॑त्त॒ मोज॒ ओजो॑ धत्त मिन्द्रि॒य मि॑न्द्रि॒यम् ध॑त्त॒ मोजः॑ । \newline
12. ध॒त्त॒ मोज॒ ओजो॑ धत्तम् धत्त॒ मोजः॑ । \newline
13. ओज॒ इत्योजः॑ । \newline
14. वि नो॑ नो॒ वि वि न॑ इन्द्रे न्द्र नो॒ वि वि न॑ इन्द्र । \newline
15. न॒ इ॒न्द्रे॒ न्द्र॒ नो॒ न॒ इ॒न्द्र॒ मृधो॒ मृध॑ इन्द्र नो न इन्द्र॒ मृधः॑ । \newline
16. इ॒न्द्र॒ मृधो॒ मृध॑ इन्द्रे न्द्र॒ मृधो॑ जहि जहि॒ मृध॑ इन्द्रे न्द्र॒ मृधो॑ जहि । \newline
17. मृधो॑ जहि जहि॒ मृधो॒ मृधो॑ जहि नी॒चा नी॒चा ज॑हि॒ मृधो॒ मृधो॑ जहि नी॒चा । \newline
18. ज॒हि॒ नी॒चा नी॒चा ज॑हि जहि नी॒चा य॑च्छ यच्छ नी॒चा ज॑हि जहि नी॒चा य॑च्छ । \newline
19. नी॒चा य॑च्छ यच्छ नी॒चा नी॒चा य॑च्छ पृतन्य॒तः पृ॑तन्य॒तो य॑च्छ नी॒चा नी॒चा य॑च्छ पृतन्य॒तः । \newline
20. य॒च्छ॒ पृ॒त॒न्य॒तः पृ॑तन्य॒तो य॑च्छ यच्छ पृतन्य॒तः । \newline
21. पृ॒त॒न्य॒त इति॑ पृतन्य॒तः । \newline
22. अ॒ध॒स्प॒दम् तम् त म॑धस्प॒द म॑धस्प॒दम् त मी॑ मी॒म् त म॑धस्प॒द म॑धस्प॒दम् त मी᳚म् । \newline
23. अ॒ध॒स्प॒दमित्य॑धः - प॒दम् । \newline
24. त मी॑ मी॒म् तम् त मी᳚म् कृधि कृधी॒म् तम् त मी᳚म् कृधि । \newline
25. ई॒म् कृ॒धि॒ कृ॒धी॒ मी॒म् कृ॒धि॒ यो यस्कृ॑धी मीम् कृधि॒ यः । \newline
26. कृ॒धि॒ यो यस्कृ॑धि कृधि॒ यो अ॒स्माꣳ अ॒स्मान्. यस्कृ॑धि कृधि॒ यो अ॒स्मान् । \newline
27. यो अ॒स्माꣳ अ॒स्मान्. यो यो अ॒स्माꣳ अ॑भि॒दास॑ त्यभि॒दास॑ त्य॒स्मान्. यो यो अ॒स्माꣳ अ॑भि॒दास॑ति । \newline
28. अ॒स्माꣳ अ॑भि॒दास॑ त्यभि॒दास॑त्य॒स्माꣳ अ॒स्माꣳ अ॑भि॒दास॑ति । \newline
29. अ॒भि॒दास॒तीत्य॑भि - दास॑ति । \newline
30. इन्द्र॑ क्ष॒त्रम् क्ष॒त्र मिन्द्रे न्द्र॑ क्ष॒त्र म॒भ्य॑भि क्ष॒त्र मिन्द्रे न्द्र॑ क्ष॒त्र म॒भि । \newline
31. क्ष॒त्र म॒भ्य॑भि क्ष॒त्रम् क्ष॒त्र म॒भि वा॒मं ॅवा॒म म॒भि क्ष॒त्रम् क्ष॒त्र म॒भि वा॒मम् । \newline
32. अ॒भि वा॒मं ॅवा॒म म॒भ्य॑भि वा॒म मोज॒ ओजो॑ वा॒म म॒भ्य॑भि वा॒म मोजः॑ । \newline
33. वा॒म मोज॒ ओजो॑ वा॒मं ॅवा॒म मोजो ऽजा॑यथा॒ अजा॑यथा॒ ओजो॑ वा॒मं ॅवा॒म मोजो ऽजा॑यथाः । \newline
34. ओजो ऽजा॑यथा॒ अजा॑यथा॒ ओज॒ ओजो ऽजा॑यथा वृषभ वृष॒भाजा॑यथा॒ ओज॒ ओजो ऽजा॑यथा वृषभ । \newline
35. अजा॑यथा वृषभ वृष॒भाजा॑यथा॒ अजा॑यथा वृषभ चर्.षणी॒नाम् च॑र्.षणी॒नां ॅवृ॑ष॒भाजा॑यथा॒ अजा॑यथा वृषभ चर्.षणी॒नाम् । \newline
36. वृ॒ष॒भ॒ च॒र्॒.ष॒णी॒नाम् च॑र्.षणी॒नां ॅवृ॑षभ वृषभ चर्.षणी॒नाम् । \newline
37. च॒र्॒.ष॒णी॒नामिति॑ चर्.षणी॒नाम् । \newline
38. अपा॑नुदो ऽनुदो॒ अपापा॑नुदो॒ जन॒म् जन॑ मनुदो॒ अपापा॑नुदो॒ जन᳚म् । \newline
39. अ॒नु॒दो॒ जन॒म् जन॑ मनुदो ऽनुदो॒ जन॑ ममित्र॒यन्त॑ ममित्र॒यन्त॒म् जन॑ मनुदो ऽनुदो॒ जन॑ ममित्र॒यन्त᳚म् । \newline
40. जन॑ ममित्र॒यन्त॑ ममित्र॒यन्त॒म् जन॒म् जन॑ ममित्र॒यन्त॑ मु॒रु मु॒रु म॑मित्र॒यन्त॒म् जन॒म् जन॑ ममित्र॒यन्त॑ मु॒रुम् । \newline
41. अ॒मि॒त्र॒यन्त॑ मु॒रु मु॒रु म॑मित्र॒यन्त॑ ममित्र॒यन्त॑ मु॒रुम् दे॒वेभ्यो॑ दे॒वेभ्य॑ उ॒रु म॑मित्र॒यन्त॑ ममित्र॒यन्त॑ मु॒रुम् दे॒वेभ्यः॑ । \newline
42. अ॒मि॒त्र॒यन्त॒मित्य॑मित्र - यन्त᳚म् । \newline
43. उ॒रुम् दे॒वेभ्यो॑ दे॒वेभ्य॑ उ॒रु मु॒रुम् दे॒वेभ्यो॑ अकृणो रकृणोर् दे॒वेभ्य॑ उ॒रु मु॒रुम् दे॒वेभ्यो॑ अकृणोः । \newline
44. दे॒वेभ्यो॑ अकृणो रकृणोर् दे॒वेभ्यो॑ दे॒वेभ्यो॑ अकृणोरु वु वकृणोर् दे॒वेभ्यो॑ दे॒वेभ्यो॑ अकृणोरु । \newline
45. अ॒कृ॒णो॒रु॒ वु॒ व॒कृ॒णो॒ र॒कृ॒णो॒रु॒ लो॒कम् ॅलो॒क मु॑ वकृणो रकृणोरु लो॒कम् । \newline
46. उ॒ लो॒कम् ॅलो॒क मु॑ वु लो॒कम् । \newline
47. लो॒कमिति॑ लो॒कम् । \newline
48. मृ॒गो न न मृ॒गो मृ॒गो न भी॒मो भी॒मो न मृ॒गो मृ॒गो न भी॒मः । \newline
49. न भी॒मो भी॒मो न न भी॒मः कु॑च॒रः कु॑च॒रो भी॒मो न न भी॒मः कु॑च॒रः । \newline
50. भी॒मः कु॑च॒रः कु॑च॒रो भी॒मो भी॒मः कु॑च॒रो गि॑रि॒ष्ठा गि॑रि॒ष्ठाः कु॑च॒रो भी॒मो भी॒मः कु॑च॒रो गि॑रि॒ष्ठाः । \newline
51. कु॒च॒रो गि॑रि॒ष्ठा गि॑रि॒ष्ठाः कु॑च॒रः कु॑च॒रो गि॑रि॒ष्ठाः प॑रा॒वतः॑ परा॒वतो॑ गिरि॒ष्ठाः कु॑च॒रः कु॑च॒रो गि॑रि॒ष्ठाः प॑रा॒वतः॑ । \newline
52. गि॒रि॒ष्ठाः प॑रा॒वतः॑ परा॒वतो॑ गिरि॒ष्ठा गि॑रि॒ष्ठाः प॑रा॒वत॒ आ प॑रा॒वतो॑ गिरि॒ष्ठा गि॑रि॒ष्ठाः प॑रा॒वत॒ आ । \newline
53. गि॒रि॒ष्ठा इति॑ गिरि - स्थाः । \newline
54. प॒रा॒वत॒ आ प॑रा॒वतः॑ परा॒वत॒ आ ज॑गाम जगा॒मा प॑रा॒वतः॑ परा॒वत॒ आ ज॑गाम । \newline
55. प॒रा॒वत॒ इति॑ परा - वतः॑ । \newline
\pagebreak
\markright{ TS 1.6.12.5  \hfill https://www.vedavms.in \hfill}

\section{ TS 1.6.12.5 }

\textbf{TS 1.6.12.5 } \newline
\textbf{Samhita Paata} \newline

आ ज॑गामा॒ पर॑स्याः । सृ॒कꣳ सꣳ॒॒शाय॑ प॒विमि॑न्द्र ति॒ग्मं ॅवि शत्रू᳚न् ताढि॒ विमृधो॑ नुदस्व ॥ वि शत्रू॒न्॒. वि मृधो॑ नुद॒ विवृ॒त्रस्य॒ हनू॑ रुज । वि म॒न्युमि॑न्द्र भामि॒तो॑ऽमित्र॑स्याऽभि॒दास॑तः ॥ त्रा॒तार॒-मिन्द्र॑-मवि॒तार॒-मिन्द्रꣳ॒॒ हवे॑हवे सु॒हवꣳ॒॒ शूर॒मिन्द्रं᳚ । हु॒वे नु श॒क्रं पु॑रुहू॒तमिन्द्रꣳ॑ स्व॒स्ति नो॑ म॒घवा॑ धा॒त्विन्द्रः॑ ॥ मा ते॑ अ॒स्याꣳ - [ ] \newline

\textbf{Pada Paata} \newline

एति॑ । ज॒गा॒म॒ । पर॑स्याः ॥ सृ॒कम् । सꣳ॒॒शायेति॑ सं-शाय॑ । प॒विम् । इ॒न्द्र॒ । ति॒ग्मम् । वीति॑ । शत्रून्॑ । ता॒ढि॒ । वीति॑ । मृधः॑ । नु॒द॒स्व॒ ॥ वीति॑ । शत्रून्॑ । वीति॑ । मृधः॑ । नु॒द॒ । वीति॑ । वृ॒त्रस्य॑ । हनू॒ इति॑ । रु॒ज॒ ॥ वीति॑ । म॒न्युम् । इ॒न्द्र॒ । भा॒मि॒तः । अ॒मित्र॑स्य । अ॒भि॒दास॑त॒ इत्य॑भि - दास॑तः ॥ त्रा॒तार᳚म् । इन्द्र᳚म् । अ॒वि॒तार᳚म् । इन्द्र᳚म् । हवे॑हव॒ इति॒ हवे᳚ - ह॒वे॒ । सु॒हव॒मिति॑ सु - हव᳚म् । शूर᳚म् । इन्द्र᳚म् ॥ हु॒वे । नु । श॒क्रम् । पु॒रु॒हू॒तमिति॑ पुरु - हू॒तम् । इन्द्र᳚म् । स्व॒स्ति । नः॒ । म॒घवेति॑ म॒घ - वा॒ । धा॒तु॒ । इन्द्रः॑ ॥ मा । ते॒ । अ॒स्याम् ।  \newline


\textbf{Krama Paata} \newline

आ ज॑गाम । ज॒गा॒मा॒ पर॑स्याः । पर॑स्या॒ इति॒ पर॑स्याः ॥ सृ॒कꣳ सꣳ॒॒शाय॑ । सꣳ॒॒शाय॑ प॒विम् । सꣳ॒॒शायेति॑ सम् - शाय॑ । प॒विमि॑न्द्र । इ॒न्द्र॒ ति॒ग्मम् । ति॒ग्मं ॅवि । वि शत्रून्॑ । शत्रू᳚न् ताढि । ता॒ढि॒ वि । वि मृधः॑ । मृधो॑ नुदस्व । नु॒द॒स्वेति॑ नुदस्व ॥ वि शत्रून्॑ । शत्रू॒न्॒. वि । वि मृधः॑ । मृधो॑ नुद । नु॒द॒ वि । वि वृ॒त्रस्य॑ । वृ॒त्रस्य॒ हनू᳚ । हनू॑ रुज । हनू॒ इति॒ हनू᳚ । रु॒जेति॑ रुज ॥ वि म॒न्युम् । म॒न्युमि॑न्द्र । इ॒न्द्र॒ भा॒मि॒तः । भा॒मि॒तो॑ऽमित्र॑स्य । अ॒मित्र॑स्याभि॒दास॑तः । अ॒भि॒दास॑त॒ इत्य॑भि - दास॑तः ॥ त्रा॒तार॒मिन्द्र᳚म् । इन्द्र॑मवि॒तार᳚म् । अ॒वि॒तार॒मिन्द्र᳚म् । इन्द्रꣳ॒॒ हवे॑हवे । हवे॑हवे सु॒हव᳚म् । हवे॑हव॒ इति॒ हवे᳚ - ह॒वे॒ । सु॒हवꣳ॒॒ शूर᳚म् । सु॒हव॒मिति॑ सु - हव᳚म् । शूर॒मिन्द्र᳚म् । इन्द्र॒मितीन्द्र᳚म् ॥ हु॒वे नु । नु श॒क्रम् । श॒क्रम् पु॑रुहू॒तम् । पु॒रु॒हू॒तमिन्द्र᳚म् । पु॒रु॒हू॒तमिति॑ पुरु - हू॒तम् । इन्द्रꣳ॑ स्व॒स्ति । स्व॒स्ति नः॑ । नो॒ म॒घवा᳚ । म॒घवा॑ धातु । म॒घवेति॑ म॒घ - वा॒ । धा॒त्विन्द्रः॑ । इन्द्र॒ इतीन्द्रः॑ ॥ मा ते᳚ । ते॒ अ॒स्याम् । अ॒स्याꣳ स॑हसावन्न् \newline

\textbf{Jatai Paata} \newline

1. आ ज॑गाम जगा॒मा ज॑गाम । \newline
2. ज॒गा॒मा॒ पर॑स्याः॒ पर॑स्या जगाम जगामा॒ पर॑स्याः । \newline
3. पर॑स्या॒ इति॒ पर॑स्याः । \newline
4. सृ॒कꣳ स॒(ग्म्॒)शाय॑ स॒(ग्म्॒)शाय॑ सृ॒कꣳ सृ॒कꣳ स॒(ग्म्॒)शाय॑ । \newline
5. स॒(ग्म्॒)शाय॑ प॒विम् प॒विꣳ स॒(ग्म्॒)शाय॑ स॒(ग्म्॒)शाय॑ प॒विम् । \newline
6. स॒(ग्म्॒)शायेति॑ सं - शाय॑ । \newline
7. प॒वि मि॑न्द्रे न्द्र प॒विम् प॒वि मि॑न्द्र । \newline
8. इ॒न्द्र॒ ति॒ग्मम् ति॒ग्म मि॑न्द्रे न्द्र ति॒ग्मम् । \newline
9. ति॒ग्मं ॅवि वि ति॒ग्मम् ति॒ग्मं ॅवि । \newline
10. वि शत्रू॒ञ् छत्रू॒न्॒. वि वि शत्रून्॑ । \newline
11. शत्रू᳚न् ताढि ताढि॒ शत्रू॒ञ् छत्रू᳚न् ताढि । \newline
12. ता॒ढि॒ वि वि ता॑ढि ताढि॒ वि । \newline
13. वि मृधो॒ मृधो॒ वि वि मृधः॑ । \newline
14. मृधो॑ नुदस्व नुदस्व॒ मृधो॒ मृधो॑ नुदस्व । \newline
15. नु॒द॒स्वेति॑ नुदस्व । \newline
16. वि शत्रू॒ञ् छत्रू॒न्॒. वि वि शत्रून्॑ । \newline
17. शत्रू॒न्॒. वि वि शत्रू॒ञ् छत्रू॒न्॒. वि । \newline
18. वि मृधो॒ मृधो॒ वि वि मृधः॑ । \newline
19. मृधो॑ नुद नुद॒ मृधो॒ मृधो॑ नुद । \newline
20. नु॒द॒ वि वि नु॑द नुद॒ वि । \newline
21. वि वृ॒त्रस्य॑ वृ॒त्रस्य॒ वि वि वृ॒त्रस्य॑ । \newline
22. वृ॒त्रस्य॒ हनू॒ हनू॑ वृ॒त्रस्य॑ वृ॒त्रस्य॒ हनू᳚ । \newline
23. हनू॑ रुज रुज॒ हनू॒ हनू॑ रुज । \newline
24. हनू॒ इति॒ हनू᳚ । \newline
25. रु॒जेति॑ रुज । \newline
26. वि म॒न्युम् म॒न्युं ॅवि वि म॒न्युम् । \newline
27. म॒न्यु मि॑न्द्रे न्द्र म॒न्युम् म॒न्यु मि॑न्द्र । \newline
28. इ॒न्द्र॒ भा॒मि॒तो भा॑मि॒त इ॑न्द्रे न्द्र भामि॒तः । \newline
29. भा॒मि॒तो॑ ऽमित्र॑स्या॒ मित्र॑स्य भामि॒तो भा॑मि॒तो॑ ऽमित्र॑स्य । \newline
30. अ॒मित्र॑स्या भि॒दास॑तो अभि॒दास॑तो अ॒मित्र॑स्या॒ मित्र॑स्या भि॒दास॑तः । \newline
31. अ॒भि॒दास॑त॒ इत्य॑भि - दास॑तः । \newline
32. त्रा॒तार॒ मिन्द्र॒ मिन्द्र॑म् त्रा॒तार॑म् त्रा॒तार॒ मिन्द्र᳚म् । \newline
33. इन्द्र॑ मवि॒तार॑ मवि॒तार॒ मिन्द्र॒ मिन्द्र॑ मवि॒तार᳚म् । \newline
34. अ॒वि॒तार॒ मिन्द्र॒ मिन्द्र॑ मवि॒तार॑ मवि॒तार॒ मिन्द्र᳚म् । \newline
35. इन्द्र॒(ग्म्॒) हवे॑हवे॒ हवे॑हव॒ इन्द्र॒ मिन्द्र॒(ग्म्॒) हवे॑हवे । \newline
36. हवे॑हवे सु॒हव(ग्म्॑) सु॒हव॒(ग्म्॒) हवे॑हवे॒ हवे॑हवे सु॒हव᳚म् । \newline
37. हवे॑हव॒ इति॒ हवे᳚ - ह॒वे॒ । \newline
38. सु॒हव॒(ग्म्॒) शूर॒(ग्म्॒) शूर(ग्म्॑) सु॒हव(ग्म्॑) सु॒हव॒(ग्म्॒) शूर᳚म् । \newline
39. सु॒हव॒मिति॑ सु - हव᳚म् । \newline
40. शूर॒ मिन्द्र॒ मिन्द्र॒(ग्म्॒) शूर॒(ग्म्॒) शूर॒ मिन्द्र᳚म् । \newline
41. इन्द्र॒मितीन्द्र᳚म् । \newline
42. हु॒वे नु नु हु॒वे हु॒वे नु । \newline
43. नु श॒क्रꣳ श॒क्रम् नु नु श॒क्रम् । \newline
44. श॒क्रम् पु॑रुहू॒तम् पु॑रुहू॒तꣳ श॒क्रꣳ श॒क्रम् पु॑रुहू॒तम् । \newline
45. पु॒रु॒हू॒त मिन्द्र॒ मिन्द्र॑म् पुरुहू॒तम् पु॑रुहू॒त मिन्द्र᳚म् । \newline
46. पु॒रु॒हू॒तमिति॑ पुरु - हू॒तम् । \newline
47. इन्द्र(ग्ग्॑) स्व॒स्ति स्व॒स्तीन्द्र॒ मिन्द्र(ग्ग्॑) स्व॒स्ति । \newline
48. स्व॒स्ति नो॑ नः स्व॒स्ति स्व॒स्ति नः॑ । \newline
49. नो॒ म॒घवा॑ म॒घवा॑ नो नो म॒घवा᳚ । \newline
50. म॒घवा॑ धातु धातु म॒घवा॑ म॒घवा॑ धातु । \newline
51. म॒घवेति॑ म॒घ - वा॒ । \newline
52. धा॒त्विन्द्र॒ इन्द्रो॑ धातु धा॒त्विन्द्रः॑ । \newline
53. इन्द्र॒ इतीन्द्रः॑ । \newline
54. मा ते॑ ते॒ मा मा ते᳚ । \newline
55. ते॒ अ॒स्या म॒स्याम् ते॑ ते अ॒स्याम् । \newline
56. अ॒स्याꣳ स॑हसावन् थ्सहसावन् न॒स्या म॒स्याꣳ स॑हसावन्न् । \newline

\textbf{Ghana Paata } \newline

1. आ ज॑गाम जगा॒मा ज॑गामा॒ पर॑स्याः॒ पर॑स्या जगा॒मा ज॑गामा॒ पर॑स्याः । \newline
2. ज॒गा॒मा॒ पर॑स्याः॒ पर॑स्या जगाम जगामा॒ पर॑स्याः । \newline
3. पर॑स्या॒ इति॒ पर॑स्याः । \newline
4. सृ॒कꣳ स॒(ग्म्॒)शाय॑ स॒(ग्म्॒)शाय॑ सृ॒कꣳ सृ॒कꣳ स॒(ग्म्॒)शाय॑ प॒विम् प॒विꣳ स॒(ग्म्॒)शाय॑ सृ॒कꣳ सृ॒कꣳ स॒(ग्म्॒)शाय॑ प॒विम् । \newline
5. स॒(ग्म्॒)शाय॑ प॒विम् प॒विꣳ स॒(ग्म्॒)शाय॑ स॒(ग्म्॒)शाय॑ प॒वि मि॑न्द्रे न्द्र प॒विꣳ स॒(ग्म्॒)शाय॑ स॒(ग्म्॒)शाय॑ प॒वि मि॑न्द्र । \newline
6. स॒(ग्म्॒)शायेति॑ सं - शाय॑ । \newline
7. प॒वि मि॑न्द्रे न्द्र प॒विम् प॒वि मि॑न्द्र ति॒ग्मम् ति॒ग्म मि॑न्द्र प॒विम् प॒वि मि॑न्द्र ति॒ग्मम् । \newline
8. इ॒न्द्र॒ ति॒ग्मम् ति॒ग्म मि॑न्द्रे न्द्र ति॒ग्मं ॅवि वि ति॒ग्म मि॑न्द्रे न्द्र ति॒ग्मं ॅवि । \newline
9. ति॒ग्मं ॅवि वि ति॒ग्मम् ति॒ग्मं ॅवि शत्रू॒ञ् छत्रू॒न्॒. वि ति॒ग्मम् ति॒ग्मं ॅवि शत्रून्॑ । \newline
10. वि शत्रू॒ञ् छत्रू॒न्॒. वि वि शत्रू᳚न् ताढि ताढि॒ शत्रू॒न्॒. वि वि शत्रू᳚न् ताढि । \newline
11. शत्रू᳚न् ताढि ताढि॒ शत्रू॒ञ् छत्रू᳚न् ताढि॒ वि वि ता॑ढि॒ शत्रू॒ञ् छत्रू᳚न् ताढि॒ वि । \newline
12. ता॒ढि॒ वि वि ता॑ढि ताढि॒ वि मृधो॒ मृधो॒ वि ता॑ढि ताढि॒ वि मृधः॑ । \newline
13. वि मृधो॒ मृधो॒ वि वि मृधो॑ नुदस्व नुदस्व॒ मृधो॒ वि वि मृधो॑ नुदस्व । \newline
14. मृधो॑ नुदस्व नुदस्व॒ मृधो॒ मृधो॑ नुदस्व । \newline
15. नु॒द॒स्वेति॑ नुदस्व । \newline
16. वि शत्रू॒ञ् छत्रू॒न्॒. वि वि शत्रू॒न्॒. वि वि शत्रू॒न्॒. वि वि शत्रू॒न्॒. वि । \newline
17. शत्रू॒न्॒. वि वि शत्रू॒ञ् छत्रू॒न्॒. वि मृधो॒ मृधो॒ वि शत्रू॒ञ् छत्रू॒न्॒. वि मृधः॑ । \newline
18. वि मृधो॒ मृधो॒ वि वि मृधो॑ नुद नुद॒ मृधो॒ वि वि मृधो॑ नुद । \newline
19. मृधो॑ नुद नुद॒ मृधो॒ मृधो॑ नुद॒ वि वि नु॑द॒ मृधो॒ मृधो॑ नुद॒ वि । \newline
20. नु॒द॒ वि वि नु॑द नुद॒ वि वृ॒त्रस्य॑ वृ॒त्रस्य॒ वि नु॑द नुद॒ वि वृ॒त्रस्य॑ । \newline
21. वि वृ॒त्रस्य॑ वृ॒त्रस्य॒ वि वि वृ॒त्रस्य॒ हनू॒ हनू॑ वृ॒त्रस्य॒ वि वि वृ॒त्रस्य॒ हनू᳚ । \newline
22. वृ॒त्रस्य॒ हनू॒ हनू॑ वृ॒त्रस्य॑ वृ॒त्रस्य॒ हनू॑ रुज रुज॒ हनू॑ वृ॒त्रस्य॑ वृ॒त्रस्य॒ हनू॑ रुज । \newline
23. हनू॑ रुज रुज॒ हनू॒ हनू॑ रुज । \newline
24. हनू॒ इति॒ हनू᳚ । \newline
25. रु॒जेति॑ रुज । \newline
26. वि म॒न्युम् म॒न्युं ॅवि वि म॒न्यु मि॑न्द्रे न्द्र म॒न्युं ॅवि वि म॒न्यु मि॑न्द्र । \newline
27. म॒न्यु मि॑न्द्रे न्द्र म॒न्युम् म॒न्यु मि॑न्द्र भामि॒तो भा॑मि॒त इ॑न्द्र म॒न्युम् म॒न्यु मि॑न्द्र भामि॒तः । \newline
28. इ॒न्द्र॒ भा॒मि॒तो भा॑मि॒त इ॑न्द्रे न्द्र भामि॒तो॑ ऽमित्र॑स्या॒मित्र॑स्य भामि॒त इ॑न्द्रे न्द्र भामि॒तो॑ ऽमित्र॑स्य । \newline
29. भा॒मि॒तो॑ ऽमित्र॑स्या॒मित्र॑स्य भामि॒तो भा॑मि॒तो॑ ऽमित्र॑स्या भि॒दास॑तो अभि॒दास॑तो अ॒मित्र॑स्य भामि॒तो भा॑मि॒तो॑ ऽमित्र॑स्या भि॒दास॑तः । \newline
30. अ॒मित्र॑स्या भि॒दास॑तो अभि॒दास॑तो अ॒मित्र॑स्या॒ मित्र॑स्या भि॒दास॑तः । \newline
31. अ॒भि॒दास॑त॒ इत्य॑भि - दास॑तः । \newline
32. त्रा॒तार॒ मिन्द्र॒ मिन्द्र॑म् त्रा॒तार॑म् त्रा॒तार॒ मिन्द्र॑ मवि॒तार॑ मवि॒तार॒ मिन्द्र॑म् त्रा॒तार॑म् त्रा॒तार॒ मिन्द्र॑ मवि॒तार᳚म् । \newline
33. इन्द्र॑ मवि॒तार॑ मवि॒तार॒ मिन्द्र॒ मिन्द्र॑ मवि॒तार॒ मिन्द्र॒ मिन्द्र॑ मवि॒तार॒ मिन्द्र॒ मिन्द्र॑ मवि॒तार॒ मिन्द्र᳚म् । \newline
34. अ॒वि॒तार॒ मिन्द्र॒ मिन्द्र॑ मवि॒तार॑ मवि॒तार॒ मिन्द्र॒(ग्म्॒) हवे॑हवे॒ हवे॑हव॒ इन्द्र॑ मवि॒तार॑ मवि॒तार॒ मिन्द्र॒(ग्म्॒) हवे॑हवे । \newline
35. इन्द्र॒(ग्म्॒) हवे॑हवे॒ हवे॑हव॒ इन्द्र॒ मिन्द्र॒(ग्म्॒) हवे॑हवे सु॒हव(ग्म्॑) सु॒हव॒(ग्म्॒) हवे॑हव॒ इन्द्र॒ मिन्द्र॒(ग्म्॒) हवे॑हवे सु॒हव᳚म् । \newline
36. हवे॑हवे सु॒हव(ग्म्॑) सु॒हव॒(ग्म्॒) हवे॑हवे॒ हवे॑हवे सु॒हव॒(ग्म्॒) शूर॒(ग्म्॒) शूर(ग्म्॑) सु॒हव॒(ग्म्॒) हवे॑हवे॒ हवे॑हवे सु॒हव॒(ग्म्॒) शूर᳚म् । \newline
37. हवे॑हव॒ इति॒ हवे᳚ - ह॒वे॒ । \newline
38. सु॒हव॒(ग्म्॒) शूर॒(ग्म्॒) शूर(ग्म्॑) सु॒हव(ग्म्॑) सु॒हव॒(ग्म्॒) शूर॒ मिन्द्र॒ मिन्द्र॒(ग्म्॒) शूर(ग्म्॑) सु॒हव(ग्म्॑) सु॒हव॒(ग्म्॒) शूर॒ मिन्द्र᳚म् । \newline
39. सु॒हव॒मिति॑ सु - हव᳚म् । \newline
40. शूर॒ मिन्द्र॒ मिन्द्र॒(ग्म्॒) शूर॒(ग्म्॒) शूर॒ मिन्द्र᳚म् । \newline
41. इन्द्र॒मितीन्द्र᳚म् । \newline
42. हु॒वे नु नु हु॒वे हु॒वे नु श॒क्रꣳ श॒क्रन्नु हु॒वे हु॒वे नु श॒क्रम् । \newline
43. नु श॒क्रꣳ श॒क्रम् नु नु श॒क्रम् पु॑रुहू॒तम् पु॑रुहू॒तꣳ श॒क्रम् नु नु श॒क्रम् पु॑रुहू॒तम् । \newline
44. श॒क्रम् पु॑रुहू॒तम् पु॑रुहू॒तꣳ श॒क्रꣳ श॒क्रम् पु॑रुहू॒त मिन्द्र॒ मिन्द्र॑म् पुरुहू॒तꣳ श॒क्रꣳ श॒क्रम् पु॑रुहू॒त मिन्द्र᳚म् । \newline
45. पु॒रु॒हू॒त मिन्द्र॒ मिन्द्र॑म् पुरुहू॒तम् पु॑रुहू॒त मिन्द्र(ग्ग्॑) स्व॒स्ति स्व॒स्तीन्द्र॑म् पुरुहू॒तम् पु॑रुहू॒त मिन्द्र(ग्ग्॑) स्व॒स्ति । \newline
46. पु॒रु॒हू॒तमिति॑ पुरु - हू॒तम् । \newline
47. इन्द्र(ग्ग्॑) स्व॒स्ति स्व॒स्तीन्द्र॒ मिन्द्र(ग्ग्॑) स्व॒स्ति नो॑ नः स्व॒स्तीन्द्र॒ मिन्द्र(ग्ग्॑) स्व॒स्ति नः॑ । \newline
48. स्व॒स्ति नो॑ नः स्व॒स्ति स्व॒स्ति नो॑ म॒घवा॑ म॒घवा॑ नः स्व॒स्ति स्व॒स्ति नो॑ म॒घवा᳚ । \newline
49. नो॒ म॒घवा॑ म॒घवा॑ नो नो म॒घवा॑ धातु धातु म॒घवा॑ नो नो म॒घवा॑ धातु । \newline
50. म॒घवा॑ धातु धातु म॒घवा॑ म॒घवा॑ धा॒त्विन्द्र॒ इन्द्रो॑ धातु म॒घवा॑ म॒घवा॑ धा॒त्विन्द्रः॑ । \newline
51. म॒घवेति॑ म॒घ - वा॒ । \newline
52. धा॒त्विन्द्र॒ इन्द्रो॑ धातु धा॒त्विन्द्रः॑ । \newline
53. इन्द्र॒ इतीन्द्रः॑ । \newline
54. मा ते॑ ते॒ मा मा ते॑ अ॒स्या म॒स्याम् ते॒ मा मा ते॑ अ॒स्याम् । \newline
55. ते॒ अ॒स्या म॒स्याम् ते॑ ते अ॒स्याꣳ स॑हसावन् थ्सहसावन् न॒स्याम् ते॑ ते अ॒स्याꣳ स॑हसावन्न् । \newline
56. अ॒स्याꣳ स॑हसावन् थ्सहसावन् न॒स्या म॒स्याꣳ स॑हसाव॒न् परि॑ष्टौ॒ परि॑ष्टौ सहसावन् न॒स्या म॒स्याꣳ स॑हसाव॒न् परि॑ष्टौ । \newline
\pagebreak
\markright{ TS 1.6.12.6  \hfill https://www.vedavms.in \hfill}

\section{ TS 1.6.12.6 }

\textbf{TS 1.6.12.6 } \newline
\textbf{Samhita Paata} \newline

स॑हसाव॒न् परि॑ष्टाव॒घाय॑ भूम हरिवः परा॒दै । त्राय॑स्व नो ऽवृ॒केभि॒र् वरू॑थै॒-स्तव॑ प्रि॒यासः॑ सू॒रिषु॑ स्याम ॥ अन॑वस्ते॒ रथ॒मश्वा॑य तक्ष॒न् त्वष्टा॒ वज्रं॑ पुरुहूत द्यु॒मन्तं᳚ । ब्र॒ह्माण॒ इन्द्रं॑ म॒हय॑न्तो अ॒र्कैरव॑र्द्धय॒न्नह॑ये॒ हन्त॒वा उ॑ ॥ वृष्णे॒ यत् ते॒ वृष॑णो अ॒र्कमर्चा॒निन्द्र॒ ग्रावा॑णो॒ अदि॑तिः स॒जोषाः᳚ । अ॒न॒श्वासो॒ ये प॒वयो॑ऽर॒था इन्द्रे॑षिता अ॒भ्यव॑र्त्त न्त॒॒ दस्यून्॑ ॥ \newline

\textbf{Pada Paata} \newline

स॒ह॒सा॒व॒न्निति॑ सहसा - व॒न्न् । परि॑ष्टौ । अ॒घाय॑ । भू॒म॒ । ह॒रि॒व॒ इति॑ हरि - वः॒ । प॒रा॒दा इति॑ परा - दै ॥ त्राय॑स्व । नः॒ । अ॒वृ॒केभिः॑ । वरू॑थैः । तव॑ । प्रि॒यासः॑ । सू॒रिषु॑ । स्या॒म॒ ॥ अन॑वः । ते॒ । रथ᳚म् । अश्वा॑य । त॒क्ष॒न्न् । त्वष्टा᳚ । वज्र᳚म् । पु॒रु॒हू॒तेति॑ पुरु - हू॒त॒ । द्यु॒मन्त॒मिति॑ द्यु - मन्त᳚म् ॥ ब्र॒ह्माणः॑ । इन्द्र᳚म् । म॒हय॑न्तः । अ॒र्कैः । अव॑र्धयन्न् । अह॑ये । हन्त॒वै । उ॒ ॥ वृष्णे᳚ । यत् । ते॒ । वृष॑णः । अ॒र्कम् । अर्चान्॑ । इन्द्र॑ । ग्रावा॑णः । अदि॑तिः । स॒जोषा॒ इति॑ स - जोषाः᳚ ॥ अ॒न॒श्वासः॑ । ये । प॒वयः॑ । अ॒र॒थाः । इन्द्रे॑षिता॒ इतीन्द्र॑ - इ॒षि॒ताः॒ । अ॒भ्यव॑र्त॒न्तेत्य॑भि - अव॑र्तन्त । दस्यून्॑ ॥  \newline


\textbf{Krama Paata} \newline

स॒ह॒सा॒व॒न्,परि॑ष्टौ । स॒ह॒सा॒व॒न्निति॑ सहसा - व॒न्न्॒ । परि॑ष्टाव॒घाय॑ । अ॒घाय॑ भूम । भू॒म॒ ह॒रि॒वः॒ । ह॒रि॒वः॒ प॒रा॒दै । ह॒रि॒व॒ इति॑ हरि - वः॒ । प॒रा॒दा इति॑ परा - दै ॥ त्राय॑स्व नः । नो॒ ऽवृ॒केभिः॑ । अ॒वृ॒केभि॒र् वरू॑थैः । वरू॑थै॒स्तव॑ । तव॑ प्रि॒यासः॑ । प्रि॒यासः॑ सू॒रिषु॑ । सू॒रिषु॑ स्याम । स्या॒मेति॑ स्याम ॥ अन॑वस्ते । ते॒ रथ᳚म् । रथ॒मश्वा॑य । अश्वा॑य तक्षन्न् । त॒क्ष॒न् त्वष्टा᳚ । त्वष्टा॒ वज्र᳚म् । वज्र॑म् पुरुहूत । पु॒रु॒हू॒त॒ द्यु॒मन्त᳚म् । पु॒रु॒हू॒तेति॑ पुरु - हू॒त॒ । द्यु॒मन्त॒मिति॑ द्यु - मन्त᳚म् ॥ ब्र॒ह्माण॒ इन्द्र᳚म् । इन्द्र॑म् म॒हय॑न्तः । म॒हय॑न्तो अ॒र्कैः । अ॒र्कैरव॑र्द्धयन्न् । अव॑र्द्धय॒न्नह॑ये । अह॑ये॒ हन्त॒वै । हन्त॒वा उ॑ । उ॒वित्यु॑ ॥ वृष्णे॒ यत् । यत् ते᳚ । ते॒ वृष॑णः । वृष॑णो अ॒र्कम् । अ॒र्कमर्चान्॑ । अर्चा॒निन्द्र॑ । इन्द्र॒ ग्रावा॑णः । ग्रावा॑णो॒ अदि॑तिः । अदि॑तिः स॒जोषाः᳚ । स॒जोषा॒ इति॑ स - जोषाः᳚ ॥ अ॒न॒श्वासो॒ ये । ये प॒वयः॑ । प॒वयो॑ऽर॒थाः । अ॒र॒था इन्द्रे॑षिताः । इन्द्रे॑षिता अ॒भ्यव॑र्तन्त । इन्द्रे॑षिता॒ इतीन्द्र॑ - इ॒षि॒ताः॒ । अ॒भ्यव॑र्तन्त॒ दस्यून्न्॑ । अ॒भ्यव॑र्त॒न्तेत्य॑भि - अव॑र्तन्त । दस्यू॒निति॒ दस्यून्न्॑ । \newline

\textbf{Jatai Paata} \newline

1. स॒ह॒सा॒व॒न् परि॑ष्टौ॒ परि॑ष्टौ सहसावन् थ्सहसाव॒न् परि॑ष्टौ । \newline
2. स॒ह॒सा॒व॒न्निति॑ सहसा - व॒न्न् । \newline
3. परि॑ष्टा व॒घाया॒ घाय॒ परि॑ष्टौ॒ परि॑ष्टा व॒घाय॑ । \newline
4. अ॒घाय॑ भूम भूमा॒घाया॒ घाय॑ भूम । \newline
5. भू॒म॒ ह॒रि॒वो॒ ह॒रि॒वो॒ भू॒म॒ भू॒म॒ ह॒रि॒वः॒ । \newline
6. ह॒रि॒वः॒ प॒रा॒दै प॑रा॒दै ह॑रिवो हरिवः परा॒दै । \newline
7. ह॒रि॒व॒ इति॑ हरि - वः॒ । \newline
8. प॒रा॒दा इति॑ परा - दै । \newline
9. त्राय॑स्व नो न॒स्त्राय॑स्व॒ त्राय॑स्व नः । \newline
10. नो॒ ऽवृ॒केभि॑ रवृ॒केभि॑र् नो नो ऽवृ॒केभिः॑ । \newline
11. अ॒वृ॒केभि॒र् वरू॑थै॒र् वरू॑थै रवृ॒केभि॑ रवृ॒केभि॒र् वरू॑थैः । \newline
12. वरू॑थै॒ स्तव॒ तव॒ वरू॑थै॒र् वरू॑थै॒ स्तव॑ । \newline
13. तव॑ प्रि॒यासः॑ प्रि॒यास॒ स्तव॒ तव॑ प्रि॒यासः॑ । \newline
14. प्रि॒यासः॑ सू॒रिषु॑ सू॒रिषु॑ प्रि॒यासः॑ प्रि॒यासः॑ सू॒रिषु॑ । \newline
15. सू॒रिषु॑ स्याम स्याम सू॒रिषु॑ सू॒रिषु॑ स्याम । \newline
16. स्या॒मेति॑ स्याम । \newline
17. अन॑व स्ते॒ ते ऽन॒वो ऽन॑वस्ते । \newline
18. ते॒ रथ॒(ग्म्॒) रथ॑म् ते ते॒ रथ᳚म् । \newline
19. रथ॒ मश्वा॒या श्वा॑य॒ रथ॒(ग्म्॒) रथ॒ मश्वा॑य । \newline
20. अश्वा॑य तक्षन् तक्ष॒न् नश्वा॒या श्वा॑य तक्षन्न् । \newline
21. त॒क्ष॒न् त्वष्टा॒ त्वष्टा॑ तक्षन् तक्ष॒न् त्वष्टा᳚ । \newline
22. त्वष्टा॒ वज्रं॒ ॅवज्र॒म् त्वष्टा॒ त्वष्टा॒ वज्र᳚म् । \newline
23. वज्र॑म् पुरुहूत पुरुहूत॒ वज्रं॒ ॅवज्र॑म् पुरुहूत । \newline
24. पु॒रु॒हू॒त॒ द्यु॒मन्त॑म् द्यु॒मन्त॑म् पुरुहूत पुरुहूत द्यु॒मन्त᳚म् । \newline
25. पु॒रु॒हू॒तेति॑ पुरु - हू॒त॒ । \newline
26. द्यु॒मन्त॒मिति॑ द्यु - मन्त᳚म् । \newline
27. ब्र॒ह्माण॒ इन्द्र॒ मिन्द्र॑म् ब्र॒ह्माणो᳚ ब्र॒ह्माण॒ इन्द्र᳚म् । \newline
28. इन्द्र॑म् म॒हय॑न्तो म॒हय॑न्त॒ इन्द्र॒ मिन्द्र॑म् म॒हय॑न्तः । \newline
29. म॒हय॑न्तो अ॒र्कै र॒र्कैर् म॒हय॑न्तो म॒हय॑न्तो अ॒र्कैः । \newline
30. अ॒र्कै रव॑र्धय॒न् नव॑र्धयन् न॒र्कै र॒र्कै रव॑र्धयन्न् । \newline
31. अव॑र्धय॒न् नह॒ये ऽह॒ये ऽव॑र्धय॒न् नव॑र्धय॒न् नह॑ये । \newline
32. अह॑ये॒ हन्त॒वै हन्त॒वा अह॒ये ऽह॑ये॒ हन्त॒वै । \newline
33. हन्त॒वा उ॑ वु॒ हन्त॒वै हन्त॒वा उ॑ । \newline
34. उ॒ वित्यु॑ । \newline
35. वृष्णे॒ यद् यद् वृष्णे॒ वृष्णे॒ यत् । \newline
36. यत् ते॑ ते॒ यद् यत् ते᳚ । \newline
37. ते॒ वृष॑णो॒ वृष॑ण स्ते ते॒ वृष॑णः । \newline
38. वृष॑णो अ॒र्क म॒र्कं ॅवृष॑णो॒ वृष॑णो अ॒र्कम् । \newline
39. अ॒र्क मर्चा॒ नर्चा॑ न॒र्क म॒र्क मर्चान्॑ । \newline
40. अर्चा॒ निन्द्रे न्द्रार्चा॒ नर्चा॒ निन्द्र॑ । \newline
41. इन्द्र॒ ग्रावा॑णो॒ ग्रावा॑ण॒ इन्द्रे न्द्र॒ ग्रावा॑णः । \newline
42. ग्रावा॑णो॒ अदि॑ति॒ रदि॑ति॒र् ग्रावा॑णो॒ ग्रावा॑णो॒ अदि॑तिः । \newline
43. अदि॑तिः स॒जोषाः᳚ स॒जोषा॒ अदि॑ति॒ रदि॑तिः स॒जोषाः᳚ । \newline
44. स॒जोषा॒ इति॑ स - जोषाः᳚ । \newline
45. अ॒न॒श्वासो॒ ये ये॑ ऽन॒श्वासो॑ ऽन॒श्वासो॒ ये । \newline
46. ये प॒वयः॑ प॒वयो॒ ये ये प॒वयः॑ । \newline
47. प॒वयो॑ ऽर॒था अ॑र॒थाः प॒वयः॑ प॒वयो॑ ऽर॒थाः । \newline
48. अ॒र॒था इन्द्रे॑षिता॒ इन्द्रे॑षिता अर॒था अ॑र॒था इन्द्रे॑षिताः । \newline
49. इन्द्रे॑षिता अ॒भ्यव॑र्तन्ता॒ भ्यव॑र्त॒न्ते न्द्रे॑षिता॒ इन्द्रे॑षिता अ॒भ्यव॑र्तन्त । \newline
50. इन्द्रे॑षिता॒ इतीन्द्र॑ - इ॒षि॒ताः॒ । \newline
51. अ॒भ्यव॑र्तन्त॒ दस्यू॒न् दस्यू॑ न॒भ्यव॑र्तन्ता॒ भ्यव॑र्तन्त॒ दस्यून्॑ । \newline
52. अ॒भ्यव॑र्त॒न्तेत्य॑भि - अव॑र्तन्त । \newline
53. दस्यू॒निति॒ दस्यून्॑ । \newline

\textbf{Ghana Paata } \newline

1. स॒ह॒सा॒व॒न् परि॑ष्टौ॒ परि॑ष्टौ सहसावन् थ्सहसाव॒न् परि॑ष्टा व॒घाया॒घाय॒ परि॑ष्टौ सहसावन् थ्सहसाव॒न् परि॑ष्टा व॒घाय॑ । \newline
2. स॒ह॒सा॒व॒न्निति॑ सहसा - व॒न्न् । \newline
3. परि॑ष्टा व॒घाया॒घाय॒ परि॑ष्टौ॒ परि॑ष्टा व॒घाय॑ भूम भूमा॒घाय॒ परि॑ष्टौ॒ परि॑ष्टा व॒घाय॑ भूम । \newline
4. अ॒घाय॑ भूम भूमा॒घाया॒घाय॑ भूम हरिवो हरिवो भूमा॒घाया॒घाय॑ भूम हरिवः । \newline
5. भू॒म॒ ह॒रि॒वो॒ ह॒रि॒वो॒ भू॒म॒ भू॒म॒ ह॒रि॒वः॒ प॒रा॒दै प॑रा॒दै ह॑रिवो भूम भूम हरिवः परा॒दै । \newline
6. ह॒रि॒वः॒ प॒रा॒दै प॑रा॒दै ह॑रिवो हरिवः परा॒दै । \newline
7. ह॒रि॒व॒ इति॑ हरि - वः॒ । \newline
8. प॒रा॒दा इति॑ परा - दै । \newline
9. त्राय॑स्व नो न॒स्त्राय॑स्व॒ त्राय॑स्व नो ऽवृ॒केभि॑ रवृ॒केभि॑र् न॒स्त्राय॑स्व॒ त्राय॑स्व नो ऽवृ॒केभिः॑ । \newline
10. नो॒ ऽवृ॒केभि॑ रवृ॒केभि॑र् नो नो ऽवृ॒केभि॒र् वरू॑थै॒र् वरू॑थै रवृ॒केभि॑र् नो नो ऽवृ॒केभि॒र् वरू॑थैः । \newline
11. अ॒वृ॒केभि॒र् वरू॑थै॒र् वरू॑थै रवृ॒केभि॑ रवृ॒केभि॒र् वरू॑थै॒स्तव॒ तव॒ वरू॑थै रवृ॒केभि॑ रवृ॒केभि॒र् वरू॑थै॒स्तव॑ । \newline
12. वरू॑थै॒स्तव॒ तव॒ वरू॑थै॒र् वरू॑थै॒स्तव॑ प्रि॒यासः॑ प्रि॒यास॒ स्तव॒ वरू॑थै॒र् वरू॑थै॒स्तव॑ प्रि॒यासः॑ । \newline
13. तव॑ प्रि॒यासः॑ प्रि॒यास॒ स्तव॒ तव॑ प्रि॒यासः॑ सू॒रिषु॑ सू॒रिषु॑ प्रि॒यास॒ स्तव॒ तव॑ प्रि॒यासः॑ सू॒रिषु॑ । \newline
14. प्रि॒यासः॑ सू॒रिषु॑ सू॒रिषु॑ प्रि॒यासः॑ प्रि॒यासः॑ सू॒रिषु॑ स्याम स्याम सू॒रिषु॑ प्रि॒यासः॑ प्रि॒यासः॑ सू॒रिषु॑ स्याम । \newline
15. सू॒रिषु॑ स्याम स्याम सू॒रिषु॑ सू॒रिषु॑ स्याम । \newline
16. स्या॒मेति॑ स्याम । \newline
17. अन॑वस्ते॒ ते ऽन॒वो ऽन॑वस्ते॒ रथ॒(ग्म्॒) रथ॒म् ते ऽन॒वो ऽन॑वस्ते॒ रथ᳚म् । \newline
18. ते॒ रथ॒(ग्म्॒) रथ॑म् ते ते॒ रथ॒ मश्वा॒याश्वा॑य॒ रथ॑म् ते ते॒ रथ॒ मश्वा॑य । \newline
19. रथ॒ मश्वा॒याश्वा॑य॒ रथ॒(ग्म्॒) रथ॒ मश्वा॑य तक्षन् तक्ष॒न् नश्वा॑य॒ रथ॒(ग्म्॒) रथ॒ मश्वा॑य तक्षन्न् । \newline
20. अश्वा॑य तक्षन् तक्ष॒न् नश्वा॒याश्वा॑य तक्ष॒न् त्वष्टा॒ त्वष्टा॑ तक्ष॒न् नश्वा॒याश्वा॑य तक्ष॒न् त्वष्टा᳚ । \newline
21. त॒क्ष॒न् त्वष्टा॒ त्वष्टा॑ तक्षन् तक्ष॒न् त्वष्टा॒ वज्रं॒ ॅवज्र॒म् त्वष्टा॑ तक्षन् तक्ष॒न् त्वष्टा॒ वज्र᳚म् । \newline
22. त्वष्टा॒ वज्रं॒ ॅवज्र॒म् त्वष्टा॒ त्वष्टा॒ वज्र॑म् पुरुहूत पुरुहूत॒ वज्र॒म् त्वष्टा॒ त्वष्टा॒ वज्र॑म् पुरुहूत । \newline
23. वज्र॑म् पुरुहूत पुरुहूत॒ वज्रं॒ ॅवज्र॑म् पुरुहूत द्यु॒मन्त॑म् द्यु॒मन्त॑म् पुरुहूत॒ वज्रं॒ ॅवज्र॑म् पुरुहूत द्यु॒मन्त᳚म् । \newline
24. पु॒रु॒हू॒त॒ द्यु॒मन्त॑म् द्यु॒मन्त॑म् पुरुहूत पुरुहूत द्यु॒मन्त᳚म् । \newline
25. पु॒रु॒हू॒तेति॑ पुरु - हू॒त॒ । \newline
26. द्यु॒मन्त॒मिति॑ द्यु - मन्त᳚म् । \newline
27. ब्र॒ह्माण॒ इन्द्र॒ मिन्द्र॑म् ब्र॒ह्माणो᳚ ब्र॒ह्माण॒ इन्द्र॑म् म॒हय॑न्तो म॒हय॑न्त॒ इन्द्र॑म् ब्र॒ह्माणो᳚ ब्र॒ह्माण॒ इन्द्र॑म् म॒हय॑न्तः । \newline
28. इन्द्र॑म् म॒हय॑न्तो म॒हय॑न्त॒ इन्द्र॒ मिन्द्र॑म् म॒हय॑न्तो अ॒र्कैर॒र्कैर् म॒हय॑न्त॒ इन्द्र॒ मिन्द्र॑म् म॒हय॑न्तो अ॒र्कैः । \newline
29. म॒हय॑न्तो अ॒र्कैर॒र्कैर् म॒हय॑न्तो म॒हय॑न्तो अ॒र्कै रव॑र्धय॒न् नव॑र्धयन् न॒र्कैर् म॒हय॑न्तो म॒हय॑न्तो अ॒र्कै रव॑र्धयन्न् । \newline
30. अ॒र्कै रव॑र्धय॒न् नव॑र्धयन् न॒र्कै र॒र्कै रव॑र्धय॒न् नह॒ये ऽह॒ये ऽव॑र्धयन् न॒र्कै र॒र्कै रव॑र्धय॒न् नह॑ये । \newline
31. अव॑र्धय॒न् नह॒ये ऽह॒ये ऽव॑र्धय॒न् नव॑र्धय॒न् नह॑ये॒ हन्त॒वै हन्त॒वा अह॒ये ऽव॑र्धय॒न् नव॑र्धय॒न् नह॑ये॒ हन्त॒वै । \newline
32. अह॑ये॒ हन्त॒वै हन्त॒वा अह॒ये ऽह॑ये॒ हन्त॒वा उ॑ वु॒ हन्त॒वा अह॒ये ऽह॑ये॒ हन्त॒वा उ॑ । \newline
33. हन्त॒वा उ॑ वु॒ हन्त॒वै हन्त॒वा उ॑ । \newline
34. उ॒ वित्यु॑ । \newline
35. वृष्णे॒ यद् यद् वृष्णे॒ वृष्णे॒ यत् ते॑ ते॒ यद् वृष्णे॒ वृष्णे॒ यत् ते᳚ । \newline
36. यत् ते॑ ते॒ यद् यत् ते॒ वृष॑णो॒ वृष॑णस्ते॒ यद् यत् ते॒ वृष॑णः । \newline
37. ते॒ वृष॑णो॒ वृष॑णस्ते ते॒ वृष॑णो अ॒र्क म॒र्कं ॅवृष॑णस्ते ते॒ वृष॑णो अ॒र्कम् । \newline
38. वृष॑णो अ॒र्क म॒र्कं ॅवृष॑णो॒ वृष॑णो अ॒र्क मर्चा॒ नर्चा॑ न॒र्कं ॅवृष॑णो॒ वृष॑णो अ॒र्क मर्चान्॑ । \newline
39. अ॒र्क मर्चा॒ नर्चा॑ न॒र्क म॒र्क मर्चा॒ निन्द्रे न्द्रार्चा॑ न॒र्क म॒र्क मर्चा॒ निन्द्र॑ । \newline
40. अर्चा॒ निन्द्रे न्द्रार्चा॒ नर्चा॒ निन्द्र॒ ग्रावा॑णो॒ ग्रावा॑ण॒ इन्द्रार्चा॒ नर्चा॒ निन्द्र॒ ग्रावा॑णः । \newline
41. इन्द्र॒ ग्रावा॑णो॒ ग्रावा॑ण॒ इन्द्रे न्द्र॒ ग्रावा॑णो॒ अदि॑ति॒ रदि॑ति॒र् ग्रावा॑ण॒ इन्द्रे न्द्र॒ ग्रावा॑णो॒ अदि॑तिः । \newline
42. ग्रावा॑णो॒ अदि॑ति॒ रदि॑ति॒र् ग्रावा॑णो॒ ग्रावा॑णो॒ अदि॑तिः स॒जोषाः᳚ स॒जोषा॒ अदि॑ति॒र् ग्रावा॑णो॒ ग्रावा॑णो॒ अदि॑तिः स॒जोषाः᳚ । \newline
43. अदि॑तिः स॒जोषाः᳚ स॒जोषा॒ अदि॑ति॒रदि॑तिः स॒जोषाः᳚ । \newline
44. स॒जोषा॒ इति॑ स - जोषाः᳚ । \newline
45. अ॒न॒श्वासो॒ ये ये॑ ऽन॒श्वासो॑ ऽन॒श्वासो॒ ये प॒वयः॑ प॒वयो॒ ये॑ ऽन॒श्वासो॑ ऽन॒श्वासो॒ ये प॒वयः॑ । \newline
46. ये प॒वयः॑ प॒वयो॒ ये ये प॒वयो॑ ऽर॒था अ॑र॒थाः प॒वयो॒ ये ये प॒वयो॑ ऽर॒थाः । \newline
47. प॒वयो॑ ऽर॒था अ॑र॒थाः प॒वयः॑ प॒वयो॑ ऽर॒था इन्द्रे॑षिता॒ इन्द्रे॑षिता अर॒थाः प॒वयः॑ प॒वयो॑ ऽर॒था इन्द्रे॑षिताः । \newline
48. अ॒र॒था इन्द्रे॑षिता॒ इन्द्रे॑षिता अर॒था अ॑र॒था इन्द्रे॑षिता अ॒भ्यव॑र्तन्ता॒ भ्यव॑र्त॒न्ते न्द्रे॑षिता अर॒था अ॑र॒था इन्द्रे॑षिता अ॒भ्यव॑र्तन्त । \newline
49. इन्द्रे॑षिता अ॒भ्यव॑र्तन्ता॒ भ्यव॑र्त॒न्ते न्द्रे॑षिता॒ इन्द्रे॑षिता अ॒भ्यव॑र्तन्त॒ दस्यू॒न् दस्यू॑ न॒भ्यव॑र्त॒न्ते न्द्रे॑षिता॒ इन्द्रे॑षिता अ॒भ्यव॑र्तन्त॒ दस्यून्॑ । \newline
50. इन्द्रे॑षिता॒ इतीन्द्र॑ - इ॒षि॒ताः॒ । \newline
51. अ॒भ्यव॑र्तन्त॒ दस्यू॒न् दस्यू॑ न॒भ्यव॑र्तन्ता॒ भ्यव॑र्तन्त॒ दस्यून्॑ । \newline
52. अ॒भ्यव॑र्त॒न्तेत्य॑भि - अव॑र्तन्त । \newline
53. दस्यू॒निति॒ दस्यून्॑ । \newline
\pagebreak


\end{document}