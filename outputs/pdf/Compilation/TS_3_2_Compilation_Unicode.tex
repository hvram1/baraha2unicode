\documentclass[17pt]{extarticle}
\usepackage{babel}
\usepackage{fontspec}
\usepackage{polyglossia}
\usepackage{extsizes}

\usepackage{color}   %May be necessary if you want to color links
\usepackage{hyperref}
\hypersetup{
    colorlinks=true, %set true if you want colored links
    linktoc=all,     %set to all if you want both sections and subsections linked
    linkcolor=black,  %choose some color if you want links to stand out
}

\setmainlanguage{sanskrit}
\setotherlanguages{english} %% or other languages
\setlength{\parindent}{0pt}
\pagestyle{myheadings}
\newfontfamily\devanagarifont[Script=Devanagari]{AdishilaVedic}
\renewcommand{\theHsection}{\thepart.section.\thesection}

\newcommand{\VAR}[1]{}
\newcommand{\BLOCK}[1]{}




\begin{document}
\begin{titlepage}
    \begin{center}
 
\begin{sanskrit}
    { \Large
    कृष्ण यजुर्वेदीय तैत्तिरीय संहिता,पद,जटा,घन पाठः 
    }
    \\
    \vspace{2.5cm}
    \mbox{ \Large
    3.2     तृतीयकाण्डे द्वितीयः प्रश्नः - पवमानग्राहादीनां व्याख्Yआनं   }
\end{sanskrit}
\end{center}

\end{titlepage}
\tableofcontents
\phantomsection
\pagebreak

\markright{ TS 3.2.1.1  \hfill https://www.vedavms.in \hfill}

\section{ TS 3.2.1.1 }

\textbf{TS 3.2.1.1 } \newline
\textbf{Samhita Paata} \newline

यो वै पव॑मानानामन्वारो॒हान्. वि॒द्वान्. यज॒तेऽनु॒ पव॑माना॒ना रो॑हति॒ न पव॑माने॒भ्यो-ऽव॑ च्छिद्यते श्ये॒नो॑ऽसि गाय॒त्रछ॑न्दा॒ अनु॒ त्वाऽऽर॑भे स्व॒स्ति मा॒ सं पा॑रय सुप॒र्णो॑ऽसि त्रि॒ष्टुप्छ॑न्दा॒ अनु॒ त्वाऽऽर॑भे स्व॒स्ति मा॒ सं पा॑रय॒ सघा॑ऽसि॒ जग॑तीछन्दा॒ अनु॒ त्वाऽऽर॑भे स्व॒स्ति मा॒ संपा॑र॒येत्या॑है॒ते- [  ] \newline

\textbf{Pada Paata} \newline

यः । वै । पव॑मानानाम् । अ॒न्वा॒रो॒हानित्य॑नु - आ॒रो॒हान् । वि॒द्वान् । यज॑ते । अन्विति॑ । पव॑मानान् । एति॑ । रो॒ह॒ति॒ । न । पव॑मानेभ्यः । अवेति॑ । छि॒द्य॒ते॒ । श्ये॒नः । अ॒सि॒ । गा॒य॒त्रछ॑न्दा॒ इति॑ गाय॒त्र-छ॒न्दाः॒ । अन्विति॑ । त्वा॒ । एति॑ । र॒भे॒ । स्व॒स्ति । मा॒ । समिति॑ । पा॒र॒य॒ । सु॒प॒र्ण इति॑ सु - प॒र्णः । अ॒सि॒ । त्रि॒ष्टुप्छ॑न्दा॒ इति॑ त्रि॒ष्टुप् - छ॒न्दाः॒ । अन्विति॑ । त्वा॒ । एति॑ । र॒भे॒ । स्व॒स्ति । मा॒ । समिति॑ । पा॒र॒य॒ । सघा᳚ । अ॒सि॒ । जग॑तीछन्दा॒ इति॒ जग॑ती - छ॒न्दाः॒ । अन्विति॑ । त्वा॒ । एति॑ । र॒भे॒ । स्व॒स्ति । मा॒ । समिति॑ । पा॒र॒य॒ । इति॑ । आ॒ह॒ । ए॒ते ।  \newline


\textbf{Krama Paata} \newline

यो वै । वै पव॑मानानाम् । पव॑मानाना,मन्वारो॒हान् । अ॒न्वा॒रो॒हान्. वि॒द्वान् । अ॒न्वा॒रो॒हानित्य॑नु - आ॒रो॒हान् । वि॒द्वान्. यज॑ते । यज॒तेऽनु॑ । अनु॒ पव॑मानान् । पव॑माना॒ना । आ रो॑हति । रो॒ह॒ति॒ न । न पव॑मानेभ्यः । पव॑माने॒भ्योऽव॑ । अव॑च्छिद्यते । छि॒द्य॒ते॒ श्ये॒नः । श्ये॒नो॑ऽसि । अ॒सि॒ गा॒य॒त्रछ॑न्दाः । गा॒य॒त्रछ॑न्दा॒ अनु॑ । गा॒य॒त्रछ॑न्दा॒ इति॑ गाय॒त्र - छ॒न्दाः॒ । अनु॑ त्वा । त्वा । आ र॑भे । र॒भे॒ स्व॒स्ति । स्व॒स्ति मा᳚ । मा॒ सम् । सम् पा॑रय । पा॒र॒य॒ सु॒प॒र्णः । सु॒प॒र्णो॑ ऽसि । सु॒प॒र्ण इति॑ सु - प॒र्णः । अ॒सि॒ त्रि॒ष्टुप्छ॑न्दाः । त्रि॒ष्टुप्छ॑न्दा॒ अनु॑ । त्रि॒ष्टुप्छ॑न्दा॒ इति॑ त्रि॒ष्टुप् - छ॒न्दाः॒ । अनु॑ त्वा । त्वा । आ र॑भे । र॒भे॒ स्व॒स्ति । स्व॒स्ति मा᳚ । 
मा॒ सम् । सम् पा॑रय । पा॒र॒य॒ सघा᳚ । सघा॑ऽसि । अ॒सि॒ जग॑तीछन्दाः । जग॑तीछन्दा॒ अनु॑ । जग॑तीछन्दा॒ इति॒ जग॑ती - छ॒न्दाः॒ । अनु॑ त्वा । त्वा । 
आ र॑भे । र॒भे॒ स्व॒स्ति । स्व॒स्ति मा᳚ । मा॒ सम् । सम् पा॑रय । 
पा॒र॒येति॑ । इत्या॑ह । आ॒है॒ते । ए॒ते वै \newline

\textbf{Jatai Paata} \newline

1. यो वै वै यो यो वै । \newline
2. वै पव॑मानाना॒म् पव॑मानानां॒ ॅवै वै पव॑मानानाम् । \newline
3. पव॑मानाना मन्वारो॒हा न॑न्वारो॒हान् पव॑मानाना॒म् पव॑मानाना मन्वारो॒हान् । \newline
4. अ॒न्वा॒रो॒हान्. वि॒द्वान्. वि॒द्वा न॑न्वारो॒हा न॑न्वारो॒हान्. वि॒द्वान् । \newline
5. अ॒न्वा॒रो॒हानित्य॑नु - आ॒रो॒हान् । \newline
6. वि॒द्वान्. यज॑ते॒ यज॑ते वि॒द्वान्. वि॒द्वान्. यज॑ते । \newline
7. यज॒ते ऽन्वनु॒ यज॑ते॒ यज॒ते ऽनु॑ । \newline
8. अनु॒ पव॑माना॒न् पव॑माना॒ नन्वनु॒ पव॑मानान् । \newline
9. पव॑माना॒ ना पव॑माना॒न् पव॑माना॒ ना । \newline
10. आ रो॑हति रोह॒त्या रो॑हति । \newline
11. रो॒ह॒ति॒ न न रो॑हति रोहति॒ न । \newline
12. न पव॑मानेभ्यः॒ पव॑मानेभ्यो॒ न न पव॑मानेभ्यः । \newline
13. पव॑माने॒भ्यो ऽवाव॒ पव॑मानेभ्यः॒ पव॑माने॒भ्यो ऽव॑ । \newline
14. अव॑च् छिद्यते छिद्य॒ते ऽवाव॑च् छिद्यते । \newline
15. छि॒द्य॒ते॒ श्ये॒नः श्ये॒न श्छि॑द्यते छिद्यते श्ये॒नः । \newline
16. श्ये॒नो᳚ ऽस्यसि श्ये॒नः श्ये॒नो॑ ऽसि । \newline
17. अ॒सि॒ गा॒य॒त्रछ॑न्दा गाय॒त्रछ॑न्दा अस्यसि गाय॒त्रछ॑न्दाः । \newline
18. गा॒य॒त्रछ॑न्दा॒ अन्वनु॑ गाय॒त्रछ॑न्दा गाय॒त्रछ॑न्दा॒ अनु॑ । \newline
19. गा॒य॒त्रछ॑न्दा॒ इति॑ गाय॒त्र - छ॒न्दाः॒ । \newline
20. अनु॑ त्वा॒ त्वा ऽन्वनु॑ त्वा । \newline
21. त्वा ऽऽत्वा॒ त्वा । \newline
22. आ र॑भे रभ॒ आ र॑भे । \newline
23. र॒भे॒ स्व॒स्ति स्व॒स्ति र॑भे रभे स्व॒स्ति । \newline
24. स्व॒स्ति मा॑ मा स्व॒स्ति स्व॒स्ति मा᳚ । \newline
25. मा॒ सꣳ सम् मा॑ मा॒ सम् । \newline
26. सम् पा॑रय पारय॒ सꣳ सम् पा॑रय । \newline
27. पा॒र॒य॒ सु॒प॒र्णः सु॑प॒र्णः पा॑रय पारय सुप॒र्णः । \newline
28. सु॒प॒र्णो᳚ ऽस्यसि सुप॒र्णः सु॑प॒र्णो॑ ऽसि । \newline
29. सु॒प॒र्ण इति॑ सु - प॒र्णः । \newline
30. अ॒सि॒ त्रि॒ष्टुप्छ॑न्दा स्त्रि॒ष्टुप्छ॑न्दा अस्यसि त्रि॒ष्टुप्छ॑न्दाः । \newline
31. त्रि॒ष्टुप्छ॑न्दा॒ अन्वनु॑ त्रि॒ष्टुप्छ॑न्दा स्त्रि॒ष्टुप्छ॑न्दा॒ अनु॑ । \newline
32. त्रि॒ष्टुप्छ॑न्दा॒ इति॑ त्रि॒ष्टुप् - छ॒न्दाः॒ । \newline
33. अनु॑ त्वा॒ त्वा ऽन्वनु॑ त्वा । \newline
34. त्वा ऽऽत्वा॒ त्वा । \newline
35. आ र॑भे रभ॒ आ र॑भे । \newline
36. र॒भे॒ स्व॒स्ति स्व॒स्ति र॑भे रभे स्व॒स्ति । \newline
37. स्व॒स्ति मा॑ मा स्व॒स्ति स्व॒स्ति मा᳚ । \newline
38. मा॒ सꣳ सम् मा॑ मा॒ सम् । \newline
39. सम् पा॑रय पारय॒ सꣳ सम् पा॑रय । \newline
40. पा॒र॒य॒ सघा॒ सघा॑ पारय पारय॒ सघा᳚ । \newline
41. सघा᳚ ऽस्यसि॒ सघा॒ सघा॑ ऽसि । \newline
42. अ॒सि॒ जग॑तीछन्दा॒ जग॑तीछन्दा अस्यसि॒ जग॑तीछन्दाः । \newline
43. जग॑तीछन्दा॒ अन्वनु॒ जग॑तीछन्दा॒ जग॑तीछन्दा॒ अनु॑ । \newline
44. जग॑तीछन्दा॒ इति॒ जग॑ती - छ॒न्दाः॒ । \newline
45. अनु॑ त्वा॒ त्वा ऽन्वनु॑ त्वा । \newline
46. त्वा ऽऽत्वा॒ त्वा । \newline
47. आ र॑भे रभ॒ आ र॑भे । \newline
48. र॒भे॒ स्व॒स्ति स्व॒स्ति र॑भे रभे स्व॒स्ति । \newline
49. स्व॒स्ति मा॑ मा स्व॒स्ति स्व॒स्ति मा᳚ । \newline
50. मा॒ सꣳ सम् मा॑ मा॒ सम् । \newline
51. सम् पा॑रय पारय॒ सꣳ सम् पा॑रय । \newline
52. पा॒र॒ये तीति॑ पारय पार॒ये ति॑ । \newline
53. इत्या॑हा॒हे तीत्या॑ह । \newline
54. आ॒है॒त ए॒त आ॑हाहै॒ते । \newline
55. ए॒ते वै वा ए॒त ए॒ते वै । \newline

\textbf{Ghana Paata } \newline

1. यो वै वै यो यो वै पव॑मानाना॒म् पव॑मानानां॒ ॅवै यो यो वै पव॑मानानाम् । \newline
2. वै पव॑मानाना॒म् पव॑मानानां॒ ॅवै वै पव॑मानाना मन्वारो॒हा न॑न्वारो॒हान् पव॑मानानां॒ ॅवै वै पव॑मानाना मन्वारो॒हान् । \newline
3. पव॑मानाना मन्वारो॒हा न॑न्वारो॒हान् पव॑मानाना॒म् पव॑मानाना मन्वारो॒हान्. वि॒द्वान्. वि॒द्वा न॑न्वारो॒हान् पव॑मानाना॒म् पव॑मानाना मन्वारो॒हान्. वि॒द्वान् । \newline
4. अ॒न्वा॒रो॒हान्. वि॒द्वान्. वि॒द्वा न॑न्वारो॒हा न॑न्वारो॒हान्. वि॒द्वान्. यज॑ते॒ यज॑ते वि॒द्वा न॑न्वारो॒हा न॑न्वारो॒हान्. वि॒द्वान्. यज॑ते । \newline
5. अ॒न्वा॒रो॒हानित्य॑नु - आ॒रो॒हान् । \newline
6. वि॒द्वान्. यज॑ते॒ यज॑ते वि॒द्वान्. वि॒द्वान्. यज॒ते ऽन्वनु॒ यज॑ते वि॒द्वान्. वि॒द्वान्. यज॒ते ऽनु॑ । \newline
7. यज॒ते ऽन्वनु॒ यज॑ते॒ यज॒ते ऽनु॒ पव॑माना॒न् पव॑माना॒ ननु॒ यज॑ते॒ यज॒ते ऽनु॒ पव॑मानान् । \newline
8. अनु॒ पव॑माना॒न् पव॑माना॒ नन्वनु॒ पव॑माना॒ ना पव॑माना॒ नन्वनु॒ पव॑माना॒ ना । \newline
9. पव॑माना॒ ना पव॑माना॒न् पव॑माना॒ ना रो॑हति रोह॒त्या पव॑माना॒न् पव॑माना॒ ना रो॑हति । \newline
10. आ रो॑हति रोह॒त्या रो॑हति॒ न न रो॑ह॒त्या रो॑हति॒ न । \newline
11. रो॒ह॒ति॒ न न रो॑हति रोहति॒ न पव॑मानेभ्यः॒ पव॑मानेभ्यो॒ न रो॑हति रोहति॒ न पव॑मानेभ्यः । \newline
12. न पव॑मानेभ्यः॒ पव॑मानेभ्यो॒ न न पव॑माने॒भ्यो ऽवाव॒ पव॑मानेभ्यो॒ न न पव॑माने॒भ्यो ऽव॑ । \newline
13. पव॑माने॒भ्यो ऽवाव॒ पव॑मानेभ्यः॒ पव॑माने॒भ्यो ऽव॑ च्छिद्यते छिद्य॒ते ऽव॒ पव॑मानेभ्यः॒ पव॑माने॒भ्यो ऽव॑ च्छिद्यते । \newline
14. अव॑ च्छिद्यते छिद्य॒ते ऽवाव॑ च्छिद्यते श्ये॒नः श्ये॒न श्छि॑द्य॒ते ऽवाव॑ च्छिद्यते श्ये॒नः । \newline
15. छि॒द्य॒ते॒ श्ये॒नः श्ये॒न श्छि॑द्यते छिद्यते श्ये॒नो᳚ ऽस्यसि श्ये॒न श्छि॑द्यते छिद्यते श्ये॒नो॑ ऽसि । \newline
16. श्ये॒नो᳚ ऽस्यसि श्ये॒नः श्ये॒नो॑ ऽसि गाय॒त्रछ॑न्दा गाय॒त्रछ॑न्दा असि श्ये॒नः श्ये॒नो॑ ऽसि गाय॒त्रछ॑न्दाः । \newline
17. अ॒सि॒ गा॒य॒त्रछ॑न्दा गाय॒त्रछ॑न्दा अस्यसि गाय॒त्रछ॑न्दा॒ अन्वनु॑ गाय॒त्रछ॑न्दा अस्यसि गाय॒त्रछ॑न्दा॒ अनु॑ । \newline
18. गा॒य॒त्रछ॑न्दा॒ अन्वनु॑ गाय॒त्रछ॑न्दा गाय॒त्रछ॑न्दा॒ अनु॑ त्वा॒ त्वा ऽनु॑ गाय॒त्रछ॑न्दा गाय॒त्रछ॑न्दा॒ अनु॑ त्वा । \newline
19. गा॒य॒त्रछ॑न्दा॒ इति॑ गाय॒त्र - छ॒न्दाः॒ । \newline
20. अनु॑ त्वा॒ त्वा ऽन्वनु॒ त्वा ऽऽत्वा ऽन्वनु॒ त्वा । \newline
21. त्वा ऽऽत्वा॒ त्वा ऽऽर॑भे रभ॒ आ त्वा॒ त्वा ऽऽर॑भे । \newline
22. आ र॑भे रभ॒ आ र॑भे स्व॒स्ति स्व॒स्ति र॑भ॒ आ र॑भे स्व॒स्ति । \newline
23. र॒भे॒ स्व॒स्ति स्व॒स्ति र॑भे रभे स्व॒स्ति मा॑ मा स्व॒स्ति र॑भे रभे स्व॒स्ति मा᳚ । \newline
24. स्व॒स्ति मा॑ मा स्व॒स्ति स्व॒स्ति मा॒ सꣳ सम् मा᳚ स्व॒स्ति स्व॒स्ति मा॒ सम् । \newline
25. मा॒ सꣳ सम् मा॑ मा॒ सम् पा॑रय पारय॒ सम् मा॑ मा॒ सम् पा॑रय । \newline
26. सम् पा॑रय पारय॒ सꣳ सम् पा॑रय सुप॒र्णः सु॑प॒र्णः पा॑रय॒ सꣳ सम् पा॑रय सुप॒र्णः । \newline
27. पा॒र॒य॒ सु॒प॒र्णः सु॑प॒र्णः पा॑रय पारय सुप॒र्णो᳚ ऽस्यसि सुप॒र्णः पा॑रय पारय सुप॒र्णो॑ ऽसि । \newline
28. सु॒प॒र्णो᳚ ऽस्यसि सुप॒र्णः सु॑प॒र्णो॑ ऽसि त्रि॒ष्टुप्छ॑न्दा स्त्रि॒ष्टुप्छ॑न्दा असि सुप॒र्णः सु॑प॒र्णो॑ ऽसि त्रि॒ष्टुप्छ॑न्दाः । \newline
29. सु॒प॒र्ण इति॑ सु - प॒र्णः । \newline
30. अ॒सि॒ त्रि॒ष्टुप्छ॑न्दा स्त्रि॒ष्टुप्छ॑न्दा अस्यसि त्रि॒ष्टुप्छ॑न्दा॒ अन्वनु॑ त्रि॒ष्टुप्छ॑न्दा अस्यसि त्रि॒ष्टुप्छ॑न्दा॒ अनु॑ । \newline
31. त्रि॒ष्टुप्छ॑न्दा॒ अन्वनु॑ त्रि॒ष्टुप्छ॑न्दा स्त्रि॒ष्टुप्छ॑न्दा॒ अनु॑ त्वा॒ त्वा ऽनु॑ त्रि॒ष्टुप्छ॑न्दा स्त्रि॒ष्टुप्छ॑न्दा॒ अनु॑ त्वा । \newline
32. त्रि॒ष्टुप्छ॑न्दा॒ इति॑ त्रि॒ष्टुप् - छ॒न्दाः॒ । \newline
33. अनु॑ त्वा॒ त्वा ऽन्वनु॒ त्वा ऽऽत्वा ऽन्वनु॒ त्वा । \newline
34. त्वा ऽऽत्वा॒ त्वा ऽऽर॑भे रभ॒ आ त्वा॒ त्वा ऽऽर॑भे । \newline
35. आ र॑भे रभ॒ आ र॑भे स्व॒स्ति स्व॒स्ति र॑भ॒ आ र॑भे स्व॒स्ति । \newline
36. र॒भे॒ स्व॒स्ति स्व॒स्ति र॑भे रभे स्व॒स्ति मा॑ मा स्व॒स्ति र॑भे रभे स्व॒स्ति मा᳚ । \newline
37. स्व॒स्ति मा॑ मा स्व॒स्ति स्व॒स्ति मा॒ सꣳ सम् मा᳚ स्व॒स्ति स्व॒स्ति मा॒ सम् । \newline
38. मा॒ सꣳ सम् मा॑ मा॒ सम् पा॑रय पारय॒ सम् मा॑ मा॒ सम् पा॑रय । \newline
39. सम् पा॑रय पारय॒ सꣳ सम् पा॑रय॒ सघा॒ सघा॑ पारय॒ सꣳ सम् पा॑रय॒ सघा᳚ । \newline
40. पा॒र॒य॒ सघा॒ सघा॑ पारय पारय॒ सघा᳚ ऽस्यसि॒ सघा॑ पारय पारय॒ सघा॑ ऽसि । \newline
41. सघा᳚ ऽस्यसि॒ सघा॒ सघा॑ ऽसि॒ जग॑तीछन्दा॒ जग॑तीछन्दा असि॒ सघा॒ सघा॑ ऽसि॒ जग॑तीछन्दाः । \newline
42. अ॒सि॒ जग॑तीछन्दा॒ जग॑तीछन्दा अस्यसि॒ जग॑तीछन्दा॒ अन्वनु॒ जग॑तीछन्दा अस्यसि॒ जग॑तीछन्दा॒ अनु॑ । \newline
43. जग॑तीछन्दा॒ अन्वनु॒ जग॑तीछन्दा॒ जग॑तीछन्दा॒ अनु॑ त्वा॒ त्वा ऽनु॒ जग॑तीछन्दा॒ जग॑तीछन्दा॒ अनु॑ त्वा । \newline
44. जग॑तीछन्दा॒ इति॒ जग॑ती - छ॒न्दाः॒ । \newline
45. अनु॑ त्वा त्वा॒ ऽन्वनु॒ त्वा ऽऽत्वा ऽन्वनु॒ त्वा । \newline
46. त्वा ऽऽत्वा॒ त्वा ऽऽर॑भे रभ॒ आ त्वा॒ त्वा ऽऽर॑भे । \newline
47. आ र॑भे रभ॒ आ र॑भे स्व॒स्ति स्व॒स्ति र॑भ॒ आ र॑भे स्व॒स्ति । \newline
48. र॒भे॒ स्व॒स्ति स्व॒स्ति र॑भे रभे स्व॒स्ति मा॑ मा स्व॒स्ति र॑भे रभे स्व॒स्ति मा᳚ । \newline
49. स्व॒स्ति मा॑ मा स्व॒स्ति स्व॒स्ति मा॒ सꣳ सम् मा᳚ स्व॒स्ति स्व॒स्ति मा॒ सम् । \newline
50. मा॒ सꣳ सम् मा॑ मा॒ सम् पा॑रय पारय॒ सम् मा॑ मा॒ सम् पा॑रय । \newline
51. सम् पा॑रय पारय॒ सꣳ सम् पा॑र॒ये तीति॑ पारय॒ सꣳ सम् पा॑र॒ये ति॑ । \newline
52. पा॒र॒ये तीति॑ पारय पार॒ये त्या॑हा॒हे ति॑ पारय पार॒ये त्या॑ह । \newline
53. इत्या॑हा॒हे तीत्या॑है॒त ए॒त आ॒हे तीत्या॑है॒ते । \newline
54. आ॒है॒त ए॒त आ॑हाहै॒ते वै वा ए॒त आ॑हाहै॒ते वै । \newline
55. ए॒ते वै वा ए॒त ए॒ते वै पव॑मानाना॒म् पव॑मानानां॒ ॅवा ए॒त ए॒ते वै पव॑मानानाम् । \newline
\pagebreak
\markright{ TS 3.2.1.2  \hfill https://www.vedavms.in \hfill}

\section{ TS 3.2.1.2 }

\textbf{TS 3.2.1.2 } \newline
\textbf{Samhita Paata} \newline

वै पव॑मानानामन्वारो॒हास्तान्. य ए॒वं ॅवि॒द्वान्. यज॒तेऽनु॒ पव॑माना॒ना रो॑हति॒ न पव॑माने॒भ्योऽव॑ च्छिद्यते॒ यो वै पव॑मानस्य॒ सन्त॑तिं॒ ॅवेद॒ सर्व॒मायु॑रेति॒ न पु॒राऽऽयु॑षः॒ प्र मी॑यते पशु॒मान् भ॑वति वि॒न्दते᳚ प्र॒जां पव॑मानस्य॒ ग्रहा॑ गृह्य॒न्तेऽथ॒ वा अ॑स्यै॒तेऽगृ॑हीता द्रोणकल॒श आ॑धव॒नीयः॑ पूत॒भृत् तान्. यदगृ॑हीत्वोपाकु॒र्यात् पव॑मानं॒ ॅवि - [  ] \newline

\textbf{Pada Paata} \newline

वै । पव॑मानानाम् । अ॒न्वा॒रो॒हा इत्य॑नु - आ॒रो॒हाः । तान् । यः । ए॒वम् । वि॒द्वान् । यज॑ते । अन्विति॑ । पव॑मानान् । एति॑ । रो॒ह॒ति॒ । न । पव॑मानेभ्यः । अवेति॑ । छि॒द्य॒ते॒ । यः । वै । पव॑मानस्य । संत॑ति॒मिति॒ सं - त॒ति॒म् । वेद॑ । सर्व᳚म् । आयुः॑ । ए॒ति॒ । न । पु॒रा । आयु॑षः । प्रेति॑ । मी॒य॒ते॒ । प॒शु॒मानिति॑ पशु-मान् । भ॒व॒ति॒ । वि॒न्दते᳚ । प्र॒जामिति॑ प्र - जाम् । पव॑मानस्य । ग्रहाः᳚ । गृ॒ह्य॒न्ते॒ । अथ॑ । वै । अ॒स्य॒ । ए॒ते । अगृ॑हीताः । द्रो॒ण॒क॒ल॒श इति॑ द्रोण - क॒ल॒शः । आ॒ध॒व॒नीय॒ इत्या᳚ - ध॒व॒नीयः॑ । पू॒त॒भृदिति॑ पूत-भृत् । तान् । यत् । अगृ॑हीत्वा । उ॒पा॒कु॒र्यादित्यु॑प - आ॒कु॒र्यात् । पव॑मानम् । वीति॑ ।  \newline


\textbf{Krama Paata} \newline

वै पव॑मनानाम् । पव॑मानाना,मन्वारो॒हाः । अ॒न्वा॒रो॒हास्तान् । अ॒न्वा॒रो॒हा इत्य॑नु - आ॒रो॒हाः । तान्. यः । य ए॒वम् । ए॒वं ॅवि॒द्वान् । वि॒द्वान्. यज॑ते । यज॒तेऽनु॑ । अनु॒ पव॑मानान् । पव॑माना॒ना । आ रो॑हति । रो॒ह॒ति॒ न । न पव॑मानेभ्यः । पव॑माने॒भ्योऽव॑ । अव॑च्छिद्यते । छि॒द्य॒ते॒ यः । यो वै । वै पव॑मानस्य । पव॑मानस्य॒ सन्त॑तिम् । सन्त॑तिं॒ ॅवेद॑ । सन्त॑ति॒मिति॒ सम् - त॒ति॒म् । वेद॒ सर्व᳚म् । सर्व॒मायुः॑ । आयु॑रेति । ए॒ति॒ न । न पु॒रा । पु॒रा ऽऽयु॑षः । आयु॑षः॒ प्र । प्र मी॑यते । मी॒य॒ते॒ प॒शु॒मान् । प॒शु॒मान् भ॑वति । प॒शु॒मानिति॑ पशु - मान् । भ॒व॒ति॒ वि॒न्दते᳚ । वि॒न्दते᳚ प्र॒जाम् । प्र॒जाम् पव॑मानस्य । प्र॒जामिति॑ प्र - जाम् । पव॑मानस्य॒ ग्रहाः᳚ । ग्रहा॑ गृह्यन्ते । गृ॒ह्य॒न्ते ऽथ॑ । अथ॒ वै । वा अ॑स्य । अ॒स्यै॒ते । ए॒ते ऽगृ॑हीताः । अगृ॑हीता द्रोणकल॒शः । द्रो॒ण॒क॒ल॒श आ॑धव॒नीयः॑ । द्रो॒ण॒क॒ल॒श इति॑ द्रोण - क॒ल॒शः । आ॒ध॒व॒नीयः॑ पूत॒भृत् । आ॒ध॒व॒नीय॒ इत्या᳚ - ध॒व॒नीयः॑ । पू॒त॒भृत् तान् । पू॒त॒भृदिति॑ पूत - भृत् । तान्. यत् । यदगृ॑हीत्वा । अगृ॑हीत्वोपाकु॒र्यात् । उ॒पा॒कु॒र्यात्,पव॑मानम् । उ॒पा॒कु॒र्यादित्यु॑प - आ॒कु॒र्यात् । पव॑मानं॒ ॅवि ( ) । वि च्छि॑न्द्यात् \newline

\textbf{Jatai Paata} \newline

1. वै पव॑मानाना॒म् पव॑मानानां॒ ॅवै वै पव॑मानानाम् । \newline
2. पव॑मानाना मन्वारो॒हा अ॑न्वारो॒हाः पव॑मानाना॒म् पव॑मानाना मन्वारो॒हाः । \newline
3. अ॒न्वा॒रो॒हा स्ताꣳ स्ता न॑न्वारो॒हा अ॑न्वारो॒हा स्तान् । \newline
4. अ॒न्वा॒रो॒हा इत्य॑नु - आ॒रो॒हाः । \newline
5. तान्. यो यस्ताꣳ स्तान्. यः । \newline
6. य ए॒व मे॒वं ॅयो य ए॒वम् । \newline
7. ए॒वं ॅवि॒द्वान्. वि॒द्वा ने॒व मे॒वं ॅवि॒द्वान् । \newline
8. वि॒द्वान्. यज॑ते॒ यज॑ते वि॒द्वान्. वि॒द्वान्. यज॑ते । \newline
9. यज॒ते ऽन्वनु॒ यज॑ते॒ यज॒ते ऽनु॑ । \newline
10. अनु॒ पव॑माना॒न् पव॑माना॒ नन्वनु॒ पव॑मानान् । \newline
11. पव॑माना॒ ना पव॑माना॒न् पव॑माना॒ ना । \newline
12. आ रो॑हति रोह॒त्या रो॑हति । \newline
13. रो॒ह॒ति॒ न न रो॑हति रोहति॒ न । \newline
14. न पव॑मानेभ्यः॒ पव॑मानेभ्यो॒ न न पव॑मानेभ्यः । \newline
15. पव॑माने॒भ्यो ऽवाव॒ पव॑मानेभ्यः॒ पव॑माने॒भ्यो ऽव॑ । \newline
16. अव॑च् छिद्यते छिद्य॒ते ऽवाव॑च् छिद्यते । \newline
17. छि॒द्य॒ते॒ यो य श्छि॑द्यते छिद्यते॒ यः । \newline
18. यो वै वै यो यो वै । \newline
19. वै पव॑मानस्य॒ पव॑मानस्य॒ वै वै पव॑मानस्य । \newline
20. पव॑मानस्य॒ सन्त॑तिꣳ॒॒ सन्त॑ति॒म् पव॑मानस्य॒ पव॑मानस्य॒ सन्त॑तिम् । \newline
21. सन्त॑तिं॒ ॅवेद॒ वेद॒ सन्त॑तिꣳ॒॒ सन्त॑तिं॒ ॅवेद॑ । \newline
22. सन्त॑ति॒मिति॒ सं - त॒ति॒म् । \newline
23. वेद॒ सर्वꣳ॒॒ सर्वं॒ ॅवेद॒ वेद॒ सर्व᳚म् । \newline
24. सर्व॒ मायु॒ रायुः॒ सर्वꣳ॒॒ सर्व॒ मायुः॑ । \newline
25. आयु॑ रेत्ये॒ त्यायु॒ रायु॑रेति । \newline
26. ए॒ति॒ न नैत्ये॑ति॒ न । \newline
27. न पु॒रा पु॒रा न न पु॒रा । \newline
28. पु॒रा ऽऽयु॑ष॒ आयु॑षः पु॒रा पु॒रा ऽऽयु॑षः । \newline
29. आयु॑षः॒ प्र प्रायु॑ष॒ आयु॑षः॒ प्र । \newline
30. प्र मी॑यते मीयते॒ प्र प्र मी॑यते । \newline
31. मी॒य॒ते॒ प॒शु॒मान् प॑शु॒मान् मी॑यते मीयते पशु॒मान् । \newline
32. प॒शु॒मान् भ॑वति भवति पशु॒मान् प॑शु॒मान् भ॑वति । \newline
33. प॒शु॒मानिति॑ पशु - मान् । \newline
34. भ॒व॒ति॒ वि॒न्दते॑ वि॒न्दते॑ भवति भवति वि॒न्दते᳚ । \newline
35. वि॒न्दते᳚ प्र॒जाम् प्र॒जां ॅवि॒न्दते॑ वि॒न्दते᳚ प्र॒जाम् । \newline
36. प्र॒जाम् पव॑मानस्य॒ पव॑मानस्य प्र॒जाम् प्र॒जाम् पव॑मानस्य । \newline
37. प्र॒जामिति॑ प्र - जाम् । \newline
38. पव॑मानस्य॒ ग्रहा॒ ग्रहाः॒ पव॑मानस्य॒ पव॑मानस्य॒ ग्रहाः᳚ । \newline
39. ग्रहा॑ गृह्यन्ते गृह्यन्ते॒ ग्रहा॒ ग्रहा॑ गृह्यन्ते । \newline
40. गृ॒ह्य॒न्ते ऽथाथ॑ गृह्यन्ते गृह्य॒न्ते ऽथ॑ । \newline
41. अथ॒ वै वा अथाथ॒ वै । \newline
42. वा अ॑स्यास्य॒ वै वा अ॑स्य । \newline
43. अ॒स्यै॒त ए॒ते᳚ ऽस्या स्यै॒ते । \newline
44. ए॒ते ऽगृ॑हीता॒ अगृ॑हीता ए॒त ए॒ते ऽगृ॑हीताः । \newline
45. अगृ॑हीता द्रोणकल॒शो द्रो॑णकल॒शो ऽगृ॑हीता॒ अगृ॑हीता द्रोणकल॒शः । \newline
46. द्रो॒ण॒क॒ल॒श आ॑धव॒नीय॑ आधव॒नीयो᳚ द्रोणकल॒शो द्रो॑णकल॒श आ॑धव॒नीयः॑ । \newline
47. द्रो॒ण॒क॒ल॒श इति॑ द्रोण - क॒ल॒शः । \newline
48. आ॒ध॒व॒नीयः॑ पूत॒भृत् पू॑त॒भृ दा॑धव॒नीय॑ आधव॒नीयः॑ पूत॒भृत् । \newline
49. आ॒ध॒व॒नीय॒ इत्या᳚ - ध॒व॒नीयः॑ । \newline
50. पू॒त॒भृत् ताꣳ स्तान् पू॑त॒भृत् पू॑त॒भृत् तान् । \newline
51. पू॒त॒भृदिति॑ पूत - भृत् । \newline
52. तान्. यद् यत् ताꣳ स्तान्. यत् । \newline
53. यदगृ॑ही॒त्वा ऽगृ॑हीत्वा॒ यद् यदगृ॑हीत्वा । \newline
54. अगृ॑ही त्वोपाकु॒र्या दु॑पाकु॒र्या दगृ॑ही॒त्वा ऽगृ॑ही त्वोपाकु॒र्यात् । \newline
55. उ॒पा॒कु॒र्यात् पव॑मान॒म् पव॑मान मुपाकु॒र्या दु॑पाकु॒र्यात् पव॑मानम् । \newline
56. उ॒पा॒कु॒र्यादित्यु॑प - आ॒कु॒र्यात् । \newline
57. पव॑मानं॒ ॅवि वि पव॑मान॒म् पव॑मानं॒ ॅवि । \newline
58. विच् छि॑न्द्याच् छिन्द्या॒द् वि विच् छि॑न्द्यात् । \newline

\textbf{Ghana Paata } \newline

1. वै पव॑मानाना॒म् पव॑मानानां॒ ॅवै वै पव॑मानाना मन्वारो॒हा अ॑न्वारो॒हाः पव॑मानानां॒ ॅवै वै पव॑मानाना मन्वारो॒हाः । \newline
2. पव॑मानाना मन्वारो॒हा अ॑न्वारो॒हाः पव॑मानाना॒म् पव॑मानाना मन्वारो॒हा स्ताꣳ स्ता न॑न्वारो॒हाः पव॑मानाना॒म् पव॑मानाना मन्वारो॒हा स्तान् । \newline
3. अ॒न्वा॒रो॒हा स्ताꣳ स्ता न॑न्वारो॒हा अ॑न्वारो॒हा स्तान्. यो यस्ता न॑न्वारो॒हा अ॑न्वारो॒हा स्तान्. यः । \newline
4. अ॒न्वा॒रो॒हा इत्य॑नु - आ॒रो॒हाः । \newline
5. तान्. यो य स्ताꣳ स्तान्. य ए॒व मे॒वं ॅय स्ताꣳ स्तान्. य ए॒वम् । \newline
6. य ए॒व मे॒वं ॅयो य ए॒वं ॅवि॒द्वान्. वि॒द्वा ने॒वं ॅयो य ए॒वं ॅवि॒द्वान् । \newline
7. ए॒वं ॅवि॒द्वान्. वि॒द्वा ने॒व मे॒वं ॅवि॒द्वान्. यज॑ते॒ यज॑ते वि॒द्वा ने॒व मे॒वं ॅवि॒द्वान्. यज॑ते । \newline
8. वि॒द्वान्. यज॑ते॒ यज॑ते वि॒द्वान्. वि॒द्वान्. यज॒ते ऽन्वनु॒ यज॑ते वि॒द्वान्. वि॒द्वान्. यज॒ते ऽनु॑ । \newline
9. यज॒ते ऽन्वनु॒ यज॑ते॒ यज॒ते ऽनु॒ पव॑माना॒न् पव॑माना॒ ननु॒ यज॑ते॒ यज॒ते ऽनु॒ पव॑मानान् । \newline
10. अनु॒ पव॑माना॒न् पव॑माना॒ नन्वनु॒ पव॑माना॒ ना पव॑माना॒ नन्वनु॒ पव॑माना॒ ना । \newline
11. पव॑माना॒ ना पव॑माना॒न् पव॑माना॒ ना रो॑हति रोह॒त्या पव॑माना॒न् पव॑माना॒ ना रो॑हति । \newline
12. आ रो॑हति रोह॒त्या रो॑हति॒ न न रो॑ह॒त्या रो॑हति॒ न । \newline
13. रो॒ह॒ति॒ न न रो॑हति रोहति॒ न पव॑मानेभ्यः॒ पव॑मानेभ्यो॒ न रो॑हति रोहति॒ न पव॑मानेभ्यः । \newline
14. न पव॑मानेभ्यः॒ पव॑मानेभ्यो॒ न न पव॑माने॒भ्यो ऽवाव॒ पव॑मानेभ्यो॒ न न पव॑माने॒भ्यो ऽव॑ । \newline
15. पव॑माने॒भ्यो ऽवाव॒ पव॑मानेभ्यः॒ पव॑माने॒भ्यो ऽव॑ च्छिद्यते छिद्य॒ते ऽव॒ पव॑मानेभ्यः॒ पव॑माने॒भ्यो ऽव॑ च्छिद्यते । \newline
16. अव॑ च्छिद्यते छिद्य॒ते ऽवाव॑ च्छिद्यते॒ यो य श्छि॑द्य॒ते ऽवाव॑ च्छिद्यते॒ यः । \newline
17. छि॒द्य॒ते॒ यो य श्छि॑द्यते छिद्यते॒ यो वै वै य श्छि॑द्यते छिद्यते॒ यो वै । \newline
18. यो वै वै यो यो वै पव॑मानस्य॒ पव॑मानस्य॒ वै यो यो वै पव॑मानस्य । \newline
19. वै पव॑मानस्य॒ पव॑मानस्य॒ वै वै पव॑मानस्य॒ सन्त॑तिꣳ॒॒ सन्त॑ति॒म् पव॑मानस्य॒ वै वै पव॑मानस्य॒ सन्त॑तिम् । \newline
20. पव॑मानस्य॒ सन्त॑तिꣳ॒॒ सन्त॑ति॒म् पव॑मानस्य॒ पव॑मानस्य॒ सन्त॑तिं॒ ॅवेद॒ वेद॒ सन्त॑ति॒म् पव॑मानस्य॒ पव॑मानस्य॒ सन्त॑तिं॒ ॅवेद॑ । \newline
21. सन्त॑तिं॒ ॅवेद॒ वेद॒ सन्त॑तिꣳ॒॒ सन्त॑तिं॒ ॅवेद॒ सर्वꣳ॒॒ सर्वं॒ ॅवेद॒ सन्त॑तिꣳ॒॒ सन्त॑तिं॒ ॅवेद॒ सर्व᳚म् । \newline
22. सन्त॑ति॒मिति॒ सं - त॒ति॒म् । \newline
23. वेद॒ सर्वꣳ॒॒ सर्वं॒ ॅवेद॒ वेद॒ सर्व॒ मायु॒ रायुः॒ सर्वं॒ ॅवेद॒ वेद॒ सर्व॒ मायुः॑ । \newline
24. सर्व॒ मायु॒ रायुः॒ सर्वꣳ॒॒ सर्व॒ मायु॑ रेत्ये॒ त्यायुः॒ सर्वꣳ॒॒ सर्व॒ मायु॑ रेति । \newline
25. आयु॑ रेत्ये॒ त्यायु॒ रायु॑ रेति॒ न नैत्यायु॒ रायु॑ रेति॒ न । \newline
26. ए॒ति॒ न नैत्ये॑ति॒ न पु॒रा पु॒रा नैत्ये॑ति॒ न पु॒रा । \newline
27. न पु॒रा पु॒रा न न पु॒रा ऽऽयु॑ष॒ आयु॑षः पु॒रा न न पु॒रा ऽऽयु॑षः । \newline
28. पु॒रा ऽऽयु॑ष॒ आयु॑षः पु॒रा पु॒रा ऽऽयु॑षः॒ प्र प्रायु॑षः पु॒रा पु॒रा ऽऽयु॑षः॒ प्र । \newline
29. आयु॑षः॒ प्र प्रायु॑ष॒ आयु॑षः॒ प्र मी॑यते मीयते॒ प्रायु॑ष॒ आयु॑षः॒ प्र मी॑यते । \newline
30. प्र मी॑यते मीयते॒ प्र प्र मी॑यते पशु॒मान् प॑शु॒मान् मी॑यते॒ प्र प्र मी॑यते पशु॒मान् । \newline
31. मी॒य॒ते॒ प॒शु॒मान् प॑शु॒मान् मी॑यते मीयते पशु॒मान् भ॑वति भवति पशु॒मान् मी॑यते मीयते पशु॒मान् भ॑वति । \newline
32. प॒शु॒मान् भ॑वति भवति पशु॒मान् प॑शु॒मान् भ॑वति वि॒न्दते॑ वि॒न्दते॑ भवति पशु॒मान् प॑शु॒मान् भ॑वति वि॒न्दते᳚ । \newline
33. प॒शु॒मानिति॑ पशु - मान् । \newline
34. भ॒व॒ति॒ वि॒न्दते॑ वि॒न्दते॑ भवति भवति वि॒न्दते᳚ प्र॒जाम् प्र॒जां ॅवि॒न्दते॑ भवति भवति वि॒न्दते᳚ प्र॒जाम् । \newline
35. वि॒न्दते᳚ प्र॒जाम् प्र॒जां ॅवि॒न्दते॑ वि॒न्दते᳚ प्र॒जाम् पव॑मानस्य॒ पव॑मानस्य प्र॒जां ॅवि॒न्दते॑ वि॒न्दते᳚ प्र॒जाम् पव॑मानस्य । \newline
36. प्र॒जाम् पव॑मानस्य॒ पव॑मानस्य प्र॒जाम् प्र॒जाम् पव॑मानस्य॒ ग्रहा॒ ग्रहाः॒ पव॑मानस्य प्र॒जाम् प्र॒जाम् पव॑मानस्य॒ ग्रहाः᳚ । \newline
37. प्र॒जामिति॑ प्र - जाम् । \newline
38. पव॑मानस्य॒ ग्रहा॒ ग्रहाः॒ पव॑मानस्य॒ पव॑मानस्य॒ ग्रहा॑ गृह्यन्ते गृह्यन्ते॒ ग्रहाः॒ पव॑मानस्य॒ पव॑मानस्य॒ ग्रहा॑ गृह्यन्ते । \newline
39. ग्रहा॑ गृह्यन्ते गृह्यन्ते॒ ग्रहा॒ ग्रहा॑ गृह्य॒न्ते ऽथाथ॑ गृह्यन्ते॒ ग्रहा॒ ग्रहा॑ गृह्य॒न्ते ऽथ॑ । \newline
40. गृ॒ह्य॒न्ते ऽथाथ॑ गृह्यन्ते गृह्य॒न्ते ऽथ॒ वै वा अथ॑ गृह्यन्ते गृह्य॒न्ते ऽथ॒ वै । \newline
41. अथ॒ वै वा अथाथ॒ वा अ॑स्यास्य॒ वा अथाथ॒ वा अ॑स्य । \newline
42. वा अ॑स्यास्य॒ वै वा अ॑स्यै॒त ए॒ते᳚ ऽस्य॒ वै वा अ॑स्यै॒ते । \newline
43. अ॒स्यै॒त ए॒ते᳚ ऽस्या स्यै॒ते ऽगृ॑हीता॒ अगृ॑हीता ए॒ते᳚ ऽस्या स्यै॒ते ऽगृ॑हीताः । \newline
44. ए॒ते ऽगृ॑हीता॒ अगृ॑हीता ए॒त ए॒ते ऽगृ॑हीता द्रोणकल॒शो द्रो॑णकल॒शो ऽगृ॑हीता ए॒त ए॒ते ऽगृ॑हीता द्रोणकल॒शः । \newline
45. अगृ॑हीता द्रोणकल॒शो द्रो॑णकल॒शो ऽगृ॑हीता॒ अगृ॑हीता द्रोणकल॒श आ॑धव॒नीय॑ आधव॒नीयो᳚ द्रोणकल॒शो ऽगृ॑हीता॒ अगृ॑हीता द्रोणकल॒श आ॑धव॒नीयः॑ । \newline
46. द्रो॒ण॒क॒ल॒श आ॑धव॒नीय॑ आधव॒नीयो᳚ द्रोणकल॒शो द्रो॑णकल॒श आ॑धव॒नीयः॑ पूत॒भृत् पू॑त॒भृ दा॑धव॒नीयो᳚ द्रोणकल॒शो द्रो॑णकल॒श आ॑धव॒नीयः॑ पूत॒भृत् । \newline
47. द्रो॒ण॒क॒ल॒श इति॑ द्रोण - क॒ल॒शः । \newline
48. आ॒ध॒व॒नीयः॑ पूत॒भृत् पू॑त॒भृ दा॑धव॒नीय॑ आधव॒नीयः॑ पूत॒भृत् ताꣳ स्तान् पू॑त॒भृ दा॑धव॒नीय॑ आधव॒नीयः॑ पूत॒भृत् तान् । \newline
49. आ॒ध॒व॒नीय॒ इत्या᳚ - ध॒व॒नीयः॑ । \newline
50. पू॒त॒भृत् ताꣳ स्तान् पू॑त॒भृत् पू॑त॒भृत् तान्. यद् यत् तान् पू॑त॒भृत् पू॑त॒भृत् तान्. यत् । \newline
51. पू॒त॒भृदिति॑ पूत - भृत् । \newline
52. तान्. यद् यत् ताꣳ स्तान्. यदगृ॑ही॒त्वा ऽगृ॑हीत्वा॒ यत् ताꣳ स्तान्. यदगृ॑हीत्वा । \newline
53. यदगृ॑ही॒त्वा ऽगृ॑हीत्वा॒ यद् यदगृ॑ही त्वोपाकु॒र्या दु॑पाकु॒र्या दगृ॑हीत्वा॒ यद् यदगृ॑ही त्वोपाकु॒र्यात् । \newline
54. अगृ॑ही त्वोपाकु॒र्या दु॑पाकु॒र्या दगृ॑ही॒त्वा ऽगृ॑ही त्वोपाकु॒र्यात् पव॑मान॒म् पव॑मान मुपाकु॒र्या दगृ॑ही॒त्वा ऽगृ॑ही त्वोपाकु॒र्यात् पव॑मानम् । \newline
55. उ॒पा॒कु॒र्यात् पव॑मान॒म् पव॑मान मुपाकु॒र्या दु॑पाकु॒र्यात् पव॑मानं॒ ॅवि वि पव॑मान मुपाकु॒र्या दु॑पाकु॒र्यात् पव॑मानं॒ ॅवि । \newline
56. उ॒पा॒कु॒र्यादित्यु॑प - आ॒कु॒र्यात् । \newline
57. पव॑मानं॒ ॅवि वि पव॑मान॒म् पव॑मानं॒ ॅवि च्छि॑न्द्या च्छिन्द्या॒द् वि पव॑मान॒म् पव॑मानं॒ 
ॅवि च्छि॑न्द्यात् । \newline
58. वि च्छि॑न्द्या च्छिन्द्या॒द् वि वि च्छि॑न्द्या॒त् तम् तम् छि॑न्द्या॒द् वि वि च्छि॑न्द्या॒त् तम् । \newline
\pagebreak
\markright{ TS 3.2.1.3  \hfill https://www.vedavms.in \hfill}

\section{ TS 3.2.1.3 }

\textbf{TS 3.2.1.3 } \newline
\textbf{Samhita Paata} \newline

च्छि॑न्द्या॒त् तं ॅवि॒च्छिद्य॑मानमद्ध्व॒र्योः प्रा॒णोऽनु॒ विच्छि॑द्ये-तोपया॒मगृ॑हीतोऽसि प्र॒जाप॑तये॒ त्वेति॑ द्रोणकल॒शम॒भि मृ॑शे॒दिन्द्रा॑य॒ त्वेत्या॑धव॒नीयं॒ ॅविश्वे᳚भ्यस्त्वा दे॒वेभ्य॒ इति॑ पूत॒भृतं॒ पव॑मानमे॒व तथ् सं त॑नोति॒ सर्व॒मायु॑रेति॒ न पु॒राऽऽयु॑षः॒ प्रमी॑यते पशु॒मान् भ॑वति वि॒न्दते᳚ प्र॒जां ॥ \newline

\textbf{Pada Paata} \newline

छि॒न्द्या॒त् । तम् । वि॒च्छिद्य॑मान॒मिति॑ वि - छिद्य॑मानम् । अ॒द्ध्व॒र्योः । प्रा॒ण इति॑ प्र -अ॒नः । अनु॑ । वीति॑ । छि॒द्ये॒त॒ । उ॒प॒या॒मगृ॑हीत॒ इत्युप॑या॒म - गृ॒ही॒तः॒ । अ॒सि॒ । प्र॒जाप॑तय॒ इति॑ प्र॒जा-प॒त॒ये॒ । त्वा॒ । इति॑ । द्रो॒ण॒क॒ल॒शमिति॑ द्रोण - क॒ल॒शम् । अ॒भीति॑ । मृ॒शे॒त् । इन्द्रा॑य । त्वा॒ । इति॑ । आ॒ध॒व॒नीय॒मित्या᳚ - ध॒व॒नीय᳚म् । विश्वे᳚भ्यः । त्वा॒ । दे॒वेभ्यः॑ । इति॑ । पू॒त॒भृत॒मिति॑ पूत - भृत᳚म् । पव॑मानम् । ए॒व । तत् । समिति॑ । त॒नो॒ति॒ । सर्व᳚म् । आयुः॑ । ए॒ति॒ । न । पु॒रा । आयु॑षः । प्रेति॑ । मी॒य॒ते॒ । प॒शु॒मानिति॑ पशु - मान् । भ॒व॒ति॒ । वि॒न्दते᳚ । प्र॒जामिति॑ प्र - जाम् ॥  \newline


\textbf{Krama Paata} \newline

छि॒न्द्या॒त् तम् । तं ॅवि॒च्छिद्य॑मानम् । वि॒च्छिद्य॑मानमद्ध्व॒र्योः । वि॒च्छिद्य॑मान॒मिति॑ वि - छिद्य॑मानम् । अ॒द्ध्व॒र्योः प्रा॒णः । प्रा॒णोऽनु॑ । प्रा॒ण इति॑ प्र - अ॒नः । अनु॒ वि । वि च्छि॑द्येत । छि॒द्ये॒तो॒प॒या॒मगृ॑हीतः । उ॒प॒या॒मगृ॑हीतोऽसि । उ॒प॒या॒मगृ॑हीत॒ इत्यु॑पया॒म - गृ॒ही॒तः॒ । अ॒सि॒ प्र॒जाप॑तये । प्र॒जाप॑तये त्वा । प्र॒जाप॑तय॒ इति॑ प्र॒जा - प॒त॒ये॒ । त्वेति॑ । इति॑ द्रोणकल॒शम् । द्रो॒ण॒क॒ल॒शम॒भि । द्रो॒ण॒क॒ल॒शमिति॑ द्रोण - क॒ल॒शम् । अ॒भि मृ॑शेत् । मृ॒शे॒दिन्द्रा॑य । इन्द्रा॑य त्वा । त्वेति॑ । इत्या॑धव॒नीय᳚म् । आ॒ध॒व॒नीयं॒ ॅविश्वे᳚भ्यः । आ॒ध॒व॒नीय॒मित्या᳚ - ध॒व॒नीय᳚म् । विश्वे᳚भ्यस्त्वा । त्वा॒ दे॒वेभ्यः॑ । दे॒वेभ्य॒ इति॑ । इति॑ पूत॒भृत᳚म् । पू॒त॒भृत॒म् पव॑मानम् । पू॒त॒भृत॒मिति॑ पूत - भृत᳚म् । पव॑मानमे॒व । ए॒व तत् । तथ् सम् । सम् त॑नोति । त॒नो॒ति॒ सर्व᳚म् । सर्व॒मायुः॑ । आयु॑रेति । ए॒ति॒ न । न पु॒रा । 
पु॒रा ऽऽयु॑षः । आयु॑षः॒ प्र । प्र मी॑यते । मी॒य॒ते॒ प॒शु॒मान् । प॒शु॒मान् भ॑वति । प॒शु॒मानिति॑ पशु - मान् । भ॒व॒ति॒ वि॒न्दते᳚ । वि॒न्दते᳚ प्र॒जाम् । प्र॒जामिति॑ प्र - जाम् । \newline

\textbf{Jatai Paata} \newline

1. छि॒न्द्या॒त् तम् तम् छि॑न्द्याच् छिन्द्या॒त् तम् । \newline
2. तं ॅवि॒च्छिद्य॑मानं ॅवि॒च्छिद्य॑मान॒म् तम् तं ॅवि॒च्छिद्य॑मानम् । \newline
3. वि॒च्छिद्य॑मान मद्ध्व॒र्यो र॑द्ध्व॒र्योर् वि॒च्छिद्य॑मानं ॅवि॒च्छिद्य॑मान मद्ध्व॒र्योः । \newline
4. वि॒च्छिद्य॑मान॒मिति॑ वि - छिद्य॑मानम् । \newline
5. अ॒द्ध्व॒र्योः प्रा॒णः प्रा॒णो᳚ ऽद्ध्व॒र्यो र॑द्ध्व॒र्योः प्रा॒णः । \newline
6. प्रा॒णो ऽन्वनु॑ प्रा॒णः प्रा॒णो ऽनु॑ । \newline
7. प्रा॒ण इति॑ प्र - अ॒नः । \newline
8. अनु॒ वि व्यन्वनु॒ वि । \newline
9. विच् छि॑द्येत छिद्येत॒ वि विच् छि॑द्येत । \newline
10. छि॒द्ये॒ तो॒प॒या॒मगृ॑हीत उपया॒मगृ॑हीत श्छिद्येत छिद्ये तोपया॒मगृ॑हीतः । \newline
11. उ॒प॒या॒मगृ॑हीतो ऽस्य स्युपया॒मगृ॑हीत उपया॒मगृ॑हीतो ऽसि । \newline
12. उ॒प॒या॒मगृ॑हीत॒ इत्युप॑या॒म - गृ॒ही॒तः॒ । \newline
13. अ॒सि॒ प्र॒जाप॑तये प्र॒जाप॑तये ऽस्यसि प्र॒जाप॑तये । \newline
14. प्र॒जाप॑तये त्वा त्वा प्र॒जाप॑तये प्र॒जाप॑तये त्वा । \newline
15. प्र॒जाप॑तय॒ इति॑ प्र॒जा - प॒त॒ये॒ । \newline
16. त्वेतीति॑ त्वा॒ त्वेति॑ । \newline
17. इति॑ द्रोणकल॒शम् द्रो॑णकल॒श मितीति॑ द्रोणकल॒शम् । \newline
18. द्रो॒ण॒क॒ल॒श म॒भ्य॑भि द्रो॑णकल॒शम् द्रो॑णकल॒श म॒भि । \newline
19. द्रो॒ण॒क॒ल॒शमिति॑ द्रोण - क॒ल॒शम् । \newline
20. अ॒भि मृ॑शेन् मृशे द॒भ्य॑भि मृ॑शेत् । \newline
21. मृ॒शे॒ दिन्द्रा॒ये न्द्रा॑य मृशेन् मृशे॒ दिन्द्रा॑य । \newline
22. इन्द्रा॑य त्वा॒ त्वेन्द्रा॒ये न्द्रा॑य त्वा । \newline
23. त्वेतीति॑ त्वा॒ त्वेति॑ । \newline
24. इत्या॑धव॒नीय॑ माधव॒नीय॒ मिती त्या॑धव॒नीय᳚म् । \newline
25. आ॒ध॒व॒नीयं॒ ॅविश्वे᳚भ्यो॒ विश्वे᳚भ्य आधव॒नीय॑ माधव॒नीयं॒ ॅविश्वे᳚भ्यः । \newline
26. आ॒ध॒व॒नीय॒मित्या᳚ - ध॒व॒नीय᳚म् । \newline
27. विश्वे᳚भ्य स्त्वा त्वा॒ विश्वे᳚भ्यो॒ विश्वे᳚भ्य स्त्वा । \newline
28. त्वा॒ दे॒वेभ्यो॑ दे॒वेभ्य॑ स्त्वा त्वा दे॒वेभ्यः॑ । \newline
29. दे॒वेभ्य॒ इतीति॑ दे॒वेभ्यो॑ दे॒वेभ्य॒ इति॑ । \newline
30. इति॑ पूत॒भृत॑म् पूत॒भृत॒ मितीति॑ पूत॒भृत᳚म् । \newline
31. पू॒त॒भृत॒म् पव॑मान॒म् पव॑मानम् पूत॒भृत॑म् पूत॒भृत॒म् पव॑मानम् । \newline
32. पू॒त॒भृत॒मिति॑ पूत - भृत᳚म् । \newline
33. पव॑मान मे॒वैव पव॑मान॒म् पव॑मान मे॒व । \newline
34. ए॒व तत् तदे॒वैव तत् । \newline
35. तथ् सꣳ सम् तत् तथ् सम् । \newline
36. सम् त॑नोति तनोति॒ सꣳ सम् त॑नोति । \newline
37. त॒नो॒ति॒ सर्वꣳ॒॒ सर्व॑म् तनोति तनोति॒ सर्व᳚म् । \newline
38. सर्व॒ मायु॒ रायुः॒ सर्वꣳ॒॒ सर्व॒ मायुः॑ । \newline
39. आयु॑ रेत्ये॒ त्यायु॒ रायु॑रेति । \newline
40. ए॒ति॒ न नैत्ये॑ति॒ न । \newline
41. न पु॒रा पु॒रा न न पु॒रा । \newline
42. पु॒रा ऽऽयु॑ष॒ आयु॑षः पु॒रा पु॒रा ऽऽयु॑षः । \newline
43. आयु॑षः॒ प्र प्रायु॑ष॒ आयु॑षः॒ प्र । \newline
44. प्र मी॑यते मीयते॒ प्र प्र मी॑यते । \newline
45. मी॒य॒ते॒ प॒शु॒मान् प॑शु॒मान् मी॑यते मीयते पशु॒मान् । \newline
46. प॒शु॒मान् भ॑वति भवति पशु॒मान् प॑शु॒मान् भ॑वति । \newline
47. प॒शु॒मानिति॑ पशु - मान् । \newline
48. भ॒व॒ति॒ वि॒न्दते॑ वि॒न्दते॑ भवति भवति वि॒न्दते᳚ । \newline
49. वि॒न्दते᳚ प्र॒जाम् प्र॒जां ॅवि॒न्दते॑ वि॒न्दते᳚ प्र॒जाम् । \newline
50. प्र॒जामिति॑ प्र - जाम् । \newline

\textbf{Ghana Paata } \newline

1. छि॒न्द्या॒त् तम् तम् छि॑न्द्या च्छिन्द्या॒त् तं ॅवि॒च्छिद्य॑मानं ॅवि॒च्छिद्य॑मान॒म् तम् छि॑न्द्या च्छिन्द्या॒त् तं ॅवि॒च्छिद्य॑मानम् । \newline
2. तं ॅवि॒च्छिद्य॑मानं ॅवि॒च्छिद्य॑मान॒म् तम् तं ॅवि॒च्छिद्य॑मान मद्ध्व॒र्यो र॑द्ध्व॒र्योर् वि॒च्छिद्य॑मान॒म् तम् तं ॅवि॒च्छिद्य॑मान मद्ध्व॒र्योः । \newline
3. वि॒च्छिद्य॑मान मद्ध्व॒र्यो र॑द्ध्व॒र्योर् वि॒च्छिद्य॑मानं ॅवि॒च्छिद्य॑मान मद्ध्व॒र्योः प्रा॒णः प्रा॒णो᳚ ऽद्ध्व॒र्योर् वि॒च्छिद्य॑मानं ॅवि॒च्छिद्य॑मान मद्ध्व॒र्योः प्रा॒णः । \newline
4. वि॒च्छिद्य॑मान॒मिति॑ वि - छिद्य॑मानम् । \newline
5. अ॒द्ध्व॒र्योः प्रा॒णः प्रा॒णो᳚ ऽद्ध्व॒र्यो र॑द्ध्व॒र्योः प्रा॒णो ऽन्वनु॑ प्रा॒णो᳚ ऽद्ध्व॒र्यो र॑द्ध्व॒र्योः प्रा॒णो ऽनु॑ । \newline
6. प्रा॒णो ऽन्वनु॑ प्रा॒णः प्रा॒णो ऽनु॒ वि व्यनु॑ प्रा॒णः प्रा॒णो ऽनु॒ वि । \newline
7. प्रा॒ण इति॑ प्र - अ॒नः । \newline
8. अनु॒ वि व्यन्वनु॒ वि च्छि॑द्येत छिद्येत॒ व्यन्वनु॒ वि च्छि॑द्येत । \newline
9. वि च्छि॑द्येत छिद्येत॒ वि वि च्छि॑द्येतोपया॒मगृ॑हीत उपया॒मगृ॑हीत श्छिद्येत॒ वि वि च्छि॑द्येतोपया॒मगृ॑हीतः । \newline
10. छि॒द्ये॒तो॒प॒या॒मगृ॑हीत उपया॒मगृ॑हीत श्छिद्येत छिद्येतोपया॒मगृ॑हीतो ऽस्य स्युपया॒मगृ॑हीत श्छिद्येत छिद्येतोपया॒मगृ॑हीतो ऽसि । \newline
11. उ॒प॒या॒मगृ॑हीतो ऽस्य स्युपया॒मगृ॑हीत उपया॒मगृ॑हीतो ऽसि प्र॒जाप॑तये प्र॒जाप॑तये ऽस्युपया॒मगृ॑हीत उपया॒मगृ॑हीतो ऽसि प्र॒जाप॑तये । \newline
12. उ॒प॒या॒मगृ॑हीत॒ इत्युप॑या॒म - गृ॒ही॒तः॒ । \newline
13. अ॒सि॒ प्र॒जाप॑तये प्र॒जाप॑तये ऽस्यसि प्र॒जाप॑तये त्वा त्वा प्र॒जाप॑तये ऽस्यसि प्र॒जाप॑तये त्वा । \newline
14. प्र॒जाप॑तये त्वा त्वा प्र॒जाप॑तये प्र॒जाप॑तये॒ त्वेतीति॑ त्वा प्र॒जाप॑तये प्र॒जाप॑तये॒ त्वेति॑ । \newline
15. प्र॒जाप॑तय॒ इति॑ प्र॒जा - प॒त॒ये॒ । \newline
16. त्वेतीति॑ त्वा॒ त्वेति॑ द्रोणकल॒शम् द्रो॑णकल॒श मिति॑ त्वा॒ त्वेति॑ द्रोणकल॒शम् । \newline
17. इति॑ द्रोणकल॒शम् द्रो॑णकल॒श मितीति॑ द्रोणकल॒श म॒भ्य॑भि द्रो॑णकल॒श मितीति॑ द्रोणकल॒श म॒भि । \newline
18. द्रो॒ण॒क॒ल॒श म॒भ्य॑भि द्रो॑णकल॒शम् द्रो॑णकल॒श म॒भि मृ॑शेन् मृशेद॒भि द्रो॑णकल॒शम् द्रो॑णकल॒श म॒भि मृ॑शेत् । \newline
19. द्रो॒ण॒क॒ल॒शमिति॑ द्रोण - क॒ल॒शम् । \newline
20. अ॒भि मृ॑शेन् मृशे द॒भ्य॑भि मृ॑शे॒ दिन्द्रा॒ये न्द्रा॑य मृशे द॒भ्य॑भि मृ॑शे॒ दिन्द्रा॑य । \newline
21. मृ॒शे॒ दिन्द्रा॒ये न्द्रा॑य मृशेन् मृशे॒ दिन्द्रा॑य त्वा॒ त्वेन्द्रा॑य मृशेन् मृशे॒ दिन्द्रा॑य त्वा । \newline
22. इन्द्रा॑य त्वा॒ त्वेन्द्रा॒ये न्द्रा॑य॒ त्वेतीति॒ त्वेन्द्रा॒ये न्द्रा॑य॒ त्वेति॑ । \newline
23. त्वेतीति॑ त्वा॒ त्वेत्या॑धव॒नीय॑ माधव॒नीय॒ मिति॑ त्वा॒ त्वेत्या॑धव॒नीय᳚म् । \newline
24. इत्या॑धव॒नीय॑ माधव॒नीय॒ मिती त्या॑धव॒नीयं॒ ॅविश्वे᳚भ्यो॒ विश्वे᳚भ्य आधव॒नीय॒ मिती त्या॑धव॒नीयं॒ ॅविश्वे᳚भ्यः । \newline
25. आ॒ध॒व॒नीयं॒ ॅविश्वे᳚भ्यो॒ विश्वे᳚भ्य आधव॒नीय॑ माधव॒नीयं॒ ॅविश्वे᳚भ्य स्त्वा त्वा॒ विश्वे᳚भ्य आधव॒नीय॑ माधव॒नीयं॒ ॅविश्वे᳚भ्य स्त्वा । \newline
26. आ॒ध॒व॒नीय॒मित्या᳚ - ध॒व॒नीय᳚म् । \newline
27. विश्वे᳚भ्य स्त्वा त्वा॒ विश्वे᳚भ्यो॒ विश्वे᳚भ्य स्त्वा दे॒वेभ्यो॑ दे॒वेभ्य॑ स्त्वा॒ विश्वे᳚भ्यो॒ विश्वे᳚भ्य स्त्वा दे॒वेभ्यः॑ । \newline
28. त्वा॒ दे॒वेभ्यो॑ दे॒वेभ्य॑ स्त्वा त्वा दे॒वेभ्य॒ इतीति॑ दे॒वेभ्य॑ स्त्वा त्वा दे॒वेभ्य॒ इति॑ । \newline
29. दे॒वेभ्य॒ इतीति॑ दे॒वेभ्यो॑ दे॒वेभ्य॒ इति॑ पूत॒भृत॑म् पूत॒भृत॒ मिति॑ दे॒वेभ्यो॑ दे॒वेभ्य॒ इति॑ पूत॒भृत᳚म् । \newline
30. इति॑ पूत॒भृत॑म् पूत॒भृत॒ मितीति॑ पूत॒भृत॒म् पव॑मान॒म् पव॑मानम् पूत॒भृत॒ मितीति॑ पूत॒भृत॒म् पव॑मानम् । \newline
31. पू॒त॒भृत॒म् पव॑मान॒म् पव॑मानम् पूत॒भृत॑म् पूत॒भृत॒म् पव॑मान मे॒वैव पव॑मानम् पूत॒भृत॑म् पूत॒भृत॒म् पव॑मान मे॒व । \newline
32. पू॒त॒भृत॒मिति॑ पूत - भृत᳚म् । \newline
33. पव॑मान मे॒वैव पव॑मान॒म् पव॑मान मे॒व तत् तदे॒व पव॑मान॒म् पव॑मान मे॒व तत् । \newline
34. ए॒व तत् तदे॒वैव तथ् सꣳ सम् तदे॒वैव त थ्सम् । \newline
35. त थ्सꣳ सम् तत् तथ् सम् त॑नोति तनोति॒ सम् तत् तथ् सम् त॑नोति । \newline
36. सम् त॑नोति तनोति॒ सꣳ सम् त॑नोति॒ सर्वꣳ॒॒ सर्व॑म् तनोति॒ सꣳ सम् त॑नोति॒ सर्व᳚म् । \newline
37. त॒नो॒ति॒ सर्वꣳ॒॒ सर्व॑म् तनोति तनोति॒ सर्व॒ मायु॒ रायुः॒ सर्व॑म् तनोति तनोति॒ सर्व॒ मायुः॑ । \newline
38. सर्व॒ मायु॒ रायुः॒ सर्वꣳ॒॒ सर्व॒ मायु॑ रेत्ये॒ त्यायुः॒ सर्वꣳ॒॒ सर्व॒ मायु॑ रेति । \newline
39. आयु॑ रेत्ये॒ त्यायु॒ रायु॑ रेति॒ न नैत्यायु॒ रायु॑ रेति॒ न । \newline
40. ए॒ति॒ न नैत्ये॑ति॒ न पु॒रा पु॒रा नैत्ये॑ति॒ न पु॒रा । \newline
41. न पु॒रा पु॒रा न न पु॒रा ऽऽयु॑ष॒ आयु॑षः पु॒रा न न पु॒रा ऽऽयु॑षः । \newline
42. पु॒रा ऽऽयु॑ष॒ आयु॑षः पु॒रा पु॒रा ऽऽयु॑षः॒ प्र प्रायु॑षः पु॒रा पु॒रा ऽऽयु॑षः॒ प्र । \newline
43. आयु॑षः॒ प्र प्रायु॑ष॒ आयु॑षः॒ प्र मी॑यते मीयते॒ प्रायु॑ष॒ आयु॑षः॒ प्र मी॑यते । \newline
44. प्र मी॑यते मीयते॒ प्र प्र मी॑यते पशु॒मान् प॑शु॒मान् मी॑यते॒ प्र प्र मी॑यते पशु॒मान् । \newline
45. मी॒य॒ते॒ प॒शु॒मान् प॑शु॒मान् मी॑यते मीयते पशु॒मान् भ॑वति भवति पशु॒मान् मी॑यते मीयते पशु॒मान् भ॑वति । \newline
46. प॒शु॒मान् भ॑वति भवति पशु॒मान् प॑शु॒मान् भ॑वति वि॒न्दते॑ वि॒न्दते॑ भवति पशु॒मान् प॑शु॒मान् भ॑वति वि॒न्दते᳚ । \newline
47. प॒शु॒मानिति॑ पशु - मान् । \newline
48. भ॒व॒ति॒ वि॒न्दते॑ वि॒न्दते॑ भवति भवति वि॒न्दते᳚ प्र॒जाम् प्र॒जां ॅवि॒न्दते॑ भवति भवति वि॒न्दते᳚ प्र॒जाम् । \newline
49. वि॒न्दते᳚ प्र॒जाम् प्र॒जां ॅवि॒न्दते॑ वि॒न्दते᳚ प्र॒जाम् । \newline
50. प्र॒जामिति॑ प्र - जाम् । \newline
\pagebreak
\markright{ TS 3.2.2.1  \hfill https://www.vedavms.in \hfill}

\section{ TS 3.2.2.1 }

\textbf{TS 3.2.2.1 } \newline
\textbf{Samhita Paata} \newline

त्रीणि॒ वाव सव॑ना॒न्यथ॑ तृ॒तीयꣳ॒॒ सव॑न॒मव॑ लुम्पन्त्यनꣳ॒॒शु कु॒र्वन्त॑ उपाꣳ॒॒शुꣳहु॒त्वोपाꣳ॑शुपा॒त्रेऽꣳ॑शुम॒वास्य॒ तन्तृ॑तीयसव॒ने॑ ऽपि॒सृज्या॒भि षु॑णुया॒द्यदा᳚प्या॒यय॑ति॒ तेनाꣳ॑शु॒मद्यद॑भिषु॒णोति॒ तेन॑र्जी॒षि सर्वा᳚ण्ये॒व तथ् सव॑नान्यꣳशु॒मन्ति॑ शु॒क्रव॑न्ति स॒माव॑द्वीर्याणि करोति॒ द्वौ स॑मु॒द्रौ वित॑तावजू॒र्यौ प॒र्याव॑र्तेते ज॒ठरे॑व॒ पादाः᳚ । तयोः॒ पश्य॑न्तो॒ अति॑ यन्त्य॒न्यमप॑श्यन्तः॒ - [  ] \newline

\textbf{Pada Paata} \newline

त्रीणि॑ । वाव । सव॑नानि । अथ॑ । तृ॒तीय᳚म् । सव॑नम् । अवेति॑ । लु॒पं॒न्ति॒ । अ॒नꣳ॒॒शु । कु॒र्वन्तः॑ । उ॒पाꣳ॒॒शुमित्यु॑प - अꣳ॒॒शुम् । हु॒त्वा । उ॒पाꣳ॒॒शु॒पा॒त्र इत्यु॑पाꣳशु - पा॒त्रे । अꣳ॒॒शुम् । अ॒वास्येत्य॑व - अस्य॑ । तम् । तृ॒ती॒य॒स॒व॒न इति॑ तृतीय - स॒व॒ने । अ॒पि॒सृज्येत्य॑पि - सृज्य॑ । अ॒भीति॑ । सु॒नु॒या॒त् । यत् । आ॒प्या॒यय॒तीया᳚ - प्या॒यय॑ति । तेन॑ । अꣳ॒॒शु॒मदित्यꣳ॑शु-मत् । यत् । अ॒भि॒षु॒णोतीत्य॑भि-सु॒नोति॑ । तेन॑ । ऋ॒जी॒षि । सर्वा॑णि । ए॒व । तत् । सव॑नानि । अꣳ॒॒शु॒मन्तीत्यꣳ॑शु - मन्ति॑ । शु॒क्रव॒न्तीति॑ शु॒क्र - व॒न्ति॒ । स॒माव॑द् वीर्या॒णीति॑ स॒माव॑त् - वी॒र्या॒णि॒ । क॒रो॒ति॒ । द्वौ । स॒मु॒द्रौ । वित॑ता॒विति॒ वि - त॒तौ॒ । अ॒जू॒र्यौ । प॒र्याव॑र्तेते॒ इति॑ परि-आव॑र्तेते । ज॒ठरा᳚ । इ॒व॒ । पादाः᳚ ॥ तयोः᳚ । पश्य॑न्तः । अतीति॑ । य॒न्ति॒ । अ॒न्यम् । अप॑श्यन्तः ।  \newline


\textbf{Krama Paata} \newline

त्रीणि॒ वाव । वाव सव॑नानि । सव॑ना॒न्यथ॑ । अथ॑ तृ॒तीय᳚म् । तृ॒तीयꣳ॒॒ सव॑नम् । सव॑न॒मव॑ । अव॑ लुम्पन्ति । लु॒म्प॒न्त्य॒नꣳ॒॒शु । अ॒नꣳ॒॒शु कु॒र्वन्तः॑ । कु॒र्वन्त॑ उपाꣳ॒॒शुम् । उ॒पाꣳ॒॒शुꣳ हु॒त्वा । उ॒पाꣳ॒॒शुमित्यु॑प - अꣳ॒॒शुम् । हु॒त्वोपाꣳ॑शुपा॒त्रे । उ॒पाꣳ॒॒शु॒पा॒त्रे ऽꣳ॑शुम् । उ॒पाꣳ॒॒शु॒पा॒त्र इत्यु॑पाꣳशु - पा॒त्रे । अꣳ॒॒शुम॒वास्य॑ । अ॒वास्य॒ तम् । अ॒वास्येत्य॑व - अस्य॑ । तम् तृ॑तीयसव॒ने । तृ॒ती॒य॒स॒व॒ने॑ऽपि॒सृज्य॑ । तृ॒ती॒य॒स॒व॒न इति॑ तृतीय - स॒व॒ने । अ॒पि॒सृज्या॒भि । अ॒पि॒सृज्येत्य॑पि - सृज्य॑ । अ॒भि षु॑णुयात् । सु॒नु॒या॒द् यत् । 
यदा᳚प्या॒यय॑ति । आ॒प्या॒यय॑ति॒ तेन॑ । आ॒प्या॒यय॒तीत्या᳚ - प्या॒यय॑ति । तेनाꣳ॑शु॒मत् । अꣳ॒॒शु॒मद् यत् । अꣳ॒॒शु॒मदित्यꣳ॑शु - मत् । यद॑भिषु॒णोति॑ । अ॒भि॒षु॒णोति॒ तेन॑ । अ॒भि॒॒षु॒णोतीत्य॑भि - सु॒नोति॑ । तेन॑र्जी॒षि । ऋ॒जी॒षि सर्वा॑णि । सर्वा᳚ण्ये॒व । ए॒व तत् । तथ् सव॑नानि । सव॑नान्यꣳशु॒मन्ति॑ । अꣳ॒॒शु॒मन्ति॑ शु॒क्रव॑न्ति । अꣳ॒॒शु॒मन्तीत्यꣳ॑शु - मन्ति॑ । शु॒क्रव॑न्ति स॒माव॑द्वीर्याणि । शु॒कव॒न्तीति॑ शु॒क्र - व॒न्ति॒ । स॒माव॑द्वीर्याणि करोति । स॒माव॑द्वीर्या॒णीति॑ स॒माव॑त् - वी॒र्या॒णि॒ । क॒रो॒ति॒ द्वौ । द्वौ स॑मु॒द्रौ । स॒मु॒द्रौ वित॑तौ । वित॑तावजू॒र्यौ । वित॑ता॒विति॒ वि - त॒तौ॒ । अ॒जू॒र्यौ प॒र्याव॑र्तेते । प॒र्याव॑र्तेते ज॒ठरा᳚ । प॒र्याव॑र्तेते॒ इति॑ परि - आव॑र्तेते । ज॒ठरे॑व । इ॒व॒ पादाः᳚ । पादा॒ इति॒ पादाः᳚ ॥ तयोः॒ पश्य॑न्तः । पश्य॑न्तो॒ अति॑ । अति॑ यन्ति । य॒न्त्य॒न्यम् । अ॒न्यमप॑श्यन्तः । अप॑श्यन्तः॒ सेतु॑ना \newline

\textbf{Jatai Paata} \newline

1. त्रीणि॒ वाव वाव त्रीणि॒ त्रीणि॒ वाव । \newline
2. वाव सव॑नानि॒ सव॑नानि॒ वाव वाव सव॑नानि । \newline
3. सव॑ ना॒न्यथाथ॒ सव॑नानि॒ सव॑ना॒ न्यथ॑ । \newline
4. अथ॑ तृ॒तीय॑म् तृ॒तीय॒ मथाथ॑ तृ॒तीय᳚म् । \newline
5. तृ॒तीयꣳ॒॒ सव॑नꣳ॒॒ सव॑नम् तृ॒तीय॑म् तृ॒तीयꣳ॒॒ सव॑नम् । \newline
6. सव॑न॒ मवाव॒ सव॑नꣳ॒॒ सव॑न॒ मव॑ । \newline
7. अव॑ लुंपन्ति लुंप॒ न्त्यवाव॑ लुंपन्ति । \newline
8. लुं॒प॒ न्त्य॒नꣳ॒॒श्व॑नꣳ॒॒शु लुं॑पन्ति लुंपन् त्यनꣳ॒॒शु । \newline
9. अ॒नꣳ॒॒शु कु॒र्वन्तः॑ कु॒र्वन्तो॑ ऽनꣳ॒॒ श्व॑नꣳ॒॒शु कु॒र्वन्तः॑ । \newline
10. कु॒र्वन्त॑ उपाꣳ॒॒शु मु॑पाꣳ॒॒शुम् कु॒र्वन्तः॑ कु॒र्वन्त॑ उपाꣳ॒॒शुम् । \newline
11. उ॒पाꣳ॒॒शुꣳ हु॒त्वा हु॒त्वोपाꣳ॒॒शु मु॑पाꣳ॒॒शुꣳ हु॒त्वा । \newline
12. उ॒पाꣳ॒॒शुमित्यु॑प - अꣳ॒॒शुम् । \newline
13. हु॒त्वोपाꣳ॑शुपा॒त्र उ॑पाꣳशुपा॒त्रे हु॒त्वा हु॒त्वोपाꣳ॑शुपा॒त्रे । \newline
14. उ॒पाꣳ॒॒शु॒पा॒त्रे ऽꣳ॑शु मꣳ॒॒शु मु॑पाꣳशुपा॒त्र उ॑पाꣳशुपा॒त्रे ऽꣳ॑शुम् । \newline
15. उ॒पाꣳ॒॒शु॒पा॒त्र इत्यु॑पाꣳशु - पा॒त्रे । \newline
16. अꣳ॒॒शु म॒वास्या॒ वास्याꣳ॒॒शु मꣳ॒॒शु म॒वास्य॑ । \newline
17. अ॒वास्य॒ तम् त म॒वास्या॒ वास्य॒ तम् । \newline
18. अ॒वास्येत्य॑व - अस्य॑ । \newline
19. तम् तृ॑तीयसव॒ने तृ॑तीयसव॒ने तम् तम् तृ॑तीयसव॒ने । \newline
20. तृ॒ती॒य॒स॒व॒ने॑ ऽपि॒सृज्या॑ पि॒सृज्य॑ तृतीयसव॒ने तृ॑तीयसव॒ने॑ ऽपि॒सृज्य॑ । \newline
21. तृ॒ती॒य॒स॒व॒न इति॑ तृतीय - स॒व॒ने । \newline
22. अ॒पि॒सृज्या॒ भ्या᳚(1॒)भ्य॑पि॒सृज्या॑ पि॒सृज्या॒भि । \newline
23. अ॒पि॒सृज्येत्य॑पि - सृज्य॑ । \newline
24. अ॒भि षु॑णुयाथ् सुनुया द॒भ्य॑भि षु॑णुयात् । \newline
25. सु॒नु॒या॒द् यद् यथ् सु॑नुयाथ् सुनुया॒द् यत् । \newline
26. यदा᳚प्या॒यय॑ त्याप्या॒यय॑ति॒ यद् यदा᳚प्या॒यय॑ति । \newline
27. आ॒प्या॒यय॑ति॒ तेन॒ तेना᳚प्या॒यय॑ त्याप्या॒यय॑ति॒ तेन॑ । \newline
28. आ॒प्या॒यय॒तीत्या᳚ - प्या॒यय॑ति । \newline
29. तेनाꣳ॑शु॒म दꣳ॑शु॒मत् तेन॒ तेनाꣳ॑शु॒मत् । \newline
30. अꣳ॒॒शु॒मद् यद् यदꣳ॑शु॒म दꣳ॑शु॒मद् यत् । \newline
31. अꣳ॒॒शु॒मदित्यꣳ॑शु - मत् । \newline
32. यद॑भिषु॒णो त्य॑भिषु॒णोति॒ यद् यद॑भिषु॒णोति॑ । \newline
33. अ॒भि॒षु॒णोति॒ तेन॒ तेना॑भिषु॒णो त्य॑भिषु॒णोति॒ तेन॑ । \newline
34. अ॒भि॒षु॒णोतीत्य॑भि - सु॒नोति॑ । \newline
35. तेन॑ र्जी॒ष्यृ॑जी॒षि तेन॒ तेन॑ र्जी॒षि । \newline
36. ऋ॒जी॒षि सर्वा॑णि॒ सर्वा᳚ ण्यृजी॒ ष्यृ॑जी॒षि सर्वा॑णि । \newline
37. सर्वा᳚ ण्ये॒वैव सर्वा॑णि॒ सर्वा᳚ण्ये॒व । \newline
38. ए॒व तत् तदे॒वैव तत् । \newline
39. तथ् सव॑नानि॒ सव॑नानि॒ तत् तथ् सव॑नानि । \newline
40. सव॑ना न्यꣳशु॒म न्त्यꣳ॑शु॒मन्ति॒ सव॑नानि॒ सव॑ना न्यꣳशु॒मन्ति॑ । \newline
41. अꣳ॒॒शु॒मन्ति॑ शु॒क्रव॑न्ति शु॒क्रव॑ न्त्यꣳशु॒म न्त्यꣳ॑शु॒मन्ति॑ शु॒क्रव॑न्ति । \newline
42. अꣳ॒॒शु॒मन्तीत्यꣳ॑शु - मन्ति॑ । \newline
43. शु॒क्रव॑न्ति स॒माव॑द्वीर्याणि स॒माव॑द्वीर्याणि शु॒क्रव॑न्ति शु॒क्रव॑न्ति स॒माव॑द्वीर्याणि । \newline
44. शु॒क्रव॒न्तीति॑ शु॒क्र - व॒न्ति॒ । \newline
45. स॒माव॑द्वीर्याणि करोति करोति स॒माव॑द्वीर्याणि स॒माव॑द्वीर्याणि करोति । \newline
46. स॒माव॑द्वीर्या॒णीति॑ स॒माव॑त् - वी॒र्या॒णि॒ । \newline
47. क॒रो॒ति॒ द्वौ द्वौ क॑रोति करोति॒ द्वौ । \newline
48. द्वौ स॑मु॒द्रौ स॑मु॒द्रौ द्वौ द्वौ स॑मु॒द्रौ । \newline
49. स॒मु॒द्रौ वित॑तौ॒ वित॑तौ समु॒द्रौ स॑मु॒द्रौ वित॑तौ । \newline
50. वित॑ता वजू॒र्या व॑जू॒र्यौ वित॑तौ॒ वित॑ता वजू॒र्यौ । \newline
51. वित॑ता॒विति॒ वि - त॒तौ॒ । \newline
52. अ॒जू॒र्यौ प॒र्याव॑र्तेते प॒र्याव॑र्तेते अजू॒र्या व॑जू॒र्यौ प॒र्याव॑र्तेते । \newline
53. प॒र्याव॑र्तेते ज॒ठरा॑ ज॒ठरा॑ प॒र्याव॑र्तेते प॒र्याव॑र्तेते ज॒ठरा᳚ । \newline
54. प॒र्याव॑र्तेते॒ इति॑ परि - आव॑र्तेते । \newline
55. ज॒ठ रे॑वे व ज॒ठरा॑ ज॒ठ रे॑व । \newline
56. इ॒व॒ पादाः॒ पादा॑ इवे व॒ पादाः᳚ । \newline
57. पादा॒ इति॒ पादाः᳚ । \newline
58. तयोः॒ पश्य॑न्तः॒ पश्य॑न्त॒ स्तयो॒ स्तयोः॒ पश्य॑न्तः । \newline
59. पश्य॑न्तो॒ अत्यति॒ पश्य॑न्तः॒ पश्य॑न्तो॒ अति॑ । \newline
60. अति॑ यन्ति य॒न्त्य त्यति॑ यन्ति । \newline
61. य॒न्त्य॒न्य म॒न्यं ॅय॑न्ति यन्त्य॒ न्यम् । \newline
62. अ॒न्य मप॑श्य॒न्तो ऽप॑श्यन्तो॒ ऽन्य म॒न्य मप॑श्यन्तः । \newline
63. अप॑श्यन्तः॒ सेतु॑ना॒ सेतु॒ना ऽप॑श्य॒न्तो ऽप॑श्यन्तः॒ सेतु॑ना । \newline

\textbf{Ghana Paata } \newline

1. त्रीणि॒ वाव वाव त्रीणि॒ त्रीणि॒ वाव सव॑नानि॒ सव॑नानि॒ वाव त्रीणि॒ त्रीणि॒ वाव सव॑नानि । \newline
2. वाव सव॑नानि॒ सव॑नानि॒ वाव वाव सव॑ना॒ न्यथाथ॒ सव॑नानि॒ वाव वाव सव॑ना॒ न्यथ॑ । \newline
3. सव॑ना॒ न्यथाथ॒ सव॑नानि॒ सव॑ना॒ न्यथ॑ तृ॒तीय॑म् तृ॒तीय॒ मथ॒ सव॑नानि॒ सव॑ना॒ न्यथ॑ तृ॒तीय᳚म् । \newline
4. अथ॑ तृ॒तीय॑म् तृ॒तीय॒ मथाथ॑ तृ॒तीयꣳ॒॒ सव॑नꣳ॒॒ सव॑नम् तृ॒तीय॒ मथाथ॑ तृ॒तीयꣳ॒॒ सव॑नम् । \newline
5. तृ॒तीयꣳ॒॒ सव॑नꣳ॒॒ सव॑नम् तृ॒तीय॑म् तृ॒तीयꣳ॒॒ सव॑न॒ मवाव॒ सव॑नम् तृ॒तीय॑म् तृ॒तीयꣳ॒॒ सव॑न॒ मव॑ । \newline
6. सव॑न॒ मवाव॒ सव॑नꣳ॒॒ सव॑न॒ मव॑ लुंपन्ति लुंप॒ न्त्यव॒ सव॑नꣳ॒॒ सव॑न॒ मव॑ लुंपन्ति । \newline
7. अव॑ लुंपन्ति लुंप॒ न्त्यवाव॑ लुंप न्त्यनꣳ॒॒ श्व॑नꣳ॒॒शु लुं॑प॒ न्त्यवाव॑ लुंप न्त्यनꣳ॒॒शु । \newline
8. लुं॒प॒ न्त्य॒नꣳ॒॒ श्व॑नꣳ॒॒शु लुं॑पन्ति लुंप न्त्यनꣳ॒॒शु कु॒र्वन्तः॑ कु॒र्वन्तो॑ ऽनꣳ॒॒शु लुं॑पन्ति लुंप न्त्यनꣳ॒॒शु कु॒र्वन्तः॑ । \newline
9. अ॒नꣳ॒॒शु कु॒र्वन्तः॑ कु॒र्वन्तो॑ ऽनꣳ॒॒ श्व॑नꣳ॒॒शु कु॒र्वन्त॑ उपाꣳ॒॒शु मु॑पाꣳ॒॒शुम् कु॒र्वन्तो॑ ऽनꣳ॒॒ श्व॑नꣳ॒॒शु कु॒र्वन्त॑ उपाꣳ॒॒शुम् । \newline
10. कु॒र्वन्त॑ उपाꣳ॒॒शु मु॑पाꣳ॒॒शुम् कु॒र्वन्तः॑ कु॒र्वन्त॑ उपाꣳ॒॒शुꣳ हु॒त्वा हु॒त्वोपाꣳ॒॒शुम् कु॒र्वन्तः॑ कु॒र्वन्त॑ उपाꣳ॒॒शुꣳ हु॒त्वा । \newline
11. उ॒पाꣳ॒॒शुꣳ हु॒त्वा हु॒त्वोपाꣳ॒॒शु मु॑पाꣳ॒॒शुꣳ हु॒त्वोपाꣳ॑शुपा॒त्र उ॑पाꣳशुपा॒त्रे 
हु॒त्वोपाꣳ॒॒शु मु॑पाꣳ॒॒शुꣳ हु॒त्वोपाꣳ॑शुपा॒त्रे । \newline
12. उ॒पाꣳ॒॒शुमित्यु॑प - अꣳ॒॒शुम् । \newline
13. हु॒त्वोपाꣳ॑शुपा॒त्र उ॑पाꣳशुपा॒त्रे हु॒त्वा हु॒त्वोपाꣳ॑शुपा॒त्रे ऽꣳ॑शु मꣳ॒॒शु मु॑पाꣳशुपा॒त्रे हु॒त्वा हु॒त्वोपाꣳ॑शुपा॒त्रे ऽꣳ॑शुम् । \newline
14. उ॒पाꣳ॒॒शु॒पा॒त्रे ऽꣳ॑??शु मꣳ॒॒शु मु॑पाꣳशुपा॒त्र उ॑पाꣳशुपा॒त्रे ऽꣳ॑शु म॒वास्या॒ 
वास्याꣳ॒॒शु मु॑पाꣳशुपा॒त्र उ॑पाꣳशुपा॒त्रे ऽꣳ॑शु म॒वास्य॑ । \newline
15. उ॒पाꣳ॒॒शु॒पा॒त्र इत्यु॑पाꣳशु - पा॒त्रे । \newline
16. अꣳ॒॒शु म॒वास्या॒ वास्याꣳ॒॒शु मꣳ॒॒शु म॒वास्य॒ तम् त म॒वास्याꣳ॒॒शु मꣳ॒॒शु म॒वास्य॒ तम् । \newline
17. अ॒वास्य॒ तम् त म॒वास्या॒ वास्य॒ तम् तृ॑तीयसव॒ने तृ॑तीयसव॒ने त म॒वास्या॒ वास्य॒ तम् तृ॑तीयसव॒ने । \newline
18. अ॒वास्येत्य॑व - अस्य॑ । \newline
19. तम् तृ॑तीयसव॒ने तृ॑तीयसव॒ने तम् तम् तृ॑तीयसव॒ने॑ ऽपि॒सृज्या॑ पि॒सृज्य॑ तृतीयसव॒ने तम् तम् तृ॑तीयसव॒ने॑ ऽपि॒सृज्य॑ । \newline
20. तृ॒ती॒य॒स॒व॒ने॑ ऽपि॒सृज्या॑ पि॒सृज्य॑ तृतीयसव॒ने तृ॑तीयसव॒ने॑ ऽपि॒सृज्या॒ भ्या᳚(1॒)भ्य॑ पि॒सृज्य॑ तृतीयसव॒ने तृ॑तीयसव॒ने॑ ऽपि॒सृज्या॒भि । \newline
21. तृ॒ती॒य॒स॒व॒न इति॑ तृतीय - स॒व॒ने । \newline
22. अ॒पि॒सृज्या॒ भ्या᳚(1॒)भ्य॑ पि॒सृज्या॑ पि॒सृज्या॒भि षु॑णुया थ्सुनुया द॒भ्य॑पि॒सृज्या॑ पि॒सृज्या॒भि षु॑णुयात् । \newline
23. अ॒पि॒सृज्येत्य॑पि - सृज्य॑ । \newline
24. अ॒भि षु॑णुयाथ् सुनुया द॒भ्य॑भि षु॑णुया॒द् यद् यथ् सु॑नुया द॒भ्य॑भि षु॑णुया॒द् यत् । \newline
25. सु॒नु॒या॒द् यद् यथ् सु॑नुयाथ् सुनुया॒द् यदा᳚ प्या॒यय॑ त्याप्या॒यय॑ति॒ यथ् सु॑नुयाथ् सुनुया॒द् यदा᳚ प्या॒यय॑ति । \newline
26. यदा᳚ प्या॒यय॑ त्याप्या॒यय॑ति॒ यद् यदा᳚ प्या॒यय॑ति॒ तेन॒ तेना᳚प्या॒यय॑ति॒ यद् यदा᳚ प्या॒यय॑ति॒ तेन॑ । \newline
27. आ॒प्या॒यय॑ति॒ तेन॒ तेना᳚ प्या॒यय॑ त्याप्या॒यय॑ति॒ तेना ꣳ॑शु॒म दꣳ॑शु॒मत् तेना᳚ प्या॒यय॑ त्याप्या॒यय॑ति॒ तेनाꣳ॑शु॒मत् । \newline
28. आ॒प्या॒यय॒तीत्या᳚ - प्या॒यय॑ति । \newline
29. तेना ꣳ॑शु॒म दꣳ॑शु॒मत् तेन॒ तेना ꣳ॑शु॒मद् यद् यदꣳ॑शु॒मत् तेन॒ तेनाꣳ॑शु॒मद् यत् । \newline
30. अꣳ॒॒शु॒मद् यद् यदꣳ॑शु॒म दꣳ॑शु॒मद् यद॑भिषु॒णो त्य॑भिषु॒णोति॒ यदꣳ॑शु॒म दꣳ॑शु॒मद् यद॑भिषु॒णोति॑ । \newline
31. अꣳ॒॒शु॒मदित्यꣳ॑शु - मत् । \newline
32. यद॑भिषु॒णो त्य॑भिषु॒णोति॒ यद् यद॑भिषु॒णोति॒ तेन॒ तेना॑भिषु॒णोति॒ यद् यद॑भिषु॒णोति॒ तेन॑ । \newline
33. अ॒भि॒षु॒णोति॒ तेन॒ तेना॑भिषु॒णो त्य॑भिषु॒णोति॒ तेन॑ र्जी॒ष्यृ॑जी॒षि तेना॑भिषु॒णो त्य॑भिषु॒णोति॒ तेन॑ र्जी॒षि । \newline
34. अ॒भि॒षु॒णोतीत्य॑भि - सु॒नोति॑ । \newline
35. तेन॑ र्जी॒ष्यृ॑जी॒षि तेन॒ तेन॑ र्जी॒षि सर्वा॑णि॒ सर्वा᳚ ण्यृजी॒षि तेन॒ तेन॑ र्जी॒षि सर्वा॑णि । \newline
36. ऋ॒जी॒षि सर्वा॑णि॒ सर्वा᳚ ण्यृजी॒ ष्यृ॑जी॒षि सर्वा᳚ ण्ये॒वैव सर्वा᳚ ण्यृजी॒ ष्यृ॑जी॒षि सर्वा᳚ण्ये॒व । \newline
37. सर्वा᳚ ण्ये॒वैव सर्वा॑णि॒ सर्वा᳚ ण्ये॒व तत् तदे॒व सर्वा॑णि॒ सर्वा᳚ ण्ये॒व तत् । \newline
38. ए॒व तत् तदे॒वैव तथ् सव॑नानि॒ सव॑नानि॒ तदे॒वैव तथ् सव॑नानि । \newline
39. तथ् सव॑नानि॒ सव॑नानि॒ तत् तथ् सव॑ना न्यꣳशु॒म न्त्यꣳ॑शु॒मन्ति॒ सव॑नानि॒ तत् तथ् सव॑ना 
न्यꣳशु॒मन्ति॑ । \newline
40. सव॑ना न्यꣳशु॒म न्त्यꣳ॑शु॒मन्ति॒ सव॑नानि॒ सव॑ नान्यꣳशु॒मन्ति॑ शु॒क्रव॑न्ति 
शु॒क्रव॑ न्त्यꣳशु॒मन्ति॒ सव॑नानि॒ सव॑ना न्यꣳशु॒मन्ति॑ शु॒क्रव॑न्ति । \newline
41. अꣳ॒॒शु॒मन्ति॑ शु॒क्रव॑न्ति शु॒क्रव॑ न्त्यꣳशु॒म न्त्यꣳ॑शु॒मन्ति॑ शु॒क्रव॑न्ति स॒माव॑द्वीर्याणि स॒माव॑द्वीर्याणि शु॒क्रव॑ न्त्यꣳशु॒म न्त्यꣳ॑शु॒मन्ति॑ शु॒क्रव॑न्ति स॒माव॑द्वीर्याणि । \newline
42. अꣳ॒॒शु॒मन्तीत्यꣳ॑शु - मन्ति॑ । \newline
43. शु॒क्रव॑न्ति स॒माव॑द्वीर्याणि स॒माव॑द्वीर्याणि शु॒क्रव॑न्ति शु॒क्रव॑न्ति स॒माव॑द्वीर्याणि करोति करोति स॒माव॑द्वीर्याणि शु॒क्रव॑न्ति शु॒क्रव॑न्ति स॒माव॑द्वीर्याणि करोति । \newline
44. शु॒क्रव॒न्तीति॑ शु॒क्र - व॒न्ति॒ । \newline
45. स॒माव॑द्वीर्याणि करोति करोति स॒माव॑द्वीर्याणि स॒माव॑द्वीर्याणि करोति॒ द्वौ द्वौ क॑रोति स॒माव॑द्वीर्याणि स॒माव॑द्वीर्याणि करोति॒ द्वौ । \newline
46. स॒माव॑द्वीर्या॒णीति॑ स॒माव॑त् - वी॒र्या॒णि॒ । \newline
47. क॒रो॒ति॒ द्वौ द्वौ क॑रोति करोति॒ द्वौ स॑मु॒द्रौ स॑मु॒द्रौ द्वौ क॑रोति करोति॒ द्वौ स॑मु॒द्रौ । \newline
48. द्वौ स॑मु॒द्रौ स॑मु॒द्रौ द्वौ द्वौ स॑मु॒द्रौ वित॑तौ॒ वित॑तौ समु॒द्रौ द्वौ द्वौ स॑मु॒द्रौ वित॑तौ । \newline
49. स॒मु॒द्रौ वित॑तौ॒ वित॑तौ समु॒द्रौ स॑मु॒द्रौ वित॑ता वजू॒र्या व॑जू॒र्यौ वित॑तौ समु॒द्रौ स॑मु॒द्रौ वित॑ता वजू॒र्यौ । \newline
50. वित॑ता वजू॒र्या व॑जू॒र्यौ वित॑तौ॒ वित॑ता वजू॒र्यौ प॒र्याव॑र्तेते प॒र्याव॑र्तेते अजू॒र्यौ वित॑तौ॒ वित॑ता वजू॒र्यौ प॒र्याव॑र्तेते । \newline
51. वित॑ता॒विति॒ वि - त॒तौ॒ । \newline
52. अ॒जू॒र्यौ प॒र्याव॑र्तेते प॒र्याव॑र्तेते अजू॒र्या व॑जू॒र्यौ प॒र्याव॑र्तेते ज॒ठरा॑ ज॒ठरा॑ प॒र्याव॑र्तेते अजू॒र्या व॑जू॒र्यौ प॒र्याव॑र्तेते ज॒ठरा᳚ । \newline
53. प॒र्याव॑र्तेते ज॒ठरा॑ ज॒ठरा॑ प॒र्याव॑र्तेते प॒र्याव॑र्तेते ज॒ठरे॑वे व ज॒ठरा॑ प॒र्याव॑र्तेते प॒र्याव॑र्तेते ज॒ठरे॑व । \newline
54. प॒र्याव॑र्तेते॒ इति॑ परि - आव॑र्तेते । \newline
55. ज॒ठरे॑वे व ज॒ठरा॑ ज॒ठरे॑व॒ पादाः॒ पादा॑ इव ज॒ठरा॑ ज॒ठरे॑व॒ पादाः᳚ । \newline
56. इ॒व॒ पादाः॒ पादा॑ इवे व॒ पादाः᳚ । \newline
57. पादा॒ इति॒ पादाः᳚ । \newline
58. तयोः॒ पश्य॑न्तः॒ पश्य॑न्त॒ स्तयो॒ स्तयोः॒ पश्य॑न्तो॒ अत्यति॒ पश्य॑न्त॒ स्तयो॒ स्तयोः॒ पश्य॑न्तो॒ अति॑ । \newline
59. पश्य॑न्तो॒ अत्यति॒ पश्य॑न्तः॒ पश्य॑न्तो॒ अति॑ यन्ति य॒न्त्यति॒ पश्य॑न्तः॒ पश्य॑न्तो॒ अति॑ यन्ति । \newline
60. अति॑ यन्ति य॒न्त्य त्यति॑ यन्त्य॒ न्यम॒न्यं ॅय॒न्त्य त्यति॑ यन्त्य॒ न्यम् । \newline
61. य॒न्त्य॒ न्य म॒न्यं ॅय॑न्ति यन्त्य॒ न्य मप॑श्य॒न्तो ऽप॑श्यन्तो॒ ऽन्यं ॅय॑न्ति यन्त्य॒न्य मप॑श्यन्तः । \newline
62. अ॒न्य मप॑श्य॒न्तो ऽप॑श्यन्तो॒ ऽन्य म॒न्य मप॑श्यन्तः॒ सेतु॑ना॒ सेतु॒ना ऽप॑श्यन्तो॒ ऽन्य म॒न्य मप॑श्यन्तः॒ सेतु॑ना । \newline
63. अप॑श्यन्तः॒ सेतु॑ना॒ सेतु॒ना ऽप॑श्य॒न्तो ऽप॑श्यन्तः॒ सेतु॒ना ऽत्यति॒ सेतु॒ना ऽप॑श्य॒न्तो ऽप॑श्यन्तः॒ सेतु॒ना ऽति॑ । \newline
\pagebreak
\markright{ TS 3.2.2.2  \hfill https://www.vedavms.in \hfill}

\section{ TS 3.2.2.2 }

\textbf{TS 3.2.2.2 } \newline
\textbf{Samhita Paata} \newline

सेतु॒नाऽति॑ यन्त्य॒न्यं ॥ द्वे द्रध॑सी स॒तती॑ वस्त॒ एकः॑ के॒शी विश्वा॒ भुव॑नानि वि॒द्वान् । ति॒रो॒धायै॒त्यसि॑तं॒ ॅवसा॑नः शु॒क्रमा द॑त्ते अनु॒हाय॑ जा॒र्यै ॥ दे॒वा वै यद्य॒ज्ञेऽकु॑र्वत॒ तदसु॑रा अकुर्वत॒ ते दे॒वा ए॒तं म॑हाय॒ज्ञ्म॑पश्य॒न् तम॑तन्वताग्निहो॒त्रं ॅव्र॒तम॑कुर्वत॒ तस्मा॒द् द्विव्र॑तः स्या॒द् द्विर्ह्य॑ग्निहो॒त्रं जुह्व॑ति पौर्णमा॒सं ॅय॒ज्ञ्म॑ग्नीषो॒मीयं॑ - [  ] \newline

\textbf{Pada Paata} \newline

सेतु॑ना । अतीति॑ । य॒न्ति॒ । अ॒न्यम् ॥ द्वे इति॑ । द्रध॑सी॒ इति॑ । स॒तती॒ इति॑ स - तती᳚ । व॒स्ते॒ । एकः॑ । के॒शी । विश्वा᳚ । भुव॑नानि । वि॒द्वान् ॥ ति॒रो॒धायेति॑ तिरः - धाय॑ । ए॒ति॒ । असि॑तम् । वसा॑नः । शु॒क्रम् । एति॑ । द॒त्ते॒ । अ॒नु॒हायेत्य॑नु - हाय॑ । जा॒र्यै ॥ दे॒वाः । वै । यत् । य॒ज्ञे । अकु॑र्वत । तत् । असु॑राः । अ॒कु॒र्व॒त॒ । ते । दे॒वाः । ए॒तम् । म॒हा॒य॒ज्ञ्मिति॑ महा - य॒ज्ञ्म् । अ॒प॒श्य॒न्न् । तम् । अ॒त॒न्व॒त॒ । अ॒ग्नि॒हो॒त्रमित्य॑ग्नि - हो॒त्रम् । व्र॒तम् । अ॒कु॒र्व॒त॒ । तस्मा᳚त् । द्विव्र॑त॒ इति॒ द्वि - व्र॒तः॒ । स्या॒त् । द्विः । हि । अ॒ग्नि॒हो॒त्रमित्य॑ग्नि - हो॒त्रम् । जुह्व॑ति । पौ॒र्ण॒मा॒समिति॑ पौर्ण - मा॒सम् । य॒ज्ञ्म् । अ॒ग्नी॒षो॒मीय॒मित्य॑ग्नी - सो॒मीय᳚म् ।  \newline


\textbf{Krama Paata} \newline

सेतु॒नाऽति॑ । अति॑ यन्ति । य॒न्त्य॒न्यम् । अ॒न्यमित्य॒न्यम् ॥ द्वे द्रध॑सी । द्वे इति॒ द्वे । द्रध॑सी स॒तती᳚ । द्रध॑सी॒ इति॒ द्रध॑सी । स॒तती॑ वस्ते । स॒तती॒ इति॑ स - तती᳚ । व॒स्त॒ एकः॑ । एकः॑ के॒शी । के॒शी विश्वा᳚ । विश्वा॒ भुव॑नानि । भुव॑नानि वि॒द्वान् । वि॒द्वानिति॑ वि॒द्वान् ॥ ति॒रो॒धायै॑ति । ति॒रो॒धायेति॑ तिरः - धाय॑ । ए॒त्यसि॑तम् । असि॑तं॒ ॅवसा॑नः । वसा॑नः शु॒क्रम् । शु॒क्रमा । आ द॑त्ते । द॒त्ते॒ अ॒नु॒हाय॑ । अ॒नु॒हाय॑ जा॒र्यै । अ॒नु॒हायेत्य॑नु - हाय॑ । जा॒र्या इति॑ जा॒र्यै ॥ दे॒वा वै । वै यत् । यद् य॒ज्ञे । य॒ज्ञे ऽकु॑र्वत । अकु॑र्वत॒ तत् । तदसु॑राः । असु॑रा अकुर्वत । अ॒कु॒र्व॒त॒ ते । ते दे॒वाः । दे॒वा ए॒तम् । ए॒तम् म॑हाय॒ज्ञ्म् । म॒हा॒य॒ज्ञ्म॑पश्यन्न् । म॒हा॒य॒ज्ञ्मिति॑ महा - य॒ज्ञ्म् । अ॒प॒श्य॒न् तम् । तम॑तन्वत । अ॒त॒न्व॒ता॒ग्नि॒हो॒त्रम् । अ॒ग्नि॒हो॒त्रं ॅव्र॒तम् । अ॒ग्नि॒हो॒त्रमित्य॑ग्नि - हो॒त्रम् । व्र॒तम॑कुर्वत । अ॒कु॒र्व॒त॒ तस्मा᳚त् । तस्मा॒द् द्विव्र॑तः । द्विव्र॑तः स्यात् । द्विव्र॑त॒ इति॒ द्वि - व्र॒तः॒ । स्या॒द् द्विः । द्विर्. हि । ह्य॑ग्निहो॒त्रम् । अ॒ग्नि॒हो॒त्रम् जुह्व॑ति । अ॒ग्नि॒हो॒त्रमित्य॑ग्नि - हो॒त्रम् । जुह्व॑ति पौर्णमा॒सम् । पौ॒र्ण॒मा॒सं ॅय॒ज्ञ्म् । पौ॒र्ण॒मा॒समिति॑ पौर्ण - मा॒सम् । य॒ज्ञ्म॑ग्नीषो॒मीय᳚म् । अ॒ग्नी॒षो॒मीय॑म् प॒शुम् । अ॒ग्नी॒षो॒मीय॒मित्य॑ग्नी - सो॒मीयम्᳚ \newline

\textbf{Jatai Paata} \newline

1. सेतु॒ना ऽत्यति॒ सेतु॑ना॒ सेतु॒ना ऽति॑ । \newline
2. अति॑ यन्ति य॒न्त्य त्यति॑ यन्ति । \newline
3. य॒ न्त्य॒न्य म॒न्यं ॅय॑न्ति य न्त्य॒न्यम् । \newline
4. अ॒न्यमित्य॒न्यम् । \newline
5. द्वे द्रध॑सी॒ द्रध॑सी॒ द्वे द्वे द्रध॑सी । \newline
6. द्वे इति॒ द्वे । \newline
7. द्रध॑सी स॒तती॑ स॒तती॒ द्रध॑सी॒ द्रध॑सी स॒तती᳚ । \newline
8. द्रध॑सी॒ इति॒ द्रध॑सी । \newline
9. स॒तती॑ वस्ते वस्ते स॒तती॑ स॒तती॑ वस्ते । \newline
10. स॒तती॒ इति॑ स - तती᳚ । \newline
11. व॒स्त॒ एक॒ एको॑ वस्ते वस्त॒ एकः॑ । \newline
12. एकः॑ के॒शी के॒श्येक॒ एकः॑ के॒शी । \newline
13. के॒शी विश्वा॒ विश्वा॑ के॒शी के॒शी विश्वा᳚ । \newline
14. विश्वा॒ भुव॑नानि॒ भुव॑नानि॒ विश्वा॒ विश्वा॒ भुव॑नानि । \newline
15. भुव॑नानि वि॒द्वान्. वि॒द्वान् भुव॑नानि॒ भुव॑नानि वि॒द्वान् । \newline
16. वि॒द्वानिति॑ वि॒द्वान् । \newline
17. ति॒रो॒धा यै᳚त्येति तिरो॒धाय॑ तिरो॒धा यै॑ति । \newline
18. ति॒रो॒धायेति॑ तिरः - धाय॑ । \newline
19. ए॒त्यसि॑त॒ मसि॑त मेत्ये॒ त्यसि॑तम् । \newline
20. असि॑तं॒ ॅवसा॑नो॒ वसा॒नो ऽसि॑त॒ मसि॑तं॒ ॅवसा॑नः । \newline
21. वसा॑नः शु॒क्रꣳ शु॒क्रं ॅवसा॑नो॒ वसा॑नः शु॒क्रम् । \newline
22. शु॒क्र मा शु॒क्रꣳ शु॒क्र मा । \newline
23. आ द॑त्ते दत्त॒ आ द॑त्ते । \newline
24. द॒त्ते॒ अ॒नु॒हाया॑ नु॒हाय॑ दत्ते दत्ते अनु॒हाय॑ । \newline
25. अ॒नु॒हाय॑ जा॒र्यै जा॒र्या अ॑नु॒हाया॑ नु॒हाय॑ जा॒र्यै । \newline
26. अ॒नु॒हायेत्य॑नु - हाय॑ । \newline
27. जा॒र्या इति॑ जा॒र्यै । \newline
28. दे॒वा वै वै दे॒वा दे॒वा वै । \newline
29. वै यद् यद् वै वै यत् । \newline
30. यद् य॒ज्ञे य॒ज्ञे यद् यद् य॒ज्ञे । \newline
31. य॒ज्ञे ऽकु॑र्व॒ता कु॑र्वत य॒ज्ञे य॒ज्ञे ऽकु॑र्वत । \newline
32. अकु॑र्वत॒ तत् तदकु॑र्व॒ता कु॑र्वत॒ तत् । \newline
33. तदसु॑रा॒ असु॑रा॒ स्तत् तदसु॑राः । \newline
34. असु॑रा अकुर्वता कुर्व॒ता सु॑रा॒ असु॑रा अकुर्वत । \newline
35. अ॒कु॒र्व॒त॒ ते ते॑ ऽकुर्वता कुर्वत॒ ते । \newline
36. ते दे॒वा दे॒वा स्ते ते दे॒वाः । \newline
37. दे॒वा ए॒त मे॒तम् दे॒वा दे॒वा ए॒तम् । \newline
38. ए॒तम् म॑हाय॒ज्ञ्म् म॑हाय॒ज्ञ् मे॒त मे॒तम् म॑हाय॒ज्ञ्म् । \newline
39. म॒हा॒य॒ज्ञ् म॑पश्यन् नपश्यन् महाय॒ज्ञ्म् म॑हाय॒ज्ञ् म॑पश्यन्न् । \newline
40. म॒हा॒य॒ज्ञ्मिति॑ महा - य॒ज्ञ्म् । \newline
41. अ॒प॒श्य॒न् तम् त म॑पश्यन् नपश्य॒न् तम् । \newline
42. त म॑तन्वता तन्वत॒ तम् त म॑तन्वत । \newline
43. अ॒त॒न्व॒ता॒ ग्नि॒हो॒त्र म॑ग्निहो॒त्र म॑तन्वता तन्वता ग्निहो॒त्रम् । \newline
44. अ॒ग्नि॒हो॒त्रं ॅव्र॒तं ॅव्र॒त म॑ग्निहो॒त्र म॑ग्निहो॒त्रं ॅव्र॒तम् । \newline
45. अ॒ग्नि॒हो॒त्रमित्य॑ग्नि - हो॒त्रम् । \newline
46. व्र॒त म॑कुर्वता कुर्वत व्र॒तं ॅव्र॒त म॑कुर्वत । \newline
47. अ॒कु॒र्व॒त॒ तस्मा॒त् तस्मा॑ दकुर्वता कुर्वत॒ तस्मा᳚त् । \newline
48. तस्मा॒द् द्विव्र॑तो॒ द्विव्र॑त॒ स्तस्मा॒त् तस्मा॒द् द्विव्र॑तः । \newline
49. द्विव्र॑तः स्याथ् स्या॒द् द्विव्र॑तो॒ द्विव्र॑तः स्यात् । \newline
50. द्विव्र॑त॒ इति॒ द्वि - व्र॒तः॒ । \newline
51. स्या॒द् द्विर् द्विः स्या᳚थ् स्या॒द् द्विः । \newline
52. द्विर्. हि हि द्विर् द्विर्. हि । \newline
53. ह्य॑ग्निहो॒त्र म॑ग्निहो॒त्रꣳ हि ह्य॑ग्निहो॒त्रम् । \newline
54. अ॒ग्नि॒हो॒त्रम् जुह्व॑ति॒ जुह्व॑ त्यग्निहो॒त्र म॑ग्निहो॒त्रम् जुह्व॑ति । \newline
55. अ॒ग्नि॒हो॒त्रमित्य॑ग्नि - हो॒त्रम् । \newline
56. जुह्व॑ति पौर्णमा॒सम् पौ᳚र्णमा॒सम् जुह्व॑ति॒ जुह्व॑ति पौर्णमा॒सम् । \newline
57. पौ॒र्ण॒मा॒सं ॅय॒ज्ञ्ं ॅय॒ज्ञ्म् पौ᳚र्णमा॒सम् पौ᳚र्णमा॒सं ॅय॒ज्ञ्म् । \newline
58. पौ॒र्ण॒मा॒समिति॑ पौर्ण - मा॒सम् । \newline
59. य॒ज्ञ् म॑ग्नीषो॒मीय॑ मग्नीषो॒मीयं॑ ॅय॒ज्ञ्ं ॅय॒ज्ञ् म॑ग्नीषो॒मीय᳚म् । \newline
60. अ॒ग्नी॒षो॒मीय॑म् प॒शुम् प॒शु म॑ग्नीषो॒मीय॑ मग्नीषो॒मीय॑म् प॒शुम् । \newline
61. अ॒ग्नी॒षो॒मीय॒मित्य॑ग्नी - सो॒मीय᳚म् । \newline

\textbf{Ghana Paata } \newline

1. सेतु॒ना ऽत्यति॒ सेतु॑ना॒ सेतु॒ना ऽति॑ यन्ति य॒न्त्यति॒ सेतु॑ना॒ सेतु॒ना ऽति॑ यन्ति । \newline
2. अति॑ यन्ति य॒ न्त्यत्यति॑ यन्त्य॒न्य म॒न्यं ॅय॒ न्त्यत्यति॑ यन्त्य॒न्यम् । \newline
3. य॒न्त्य॒न्य म॒न्यं ॅय॑न्ति यन्त्य॒न्यम् । \newline
4. अ॒न्यमित्य॒न्यम् । \newline
5. द्वे द्रध॑सी॒ द्रध॑सी॒ द्वे द्वे द्रध॑सी स॒तती॑ स॒तती॒ द्रध॑सी॒ द्वे द्वे द्रध॑सी स॒तती᳚ । \newline
6. द्वे इति॒ द्वे । \newline
7. द्रध॑सी स॒तती॑ स॒तती॒ द्रध॑सी॒ द्रध॑सी स॒तती॑ वस्ते वस्ते स॒तती॒ द्रध॑सी॒ द्रध॑सी स॒तती॑ वस्ते । \newline
8. द्रध॑सी॒ इति॒ द्रध॑सी । \newline
9. स॒तती॑ वस्ते वस्ते स॒तती॑ स॒तती॑ वस्त॒ एक॒ एको॑ वस्ते स॒तती॑ स॒तती॑ वस्त॒ एकः॑ । \newline
10. स॒तती॒ इति॑ स - तती᳚ । \newline
11. व॒स्त॒ एक॒ एको॑ वस्ते वस्त॒ एकः॑ के॒शी के॒श्येको॑ वस्ते वस्त॒ एकः॑ के॒शी । \newline
12. एकः॑ के॒शी के॒श्येक॒ एकः॑ के॒शी विश्वा॒ विश्वा॑ के॒श्येक॒ एकः॑ के॒शी विश्वा᳚ । \newline
13. के॒शी विश्वा॒ विश्वा॑ के॒शी के॒शी विश्वा॒ भुव॑नानि॒ भुव॑नानि॒ विश्वा॑ के॒शी के॒शी विश्वा॒ भुव॑नानि । \newline
14. विश्वा॒ भुव॑नानि॒ भुव॑नानि॒ विश्वा॒ विश्वा॒ भुव॑नानि वि॒द्वान्. वि॒द्वान् भुव॑नानि॒ विश्वा॒ विश्वा॒ भुव॑नानि वि॒द्वान् । \newline
15. भुव॑नानि वि॒द्वान्. वि॒द्वान् भुव॑नानि॒ भुव॑नानि वि॒द्वान् । \newline
16. वि॒द्वानिति॑ वि॒द्वान् । \newline
17. ति॒रो॒धा यै᳚त्येति तिरो॒धाय॑ तिरो॒धा यै॒त्यसि॑त॒ मसि॑त मेति तिरो॒धाय॑ तिरो॒धा यै॒त्यसि॑तम् । \newline
18. ति॒रो॒धायेति॑ तिरः - धाय॑ । \newline
19. ए॒त्यसि॑त॒ मसि॑त मेत्ये॒ त्यसि॑तं॒ ॅवसा॑नो॒ वसा॒नो ऽसि॑त मेत्ये॒ त्यसि॑तं॒ ॅवसा॑नः । \newline
20. असि॑तं॒ ॅवसा॑नो॒ वसा॒नो ऽसि॑त॒ मसि॑तं॒ ॅवसा॑नः शु॒क्रꣳ शु॒क्रं ॅवसा॒नो ऽसि॑त॒ मसि॑तं॒ ॅवसा॑नः शु॒क्रम् । \newline
21. वसा॑नः शु॒क्रꣳ शु॒क्रं ॅवसा॑नो॒ वसा॑नः शु॒क्र मा शु॒क्रं ॅवसा॑नो॒ वसा॑नः शु॒क्र मा । \newline
22. शु॒क्र मा शु॒क्रꣳ शु॒क्र मा द॑त्ते दत्त॒ आ शु॒क्रꣳ शु॒क्र मा द॑त्ते । \newline
23. आ द॑त्ते दत्त॒ आ द॑त्ते अनु॒हाया॑ नु॒हाय॑ दत्त॒ आ द॑त्ते अनु॒हाय॑ । \newline
24. द॒त्ते॒ अ॒नु॒हाया॑ नु॒हाय॑ दत्ते दत्ते अनु॒हाय॑ जा॒र्यै जा॒र्या अ॑नु॒हाय॑ दत्ते दत्ते अनु॒हाय॑ जा॒र्यै । \newline
25. अ॒नु॒हाय॑ जा॒र्यै जा॒र्या अ॑नु॒हाया॑ नु॒हाय॑ जा॒र्यै । \newline
26. अ॒नु॒हायेत्य॑नु - हाय॑ । \newline
27. जा॒र्या इति॑ जा॒र्यै । \newline
28. दे॒वा वै वै दे॒वा दे॒वा वै यद् यद् वै दे॒वा दे॒वा वै यत् । \newline
29. वै यद् यद् वै वै यद् य॒ज्ञे य॒ज्ञे यद् वै वै यद् य॒ज्ञे । \newline
30. यद् य॒ज्ञे य॒ज्ञे यद् यद् य॒ज्ञे ऽकु॑र्व॒ता कु॑र्वत य॒ज्ञे यद् यद् य॒ज्ञे ऽकु॑र्वत । \newline
31. य॒ज्ञे ऽकु॑र्व॒ता कु॑र्वत य॒ज्ञे य॒ज्ञे ऽकु॑र्वत॒ तत् तदकु॑र्वत य॒ज्ञे य॒ज्ञे ऽकु॑र्वत॒ तत् । \newline
32. अकु॑र्वत॒ तत् तदकु॑र्व॒ता कु॑र्वत॒ तदसु॑रा॒ असु॑रा॒ स्तदकु॑र्व॒ता कु॑र्वत॒ तदसु॑राः । \newline
33. तदसु॑रा॒ असु॑रा॒ स्तत् तदसु॑रा अकुर्वता कुर्व॒ता सु॑रा॒ स्तत् तदसु॑रा अकुर्वत । \newline
34. असु॑रा अकुर्वता कुर्व॒ता सु॑रा॒ असु॑रा अकुर्वत॒ ते ते॑ ऽकुर्व॒ता सु॑रा॒ असु॑रा अकुर्वत॒ ते । \newline
35. अ॒कु॒र्व॒त॒ ते ते॑ ऽकुर्वता कुर्वत॒ ते दे॒वा दे॒वा स्ते॑ ऽकुर्वता कुर्वत॒ ते दे॒वाः । \newline
36. ते दे॒वा दे॒वा स्ते ते दे॒वा ए॒त मे॒तम् दे॒वा स्ते ते दे॒वा ए॒तम् । \newline
37. दे॒वा ए॒त मे॒तम् दे॒वा दे॒वा ए॒तम् म॑हाय॒ज्ञ्म् म॑हाय॒ज्ञ् मे॒तम् दे॒वा दे॒वा ए॒तम् म॑हाय॒ज्ञ्म् । \newline
38. ए॒तम् म॑हाय॒ज्ञ्म् म॑हाय॒ज्ञ् मे॒त मे॒तम् म॑हाय॒ज्ञ् म॑पश्यन् नपश्यन् महाय॒ज्ञ् मे॒त मे॒तम् म॑हाय॒ज्ञ् म॑पश्यन्न् । \newline
39. म॒हा॒य॒ज्ञ् म॑पश्यन् नपश्यन् महाय॒ज्ञ्म् म॑हाय॒ज्ञ् म॑पश्य॒न् तम् त म॑पश्यन् महाय॒ज्ञ्म् म॑हाय॒ज्ञ् म॑पश्य॒न् तम् । \newline
40. म॒हा॒य॒ज्ञ्मिति॑ महा - य॒ज्ञ्म् । \newline
41. अ॒प॒श्य॒न् तम् त म॑पश्यन् नपश्य॒न् त म॑तन्वता तन्वत॒ त म॑पश्यन् नपश्य॒न् त म॑तन्वत । \newline
42. त म॑तन्वता तन्वत॒ तम् त म॑तन्वता ग्निहो॒त्र म॑ग्निहो॒त्र म॑तन्वत॒ तम् त म॑तन्वता ग्निहो॒त्रम् । \newline
43. अ॒त॒न्व॒ता॒ ग्नि॒हो॒त्र म॑ग्निहो॒त्र म॑तन्वता तन्वता ग्निहो॒त्रं ॅव्र॒तं ॅव्र॒त म॑ग्निहो॒त्र म॑तन्वता तन्वता ग्निहो॒त्रं ॅव्र॒तम् । \newline
44. अ॒ग्नि॒हो॒त्रं ॅव्र॒तं ॅव्र॒त म॑ग्निहो॒त्र म॑ग्निहो॒त्रं ॅव्र॒त म॑कुर्वता कुर्वत व्र॒त म॑ग्निहो॒त्र म॑ग्निहो॒त्रं ॅव्र॒त म॑कुर्वत । \newline
45. अ॒ग्नि॒हो॒त्रमित्य॑ग्नि - हो॒त्रम् । \newline
46. व्र॒त म॑कुर्वता कुर्वत व्र॒तं ॅव्र॒त म॑कुर्वत॒ तस्मा॒त् तस्मा॑ दकुर्वत व्र॒तं ॅव्र॒त म॑कुर्वत॒ तस्मा᳚त् । \newline
47. अ॒कु॒र्व॒त॒ तस्मा॒त् तस्मा॑ दकुर्वता कुर्वत॒ तस्मा॒द् द्विव्र॑तो॒ द्विव्र॑त॒ स्तस्मा॑ दकुर्वता कुर्वत॒ तस्मा॒द् द्विव्र॑तः । \newline
48. तस्मा॒द् द्विव्र॑तो॒ द्विव्र॑त॒ स्तस्मा॒त् तस्मा॒द् द्विव्र॑तः स्याथ् स्या॒द् द्विव्र॑त॒ स्तस्मा॒त् तस्मा॒द् द्विव्र॑तः स्यात् । \newline
49. द्विव्र॑तः स्याथ् स्या॒द् द्विव्र॑तो॒ द्विव्र॑तः स्या॒द् द्विर् द्विः स्या॒द् द्विव्र॑तो॒ द्विव्र॑तः स्या॒द् द्विः । \newline
50. द्विव्र॑त॒ इति॒ द्वि - व्र॒तः॒ । \newline
51. स्या॒द् द्विर् द्विः स्या᳚थ् स्या॒द् द्विर्. हि हि द्विः स्या᳚थ् स्या॒द् द्विर्. हि । \newline
52. द्विर्. हि हि द्विर् द्विर् ह्य॑ग्निहो॒त्र म॑ग्निहो॒त्रꣳ हि द्विर् द्विर् ह्य॑ग्निहो॒त्रम् । \newline
53. ह्य॑ग्निहो॒त्र म॑ग्निहो॒त्रꣳ हि ह्य॑ग्निहो॒त्रम् जुह्व॑ति॒ जुह्व॑ त्यग्निहो॒त्रꣳ हि ह्य॑ग्निहो॒त्रम् जुह्व॑ति । \newline
54. अ॒ग्नि॒हो॒त्रम् जुह्व॑ति॒ जुह्व॑ त्यग्निहो॒त्र म॑ग्निहो॒त्रम् जुह्व॑ति पौर्णमा॒सम् पौ᳚र्णमा॒सम् जुह्व॑ त्यग्निहो॒त्र म॑ग्निहो॒त्रम् जुह्व॑ति पौर्णमा॒सम् । \newline
55. अ॒ग्नि॒हो॒त्रमित्य॑ग्नि - हो॒त्रम् । \newline
56. जुह्व॑ति पौर्णमा॒सम् पौ᳚र्णमा॒सम् जुह्व॑ति॒ जुह्व॑ति पौर्णमा॒सं ॅय॒ज्ञ्ं ॅय॒ज्ञ्म् पौ᳚र्णमा॒सम् जुह्व॑ति॒ जुह्व॑ति पौर्णमा॒सं ॅय॒ज्ञ्म् । \newline
57. पौ॒र्ण॒मा॒सं ॅय॒ज्ञ्ं ॅय॒ज्ञ्म् पौ᳚र्णमा॒सम् पौ᳚र्णमा॒सं ॅय॒ज्ञ् म॑ग्नीषो॒मीय॑ मग्नीषो॒मीयं॑ ॅय॒ज्ञ्म् पौ᳚र्णमा॒सम् पौ᳚र्णमा॒सं ॅय॒ज्ञ् म॑ग्नीषो॒मीय᳚म् । \newline
58. पौ॒र्ण॒मा॒समिति॑ पौर्ण - मा॒सम् । \newline
59. य॒ज्ञ् म॑ग्नीषो॒मीय॑ मग्नीषो॒मीयं॑ ॅय॒ज्ञ्ं ॅय॒ज्ञ् म॑ग्नीषो॒मीय॑म् प॒शुम् प॒शु म॑ग्नीषो॒मीयं॑ ॅय॒ज्ञ्ं ॅय॒ज्ञ् म॑ग्नीषो॒मीय॑म् प॒शुम् । \newline
60. अ॒ग्नी॒षो॒मीय॑म् प॒शुम् प॒शु म॑ग्नीषो॒मीय॑ मग्नीषो॒मीय॑म् प॒शु म॑कुर्वता कुर्वत प॒शु म॑ग्नीषो॒मीय॑ मग्नीषो॒मीय॑म् प॒शु म॑कुर्वत । \newline
61. अ॒ग्नी॒षो॒मीय॒मित्य॑ग्नी - सो॒मीय᳚म् । \newline
\pagebreak
\markright{ TS 3.2.2.3  \hfill https://www.vedavms.in \hfill}

\section{ TS 3.2.2.3 }

\textbf{TS 3.2.2.3 } \newline
\textbf{Samhita Paata} \newline

प॒शुम॑कुर्वत दा॒र्श्यं ॅय॒ज्ञ्मा᳚ग्ने॒यं प॒शुम॑कुर्वत वैश्वदे॒वं प्रा॑तस्सव॒न -म॑कुर्वत वरुणप्रघा॒सान् माद्ध्य॑दिंनꣳ॒॒ सव॑नꣳ साकमे॒धान् पि॑तृय॒ज्ञ्ं त्र्य॑म्बकाꣳ-स्तृतीयसव॒नम॑कुर्वत॒ तमे॑षा॒मसु॑रा य॒ज्ञ् -म॒न्ववा॑जिगाꣳस॒न् तं नाऽन्ववा॑य॒न् ते᳚ऽब्रुवन्नद्ध्वर्त॒व्या वा इ॒मे दे॒वा अ॑भूव॒न्निति॒ तद॑द्ध्व॒रस्या᳚ ऽद्ध्वर॒त्वं ततो॑ दे॒वा अभ॑व॒न् पराऽसु॑रा॒ य ए॒वं ॅवि॒द्वान्थ् सोमे॑न॒ यज॑ते॒ भव॑त्या॒त्मना॒ परा᳚ ( ) ऽस्य॒ भ्रातृ॑व्यो भवति ॥ \newline

\textbf{Pada Paata} \newline

प॒शुम् । अ॒कु॒र्व॒त॒ । दा॒र्श्यम् । य॒ज्ञ्म् । आ॒ग्ने॒यम् । प॒शुम् । अ॒कु॒र्व॒त॒ । वै॒श्व॒दे॒वमिति॑ वैश्व - दे॒वम् । प्रा॒त॒स्स॒व॒नमिति॑ प्रातः - स॒व॒नम् । अ॒कु॒र्व॒त॒ । व॒रु॒ण॒प्र॒घा॒सानिति॑ वरुण - प्र॒घा॒सान् । माद्ध्य॑न्दिनम् । सव॑नम् । सा॒क॒मे॒धानिति॑ साक - मे॒धान् । पि॒तृ॒य॒ज्ञ्मिति॑ पितृ - य॒ज्ञ्ं । त्र्य॑बंका॒निति॒ त्रि - अ॒बं॒का॒न् । तृ॒ती॒य॒स॒व॒नमिति॑ तृतीय - स॒व॒नम् । अ॒कु॒र्व॒त॒ । तम् । ए॒षा॒म् । असु॑राः । य॒ज्ञ्म् । अ॒न्ववा॑जिगाꣳस॒न्नित्य॑नु-अवा॑जिगाꣳसन्न् । तम् । न । अ॒न्ववा॑य॒न्नित्य॑नु-अवा॑यन्न् । ते । अ॒ब्रु॒व॒न्न् । अ॒द्ध्व॒र्त॒व्याः । वै । इ॒मे । दे॒वाः । अ॒भू॒व॒न्न् । इति॑ । तत् । अ॒द्ध्व॒रस्य॑ । अ॒द्ध्व॒र॒त्वमित्य॑द्ध्वर - त्वम् । ततः॑ । दे॒वाः । अभ॑वन्न् । परेति॑ । असु॑राः । यः । ए॒वम् । वि॒द्वान् । सोमे॑न । यज॑ते । भव॑ति । आ॒त्मना᳚ । परेति॑ ( ) । अ॒स्य॒ । भ्रातृ॑व्यः । भ॒व॒ति॒ ॥  \newline


\textbf{Krama Paata} \newline

प॒शुम॑कुर्वत । अ॒कु॒र्व॒त॒ दा॒र्श्यम् । दा॒र्श्यं ॅय॒ज्ञ्म् । य॒ज्ञ्मा᳚ग्ने॒यम् । आ॒ग्ने॒यम् प॒शुम् । प॒शुम॑कुर्वत । अ॒कु॒र्व॒त॒ वै॒श्व॒दे॒वम् । वै॒श्व॒दे॒वम् प्रा॑तस्सव॒नम् । वै॒श्व॒दे॒वमिति॑ वैश्व - दे॒वम् । प्रा॒त॒स्स॒व॒नम॑कुर्वत । प्रा॒त॒स्स॒व॒नमिति॑ प्रातः - स॒व॒नम् । अ॒कु॒र्व॒त॒ व॒रु॒ण॒प्र॒घा॒सान् । व॒रु॒ण॒प्र॒घा॒सान्,माद्ध्य॑न्दिनम् । व॒रु॒ण॒प्र॒घा॒सानिति॑ वरुण - प्र॒घा॒सान् । माद्ध्य॑न्दिनꣳ॒॒ सव॑नम् । सव॑नꣳ साकमे॒धान् । सा॒क॒मे॒धान् पि॑तृय॒ज्ञ्म् । सा॒क॒मे॒धानिति॑ साक - मे॒धान् । पि॒तृ॒य॒ज्ञ्म् त्र्य॑म्बकान् । पि॒तृ॒य॒ज्ञ्मिति॑ पितृ - य॒ज्ञ्म् । त्र्य॑म्बकाꣳ तृतीयसव॒नम् । त्र्य॑म्बका॒निति॒ त्रि - अ॒म्ब॒का॒न्॒ । तृ॒ती॒य॒स॒व॒नम॑कुर्वत । तृ॒ती॒य॒स॒व॒नमिति॑ तृतीय - स॒व॒नम् । अ॒कु॒र्व॒त॒ तम् । तमे॑षाम् । ए॒षा॒मसु॑राः । असु॑रा य॒ज्ञ्म् । य॒ज्ञ्म॒न्ववा॑जिगाꣳसन्न् । अ॒न्ववा॑जिगाꣳस॒न् तम् । अ॒न्ववा॑जिगाꣳस॒न्नित्य॑नु - अवा॑जिगाꣳसन्न् । तम् न । नान्ववा॑यन्न् । अ॒न्ववा॑य॒न् ते । अ॒न्ववा॑य॒न्नित्य॑नु - अवा॑यन्न् । ते᳚ ऽब्रुवन्न् । अ॒ब्रु॒व॒न्न॒द्ध्व॒र्त॒व्याः । अ॒द्ध्व॒र्त॒व्या वै । वा इ॒मे । इ॒मे दे॒वाः । दे॒वा अ॑भूवन्न् । अ॒भू॒व॒न्निति॑ । इति॒ तत् । तद॑द्ध्व॒रस्य॑ । अ॒द्ध्व॒रस्या᳚द्ध्वर॒त्वम् । अ॒द्ध्व॒र॒त्वम् ततः॑ । अ॒द्ध्व॒र॒त्वमित्य॑द्ध्वर - त्वम् । ततो॑ दे॒वाः । दे॒वा अभ॑वन्न् । अभ॑व॒न् परा᳚ । परा ऽसु॑राः । असु॑रा॒ यः । य ए॒वम् । ए॒वं ॅवि॒द्वान् । वि॒द्वान्थ् सोमे॑न । सोमे॑न॒ यज॑ते । यज॑ते॒ भव॑ति । भव॑त्या॒त्मना᳚ । आ॒त्मना॒ परा᳚ ( ) । परा᳚ ऽस्य । अ॒स्य॒ भ्रातृ॑व्यः । भ्रातृ॑व्यो भवति । भ॒व॒तीति॑ भवति । \newline

\textbf{Jatai Paata} \newline

1. प॒शु म॑कुर्वता कुर्वत प॒शुम् प॒शु म॑कुर्वत । \newline
2. अ॒कु॒र्व॒त॒ दा॒र्श्यम् दा॒र्श्य म॑कुर्वता कुर्वत दा॒र्श्यम् । \newline
3. दा॒र्श्यं ॅय॒ज्ञ्ं ॅय॒ज्ञ्म् दा॒र्श्यम् दा॒र्श्यं ॅय॒ज्ञ्म् । \newline
4. य॒ज्ञ् मा᳚ग्ने॒य मा᳚ग्ने॒यं ॅय॒ज्ञ्ं ॅय॒ज्ञ् मा᳚ग्ने॒यम् । \newline
5. आ॒ग्ने॒यम् प॒शुम् प॒शु मा᳚ग्ने॒य मा᳚ग्ने॒यम् प॒शुम् । \newline
6. प॒शु म॑कुर्वता कुर्वत प॒शुम् प॒शु म॑कुर्वत । \newline
7. अ॒कु॒र्व॒त॒ वै॒श्व॒दे॒वं ॅवै᳚श्वदे॒व म॑कुर्वता कुर्वत वैश्वदे॒वम् । \newline
8. वै॒श्व॒दे॒वम् प्रा॑तस्सव॒नम् प्रा॑तस्सव॒नं ॅवै᳚श्वदे॒वं ॅवै᳚श्वदे॒वम् प्रा॑तस्सव॒नम् । \newline
9. वै॒श्व॒दे॒वमिति॑ वैश्व - दे॒वम् । \newline
10. प्रा॒त॒स्स॒व॒न म॑कुर्वताकुर्वत प्रातस्सव॒नम् प्रा॑तस्सव॒न म॑कुर्वत । \newline
11. प्रा॒त॒स्स॒व॒नमिति॑ प्रातः - स॒व॒नम् । \newline
12. अ॒कु॒र्व॒त॒ व॒रु॒ण॒प्र॒घा॒सान्. व॑रुणप्रघा॒सा न॑कुर्वता कुर्वत वरुणप्रघा॒सान् । \newline
13. व॒रु॒ण॒प्र॒घा॒सान् माद्ध्य॑न्दिन॒म् माद्ध्य॑न्दिनं ॅवरुणप्रघा॒सान्. व॑रुणप्रघा॒सान् माद्ध्य॑न्दिनम् । \newline
14. व॒रु॒ण॒प्र॒घा॒सानिति॑ वरुण - प्र॒घा॒सान् । \newline
15. माद्ध्य॑न्दिनꣳ॒॒ सव॑नꣳ॒॒ सव॑न॒म् माद्ध्य॑न्दिन॒म् माद्ध्य॑न्दिनꣳ॒॒ सव॑नम् । \newline
16. सव॑नꣳ साकमे॒धान् थ्सा॑कमे॒धान् थ्सव॑नꣳ॒॒ सव॑नꣳ साकमे॒धान् । \newline
17. सा॒क॒मे॒धान् पि॑तृय॒ज्ञ्म् पि॑तृय॒ज्ञ्ꣳ सा॑कमे॒धान् थ्सा॑कमे॒धान् पि॑तृय॒ज्ञ्म् । \newline
18. सा॒क॒मे॒धानिति॑ साक - मे॒धान् । \newline
19. पि॒तृ॒य॒ज्ञ्म् त्र्यं॑बकाꣳ॒॒ स्त्र्यं॑बकान् पितृय॒ज्ञ्म् पि॑तृय॒ज्ञ्म् त्र्यं॑बकान् । \newline
20. पि॒तृ॒य॒ज्ञ्मिति॑ पितृ - य॒ज्ञ्ं । \newline
21. त्र्यं॑बकाꣳ स्तृतीयसव॒नम् तृ॑तीयसव॒नम् त्र्यं॑बकाꣳ॒॒ स्त्र्यं॑बकाꣳ स्तृतीयसव॒नम् । \newline
22. त्र्यं॑बका॒निति॒ त्रि - अं॒ब॒का॒न् । \newline
23. तृ॒ती॒य॒स॒व॒न म॑कुर्वता कुर्वत तृतीयसव॒नम् तृ॑तीयसव॒न म॑कुर्वत । \newline
24. तृ॒ती॒य॒स॒व॒नमिति॑ तृतीय - स॒व॒नम् । \newline
25. अ॒कु॒र्व॒त॒ तम् त म॑कुर्वता कुर्वत॒ तम् । \newline
26. त मे॑षा मेषा॒म् तम् त मे॑षाम् । \newline
27. ए॒षा॒ मसु॑रा॒ असु॑रा एषा मेषा॒ मसु॑राः । \newline
28. असु॑रा य॒ज्ञ्ं ॅय॒ज्ञ् मसु॑रा॒ असु॑रा य॒ज्ञ्म् । \newline
29. य॒ज्ञ् म॒न्ववा॑जिगाꣳसन् न॒न्ववा॑जिगाꣳसन्. य॒ज्ञ्ं ॅय॒ज्ञ् म॒न्ववा॑जिगाꣳसन्न् । \newline
30. अ॒न्ववा॑जिगाꣳस॒न् तम् त म॒न्ववा॑जिगाꣳसन् न॒न्ववा॑जिगाꣳस॒न् तम् । \newline
31. अ॒न्ववा॑जिगाꣳस॒न्नित्य॑नु - अवा॑जिगाꣳसन्न् । \newline
32. तम् न न तम् तम् न । \newline
33. नान्ववा॑यन् न॒न्ववा॑य॒न् न नान्ववा॑यन्न् । \newline
34. अ॒न्ववा॑य॒न् ते ते᳚ ऽन्ववा॑यन् न॒न्ववा॑य॒न् ते । \newline
35. अ॒न्ववा॑य॒न्नित्य॑नु - अवा॑यन्न् । \newline
36. ते᳚ ऽब्रुवन् नब्रुव॒न् ते ते᳚ ऽब्रुवन्न् । \newline
37. अ॒ब्रु॒व॒न् न॒द्ध्व॒र्त॒व्या अ॑द्ध्वर्त॒व्या अ॑ब्रुवन् नब्रुवन् नद्ध्वर्त॒व्याः । \newline
38. अ॒द्ध्व॒र्त॒व्या वै वा अ॑द्ध्वर्त॒व्या अ॑द्ध्वर्त॒व्या वै । \newline
39. वा इ॒म इ॒मे वै वा इ॒मे । \newline
40. इ॒मे दे॒वा दे॒वा इ॒म इ॒मे दे॒वाः । \newline
41. दे॒वा अ॑भूवन् नभूवन् दे॒वा दे॒वा अ॑भूवन्न् । \newline
42. अ॒भू॒व॒न् निती त्य॑भूवन् नभूव॒न् निति॑ । \newline
43. इति॒ तत् तदितीति॒ तत् । \newline
44. तद॑द्ध्व॒रस्या᳚ द्ध्व॒रस्य॒ तत् तद॑द्ध्व॒रस्य॑ । \newline
45. अ॒द्ध्व॒रस्या᳚ द्ध्वर॒त्व म॑द्ध्वर॒त्व म॑द्ध्व॒रस्या᳚ द्ध्व॒रस्या᳚ द्ध्वर॒त्वम् । \newline
46. अ॒द्ध्व॒र॒त्वम् तत॒ स्ततो॑ अद्ध्वर॒त्व म॑द्ध्वर॒त्वम् ततः॑ । \newline
47. अ॒द्ध्व॒र॒त्वमित्य॑द्ध्वर - त्वम् । \newline
48. ततो॑ दे॒वा दे॒वा स्तत॒ स्ततो॑ दे॒वाः । \newline
49. दे॒वा अभ॑व॒न् नभ॑वन् दे॒वा दे॒वा अभ॑वन्न् । \newline
50. अभ॑व॒न् परा॒ परा ऽभ॑व॒न् नभ॑व॒न् परा᳚ । \newline
51. परा ऽसु॑रा॒ असु॑राः॒ परा॒ परा ऽसु॑राः । \newline
52. असु॑रा॒ यो यो ऽसु॑रा॒ असु॑रा॒ यः । \newline
53. य ए॒व मे॒वं ॅयो य ए॒वम् । \newline
54. ए॒वं ॅवि॒द्वान्. वि॒द्वा ने॒व मे॒वं ॅवि॒द्वान् । \newline
55. वि॒द्वान् थ्सोमे॑न॒ सोमे॑न वि॒द्वान्. वि॒द्वान् थ्सोमे॑न । \newline
56. सोमे॑न॒ यज॑ते॒ यज॑ते॒ सोमे॑न॒ सोमे॑न॒ यज॑ते । \newline
57. यज॑ते॒ भव॑ति॒ भव॑ति॒ यज॑ते॒ यज॑ते॒ भव॑ति । \newline
58. भव॑ त्या॒त्मना॒ ऽऽत्मना॒ भव॑ति॒ भव॑ त्या॒त्मना᳚ । \newline
59. आ॒त्मना॒ परा॒ परा॒ ऽऽत्मना॒ ऽऽत्मना॒ परा᳚ । \newline
60. परा᳚ ऽस्यास्य॒ परा॒ परा᳚ ऽस्य । \newline
61. अ॒स्य॒ भ्रातृ॑व्यो॒ भ्रातृ॑व्यो ऽस्यास्य॒ भ्रातृ॑व्यः । \newline
62. भ्रातृ॑व्यो भवति भवति॒ भ्रातृ॑व्यो॒ भ्रातृ॑व्यो भवति । \newline
63. भ॒व॒तीति॑ भवति । \newline

\textbf{Ghana Paata } \newline

1. प॒शु म॑कुर्वता कुर्वत प॒शुम् प॒शु म॑कुर्वत दा॒र्श्यम् दा॒र्श्य म॑कुर्वत प॒शुम् प॒शु म॑कुर्वत दा॒र्श्यम् । \newline
2. अ॒कु॒र्व॒त॒ दा॒र्श्यम् दा॒र्श्य म॑कुर्वता कुर्वत दा॒र्श्यं ॅय॒ज्ञ्ं ॅय॒ज्ञ्म् दा॒र्श्य म॑कुर्वता कुर्वत दा॒र्श्यं ॅय॒ज्ञ्म् । \newline
3. दा॒र्श्यं ॅय॒ज्ञ्ं ॅय॒ज्ञ्म् दा॒र्श्यम् दा॒र्श्यं ॅय॒ज्ञ् मा᳚ग्ने॒य मा᳚ग्ने॒यं ॅय॒ज्ञ्म् दा॒र्श्यम् दा॒र्श्यं ॅय॒ज्ञ् मा᳚ग्ने॒यम् । \newline
4. य॒ज्ञ् मा᳚ग्ने॒य मा᳚ग्ने॒यं ॅय॒ज्ञ्ं ॅय॒ज्ञ् मा᳚ग्ने॒यम् प॒शुम् प॒शु मा᳚ग्ने॒यं ॅय॒ज्ञ्ं ॅय॒ज्ञ् मा᳚ग्ने॒यम् प॒शुम् । \newline
5. आ॒ग्ने॒यम् प॒शुम् प॒शु मा᳚ग्ने॒य मा᳚ग्ने॒यम् प॒शु म॑कुर्वता कुर्वत प॒शु मा᳚ग्ने॒य मा᳚ग्ने॒यम् प॒शु म॑कुर्वत । \newline
6. प॒शु म॑कुर्वता कुर्वत प॒शुम् प॒शु म॑कुर्वत वैश्वदे॒वं ॅवै᳚श्वदे॒व म॑कुर्वत प॒शुम् प॒शु म॑कुर्वत वैश्वदे॒वम् । \newline
7. अ॒कु॒र्व॒त॒ वै॒श्व॒दे॒वं ॅवै᳚श्वदे॒व म॑कुर्वता कुर्वत वैश्वदे॒वम् प्रा॑तस्सव॒नम् प्रा॑तस्सव॒नं ॅवै᳚श्वदे॒व म॑कुर्वता कुर्वत वैश्वदे॒वम् प्रा॑तस्सव॒नम् । \newline
8. वै॒श्व॒दे॒वम् प्रा॑तस्सव॒नम् प्रा॑तस्सव॒नं ॅवै᳚श्वदे॒वं ॅवै᳚श्वदे॒वम् प्रा॑तस्सव॒न म॑कुर्वता कुर्वत प्रातस्सव॒नं ॅवै᳚श्वदे॒वं ॅवै᳚श्वदे॒वम् प्रा॑तस्सव॒न म॑कुर्वत । \newline
9. वै॒श्व॒दे॒वमिति॑ वैश्व - दे॒वम् । \newline
10. प्रा॒त॒स्स॒व॒न म॑कुर्वता कुर्वत प्रातस्सव॒नम् प्रा॑तस्सव॒न म॑कुर्वत वरुणप्रघा॒सान्. व॑रुणप्रघा॒सा न॑कुर्वत प्रातस्सव॒नम् प्रा॑तस्सव॒न म॑कुर्वत वरुणप्रघा॒सान् । \newline
11. प्रा॒त॒स्स॒व॒नमिति॑ प्रातः - स॒व॒नम् । \newline
12. अ॒कु॒र्व॒त॒ व॒रु॒ण॒प्र॒घा॒सान्. व॑रुणप्रघा॒सा न॑कुर्वता कुर्वत वरुणप्रघा॒सान् माद्ध्य॑न्दिन॒म् माद्ध्य॑न्दिनं ॅवरुणप्रघा॒सा न॑कुर्वता कुर्वत वरुणप्रघा॒सान् माद्ध्य॑न्दिनम् । \newline
13. व॒रु॒ण॒प्र॒घा॒सान् माद्ध्य॑न्दिन॒म् माद्ध्य॑न्दिनं ॅवरुणप्रघा॒सान्. व॑रुणप्रघा॒सान् माद्ध्य॑न्दिनꣳ॒॒ सव॑नꣳ॒॒ सव॑न॒म् माद्ध्य॑न्दिनं ॅवरुणप्रघा॒सान्. व॑रुणप्रघा॒सान् माद्ध्य॑न्दिनꣳ॒॒ सव॑नम् । \newline
14. व॒रु॒ण॒प्र॒घा॒सानिति॑ वरुण - प्र॒घा॒सान् । \newline
15. माद्ध्य॑न्दिनꣳ॒॒ सव॑नꣳ॒॒ सव॑न॒म् माद्ध्य॑न्दिन॒म् माद्ध्य॑न्दिनꣳ॒॒ सव॑नꣳ साकमे॒धान् थ्सा॑कमे॒धान् थ्सव॑न॒म् माद्ध्य॑न्दिन॒म् माद्ध्य॑न्दिनꣳ॒॒ सव॑नꣳ साकमे॒धान् । \newline
16. सव॑नꣳ साकमे॒धान् थ्सा॑कमे॒धान् थ्सव॑नꣳ॒॒ सव॑नꣳ साकमे॒धान् पि॑तृय॒ज्ञ्म् पि॑तृय॒ज्ञ्ꣳ सा॑कमे॒धान् थ्सव॑नꣳ॒॒ सव॑नꣳ साकमे॒धान् पि॑तृय॒ज्ञ्म् । \newline
17. सा॒क॒मे॒धान् पि॑तृय॒ज्ञ्म् पि॑तृय॒ज्ञ्ꣳ सा॑कमे॒धान् थ्सा॑कमे॒धान् पि॑तृय॒ज्ञ्म् त्र्यं॑बकाꣳ॒॒ स्त्र्यं॑बकान् पितृय॒ज्ञ्ꣳ सा॑कमे॒धान् थ्सा॑कमे॒धान् पि॑तृय॒ज्ञ्म् त्र्यं॑बकान् । \newline
18. सा॒क॒मे॒धानिति॑ साक - मे॒धान् । \newline
19. पि॒तृ॒य॒ज्ञ्म् त्र्यं॑बकाꣳ॒॒ स्त्र्यं॑बकान् पितृय॒ज्ञ्म् पि॑तृय॒ज्ञ्म् त्र्यं॑बकाꣳ स्तृतीयसव॒नम् तृ॑तीयसव॒नम् त्र्यं॑बकान् पितृय॒ज्ञ्म् पि॑तृय॒ज्ञ्म् त्र्यं॑बकाꣳ स्तृतीयसव॒नम् । \newline
20. पि॒तृ॒य॒ज्ञ्मिति॑ पितृ - य॒ज्ञ्ं । \newline
21. त्र्यं॑बकाꣳ स्तृतीयसव॒नम् तृ॑तीयसव॒नम् त्र्यं॑बकाꣳ॒॒ स्त्र्यं॑बकाꣳ स्तृतीयसव॒न म॑कुर्वता कुर्वत तृतीयसव॒नम् त्र्यं॑बकाꣳ॒॒ स्त्र्यं॑बकाꣳ स्तृतीयसव॒न म॑कुर्वत । \newline
22. त्र्यं॑बका॒निति॒ त्रि - अं॒ब॒का॒न् । \newline
23. तृ॒ती॒य॒स॒व॒न म॑कुर्वता कुर्वत तृतीयसव॒नम् तृ॑तीयसव॒न म॑कुर्वत॒ तम् त म॑कुर्वत तृतीयसव॒नम् तृ॑तीयसव॒न म॑कुर्वत॒ तम् । \newline
24. तृ॒ती॒य॒स॒व॒नमिति॑ तृतीय - स॒व॒नम् । \newline
25. अ॒कु॒र्व॒त॒ तम् त म॑कुर्वता कुर्वत॒ त मे॑षा मेषा॒म् त म॑कुर्वता कुर्वत॒ त मे॑षाम् । \newline
26. त मे॑षा मेषा॒म् तम् त मे॑षा॒ मसु॑रा॒ असु॑रा एषा॒म् तम् त मे॑षा॒ मसु॑राः । \newline
27. ए॒षा॒ मसु॑रा॒ असु॑रा एषा मेषा॒ मसु॑रा य॒ज्ञ्ं ॅय॒ज्ञ् मसु॑रा एषा मेषा॒ मसु॑रा य॒ज्ञ्म् । \newline
28. असु॑रा य॒ज्ञ्ं ॅय॒ज्ञ् मसु॑रा॒ असु॑रा य॒ज्ञ् म॒न्ववा॑जिगाꣳसन् न॒न्ववा॑जिगाꣳसन्. य॒ज्ञ् मसु॑रा॒ असु॑रा य॒ज्ञ् म॒न्ववा॑जिगाꣳसन्न् । \newline
29. य॒ज्ञ् म॒न्ववा॑जिगाꣳसन् न॒न्ववा॑जिगाꣳसन्. य॒ज्ञ्ं ॅय॒ज्ञ् म॒न्ववा॑जिगाꣳस॒न् तम् त म॒न्ववा॑जिगाꣳसन्. य॒ज्ञ्ं ॅय॒ज्ञ् म॒न्ववा॑जिगाꣳस॒न् तम् । \newline
30. अ॒न्ववा॑जिगाꣳस॒न् तम् त म॒न्ववा॑जिगाꣳसन् न॒न्ववा॑जिगाꣳस॒न् तम् न न त म॒न्ववा॑जिगाꣳसन् न॒न्ववा॑जिगाꣳस॒न् तम् न । \newline
31. अ॒न्ववा॑जिगाꣳस॒न्नित्य॑नु - अवा॑जिगाꣳसन्न् । \newline
32. तम् न न तम् तम् नान्ववा॑यन् न॒न्ववा॑य॒न् न तम् तम् नान्ववा॑यन्न् । \newline
33. नान्ववा॑यन् न॒न्ववा॑य॒न् न नान्ववा॑य॒न् ते ते᳚ ऽन्ववा॑य॒न् न नान्ववा॑य॒न् ते । \newline
34. अ॒न्ववा॑य॒न् ते ते᳚ ऽन्ववा॑यन् न॒न्ववा॑य॒न् ते᳚ ऽब्रुवन् नब्रुव॒न् ते᳚ ऽन्ववा॑यन् न॒न्ववा॑य॒न् ते᳚ ऽब्रुवन्न् । \newline
35. अ॒न्ववा॑य॒न्नित्य॑नु - अवा॑यन्न् । \newline
36. ते᳚ ऽब्रुवन् नब्रुव॒न् ते ते᳚ ऽब्रुवन् नद्ध्वर्त॒व्या अ॑द्ध्वर्त॒व्या अ॑ब्रुव॒न् ते ते᳚ ऽब्रुवन् नद्ध्वर्त॒व्याः । \newline
37. अ॒ब्रु॒व॒न् न॒द्ध्व॒र्त॒व्या अ॑द्ध्वर्त॒व्या अ॑ब्रुवन् नब्रुवन् नद्ध्वर्त॒व्या वै वा अ॑द्ध्वर्त॒व्या अ॑ब्रुवन् नब्रुवन् नद्ध्वर्त॒व्या वै । \newline
38. अ॒द्ध्व॒र्त॒व्या वै वा अ॑द्ध्वर्त॒व्या अ॑द्ध्वर्त॒व्या वा इ॒म इ॒मे वा अ॑द्ध्वर्त॒व्या अ॑द्ध्वर्त॒व्या वा इ॒मे । \newline
39. वा इ॒म इ॒मे वै वा इ॒मे दे॒वा दे॒वा इ॒मे वै वा इ॒मे दे॒वाः । \newline
40. इ॒मे दे॒वा दे॒वा इ॒म इ॒मे दे॒वा अ॑भूवन् नभूवन् दे॒वा इ॒म इ॒मे दे॒वा अ॑भूवन्न् । \newline
41. दे॒वा अ॑भूवन् नभूवन् दे॒वा दे॒वा अ॑भूव॒न् निती त्य॑भूवन् दे॒वा दे॒वा अ॑भूव॒न् निति॑ । \newline
42. अ॒भू॒व॒न् निती त्य॑भूवन् नभूव॒न् निति॒ तत् तदि त्य॑भूवन् नभूव॒न् निति॒ तत् । \newline
43. इति॒ तत् तदितीति॒ तद॑द्ध्व॒रस्या᳚ द्ध्व॒रस्य॒ तदितीति॒ तद॑द्ध्व॒रस्य॑ । \newline
44. तद॑द्ध्व॒रस्या᳚ द्ध्व॒रस्य॒ तत् तद॑द्ध्व॒रस्या᳚ द्ध्वर॒त्व म॑द्ध्वर॒त्व म॑द्ध्व॒रस्य॒ तत् तद॑द्ध्व॒रस्या᳚ द्ध्वर॒त्वम् । \newline
45. अ॒द्ध्व॒रस्या᳚ द्ध्वर॒त्व म॑द्ध्वर॒त्व म॑द्ध्व॒रस्या᳚ द्ध्व॒रस्या᳚ द्ध्वर॒त्वम् तत॒ स्ततो॑ अद्ध्वर॒त्व म॑द्ध्व॒रस्या᳚ द्ध्व॒रस्या᳚ द्ध्वर॒त्वम् ततः॑ । \newline
46. अ॒द्ध्व॒र॒त्वम् तत॒ स्ततो॑ अद्ध्वर॒त्व म॑द्ध्वर॒त्वम् ततो॑ दे॒वा दे॒वा स्ततो॑ अद्ध्वर॒त्व म॑द्ध्वर॒त्वम् ततो॑ दे॒वाः । \newline
47. अ॒द्ध्व॒र॒त्वमित्य॑द्ध्वर - त्वम् । \newline
48. ततो॑ दे॒वा दे॒वा स्तत॒ स्ततो॑ दे॒वा अभ॑व॒न् नभ॑वन् दे॒वा स्तत॒ स्ततो॑ दे॒वा अभ॑वन्न् । \newline
49. दे॒वा अभ॑व॒न् नभ॑वन् दे॒वा दे॒वा अभ॑व॒न् परा॒ परा ऽभ॑वन् दे॒वा दे॒वा अभ॑व॒न् परा᳚ । \newline
50. अभ॑व॒न् परा॒ परा ऽभ॑व॒न् नभ॑व॒न् परा ऽसु॑रा॒ असु॑राः॒ परा ऽभ॑व॒न् नभ॑व॒न् परा ऽसु॑राः । \newline
51. परा ऽसु॑रा॒ असु॑राः॒ परा॒ परा ऽसु॑रा॒ यो यो ऽसु॑राः॒ परा॒ परा ऽसु॑रा॒ यः । \newline
52. असु॑रा॒ यो यो ऽसु॑रा॒ असु॑रा॒ य ए॒व मे॒वं ॅयो ऽसु॑रा॒ असु॑रा॒ य ए॒वम् । \newline
53. य ए॒व मे॒वं ॅयो य ए॒वं ॅवि॒द्वान्. वि॒द्वा ने॒वं ॅयो य ए॒वं ॅवि॒द्वान् । \newline
54. ए॒वं ॅवि॒द्वान्. वि॒द्वा ने॒व मे॒वं ॅवि॒द्वान् थ्सोमे॑न॒ सोमे॑न वि॒द्वा ने॒व मे॒वं ॅवि॒द्वान् थ्सोमे॑न । \newline
55. वि॒द्वान् थ्सोमे॑न॒ सोमे॑न वि॒द्वान्. वि॒द्वान् थ्सोमे॑न॒ यज॑ते॒ यज॑ते॒ सोमे॑न वि॒द्वान्. वि॒द्वान् थ्सोमे॑न॒ यज॑ते । \newline
56. सोमे॑न॒ यज॑ते॒ यज॑ते॒ सोमे॑न॒ सोमे॑न॒ यज॑ते॒ भव॑ति॒ भव॑ति॒ यज॑ते॒ सोमे॑न॒ सोमे॑न॒ यज॑ते॒ भव॑ति । \newline
57. यज॑ते॒ भव॑ति॒ भव॑ति॒ यज॑ते॒ यज॑ते॒ भव॑ त्या॒त्मना॒ ऽऽत्मना॒ भव॑ति॒ यज॑ते॒ यज॑ते॒ भव॑ त्या॒त्मना᳚ । \newline
58. भव॑ त्या॒त्मना॒ ऽऽत्मना॒ भव॑ति॒ भव॑ त्या॒त्मना॒ परा॒ परा॒ ऽऽत्मना॒ भव॑ति॒ भव॑ त्या॒त्मना॒ परा᳚ । \newline
59. आ॒त्मना॒ परा॒ परा॒ ऽऽत्मना॒ ऽऽत्मना॒ परा᳚ ऽस्यास्य॒ परा॒ ऽऽत्मना॒ ऽऽत्मना॒ परा᳚ ऽस्य । \newline
60. परा᳚ ऽस्यास्य॒ परा॒ परा᳚ ऽस्य॒ भ्रातृ॑व्यो॒ भ्रातृ॑व्यो ऽस्य॒ परा॒ परा᳚ ऽस्य॒ भ्रातृ॑व्यः । \newline
61. अ॒स्य॒ भ्रातृ॑व्यो॒ भ्रातृ॑व्यो ऽस्यास्य॒ भ्रातृ॑व्यो भवति भवति॒ भ्रातृ॑व्यो ऽस्यास्य॒ भ्रातृ॑व्यो भवति । \newline
62. भ्रातृ॑व्यो भवति भवति॒ भ्रातृ॑व्यो॒ भ्रातृ॑व्यो भवति । \newline
63. भ॒व॒तीति॑ भवति । \newline
\pagebreak
\markright{ TS 3.2.3.1  \hfill https://www.vedavms.in \hfill}

\section{ TS 3.2.3.1 }

\textbf{TS 3.2.3.1 } \newline
\textbf{Samhita Paata} \newline

प॒रि॒भूर॒ग्निं प॑रि॒भूरिन्द्रं॑ परि॒भूर्विश्वा᳚न् दे॒वान् प॑रि॒भूर्माꣳ स॒ह ब्र॑ह्मवर्च॒सेन॒ स नः॑ पवस्व॒ शं गवे॒ शं जना॑य॒ शमर्व॑ते॒ शꣳ रा॑ज॒न्नोष॑धी॒भ्यो ऽच्छि॑न्नस्य ते रयिपते सु॒वीर्य॑स्य रा॒यस्पोष॑स्य ददि॒तारः॑ स्याम । तस्य॑ मे रास्व॒ तस्य॑ ते भक्षीय॒ तस्य॑ त इ॒दमुन्मृ॑जे ॥ प्रा॒णाय॑ मे वर्चो॒दा वर्च॑से पवस्वा पा॒नाय॑ व्या॒नाय॑ वा॒चे - [  ] \newline

\textbf{Pada Paata} \newline

प॒रि॒भूरिति॑ परि - भूः । अ॒ग्निम् । प॒रि॒भूरिति॑ परि - भूः । इन्द्र᳚म् । प॒रि॒भूरिति॑ परि - भूः । विश्वान्॑ । दे॒वान् । प॒रि॒भूरिति॑ परि - भूः । माम् । स॒ह । ब्र॒ह्म॒व॒र्च॒सेनेति॑ ब्रह्म - व॒र्च॒सेन॑ । सः । नः॒ । प॒व॒स्व॒ । शम् । गवे᳚ । शम् । जना॑य । शम् । अर्व॑ते । शम् । रा॒ज॒न्न् । ओष॑धीभ्य॒ इत्योष॑धि - भ्यः॒ । अच्छि॑न्नस्य । ते॒ । र॒यि॒प॒त॒ इति॑ रयि - प॒ते॒ । सु॒वीर्य॒स्येति॑ सु - वीर्य॑स्य । रा॒यः । पोष॑स्य । द॒दि॒तारः॑ । स्या॒म॒ ॥ तस्य॑ । मे॒ । रा॒स्व॒ । तस्य॑ । ते॒ । भ॒क्षी॒य॒ । तस्य॑ । ते॒ । इ॒दम् । उदिति॑ । मृ॒जे॒ ॥ प्रा॒णायेति॑ प्र -अ॒नाय॑ । मे॒ । व॒र्चो॒दा इति॑ वर्चः - दाः । वर्च॑से । प॒व॒स्व॒ । अ॒पा॒नायेत्यप॑ - अ॒नाय॑ । व्या॒नायेति॑ वि - अ॒नाय॑ । वा॒चे ।  \newline


\textbf{Krama Paata} \newline

प॒रि॒भूर॒ग्निम् । प॒रि॒भूरिति॑ परि - भूः । अ॒ग्निम् प॑रि॒भूः । प॒रि॒भूरिन्द्र᳚म् । प॒रि॒भूरिति॑ परि - भूः । इन्द्र॑म् परि॒भूः । प॒रि॒भूर् विश्वान्॑ । प॒रि॒भूरिति॑ परि - भूः । विश्वा᳚न् दे॒वान् । दे॒वान् प॑रि॒भूः । 
प॒रि॒भूर् माम् । प॒रि॒भूरिति॑ परि - भूः । माꣳ स॒ह । स॒ह ब्र॑ह्मवर्च॒सेन॑ । ब्र॒ह्म॒व॒र्च॒सेन॒ सः । ब्र॒ह्म॒व॒र्च॒सेनेति॑ ब्रह्म - व॒र्च॒सेन॑ । स नः॑ । नः॒ प॒व॒स्व॒ । प॒व॒स्व॒ शम् । शम् गवे᳚ । गवे॒ शम् । शम् जना॑य । जना॑य॒ शम् । शमर्व॑ते । अर्व॑ते॒ शम् । शꣳ रा॑जन्न् । रा॒ज॒न्नोष॑धीभ्यः । ओष॑धी॒भ्यो ऽच्छि॑न्नस्य । ओष॑धीभ्य॒ इत्योष॑धि - भ्यः॒ । अच्छि॑न्नस्य ते । ते॒ र॒यि॒प॒ते॒ । र॒यि॒प॒ते॒ सु॒वीर्य॑स्य । र॒यि॒प॒त॒ इति॑ रयि - प॒ते॒ । सु॒वीर्य॑स्य रा॒यः । सु॒वीर्य॒स्येति॑ सु - वीर्य॑स्य । रा॒यस्पोष॑स्य । पोष॑स्य ददि॒तारः॑ । द॒दि॒तारः॑ स्याम । स्या॒मेति॑ स्याम ॥ तस्य॑ मे । मे॒ रा॒स्व॒ । रा॒स्व॒ तस्य॑ । तस्य॑ ते । ते॒ भ॒क्षी॒य॒ । भ॒क्षी॒य॒ तस्य॑ । तस्य॑ ते । त॒ इ॒दम् । इ॒दमुत् । उन् मृ॑जे । मृ॒ज॒ इति॑ मृजे ॥ प्रा॒णाय॑ मे । प्रा॒णायेति॑ प्र - अ॒नाय॑ । मे॒ व॒र्चो॒दाः । व॒र्चो॒दा वर्च॑से । व॒र्चो॒दा इति॑ वर्चः - दाः । वर्च॑से पवस्व । प॒व॒स्वा॒पा॒नाय॑ । अ॒पा॒नाय॑ व्या॒नाय॑ । अ॒पा॒नायेत्य॑प - अ॒नाय॑ । व्या॒नाय॑ वा॒चे । व्या॒नायेति॑ वि - अ॒नाय॑ । वा॒चे द॑क्षक्र॒तुभ्या᳚म् \newline

\textbf{Jatai Paata} \newline

1. प॒रि॒भू र॒ग्नि म॒ग्निम् प॑रि॒भूः प॑रि॒भू र॒ग्निम् । \newline
2. प॒रि॒भूरिति॑ परि - भूः । \newline
3. अ॒ग्निम् प॑रि॒भूः प॑रि॒भू र॒ग्नि म॒ग्निम् प॑रि॒भूः । \newline
4. प॒रि॒भू रिन्द्र॒ मिन्द्र॑म् परि॒भूः प॑रि॒भू रिन्द्र᳚म् । \newline
5. प॒रि॒भूरिति॑ परि - भूः । \newline
6. इन्द्र॑म् परि॒भूः प॑रि॒भू रिन्द्र॒ मिन्द्र॑म् परि॒भूः । \newline
7. प॒रि॒भूर् विश्वा॒न्॒. विश्वा᳚न् परि॒भूः प॑रि॒भूर् विश्वान्॑ । \newline
8. प॒रि॒भूरिति॑ परि - भूः । \newline
9. विश्वा᳚न् दे॒वान् दे॒वान्. विश्वा॒न्॒. विश्वा᳚न् दे॒वान् । \newline
10. दे॒वान् प॑रि॒भूः प॑रि॒भूर् दे॒वान् दे॒वान् प॑रि॒भूः । \newline
11. प॒रि॒भूर् माम् माम् प॑रि॒भूः प॑रि॒भूर् माम् । \newline
12. प॒रि॒भूरिति॑ परि - भूः । \newline
13. माꣳ स॒ह स॒ह माम् माꣳ स॒ह । \newline
14. स॒ह ब्र॑ह्मवर्च॒सेन॑ ब्रह्मवर्च॒सेन॑ स॒ह स॒ह ब्र॑ह्मवर्च॒सेन॑ । \newline
15. ब्र॒ह्म॒व॒र्च॒सेन॒ स स ब्र॑ह्मवर्च॒सेन॑ ब्रह्मवर्च॒सेन॒ सः । \newline
16. ब्र॒ह्म॒व॒र्च॒सेनेति॑ ब्रह्म - व॒र्च॒सेन॑ । \newline
17. स नो॑ नः॒ स स नः॑ । \newline
18. नः॒ प॒व॒स्व॒ प॒व॒स्व॒ नो॒ नः॒ प॒व॒स्व॒ । \newline
19. प॒व॒स्व॒ शꣳ शम् प॑वस्व पवस्व॒ शम् । \newline
20. शम् गवे॒ गवे॒ शꣳ शम् गवे᳚ । \newline
21. गवे॒ शꣳ शम् गवे॒ गवे॒ शम् । \newline
22. शम् जना॑य॒ जना॑य॒ शꣳ शम् जना॑य । \newline
23. जना॑य॒ शꣳ शम् जना॑य॒ जना॑य॒ शम् । \newline
24. श मर्व॒ते ऽर्व॑ते॒ शꣳ श मर्व॑ते । \newline
25. अर्व॑ते॒ शꣳ श मर्व॒ते ऽर्व॑ते॒ शम् । \newline
26. शꣳ रा॑जन् राज॒ञ् छꣳ शꣳ रा॑जन्न् । \newline
27. रा॒ज॒न् नोष॑धीभ्य॒ ओष॑धीभ्यो राजन् राज॒न् नोष॑धीभ्यः । \newline
28. ओष॑धी॒भ्यो ऽच्छि॑न्न॒स्या च्छि॑न्न॒ स्यौष॑धीभ्य॒ ओष॑धी॒भ्यो ऽच्छि॑न्नस्य । \newline
29. ओष॑धीभ्य॒ इत्योष॑धि - भ्यः॒ । \newline
30. अच्छि॑न्नस्य ते॒ ते ऽच्छि॑न्न॒स्या च्छि॑न्नस्य ते । \newline
31. ते॒ र॒यि॒प॒ते॒ र॒यि॒प॒ते॒ ते॒ ते॒ र॒यि॒प॒ते॒ । \newline
32. र॒यि॒प॒ते॒ सु॒वीर्य॑स्य सु॒वीर्य॑स्य रयिपते रयिपते सु॒वीर्य॑स्य । \newline
33. र॒यि॒प॒त॒ इति॑ रयि - प॒ते॒ । \newline
34. सु॒वीर्य॑स्य रा॒यो रा॒यः सु॒वीर्य॑स्य सु॒वीर्य॑स्य रा॒यः । \newline
35. सु॒वीर्य॒स्येति॑ सु - वीर्य॑स्य । \newline
36. रा॒य स्पोष॑स्य॒ पोष॑स्य रा॒यो रा॒य स्पोष॑स्य । \newline
37. पोष॑स्य ददि॒तारो॑ ददि॒तारः॒ पोष॑स्य॒ पोष॑स्य ददि॒तारः॑ । \newline
38. द॒दि॒तारः॑ स्याम स्याम ददि॒तारो॑ ददि॒तारः॑ स्याम । \newline
39. स्या॒मेति॑ स्याम । \newline
40. तस्य॑ मे मे॒ तस्य॒ तस्य॑ मे । \newline
41. मे॒ रा॒स्व॒ रा॒स्व॒ मे॒ मे॒ रा॒स्व॒ । \newline
42. रा॒स्व॒ तस्य॒ तस्य॑ रास्व रास्व॒ तस्य॑ । \newline
43. तस्य॑ ते ते॒ तस्य॒ तस्य॑ ते । \newline
44. ते॒ भ॒क्षी॒य॒ भ॒क्षी॒य॒ ते॒ ते॒ भ॒क्षी॒य॒ । \newline
45. भ॒क्षी॒य॒ तस्य॒ तस्य॑ भक्षीय भक्षीय॒ तस्य॑ । \newline
46. तस्य॑ ते ते॒ तस्य॒ तस्य॑ ते । \newline
47. त॒ इ॒द मि॒दम् ते॑ त इ॒दम् । \newline
48. इ॒द मुदु दि॒द मि॒द मुत् । \newline
49. उन् मृ॑जे मृज॒ उदुन् मृ॑जे । \newline
50. मृ॒ज॒ इति॑ मृजे । \newline
51. प्रा॒णाय॑ मे मे प्रा॒णाय॑ प्रा॒णाय॑ मे । \newline
52. प्रा॒णायेति॑ प्र - अ॒नाय॑ । \newline
53. मे॒ व॒र्चो॒दा व॑र्चो॒दा मे॑ मे वर्चो॒दाः । \newline
54. व॒र्चो॒दा वर्च॑से॒ वर्च॑से वर्चो॒दा व॑र्चो॒दा वर्च॑से । \newline
55. व॒र्चो॒दा इति॑ वर्चः - दाः । \newline
56. वर्च॑से पवस्व पवस्व॒ वर्च॑से॒ वर्च॑से पवस्व । \newline
57. प॒व॒स्वा॒ पा॒नाया॑ पा॒नाय॑ पवस्व पवस्वा पा॒नाय॑ । \newline
58. अ॒पा॒नाय॑ व्या॒नाय॑ व्या॒नाया॑ पा॒नाया॑ पा॒नाय॑ व्या॒नाय॑ । \newline
59. अ॒पा॒नायेत्यप॑ - अ॒नाय॑ । \newline
60. व्या॒नाय॑ वा॒चे वा॒चे व्या॒नाय॑ व्या॒नाय॑ वा॒चे । \newline
61. व्या॒नायेति॑ वि - अ॒नाय॑ । \newline
62. वा॒चे द॑क्षक्र॒तुभ्या᳚म् दक्षक्र॒तुभ्यां᳚ ॅवा॒चे वा॒चे द॑क्षक्र॒तुभ्या᳚म् । \newline

\textbf{Ghana Paata } \newline

1. प॒रि॒भू र॒ग्नि म॒ग्निम् प॑रि॒भूः प॑रि॒भू र॒ग्निम् प॑रि॒भूः प॑रि॒भू र॒ग्निम् प॑रि॒भूः प॑रि॒भू र॒ग्निम् प॑रि॒भूः । \newline
2. प॒रि॒भूरिति॑ परि - भूः । \newline
3. अ॒ग्निम् प॑रि॒भूः प॑रि॒भू र॒ग्नि म॒ग्निम् प॑रि॒भू रिन्द्र॒ मिन्द्र॑म् परि॒भू र॒ग्नि म॒ग्निम् प॑रि॒भू रिन्द्र᳚म् । \newline
4. प॒रि॒भू रिन्द्र॒ मिन्द्र॑म् परि॒भूः प॑रि॒भू रिन्द्र॑म् परि॒भूः प॑रि॒भू रिन्द्र॑म् परि॒भूः प॑रि॒भू रिन्द्र॑म् परि॒भूः । \newline
5. प॒रि॒भूरिति॑ परि - भूः । \newline
6. इन्द्र॑म् परि॒भूः प॑रि॒भू रिन्द्र॒ मिन्द्र॑म् परि॒भूर् विश्वा॒न्॒. विश्वा᳚न् परि॒भू रिन्द्र॒ मिन्द्र॑म् परि॒भूर् विश्वान्॑ । \newline
7. प॒रि॒भूर् विश्वा॒न्॒. विश्वा᳚न् परि॒भूः प॑रि॒भूर् विश्वा᳚न् दे॒वान् दे॒वान्. विश्वा᳚न् परि॒भूः प॑रि॒भूर् विश्वा᳚न् दे॒वान् । \newline
8. प॒रि॒भूरिति॑ परि - भूः । \newline
9. विश्वा᳚न् दे॒वान् दे॒वान्. विश्वा॒न्॒. विश्वा᳚न् दे॒वान् प॑रि॒भूः प॑रि॒भूर् दे॒वान्. विश्वा॒न्॒. विश्वा᳚न् दे॒वान् प॑रि॒भूः । \newline
10. दे॒वान् प॑रि॒भूः प॑रि॒भूर् दे॒वान् दे॒वान् प॑रि॒भूर् माम् माम् प॑रि॒भूर् दे॒वान् दे॒वान् प॑रि॒भूर् माम् । \newline
11. प॒रि॒भूर् माम् माम् प॑रि॒भूः प॑रि॒भूर् माꣳ स॒ह स॒ह माम् प॑रि॒भूः प॑रि॒भूर् माꣳ स॒ह । \newline
12. प॒रि॒भूरिति॑ परि - भूः । \newline
13. माꣳ स॒ह स॒ह माम् माꣳ स॒ह ब्र॑ह्मवर्च॒सेन॑ ब्रह्मवर्च॒सेन॑ स॒ह माम् माꣳ स॒ह ब्र॑ह्मवर्च॒सेन॑ । \newline
14. स॒ह ब्र॑ह्मवर्च॒सेन॑ ब्रह्मवर्च॒सेन॑ स॒ह स॒ह ब्र॑ह्मवर्च॒सेन॒ स स ब्र॑ह्मवर्च॒सेन॑ स॒ह स॒ह ब्र॑ह्मवर्च॒सेन॒ सः । \newline
15. ब्र॒ह्म॒व॒र्च॒सेन॒ स स ब्र॑ह्मवर्च॒सेन॑ ब्रह्मवर्च॒सेन॒ स नो॑ नः॒ स ब्र॑ह्मवर्च॒सेन॑ ब्रह्मवर्च॒सेन॒ स नः॑ । \newline
16. ब्र॒ह्म॒व॒र्च॒सेनेति॑ ब्रह्म - व॒र्च॒सेन॑ । \newline
17. स नो॑ नः॒ स स नः॑ पवस्व पवस्व नः॒ स स नः॑ पवस्व । \newline
18. नः॒ प॒व॒स्व॒ प॒व॒स्व॒ नो॒ नः॒ प॒व॒स्व॒ शꣳ शम् प॑वस्व नो नः पवस्व॒ शम् । \newline
19. प॒व॒स्व॒ शꣳ शम् प॑वस्व पवस्व॒ शम् गवे॒ गवे॒ शम् प॑वस्व पवस्व॒ शम् गवे᳚ । \newline
20. शम् गवे॒ गवे॒ शꣳ शम् गवे॒ शꣳ शम् गवे॒ शꣳ शम् गवे॒ शम् । \newline
21. गवे॒ शꣳ शम् गवे॒ गवे॒ शम् जना॑य॒ जना॑य॒ शम् गवे॒ गवे॒ शम् जना॑य । \newline
22. शम् जना॑य॒ जना॑य॒ शꣳ शम् जना॑य॒ शꣳ शम् जना॑य॒ शꣳ शम् जना॑य॒ शम् । \newline
23. जना॑य॒ शꣳ शम् जना॑य॒ जना॑य॒ श मर्व॒ते ऽर्व॑ते॒ शम् जना॑य॒ जना॑य॒ श मर्व॑ते । \newline
24. श मर्व॒ते ऽर्व॑ते॒ शꣳ श मर्व॑ते॒ शꣳ श मर्व॑ते॒ शꣳ श मर्व॑ते॒ शम् । \newline
25. अर्व॑ते॒ शꣳ श मर्व॒ते ऽर्व॑ते॒ शꣳ रा॑जन् राज॒ञ् छमर्व॒ते ऽर्व॑ते॒ शꣳ रा॑जन्न् । \newline
26. शꣳ रा॑जन् राज॒ञ् छꣳ शꣳ रा॑ज॒न् नोष॑धीभ्य॒ ओष॑धीभ्यो राज॒ञ् छꣳ शꣳ रा॑ज॒न् नोष॑धीभ्यः । \newline
27. रा॒ज॒न् नोष॑धीभ्य॒ ओष॑धीभ्यो राजन् राज॒न् नोष॑धी॒भ्यो ऽच्छि॑न्न॒स्या च्छि॑न्न॒ स्यौष॑धीभ्यो राजन् राज॒न् नोष॑धी॒भ्यो ऽच्छि॑न्नस्य । \newline
28. ओष॑धी॒भ्यो ऽच्छि॑न्न॒स्या च्छि॑न्न॒ स्यौष॑धीभ्य॒ ओष॑धी॒भ्यो ऽच्छि॑न्नस्य ते॒ ते ऽच्छि॑न्न॒ स्यौष॑धीभ्य॒ ओष॑धी॒भ्यो ऽच्छि॑न्नस्य ते । \newline
29. ओष॑धीभ्य॒ इत्योष॑धि - भ्यः॒ । \newline
30. अच्छि॑न्नस्य ते॒ ते ऽच्छि॑न्न॒स्या च्छि॑न्नस्य ते रयिपते रयिपते॒ ते ऽच्छि॑न्न॒स्या च्छि॑न्नस्य ते रयिपते । \newline
31. ते॒ र॒यि॒प॒ते॒ र॒यि॒प॒ते॒ ते॒ ते॒ र॒यि॒प॒ते॒ सु॒वीर्य॑स्य सु॒वीर्य॑स्य रयिपते ते ते रयिपते सु॒वीर्य॑स्य । \newline
32. र॒यि॒प॒ते॒ सु॒वीर्य॑स्य सु॒वीर्य॑स्य रयिपते रयिपते सु॒वीर्य॑स्य रा॒यो रा॒यः सु॒वीर्य॑स्य रयिपते रयिपते सु॒वीर्य॑स्य रा॒यः । \newline
33. र॒यि॒प॒त॒ इति॑ रयि - प॒ते॒ । \newline
34. सु॒वीर्य॑स्य रा॒यो रा॒यः सु॒वीर्य॑स्य सु॒वीर्य॑स्य रा॒य स्पोष॑स्य॒ पोष॑स्य रा॒यः सु॒वीर्य॑स्य सु॒वीर्य॑स्य रा॒य स्पोष॑स्य । \newline
35. सु॒वीर्य॒स्येति॑ सु - वीर्य॑स्य । \newline
36. रा॒य स्पोष॑स्य॒ पोष॑स्य रा॒यो रा॒य स्पोष॑स्य ददि॒तारो॑ ददि॒तारः॒ पोष॑स्य रा॒यो रा॒य स्पोष॑स्य ददि॒तारः॑ । \newline
37. पोष॑स्य ददि॒तारो॑ ददि॒तारः॒ पोष॑स्य॒ पोष॑स्य ददि॒तारः॑ स्याम स्याम ददि॒तारः॒ पोष॑स्य॒ पोष॑स्य ददि॒तारः॑ स्याम । \newline
38. द॒दि॒तारः॑ स्याम स्याम ददि॒तारो॑ ददि॒तारः॑ स्याम । \newline
39. स्या॒मेति॑ स्याम । \newline
40. तस्य॑ मे मे॒ तस्य॒ तस्य॑ मे रास्व रास्व मे॒ तस्य॒ तस्य॑ मे रास्व । \newline
41. मे॒ रा॒स्व॒ रा॒स्व॒ मे॒ मे॒ रा॒स्व॒ तस्य॒ तस्य॑ रास्व मे मे रास्व॒ तस्य॑ । \newline
42. रा॒स्व॒ तस्य॒ तस्य॑ रास्व रास्व॒ तस्य॑ ते ते॒ तस्य॑ रास्व रास्व॒ तस्य॑ ते । \newline
43. तस्य॑ ते ते॒ तस्य॒ तस्य॑ ते भक्षीय भक्षीय ते॒ तस्य॒ तस्य॑ ते भक्षीय । \newline
44. ते॒ भ॒क्षी॒य॒ भ॒क्षी॒य॒ ते॒ ते॒ भ॒क्षी॒य॒ तस्य॒ तस्य॑ भक्षीय ते ते भक्षीय॒ तस्य॑ । \newline
45. भ॒क्षी॒य॒ तस्य॒ तस्य॑ भक्षीय भक्षीय॒ तस्य॑ ते ते॒ तस्य॑ भक्षीय भक्षीय॒ तस्य॑ ते । \newline
46. तस्य॑ ते ते॒ तस्य॒ तस्य॑ त इ॒द मि॒दम् ते॒ तस्य॒ तस्य॑ त इ॒दम् । \newline
47. त॒ इ॒द मि॒दम् ते॑ त इ॒द मुदु दि॒दम् ते॑ त इ॒द मुत् । \newline
48. इ॒द मुदु दि॒द मि॒द मुन् मृ॑जे मृज॒ उदि॒द मि॒द मुन् मृ॑जे । \newline
49. उन् मृ॑जे मृज॒ उदुन् मृ॑जे । \newline
50. मृ॒ज॒ इति॑ मृजे । \newline
51. प्रा॒णाय॑ मे मे प्रा॒णाय॑ प्रा॒णाय॑ मे वर्चो॒दा व॑र्चो॒दा मे᳚ प्रा॒णाय॑ प्रा॒णाय॑ मे वर्चो॒दाः । \newline
52. प्रा॒णायेति॑ प्र - अ॒नाय॑ । \newline
53. मे॒ व॒र्चो॒दा व॑र्चो॒दा मे॑ मे वर्चो॒दा वर्च॑से॒ वर्च॑से वर्चो॒दा मे॑ मे वर्चो॒दा वर्च॑से । \newline
54. व॒र्चो॒दा वर्च॑से॒ वर्च॑से वर्चो॒दा व॑र्चो॒दा वर्च॑से पवस्व पवस्व॒ वर्च॑से वर्चो॒दा व॑र्चो॒दा वर्च॑से पवस्व । \newline
55. व॒र्चो॒दा इति॑ वर्चः - दाः । \newline
56. वर्च॑से पवस्व पवस्व॒ वर्च॑से॒ वर्च॑से पवस्वा पा॒नाया॑ पा॒नाय॑ पवस्व॒ वर्च॑से॒ वर्च॑से पवस्वा पा॒नाय॑ । \newline
57. प॒व॒स्वा॒ पा॒नाया॑ पा॒नाय॑ पवस्व पवस्वा पा॒नाय॑ व्या॒नाय॑ व्या॒नाया॑ पा॒नाय॑ पवस्व पवस्वा पा॒नाय॑ व्या॒नाय॑ । \newline
58. अ॒पा॒नाय॑ व्या॒नाय॑ व्या॒नाया॑ पा॒नाया॑ पा॒नाय॑ व्या॒नाय॑ वा॒चे वा॒चे व्या॒नाया॑ पा॒नाया॑ पा॒नाय॑ व्या॒नाय॑ वा॒चे । \newline
59. अ॒पा॒नायेत्यप॑ - अ॒नाय॑ । \newline
60. व्या॒नाय॑ वा॒चे वा॒चे व्या॒नाय॑ व्या॒नाय॑ वा॒चे द॑क्षक्र॒तुभ्या᳚म् दक्षक्र॒तुभ्यां᳚ ॅवा॒चे व्या॒नाय॑ व्या॒नाय॑ वा॒चे द॑क्षक्र॒तुभ्या᳚म् । \newline
61. व्या॒नायेति॑ वि - अ॒नाय॑ । \newline
62. वा॒चे द॑क्षक्र॒तुभ्या᳚म् दक्षक्र॒तुभ्यां᳚ ॅवा॒चे वा॒चे द॑क्षक्र॒तुभ्या॒म् चक्षु॑र्भ्या॒म् चक्षु॑र्भ्याम् दक्षक्र॒तुभ्यां᳚ ॅवा॒चे वा॒चे द॑क्षक्र॒तुभ्या॒म् चक्षु॑र्भ्याम् । \newline
\pagebreak
\markright{ TS 3.2.3.2  \hfill https://www.vedavms.in \hfill}

\section{ TS 3.2.3.2 }

\textbf{TS 3.2.3.2 } \newline
\textbf{Samhita Paata} \newline

द॑क्षक्र॒तुभ्यां॒ चक्षु॑र्भ्यां मे वर्चो॒दौ वर्च॑से पवेथाꣳ॒॒ श्रोत्रा॑या॒ ऽऽ*त्मने ऽङ्गे᳚भ्य॒ आयु॑षे वी॒र्या॑य॒ विष्णो॒रिन्द्र॑स्य॒ विश्वे॑षां दे॒वानां᳚ ज॒ठर॑मसि वर्चो॒दा मे॒ वर्च॑से पवस्व॒ को॑ऽसि॒ को नाम॒ कस्मै᳚ त्वा॒ काय॑ त्वा॒ यं त्वा॒ सोमे॒नाती॑तृपं॒ ॅयं त्वा॒ सोमे॒नामी॑मदꣳ सुप्र॒जाः प्र॒जया॑ भूयासꣳ सु॒वीरो॑ वी॒रैः सु॒वर्चा॒ वर्च॑सा सु॒पोषः॒ पोषै॒-र्विश्वे᳚भ्यो मे रू॒पेभ्यो॑ वर्चो॒दा- [  ] \newline

\textbf{Pada Paata} \newline

द॒क्ष॒क्र॒तुभ्या॒मिति॑ दक्षक्र॒तु - भ्या॒म् । चक्षु॑र्भ्या॒मिति॒ चक्षुः॑ - भ्या॒म् । मे॒ । व॒र्चो॒दाविति॑ वर्चः-दौ । वर्च॑से । प॒वे॒था॒म् । श्रोत्रा॑य । आ॒त्मने᳚ । अङ्गे᳚भ्यः । आयु॑षे । वी॒र्या॑य । विष्णोः᳚ । इन्द्र॑स्य । विश्वे॑षाम् । दे॒वाना᳚म् । ज॒ठर᳚म् । अ॒सि॒ । व॒र्चो॒दा इति॑ वर्चः-दाः । मे॒ । वर्च॑से । प॒व॒स्व॒ । कः । अ॒सि॒ । कः । नाम॑ । कस्मै᳚ । त्वा॒ । काय॑ । त्वा॒ । यम् । त्वा॒ । सोमे॑न । अती॑तृपम् । यम् । त्वा॒ । सोमे॑न । अमी॑मदम् । सु॒प्र॒जा इति॑ सु - प्र॒जाः । प्र॒जयेति॑ प्र - जया᳚ । भू॒या॒स॒म् । सु॒वीर॒ इति॑ सु - वीरः॑ । वी॒रैः । सु॒वर्चा॒ इति॑ सु - वर्चाः᳚ । वर्च॑सा । सु॒पोष॒ इति॑ सु - पोषः॑ । पोषैः᳚ । विश्वे᳚भ्यः । मे॒ । रू॒पेभ्यः॑ । व॒र्चो॒दा इति॑ वर्चः - दाः ।  \newline


\textbf{Krama Paata} \newline

द॒क्ष॒क्र॒तुभ्या॒म् चक्षु॑र्भ्याम् । द॒क्ष॒क्र॒तुभ्या॒मिति॑ दक्षक्र॒तु - भ्या॒म् । चक्षु॑र्भ्याम् मे । चक्षु॑र्भ्या॒मिति॒ चक्षुः॑ - भ्या॒म् । मे॒ व॒र्चो॒दौ । व॒र्चो॒दौ वर्च॑से । व॒र्चो॒दाविति॑ वर्चः - दौ । वर्च॑से पवेथाम् । प॒वे॒थाꣳ॒॒ श्रोत्रा॑य । श्रोत्रा॑या॒त्मने᳚ । आ॒त्मने ऽङ्गे᳚भ्यः । अङ्गे᳚भ्य॒ आयु॑षे । आयु॑षे वी॒र्या॑य । वी॒र्या॑य॒ विष्णोः᳚ । विष्णो॒रिन्द्र॑स्य । इन्द्र॑स्य॒ विश्वे॑षाम् । विश्वे॑षाम् दे॒वाना᳚म् । दे॒वाना᳚म् ज॒ठर᳚म् । ज॒ठर॑मसि । अ॒सि॒ व॒र्चो॒दाः । व॒र्चो॒दा मे᳚ । व॒र्चो॒दा इति॑ वर्चः - दाः । मे॒ वर्च॑से । वर्च॑से पवस्व । प॒व॒स्व॒ कः । को॑ ऽसि । अ॒सि॒ कः । को नाम॑ । नाम॒ कस्मै᳚ । कस्मै᳚ त्वा । त्वा॒ काय॑ । काय॑ त्वा । त्वा॒ यम् । यम् त्वा᳚ । त्वा॒ सोमे॑न । सोमे॒नाती॑तृपम् । अती॑तृपं॒ ॅयम् । यम् त्वा᳚ । त्वा॒ सोमे॑न । सोमे॒नामी॑मदम् । अमी॑मदꣳ सुप्र॒जाः । सु॒प्र॒जाः प्र॒जया᳚ । सु॒प्र॒जा इति॑ सु - प्र॒जाः । प्र॒जया॑ भूयासम् । प्र॒जयेति॑ प्र - जया᳚ । भू॒या॒सꣳ॒॒ सु॒वीरः॑ । सु॒वीरो॑ वी॒रैः । सु॒वीर॒ इति॑ सु - वीरः॑ । वी॒रैः सु॒वर्चाः᳚ । सु॒वर्चा॒ वर्च॑सा । सु॒वर्चा॒ इति॑ सु - वर्चाः᳚ । वर्च॑सा सु॒पोषः॑ । सु॒पोषः॒ पोषैः᳚ । सु॒पोष॒ इति॑ सु - पोषः॑ । पोषै॒र् विश्वे᳚भ्यः । विश्वे᳚भ्यो मे । मे॒ रू॒पेभ्यः॑ । रू॒पेभ्यो॑ वर्चो॒दाः । व॒र्चो॒दा वर्च॑से । व॒र्चो॒दा इति॑ वर्चः - दाः \newline

\textbf{Jatai Paata} \newline

1. द॒क्ष॒क्र॒तुभ्या॒म् चक्षु॑र्भ्या॒म् चक्षु॑र्भ्याम् दक्षक्र॒तुभ्या᳚म् दक्षक्र॒तुभ्या॒म् चक्षु॑र्भ्याम् । \newline
2. द॒क्ष॒क्र॒तुभ्या॒मिति॑ दक्षक्र॒तु - भ्या॒म् । \newline
3. चक्षु॑र्भ्याम् मे मे॒ चक्षु॑र्भ्या॒म् चक्षु॑र्भ्याम् मे । \newline
4. चक्षु॑र्भ्या॒मिति॒ चक्षुः॑ - भ्या॒म् । \newline
5. मे॒ व॒र्चो॒दौ व॑र्चो॒दौ मे॑ मे वर्चो॒दौ । \newline
6. व॒र्चो॒दौ वर्च॑से॒ वर्च॑से वर्चो॒दौ व॑र्चो॒दौ वर्च॑से । \newline
7. व॒र्चो॒दाविति॑ वर्चः - दौ । \newline
8. वर्च॑से पवेथाम् पवेथां॒ ॅवर्च॑से॒ वर्च॑से पवेथाम् । \newline
9. प॒वे॒थाꣳ॒॒ श्रोत्रा॑य॒ श्रोत्रा॑य पवेथाम् पवेथाꣳ॒॒ श्रोत्रा॑य । \newline
10. श्रोत्रा॑या॒त्मन॑ आ॒त्मने॒ श्रोत्रा॑य॒ श्रोत्रा॑या॒त्मने᳚ । \newline
11. आ॒त्मने ऽङ्गे॒भ्यो ऽङ्गे᳚भ्य आ॒त्मन॑ आ॒त्मने ऽङ्गे᳚भ्यः । \newline
12. अङ्गे᳚भ्य॒ आयु॑ष॒ आयु॒षे ऽङ्गे॒भ्यो ऽङ्गे᳚भ्य॒ आयु॑षे । \newline
13. आयु॑षे वी॒र्या॑य वी॒र्या॑या यु॑ष॒ आयु॑षे वी॒र्या॑य । \newline
14. वी॒र्या॑य॒ विष्णो॒र् विष्णो᳚र् वी॒र्या॑य वी॒र्या॑य॒ विष्णोः᳚ । \newline
15. विष्णो॒ रिन्द्र॒स्ये न्द्र॑स्य॒ विष्णो॒र् विष्णो॒ रिन्द्र॑स्य । \newline
16. इन्द्र॑स्य॒ विश्वे॑षां॒ ॅविश्वे॑षा॒ मिन्द्र॒स्ये न्द्र॑स्य॒ विश्वे॑षाम् । \newline
17. विश्वे॑षाम् दे॒वाना᳚म् दे॒वानां॒ ॅविश्वे॑षां॒ ॅविश्वे॑षाम् दे॒वाना᳚म् । \newline
18. दे॒वाना᳚म् ज॒ठर॑म् ज॒ठर॑म् दे॒वाना᳚म् दे॒वाना᳚म् ज॒ठर᳚म् । \newline
19. ज॒ठर॑ मस्यसि ज॒ठर॑म् ज॒ठर॑ मसि । \newline
20. अ॒सि॒ व॒र्चो॒दा व॑र्चो॒दा अ॑स्यसि वर्चो॒दाः । \newline
21. व॒र्चो॒दा मे॑ मे वर्चो॒दा व॑र्चो॒दा मे᳚ । \newline
22. व॒र्चो॒दा इति॑ वर्चः - दाः । \newline
23. मे॒ वर्च॑से॒ वर्च॑से मे मे॒ वर्च॑से । \newline
24. वर्च॑से पवस्व पवस्व॒ वर्च॑से॒ वर्च॑से पवस्व । \newline
25. प॒व॒स्व॒ कः कः प॑वस्व पवस्व॒ कः । \newline
26. को᳚ ऽस्यसि॒ कः को॑ ऽसि । \newline
27. अ॒सि॒ कः को᳚ ऽस्यसि॒ कः । \newline
28. को नाम॒ नाम॒ कः को नाम॑ । \newline
29. नाम॒ कस्मै॒ कस्मै॒ नाम॒ नाम॒ कस्मै᳚ । \newline
30. कस्मै᳚ त्वा त्वा॒ कस्मै॒ कस्मै᳚ त्वा । \newline
31. त्वा॒ काय॒ काय॑ त्वा त्वा॒ काय॑ । \newline
32. काय॑ त्वा त्वा॒ काय॒ काय॑ त्वा । \newline
33. त्वा॒ यं ॅयम् त्वा᳚ त्वा॒ यम् । \newline
34. यम् त्वा᳚ त्वा॒ यं ॅयम् त्वा᳚ । \newline
35. त्वा॒ सोमे॑न॒ सोमे॑न त्वा त्वा॒ सोमे॑न । \newline
36. सोमे॒ना ती॑तृप॒ मती॑तृपꣳ॒॒ सोमे॑न॒ सोमे॒ना ती॑तृपम् । \newline
37. अती॑तृपं॒ ॅयं ॅय मती॑तृप॒ मती॑तृपं॒ ॅयम् । \newline
38. यम् त्वा᳚ त्वा॒ यं ॅयम् त्वा᳚ । \newline
39. त्वा॒ सोमे॑न॒ सोमे॑न त्वा त्वा॒ सोमे॑न । \newline
40. सोमे॒ना मी॑मद॒ ममी॑मदꣳ॒॒ सोमे॑न॒ सोमे॒ना मी॑मदम् । \newline
41. अमी॑मदꣳ सुप्र॒जाः सु॑प्र॒जा अमी॑मद॒ ममी॑मदꣳ सुप्र॒जाः । \newline
42. सु॒प्र॒जाः प्र॒जया᳚ प्र॒जया॑ सुप्र॒जाः सु॑प्र॒जाः प्र॒जया᳚ । \newline
43. सु॒प्र॒जा इति॑ सु - प्र॒जाः । \newline
44. प्र॒जया॑ भूयासम् भूयासम् प्र॒जया᳚ प्र॒जया॑ भूयासम् । \newline
45. प्र॒जयेति॑ प्र - जया᳚ । \newline
46. भू॒या॒सꣳ॒॒ सु॒वीरः॑ सु॒वीरो॑ भूयासम् भूयासꣳ सु॒वीरः॑ । \newline
47. सु॒वीरो॑ वी॒रैर् वी॒रैः सु॒वीरः॑ सु॒वीरो॑ वी॒रैः । \newline
48. सु॒वीर॒ इति॑ सु - वीरः॑ । \newline
49. वी॒रैः सु॒वर्चाः᳚ सु॒वर्चा॑ वी॒रैर् वी॒रैः सु॒वर्चाः᳚ । \newline
50. सु॒वर्चा॒ वर्च॑सा॒ वर्च॑सा सु॒वर्चाः᳚ सु॒वर्चा॒ वर्च॑सा । \newline
51. सु॒वर्चा॒ इति॑ सु - वर्चाः᳚ । \newline
52. वर्च॑सा सु॒पोषः॑ सु॒पोषो॒ वर्च॑सा॒ वर्च॑सा सु॒पोषः॑ । \newline
53. सु॒पोषः॒ पोषैः॒ पोषैः᳚ सु॒पोषः॑ सु॒पोषः॒ पोषैः᳚ । \newline
54. सु॒पोष॒ इति॑ सु - पोषः॑ । \newline
55. पोषै॒र् विश्वे᳚भ्यो॒ विश्वे᳚भ्यः॒ पोषैः॒ पोषै॒र् विश्वे᳚भ्यः । \newline
56. विश्वे᳚भ्यो मे मे॒ विश्वे᳚भ्यो॒ विश्वे᳚भ्यो मे । \newline
57. मे॒ रू॒पेभ्यो॑ रू॒पेभ्यो॑ मे मे रू॒पेभ्यः॑ । \newline
58. रू॒पेभ्यो॑ वर्चो॒दा व॑र्चो॒दा रू॒पेभ्यो॑ रू॒पेभ्यो॑ वर्चो॒दाः । \newline
59. व॒र्चो॒दा वर्च॑से॒ वर्च॑से वर्चो॒दा व॑र्चो॒दा वर्च॑से । \newline
60. व॒र्चो॒दा इति॑ वर्चः - दाः । \newline

\textbf{Ghana Paata } \newline

1. द॒क्ष॒क्र॒तुभ्या॒म् चक्षु॑र्भ्या॒म् चक्षु॑र्भ्याम् दक्षक्र॒तुभ्या᳚म् दक्षक्र॒तुभ्या॒म् चक्षु॑र्भ्याम् मे मे॒ चक्षु॑र्भ्याम् दक्षक्र॒तुभ्या᳚म् दक्षक्र॒तुभ्या॒म् चक्षु॑र्भ्याम् मे । \newline
2. द॒क्ष॒क्र॒तुभ्या॒मिति॑ दक्षक्र॒तु - भ्या॒म् । \newline
3. चक्षु॑र्भ्याम् मे मे॒ चक्षु॑र्भ्या॒म् चक्षु॑र्भ्याम् मे वर्चो॒दौ व॑र्चो॒दौ मे॒ चक्षु॑र्भ्या॒म् चक्षु॑र्भ्याम् मे वर्चो॒दौ । \newline
4. चक्षु॑र्भ्या॒मिति॒ चक्षुः॑ - भ्या॒म् । \newline
5. मे॒ व॒र्चो॒दौ व॑र्चो॒दौ मे॑ मे वर्चो॒दौ वर्च॑से॒ वर्च॑से वर्चो॒दौ मे॑ मे वर्चो॒दौ वर्च॑से । \newline
6. व॒र्चो॒दौ वर्च॑से॒ वर्च॑से वर्चो॒दौ व॑र्चो॒दौ वर्च॑से पवेथाम् पवेथां॒ ॅवर्च॑से वर्चो॒दौ व॑र्चो॒दौ वर्च॑से पवेथाम् । \newline
7. व॒र्चो॒दाविति॑ वर्चः - दौ । \newline
8. वर्च॑से पवेथाम् पवेथां॒ ॅवर्च॑से॒ वर्च॑से पवेथाꣳ॒॒ श्रोत्रा॑य॒ श्रोत्रा॑य पवेथां॒ ॅवर्च॑से॒ वर्च॑से पवेथाꣳ॒॒ श्रोत्रा॑य । \newline
9. प॒वे॒थाꣳ॒॒ श्रोत्रा॑य॒ श्रोत्रा॑य पवेथाम् पवेथाꣳ॒॒ श्रोत्रा॑या॒ त्मन॑ आ॒त्मने॒ श्रोत्रा॑य पवेथाम् पवेथाꣳ॒॒ श्रोत्रा॑या॒ त्मने᳚ । \newline
10. श्रोत्रा॑या॒ त्मन॑ आ॒त्मने॒ श्रोत्रा॑य॒ श्रोत्रा॑या॒ त्मने ऽङ्गे॒भ्यो ऽङ्गे᳚भ्य आ॒त्मने॒ श्रोत्रा॑य॒ श्रोत्रा॑या॒ त्मने ऽङ्गे᳚भ्यः । \newline
11. आ॒त्मने ऽङ्गे॒भ्यो ऽङ्गे᳚भ्य आ॒त्मन॑ आ॒त्मने ऽङ्गे᳚भ्य॒ आयु॑ष॒ आयु॒षे ऽङ्गे᳚भ्य आ॒त्मन॑ आ॒त्मने ऽङ्गे᳚भ्य॒ आयु॑षे । \newline
12. अङ्गे᳚भ्य॒ आयु॑ष॒ आयु॒षे ऽङ्गे॒भ्यो ऽङ्गे᳚भ्य॒ आयु॑षे वी॒र्या॑य वी॒र्या॑या यु॒षे ऽङ्गे॒भ्यो ऽङ्गे᳚भ्य॒ आयु॑षे वी॒र्या॑य । \newline
13. आयु॑षे वी॒र्या॑य वी॒र्या॑या यु॑ष॒ आयु॑षे वी॒र्या॑य॒ विष्णो॒र् विष्णो᳚र् वी॒र्या॑या यु॑ष॒ आयु॑षे वी॒र्या॑य॒ विष्णोः᳚ । \newline
14. वी॒र्या॑य॒ विष्णो॒र् विष्णो᳚र् वी॒र्या॑य वी॒र्या॑य॒ विष्णो॒ रिन्द्र॒स्ये न्द्र॑स्य॒ विष्णो᳚र् वी॒र्या॑य वी॒र्या॑य॒ विष्णो॒ रिन्द्र॑स्य । \newline
15. विष्णो॒ रिन्द्र॒स्ये न्द्र॑स्य॒ विष्णो॒र् विष्णो॒ रिन्द्र॑स्य॒ विश्वे॑षां॒ ॅविश्वे॑षा॒ मिन्द्र॑स्य॒ विष्णो॒र् विष्णो॒ रिन्द्र॑स्य॒ विश्वे॑षाम् । \newline
16. इन्द्र॑स्य॒ विश्वे॑षां॒ ॅविश्वे॑षा॒ मिन्द्र॒स्ये न्द्र॑स्य॒ विश्वे॑षाम् दे॒वाना᳚म् दे॒वानां॒ ॅविश्वे॑षा॒ मिन्द्र॒स्ये न्द्र॑स्य॒ विश्वे॑षाम् दे॒वाना᳚म् । \newline
17. विश्वे॑षाम् दे॒वाना᳚म् दे॒वानां॒ ॅविश्वे॑षां॒ ॅविश्वे॑षाम् दे॒वाना᳚म् ज॒ठर॑म् ज॒ठर॑म् दे॒वानां॒ ॅविश्वे॑षां॒ ॅविश्वे॑षाम् दे॒वाना᳚म् ज॒ठर᳚म् । \newline
18. दे॒वाना᳚म् ज॒ठर॑म् ज॒ठर॑म् दे॒वाना᳚म् दे॒वाना᳚म् ज॒ठर॑ मस्यसि ज॒ठर॑म् दे॒वाना᳚म् दे॒वाना᳚म् ज॒ठर॑ मसि । \newline
19. ज॒ठर॑ मस्यसि ज॒ठर॑म् ज॒ठर॑ मसि वर्चो॒दा व॑र्चो॒दा अ॑सि ज॒ठर॑म् ज॒ठर॑ मसि वर्चो॒दाः । \newline
20. अ॒सि॒ व॒र्चो॒दा व॑र्चो॒दा अ॑स्यसि वर्चो॒दा मे॑ मे वर्चो॒दा अ॑स्यसि वर्चो॒दा मे᳚ । \newline
21. व॒र्चो॒दा मे॑ मे वर्चो॒दा व॑र्चो॒दा मे॒ वर्च॑से॒ वर्च॑से मे वर्चो॒दा व॑र्चो॒दा मे॒ वर्च॑से । \newline
22. व॒र्चो॒दा इति॑ वर्चः - दाः । \newline
23. मे॒ वर्च॑से॒ वर्च॑से मे मे॒ वर्च॑से पवस्व पवस्व॒ वर्च॑से मे मे॒ वर्च॑से पवस्व । \newline
24. वर्च॑से पवस्व पवस्व॒ वर्च॑से॒ वर्च॑से पवस्व॒ कः कः प॑वस्व॒ वर्च॑से॒ वर्च॑से पवस्व॒ कः । \newline
25. प॒व॒स्व॒ कः कः प॑वस्व पवस्व॒ को᳚ ऽस्यसि॒ कः प॑वस्व पवस्व॒ को॑ ऽसि । \newline
26. को᳚ ऽस्यसि॒ कः को॑ ऽसि॒ कः को॑ ऽसि॒ कः को॑ ऽसि॒ कः । \newline
27. अ॒सि॒ कः को᳚ ऽस्यसि॒ को नाम॒ नाम॒ को᳚ ऽस्यसि॒ को नाम॑ । \newline
28. को नाम॒ नाम॒ कः को नाम॒ कस्मै॒ कस्मै॒ नाम॒ कः को नाम॒ कस्मै᳚ । \newline
29. नाम॒ कस्मै॒ कस्मै॒ नाम॒ नाम॒ कस्मै᳚ त्वा त्वा॒ कस्मै॒ नाम॒ नाम॒ कस्मै᳚ त्वा । \newline
30. कस्मै᳚ त्वा त्वा॒ कस्मै॒ कस्मै᳚ त्वा॒ काय॒ काय॑ त्वा॒ कस्मै॒ कस्मै᳚ त्वा॒ काय॑ । \newline
31. त्वा॒ काय॒ काय॑ त्वा त्वा॒ काय॑ त्वा त्वा॒ काय॑ त्वा त्वा॒ काय॑ त्वा । \newline
32. काय॑ त्वा त्वा॒ काय॒ काय॑ त्वा॒ यं ॅयम् त्वा॒ काय॒ काय॑ त्वा॒ यम् । \newline
33. त्वा॒ यं ॅयम् त्वा᳚ त्वा॒ यम् त्वा᳚ त्वा॒ यम् त्वा᳚ त्वा॒ यम् त्वा᳚ । \newline
34. यम् त्वा᳚ त्वा॒ यं ॅयम् त्वा॒ सोमे॑न॒ सोमे॑न त्वा॒ यं ॅयम् त्वा॒ सोमे॑न । \newline
35. त्वा॒ सोमे॑न॒ सोमे॑न त्वा त्वा॒ सोमे॒ना ती॑तृप॒ मती॑तृपꣳ॒॒ सोमे॑न त्वा त्वा॒ सोमे॒ना ती॑तृपम् । \newline
36. सोमे॒ना ती॑तृप॒ मती॑तृपꣳ॒॒ सोमे॑न॒ सोमे॒ना ती॑तृपं॒ ॅयं ॅय मती॑तृपꣳ॒॒ सोमे॑न॒ सोमे॒ना ती॑तृपं॒ ॅयम् । \newline
37. अती॑तृपं॒ ॅयं ॅय मती॑तृप॒ मती॑तृपं॒ ॅयम् त्वा᳚ त्वा॒ य मती॑तृप॒ मती॑तृपं॒ ॅयम् त्वा᳚ । \newline
38. यम् त्वा᳚ त्वा॒ यं ॅयम् त्वा॒ सोमे॑न॒ सोमे॑न त्वा॒ यं ॅयम् त्वा॒ सोमे॑न । \newline
39. त्वा॒ सोमे॑न॒ सोमे॑न त्वा त्वा॒ सोमे॒ना मी॑मद॒ ममी॑मदꣳ॒॒ सोमे॑न त्वा त्वा॒ सोमे॒ना मी॑मदम् । \newline
40. सोमे॒ना मी॑मद॒ ममी॑मदꣳ॒॒ सोमे॑न॒ सोमे॒ना मी॑मदꣳ सुप्र॒जाः सु॑प्र॒जा अमी॑मदꣳ॒॒ सोमे॑न॒ सोमे॒ना मी॑मदꣳ सुप्र॒जाः । \newline
41. अमी॑मदꣳ सुप्र॒जाः सु॑प्र॒जा अमी॑मद॒ ममी॑मदꣳ सुप्र॒जाः प्र॒जया᳚ प्र॒जया॑ सुप्र॒जा अमी॑मद॒ ममी॑मदꣳ सुप्र॒जाः प्र॒जया᳚ । \newline
42. सु॒प्र॒जाः प्र॒जया᳚ प्र॒जया॑ सुप्र॒जाः सु॑प्र॒जाः प्र॒जया॑ भूयासम् भूयासम् प्र॒जया॑ सुप्र॒जाः सु॑प्र॒जाः प्र॒जया॑ भूयासम् । \newline
43. सु॒प्र॒जा इति॑ सु - प्र॒जाः । \newline
44. प्र॒जया॑ भूयासम् भूयासम् प्र॒जया᳚ प्र॒जया॑ भूयासꣳ सु॒वीरः॑ सु॒वीरो॑ भूयासम् प्र॒जया᳚ प्र॒जया॑ भूयासꣳ सु॒वीरः॑ । \newline
45. प्र॒जयेति॑ प्र - जया᳚ । \newline
46. भू॒या॒सꣳ॒॒ सु॒वीरः॑ सु॒वीरो॑ भूयासम् भूयासꣳ सु॒वीरो॑ वी॒रैर् वी॒रैः सु॒वीरो॑ भूयासम् भूयासꣳ सु॒वीरो॑ वी॒रैः । \newline
47. सु॒वीरो॑ वी॒रैर् वी॒रैः सु॒वीरः॑ सु॒वीरो॑ वी॒रैः सु॒वर्चाः᳚ सु॒वर्चा॑ वी॒रैः सु॒वीरः॑ सु॒वीरो॑ वी॒रैः सु॒वर्चाः᳚ । \newline
48. सु॒वीर॒ इति॑ सु - वीरः॑ । \newline
49. वी॒रैः सु॒वर्चाः᳚ सु॒वर्चा॑ वी॒रैर् वी॒रैः सु॒वर्चा॒ वर्च॑सा॒ वर्च॑सा सु॒वर्चा॑ वी॒रैर् वी॒रैः सु॒वर्चा॒ वर्च॑सा । \newline
50. सु॒वर्चा॒ वर्च॑सा॒ वर्च॑सा सु॒वर्चाः᳚ सु॒वर्चा॒ वर्च॑सा सु॒पोषः॑ सु॒पोषो॒ वर्च॑सा सु॒वर्चाः᳚ सु॒वर्चा॒ वर्च॑सा सु॒पोषः॑ । \newline
51. सु॒वर्चा॒ इति॑ सु - वर्चाः᳚ । \newline
52. वर्च॑सा सु॒पोषः॑ सु॒पोषो॒ वर्च॑सा॒ वर्च॑सा सु॒पोषः॒ पोषैः॒ पोषैः᳚ सु॒पोषो॒ वर्च॑सा॒ वर्च॑सा सु॒पोषः॒ पोषैः᳚ । \newline
53. सु॒पोषः॒ पोषैः॒ पोषैः᳚ सु॒पोषः॑ सु॒पोषः॒ पोषै॒र् विश्वे᳚भ्यो॒ विश्वे᳚भ्यः॒ पोषैः᳚ सु॒पोषः॑ सु॒पोषः॒ पोषै॒र् विश्वे᳚भ्यः । \newline
54. सु॒पोष॒ इति॑ सु - पोषः॑ । \newline
55. पोषै॒र् विश्वे᳚भ्यो॒ विश्वे᳚भ्यः॒ पोषैः॒ पोषै॒र् विश्वे᳚भ्यो मे मे॒ विश्वे᳚भ्यः॒ पोषैः॒ पोषै॒र् विश्वे᳚भ्यो मे । \newline
56. विश्वे᳚भ्यो मे मे॒ विश्वे᳚भ्यो॒ विश्वे᳚भ्यो मे रू॒पेभ्यो॑ रू॒पेभ्यो॑ मे॒ विश्वे᳚भ्यो॒ विश्वे᳚भ्यो मे रू॒पेभ्यः॑ । \newline
57. मे॒ रू॒पेभ्यो॑ रू॒पेभ्यो॑ मे मे रू॒पेभ्यो॑ वर्चो॒दा व॑र्चो॒दा रू॒पेभ्यो॑ मे मे रू॒पेभ्यो॑ वर्चो॒दाः । \newline
58. रू॒पेभ्यो॑ वर्चो॒दा व॑र्चो॒दा रू॒पेभ्यो॑ रू॒पेभ्यो॑ वर्चो॒दा वर्च॑से॒ वर्च॑से वर्चो॒दा रू॒पेभ्यो॑ रू॒पेभ्यो॑ वर्चो॒दा वर्च॑से । \newline
59. व॒र्चो॒दा वर्च॑से॒ वर्च॑से वर्चो॒दा व॑र्चो॒दा वर्च॑से पवस्व पवस्व॒ वर्च॑से वर्चो॒दा व॑र्चो॒दा वर्च॑से पवस्व । \newline
60. व॒र्चो॒दा इति॑ वर्चः - दाः । \newline
\pagebreak
\markright{ TS 3.2.3.3  \hfill https://www.vedavms.in \hfill}

\section{ TS 3.2.3.3 }

\textbf{TS 3.2.3.3 } \newline
\textbf{Samhita Paata} \newline

वर्च॑से पवस्व॒ तस्य॑ मे रास्व॒ तस्य॑ ते भक्षीय॒ तस्य॑ त इ॒दमुन्मृ॑जे ॥ बुभू॑ष॒न्नवे᳚क्षेतै॒ष वै पात्रि॑यः प्र॒जाप॑तिर्य॒ज्ञ्ः प्र॒जाप॑ति॒स्तमे॒व त॑र्पयति॒ स ए॑नं तृ॒प्तो भूत्या॒ऽभि प॑वते ब्रह्मवर्च॒सका॒मो-ऽवे᳚क्षेतै॒ष वै पात्रि॑यः प्र॒जाप॑तिर्य॒ज्ञ्ः प्र॒जाप॑ति॒स्तमे॒व त॑र्पयति॒ स ए॑नं तृ॒प्तो ब्र॑ह्मवर्च॒सेना॒भि प॑वत आमया॒व्य - [  ] \newline

\textbf{Pada Paata} \newline

वर्च॑से । प॒व॒स्व॒ । तस्य॑ । मे॒ । रा॒स्व॒ । तस्य॑ । ते॒ । भ॒क्षी॒य॒ । तस्य॑ । ते॒ । इ॒दम् । उदिति॑ । मृ॒जे॒ ॥ बुभू॑षन्न् । अवेति॑ । ई॒क्षे॒त॒ । ए॒षः । वै । पात्रि॑यः । प्र॒जाप॑ति॒रिति॑ प्र॒जा - प॒तिः॒ । य॒ज्ञ्ः । प्र॒जाप॑ति॒रिति॑ प्र॒जा - प॒तिः॒ । तम् । ए॒व । त॒र्प॒य॒ति॒ । सः । ए॒न॒म् । तृ॒प्तः । भूत्या᳚ । अ॒भीति॑ । प॒व॒ते॒ । ब्र॒ह्म॒व॒र्च॒सका॑म॒ इति॑ ब्रह्मवर्च॒स-का॒मः॒ । अवेति॑ । ई॒क्षे॒त॒ । ए॒षः । वै । पात्रि॑यः । प्र॒जाप॑ति॒रिति॑ प्र॒जा - प॒तिः॒ । य॒ज्ञ्ः । प्र॒जाप॑ति॒रिति॑ प्र॒जा - प॒तिः॒ । तम् । ए॒व । त॒र्प॒य॒ति॒ । सः । ए॒न॒म् । तृ॒प्तः । ब्र॒ह्म॒व॒र्च॒सेनेति॑ ब्रह्म - व॒र्च॒सेन॑ । अ॒भीति॑ । प॒व॒ते॒ । आ॒म॒या॒वी ।  \newline


\textbf{Krama Paata} \newline

वर्च॑से पवस्व । प॒व॒स्व॒ तस्य॑ । तस्य॑ मे । मे॒ रा॒स्व॒ । रा॒स्व॒ तस्य॑ । तस्य॑ ते । ते॒ भ॒क्षी॒य॒ । भ॒क्षी॒य॒ तस्य॑ । तस्य॑ ते । त॒ इ॒दम् । इ॒दमुत् । उन् मृ॑जे । मृ॒ज॒ इति॑ मृजे ॥ बुभू॑ष॒न्नव॑ । अवे᳚क्षेत । ई॒क्षे॒तै॒षः । ए॒ष वै । वै पात्रि॑यः । पात्रि॑यः प्र॒जाप॑तिः । प्र॒जाप॑तिर् य॒ज्ञ्ः । प्र॒जाप॑ति॒रिति॑ प्र॒जा - प॒तिः॒ । य॒ज्ञ्ः प्र॒जाप॑तिः । प्र॒जाप॑ति॒स्तम् । प्र॒जाप॑ति॒रिति॑ प्र॒जा - प॒तिः॒ । तमे॒व । ए॒व त॑र्पयति । त॒र्प॒य॒ति॒ सः । स ए॑नम् । ए॒न॒म् तृ॒प्तः । तृ॒प्तो भूत्या᳚ । भूत्या॒ ऽभि । अ॒भि प॑वते । प॒व॒ते॒ ब्र॒ह्म॒व॒र्च॒सका॑मः । ब्र॒ह्म॒व॒र्च॒सका॒मो ऽव॑ । ब्र॒ह्म॒व॒र्च॒सका॑म॒ इति॑ ब्रह्मवर्च॒स - का॒मः॒ । अवे᳚क्षेत । ई॒क्षे॒तै॒षः । ए॒ष वै । वै पात्रि॑यः । पात्रि॑यः प्र॒जाप॑तिः । प्र॒जाप॑तिर् य॒ज्ञ्ः । प्र॒जाप॑ति॒रिति॑ प्र॒जा - प॒तिः॒ । य॒ज्ञ्ः प्र॒जाप॑तिः । प्र॒जाप॑ति॒स्तम् । प्र॒जाप॑ति॒रिति॑ प्र॒जा - प॒तिः॒ । तमे॒व । ए॒व त॑र्पयति । त॒र्प॒य॒ति॒ सः । स ए॑नम् । ए॒न॒म् तृ॒प्तः । तृ॒प्तो ब्र॑ह्मवर्च॒सेन॑ । ब्र॒ह्म॒व॒र्च॒सेना॒भि । ब्र॒ह्म॒व॒र्च॒सेनेति॑ ब्रह्म - व॒र्च॒सेन॑ । अ॒भि प॑वते । प॒व॒त॒ आ॒म॒या॒वी ( ) । आ॒म॒या॒व्यव॑ \newline

\textbf{Jatai Paata} \newline

1. वर्च॑से पवस्व पवस्व॒ वर्च॑से॒ वर्च॑से पवस्व । \newline
2. प॒व॒स्व॒ तस्य॒ तस्य॑ पवस्व पवस्व॒ तस्य॑ । \newline
3. तस्य॑ मे मे॒ तस्य॒ तस्य॑ मे । \newline
4. मे॒ रा॒स्व॒ रा॒स्व॒ मे॒ मे॒ रा॒स्व॒ । \newline
5. रा॒स्व॒ तस्य॒ तस्य॑ रास्व रास्व॒ तस्य॑ । \newline
6. तस्य॑ ते ते॒ तस्य॒ तस्य॑ ते । \newline
7. ते॒ भ॒क्षी॒य॒ भ॒क्षी॒य॒ ते॒ ते॒ भ॒क्षी॒य॒ । \newline
8. भ॒क्षी॒य॒ तस्य॒ तस्य॑ भक्षीय भक्षीय॒ तस्य॑ । \newline
9. तस्य॑ ते ते॒ तस्य॒ तस्य॑ ते । \newline
10. त॒ इ॒द मि॒दम् ते॑ त इ॒दम् । \newline
11. इ॒द मुदु दि॒द मि॒द मुत् । \newline
12. उन् मृ॑जे मृज॒ उदुन् मृ॑जे । \newline
13. मृ॒ज॒ इति॑ मृजे । \newline
14. बुभू॑ष॒न् नवाव॒ बुभू॑ष॒न् बुभू॑ष॒न् नव॑ । \newline
15. अवे᳚क्षे तेक्षे॒ता वावे᳚क्षेत । \newline
16. ई॒क्षे॒तै॒ष ए॒ष ई᳚क्षे तेक्षे तै॒षः । \newline
17. ए॒ष वै वा ए॒ष ए॒ष वै । \newline
18. वै पात्रि॑यः॒ पात्रि॑यो॒ वै वै पात्रि॑यः । \newline
19. पात्रि॑यः प्र॒जाप॑तिः प्र॒जाप॑तिः॒ पात्रि॑यः॒ पात्रि॑यः प्र॒जाप॑तिः । \newline
20. प्र॒जाप॑तिर् य॒ज्ञो य॒ज्ञ्ः प्र॒जाप॑तिः प्र॒जाप॑तिर् य॒ज्ञ्ः । \newline
21. प्र॒जाप॑ति॒रिति॑ प्र॒जा - प॒तिः॒ । \newline
22. य॒ज्ञ्ः प्र॒जाप॑तिः प्र॒जाप॑तिर् य॒ज्ञो य॒ज्ञ्ः प्र॒जाप॑तिः । \newline
23. प्र॒जाप॑ति॒ स्तम् तम् प्र॒जाप॑तिः प्र॒जाप॑ति॒ स्तम् । \newline
24. प्र॒जाप॑ति॒रिति॑ प्र॒जा - प॒तिः॒ । \newline
25. त मे॒वैव तम् त मे॒व । \newline
26. ए॒व त॑र्पयति तर्पय त्ये॒वैव त॑र्पयति । \newline
27. त॒र्प॒य॒ति॒ स स त॑र्पयति तर्पयति॒ सः । \newline
28. स ए॑न मेनꣳ॒॒ स स ए॑नम् । \newline
29. ए॒न॒म् तृ॒प्त स्तृ॒प्त ए॑न मेनम् तृ॒प्तः । \newline
30. तृ॒प्तो भूत्या॒ भूत्या॑ तृ॒प्त स्तृ॒प्तो भूत्या᳚ । \newline
31. भूत्या॒ ऽभ्य॑भि भूत्या॒ भूत्या॒ ऽभि । \newline
32. अ॒भि प॑वते पवते॒ ऽभ्य॑भि प॑वते । \newline
33. प॒व॒ते॒ ब्र॒ह्म॒व॒र्च॒सका॑मो ब्रह्मवर्च॒सका॑मः पवते पवते ब्रह्मवर्च॒सका॑मः । \newline
34. ब्र॒ह्म॒व॒र्च॒सका॒मो ऽवाव॑ ब्रह्मवर्च॒सका॑मो ब्रह्मवर्च॒सका॒मो ऽव॑ । \newline
35. ब्र॒ह्म॒व॒र्च॒सका॑म॒ इति॑ ब्रह्मवर्च॒स - का॒मः॒ । \newline
36. अवे᳚क्षे तेक्षे॒ता वावे᳚क्षेत । \newline
37. ई॒क्षे॒तै॒ष ए॒ष ई᳚क्षे तेक्षे तै॒षः । \newline
38. ए॒ष वै वा ए॒ष ए॒ष वै । \newline
39. वै पात्रि॑यः॒ पात्रि॑यो॒ वै वै पात्रि॑यः । \newline
40. पात्रि॑यः प्र॒जाप॑तिः प्र॒जाप॑तिः॒ पात्रि॑यः॒ पात्रि॑यः प्र॒जाप॑तिः । \newline
41. प्र॒जाप॑तिर् य॒ज्ञो य॒ज्ञ्ः प्र॒जाप॑तिः प्र॒जाप॑तिर् य॒ज्ञ्ः । \newline
42. प्र॒जाप॑ति॒रिति॑ प्र॒जा - प॒तिः॒ । \newline
43. य॒ज्ञ्ः प्र॒जाप॑तिः प्र॒जाप॑तिर् य॒ज्ञो य॒ज्ञ्ः प्र॒जाप॑तिः । \newline
44. प्र॒जाप॑ति॒स्तम् तम् प्र॒जाप॑तिः प्र॒जाप॑ति॒स्तम् । \newline
45. प्र॒जाप॑ति॒रिति॑ प्र॒जा - प॒तिः॒ । \newline
46. त मे॒वैव तम् त मे॒व । \newline
47. ए॒व त॑र्पयति तर्पय त्ये॒वैव त॑र्पयति । \newline
48. त॒र्प॒य॒ति॒ स स त॑र्पयति तर्पयति॒ सः । \newline
49. स ए॑न मेनꣳ॒॒ स स ए॑नम् । \newline
50. ए॒न॒म् तृ॒प्त स्तृ॒प्त ए॑न मेनम् तृ॒प्तः । \newline
51. तृ॒प्तो ब्र॑ह्मवर्च॒सेन॑ ब्रह्मवर्च॒सेन॑ तृ॒प्त स्तृ॒प्तो ब्र॑ह्मवर्च॒सेन॑ । \newline
52. ब्र॒ह्म॒व॒र्च॒सेना॒ भ्य॑भि ब्र॑ह्मवर्च॒सेन॑ ब्रह्मवर्च॒सेना॒भि । \newline
53. ब्र॒ह्म॒व॒र्च॒सेनेति॑ ब्रह्म - व॒र्च॒सेन॑ । \newline
54. अ॒भि प॑वते पवते॒ ऽभ्य॑भि प॑वते । \newline
55. प॒व॒त॒ आ॒म॒या॒ व्या॑मया॒वी प॑वते पवत आमया॒वी । \newline
56. आ॒म॒या॒ व्यवावा॑ मया॒ व्या॑मया॒ व्यव॑ । \newline

\textbf{Ghana Paata } \newline

1. वर्च॑से पवस्व पवस्व॒ वर्च॑से॒ वर्च॑से पवस्व॒ तस्य॒ तस्य॑ पवस्व॒ वर्च॑से॒ वर्च॑से पवस्व॒ तस्य॑ । \newline
2. प॒व॒स्व॒ तस्य॒ तस्य॑ पवस्व पवस्व॒ तस्य॑ मे मे॒ तस्य॑ पवस्व पवस्व॒ तस्य॑ मे । \newline
3. तस्य॑ मे मे॒ तस्य॒ तस्य॑ मे रास्व रास्व मे॒ तस्य॒ तस्य॑ मे रास्व । \newline
4. मे॒ रा॒स्व॒ रा॒स्व॒ मे॒ मे॒ रा॒स्व॒ तस्य॒ तस्य॑ रास्व मे मे रास्व॒ तस्य॑ । \newline
5. रा॒स्व॒ तस्य॒ तस्य॑ रास्व रास्व॒ तस्य॑ ते ते॒ तस्य॑ रास्व रास्व॒ तस्य॑ ते । \newline
6. तस्य॑ ते ते॒ तस्य॒ तस्य॑ ते भक्षीय भक्षीय ते॒ तस्य॒ तस्य॑ ते भक्षीय । \newline
7. ते॒ भ॒क्षी॒य॒ भ॒क्षी॒य॒ ते॒ ते॒ भ॒क्षी॒य॒ तस्य॒ तस्य॑ भक्षीय ते ते भक्षीय॒ तस्य॑ । \newline
8. भ॒क्षी॒य॒ तस्य॒ तस्य॑ भक्षीय भक्षीय॒ तस्य॑ ते ते॒ तस्य॑ भक्षीय भक्षीय॒ तस्य॑ ते । \newline
9. तस्य॑ ते ते॒ तस्य॒ तस्य॑ त इ॒द मि॒दम् ते॒ तस्य॒ तस्य॑ त इ॒दम् । \newline
10. त॒ इ॒द मि॒दम् ते॑ त इ॒द मुदु दि॒दम् ते॑ त इ॒द मुत् । \newline
11. इ॒द मुदु दि॒द मि॒द मुन् मृ॑जे मृज॒ उदि॒द मि॒द मुन् मृ॑जे । \newline
12. उन् मृ॑जे मृज॒ उदुन् मृ॑जे । \newline
13. मृ॒ज॒ इति॑ मृजे । \newline
14. बुभू॑ष॒न् नवाव॒ बुभू॑ष॒न् बुभू॑ष॒न् नवे᳚क्षे तेक्षे॒ ताव॒ बुभू॑ष॒न् बुभू॑ष॒न् नवे᳚क्षेत । \newline
15. अवे᳚क्षे तेक्षे॒ता वावे᳚क्षे तै॒ष ए॒ष ई᳚क्षे॒ तावा वे᳚क्षे तै॒षः । \newline
16. ई॒क्षे॒ तै॒ष ए॒ष ई᳚क्षे तेक्षे तै॒ष वै वा ए॒ष ई᳚क्षे तेक्षे तै॒ष वै । \newline
17. ए॒ष वै वा ए॒ष ए॒ष वै पात्रि॑यः॒ पात्रि॑यो॒ वा ए॒ष ए॒ष वै पात्रि॑यः । \newline
18. वै पात्रि॑यः॒ पात्रि॑यो॒ वै वै पात्रि॑यः प्र॒जाप॑तिः प्र॒जाप॑तिः॒ पात्रि॑यो॒ वै वै पात्रि॑यः प्र॒जाप॑तिः । \newline
19. पात्रि॑यः प्र॒जाप॑तिः प्र॒जाप॑तिः॒ पात्रि॑यः॒ पात्रि॑यः प्र॒जाप॑तिर् य॒ज्ञो य॒ज्ञ्ः प्र॒जाप॑तिः॒ पात्रि॑यः॒ पात्रि॑यः प्र॒जाप॑तिर् य॒ज्ञ्ः । \newline
20. प्र॒जाप॑तिर् य॒ज्ञो य॒ज्ञ्ः प्र॒जाप॑तिः प्र॒जाप॑तिर् य॒ज्ञ्ः प्र॒जाप॑तिः प्र॒जाप॑तिर् य॒ज्ञ्ः प्र॒जाप॑तिः प्र॒जाप॑तिर् य॒ज्ञ्ः प्र॒जाप॑तिः । \newline
21. प्र॒जाप॑ति॒रिति॑ प्र॒जा - प॒तिः॒ । \newline
22. य॒ज्ञ्ः प्र॒जाप॑तिः प्र॒जाप॑तिर् य॒ज्ञो य॒ज्ञ्ः प्र॒जाप॑ति॒ स्तम् तम् प्र॒जाप॑तिर् य॒ज्ञो य॒ज्ञ्ः प्र॒जाप॑ति॒ स्तम् । \newline
23. प्र॒जाप॑ति॒ स्तम् तम् प्र॒जाप॑तिः प्र॒जाप॑ति॒ स्त मे॒वैव तम् प्र॒जाप॑तिः प्र॒जाप॑ति॒ स्त मे॒व । \newline
24. प्र॒जाप॑ति॒रिति॑ प्र॒जा - प॒तिः॒ । \newline
25. त मे॒वैव तम् त मे॒व त॑र्पयति तर्पय त्ये॒व तम् त मे॒व त॑र्पयति । \newline
26. ए॒व त॑र्पयति तर्पय त्ये॒वैव त॑र्पयति॒ स स त॑र्पय त्ये॒वैव त॑र्पयति॒ सः । \newline
27. त॒र्प॒य॒ति॒ स स त॑र्पयति तर्पयति॒ स ए॑न मेनꣳ॒॒ स त॑र्पयति तर्पयति॒ स ए॑नम् । \newline
28. स ए॑न मेनꣳ॒॒ स स ए॑नम् तृ॒प्त स्तृ॒प्त ए॑नꣳ॒॒ स स ए॑नम् तृ॒प्तः । \newline
29. ए॒न॒म् तृ॒प्त स्तृ॒प्त ए॑न मेनम् तृ॒प्तो भूत्या॒ भूत्या॑ तृ॒प्त ए॑न मेनम् तृ॒प्तो भूत्या᳚ । \newline
30. तृ॒प्तो भूत्या॒ भूत्या॑ तृ॒प्त स्तृ॒प्तो भूत्या॒ ऽभ्य॑भि भूत्या॑ तृ॒प्त स्तृ॒प्तो भूत्या॒ ऽभि । \newline
31. भूत्या॒ ऽभ्य॑भि भूत्या॒ भूत्या॒ ऽभि प॑वते पवते॒ ऽभि भूत्या॒ भूत्या॒ ऽभि प॑वते । \newline
32. अ॒भि प॑वते पवते॒ ऽभ्य॑भि प॑वते ब्रह्मवर्च॒सका॑मो ब्रह्मवर्च॒सका॑मः पवते॒ ऽभ्य॑भि प॑वते ब्रह्मवर्च॒सका॑मः । \newline
33. प॒व॒ते॒ ब्र॒ह्म॒व॒र्च॒सका॑मो ब्रह्मवर्च॒सका॑मः पवते पवते ब्रह्मवर्च॒सका॒मो ऽवाव॑ ब्रह्मवर्च॒सका॑मः पवते पवते ब्रह्मवर्च॒सका॒मो ऽव॑ । \newline
34. ब्र॒ह्म॒व॒र्च॒सका॒मो ऽवाव॑ ब्रह्मवर्च॒सका॑मो ब्रह्मवर्च॒सका॒मो ऽवे᳚क्षे तेक्षे॒ ताव॑ ब्रह्मवर्च॒सका॑मो ब्रह्मवर्च॒सका॒मो ऽवे᳚क्षेत । \newline
35. ब्र॒ह्म॒व॒र्च॒सका॑म॒ इति॑ ब्रह्मवर्च॒स - का॒मः॒ । \newline
36. अवे᳚क्षे तेक्षे॒ता वावे᳚क्षे तै॒ष ए॒ष ई᳚क्षे॒ तावा वे᳚क्षेतै॒षः । \newline
37. ई॒क्षे॒ तै॒ष ए॒ष ई᳚क्षे तेक्षे तै॒ष वै वा ए॒ष ई᳚क्षे तेक्षे तै॒ष वै । \newline
38. ए॒ष वै वा ए॒ष ए॒ष वै पात्रि॑यः॒ पात्रि॑यो॒ वा ए॒ष ए॒ष वै पात्रि॑यः । \newline
39. वै पात्रि॑यः॒ पात्रि॑यो॒ वै वै पात्रि॑यः प्र॒जाप॑तिः प्र॒जाप॑तिः॒ पात्रि॑यो॒ वै वै पात्रि॑यः प्र॒जाप॑तिः । \newline
40. पात्रि॑यः प्र॒जाप॑तिः प्र॒जाप॑तिः॒ पात्रि॑यः॒ पात्रि॑यः प्र॒जाप॑तिर् य॒ज्ञो य॒ज्ञ्ः प्र॒जाप॑तिः॒ पात्रि॑यः॒ पात्रि॑यः प्र॒जाप॑तिर् य॒ज्ञ्ः । \newline
41. प्र॒जाप॑तिर् य॒ज्ञो य॒ज्ञ्ः प्र॒जाप॑तिः प्र॒जाप॑तिर् य॒ज्ञ्ः प्र॒जाप॑तिः प्र॒जाप॑तिर् य॒ज्ञ्ः प्र॒जाप॑तिः प्र॒जाप॑तिर् य॒ज्ञ्ः प्र॒जाप॑तिः । \newline
42. प्र॒जाप॑ति॒रिति॑ प्र॒जा - प॒तिः॒ । \newline
43. य॒ज्ञ्ः प्र॒जाप॑तिः प्र॒जाप॑तिर् य॒ज्ञो य॒ज्ञ्ः प्र॒जाप॑ति॒ स्तम् तम् प्र॒जाप॑तिर् य॒ज्ञो य॒ज्ञ्ः प्र॒जाप॑ति॒ स्तम् । \newline
44. प्र॒जाप॑ति॒ स्तम् तम् प्र॒जाप॑तिः प्र॒जाप॑ति॒ स्त मे॒वैव तम् प्र॒जाप॑तिः प्र॒जाप॑ति॒ स्त मे॒व । \newline
45. प्र॒जाप॑ति॒रिति॑ प्र॒जा - प॒तिः॒ । \newline
46. त मे॒वैव तम् त मे॒व त॑र्पयति तर्पय त्ये॒व तम् त मे॒व त॑र्पयति । \newline
47. ए॒व त॑र्पयति तर्पय त्ये॒वैव त॑र्पयति॒ स स त॑र्पय त्ये॒वैव त॑र्पयति॒ सः । \newline
48. त॒र्प॒य॒ति॒ स स त॑र्पयति तर्पयति॒ स ए॑न मेनꣳ॒॒ स त॑र्पयति तर्पयति॒ स ए॑नम् । \newline
49. स ए॑न मेनꣳ॒॒ स स ए॑नम् तृ॒प्त स्तृ॒प्त ए॑नꣳ॒॒ स स ए॑नम् तृ॒प्तः । \newline
50. ए॒न॒म् तृ॒प्त स्तृ॒प्त ए॑न मेनम् तृ॒प्तो ब्र॑ह्मवर्च॒सेन॑ ब्रह्मवर्च॒सेन॑ तृ॒प्त ए॑न मेनम् तृ॒प्तो ब्र॑ह्मवर्च॒सेन॑ । \newline
51. तृ॒प्तो ब्र॑ह्मवर्च॒सेन॑ ब्रह्मवर्च॒सेन॑ तृ॒प्त स्तृ॒प्तो ब्र॑ह्मवर्च॒सेना॒ भ्य॑भि ब्र॑ह्मवर्च॒सेन॑ तृ॒प्त स्तृ॒प्तो ब्र॑ह्मवर्च॒सेना॒भि । \newline
52. ब्र॒ह्म॒व॒र्च॒सेना॒ भ्य॑भि ब्र॑ह्मवर्च॒सेन॑ ब्रह्मवर्च॒सेना॒भि प॑वते पवते॒ ऽभि ब्र॑ह्मवर्च॒सेन॑ ब्रह्मवर्च॒सेना॒भि प॑वते । \newline
53. ब्र॒ह्म॒व॒र्च॒सेनेति॑ ब्रह्म - व॒र्च॒सेन॑ । \newline
54. अ॒भि प॑वते पवते॒ ऽभ्य॑भि प॑वत आमया॒ व्या॑मया॒वी प॑वते॒ ऽभ्य॑भि प॑वत आमया॒वी । \newline
55. प॒व॒त॒ आ॒म॒या॒ व्या॑मया॒वी प॑वते पवत आमया॒ व्यवावा॑ मया॒वी प॑वते पवत आमया॒ व्यव॑ । \newline
56. आ॒म॒या॒ व्यवावा॑ मया॒ व्या॑मया॒ व्यवे᳚क्षे तेक्षे॒ता वा॑मया॒ व्या॑मया॒ व्यवे᳚क्षेत । \newline
\pagebreak
\markright{ TS 3.2.3.4  \hfill https://www.vedavms.in \hfill}

\section{ TS 3.2.3.4 }

\textbf{TS 3.2.3.4 } \newline
\textbf{Samhita Paata} \newline

वे᳚क्षेतै॒ष वै पात्रि॑यः प्र॒जाप॑तिर्य॒ज्ञ्ः प्र॒जाप॑ति॒स्तमे॒व त॑र्पयति॒ स ए॑नं तृ॒प्त आयु॑षा॒ऽभि प॑वतेऽभि॒चर॒न्नवे᳚क्षेतै॒ष वै पात्रि॑यः प्र॒जाप॑तिर्य॒ज्ञ्ः प्र॒जाप॑ति॒स्तमे॒व त॑र्पयति॒ स ए॑नं तृ॒प्तः प्रा॑णापा॒नाभ्यां᳚ ॅवा॒चो द॑क्षक्र॒तुभ्यां॒ चक्षु॑र्भ्याꣳ॒॒ श्रोत्रा᳚भ्या-मा॒त्मनोऽङ्गे᳚भ्य॒ आयु॑षो॒ऽन्तरे॑ति ता॒जक् प्र ध॑न्वति ॥ \newline

\textbf{Pada Paata} \newline

अवेति॑ । ई॒क्षे॒त॒ । ए॒षः । वै । पात्रि॑यः । प्र॒जाप॑ति॒रिति॑ प्र॒जा-प॒तिः॒ । य॒ज्ञ्ः । प्र॒जाप॑ति॒रिति॑ प्र॒जा - प॒तिः॒ । तम् । ए॒व । त॒र्प॒य॒ति॒ । सः । ए॒न॒म् । तृ॒प्तः । आयु॑षा । अ॒भीति॑ । प॒व॒ते॒ । अ॒भि॒चर॒न्नित्य॑भि - चरन्न्॑ । अवेति॑ । ई॒क्षे॒त॒ । ए॒षः । वै । पात्रि॑यः । प्र॒जाप॑ति॒रिति॑ प्र॒जा - प॒तिः॒ । य॒ज्ञ्ः । प्र॒जाप॑ति॒रिति॑ प्र॒जा - प॒तिः॒ । तम् । ए॒व । त॒र्प॒य॒ति॒ । सः । ए॒न॒म् । तृ॒प्तः । प्रा॒णा॒पा॒नाभ्या॒मिति॑ प्राण - अ॒पा॒नाभ्या᳚म् । वा॒चः । द॒क्ष॒क्र॒तुभ्या॒मिति॑ दक्षक्र॒तु - भ्या॒म् । चक्षु॑र्भ्या॒मिति॒ चक्षुः॑ - भ्या॒म् । श्रोत्रा᳚भ्याम् । आ॒त्मनः॑ । अङ्गे᳚भ्यः । आयु॑षः । अ॒न्तः । ए॒ति॒ । ता॒जक् । प्रेति॑ । ध॒न्व॒ति॒ ॥  \newline


\textbf{Krama Paata} \newline

अवे᳚क्षेत । ई॒क्षे॒तै॒षः । ए॒ष वै । वै पात्रि॑यः । पात्रि॑यः प्र॒जाप॑तिः । प्र॒जाप॑तिर् य॒ज्ञ्ः । प्र॒जाप॑ति॒रिति॑ प्र॒जा - प॒तिः॒ । य॒ज्ञ्ः प्र॒जाप॑तिः । प्र॒जाप॑ति॒स्तम् । प्र॒जाप॑ति॒रिति॑ प्र॒जा - प॒तिः॒ । तमे॒व । एव त॑र्पयति । त॒र्प॒य॒ति॒ सः । स ए॑नम् । ए॒न॒म् तृ॒प्तः । तृ॒प्त आयु॑षा । आयु॑षा॒ ऽभि । अ॒भि प॑वते । प॒व॒ते॒ ऽभि॒चरन्न्॑ । अ॒भि॒चर॒न्नव॑ । अ॒भि॒चर॒न्नित्य॑भि - चरन्न्॑ । अवे᳚क्षेत । ई॒क्षे॒तै॒षः । ए॒ष वै । वै पात्रि॑यः । पात्रि॑यः प्र॒जाप॑तिः । प्र॒जाप॑तिर् य॒ज्ञ्ः । प्र॒जाप॑ति॒रिति॑ प्र॒जा - प॒तिः॒ । य॒ज्ञ्ः प्र॒जाप॑तिः । प्र॒जाप॑ति॒स्तम् । प्र॒जाप॑ति॒रिति॑ प्र॒जा - प॒तिः॒ । तमे॒व । ए॒व त॑र्पयति । त॒र्प॒य॒ति॒ सः । स ए॑नम् । ए॒न॒म् तृ॒प्तः । तृ॒प्तः प्रा॑णापा॒नाभ्या᳚म् । प्रा॒णा॒पा॒नाभ्यां᳚ ॅवा॒चः । प्रा॒णा॒पा॒नाभ्या॒मिति॑ प्राण - अ॒पा॒नाभ्या᳚म् । वा॒चो द॑क्षक्र॒तुभ्या᳚म् । द॒क्ष॒क्र॒तुभ्या॒म् चक्षु॑र्भ्याम् । द॒क्ष॒क्र॒तुभ्या॒मिति॑ दक्षक्र॒तु - भ्या॒म् । चक्षु॑र्भ्याꣳ॒॒ श्रोत्रा᳚भ्याम् । चक्षु॑र्भ्या॒मिति॒ चक्षुः॑ - भ्या॒म् । श्रोत्रा᳚भ्यामा॒त्मनः॑ । आ॒त्मनो ऽङ्गे᳚भ्यः । अङ्गे᳚भ्य॒ आयु॑षः । आयु॑षो॒ ऽन्तः । अ॒न्तरे॑ति । ए॒ति॒ ता॒जक् । ता॒जक् प्र । प्र ध॑न्वति । ध॒न्व॒तीति॑ धन्वति । \newline

\textbf{Jatai Paata} \newline

1. अवे᳚क्षे तेक्षे॒ता वावे᳚क्षेत । \newline
2. ई॒क्षे॒तै॒ष ए॒ष ई᳚क्षे तेक्षे तै॒षः । \newline
3. ए॒ष वै वा ए॒ष ए॒ष वै । \newline
4. वै पात्रि॑यः॒ पात्रि॑यो॒ वै वै पात्रि॑यः । \newline
5. पात्रि॑यः प्र॒जाप॑तिः प्र॒जाप॑तिः॒ पात्रि॑यः॒ पात्रि॑यः प्र॒जाप॑तिः । \newline
6. प्र॒जाप॑तिर् य॒ज्ञो य॒ज्ञ्ः प्र॒जाप॑तिः प्र॒जाप॑तिर् य॒ज्ञ्ः । \newline
7. प्र॒जाप॑ति॒रिति॑ प्र॒जा-प॒तिः॒ । \newline
8. य॒ज्ञ्ः प्र॒जाप॑तिः प्र॒जाप॑तिर् य॒ज्ञो य॒ज्ञ्ः प्र॒जाप॑तिः । \newline
9. प्र॒जाप॑ति॒ स्तम् तम् प्र॒जाप॑तिः प्र॒जाप॑ति॒ स्तम् । \newline
10. प्र॒जाप॑ति॒रिति॑ प्र॒जा - प॒तिः॒ । \newline
11. त मे॒वैव तम् त मे॒व । \newline
12. ए॒व त॑र्पयति तर्पय त्ये॒वैव त॑र्पयति । \newline
13. त॒र्प॒य॒ति॒ स स त॑र्पयति तर्पयति॒ सः । \newline
14. स ए॑न मेनꣳ॒॒ स स ए॑नम् । \newline
15. ए॒न॒म् तृ॒प्त स्तृ॒प्त ए॑न मेनम् तृ॒प्तः । \newline
16. तृ॒प्त आयु॒षा ऽऽयु॑षा तृ॒प्त स्तृ॒प्त आयु॑षा । \newline
17. आयु॑षा॒ ऽभ्य॑ भ्यायु॒षा ऽऽयु॑षा॒ ऽभि । \newline
18. अ॒भि प॑वते पवते॒ ऽभ्य॑भि प॑वते । \newline
19. प॒व॒ते॒ ऽभि॒चर॑न् नभि॒चर॑न् पवते पवते ऽभि॒चरन्न्॑ । \newline
20. अ॒भि॒चर॒न् नवावा॑ भि॒चर॑न् नभि॒चर॒न् नव॑ । \newline
21. अ॒भि॒चर॒न्नित्य॑भि - चरन्न्॑ । \newline
22. अवे᳚क्षे तेक्षे॒ता वावे᳚क्षेत । \newline
23. ई॒क्षे॒तै॒ष ए॒ष ई᳚क्षे तेक्षे तै॒षः । \newline
24. ए॒ष वै वा ए॒ष ए॒ष वै । \newline
25. वै पात्रि॑यः॒ पात्रि॑यो॒ वै वै पात्रि॑यः । \newline
26. पात्रि॑यः प्र॒जाप॑तिः प्र॒जाप॑तिः॒ पात्रि॑यः॒ पात्रि॑यः प्र॒जाप॑तिः । \newline
27. प्र॒जाप॑तिर् य॒ज्ञो य॒ज्ञ्ः प्र॒जाप॑तिः प्र॒जाप॑तिर् य॒ज्ञ्ः । \newline
28. प्र॒जाप॑ति॒रिति॑ प्र॒जा - प॒तिः॒ । \newline
29. य॒ज्ञ्ः प्र॒जाप॑तिः प्र॒जाप॑तिर् य॒ज्ञो य॒ज्ञ्ः प्र॒जाप॑तिः । \newline
30. प्र॒जाप॑ति॒ स्तम् तम् प्र॒जाप॑तिः प्र॒जाप॑ति॒ स्तम् । \newline
31. प्र॒जाप॑ति॒रिति॑ प्र॒जा - प॒तिः॒ । \newline
32. त मे॒वैव तम् त मे॒व । \newline
33. ए॒व त॑र्पयति तर्पय त्ये॒वैव त॑र्पयति । \newline
34. त॒र्प॒य॒ति॒ स स त॑र्पयति तर्पयति॒ सः । \newline
35. स ए॑न मेनꣳ॒॒ स स ए॑नम् । \newline
36. ए॒न॒म् तृ॒प्त स्तृ॒प्त ए॑न मेनम् तृ॒प्तः । \newline
37. तृ॒प्तः प्रा॑णापा॒नाभ्या᳚म् प्राणापा॒नाभ्या᳚म् तृ॒प्त स्तृ॒प्तः प्रा॑णापा॒नाभ्या᳚म् । \newline
38. प्रा॒णा॒पा॒नाभ्यां᳚ ॅवा॒चो वा॒चः प्रा॑णापा॒नाभ्या᳚म् प्राणापा॒नाभ्यां᳚ ॅवा॒चः । \newline
39. प्रा॒णा॒पा॒नाभ्या॒मिति॑ प्राण - अ॒पा॒नाभ्या᳚म् । \newline
40. वा॒चो द॑क्षक्र॒तुभ्या᳚म् दक्षक्र॒तुभ्यां᳚ ॅवा॒चो वा॒चो द॑क्षक्र॒तुभ्या᳚म् । \newline
41. द॒क्ष॒क्र॒तुभ्या॒म् चक्षु॑र्भ्या॒म् चक्षु॑र्भ्याम् दक्षक्र॒तुभ्या᳚म् दक्षक्र॒तुभ्या॒म् चक्षु॑र्भ्याम् । \newline
42. द॒क्ष॒क्र॒तुभ्या॒मिति॑ दक्षक्र॒तु - भ्या॒म् । \newline
43. चक्षु॑र्भ्याꣳ॒॒ श्रोत्रा᳚भ्याꣳ॒॒ श्रोत्रा᳚भ्या॒म् चक्षु॑र्भ्या॒म् चक्षु॑र्भ्याꣳ॒॒ श्रोत्रा᳚भ्याम् । \newline
44. चक्षु॑र्भ्या॒मिति॒ चक्षुः॑ - भ्या॒म् । \newline
45. श्रोत्रा᳚भ्या मा॒त्मन॑ आ॒त्मनः॒ श्रोत्रा᳚भ्याꣳ॒॒ श्रोत्रा᳚भ्या मा॒त्मनः॑ । \newline
46. आ॒त्मनो ऽङ्गे॒भ्यो ऽङ्गे᳚भ्य आ॒त्मन॑ आ॒त्मनो ऽङ्गे᳚भ्यः । \newline
47. अङ्गे᳚भ्य॒ आयु॑ष॒ आयु॒षो ऽङ्गे॒भ्यो ऽङ्गे᳚भ्य॒ आयु॑षः । \newline
48. आयु॑षो॒ ऽन्त र॒न्त रायु॑ष॒ आयु॑षो॒ ऽन्तः । \newline
49. अ॒न्त रे᳚त्ये त्य॒न्त र॒न्त रे॑ति । \newline
50. ए॒ति॒ ता॒जक् ता॒जगे᳚ त्येति ता॒जक् । \newline
51. ता॒जक् प्र प्र ता॒जक् ता॒जक् प्र । \newline
52. प्र ध॑न्वति धन्वति॒ प्र प्र ध॑न्वति । \newline
53. ध॒न्व॒तीति॑ धन्वति । \newline

\textbf{Ghana Paata } \newline

1. अवे᳚क्षे तेक्षे॒ता वावे᳚क्षे तै॒ष ए॒ष ई᳚क्षे॒ता वावे᳚क्षे तै॒षः । \newline
2. ई॒क्षे॒ तै॒ष ए॒ष ई᳚क्षे तेक्षे तै॒ष वै वा ए॒ष ई᳚क्षे तेक्षे तै॒ष वै । \newline
3. ए॒ष वै वा ए॒ष ए॒ष वै पात्रि॑यः॒ पात्रि॑यो॒ वा ए॒ष ए॒ष वै पात्रि॑यः । \newline
4. वै पात्रि॑यः॒ पात्रि॑यो॒ वै वै पात्रि॑यः प्र॒जाप॑तिः प्र॒जाप॑तिः॒ पात्रि॑यो॒ वै वै पात्रि॑यः प्र॒जाप॑तिः । \newline
5. पात्रि॑यः प्र॒जाप॑तिः प्र॒जाप॑तिः॒ पात्रि॑यः॒ पात्रि॑यः प्र॒जाप॑तिर् य॒ज्ञो य॒ज्ञ्ः प्र॒जाप॑तिः॒ पात्रि॑यः॒ पात्रि॑यः प्र॒जाप॑तिर् य॒ज्ञ्ः । \newline
6. प्र॒जाप॑तिर् य॒ज्ञो य॒ज्ञ्ः प्र॒जाप॑तिः प्र॒जाप॑तिर् य॒ज्ञ्ः प्र॒जाप॑तिः प्र॒जाप॑तिर् य॒ज्ञ्ः प्र॒जाप॑तिः प्र॒जाप॑तिर् य॒ज्ञ्ः प्र॒जाप॑तिः । \newline
7. प्र॒जाप॑ति॒रिति॑ प्र॒जा - प॒तिः॒ । \newline
8. य॒ज्ञ्ः प्र॒जाप॑तिः प्र॒जाप॑तिर् य॒ज्ञो य॒ज्ञ्ः प्र॒जाप॑ति॒ स्तम् तम् प्र॒जाप॑तिर् य॒ज्ञो य॒ज्ञ्ः प्र॒जाप॑ति॒ स्तम् । \newline
9. प्र॒जाप॑ति॒ स्तम् तम् प्र॒जाप॑तिः प्र॒जाप॑ति॒ स्त मे॒वैव तम् प्र॒जाप॑तिः प्र॒जाप॑ति॒ स्त मे॒व । \newline
10. प्र॒जाप॑ति॒रिति॑ प्र॒जा - प॒तिः॒ । \newline
11. त मे॒वैव तम् त मे॒व त॑र्पयति तर्पय त्ये॒व तम् त मे॒व त॑र्पयति । \newline
12. ए॒व त॑र्पयति तर्पय त्ये॒वैव त॑र्पयति॒ स स त॑र्पय त्ये॒वैव त॑र्पयति॒ सः । \newline
13. त॒र्प॒य॒ति॒ स स त॑र्पयति तर्पयति॒ स ए॑न मेनꣳ॒॒ स त॑र्पयति तर्पयति॒ स ए॑नम् । \newline
14. स ए॑न मेनꣳ॒॒ स स ए॑नम् तृ॒प्त स्तृ॒प्त ए॑नꣳ॒॒ स स ए॑नम् तृ॒प्तः । \newline
15. ए॒न॒म् तृ॒प्त स्तृ॒प्त ए॑न मेनम् तृ॒प्त आयु॒षा ऽऽयु॑षा तृ॒प्त ए॑न मेनम् तृ॒प्त आयु॑षा । \newline
16. तृ॒प्त आयु॒षा ऽऽयु॑षा तृ॒प्त स्तृ॒प्त आयु॑षा॒ ऽभ्य॑ भ्यायु॑षा तृ॒प्त स्तृ॒प्त आयु॑षा॒ ऽभि । \newline
17. आयु॑षा॒ ऽभ्य॑ भ्यायु॒षा ऽऽयु॑षा॒ ऽभि प॑वते पवते॒ ऽभ्यायु॒षा ऽऽयु॑षा॒ ऽभि प॑वते । \newline
18. अ॒भि प॑वते पवते॒ ऽभ्य॑भि प॑वते ऽभि॒चर॑न् नभि॒चर॑न् पवते॒ ऽभ्य॑भि प॑वते ऽभि॒चरन्न्॑ । \newline
19. प॒व॒ते॒ ऽभि॒चर॑न् नभि॒चर॑न् पवते पवते ऽभि॒चर॒न् नवावा॑ भि॒चर॑न् पवते पवते ऽभि॒चर॒न् नव॑ । \newline
20. अ॒भि॒चर॒न् नवावा॑ भि॒चर॑न् नभि॒चर॒न् नवे᳚क्षे तेक्षे॒तावा॑ भि॒चर॑न् नभि॒चर॒न् नवे᳚क्षेत । \newline
21. अ॒भि॒चर॒न्नित्य॑भि - चरन्न्॑ । \newline
22. अवे᳚क्षे तेक्षे॒ता वावे᳚क्षे तै॒ष ए॒ष ई᳚क्षे॒ता वावे᳚क्षे तै॒षः । \newline
23. ई॒क्षे॒ तै॒ष ए॒ष ई᳚क्षे तेक्षे तै॒ष वै वा ए॒ष ई᳚क्षे तेक्षे तै॒ष वै । \newline
24. ए॒ष वै वा ए॒ष ए॒ष वै पात्रि॑यः॒ पात्रि॑यो॒ वा ए॒ष ए॒ष वै पात्रि॑यः । \newline
25. वै पात्रि॑यः॒ पात्रि॑यो॒ वै वै पात्रि॑यः प्र॒जाप॑तिः प्र॒जाप॑तिः॒ पात्रि॑यो॒ वै वै पात्रि॑यः प्र॒जाप॑तिः । \newline
26. पात्रि॑यः प्र॒जाप॑तिः प्र॒जाप॑तिः॒ पात्रि॑यः॒ पात्रि॑यः प्र॒जाप॑तिर् य॒ज्ञो य॒ज्ञ्ः प्र॒जाप॑तिः॒ पात्रि॑यः॒ पात्रि॑यः प्र॒जाप॑तिर् य॒ज्ञ्ः । \newline
27. प्र॒जाप॑तिर् य॒ज्ञो य॒ज्ञ्ः प्र॒जाप॑तिः प्र॒जाप॑तिर् य॒ज्ञ्ः प्र॒जाप॑तिः प्र॒जाप॑तिर् य॒ज्ञ्ः प्र॒जाप॑तिः प्र॒जाप॑तिर् य॒ज्ञ्ः प्र॒जाप॑तिः । \newline
28. प्र॒जाप॑ति॒रिति॑ प्र॒जा - प॒तिः॒ । \newline
29. य॒ज्ञ्ः प्र॒जाप॑तिः प्र॒जाप॑तिर् य॒ज्ञो य॒ज्ञ्ः प्र॒जाप॑ति॒ स्तम् तम् प्र॒जाप॑तिर् य॒ज्ञो य॒ज्ञ्ः प्र॒जाप॑ति॒ स्तम् । \newline
30. प्र॒जाप॑ति॒ स्तम् तम् प्र॒जाप॑तिः प्र॒जाप॑ति॒ स्त मे॒वैव तम् प्र॒जाप॑तिः प्र॒जाप॑ति॒ स्त मे॒व । \newline
31. प्र॒जाप॑ति॒रिति॑ प्र॒जा - प॒तिः॒ । \newline
32. त मे॒वैव तम् त मे॒व त॑र्पयति तर्पय त्ये॒व तम् त मे॒व त॑र्पयति । \newline
33. ए॒व त॑र्पयति तर्पय त्ये॒वैव त॑र्पयति॒ स स त॑र्पय त्ये॒वैव त॑र्पयति॒ सः । \newline
34. त॒र्प॒य॒ति॒ स स त॑र्पयति तर्पयति॒ स ए॑न मेनꣳ॒॒ स त॑र्पयति तर्पयति॒ स ए॑नम् । \newline
35. स ए॑न मेनꣳ॒॒ स स ए॑नम् तृ॒प्त स्तृ॒प्त ए॑नꣳ॒॒ स स ए॑नम् तृ॒प्तः । \newline
36. ए॒न॒म् तृ॒प्त स्तृ॒प्त ए॑न मेनम् तृ॒प्तः प्रा॑णापा॒नाभ्या᳚म् प्राणापा॒नाभ्या᳚म् तृ॒प्त ए॑न मेनम् तृ॒प्तः प्रा॑णापा॒नाभ्या᳚म् । \newline
37. तृ॒प्तः प्रा॑णापा॒नाभ्या᳚म् प्राणापा॒नाभ्या᳚म् तृ॒प्त स्तृ॒प्तः प्रा॑णापा॒नाभ्यां᳚ ॅवा॒चो वा॒चः प्रा॑णापा॒नाभ्या᳚म् तृ॒प्त स्तृ॒प्तः प्रा॑णापा॒नाभ्यां᳚ ॅवा॒चः । \newline
38. प्रा॒णा॒पा॒नाभ्यां᳚ ॅवा॒चो वा॒चः प्रा॑णापा॒नाभ्या᳚म् प्राणापा॒नाभ्यां᳚ ॅवा॒चो द॑क्षक्र॒तुभ्या᳚म् दक्षक्र॒तुभ्यां᳚ ॅवा॒चः प्रा॑णापा॒नाभ्या᳚म् प्राणापा॒नाभ्यां᳚ ॅवा॒चो द॑क्षक्र॒तुभ्या᳚म् । \newline
39. प्रा॒णा॒पा॒नाभ्या॒मिति॑ प्राण - अ॒पा॒नाभ्या᳚म् । \newline
40. वा॒चो द॑क्षक्र॒तुभ्या᳚म् दक्षक्र॒तुभ्यां᳚ ॅवा॒चो वा॒चो द॑क्षक्र॒तुभ्या॒म् चक्षु॑र्भ्या॒म् चक्षु॑र्भ्याम् दक्षक्र॒तुभ्यां᳚ ॅवा॒चो वा॒चो द॑क्षक्र॒तुभ्या॒म् चक्षु॑र्भ्याम् । \newline
41. द॒क्ष॒क्र॒तुभ्या॒म् चक्षु॑र्भ्या॒म् चक्षु॑र्भ्याम् दक्षक्र॒तुभ्या᳚म् दक्षक्र॒तुभ्या॒म् चक्षु॑र्भ्याꣳ॒॒ श्रोत्रा᳚भ्याꣳ॒॒ श्रोत्रा᳚भ्या॒म् चक्षु॑र्भ्याम् दक्षक्र॒तुभ्या᳚म् दक्षक्र॒तुभ्या॒म् चक्षु॑र्भ्याꣳ॒॒ श्रोत्रा᳚भ्याम् । \newline
42. द॒क्ष॒क्र॒तुभ्या॒मिति॑ दक्षक्र॒तु - भ्या॒म् । \newline
43. चक्षु॑र्भ्याꣳ॒॒ श्रोत्रा᳚भ्याꣳ॒॒ श्रोत्रा᳚भ्या॒म् चक्षु॑र्भ्या॒म् चक्षु॑र्भ्याꣳ॒॒ श्रोत्रा᳚भ्या मा॒त्मन॑ आ॒त्मनः॒ श्रोत्रा᳚भ्या॒म् चक्षु॑र्भ्या॒म् चक्षु॑र्भ्याꣳ॒॒ श्रोत्रा᳚भ्या मा॒त्मनः॑ । \newline
44. चक्षु॑र्भ्या॒मिति॒ चक्षुः॑ - भ्या॒म् । \newline
45. श्रोत्रा᳚भ्या मा॒त्मन॑ आ॒त्मनः॒ श्रोत्रा᳚भ्याꣳ॒॒ श्रोत्रा᳚भ्या मा॒त्मनो ऽङ्गे॒भ्यो ऽङ्गे᳚भ्य आ॒त्मनः॒ श्रोत्रा᳚भ्याꣳ॒॒ श्रोत्रा᳚भ्या मा॒त्मनो ऽङ्गे᳚भ्यः । \newline
46. आ॒त्मनो ऽङ्गे॒भ्यो ऽङ्गे᳚भ्य आ॒त्मन॑ आ॒त्मनो ऽङ्गे᳚भ्य॒ आयु॑ष॒ आयु॒षो ऽङ्गे᳚भ्य आ॒त्मन॑ आ॒त्मनो ऽङ्गे᳚भ्य॒ आयु॑षः । \newline
47. अङ्गे᳚भ्य॒ आयु॑ष॒ आयु॒षो ऽङ्गे॒भ्यो ऽङ्गे᳚भ्य॒ आयु॑षो॒ ऽन्त र॒न्त रायु॒षो ऽङ्गे॒भ्यो ऽङ्गे᳚भ्य॒ आयु॑षो॒ ऽन्तः । \newline
48. आयु॑षो॒ ऽन्त र॒न्त रायु॑ष॒ आयु॑षो॒ ऽन्त रे᳚त्ये त्य॒न्त रायु॑ष॒ आयु॑षो॒ ऽन्त रे॑ति । \newline
49. अ॒न्त रे᳚त्ये त्य॒न्त र॒न्त रे॑ति ता॒जक् ता॒जगे᳚ त्य॒न्त र॒न्त रे॑ति ता॒जक् । \newline
50. ए॒ति॒ ता॒जक् ता॒जगे᳚ त्येति ता॒जक् प्र प्र ता॒जगे᳚ त्येति ता॒जक् प्र । \newline
51. ता॒जक् प्र प्र ता॒जक् ता॒जक् प्र ध॑न्वति धन्वति॒ प्र ता॒जक् ता॒जक् प्र ध॑न्वति । \newline
52. प्र ध॑न्वति धन्वति॒ प्र प्र ध॑न्वति । \newline
53. ध॒न्व॒तीति॑ धन्वति । \newline
\pagebreak
\markright{ TS 3.2.4.1  \hfill https://www.vedavms.in \hfill}

\section{ TS 3.2.4.1 }

\textbf{TS 3.2.4.1 } \newline
\textbf{Samhita Paata} \newline

स्फ्यः स्व॒स्तिर्वि॑घ॒नः स्व॒स्तिः पर्.शु॒र्वेदिः॑ पर॒शुर्नः॑ स्व॒स्तिः । य॒ज्ञिया॑ यज्ञ्॒कृतः॑ स्थ॒ ते मा॒स्मिन् य॒ज्ञ् उप॑ ह्वयद्ध्व॒मुप॑ मा॒ द्यावा॑पृथि॒वी ह्व॑येता॒मुपा᳚ऽऽ*स्ता॒वः क॒लशः॒ सोमो॑ अ॒ग्निरुप॑ दे॒वा उप॑ य॒ज्ञ् उप॑ मा॒ होत्रा॑ उपह॒वे ह्व॑यन्तां॒ नमो॒ऽग्नये॑ मख॒घ्नेम॒खस्य॑ मा॒ यशो᳚ऽर्या॒दित्या॑हव॒नीय॒मुप॑ तिष्ठते य॒ज्ञो वै म॒खो - [  ] \newline

\textbf{Pada Paata} \newline

स्फ्यः । स्व॒स्तिः । वि॒घ॒न इति॑ वि - घ॒नः । स्व॒स्तिः । पर्.शुः॑ । वेदिः॑ । प॒र॒शुः । नः॒ । स्व॒स्तिः ॥ य॒ज्ञियाः᳚ । य॒ज्ञ्॒कृत॒ इति॑ यज्ञ्-कृतः॑ । स्थ॒ । ते । मा॒ । अ॒स्मिन्न् । य॒ज्ञे । उपेति॑ । ह्व॒य॒द्ध्व॒म् । उपेति॑ । मा॒ । द्यावा॑पृथि॒वी इति॒ द्यावा᳚-पृ॒थि॒वी । ह्व॒ये॒ता॒म् । उपेति॑ । आ॒स्ता॒व इत्या᳚ - स्ता॒वः । क॒लशः॑ । सोमः॑ । अ॒ग्निः । उपेति॑ । दे॒वाः । उपेति॑ । य॒ज्ञ्ः । उपेति॑ । मा॒ । होत्राः᳚ । उ॒प॒ह॒व इत्यु॑प-ह॒वे । ह्व॒य॒न्ता॒म् । नमः॑ । अ॒ग्नये᳚ । म॒ख॒घ्न इति॑ मख-घ्ने । म॒खस्य॑ । मा॒ । यशः॑ । अ॒र्या॒त् । इति॑ । आ॒ह॒व॒नीय॒मित्या᳚ - ह॒व॒नीय᳚म् । उपेति॑ । ति॒ष्ठ॒ते॒ । य॒ज्ञ्ः । वै । म॒खः ।  \newline


\textbf{Krama Paata} \newline

स्फ्यः स्व॒स्तिः । स्व॒स्तिर् वि॑घ॒नः । वि॒घ॒नः स्व॒स्तिः । वि॒घ॒न इति॑ वि - घ॒नः । स्व॒स्तिः पर्.शुः॑ । पर्.शु॒र् वेदिः॑ । वेदिः॑ पर॒शुः । प॒र॒शुर् नः॑ । नः॒ स्व॒स्तिः । स्व॒स्तिरिति॑ स्व॒स्तिः ॥ य॒ज्ञिया॑ यज्ञ्॒कृतः॑ । य॒ज्ञ्॒कृतः॑ स्थ । य॒ज्ञ्॒कृत॒ इति॑ यज्ञ् - कृतः॑ । स्थ॒ ते । ते मा᳚ । मा॒ ऽस्मिन्न् । अ॒स्मिन्. य॒ज्ञे । य॒ज्ञ् उप॑ । उप॑ ह्वयद्ध्वम् । ह्व॒य॒द्ध्व॒मुप॑ । उप॑ मा । मा॒ द्यावा॑पृथि॒वी । द्यावा॑पृथि॒वी ह्व॑येताम् । द्यावा॑पृथि॒वी इति॒ द्यावा᳚ - पृ॒थि॒वी । ह्व॒ये॒ता॒मुप॑ । उपा᳚स्ता॒वः । आ॒स्ता॒वः क॒लशः॑ । आ॒स्ता॒व इत्या᳚ - स्ता॒वः । क॒लशः॒ सोमः॑ । सोमो॑ अ॒ग्निः । अ॒ग्निरुप॑ । उप॑ दे॒वाः । दे॒वा उप॑ । उप॑ य॒ज्ञ्ः । य॒ज्ञ् उप॑ । उप॑ मा । मा॒ होत्राः᳚ । होत्रा॑ उपह॒वे । उ॒प॒ह॒वे ह्व॑यन्ताम् । उ॒प॒ह॒व इत्यु॑प - ह॒वे । ह्व॒य॒न्ता॒म् नमः॑ । नमो॒ ऽग्नये᳚ । अ॒ग्नये॑ मख॒घ्ने । म॒ख॒घ्ने म॒खस्य॑ । म॒ख॒घ्न इति॑ मख - घ्ने । म॒खस्य॑ मा । मा॒ यशः॑ । यशो᳚ ऽर्यात् । अ॒र्या॒दिति॑ । इत्या॑हव॒नीय᳚म् । आ॒ह॒व॒नीय॒मुप॑ । आ॒ह॒व॒नीय॒मित्या᳚ - ह॒व॒नीय᳚म् । उप॑ तिष्ठते । ति॒ष्ठ॒ते॒ य॒ज्ञ्ः । य॒ज्ञो वै । वै म॒खः । म॒खो य॒ज्ञ्म् \newline

\textbf{Jatai Paata} \newline

1. स्फ्यः स्व॒स्तिः स्व॒स्तिः स्फ्यः स्फ्यः स्व॒स्तिः । \newline
2. स्व॒स्तिर् वि॑घ॒नो वि॑घ॒नः स्व॒स्तिः स्व॒स्तिर् वि॑घ॒नः । \newline
3. वि॒घ॒नः स्व॒स्तिः स्व॒स्तिर् वि॑घ॒नो वि॑घ॒नः स्व॒स्तिः । \newline
4. वि॒घ॒न इति॑ वि - घ॒नः । \newline
5. स्व॒स्तिः पर्.शुः॒ पर्.शुः॑ स्व॒स्तिः स्व॒स्तिः पर्.शुः॑ । \newline
6. पर्.शु॒र् वेदि॒र् वेदिः॒ पर्.शुः॒ पर्.शु॒र् वेदिः॑ । \newline
7. वेदिः॑ पर॒शुः प॑र॒शुर् वेदि॒र् वेदिः॑ पर॒शुः । \newline
8. प॒र॒शुर् नो॑ नः पर॒शुः प॑र॒शुर् नः॑ । \newline
9. नः॒ स्व॒स्तिः स्व॒स्तिर् नो॑ नः स्व॒स्तिः । \newline
10. स्व॒स्तिरिति॑ स्व॒स्तिः । \newline
11. य॒ज्ञिया॑ यज्ञ्॒कृतो॑ यज्ञ्॒कृतो॑ य॒ज्ञिया॑ य॒ज्ञिया॑ यज्ञ्॒कृतः॑ । \newline
12. य॒ज्ञ्॒कृतः॑ स्थ स्थ यज्ञ्॒कृतो॑ यज्ञ्॒कृतः॑ स्थ । \newline
13. य॒ज्ञ्॒कृत॒ इति॑ यज्ञ् - कृतः॑ । \newline
14. स्थ॒ ते ते स्थ॑ स्थ॒ ते । \newline
15. ते मा॑ मा॒ ते ते मा᳚ । \newline
16. मा॒ ऽस्मिन् न॒स्मिन् मा॑ मा॒ ऽस्मिन्न् । \newline
17. अ॒स्मिन्. य॒ज्ञे य॒ज्ञे᳚ ऽस्मिन् न॒स्मिन्. य॒ज्ञे । \newline
18. य॒ज्ञ् उपोप॑ य॒ज्ञे य॒ज्ञ् उप॑ । \newline
19. उप॑ ह्वयद्ध्वꣳ ह्वयद्ध्व॒ मुपोप॑ ह्वयद्ध्वम् । \newline
20. ह्व॒य॒द्ध्व॒ मुपोप॑ ह्वयद्ध्वꣳ ह्वयद्ध्व॒ मुप॑ । \newline
21. उप॑ मा॒ मोपोप॑ मा । \newline
22. मा॒ द्यावा॑पृथि॒वी द्यावा॑पृथि॒वी मा॑ मा॒ द्यावा॑पृथि॒वी । \newline
23. द्यावा॑पृथि॒वी ह्व॑येताꣳ ह्वयेता॒म् द्यावा॑पृथि॒वी द्यावा॑पृथि॒वी ह्व॑येताम् । \newline
24. द्यावा॑पृथि॒वी इति॒ द्यावा᳚ - पृ॒थि॒वी । \newline
25. ह्व॒ये॒ता॒ मुपोप॑ ह्वयेताꣳ ह्वयेता॒ मुप॑ । \newline
26. उपा᳚स्ता॒व आ᳚स्ता॒व उपोपा᳚स्ता॒वः । \newline
27. आ॒स्ता॒वः क॒लशः॑ क॒लश॑ आस्ता॒व आ᳚स्ता॒वः क॒लशः॑ । \newline
28. आ॒स्ता॒व इत्या᳚ - स्ता॒वः । \newline
29. क॒लशः॒ सोमः॒ सोमः॑ क॒लशः॑ क॒लशः॒ सोमः॑ । \newline
30. सोमो॑ अ॒ग्नि र॒ग्निः सोमः॒ सोमो॑ अ॒ग्निः । \newline
31. अ॒ग्नि रुपोपा॒ ग्नि र॒ग्नि रुप॑ । \newline
32. उप॑ दे॒वा दे॒वा उपोप॑ दे॒वाः । \newline
33. दे॒वा उपोप॑ दे॒वा दे॒वा उप॑ । \newline
34. उप॑ य॒ज्ञो य॒ज्ञ् उपोप॑ य॒ज्ञ्ः । \newline
35. य॒ज्ञ् उपोप॑ य॒ज्ञो य॒ज्ञ् उप॑ । \newline
36. उप॑ मा॒ मोपोप॑ मा । \newline
37. मा॒ होत्रा॒ होत्रा॑ मा मा॒ होत्राः᳚ । \newline
38. होत्रा॑ उपह॒व उ॑पह॒वे होत्रा॒ होत्रा॑ उपह॒वे । \newline
39. उ॒प॒ह॒वे ह्व॑यन्ताꣳ ह्वयन्ता मुपह॒व उ॑पह॒वे ह्व॑यन्ताम् । \newline
40. उ॒प॒ह॒व इत्यु॑प - ह॒वे । \newline
41. ह्व॒य॒न्ता॒म् नमो॒ नमो᳚ ह्वयन्ताꣳ ह्वयन्ता॒म् नमः॑ । \newline
42. नमो॒ ऽग्नये॒ ऽग्नये॒ नमो॒ नमो॒ ऽग्नये᳚ । \newline
43. अ॒ग्नये॑ मख॒घ्ने म॑ख॒घ्ने᳚ ऽग्नये॒ ऽग्नये॑ मख॒घ्ने । \newline
44. म॒ख॒घ्ने म॒खस्य॑ म॒खस्य॑ मख॒घ्ने म॑ख॒घ्ने म॒खस्य॑ । \newline
45. म॒ख॒घ्न इति॑ मख-घ्ने । \newline
46. म॒खस्य॑ मा मा म॒खस्य॑ म॒खस्य॑ मा । \newline
47. मा॒ यशो॒ यशो॑ मा मा॒ यशः॑ । \newline
48. यशो᳚ ऽर्या दर्या॒द् यशो॒ यशो᳚ ऽर्यात् । \newline
49. अ॒र्या॒ दिती त्य॑र्या दर्या॒ दिति॑ । \newline
50. इत्या॑हव॒नीय॑ माहव॒नीय॒ मिती त्या॑हव॒नीय᳚म् । \newline
51. आ॒ह॒व॒नीय॒ मुपोपा॑हव॒नीय॑ माहव॒नीय॒ मुप॑ । \newline
52. आ॒ह॒व॒नीय॒मित्या᳚ - ह॒व॒नीय᳚म् । \newline
53. उप॑ तिष्ठते तिष्ठत॒ उपोप॑ तिष्ठते । \newline
54. ति॒ष्ठ॒ते॒ य॒ज्ञो य॒ज्ञ् स्ति॑ष्ठते तिष्ठते य॒ज्ञ्ः । \newline
55. य॒ज्ञो वै वै य॒ज्ञो य॒ज्ञो वै । \newline
56. वै म॒खो म॒खो वै वै म॒खः । \newline
57. म॒खो य॒ज्ञ्ं ॅय॒ज्ञ्म् म॒खो म॒खो य॒ज्ञ्म् । \newline

\textbf{Ghana Paata } \newline

1. स्फ्यः स्व॒स्तिः स्व॒स्तिः स्फ्यः स्फ्यः स्व॒स्तिर् वि॑घ॒नो वि॑घ॒नः स्व॒स्तिः स्फ्यः स्फ्यः स्व॒स्तिर् वि॑घ॒नः । \newline
2. स्व॒स्तिर् वि॑घ॒नो वि॑घ॒नः स्व॒स्तिः स्व॒स्तिर् वि॑घ॒नः स्व॒स्तिः स्व॒स्तिर् वि॑घ॒नः स्व॒स्तिः स्व॒स्तिर् वि॑घ॒नः स्व॒स्तिः । \newline
3. वि॒घ॒नः स्व॒स्तिः स्व॒स्तिर् वि॑घ॒नो वि॑घ॒नः स्व॒स्तिः पर्.शुः॒ पर्.शुः॑ स्व॒स्तिर् वि॑घ॒नो वि॑घ॒नः स्व॒स्तिः पर्.शुः॑ । \newline
4. वि॒घ॒न इति॑ वि - घ॒नः । \newline
5. स्व॒स्तिः पर्.शुः॒ पर्.शुः॑ स्व॒स्तिः स्व॒स्तिः पर्.शु॒र् वेदि॒र् वेदिः॒ पर्.शुः॑ स्व॒स्तिः स्व॒स्तिः पर्.शु॒र् वेदिः॑ । \newline
6. पर्.शु॒र् वेदि॒र् वेदिः॒ पर्.शुः॒ पर्.शु॒र् वेदिः॑ पर॒शुः प॑र॒शुर् वेदिः॒ पर्.शुः॒ पर्.शु॒र् वेदिः॑ पर॒शुः । \newline
7. वेदिः॑ पर॒शुः प॑र॒शुर् वेदि॒र् वेदिः॑ पर॒शुर् नो॑ नः पर॒शुर् वेदि॒र् वेदिः॑ पर॒शुर् नः॑ । \newline
8. प॒र॒शुर् नो॑ नः पर॒शुः प॑र॒शुर् नः॑ स्व॒स्तिः स्व॒स्तिर् नः॑ पर॒शुः प॑र॒शुर् नः॑ स्व॒स्तिः । \newline
9. नः॒ स्व॒स्तिः स्व॒स्तिर् नो॑ नः स्व॒स्तिः । \newline
10. स्व॒स्तिरिति॑ स्व॒स्तिः । \newline
11. य॒ज्ञिया॑ यज्ञ्॒कृतो॑ यज्ञ्॒कृतो॑ य॒ज्ञिया॑ य॒ज्ञिया॑ यज्ञ्॒कृतः॑ स्थ स्थ यज्ञ्॒कृतो॑ य॒ज्ञिया॑ य॒ज्ञिया॑ यज्ञ्॒कृतः॑ स्थ । \newline
12. य॒ज्ञ्॒कृतः॑ स्थ स्थ यज्ञ्॒कृतो॑ यज्ञ्॒कृतः॑ स्थ॒ ते ते स्थ॑ यज्ञ्॒कृतो॑ यज्ञ्॒कृतः॑ स्थ॒ ते । \newline
13. य॒ज्ञ्॒कृत॒ इति॑ यज्ञ् - कृतः॑ । \newline
14. स्थ॒ ते ते स्थ॑ स्थ॒ ते मा॑ मा॒ ते स्थ॑ स्थ॒ ते मा᳚ । \newline
15. ते मा॑ मा॒ ते ते मा॒ ऽस्मिन् न॒स्मिन् मा॒ ते ते मा॒ ऽस्मिन्न् । \newline
16. मा॒ ऽस्मिन् न॒स्मिन् मा॑ मा॒ ऽस्मिन्. य॒ज्ञे य॒ज्ञे᳚ ऽस्मिन् मा॑ मा॒ ऽस्मिन्. य॒ज्ञे । \newline
17. अ॒स्मिन्. य॒ज्ञे य॒ज्ञे᳚ ऽस्मिन् न॒स्मिन्. य॒ज्ञ् उपोप॑ य॒ज्ञे᳚ ऽस्मिन् न॒स्मिन्. य॒ज्ञ् उप॑ । \newline
18. य॒ज्ञ् उपोप॑ य॒ज्ञे य॒ज्ञ् उप॑ ह्वयद्ध्वꣳ ह्वयद्ध्व॒ मुप॑ य॒ज्ञे य॒ज्ञ् उप॑ ह्वयद्ध्वम् । \newline
19. उप॑ ह्वयद्ध्वꣳ ह्वयद्ध्व॒ मुपोप॑ ह्वयद्ध्व॒ मुपोप॑ ह्वयद्ध्व॒ मुपोप॑ ह्वयद्ध्व॒ मुप॑ । \newline
20. ह्व॒य॒द्ध्व॒ मुपोप॑ ह्वयद्ध्वꣳ ह्वयद्ध्व॒ मुप॑ मा॒ मोप॑ ह्वयद्ध्वꣳ ह्वयद्ध्व॒ मुप॑ मा । \newline
21. उप॑ मा॒ मोपोप॑ मा॒ द्यावा॑पृथि॒वी द्यावा॑पृथि॒वी मोपोप॑ मा॒ द्यावा॑पृथि॒वी । \newline
22. मा॒ द्यावा॑पृथि॒वी द्यावा॑पृथि॒वी मा॑ मा॒ द्यावा॑पृथि॒वी ह्व॑येताꣳ ह्वयेता॒म् द्यावा॑पृथि॒वी मा॑ मा॒ द्यावा॑पृथि॒वी ह्व॑येताम् । \newline
23. द्यावा॑पृथि॒वी ह्व॑येताꣳ ह्वयेता॒म् द्यावा॑पृथि॒वी द्यावा॑पृथि॒वी ह्व॑येता॒ मुपोप॑ ह्वयेता॒म् द्यावा॑पृथि॒वी द्यावा॑पृथि॒वी ह्व॑येता॒ मुप॑ । \newline
24. द्यावा॑पृथि॒वी इति॒ द्यावा᳚ - पृ॒थि॒वी । \newline
25. ह्व॒ये॒ता॒ मुपोप॑ ह्वयेताꣳ ह्वयेता॒ मुपा᳚स्ता॒व आ᳚स्ता॒व उप॑ ह्वयेताꣳ ह्वयेता॒ मुपा᳚स्ता॒वः । \newline
26. उपा᳚स्ता॒व आ᳚स्ता॒व उपोपा᳚स्ता॒वः क॒लशः॑ क॒लश॑ आस्ता॒व उपोपा᳚स्ता॒वः क॒लशः॑ । \newline
27. आ॒स्ता॒वः क॒लशः॑ क॒लश॑ आस्ता॒व आ᳚स्ता॒वः क॒लशः॒ सोमः॒ सोमः॑ क॒लश॑ आस्ता॒व आ᳚स्ता॒वः क॒लशः॒ सोमः॑ । \newline
28. आ॒स्ता॒व इत्या᳚ - स्ता॒वः । \newline
29. क॒लशः॒ सोमः॒ सोमः॑ क॒लशः॑ क॒लशः॒ सोमो॑ अ॒ग्नि र॒ग्निः सोमः॑ क॒लशः॑ क॒लशः॒ सोमो॑ अ॒ग्निः । \newline
30. सोमो॑ अ॒ग्नि र॒ग्निः सोमः॒ सोमो॑ अ॒ग्नि रुपोपा॒ग्निः सोमः॒ सोमो॑ अ॒ग्निरुप॑ । \newline
31. अ॒ग्नि रुपोपा॒ग्नि र॒ग्नि रुप॑ दे॒वा दे॒वा उपा॒ग्नि र॒ग्नि रुप॑ दे॒वाः । \newline
32. उप॑ दे॒वा दे॒वा उपोप॑ दे॒वा उपोप॑ दे॒वा उपोप॑ दे॒वा उप॑ । \newline
33. दे॒वा उपोप॑ दे॒वा दे॒वा उप॑ य॒ज्ञो य॒ज्ञ् उप॑ दे॒वा दे॒वा उप॑ य॒ज्ञ्ः । \newline
34. उप॑ य॒ज्ञो य॒ज्ञ् उपोप॑ य॒ज्ञ् उपोप॑ य॒ज्ञ् उपोप॑ य॒ज्ञ् उप॑ । \newline
35. य॒ज्ञ् उपोप॑ य॒ज्ञो य॒ज्ञ् उप॑ मा॒ मोप॑ य॒ज्ञो य॒ज्ञ् उप॑ मा । \newline
36. उप॑ मा॒ मोपोप॑ मा॒ होत्रा॒ होत्रा॒ मोपोप॑ मा॒ होत्राः᳚ । \newline
37. मा॒ होत्रा॒ होत्रा॑ मा मा॒ होत्रा॑ उपह॒व उ॑पह॒वे होत्रा॑ मा मा॒ होत्रा॑ उपह॒वे । \newline
38. होत्रा॑ उपह॒व उ॑पह॒वे होत्रा॒ होत्रा॑ उपह॒वे ह्व॑यन्ताꣳ ह्वयन्ता मुपह॒वे होत्रा॒ होत्रा॑ उपह॒वे ह्व॑यन्ताम् । \newline
39. उ॒प॒ह॒वे ह्व॑यन्ताꣳ ह्वयन्ता मुपह॒व उ॑पह॒वे ह्व॑यन्ता॒म् नमो॒ नमो᳚ ह्वयन्ता मुपह॒व उ॑पह॒वे ह्व॑यन्ता॒म् नमः॑ । \newline
40. उ॒प॒ह॒व इत्यु॑प - ह॒वे । \newline
41. ह्व॒य॒न्ता॒म् नमो॒ नमो᳚ ह्वयन्ताꣳ ह्वयन्ता॒म् नमो॒ ऽग्नये॒ ऽग्नये॒ नमो᳚ ह्वयन्ताꣳ ह्वयन्ता॒म् नमो॒ ऽग्नये᳚ । \newline
42. नमो॒ ऽग्नये॒ ऽग्नये॒ नमो॒ नमो॒ ऽग्नये॑ मख॒घ्ने म॑ख॒घ्ने᳚ ऽग्नये॒ नमो॒ नमो॒ ऽग्नये॑ मख॒घ्ने । \newline
43. अ॒ग्नये॑ मख॒घ्ने म॑ख॒घ्ने᳚ ऽग्नये॒ ऽग्नये॑ मख॒घ्ने म॒खस्य॑ म॒खस्य॑ मख॒घ्ने᳚ ऽग्नये॒ ऽग्नये॑ मख॒घ्ने म॒खस्य॑ । \newline
44. म॒ख॒घ्ने म॒खस्य॑ म॒खस्य॑ मख॒घ्ने म॑ख॒घ्ने म॒खस्य॑ मा मा म॒खस्य॑ मख॒घ्ने म॑ख॒घ्ने म॒खस्य॑ मा । \newline
45. म॒ख॒घ्न इति॑ मख - घ्ने । \newline
46. म॒खस्य॑ मा मा म॒खस्य॑ म॒खस्य॑ मा॒ यशो॒ यशो॑ मा म॒खस्य॑ म॒खस्य॑ मा॒ यशः॑ । \newline
47. मा॒ यशो॒ यशो॑ मा मा॒ यशो᳚ ऽर्या दर्या॒द् यशो॑ मा मा॒ यशो᳚ ऽर्यात् । \newline
48. यशो᳚ ऽर्या दर्या॒द् यशो॒ यशो᳚ ऽर्या॒ दिती त्य॑र्या॒द् यशो॒ यशो᳚ ऽर्या॒दिति॑ । \newline
49. अ॒र्या॒ दिती त्य॑र्या दर्या॒ दि त्या॑हव॒नीय॑ माहव॒नीय॒ मित्य॑र्या दर्या॒ दित्या॑हव॒नीय᳚म् । \newline
50. इत्या॑हव॒नीय॑ माहव॒नीय॒ मिती त्या॑हव॒नीय॒ मुपोपा॑हव॒नीय॒ मिती त्या॑हव॒नीय॒ मुप॑ । \newline
51. आ॒ह॒व॒नीय॒ मुपोपा॑हव॒नीय॑ माहव॒नीय॒ मुप॑ तिष्ठते तिष्ठत॒ उपा॑हव॒नीय॑ माहव॒नीय॒ मुप॑ तिष्ठते । \newline
52. आ॒ह॒व॒नीय॒मित्या᳚ - ह॒व॒नीय᳚म् । \newline
53. उप॑ तिष्ठते तिष्ठत॒ उपोप॑ तिष्ठते य॒ज्ञो य॒ज्ञ् स्ति॑ष्ठत॒ उपोप॑ तिष्ठते य॒ज्ञ्ः । \newline
54. ति॒ष्ठ॒ते॒ य॒ज्ञो य॒ज्ञ् स्ति॑ष्ठते तिष्ठते य॒ज्ञो वै वै य॒ज्ञ् स्ति॑ष्ठते तिष्ठते य॒ज्ञो वै । \newline
55. य॒ज्ञो वै वै य॒ज्ञो य॒ज्ञो वै म॒खो म॒खो वै य॒ज्ञो य॒ज्ञो वै म॒खः । \newline
56. वै म॒खो म॒खो वै वै म॒खो य॒ज्ञ्ं ॅय॒ज्ञ्म् म॒खो वै वै म॒खो य॒ज्ञ्म् । \newline
57. म॒खो य॒ज्ञ्ं ॅय॒ज्ञ्म् म॒खो म॒खो य॒ज्ञ्ं ॅवाव वाव य॒ज्ञ्म् म॒खो म॒खो य॒ज्ञ्ं ॅवाव । \newline
\pagebreak
\markright{ TS 3.2.4.2  \hfill https://www.vedavms.in \hfill}

\section{ TS 3.2.4.2 }

\textbf{TS 3.2.4.2 } \newline
\textbf{Samhita Paata} \newline

य॒ज्ञ्ं ॅवाव स तद॑ह॒न् तस्मा॑ ए॒व न॑म॒स्कृत्य॒ सदः॒ प्रस॑र्पत्या॒त्मनोऽना᳚र्त्यै॒ नमो॑ रु॒द्राय॑ मख॒घ्ने नम॑स्कृत्या मा पा॒हीत्याग्नी᳚द्ध्रं॒ तस्मा॑ ए॒व न॑म॒स्कृत्य॒ सदः॒ प्रस॑र्पत्या॒त्मनोऽना᳚र्त्यै॒ नम॒ इन्द्रा॑य मख॒घ्न इ॑न्द्रि॒यं मे॑ वी॒र्यं॑ मा निर्व॑धी॒रिति॑ हो॒त्रीय॑मा॒शिष॑मे॒वैतामा शा᳚स्तैन्द्रि॒यस्य॑ वी॒र्य॑स्यानि॑र्घाताय॒ या वै - [  ] \newline

\textbf{Pada Paata} \newline

य॒ज्ञ्म् । वाव । सः । तत् । अ॒ह॒न्न् । तस्मै᳚ । ए॒व । न॒म॒स्कृत्येति॑ नमः - कृत्य॑ । सदः॑ । प्रेति॑ । स॒र्प॒ति॒ । आ॒त्मनः॑ । अना᳚र्त्यै । नमः॑ । रु॒द्राय॑ । म॒ख॒घ्न इति॑ मख - घ्ने । नम॑स्कृ॒त्येति॒ नमः॑-कृ॒त्या॒ । मा॒ । पा॒हि॒ । इति॑ । आग्नी᳚द्ध्र॒मित्याग्नि॑ - इ॒द्ध्र॒म् । तस्मै᳚ । ए॒व । न॒म॒स्कृत्येति॑ नमः - कृत्य॑ । सदः॑ । प्रेति॑ । स॒र्प॒ति॒ । आ॒त्मनः॑ । अना᳚र्त्यै । नमः॑ । इन्द्रा॑य । म॒ख॒घ्न इति॑ मख - घ्ने । इ॒न्द्रि॒यम् । मे॒ । वी॒र्य᳚म् । मा । निरिति॑ । व॒धीः॒ । इति॑ । हो॒त्रीय᳚म् । आ॒शिष॒मित्या᳚ - शिष᳚म् । ए॒व । ए॒ताम् । एति॑ । शा॒स्ते॒ । इ॒न्द्रि॒यस्य॑ । वी॒र्य॑स्य । अनि॑र्घाता॒येत्यनिः॑ - घा॒ता॒य॒ । याः । वै ।  \newline


\textbf{Krama Paata} \newline

य॒ज्ञ्ं ॅवाव । वाव सः । स तत् । तद॑हन्न् । अ॒ह॒न् तस्मै᳚ । तस्मा॑ ए॒व । ए॒व न॑म॒स्कृत्य॑ । न॒म॒स्कृत्य॒ सदः॑ । न॒म॒स्कृत्येति॑ नमः - कृत्य॑ । सदः॒ प्र । प्र स॑र्पति । स॒र्प॒त्या॒त्मनः॑ । आ॒त्मनो ऽना᳚र्त्यै । अना᳚र्त्यै॒ नमः॑ । नमो॑ रु॒द्राय॑ । रु॒द्राय॑ मख॒घ्ने । म॒ख॒घ्ने नम॑स्कृत्या । म॒ख॒घ्न इति॑ मख - घ्ने । नम॑स्कृत्या मा । नम॑स्कृ॒त्येति॒ नमः॑ - कृ॒त्या॒ । मा॒ पा॒हि॒ । पा॒हीति॑ । इत्याग्नी᳚ध्रम् । आग्नी᳚ध्र॒म् तस्मै᳚ । आग्नी᳚ध्र॒मित्याग्नि॑ - इ॒ध्र॒म् । तस्मा॑ ए॒व । ए॒व न॑म॒स्कृत्य॑ । न॒म॒स्कृत्य॒ सदः॑ । न॒म॒स्कृत्येति॑ नमः - कृत्य॑ । सदः॒ प्र । प्र स॑र्पति । स॒र्प॒त्या॒त्मनः॑ । आ॒त्मनो ऽना᳚र्त्यै । अना᳚र्त्यै॒ नमः॑ । नम॒ इन्द्रा॑य । इन्द्रा॑य मख॒घ्ने । म॒ख॒घ्न इ॑न्द्रि॒यम् । म॒ख॒घ्न इति॑ मख - घ्ने । इ॒न्द्रि॒यम् मे᳚ । मे॒ वी॒र्य᳚म् । वी॒र्य॑म् मा । मा निः । निर् व॑धीः । व॒धी॒रिति॑ । इति॑ हो॒त्रीय᳚म् । हो॒त्रीय॑मा॒शिष᳚म् । आ॒शिष॑मे॒व । आ॒शिष॒मित्या᳚ - शिष᳚म् । ए॒वैताम् । ए॒तामा । आ शा᳚स्ते । शा॒स्त॒ इ॒न्द्रि॒यस्य॑ । इ॒न्द्रि॒यस्य॑ वी॒र्य॑स्य । वी॒र्य॑स्यानि॑र्घाताय । अनि॑र्घाताय॒ याः । अनि॑र्घाता॒येत्यनिः॑ - घा॒ता॒य॒ । या वै । वै दे॒वताः᳚ \newline

\textbf{Jatai Paata} \newline

1. य॒ज्ञ्ं ॅवाव वाव य॒ज्ञ्ं ॅय॒ज्ञ्ं ॅवाव । \newline
2. वाव स स वाव वाव सः । \newline
3. स तत् तथ् स स तत् । \newline
4. तद॑हन् नह॒न् तत् तद॑हन्न् । \newline
5. अ॒ह॒न् तस्मै॒ तस्मा॑ अहन् नह॒न् तस्मै᳚ । \newline
6. तस्मा॑ ए॒वैव तस्मै॒ तस्मा॑ ए॒व । \newline
7. ए॒व न॑म॒स्कृत्य॑ नम॒स्कृ त्यै॒वैव न॑म॒स्कृत्य॑ । \newline
8. न॒म॒स्कृत्य॒ सदः॒ सदो॑ नम॒स्कृत्य॑ नम॒स्कृत्य॒ सदः॑ । \newline
9. न॒म॒स्कृत्येति॑ नमः - कृत्य॑ । \newline
10. सदः॒ प्र प्र सदः॒ सदः॒ प्र । \newline
11. प्र स॑र्पति सर्पति॒ प्र प्र स॑र्पति । \newline
12. स॒र्प॒ त्या॒त्मन॑ आ॒त्मनः॑ सर्पति सर्प त्या॒त्मनः॑ । \newline
13. आ॒त्मनो ऽना᳚र्त्या॒ अना᳚र्त्या आ॒त्मन॑ आ॒त्मनो ऽना᳚र्त्यै । \newline
14. अना᳚र्त्यै॒ नमो॒ नमो ऽना᳚र्त्या॒ अना᳚र्त्यै॒ नमः॑ । \newline
15. नमो॑ रु॒द्राय॑ रु॒द्राय॒ नमो॒ नमो॑ रु॒द्राय॑ । \newline
16. रु॒द्राय॑ मख॒घ्ने म॑ख॒घ्ने रु॒द्राय॑ रु॒द्राय॑ मख॒घ्ने । \newline
17. म॒ख॒घ्ने नम॑स्कृत्या॒ नम॑स्कृत्या मख॒घ्ने म॑ख॒घ्ने नम॑स्कृत्या । \newline
18. म॒ख॒घ्न इति॑ मख - घ्ने । \newline
19. नम॑स्कृत्या मा मा॒ नम॑स्कृत्या॒ नम॑स्कृत्या मा । \newline
20. नम॑स्कृ॒त्येति॒ नमः॑ - कृ॒त्या॒ । \newline
21. मा॒ पा॒हि॒ पा॒हि॒ मा॒ मा॒ पा॒हि॒ । \newline
22. पा॒हीतीति॑ पाहि पा॒हीति॑ । \newline
23. इत्याग्नी᳚द्ध्र॒ माग्नी᳚द्ध्र॒ मिती त्याग्नी᳚द्ध्रम् । \newline
24. आग्नी᳚द्ध्र॒म् तस्मै॒ तस्मा॒ आग्नी᳚द्ध्र॒ माग्नी᳚द्ध्र॒म् तस्मै᳚ । \newline
25. आग्नी᳚द्ध्र॒मित्याग्नि॑ - इ॒द्ध्र॒म् । \newline
26. तस्मा॑ ए॒वैव तस्मै॒ तस्मा॑ ए॒व । \newline
27. ए॒व न॑म॒स्कृत्य॑ नम॒स्कृ त्यै॒वैव न॑म॒स्कृत्य॑ । \newline
28. न॒म॒स्कृत्य॒ सदः॒ सदो॑ नम॒स्कृत्य॑ नम॒स्कृत्य॒ सदः॑ । \newline
29. न॒म॒स्कृत्येति॑ नमः - कृत्य॑ । \newline
30. सदः॒ प्र प्र सदः॒ सदः॒ प्र । \newline
31. प्र स॑र्पति सर्पति॒ प्र प्र स॑र्पति । \newline
32. स॒र्प॒ त्या॒त्मन॑ आ॒त्मनः॑ सर्पति सर्प त्या॒त्मनः॑ । \newline
33. आ॒त्मनो ऽना᳚र्त्या॒ अना᳚र्त्या आ॒त्मन॑ आ॒त्मनो ऽना᳚र्त्यै । \newline
34. अना᳚र्त्यै॒ नमो॒ नमो ऽना᳚र्त्या॒ अना᳚र्त्यै॒ नमः॑ । \newline
35. नम॒ इन्द्रा॒ये न्द्रा॑य॒ नमो॒ नम॒ इन्द्रा॑य । \newline
36. इन्द्रा॑य मख॒घ्ने म॑ख॒घ्न इन्द्रा॒ये न्द्रा॑य मख॒घ्ने । \newline
37. म॒ख॒घ्न इ॑न्द्रि॒य मि॑न्द्रि॒यम् म॑ख॒घ्ने म॑ख॒घ्न इ॑न्द्रि॒यम् । \newline
38. म॒ख॒घ्न इति॑ मख - घ्ने । \newline
39. इ॒न्द्रि॒यम् मे॑ म इन्द्रि॒य मि॑न्द्रि॒यम् मे᳚ । \newline
40. मे॒ वी॒र्यं॑ ॅवी॒र्य॑म् मे मे वी॒र्य᳚म् । \newline
41. वी॒र्य॑म् मा मा वी॒र्यं॑ ॅवी॒र्य॑म् मा । \newline
42. मा निर् णिर् मा मा निः । \newline
43. निर् व॑धीर् वधी॒र् निर् णिर् व॑धीः । \newline
44. व॒धी॒ रितीति॑ वधीर् वधी॒ रिति॑ । \newline
45. इति॑ हो॒त्रीयꣳ॑ हो॒त्रीय॒ मितीति॑ हो॒त्रीय᳚म् । \newline
46. हो॒त्रीय॑ मा॒शिष॑ मा॒शिषꣳ॑ हो॒त्रीयꣳ॑ हो॒त्रीय॑ मा॒शिष᳚म् । \newline
47. आ॒शिष॑ मे॒वैवाशिष॑ मा॒शिष॑ मे॒व । \newline
48. आ॒शिष॒मित्या᳚ - शिष᳚म् । \newline
49. ए॒वैता मे॒ता मे॒वैवैताम् । \newline
50. ए॒ता मैता मे॒ता मा । \newline
51. आ शा᳚स्ते शास्त॒ आ शा᳚स्ते । \newline
52. शा॒स्त॒ इ॒न्द्रि॒यस्ये᳚ न्द्रि॒यस्य॑ शास्ते शास्त इन्द्रि॒यस्य॑ । \newline
53. इ॒न्द्रि॒यस्य॑ वी॒र्य॑स्य वी॒र्य॑स्ये न्द्रि॒यस्ये᳚ न्द्रि॒यस्य॑ वी॒र्य॑स्य । \newline
54. वी॒र्य॑स्या नि॑र्घाता॒या नि॑र्घाताय वी॒र्य॑स्य वी॒र्य॑स्या नि॑र्घाताय । \newline
55. अनि॑र्घाताय॒ या या अनि॑र्घाता॒या नि॑र्घाताय॒ याः । \newline
56. अनि॑र्घाता॒येत्यनिः॑ - घा॒ता॒य॒ । \newline
57. या वै वै या या वै । \newline
58. वै दे॒वता॑ दे॒वता॒ वै वै दे॒वताः᳚ । \newline

\textbf{Ghana Paata } \newline

1. य॒ज्ञ्ं ॅवाव वाव य॒ज्ञ्ं ॅय॒ज्ञ्ं ॅवाव स स वाव य॒ज्ञ्ं ॅय॒ज्ञ्ं ॅवाव सः । \newline
2. वाव स स वाव वाव स तत् तथ् स वाव वाव स तत् । \newline
3. स तत् तथ् स स तद॑हन् नह॒न् तथ् स स तद॑हन्न् । \newline
4. तद॑हन् नह॒न् तत् तद॑ह॒न् तस्मै॒ तस्मा॑ अह॒न् तत् तद॑ह॒न् तस्मै᳚ । \newline
5. अ॒ह॒न् तस्मै॒ तस्मा॑ अहन् नह॒न् तस्मा॑ ए॒वैव तस्मा॑ अहन् नह॒न् तस्मा॑ ए॒व । \newline
6. तस्मा॑ ए॒वैव तस्मै॒ तस्मा॑ ए॒व न॑म॒स्कृत्य॑ नम॒स्कृत्यै॒व तस्मै॒ तस्मा॑ ए॒व न॑म॒स्कृत्य॑ । \newline
7. ए॒व न॑म॒स्कृत्य॑ नम॒स्कृ त्यै॒वैव न॑म॒स्कृत्य॒ सदः॒ सदो॑ नम॒स्कृ त्यै॒वैव न॑म॒स्कृत्य॒ सदः॑ । \newline
8. न॒म॒स्कृत्य॒ सदः॒ सदो॑ नम॒स्कृत्य॑ नम॒स्कृत्य॒ सदः॒ प्र प्र सदो॑ नम॒स्कृत्य॑ नम॒स्कृत्य॒ सदः॒ प्र । \newline
9. न॒म॒स्कृत्येति॑ नमः - कृत्य॑ । \newline
10. सदः॒ प्र प्र सदः॒ सदः॒ प्र स॑र्पति सर्पति॒ प्र सदः॒ सदः॒ प्र स॑र्पति । \newline
11. प्र स॑र्पति सर्पति॒ प्र प्र स॑र्प त्या॒त्मन॑ आ॒त्मनः॑ सर्पति॒ प्र प्र स॑र्प त्या॒त्मनः॑ । \newline
12. स॒र्प॒ त्या॒त्मन॑ आ॒त्मनः॑ सर्पति सर्प त्या॒त्मनो ऽना᳚र्त्या॒ अना᳚र्त्या आ॒त्मनः॑ सर्पति सर्प त्या॒त्मनो ऽना᳚र्त्यै । \newline
13. आ॒त्मनो ऽना᳚र्त्या॒ अना᳚र्त्या आ॒त्मन॑ आ॒त्मनो ऽना᳚र्त्यै॒ नमो॒ नमो ऽना᳚र्त्या आ॒त्मन॑ आ॒त्मनो ऽना᳚र्त्यै॒ नमः॑ । \newline
14. अना᳚र्त्यै॒ नमो॒ नमो ऽना᳚र्त्या॒ अना᳚र्त्यै॒ नमो॑ रु॒द्राय॑ रु॒द्राय॒ नमो ऽना᳚र्त्या॒ अना᳚र्त्यै॒ नमो॑ रु॒द्राय॑ । \newline
15. नमो॑ रु॒द्राय॑ रु॒द्राय॒ नमो॒ नमो॑ रु॒द्राय॑ मख॒घ्ने म॑ख॒घ्ने रु॒द्राय॒ नमो॒ नमो॑ रु॒द्राय॑ मख॒घ्ने । \newline
16. रु॒द्राय॑ मख॒घ्ने म॑ख॒घ्ने रु॒द्राय॑ रु॒द्राय॑ मख॒घ्ने नम॑स्कृत्या॒ नम॑स्कृत्या मख॒घ्ने रु॒द्राय॑ रु॒द्राय॑ मख॒घ्ने नम॑स्कृत्या । \newline
17. म॒ख॒घ्ने नम॑स्कृत्या॒ नम॑स्कृत्या मख॒घ्ने म॑ख॒घ्ने नम॑स्कृत्या मा मा॒ नम॑स्कृत्या मख॒घ्ने म॑ख॒घ्ने नम॑स्कृत्या मा । \newline
18. म॒ख॒घ्न इति॑ मख - घ्ने । \newline
19. नम॑स्कृत्या मा मा॒ नम॑स्कृत्या॒ नम॑स्कृत्या मा पाहि पाहि मा॒ नम॑स्कृत्या॒ नम॑स्कृत्या मा पाहि । \newline
20. नम॑स्कृ॒त्येति॒ नमः॑ - कृ॒त्या॒ । \newline
21. मा॒ पा॒हि॒ पा॒हि॒ मा॒ मा॒ पा॒हीतीति॑ पाहि मा मा पा॒हीति॑ । \newline
22. पा॒हीतीति॑ पाहि पा॒ही त्याग्नी᳚द्ध्र॒ माग्नी᳚द्ध्र॒ मिति॑ पाहि पा॒ही त्याग्नी᳚द्ध्रम् । \newline
23. इत्याग्नी᳚द्ध्र॒ माग्नी᳚द्ध्र॒ मिती त्याग्नी᳚द्ध्र॒म् तस्मै॒ तस्मा॒ आग्नी᳚द्ध्र॒ मिती त्याग्नी᳚द्ध्र॒म् तस्मै᳚ । \newline
24. आग्नी᳚द्ध्र॒म् तस्मै॒ तस्मा॒ आग्नी᳚द्ध्र॒ माग्नी᳚द्ध्र॒म् तस्मा॑ ए॒वैव तस्मा॒ आग्नी᳚द्ध्र॒ माग्नी᳚द्ध्र॒म् तस्मा॑ ए॒व । \newline
25. आग्नी᳚द्ध्र॒मित्याग्नि॑ - इ॒द्ध्र॒म् । \newline
26. तस्मा॑ ए॒वैव तस्मै॒ तस्मा॑ ए॒व न॑म॒स्कृत्य॑ नम॒स्कृत्यै॒व तस्मै॒ तस्मा॑ ए॒व न॑म॒स्कृत्य॑ । \newline
27. ए॒व न॑म॒स्कृत्य॑ नम॒स्कृ त्यै॒वैव न॑म॒स्कृत्य॒ सदः॒ सदो॑ नम॒स्कृ त्यै॒वैव न॑म॒स्कृत्य॒ सदः॑ । \newline
28. न॒म॒स्कृत्य॒ सदः॒ सदो॑ नम॒स्कृत्य॑ नम॒स्कृत्य॒ सदः॒ प्र प्र सदो॑ नम॒स्कृत्य॑ नम॒स्कृत्य॒ सदः॒ प्र । \newline
29. न॒म॒स्कृत्येति॑ नमः - कृत्य॑ । \newline
30. सदः॒ प्र प्र सदः॒ सदः॒ प्र स॑र्पति सर्पति॒ प्र सदः॒ सदः॒ प्र स॑र्पति । \newline
31. प्र स॑र्पति सर्पति॒ प्र प्र स॑र्प त्या॒त्मन॑ आ॒त्मनः॑ सर्पति॒ प्र प्र स॑र्प त्या॒त्मनः॑ । \newline
32. स॒र्प॒ त्या॒त्मन॑ आ॒त्मनः॑ सर्पति सर्प त्या॒त्मनो ऽना᳚र्त्या॒ अना᳚र्त्या आ॒त्मनः॑ सर्पति सर्प त्या॒त्मनो ऽना᳚र्त्यै । \newline
33. आ॒त्मनो ऽना᳚र्त्या॒ अना᳚र्त्या आ॒त्मन॑ आ॒त्मनो ऽना᳚र्त्यै॒ नमो॒ नमो ऽना᳚र्त्या आ॒त्मन॑ आ॒त्मनो ऽना᳚र्त्यै॒ नमः॑ । \newline
34. अना᳚र्त्यै॒ नमो॒ नमो ऽना᳚र्त्या॒ अना᳚र्त्यै॒ नम॒ इन्द्रा॒ये न्द्रा॑य॒ नमो ऽना᳚र्त्या॒ अना᳚र्त्यै॒ नम॒ इन्द्रा॑य । \newline
35. नम॒ इन्द्रा॒ये न्द्रा॑य॒ नमो॒ नम॒ इन्द्रा॑य मख॒घ्ने म॑ख॒घ्न इन्द्रा॑य॒ नमो॒ नम॒ इन्द्रा॑य मख॒घ्ने । \newline
36. इन्द्रा॑य मख॒घ्ने म॑ख॒घ्न इन्द्रा॒ये न्द्रा॑य मख॒घ्न इ॑न्द्रि॒य मि॑न्द्रि॒यम् म॑ख॒घ्न इन्द्रा॒ये न्द्रा॑य मख॒घ्न इ॑न्द्रि॒यम् । \newline
37. म॒ख॒घ्न इ॑न्द्रि॒य मि॑न्द्रि॒यम् म॑ख॒घ्ने म॑ख॒घ्न इ॑न्द्रि॒यम् मे॑ म इन्द्रि॒यम् म॑ख॒घ्ने म॑ख॒घ्न इ॑न्द्रि॒यम् मे᳚ । \newline
38. म॒ख॒घ्न इति॑ मख - घ्ने । \newline
39. इ॒न्द्रि॒यम् मे॑ म इन्द्रि॒य मि॑न्द्रि॒यम् मे॑ वी॒र्यं॑ ॅवी॒र्य॑म् म इन्द्रि॒य मि॑न्द्रि॒यम् मे॑ वी॒र्य᳚म् । \newline
40. मे॒ वी॒र्यं॑ ॅवी॒र्य॑म् मे मे वी॒र्य॑म् मा मा वी॒र्य॑म् मे मे वी॒र्य॑म् मा । \newline
41. वी॒र्य॑म् मा मा वी॒र्यं॑ ॅवी॒र्य॑म् मा निर् णिर् मा वी॒र्यं॑ ॅवी॒र्य॑म् मा निः । \newline
42. मा निर् णिर् मा मा निर् व॑धीर् वधी॒र् निर् मा मा निर् व॑धीः । \newline
43. निर् व॑धीर् वधी॒र् निर् णिर् व॑धी॒ रितीति॑ वधी॒र् निर् णिर् व॑धी॒रिति॑ । \newline
44. व॒धी॒रितीति॑ वधीर् वधी॒रिति॑ हो॒त्रीयꣳ॑ हो॒त्रीय॒ मिति॑ वधीर् वधी॒रिति॑ हो॒त्रीय᳚म् । \newline
45. इति॑ हो॒त्रीयꣳ॑ हो॒त्रीय॒ मितीति॑ हो॒त्रीय॑ मा॒शिष॑ मा॒शिषꣳ॑ हो॒त्रीय॒ मितीति॑ हो॒त्रीय॑ मा॒शिष᳚म् । \newline
46. हो॒त्रीय॑ मा॒शिष॑ मा॒शिषꣳ॑ हो॒त्रीयꣳ॑ हो॒त्रीय॑ मा॒शिष॑ मे॒वै वाशिषꣳ॑ हो॒त्रीयꣳ॑ हो॒त्रीय॑ मा॒शिष॑ मे॒व । \newline
47. आ॒शिष॑ मे॒वैवाशिष॑ मा॒शिष॑ मे॒वैता मे॒ता मे॒वाशिष॑ मा॒शिष॑ मे॒वैताम् । \newline
48. आ॒शिष॒मित्या᳚ - शिष᳚म् । \newline
49. ए॒वैता मे॒ता मे॒वै वैता मैता मे॒वैवैता मा । \newline
50. ए॒ता मैता मे॒ता मा शा᳚स्ते शास्त॒ ऐता मे॒ता मा शा᳚स्ते । \newline
51. आ शा᳚स्ते शास्त॒ आ शा᳚स्त इन्द्रि॒यस्ये᳚ न्द्रि॒यस्य॑ शास्त॒ आ शा᳚स्त इन्द्रि॒यस्य॑ । \newline
52. शा॒स्त॒ इ॒न्द्रि॒यस्ये᳚ न्द्रि॒यस्य॑ शास्ते शास्त इन्द्रि॒यस्य॑ वी॒र्य॑स्य वी॒र्य॑स्ये न्द्रि॒यस्य॑ शास्ते शास्त इन्द्रि॒यस्य॑ वी॒र्य॑स्य । \newline
53. इ॒न्द्रि॒यस्य॑ वी॒र्य॑स्य वी॒र्य॑स्ये न्द्रि॒यस्ये᳚ न्द्रि॒यस्य॑ वी॒र्य॑स्या नि॑र्घाता॒या नि॑र्घाताय वी॒र्य॑स्ये न्द्रि॒यस्ये᳚ न्द्रि॒यस्य॑ वी॒र्य॑स्या नि॑र्घाताय । \newline
54. वी॒र्य॑स्या नि॑र्घाता॒या नि॑र्घाताय वी॒र्य॑स्य वी॒र्य॑स्या नि॑र्घाताय॒ या या अनि॑र्घाताय वी॒र्य॑स्य वी॒र्य॑स्या नि॑र्घाताय॒ याः । \newline
55. अनि॑र्घाताय॒ या या अनि॑र्घाता॒या नि॑र्घाताय॒ या वै वै या अनि॑र्घाता॒या नि॑र्घाताय॒ या वै । \newline
56. अनि॑र्घाता॒येत्यनिः॑ - घा॒ता॒य॒ । \newline
57. या वै वै या या वै दे॒वता॑ दे॒वता॒ वै या या वै दे॒वताः᳚ । \newline
58. वै दे॒वता॑ दे॒वता॒ वै वै दे॒वताः॒ सद॑सि॒ सद॑सि दे॒वता॒ वै वै दे॒वताः॒ सद॑सि । \newline
\pagebreak
\markright{ TS 3.2.4.3  \hfill https://www.vedavms.in \hfill}

\section{ TS 3.2.4.3 }

\textbf{TS 3.2.4.3 } \newline
\textbf{Samhita Paata} \newline

दे॒वताः॒ सद॒स्यार्ति॑मा॒र्पय॑न्ति॒ यस्ता वि॒द्वान् प्र॒सर्प॑ति॒ न सद॒स्यार्ति॒मार्च्छ॑ति॒ नमो॒ऽग्नये॑ मख॒घ्न इत्या॑है॒ता वै दे॒वताः॒ सद॒स्यार्ति॒माऽर्प॑यन्ति॒ ता य ए॒वं ॅवि॒द्वान् प्र॒सर्प॑ति॒ न सद॒स्यार्ति॒मार्च्छ॑ति द्दृ॒ढे स्थः॑ शिथि॒रे स॒मीची॒ माऽꣳह॑सस्पातꣳ॒॒ सूर्यो॑ मा दे॒वो दि॒व्यादꣳह॑सस्पातु वा॒युर॒न्तरि॑क्षा - [  ] \newline

\textbf{Pada Paata} \newline

दे॒वताः᳚ । सद॑सि । आर्ति᳚म् । आ॒र्पय॒न्तीत्या᳚ - अ॒र्पय॑न्ति । यः । ताः । वि॒द्वान् । प्र॒सर्प॒तीति॑ प्र - सर्प॑ति । न । सद॑सि । आर्ति᳚म् । एति॑ । ऋ॒च्छ॒ति॒ । नमः॑ । अ॒ग्नये᳚ । म॒ख॒घ्न इति॑ मख - घ्ने । इति॑ । आ॒ह॒ । ए॒ताः । वै । दे॒वताः᳚ । सद॑सि । आर्ति᳚म् । एति॑ । अ॒र्प॒य॒न्ति॒ । ताः । यः । ए॒वम् । वि॒द्वान् । प्र॒सर्प॒तीति॑ प्र - सर्प॑ति । न । सद॑सि । आर्ति᳚म् । एति॑ । ऋ॒च्छ॒ति॒ । दृ॒ढे इति॑ । स्थः॒ । शि॒थि॒रे इति॑ । स॒मीची॒ इति॑ । मा॒ । अꣳह॑सः । पा॒त॒म् । सूर्यः॑ । मा॒ । दे॒वः । दि॒व्यात् । अꣳह॑सः । पा॒तु॒ । वा॒युः । अ॒न्तरि॑क्षात् ।  \newline


\textbf{Krama Paata} \newline

दे॒वताः॒ सद॑सि । सद॒स्यार्ति᳚म् । आर्ति॑मा॒र्पय॑न्ति । आ॒र्पय॑न्ति॒ यः । आ॒र्पय॒न्तीत्या᳚ - अ॒र्पय॑न्ति । यस्ताः । ता वि॒द्वान् । वि॒द्वान् प्र॒सर्प॑ति । प्र॒सर्प॑ति॒ न । प्र॒सर्प॒तीति॑ प्र - सर्प॑ति । न सद॑सि । सद॒स्यार्ति᳚म् । आर्ति॒मा । आर्च्छ॑ति । ऋ॒च्छ॒ति॒ नमः॑ । नमो॒ ऽग्नये᳚ । अ॒ग्नये॑ मख॒घ्ने । म॒ख॒घ्न इति॑ । म॒ख॒घ्न इति॑ मख - घ्ने । इत्या॑ह । आ॒है॒ताः । ए॒ता वै । वै दे॒वताः᳚ । दे॒वताः॒ सद॑सि । सद॒स्यार्ति᳚म् । आर्ति॒मा । आ ऽर्प॑यन्ति । अ॒र्प॒य॒न्ति॒ ताः । ता यः । य ए॒वम् । ए॒वं ॅवि॒द्वान् । वि॒द्वान् प्र॒सर्प॑ति । प्र॒सर्प॑ति॒ न । प्र॒सर्प॒तीति॑ प्र - सर्प॑ति । न सद॑सि । सद॒स्यार्ति᳚म् । आर्ति॒मा । आर्च्छ॑ति । ऋ॒च्छ॒ति॒ दृ॒ढे । दृ॒ढे स्थः॑ । दृ॒ढे इति॑ दृ॒ढे । स्थः॒ शि॒थि॒रे । शि॒थि॒रे स॒मीची᳚ । शि॒थि॒रे इति॑ शिथि॒रे । स॒मीची॑ मा । स॒मीची॒ इति॑ स॒मीची᳚ । मा ऽꣳह॑सः । अꣳह॑सस्पातम् । पा॒तꣳ॒॒ सूर्यः॑ । सूर्यो॑ मा । मा॒ दे॒वः । दे॒वो दि॒व्यात् । दि॒व्यादꣳह॑सः । अꣳह॑सस्पातु । पा॒तु॒ वा॒युः । वा॒युर॒न्तरि॑क्षात् । अ॒न्तरि॑क्षाद॒ग्निः \newline

\textbf{Jatai Paata} \newline

1. दे॒वताः॒ सद॑सि॒ सद॑सि दे॒वता॑ दे॒वताः॒ सद॑सि । \newline
2. सद॒स्यार्ति॒ मार्तिꣳ॒॒ सद॑सि॒ सद॒स्यार्ति᳚म् । \newline
3. आर्ति॑ मा॒र्पय॑ न्त्या॒र्पय॒ न्त्यार्ति॒ मार्ति॑ मा॒र्पय॑न्ति । \newline
4. आ॒र्पय॑न्ति॒ यो य आ॒र्पय॑ न्त्या॒र्पय॑न्ति॒ यः । \newline
5. आ॒र्पय॒न्तीत्या᳚ - अ॒र्पय॑न्ति । \newline
6. य स्ता स्ता यो य स्ताः । \newline
7. ता वि॒द्वान्. वि॒द्वाꣳ स्ता स्ता वि॒द्वान् । \newline
8. वि॒द्वान् प्र॒सर्प॑ति प्र॒सर्प॑ति वि॒द्वान्. वि॒द्वान् प्र॒सर्प॑ति । \newline
9. प्र॒सर्प॑ति॒ न न प्र॒सर्प॑ति प्र॒सर्प॑ति॒ न । \newline
10. प्र॒सर्प॒तीति॑ प्र - सर्प॑ति । \newline
11. न सद॑सि॒ सद॑सि॒ न न सद॑सि । \newline
12. सद॒ स्यार्ति॒ मार्तिꣳ॒॒ सद॑सि॒ सद॒ स्यार्ति᳚म् । \newline
13. आर्ति॒ मा ऽऽर्ति॒ मार्ति॒ मा । \newline
14. आर्च्छ॑ त्यृच्छ॒ त्यार्च्छ॑ति । \newline
15. ऋ॒च्छ॒ति॒ नमो॒ नम॑ ऋच्छ त्यृच्छति॒ नमः॑ । \newline
16. नमो॒ ऽग्नये॒ ऽग्नये॒ नमो॒ नमो॒ ऽग्नये᳚ । \newline
17. अ॒ग्नये॑ मख॒घ्ने म॑ख॒घ्ने᳚ ऽग्नये॒ ऽग्नये॑ मख॒घ्ने । \newline
18. म॒ख॒घ्न इतीति॑ मख॒घ्ने म॑ख॒घ्न इति॑ । \newline
19. म॒ख॒घ्न इति॑ मख - घ्ने । \newline
20. इत्या॑हा॒हे तीत्या॑ह । \newline
21. आ॒है॒ता ए॒ता आ॑हाहै॒ताः । \newline
22. ए॒ता वै वा ए॒ता ए॒ता वै । \newline
23. वै दे॒वता॑ दे॒वता॒ वै वै दे॒वताः᳚ । \newline
24. दे॒वताः॒ सद॑सि॒ सद॑सि दे॒वता॑ दे॒वताः॒ सद॑सि । \newline
25. सद॒ स्यार्ति॒ मार्तिꣳ॒॒ सद॑सि॒ सद॒ स्यार्ति᳚म् । \newline
26. आर्ति॒ मा ऽऽर्ति॒ मार्ति॒ मा । \newline
27. आ ऽर्प॑य न्त्यर्पय॒ न्त्या ऽर्प॑यन्ति । \newline
28. अ॒र्प॒य॒न्ति॒ तास्ता अ॑र्पय न्त्यर्पयन्ति॒ ताः । \newline
29. ता यो य स्ता स्ता यः । \newline
30. य ए॒व मे॒वं ॅयो य ए॒वम् । \newline
31. ए॒वं ॅवि॒द्वान्. वि॒द्वा ने॒व मे॒वं ॅवि॒द्वान् । \newline
32. वि॒द्वान् प्र॒सर्प॑ति प्र॒सर्प॑ति वि॒द्वान्. वि॒द्वान् प्र॒सर्प॑ति । \newline
33. प्र॒सर्प॑ति॒ न न प्र॒सर्प॑ति प्र॒सर्प॑ति॒ न । \newline
34. प्र॒सर्प॒तीति॑ प्र - सर्प॑ति । \newline
35. न सद॑सि॒ सद॑सि॒ न न सद॑सि । \newline
36. सद॒ स्यार्ति॒ मार्तिꣳ॒॒ सद॑सि॒ सद॒ स्यार्ति᳚म् । \newline
37. आर्ति॒ मा ऽऽर्ति॒ मार्ति॒ मा । \newline
38. आर्च्छ॑ त्यृच्छ॒ त्यार्च्छ॑ति । \newline
39. ऋ॒च्छ॒ति॒ दृ॒ढे दृ॒ढे ऋ॑च्छ त्यृच्छति दृ॒ढे । \newline
40. दृ॒ढे स्थः॑ स्थो दृ॒ढे दृ॒ढे स्थः॑ । \newline
41. दृ॒ढे इति॑ दृ॒ढे । \newline
42. स्थः॒ शि॒थि॒रे शि॑थि॒रे स्थः॑ स्थः शिथि॒रे । \newline
43. शि॒थि॒रे स॒मीची॑ स॒मीची॑ शिथि॒रे शि॑थि॒रे स॒मीची᳚ । \newline
44. शि॒थि॒रे इति॑ शिथि॒रे । \newline
45. स॒मीची॑ मा मा स॒मीची॑ स॒मीची॑ मा । \newline
46. स॒मीची॒ इति॑ स॒मीची᳚ । \newline
47. मा ऽꣳह॒सो ऽꣳह॑सो मा॒ मा ऽꣳह॑सः । \newline
48. अꣳह॑स स्पातम् पात॒ मꣳह॒सो ऽꣳह॑स स्पातम् । \newline
49. पा॒तꣳ॒॒ सूर्यः॒ सूर्यः॑ पातम् पातꣳ॒॒ सूर्यः॑ । \newline
50. सूर्यो॑ मा मा॒ सूर्यः॒ सूर्यो॑ मा । \newline
51. मा॒ दे॒वो दे॒वो मा॑ मा दे॒वः । \newline
52. दे॒वो दि॒व्याद् दि॒व्याद् दे॒वो दे॒वो दि॒व्यात् । \newline
53. दि॒व्या दꣳह॒सो ऽꣳह॑सो दि॒व्याद् दि॒व्या दꣳह॑सः । \newline
54. अꣳह॑स स्पातु पा॒त्वꣳह॒सो ऽꣳह॑स स्पातु । \newline
55. पा॒तु॒ वा॒युर् वा॒युः पा॑तु पातु वा॒युः । \newline
56. वा॒यु र॒न्तरि॑क्षा द॒न्तरि॑क्षाद् वा॒युर् वा॒यु र॒न्तरि॑क्षात् । \newline
57. अ॒न्तरि॑क्षा द॒ग्नि र॒ग्नि र॒न्तरि॑क्षा द॒न्तरि॑क्षा द॒ग्निः । \newline

\textbf{Ghana Paata } \newline

1. दे॒वताः॒ सद॑सि॒ सद॑सि दे॒वता॑ दे॒वताः॒ सद॒स्यार्ति॒ मार्तिꣳ॒॒ सद॑सि दे॒वता॑ दे॒वताः॒ सद॒स्यार्ति᳚म् । \newline
2. सद॒स्यार्ति॒ मार्तिꣳ॒॒ सद॑सि॒ सद॒स्यार्ति॑ मा॒र्पय॑ न्त्या॒र्पय॒ न्त्यार्तिꣳ॒॒ सद॑सि॒ सद॒स्यार्ति॑ मा॒र्पय॑न्ति । \newline
3. आर्ति॑ मा॒र्पय॑ न्त्या॒र्पय॒ न्त्यार्ति॒ मार्ति॑ मा॒र्पय॑न्ति॒ यो य आ॒र्पय॒ न्त्यार्ति॒ मार्ति॑ मा॒र्पय॑न्ति॒ यः । \newline
4. आ॒र्पय॑न्ति॒ यो य आ॒र्पय॑ न्त्या॒र्पय॑न्ति॒ यस्ता स्ता य आ॒र्पय॑ न्त्या॒र्पय॑न्ति॒ यस्ताः । \newline
5. आ॒र्पय॒न्तीत्या᳚ - अ॒र्पय॑न्ति । \newline
6. यस्ता स्ता यो यस्ता वि॒द्वान्. वि॒द्वाꣳ स्ता यो यस्ता वि॒द्वान् । \newline
7. ता वि॒द्वान्. वि॒द्वाꣳ स्ता स्ता वि॒द्वान् प्र॒सर्प॑ति प्र॒सर्प॑ति वि॒द्वाꣳ स्ता स्ता वि॒द्वान् प्र॒सर्प॑ति । \newline
8. वि॒द्वान् प्र॒सर्प॑ति प्र॒सर्प॑ति वि॒द्वान्. वि॒द्वान् प्र॒सर्प॑ति॒ न न प्र॒सर्प॑ति वि॒द्वान्. वि॒द्वान् प्र॒सर्प॑ति॒ न । \newline
9. प्र॒सर्प॑ति॒ न न प्र॒सर्प॑ति प्र॒सर्प॑ति॒ न सद॑सि॒ सद॑सि॒ न प्र॒सर्प॑ति प्र॒सर्प॑ति॒ न सद॑सि । \newline
10. प्र॒सर्प॒तीति॑ प्र - सर्प॑ति । \newline
11. न सद॑सि॒ सद॑सि॒ न न सद॒स्यार्ति॒ मार्तिꣳ॒॒ सद॑सि॒ न न सद॒स्यार्ति᳚म् । \newline
12. सद॒स्यार्ति॒ मार्तिꣳ॒॒ सद॑सि॒ सद॒स्यार्ति॒ मा ऽऽर्तिꣳ॒॒ सद॑सि॒ सद॒स्यार्ति॒ मा । \newline
13. आर्ति॒ मा ऽऽर्ति॒ मार्ति॒ मार्च्छ॑ त्यृच्छ॒त्या ऽऽर्ति॒ मार्ति॒ मार्च्छ॑ति । \newline
14. आर्च्छ॑ त्यृच्छ॒ त्यार्च्छ॑ति॒ नमो॒ नम॑ ऋच्छ॒ त्यार्च्छ॑ति॒ नमः॑ । \newline
15. ऋ॒च्छ॒ति॒ नमो॒ नम॑ ऋच्छ त्यृच्छति॒ नमो॒ ऽग्नये॒ ऽग्नये॒ नम॑ ऋच्छ त्यृच्छति॒ नमो॒ ऽग्नये᳚ । \newline
16. नमो॒ ऽग्नये॒ ऽग्नये॒ नमो॒ नमो॒ ऽग्नये॑ मख॒घ्ने म॑ख॒घ्ने᳚ ऽग्नये॒ नमो॒ नमो॒ ऽग्नये॑ मख॒घ्ने । \newline
17. अ॒ग्नये॑ मख॒घ्ने म॑ख॒घ्ने᳚ ऽग्नये॒ ऽग्नये॑ मख॒घ्न इतीति॑ मख॒घ्ने᳚ ऽग्नये॒ ऽग्नये॑ मख॒घ्न इति॑ । \newline
18. म॒ख॒घ्न इतीति॑ मख॒घ्ने म॑ख॒घ्न इत्या॑हा॒हे ति॑ मख॒घ्ने म॑ख॒घ्न इत्या॑ह । \newline
19. म॒ख॒घ्न इति॑ मख - घ्ने । \newline
20. इत्या॑हा॒हे तीत्या॑ है॒ता ए॒ता आ॒हे तीत्या॑है॒ताः । \newline
21. आ॒है॒ता ए॒ता आ॑हा है॒ता वै वा ए॒ता आ॑हा है॒ता वै । \newline
22. ए॒ता वै वा ए॒ता ए॒ता वै दे॒वता॑ दे॒वता॒ वा ए॒ता ए॒ता वै दे॒वताः᳚ । \newline
23. वै दे॒वता॑ दे॒वता॒ वै वै दे॒वताः॒ सद॑सि॒ सद॑सि दे॒वता॒ वै वै दे॒वताः॒ सद॑सि । \newline
24. दे॒वताः॒ सद॑सि॒ सद॑सि दे॒वता॑ दे॒वताः॒ सद॒स्यार्ति॒ मार्तिꣳ॒॒ सद॑सि दे॒वता॑ दे॒वताः॒ सद॒स्यार्ति᳚म् । \newline
25. सद॒स्यार्ति॒ मार्तिꣳ॒॒ सद॑सि॒ सद॒स्यार्ति॒ मा ऽऽर्तिꣳ॒॒ सद॑सि॒ सद॒स्यार्ति॒ मा । \newline
26. आर्ति॒ मा ऽऽर्ति॒ मार्ति॒ मा ऽर्प॑य न्त्यर्पय॒ न्त्या ऽऽर्ति॒ मार्ति॒ मा ऽर्प॑यन्ति । \newline
27. आ ऽर्प॑य न्त्यर्पय॒ न्त्या ऽर्प॑यन्ति॒ ता स्ता अ॑र्पय॒ न्त्या ऽर्प॑यन्ति॒ ताः । \newline
28. अ॒र्प॒य॒न्ति॒ ता स्ता अ॑र्पय न्त्यर्पयन्ति॒ ता यो यस्ता अ॑र्पय न्त्यर्पयन्ति॒ ता यः । \newline
29. ता यो य स्ता स्ता य ए॒व मे॒वं ॅय स्ता स्ता य ए॒वम् । \newline
30. य ए॒व मे॒वं ॅयो य ए॒वं ॅवि॒द्वान्. वि॒द्वा ने॒वं ॅयो य ए॒वं ॅवि॒द्वान् । \newline
31. ए॒वं ॅवि॒द्वान्. वि॒द्वा ने॒व मे॒वं ॅवि॒द्वान् प्र॒सर्प॑ति प्र॒सर्प॑ति वि॒द्वा ने॒व मे॒वं ॅवि॒द्वान् प्र॒सर्प॑ति । \newline
32. वि॒द्वान् प्र॒सर्प॑ति प्र॒सर्प॑ति वि॒द्वान्. वि॒द्वान् प्र॒सर्प॑ति॒ न न प्र॒सर्प॑ति वि॒द्वान्. वि॒द्वान् प्र॒सर्प॑ति॒ न । \newline
33. प्र॒सर्प॑ति॒ न न प्र॒सर्प॑ति प्र॒सर्प॑ति॒ न सद॑सि॒ सद॑सि॒ न प्र॒सर्प॑ति प्र॒सर्प॑ति॒ न सद॑सि । \newline
34. प्र॒सर्प॒तीति॑ प्र - सर्प॑ति । \newline
35. न सद॑सि॒ सद॑सि॒ न न सद॒स्यार्ति॒ मार्तिꣳ॒॒ सद॑सि॒ न न सद॒स्यार्ति᳚म् । \newline
36. सद॒स्यार्ति॒ मार्तिꣳ॒॒ सद॑सि॒ सद॒स्यार्ति॒ मा ऽऽर्तिꣳ॒॒ सद॑सि॒ सद॒स्यार्ति॒ मा । \newline
37. आर्ति॒ मा ऽऽर्ति॒ मार्ति॒ मार्च्छ॑ त्यृच्छ॒त्या ऽऽर्ति॒ मार्ति॒ मार्च्छ॑ति । \newline
38. आर्च्छ॑ त्यृच्छ॒ त्यार्च्छ॑ति दृ॒ढे दृ॒ढे ऋ॑च्छ॒ त्यार्च्छ॑ति दृ॒ढे । \newline
39. ऋ॒च्छ॒ति॒ दृ॒ढे दृ॒ढे ऋ॑च्छ त्यृच्छति दृ॒ढे स्थः॑ स्थो दृ॒ढे ऋ॑च्छ त्यृच्छति दृ॒ढे स्थः॑ । \newline
40. दृ॒ढे स्थः॑ स्थो दृ॒ढे दृ॒ढे स्थः॑ शिथि॒रे शि॑थि॒रे स्थो॑ दृ॒ढे दृ॒ढे स्थः॑ शिथि॒रे । \newline
41. दृ॒ढे इति॑ दृ॒ढे । \newline
42. स्थः॒ शि॒थि॒रे शि॑थि॒रे स्थः॑ स्थः शिथि॒रे स॒मीची॑ स॒मीची॑ शिथि॒रे स्थः॑ स्थः शिथि॒रे स॒मीची᳚ । \newline
43. शि॒थि॒रे स॒मीची॑ स॒मीची॑ शिथि॒रे शि॑थि॒रे स॒मीची॑ मा मा स॒मीची॑ शिथि॒रे शि॑थि॒रे स॒मीची॑ मा । \newline
44. शि॒थि॒रे इति॑ शिथि॒रे । \newline
45. स॒मीची॑ मा मा स॒मीची॑ स॒मीची॒ मा ऽꣳह॒सो ऽꣳह॑सो मा स॒मीची॑ स॒मीची॒ मा ऽꣳह॑सः । \newline
46. स॒मीची॒ इति॑ स॒मीची᳚ । \newline
47. मा ऽꣳह॒सो ऽꣳह॑सो मा॒ मा ऽꣳह॑स स्पातम् पात॒ मꣳह॑सो मा॒ मा ऽꣳह॑स स्पातम् । \newline
48. अꣳह॑स स्पातम् पात॒ मꣳह॒सो ऽꣳह॑स स्पातꣳ॒॒ सूर्यः॒ सूर्यः॑ पात॒ मꣳह॒सो 
ऽꣳह॑स स्पातꣳ॒॒ सूर्यः॑ । \newline
49. पा॒तꣳ॒॒ सूर्यः॒ सूर्यः॑ पातम् पातꣳ॒॒ सूर्यो॑ मा मा॒ सूर्यः॑ पातम् पातꣳ॒॒ सूर्यो॑ मा । \newline
50. सूर्यो॑ मा मा॒ सूर्यः॒ सूर्यो॑ मा दे॒वो दे॒वो मा॒ सूर्यः॒ सूर्यो॑ मा दे॒वः । \newline
51. मा॒ दे॒वो दे॒वो मा॑ मा दे॒वो दि॒व्याद् दि॒व्याद् दे॒वो मा॑ मा दे॒वो दि॒व्यात् । \newline
52. दे॒वो दि॒व्याद् दि॒व्याद् दे॒वो दे॒वो दि॒व्या दꣳह॒सो ऽꣳह॑सो दि॒व्याद् दे॒वो दे॒वो दि॒व्या दꣳह॑सः । \newline
53. दि॒व्या दꣳह॒सो ऽꣳह॑सो दि॒व्याद् दि॒व्या दꣳह॑स स्पातु पा॒त्वꣳह॑सो दि॒व्याद् दि॒व्या दꣳह॑स स्पातु । \newline
54. अꣳह॑स स्पातु पा॒त्वꣳह॒सो ऽꣳह॑स स्पातु वा॒युर् वा॒युः पा॒त्वꣳह॒सो ऽꣳह॑स स्पातु वा॒युः । \newline
55. पा॒तु॒ वा॒युर् वा॒युः पा॑तु पातु वा॒यु र॒न्तरि॑क्षा द॒न्तरि॑क्षाद् वा॒युः पा॑तु पातु वा॒यु र॒न्तरि॑क्षात् । \newline
56. वा॒यु र॒न्तरि॑क्षा द॒न्तरि॑क्षाद् वा॒युर् वा॒यु र॒न्तरि॑क्षा द॒ग्नि र॒ग्नि र॒न्तरि॑क्षाद् वा॒युर् वा॒यु र॒न्तरि॑क्षा द॒ग्निः । \newline
57. अ॒न्तरि॑क्षा द॒ग्नि र॒ग्नि र॒न्तरि॑क्षा द॒न्तरि॑क्षा द॒ग्निः पृ॑थि॒व्याः पृ॑थि॒व्या अ॒ग्नि र॒न्तरि॑क्षा द॒न्तरि॑क्षा द॒ग्निः पृ॑थि॒व्याः । \newline
\pagebreak
\markright{ TS 3.2.4.4  \hfill https://www.vedavms.in \hfill}

\section{ TS 3.2.4.4 }

\textbf{TS 3.2.4.4 } \newline
\textbf{Samhita Paata} \newline

द॒ग्निः पृ॑थि॒व्या य॒मः पि॒तृभ्यः॒ सर॑स्वती मनु॒ष्ये᳚भ्यो॒ देवी᳚ द्वारौ॒ मा मा॒ सं ता᳚प्तं॒ नमः॒ सद॑से॒ नमः॒ सद॑स॒स्पत॑ये॒ नमः॒ सखी॑नां पुरो॒गाणां॒ चक्षु॑षे॒ नमो॑ दि॒वे नमः॑ पृथि॒व्या अहे॑ दैधिष॒व्योदत॑स्तिष्ठा॒ऽन्यस्य॒ सद॑ने सीद॒ यो᳚ऽस्मत् पाक॑तर॒ उन्नि॒वत॒ उदु॒द्वत॑श्च गेषं पा॒तं मा᳚ द्यावापृथिवी अ॒द्याह्नः॒ सदो॒ वै प्र॒सर्प॑न्तं - [  ] \newline

\textbf{Pada Paata} \newline

अ॒ग्निः । पृ॒थि॒व्याः । य॒मः । पि॒तृभ्य॒ इति॑ पि॒तृ - भ्यः॒ । सर॑स्वती । म॒नु॒ष्ये᳚भ्यः । देवी॒ इति॑ । द्वा॒रौ॒ । मा । मा॒ । समिति॑ । ता॒प्त॒म् । नमः॑ । सद॑से । नमः॑ । सद॑सः । पत॑ये । नमः॑ । सखी॑नाम् । पु॒रो॒गाणा॒मिति॑ पुरः - गाना᳚म् । चक्षु॑षे । नमः॑ । दि॒वे । नमः॑ । पृ॒थि॒व्यै । अहे᳚ । दै॒धि॒ष॒व्य॒ । उदिति॑ । अतः॑ । ति॒ष्ठ॒ । अ॒न्यस्य॑ । सद॑ने । सी॒द॒ । यः । अ॒स्मत् । पाक॑तर॒ इति॒ पाक॑ - त॒रः॒ । उदिति॑ । नि॒वत॒ इति॑ नि-वतः॑ । उदिति॑ । उ॒द्वत॒ इत्यु॑त् - वतः॑ । च॒ । गे॒ष॒म् । पा॒तम् । मा॒ । द्या॒वा॒पृ॒थि॒वी॒ इति॑ द्यावा - पृ॒थि॒वी॒ । अ॒द्य । अह्नः॑ । सदः॑ । वै । प्र॒सर्प॑न्त॒मिति॑ प्र - सर्प॑न्तम् ।  \newline


\textbf{Krama Paata} \newline

अ॒ग्निः पृ॑थि॒व्याः । पृ॒थि॒व्या य॒मः । य॒मः पि॒तृभ्यः॑ । पि॒तृभ्यः॒ सर॑स्वती । पि॒तृभ्य॒ इति॑ पि॒तृ - भ्यः॒ । सर॑स्वती मनु॒ष्ये᳚भ्यः । म॒नु॒ष्ये᳚भ्यो॒ देवी᳚ । देवी᳚ द्वारौ । देवी॒ इति॒ देवी᳚ । द्वा॒रौ॒ मा । मा मा᳚ । मा॒ सम् । सम् ता᳚प्तम् । ता॒प्त॒म् नमः॑ । नमः॒ सद॑से । सद॑से॒ नमः॑ । नमः॒ सद॑सः । सद॑स॒स्पत॑ये । पत॑ये॒ नमः॑ । नमः॒ सखी॑नाम् । सखी॑नाम् पुरो॒गाणा᳚म् । पु॒रो॒गाणा॒म् चक्षु॑षे । पु॒रो॒गाणा॒मिति॑ पुरः - गाना᳚म् । चक्षु॑षे॒ नमः॑ । नमो॑ दि॒वे । दि॒वे नमः॑ । नमः॑ पृथि॒व्यै । पृ॒थि॒व्या अहे᳚ । अहे॑ दैधिषव्य । दै॒धि॒॒ष॒व्योत् । उदतः॑ । अत॑स्तिष्ठ । ति॒ष्ठा॒न्यस्य॑ । अ॒न्यस्य॒ सद॑ने । सद॑ने सीद । सी॒द॒ यः । यो᳚ ऽस्मत् । अ॒स्मत् पाक॑तरः । पाक॑तर॒ उत् । पाक॑तर॒ इति॒ पाक॑ - त॒रः॒ । उन् नि॒वतः॑ । नि॒वत॒ उत् । नि॒वत॒ इति॑ नि - वतः॑ । उदु॒द्वतः॑ । उ॒द्वत॑श्च । उ॒द्वत॒ इत्यु॑त् - वतः॑ । च॒ गे॒ष॒म् । गे॒ष॒म् पा॒तम् । पा॒तम् मा᳚ । मा॒ द्या॒वा॒पृ॒थि॒वी॒ । द्या॒वा॒पृ॒थि॒वी॒ अ॒द्य । द्या॒वा॒पृ॒थि॒वी॒ इति॑ द्यावा - पृ॒थि॒वी॒ । अ॒द्याह्नः॑ । अह्नः॒ सदः॑ । सदो॒ वै । वै प्र॒सर्प॑न्तम् ( ) । प्र॒सर्प॑न्तम् पि॒तरः॑ । प्र॒सर्प॑न्त॒मिति॑ प्र - सर्प॑न्तम् \newline

\textbf{Jatai Paata} \newline

1. अ॒ग्निः पृ॑थि॒व्याः पृ॑थि॒व्या अ॒ग्नि र॒ग्निः पृ॑थि॒व्याः । \newline
2. पृ॒थि॒व्या य॒मो य॒मः पृ॑थि॒व्याः पृ॑थि॒व्या य॒मः । \newline
3. य॒मः पि॒तृभ्यः॑ पि॒तृभ्यो॑ य॒मो य॒मः पि॒तृभ्यः॑ । \newline
4. पि॒तृभ्यः॒ सर॑स्वती॒ सर॑स्वती पि॒तृभ्यः॑ पि॒तृभ्यः॒ सर॑स्वती । \newline
5. पि॒तृभ्य॒ इति॑ पि॒तृ - भ्यः॒ । \newline
6. सर॑स्वती मनु॒ष्ये᳚भ्यो मनु॒ष्ये᳚भ्यः॒ सर॑स्वती॒ सर॑स्वती मनु॒ष्ये᳚भ्यः । \newline
7. म॒नु॒ष्ये᳚भ्यो॒ देवी॒ देवी॑ मनु॒ष्ये᳚भ्यो मनु॒ष्ये᳚भ्यो॒ देवी᳚ । \newline
8. देवी᳚ द्वारौ द्वारौ॒ देवी॒ देवी᳚ द्वारौ । \newline
9. देवी॒ इति॒ देवी᳚ । \newline
10. द्वा॒रौ॒ मा मा द्वा॑रौ द्वारौ॒ मा । \newline
11. मा मा॑ मा॒ मा मा मा᳚ । \newline
12. मा॒ सꣳ सम् मा॑ मा॒ सम् । \newline
13. सम् ता᳚प्तम् ताप्तꣳ॒॒ सꣳ सम् ता᳚प्तम् । \newline
14. ता॒प्त॒म् नमो॒ नम॑ स्ताप्तम् ताप्त॒म् नमः॑ । \newline
15. नमः॒ सद॑से॒ सद॑से॒ नमो॒ नमः॒ सद॑से । \newline
16. सद॑से॒ नमो॒ नमः॒ सद॑से॒ सद॑से॒ नमः॑ । \newline
17. नमः॒ सद॑सः॒ सद॑सो॒ नमो॒ नमः॒ सद॑सः । \newline
18. सद॑स॒ स्पत॑ये॒ पत॑ये॒ सद॑सः॒ सद॑स॒ स्पत॑ये । \newline
19. पत॑ये॒ नमो॒ नम॒ स्पत॑ये॒ पत॑ये॒ नमः॑ । \newline
20. नमः॒ सखी॑नाꣳ॒॒ सखी॑ना॒म् नमो॒ नमः॒ सखी॑नाम् । \newline
21. सखी॑नाम् पुरो॒गाणा᳚म् पुरो॒गाणाꣳ॒॒ सखी॑नाꣳ॒॒ सखी॑नाम् पुरो॒गाणा᳚म् । \newline
22. पु॒रो॒गाणा॒म् चक्षु॑षे॒ चक्षु॑षे पुरो॒गाणा᳚म् पुरो॒गाणा॒म् चक्षु॑षे । \newline
23. पु॒रो॒गाणा॒मिति॑ पुरः - गाना᳚म् । \newline
24. चक्षु॑षे॒ नमो॒ नम॒ श्चक्षु॑षे॒ चक्षु॑षे॒ नमः॑ । \newline
25. नमो॑ दि॒वे दि॒वे नमो॒ नमो॑ दि॒वे । \newline
26. दि॒वे नमो॒ नमो॑ दि॒वे दि॒वे नमः॑ । \newline
27. नमः॑ पृथि॒व्यै पृ॑थि॒व्यै नमो॒ नमः॑ पृथि॒व्यै । \newline
28. पृ॒थि॒व्या अहे ऽहे॑ पृथि॒व्यै पृ॑थि॒व्या अहे᳚ । \newline
29. अहे॑ दैधिषव्य दैधिष॒व्या हे ऽहे॑ दैधिषव्य । \newline
30. दै॒धि॒ष॒व्यो दुद् दै॑धिषव्य दैधिष॒व्योत् । \newline
31. उदतो ऽत॒ उदुदतः॑ । \newline
32. अत॑ स्तिष्ठ ति॒ष्ठातो ऽत॑ स्तिष्ठ । \newline
33. ति॒ष्ठा॒ न्यस्या॒ न्यस्य॑ तिष्ठ तिष्ठा॒ न्यस्य॑ । \newline
34. अ॒न्यस्य॒ सद॑ने॒ सद॑ने॒ ऽन्यस्या॒ न्यस्य॒ सद॑ने । \newline
35. सद॑ने सीद सीद॒ सद॑ने॒ सद॑ने सीद । \newline
36. सी॒द॒ यो यः सी॑द सीद॒ यः । \newline
37. यो᳚ ऽस्म द॒स्मद् यो यो᳚ ऽस्मत् । \newline
38. अ॒स्मत् पाक॑तरः॒ पाक॑तरो॒ ऽस्म द॒स्मत् पाक॑तरः । \newline
39. पाक॑तर॒ उदुत् पाक॑तरः॒ पाक॑तर॒ उत् । \newline
40. पाक॑तर॒ इति॒ पाक॑ - त॒रः॒ । \newline
41. उन् नि॒वतो॑ नि॒वत॒ उदुन् नि॒वतः॑ । \newline
42. नि॒वत॒ उदुन् नि॒वतो॑ नि॒वत॒ उत् । \newline
43. नि॒वत॒ इति॑ नि-वतः॑ । \newline
44. उदु॒द्वत॑ उ॒द्वत॒ उदुदु॒द्वतः॑ । \newline
45. उ॒द्वत॑श्च चो॒द्वत॑ उ॒द्वत॑श्च । \newline
46. उ॒द्वत॒ इत्यु॑त् - वतः॑ । \newline
47. च॒ गे॒ष॒म् गे॒ष॒म् च॒ च॒ गे॒ष॒म् । \newline
48. गे॒ष॒म् पा॒तम् पा॒तम् गे॑षम् गेषम् पा॒तम् । \newline
49. पा॒तम् मा॑ मा पा॒तम् पा॒तम् मा᳚ । \newline
50. मा॒ द्या॒वा॒पृ॒थि॒वी॒ द्या॒वा॒पृ॒थि॒वी॒ मा॒ मा॒ द्या॒वा॒पृ॒थि॒वी॒ । \newline
51. द्या॒वा॒पृ॒थि॒वी॒ अ॒द्याद्य द्या॑वापृथिवी द्यावापृथिवी अ॒द्य । \newline
52. द्या॒वा॒पृ॒थि॒वी॒ इति॑ द्यावा - पृ॒थि॒वी॒ । \newline
53. अ॒द्या ह्नो ऽह्नो॒ ऽद्या द्याह्नः॑ । \newline
54. अह्नः॒ सदः॒ सदो ऽह्नो ऽह्नः॒ सदः॑ । \newline
55. सदो॒ वै वै सदः॒ सदो॒ वै । \newline
56. वै प्र॒सर्प॑न्तम् प्र॒सर्प॑न्तं॒ ॅवै वै प्र॒सर्प॑न्तम् । \newline
57. प्र॒सर्प॑न्तम् पि॒तरः॑ पि॒तरः॑ प्र॒सर्प॑न्तम् प्र॒सर्प॑न्तम् पि॒तरः॑ । \newline
58. प्र॒सर्प॑न्त॒मिति॑ प्र - सर्प॑न्तम् । \newline

\textbf{Ghana Paata } \newline

1. अ॒ग्निः पृ॑थि॒व्याः पृ॑थि॒व्या अ॒ग्नि र॒ग्निः पृ॑थि॒व्या य॒मो य॒मः पृ॑थि॒व्या अ॒ग्नि र॒ग्निः पृ॑थि॒व्या य॒मः । \newline
2. पृ॒थि॒व्या य॒मो य॒मः पृ॑थि॒व्याः पृ॑थि॒व्या य॒मः पि॒तृभ्यः॑ पि॒तृभ्यो॑ य॒मः पृ॑थि॒व्याः पृ॑थि॒व्या य॒मः पि॒तृभ्यः॑ । \newline
3. य॒मः पि॒तृभ्यः॑ पि॒तृभ्यो॑ य॒मो य॒मः पि॒तृभ्यः॒ सर॑स्वती॒ सर॑स्वती पि॒तृभ्यो॑ य॒मो य॒मः पि॒तृभ्यः॒ सर॑स्वती । \newline
4. पि॒तृभ्यः॒ सर॑स्वती॒ सर॑स्वती पि॒तृभ्यः॑ पि॒तृभ्यः॒ सर॑स्वती मनु॒ष्ये᳚भ्यो मनु॒ष्ये᳚भ्यः॒ सर॑स्वती पि॒तृभ्यः॑ पि॒तृभ्यः॒ सर॑स्वती मनु॒ष्ये᳚भ्यः । \newline
5. पि॒तृभ्य॒ इति॑ पि॒तृ - भ्यः॒ । \newline
6. सर॑स्वती मनु॒ष्ये᳚भ्यो मनु॒ष्ये᳚भ्यः॒ सर॑स्वती॒ सर॑स्वती मनु॒ष्ये᳚भ्यो॒ देवी॒ देवी॑ मनु॒ष्ये᳚भ्यः॒ सर॑स्वती॒ सर॑स्वती मनु॒ष्ये᳚भ्यो॒ देवी᳚ । \newline
7. म॒नु॒ष्ये᳚भ्यो॒ देवी॒ देवी॑ मनु॒ष्ये᳚भ्यो मनु॒ष्ये᳚भ्यो॒ देवी᳚ द्वारौ द्वारौ॒ देवी॑ मनु॒ष्ये᳚भ्यो मनु॒ष्ये᳚भ्यो॒ देवी᳚ द्वारौ । \newline
8. देवी᳚ द्वारौ द्वारौ॒ देवी॒ देवी᳚ द्वारौ॒ मा मा द्वा॑रौ॒ देवी॒ देवी᳚ द्वारौ॒ मा । \newline
9. देवी॒ इति॒ देवी᳚ । \newline
10. द्वा॒रौ॒ मा मा द्वा॑रौ द्वारौ॒ मा मा॑ मा॒ मा द्वा॑रौ द्वारौ॒ मा मा᳚ । \newline
11. मा मा॑ मा॒ मा मा मा॒ सꣳ सम् मा॒ मा मा मा॒ सम् । \newline
12. मा॒ सꣳ सम् मा॑ मा॒ सम् ता᳚प्तम् ताप्तꣳ॒॒ सम् मा॑ मा॒ सम् ता᳚प्तम् । \newline
13. सम् ता᳚प्तम् ताप्तꣳ॒॒ सꣳ सम् ता᳚प्त॒म् नमो॒ नम॑ स्ताप्तꣳ॒॒ सꣳ सम् ता᳚प्त॒म् नमः॑ । \newline
14. ता॒प्त॒म् नमो॒ नम॑ स्ताप्तम् ताप्त॒म् नमः॒ सद॑से॒ सद॑से॒ नम॑ स्ताप्तम् ताप्त॒म् नमः॒ सद॑से । \newline
15. नमः॒ सद॑से॒ सद॑से॒ नमो॒ नमः॒ सद॑से॒ नमो॒ नमः॒ सद॑से॒ नमो॒ नमः॒ सद॑से॒ नमः॑ । \newline
16. सद॑से॒ नमो॒ नमः॒ सद॑से॒ सद॑से॒ नमः॒ सद॑सः॒ सद॑सो॒ नमः॒ सद॑से॒ सद॑से॒ नमः॒ सद॑सः । \newline
17. नमः॒ सद॑सः॒ सद॑सो॒ नमो॒ नमः॒ सद॑स॒ स्पत॑ये॒ पत॑ये॒ सद॑सो॒ नमो॒ नमः॒ सद॑स॒ स्पत॑ये । \newline
18. सद॑स॒ स्पत॑ये॒ पत॑ये॒ सद॑सः॒ सद॑स॒ स्पत॑ये॒ नमो॒ नम॒ स्पत॑ये॒ सद॑सः॒ सद॑स॒ स्पत॑ये॒ नमः॑ । \newline
19. पत॑ये॒ नमो॒ नम॒ स्पत॑ये॒ पत॑ये॒ नमः॒ सखी॑नाꣳ॒॒ सखी॑ना॒म् नम॒ स्पत॑ये॒ पत॑ये॒ नमः॒ सखी॑नाम् । \newline
20. नमः॒ सखी॑नाꣳ॒॒ सखी॑ना॒म् नमो॒ नमः॒ सखी॑नाम् पुरो॒गाणा᳚म् पुरो॒गाणाꣳ॒॒ सखी॑ना॒म् नमो॒ नमः॒ सखी॑नाम् पुरो॒गाणा᳚म् । \newline
21. सखी॑नाम् पुरो॒गाणा᳚म् पुरो॒गाणाꣳ॒॒ सखी॑नाꣳ॒॒ सखी॑नाम् पुरो॒गाणा॒म् चक्षु॑षे॒ चक्षु॑षे पुरो॒गाणाꣳ॒॒ सखी॑नाꣳ॒॒ सखी॑नाम् पुरो॒गाणा॒म् चक्षु॑षे । \newline
22. पु॒रो॒गाणा॒म् चक्षु॑षे॒ चक्षु॑षे पुरो॒गाणा᳚म् पुरो॒गाणा॒म् चक्षु॑षे॒ नमो॒ नम॒ श्चक्षु॑षे पुरो॒गाणा᳚म् पुरो॒गाणा॒म् चक्षु॑षे॒ नमः॑ । \newline
23. पु॒रो॒गाणा॒मिति॑ पुरः - गाना᳚म् । \newline
24. चक्षु॑षे॒ नमो॒ नम॒ श्चक्षु॑षे॒ चक्षु॑षे॒ नमो॑ दि॒वे दि॒वे नम॒ श्चक्षु॑षे॒ चक्षु॑षे॒ नमो॑ दि॒वे । \newline
25. नमो॑ दि॒वे दि॒वे नमो॒ नमो॑ दि॒वे नमो॒ नमो॑ दि॒वे नमो॒ नमो॑ दि॒वे नमः॑ । \newline
26. दि॒वे नमो॒ नमो॑ दि॒वे दि॒वे नमः॑ पृथि॒व्यै पृ॑थि॒व्यै नमो॑ दि॒वे दि॒वे नमः॑ पृथि॒व्यै । \newline
27. नमः॑ पृथि॒व्यै पृ॑थि॒व्यै नमो॒ नमः॑ पृथि॒व्या अहे ऽहे॑ पृथि॒व्यै नमो॒ नमः॑ पृथि॒व्या अहे᳚ । \newline
28. पृ॒थि॒व्या अहे ऽहे॑ पृथि॒व्यै पृ॑थि॒व्या अहे॑ दैधिषव्य दैधिष॒व्याहे॑ पृथि॒व्यै पृ॑थि॒व्या अहे॑ दैधिषव्य । \newline
29. अहे॑ दैधिषव्य दैधिष॒व्याहे ऽहे॑ दैधिष॒व्यो दुद् दै॑धिष॒व्याहे ऽहे॑ दैधिष॒व्योत् । \newline
30. दै॒धि॒ष॒व्यो दुद् दै॑धिषव्य दैधिष॒व्यो दतो ऽत॒ उद् दै॑धिषव्य दैधिष॒व्यो दतः॑ । \newline
31. उदतो ऽत॒ उदु दत॑ स्तिष्ठ ति॒ष्ठात॒ उदु दत॑ स्तिष्ठ । \newline
32. अत॑ स्तिष्ठ ति॒ष्ठातो ऽत॑ स्तिष्ठा॒ न्यस्या॒ न्यस्य॑ ति॒ष्ठातो ऽत॑ स्तिष्ठा॒ न्यस्य॑ । \newline
33. ति॒ष्ठा॒ न्यस्या॒ न्यस्य॑ तिष्ठ तिष्ठा॒ न्यस्य॒ सद॑ने॒ सद॑ने॒ ऽन्यस्य॑ तिष्ठ तिष्ठा॒ न्यस्य॒ सद॑ने । \newline
34. अ॒न्यस्य॒ सद॑ने॒ सद॑ने॒ ऽन्यस्या॒ न्यस्य॒ सद॑ने सीद सीद॒ सद॑ने॒ ऽन्यस्या॒ न्यस्य॒ सद॑ने सीद । \newline
35. सद॑ने सीद सीद॒ सद॑ने॒ सद॑ने सीद॒ यो यः सी॑द॒ सद॑ने॒ सद॑ने सीद॒ यः । \newline
36. सी॒द॒ यो यः सी॑द सीद॒ यो᳚ ऽस्म द॒स्मद् यः सी॑द सीद॒ यो᳚ ऽस्मत् । \newline
37. यो᳚ ऽस्म द॒स्मद् यो यो᳚ ऽस्मत् पाक॑तरः॒ पाक॑तरो॒ ऽस्मद् यो यो᳚ ऽस्मत् पाक॑तरः । \newline
38. अ॒स्मत् पाक॑तरः॒ पाक॑तरो॒ ऽस्म द॒स्मत् पाक॑तर॒ उदुत् पाक॑तरो॒ ऽस्म द॒स्मत् पाक॑तर॒ उत् । \newline
39. पाक॑तर॒ उदुत् पाक॑तरः॒ पाक॑तर॒ उन् नि॒वतो॑ नि॒वत॒ उत् पाक॑तरः॒ पाक॑तर॒ उन् नि॒वतः॑ । \newline
40. पाक॑तर॒ इति॒ पाक॑ - त॒रः॒ । \newline
41. उन् नि॒वतो॑ नि॒वत॒ उदुन् नि॒वत॒ उदुन् नि॒वत॒ उदुन् नि॒वत॒ उत् । \newline
42. नि॒वत॒ उदुन् नि॒वतो॑ नि॒वत॒ उदु॒द्वत॑ उ॒द्वत॒ उन् नि॒वतो॑ नि॒वत॒ उदु॒द्वतः॑ । \newline
43. नि॒वत॒ इति॑ नि - वतः॑ । \newline
44. उदु॒ द्वत॑ उ॒द्वत॒ उदु दु॒द्वत॑श्च चो॒द्वत॒ उदु दु॒द्वत॑श्च । \newline
45. उ॒द्वत॑श्च चो॒द्वत॑ उ॒द्वत॑श्च गेषम् गेषम् चो॒द्वत॑ उ॒द्वत॑श्च गेषम् । \newline
46. उ॒द्वत॒ इत्यु॑त् - वतः॑ । \newline
47. च॒ गे॒ष॒म् गे॒ष॒म् च॒ च॒ गे॒ष॒म् पा॒तम् पा॒तम् गे॑षम् च च गेषम् पा॒तम् । \newline
48. गे॒ष॒म् पा॒तम् पा॒तम् गे॑षम् गेषम् पा॒तम् मा॑ मा पा॒तम् गे॑षम् गेषम् पा॒तम् मा᳚ । \newline
49. पा॒तम् मा॑ मा पा॒तम् पा॒तम् मा᳚ द्यावापृथिवी द्यावापृथिवी मा पा॒तम् पा॒तम् मा᳚ द्यावापृथिवी । \newline
50. मा॒ द्या॒वा॒पृ॒थि॒वी॒ द्या॒वा॒पृ॒थि॒वी॒ मा॒ मा॒ द्या॒वा॒पृ॒थि॒वी॒ अ॒द्याद्य द्या॑वापृथिवी मा मा द्यावापृथिवी अ॒द्य । \newline
51. द्या॒वा॒पृ॒थि॒वी॒ अ॒द्याद्य द्या॑वापृथिवी द्यावापृथिवी अ॒द्याह्नो ऽह्नो॒ ऽद्य द्या॑वापृथिवी द्यावापृथिवी अ॒द्याह्नः॑ । \newline
52. द्या॒वा॒पृ॒थि॒वी॒ इति॑ द्यावा - पृ॒थि॒वी॒ । \newline
53. अ॒द्याह्नो ऽह्नो॒ ऽद्या द्याह्नः॒ सदः॒ सदो ऽह्नो॒ ऽद्या द्याह्नः॒ सदः॑ । \newline
54. अह्नः॒ सदः॒ सदो ऽह्नो ऽह्नः॒ सदो॒ वै वै सदो ऽह्नो ऽह्नः॒ सदो॒ वै । \newline
55. सदो॒ वै वै सदः॒ सदो॒ वै प्र॒सर्प॑न्तम् प्र॒सर्प॑न्तं॒ ॅवै सदः॒ सदो॒ वै प्र॒सर्प॑न्तम् । \newline
56. वै प्र॒सर्प॑न्तम् प्र॒सर्प॑न्तं॒ ॅवै वै प्र॒सर्प॑न्तम् पि॒तरः॑ पि॒तरः॑ प्र॒सर्प॑न्तं॒ ॅवै वै प्र॒सर्प॑न्तम् पि॒तरः॑ । \newline
57. प्र॒सर्प॑न्तम् पि॒तरः॑ पि॒तरः॑ प्र॒सर्प॑न्तम् प्र॒सर्प॑न्तम् पि॒तरो ऽन्वनु॑ पि॒तरः॑ प्र॒सर्प॑न्तम् प्र॒सर्प॑न्तम् पि॒तरो ऽनु॑ । \newline
58. प्र॒सर्प॑न्त॒मिति॑ प्र - सर्प॑न्तम् । \newline
\pagebreak
\markright{ TS 3.2.4.5  \hfill https://www.vedavms.in \hfill}

\section{ TS 3.2.4.5 }

\textbf{TS 3.2.4.5 } \newline
\textbf{Samhita Paata} \newline

पि॒तरोऽनु॒ प्रस॑र्पन्ति॒ त ए॑नमीश्व॒रा हिꣳसि॑तोः॒ सदः॑ प्र॒सृप्य॑ दक्षिणा॒र्द्धं परे᳚क्षे॒ताऽग॑न्त पितरः पितृ॒मान॒हं ॅयु॒ष्माभि॑र्भूयासꣳ सुप्र॒जसो॒ मया॑ यू॒यं भू॑या॒स्तेति॒ तेभ्य॑ ए॒व न॑म॒स्कृत्य॒ सदः॒ प्रस॑र्पत्या॒त्मनोऽना᳚र्त्यै ॥ \newline

\textbf{Pada Paata} \newline

पि॒तरः॑ । अनु॑ । प्रेति॑ । स॒र्प॒न्ति॒ । ते । ए॒न॒म् । ई॒श्व॒राः । हिꣳसि॑तोः । सदः॑ । प्र॒सृप्येति॑ प्र - सृप्य॑ । द॒क्षि॒णा॒र्द्धमिति॑ दक्षिण - अ॒र्द्धम् । परेति॑ । ई॒क्षे॒त॒ । एति॑ । अ॒ग॒न्त॒ । पि॒त॒रः॒ । पि॒तृ॒मानिति॑ पितृ - मान् । अ॒हम् । यु॒ष्माभिः॑ । भू॒या॒स॒म् । सु॒प्र॒जस॒ इति॑ सु-प्र॒जसः॑ । मया᳚ । यू॒यम् । भू॒या॒स्त॒ । इति॑ । तेभ्यः॑ । ए॒व । न॒म॒स्कृत्येति॑ नमः-कृत्य॑ । सदः॑ । प्रेति॑ । स॒र्प॒ति॒ । आ॒त्मनः॑ । अना᳚र्त्यै ॥  \newline


\textbf{Krama Paata} \newline

पि॒तरो ऽनु॑ । अनु॒ प्र । प्र स॑र्पन्ति । स॒र्प॒न्ति॒ ते । त ए॑नम् । ए॒न॒मी॒श्व॒राः । ई॒श्व॒रा हिꣳसि॑तोः । हिꣳसि॑तोः॒ सदः॑ । सदः॑ प्र॒सृप्य॑ । प्र॒सृप्य॑ दक्षिणा॒र्द्धम् । प्र॒सृप्येति॑ प्र - सृप्य॑ । द॒क्षि॒णा॒र्द्धम् परा᳚ । द॒क्षि॒णा॒र्द्धमिति॑ दक्षिण - अ॒र्द्धम् । परे᳚क्षेत । ई॒क्षे॒ता । आ ऽग॑न्त । अ॒ग॒न्त॒ पि॒त॒रः॒ । पि॒त॒रः॒ पि॒तृ॒मान् । पि॒तृ॒मान॒हम् । पि॒तृ॒मानिति॑ पितृ - मान् । अ॒हं ॅयु॒ष्माभिः॑ । यु॒ष्माभि॑र् भूयासम् । भू॒या॒सꣳ॒॒ सु॒प्र॒जसः॑ । सु॒प्र॒जसो॒ मया᳚ । सु॒प्र॒जस॒ इति॑ सु - प्र॒जसः॑ । मया॑ यू॒यम् । यू॒यम् भू॑यास्त । भू॒या॒स्तेति॑ । इति॒ तेभ्यः॑ । तेभ्य॑ ए॒व । ए॒व न॑म॒स्कृत्य॑ । न॒म॒स्कृत्य॒ सदः॑ । न॒म॒स्कृत्येति॑ नमः - कृत्य॑ । सदः॒ प्र । प्र स॑र्पति । स॒र्प॒त्या॒त्मनः॑ । आ॒त्मनो ऽना᳚र्त्यै । अना᳚र्त्या॒ इत्यना᳚र्त्यै । \newline

\textbf{Jatai Paata} \newline

1. पि॒तरो ऽन्वनु॑ पि॒तरः॑ पि॒तरो ऽनु॑ । \newline
2. अनु॒ प्र प्राण्वनु॒ प्र । \newline
3. प्र स॑र्पन्ति सर्पन्ति॒ प्र प्र स॑र्पन्ति । \newline
4. स॒र्प॒न्ति॒ ते ते स॑र्पन्ति सर्पन्ति॒ ते । \newline
5. त ए॑न मेन॒म् ते त ए॑नम् । \newline
6. ए॒न॒ मी॒श्व॒रा ई᳚श्व॒रा ए॑न मेन मीश्व॒राः । \newline
7. ई॒श्व॒रा हिꣳसि॑तो॒र्॒. हिꣳसि॑तो रीश्व॒रा ई᳚श्व॒रा हिꣳसि॑तोः । \newline
8. हिꣳसि॑तोः॒ सदः॒ सदो॒ हिꣳसि॑तो॒र्॒. हिꣳसि॑तोः॒ सदः॑ । \newline
9. सदः॑ प्र॒सृप्य॑ प्र॒सृप्य॒ सदः॒ सदः॑ प्र॒सृप्य॑ । \newline
10. प्र॒सृप्य॑ दक्षिणा॒र्द्धम् द॑क्षिणा॒र्द्धम् प्र॒सृप्य॑ प्र॒सृप्य॑ दक्षिणा॒र्द्धम् । \newline
11. प्र॒सृप्येति॑ प्र - सृप्य॑ । \newline
12. द॒क्षि॒णा॒र्द्धम् परा॒ परा॑ दक्षिणा॒र्द्धम् द॑क्षिणा॒र्द्धम् परा᳚ । \newline
13. द॒क्षि॒णा॒र्द्धमिति॑ दक्षिण - अ॒र्द्धम् । \newline
14. परे᳚क्षे तेक्षेत॒ परा॒ परे᳚क्षेत । \newline
15. ई॒क्षे॒ तेक्षे॑ तेक्षे॒ता । \newline
16. आ ऽग॑न्ता ग॒न्ता ऽग॑न्त । \newline
17. अ॒ग॒न्त॒ पि॒त॒रः॒ पि॒त॒रो॒ ऽग॒न्ता॒ ग॒न्त॒ पि॒त॒रः॒ । \newline
18. पि॒त॒रः॒ पि॒तृ॒मान् पि॑तृ॒मान् पि॑तरः पितरः पितृ॒मान् । \newline
19. पि॒तृ॒मा न॒ह म॒हम् पि॑तृ॒मान् पि॑तृ॒मा न॒हम् । \newline
20. पि॒तृ॒मानिति॑ पितृ - मान् । \newline
21. अ॒हं ॅयु॒ष्माभि॑र् यु॒ष्माभि॑ र॒ह म॒हं ॅयु॒ष्माभिः॑ । \newline
22. यु॒ष्माभि॑र् भूयासम् भूयासं ॅयु॒ष्माभि॑र् यु॒ष्माभि॑र् भूयासम् । \newline
23. भू॒या॒सꣳ॒॒ सु॒प्र॒जसः॑ सुप्र॒जसो॑ भूयासम् भूयासꣳ सुप्र॒जसः॑ । \newline
24. सु॒प्र॒जसो॒ मया॒ मया॑ सुप्र॒जसः॑ सुप्र॒जसो॒ मया᳚ । \newline
25. सु॒प्र॒जस॒ इति॑ सु - प्र॒जसः॑ । \newline
26. मया॑ यू॒यं ॅयू॒यम् मया॒ मया॑ यू॒यम् । \newline
27. यू॒यम् भू॑यास्त भूयास्त यू॒यं ॅयू॒यम् भू॑यास्त । \newline
28. भू॒या॒स्ते तीति॑ भूयास्त भूया॒स्ते ति॑ । \newline
29. इति॒ तेभ्य॒ स्तेभ्य॒ इतीति॒ तेभ्यः॑ । \newline
30. तेभ्य॑ ए॒वैव तेभ्य॒ स्तेभ्य॑ ए॒व । \newline
31. ए॒व न॑म॒स्कृत्य॑ नम॒स्कृ त्यै॒वैव न॑म॒स्कृत्य॑ । \newline
32. न॒म॒स्कृत्य॒ सदः॒ सदो॑ नम॒स्कृत्य॑ नम॒स्कृत्य॒ सदः॑ । \newline
33. न॒म॒स्कृत्येति॑ नमः - कृत्य॑ । \newline
34. सदः॒ प्र प्र सदः॒ सदः॒ प्र । \newline
35. प्र स॑र्पति सर्पति॒ प्र प्र स॑र्पति । \newline
36. स॒र्प॒ त्या॒त्मन॑ आ॒त्मनः॑ सर्पति सर्प त्या॒त्मनः॑ । \newline
37. आ॒त्मनो ऽना᳚र्त्या॒ अना᳚र्त्या आ॒त्मन॑ आ॒त्मनो ऽना᳚र्त्यै । \newline
38. अना᳚र्त्या॒ इत्यना᳚र्त्यै । \newline

\textbf{Ghana Paata } \newline

1. पि॒तरो ऽन्वनु॑ पि॒तरः॑ पि॒तरो ऽनु॒ प्र प्राणु॑ पि॒तरः॑ पि॒तरो ऽनु॒ प्र । \newline
2. अनु॒ प्र प्राण्वनु॒ प्र स॑र्पन्ति सर्पन्ति॒ प्राण्वनु॒ प्र स॑र्पन्ति । \newline
3. प्र स॑र्पन्ति सर्पन्ति॒ प्र प्र स॑र्पन्ति॒ ते ते स॑र्पन्ति॒ प्र प्र स॑र्पन्ति॒ ते । \newline
4. स॒र्प॒न्ति॒ ते ते स॑र्पन्ति सर्पन्ति॒ त ए॑न मेन॒म् ते स॑र्पन्ति सर्पन्ति॒ त ए॑नम् । \newline
5. त ए॑न मेन॒म् ते त ए॑न मीश्व॒रा ई᳚श्व॒रा ए॑न॒म् ते त ए॑न मीश्व॒राः । \newline
6. ए॒न॒ मी॒श्व॒रा ई᳚श्व॒रा ए॑न मेन मीश्व॒रा हिꣳसि॑तो॒र्॒. हिꣳसि॑तो रीश्व॒रा ए॑न मेन मीश्व॒रा हिꣳसि॑तोः । \newline
7. ई॒श्व॒रा हिꣳसि॑तो॒र्॒. हिꣳसि॑तो रीश्व॒रा ई᳚श्व॒रा हिꣳसि॑तोः॒ सदः॒ सदो॒ हिꣳसि॑तो रीश्व॒रा ई᳚श्व॒रा हिꣳसि॑तोः॒ सदः॑ । \newline
8. हिꣳसि॑तोः॒ सदः॒ सदो॒ हिꣳसि॑तो॒र्॒. हिꣳसि॑तोः॒ सदः॑ प्र॒सृप्य॑ प्र॒सृप्य॒ सदो॒ हिꣳसि॑तो॒र्॒. हिꣳसि॑तोः॒ सदः॑ प्र॒सृप्य॑ । \newline
9. सदः॑ प्र॒सृप्य॑ प्र॒सृप्य॒ सदः॒ सदः॑ प्र॒सृप्य॑ दक्षिणा॒र्द्धम् द॑क्षिणा॒र्द्धम् प्र॒सृप्य॒ सदः॒ सदः॑ प्र॒सृप्य॑ दक्षिणा॒र्द्धम् । \newline
10. प्र॒सृप्य॑ दक्षिणा॒र्द्धम् द॑क्षिणा॒र्द्धम् प्र॒सृप्य॑ प्र॒सृप्य॑ दक्षिणा॒र्द्धम् परा॒ परा॑ दक्षिणा॒र्द्धम् प्र॒सृप्य॑ प्र॒सृप्य॑ दक्षिणा॒र्द्धम् परा᳚ । \newline
11. प्र॒सृप्येति॑ प्र - सृप्य॑ । \newline
12. द॒क्षि॒णा॒र्द्धम् परा॒ परा॑ दक्षिणा॒र्द्धम् द॑क्षिणा॒र्द्धम् परे᳚ क्षेते क्षेत॒ परा॑ दक्षिणा॒र्द्धम् द॑क्षिणा॒र्द्धम् परे᳚क्षेत । \newline
13. द॒क्षि॒णा॒र्द्धमिति॑ दक्षिण - अ॒र्द्धम् । \newline
14. परे᳚ क्षेते क्षेत॒ परा॒ परे᳚ क्षे॒ते क्षे॑त॒ परा॒ परे᳚क्षे॒ता । \newline
15. ई॒क्षे॒ तेक्षे॑ तेक्षे॒ता ऽग॑न्ता ग॒न्तेक्षे॑ तेक्षे॒ता ऽग॑न्त । \newline
16. आ ऽग॑न्ता ग॒न्ता ऽग॑न्त पितरः पितरो ऽग॒न्ता ऽग॑न्त पितरः । \newline
17. अ॒ग॒न्त॒ पि॒त॒रः॒ पि॒त॒रो॒ ऽग॒न्ता॒ ग॒न्त॒ पि॒त॒रः॒ पि॒तृ॒मान् पि॑तृ॒मान् पि॑तरो ऽगन्ता गन्त पितरः पितृ॒मान् । \newline
18. पि॒त॒रः॒ पि॒तृ॒मान् पि॑तृ॒मान् पि॑तरः पितरः पितृ॒मा न॒ह म॒हम् पि॑तृ॒मान् पि॑तरः पितरः पितृ॒मा न॒हम् । \newline
19. पि॒तृ॒मा न॒ह म॒हम् पि॑तृ॒मान् पि॑तृ॒मा न॒हं ॅयु॒ष्माभि॑र् यु॒ष्माभि॑ र॒हम् पि॑तृ॒मान् पि॑तृ॒मा न॒हं ॅयु॒ष्माभिः॑ । \newline
20. पि॒तृ॒मानिति॑ पितृ - मान् । \newline
21. अ॒हं ॅयु॒ष्माभि॑र् यु॒ष्माभि॑ र॒ह म॒हं ॅयु॒ष्माभि॑र् भूयासम् भूयासं ॅयु॒ष्माभि॑ र॒ह म॒हं ॅयु॒ष्माभि॑र् भूयासम् । \newline
22. यु॒ष्माभि॑र् भूयासम् भूयासं ॅयु॒ष्माभि॑र् यु॒ष्माभि॑र् भूयासꣳ सुप्र॒जसः॑ सुप्र॒जसो॑ भूयासं ॅयु॒ष्माभि॑र् यु॒ष्माभि॑र् भूयासꣳ सुप्र॒जसः॑ । \newline
23. भू॒या॒सꣳ॒॒ सु॒प्र॒जसः॑ सुप्र॒जसो॑ भूयासम् भूयासꣳ सुप्र॒जसो॒ मया॒ मया॑ सुप्र॒जसो॑ भूयासम् भूयासꣳ सुप्र॒जसो॒ मया᳚ । \newline
24. सु॒प्र॒जसो॒ मया॒ मया॑ सुप्र॒जसः॑ सुप्र॒जसो॒ मया॑ यू॒यं ॅयू॒यम् मया॑ सुप्र॒जसः॑ सुप्र॒जसो॒ मया॑ यू॒यम् । \newline
25. सु॒प्र॒जस॒ इति॑ सु - प्र॒जसः॑ । \newline
26. मया॑ यू॒यं ॅयू॒यम् मया॒ मया॑ यू॒यम् भू॑यास्त भूयास्त यू॒यम् मया॒ मया॑ यू॒यम् भू॑यास्त । \newline
27. यू॒यम् भू॑यास्त भूयास्त यू॒यं ॅयू॒यम् भू॑या॒स्ते तीति॑ भूयास्त यू॒यं ॅयू॒यम् भू॑या॒स्ते ति॑ । \newline
28. भू॒या॒स्ते तीति॑ भूयास्त भूया॒स्ते ति॒ तेभ्य॒ स्तेभ्य॒ इति॑ भूयास्त भूया॒स्ते ति॒ तेभ्यः॑ । \newline
29. इति॒ तेभ्य॒ स्तेभ्य॒ इतीति॒ तेभ्य॑ ए॒वैव तेभ्य॒ इतीति॒ तेभ्य॑ ए॒व । \newline
30. तेभ्य॑ ए॒वैव तेभ्य॒ स्तेभ्य॑ ए॒व न॑म॒स्कृत्य॑ नम॒स्कृ???त्यै॒व तेभ्य॒ स्तेभ्य॑ ए॒व न॑म॒स्कृत्य॑ । \newline
31. ए॒व न॑म॒स्कृत्य॑ नम॒स्कृ त्यै॒वैव न॑म॒स्कृत्य॒ सदः॒ सदो॑ नम॒स्कृ त्यै॒वैव न॑म॒स्कृत्य॒ सदः॑ । \newline
32. न॒म॒स्कृत्य॒ सदः॒ सदो॑ नम॒स्कृत्य॑ नम॒स्कृत्य॒ सदः॒ प्र प्र सदो॑ नम॒स्कृत्य॑ नम॒स्कृत्य॒ सदः॒ प्र । \newline
33. न॒म॒स्कृत्येति॑ नमः - कृत्य॑ । \newline
34. सदः॒ प्र प्र सदः॒ सदः॒ प्र स॑र्पति सर्पति॒ प्र सदः॒ सदः॒ प्र स॑र्पति । \newline
35. प्र स॑र्पति सर्पति॒ प्र प्र स॑र्प त्या॒त्मन॑ आ॒त्मनः॑ सर्पति॒ प्र प्र स॑र्प त्या॒त्मनः॑ । \newline
36. स॒र्प॒ त्या॒त्मन॑ आ॒त्मनः॑ सर्पति सर्प त्या॒त्मनो ऽना᳚र्त्या॒ अना᳚र्त्या आ॒त्मनः॑ सर्पति सर्प त्या॒त्मनो ऽना᳚र्त्यै । \newline
37. आ॒त्मनो ऽना᳚र्त्या॒ अना᳚र्त्या आ॒त्मन॑ आ॒त्मनो ऽना᳚र्त्यै । \newline
38. अना᳚र्त्या॒ इत्यना᳚र्त्यै । \newline
\pagebreak
\markright{ TS 3.2.5.1  \hfill https://www.vedavms.in \hfill}

\section{ TS 3.2.5.1 }

\textbf{TS 3.2.5.1 } \newline
\textbf{Samhita Paata} \newline

भक्षेहि॒ मा ऽऽवि॑श दीर्घायु॒त्वाय॑ शन्तनु॒त्वाय॑ रा॒यस्पोषा॑य॒ वर्च॑से सुप्रजा॒स्त्वायेहि॑ वसो पुरो वसो प्रि॒यो मे॑ हृ॒दो᳚ऽस्य॒श्विनो᳚स्त्वा बा॒हुभ्याꣳ॑ सघ्यासं नृ॒चक्ष॑सं त्वा देव सोम सु॒चक्षा॒ अव॑ ख्येषं म॒न्द्राऽभिभू॑तिः के॒तुर्य॒ज्ञानां॒ ॅवाग्जु॑षा॒णा सोम॑स्य तृप्यतु म॒न्द्रा स्व॑र्वा॒च्यदि॑ति॒रना॑हत शीर्ष्णी॒ वाग्जु॑षा॒णा सोम॑स्य तृप्य॒त्वेहि॑ विश्वचर्.षणे - [  ] \newline

\textbf{Pada Paata} \newline

भक्ष॑ । एति॑ । इ॒हि॒ । मा॒ । एति॑ । वि॒श॒ । दी॒र्घा॒यु॒त्वायेति॑ दीर्घायु - त्वाय॑ । श॒न्त॒नु॒त्वायेति॑ शन्तनु - त्वाय॑ । रा॒यः । पोषा॑य । वर्च॑से । सु॒प्र॒जा॒स्त्वायेति॑ सुप्रजाः-त्वाय॑ । एति॑ । इ॒हि॒ । व॒सो॒ इति॑ । पु॒रो॒व॒सो॒ इति॑ पुरः - व॒सो॒ । प्रि॒यः । मे॒ । हृ॒दः । अ॒सि॒ । अ॒श्विनोः᳚ । त्वा॒ । बा॒हुभ्या॒मिति॑ बा॒हु- भ्या॒म् । स॒घ्या॒स॒म् । नृ॒चक्ष॑स॒मिति॑ नृ - चक्ष॑सम् । त्वा॒ । दे॒व॒ । सो॒म॒ । सु॒चक्षा॒ इति॑ सु - चक्षाः᳚ । अवेति॑ । ख्ये॒ष॒म् । म॒न्द्रा । अ॒भिभू॑ति॒रित्य॒भि - भू॒तिः॒ । के॒तुः । य॒ज्ञाना᳚म् । वाक् । जु॒षा॒णा । सोम॑स्य । तृ॒प्य॒तु॒ । म॒न्द्रा । स्व॑र्वा॒चीति॒ सु - अ॒र्वा॒ची॒ । अदि॑तिः । अना॑हतशी॒र्ष्णीत्यना॑हत - शी॒र्ष्णी॒ । वाक् । जु॒षा॒णा । सोम॑स्य । तृ॒प्य॒तु॒ । एति॑ । इ॒हि॒ । वि॒श्व॒च॒र्॒.ष॒ण॒ इति॑ विश्व - च॒र्॒.ष॒णे॒ ।  \newline


\textbf{Krama Paata} \newline

भक्षा । एहि॑ । इ॒हि॒ मा॒ । मा । आ वि॑श । वि॒श॒ दी॒र्घा॒यु॒त्वाय॑ । दी॒र्घा॒यु॒त्वाय॑ शन्तनु॒त्वाय॑ । दी॒र्घा॒यु॒त्वायेति॑ दीर्घायु - त्वाय॑ । श॒न्त॒नु॒त्वाय॑ रा॒यः । श॒न्त॒नु॒त्वायेति॑ शन्तनु - त्वाय॑ । रा॒यस्पोषा॑य । पोषा॑य॒ वर्च॑से । वर्च॑से सुप्रजा॒स्त्वाय॑ । सु॒प्र॒जा॒स्त्वाया । सु॒प्र॒जा॒स्त्वायेति॑ सुप्रजाः - त्वाय॑ । एहि॑ । इ॒हि॒ व॒सो॒ । व॒सो॒ पु॒रो॒व॒सो॒ । व॒सो॒ इति॑ वसो । पु॒रो॒व॒सो॒ प्रि॒यः । पु॒रो॒व॒सो॒ इति॑ पुरः - व॒सो॒ । प्रि॒यो मे᳚ । मे॒ हृ॒दः । हृ॒दो॑ ऽसि । अ॒स्य॒श्विनोः᳚ । अ॒श्विनो᳚स्त्वा । त्वा॒ बा॒हुभ्या᳚म् । बा॒हुभ्याꣳ॑ सघ्यासम् । बा॒हुभ्या॒मिति॑ बा॒हु - भ्या॒म् । स॒घ्या॒स॒म् नृ॒चक्ष॑सम् । नृ॒चक्ष॑सम् त्वा । नृ॒चक्ष॑स॒मिति॑ नृ - चक्ष॑सम् । त्वा॒ दे॒व॒ । दे॒व॒ सो॒म॒ । सो॒म॒ सु॒चक्षाः᳚ । सु॒चक्षा॒ अव॑ । सु॒चक्षा॒ इति॑ सु - चक्षाः᳚ । अव॑ ख्येषम् । ख्ये॒ष॒म् म॒न्द्रा । म॒न्द्रा ऽभिभू॑तिः । अ॒भिभू॑तिः के॒तुः । अ॒भिभू॑ति॒रित्य॒भि - भू॒तिः॒ । के॒तुर् य॒ज्ञाना᳚म् । य॒ज्ञानां॒ ॅवाक् । वाग् जु॑षा॒णा । जु॒षा॒णा सोम॑स्य । सोम॑स्य तृप्यतु । तृ॒प्य॒तु॒ म॒न्द्रा । म॒न्द्रा स्व॑र्वाची । स्व॑र्वा॒च्यदि॑तिः । स्व॑र्वा॒चीति॒ सु - अ॒र्वा॒ची॒ । अदि॑ति॒रना॑हतशीर्ष्णी । अना॑हतशीर्ष्णी॒ वाक् । अना॑हतशी॒र्ष्णीत्यना॑हत - शी॒र्ष्णी॒ । वाग् जु॑षा॒णा । जु॒षा॒णा सोम॑स्य । सोम॑स्य तृप्यतु । तृ॒प्य॒त्वा । एहि॑ । इ॒हि॒ वि॒श्व॒च॒र्॒.ष॒णे॒ । वि॒श्व॒च॒र्॒.ष॒णे॒ श॒म्भूः । वि॒श्व॒च॒र्॒.ष॒ण इति॑ विश्व - च॒र्॒.ष॒णे॒ \newline

\textbf{Jatai Paata} \newline

1. भक्षा भक्ष॒ भक्षा । \newline
2. एही॒ह्येहि॑ । \newline
3. इ॒हि॒ मा॒ मे॒ही॒हि॒ मा॒ । \newline
4. माऽऽ मा॒ मा । \newline
5. आ वि॑श वि॒शा वि॑श । \newline
6. वि॒श॒ दी॒र्घा॒यु॒त्वाय॑ दीर्घायु॒त्वाय॑ विश विश दीर्घायु॒त्वाय॑ । \newline
7. दी॒र्घा॒यु॒त्वाय॑ शन्तनु॒त्वाय॑ शन्तनु॒त्वाय॑ दीर्घायु॒त्वाय॑ दीर्घायु॒त्वाय॑ शन्तनु॒त्वाय॑ । \newline
8. दी॒र्घा॒यु॒त्वायेति॑ दीर्घायु - त्वाय॑ । \newline
9. श॒न्त॒नु॒त्वाय॑ रा॒यो रा॒यः श॑न्तनु॒त्वाय॑ शन्तनु॒त्वाय॑ रा॒यः । \newline
10. श॒न्त॒नु॒त्वायेति॑ शन्तनु - त्वाय॑ । \newline
11. रा॒य स्पोषा॑य॒ पोषा॑य रा॒यो रा॒य स्पोषा॑य । \newline
12. पोषा॑य॒ वर्च॑से॒ वर्च॑से॒ पोषा॑य॒ पोषा॑य॒ वर्च॑से । \newline
13. वर्च॑से सुप्रजा॒स्त्वाय॑ सुप्रजा॒स्त्वाय॒ वर्च॑से॒ वर्च॑से सुप्रजा॒स्त्वाय॑ । \newline
14. सु॒प्र॒जा॒स्त्वाया सु॑प्रजा॒स्त्वाय॑ सुप्रजा॒स्त्वाया । \newline
15. सु॒प्र॒जा॒स्त्वायेति॑ सुप्रजाः - त्वाय॑ । \newline
16. एही॒ह्येहि॑ । \newline
17. इ॒हि॒ व॒सो॒ व॒सो॒ इ॒ही॒हि॒ व॒सो॒ । \newline
18. व॒सो॒ पु॒रो॒व॒सो॒ पु॒रो॒व॒सो॒ व॒सो॒ व॒सो॒ पु॒रो॒व॒सो॒ । \newline
19. व॒सो॒ इति॑ वसो । \newline
20. पु॒रो॒व॒सो॒ प्रि॒यः प्रि॒यः पु॑रोवसो पुरोवसो प्रि॒यः । \newline
21. पु॒रो॒व॒सो॒ इति॑ पुरः - व॒सो॒ । \newline
22. प्रि॒यो मे॑ मे प्रि॒यः प्रि॒यो मे᳚ । \newline
23. मे॒ हृ॒दो हृ॒दो मे॑ मे हृ॒दः । \newline
24. हृ॒दो᳚ ऽस्यसि हृ॒दो हृ॒दो॑ ऽसि । \newline
25. अ॒स्य॒श्विनो॑ र॒श्विनो॑ रस्य स्य॒श्विनोः᳚ । \newline
26. अ॒श्विनो᳚ स्त्वा त्वा॒ ऽश्विनो॑ र॒श्विनो᳚ स्त्वा । \newline
27. त्वा॒ बा॒हुभ्या᳚म् बा॒हुभ्या᳚म् त्वा त्वा बा॒हुभ्या᳚म् । \newline
28. बा॒हुभ्याꣳ॑ सघ्यासꣳ सघ्यासम् बा॒हुभ्या᳚म् बा॒हुभ्याꣳ॑ सघ्यासम् । \newline
29. बा॒हुभ्या॒मिति॑ बा॒हु- भ्या॒म् । \newline
30. स॒घ्या॒स॒म् नृ॒चक्ष॑सम् नृ॒चक्ष॑सꣳ सघ्यासꣳ सघ्यासम् नृ॒चक्ष॑सम् । \newline
31. नृ॒चक्ष॑सम् त्वा त्वा नृ॒चक्ष॑सम् नृ॒चक्ष॑सम् त्वा । \newline
32. नृ॒चक्ष॑स॒मिति॑ नृ - चक्ष॑सम् । \newline
33. त्वा॒ दे॒व॒ दे॒व॒ त्वा॒ त्वा॒ दे॒व॒ । \newline
34. दे॒व॒ सो॒म॒ सो॒म॒ दे॒व॒ दे॒व॒ सो॒म॒ । \newline
35. सो॒म॒ सु॒चक्षाः᳚ सु॒चक्षाः᳚ सोम सोम सु॒चक्षाः᳚ । \newline
36. सु॒चक्षा॒ अवाव॑ सु॒चक्षाः᳚ सु॒चक्षा॒ अव॑ । \newline
37. सु॒चक्षा॒ इति॑ सु - चक्षाः᳚ । \newline
38. अव॑ ख्येषम् ख्येष॒ मवाव॑ ख्येषम् । \newline
39. ख्ये॒ष॒म् म॒न्द्रा म॒न्द्रा ख्ये॑षम् ख्येषम् म॒न्द्रा । \newline
40. म॒न्द्रा ऽभिभू॑ति र॒भिभू॑तिर् म॒न्द्रा म॒न्द्रा ऽभिभू॑तिः । \newline
41. अ॒भिभू॑तिः के॒तुः के॒तु र॒भिभू॑ति र॒भिभू॑तिः के॒तुः । \newline
42. अ॒भिभू॑ति॒रित्य॒भि - भू॒तिः॒ । \newline
43. के॒तुर् य॒ज्ञानां᳚ ॅय॒ज्ञाना᳚म् के॒तुः के॒तुर् य॒ज्ञाना᳚म् । \newline
44. य॒ज्ञानां॒ ॅवाग् वाग् य॒ज्ञानां᳚ ॅय॒ज्ञानां॒ ॅवाक् । \newline
45. वाग् जु॑षा॒णा जु॑षा॒णा वाग् वाग् जु॑षा॒णा । \newline
46. जु॒षा॒णा सोम॑स्य॒ सोम॑स्य जुषा॒णा जु॑षा॒णा सोम॑स्य । \newline
47. सोम॑स्य तृप्यतु तृप्यतु॒ सोम॑स्य॒ सोम॑स्य तृप्यतु । \newline
48. तृ॒प्य॒तु॒ म॒न्द्रा म॒न्द्रा तृ॑प्यतु तृप्यतु म॒न्द्रा । \newline
49. म॒न्द्रा स्व॑र्वाची॒ स्व॑र्वाची म॒न्द्रा म॒न्द्रा स्व॑र्वाची । \newline
50. स्व॑र्वा॒ च्यदि॑ति॒ रदि॑तिः॒ स्व॑र्वाची॒ स्व॑र्वा॒ च्यदि॑तिः । \newline
51. स्व॑र्वा॒चीति॒ सु - अ॒र्वा॒ची॒ । \newline
52. अदि॑ति॒ रना॑हतशी॒र् ‌ष्ण्यना॑हतशी॒र्‌ ष्ण्यदि॑ति॒ रदि॑ति॒ रना॑हतशीर्ष्णी । \newline
53. अना॑हतशीर्ष्णी॒ वाग् वागना॑हतशी॒र्‌ ष्ण्यना॑हतशीर्ष्णी॒ वाक् । \newline
54. अना॑हतशी॒र्ष्णीत्यना॑हत - शी॒र्ष्णी॒ । \newline
55. वाग् जु॑षा॒णा जु॑षा॒णा वाग् वाग् जु॑षा॒णा । \newline
56. जु॒षा॒णा सोम॑स्य॒ सोम॑स्य जुषा॒णा जु॑षा॒णा सोम॑स्य । \newline
57. सोम॑स्य तृप्यतु तृप्यतु॒ सोम॑स्य॒ सोम॑स्य तृप्यतु । \newline
58. तृ॒प्य॒त्वा तृ॑प्यतु तृप्य॒त्वा । \newline
59. एही॒ह्येहि॑ । \newline
60. इ॒हि॒ वि॒श्व॒च॒र्॒.ष॒णे॒ वि॒श्व॒च॒र्॒.ष॒ण॒ इ॒ही॒हि॒ वि॒श्व॒च॒र्॒.ष॒णे॒ । \newline
61. वि॒श्व॒च॒र्॒.ष॒णे॒ शं॒भूः शं॒भूर् वि॑श्वचर्.षणे विश्वचर्.षणे शं॒भूः । \newline
62. वि॒श्व॒च॒र्॒.ष॒ण॒ इति॑ विश्व - च॒र्॒.ष॒णे॒ । \newline

\textbf{Ghana Paata } \newline

1. भक्षा भक्ष॒ भक्षेही॒ ह्या भक्ष॒ भक्षेहि॑ । \newline
2. एही॒ह्येहि॑ मा मे॒ह्येहि॑ मा । \newline
3. इ॒हि॒ मा॒ मे॒ही॒हि॒ माऽऽ मे॑हीहि॒ मा । \newline
4. माऽऽ मा॒ माऽऽ वि॑श वि॒शा मा॒ माऽऽ वि॑श । \newline
5. आ वि॑श वि॒शा वि॑श दीर्घायु॒त्वाय॑ दीर्घायु॒त्वाय॑ वि॒शा वि॑श दीर्घायु॒त्वाय॑ । \newline
6. वि॒श॒ दी॒र्घा॒यु॒त्वाय॑ दीर्घायु॒त्वाय॑ विश विश दीर्घायु॒त्वाय॑ शन्तनु॒त्वाय॑ शन्तनु॒त्वाय॑ दीर्घायु॒त्वाय॑ विश विश दीर्घायु॒त्वाय॑ शन्तनु॒त्वाय॑ । \newline
7. दी॒र्घा॒यु॒त्वाय॑ शन्तनु॒त्वाय॑ शन्तनु॒त्वाय॑ दीर्घायु॒त्वाय॑ दीर्घायु॒त्वाय॑ शन्तनु॒त्वाय॑ रा॒यो रा॒यः श॑न्तनु॒त्वाय॑ दीर्घायु॒त्वाय॑ दीर्घायु॒त्वाय॑ शन्तनु॒त्वाय॑ रा॒यः । \newline
8. दी॒र्घा॒यु॒त्वायेति॑ दीर्घायु - त्वाय॑ । \newline
9. श॒न्त॒नु॒त्वाय॑ रा॒यो रा॒यः श॑न्तनु॒त्वाय॑ शन्तनु॒त्वाय॑ रा॒य स्पोषा॑य॒ पोषा॑य रा॒यः श॑न्तनु॒त्वाय॑ शन्तनु॒त्वाय॑ रा॒य स्पोषा॑य । \newline
10. श॒न्त॒नु॒त्वायेति॑ शन्तनु - त्वाय॑ । \newline
11. रा॒य स्पोषा॑य॒ पोषा॑य रा॒यो रा॒य स्पोषा॑य॒ वर्च॑से॒ वर्च॑से॒ पोषा॑य रा॒यो रा॒य स्पोषा॑य॒ वर्च॑से । \newline
12. पोषा॑य॒ वर्च॑से॒ वर्च॑से॒ पोषा॑य॒ पोषा॑य॒ वर्च॑से सुप्रजा॒स्त्वाय॑ सुप्रजा॒स्त्वाय॒ वर्च॑से॒ पोषा॑य॒ पोषा॑य॒ वर्च॑से सुप्रजा॒स्त्वाय॑ । \newline
13. वर्च॑से सुप्रजा॒स्त्वाय॑ सुप्रजा॒स्त्वाय॒ वर्च॑से॒ वर्च॑से सुप्रजा॒स्त्वाया सु॑प्रजा॒स्त्वाय॒ वर्च॑से॒ वर्च॑से सुप्रजा॒स्त्वाया । \newline
14. सु॒प्र॒जा॒स्त्वाया सु॑प्रजा॒स्त्वाय॑ सुप्रजा॒स्त्वाये ही॒ह्या सु॑प्रजा॒स्त्वाय॑ सुप्रजा॒स्त्वायेहि॑ । \newline
15. सु॒प्र॒जा॒स्त्वायेति॑ सुप्रजाः - त्वाय॑ । \newline
16. एही॒ ह्येहि॑ वसो वसो इ॒ह्येहि॑ वसो । \newline
17. इ॒हि॒ व॒सो॒ व॒सो॒ इ॒ही॒हि॒ व॒सो॒ पु॒रो॒व॒सो॒ पु॒रो॒व॒सो॒ व॒सो॒ इ॒ही॒हि॒ व॒सो॒ पु॒रो॒व॒सो॒ । \newline
18. व॒सो॒ पु॒रो॒व॒सो॒ पु॒रो॒व॒सो॒ व॒सो॒ व॒सो॒ पु॒रो॒व॒सो॒ प्रि॒यः प्रि॒यः पु॑रोवसो वसो वसो पुरोवसो प्रि॒यः । \newline
19. व॒सो॒ इति॑ वसो । \newline
20. पु॒रो॒व॒सो॒ प्रि॒यः प्रि॒यः पु॑रोवसो पुरोवसो प्रि॒यो मे॑ मे प्रि॒यः पु॑रोवसो पुरोवसो प्रि॒यो मे᳚ । \newline
21. पु॒रो॒व॒सो॒ इति॑ पुरः - व॒सो॒ । \newline
22. प्रि॒यो मे॑ मे प्रि॒यः प्रि॒यो मे॑ हृ॒दो हृ॒दो मे᳚ प्रि॒यः प्रि॒यो मे॑ हृ॒दः । \newline
23. मे॒ हृ॒दो हृ॒दो मे॑ मे हृ॒दो᳚ ऽस्यसि हृ॒दो मे॑ मे हृ॒दो॑ ऽसि । \newline
24. हृ॒दो᳚ ऽस्यसि हृ॒दो हृ॒दो᳚ ऽस्य॒श्विनो॑ र॒श्विनो॑ रसि हृ॒दो हृ॒दो᳚ ऽस्य॒श्विनोः᳚ । \newline
25. अ॒स्य॒श्विनो॑ र॒श्विनो॑ रस्यस्य॒श्विनो᳚ स्त्वा त्वा॒ ऽश्विनो॑ रस्यस्य॒श्विनो᳚ स्त्वा । \newline
26. अ॒श्विनो᳚ स्त्वा त्वा॒ ऽश्विनो॑ र॒श्विनो᳚ स्त्वा बा॒हुभ्या᳚म् बा॒हुभ्या᳚म् त्वा॒ ऽश्विनो॑ र॒श्विनो᳚ स्त्वा बा॒हुभ्या᳚म् । \newline
27. त्वा॒ बा॒हुभ्या᳚म् बा॒हुभ्या᳚म् त्वा त्वा बा॒हुभ्याꣳ॑ सघ्यासꣳ सघ्यासम् बा॒हुभ्या᳚म् त्वा त्वा बा॒हुभ्याꣳ॑ सघ्यासम् । \newline
28. बा॒हुभ्याꣳ॑ सघ्यासꣳ सघ्यासम् बा॒हुभ्या᳚म् बा॒हुभ्याꣳ॑ सघ्यासम् नृ॒चक्ष॑सम् नृ॒चक्ष॑सꣳ सघ्यासम् बा॒हुभ्या᳚म् बा॒हुभ्याꣳ॑ सघ्यासम् नृ॒चक्ष॑सम् । \newline
29. बा॒हुभ्या॒मिति॑ बा॒हु- भ्या॒म् । \newline
30. स॒घ्या॒स॒म् नृ॒चक्ष॑सम् नृ॒चक्ष॑सꣳ सघ्यासꣳ सघ्यासम् नृ॒चक्ष॑सम् त्वा त्वा नृ॒चक्ष॑सꣳ सघ्यासꣳ सघ्यासम् नृ॒चक्ष॑सम् त्वा । \newline
31. नृ॒चक्ष॑सम् त्वा त्वा नृ॒चक्ष॑सम् नृ॒चक्ष॑सम् त्वा देव देव त्वा नृ॒चक्ष॑सम् नृ॒चक्ष॑सम् त्वा देव । \newline
32. नृ॒चक्ष॑स॒मिति॑ नृ - चक्ष॑सम् । \newline
33. त्वा॒ दे॒व॒ दे॒व॒ त्वा॒ त्वा॒ दे॒व॒ सो॒म॒ सो॒म॒ दे॒व॒ त्वा॒ त्वा॒ दे॒व॒ सो॒म॒ । \newline
34. दे॒व॒ सो॒म॒ सो॒म॒ दे॒व॒ दे॒व॒ सो॒म॒ सु॒चक्षाः᳚ सु॒चक्षाः᳚ सोम देव देव सोम सु॒चक्षाः᳚ । \newline
35. सो॒म॒ सु॒चक्षाः᳚ सु॒चक्षाः᳚ सोम सोम सु॒चक्षा॒ अवाव॑ सु॒चक्षाः᳚ सोम सोम सु॒चक्षा॒ अव॑ । \newline
36. सु॒चक्षा॒ अवाव॑ सु॒चक्षाः᳚ सु॒चक्षा॒ अव॑ ख्येषम् ख्येष॒ मव॑ सु॒चक्षाः᳚ सु॒चक्षा॒ अव॑ ख्येषम् । \newline
37. सु॒चक्षा॒ इति॑ सु - चक्षाः᳚ । \newline
38. अव॑ ख्येषम् ख्येष॒ मवाव॑ ख्येषम् म॒न्द्रा म॒न्द्रा ख्ये॑ष॒ मवाव॑ ख्येषम् म॒न्द्रा । \newline
39. ख्ये॒ष॒म् म॒न्द्रा म॒न्द्रा ख्ये॑षम् ख्येषम् म॒न्द्रा ऽभिभू॑ति र॒भिभू॑तिर् म॒न्द्रा ख्ये॑षम् ख्येषम् म॒न्द्रा ऽभिभू॑तिः । \newline
40. म॒न्द्रा ऽभिभू॑ति र॒भिभू॑तिर् म॒न्द्रा म॒न्द्रा ऽभिभू॑तिः के॒तुः के॒तु र॒भिभू॑तिर् म॒न्द्रा म॒न्द्रा ऽभिभू॑तिः के॒तुः । \newline
41. अ॒भिभू॑तिः के॒तुः के॒तु र॒भिभू॑ति र॒भिभू॑तिः के॒तुर् य॒ज्ञानां᳚ ॅय॒ज्ञाना᳚म् के॒तु र॒भिभू॑ति र॒भिभू॑तिः के॒तुर् य॒ज्ञाना᳚म् । \newline
42. अ॒भिभू॑ति॒रित्य॒भि - भू॒तिः॒ । \newline
43. के॒तुर् य॒ज्ञानां᳚ ॅय॒ज्ञाना᳚म् के॒तुः के॒तुर् य॒ज्ञानां॒ ॅवाग् वाग् य॒ज्ञाना᳚म् के॒तुः के॒तुर् य॒ज्ञानां॒ ॅवाक् । \newline
44. य॒ज्ञानां॒ ॅवाग् वाग् य॒ज्ञानां᳚ ॅय॒ज्ञानां॒ ॅवाग् जु॑षा॒णा जु॑षा॒णा वाग् य॒ज्ञानां᳚ ॅय॒ज्ञानां॒ ॅवाग् जु॑षा॒णा । \newline
45. वाग् जु॑षा॒णा जु॑षा॒णा वाग् वाग् जु॑षा॒णा सोम॑स्य॒ सोम॑स्य जुषा॒णा वाग् वाग् जु॑षा॒णा सोम॑स्य । \newline
46. जु॒षा॒णा सोम॑स्य॒ सोम॑स्य जुषा॒णा जु॑षा॒णा सोम॑स्य तृप्यतु तृप्यतु॒ सोम॑स्य जुषा॒णा जु॑षा॒णा सोम॑स्य तृप्यतु । \newline
47. सोम॑स्य तृप्यतु तृप्यतु॒ सोम॑स्य॒ सोम॑स्य तृप्यतु म॒न्द्रा म॒न्द्रा तृ॑प्यतु॒ सोम॑स्य॒ सोम॑स्य तृप्यतु म॒न्द्रा । \newline
48. तृ॒प्य॒तु॒ म॒न्द्रा म॒न्द्रा तृ॑प्यतु तृप्यतु म॒न्द्रा स्व॑र्वाची॒ स्व॑र्वाची म॒न्द्रा तृ॑प्यतु तृप्यतु म॒न्द्रा स्व॑र्वाची । \newline
49. म॒न्द्रा स्व॑र्वाची॒ स्व॑र्वाची म॒न्द्रा म॒न्द्रा स्व॑र्वा॒ च्यदि॑ति॒ रदि॑तिः॒ स्व॑र्वाची म॒न्द्रा म॒न्द्रा स्व॑र्वा॒ च्यदि॑तिः । \newline
50. स्व॑र्वा॒ च्यदि॑ति॒ रदि॑तिः॒ स्व॑र्वाची॒ स्व॑र्वा॒ च्यदि॑ति॒ रना॑हतशी॒र् ष्ण्यना॑हतशी॒र् ष्ण्यदि॑तिः॒ स्व॑र्वाची॒ स्व॑र्वा॒ च्यदि॑ति॒ रना॑हतशीर्ष्णी । \newline
51. स्व॑र्वा॒चीति॒ सु - अ॒र्वा॒ची॒ । \newline
52. अदि॑ति॒ रना॑हतशी॒र् ष्ण्यना॑हतशी॒र् ष्ण्यदि॑ति॒ रदि॑ति॒ रना॑हतशीर्ष्णी॒ वाग् वागना॑हतशी॒र् ष्ण्यदि॑ति॒ रदि॑ति॒ रना॑हतशीर्ष्णी॒ वाक् । \newline
53. अना॑हतशीर्ष्णी॒ वाग् वागना॑हतशी॒र्‌ ष्ण्यना॑हतशीर्ष्णी॒ वाग् जु॑षा॒णा जु॑षा॒णा वागना॑हतशी॒ र्ष्ण्यना॑हतशीर्ष्णी॒ वाग् जु॑षा॒णा । \newline
54. अना॑हतशी॒र्ष्णीत्यना॑हत - शी॒र्ष्णी॒ । \newline
55. वाग् जु॑षा॒णा जु॑षा॒णा वाग् वाग् जु॑षा॒णा सोम॑स्य॒ सोम॑स्य जुषा॒णा वाग् वाग् जु॑षा॒णा सोम॑स्य । \newline
56. जु॒षा॒णा सोम॑स्य॒ सोम॑स्य जुषा॒णा जु॑षा॒णा सोम॑स्य तृप्यतु तृप्यतु॒ सोम॑स्य जुषा॒णा जु॑षा॒णा सोम॑स्य तृप्यतु । \newline
57. सोम॑स्य तृप्यतु तृप्यतु॒ सोम॑स्य॒ सोम॑स्य तृप्य॒त्वा तृ॑प्यतु॒ सोम॑स्य॒ सोम॑स्य तृप्य॒त्वा । \newline
58. तृ॒प्य॒त्वा तृ॑प्यतु तृप्य॒ त्वेही॒ह्या तृ॑प्यतु तृप्य॒त्वेहि॑ । \newline
59. एही॒ ह्येहि॑ विश्वचर्.षणे विश्वचर्.षण इ॒ह्येहि॑ विश्वचर्.षणे । \newline
60. इ॒हि॒ वि॒श्व॒च॒र्॒.ष॒णे॒ वि॒श्व॒च॒र्॒.ष॒ण॒ इ॒ही॒हि॒ वि॒श्व॒च॒र्॒.ष॒णे॒ शं॒भूः शं॒भूर् वि॑श्वचर्.षण इहीहि विश्वचर्.षणे शं॒भूः । \newline
61. वि॒श्व॒च॒र्॒.ष॒णे॒ शं॒भूः शं॒भूर् वि॑श्वचर्.षणे विश्वचर्.षणे शं॒भूर् म॑यो॒भूर् म॑यो॒भूः शं॒भूर् वि॑श्वचर्.षणे विश्वचर्.षणे शं॒भूर् म॑यो॒भूः । \newline
62. वि॒श्व॒च॒र्॒.ष॒ण॒ इति॑ विश्व - च॒र्॒.ष॒णे॒ । \newline
\pagebreak
\markright{ TS 3.2.5.2  \hfill https://www.vedavms.in \hfill}

\section{ TS 3.2.5.2 }

\textbf{TS 3.2.5.2 } \newline
\textbf{Samhita Paata} \newline

श॒म्भूर्म॑यो॒भूः स्व॒स्ति मा॑ हरिवर्ण॒ प्रच॑र॒ क्रत्वे॒ दक्षा॑य रा॒यस्पोषा॑य सुवी॒रता॑यै॒ मा मा॑ राज॒न्. वि बी॑भिषो॒ मा मे॒ हार्दि॑ त्वि॒षा व॑धीः । वृष॑णे॒ शुष्मा॒याऽऽयु॑षे॒ वर्च॑से ॥ वसु॑मद्-गणस्य सोम देव ते मति॒विदः॑ प्रात॒स्सव॒नस्य॑ गाय॒त्रछ॑न्दस॒ इन्द्र॑पीतस्य॒ नरा॒शꣳस॑पीतस्य पि॒तृपी॑तस्य॒ मधु॑मत॒ उप॑हूत॒स्योप॑हूतो भक्षयामि रु॒द्रव॑द्-गणस्य सोम देव ते मति॒विदो॒ माद्ध्य॑न्दिनस्य॒ सव॑नस्य त्रि॒ष्टुप्छ॑न्दस॒ इन्द्र॑पीतस्य॒ नरा॒शꣳ स॑पीतस्य - [  ] \newline

\textbf{Pada Paata} \newline

श॒भूंरिति॑ शं - भूः । म॒यो॒भूरिति॑ मयः - भूः । स्व॒स्ति । मा॒ । ह॒रि॒व॒र्णेति॑ हरि - व॒र्ण॒ । प्रेति॑ । च॒र॒ । क्रत्वे᳚ । दक्षा॑य । रा॒यः । पोषा॑य । सु॒वी॒रता॑या॒ इति॑ सु-वी॒रता॑यै । मा । मा॒ । रा॒ज॒न्न् । वीति॑ । बी॒भि॒षः॒ । मा । मे॒ । हार्दि॑ । त्वि॒षा । व॒धीः॒ ॥ वृष॑णे । शुष्मा॑य । आयु॑षे । वर्च॑से ॥ वसु॑मद् गण॒स्येति॒ वसु॑मत् - ग॒ण॒स्य॒ । सो॒म॒ । दे॒व॒ । ते॒ । म॒ति॒विद॒ इति॑ मति - विदः॑ । प्रा॒त॒स्स॒व॒नस्येति॑ प्रातः - स॒व॒नस्य॑ । गा॒य॒त्रछ॑न्दस॒ इति॑ गाय॒त्र - छ॒न्द॒सः॒ । इन्द्र॑पीत॒स्येतीन्द्र॑ - पी॒त॒स्य॒ । नरा॒शꣳस॑पीत॒स्येति॒ नरा॒शꣳस॑ - पी॒त॒स्य॒ । पि॒तृपी॑त॒स्येति॑ पि॒तृ - पी॒त॒स्य॒ । मधु॑मत॒ इति॒ मधु॑ - म॒तः॒ । उप॑हूत॒स्येत्युप॑ - हू॒त॒स्य॒ । उप॑हूत॒ इत्युप॑ - हू॒तः॒ । भ॒क्ष॒या॒मि॒ । रु॒द्रव॑द्गण॒स्येति॑ रु॒द्रव॑त् - ग॒ण॒स्य॒ । सो॒म॒ । दे॒व॒ । ते॒ । म॒ति॒विद॒ इति॑ मति - विदः॑ । माद्ध्य॑न्दिनस्य । सव॑नस्य । त्रि॒ष्टुप्छ॑न्दस॒ इति॑ त्रि॒ष्टुप् - छ॒न्द॒सः॒ । इन्द्र॑पीत॒स्येतीन्द्र॑ - पी॒त॒स्य॒ । नरा॒शꣳस॑पीत॒स्येति॒ नरा॒शꣳस॑ - पी॒त॒स्य॒ ।  \newline


\textbf{Krama Paata} \newline

श॒म्भूर् म॑यो॒भूः । श॒म्भूरिति॑ शम् - भूः । म॒यो॒भूः स्व॒स्ति । म॒यो॒भूरिति॑ मयः - भूः । स्व॒स्ति मा᳚ । मा॒ ह॒रि॒व॒र्ण॒ । ह॒रि॒व॒र्ण॒ प्र । ह॒रि॒व॒र्णेति॑ हरि - व॒र्ण॒ । प्र च॑र । च॒र॒ क्रत्वे᳚ । क्रत्वे॒ दक्षा॑य । दक्षा॑य रा॒यः । रा॒यस्पोषा॑य । पोषा॑य सुवी॒रता॑यै । सु॒वी॒रता॑यै॒ मा । सु॒वी॒रता॑या॒ इति॑ सु - वी॒रता॑यै । मा मा᳚ । मा॒ रा॒ज॒न्न्॒ । रा॒ज॒न् वि । वि बी॑भिषः । बी॒भि॒षो॒ मा । मा मे᳚ । मे॒ हार्दि॑ । हार्दि॑ त्वि॒षा । त्वि॒षा व॑धीः । व॒धी॒रिति॑ वधीः ॥ वृष॑णे॒ शुष्मा॑य । शुष्मा॒यायु॑षे । आयु॑षे॒ वर्च॑से । वर्च॑स॒ इति॒ वर्च॑से ॥ वसु॑मद्गणस्य सोम । वसु॑मद्गण॒स्येति॒ वसु॑मत् - ग॒ण॒स्य॒ । सो॒म॒ दे॒व॒ । दे॒व॒ ते॒ । ते॒ म॒ति॒विदः॑ । म॒ति॒विदः॑ प्रातस्सव॒नस्य॑ । म॒ति॒विद॒ इति॑ मति - विदः॑ । प्रा॒त॒स्स॒व॒नस्य॑ गाय॒त्रछ॑न्दसः । प्रा॒त॒स्स॒व॒नस्येति॑ प्रातः - स॒व॒नस्य॑ । गा॒य॒त्रछ॑न्दस॒ इन्द्र॑पीतस्य । गा॒य॒त्रछ॑न्दस॒ इति॑ गाय॒त्र - छ॒न्द॒सः॒ । इन्द्र॑पीतस्य॒ नरा॒शꣳस॑पीतस्य । इन्द्र॑पीत॒स्येतीन्द्र॑ - पी॒त॒स्य॒ । नरा॒शꣳस॑पीतस्य पि॒तृपी॑तस्य । नरा॒शꣳस॑पीत॒स्येति॒ नरा॒शꣳस॑ - पी॒त॒स्य॒ । पि॒तृपी॑तस्य॒ मधु॑मतः । पि॒तृपी॑त॒स्येति॑ पि॒तृ - पी॒त॒स्य॒ । मधु॑मत॒ उप॑हूतस्य । मधु॑मत॒ इति॒ मधु॑ - म॒तः॒ । उप॑हूत॒स्योप॑हूतः । उप॑हूत॒स्येत्युप॑ - हू॒त॒स्य॒ । उप॑हूतो भक्षयामि । उप॑हूत॒ इत्युप॑ - हू॒तः॒ । भ॒क्ष॒या॒मि॒ रु॒द्रव॑द्गणस्य । रु॒द्रव॑द्गणस्य सोम । रु॒द्रव॑द्गण॒स्येति॑ रु॒द्रव॑त् - ग॒ण॒स्य॒ । सो॒म॒ दे॒व॒ । दे॒व॒ ते॒ । ते॒ म॒ति॒विदः॑ । म॒ति॒विदो॒ माद्ध्य॑न्दिनस्य । म॒ति॒विद॒ इति॑ मति - विदः॑ । माद्ध्य॑न्दिनस्य॒ सव॑नस्य । सव॑नस्य त्रि॒ष्टुप्छ॑न्दसः । त्रि॒ष्टुप्छ॑न्दस॒ इन्द्र॑पीतस्य । त्रि॒ष्टुप्छ॑न्दस॒ इति॑ त्रि॒ष्टुप् - छ॒न्द॒सः॒ । इन्द्र॑पीतस्य॒ नरा॒शꣳस॑पीतस्य । इन्द्र॑पीत॒स्येतीन्द्र॑ - पी॒त॒स्य॒ । नरा॒शꣳस॑पीतस्य पि॒तृपी॑तस्य । नरा॒शꣳस॑पीत॒स्येति॒ नरा॒शꣳस॑ - पी॒त॒स्य॒ \newline

\textbf{Jatai Paata} \newline

1. शं॒भूर् म॑यो॒भूर् म॑यो॒भूः शं॒भूः शं॒भूर् म॑यो॒भूः । \newline
2. शं॒भूरिति॑ शं - भूः । \newline
3. म॒यो॒भूः स्व॒स्ति स्व॒स्ति म॑यो॒भूर् म॑यो॒भूः स्व॒स्ति । \newline
4. म॒यो॒भूरिति॑ मयः - भूः । \newline
5. स्व॒स्ति मा॑ मा स्व॒स्ति स्व॒स्ति मा᳚ । \newline
6. मा॒ ह॒रि॒व॒र्ण॒ ह॒रि॒व॒र्ण॒ मा॒ मा॒ ह॒रि॒व॒र्ण॒ । \newline
7. ह॒रि॒व॒र्ण॒ प्र प्र ह॑रिवर्ण हरिवर्ण॒ प्र । \newline
8. ह॒रि॒व॒र्णेति॑ हरि - व॒र्ण॒ । \newline
9. प्र च॑र चर॒ प्र प्र च॑र । \newline
10. च॒र॒ क्रत्वे॒ क्रत्वे॑ चर चर॒ क्रत्वे᳚ । \newline
11. क्रत्वे॒ दक्षा॑य॒ दक्षा॑य॒ क्रत्वे॒ क्रत्वे॒ दक्षा॑य । \newline
12. दक्षा॑य रा॒यो रा॒यो दक्षा॑य॒ दक्षा॑य रा॒यः । \newline
13. रा॒य स्पोषा॑य॒ पोषा॑य रा॒यो रा॒य स्पोषा॑य । \newline
14. पोषा॑य सुवी॒रता॑यै सुवी॒रता॑यै॒ पोषा॑य॒ पोषा॑य सुवी॒रता॑यै । \newline
15. सु॒वी॒रता॑यै॒ मा मा सु॑वी॒रता॑यै सुवी॒रता॑यै॒ मा । \newline
16. सु॒वी॒रता॑या॒ इति॑ सु - वी॒रता॑यै । \newline
17. मा मा॑ मा॒ मा मा मा᳚ । \newline
18. मा॒ रा॒ज॒न् रा॒ज॒न् मा॒ मा॒ रा॒ज॒न्न् । \newline
19. रा॒ज॒न् वि वि रा॑जन् राज॒न् वि । \newline
20. वि बी॑भिषो बीभिषो॒ वि वि बी॑भिषः । \newline
21. बी॒भि॒षो॒ मा मा बी॑भिषो बीभिषो॒ मा । \newline
22. मा मे॑ मे॒ मा मा मे᳚ । \newline
23. मे॒ हार्दि॒ हार्दि॑ मे मे॒ हार्दि॑ । \newline
24. हार्दि॑ त्वि॒षा त्वि॒षा हार्दि॒ हार्दि॑ त्वि॒षा । \newline
25. त्वि॒षा व॑धीर् वधी स्त्वि॒षा त्वि॒षा व॑धीः । \newline
26. व॒धी॒रिति॑ वधीः । \newline
27. वृष॑णे॒ शुष्मा॑य॒ शुष्मा॑य॒ वृष॑णे॒ वृष॑णे॒ शुष्मा॑य । \newline
28. शुष्मा॒ यायु॑ष॒ आयु॑षे॒ शुष्मा॑य॒ शुष्मा॒ यायु॑षे । \newline
29. आयु॑षे॒ वर्च॑से॒ वर्च॑स॒ आयु॑ष॒ आयु॑षे॒ वर्च॑से । \newline
30. वर्च॑स॒ इति॒ वर्च॑से । \newline
31. वसु॑मद्‍गणस्य सोम सोम॒ वसु॑मद्‍गणस्य॒ वसु॑मद्‍गणस्य सोम । \newline
32. वसु॑मद् गण॒स्येति॒ वसु॑मत् - ग॒ण॒स्य॒ । \newline
33. सो॒म॒ दे॒व॒ दे॒व॒ सो॒म॒ सो॒म॒ दे॒व॒ । \newline
34. दे॒व॒ ते॒ ते॒ दे॒व॒ दे॒व॒ ते॒ । \newline
35. ते॒ म॒ति॒विदो॑ मति॒विद॑ स्ते ते मति॒विदः॑ । \newline
36. म॒ति॒विदः॑ प्रातस्सव॒नस्य॑ प्रातस्सव॒नस्य॑ मति॒विदो॑ मति॒विदः॑ प्रातस्सव॒नस्य॑ । \newline
37. म॒ति॒विद॒ इति॑ मति - विदः॑ । \newline
38. प्रा॒त॒स्स॒व॒नस्य॑ गाय॒त्रछ॑न्दसो गाय॒त्रछ॑न्दसः प्रातस्सव॒नस्य॑ प्रातस्सव॒नस्य॑ गाय॒त्रछ॑न्दसः । \newline
39. प्रा॒त॒स्स॒व॒नस्येति॑ प्रातः - स॒व॒नस्य॑ । \newline
40. गा॒य॒त्रछ॑न्दस॒ इन्द्र॑पीत॒स्ये न्द्र॑पीतस्य गाय॒त्रछ॑न्दसो गाय॒त्रछ॑न्दस॒ इन्द्र॑पीतस्य । \newline
41. गा॒य॒त्रछ॑न्दस॒ इति॑ गाय॒त्र - छ॒न्द॒सः॒ । \newline
42. इन्द्र॑पीतस्य॒ नरा॒शꣳस॑पीतस्य॒ नरा॒शꣳस॑पीत॒ स्येन्द्र॑पीत॒स्ये न्द्र॑पीतस्य॒ नरा॒शꣳस॑पीतस्य । \newline
43. इन्द्र॑पीत॒स्येतीन्द्र॑ - पी॒त॒स्य॒ । \newline
44. नरा॒शꣳस॑पीतस्य पि॒तृपी॑तस्य पि॒तृपी॑तस्य॒ नरा॒शꣳस॑पीतस्य॒ नरा॒शꣳस॑पीतस्य पि॒तृपी॑तस्य । \newline
45. नरा॒शꣳस॑पीत॒स्येति॒ नरा॒शꣳस॑ - पी॒त॒स्य॒ । \newline
46. पि॒तृपी॑तस्य॒ मधु॑मतो॒ मधु॑मतः पि॒तृपी॑तस्य पि॒तृपी॑तस्य॒ मधु॑मतः । \newline
47. पि॒तृपी॑त॒स्येति॑ पि॒तृ - पी॒त॒स्य॒ । \newline
48. मधु॑मत॒ उप॑हूत॒ स्योप॑हूतस्य॒ मधु॑मतो॒ मधु॑मत॒ उप॑हूतस्य । \newline
49. मधु॑मत॒ इति॒ मधु॑ - म॒तः॒ । \newline
50. उप॑हूत॒ स्योप॑हूत॒ उप॑हूत॒ उप॑हूत॒ स्योप॑हूत॒ स्योप॑हूतः । \newline
51. उप॑हूत॒स्येत्युप॑ - हू॒त॒स्य॒ । \newline
52. उप॑हूतो भक्षयामि भक्षया॒ म्युप॑हूत॒ उप॑हूतो भक्षयामि । \newline
53. उप॑हूत॒ इत्युप॑ - हू॒तः॒ । \newline
54. भ॒क्ष॒या॒मि॒ रु॒द्रव॑द्‍गणस्य रु॒द्रव॑द्‍गणस्य भक्षयामि भक्षयामि रु॒द्रव॑द्‍गणस्य । \newline
55. रु॒द्रव॑द्‍गणस्य सोम सोम रु॒द्रव॑द्‍गणस्य रु॒द्रव॑द्‍गणस्य सोम । \newline
56. रु॒द्रव॑द्‍गण॒स्येति॑ रु॒द्रव॑त् - ग॒ण॒स्य॒ । \newline
57. सो॒म॒ दे॒व॒ दे॒व॒ सो॒म॒ सो॒म॒ दे॒व॒ । \newline
58. दे॒व॒ ते॒ ते॒ दे॒व॒ दे॒व॒ ते॒ । \newline
59. ते॒ म॒ति॒विदो॑ मति॒विद॑ स्ते ते मति॒विदः॑ । \newline
60. म॒ति॒विदो॒ माद्ध्य॑न्दिनस्य॒ माद्ध्य॑न्दिनस्य मति॒विदो॑ मति॒विदो॒ माद्ध्य॑न्दिनस्य । \newline
61. म॒ति॒विद॒ इति॑ मति - विदः॑ । \newline
62. माद्ध्य॑न्दिनस्य॒ सव॑नस्य॒ सव॑नस्य॒ माद्ध्य॑न्दिनस्य॒ माद्ध्य॑न्दिनस्य॒ सव॑नस्य । \newline
63. सव॑नस्य त्रि॒ष्टुप्छ॑न्दस स्त्रि॒ष्टुप्छ॑न्दसः॒ सव॑नस्य॒ सव॑नस्य त्रि॒ष्टुप्छ॑न्दसः । \newline
64. त्रि॒ष्टुप्छ॑न्दस॒ इन्द्र॑पीत॒स्ये न्द्र॑पीतस्य त्रि॒ष्टुप्छ॑न्दस स्त्रि॒ष्टुप्छ॑न्दस॒ इन्द्र॑पीतस्य । \newline
65. त्रि॒ष्टुप्छ॑न्दस॒ इति॑ त्रि॒ष्टुप् - छ॒न्द॒सः॒ । \newline
66. इन्द्र॑पीतस्य॒ नरा॒शꣳस॑पीतस्य॒ नरा॒शꣳस॑पीत॒स्ये न्द्र॑पीत॒स्ये न्द्र॑पीतस्य॒ नरा॒शꣳस॑पीतस्य । \newline
67. इन्द्र॑पीत॒स्येतीन्द्र॑ - पी॒त॒स्य॒ । \newline
68. नरा॒शꣳस॑पीतस्य पि॒तृपी॑तस्य पि॒तृपी॑तस्य॒ नरा॒शꣳस॑पीतस्य॒ नरा॒शꣳस॑पीतस्य पि॒तृपी॑तस्य । \newline
69. नरा॒शꣳस॑पीत॒स्येति॒ नरा॒शꣳस॑ - पी॒त॒स्य॒ । \newline

\textbf{Ghana Paata } \newline

1. शं॒भूर् म॑यो॒भूर् म॑यो॒भूः शं॒भूः शं॒भूर् म॑यो॒भूः स्व॒स्ति स्व॒स्ति म॑यो॒भूः शं॒भूः शं॒भूर् म॑यो॒भूः स्व॒स्ति । \newline
2. शं॒भूरिति॑ शं - भूः । \newline
3. म॒यो॒भूः स्व॒स्ति स्व॒स्ति म॑यो॒भूर् म॑यो॒भूः स्व॒स्ति मा॑ मा स्व॒स्ति म॑यो॒भूर् म॑यो॒भूः स्व॒स्ति मा᳚ । \newline
4. म॒यो॒भूरिति॑ मयः - भूः । \newline
5. स्व॒स्ति मा॑ मा स्व॒स्ति स्व॒स्ति मा॑ हरिवर्ण हरिवर्ण मा स्व॒स्ति स्व॒स्ति मा॑ हरिवर्ण । \newline
6. मा॒ ह॒रि॒व॒र्ण॒ ह॒रि॒व॒र्ण॒ मा॒ मा॒ ह॒रि॒व॒र्ण॒ प्र प्र ह॑रिवर्ण मा मा हरिवर्ण॒ प्र । \newline
7. ह॒रि॒व॒र्ण॒ प्र प्र ह॑रिवर्ण हरिवर्ण॒ प्र च॑र चर॒ प्र ह॑रिवर्ण हरिवर्ण॒ प्र च॑र । \newline
8. ह॒रि॒व॒र्णेति॑ हरि - व॒र्ण॒ । \newline
9. प्र च॑र चर॒ प्र प्र च॑र॒ क्रत्वे॒ क्रत्वे॑ चर॒ प्र प्र च॑र॒ क्रत्वे᳚ । \newline
10. च॒र॒ क्रत्वे॒ क्रत्वे॑ चर चर॒ क्रत्वे॒ दक्षा॑य॒ दक्षा॑य॒ क्रत्वे॑ चर चर॒ क्रत्वे॒ दक्षा॑य । \newline
11. क्रत्वे॒ दक्षा॑य॒ दक्षा॑य॒ क्रत्वे॒ क्रत्वे॒ दक्षा॑य रा॒यो रा॒यो दक्षा॑य॒ क्रत्वे॒ क्रत्वे॒ दक्षा॑य रा॒यः । \newline
12. दक्षा॑य रा॒यो रा॒यो दक्षा॑य॒ दक्षा॑य रा॒य स्पोषा॑य॒ पोषा॑य रा॒यो दक्षा॑य॒ दक्षा॑य रा॒य स्पोषा॑य । \newline
13. रा॒य स्पोषा॑य॒ पोषा॑य रा॒यो रा॒य स्पोषा॑य सुवी॒रता॑यै सुवी॒रता॑यै॒ पोषा॑य रा॒यो रा॒य स्पोषा॑य सुवी॒रता॑यै । \newline
14. पोषा॑य सुवी॒रता॑यै सुवी॒रता॑यै॒ पोषा॑य॒ पोषा॑य सुवी॒रता॑यै॒ मा मा सु॑वी॒रता॑यै॒ पोषा॑य॒ पोषा॑य सुवी॒रता॑यै॒ मा । \newline
15. सु॒वी॒रता॑यै॒ मा मा सु॑वी॒रता॑यै सुवी॒रता॑यै॒ मा मा॑ मा॒ मा सु॑वी॒रता॑यै सुवी॒रता॑यै॒ मा मा᳚ । \newline
16. सु॒वी॒रता॑या॒ इति॑ सु - वी॒रता॑यै । \newline
17. मा मा॑ मा॒ मा मा मा॑ राजन् राजन् मा॒ मा मा मा॑ राजन्न् । \newline
18. मा॒ रा॒ज॒न् रा॒ज॒न् मा॒ मा॒ रा॒ज॒न् वि वि रा॑जन् मा मा राज॒न् वि । \newline
19. रा॒ज॒न् वि वि रा॑जन् राज॒न् वि बी॑भिषो बीभिषो॒ वि रा॑जन् राज॒न् वि बी॑भिषः । \newline
20. वि बी॑भिषो बीभिषो॒ वि वि बी॑भिषो॒ मा मा बी॑भिषो॒ वि वि बी॑भिषो॒ मा । \newline
21. बी॒भि॒षो॒ मा मा बी॑भिषो बीभिषो॒ मा मे॑ मे॒ मा बी॑भिषो बीभिषो॒ मा मे᳚ । \newline
22. मा मे॑ मे॒ मा मा मे॒ हार्दि॒ हार्दि॑ मे॒ मा मा मे॒ हार्दि॑ । \newline
23. मे॒ हार्दि॒ हार्दि॑ मे मे॒ हार्दि॑ त्वि॒षा त्वि॒षा हार्दि॑ मे मे॒ हार्दि॑ त्वि॒षा । \newline
24. हार्दि॑ त्वि॒षा त्वि॒षा हार्दि॒ हार्दि॑ त्वि॒षा व॑धीर् वधी स्त्वि॒षा हार्दि॒ हार्दि॑ त्वि॒षा व॑धीः । \newline
25. त्वि॒षा व॑धीर् वधी स्त्वि॒षा त्वि॒षा व॑धीः । \newline
26. व॒धी॒रिति॑ वधीः । \newline
27. वृष॑णे॒ शुष्मा॑य॒ शुष्मा॑य॒ वृष॑णे॒ वृष॑णे॒ शुष्मा॒यायु॑ष॒ आयु॑षे॒ शुष्मा॑य॒ वृष॑णे॒ वृष॑णे॒ शुष्मा॒यायु॑षे । \newline
28. शुष्मा॒यायु॑ष॒ आयु॑षे॒ शुष्मा॑य॒ शुष्मा॒यायु॑षे॒ वर्च॑से॒ वर्च॑स॒ आयु॑षे॒ शुष्मा॑य॒ शुष्मा॒यायु॑षे॒ वर्च॑से । \newline
29. आयु॑षे॒ वर्च॑से॒ वर्च॑स॒ आयु॑ष॒ आयु॑षे॒ वर्च॑से । \newline
30. वर्च॑स॒ इति॒ वर्च॑से । \newline
31. वसु॑मद्‍गणस्य सोम सोम॒ वसु॑मद्‍गणस्य॒ वसु॑मद्‍गणस्य सोम देव देव सोम॒ वसु॑मद्‍गणस्य॒ वसु॑मद्‍गणस्य सोम देव । \newline
32. वसु॑मद्‍गण॒स्येति॒ वसु॑मत् - ग॒ण॒स्य॒ । \newline
33. सो॒म॒ दे॒व॒ दे॒व॒ सो॒म॒ सो॒म॒ दे॒व॒ ते॒ ते॒ दे॒व॒ सो॒म॒ सो॒म॒ दे॒व॒ ते॒ । \newline
34. दे॒व॒ ते॒ ते॒ दे॒व॒ दे॒व॒ ते॒ म॒ति॒विदो॑ मति॒विद॑ स्ते देव देव ते मति॒विदः॑ । \newline
35. ते॒ म॒ति॒विदो॑ मति॒विद॑ स्ते ते मति॒विदः॑ प्रातस्सव॒नस्य॑ प्रातस्सव॒नस्य॑ मति॒विद॑ स्ते ते मति॒विदः॑ प्रातस्सव॒नस्य॑ । \newline
36. म॒ति॒विदः॑ प्रातस्सव॒नस्य॑ प्रातस्सव॒नस्य॑ मति॒विदो॑ मति॒विदः॑ प्रातस्सव॒नस्य॑ गाय॒त्रछ॑न्दसो गाय॒त्रछ॑न्दसः प्रातस्सव॒नस्य॑ मति॒विदो॑ मति॒विदः॑ प्रातस्सव॒नस्य॑ गाय॒त्रछ॑न्दसः । \newline
37. म॒ति॒विद॒ इति॑ मति - विदः॑ । \newline
38. प्रा॒त॒स्स॒व॒नस्य॑ गाय॒त्रछ॑न्दसो गाय॒त्रछ॑न्दसः प्रातस्सव॒नस्य॑ प्रातस्सव॒नस्य॑ गाय॒त्रछ॑न्दस॒ इन्द्र॑पीत॒स्ये न्द्र॑पीतस्य गाय॒त्रछ॑न्दसः प्रातस्सव॒नस्य॑ प्रातस्सव॒नस्य॑ गाय॒त्रछ॑न्दस॒ इन्द्र॑पीतस्य । \newline
39. प्रा॒त॒स्स॒व॒नस्येति॑ प्रातः - स॒व॒नस्य॑ । \newline
40. गा॒य॒त्रछ॑न्दस॒ इन्द्र॑पीत॒स्ये न्द्र॑पीतस्य गाय॒त्रछ॑न्दसो गाय॒त्रछ॑न्दस॒ इन्द्र॑पीतस्य॒ नरा॒शꣳस॑पीतस्य॒ नरा॒शꣳस॑पीत॒स्ये न्द्र॑पीतस्य गाय॒त्रछ॑न्दसो गाय॒त्रछ॑न्दस॒ इन्द्र॑पीतस्य॒ नरा॒शꣳस॑पीतस्य । \newline
41. गा॒य॒त्रछ॑न्दस॒ इति॑ गाय॒त्र - छ॒न्द॒सः॒ । \newline
42. इन्द्र॑पीतस्य॒ नरा॒शꣳस॑पीतस्य॒ नरा॒शꣳस॑पीत॒स्ये न्द्र॑पीत॒स्ये न्द्र॑पीतस्य॒ नरा॒शꣳस॑पीतस्य पि॒तृपी॑तस्य पि॒तृपी॑तस्य॒ नरा॒शꣳस॑पीत॒स्ये न्द्र॑पीत॒स्ये न्द्र॑पीतस्य॒ नरा॒शꣳस॑पीतस्य पि॒तृपी॑तस्य । \newline
43. इन्द्र॑पीत॒स्येतीन्द्र॑ - पी॒त॒स्य॒ । \newline
44. नरा॒शꣳस॑पीतस्य पि॒तृपी॑तस्य पि॒तृपी॑तस्य॒ नरा॒शꣳस॑पीतस्य॒ नरा॒शꣳस॑पीतस्य पि॒तृपी॑तस्य॒ मधु॑मतो॒ मधु॑मतः पि॒तृपी॑तस्य॒ नरा॒शꣳस॑पीतस्य॒ नरा॒शꣳस॑पीतस्य पि॒तृपी॑तस्य॒ मधु॑मतः । \newline
45. नरा॒शꣳस॑पीत॒स्येति॒ नरा॒शꣳस॑ - पी॒त॒स्य॒ । \newline
46. पि॒तृपी॑तस्य॒ मधु॑मतो॒ मधु॑मतः पि॒तृपी॑तस्य पि॒तृपी॑तस्य॒ मधु॑मत॒ उप॑हूत॒ स्योप॑हूतस्य॒ मधु॑मतः पि॒तृपी॑तस्य पि॒तृपी॑तस्य॒ मधु॑मत॒ उप॑हूतस्य । \newline
47. पि॒तृपी॑त॒स्येति॑ पि॒तृ - पी॒त॒स्य॒ । \newline
48. मधु॑मत॒ उप॑हूत॒ स्योप॑हूतस्य॒ मधु॑मतो॒ मधु॑मत॒ उप॑हूत॒ स्योप॑हूत॒ उप॑हूत॒ उप॑हूतस्य॒ मधु॑मतो॒ मधु॑मत॒ उप॑हूत॒ स्योप॑हूतः । \newline
49. मधु॑मत॒ इति॒ मधु॑ - म॒तः॒ । \newline
50. उप॑हूत॒ स्योप॑हूत॒ उप॑हूत॒ उप॑हूत॒ स्योप॑हूत॒ स्योप॑हूतो भक्षयामि भक्षया॒ म्युप॑हूत॒ उप॑हूत॒ स्योप॑हूत॒ स्योप॑हूतो भक्षयामि । \newline
51. उप॑हूत॒स्येत्युप॑ - हू॒त॒स्य॒ । \newline
52. उप॑हूतो भक्षयामि भक्षया॒ म्युप॑हूत॒ उप॑हूतो भक्षयामि रु॒द्रव॑द्‍गणस्य रु॒द्रव॑द्‍गणस्य भक्षया॒ म्युप॑हूत॒ उप॑हूतो भक्षयामि रु॒द्रव॑द्‍गणस्य । \newline
53. उप॑हूत॒ इत्युप॑ - हू॒तः॒ । \newline
54. भ॒क्ष॒या॒मि॒ रु॒द्रव॑द्‍गणस्य रु॒द्रव॑द्‍गणस्य भक्षयामि भक्षयामि रु॒द्रव॑द्‍गणस्य सोम सोम रु॒द्रव॑द्‍गणस्य भक्षयामि भक्षयामि रु॒द्रव॑द्‍गणस्य सोम । \newline
55. रु॒द्रव॑द्‍गणस्य सोम सोम रु॒द्रव॑द्‍गणस्य रु॒द्रव॑द्‍गणस्य सोम देव देव सोम रु॒द्रव॑द्‍गणस्य रु॒द्रव॑द्‍गणस्य सोम देव । \newline
56. रु॒द्रव॑द्‍गण॒स्येति॑ रु॒द्रव॑त् - ग॒ण॒स्य॒ । \newline
57. सो॒म॒ दे॒व॒ दे॒व॒ सो॒म॒ सो॒म॒ दे॒व॒ ते॒ ते॒ दे॒व॒ सो॒म॒ सो॒म॒ दे॒व॒ ते॒ । \newline
58. दे॒व॒ ते॒ ते॒ दे॒व॒ दे॒व॒ ते॒ म॒ति॒विदो॑ मति॒विद॑ स्ते देव देव ते मति॒विदः॑ । \newline
59. ते॒ म॒ति॒विदो॑ मति॒विद॑ स्ते ते मति॒विदो॒ माद्ध्य॑न्दिनस्य॒ माद्ध्य॑न्दिनस्य मति॒विद॑ स्ते ते मति॒विदो॒ माद्ध्य॑न्दिनस्य । \newline
60. म॒ति॒विदो॒ माद्ध्य॑न्दिनस्य॒ माद्ध्य॑न्दिनस्य मति॒विदो॑ मति॒विदो॒ माद्ध्य॑न्दिनस्य॒ सव॑नस्य॒ सव॑नस्य॒ माद्ध्य॑न्दिनस्य मति॒विदो॑ मति॒विदो॒ माद्ध्य॑न्दिनस्य॒ सव॑नस्य । \newline
61. म॒ति॒विद॒ इति॑ मति - विदः॑ । \newline
62. माद्ध्य॑न्दिनस्य॒ सव॑नस्य॒ सव॑नस्य॒ माद्ध्य॑न्दिनस्य॒ माद्ध्य॑न्दिनस्य॒ सव॑नस्य त्रि॒ष्टुप्छ॑न्दस स्त्रि॒ष्टुप्छ॑न्दसः॒ सव॑नस्य॒ माद्ध्य॑न्दिनस्य॒ माद्ध्य॑न्दिनस्य॒ सव॑नस्य त्रि॒ष्टुप्छ॑न्दसः । \newline
63. सव॑नस्य त्रि॒ष्टुप्छ॑न्दस स्त्रि॒ष्टुप्छ॑न्दसः॒ सव॑नस्य॒ सव॑नस्य त्रि॒ष्टुप्छ॑न्दस॒ इन्द्र॑पीत॒स्ये न्द्र॑पीतस्य त्रि॒ष्टुप्छ॑न्दसः॒ सव॑नस्य॒ सव॑नस्य त्रि॒ष्टुप्छ॑न्दस॒ इन्द्र॑पीतस्य । \newline
64. त्रि॒ष्टुप्छ॑न्दस॒ इन्द्र॑पीत॒स्ये न्द्र॑पीतस्य त्रि॒ष्टुप्छ॑न्दस स्त्रि॒ष्टुप्छ॑न्दस॒ इन्द्र॑पीतस्य॒ नरा॒शꣳस॑पीतस्य॒ नरा॒शꣳस॑पीत॒स्ये न्द्र॑पीतस्य त्रि॒ष्टुप्छ॑न्दस स्त्रि॒ष्टुप्छ॑न्दस॒ इन्द्र॑पीतस्य॒ नरा॒शꣳस॑पीतस्य । \newline
65. त्रि॒ष्टुप्छ॑न्दस॒ इति॑ त्रि॒ष्टुप् - छ॒न्द॒सः॒ । \newline
66. इन्द्र॑पीतस्य॒ नरा॒शꣳस॑पीतस्य॒ नरा॒शꣳस॑पीत॒स्ये न्द्र॑पीत॒स्ये न्द्र॑पीतस्य॒ नरा॒शꣳस॑पीतस्य पि॒तृपी॑तस्य पि॒तृपी॑तस्य॒ नरा॒शꣳस॑पीत॒स्ये न्द्र॑पीत॒स्ये न्द्र॑पीतस्य॒ नरा॒शꣳस॑पीतस्य पि॒तृपी॑तस्य । \newline
67. इन्द्र॑पीत॒स्येतीन्द्र॑ - पी॒त॒स्य॒ । \newline
68. नरा॒शꣳस॑पीतस्य पि॒तृपी॑तस्य पि॒तृपी॑तस्य॒ नरा॒शꣳस॑पीतस्य॒ नरा॒शꣳस॑पीतस्य पि॒तृपी॑तस्य॒ मधु॑मतो॒ मधु॑मतः पि॒तृपी॑तस्य॒ नरा॒शꣳस॑पीतस्य॒ नरा॒शꣳस॑पीतस्य पि॒तृपी॑तस्य॒ मधु॑मतः । \newline
69. नरा॒शꣳस॑पीत॒स्येति॒ नरा॒शꣳस॑ - पी॒त॒स्य॒ । \newline
\pagebreak
\markright{ TS 3.2.5.3  \hfill https://www.vedavms.in \hfill}

\section{ TS 3.2.5.3 }

\textbf{TS 3.2.5.3 } \newline
\textbf{Samhita Paata} \newline

पि॒तृपी॑तस्य॒ मधु॑मत॒ उप॑हूत॒स्योप॑हूतो भक्षयाम्यादि॒त्यव॑द्-गणस्य सोम देव ते मति॒विद॑स्तृ॒तीय॑स्य॒ सव॑नस्य॒ जग॑तीछन्दस॒ इन्द्र॑पीतस्य॒ नरा॒शꣳ स॑पीतस्य पि॒तृपी॑तस्य॒ मधु॑मत॒ उप॑हूत॒स्योप॑हूतो भक्षयामि ॥ आप्या॑यस्व॒ समे॑तु ते वि॒श्वतः॑ सोम॒ वृष्णि॑यं । भवा॒ वाज॑स्य सङ्ग॒थे ॥ हिन्व॑ मे॒ गात्रा॑ हरिवो ग॒णान् मे॒ मा विती॑तृषः । शि॒वो मे॑ सप्त॒र्॒.षीनुप॑ तिष्ठस्व॒ मा मेऽवा॒ङ्नाभि॒मति॑ - [  ] \newline

\textbf{Pada Paata} \newline

पि॒तृपी॑त॒स्येति॑ पि॒तृ - पी॒त॒स्य॒ । मधु॑मत॒ इति॒ मधु॑ - म॒तः॒ । उप॑हूत॒स्येत्युप॑ - हू॒त॒स्य॒ । उप॑हूत॒ इत्युप॑ - हू॒तः॒ । भ॒क्ष॒या॒मि॒ । आ॒दि॒त्यव॑द्गण॒स्येत्या॑दि॒त्यव॑त् - ग॒ण॒स्य॒ । सो॒म॒ । दे॒व॒ । ते॒ । म॒ति॒विद॒ इति॑ मति - विदः॑ । तृ॒तीय॑स्य । सव॑नस्य । जग॑तीछन्दस॒ इति॒ जग॑ती - छ॒न्द॒सः॒ । इन्द्र॑पीत॒स्येतीन्द्र॑ - पी॒त॒स्य॒ । नरा॒शꣳस॑पीत॒स्येति॒ नरा॒शꣳस॑ - पी॒त॒स्य॒ । पि॒तृपी॑त॒स्येति॑ पि॒तृ - पी॒त॒स्य॒ । मधु॑मत॒ इति॒ मधु॑ - म॒तः॒ । उप॑हूत॒स्येत्युप॑ - हू॒त॒स्य॒ । उप॑हूत॒ इत्युप॑-हू॒तः॒ । भ॒क्ष॒या॒मि॒ ॥ एति॑ । प्या॒य॒स्व॒ । समिति॑ । ए॒तु॒ । ते॒ । वि॒श्वतः॑ । सो॒म॒ । वृष्णि॑यम् ॥ भव॑ । वाज॑स्य । स॒ङ्ग॒थ इति॑ सं - ग॒थे ॥ हिन्व॑ । मे॒ । गात्रा᳚ । ह॒रि॒व॒ इति॑ हरि - वः॒ । ग॒णान् । मे॒ । मा । वीति॑ । ती॒तृ॒षः॒ ॥ शि॒वः । म॒ । स॒प्त॒र्॒.षीनिति॑ सप्त-ऋ॒षीन् । उपेति॑ । ति॒ष्ठ॒स्व॒ । मा । मे॒ । अवाङ्॑ । नाभि᳚म् । अतीति॑ ।  \newline


\textbf{Krama Paata} \newline

पि॒तृपी॑तस्य॒ मधु॑मतः । पि॒तृपी॑त॒स्येति॑ पि॒तृ - पी॒त॒स्य॒ । मधु॑मत॒ उप॑हूतस्य । मधु॑मत॒ इति॒ मधु॑ - म॒तः॒ । उप॑हूत॒स्योप॑हूतः । उप॑हूत॒स्येत्युप॑ - हू॒त॒स्य॒ । उप॑हूतो भक्षयामि । उप॑हूत॒ इत्युप॑ - हू॒तः॒ । भ॒क्ष॒या॒म्या॒दि॒त्यव॑द्गणस्य । आ॒दि॒त्यव॑द्गणस्य सोम । आ॒दि॒त्यव॑द्गण॒स्येत्या॑दि॒त्यव॑त् - ग॒ण॒स्य॒ । सो॒म॒ दे॒व॒ । दे॒व॒ ते॒ । ते॒ म॒ति॒विदः॑ । म॒ति॒विद॑स्तृ॒तीय॑स्य । म॒ति॒विद॒ इति॑ मति - विदः॑ । तृ॒तीय॑स्य॒ सव॑नस्य । सव॑नस्य॒ जग॑तीछन्दसः । जग॑तीछन्दस॒ इन्द्र॑पीतस्य । जग॑तीछन्दस॒ इति॒ जग॑ती - छ॒न्द॒सः॒ । इन्द्र॑पीतस्य॒ नरा॒शꣳस॑पीतस्य । इन्द्र॑पीत॒स्येतीन्द्र॑ - पी॒त॒स्य॒ । नरा॒शꣳस॑पीतस्य पि॒तृपी॑तस्य । नरा॒शꣳस॑पीत॒स्येति॒ नरा॒शꣳस॑ - पी॒त॒स्य॒ । पि॒तृपी॑तस्य॒ मधु॑मतः । पि॒तृपी॑त॒स्येति॑ पि॒तृ - पी॒त॒स्य॒ । मधु॑मत॒ उप॑हूतस्य । मधु॑मत॒ इति॒ मधु॑ - म॒तः॒ । उप॑हूत॒स्योप॑हूतः । उप॑हूत॒स्येत्युप॑ - हू॒त॒स्य॒ । उप॑हूतो भक्षयामि । उप॑हूत॒ इत्युप॑ - हू॒तः॒ । भ॒क्ष॒या॒मीति॑ भक्षयामि ॥ आ प्या॑यस्व । प्या॒य॒स्व॒ सम् । समे॑तु । ए॒तु॒ ते॒ । ते॒ वि॒श्वतः॑ । वि॒श्वतः॑ सोम । सो॒म॒ वृष्णि॑यम् । वृष्णि॑य॒मिति॒ वृष्णि॑यम् ॥ भवा॒ वाज॑स्य । वाज॑स्य सङ्ग॒थे । स॒ङ्ग॒थ इति॑ सङ्ग॒थे ॥ हिन्व॑ मे । मे॒ गात्रा᳚ । गात्रा॑ हरिवः । ह॒रि॒वो॒ ग॒णान् । ह॒रि॒व॒ इति॑ हरि - वः॒ । ग॒णान् मे᳚ । मे॒ मा । मा वि । वि ती॑तृषः । ती॒तृ॒ष॒ इति॑ तीतृषः ॥ शि॒वो मे᳚ । मे॒ स॒प्त॒र्.॒षीन् । स॒प्त॒र्.॒षीनुप॑ । स॒प्त॒र्.॒षीनिति॑ सप्त - ऋ॒षीन् । उप॑ तिष्ठस्व । ति॒ष्ठ॒स्व॒ मा । मा मे᳚ । मे ऽवाङ्॑ । अवा॒ङ् नाभि᳚म् । नाभि॒मति॑ । अति॑ गाः \newline

\textbf{Jatai Paata} \newline

1. पि॒तृपी॑तस्य॒ मधु॑मतो॒ मधु॑मतः पि॒तृपी॑तस्य पि॒तृपी॑तस्य॒ मधु॑मतः । \newline
2. पि॒तृपी॑त॒स्येति॑ पि॒तृ - पी॒त॒स्य॒ । \newline
3. मधु॑मत॒ उप॑हूत॒ स्योप॑हूतस्य॒ मधु॑मतो॒ मधु॑मत॒ उप॑हूतस्य । \newline
4. मधु॑मत॒ इति॒ मधु॑ - म॒तः॒ । \newline
5. उप॑हूत॒ स्योप॑हूत॒ उप॑हूत॒ उप॑हूत॒ स्योप॑हूत॒ स्योप॑हूतः । \newline
6. उप॑हूत॒स्येत्युप॑ - हू॒त॒स्य॒ । \newline
7. उप॑हूतो भक्षयामि भक्षया॒ म्युप॑हूत॒ उप॑हूतो भक्षयामि । \newline
8. उप॑हूत॒ इत्युप॑ - हू॒तः॒ । \newline
9. भ॒क्ष॒या॒ म्या॒दि॒त्यव॑द्‍गणस्या दि॒त्यव॑द्‍गणस्य भक्षयामि भक्षया म्यादि॒त्यव॑द्‍गणस्य । \newline
10. आ॒दि॒त्यव॑द्‍गणस्य सोम सोमा दि॒त्यव॑द्‍गणस्या दि॒त्यव॑द्‍गणस्य सोम । \newline
11. आ॒दि॒त्यव॑द्‍गण॒स्येत्या॑दि॒त्यव॑त् - ग॒ण॒स्य॒ । \newline
12. सो॒म॒ दे॒व॒ दे॒व॒ सो॒म॒ सो॒म॒ दे॒व॒ । \newline
13. दे॒व॒ ते॒ ते॒ दे॒व॒ दे॒व॒ ते॒ । \newline
14. ते॒ म॒ति॒विदो॑ मति॒विद॑ स्ते ते मति॒विदः॑ । \newline
15. म॒ति॒विद॑ स्तृ॒तीय॑स्य तृ॒तीय॑स्य मति॒विदो॑ मति॒विद॑ स्तृ॒तीय॑स्य । \newline
16. म॒ति॒विद॒ इति॑ मति - विदः॑ । \newline
17. तृ॒तीय॑स्य॒ सव॑नस्य॒ सव॑नस्य तृ॒तीय॑स्य तृ॒तीय॑स्य॒ सव॑नस्य । \newline
18. सव॑नस्य॒ जग॑तीछन्दसो॒ जग॑तीछन्दसः॒ सव॑नस्य॒ सव॑नस्य॒ जग॑तीछन्दसः । \newline
19. जग॑तीछन्दस॒ इन्द्र॑पीत॒स्ये न्द्र॑पीतस्य॒ जग॑तीछन्दसो॒ जग॑तीछन्दस॒ इन्द्र॑पीतस्य । \newline
20. जग॑तीछन्दस॒ इति॒ जग॑ती - छ॒न्द॒सः॒ । \newline
21. इन्द्र॑पीतस्य॒ नरा॒शꣳस॑पीतस्य॒ नरा॒शꣳस॑पीत॒ स्येन्द्र॑पीत॒स्ये न्द्र॑पीतस्य॒ नरा॒शꣳस॑पीतस्य । \newline
22. इन्द्र॑पीत॒स्येतीन्द्र॑ - पी॒त॒स्य॒ । \newline
23. नरा॒शꣳस॑पीतस्य पि॒तृपी॑तस्य पि॒तृपी॑तस्य॒ नरा॒शꣳस॑पीतस्य॒ नरा॒शꣳस॑पीतस्य पि॒तृपी॑तस्य । \newline
24. नरा॒शꣳस॑पीत॒स्येति॒ नरा॒शꣳस॑ - पी॒त॒स्य॒ । \newline
25. पि॒तृपी॑तस्य॒ मधु॑मतो॒ मधु॑मतः पि॒तृपी॑तस्य पि॒तृपी॑तस्य॒ मधु॑मतः । \newline
26. पि॒तृपी॑त॒स्येति॑ पि॒तृ - पी॒त॒स्य॒ । \newline
27. मधु॑मत॒ उप॑हूत॒ स्योप॑हूतस्य॒ मधु॑मतो॒ मधु॑मत॒ उप॑हूतस्य । \newline
28. मधु॑मत॒ इति॒ मधु॑ - म॒तः॒ । \newline
29. उप॑हूत॒ स्योप॑हूत॒ उप॑हूत॒ उप॑हूत॒ स्योप॑हूत॒ स्योप॑हूतः । \newline
30. उप॑हूत॒स्येत्युप॑ - हू॒त॒स्य॒ । \newline
31. उप॑हूतो भक्षयामि भक्षया॒ म्युप॑हूत॒ उप॑हूतो भक्षयामि । \newline
32. उप॑हूत॒ इत्युप॑ - हू॒तः॒ । \newline
33. भ॒क्ष॒या॒मीति॑ भक्षयामि । \newline
34. आ प्या॑यस्व प्याय॒स्वा प्या॑यस्व । \newline
35. प्या॒य॒स्व॒ सꣳ सम् प्या॑यस्व प्यायस्व॒ सम् । \newline
36. स मे᳚त्वेतु॒ सꣳ स मे॑तु । \newline
37. ए॒तु॒ ते॒ त॒ ए॒त्वे॒तु॒ ते॒ । \newline
38. ते॒ वि॒श्वतो॑ वि॒श्वत॑ स्ते ते वि॒श्वतः॑ । \newline
39. वि॒श्वतः॑ सोम सोम वि॒श्वतो॑ वि॒श्वतः॑ सोम । \newline
40. सो॒म॒ वृष्णि॑यं॒ ॅवृष्णि॑यꣳ सोम सोम॒ वृष्णि॑यम् । \newline
41. वृष्णि॑य॒मिति॒ वृष्णि॑यम् । \newline
42. भवा॒ वाज॑स्य॒ वाज॑स्य॒ भव॒ भवा॒ वाज॑स्य । \newline
43. वाज॑स्य सङ्ग॒थे स॑ङ्ग॒थे वाज॑स्य॒ वाज॑स्य सङ्ग॒थे । \newline
44. स॒ङ्ग॒थ इति॑ सं - ग॒थे । \newline
45. हिन्व॑ मे मे॒ हिन्व॒ हिन्व॑ मे । \newline
46. मे॒ गात्रा॒ गात्रा॑ मे मे॒ गात्रा᳚ । \newline
47. गात्रा॑ हरिवो हरिवो॒ गात्रा॒ गात्रा॑ हरिवः । \newline
48. ह॒रि॒वो॒ ग॒णान् ग॒णान्. ह॑रिवो हरिवो ग॒णान् । \newline
49. ह॒रि॒व॒ इति॑ हरि - वः॒ । \newline
50. ग॒णान् मे॑ मे ग॒णान् ग॒णान् मे᳚ । \newline
51. मे॒ मा मा मे॑ मे॒ मा । \newline
52. मा वि वि मा मा वि । \newline
53. वि ती॑तृष स्तीतृषो॒ वि वि ती॑तृषः । \newline
54. ती॒तृ॒ष॒ इति॑ तीतृषः । \newline
55. शि॒वो मे॑ मे शि॒वः शि॒वो मे᳚ । \newline
56. मे॒ स॒प्त॒र्॒.षीन् थ्स॑प्त॒र्॒.षीन् मे॑ मे सप्त॒र्॒.षीन् । \newline
57. स॒प्त॒र्॒.षी नुपोप॑ सप्त॒र्॒.षीन् थ्स॑प्त॒र्॒.षी नुप॑ । \newline
58. स॒प्त॒र्॒.षीनिति॑ सप्त - ऋ॒षीन् । \newline
59. उप॑ तिष्ठस्व तिष्ठ॒ स्वोपोप॑ तिष्ठस्व । \newline
60. ति॒ष्ठ॒स्व॒ मा मा ति॑ष्ठस्व तिष्ठस्व॒ मा । \newline
61. मा मे॑ मे॒ मा मा मे᳚ । \newline
62. मे ऽवा॒ङ् ङवा᳚ङ् मे॒ मे ऽवाङ्॑ । \newline
63. अवा॒ङ् नाभि॒म् नाभि॒ मवा॒ङ् ङवा॒ङ् नाभि᳚म् । \newline
64. नाभि॒ मत्यति॒ नाभि॒म् नाभि॒ मति॑ । \newline
65. अति॑ गा गा॒ अत्यति॑ गाः । \newline

\textbf{Ghana Paata } \newline

1. पि॒तृपी॑तस्य॒ मधु॑मतो॒ मधु॑मतः पि॒तृपी॑तस्य पि॒तृपी॑तस्य॒ मधु॑मत॒ उप॑हूत॒ स्योप॑हूतस्य॒ मधु॑मतः पि॒तृपी॑तस्य पि॒तृपी॑तस्य॒ मधु॑मत॒ उप॑हूतस्य । \newline
2. पि॒तृपी॑त॒स्येति॑ पि॒तृ - पी॒त॒स्य॒ । \newline
3. मधु॑मत॒ उप॑हूत॒ स्योप॑हूतस्य॒ मधु॑मतो॒ मधु॑मत॒ उप॑हूत॒ स्योप॑हूत॒ उप॑हूत॒ उप॑हूतस्य॒ मधु॑मतो॒ मधु॑मत॒ उप॑हूत॒ स्योप॑हूतः । \newline
4. मधु॑मत॒ इति॒ मधु॑ - म॒तः॒ । \newline
5. उप॑हूत॒ स्योप॑हूत॒ उप॑हूत॒ उप॑हूत॒ स्योप॑हूत॒ स्योप॑हूतो भक्षयामि भक्षया॒ म्युप॑हूत॒ उप॑हूत॒ स्योप॑हूत॒ स्योप॑हूतो भक्षयामि । \newline
6. उप॑हूत॒स्येत्युप॑ - हू॒त॒स्य॒ । \newline
7. उप॑हूतो भक्षयामि भक्षया॒ म्युप॑हूत॒ उप॑हूतो भक्षया म्यादि॒त्यव॑द्‍गणस्या दि॒त्यव॑द्‍गणस्य भक्षया॒ म्युप॑हूत॒ उप॑हूतो भक्षया म्यादि॒त्यव॑द्‍गणस्य । \newline
8. उप॑हूत॒ इत्युप॑ - हू॒तः॒ । \newline
9. भ॒क्ष॒या॒ म्या॒दि॒त्यव॑द्‍गणस्या दि॒त्यव॑द्‍गणस्य भक्षयामि भक्षया म्यादि॒त्यव॑द्‍गणस्य सोम सोमा दि॒त्यव॑द्‍गणस्य भक्षयामि भक्षया म्यादि॒त्यव॑द्‍गणस्य सोम । \newline
10. आ॒दि॒त्यव॑द्‍गणस्य सोम सोमादि॒त्यव॑द्‍गणस्या दि॒त्यव॑द्‍गणस्य सोम देव देव सोमादि॒त्यव॑द्‍गणस्या दि॒त्यव॑द्‍गणस्य सोम देव । \newline
11. आ॒दि॒त्यव॑द्‍गण॒स्येत्या॑दि॒त्यव॑त् - ग॒ण॒स्य॒ । \newline
12. सो॒म॒ दे॒व॒ दे॒व॒ सो॒म॒ सो॒म॒ दे॒व॒ ते॒ ते॒ दे॒व॒ सो॒म॒ सो॒म॒ दे॒व॒ ते॒ । \newline
13. दे॒व॒ ते॒ ते॒ दे॒व॒ दे॒व॒ ते॒ म॒ति॒विदो॑ मति॒विद॑ स्ते देव देव ते मति॒विदः॑ । \newline
14. ते॒ म॒ति॒विदो॑ मति॒विद॑स्ते ते मति॒विद॑ स्तृ॒तीय॑स्य तृ॒तीय॑स्य मति॒विद॑ स्ते ते मति॒विद॑ स्तृ॒तीय॑स्य । \newline
15. म॒ति॒विद॑ स्तृ॒तीय॑स्य तृ॒तीय॑स्य मति॒विदो॑ मति॒विद॑ स्तृ॒तीय॑स्य॒ सव॑नस्य॒ सव॑नस्य तृ॒तीय॑स्य मति॒विदो॑ मति॒विद॑ स्तृ॒तीय॑स्य॒ सव॑नस्य । \newline
16. म॒ति॒विद॒ इति॑ मति - विदः॑ । \newline
17. तृ॒तीय॑स्य॒ सव॑नस्य॒ सव॑नस्य तृ॒तीय॑स्य तृ॒तीय॑स्य॒ सव॑नस्य॒ जग॑तीछन्दसो॒ जग॑तीछन्दसः॒ सव॑नस्य तृ॒तीय॑स्य तृ॒तीय॑स्य॒ सव॑नस्य॒ जग॑तीछन्दसः । \newline
18. सव॑नस्य॒ जग॑तीछन्दसो॒ जग॑तीछन्दसः॒ सव॑नस्य॒ सव॑नस्य॒ जग॑तीछन्दस॒ इन्द्र॑पीत॒स्ये न्द्र॑पीतस्य॒ जग॑तीछन्दसः॒ सव॑नस्य॒ सव॑नस्य॒ जग॑तीछन्दस॒ इन्द्र॑पीतस्य । \newline
19. जग॑तीछन्दस॒ इन्द्र॑पीत॒स्ये न्द्र॑पीतस्य॒ जग॑तीछन्दसो॒ जग॑तीछन्दस॒ इन्द्र॑पीतस्य॒ नरा॒शꣳस॑पीतस्य॒ नरा॒शꣳस॑पीत॒स्ये न्द्र॑पीतस्य॒ जग॑तीछन्दसो॒ जग॑तीछन्दस॒ इन्द्र॑पीतस्य॒ नरा॒शꣳस॑पीतस्य । \newline
20. जग॑तीछन्दस॒ इति॒ जग॑ती - छ॒न्द॒सः॒ । \newline
21. इन्द्र॑पीतस्य॒ नरा॒शꣳस॑पीतस्य॒ नरा॒शꣳस॑पीत॒स्ये न्द्र॑पीत॒स्ये न्द्र॑पीतस्य॒ नरा॒शꣳस॑पीतस्य पि॒तृपी॑तस्य पि॒तृपी॑तस्य॒ नरा॒शꣳस॑पीत॒स्ये न्द्र॑पीत॒स्ये न्द्र॑पीतस्य॒ नरा॒शꣳस॑पीतस्य पि॒तृपी॑तस्य । \newline
22. इन्द्र॑पीत॒स्येतीन्द्र॑ - पी॒त॒स्य॒ । \newline
23. नरा॒शꣳस॑पीतस्य पि॒तृपी॑तस्य पि॒तृपी॑तस्य॒ नरा॒शꣳस॑पीतस्य॒ नरा॒शꣳस॑पीतस्य पि॒तृपी॑तस्य॒ मधु॑मतो॒ मधु॑मतः पि॒तृपी॑तस्य॒ नरा॒शꣳस॑पीतस्य॒ नरा॒शꣳस॑पीतस्य पि॒तृपी॑तस्य॒ मधु॑मतः । \newline
24. नरा॒शꣳस॑पीत॒स्येति॒ नरा॒शꣳस॑ - पी॒त॒स्य॒ । \newline
25. पि॒तृपी॑तस्य॒ मधु॑मतो॒ मधु॑मतः पि॒तृपी॑तस्य पि॒तृपी॑तस्य॒ मधु॑मत॒ उप॑हूत॒ स्योप॑हूतस्य॒ मधु॑मतः पि॒तृपी॑तस्य पि॒तृपी॑तस्य॒ मधु॑मत॒ उप॑हूतस्य । \newline
26. पि॒तृपी॑त॒स्येति॑ पि॒तृ - पी॒त॒स्य॒ । \newline
27. मधु॑मत॒ उप॑हूत॒ स्योप॑हूतस्य॒ मधु॑मतो॒ मधु॑मत॒ उप॑हूत॒ स्योप॑हूत॒ उप॑हूत॒ उप॑हूतस्य॒ मधु॑मतो॒ मधु॑मत॒ उप॑हूत॒ स्योप॑हूतः॒ । \newline
28. मधु॑मत॒ इति॒ मधु॑ - म॒तः॒ । \newline
29. उप॑हूत॒ स्योप॑हूत॒ उप॑हूत॒ उप॑हूत॒ स्योप॑हूत॒ स्योप॑हूतो भक्षयामि भक्षया॒ म्युप॑हूत॒ उप॑हूत॒ स्योप॑हूत॒ स्योप॑हूतो भक्षयामि । \newline
30. उप॑हूत॒स्येत्युप॑ - हू॒त॒स्य॒ । \newline
31. उप॑हूतो भक्षयामि भक्षया॒ म्युप॑हूत॒ उप॑हूतो भक्षयामि । \newline
32. उप॑हूत॒ इत्युप॑-हू॒तः॒ । \newline
33. भ॒क्ष॒या॒मीति॑ भक्षयामि । \newline
34. आ प्या॑यस्व प्याय॒स्वा प्या॑यस्व॒ सꣳ सम् प्या॑य॒स्वा प्या॑यस्व॒ सम् । \newline
35. प्या॒य॒स्व॒ सꣳ सम् प्या॑यस्व प्यायस्व॒ स मे᳚त्वेतु॒ सम् प्या॑यस्व प्यायस्व॒ स मे॑तु । \newline
36. स मे᳚त्वेतु॒ सꣳ स मे॑तु ते त एतु॒ सꣳ स मे॑तु ते । \newline
37. ए॒तु॒ ते॒ त॒ ए॒त्वे॒तु॒ ते॒ वि॒श्वतो॑ वि॒श्वत॑ स्त एत्वेतु ते वि॒श्वतः॑ । \newline
38. ते॒ वि॒श्वतो॑ वि॒श्वत॑ स्ते ते वि॒श्वतः॑ सोम सोम वि॒श्वत॑ स्ते ते वि॒श्वतः॑ सोम । \newline
39. वि॒श्वतः॑ सोम सोम वि॒श्वतो॑ वि॒श्वतः॑ सोम॒ वृष्णि॑यं॒ ॅवृष्णि॑यꣳ सोम वि॒श्वतो॑ वि॒श्वतः॑ सोम॒ वृष्णि॑यम् । \newline
40. सो॒म॒ वृष्णि॑यं॒ ॅवृष्णि॑यꣳ सोम सोम॒ वृष्णि॑यम् । \newline
41. वृष्णि॑य॒मिति॒ वृष्णि॑यम् । \newline
42. भवा॒ वाज॑स्य॒ वाज॑स्य॒ भव॒ भवा॒ वाज॑स्य सङ्ग॒थे स॑ङ्ग॒थे वाज॑स्य॒ भव॒ भवा॒ वाज॑स्य सङ्ग॒थे । \newline
43. वाज॑स्य सङ्ग॒थे स॑ङ्ग॒थे वाज॑स्य॒ वाज॑स्य सङ्ग॒थे । \newline
44. स॒ङ्ग॒थ इति॑ सं - ग॒थे । \newline
45. हिन्व॑ मे मे॒ हिन्व॒ हिन्व॑ मे॒ गात्रा॒ गात्रा॑ मे॒ हिन्व॒ हिन्व॑ मे॒ गात्रा᳚ । \newline
46. मे॒ गात्रा॒ गात्रा॑ मे मे॒ गात्रा॑ हरिवो हरिवो॒ गात्रा॑ मे मे॒ गात्रा॑ हरिवः । \newline
47. गात्रा॑ हरिवो हरिवो॒ गात्रा॒ गात्रा॑ हरिवो ग॒णान् ग॒णान्. ह॑रिवो॒ गात्रा॒ गात्रा॑ हरिवो ग॒णान् । \newline
48. ह॒रि॒वो॒ ग॒णान् ग॒णान्. ह॑रिवो हरिवो ग॒णान् मे॑ मे ग॒णान्. ह॑रिवो हरिवो ग॒णान् मे᳚ । \newline
49. ह॒रि॒व॒ इति॑ हरि - वः॒ । \newline
50. ग॒णान् मे॑ मे ग॒णान् ग॒णान् मे॒ मा मा मे॑ ग॒णान् ग॒णान् मे॒ मा । \newline
51. मे॒ मा मा मे॑ मे॒ मा वि वि मा मे॑ मे॒ मा वि । \newline
52. मा वि वि मा मा वि ती॑तृष स्तीतृषो॒ वि मा मा वि ती॑तृषः । \newline
53. वि ती॑तृष स्तीतृषो॒ वि वि ती॑तृषः । \newline
54. ती॒तृ॒ष॒ इति॑ तीतृषः । \newline
55. शि॒वो मे॑ मे शि॒वः शि॒वो मे॑ सप्त॒र्॒.षीन् थ्स॑प्त॒र्॒.षीन् मे॑ शि॒वः शि॒वो मे॑ सप्त॒र्॒.षीन् । \newline
56. मे॒ स॒प्त॒र्॒.षीन् थ्स॑प्त॒र्॒.षीन् मे॑ मे सप्त॒र्॒.षी नुपोप॑ सप्त॒र्॒.षीन् मे॑ मे सप्त॒र्॒.षी नुप॑ । \newline
57. स॒प्त॒र्॒.षी नुपोप॑ सप्त॒र्॒.षीन् थ्स॑प्त॒र्॒.षी नुप॑ तिष्ठस्व तिष्ठ॒स्वोप॑ सप्त॒र्॒.षीन् थ्स॑प्त॒र्॒.षी नुप॑ तिष्ठस्व । \newline
58. स॒प्त॒र्॒.षीनिति॑ सप्त - ऋ॒षीन् । \newline
59. उप॑ तिष्ठस्व तिष्ठ॒ स्वोपोप॑ तिष्ठस्व॒ मा मा ति॑ष्ठ॒ स्वोपोप॑ तिष्ठस्व॒ मा । \newline
60. ति॒ष्ठ॒स्व॒ मा मा ति॑ष्ठस्व तिष्ठस्व॒ मा मे॑ मे॒ मा ति॑ष्ठस्व तिष्ठस्व॒ मा मे᳚ । \newline
61. मा मे॑ मे॒ मा मा मे ऽवा॒ङ् ङवा᳚ङ् मे॒ मा मा मे ऽवाङ्॑ । \newline
62. मे ऽवा॒ङ् ङवा᳚ङ् मे॒ मे ऽवा॒ङ् नाभि॒म् नाभि॒ मवा᳚ङ् मे॒ मे ऽवा॒ङ् नाभि᳚म् । \newline
63. अवा॒ङ् नाभि॒म् नाभि॒ मवा॒ङ् ङवा॒ङ् नाभि॒ मत्यति॒ नाभि॒ मवा॒ङ् ङवा॒ङ् नाभि॒ मति॑ । \newline
64. नाभि॒ मत्यति॒ नाभि॒म् नाभि॒ मति॑ गा गा॒ अति॒ नाभि॒म् नाभि॒ मति॑ गाः । \newline
65. अति॑ गा गा॒ अत्यति॑ गाः । \newline
\pagebreak
\markright{ TS 3.2.5.4  \hfill https://www.vedavms.in \hfill}

\section{ TS 3.2.5.4 }

\textbf{TS 3.2.5.4 } \newline
\textbf{Samhita Paata} \newline

गाः ॥ अपा॑म॒ सोम॑म॒मृता॑ अभू॒माऽद॑र्श्म॒ ज्योति॒रवि॑दाम दे॒वान् । किम॒स्मान् कृ॑णव॒दरा॑तिः॒ किमु॑ धू॒र्तिर॑मृत॒ मर्त्य॑स्य ॥यन्म॑ आ॒त्मनो॑ मि॒न्दाऽभू॑द॒ग्निस्तत् पुन॒राऽहा᳚र्जा॒तवे॑दा॒ विच॑र्.षणिः ॥ पुन॑र॒ग्निश्चक्षु॑रदा॒त्-पुन॒रिन्द्रो॒ बृह॒स्पतिः॑ । पुन॑र्मे अश्विना यु॒वं चक्षु॒रा ध॑त्तम॒क्ष्योः ॥ इ॒ष्टय॑जुषस्ते देव सोम स्तु॒तस्तो॑मस्य - [  ] \newline

\textbf{Pada Paata} \newline

गाः॒ ॥ अपा॑म । सोम᳚म् । अ॒मृताः᳚ । अ॒भू॒म॒ । अद॑र्श्म । ज्योतिः॑ । अवि॑दाम । दे॒वान् ॥ किम् । अ॒स्मान् । कृ॒ण॒व॒त् । अरा॑तिः । किम् । उ॒ । धू॒र्तिः । अ॒मृ॒त॒ । मर्त्य॑स्य ॥ यत् । मे॒ । आ॒त्मनः॑ । मि॒न्दा । अभू᳚त् । अ॒ग्निः । तत् । पुनः॑ । एति॑ । अ॒हाः॒ । जा॒तवे॑दा॒ इति॑ जा॒त - वे॒दाः॒ । विच॑र्.षणि॒रिति॒ वि - च॒र्॒.ष॒णिः॒ ॥ पुनः॑ । अ॒ग्निः । चक्षुः॑ । अ॒दा॒त् । पुनः॑ । इन्द्रः॑ । बृह॒स्पतिः॑ ॥ पुनः॑ । मे॒ । अ॒श्वि॒ना॒ । यु॒वम् । चक्षुः॑ । एति॑ । ध॒त्त॒म् । अ॒क्ष्योः ॥ इ॒ष्टय॑जुष॒ इती॒ष्ट-य॒जु॒षः॒ । ते॒ । दे॒व॒ । सो॒म॒ । स्तु॒तस्तो॑म॒स्येति॑ स्तु॒त - स्तो॒म॒स्य॒ ।  \newline


\textbf{Krama Paata} \newline

गा॒ इति॑ गाः ॥ अपा॑म॒ सोम᳚म् । सोम॑म॒मृताः᳚ । अ॒मृता॑ अभूम । अ॒भू॒माद॑र्श्म । अद॑र्श्म॒ ज्योतिः॑ । ज्योति॒रवि॑दाम । अवि॑दाम दे॒वान् । दे॒वानिति॑ दे॒वान् ॥ किम॒स्मान् । अ॒स्मान् कृ॑णवत् । कृ॒ण॒व॒दरा॑तिः । अरा॑तिः॒ किम् । किमु॑ । उ॒ धू॒र्तिः । धू॒र्तिर॑मृत । अ॒मृ॒त॒ मर्त्य॑स्य । मर्त्य॒स्येति॒ मर्त्य॑स्य ॥ यन् मे᳚ । म॒ आ॒त्मनः॑ । आ॒त्मनो॑ मि॒न्दा । मि॒न्दा ऽभू᳚त् । अभू॑द॒ग्निः । अ॒ग्निस्तत् । तत् पुनः॑ । पुन॒रा । आ ऽहाः᳚ । अ॒हा॒र् जा॒तवे॑दाः । जा॒तवे॑दा॒ विच॑र्.षणिः । जा॒तवे॑दा॒ इति॑ जा॒त - वे॒दाः॒ । विच॑र्.षणि॒रिति॒ वि - च॒र्॒ष॒णिः॒ ॥ पुन॑र॒ग्निः । अ॒ग्निश्चक्षुः॑ । चक्षु॑रदात् । अ॒दा॒त् पुनः॑ । पुन॒रिन्द्रः॑ । इन्द्रो॒ बृह॒स्पतिः॑ । बृह॒स्पति॒रिति॑ बृह॒स्पतिः॑ ॥ पुन॑र् मे । मे॒ अ॒श्वि॒ना॒ । अ॒श्वि॒ना॒ यु॒वम् । यु॒वम् चक्षुः॑ । चक्षु॒रा । आ ध॑त्तम् । ध॒त्त॒म॒क्ष्योः । अ॒क्ष्योरित्य॒क्ष्योः ॥ इ॒ष्टय॑जुषस्ते । इ॒ष्टय॑जुष॒ इती॒ष्ट - य॒जु॒षः॒ । ते॒ दे॒व॒ । दे॒व॒ सो॒म॒ । सो॒म॒ स्तु॒तस्तो॑मस्य । स्तु॒तस्तो॑मस्य श॒स्तोक्थ॑स्य । स्तु॒तस्तो॑म॒स्येति॑ स्तु॒त - स्तो॒म॒स्य॒ \newline

\textbf{Jatai Paata} \newline

1. गा॒ इति॑ गाः । \newline
2. अपा॑म॒ सोमꣳ॒॒ सोम॒ मपा॒मा पा॑म॒ सोम᳚म् । \newline
3. सोम॑ म॒मृता॑ अ॒मृताः॒ सोमꣳ॒॒ सोम॑ म॒मृताः᳚ । \newline
4. अ॒मृता॑ अभूमा भूमा॒ मृता॑ अ॒मृता॑ अभूम । \newline
5. अ॒भू॒मा द॒र्श्मा द॑र्श्मा भूमा भू॒मा द॑र्श्म । \newline
6. अद॑र्श्म॒ ज्योति॒र् ज्योति॒ रद॒र्श्मा द॑र्श्म॒ ज्योतिः॑ । \newline
7. ज्योति॒ रवि॑दा॒मा वि॑दाम॒ ज्योति॒र् ज्योति॒ रवि॑दाम । \newline
8. अवि॑दाम दे॒वान् दे॒वा नवि॑दा॒मा वि॑दाम दे॒वान् । \newline
9. दे॒वानिति॑ दे॒वान् । \newline
10. कि म॒स्मा न॒स्मान् किम् कि म॒स्मान् । \newline
11. अ॒स्मान् कृ॑णवत् कृणव द॒स्मा न॒स्मान् कृ॑णवत् । \newline
12. कृ॒ण॒व॒ दरा॑ति॒ ररा॑तिः कृणवत् कृणव॒ दरा॑तिः । \newline
13. अरा॑तिः॒ किम् कि मरा॑ति॒ ररा॑तिः॒ किम् । \newline
14. कि मु॑ वु॒ किम् कि मु॑ । \newline
15. उ॒ धू॒र्तिर् धू॒र्तिरु॑ वु धू॒र्तिः । \newline
16. धू॒र्ति र॑मृता मृत धू॒र्तिर् धू॒र्ति र॑मृत । \newline
17. अ॒मृ॒त॒ मर्त्य॑स्य॒ मर्त्य॑स्या मृता मृत॒ मर्त्य॑स्य । \newline
18. मर्त्य॒स्येति॒ मर्त्य॑स्य । \newline
19. यन् मे॑ मे॒ यद् यन् मे᳚ । \newline
20. म॒ आ॒त्मन॑ आ॒त्मनो॑ मे म आ॒त्मनः॑ । \newline
21. आ॒त्मनो॑ मि॒न्दा मि॒न्दा ऽऽत्मन॑ आ॒त्मनो॑ मि॒न्दा । \newline
22. मि॒न्दा ऽभू॒दभू᳚न् मि॒न्दा मि॒न्दा ऽभू᳚त् । \newline
23. अभू॑ द॒ग्नि र॒ग्नि रभू॒ दभू॑ द॒ग्निः । \newline
24. अ॒ग्नि स्तत् तद॒ग्नि र॒ग्नि स्तत् । \newline
25. तत् पुनः॒ पुन॒ स्तत् तत् पुनः॑ । \newline
26. पुन॒ रा पुनः॒ पुन॒ रा । \newline
27. आ ऽहा॑ रहा॒ रा ऽहाः᳚ । \newline
28. अ॒हा॒र् जा॒तवे॑दा जा॒तवे॑दा अहा रहार् जा॒तवे॑दाः । \newline
29. जा॒तवे॑दा॒ विच॑र्.षणि॒र् विच॑र्.षणिर् जा॒तवे॑दा जा॒तवे॑दा॒ विच॑र्.षणिः । \newline
30. जा॒तवे॑दा॒ इति॑ जा॒त - वे॒दाः॒ । \newline
31. विच॑र्.षणि॒रिति॒ वि - च॒र्॒.ष॒णिः॒ । \newline
32. पुन॑ र॒ग्नि र॒ग्निः पुनः॒ पुन॑ र॒ग्निः । \newline
33. अ॒ग्नि श्चक्षु॒ श्चक्षु॑ र॒ग्नि र॒ग्नि श्चक्षुः॑ । \newline
34. चक्षु॑ रदा ददा॒च् चक्षु॒ श्चक्षु॑ रदात् । \newline
35. अ॒दा॒त् पुनः॒ पुन॑ रदा ददा॒त् पुनः॑ । \newline
36. पुन॒ रिन्द्र॒ इन्द्रः॒ पुनः॒ पुन॒ रिन्द्रः॑ । \newline
37. इन्द्रो॒ बृह॒स्पति॒र् बृह॒स्पति॒ रिन्द्र॒ इन्द्रो॒ बृह॒स्पतिः॑ । \newline
38. बृह॒स्पति॒रिति॑ बृह॒स्पतिः॑ । \newline
39. पुन॑र् मे मे॒ पुनः॒ पुन॑र् मे । \newline
40. मे॒ अ॒श्वि॒ना॒ ऽश्वि॒ना॒ मे॒ मे॒ अ॒श्वि॒ना॒ । \newline
41. अ॒श्वि॒ना॒ यु॒वं ॅयु॒व म॑श्विना ऽश्विना यु॒वम् । \newline
42. यु॒वम् चक्षु॒ श्चक्षु॑र् यु॒वं ॅयु॒वम् चक्षुः॑ । \newline
43. चक्षु॒रा चक्षु॒ श्चक्षु॒रा । \newline
44. आ ध॑त्तम् धत्त॒ मा ध॑त्तम् । \newline
45. ध॒त्त॒ म॒क्ष्यो र॒क्ष्योर् ध॑त्तम् धत्त म॒क्ष्योः । \newline
46. अ॒क्ष्योरित्य॒क्ष्योः । \newline
47. इ॒ष्टय॑जुष स्ते त इ॒ष्टय॑जुष इ॒ष्टय॑जुष स्ते । \newline
48. इ॒ष्टय॑जुष॒ इती॒ष्ट - य॒जु॒षः॒ । \newline
49. ते॒ दे॒व॒ दे॒व॒ ते॒ ते॒ दे॒व॒ । \newline
50. दे॒व॒ सो॒म॒ सो॒म॒ दे॒व॒ दे॒व॒ सो॒म॒ । \newline
51. सो॒म॒ स्तु॒तस्तो॑मस्य स्तु॒तस्तो॑मस्य सोम सोम स्तु॒तस्तो॑मस्य । \newline
52. स्तु॒तस्तो॑मस्य श॒स्तोक्थ॑स्य श॒स्तोक्थ॑स्य स्तु॒तस्तो॑मस्य स्तु॒तस्तो॑मस्य श॒स्तोक्थ॑स्य । \newline
53. स्तु॒तस्तो॑म॒स्येति॑ स्तु॒त - स्तो॒म॒स्य॒ । \newline

\textbf{Ghana Paata } \newline

1. गा॒ इति॑ गाः । \newline
2. अपा॑म॒ सोमꣳ॒॒ सोम॒ मपा॒मापा॑म॒ सोम॑ म॒मृता॑ अ॒मृताः॒ सोम॒ मपा॒मापा॑म॒ सोम॑ म॒मृताः᳚ । \newline
3. सोम॑ म॒मृता॑ अ॒मृताः॒ सोमꣳ॒॒ सोम॑ म॒मृता॑ अभूमा भूमा॒ मृताः॒ सोमꣳ॒॒ सोम॑ म॒मृता॑ अभूम । \newline
4. अ॒मृता॑ अभूमा भूमा॒ मृता॑ अ॒मृता॑ अभू॒मा द॒र्श्मा द॑र्श्मा भूमा॒ मृता॑ अ॒मृता॑ अभू॒मा द॑र्श्म । \newline
5. अ॒भू॒मा द॒र्श्मा द॑र्श्मा भूमा भू॒मा द॑र्श्म॒ ज्योति॒र् ज्योति॒ रद॑र्श्मा भूमा भू॒मा द॑र्श्म॒ ज्योतिः॑ । \newline
6. अद॑र्श्म॒ ज्योति॒र् ज्योति॒ रद॒र्श्मा द॑र्श्म॒ ज्योति॒ रवि॑दा॒मा वि॑दाम॒ ज्योति॒ रद॒र्श्मा द॑र्श्म॒ ज्योति॒ रवि॑दाम । \newline
7. ज्योति॒ रवि॑दा॒मा वि॑दाम॒ ज्योति॒र् ज्योति॒ रवि॑दाम दे॒वान् दे॒वा नवि॑दाम॒ ज्योति॒र् ज्योति॒ रवि॑दाम दे॒वान् । \newline
8. अवि॑दाम दे॒वान् दे॒वा नवि॑दा॒मा वि॑दाम दे॒वान् । \newline
9. दे॒वानिति॑ दे॒वान् । \newline
10. कि म॒स्मा न॒स्मान् किम् कि म॒स्मान् कृ॑णवत् कृणव द॒स्मान् किम् कि म॒स्मान् कृ॑णवत् । \newline
11. अ॒स्मान् कृ॑णवत् कृणव द॒स्मा न॒स्मान् कृ॑णव॒ दरा॑ति॒ ररा॑तिः कृणव द॒स्मा न॒स्मान् कृ॑णव॒ दरा॑तिः । \newline
12. कृ॒ण॒व॒ दरा॑ति॒ ररा॑तिः कृणवत् कृणव॒ दरा॑तिः॒ किम् कि मरा॑तिः कृणवत् कृणव॒ दरा॑तिः॒ किम् । \newline
13. अरा॑तिः॒ किम् कि मरा॑ति॒ ररा॑तिः॒ कि मु॑ वु॒ कि मरा॑ति॒ ररा॑तिः॒ कि मु॑ । \newline
14. कि मु॑ वु॒ किम् कि मु॑ धू॒र्तिर् धू॒र्तिरु॒ किम् कि मु॑ धू॒र्तिः । \newline
15. उ॒ धू॒र्तिर् धू॒र्तिरु॑ वु धू॒र्ति र॑मृता मृत धू॒र्तिरु॑ वु धू॒र्तिर॑मृत । \newline
16. धू॒र्ति र॑मृता मृत धू॒र्तिर् धू॒र्तिर॑मृत॒ मर्त्य॑स्य॒ मर्त्य॑स्यामृत धू॒र्तिर् धू॒र्तिर॑मृत॒ मर्त्य॑स्य । \newline
17. अ॒मृ॒त॒ मर्त्य॑स्य॒ मर्त्य॑स्या मृता मृत॒ मर्त्य॑स्य । \newline
18. मर्त्य॒स्येति॒ मर्त्य॑स्य । \newline
19. यन् मे॑ मे॒ यद् यन् म॑ आ॒त्मन॑ आ॒त्मनो॑ मे॒ यद् यन् म॑ आ॒त्मनः॑ । \newline
20. म॒ आ॒त्मन॑ आ॒त्मनो॑ मे म आ॒त्मनो॑ मि॒न्दा मि॒न्दा ऽऽत्मनो॑ मे म आ॒त्मनो॑ मि॒न्दा । \newline
21. आ॒त्मनो॑ मि॒न्दा मि॒न्दा ऽऽत्मन॑ आ॒त्मनो॑ मि॒न्दा ऽभू॒दभू᳚न् मि॒न्दा ऽऽत्मन॑ आ॒त्मनो॑ मि॒न्दा ऽभू᳚त् । \newline
22. मि॒न्दा ऽभू॒दभू᳚न् मि॒न्दा मि॒न्दा ऽभू॑द॒ग्नि र॒ग्नि रभू᳚न् मि॒न्दा मि॒न्दा ऽभू॑द॒ग्निः । \newline
23. अभू॑ द॒ग्नि र॒ग्नि रभू॒ दभू॑ द॒ग्नि स्तत् तद॒ग्नि रभू॒ दभू॑ द॒ग्नि स्तत् । \newline
24. अ॒ग्नि स्तत् तद॒ग्नि र॒ग्नि स्तत् पुनः॒ पुन॒ स्त द॒ग्नि र॒ग्नि स्तत् पुनः॑ । \newline
25. तत् पुनः॒ पुन॒ स्तत् तत् पुन॒ रा पुन॒ स्तत् तत् पुन॒ रा । \newline
26. पुन॒ रा पुनः॒ पुन॒ रा ऽहा॑ रहा॒ रा पुनः॒ पुन॒ रा ऽहाः᳚ । \newline
27. आ ऽहा॑ रहा॒ रा ऽहा᳚र् जा॒तवे॑दा जा॒तवे॑दा अहा॒ रा ऽहा᳚र् जा॒तवे॑दाः । \newline
28. अ॒हा॒र् जा॒तवे॑दा जा॒तवे॑दा अहा रहार् जा॒तवे॑दा॒ विच॑र्.षणि॒र् विच॑र्.षणिर् जा॒तवे॑दा अहा रहार् जा॒तवे॑दा॒ विच॑र्.षणिः । \newline
29. जा॒तवे॑दा॒ विच॑र्.षणि॒र् विच॑र्.षणिर् जा॒तवे॑दा जा॒तवे॑दा॒ विच॑र्.षणिः । \newline
30. जा॒तवे॑दा॒ इति॑ जा॒त - वे॒दाः॒ । \newline
31. विच॑र्.षणि॒रिति॒ वि - च॒र्॒.ष॒णिः॒ । \newline
32. पुन॑ र॒ग्नि र॒ग्निः पुनः॒ पुन॑ र॒ग्नि श्चक्षु॒ श्चक्षु॑ र॒ग्निः पुनः॒ पुन॑ र॒ग्नि श्चक्षुः॑ । \newline
33. अ॒ग्नि श्चक्षु॒ श्चक्षु॑ र॒ग्नि र॒ग्नि श्चक्षु॑ रदा ददा॒च् चक्षु॑ र॒ग्नि र॒ग्नि श्चक्षु॑ रदात् । \newline
34. चक्षु॑ रदा ददा॒च् चक्षु॒ श्चक्षु॑ रदा॒त् पुनः॒ पुन॑ रदा॒च् चक्षु॒ श्चक्षु॑ रदा॒त् पुनः॑ । \newline
35. अ॒दा॒त् पुनः॒ पुन॑ रदा ददा॒त् पुन॒ रिन्द्र॒ इन्द्रः॒ पुन॑ रदा ददा॒त् पुन॒ रिन्द्रः॑ । \newline
36. पुन॒ रिन्द्र॒ इन्द्रः॒ पुनः॒ पुन॒ रिन्द्रो॒ बृह॒स्पति॒र् बृह॒स्पति॒ रिन्द्रः॒ पुनः॒ पुन॒ रिन्द्रो॒ बृह॒स्पतिः॑ । \newline
37. इन्द्रो॒ बृह॒स्पति॒र् बृह॒स्पति॒ रिन्द्र॒ इन्द्रो॒ बृह॒स्पतिः॑ । \newline
38. बृह॒स्पति॒रिति॑ बृह॒स्पतिः॑ । \newline
39. पुन॑र् मे मे॒ पुनः॒ पुन॑र् मे अश्विना ऽश्विना मे॒ पुनः॒ पुन॑र् मे अश्विना । \newline
40. मे॒ अ॒श्वि॒ना॒ ऽश्वि॒ना॒ मे॒ मे॒ अ॒श्वि॒ना॒ यु॒वं ॅयु॒व म॑श्विना मे मे अश्विना यु॒वम् । \newline
41. अ॒श्वि॒ना॒ यु॒वं ॅयु॒व म॑श्विना ऽश्विना यु॒वम् चक्षु॒ श्चक्षु॑र् यु॒व म॑श्विना ऽश्विना यु॒वम् चक्षुः॑ । \newline
42. यु॒वम् चक्षु॒ श्चक्षु॑र् यु॒वं ॅयु॒वम् चक्षु॒रा चक्षु॑र् यु॒वं ॅयु॒वम् चक्षु॒रा । \newline
43. चक्षु॒रा चक्षु॒ श्चक्षु॒रा ध॑त्तम् धत्त॒ मा चक्षु॒ श्चक्षु॒रा ध॑त्तम् । \newline
44. आ ध॑त्तम् धत्त॒ मा ध॑त्त म॒क्ष्यो र॒क्ष्योर् ध॑त्त॒ मा ध॑त्त म॒क्ष्योः । \newline
45. ध॒त्त॒ म॒क्ष्यो र॒क्ष्योर् ध॑त्तम् धत्त म॒क्ष्योः । \newline
46. अ॒क्ष्योरित्य॒क्ष्योः । \newline
47. इ॒ष्टय॑जुष स्ते त इ॒ष्टय॑जुष इ॒ष्टय॑जुष स्ते देव देव त इ॒ष्टय॑जुष इ॒ष्टय॑जुष स्ते देव । \newline
48. इ॒ष्टय॑जुष॒ इती॒ष्ट - य॒जु॒षः॒ । \newline
49. ते॒ दे॒व॒ दे॒व॒ ते॒ ते॒ दे॒व॒ सो॒म॒ सो॒म॒ दे॒व॒ ते॒ ते॒ दे॒व॒ सो॒म॒ । \newline
50. दे॒व॒ सो॒म॒ सो॒म॒ दे॒व॒ दे॒व॒ सो॒म॒ स्तु॒तस्तो॑मस्य स्तु॒तस्तो॑मस्य सोम देव देव सोम स्तु॒तस्तो॑मस्य । \newline
51. सो॒म॒ स्तु॒तस्तो॑मस्य स्तु॒तस्तो॑मस्य सोम सोम स्तु॒तस्तो॑मस्य श॒स्तोक्थ॑स्य श॒स्तोक्थ॑स्य स्तु॒तस्तो॑मस्य सोम सोम स्तु॒तस्तो॑मस्य श॒स्तोक्थ॑स्य । \newline
52. स्तु॒तस्तो॑मस्य श॒स्तोक्थ॑स्य श॒स्तोक्थ॑स्य स्तु॒तस्तो॑मस्य स्तु॒तस्तो॑मस्य श॒स्तोक्थ॑स्य॒ हरि॑वतो॒ हरि॑वतः श॒स्तोक्थ॑स्य स्तु॒तस्तो॑मस्य स्तु॒तस्तो॑मस्य श॒स्तोक्थ॑स्य॒ हरि॑वतः । \newline
53. स्तु॒तस्तो॑म॒स्येति॑ स्तु॒त - स्तो॒म॒स्य॒ । \newline
\pagebreak
\markright{ TS 3.2.5.5  \hfill https://www.vedavms.in \hfill}

\section{ TS 3.2.5.5 }

\textbf{TS 3.2.5.5 } \newline
\textbf{Samhita Paata} \newline

श॒स्तोक्थ॑स्य॒ हरि॑वत॒ इन्द्र॑पीतस्य॒ मधु॑मत॒ उप॑हूत॒स्योप॑हूतो भक्षयामि ॥ आ॒पूर्याः॒ स्थाऽऽमा॑ पूरयत प्र॒जया॑ च॒ धने॑न च ॥ ए॒तत् ते॑ तत॒ ये च॒ त्वामन्वे॒तत् ते॑ पितामह प्रपितामह॒ ये च॒ त्वामन्वत्र॑ पितरो यथाभा॒गं म॑न्दद्ध्वं॒ नमो॑ वः पितरो॒ रसा॑य॒ नमो॑ वः पितरः॒ शुष्मा॑य॒ नमो॑ वः पितरो जी॒वाय॒ नमो॑ वः पितरः - [  ] \newline

\textbf{Pada Paata} \newline

श॒स्तोक्थ॒स्येति॑ श॒स्त - उ॒क्थ॒स्य॒ । हरि॑वत॒ इति॒ हरि॑ - व॒तः॒ । इन्द्र॑पीत॒स्येतीन्द्र॑ - पी॒त॒स्य॒ । मधु॑मत॒ इति॒ मधु॑ - म॒तः॒ । उप॑हूत॒स्येत्युप॑ - हू॒त॒स्य॒ । उप॑हूत॒ इत्युप॑ - हू॒तः॒ । भ॒क्ष॒या॒मि॒ ॥ आ॒पूर्या॒ इत्या᳚ - पूर्याः᳚ । स्थ॒ । एति॑ । मा॒ । पू॒र॒य॒त॒ । प्र॒जयेति॑ प्र-जया᳚ । च॒ । धने॑न । च॒ ॥ ए॒तत् । ते॒ । त॒त॒ । ये । च॒ । त्वाम् । अन्विति॑ । ए॒तत् । ते॒ । पि॒ता॒म॒ह॒ । प्र॒पि॒ता॒म॒हेति॑ प्र - पि॒ता॒म॒ह॒ । ये । च॒ । त्वाम् । अन्विति॑ । अत्र॑ । पि॒त॒रः॒ । य॒था॒भा॒गमिति॑ यथा-भा॒गम् । म॒न्द॒द्ध्व॒म् । नमः॑ । वः॒ । पि॒त॒रः॒ । रसा॑य । नमः॑ । वः॒ । पि॒त॒रः॒ । शुष्मा॑य । नमः॑ । वः॒ । पि॒त॒रः॒ । जी॒वाय॑ । नमः॑ । वः॒ । पि॒त॒रः॒ ।  \newline


\textbf{Krama Paata} \newline

श॒स्तोक्थ॑स्य॒ हरि॑वतः । श॒स्तोक्थ॒स्येति॑ श॒स्त - उ॒क्थ॒स्य॒ । हरि॑वत॒ इन्द्र॑पीतस्य । हरि॑वत॒ इति॒ हरि॑ - व॒तः॒ । इन्द्र॑पीतस्य॒ मधु॑मतः । इन्द्र॑पीत॒स्येतीन्द्र॑ - पी॒त॒स्य॒ । मधु॑मत॒ उप॑हूतस्य । मधु॑मत॒ इति॒ मधु॑ - म॒तः॒ । उप॑हूत॒स्योप॑हूतः । उप॑हूत॒स्येत्युप॑ - हू॒त॒स्य॒ । उप॑हूतो भक्षयामि । उप॑हूत॒ इत्युप॑ - हू॒तः॒ । भ॒क्ष॒या॒मीति॑ भक्षयामि ॥ आ॒पूर्याः᳚ स्थ । आ॒पूर्या॒ इत्या᳚ - पूर्याः᳚ । स्था । आ मा᳚ । मा॒ पू॒र॒य॒त॒ । पू॒र॒य॒त॒ प्र॒जया᳚ । प्र॒जया॑ च । प्र॒जयेति॑ प्र - जया᳚ । च॒ धने॑न । धने॑न च । चेति॑ च ॥ ए॒तत् ते᳚ । ते॒ त॒त॒ । त॒त॒ ये । ये च॑ । च॒ त्वाम् । त्वामनु॑ । अन्वे॒तत् । ए॒तत् ते᳚ । ते॒ पि॒ता॒म॒ह॒ । पि॒ता॒म॒ह॒ प्र॒पि॒ता॒म॒ह॒ । प्र॒पि॒ता॒म॒ह॒ ये । प्र॒पि॒ता॒म॒हेति॑ प्र - पि॒ता॒म॒ह॒ । ये च॑ । च॒ त्वाम् । त्वामनु॑ । अन्वत्र॑ । अत्र॑ पितरः । पि॒त॒रो॒ य॒था॒भा॒गम् । य॒था॒भा॒गम् म॑न्दद्ध्वम् । य॒था॒भा॒गमिति॑ यथा - भा॒गम् । म॒न्द॒द्ध्व॒म् नमः॑ । नमो॑ वः । वः॒ पि॒त॒रः॒ । पि॒त॒रो॒ रसा॑य । रसा॑य॒ नमः॑ । नमो॑ वः । वः॒ पि॒त॒रः॒ । पि॒त॒रः॒ शुष्मा॑य । शुष्मा॑य॒ नमः॑ । नमो॑ वः । वः॒ पि॒त॒रः॒ । पि॒त॒रो॒ जी॒वाय॑ । जी॒वाय॒ नमः॑ । नमो॑ वः । वः॒ पि॒त॒रः॒ । पि॒त॒रः॒ स्व॒धायै᳚ \newline

\textbf{Jatai Paata} \newline

1. श॒स्तोक्थ॑स्य॒ हरि॑वतो॒ हरि॑वतः श॒स्तोक्थ॑स्य श॒स्तोक्थ॑स्य॒ हरि॑वतः । \newline
2. श॒स्तोक्थ॒स्येति॑ श॒स्त - उ॒क्थ॒स्य॒ । \newline
3. हरि॑वत॒ इन्द्र॑पीत॒ स्येन्द्र॑पीतस्य॒ हरि॑वतो॒ हरि॑वत॒ इन्द्र॑पीतस्य । \newline
4. हरि॑वत॒ इति॒ हरि॑ - व॒तः॒ । \newline
5. इन्द्र॑पीतस्य॒ मधु॑मतो॒ मधु॑मत॒ इन्द्र॑पीत॒ स्येन्द्र॑पीतस्य॒ मधु॑मतः । \newline
6. इन्द्र॑पीत॒स्येतीन्द्र॑ - पी॒त॒स्य॒ । \newline
7. मधु॑मत॒ उप॑हूत॒ स्योप॑हूतस्य॒ मधु॑मतो॒ मधु॑मत॒ उप॑हूतस्य । \newline
8. मधु॑मत॒ इति॒ मधु॑ - म॒तः॒ । \newline
9. उप॑हूत॒ स्योप॑हूत॒ उप॑हूत॒ उप॑हूत॒ स्योप॑हूत॒ स्योप॑हूतः । \newline
10. उप॑हूत॒स्येत्युप॑ - हू॒त॒स्य॒ । \newline
11. उप॑हूतो भक्षयामि भक्षया॒ म्युप॑हूत॒ उप॑हूतो भक्षयामि । \newline
12. उप॑हूत॒ इत्युप॑ - हू॒तः॒ । \newline
13. भ॒क्ष॒या॒मीति॑ भक्षयामि । \newline
14. आ॒पूर्याः᳚ स्थ स्था॒ पूर्या॑ आ॒पूर्याः᳚ स्थ । \newline
15. आ॒पूर्या॒ इत्या᳚ - पूर्याः᳚ । \newline
16. स्था स्थ॒ स्था । \newline
17. आ मा॒ मा ऽऽमा᳚ । \newline
18. मा॒ पू॒र॒य॒त॒ पू॒र॒य॒त॒ मा॒ मा॒ पू॒र॒य॒त॒ । \newline
19. पू॒र॒य॒त॒ प्र॒जया᳚ प्र॒जया॑ पूरयत पूरयत प्र॒जया᳚ । \newline
20. प्र॒जया॑ च च प्र॒जया᳚ प्र॒जया॑ च । \newline
21. प्र॒जयेति॑ प्र - जया᳚ । \newline
22. च॒ धने॑न॒ धने॑न च च॒ धने॑न । \newline
23. धने॑न च च॒ धने॑न॒ धने॑न च । \newline
24. चेति॑ च । \newline
25. ए॒तत् ते॑ त ए॒त दे॒तत् ते᳚ । \newline
26. ते॒ त॒त॒ त॒त॒ ते॒ ते॒ त॒त॒ । \newline
27. त॒त॒ ये ये त॑त तत॒ ये । \newline
28. ये च॑ च॒ ये ये च॑ । \newline
29. च॒ त्वाम् त्वाम् च॑ च॒ त्वाम् । \newline
30. त्वा मन्वनु॒ त्वाम् त्वा मनु॑ । \newline
31. अन्वे॒त दे॒त दन्वन् वे॒तत् । \newline
32. ए॒तत् ते॑ त ए॒त दे॒तत् ते᳚ । \newline
33. ते॒ पि॒ता॒म॒ह॒ पि॒ता॒म॒ह॒ ते॒ ते॒ पि॒ता॒म॒ह॒ । \newline
34. पि॒ता॒म॒ह॒ प्र॒पि॒ता॒म॒ह॒ प्र॒पि॒ता॒म॒ह॒ पि॒ता॒म॒ह॒ पि॒ता॒म॒ह॒ प्र॒पि॒ता॒म॒ह॒ । \newline
35. प्र॒पि॒ता॒म॒ह॒ ये ये प्र॑पितामह प्रपितामह॒ ये । \newline
36. प्र॒पि॒ता॒म॒हेति॑ प्र - पि॒ता॒म॒ह॒ । \newline
37. ये च॑ च॒ ये ये च॑ । \newline
38. च॒ त्वाम् त्वाम् च॑ च॒ त्वाम् । \newline
39. त्वा मन्वनु॒ त्वाम् त्वा मनु॑ । \newline
40. अन्वत्रा त्रा न्व न्वत्र॑ । \newline
41. अत्र॑ पितरः पित॒रो ऽत्रात्र॑ पितरः । \newline
42. पि॒त॒रो॒ य॒था॒भा॒गं ॅय॑थाभा॒गम् पि॑तरः पितरो यथाभा॒गम् । \newline
43. य॒था॒भा॒गम् म॑न्दद्ध्वम् मन्दद्ध्वं ॅयथाभा॒गं ॅय॑थाभा॒गम् म॑न्दद्ध्वम् । \newline
44. य॒था॒भा॒गमिति॑ यथा - भा॒गम् । \newline
45. म॒न्द॒द्ध्व॒म् नमो॒ नमो॑ मन्दद्ध्वम् मन्दद्ध्व॒म् नमः॑ । \newline
46. नमो॑ वो वो॒ नमो॒ नमो॑ वः । \newline
47. वः॒ पि॒त॒रः॒ पि॒त॒रो॒ वो॒ वः॒ पि॒त॒रः॒ । \newline
48. पि॒त॒रो॒ रसा॑य॒ रसा॑य पितरः पितरो॒ रसा॑य । \newline
49. रसा॑य॒ नमो॒ नमो॒ रसा॑य॒ रसा॑य॒ नमः॑ । \newline
50. नमो॑ वो वो॒ नमो॒ नमो॑ वः । \newline
51. वः॒ पि॒त॒रः॒ पि॒त॒रो॒ वो॒ वः॒ पि॒त॒रः॒ । \newline
52. पि॒त॒रः॒ शुष्मा॑य॒ शुष्मा॑य पितरः पितरः॒ शुष्मा॑य । \newline
53. शुष्मा॑य॒ नमो॒ नमः॒ शुष्मा॑य॒ शुष्मा॑य॒ नमः॑ । \newline
54. नमो॑ वो वो॒ नमो॒ नमो॑ वः । \newline
55. वः॒ पि॒त॒रः॒ पि॒त॒रो॒ वो॒ वः॒ पि॒त॒रः॒ । \newline
56. पि॒त॒रो॒ जी॒वाय॑ जी॒वाय॑ पितरः पितरो जी॒वाय॑ । \newline
57. जी॒वाय॒ नमो॒ नमो॑ जी॒वाय॑ जी॒वाय॒ नमः॑ । \newline
58. नमो॑ वो वो॒ नमो॒ नमो॑ वः । \newline
59. वः॒ पि॒त॒रः॒ पि॒त॒रो॒ वो॒ वः॒ पि॒त॒रः॒ । \newline
60. पि॒त॒रः॒ स्व॒धायै᳚ स्व॒धायै॑ पितरः पितरः स्व॒धायै᳚ । \newline

\textbf{Ghana Paata } \newline

1. श॒स्तोक्थ॑स्य॒ हरि॑वतो॒ हरि॑वतः श॒स्तोक्थ॑स्य श॒स्तोक्थ॑स्य॒ हरि॑वत॒ इन्द्र॑पीत॒स्ये न्द्र॑पीतस्य॒ हरि॑वतः श॒स्तोक्थ॑स्य श॒स्तोक्थ॑स्य॒ हरि॑वत॒ इन्द्र॑पीतस्य । \newline
2. श॒स्तोक्थ॒स्येति॑ श॒स्त - उ॒क्थ॒स्य॒ । \newline
3. हरि॑वत॒ इन्द्र॑पीत॒स्ये न्द्र॑पीतस्य॒ हरि॑वतो॒ हरि॑वत॒ इन्द्र॑पीतस्य॒ मधु॑मतो॒ मधु॑मत॒ इन्द्र॑पीतस्य॒ हरि॑वतो॒ हरि॑वत॒ इन्द्र॑पीतस्य॒ मधु॑मतः । \newline
4. हरि॑वत॒ इति॒ हरि॑ - व॒तः॒ । \newline
5. इन्द्र॑पीतस्य॒ मधु॑मतो॒ मधु॑मत॒ इन्द्र॑पीत॒स्ये न्द्र॑पीतस्य॒ मधु॑मत॒ उप॑हूत॒ स्योप॑हूतस्य॒ मधु॑मत॒ इन्द्र॑पीत॒स्ये न्द्र॑पीतस्य॒ मधु॑मत॒ उप॑हूतस्य । \newline
6. इन्द्र॑पीत॒स्येतीन्द्र॑ - पी॒त॒स्य॒ । \newline
7. मधु॑मत॒ उप॑हूत॒ स्योप॑हूतस्य॒ मधु॑मतो॒ मधु॑मत॒ उप॑हूत॒ स्योप॑हूत॒ उप॑हूत॒ उप॑हूतस्य॒ मधु॑मतो॒ मधु॑मत॒ उप॑हूत॒ स्योप॑हूतः । \newline
8. मधु॑मत॒ इति॒ मधु॑ - म॒तः॒ । \newline
9. उप॑हूत॒ स्योप॑हूत॒ उप॑हूत॒ उप॑हूत॒ स्योप॑हूत॒ स्योप॑हूतो भक्षयामि भक्षया॒ म्युप॑हूत॒ उप॑हूत॒ स्योप॑हूत॒ स्योप॑हूतो भक्षयामि । \newline
10. उप॑हूत॒स्येत्युप॑ - हू॒त॒स्य॒ । \newline
11. उप॑हूतो भक्षयामि भक्षया॒ म्युप॑हूत॒ उप॑हूतो भक्षयामि । \newline
12. उप॑हूत॒ इत्युप॑ - हू॒तः॒ । \newline
13. भ॒क्ष॒या॒मीति॑ भक्षयामि । \newline
14. आ॒पूर्याः᳚ स्थ स्था॒पूर्या॑ आ॒पूर्याः॒ स्था स्था॒पूर्या॑ आ॒पूर्याः॒ स्था । \newline
15. आ॒पूर्या॒ इत्या᳚ - पूर्याः᳚ । \newline
16. स्था स्थ॒ स्था मा॒ मा ऽऽस्थ॒ स्था मा᳚ । \newline
17. आ मा॒ मा ऽऽमा॑ पूरयत पूरयत॒ मा ऽऽमा॑ पूरयत । \newline
18. मा॒ पू॒र॒य॒त॒ पू॒र॒य॒त॒ मा॒ मा॒ पू॒र॒य॒त॒ प्र॒जया᳚ प्र॒जया॑ पूरयत मा मा पूरयत प्र॒जया᳚ । \newline
19. पू॒र॒य॒त॒ प्र॒जया᳚ प्र॒जया॑ पूरयत पूरयत प्र॒जया॑ च च प्र॒जया॑ पूरयत पूरयत प्र॒जया॑ च । \newline
20. प्र॒जया॑ च च प्र॒जया᳚ प्र॒जया॑ च॒ धने॑न॒ धने॑न च प्र॒जया᳚ प्र॒जया॑ च॒ धने॑न । \newline
21. प्र॒जयेति॑ प्र - जया᳚ । \newline
22. च॒ धने॑न॒ धने॑न च च॒ धने॑न च च॒ धने॑न च च॒ धने॑न च । \newline
23. धने॑न च च॒ धने॑न॒ धने॑न च । \newline
24. चेति॑ च । \newline
25. ए॒तत् ते॑ त ए॒त दे॒तत् ते॑ तत तत त ए॒त दे॒तत् ते॑ तत । \newline
26. ते॒ त॒त॒ त॒त॒ ते॒ ते॒ त॒त॒ ये ये त॑त ते ते तत॒ ये । \newline
27. त॒त॒ ये ये त॑त तत॒ ये च॑ च॒ ये त॑त तत॒ ये च॑ । \newline
28. ये च॑ च॒ ये ये च॒ त्वाम् त्वाम् च॒ ये ये च॒ त्वाम् । \newline
29. च॒ त्वाम् त्वाम् च॑ च॒ त्वा मन्वनु॒ त्वाम् च॑ च॒ त्वा मनु॑ । \newline
30. त्वा मन्वनु॒ त्वाम् त्वा मन्वे॒त दे॒तदनु॒ त्वाम् त्वा मन्वे॒तत् । \newline
31. अन्वे॒त दे॒त दन्व न्वे॒तत् ते॑ त ए॒त दन्व न्वे॒तत् ते᳚ । \newline
32. ए॒तत् ते॑ त ए॒त दे॒तत् ते॑ पितामह पितामह त ए॒त दे॒तत् ते॑ पितामह । \newline
33. ते॒ पि॒ता॒म॒ह॒ पि॒ता॒म॒ह॒ ते॒ ते॒ पि॒ता॒म॒ह॒ प्र॒पि॒ता॒म॒ह॒ प्र॒पि॒ता॒म॒ह॒ पि॒ता॒म॒ह॒ ते॒ ते॒ पि॒ता॒म॒ह॒ प्र॒पि॒ता॒म॒ह॒ । \newline
34. पि॒ता॒म॒ह॒ प्र॒पि॒ता॒म॒ह॒ प्र॒पि॒ता॒म॒ह॒ पि॒ता॒म॒ह॒ पि॒ता॒म॒ह॒ प्र॒पि॒ता॒म॒ह॒ ये ये प्र॑पितामह पितामह पितामह प्रपितामह॒ ये । \newline
35. प्र॒पि॒ता॒म॒ह॒ ये ये प्र॑पितामह प्रपितामह॒ ये च॑ च॒ ये प्र॑पितामह प्रपितामह॒ ये च॑ । \newline
36. प्र॒पि॒ता॒म॒हेति॑ प्र - पि॒ता॒म॒ह॒ । \newline
37. ये च॑ च॒ ये ये च॒ त्वाम् त्वाम् च॒ ये ये च॒ त्वाम् । \newline
38. च॒ त्वाम् त्वाम् च॑ च॒ त्वा मन्वनु॒ त्वाम् च॑ च॒ त्वा मनु॑ । \newline
39. त्वा मन्वनु॒ त्वाम् त्वा मन्वत्रा त्रानु॒ त्वाम् त्वा मन्वत्र॑ । \newline
40. अन्वत्रा त्रान्व न्वत्र॑ पितरः पित॒रो ऽत्रान्व न्वत्र॑ पितरः । \newline
41. अत्र॑ पितरः पित॒रो ऽत्रात्र॑ पितरो यथाभा॒गं ॅय॑थाभा॒गम् पि॑त॒रो ऽत्रात्र॑ पितरो यथाभा॒गम् । \newline
42. पि॒त॒रो॒ य॒था॒भा॒गं ॅय॑थाभा॒गम् पि॑तरः पितरो यथाभा॒गम् म॑न्दद्ध्वम् मन्दद्ध्वं ॅयथाभा॒गम् पि॑तरः पितरो यथाभा॒गम् म॑न्दद्ध्वम् । \newline
43. य॒था॒भा॒गम् म॑न्दद्ध्वम् मन्दद्ध्वं ॅयथाभा॒गं ॅय॑थाभा॒गम् म॑न्दद्ध्व॒म् नमो॒ नमो॑ मन्दद्ध्वं ॅयथाभा॒गं ॅय॑थाभा॒गम् म॑न्दद्ध्व॒म् नमः॑ । \newline
44. य॒था॒भा॒गमिति॑ यथा - भा॒गम् । \newline
45. म॒न्द॒द्ध्व॒म् नमो॒ नमो॑ मन्दद्ध्वम् मन्दद्ध्व॒म् नमो॑ वो वो॒ नमो॑ मन्दद्ध्वम् मन्दद्ध्व॒म् नमो॑ वः । \newline
46. नमो॑ वो वो॒ नमो॒ नमो॑ वः पितरः पितरो वो॒ नमो॒ नमो॑ वः पितरः । \newline
47. वः॒ पि॒त॒रः॒ पि॒त॒रो॒ वो॒ वः॒ पि॒त॒रो॒ रसा॑य॒ रसा॑य पितरो वो वः पितरो॒ रसा॑य । \newline
48. पि॒त॒रो॒ रसा॑य॒ रसा॑य पितरः पितरो॒ रसा॑य॒ नमो॒ नमो॒ रसा॑य पितरः पितरो॒ रसा॑य॒ नमः॑ । \newline
49. रसा॑य॒ नमो॒ नमो॒ रसा॑य॒ रसा॑य॒ नमो॑ वो वो॒ नमो॒ रसा॑य॒ रसा॑य॒ नमो॑ वः । \newline
50. नमो॑ वो वो॒ नमो॒ नमो॑ वः पितरः पितरो वो॒ नमो॒ नमो॑ वः पितरः । \newline
51. वः॒ पि॒त॒रः॒ पि॒त॒रो॒ वो॒ वः॒ पि॒त॒रः॒ शुष्मा॑य॒ शुष्मा॑य पितरो वो वः पितरः॒ शुष्मा॑य । \newline
52. पि॒त॒रः॒ शुष्मा॑य॒ शुष्मा॑य पितरः पितरः॒ शुष्मा॑य॒ नमो॒ नमः॒ शुष्मा॑य पितरः पितरः॒ शुष्मा॑य॒ नमः॑ । \newline
53. शुष्मा॑य॒ नमो॒ नमः॒ शुष्मा॑य॒ शुष्मा॑य॒ नमो॑ वो वो॒ नमः॒ शुष्मा॑य॒ शुष्मा॑य॒ नमो॑ वः । \newline
54. नमो॑ वो वो॒ नमो॒ नमो॑ वः पितरः पितरो वो॒ नमो॒ नमो॑ वः पितरः । \newline
55. वः॒ पि॒त॒रः॒ पि॒त॒रो॒ वो॒ वः॒ पि॒त॒रो॒ जी॒वाय॑ जी॒वाय॑ पितरो वो वः पितरो जी॒वाय॑ । \newline
56. पि॒त॒रो॒ जी॒वाय॑ जी॒वाय॑ पितरः पितरो जी॒वाय॒ नमो॒ नमो॑ जी॒वाय॑ पितरः पितरो जी॒वाय॒ नमः॑ । \newline
57. जी॒वाय॒ नमो॒ नमो॑ जी॒वाय॑ जी॒वाय॒ नमो॑ वो वो॒ नमो॑ जी॒वाय॑ जी॒वाय॒ नमो॑ वः । \newline
58. नमो॑ वो वो॒ नमो॒ नमो॑ वः पितरः पितरो वो॒ नमो॒ नमो॑ वः पितरः । \newline
59. वः॒ पि॒त॒रः॒ पि॒त॒रो॒ वो॒ वः॒ पि॒त॒रः॒ स्व॒धायै᳚ स्व॒धायै॑ पितरो वो वः पितरः स्व॒धायै᳚ । \newline
60. पि॒त॒रः॒ स्व॒धायै᳚ स्व॒धायै॑ पितरः पितरः स्व॒धायै॒ नमो॒ नमः॑ स्व॒धायै॑ पितरः पितरः स्व॒धायै॒ नमः॑ । \newline
\pagebreak
\markright{ TS 3.2.5.6  \hfill https://www.vedavms.in \hfill}

\section{ TS 3.2.5.6 }

\textbf{TS 3.2.5.6 } \newline
\textbf{Samhita Paata} \newline

स्व॒धायै॒ नमो॑ वः पितरो म॒न्यवे॒ नमो॑ वः पितरो घो॒राय॒ पित॑रो॒ नमो॑ वो॒ य ए॒तस्मि॑न् ॅलो॒केस्थ यु॒ष्माꣳस्तेऽनु॒ ये᳚ऽस्मिन् ॅलो॒के मां तेऽनु॒ य ए॒तस्मि॑न् ॅलो॒के स्थ यू॒यं तेषां॒ ॅवसि॑ष्ठा भूयास्त॒ ये᳚ऽस्मिन् ॅलो॒के॑ऽहं तेषां॒ ॅवसि॑ष्ठो भूयासं॒ प्रजा॑पते॒ न त्वदे॒तान्य॒न्यो विश्वा॑ जा॒तानि॒ परि॒ता ब॑भूव । \newline

\textbf{Pada Paata} \newline

स्व॒धाया॒ इति॑ स्व - धायै᳚ । नमः॑ । वः॒ । पि॒त॒रः॒ । म॒न्यवे᳚ । नमः॑ । वः॒ । पि॒त॒रः॒ । घो॒राय॑ । पित॑रः । नमः॑ । वः॒ । ये । ए॒तस्मिन्न्॑ । लो॒के । स्थ । यु॒ष्मान् । ते । अन्विति॑ । ये । अ॒स्मिन्न् । लो॒के । माम् । ते । अन्विति॑ । ये । ए॒तस्मिन्न्॑ । लो॒के । स्थ । यू॒यम् । तेषा᳚म् । वसि॑ष्ठाः । भू॒या॒स्त॒ । ये । अ॒स्मिन्न् । लो॒के । अ॒हम् । तेषा᳚म् । वसि॑ष्ठः । भू॒या॒स॒म् । प्रजा॑पत॒ इति॒ प्रजा᳚-प॒ते॒ । न । त्वत् । ए॒तानि॑ । अ॒न्यः । विश्वा᳚ । जा॒तानि॑ । परीति॑ । ता । ब॒भू॒व॒ ॥  \newline


\textbf{Krama Paata} \newline

स्व॒धायै॒ नमः॑ । स्व॒धाया॒ इति॑ स्व - धायै᳚ । नमो॑ वः । वः॒ पि॒त॒रः॒ । पि॒त॒रो॒ म॒न्यवे᳚ । म॒न्यवे॒ नमः॑ । नमो॑ वः । वः॒ पि॒त॒रः॒ । पि॒त॒रो॒ घो॒राय॑ । घो॒राय॒ पित॑रः । पित॑रो॒ नमः॑ । नमो॑ वः । वो॒ ये । य ए॒तस्मिन्न्॑ । ए॒तस्मि॑न् ॅलो॒के । लो॒के स्थ । स्थ यु॒ष्मान् । यु॒ष्माꣳस्ते । ते ऽनु॑ । अनु॒ ये । ये᳚ ऽस्मिन्न् । अ॒स्मिन् ॅलो॒के । लो॒के माम् । माम् ते । ते ऽनु॑ । अनु॒ ये । य ए॒तस्मिन्न्॑ । ए॒तस्मि॑न् ॅलो॒के । लो॒के स्थ । स्थ यू॒यम् । यू॒यम् तेषा᳚म् । तेषां॒ ॅवसि॑ष्ठाः । वसि॑ष्ठा भूयास्त । भू॒या॒स्त॒ ये । ये᳚ ऽस्मिन्न् । अ॒स्मिन् ॅलो॒के । लो॒के॑ ऽहम् । अ॒हम् तेषा᳚म् । तेषां॒ ॅवसि॑ष्ठः । वसि॑ष्ठो भूयासम् । भू॒या॒स॒म् प्रजा॑पते । प्रजा॑पते॒ न । प्रजा॑पत॒ इति॒ प्रजा᳚ - प॒ते॒ । न त्वत् । त्वदे॒तानि॑ । ए॒तान्य॒न्यः । अ॒न्यो विश्वा᳚ । विश्वा॑ जा॒तानि॑ । जा॒तानि॒ परि॑ । परि॒ ता । ता ब॑भूव ( ) । ब॒भू॒वेति॑ बभूव । \newline

\textbf{Jatai Paata} \newline

1. स्व॒धायै॒ नमो॒ नमः॑ स्व॒धायै᳚ स्व॒धायै॒ नमः॑ । \newline
2. स्व॒धाया॒ इति॑ स्व - धायै᳚ । \newline
3. नमो॑ वो वो॒ नमो॒ नमो॑ वः । \newline
4. वः॒ पि॒त॒रः॒ पि॒त॒रो॒ वो॒ वः॒ पि॒त॒रः॒ । \newline
5. पि॒त॒रो॒ म॒न्यवे॑ म॒न्यवे॑ पितरः पितरो म॒न्यवे᳚ । \newline
6. म॒न्यवे॒ नमो॒ नमो॑ म॒न्यवे॑ म॒न्यवे॒ नमः॑ । \newline
7. नमो॑ वो वो॒ नमो॒ नमो॑ वः । \newline
8. वः॒ पि॒त॒रः॒ पि॒त॒रो॒ वो॒ वः॒ पि॒त॒रः॒ । \newline
9. पि॒त॒रो॒ घो॒राय॑ घो॒राय॑ पितरः पितरो घो॒राय॑ । \newline
10. घो॒राय॒ पित॑रः॒ पित॑रो घो॒राय॑ घो॒राय॒ पित॑रः । \newline
11. पित॑रो॒ नमो॒ नमः॒ पित॑रः॒ पित॑रो॒ नमः॑ । \newline
12. नमो॑ वो वो॒ नमो॒ नमो॑ वः । \newline
13. वो॒ ये ये वो॑ वो॒ ये । \newline
14. य ए॒तस्मि॑न् ने॒तस्मि॒न्॒. ये य ए॒तस्मिन्न्॑ । \newline
15. ए॒तस्मि॑न् ॅलो॒के लो॒क ए॒तस्मि॑न् ने॒तस्मि॑न् ॅलो॒के । \newline
16. लो॒के स्थ स्थ लो॒के लो॒के स्थ । \newline
17. स्थ यु॒ष्मान्. यु॒ष्मान् थ्स्थ स्थ यु॒ष्मान् । \newline
18. यु॒ष्माꣳ स्ते ते यु॒ष्मान्. यु॒ष्माꣳ स्ते । \newline
19. ते ऽन्वनु॒ ते ते ऽनु॑ । \newline
20. अनु॒ ये ये ऽन्वनु॒ ये । \newline
21. ये᳚ ऽस्मिन् न॒स्मिन्. ये ये᳚ ऽस्मिन्न् । \newline
22. अ॒स्मिन् ॅलो॒के लो॒के᳚ ऽस्मिन् न॒स्मिन् ॅलो॒के । \newline
23. लो॒के माम् माम् ॅलो॒के लो॒के माम् । \newline
24. माम् ते ते माम् माम् ते । \newline
25. ते ऽन्वनु॒ ते ते ऽनु॑ । \newline
26. अनु॒ ये ये ऽन्वनु॒ ये । \newline
27. य ए॒तस्मि॑न् ने॒तस्मि॒न्॒. ये य ए॒तस्मिन्न्॑ । \newline
28. ए॒तस्मि॑न् ॅलो॒के लो॒क ए॒तस्मि॑न् ने॒तस्मि॑न् ॅलो॒के । \newline
29. लो॒के स्थ स्थ लो॒के लो॒के स्थ । \newline
30. स्थ यू॒यं ॅयू॒यꣳ स्थ स्थ यू॒यम् । \newline
31. यू॒यम् तेषा॒म् तेषां᳚ ॅयू॒यं ॅयू॒यम् तेषा᳚म् । \newline
32. तेषां॒ ॅवसि॑ष्ठा॒ वसि॑ष्ठा॒ स्तेषा॒म् तेषां॒ ॅवसि॑ष्ठाः । \newline
33. वसि॑ष्ठा भूयास्त भूयास्त॒ वसि॑ष्ठा॒ वसि॑ष्ठा भूयास्त । \newline
34. भू॒या॒स्त॒ ये ये भू॑यास्त भूयास्त॒ ये । \newline
35. ये᳚ ऽस्मिन् न॒स्मिन्. ये ये᳚ ऽस्मिन्न् । \newline
36. अ॒स्मिन् ॅलो॒के लो॒के᳚ ऽस्मिन् न॒स्मिन् ॅलो॒के । \newline
37. लो॒के॑ ऽह म॒हम् ॅलो॒के लो॒के॑ ऽहम् । \newline
38. अ॒हम् तेषा॒म् तेषा॑ म॒ह म॒हम् तेषा᳚म् । \newline
39. तेषां॒ ॅवसि॑ष्ठो॒ वसि॑ष्ठ॒ स्तेषा॒म् तेषां॒ ॅवसि॑ष्ठः । \newline
40. वसि॑ष्ठो भूयासम् भूयासं॒ ॅवसि॑ष्ठो॒ वसि॑ष्ठो भूयासम् । \newline
41. भू॒या॒स॒म् प्रजा॑पते॒ प्रजा॑पते भूयासम् भूयास॒म् प्रजा॑पते । \newline
42. प्रजा॑पते॒ न न प्रजा॑पते॒ प्रजा॑पते॒ न । \newline
43. प्रजा॑पत॒ इति॒ प्रजा᳚ - प॒ते॒ । \newline
44. न त्वत् त्वन् न न त्वत् । \newline
45. त्व दे॒ता न्ये॒तानि॒ त्वत् त्व दे॒तानि॑ । \newline
46. ए॒ता न्य॒न्यो᳚ ऽन्य ए॒ता न्ये॒ता न्य॒न्यः । \newline
47. अ॒न्यो विश्वा॒ विश्वा॒ ऽन्यो᳚ ऽन्यो विश्वा᳚ । \newline
48. विश्वा॑ जा॒तानि॑ जा॒तानि॒ विश्वा॒ विश्वा॑ जा॒तानि॑ । \newline
49. जा॒तानि॒ परि॒ परि॑ जा॒तानि॑ जा॒तानि॒ परि॑ । \newline
50. परि॒ ता ता परि॒ परि॒ ता । \newline
51. ता ब॑भूव बभूव॒ ता ता ब॑भूव । \newline
52. ब॒भू॒वेति॑ बभूव । \newline

\textbf{Ghana Paata } \newline

1. स्व॒धायै॒ नमो॒ नमः॑ स्व॒धायै᳚ स्व॒धायै॒ नमो॑ वो वो॒ नमः॑ स्व॒धायै᳚ स्व॒धायै॒ नमो॑ वः । \newline
2. स्व॒धाया॒ इति॑ स्व - धायै᳚ । \newline
3. नमो॑ वो वो॒ नमो॒ नमो॑ वः पितरः पितरो वो॒ नमो॒ नमो॑ वः पितरः । \newline
4. वः॒ पि॒त॒रः॒ पि॒त॒रो॒ वो॒ वः॒ पि॒त॒रो॒ म॒न्यवे॑ म॒न्यवे॑ पितरो वो वः पितरो म॒न्यवे᳚ । \newline
5. पि॒त॒रो॒ म॒न्यवे॑ म॒न्यवे॑ पितरः पितरो म॒न्यवे॒ नमो॒ नमो॑ म॒न्यवे॑ पितरः पितरो म॒न्यवे॒ नमः॑ । \newline
6. म॒न्यवे॒ नमो॒ नमो॑ म॒न्यवे॑ म॒न्यवे॒ नमो॑ वो वो॒ नमो॑ म॒न्यवे॑ म॒न्यवे॒ नमो॑ वः । \newline
7. नमो॑ वो वो॒ नमो॒ नमो॑ वः पितरः पितरो वो॒ नमो॒ नमो॑ वः पितरः । \newline
8. वः॒ पि॒त॒रः॒ पि॒त॒रो॒ वो॒ वः॒ पि॒त॒रो॒ घो॒राय॑ घो॒राय॑ पितरो वो वः पितरो घो॒राय॑ । \newline
9. पि॒त॒रो॒ घो॒राय॑ घो॒राय॑ पितरः पितरो घो॒राय॒ पित॑रः॒ पित॑रो घो॒राय॑ पितरः पितरो घो॒राय॒ पित॑रः । \newline
10. घो॒राय॒ पित॑रः॒ पित॑रो घो॒राय॑ घो॒राय॒ पित॑रो॒ नमो॒ नमः॒ पित॑रो घो॒राय॑ घो॒राय॒ पित॑रो॒ नमः॑ । \newline
11. पित॑रो॒ नमो॒ नमः॒ पित॑रः॒ पित॑रो॒ नमो॑ वो वो॒ नमः॒ पित॑रः॒ पित॑रो॒ नमो॑ वः । \newline
12. नमो॑ वो वो॒ नमो॒ नमो॑ वो॒ ये ये वो॒ नमो॒ नमो॑ वो॒ ये । \newline
13. वो॒ ये ये वो॑ वो॒ य ए॒तस्मि॑न् ने॒तस्मि॒न्॒. ये वो॑ वो॒ य ए॒तस्मिन्न्॑ । \newline
14. य ए॒तस्मि॑न् ने॒तस्मि॒न्॒. ये य ए॒तस्मि॑न् ॅलो॒के लो॒क ए॒तस्मि॒न्॒. ये य ए॒तस्मि॑न् ॅलो॒के । \newline
15. ए॒तस्मि॑न् ॅलो॒के लो॒क ए॒तस्मि॑न् ने॒तस्मि॑न् ॅलो॒के स्थ स्थ लो॒क ए॒तस्मि॑न् ने॒तस्मि॑न् ॅलो॒के स्थ । \newline
16. लो॒के स्थ स्थ लो॒के लो॒के स्थ यु॒ष्मान्. यु॒ष्मान् थ्स्थ लो॒के लो॒के स्थ यु॒ष्मान् । \newline
17. स्थ यु॒ष्मान्. यु॒ष्मान् थ्स्थ स्थ यु॒ष्माꣳ स्ते ते यु॒ष्मान् थ्स्थ स्थ यु॒ष्माꣳ स्ते । \newline
18. यु॒ष्माꣳ स्ते ते यु॒ष्मान्. यु॒ष्माꣳ स्ते ऽन्वनु॒ ते यु॒ष्मान्. यु॒ष्माꣳ स्ते ऽनु॑ । \newline
19. ते ऽन्वनु॒ ते ते ऽनु॒ ये ये ऽनु॒ ते ते ऽनु॒ ये । \newline
20. अनु॒ ये ये ऽन्वनु॒ ये᳚ ऽस्मिन् न॒स्मिन्. ये ऽन्वनु॒ ये᳚ ऽस्मिन्न् । \newline
21. ये᳚ ऽस्मिन् न॒स्मिन्. ये ये᳚ ऽस्मिन् ॅलो॒के लो॒के᳚ ऽस्मिन्. ये ये᳚ ऽस्मिन् ॅलो॒के । \newline
22. अ॒स्मिन् ॅलो॒के लो॒के᳚ ऽस्मिन् न॒स्मिन् ॅलो॒के माम् माम् ॅलो॒के᳚ ऽस्मिन् न॒स्मिन् ॅलो॒के माम् । \newline
23. लो॒के माम् माम् ॅलो॒के लो॒के माम् ते ते माम् ॅलो॒के लो॒के माम् ते । \newline
24. माम् ते ते माम् माम् ते ऽन्वनु॒ ते माम् माम् ते ऽनु॑ । \newline
25. ते ऽन्वनु॒ ते ते ऽनु॒ ये ये ऽनु॒ ते ते ऽनु॒ ये । \newline
26. अनु॒ ये ये ऽन्वनु॒ य ए॒तस्मि॑न् ने॒तस्मि॒न्॒. ये ऽन्वनु॒ य ए॒तस्मिन्न्॑ । \newline
27. य ए॒तस्मि॑न् ने॒तस्मि॒न्॒. ये य ए॒तस्मि॑न् ॅलो॒के लो॒क ए॒तस्मि॒न्॒. ये य ए॒तस्मि॑न् ॅलो॒के । \newline
28. ए॒तस्मि॑न् ॅलो॒के लो॒क ए॒तस्मि॑न् ने॒तस्मि॑न् ॅलो॒के स्थ स्थ लो॒क ए॒तस्मि॑न् ने॒तस्मि॑न् ॅलो॒के स्थ । \newline
29. लो॒के स्थ स्थ लो॒के लो॒के स्थ यू॒यं ॅयू॒यꣳ स्थ लो॒के लो॒के स्थ यू॒यम् । \newline
30. स्थ यू॒यं ॅयू॒यꣳ स्थ स्थ यू॒यम् तेषा॒म् तेषां᳚ ॅयू॒यꣳ स्थ स्थ यू॒यम् तेषा᳚म् । \newline
31. यू॒यम् तेषा॒म् तेषां᳚ ॅयू॒यं ॅयू॒यम् तेषां॒ ॅवसि॑ष्ठा॒ वसि॑ष्ठा॒ स्तेषां᳚ ॅयू॒यं ॅयू॒यम् तेषां॒ ॅवसि॑ष्ठाः । \newline
32. तेषां॒ ॅवसि॑ष्ठा॒ वसि॑ष्ठा॒ स्तेषा॒म् तेषां॒ ॅवसि॑ष्ठा भूयास्त भूयास्त॒ वसि॑ष्ठा॒ स्तेषा॒म् तेषां॒ ॅवसि॑ष्ठा भूयास्त । \newline
33. वसि॑ष्ठा भूयास्त भूयास्त॒ वसि॑ष्ठा॒ वसि॑ष्ठा भूयास्त॒ ये ये भू॑यास्त॒ वसि॑ष्ठा॒ वसि॑ष्ठा भूयास्त॒ ये । \newline
34. भू॒या॒स्त॒ ये ये भू॑यास्त भूयास्त॒ ये᳚ ऽस्मिन् न॒स्मिन्. ये भू॑यास्त भूयास्त॒ ये᳚ ऽस्मिन्न् । \newline
35. ये᳚ ऽस्मिन् न॒स्मिन्. ये ये᳚ ऽस्मिन् ॅलो॒के लो॒के᳚ ऽस्मिन्. ये ये᳚ ऽस्मिन् ॅलो॒के । \newline
36. अ॒स्मिन् ॅलो॒के लो॒के᳚ ऽस्मिन् न॒स्मिन् ॅलो॒के॑ ऽह म॒हम् ॅलो॒के᳚ ऽस्मिन् न॒स्मिन् ॅलो॒के॑ ऽहम् । \newline
37. लो॒के॑ ऽह म॒हम् ॅलो॒के लो॒के॑ ऽहम् तेषा॒म् तेषा॑ म॒हम् ॅलो॒के लो॒के॑ ऽहम् तेषा᳚म् । \newline
38. अ॒हम् तेषा॒म् तेषा॑ म॒ह म॒हम् तेषां॒ ॅवसि॑ष्ठो॒ वसि॑ष्ठ॒ स्तेषा॑ म॒ह म॒हम् तेषां॒ ॅवसि॑ष्ठः । \newline
39. तेषां॒ ॅवसि॑ष्ठो॒ वसि॑ष्ठ॒ स्तेषा॒म् तेषां॒ ॅवसि॑ष्ठो भूयासम् भूयासं॒ ॅवसि॑ष्ठ॒ स्तेषा॒म् तेषां॒ ॅवसि॑ष्ठो भूयासम् । \newline
40. वसि॑ष्ठो भूयासम् भूयासं॒ ॅवसि॑ष्ठो॒ वसि॑ष्ठो भूयास॒म् प्रजा॑पते॒ प्रजा॑पते भूयासं॒ ॅवसि॑ष्ठो॒ वसि॑ष्ठो भूयास॒म् प्रजा॑पते । \newline
41. भू॒या॒स॒म् प्रजा॑पते॒ प्रजा॑पते भूयासम् भूयास॒म् प्रजा॑पते॒ न न प्रजा॑पते भूयासम् भूयास॒म् प्रजा॑पते॒ न । \newline
42. प्रजा॑पते॒ न न प्रजा॑पते॒ प्रजा॑पते॒ न त्वत् त्वन् न प्रजा॑पते॒ प्रजा॑पते॒ न त्वत् । \newline
43. प्रजा॑पत॒ इति॒ प्रजा᳚-प॒ते॒ । \newline
44. न त्वत् त्वन् न न त्वदे॒ता न्ये॒तानि॒ त्वन् न न त्वदे॒तानि॑ । \newline
45. त्वदे॒ता न्ये॒तानि॒ त्वत् त्वदे॒ता न्य॒न्यो᳚ ऽन्य ए॒तानि॒ त्वत् त्वदे॒तान्य॒न्यः । \newline
46. ए॒तान्य॒न्यो᳚ ऽन्य ए॒ता न्ये॒ता न्य॒न्यो विश्वा॒ विश्वा॒ ऽन्य ए॒ता न्ये॒ता न्य॒न्यो विश्वा᳚ । \newline
47. अ॒न्यो विश्वा॒ विश्वा॒ ऽन्यो᳚ ऽन्यो विश्वा॑ जा॒तानि॑ जा॒तानि॒ विश्वा॒ ऽन्यो᳚ ऽन्यो विश्वा॑ जा॒तानि॑ । \newline
48. विश्वा॑ जा॒तानि॑ जा॒तानि॒ विश्वा॒ विश्वा॑ जा॒तानि॒ परि॒ परि॑ जा॒तानि॒ विश्वा॒ विश्वा॑ जा॒तानि॒ परि॑ । \newline
49. जा॒तानि॒ परि॒ परि॑ जा॒तानि॑ जा॒तानि॒ परि॒ ता ता परि॑ जा॒तानि॑ जा॒तानि॒ परि॒ ता । \newline
50. परि॒ ता ता परि॒ परि॒ ता ब॑भूव बभूव॒ ता परि॒ परि॒ ता ब॑भूव । \newline
51. ता ब॑भूव बभूव॒ ता ता ब॑भूव । \newline
52. ब॒भू॒वेति॑ बभूव । \newline
\pagebreak
\markright{ TS 3.2.5.7  \hfill https://www.vedavms.in \hfill}

\section{ TS 3.2.5.7 }

\textbf{TS 3.2.5.7 } \newline
\textbf{Samhita Paata} \newline

यत् का॑मास्ते जुहु॒मस्तन्नो॑ अस्तु व॒यꣳ स्या॑म॒ पत॑यो रयी॒णां ॥ दे॒वकृ॑त॒स्यैन॑सो ऽव॒यज॑नमसि मनु॒ष्य॑कृत॒स्यैन॑सो ऽव॒यज॑नमसि पि॒तृकृ॑त॒स्यैन॑सो ऽव॒यज॑नमस्य॒फ्सु धौ॒तस्य॑ सोम देव ते॒ नृभिः॑ सु॒तस्ये॒ष्ट य॑जुषः स्तु॒तस्तो॑मस्य श॒स्तोक्थ॑स्य॒ यो भ॒क्षोअ॑श्व॒सनि॒र्यो गो॒सनि॒स्तस्य॑ ते पि॒तृभि॑र्भ॒क्षं कृ॑त॒स्यो-प॑हूत॒स्योप॑हूतो भक्षयामि ॥ \newline

\textbf{Pada Paata} \newline

यत्का॑मा॒ इति॒ यत् - का॒माः॒ । ते॒ । जु॒हु॒मः । तत् । नः॒ । अ॒स्तु॒ । व॒यम् । स्या॒म॒ । पत॑यः । र॒यी॒णाम् ॥ दे॒वकृ॑त॒स्येति॑ दे॒व - कृ॒त॒स्य॒ । एन॑सः । अ॒व॒यज॑न॒मित्य॑व - यज॑नम् । अ॒सि॒ । म॒नु॒ष्य॑कृत॒स्येति॑ मनु॒ष्य॑ - कृ॒त॒स्य॒ । एन॑सः । अ॒व॒यज॑न॒मित्य॑व - यज॑नम् । अ॒सि॒ । पि॒तृकृ॑त॒स्येति॑ पि॒तृ - कृ॒त॒स्य॒ । एन॑सः । अ॒व॒यज॑न॒मित्य॑व - यज॑नम् । अ॒सि॒ । अ॒फ्स्वित्य॑प्-सु । धौ॒तस्य॑ । सो॒म॒ । दे॒व॒ । ते॒ । नृभि॒रिति॒ नृ - भिः॒ । सु॒तस्य॑ । इ॒ष्टय॑जुष॒ इती॒ष्ट - य॒जु॒षः॒ । स्तु॒तस्तो॑म॒स्येति॑ स्तु॒त - स्तो॒म॒स्य॒ । श॒स्तोक्थ॒स्येति॑ श॒स्त - उ॒क्थ॒स्य॒ । यः । भ॒क्षः । अ॒श्व॒सनि॒रित्य॑श्व - सनिः॑ । यः । गो॒सनि॒रिति॑ गो - सनिः॑ । तस्य॑ । ते॒ । पि॒तृभि॒रिति॑ पि॒तृ - भिः॒ । भ॒क्षं कृ॑त॒स्येति॑ भ॒क्षं - कृ॒त॒स्य॒ । उप॑हूत॒स्येत्युप॑ - हू॒त॒स्य॒ । उप॑हूत॒ इत्युप॑ - हू॒तः॒ । भ॒क्ष॒या॒मि॒ ॥  \newline


\textbf{Krama Paata} \newline

यत्का॑मास्ते । यत्का॑मा॒ इति॒ यत् - का॒माः॒ । ते॒ जु॒हु॒मः । जु॒हु॒म स्तत् । तन् नः॑ । नो॒ अ॒स्तु॒ । अ॒स्तु॒ व॒यम् । व॒यꣳ स्या॑म । स्या॒म॒ पत॑यः । पत॑यो रयी॒णाम् । र॒यी॒णामिति॑ रयी॒णाम् ॥ दे॒वकृ॑त॒स्यैन॑सः । दे॒वकृ॑त॒स्येति॑ दे॒व - कृ॒त॒स्य॒ । एन॑सो ऽव॒यज॑नम् । अ॒व॒यज॑नमसि । अ॒व॒यज॑न॒मित्य॑व - यज॑नम् । अ॒सि॒ म॒नु॒ष्य॑कृतस्य । म॒नु॒ष्य॑कृत॒स्यैन॑सः । म॒नु॒ष्य॑कृत॒स्येति॑ मनु॒ष्य॑ - कृ॒त॒स्य॒ । एन॑सो ऽव॒यज॑नम् । अ॒व॒यज॑नमसि । अ॒व॒यज॑न॒मित्य॑व - यज॑नम् । अ॒सि॒ पि॒तृकृ॑तस्य । पि॒तृकृ॑त॒स्यैन॑सः । पि॒तृकृ॑त॒स्येति॑ पि॒तृ - कृ॒त॒स्य॒ । एन॑सो ऽव॒यज॑नम् । अ॒व॒यज॑नमसि । अ॒व॒यज॑न॒मित्य॑व - यज॑नम् । अ॒स्य॒फ्सु । अ॒फ्सु धौ॒तस्य॑ । अ॒फ्स्वित्य॑प् - सु । धौ॒तस्य॑ सोम । सो॒म॒ दे॒व॒ । दे॒व॒ ते॒ । ते॒ नृभिः॑ । नृभिः॑ सु॒तस्य॑ । नृभि॒रिति॒ नृ - भिः॒ । सु॒तस्ये॒ष्टय॑जुषः । इ॒ष्टय॑जुषः स्तु॒तस्तो॑मस्य । इ॒ष्टय॑जुष॒ इती॒ष्ट - य॒जु॒षः॒ । स्तु॒तस्तो॑मस्य श॒स्तोक्थ॑स्य । स्तु॒तस्तो॑म॒स्येति॑ स्तु॒त - स्तो॒म॒स्य॒ । श॒स्तोक्थ॑स्य॒ यः । श॒स्तोक्थ॒स्येति॑ श॒स्त - उ॒क्थ॒स्य॒ । यो भ॒क्षः । भ॒क्षो अ॑श्व॒सनिः॑ । अ॒श्व॒सनि॒र् यः । अ॒श्व॒सनि॒रित्य॑श्व - सनिः॑ । यो गो॒सनिः॑ । गो॒सनि॒स्तस्य॑ । गो॒सनि॒रिति॑ गो - सनिः॑ । तस्य॑ ते । ते॒ पि॒तृभिः॑ । पि॒तृभि॑र् भ॒क्षङ्कृ॑तस्य । पि॒तृभि॒रिति॑ पि॒तृ - भिः॒ । भ॒क्षङ्कृ॑त॒,स्योप॑हूतस्य । भ॒क्षङ्कृ॑त॒स्येति॑ भ॒क्षम् - कृ॒त॒स्य॒ । उप॑हूत॒स्योप॑हूतः । उप॑हूत॒स्येत्युप॑ - हू॒त॒स्य॒ । उप॑हूतो भक्षयामि । उप॑हूत॒ इत्युप॑ - हू॒तः॒ । भ॒क्ष॒या॒मीति॑ भक्षयामि । \newline

\textbf{Jatai Paata} \newline

1. यत्का॑मा स्ते ते॒ यत्का॑मा॒ यत्का॑मा स्ते । \newline
2. यत्का॑मा॒ इति॒ यत् - का॒माः॒ । \newline
3. ते॒ जु॒हु॒मो जु॑हु॒म स्ते॑ ते जुहु॒मः । \newline
4. जु॒हु॒म स्तत् तज् जु॑हु॒मो जु॑हु॒म स्तत् । \newline
5. तन् नो॑ न॒ स्तत् तन् नः॑ । \newline
6. नो॒ अ॒स्त्व॒स्तु॒ नो॒ नो॒ अ॒स्तु॒ । \newline
7. अ॒स्तु॒ व॒यं ॅव॒य म॑स्त्वस्तु व॒यम् । \newline
8. व॒यꣳ स्या॑म स्याम व॒यं ॅव॒यꣳ स्या॑म । \newline
9. स्या॒म॒ पत॑यः॒ पत॑यः स्याम स्याम॒ पत॑यः । \newline
10. पत॑यो रयी॒णाꣳ र॑यी॒णाम् पत॑यः॒ पत॑यो रयी॒णाम् । \newline
11. र॒यी॒णामिति॑ रयी॒णाम् । \newline
12. दे॒वकृ॑त॒ स्यैन॑स॒ एन॑सो दे॒वकृ॑तस्य दे॒वकृ॑त॒ स्यैन॑सः । \newline
13. दे॒वकृ॑त॒स्येति॑ दे॒व - कृ॒त॒स्य॒ । \newline
14. एन॑सो ऽव॒यज॑न मव॒यज॑न॒ मेन॑स॒ एन॑सो ऽव॒यज॑नम् । \newline
15. अ॒व॒यज॑न मस्य स्यव॒यज॑न मव॒यज॑न मसि । \newline
16. अ॒व॒यज॑न॒मित्य॑व - यज॑नम् । \newline
17. अ॒सि॒ म॒नु॒ष्य॑कृतस्य मनु॒ष्य॑कृत स्यास्यसि मनु॒ष्य॑कृतस्य । \newline
18. म॒नु॒ष्य॑कृत॒ स्यैन॑स॒ एन॑सो मनु॒ष्य॑कृतस्य मनु॒ष्य॑कृत॒ स्यैन॑सः । \newline
19. म॒नु॒ष्य॑कृत॒स्येति॑ मनु॒ष्य॑ - कृ॒त॒स्य॒ । \newline
20. एन॑सो ऽव॒यज॑न मव॒यज॑न॒ मेन॑स॒ एन॑सो ऽव॒यज॑नम् । \newline
21. अ॒व॒यज॑न मस्य स्यव॒यज॑न मव॒यज॑न मसि । \newline
22. अ॒व॒यज॑न॒मित्य॑व - यज॑नम् । \newline
23. अ॒सि॒ पि॒तृकृ॑तस्य पि॒तृकृ॑त स्यास्यसि पि॒तृकृ॑तस्य । \newline
24. पि॒तृकृ॑त॒ स्यैन॑स॒ एन॑सः पि॒तृकृ॑तस्य पि॒तृकृ॑त॒ स्यैन॑सः । \newline
25. पि॒तृकृ॑त॒स्येति॑ पि॒तृ - कृ॒त॒स्य॒ । \newline
26. एन॑सो ऽव॒यज॑न मव॒यज॑न॒ मेन॑स॒ एन॑सो ऽव॒यज॑नम् । \newline
27. अ॒व॒यज॑न मस्य स्यव॒यज॑न मव॒यज॑न मसि । \newline
28. अ॒व॒यज॑न॒मित्य॑व - यज॑नम् । \newline
29. अ॒स्य॒ फ्स्वा᳚(1॒)फ्स्व॑ स्यस्य॒फ्सु । \newline
30. अ॒फ्सु धौ॒तस्य॑ धौ॒तस्या॒ फ्स्व॑फ्सु धौ॒तस्य॑ । \newline
31. अ॒फ्स्वित्य॑प् - सु । \newline
32. धौ॒तस्य॑ सोम सोम धौ॒तस्य॑ धौ॒तस्य॑ सोम । \newline
33. सो॒म॒ दे॒व॒ दे॒व॒ सो॒म॒ सो॒म॒ दे॒व॒ । \newline
34. दे॒व॒ ते॒ ते॒ दे॒व॒ दे॒व॒ ते॒ । \newline
35. ते॒ नृभि॒र् नृभि॑ स्ते ते॒ नृभिः॑ । \newline
36. नृभिः॑ सु॒तस्य॑ सु॒तस्य॒ नृभि॒र् नृभिः॑ सु॒तस्य॑ । \newline
37. नृभि॒रिति॒ नृ - भिः॒ । \newline
38. सु॒तस्ये॒ ष्टय॑जुष इ॒ष्टय॑जुषः सु॒तस्य॑ सु॒तस्ये॒ ष्टय॑जुषः । \newline
39. इ॒ष्टय॑जुषः स्तु॒तस्तो॑मस्य स्तु॒तस्तो॑म स्ये॒ष्टय॑जुष इ॒ष्टय॑जुषः स्तु॒तस्तो॑मस्य । \newline
40. इ॒ष्टय॑जुष॒ इती॒ष्ट - य॒जु॒षः॒ । \newline
41. स्तु॒तस्तो॑मस्य श॒स्तोक्थ॑स्य श॒स्तोक्थ॑स्य स्तु॒तस्तो॑मस्य स्तु॒तस्तो॑मस्य श॒स्तोक्थ॑स्य । \newline
42. स्तु॒तस्तो॑म॒स्येति॑ स्तु॒त - स्तो॒म॒स्य॒ । \newline
43. श॒स्तोक्थ॑स्य॒ यो यः श॒स्तोक्थ॑स्य श॒स्तोक्थ॑स्य॒ यः । \newline
44. श॒स्तोक्थ॒स्येति॑ श॒स्त - उ॒क्थ॒स्य॒ । \newline
45. यो भ॒क्षो भ॒क्षो यो यो भ॒क्षः । \newline
46. भ॒क्षो अ॑श्व॒सनि॑ रश्व॒सनि॑र् भ॒क्षो भ॒क्षो अ॑श्व॒सनिः॑ । \newline
47. अ॒श्व॒सनि॒र् यो यो अ॑श्व॒सनि॑ रश्व॒सनि॒र् यः । \newline
48. अ॒श्व॒सनि॒रित्य॑श्व - सनिः॑ । \newline
49. यो गो॒सनि॑र् गो॒सनि॒र् यो यो गो॒सनिः॑ । \newline
50. गो॒सनि॒ स्तस्य॒ तस्य॑ गो॒सनि॑र् गो॒सनि॒ स्तस्य॑ । \newline
51. गो॒सनि॒रिति॑ गो - सनिः॑ । \newline
52. तस्य॑ ते ते॒ तस्य॒ तस्य॑ ते । \newline
53. ते॒ पि॒तृभिः॑ पि॒तृभि॑ स्ते ते पि॒तृभिः॑ । \newline
54. पि॒तृभि॑र् भ॒क्षङ्‍कृ॑तस्य भ॒क्षङ्‍कृ॑तस्य पि॒तृभिः॑ पि॒तृभि॑र् भ॒क्षङ्‍कृ॑तस्य । \newline
55. पि॒तृभि॒रिति॑ पि॒तृ - भिः॒ । \newline
56. भ॒क्षङ्‍कृ॑त॒ स्योप॑हूत॒ स्योप॑हूतस्य भ॒क्षङ्‍कृ॑तस्य भ॒क्षङ्‍कृ॑त॒ स्योप॑हूतस्य । \newline
57. भ॒क्षङ्‍कृ॑त॒स्येति॑ भ॒क्षं - कृ॒त॒स्य॒ । \newline
58. उप॑हूत॒ स्योप॑हूत॒ उप॑हूत॒ उप॑हूत॒ स्योप॑हूत॒ स्योप॑हूतः । \newline
59. उप॑हूत॒स्येत्युप॑ - हू॒त॒स्य॒ । \newline
60. उप॑हूतो भक्षयामि भक्षया॒ म्युप॑हूत॒ उप॑हूतो भक्षयामि । \newline
61. उप॑हूत॒ इत्युप॑ - हू॒तः॒ । \newline
62. भ॒क्ष॒या॒मीति॑ भक्षयामि । \newline

\textbf{Ghana Paata } \newline

1. यत्का॑मा स्ते ते॒ यत्का॑मा॒ यत्का॑मा स्ते जुहु॒मो जु॑हु॒म स्ते॒ यत्का॑मा॒ यत्का॑मा स्ते जुहु॒मः । \newline
2. यत्का॑मा॒ इति॒ यत् - का॒माः॒ । \newline
3. ते॒ जु॒हु॒मो जु॑हु॒म स्ते॑ ते जुहु॒म स्तत् तज् जु॑हु॒म स्ते॑ ते जुहु॒म स्तत् । \newline
4. जु॒हु॒म स्तत् तज् जु॑हु॒मो जु॑हु॒म स्तन् नो॑ न॒ स्तज् जु॑हु॒मो जु॑हु॒म स्तन् नः॑ । \newline
5. तन् नो॑ न॒ स्तत् तन् नो॑ अस्त्वस्तु न॒ स्तत् तन् नो॑ अस्तु । \newline
6. नो॒ अ॒स्त्व॒स्तु॒ नो॒ नो॒ अ॒स्तु॒ व॒यं ॅव॒य म॑स्तु नो नो अस्तु व॒यम् । \newline
7. अ॒स्तु॒ व॒यं ॅव॒य म॑स्त्वस्तु व॒यꣳ स्या॑म स्याम व॒य म॑स्त्वस्तु व॒यꣳ स्या॑म । \newline
8. व॒यꣳ स्या॑म स्याम व॒यं ॅव॒यꣳ स्या॑म॒ पत॑यः॒ पत॑यः स्याम व॒यं ॅव॒यꣳ स्या॑म॒ पत॑यः । \newline
9. स्या॒म॒ पत॑यः॒ पत॑यः स्याम स्याम॒ पत॑यो रयी॒णाꣳ र॑यी॒णाम् पत॑यः स्याम स्याम॒ पत॑यो रयी॒णाम् । \newline
10. पत॑यो रयी॒णाꣳ र॑यी॒णाम् पत॑यः॒ पत॑यो रयी॒णाम् । \newline
11. र॒यी॒णामिति॑ रयी॒णाम् । \newline
12. दे॒वकृ॑त॒ स्यैन॑स॒ एन॑सो दे॒वकृ॑तस्य दे॒वकृ॑त॒ स्यैन॑सो ऽव॒यज॑न मव॒यज॑न॒ मेन॑सो दे॒वकृ॑तस्य दे॒वकृ॑त॒ स्यैन॑सो ऽव॒यज॑नम् । \newline
13. दे॒वकृ॑त॒स्येति॑ दे॒व - कृ॒त॒स्य॒ । \newline
14. एन॑सो ऽव॒यज॑न मव॒यज॑न॒ मेन॑स॒ एन॑सो ऽव॒यज॑न मस्य स्यव॒यज॑न॒ मेन॑स॒ एन॑सो ऽव॒यज॑न मसि । \newline
15. अ॒व॒यज॑न मस्य स्यव॒यज॑न मव॒यज॑न मसि मनु॒ष्य॑कृतस्य मनु॒ष्य॑कृतस्या स्यव॒यज॑न मव॒यज॑न मसि मनु॒ष्य॑कृतस्य । \newline
16. अ॒व॒यज॑न॒मित्य॑व - यज॑नम् । \newline
17. अ॒सि॒ म॒नु॒ष्य॑कृतस्य मनु॒ष्य॑कृतस्या स्यसि मनु॒ष्य॑कृत॒ स्यैन॑स॒ एन॑सो मनु॒ष्य॑कृतस्या स्यसि मनु॒ष्य॑कृत॒ स्यैन॑सः । \newline
18. म॒नु॒ष्य॑कृत॒ स्यैन॑स॒ एन॑सो मनु॒ष्य॑कृतस्य मनु॒ष्य॑कृत॒ स्यैन॑सो ऽव॒यज॑न मव॒यज॑न॒ मेन॑सो मनु॒ष्य॑कृतस्य मनु॒ष्य॑कृत॒ स्यैन॑सो ऽव॒यज॑नम् । \newline
19. म॒नु॒ष्य॑कृत॒स्येति॑ मनु॒ष्य॑ - कृ॒त॒स्य॒ । \newline
20. एन॑सो ऽव॒यज॑न मव॒यज॑न॒ मेन॑स॒ एन॑सो ऽव॒यज॑न मस्य स्यव॒यज॑न॒ मेन॑स॒ एन॑सो ऽव॒यज॑न मसि । \newline
21. अ॒व॒यज॑न मस्य स्यव॒यज॑न मव॒यज॑न मसि पि॒तृकृ॑तस्य पि॒तृकृ॑तस्या स्यव॒यज॑न मव॒यज॑न मसि पि॒तृकृ॑तस्य । \newline
22. अ॒व॒यज॑न॒मित्य॑व - यज॑नम् । \newline
23. अ॒सि॒ पि॒तृकृ॑तस्य पि॒तृकृ॑तस्या स्यसि पि॒तृकृ॑त॒ स्यैन॑स॒ एन॑सः पि॒तृकृ॑तस्या स्यसि पि॒तृकृ॑त॒ स्यैन॑सः । \newline
24. पि॒तृकृ॑त॒ स्यैन॑स॒ एन॑सः पि॒तृकृ॑तस्य पि॒तृकृ॑त॒ स्यैन॑सो ऽव॒यज॑न मव॒यज॑न॒ मेन॑सः पि॒तृकृ॑तस्य पि॒तृकृ॑त॒ स्यैन॑सो ऽव॒यज॑नम् । \newline
25. पि॒तृकृ॑त॒स्येति॑ पि॒तृ - कृ॒त॒स्य॒ । \newline
26. एन॑सो ऽव॒यज॑न मव॒यज॑न॒ मेन॑स॒ एन॑सो ऽव॒यज॑न मस्य स्यव॒यज॑न॒ मेन॑स॒ एन॑सो ऽव॒यज॑न मसि । \newline
27. अ॒व॒यज॑न मस्य स्यव॒यज॑न मव॒यज॑न मस्य॒फ्स्वा᳚(1॒)फ्स्व॑ स्यव॒यज॑न मव॒यज॑न मस्य॒फ्सु । \newline
28. अ॒व॒यज॑न॒मित्य॑व - यज॑नम् । \newline
29. अ॒स्य॒फ्स्वा᳚(1॒)फ्स्व॑स्य स्य॒फ्सु धौ॒तस्य॑ धौ॒तस्या॒ फ्स्व॑स्य स्य॒फ्सु धौ॒तस्य॑ । \newline
30. अ॒फ्सु धौ॒तस्य॑ धौ॒तस्या॒ फ्स्व॑फ्सु धौ॒तस्य॑ सोम सोम धौ॒तस्या॒ फ्स्व॑फ्सु धौ॒तस्य॑ सोम । \newline
31. अ॒फ्स्वित्य॑प् - सु । \newline
32. धौ॒तस्य॑ सोम सोम धौ॒तस्य॑ धौ॒तस्य॑ सोम देव देव सोम धौ॒तस्य॑ धौ॒तस्य॑ सोम देव । \newline
33. सो॒म॒ दे॒व॒ दे॒व॒ सो॒म॒ सो॒म॒ दे॒व॒ ते॒ ते॒ दे॒व॒ सो॒म॒ सो॒म॒ दे॒व॒ ते॒ । \newline
34. दे॒व॒ ते॒ ते॒ दे॒व॒ दे॒व॒ ते॒ नृभि॒र् नृभि॑ स्ते देव देव ते॒ नृभिः॑ । \newline
35. ते॒ नृभि॒र् नृभि॑ स्ते ते॒ नृभिः॑ सु॒तस्य॑ सु॒तस्य॒ नृभि॑ स्ते ते॒ नृभिः॑ सु॒तस्य॑ । \newline
36. नृभिः॑ सु॒तस्य॑ सु॒तस्य॒ नृभि॒र् नृभिः॑ सु॒त स्ये॒ष्टय॑जुष इ॒ष्टय॑जुषः सु॒तस्य॒ नृभि॒र् नृभिः॑ सु॒त स्ये॒ष्टय॑जुषः । \newline
37. नृभि॒रिति॒ नृ - भिः॒ । \newline
38. सु॒त स्ये॒ष्टय॑जुष इ॒ष्टय॑जुषः सु॒तस्य॑ सु॒त स्ये॒ष्टय॑जुषः स्तु॒तस्तो॑मस्य 
स्तु॒तस्तो॑म स्ये॒ष्टय॑जुषः सु॒तस्य॑ सु॒त स्ये॒ष्टय॑जुषः स्तु॒तस्तो॑मस्य । \newline
39. इ॒ष्टय॑जुषः स्तु॒तस्तो॑मस्य स्तु॒तस्तो॑म स्ये॒ष्टय॑जुष इ॒ष्टय॑जुषः स्तु॒तस्तो॑मस्य श॒स्तोक्थ॑स्य श॒स्तोक्थ॑स्य स्तु॒तस्तो॑म स्ये॒ष्टय॑जुष इ॒ष्टय॑जुषः स्तु॒तस्तो॑मस्य श॒स्तोक्थ॑स्य । \newline
40. इ॒ष्टय॑जुष॒ इती॒ष्ट - य॒जु॒षः॒ । \newline
41. स्तु॒तस्तो॑मस्य श॒स्तोक्थ॑स्य श॒स्तोक्थ॑स्य स्तु॒तस्तो॑मस्य स्तु॒तस्तो॑मस्य श॒स्तोक्थ॑स्य॒ यो यः श॒स्तोक्थ॑स्य स्तु॒तस्तो॑मस्य स्तु॒तस्तो॑मस्य श॒स्तोक्थ॑स्य॒ यः । \newline
42. स्तु॒तस्तो॑म॒स्येति॑ स्तु॒त - स्तो॒म॒स्य॒ । \newline
43. श॒स्तोक्थ॑स्य॒ यो यः श॒स्तोक्थ॑स्य श॒स्तोक्थ॑स्य॒ यो भ॒क्षो भ॒क्षो यः श॒स्तोक्थ॑स्य श॒स्तोक्थ॑स्य॒ यो भ॒क्षः । \newline
44. श॒स्तोक्थ॒स्येति॑ श॒स्त - उ॒क्थ॒स्य॒ । \newline
45. यो भ॒क्षो भ॒क्षो यो यो भ॒क्षो अ॑श्व॒सनि॑ रश्व॒सनि॑र् भ॒क्षो यो यो भ॒क्षो अ॑श्व॒सनिः॑ । \newline
46. भ॒क्षो अ॑श्व॒सनि॑ रश्व॒सनि॑र् भ॒क्षो भ॒क्षो अ॑श्व॒सनि॒र् यो यो अ॑श्व॒सनि॑र् भ॒क्षो भ॒क्षो अ॑श्व॒सनि॒र् यः । \newline
47. अ॒श्व॒सनि॒र् यो यो अ॑श्व॒सनि॑ रश्व॒सनि॒र् यो गो॒सनि॑र् गो॒सनि॒र् यो अ॑श्व॒सनि॑ रश्व॒सनि॒र् यो गो॒सनिः॑ । \newline
48. अ॒श्व॒सनि॒रित्य॑श्व - सनिः॑ । \newline
49. यो गो॒सनि॑र् गो॒सनि॒र् यो यो गो॒सनि॒ स्तस्य॒ तस्य॑ गो॒सनि॒र् यो यो गो॒सनि॒ स्तस्य॑ । \newline
50. गो॒सनि॒ स्तस्य॒ तस्य॑ गो॒सनि॑र् गो॒सनि॒ स्तस्य॑ ते ते॒ तस्य॑ गो॒सनि॑र् गो॒सनि॒ स्तस्य॑ ते । \newline
51. गो॒सनि॒रिति॑ गो - सनिः॑ । \newline
52. तस्य॑ ते ते॒ तस्य॒ तस्य॑ ते पि॒तृभिः॑ पि॒तृभि॑ स्ते॒ तस्य॒ तस्य॑ ते पि॒तृभिः॑ । \newline
53. ते॒ पि॒तृभिः॑ पि॒तृभि॑ स्ते ते पि॒तृभि॑र् भ॒क्षङ्‍कृ॑तस्य भ॒क्षङ्‍कृ॑तस्य पि॒तृभि॑ स्ते ते पि॒तृभि॑र् भ॒क्षङ्‍कृ॑तस्य । \newline
54. पि॒तृभि॑र् भ॒क्षङ्‍कृ॑तस्य भ॒क्षङ्‍कृ॑तस्य पि॒तृभिः॑ पि॒तृभि॑र् भ॒क्षङ्‍कृ॑त॒ स्योप॑हूत॒ स्योप॑हूतस्य भ॒क्षङ्‍कृ॑तस्य पि॒तृभिः॑ पि॒तृभि॑र् भ॒क्षङ्‍कृ॑त॒ स्योप॑हूतस्य । \newline
55. पि॒तृभि॒रिति॑ पि॒तृ - भिः॒ । \newline
56. भ॒क्षङ्‍कृ॑त॒ स्योप॑हूत॒ स्योप॑हूतस्य भ॒क्षङ्‍कृ॑तस्य भ॒क्षङ्‍कृ॑त॒ स्योप॑हूत॒ स्योप॑हूत॒ उप॑हूत॒ उप॑हूतस्य भ॒क्षङ्‍कृ॑तस्य भ॒क्षङ्‍कृ॑त॒ स्योप॑हूत॒ स्योप॑हूतः । \newline
57. भ॒क्षङ्‍कृ॑त॒स्येति॑ भ॒क्षं - कृ॒त॒स्य॒ । \newline
58. उप॑हूत॒ स्योप॑हूत॒ उप॑हूत॒ उप॑हूत॒ स्योप॑हूत॒ स्योप॑हूतो भक्षयामि भक्षया॒ म्युप॑हूत॒ उप॑हूत॒ स्योप॑हूत॒ स्योप॑हूतो भक्षयामि । \newline
59. उप॑हूत॒स्येत्युप॑ - हू॒त॒स्य॒ । \newline
60. उप॑हूतो भक्षयामि भक्षया॒ म्युप॑हूत॒ उप॑हूतो भक्षयामि । \newline
61. उप॑हूत॒ इत्युप॑ - हू॒तः॒ । \newline
62. भ॒क्ष॒या॒मीति॑ भक्षयामि । \newline
\pagebreak
\markright{ TS 3.2.6.1  \hfill https://www.vedavms.in \hfill}

\section{ TS 3.2.6.1 }

\textbf{TS 3.2.6.1 } \newline
\textbf{Samhita Paata} \newline

म॒ही॒नां पयो॑ऽसि॒ विश्वे॑षां दे॒वानां᳚ त॒नूर् ऋ॒द्ध्यास॑म॒द्य पृष॑तीनां॒ ग्रहं॒ पृष॑तीनां॒ ग्रहो॑ऽसि॒ विष्णो॒र्॒.हृद॑यम॒स्येक॑मिष॒ विष्णु॒स्त्वाऽनु॒ विच॑क्रमे भू॒तिर्द॒द्ध्ना घृ॒तेन॑ वर्द्धतां॒ तस्य॑ मे॒ष्टस्य॑ वी॒तस्य॒ द्रवि॑ण॒मा ग॑म्या॒ज्ज्योति॑रसि वैश्वान॒रं पृश्ञि॑यै दु॒ग्धं ॅयाव॑ती॒ द्यावा॑पृथि॒वी म॑हि॒त्वा याव॑च्च स॒प्त सिन्ध॑वो वित॒स्थुः । ताव॑न्तमिन्द्र ते॒ - [  ] \newline

\textbf{Pada Paata} \newline

म॒ही॒नाम् । पयः॑ । अ॒सि॒ । विश्वे॑षाम् । दे॒वाना᳚म् । त॒नूः । ऋ॒द्ध्यास᳚म् । अ॒द्य । पृष॑तीनाम् । ग्रह᳚म् । पृष॑तीनाम् । ग्रहः॑ । अ॒सि॒ । विष्णोः᳚ । हृद॑यम् । अ॒सि॒ । एक᳚म् । इ॒ष॒ । विष्णुः॑ । त्वा॒ । अनु॑ । वीति॑ । च॒क्र॒मे॒ । भू॒तिः । द॒द्ध्ना । घृ॒तेन॑ । व॒र्द्ध॒ता॒म् । तस्य॑ । मा॒ । इ॒ष्टस्य॑ । वी॒तस्य॑ । द्रवि॑णम् । एति॑ । ग॒म्या॒त् । ज्योतिः॑ । अ॒सि॒ । वै॒श्वा॒न॒रम् । पृश्नि॑यै । दु॒ग्धम् । याव॑ती॒ इति॑ । द्यावा॑पृथि॒वी इति॒ द्यावा᳚-पृ॒थि॒वी । म॒हि॒त्वेति॑ महि - त्वा । याव॑त् । च॒ । स॒प्त । सिन्ध॑वः । वि॒त॒स्थुरिति॑ वि - त॒स्थुः ॥ ताव॑न्तम् । इ॒न्द्र॒ । ते॒ ।  \newline


\textbf{Krama Paata} \newline

म॒ही॒नाम् पयः॑ । पयो॑ ऽसि । अ॒सि॒ विश्वे॑षाम् । विश्वे॑षाम् दे॒वाना᳚म् । दे॒वाना᳚म् त॒नूः । त॒नूर्. ऋ॒द्ध्यास᳚म् । ऋ॒द्ध्यास॑म॒द्य । अ॒द्य पृष॑तीनाम् । पृष॑तीना॒म् ग्रह᳚म् । ग्रह॒म् पृष॑तीनाम् । पृष॑तीना॒म् ग्रहः॑ । ग्रहो॑ ऽसि । अ॒सि॒ विष्णोः᳚ । विष्णो॒र्.॒ हृद॑यम् । हृद॑यमसि । अ॒स्येक᳚म् । एक॑मिष । इ॒ष॒ विष्णुः॑ । विष्णु॑स्त्वा । त्वा ऽनु॑ । अनु॒ वि । वि च॑क्रमे । च॒क्र॒मे॒ भू॒तिः । भू॒तिर् द॒द्ध्ना । द॒द्ध्ना घृ॒तेन॑ । घृ॒तेन॑ वर्द्धताम् । व॒र्द्ध॒ता॒म् तस्य॑ । तस्य॑ मा । मे॒ष्टस्य॑ । इ॒ष्टस्य॑ वी॒तस्य॑ । वी॒तस्य॒ द्रवि॑णम् । द्रवि॑ण॒मा । आ ग॑म्यात् । ग॒म्या॒ज् ज्योतिः॑ । ज्योति॑रसि । अ॒सि॒ वै॒श्वा॒न॒रम् । वै॒श्वा॒न॒रम् पृश्ञि॑यै । पृश्ञि॑यै दु॒ग्धम् । दु॒ग्धं ॅयाव॑ती । याव॑ती॒ द्यावा॑पृथि॒वी । याव॑ती॒ इति॒ याव॑ती । द्यावा॑पृथि॒वी म॑हि॒त्वा । द्यावा॑पृथि॒वी इति॒ द्यावा᳚ - पृ॒थि॒वी । म॒हि॒त्वा याव॑त् । म॒हि॒त्वेति॑ महि - त्वा । याव॑च् च । च॒ स॒प्त । स॒प्त सिन्ध॑वः । सिन्ध॑वो वित॒स्थुः । वि॒त॒स्थुरि॑ति वि - त॒स्थुः ॥ ताव॑न्तमिन्द्र । इ॒न्द्र॒ ते॒ । ते॒ ग्रह᳚म् \newline

\textbf{Jatai Paata} \newline

1. म॒ही॒नाम् पयः॒ पयो॑ मही॒नाम् म॑ही॒नाम् पयः॑ । \newline
2. पयो᳚ ऽस्यसि॒ पयः॒ पयो॑ ऽसि । \newline
3. अ॒सि॒ विश्वे॑षां॒ ॅविश्वे॑षा मस्यसि॒ विश्वे॑षाम् । \newline
4. विश्वे॑षाम् दे॒वाना᳚म् दे॒वानां॒ ॅविश्वे॑षां॒ ॅविश्वे॑षाम् दे॒वाना᳚म् । \newline
5. दे॒वाना᳚म् त॒नू स्त॒नूर् दे॒वाना᳚म् दे॒वाना᳚म् त॒नूः । \newline
6. त॒नूर्. ऋ॒द्ध्यास॑ मृ॒द्ध्यास॑म् त॒नू स्त॒नूर्. ऋ॒द्ध्यास᳚म् । \newline
7. ऋ॒द्ध्यास॑ म॒द्याद्य र्‌द्ध्यास॑ मृ॒द्ध्यास॑ म॒द्य । \newline
8. अ॒द्य पृष॑तीना॒म् पृष॑तीना म॒द्याद्य पृष॑तीनाम् । \newline
9. पृष॑तीना॒म् ग्रह॒म् ग्रह॒म् पृष॑तीना॒म् पृष॑तीना॒म् ग्रह᳚म् । \newline
10. ग्रह॒म् पृष॑तीना॒म् पृष॑तीना॒म् ग्रह॒म् ग्रह॒म् पृष॑तीनाम् । \newline
11. पृष॑तीना॒म् ग्रहो॒ ग्रहः॒ पृष॑तीना॒म् पृष॑तीना॒म् ग्रहः॑ । \newline
12. ग्रहो᳚ ऽस्यसि॒ ग्रहो॒ ग्रहो॑ ऽसि । \newline
13. अ॒सि॒ विष्णो॒र् विष्णो॑ रस्यसि॒ विष्णोः᳚ । \newline
14. विष्णो॒र्॒. हृद॑यꣳ॒॒ हृद॑यं॒ ॅविष्णो॒र् विष्णो॒र्॒. हृद॑यम् । \newline
15. हृद॑य मस्यसि॒ हृद॑यꣳ॒॒ हृद॑य मसि । \newline
16. अ॒स्येक॒ मेक॑ मस्य॒ स्येक᳚म् । \newline
17. एक॑ मिषे॒ षैक॒ मेक॑ मिष । \newline
18. इ॒ष॒ विष्णु॒र् विष्णु॑ रिषे ष॒ विष्णुः॑ । \newline
19. विष्णु॑ स्त्वा त्वा॒ विष्णु॒र् विष्णु॑ स्त्वा । \newline
20. त्वा ऽन्वनु॑ त्वा॒ त्वा ऽनु॑ । \newline
21. अनु॒ वि व्यन्वनु॒ वि । \newline
22. वि च॑क्रमे चक्रमे॒ वि वि च॑क्रमे । \newline
23. च॒क्र॒मे॒ भू॒तिर् भू॒ति श्च॑क्रमे चक्रमे भू॒तिः । \newline
24. भू॒तिर् द॒द्ध्ना द॒द्ध्ना भू॒तिर् भू॒तिर् द॒द्ध्ना । \newline
25. द॒द्ध्ना घृ॒तेन॑ घृ॒तेन॑ द॒द्ध्ना द॒द्ध्ना घृ॒तेन॑ । \newline
26. घृ॒तेन॑ वर्द्धतां ॅवर्द्धताम् घृ॒तेन॑ घृ॒तेन॑ वर्द्धताम् । \newline
27. व॒र्द्ध॒ता॒म् तस्य॒ तस्य॑ वर्द्धतां ॅवर्द्धता॒म् तस्य॑ । \newline
28. तस्य॑ मा मा॒ तस्य॒ तस्य॑ मा । \newline
29. मे॒ष्ट स्ये॒ष्टस्य॑ मा मे॒ष्टस्य॑ । \newline
30. इ॒ष्टस्य॑ वी॒तस्य॑ वी॒त स्ये॒ष्ट स्ये॒ष्टस्य॑ वी॒तस्य॑ । \newline
31. वी॒तस्य॒ द्रवि॑ण॒म् द्रवि॑णं ॅवी॒तस्य॑ वी॒तस्य॒ द्रवि॑णम् । \newline
32. द्रवि॑ण॒ मा द्रवि॑ण॒म् द्रवि॑ण॒ मा । \newline
33. आ ग॑म्याद् गम्या॒दा ग॑म्यात् । \newline
34. ग॒म्या॒ज् ज्योति॒र् ज्योति॑र् गम्याद् गम्या॒ज् ज्योतिः॑ । \newline
35. ज्योति॑ रस्यसि॒ ज्योति॒र् ज्योति॑ रसि । \newline
36. अ॒सि॒ वै॒श्वा॒न॒रं ॅवै᳚श्वान॒र म॑स्यसि वैश्वान॒रम् । \newline
37. वै॒श्वा॒न॒रम् पृश्ञि॑यै॒ पृश्ञि॑यै वैश्वान॒रं ॅवै᳚श्वान॒रम् पृश्ञि॑यै । \newline
38. पृश्ञि॑यै दु॒ग्धम् दु॒ग्धम् पृश्ञि॑यै॒ पृश्ञि॑यै दु॒ग्धम् । \newline
39. दु॒ग्धं ॅयाव॑ती॒ याव॑ती दु॒ग्धम् दु॒ग्धं ॅयाव॑ती । \newline
40. याव॑ती॒ द्यावा॑पृथि॒वी द्यावा॑पृथि॒वी याव॑ती॒ याव॑ती॒ द्यावा॑पृथि॒वी । \newline
41. याव॑ती॒ इति॒ याव॑ती । \newline
42. द्यावा॑पृथि॒वी म॑हि॒त्वा म॑हि॒त्वा द्यावा॑पृथि॒वी द्यावा॑पृथि॒वी म॑हि॒त्वा । \newline
43. द्यावा॑पृथि॒वी इति॒ द्यावा᳚ - पृ॒थि॒वी । \newline
44. म॒हि॒त्वा याव॒द् याव॑न् महि॒त्वा म॑हि॒त्वा याव॑त् । \newline
45. म॒हि॒त्वेति॑ महि - त्वा । \newline
46. याव॑च् च च॒ याव॒द् याव॑च् च । \newline
47. च॒ स॒प्त स॒प्त च॑ च स॒प्त । \newline
48. स॒प्त सिन्ध॑वः॒ सिन्ध॑वः स॒प्त स॒प्त सिन्ध॑वः । \newline
49. सिन्ध॑वो वित॒स्थुर् वि॑त॒स्थुः सिन्ध॑वः॒ सिन्ध॑वो वित॒स्थुः । \newline
50. वि॒त॒स्थुरिति॑ वि - त॒स्थुः । \newline
51. ताव॑न्त मिन्द्रे न्द्र॒ ताव॑न्त॒म् ताव॑न्त मिन्द्र । \newline
52. इ॒न्द्र॒ ते॒ त॒ इ॒न्द्रे॒ न्द्र॒ ते॒ । \newline
53. ते॒ ग्रह॒म् ग्रह॑म् ते ते॒ ग्रह᳚म् । \newline

\textbf{Ghana Paata } \newline

1. म॒ही॒नाम् पयः॒ पयो॑ मही॒नाम् म॑ही॒नाम् पयो᳚ ऽस्यसि॒ पयो॑ मही॒नाम् म॑ही॒नाम् पयो॑ ऽसि । \newline
2. पयो᳚ ऽस्यसि॒ पयः॒ पयो॑ ऽसि॒ विश्वे॑षां॒ ॅविश्वे॑षा मसि॒ पयः॒ पयो॑ ऽसि॒ विश्वे॑षाम् । \newline
3. अ॒सि॒ विश्वे॑षां॒ ॅविश्वे॑षा मस्यसि॒ विश्वे॑षाम् दे॒वाना᳚म् दे॒वानां॒ ॅविश्वे॑षा मस्यसि॒ विश्वे॑षाम् दे॒वाना᳚म् । \newline
4. विश्वे॑षाम् दे॒वाना᳚म् दे॒वानां॒ ॅविश्वे॑षां॒ ॅविश्वे॑षाम् दे॒वाना᳚म् त॒नू स्त॒नूर् दे॒वानां॒ ॅविश्वे॑षां॒ ॅविश्वे॑षाम् दे॒वाना᳚म् त॒नूः । \newline
5. दे॒वाना᳚म् त॒नू स्त॒नूर् दे॒वाना᳚म् दे॒वाना᳚म् त॒नूर्. ऋ॒द्ध्यास॑ मृ॒द्ध्यास॑म् त॒नूर् दे॒वाना᳚म् दे॒वाना᳚म् त॒नूर्. ऋ॒द्ध्यास᳚म् । \newline
6. त॒नूर्. ऋ॒द्ध्यास॑ मृ॒द्ध्यास॑म् त॒नू स्त॒नूर्. ऋ॒द्ध्यास॑ म॒द्याद्य र्‌द्ध्यास॑म् त॒नू स्त॒नूर्. ऋ॒द्ध्यास॑ म॒द्य । \newline
7. ऋ॒द्ध्यास॑ म॒द्याद्य र्‌द्ध्यास॑ मृ॒द्ध्यास॑ म॒द्य पृष॑तीना॒म् पृष॑तीना म॒द्य र्‌द्ध्यास॑ मृ॒द्ध्यास॑ म॒द्य पृष॑तीनाम् । \newline
8. अ॒द्य पृष॑तीना॒म् पृष॑तीना म॒द्याद्य पृष॑तीना॒म् ग्रह॒म् ग्रह॒म् पृष॑तीना म॒द्याद्य पृष॑तीना॒म् ग्रह᳚म् । \newline
9. पृष॑तीना॒म् ग्रह॒म् ग्रह॒म् पृष॑तीना॒म् पृष॑तीना॒म् ग्रह॒म् पृष॑तीना॒म् पृष॑तीना॒म् ग्रह॒म् पृष॑तीना॒म् पृष॑तीना॒म् ग्रह॒म् पृष॑तीनाम् । \newline
10. ग्रह॒म् पृष॑तीना॒म् पृष॑तीना॒म् ग्रह॒म् ग्रह॒म् पृष॑तीना॒म् ग्रहो॒ ग्रहः॒ पृष॑तीना॒म् ग्रह॒म् ग्रह॒म् पृष॑तीना॒म् ग्रहः॑ । \newline
11. पृष॑तीना॒म् ग्रहो॒ ग्रहः॒ पृष॑तीना॒म् पृष॑तीना॒म् ग्रहो᳚ ऽस्यसि॒ ग्रहः॒ पृष॑तीना॒म् पृष॑तीना॒म् ग्रहो॑ ऽसि । \newline
12. ग्रहो᳚ ऽस्यसि॒ ग्रहो॒ ग्रहो॑ ऽसि॒ विष्णो॒र् विष्णो॑ रसि॒ ग्रहो॒ ग्रहो॑ ऽसि॒ विष्णोः᳚ । \newline
13. अ॒सि॒ विष्णो॒र् विष्णो॑ रस्यसि॒ विष्णो॒र्॒. हृद॑यꣳ॒॒ हृद॑यं॒ ॅविष्णो॑ रस्यसि॒ विष्णो॒र्॒. हृद॑यम् । \newline
14. विष्णो॒र्॒. हृद॑यꣳ॒॒ हृद॑यं॒ ॅविष्णो॒र् विष्णो॒र्॒. हृद॑य मस्यसि॒ हृद॑यं॒ ॅविष्णो॒र् 
विष्णो॒र्॒. हृद॑य मसि । \newline
15. हृद॑य मस्यसि॒ हृद॑यꣳ॒॒ हृद॑य म॒स्येक॒ मेक॑ मसि॒ हृद॑यꣳ॒॒ हृद॑य म॒स्येक᳚म् । \newline
16. अ॒स्येक॒ मेक॑ मस्य॒स्येक॑ मिषे॒ षैक॑ मस्य॒स्येक॑ मिष । \newline
17. एक॑ मिषे॒ षैक॒ मेक॑ मिष॒ विष्णु॒र् विष्णु॑ रि॒षैक॒ मेक॑ मिष॒ विष्णुः॑ । \newline
18. इ॒ष॒ विष्णु॒र् विष्णु॑ रिषे ष॒ विष्णु॑ स्त्वा त्वा॒ विष्णु॑रिषे ष॒ विष्णु॑ स्त्वा । \newline
19. विष्णु॑ स्त्वा त्वा॒ विष्णु॒र् विष्णु॒ स्त्वा ऽन्वनु॑ त्वा॒ विष्णु॒र् विष्णु॒ स्त्वा ऽनु॑ । \newline
20. त्वा ऽन्वनु॑ त्वा॒ त्वा ऽनु॒ वि व्यनु॑ त्वा॒ त्वा ऽनु॒ वि । \newline
21. अनु॒ वि व्यन्वनु॒ वि च॑क्रमे चक्रमे॒ व्यन्वनु॒ वि च॑क्रमे । \newline
22. वि च॑क्रमे चक्रमे॒ वि वि च॑क्रमे भू॒तिर् भू॒ति श्च॑क्रमे॒ वि वि च॑क्रमे भू॒तिः । \newline
23. च॒क्र॒मे॒ भू॒तिर् भू॒ति श्च॑क्रमे चक्रमे भू॒तिर् द॒द्ध्ना द॒द्ध्ना भू॒ति श्च॑क्रमे चक्रमे भू॒तिर् द॒द्ध्ना । \newline
24. भू॒तिर् द॒द्ध्ना द॒द्ध्ना भू॒तिर् भू॒तिर् द॒द्ध्ना घृ॒तेन॑ घृ॒तेन॑ द॒द्ध्ना भू॒तिर् भू॒तिर् द॒द्ध्ना घृ॒तेन॑ । \newline
25. द॒द्ध्ना घृ॒तेन॑ घृ॒तेन॑ द॒द्ध्ना द॒द्ध्ना घृ॒तेन॑ वर्द्धतां ॅवर्द्धताम् घृ॒तेन॑ द॒द्ध्ना द॒द्ध्ना घृ॒तेन॑ वर्द्धताम् । \newline
26. घृ॒तेन॑ वर्द्धतां ॅवर्द्धताम् घृ॒तेन॑ घृ॒तेन॑ वर्द्धता॒म् तस्य॒ तस्य॑ वर्द्धताम् घृ॒तेन॑ घृ॒तेन॑ वर्द्धता॒म् तस्य॑ । \newline
27. व॒र्द्ध॒ता॒म् तस्य॒ तस्य॑ वर्द्धतां ॅवर्द्धता॒म् तस्य॑ मा मा॒ तस्य॑ वर्द्धतां ॅवर्द्धता॒म् तस्य॑ मा । \newline
28. तस्य॑ मा मा॒ तस्य॒ तस्य॑ मे॒ष्ट स्ये॒ष्टस्य॑ मा॒ तस्य॒ तस्य॑ मे॒ष्टस्य॑ । \newline
29. मे॒ष्ट स्ये॒ष्टस्य॑ मा मे॒ष्टस्य॑ वी॒तस्य॑ वी॒त स्ये॒ष्टस्य॑ मा मे॒ष्टस्य॑ वी॒तस्य॑ । \newline
30. इ॒ष्टस्य॑ वी॒तस्य॑ वी॒त स्ये॒ष्ट स्ये॒ष्टस्य॑ वी॒तस्य॒ द्रवि॑ण॒म् द्रवि॑णं ॅवी॒त स्ये॒ष्ट स्ये॒ष्टस्य॑ वी॒तस्य॒ द्रवि॑णम् । \newline
31. वी॒तस्य॒ द्रवि॑ण॒म् द्रवि॑णं ॅवी॒तस्य॑ वी॒तस्य॒ द्रवि॑ण॒ मा द्रवि॑णं ॅवी॒तस्य॑ वी॒तस्य॒ द्रवि॑ण॒ मा । \newline
32. द्रवि॑ण॒ मा द्रवि॑ण॒म् द्रवि॑ण॒ मा ग॑म्याद् गम्या॒दा द्रवि॑ण॒म् द्रवि॑ण॒ मा ग॑म्यात् । \newline
33. आ ग॑म्याद् गम्या॒दा ग॑म्या॒ज् ज्योति॒र् ज्योति॑र् गम्या॒दा ग॑म्या॒ज् ज्योतिः॑ । \newline
34. ग॒म्या॒ज् ज्योति॒र् ज्योति॑र् गम्याद् गम्या॒ज् ज्योति॑ रस्यसि॒ ज्योति॑र् गम्याद् गम्या॒ज् ज्योति॑रसि । \newline
35. ज्योति॑ रस्यसि॒ ज्योति॒र् ज्योति॑ रसि वैश्वान॒रं ॅवै᳚श्वान॒र म॑सि॒ ज्योति॒र् ज्योति॑ रसि वैश्वान॒रम् । \newline
36. अ॒सि॒ वै॒श्वा॒न॒रं ॅवै᳚श्वान॒र म॑स्यसि वैश्वान॒रम् पृश्ञि॑यै॒ पृश्ञि॑यै वैश्वान॒र म॑स्यसि वैश्वान॒रम् पृश्ञि॑यै । \newline
37. वै॒श्वा॒न॒रम् पृश्ञि॑यै॒ पृश्ञि॑यै वैश्वान॒रं ॅवै᳚श्वान॒रम् पृश्ञि॑यै दु॒ग्धम् दु॒ग्धम् पृश्ञि॑यै वैश्वान॒रं ॅवै᳚श्वान॒रम् पृश्ञि॑यै दु॒ग्धम् । \newline
38. पृश्ञि॑यै दु॒ग्धम् दु॒ग्धम् पृश्ञि॑यै॒ पृश्ञि॑यै दु॒ग्धं ॅयाव॑ती॒ याव॑ती दु॒ग्धम् पृश्ञि॑यै॒ पृश्ञि॑यै दु॒ग्धं ॅयाव॑ती । \newline
39. दु॒ग्धं ॅयाव॑ती॒ याव॑ती दु॒ग्धम् दु॒ग्धं ॅयाव॑ती॒ द्यावा॑पृथि॒वी द्यावा॑पृथि॒वी याव॑ती दु॒ग्धम् दु॒ग्धं ॅयाव॑ती॒ द्यावा॑पृथि॒वी । \newline
40. याव॑ती॒ द्यावा॑पृथि॒वी द्यावा॑पृथि॒वी याव॑ती॒ याव॑ती॒ द्यावा॑पृथि॒वी म॑हि॒त्वा म॑हि॒त्वा द्यावा॑पृथि॒वी याव॑ती॒ याव॑ती॒ द्यावा॑पृथि॒वी म॑हि॒त्वा । \newline
41. याव॑ती॒ इति॒ याव॑ती । \newline
42. द्यावा॑पृथि॒वी म॑हि॒त्वा म॑हि॒त्वा द्यावा॑पृथि॒वी द्यावा॑पृथि॒वी म॑हि॒त्वा याव॒द् याव॑न् महि॒त्वा द्यावा॑पृथि॒वी द्यावा॑पृथि॒वी म॑हि॒त्वा याव॑त् । \newline
43. द्यावा॑पृथि॒वी इति॒ द्यावा᳚-पृ॒थि॒वी । \newline
44. म॒हि॒त्वा याव॒द् याव॑न् महि॒त्वा म॑हि॒त्वा याव॑च् च च॒ याव॑न् महि॒त्वा म॑हि॒त्वा याव॑च् च । \newline
45. म॒हि॒त्वेति॑ महि - त्वा । \newline
46. याव॑च् च च॒ याव॒द् याव॑च् च स॒प्त स॒प्त च॒ याव॒द् याव॑च् च स॒प्त । \newline
47. च॒ स॒प्त स॒प्त च॑ च स॒प्त सिन्ध॑वः॒ सिन्ध॑वः स॒प्त च॑ च स॒प्त सिन्ध॑वः । \newline
48. स॒प्त सिन्ध॑वः॒ सिन्ध॑वः स॒प्त स॒प्त सिन्ध॑वो वित॒स्थुर् वि॑त॒स्थुः सिन्ध॑वः स॒प्त स॒प्त सिन्ध॑वो वित॒स्थुः । \newline
49. सिन्ध॑वो वित॒स्थुर् वि॑त॒स्थुः सिन्ध॑वः॒ सिन्ध॑वो वित॒स्थुः । \newline
50. वि॒त॒स्थुरिति॑ वि - त॒स्थुः । \newline
51. ताव॑न्त मिन्द्रे न्द्र॒ ताव॑न्त॒म् ताव॑न्त मिन्द्र ते त इन्द्र॒ ताव॑न्त॒म् ताव॑न्त मिन्द्र ते । \newline
52. इ॒न्द्र॒ ते॒ त॒ इ॒न्द्रे॒ न्द्र॒ ते॒ ग्रह॒म् ग्रह॑म् त इन्द्रे न्द्र ते॒ ग्रह᳚म् । \newline
53. ते॒ ग्रह॒म् ग्रह॑म् ते ते॒ ग्रहꣳ॑ स॒ह स॒ह ग्रह॑म् ते ते॒ ग्रहꣳ॑ स॒ह । \newline
\pagebreak
\markright{ TS 3.2.6.2  \hfill https://www.vedavms.in \hfill}

\section{ TS 3.2.6.2 }

\textbf{TS 3.2.6.2 } \newline
\textbf{Samhita Paata} \newline

ग्रहꣳ॑ स॒होर्जा गृ॑ह्णा॒म्यस्तृ॑तं ॥ यत् कृ॑ष्णशकु॒नः पृ॑षदा॒ज्यम॑वमृ॒शेच्छू॒द्रा अ॑स्य प्र॒मायु॑काः स्यु॒र्यच्छ्वा ऽव॑मृ॒शेच्चतु॑ष्पादोऽस्य प॒शवः॑ प्र॒मायु॑काः स्यु॒र्यथ् स्कन्दे॒द्-यज॑मानः प्र॒मायु॑कः स्यात् प॒शवो॒ वै पृ॑षदा॒ज्यं प॒शवो॒ वा ए॒तस्य॑ स्कन्दन्ति॒ यस्य॑ पृषदा॒ज्यꣳ स्कन्द॑ति॒ यत् पृ॑षदा॒ज्यं पुन॑र्गृ॒ह्णाति॑ प॒शूने॒वास्मै॒ पुन॑र्गृह्णाति प्रा॒णो वै पृ॑षदा॒ज्यं प्रा॒णो वा - [  ] \newline

\textbf{Pada Paata} \newline

ग्रह᳚म् । स॒ह । ऊ॒र्जा । गृ॒ह्णा॒मि॒ । अस्तृ॑तम् ॥ यत् । कृ॒ष्ण॒श॒कु॒न इति॑ कृष्ण - श॒कु॒नः । पृ॒ष॒दा॒ज्यमिति॑ पृषत् - आ॒ज्यम् । अ॒व॒मृ॒शेदित्य॑व - मृ॒शेत् । शू॒द्राः । अ॒स्य॒ । प्र॒मायु॑का॒ इति॑ प्र - मायु॑काः । स्युः॒ । यत् । श्वा । अ॒व॒मृ॒शेदित्य॑व - मृ॒शेत् । चतु॑ष्पाद॒ इति॒ चतुः॑ - पा॒दः॒ । अ॒स्य॒ । प॒शवः॑ । प्र॒मायु॑का॒ इति॑ प्र - मायु॑काः । स्युः॒ । यत् । स्कन्दे᳚त् । यज॑मानः । प्र॒मायु॑क॒ इति॑ प्र - मायु॑कः । स्या॒त् । प॒शवः॑ । वै । पृ॒ष॒दा॒ज्यमिति॑ पृषत्-आ॒ज्यम् । प॒शवः॑ । वै । ए॒तस्य॑ । स्क॒न्द॒न्ति॒ । यस्य॑ । पृ॒ष॒दा॒ज्यमिति॑ पृषत् - आ॒ज्यम् । स्कन्द॑ति । यत् । पृ॒ष॒दा॒ज्यमिति॑ पृषत् - आ॒ज्यम् । पुनः॑ । गृ॒ह्णाति॑ । प॒शून् । ए॒व । अ॒स्मै॒ । पुनः॑ । गृ॒ह्णा॒ति॒ । प्रा॒ण इति॑ प्र - अ॒नः । वै । पृ॒ष॒दा॒ज्यमिति॑ पृषत् - आ॒ज्यम् । प्रा॒ण इति॑ प्र - अ॒नः । वै ॥  \newline


\textbf{Krama Paata} \newline

ग्रहꣳ॑ स॒ह । स॒होर्जा । ऊ॒र्जा गृ॑ह्णामि । गृ॒ह्णा॒म्यस्तृ॑तम् । अस्तृ॑त॒मित्यस्तृ॑तम् ॥ यत् कृ॑ष्णशकु॒नः । कृ॒ष्ण॒श॒कु॒नः पृ॑षदा॒ज्यम् । कृ॒ष्ण॒श॒कु॒न इति॑ कृष्ण - श॒कु॒नः । पृ॒ष॒दा॒ज्य,म॑वमृ॒शेत् । पृ॒ष॒दा॒ज्यमिति॑ पृषत् - आ॒ज्यम् । अ॒व॒मृ॒शेच्छू॒द्राः । अ॒व॒मृ॒शेदित्य॑व - मृ॒शेत् । शू॒द्रा अ॑स्य । अ॒स्य॒ प्र॒मायु॑काः । प्र॒मायु॑काः स्युः । प्र॒मायु॑का॒ इति॑ प्र - मायु॑काः । स्यु॒र् यत् । यच्छ्वा । श्वा ऽव॑मृ॒शेत् । अ॒व॒मृ॒शेच् चतु॑ष्पादः । अ॒व॒मृ॒शेदित्य॑व - मृ॒शेत् । चतु॑ष्पादो ऽस्य । चतु॑ष्पाद॒ इति॒ चतुः॑ - पा॒दः॒ । अ॒स्य॒ प॒शवः॑ । प॒शवः॑ प्र॒मायु॑काः । प्र॒मायु॑काः स्युः । प्र॒मायु॑का॒ इति॑ प्र - मायु॑काः । स्यु॒र् यत् । यथ् स्कन्दे᳚त् । स्कन्दे॒द् यज॑मानः । यज॑मानः प्र॒मायु॑कः । प्र॒मायु॑कः स्यात् । प्र॒मायु॑क॒ इति॑ प्र - मायु॑कः । स्या॒त् प॒शवः॑ । प॒शवो॒ वै । वै पृ॑षदा॒ज्यम् । पृ॒ष॒दा॒ज्यम् प॒शवः॑ । पृ॒ष॒दा॒ज्यमिति॑ पृषत् - आ॒ज्यम् । प॒शवो॒ वै । वा ए॒तस्य॑ । ए॒तस्य॑ स्कन्दन्ति । स्क॒न्द॒न्ति॒ यस्य॑ । यस्य॑ पृषदा॒ज्यम् । पृ॒ष॒दा॒ज्यꣳ स्कन्द॑ति । पृ॒ष॒दा॒ज्यमिति॑ पृषत् - आ॒ज्यम् । स्कन्द॑ति॒ यत् । यत् पृ॑षदा॒ज्यम् । पृ॒ष॒दा॒ज्यम् पुनः॑ । पृ॒ष॒दा॒ज्यमिति॑ पृषत् - आ॒ज्यम् । पुन॑र् गृ॒ह्णाति॑ । गृ॒ह्णाति॑ प॒शून् । प॒शूने॒व । ए॒वास्मै᳚ । अ॒स्मै॒ पुनः॑ । पुन॑र् गृह्णाति । गृ॒ह्णा॒ति॒ प्रा॒णः । प्रा॒णो वै । प्रा॒ण इति॑ प्र - अ॒नः । वै पृ॑षदा॒ज्यम् । पृ॒ष॒दा॒ज्यम् प्रा॒णः । पृ॒ष॒दा॒ज्यमिति॑ पृषत् - आ॒ज्यम् । प्रा॒णो वै । प्रा॒ण इति॑ प्र - अ॒नः । वा ए॒तस्य॑ \newline

\textbf{Jatai Paata} \newline

1. ग्रहꣳ॑ स॒ह स॒ह ग्रह॒म् ग्रहꣳ॑ स॒ह । \newline
2. स॒होर्जो र्जा स॒ह स॒होर्जा । \newline
3. ऊ॒र्जा गृ॑ह्णामि गृह्णा म्यू॒र्जोर्जा गृ॑ह्णामि । \newline
4. गृ॒ह्णा॒ म्यस्तृ॑त॒ मस्तृ॑तम् गृह्णामि गृह्णा॒ म्यस्तृ॑तम् । \newline
5. अस्तृ॑त॒मित्यस्तृ॑तम् । \newline
6. यत् कृ॑ष्णशकु॒नः कृ॑ष्णशकु॒नो यद् यत् कृ॑ष्णशकु॒नः । \newline
7. कृ॒ष्ण॒श॒कु॒नः पृ॑षदा॒ज्यम् पृ॑षदा॒ज्यम् कृ॑ष्णशकु॒नः कृ॑ष्णशकु॒नः पृ॑षदा॒ज्यम् । \newline
8. कृ॒ष्ण॒श॒कु॒न इति॑ कृष्ण - श॒कु॒नः । \newline
9. पृ॒ष॒दा॒ज्य म॑वमृ॒शे द॑वमृ॒शेत् पृ॑षदा॒ज्यम् पृ॑षदा॒ज्य म॑वमृ॒शेत् । \newline
10. पृ॒ष॒दा॒ज्यमिति॑ पृषत् - आ॒ज्यम् । \newline
11. अ॒व॒मृ॒शे च्छू॒द्राः शू॒द्रा अ॑वमृ॒शे द॑वमृ॒शे च्छू॒द्राः । \newline
12. अ॒व॒मृ॒शेदित्य॑व - मृ॒शेत् । \newline
13. शू॒द्रा अ॑स्यास्य शू॒द्राः शू॒द्रा अ॑स्य । \newline
14. अ॒स्य॒ प्र॒मायु॑काः प्र॒मायु॑का अस्यास्य प्र॒मायु॑काः । \newline
15. प्र॒मायु॑काः स्युः स्युः प्र॒मायु॑काः प्र॒मायु॑काः स्युः । \newline
16. प्र॒मायु॑का॒ इति॑ प्र - मायु॑काः । \newline
17. स्यु॒र् यद् यथ् स्युः॑ स्यु॒र् यत् । \newline
18. यच्छ्‌वा श्वा यद् यच्छ्‌वा । \newline
19. श्वा ऽव॑मृ॒शे द॑वमृ॒शे च्छ्‌वा श्वा ऽव॑मृ॒शेत् । \newline
20. अ॒व॒मृ॒शेच् चतु॑ष्पाद॒ श्चतु॑ष्पादो ऽवमृ॒शे द॑वमृ॒शेच् चतु॑ष्पादः । \newline
21. अ॒व॒मृ॒शेदित्य॑व - मृ॒शेत् । \newline
22. चतु॑ष्पादो ऽस्यास्य॒ चतु॑ष्पाद॒ श्चतु॑ष्पादो ऽस्य । \newline
23. चतु॑ष्पाद॒ इति॒ चतुः॑ - पा॒दः॒ । \newline
24. अ॒स्य॒ प॒शवः॑ प॒शवो᳚ ऽस्यास्य प॒शवः॑ । \newline
25. प॒शवः॑ प्र॒मायु॑काः प्र॒मायु॑काः प॒शवः॑ प॒शवः॑ प्र॒मायु॑काः । \newline
26. प्र॒मायु॑काः स्युः स्युः प्र॒मायु॑काः प्र॒मायु॑काः स्युः । \newline
27. प्र॒मायु॑का॒ इति॑ प्र - मायु॑काः । \newline
28. स्यु॒र् यद् यथ् स्युः॑ स्यु॒र् यत् । \newline
29. यथ् स्कन्दे॒थ् स्कन्दे॒द् यद् यथ् स्कन्दे᳚त् । \newline
30. स्कन्दे॒द् यज॑मानो॒ यज॑मानः॒ स्कन्दे॒थ् स्कन्दे॒द् यज॑मानः । \newline
31. यज॑मानः प्र॒मायु॑कः प्र॒मायु॑को॒ यज॑मानो॒ यज॑मानः प्र॒मायु॑कः । \newline
32. प्र॒मायु॑कः स्याथ् स्यात् प्र॒मायु॑कः प्र॒मायु॑कः स्यात् । \newline
33. प्र॒मायु॑क॒ इति॑ प्र - मायु॑कः । \newline
34. स्या॒त् प॒शवः॑ प॒शवः॑ स्याथ् स्यात् प॒शवः॑ । \newline
35. प॒शवो॒ वै वै प॒शवः॑ प॒शवो॒ वै । \newline
36. वै पृ॑षदा॒ज्यम् पृ॑षदा॒ज्यं ॅवै वै पृ॑षदा॒ज्यम् । \newline
37. पृ॒ष॒दा॒ज्यम् प॒शवः॑ प॒शवः॑ पृषदा॒ज्यम् पृ॑षदा॒ज्यम् प॒शवः॑ । \newline
38. पृ॒ष॒दा॒ज्यमिति॑ पृषत् - आ॒ज्यम् । \newline
39. प॒शवो॒ वै वै प॒शवः॑ प॒शवो॒ वै । \newline
40. वा ए॒त स्यै॒तस्य॒ वै वा ए॒तस्य॑ । \newline
41. ए॒तस्य॑ स्कन्दन्ति स्कन्द न्त्ये॒त स्यै॒तस्य॑ स्कन्दन्ति । \newline
42. स्क॒न्द॒न्ति॒ यस्य॒ यस्य॑ स्कन्दन्ति स्कन्दन्ति॒ यस्य॑ । \newline
43. यस्य॑ पृषदा॒ज्यम् पृ॑षदा॒ज्यं ॅयस्य॒ यस्य॑ पृषदा॒ज्यम् । \newline
44. पृ॒ष॒दा॒ज्यꣳ स्कन्द॑ति॒ स्कन्द॑ति पृषदा॒ज्यम् पृ॑षदा॒ज्यꣳ स्कन्द॑ति । \newline
45. पृ॒ष॒दा॒ज्यमिति॑ पृषत् - आ॒ज्यम् । \newline
46. स्कन्द॑ति॒ यद् यथ् स्कन्द॑ति॒ स्कन्द॑ति॒ यत् । \newline
47. यत् पृ॑षदा॒ज्यम् पृ॑षदा॒ज्यं ॅयद् यत् पृ॑षदा॒ज्यम् । \newline
48. पृ॒ष॒दा॒ज्यम् पुनः॒ पुनः॑ पृषदा॒ज्यम् पृ॑षदा॒ज्यम् पुनः॑ । \newline
49. पृ॒ष॒दा॒ज्यमिति॑ पृषत् - आ॒ज्यम् । \newline
50. पुन॑र् गृ॒ह्णाति॑ गृ॒ह्णाति॒ पुनः॒ पुन॑र् गृ॒ह्णाति॑ । \newline
51. गृ॒ह्णाति॑ प॒शून् प॒शून् गृ॒ह्णाति॑ गृ॒ह्णाति॑ प॒शून् । \newline
52. प॒शू ने॒वैव प॒शून् प॒शू ने॒व । \newline
53. ए॒वास्मा॑ अस्मा ए॒वैवास्मै᳚ । \newline
54. अ॒स्मै॒ पुनः॒ पुन॑ रस्मा अस्मै॒ पुनः॑ । \newline
55. पुन॑र् गृह्णाति गृह्णाति॒ पुनः॒ पुन॑र् गृह्णाति । \newline
56. गृ॒ह्णा॒ति॒ प्रा॒णः प्रा॒णो गृ॑ह्णाति गृह्णाति प्रा॒णः । \newline
57. प्रा॒णो वै वै प्रा॒णः प्रा॒णो वै । \newline
58. प्रा॒ण इति॑ प्र - अ॒नः । \newline
59. वै पृ॑षदा॒ज्यम् पृ॑षदा॒ज्यं ॅवै वै पृ॑षदा॒ज्यम् । \newline
60. पृ॒ष॒दा॒ज्यम् प्रा॒णः प्रा॒णः पृ॑षदा॒ज्यम् पृ॑षदा॒ज्यम् प्रा॒णः । \newline
61. पृ॒ष॒दा॒ज्यमिति॑ पृषत् - आ॒ज्यम् । \newline
62. प्रा॒णो वै वै प्रा॒णः प्रा॒णो वै । \newline
63. प्रा॒ण इति॑ प्र - अ॒नः । \newline
64. वा ए॒त स्यै॒तस्य॒ वै वा ए॒तस्य॑ । \newline

\textbf{Ghana Paata } \newline

1. ग्रहꣳ॑ स॒ह स॒ह ग्रह॒म् ग्रहꣳ॑ स॒होर्जोर्जा स॒ह ग्रह॒म् ग्रहꣳ॑ स॒होर्जा । \newline
2. स॒होर्जोर्जा स॒ह स॒होर्जा गृ॑ह्णामि गृह्णा म्यू॒र्जा स॒ह स॒होर्जा गृ॑ह्णामि । \newline
3. ऊ॒र्जा गृ॑ह्णामि गृह्णा म्यू॒र्जोर्जा गृ॑ह्णा॒ म्यस्तृ॑त॒ मस्तृ॑तम् गृह्णा म्यू॒र्जोर्जा गृ॑ह्णा॒ म्यस्तृ॑तम् । \newline
4. गृ॒ह्णा॒ म्यस्तृ॑त॒ मस्तृ॑तम् गृह्णामि गृह्णा॒ म्यस्तृ॑तम् । \newline
5. अस्तृ॑त॒मित्यस्तृ॑तम् । \newline
6. यत् कृ॑ष्णशकु॒नः कृ॑ष्णशकु॒नो यद् यत् कृ॑ष्णशकु॒नः पृ॑षदा॒ज्यम् पृ॑षदा॒ज्यम् कृ॑ष्णशकु॒नो यद् यत् कृ॑ष्णशकु॒नः पृ॑षदा॒ज्यम् । \newline
7. कृ॒ष्ण॒श॒कु॒नः पृ॑षदा॒ज्यम् पृ॑षदा॒ज्यम् कृ॑ष्णशकु॒नः कृ॑ष्णशकु॒नः पृ॑षदा॒ज्य म॑वमृ॒शे द॑वमृ॒शेत् पृ॑षदा॒ज्यम् कृ॑ष्णशकु॒नः कृ॑ष्णशकु॒नः पृ॑षदा॒ज्य म॑वमृ॒शेत् । \newline
8. कृ॒ष्ण॒श॒कु॒न इति॑ कृष्ण - श॒कु॒नः । \newline
9. पृ॒ष॒दा॒ज्य म॑वमृ॒शे द॑वमृ॒शेत् पृ॑षदा॒ज्यम् पृ॑षदा॒ज्य म॑वमृ॒शे च्छू॒द्राः शू॒द्रा अ॑वमृ॒शेत् पृ॑षदा॒ज्यम् पृ॑षदा॒ज्य म॑वमृ॒शे च्छू॒द्राः । \newline
10. पृ॒ष॒दा॒ज्यमिति॑ पृषत् - आ॒ज्यम् । \newline
11. अ॒व॒मृ॒शे च्छू॒द्राः शू॒द्रा अ॑वमृ॒शे द॑वमृ॒शे च्छू॒द्रा अ॑स्यास्य शू॒द्रा अ॑वमृ॒शे द॑वमृ॒शे च्छू॒द्रा अ॑स्य । \newline
12. अ॒व॒मृ॒शेदित्य॑व - मृ॒शेत् । \newline
13. शू॒द्रा अ॑स्यास्य शू॒द्राः शू॒द्रा अ॑स्य प्र॒मायु॑काः प्र॒मायु॑का अस्य शू॒द्राः शू॒द्रा अ॑स्य प्र॒मायु॑काः । \newline
14. अ॒स्य॒ प्र॒मायु॑काः प्र॒मायु॑का अस्यास्य प्र॒मायु॑काः स्युः स्युः प्र॒मायु॑का अस्यास्य प्र॒मायु॑काः स्युः । \newline
15. प्र॒मायु॑काः स्युः स्युः प्र॒मायु॑काः प्र॒मायु॑काः स्यु॒र् यद् यथ् स्युः॑ प्र॒मायु॑काः प्र॒मायु॑काः स्यु॒र् यत् । \newline
16. प्र॒मायु॑का॒ इति॑ प्र - मायु॑काः । \newline
17. स्यु॒र् यद् यथ् स्युः॑ स्यु॒र् यच्छ्‌वा श्वा यथ् स्युः॑ स्यु॒र् यच्छ्‌वा । \newline
18. यच्छ्‌वा श्वा यद् यच्छ्‌वा ऽव॑मृ॒शे द॑वमृ॒शे च्छ्‌वा यद् यच्छ्‌वा ऽव॑मृ॒शेत् । \newline
19. श्वा ऽव॑मृ॒शे द॑वमृ॒शे च्छ्‌वा श्वा ऽव॑मृ॒शेच् चतु॑ष्पाद॒ श्चतु॑ष्पादो ऽवमृ॒शे च्छ्‌वा श्वा ऽव॑मृ॒शेच् चतु॑ष्पादः । \newline
20. अ॒व॒मृ॒शेच् चतु॑ष्पाद॒ श्चतु॑ष्पादो ऽवमृ॒शे द॑वमृ॒शेच् चतु॑ष्पादो ऽस्यास्य॒ चतु॑ष्पादो ऽवमृ॒शे द॑वमृ॒शेच् चतु॑ष्पादो ऽस्य । \newline
21. अ॒व॒मृ॒शेदित्य॑व - मृ॒शेत् । \newline
22. चतु॑ष्पादो ऽस्यास्य॒ चतु॑ष्पाद॒ श्चतु॑ष्पादो ऽस्य प॒शवः॑ प॒शवो᳚ ऽस्य॒ चतु॑ष्पाद॒ श्चतु॑ष्पादो ऽस्य प॒शवः॑ । \newline
23. चतु॑ष्पाद॒ इति॒ चतुः॑ - पा॒दः॒ । \newline
24. अ॒स्य॒ प॒शवः॑ प॒शवो᳚ ऽस्यास्य प॒शवः॑ प्र॒मायु॑काः प्र॒मायु॑काः प॒शवो᳚ ऽस्यास्य प॒शवः॑ प्र॒मायु॑काः । \newline
25. प॒शवः॑ प्र॒मायु॑काः प्र॒मायु॑काः प॒शवः॑ प॒शवः॑ प्र॒मायु॑काः स्युः स्युः प्र॒मायु॑काः प॒शवः॑ प॒शवः॑ प्र॒मायु॑काः स्युः । \newline
26. प्र॒मायु॑काः स्युः स्युः प्र॒मायु॑काः प्र॒मायु॑काः स्यु॒र् यद् यथ् स्युः॑ प्र॒मायु॑काः प्र॒मायु॑काः स्यु॒र् यत् । \newline
27. प्र॒मायु॑का॒ इति॑ प्र - मायु॑काः । \newline
28. स्यु॒र् यद् यथ् स्युः॑ स्यु॒र् यथ् स्कन्दे॒थ् स्कन्दे॒द् यथ् स्युः॑ स्यु॒र् यथ् स्कन्दे᳚त् । \newline
29. यथ् स्कन्दे॒थ् स्कन्दे॒द् यद् यथ् स्कन्दे॒द् यज॑मानो॒ यज॑मानः॒ स्कन्दे॒द् यद् यथ् स्कन्दे॒द् यज॑मानः । \newline
30. स्कन्दे॒द् यज॑मानो॒ यज॑मानः॒ स्कन्दे॒थ् स्कन्दे॒द् यज॑मानः प्र॒मायु॑कः प्र॒मायु॑को॒ यज॑मानः॒ स्कन्दे॒थ् स्कन्दे॒द् यज॑मानः प्र॒मायु॑कः । \newline
31. यज॑मानः प्र॒मायु॑कः प्र॒मायु॑को॒ यज॑मानो॒ यज॑मानः प्र॒मायु॑कः स्याथ् स्यात् प्र॒मायु॑को॒ यज॑मानो॒ यज॑मानः प्र॒मायु॑कः स्यात् । \newline
32. प्र॒मायु॑कः स्याथ् स्यात् प्र॒मायु॑कः प्र॒मायु॑कः स्यात् प॒शवः॑ प॒शवः॑ स्यात् प्र॒मायु॑कः प्र॒मायु॑कः स्यात् प॒शवः॑ । \newline
33. प्र॒मायु॑क॒ इति॑ प्र - मायु॑कः । \newline
34. स्या॒त् प॒शवः॑ प॒शवः॑ स्याथ् स्यात् प॒शवो॒ वै वै प॒शवः॑ स्याथ् स्यात् प॒शवो॒ वै । \newline
35. प॒शवो॒ वै वै प॒शवः॑ प॒शवो॒ वै पृ॑षदा॒ज्यम् पृ॑षदा॒ज्यं ॅवै प॒शवः॑ प॒शवो॒ वै पृ॑षदा॒ज्यम् । \newline
36. वै पृ॑षदा॒ज्यम् पृ॑षदा॒ज्यं ॅवै वै पृ॑षदा॒ज्यम् प॒शवः॑ प॒शवः॑ पृषदा॒ज्यं ॅवै वै पृ॑षदा॒ज्यम् प॒शवः॑ । \newline
37. पृ॒ष॒दा॒ज्यम् प॒शवः॑ प॒शवः॑ पृषदा॒ज्यम् पृ॑षदा॒ज्यम् प॒शवो॒ वै वै प॒शवः॑ पृषदा॒ज्यम् पृ॑षदा॒ज्यम् प॒शवो॒ वै । \newline
38. पृ॒ष॒दा॒ज्यमिति॑ पृषत् - आ॒ज्यम् । \newline
39. प॒शवो॒ वै वै प॒शवः॑ प॒शवो॒ वा ए॒त स्यै॒तस्य॒ वै प॒शवः॑ प॒शवो॒ वा ए॒तस्य॑ । \newline
40. वा ए॒त स्यै॒तस्य॒ वै वा ए॒तस्य॑ स्कन्दन्ति स्कन्द न्त्ये॒तस्य॒ वै वा ए॒तस्य॑ स्कन्दन्ति । \newline
41. ए॒तस्य॑ स्कन्दन्ति स्कन्द न्त्ये॒त स्यै॒तस्य॑ स्कन्दन्ति॒ यस्य॒ यस्य॑ स्कन्द न्त्ये॒त स्यै॒तस्य॑ स्कन्दन्ति॒ यस्य॑ । \newline
42. स्क॒न्द॒न्ति॒ यस्य॒ यस्य॑ स्कन्दन्ति स्कन्दन्ति॒ यस्य॑ पृषदा॒ज्यम् पृ॑षदा॒ज्यं ॅयस्य॑ स्कन्दन्ति स्कन्दन्ति॒ यस्य॑ पृषदा॒ज्यम् । \newline
43. यस्य॑ पृषदा॒ज्यम् पृ॑षदा॒ज्यं ॅयस्य॒ यस्य॑ पृषदा॒ज्यꣳ स्कन्द॑ति॒ स्कन्द॑ति पृषदा॒ज्यं ॅयस्य॒ यस्य॑ पृषदा॒ज्यꣳ स्कन्द॑ति । \newline
44. पृ॒ष॒दा॒ज्यꣳ स्कन्द॑ति॒ स्कन्द॑ति पृषदा॒ज्यम् पृ॑षदा॒ज्यꣳ स्कन्द॑ति॒ यद् यथ् स्कन्द॑ति पृषदा॒ज्यम् पृ॑षदा॒ज्यꣳ स्कन्द॑ति॒ यत् । \newline
45. पृ॒ष॒दा॒ज्यमिति॑ पृषत् - आ॒ज्यम् । \newline
46. स्कन्द॑ति॒ यद् यथ् स्कन्द॑ति॒ स्कन्द॑ति॒ यत् पृ॑षदा॒ज्यम् पृ॑षदा॒ज्यं ॅयथ् स्कन्द॑ति॒ स्कन्द॑ति॒ यत् पृ॑षदा॒ज्यम् । \newline
47. यत् पृ॑षदा॒ज्यम् पृ॑षदा॒ज्यं ॅयद् यत् पृ॑षदा॒ज्यम् पुनः॒ पुनः॑ पृषदा॒ज्यं ॅयद् यत् पृ॑षदा॒ज्यम् पुनः॑ । \newline
48. पृ॒ष॒दा॒ज्यम् पुनः॒ पुनः॑ पृषदा॒ज्यम् पृ॑षदा॒ज्यम् पुन॑र् गृ॒ह्णाति॑ गृ॒ह्णाति॒ पुनः॑ पृषदा॒ज्यम् पृ॑षदा॒ज्यम् पुन॑र् गृ॒ह्णाति॑ । \newline
49. पृ॒ष॒दा॒ज्यमिति॑ पृषत् - आ॒ज्यम् । \newline
50. पुन॑र् गृ॒ह्णाति॑ गृ॒ह्णाति॒ पुनः॒ पुन॑र् गृ॒ह्णाति॑ प॒शून् प॒शून् गृ॒ह्णाति॒ पुनः॒ पुन॑र् गृ॒ह्णाति॑ प॒शून् । \newline
51. गृ॒ह्णाति॑ प॒शून् प॒शून् गृ॒ह्णाति॑ गृ॒ह्णाति॑ प॒शू ने॒वैव प॒शून् गृ॒ह्णाति॑ गृ॒ह्णाति॑ प॒शू ने॒व । \newline
52. प॒शू ने॒वैव प॒शून् प॒शू ने॒वास्मा॑ अस्मा ए॒व प॒शून् प॒शू ने॒वास्मै᳚ । \newline
53. ए॒वास्मा॑ अस्मा ए॒वैवास्मै॒ पुनः॒ पुन॑ रस्मा ए॒वैवास्मै॒ पुनः॑ । \newline
54. अ॒स्मै॒ पुनः॒ पुन॑ रस्मा अस्मै॒ पुन॑र् गृह्णाति गृह्णाति॒ पुन॑ रस्मा अस्मै॒ पुन॑र् गृह्णाति । \newline
55. पुन॑र् गृह्णाति गृह्णाति॒ पुनः॒ पुन॑र् गृह्णाति प्रा॒णः प्रा॒णो गृ॑ह्णाति॒ पुनः॒ पुन॑र् गृह्णाति प्रा॒णः । \newline
56. गृ॒ह्णा॒ति॒ प्रा॒णः प्रा॒णो गृ॑ह्णाति गृह्णाति प्रा॒णो वै वै प्रा॒णो गृ॑ह्णाति गृह्णाति प्रा॒णो वै । \newline
57. प्रा॒णो वै वै प्रा॒णः प्रा॒णो वै पृ॑षदा॒ज्यम् पृ॑षदा॒ज्यं ॅवै प्रा॒णः प्रा॒णो वै पृ॑षदा॒ज्यम् । \newline
58. प्रा॒ण इति॑ प्र - अ॒नः । \newline
59. वै पृ॑षदा॒ज्यम् पृ॑षदा॒ज्यं ॅवै वै पृ॑षदा॒ज्यम् प्रा॒णः प्रा॒णः पृ॑षदा॒ज्यं ॅवै वै पृ॑षदा॒ज्यम् प्रा॒णः । \newline
60. पृ॒ष॒दा॒ज्यम् प्रा॒णः प्रा॒णः पृ॑षदा॒ज्यम् पृ॑षदा॒ज्यम् प्रा॒णो वै वै प्रा॒णः पृ॑षदा॒ज्यम् पृ॑षदा॒ज्यम् प्रा॒णो वै । \newline
61. पृ॒ष॒दा॒ज्यमिति॑ पृषत् - आ॒ज्यम् । \newline
62. प्रा॒णो वै वै प्रा॒णः प्रा॒णो वा ए॒त स्यै॒तस्य॒ वै प्रा॒णः प्रा॒णो वा ए॒तस्य॑ । \newline
63. प्रा॒ण इति॑ प्र - अ॒नः । \newline
64. वा ए॒त स्यै॒तस्य॒ वै वा ए॒तस्य॑ स्कन्दति स्कन्द त्ये॒तस्य॒ वै वा ए॒तस्य॑ स्कन्दति । \newline
\pagebreak
\markright{ TS 3.2.6.3  \hfill https://www.vedavms.in \hfill}

\section{ TS 3.2.6.3 }

\textbf{TS 3.2.6.3 } \newline
\textbf{Samhita Paata} \newline

ए॒तस्य॑ स्कन्दति॒ यस्य॑ पृषदा॒ज्यꣳ स्कन्द॑ति॒ यत् पृ॑षदा॒ज्यं पुन॑र्गृ॒ह्णाति॑ प्रा॒णमे॒वास्मै॒ पुन॑र्गृह्णाति॒ हिर॑ण्यमव॒धाय॑ गृह्णात्य॒मृतं॒ ॅवै हिर॑ण्यं प्रा॒णः पृ॑षदा॒ज्यम॒मृत॑मे॒वास्य॑ प्रा॒णे द॑धाति श॒तमा॑नं भवति श॒तायुः॒ पुरु॑षः श॒तेन्द्रि॑य॒ आयु॑ष्ये॒वेन्द्रि॒ये प्रति॑तिष्ठ॒त्यश्व॒मव॑ घ्रापयति प्राजाप॒त्यो वा अश्वः॑ प्राजाप॒त्यः प्रा॒णः स्वादे॒वास्मै॒ योनेः᳚ प्रा॒णं ( ) निर्मि॑मीते॒ वि वा ए॒तस्य॑ य॒ज्ञ्श्छि॑द्यते॒ यस्य॑ पृषदा॒ज्यꣳ स्कन्द॑ति वैष्ण॒व्यर्चा पुन॑र्गृह्णाति य॒ज्ञो वै विष्णु॑र्य॒ज्ञेनै॒व य॒ज्ञ्ꣳ सं त॑नोति ॥ \newline

\textbf{Pada Paata} \newline

ए॒तस्य॑ । स्क॒न्द॒ति॒ । यस्य॑ । पृ॒ष॒दा॒ज्यमिति॑ पृषत् - आ॒ज्यम् । स्कन्द॑ति । यत् । पृ॒ष॒दा॒ज्यमिति॑ पृषत् - आ॒ज्यम् । पुनः॑ । गृ॒ह्णाति॑ । प्रा॒णमिति॑ प्र - अ॒नम् । ए॒व । अ॒स्मै॒ । पुनः॑ । गृ॒ह्णा॒ति॒ । हिर॑ण्यम् । अ॒व॒धायेत्य॑व - धाय॑ । गृ॒ह्णा॒ति॒ । अ॒मृत᳚म् । वै । हिर॑ण्यम् । प्रा॒ण इति॑ प्र - अ॒नः । पृ॒ष॒दा॒ज्यमिति॑ पृषत् - आ॒ज्यम् । अ॒मृत᳚म् । ए॒व । अ॒स्य॒ । प्रा॒ण इति॑ प्र - अ॒ने । द॒धा॒ति॒ । श॒तमा॑न॒मिति॑ श॒त-मा॒न॒म् । भ॒व॒ति॒ । श॒तायु॒रिति॑ श॒त - आ॒युः॒ । पुरु॑षः । श॒तेन्द्रि॑य॒ इति॑ श॒त-इ॒न्द्रि॒यः॒ । आयु॑षि । ए॒व । इ॒न्द्रि॒ये । प्रतीति॑ । ति॒ष्ठ॒ति॒ । अश्व᳚म् । अवेति॑ । घ्रा॒प॒य॒ति॒ । प्रा॒जा॒प॒त्य इति॑ प्राजा - प॒त्यः । वै । अश्वः॑ । प्रा॒जा॒प॒त्य इति॑ प्राजा - प॒त्यः । प्रा॒ण इति॑ प्र-अ॒नः । स्वात् । ए॒व । अ॒स्मै॒ । योनेः᳚ । प्रा॒णमिति॑ प्र - अ॒नम् ( ) । निरिति॑ । मि॒मी॒ते॒ । वीति॑ । वै । ए॒तस्य॑ । य॒ज्ञ्ः । छि॒द्य॒ते॒ । यस्य॑ । पृ॒ष॒दा॒ज्यमिति॑ पृषत् - आ॒ज्यम् । स्कन्द॑ति । वै॒ष्ण॒व्या । ऋ॒चा । पुनः॑ । गृ॒ह्णा॒ति॒ । य॒ज्ञ्ः । वै । विष्णुः॑ । य॒ज्ञेन॑ । ए॒व । य॒ज्ञ्म् । समिति॑ । त॒नो॒ति॒ ॥  \newline


\textbf{Krama Paata} \newline

ए॒तस्य॑ स्कन्दति । स्क॒न्द॒ति॒ यस्य॑ । यस्य॑ पृषदा॒ज्यम् । पृ॒ष॒दा॒ज्यꣳ स्कन्द॑ति । पृ॒ष॒दा॒ज्यमिति॑ पृषत् - आ॒ज्यम् । स्कन्द॑ति॒ यत् । यत् पृ॑षदा॒ज्यम् । पृ॒ष॒दा॒ज्यम् पुनः॑ । पृ॒ष॒दा॒ज्यमिति॑ पृषत् - आ॒ज्यम् । पुन॑र् गृ॒ह्णाति॑ । गृ॒ह्णाति॑ प्रा॒णम् । प्रा॒णमे॒व । प्रा॒णमिति॑ प्र - अ॒नम् । ए॒वास्मै᳚ । अ॒स्मै॒ पुनः॑ । पुन॑र् गृह्णाति । गृ॒ह्णा॒ति॒ हिर॑ण्यम् । हिर॑ण्यमव॒धाय॑ । अ॒व॒धाय॑ गृह्णाति । अ॒व॒धायेत्य॑व - धाय॑ । गृ॒ह्णा॒त्य॒मृत᳚म् । अ॒मृतं॒ ॅवै । वै हिर॑ण्यम् । हिर॑ण्यम् प्रा॒णः । प्रा॒णः पृ॑षदा॒ज्यम् । प्रा॒ण इति॑ प्र - अ॒नः । पृ॒ष॒दा॒ज्यम॒मृत᳚म् । पृ॒ष॒दा॒ज्यमिति॑ पृषत् - आ॒ज्यम् । अ॒मृत॑मे॒व । ए॒वास्य॑ । अ॒स्य॒ प्रा॒णे । प्रा॒णे द॑धाति । प्रा॒ण इति॑ प्र - अ॒ने । द॒धा॒ति॒ श॒तमा॑नम् । श॒तमा॑नम् भवति । श॒तमा॑न॒मिति॑ श॒त - मा॒न॒म् । भ॒व॒ति॒ श॒तायुः॑ । श॒तायुः॒ पुरु॑षः । श॒तायु॒रिति॑ श॒त - आ॒युः॒ । पुरु॑षः श॒तेन्द्रि॑यः । श॒तेन्द्रि॑य॒ आयु॑षि । श॒तेन्द्रि॑य॒ इति॑ श॒त - इ॒न्द्रि॒यः॒ । आयु॑ष्ये॒व । ए॒वेन्द्रि॒ये । इ॒न्द्रि॒ये प्रति॑ । प्रति॑ तिष्ठति । ति॒ष्ठ॒त्यश्व᳚म् । अश्व॒मव॑ । अव॑ घ्रापयति । घ्रा॒प॒य॒ति॒ प्रा॒जा॒प॒त्यः । प्रा॒जा॒प॒त्यो वै । प्रा॒जा॒प॒त्य इति॑ प्राजा - प॒त्यः । वा अश्वः॑ । अश्वः॑ प्राजाप॒त्यः । प्रा॒जा॒प॒त्यः प्रा॒णः । प्रा॒जा॒प॒त्य इति॑ प्राजा - प॒त्यः । प्रा॒णः स्वात् । प्रा॒ण इति॑ प्र - अ॒नः । स्वादे॒व । ए॒वास्मै᳚ । अ॒स्मै॒ योनेः᳚ । योनेः᳚ प्रा॒णम् ( ) । प्रा॒णम् निः । प्रा॒णमिति॑ प्र - अ॒नम् । निर् मि॑मीते । मि॒मी॒ते॒ वि । वि वै । वा ए॒तस्य॑ । ए॒तस्य॑ य॒ज्ञ्ः । य॒ज्ञ् श्छि॑द्यते । छि॒द्य॒ते॒ यस्य॑ । यस्य॑ पृषदा॒ज्यम् । पृ॒ष॒दा॒ज्यꣳ स्कन्द॑ति । पृ॒ष॒दा॒ज्यमिति॑ पृषत् - आ॒ज्यम् । स्कन्द॑ति वैष्ण॒व्या । वै॒ष्ण॒व्यर्चा । ऋ॒चा पुनः॑ । पुन॑र् गृह्णाति । गृ॒ह्णा॒ति॒ य॒ज्ञ्ः । य॒ज्ञो वै । वै विष्णुः॑ । विष्णु॑र् य॒ज्ञेन॑ । य॒ज्ञेनै॒व । ए॒व य॒ज्ञ्म् । य॒ज्ञ्ꣳ सम् । सम् त॑नोति । त॒नो॒तीति॑ तनोति । \newline

\textbf{Jatai Paata} \newline

1. ए॒तस्य॑ स्कन्दति स्कन्द त्ये॒त स्यै॒तस्य॑ स्कन्दति । \newline
2. स्क॒न्द॒ति॒ यस्य॒ यस्य॑ स्कन्दति स्कन्दति॒ यस्य॑ । \newline
3. यस्य॑ पृषदा॒ज्यम् पृ॑षदा॒ज्यं ॅयस्य॒ यस्य॑ पृषदा॒ज्यम् । \newline
4. पृ॒ष॒दा॒ज्यꣳ स्कन्द॑ति॒ स्कन्द॑ति पृषदा॒ज्यम् पृ॑षदा॒ज्यꣳ स्कन्द॑ति । \newline
5. पृ॒ष॒दा॒ज्यमिति॑ पृषत् - आ॒ज्यम् । \newline
6. स्कन्द॑ति॒ यद् यथ् स्कन्द॑ति॒ स्कन्द॑ति॒ यत् । \newline
7. यत् पृ॑षदा॒ज्यम् पृ॑षदा॒ज्यं ॅयद् यत् पृ॑षदा॒ज्यम् । \newline
8. पृ॒ष॒दा॒ज्यम् पुनः॒ पुनः॑ पृषदा॒ज्यम् पृ॑षदा॒ज्यम् पुनः॑ । \newline
9. पृ॒ष॒दा॒ज्यमिति॑ पृषत् - आ॒ज्यम् । \newline
10. पुन॑र् गृ॒ह्णाति॑ गृ॒ह्णाति॒ पुनः॒ पुन॑र् गृ॒ह्णाति॑ । \newline
11. गृ॒ह्णाति॑ प्रा॒णम् प्रा॒णम् गृ॒ह्णाति॑ गृ॒ह्णाति॑ प्रा॒णम् । \newline
12. प्रा॒ण मे॒वैव प्रा॒णम् प्रा॒ण मे॒व । \newline
13. प्रा॒णमिति॑ प्र - अ॒नम् । \newline
14. ए॒वास्मा॑ अस्मा ए॒वैवास्मै᳚ । \newline
15. अ॒स्मै॒ पुनः॒ पुन॑ रस्मा अस्मै॒ पुनः॑ । \newline
16. पुन॑र् गृह्णाति गृह्णाति॒ पुनः॒ पुन॑र् गृह्णाति । \newline
17. गृ॒ह्णा॒ति॒ हिर॑ण्यꣳ॒॒ हिर॑ण्यम् गृह्णाति गृह्णाति॒ हिर॑ण्यम् । \newline
18. हिर॑ण्य मव॒धाया॑ व॒धाय॒ हिर॑ण्यꣳ॒॒ हिर॑ण्य मव॒धाय॑ । \newline
19. अ॒व॒धाय॑ गृह्णाति गृह्णा त्यव॒धाया॑ व॒धाय॑ गृह्णाति । \newline
20. अ॒व॒धायेत्य॑व - धाय॑ । \newline
21. गृ॒ह्णा॒ त्य॒मृत॑ म॒मृत॑म् गृह्णाति गृह्णा त्य॒मृत᳚म् । \newline
22. अ॒मृतं॒ ॅवै वा अ॒मृत॑ म॒मृतं॒ ॅवै । \newline
23. वै हिर॑ण्यꣳ॒॒ हिर॑ण्यं॒ ॅवै वै हिर॑ण्यम् । \newline
24. हिर॑ण्यम् प्रा॒णः प्रा॒णो हिर॑ण्यꣳ॒॒ हिर॑ण्यम् प्रा॒णः । \newline
25. प्रा॒णः पृ॑षदा॒ज्यम् पृ॑षदा॒ज्यम् प्रा॒णः प्रा॒णः पृ॑षदा॒ज्यम् । \newline
26. प्रा॒ण इति॑ प्र - अ॒नः । \newline
27. पृ॒ष॒दा॒ज्य म॒मृत॑ म॒मृत॑म् पृषदा॒ज्यम् पृ॑षदा॒ज्य म॒मृत᳚म् । \newline
28. पृ॒ष॒दा॒ज्यमिति॑ पृषत् - आ॒ज्यम् । \newline
29. अ॒मृत॑ मे॒वैवामृत॑ म॒मृत॑ मे॒व । \newline
30. ए॒वास्या᳚ स्यै॒वैवास्य॑ । \newline
31. अ॒स्य॒ प्रा॒णे प्रा॒णे᳚ ऽस्यास्य प्रा॒णे । \newline
32. प्रा॒णे द॑धाति दधाति प्रा॒णे प्रा॒णे द॑धाति । \newline
33. प्रा॒ण इति॑ प्र - अ॒ने । \newline
34. द॒धा॒ति॒ श॒तमा॑नꣳ श॒तमा॑नम् दधाति दधाति श॒तमा॑नम् । \newline
35. श॒तमा॑नम् भवति भवति श॒तमा॑नꣳ श॒तमा॑नम् भवति । \newline
36. श॒तमा॑न॒मिति॑ श॒त - मा॒न॒म् । \newline
37. भ॒व॒ति॒ श॒तायुः॑ श॒तायु॑र् भवति भवति श॒तायुः॑ । \newline
38. श॒तायुः॒ पुरु॑षः॒ पुरु॑षः श॒तायुः॑ श॒तायुः॒ पुरु॑षः । \newline
39. श॒तायु॒रिति॑ श॒त - आ॒युः॒ । \newline
40. पुरु॑षः श॒तेन्द्रि॑यः श॒तेन्द्रि॑यः॒ पुरु॑षः॒ पुरु॑षः श॒तेन्द्रि॑यः । \newline
41. श॒तेन्द्रि॑य॒ आयु॒ ष्यायु॑षि श॒तेन्द्रि॑यः श॒तेन्द्रि॑य॒ आयु॑षि । \newline
42. श॒तेन्द्रि॑य॒ इति॑ श॒त - इ॒न्द्रि॒यः॒ । \newline
43. आयु॑ ष्ये॒वै वायु॒ ष्यायु॑ ष्ये॒व । \newline
44. ए॒वे न्द्रि॒य इ॑न्द्रि॒य ए॒वैवे न्द्रि॒ये । \newline
45. इ॒न्द्रि॒ये प्रति॒ प्रती᳚न्द्रि॒य इ॑न्द्रि॒ये प्रति॑ । \newline
46. प्रति॑ तिष्ठति तिष्ठति॒ प्रति॒ प्रति॑ तिष्ठति । \newline
47. ति॒ष्ठ॒ त्यश्व॒ मश्व॑म् तिष्ठति तिष्ठ॒ त्यश्व᳚म् । \newline
48. अश्व॒ मवा वाश्व॒ मश्व॒ मव॑ । \newline
49. अव॑ घ्रापयति घ्रापय॒ त्यवाव॑ घ्रापयति । \newline
50. घ्रा॒प॒य॒ति॒ प्रा॒जा॒प॒त्यः प्रा॑जाप॒त्यो घ्रा॑पयति घ्रापयति प्राजाप॒त्यः । \newline
51. प्रा॒जा॒प॒त्यो वै वै प्रा॑जाप॒त्यः प्रा॑जाप॒त्यो वै । \newline
52. प्रा॒जा॒प॒त्य इति॑ प्राजा - प॒त्यः । \newline
53. वा अश्वो ऽश्वो॒ वै वा अश्वः॑ । \newline
54. अश्वः॑ प्राजाप॒त्यः प्रा॑जाप॒त्यो ऽश्वो ऽश्वः॑ प्राजाप॒त्यः । \newline
55. प्रा॒जा॒प॒त्यः प्रा॒णः प्रा॒णः प्रा॑जाप॒त्यः प्रा॑जाप॒त्यः प्रा॒णः । \newline
56. प्रा॒जा॒प॒त्य इति॑ प्राजा - प॒त्यः । \newline
57. प्रा॒णः स्वाथ् स्वात् प्रा॒णः प्रा॒णः स्वात् । \newline
58. प्रा॒ण इति॑ प्र - अ॒नः । \newline
59. स्वा दे॒वैव स्वाथ् स्वा दे॒व । \newline
60. ए॒वास्मा॑ अस्मा ए॒वैवास्मै᳚ । \newline
61. अ॒स्मै॒ योने॒र् योने॑ रस्मा अस्मै॒ योनेः᳚ । \newline
62. योनेः᳚ प्रा॒णम् प्रा॒णं ॅयोने॒र् योनेः᳚ प्रा॒णम् । \newline
63. प्रा॒णम् निर् णिष् प्रा॒णम् प्रा॒णम् निः । \newline
64. प्रा॒णमिति॑ प्र - अ॒नम् । \newline
65. निर् मि॑मीते मिमीते॒ निर् णिर् मि॑मीते । \newline
66. मि॒मी॒ते॒ वि वि मि॑मीते मिमीते॒ वि । \newline
67. वि वै वै वि वि वै । \newline
68. वा ए॒त स्यै॒तस्य॒ वै वा ए॒तस्य॑ । \newline
69. ए॒तस्य॑ य॒ज्ञो य॒ज्ञ् ए॒त स्यै॒तस्य॑ य॒ज्ञ्ः । \newline
70. य॒ज्ञ् श्छि॑द्यते छिद्यते य॒ज्ञो य॒ज्ञ् श्छि॑द्यते । \newline
71. छि॒द्य॒ते॒ यस्य॒ यस्य॑ छिद्यते छिद्यते॒ यस्य॑ । \newline
72. यस्य॑ पृषदा॒ज्यम् पृ॑षदा॒ज्यं ॅयस्य॒ यस्य॑ पृषदा॒ज्यम् । \newline
73. पृ॒ष॒दा॒ज्यꣳ स्कन्द॑ति॒ स्कन्द॑ति पृषदा॒ज्यम् पृ॑षदा॒ज्यꣳ स्कन्द॑ति । \newline
74. पृ॒ष॒दा॒ज्यमिति॑ पृषत् - आ॒ज्यम् । \newline
75. स्कन्द॑ति वैष्ण॒व्या वै᳚ष्ण॒व्या स्कन्द॑ति॒ स्कन्द॑ति वैष्ण॒व्या । \newline
76. वै॒ष्ण॒ व्यर्चर्चा वै᳚ष्ण॒व्या वै᳚ष्ण॒ व्यर्चा । \newline
77. ऋ॒चा पुनः॒ पुनर्॑. ऋ॒चर्चा पुनः॑ । \newline
78. पुन॑र् गृह्णाति गृह्णाति॒ पुनः॒ पुन॑र् गृह्णाति । \newline
79. गृ॒ह्णा॒ति॒ य॒ज्ञो य॒ज्ञो गृ॑ह्णाति गृह्णाति य॒ज्ञ्ः । \newline
80. य॒ज्ञो वै वै य॒ज्ञो य॒ज्ञो वै । \newline
81. वै विष्णु॒र् विष्णु॒र् वै वै विष्णुः॑ । \newline
82. विष्णु॑र् य॒ज्ञेन॑ य॒ज्ञेन॒ विष्णु॒र् विष्णु॑र् य॒ज्ञेन॑ । \newline
83. य॒ज्ञे नै॒वैव य॒ज्ञेन॑ य॒ज्ञे नै॒व । \newline
84. ए॒व य॒ज्ञ्ं ॅय॒ज्ञ् मे॒वैव य॒ज्ञ्म् । \newline
85. य॒ज्ञ्ꣳ सꣳ सं ॅय॒ज्ञ्ं ॅय॒ज्ञ्ꣳ सम् । \newline
86. सम् त॑नोति तनोति॒ सꣳ सम् त॑नोति । \newline
87. त॒नो॒तीति॑ तनोति । \newline

\textbf{Ghana Paata } \newline

1. ए॒तस्य॑ स्कन्दति स्कन्द त्ये॒त स्यै॒तस्य॑ स्कन्दति॒ यस्य॒ यस्य॑ स्कन्द त्ये॒त स्यै॒तस्य॑ स्कन्दति॒ यस्य॑ । \newline
2. स्क॒न्द॒ति॒ यस्य॒ यस्य॑ स्कन्दति स्कन्दति॒ यस्य॑ पृषदा॒ज्यम् पृ॑षदा॒ज्यं ॅयस्य॑ स्कन्दति स्कन्दति॒ यस्य॑ पृषदा॒ज्यम् । \newline
3. यस्य॑ पृषदा॒ज्यम् पृ॑षदा॒ज्यं ॅयस्य॒ यस्य॑ पृषदा॒ज्यꣳ स्कन्द॑ति॒ स्कन्द॑ति पृषदा॒ज्यं ॅयस्य॒ यस्य॑ पृषदा॒ज्यꣳ स्कन्द॑ति । \newline
4. पृ॒ष॒दा॒ज्यꣳ स्कन्द॑ति॒ स्कन्द॑ति पृषदा॒ज्यम् पृ॑षदा॒ज्यꣳ स्कन्द॑ति॒ यद् यथ् स्कन्द॑ति पृषदा॒ज्यम् पृ॑षदा॒ज्यꣳ स्कन्द॑ति॒ यत् । \newline
5. पृ॒ष॒दा॒ज्यमिति॑ पृषत् - आ॒ज्यम् । \newline
6. स्कन्द॑ति॒ यद् यथ् स्कन्द॑ति॒ स्कन्द॑ति॒ यत् पृ॑षदा॒ज्यम् पृ॑षदा॒ज्यं ॅयथ् स्कन्द॑ति॒ स्कन्द॑ति॒ यत् पृ॑षदा॒ज्यम् । \newline
7. यत् पृ॑षदा॒ज्यम् पृ॑षदा॒ज्यं ॅयद् यत् पृ॑षदा॒ज्यम् पुनः॒ पुनः॑ पृषदा॒ज्यं ॅयद् यत् पृ॑षदा॒ज्यम् पुनः॑ । \newline
8. पृ॒ष॒दा॒ज्यम् पुनः॒ पुनः॑ पृषदा॒ज्यम् पृ॑षदा॒ज्यम् पुन॑र् गृ॒ह्णाति॑ गृ॒ह्णाति॒ पुनः॑ पृषदा॒ज्यम् पृ॑षदा॒ज्यम् पुन॑र् गृ॒ह्णाति॑ । \newline
9. पृ॒ष॒दा॒ज्यमिति॑ पृषत् - आ॒ज्यम् । \newline
10. पुन॑र् गृ॒ह्णाति॑ गृ॒ह्णाति॒ पुनः॒ पुन॑र् गृ॒ह्णाति॑ प्रा॒णम् प्रा॒णम् गृ॒ह्णाति॒ पुनः॒ पुन॑र् गृ॒ह्णाति॑ प्रा॒णम् । \newline
11. गृ॒ह्णाति॑ प्रा॒णम् प्रा॒णम् गृ॒ह्णाति॑ गृ॒ह्णाति॑ प्रा॒ण मे॒वैव प्रा॒णम् गृ॒ह्णाति॑ गृ॒ह्णाति॑ प्रा॒ण मे॒व । \newline
12. प्रा॒ण मे॒वैव प्रा॒णम् प्रा॒ण मे॒वास्मा॑ अस्मा ए॒व प्रा॒णम् प्रा॒ण मे॒वास्मै᳚ । \newline
13. प्रा॒णमिति॑ प्र - अ॒नम् । \newline
14. ए॒वास्मा॑ अस्मा ए॒वैवास्मै॒ पुनः॒ पुन॑ रस्मा ए॒वैवास्मै॒ पुनः॑ । \newline
15. अ॒स्मै॒ पुनः॒ पुन॑ रस्मा अस्मै॒ पुन॑र् गृह्णाति गृह्णाति॒ पुन॑ रस्मा अस्मै॒ पुन॑र् गृह्णाति । \newline
16. पुन॑र् गृह्णाति गृह्णाति॒ पुनः॒ पुन॑र् गृह्णाति॒ हिर॑ण्यꣳ॒॒ हिर॑ण्यम् गृह्णाति॒ पुनः॒ पुन॑र् गृह्णाति॒ हिर॑ण्यम् । \newline
17. गृ॒ह्णा॒ति॒ हिर॑ण्यꣳ॒॒ हिर॑ण्यम् गृह्णाति गृह्णाति॒ हिर॑ण्य मव॒धाया॑ व॒धाय॒ हिर॑ण्यम् गृह्णाति गृह्णाति॒ हिर॑ण्य मव॒धाय॑ । \newline
18. हिर॑ण्य मव॒धाया॑ व॒धाय॒ हिर॑ण्यꣳ॒॒ हिर॑ण्य मव॒धाय॑ गृह्णाति गृह्णा त्यव॒धाय॒ हिर॑ण्यꣳ॒॒ हिर॑ण्य मव॒धाय॑ गृह्णाति । \newline
19. अ॒व॒धाय॑ गृह्णाति गृह्णा त्यव॒धाया॑ व॒धाय॑ गृह्णा त्य॒मृत॑ म॒मृत॑म् गृह्णा त्यव॒धाया॑ व॒धाय॑ गृह्णात्य॒मृत᳚म् । \newline
20. अ॒व॒धायेत्य॑व - धाय॑ । \newline
21. गृ॒ह्णा॒ त्य॒मृत॑ म॒मृत॑म् गृह्णाति गृह्णा त्य॒मृतं॒ ॅवै वा अ॒मृत॑म् गृह्णाति गृह्णा त्य॒मृतं॒ ॅवै । \newline
22. अ॒मृतं॒ ॅवै वा अ॒मृत॑ म॒मृतं॒ ॅवै हिर॑ण्यꣳ॒॒ हिर॑ण्यं॒ ॅवा अ॒मृत॑ म॒मृतं॒ ॅवै हिर॑ण्यम् । \newline
23. वै हिर॑ण्यꣳ॒॒ हिर॑ण्यं॒ ॅवै वै हिर॑ण्यम् प्रा॒णः प्रा॒णो हिर॑ण्यं॒ ॅवै वै हिर॑ण्यम् प्रा॒णः । \newline
24. हिर॑ण्यम् प्रा॒णः प्रा॒णो हिर॑ण्यꣳ॒॒ हिर॑ण्यम् प्रा॒णः पृ॑षदा॒ज्यम् पृ॑षदा॒ज्यम् प्रा॒णो हिर॑ण्यꣳ॒॒ हिर॑ण्यम् प्रा॒णः पृ॑षदा॒ज्यम् । \newline
25. प्रा॒णः पृ॑षदा॒ज्यम् पृ॑षदा॒ज्यम् प्रा॒णः प्रा॒णः पृ॑षदा॒ज्य म॒मृत॑ म॒मृत॑म् पृषदा॒ज्यम् प्रा॒णः प्रा॒णः पृ॑षदा॒ज्य म॒मृत᳚म् । \newline
26. प्रा॒ण इति॑ प्र - अ॒नः । \newline
27. पृ॒ष॒दा॒ज्य म॒मृत॑ म॒मृत॑म् पृषदा॒ज्यम् पृ॑षदा॒ज्य म॒मृत॑ मे॒वैवा मृत॑म् पृषदा॒ज्यम् पृ॑षदा॒ज्य म॒मृत॑ मे॒व । \newline
28. पृ॒ष॒दा॒ज्यमिति॑ पृषत् - आ॒ज्यम् । \newline
29. अ॒मृत॑ मे॒वैवा मृत॑ म॒मृत॑ मे॒वास्या᳚ स्यै॒वामृत॑ म॒मृत॑ मे॒वास्य॑ । \newline
30. ए॒वास्या᳚ स्यै॒वैवास्य॑ प्रा॒णे प्रा॒णे᳚ ऽस्यै॒वैवास्य॑ प्रा॒णे । \newline
31. अ॒स्य॒ प्रा॒णे प्रा॒णे᳚ ऽस्यास्य प्रा॒णे द॑धाति दधाति प्रा॒णे᳚ ऽस्यास्य प्रा॒णे द॑धाति । \newline
32. प्रा॒णे द॑धाति दधाति प्रा॒णे प्रा॒णे द॑धाति श॒तमा॑नꣳ श॒तमा॑नम् दधाति प्रा॒णे प्रा॒णे द॑धाति श॒तमा॑नम् । \newline
33. प्रा॒ण इति॑ प्र - अ॒ने । \newline
34. द॒धा॒ति॒ श॒तमा॑नꣳ श॒तमा॑नम् दधाति दधाति श॒तमा॑नम् भवति भवति श॒तमा॑नम् दधाति दधाति श॒तमा॑नम् भवति । \newline
35. श॒तमा॑नम् भवति भवति श॒तमा॑नꣳ श॒तमा॑नम् भवति श॒तायुः॑ श॒तायु॑र् भवति श॒तमा॑नꣳ श॒तमा॑नम् भवति श॒तायुः॑ । \newline
36. श॒तमा॑न॒मिति॑ श॒त - मा॒न॒म् । \newline
37. भ॒व॒ति॒ श॒तायुः॑ श॒तायु॑र् भवति भवति श॒तायुः॒ पुरु॑षः॒ पुरु॑षः श॒तायु॑र् भवति भवति श॒तायुः॒ पुरु॑षः । \newline
38. श॒तायुः॒ पुरु॑षः॒ पुरु॑षः श॒तायुः॑ श॒तायुः॒ पुरु॑षः श॒तेन्द्रि॑यः श॒तेन्द्रि॑यः॒ पुरु॑षः श॒तायुः॑ श॒तायुः॒ पुरु॑षः श॒तेन्द्रि॑यः । \newline
39. श॒तायु॒रिति॑ श॒त - आ॒युः॒ । \newline
40. पुरु॑षः श॒तेन्द्रि॑यः श॒तेन्द्रि॑यः॒ पुरु॑षः॒ पुरु॑षः श॒तेन्द्रि॑य॒ आयु॒ ष्यायु॑षि श॒तेन्द्रि॑यः॒ पुरु॑षः॒ पुरु॑षः श॒तेन्द्रि॑य॒ आयु॑षि । \newline
41. श॒तेन्द्रि॑य॒ आयु॒ ष्यायु॑षि श॒तेन्द्रि॑यः श॒तेन्द्रि॑य॒ आयु॑ ष्ये॒वैवायु॑षि श॒तेन्द्रि॑यः श॒तेन्द्रि॑य॒ आयु॑ष्ये॒व । \newline
42. श॒तेन्द्रि॑य॒ इति॑ श॒त - इ॒न्द्रि॒यः॒ । \newline
43. आयु॑ ष्ये॒वैवायु॒ ष्यायु॑ ष्ये॒वेन्द्रि॒य इ॑न्द्रि॒य ए॒वायु॒ ष्यायु॑ ष्ये॒वेन्द्रि॒ये । \newline
44. ए॒वे न्द्रि॒य इ॑न्द्रि॒य ए॒वै वेन्द्रि॒ये प्रति॒ प्रती᳚न्द्रि॒य ए॒वै वेन्द्रि॒ये प्रति॑ । \newline
45. इ॒न्द्रि॒ये प्रति॒ प्रती᳚न्द्रि॒य इ॑न्द्रि॒ये प्रति॑ तिष्ठति तिष्ठति॒ प्रती᳚न्द्रि॒य इ॑न्द्रि॒ये प्रति॑ तिष्ठति । \newline
46. प्रति॑ तिष्ठति तिष्ठति॒ प्रति॒ प्रति॑ तिष्ठ॒ त्यश्व॒ मश्व॑म् तिष्ठति॒ प्रति॒ प्रति॑ तिष्ठ॒ त्यश्व᳚म् । \newline
47. ति॒ष्ठ॒ त्यश्व॒ मश्व॑म् तिष्ठति तिष्ठ॒ त्यश्व॒ मवावाश्व॑म् तिष्ठति तिष्ठ॒ त्यश्व॒ मव॑ । \newline
48. अश्व॒ मवावाश्व॒ मश्व॒ मव॑ घ्रापयति घ्रापय॒ त्यवाश्व॒ मश्व॒ मव॑ घ्रापयति । \newline
49. अव॑ घ्रापयति घ्रापय॒ त्यवाव॑ घ्रापयति प्राजाप॒त्यः प्रा॑जाप॒त्यो घ्रा॑पय॒ त्यवाव॑ घ्रापयति प्राजाप॒त्यः । \newline
50. घ्रा॒प॒य॒ति॒ प्रा॒जा॒प॒त्यः प्रा॑जाप॒त्यो घ्रा॑पयति घ्रापयति प्राजाप॒त्यो वै वै प्रा॑जाप॒त्यो घ्रा॑पयति घ्रापयति प्राजाप॒त्यो वै । \newline
51. प्रा॒जा॒प॒त्यो वै वै प्रा॑जाप॒त्यः प्रा॑जाप॒त्यो वा अश्वो ऽश्वो॒ वै प्रा॑जाप॒त्यः प्रा॑जाप॒त्यो वा अश्वः॑ । \newline
52. प्रा॒जा॒प॒त्य इति॑ प्राजा - प॒त्यः । \newline
53. वा अश्वो ऽश्वो॒ वै वा अश्वः॑ प्राजाप॒त्यः प्रा॑जाप॒त्यो ऽश्वो॒ वै वा अश्वः॑ प्राजाप॒त्यः । \newline
54. अश्वः॑ प्राजाप॒त्यः प्रा॑जाप॒त्यो ऽश्वो ऽश्वः॑ प्राजाप॒त्यः प्रा॒णः प्रा॒णः प्रा॑जाप॒त्यो ऽश्वो ऽश्वः॑ प्राजाप॒त्यः प्रा॒णः । \newline
55. प्रा॒जा॒प॒त्यः प्रा॒णः प्रा॒णः प्रा॑जाप॒त्यः प्रा॑जाप॒त्यः प्रा॒णः स्वाथ् स्वात् प्रा॒णः प्रा॑जाप॒त्यः प्रा॑जाप॒त्यः प्रा॒णः स्वात् । \newline
56. प्रा॒जा॒प॒त्य इति॑ प्राजा - प॒त्यः । \newline
57. प्रा॒णः स्वाथ् स्वात् प्रा॒णः प्रा॒णः स्वा दे॒वैव स्वात् प्रा॒णः प्रा॒णः स्वादे॒व । \newline
58. प्रा॒ण इति॑ प्र - अ॒नः । \newline
59. स्वादे॒वैव स्वाथ् स्वा दे॒वास्मा॑ अस्मा ए॒व स्वाथ् स्वा दे॒वास्मै᳚ । \newline
60. ए॒वास्मा॑ अस्मा ए॒वैवास्मै॒ योने॒र् योने॑ रस्मा ए॒वैवास्मै॒ योनेः᳚ । \newline
61. अ॒स्मै॒ योने॒र् योने॑ रस्मा अस्मै॒ योनेः᳚ प्रा॒णम् प्रा॒णं ॅयोने॑ रस्मा अस्मै॒ योनेः᳚ प्रा॒णम् । \newline
62. योनेः᳚ प्रा॒णम् प्रा॒णं ॅयोने॒र् योनेः᳚ प्रा॒णम् निर् णिष् प्रा॒णं ॅयोने॒र् योनेः᳚ प्रा॒णम् निः । \newline
63. प्रा॒णम् निर् णिष् प्रा॒णम् प्रा॒णम् निर् मि॑मीते मिमीते॒ निष् प्रा॒णम् प्रा॒णम् निर् मि॑मीते । \newline
64. प्रा॒णमिति॑ प्र - अ॒नम् । \newline
65. निर् मि॑मीते मिमीते॒ निर् णिर् मि॑मीते॒ वि वि मि॑मीते॒ निर् णिर् मि॑मीते॒ वि । \newline
66. मि॒मी॒ते॒ वि वि मि॑मीते मिमीते॒ वि वै वै वि मि॑मीते मिमीते॒ वि वै । \newline
67. वि वै वै वि वि वा ए॒त स्यै॒तस्य॒ वै वि वि वा ए॒तस्य॑ । \newline
68. वा ए॒त स्यै॒तस्य॒ वै वा ए॒तस्य॑ य॒ज्ञो य॒ज्ञ् ए॒तस्य॒ वै वा ए॒तस्य॑ य॒ज्ञ्ः । \newline
69. ए॒तस्य॑ य॒ज्ञो य॒ज्ञ् ए॒त स्यै॒तस्य॑ य॒ज्ञ् श्छि॑द्यते छिद्यते य॒ज्ञ् ए॒त स्यै॒तस्य॑ य॒ज्ञ् श्छि॑द्यते । \newline
70. य॒ज्ञ् श्छि॑द्यते छिद्यते य॒ज्ञो य॒ज्ञ् श्छि॑द्यते॒ यस्य॒ यस्य॑ छिद्यते य॒ज्ञो य॒ज्ञ् श्छि॑द्यते॒ यस्य॑ । \newline
71. छि॒द्य॒ते॒ यस्य॒ यस्य॑ छिद्यते छिद्यते॒ यस्य॑ पृषदा॒ज्यम् पृ॑षदा॒ज्यं ॅयस्य॑ छिद्यते छिद्यते॒ यस्य॑ पृषदा॒ज्यम् । \newline
72. यस्य॑ पृषदा॒ज्यम् पृ॑षदा॒ज्यं ॅयस्य॒ यस्य॑ पृषदा॒ज्यꣳ स्कन्द॑ति॒ स्कन्द॑ति पृषदा॒ज्यं ॅयस्य॒ यस्य॑ पृषदा॒ज्यꣳ स्कन्द॑ति । \newline
73. पृ॒ष॒दा॒ज्यꣳ स्कन्द॑ति॒ स्कन्द॑ति पृषदा॒ज्यम् पृ॑षदा॒ज्यꣳ स्कन्द॑ति वैष्ण॒व्या वै᳚ष्ण॒व्या स्कन्द॑ति पृषदा॒ज्यम् पृ॑षदा॒ज्यꣳ स्कन्द॑ति वैष्ण॒व्या । \newline
74. पृ॒ष॒दा॒ज्यमिति॑ पृषत् - आ॒ज्यम् । \newline
75. स्कन्द॑ति वैष्ण॒व्या वै᳚ष्ण॒व्या स्कन्द॑ति॒ स्कन्द॑ति वैष्ण॒व्यर्चर्चा वै᳚ष्ण॒व्या स्कन्द॑ति॒ स्कन्द॑ति वैष्ण॒व्यर्चा । \newline
76. वै॒ष्ण॒व्यर्चर्चा वै᳚ष्ण॒व्या वै᳚ष्ण॒व्यर्चा पुनः॒ पुनर्॑. ऋ॒चा वै᳚ष्ण॒व्या वै᳚ष्ण॒व्यर्चा पुनः॑ । \newline
77. ऋ॒चा पुनः॒ पुनर्॑. ऋ॒चर्चा पुन॑र् गृह्णाति गृह्णाति॒ पुनर्॑. ऋ॒चर्चा पुन॑र् गृह्णाति । \newline
78. पुन॑र् गृह्णाति गृह्णाति॒ पुनः॒ पुन॑र् गृह्णाति य॒ज्ञो य॒ज्ञो गृ॑ह्णाति॒ पुनः॒ पुन॑र् गृह्णाति य॒ज्ञ्ः । \newline
79. गृ॒ह्णा॒ति॒ य॒ज्ञो य॒ज्ञो गृ॑ह्णाति गृह्णाति य॒ज्ञो वै वै य॒ज्ञो गृ॑ह्णाति गृह्णाति य॒ज्ञो वै । \newline
80. य॒ज्ञो वै वै य॒ज्ञो य॒ज्ञो वै विष्णु॒र् विष्णु॒र् वै य॒ज्ञो य॒ज्ञो वै विष्णुः॑ । \newline
81. वै विष्णु॒र् विष्णु॒र् वै वै विष्णु॑र् य॒ज्ञेन॑ य॒ज्ञेन॒ विष्णु॒र् वै वै विष्णु॑र् य॒ज्ञेन॑ । \newline
82. विष्णु॑र् य॒ज्ञेन॑ य॒ज्ञेन॒ विष्णु॒र् विष्णु॑र् य॒ज्ञेनै॒वैव य॒ज्ञेन॒ विष्णु॒र् विष्णु॑र् य॒ज्ञेनै॒व । \newline
83. य॒ज्ञे नै॒वैव य॒ज्ञेन॑ य॒ज्ञेनै॒व य॒ज्ञ्ं ॅय॒ज्ञ् मे॒व य॒ज्ञेन॑ य॒ज्ञेनै॒व य॒ज्ञ्म् । \newline
84. ए॒व य॒ज्ञ्ं ॅय॒ज्ञ् मे॒वैव य॒ज्ञ्ꣳ सꣳ सं ॅय॒ज्ञ् मे॒वैव य॒ज्ञ्ꣳ सम् । \newline
85. य॒ज्ञ्ꣳ सꣳ सं ॅय॒ज्ञ्ं ॅय॒ज्ञ्ꣳ सम् त॑नोति तनोति॒ सं ॅय॒ज्ञ्ं ॅय॒ज्ञ्ꣳ सम् त॑नोति । \newline
86. सम् त॑नोति तनोति॒ सꣳ सम् त॑नोति । \newline
87. त॒नो॒तीति॑ तनोति । \newline
\pagebreak
\markright{ TS 3.2.7.1  \hfill https://www.vedavms.in \hfill}

\section{ TS 3.2.7.1 }

\textbf{TS 3.2.7.1 } \newline
\textbf{Samhita Paata} \newline

देव॑ सवितरे॒तत् ते॒ प्राऽऽ*ह॒ तत् प्र च॑ सु॒व प्र च॑ यज॒ बृह॒स्पति॑र्ब्र॒ह्मा ऽऽयु॑ष्मत्या ऋ॒चो मा गा॑त तनू॒पाथ् साम्नः॑ स॒त्या व॑ आ॒शिषः॑ सन्तु स॒त्या आकू॑तय ऋ॒तं च॑ स॒त्यं च॑ वदत स्तु॒त दे॒वस्य॑ सवि॒तुः प्र॑स॒वे स्तु॒तस्य॑ स्तु॒तम॒स्यूर्जं॒ मह्यꣳ॑ स्तु॒तं दु॑हा॒मा मा᳚ स्तु॒तस्य॑ स्तु॒तं ग॑म्याच्छ॒स्त्रस्य॑ श॒स्त्र - [  ] \newline

\textbf{Pada Paata} \newline

देव॑ । स॒वि॒तः॒ । ए॒तत् । ते॒ । प्रेति॑ । आ॒ह॒ । तत् । प्रेति॑ । च॒ । सु॒व । प्रेति॑ । च॒ । य॒ज॒ । बृह॒स्पतिः॑ । ब्र॒ह्मा । आयु॑ष्मत्याः । ऋ॒चः । मा । गा॒त॒ । त॒नू॒पादिति॑ तनू - पात् । साम्नः॑ । स॒त्याः । वः॒ । आ॒शिष॒ इत्या᳚ - शिषः॑ । स॒न्तु॒ । स॒त्याः । आकू॑तय॒ इत्या-कू॒त॒यः॒ । ऋ॒तम् । च॒ । स॒त्यम् । च॒ । व॒द॒त॒ । स्तु॒त । दे॒वस्य॑ । स॒वि॒तुः । प्र॒स॒व इति॑ प्र - स॒वे । स्तु॒तस्य॑ । स्तु॒तम् । अ॒सि॒ । ऊर्ज᳚म् । मह्य᳚म् । स्तु॒तम् । दु॒हा॒म् । एति॑ । मा॒ । स्तु॒तस्य॑ । स्तु॒तम् । ग॒म्या॒त् । श॒स्त्रस्य॑ । श॒स्त्रम् ।  \newline


\textbf{Krama Paata} \newline

देव॑ सवितः । स॒वि॒त॒रे॒तत् । ए॒तत् ते᳚ । ते॒ प्र । प्राह॑ । आ॒ह॒ तत् । तत् प्र । प्र च॑ । च॒ सु॒व । सु॒व प्र । प्र च॑ । च॒ य॒ज॒ । य॒ज॒ बृह॒स्पतिः॑ । बृह॒स्पति॑र् ब्र॒ह्मा । ब्र॒ह्मा ऽऽयु॑ष्मत्याः । आयु॑ष्मत्या ऋ॒चः । ऋ॒चो मा । मा गा॑त । गा॒त॒ त॒नू॒पात् । त॒नू॒पाथ् साम्नः॑ । त॒नू॒पादिति॑ तनू - पात् । साम्नः॑ स॒त्याः । स॒त्या वः॑ । व॒ आ॒शिषः॑ । आ॒शिषः॑ सन्तु । आ॒शिष॒ इत्या᳚ - शिषः॑ । स॒न्तु॒ स॒त्याः । स॒त्या आकू॑तयः । आकू॑तय ऋ॒तम् । आकू॑तय॒ इत्या - कू॒त॒यः॒ । ऋ॒तम् च॑ । च॒ स॒त्यम् । स॒त्यम् च॑ । च॒ व॒द॒त॒ । व॒द॒त॒ स्तु॒त । स्तु॒त दे॒वस्य॑ । दे॒वस्य॑ सवि॒तुः । स॒वि॒तुः प्र॑स॒वे । प्र॒स॒वे स्तु॒तस्य॑ । प्र॒स॒व इति॑ प्र - स॒वे । स्तु॒तस्य॑ स्तु॒तम् । स्तु॒तम॑सि । अ॒स्यूर्ज᳚म् । ऊर्ज॒म् मह्य᳚म् । मह्यꣳ॑ स्तु॒तम् । स्तु॒तम् दु॑हाम् । दु॒हा॒मा । आ मा᳚ । मा॒ स्तु॒तस्य॑ । स्तु॒तस्य॑ स्तु॒तम् । स्तु॒तम् ग॑म्यात् । ग॒म्या॒च्छ॒स्त्रस्य॑ । श॒स्त्रस्य॑ श॒स्त्रम् । श॒स्त्रम॑सि \newline

\textbf{Jatai Paata} \newline

1. देव॑ सवितः सवित॒र् देव॒ देव॑ सवितः । \newline
2. स॒वि॒त॒ रे॒त दे॒तथ् स॑वितः सवित रे॒तत् । \newline
3. ए॒तत् ते॑ त ए॒त दे॒तत् ते᳚ । \newline
4. ते॒ प्र प्र ते॑ ते॒ प्र । \newline
5. प्राहा॑ह॒ प्र प्राह॑ । \newline
6. आ॒ह॒ तत् तदा॑ हाह॒ तत् । \newline
7. तत् प्र प्र तत् तत् प्र । \newline
8. प्र च॑ च॒ प्र प्र च॑ । \newline
9. च॒ सु॒व सु॒व च॑ च सु॒व । \newline
10. सु॒व प्र प्र सु॒व सु॒व प्र । \newline
11. प्र च॑ च॒ प्र प्र च॑ । \newline
12. च॒ य॒ज॒ य॒ज॒ च॒ च॒ य॒ज॒ । \newline
13. य॒ज॒ बृह॒स्पति॒र् बृह॒स्पति॑र् यज यज॒ बृह॒स्पतिः॑ । \newline
14. बृह॒स्पति॑र् ब्र॒ह्मा ब्र॒ह्मा बृह॒स्पति॒र् बृह॒स्पति॑र् ब्र॒ह्मा । \newline
15. ब्र॒ह्मा ऽऽयु॑ष्मत्या॒ आयु॑ष्मत्या ब्र॒ह्मा ब्र॒ह्मा ऽऽयु॑ष्मत्याः । \newline
16. आयु॑ष्मत्या ऋ॒च ऋ॒च आयु॑ष्मत्या॒ आयु॑ष्मत्या ऋ॒चः । \newline
17. ऋ॒चो मा मर्च ऋ॒चो मा । \newline
18. मा गा॑त गात॒ मा मा गा॑त । \newline
19. गा॒त॒ त॒नू॒पात् त॑नू॒पाद् गा॑त गात तनू॒पात् । \newline
20. त॒नू॒पाथ् साम्नः॒ साम्न॑ स्तनू॒पात् त॑नू॒पाथ् साम्नः॑ । \newline
21. त॒नू॒पादिति॑ तनू - पात् । \newline
22. साम्नः॑ स॒त्याः स॒त्याः साम्नः॒ साम्नः॑ स॒त्याः । \newline
23. स॒त्या वो॑ वः स॒त्याः स॒त्या वः॑ । \newline
24. व॒ आ॒शिष॑ आ॒शिषो॑ वो व आ॒शिषः॑ । \newline
25. आ॒शिषः॑ सन्तु सन् त्वा॒शिष॑ आ॒शिषः॑ सन्तु । \newline
26. आ॒शिष॒ इत्या᳚ - शिषः॑ । \newline
27. स॒न्तु॒ स॒त्याः स॒त्याः स॑न्तु सन्तु स॒त्याः । \newline
28. स॒त्या आकू॑तय॒ आकू॑तयः स॒त्याः स॒त्या आकू॑तयः । \newline
29. आकू॑तय ऋ॒त मृ॒त माकू॑तय॒ आकू॑तय ऋ॒तम् । \newline
30. आकू॑तय॒ इत्या - कू॒त॒यः॒ । \newline
31. ऋ॒तम् च॑ च॒ र्त मृ॒तम् च॑ । \newline
32. च॒ स॒त्यꣳ स॒त्यम् च॑ च स॒त्यम् । \newline
33. स॒त्यम् च॑ च स॒त्यꣳ स॒त्यम् च॑ । \newline
34. च॒ व॒द॒त॒ व॒द॒त॒ च॒ च॒ व॒द॒त॒ । \newline
35. व॒द॒त॒ स्तु॒त स्तु॒त व॑दत वदत स्तु॒त । \newline
36. स्तु॒त दे॒वस्य॑ दे॒वस्य॑ स्तु॒त स्तु॒त दे॒वस्य॑ । \newline
37. दे॒वस्य॑ सवि॒तुः स॑वि॒तुर् दे॒वस्य॑ दे॒वस्य॑ सवि॒तुः । \newline
38. स॒वि॒तुः प्र॑स॒वे प्र॑स॒वे स॑वि॒तुः स॑वि॒तुः प्र॑स॒वे । \newline
39. प्र॒स॒वे स्तु॒तस्य॑ स्तु॒तस्य॑ प्रस॒वे प्र॑स॒वे स्तु॒तस्य॑ । \newline
40. प्र॒स॒व इति॑ प्र - स॒वे । \newline
41. स्तु॒तस्य॑ स्तु॒तꣳ स्तु॒तꣳ स्तु॒तस्य॑ स्तु॒तस्य॑ स्तु॒तम् । \newline
42. स्तु॒त म॑स्यसि स्तु॒तꣳ स्तु॒त म॑सि । \newline
43. अ॒स्यूर्ज॒ मूर्ज॑ मस्य॒ स्यूर्ज᳚म् । \newline
44. ऊर्ज॒म् मह्य॒म् मह्य॒ मूर्ज॒ मूर्ज॒म् मह्य᳚म् । \newline
45. मह्यꣳ॑ स्तु॒तꣳ स्तु॒तम् मह्य॒म् मह्यꣳ॑ स्तु॒तम् । \newline
46. स्तु॒तम् दु॑हाम् दुहाꣳ स्तु॒तꣳ स्तु॒तम् दु॑हाम् । \newline
47. दु॒हा॒ मा दु॑हाम् दुहा॒ मा । \newline
48. आ मा॒ मा ऽऽमा᳚ । \newline
49. मा॒ स्तु॒तस्य॑ स्तु॒तस्य॑ मा मा स्तु॒तस्य॑ । \newline
50. स्तु॒तस्य॑ स्तु॒तꣳ स्तु॒तꣳ स्तु॒तस्य॑ स्तु॒तस्य॑ स्तु॒तम् । \newline
51. स्तु॒तम् ग॑म्याद् गम्याथ् स्तु॒तꣳ स्तु॒तम् ग॑म्यात् । \newline
52. ग॒म्या॒ च्छ॒स्त्रस्य॑ श॒स्त्रस्य॑ गम्याद् गम्या च्छ॒स्त्रस्य॑ । \newline
53. श॒स्त्रस्य॑ श॒स्त्रꣳ श॒स्त्रꣳ श॒स्त्रस्य॑ श॒स्त्रस्य॑ श॒स्त्रम् । \newline
54. श॒स्त्र म॑स्यसि श॒स्त्रꣳ श॒स्त्र म॑सि । \newline

\textbf{Ghana Paata } \newline

1. देव॑ सवितः सवित॒र् देव॒ देव॑ सवित रे॒त दे॒तथ् स॑वित॒र् देव॒ देव॑ सवित रे॒तत् । \newline
2. स॒वि॒त॒ रे॒त दे॒तथ् स॑वितः सवित रे॒तत् ते॑ त ए॒तथ् स॑वितः सवित रे॒तत् ते᳚ । \newline
3. ए॒तत् ते॑ त ए॒त दे॒तत् ते॒ प्र प्र त॑ ए॒त दे॒तत् ते॒ प्र । \newline
4. ते॒ प्र प्र ते॑ ते॒ प्राहा॑ह॒ प्र ते॑ ते॒ प्राह॑ । \newline
5. प्राहा॑ह॒ प्र प्राह॒ तत् तदा॑ह॒ प्र प्राह॒ तत् । \newline
6. आ॒ह॒ तत् तदा॑ हाह॒ तत् प्र प्र तदा॑ हाह॒ तत् प्र । \newline
7. तत् प्र प्र तत् तत् प्र च॑ च॒ प्र तत् तत् प्र च॑ । \newline
8. प्र च॑ च॒ प्र प्र च॑ सु॒व सु॒व च॒ प्र प्र च॑ सु॒व । \newline
9. च॒ सु॒व सु॒व च॑ च सु॒व प्र प्र सु॒व च॑ च सु॒व प्र । \newline
10. सु॒व प्र प्र सु॒व सु॒व प्र च॑ च॒ प्र सु॒व सु॒व प्र च॑ । \newline
11. प्र च॑ च॒ प्र प्र च॑ यज यज च॒ प्र प्र च॑ यज । \newline
12. च॒ य॒ज॒ य॒ज॒ च॒ च॒ य॒ज॒ बृह॒स्पति॒र् बृह॒स्पति॑र् यज च च यज॒ बृह॒स्पतिः॑ । \newline
13. य॒ज॒ बृह॒स्पति॒र् बृह॒स्पति॑र् यज यज॒ बृह॒स्पति॑र् ब्र॒ह्मा ब्र॒ह्मा बृह॒स्पति॑र् यज यज॒ बृह॒स्पति॑र् ब्र॒ह्मा । \newline
14. बृह॒स्पति॑र् ब्र॒ह्मा ब्र॒ह्मा बृह॒स्पति॒र् बृह॒स्पति॑र् ब्र॒ह्मा ऽऽयु॑ष्मत्या॒ आयु॑ष्मत्या ब्र॒ह्मा बृह॒स्पति॒र् बृह॒स्पति॑र् ब्र॒ह्मा ऽऽयु॑ष्मत्याः । \newline
15. ब्र॒ह्मा ऽऽयु॑ष्मत्या॒ आयु॑ष्मत्या ब्र॒ह्मा ब्र॒ह्मा ऽऽयु॑ष्मत्या ऋ॒च ऋ॒च आयु॑ष्मत्या ब्र॒ह्मा ब्र॒ह्मा ऽऽयु॑ष्मत्या ऋ॒चः । \newline
16. आयु॑ष्मत्या ऋ॒च ऋ॒च आयु॑ष्मत्या॒ आयु॑ष्मत्या ऋ॒चो मा मर्च आयु॑ष्मत्या॒ आयु॑ष्मत्या ऋ॒चो मा । \newline
17. ऋ॒चो मा मर्च ऋ॒चो मा गा॑त गात॒ मर्च ऋ॒चो मा गा॑त । \newline
18. मा गा॑त गात॒ मा मा गा॑त तनू॒पात् त॑नू॒पाद् गा॑त॒ मा मा गा॑त तनू॒पात् । \newline
19. गा॒त॒ त॒नू॒पात् त॑नू॒पाद् गा॑त गात तनू॒पाथ् साम्नः॒ साम्न॑ स्तनू॒पाद् गा॑त गात तनू॒पाथ् साम्नः॑ । \newline
20. त॒नू॒पाथ् साम्नः॒ साम्न॑ स्तनू॒पात् त॑नू॒पाथ् साम्नः॑ स॒त्याः स॒त्याः साम्न॑ स्तनू॒पात् त॑नू॒पाथ् साम्नः॑ स॒त्याः । \newline
21. त॒नू॒पादिति॑ तनू - पात् । \newline
22. साम्नः॑ स॒त्याः स॒त्याः साम्नः॒ साम्नः॑ स॒त्या वो॑ वः स॒त्याः साम्नः॒ साम्नः॑ स॒त्या वः॑ । \newline
23. स॒त्या वो॑ वः स॒त्याः स॒त्या व॑ आ॒शिष॑ आ॒शिषो॑ वः स॒त्याः स॒त्या व॑ आ॒शिषः॑ । \newline
24. व॒ आ॒शिष॑ आ॒शिषो॑ वो व आ॒शिषः॑ सन्तु स न्त्वा॒शिषो॑ वो व आ॒शिषः॑ सन्तु । \newline
25. आ॒शिषः॑ सन्तु स न्त्वा॒शिष॑ आ॒शिषः॑ सन्तु स॒त्याः स॒त्याः स॑ न्त्वा॒शिष॑ आ॒शिषः॑ सन्तु स॒त्याः । \newline
26. आ॒शिष॒ इत्या᳚ - शिषः॑ । \newline
27. स॒न्तु॒ स॒त्याः स॒त्याः स॑न्तु सन्तु स॒त्या आकू॑तय॒ आकू॑तयः स॒त्याः स॑न्तु सन्तु स॒त्या आकू॑तयः । \newline
28. स॒त्या आकू॑तय॒ आकू॑तयः स॒त्याः स॒त्या आकू॑तय ऋ॒त मृ॒त माकू॑तयः स॒त्याः स॒त्या आकू॑तय ऋ॒तम् । \newline
29. आकू॑तय ऋ॒त मृ॒त माकू॑तय॒ आकू॑तय ऋ॒तम् च॑ च॒ र्त माकू॑तय॒ आकू॑तय ऋ॒तम् च॑ । \newline
30. आकू॑तय॒ इत्या - कू॒त॒यः॒ । \newline
31. ऋ॒तम् च॑ च॒ र्त मृ॒तम् च॑ स॒त्यꣳ स॒त्यम् च॒ र्त मृ॒तम् च॑ स॒त्यम् । \newline
32. च॒ स॒त्यꣳ स॒त्यम् च॑ च स॒त्यम् च॑ च स॒त्यम् च॑ च स॒त्यम् च॑ । \newline
33. स॒त्यम् च॑ च स॒त्यꣳ स॒त्यम् च॑ वदत वदत च स॒त्यꣳ स॒त्यम् च॑ वदत । \newline
34. च॒ व॒द॒त॒ व॒द॒त॒ च॒ च॒ व॒द॒त॒ स्तु॒त स्तु॒त व॑दत च च वदत स्तु॒त । \newline
35. व॒द॒त॒ स्तु॒त स्तु॒त व॑दत वदत स्तु॒त दे॒वस्य॑ दे॒वस्य॑ स्तु॒त व॑दत वदत स्तु॒त दे॒वस्य॑ । \newline
36. स्तु॒त दे॒वस्य॑ दे॒वस्य॑ स्तु॒त स्तु॒त दे॒वस्य॑ सवि॒तुः स॑वि॒तुर् दे॒वस्य॑ स्तु॒त स्तु॒त दे॒वस्य॑ सवि॒तुः । \newline
37. दे॒वस्य॑ सवि॒तुः स॑वि॒तुर् दे॒वस्य॑ दे॒वस्य॑ सवि॒तुः प्र॑स॒वे प्र॑स॒वे स॑वि॒तुर् दे॒वस्य॑ दे॒वस्य॑ सवि॒तुः प्र॑स॒वे । \newline
38. स॒वि॒तुः प्र॑स॒वे प्र॑स॒वे स॑वि॒तुः स॑वि॒तुः प्र॑स॒वे स्तु॒तस्य॑ स्तु॒तस्य॑ प्रस॒वे स॑वि॒तुः स॑वि॒तुः प्र॑स॒वे स्तु॒तस्य॑ । \newline
39. प्र॒स॒वे स्तु॒तस्य॑ स्तु॒तस्य॑ प्रस॒वे प्र॑स॒वे स्तु॒तस्य॑ स्तु॒तꣳ स्तु॒तꣳ स्तु॒तस्य॑ प्रस॒वे प्र॑स॒वे स्तु॒तस्य॑ स्तु॒तम् । \newline
40. प्र॒स॒व इति॑ प्र - स॒वे । \newline
41. स्तु॒तस्य॑ स्तु॒तꣳ स्तु॒तꣳ स्तु॒तस्य॑ स्तु॒तस्य॑ स्तु॒त म॑स्यसि स्तु॒तꣳ स्तु॒तस्य॑ स्तु॒तस्य॑ स्तु॒त म॑सि । \newline
42. स्तु॒त म॑स्यसि स्तु॒तꣳ स्तु॒त म॒स्यूर्ज॒ मूर्ज॑ मसि स्तु॒तꣳ स्तु॒त म॒स्यूर्ज᳚म् । \newline
43. अ॒स्यूर्ज॒ मूर्ज॑ मस्य॒ स्यूर्ज॒म् मह्य॒म् मह्य॒ मूर्ज॑ मस्य॒ स्यूर्ज॒म् मह्य᳚म् । \newline
44. ऊर्ज॒म् मह्य॒म् मह्य॒ मूर्ज॒ मूर्ज॒म् मह्यꣳ॑ स्तु॒तꣳ स्तु॒तम् मह्य॒ मूर्ज॒ मूर्ज॒म् मह्यꣳ॑ स्तु॒तम् । \newline
45. मह्यꣳ॑ स्तु॒तꣳ स्तु॒तम् मह्य॒म् मह्यꣳ॑ स्तु॒तम् दु॑हाम् दुहाꣳ स्तु॒तम् मह्य॒म् मह्यꣳ॑ स्तु॒तम् दु॑हाम् । \newline
46. स्तु॒तम् दु॑हाम् दुहाꣳ स्तु॒तꣳ स्तु॒तम् दु॑हा॒ मा दु॑हाꣳ स्तु॒तꣳ स्तु॒तम् दु॑हा॒ मा । \newline
47. दु॒हा॒ मा दु॑हाम् दुहा॒ मा मा॒ मा ऽऽदु॑हाम् दुहा॒ मा मा᳚ । \newline
48. आ मा॒ मा ऽऽमा᳚ स्तु॒तस्य॑ स्तु॒तस्य॒ मा ऽऽमा᳚ स्तु॒तस्य॑ । \newline
49. मा॒ स्तु॒तस्य॑ स्तु॒तस्य॑ मा मा स्तु॒तस्य॑ स्तु॒तꣳ स्तु॒तꣳ स्तु॒तस्य॑ मा मा स्तु॒तस्य॑ स्तु॒तम् । \newline
50. स्तु॒तस्य॑ स्तु॒तꣳ स्तु॒तꣳ स्तु॒तस्य॑ स्तु॒तस्य॑ स्तु॒तम् ग॑म्याद् गम्याथ् स्तु॒तꣳ स्तु॒तस्य॑ स्तु॒तस्य॑ स्तु॒तम् ग॑म्यात् । \newline
51. स्तु॒तम् ग॑म्याद् गम्याथ् स्तु॒तꣳ स्तु॒तम् ग॑म्या च्छ॒स्त्रस्य॑ श॒स्त्रस्य॑ गम्याथ् स्तु॒तꣳ स्तु॒तम् ग॑म्या च्छ॒स्त्रस्य॑ । \newline
52. ग॒म्या॒ च्छ॒स्त्रस्य॑ श॒स्त्रस्य॑ गम्याद् गम्या च्छ॒स्त्रस्य॑ श॒स्त्रꣳ श॒स्त्रꣳ श॒स्त्रस्य॑ गम्याद् गम्या च्छ॒स्त्रस्य॑ श॒स्त्रम् । \newline
53. श॒स्त्रस्य॑ श॒स्त्रꣳ श॒स्त्रꣳ श॒स्त्रस्य॑ श॒स्त्रस्य॑ श॒स्त्र म॑स्यसि श॒स्त्रꣳ श॒स्त्रस्य॑ श॒स्त्रस्य॑ श॒स्त्र म॑सि । \newline
54. श॒स्त्र म॑स्यसि श॒स्त्रꣳ श॒स्त्र म॒स्यूर्ज॒ मूर्ज॑ मसि श॒स्त्रꣳ श॒स्त्र म॒स्यूर्ज᳚म् । \newline
\pagebreak
\markright{ TS 3.2.7.2  \hfill https://www.vedavms.in \hfill}

\section{ TS 3.2.7.2 }

\textbf{TS 3.2.7.2 } \newline
\textbf{Samhita Paata} \newline

म॒स्यूर्जं॒ मह्यꣳ॑ श॒स्त्रं दु॑हा॒मा मा॑ श॒स्त्रस्य॑ श॒स्त्रं ग॑म्या-दिन्द्रि॒याव॑न्तो वनामहे धुक्षी॒महि॑ प्र॒जामिषं᳚ ॥ सा मे॑ स॒त्याऽऽशीर्दे॒वेषु॑ भूयाद्-ब्रह्मवर्च॒सं माऽऽ ग॑म्यात् ॥ य॒ज्ञो ब॑भूव॒ स आ ब॑भूव॒ सप्रज॑ज्ञे॒ स वा॑वृधे । स दे॒वाना॒मधि॑-पतिर्बभूव॒ सो अ॒स्माꣳ अधि॑पतीन् करोतु व॒यꣳ स्या॑म॒ पत॑यो रयी॒णां ॥ य॒ज्ञो वा॒ वै - [  ] \newline

\textbf{Pada Paata} \newline

अ॒सि॒ । ऊर्ज᳚म् । मह्य᳚म् । श॒स्त्रम् । दु॒हा॒म् । एति॑ । मा॒ । श॒स्त्रस्य॑ । श॒स्त्रम् । ग॒म्या॒त् । इ॒न्द्रि॒याव॑न्त॒ इती᳚न्द्रि॒य - व॒न्तः॒ । व॒ना॒म॒हे॒ । धु॒क्षी॒महि॑ । प्र॒जामिति॑ प्र - जाम् । इष᳚म् ॥ सा । मे॒ । स॒त्या । आ॒शीरित्या᳚ - शीः । दे॒वेषु॑ । भू॒या॒त् । ब्र॒ह्म॒व॒र्च॒समिति॑ ब्रह्म - व॒र्च॒सम् । मा॒ । एति॑ । ग॒म्या॒त् ॥ य॒ज्ञ्ः । ब॒भू॒व॒ । सः । एति॑ । ब॒भू॒व॒ । सः । प्रेति॑ । ज॒ज्ञे॒ । सः । वा॒वृ॒धे॒ ॥ सः । दे॒वाना᳚म् । अधि॑पति॒रित्यधि॑ - प॒तिः॒ । ब॒भू॒व॒ । सः । अ॒स्मान् । अधि॑पती॒नित्यधि॑ - प॒ती॒न् । क॒रो॒तु॒ । व॒यम् । स्या॒म॒ । पत॑यः । र॒यी॒णाम् ॥ य॒ज्ञ्ः । वा॒ । वै ।  \newline


\textbf{Krama Paata} \newline

अ॒स्यूर्ज᳚म् । ऊर्ज॒म् मह्य᳚म् । मह्यꣳ॑ श॒स्त्रम् । श॒स्त्रम् दु॑हाम् । दु॒हा॒मा । आ मा᳚ । मा॒ श॒स्त्रस्य॑ । श॒स्त्रस्य॑ श॒स्त्रम् । श॒स्त्रम् ग॑म्यात् । ग॒म्या॒दि॒न्द्रि॒याव॑न्तः । इ॒न्द्रि॒याव॑न्तो वनामहे । इ॒न्द्रि॒याव॑न्त॒ इती᳚न्द्रि॒य - व॒न्तः॒ । व॒ना॒म॒हे॒ धु॒क्षी॒महि॑ । धु॒क्षी॒महि॑ प्र॒जाम् । प्र॒जामिष᳚म् । प्र॒जामिति॑ प्र - जाम् । इष॒मितीष᳚म् ॥ सा मे᳚ । मे॒ स॒त्या । स॒त्या ऽऽशीः । आ॒शीर् दे॒वेषु॑ । आ॒शीरित्या᳚ - शीः । दे॒वेषु॑ भूयात् । भू॒या॒द् ब्र॒ह्म॒व॒र्च॒सम् । ब्र॒ह्म॒व॒र्च॒सम् मा᳚ । ब्र॒ह्म॒व॒र्च॒समिति॑ ब्रह्म - व॒र्च॒सम् । मा । आ ग॑म्यात् । ग॒म्या॒दिति॑ गम्यात् ॥ य॒ज्ञो ब॑भूव । ब॒भू॒व॒ सः । स आ । आ ब॑भूव । ब॒भू॒व॒ सः । स प्र । प्र ज॑ज्ञे । ज॒ज्ञे॒ सः । स वा॑वृधे । वा॒वृ॒ध॒ इति॑ वावृधे ॥ स दे॒वाना᳚म् । दे॒वाना॒मधि॑पतिः । अधि॑पतिर् बभूव । अधि॑पति॒रित्यधि॑ - प॒तिः॒ । ब॒भू॒व॒ सः । सो अ॒स्मान् । अ॒स्माꣳ अधि॑पतीन् । अधि॑पतीन् करोतु । अधि॑पती॒नित्यधि॑ - प॒ती॒न्॒ । क॒रो॒तु॒ व॒यम् । व॒यꣳ स्या॑म । स्या॒म॒ पत॑यः । पत॑यो रयी॒णाम् । र॒यी॒णामिति॑ रयी॒णाम् ॥ य॒ज्ञो वा᳚ । वा॒ वै । वै य॒ज्ञ्प॑तिम् \newline

\textbf{Jatai Paata} \newline

1. अ॒स्यूर्ज॒ मूर्ज॑ मस्य॒ स्यूर्ज᳚म् । \newline
2. ऊर्ज॒म् मह्य॒म् मह्य॒ मूर्ज॒ मूर्ज॒म् मह्य᳚म् । \newline
3. मह्यꣳ॑ श॒स्त्रꣳ श॒स्त्रम् मह्य॒म् मह्यꣳ॑ श॒स्त्रम् । \newline
4. श॒स्त्रम् दु॑हाम् दुहाꣳ श॒स्त्रꣳ श॒स्त्रम् दु॑हाम् । \newline
5. दु॒हा॒ मा दु॑हाम् दुहा॒ मा । \newline
6. आ मा॒ मा ऽऽमा᳚ । \newline
7. मा॒ श॒स्त्रस्य॑ श॒स्त्रस्य॑ मा मा श॒स्त्रस्य॑ । \newline
8. श॒स्त्रस्य॑ श॒स्त्रꣳ श॒स्त्रꣳ श॒स्त्रस्य॑ श॒स्त्रस्य॑ श॒स्त्रम् । \newline
9. श॒स्त्रम् ग॑म्याद् गम्या च्छ॒स्त्रꣳ श॒स्त्रम् ग॑म्यात् । \newline
10. ग॒म्या॒ दि॒न्द्रि॒याव॑न्त इन्द्रि॒याव॑न्तो गम्याद् गम्या दिन्द्रि॒याव॑न्तः । \newline
11. इ॒न्द्रि॒याव॑न्तो वनामहे वनामह इन्द्रि॒याव॑न्त इन्द्रि॒याव॑न्तो वनामहे । \newline
12. इ॒न्द्रि॒याव॑न्त॒ इती᳚न्द्रि॒य - व॒न्तः॒ । \newline
13. व॒ना॒म॒हे॒ धु॒क्षी॒महि॑ धुक्षी॒महि॑ वनामहे वनामहे धुक्षी॒महि॑ । \newline
14. धु॒क्षी॒महि॑ प्र॒जाम् प्र॒जाम् धु॑क्षी॒महि॑ धुक्षी॒महि॑ प्र॒जाम् । \newline
15. प्र॒जा मिष॒ मिष॑म् प्र॒जाम् प्र॒जा मिष᳚म् । \newline
16. प्र॒जामिति॑ प्र - जाम् । \newline
17. इष॒मितीष᳚म् । \newline
18. सा मे॑ मे॒ सा सा मे᳚ । \newline
19. मे॒ स॒त्या स॒त्या मे॑ मे स॒त्या । \newline
20. स॒त्या ऽऽशी रा॒शीः स॒त्या स॒त्या ऽऽशीः । \newline
21. आ॒शीर् दे॒वेषु॑ दे॒वे ष्वा॒शी रा॒शीर् दे॒वेषु॑ । \newline
22. आ॒शीरित्या᳚ - शीः । \newline
23. दे॒वेषु॑ भूयाद् भूयाद् दे॒वेषु॑ दे॒वेषु॑ भूयात् । \newline
24. भू॒या॒द् ब्र॒ह्म॒व॒र्च॒सम् ब्र॑ह्मवर्च॒सम् भू॑याद् भूयाद् ब्रह्मवर्च॒सम् । \newline
25. ब्र॒ह्म॒व॒र्च॒सम् मा॑ मा ब्रह्मवर्च॒सम् ब्र॑ह्मवर्च॒सम् मा᳚ । \newline
26. ब्र॒ह्म॒व॒र्च॒समिति॑ ब्रह्म - व॒र्च॒सम् । \newline
27. मा ऽऽमा॒ मा । \newline
28. आ ग॑म्याद् गम्या॒ दा ग॑म्यात् । \newline
29. ग॒म्या॒दिति॑ गम्यात् । \newline
30. य॒ज्ञो ब॑भूव बभूव य॒ज्ञो य॒ज्ञो ब॑भूव । \newline
31. ब॒भू॒व॒ स स ब॑भूव बभूव॒ सः । \newline
32. स आ स स आ । \newline
33. आ ब॑भूव बभू॒वा ब॑भूव । \newline
34. ब॒भू॒व॒ स स ब॑भूव बभूव॒ सः । \newline
35. स प्र प्र स स प्र । \newline
36. प्र ज॑ज्ञे जज्ञे॒ प्र प्र ज॑ज्ञे । \newline
37. ज॒ज्ञे॒ स स ज॑ज्ञे जज्ञे॒ सः । \newline
38. स वा॑वृधे वावृधे॒ स स वा॑वृधे । \newline
39. वा॒वृ॒ध॒ इति॑ वावृधे । \newline
40. स दे॒वाना᳚म् दे॒वानाꣳ॒॒ स स दे॒वाना᳚म् । \newline
41. दे॒वाना॒ मधि॑पति॒ रधि॑पतिर् दे॒वाना᳚म् दे॒वाना॒ मधि॑पतिः । \newline
42. अधि॑पतिर् बभूव बभू॒वा धि॑पति॒ रधि॑पतिर् बभूव । \newline
43. अधि॑पति॒रित्यधि॑ - प॒तिः॒ । \newline
44. ब॒भू॒व॒ स स ब॑भूव बभूव॒ सः । \newline
45. सो अ॒स्माꣳ अ॒स्मान् थ्स सो अ॒स्मान् । \newline
46. अ॒स्माꣳ अधि॑पती॒ नधि॑पती न॒स्माꣳ अ॒स्माꣳ अधि॑पतीन् । \newline
47. अधि॑पतीन् करोतु करो॒ त्वधि॑पती॒ नधि॑पतीन् करोतु । \newline
48. अधि॑पती॒नित्यधि॑ - प॒ती॒न् । \newline
49. क॒रो॒तु॒ व॒यं ॅव॒यम् क॑रोतु करोतु व॒यम् । \newline
50. व॒यꣳ स्या॑म स्याम व॒यं ॅव॒यꣳ स्या॑म । \newline
51. स्या॒म॒ पत॑यः॒ पत॑यः स्याम स्याम॒ पत॑यः । \newline
52. पत॑यो रयी॒णाꣳ र॑यी॒णाम् पत॑यः॒ पत॑यो रयी॒णाम् । \newline
53. र॒यी॒णामिति॑ रयी॒णाम् । \newline
54. य॒ज्ञो वा॑ वा य॒ज्ञो य॒ज्ञो वा᳚ । \newline
55. वा॒ वै वै वा॑ वा॒ वै । \newline
56. वै य॒ज्ञ्प॑तिं ॅय॒ज्ञ्प॑तिं॒ ॅवै वै य॒ज्ञ्प॑तिम् । \newline

\textbf{Ghana Paata } \newline

1. अ॒स्यूर्ज॒ मूर्ज॑ मस्य॒ स्यूर्ज॒म् मह्य॒म् मह्य॒ मूर्ज॑ मस्य॒ स्यूर्ज॒म् मह्य᳚म् । \newline
2. ऊर्ज॒म् मह्य॒म् मह्य॒ मूर्ज॒ मूर्ज॒म् मह्यꣳ॑ श॒स्त्रꣳ श॒स्त्रम् मह्य॒ मूर्ज॒ मूर्ज॒म् मह्यꣳ॑ श॒स्त्रम् । \newline
3. मह्यꣳ॑ श॒स्त्रꣳ श॒स्त्रम् मह्य॒म् मह्यꣳ॑ श॒स्त्रम् दु॑हाम् दुहाꣳ श॒स्त्रम् मह्य॒म् मह्यꣳ॑ श॒स्त्रम् दु॑हाम् । \newline
4. श॒स्त्रम् दु॑हाम् दुहाꣳ श॒स्त्रꣳ श॒स्त्रम् दु॑हा॒ मा दु॑हाꣳ श॒स्त्रꣳ श॒स्त्रम् दु॑हा॒ मा । \newline
5. दु॒हा॒ मा दु॑हाम् दुहा॒ मा मा॒ मा ऽऽदु॑हाम् दुहा॒ मा मा᳚ । \newline
6. आ मा॒ मा ऽऽमा॑ श॒स्त्रस्य॑ श॒स्त्रस्य॒ मा ऽऽमा॑ श॒स्त्रस्य॑ । \newline
7. मा॒ श॒स्त्रस्य॑ श॒स्त्रस्य॑ मा मा श॒स्त्रस्य॑ श॒स्त्रꣳ श॒स्त्रꣳ श॒स्त्रस्य॑ मा मा श॒स्त्रस्य॑ श॒स्त्रम् । \newline
8. श॒स्त्रस्य॑ श॒स्त्रꣳ श॒स्त्रꣳ श॒स्त्रस्य॑ श॒स्त्रस्य॑ श॒स्त्रम् ग॑म्याद् गम्या च्छ॒स्त्रꣳ श॒स्त्रस्य॑ श॒स्त्रस्य॑ श॒स्त्रम् ग॑म्यात् । \newline
9. श॒स्त्रम् ग॑म्याद् गम्या च्छ॒स्त्रꣳ श॒स्त्रम् ग॑म्या दिन्द्रि॒याव॑न्त इन्द्रि॒याव॑न्तो गम्या च्छ॒स्त्रꣳ श॒स्त्रम् ग॑म्या दिन्द्रि॒याव॑न्तः । \newline
10. ग॒म्या॒ दि॒न्द्रि॒याव॑न्त इन्द्रि॒याव॑न्तो गम्याद् गम्या दिन्द्रि॒याव॑न्तो वनामहे वनामह इन्द्रि॒याव॑न्तो गम्याद् गम्या दिन्द्रि॒याव॑न्तो वनामहे । \newline
11. इ॒न्द्रि॒याव॑न्तो वनामहे वनामह इन्द्रि॒याव॑न्त इन्द्रि॒याव॑न्तो वनामहे धुक्षी॒महि॑ धुक्षी॒महि॑ वनामह इन्द्रि॒याव॑न्त इन्द्रि॒याव॑न्तो वनामहे धुक्षी॒महि॑ । \newline
12. इ॒न्द्रि॒याव॑न्त॒ इती᳚न्द्रि॒य - व॒न्तः॒ । \newline
13. व॒ना॒म॒हे॒ धु॒क्षी॒महि॑ धुक्षी॒महि॑ वनामहे वनामहे धुक्षी॒महि॑ प्र॒जाम् प्र॒जाम् धु॑क्षी॒महि॑ वनामहे वनामहे धुक्षी॒महि॑ प्र॒जाम् । \newline
14. धु॒क्षी॒महि॑ प्र॒जाम् प्र॒जाम् धु॑क्षी॒महि॑ धुक्षी॒महि॑ प्र॒जा मिष॒ मिष॑म् प्र॒जाम् धु॑क्षी॒महि॑ धुक्षी॒महि॑ प्र॒जा मिष᳚म् । \newline
15. प्र॒जा मिष॒ मिष॑म् प्र॒जाम् प्र॒जा मिष᳚म् । \newline
16. प्र॒जामिति॑ प्र - जाम् । \newline
17. इष॒मितीष᳚म् । \newline
18. सा मे॑ मे॒ सा सा मे॑ स॒त्या स॒त्या मे॒ सा सा मे॑ स॒त्या । \newline
19. मे॒ स॒त्या स॒त्या मे॑ मे स॒त्या ऽऽशी रा॒शीः स॒त्या मे॑ मे स॒त्या ऽऽशीः । \newline
20. स॒त्या ऽऽशी रा॒शीः स॒त्या स॒त्या ऽऽशीर् दे॒वेषु॑ दे॒वे ष्वा॒शीः स॒त्या स॒त्या ऽऽशीर् दे॒वेषु॑ । \newline
21. आ॒शीर् दे॒वेषु॑ दे॒वे ष्वा॒शी रा॒शीर् दे॒वेषु॑ भूयाद् भूयाद् दे॒वे ष्वा॒शी रा॒शीर् दे॒वेषु॑ भूयात् । \newline
22. आ॒शीरित्या᳚ - शीः । \newline
23. दे॒वेषु॑ भूयाद् भूयाद् दे॒वेषु॑ दे॒वेषु॑ भूयाद् ब्रह्मवर्च॒सम् ब्र॑ह्मवर्च॒सम् भू॑याद् दे॒वेषु॑ दे॒वेषु॑ भूयाद् ब्रह्मवर्च॒सम् । \newline
24. भू॒या॒द् ब्र॒ह्म॒व॒र्च॒सम् ब्र॑ह्मवर्च॒सम् भू॑याद् भूयाद् ब्रह्मवर्च॒सम् मा॑ मा ब्रह्मवर्च॒सम् भू॑याद् भूयाद् ब्रह्मवर्च॒सम् मा᳚ । \newline
25. ब्र॒ह्म॒व॒र्च॒सम् मा॑ मा ब्रह्मवर्च॒सम् ब्र॑ह्मवर्च॒सम् मा ऽऽमा᳚ ब्रह्मवर्च॒सम् ब्र॑ह्मवर्च॒सम् मा । \newline
26. ब्र॒ह्म॒व॒र्च॒समिति॑ ब्रह्म - व॒र्च॒सम् । \newline
27. मा ऽऽमा॒ मा ऽऽग॑म्याद् गम्या॒ दा मा॒ मा ऽऽग॑म्यात् । \newline
28. आ ग॑म्याद् गम्या॒ दा ग॑म्यात् । \newline
29. ग॒म्या॒दिति॑ गम्यात् । \newline
30. य॒ज्ञो ब॑भूव बभूव य॒ज्ञो य॒ज्ञो ब॑भूव॒ स स ब॑भूव य॒ज्ञो य॒ज्ञो ब॑भूव॒ सः । \newline
31. ब॒भू॒व॒ स स ब॑भूव बभूव॒ स आ स ब॑भूव बभूव॒ स आ । \newline
32. स आ स स आ ब॑भूव बभू॒वा स स आ ब॑भूव । \newline
33. आ ब॑भूव बभू॒वा ब॑भूव॒ स स ब॑भू॒वा ब॑भूव॒ सः । \newline
34. ब॒भू॒व॒ स स ब॑भूव बभूव॒ स प्र प्र स ब॑भूव बभूव॒ स प्र । \newline
35. स प्र प्र स स प्र ज॑ज्ञे जज्ञे॒ प्र स स प्र ज॑ज्ञे । \newline
36. प्र ज॑ज्ञे जज्ञे॒ प्र प्र ज॑ज्ञे॒ स स ज॑ज्ञे॒ प्र प्र ज॑ज्ञे॒ सः । \newline
37. ज॒ज्ञे॒ स स ज॑ज्ञे जज्ञे॒ स वा॑वृधे वावृधे॒ स ज॑ज्ञे जज्ञे॒ स वा॑वृधे । \newline
38. स वा॑वृधे वावृधे॒ स स वा॑वृधे । \newline
39. वा॒वृ॒ध॒ इति॑ वावृधे । \newline
40. स दे॒वाना᳚म् दे॒वानाꣳ॒॒ स स दे॒वाना॒ मधि॑पति॒ रधि॑पतिर् दे॒वानाꣳ॒॒ स स दे॒वाना॒ मधि॑पतिः । \newline
41. दे॒वाना॒ मधि॑पति॒ रधि॑पतिर् दे॒वाना᳚म् दे॒वाना॒ मधि॑पतिर् बभूव बभू॒वा धि॑पतिर् दे॒वाना᳚म् दे॒वाना॒ मधि॑पतिर् बभूव । \newline
42. अधि॑पतिर् बभूव बभू॒वा धि॑पति॒ रधि॑पतिर् बभूव॒ स स ब॑भू॒वा धि॑पति॒ रधि॑पतिर् बभूव॒ सः । \newline
43. अधि॑पति॒रित्यधि॑ - प॒तिः॒ । \newline
44. ब॒भू॒व॒ स स ब॑भूव बभूव॒ सो अ॒स्माꣳ अ॒स्मान् थ्स ब॑भूव बभूव॒ सो अ॒स्मान् । \newline
45. सो अ॒स्माꣳ अ॒स्मान् थ्स सो अ॒स्माꣳ अधि॑पती॒ नधि॑पती न॒स्मान् थ्स सो अ॒स्माꣳ अधि॑पतीन् । \newline
46. अ॒स्माꣳ अधि॑पती॒ नधि॑पती न॒स्माꣳ अ॒स्माꣳ अधि॑पतीन् करोतु करो॒ त्वधि॑पती न॒स्माꣳ अ॒स्माꣳ अधि॑पतीन् करोतु । \newline
47. अधि॑पतीन् करोतु करो॒ त्वधि॑पती॒ नधि॑पतीन् करोतु व॒यं ॅव॒यम् क॑रो॒ त्वधि॑पती॒ नधि॑पतीन् करोतु व॒यम् । \newline
48. अधि॑पती॒नित्यधि॑ - प॒ती॒न् । \newline
49. क॒रो॒तु॒ व॒यं ॅव॒यम् क॑रोतु करोतु व॒यꣳ स्या॑म स्याम व॒यम् क॑रोतु करोतु व॒यꣳ स्या॑म । \newline
50. व॒यꣳ स्या॑म स्याम व॒यं ॅव॒यꣳ स्या॑म॒ पत॑यः॒ पत॑यः स्याम व॒यं ॅव॒यꣳ स्या॑म॒ पत॑यः । \newline
51. स्या॒म॒ पत॑यः॒ पत॑यः स्याम स्याम॒ पत॑यो रयी॒णाꣳ र॑यी॒णाम् पत॑यः स्याम स्याम॒ पत॑यो रयी॒णाम् । \newline
52. पत॑यो रयी॒णाꣳ र॑यी॒णाम् पत॑यः॒ पत॑यो रयी॒णाम् । \newline
53. र॒यी॒णामिति॑ रयी॒णाम् । \newline
54. य॒ज्ञो वा॑ वा य॒ज्ञो य॒ज्ञो वा॒ वै वै वा॑ य॒ज्ञो य॒ज्ञो वा॒ वै । \newline
55. वा॒ वै वै वा॑ वा॒ वै य॒ज्ञ्प॑तिं ॅय॒ज्ञ्प॑तिं॒ ॅवै वा॑ वा॒ वै य॒ज्ञ्प॑तिम् । \newline
56. वै य॒ज्ञ्प॑तिं ॅय॒ज्ञ्प॑तिं॒ ॅवै वै य॒ज्ञ्प॑तिम् दु॒हे दु॒हे य॒ज्ञ्प॑तिं॒ ॅवै वै य॒ज्ञ्प॑तिम् दु॒हे । \newline
\pagebreak
\markright{ TS 3.2.7.3  \hfill https://www.vedavms.in \hfill}

\section{ TS 3.2.7.3 }

\textbf{TS 3.2.7.3 } \newline
\textbf{Samhita Paata} \newline

य॒ज्ञ्प॑तिं दु॒हे य॒ज्ञ्प॑तिर्वा य॒ज्ञ्ं दु॑हे॒ स यः स्तु॑तश॒स्त्रयो॒र्दोह॒म वि॑द्वा॒न॒. यज॑ते॒ तं ॅय॒ज्ञो दु॑हे॒ स इ॒ष्ट्वा पापी॑यान् भवति॒ य ए॑नयो॒र्दोहं॑ ॅवि॒द्वान्. यज॑ते॒ स य॒ज्ञ्ं दु॑हे॒ स इ॒ष्ट्वा वसी॑यान् भवति स्तु॒तस्य॑ स्तु॒तम॒स्यूर्जं॒ मह्यꣳ॑ स्तु॒तं दु॑हा॒मा मा᳚ स्तु॒तस्य॑ स्तु॒तं ग॑म्याच्छ॒स्त्रस्य॑ श॒स्त्रम॒स्यूर्जं॒ मह्यꣳ॑ श॒स्त्रं दु॑हा॒ ( ) मा मा॑ श॒स्त्रस्य॑ श॒स्त्रं ग॑म्या॒दित्या॑है॒ष वै स्तु॑तश॒स्त्रयो॒र्दोह॒स्तं ॅय ए॒वं ॅवि॒द्वान्. यज॑ते दु॒ह ए॒व य॒ज्ञ्मि॒ष्ट्वा वसी॑यान् भवति ॥ \newline

\textbf{Pada Paata} \newline

य॒ज्ञ्प॑ति॒मिति॑ य॒ज्ञ् - प॒ति॒म् । दु॒हे । य॒ज्ञ्प॑ति॒रिति॑ य॒ज्ञ्-प॒तिः॒ । वा॒ । य॒ज्ञ्म् । दु॒हे॒ । सः । यः । स्तु॒त॒श॒स्त्रयो॒रिति॑ स्तुत - श॒स्त्रयोः᳚ । दोह᳚म् । अवि॑द्वान् । यज॑ते । तम् । य॒ज्ञ्ः । दु॒हे॒ । सः । इ॒ष्ट्वा । पापी॑यान् । भ॒व॒ति॒ । यः । ए॒न॒योः॒ । दोह᳚म् । वि॒द्वान् । यज॑ते । सः । य॒ज्ञ्म् । दु॒हे॒ । सः । इ॒ष्ट्वा । वसी॑यान् । भ॒व॒ति॒ । स्तु॒तस्य॑ । स्तु॒तम् । अ॒सि॒ । ऊर्ज᳚म् । मह्य᳚म् । स्तु॒तम् । दु॒हा॒म् । एति॑ । मा॒ । स्तु॒तस्य॑ । स्तु॒तम् । ग॒म्या॒त् । श॒स्त्रस्य॑ । श॒स्त्रम् । अ॒सि॒ । ऊर्ज᳚म् । मह्य᳚म् । श॒स्त्रम् । दु॒हा॒म् ( ) । एति॑ । मा॒ । श॒स्त्रस्य॑ । श॒स्त्रम् । ग॒म्या॒त् । इति॑ । आ॒ह॒ । ए॒षः । वै । स्तु॒त॒श॒स्त्रयो॒रिति॑ स्तुत - श॒स्त्रयोः᳚ । दोहः॑ । तम् । यः । ए॒वम् । वि॒द्वान् । यज॑ते । दु॒हे । ए॒व । य॒ज्ञ्म् । इ॒ष्ट्वा । वसी॑यान् । भ॒व॒ति॒ ॥  \newline


\textbf{Krama Paata} \newline

य॒ज्ञ्प॑तिम् दु॒हे । य॒ज्ञ्प॑ति॒मिति॑ य॒ज्ञ् - प॒ति॒म् । दु॒हे य॒ज्ञ्प॑तिः । य॒ज्ञ्प॑तिर् वा । य॒ज्ञ्प॑ति॒रिति॑ य॒ज्ञ् - प॒तिः॒ । वा॒ य॒ज्ञ्म् । य॒ज्ञ्म् दु॑हे । दु॒हे॒ सः । स यः । यः स्तु॑तश॒स्त्रयोः᳚ । स्तु॒त॒श॒स्त्रयो॒र् दोह᳚म् । स्तु॒त॒श॒स्त्रयो॒रिति॑ स्तुत - श॒स्त्रयोः᳚ । दोह॒मवि॑द्वान् । अवि॑द्वा॒न्॒. यज॑ते । यज॑ते॒ तम् । तं ॅय॒ज्ञ्ः । य॒ज्ञो दु॑हे । दु॒हे॒ सः । स इ॒ष्ट्वा । इ॒ष्ट्वा पापी॑यान् । पापी॑यान् भवति । भ॒व॒ति॒ यः । य ए॑नयोः । ए॒न॒यो॒र् दोह᳚म् । दोहं॑ ॅवि॒द्वान् । वि॒द्वान्. यज॑ते । यज॑ते॒ सः । स य॒ज्ञ्म् । य॒ज्ञ्म् दु॑हे । दु॒हे॒ सः । स इ॒ष्ट्वा । इ॒ष्ट्वा वसी॑यान् । वसी॑यान् भवति । भ॒व॒ति॒ स्तु॒तस्य॑ । स्तु॒तस्य॑ स्तु॒तम् । स्तु॒तम॑सि । अ॒स्यूर्ज᳚म् । ऊर्ज॒म् मह्य᳚म् । मह्यꣳ॑ स्तु॒तम् । स्तु॒तम् दु॑हाम् । दु॒हा॒मा । आ मा᳚ । मा॒ स्तु॒तस्य॑ । स्तु॒तस्य॑ स्तु॒तम् । स्तु॒तम् ग॑म्यात् । ग॒म्या॒च्छ॒स्त्रस्य॑ । श॒स्त्रस्य॑ श॒स्त्रम् । श॒स्त्रम॑सि । अ॒स्यूर्ज᳚म् । ऊर्ज॒म् मह्य᳚म् । मह्यꣳ॑ श॒स्त्रम् । श॒स्त्रम् दु॑हाम् ( ) । दु॒हा॒मा । आ मा᳚ । मा॒ श॒स्त्रस्य॑ । श॒स्त्रस्य॑ श॒स्त्रम् । श॒स्त्रम् ग॑म्यात् । ग॒म्या॒दिति॑ । इत्या॑ह । आ॒है॒षः । ए॒ष वै । वै स्तु॑तश॒स्त्रयोः᳚ । स्तु॒त॒श॒स्त्रयो॒र् दोहः॑ । स्तु॒त॒श॒स्त्रयो॒रिति॑ स्तुत - श॒स्त्रयोः᳚ । दोह॒स्तम् । तं ॅयः । य ए॒वम् । ए॒वं ॅवि॒द्वान् । वि॒द्वान्. यज॑ते । यज॑ते दु॒हे । दु॒ह ए॒व । ए॒व य॒ज्ञ्म् । य॒ज्ञ्मि॒ष्ट्वा । इ॒ष्ट्वा वसी॑यान् । वसी॑यान् भवति । भ॒व॒तीति॑ भवति । \newline

\textbf{Jatai Paata} \newline

1. य॒ज्ञ्प॑तिम् दु॒हे दु॒हे य॒ज्ञ्प॑तिं ॅय॒ज्ञ्प॑तिम् दु॒हे । \newline
2. य॒ज्ञ्प॑ति॒मिति॑ य॒ज्ञ् - प॒ति॒म् । \newline
3. दु॒हे य॒ज्ञ्प॑तिर् य॒ज्ञ्प॑तिर् दु॒हे दु॒हे य॒ज्ञ्प॑तिः । \newline
4. य॒ज्ञ्प॑तिर् वा वा य॒ज्ञ्प॑तिर् य॒ज्ञ्प॑तिर् वा । \newline
5. य॒ज्ञ्प॑ति॒रिति॑ य॒ज्ञ् - प॒तिः॒ । \newline
6. वा॒ य॒ज्ञ्ं ॅय॒ज्ञ्ं ॅवा॑ वा य॒ज्ञ्म् । \newline
7. य॒ज्ञ्म् दु॑हे दुहे य॒ज्ञ्ं ॅय॒ज्ञ्म् दु॑हे । \newline
8. दु॒हे॒ स स दु॑हे दुहे॒ सः । \newline
9. स यो यः स स यः । \newline
10. यः स्तु॑तश॒स्त्रयोः᳚ स्तुतश॒स्त्रयो॒र् यो यः स्तु॑तश॒स्त्रयोः᳚ । \newline
11. स्तु॒त॒श॒स्त्रयो॒र् दोह॒म् दोहꣳ॑ स्तुतश॒स्त्रयोः᳚ स्तुतश॒स्त्रयो॒र् दोह᳚म् । \newline
12. स्तु॒त॒श॒स्त्रयो॒रिति॑ स्तुत - श॒स्त्रयोः᳚ । \newline
13. दोह॒ मवि॑द्वा॒ नवि॑द्वा॒न् दोह॒म् दोह॒ मवि॑द्वान् । \newline
14. अवि॑द्वा॒न्॒. यज॑ते॒ यज॒ते ऽवि॑द्वा॒ नवि॑द्वा॒न्॒. यज॑ते । \newline
15. यज॑ते॒ तम् तं ॅयज॑ते॒ यज॑ते॒ तम् । \newline
16. तं ॅय॒ज्ञो य॒ज्ञ् स्तम् तं ॅय॒ज्ञ्ः । \newline
17. य॒ज्ञो दु॑हे दुहे य॒ज्ञो य॒ज्ञो दु॑हे । \newline
18. दु॒हे॒ स स दु॑हे दुहे॒ सः । \newline
19. स इ॒ष्ट्वे ष्ट्वा स स इ॒ष्ट्वा । \newline
20. इ॒ष्ट्वा पापी॑या॒न् पापी॑या नि॒ष्ट्वे ष्ट्वा पापी॑यान् । \newline
21. पापी॑यान् भवति भवति॒ पापी॑या॒न् पापी॑यान् भवति । \newline
22. भ॒व॒ति॒ यो यो भ॑वति भवति॒ यः । \newline
23. य ए॑नयो रेनयो॒र् यो य ए॑नयोः । \newline
24. ए॒न॒यो॒र् दोह॒म् दोह॑ मेनयो रेनयो॒र् दोह᳚म् । \newline
25. दोहं॑ ॅवि॒द्वान्. वि॒द्वान् दोह॒म् दोहं॑ ॅवि॒द्वान् । \newline
26. वि॒द्वान्. यज॑ते॒ यज॑ते वि॒द्वान्. वि॒द्वान्. यज॑ते । \newline
27. यज॑ते॒ स स यज॑ते॒ यज॑ते॒ सः । \newline
28. स य॒ज्ञ्ं ॅय॒ज्ञ्ꣳ स स य॒ज्ञ्म् । \newline
29. य॒ज्ञ्म् दु॑हे दुहे य॒ज्ञ्ं ॅय॒ज्ञ्म् दु॑हे । \newline
30. दु॒हे॒ स स दु॑हे दुहे॒ सः । \newline
31. स इ॒ष्ट्वे ष्ट्वा स स इ॒ष्ट्वा । \newline
32. इ॒ष्ट्वा वसी॑या॒न्॒. वसी॑या नि॒ष्ट्वे ष्ट्वा वसी॑यान् । \newline
33. वसी॑यान् भवति भवति॒ वसी॑या॒न्॒. वसी॑यान् भवति । \newline
34. भ॒व॒ति॒ स्तु॒तस्य॑ स्तु॒तस्य॑ भवति भवति स्तु॒तस्य॑ । \newline
35. स्तु॒तस्य॑ स्तु॒तꣳ स्तु॒तꣳ स्तु॒तस्य॑ स्तु॒तस्य॑ स्तु॒तम् । \newline
36. स्तु॒त म॑स्यसि स्तु॒तꣳ स्तु॒त म॑सि । \newline
37. अ॒स्यूर्ज॒ मूर्ज॑ मस्य॒ स्यूर्ज᳚म् । \newline
38. ऊर्ज॒म् मह्य॒म् मह्य॒ मूर्ज॒ मूर्ज॒म् मह्य᳚म् । \newline
39. मह्यꣳ॑ स्तु॒तꣳ स्तु॒तम् मह्य॒म् मह्यꣳ॑ स्तु॒तम् । \newline
40. स्तु॒तम् दु॑हाम् दुहाꣳ स्तु॒तꣳ स्तु॒तम् दु॑हाम् । \newline
41. दु॒हा॒ मा दु॑हाम् दुहा॒ मा । \newline
42. आ मा॒ मा ऽऽमा᳚ । \newline
43. मा॒ स्तु॒तस्य॑ स्तु॒तस्य॑ मा मा स्तु॒तस्य॑ । \newline
44. स्तु॒तस्य॑ स्तु॒तꣳ स्तु॒तꣳ स्तु॒तस्य॑ स्तु॒तस्य॑ स्तु॒तम् । \newline
45. स्तु॒तम् ग॑म्याद् गम्याथ् स्तु॒तꣳ स्तु॒तम् ग॑म्यात् । \newline
46. ग॒म्या॒ च्छ॒स्त्रस्य॑ श॒स्त्रस्य॑ गम्याद् गम्या च्छ॒स्त्रस्य॑ । \newline
47. श॒स्त्रस्य॑ श॒स्त्रꣳ श॒स्त्रꣳ श॒स्त्रस्य॑ श॒स्त्रस्य॑ श॒स्त्रम् । \newline
48. श॒स्त्र म॑स्यसि श॒स्त्रꣳ श॒स्त्र म॑सि । \newline
49. अ॒स्यूर्ज॒ मूर्ज॑ मस्य॒ स्यूर्ज᳚म् । \newline
50. ऊर्ज॒म् मह्य॒म् मह्य॒ मूर्ज॒ मूर्ज॒म् मह्य᳚म् । \newline
51. मह्यꣳ॑ श॒स्त्रꣳ श॒स्त्रम् मह्य॒म् मह्यꣳ॑ श॒स्त्रम् । \newline
52. श॒स्त्रम् दु॑हाम् दुहाꣳ श॒स्त्रꣳ श॒स्त्रम् दु॑हाम् । \newline
53. दु॒हा॒ मा दु॑हाम् दुहा॒ मा । \newline
54. आ मा॒ मा ऽऽमा᳚ । \newline
55. मा॒ श॒स्त्रस्य॑ श॒स्त्रस्य॑ मा मा श॒स्त्रस्य॑ । \newline
56. श॒स्त्रस्य॑ श॒स्त्रꣳ श॒स्त्रꣳ श॒स्त्रस्य॑ श॒स्त्रस्य॑ श॒स्त्रम् । \newline
57. श॒स्त्रम् ग॑म्याद् गम्या च्छ॒स्त्रꣳ श॒स्त्रम् ग॑म्यात् । \newline
58. ग॒म्या॒ दितीति॑ गम्याद् गम्या॒दिति॑ । \newline
59. इत्या॑हा॒हे तीत्या॑ह । \newline
60. आ॒है॒ष ए॒ष आ॑हाहै॒षः । \newline
61. ए॒ष वै वा ए॒ष ए॒ष वै । \newline
62. वै स्तु॑तश॒स्त्रयोः᳚ स्तुतश॒स्त्रयो॒र् वै वै स्तु॑तश॒स्त्रयोः᳚ । \newline
63. स्तु॒त॒श॒स्त्रयो॒र् दोहो॒ दोहः॑ स्तुतश॒स्त्रयोः᳚ स्तुतश॒स्त्रयो॒र् दोहः॑ । \newline
64. स्तु॒त॒श॒स्त्रयो॒रिति॑ स्तुत - श॒स्त्रयोः᳚ । \newline
65. दोह॒ स्तम् तम् दोहो॒ दोह॒ स्तम् । \newline
66. तं ॅयो य स्तम् तं ॅयः । \newline
67. य ए॒व मे॒वं ॅयो य ए॒वम् । \newline
68. ए॒वं ॅवि॒द्वान्. वि॒द्वा ने॒व मे॒वं ॅवि॒द्वान् । \newline
69. वि॒द्वान्. यज॑ते॒ यज॑ते वि॒द्वान्. वि॒द्वान्. यज॑ते । \newline
70. यज॑ते दु॒हे दु॒हे यज॑ते॒ यज॑ते दु॒हे । \newline
71. दु॒ह ए॒वैव दु॒हे दु॒ह ए॒व । \newline
72. ए॒व य॒ज्ञ्ं ॅय॒ज्ञ् मे॒वैव य॒ज्ञ्म् । \newline
73. य॒ज्ञ् मि॒ष्ट्वे ष्ट्वा य॒ज्ञ्ं ॅय॒ज्ञ् मि॒ष्ट्वा । \newline
74. इ॒ष्ट्वा वसी॑या॒न्॒. वसी॑या नि॒ष्ट्वे ष्ट्वा वसी॑यान् । \newline
75. वसी॑यान् भवति भवति॒ वसी॑या॒न्॒. वसी॑यान् भवति । \newline
76. भ॒व॒तीति॑ भवति । \newline

\textbf{Ghana Paata } \newline

1. य॒ज्ञ्प॑तिम् दु॒हे दु॒हे य॒ज्ञ्प॑तिं ॅय॒ज्ञ्प॑तिम् दु॒हे य॒ज्ञ्प॑तिर् य॒ज्ञ्प॑तिर् दु॒हे य॒ज्ञ्प॑तिं ॅय॒ज्ञ्प॑तिम् दु॒हे य॒ज्ञ्प॑तिः । \newline
2. य॒ज्ञ्प॑ति॒मिति॑ य॒ज्ञ् - प॒ति॒म् । \newline
3. दु॒हे य॒ज्ञ्प॑तिर् य॒ज्ञ्प॑तिर् दु॒हे दु॒हे य॒ज्ञ्प॑तिर् वा वा य॒ज्ञ्प॑तिर् दु॒हे दु॒हे य॒ज्ञ्प॑तिर् वा । \newline
4. य॒ज्ञ्प॑तिर् वा वा य॒ज्ञ्प॑तिर् य॒ज्ञ्प॑तिर् वा य॒ज्ञ्ं ॅय॒ज्ञ्ं ॅवा॑ य॒ज्ञ्प॑तिर् य॒ज्ञ्प॑तिर् वा य॒ज्ञ्म् । \newline
5. य॒ज्ञ्प॑ति॒रिति॑ य॒ज्ञ् - प॒तिः॒ । \newline
6. वा॒ य॒ज्ञ्ं ॅय॒ज्ञ्ं ॅवा॑ वा य॒ज्ञ्म् दु॑हे दुहे य॒ज्ञ्ं ॅवा॑ वा य॒ज्ञ्म् दु॑हे । \newline
7. य॒ज्ञ्म् दु॑हे दुहे य॒ज्ञ्ं ॅय॒ज्ञ्म् दु॑हे॒ स स दु॑हे य॒ज्ञ्ं ॅय॒ज्ञ्म् दु॑हे॒ सः । \newline
8. दु॒हे॒ स स दु॑हे दुहे॒ स यो यः स दु॑हे दुहे॒ स यः । \newline
9. स यो यः स स यः स्तु॑तश॒स्त्रयोः᳚ स्तुतश॒स्त्रयो॒र् यः स स यः स्तु॑तश॒स्त्रयोः᳚ । \newline
10. यः स्तु॑तश॒स्त्रयोः᳚ स्तुतश॒स्त्रयो॒र् यो यः स्तु॑तश॒स्त्रयो॒र् दोह॒म् दोहꣳ॑ स्तुतश॒स्त्रयो॒र् यो यः स्तु॑तश॒स्त्रयो॒र् दोह᳚म् । \newline
11. स्तु॒त॒श॒स्त्रयो॒र् दोह॒म् दोहꣳ॑ स्तुतश॒स्त्रयोः᳚ स्तुतश॒स्त्रयो॒र् दोह॒ मवि॑द्वा॒ नवि॑द्वा॒न् दोहꣳ॑ स्तुतश॒स्त्रयोः᳚ स्तुतश॒स्त्रयो॒र् दोह॒ मवि॑द्वान् । \newline
12. स्तु॒त॒श॒स्त्रयो॒रिति॑ स्तुत - श॒स्त्रयोः᳚ । \newline
13. दोह॒ मवि॑द्वा॒ नवि॑द्वा॒न् दोह॒म् दोह॒ मवि॑द्वा॒न्॒. यज॑ते॒ यज॒ते ऽवि॑द्वा॒न् दोह॒म् दोह॒ मवि॑द्वा॒न्॒. यज॑ते । \newline
14. अवि॑द्वा॒न्॒. यज॑ते॒ यज॒ते ऽवि॑द्वा॒ नवि॑द्वा॒न्॒. यज॑ते॒ तम् तं ॅयज॒ते ऽवि॑द्वा॒ नवि॑द्वा॒न्॒. यज॑ते॒ तम् । \newline
15. यज॑ते॒ तम् तं ॅयज॑ते॒ यज॑ते॒ तं ॅय॒ज्ञो य॒ज्ञ् स्तं ॅयज॑ते॒ यज॑ते॒ तं ॅय॒ज्ञ्ः । \newline
16. तं ॅय॒ज्ञो य॒ज्ञ्स्तम् तं ॅय॒ज्ञो दु॑हे दुहे य॒ज्ञ्स्तम् तं ॅय॒ज्ञो दु॑हे । \newline
17. य॒ज्ञो दु॑हे दुहे य॒ज्ञो य॒ज्ञो दु॑हे॒ स स दु॑हे य॒ज्ञो य॒ज्ञो दु॑हे॒ सः । \newline
18. दु॒हे॒ स स दु॑हे दुहे॒ स इ॒ष्ट्वे ष्ट्वा स दु॑हे दुहे॒ स इ॒ष्ट्वा । \newline
19. स इ॒ष्ट्वे ष्ट्वा स स इ॒ष्ट्वा पापी॑या॒न् पापी॑या नि॒ष्ट्वा स स इ॒ष्ट्वा पापी॑यान् । \newline
20. इ॒ष्ट्वा पापी॑या॒न् पापी॑या नि॒ष्ट्वे ष्ट्वा पापी॑यान् भवति भवति॒ पापी॑या नि॒ष्ट्वे ष्ट्वा पापी॑यान् भवति । \newline
21. पापी॑यान् भवति भवति॒ पापी॑या॒न् पापी॑यान् भवति॒ यो यो भ॑वति॒ पापी॑या॒न् पापी॑यान् भवति॒ यः । \newline
22. भ॒व॒ति॒ यो यो भ॑वति भवति॒ य ए॑नयो रेनयो॒र् यो भ॑वति भवति॒ य ए॑नयोः । \newline
23. य ए॑नयो रेनयो॒र् यो य ए॑नयो॒र् दोह॒म् दोह॑ मेनयो॒र् यो य ए॑नयो॒र् दोह᳚म् । \newline
24. ए॒न॒यो॒र् दोह॒म् दोह॑ मेनयो रेनयो॒र् दोहं॑ ॅवि॒द्वान्. वि॒द्वान् दोह॑ मेनयो रेनयो॒र् दोहं॑ ॅवि॒द्वान् । \newline
25. दोहं॑ ॅवि॒द्वान्. वि॒द्वान् दोह॒म् दोहं॑ ॅवि॒द्वान्. यज॑ते॒ यज॑ते वि॒द्वान् दोह॒म् दोहं॑ ॅवि॒द्वान्. यज॑ते । \newline
26. वि॒द्वान्. यज॑ते॒ यज॑ते वि॒द्वान्. वि॒द्वान्. यज॑ते॒ स स यज॑ते वि॒द्वान्. वि॒द्वान्. यज॑ते॒ सः । \newline
27. यज॑ते॒ स स यज॑ते॒ यज॑ते॒ स य॒ज्ञ्ं ॅय॒ज्ञ्ꣳ स यज॑ते॒ यज॑ते॒ स य॒ज्ञ्म् । \newline
28. स य॒ज्ञ्ं ॅय॒ज्ञ्ꣳ स स य॒ज्ञ्म् दु॑हे दुहे य॒ज्ञ्ꣳ स स य॒ज्ञ्म् दु॑हे । \newline
29. य॒ज्ञ्म् दु॑हे दुहे य॒ज्ञ्ं ॅय॒ज्ञ्म् दु॑हे॒ स स दु॑हे य॒ज्ञ्ं ॅय॒ज्ञ्म् दु॑हे॒ सः । \newline
30. दु॒हे॒ स स दु॑हे दुहे॒ स इ॒ष्ट्वे ष्ट्वा स दु॑हे दुहे॒ स इ॒ष्ट्वा । \newline
31. स इ॒ष्ट्वे ष्ट्वा स स इ॒ष्ट्वा वसी॑या॒न्॒. वसी॑या नि॒ष्ट्वा स स इ॒ष्ट्वा वसी॑यान् । \newline
32. इ॒ष्ट्वा वसी॑या॒न्॒. वसी॑या नि॒ष्ट्वे ष्ट्वा वसी॑यान् भवति भवति॒ वसी॑या नि॒ष्ट्वे ष्ट्वा वसी॑यान् भवति । \newline
33. वसी॑यान् भवति भवति॒ वसी॑या॒न्॒. वसी॑यान् भवति स्तु॒तस्य॑ स्तु॒तस्य॑ भवति॒ वसी॑या॒न्॒. वसी॑यान् भवति स्तु॒तस्य॑ । \newline
34. भ॒व॒ति॒ स्तु॒तस्य॑ स्तु॒तस्य॑ भवति भवति स्तु॒तस्य॑ स्तु॒तꣳ स्तु॒तꣳ स्तु॒तस्य॑ भवति भवति स्तु॒तस्य॑ स्तु॒तम् । \newline
35. स्तु॒तस्य॑ स्तु॒तꣳ स्तु॒तꣳ स्तु॒तस्य॑ स्तु॒तस्य॑ स्तु॒त म॑स्यसि स्तु॒तꣳ स्तु॒तस्य॑ स्तु॒तस्य॑ स्तु॒त म॑सि । \newline
36. स्तु॒त म॑स्यसि स्तु॒तꣳ स्तु॒त म॒स्यूर्ज॒ मूर्ज॑ मसि स्तु॒तꣳ स्तु॒त म॒स्यूर्ज᳚म् । \newline
37. अ॒स्यूर्ज॒ मूर्ज॑ मस्य॒ स्यूर्ज॒म् मह्य॒म् मह्य॒ मूर्ज॑ मस्य॒ स्यूर्ज॒म् मह्य᳚म् । \newline
38. ऊर्ज॒म् मह्य॒म् मह्य॒ मूर्ज॒ मूर्ज॒म् मह्यꣳ॑ स्तु॒तꣳ स्तु॒तम् मह्य॒ मूर्ज॒ मूर्ज॒म् मह्यꣳ॑ स्तु॒तम् । \newline
39. मह्यꣳ॑ स्तु॒तꣳ स्तु॒तम् मह्य॒म् मह्यꣳ॑ स्तु॒तम् दु॑हाम् दुहाꣳ स्तु॒तम् मह्य॒म् मह्यꣳ॑ स्तु॒तम् दु॑हाम् । \newline
40. स्तु॒तम् दु॑हाम् दुहाꣳ स्तु॒तꣳ स्तु॒तम् दु॑हा॒ मा दु॑हाꣳ स्तु॒तꣳ स्तु॒तम् दु॑हा॒ मा । \newline
41. दु॒हा॒ मा दु॑हाम् दुहा॒ मा मा॒ मा ऽऽदु॑हाम् दुहा॒ मा मा᳚ । \newline
42. आ मा॒ मा ऽऽमा᳚ स्तु॒तस्य॑ स्तु॒तस्य॒ मा ऽऽमा᳚ स्तु॒तस्य॑ । \newline
43. मा॒ स्तु॒तस्य॑ स्तु॒तस्य॑ मा मा स्तु॒तस्य॑ स्तु॒तꣳ स्तु॒तꣳ स्तु॒तस्य॑ मा मा स्तु॒तस्य॑ स्तु॒तम् । \newline
44. स्तु॒तस्य॑ स्तु॒तꣳ स्तु॒तꣳ स्तु॒तस्य॑ स्तु॒तस्य॑ स्तु॒तम् ग॑म्याद् गम्याथ् स्तु॒तꣳ स्तु॒तस्य॑ स्तु॒तस्य॑ स्तु॒तम् ग॑म्यात् । \newline
45. स्तु॒तम् ग॑म्याद् गम्याथ् स्तु॒तꣳ स्तु॒तम् ग॑म्या च्छ॒स्त्रस्य॑ श॒स्त्रस्य॑ गम्याथ् स्तु॒तꣳ स्तु॒तम् ग॑म्या च्छ॒स्त्रस्य॑ । \newline
46. ग॒म्या॒ च्छ॒स्त्रस्य॑ श॒स्त्रस्य॑ गम्याद् गम्या च्छ॒स्त्रस्य॑ श॒स्त्रꣳ श॒स्त्रꣳ श॒स्त्रस्य॑ गम्याद् गम्या च्छ॒स्त्रस्य॑ श॒स्त्रम् । \newline
47. श॒स्त्रस्य॑ श॒स्त्रꣳ श॒स्त्रꣳ श॒स्त्रस्य॑ श॒स्त्रस्य॑ श॒स्त्र म॑स्यसि श॒स्त्रꣳ श॒स्त्रस्य॑ श॒स्त्रस्य॑ श॒स्त्र म॑सि । \newline
48. श॒स्त्र म॑स्यसि श॒स्त्रꣳ श॒स्त्र म॒स्यूर्ज॒ मूर्ज॑ मसि श॒स्त्रꣳ श॒स्त्र म॒स्यूर्ज᳚म् । \newline
49. अ॒स्यूर्ज॒ मूर्ज॑ मस्य॒ स्यूर्ज॒म् मह्य॒म् मह्य॒ मूर्ज॑ मस्य॒ स्यूर्ज॒म् मह्य᳚म् । \newline
50. ऊर्ज॒म् मह्य॒म् मह्य॒ मूर्ज॒ मूर्ज॒म् मह्यꣳ॑ श॒स्त्रꣳ श॒स्त्रम् मह्य॒ मूर्ज॒ मूर्ज॒म् मह्यꣳ॑ श॒स्त्रम् । \newline
51. मह्यꣳ॑ श॒स्त्रꣳ श॒स्त्रम् मह्य॒म् मह्यꣳ॑ श॒स्त्रम् दु॑हाम् दुहाꣳ श॒स्त्रम् मह्य॒म् मह्यꣳ॑ श॒स्त्रम् दु॑हाम् । \newline
52. श॒स्त्रम् दु॑हाम् दुहाꣳ श॒स्त्रꣳ श॒स्त्रम् दु॑हा॒ मा दु॑हाꣳ श॒स्त्रꣳ श॒स्त्रम् दु॑हा॒ मा । \newline
53. दु॒हा॒ मा दु॑हाम् दुहा॒ मा मा॒ मा ऽऽदु॑हाम् दुहा॒ मा मा᳚ । \newline
54. आ मा॒ मा ऽऽमा॑ श॒स्त्रस्य॑ श॒स्त्रस्य॒ मा ऽऽमा॑ श॒स्त्रस्य॑ । \newline
55. मा॒ श॒स्त्रस्य॑ श॒स्त्रस्य॑ मा मा श॒स्त्रस्य॑ श॒स्त्रꣳ श॒स्त्रꣳ श॒स्त्रस्य॑ मा मा श॒स्त्रस्य॑ श॒स्त्रम् । \newline
56. श॒स्त्रस्य॑ श॒स्त्रꣳ श॒स्त्रꣳ श॒स्त्रस्य॑ श॒स्त्रस्य॑ श॒स्त्रम् ग॑म्याद् गम्या च्छ॒स्त्रꣳ श॒स्त्रस्य॑ श॒स्त्रस्य॑ श॒स्त्रम् ग॑म्यात् । \newline
57. श॒स्त्रम् ग॑म्याद् गम्या च्छ॒स्त्रꣳ श॒स्त्रम् ग॑म्या॒ दितीति॑ गम्या च्छ॒स्त्रꣳ श॒स्त्रम् ग॑म्या॒ दिति॑ । \newline
58. ग॒म्या॒ दितीति॑ गम्याद् गम्या॒ दित्या॑हा॒हे ति॑ गम्याद् गम्या॒ दित्या॑ह । \newline
59. इत्या॑हा॒हे तीत्या॑है॒ष ए॒ष आ॒हे तीत्या॑है॒षः । \newline
60. आ॒है॒ष ए॒ष आ॑हाहै॒ष वै वा ए॒ष आ॑हाहै॒ष वै । \newline
61. ए॒ष वै वा ए॒ष ए॒ष वै स्तु॑तश॒स्त्रयोः᳚ स्तुतश॒स्त्रयो॒र् वा ए॒ष ए॒ष वै स्तु॑तश॒स्त्रयोः᳚ । \newline
62. वै स्तु॑तश॒स्त्रयोः᳚ स्तुतश॒स्त्रयो॒र् वै वै स्तु॑तश॒स्त्रयो॒र् दोहो॒ दोहः॑ स्तुतश॒स्त्रयो॒र् वै वै स्तु॑तश॒स्त्रयो॒र् दोहः॑ । \newline
63. स्तु॒त॒श॒स्त्रयो॒र् दोहो॒ दोहः॑ स्तुतश॒स्त्रयोः᳚ स्तुतश॒स्त्रयो॒र् दोह॒ स्तम् तम् दोहः॑ स्तुतश॒स्त्रयोः᳚ स्तुतश॒स्त्रयो॒र् दोह॒ स्तम् । \newline
64. स्तु॒त॒श॒स्त्रयो॒रिति॑ स्तुत - श॒स्त्रयोः᳚ । \newline
65. दोह॒ स्तम् तम् दोहो॒ दोह॒ स्तं ॅयो यस्तम् दोहो॒ दोह॒ स्तं ॅयः । \newline
66. तं ॅयो य स्तम् तं ॅय ए॒व मे॒वं ॅय स्तम् तं ॅय ए॒वम् । \newline
67. य ए॒व मे॒वं ॅयो य ए॒वं ॅवि॒द्वान्. वि॒द्वा ने॒वं ॅयो य ए॒वं ॅवि॒द्वान् । \newline
68. ए॒वं ॅवि॒द्वान्. वि॒द्वा ने॒व मे॒वं ॅवि॒द्वान्. यज॑ते॒ यज॑ते वि॒द्वा ने॒व मे॒वं ॅवि॒द्वान्. यज॑ते । \newline
69. वि॒द्वान्. यज॑ते॒ यज॑ते वि॒द्वान्. वि॒द्वान्. यज॑ते दु॒हे दु॒हे यज॑ते वि॒द्वान्. वि॒द्वान्. यज॑ते दु॒हे । \newline
70. यज॑ते दु॒हे दु॒हे यज॑ते॒ यज॑ते दु॒ह ए॒वैव दु॒हे यज॑ते॒ यज॑ते दु॒ह ए॒व । \newline
71. दु॒ह ए॒वैव दु॒हे दु॒ह ए॒व य॒ज्ञ्ं ॅय॒ज्ञ् मे॒व दु॒हे दु॒ह ए॒व य॒ज्ञ्म् । \newline
72. ए॒व य॒ज्ञ्ं ॅय॒ज्ञ् मे॒वैव य॒ज्ञ् मि॒ष्ट्वे ष्ट्वा य॒ज्ञ् मे॒वैव य॒ज्ञ् मि॒ष्ट्वा । \newline
73. य॒ज्ञ् मि॒ष्ट्वे ष्ट्वा य॒ज्ञ्ं ॅय॒ज्ञ् मि॒ष्ट्वा वसी॑या॒न्॒. वसी॑या नि॒ष्ट्वा य॒ज्ञ्ं ॅय॒ज्ञ् मि॒ष्ट्वा वसी॑यान् । \newline
74. इ॒ष्ट्वा वसी॑या॒न्॒. वसी॑या नि॒ष्ट्वे ष्ट्वा वसी॑यान् भवति भवति॒ वसी॑या नि॒ष्ट्वे ष्ट्वा वसी॑यान् भवति । \newline
75. वसी॑यान् भवति भवति॒ वसी॑या॒न्॒. वसी॑यान् भवति । \newline
76. भ॒व॒तीति॑ भवति । \newline
\pagebreak
\markright{ TS 3.2.8.1  \hfill https://www.vedavms.in \hfill}

\section{ TS 3.2.8.1 }

\textbf{TS 3.2.8.1 } \newline
\textbf{Samhita Paata} \newline

श्ये॒नाय॒ पत्व॑ने॒ स्वाहा॒ वट्थ्स्व॒यम॑भिगूर्ताय॒ नमो॑ विष्ट॒म्भाय॒ धर्म॑णे॒ स्वाहा॒ वट्थ्स्व॒यम॑भिगूर्ताय॒ नमः॑ परि॒धये॑ जन॒प्रथ॑नाय॒ स्वाहा॒ वट्थ्स्व॒यम॑भिगूर्ताय॒ नम॑ ऊ॒र्जे होत्रा॑णाꣳ॒॒ स्वाहा॒ वट्थ्स्व॒यम॑भिगूर्ताय॒ नमः॒ पय॑से॒ होत्रा॑णाꣳ॒॒ स्वाहा॒ वट्थ्स्व॒यम॑भिगूर्ताय॒ नमः॑ प्र॒जाप॑तये॒ मन॑वे॒ स्वाहा॒ वट्थ्स्व॒यम॑भिगूर्ताय॒ नम॑ ऋ॒तमृ॑तपाः सुवर्वा॒ट्थ्स्वाहा॒ वट्थ्स्व॒यम॑भिगूर्ताय॒ नम॑स्तृं॒पन्ताꣳ॒॒ होत्रा॒ मधो᳚र्घृ॒तस्य॑ य॒ज्ञ्प॑ति॒मृष॑य॒ एन॑सा - [  ] \newline

\textbf{Pada Paata} \newline

श्ये॒नाय॑ । पत्व॑ने । स्वाहा᳚ । वट् । स्व॒यम॑भिगूर्ता॒येति॑ स्व॒यं - अ॒भि॒गू॒र्ता॒य॒ । नमः॑ । वि॒ष्ट॒भांयेति॑ वि - स्त॒भांय॑ । धर्म॑णे । स्वाहा᳚ । वट् । स्व॒यम॑भिगूर्ता॒येति॑ स्व॒यं - अ॒भि॒गू॒र्ता॒य॒ । नमः॑ । प॒रि॒धय॒ इति॑ परि - धये᳚ । ज॒न॒प्रथ॑ना॒येति॑ जन - प्रथ॑नाय । स्वाहा᳚ । वट् । स्व॒यम॑भिगूर्ता॒येति॑ स्व॒यं - अ॒भि॒गू॒र्ता॒य॒ । नमः॑ । ऊ॒र्जे । होत्रा॑णाम् । स्वाहा᳚ । वट् । स्व॒यम॑भिगूर्ता॒येति॑ स्व॒यं - अ॒भि॒गू॒र्ता॒य॒ । नमः॑ । पय॑से । होत्रा॑णाम् । स्वाहा᳚ । वट् । स्व॒यम॑भिगूर्ता॒येति॑ स्व॒यं - अ॒भि॒गू॒र्ता॒य॒ । नमः॑ । प्र॒जाप॑तय॒ इति॑ प्र॒जा-प॒त॒ये॒ । मन॑वे । स्वाहा᳚ । वट् । स्व॒यम॑भिगूर्ता॒येति॑ स्व॒यं - अ॒भि॒गू॒र्ता॒य॒ । नमः॑ । ऋ॒तम् । ऋ॒त॒पा॒ इत्यृ॑त - पाः॒ । सु॒व॒र्वा॒डिति॑ सुवः - वा॒ट् । स्वाहा᳚ । वट् । स्व॒यम॑भिगूर्ता॒येति॑ स्व॒यं - अ॒भि॒गू॒र्ता॒य॒ । नमः॑ । तृ॒पंन्ता᳚म् । होत्राः᳚ । मधोः᳚ । घृ॒तस्य॑ । य॒ज्ञ्प॑ति॒मिति॑ य॒ज्ञ् - प॒ति॒म् । ऋष॑यः । एन॑सा ।  \newline


\textbf{Krama Paata} \newline

श्ये॒नाय॒ पत्व॑ने । पत्व॑ने॒ स्वाहा᳚ । स्वाहा॒ वट् । वट्थ् स्व॒यम॑भिगूर्ताय । स्व॒यम॑भिगूर्ताय॒ नमः॑ । स्व॒यम॑भिगूर्ता॒येति॑ स्व॒यम् - अ॒भि॒गू॒र्ता॒य॒ । नमो॑ विष्ट॒म्भाय॑ । वि॒ष्ट॒म्भाय॒ धर्म॑णे । वि॒ष्ट॒म्भायेति॑ वि - स्त॒म्भाय॑ । धर्म॑णे॒ स्वाहा᳚ । स्वाहा॒ वट् । वट्थ् स्व॒यम॑भिगूर्ताय । स्व॒यम॑भिगूर्ताय॒ नमः॑ । स्व॒यम॑भिगूर्ता॒येति॑ स्व॒यम् - अ॒भि॒गू॒र्ता॒य॒ । नमः॑ परि॒धये᳚ । प॒रि॒धये॑ जन॒प्रथ॑नाय । प॒रि॒धय॒ इति॑ परि - धये᳚ । ज॒न॒प्रथ॑नाय॒ स्वाहा᳚ । ज॒न॒प्रथ॑ना॒येति॑ जन - प्रथ॑नाय । स्वाहा॒ वट् । वट्थ् स्व॒यम॑भिगूर्ताय । स्व॒यम॑भिगूर्ताय॒ नमः॑ । स्व॒यम॑भिगूर्ता॒येति॑ स्व॒यम् - अ॒भि॒गू॒र्ता॒य॒ । नम॑ ऊ॒र्जे । ऊ॒र्जे होत्रा॑णाम् । होत्रा॑णाꣳ॒॒ स्वाहा᳚ । स्वाहा॒ वट् । वट्थ् स्व॒यम॑भिगूर्ताय । स्व॒यम॑भिगूर्ताय॒ नमः॑ । स्व॒यम॑भिगूर्ता॒येति॑ स्व॒यम् - अ॒भि॒गू॒र्ता॒य॒ । नमः॒ पय॑से । पय॑से॒ होत्रा॑णाम् । होत्रा॑णाꣳ॒॒ स्वाहा᳚ । स्वाहा॒ वट् । वट्थ् स्व॒यम॑भिगूर्ताय । स्व॒यम॑भिगूर्ताय॒ नमः॑ । स्व॒यम॑भिगूर्ता॒येति॑ स्व॒यम् - अ॒भि॒गू॒र्ता॒य॒ । नमः॑ प्र॒जाप॑तये । प्र॒जाप॑तये॒ मन॑वे । प्र॒जाप॑तय॒ इति॑ प्र॒जा - प॒त॒ये॒ । मन॑वे॒ स्वाहा᳚ । स्वाहा॒ वट् । वट्थ् स्व॒यम॑भिगूर्ताय । स्व॒यम॑भिगूर्ताय॒ नमः॑ । स्व॒यम॑भिगूर्ता॒येति॑ स्व॒यम् - अ॒भि॒गू॒र्ता॒य॒ । नम॑ ऋ॒तम् । ऋ॒तमृ॑तपाः । ऋ॒त॒पाः॒ सु॒व॒र्वा॒ट्॒ । ऋ॒त॒पा॒ इत्यृ॑त - पाः॒ । सु॒व॒र्वा॒ट्थ् स्वाहा᳚ । सु॒व॒र्वा॒डिति॑ सुवः - वा॒ट्॒ । स्वाहा॒ वट् । वट्थ् स्व॒यम॑भिगूर्ताय । स्व॒यम॑भिगूर्ताय॒ नमः॑ । स्व॒यम॑भिगूर्ता॒येति॑ स्व॒यम् - अ॒भि॒गू॒र्ता॒य॒ । नम॑ स्तृ॒म्पन्ता᳚म् । तृ॒म्पताꣳ॒॒ होत्राः᳚ । होत्रा॒ मधोः᳚ । मधो᳚र् घृ॒तस्य॑ । घृ॒तस्य॑ य॒ज्ञ्प॑तिम् । य॒ज्ञ्प॑ति॒मृष॑यः । य॒ज्ञ्प॑ति॒मिति॑ य॒ज्ञ् - प॒ति॒म् । ऋष॑य॒ एन॑सा । एन॑सा ऽऽहुः \newline

\textbf{Jatai Paata} \newline

1. श्ये॒नाय॒ पत्व॑ने॒ पत्व॑ने श्ये॒नाय॑ श्ये॒नाय॒ पत्व॑ने । \newline
2. पत्व॑ने॒ स्वाहा॒ स्वाहा॒ पत्व॑ने॒ पत्व॑ने॒ स्वाहा᳚ । \newline
3. स्वाहा॒ वड् वट् थ्स्वाहा॒ स्वाहा॒ वट् । \newline
4. वट् थ्स्व॒यम॑भिगूर्ताय स्व॒यम॑भिगूर्ताय॒ वड् वट् थ्स्व॒यम॑भिगूर्ताय । \newline
5. स्व॒यम॑भिगूर्ताय॒ नमो॒ नमः॑ स्व॒यम॑भिगूर्ताय स्व॒यम॑भिगूर्ताय॒ नमः॑ । \newline
6. स्व॒यम॑भिगूर्ता॒येति॑ स्व॒यं - अ॒भि॒गू॒र्ता॒य॒ । \newline
7. नमो॑ विष्टं॒भाय॑ विष्टं॒भाय॒ नमो॒ नमो॑ विष्टं॒भाय॑ । \newline
8. वि॒ष्टं॒भाय॒ धर्म॑णे॒ धर्म॑णे विष्टं॒भाय॑ विष्टं॒भाय॒ धर्म॑णे । \newline
9. वि॒ष्टं॒भायेति॑ वि - स्तं॒भाय॑ । \newline
10. धर्म॑णे॒ स्वाहा॒ स्वाहा॒ धर्म॑णे॒ धर्म॑णे॒ स्वाहा᳚ । \newline
11. स्वाहा॒ वड् वट् थ्स्वाहा॒ स्वाहा॒ वट् । \newline
12. वट् थ्स्व॒यम॑भिगूर्ताय स्व॒यम॑भिगूर्ताय॒ वड् वट् थ्स्व॒यम॑भिगूर्ताय । \newline
13. स्व॒यम॑भिगूर्ताय॒ नमो॒ नमः॑ स्व॒यम॑भिगूर्ताय स्व॒यम॑भिगूर्ताय॒ नमः॑ । \newline
14. स्व॒यम॑भिगूर्ता॒येति॑ स्व॒यं - अ॒भि॒गू॒र्ता॒य॒ । \newline
15. नमः॑ परि॒धये॑ परि॒धये॒ नमो॒ नमः॑ परि॒धये᳚ । \newline
16. प॒रि॒धये॑ जन॒प्रथ॑नाय जन॒प्रथ॑नाय परि॒धये॑ परि॒धये॑ जन॒प्रथ॑नाय । \newline
17. प॒रि॒धय॒ इति॑ परि - धये᳚ । \newline
18. ज॒न॒प्रथ॑नाय॒ स्वाहा॒ स्वाहा॑ जन॒प्रथ॑नाय जन॒प्रथ॑नाय॒ स्वाहा᳚ । \newline
19. ज॒न॒प्रथ॑ना॒येति॑ जन - प्रथ॑नाय । \newline
20. स्वाहा॒ वड् वट् थ्स्वाहा॒ स्वाहा॒ वट् । \newline
21. वट् थ्स्व॒यम॑भिगूर्ताय स्व॒यम॑भिगूर्ताय॒ वड् वट् थ्स्व॒यम॑भिगूर्ताय । \newline
22. स्व॒यम॑भिगूर्ताय॒ नमो॒ नमः॑ स्व॒यम॑भिगूर्ताय स्व॒यम॑भिगूर्ताय॒ नमः॑ । \newline
23. स्व॒यम॑भिगूर्ता॒येति॑ स्व॒यं - अ॒भि॒गू॒र्ता॒य॒ । \newline
24. नम॑ ऊ॒र्ज ऊ॒र्जे नमो॒ नम॑ ऊ॒र्जे । \newline
25. ऊ॒र्जे होत्रा॑णाꣳ॒॒ होत्रा॑णा मू॒र्ज ऊ॒र्जे होत्रा॑णाम् । \newline
26. होत्रा॑णाꣳ॒॒ स्वाहा॒ स्वाहा॒ होत्रा॑णाꣳ॒॒ होत्रा॑णाꣳ॒॒ स्वाहा᳚ । \newline
27. स्वाहा॒ वड् वट् थ्स्वाहा॒ स्वाहा॒ वट् । \newline
28. वट् थ्स्व॒यम॑भिगूर्ताय स्व॒यम॑भिगूर्ताय॒ वड् वट् थ्स्व॒यम॑भिगूर्ताय । \newline
29. स्व॒यम॑भिगूर्ताय॒ नमो॒ नमः॑ स्व॒यम॑भिगूर्ताय स्व॒यम॑भिगूर्ताय॒ नमः॑ । \newline
30. स्व॒यम॑भिगूर्ता॒येति॑ स्व॒यं - अ॒भि॒गू॒र्ता॒य॒ । \newline
31. नमः॒ पय॑से॒ पय॑से॒ नमो॒ नमः॒ पय॑से । \newline
32. पय॑से॒ होत्रा॑णाꣳ॒॒ होत्रा॑णा॒म् पय॑से॒ पय॑से॒ होत्रा॑णाम् । \newline
33. होत्रा॑णाꣳ॒॒ स्वाहा॒ स्वाहा॒ होत्रा॑णाꣳ॒॒ होत्रा॑णाꣳ॒॒ स्वाहा᳚ । \newline
34. स्वाहा॒ वड् वट् थ्स्वाहा॒ स्वाहा॒ वट् । \newline
35. वट् थ्स्व॒यम॑भिगूर्ताय स्व॒यम॑भिगूर्ताय॒ वड् वट् थ्स्व॒यम॑भिगूर्ताय । \newline
36. स्व॒यम॑भिगूर्ताय॒ नमो॒ नमः॑ स्व॒यम॑भिगूर्ताय स्व॒यम॑भिगूर्ताय॒ नमः॑ । \newline
37. स्व॒यम॑भिगूर्ता॒येति॑ स्व॒यं - अ॒भि॒गू॒र्ता॒य॒ । \newline
38. नमः॑ प्र॒जाप॑तये प्र॒जाप॑तये॒ नमो॒ नमः॑ प्र॒जाप॑तये । \newline
39. प्र॒जाप॑तये॒ मन॑वे॒ मन॑वे प्र॒जाप॑तये प्र॒जाप॑तये॒ मन॑वे । \newline
40. प्र॒जाप॑तय॒ इति॑ प्र॒जा - प॒त॒ये॒ । \newline
41. मन॑वे॒ स्वाहा॒ स्वाहा॒ मन॑वे॒ मन॑वे॒ स्वाहा᳚ । \newline
42. स्वाहा॒ वड् वट् थ्स्वाहा॒ स्वाहा॒ वट् । \newline
43. वट् थ्स्व॒यम॑भिगूर्ताय स्व॒यम॑भिगूर्ताय॒ वड् वट् थ्स्व॒यम॑भिगूर्ताय । \newline
44. स्व॒यम॑भिगूर्ताय॒ नमो॒ नमः॑ स्व॒यम॑भिगूर्ताय स्व॒यम॑भिगूर्ताय॒ नमः॑ । \newline
45. स्व॒यम॑भिगूर्ता॒येति॑ स्व॒यं - अ॒भि॒गू॒र्ता॒य॒ । \newline
46. नम॑ ऋ॒त मृ॒तम् नमो॒ नम॑ ऋ॒तम् । \newline
47. ऋ॒त मृ॑तपा ऋतपा ऋ॒त मृ॒त मृ॑तपाः । \newline
48. ऋ॒त॒पाः॒ सु॒व॒र्वा॒ट् थ्सु॒व॒र्वा॒ डृ॒त॒पा॒ ऋ॒त॒पाः॒ सु॒व॒र्वा॒ट् । \newline
49. ऋ॒त॒पा॒ इत्यृ॑त - पाः॒ । \newline
50. सु॒व॒र्वा॒ट् थ्स्वाहा॒ स्वाहा॑ सुवर्वाट् थ्सुवर्वा॒ट् थ्स्वाहा᳚ । \newline
51. सु॒व॒र्वा॒डिति॑ सुवः - वा॒ट् । \newline
52. स्वाहा॒ वड् वट् थ्स्वाहा॒ स्वाहा॒ वट् । \newline
53. वट् थ्स्व॒यम॑भिगूर्ताय स्व॒यम॑भिगूर्ताय॒ वड् वट् थ्स्व॒यम॑भिगूर्ताय । \newline
54. स्व॒यम॑भिगूर्ताय॒ नमो॒ नमः॑ स्व॒यम॑भिगूर्ताय स्व॒यम॑भिगूर्ताय॒ नमः॑ । \newline
55. स्व॒यम॑भिगूर्ता॒येति॑ स्व॒यं - अ॒भि॒गू॒र्ता॒य॒ । \newline
56. नम॑ स्तृं॒पन्ता᳚म् तृं॒पन्ता॒म् नमो॒ नम॑ स्तृं॒पन्ता᳚म् । \newline
57. तृं॒पन्ताꣳ॒॒ होत्रा॒ होत्रा᳚ स्तृं॒पन्ता᳚म् तृं॒पन्ताꣳ॒॒ होत्राः᳚ । \newline
58. होत्रा॒ मधो॒र् मधो॒र्॒. होत्रा॒ होत्रा॒ मधोः᳚ । \newline
59. मधो᳚र् घृ॒तस्य॑ घृ॒तस्य॒ मधो॒र् मधो᳚र् घृ॒तस्य॑ । \newline
60. घृ॒तस्य॑ य॒ज्ञ्प॑तिं ॅय॒ज्ञ्प॑तिम् घृ॒तस्य॑ घृ॒तस्य॑ य॒ज्ञ्प॑तिम् । \newline
61. य॒ज्ञ्प॑ति॒ मृष॑य॒ ऋष॑यो य॒ज्ञ्प॑तिं ॅय॒ज्ञ्प॑ति॒ मृष॑यः । \newline
62. य॒ज्ञ्प॑ति॒मिति॑ य॒ज्ञ् - प॒ति॒म् । \newline
63. ऋष॑य॒ एन॒ सैन॒स र्.ष॑य॒ ऋष॑य॒ एन॑सा । \newline
64. एन॑सा ऽऽहु राहु॒ रेन॒ सैन॑सा ऽऽहुः । \newline

\textbf{Ghana Paata } \newline

1. श्ये॒नाय॒ पत्व॑ने॒ पत्व॑ने श्ये॒नाय॑ श्ये॒नाय॒ पत्व॑ने॒ स्वाहा॒ स्वाहा॒ पत्व॑ने श्ये॒नाय॑ श्ये॒नाय॒ पत्व॑ने॒ स्वाहा᳚ । \newline
2. पत्व॑ने॒ स्वाहा॒ स्वाहा॒ पत्व॑ने॒ पत्व॑ने॒ स्वाहा॒ वड् वट् थ्स्वाहा॒ पत्व॑ने॒ पत्व॑ने॒ स्वाहा॒ वट् । \newline
3. स्वाहा॒ वड् वट् थ्स्वाहा॒ स्वाहा॒ वट् थ्स्व॒यम॑भिगूर्ताय स्व॒यम॑भिगूर्ताय॒ वट् थ्स्वाहा॒ स्वाहा॒ 
वट् थ्स्व॒यम॑भिगूर्ताय । \newline
4. वट् थ्स्व॒यम॑भिगूर्ताय स्व॒यम॑भिगूर्ताय॒ वड् वट् थ्स्व॒यम॑भिगूर्ताय॒ नमो॒ नमः॑ स्व॒यम॑भिगूर्ताय॒ वड् वट् थ्स्व॒यम॑भिगूर्ताय॒ नमः॑ । \newline
5. स्व॒यम॑भिगूर्ताय॒ नमो॒ नमः॑ स्व॒यम॑भिगूर्ताय स्व॒यम॑भिगूर्ताय॒ नमो॑ विष्टं॒भाय॑ विष्टं॒भाय॒ नमः॑ स्व॒यम॑भिगूर्ताय स्व॒यम॑भिगूर्ताय॒ नमो॑ विष्टं॒भाय॑ । \newline
6. स्व॒यम॑भिगूर्ता॒येति॑ स्व॒यं - अ॒भि॒गू॒र्ता॒य॒ । \newline
7. नमो॑ विष्टं॒भाय॑ विष्टं॒भाय॒ नमो॒ नमो॑ विष्टं॒भाय॒ धर्म॑णे॒ धर्म॑णे विष्टं॒भाय॒ नमो॒ नमो॑ विष्टं॒भाय॒ धर्म॑णे । \newline
8. वि॒ष्टं॒भाय॒ धर्म॑णे॒ धर्म॑णे विष्टं॒भाय॑ विष्टं॒भाय॒ धर्म॑णे॒ स्वाहा॒ स्वाहा॒ धर्म॑णे विष्टं॒भाय॑ विष्टं॒भाय॒ धर्म॑णे॒ स्वाहा᳚ । \newline
9. वि॒ष्टं॒भायेति॑ वि - स्तं॒भाय॑ । \newline
10. धर्म॑णे॒ स्वाहा॒ स्वाहा॒ धर्म॑णे॒ धर्म॑णे॒ स्वाहा॒ वड् वट् थ्स्वाहा॒ धर्म॑णे॒ धर्म॑णे॒ स्वाहा॒ वट् । \newline
11. स्वाहा॒ वड् वट् थ्स्वाहा॒ स्वाहा॒ वट् थ्स्व॒यम॑भिगूर्ताय स्व॒यम॑भिगूर्ताय॒ वट् थ्स्वाहा॒ स्वाहा॒ वट् थ्स्व॒यम॑भिगूर्ताय । \newline
12. वट् थ्स्व॒यम॑भिगूर्ताय स्व॒यम॑भिगूर्ताय॒ वड् वट् थ्स्व॒यम॑भिगूर्ताय॒ नमो॒ नमः॑ स्व॒यम॑भिगूर्ताय॒ वड् वट् थ्स्व॒यम॑भिगूर्ताय॒ नमः॑ । \newline
13. स्व॒यम॑भिगूर्ताय॒ नमो॒ नमः॑ स्व॒यम॑भिगूर्ताय स्व॒यम॑भिगूर्ताय॒ नमः॑ परि॒धये॑ परि॒धये॒ नमः॑ स्व॒यम॑भिगूर्ताय स्व॒यम॑भिगूर्ताय॒ नमः॑ परि॒धये᳚ । \newline
14. स्व॒यम॑भिगूर्ता॒येति॑ स्व॒यं - अ॒भि॒गू॒र्ता॒य॒ । \newline
15. नमः॑ परि॒धये॑ परि॒धये॒ नमो॒ नमः॑ परि॒धये॑ जन॒प्रथ॑नाय जन॒प्रथ॑नाय परि॒धये॒ नमो॒ नमः॑ परि॒धये॑ जन॒प्रथ॑नाय । \newline
16. प॒रि॒धये॑ जन॒प्रथ॑नाय जन॒प्रथ॑नाय परि॒धये॑ परि॒धये॑ जन॒प्रथ॑नाय॒ स्वाहा॒ स्वाहा॑ जन॒प्रथ॑नाय परि॒धये॑ परि॒धये॑ जन॒प्रथ॑नाय॒ स्वाहा᳚ । \newline
17. प॒रि॒धय॒ इति॑ परि - धये᳚ । \newline
18. ज॒न॒प्रथ॑नाय॒ स्वाहा॒ स्वाहा॑ जन॒प्रथ॑नाय जन॒प्रथ॑नाय॒ स्वाहा॒ वड् वट् थ्स्वाहा॑ जन॒प्रथ॑नाय जन॒प्रथ॑नाय॒ स्वाहा॒ वट् । \newline
19. ज॒न॒प्रथ॑ना॒येति॑ जन - प्रथ॑नाय । \newline
20. स्वाहा॒ वड् वट् थ्स्वाहा॒ स्वाहा॒ वट् थ्स्व॒यम॑भिगूर्ताय स्व॒यम॑भिगूर्ताय॒ वट् थ्स्वाहा॒ स्वाहा॒ वट् थ्स्व॒यम॑भिगूर्ताय । \newline
21. वट् थ्स्व॒यम॑भिगूर्ताय स्व॒यम॑भिगूर्ताय॒ वड् वट् थ्स्व॒यम॑भिगूर्ताय॒ नमो॒ नमः॑ स्व॒यम॑भिगूर्ताय॒ वड् वट् थ्स्व॒यम॑भिगूर्ताय॒ नमः॑ । \newline
22. स्व॒यम॑भिगूर्ताय॒ नमो॒ नमः॑ स्व॒यम॑भिगूर्ताय स्व॒यम॑भिगूर्ताय॒ नम॑ ऊ॒र्ज ऊ॒र्जे नमः॑ स्व॒यम॑भिगूर्ताय स्व॒यम॑भिगूर्ताय॒ नम॑ ऊ॒र्जे । \newline
23. स्व॒यम॑भिगूर्ता॒येति॑ स्व॒यं - अ॒भि॒गू॒र्ता॒य॒ । \newline
24. नम॑ ऊ॒र्ज ऊ॒र्जे नमो॒ नम॑ ऊ॒र्जे होत्रा॑णाꣳ॒॒ होत्रा॑णा मू॒र्जे नमो॒ नम॑ ऊ॒र्जे होत्रा॑णाम् । \newline
25. ऊ॒र्जे होत्रा॑णाꣳ॒॒ होत्रा॑णा मू॒र्ज ऊ॒र्जे होत्रा॑णाꣳ॒॒ स्वाहा॒ स्वाहा॒ होत्रा॑णा मू॒र्ज ऊ॒र्जे होत्रा॑णाꣳ॒॒ स्वाहा᳚ । \newline
26. होत्रा॑णाꣳ॒॒ स्वाहा॒ स्वाहा॒ होत्रा॑णाꣳ॒॒ होत्रा॑णाꣳ॒॒ स्वाहा॒ वड् वट् थ्स्वाहा॒ होत्रा॑णाꣳ॒॒ होत्रा॑णाꣳ॒॒ स्वाहा॒ वट् । \newline
27. स्वाहा॒ वड् वट् थ्स्वाहा॒ स्वाहा॒ वट् थ्स्व॒यम॑भिगूर्ताय स्व॒यम॑भिगूर्ताय॒ वट् थ्स्वाहा॒ स्वाहा॒ वट् थ्स्व॒यम॑भिगूर्ताय । \newline
28. वट् थ्स्व॒यम॑भिगूर्ताय स्व॒यम॑भिगूर्ताय॒ वड् वट् थ्स्व॒यम॑भिगूर्ताय॒ नमो॒ नमः॑ स्व॒यम॑भिगूर्ताय॒ वड् वट् थ्स्व॒यम॑भिगूर्ताय॒ नमः॑ । \newline
29. स्व॒यम॑भिगूर्ताय॒ नमो॒ नमः॑ स्व॒यम॑भिगूर्ताय स्व॒यम॑भिगूर्ताय॒ नमः॒ पय॑से॒ पय॑से॒ नमः॑ स्व॒यम॑भिगूर्ताय स्व॒यम॑भिगूर्ताय॒ नमः॒ पय॑से । \newline
30. स्व॒यम॑भिगूर्ता॒येति॑ स्व॒यं - अ॒भि॒गू॒र्ता॒य॒ । \newline
31. नमः॒ पय॑से॒ पय॑से॒ नमो॒ नमः॒ पय॑से॒ होत्रा॑णाꣳ॒॒ होत्रा॑णा॒म् पय॑से॒ नमो॒ नमः॒ पय॑से॒ होत्रा॑णाम् । \newline
32. पय॑से॒ होत्रा॑णाꣳ॒॒ होत्रा॑णा॒म् पय॑से॒ पय॑से॒ होत्रा॑णाꣳ॒॒ स्वाहा॒ स्वाहा॒ होत्रा॑णा॒म् पय॑से॒ पय॑से॒ होत्रा॑णाꣳ॒॒ स्वाहा᳚ । \newline
33. होत्रा॑णाꣳ॒॒ स्वाहा॒ स्वाहा॒ होत्रा॑णाꣳ॒॒ होत्रा॑णाꣳ॒॒ स्वाहा॒ वड् वट् थ्स्वाहा॒ होत्रा॑णाꣳ॒॒ होत्रा॑णाꣳ॒॒ स्वाहा॒ वट् । \newline
34. स्वाहा॒ वड् वट् थ्स्वाहा॒ स्वाहा॒ वट् थ्स्व॒यम॑भिगूर्ताय स्व॒यम॑भिगूर्ताय॒ वट् थ्स्वाहा॒ स्वाहा॒ वट् थ्स्व॒यम॑भिगूर्ताय । \newline
35. वट् थ्स्व॒यम॑भिगूर्ताय स्व॒यम॑भिगूर्ताय॒ वड् वट् थ्स्व॒यम॑भिगूर्ताय॒ नमो॒ नमः॑ स्व॒यम॑भिगूर्ताय॒ वड् वट् थ्स्व॒यम॑भिगूर्ताय॒ नमः॑ । \newline
36. स्व॒यम॑भिगूर्ताय॒ नमो॒ नमः॑ स्व॒यम॑भिगूर्ताय स्व॒यम॑भिगूर्ताय॒ नमः॑ प्र॒जाप॑तये प्र॒जाप॑तये॒ नमः॑ स्व॒यम॑भिगूर्ताय स्व॒यम॑भिगूर्ताय॒ नमः॑ प्र॒जाप॑तये । \newline
37. स्व॒यम॑भिगूर्ता॒येति॑ स्व॒यं - अ॒भि॒गू॒र्ता॒य॒ । \newline
38. नमः॑ प्र॒जाप॑तये प्र॒जाप॑तये॒ नमो॒ नमः॑ प्र॒जाप॑तये॒ मन॑वे॒ मन॑वे प्र॒जाप॑तये॒ नमो॒ नमः॑ प्र॒जाप॑तये॒ मन॑वे । \newline
39. प्र॒जाप॑तये॒ मन॑वे॒ मन॑वे प्र॒जाप॑तये प्र॒जाप॑तये॒ मन॑वे॒ स्वाहा॒ स्वाहा॒ मन॑वे प्र॒जाप॑तये प्र॒जाप॑तये॒ मन॑वे॒ स्वाहा᳚ । \newline
40. प्र॒जाप॑तय॒ इति॑ प्र॒जा - प॒त॒ये॒ । \newline
41. मन॑वे॒ स्वाहा॒ स्वाहा॒ मन॑वे॒ मन॑वे॒ स्वाहा॒ वड् वट् थ्स्वाहा॒ मन॑वे॒ मन॑वे॒ स्वाहा॒ वट् । \newline
42. स्वाहा॒ वड् वट् थ्स्वाहा॒ स्वाहा॒ वट् थ्स्व॒यम॑भिगूर्ताय स्व॒यम॑भिगूर्ताय॒ वट् थ्स्वाहा॒ स्वाहा॒ वट् थ्स्व॒यम॑भिगूर्ताय । \newline
43. वट् थ्स्व॒यम॑भिगूर्ताय स्व॒यम॑भिगूर्ताय॒ वड् वट् थ्स्व॒यम॑भिगूर्ताय॒ नमो॒ नमः॑ स्व॒यम॑भिगूर्ताय॒ वड् वट् थ्स्व॒यम॑भिगूर्ताय॒ नमः॑ । \newline
44. स्व॒यम॑भिगूर्ताय॒ नमो॒ नमः॑ स्व॒यम॑भिगूर्ताय स्व॒यम॑भिगूर्ताय॒ नम॑ ऋ॒त मृ॒तम् नमः॑ स्व॒यम॑भिगूर्ताय स्व॒यम॑भिगूर्ताय॒ नम॑ ऋ॒तम् । \newline
45. स्व॒यम॑भिगूर्ता॒येति॑ स्व॒यं - अ॒भि॒गू॒र्ता॒य॒ । \newline
46. नम॑ ऋ॒त मृ॒तम् नमो॒ नम॑ ऋ॒त मृ॑तपा ऋतपा ऋ॒तम् नमो॒ नम॑ ऋ॒त मृ॑तपाः । \newline
47. ऋ॒त मृ॑तपा ऋतपा ऋ॒त मृ॒त मृ॑तपाः सुवर्वाट् थ्सुवर्वा डृतपा ऋ॒त मृ॒त मृ॑तपाः सुवर्वाट् । \newline
48. ऋ॒त॒पाः॒ सु॒व॒र्वा॒ट् थ्सु॒व॒र्वा॒ डृ॒त॒पा॒ ऋ॒त॒पाः॒ सु॒व॒र्वा॒ट् थ्स्वाहा॒ स्वाहा॑ सुवर्वा डृतपा ऋतपाः सुवर्वा॒ट् थ्स्वाहा᳚ । \newline
49. ऋ॒त॒पा॒ इत्यृ॑त - पाः॒ । \newline
50. सु॒व॒र्वा॒ट् थ्स्वाहा॒ स्वाहा॑ सुवर्वाट् थ्सुवर्वा॒ट् थ्स्वाहा॒ वड् वट् थ्स्वाहा॑ सुवर्वाट् थ्सु॑वर्वा॒ट् थ्स्वाहा॒ वट् । \newline
51. सु॒व॒र्वा॒डिति॑ सुवः - वा॒ट् । \newline
52. स्वाहा॒ वड् वट् थ्स्वाहा॒ स्वाहा॒ वट् थ्स्व॒यम॑भिगूर्ताय स्व॒यम॑भिगूर्ताय॒ वट् थ्स्वाहा॒ स्वाहा॒ वट् थ्स्व॒यम॑भिगूर्ताय । \newline
53. वट् थ्स्व॒यम॑भिगूर्ताय स्व॒यम॑भिगूर्ताय॒ वड् वट् थ्स्व॒यम॑भिगूर्ताय॒ नमो॒ नमः॑ स्व॒यम॑भिगूर्ताय॒ वड् वट् थ्स्व॒यम॑भिगूर्ताय॒ नमः॑ । \newline
54. स्व॒यम॑भिगूर्ताय॒ नमो॒ नमः॑ स्व॒यम॑भिगूर्ताय स्व॒यम॑भिगूर्ताय॒ नम॑ स्तृं॒पन्ता᳚म् तृं॒पन्ता॒म् नमः॑ स्व॒यम॑भिगूर्ताय स्व॒यम॑भिगूर्ताय॒ नम॑ स्तृं॒पन्ता᳚म् । \newline
55. स्व॒यम॑भिगूर्ता॒येति॑ स्व॒यं - अ॒भि॒गू॒र्ता॒य॒ । \newline
56. नम॑ स्तृं॒पन्ता᳚म् तृं॒पन्ता॒म् नमो॒ नम॑ स्तृं॒पन्ताꣳ॒॒ होत्रा॒ होत्रा᳚ स्तृं॒पन्ता॒म् नमो॒ नम॑ स्तृं॒पन्ताꣳ॒॒ होत्राः᳚ । \newline
57. तृं॒पन्ताꣳ॒॒ होत्रा॒ होत्रा᳚ स्तृं॒पन्ता᳚म् तृं॒पन्ताꣳ॒॒ होत्रा॒ मधो॒र् मधो॒र्॒. होत्रा᳚ स्तृं॒पन्ता᳚म् तृं॒पन्ताꣳ॒॒ होत्रा॒ मधोः᳚ । \newline
58. होत्रा॒ मधो॒र् मधो॒र्॒. होत्रा॒ होत्रा॒ मधो᳚र् घृ॒तस्य॑ घृ॒तस्य॒ मधो॒र्॒. होत्रा॒ होत्रा॒ मधो᳚र् घृ॒तस्य॑ । \newline
59. मधो᳚र् घृ॒तस्य॑ घृ॒तस्य॒ मधो॒र् मधो᳚र् घृ॒तस्य॑ य॒ज्ञ्प॑तिं ॅय॒ज्ञ्प॑तिम् घृ॒तस्य॒ मधो॒र् मधो᳚र् घृ॒तस्य॑ य॒ज्ञ्प॑तिम् । \newline
60. घृ॒तस्य॑ य॒ज्ञ्प॑तिं ॅय॒ज्ञ्प॑तिम् घृ॒तस्य॑ घृ॒तस्य॑ य॒ज्ञ्प॑ति॒ मृष॑य॒ ऋष॑यो य॒ज्ञ्प॑तिम् घृ॒तस्य॑ घृ॒तस्य॑ य॒ज्ञ्प॑ति॒ मृष॑यः । \newline
61. य॒ज्ञ्प॑ति॒ मृष॑य॒ ऋष॑यो य॒ज्ञ्प॑तिं ॅय॒ज्ञ्प॑ति॒ मृष॑य॒ एन॒ सैन॒स र्.ष॑यो य॒ज्ञ्प॑तिं ॅय॒ज्ञ्प॑ति॒ मृष॑य॒ एन॑सा । \newline
62. य॒ज्ञ्प॑ति॒मिति॑ य॒ज्ञ् - प॒ति॒म् । \newline
63. ऋष॑य॒ एन॒ सैन॒स र्.ष॑य॒ ऋष॑य॒ एन॑सा ऽऽहु राहु॒ रेन॒स र्.ष॑य॒ ऋष॑य॒ एन॑सा ऽऽहुः । \newline
64. एन॑सा ऽऽहु राहु॒ रेन॒ सैन॑सा ऽऽहुः । \newline
\pagebreak
\markright{ TS 3.2.8.2  \hfill https://www.vedavms.in \hfill}

\section{ TS 3.2.8.2 }

\textbf{TS 3.2.8.2 } \newline
\textbf{Samhita Paata} \newline

ऽऽहुः । प्र॒जा निर्भ॑क्ता अनुत॒प्यमा॑ना मध॒व्यौ᳚ स्तो॒कावप॒ तौ र॑राध ॥ सं न॒स्ताभ्याꣳ॑ सृजतुवि॒श्वक॑र्मा घो॒रा ऋष॑यो॒ नमो॑ अस्त्वेभ्यः । चक्षु॑ष एषां॒ मन॑सश्च स॒न्धौ बृह॒स्पत॑ये॒ महि॒ षद् द्यु॒मन्नमः॑ ॥ नमो॑ वि॒श्वक॑र्मणे॒ स उ॑ पात्व॒स्मान॑न॒न्यान्थ्-सो॑म॒पान् मन्य॑मानः । प्रा॒णस्य॑ वि॒द्वान्थ् स॑म॒रे न धीर॒ एन॑श्चकृ॒वान् महि॑ ब॒द्ध ए॑षां ॥ तं ॅवि॑श्वकर्म॒न् - [  ] \newline

\textbf{Pada Paata} \newline

आ॒हुः॒ ॥ प्र॒जा इति॑ प्र - जाः । निर्भ॑क्ता॒ इति॒ निः - भ॒क्ताः॒ । अ॒नु॒त॒प्यमा॑ना॒ इत्य॑नु-त॒प्यमा॑नाः । म॒ध॒व्यौ᳚ । स्तो॒कौ । अपेति॑ । तौ । र॒रा॒ध॒ ॥ समिति॑ । नः॒ । ताभ्या᳚म् । सृ॒ज॒तु॒ । वि॒श्वक॒र्मेति॑ वि॒श्व - क॒र्मा॒ । घो॒राः । ऋष॑यः । नमः॑ । अ॒स्तु॒ । ए॒भ्यः॒ ॥ चक्षु॑षः । ए॒षा॒म् । मन॑सः । च॒ । स॒धांविति॑ सं - धौ । बृह॒स्पत॑ये । महि॑ । सत् । द्यु॒मदिति॑ द्यु - मत् । नमः॑ ॥ नमः॑ । वि॒श्वक॑र्मण॒ इति॑ वि॒श्व - क॒र्म॒णे॒ । सः । उ॒ । पा॒तु॒ । अ॒स्मान् । अ॒न॒न्यान् । सो॒म॒पानिति॑ सोम - पान् । मन्य॑मानः ॥ प्रा॒णस्येति॑ प्र - अ॒नस्य॑ । वि॒द्वान् । स॒म॒र इति॑ सं-अ॒रे । न । धीरः॑ । एनः॑ । च॒कृ॒वान् । महि॑ । ब॒द्धः । ए॒षा॒म् ॥ तम् । वि॒श्व॒क॒र्म॒न्निति॑ विश्व - क॒र्म॒न्न् ।  \newline


\textbf{Krama Paata} \newline

आ॒हु॒रित्या॑हुः ॥ प्र॒जा निर्भ॑क्ताः । प्र॒जा इति॑ प्र - जाः । निर्भ॑क्ता अनुत॒प्यमा॑नाः । निर्भ॑क्ता॒ इति॒ निः - भ॒क्ताः॒ । अ॒नु॒त॒प्यमा॑ना मध॒व्यौ᳚ । अ॒नु॒त॒प्यमा॑ना॒ इत्य॑नु - त॒प्यमा॑नाः । म॒ध॒व्यौ᳚ स्तो॒कौ । स्तो॒कावप॑ । अप॒ तौ । तौ र॑राध । र॒रा॒धेति॑ रराध ॥ सम् नः॑ । न॒स्ताभ्या᳚म् । ताभ्याꣳ॑ सृजतु । सृ॒ज॒तु॒ वि॒श्वक॑र्मा । वि॒श्वक॑र्मा घो॒राः । वि॒श्वक॒र्मेति॑ वि॒श्व - क॒र्मा॒ । घो॒रा ऋष॑यः । ऋष॑यो॒ नमः॑ । नमो॑ अस्तु । अ॒स्त्वे॒भ्यः॒ । ए॒भ्य॒ इत्ये᳚भ्यः ॥ चक्षु॑ष एषाम् । ए॒षा॒म् मन॑सः । मन॑सश्च । च॒ स॒न्धौ । स॒न्धौ बृह॒स्पत॑ये । स॒न्धाविति॑ सम् - धौ । बृह॒स्पत॑ये॒ महि॑ । महि॒ षत् । सद् द्यु॒मत् । द्यु॒मन् नमः॑ । द्यु॒मदिति॑ द्यु - मत् । नम॒ इति॒ नमः॑ ॥ नमो॑ वि॒श्वक॑र्मणे । वि॒श्वक॑र्मणे॒ सः । वि॒श्वक॑र्मण॒ इति॑ वि॒श्व - क॒र्म॒णे॒ । स उ॑ । उ॒ पा॒तु॒ । पा॒त्व॒स्मान् । अ॒स्मान॑न॒न्यान् । अ॒न॒न्यान्थ् सो॑म॒पान् । सो॒म॒पान् मन्य॑मानः । सो॒म॒पानिति॑ सोम - पान् । मन्य॑मान॒ इति॒ मन्य॑मानः ॥ प्रा॒णस्य॑ वि॒द्वान् । प्रा॒णस्येति॑ प्र - अ॒नस्य॑ । वि॒द्वान्थ् स॑म॒रे । स॒म॒रे न । स॒म॒र इति॑ सम् - अ॒रे । न धीरः॑ । धीर॒ एनः॑ । एन॑श्चकृ॒वान् । च॒कृ॒वान् महि॑ । महि॑ ब॒द्धः । ब॒द्ध ए॑षाम् । ए॒षा॒मित्ये॑षाम् ॥ तं ॅवि॑श्वकर्मन्न् । वि॒श्व॒क॒र्म॒न् प्र । वि॒श्व॒क॒र्म॒न्निति॑ विश्व - क॒र्म॒न्न्॒ \newline

\textbf{Jatai Paata} \newline

1. आ॒हु॒रित्या॑हुः । \newline
2. प्र॒जा निर्भ॑क्ता॒ निर्भ॑क्ताः प्र॒जाः प्र॒जा निर्भ॑क्ताः । \newline
3. प्र॒जा इति॑ प्र - जाः । \newline
4. निर्भ॑क्ता अनुत॒प्यमा॑ना अनुत॒प्यमा॑ना॒ निर्भ॑क्ता॒ निर्भ॑क्ता अनुत॒प्यमा॑नाः । \newline
5. निर्भ॑क्ता॒ इति॒ निः - भ॒क्ताः॒ । \newline
6. अ॒नु॒त॒प्यमा॑ना मध॒व्यौ॑ मध॒व्या॑ वनुत॒प्यमा॑ना अनुत॒प्यमा॑ना मध॒व्यौ᳚ । \newline
7. अ॒नु॒त॒प्यमा॑ना॒ इत्य॑नु - त॒प्यमा॑नाः । \newline
8. म॒ध॒व्यौ᳚ स्तो॒कौ स्तो॒कौ म॑ध॒व्यौ॑ मध॒व्यौ᳚ स्तो॒कौ । \newline
9. स्तो॒का वपाप॑ स्तो॒कौ स्तो॒का वप॑ । \newline
10. अप॒ तौ ता वपाप॒ तौ । \newline
11. तौ र॑राध रराध॒ तौ तौ र॑राध । \newline
12. र॒रा॒धेति॑ रराध । \newline
13. सम् नो॑ नः॒ सꣳ सम् नः॑ । \newline
14. न॒ स्ताभ्या॒म् ताभ्या᳚म् नो न॒ स्ताभ्या᳚म् । \newline
15. ताभ्याꣳ॑ सृजतु सृजतु॒ ताभ्या॒म् ताभ्याꣳ॑ सृजतु । \newline
16. सृ॒ज॒तु॒ वि॒श्वक॑र्मा वि॒श्वक॑र्मा सृजतु सृजतु वि॒श्वक॑र्मा । \newline
17. वि॒श्वक॑र्मा घो॒रा घो॒रा वि॒श्वक॑र्मा वि॒श्वक॑र्मा घो॒राः । \newline
18. वि॒श्वक॒र्मेति॑ वि॒श्व - क॒र्मा॒ । \newline
19. घो॒रा ऋष॑य॒ ऋष॑यो घो॒रा घो॒रा ऋष॑यः । \newline
20. ऋष॑यो॒ नमो॒ नम॒ ऋष॑य॒ ऋष॑यो॒ नमः॑ । \newline
21. नमो॑ अस्त्व स्तु॒ नमो॒ नमो॑ अस्तु । \newline
22. अ॒स्त्वे॒भ्य॒ ए॒भ्यो॒ अ॒स्त्व॒ स्त्वे॒भ्यः॒ । \newline
23. ए॒भ्य॒ इत्ये᳚भ्यः । \newline
24. चक्षु॑ष एषा मेषा॒म् चक्षु॑ष॒ श्चक्षु॑ष एषाम् । \newline
25. ए॒षा॒म् मन॑सो॒ मन॑स एषा मेषा॒म् मन॑सः । \newline
26. मन॑सश्च च॒ मन॑सो॒ मन॑सश्च । \newline
27. च॒ स॒न्धौ स॒न्धौ च॑ च स॒न्धौ । \newline
28. स॒न्धौ बृह॒स्पत॑ये॒ बृह॒स्पत॑ये स॒न्धौ स॒न्धौ बृह॒स्पत॑ये । \newline
29. स॒न्धाविति॑ सं - धौ । \newline
30. बृह॒स्पत॑ये॒ महि॒ महि॒ बृह॒स्पत॑ये॒ बृह॒स्पत॑ये॒ महि॑ । \newline
31. महि॒ षथ् सन् महि॒ महि॒ षत् । \newline
32. सद् द्यु॒मद् द्यु॒मथ् सथ् सद् द्यु॒मत् । \newline
33. द्यु॒मन् नमो॒ नमो᳚ द्यु॒मद् द्यु॒मन् नमः॑ । \newline
34. द्यु॒मदिति॑ द्यु - मत् । \newline
35. नम॒ इति॒ नमः॑ । \newline
36. नमो॑ वि॒श्वक॑र्मणे वि॒श्वक॑र्मणे॒ नमो॒ नमो॑ वि॒श्वक॑र्मणे । \newline
37. वि॒श्वक॑र्मणे॒ स स वि॒श्वक॑र्मणे वि॒श्वक॑र्मणे॒ सः । \newline
38. वि॒श्वक॑र्मण॒ इति॑ वि॒श्व - क॒र्म॒णे॒ । \newline
39. स उ॑ वु॒ स स उ॑ । \newline
40. उ॒ पा॒तु॒ पा॒तू॒ पा॒तु॒ । \newline
41. पा॒त्व॒स्मा न॒स्मान् पा॑तु पात्व॒स्मान् । \newline
42. अ॒स्मा न॑न॒न्या न॑न॒न्या न॒स्मा न॒स्मा न॑न॒न्यान् । \newline
43. अ॒न॒न्यान् थ्सो॑म॒पान् थ्सो॑म॒पा न॑न॒न्या न॑न॒न्यान् थ्सो॑म॒पान् । \newline
44. सो॒म॒पान् मन्य॑मानो॒ मन्य॑मानः सोम॒पान् थ्सो॑म॒पान् मन्य॑मानः । \newline
45. सो॒म॒पानिति॑ सोम - पान् । \newline
46. मन्य॑मान॒ इति॒ मन्य॑मानः । \newline
47. प्रा॒णस्य॑ वि॒द्वान्. वि॒द्वान् प्रा॒णस्य॑ प्रा॒णस्य॑ वि॒द्वान् । \newline
48. प्रा॒णस्येति॑ प्र - अ॒नस्य॑ । \newline
49. वि॒द्वान् थ्स॑म॒रे स॑म॒रे वि॒द्वान्. वि॒द्वान् थ्स॑म॒रे । \newline
50. स॒म॒रे न न स॑म॒रे स॑म॒रे न । \newline
51. स॒म॒र इति॑ सं - अ॒रे । \newline
52. न धीरो॒ धीरो॒ न न धीरः॑ । \newline
53. धीर॒ एन॒ एनो॒ धीरो॒ धीर॒ एनः॑ । \newline
54. एन॑ श्चकृ॒वाꣳ श्च॑कृ॒वा नेन॒ एन॑ श्चकृ॒वान् । \newline
55. च॒कृ॒वान् महि॒ महि॑ चकृ॒वाꣳ श्च॑कृ॒वान् महि॑ । \newline
56. महि॑ ब॒द्धो ब॒द्धो महि॒ महि॑ ब॒द्धः । \newline
57. ब॒द्ध ए॑षा मेषाम् ब॒द्धो ब॒द्ध ए॑षाम् । \newline
58. ए॒षा॒मित्ये॑षाम् । \newline
59. तं ॅवि॑श्वकर्मन्. विश्वकर्म॒न् तम् तं ॅवि॑श्वकर्मन्न् । \newline
60. वि॒श्व॒क॒र्म॒न् प्र प्र वि॑श्वकर्मन्. विश्वकर्म॒न् प्र । \newline
61. वि॒श्व॒क॒र्म॒न्निति॑ विश्व - क॒र्म॒न्न् । \newline

\textbf{Ghana Paata } \newline

1. आ॒हु॒रित्या॑हुः । \newline
2. प्र॒जा निर्भ॑क्ता॒ निर्भ॑क्ताः प्र॒जाः प्र॒जा निर्भ॑क्ता अनुत॒प्यमा॑ना अनुत॒प्यमा॑ना॒ निर्भ॑क्ताः प्र॒जाः प्र॒जा निर्भ॑क्ता अनुत॒प्यमा॑नाः । \newline
3. प्र॒जा इति॑ प्र - जाः । \newline
4. निर्भ॑क्ता अनुत॒प्यमा॑ना अनुत॒प्यमा॑ना॒ निर्भ॑क्ता॒ निर्भ॑क्ता अनुत॒प्यमा॑ना मध॒व्यौ॑ मध॒व्या॑ वनुत॒प्यमा॑ना॒ निर्भ॑क्ता॒ निर्भ॑क्ता अनुत॒प्यमा॑ना मध॒व्यौ᳚ । \newline
5. निर्भ॑क्ता॒ इति॒ निः - भ॒क्ताः॒ । \newline
6. अ॒नु॒त॒प्यमा॑ना मध॒व्यौ॑ मध॒व्या॑ वनुत॒प्यमा॑ना अनुत॒प्यमा॑ना मध॒व्यौ᳚ स्तो॒कौ स्तो॒कौ म॑ध॒व्या॑ वनुत॒प्यमा॑ना अनुत॒प्यमा॑ना मध॒व्यौ᳚ स्तो॒कौ । \newline
7. अ॒नु॒त॒प्यमा॑ना॒ इत्य॑नु - त॒प्यमा॑नाः । \newline
8. म॒ध॒व्यौ᳚ स्तो॒कौ स्तो॒कौ म॑ध॒व्यौ॑ मध॒व्यौ᳚ स्तो॒का वपाप॑ स्तो॒कौ म॑ध॒व्यौ॑ मध॒व्यौ᳚ स्तो॒का वप॑ । \newline
9. स्तो॒का वपाप॑ स्तो॒कौ स्तो॒का वप॒ तौ ता वप॑ स्तो॒कौ स्तो॒का वप॒ तौ । \newline
10. अप॒ तौ ता वपाप॒ तौ र॑राध रराध॒ ता वपाप॒ तौ र॑राध । \newline
11. तौ र॑राध रराध॒ तौ तौ र॑राध । \newline
12. र॒रा॒धेति॑ रराध । \newline
13. सम् नो॑ नः॒ सꣳ सम् न॒ स्ताभ्या॒म् ताभ्या᳚म् नः॒ सꣳ सम् न॒ स्ताभ्या᳚म् । \newline
14. न॒ स्ताभ्या॒म् ताभ्या᳚म् नो न॒ स्ताभ्याꣳ॑ सृजतु सृजतु॒ ताभ्या᳚म् नो न॒ स्ताभ्याꣳ॑ सृजतु । \newline
15. ताभ्याꣳ॑ सृजतु सृजतु॒ ताभ्या॒म् ताभ्याꣳ॑ सृजतु वि॒श्वक॑र्मा वि॒श्वक॑र्मा सृजतु॒ ताभ्या॒म् ताभ्याꣳ॑ सृजतु वि॒श्वक॑र्मा । \newline
16. सृ॒ज॒तु॒ वि॒श्वक॑र्मा वि॒श्वक॑र्मा सृजतु सृजतु वि॒श्वक॑र्मा घो॒रा घो॒रा वि॒श्वक॑र्मा सृजतु सृजतु वि॒श्वक॑र्मा घो॒राः । \newline
17. वि॒श्वक॑र्मा घो॒रा घो॒रा वि॒श्वक॑र्मा वि॒श्वक॑र्मा घो॒रा ऋष॑य॒ ऋष॑यो घो॒रा वि॒श्वक॑र्मा 
वि॒श्वक॑र्मा घो॒रा ऋष॑यः । \newline
18. वि॒श्वक॒र्मेति॑ वि॒श्व - क॒र्मा॒ । \newline
19. घो॒रा ऋष॑य॒ ऋष॑यो घो॒रा घो॒रा ऋष॑यो॒ नमो॒ नम॒ ऋष॑यो घो॒रा घो॒रा ऋष॑यो॒ नमः॑ । \newline
20. ऋष॑यो॒ नमो॒ नम॒ ऋष॑य॒ ऋष॑यो॒ नमो॑ अस्त्वस्तु॒ नम॒ ऋष॑य॒ ऋष॑यो॒ नमो॑ अस्तु । \newline
21. नमो॑ अस्त्वस्तु॒ नमो॒ नमो॑ अस्त्वेभ्य एभ्यो अस्तु॒ नमो॒ नमो॑ अस्त्वेभ्यः । \newline
22. अ॒स्त्वे॒भ्य॒ ए॒भ्यो॒ अ॒स्त्व॒ स्त्वे॒भ्यः॒ । \newline
23. ए॒भ्य॒ इत्ये᳚भ्यः । \newline
24. चक्षु॑ष एषा मेषा॒म् चक्षु॑ष॒ श्चक्षु॑ष एषा॒म् मन॑सो॒ मन॑स एषा॒म् चक्षु॑ष॒ श्चक्षु॑ष एषा॒म् मन॑सः । \newline
25. ए॒षा॒म् मन॑सो॒ मन॑स एषा मेषा॒म् मन॑सश्च च॒ मन॑स एषा मेषा॒म् मन॑सश्च । \newline
26. मन॑सश्च च॒ मन॑सो॒ मन॑सश्च स॒न्धौ स॒न्धौ च॒ मन॑सो॒ मन॑सश्च स॒न्धौ । \newline
27. च॒ स॒न्धौ स॒न्धौ च॑ च स॒न्धौ बृह॒स्पत॑ये॒ बृह॒स्पत॑ये स॒न्धौ च॑ च स॒न्धौ बृह॒स्पत॑ये । \newline
28. स॒न्धौ बृह॒स्पत॑ये॒ बृह॒स्पत॑ये स॒न्धौ स॒न्धौ बृह॒स्पत॑ये॒ महि॒ महि॒ बृह॒स्पत॑ये स॒न्धौ स॒न्धौ बृह॒स्पत॑ये॒ महि॑ । \newline
29. स॒न्धाविति॑ सं - धौ । \newline
30. बृह॒स्पत॑ये॒ महि॒ महि॒ बृह॒स्पत॑ये॒ बृह॒स्पत॑ये॒ महि॒ ष थ्सन् महि॒ बृह॒स्पत॑ये॒ बृह॒स्पत॑ये॒ महि॒ षत् । \newline
31. महि॒ षथ् सन् महि॒ महि॒ षद् द्यु॒मद् द्यु॒मथ् सन् महि॒ महि॒ षद् द्यु॒मत् । \newline
32. सद् द्यु॒मद् द्यु॒मथ् सथ् सद् द्यु॒मन् नमो॒ नमो᳚ द्यु॒मथ् सथ् सद् द्यु॒मन् नमः॑ । \newline
33. द्यु॒मन् नमो॒ नमो᳚ द्यु॒मद् द्यु॒मन् नमः॑ । \newline
34. द्यु॒मदिति॑ द्यु - मत् । \newline
35. नम॒ इति॒ नमः॑ । \newline
36. नमो॑ वि॒श्वक॑र्मणे वि॒श्वक॑र्मणे॒ नमो॒ नमो॑ वि॒श्वक॑र्मणे॒ स स वि॒श्वक॑र्मणे॒ नमो॒ नमो॑ वि॒श्वक॑र्मणे॒ सः । \newline
37. वि॒श्वक॑र्मणे॒ स स वि॒श्वक॑र्मणे वि॒श्वक॑र्मणे॒ स उ॑ वु॒ स वि॒श्वक॑र्मणे वि॒श्वक॑र्मणे॒ स उ॑ । \newline
38. वि॒श्वक॑र्मण॒ इति॑ वि॒श्व - क॒र्म॒णे॒ । \newline
39. स उ॑ वु॒ स स उ॑ पातु पातू॒ स स उ॑ पातु । \newline
40. उ॒ पा॒तु॒ पा॒तू॒ पा॒त्व॒स्मा न॒स्मान् पा॑तू पात्व॒स्मान् । \newline
41. पा॒त्व॒स्मा न॒स्मान् पा॑तु पात्व॒स्मा न॑न॒न्या न॑न॒न्या न॒स्मान् पा॑तु पात्व॒स्मा न॑न॒न्यान् । \newline
42. अ॒स्मा न॑न॒न्या न॑न॒न्या न॒स्मा न॒स्मा न॑न॒न्यान् थ्सो॑म॒पान् थ्सो॑म॒पा न॑न॒न्या न॒स्मा न॒स्मा न॑न॒न्यान् थ्सो॑म॒पान् । \newline
43. अ॒न॒न्यान् थ्सो॑म॒पान् थ्सो॑म॒पा न॑न॒न्या न॑न॒न्यान् थ्सो॑म॒पान् मन्य॑मानो॒ मन्य॑मानः सोम॒पा न॑न॒न्या न॑न॒न्यान् थ्सो॑म॒पान् मन्य॑मानः । \newline
44. सो॒म॒पान् मन्य॑मानो॒ मन्य॑मानः सोम॒पान् थ्सो॑म॒पान् मन्य॑मानः । \newline
45. सो॒म॒पानिति॑ सोम - पान् । \newline
46. मन्य॑मान॒ इति॒ मन्य॑मानः । \newline
47. प्रा॒णस्य॑ वि॒द्वान्. वि॒द्वान् प्रा॒णस्य॑ प्रा॒णस्य॑ वि॒द्वान् थ्स॑म॒रे स॑म॒रे वि॒द्वान् प्रा॒णस्य॑ प्रा॒णस्य॑ वि॒द्वान् थ्स॑म॒रे । \newline
48. प्रा॒णस्येति॑ प्र - अ॒नस्य॑ । \newline
49. वि॒द्वान् थ्स॑म॒रे स॑म॒रे वि॒द्वान्. वि॒द्वान् थ्स॑म॒रे न न स॑म॒रे वि॒द्वान्. वि॒द्वान् थ्स॑म॒रे न । \newline
50. स॒म॒रे न न स॑म॒रे स॑म॒रे न धीरो॒ धीरो॒ न स॑म॒रे स॑म॒रे न धीरः॑ । \newline
51. स॒म॒र इति॑ सं - अ॒रे । \newline
52. न धीरो॒ धीरो॒ न न धीर॒ एन॒ एनो॒ धीरो॒ न न धीर॒ एनः॑ । \newline
53. धीर॒ एन॒ एनो॒ धीरो॒ धीर॒ एन॑ श्चकृ॒वाꣳ श्च॑कृ॒वा नेनो॒ धीरो॒ धीर॒ एन॑ श्चकृ॒वान् । \newline
54. एन॑ श्चकृ॒वाꣳ श्च॑कृ॒वा नेन॒ एन॑ श्चकृ॒वान् महि॒ महि॑ चकृ॒वा नेन॒ एन॑ श्चकृ॒वान् महि॑ । \newline
55. च॒कृ॒वान् महि॒ महि॑ चकृ॒वाꣳ श्च॑कृ॒वान् महि॑ ब॒द्धो ब॒द्धो महि॑ चकृ॒वाꣳ श्च॑कृ॒वान् महि॑ ब॒द्धः । \newline
56. महि॑ ब॒द्धो ब॒द्धो महि॒ महि॑ ब॒द्ध ए॑षा मेषाम् ब॒द्धो महि॒ महि॑ ब॒द्ध ए॑षाम् । \newline
57. ब॒द्ध ए॑षा मेषाम् ब॒द्धो ब॒द्ध ए॑षाम् । \newline
58. ए॒षा॒मित्ये॑षाम् । \newline
59. तं ॅवि॑श्वकर्मन्. विश्वकर्म॒न् तम् तं ॅवि॑श्वकर्म॒न् प्र प्र वि॑श्वकर्म॒न् तम् तं ॅवि॑श्वकर्म॒न् प्र । \newline
60. वि॒श्व॒क॒र्म॒न् प्र प्र वि॑श्वकर्मन्. विश्वकर्म॒न् प्र मु॑ञ्च मुञ्च॒ प्र वि॑श्वकर्मन्. विश्वकर्म॒न् प्र मु॑ञ्च । \newline
61. वि॒श्व॒क॒र्म॒न्निति॑ विश्व - क॒र्म॒न्न् । \newline
\pagebreak
\markright{ TS 3.2.8.3  \hfill https://www.vedavms.in \hfill}

\section{ TS 3.2.8.3 }

\textbf{TS 3.2.8.3 } \newline
\textbf{Samhita Paata} \newline

-प्र मु॑ञ्चा स्व॒स्तये॒ ये भ॒क्षय॑न्तो॒ न वसू᳚न्यानृ॒हुः । यान॒ग्नयो॒ऽन्वत॑प्यन्त॒ धिष्णि॑या इ॒यं तेषा॑मव॒या दुरि॑ष्ट्यै॒ स्वि॑ष्टिं न॒स्तां कृ॑णोतु वि॒श्वक॑र्मा ॥ नमः॑ पि॒तृभ्यो॑ अ॒भि ये नो॒ अख्य॑न्. यज्ञ्॒कृतो॑ य॒ज्ञ्का॑माः सुदे॒वा अ॑का॒मा वो॒ दक्षि॑णां॒ न नी॑निम॒ मा न॒स्तस्मा॒ देन॑सः पापयिष्ट । याव॑न्तो॒ वै स॑द॒स्या᳚स्ते सर्वे॑ दक्षि॒ण्या᳚स्तेभ्यो॒ यो दक्षि॑णां॒ न-  [  ] \newline

\textbf{Pada Paata} \newline

प्रेति॑ । मु॒ञ्च॒ । स्व॒स्तये᳚ । ये । भ॒क्षय॑न्तः । न । वसू॑नि । आ॒नृ॒हुः ॥ यान् । अ॒ग्नयः॑ । अ॒न्वत॑प्य॒न्तेत्य॑नु - अत॑प्यन्त । धिष्णि॑याः । इ॒यम् । तेषा᳚म् । अ॒व॒या । दुरि॑ष्ट्या॒ इति॒ दुः - इ॒ष्ट्यै॒ । स्वि॑ष्टि॒मिति॒ सु - इ॒ष्टि॒म् । नः॒ । ताम् । कृ॒णो॒तु॒ । वि॒श्वक॒र्मेति॑ वि॒श्व - क॒र्मा॒ ॥ नमः॑ । पि॒तृभ्य॒ इति॑ पि॒तृ - भ्यः॒ । अ॒भीति॑ । ये । नः॒ । अख्यन्न्॑ । य॒ज्ञ्॒कृत॒ इति॑ यज्ञ् - कृतः॑ । य॒ज्ञ्का॑मा॒ इति॑ य॒ज्ञ् - का॒माः॒ । सु॒दे॒वा इति॑ सु - दे॒वाः । अ॒का॒माः । वः॒ । दक्षि॑णाम् । न । नी॒नि॒म॒ । मा । नः॒ । तस्मा᳚त् । एन॑सः । पा॒प॒यि॒ष्ट॒ ॥ याव॑न्तः । वै । स॒द॒स्याः᳚ । ते । सर्वे᳚ । द॒क्षि॒ण्याः᳚ । तेभ्यः॑ । यः । दक्षि॑णाम् । न ।  \newline


\textbf{Krama Paata} \newline

प्र मु॑ञ्च । मु॒ञ्चा॒ स्व॒स्तये᳚ । स्व॒स्तये॒ ये । ये भ॒क्षय॑न्तः । भ॒क्षय॑न्तो॒ न । न वसू॑नि । वसू᳚न्यानृ॒हुः । आ॒नृ॒हुरित्या॑नृ॒हुः ॥ यान॒ग्नयः॑ । अ॒ग्नयो॒ ऽन्वत॑प्यन्त । अ॒न्वत॑प्यन्त॒ धिष्णि॑याः । अ॒न्वत॑प्य॒न्तेत्य॑नु - अत॑प्यन्त । धिष्णि॑या इ॒यम् । इ॒यम् तेषा᳚म् । तेषा॑मव॒या । अ॒व॒या दुरि॑ष्ट्यै । दुरि॑ष्ट्यै॒ स्वि॑ष्टिम् । दुरि॑ष्ट्या॒ इति॒ दुः - इ॒ष्ट्यै॒ । स्वि॑ष्टिम् नः । स्वि॑ष्टि॒मिति॒ सु - इ॒ष्टि॒म् । न॒स्ताम् । ताम् कृ॑णोतु । कृ॒णो॒तु॒ वि॒श्वक॑र्मा । वि॒श्वक॑र्मा॒ नमः॑ । वि॒श्वक॒र्मेति॑ वि॒श्व - क॒र्मा॒ ॥ नमः॑ पि॒तृभ्यः॑ । पि॒तृभ्यो॑ अ॒भि । पि॒तृभ्य॒ इति॑ पि॒तृ - भ्यः॒ । अ॒भि ये । ये नः॑ । नो॒ अख्यन्न्॑ । अख्य॑न्. यज्ञ्॒कृतः॑ । य॒ज्ञ्॒कृतो॑ य॒ज्ञ्का॑माः । य॒ज्ञ्॒कृत॒ इति॑ यज्ञ् - कृतः॑ । य॒ज्ञ्का॑माः सुदे॒वाः । य॒ज्ञ्का॑म॒ इति॑ य॒ज्ञ् - का॒माः॒ । सु॒दे॒वा अ॑का॒माः । सु॒दे॒वा इति॑ सु - दे॒वाः । अ॒का॒मा वः॑ । वो॒ दक्षि॑णाम् । दक्षि॑णा॒म् न । न नी॑निम । नी॒नि॒म॒ मा । मा नः॑ । न॒ स्तस्मा᳚त् । तस्मा॒देन॑सः । एन॑सः पापयिष्ट । पा॒प॒यि॒ष्टेति॑ पापयिष्ट ॥ याव॑न्तो॒ वै । वै स॑द॒स्याः᳚ । स॒द॒स्या᳚स्ते । ते॒ सर्वे᳚ । सर्वे॑ दक्षि॒ण्याः᳚ । द॒क्षि॒ण्या᳚ स्तेभ्यः॑ । तेभ्यो॒ यः । यो दक्षि॑णाम् । दक्षि॑णा॒म् न । न नये᳚त् \newline

\textbf{Jatai Paata} \newline

1. प्र मु॑ञ्च मुञ्च॒ प्र प्र मु॑ञ्च । \newline
2. मु॒ञ्चा॒ स्व॒स्तये᳚ स्व॒स्तये॑ मुञ्च मुञ्चा स्व॒स्तये᳚ । \newline
3. स्व॒स्तये॒ ये ये स्व॒स्तये᳚ स्व॒स्तये॒ ये । \newline
4. ये भ॒क्षय॑न्तो भ॒क्षय॑न्तो॒ ये ये भ॒क्षय॑न्तः । \newline
5. भ॒क्षय॑न्तो॒ न न भ॒क्षय॑न्तो भ॒क्षय॑न्तो॒ न । \newline
6. न वसू॑नि॒ वसू॑नि॒ न न वसू॑नि । \newline
7. वसू᳚ न्यानृ॒हु रा॑नृ॒हुर् वसू॑नि॒ वसू᳚ न्यानृ॒हुः । \newline
8. आ॒नृ॒हुरित्या॑नृ॒हुः । \newline
9. या न॒ग्नयो॑ अ॒ग्नयो॒ यान्. या न॒ग्नयः॑ । \newline
10. अ॒ग्नयो॒ ऽन्वत॑प्यन्ता॒ न्वत॑प्यन्ता॒ ग्नयो॑ अ॒ग्नयो॒ ऽन्वत॑प्यन्त । \newline
11. अ॒न्वत॑प्यन्त॒ धिष्णि॑या॒ धिष्णि॑या अ॒न्वत॑प्यन्ता॒ न्वत॑प्यन्त॒ धिष्णि॑याः । \newline
12. अ॒न्वत॑प्य॒न्तेत्य॑नु - अत॑प्यन्त । \newline
13. धिष्णि॑या इ॒य मि॒यम् धिष्णि॑या॒ धिष्णि॑या इ॒यम् । \newline
14. इ॒यम् तेषा॒म् तेषा॑ मि॒य मि॒यम् तेषा᳚म् । \newline
15. तेषा॑ मव॒या ऽव॒या तेषा॒म् तेषा॑ मव॒या । \newline
16. अ॒व॒या दुरि॑ष्ट्यै॒ दुरि॑ष्ट्या अव॒या ऽव॒या दुरि॑ष्ट्यै । \newline
17. दुरि॑ष्ट्यै॒ स्वि॑ष्टिꣳ॒॒ स्वि॑ष्टि॒म् दुरि॑ष्ट्यै॒ दुरि॑ष्ट्यै॒ स्वि॑ष्टिम् । \newline
18. दुरि॑ष्ट्या॒ इति॒ दुः - इ॒ष्ट्यै॒ । \newline
19. स्वि॑ष्टिम् नो नः॒ स्वि॑ष्टिꣳ॒॒ स्वि॑ष्टिम् नः । \newline
20. स्वि॑ष्टि॒मिति॒ सु - इ॒ष्टि॒म् । \newline
21. न॒ स्ताम् ताम् नो॑ न॒ स्ताम् । \newline
22. ताम् कृ॑णोतु कृणोतु॒ ताम् ताम् कृ॑णोतु । \newline
23. कृ॒णो॒तु॒ वि॒श्वक॑र्मा वि॒श्वक॑र्मा कृणोतु कृणोतु वि॒श्वक॑र्मा । \newline
24. वि॒श्वक॒र्मेति॑ वि॒श्व - क॒र्मा॒ । \newline
25. नमः॑ पि॒तृभ्यः॑ पि॒तृभ्यो॒ नमो॒ नमः॑ पि॒तृभ्यः॑ । \newline
26. पि॒तृभ्यो॑ अ॒भ्य॑भि पि॒तृभ्यः॑ पि॒तृभ्यो॑ अ॒भि । \newline
27. पि॒तृभ्य॒ इति॑ पि॒तृ - भ्यः॒ । \newline
28. अ॒भि ये ये अ॒भ्य॑भि ये । \newline
29. ये नो॑ नो॒ ये ये नः॑ । \newline
30. नो॒ अख्य॒न् नख्य॑न् नो नो॒ अख्यन्न्॑ । \newline
31. अख्य॑न्. यज्ञ्॒कृतो॑ यज्ञ्॒कृतो॒ अख्य॒न् नख्य॑न्. यज्ञ्॒कृतः॑ । \newline
32. य॒ज्ञ्॒कृतो॑ य॒ज्ञ्का॑मा य॒ज्ञ्का॑मा यज्ञ्॒कृतो॑ यज्ञ्॒कृतो॑ य॒ज्ञ्का॑माः । \newline
33. य॒ज्ञ्॒कृत॒ इति॑ यज्ञ् - कृतः॑ । \newline
34. य॒ज्ञ्का॑माः सुदे॒वाः सु॑दे॒वा य॒ज्ञ्का॑मा य॒ज्ञ्का॑माः सुदे॒वाः । \newline
35. य॒ज्ञ्का॑मा॒ इति॑ य॒ज्ञ् - का॒माः॒ । \newline
36. सु॒दे॒वा अ॑का॒मा अ॑का॒माः सु॑दे॒वाः सु॑दे॒वा अ॑का॒माः । \newline
37. सु॒दे॒वा इति॑ सु - दे॒वाः । \newline
38. अ॒का॒मा वो॑ वो अका॒मा अ॑का॒मा वः॑ । \newline
39. वो॒ दक्षि॑णा॒म् दक्षि॑णां ॅवो वो॒ दक्षि॑णाम् । \newline
40. दक्षि॑णा॒म् न न दक्षि॑णा॒म् दक्षि॑णा॒म् न । \newline
41. न नी॑निम नीनिम॒ न न नी॑निम । \newline
42. नी॒नि॒म॒ मा मा नी॑निम नीनिम॒ मा । \newline
43. मा नो॑ नो॒ मा मा नः॑ । \newline
44. न॒ स्तस्मा॒त् तस्मा᳚न् नो न॒ स्तस्मा᳚त् । \newline
45. तस्मा॒ देन॑स॒ एन॑स॒ स्तस्मा॒त् तस्मा॒ देन॑सः । \newline
46. एन॑सः पापयिष्ट पापयि॒ ष्टैन॑स॒ एन॑सः पापयिष्ट । \newline
47. पा॒प॒यि॒ष्टेति॑ पापयिष्ट । \newline
48. याव॑न्तो॒ वै वै याव॑न्तो॒ याव॑न्तो॒ वै । \newline
49. वै स॑द॒स्याः᳚ सद॒स्या॑ वै वै स॑द॒स्याः᳚ । \newline
50. स॒द॒स्या᳚ स्ते ते स॑द॒स्याः᳚ सद॒स्या᳚ स्ते । \newline
51. ते सर्वे॒ सर्वे॒ ते ते सर्वे᳚ । \newline
52. सर्वे॑ दक्षि॒ण्या॑ दक्षि॒ण्याः᳚ सर्वे॒ सर्वे॑ दक्षि॒ण्याः᳚ । \newline
53. द॒क्षि॒ण्या᳚ स्तेभ्य॒ स्तेभ्यो॑ दक्षि॒ण्या॑ दक्षि॒ण्या᳚ स्तेभ्यः॑ । \newline
54. तेभ्यो॒ यो य स्तेभ्य॒ स्तेभ्यो॒ यः । \newline
55. यो दक्षि॑णा॒म् दक्षि॑णां॒ ॅयो यो दक्षि॑णाम् । \newline
56. दक्षि॑णा॒म् न न दक्षि॑णा॒म् दक्षि॑णा॒म् न । \newline
57. न नये॒न् नये॒न् न न नये᳚त् । \newline

\textbf{Ghana Paata } \newline

1. प्र मु॑ञ्च मुञ्च॒ प्र प्र मु॑ञ्चा स्व॒स्तये᳚ स्व॒स्तये॑ मुञ्च॒ प्र प्र मु॑ञ्चा स्व॒स्तये᳚ । \newline
2. मु॒ञ्चा॒ स्व॒स्तये᳚ स्व॒स्तये॑ मुञ्च मुञ्चा स्व॒स्तये॒ ये ये स्व॒स्तये॑ मुञ्च मुञ्चा स्व॒स्तये॒ ये । \newline
3. स्व॒स्तये॒ ये ये स्व॒स्तये᳚ स्व॒स्तये॒ ये भ॒क्षय॑न्तो भ॒क्षय॑न्तो॒ ये स्व॒स्तये᳚ स्व॒स्तये॒ ये भ॒क्षय॑न्तः । \newline
4. ये भ॒क्षय॑न्तो भ॒क्षय॑न्तो॒ ये ये भ॒क्षय॑न्तो॒ न न भ॒क्षय॑न्तो॒ ये ये भ॒क्षय॑न्तो॒ न । \newline
5. भ॒क्षय॑न्तो॒ न न भ॒क्षय॑न्तो भ॒क्षय॑न्तो॒ न वसू॑नि॒ वसू॑नि॒ न भ॒क्षय॑न्तो भ॒क्षय॑न्तो॒ न वसू॑नि । \newline
6. न वसू॑नि॒ वसू॑नि॒ न न वसू᳚ न्यानृ॒हु रा॑नृ॒हुर् वसू॑नि॒ न न वसू᳚ न्यानृ॒हुः । \newline
7. वसू᳚ न्यानृ॒हु रा॑नृ॒हुर् वसू॑नि॒ वसू᳚ न्यानृ॒हुः । \newline
8. आ॒नृ॒हुरित्या॑नृ॒हुः । \newline
9. या न॒ग्नयो॑ अ॒ग्नयो॒ यान्. या न॒ग्नयो॒ ऽन्वत॑प्यन्ता॒ न्वत॑प्यन्ता॒ ग्नयो॒ यान्. या न॒ग्नयो॒ ऽन्वत॑प्यन्त । \newline
10. अ॒ग्नयो॒ ऽन्वत॑प्यन्ता॒ न्वत॑प्यन्ता॒ ग्नयो॑ अ॒ग्नयो॒ ऽन्वत॑प्यन्त॒ धिष्णि॑या॒ धिष्णि॑या अ॒न्वत॑प्यन्ता॒ ग्नयो॑ अ॒ग्नयो॒ ऽन्वत॑प्यन्त॒ धिष्णि॑याः । \newline
11. अ॒न्वत॑प्यन्त॒ धिष्णि॑या॒ धिष्णि॑या अ॒न्वत॑प्यन्ता॒ न्वत॑प्यन्त॒ धिष्णि॑या इ॒य मि॒यम् धिष्णि॑या अ॒न्वत॑प्यन्ता॒ न्वत॑प्यन्त॒ धिष्णि॑या इ॒यम् । \newline
12. अ॒न्वत॑प्य॒न्तेत्य॑नु - अत॑प्यन्त । \newline
13. धिष्णि॑या इ॒य मि॒यम् धिष्णि॑या॒ धिष्णि॑या इ॒यम् तेषा॒म् तेषा॑ मि॒यम् धिष्णि॑या॒ धिष्णि॑या इ॒यम् तेषा᳚म् । \newline
14. इ॒यम् तेषा॒म् तेषा॑ मि॒य मि॒यम् तेषा॑ मव॒या ऽव॒या तेषा॑ मि॒य मि॒यम् तेषा॑ मव॒या । \newline
15. तेषा॑ मव॒या ऽव॒या तेषा॒म् तेषा॑ मव॒या दुरि॑ष्ट्यै॒ दुरि॑ष्ट्या अव॒या तेषा॒म् तेषा॑ मव॒या दुरि॑ष्ट्यै । \newline
16. अ॒व॒या दुरि॑ष्ट्यै॒ दुरि॑ष्ट्या अव॒या ऽव॒या दुरि॑ष्ट्यै॒ स्वि॑ष्टिꣳ॒॒ स्वि॑ष्टि॒म् दुरि॑ष्ट्या अव॒या ऽव॒या दुरि॑ष्ट्यै॒ स्वि॑ष्टिम् । \newline
17. दुरि॑ष्ट्यै॒ स्वि॑ष्टिꣳ॒॒ स्वि॑ष्टि॒म् दुरि॑ष्ट्यै॒ दुरि॑ष्ट्यै॒ स्वि॑ष्टिम् नो नः॒ स्वि॑ष्टि॒म् दुरि॑ष्ट्यै॒ दुरि॑ष्ट्यै॒ स्वि॑ष्टिम् नः । \newline
18. दुरि॑ष्ट्या॒ इति॒ दुः - इ॒ष्ट्यै॒ । \newline
19. स्वि॑ष्टिम् नो नः॒ स्वि॑ष्टिꣳ॒॒ स्वि॑ष्टिम् न॒ स्ताम् ताम् नः॒ स्वि॑ष्टिꣳ॒॒ स्वि॑ष्टिम् न॒ स्ताम् । \newline
20. स्वि॑ष्टि॒मिति॒ सु - इ॒ष्टि॒म् । \newline
21. न॒स्ताम् ताम् नो॑ न॒ स्ताम् कृ॑णोतु कृणोतु॒ ताम् नो॑ न॒ स्ताम् कृ॑णोतु । \newline
22. ताम् कृ॑णोतु कृणोतु॒ ताम् ताम् कृ॑णोतु वि॒श्वक॑र्मा वि॒श्वक॑र्मा कृणोतु॒ ताम् ताम् कृ॑णोतु वि॒श्वक॑र्मा । \newline
23. कृ॒णो॒तु॒ वि॒श्वक॑र्मा वि॒श्वक॑र्मा कृणोतु कृणोतु वि॒श्वक॑र्मा । \newline
24. वि॒श्वक॒र्मेति॑ वि॒श्व - क॒र्मा॒ । \newline
25. नमः॑ पि॒तृभ्यः॑ पि॒तृभ्यो॒ नमो॒ नमः॑ पि॒तृभ्यो॑ अ॒भ्य॑भि पि॒तृभ्यो॒ नमो॒ नमः॑ पि॒तृभ्यो॑ अ॒भि । \newline
26. पि॒तृभ्यो॑ अ॒भ्य॑भि पि॒तृभ्यः॑ पि॒तृभ्यो॑ अ॒भि ये ये अ॒भि पि॒तृभ्यः॑ पि॒तृभ्यो॑ अ॒भि ये । \newline
27. पि॒तृभ्य॒ इति॑ पि॒तृ - भ्यः॒ । \newline
28. अ॒भि ये ये अ॒भ्य॑भि ये नो॑ नो॒ ये अ॒भ्य॑भि ये नः॑ । \newline
29. ये नो॑ नो॒ ये ये नो॒ अख्य॒न् नख्य॑न् नो॒ ये ये नो॒ अख्यन्न्॑ । \newline
30. नो॒ अख्य॒न् नख्य॑न् नो नो॒ अख्य॑न्. यज्ञ्॒कृतो॑ यज्ञ्॒कृतो॒ अख्य॑न् नो नो॒ अख्य॑न्. यज्ञ्॒कृतः॑ । \newline
31. अख्य॑न्. यज्ञ्॒कृतो॑ यज्ञ्॒कृतो॒ अख्य॒न् नख्य॑न्. यज्ञ्॒कृतो॑ य॒ज्ञ्का॑मा य॒ज्ञ्का॑मा यज्ञ्॒कृतो॒ अख्य॒न् नख्य॑न्. यज्ञ्॒कृतो॑ य॒ज्ञ्का॑माः । \newline
32. य॒ज्ञ्॒कृतो॑ य॒ज्ञ्का॑मा य॒ज्ञ्का॑मा यज्ञ्॒कृतो॑ यज्ञ्॒कृतो॑ य॒ज्ञ्का॑माः सुदे॒वाः सु॑दे॒वा य॒ज्ञ्का॑मा यज्ञ्॒कृतो॑ यज्ञ्॒कृतो॑ य॒ज्ञ्का॑माः सुदे॒वाः । \newline
33. य॒ज्ञ्॒कृत॒ इति॑ यज्ञ् - कृतः॑ । \newline
34. य॒ज्ञ्का॑माः सुदे॒वाः सु॑दे॒वा य॒ज्ञ्का॑मा य॒ज्ञ्का॑माः सुदे॒वा अ॑का॒मा अ॑का॒माः सु॑दे॒वा य॒ज्ञ्का॑मा य॒ज्ञ्का॑माः सुदे॒वा अ॑का॒माः । \newline
35. य॒ज्ञ्का॑मा॒ इति॑ य॒ज्ञ् - का॒माः॒ । \newline
36. सु॒दे॒वा अ॑का॒मा अ॑का॒माः सु॑दे॒वाः सु॑दे॒वा अ॑का॒मा वो॑ वो अका॒माः सु॑दे॒वाः सु॑दे॒वा अ॑का॒मा वः॑ । \newline
37. सु॒दे॒वा इति॑ सु - दे॒वाः । \newline
38. अ॒का॒मा वो॑ वो अका॒मा अ॑का॒मा वो॒ दक्षि॑णा॒म् दक्षि॑णां ॅवो अका॒मा अ॑का॒मा वो॒ दक्षि॑णाम् । \newline
39. वो॒ दक्षि॑णा॒म् दक्षि॑णां ॅवो वो॒ दक्षि॑णा॒म् न न दक्षि॑णां ॅवो वो॒ दक्षि॑णा॒म् न । \newline
40. दक्षि॑णा॒म् न न दक्षि॑णा॒म् दक्षि॑णा॒म् न नी॑निम नीनिम॒ न दक्षि॑णा॒म् दक्षि॑णा॒म् न नी॑निम । \newline
41. न नी॑निम नीनिम॒ न न नी॑निम॒ मा मा नी॑निम॒ न न नी॑निम॒ मा । \newline
42. नी॒नि॒म॒ मा मा नी॑निम नीनिम॒ मा नो॑ नो॒ मा नी॑निम नीनिम॒ मा नः॑ । \newline
43. मा नो॑ नो॒ मा मा न॒ स्तस्मा॒त् तस्मा᳚न् नो॒ मा मा न॒ स्तस्मा᳚त् । \newline
44. न॒ स्तस्मा॒त् तस्मा᳚न् नो न॒ स्तस्मा॒ देन॑स॒ एन॑स॒ स्तस्मा᳚न् नो न॒ स्तस्मा॒ देन॑सः । \newline
45. तस्मा॒ देन॑स॒ एन॑स॒ स्तस्मा॒त् तस्मा॒ देन॑सः पापयिष्ट पापयि॒ ष्टैन॑स॒ स्तस्मा॒त् तस्मा॒ देन॑सः पापयिष्ट । \newline
46. एन॑सः पापयिष्ट पापयि॒ ष्टैन॑स॒ एन॑सः पापयिष्ट । \newline
47. पा॒प॒यि॒ष्टेति॑ पापयिष्ट । \newline
48. याव॑न्तो॒ वै वै याव॑न्तो॒ याव॑न्तो॒ वै स॑द॒स्याः᳚ सद॒स्या॑ वै याव॑न्तो॒ याव॑न्तो॒ वै स॑द॒स्याः᳚ । \newline
49. वै स॑द॒स्याः᳚ सद॒स्या॑ वै वै स॑द॒स्या᳚ स्ते ते स॑द॒स्या॑ वै वै स॑द॒स्या᳚ स्ते । \newline
50. स॒द॒स्या᳚ स्ते ते स॑द॒स्याः᳚ सद॒स्या᳚ स्ते सर्वे॒ सर्वे॒ ते स॑द॒स्याः᳚ सद॒स्या᳚ स्ते सर्वे᳚ । \newline
51. ते सर्वे॒ सर्वे॒ ते ते सर्वे॑ दक्षि॒ण्या॑ दक्षि॒ण्याः᳚ सर्वे॒ ते ते सर्वे॑ दक्षि॒ण्याः᳚ । \newline
52. सर्वे॑ दक्षि॒ण्या॑ दक्षि॒ण्याः᳚ सर्वे॒ सर्वे॑ दक्षि॒ण्या᳚ स्तेभ्य॒ स्तेभ्यो॑ दक्षि॒ण्याः᳚ सर्वे॒ सर्वे॑ दक्षि॒ण्या᳚ स्तेभ्यः॑ । \newline
53. द॒क्षि॒ण्या᳚ स्तेभ्य॒ स्तेभ्यो॑ दक्षि॒ण्या॑ दक्षि॒ण्या᳚ स्तेभ्यो॒ यो य स्तेभ्यो॑ दक्षि॒ण्या॑ दक्षि॒ण्या᳚ स्तेभ्यो॒ यः । \newline
54. तेभ्यो॒ यो यस्तेभ्य॒ स्तेभ्यो॒ यो दक्षि॑णा॒म् दक्षि॑णां॒ ॅय स्तेभ्य॒ स्तेभ्यो॒ यो दक्षि॑णाम् । \newline
55. यो दक्षि॑णा॒म् दक्षि॑णां॒ ॅयो यो दक्षि॑णा॒म् न न दक्षि॑णां॒ ॅयो यो दक्षि॑णा॒म् न । \newline
56. दक्षि॑णा॒म् न न दक्षि॑णा॒म् दक्षि॑णा॒म् न नये॒न् नये॒न् न दक्षि॑णा॒म् दक्षि॑णा॒म् न नये᳚त् । \newline
57. न नये॒न् नये॒न् न न नये॒दा नये॒न् न न नये॒दा । \newline
\pagebreak
\markright{ TS 3.2.8.4  \hfill https://www.vedavms.in \hfill}

\section{ TS 3.2.8.4 }

\textbf{TS 3.2.8.4 } \newline
\textbf{Samhita Paata} \newline

नये॒दैभ्यो॑ वृश्च्येत॒ यद्-वै᳚श्वकर्म॒णानि॑ जु॒होति॑ सद॒स्या॑ने॒व तत् प्री॑णात्य॒स्मे दे॑वासो॒ वपु॑षे चिकिथ्सत॒ यमा॒शिरा॒ दम्प॑ती वा॒मम॑श्ञु॒तः । पुमा᳚न् पु॒त्रो जा॑यते वि॒न्दते॒ वस्वथ॒ विश्वे॑ अर॒पा ए॑धते गृ॒हः ॥ आ॒शी॒र्दा॒या दम्प॑ती वा॒मम॑श्ञुता॒मरि॑ष्टो॒ रायः॑ सचताꣳ॒॒ समो॑कसा । य आऽसि॑च॒थ् सं दु॑ग्धं कु॒म्भ्या स॒हेष्टेन॒ याम॒न्नम॑तिं जहातु॒ सः ॥ स॒र्पि॒र्ग्री॒वी - [  ] \newline

\textbf{Pada Paata} \newline

नये᳚त् । एति॑ । ए॒भ्यः॒ । वृ॒श्च्ये॒त॒ । यत् । वै॒श्व॒क॒र्म॒णानीति॑ वैश्व - क॒र्म॒णानि॑ । जु॒होति॑ । स॒द॒स्यान्॑ । ए॒व । तत् । प्री॒णा॒ति॒ । अ॒स्मे इति॑ । दे॒वा॒सः॒ । वपु॑षे । चि॒कि॒थ्स॒त॒ । यम् । आ॒शिरा᳚ । दंप॑ती॒ इति॑ । वा॒मम् । अ॒श्नु॒तः ॥ पुमान्॑ । पु॒त्रः । जा॒य॒ते॒ । वि॒न्दते᳚ । वसु॑ । अथ॑ । विश्वे᳚ । अ॒र॒पाः । ए॒ध॒ते॒ । गृ॒हः ॥ आ॒शी॒र्दा॒येत्या॑शीः - दा॒या । दंप॑ती॒ इति॑ । वा॒मम् । अ॒श्नु॒ता॒म् । अरि॑ष्टः । रायः॑ । स॒च॒ता॒म् । समो॑क॒सेति॒ सं - ओ॒क॒सा॒ ॥ यः । एति॑ । असि॑चत् । संदु॑ग्ध॒मिति॒ सं - दु॒ग्ध॒म् । कु॒भ्यां । स॒ह । इ॒ष्टेन॑ । यामन्न्॑ । अम॑तिम् । ज॒हा॒तु॒ । सः ॥ स॒र्पि॒र्ग्री॒वीति॑ सर्पिः - ग्री॒वी ।  \newline


\textbf{Krama Paata} \newline

नये॒दा । ऐभ्यः॑ । ए॒भ्यो॒ वृ॒श्च्ये॒त॒ । वृ॒श्च्ये॒त॒ यत् । यद् वै᳚श्वकर्म॒णानि॑ । वै॒श्व॒क॒र्म॒णानि॑ जु॒होति॑ । वै॒श्व॒क॒र्म॒णानीति॑ वैश्व - क॒र्म॒णानि॑ । जु॒होति॑ सद॒स्यान्॑ । स॒द॒स्या॑ने॒व । ए॒व तत् । तत् प्री॑णाति । प्री॒णा॒त्य॒स्मे । अ॒स्मे दे॑वासः । अ॒स्मे इत्य॒स्मे । दे॒वा॒सो॒ वपु॑षे । वपु॑षे चिकिथ्सत । चि॒कि॒थ्स॒त॒ यम् । यमा॒शिरा᳚ । आ॒शिरा॒ दम्प॑ती । दम्प॑ती वा॒मम् । दम्प॑ती॒ इति॒ दम्प॑ती । वा॒मम॑श्ञु॒तः । अ॒श्ञु॒त इत्य॑श्ञु॒तः ॥ पुमा᳚न् पु॒त्रः । पु॒त्रो जा॑यते । जा॒य॒ते॒ वि॒न्दते᳚ । वि॒न्दते॒ वसु॑ । वस्वथ॑ । अथ॒ विश्वे᳚ । विश्वे॑ अर॒पाः । अ॒र॒पा ए॑धते । ए॒ध॒ते॒ गृ॒हः । गृ॒ह इति॑ गृ॒हः ॥ आ॒शी॒र्दा॒या दम्प॑ती । आ॒शी॒र्दा॒येत्या॑शीः - दा॒या । दम्प॑ती वा॒मम् । दम्प॑ती॒ इति॒ दम्प॑ती । वा॒मम॑श्ञुताम् । अ॒श्ञु॒ता॒मरि॑ष्टः । अरि॑ष्टो॒ रायः॑ । रायः॑ सचताम् । स॒च॒ताꣳ॒॒ समो॑कसा । समो॑क॒सेति॒ सम् - ओ॒क॒सा॒ ॥ य आ । आ ऽसि॑चत् । असि॑च॒थ् सन्दु॑ग्धम् । सन्दु॑ग्धम् कु॒म्भ्या । सन्दु॑ग्ध॒मिति॒ सम् - दु॒ग्ध॒म् । कु॒म्भ्या स॒ह । स॒हेष्टेन॑ । इ॒ष्टेन॒ यामन्न्॑ । याम॒न्नम॑तिम् । अम॑तिम् जहातु । ज॒हा॒तु॒ सः । स इति॒ सः ॥ स॒र्पि॒र्ग्री॒वी पीव॑री । स॒र्पि॒र्ग्री॒वीति॑ सर्पिः - ग्री॒वी \newline

\textbf{Jatai Paata} \newline

1. नये॒दा नये॒न् नये॒दा । \newline
2. ऐभ्य॑ एभ्य॒ ऐभ्यः॑ । \newline
3. ए॒भ्यो॒ वृ॒श्च्ये॒त॒ वृ॒श्च्ये॒ तै॒भ्य॒ ए॒भ्यो॒ वृ॒श्च्ये॒त॒ । \newline
4. वृ॒श्च्ये॒त॒ यद् यद् वृ॑श्च्येत वृश्च्येत॒ यत् । \newline
5. यद् वै᳚श्वकर्म॒णानि॑ वैश्वकर्म॒णानि॒ यद् यद् वै᳚श्वकर्म॒णानि॑ । \newline
6. वै॒श्व॒क॒र्म॒णानि॑ जु॒होति॑ जु॒होति॑ वैश्वकर्म॒णानि॑ वैश्वकर्म॒णानि॑ जु॒होति॑ । \newline
7. वै॒श्व॒क॒र्म॒णानीति॑ वैश्व - क॒र्म॒णानि॑ । \newline
8. जु॒होति॑ सद॒स्या᳚न् थ्सद॒स्या᳚न् जु॒होति॑ जु॒होति॑ सद॒स्यान्॑ । \newline
9. स॒द॒स्या॑ ने॒वैव स॑द॒स्या᳚न् थ्सद॒स्या॑ ने॒व । \newline
10. ए॒व तत् तदे॒वैव तत् । \newline
11. तत् प्री॑णाति प्रीणाति॒ तत् तत् प्री॑णाति । \newline
12. प्री॒णा॒ त्य॒स्मे अ॒स्मे प्री॑णाति प्रीणा त्य॒स्मे । \newline
13. अ॒स्मे दे॑वासो देवासो अ॒स्मे अ॒स्मे दे॑वासः । \newline
14. अ॒स्मे इत्य॒स्मे । \newline
15. दे॒वा॒सो॒ वपु॑षे॒ वपु॑षे देवासो देवासो॒ वपु॑षे । \newline
16. वपु॑षे चिकिथ्सत चिकिथ्सत॒ वपु॑षे॒ वपु॑षे चिकिथ्सत । \newline
17. चि॒कि॒थ्स॒त॒ यं ॅयम् चि॑किथ्सत चिकिथ्सत॒ यम् । \newline
18. य मा॒शिरा॒ ऽऽशिरा॒ यं ॅय मा॒शिरा᳚ । \newline
19. आ॒शिरा॒ दंप॑ती॒ दंप॑ती आ॒शिरा॒ ऽऽशिरा॒ दंप॑ती । \newline
20. दंप॑ती वा॒मं ॅवा॒मम् दंप॑ती॒ दंप॑ती वा॒मम् । \newline
21. दंप॑ती॒ इति॒ दंप॑ती । \newline
22. वा॒म म॑श्ञु॒तो अ॑श्ञु॒तो वा॒मं ॅवा॒म म॑श्ञु॒तः । \newline
23. अ॒श्ञु॒त इत्य॑श्ञु॒तः । \newline
24. पुमा᳚न् पु॒त्रः पु॒त्रः पुमा॒न् पुमा᳚न् पु॒त्रः । \newline
25. पु॒त्रो जा॑यते जायते पु॒त्रः पु॒त्रो जा॑यते । \newline
26. जा॒य॒ते॒ वि॒न्दते॑ वि॒न्दते॑ जायते जायते वि॒न्दते᳚ । \newline
27. वि॒न्दते॒ वसु॒ वसु॑ वि॒न्दते॑ वि॒न्दते॒ वसु॑ । \newline
28. वस्वथा थ॒ वसु॒ वस्वथ॑ । \newline
29. अथ॒ विश्वे॒ विश्वे॒ अथाथ॒ विश्वे᳚ । \newline
30. विश्वे॑ अर॒पा अ॑र॒पा विश्वे॒ विश्वे॑ अर॒पाः । \newline
31. अ॒र॒पा ए॑धत एधते अर॒पा अ॑र॒पा ए॑धते । \newline
32. ए॒ध॒ते॒ गृ॒हो गृ॒ह ए॑धत एधते गृ॒हः । \newline
33. गृ॒ह इति॑ गृ॒हः । \newline
34. आ॒शी॒र्दा॒या दंप॑ती॒ दंप॑ती आशीर्दा॒या ऽऽशी᳚र्दा॒या दंप॑ती । \newline
35. आ॒शी॒र्दा॒येत्या॑शीः - दा॒या । \newline
36. दंप॑ती वा॒मं ॅवा॒मम् दंप॑ती॒ दंप॑ती वा॒मम् । \newline
37. दंप॑ती॒ इति॒ दंप॑ती । \newline
38. वा॒म म॑श्ञुता मश्ञुतां ॅवा॒मं ॅवा॒म म॑श्ञुताम् । \newline
39. अ॒श्ञु॒ता॒ मरि॑ष्टो॒ अरि॑ष्टो अश्ञुता मश्ञुता॒ मरि॑ष्टः । \newline
40. अरि॑ष्टो॒ रायो॒ रायो॒ अरि॑ष्टो॒ अरि॑ष्टो॒ रायः॑ । \newline
41. रायः॑ सचताꣳ सचताꣳ॒॒ रायो॒ रायः॑ सचताम् । \newline
42. स॒च॒ताꣳ॒॒ समो॑कसा॒ समो॑कसा सचताꣳ सचताꣳ॒॒ समो॑कसा । \newline
43. समो॑क॒सेति॒ सं - ओ॒क॒सा॒ । \newline
44. य आ यो य आ । \newline
45. आ ऽसि॑च॒ दसि॑च॒दा ऽसि॑चत् । \newline
46. असि॑च॒थ् सन्दु॑ग्धꣳ॒॒ सन्दु॑ग्ध॒ मसि॑च॒ दसि॑च॒थ् सन्दु॑ग्धम् । \newline
47. सन्दु॑ग्धम् कु॒म्भ्या कु॒म्भ्या सन्दु॑ग्धꣳ॒॒ सन्दु॑ग्धम् कु॒म्भ्या । \newline
48. सन्दु॑ग्ध॒मिति॒ सं - दु॒ग्ध॒म् । \newline
49. कु॒म्भ्या स॒ह स॒ह कु॒म्भ्या कु॒म्भ्या स॒ह । \newline
50. स॒हे ष्टेने॒ ष्टेन॑ स॒ह स॒हे ष्टेन॑ । \newline
51. इ॒ष्टेन॒ याम॒न्॒. याम॑न् नि॒ष्टेने॒ ष्टेन॒ यामन्न्॑ । \newline
52. याम॒न् नम॑ति॒ मम॑तिं॒ ॅयाम॒न्॒. याम॒न् नम॑तिम् । \newline
53. अम॑तिम् जहातु जहा॒ त्वम॑ति॒ मम॑तिम् जहातु । \newline
54. ज॒हा॒तु॒ स स ज॑हातु जहातु॒ सः । \newline
55. स इति॒ सः । \newline
56. स॒र्पि॒र्ग्री॒वी पीव॑री॒ पीव॑री सर्पिर्ग्री॒वी स॑र्पिर्ग्री॒वी पीव॑री । \newline
57. स॒र्पि॒र्ग्री॒वीति॑ सर्पिः - ग्री॒वी । \newline

\textbf{Ghana Paata } \newline

1. नये॒दा नये॒न् नये॒ दैभ्य॑ एभ्य॒ आ नये॒न् नये॒ दैभ्यः॑ । \newline
2. ऐभ्य॑ एभ्य॒ ऐभ्यो॑ वृश्च्येत वृश्च्ये तैभ्य॒ ऐभ्यो॑ वृश्च्येत । \newline
3. ए॒भ्यो॒ वृ॒श्च्ये॒त॒ वृ॒श्च्ये॒ तै॒भ्य॒ ए॒भ्यो॒ वृ॒श्च्ये॒त॒ यद् यद् वृ॑श्च्ये तैभ्य एभ्यो वृश्च्येत॒ यत् । \newline
4. वृ॒श्च्ये॒त॒ यद् यद् वृ॑श्च्येत वृश्च्येत॒ यद् वै᳚श्वकर्म॒णानि॑ वैश्वकर्म॒णानि॒ यद् वृ॑श्च्येत वृश्च्येत॒ यद् वै᳚श्वकर्म॒णानि॑ । \newline
5. यद् वै᳚श्वकर्म॒णानि॑ वैश्वकर्म॒णानि॒ यद् यद् वै᳚श्वकर्म॒णानि॑ जु॒होति॑ जु॒होति॑ वैश्वकर्म॒णानि॒ यद् यद् वै᳚श्वकर्म॒णानि॑ जु॒होति॑ । \newline
6. वै॒श्व॒क॒र्म॒णानि॑ जु॒होति॑ जु॒होति॑ वैश्वकर्म॒णानि॑ वैश्वकर्म॒णानि॑ जु॒होति॑ सद॒स्या᳚न् थ्सद॒स्या᳚न् जु॒होति॑ वैश्वकर्म॒णानि॑ वैश्वकर्म॒णानि॑ जु॒होति॑ सद॒स्यान्॑ । \newline
7. वै॒श्व॒क॒र्म॒णानीति॑ वैश्व - क॒र्म॒णानि॑ । \newline
8. जु॒होति॑ सद॒स्या᳚न् थ्सद॒स्या᳚न् जु॒होति॑ जु॒होति॑ सद॒स्या॑ ने॒वैव स॑द॒स्या᳚न् जु॒होति॑ जु॒होति॑ सद॒स्या॑ ने॒व । \newline
9. स॒द॒स्या॑ ने॒वैव स॑द॒स्या᳚न् थ्सद॒स्या॑ ने॒व तत् तदे॒व स॑द॒स्या᳚न् थ्सद॒स्या॑ ने॒व तत् । \newline
10. ए॒व तत् तदे॒वैव तत् प्री॑णाति प्रीणाति॒ तदे॒वैव तत् प्री॑णाति । \newline
11. तत् प्री॑णाति प्रीणाति॒ तत् तत् प्री॑णा त्य॒स्मे अ॒स्मे प्री॑णाति॒ तत् तत् प्री॑णा त्य॒स्मे । \newline
12. प्री॒णा॒ त्य॒स्मे अ॒स्मे प्री॑णाति प्रीणा त्य॒स्मे दे॑वासो देवासो अ॒स्मे प्री॑णाति प्रीणा त्य॒स्मे दे॑वासः । \newline
13. अ॒स्मे दे॑वासो देवासो अ॒स्मे अ॒स्मे दे॑वासो॒ वपु॑षे॒ वपु॑षे देवासो अ॒स्मे अ॒स्मे दे॑वासो॒ वपु॑षे । \newline
14. अ॒स्मे इत्य॒स्मे । \newline
15. दे॒वा॒सो॒ वपु॑षे॒ वपु॑षे देवासो देवासो॒ वपु॑षे चिकिथ्सत चिकिथ्सत॒ वपु॑षे देवासो देवासो॒ वपु॑षे चिकिथ्सत । \newline
16. वपु॑षे चिकिथ्सत चिकिथ्सत॒ वपु॑षे॒ वपु॑षे चिकिथ्सत॒ यं ॅयम् चि॑किथ्सत॒ वपु॑षे॒ वपु॑षे चिकिथ्सत॒ यम् । \newline
17. चि॒कि॒थ्स॒त॒ यं ॅयम् चि॑किथ्सत चिकिथ्सत॒ य मा॒शिरा॒ ऽऽशिरा॒ यम् चि॑किथ्सत चिकिथ्सत॒ य मा॒शिरा᳚ । \newline
18. य मा॒शिरा॒ ऽऽशिरा॒ यं ॅय मा॒शिरा॒ दंप॑ती॒ दंप॑ती आ॒शिरा॒ यं ॅय मा॒शिरा॒ दंप॑ती । \newline
19. आ॒शिरा॒ दंप॑ती॒ दंप॑ती आ॒शिरा॒ ऽऽशिरा॒ दंप॑ती वा॒मं ॅवा॒मम् दंप॑ती आ॒शिरा॒ ऽऽशिरा॒ दंप॑ती वा॒मम् । \newline
20. दंप॑ती वा॒मं ॅवा॒मम् दंप॑ती॒ दंप॑ती वा॒म म॑श्ञु॒तो अ॑श्ञु॒तो वा॒मम् दंप॑ती॒ दंप॑ती वा॒म म॑श्ञु॒तः । \newline
21. दंप॑ती॒ इति॒ दंप॑ती । \newline
22. वा॒म म॑श्ञु॒तो अ॑श्ञु॒तो वा॒मं ॅवा॒म म॑श्ञु॒तः । \newline
23. अ॒श्ञु॒त इत्य॑श्ञु॒तः । \newline
24. पुमा᳚न् पु॒त्रः पु॒त्रः पुमा॒न् पुमा᳚न् पु॒त्रो जा॑यते जायते पु॒त्रः पुमा॒न् पुमा᳚न् पु॒त्रो जा॑यते । \newline
25. पु॒त्रो जा॑यते जायते पु॒त्रः पु॒त्रो जा॑यते वि॒न्दते॑ वि॒न्दते॑ जायते पु॒त्रः पु॒त्रो जा॑यते वि॒न्दते᳚ । \newline
26. जा॒य॒ते॒ वि॒न्दते॑ वि॒न्दते॑ जायते जायते वि॒न्दते॒ वसु॒ वसु॑ वि॒न्दते॑ जायते जायते वि॒न्दते॒ वसु॑ । \newline
27. वि॒न्दते॒ वसु॒ वसु॑ वि॒न्दते॑ वि॒न्दते॒ वस्वथाथ॒ वसु॑ वि॒न्दते॑ वि॒न्दते॒ वस्वथ॑ । \newline
28. वस्वथाथ॒ वसु॒ वस्वथ॒ विश्वे॒ विश्वे॒ अथ॒ वसु॒ वस्वथ॒ विश्वे᳚ । \newline
29. अथ॒ विश्वे॒ विश्वे॒ अथाथ॒ विश्वे॑ अर॒पा अ॑र॒पा विश्वे॒ अथाथ॒ विश्वे॑ अर॒पाः । \newline
30. विश्वे॑ अर॒पा अ॑र॒पा विश्वे॒ विश्वे॑ अर॒पा ए॑धत एधते अर॒पा विश्वे॒ विश्वे॑ अर॒पा ए॑धते । \newline
31. अ॒र॒पा ए॑धत एधते अर॒पा अ॑र॒पा ए॑धते गृ॒हो गृ॒ह ए॑धते अर॒पा अ॑र॒पा ए॑धते गृ॒हः । \newline
32. ए॒ध॒ते॒ गृ॒हो गृ॒ह ए॑धत एधते गृ॒हः । \newline
33. गृ॒ह इति॑ गृ॒हः । \newline
34. आ॒शी॒र्दा॒या दंप॑ती॒ दंप॑ती आशीर्दा॒या ऽऽशी᳚र्दा॒या दंप॑ती वा॒मं ॅवा॒मम् दंप॑ती आशीर्दा॒या ऽऽशी᳚र्दा॒या दंप॑ती वा॒मम् । \newline
35. आ॒शी॒र्दा॒येत्या॑शीः - दा॒या । \newline
36. दंप॑ती वा॒मं ॅवा॒मम् दंप॑ती॒ दंप॑ती वा॒म म॑श्ञुता मश्ञुतां ॅवा॒मम् दंप॑ती॒ दंप॑ती वा॒म म॑श्ञुताम् । \newline
37. दंप॑ती॒ इति॒ दंप॑ती । \newline
38. वा॒म म॑श्ञुता मश्ञुतां ॅवा॒मं ॅवा॒म म॑श्ञुता॒ मरि॑ष्टो॒ अरि॑ष्टो अश्ञुतां ॅवा॒मं ॅवा॒म म॑श्ञुता॒ मरि॑ष्टः । \newline
39. अ॒श्ञु॒ता॒ मरि॑ष्टो॒ अरि॑ष्टो अश्ञुता मश्ञुता॒ मरि॑ष्टो॒ रायो॒ रायो॒ अरि॑ष्टो अश्ञुता मश्ञुता॒ मरि॑ष्टो॒ रायः॑ । \newline
40. अरि॑ष्टो॒ रायो॒ रायो॒ अरि॑ष्टो॒ अरि॑ष्टो॒ रायः॑ सचताꣳ सचताꣳ॒॒ रायो॒ अरि॑ष्टो॒ अरि॑ष्टो॒ रायः॑ सचताम् । \newline
41. रायः॑ सचताꣳ सचताꣳ॒॒ रायो॒ रायः॑ सचताꣳ॒॒ समो॑कसा॒ समो॑कसा सचताꣳ॒॒ रायो॒ रायः॑ सचताꣳ॒॒ समो॑कसा । \newline
42. स॒च॒ताꣳ॒॒ समो॑कसा॒ समो॑कसा सचताꣳ सचताꣳ॒॒ समो॑कसा । \newline
43. समो॑क॒सेति॒ सं - ओ॒क॒सा॒ । \newline
44. य आ यो य आ ऽसि॑च॒ दसि॑च॒ दा यो य आ ऽसि॑चत् । \newline
45. आ ऽसि॑च॒ दसि॑च॒दा ऽसि॑च॒थ् सन्दु॑ग्धꣳ॒॒ सन्दु॑ग्ध॒ मसि॑च॒दा ऽसि॑च॒थ् सन्दु॑ग्धम् । \newline
46. असि॑च॒थ् सन्दु॑ग्धꣳ॒॒ सन्दु॑ग्ध॒ मसि॑च॒ दसि॑च॒थ् सन्दु॑ग्धम् कुं॒भ्या कुं॒भ्या सन्दु॑ग्ध॒ मसि॑च॒ दसि॑च॒थ् सन्दु॑ग्धम् कुं॒भ्या । \newline
47. सन्दु॑ग्धम् कुं॒भ्या कुं॒भ्या सन्दु॑ग्धꣳ॒॒ सन्दु॑ग्धम् कुं॒भ्या स॒ह स॒ह कुं॒भ्या सन्दु॑ग्धꣳ॒॒ सन्दु॑ग्धम् कुं॒भ्या स॒ह । \newline
48. सन्दु॑ग्ध॒मिति॒ सं - दु॒ग्ध॒म् । \newline
49. कुं॒भ्या स॒ह स॒ह कुं॒भ्या कुं॒भ्या स॒हे ष्टेने॒ ष्टेन॑ स॒ह कुं॒भ्या कुं॒भ्या स॒हे ष्टेन॑ । \newline
50. स॒हे ष्टेने॒ ष्टेन॑ स॒ह स॒हे ष्टेन॒ याम॒न्॒. याम॑न् नि॒ष्टेन॑ स॒ह स॒हे ष्टेन॒ यामन्न्॑ । \newline
51. इ॒ष्टेन॒ याम॒न्॒. याम॑न् नि॒ष्टेने॒ ष्टेन॒ याम॒न् नम॑ति॒ मम॑तिं॒ ॅयाम॑न् नि॒ष्टेने॒ ष्टेन॒ याम॒न् नम॑तिम् । \newline
52. याम॒न् नम॑ति॒ मम॑तिं॒ ॅयाम॒न्॒. याम॒न् नम॑तिम् जहातु जहा॒ त्वम॑तिं॒ ॅयाम॒न्॒. याम॒न् नम॑तिम् जहातु । \newline
53. अम॑तिम् जहातु जहा॒ त्वम॑ति॒ मम॑तिम् जहातु॒ स स ज॑हा॒ त्वम॑ति॒ मम॑तिम् जहातु॒ सः । \newline
54. ज॒हा॒तु॒ स स ज॑हातु जहातु॒ सः । \newline
55. स इति॒ सः । \newline
56. स॒र्पि॒र्ग्री॒वी पीव॑री॒ पीव॑री सर्पिर्ग्री॒वी स॑र्पिर्ग्री॒वी पीव॑र्यस्यास्य॒ पीव॑री सर्पिर्ग्री॒वी स॑र्पिर्ग्री॒वी पीव॑र्यस्य । \newline
57. स॒र्पि॒र्ग्री॒वीति॑ सर्पिः - ग्री॒वी । \newline
\pagebreak
\markright{ TS 3.2.8.5  \hfill https://www.vedavms.in \hfill}

\section{ TS 3.2.8.5 }

\textbf{TS 3.2.8.5 } \newline
\textbf{Samhita Paata} \newline

पीव॑र्यस्य जा॒या पीवा॑नः पु॒त्रा अकृ॑शासो अस्य । स॒हजा॑नि॒र्यः सु॑मख॒स्यमा॑न॒ इन्द्रा॑या॒ऽऽ*शिरꣳ॑ स॒ह कु॒म्भ्याऽदा᳚त् ॥ आ॒शीर्म॒ ऊर्ज॑मु॒त सु॑प्रजा॒स्त्वमिषं॑ दधातु॒ द्रवि॑णꣳ॒॒ सव॑र्चसं । सं॒ जय॒न् क्षेत्रा॑णि॒ सह॑सा॒ऽहमि॑न्द्र कृण्वा॒नो अ॒न्याꣳ अध॑रान्थ्स॒पत्नान्॑ ॥ भू॒तम॑सि भू॒ते मा॑ धा॒ मुख॑मसि॒ मुखं॑ भूयासं॒ द्यावा॑पृथि॒वीभ्यां᳚ त्वा॒ परि॑गृह्णामि॒ विश्वे᳚ त्वा दे॒वा वै᳚श्वान॒राः - [  ] \newline

\textbf{Pada Paata} \newline

पीव॑री । अ॒स्य॒ । जा॒या । पीवा॑नः । पु॒त्राः । अकृ॑शासः । अ॒स्य॒ ॥ स॒हजा॑नि॒रिति॑ स॒ह - जा॒निः॒ । यः । सु॒म॒ख॒स्यमा॑न॒ इति॑ सु - म॒ख॒स्यमा॑नः । इन्द्रा॑य । आ॒शिर᳚म् । स॒ह । कु॒भ्यां । अदा᳚त् ॥ आ॒शीरित्या᳚ - शीः । मे॒ । ऊर्ज᳚म् । उ॒त । सु॒प्र॒जा॒स्त्वमिति॑ सुप्रजाः - त्वम् । इष᳚म् । द॒धा॒तु॒ । द्रवि॑णम् । सव॑र्चस॒मिति॒ स - व॒र्च॒स॒म् ॥ स॒जंय॒न्निति ॑ सं - जयन्न्॑ । क्षेत्रा॑णि । सह॑सा । अ॒हम् । इ॒न्द्र॒ । कृ॒ण्वा॒नः । अ॒न्यान् । अध॑रान् । स॒पत्नान्॑ ॥ भू॒तम् । अ॒सि॒ । भू॒ते । मा॒ । धाः॒ । मुख᳚म् । अ॒सि॒ । मुख᳚म् । भू॒या॒स॒म् । द्यावा॑पृथि॒वीभ्या॒मिति॒ द्यावा᳚ - पृ॒थि॒वीभ्या᳚म् । त्वा॒ । परीति॑ । गृ॒ह्णा॒मि॒ । विश्वे᳚ । त्वा॒ । दे॒वाः । वै॒श्वा॒न॒राः ।  \newline


\textbf{Krama Paata} \newline

पीव॑र्यस्य । अ॒स्य॒ जा॒या । जा॒या पीवा॑नः । पीवा॑नः पु॒त्राः । पु॒त्रा अकृ॑शासः । अकृ॑शासो अस्य । अ॒स्येत्य॑स्य ॥ स॒हजा॑नि॒र् यः । स॒हजा॑नि॒रिति॑ स॒ह - जा॒निः॒ । यः सु॑मख॒स्यमा॑नः । सु॒म॒ख॒स्यमा॑न॒ इन्द्रा॑य । सु॒म॒ख॒स्यमा॑न॒ इति॑ सु - म॒ख॒स्यमा॑नः । इन्द्रा॑या॒शिर᳚म् । आ॒शिरꣳ॑ स॒ह । स॒ह कु॒म्भ्या । कु॒म्भ्या ऽदा᳚त् । अदा॒दित्यदा᳚त् ॥ आ॒शीर् मे᳚ । आ॒शीरित्या᳚ - शीः । म॒ ऊर्ज᳚म् । ऊर्ज॑मु॒त । उ॒त सु॑प्रजा॒स्त्वम् । सु॒प्र॒जा॒,स्त्वमिष᳚म् । सु॒प्र॒जा॒स्त्वमिति॑ सुप्रजाः - त्वम् । इष॑म् दधातु । द॒धा॒तु॒ द्रवि॑णम् । द्रवि॑णꣳ॒॒ सव॑र्चसम् । सव॑र्चस॒मिति॒ स - व॒र्च॒स॒म् ॥ स॒ञ्जय॒न् क्षेत्रा॑णि । स॒ञ्जय॒न्निति॑ सम् - जयन्न्॑ । क्षेत्रा॑णि॒ सह॑सा । सह॑सा॒ ऽहम् । अ॒हमि॑न्द्र । इ॒न्द्र॒ कृ॒ण्वा॒नः । कृ॒ण्वा॒नो अ॒न्यान् । अ॒न्याꣳ अध॑रान् । अध॑रान्थ् स॒पत्नान्॑ । स॒पत्ना॒निति॑ स॒पत्नान्॑ ॥ भू॒तम॑सि । अ॒सि॒ भू॒ते । भू॒ते मा᳚ । मा॒ धाः॒ । धा॒ मुख᳚म् । मुख॑मसि । अ॒सि॒ मुख᳚म् । मुख॑म् भूयासम् । भू॒या॒स॒म् द्यावा॑पृथि॒वीभ्या᳚म् । द्यावा॑पृथि॒वीभ्या᳚म् त्वा । द्यावा॑पृथि॒वीभ्या॒मिति॒ द्यावा᳚ - पृ॒थि॒वीभ्या᳚म् । त्वा॒ परि॑ । परि॑ गृह्णामि । गृ॒ह्णा॒मि॒ विश्वे᳚ । विश्वे᳚ त्वा । त्वा॒ दे॒वाः । दे॒वा वै᳚श्वान॒राः ( ) । वै॒श्वा॒न॒राः प्र \newline
प्र च्या॑वयन्तु । च्या॒व॒य॒न्तु॒ दि॒वि । दि॒वि दे॒वान् । दे॒वान् दृꣳ॑ह । दृꣳ॒॒हा॒न्तरि॑क्षे । अ॒न्तरि॑क्षे॒ वयाꣳ॑सि । वयाꣳ॑सि पृथि॒व्याम् । पृ॒थि॒व्याम् पार्त्थि॑वान् । पार्त्थि॑वान् ध्रु॒वम् । ध्रु॒वम् ध्रु॒वेण॑ । ध्रु॒वेण॑ ह॒विषा᳚ । ह॒विषा ऽव॑ । अव॒ सोम᳚म् । सोम॑म् नयामसि । न॒या॒म॒सीति॑ नयामसि ॥ यथा॑ नः । नः॒ सर्व᳚म् । सर्व॒मित् । इज् जग॑त् । जग॑दय॒क्ष्मम् । अ॒य॒क्ष्मꣳ सु॒मनाः᳚ । सु॒मना॒ अस॑त् । सु॒मना॒ इति॑ सु - मनाः᳚ । अस॒दित्यस॑त् ॥ यथा॑ नः । न॒ इन्द्रः॑ । इन्द्र॒ इत् । इद् विशः॑ । विशः॒ केव॑लीः । केव॑लीः॒ सर्वाः᳚ । सर्वाः॒ सम॑नसः । सम॑नसः॒ कर॑त् । सम॑नस॒ इति॒ स - म॒न॒सः॒ । कर॒दिति॒ कर॑त् ॥ यथा॑ नः । नः॒ सर्वाः᳚ । सर्वा॒ इत् । इद् दिशः॑ । दिशो॒ ऽस्माक᳚म् । अ॒स्माक॒म् केव॑लीः । केव॑ली॒रसन्न्॑ । अस॒न्नित्यसन्न्॑ । \newline

\textbf{Jatai Paata} \newline

1. पीव॑र्यस्यास्य॒ पीव॑री॒ पीव॑र्यस्य । \newline
2. अ॒स्य॒ जा॒या जा॒या ऽस्या᳚स्य जा॒या । \newline
3. जा॒या पीवा॑नः॒ पीवा॑नो जा॒या जा॒या पीवा॑नः । \newline
4. पीवा॑नः पु॒त्राः पु॒त्राः पीवा॑नः॒ पीवा॑नः पु॒त्राः । \newline
5. पु॒त्रा अकृ॑शासो॒ अकृ॑शासः पु॒त्राः पु॒त्रा अकृ॑शासः । \newline
6. अकृ॑शासो अस्या॒स्या कृ॑शासो॒ अकृ॑शासो अस्य । \newline
7. अ॒स्येत्य॑स्य । \newline
8. स॒हजा॑नि॒र् यो यः स॒हजा॑निः स॒हजा॑नि॒र् यः । \newline
9. स॒हजा॑नि॒रिति॑ स॒ह - जा॒निः॒ । \newline
10. यः सु॑मख॒स्यमा॑नः सुमख॒स्यमा॑नो॒ यो यः सु॑मख॒स्यमा॑नः । \newline
11. सु॒म॒ख॒स्यमा॑न॒ इन्द्रा॒ येन्द्रा॑य सुमख॒स्यमा॑नः सुमख॒स्यमा॑न॒ इन्द्रा॑य । \newline
12. सु॒म॒ख॒स्यमा॑न॒ इति॑ सु - म॒ख॒स्यमा॑नः । \newline
13. इन्द्रा॑या॒ शिर॑ मा॒शिर॒ मिन्द्रा॒ येन्द्रा॑या॒ शिर᳚म् । \newline
14. आ॒शिरꣳ॑ स॒ह स॒हा शिर॑ मा॒शिरꣳ॑ स॒ह । \newline
15. स॒ह कु॒म्भ्या कु॒म्भ्या स॒ह स॒ह कु॒म्भ्या । \newline
16. कु॒म्भ्या ऽदा॒ ददा᳚त् कु॒म्भ्या कु॒म्भ्या ऽदा᳚त् । \newline
17. अदा॒दित्यदा᳚त् । \newline
18. आ॒शीर् मे॑ म आ॒शी रा॒शीर् मे᳚ । \newline
19. आ॒शीरित्या᳚ - शीः । \newline
20. म॒ ऊर्ज॒ मूर्ज॑म् मे म॒ ऊर्ज᳚म् । \newline
21. ऊर्ज॑ मु॒तो तोर्ज॒ मूर्ज॑ मु॒त । \newline
22. उ॒त सु॑प्रजा॒स्त्वꣳ सु॑प्रजा॒स्त्व मु॒तोत सु॑प्रजा॒स्त्वम् । \newline
23. सु॒प्र॒जा॒स्त्व मिष॒ मिषꣳ॑ सुप्रजा॒स्त्वꣳ सु॑प्रजा॒स्त्व मिष᳚म् । \newline
24. सु॒प्र॒जा॒स्त्वमिति॑ सुप्रजाः - त्वम् । \newline
25. इष॑म् दधातु दधा॒ त्विष॒ मिष॑म् दधातु । \newline
26. द॒धा॒तु॒ द्रवि॑ण॒म् द्रवि॑णम् दधातु दधातु॒ द्रवि॑णम् । \newline
27. द्रवि॑णꣳ॒॒ सव॑र्चसꣳ॒॒ सव॑र्चस॒म् द्रवि॑ण॒म् द्रवि॑णꣳ॒॒ सव॑र्चसम् । \newline
28. सव॑र्चस॒मिति॒ स - व॒र्च॒स॒म् । \newline
29. स॒ञ्जय॒न् क्षेत्रा॑णि॒ क्षेत्रा॑णि स॒ञ्जयन्᳚ थ्स॒ञ्जय॒न् क्षेत्रा॑णि । \newline
30. स॒ञ्जय॒न्निति ॑ सं - जयन्न्॑ । \newline
31. क्षेत्रा॑णि॒ सह॑सा॒ सह॑सा॒ क्षेत्रा॑णि॒ क्षेत्रा॑णि॒ सह॑सा । \newline
32. सह॑सा॒ ऽह म॒हꣳ सह॑सा॒ सह॑सा॒ ऽहम् । \newline
33. अ॒ह मि॑न्द्रे न्द्रा॒ह म॒ह मि॑न्द्र । \newline
34. इ॒न्द्र॒ कृ॒ण्वा॒नः कृ॑ण्वा॒न इ॑न्द्रे न्द्र कृण्वा॒नः । \newline
35. कृ॒ण्वा॒नो अ॒न्याꣳ अ॒न्यान् कृ॑ण्वा॒नः कृ॑ण्वा॒नो अ॒न्यान् । \newline
36. अ॒न्याꣳ अध॑रा॒ नध॑रा न॒न्याꣳ अ॒न्याꣳ अध॑रान् । \newline
37. अध॑रान् थ्स॒पत्ना᳚न् थ्स॒पत्ना॒ नध॑रा॒ नध॑रान् थ्स॒पत्नान्॑ । \newline
38. स॒पत्ना॒निति॑ स॒पत्नान्॑ । \newline
39. भू॒त म॑स्यसि भू॒तम् भू॒त म॑सि । \newline
40. अ॒सि॒ भू॒ते भू॒ते᳚ ऽस्यसि भू॒ते । \newline
41. भू॒ते मा॑ मा भू॒ते भू॒ते मा᳚ । \newline
42. मा॒ धा॒ धा॒ मा॒ मा॒ धाः॒ । \newline
43. धा॒ मुख॒म् मुख॑म् धा धा॒ मुख᳚म् । \newline
44. मुख॑ मस्यसि॒ मुख॒म् मुख॑ मसि । \newline
45. अ॒सि॒ मुख॒म् मुख॑ मस्यसि॒ मुख᳚म् । \newline
46. मुख॑म् भूयासम् भूयास॒म् मुख॒म् मुख॑म् भूयासम् । \newline
47. भू॒या॒स॒म् द्यावा॑पृथि॒वीभ्या॒म् द्यावा॑पृथि॒वीभ्या᳚म् भूयासम् भूयास॒म् द्यावा॑पृथि॒वीभ्या᳚म् । \newline
48. द्यावा॑पृथि॒वीभ्या᳚म् त्वा त्वा॒ द्यावा॑पृथि॒वीभ्या॒म् द्यावा॑पृथि॒वीभ्या᳚म् त्वा । \newline
49. द्यावा॑पृथि॒वीभ्या॒मिति॒ द्यावा᳚ - पृ॒थि॒वीभ्या᳚म् । \newline
50. त्वा॒ परि॒ परि॑ त्वा त्वा॒ परि॑ । \newline
51. परि॑ गृह्णामि गृह्णामि॒ परि॒ परि॑ गृह्णामि । \newline
52. गृ॒ह्णा॒मि॒ विश्वे॒ विश्वे॑ गृह्णामि गृह्णामि॒ विश्वे᳚ । \newline
53. विश्वे᳚ त्वा त्वा॒ विश्वे॒ विश्वे᳚ त्वा । \newline
54. त्वा॒ दे॒वा दे॒वा स्त्वा᳚ त्वा दे॒वाः । \newline
55. दे॒वा वै᳚श्वान॒रा वै᳚श्वान॒रा दे॒वा दे॒वा वै᳚श्वान॒राः । \newline
56. वै॒श्वा॒न॒राः प्र प्र वै᳚श्वान॒रा वै᳚श्वान॒राः प्र । \newline

\textbf{Ghana Paata } \newline

1. पीव॑र्यस्यास्य॒ पीव॑री॒ पीव॑र्यस्य जा॒या जा॒या ऽस्य॒ पीव॑री॒ पीव॑र्यस्य जा॒या । \newline
2. अ॒स्य॒ जा॒या जा॒या ऽस्या᳚स्य जा॒या पीवा॑नः॒ पीवा॑नो जा॒या ऽस्या᳚स्य जा॒या पीवा॑नः । \newline
3. जा॒या पीवा॑नः॒ पीवा॑नो जा॒या जा॒या पीवा॑नः पु॒त्राः पु॒त्राः पीवा॑नो जा॒या जा॒या पीवा॑नः पु॒त्राः । \newline
4. पीवा॑नः पु॒त्राः पु॒त्राः पीवा॑नः॒ पीवा॑नः पु॒त्रा अकृ॑शासो॒ अकृ॑शासः पु॒त्राः पीवा॑नः॒ पीवा॑नः पु॒त्रा अकृ॑शासः । \newline
5. पु॒त्रा अकृ॑शासो॒ अकृ॑शासः पु॒त्राः पु॒त्रा अकृ॑शासो अस्या॒स्या कृ॑शासः पु॒त्राः पु॒त्रा अकृ॑शासो अस्य । \newline
6. अकृ॑शासो अस्या॒स्या कृ॑शासो॒ अकृ॑शासो अस्य । \newline
7. अ॒स्येत्य॑स्य । \newline
8. स॒हजा॑नि॒र् यो यः स॒हजा॑निः स॒हजा॑नि॒र् यः सु॑मख॒स्यमा॑नः सुमख॒स्यमा॑नो॒ यः स॒हजा॑निः स॒हजा॑नि॒र् यः सु॑मख॒स्यमा॑नः । \newline
9. स॒हजा॑नि॒रिति॑ स॒ह - जा॒निः॒ । \newline
10. यः सु॑मख॒स्यमा॑नः सुमख॒स्यमा॑नो॒ यो यः सु॑मख॒स्यमा॑न॒ इन्द्रा॒ये न्द्रा॑य सुमख॒स्यमा॑नो॒ यो यः सु॑मख॒स्यमा॑न॒ इन्द्रा॑य । \newline
11. सु॒म॒ख॒स्यमा॑न॒ इन्द्रा॒ये न्द्रा॑य सुमख॒स्यमा॑नः सुमख॒स्यमा॑न॒ इन्द्रा॑या॒ शिर॑ मा॒शिर॒ मिन्द्रा॑य सुमख॒स्यमा॑नः सुमख॒स्यमा॑न॒ इन्द्रा॑या॒ शिर᳚म् । \newline
12. सु॒म॒ख॒स्यमा॑न॒ इति॑ सु - म॒ख॒स्यमा॑नः । \newline
13. इन्द्रा॑या॒ शिर॑ मा॒शिर॒ मिन्द्रा॒ये न्द्रा॑या॒ शिरꣳ॑ स॒ह स॒हाशिर॒ मिन्द्रा॒ये न्द्रा॑या॒ शिरꣳ॑ स॒ह । \newline
14. आ॒शिरꣳ॑ स॒ह स॒हाशिर॑ मा॒शिरꣳ॑ स॒ह कुं॒भ्या कुं॒भ्या स॒हाशिर॑ मा॒शिरꣳ॑ स॒ह कुं॒भ्या । \newline
15. स॒ह कुं॒भ्या कुं॒भ्या स॒ह स॒ह कुं॒भ्या ऽदा॒ ददा᳚त् कुं॒भ्या स॒ह स॒ह कुं॒भ्या ऽदा᳚त् । \newline
16. कुं॒भ्या ऽदा॒ ददा᳚त् कुं॒भ्या कुं॒भ्या ऽदा᳚त् । \newline
17. अदा॒दित्यदा᳚त् । \newline
18. आ॒शीर् मे॑ म आ॒शी रा॒शीर् म॒ ऊर्ज॒ मूर्ज॑म् म आ॒शी रा॒शीर् म॒ ऊर्ज᳚म् । \newline
19. आ॒शीरित्या᳚ - शीः । \newline
20. म॒ ऊर्ज॒ मूर्ज॑म् मे म॒ ऊर्ज॑ मु॒तो तोर्ज॑म् मे म॒ ऊर्ज॑ मु॒त । \newline
21. ऊर्ज॑ मु॒तोतोर्ज॒ मूर्ज॑ मु॒त सु॑प्रजा॒स्त्वꣳ सु॑प्रजा॒स्त्व मु॒तोर्ज॒ मूर्ज॑ मु॒त सु॑प्रजा॒स्त्वम् । \newline
22. उ॒त सु॑प्रजा॒स्त्वꣳ सु॑प्रजा॒स्त्व मु॒तोत सु॑प्रजा॒स्त्व मिष॒ मिषꣳ॑ सुप्रजा॒स्त्व मु॒तोत सु॑प्रजा॒स्त्व मिष᳚म् । \newline
23. सु॒प्र॒जा॒स्त्व मिष॒ मिषꣳ॑ सुप्रजा॒स्त्वꣳ सु॑प्रजा॒स्त्व मिष॑म् दधातु दधा॒त्विषꣳ॑ सुप्रजा॒स्त्वꣳ सु॑प्रजा॒स्त्व मिष॑म् दधातु । \newline
24. सु॒प्र॒जा॒स्त्वमिति॑ सुप्रजाः - त्वम् । \newline
25. इष॑म् दधातु दधा॒त्विष॒ मिष॑म् दधातु॒ द्रवि॑ण॒म् द्रवि॑णम् दधा॒त्विष॒ मिष॑म् दधातु॒ द्रवि॑णम् । \newline
26. द॒धा॒तु॒ द्रवि॑ण॒म् द्रवि॑णम् दधातु दधातु॒ द्रवि॑णꣳ॒॒ सव॑र्चसꣳ॒॒ सव॑र्चस॒म् द्रवि॑णम् दधातु दधातु॒ द्रवि॑णꣳ॒॒ सव॑र्चसम् । \newline
27. द्रवि॑णꣳ॒॒ सव॑र्चसꣳ॒॒ सव॑र्चस॒म् द्रवि॑ण॒म् द्रवि॑णꣳ॒॒ सव॑र्चसम् । \newline
28. सव॑र्चस॒मिति॒ स - व॒र्च॒स॒म् । \newline
29. स॒ञ्जय॒न् क्षेत्रा॑णि॒ क्षेत्रा॑णि स॒ञ्जयन्᳚ थ्स॒ञ्जय॒न् क्षेत्रा॑णि॒ सह॑सा॒ सह॑सा॒ क्षेत्रा॑णि स॒ञ्जयन्᳚ थ्स॒ञ्जय॒न् क्षेत्रा॑णि॒ सह॑सा । \newline
30. स॒ञ्जय॒न्निति ॑ सं - जयन्न्॑ । \newline
31. क्षेत्रा॑णि॒ सह॑सा॒ सह॑सा॒ क्षेत्रा॑णि॒ क्षेत्रा॑णि॒ सह॑सा॒ ऽह म॒हꣳ सह॑सा॒ क्षेत्रा॑णि॒ क्षेत्रा॑णि॒ सह॑सा॒ ऽहम् । \newline
32. सह॑सा॒ ऽह म॒हꣳ सह॑सा॒ सह॑सा॒ ऽह मि॑न्द्रे न्द्रा॒हꣳ सह॑सा॒ सह॑सा॒ ऽह मि॑न्द्र । \newline
33. अ॒ह मि॑न्द्रे न्द्रा॒ह म॒ह मि॑न्द्र कृण्वा॒नः कृ॑ण्वा॒न इ॑न्द्रा॒ह म॒ह मि॑न्द्र कृण्वा॒नः । \newline
34. इ॒न्द्र॒ कृ॒ण्वा॒नः कृ॑ण्वा॒न इ॑न्द्रे न्द्र कृण्वा॒नो अ॒न्याꣳ अ॒न्यान् कृ॑ण्वा॒न इ॑न्द्रे न्द्र कृण्वा॒नो अ॒न्यान् । \newline
35. कृ॒ण्वा॒नो अ॒न्याꣳ अ॒न्यान् कृ॑ण्वा॒नः कृ॑ण्वा॒नो अ॒न्याꣳ अध॑रा॒ नध॑रा न॒न्यान् कृ॑ण्वा॒नः कृ॑ण्वा॒नो अ॒न्याꣳ अध॑रान् । \newline
36. अ॒न्याꣳ अध॑रा॒ नध॑रा न॒न्याꣳ अ॒न्याꣳ अध॑रान् थ्स॒पत्ना᳚न् थ्स॒पत्ना॒ नध॑रा न॒न्याꣳ अ॒न्याꣳ अध॑रान् थ्स॒पत्नान्॑ । \newline
37. अध॑रान् थ्स॒पत्ना᳚न् थ्स॒पत्ना॒ नध॑रा॒ नध॑रान् थ्स॒पत्नान्॑ । \newline
38. स॒पत्ना॒निति॑ स॒पत्नान्॑ । \newline
39. भू॒त म॑स्यसि भू॒तम् भू॒त म॑सि भू॒ते भू॒ते॑ ऽसि भू॒तम् भू॒त म॑सि भू॒ते । \newline
40. अ॒सि॒ भू॒ते भू॒ते᳚ ऽस्यसि भू॒ते मा॑ मा भू॒ते᳚ ऽस्यसि भू॒ते मा᳚ । \newline
41. भू॒ते मा॑ मा भू॒ते भू॒ते मा॑ धा धा मा भू॒ते भू॒ते मा॑ धाः । \newline
42. मा॒ धा॒ धा॒ मा॒ मा॒ धा॒ मुख॒म् मुख॑म् धा मा मा धा॒ मुख᳚म् । \newline
43. धा॒ मुख॒म् मुख॑म् धा धा॒ मुख॑ मस्यसि॒ मुख॑म् धा धा॒ मुख॑ मसि । \newline
44. मुख॑ मस्यसि॒ मुख॒म् मुख॑ मसि॒ मुख॒म् मुख॑ मसि॒ मुख॒म् मुख॑ मसि॒ मुख᳚म् । \newline
45. अ॒सि॒ मुख॒म् मुख॑ मस्यसि॒ मुख॑म् भूयासम् भूयास॒म् मुख॑ मस्यसि॒ मुख॑म् भूयासम् । \newline
46. मुख॑म् भूयासम् भूयास॒म् मुख॒म् मुख॑म् भूयास॒म् द्यावा॑पृथि॒वीभ्या॒म् द्यावा॑पृथि॒वीभ्या᳚म् भूयास॒म् मुख॒म् मुख॑म् भूयास॒म् द्यावा॑पृथि॒वीभ्या᳚म् । \newline
47. भू॒या॒स॒म् द्यावा॑पृथि॒वीभ्या॒म् द्यावा॑पृथि॒वीभ्या᳚म् भूयासम् भूयास॒म् द्यावा॑पृथि॒वीभ्या᳚म् त्वा त्वा॒ द्यावा॑पृथि॒वीभ्या᳚म् भूयासम् भूयास॒म् द्यावा॑पृथि॒वीभ्या᳚म् त्वा । \newline
48. द्यावा॑पृथि॒वीभ्या᳚म् त्वा त्वा॒ द्यावा॑पृथि॒वीभ्या॒म् द्यावा॑पृथि॒वीभ्या᳚म् त्वा॒ परि॒ परि॑ त्वा॒ द्यावा॑पृथि॒वीभ्या॒म् द्यावा॑पृथि॒वीभ्या᳚म् त्वा॒ परि॑ । \newline
49. द्यावा॑पृथि॒वीभ्या॒मिति॒ द्यावा᳚ - पृ॒थि॒वीभ्या᳚म् । \newline
50. त्वा॒ परि॒ परि॑ त्वा त्वा॒ परि॑ गृह्णामि गृह्णामि॒ परि॑ त्वा त्वा॒ परि॑ गृह्णामि । \newline
51. परि॑ गृह्णामि गृह्णामि॒ परि॒ परि॑ गृह्णामि॒ विश्वे॒ विश्वे॑ गृह्णामि॒ परि॒ परि॑ गृह्णामि॒ विश्वे᳚ । \newline
52. गृ॒ह्णा॒मि॒ विश्वे॒ विश्वे॑ गृह्णामि गृह्णामि॒ विश्वे᳚ त्वा त्वा॒ विश्वे॑ गृह्णामि गृह्णामि॒ विश्वे᳚ त्वा । \newline
53. विश्वे᳚ त्वा त्वा॒ विश्वे॒ विश्वे᳚ त्वा दे॒वा दे॒वा स्त्वा॒ विश्वे॒ विश्वे᳚ त्वा दे॒वाः । \newline
54. त्वा॒ दे॒वा दे॒वा स्त्वा᳚ त्वा दे॒वा वै᳚श्वान॒रा वै᳚श्वान॒रा दे॒वा स्त्वा᳚ त्वा दे॒वा वै᳚श्वान॒राः । \newline
55. दे॒वा वै᳚श्वान॒रा वै᳚श्वान॒रा दे॒वा दे॒वा वै᳚श्वान॒राः प्र प्र वै᳚श्वान॒रा दे॒वा दे॒वा वै᳚श्वान॒राः प्र । \newline
56. वै॒श्वा॒न॒राः प्र प्र वै᳚श्वान॒रा वै᳚श्वान॒राः प्र च्या॑वयन्तु च्यावयन्तु॒ प्र वै᳚श्वान॒रा वै᳚श्वान॒राः प्र च्या॑वयन्तु । \newline
\pagebreak
\markright{ TS 3.2.8.6  \hfill https://www.vedavms.in \hfill}

\section{ TS 3.2.8.6 }

\textbf{TS 3.2.8.6 } \newline
\textbf{Samhita Paata} \newline

प्रच्या॑वयन्तु दि॒वि दे॒वान् दृꣳ॑हा॒न्तरि॑क्षे॒ वयाꣳ॑सि पृथि॒व्यां पार्थि॑वान् ध्रु॒वं ध्रु॒वेण॑ ह॒विषाऽव॒ सोमं॑ नयामसि । यथा॑ नः॒ सर्व॒मिज्जग॑दय॒क्ष्मꣳ सु॒मना॒ अस॑त् । यथा॑ न॒ इन्द्र॒ इद्विशः॒ केव॑लीः॒ सर्वाः॒ सम॑नसः॒ कर॑त् । यथा॑ नः॒ सर्वा॒ इद्दिशो॒ऽस्माकं॒ केव॑ली॒रसन्न्॑ ॥ \newline

\textbf{Pada Paata} \newline

प्रेति॑ । च्या॒व॒य॒न्तु॒ । दि॒वि । दे॒वान् । दृꣳ॒॒ह॒ । अ॒न्तरि॑क्षे । वयाꣳ॑सि । पृ॒थि॒व्याम् । पार्थि॑वान् । ध्रु॒वम् । ध्रु॒वेण॑ । ह॒विषा᳚ । अवेति॑ । सोम᳚म् । न॒या॒म॒सि॒ ॥ यथा᳚ । नः॒ । सर्व᳚म् । इत् । जग॑त् । अ॒य॒क्ष्मम् । सु॒मना॒ इति॑ सु - मनाः᳚ । अस॑त् ॥ यथा᳚ । नः॒ । इन्द्रः॑ । इत् । विशः॑ । केव॑लीः । सर्वाः᳚ । सम॑नस॒ इति॒ स - म॒न॒सः॒ । कर॑त् ॥ यथा᳚ । नः॒ । सर्वाः᳚ । इत् । दिशः॑ । अ॒स्माक᳚म् । केव॑लीः । असन्न्॑ ॥  \newline



\textbf{Jatai Paata} \newline

1. प्र च्या॑वयन्तु च्यावयन्तु॒ प्र प्र च्या॑वयन्तु । \newline
2. च्या॒व॒य॒न्तु॒ दि॒वि दि॒वि च्या॑वयन्तु च्यावयन्तु दि॒वि । \newline
3. दि॒वि दे॒वान् दे॒वान् दि॒वि दि॒वि दे॒वान् । \newline
4. दे॒वान् दृꣳ॑ह दृꣳह दे॒वान् दे॒वान् दृꣳ॑ह । \newline
5. दृꣳ॒॒हा॒ न्तरि॑क्षे अ॒न्तरि॑क्षे दृꣳह दृꣳहा॒ न्तरि॑क्षे । \newline
6. अ॒न्तरि॑क्षे॒ वयाꣳ॑सि॒ वयाꣳ॑ स्य॒न्तरि॑क्षे अ॒न्तरि॑क्षे॒ वयाꣳ॑सि । \newline
7. वयाꣳ॑सि पृथि॒व्याम् पृ॑थि॒व्यां ॅवयाꣳ॑सि॒ वयाꣳ॑सि पृथि॒व्याम् । \newline
8. पृ॒थि॒व्याम् पार्थि॑वा॒न् पार्थि॑वान् पृथि॒व्याम् पृ॑थि॒व्याम् पार्थि॑वान् । \newline
9. पार्थि॑वान् ध्रु॒वम् ध्रु॒वम् पार्थि॑वा॒न् पार्थि॑वान् ध्रु॒वम् । \newline
10. ध्रु॒वम् ध्रु॒वेण॑ ध्रु॒वेण॑ ध्रु॒वम् ध्रु॒वम् ध्रु॒वेण॑ । \newline
11. ध्रु॒वेण॑ ह॒विषा॑ ह॒विषा᳚ ध्रु॒वेण॑ ध्रु॒वेण॑ ह॒विषा᳚ । \newline
12. ह॒विषा ऽवाव॑ ह॒विषा॑ ह॒विषा ऽव॑ । \newline
13. अव॒ सोमꣳ॒॒ सोम॒ मवाव॒ सोम᳚म् । \newline
14. सोम॑म् नयामसि नयामसि॒ सोमꣳ॒॒ सोम॑म् नयामसि । \newline
15. न॒या॒म॒सीति॑ नयामसि । \newline
16. यथा॑ नो नो॒ यथा॒ यथा॑ नः । \newline
17. नः॒ सर्वꣳ॒॒ सर्व॑म् नो नः॒ सर्व᳚म् । \newline
18. सर्व॒ मिदिथ् सर्वꣳ॒॒ सर्व॒ मित् । \newline
19. इज् जग॒ज् जग॒ दिदिज् जग॑त् । \newline
20. जग॑ दय॒क्ष्म म॑य॒क्ष्मम् जग॒ज् जग॑ दय॒क्ष्मम् । \newline
21. अ॒य॒क्ष्मꣳ सु॒मनाः᳚ सु॒मना॑ अय॒क्ष्म म॑य॒क्ष्मꣳ सु॒मनाः᳚ । \newline
22. सु॒मना॒ अस॒ दस॑थ् सु॒मनाः᳚ सु॒मना॒ अस॑त् । \newline
23. सु॒मना॒ इति॑ सु - मनाः᳚ । \newline
24. अस॒दित्यस॑त् । \newline
25. यथा॑ नो नो॒ यथा॒ यथा॑ नः । \newline
26. न॒ इन्द्र॒ इन्द्रो॑ नो न॒ इन्द्रः॑ । \newline
27. इन्द्र॒ इदि दिन्द्र॒ इन्द्र॒ इत् । \newline
28. इद् विशो॒ विश॒ इदिद् विशः॑ । \newline
29. विशः॒ केव॑लीः॒ केव॑ली॒र् विशो॒ विशः॒ केव॑लीः । \newline
30. केव॑लीः॒ सर्वाः॒ सर्वाः॒ केव॑लीः॒ केव॑लीः॒ सर्वाः᳚ । \newline
31. सर्वाः॒ सम॑नसः॒ सम॑नसः॒ सर्वाः॒ सर्वाः॒ सम॑नसः । \newline
32. सम॑नसः॒ कर॒त् कर॒थ् सम॑नसः॒ सम॑नसः॒ कर॑त् । \newline
33. सम॑नस॒ इति॒ स - म॒न॒सः॒ । \newline
34. कर॒दिति॒ कर॑त् । \newline
35. यथा॑ नो नो॒ यथा॒ यथा॑ नः । \newline
36. नः॒ सर्वाः॒ सर्वा॑ नो नः॒ सर्वाः᳚ । \newline
37. सर्वा॒ इदिथ् सर्वाः॒ सर्वा॒ इत् । \newline
38. इद् दिशो॒ दिश॒ इदिद् दिशः॑ । \newline
39. दिशो॒ ऽस्माक॑ म॒स्माक॒म् दिशो॒ दिशो॒ ऽस्माक᳚म् । \newline
40. अ॒स्माक॒म् केव॑लीः॒ केव॑लीर॒स्माक॑ म॒स्माक॒म् केव॑लीः । \newline
41. केव॑ली॒ रस॒न् नस॒न् केव॑लीः॒ केव॑ली॒ रसन्न्॑ । \newline
42. अस॒न्नित्यसन्न्॑ । \newline

\textbf{Ghana Paata } \newline

1. प्र च्या॑वयन्तु च्यावयन्तु॒ प्र प्र च्या॑वयन्तु दि॒वि दि॒वि च्या॑वयन्तु॒ प्र प्र च्या॑वयन्तु दि॒वि । \newline
2. च्या॒व॒य॒न्तु॒ दि॒वि दि॒वि च्या॑वयन्तु च्यावयन्तु दि॒वि दे॒वान् दे॒वान् दि॒वि च्या॑वयन्तु च्यावयन्तु दि॒वि दे॒वान् । \newline
3. दि॒वि दे॒वान् दे॒वान् दि॒वि दि॒वि दे॒वान् दृꣳ॑ह दृꣳह दे॒वान् दि॒वि दि॒वि दे॒वान् दृꣳ॑ह । \newline
4. दे॒वान् दृꣳ॑ह दृꣳह दे॒वान् दे॒वान् दृꣳ॑हा॒ न्तरि॑क्षे अ॒न्तरि॑क्षे दृꣳह दे॒वान् दे॒वान् दृꣳ॑हा॒ न्तरि॑क्षे । \newline
5. दृꣳ॒॒हा॒ न्तरि॑क्षे अ॒न्तरि॑क्षे दृꣳह दृꣳहा॒ न्तरि॑क्षे॒ वयाꣳ॑सि॒ वयाꣳ॑ स्य॒न्तरि॑क्षे दृꣳह दृꣳहा॒ न्तरि॑क्षे॒ वयाꣳ॑सि । \newline
6. अ॒न्तरि॑क्षे॒ वयाꣳ॑सि॒ वयाꣳ॑ स्य॒न्तरि॑क्षे अ॒न्तरि॑क्षे॒ वयाꣳ॑सि पृथि॒व्याम् पृ॑थि॒व्यां ॅवयाꣳ॑ स्य॒न्तरि॑क्षे अ॒न्तरि॑क्षे॒ वयाꣳ॑सि पृथि॒व्याम् । \newline
7. वयाꣳ॑सि पृथि॒व्याम् पृ॑थि॒व्यां ॅवयाꣳ॑सि॒ वयाꣳ॑सि पृथि॒व्याम् पार्थि॑वा॒न् पार्थि॑वान् पृथि॒व्यां ॅवयाꣳ॑सि॒ वयाꣳ॑सि पृथि॒व्याम् पार्थि॑वान् । \newline
8. पृ॒थि॒व्याम् पार्थि॑वा॒न् पार्थि॑वान् पृथि॒व्याम् पृ॑थि॒व्याम् पार्थि॑वान् ध्रु॒वम् ध्रु॒वम् पार्थि॑वान् पृथि॒व्याम् पृ॑थि॒व्याम् पार्थि॑वान् ध्रु॒वम् । \newline
9. पार्थि॑वान् ध्रु॒वम् ध्रु॒वम् पार्थि॑वा॒न् पार्थि॑वान् ध्रु॒वम् ध्रु॒वेण॑ ध्रु॒वेण॑ ध्रु॒वम् पार्थि॑वा॒न् पार्थि॑वान् ध्रु॒वम् ध्रु॒वेण॑ । \newline
10. ध्रु॒वम् ध्रु॒वेण॑ ध्रु॒वेण॑ ध्रु॒वम् ध्रु॒वम् ध्रु॒वेण॑ ह॒विषा॑ ह॒विषा᳚ ध्रु॒वेण॑ ध्रु॒वम् ध्रु॒वम् ध्रु॒वेण॑ ह॒विषा᳚ । \newline
11. ध्रु॒वेण॑ ह॒विषा॑ ह॒विषा᳚ ध्रु॒वेण॑ ध्रु॒वेण॑ ह॒विषा ऽवाव॑ ह॒विषा᳚ ध्रु॒वेण॑ ध्रु॒वेण॑ ह॒विषा ऽव॑ । \newline
12. ह॒विषा ऽवाव॑ ह॒विषा॑ ह॒विषा ऽव॒ सोमꣳ॒॒ सोम॒ मव॑ ह॒विषा॑ ह॒विषा ऽव॒ सोम᳚म् । \newline
13. अव॒ सोमꣳ॒॒ सोम॒ मवाव॒ सोम॑म् नयामसि नयामसि॒ सोम॒ मवाव॒ सोम॑म् नयामसि । \newline
14. सोम॑म् नयामसि नयामसि॒ सोमꣳ॒॒ सोम॑म् नयामसि । \newline
15. न॒या॒म॒सीति॑ नयामसि । \newline
16. यथा॑ नो नो॒ यथा॒ यथा॑ नः॒ सर्वꣳ॒॒ सर्व॑म् नो॒ यथा॒ यथा॑ नः॒ सर्व᳚म् । \newline
17. नः॒ सर्वꣳ॒॒ सर्व॑म् नो नः॒ सर्व॒ मिदिथ् सर्व॑म् नो नः॒ सर्व॒ मित् । \newline
18. सर्व॒ मिदिथ् सर्वꣳ॒॒ सर्व॒ मिज् जग॒ज् जग॒दिथ् सर्वꣳ॒॒ सर्व॒ मिज् जग॑त् । \newline
19. इज् जग॒ज् जग॒ दिदिज् जग॑ दय॒क्ष्म म॑य॒क्ष्मम् जग॒ दिदिज् जग॑ दय॒क्ष्मम् । \newline
20. जग॑ दय॒क्ष्म म॑य॒क्ष्मम् जग॒ज् जग॑ दय॒क्ष्मꣳ सु॒मनाः᳚ सु॒मना॑ अय॒क्ष्मम् जग॒ज् जग॑ दय॒क्ष्मꣳ सु॒मनाः᳚ । \newline
21. अ॒य॒क्ष्मꣳ सु॒मनाः᳚ सु॒मना॑ अय॒क्ष्म म॑य॒क्ष्मꣳ सु॒मना॒ अस॒ दस॑थ् सु॒मना॑ अय॒क्ष्म म॑य॒क्ष्मꣳ सु॒मना॒ अस॑त् । \newline
22. सु॒मना॒ अस॒ दस॑थ् सु॒मनाः᳚ सु॒मना॒ अस॑त् । \newline
23. सु॒मना॒ इति॑ सु - मनाः᳚ । \newline
24. अस॒दित्यस॑त् । \newline
25. यथा॑ नो नो॒ यथा॒ यथा॑ न॒ इन्द्र॒ इन्द्रो॑ नो॒ यथा॒ यथा॑ न॒ इन्द्रः॑ । \newline
26. न॒ इन्द्र॒ इन्द्रो॑ नो न॒ इन्द्र॒ इदि दिन्द्रो॑ नो न॒ इन्द्र॒ इत् । \newline
27. इन्द्र॒ इदि दिन्द्र॒ इन्द्र॒ इद् विशो॒ विश॒ इदिन्द्र॒ इन्द्र॒ इद् विशः॑ । \newline
28. इद् विशो॒ विश॒ इदिद् विशः॒ केव॑लीः॒ केव॑ली॒र् विश॒ इदिद् विशः॒ केव॑लीः । \newline
29. विशः॒ केव॑लीः॒ केव॑ली॒र् विशो॒ विशः॒ केव॑लीः॒ सर्वाः॒ सर्वाः॒ केव॑ली॒र् विशो॒ विशः॒ केव॑लीः॒ सर्वाः᳚ । \newline
30. केव॑लीः॒ सर्वाः॒ सर्वाः॒ केव॑लीः॒ केव॑लीः॒ सर्वाः॒ सम॑नसः॒ सम॑नसः॒ सर्वाः॒ केव॑लीः॒ केव॑लीः॒ सर्वाः॒ सम॑नसः । \newline
31. सर्वाः॒ सम॑नसः॒ सम॑नसः॒ सर्वाः॒ सर्वाः॒ सम॑नसः॒ कर॒त् कर॒थ् सम॑नसः॒ सर्वाः॒ सर्वाः॒ सम॑नसः॒ कर॑त् । \newline
32. सम॑नसः॒ कर॒त् कर॒थ् सम॑नसः॒ सम॑नसः॒ कर॑त् । \newline
33. सम॑नस॒ इति॒ स - म॒न॒सः॒ । \newline
34. कर॒दिति॒ कर॑त् । \newline
35. यथा॑ नो नो॒ यथा॒ यथा॑ नः॒ सर्वाः॒ सर्वा॑ नो॒ यथा॒ यथा॑ नः॒ सर्वाः᳚ । \newline
36. नः॒ सर्वाः॒ सर्वा॑ नो नः॒ सर्वा॒ इदिथ् सर्वा॑ नो नः॒ सर्वा॒ इत् । \newline
37. सर्वा॒ इदिथ् सर्वाः॒ सर्वा॒ इद् दिशो॒ दिश॒ इथ् सर्वाः॒ सर्वा॒ इद् दिशः॑ । \newline
38. इद् दिशो॒ दिश॒ इदिद् दिशो॒ ऽस्माक॑ म॒स्माक॒म् दिश॒ इदिद् दिशो॒ ऽस्माक᳚म् । \newline
39. दिशो॒ ऽस्माक॑ म॒स्माक॒म् दिशो॒ दिशो॒ ऽस्माक॒म् केव॑लीः॒ केव॑ली र॒स्माक॒म् दिशो॒ दिशो॒ ऽस्माक॒म् केव॑लीः । \newline
40. अ॒स्माक॒म् केव॑लीः॒ केव॑ली र॒स्माक॑ म॒स्माक॒म् केव॑ली॒ रस॒न् नस॒न् केव॑ली र॒स्माक॑ म॒स्माक॒म् केव॑ली॒ रसन्न्॑ । \newline
41. केव॑ली॒ रस॒न् नस॒न् केव॑लीः॒ केव॑ली॒ रसन्न्॑ । \newline
42. अस॒न्नित्यसन्न्॑ । \newline
\pagebreak
\markright{ TS 3.2.9.1  \hfill https://www.vedavms.in \hfill}

\section{ TS 3.2.9.1 }

\textbf{TS 3.2.9.1 } \newline
\textbf{Samhita Paata} \newline

यद्वै होता᳚ऽद्ध्व॒र्युम॑भ्या॒ह्वय॑ते॒ वज्र॑मेनम॒भि प्रव॑र्तय॒त्युक्थ॑शा॒ इत्या॑ह प्रातस्सव॒नं प्र॑ति॒गीर्य॒ त्रीण्ये॒तान्य॒क्षरा॑णि त्रि॒पदा॑ गाय॒त्री गा॑य॒त्रं प्रा॑तस्सव॒नं गा॑यत्रि॒यैव प्रा॑तस्सव॒ने वज्र॑म॒न्तर्द्ध॑त्त उ॒क्थं ॅवा॒चीत्या॑ह॒ माद्ध्य॑दिंनꣳ॒॒ सव॑नं प्रति॒गीर्य॑ च॒त्वार्ये॒तान्य॒-क्षरा॑णि॒ चतु॑ष्पदा त्रि॒ष्टुप् त्रैष्टु॑भं॒ माद्ध्य॑दिंनꣳ॒॒ सव॑नं त्रि॒ष्टुभै॒व माद्ध्य॑न्दिने॒ सव॑ने॒ वज्र॑म॒न्तर्द्ध॑त्त-  [  ] \newline

\textbf{Pada Paata} \newline

यत् । वै । होता᳚ । अ॒द्ध्व॒र्युम् । अ॒भ्या॒ह्वय॑त॒ इत्य॑भि - आ॒ह्वय॑ते । वज्र᳚म् । ए॒न॒म् । अ॒भि । प्रेति॑ । व॒र्त॒य॒ति॒ । उक्थ॑शा॒ इत्युक्थ॑-शाः॒ । इति॑ । आ॒ह॒ । प्रा॒त॒स्स॒व॒नमिति॑ प्रातः - स॒व॒नम् । प्र॒ति॒गीर्येति॑ प्रति - गीर्य॑ । त्रीणि॑ । ए॒तानि॑ । अ॒क्षरा॑णि । त्रि॒पदेति॑ त्रि - पदा᳚ । गा॒य॒त्री । गा॒य॒त्रम् । प्रा॒त॒स्स॒व॒नमिति॑ प्रातः - स॒व॒नम् । गा॒य॒त्रि॒या । ए॒व । प्रा॒त॒स्स॒व॒न इति॑ प्रातः - स॒व॒ने । वज्र᳚म् । अ॒न्तः । ध॒त्ते॒ । उ॒क्थम् । वा॒चि । इति॑ । आ॒ह॒ । माद्ध्य॑न्दिनम् । सव॑नम् । प्र॒ति॒गीर्येति॑ प्रति - गीर्य॑ । च॒त्वारि॑ । ए॒तानि॑ । अ॒क्षरा॑णि । चतु॑ष्प॒देति॒ चतुः॑ - प॒दा॒ । त्रि॒ष्टुप् । त्रैष्टु॑भम् । माद्ध्य॑न्दिनम् । सव॑नम् । त्रि॒ष्टुभा᳚ । ए॒व । माद्ध्य॑न्दिने । सव॑ने । वज्र᳚म् । अ॒न्तः । ध॒त्ते॒ ।  \newline


\textbf{Krama Paata} \newline

यद् वै । वै होता᳚ । होता᳚ ऽद्ध्व॒र्युम् । अ॒द्ध्व॒र्यु,म॑भ्या॒ह्वय॑ते । अ॒भ्या॒ह्वय॑ते॒ वज्र᳚म् । अ॒भ्या॒ह्वय॑त॒ इत्य॑भि - आ॒ह्वय॑ते । वज्र॑मेनम् । ए॒न॒म॒भि । अ॒भि प्र । प्र व॑र्तयति । व॒र्त॒य॒त्युक्थ॑शाः । उक्थ॑शा॒ इति॑ । उक्थ॑शा॒ इत्युक्थ॑ - शाः॒ । इत्या॑ह । आ॒ह॒ प्रा॒त॒स्स॒व॒नम् । प्रा॒त॒स्स॒व॒नम् प्र॑ति॒गीर्य॑ । प्रा॒त॒स्स॒व॒नमिति॑ प्रातः - स॒व॒नम् । प्र॒ति॒गीर्य॒ त्रीणि॑ । प्र॒ति॒गीर्येति॑ प्रति - गीर्य॑ । त्रीण्ये॒तानि॑ । ए॒तान्य॒क्षरा॑णि । अ॒क्षरा॑णि त्रि॒पदा᳚ । त्रि॒पदा॑ गाय॒त्री । त्रि॒पदेति॑ त्रि - पदा᳚ । गा॒य॒त्री गा॑य॒त्रम् । गा॒य॒त्रम् प्रा॑तस्सव॒नम् । प्रा॒त॒स्स॒व॒नम् गा॑यत्रि॒या । प्रा॒त॒स्स॒व॒नमिति॑ प्रातः - स॒व॒नम् । गा॒य॒त्रि॒यैव । ए॒व प्रा॑तस्सव॒ने । प्रा॒त॒स्स॒व॒ने वज्र᳚म् । प्रा॒त॒स्स॒व॒न इति॑ प्रातः - स॒व॒ने । वज्र॑म॒न्तः । अ॒न्तर् ध॑त्ते । ध॒त्त॒ उ॒क्थम् । उ॒क्थं ॅवा॒चि । वा॒चीति॑ । इत्या॑ह । आ॒ह॒ माद्ध्य॑न्दिनम् । माद्ध्य॑न्दिनꣳ॒॒ सव॑नम् । सव॑नम् प्रति॒गीर्य॑ । प्र॒ति॒गीर्य॑ च॒त्वारि॑ । प्र॒ति॒गीर्येति॑ प्रति - गीर्य॑ । च॒त्वार्ये॒तानि॑ । ए॒तान्य॒क्षरा॑णि । अ॒क्षरा॑णि॒ चतु॑ष्पदा । चतु॑ष्पदा त्रि॒ष्टुप् । चतु॑ष्प॒देति॒ चतुः॑ - प॒दा॒ । त्रि॒ष्टुप् त्रैष्टु॑भम् । त्रैष्टु॑भ॒म् माद्ध्य॑न्दिनम् । माद्ध्य॑न्दिनꣳ॒॒ सव॑नम् । सव॑नम् त्रि॒ष्टुभा᳚ । त्रि॒ष्टुभै॒व । ए॒व माद्ध्य॑न्दिने । माद्ध्य॑न्दिने॒ सव॑ने । सव॑ने॒ वज्र᳚म् । वज्र॑म॒न्तः । अ॒न्तर् ध॑त्ते । ध॒त्त॒ उ॒क्थम् \newline

\textbf{Jatai Paata} \newline

1. यद् वै वै यद् यद् वै । \newline
2. वै होता॒ होता॒ वै वै होता᳚ । \newline
3. होता᳚ ऽद्ध्व॒र्यु म॑द्ध्व॒र्युꣳ होता॒ होता᳚ ऽद्ध्व॒र्युम् । \newline
4. अ॒द्ध्व॒र्यु म॑भ्या॒ह्वय॑ते ऽभ्या॒ह्वय॑ते ऽद्ध्व॒र्यु म॑द्ध्व॒र्यु म॑भ्या॒ह्वय॑ते । \newline
5. अ॒भ्या॒ह्वय॑ते॒ वज्रं॒ ॅवज्र॑ मभ्या॒ह्वय॑ते ऽभ्या॒ह्वय॑ते॒ वज्र᳚म् । \newline
6. अ॒भ्या॒ह्वय॑त॒ इत्य॑भि - आ॒ह्वय॑ते । \newline
7. वज्र॑ मेन मेनं॒ ॅवज्रं॒ ॅवज्र॑ मेनम् । \newline
8. ए॒न॒ म॒भ्या᳚(1॒)भ्ये॑न मेन म॒भि । \newline
9. अ॒भि प्र प्राभ्य॑भि प्र । \newline
10. प्र व॑र्तयति वर्तयति॒ प्र प्र व॑र्तयति । \newline
11. व॒र्त॒य॒ त्युक्थ॑शा॒ उक्थ॑शा वर्तयति वर्तय॒ त्युक्थ॑शाः । \newline
12. उक्थ॑शा॒ इती त्युक्थ॑शा॒ उक्थ॑शा॒ इति॑ । \newline
13. उक्थ॑शा॒ इत्युक्थ॑ - शाः॒ । \newline
14. इत्या॑हा॒हे तीत्या॑ह । \newline
15. आ॒ह॒ प्रा॒त॒स्स॒व॒नम् प्रा॑तस्सव॒न मा॑हाह प्रातस्सव॒नम् । \newline
16. प्रा॒त॒स्स॒व॒नम् प्र॑ति॒गीर्य॑ प्रति॒गीर्य॑ प्रातस्सव॒नम् प्रा॑तस्सव॒नम् प्र॑ति॒गीर्य॑ । \newline
17. प्रा॒त॒स्स॒व॒नमिति॑ प्रातः - स॒व॒नम् । \newline
18. प्र॒ति॒गीर्य॒ त्रीणि॒ त्रीणि॑ प्रति॒गीर्य॑ प्रति॒गीर्य॒ त्रीणि॑ । \newline
19. प्र॒ति॒गीर्येति॑ प्रति - गीर्य॑ । \newline
20. त्री ण्ये॒ता न्ये॒तानि॒ त्रीणि॒ त्री ण्ये॒तानि॑ । \newline
21. ए॒ता न्य॒क्षरा᳚ ण्य॒क्षरा᳚ ण्ये॒ता न्ये॒ता न्य॒क्षरा॑णि । \newline
22. अ॒क्षरा॑णि त्रि॒पदा᳚ त्रि॒पदा॒ ऽक्षरा᳚ ण्य॒क्षरा॑णि त्रि॒पदा᳚ । \newline
23. त्रि॒पदा॑ गाय॒त्री गा॑य॒त्री त्रि॒पदा᳚ त्रि॒पदा॑ गाय॒त्री । \newline
24. त्रि॒पदेति॑ त्रि - पदा᳚ । \newline
25. गा॒य॒त्री गा॑य॒त्रम् गा॑य॒त्रम् गा॑य॒त्री गा॑य॒त्री गा॑य॒त्रम् । \newline
26. गा॒य॒त्रम् प्रा॑तस्सव॒नम् प्रा॑तस्सव॒नम् गा॑य॒त्रम् गा॑य॒त्रम् प्रा॑तस्सव॒नम् । \newline
27. प्रा॒त॒स्स॒व॒नम् गा॑यत्रि॒या गा॑यत्रि॒या प्रा॑तस्सव॒नम् प्रा॑तस्सव॒नम् गा॑यत्रि॒या । \newline
28. प्रा॒त॒स्स॒व॒नमिति॑ प्रातः - स॒व॒नम् । \newline
29. गा॒य॒त्रि॒ यैवैव गा॑यत्रि॒या गा॑यत्रि॒ यैव । \newline
30. ए॒व प्रा॑तस्सव॒ने प्रा॑तस्सव॒न ए॒वैव प्रा॑तस्सव॒ने । \newline
31. प्रा॒त॒स्स॒व॒ने वज्रं॒ ॅवज्र॑म् प्रातस्सव॒ने प्रा॑तस्सव॒ने वज्र᳚म् । \newline
32. प्रा॒त॒स्स॒व॒न इति॑ प्रातः - स॒व॒ने । \newline
33. वज्र॑ म॒न्त र॒न्तर् वज्रं॒ ॅवज्र॑ म॒न्तः । \newline
34. अ॒न्तर् ध॑त्ते धत्ते॒ ऽन्त र॒न्तर् ध॑त्ते । \newline
35. ध॒त्त॒ उ॒क्थ मु॒क्थम् ध॑त्ते धत्त उ॒क्थम् । \newline
36. उ॒क्थं ॅवा॒चि वा॒च्यु॑क्थ मु॒क्थं ॅवा॒चि । \newline
37. वा॒चीतीति॑ वा॒चि वा॒चीति॑ । \newline
38. इत्या॑हा॒हे तीत्या॑ह । \newline
39. आ॒ह॒ माद्ध्य॑न्दिन॒म् माद्ध्य॑न्दिन माहाह॒ माद्ध्य॑न्दिनम् । \newline
40. माद्ध्य॑न्दिनꣳ॒॒ सव॑नꣳ॒॒ सव॑न॒म् माद्ध्य॑न्दिन॒म् माद्ध्य॑न्दिनꣳ॒॒ सव॑नम् । \newline
41. सव॑नम् प्रति॒गीर्य॑ प्रति॒गीर्य॒ सव॑नꣳ॒॒ सव॑नम् प्रति॒गीर्य॑ । \newline
42. प्र॒ति॒गीर्य॑ च॒त्वारि॑ च॒त्वारि॑ प्रति॒गीर्य॑ प्रति॒गीर्य॑ च॒त्वारि॑ । \newline
43. प्र॒ति॒गीर्येति॑ प्रति - गीर्य॑ । \newline
44. च॒त्वा र्ये॒ता न्ये॒तानि॑ च॒त्वारि॑ च॒त्वा र्ये॒तानि॑ । \newline
45. ए॒ता न्य॒क्षरा᳚ ण्य॒क्षरा᳚ ण्ये॒ता न्ये॒ता न्य॒क्षरा॑णि । \newline
46. अ॒क्षरा॑णि॒ चतु॑ष्पदा॒ चतु॑ष्पदा॒ ऽक्षरा᳚ ण्य॒क्षरा॑णि॒ चतु॑ष्पदा । \newline
47. चतु॑ष्पदा त्रि॒ष्टुप् त्रि॒ष्टुप् चतु॑ष्पदा॒ चतु॑ष्पदा त्रि॒ष्टुप् । \newline
48. चतु॑ष्प॒देति॒ चतुः॑ - प॒दा॒ । \newline
49. त्रि॒ष्टुप् त्रैष्टु॑भ॒म् त्रैष्टु॑भम् त्रि॒ष्टुप् त्रि॒ष्टुप् त्रैष्टु॑भम् । \newline
50. त्रैष्टु॑भ॒म् माद्ध्य॑न्दिन॒म् माद्ध्य॑न्दिन॒म् त्रैष्टु॑भ॒म् त्रैष्टु॑भ॒म् माद्ध्य॑न्दिनम् । \newline
51. माद्ध्य॑न्दिनꣳ॒॒ सव॑नꣳ॒॒ सव॑न॒म् माद्ध्य॑न्दिन॒म् माद्ध्य॑न्दिनꣳ॒॒ सव॑नम् । \newline
52. सव॑नम् त्रि॒ष्टुभा᳚ त्रि॒ष्टुभा॒ सव॑नꣳ॒॒ सव॑नम् त्रि॒ष्टुभा᳚ । \newline
53. त्रि॒ष्टुभै॒ वैव त्रि॒ष्टुभा᳚ त्रि॒ष्टुभै॒व । \newline
54. ए॒व माद्ध्य॑न्दिने॒ माद्ध्य॑न्दिन ए॒वैव माद्ध्य॑न्दिने । \newline
55. माद्ध्य॑न्दिने॒ सव॑ने॒ सव॑ने॒ माद्ध्य॑न्दिने॒ माद्ध्य॑न्दिने॒ सव॑ने । \newline
56. सव॑ने॒ वज्रं॒ ॅवज्रꣳ॒॒ सव॑ने॒ सव॑ने॒ वज्र᳚म् । \newline
57. वज्र॑ म॒न्त र॒न्तर् वज्रं॒ ॅवज्र॑ म॒न्तः । \newline
58. अ॒न्तर् ध॑त्ते धत्ते॒ ऽन्त र॒न्तर् ध॑त्ते । \newline
59. ध॒त्त॒ उ॒क्थ मु॒क्थम् ध॑त्ते धत्त उ॒क्थम् । \newline

\textbf{Ghana Paata } \newline

1. यद् वै वै यद् यद् वै होता॒ होता॒ वै यद् यद् वै होता᳚ । \newline
2. वै होता॒ होता॒ वै वै होता᳚ ऽद्ध्व॒र्यु म॑द्ध्व॒र्युꣳ होता॒ वै वै होता᳚ ऽद्ध्व॒र्युम् । \newline
3. होता᳚ ऽद्ध्व॒र्यु म॑द्ध्व॒र्युꣳ होता॒ होता᳚ ऽद्ध्व॒र्यु म॑भ्या॒ह्वय॑ते ऽभ्या॒ह्वय॑ते ऽद्ध्व॒र्युꣳ होता॒ होता᳚ ऽद्ध्व॒र्यु म॑भ्या॒ह्वय॑ते । \newline
4. अ॒द्ध्व॒र्यु म॑भ्या॒ह्वय॑ते ऽभ्या॒ह्वय॑ते ऽद्ध्व॒र्यु म॑द्ध्व॒र्यु म॑भ्या॒ह्वय॑ते॒ वज्रं॒ ॅवज्र॑ मभ्या॒ह्वय॑ते ऽद्ध्व॒र्यु म॑द्ध्व॒र्यु म॑भ्या॒ह्वय॑ते॒ वज्र᳚म् । \newline
5. अ॒भ्या॒ह्वय॑ते॒ वज्रं॒ ॅवज्र॑ मभ्या॒ह्वय॑ते ऽभ्या॒ह्वय॑ते॒ वज्र॑ मेन मेनं॒ ॅवज्र॑ मभ्या॒ह्वय॑ते ऽभ्या॒ह्वय॑ते॒ वज्र॑ मेनम् । \newline
6. अ॒भ्या॒ह्वय॑त॒ इत्य॑भि - आ॒ह्वय॑ते । \newline
7. वज्र॑ मेन मेनं॒ ॅवज्रं॒ ॅवज्र॑ मेन म॒भ्या᳚(1॒)भ्ये॑नं॒ ॅवज्रं॒ ॅवज्र॑ मेन म॒भि । \newline
8. ए॒न॒ म॒भ्या᳚(1॒)भ्ये॑न मेन म॒भि प्र प्राभ्ये॑न मेन म॒भि प्र । \newline
9. अ॒भि प्र प्राभ्य॑भि प्र व॑र्तयति वर्तयति॒ प्राभ्य॑भि प्र व॑र्तयति । \newline
10. प्र व॑र्तयति वर्तयति॒ प्र प्र व॑र्तय॒ त्युक्थ॑शा॒ उक्थ॑शा वर्तयति॒ प्र प्र व॑र्तय॒ त्युक्थ॑शाः । \newline
11. व॒र्त॒य॒ त्युक्थ॑शा॒ उक्थ॑शा वर्तयति वर्तय॒ त्युक्थ॑शा॒ इती त्युक्थ॑शा वर्तयति वर्तय॒ त्युक्थ॑शा॒ इति॑ । \newline
12. उक्थ॑शा॒ इती त्युक्थ॑शा॒ उक्थ॑शा॒ इत्या॑हा॒हे त्युक्थ॑शा॒ उक्थ॑शा॒ इत्या॑ह । \newline
13. उक्थ॑शा॒ इत्युक्थ॑-शाः॒ । \newline
14. इत्या॑हा॒हे तीत्या॑ह प्रातस्सव॒नम् प्रा॑तस्सव॒न मा॒हे तीत्या॑ह प्रातस्सव॒नम् । \newline
15. आ॒ह॒ प्रा॒त॒स्स॒व॒नम् प्रा॑तस्सव॒न मा॑हाह प्रातस्सव॒नम् प्र॑ति॒गीर्य॑ प्रति॒गीर्य॑ प्रातस्सव॒न मा॑हाह प्रातस्सव॒नम् प्र॑ति॒गीर्य॑ । \newline
16. प्रा॒त॒स्स॒व॒नम् प्र॑ति॒गीर्य॑ प्रति॒गीर्य॑ प्रातस्सव॒नम् प्रा॑तस्सव॒नम् प्र॑ति॒गीर्य॒ त्रीणि॒ त्रीणि॑ प्रति॒गीर्य॑ प्रातस्सव॒नम् प्रा॑तस्सव॒नम् प्र॑ति॒गीर्य॒ त्रीणि॑ । \newline
17. प्रा॒त॒स्स॒व॒नमिति॑ प्रातः - स॒व॒नम् । \newline
18. प्र॒ति॒गीर्य॒ त्रीणि॒ त्रीणि॑ प्रति॒गीर्य॑ प्रति॒गीर्य॒ त्रीण्ये॒ता न्ये॒तानि॒ त्रीणि॑ प्रति॒गीर्य॑ प्रति॒गीर्य॒ त्रीण्ये॒तानि॑ । \newline
19. प्र॒ति॒गीर्येति॑ प्रति - गीर्य॑ । \newline
20. त्रीण्ये॒ता न्ये॒तानि॒ त्रीणि॒ त्रीण्ये॒ता न्य॒क्षरा᳚ ण्य॒क्षरा᳚ ण्ये॒तानि॒ त्रीणि॒ त्रीण्ये॒ता न्य॒क्षरा॑णि । \newline
21. ए॒ता न्य॒क्षरा᳚ ण्य॒क्षरा᳚ ण्ये॒ता न्ये॒ता न्य॒क्षरा॑णि त्रि॒पदा᳚ त्रि॒पदा॒ ऽक्षरा᳚ ण्ये॒ता न्ये॒ता न्य॒क्षरा॑णि त्रि॒पदा᳚ । \newline
22. अ॒क्षरा॑णि त्रि॒पदा᳚ त्रि॒पदा॒ ऽक्षरा᳚ ण्य॒क्षरा॑णि त्रि॒पदा॑ गाय॒त्री गा॑य॒त्री त्रि॒पदा॒ ऽक्षरा᳚ ण्य॒क्षरा॑णि त्रि॒पदा॑ गाय॒त्री । \newline
23. त्रि॒पदा॑ गाय॒त्री गा॑य॒त्री त्रि॒पदा᳚ त्रि॒पदा॑ गाय॒त्री गा॑य॒त्रम् गा॑य॒त्रम् गा॑य॒त्री त्रि॒पदा᳚ त्रि॒पदा॑ गाय॒त्री गा॑य॒त्रम् । \newline
24. त्रि॒पदेति॑ त्रि - पदा᳚ । \newline
25. गा॒य॒त्री गा॑य॒त्रम् गा॑य॒त्रम् गा॑य॒त्री गा॑य॒त्री गा॑य॒त्रम् प्रा॑तस्सव॒नम् प्रा॑तस्सव॒नम् गा॑य॒त्रम् गा॑य॒त्री गा॑य॒त्री गा॑य॒त्रम् प्रा॑तस्सव॒नम् । \newline
26. गा॒य॒त्रम् प्रा॑तस्सव॒नम् प्रा॑तस्सव॒नम् गा॑य॒त्रम् गा॑य॒त्रम् प्रा॑तस्सव॒नम् गा॑यत्रि॒या गा॑यत्रि॒या प्रा॑तस्सव॒नम् गा॑य॒त्रम् गा॑य॒त्रम् प्रा॑तस्सव॒नम् गा॑यत्रि॒या । \newline
27. प्रा॒त॒स्स॒व॒नम् गा॑यत्रि॒या गा॑यत्रि॒या प्रा॑तस्सव॒नम् प्रा॑तस्सव॒नम् गा॑यत्रि॒यैवैव गा॑यत्रि॒या प्रा॑तस्सव॒नम् प्रा॑तस्सव॒नम् गा॑यत्रि॒यैव । \newline
28. प्रा॒त॒स्स॒व॒नमिति॑ प्रातः - स॒व॒नम् । \newline
29. गा॒य॒त्रि॒यैवैव गा॑यत्रि॒या गा॑यत्रि॒यैव प्रा॑तस्सव॒ने प्रा॑तस्सव॒न ए॒व गा॑यत्रि॒या गा॑यत्रि॒यैव प्रा॑तस्सव॒ने । \newline
30. ए॒व प्रा॑तस्सव॒ने प्रा॑तस्सव॒न ए॒वैव प्रा॑तस्सव॒ने वज्रं॒ ॅवज्र॑म् प्रातस्सव॒न ए॒वैव प्रा॑तस्सव॒ने वज्र᳚म् । \newline
31. प्रा॒त॒स्स॒व॒ने वज्रं॒ ॅवज्र॑म् प्रातस्सव॒ने प्रा॑तस्सव॒ने वज्र॑ म॒न्त र॒न्तर् वज्र॑म् प्रातस्सव॒ने प्रा॑तस्सव॒ने वज्र॑ म॒न्तः । \newline
32. प्रा॒त॒स्स॒व॒न इति॑ प्रातः - स॒व॒ने । \newline
33. वज्र॑ म॒न्त र॒न्तर् वज्रं॒ ॅवज्र॑ म॒न्तर् ध॑त्ते धत्ते॒ ऽन्तर् वज्रं॒ ॅवज्र॑ म॒न्तर् ध॑त्ते । \newline
34. अ॒न्तर् ध॑त्ते धत्ते॒ ऽन्त र॒न्तर् ध॑त्त उ॒क्थ मु॒क्थम् ध॑त्ते॒ ऽन्त र॒न्तर् ध॑त्त उ॒क्थम् । \newline
35. ध॒त्त॒ उ॒क्थ मु॒क्थम् ध॑त्ते धत्त उ॒क्थं ॅवा॒चि वा॒च्यु॑क्थम् ध॑त्ते धत्त उ॒क्थं ॅवा॒चि । \newline
36. उ॒क्थं ॅवा॒चि वा॒च्यु॑क्थ मु॒क्थं ॅवा॒चीतीति॑ वा॒च्यु॑क्थ मु॒क्थं ॅवा॒चीति॑ । \newline
37. वा॒चीतीति॑ वा॒चि वा॒ची त्या॑हा॒हे ति॑ वा॒चि वा॒ची त्या॑ह । \newline
38. इत्या॑हा॒हे तीत्या॑ह॒ माद्ध्य॑न्दिन॒म् माद्ध्य॑न्दिन मा॒हे तीत्या॑ह॒ माद्ध्य॑न्दिनम् । \newline
39. आ॒ह॒ माद्ध्य॑न्दिन॒म् माद्ध्य॑न्दिन माहाह॒ माद्ध्य॑न्दिनꣳ॒॒ सव॑नꣳ॒॒ सव॑न॒म् माद्ध्य॑न्दिन माहाह॒ माद्ध्य॑न्दिनꣳ॒॒ सव॑नम् । \newline
40. माद्ध्य॑न्दिनꣳ॒॒ सव॑नꣳ॒॒ सव॑न॒म् माद्ध्य॑न्दिन॒म् माद्ध्य॑न्दिनꣳ॒॒ सव॑नम् प्रति॒गीर्य॑ प्रति॒गीर्य॒ सव॑न॒म् माद्ध्य॑न्दिन॒म् माद्ध्य॑न्दिनꣳ॒॒ सव॑नम् प्रति॒गीर्य॑ । \newline
41. सव॑नम् प्रति॒गीर्य॑ प्रति॒गीर्य॒ सव॑नꣳ॒॒ सव॑नम् प्रति॒गीर्य॑ च॒त्वारि॑ च॒त्वारि॑ प्रति॒गीर्य॒ सव॑नꣳ॒॒ सव॑नम् प्रति॒गीर्य॑ च॒त्वारि॑ । \newline
42. प्र॒ति॒गीर्य॑ च॒त्वारि॑ च॒त्वारि॑ प्रति॒गीर्य॑ प्रति॒गीर्य॑ च॒त्वार्ये॒ता न्ये॒तानि॑ च॒त्वारि॑ प्रति॒गीर्य॑ प्रति॒गीर्य॑ च॒त्वार्ये॒तानि॑ । \newline
43. प्र॒ति॒गीर्येति॑ प्रति - गीर्य॑ । \newline
44. च॒त्वार्ये॒ता न्ये॒तानि॑ च॒त्वारि॑ च॒त्वार्ये॒ता न्य॒क्षरा᳚ ण्य॒क्षरा᳚ ण्ये॒तानि॑ च॒त्वारि॑ च॒त्वार्ये॒ता न्य॒क्षरा॑णि । \newline
45. ए॒ता न्य॒क्षरा᳚ ण्य॒क्षरा᳚ ण्ये॒ता न्ये॒ता न्य॒क्षरा॑णि॒ चतु॑ष्पदा॒ चतु॑ष्पदा॒ ऽक्षरा᳚ण्ये॒ता न्ये॒ता न्य॒क्षरा॑णि॒ चतु॑ष्पदा । \newline
46. अ॒क्षरा॑णि॒ चतु॑ष्पदा॒ चतु॑ष्पदा॒ ऽक्षरा᳚ ण्य॒क्षरा॑णि॒ चतु॑ष्पदा त्रि॒ष्टुप् त्रि॒ष्टुप् 
चतु॑ष्पदा॒ ऽक्षरा᳚ ण्य॒क्षरा॑णि॒ चतु॑ष्पदा त्रि॒ष्टुप् । \newline
47. चतु॑ष्पदा त्रि॒ष्टुप् त्रि॒ष्टुप् चतु॑ष्पदा॒ चतु॑ष्पदा त्रि॒ष्टुप् त्रैष्टु॑भ॒म् त्रैष्टु॑भम् त्रि॒ष्टुप् चतु॑ष्पदा॒ चतु॑ष्पदा त्रि॒ष्टुप् त्रैष्टु॑भम् । \newline
48. चतु॑ष्प॒देति॒ चतुः॑ - प॒दा॒ । \newline
49. त्रि॒ष्टुप् त्रैष्टु॑भ॒म् त्रैष्टु॑भम् त्रि॒ष्टुप् त्रि॒ष्टुप् त्रैष्टु॑भ॒म् माद्ध्य॑न्दिन॒म् माद्ध्य॑न्दिन॒म् त्रैष्टु॑भम् त्रि॒ष्टुप् त्रि॒ष्टुप् त्रैष्टु॑भ॒म् माद्ध्य॑न्दिनम् । \newline
50. त्रैष्टु॑भ॒म् माद्ध्य॑न्दिन॒म् माद्ध्य॑न्दिन॒म् त्रैष्टु॑भ॒म् त्रैष्टु॑भ॒म् माद्ध्य॑न्दिनꣳ॒॒ सव॑नꣳ॒॒ सव॑न॒म् माद्ध्य॑न्दिन॒म् त्रैष्टु॑भ॒म् त्रैष्टु॑भ॒म् माद्ध्य॑न्दिनꣳ॒॒ सव॑नम् । \newline
51. माद्ध्य॑न्दिनꣳ॒॒ सव॑नꣳ॒॒ सव॑न॒म् माद्ध्य॑न्दिन॒म् माद्ध्य॑न्दिनꣳ॒॒ सव॑नम् त्रि॒ष्टुभा᳚ त्रि॒ष्टुभा॒ सव॑न॒म् माद्ध्य॑न्दिन॒म् माद्ध्य॑न्दिनꣳ॒॒ सव॑नम् त्रि॒ष्टुभा᳚ । \newline
52. सव॑नम् त्रि॒ष्टुभा᳚ त्रि॒ष्टुभा॒ सव॑नꣳ॒॒ सव॑नम् त्रि॒ष्टुभै॒वैव त्रि॒ष्टुभा॒ सव॑नꣳ॒॒ सव॑नम् त्रि॒ष्टुभै॒व । \newline
53. त्रि॒ष्टुभै॒वैव त्रि॒ष्टुभा᳚ त्रि॒ष्टुभै॒व माद्ध्य॑न्दिने॒ माद्ध्य॑न्दिन ए॒व त्रि॒ष्टुभा᳚ त्रि॒ष्टुभै॒व माद्ध्य॑न्दिने । \newline
54. ए॒व माद्ध्य॑न्दिने॒ माद्ध्य॑न्दिन ए॒वैव माद्ध्य॑न्दिने॒ सव॑ने॒ सव॑ने॒ माद्ध्य॑न्दिन ए॒वैव माद्ध्य॑न्दिने॒ सव॑ने । \newline
55. माद्ध्य॑न्दिने॒ सव॑ने॒ सव॑ने॒ माद्ध्य॑न्दिने॒ माद्ध्य॑न्दिने॒ सव॑ने॒ वज्रं॒ ॅवज्रꣳ॒॒ सव॑ने॒ माद्ध्य॑न्दिने॒ माद्ध्य॑न्दिने॒ सव॑ने॒ वज्र᳚म् । \newline
56. सव॑ने॒ वज्रं॒ ॅवज्रꣳ॒॒ सव॑ने॒ सव॑ने॒ वज्र॑ म॒न्त र॒न्तर् वज्रꣳ॒॒ सव॑ने॒ सव॑ने॒ वज्र॑ म॒न्तः । \newline
57. वज्र॑ म॒न्त र॒न्तर् वज्रं॒ ॅवज्र॑ म॒न्तर् ध॑त्ते धत्ते॒ ऽन्तर् वज्रं॒ ॅवज्र॑ म॒न्तर् ध॑त्ते । \newline
58. अ॒न्तर् ध॑त्ते धत्ते॒ ऽन्त र॒न्तर् ध॑त्त उ॒क्थ मु॒क्थम् ध॑त्ते॒ ऽन्त र॒न्तर् ध॑त्त उ॒क्थम् । \newline
59. ध॒त्त॒ उ॒क्थ मु॒क्थम् ध॑त्ते धत्त उ॒क्थं ॅवा॒चि वा॒च्यु॑क्थम् ध॑त्ते धत्त उ॒क्थं ॅवा॒चि । \newline
\pagebreak
\markright{ TS 3.2.9.2  \hfill https://www.vedavms.in \hfill}

\section{ TS 3.2.9.2 }

\textbf{TS 3.2.9.2 } \newline
\textbf{Samhita Paata} \newline

उ॒क्थं ॅवा॒चीन्द्रा॒येत्या॑ह तृतीयसव॒नं प्र॑ति॒गीर्य॑ स॒प्तैतान्य॒क्षरा॑णि स॒प्तप॑दा॒ शक्व॑री शाक्व॒रो वज्रो॒ वज्रे॑णै॒व तृ॑तीयसव॒ने वज्र॑म॒न्तर्द्ध॑त्ते ब्रह्मवा॒दिनो॑ वदन्ति॒ स त्वा अ॑द्ध्व॒र्युः स्या॒द्यो य॑थासव॒नं प्र॑तिग॒रे छन्दाꣳ॑सि सम्पा॒दये॒त् तेजः॑ प्रातः सव॒न आ॒त्मन् दधी॑तेन्द्रि॒यं माद्ध्य॑न्दिने॒ सव॑ने प॒शूꣳ स्तृ॑तीयसव॒न इत्युक्थ॑शा॒ इत्या॑ह प्रातस्सव॒नं प्र॑ति॒गीर्य॒ त्रीण्ये॒तान्य॒क्षरा॑णि - [  ] \newline

\textbf{Pada Paata} \newline

उ॒क्थम् । वा॒चि । इन्द्रा॑य । इति॑ । आ॒ह॒ । तृ॒ती॒य॒स॒व॒नमिति॑ तृतीय -स॒व॒नम् । प्र॒ति॒गीर्येति॑ प्रति - गीर्य॑ । स॒प्त । ए॒तानि॑ । अ॒क्षरा॑णि । स॒प्तप॒देति॑ स॒प्त - प॒दा॒ । शक्व॑री । शा॒क्व॒रः । वज्रः॑ । वज्रे॑ण । ए॒व । तृ॒ती॒य॒स॒व॒न इति॑ तृतीय - स॒व॒ने । वज्र᳚म् । अ॒न्तः । ध॒त्ते॒ । ब्र॒ह्म॒वा॒दिन॒ इति॑ ब्रह्म - वा॒दिनः॑ । व॒द॒न्ति॒ । सः । तु । वै । अ॒ध्व॒र्युः । स्या॒त् । यः । य॒था॒स॒व॒नमिति॑ यथा - स॒व॒नम् । प्र॒ति॒ग॒र इति॑ प्रति - ग॒रे । छन्दाꣳ॑सि । स॒पां॒दये॒दिति॑ सं - पा॒दये᳚त् । तेजः॑ । प्रा॒त॒स्स॒व॒न इति॑ प्रातः - स॒व॒ने । आ॒त्मन्न् । दधी॑त । इ॒न्द्रि॒यम् । माध्य॑न्दिने । सव॑ने । प॒शून् । तृ॒ती॒य॒स॒व॒न इति॑ तृतीय - स॒व॒ने । इति॑ । उक्थ॑शा॒ इत्युक्थ॑ - शाः॒ । इति॑ । आ॒ह॒ । प्रा॒त॒स्स॒व॒नमिति॑ प्रातः - स॒व॒नम् । प्र॒ति॒गीर्येति॑ प्रति - गीर्य॑ । त्रीणि॑ । ए॒तानि॑ । अ॒क्षरा॑णि ।  \newline


\textbf{Krama Paata} \newline

उ॒क्थं ॅवा॒चि । वा॒चीन्द्रा॑य । इन्द्रा॒येति॑ । इत्या॑ह । आ॒ह॒ तृ॒ती॒य॒स॒व॒नम् । तृ॒ती॒य॒स॒व॒नम् प्र॑ति॒गीर्य॑ । तृ॒ती॒य॒स॒व॒नमिति॑ तृतीय - स॒व॒नम् । प्र॒ति॒गीर्य॑ स॒प्त । प्र॒ति॒गीर्येति॑ प्रति - गीर्य॑ । स॒प्तैतानि॑ । ए॒तान्य॒क्षरा॑णि । अ॒क्षरा॑णि स॒प्तप॑दा । स॒प्तप॑दा॒ शक्व॑री । स॒प्तप॒देति॑ स॒प्त - प॒दा॒ । शक्व॑री शाक्व॒रः । शा॒क्व॒रो वज्रः॑ । वज्रो॒ वज्रे॑ण । वज्रे॑णै॒व । ए॒व तृ॑तीयसव॒ने । तृ॒ती॒य॒स॒व॒ने वज्र᳚म् । तृ॒ती॒य॒स॒व॒न इति॑ तृतीय - स॒व॒ने । वज्र॑म॒न्तः । अ॒न्तर् ध॑त्ते । ध॒त्ते॒ ब्र॒ह्म॒वा॒दिनः॑ । ब्र॒ह्म॒वा॒दिनो॑ वदन्ति । ब॒ह्म॒वा॒दिन॒ इति॑ ब्रह्म - वा॒दिनः॑ । व॒द॒न्ति॒ सः । स तु । त्वै । वा अ॑द्ध्व॒र्युः । अ॒द्ध्व॒र्युः स्या᳚त् । स्या॒द् यः । यो य॑थासव॒नम् । य॒था॒स॒व॒नम् प्र॑तिग॒रे । य॒था॒स॒व॒नमिति॑ यथा - स॒व॒नम् । प्र॒ति॒ग॒रे छन्दाꣳ॑सि । प्र॒ति॒ग॒र इति॑ प्रति - ग॒रे । छन्दाꣳ॑सि सम्पा॒दये᳚त् । स॒म्पा॒दये॒त् तेजः॑ । स॒म्पा॒दये॒दिति॑ सम् - पा॒दये᳚त् । तेजः॑ प्रातस्सव॒ने । प्रा॒त॒स्स॒व॒न आ॒त्मन्न् । प्रा॒त॒स्स॒व॒न इति॑ प्रातः - स॒व॒ने । आ॒त्मन् दधी॑त । दधी॑तेन्द्रि॒यम् । इ॒न्द्रि॒यम् माद्ध्य॑न्दिने । माद्ध्य॑न्दिने॒ सव॑ने । सव॑ने प॒शून् । प॒शूꣳ स्तृ॑तीयसव॒ने । तृ॒ती॒य॒स॒व॒न इति॑ । तृ॒ती॒य॒स॒व॒न इति॑ तृतीय - स॒व॒ने । इत्युक्थ॑शाः । उक्थ॑शा॒ इति॑ । उक्थ॑शा॒ इत्युक्थ॑ - शाः॒ । इत्या॑ह । आ॒ह॒ प्रा॒त॒स्स॒व॒नम् । प्रा॒त॒स्स॒व॒नम् प्र॑ति॒गीर्य॑ । प्रा॒त॒स॒व॒नमिति॑ प्रातः - स॒व॒नम् । प्र॒ति॒गीर्य॒ त्रीणि॑ । प्र॒ति॒गीर्येति॑ प्रति - गीर्य॑ । त्रीण्ये॒तानि॑ । ए॒तान्य॒क्षरा॑णि । अ॒क्षरा॑णि त्रि॒पदा᳚ \newline

\textbf{Jatai Paata} \newline

1. उ॒क्थं ॅवा॒चि वा॒च्यु॑क्थ मु॒क्थं ॅवा॒चि । \newline
2. वा॒चीन्द्रा॒ये न्द्रा॑य वा॒चि वा॒चीन्द्रा॑य । \newline
3. इन्द्रा॒ये तीतीन्द्रा॒ये न्द्रा॒ये ति॑ । \newline
4. इत्या॑हा॒हे तीत्या॑ह । \newline
5. आ॒ह॒ तृ॒ती॒य॒स॒व॒नम् तृ॑तीयसव॒न मा॑हाह तृतीयसव॒नम् । \newline
6. तृ॒ती॒य॒स॒व॒नम् प्र॑ति॒गीर्य॑ प्रति॒गीर्य॑ तृतीयसव॒नम् तृ॑तीयसव॒नम् प्र॑ति॒गीर्य॑ । \newline
7. तृ॒ती॒य॒स॒व॒नमिति॑ तृतीय -स॒व॒नम् । \newline
8. प्र॒ति॒गीर्य॑ स॒प्त स॒प्त प्र॑ति॒गीर्य॑ प्रति॒गीर्य॑ स॒प्त । \newline
9. प्र॒ति॒गीर्येति॑ प्रति - गीर्य॑ । \newline
10. स॒प्तैता न्ये॒तानि॑ स॒प्त स॒प्तैतानि॑ । \newline
11. ए॒ता न्य॒क्षरा᳚ ण्य॒क्षरा᳚ ण्ये॒ता न्ये॒ता न्य॒क्षरा॑णि । \newline
12. अ॒क्षरा॑णि स॒प्तप॑दा स॒प्तप॑दा॒ ऽक्षरा᳚ ण्य॒क्षरा॑णि स॒प्तप॑दा । \newline
13. स॒प्तप॑दा॒ शक्व॑री॒ शक्व॑री स॒प्तप॑दा स॒प्तप॑दा॒ शक्व॑री । \newline
14. स॒प्तप॒देति॑ स॒प्त - प॒दा॒ । \newline
15. शक्व॑री शाक्व॒रः शा᳚क्व॒रः शक्व॑री॒ शक्व॑री शाक्व॒रः । \newline
16. शा॒क्व॒रो वज्रो॒ वज्रः॑ शाक्व॒रः शा᳚क्व॒रो वज्रः॑ । \newline
17. वज्रो॒ वज्रे॑ण॒ वज्रे॑ण॒ वज्रो॒ वज्रो॒ वज्रे॑ण । \newline
18. वज्रे॑णै॒ वैव वज्रे॑ण॒ वज्रे॑णै॒व । \newline
19. ए॒व तृ॑तीयसव॒ने तृ॑तीयसव॒न ए॒वैव तृ॑तीयसव॒ने । \newline
20. तृ॒ती॒य॒स॒व॒ने वज्रं॒ ॅवज्र॑म् तृतीयसव॒ने तृ॑तीयसव॒ने वज्र᳚म् । \newline
21. तृ॒ती॒य॒स॒व॒न इति॑ तृतीय - स॒व॒ने । \newline
22. वज्र॑ म॒न्त र॒न्तर् वज्रं॒ ॅवज्र॑ म॒न्तः । \newline
23. अ॒न्तर् ध॑त्ते धत्ते॒ ऽन्त र॒न्तर् ध॑त्ते । \newline
24. ध॒त्ते॒ ब्र॒ह्म॒वा॒दिनो᳚ ब्रह्मवा॒दिनो॑ धत्ते धत्ते ब्रह्मवा॒दिनः॑ । \newline
25. ब्र॒ह्म॒वा॒दिनो॑ वदन्ति वदन्ति ब्रह्मवा॒दिनो᳚ ब्रह्मवा॒दिनो॑ वदन्ति । \newline
26. ब्र॒ह्म॒वा॒दिन॒ इति॑ ब्रह्म - वा॒दिनः॑ । \newline
27. व॒द॒न्ति॒ स स व॑दन्ति वदन्ति॒ सः । \newline
28. स तु तु स स तु । \newline
29. त्वै वै तु त्वै । \newline
30. वा अ॑द्ध्व॒र्यु र॑द्ध्व॒र्युर् वै वा अ॑द्ध्व॒र्युः । \newline
31. अ॒द्ध्व॒र्युः स्या᳚थ् स्यादद्ध्व॒र्यु र॑द्ध्व॒र्युः स्या᳚त् । \newline
32. स्या॒द् यो यः स्या᳚थ् स्या॒द् यः । \newline
33. यो य॑थासव॒नं ॅय॑थासव॒नं ॅयो यो य॑थासव॒नम् । \newline
34. य॒था॒स॒व॒नम् प्र॑तिग॒रे प्र॑तिग॒रे य॑थासव॒नं ॅय॑थासव॒नम् प्र॑तिग॒रे । \newline
35. य॒था॒स॒व॒नमिति॑ यथा - स॒व॒नम् । \newline
36. प्र॒ति॒ग॒रे छन्दाꣳ॑सि॒ छन्दाꣳ॑सि प्रतिग॒रे प्र॑तिग॒रे छन्दाꣳ॑सि । \newline
37. प्र॒ति॒ग॒र इति॑ प्रति - ग॒रे । \newline
38. छन्दाꣳ॑सि संपा॒दये᳚थ् संपा॒दये॒च् छन्दाꣳ॑सि॒ छन्दाꣳ॑सि संपा॒दये᳚त् । \newline
39. सं॒पा॒दये॒त् तेज॒ स्तेजः॑ संपा॒दये᳚थ् संपा॒दये॒त् तेजः॑ । \newline
40. सं॒पा॒दये॒दिति॑ सं - पा॒दये᳚त् । \newline
41. तेजः॑ प्रातस्सव॒ने प्रा॑तस्सव॒ने तेज॒ स्तेजः॑ प्रातस्सव॒ने । \newline
42. प्रा॒त॒स्स॒व॒न आ॒त्मन् ना॒त्मन् प्रा॑तस्सव॒ने प्रा॑तस्सव॒न आ॒त्मन्न् । \newline
43. प्रा॒त॒स्स॒व॒न इति॑ प्रातः - स॒व॒ने । \newline
44. आ॒त्मन् दधी॑त॒ दधी॑ ता॒त्मन् ना॒त्मन् दधी॑त । \newline
45. दधी॑ तेन्द्रि॒य मि॑न्द्रि॒यम् दधी॑त॒ दधी॑ तेन्द्रि॒यम् । \newline
46. इ॒न्द्रि॒यम् माद्ध्य॑न्दिने॒ माद्ध्य॑न्दिन इन्द्रि॒य मि॑न्द्रि॒यम् माद्ध्य॑न्दिने । \newline
47. माद्ध्य॑न्दिने॒ सव॑ने॒ सव॑ने॒ माद्ध्य॑न्दिने॒ माद्ध्य॑न्दिने॒ सव॑ने । \newline
48. सव॑ने प॒शून् प॒शून् थ्सव॑ने॒ सव॑ने प॒शून् । \newline
49. प॒शूꣳ स्तृ॑तीयसव॒ने तृ॑तीयसव॒ने प॒शून् प॒शूꣳ स्तृ॑तीयसव॒ने । \newline
50. तृ॒ती॒य॒स॒व॒न इतीति॑ तृतीयसव॒ने तृ॑तीयसव॒न इति॑ । \newline
51. तृ॒ती॒य॒स॒व॒न इति॑ तृतीय - स॒व॒ने । \newline
52. इत्युक्थ॑शा॒ उक्थ॑शा॒ इती त्युक्थ॑शाः । \newline
53. उक्थ॑शा॒ इती त्युक्थ॑शा॒ उक्थ॑शा॒ इति॑ । \newline
54. उक्थ॑शा॒ इत्युक्थ॑ - शाः॒ । \newline
55. इत्या॑हा॒हे तीत्या॑ह । \newline
56. आ॒ह॒ प्रा॒त॒स्स॒व॒नम् प्रा॑तस्सव॒न मा॑हाह प्रातस्सव॒नम् । \newline
57. प्रा॒त॒स्स॒व॒नम् प्र॑ति॒गीर्य॑ प्रति॒गीर्य॑ प्रातस्सव॒नम् प्रा॑तस्सव॒नम् प्र॑ति॒गीर्य॑ । \newline
58. प्रा॒त॒स्स॒व॒नमिति॑ प्रातः - स॒व॒नम् । \newline
59. प्र॒ति॒गीर्य॒ त्रीणि॒ त्रीणि॑ प्रति॒गीर्य॑ प्रति॒गीर्य॒ त्रीणि॑ । \newline
60. प्र॒ति॒गीर्येति॑ प्रति - गीर्य॑ । \newline
61. त्री ण्ये॒ता न्ये॒तानि॒ त्रीणि॒ त्रीण्ये॒तानि॑ । \newline
62. ए॒ता न्य॒क्षरा᳚ ण्य॒क्षरा᳚ ण्ये॒ता न्ये॒ता न्य॒क्षरा॑णि । \newline
63. अ॒क्षरा॑णि त्रि॒पदा᳚ त्रि॒पदा॒ ऽक्षरा᳚ ण्य॒क्षरा॑णि त्रि॒पदा᳚ । \newline

\textbf{Ghana Paata } \newline

1. उ॒क्थं ॅवा॒चि वा॒च्यु॑क्थ मु॒क्थं ॅवा॒चीन्द्रा॒ये न्द्रा॑य वा॒च्यु॑क्थ मु॒क्थं ॅवा॒चीन्द्रा॑य । \newline
2. वा॒चीन्द्रा॒ये न्द्रा॑य वा॒चि वा॒चीन्द्रा॒ये तीतीन्द्रा॑य वा॒चि वा॒चीन्द्रा॒ये ति॑ । \newline
3. इन्द्रा॒ये तीतीन्द्रा॒ये न्द्रा॒ये त्या॑हा॒हे तीन्द्रा॒ये न्द्रा॒ये त्या॑ह । \newline
4. इत्या॑हा॒हे तीत्या॑ह तृतीयसव॒नम् तृ॑तीयसव॒न मा॒हे तीत्या॑ह तृतीयसव॒नम् । \newline
5. आ॒ह॒ तृ॒ती॒य॒स॒व॒नम् तृ॑तीयसव॒न मा॑हाह तृतीयसव॒नम् प्र॑ति॒गीर्य॑ प्रति॒गीर्य॑ तृतीयसव॒न मा॑हाह तृतीयसव॒नम् प्र॑ति॒गीर्य॑ । \newline
6. तृ॒ती॒य॒स॒व॒नम् प्र॑ति॒गीर्य॑ प्रति॒गीर्य॑ तृतीयसव॒नम् तृ॑तीयसव॒नम् प्र॑ति॒गीर्य॑ स॒प्त स॒प्त प्र॑ति॒गीर्य॑ तृतीयसव॒नम् तृ॑तीयसव॒नम् प्र॑ति॒गीर्य॑ स॒प्त । \newline
7. तृ॒ती॒य॒स॒व॒नमिति॑ तृतीय - स॒व॒नम् । \newline
8. प्र॒ति॒गीर्य॑ स॒प्त स॒प्त प्र॑ति॒गीर्य॑ प्रति॒गीर्य॑ स॒प्तैता न्ये॒तानि॑ स॒प्त प्र॑ति॒गीर्य॑ प्रति॒गीर्य॑ स॒प्तैतानि॑ । \newline
9. प्र॒ति॒गीर्येति॑ प्रति - गीर्य॑ । \newline
10. स॒प्तैता न्ये॒तानि॑ स॒प्त स॒प्तैता न्य॒क्षरा᳚ ण्य॒क्षरा᳚ ण्ये॒तानि॑ स॒प्त स॒प्तैता न्य॒क्षरा॑णि । \newline
11. ए॒ता न्य॒क्षरा᳚ ण्य॒क्षरा᳚ ण्ये॒ता न्ये॒ता न्य॒क्षरा॑णि स॒प्तप॑दा 
स॒प्तप॑दा॒ ऽक्षरा᳚ ण्ये॒ता न्ये॒ता न्य॒क्षरा॑णि स॒प्तप॑दा । \newline
12. अ॒क्षरा॑णि स॒प्तप॑दा स॒प्तप॑दा॒ ऽक्षरा᳚ ण्य॒क्षरा॑णि स॒प्तप॑दा॒ शक्व॑री॒ शक्व॑री 
स॒प्तप॑दा॒ ऽक्षरा᳚ ण्य॒क्षरा॑णि स॒प्तप॑दा॒ शक्व॑री । \newline
13. स॒प्तप॑दा॒ शक्व॑री॒ शक्व॑री स॒प्तप॑दा स॒प्तप॑दा॒ शक्व॑री शाक्व॒रः शा᳚क्व॒रः शक्व॑री स॒प्तप॑दा स॒प्तप॑दा॒ शक्व॑री शाक्व॒रः । \newline
14. स॒प्तप॒देति॑ स॒प्त - प॒दा॒ । \newline
15. शक्व॑री शाक्व॒रः शा᳚क्व॒रः शक्व॑री॒ शक्व॑री शाक्व॒रो वज्रो॒ वज्रः॑ शाक्व॒रः शक्व॑री॒ शक्व॑री शाक्व॒रो वज्रः॑ । \newline
16. शा॒क्व॒रो वज्रो॒ वज्रः॑ शाक्व॒रः शा᳚क्व॒रो वज्रो॒ वज्रे॑ण॒ वज्रे॑ण॒ वज्रः॑ शाक्व॒रः शा᳚क्व॒रो वज्रो॒ वज्रे॑ण । \newline
17. वज्रो॒ वज्रे॑ण॒ वज्रे॑ण॒ वज्रो॒ वज्रो॒ वज्रे॑ णै॒वैव वज्रे॑ण॒ वज्रो॒ वज्रो॒ वज्रे॑णै॒व । \newline
18. वज्रे॑णै॒वैव वज्रे॑ण॒ वज्रे॑णै॒व तृ॑तीयसव॒ने तृ॑तीयसव॒न ए॒व वज्रे॑ण॒ वज्रे॑णै॒व तृ॑तीयसव॒ने । \newline
19. ए॒व तृ॑तीयसव॒ने तृ॑तीयसव॒न ए॒वैव तृ॑तीयसव॒ने वज्रं॒ ॅवज्र॑म् तृतीयसव॒न ए॒वैव तृ॑तीयसव॒ने वज्र᳚म् । \newline
20. तृ॒ती॒य॒स॒व॒ने वज्रं॒ ॅवज्र॑म् तृतीयसव॒ने तृ॑तीयसव॒ने वज्र॑ म॒न्त र॒न्तर् वज्र॑म् तृतीयसव॒ने तृ॑तीयसव॒ने वज्र॑ म॒न्तः । \newline
21. तृ॒ती॒य॒स॒व॒न इति॑ तृतीय - स॒व॒ने । \newline
22. वज्र॑ म॒न्त र॒न्तर् वज्रं॒ ॅवज्र॑ म॒न्तर् ध॑त्ते धत्ते॒ ऽन्तर् वज्रं॒ ॅवज्र॑ म॒न्तर् ध॑त्ते । \newline
23. अ॒न्तर् ध॑त्ते धत्ते॒ ऽन्त र॒न्तर् ध॑त्ते ब्रह्मवा॒दिनो᳚ ब्रह्मवा॒दिनो॑ धत्ते॒ ऽन्त र॒न्तर् ध॑त्ते ब्रह्मवा॒दिनः॑ । \newline
24. ध॒त्ते॒ ब्र॒ह्म॒वा॒दिनो᳚ ब्रह्मवा॒दिनो॑ धत्ते धत्ते ब्रह्मवा॒दिनो॑ वदन्ति वदन्ति ब्रह्मवा॒दिनो॑ धत्ते धत्ते ब्रह्मवा॒दिनो॑ वदन्ति । \newline
25. ब्र॒ह्म॒वा॒दिनो॑ वदन्ति वदन्ति ब्रह्मवा॒दिनो᳚ ब्रह्मवा॒दिनो॑ वदन्ति॒ स स व॑दन्ति ब्रह्मवा॒दिनो᳚ ब्रह्मवा॒दिनो॑ वदन्ति॒ सः । \newline
26. ब्र॒ह्म॒वा॒दिन॒ इति॑ ब्रह्म - वा॒दिनः॑ । \newline
27. व॒द॒न्ति॒ स स व॑दन्ति वदन्ति॒ स तु तु स व॑दन्ति वदन्ति॒ स तु । \newline
28. स तु तु स सत्वै वै तु स सत्वै । \newline
29. त्वै वै तुत् वा अ॑द्ध्व॒र्यु र॑द्ध्व॒र्युर् वै तुत् वा अ॑द्ध्व॒र्युः । \newline
30. वा अ॑द्ध्व॒र्यु र॑द्ध्व॒र्युर् वै वा अ॑द्ध्व॒र्युः स्या᳚थ् स्या दद्ध्व॒र्युर् वै वा अ॑द्ध्व॒र्युः स्या᳚त् । \newline
31. अ॒द्ध्व॒र्युः स्या᳚थ् स्या दद्ध्व॒र्यु र॑द्ध्व॒र्युः स्या॒द् यो यः स्या॑ दद्ध्व॒र्यु र॑द्ध्व॒र्युः स्या॒द् यः । \newline
32. स्या॒द् यो यः स्या᳚थ् स्या॒द् यो य॑थासव॒नं ॅय॑थासव॒नं ॅयः स्या᳚थ् स्या॒द् यो य॑थासव॒नम् । \newline
33. यो य॑थासव॒नं ॅय॑थासव॒नं ॅयो यो य॑थासव॒नम् प्र॑तिग॒रे प्र॑तिग॒रे य॑थासव॒नं ॅयो यो य॑थासव॒नम् प्र॑तिग॒रे । \newline
34. य॒था॒स॒व॒नम् प्र॑तिग॒रे प्र॑तिग॒रे य॑थासव॒नं ॅय॑थासव॒नम् प्र॑तिग॒रे छन्दाꣳ॑सि॒ छन्दाꣳ॑सि प्रतिग॒रे य॑थासव॒नं ॅय॑थासव॒नम् प्र॑तिग॒रे छन्दाꣳ॑सि । \newline
35. य॒था॒स॒व॒नमिति॑ यथा - स॒व॒नम् । \newline
36. प्र॒ति॒ग॒रे छन्दाꣳ॑सि॒ छन्दाꣳ॑सि प्रतिग॒रे प्र॑तिग॒रे छन्दाꣳ॑सि संपा॒दये᳚थ् संपा॒दये॒ च्छन्दाꣳ॑सि प्रतिग॒रे प्र॑तिग॒रे छन्दाꣳ॑सि संपा॒दये᳚त् । \newline
37. प्र॒ति॒ग॒र इति॑ प्रति - ग॒रे । \newline
38. छन्दाꣳ॑सि संपा॒दये᳚थ् संपा॒दये॒ च्छन्दाꣳ॑सि॒ छन्दाꣳ॑सि संपा॒दये॒त् तेज॒ स्तेजः॑ संपा॒दये॒ च्छन्दाꣳ॑सि॒ छन्दाꣳ॑सि संपा॒दये॒त् तेजः॑ । \newline
39. सं॒पा॒दये॒त् तेज॒ स्तेजः॑ संपा॒दये᳚थ् संपा॒दये॒त् तेजः॑ प्रातस्सव॒ने प्रा॑तस्सव॒ने तेजः॑ संपा॒दये᳚थ् संपा॒दये॒त् तेजः॑ प्रातस्सव॒ने । \newline
40. सं॒पा॒दये॒दिति॑ सं - पा॒दये᳚त् । \newline
41. तेजः॑ प्रातस्सव॒ने प्रा॑तस्सव॒ने तेज॒ स्तेजः॑ प्रातस्सव॒न आ॒त्मन् ना॒त्मन् प्रा॑तस्सव॒ने तेज॒ स्तेजः॑ प्रातस्सव॒न आ॒त्मन्न् । \newline
42. प्रा॒त॒स्स॒व॒न आ॒त्मन् ना॒त्मन् प्रा॑तस्सव॒ने प्रा॑तस्सव॒न आ॒त्मन् दधी॑त॒ दधी॑ता॒त्मन् प्रा॑तस्सव॒ने प्रा॑तस्सव॒न आ॒त्मन् दधी॑त । \newline
43. प्रा॒त॒स्स॒व॒न इति॑ प्रातः - स॒व॒ने । \newline
44. आ॒त्मन् दधी॑त॒ दधी॑ता॒त्मन् ना॒त्मन् दधी॑ते न्द्रि॒य मि॑न्द्रि॒यम् दधी॑ता॒त्मन् ना॒त्मन् दधी॑ते न्द्रि॒यम् । \newline
45. दधी॑ते न्द्रि॒य मि॑न्द्रि॒यम् दधी॑त॒ दधी॑ते न्द्रि॒यम् माद्ध्य॑न्दिने॒ माद्ध्य॑न्दिन इन्द्रि॒यम् दधी॑त॒ दधी॑ते न्द्रि॒यम् माद्ध्य॑न्दिने । \newline
46. इ॒न्द्रि॒यम् माद्ध्य॑न्दिने॒ माद्ध्य॑न्दिन इन्द्रि॒य मि॑न्द्रि॒यम् माद्ध्य॑न्दिने॒ सव॑ने॒ सव॑ने॒ माद्ध्य॑न्दिन इन्द्रि॒य मि॑न्द्रि॒यम् माद्ध्य॑न्दिने॒ सव॑ने । \newline
47. माद्ध्य॑न्दिने॒ सव॑ने॒ सव॑ने॒ माद्ध्य॑न्दिने॒ माद्ध्य॑न्दिने॒ सव॑ने प॒शून् प॒शून् थ्सव॑ने॒ माद्ध्य॑न्दिने॒ माद्ध्य॑न्दिने॒ सव॑ने प॒शून् । \newline
48. सव॑ने प॒शून् प॒शून् थ्सव॑ने॒ सव॑ने प॒शूꣳ स्तृ॑तीयसव॒ने तृ॑तीयसव॒ने प॒शून् थ्सव॑ने॒ सव॑ने प॒शूꣳ स्तृ॑तीयसव॒ने । \newline
49. प॒शूꣳ स्तृ॑तीयसव॒ने तृ॑तीयसव॒ने प॒शून् प॒शूꣳ स्तृ॑तीयसव॒न इतीति॑ तृतीयसव॒ने प॒शून् प॒शूꣳ स्तृ॑तीयसव॒न इति॑ । \newline
50. तृ॒ती॒य॒स॒व॒न इतीति॑ तृतीयसव॒ने तृ॑तीयसव॒न इत्युक्थ॑शा॒ उक्थ॑शा॒ इति॑ तृतीयसव॒ने तृ॑तीयसव॒न इत्युक्थ॑शाः । \newline
51. तृ॒ती॒य॒स॒व॒न इति॑ तृतीय - स॒व॒ने । \newline
52. इत्युक्थ॑शा॒ उक्थ॑शा॒ इती त्युक्थ॑शा॒ इती त्युक्थ॑शा॒ इती त्युक्थ॑शा॒ इति॑ । \newline
53. उक्थ॑शा॒ इती त्युक्थ॑शा॒ उक्थ॑शा॒ इत्या॑हा॒हे त्युक्थ॑शा॒ उक्थ॑शा॒ इत्या॑ह । \newline
54. उक्थ॑शा॒ इत्युक्थ॑ - शाः॒ । \newline
55. इत्या॑हा॒हे तीत्या॑ह प्रातस्सव॒नम् प्रा॑तस्सव॒न मा॒हे तीत्या॑ह प्रातस्सव॒नम् । \newline
56. आ॒ह॒ प्रा॒त॒स्स॒व॒नम् प्रा॑तस्सव॒न मा॑हाह प्रातस्सव॒नम् प्र॑ति॒गीर्य॑ प्रति॒गीर्य॑ प्रातस्सव॒न मा॑हाह प्रातस्सव॒नम् प्र॑ति॒गीर्य॑ । \newline
57. प्रा॒त॒स्स॒व॒नम् प्र॑ति॒गीर्य॑ प्रति॒गीर्य॑ प्रातस्सव॒नम् प्रा॑तस्सव॒नम् प्र॑ति॒गीर्य॒ त्रीणि॒ त्रीणि॑ प्रति॒गीर्य॑ प्रातस्सव॒नम् प्रा॑तस्सव॒नम् प्र॑ति॒गीर्य॒ त्रीणि॑ । \newline
58. प्रा॒त॒स्स॒व॒नमिति॑ प्रातः - स॒व॒नम् । \newline
59. प्र॒ति॒गीर्य॒ त्रीणि॒ त्रीणि॑ प्रति॒गीर्य॑ प्रति॒गीर्य॒ त्रीण्ये॒ता न्ये॒तानि॒ त्रीणि॑ प्रति॒गीर्य॑ प्रति॒गीर्य॒ त्रीण्ये॒तानि॑ । \newline
60. प्र॒ति॒गीर्येति॑ प्रति - गीर्य॑ । \newline
61. त्रीण्ये॒ता न्ये॒तानि॒ त्रीणि॒ त्रीण्ये॒ता न्य॒क्षरा᳚ ण्य॒क्षरा᳚ ण्ये॒तानि॒ त्रीणि॒ त्रीण्ये॒ता न्य॒क्षरा॑णि । \newline
62. ए॒ता न्य॒क्षरा᳚ ण्य॒क्षरा᳚ ण्ये॒ता न्ये॒ता न्य॒क्षरा॑णि त्रि॒पदा᳚ त्रि॒पदा॒ ऽक्षरा᳚ ण्ये॒ता न्ये॒ता न्य॒क्षरा॑णि त्रि॒पदा᳚ । \newline
63. अ॒क्षरा॑णि त्रि॒पदा᳚ त्रि॒पदा॒ ऽक्षरा᳚ ण्य॒क्षरा॑णि त्रि॒पदा॑ गाय॒त्री गा॑य॒त्री 
त्रि॒पदा॒ ऽक्षरा᳚ ण्य॒क्षरा॑णि त्रि॒पदा॑ गाय॒त्री । \newline
\pagebreak
\markright{ TS 3.2.9.3  \hfill https://www.vedavms.in \hfill}

\section{ TS 3.2.9.3 }

\textbf{TS 3.2.9.3 } \newline
\textbf{Samhita Paata} \newline

त्रि॒पदा॑ गाय॒त्री गा॑य॒त्रं प्रा॑तस्सव॒नं प्रा॑तस्सव॒न ए॒व प्र॑तिग॒रे छन्दाꣳ॑सि॒ संपा॑दय॒त्यथो॒ तेजो॒ वै गा॑य॒त्री तेजः॑ प्रातः सव॒नं तेज॑ ए॒व प्रा॑तस्सव॒न आ॒त्मन् ध॑त्त उ॒क्थं ॅवा॒चीत्या॑ह॒ माद्ध्य॑न्दिनꣳ॒॒ सव॑नं प्रति॒गीर्य॑ च॒त्वार्ये॒तान्य॒क्षरा॑णि॒ चतु॑ष्पदा त्रि॒ष्टुप् त्रैष्टु॑भं॒ माद्ध्य॑न्दिनꣳ॒॒ सव॑नं॒ माद्ध्य॑दिंन ए॒व सव॑ने प्रतिग॒रे छन्दाꣳ॑सि॒ संपा॑दय॒त्यथो॑ इन्द्रि॒यं ॅवै त्रि॒ष्टुगि॑न्द्रि॒यं माद्ध्य॑दिंनꣳ॒॒ सव॑न - [  ] \newline

\textbf{Pada Paata} \newline

त्रि॒पदेति॑ त्रि - पदा᳚ । गा॒य॒त्री । गा॒य॒त्रम् । प्रा॒त॒स्स॒व॒नमिति॑ प्रातः - स॒व॒नम् । प्रा॒त॒स्स॒व॒न इति॑ प्रातः - स॒व॒ने । ए॒व । प्र॒ति॒ग॒र इति॑ प्रति - ग॒रे । छन्दाꣳ॑सि । समिति॑ । पा॒द॒य॒ति॒ । अथो॒ इति॑ । तेजः॑ । वै । गा॒य॒त्री । तेजः॑ । प्रा॒त॒स्स॒व॒नमिति॑ प्रातः - स॒व॒नम् । तेजः॑ । ए॒व । प्रा॒त॒स्स॒व॒न इति॑ प्रातः - स॒व॒ने । आ॒त्मन्न् । ध॒त्ते॒ । उ॒क्थम् । वा॒चि । इति॑ । आ॒ह॒ । माद्ध्य॑न्दिनम् । सव॑नम् । प्र॒ति॒गीर्येति॑ प्रति - गीर्य॑ । च॒त्वारि॑ । ए॒तानि॑ । अ॒क्षरा॑णि । चतु॑ष्प॒देति॒ चतुः॑ - प॒दा॒ । त्रि॒ष्टुप् । त्रैष्टु॑भम् । माद्ध्य॑न्दिनम् । सव॑नम् । माद्ध्य॑न्दिने । ए॒व । सव॑ने । प्र॒ति॒ग॒र इति॑ प्रति - ग॒रे । छन्दाꣳ॑सि । समिति॑ । पा॒द॒य॒ति॒ । अथो॒ इति॑ । इ॒न्द्रि॒यम् । वै । त्रि॒ष्टुक् । इ॒न्द्रि॒यम् । माद्ध्य॑न्दिनम् । सव॑नम् ।  \newline


\textbf{Krama Paata} \newline

त्रि॒पदा॑ गाय॒त्री । त्रि॒पदेति॑ त्रि - पदा᳚ । गा॒य॒त्री गा॑य॒त्रम् । गा॒य॒त्रम् प्रा॑तस्सव॒नम् । प्रा॒त॒स्स॒व॒नम् प्रा॑तस्सव॒ने । प्रा॒त॒स्स॒व॒नमिति॑ प्रातः - स॒व॒नम् । प्रा॒त॒स्स॒व॒न ए॒व । प्रा॒त॒स्स॒व॒न इति॑ प्रातः - स॒व॒ने । ए॒व प्र॑तिग॒रे । प्र॒ति॒ग॒रे छन्दाꣳ॑सि । प्र॒ति॒ग॒र इति॑ प्रति - ग॒रे । छन्दाꣳ॑सि॒ सम् । सम् पा॑दयति । पा॒द॒य॒त्यथो᳚ । अथो॒ तेजः॑ । अथो॒ इत्यथो᳚ । तेजो॒ वै । वै गा॑य॒त्री । गा॒य॒त्री तेजः॑ । तेजः॑ प्रातस्सव॒नम् । प्रा॒त॒स्स॒व॒नम् तेजः॑ । प्रा॒त॒स्स॒व॒नमिति॑ प्रातः - स॒व॒नम् । तेज॑ ए॒व । ए॒व प्रा॑तस्सव॒ने । प्रा॒त॒स्स॒व॒न आ॒त्मन्न् । प्रा॒त॒स्स॒व॒न इति॑ प्रातः - स॒व॒ने । आ॒त्मन् ध॑त्ते । ध॒त्त॒ उ॒क्थम् । उ॒क्थं ॅवा॒चि । वा॒चीति॑ । इत्या॑ह । आ॒ह॒ माद्ध्य॑न्दिनम् । माद्ध्य॑न्दिनꣳ॒॒ सव॑नम् । सव॑नम् प्रति॒गीर्य॑ । प्र॒ति॒गीर्य॑ च॒त्वारि॑ । प्र॒ति॒गीर्येति॑ प्रति - गीर्य॑ । च॒त्वार्ये॒तानि॑ । ए॒तान्य॒क्षरा॑णि । अ॒क्षरा॑णि॒ चतु॑ष्पदा । चतु॑ष्पदा त्रि॒ष्टुप् । चतु॑ष्प॒देति॒ चतुः॑ - प॒दा॒ । त्रि॒ष्टुप् त्रैष्टु॑भम् । त्रैष्टु॑भ॒म् माद्ध्य॑न्दिनम् । माद्ध्य॑न्दिनꣳ॒॒ सव॑नम् । सव॑न॒म् माद्ध्य॑न्दिने । माद्ध्य॑न्दिन ए॒व । ए॒व सव॑ने । सव॑ने प्रतिग॒रे । प्र॒ति॒ग॒रे छन्दाꣳ॑सि । प्र॒ति॒ग॒र इति॑ प्रति - ग॒रे । छन्दाꣳ॑सि॒ सम् । सम् पा॑दयति । पा॒द॒य॒त्यथो᳚ । अथो॑ इन्द्रि॒यम् । अथो॒ इत्यथो᳚ । इ॒न्द्रि॒यं ॅवै । वै त्रि॒ष्टुक् । त्रि॒ष्टुगि॑न्द्रि॒यम् । इ॒न्द्रि॒यम् माद्ध्य॑न्दिनम् । माद्ध्य॑न्दिनꣳ॒॒ सव॑नम् । सव॑नमिन्द्रि॒यम् \newline

\textbf{Jatai Paata} \newline

1. त्रि॒पदा॑ गाय॒त्री गा॑य॒त्री त्रि॒पदा᳚ त्रि॒पदा॑ गाय॒त्री । \newline
2. त्रि॒पदेति॑ त्रि - पदा᳚ । \newline
3. गा॒य॒त्री गा॑य॒त्रम् गा॑य॒त्रम् गा॑य॒त्री गा॑य॒त्री गा॑य॒त्रम् । \newline
4. गा॒य॒त्रम् प्रा॑तस्सव॒नम् प्रा॑तस्सव॒नम् गा॑य॒त्रम् गा॑य॒त्रम् प्रा॑तस्सव॒नम् । \newline
5. प्रा॒त॒स्स॒व॒नम् प्रा॑तस्सव॒ने प्रा॑तस्सव॒ने प्रा॑तस्सव॒नम् प्रा॑तस्सव॒नम् प्रा॑तस्सव॒ने । \newline
6. प्रा॒त॒स्स॒व॒नमिति॑ प्रातः - स॒व॒नम् । \newline
7. प्रा॒त॒स्स॒व॒न ए॒वैव प्रा॑तस्सव॒ने प्रा॑तस्सव॒न ए॒व । \newline
8. प्रा॒त॒स्स॒व॒न इति॑ प्रातः - स॒व॒ने । \newline
9. ए॒व प्र॑तिग॒रे प्र॑तिग॒र ए॒वैव प्र॑तिग॒रे । \newline
10. प्र॒ति॒ग॒रे छन्दाꣳ॑सि॒ छन्दाꣳ॑सि प्रतिग॒रे प्र॑तिग॒रे छन्दाꣳ॑सि । \newline
11. प्र॒ति॒ग॒र इति॑ प्रति - ग॒रे । \newline
12. छन्दाꣳ॑सि॒ सꣳ सम् छन्दाꣳ॑सि॒ छन्दाꣳ॑सि॒ सम् । \newline
13. सम् पा॑दयति पादयति॒ सꣳ सम् पा॑दयति । \newline
14. पा॒द॒य॒ त्यथो॒ अथो॑ पादयति पादय॒ त्यथो᳚ । \newline
15. अथो॒ तेज॒ स्तेजो ऽथो॒ अथो॒ तेजः॑ । \newline
16. अथो॒ इत्यथो᳚ । \newline
17. तेजो॒ वै वै तेज॒ स्तेजो॒ वै । \newline
18. वै गा॑य॒त्री गा॑य॒त्री वै वै गा॑य॒त्री । \newline
19. गा॒य॒त्री तेज॒ स्तेजो॑ गाय॒त्री गा॑य॒त्री तेजः॑ । \newline
20. तेजः॑ प्रातस्सव॒नम् प्रा॑तस्सव॒नम् तेज॒ स्तेजः॑ प्रातस्सव॒नम् । \newline
21. प्रा॒त॒स्स॒व॒नम् तेज॒ स्तेजः॑ प्रातस्सव॒नम् प्रा॑तस्सव॒नम् तेजः॑ । \newline
22. प्रा॒त॒स्स॒व॒नमिति॑ प्रातः - स॒व॒नम् । \newline
23. तेज॑ ए॒वैव तेज॒ स्तेज॑ ए॒व । \newline
24. ए॒व प्रा॑तस्सव॒ने प्रा॑तस्सव॒न ए॒वैव प्रा॑तस्सव॒ने । \newline
25. प्रा॒त॒स्स॒व॒न आ॒त्मन् ना॒त्मन् प्रा॑तस्सव॒ने प्रा॑तस्सव॒न आ॒त्मन्न् । \newline
26. प्रा॒त॒स्स॒व॒न इति॑ प्रातः - स॒व॒ने । \newline
27. आ॒त्मन् ध॑त्ते धत्त आ॒त्मन् ना॒त्मन् ध॑त्ते । \newline
28. ध॒त्त॒ उ॒क्थ मु॒क्थम् ध॑त्ते धत्त उ॒क्थम् । \newline
29. उ॒क्थं ॅवा॒चि वा॒च्यु॑क्थ मु॒क्थं ॅवा॒चि । \newline
30. वा॒चीतीति॑ वा॒चि वा॒चीति॑ । \newline
31. इत्या॑हा॒हे तीत्या॑ह । \newline
32. आ॒ह॒ माद्ध्य॑न्दिन॒म् माद्ध्य॑न्दिन माहाह॒ माद्ध्य॑न्दिनम् । \newline
33. माद्ध्य॑न्दिनꣳ॒॒ सव॑नꣳ॒॒ सव॑न॒म् माद्ध्य॑न्दिन॒म् माद्ध्य॑न्दिनꣳ॒॒ सव॑नम् । \newline
34. सव॑नम् प्रति॒गीर्य॑ प्रति॒गीर्य॒ सव॑नꣳ॒॒ सव॑नम् प्रति॒गीर्य॑ । \newline
35. प्र॒ति॒गीर्य॑ च॒त्वारि॑ च॒त्वारि॑ प्रति॒गीर्य॑ प्रति॒गीर्य॑ च॒त्वारि॑ । \newline
36. प्र॒ति॒गीर्येति॑ प्रति - गीर्य॑ । \newline
37. च॒त्वा र्ये॒ता न्ये॒तानि॑ च॒त्वारि॑ च॒त्वा र्ये॒तानि॑ । \newline
38. ए॒ता न्य॒क्षरा᳚ ण्य॒क्षरा᳚ ण्ये॒ता न्ये॒ता न्य॒क्षरा॑णि । \newline
39. अ॒क्षरा॑णि॒ चतु॑ष्पदा॒ चतु॑ष्पदा॒ ऽक्षरा᳚ ण्य॒क्षरा॑णि॒ चतु॑ष्पदा । \newline
40. चतु॑ष्पदा त्रि॒ष्टुप् त्रि॒ष्टुप् चतु॑ष्पदा॒ चतु॑ष्पदा त्रि॒ष्टुप् । \newline
41. चतु॑ष्प॒देति॒ चतुः॑ - प॒दा॒ । \newline
42. त्रि॒ष्टुप् त्रैष्टु॑भ॒म् त्रैष्टु॑भम् त्रि॒ष्टुप् त्रि॒ष्टुप् त्रैष्टु॑भम् । \newline
43. त्रैष्टु॑भ॒म् माद्ध्य॑न्दिन॒म् माद्ध्य॑न्दिन॒म् त्रैष्टु॑भ॒म् त्रैष्टु॑भ॒म् माद्ध्य॑न्दिनम् । \newline
44. माद्ध्य॑न्दिनꣳ॒॒ सव॑नꣳ॒॒ सव॑न॒म् माद्ध्य॑न्दिन॒म् माद्ध्य॑न्दिनꣳ॒॒ सव॑नम् । \newline
45. सव॑न॒म् माद्ध्य॑न्दिने॒ माद्ध्य॑न्दिने॒ सव॑नꣳ॒॒ सव॑न॒म् माद्ध्य॑न्दिने । \newline
46. माद्ध्य॑न्दिन ए॒वैव माद्ध्य॑न्दिने॒ माद्ध्य॑न्दिन ए॒व । \newline
47. ए॒व सव॑ने॒ सव॑न ए॒वैव सव॑ने । \newline
48. सव॑ने प्रतिग॒रे प्र॑तिग॒रे सव॑ने॒ सव॑ने प्रतिग॒रे । \newline
49. प्र॒ति॒ग॒रे छन्दाꣳ॑सि॒ छन्दाꣳ॑सि प्रतिग॒रे प्र॑तिग॒रे छन्दाꣳ॑सि । \newline
50. प्र॒ति॒ग॒र इति॑ प्रति - ग॒रे । \newline
51. छन्दाꣳ॑सि॒ सꣳ सम् छन्दाꣳ॑सि॒ छन्दाꣳ॑सि॒ सम् । \newline
52. सम् पा॑दयति पादयति॒ सꣳ सम् पा॑दयति । \newline
53. पा॒द॒य॒ त्यथो॒ अथो॑ पादयति पादय॒ त्यथो᳚ । \newline
54. अथो॑ इन्द्रि॒य मि॑न्द्रि॒य मथो॒ अथो॑ इन्द्रि॒यम् । \newline
55. अथो॒ इत्यथो᳚ । \newline
56. इ॒न्द्रि॒यं ॅवै वा इ॑न्द्रि॒य मि॑न्द्रि॒यं ॅवै । \newline
57. वै त्रि॒ष्टुक् त्रि॒ष्टुग् वै वै त्रि॒ष्टुक् । \newline
58. त्रि॒ष्टु गि॑न्द्रि॒य मि॑न्द्रि॒यम् त्रि॒ष्टुक् त्रि॒ष्टु गि॑न्द्रि॒यम् । \newline
59. इ॒न्द्रि॒यम् माद्ध्य॑न्दिन॒म् माद्ध्य॑न्दिन मिन्द्रि॒य मि॑न्द्रि॒यम् माद्ध्य॑न्दिनम् । \newline
60. माद्ध्य॑न्दिनꣳ॒॒ सव॑नꣳ॒॒ सव॑न॒म् माद्ध्य॑न्दिन॒म् माद्ध्य॑न्दिनꣳ॒॒ सव॑नम् । \newline
61. सव॑न मिन्द्रि॒य मि॑न्द्रि॒यꣳ सव॑नꣳ॒॒ सव॑न मिन्द्रि॒यम् । \newline

\textbf{Ghana Paata } \newline

1. त्रि॒पदा॑ गाय॒त्री गा॑य॒त्री त्रि॒पदा᳚ त्रि॒पदा॑ गाय॒त्री गा॑य॒त्रम् गा॑य॒त्रम् गा॑य॒त्री त्रि॒पदा᳚ त्रि॒पदा॑ गाय॒त्री गा॑य॒त्रम् । \newline
2. त्रि॒पदेति॑ त्रि - पदा᳚ । \newline
3. गा॒य॒त्री गा॑य॒त्रम् गा॑य॒त्रम् गा॑य॒त्री गा॑य॒त्री गा॑य॒त्रम् प्रा॑तस्सव॒नम् प्रा॑तस्सव॒नम् गा॑य॒त्रम् गा॑य॒त्री गा॑य॒त्री गा॑य॒त्रम् प्रा॑तस्सव॒नम् । \newline
4. गा॒य॒त्रम् प्रा॑तस्सव॒नम् प्रा॑तस्सव॒नम् गा॑य॒त्रम् गा॑य॒त्रम् प्रा॑तस्सव॒नम् प्रा॑तस्सव॒ने प्रा॑तस्सव॒ने प्रा॑तस्सव॒नम् गा॑य॒त्रम् गा॑य॒त्रम् प्रा॑तस्सव॒नम् प्रा॑तस्सव॒ने । \newline
5. प्रा॒त॒स्स॒व॒नम् प्रा॑तस्सव॒ने प्रा॑तस्सव॒ने प्रा॑तस्सव॒नम् प्रा॑तस्सव॒नम् प्रा॑तस्सव॒न ए॒वैव प्रा॑तस्सव॒ने प्रा॑तस्सव॒नम् प्रा॑तस्सव॒नम् प्रा॑तस्सव॒न ए॒व । \newline
6. प्रा॒त॒स्स॒व॒नमिति॑ प्रातः - स॒व॒नम् । \newline
7. प्रा॒त॒स्स॒व॒न ए॒वैव प्रा॑तस्सव॒ने प्रा॑तस्सव॒न ए॒व प्र॑तिग॒रे प्र॑तिग॒र ए॒व प्रा॑तस्सव॒ने प्रा॑तस्सव॒न ए॒व प्र॑तिग॒रे । \newline
8. प्रा॒त॒स्स॒व॒न इति॑ प्रातः - स॒व॒ने । \newline
9. ए॒व प्र॑तिग॒रे प्र॑तिग॒र ए॒वैव प्र॑तिग॒रे छन्दाꣳ॑सि॒ छन्दाꣳ॑सि प्रतिग॒र ए॒वैव प्र॑तिग॒रे छन्दाꣳ॑सि । \newline
10. प्र॒ति॒ग॒रे छन्दाꣳ॑सि॒ छन्दाꣳ॑सि प्रतिग॒रे प्र॑तिग॒रे छन्दाꣳ॑सि॒ सꣳ सम् छन्दाꣳ॑सि प्रतिग॒रे प्र॑तिग॒रे छन्दाꣳ॑सि॒ सम् । \newline
11. प्र॒ति॒ग॒र इति॑ प्रति - ग॒रे । \newline
12. छन्दाꣳ॑सि॒ सꣳ सम् छन्दाꣳ॑सि॒ छन्दाꣳ॑सि॒ सम् पा॑दयति पादयति॒ सम् छन्दाꣳ॑सि॒ छन्दाꣳ॑सि॒ सम् पा॑दयति । \newline
13. सम् पा॑दयति पादयति॒ सꣳ सम् पा॑दय॒ त्यथो॒ अथो॑ पादयति॒ सꣳ सम् पा॑दय॒ त्यथो᳚ । \newline
14. पा॒द॒य॒ त्यथो॒ अथो॑ पादयति पादय॒ त्यथो॒ तेज॒ स्तेजो ऽथो॑ पादयति पादय॒ त्यथो॒ तेजः॑ । \newline
15. अथो॒ तेज॒ स्तेजो ऽथो॒ अथो॒ तेजो॒ वै वै तेजो ऽथो॒ अथो॒ तेजो॒ वै । \newline
16. अथो॒ इत्यथो᳚ । \newline
17. तेजो॒ वै वै तेज॒ स्तेजो॒ वै गा॑य॒त्री गा॑य॒त्री वै तेज॒ स्तेजो॒ वै गा॑य॒त्री । \newline
18. वै गा॑य॒त्री गा॑य॒त्री वै वै गा॑य॒त्री तेज॒ स्तेजो॑ गाय॒त्री वै वै गा॑य॒त्री तेजः॑ । \newline
19. गा॒य॒त्री तेज॒ स्तेजो॑ गाय॒त्री गा॑य॒त्री तेजः॑ प्रातस्सव॒नम् प्रा॑तस्सव॒नम् तेजो॑ गाय॒त्री गा॑य॒त्री तेजः॑ प्रातस्सव॒नम् । \newline
20. तेजः॑ प्रातस्सव॒नम् प्रा॑तस्सव॒नम् तेज॒ स्तेजः॑ प्रातस्सव॒नम् तेज॒ स्तेजः॑ प्रातस्सव॒नम् तेज॒ स्तेजः॑ प्रातस्सव॒नम् तेजः॑ । \newline
21. प्रा॒त॒स्स॒व॒नम् तेज॒ स्तेजः॑ प्रातस्सव॒नम् प्रा॑तस्सव॒नम् तेज॑ ए॒वैव तेजः॑ प्रातस्सव॒नम् प्रा॑तस्सव॒नम् तेज॑ ए॒व । \newline
22. प्रा॒त॒स्स॒व॒नमिति॑ प्रातः - स॒व॒नम् । \newline
23. तेज॑ ए॒वैव तेज॒ स्तेज॑ ए॒व प्रा॑तस्सव॒ने प्रा॑तस्सव॒न ए॒व तेज॒ स्तेज॑ ए॒व प्रा॑तस्सव॒ने । \newline
24. ए॒व प्रा॑तस्सव॒ने प्रा॑तस्सव॒न ए॒वैव प्रा॑तस्सव॒न आ॒त्मन् ना॒त्मन् प्रा॑तस्सव॒न ए॒वैव प्रा॑तस्सव॒न आ॒त्मन्न् । \newline
25. प्रा॒त॒स्स॒व॒न आ॒त्मन् ना॒त्मन् प्रा॑तस्सव॒ने प्रा॑तस्सव॒न आ॒त्मन् ध॑त्ते धत्त आ॒त्मन् प्रा॑तस्सव॒ने प्रा॑तस्सव॒न आ॒त्मन् ध॑त्ते । \newline
26. प्रा॒त॒स्स॒व॒न इति॑ प्रातः - स॒व॒ने । \newline
27. आ॒त्मन् ध॑त्ते धत्त आ॒त्मन् ना॒त्मन् ध॑त्त उ॒क्थ मु॒क्थम् ध॑त्त आ॒त्मन् ना॒त्मन् ध॑त्त उ॒क्थम् । \newline
28. ध॒त्त॒ उ॒क्थ मु॒क्थम् ध॑त्ते धत्त उ॒क्थं ॅवा॒चि वा॒च्यु॑क्थम् ध॑त्ते धत्त उ॒क्थं ॅवा॒चि । \newline
29. उ॒क्थं ॅवा॒चि वा॒च्यु॑क्थ मु॒क्थं ॅवा॒चीतीति॑ वा॒च्यु॑क्थ मु॒क्थं ॅवा॒चीति॑ । \newline
30. वा॒चीतीति॑ वा॒चि वा॒ची त्या॑हा॒हे ति॑ वा॒चि वा॒चीत्या॑ह । \newline
31. इत्या॑हा॒हे तीत्या॑ह॒ माद्ध्य॑न्दिन॒म् माद्ध्य॑न्दिन मा॒हे तीत्या॑ह॒ माद्ध्य॑न्दिनम् । \newline
32. आ॒ह॒ माद्ध्य॑न्दिन॒म् माद्ध्य॑न्दिन माहाह॒ माद्ध्य॑न्दिनꣳ॒॒ सव॑नꣳ॒॒ सव॑न॒म् माद्ध्य॑न्दिन माहाह॒ माद्ध्य॑न्दिनꣳ॒॒ सव॑नम् । \newline
33. माद्ध्य॑न्दिनꣳ॒॒ सव॑नꣳ॒॒ सव॑न॒म् माद्ध्य॑न्दिन॒म् माद्ध्य॑न्दिनꣳ॒॒ सव॑नम् प्रति॒गीर्य॑ प्रति॒गीर्य॒ सव॑न॒म् माद्ध्य॑न्दिन॒म् माद्ध्य॑न्दिनꣳ॒॒ सव॑नम् प्रति॒गीर्य॑ । \newline
34. सव॑नम् प्रति॒गीर्य॑ प्रति॒गीर्य॒ सव॑नꣳ॒॒ सव॑नम् प्रति॒गीर्य॑ च॒त्वारि॑ च॒त्वारि॑ प्रति॒गीर्य॒ सव॑नꣳ॒॒ सव॑नम् प्रति॒गीर्य॑ च॒त्वारि॑ । \newline
35. प्र॒ति॒गीर्य॑ च॒त्वारि॑ च॒त्वारि॑ प्रति॒गीर्य॑ प्रति॒गीर्य॑ च॒त्वा र्ये॒ता न्ये॒तानि॑ च॒त्वारि॑ प्रति॒गीर्य॑ प्रति॒गीर्य॑ च॒त्वा र्ये॒तानि॑ । \newline
36. प्र॒ति॒गीर्येति॑ प्रति - गीर्य॑ । \newline
37. च॒त्वा र्ये॒ता न्ये॒तानि॑ च॒त्वारि॑ च॒त्वा र्ये॒ता न्य॒क्षरा᳚ ण्य॒क्षरा᳚ ण्ये॒तानि॑ च॒त्वारि॑ च॒त्वा र्ये॒ता न्य॒क्षरा॑णि । \newline
38. ए॒ता न्य॒क्षरा᳚ ण्य॒क्षरा᳚ ण्ये॒ता न्ये॒ता न्य॒क्षरा॑णि॒ चतु॑ष्पदा॒ चतु॑ष्पदा॒ ऽक्षरा᳚ण्ये॒ता न्ये॒ता न्य॒क्षरा॑णि॒ चतु॑ष्पदा । \newline
39. अ॒क्षरा॑णि॒ चतु॑ष्पदा॒ चतु॑ष्पदा॒ ऽक्षरा᳚ ण्य॒क्षरा॑णि॒ चतु॑ष्पदा त्रि॒ष्टुप् त्रि॒ष्टुप् 
चतु॑ष्पदा॒ ऽक्षरा᳚ ण्य॒क्षरा॑णि॒ चतु॑ष्पदा त्रि॒ष्टुप् । \newline
40. चतु॑ष्पदा त्रि॒ष्टुप् त्रि॒ष्टुप् चतु॑ष्पदा॒ चतु॑ष्पदा त्रि॒ष्टुप् त्रैष्टु॑भ॒म् त्रैष्टु॑भम् त्रि॒ष्टुप् चतु॑ष्पदा॒ चतु॑ष्पदा त्रि॒ष्टुप् त्रैष्टु॑भम् । \newline
41. चतु॑ष्प॒देति॒ चतुः॑ - प॒दा॒ । \newline
42. त्रि॒ष्टुप् त्रैष्टु॑भ॒म् त्रैष्टु॑भम् त्रि॒ष्टुप् त्रि॒ष्टुप् त्रैष्टु॑भ॒म् माद्ध्य॑न्दिन॒म् माद्ध्य॑न्दिन॒म् त्रैष्टु॑भम् त्रि॒ष्टुप् त्रि॒ष्टुप् त्रैष्टु॑भ॒म् माद्ध्य॑न्दिनम् । \newline
43. त्रैष्टु॑भ॒म् माद्ध्य॑न्दिन॒म् माद्ध्य॑न्दिन॒म् त्रैष्टु॑भ॒म् त्रैष्टु॑भ॒म् माद्ध्य॑न्दिनꣳ॒॒ सव॑नꣳ॒॒ सव॑न॒म् माद्ध्य॑न्दिन॒म् त्रैष्टु॑भ॒म् त्रैष्टु॑भ॒म् माद्ध्य॑न्दिनꣳ॒॒ सव॑नम् । \newline
44. माद्ध्य॑न्दिनꣳ॒॒ सव॑नꣳ॒॒ सव॑न॒म् माद्ध्य॑न्दिन॒म् माद्ध्य॑न्दिनꣳ॒॒ सव॑न॒म् माद्ध्य॑न्दिने॒ माद्ध्य॑न्दिने॒ सव॑न॒म् माद्ध्य॑न्दिन॒म् माद्ध्य॑न्दिनꣳ॒॒ सव॑न॒म् माद्ध्य॑न्दिने । \newline
45. सव॑न॒म् माद्ध्य॑न्दिने॒ माद्ध्य॑न्दिने॒ सव॑नꣳ॒॒ सव॑न॒म् माद्ध्य॑न्दिन ए॒वैव माद्ध्य॑न्दिने॒ सव॑नꣳ॒॒ सव॑न॒म् माद्ध्य॑न्दिन ए॒व । \newline
46. माद्ध्य॑न्दिन ए॒वैव माद्ध्य॑न्दिने॒ माद्ध्य॑न्दिन ए॒व सव॑ने॒ सव॑न ए॒व माद्ध्य॑न्दिने॒ माद्ध्य॑न्दिन ए॒व सव॑ने । \newline
47. ए॒व सव॑ने॒ सव॑न ए॒वैव सव॑ने प्रतिग॒रे प्र॑तिग॒रे सव॑न ए॒वैव सव॑ने प्रतिग॒रे । \newline
48. सव॑ने प्रतिग॒रे प्र॑तिग॒रे सव॑ने॒ सव॑ने प्रतिग॒रे छन्दाꣳ॑सि॒ छन्दाꣳ॑सि प्रतिग॒रे सव॑ने॒ सव॑ने प्रतिग॒रे छन्दाꣳ॑सि । \newline
49. प्र॒ति॒ग॒रे छन्दाꣳ॑सि॒ छन्दाꣳ॑सि प्रतिग॒रे प्र॑तिग॒रे छन्दाꣳ॑सि॒ सꣳ सम् छन्दाꣳ॑सि प्रतिग॒रे प्र॑तिग॒रे छन्दाꣳ॑सि॒ सम् । \newline
50. प्र॒ति॒ग॒र इति॑ प्रति - ग॒रे । \newline
51. छन्दाꣳ॑सि॒ सꣳ सम् छन्दाꣳ॑सि॒ छन्दाꣳ॑सि॒ सम् पा॑दयति पादयति॒ सम् छन्दाꣳ॑सि॒ छन्दाꣳ॑सि॒ सम् पा॑दयति । \newline
52. सम् पा॑दयति पादयति॒ सꣳ सम् पा॑दय॒ त्यथो॒ अथो॑ पादयति॒ सꣳ सम् पा॑दय॒ त्यथो᳚ । \newline
53. पा॒द॒य॒ त्यथो॒ अथो॑ पादयति पादय॒ त्यथो॑ इन्द्रि॒य मि॑न्द्रि॒य मथो॑ पादयति पादय॒ त्यथो॑ इन्द्रि॒यम् । \newline
54. अथो॑ इन्द्रि॒य मि॑न्द्रि॒य मथो॒ अथो॑ इन्द्रि॒यं ॅवै वा इ॑न्द्रि॒य मथो॒ अथो॑ इन्द्रि॒यं ॅवै । \newline
55. अथो॒ इत्यथो᳚ । \newline
56. इ॒न्द्रि॒यं ॅवै वा इ॑न्द्रि॒य मि॑न्द्रि॒यं ॅवै त्रि॒ष्टुक् त्रि॒ष्टुग् वा इ॑न्द्रि॒य मि॑न्द्रि॒यं ॅवै त्रि॒ष्टुक् । \newline
57. वै त्रि॒ष्टुक् त्रि॒ष्टुग् वै वै त्रि॒ष्टु गि॑न्द्रि॒य मि॑न्द्रि॒यम् त्रि॒ष्टुग् वै वै त्रि॒ष्टु गि॑न्द्रि॒यम् । \newline
58. त्रि॒ष्टु गि॑न्द्रि॒य मि॑न्द्रि॒यम् त्रि॒ष्टुक् त्रि॒ष्टु गि॑न्द्रि॒यम् माद्ध्य॑न्दिन॒म् माद्ध्य॑न्दिन मिन्द्रि॒यम् त्रि॒ष्टुक् त्रि॒ष्टु गि॑न्द्रि॒यम् माद्ध्य॑न्दिनम् । \newline
59. इ॒न्द्रि॒यम् माद्ध्य॑न्दिन॒म् माद्ध्य॑न्दिन मिन्द्रि॒य मि॑न्द्रि॒यम् माद्ध्य॑न्दिनꣳ॒॒ सव॑नꣳ॒॒ सव॑न॒म् माद्ध्य॑न्दिन मिन्द्रि॒य मि॑न्द्रि॒यम् माद्ध्य॑न्दिनꣳ॒॒ सव॑नम् । \newline
60. माद्ध्य॑न्दिनꣳ॒॒ सव॑नꣳ॒॒ सव॑न॒म् माद्ध्य॑न्दिन॒म् माद्ध्य॑न्दिनꣳ॒॒ सव॑न मिन्द्रि॒य मि॑न्द्रि॒यꣳ सव॑न॒म् माद्ध्य॑न्दिन॒म् माद्ध्य॑न्दिनꣳ॒॒ सव॑न मिन्द्रि॒यम् । \newline
61. सव॑न मिन्द्रि॒य मि॑न्द्रि॒यꣳ सव॑नꣳ॒॒ सव॑न मिन्द्रि॒य मे॒वैवे न्द्रि॒यꣳ सव॑नꣳ॒॒ सव॑न मिन्द्रि॒य मे॒व । \newline
\pagebreak
\markright{ TS 3.2.9.4  \hfill https://www.vedavms.in \hfill}

\section{ TS 3.2.9.4 }

\textbf{TS 3.2.9.4 } \newline
\textbf{Samhita Paata} \newline

-मिन्द्रि॒यमे॒व माद्ध्य॑न्दिने॒ सव॑न आ॒त्मन् ध॑त्त उ॒क्थं ॅवा॒चीन्द्रा॒येत्या॑ह तृतीयसव॒नं प्र॑ति॒गीर्य॑ स॒प्तैतान्य॒क्षरा॑णि स॒प्तप॑दा॒ शक्व॑री शाक्व॒राः प॒शवो॒ जाग॑तं तृतीयसव॒नं तृ॑तीयसव॒न ए॒व प्र॑तिग॒रे छन्दाꣳ॑सि॒ संपा॑दय॒त्यथो॑ प॒शवो॒ वै जग॑ती प॒शव॑स्तृतीयसव॒नं प॒शूने॒व तृ॑तीयसव॒न आ॒त्मन् ध॑त्ते॒ यद्वै होता᳚ऽद्ध्व॒र्युम॑भ्या॒ह्वय॑त आ॒व्य॑मस्मिन् दधाति॒ तद्यन्ना - [  ] \newline

\textbf{Pada Paata} \newline

इ॒न्द्रि॒यम् । ए॒व । माद्ध्य॑न्दिने । सव॑ने । आ॒त्मन्न् । ध॒त्ते॒ । उ॒क्थम् । वा॒चि । इन्द्रा॑य । इति॑ । आ॒ह॒ । तृ॒ती॒य॒स॒व॒नमिति॑ तृतीय - स॒व॒नम् । प्र॒ति॒गीर्येति॑ प्रति - गीर्य॑ । स॒प्त । ए॒तानि॑ । अ॒क्षरा॑णि । स॒प्तप॒देति॑ स॒प्त - प॒दा॒ । शक्व॑री । शा॒क्व॒राः । प॒शवः॑ । जाग॑तम् । तृ॒ती॒य॒स॒व॒नमिति॑ तृतीय - स॒व॒नम् । तृ॒ती॒य॒स॒व॒न इति॑ तृतीय - स॒व॒ने । ए॒व । प्र॒ति॒ग॒र इति॑ प्रति - ग॒रे । छन्दाꣳ॑सि । समिति॑ । पा॒द॒य॒ति॒ । अथो॒ इति॑ । प॒शवः॑ । वै । जग॑ती । प॒शवः॑ । तृ॒ती॒य॒स॒व॒नमिति॑ तृतीय - स॒व॒नम् । प॒शून् । ए॒व । तृ॒ती॒य॒स॒व॒न इति॑ तृतीय - स॒व॒ने । आ॒त्मन्न् । ध॒त्ते॒ । यत् । वै । होता᳚ । अ॒द्ध्व॒र्युम् । अ॒भ्या॒ह्वय॑त॒ इत्य॑भि - आ॒ह्वय॑ते । आ॒व्य᳚म् । अ॒स्मि॒न्न् । द॒धा॒ति॒ । तत् । यत् । न ।  \newline


\textbf{Krama Paata} \newline

इ॒न्द्रि॒यमे॒व । ए॒व माद्ध्य॑न्दिने । माद्ध्य॑न्दिने॒ सव॑ने । सव॑न आ॒त्मन्न् । आ॒त्मन् ध॑त्ते । ध॒त्त॒ उ॒क्थम् । उ॒क्थं ॅवा॒चि । वा॒चीन्द्रा॑य । इन्द्रा॒येति॑ । इत्या॑ह । आ॒ह॒ तृ॒ती॒य॒स॒व॒नम् । तृ॒ती॒य॒स॒व॒नम् प्र॑ति॒गीर्य॑ । तृ॒ती॒य॒स॒व॒नमिति॑ तृतीय - स॒व॒नम् । प्र॒ति॒गीर्य॑ स॒प्त । प्र॒ति॒गीर्येति॑ प्रति - गीर्य॑ । स॒प्तैतानि॑ । ए॒तान्य॒क्षरा॑णि । अ॒क्षरा॑णि स॒प्तप॑दा । स॒प्तप॑दा॒ शक्व॑री । स॒प्तप॒देति॑ स॒प्त - प॒दा॒ । शक्व॑री शाक्व॒राः । शा॒क्व॒राः प॒शवः॑ । प॒शवो॒ जाग॑तम् । जाग॑तम् तृतीयसव॒नम् । तृ॒ती॒य॒स॒व॒नम् तृ॑तीयसव॒ने । तृ॒ती॒य॒स॒व॒नमिति॑ तृतीय - स॒व॒नम् । तृ॒ती॒य॒स॒व॒न ए॒व । तृ॒ती॒य॒स॒व॒न इति॑ तृतीय - स॒व॒ने । ए॒व प्र॑तिग॒रे । प्र॒ति॒ग॒रे छन्दाꣳ॑सि । प्र॒ति॒ग॒र इति॑ प्रति - ग॒रे । छन्दाꣳ॑सि॒ सम् । सम् पा॑दयति । पा॒द॒य॒त्यथो᳚ । अथो॑ प॒शवः॑ । अथो॒ इत्यथो᳚ । प॒शवो॒ वै । वै जग॑ती । जग॑ती प॒शवः॑ । प॒शव॑ स्तृतीयसव॒नम् । तृ॒ती॒य॒स॒व॒नम् प॒शून् । तृ॒ती॒य॒स॒व॒नमिति॑ तृतीय - स॒व॒नम् । प॒शूने॒व । ए॒व तृ॑तीयसव॒ने । तृ॒ती॒य॒स॒व॒न आ॒त्मन्न् । तृ॒ती॒य॒स॒व॒न इति॑ तृतीय - स॒व॒ने । आ॒त्मन् ध॑त्ते । ध॒त्ते॒ यत् । यद् वै । वै होता᳚ । होता᳚ ऽद्ध्व॒र्युम् । अ॒द्ध्व॒र्यु,म॑भ्या॒ह्वय॑ते । अ॒भ्या॒ह्वय॑त आ॒व्य᳚म् । अ॒भ्या॒ह्वय॑त॒ इत्य॑भि - आ॒ह्वय॑ते । आ॒व्य॑मस्मिन्न् । अ॒स्मि॒न् द॒धा॒ति॒ । द॒धा॒ति॒ तत् । तद् यत् । यन् न । ना॒प॒हनी॑त \newline

\textbf{Jatai Paata} \newline

1. इ॒न्द्रि॒य मे॒वैवे न्द्रि॒य मि॑न्द्रि॒य मे॒व । \newline
2. ए॒व माद्ध्य॑न्दिने॒ माद्ध्य॑न्दिन ए॒वैव माद्ध्य॑न्दिने । \newline
3. माद्ध्य॑न्दिने॒ सव॑ने॒ सव॑ने॒ माद्ध्य॑न्दिने॒ माद्ध्य॑न्दिने॒ सव॑ने । \newline
4. सव॑न आ॒त्मन् ना॒त्मन् थ्सव॑ने॒ सव॑न आ॒त्मन्न् । \newline
5. आ॒त्मन् ध॑त्ते धत्त आ॒त्मन् ना॒त्मन् ध॑त्ते । \newline
6. ध॒त्त॒ उ॒क्थ मु॒क्थम् ध॑त्ते धत्त उ॒क्थम् । \newline
7. उ॒क्थं ॅवा॒चि वा॒च्यु॑क्थ मु॒क्थं ॅवा॒चि । \newline
8. वा॒चीन्द्रा॒ येन्द्रा॑य वा॒चि वा॒चीन्द्रा॑य । \newline
9. इन्द्रा॒ येतीतीन्द्रा॒ येन्द्रा॒ये ति॑ । \newline
10. इत्या॑हा॒हे तीत्या॑ह । \newline
11. आ॒ह॒ तृ॒ती॒य॒स॒व॒नम् तृ॑तीयसव॒न मा॑हाह तृतीयसव॒नम् । \newline
12. तृ॒ती॒य॒स॒व॒नम् प्र॑ति॒गीर्य॑ प्रति॒गीर्य॑ तृतीयसव॒नम् तृ॑तीयसव॒नम् प्र॑ति॒गीर्य॑ । \newline
13. तृ॒ती॒य॒स॒व॒नमिति॑ तृतीय - स॒व॒नम् । \newline
14. प्र॒ति॒गीर्य॑ स॒प्त स॒प्त प्र॑ति॒गीर्य॑ प्रति॒गीर्य॑ स॒प्त । \newline
15. प्र॒ति॒गीर्येति॑ प्रति - गीर्य॑ । \newline
16. स॒प्तैता न्ये॒तानि॑ स॒प्त स॒प्तैतानि॑ । \newline
17. ए॒ता न्य॒क्षरा᳚ ण्य॒क्षरा᳚ ण्ये॒ता न्ये॒ता न्य॒क्षरा॑णि । \newline
18. अ॒क्षरा॑णि स॒प्तप॑दा स॒प्तप॑दा॒ ऽक्षरा᳚ ण्य॒क्षरा॑णि स॒प्तप॑दा । \newline
19. स॒प्तप॑दा॒ शक्व॑री॒ शक्व॑री स॒प्तप॑दा स॒प्तप॑दा॒ शक्व॑री । \newline
20. स॒प्तप॒देति॑ स॒प्त - प॒दा॒ । \newline
21. शक्व॑री शाक्व॒राः शा᳚क्व॒राः शक्व॑री॒ शक्व॑री शाक्व॒राः । \newline
22. शा॒क्व॒राः प॒शवः॑ प॒शवः॑ शाक्व॒राः शा᳚क्व॒राः प॒शवः॑ । \newline
23. प॒शवो॒ जाग॑त॒म् जाग॑तम् प॒शवः॑ प॒शवो॒ जाग॑तम् । \newline
24. जाग॑तम् तृतीयसव॒नम् तृ॑तीयसव॒नम् जाग॑त॒म् जाग॑तम् तृतीयसव॒नम् । \newline
25. तृ॒ती॒य॒स॒व॒नम् तृ॑तीयसव॒ने तृ॑तीयसव॒ने तृ॑तीयसव॒नम् तृ॑तीयसव॒नम् तृ॑तीयसव॒ने । \newline
26. तृ॒ती॒य॒स॒व॒नमिति॑ तृतीय - स॒व॒नम् । \newline
27. तृ॒ती॒य॒स॒व॒न ए॒वैव तृ॑तीयसव॒ने तृ॑तीयसव॒न ए॒व । \newline
28. तृ॒ती॒य॒स॒व॒न इति॑ तृतीय - स॒व॒ने । \newline
29. ए॒व प्र॑तिग॒रे प्र॑तिग॒र ए॒वैव प्र॑तिग॒रे । \newline
30. प्र॒ति॒ग॒रे छन्दाꣳ॑सि॒ छन्दाꣳ॑सि प्रतिग॒रे प्र॑तिग॒रे छन्दाꣳ॑सि । \newline
31. प्र॒ति॒ग॒र इति॑ प्रति - ग॒रे । \newline
32. छन्दाꣳ॑सि॒ सꣳ सम् छन्दाꣳ॑सि॒ छन्दाꣳ॑सि॒ सम् । \newline
33. सम् पा॑दयति पादयति॒ सꣳ सम् पा॑दयति । \newline
34. पा॒द॒य॒ त्यथो॒ अथो॑ पादयति पादय॒ त्यथो᳚ । \newline
35. अथो॑ प॒शवः॑ प॒शवो ऽथो॒ अथो॑ प॒शवः॑ । \newline
36. अथो॒ इत्यथो᳚ । \newline
37. प॒शवो॒ वै वै प॒शवः॑ प॒शवो॒ वै । \newline
38. वै जग॑ती॒ जग॑ती॒ वै वै जग॑ती । \newline
39. जग॑ती प॒शवः॑ प॒शवो॒ जग॑ती॒ जग॑ती प॒शवः॑ । \newline
40. प॒शव॑ स्तृतीयसव॒नम् तृ॑तीयसव॒नम् प॒शवः॑ प॒शव॑ स्तृतीयसव॒नम् । \newline
41. तृ॒ती॒य॒स॒व॒नम् प॒शून् प॒शून् तृ॑तीयसव॒नम् तृ॑तीयसव॒नम् प॒शून् । \newline
42. तृ॒ती॒य॒स॒व॒नमिति॑ तृतीय - स॒व॒नम् । \newline
43. प॒शू ने॒वैव प॒शून् प॒शू ने॒व । \newline
44. ए॒व तृ॑तीयसव॒ने तृ॑तीयसव॒न ए॒वैव तृ॑तीयसव॒ने । \newline
45. तृ॒ती॒य॒स॒व॒न आ॒त्मन् ना॒त्मन् तृ॑तीयसव॒ने तृ॑तीयसव॒न आ॒त्मन्न् । \newline
46. तृ॒ती॒य॒स॒व॒न इति॑ तृतीय - स॒व॒ने । \newline
47. आ॒त्मन् ध॑त्ते धत्त आ॒त्मन् ना॒त्मन् ध॑त्ते । \newline
48. ध॒त्ते॒ यद् यद् ध॑त्ते धत्ते॒ यत् । \newline
49. यद् वै वै यद् यद् वै । \newline
50. वै होता॒ होता॒ वै वै होता᳚ । \newline
51. होता᳚ ऽद्ध्व॒र्यु म॑द्ध्व॒र्युꣳ होता॒ होता᳚ ऽद्ध्व॒र्युम् । \newline
52. अ॒द्ध्व॒र्यु म॑भ्या॒ह्वय॑ते ऽभ्या॒ह्वय॑ते ऽद्ध्व॒र्यु म॑द्ध्व॒र्यु म॑भ्या॒ह्वय॑ते । \newline
53. अ॒भ्या॒ह्वय॑त आ॒व्य॑ मा॒व्य॑ मभ्या॒ह्वय॑ते ऽभ्या॒ह्वय॑त आ॒व्य᳚म् । \newline
54. अ॒भ्या॒ह्वय॑त॒ इत्य॑भि - आ॒ह्वय॑ते । \newline
55. आ॒व्य॑ मस्मिन् नस्मिन् ना॒व्य॑ मा॒व्य॑ मस्मिन्न् । \newline
56. अ॒स्मि॒न् द॒धा॒ति॒ द॒धा॒ त्य॒स्मि॒न् न॒स्मि॒न् द॒धा॒ति॒ । \newline
57. द॒धा॒ति॒ तत् तद् द॑धाति दधाति॒ तत् । \newline
58. तद् यद् यत् तत् तद् यत् । \newline
59. यन् न न यद् यन् न । \newline
60. नाप॒हनी॑ता प॒हनी॑त॒ न नाप॒हनी॑त । \newline

\textbf{Ghana Paata } \newline

1. इ॒न्द्रि॒य मे॒वैवे न्द्रि॒य मि॑न्द्रि॒य मे॒व माद्ध्य॑न्दिने॒ माद्ध्य॑न्दिन ए॒वे न्द्रि॒य मि॑न्द्रि॒य मे॒व माद्ध्य॑न्दिने । \newline
2. ए॒व माद्ध्य॑न्दिने॒ माद्ध्य॑न्दिन ए॒वैव माद्ध्य॑न्दिने॒ सव॑ने॒ सव॑ने॒ माद्ध्य॑न्दिन ए॒वैव माद्ध्य॑न्दिने॒ सव॑ने । \newline
3. माद्ध्य॑न्दिने॒ सव॑ने॒ सव॑ने॒ माद्ध्य॑न्दिने॒ माद्ध्य॑न्दिने॒ सव॑न आ॒त्मन् ना॒त्मन् थ्सव॑ने॒ माद्ध्य॑न्दिने॒ माद्ध्य॑न्दिने॒ सव॑न आ॒त्मन्न् । \newline
4. सव॑न आ॒त्मन् ना॒त्मन् थ्सव॑ने॒ सव॑न आ॒त्मन् ध॑त्ते धत्त आ॒त्मन् थ्सव॑ने॒ सव॑न आ॒त्मन् ध॑त्ते । \newline
5. आ॒त्मन् ध॑त्ते धत्त आ॒त्मन् ना॒त्मन् ध॑त्त उ॒क्थ मु॒क्थम् ध॑त्त आ॒त्मन् ना॒त्मन् ध॑त्त उ॒क्थम् । \newline
6. ध॒त्त॒ उ॒क्थ मु॒क्थम् ध॑त्ते धत्त उ॒क्थं ॅवा॒चि वा॒च्यु॑क्थम् ध॑त्ते धत्त उ॒क्थं ॅवा॒चि । \newline
7. उ॒क्थं ॅवा॒चि वा॒च्यु॑क्थ मु॒क्थं ॅवा॒चीन्द्रा॒ये न्द्रा॑य वा॒च्यु॑क्थ मु॒क्थं ॅवा॒चीन्द्रा॑य । \newline
8. वा॒चीन्द्रा॒ये न्द्रा॑य वा॒चि वा॒चीन्द्रा॒ये तीतीन्द्रा॑य वा॒चि वा॒चीन्द्रा॒ये ति॑ । \newline
9. इन्द्रा॒ये तीतीन्द्रा॒ये न्द्रा॒ये त्या॑हा॒हे तीन्द्रा॒ये न्द्रा॒ये त्या॑ह । \newline
10. इत्या॑हा॒हे तीत्या॑ह तृतीयसव॒नम् तृ॑तीयसव॒न मा॒हे तीत्या॑ह तृतीयसव॒नम् । \newline
11. आ॒ह॒ तृ॒ती॒य॒स॒व॒नम् तृ॑तीयसव॒न मा॑हाह तृतीयसव॒नम् प्र॑ति॒गीर्य॑ प्रति॒गीर्य॑ तृतीयसव॒न मा॑हाह तृतीयसव॒नम् प्र॑ति॒गीर्य॑ । \newline
12. तृ॒ती॒य॒स॒व॒नम् प्र॑ति॒गीर्य॑ प्रति॒गीर्य॑ तृतीयसव॒नम् तृ॑तीयसव॒नम् प्र॑ति॒गीर्य॑ स॒प्त स॒प्त प्र॑ति॒गीर्य॑ तृतीयसव॒नम् तृ॑तीयसव॒नम् प्र॑ति॒गीर्य॑ स॒प्त । \newline
13. तृ॒ती॒य॒स॒व॒नमिति॑ तृतीय - स॒व॒नम् । \newline
14. प्र॒ति॒गीर्य॑ स॒प्त स॒प्त प्र॑ति॒गीर्य॑ प्रति॒गीर्य॑ स॒प्तैता न्ये॒तानि॑ स॒प्त प्र॑ति॒गीर्य॑ प्रति॒गीर्य॑ स॒प्तैतानि॑ । \newline
15. प्र॒ति॒गीर्येति॑ प्रति - गीर्य॑ । \newline
16. स॒प्तैता न्ये॒तानि॑ स॒प्त स॒प्तैता न्य॒क्षरा᳚ ण्य॒क्षरा᳚ ण्ये॒तानि॑ स॒प्त स॒प्तैता न्य॒क्षरा॑णि । \newline
17. ए॒ता न्य॒क्षरा᳚ ण्य॒क्षरा᳚ ण्ये॒ता न्ये॒ता न्य॒क्षरा॑णि स॒प्तप॑दा 
स॒प्तप॑दा॒ ऽक्षरा᳚ ण्ये॒ता न्ये॒ता न्य॒क्षरा॑णि स॒प्तप॑दा । \newline
18. अ॒क्षरा॑णि स॒प्तप॑दा स॒प्तप॑दा॒ ऽक्षरा᳚ ण्य॒क्षरा॑णि स॒प्तप॑दा॒ शक्व॑री॒ शक्व॑री 
स॒प्तप॑दा॒ ऽक्षरा᳚ ण्य॒क्षरा॑णि स॒प्तप॑दा॒ शक्व॑री । \newline
19. स॒प्तप॑दा॒ शक्व॑री॒ शक्व॑री स॒प्तप॑दा स॒प्तप॑दा॒ शक्व॑री शाक्व॒राः शा᳚क्व॒राः शक्व॑री स॒प्तप॑दा स॒प्तप॑दा॒ शक्व॑री शाक्व॒राः । \newline
20. स॒प्तप॒देति॑ स॒प्त - प॒दा॒ । \newline
21. शक्व॑री शाक्व॒राः शा᳚क्व॒राः शक्व॑री॒ शक्व॑री शाक्व॒राः प॒शवः॑ प॒शवः॑ शाक्व॒राः शक्व॑री॒ शक्व॑री शाक्व॒राः प॒शवः॑ । \newline
22. शा॒क्व॒राः प॒शवः॑ प॒शवः॑ शाक्व॒राः शा᳚क्व॒राः प॒शवो॒ जाग॑त॒म् जाग॑तम् प॒शवः॑ शाक्व॒राः शा᳚क्व॒राः प॒शवो॒ जाग॑तम् । \newline
23. प॒शवो॒ जाग॑त॒म् जाग॑तम् प॒शवः॑ प॒शवो॒ जाग॑तम् तृतीयसव॒नम् तृ॑तीयसव॒नम् जाग॑तम् प॒शवः॑ प॒शवो॒ जाग॑तम् तृतीयसव॒नम् । \newline
24. जाग॑तम् तृतीयसव॒नम् तृ॑तीयसव॒नम् जाग॑त॒म् जाग॑तम् तृतीयसव॒नम् तृ॑तीयसव॒ने तृ॑तीयसव॒ने तृ॑तीयसव॒नम् जाग॑त॒म् जाग॑तम् तृतीयसव॒नम् तृ॑तीयसव॒ने । \newline
25. तृ॒ती॒य॒स॒व॒नम् तृ॑तीयसव॒ने तृ॑तीयसव॒ने तृ॑तीयसव॒नम् तृ॑तीयसव॒नम् तृ॑तीयसव॒न ए॒वैव तृ॑तीयसव॒ने तृ॑तीयसव॒नम् तृ॑तीयसव॒नम् तृ॑तीयसव॒न ए॒व । \newline
26. तृ॒ती॒य॒स॒व॒नमिति॑ तृतीय - स॒व॒नम् । \newline
27. तृ॒ती॒य॒स॒व॒न ए॒वैव तृ॑तीयसव॒ने तृ॑तीयसव॒न ए॒व प्र॑तिग॒रे प्र॑तिग॒र ए॒व तृ॑तीयसव॒ने तृ॑तीयसव॒न ए॒व प्र॑तिग॒रे । \newline
28. तृ॒ती॒य॒स॒व॒न इति॑ तृतीय - स॒व॒ने । \newline
29. ए॒व प्र॑तिग॒रे प्र॑तिग॒र ए॒वैव प्र॑तिग॒रे छन्दाꣳ॑सि॒ छन्दाꣳ॑सि प्रतिग॒र ए॒वैव प्र॑तिग॒रे छन्दाꣳ॑सि । \newline
30. प्र॒ति॒ग॒रे छन्दाꣳ॑सि॒ छन्दाꣳ॑सि प्रतिग॒रे प्र॑तिग॒रे छन्दाꣳ॑सि॒ सꣳ सम् छन्दाꣳ॑सि प्रतिग॒रे प्र॑तिग॒रे छन्दाꣳ॑सि॒ सम् । \newline
31. प्र॒ति॒ग॒र इति॑ प्रति - ग॒रे । \newline
32. छन्दाꣳ॑सि॒ सꣳ सम् छन्दाꣳ॑सि॒ छन्दाꣳ॑सि॒ सम् पा॑दयति पादयति॒ सम् छन्दाꣳ॑सि॒ छन्दाꣳ॑सि॒ सम् पा॑दयति । \newline
33. सम् पा॑दयति पादयति॒ सꣳ सम् पा॑दय॒ त्यथो॒ अथो॑ पादयति॒ सꣳ सम् पा॑दय॒ त्यथो᳚ । \newline
34. पा॒द॒य॒ त्यथो॒ अथो॑ पादयति पादय॒ त्यथो॑ प॒शवः॑ प॒शवो ऽथो॑ पादयति पादय॒ त्यथो॑ प॒शवः॑ । \newline
35. अथो॑ प॒शवः॑ प॒शवो ऽथो॒ अथो॑ प॒शवो॒ वै वै प॒शवो ऽथो॒ अथो॑ प॒शवो॒ वै । \newline
36. अथो॒ इत्यथो᳚ । \newline
37. प॒शवो॒ वै वै प॒शवः॑ प॒शवो॒ वै जग॑ती॒ जग॑ती॒ वै प॒शवः॑ प॒शवो॒ वै जग॑ती । \newline
38. वै जग॑ती॒ जग॑ती॒ वै वै जग॑ती प॒शवः॑ प॒शवो॒ जग॑ती॒ वै वै जग॑ती प॒शवः॑ । \newline
39. जग॑ती प॒शवः॑ प॒शवो॒ जग॑ती॒ जग॑ती प॒शव॑ स्तृतीयसव॒नम् तृ॑तीयसव॒नम् प॒शवो॒ जग॑ती॒ जग॑ती प॒शव॑ स्तृतीयसव॒नम् । \newline
40. प॒शव॑ स्तृतीयसव॒नम् तृ॑तीयसव॒नम् प॒शवः॑ प॒शव॑ स्तृतीयसव॒नम् प॒शून् प॒शून् तृ॑तीयसव॒नम् प॒शवः॑ प॒शव॑ स्तृतीयसव॒नम् प॒शून् । \newline
41. तृ॒ती॒य॒स॒व॒नम् प॒शून् प॒शून् तृ॑तीयसव॒नम् तृ॑तीयसव॒नम् प॒शू ने॒वैव प॒शून् तृ॑तीयसव॒नम् तृ॑तीयसव॒नम् प॒शू ने॒व । \newline
42. तृ॒ती॒य॒स॒व॒नमिति॑ तृतीय - स॒व॒नम् । \newline
43. प॒शू ने॒वैव प॒शून् प॒शू ने॒व तृ॑तीयसव॒ने तृ॑तीयसव॒न ए॒व प॒शून् प॒शू ने॒व तृ॑तीयसव॒ने । \newline
44. ए॒व तृ॑तीयसव॒ने तृ॑तीयसव॒न ए॒वैव तृ॑तीयसव॒न आ॒त्मन् ना॒त्मन् तृ॑तीयसव॒न ए॒वैव तृ॑तीयसव॒न आ॒त्मन्न् । \newline
45. तृ॒ती॒य॒स॒व॒न आ॒त्मन् ना॒त्मन् तृ॑तीयसव॒ने तृ॑तीयसव॒न आ॒त्मन् ध॑त्ते धत्त आ॒त्मन् तृ॑तीयसव॒ने तृ॑तीयसव॒न आ॒त्मन् ध॑त्ते । \newline
46. तृ॒ती॒य॒स॒व॒न इति॑ तृतीय - स॒व॒ने । \newline
47. आ॒त्मन् ध॑त्ते धत्त आ॒त्मन् ना॒त्मन् ध॑त्ते॒ यद् यद् ध॑त्त आ॒त्मन् ना॒त्मन् ध॑त्ते॒ यत् । \newline
48. ध॒त्ते॒ यद् यद् ध॑त्ते धत्ते॒ यद् वै वै यद् ध॑त्ते धत्ते॒ यद् वै । \newline
49. यद् वै वै यद् यद् वै होता॒ होता॒ वै यद् यद् वै होता᳚ । \newline
50. वै होता॒ होता॒ वै वै होता᳚ ऽद्ध्व॒र्यु म॑द्ध्व॒र्युꣳ होता॒ वै वै होता᳚ ऽद्ध्व॒र्युम् । \newline
51. होता᳚ ऽद्ध्व॒र्यु म॑द्ध्व॒र्युꣳ होता॒ होता᳚ ऽद्ध्व॒र्यु म॑भ्या॒ह्वय॑ते ऽभ्या॒ह्वय॑ते ऽद्ध्व॒र्युꣳ होता॒ होता᳚ ऽद्ध्व॒र्यु म॑भ्या॒ह्वय॑ते । \newline
52. अ॒द्ध्व॒र्यु म॑भ्या॒ह्वय॑ते ऽभ्या॒ह्वय॑ते ऽद्ध्व॒र्यु म॑द्ध्व॒र्यु म॑भ्या॒ह्वय॑त आ॒व्य॑ मा॒व्य॑ मभ्या॒ह्वय॑ते ऽद्ध्व॒र्यु म॑द्ध्व॒र्यु म॑भ्या॒ह्वय॑त आ॒व्य᳚म् । \newline
53. अ॒भ्या॒ह्वय॑त आ॒व्य॑ मा॒व्य॑ मभ्या॒ह्वय॑ते ऽभ्या॒ह्वय॑त आ॒व्य॑ मस्मिन् नस्मिन् ना॒व्य॑ मभ्या॒ह्वय॑ते ऽभ्या॒ह्वय॑त आ॒व्य॑ मस्मिन्न् । \newline
54. अ॒भ्या॒ह्वय॑त॒ इत्य॑भि - आ॒ह्वय॑ते । \newline
55. आ॒व्य॑ मस्मिन् नस्मिन् ना॒व्य॑ मा॒व्य॑ मस्मिन् दधाति दधा त्यस्मिन् ना॒व्य॑ मा॒व्य॑ मस्मिन् दधाति । \newline
56. अ॒स्मि॒न् द॒धा॒ति॒ द॒धा॒ त्य॒स्मि॒न् न॒स्मि॒न् द॒धा॒ति॒ तत् तद् द॑धा त्यस्मिन् नस्मिन् दधाति॒ तत् । \newline
57. द॒धा॒ति॒ तत् तद् द॑धाति दधाति॒ तद् यद् यत् तद् द॑धाति दधाति॒ तद् यत् । \newline
58. तद् यद् यत् तत् तद् यन् न न यत् तत् तद् यन् न । \newline
59. यन् न न यद् यन् नाप॒हनी॑ता प॒हनी॑त॒ न यद् यन् नाप॒हनी॑त । \newline
60. नाप॒हनी॑ता प॒हनी॑त॒ न नाप॒हनी॑त पु॒रा पु॒रा ऽप॒हनी॑त॒ न नाप॒हनी॑त पु॒रा । \newline
\pagebreak
\markright{ TS 3.2.9.5  \hfill https://www.vedavms.in \hfill}

\section{ TS 3.2.9.5 }

\textbf{TS 3.2.9.5 } \newline
\textbf{Samhita Paata} \newline

प॒हनी॑त पु॒राऽस्य॑ संॅवथ्स॒राद्-गृ॒ह आ वे॑वीर॒ञ्छोꣳसा॒ मोद॑ इ॒वेति॑ प्र॒त्याह्व॑यते॒ तेनै॒व तदप॑ हते॒ यथा॒ वा आय॑तां प्र॒तीक्ष॑त ए॒वम॑द्ध्व॒र्युः प्र॑तिग॒रं प्रती᳚क्षते॒ यद॑भि प्रतिगृणी॒याद्यथा ऽऽय॑तया समृ॒च्छते॑ ता॒दृगे॒व तद्यद॑र्द्ध॒र्चाल्लुप्ये॑त॒ यथा॒ धाव॑द्भ्यो॒ हीय॑ते ता॒दृगे॒व तत् प्र॒बाहु॒ग्वा ऋ॒त्विजा॑मुद्गी॒था उ॑द्गी॒थ ए॒वोद्-गा॑तृ॒णा - [  ] \newline

\textbf{Pada Paata} \newline

अ॒प॒हनी॒तेत्यप॑ - हनी॑त । पु॒रा । अ॒स्य॒ । सं॒ॅव॒थ्स॒रादिति॑ सं - व॒थ्स॒रात् । गृ॒हे । एति॑ । वे॒वी॒र॒न्न् । शोꣳसा᳚ । मोदः॑ । इ॒व॒ । इति॑ । प्र॒त्याह्व॑यत॒ इति॑ प्रति - आह्व॑यते । तेन॑ । ए॒व । तत् । अपेति॑ । ह॒ते॒ । यथा᳚ । वै । आय॑ता॒मित्या - य॒ता॒म् । प्र॒तीक्ष॑त॒ इति॑ प्रति - ईक्ष॑ते । ए॒वम् । अ॒द्ध्व॒र्युः । प्र॒ति॒ग॒रमिति॑ प्रति - ग॒रम् । प्रतीति॑ । ई॒क्ष॒ते॒ । यत् । अ॒भि॒प्र॒ति॒गृ॒णी॒यादित्य॑भि - प्र॒ति॒गृ॒णी॒यात् । यथा᳚ । आय॑त॒येत्या - य॒त॒या॒ । स॒मृ॒च्छत॒ इति॑ सं - ऋ॒च्छते᳚ । ता॒दृक् । ए॒व । तत् । यत् । अ॒र्द्ध॒र्चादित्य॑र्द्ध-ऋ॒चात् । लुप्ये॑त । यथा᳚ । धाव॑द्भ्य॒ इति॒ धाव॑त् - भ्यः॒ । हीय॑ते । ता॒दृक् । ए॒व । तत् । प्र॒बाहु॒गिति॑ प्र - बाहु॑क् । वै । ऋ॒त्विजा᳚म् । उ॒द्गी॒था इत्यु॑त्-गी॒थाः । उ॒द्गी॒थ इत्यु॑त् - गी॒थः । ए॒व । उ॒द्गा॒तृ॒णामित्यु॑त्-गा॒तृ॒णाम् ।  \newline


\textbf{Krama Paata} \newline

अ॒प॒हनी॑त पु॒रा । अ॒प॒हनी॒तेत्य॑प - हनी॑त । पु॒रा ऽस्य॑ । अ॒स्य॒ स॒म्ॅव॒थ्स॒रात् । स॒म्ॅव॒थ्स॒राद् गृ॒हे । स॒म्ॅव॒थ्स॒रादिति॑ सं - व॒थ्स॒रात् । गृ॒ह आ । आ वे॑वीरन्न् । वे॒वी॒र॒ञ्छोꣳसा᳚ । शोꣳसा॒ मोदः॑ । मोद॑ इव । इ॒वेति॑ । इति॑ प्र॒त्याह्व॑यते । प्र॒त्याह्व॑यते॒ तेन॑ । प्र॒त्याह्व॑यत॒ इति॑ प्रति - आह्व॑यते । तेनै॒व । ए॒व तत् । तदप॑ । अप॑ हते । ह॒ते॒ यथा᳚ । यथा॒ वै । वा आय॑ताम् । आय॑ताम् प्र॒तीक्ष॑ते । आय॑ता॒मित्या - य॒ता॒म् । प्र॒तीक्ष॑त ए॒वम् । प्र॒तीक्ष॑त॒ इति॑ प्रति - ईक्ष॑ते । ए॒वम॑द्ध्व॒र्युः । अ॒द्ध्व॒र्युः प्र॑तिग॒रम् । प्र॒ति॒ग॒रम् प्रति॑ । प्र॒ति॒ग॒रमिति॑ प्रति - ग॒रम् । प्रती᳚क्षते । ई॒क्ष॒ते॒ यत् । यद॑भिप्रतिगृणी॒यात् । अ॒भि॒प्र॒ति॒गृ॒णी॒याद् यथा᳚ । अ॒भि॒प्र॒ति॒गृ॒णी॒यादित्य॑भि - प्र॒ति॒गृ॒णी॒यात् । यथा ऽऽय॑तया । आय॑तया समृ॒च्छते᳚ । आय॑त॒येत्या - य॒त॒या॒ । स॒मृ॒च्छते॑ ता॒दृक् । स॒मृ॒च्छत॒ इति॑ सम् - ऋ॒च्छते᳚ । ता॒दृगे॒व । ए॒व तत् । तद् यत् । यद॑र्द्ध॒र्चात् । अ॒र्द्ध॒र्चा ल्लुप्ये॑त । अ॒र्द्ध॒र्चादित्य॑र्द्ध - ऋ॒चात् । लुप्ये॑त॒ यथा᳚ । यथा॒ धाव॑द्भ्यः । धाव॑द्भ्यो॒ हीय॑ते । धाव॑द्भ्य॒ इति॒ धाव॑त् - भ्यः॒ । हीय॑ते ता॒दृक् । ता॒दृगे॒व । ए॒व तत् । तत् प्र॒बाहु॑क् । प्र॒बाहु॒ग् वै । प्र॒बाहु॒गिति॑ प्र - बाहु॑क् । वा ऋ॒त्विजा᳚म् । ऋ॒त्विजा॑मुद्गी॒थाः । उ॒द्गी॒था उ॑द्गी॒थः । उ॒द्गी॒था इत्यु॑त् - गी॒थाः । उ॒द्गी॒थ ए॒व । उ॒द्गी॒थ इत्यु॑त् - गी॒थः । ए॒वोद्गा॑तृ॒णाम् । उ॒द्गा॒तृ॒णामृ॒चः । उ॒द्गा॒तृ॒णामित्यु॑त् - गा॒तृ॒णाम् \newline

\textbf{Jatai Paata} \newline

1. अ॒प॒हनी॑त पु॒रा पु॒रा ऽप॒हनी॑ता प॒हनी॑त पु॒रा । \newline
2. अ॒प॒हनी॒तेत्यप॑ - हनी॑त । \newline
3. पु॒रा ऽस्या᳚स्य पु॒रा पु॒रा ऽस्य॑ । \newline
4. अ॒स्य॒ सं॒ॅव॒थ्स॒राथ् सं॑ॅवथ्स॒रा द॑स्यास्य संॅवथ्स॒रात् । \newline
5. सं॒ॅव॒थ्स॒राद् गृ॒हे गृ॒हे सं॑ॅवथ्स॒राथ् सं॑ॅवथ्स॒राद् गृ॒हे । \newline
6. सं॒ॅव॒थ्स॒रादिति॑ सं - व॒थ्स॒रात् । \newline
7. गृ॒ह आ गृ॒हे गृ॒ह आ । \newline
8. आ वे॑वीरन्. वेवीर॒न् ना वे॑वीरन्न् । \newline
9. वे॒वी॒र॒ञ् छोꣳसा॒ शोꣳसा॑ वेवीरन्. वेवीर॒ञ् छोꣳसा᳚ । \newline
10. शोꣳसा॒ मोदो॒ मोदः॒ शोꣳसा॒ शोꣳसा॒ मोदः॑ । \newline
11. मोद॑ इवे व॒ मोदो॒ मोद॑ इव । \newline
12. इ॒वे तीती॑वे॒ वे ति॑ । \newline
13. इति॑ प्र॒त्याह्व॑यते प्र॒त्याह्व॑यत॒ इतीति॑ प्र॒त्याह्व॑यते । \newline
14. प्र॒त्याह्व॑यते॒ तेन॒ तेन॑ प्र॒त्याह्व॑यते प्र॒त्याह्व॑यते॒ तेन॑ । \newline
15. प्र॒त्याह्व॑यत॒ इति॑ प्रति - आह्व॑यते । \newline
16. तेनै॒वैव तेन॒ तेनै॒व । \newline
17. ए॒व तत् तदे॒वैव तत् । \newline
18. तदपाप॒ तत् तदप॑ । \newline
19. अप॑ हते ह॒ते ऽपाप॑ हते । \newline
20. ह॒ते॒ यथा॒ यथा॑ हते हते॒ यथा᳚ । \newline
21. यथा॒ वै वै यथा॒ यथा॒ वै । \newline
22. वा आय॑ता॒ माय॑तां॒ ॅवै वा आय॑ताम् । \newline
23. आय॑ताम् प्र॒तीक्ष॑ते प्र॒तीक्ष॑त॒ आय॑ता॒ माय॑ताम् प्र॒तीक्ष॑ते । \newline
24. आय॑ता॒मित्या - य॒ता॒म् । \newline
25. प्र॒तीक्ष॑त ए॒व मे॒वम् प्र॒तीक्ष॑ते प्र॒तीक्ष॑त ए॒वम् । \newline
26. प्र॒तीक्ष॑त॒ इति॑ प्रति - ईक्ष॑ते । \newline
27. ए॒व म॑द्ध्व॒र्यु र॑द्ध्व॒र्यु रे॒व मे॒व म॑द्ध्व॒र्युः । \newline
28. अ॒द्ध्व॒र्युः प्र॑तिग॒रम् प्र॑तिग॒र म॑द्ध्व॒र्यु र॑द्ध्व॒र्युः प्र॑तिग॒रम् । \newline
29. प्र॒ति॒ग॒रम् प्रति॒ प्रति॑ प्रतिग॒रम् प्र॑तिग॒रम् प्रति॑ । \newline
30. प्र॒ति॒ग॒रमिति॑ प्रति - ग॒रम् । \newline
31. प्रती᳚क्षत ईक्षते॒ प्रति॒ प्रती᳚क्षते । \newline
32. ई॒क्ष॒ते॒ यद् यदी᳚क्षत ईक्षते॒ यत् । \newline
33. यद॑भिप्रतिगृणी॒या द॑भिप्रतिगृणी॒याद् यद् यद॑भिप्रतिगृणी॒यात् । \newline
34. अ॒भि॒प्र॒ति॒गृ॒णी॒याद् यथा॒ यथा॑ ऽभिप्रतिगृणी॒या द॑भिप्रतिगृणी॒याद् यथा᳚ । \newline
35. अ॒भि॒प्र॒ति॒गृ॒णी॒यादित्य॑भि - प्र॒ति॒गृ॒णी॒यात् । \newline
36. यथा ऽऽय॑त॒या ऽऽय॑तया॒ यथा॒ यथा ऽऽय॑तया । \newline
37. आय॑तया समृ॒च्छते॑ समृ॒च्छत॒ आय॑त॒या ऽऽय॑तया समृ॒च्छते᳚ । \newline
38. आय॑त॒येत्या - य॒त॒या॒ । \newline
39. स॒मृ॒च्छते॑ ता॒दृक् ता॒दृख् स॑मृ॒च्छते॑ समृ॒च्छते॑ ता॒दृक् । \newline
40. स॒मृ॒च्छत॒ इति॑ सं - ऋ॒च्छते᳚ । \newline
41. ता॒दृ गे॒वैव ता॒दृक् ता॒दृ गे॒व । \newline
42. ए॒व तत् तदे॒वैव तत् । \newline
43. तद् यद् यत् तत् तद् यत् । \newline
44. यद॑र्द्ध॒र्चा द॑र्द्ध॒र्चाद् यद् यद॑र्द्ध॒र्चात् । \newline
45. अ॒र्द्ध॒र्चा ल्लुप्ये॑त॒ लुप्ये॑ता र्द्ध॒र्चा द॑र्द्ध॒र्चा ल्लुप्ये॑त । \newline
46. अ॒द्‌र्ध॒र्चादित्य॑द्‌र्ध - ऋ॒चात् । \newline
47. लुप्ये॑त॒ यथा॒ यथा॒ लुप्ये॑त॒ लुप्ये॑त॒ यथा᳚ । \newline
48. यथा॒ धाव॑द्भ्यो॒ धाव॑द्भ्यो॒ यथा॒ यथा॒ धाव॑द्भ्यः । \newline
49. धाव॑द्भ्यो॒ हीय॑ते॒ हीय॑ते॒ धाव॑द्भ्यो॒ धाव॑द्भ्यो॒ हीय॑ते । \newline
50. धाव॑द्भ्य॒ इति॒ धाव॑त् - भ्यः॒ । \newline
51. हीय॑ते ता॒दृक् ता॒दृग् घीय॑ते॒ हीय॑ते ता॒दृक् । \newline
52. ता॒दृगे॒वैव ता॒दृक् ता॒दृगे॒व । \newline
53. ए॒व तत् तदे॒वैव तत् । \newline
54. तत् प्र॒बाहु॑क् प्र॒बाहु॒क् तत् तत् प्र॒बाहु॑क् । \newline
55. प्र॒बाहु॒ग् वै वै प्र॒बाहु॑क् प्र॒बाहु॒ग् वै । \newline
56. प्र॒बाहु॒गिति॑ प्र - बाहु॑क् । \newline
57. वा ऋ॒त्विजा॑ मृ॒त्विजां॒ ॅवै वा ऋ॒त्विजा᳚म् । \newline
58. ऋ॒त्विजा॑ मुद्‍गी॒था उ॑द्‍गी॒था ऋ॒त्विजा॑ मृ॒त्विजा॑ मुद्‍गी॒थाः । \newline
59. उ॒द्‍गी॒था उ॑द्‍गी॒थ उ॑द्‍गी॒थ उ॑द्‍गी॒था उ॑द्‍गी॒था उ॑द्‍गी॒थः । \newline
60. उ॒द्‍गी॒था इत्यु॑त् - गी॒थाः । \newline
61. उ॒द्‍गी॒थ ए॒वै वोद्‍गी॒थ उ॑द्‍गी॒थ ए॒व । \newline
62. उ॒द्‍गी॒थ इत्यु॑त् - गी॒थः । \newline
63. ए॒वोद्‍गा॑तृ॒णा मु॑द्‍गातृ॒णा मे॒वै वोद्‍गा॑तृ॒णाम् । \newline
64. उ॒द्‍गा॒तृ॒णा मृ॒च ऋ॒च उ॑द्‍गातृ॒णा मु॑द्‍गातृ॒णा मृ॒चः । \newline
65. उ॒द्‍गा॒तृ॒णामित्यु॑त् - गा॒तृ॒णाम् । \newline

\textbf{Ghana Paata } \newline

1. अ॒प॒हनी॑त पु॒रा पु॒रा ऽप॒हनी॑ता प॒हनी॑त पु॒रा ऽस्या᳚स्य पु॒रा ऽप॒हनी॑ता प॒हनी॑त पु॒रा ऽस्य॑ । \newline
2. अ॒प॒हनी॒तेत्यप॑ - हनी॑त । \newline
3. पु॒रा ऽस्या᳚स्य पु॒रा पु॒रा ऽस्य॑ संॅवथ्स॒राथ् सं॑ॅवथ्स॒रा द॑स्य पु॒रा पु॒रा ऽस्य॑ संॅवथ्स॒रात् । \newline
4. अ॒स्य॒ सं॒ॅव॒थ्स॒राथ् सं॑ॅवथ्स॒रा द॑स्यास्य संॅवथ्स॒राद् गृ॒हे गृ॒हे सं॑ॅवथ्स॒रा द॑स्यास्य संॅवथ्स॒राद् गृ॒हे । \newline
5. सं॒ॅव॒थ्स॒राद् गृ॒हे गृ॒हे सं॑ॅवथ्स॒राथ् सं॑ॅवथ्स॒राद् गृ॒ह आ गृ॒हे सं॑ॅवथ्स॒राथ् सं॑ॅवथ्स॒राद् गृ॒ह आ । \newline
6. सं॒ॅव॒थ्स॒रादिति॑ सं - व॒थ्स॒रात् । \newline
7. गृ॒ह आ गृ॒हे गृ॒ह आ वे॑वीरन्. वेवीर॒न् ना गृ॒हे गृ॒ह आ वे॑वीरन्न् । \newline
8. आ वे॑वीरन्. वेवीर॒न् ना वे॑वीर॒ञ् छोꣳसा॒ शोꣳसा॑ वेवीर॒न् ना वे॑वीर॒ञ् छोꣳसा᳚ । \newline
9. वे॒वी॒र॒ञ् छोꣳसा॒ शोꣳसा॑ वेवीरन्. वेवीर॒ञ् छोꣳसा॒ मोदो॒ मोदः॒ शोꣳसा॑ वेवीरन्. वेवीर॒ञ् छोꣳसा॒ मोदः॑ । \newline
10. शोꣳसा॒ मोदो॒ मोदः॒ शोꣳसा॒ शोꣳसा॒ मोद॑ इवे व॒ मोदः॒ शोꣳसा॒ शोꣳसा॒ मोद॑ इव । \newline
11. मोद॑ इवे व॒ मोदो॒ मोद॑ इ॒वे तीती॑व॒ मोदो॒ मोद॑ इ॒वे ति॑ । \newline
12. इ॒वे तीती॑वे॒ वे ति॑ प्र॒त्याह्व॑यते प्र॒त्याह्व॑यत॒ इती॑वे॒ वे ति॑ प्र॒त्याह्व॑यते । \newline
13. इति॑ प्र॒त्याह्व॑यते प्र॒त्याह्व॑यत॒ इतीति॑ प्र॒त्याह्व॑यते॒ तेन॒ तेन॑ प्र॒त्याह्व॑यत॒ इतीति॑ प्र॒त्याह्व॑यते॒ तेन॑ । \newline
14. प्र॒त्याह्व॑यते॒ तेन॒ तेन॑ प्र॒त्याह्व॑यते प्र॒त्याह्व॑यते॒ तेनै॒वैव तेन॑ प्र॒त्याह्व॑यते प्र॒त्याह्व॑यते॒ तेनै॒व । \newline
15. प्र॒त्याह्व॑यत॒ इति॑ प्रति - आह्व॑यते । \newline
16. तेनै॒वैव तेन॒ तेनै॒व तत् तदे॒व तेन॒ तेनै॒व तत् । \newline
17. ए॒व तत् तदे॒वैव तदपाप॒ तदे॒वैव तदप॑ । \newline
18. तदपाप॒ तत् तदप॑ हते ह॒ते ऽप॒ तत् तदप॑ हते । \newline
19. अप॑ हते ह॒ते ऽपाप॑ हते॒ यथा॒ यथा॑ ह॒ते ऽपाप॑ हते॒ यथा᳚ । \newline
20. ह॒ते॒ यथा॒ यथा॑ हते हते॒ यथा॒ वै वै यथा॑ हते हते॒ यथा॒ वै । \newline
21. यथा॒ वै वै यथा॒ यथा॒ वा आय॑ता॒ माय॑तां॒ ॅवै यथा॒ यथा॒ वा आय॑ताम् । \newline
22. वा आय॑ता॒ माय॑तां॒ ॅवै वा आय॑ताम् प्र॒तीक्ष॑ते प्र॒तीक्ष॑त॒ आय॑तां॒ ॅवै वा आय॑ताम् प्र॒तीक्ष॑ते । \newline
23. आय॑ताम् प्र॒तीक्ष॑ते प्र॒तीक्ष॑त॒ आय॑ता॒ माय॑ताम् प्र॒तीक्ष॑त ए॒व मे॒वम् प्र॒तीक्ष॑त॒ आय॑ता॒ माय॑ताम् प्र॒तीक्ष॑त ए॒वम् । \newline
24. आय॑ता॒मित्या - य॒ता॒म् । \newline
25. प्र॒तीक्ष॑त ए॒व मे॒वम् प्र॒तीक्ष॑ते प्र॒तीक्ष॑त ए॒व म॑द्ध्व॒र्यु र॑द्ध्व॒र्यु रे॒वम् प्र॒तीक्ष॑ते प्र॒तीक्ष॑त ए॒व म॑द्ध्व॒र्युः । \newline
26. प्र॒तीक्ष॑त॒ इति॑ प्रति - ईक्ष॑ते । \newline
27. ए॒व म॑द्ध्व॒र्यु र॑द्ध्व॒र्यु रे॒व मे॒व म॑द्ध्व॒र्युः प्र॑तिग॒रम् प्र॑तिग॒र म॑द्ध्व॒र्यु रे॒व मे॒व म॑द्ध्व॒र्युः प्र॑तिग॒रम् । \newline
28. अ॒द्ध्व॒र्युः प्र॑तिग॒रम् प्र॑तिग॒र म॑द्ध्व॒र्यु र॑द्ध्व॒र्युः प्र॑तिग॒रम् प्रति॒ प्रति॑ प्रतिग॒र म॑द्ध्व॒र्यु र॑द्ध्व॒र्युः प्र॑तिग॒रम् प्रति॑ । \newline
29. प्र॒ति॒ग॒रम् प्रति॒ प्रति॑ प्रतिग॒रम् प्र॑तिग॒रम् प्रती᳚क्षत ईक्षते॒ प्रति॑ प्रतिग॒रम् प्र॑तिग॒रम् प्रती᳚क्षते । \newline
30. प्र॒ति॒ग॒रमिति॑ प्रति - ग॒रम् । \newline
31. प्रती᳚क्षत ईक्षते॒ प्रति॒ प्रती᳚क्षते॒ यद् यदी᳚क्षते॒ प्रति॒ प्रती᳚क्षते॒ यत् । \newline
32. ई॒क्ष॒ते॒ यद् यदी᳚क्षत ईक्षते॒ यद॑भिप्रतिगृणी॒या द॑भिप्रतिगृणी॒याद् यदी᳚क्षत ईक्षते॒ यद॑भिप्रतिगृणी॒यात् । \newline
33. यद॑भिप्रतिगृणी॒या द॑भिप्रतिगृणी॒याद् यद् यद॑भिप्रतिगृणी॒याद् यथा॒ यथा॑ ऽभिप्रतिगृणी॒याद् यद् यद॑भिप्रतिगृणी॒याद् यथा᳚ । \newline
34. अ॒भि॒प्र॒ति॒गृ॒णी॒याद् यथा॒ यथा॑ ऽभिप्रतिगृणी॒या द॑भिप्रतिगृणी॒याद् यथा ऽऽय॑त॒या ऽऽय॑तया॒ यथा॑ ऽभिप्रतिगृणी॒या द॑भिप्रतिगृणी॒याद् यथा ऽऽय॑तया । \newline
35. अ॒भि॒प्र॒ति॒गृ॒णी॒यादित्य॑भि - प्र॒ति॒गृ॒णी॒यात् । \newline
36. यथा ऽऽय॑त॒या ऽऽय॑तया॒ यथा॒ यथा ऽऽय॑तया समृ॒च्छते॑ समृ॒च्छत॒ आय॑तया॒ यथा॒ यथा ऽऽय॑तया समृ॒च्छते᳚ । \newline
37. आय॑तया समृ॒च्छते॑ समृ॒च्छत॒ आय॑त॒या ऽऽय॑तया समृ॒च्छते॑ ता॒दृक् ता॒दृख् स॑मृ॒च्छत॒ आय॑त॒या ऽऽय॑तया समृ॒च्छते॑ ता॒दृक् । \newline
38. आय॑त॒येत्या - य॒त॒या॒ । \newline
39. स॒मृ॒च्छते॑ ता॒दृक् ता॒दृख् स॑मृ॒च्छते॑ समृ॒च्छते॑ ता॒दृगे॒वैव ता॒दृख् स॑मृ॒च्छते॑ समृ॒च्छते॑ ता॒दृगे॒व । \newline
40. स॒मृ॒च्छत॒ इति॑ सं - ऋ॒च्छते᳚ । \newline
41. ता॒दृगे॒वैव ता॒दृक् ता॒दृगे॒व तत् तदे॒व ता॒दृक् ता॒दृगे॒व तत् । \newline
42. ए॒व तत् तदे॒वैव तद् यद् यत् तदे॒वैव तद् यत् । \newline
43. तद् यद् यत् तत् तद् यद॑र्द्ध॒र्चा द॑र्द्ध॒र्चाद् यत् तत् तद् यद॑र्द्ध॒र्चात् । \newline
44. यद॑र्द्ध॒र्चा द॑र्द्ध॒र्चाद् यद् यद॑र्द्ध॒र्चा ल्लुप्ये॑त॒ लुप्ये॑ता र्द्ध॒र्चाद् यद् यद॑र्द्ध॒र्चा ल्लुप्ये॑त । \newline
45. अ॒र्द्ध॒र्चा ल्लुप्ये॑त॒ लुप्ये॑ता र्द्ध॒र्चा द॑र्द्ध॒र्चा ल्लुप्ये॑त॒ यथा॒ यथा॒ लुप्ये॑ता र्द्ध॒र्चा द॑र्द्ध॒र्चा ल्लुप्ये॑त॒ यथा᳚ । \newline
46. अ॒द्‌र्ध॒र्चादित्य॑द्‌र्ध - ऋ॒चात् । \newline
47. लुप्ये॑त॒ यथा॒ यथा॒ लुप्ये॑त॒ लुप्ये॑त॒ यथा॒ धाव॑द्भ्यो॒ धाव॑द्भ्यो॒ यथा॒ लुप्ये॑त॒ लुप्ये॑त॒ यथा॒ धाव॑द्भ्यः । \newline
48. यथा॒ धाव॑द्भ्यो॒ धाव॑द्भ्यो॒ यथा॒ यथा॒ धाव॑द्भ्यो॒ हीय॑ते॒ हीय॑ते॒ धाव॑द्भ्यो॒ यथा॒ यथा॒ धाव॑द्भ्यो॒ हीय॑ते । \newline
49. धाव॑द्भ्यो॒ हीय॑ते॒ हीय॑ते॒ धाव॑द्भ्यो॒ धाव॑द्भ्यो॒ हीय॑ते ता॒दृक् ता॒दृग्घीय॑ते॒ धाव॑द्भ्यो॒ धाव॑द्भ्यो॒ हीय॑ते ता॒दृक् । \newline
50. धाव॑द्भ्य॒ इति॒ धाव॑त् - भ्यः॒ । \newline
51. हीय॑ते ता॒दृक् ता॒दृग् घीय॑ते॒ हीय॑ते ता॒दृगे॒वैव ता॒दृग् घीय॑ते॒ हीय॑ते ता॒दृगे॒व । \newline
52. ता॒दृगे॒वैव ता॒दृक् ता॒दृगे॒व तत् तदे॒व ता॒दृक् ता॒दृगे॒व तत् । \newline
53. ए॒व तत् तदे॒वैव तत् प्र॒बाहु॑क् प्र॒बाहु॒क् तदे॒वैव तत् प्र॒बाहु॑क् । \newline
54. तत् प्र॒बाहु॑क् प्र॒बाहु॒क् तत् तत् प्र॒बाहु॒ग् वै वै प्र॒बाहु॒क् तत् तत् प्र॒बाहु॒ग् वै । \newline
55. प्र॒बाहु॒ग् वै वै प्र॒बाहु॑क् प्र॒बाहु॒ग् वा ऋ॒त्विजा॑ मृ॒त्विजां॒ ॅवै प्र॒बाहु॑क् प्र॒बाहु॒ग् वा ऋ॒त्विजा᳚म् । \newline
56. प्र॒बाहु॒गिति॑ प्र - बाहु॑क् । \newline
57. वा ऋ॒त्विजा॑ मृ॒त्विजां॒ ॅवै वा ऋ॒त्विजा॑ मुद्‍गी॒था उ॑द्‍गी॒था ऋ॒त्विजां॒ ॅवै वा ऋ॒त्विजा॑ मुद्‍गी॒थाः । \newline
58. ऋ॒त्विजा॑ मुद्‍गी॒था उ॑द्‍गी॒था ऋ॒त्विजा॑ मृ॒त्विजा॑ मुद्‍गी॒था उ॑द्‍गी॒थ उ॑द्‍गी॒थ उ॑द्‍गी॒था ऋ॒त्विजा॑ मृ॒त्विजा॑ मुद्‍गी॒था उ॑द्‍गी॒थः । \newline
59. उ॒द्‍गी॒था उ॑द्‍गी॒थ उ॑द्‍गी॒थ उ॑द्‍गी॒था उ॑द्‍गी॒था उ॑द्‍गी॒थ ए॒वैवोद्‍गी॒थ उ॑द्‍गी॒था उ॑द्‍गी॒था उ॑द्‍गी॒थ ए॒व । \newline
60. उ॒द्‍गी॒था इत्यु॑त् - गी॒थाः । \newline
61. उ॒द्‍गी॒थ ए॒वैवोद्‍गी॒थ उ॑द्‍गी॒थ ए॒वोद्‍गा॑तृ॒णा मु॑द्‍गातृ॒णा मे॒वोद्‍गी॒थ उ॑द्‍गी॒थ ए॒वोद्‍गा॑तृ॒णाम् । \newline
62. उ॒द्‍गी॒थ इत्यु॑त् - गी॒थः । \newline
63. ए॒वोद्‍गा॑तृ॒णा मु॑द्‍गातृ॒णा मे॒वैवोद्‍गा॑तृ॒णा मृ॒च ऋ॒च उ॑द्‍गातृ॒णा मे॒वैवोद्‍गा॑तृ॒णा मृ॒चः । \newline
64. उ॒द्‍गा॒तृ॒णा मृ॒च ऋ॒च उ॑द्‍गातृ॒णा मु॑द्‍गातृ॒णा मृ॒चः प्र॑ण॒वः प्र॑ण॒व ऋ॒च उ॑द्‍गातृ॒णा मु॑द्‍गातृ॒णा मृ॒चः प्र॑ण॒वः । \newline
65. उ॒द्‍गा॒तृ॒णामित्यु॑त् - गा॒तृ॒णाम् । \newline
\pagebreak
\markright{ TS 3.2.9.6  \hfill https://www.vedavms.in \hfill}

\section{ TS 3.2.9.6 }

\textbf{TS 3.2.9.6 } \newline
\textbf{Samhita Paata} \newline

मृ॒चः प्र॑ण॒व उ॑क्थशꣳ॒॒सिनां᳚ प्रतिग॒रो᳚ऽद्ध्वर्यू॒णां ॅय ए॒वं ॅवि॒द्वान् प्र॑तिगृ॒णात्य॑न्ना॒द ए॒व भ॑व॒त्याऽस्य॑ प्र॒जायां᳚ ॅवा॒जी जा॑यत इ॒यं ॅवै होता॒ऽसाव॑द्ध्व॒र्युर्यदासी॑नः॒ शꣳ स॑त्य॒स्या ए॒व तद्धोता॒ नैत्यास्त॑ इव॒ हीयमथो॑ इ॒मामे॒व तेन॒ यज॑मानो दुहे॒ यत् तिष्ठ॑न् प्रतिगृ॒णात्य॒मुष्या॑ ए॒व तद॑द्ध्व॒र्युर्नैति॒ - [  ] \newline

\textbf{Pada Paata} \newline

ऋ॒चः । प्र॒ण॒व इति॑ प्र - न॒वः । उ॒क्थ॒शꣳ॒॒सिना॒मित्यु॑क्थ - शꣳ॒॒सिना᳚म् । प्र॒ति॒ग॒र इति॑ प्रति - ग॒रः । अ॒द्ध्व॒र्यू॒णाम् । यः । ए॒वम् । वि॒द्वान् । प्र॒ति॒गृ॒णातीति॑ प्रति-गृ॒णाति॑ । अ॒न्ना॒द इत्य॑न्न - अ॒दः । ए॒व । भ॒व॒ति॒ । एति॑ । अ॒स्य॒ । प्र॒जाया॒मिति॑ प्र - जाया᳚म् । वा॒जी । जा॒य॒ते॒ । इ॒यम् । वै । होता᳚ । अ॒सौ । अ॒द्ध्व॒र्युः । यत् । आसी॑नः । शꣳस॑ति । अ॒स्याः । ए॒व । तत् । होता᳚ । न । ए॒ति॒ । आस्ते᳚ । इ॒व॒ । हि । इ॒यम् । अथो॒ इति॑ । इ॒माम् । ए॒व । तेन॑ । यज॑मानः । दु॒हे॒ । यत् । तिष्ठन्न्॑ । प्र॒ति॒गृ॒णातीति॑ प्रति - गृ॒णाति॑ । अ॒मुष्याः᳚ । ए॒व । तत् । अ॒द्ध्व॒र्युः । न । ए॒ति॒ ।  \newline


\textbf{Krama Paata} \newline

ऋ॒चः प्र॑ण॒वः । प्र॒ण॒व उ॑क्थशꣳ॒॒सिना᳚म् । प्र॒ण॒व इति॑ प्र - न॒वः । उ॒क्थ॒शꣳ॒॒सिना᳚म् प्रतिग॒रः । उ॒क्थ॒शꣳ॒॒सिना॒मित्यु॑क्थ - शꣳ॒॒सिना᳚म् । प्र॒ति॒ग॒रो᳚ ऽद्ध्वर्यू॒णाम् । प्र॒ति॒ग॒र इति॑ प्रति - ग॒रः । अ॒द्ध्व॒र्यू॒णां ॅयः । य ए॒वम् । ए॒वं ॅवि॒द्वान् । वि॒द्वान् प्र॑तिगृ॒णाति॑ । प्र॒ति॒गृ॒णात्य॑न्ना॒दः । प्र॒ति॒गृ॒णातीति॑ प्रति - गृ॒णाति॑ । अ॒न्ना॒द ए॒व । अ॒न्ना॒द इत्य॑न्न - अ॒दः । ए॒व भ॑वति । भ॒व॒त्या । आ ऽस्य॑ । अ॒स्य॒ प्र॒जाया᳚म् । प्र॒जायां᳚ ॅवा॒जी । प्र॒जाया॒मिति॑ प्र - जाया᳚म् । वा॒जी जा॑यते । जा॒य॒त॒ इ॒यम् । इ॒यं ॅवै । वै होता᳚ । होता॒ ऽसौ । अ॒साव॑द्ध्व॒र्युः । अ॒द्ध्व॒र्युर् यत् । यदासी॑नः । आसी॑नः॒ शꣳस॑ति । शꣳस॑त्य॒स्याः । अ॒स्या ए॒व । ए॒व तत् । तद्धोता᳚ । होता॒ न । नैति॑ । ए॒त्यास्ते᳚ । आस्त॑ इव । इ॒व॒ हि । हीयम् । इ॒यमथो᳚ । अथो॑ इ॒माम् । अथो॒ इत्यथो᳚ । इ॒मामे॒व । ए॒व तेन॑ । तेन॒ यज॑मानः । यज॑मानो दुहे । दु॒हे॒ यत् । यत् तिष्ठन्न्॑ । तिष्ठ॑न् प्रतिगृ॒णाति॑ । प्र॒ति॒गृ॒णात्य॒मुष्याः᳚ । प्र॒ति॒गृ॒णातीति॑ प्रति - गृ॒णाति॑ । अ॒मुष्या॑ ए॒व । ए॒व तत् । तद॑द्ध्व॒र्युः । अ॒द्ध्व॒र्युर् न । नैति॑ । ए॒ति॒ तिष्ठ॑ति \newline

\textbf{Jatai Paata} \newline

1. ऋ॒चः प्र॑ण॒वः प्र॑ण॒व ऋ॒च ऋ॒चः प्र॑ण॒वः । \newline
2. प्र॒ण॒व उ॑क्थशꣳ॒॒सिना॑ मुक्थशꣳ॒॒सिना᳚म् प्रण॒वः प्र॑ण॒व उ॑क्थशꣳ॒॒सिना᳚म् । \newline
3. प्र॒ण॒व इति॑ प्र - न॒वः । \newline
4. उ॒क्थ॒शꣳ॒॒सिना᳚म् प्रतिग॒रः प्र॑तिग॒र उ॑क्थशꣳ॒॒सिना॑ मुक्थशꣳ॒॒सिना᳚म् प्रतिग॒रः । \newline
5. उ॒क्थ॒शꣳ॒॒सिना॒मित्यु॑क्थ - शꣳ॒॒सिना᳚म् । \newline
6. प्र॒ति॒ग॒रो᳚ ऽद्ध्वर्यू॒णा म॑द्ध्वर्यू॒णाम् प्र॑तिग॒रः प्र॑तिग॒रो᳚ ऽद्ध्वर्यू॒णाम् । \newline
7. प्र॒ति॒ग॒र इति॑ प्रति - ग॒रः । \newline
8. अ॒द्ध्व॒र्यू॒णां ॅयो यो᳚ ऽद्ध्वर्यू॒णा म॑द्ध्वर्यू॒णां ॅयः । \newline
9. य ए॒व मे॒वं ॅयो य ए॒वम् । \newline
10. ए॒वं ॅवि॒द्वान्. वि॒द्वा ने॒व मे॒वं ॅवि॒द्वान् । \newline
11. वि॒द्वान् प्र॑तिगृ॒णाति॑ प्रतिगृ॒णाति॑ वि॒द्वान्. वि॒द्वान् प्र॑तिगृ॒णाति॑ । \newline
12. प्र॒ति॒गृ॒णा त्य॑न्ना॒दो᳚ ऽन्ना॒दः प्र॑तिगृ॒णाति॑ प्रतिगृ॒णा त्य॑न्ना॒दः । \newline
13. प्र॒ति॒गृ॒णातीति॑ प्रति - गृ॒णाति॑ । \newline
14. अ॒न्ना॒द ए॒वैवा न्ना॒दो᳚ ऽन्ना॒द ए॒व । \newline
15. अ॒न्ना॒द इत्य॑न्न - अ॒दः । \newline
16. ए॒व भ॑वति भव त्ये॒वैव भ॑वति । \newline
17. भ॒व॒त्या भ॑वति भव॒त्या । \newline
18. आ ऽस्या॒स्या ऽस्य॑ । \newline
19. अ॒स्य॒ प्र॒जाया᳚म् प्र॒जाया॑ मस्यास्य प्र॒जाया᳚म् । \newline
20. प्र॒जायां᳚ ॅवा॒जी वा॒जी प्र॒जाया᳚म् प्र॒जायां᳚ ॅवा॒जी । \newline
21. प्र॒जाया॒मिति॑ प्र - जाया᳚म् । \newline
22. वा॒जी जा॑यते जायते वा॒जी वा॒जी जा॑यते । \newline
23. जा॒य॒त॒ इ॒य मि॒यम् जा॑यते जायत इ॒यम् । \newline
24. इ॒यं ॅवै वा इ॒य मि॒यं ॅवै । \newline
25. वै होता॒ होता॒ वै वै होता᳚ । \newline
26. होता॒ ऽसा व॒सौ होता॒ होता॒ ऽसौ । \newline
27. अ॒सा व॑द्ध्व॒र्यु र॑द्ध्व॒र्यु र॒सा व॒सा व॑द्ध्व॒र्युः । \newline
28. अ॒द्ध्व॒र्युर् यद् यद॑द्ध्व॒र्यु र॑द्ध्व॒र्युर् यत् । \newline
29. यदासी॑न॒ आसी॑नो॒ यद् यदासी॑नः । \newline
30. आसी॑नः॒ शꣳस॑ति॒ शꣳस॒ त्यासी॑न॒ आसी॑नः॒ शꣳस॑ति । \newline
31. शꣳस॑ त्य॒स्या अ॒स्याः शꣳस॑ति॒ शꣳस॑त्य॒स्याः । \newline
32. अ॒स्या ए॒वैवास्या अ॒स्या ए॒व । \newline
33. ए॒व तत् तदे॒वैव तत् । \newline
34. तद्धोता॒ होता॒ तत् तद्धोता᳚ । \newline
35. होता॒ न न होता॒ होता॒ न । \newline
36. नैत्ये॑ति॒ न नैति॑ । \newline
37. ए॒त्या स्त॒ आस्त॑ एत्ये॒ त्यास्ते᳚ । \newline
38. आस्त॑ इवे॒ वास्त॒ आस्त॑ इव । \newline
39. इ॒व॒ हि हीवे॑ व॒ हि । \newline
40. हीय मि॒यꣳ हि हीयम् । \newline
41. इ॒य मथो॒ अथो॑ इ॒य मि॒य मथो᳚ । \newline
42. अथो॑ इ॒मा मि॒मा मथो॒ अथो॑ इ॒माम् । \newline
43. अथो॒ इत्यथो᳚ । \newline
44. इ॒मा मे॒वैवे मा मि॒मा मे॒व । \newline
45. ए॒व तेन॒ तेनै॒वैव तेन॑ । \newline
46. तेन॒ यज॑मानो॒ यज॑मान॒ स्तेन॒ तेन॒ यज॑मानः । \newline
47. यज॑मानो दुहे दुहे॒ यज॑मानो॒ यज॑मानो दुहे । \newline
48. दु॒हे॒ यद् यद् दु॑हे दुहे॒ यत् । \newline
49. यत् तिष्ठꣳ॒॒ स्तिष्ठ॒न्॒. यद् यत् तिष्ठन्न्॑ । \newline
50. तिष्ठ॑न् प्रतिगृ॒णाति॑ प्रतिगृ॒णाति॒ तिष्ठꣳ॒॒ स्तिष्ठ॑न् प्रतिगृ॒णाति॑ । \newline
51. प्र॒ति॒गृ॒णा त्य॒मुष्या॑ अ॒मुष्याः᳚ प्रतिगृ॒णाति॑ प्रतिगृ॒णा त्य॒मुष्याः᳚ । \newline
52. प्र॒ति॒गृ॒णातीति॑ प्रति - गृ॒णाति॑ । \newline
53. अ॒मुष्या॑ ए॒वैवा मुष्या॑ अ॒मुष्या॑ ए॒व । \newline
54. ए॒व तत् तदे॒वैव तत् । \newline
55. तद॑द्ध्व॒र्यु र॑द्ध्व॒र्यु स्तत् तद॑द्ध्व॒र्युः । \newline
56. अ॒द्ध्व॒र्युर् न नाद्ध्व॒र्यु र॑द्ध्व॒र्युर् न । \newline
57. नैत्ये॑ति॒ न नैति॑ । \newline
58. ए॒ति॒ तिष्ठ॑ति॒ तिष्ठ॑ त्येत्येति॒ तिष्ठ॑ति । \newline

\textbf{Ghana Paata } \newline

1. ऋ॒चः प्र॑ण॒वः प्र॑ण॒व ऋ॒च ऋ॒चः प्र॑ण॒व उ॑क्थशꣳ॒॒सिना॑ मुक्थशꣳ॒॒सिना᳚म् प्रण॒व ऋ॒च ऋ॒चः प्र॑ण॒व उ॑क्थशꣳ॒॒सिना᳚म् । \newline
2. प्र॒ण॒व उ॑क्थशꣳ॒॒सिना॑ मुक्थशꣳ॒॒सिना᳚म् प्रण॒वः प्र॑ण॒व उ॑क्थशꣳ॒॒सिना᳚म् प्रतिग॒रः प्र॑तिग॒र उ॑क्थशꣳ॒॒सिना᳚म् प्रण॒वः प्र॑ण॒व उ॑क्थशꣳ॒॒सिना᳚म् प्रतिग॒रः । \newline
3. प्र॒ण॒व इति॑ प्र - न॒वः । \newline
4. उ॒क्थ॒शꣳ॒॒सिना᳚म् प्रतिग॒रः प्र॑तिग॒र उ॑क्थशꣳ॒॒सिना॑ मुक्थशꣳ॒॒सिना᳚म् प्रतिग॒रो᳚ ऽद्ध्वर्यू॒णा म॑द्ध्वर्यू॒णाम् प्र॑तिग॒र उ॑क्थशꣳ॒॒सिना॑ मुक्थशꣳ॒॒सिना᳚म् प्रतिग॒रो᳚ ऽद्ध्वर्यू॒णाम् । \newline
5. उ॒क्थ॒शꣳ॒॒सिना॒मित्यु॑क्थ - शꣳ॒॒सिना᳚म् । \newline
6. प्र॒ति॒ग॒रो᳚ ऽद्ध्वर्यू॒णा म॑द्ध्वर्यू॒णाम् प्र॑तिग॒रः प्र॑तिग॒रो᳚ ऽद्ध्वर्यू॒णां ॅयो यो᳚ ऽद्ध्वर्यू॒णाम् प्र॑तिग॒रः प्र॑तिग॒रो᳚ ऽद्ध्वर्यू॒णां ॅयः । \newline
7. प्र॒ति॒ग॒र इति॑ प्रति - ग॒रः । \newline
8. अ॒द्ध्व॒र्यू॒णां ॅयो यो᳚ ऽद्ध्वर्यू॒णा म॑द्ध्वर्यू॒णां ॅय ए॒व मे॒वं ॅयो᳚ ऽद्ध्वर्यू॒णा म॑द्ध्वर्यू॒णां ॅय ए॒वम् । \newline
9. य ए॒व मे॒वं ॅयो य ए॒वं ॅवि॒द्वान्. वि॒द्वा ने॒वं ॅयो य ए॒वं ॅवि॒द्वान् । \newline
10. ए॒वं ॅवि॒द्वान्. वि॒द्वा ने॒व मे॒वं ॅवि॒द्वान् प्र॑तिगृ॒णाति॑ प्रतिगृ॒णाति॑ वि॒द्वा ने॒व मे॒वं ॅवि॒द्वान् प्र॑तिगृ॒णाति॑ । \newline
11. वि॒द्वान् प्र॑तिगृ॒णाति॑ प्रतिगृ॒णाति॑ वि॒द्वान्. वि॒द्वान् प्र॑तिगृ॒णा त्य॑न्ना॒दो᳚ ऽन्ना॒दः प्र॑तिगृ॒णाति॑ वि॒द्वान्. वि॒द्वान् प्र॑तिगृ॒णा त्य॑न्ना॒दः । \newline
12. प्र॒ति॒गृ॒णा त्य॑न्ना॒दो᳚ ऽन्ना॒दः प्र॑तिगृ॒णाति॑ प्रतिगृ॒णा त्य॑न्ना॒द ए॒वैवा न्ना॒दः प्र॑तिगृ॒णाति॑ प्रतिगृ॒णा त्य॑न्ना॒द ए॒व । \newline
13. प्र॒ति॒गृ॒णातीति॑ प्रति - गृ॒णाति॑ । \newline
14. अ॒न्ना॒द ए॒वैवा न्ना॒दो᳚ ऽन्ना॒द ए॒व भ॑वति भव त्ये॒वान्ना॒दो᳚ ऽन्ना॒द ए॒व भ॑वति । \newline
15. अ॒न्ना॒द इत्य॑न्न - अ॒दः । \newline
16. ए॒व भ॑वति भव त्ये॒वैव भ॑व॒त्या भ॑व त्ये॒वैव भ॑व॒त्या । \newline
17. भ॒व॒त्या भ॑वति भव॒त्या ऽस्या॒स्या भ॑वति भव॒त्या ऽस्य॑ । \newline
18. आ ऽस्या॒स्या ऽस्य॑ प्र॒जाया᳚म् प्र॒जाया॑ म॒स्या ऽस्य॑ प्र॒जाया᳚म् । \newline
19. अ॒स्य॒ प्र॒जाया᳚म् प्र॒जाया॑ मस्यास्य प्र॒जायां᳚ ॅवा॒जी वा॒जी प्र॒जाया॑ मस्यास्य प्र॒जायां᳚ ॅवा॒जी । \newline
20. प्र॒जायां᳚ ॅवा॒जी वा॒जी प्र॒जाया᳚म् प्र॒जायां᳚ ॅवा॒जी जा॑यते जायते वा॒जी प्र॒जाया᳚म् प्र॒जायां᳚ ॅवा॒जी जा॑यते । \newline
21. प्र॒जाया॒मिति॑ प्र - जाया᳚म् । \newline
22. वा॒जी जा॑यते जायते वा॒जी वा॒जी जा॑यत इ॒य मि॒यम् जा॑यते वा॒जी वा॒जी जा॑यत इ॒यम् । \newline
23. जा॒य॒त॒ इ॒य मि॒यम् जा॑यते जायत इ॒यं ॅवै वा इ॒यम् जा॑यते जायत इ॒यं ॅवै । \newline
24. इ॒यं ॅवै वा इ॒य मि॒यं ॅवै होता॒ होता॒ वा इ॒य मि॒यं ॅवै होता᳚ । \newline
25. वै होता॒ होता॒ वै वै होता॒ ऽसा व॒सौ होता॒ वै वै होता॒ ऽसौ । \newline
26. होता॒ ऽसा व॒सौ होता॒ होता॒ ऽसा व॑द्ध्व॒र्यु र॑द्ध्व॒र्यु र॒सौ होता॒ होता॒ ऽसा व॑द्ध्व॒र्युः । \newline
27. अ॒सा व॑द्ध्व॒र्यु र॑द्ध्व॒र्यु र॒सा व॒सा व॑द्ध्व॒र्युर् यद् यद॑द्ध्व॒र्यु र॒सा व॒सा व॑द्ध्व॒र्युर् यत् । \newline
28. अ॒द्ध्व॒र्युर् यद् यद॑द्ध्व॒र्यु र॑द्ध्व॒र्युर् यदासी॑न॒ आसी॑नो॒ यद॑द्ध्व॒र्यु र॑द्ध्व॒र्युर् यदासी॑नः । \newline
29. यदासी॑न॒ आसी॑नो॒ यद् यदासी॑नः॒ शꣳस॑ति॒ शꣳस॒ त्यासी॑नो॒ यद् यदासी॑नः॒ शꣳस॑ति । \newline
30. आसी॑नः॒ शꣳस॑ति॒ शꣳस॒ त्यासी॑न॒ आसी॑नः॒ शꣳस॑त्य॒स्या अ॒स्याः शꣳस॒ त्यासी॑न॒ आसी॑नः॒ शꣳस॑त्य॒स्याः । \newline
31. शꣳस॑त्य॒स्या अ॒स्याः शꣳस॑ति॒ शꣳस॑ त्य॒स्या ए॒वैवास्याः शꣳस॑ति॒ शꣳस॑ त्य॒स्या ए॒व । \newline
32. अ॒स्या ए॒वैवास्या अ॒स्या ए॒व तत् तदे॒वास्या अ॒स्या ए॒व तत् । \newline
33. ए॒व तत् तदे॒वैव तद्धोता॒ होता॒ तदे॒वैव तद्धोता᳚ । \newline
34. तद्धोता॒ होता॒ तत् तद्धोता॒ न न होता॒ तत् तद्धोता॒ न । \newline
35. होता॒ न न होता॒ होता॒ नैत्ये॑ति॒ न होता॒ होता॒ नैति॑ । \newline
36. नैत्ये॑ति॒ न नैत्या स्त॒ आस्त॑ एति॒ न नैत्यास्ते᳚ । \newline
37. ए॒त्यास्त॒ आस्त॑ एत्ये॒त्यास्त॑ इवे॒ वास्त॑ एत्ये॒ त्यास्त॑ इव । \newline
38. आस्त॑ इवे॒ वास्त॒ आस्त॑ इव॒ हि हीवास्त॒ आस्त॑ इव॒ हि । \newline
39. इ॒व॒ हि हीवे॑ व॒ हीय मि॒यꣳ हीवे॑ व॒ हीयम् । \newline
40. हीय मि॒यꣳ हि हीय मथो॒ अथो॑ इ॒यꣳ हि हीय मथो᳚ । \newline
41. इ॒य मथो॒ अथो॑ इ॒य मि॒य मथो॑ इ॒मा मि॒मा मथो॑ इ॒य मि॒य मथो॑ इ॒माम् । \newline
42. अथो॑ इ॒मा मि॒मा मथो॒ अथो॑ इ॒मा मे॒वैवे मा मथो॒ अथो॑ इ॒मा मे॒व । \newline
43. अथो॒ इत्यथो᳚ । \newline
44. इ॒मा मे॒वैवे मा मि॒मा मे॒व तेन॒ तेनै॒वे मा मि॒मा मे॒व तेन॑ । \newline
45. ए॒व तेन॒ तेनै॒वैव तेन॒ यज॑मानो॒ यज॑मान॒ स्तेनै॒वैव तेन॒ यज॑मानः । \newline
46. तेन॒ यज॑मानो॒ यज॑मान॒ स्तेन॒ तेन॒ यज॑मानो दुहे दुहे॒ यज॑मान॒ स्तेन॒ तेन॒ यज॑मानो दुहे । \newline
47. यज॑मानो दुहे दुहे॒ यज॑मानो॒ यज॑मानो दुहे॒ यद् यद् दु॑हे॒ यज॑मानो॒ यज॑मानो दुहे॒ यत् । \newline
48. दु॒हे॒ यद् यद् दु॑हे दुहे॒ यत् तिष्ठꣳ॒॒ स्तिष्ठ॒न्॒. यद् दु॑हे दुहे॒ यत् तिष्ठन्न्॑ । \newline
49. यत् तिष्ठꣳ॒॒ स्तिष्ठ॒न्॒. यद् यत् तिष्ठ॑न् प्रतिगृ॒णाति॑ प्रतिगृ॒णाति॒ तिष्ठ॒न्॒. यद् यत् तिष्ठ॑न् प्रतिगृ॒णाति॑ । \newline
50. तिष्ठ॑न् प्रतिगृ॒णाति॑ प्रतिगृ॒णाति॒ तिष्ठꣳ॒॒ स्तिष्ठ॑न् प्रतिगृ॒णा त्य॒मुष्या॑ अ॒मुष्याः᳚ प्रतिगृ॒णाति॒ तिष्ठꣳ॒॒ स्तिष्ठ॑न् प्रतिगृ॒णा त्य॒मुष्याः᳚ । \newline
51. प्र॒ति॒गृ॒णा त्य॒मुष्या॑ अ॒मुष्याः᳚ प्रतिगृ॒णाति॑ प्रतिगृ॒णा त्य॒मुष्या॑ ए॒वैवामुष्याः᳚ प्रतिगृ॒णाति॑ प्रतिगृ॒णा त्य॒मुष्या॑ ए॒व । \newline
52. प्र॒ति॒गृ॒णातीति॑ प्रति - गृ॒णाति॑ । \newline
53. अ॒मुष्या॑ ए॒वैवा मुष्या॑ अ॒मुष्या॑ ए॒व तत् तदे॒वा मुष्या॑ अ॒मुष्या॑ ए॒व तत् । \newline
54. ए॒व तत् तदे॒वैव तद॑द्ध्व॒र्यु र॑द्ध्व॒र्यु स्तदे॒वैव तद॑द्ध्व॒र्युः । \newline
55. तद॑द्ध्व॒र्यु र॑द्ध्व॒र्यु स्तत् तद॑द्ध्व॒र्युर् न नाद्ध्व॒र्यु स्तत् तद॑द्ध्व॒र्युर् न । \newline
56. अ॒द्ध्व॒र्युर् न नाद्ध्व॒र्यु र॑द्ध्व॒र्युर् नैत्ये॑ति॒ नाद्ध्व॒र्यु र॑द्ध्व॒र्युर् नैति॑ । \newline
57. नैत्ये॑ति॒ न नैति॒ तिष्ठ॑ति॒ तिष्ठ॑ त्येति॒ न नैति॒ तिष्ठ॑ति । \newline
58. ए॒ति॒ तिष्ठ॑ति॒ तिष्ठ॑ त्येत्येति॒ तिष्ठ॑तीवे व॒ तिष्ठ॑ त्येत्येति॒ तिष्ठ॑तीव । \newline
\pagebreak
\markright{ TS 3.2.9.7  \hfill https://www.vedavms.in \hfill}

\section{ TS 3.2.9.7 }

\textbf{TS 3.2.9.7 } \newline
\textbf{Samhita Paata} \newline

तिष्ठ॑तीव॒ ह्य॑सावथो॑ अ॒मूमे॒व तेन॒ यज॑मानो दुहे॒ यदासी॑नः॒ शꣳस॑ति॒ तस्मा॑दि॒तः प्र॑दानं दे॒वा उप॑ जीवन्ति॒ यत् तिष्ठ॑न् प्रतिगृ॒णाति॒ तस्मा॑द॒मुतः॑ प्रदानं मनु॒ष्या॑ उप॑ जीवन्ति॒ यत् प्राङासी॑नः॒ शꣳस॑ति प्र॒त्यङ् तिष्ठ॑न् प्रतिगृ॒णाति॒ तस्मा᳚त् प्रा॒चीनꣳ॒॒ रेतो॑ धीयते प्र॒तीचीः᳚ प्र॒जा जा॑यन्ते॒ यद्वै होता᳚ऽद्ध्व॒र्युम॑भ्या॒ह्वय॑ते॒ वज्र॑मेनम॒भि प्रव॑र्तयति॒ ( ) परा॒ङा व॑र्तते॒ वज्र॑मे॒व तन्नि क॑रोति ॥ \newline

\textbf{Pada Paata} \newline

तिष्ठ॑ति । इ॒व॒ । हि । अ॒सौ । अथो॒ इति॑ । अ॒मूम् । ए॒व । तेन॑ । यज॑मानः । दु॒हे॒ । यत् । आसी॑नः । शꣳस॑ति । तस्मा᳚त् । इ॒तः प्र॑दान॒मिती॒तः - प्र॒दा॒न॒म् । दे॒वाः । उपेति॑ । जी॒व॒न्ति॒ । यत् । तिष्ठन्न्॑ । प्र॒ति॒गृ॒णातीति॑ प्रति - गृ॒णाति॑ । तस्मा᳚त् । अ॒मुतः॑ प्रदान॒मित्य॒मुतः॑ - प्र॒दा॒न॒म् । म॒नु॒ष्याः᳚ । उपेति॑ । जी॒व॒न्ति॒ । यत् । प्राङ् । आसी॑नः । शꣳस॑ति । प्र॒त्यङ् । तिष्ठन्न्॑ । प्र॒ति॒गृ॒णातीति॑ प्रति - गृ॒णाति॑ । तस्मा᳚त् । प्रा॒चीन᳚म् । रेतः॑ । धी॒य॒ते॒ । प्र॒तीचीः᳚ । प्र॒जा इति॑ प्र - जाः । जा॒य॒न्ते॒ । यत् । वै । होता᳚ । अ॒द्ध्व॒र्युम् । अ॒भ्या॒ह्वय॑त॒ इत्य॑भि - आ॒ह्वय॑ते । वज्र᳚म् । ए॒न॒म् । अ॒भि । प्रेति॑ । व॒र्त॒य॒ति॒ ( ) । पराङ्॑ । एति॑ । व॒र्त॒ते॒ । वज्र᳚म् । ए॒व । तत् । नीति॑ । क॒रो॒ति॒ ॥  \newline


\textbf{Krama Paata} \newline

तिष्ठ॑तीव । इ॒व॒ हि । ह्य॑सौ । अ॒सावथो᳚ । अथो॑ अ॒मूम् । अथो॒ इत्यथो᳚ । अ॒मूमे॒व । ए॒व तेन॑ । तेन॒ यज॑मानः । यज॑मानो दुहे । दु॒हे॒ यत् । यदासी॑नः । आसी॑नः॒ शꣳस॑ति । शꣳस॑ति॒ तस्मा᳚त् । तस्मा॑दि॒तःप्र॑दानम् । इ॒तःप्र॑दानम् दे॒वाः । इ॒तःप्र॑दान॒मिती॒तः - प्र॒दा॒न॒म् । दे॒वा उप॑ । उप॑ जीवन्ति । जी॒व॒न्ति॒ यत् । यत् तिष्ठन्न्॑ । तिष्ठ॑न् प्रतिगृ॒णाति॑ । प्र॒ति॒गृ॒णाति॒ तस्मा᳚त् । प्र॒ति॒गृ॒णातीति॑ प्रति - गृ॒णाति॑ । तस्मा॑द॒मुतः॑प्रदानम् । अ॒मुतः॑प्रदानम् मनु॒ष्याः᳚ । अ॒मुतः॑प्रदान॒मित्य॒मुतः॑ - प्र॒दा॒न॒म् । म॒नु॒ष्या॑ उप॑ । उप॑ जिवन्ति । जी॒व॒न्ति॒ यत् । यत् प्राङ् । प्राङासी॑नः । आसी॑नः॒ शꣳस॑ति । शꣳस॑ति प्र॒त्यङ् । प्र॒त्यङ् तिष्ठन्न्॑ । तिष्ठ॑न् प्रतिगृ॒णाति॑ । प्र॒ति॒गृ॒णाति॒ तस्मा᳚त् । प्र॒ति॒गृ॒णातीति॑ प्रति - गृ॒णाति॑ । तस्मा᳚त् प्रा॒चीन᳚म् । प्रा॒चीनꣳ॒॒ रेतः॑ । रेतो॑ धीयते । धी॒य॒ते॒ प्र॒तीचीः᳚ । प्र॒तीचीः᳚ प्र॒जाः । प्र॒जा जा॑यन्ते । प्र॒जा इति॑ प्र - जाः । जा॒य॒न्ते॒ यत् । यद् वै । वै होता᳚ । होता᳚ ऽद्ध्व॒र्युम् । अ॒द्ध्व॒र्युम॑भ्या॒ह्वय॑ते । अ॒भ्या॒ह्वय॑ते॒ वज्र᳚म् । अ॒भ्या॒ह्वय॑त॒ इत्य॑भि - आ॒ह्वय॑ते । वज्र॑मेनम् । ए॒न॒म॒भि । अ॒भि प्र । प्र व॑र्तयति ( ) । व॒र्त॒य॒ति॒ पराङ्॑ । परा॒ङा । आ व॑र्तते । व॒र्त॒ते॒ वज्र᳚म् । वज्र॑मे॒व । ए॒व तत् । तन् नि । नि क॑रोति । क॒रो॒तीति॑ करोति । \newline

\textbf{Jatai Paata} \newline

1. तिष्ठ॑तीवे व॒ तिष्ठ॑ति॒ तिष्ठ॑तीव । \newline
2. इ॒व॒ हि हीवे॑ व॒ हि । \newline
3. ह्य॑सा व॒सौ हि ह्य॑सौ । \newline
4. अ॒सा वथो॒ अथो॑ अ॒सा व॒सा वथो᳚ । \newline
5. अथो॑ अ॒मू म॒मू मथो॒ अथो॑ अ॒मूम् । \newline
6. अथो॒ इत्यथो᳚ । \newline
7. अ॒मू मे॒वैवामू म॒मू मे॒व । \newline
8. ए॒व तेन॒ तेनै॒वैव तेन॑ । \newline
9. तेन॒ यज॑मानो॒ यज॑मान॒ स्तेन॒ तेन॒ यज॑मानः । \newline
10. यज॑मानो दुहे दुहे॒ यज॑मानो॒ यज॑मानो दुहे । \newline
11. दु॒हे॒ यद् यद् दु॑हे दुहे॒ यत् । \newline
12. यदासी॑न॒ आसी॑नो॒ यद् यदासी॑नः । \newline
13. आसी॑नः॒ शꣳस॑ति॒ शꣳस॒ त्यासी॑न॒ आसी॑नः॒ शꣳस॑ति । \newline
14. शꣳस॑ति॒ तस्मा॒त् तस्मा॒ च्छꣳस॑ति॒ शꣳस॑ति॒ तस्मा᳚त् । \newline
15. तस्मा॑ दि॒तःप्र॑दान मि॒तःप्र॑दान॒म् तस्मा॒त् तस्मा॑ दि॒तःप्र॑दानम् । \newline
16. इ॒तःप्र॑दानम् दे॒वा दे॒वा इ॒तःप्र॑दान मि॒तःप्र॑दानम् दे॒वाः । \newline
17. इ॒तःप्र॑दान॒मिती॒तः - प्र॒दा॒न॒म् । \newline
18. दे॒वा उपोप॑ दे॒वा दे॒वा उप॑ । \newline
19. उप॑ जीवन्ति जीव॒ न्त्युपोप॑ जीवन्ति । \newline
20. जी॒व॒न्ति॒ यद् यज् जी॑वन्ति जीवन्ति॒ यत् । \newline
21. यत् तिष्ठꣳ॒॒ स्तिष्ठ॒न्॒. यद् यत् तिष्ठन्न्॑ । \newline
22. तिष्ठ॑न् प्रतिगृ॒णाति॑ प्रतिगृ॒णाति॒ तिष्ठꣳ॒॒ स्तिष्ठ॑न् प्रतिगृ॒णाति॑ । \newline
23. प्र॒ति॒गृ॒णाति॒ तस्मा॒त् तस्मा᳚त् प्रतिगृ॒णाति॑ प्रतिगृ॒णाति॒ तस्मा᳚त् । \newline
24. प्र॒ति॒गृ॒णातीति॑ प्रति - गृ॒णाति॑ । \newline
25. तस्मा॑ द॒मुतः॑प्रदान म॒मुतः॑प्रदान॒म् तस्मा॒त् तस्मा॑ द॒मुतः॑प्रदानम् । \newline
26. अ॒मुतः॑प्रदानम् मनु॒ष्या॑ मनु॒ष्या॑ अ॒मुतः॑प्रदान म॒मुतः॑प्रदानम् मनु॒ष्याः᳚ । \newline
27. अ॒मुतः॑प्रदान॒मित्य॒मुतः॑ - प्र॒दा॒न॒म् । \newline
28. म॒नु॒ष्या॑ उपोप॑ मनु॒ष्या॑ मनु॒ष्या॑ उप॑ । \newline
29. उप॑ जीवन्ति जीव॒ न्त्युपोप॑ जीवन्ति । \newline
30. जी॒व॒न्ति॒ यद् यज् जी॑वन्ति जीवन्ति॒ यत् । \newline
31. यत् प्राङ् प्राङ् यद् यत् प्राङ् । \newline
32. प्राङासी॑न॒ आसी॑नः॒ प्राङ् प्राङासी॑नः । \newline
33. आसी॑नः॒ शꣳस॑ति॒ शꣳस॒ त्यासी॑न॒ आसी॑नः॒ शꣳस॑ति । \newline
34. शꣳस॑ति प्र॒त्यङ् प्र॒त्यङ् छꣳस॑ति॒ शꣳस॑ति प्र॒त्यङ् । \newline
35. प्र॒त्यङ् तिष्ठꣳ॒॒ स्तिष्ठ॑न् प्र॒त्यङ् प्र॒त्यङ् तिष्ठन्न्॑ । \newline
36. तिष्ठ॑न् प्रतिगृ॒णाति॑ प्रतिगृ॒णाति॒ तिष्ठꣳ॒॒ स्तिष्ठ॑न् प्रतिगृ॒णाति॑ । \newline
37. प्र॒ति॒गृ॒णाति॒ तस्मा॒त् तस्मा᳚त् प्रतिगृ॒णाति॑ प्रतिगृ॒णाति॒ तस्मा᳚त् । \newline
38. प्र॒ति॒गृ॒णातीति॑ प्रति - गृ॒णाति॑ । \newline
39. तस्मा᳚त् प्रा॒चीन॑म् प्रा॒चीन॒म् तस्मा॒त् तस्मा᳚त् प्रा॒चीन᳚म् । \newline
40. प्रा॒चीनꣳ॒॒ रेतो॒ रेतः॑ प्रा॒चीन॑म् प्रा॒चीनꣳ॒॒ रेतः॑ । \newline
41. रेतो॑ धीयते धीयते॒ रेतो॒ रेतो॑ धीयते । \newline
42. धी॒य॒ते॒ प्र॒तीचीः᳚ प्र॒तीची᳚र् धीयते धीयते प्र॒तीचीः᳚ । \newline
43. प्र॒तीचीः᳚ प्र॒जाः प्र॒जाः प्र॒तीचीः᳚ प्र॒तीचीः᳚ प्र॒जाः । \newline
44. प्र॒जा जा॑यन्ते जायन्ते प्र॒जाः प्र॒जा जा॑यन्ते । \newline
45. प्र॒जा इति॑ प्र - जाः । \newline
46. जा॒य॒न्ते॒ यद् यज् जा॑यन्ते जायन्ते॒ यत् । \newline
47. यद् वै वै यद् यद् वै । \newline
48. वै होता॒ होता॒ वै वै होता᳚ । \newline
49. होता᳚ ऽद्ध्व॒र्यु म॑द्ध्व॒र्युꣳ होता॒ होता᳚ ऽद्ध्व॒र्युम् । \newline
50. अ॒द्ध्व॒र्यु म॑भ्या॒ह्वय॑ते ऽभ्या॒ह्वय॑ते ऽद्ध्व॒र्यु म॑द्ध्व॒र्यु म॑भ्या॒ह्वय॑ते । \newline
51. अ॒भ्या॒ह्वय॑ते॒ वज्रं॒ ॅवज्र॑ मभ्या॒ह्वय॑ते ऽभ्या॒ह्वय॑ते॒ वज्र᳚म् । \newline
52. अ॒भ्या॒ह्वय॑त॒ इत्य॑भि - आ॒ह्वय॑ते । \newline
53. वज्र॑ मेन मेनं॒ ॅवज्रं॒ ॅवज्र॑ मेनम् । \newline
54. ए॒न॒ म॒भ्या᳚(1॒)भ्ये॑न मेन म॒भि । \newline
55. अ॒भि प्र प्राभ्य॑भि प्र । \newline
56. प्र व॑र्तयति वर्तयति॒ प्र प्र व॑र्तयति । \newline
57. व॒र्त॒य॒ति॒ परा॒ङ् पराङ्॑ वर्तयति वर्तयति॒ पराङ्॑ । \newline
58. परा॒ङा परा॒ङ् परा॒ङा । \newline
59. आ व॑र्तते वर्तत॒ आ व॑र्तते । \newline
60. व॒र्त॒ते॒ वज्रं॒ ॅवज्रं॑ ॅवर्तते वर्तते॒ वज्र᳚म् । \newline
61. वज्र॑ मे॒वैव वज्रं॒ ॅवज्र॑ मे॒व । \newline
62. ए॒व तत् तदे॒वैव तत् । \newline
63. तन् नि नि तत् तन् नि । \newline
64. नि क॑रोति करोति॒ नि नि क॑रोति । \newline
65. क॒रो॒तीति॑ करोति । \newline

\textbf{Ghana Paata } \newline

1. तिष्ठ॑तीवे व॒ तिष्ठ॑ति॒ तिष्ठ॑तीव॒ हि हीव॒ तिष्ठ॑ति॒ तिष्ठ॑तीव॒ हि । \newline
2. इ॒व॒ हि हीवे॑ व॒ ह्य॑सा व॒सौ हीवे॑ व॒ ह्य॑सौ । \newline
3. ह्य॑सा व॒सौ हि ह्य॑सा वथो॒ अथो॑ अ॒सौ हि ह्य॑सा वथो᳚ । \newline
4. अ॒सा वथो॒ अथो॑ अ॒सा व॒सा वथो॑ अ॒मू म॒मू मथो॑ अ॒सा व॒सा वथो॑ अ॒मूम् । \newline
5. अथो॑ अ॒मू म॒मू मथो॒ अथो॑ अ॒मू मे॒वैवामू मथो॒ अथो॑ अ॒मू मे॒व । \newline
6. अथो॒ इत्यथो᳚ । \newline
7. अ॒मू मे॒वैवामू म॒मू मे॒व तेन॒ तेनै॒वामू म॒मू मे॒व तेन॑ । \newline
8. ए॒व तेन॒ तेनै॒वैव तेन॒ यज॑मानो॒ यज॑मान॒ स्तेनै॒वैव तेन॒ यज॑मानः । \newline
9. तेन॒ यज॑मानो॒ यज॑मान॒ स्तेन॒ तेन॒ यज॑मानो दुहे दुहे॒ यज॑मान॒ स्तेन॒ तेन॒ यज॑मानो दुहे । \newline
10. यज॑मानो दुहे दुहे॒ यज॑मानो॒ यज॑मानो दुहे॒ यद् यद् दु॑हे॒ यज॑मानो॒ यज॑मानो दुहे॒ यत् । \newline
11. दु॒हे॒ यद् यद् दु॑हे दुहे॒ यदासी॑न॒ आसी॑नो॒ यद् दु॑हे दुहे॒ यदासी॑नः । \newline
12. यदासी॑न॒ आसी॑नो॒ यद् यदासी॑नः॒ शꣳस॑ति॒ शꣳस॒ त्यासी॑नो॒ यद् यदासी॑नः॒ शꣳस॑ति । \newline
13. आसी॑नः॒ शꣳस॑ति॒ शꣳस॒ त्यासी॑न॒ आसी॑नः॒ शꣳस॑ति॒ तस्मा॒त् तस्मा॒ च्छꣳस॒ त्यासी॑न॒ आसी॑नः॒ शꣳस॑ति॒ तस्मा᳚त् । \newline
14. शꣳस॑ति॒ तस्मा॒त् तस्मा॒ च्छꣳस॑ति॒ शꣳस॑ति॒ तस्मा॑ दि॒तःप्र॑दान मि॒तःप्र॑दान॒म् तस्मा॒ च्छꣳस॑ति॒ शꣳस॑ति॒ तस्मा॑ दि॒तःप्र॑दानम् । \newline
15. तस्मा॑ दि॒तःप्र॑दान मि॒तःप्र॑दान॒म् तस्मा॒त् तस्मा॑ दि॒तःप्र॑दानम् दे॒वा दे॒वा इ॒तःप्र॑दान॒म् तस्मा॒त् तस्मा॑ दि॒तःप्र॑दानम् दे॒वाः । \newline
16. इ॒तःप्र॑दानम् दे॒वा दे॒वा इ॒तःप्र॑दान मि॒तःप्र॑दानम् दे॒वा उपोप॑ दे॒वा इ॒तःप्र॑दान मि॒तःप्र॑दानम् दे॒वा उप॑ । \newline
17. इ॒तःप्र॑दान॒मिती॒तः - प्र॒दा॒न॒म् । \newline
18. दे॒वा उपोप॑ दे॒वा दे॒वा उप॑ जीवन्ति जीव॒ न्त्युप॑ दे॒वा दे॒वा उप॑ जीवन्ति । \newline
19. उप॑ जीवन्ति जीव॒ न्त्युपोप॑ जीवन्ति॒ यद् यज् जी॑व॒ न्त्युपोप॑ जीवन्ति॒ यत् । \newline
20. जी॒व॒न्ति॒ यद् यज् जी॑वन्ति जीवन्ति॒ यत् तिष्ठꣳ॒॒ स्तिष्ठ॒न्॒. यज् जी॑वन्ति जीवन्ति॒ यत् तिष्ठन्न्॑ । \newline
21. यत् तिष्ठꣳ॒॒ स्तिष्ठ॒न्॒. यद् यत् तिष्ठ॑न् प्रतिगृ॒णाति॑ प्रतिगृ॒णाति॒ तिष्ठ॒न्॒. यद् यत् तिष्ठ॑न् प्रतिगृ॒णाति॑ । \newline
22. तिष्ठ॑न् प्रतिगृ॒णाति॑ प्रतिगृ॒णाति॒ तिष्ठꣳ॒॒ स्तिष्ठ॑न् प्रतिगृ॒णाति॒ तस्मा॒त् तस्मा᳚त् प्रतिगृ॒णाति॒ तिष्ठꣳ॒॒ स्तिष्ठ॑न् प्रतिगृ॒णाति॒ तस्मा᳚त् । \newline
23. प्र॒ति॒गृ॒णाति॒ तस्मा॒त् तस्मा᳚त् प्रतिगृ॒णाति॑ प्रतिगृ॒णाति॒ तस्मा॑ द॒मुतः॑प्रदान म॒मुतः॑प्रदान॒म् तस्मा᳚त् प्रतिगृ॒णाति॑ प्रतिगृ॒णाति॒ तस्मा॑ द॒मुतः॑प्रदानम् । \newline
24. प्र॒ति॒गृ॒णातीति॑ प्रति - गृ॒णाति॑ । \newline
25. तस्मा॑ द॒मुतः॑प्रदान म॒मुतः॑प्रदान॒म् तस्मा॒त् तस्मा॑ द॒मुतः॑प्रदानम् मनु॒ष्या॑ मनु॒ष्या॑ 
अ॒मुतः॑प्रदान॒म् तस्मा॒त् तस्मा॑ द॒मुतः॑प्रदानम् मनु॒ष्याः᳚ । \newline
26. अ॒मुतः॑प्रदानम् मनु॒ष्या॑ मनु॒ष्या॑ अ॒मुतः॑प्रदान म॒मुतः॑प्रदानम् मनु॒ष्या॑ उपोप॑ 
मनु॒ष्या॑ अ॒मुतः॑प्रदान म॒मुतः॑प्रदानम् मनु॒ष्या॑ उप॑ । \newline
27. अ॒मुतः॑प्रदान॒मित्य॒मुतः॑ - प्र॒दा॒न॒म् । \newline
28. म॒नु॒ष्या॑ उपोप॑ मनु॒ष्या॑ मनु॒ष्या॑ उप॑ जीवन्ति जीव॒ न्त्युप॑ मनु॒ष्या॑ मनु॒ष्या॑ उप॑ जीवन्ति । \newline
29. उप॑ जीवन्ति जीव॒ न्त्युपोप॑ जीवन्ति॒ यद् यज् जी॑व॒ न्त्युपोप॑ जीवन्ति॒ यत् । \newline
30. जी॒व॒न्ति॒ यद् यज् जी॑वन्ति जीवन्ति॒ यत् प्राङ् प्राङ् यज् जी॑वन्ति जीवन्ति॒ यत् प्राङ् । \newline
31. यत् प्राङ् प्राङ् यद् यत् प्राङासी॑न॒ आसी॑नः॒ प्राङ् यद् यत् प्राङासी॑नः । \newline
32. प्राङासी॑न॒ आसी॑नः॒ प्राङ् प्राङासी॑नः॒ शꣳस॑ति॒ शꣳस॒ त्यासी॑नः॒ प्राङ् प्राङासी॑नः॒ शꣳस॑ति । \newline
33. आसी॑नः॒ शꣳस॑ति॒ शꣳस॒ त्यासी॑न॒ आसी॑नः॒ शꣳस॑ति प्र॒त्यङ् प्र॒त्यङ् छꣳस॒ त्यासी॑न॒ आसी॑नः॒ शꣳस॑ति प्र॒त्यङ् । \newline
34. शꣳस॑ति प्र॒त्यङ् प्र॒त्यङ् छꣳस॑ति॒ शꣳस॑ति प्र॒त्यङ् तिष्ठꣳ॒॒ स्तिष्ठ॑न् प्र॒त्यङ् छꣳस॑ति॒ शꣳस॑ति प्र॒त्यङ् तिष्ठन्न्॑ । \newline
35. प्र॒त्यङ् तिष्ठꣳ॒॒ स्तिष्ठ॑न् प्र॒त्यङ् प्र॒त्यङ् तिष्ठ॑न् प्रतिगृ॒णाति॑ प्रतिगृ॒णाति॒ तिष्ठ॑न् प्र॒त्यङ् प्र॒त्यङ् तिष्ठ॑न् प्रतिगृ॒णाति॑ । \newline
36. तिष्ठ॑न् प्रतिगृ॒णाति॑ प्रतिगृ॒णाति॒ तिष्ठꣳ॒॒ स्तिष्ठ॑न् प्रतिगृ॒णाति॒ तस्मा॒त् तस्मा᳚त् प्रतिगृ॒णाति॒ तिष्ठꣳ॒॒ स्तिष्ठ॑न् प्रतिगृ॒णाति॒ तस्मा᳚त् । \newline
37. प्र॒ति॒गृ॒णाति॒ तस्मा॒त् तस्मा᳚त् प्रतिगृ॒णाति॑ प्रतिगृ॒णाति॒ तस्मा᳚त् प्रा॒चीन॑म् प्रा॒चीन॒म् तस्मा᳚त् प्रतिगृ॒णाति॑ प्रतिगृ॒णाति॒ तस्मा᳚त् प्रा॒चीन᳚म् । \newline
38. प्र॒ति॒गृ॒णातीति॑ प्रति - गृ॒णाति॑ । \newline
39. तस्मा᳚त् प्रा॒चीन॑म् प्रा॒चीन॒म् तस्मा॒त् तस्मा᳚त् प्रा॒चीनꣳ॒॒ रेतो॒ रेतः॑ प्रा॒चीन॒म् तस्मा॒त् तस्मा᳚त् प्रा॒चीनꣳ॒॒ रेतः॑ । \newline
40. प्रा॒चीनꣳ॒॒ रेतो॒ रेतः॑ प्रा॒चीन॑म् प्रा॒चीनꣳ॒॒ रेतो॑ धीयते धीयते॒ रेतः॑ प्रा॒चीन॑म् प्रा॒चीनꣳ॒॒ रेतो॑ धीयते । \newline
41. रेतो॑ धीयते धीयते॒ रेतो॒ रेतो॑ धीयते प्र॒तीचीः᳚ प्र॒तीची᳚र् धीयते॒ रेतो॒ रेतो॑ धीयते प्र॒तीचीः᳚ । \newline
42. धी॒य॒ते॒ प्र॒तीचीः᳚ प्र॒तीची᳚र् धीयते धीयते प्र॒तीचीः᳚ प्र॒जाः प्र॒जाः प्र॒तीची᳚र् धीयते धीयते प्र॒तीचीः᳚ प्र॒जाः । \newline
43. प्र॒तीचीः᳚ प्र॒जाः प्र॒जाः प्र॒तीचीः᳚ प्र॒तीचीः᳚ प्र॒जा जा॑यन्ते जायन्ते प्र॒जाः प्र॒तीचीः᳚ प्र॒तीचीः᳚ प्र॒जा जा॑यन्ते । \newline
44. प्र॒जा जा॑यन्ते जायन्ते प्र॒जाः प्र॒जा जा॑यन्ते॒ यद् यज् जा॑यन्ते प्र॒जाः प्र॒जा जा॑यन्ते॒ यत् । \newline
45. प्र॒जा इति॑ प्र - जाः । \newline
46. जा॒य॒न्ते॒ यद् यज् जा॑यन्ते जायन्ते॒ यद् वै वै यज् जा॑यन्ते जायन्ते॒ यद् वै । \newline
47. यद् वै वै यद् यद् वै होता॒ होता॒ वै यद् यद् वै होता᳚ । \newline
48. वै होता॒ होता॒ वै वै होता᳚ ऽद्ध्व॒र्यु म॑द्ध्व॒र्युꣳ होता॒ वै वै होता᳚ ऽद्ध्व॒र्युम् । \newline
49. होता᳚ ऽद्ध्व॒र्यु म॑द्ध्व॒र्युꣳ होता॒ होता᳚ ऽद्ध्व॒र्यु म॑भ्या॒ह्वय॑ते ऽभ्या॒ह्वय॑ते ऽद्ध्व॒र्युꣳ होता॒ होता᳚ ऽद्ध्व॒र्यु म॑भ्या॒ह्वय॑ते । \newline
50. अ॒द्ध्व॒र्यु म॑भ्या॒ह्वय॑ते ऽभ्या॒ह्वय॑ते ऽद्ध्व॒र्यु म॑द्ध्व॒र्यु म॑भ्या॒ह्वय॑ते॒ वज्रं॒ ॅवज्र॑ मभ्या॒ह्वय॑ते ऽद्ध्व॒र्यु म॑द्ध्व॒र्यु म॑भ्या॒ह्वय॑ते॒ वज्र᳚म् । \newline
51. अ॒भ्या॒ह्वय॑ते॒ वज्रं॒ ॅवज्र॑ मभ्या॒ह्वय॑ते ऽभ्या॒ह्वय॑ते॒ वज्र॑ मेन मेनं॒ ॅवज्र॑ मभ्या॒ह्वय॑ते ऽभ्या॒ह्वय॑ते॒ वज्र॑ मेनम् । \newline
52. अ॒भ्या॒ह्वय॑त॒ इत्य॑भि - आ॒ह्वय॑ते । \newline
53. वज्र॑ मेन मेनं॒ ॅवज्रं॒ ॅवज्र॑ मेन म॒भ्या᳚(1॒)भ्ये॑नं॒ ॅवज्रं॒ ॅवज्र॑ मेन म॒भि । \newline
54. ए॒न॒ म॒भ्या᳚(1॒)भ्ये॑न मेन म॒भि प्र प्राभ्ये॑न मेन म॒भि प्र । \newline
55. अ॒भि प्र प्राभ्य॑भि प्र व॑र्तयति वर्तयति॒ प्राभ्य॑भि प्र व॑र्तयति । \newline
56. प्र व॑र्तयति वर्तयति॒ प्र प्र व॑र्तयति॒ परा॒ङ् पराङ्॑ वर्तयति॒ प्र प्र व॑र्तयति॒ पराङ्॑ । \newline
57. व॒र्त॒य॒ति॒ परा॒ङ् पराङ्॑ वर्तयति वर्तयति॒ परा॒ङा पराङ्॑ वर्तयति वर्तयति॒ परा॒ङा । \newline
58. परा॒ङा परा॒ङ् परा॒ङा व॑र्तते वर्तत॒ आ परा॒ङ् परा॒ङा व॑र्तते । \newline
59. आ व॑र्तते वर्तत॒ आ व॑र्तते॒ वज्रं॒ ॅवज्रं॑ ॅवर्तत॒ आ व॑र्तते॒ वज्र᳚म् । \newline
60. व॒र्त॒ते॒ वज्रं॒ ॅवज्रं॑ ॅवर्तते वर्तते॒ वज्र॑ मे॒वैव वज्रं॑ ॅवर्तते वर्तते॒ वज्र॑ मे॒व । \newline
61. वज्र॑ मे॒वैव वज्रं॒ ॅवज्र॑ मे॒व तत् तदे॒व वज्रं॒ ॅवज्र॑ मे॒व तत् । \newline
62. ए॒व तत् तदे॒वैव तन् नि नि तदे॒वैव तन् नि । \newline
63. तन् नि नि तत् तन् नि क॑रोति करोति॒ नि तत् तन् नि क॑रोति । \newline
64. नि क॑रोति करोति॒ नि नि क॑रोति । \newline
65. क॒रो॒तीति॑ करोति । \newline
\pagebreak
\markright{ TS 3.2.10.1  \hfill https://www.vedavms.in \hfill}

\section{ TS 3.2.10.1 }

\textbf{TS 3.2.10.1 } \newline
\textbf{Samhita Paata} \newline

उ॒प॒या॒मगृ॑हीतोऽसि वाक्ष॒सद॑सि वा॒क्पाभ्यां᳚ त्वा क्रतु॒पाभ्या॑म॒स्य य॒ज्ञ्स्य॑ ध्रु॒वस्याऽद्ध्य॑-क्षाभ्यां गृह्णा-म्युपया॒मगृ॑हीतोऽस्यृत॒सद॑सि चक्षु॒ष्पाभ्यां᳚ त्वा क्रतु॒पाभ्या॑म॒स्य य॒ज्ञ्स्य॑ ध्रु॒वस्याऽद्ध्य॑क्षाभ्यां गृह्णाम्युपया॒मगृ॑हीतोऽसि श्रुत॒सद॑सि श्रोत्र॒पाभ्यां᳚ त्वा क्रतु॒पाभ्या॑म॒स्य य॒ज्ञ्स्य॑ ध्रु॒वस्याऽद्ध्य॑क्षाभ्यां गृह्णामि दे॒वेभ्य॑स्त्वा वि॒श्वदे॑वेभ्यस्त्वा॒ विश्वे᳚भ्यस्त्वा दे॒वेभ्यो॒ विष्ण॑वुरुक्रमै॒ष ते॒ सोम॒स्तꣳ र॑क्षस्व॒ - [  ] \newline

\textbf{Pada Paata} \newline

उ॒प॒या॒मगृ॑हीत॒ इत्यु॑पया॒म - गृ॒ही॒तः॒ । अ॒सि॒ । वा॒क्ष॒सदिति॑ वाक्ष - सत् । अ॒सि॒ । वा॒क्पाभ्या॒मिति॑ वाक् - पाभ्या᳚म् । त्वा॒ । क्र॒तु॒पाभ्या॒मिति॑ क्रतु - पाभ्या᳚म् । अ॒स्य । य॒ज्ञ्स्य॑ । ध्रु॒वस्य॑ । अद्ध्य॑क्षाभ्या॒मित्यधि॑ - अ॒क्षा॒भ्या॒म् । गृ॒ह्णा॒मि॒ । उ॒प॒या॒मगृ॑हीत॒ इत्यु॑पया॒म - गृ॒ही॒तः॒ । अ॒सि॒ । ऋ॒त॒सदित्यृ॑त - सत् । अ॒सि॒ । च॒क्षु॒ष्पाभ्या॒मिति॑ चक्षुः - पाभ्या᳚म् । त्वा॒ । क्र॒तु॒पाभ्या॒मिति॑ क्रतु - पाभ्या᳚म् । अ॒स्य । य॒ज्ञ्स्य॑ । ध्रु॒वस्य॑ । अद्ध्य॑क्षाभ्या॒मित्यधि॑ - अ॒क्षा॒भ्या॒म् । गृ॒ह्णा॒मि॒ । उ॒प॒या॒मगृ॑हीत॒ इत्यु॑पया॒म - गृ॒ही॒तः॒ । अ॒सि॒ । श्रु॒त॒सदिति॑ श्रुत - सत् । अ॒सि॒ । श्रो॒त्र॒पाभ्या॒मिति॑ श्रोत्र - पाभ्या᳚म् । त्वा॒ । क्र॒तु॒पाभ्या॒मिति॑ क्रतु - पाभ्या᳚म् । अ॒स्य । य॒ज्ञ्स्य॑ । ध्रु॒वस्य॑ । अद्ध्य॑क्षाभ्या॒मित्यधि॑ - अ॒क्षा॒भ्या॒म् । गृ॒ह्णा॒मि॒ । दे॒वेभ्यः॑ । त्वा॒ । वि॒श्वदे॑वेभ्य॒ इति॑ वि॒श्व-दे॒वे॒भ्यः॒ । त्वा॒ । विश्वे᳚भ्यः । त्वा॒ । दे॒वेभ्यः॑ । विष्णो᳚ । उ॒रु॒क्र॒मेत्यु॑रु - क्र॒म॒ । ए॒षः । ते॒ । सोमः॑ । तम् । र॒क्ष॒स्व॒ ।  \newline


\textbf{Krama Paata} \newline

उ॒प॒या॒मगृ॑हीतो ऽसि । उ॒प॒या॒मगृ॑हीत॒ इत्यु॑पया॒म - गृ॒ही॒तः॒ । अ॒सि॒ वा॒क्ष॒सत् । वा॒क्ष॒सद॑सि । वा॒क्ष॒सदिति॑ वाक्ष - सत् । अ॒सि॒ वा॒क्पाभ्या᳚म् । वा॒क्पाभ्या᳚म् त्वा । वा॒क्पाभ्या॒मिति॑ वाक् - पाभ्या᳚म् । त्वा॒ क्र॒तु॒पाभ्या᳚म् । क्र॒तु॒पाभ्या॑म॒स्य । क्र॒तु॒पाभ्या॒मिति॑ क्रतु - पाभ्या᳚म् । अ॒स्य य॒ज्ञ्स्य॑ । य॒ज्ञ्स्य॑ ध्रु॒वस्य॑ । ध्रु॒वस्याद्ध्य॑क्षाभ्याम् । अद्ध्य॑क्षाभ्याम् गृह्णामि । अद्ध्य॑क्षाभ्या॒मित्यधि॑ - अ॒क्षा॒भ्या॒म् । गृ॒ह्णा॒म्यु॒प॒या॒मगृ॑हीतः । उ॒प॒या॒मगृ॑हीतो ऽसि । उ॒प॒या॒मगृ॑हीत॒ इत्यु॑पया॒म - गृ॒ही॒तः॒ । अ॒स्यृ॒त॒सत् । ऋ॒त॒सद॑सि । ऋ॒त॒सदित्यृ॑त - सत् । अ॒सि॒ च॒क्षु॒ष्पाभ्या᳚म् । च॒क्षु॒ष्पाभ्या᳚म् त्वा । च॒क्षु॒ष्पाभ्या॒मिति॑ चक्षुः - पाभ्या᳚म् । त्वा॒ क्र॒तु॒पाभ्या᳚म् । क्र॒तु॒पाभ्या॑म॒स्य । क्र॒तु॒पाभ्या॒मिति॑ क्रतु - पाभ्या᳚म् । अ॒स्य॒ य॒ज्ञ्स्य॑ । य॒ज्ञ्स्य॑ ध्रु॒वस्य॑ । ध्रु॒वस्याद्ध्य॑क्षाभ्याम् । अद्ध्य॑क्षाभ्याम् गृह्णामि । अद्ध्य॑क्षाभ्या॒मित्यधि॑ - अ॒क्षा॒भ्या॒म् । गृ॒ह्णा॒म्यु॒प॒या॒मगृ॑हीतः । उ॒प॒या॒मगृ॑हीतो ऽसि । उ॒प॒या॒मगृ॑हीत॒ इत्यु॑पया॒म - गृ॒ही॒तः॒ । अ॒सि॒ श्रु॒त॒सत् । श्रु॒त॒सद॑सि । श्रु॒त॒सदिति॑ श्रुत - सत् । अ॒सि॒ श्रो॒त्र॒पाभ्या᳚म् । श्रो॒त्र॒पाभ्या᳚म् त्वा । श्रो॒त्र॒पाभ्या॒मिति॑ श्रोत्र - पाभ्या᳚म् । त्वा॒ क्र॒तु॒पाभ्या᳚म् । क्र॒तु॒पाभ्या॑म॒स्य । क्र॒तु॒पाभ्या॒मिति॑ क्रतु - पाभ्या᳚म् ॥ ?अ॒स्य य॒ज्ञ्स्य॑ । य॒ज्ञ्स्य॑ ध्रु॒वस्य॑ । ध्रु॒वस्याद्ध्य॑क्षाभ्याम् । अद्ध्य॑क्षाभ्याम् गृह्णामि । अद्ध्य॑क्षाभ्या॒मित्यधि॑ - अ॒क्षा॒भ्या॒म् । गृ॒ह्णा॒मि॒ दे॒वेभ्यः॑ । दे॒वेभ्य॑स्त्वा । त्वा॒ वि॒श्वदे॑वेभ्यः । वि॒श्वदे॑वेभ्यस्त्वा । वि॒श्वदे॑वेभ्य॒ इति॑ वि॒श्व - दे॒वे॒भ्यः॒ । त्वा॒ विश्वे᳚भ्यः । विश्वे᳚भ्यस्त्वा । त्वा॒ दे॒वेभ्यः॑ । दे॒वेभ्यो॒ विष्णो᳚ । विष्ण॑वुरुक्रम । उ॒रु॒क्र॒मै॒षः । उ॒रु॒क्र॒मेत्यु॑रु - क्र॒म॒ । ए॒ष ते᳚ । ते॒ सोमः॑ । सोम॒स्तम् । तꣳ र॑क्षस्व । र॒क्ष॒स्व॒ तम् \newline

\textbf{Jatai Paata} \newline

1. उ॒प॒या॒मगृ॑हीतो ऽस्य स्युपया॒मगृ॑हीत उपया॒मगृ॑हीतो ऽसि । \newline
2. उ॒प॒या॒मगृ॑हीत॒ इत्यु॑पया॒म - गृ॒ही॒तः॒ । \newline
3. अ॒सि॒ वा॒क्ष॒सद् वा᳚क्ष॒स द॑स्यसि वाक्ष॒सत् । \newline
4. वा॒क्ष॒स द॑स्यसि वाक्ष॒सद् वा᳚क्ष॒स द॑सि । \newline
5. वा॒क्ष॒सदिति॑ वाक्ष - सत् । \newline
6. अ॒सि॒ वा॒क्पाभ्यां᳚ ॅवा॒क्पाभ्या॑ मस्यसि वा॒क्पाभ्या᳚म् । \newline
7. वा॒क्पाभ्या᳚म् त्वा त्वा वा॒क्पाभ्यां᳚ ॅवा॒क्पाभ्या᳚म् त्वा । \newline
8. वा॒क्पाभ्या॒मिति॑ वाक् - पाभ्या᳚म् । \newline
9. त्वा॒ क्र॒तु॒पाभ्या᳚म् क्रतु॒पाभ्या᳚म् त्वा त्वा क्रतु॒पाभ्या᳚म् । \newline
10. क्र॒तु॒पाभ्या॑ म॒स्यास्य क्र॑तु॒पाभ्या᳚म् क्रतु॒पाभ्या॑ म॒स्य । \newline
11. क्र॒तु॒पाभ्या॒मिति॑ क्रतु - पाभ्या᳚म् । \newline
12. अ॒स्य य॒ज्ञ्स्य॑ य॒ज्ञ्स्या॒ स्यास्य य॒ज्ञ्स्य॑ । \newline
13. य॒ज्ञ्स्य॑ ध्रु॒वस्य॑ ध्रु॒वस्य॑ य॒ज्ञ्स्य॑ य॒ज्ञ्स्य॑ ध्रु॒वस्य॑ । \newline
14. ध्रु॒वस्या द्ध्य॑क्षाभ्या॒ मद्ध्य॑क्षाभ्याम् ध्रु॒वस्य॑ ध्रु॒वस्या द्ध्य॑क्षाभ्याम् । \newline
15. अद्ध्य॑क्षाभ्याम् गृह्णामि गृह्णा॒ म्यद्ध्य॑क्षाभ्या॒ मद्ध्य॑क्षाभ्याम् गृह्णामि । \newline
16. अद्ध्य॑क्षाभ्या॒मित्यधि॑ - अ॒क्षा॒भ्या॒म् । \newline
17. गृ॒ह्णा॒ म्यु॒प॒या॒मगृ॑हीत उपया॒मगृ॑हीतो गृह्णामि गृह्णा म्युपया॒मगृ॑हीतः । \newline
18. उ॒प॒या॒मगृ॑हीतो ऽस्य स्युपया॒मगृ॑हीत उपया॒मगृ॑हीतो ऽसि । \newline
19. उ॒प॒या॒मगृ॑हीत॒ इत्यु॑पया॒म - गृ॒ही॒तः॒ । \newline
20. अ॒स्यृ॒त॒स दृ॑त॒स द॑स्य स्यृत॒सत् । \newline
21. ऋ॒त॒स द॑स्य स्यृत॒स दृ॑त॒स द॑सि । \newline
22. ऋ॒त॒सदित्यृ॑त - सत् । \newline
23. अ॒सि॒ च॒क्षु॒ष्पाभ्या᳚म् चक्षु॒ष्पाभ्या॑ मस्यसि चक्षु॒ष्पाभ्या᳚म् । \newline
24. च॒क्षु॒ष्पाभ्या᳚म् त्वा त्वा चक्षु॒ष्पाभ्या᳚म् चक्षु॒ष्पाभ्या᳚म् त्वा । \newline
25. च॒क्षु॒ष्पाभ्या॒मिति॑ चक्षुः - पाभ्या᳚म् । \newline
26. त्वा॒ क्र॒तु॒पाभ्या᳚म् क्रतु॒पाभ्या᳚म् त्वा त्वा क्रतु॒पाभ्या᳚म् । \newline
27. क्र॒तु॒पाभ्या॑ म॒स्यास्य क्र॑तु॒पाभ्या᳚म् क्रतु॒पाभ्या॑ म॒स्य । \newline
28. क्र॒तु॒पाभ्या॒मिति॑ क्रतु - पाभ्या᳚म् । \newline
29. अ॒स्य य॒ज्ञ्स्य॑ य॒ज्ञ्स्या॒ स्यास्य य॒ज्ञ्स्य॑ । \newline
30. य॒ज्ञ्स्य॑ ध्रु॒वस्य॑ ध्रु॒वस्य॑ य॒ज्ञ्स्य॑ य॒ज्ञ्स्य॑ ध्रु॒वस्य॑ । \newline
31. ध्रु॒वस्या द्ध्य॑क्षाभ्या॒ मद्ध्य॑क्षाभ्याम् ध्रु॒वस्य॑ ध्रु॒वस्या द्ध्य॑क्षाभ्याम् । \newline
32. अद्ध्य॑क्षाभ्याम् गृह्णामि गृह्णा॒ म्यद्ध्य॑क्षाभ्या॒ मद्ध्य॑क्षाभ्याम् गृह्णामि । \newline
33. अद्ध्य॑क्षाभ्या॒मित्यधि॑ - अ॒क्षा॒भ्या॒म् । \newline
34. गृ॒ह्णा॒ म्यु॒प॒या॒मगृ॑हीत उपया॒मगृ॑हीतो गृह्णामि गृह्णा म्युपया॒मगृ॑हीतः । \newline
35. उ॒प॒या॒मगृ॑हीतो ऽस्य स्युपया॒मगृ॑हीत उपया॒मगृ॑हीतो ऽसि । \newline
36. उ॒प॒या॒मगृ॑हीत॒ इत्यु॑पया॒म - गृ॒ही॒तः॒ । \newline
37. अ॒सि॒ श्रु॒त॒स च्छ्रु॑त॒स द॑स्यसि श्रुत॒सत् । \newline
38. श्रु॒त॒स द॑स्यसि श्रुत॒स च्छ्रु॑त॒स द॑सि । \newline
39. श्रु॒त॒सदिति॑ श्रुत - सत् । \newline
40. अ॒सि॒ श्रो॒त्र॒पाभ्याꣳ॑ श्रोत्र॒पाभ्या॑ मस्यसि श्रोत्र॒पाभ्या᳚म् । \newline
41. श्रो॒त्र॒पाभ्या᳚म् त्वा त्वा श्रोत्र॒पाभ्याꣳ॑ श्रोत्र॒पाभ्या᳚म् त्वा । \newline
42. श्रो॒त्र॒पाभ्या॒मिति॑ श्रोत्र - पाभ्या᳚म् । \newline
43. त्वा॒ क्र॒तु॒पाभ्या᳚म् क्रतु॒पाभ्या᳚म् त्वा त्वा क्रतु॒पाभ्या᳚म् । \newline
44. क्र॒तु॒पाभ्या॑ म॒स्यास्य क्र॑तु॒पाभ्या᳚म् क्रतु॒पाभ्या॑ म॒स्य । \newline
45. क्र॒तु॒पाभ्या॒मिति॑ क्रतु - पाभ्या᳚म् । \newline
46. अ॒स्य य॒ज्ञ्स्य॑ य॒ज्ञ्स्या॒ स्यास्य य॒ज्ञ्स्य॑ । \newline
47. य॒ज्ञ्स्य॑ ध्रु॒वस्य॑ ध्रु॒वस्य॑ य॒ज्ञ्स्य॑ य॒ज्ञ्स्य॑ ध्रु॒वस्य॑ । \newline
48. ध्रु॒वस्या द्ध्य॑क्षाभ्या॒ मद्ध्य॑क्षाभ्याम् ध्रु॒वस्य॑ ध्रु॒वस्या द्ध्य॑क्षाभ्याम् । \newline
49. अद्ध्य॑क्षाभ्याम् गृह्णामि गृह्णा॒ म्यद्ध्य॑क्षाभ्या॒ मद्ध्य॑क्षाभ्याम् गृह्णामि । \newline
50. अद्ध्य॑क्षाभ्या॒मित्यधि॑ - अ॒क्षा॒भ्या॒म् । \newline
51. गृ॒ह्णा॒मि॒ दे॒वेभ्यो॑ दे॒वेभ्यो॑ गृह्णामि गृह्णामि दे॒वेभ्यः॑ । \newline
52. दे॒वेभ्य॑ स्त्वा त्वा दे॒वेभ्यो॑ दे॒वेभ्य॑ स्त्वा । \newline
53. त्वा॒ वि॒श्वदे॑वेभ्यो वि॒श्वदे॑वेभ्य स्त्वा त्वा वि॒श्वदे॑वेभ्यः । \newline
54. वि॒श्वदे॑वेभ्य स्त्वा त्वा वि॒श्वदे॑वेभ्यो वि॒श्वदे॑वेभ्य स्त्वा । \newline
55. वि॒श्वदे॑वेभ्य॒ इति॑ वि॒श्व - दे॒वे॒भ्यः॒ । \newline
56. त्वा॒ विश्वे᳚भ्यो॒ विश्वे᳚भ्य स्त्वा त्वा॒ विश्वे᳚भ्यः । \newline
57. विश्वे᳚भ्य स्त्वा त्वा॒ विश्वे᳚भ्यो॒ विश्वे᳚भ्य स्त्वा । \newline
58. त्वा॒ दे॒वेभ्यो॑ दे॒वेभ्य॑ स्त्वा त्वा दे॒वेभ्यः॑ । \newline
59. दे॒वेभ्यो॒ विष्णो॒ विष्णो॑ दे॒वेभ्यो॑ दे॒वेभ्यो॒ विष्णो᳚ । \newline
60. विष्ण॑ वुरुक्रमो रुक्रम॒ विष्णो॒ विष्ण॑ वुरुक्रम । \newline
61. उ॒रु॒क्र॒ मै॒ष ए॒ष उ॑रुक्रमो रुक्रमै॒षः । \newline
62. उ॒रु॒क्र॒मेत्यु॑रु - क्र॒म॒ । \newline
63. ए॒ष ते॑ त ए॒ष ए॒ष ते᳚ । \newline
64. ते॒ सोमः॒ सोम॑ स्ते ते॒ सोमः॑ । \newline
65. सोम॒ स्तम् तꣳ सोमः॒ सोम॒ स्तम् । \newline
66. तꣳ र॑क्षस्व रक्षस्व॒ तम् तꣳ र॑क्षस्व । \newline
67. र॒क्ष॒स्व॒ तम् तꣳ र॑क्षस्व रक्षस्व॒ तम् । \newline

\textbf{Ghana Paata } \newline

1. उ॒प॒या॒मगृ॑हीतो ऽस्य स्युपया॒मगृ॑हीत उपया॒मगृ॑हीतो ऽसि वाक्ष॒सद् वा᳚क्ष॒स द॑स्युपया॒मगृ॑हीत उपया॒मगृ॑हीतो ऽसि वाक्ष॒सत् । \newline
2. उ॒प॒या॒मगृ॑हीत॒ इत्यु॑पया॒म - गृ॒ही॒तः॒ । \newline
3. अ॒सि॒ वा॒क्ष॒सद् वा᳚क्ष॒स द॑स्यसि वाक्ष॒स द॑स्यसि वाक्ष॒स द॑स्यसि वाक्ष॒स द॑सि । \newline
4. वा॒क्ष॒स द॑स्यसि वाक्ष॒सद् वा᳚क्ष॒स द॑सि वा॒क्पाभ्यां᳚ ॅवा॒क्पाभ्या॑ मसि वाक्ष॒सद् वा᳚क्ष॒स द॑सि वा॒क्पाभ्या᳚म् । \newline
5. वा॒क्ष॒सदिति॑ वाक्ष - सत् । \newline
6. अ॒सि॒ वा॒क्पाभ्यां᳚ ॅवा॒क्पाभ्या॑ मस्यसि वा॒क्पाभ्या᳚म् त्वा त्वा वा॒क्पाभ्या॑ मस्यसि वा॒क्पाभ्या᳚म् त्वा । \newline
7. वा॒क्पाभ्या᳚म् त्वा त्वा वा॒क्पाभ्यां᳚ ॅवा॒क्पाभ्या᳚म् त्वा क्रतु॒पाभ्या᳚म् क्रतु॒पाभ्या᳚म् त्वा वा॒क्पाभ्यां᳚ ॅवा॒क्पाभ्या᳚म् त्वा क्रतु॒पाभ्या᳚म् । \newline
8. वा॒क्पाभ्या॒मिति॑ वाक् - पाभ्या᳚म् । \newline
9. त्वा॒ क्र॒तु॒पाभ्या᳚म् क्रतु॒पाभ्या᳚म् त्वा त्वा क्रतु॒पाभ्या॑ म॒स्यास्य क्र॑तु॒पाभ्या᳚म् त्वा त्वा क्रतु॒पाभ्या॑ म॒स्य । \newline
10. क्र॒तु॒पाभ्या॑ म॒स्यास्य क्र॑तु॒पाभ्या᳚म् क्रतु॒पाभ्या॑ म॒स्य य॒ज्ञ्स्य॑ य॒ज्ञ्स्या॒स्य क्र॑तु॒पाभ्या᳚म् क्रतु॒पाभ्या॑ म॒स्य य॒ज्ञ्स्य॑ । \newline
11. क्र॒तु॒पाभ्या॒मिति॑ क्रतु - पाभ्या᳚म् । \newline
12. अ॒स्य य॒ज्ञ्स्य॑ य॒ज्ञ्स्या॒ स्यास्य य॒ज्ञ्स्य॑ ध्रु॒वस्य॑ ध्रु॒वस्य॑ य॒ज्ञ्स्या॒ स्यास्य य॒ज्ञ्स्य॑ ध्रु॒वस्य॑ । \newline
13. य॒ज्ञ्स्य॑ ध्रु॒वस्य॑ ध्रु॒वस्य॑ य॒ज्ञ्स्य॑ य॒ज्ञ्स्य॑ ध्रु॒वस्या द्ध्य॑क्षाभ्या॒ मद्ध्य॑क्षाभ्याम् ध्रु॒वस्य॑ य॒ज्ञ्स्य॑ य॒ज्ञ्स्य॑ ध्रु॒वस्या द्ध्य॑क्षाभ्याम् । \newline
14. ध्रु॒वस्या द्ध्य॑क्षाभ्या॒ मद्ध्य॑क्षाभ्याम् ध्रु॒वस्य॑ ध्रु॒वस्या द्ध्य॑क्षाभ्याम् गृह्णामि गृह्णा॒ म्यद्ध्य॑क्षाभ्याम् ध्रु॒वस्य॑ ध्रु॒वस्या द्ध्य॑क्षाभ्याम् गृह्णामि । \newline
15. अद्ध्य॑क्षाभ्याम् गृह्णामि गृह्णा॒ म्यद्ध्य॑क्षाभ्या॒ मद्ध्य॑क्षाभ्याम् गृह्णा म्युपया॒मगृ॑हीत उपया॒मगृ॑हीतो गृह्णा॒ म्यद्ध्य॑क्षाभ्या॒ मद्ध्य॑क्षाभ्याम् गृह्णा म्युपया॒मगृ॑हीतः । \newline
16. अद्ध्य॑क्षाभ्या॒मित्यधि॑ - अ॒क्षा॒भ्या॒म् । \newline
17. गृ॒ह्णा॒ म्यु॒प॒या॒मगृ॑हीत उपया॒मगृ॑हीतो गृह्णामि गृह्णा म्युपया॒मगृ॑हीतो ऽस्य स्युपया॒मगृ॑हीतो गृह्णामि गृह्णा म्युपया॒मगृ॑हीतो ऽसि । \newline
18. उ॒प॒या॒मगृ॑हीतो ऽस्य स्युपया॒मगृ॑हीत उपया॒मगृ॑हीतो ऽस्यृत॒स दृ॑त॒सद॑ स्युपया॒मगृ॑हीत उपया॒मगृ॑हीतो ऽस्यृत॒सत् । \newline
19. उ॒प॒या॒मगृ॑हीत॒ इत्यु॑पया॒म - गृ॒ही॒तः॒ । \newline
20. अ॒स्यृ॒त॒स दृ॑त॒स द॑स्य स्यृत॒स द॑स्य स्यृत॒स द॑स्य स्यृत॒स द॑सि । \newline
21. ऋ॒त॒स द॑स्य स्यृत॒स दृ॑त॒स द॑सि चक्षु॒ष्पाभ्या᳚म् चक्षु॒ष्पाभ्या॑ मस्यृत॒स दृ॑त॒सद॑सि चक्षु॒ष्पाभ्या᳚म् । \newline
22. ऋ॒त॒सदित्यृ॑त - सत् । \newline
23. अ॒सि॒ च॒क्षु॒ष्पाभ्या᳚म् चक्षु॒ष्पाभ्या॑ मस्यसि चक्षु॒ष्पाभ्या᳚म् त्वा त्वा चक्षु॒ष्पाभ्या॑ मस्यसि चक्षु॒ष्पाभ्या᳚म् त्वा । \newline
24. च॒क्षु॒ष्पाभ्या᳚म् त्वा त्वा चक्षु॒ष्पाभ्या᳚म् चक्षु॒ष्पाभ्या᳚म् त्वा क्रतु॒पाभ्या᳚म् क्रतु॒पाभ्या᳚म् त्वा चक्षु॒ष्पाभ्या᳚म् चक्षु॒ष्पाभ्या᳚म् त्वा क्रतु॒पाभ्या᳚म् । \newline
25. च॒क्षु॒ष्पाभ्या॒मिति॑ चक्षुः - पाभ्या᳚म् । \newline
26. त्वा॒ क्र॒तु॒पाभ्या᳚म् क्रतु॒पाभ्या᳚म् त्वा त्वा क्रतु॒पाभ्या॑ म॒स्यास्य क्र॑तु॒पाभ्या᳚म् त्वा त्वा क्रतु॒पाभ्या॑ म॒स्य । \newline
27. क्र॒तु॒पाभ्या॑ म॒स्यास्य क्र॑तु॒पाभ्या᳚म् क्रतु॒पाभ्या॑ म॒स्य य॒ज्ञ्स्य॑ य॒ज्ञ्स्या॒स्य क्र॑तु॒पाभ्या᳚म् क्रतु॒पाभ्या॑ म॒स्य य॒ज्ञ्स्य॑ । \newline
28. क्र॒तु॒पाभ्या॒मिति॑ क्रतु - पाभ्या᳚म् । \newline
29. अ॒स्य य॒ज्ञ्स्य॑ य॒ज्ञ्स्या॒ स्यास्य य॒ज्ञ्स्य॑ ध्रु॒वस्य॑ ध्रु॒वस्य॑ य॒ज्ञ्स्या॒ स्यास्य य॒ज्ञ्स्य॑ ध्रु॒वस्य॑ । \newline
30. य॒ज्ञ्स्य॑ ध्रु॒वस्य॑ ध्रु॒वस्य॑ य॒ज्ञ्स्य॑ य॒ज्ञ्स्य॑ ध्रु॒वस्या द्ध्य॑क्षाभ्या॒ मद्ध्य॑क्षाभ्याम् ध्रु॒वस्य॑ य॒ज्ञ्स्य॑ य॒ज्ञ्स्य॑ ध्रु॒वस्या द्ध्य॑क्षाभ्याम् । \newline
31. ध्रु॒वस्या द्ध्य॑क्षाभ्या॒ मद्ध्य॑क्षाभ्याम् ध्रु॒वस्य॑ ध्रु॒वस्या द्ध्य॑क्षाभ्याम् गृह्णामि गृह्णा॒ म्यद्ध्य॑क्षाभ्याम् ध्रु॒वस्य॑ ध्रु॒वस्या द्ध्य॑क्षाभ्याम् गृह्णामि । \newline
32. अद्ध्य॑क्षाभ्याम् गृह्णामि गृह्णा॒ म्यद्ध्य॑क्षाभ्या॒ मद्ध्य॑क्षाभ्याम् गृह्णा म्युपया॒मगृ॑हीत उपया॒मगृ॑हीतो गृह्णा॒ म्यद्ध्य॑क्षाभ्या॒ मद्ध्य॑क्षाभ्याम् गृह्णा म्युपया॒मगृ॑हीतः । \newline
33. अद्ध्य॑क्षाभ्या॒मित्यधि॑ - अ॒क्षा॒भ्या॒म् । \newline
34. गृ॒ह्णा॒ म्यु॒प॒या॒मगृ॑हीत उपया॒मगृ॑हीतो गृह्णामि गृह्णा म्युपया॒मगृ॑हीतो ऽस्य स्युपया॒मगृ॑हीतो गृह्णामि गृह्णा म्युपया॒मगृ॑हीतो ऽसि । \newline
35. उ॒प॒या॒मगृ॑हीतो ऽस्य स्युपया॒मगृ॑हीत उपया॒मगृ॑हीतो ऽसि श्रुत॒स च्छ्रु॑त॒स द॑स्युपया॒मगृ॑हीत उपया॒मगृ॑हीतो ऽसि श्रुत॒सत् । \newline
36. उ॒प॒या॒मगृ॑हीत॒ इत्यु॑पया॒म - गृ॒ही॒तः॒ । \newline
37. अ॒सि॒ श्रु॒त॒स च्छ्रु॑त॒स द॑स्यसि श्रुत॒स द॑स्यसि श्रुत॒स द॑स्यसि श्रुत॒सद॑सि । \newline
38. श्रु॒त॒स द॑स्यसि श्रुत॒स च्छ्रु॑त॒सद॑सि श्रोत्र॒पाभ्याꣳ॑ श्रोत्र॒पाभ्या॑ मसि श्रुत॒स च्छ्रु॑त॒सद॑सि श्रोत्र॒पाभ्या᳚म् । \newline
39. श्रु॒त॒सदिति॑ श्रुत - सत् । \newline
40. अ॒सि॒ श्रो॒त्र॒पाभ्याꣳ॑ श्रोत्र॒पाभ्या॑ मस्यसि श्रोत्र॒पाभ्या᳚म् त्वा त्वा श्रोत्र॒पाभ्या॑ मस्यसि श्रोत्र॒पाभ्या᳚म् त्वा । \newline
41. श्रो॒त्र॒पाभ्या᳚म् त्वा त्वा श्रोत्र॒पाभ्याꣳ॑ श्रोत्र॒पाभ्या᳚म् त्वा क्रतु॒पाभ्या᳚म् क्रतु॒पाभ्या᳚म् त्वा श्रोत्र॒पाभ्याꣳ॑ श्रोत्र॒पाभ्या᳚म् त्वा क्रतु॒पाभ्या᳚म् । \newline
42. श्रो॒त्र॒पाभ्या॒मिति॑ श्रोत्र - पाभ्या᳚म् । \newline
43. त्वा॒ क्र॒तु॒पाभ्या᳚म् क्रतु॒पाभ्या᳚म् त्वा त्वा क्रतु॒पाभ्या॑ म॒स्यास्य क्र॑तु॒पाभ्या᳚म् त्वा त्वा क्रतु॒पाभ्या॑ म॒स्य । \newline
44. क्र॒तु॒पाभ्या॑ म॒स्यास्य क्र॑तु॒पाभ्या᳚म् क्रतु॒पाभ्या॑ म॒स्य य॒ज्ञ्स्य॑ य॒ज्ञ्स्या॒स्य क्र॑तु॒पाभ्या᳚म् क्रतु॒पाभ्या॑ म॒स्य य॒ज्ञ्स्य॑ । \newline
45. क्र॒तु॒पाभ्या॒मिति॑ क्रतु - पाभ्या᳚म् । \newline
46. अ॒स्य य॒ज्ञ्स्य॑ य॒ज्ञ्स्या॒ स्यास्य य॒ज्ञ्स्य॑ ध्रु॒वस्य॑ ध्रु॒वस्य॑ य॒ज्ञ्स्या॒ स्यास्य य॒ज्ञ्स्य॑ ध्रु॒वस्य॑ । \newline
47. य॒ज्ञ्स्य॑ ध्रु॒वस्य॑ ध्रु॒वस्य॑ य॒ज्ञ्स्य॑ य॒ज्ञ्स्य॑ ध्रु॒वस्या द्ध्य॑क्षाभ्या॒ मद्ध्य॑क्षाभ्याम् ध्रु॒वस्य॑ य॒ज्ञ्स्य॑ य॒ज्ञ्स्य॑ ध्रु॒वस्या द्ध्य॑क्षाभ्याम् । \newline
48. ध्रु॒वस्या द्ध्य॑क्षाभ्या॒ मद्ध्य॑क्षाभ्याम् ध्रु॒वस्य॑ ध्रु॒वस्या द्ध्य॑क्षाभ्याम् गृह्णामि गृह्णा॒ म्यद्ध्य॑क्षाभ्याम् ध्रु॒वस्य॑ ध्रु॒वस्या द्ध्य॑क्षाभ्याम् गृह्णामि । \newline
49. अद्ध्य॑क्षाभ्याम् गृह्णामि गृह्णा॒ म्यद्ध्य॑क्षाभ्या॒ मद्ध्य॑क्षाभ्याम् गृह्णामि दे॒वेभ्यो॑ दे॒वेभ्यो॑ गृह्णा॒ म्यद्ध्य॑क्षाभ्या॒ मद्ध्य॑क्षाभ्याम् गृह्णामि दे॒वेभ्यः॑ । \newline
50. अद्ध्य॑क्षाभ्या॒मित्यधि॑ - अ॒क्षा॒भ्या॒म् । \newline
51. गृ॒ह्णा॒मि॒ दे॒वेभ्यो॑ दे॒वेभ्यो॑ गृह्णामि गृह्णामि दे॒वेभ्य॑ स्त्वा त्वा दे॒वेभ्यो॑ गृह्णामि गृह्णामि दे॒वेभ्य॑ स्त्वा । \newline
52. दे॒वेभ्य॑ स्त्वा त्वा दे॒वेभ्यो॑ दे॒वेभ्य॑ स्त्वा वि॒श्वदे॑वेभ्यो वि॒श्वदे॑वेभ्य स्त्वा दे॒वेभ्यो॑ दे॒वेभ्य॑ स्त्वा वि॒श्वदे॑वेभ्यः । \newline
53. त्वा॒ वि॒श्वदे॑वेभ्यो वि॒श्वदे॑वेभ्य स्त्वा त्वा वि॒श्वदे॑वेभ्य स्त्वा त्वा वि॒श्वदे॑वेभ्य स्त्वा त्वा वि॒श्वदे॑वेभ्य स्त्वा । \newline
54. वि॒श्वदे॑वेभ्य स्त्वा त्वा वि॒श्वदे॑वेभ्यो वि॒श्वदे॑वेभ्य स्त्वा॒ विश्वे᳚भ्यो॒ विश्वे᳚भ्य स्त्वा वि॒श्वदे॑वेभ्यो वि॒श्वदे॑वेभ्य स्त्वा॒ विश्वे᳚भ्यः । \newline
55. वि॒श्वदे॑वेभ्य॒ इति॑ वि॒श्व - दे॒वे॒भ्यः॒ । \newline
56. त्वा॒ विश्वे᳚भ्यो॒ विश्वे᳚भ्य स्त्वा त्वा॒ विश्वे᳚भ्य स्त्वा त्वा॒ विश्वे᳚भ्य स्त्वा त्वा॒ विश्वे᳚भ्य स्त्वा । \newline
57. विश्वे᳚भ्य स्त्वा त्वा॒ विश्वे᳚भ्यो॒ विश्वे᳚भ्य स्त्वा दे॒वेभ्यो॑ दे॒वेभ्य॑ स्त्वा॒ विश्वे᳚भ्यो॒ विश्वे᳚भ्य स्त्वा दे॒वेभ्यः॑ । \newline
58. त्वा॒ दे॒वेभ्यो॑ दे॒वेभ्य॑ स्त्वा त्वा दे॒वेभ्यो॒ विष्णो॒ विष्णो॑ दे॒वेभ्य॑ स्त्वा त्वा दे॒वेभ्यो॒ विष्णो᳚ । \newline
59. दे॒वेभ्यो॒ विष्णो॒ विष्णो॑ दे॒वेभ्यो॑ दे॒वेभ्यो॒ विष्ण॑ वुरुक्रमो रुक्रम॒ विष्णो॑ दे॒वेभ्यो॑ दे॒वेभ्यो॒ 
विष्ण॑ वुरुक्रम । \newline
60. विष्ण॑ वुरुक्रमो रुक्रम॒ विष्णो॒ विष्ण॑ वुरुक्र मै॒ष ए॒ष उ॑रुक्रम॒ विष्णो॒ विष्ण॑ वुरुक्र मै॒षः । \newline
61. उ॒रु॒क्र॒मै॒ष ए॒ष उ॑रुक्रमो रुक्रमै॒ष ते॑ त ए॒ष उ॑रुक्रमो रुक्रमै॒ष ते᳚ । \newline
62. उ॒रु॒क्र॒मेत्यु॑रु - क्र॒म॒ । \newline
63. ए॒ष ते॑ त ए॒ष ए॒ष ते॒ सोमः॒ सोम॑ स्त ए॒ष ए॒ष ते॒ सोमः॑ । \newline
64. ते॒ सोमः॒ सोम॑ स्ते ते॒ सोम॒ स्तम् तꣳ सोम॑ स्ते ते॒ सोम॒ स्तम् । \newline
65. सोम॒ स्तम् तꣳ सोमः॒ सोम॒ स्तꣳ र॑क्षस्व रक्षस्व॒ तꣳ सोमः॒ सोम॒ स्तꣳ र॑क्षस्व । \newline
66. तꣳ र॑क्षस्व रक्षस्व॒ तम् तꣳ र॑क्षस्व॒ तम् तꣳ र॑क्षस्व॒ तम् तꣳ र॑क्षस्व॒ तम् । \newline
67. र॒क्ष॒स्व॒ तम् तꣳ र॑क्षस्व रक्षस्व॒ तम् ते॑ ते॒ तꣳ र॑क्षस्व रक्षस्व॒ तम् ते᳚ । \newline
\pagebreak
\markright{ TS 3.2.10.2  \hfill https://www.vedavms.in \hfill}

\section{ TS 3.2.10.2 }

\textbf{TS 3.2.10.2 } \newline
\textbf{Samhita Paata} \newline

तं ते॑ दु॒श्चक्षा॒ माऽव॑ ख्य॒न्मयि॒ वसुः॑ पुरो॒वसु॑र्वा॒क्पा वाचं॑ मे पाहि॒ मयि॒ वसु॑र्वि॒दद्व॑सुश्चक्षु॒ष्पाश्चक्षु॑र् मे पाहि॒ मयि॒ वसुः॑ सं॒ॅयद्व॑सुः श्रोत्र॒पाः श्रोत्रं॑ मे पाहि॒ भूर॑सि॒ श्रेष्ठो॑ रश्मी॒नां प्रा॑ण॒पाः प्रा॒णं मे॑ पाहि॒ धूर॑सि॒ श्रेष्ठो॑ रश्मी॒नाम॑पान॒पा अ॑पा॒नं मे॑ पाहि॒ यो न॑ इन्द्रवायू मित्रावरुणा-वश्विनावभि॒दास॑ति॒ भ्रातृ॑व्य ( ) उ॒त्पिपी॑ते शुभस्पती इ॒दम॒हं तमध॑रं पादयामि॒ यथे᳚न्द्रा॒ऽहमु॑त्त॒मश्चे॒तया॑नि ॥ \newline

\textbf{Pada Paata} \newline

तम् । ते॒ । दु॒श्चक्षा॒ इति॑ दुः - चक्षाः᳚ । मा । अवेति॑ । ख्य॒त् । मयि॑ । वसुः॑ । पु॒रो॒वसु॒रिति॑ पुरः - वसुः॑ । वा॒क्पा इति॑ वाक् - पाः । वाच᳚म् । मे॒ । पा॒हि॒ । मयि॑ । वसुः॑ । वि॒दद्व॑सु॒रिति॑ वि॒दत् - व॒सुः॒ । च॒क्षु॒ष्पा इति॑ चक्षुः - पाः । चक्षुः॑ । मे॒ । पा॒हि॒ । मयि॑ । वसुः॑ । सं॒ॅयद् व॑सु॒रिति॑ सं॒ॅयत् - व॒सुः॒ । श्रो॒त्र॒पा इति॑ श्रोत्र-पाः । श्रोत्र᳚म् । मे॒ । पा॒हि॒ । भूः । अ॒सि॒ । श्रेष्ठः॑ । र॒श्मी॒नाम् । प्रा॒ण॒पा इति॑ प्राण - पाः । प्रा॒णमिति॑ प्र- अ॒नम् । मे॒ । पा॒हि॒ । धूः । अ॒सि॒ । श्रेष्ठः॑ । र॒श्मी॒नाम् । अ॒पा॒न॒पा इत्य॑पान - पाः । अ॒पा॒नमित्य॑प - अ॒नम् । मे॒ । पा॒हि॒ । यः । नः॒ । इ॒न्द्र॒वा॒यू॒ इती᳚न्द्र - वा॒यू॒ । मि॒त्रा॒व॒रु॒णा॒विति॑ मित्रा - व॒रु॒णौ॒ । अ॒श्वि॒नौ॒ । अ॒भि॒दास॒तीत्य॑भि - दास॑ति । भ्रातृ॑व्यः ( ) । उ॒त्पिपी॑त॒ इत्यु॑त् - पिपी॑ते । शु॒भः॒ । प॒ती॒ इति॑ । इ॒दम् । अ॒हम् । तम् । अध॑रम् । पा॒द॒या॒मि॒ । यथा᳚ । इ॒न्द्र॒ । अ॒हम् । उ॒त्त॒म इत्यु॑त् - त॒मः । चे॒तया॑नि ॥  \newline


\textbf{Krama Paata} \newline

तम् ते᳚ । ते॒ दु॒श्चक्षाः᳚ । दु॒श्चक्षा॒ मा । दु॒श्चक्षा॒ इति॑ दुः - चक्षाः᳚ । मा ऽव॑ । अव॑ ख्यत् । ख्य॒न् मयि॑ । मयि॒ वसुः॑ । वसुः॑ पुरो॒वसुः॑ । पु॒रो॒वसु॑र् वा॒क्पाः । पु॒रो॒वसु॒रिति॑ पुरः - वसुः॑ । वा॒क्पा वाच᳚म् । वा॒क्पा इति॑ वाक् - पाः । वाच॑म् मे । मे॒ पा॒हि॒ । पा॒हि॒ मयि॑ । मयि॒ वसुः॑ । वसु॑र् वि॒दद्व॑सुः । वि॒दद्व॑सु श्चक्षु॒ष्पाः । वि॒दद्व॑सु॒रिति॑ वि॒दत् - व॒सुः॒ । च॒क्षु॒ष्पा श्चक्षुः॑ । च॒क्षु॒ष्पा इति॑ चक्षुः - पाः । चक्षु॑र् मे । मे॒ पा॒हि॒ । पा॒हि॒ मयि॑ । मयि॒ वसुः॑ । वसुः॑ स॒म्ॅयद्व॑सुः । स॒म्ॅयद्व॑सुः श्रोत्र॒पाः । स॒म्ॅयद्व॑सु॒रिति॑ स॒म्ॅयत् - व॒सुः॒ । श्रो॒त्र॒पाः श्रोत्र᳚म् । श्रो॒त्र॒पा इति॑ श्रोत्र - पाः । श्रोत्रं॑ मे । मे॒ पा॒हि॒ । पा॒हि॒ भूः । भूर॑सि । अ॒सि॒ श्रेष्ठः॑ । श्रेष्ठो॑ रश्मी॒नाम् । र॒श्मी॒नाम् प्रा॑ण॒पाः । प्रा॒ण॒पाः प्रा॒णम् । प्रा॒ण॒पा इति॑ प्राण - पाः । प्रा॒णम् मे᳚ । प्रा॒णमिति॑ प्र - अ॒नम् । मे॒ पा॒हि॒ । पा॒हि॒ धूः । धूर॑सि । अ॒सि॒ श्रेष्ठः॑ । श्रेष्ठो॑ रश्मी॒नाम् । र॒श्मि॒नाम॑पान॒पाः । अ॒पा॒न॒पा अ॑पा॒नम् । अ॒पा॒न॒पा इत्य॑पान - पाः । अ॒पा॒नम् मे᳚ । अ॒पा॒नमित्य॑प - अ॒नम् । मे॒ पा॒हि॒ । पा॒हि॒ यः । यो नः॑ । न॒ इ॒न्द्र॒वा॒यू॒ । इ॒न्द्र॒वा॒यू॒ मि॒त्रा॒व॒रु॒णौ॒ । इ॒न्द्र॒॒वा॒यू॒ इती᳚न्द्र - वा॒यू॒ । मि॒त्रा॒व॒रु॒णा॒व॒श्वि॒नौ॒ । मि॒त्रा॒व॒रु॒णा॒विति॑ मित्रा - व॒रु॒णौ॒ । अ॒श्वि॒ना॒व॒भि॒दास॑ति । अ॒भि॒दास॑ति॒ भ्रातृ॑व्यः ( ) । अ॒भि॒दास॒तीत्य॑भि - दास॑ति । भ्रातृ॑व्य उ॒त्पिपी॑ते । उ॒त्पिपी॑ते शुभः । उ॒त्पिपी॑त॒ इत्यु॑त् - पिपी॑ते । शु॒भ॒स्प॒ती॒ । प॒ती॒ इ॒दम् । प॒ती॒ इति॑ पती । इ॒दम॒हम् । अ॒हम् तम् । तमध॑रम् । अध॑रम् पादयामि । पा॒द॒या॒मि॒ यथा᳚ । यथे᳚न्द्र । इ॒न्द्रा॒हम् । अ॒हमु॑त्त॒मः । उ॒त्त॒मश्चे॒तया॑नि । उ॒त्त॒म इत्यु॑त् - त॒मः । चे॒तया॒नीति॑ चे॒तया॑नि । \newline

\textbf{Jatai Paata} \newline

1. तम् ते॑ ते॒ तम् तम् ते᳚ । \newline
2. ते॒ दु॒श्चक्षा॑ दु॒श्चक्षा᳚ स्ते ते दु॒श्चक्षाः᳚ । \newline
3. दु॒श्चक्षा॒ मा मा दु॒श्चक्षा॑ दु॒श्चक्षा॒ मा । \newline
4. दु॒श्चक्षा॒ इति॑ दुः - चक्षाः᳚ । \newline
5. मा ऽवाव॒ मा मा ऽव॑ । \newline
6. अव॑ ख्यत् ख्य॒ दवाव॑ ख्यत् । \newline
7. ख्य॒न् मयि॒ मयि॑ ख्यत् ख्य॒न् मयि॑ । \newline
8. मयि॒ वसु॒र् वसु॒र् मयि॒ मयि॒ वसुः॑ । \newline
9. वसुः॑ पुरो॒वसुः॑ पुरो॒वसु॒र् वसु॒र् वसुः॑ पुरो॒वसुः॑ । \newline
10. पु॒रो॒वसु॑र् वा॒क्पा वा॒क्पाः पु॑रो॒वसुः॑ पुरो॒वसु॑र् वा॒क्पाः । \newline
11. पु॒रो॒वसु॒रिति॑ पुरः - वसुः॑ । \newline
12. वा॒क्पा वाचं॒ ॅवाचं॑ ॅवा॒क्पा वा॒क्पा वाच᳚म् । \newline
13. वा॒क्पा इति॑ वाक् - पाः । \newline
14. वाच॑म् मे मे॒ वाचं॒ ॅवाच॑म् मे । \newline
15. मे॒ पा॒हि॒ पा॒हि॒ मे॒ मे॒ पा॒हि॒ । \newline
16. पा॒हि॒ मयि॒ मयि॑ पाहि पाहि॒ मयि॑ । \newline
17. मयि॒ वसु॒र् वसु॒र् मयि॒ मयि॒ वसुः॑ । \newline
18. वसु॑र् वि॒दद्व॑सुर् वि॒दद्व॑सु॒र् वसु॒र् वसु॑र् वि॒दद्व॑सुः । \newline
19. वि॒दद्व॑सु श्चक्षु॒ष्पा श्च॑क्षु॒ष्पा वि॒दद्व॑सुर् वि॒दद्व॑सु श्चक्षु॒ष्पाः । \newline
20. वि॒दद्व॑सु॒रिति॑ वि॒दत् - व॒सुः॒ । \newline
21. च॒क्षु॒ष्पा श्चक्षु॒ श्चक्षु॑ श्चक्षु॒ष्पा श्च॑क्षु॒ष्पा श्चक्षुः॑ । \newline
22. च॒क्षु॒ष्पा इति॑ चक्षुः - पाः । \newline
23. चक्षु॑र् मे मे॒ चक्षु॒ श्चक्षु॑र् मे । \newline
24. मे॒ पा॒हि॒ पा॒हि॒ मे॒ मे॒ पा॒हि॒ । \newline
25. पा॒हि॒ मयि॒ मयि॑ पाहि पाहि॒ मयि॑ । \newline
26. मयि॒ वसु॒र् वसु॒र् मयि॒ मयि॒ वसुः॑ । \newline
27. वसुः॑ सं॒ॅयद्व॑सुः सं॒ॅयद्व॑सु॒र् वसु॒र् वसुः॑ सं॒ॅयद्व॑सुः । \newline
28. सं॒ॅयद्व॑सुः श्रोत्र॒पाः श्रो᳚त्र॒पाः सं॒ॅयद्व॑सुः सं॒ॅयद्व॑सुः श्रोत्र॒पाः । \newline
29. सं॒ॅयद्व॑सु॒रिति॑ सं॒ॅयत् - व॒सुः॒ । \newline
30. श्रो॒त्र॒पाः श्रोत्रꣳ॒॒ श्रोत्रꣳ॑ श्रोत्र॒पाः श्रो᳚त्र॒पाः श्रोत्र᳚म् । \newline
31. श्रो॒त्र॒पा इति॑ श्रोत्र - पाः । \newline
32. श्रोत्र॑म् मे मे॒ श्रोत्रꣳ॒॒ श्रोत्र॑म् मे । \newline
33. मे॒ पा॒हि॒ पा॒हि॒ मे॒ मे॒ पा॒हि॒ । \newline
34. पा॒हि॒ भूर् भूः पा॑हि पाहि॒ भूः । \newline
35. भू र॑स्यसि॒ भूर् भूर॑सि । \newline
36. अ॒सि॒ श्रेष्ठः॒ श्रेष्ठो᳚ ऽस्यसि॒ श्रेष्ठः॑ । \newline
37. श्रेष्ठो॑ रश्मी॒नाꣳ र॑श्मी॒नाꣳ श्रेष्ठः॒ श्रेष्ठो॑ रश्मी॒नाम् । \newline
38. र॒श्मी॒नाम् प्रा॑ण॒पाः प्रा॑ण॒पा र॑श्मी॒नाꣳ र॑श्मी॒नाम् प्रा॑ण॒पाः । \newline
39. प्रा॒ण॒पाः प्रा॒णम् प्रा॒णम् प्रा॑ण॒पाः प्रा॑ण॒पाः प्रा॒णम् । \newline
40. प्रा॒ण॒पा इति॑ प्राण - पाः । \newline
41. प्रा॒णम् मे॑ मे प्रा॒णम् प्रा॒णम् मे᳚ । \newline
42. प्रा॒णमिति॑ प्र - अ॒नम् । \newline
43. मे॒ पा॒हि॒ पा॒हि॒ मे॒ मे॒ पा॒हि॒ । \newline
44. पा॒हि॒ धूर् धूः पा॑हि पाहि॒ धूः । \newline
45. धू र॑स्यसि॒ धूर् धूर॑सि । \newline
46. अ॒सि॒ श्रेष्ठः॒ श्रेष्ठो᳚ ऽस्यसि॒ श्रेष्ठः॑ । \newline
47. श्रेष्ठो॑ रश्मी॒नाꣳ र॑श्मी॒नाꣳ श्रेष्ठः॒ श्रेष्ठो॑ रश्मी॒नाम् । \newline
48. र॒श्मी॒ना म॑पान॒पा अ॑पान॒पा र॑श्मी॒नाꣳ र॑श्मी॒ना म॑पान॒पाः । \newline
49. अ॒पा॒न॒पा अ॑पा॒न म॑पा॒न म॑पान॒पा अ॑पान॒पा अ॑पा॒नम् । \newline
50. अ॒पा॒न॒पा इत्य॑पान - पाः । \newline
51. अ॒पा॒नम् मे॑ मे ऽपा॒न म॑पा॒नम् मे᳚ । \newline
52. अ॒पा॒नमित्य॑प - अ॒नम् । \newline
53. मे॒ पा॒हि॒ पा॒हि॒ मे॒ मे॒ पा॒हि॒ । \newline
54. पा॒हि॒ यो यः पा॑हि पाहि॒ यः । \newline
55. यो नो॑ नो॒ यो यो नः॑ । \newline
56. न॒ इ॒न्द्र॒वा॒यू॒ इ॒न्द्र॒वा॒यू॒ नो॒ न॒ इ॒न्द्र॒वा॒यू॒ । \newline
57. इ॒न्द्र॒वा॒यू॒ मि॒त्रा॒व॒रु॒णौ॒ मि॒त्रा॒व॒रु॒णा॒ वि॒न्द्र॒वा॒॒यू॒ इ॒न्द्र॒वा॒यू॒ मि॒त्रा॒व॒रु॒णौ॒  । \newline
58. इ॒न्द्र॒वा॒यू॒ इती᳚न्द्र - वा॒यू॒ । \newline
59. मि॒त्रा॒व॒रु॒णा॒ व॒श्वि॒ना॒ व॒श्वि॒नौ॒ मि॒त्रा॒व॒रु॒णौ॒ मि॒त्रा॒व॒रु॒णा॒ व॒श्वि॒नौ॒ । \newline
60. मि॒त्रा॒व॒रु॒णा॒विति॑ मित्रा - व॒रु॒णौ॒ । \newline
61. अ॒श्वि॒ना॒ व॒भि॒दास॑ त्यभि॒दास॑ त्यश्विना वश्विना वभि॒दास॑ति । \newline
62. अ॒भि॒दास॑ति॒ भ्रातृ॑व्यो॒ भ्रातृ॑व्यो ऽभि॒दास॑ त्यभि॒दास॑ति॒ भ्रातृ॑व्यः । \newline
63. अ॒भि॒दास॒तीत्य॑भि - दास॑ति । \newline
64. भ्रातृ॑व्य उ॒त्पिपी॑त उ॒त्पिपी॑ते॒ भ्रातृ॑व्यो॒ भ्रातृ॑व्य उ॒त्पिपी॑ते । \newline
65. उ॒त्पिपी॑ते शुभः शुभ उ॒त्पिपी॑त उ॒त्पिपी॑ते शुभः । \newline
66. उ॒त्पिपी॑त॒ इत्यु॑त् - पिपी॑ते । \newline
67. शु॒भ॒ स्प॒ती॒ प॒ती॒ शु॒भः॒ शु॒भ॒ स्प॒ती॒ । \newline
68. प॒ती॒ इ॒द मि॒दम् प॑ती पती इ॒दम् । \newline
69. प॒ती॒ इति॑ पती । \newline
70. इ॒द म॒ह म॒ह मि॒द मि॒द म॒हम् । \newline
71. अ॒हम् तम् त म॒ह म॒हम् तम् । \newline
72. त मध॑र॒ मध॑र॒म् तम् त मध॑रम् । \newline
73. अध॑रम् पादयामि पादया॒ म्यध॑र॒ मध॑रम् पादयामि । \newline
74. पा॒द॒या॒मि॒ यथा॒ यथा॑ पादयामि पादयामि॒ यथा᳚ । \newline
75. यथे᳚न्द्रे न्द्र॒ यथा॒ यथे᳚न्द्र । \newline
76. इ॒न्द्रा॒ह म॒ह मि॑न्द्रे न्द्रा॒हम् । \newline
77. अ॒ह मु॑त्त॒म उ॑त्त॒मो॑ ऽह म॒ह मु॑त्त॒मः । \newline
78. उ॒त्त॒म श्चे॒तया॑नि चे॒तया᳚ न्युत्त॒म उ॑त्त॒म श्चे॒तया॑नि । \newline
79. उ॒त्त॒म इत्यु॑त् - त॒मः । \newline
80. चे॒तया॒नीति॑ चे॒तया॑नि । \newline

\textbf{Ghana Paata } \newline

1. तम् ते॑ ते॒ तम् तम् ते॑ दु॒श्चक्षा॑ दु॒श्चक्षा᳚ स्ते॒ तम् तम् ते॑ दु॒श्चक्षाः᳚ । \newline
2. ते॒ दु॒श्चक्षा॑ दु॒श्चक्षा᳚ स्ते ते दु॒श्चक्षा॒ मा मा दु॒श्चक्षा᳚ स्ते ते दु॒श्चक्षा॒ मा । \newline
3. दु॒श्चक्षा॒ मा मा दु॒श्चक्षा॑ दु॒श्चक्षा॒ मा ऽवाव॒ मा दु॒श्चक्षा॑ दु॒श्चक्षा॒ मा ऽव॑ । \newline
4. दु॒श्चक्षा॒ इति॑ दुः - चक्षाः᳚ । \newline
5. मा ऽवाव॒ मा मा ऽव॑ ख्यत् ख्य॒दव॒ मा मा ऽव॑ ख्यत् । \newline
6. अव॑ ख्यत् ख्य॒ दवाव॑ ख्य॒न् मयि॒ मयि॑ ख्य॒ दवाव॑ ख्य॒न् मयि॑ । \newline
7. ख्य॒न् मयि॒ मयि॑ ख्यत् ख्य॒न् मयि॒ वसु॒र् वसु॒र् मयि॑ ख्यत् ख्य॒न् मयि॒ वसुः॑ । \newline
8. मयि॒ वसु॒र् वसु॒र् मयि॒ मयि॒ वसुः॑ पुरो॒वसुः॑ पुरो॒वसु॒र् वसु॒र् मयि॒ मयि॒ वसुः॑ पुरो॒वसुः॑ । \newline
9. वसुः॑ पुरो॒वसुः॑ पुरो॒वसु॒र् वसु॒र् वसुः॑ पुरो॒वसु॑र् वा॒क्पा वा॒क्पाः पु॑रो॒वसु॒र् वसु॒र् वसुः॑ पुरो॒वसु॑र् वा॒क्पाः । \newline
10. पु॒रो॒वसु॑र् वा॒क्पा वा॒क्पाः पु॑रो॒वसुः॑ पुरो॒वसु॑र् वा॒क्पा वाचं॒ ॅवाचं॑ ॅवा॒क्पाः पु॑रो॒वसुः॑ पुरो॒वसु॑र् वा॒क्पा वाच᳚म् । \newline
11. पु॒रो॒वसु॒रिति॑ पुरः - वसुः॑ । \newline
12. वा॒क्पा वाचं॒ ॅवाचं॑ ॅवा॒क्पा वा॒क्पा वाच॑म् मे मे॒ वाचं॑ ॅवा॒क्पा वा॒क्पा वाच॑म् मे । \newline
13. वा॒क्पा इति॑ वाक् - पाः । \newline
14. वाच॑म् मे मे॒ वाचं॒ ॅवाच॑म् मे पाहि पाहि मे॒ वाचं॒ ॅवाच॑म् मे पाहि । \newline
15. मे॒ पा॒हि॒ पा॒हि॒ मे॒ मे॒ पा॒हि॒ मयि॒ मयि॑ पाहि मे मे पाहि॒ मयि॑ । \newline
16. पा॒हि॒ मयि॒ मयि॑ पाहि पाहि॒ मयि॒ वसु॒र् वसु॒र् मयि॑ पाहि पाहि॒ मयि॒ वसुः॑ । \newline
17. मयि॒ वसु॒र् वसु॒र् मयि॒ मयि॒ वसु॑र् वि॒दद्व॑सुर् वि॒दद्व॑सु॒र् वसु॒र् मयि॒ मयि॒ वसु॑र् वि॒दद्व॑सुः । \newline
18. वसु॑र् वि॒दद्व॑सुर् वि॒दद्व॑सु॒र् वसु॒र् वसु॑र् वि॒दद्व॑सु श्चक्षु॒ष्पा श्च॑क्षु॒ष्पा वि॒दद्व॑सु॒र् वसु॒र् वसु॑र् वि॒दद्व॑सु श्चक्षु॒ष्पाः । \newline
19. वि॒दद्व॑सु श्चक्षु॒ष्पा श्च॑क्षु॒ष्पा वि॒दद्व॑सुर् वि॒दद्व॑सु श्चक्षु॒ष्पा श्चक्षु॒ श्चक्षु॑ श्चक्षु॒ष्पा वि॒दद्व॑सुर् वि॒दद्व॑सु श्चक्षु॒ष्पा श्चक्षुः॑ । \newline
20. वि॒दद्व॑सु॒रिति॑ वि॒दत् - व॒सुः॒ । \newline
21. च॒क्षु॒ष्पा श्चक्षु॒ श्चक्षु॑ श्चक्षु॒ष्पा श्च॑क्षु॒ष्पा श्चक्षु॑र् मे मे॒ चक्षु॑ 
श्चक्षु॒ष्पा श्च॑क्षु॒ष्पा श्चक्षु॑र् मे । \newline
22. च॒क्षु॒ष्पा इति॑ चक्षुः - पाः । \newline
23. चक्षु॑र् मे मे॒ चक्षु॒ श्चक्षु॑र् मे पाहि पाहि मे॒ चक्षु॒ श्चक्षु॑र् मे पाहि । \newline
24. मे॒ पा॒हि॒ पा॒हि॒ मे॒ मे॒ पा॒हि॒ मयि॒ मयि॑ पाहि मे मे पाहि॒ मयि॑ । \newline
25. पा॒हि॒ मयि॒ मयि॑ पाहि पाहि॒ मयि॒ वसु॒र् वसु॒र् मयि॑ पाहि पाहि॒ मयि॒ वसुः॑ । \newline
26. मयि॒ वसु॒र् वसु॒र् मयि॒ मयि॒ वसुः॑ सं॒ॅयद्व॑सुः सं॒ॅयद्व॑सु॒र् वसु॒र् मयि॒ मयि॒ वसुः॑ सं॒ॅयद्व॑सुः । \newline
27. वसुः॑ सं॒ॅयद्व॑सुः सं॒ॅयद्व॑सु॒र् वसु॒र् वसुः॑ सं॒ॅयद्व॑सुः श्रोत्र॒पाः श्रो᳚त्र॒पाः सं॒ॅयद्व॑सु॒र् वसु॒र् वसुः॑ सं॒ॅयद्व॑सुः श्रोत्र॒पाः । \newline
28. सं॒ॅयद्व॑सुः श्रोत्र॒पाः श्रो᳚त्र॒पाः सं॒ॅयद्व॑सुः सं॒ॅयद्व॑सुः श्रोत्र॒पाः श्रोत्रꣳ॒॒ श्रोत्रꣳ॑ श्रोत्र॒पाः सं॒ॅयद्व॑सुः सं॒ॅयद्व॑सुः श्रोत्र॒पाः श्रोत्र᳚म् । \newline
29. सं॒ॅयद्व॑सु॒रिति॑ सं॒ॅयत् - व॒सुः॒ । \newline
30. श्रो॒त्र॒पाः श्रोत्रꣳ॒॒ श्रोत्रꣳ॑ श्रोत्र॒पाः श्रो᳚त्र॒पाः श्रोत्र॑म् मे मे॒ श्रोत्रꣳ॑ श्रोत्र॒पाः श्रो᳚त्र॒पाः श्रोत्र॑म् मे । \newline
31. श्रो॒त्र॒पा इति॑ श्रोत्र - पाः । \newline
32. श्रोत्र॑म् मे मे॒ श्रोत्रꣳ॒॒ श्रोत्र॑म् मे पाहि पाहि मे॒ श्रोत्रꣳ॒॒ श्रोत्र॑म् मे पाहि । \newline
33. मे॒ पा॒हि॒ पा॒हि॒ मे॒ मे॒ पा॒हि॒ भूर् भूः पा॑हि मे मे पाहि॒ भूः । \newline
34. पा॒हि॒ भूर् भूः पा॑हि पाहि॒ भू र॑स्यसि॒ भूः पा॑हि पाहि॒ भूर॑सि । \newline
35. भू र॑स्यसि॒ भूर् भूर॑सि॒ श्रेष्ठः॒ श्रेष्ठो॑ ऽसि॒ भूर् भूर॑सि॒ श्रेष्ठः॑ । \newline
36. अ॒सि॒ श्रेष्ठः॒ श्रेष्ठो᳚ ऽस्यसि॒ श्रेष्ठो॑ रश्मी॒नाꣳ र॑श्मी॒नाꣳ श्रेष्ठो᳚ ऽस्यसि॒ श्रेष्ठो॑ रश्मी॒नाम् । \newline
37. श्रेष्ठो॑ रश्मी॒नाꣳ र॑श्मी॒नाꣳ श्रेष्ठः॒ श्रेष्ठो॑ रश्मी॒नाम् प्रा॑ण॒पाः प्रा॑ण॒पा र॑श्मी॒नाꣳ श्रेष्ठः॒ श्रेष्ठो॑ रश्मी॒नाम् प्रा॑ण॒पाः । \newline
38. र॒श्मी॒नाम् प्रा॑ण॒पाः प्रा॑ण॒पा र॑श्मी॒नाꣳ र॑श्मी॒नाम् प्रा॑ण॒पाः प्रा॒णम् प्रा॒णम् प्रा॑ण॒पा र॑श्मी॒नाꣳ र॑श्मी॒नाम् प्रा॑ण॒पाः प्रा॒णम् । \newline
39. प्रा॒ण॒पाः प्रा॒णम् प्रा॒णम् प्रा॑ण॒पाः प्रा॑ण॒पाः प्रा॒णम् मे॑ मे प्रा॒णम् प्रा॑ण॒पाः प्रा॑ण॒पाः प्रा॒णम् मे᳚ । \newline
40. प्रा॒ण॒पा इति॑ प्राण - पाः । \newline
41. प्रा॒णम् मे॑ मे प्रा॒णम् प्रा॒णम् मे॑ पाहि पाहि मे प्रा॒णम् प्रा॒णम् मे॑ पाहि । \newline
42. प्रा॒णमिति॑ प्र- अ॒नम् । \newline
43. मे॒ पा॒हि॒ पा॒हि॒ मे॒ मे॒ पा॒हि॒ धूर् धूः पा॑हि मे मे पाहि॒ धूः । \newline
44. पा॒हि॒ धूर् धूः पा॑हि पाहि॒ धू र॑स्यसि॒ धूः पा॑हि पाहि॒ धूर॑सि । \newline
45. धू र॑स्यसि॒ धूर् धूर॑सि॒ श्रेष्ठः॒ श्रेष्ठो॑ ऽसि॒ धूर् धूर॑सि॒ श्रेष्ठः॑ । \newline
46. अ॒सि॒ श्रेष्ठः॒ श्रेष्ठो᳚ ऽस्यसि॒ श्रेष्ठो॑ रश्मी॒नाꣳ र॑श्मी॒नाꣳ श्रेष्ठो᳚ ऽस्यसि॒ श्रेष्ठो॑ रश्मी॒नाम् । \newline
47. श्रेष्ठो॑ रश्मी॒नाꣳ र॑श्मी॒नाꣳ श्रेष्ठः॒ श्रेष्ठो॑ रश्मी॒ना म॑पान॒पा अ॑पान॒पा र॑श्मी॒नाꣳ श्रेष्ठः॒ श्रेष्ठो॑ रश्मी॒ना म॑पान॒पाः । \newline
48. र॒श्मी॒ना म॑पान॒पा अ॑पान॒पा र॑श्मी॒नाꣳ र॑श्मी॒ना म॑पान॒पा अ॑पा॒न म॑पा॒न म॑पान॒पा र॑श्मी॒नाꣳ र॑श्मी॒ना म॑पान॒पा अ॑पा॒नम् । \newline
49. अ॒पा॒न॒पा अ॑पा॒न म॑पा॒न म॑पान॒पा अ॑पान॒पा अ॑पा॒नम् मे॑ मे ऽपा॒न म॑पान॒पा अ॑पान॒पा अ॑पा॒नम् मे᳚ । \newline
50. अ॒पा॒न॒पा इत्य॑पान - पाः । \newline
51. अ॒पा॒नम् मे॑ मे ऽपा॒न म॑पा॒नम् मे॑ पाहि पाहि मे ऽपा॒न म॑पा॒नम् मे॑ पाहि । \newline
52. अ॒पा॒नमित्य॑प - अ॒नम् । \newline
53. मे॒ पा॒हि॒ पा॒हि॒ मे॒ मे॒ पा॒हि॒ यो यः पा॑हि मे मे पाहि॒ यः । \newline
54. पा॒हि॒ यो यः पा॑हि पाहि॒ यो नो॑ नो॒ यः पा॑हि पाहि॒ यो नः॑ । \newline
55. यो नो॑ नो॒ यो यो न॑ इन्द्रवायू इन्द्रवायू नो॒ यो यो न॑ इन्द्रवायू । \newline
56. न॒ इ॒न्द्र॒वा॒यू॒ इ॒न्द्र॒वा॒यू॒ नो॒ न॒ इ॒न्द्र॒वा॒यू॒ मि॒त्रा॒व॒रु॒णौ॒ मि॒त्रा॒व॒रु॒णा॒ वि॒न्द्र॒वा॒यू॒ नो॒ न॒ इ॒न्द्र॒वा॒यू॒ मि॒त्रा॒व॒रु॒णौ॒ । \newline
57. इ॒न्द्र॒वा॒यू॒ मि॒त्रा॒व॒रु॒णौ॒ मि॒त्रा॒व॒रु॒णा॒ वि॒न्द्र॒वा॒॒यू॒ इ॒न्द्र॒वा॒यू॒ मि॒त्रा॒व॒रु॒णा॒ व॒श्वि॒ना॒ व॒श्वि॒नौ॒ मि॒त्रा॒व॒रु॒णा॒ वि॒न्द्र॒वा॒यू॒ इ॒न्द्र॒वा॒यू॒ मि॒त्रा॒व॒रु॒णा॒ व॒श्वि॒नौ॒ । \newline
58. इ॒न्द्र॒वा॒यू॒ इती᳚न्द्र - वा॒यू॒ । \newline
59. मि॒त्रा॒व॒रु॒णा॒ व॒श्वि॒ना॒ व॒श्वि॒नौ॒ मि॒त्रा॒व॒रु॒णौ॒ मि॒त्रा॒व॒रु॒णा॒ व॒श्वि॒ना॒ व॒भि॒दास॑ त्यभि॒दास॑ त्यश्विनौ मित्रावरुणौ मित्रावरुणा वश्विना वभि॒दास॑ति । \newline
60. मि॒त्रा॒व॒रु॒णा॒विति॑ मित्रा - व॒रु॒णौ॒ । \newline
61. अ॒श्वि॒ना॒ व॒भि॒दास॑ त्यभि॒दास॑ त्यश्विना वश्विना वभि॒दास॑ति॒ भ्रातृ॑व्यो॒ भ्रातृ॑व्यो ऽभि॒दास॑ त्यश्विना वश्विना वभि॒दास॑ति॒ भ्रातृ॑व्यः । \newline
62. अ॒भि॒दास॑ति॒ भ्रातृ॑व्यो॒ भ्रातृ॑व्यो ऽभि॒दास॑ त्यभि॒दास॑ति॒ भ्रातृ॑व्य उ॒त्पिपी॑त उ॒त्पिपी॑ते॒ भ्रातृ॑व्यो ऽभि॒दास॑ त्यभि॒दास॑ति॒ भ्रातृ॑व्य उ॒त्पिपी॑ते । \newline
63. अ॒भि॒दास॒तीत्य॑भि - दास॑ति । \newline
64. भ्रातृ॑व्य उ॒त्पिपी॑त उ॒त्पिपी॑ते॒ भ्रातृ॑व्यो॒ भ्रातृ॑व्य उ॒त्पिपी॑ते शुभः शुभ उ॒त्पिपी॑ते॒ भ्रातृ॑व्यो॒ भ्रातृ॑व्य उ॒त्पिपी॑ते शुभः । \newline
65. उ॒त्पिपी॑ते शुभः शुभ उ॒त्पिपी॑त उ॒त्पिपी॑ते शुभ स्पती पती शुभ उ॒त्पिपी॑त उ॒त्पिपी॑ते शुभ स्पती । \newline
66. उ॒त्पिपी॑त॒ इत्यु॑त् - पिपी॑ते । \newline
67. शु॒भ॒ स्प॒ती॒ प॒ती॒ शु॒भः॒ शु॒भ॒ स्प॒ती॒ इ॒द मि॒दम् प॑ती शुभः शुभ स्पती इ॒दम् । \newline
68. प॒ती॒ इ॒द मि॒दम् प॑ती पती इ॒द म॒ह म॒ह मि॒दम् प॑ती पती इ॒द म॒हम् । \newline
69. प॒ती॒ इति॑ पती । \newline
70. इ॒द म॒ह म॒ह मि॒द मि॒द म॒हम् तम् त म॒ह मि॒द मि॒द म॒हम् तम् । \newline
71. अ॒हम् तम् त म॒ह म॒हम् त मध॑र॒ मध॑र॒म् त म॒ह म॒हम् त मध॑रम् । \newline
72. त मध॑र॒ मध॑र॒म् तम् त मध॑रम् पादयामि पादया॒ म्यध॑र॒म् तम् त मध॑रम् पादयामि । \newline
73. अध॑रम् पादयामि पादया॒ म्यध॑र॒ मध॑रम् पादयामि॒ यथा॒ यथा॑ पादया॒ म्यध॑र॒ मध॑रम् पादयामि॒ यथा᳚ । \newline
74. पा॒द॒या॒मि॒ यथा॒ यथा॑ पादयामि पादयामि॒ यथे᳚न्द्रे न्द्र॒ यथा॑ पादयामि पादयामि॒ यथे᳚न्द्र । \newline
75. यथे᳚न्द्रे न्द्र॒ यथा॒ यथे᳚न्द्रा॒ह म॒ह मि॑न्द्र॒ यथा॒ यथे᳚न्द्रा॒हम् । \newline
76. इ॒न्द्रा॒ह म॒ह मि॑न्द्रे न्द्रा॒ह मु॑त्त॒म उ॑त्त॒मो॑ ऽह मि॑न्द्रे न्द्रा॒ह मु॑त्त॒मः । \newline
77. अ॒ह मु॑त्त॒म उ॑त्त॒मो॑ ऽह म॒ह मु॑त्त॒म श्चे॒तया॑नि चे॒तया᳚ न्युत्त॒मो॑ ऽह म॒ह मु॑त्त॒म श्चे॒तया॑नि । \newline
78. उ॒त्त॒म श्चे॒तया॑नि चे॒तया᳚ न्युत्त॒म उ॑त्त॒म श्चे॒तया॑नि । \newline
79. उ॒त्त॒म इत्यु॑त् - त॒मः । \newline
80. चे॒तया॒नीति॑ चे॒तया॑नि । \newline
\pagebreak
\markright{ TS 3.2.11.1  \hfill https://www.vedavms.in \hfill}

\section{ TS 3.2.11.1 }

\textbf{TS 3.2.11.1 } \newline
\textbf{Samhita Paata} \newline

प्र सो अ॑ग्ने॒ तवो॒तिभिः॑ सु॒वीरा॑भिस्तरति॒ वाज॑कर्मभिः । यस्य॒ त्वꣳ स॒ख्यमावि॑थ ॥ प्र होत्रे॑ पू॒र्व्यं ॅवचो॒ऽग्नये॑ भरता बृ॒हत् । वि॒पां ज्योतीꣳ॑षि॒ बिभ्र॑ते॒ न वे॒धसे᳚ ॥ अग्ने॒ त्री ते॒ वाजि॑ना॒ त्री ष॒धस्था॑ ति॒स्रस्ते॑ जि॒ह्वा ऋ॑तजात पू॒र्वीः । ति॒स्र उ॑ ते त॒नुवो॑ दे॒ववा॑ता॒स्ताभि॑र्नः पाहि॒ गिरो॒ अप्र॑युच्छन्न् ॥ सं ॅवा॒म् कर्म॑णा॒ समि॒षा - [  ] \newline

\textbf{Pada Paata} \newline

प्रेति॑ । सः । अ॒ग्ने॒ । तव॑ । ऊ॒तिभि॒रित्यू॒ति - भिः॒ । सु॒वीरा॑भि॒रिति॑ सु - वीरा॑भिः । त॒र॒ति॒ । वाज॑कर्मभि॒रिति॒ वाज॑कर्म - भिः॒ ॥ यस्य॑ । त्वम् । स॒ख्यम् । आवि॑थ ॥ प्रेति॑ । होत्रे᳚ । पू॒र्व्यम् । वचः॑ । अ॒ग्नये᳚ । भ॒र॒त॒ । बृ॒हत् ॥ वि॒पामिति॑ वि - पाम् । ज्योतीꣳ॑षि । बिभ्र॑ते । न । वे॒धसे᳚ ॥ अग्ने᳚ । त्री । ते॒ । वाजि॑ना । त्री । स॒धस्थेति॑ स॒ध - स्था॒ । ति॒स्रः । ते॒ । जि॒ह्वाः । ऋ॒त॒जा॒तेत्यृ॑त - जा॒त॒ । पू॒र्वीः ॥ ति॒स्रः । उ॒ । ते॒ । त॒नुवः॑ । दे॒ववा॑ता॒ इति॑ दे॒व - वा॒ताः॒ । ताभिः॑ । नः॒ । पा॒हि॒ । गिरः॑ । अप्र॑युच्छ॒न्नित्यप्र॑ - यु॒च्छ॒न्न् ॥ समिति॑ । वा॒म् । कर्म॑णा । समिति॑ । इ॒षा ।  \newline


\textbf{Krama Paata} \newline

प्र सः । सो अ॑ग्ने । अ॒ग्ने॒ तव॑ । तवो॒तिभिः॑ । ऊ॒तिभिः॑ सु॒वीरा॑भिः । ऊ॒तिभि॒रित्यू॒ति - भिः॒ । सु॒वीरा॑भिस्तरति । सु॒वीरा॑भि॒रिति॑ सु - वीरा॑भिः । त॒र॒ति॒ वाज॑कर्मभिः । वाज॑कर्मभि॒रिति॒ वाज॑कर्म - भिः॒ ॥ यस्य॒ त्वम् । त्वꣳ स॒ख्यम् । स॒ख्यमावि॑थ । आवि॒थेत्यावि॑थ ॥ प्र होत्रे᳚ । होत्रे॑ पू॒र्व्यम् । पू॒र्व्यं ॅवचः॑ । वचो॒ ऽग्नये᳚ । अ॒ग्नये॑ भरत । भ॒र॒ता॒ बृ॒हत् । बृ॒हदिति॑ बृ॒हत् ॥ वि॒पाम् ज्योतीꣳ॑षि । वि॒पामिति॑ वि - पाम् । ज्योतीꣳ॑षि॒ बिभ्र॑ते । बिभ्र॑ते॒ न । न वे॒धसे᳚ । वे॒धस॒ इति॑ वे॒धसे᳚ ॥ अग्ने॒ त्री । त्री ते᳚ । ते॒ वाजि॑ना । वाजि॑ना॒ त्री । त्री ष॒धस्था᳚ । स॒धस्था॑ ति॒स्रः । स॒धस्थेति॑ स॒ध - स्था॒ । ति॒स्रस्ते᳚ । ते॒ जि॒ह्वाः । जि॒ह्वा ऋ॑तजात । ऋ॒त॒जा॒त॒ पू॒र्वीः । ऋ॒त॒जा॒तेत्यृ॑त - जा॒त॒ । पू॒र्वीरिति॑ पू॒र्वीः ॥ ति॒स्र उ॑ । उ॒ ते॒ । ते॒ त॒नुवः॑ । त॒नुवो॑ दे॒ववा॑ताः । दे॒ववा॑ता॒स्ताभिः॑ । दे॒ववा॑ता॒ इति॑ दे॒व - वा॒ताः॒ । ताभि॑र् नः । नः॒ पा॒हि॒ । पा॒हि॒ गिरः॑ । गिरो॒ अप्र॑युच्छन्न् । अप्र॑युच्छ॒नित्यप्र॑ - यु॒च्छ॒न्न्॒ ॥ सं ॅवा᳚म् । वां॒ कर्म॑णा । कर्म॑णा॒ सम् । समि॒षा । इ॒षा हि॑नोमि \newline

\textbf{Jatai Paata} \newline

1. प्र स स प्र प्र सः । \newline
2. सो अ॑ग्ने अग्ने॒ स सो अ॑ग्ने । \newline
3. अ॒ग्ने॒ तव॒ तवा᳚ग्ने अग्ने॒ तव॑ । \newline
4. तवो॒तिभि॑ रू॒तिभि॒ स्तव॒ तवो॒तिभिः॑ । \newline
5. ऊ॒तिभिः॑ सु॒वीरा॑भिः सु॒वीरा॑भि रू॒तिभि॑ रू॒तिभिः॑ सु॒वीरा॑भिः । \newline
6. ऊ॒तिभि॒रित्यू॒ति - भिः॒ । \newline
7. सु॒वीरा॑भि स्तरति तरति सु॒वीरा॑भिः सु॒वीरा॑भि स्तरति । \newline
8. सु॒वीरा॑भि॒रिति॑ सु - वीरा॑भिः । \newline
9. त॒र॒ति॒ वाज॑कर्मभि॒र् वाज॑कर्मभि स्तरति तरति॒ वाज॑कर्मभिः । \newline
10. वाज॑कर्मभि॒रिति॒ वाज॑कर्म - भिः॒ । \newline
11. यस्य॒ त्वम् त्वं ॅयस्य॒ यस्य॒ त्वम् । \newline
12. त्वꣳ स॒ख्यꣳ स॒ख्यम् त्वम् त्वꣳ स॒ख्यम् । \newline
13. स॒ख्य मावि॒था वि॑थ स॒ख्यꣳ स॒ख्य मावि॑थ । \newline
14. आवि॒थेत्यावि॑थ । \newline
15. प्र होत्रे॒ होत्रे॒ प्र प्र होत्रे᳚ । \newline
16. होत्रे॑ पू॒र्व्यम् पू॒र्व्यꣳ होत्रे॒ होत्रे॑ पू॒र्व्यम् । \newline
17. पू॒र्व्यं ॅवचो॒ वचः॑ पू॒र्व्यम् पू॒र्व्यं ॅवचः॑ । \newline
18. वचो॒ ऽग्नये॑ अ॒ग्नये॒ वचो॒ वचो॒ ऽग्नये᳚ । \newline
19. अ॒ग्नये॑ भरत भरता॒ ग्नये॑ अ॒ग्नये॑ भरत । \newline
20. भ॒र॒ता॒ बृ॒हद् बृ॒हद् भ॑रत भरता बृ॒हत् । \newline
21. बृ॒हदिति॑ बृ॒हत् । \newline
22. वि॒पाम् ज्योतीꣳ॑षि॒ ज्योतीꣳ॑षि वि॒पां ॅवि॒पाम् ज्योतीꣳ॑षि । \newline
23. वि॒पामिति॑ वि - पाम् । \newline
24. ज्योतीꣳ॑षि॒ बिभ्र॑ते॒ बिभ्र॑ते॒ ज्योतीꣳ॑षि॒ ज्योतीꣳ॑षि॒ बिभ्र॑ते । \newline
25. बिभ्र॑ते॒ न न बिभ्र॑ते॒ बिभ्र॑ते॒ न । \newline
26. न वे॒धसे॑ वे॒धसे॒ न न वे॒धसे᳚ । \newline
27. वे॒धस॒ इति॑ वे॒धसे᳚ । \newline
28. अग्ने॒ त्री त्र्यग्ने ऽग्ने॒ त्री । \newline
29. त्री ते॑ ते॒ त्री त्री ते᳚ । \newline
30. ते॒ वाजि॑ना॒ वाजि॑ना ते ते॒ वाजि॑ना । \newline
31. वाजि॑ना॒ त्री त्री वाजि॑ना॒ वाजि॑ना॒ त्री । \newline
32. त्री ष॒धस्था॑ स॒धस्था॒ त्री त्री ष॒धस्था᳚ । \newline
33. स॒धस्था॑ ति॒स्र स्ति॒स्रः स॒धस्था॑ स॒धस्था॑ ति॒स्रः । \newline
34. स॒धस्थेति॑ स॒ध - स्था॒ । \newline
35. ति॒स्र स्ते॑ ते ति॒स्र स्ति॒स्र स्ते᳚ । \newline
36. ते॒ जि॒ह्वा जि॒ह्वा स्ते॑ ते जि॒ह्वाः । \newline
37. जि॒ह्वा ऋ॑तजात र्तजात जि॒ह्वा जि॒ह्वा ऋ॑तजात । \newline
38. ऋ॒त॒जा॒त॒ पू॒र्वीः पू॒र्वीर्. ऋ॑तजात र्तजात पू॒र्वीः । \newline
39. ऋ॒त॒जा॒तेत्यृ॑त - जा॒त॒ । \newline
40. पू॒र्वीरिति॑ पू॒र्वीः । \newline
41. ति॒स्र उ॑ वु ति॒स्र स्ति॒स्र उ॑ । \newline
42. उ॒ ते॒ त॒ उ॒ वु॒ ते॒ । \newline
43. ते॒ त॒नुव॑ स्त॒नुव॑ स्ते ते त॒नुवः॑ । \newline
44. त॒नुवो॑ दे॒ववा॑ता दे॒ववा॑ता स्त॒नुव॑ स्त॒नुवो॑ दे॒ववा॑ताः । \newline
45. दे॒ववा॑ता॒ स्ताभि॒ स्ताभि॑र् दे॒ववा॑ता दे॒ववा॑ता॒ स्ताभिः॑ । \newline
46. दे॒ववा॑ता॒ इति॑ दे॒व - वा॒ताः॒ । \newline
47. ताभि॑र् नो न॒ स्ताभि॒ स्ताभि॑र् नः । \newline
48. नः॒ पा॒हि॒ पा॒हि॒ नो॒ नः॒ पा॒हि॒ । \newline
49. पा॒हि॒ गिरो॒ गिरः॑ पाहि पाहि॒ गिरः॑ । \newline
50. गिरो॒ अप्र॑युच्छ॒न् नप्र॑युच्छ॒न् गिरो॒ गिरो॒ अप्र॑युच्छन्न् । \newline
51. अप्र॑युच्छ॒न्नित्यप्र॑ - यु॒च्छ॒न्न् । \newline
52. सं ॅवां᳚ ॅवाꣳ॒॒ सꣳ सं ॅवा᳚म् । \newline
53. वा॒म् कर्म॑णा॒ कर्म॑णा वां ॅवा॒म् कर्म॑णा । \newline
54. कर्म॑णा॒ सꣳ सम् कर्म॑णा॒ कर्म॑णा॒ सम् । \newline
55. स मि॒षेषा सꣳ स मि॒षा । \newline
56. इ॒षा हि॑नोमि हिनोमी॒ षेषा हि॑नोमि । \newline

\textbf{Ghana Paata } \newline

1. प्र स स प्र प्र सो अ॑ग्ने अग्ने॒ स प्र प्र सो अ॑ग्ने । \newline
2. सो अ॑ग्ने अग्ने॒ स सो अ॑ग्ने॒ तव॒ तवा᳚ग्ने॒ स सो अ॑ग्ने॒ तव॑ । \newline
3. अ॒ग्ने॒ तव॒ तवा᳚ग्ने अग्ने॒ तवो॒तिभि॑ रू॒तिभि॒ स्तवा᳚ग्ने अग्ने॒ तवो॒तिभिः॑ । \newline
4. तवो॒तिभि॑ रू॒तिभि॒ स्तव॒ तवो॒तिभिः॑ सु॒वीरा॑भिः सु॒वीरा॑भि रू॒तिभि॒ स्तव॒ तवो॒तिभिः॑ सु॒वीरा॑भिः । \newline
5. ऊ॒तिभिः॑ सु॒वीरा॑भिः सु॒वीरा॑भि रू॒तिभि॑ रू॒तिभिः॑ सु॒वीरा॑भि स्तरति तरति सु॒वीरा॑भि रू॒तिभि॑ रू॒तिभिः॑ सु॒वीरा॑भि स्तरति । \newline
6. ऊ॒तिभि॒रित्यू॒ति - भिः॒ । \newline
7. सु॒वीरा॑भि स्तरति तरति सु॒वीरा॑भिः सु॒वीरा॑भि स्तरति॒ वाज॑कर्मभि॒र् वाज॑कर्मभि स्तरति सु॒वीरा॑भिः सु॒वीरा॑भि स्तरति॒ वाज॑कर्मभिः । \newline
8. सु॒वीरा॑भि॒रिति॑ सु - वीरा॑भिः । \newline
9. त॒र॒ति॒ वाज॑कर्मभि॒र् वाज॑कर्मभि स्तरति तरति॒ वाज॑कर्मभिः । \newline
10. वाज॑कर्मभि॒रिति॒ वाज॑कर्म - भिः॒ । \newline
11. यस्य॒ त्वम् त्वं ॅयस्य॒ यस्य॒ त्वꣳ स॒ख्यꣳ स॒ख्यम् त्वं ॅयस्य॒ यस्य॒ त्वꣳ स॒ख्यम् । \newline
12. त्वꣳ स॒ख्यꣳ स॒ख्यम् त्वम् त्वꣳ स॒ख्य मावि॒था वि॑थ स॒ख्यम् त्वम् त्वꣳ स॒ख्य मावि॑थ । \newline
13. स॒ख्य मावि॒था वि॑थ स॒ख्यꣳ स॒ख्य मावि॑थ । \newline
14. आवि॒थेत्यावि॑थ । \newline
15. प्र होत्रे॒ होत्रे॒ प्र प्र होत्रे॑ पू॒र्व्यम् पू॒र्व्यꣳ होत्रे॒ प्र प्र होत्रे॑ पू॒र्व्यम् । \newline
16. होत्रे॑ पू॒र्व्यम् पू॒र्व्यꣳ होत्रे॒ होत्रे॑ पू॒र्व्यं ॅवचो॒ वचः॑ पू॒र्व्यꣳ होत्रे॒ होत्रे॑ पू॒र्व्यं ॅवचः॑ । \newline
17. पू॒र्व्यं ॅवचो॒ वचः॑ पू॒र्व्यम् पू॒र्व्यं ॅवचो॒ ऽग्नये॑ अ॒ग्नये॒ वचः॑ पू॒र्व्यम् पू॒र्व्यं ॅवचो॒ ऽग्नये᳚ । \newline
18. वचो॒ ऽग्नये॑ अ॒ग्नये॒ वचो॒ वचो॒ ऽग्नये॑ भरत भरता॒ ग्नये॒ वचो॒ वचो॒ ऽग्नये॑ भरत । \newline
19. अ॒ग्नये॑ भरत भरता॒ ग्नये॑ अ॒ग्नये॑ भरता बृ॒हद् बृ॒हद् भ॑रता॒ ग्नये॑ अ॒ग्नये॑ भरता बृ॒हत् । \newline
20. भ॒र॒ता॒ बृ॒हद् बृ॒हद् भ॑रत भरता बृ॒हत् । \newline
21. बृ॒हदिति॑ बृ॒हत् । \newline
22. वि॒पाम् ज्योतीꣳ॑षि॒ ज्योतीꣳ॑षि वि॒पां ॅवि॒पाम् ज्योतीꣳ॑षि॒ बिभ्र॑ते॒ बिभ्र॑ते॒ ज्योतीꣳ॑षि वि॒पां ॅवि॒पाम् ज्योतीꣳ॑षि॒ बिभ्र॑ते । \newline
23. वि॒पामिति॑ वि - पाम् । \newline
24. ज्योतीꣳ॑षि॒ बिभ्र॑ते॒ बिभ्र॑ते॒ ज्योतीꣳ॑षि॒ ज्योतीꣳ॑षि॒ बिभ्र॑ते॒ न न बिभ्र॑ते॒ ज्योतीꣳ॑षि॒ ज्योतीꣳ॑षि॒ बिभ्र॑ते॒ न । \newline
25. बिभ्र॑ते॒ न न बिभ्र॑ते॒ बिभ्र॑ते॒ न वे॒धसे॑ वे॒धसे॒ न बिभ्र॑ते॒ बिभ्र॑ते॒ न वे॒धसे᳚ । \newline
26. न वे॒धसे॑ वे॒धसे॒ न न वे॒धसे᳚ । \newline
27. वे॒धस॒ इति॑ वे॒धसे᳚ । \newline
28. अग्ने॒ त्री त्र्यग्ने ऽग्ने॒ त्री ते॑ ते॒ त्र्यग्ने ऽग्ने॒ त्री ते᳚ । \newline
29. त्री ते॑ ते॒ त्री त्री ते॒ वाजि॑ना॒ वाजि॑ना ते॒ त्री त्री ते॒ वाजि॑ना । \newline
30. ते॒ वाजि॑ना॒ वाजि॑ना ते ते॒ वाजि॑ना॒ त्री त्री वाजि॑ना ते ते॒ वाजि॑ना॒ त्री । \newline
31. वाजि॑ना॒ त्री त्री वाजि॑ना॒ वाजि॑ना॒ त्री ष॒धस्था॑ स॒धस्था॒ त्री वाजि॑ना॒ वाजि॑ना॒ त्री ष॒धस्था᳚ । \newline
32. त्री ष॒धस्था॑ स॒धस्था॒ त्री त्री ष॒धस्था॑ ति॒स्र स्ति॒स्रः स॒धस्था॒ त्री त्री ष॒धस्था॑ ति॒स्रः । \newline
33. स॒धस्था॑ ति॒स्र स्ति॒स्रः स॒धस्था॑ स॒धस्था॑ ति॒स्रस्ते॑ ते ति॒स्रः स॒धस्था॑ स॒धस्था॑ ति॒स्र स्ते᳚ । \newline
34. स॒धस्थेति॑ स॒ध - स्था॒ । \newline
35. ति॒स्र स्ते॑ ते ति॒स्र स्ति॒स्र स्ते॑ जि॒ह्वा जि॒ह्वास्ते॑ ति॒स्र स्ति॒स्र स्ते॑ जि॒ह्वाः । \newline
36. ते॒ जि॒ह्वा जि॒ह्वा स्ते॑ ते जि॒ह्वा ऋ॑तजात र्तजात जि॒ह्वा स्ते॑ ते जि॒ह्वा ऋ॑तजात । \newline
37. जि॒ह्वा ऋ॑तजात र्तजात जि॒ह्वा जि॒ह्वा ऋ॑तजात पू॒र्वीः पू॒र्वीर्. ऋ॑तजात जि॒ह्वा जि॒ह्वा ऋ॑तजात पू॒र्वीः । \newline
38. ऋ॒त॒जा॒त॒ पू॒र्वीः पू॒र्वीर्. ऋ॑तजात र्तजात पू॒र्वीः । \newline
39. ऋ॒त॒जा॒तेत्यृ॑त - जा॒त॒ । \newline
40. पू॒र्वीरिति॑ पू॒र्वीः । \newline
41. ति॒स्र उ॑ वु ति॒स्र स्ति॒स्र उ॑ ते त उ ति॒स्र स्ति॒स्र उ॑ ते । \newline
42. उ॒ ते॒ त॒ उ॒ वु॒ ते॒ त॒नुव॑ स्त॒नुव॑ स्त उ वु ते त॒नुवः॑ । \newline
43. ते॒ त॒नुव॑ स्त॒नुव॑ स्ते ते त॒नुवो॑ दे॒ववा॑ता दे॒ववा॑ता स्त॒नुव॑ स्ते ते त॒नुवो॑ दे॒ववा॑ताः । \newline
44. त॒नुवो॑ दे॒ववा॑ता दे॒ववा॑ता स्त॒नुव॑ स्त॒नुवो॑ दे॒ववा॑ता॒ स्ताभि॒ स्ताभि॑र् दे॒ववा॑ता स्त॒नुव॑ स्त॒नुवो॑ दे॒ववा॑ता॒ स्ताभिः॑ । \newline
45. दे॒ववा॑ता॒ स्ताभि॒ स्ताभि॑र् दे॒ववा॑ता दे॒ववा॑ता॒ स्ताभि॑र् नो न॒ स्ताभि॑र् दे॒ववा॑ता दे॒ववा॑ता॒ स्ताभि॑र् नः । \newline
46. दे॒ववा॑ता॒ इति॑ दे॒व - वा॒ताः॒ । \newline
47. ताभि॑र् नो न॒ स्ताभि॒ स्ताभि॑र् नः पाहि पाहि न॒ स्ताभि॒ स्ताभि॑र् नः पाहि । \newline
48. नः॒ पा॒हि॒ पा॒हि॒ नो॒ नः॒ पा॒हि॒ गिरो॒ गिरः॑ पाहि नो नः पाहि॒ गिरः॑ । \newline
49. पा॒हि॒ गिरो॒ गिरः॑ पाहि पाहि॒ गिरो॒ अप्र॑युच्छ॒न् नप्र॑युच्छ॒न् गिरः॑ पाहि पाहि॒ गिरो॒ अप्र॑युच्छन्न् । \newline
50. गिरो॒ अप्र॑युच्छ॒न् नप्र॑युच्छ॒न् गिरो॒ गिरो॒ अप्र॑युच्छन्न् । \newline
51. अप्र॑युच्छ॒न्नित्यप्र॑ - यु॒च्छ॒न्न् । \newline
52. सं ॅवां᳚ ॅवाꣳ॒॒ सꣳ सं ॅवा॒म् कर्म॑णा॒ कर्म॑णा वाꣳ॒॒ सꣳ सं ॅवा॒म् कर्म॑णा । \newline
53. वा॒म् कर्म॑णा॒ कर्म॑णा वां ॅवा॒म् कर्म॑णा॒ सꣳ सम् कर्म॑णा वां ॅवा॒म् कर्म॑णा॒ सम् । \newline
54. कर्म॑णा॒ सꣳ सम् कर्म॑णा॒ कर्म॑णा॒ स मि॒षेषा सम् कर्म॑णा॒ कर्म॑णा॒ स मि॒षा । \newline
55. स मि॒षेषा सꣳ स मि॒षा हि॑नोमि हिनोमी॒षा सꣳ स मि॒षा हि॑नोमि । \newline
56. इ॒षा हि॑नोमि हिनो मी॒षेषा हि॑नो॒ मीन्द्रा॑विष्णू॒ इन्द्रा॑विष्णू हिनो मी॒षेषा हि॑नो॒ मीन्द्रा॑विष्णू । \newline
\pagebreak
\markright{ TS 3.2.11.2  \hfill https://www.vedavms.in \hfill}

\section{ TS 3.2.11.2 }

\textbf{TS 3.2.11.2 } \newline
\textbf{Samhita Paata} \newline

हि॑नो॒मीन्द्रा॑-विष्णू॒ अप॑सस्पा॒रे अ॒स्य । जु॒षेथां᳚ ॅय॒ज्ञ्ं द्रवि॑णं च धत्त॒मरि॑ष्टैर्नः प॒थिभिः॑ पा॒रय॑न्ता ॥ उ॒भा जि॑ग्यथु॒र्न परा॑ जयेथे॒ न परा॑ जिग्ये कत॒रश्च॒नैनोः᳚ । इन्द्र॑श्च विष्णो॒ यदप॑स्पृधेथां त्रे॒धा स॒हस्रं॒ ॅवि तदै॑रयेथां ॥ त्रीण्यायूꣳ॑षि॒ तव॑ जातवेदस्ति॒स्र आ॒जानी॑रु॒षस॑स्ते अग्ने । ताभि॑र्दे॒वाना॒मवो॑ यक्षि वि॒द्वानथा॑ - [  ] \newline

\textbf{Pada Paata} \newline

हि॒नो॒मि॒ । इन्द्रा॑विष्णू॒ इतीन्द्रा᳚ - वि॒ष्णू॒ । अप॑सः । पा॒रे । अ॒स्य ॥ जु॒षेथा᳚म् । य॒ज्ञ्म् । द्रवि॑णम् । च॒ । ध॒त्त॒म् । अरि॑ष्टैः । नः॒ । प॒थिभि॒रिति॑ प॒थि-भिः॒ । पा॒रय॑न्ता ॥ उ॒भा । जि॒ग्य॒थुः॒ । न । परेति॑ । ज॒ये॒थे॒ इति॑ । न । परेति॑ । जि॒ग्ये॒ । क॒त॒रः । च॒न । ए॒नोः॒ ॥ इन्द्रः॑ । च॒ । वि॒ष्णो॒ इति॑ । यत् । अप॑स्पृधेथाम् । त्रे॒धा । स॒हस्र᳚म् । वीति॑ । तत् । ऐ॒र॒ये॒था॒म् ॥ त्रीणि॑ । आयूꣳ॑षि । तव॑ । जा॒त॒वे॒द॒ इति॑ जात - वे॒दः॒ । ति॒स्रः । आ॒जानी॒रित्या᳚ - जानीः᳚ । उ॒षसः॑ । ते॒ । अ॒ग्ने॒ । ताभिः॑ ॥ दे॒वाना᳚म् । अवः॑ । य॒क्षि॒ । वि॒द्वान् । अथ॑ ।  \newline


\textbf{Krama Paata} \newline

हि॒नो॒मीन्द्रा॑विष्णू । इन्द्रा॑विष्णू॒ अप॑सः । इन्द्रा॑विष्णू॒ इतीन्द्रा᳚ - वि॒ष्णू॒ । अप॑सस्पा॒रे । पा॒रे अ॒स्य । अ॒स्येत्य॒स्य ॥ जु॒षेथां᳚ ॅय॒ज्ञ्म् । य॒ज्ञ्म् द्रवि॑णम् । द्रवि॑णम् च । च॒ ध॒त्त॒म् । ध॒त्त॒मरि॑ष्टैः । अरि॑ष्टैर् नः । नः॒ प॒थिभिः॑ । प॒थिभिः॑ पा॒रय॑न्ता । प॒थिभि॒रिति॑ प॒थि - भिः॒ । पा॒रय॒न्तेति॑ पा॒रय॑न्ता ॥ उ॒भा जि॑ग्यथुः । जि॒ग्य॒थु॒र् न । न परा᳚ । परा॑ जयेथे । ज॒ये॒थे॒ न । ज॒ये॒थे॒ इति॑ जयेथे । न परा᳚ । परा॑ जिग्ये । जि॒ग्ये॒ क॒त॒रः । क॒त॒रश्च॒न । च॒नैनोः᳚ । ए॒नो॒रित्ये॑नोः ॥ इन्द्र॑श्च । च॒ वि॒ष्णो॒ । वि॒ष्णो॒ यत् । वि॒ष्णो॒ इति॑ विष्णो । यदप॑स्पृधेथाम् । अप॑स्पृधेथाम् त्रे॒धा । त्रे॒धा स॒हस्र᳚म् । स॒हस्रं॒ ॅवि । वि तत् । तदै॑रयेथाम् । ऐ॒र॒ये॒था॒मित्यै॑रयेथाम् ॥ त्रीण्यायूꣳ॑षि । आयूꣳ॑षि॒ तव॑ । तव॑ जातवेदः । जा॒त॒वे॒द॒स्ति॒स्रः । जा॒त॒वे॒द॒ इति॑ जात - वे॒दः॒ । ति॒स्र आ॒जानीः᳚ । आ॒जानी॑रु॒षसः॑ । आ॒जानी॒रित्या᳚ - जानीः᳚ । उ॒षस॑स्ते । ते॒ अ॒ग्ने॒ । अ॒ग्न॒ इत्य॑ग्ने ॥ ताभि॑र् दे॒वाना᳚म् । दे॒वाना॒मवः॑ । अवो॑ यक्षि । य॒क्षि॒ वि॒द्वान् । वि॒द्वानथ॑ । अथा॑ भव \newline

\textbf{Jatai Paata} \newline

1. हि॒नो॒ मीन्द्रा॑विष्णू॒ इन्द्रा॑विष्णू हिनोमि हिनो॒ मीन्द्रा॑विष्णू । \newline
2. इन्द्रा॑विष्णू॒ अप॑सो॒ अप॑स॒ इन्द्रा॑विष्णू॒ इन्द्रा॑विष्णू॒ अप॑सः । \newline
3. इन्द्रा॑विष्णू॒ इतीन्द्रा᳚ - वि॒ष्णू॒ । \newline
4. अप॑स स्पा॒रे पा॒रे अप॑सो॒ अप॑स स्पा॒रे । \newline
5. पा॒रे अ॒स्यास्य पा॒रे पा॒रे अ॒स्य । \newline
6. अ॒स्येत्य॒स्य । \newline
7. जु॒षेथां᳚ ॅय॒ज्ञ्ं ॅय॒ज्ञ्म् जु॒षेथा᳚म् जु॒षेथां᳚ ॅय॒ज्ञ्म् । \newline
8. य॒ज्ञ्म् द्रवि॑ण॒म् द्रवि॑णं ॅय॒ज्ञ्ं ॅय॒ज्ञ्म् द्रवि॑णम् । \newline
9. द्रवि॑णम् च च॒ द्रवि॑ण॒म् द्रवि॑णम् च । \newline
10. च॒ ध॒त्त॒म् ध॒त्त॒म् च॒ च॒ ध॒त्त॒म् । \newline
11. ध॒त्त॒ मरि॑ष्टै॒ ररि॑ष्टैर् धत्तम् धत्त॒ मरि॑ष्टैः । \newline
12. अरि॑ष्टैर् नो नो॒ अरि॑ष्टै॒ ररि॑ष्टैर् नः । \newline
13. नः॒ प॒थिभिः॑ प॒थिभि॑र् नो नः प॒थिभिः॑ । \newline
14. प॒थिभिः॑ पा॒रय॑न्ता पा॒रय॑न्ता प॒थिभिः॑ प॒थिभिः॑ पा॒रय॑न्ता । \newline
15. प॒थिभि॒रिति॑ प॒थि - भिः॒ । \newline
16. पा॒रय॒न्तेति॑ पा॒रय॑न्ता । \newline
17. उ॒भा जि॑ग्यथुर् जिग्यथु रु॒भोभा जि॑ग्यथुः । \newline
18. जि॒ग्य॒थु॒र् न न जि॑ग्यथुर् जिग्यथु॒र् न । \newline
19. न परा॒ परा॒ न न परा᳚ । \newline
20. परा॑ जयेथे जयेथे॒ परा॒ परा॑ जयेथे । \newline
21. ज॒ये॒थे॒ न न ज॑येथे जयेथे॒ न । \newline
22. ज॒ये॒थे॒ इति॑ जयेथे । \newline
23. न परा॒ परा॒ न न परा᳚ । \newline
24. परा॑ जिग्ये जिग्ये॒ परा॒ परा॑ जिग्ये । \newline
25. जि॒ग्ये॒ क॒त॒रः क॑त॒रो जि॑ग्ये जिग्ये कत॒रः । \newline
26. क॒त॒र श्च॒न च॒न क॑त॒रः क॑त॒र श्च॒न । \newline
27. च॒नैनो॑ रेनो श्च॒न च॒नैनोः᳚ । \newline
28. ए॒नो॒रित्ये॑नोः । \newline
29. इन्द्र॑श्च॒ चे न्द्र॒ इन्द्र॑श्च । \newline
30. च॒ वि॒ष्णो॒ वि॒ष्णो॒ च॒ च॒ वि॒ष्णो॒ । \newline
31. वि॒ष्णो॒ यद् यद् वि॑ष्णो विष्णो॒ यत् । \newline
32. वि॒ष्णो॒ इति॑ विष्णो । \newline
33. यदप॑स्पृधेथा॒ मप॑स्पृधेथां॒ ॅयद् यदप॑स्पृधेथाम् । \newline
34. अप॑स्पृधेथाम् त्रे॒धा त्रे॒धा ऽप॑स्पृधेथा॒ मप॑स्पृधेथाम् त्रे॒धा । \newline
35. त्रे॒धा स॒हस्रꣳ॑ स॒हस्र॑म् त्रे॒धा त्रे॒धा स॒हस्र᳚म् । \newline
36. स॒हस्रं॒ ॅवि वि स॒हस्रꣳ॑ स॒हस्रं॒ ॅवि । \newline
37. वि तत् तद् वि वि तत् । \newline
38. तदै॑रयेथा मैरयेथा॒म् तत् तदै॑रयेथाम् । \newline
39. ऐ॒र॒ये॒था॒ मित्यै॑रयेथाम् । \newline
40. त्रीण्यायूꣳ॒॒ ष्यायूꣳ॑षि॒ त्रीणि॒ त्रीण्यायूꣳ॑षि । \newline
41. आयूꣳ॑षि॒ तव॒ तवायूꣳ॒॒ ष्यायूꣳ॑षि॒ तव॑ । \newline
42. तव॑ जातवेदो जातवेद॒ स्तव॒ तव॑ जातवेदः । \newline
43. जा॒त॒वे॒द॒ स्ति॒स्र स्ति॒स्रो जा॑तवेदो जातवेद स्ति॒स्रः । \newline
44. जा॒त॒वे॒द॒ इति॑ जात - वे॒दः॒ । \newline
45. ति॒स्र आ॒जानी॑ रा॒जानी᳚ स्ति॒स्र स्ति॒स्र आ॒जानीः᳚ । \newline
46. आ॒जानी॑ रु॒षस॑ उ॒षस॑ आ॒जानी॑ रा॒जानी॑ रु॒षसः॑ । \newline
47. आ॒जानी॒रित्या᳚ - जानीः᳚ । \newline
48. उ॒षस॑ स्ते त उ॒षस॑ उ॒षस॑ स्ते । \newline
49. ते॒ अ॒ग्ने॒ अ॒ग्ने॒ ते॒ ते॒ अ॒ग्ने॒ । \newline
50. अ॒ग्न॒ इत्य॑ग्ने । \newline
51. ताभि॑र् दे॒वाना᳚म् दे॒वाना॒म् ताभि॒ स्ताभि॑र् दे॒वाना᳚म् । \newline
52. दे॒वाना॒ मवो ऽवो॑ दे॒वाना᳚म् दे॒वाना॒ मवः॑ । \newline
53. अवो॑ यक्षि य॒क्ष्यवो ऽवो॑ यक्षि । \newline
54. य॒क्षि॒ वि॒द्वान्. वि॒द्वान्. य॑क्षि यक्षि वि॒द्वान् । \newline
55. वि॒द्वा नथाथ॑ वि॒द्वान्. वि॒द्वा नथ॑ । \newline
56. अथा॑ भव भ॒वाथाथा॑ भव । \newline

\textbf{Ghana Paata } \newline

1. हि॒नो॒ मीन्द्रा॑विष्णू॒ इन्द्रा॑विष्णू हिनोमि हिनो॒ मीन्द्रा॑विष्णू॒ अप॑सो॒ अप॑स॒ इन्द्रा॑विष्णू हिनोमि हिनो॒ मीन्द्रा॑विष्णू॒ अप॑सः । \newline
2. इन्द्रा॑विष्णू॒ अप॑सो॒ अप॑स॒ इन्द्रा॑विष्णू॒ इन्द्रा॑विष्णू॒ अप॑स स्पा॒रे पा॒रे अप॑स॒ इन्द्रा॑विष्णू॒ इन्द्रा॑विष्णू॒ अप॑स स्पा॒रे । \newline
3. इन्द्रा॑विष्णू॒ इतीन्द्रा᳚ - वि॒ष्णू॒ । \newline
4. अप॑स स्पा॒रे पा॒रे अप॑सो॒ अप॑स स्पा॒रे अ॒स्यास्य पा॒रे अप॑सो॒ अप॑स स्पा॒रे अ॒स्य । \newline
5. पा॒रे अ॒स्यास्य पा॒रे पा॒रे अ॒स्य । \newline
6. अ॒स्येत्य॒स्य । \newline
7. जु॒षेथां᳚ ॅय॒ज्ञ्ं ॅय॒ज्ञ्म् जु॒षेथा᳚म् जु॒षेथां᳚ ॅय॒ज्ञ्म् द्रवि॑ण॒म् द्रवि॑णं ॅय॒ज्ञ्म् जु॒षेथा᳚म् जु॒षेथां᳚ ॅय॒ज्ञ्म् द्रवि॑णम् । \newline
8. य॒ज्ञ्म् द्रवि॑ण॒म् द्रवि॑णं ॅय॒ज्ञ्ं ॅय॒ज्ञ्म् द्रवि॑णम् च च॒ द्रवि॑णं ॅय॒ज्ञ्ं ॅय॒ज्ञ्म् द्रवि॑णम् च । \newline
9. द्रवि॑णम् च च॒ द्रवि॑ण॒म् द्रवि॑णम् च धत्तम् धत्तम् च॒ द्रवि॑ण॒म् द्रवि॑णम् च धत्तम् । \newline
10. च॒ ध॒त्त॒म् ध॒त्त॒म् च॒ च॒ ध॒त्त॒ मरि॑ष्टै॒ ररि॑ष्टैर् धत्तम् च च धत्त॒ मरि॑ष्टैः । \newline
11. ध॒त्त॒ मरि॑ष्टै॒ ररि॑ष्टैर् धत्तम् धत्त॒ मरि॑ष्टैर् नो नो॒ अरि॑ष्टैर् धत्तम् धत्त॒ मरि॑ष्टैर् नः । \newline
12. अरि॑ष्टैर् नो नो॒ अरि॑ष्टै॒ ररि॑ष्टैर् नः प॒थिभिः॑ प॒थिभि॑र् नो॒ अरि॑ष्टै॒ ररि॑ष्टैर् नः प॒थिभिः॑ । \newline
13. नः॒ प॒थिभिः॑ प॒थिभि॑र् नो नः प॒थिभिः॑ पा॒रय॑न्ता पा॒रय॑न्ता प॒थिभि॑र् नो नः प॒थिभिः॑ पा॒रय॑न्ता । \newline
14. प॒थिभिः॑ पा॒रय॑न्ता पा॒रय॑न्ता प॒थिभिः॑ प॒थिभिः॑ पा॒रय॑न्ता । \newline
15. प॒थिभि॒रिति॑ प॒थि - भिः॒ । \newline
16. पा॒रय॒न्तेति॑ पा॒रय॑न्ता । \newline
17. उ॒भा जि॑ग्यथुर् जिग्यथु रु॒भोभा जि॑ग्यथु॒र् न न जि॑ग्यथु रु॒भोभा जि॑ग्यथु॒र् न । \newline
18. जि॒ग्य॒थु॒र् न न जि॑ग्यथुर् जिग्यथु॒र् न परा॒ परा॒ न जि॑ग्यथुर् जिग्यथु॒र् न परा᳚ । \newline
19. न परा॒ परा॒ न न परा॑ जयेथे जयेथे॒ परा॒ न न परा॑ जयेथे । \newline
20. परा॑ जयेथे जयेथे॒ परा॒ परा॑ जयेथे॒ न न ज॑येथे॒ परा॒ परा॑ जयेथे॒ न । \newline
21. ज॒ये॒थे॒ न न ज॑येथे जयेथे॒ न परा॒ परा॒ न ज॑येथे जयेथे॒ न परा᳚ । \newline
22. ज॒ये॒थे॒ इति॑ जयेथे । \newline
23. न परा॒ परा॒ न न परा॑ जिग्ये जिग्ये॒ परा॒ न न परा॑ जिग्ये । \newline
24. परा॑ जिग्ये जिग्ये॒ परा॒ परा॑ जिग्ये कत॒रः क॑त॒रो जि॑ग्ये॒ परा॒ परा॑ जिग्ये कत॒रः । \newline
25. जि॒ग्ये॒ क॒त॒रः क॑त॒रो जि॑ग्ये जिग्ये कत॒र श्च॒न च॒न क॑त॒रो जि॑ग्ये जिग्ये कत॒र श्च॒न । \newline
26. क॒त॒र श्च॒न च॒न क॑त॒रः क॑त॒र श्च॒नैनो॑ रेनो श्च॒न क॑त॒रः क॑त॒र श्च॒नैनोः᳚ । \newline
27. च॒नैनो॑ रेनो श्च॒न च॒नैनोः᳚ । \newline
28. ए॒नो॒रित्ये॑नोः । \newline
29. इन्द्र॑श्च॒ चे न्द्र॒ इन्द्र॑श्च विष्णो विष्णो॒ चे न्द्र॒ इन्द्र॑श्च विष्णो । \newline
30. च॒ वि॒ष्णो॒ वि॒ष्णो॒ च॒ च॒ वि॒ष्णो॒ यद् यद् वि॑ष्णो च च विष्णो॒ यत् । \newline
31. वि॒ष्णो॒ यद् यद् वि॑ष्णो विष्णो॒ यदप॑स्पृधेथा॒ मप॑स्पृधेथां॒ ॅयद् वि॑ष्णो विष्णो॒ यदप॑स्पृधेथाम् । \newline
32. वि॒ष्णो॒ इति॑ विष्णो । \newline
33. यदप॑स्पृधेथा॒ मप॑स्पृधेथां॒ ॅयद् यदप॑स्पृधेथाम् त्रे॒धा त्रे॒धा ऽप॑स्पृधेथां॒ ॅयद् यदप॑स्पृधेथाम् त्रे॒धा । \newline
34. अप॑स्पृधेथाम् त्रे॒धा त्रे॒धा ऽप॑स्पृधेथा॒ मप॑स्पृधेथाम् त्रे॒धा स॒हस्रꣳ॑ स॒हस्र॑म् त्रे॒धा ऽप॑स्पृधेथा॒ मप॑स्पृधेथाम् त्रे॒धा स॒हस्र᳚म् । \newline
35. त्रे॒धा स॒हस्रꣳ॑ स॒हस्र॑म् त्रे॒धा त्रे॒धा स॒हस्रं॒ ॅवि वि स॒हस्र॑म् त्रे॒धा त्रे॒धा स॒हस्रं॒ ॅवि । \newline
36. स॒हस्रं॒ ॅवि वि स॒हस्रꣳ॑ स॒हस्रं॒ ॅवि तत् तद् वि स॒हस्रꣳ॑ स॒हस्रं॒ ॅवि तत् । \newline
37. वि तत् तद् वि वि तदै॑रयेथा मैरयेथा॒म् तद् वि वि तदै॑रयेथाम् । \newline
38. तदै॑रयेथा मैरयेथा॒म् तत् तदै॑रयेथाम् । \newline
39. ऐ॒र॒ये॒था॒मित्यै॑रयेथाम् । \newline
40. त्रीण्यायूꣳ॒॒ ष्यायूꣳ॑षि॒ त्रीणि॒ त्रीण्यायूꣳ॑षि॒ तव॒ तवायूꣳ॑षि॒ त्रीणि॒ त्रीण्यायूꣳ॑षि॒ तव॑ । \newline
41. आयूꣳ॑षि॒ तव॒ तवायूꣳ॒॒ ष्यायूꣳ॑षि॒ तव॑ जातवेदो जातवेद॒ स्तवायूꣳ॒॒ ष्यायूꣳ॑षि॒ तव॑ जातवेदः । \newline
42. तव॑ जातवेदो जातवेद॒ स्तव॒ तव॑ जातवेद स्ति॒स्र स्ति॒स्रो जा॑तवेद॒ स्तव॒ तव॑ जातवेद स्ति॒स्रः । \newline
43. जा॒त॒वे॒द॒ स्ति॒स्र स्ति॒स्रो जा॑तवेदो जातवेद स्ति॒स्र आ॒जानी॑ रा॒जानी᳚ स्ति॒स्रो जा॑तवेदो जातवेद स्ति॒स्र आ॒जानीः᳚ । \newline
44. जा॒त॒वे॒द॒ इति॑ जात - वे॒दः॒ । \newline
45. ति॒स्र आ॒जानी॑ रा॒जानी᳚ स्ति॒स्र स्ति॒स्र आ॒जानी॑ रु॒षस॑ उ॒षस॑ आ॒जानी᳚ स्ति॒स्र स्ति॒स्र आ॒जानी॑ रु॒षसः॑ । \newline
46. आ॒जानी॑ रु॒षस॑ उ॒षस॑ आ॒जानी॑ रा॒जानी॑ रु॒षस॑ स्ते त उ॒षस॑ आ॒जानी॑ रा॒जानी॑ रु॒षस॑ स्ते । \newline
47. आ॒जानी॒रित्या᳚ - जानीः᳚ । \newline
48. उ॒षस॑ स्ते त उ॒षस॑ उ॒षस॑ स्ते अग्ने अग्ने त उ॒षस॑ उ॒षस॑ स्ते अग्ने । \newline
49. ते॒ अ॒ग्ने॒ अ॒ग्ने॒ ते॒ ते॒ अ॒ग्ने॒ । \newline
50. अ॒ग्न॒ इत्य॑ग्ने । \newline
51. ताभि॑र् दे॒वाना᳚म् दे॒वाना॒म् ताभि॒ स्ताभि॑र् दे॒वाना॒ मवो ऽवो॑ दे॒वाना॒म् ताभि॒ स्ताभि॑र् दे॒वाना॒ मवः॑ । \newline
52. दे॒वाना॒ मवो ऽवो॑ दे॒वाना᳚म् दे॒वाना॒ मवो॑ यक्षि य॒क्ष्यवो॑ दे॒वाना᳚म् दे॒वाना॒ मवो॑ यक्षि । \newline
53. अवो॑ यक्षि य॒क्ष्यवो ऽवो॑ यक्षि वि॒द्वान्. वि॒द्वान्. य॒क्ष्यवो ऽवो॑ यक्षि वि॒द्वान् । \newline
54. य॒क्षि॒ वि॒द्वान्. वि॒द्वान्. य॑क्षि यक्षि वि॒द्वा नथाथ॑ वि॒द्वान्. य॑क्षि यक्षि वि॒द्वा नथ॑ । \newline
55. वि॒द्वा नथाथ॑ वि॒द्वान्. वि॒द्वा नथा॑ भव भ॒वाथ॑ वि॒द्वान्. वि॒द्वा नथा॑ भव । \newline
56. अथा॑ भव भ॒वाथाथा॑ भव॒ यज॑मानाय॒ यज॑मानाय भ॒वाथाथा॑ भव॒ यज॑मानाय । \newline
\pagebreak
\markright{ TS 3.2.11.3  \hfill https://www.vedavms.in \hfill}

\section{ TS 3.2.11.3 }

\textbf{TS 3.2.11.3 } \newline
\textbf{Samhita Paata} \newline

-भव॒ यज॑मानाय॒ शंॅयोः ॥ अ॒ग्निस्त्रीणि॑ त्रि॒धातू॒न्या क्षे॑ति वि॒दथा॑ क॒विः । स त्रीꣳरे॑काद॒शाꣳ इ॒ह ॥ यक्ष॑च्च पि॒प्रय॑च्च नो॒ विप्रो॑ दू॒तः परि॑ष्कृतः । नभ॑न्तामन्य॒के स॑मे ॥ इन्द्रा॑विष्णू दृꣳहि॒ताः शम्ब॑रस्य॒ नव॒ पुरो॑ नव॒तिं च॑- श्ञथिष्टं । श॒तं ॅव॒र्चिनः॑ स॒हस्रं॑ च सा॒कꣳ ह॒थो अ॑प्र॒त्यसु॑रस्य वी॒रान् ॥ उ॒त मा॒ता म॑हि॒ष मन्व॑वेनद॒मी त्वा॑ ( ) जहति पुत्र दे॒वाः । अथा᳚ब्रवीद्-वृ॒त्रमिन्द्रो॑ हनि॒ष्यन्थ्-सखे॑ विष्णो वित॒रं ॅविक्र॑मस्व ॥ \newline

\textbf{Pada Paata} \newline

भ॒व॒ । यज॑मानाय । शम् । योः ॥ अ॒ग्निः । त्रीणि॑ । त्रि॒धातू॒नीति॑ त्रि - धातू॑नि । एति॑ । क्षे॒ति॒ । वि॒दथा᳚ । क॒विः ॥ सः । त्रीन् । ए॒का॒द॒शान् । इ॒ह ॥ यक्ष॑त् । च॒ । पि॒प्रय॑त् । च॒ । नः॒ । विप्रः॑ । दू॒तः । परि॑ष्कृतः ॥ नभ॑न्ताम् । अ॒न्य॒के । स॒मे॒ ॥ इन्द्रा॑विष्णू॒ इतीन्द्रा᳚-वि॒ष्णू॒ । दृꣳ॒॒हि॒ताः । शंब॑रस्य । नव॑ । पुरः॑ । न॒व॒तिम् । च॒ । श्न॒थि॒ष्ट॒म् ॥ श॒तम् । व॒र्चिनः॑ । स॒हस्र᳚म् । च॒ । सा॒कम् । ह॒थः । अ॒प्र॒ति । असु॑रस्य । वी॒रान् ॥ उ॒त । मा॒ता । म॒हि॒षम् । अन्विति॑ । अ॒वे॒न॒त् । अ॒मी इति॑ । त्वा॒ ( ) । ज॒ह॒ति॒ । पु॒त्र॒ । दे॒वाः ॥ अथ॑ । अ॒ब्र॒वी॒त् । वृ॒त्रम् । इन्द्रः॑ । ह॒नि॒ष्यन्न् । सखे᳚ । वि॒ष्णो॒ इति॑ । वि॒त॒रमिति॑ वि - त॒रम् । वीति॑ । क्र॒म॒स्व॒ ॥  \newline


\textbf{Krama Paata} \newline

भ॒व॒ यज॑मानाय । यज॑मानाय॒ शम् । शं ॅयोः । योरिति॒ योः ॥ अ॒ग्निस्त्रीणि॑ । त्रीणि॑ त्रि॒धातू॑नि । त्रि॒धातू॒न्या । त्रि॒धातू॒नीति॑ त्रि - धातू॑नि । आ क्षे॑ति । क्षे॒ति॒ वि॒दथा᳚ । वि॒दथा॑ क॒विः । क॒विरिति॑ क॒विः ॥ स त्रीन् । त्रीꣳरे॑काद॒शान् । ए॒का॒द॒शाꣳ इ॒ह । इ॒हेती॒ह ॥ यक्ष॑च् च । च॒ पि॒प्रय॑त् । पि॒प्रय॑च् च । च॒ नः॒ । नो॒ विप्रः॑ । विप्रो॑ दू॒तः । दू॒तः परि॑ष्कृतः । परि॑ष्कृत॒ इति॒ परि॑ष्कृतः ॥ नभ॑न्तामन्य॒के । अ॒न्य॒के स॑मे । स॒म॒ इति॑ समे ॥ इन्द्रा॑विष्णू दृꣳहि॒ताः । इन्द्रा॑विष्णू॒ इतीन्द्रा᳚ - वि॒ष्णू॒ । दृꣳ॒॒हि॒ताः शम्ब॑रस्य । शम्ब॑रस्य॒ नव॑ । नव॒ पुरः॑ । पुरो॑ नव॒तिम् । न॒व॒तिम् च॑ । च॒ श्ञ॒थि॒ष्ट॒म् । श्ञ॒थि॒ष्ट॒मिति॑ श्ञथिष्टम् ॥ श॒तं ॅव॒र्चिनः॑ । व॒र्चिनः॑ स॒हस्र᳚म् । स॒हस्र॑म् च । च॒ सा॒कम् । सा॒कꣳ ह॒थः । ह॒थो अ॑प्र॒ति । अ॒प्र॒त्यसु॑रस्य । असु॑रस्य वी॒रान् । वी॒रानिति॑ वी॒रान् ॥ उ॒त मा॒ता । मा॒ता म॑हि॒षम् । म॒हि॒षमनु॑ । अन्व॑वेनत् । अ॒वे॒न॒द॒मी । अ॒मी त्वा᳚ ( ) । अ॒मी इत्य॒मी । त्वा॒ ज॒ह॒ति॒ । ज॒ह॒ति॒ पु॒त्र॒ । पु॒त्र॒ दे॒वाः । दे॒वा इति॑ दे॒वाः ॥ अथा᳚ब्रवीत् । अ॒ब्र॒वी॒द् वृ॒त्रम् । वृ॒त्रमिन्द्रः॑ । इन्द्रो॑ हनि॒ष्यन्न् । ह॒नि॒ष्यन्थ् सखे᳚ । सखे॑ विष्णो । वि॒ष्णो॒ वि॒त॒रम् । वि॒ष्णो॒ इति॑ विष्णो । वि॒त॒रं ॅवि । वि॒त॒रमिति॑ वि - त॒रम् । वि क्र॑मस्व । क्र॒म॒स्वेति॑ क्रमस्व । \newline

\textbf{Jatai Paata} \newline

1. भ॒व॒ यज॑मानाय॒ यज॑मानाय भव भव॒ यज॑मानाय । \newline
2. यज॑मानाय॒ शꣳ शं ॅयज॑मानाय॒ यज॑मानाय॒ शम् । \newline
3. शं ॅयोर् योः शꣳ शं ॅयोः । \newline
4. योरिति॒ योः । \newline
5. अ॒ग्नि स्त्रीणि॒ त्रीण्य॒ग्नि र॒ग्नि स्त्रीणि॑ । \newline
6. त्रीणि॑ त्रि॒धातू॑नि त्रि॒धातू॑नि॒ त्रीणि॒ त्रीणि॑ त्रि॒धातू॑नि । \newline
7. त्रि॒धातू॒न्या त्रि॒धातू॑नि त्रि॒धातू॒न्या । \newline
8. त्रि॒धातू॒नीति॑ त्रि - धातू॑नि । \newline
9. आ क्षे॑ति क्षे॒त्या क्षे॑ति । \newline
10. क्षे॒ति॒ वि॒दथा॑ वि॒दथा᳚ क्षेति क्षेति वि॒दथा᳚ । \newline
11. वि॒दथा॑ क॒विः क॒विर् वि॒दथा॑ वि॒दथा॑ क॒विः । \newline
12. क॒विरिति॑ क॒विः । \newline
13. स त्रीꣳ स्त्रीन् थ्स स त्रीन् । \newline
14. त्रीꣳ रे॑काद॒शाꣳ ए॑काद॒शान् त्रीꣳ स्त्रीꣳ रे॑काद॒शान् । \newline
15. ए॒का॒द॒शाꣳ इ॒हे हैका॑द॒शाꣳ ए॑काद॒शाꣳ इ॒ह । \newline
16. इ॒हेती॒ह । \newline
17. यक्ष॑च् च च॒ यक्ष॒द् यक्ष॑च् च । \newline
18. च॒ पि॒प्रय॑त् पि॒प्रय॑च् च च पि॒प्रय॑त् । \newline
19. पि॒प्रय॑च् च च पि॒प्रय॑त् पि॒प्रय॑च् च । \newline
20. च॒ नो॒ न॒श्च॒ च॒ नः॒ । \newline
21. नो॒ विप्रो॒ विप्रो॑ नो नो॒ विप्रः॑ । \newline
22. विप्रो॑ दू॒तो दू॒तो विप्रो॒ विप्रो॑ दू॒तः । \newline
23. दू॒तः परि॑ष्कृतः॒ परि॑ष्कृतो दू॒तो दू॒तः परि॑ष्कृतः । \newline
24. परि॑ष्कृत॒ इति॒ परि॑ष्कृतः । \newline
25. नभ॑न्ता मन्य॒के अ॑न्य॒के नभ॑न्ता॒म् नभ॑न्ता मन्य॒के । \newline
26. अ॒न्य॒के स॑मे समे अन्य॒के अ॑न्य॒के स॑मे । \newline
27. स॒म॒ इति॑ समे । \newline
28. इन्द्रा॑विष्णू दृꣳहि॒ता दृꣳ॑हि॒ता इन्द्रा॑विष्णू॒ इन्द्रा॑विष्णू दृꣳहि॒ताः । \newline
29. इन्द्रा॑विष्णू॒ इतीन्द्रा᳚ - वि॒ष्णू॒ । \newline
30. दृꣳ॒॒हि॒ताः शम्ब॑रस्य॒ शम्ब॑रस्य दृꣳहि॒ता दृꣳ॑हि॒ताः शम्ब॑रस्य । \newline
31. शम्ब॑रस्य॒ नव॒ नव॒ शम्ब॑रस्य॒ शम्ब॑रस्य॒ नव॑ । \newline
32. नव॒ पुरः॒ पुरो॒ नव॒ नव॒ पुरः॑ । \newline
33. पुरो॑ नव॒तिम् न॑व॒तिम् पुरः॒ पुरो॑ नव॒तिम् । \newline
34. न॒व॒तिम् च॑ च नव॒तिम् न॑व॒तिम् च॑ । \newline
35. च॒ श्ञ॒थि॒ष्टꣳ॒॒ श्ञ॒थि॒ष्ट॒म् च॒ च॒ श्ञ॒थि॒ष्ट॒म् । \newline
36. श्ञ॒थि॒ष्ट॒मिति॑ श्ञथिष्टम् । \newline
37. श॒तं ॅव॒र्चिनो॑ व॒र्चिनः॑ श॒तꣳ श॒तं ॅव॒र्चिनः॑ । \newline
38. व॒र्चिनः॑ स॒हस्रꣳ॑ स॒हस्रं॑ ॅव॒र्चिनो॑ व॒र्चिनः॑ स॒हस्र᳚म् । \newline
39. स॒हस्र॑म् च च स॒हस्रꣳ॑ स॒हस्र॑म् च । \newline
40. च॒ सा॒कꣳ सा॒कम् च॑ च सा॒कम् । \newline
41. सा॒कꣳ ह॒थो ह॒थः सा॒कꣳ सा॒कꣳ ह॒थः । \newline
42. ह॒थो अ॑प्र॒ त्य॑प्र॒ति ह॒थो ह॒थो अ॑प्र॒ति । \newline
43. अ॒प्र॒ त्यसु॑र॒स्या सु॑रस्याप्र॒ त्य॑प्र॒ त्यसु॑रस्य । \newline
44. असु॑रस्य वी॒रान्. वी॒रा नसु॑र॒स्या सु॑रस्य वी॒रान् । \newline
45. वी॒रानिति॑ वी॒रान् । \newline
46. उ॒त मा॒ता मा॒तोतोत मा॒ता । \newline
47. मा॒ता म॑हि॒षम् म॑हि॒षम् मा॒ता मा॒ता म॑हि॒षम् । \newline
48. म॒हि॒ष मन्वनु॑ महि॒षम् म॑हि॒ष मनु॑ । \newline
49. अन्व॑वेन दवेन॒ दन्वन्व॑ वेनत् । \newline
50. अ॒वे॒न॒ द॒मी अ॒मी अ॑वेन दवेन द॒मी । \newline
51. अ॒मी त्वा᳚ त्वा॒ ऽमी अ॒मी त्वा᳚ । \newline
52. अ॒मी इत्य॒मी । \newline
53. त्वा॒ ज॒ह॒ति॒ ज॒ह॒ति॒ त्वा॒ त्वा॒ ज॒ह॒ति॒ । \newline
54. ज॒ह॒ति॒ पु॒त्र॒ पु॒त्र॒ ज॒ह॒ति॒ ज॒ह॒ति॒ पु॒त्र॒ । \newline
55. पु॒त्र॒ दे॒वा दे॒वाः पु॑त्र पुत्र दे॒वाः । \newline
56. दे॒वा इति॑ दे॒वाः । \newline
57. अथा᳚ब्रवी दब्रवी॒ दथाथा᳚ ब्रवीत् । \newline
58. अ॒ब्र॒वी॒द् वृ॒त्रं ॅवृ॒त्र म॑ब्रवी दब्रवीद् वृ॒त्रम् । \newline
59. वृ॒त्र मिन्द्र॒ इन्द्रो॑ वृ॒त्रं ॅवृ॒त्र मिन्द्रः॑ । \newline
60. इन्द्रो॑ हनि॒ष्यन्. ह॑नि॒ष्यन् निन्द्र॒ इन्द्रो॑ हनि॒ष्यन्न् । \newline
61. ह॒नि॒ष्यन् थ्सखे॒ सखे॑ हनि॒ष्यन्. ह॑नि॒ष्यन् थ्सखे᳚ । \newline
62. सखे॑ विष्णो विष्णो॒ सखे॒ सखे॑ विष्णो । \newline
63. वि॒ष्णो॒ वि॒त॒रं ॅवि॑त॒रं ॅवि॑ष्णो विष्णो वित॒रम् । \newline
64. वि॒ष्णो॒ इति॑ विष्णो । \newline
65. वि॒त॒रं ॅवि वि वि॑त॒रं ॅवि॑त॒रं ॅवि । \newline
66. वि॒त॒रमिति॑ वि - त॒रम् । \newline
67. वि क्र॑मस्व क्रमस्व॒ वि वि क्र॑मस्व । \newline
68. क्र॒म॒स्वेति॑ क्रमस्व । \newline

\textbf{Ghana Paata } \newline

1. भ॒व॒ यज॑मानाय॒ यज॑मानाय भव भव॒ यज॑मानाय॒ शꣳ शं ॅयज॑मानाय भव भव॒ यज॑मानाय॒ शम् । \newline
2. यज॑मानाय॒ शꣳ शं ॅयज॑मानाय॒ यज॑मानाय॒ शं ॅयोर् योः शं ॅयज॑मानाय॒ यज॑मानाय॒ शं ॅयोः । \newline
3. शं ॅयोर् योः शꣳ शं ॅयोः । \newline
4. योरिति॒ योः । \newline
5. अ॒ग्नि स्त्रीणि॒ त्रीण्य॒ग्नि र॒ग्नि स्त्रीणि॑ त्रि॒धातू॑नि त्रि॒धातू॑नि॒ त्रीण्य॒ग्नि र॒ग्नि स्त्रीणि॑ त्रि॒धातू॑नि । \newline
6. त्रीणि॑ त्रि॒धातू॑नि त्रि॒धातू॑नि॒ त्रीणि॒ त्रीणि॑ त्रि॒धातू॒ न्या त्रि॒धातू॑नि॒ त्रीणि॒ त्रीणि॑ त्रि॒धातू॒ न्या । \newline
7. त्रि॒धातू॒ न्या त्रि॒धातू॑नि त्रि॒धातू॒ न्या क्षे॑ति क्षे॒त्या त्रि॒धातू॑नि त्रि॒धातू॒ न्या क्षे॑ति । \newline
8. त्रि॒धातू॒नीति॑ त्रि - धातू॑नि । \newline
9. आ क्षे॑ति क्षे॒त्या क्षे॑ति वि॒दथा॑ वि॒दथा᳚ क्षे॒त्या क्षे॑ति वि॒दथा᳚ । \newline
10. क्षे॒ति॒ वि॒दथा॑ वि॒दथा᳚ क्षेति क्षेति वि॒दथा॑ क॒विः क॒विर् वि॒दथा᳚ क्षेति क्षेति वि॒दथा॑ क॒विः । \newline
11. वि॒दथा॑ क॒विः क॒विर् वि॒दथा॑ वि॒दथा॑ क॒विः । \newline
12. क॒विरिति॑ क॒विः । \newline
13. स त्रीꣳ स्त्रीन् थ्स स त्रीꣳ रे॑काद॒शाꣳ ए॑काद॒शान् त्रीन् थ्स स त्रीꣳ रे॑काद॒शान् । \newline
14. त्रीꣳ रे॑काद॒शाꣳ ए॑काद॒शान् त्रीꣳ स्त्रीꣳ रे॑काद॒शाꣳ इ॒हे हैका॑द॒शान् त्रीꣳ स्त्रीꣳ रे॑काद॒शाꣳ इ॒ह । \newline
15. ए॒का॒द॒शाꣳ इ॒हे हैका॑द॒शाꣳ ए॑काद॒शाꣳ इ॒ह । \newline
16. इ॒हेती॒ह । \newline
17. यक्ष॑च् च च॒ यक्ष॒द् यक्ष॑च् च पि॒प्रय॑त् पि॒प्रय॑च् च॒ यक्ष॒द् यक्ष॑च् च पि॒प्रय॑त् । \newline
18. च॒ पि॒प्रय॑त् पि॒प्रय॑च् च च पि॒प्रय॑च् च च पि॒प्रय॑च् च च पि॒प्रय॑च् च । \newline
19. पि॒प्रय॑च् च च पि॒प्रय॑त् पि॒प्रय॑च् च नो नश्च पि॒प्रय॑त् पि॒प्रय॑च् च नः । \newline
20. च॒ नो॒ न॒श्च॒ च॒ नो॒ विप्रो॒ विप्रो॑ नश्च च नो॒ विप्रः॑ । \newline
21. नो॒ विप्रो॒ विप्रो॑ नो नो॒ विप्रो॑ दू॒तो दू॒तो विप्रो॑ नो नो॒ विप्रो॑ दू॒तः । \newline
22. विप्रो॑ दू॒तो दू॒तो विप्रो॒ विप्रो॑ दू॒तः परि॑ष्कृतः॒ परि॑ष्कृतो दू॒तो विप्रो॒ विप्रो॑ दू॒तः परि॑ष्कृतः । \newline
23. दू॒तः परि॑ष्कृतः॒ परि॑ष्कृतो दू॒तो दू॒तः परि॑ष्कृतः । \newline
24. परि॑ष्कृत॒ इति॒ परि॑ष्कृतः । \newline
25. नभ॑न्ता मन्य॒के अ॑न्य॒के नभ॑न्ता॒म् नभ॑न्ता मन्य॒के स॑मे समे अन्य॒के नभ॑न्ता॒म् नभ॑न्ता मन्य॒के स॑मे । \newline
26. अ॒न्य॒के स॑मे समे अन्य॒के अ॑न्य॒के स॑मे । \newline
27. स॒म॒ इति॑ समे । \newline
28. इन्द्रा॑विष्णू दृꣳहि॒ता दृꣳ॑हि॒ता इन्द्रा॑विष्णू॒ इन्द्रा॑विष्णू दृꣳहि॒ताः शम्ब॑रस्य॒ शम्ब॑रस्य दृꣳहि॒ता इन्द्रा॑विष्णू॒ इन्द्रा॑विष्णू दृꣳहि॒ताः शम्ब॑रस्य । \newline
29. इन्द्रा॑विष्णू॒ इतीन्द्रा᳚ - वि॒ष्णू॒ । \newline
30. दृꣳ॒॒हि॒ताः शम्ब॑रस्य॒ शम्ब॑रस्य दृꣳहि॒ता दृꣳ॑हि॒ताः शम्ब॑रस्य॒ नव॒ नव॒ शम्ब॑रस्य दृꣳहि॒ता दृꣳ॑हि॒ताः शम्ब॑रस्य॒ नव॑ । \newline
31. शम्ब॑रस्य॒ नव॒ नव॒ शम्ब॑रस्य॒ शम्ब॑रस्य॒ नव॒ पुरः॒ पुरो॒ नव॒ शम्ब॑रस्य॒ शम्ब॑रस्य॒ नव॒ पुरः॑ । \newline
32. नव॒ पुरः॒ पुरो॒ नव॒ नव॒ पुरो॑ नव॒तिम् न॑व॒तिम् पुरो॒ नव॒ नव॒ पुरो॑ नव॒तिम् । \newline
33. पुरो॑ नव॒तिम् न॑व॒तिम् पुरः॒ पुरो॑ नव॒तिम् च॑ च नव॒तिम् पुरः॒ पुरो॑ नव॒तिम् च॑ । \newline
34. न॒व॒तिम् च॑ च नव॒तिम् न॑व॒तिम् च॑ श्ञथिष्टꣳ श्ञथिष्टम् च नव॒तिम् न॑व॒तिम् च॑ श्ञथिष्टम् । \newline
35. च॒ श्ञ॒थि॒ष्टꣳ॒॒ श्ञ॒थि॒ष्ट॒म् च॒ च॒ श्ञ॒थि॒ष्ट॒म् । \newline
36. श्ञ॒थि॒ष्ट॒मिति॑ श्ञथिष्टम् । \newline
37. श॒तं ॅव॒र्चिनो॑ व॒र्चिनः॑ श॒तꣳ श॒तं ॅव॒र्चिनः॑ स॒हस्रꣳ॑ स॒हस्रं॑ ॅव॒र्चिनः॑ श॒तꣳ श॒तं ॅव॒र्चिनः॑ स॒हस्र᳚म् । \newline
38. व॒र्चिनः॑ स॒हस्रꣳ॑ स॒हस्रं॑ ॅव॒र्चिनो॑ व॒र्चिनः॑ स॒हस्र॑म् च च स॒हस्रं॑ ॅव॒र्चिनो॑ व॒र्चिनः॑ स॒हस्र॑म् च । \newline
39. स॒हस्र॑म् च च स॒हस्रꣳ॑ स॒हस्र॑म् च सा॒कꣳ सा॒कम् च॑ स॒हस्रꣳ॑ स॒हस्र॑म् च सा॒कम् । \newline
40. च॒ सा॒कꣳ सा॒कम् च॑ च सा॒कꣳ ह॒थो ह॒थः सा॒कम् च॑ च सा॒कꣳ ह॒थः । \newline
41. सा॒कꣳ ह॒थो ह॒थः सा॒कꣳ सा॒कꣳ ह॒थो अ॑प्र॒त्य॑प्र॒ति ह॒थः सा॒कꣳ सा॒कꣳ ह॒थो अ॑प्र॒ति । \newline
42. ह॒थो अ॑प्र॒ त्य॑प्र॒ति ह॒थो ह॒थो अ॑प्र॒ त्यसु॑र॒स्या सु॑रस्या प्र॒ति ह॒थो ह॒थो अ॑प्र॒ त्यसु॑रस्य । \newline
43. अ॒प्र॒ त्यसु॑र॒स्या सु॑रस्या प्र॒त्य॑ प्र॒त्यसु॑रस्य वी॒रान्. वी॒रा नसु॑रस्या प्र॒त्य॑ प्र॒त्यसु॑रस्य वी॒रान् । \newline
44. असु॑रस्य वी॒रान्. वी॒रा नसु॑र॒स्या सु॑रस्य वी॒रान् । \newline
45. वी॒रानिति॑ वी॒रान् । \newline
46. उ॒त मा॒ता मा॒तोतोत मा॒ता म॑हि॒षम् म॑हि॒षम् मा॒तोतोत मा॒ता म॑हि॒षम् । \newline
47. मा॒ता म॑हि॒षम् म॑हि॒षम् मा॒ता मा॒ता म॑हि॒ष मन्वनु॑ महि॒षम् मा॒ता मा॒ता म॑हि॒ष मनु॑ । \newline
48. म॒हि॒ष मन्वनु॑ महि॒षम् म॑हि॒ष मन्व॑वे नदवेन॒ दनु॑ महि॒षम् म॑हि॒ष मन्व॑वेनत् । \newline
49. अन्व॑वेन दवेन॒ दन्वन्व॑ वेनद॒मी अ॒मी अ॑वेन॒ दन्वन्व॑वेन द॒मी । \newline
50. अ॒वे॒न॒द॒मी अ॒मी अ॑वेन दवेन द॒मी त्वा᳚ त्वा॒ ऽमी अ॑वेन दवेन द॒मी त्वा᳚ । \newline
51. अ॒मी त्वा᳚ त्वा॒ ऽमी अ॒मी त्वा॑ जहति जहति त्वा॒ ऽमी अ॒मी त्वा॑ जहति । \newline
52. अ॒मी इत्य॒मी । \newline
53. त्वा॒ ज॒ह॒ति॒ ज॒ह॒ति॒ त्वा॒ त्वा॒ ज॒ह॒ति॒ पु॒त्र॒ पु॒त्र॒ ज॒ह॒ति॒ त्वा॒ त्वा॒ ज॒ह॒ति॒ पु॒त्र॒ । \newline
54. ज॒ह॒ति॒ पु॒त्र॒ पु॒त्र॒ ज॒ह॒ति॒ ज॒ह॒ति॒ पु॒त्र॒ दे॒वा दे॒वाः पु॑त्र जहति जहति पुत्र दे॒वाः । \newline
55. पु॒त्र॒ दे॒वा दे॒वाः पु॑त्र पुत्र दे॒वाः । \newline
56. दे॒वा इति॑ दे॒वाः । \newline
57. अथा᳚ब्रवी दब्रवी॒ दथाथा᳚ ब्रवीद् वृ॒त्रं ॅवृ॒त्र म॑ब्रवी॒ दथाथा᳚ ब्रवीद् वृ॒त्रम् । \newline
58. अ॒ब्र॒वी॒द् वृ॒त्रं ॅवृ॒त्र म॑ब्रवी दब्रवीद् वृ॒त्र मिन्द्र॒ इन्द्रो॑ वृ॒त्र म॑ब्रवी दब्रवीद् वृ॒त्र मिन्द्रः॑ । \newline
59. वृ॒त्र मिन्द्र॒ इन्द्रो॑ वृ॒त्रं ॅवृ॒त्र मिन्द्रो॑ हनि॒ष्यन्. ह॑नि॒ष्यन् निन्द्रो॑ वृ॒त्रं ॅवृ॒त्र मिन्द्रो॑ हनि॒ष्यन्न् । \newline
60. इन्द्रो॑ हनि॒ष्यन्. ह॑नि॒ष्यन् निन्द्र॒ इन्द्रो॑ हनि॒ष्यन् थ्सखे॒ सखे॑ हनि॒ष्यन् निन्द्र॒ इन्द्रो॑ हनि॒ष्यन् थ्सखे᳚ । \newline
61. ह॒नि॒ष्यन् थ्सखे॒ सखे॑ हनि॒ष्यन्. ह॑नि॒ष्यन् थ्सखे॑ विष्णो विष्णो॒ सखे॑ हनि॒ष्यन्. ह॑नि॒ष्यन् थ्सखे॑ विष्णो । \newline
62. सखे॑ विष्णो विष्णो॒ सखे॒ सखे॑ विष्णो वित॒रं ॅवि॑त॒रं ॅवि॑ष्णो॒ सखे॒ सखे॑ विष्णो वित॒रम् । \newline
63. वि॒ष्णो॒ वि॒त॒रं ॅवि॑त॒रं ॅवि॑ष्णो विष्णो वित॒रं ॅवि वि वि॑त॒रं ॅवि॑ष्णो विष्णो वित॒रं ॅवि । \newline
64. वि॒ष्णो॒ इति॑ विष्णो । \newline
65. वि॒त॒रं ॅवि वि वि॑त॒रं ॅवि॑त॒रं ॅवि क्र॑मस्व क्रमस्व॒ वि वि॑त॒रं ॅवि॑त॒रं ॅवि क्र॑मस्व । \newline
66. वि॒त॒रमिति॑ वि - त॒रम् । \newline
67. वि क्र॑मस्व क्रमस्व॒ वि वि क्र॑मस्व । \newline
68. क्र॒म॒स्वेति॑ क्रमस्व । \newline
\pagebreak


\end{document}