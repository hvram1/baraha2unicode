\documentclass[17pt]{extarticle}
\usepackage{babel}
\usepackage{fontspec}
\usepackage{polyglossia}
\usepackage{extsizes}

\usepackage{color}   %May be necessary if you want to color links
\usepackage{hyperref}
\hypersetup{
    colorlinks=true, %set true if you want colored links
    linktoc=all,     %set to all if you want both sections and subsections linked
    linkcolor=black,  %choose some color if you want links to stand out
}

\setmainlanguage{sanskrit}
\setotherlanguages{english} %% or other languages
\setlength{\parindent}{0pt}
\pagestyle{myheadings}
\newfontfamily\devanagarifont[Script=Devanagari]{AdishilaVedic}
\renewcommand{\theHsection}{\thepart.section.\thesection}

\newcommand{\VAR}[1]{}
\newcommand{\BLOCK}[1]{}




\begin{document}
\begin{titlepage}
    \begin{center}
 
\begin{sanskrit}
    { \Large
    कृष्ण यजुर्वेदीय तैत्तिरीय संहिता,पद,जटा,घन पाठः 
    }
    \\
    \vspace{2.5cm}
    \mbox{ \Large
    3.2     तृतीयकाण्डे द्वितीयः प्रश्नः - पवमानग्राहादीनां व्याख्Yआनं   }
\end{sanskrit}
\end{center}

\end{titlepage}
\tableofcontents
\phantomsection
\pagebreak

\markright{ TS 3.2.1.1  \hfill https://www.vedavms.in \hfill}

\section{ TS 3.2.1.1 }

\textbf{TS 3.2.1.1 } \newline
\textbf{Samhita Paata} \newline

यो वै पव॑मानानामन्वारो॒हान्. वि॒द्वान्. यज॒तेऽनु॒ पव॑माना॒ना रो॑हति॒ न पव॑माने॒भ्यो-ऽव॑ च्छिद्यते श्ये॒नो॑ऽसि गाय॒त्रछ॑न्दा॒ अनु॒ त्वाऽऽर॑भे स्व॒स्ति मा॒ सं पा॑रय सुप॒र्णो॑ऽसि त्रि॒ष्टुप्छ॑न्दा॒ अनु॒ त्वाऽऽर॑भे स्व॒स्ति मा॒ सं पा॑रय॒ सघा॑ऽसि॒ जग॑तीछन्दा॒ अनु॒ त्वाऽऽर॑भे स्व॒स्ति मा॒ संपा॑र॒येत्या॑है॒ते- [  ] \newline

\textbf{Pada Paata} \newline

यः । वै । पव॑मानानाम् । अ॒न्वा॒रो॒हानित्य॑नु - आ॒रो॒हान् । वि॒द्वान् । यज॑ते । अन्विति॑ । पव॑मानान् । एति॑ । रो॒ह॒ति॒ । न । पव॑मानेभ्यः । अवेति॑ । छि॒द्य॒ते॒ । श्ये॒नः । अ॒सि॒ । गा॒य॒त्रछ॑न्दा॒ इति॑ गाय॒त्र-छ॒न्दाः॒ । अन्विति॑ । त्वा॒ । एति॑ । र॒भे॒ । स्व॒स्ति । मा॒ । समिति॑ । पा॒र॒य॒ । सु॒प॒र्ण इति॑ सु - प॒र्णः । अ॒सि॒ । त्रि॒ष्टुप्छ॑न्दा॒ इति॑ त्रि॒ष्टुप् - छ॒न्दाः॒ । अन्विति॑ । त्वा॒ । एति॑ । र॒भे॒ । स्व॒स्ति । मा॒ । समिति॑ । पा॒र॒य॒ । सघा᳚ । अ॒सि॒ । जग॑तीछन्दा॒ इति॒ जग॑ती - छ॒न्दाः॒ । अन्विति॑ । त्वा॒ । एति॑ । र॒भे॒ । स्व॒स्ति । मा॒ । समिति॑ । पा॒र॒य॒ । इति॑ । आ॒ह॒ । ए॒ते ।  \newline




\markright{ TS 3.2.1.2  \hfill https://www.vedavms.in \hfill}

\section{ TS 3.2.1.2 }

\textbf{TS 3.2.1.2 } \newline
\textbf{Samhita Paata} \newline

वै पव॑मानानामन्वारो॒हास्तान्. य ए॒वं ॅवि॒द्वान्. यज॒तेऽनु॒ पव॑माना॒ना रो॑हति॒ न पव॑माने॒भ्योऽव॑ च्छिद्यते॒ यो वै पव॑मानस्य॒ सन्त॑तिं॒ ॅवेद॒ सर्व॒मायु॑रेति॒ न पु॒राऽऽयु॑षः॒ प्र मी॑यते पशु॒मान् भ॑वति वि॒न्दते᳚ प्र॒जां पव॑मानस्य॒ ग्रहा॑ गृह्य॒न्तेऽथ॒ वा अ॑स्यै॒तेऽगृ॑हीता द्रोणकल॒श आ॑धव॒नीयः॑ पूत॒भृत् तान्. यदगृ॑हीत्वोपाकु॒र्यात् पव॑मानं॒ ॅवि - [  ] \newline

\textbf{Pada Paata} \newline

वै । पव॑मानानाम् । अ॒न्वा॒रो॒हा इत्य॑नु - आ॒रो॒हाः । तान् । यः । ए॒वम् । वि॒द्वान् । यज॑ते । अन्विति॑ । पव॑मानान् । एति॑ । रो॒ह॒ति॒ । न । पव॑मानेभ्यः । अवेति॑ । छि॒द्य॒ते॒ । यः । वै । पव॑मानस्य । संत॑ति॒मिति॒ सं - त॒ति॒म् । वेद॑ । सर्व᳚म् । आयुः॑ । ए॒ति॒ । न । पु॒रा । आयु॑षः । प्रेति॑ । मी॒य॒ते॒ । प॒शु॒मानिति॑ पशु-मान् । भ॒व॒ति॒ । वि॒न्दते᳚ । प्र॒जामिति॑ प्र - जाम् । पव॑मानस्य । ग्रहाः᳚ । गृ॒ह्य॒न्ते॒ । अथ॑ । वै । अ॒स्य॒ । ए॒ते । अगृ॑हीताः । द्रो॒ण॒क॒ल॒श इति॑ द्रोण - क॒ल॒शः । आ॒ध॒व॒नीय॒ इत्या᳚ - ध॒व॒नीयः॑ । पू॒त॒भृदिति॑ पूत-भृत् । तान् । यत् । अगृ॑हीत्वा । उ॒पा॒कु॒र्यादित्यु॑प - आ॒कु॒र्यात् । पव॑मानम् । वीति॑ ।  \newline




\markright{ TS 3.2.1.3  \hfill https://www.vedavms.in \hfill}

\section{ TS 3.2.1.3 }

\textbf{TS 3.2.1.3 } \newline
\textbf{Samhita Paata} \newline

च्छि॑न्द्या॒त् तं ॅवि॒च्छिद्य॑मानमद्ध्व॒र्योः प्रा॒णोऽनु॒ विच्छि॑द्ये-तोपया॒मगृ॑हीतोऽसि प्र॒जाप॑तये॒ त्वेति॑ द्रोणकल॒शम॒भि मृ॑शे॒दिन्द्रा॑य॒ त्वेत्या॑धव॒नीयं॒ ॅविश्वे᳚भ्यस्त्वा दे॒वेभ्य॒ इति॑ पूत॒भृतं॒ पव॑मानमे॒व तथ् सं त॑नोति॒ सर्व॒मायु॑रेति॒ न पु॒राऽऽयु॑षः॒ प्रमी॑यते पशु॒मान् भ॑वति वि॒न्दते᳚ प्र॒जां ॥ \newline

\textbf{Pada Paata} \newline

छि॒न्द्या॒त् । तम् । वि॒च्छिद्य॑मान॒मिति॑ वि - छिद्य॑मानम् । अ॒द्ध्व॒र्योः । प्रा॒ण इति॑ प्र -अ॒नः । अनु॑ । वीति॑ । छि॒द्ये॒त॒ । उ॒प॒या॒मगृ॑हीत॒ इत्युप॑या॒म - गृ॒ही॒तः॒ । अ॒सि॒ । प्र॒जाप॑तय॒ इति॑ प्र॒जा-प॒त॒ये॒ । त्वा॒ । इति॑ । द्रो॒ण॒क॒ल॒शमिति॑ द्रोण - क॒ल॒शम् । अ॒भीति॑ । मृ॒शे॒त् । इन्द्रा॑य । त्वा॒ । इति॑ । आ॒ध॒व॒नीय॒मित्या᳚ - ध॒व॒नीय᳚म् । विश्वे᳚भ्यः । त्वा॒ । दे॒वेभ्यः॑ । इति॑ । पू॒त॒भृत॒मिति॑ पूत - भृत᳚म् । पव॑मानम् । ए॒व । तत् । समिति॑ । त॒नो॒ति॒ । सर्व᳚म् । आयुः॑ । ए॒ति॒ । न । पु॒रा । आयु॑षः । प्रेति॑ । मी॒य॒ते॒ । प॒शु॒मानिति॑ पशु - मान् । भ॒व॒ति॒ । वि॒न्दते᳚ । प्र॒जामिति॑ प्र - जाम् ॥  \newline




\markright{ TS 3.2.2.1  \hfill https://www.vedavms.in \hfill}

\section{ TS 3.2.2.1 }

\textbf{TS 3.2.2.1 } \newline
\textbf{Samhita Paata} \newline

त्रीणि॒ वाव सव॑ना॒न्यथ॑ तृ॒तीयꣳ॒॒ सव॑न॒मव॑ लुम्पन्त्यनꣳ॒॒शु कु॒र्वन्त॑ उपाꣳ॒॒शुꣳहु॒त्वोपाꣳ॑शुपा॒त्रेऽꣳ॑शुम॒वास्य॒ तन्तृ॑तीयसव॒ने॑ ऽपि॒सृज्या॒भि षु॑णुया॒द्यदा᳚प्या॒यय॑ति॒ तेनाꣳ॑शु॒मद्यद॑भिषु॒णोति॒ तेन॑र्जी॒षि सर्वा᳚ण्ये॒व तथ् सव॑नान्यꣳशु॒मन्ति॑ शु॒क्रव॑न्ति स॒माव॑द्वीर्याणि करोति॒ द्वौ स॑मु॒द्रौ वित॑तावजू॒र्यौ प॒र्याव॑र्तेते ज॒ठरे॑व॒ पादाः᳚ । तयोः॒ पश्य॑न्तो॒ अति॑ यन्त्य॒न्यमप॑श्यन्तः॒ - [  ] \newline

\textbf{Pada Paata} \newline

त्रीणि॑ । वाव । सव॑नानि । अथ॑ । तृ॒तीय᳚म् । सव॑नम् । अवेति॑ । लु॒पं॒न्ति॒ । अ॒नꣳ॒॒शु । कु॒र्वन्तः॑ । उ॒पाꣳ॒॒शुमित्यु॑प - अꣳ॒॒शुम् । हु॒त्वा । उ॒पाꣳ॒॒शु॒पा॒त्र इत्यु॑पाꣳशु - पा॒त्रे । अꣳ॒॒शुम् । अ॒वास्येत्य॑व - अस्य॑ । तम् । तृ॒ती॒य॒स॒व॒न इति॑ तृतीय - स॒व॒ने । अ॒पि॒सृज्येत्य॑पि - सृज्य॑ । अ॒भीति॑ । सु॒नु॒या॒त् । यत् । आ॒प्या॒यय॒तीया᳚ - प्या॒यय॑ति । तेन॑ । अꣳ॒॒शु॒मदित्यꣳ॑शु-मत् । यत् । अ॒भि॒षु॒णोतीत्य॑भि-सु॒नोति॑ । तेन॑ । ऋ॒जी॒षि । सर्वा॑णि । ए॒व । तत् । सव॑नानि । अꣳ॒॒शु॒मन्तीत्यꣳ॑शु - मन्ति॑ । शु॒क्रव॒न्तीति॑ शु॒क्र - व॒न्ति॒ । स॒माव॑द् वीर्या॒णीति॑ स॒माव॑त् - वी॒र्या॒णि॒ । क॒रो॒ति॒ । द्वौ । स॒मु॒द्रौ । वित॑ता॒विति॒ वि - त॒तौ॒ । अ॒जू॒र्यौ । प॒र्याव॑र्तेते॒ इति॑ परि-आव॑र्तेते । ज॒ठरा᳚ । इ॒व॒ । पादाः᳚ ॥ तयोः᳚ । पश्य॑न्तः । अतीति॑ । य॒न्ति॒ । अ॒न्यम् । अप॑श्यन्तः ।  \newline




\markright{ TS 3.2.2.2  \hfill https://www.vedavms.in \hfill}

\section{ TS 3.2.2.2 }

\textbf{TS 3.2.2.2 } \newline
\textbf{Samhita Paata} \newline

सेतु॒नाऽति॑ यन्त्य॒न्यं ॥ द्वे द्रध॑सी स॒तती॑ वस्त॒ एकः॑ के॒शी विश्वा॒ भुव॑नानि वि॒द्वान् । ति॒रो॒धायै॒त्यसि॑तं॒ ॅवसा॑नः शु॒क्रमा द॑त्ते अनु॒हाय॑ जा॒र्यै ॥ दे॒वा वै यद्य॒ज्ञेऽकु॑र्वत॒ तदसु॑रा अकुर्वत॒ ते दे॒वा ए॒तं म॑हाय॒ज्ञ्म॑पश्य॒न् तम॑तन्वताग्निहो॒त्रं ॅव्र॒तम॑कुर्वत॒ तस्मा॒द् द्विव्र॑तः स्या॒द् द्विर्ह्य॑ग्निहो॒त्रं जुह्व॑ति पौर्णमा॒सं ॅय॒ज्ञ्म॑ग्नीषो॒मीयं॑ - [  ] \newline

\textbf{Pada Paata} \newline

सेतु॑ना । अतीति॑ । य॒न्ति॒ । अ॒न्यम् ॥ द्वे इति॑ । द्रध॑सी॒ इति॑ । स॒तती॒ इति॑ स - तती᳚ । व॒स्ते॒ । एकः॑ । के॒शी । विश्वा᳚ । भुव॑नानि । वि॒द्वान् ॥ ति॒रो॒धायेति॑ तिरः - धाय॑ । ए॒ति॒ । असि॑तम् । वसा॑नः । शु॒क्रम् । एति॑ । द॒त्ते॒ । अ॒नु॒हायेत्य॑नु - हाय॑ । जा॒र्यै ॥ दे॒वाः । वै । यत् । य॒ज्ञे । अकु॑र्वत । तत् । असु॑राः । अ॒कु॒र्व॒त॒ । ते । दे॒वाः । ए॒तम् । म॒हा॒य॒ज्ञ्मिति॑ महा - य॒ज्ञ्म् । अ॒प॒श्य॒न्न् । तम् । अ॒त॒न्व॒त॒ । अ॒ग्नि॒हो॒त्रमित्य॑ग्नि - हो॒त्रम् । व्र॒तम् । अ॒कु॒र्व॒त॒ । तस्मा᳚त् । द्विव्र॑त॒ इति॒ द्वि - व्र॒तः॒ । स्या॒त् । द्विः । हि । अ॒ग्नि॒हो॒त्रमित्य॑ग्नि - हो॒त्रम् । जुह्व॑ति । पौ॒र्ण॒मा॒समिति॑ पौर्ण - मा॒सम् । य॒ज्ञ्म् । अ॒ग्नी॒षो॒मीय॒मित्य॑ग्नी - सो॒मीय᳚म् ।  \newline




\markright{ TS 3.2.2.3  \hfill https://www.vedavms.in \hfill}

\section{ TS 3.2.2.3 }

\textbf{TS 3.2.2.3 } \newline
\textbf{Samhita Paata} \newline

प॒शुम॑कुर्वत दा॒र्श्यं ॅय॒ज्ञ्मा᳚ग्ने॒यं प॒शुम॑कुर्वत वैश्वदे॒वं प्रा॑तस्सव॒न -म॑कुर्वत वरुणप्रघा॒सान् माद्ध्य॑दिंनꣳ॒॒ सव॑नꣳ साकमे॒धान् पि॑तृय॒ज्ञ्ं त्र्य॑म्बकाꣳ-स्तृतीयसव॒नम॑कुर्वत॒ तमे॑षा॒मसु॑रा य॒ज्ञ् -म॒न्ववा॑जिगाꣳस॒न् तं नाऽन्ववा॑य॒न् ते᳚ऽब्रुवन्नद्ध्वर्त॒व्या वा इ॒मे दे॒वा अ॑भूव॒न्निति॒ तद॑द्ध्व॒रस्या᳚ ऽद्ध्वर॒त्वं ततो॑ दे॒वा अभ॑व॒न् पराऽसु॑रा॒ य ए॒वं ॅवि॒द्वान्थ् सोमे॑न॒ यज॑ते॒ भव॑त्या॒त्मना॒ परा᳚ ( ) ऽस्य॒ भ्रातृ॑व्यो भवति ॥ \newline

\textbf{Pada Paata} \newline

प॒शुम् । अ॒कु॒र्व॒त॒ । दा॒र्श्यम् । य॒ज्ञ्म् । आ॒ग्ने॒यम् । प॒शुम् । अ॒कु॒र्व॒त॒ । वै॒श्व॒दे॒वमिति॑ वैश्व - दे॒वम् । प्रा॒त॒स्स॒व॒नमिति॑ प्रातः - स॒व॒नम् । अ॒कु॒र्व॒त॒ । व॒रु॒ण॒प्र॒घा॒सानिति॑ वरुण - प्र॒घा॒सान् । माद्ध्य॑न्दिनम् । सव॑नम् । सा॒क॒मे॒धानिति॑ साक - मे॒धान् । पि॒तृ॒य॒ज्ञ्मिति॑ पितृ - य॒ज्ञ्ं । त्र्य॑बंका॒निति॒ त्रि - अ॒बं॒का॒न् । तृ॒ती॒य॒स॒व॒नमिति॑ तृतीय - स॒व॒नम् । अ॒कु॒र्व॒त॒ । तम् । ए॒षा॒म् । असु॑राः । य॒ज्ञ्म् । अ॒न्ववा॑जिगाꣳस॒न्नित्य॑नु-अवा॑जिगाꣳसन्न् । तम् । न । अ॒न्ववा॑य॒न्नित्य॑नु-अवा॑यन्न् । ते । अ॒ब्रु॒व॒न्न् । अ॒द्ध्व॒र्त॒व्याः । वै । इ॒मे । दे॒वाः । अ॒भू॒व॒न्न् । इति॑ । तत् । अ॒द्ध्व॒रस्य॑ । अ॒द्ध्व॒र॒त्वमित्य॑द्ध्वर - त्वम् । ततः॑ । दे॒वाः । अभ॑वन्न् । परेति॑ । असु॑राः । यः । ए॒वम् । वि॒द्वान् । सोमे॑न । यज॑ते । भव॑ति । आ॒त्मना᳚ । परेति॑ ( ) । अ॒स्य॒ । भ्रातृ॑व्यः । भ॒व॒ति॒ ॥  \newline




\markright{ TS 3.2.3.1  \hfill https://www.vedavms.in \hfill}

\section{ TS 3.2.3.1 }

\textbf{TS 3.2.3.1 } \newline
\textbf{Samhita Paata} \newline

प॒रि॒भूर॒ग्निं प॑रि॒भूरिन्द्रं॑ परि॒भूर्विश्वा᳚न् दे॒वान् प॑रि॒भूर्माꣳ स॒ह ब्र॑ह्मवर्च॒सेन॒ स नः॑ पवस्व॒ शं गवे॒ शं जना॑य॒ शमर्व॑ते॒ शꣳ रा॑ज॒न्नोष॑धी॒भ्यो ऽच्छि॑न्नस्य ते रयिपते सु॒वीर्य॑स्य रा॒यस्पोष॑स्य ददि॒तारः॑ स्याम । तस्य॑ मे रास्व॒ तस्य॑ ते भक्षीय॒ तस्य॑ त इ॒दमुन्मृ॑जे ॥ प्रा॒णाय॑ मे वर्चो॒दा वर्च॑से पवस्वा पा॒नाय॑ व्या॒नाय॑ वा॒चे - [  ] \newline

\textbf{Pada Paata} \newline

प॒रि॒भूरिति॑ परि - भूः । अ॒ग्निम् । प॒रि॒भूरिति॑ परि - भूः । इन्द्र᳚म् । प॒रि॒भूरिति॑ परि - भूः । विश्वान्॑ । दे॒वान् । प॒रि॒भूरिति॑ परि - भूः । माम् । स॒ह । ब्र॒ह्म॒व॒र्च॒सेनेति॑ ब्रह्म - व॒र्च॒सेन॑ । सः । नः॒ । प॒व॒स्व॒ । शम् । गवे᳚ । शम् । जना॑य । शम् । अर्व॑ते । शम् । रा॒ज॒न्न् । ओष॑धीभ्य॒ इत्योष॑धि - भ्यः॒ । अच्छि॑न्नस्य । ते॒ । र॒यि॒प॒त॒ इति॑ रयि - प॒ते॒ । सु॒वीर्य॒स्येति॑ सु - वीर्य॑स्य । रा॒यः । पोष॑स्य । द॒दि॒तारः॑ । स्या॒म॒ ॥ तस्य॑ । मे॒ । रा॒स्व॒ । तस्य॑ । ते॒ । भ॒क्षी॒य॒ । तस्य॑ । ते॒ । इ॒दम् । उदिति॑ । मृ॒जे॒ ॥ प्रा॒णायेति॑ प्र -अ॒नाय॑ । मे॒ । व॒र्चो॒दा इति॑ वर्चः - दाः । वर्च॑से । प॒व॒स्व॒ । अ॒पा॒नायेत्यप॑ - अ॒नाय॑ । व्या॒नायेति॑ वि - अ॒नाय॑ । वा॒चे ।  \newline




\markright{ TS 3.2.3.2  \hfill https://www.vedavms.in \hfill}

\section{ TS 3.2.3.2 }

\textbf{TS 3.2.3.2 } \newline
\textbf{Samhita Paata} \newline

द॑क्षक्र॒तुभ्यां॒ चक्षु॑र्भ्यां मे वर्चो॒दौ वर्च॑से पवेथाꣳ॒॒ श्रोत्रा॑या॒ ऽऽ*त्मने ऽङ्गे᳚भ्य॒ आयु॑षे वी॒र्या॑य॒ विष्णो॒रिन्द्र॑स्य॒ विश्वे॑षां दे॒वानां᳚ ज॒ठर॑मसि वर्चो॒दा मे॒ वर्च॑से पवस्व॒ को॑ऽसि॒ को नाम॒ कस्मै᳚ त्वा॒ काय॑ त्वा॒ यं त्वा॒ सोमे॒नाती॑तृपं॒ ॅयं त्वा॒ सोमे॒नामी॑मदꣳ सुप्र॒जाः प्र॒जया॑ भूयासꣳ सु॒वीरो॑ वी॒रैः सु॒वर्चा॒ वर्च॑सा सु॒पोषः॒ पोषै॒-र्विश्वे᳚भ्यो मे रू॒पेभ्यो॑ वर्चो॒दा- [  ] \newline

\textbf{Pada Paata} \newline

द॒क्ष॒क्र॒तुभ्या॒मिति॑ दक्षक्र॒तु - भ्या॒म् । चक्षु॑र्भ्या॒मिति॒ चक्षुः॑ - भ्या॒म् । मे॒ । व॒र्चो॒दाविति॑ वर्चः-दौ । वर्च॑से । प॒वे॒था॒म् । श्रोत्रा॑य । आ॒त्मने᳚ । अङ्गे᳚भ्यः । आयु॑षे । वी॒र्या॑य । विष्णोः᳚ । इन्द्र॑स्य । विश्वे॑षाम् । दे॒वाना᳚म् । ज॒ठर᳚म् । अ॒सि॒ । व॒र्चो॒दा इति॑ वर्चः-दाः । मे॒ । वर्च॑से । प॒व॒स्व॒ । कः । अ॒सि॒ । कः । नाम॑ । कस्मै᳚ । त्वा॒ । काय॑ । त्वा॒ । यम् । त्वा॒ । सोमे॑न । अती॑तृपम् । यम् । त्वा॒ । सोमे॑न । अमी॑मदम् । सु॒प्र॒जा इति॑ सु - प्र॒जाः । प्र॒जयेति॑ प्र - जया᳚ । भू॒या॒स॒म् । सु॒वीर॒ इति॑ सु - वीरः॑ । वी॒रैः । सु॒वर्चा॒ इति॑ सु - वर्चाः᳚ । वर्च॑सा । सु॒पोष॒ इति॑ सु - पोषः॑ । पोषैः᳚ । विश्वे᳚भ्यः । मे॒ । रू॒पेभ्यः॑ । व॒र्चो॒दा इति॑ वर्चः - दाः ।  \newline




\markright{ TS 3.2.3.3  \hfill https://www.vedavms.in \hfill}

\section{ TS 3.2.3.3 }

\textbf{TS 3.2.3.3 } \newline
\textbf{Samhita Paata} \newline

वर्च॑से पवस्व॒ तस्य॑ मे रास्व॒ तस्य॑ ते भक्षीय॒ तस्य॑ त इ॒दमुन्मृ॑जे ॥ बुभू॑ष॒न्नवे᳚क्षेतै॒ष वै पात्रि॑यः प्र॒जाप॑तिर्य॒ज्ञ्ः प्र॒जाप॑ति॒स्तमे॒व त॑र्पयति॒ स ए॑नं तृ॒प्तो भूत्या॒ऽभि प॑वते ब्रह्मवर्च॒सका॒मो-ऽवे᳚क्षेतै॒ष वै पात्रि॑यः प्र॒जाप॑तिर्य॒ज्ञ्ः प्र॒जाप॑ति॒स्तमे॒व त॑र्पयति॒ स ए॑नं तृ॒प्तो ब्र॑ह्मवर्च॒सेना॒भि प॑वत आमया॒व्य - [  ] \newline

\textbf{Pada Paata} \newline

वर्च॑से । प॒व॒स्व॒ । तस्य॑ । मे॒ । रा॒स्व॒ । तस्य॑ । ते॒ । भ॒क्षी॒य॒ । तस्य॑ । ते॒ । इ॒दम् । उदिति॑ । मृ॒जे॒ ॥ बुभू॑षन्न् । अवेति॑ । ई॒क्षे॒त॒ । ए॒षः । वै । पात्रि॑यः । प्र॒जाप॑ति॒रिति॑ प्र॒जा - प॒तिः॒ । य॒ज्ञ्ः । प्र॒जाप॑ति॒रिति॑ प्र॒जा - प॒तिः॒ । तम् । ए॒व । त॒र्प॒य॒ति॒ । सः । ए॒न॒म् । तृ॒प्तः । भूत्या᳚ । अ॒भीति॑ । प॒व॒ते॒ । ब्र॒ह्म॒व॒र्च॒सका॑म॒ इति॑ ब्रह्मवर्च॒स-का॒मः॒ । अवेति॑ । ई॒क्षे॒त॒ । ए॒षः । वै । पात्रि॑यः । प्र॒जाप॑ति॒रिति॑ प्र॒जा - प॒तिः॒ । य॒ज्ञ्ः । प्र॒जाप॑ति॒रिति॑ प्र॒जा - प॒तिः॒ । तम् । ए॒व । त॒र्प॒य॒ति॒ । सः । ए॒न॒म् । तृ॒प्तः । ब्र॒ह्म॒व॒र्च॒सेनेति॑ ब्रह्म - व॒र्च॒सेन॑ । अ॒भीति॑ । प॒व॒ते॒ । आ॒म॒या॒वी ।  \newline




\markright{ TS 3.2.3.4  \hfill https://www.vedavms.in \hfill}

\section{ TS 3.2.3.4 }

\textbf{TS 3.2.3.4 } \newline
\textbf{Samhita Paata} \newline

वे᳚क्षेतै॒ष वै पात्रि॑यः प्र॒जाप॑तिर्य॒ज्ञ्ः प्र॒जाप॑ति॒स्तमे॒व त॑र्पयति॒ स ए॑नं तृ॒प्त आयु॑षा॒ऽभि प॑वतेऽभि॒चर॒न्नवे᳚क्षेतै॒ष वै पात्रि॑यः प्र॒जाप॑तिर्य॒ज्ञ्ः प्र॒जाप॑ति॒स्तमे॒व त॑र्पयति॒ स ए॑नं तृ॒प्तः प्रा॑णापा॒नाभ्यां᳚ ॅवा॒चो द॑क्षक्र॒तुभ्यां॒ चक्षु॑र्भ्याꣳ॒॒ श्रोत्रा᳚भ्या-मा॒त्मनोऽङ्गे᳚भ्य॒ आयु॑षो॒ऽन्तरे॑ति ता॒जक् प्र ध॑न्वति ॥ \newline

\textbf{Pada Paata} \newline

अवेति॑ । ई॒क्षे॒त॒ । ए॒षः । वै । पात्रि॑यः । प्र॒जाप॑ति॒रिति॑ प्र॒जा-प॒तिः॒ । य॒ज्ञ्ः । प्र॒जाप॑ति॒रिति॑ प्र॒जा - प॒तिः॒ । तम् । ए॒व । त॒र्प॒य॒ति॒ । सः । ए॒न॒म् । तृ॒प्तः । आयु॑षा । अ॒भीति॑ । प॒व॒ते॒ । अ॒भि॒चर॒न्नित्य॑भि - चरन्न्॑ । अवेति॑ । ई॒क्षे॒त॒ । ए॒षः । वै । पात्रि॑यः । प्र॒जाप॑ति॒रिति॑ प्र॒जा - प॒तिः॒ । य॒ज्ञ्ः । प्र॒जाप॑ति॒रिति॑ प्र॒जा - प॒तिः॒ । तम् । ए॒व । त॒र्प॒य॒ति॒ । सः । ए॒न॒म् । तृ॒प्तः । प्रा॒णा॒पा॒नाभ्या॒मिति॑ प्राण - अ॒पा॒नाभ्या᳚म् । वा॒चः । द॒क्ष॒क्र॒तुभ्या॒मिति॑ दक्षक्र॒तु - भ्या॒म् । चक्षु॑र्भ्या॒मिति॒ चक्षुः॑ - भ्या॒म् । श्रोत्रा᳚भ्याम् । आ॒त्मनः॑ । अङ्गे᳚भ्यः । आयु॑षः । अ॒न्तः । ए॒ति॒ । ता॒जक् । प्रेति॑ । ध॒न्व॒ति॒ ॥  \newline




\markright{ TS 3.2.4.1  \hfill https://www.vedavms.in \hfill}

\section{ TS 3.2.4.1 }

\textbf{TS 3.2.4.1 } \newline
\textbf{Samhita Paata} \newline

स्फ्यः स्व॒स्तिर्वि॑घ॒नः स्व॒स्तिः पर्.शु॒र्वेदिः॑ पर॒शुर्नः॑ स्व॒स्तिः । य॒ज्ञिया॑ यज्ञ्॒कृतः॑ स्थ॒ ते मा॒स्मिन् य॒ज्ञ् उप॑ ह्वयद्ध्व॒मुप॑ मा॒ द्यावा॑पृथि॒वी ह्व॑येता॒मुपा᳚ऽऽ*स्ता॒वः क॒लशः॒ सोमो॑ अ॒ग्निरुप॑ दे॒वा उप॑ य॒ज्ञ् उप॑ मा॒ होत्रा॑ उपह॒वे ह्व॑यन्तां॒ नमो॒ऽग्नये॑ मख॒घ्नेम॒खस्य॑ मा॒ यशो᳚ऽर्या॒दित्या॑हव॒नीय॒मुप॑ तिष्ठते य॒ज्ञो वै म॒खो - [  ] \newline

\textbf{Pada Paata} \newline

स्फ्यः । स्व॒स्तिः । वि॒घ॒न इति॑ वि - घ॒नः । स्व॒स्तिः । पर्.शुः॑ । वेदिः॑ । प॒र॒शुः । नः॒ । स्व॒स्तिः ॥ य॒ज्ञियाः᳚ । य॒ज्ञ्॒कृत॒ इति॑ यज्ञ्-कृतः॑ । स्थ॒ । ते । मा॒ । अ॒स्मिन्न् । य॒ज्ञे । उपेति॑ । ह्व॒य॒द्ध्व॒म् । उपेति॑ । मा॒ । द्यावा॑पृथि॒वी इति॒ द्यावा᳚-पृ॒थि॒वी । ह्व॒ये॒ता॒म् । उपेति॑ । आ॒स्ता॒व इत्या᳚ - स्ता॒वः । क॒लशः॑ । सोमः॑ । अ॒ग्निः । उपेति॑ । दे॒वाः । उपेति॑ । य॒ज्ञ्ः । उपेति॑ । मा॒ । होत्राः᳚ । उ॒प॒ह॒व इत्यु॑प-ह॒वे । ह्व॒य॒न्ता॒म् । नमः॑ । अ॒ग्नये᳚ । म॒ख॒घ्न इति॑ मख-घ्ने । म॒खस्य॑ । मा॒ । यशः॑ । अ॒र्या॒त् । इति॑ । आ॒ह॒व॒नीय॒मित्या᳚ - ह॒व॒नीय᳚म् । उपेति॑ । ति॒ष्ठ॒ते॒ । य॒ज्ञ्ः । वै । म॒खः ।  \newline




\markright{ TS 3.2.4.2  \hfill https://www.vedavms.in \hfill}

\section{ TS 3.2.4.2 }

\textbf{TS 3.2.4.2 } \newline
\textbf{Samhita Paata} \newline

य॒ज्ञ्ं ॅवाव स तद॑ह॒न् तस्मा॑ ए॒व न॑म॒स्कृत्य॒ सदः॒ प्रस॑र्पत्या॒त्मनोऽना᳚र्त्यै॒ नमो॑ रु॒द्राय॑ मख॒घ्ने नम॑स्कृत्या मा पा॒हीत्याग्नी᳚द्ध्रं॒ तस्मा॑ ए॒व न॑म॒स्कृत्य॒ सदः॒ प्रस॑र्पत्या॒त्मनोऽना᳚र्त्यै॒ नम॒ इन्द्रा॑य मख॒घ्न इ॑न्द्रि॒यं मे॑ वी॒र्यं॑ मा निर्व॑धी॒रिति॑ हो॒त्रीय॑मा॒शिष॑मे॒वैतामा शा᳚स्तैन्द्रि॒यस्य॑ वी॒र्य॑स्यानि॑र्घाताय॒ या वै - [  ] \newline

\textbf{Pada Paata} \newline

य॒ज्ञ्म् । वाव । सः । तत् । अ॒ह॒न्न् । तस्मै᳚ । ए॒व । न॒म॒स्कृत्येति॑ नमः - कृत्य॑ । सदः॑ । प्रेति॑ । स॒र्प॒ति॒ । आ॒त्मनः॑ । अना᳚र्त्यै । नमः॑ । रु॒द्राय॑ । म॒ख॒घ्न इति॑ मख - घ्ने । नम॑स्कृ॒त्येति॒ नमः॑-कृ॒त्या॒ । मा॒ । पा॒हि॒ । इति॑ । आग्नी᳚द्ध्र॒मित्याग्नि॑ - इ॒द्ध्र॒म् । तस्मै᳚ । ए॒व । न॒म॒स्कृत्येति॑ नमः - कृत्य॑ । सदः॑ । प्रेति॑ । स॒र्प॒ति॒ । आ॒त्मनः॑ । अना᳚र्त्यै । नमः॑ । इन्द्रा॑य । म॒ख॒घ्न इति॑ मख - घ्ने । इ॒न्द्रि॒यम् । मे॒ । वी॒र्य᳚म् । मा । निरिति॑ । व॒धीः॒ । इति॑ । हो॒त्रीय᳚म् । आ॒शिष॒मित्या᳚ - शिष᳚म् । ए॒व । ए॒ताम् । एति॑ । शा॒स्ते॒ । इ॒न्द्रि॒यस्य॑ । वी॒र्य॑स्य । अनि॑र्घाता॒येत्यनिः॑ - घा॒ता॒य॒ । याः । वै ।  \newline




\markright{ TS 3.2.4.3  \hfill https://www.vedavms.in \hfill}

\section{ TS 3.2.4.3 }

\textbf{TS 3.2.4.3 } \newline
\textbf{Samhita Paata} \newline

दे॒वताः॒ सद॒स्यार्ति॑मा॒र्पय॑न्ति॒ यस्ता वि॒द्वान् प्र॒सर्प॑ति॒ न सद॒स्यार्ति॒मार्च्छ॑ति॒ नमो॒ऽग्नये॑ मख॒घ्न इत्या॑है॒ता वै दे॒वताः॒ सद॒स्यार्ति॒माऽर्प॑यन्ति॒ ता य ए॒वं ॅवि॒द्वान् प्र॒सर्प॑ति॒ न सद॒स्यार्ति॒मार्च्छ॑ति द्दृ॒ढे स्थः॑ शिथि॒रे स॒मीची॒ माऽꣳह॑सस्पातꣳ॒॒ सूर्यो॑ मा दे॒वो दि॒व्यादꣳह॑सस्पातु वा॒युर॒न्तरि॑क्षा - [  ] \newline

\textbf{Pada Paata} \newline

दे॒वताः᳚ । सद॑सि । आर्ति᳚म् । आ॒र्पय॒न्तीत्या᳚ - अ॒र्पय॑न्ति । यः । ताः । वि॒द्वान् । प्र॒सर्प॒तीति॑ प्र - सर्प॑ति । न । सद॑सि । आर्ति᳚म् । एति॑ । ऋ॒च्छ॒ति॒ । नमः॑ । अ॒ग्नये᳚ । म॒ख॒घ्न इति॑ मख - घ्ने । इति॑ । आ॒ह॒ । ए॒ताः । वै । दे॒वताः᳚ । सद॑सि । आर्ति᳚म् । एति॑ । अ॒र्प॒य॒न्ति॒ । ताः । यः । ए॒वम् । वि॒द्वान् । प्र॒सर्प॒तीति॑ प्र - सर्प॑ति । न । सद॑सि । आर्ति᳚म् । एति॑ । ऋ॒च्छ॒ति॒ । दृ॒ढे इति॑ । स्थः॒ । शि॒थि॒रे इति॑ । स॒मीची॒ इति॑ । मा॒ । अꣳह॑सः । पा॒त॒म् । सूर्यः॑ । मा॒ । दे॒वः । दि॒व्यात् । अꣳह॑सः । पा॒तु॒ । वा॒युः । अ॒न्तरि॑क्षात् ।  \newline




\markright{ TS 3.2.4.4  \hfill https://www.vedavms.in \hfill}

\section{ TS 3.2.4.4 }

\textbf{TS 3.2.4.4 } \newline
\textbf{Samhita Paata} \newline

द॒ग्निः पृ॑थि॒व्या य॒मः पि॒तृभ्यः॒ सर॑स्वती मनु॒ष्ये᳚भ्यो॒ देवी᳚ द्वारौ॒ मा मा॒ सं ता᳚प्तं॒ नमः॒ सद॑से॒ नमः॒ सद॑स॒स्पत॑ये॒ नमः॒ सखी॑नां पुरो॒गाणां॒ चक्षु॑षे॒ नमो॑ दि॒वे नमः॑ पृथि॒व्या अहे॑ दैधिष॒व्योदत॑स्तिष्ठा॒ऽन्यस्य॒ सद॑ने सीद॒ यो᳚ऽस्मत् पाक॑तर॒ उन्नि॒वत॒ उदु॒द्वत॑श्च गेषं पा॒तं मा᳚ द्यावापृथिवी अ॒द्याह्नः॒ सदो॒ वै प्र॒सर्प॑न्तं - [  ] \newline

\textbf{Pada Paata} \newline

अ॒ग्निः । पृ॒थि॒व्याः । य॒मः । पि॒तृभ्य॒ इति॑ पि॒तृ - भ्यः॒ । सर॑स्वती । म॒नु॒ष्ये᳚भ्यः । देवी॒ इति॑ । द्वा॒रौ॒ । मा । मा॒ । समिति॑ । ता॒प्त॒म् । नमः॑ । सद॑से । नमः॑ । सद॑सः । पत॑ये । नमः॑ । सखी॑नाम् । पु॒रो॒गाणा॒मिति॑ पुरः - गाना᳚म् । चक्षु॑षे । नमः॑ । दि॒वे । नमः॑ । पृ॒थि॒व्यै । अहे᳚ । दै॒धि॒ष॒व्य॒ । उदिति॑ । अतः॑ । ति॒ष्ठ॒ । अ॒न्यस्य॑ । सद॑ने । सी॒द॒ । यः । अ॒स्मत् । पाक॑तर॒ इति॒ पाक॑ - त॒रः॒ । उदिति॑ । नि॒वत॒ इति॑ नि-वतः॑ । उदिति॑ । उ॒द्वत॒ इत्यु॑त् - वतः॑ । च॒ । गे॒ष॒म् । पा॒तम् । मा॒ । द्या॒वा॒पृ॒थि॒वी॒ इति॑ द्यावा - पृ॒थि॒वी॒ । अ॒द्य । अह्नः॑ । सदः॑ । वै । प्र॒सर्प॑न्त॒मिति॑ प्र - सर्प॑न्तम् ।  \newline




\markright{ TS 3.2.4.5  \hfill https://www.vedavms.in \hfill}

\section{ TS 3.2.4.5 }

\textbf{TS 3.2.4.5 } \newline
\textbf{Samhita Paata} \newline

पि॒तरोऽनु॒ प्रस॑र्पन्ति॒ त ए॑नमीश्व॒रा हिꣳसि॑तोः॒ सदः॑ प्र॒सृप्य॑ दक्षिणा॒र्द्धं परे᳚क्षे॒ताऽग॑न्त पितरः पितृ॒मान॒हं ॅयु॒ष्माभि॑र्भूयासꣳ सुप्र॒जसो॒ मया॑ यू॒यं भू॑या॒स्तेति॒ तेभ्य॑ ए॒व न॑म॒स्कृत्य॒ सदः॒ प्रस॑र्पत्या॒त्मनोऽना᳚र्त्यै ॥ \newline

\textbf{Pada Paata} \newline

पि॒तरः॑ । अनु॑ । प्रेति॑ । स॒र्प॒न्ति॒ । ते । ए॒न॒म् । ई॒श्व॒राः । हिꣳसि॑तोः । सदः॑ । प्र॒सृप्येति॑ प्र - सृप्य॑ । द॒क्षि॒णा॒र्द्धमिति॑ दक्षिण - अ॒र्द्धम् । परेति॑ । ई॒क्षे॒त॒ । एति॑ । अ॒ग॒न्त॒ । पि॒त॒रः॒ । पि॒तृ॒मानिति॑ पितृ - मान् । अ॒हम् । यु॒ष्माभिः॑ । भू॒या॒स॒म् । सु॒प्र॒जस॒ इति॑ सु-प्र॒जसः॑ । मया᳚ । यू॒यम् । भू॒या॒स्त॒ । इति॑ । तेभ्यः॑ । ए॒व । न॒म॒स्कृत्येति॑ नमः-कृत्य॑ । सदः॑ । प्रेति॑ । स॒र्प॒ति॒ । आ॒त्मनः॑ । अना᳚र्त्यै ॥  \newline




\markright{ TS 3.2.5.1  \hfill https://www.vedavms.in \hfill}

\section{ TS 3.2.5.1 }

\textbf{TS 3.2.5.1 } \newline
\textbf{Samhita Paata} \newline

भक्षेहि॒ मा ऽऽवि॑श दीर्घायु॒त्वाय॑ शन्तनु॒त्वाय॑ रा॒यस्पोषा॑य॒ वर्च॑से सुप्रजा॒स्त्वायेहि॑ वसो पुरो वसो प्रि॒यो मे॑ हृ॒दो᳚ऽस्य॒श्विनो᳚स्त्वा बा॒हुभ्याꣳ॑ सघ्यासं नृ॒चक्ष॑सं त्वा देव सोम सु॒चक्षा॒ अव॑ ख्येषं म॒न्द्राऽभिभू॑तिः के॒तुर्य॒ज्ञानां॒ ॅवाग्जु॑षा॒णा सोम॑स्य तृप्यतु म॒न्द्रा स्व॑र्वा॒च्यदि॑ति॒रना॑हत शीर्ष्णी॒ वाग्जु॑षा॒णा सोम॑स्य तृप्य॒त्वेहि॑ विश्वचर्.षणे - [  ] \newline

\textbf{Pada Paata} \newline

भक्ष॑ । एति॑ । इ॒हि॒ । मा॒ । एति॑ । वि॒श॒ । दी॒र्घा॒यु॒त्वायेति॑ दीर्घायु - त्वाय॑ । श॒न्त॒नु॒त्वायेति॑ शन्तनु - त्वाय॑ । रा॒यः । पोषा॑य । वर्च॑से । सु॒प्र॒जा॒स्त्वायेति॑ सुप्रजाः-त्वाय॑ । एति॑ । इ॒हि॒ । व॒सो॒ इति॑ । पु॒रो॒व॒सो॒ इति॑ पुरः - व॒सो॒ । प्रि॒यः । मे॒ । हृ॒दः । अ॒सि॒ । अ॒श्विनोः᳚ । त्वा॒ । बा॒हुभ्या॒मिति॑ बा॒हु- भ्या॒म् । स॒घ्या॒स॒म् । नृ॒चक्ष॑स॒मिति॑ नृ - चक्ष॑सम् । त्वा॒ । दे॒व॒ । सो॒म॒ । सु॒चक्षा॒ इति॑ सु - चक्षाः᳚ । अवेति॑ । ख्ये॒ष॒म् । म॒न्द्रा । अ॒भिभू॑ति॒रित्य॒भि - भू॒तिः॒ । के॒तुः । य॒ज्ञाना᳚म् । वाक् । जु॒षा॒णा । सोम॑स्य । तृ॒प्य॒तु॒ । म॒न्द्रा । स्व॑र्वा॒चीति॒ सु - अ॒र्वा॒ची॒ । अदि॑तिः । अना॑हतशी॒र्ष्णीत्यना॑हत - शी॒र्ष्णी॒ । वाक् । जु॒षा॒णा । सोम॑स्य । तृ॒प्य॒तु॒ । एति॑ । इ॒हि॒ । वि॒श्व॒च॒र्॒.ष॒ण॒ इति॑ विश्व - च॒र्॒.ष॒णे॒ ।  \newline




\markright{ TS 3.2.5.2  \hfill https://www.vedavms.in \hfill}

\section{ TS 3.2.5.2 }

\textbf{TS 3.2.5.2 } \newline
\textbf{Samhita Paata} \newline

श॒म्भूर्म॑यो॒भूः स्व॒स्ति मा॑ हरिवर्ण॒ प्रच॑र॒ क्रत्वे॒ दक्षा॑य रा॒यस्पोषा॑य सुवी॒रता॑यै॒ मा मा॑ राज॒न्. वि बी॑भिषो॒ मा मे॒ हार्दि॑ त्वि॒षा व॑धीः । वृष॑णे॒ शुष्मा॒याऽऽयु॑षे॒ वर्च॑से ॥ वसु॑मद्-गणस्य सोम देव ते मति॒विदः॑ प्रात॒स्सव॒नस्य॑ गाय॒त्रछ॑न्दस॒ इन्द्र॑पीतस्य॒ नरा॒शꣳस॑पीतस्य पि॒तृपी॑तस्य॒ मधु॑मत॒ उप॑हूत॒स्योप॑हूतो भक्षयामि रु॒द्रव॑द्-गणस्य सोम देव ते मति॒विदो॒ माद्ध्य॑न्दिनस्य॒ सव॑नस्य त्रि॒ष्टुप्छ॑न्दस॒ इन्द्र॑पीतस्य॒ नरा॒शꣳ स॑पीतस्य - [  ] \newline

\textbf{Pada Paata} \newline

श॒भूंरिति॑ शं - भूः । म॒यो॒भूरिति॑ मयः - भूः । स्व॒स्ति । मा॒ । ह॒रि॒व॒र्णेति॑ हरि - व॒र्ण॒ । प्रेति॑ । च॒र॒ । क्रत्वे᳚ । दक्षा॑य । रा॒यः । पोषा॑य । सु॒वी॒रता॑या॒ इति॑ सु-वी॒रता॑यै । मा । मा॒ । रा॒ज॒न्न् । वीति॑ । बी॒भि॒षः॒ । मा । मे॒ । हार्दि॑ । त्वि॒षा । व॒धीः॒ ॥ वृष॑णे । शुष्मा॑य । आयु॑षे । वर्च॑से ॥ वसु॑मद् गण॒स्येति॒ वसु॑मत् - ग॒ण॒स्य॒ । सो॒म॒ । दे॒व॒ । ते॒ । म॒ति॒विद॒ इति॑ मति - विदः॑ । प्रा॒त॒स्स॒व॒नस्येति॑ प्रातः - स॒व॒नस्य॑ । गा॒य॒त्रछ॑न्दस॒ इति॑ गाय॒त्र - छ॒न्द॒सः॒ । इन्द्र॑पीत॒स्येतीन्द्र॑ - पी॒त॒स्य॒ । नरा॒शꣳस॑पीत॒स्येति॒ नरा॒शꣳस॑ - पी॒त॒स्य॒ । पि॒तृपी॑त॒स्येति॑ पि॒तृ - पी॒त॒स्य॒ । मधु॑मत॒ इति॒ मधु॑ - म॒तः॒ । उप॑हूत॒स्येत्युप॑ - हू॒त॒स्य॒ । उप॑हूत॒ इत्युप॑ - हू॒तः॒ । भ॒क्ष॒या॒मि॒ । रु॒द्रव॑द्गण॒स्येति॑ रु॒द्रव॑त् - ग॒ण॒स्य॒ । सो॒म॒ । दे॒व॒ । ते॒ । म॒ति॒विद॒ इति॑ मति - विदः॑ । माद्ध्य॑न्दिनस्य । सव॑नस्य । त्रि॒ष्टुप्छ॑न्दस॒ इति॑ त्रि॒ष्टुप् - छ॒न्द॒सः॒ । इन्द्र॑पीत॒स्येतीन्द्र॑ - पी॒त॒स्य॒ । नरा॒शꣳस॑पीत॒स्येति॒ नरा॒शꣳस॑ - पी॒त॒स्य॒ ।  \newline




\markright{ TS 3.2.5.3  \hfill https://www.vedavms.in \hfill}

\section{ TS 3.2.5.3 }

\textbf{TS 3.2.5.3 } \newline
\textbf{Samhita Paata} \newline

पि॒तृपी॑तस्य॒ मधु॑मत॒ उप॑हूत॒स्योप॑हूतो भक्षयाम्यादि॒त्यव॑द्-गणस्य सोम देव ते मति॒विद॑स्तृ॒तीय॑स्य॒ सव॑नस्य॒ जग॑तीछन्दस॒ इन्द्र॑पीतस्य॒ नरा॒शꣳ स॑पीतस्य पि॒तृपी॑तस्य॒ मधु॑मत॒ उप॑हूत॒स्योप॑हूतो भक्षयामि ॥ आप्या॑यस्व॒ समे॑तु ते वि॒श्वतः॑ सोम॒ वृष्णि॑यं । भवा॒ वाज॑स्य सङ्ग॒थे ॥ हिन्व॑ मे॒ गात्रा॑ हरिवो ग॒णान् मे॒ मा विती॑तृषः । शि॒वो मे॑ सप्त॒र्॒.षीनुप॑ तिष्ठस्व॒ मा मेऽवा॒ङ्नाभि॒मति॑ - [  ] \newline

\textbf{Pada Paata} \newline

पि॒तृपी॑त॒स्येति॑ पि॒तृ - पी॒त॒स्य॒ । मधु॑मत॒ इति॒ मधु॑ - म॒तः॒ । उप॑हूत॒स्येत्युप॑ - हू॒त॒स्य॒ । उप॑हूत॒ इत्युप॑ - हू॒तः॒ । भ॒क्ष॒या॒मि॒ । आ॒दि॒त्यव॑द्गण॒स्येत्या॑दि॒त्यव॑त् - ग॒ण॒स्य॒ । सो॒म॒ । दे॒व॒ । ते॒ । म॒ति॒विद॒ इति॑ मति - विदः॑ । तृ॒तीय॑स्य । सव॑नस्य । जग॑तीछन्दस॒ इति॒ जग॑ती - छ॒न्द॒सः॒ । इन्द्र॑पीत॒स्येतीन्द्र॑ - पी॒त॒स्य॒ । नरा॒शꣳस॑पीत॒स्येति॒ नरा॒शꣳस॑ - पी॒त॒स्य॒ । पि॒तृपी॑त॒स्येति॑ पि॒तृ - पी॒त॒स्य॒ । मधु॑मत॒ इति॒ मधु॑ - म॒तः॒ । उप॑हूत॒स्येत्युप॑ - हू॒त॒स्य॒ । उप॑हूत॒ इत्युप॑-हू॒तः॒ । भ॒क्ष॒या॒मि॒ ॥ एति॑ । प्या॒य॒स्व॒ । समिति॑ । ए॒तु॒ । ते॒ । वि॒श्वतः॑ । सो॒म॒ । वृष्णि॑यम् ॥ भव॑ । वाज॑स्य । स॒ङ्ग॒थ इति॑ सं - ग॒थे ॥ हिन्व॑ । मे॒ । गात्रा᳚ । ह॒रि॒व॒ इति॑ हरि - वः॒ । ग॒णान् । मे॒ । मा । वीति॑ । ती॒तृ॒षः॒ ॥ शि॒वः । म॒ । स॒प्त॒र्॒.षीनिति॑ सप्त-ऋ॒षीन् । उपेति॑ । ति॒ष्ठ॒स्व॒ । मा । मे॒ । अवाङ्॑ । नाभि᳚म् । अतीति॑ ।  \newline




\markright{ TS 3.2.5.4  \hfill https://www.vedavms.in \hfill}

\section{ TS 3.2.5.4 }

\textbf{TS 3.2.5.4 } \newline
\textbf{Samhita Paata} \newline

गाः ॥ अपा॑म॒ सोम॑म॒मृता॑ अभू॒माऽद॑र्श्म॒ ज्योति॒रवि॑दाम दे॒वान् । किम॒स्मान् कृ॑णव॒दरा॑तिः॒ किमु॑ धू॒र्तिर॑मृत॒ मर्त्य॑स्य ॥यन्म॑ आ॒त्मनो॑ मि॒न्दाऽभू॑द॒ग्निस्तत् पुन॒राऽहा᳚र्जा॒तवे॑दा॒ विच॑र्.षणिः ॥ पुन॑र॒ग्निश्चक्षु॑रदा॒त्-पुन॒रिन्द्रो॒ बृह॒स्पतिः॑ । पुन॑र्मे अश्विना यु॒वं चक्षु॒रा ध॑त्तम॒क्ष्योः ॥ इ॒ष्टय॑जुषस्ते देव सोम स्तु॒तस्तो॑मस्य - [  ] \newline

\textbf{Pada Paata} \newline

गाः॒ ॥ अपा॑म । सोम᳚म् । अ॒मृताः᳚ । अ॒भू॒म॒ । अद॑र्श्म । ज्योतिः॑ । अवि॑दाम । दे॒वान् ॥ किम् । अ॒स्मान् । कृ॒ण॒व॒त् । अरा॑तिः । किम् । उ॒ । धू॒र्तिः । अ॒मृ॒त॒ । मर्त्य॑स्य ॥ यत् । मे॒ । आ॒त्मनः॑ । मि॒न्दा । अभू᳚त् । अ॒ग्निः । तत् । पुनः॑ । एति॑ । अ॒हाः॒ । जा॒तवे॑दा॒ इति॑ जा॒त - वे॒दाः॒ । विच॑र्.षणि॒रिति॒ वि - च॒र्॒.ष॒णिः॒ ॥ पुनः॑ । अ॒ग्निः । चक्षुः॑ । अ॒दा॒त् । पुनः॑ । इन्द्रः॑ । बृह॒स्पतिः॑ ॥ पुनः॑ । मे॒ । अ॒श्वि॒ना॒ । यु॒वम् । चक्षुः॑ । एति॑ । ध॒त्त॒म् । अ॒क्ष्योः ॥ इ॒ष्टय॑जुष॒ इती॒ष्ट-य॒जु॒षः॒ । ते॒ । दे॒व॒ । सो॒म॒ । स्तु॒तस्तो॑म॒स्येति॑ स्तु॒त - स्तो॒म॒स्य॒ ।  \newline




\markright{ TS 3.2.5.5  \hfill https://www.vedavms.in \hfill}

\section{ TS 3.2.5.5 }

\textbf{TS 3.2.5.5 } \newline
\textbf{Samhita Paata} \newline

श॒स्तोक्थ॑स्य॒ हरि॑वत॒ इन्द्र॑पीतस्य॒ मधु॑मत॒ उप॑हूत॒स्योप॑हूतो भक्षयामि ॥ आ॒पूर्याः॒ स्थाऽऽमा॑ पूरयत प्र॒जया॑ च॒ धने॑न च ॥ ए॒तत् ते॑ तत॒ ये च॒ त्वामन्वे॒तत् ते॑ पितामह प्रपितामह॒ ये च॒ त्वामन्वत्र॑ पितरो यथाभा॒गं म॑न्दद्ध्वं॒ नमो॑ वः पितरो॒ रसा॑य॒ नमो॑ वः पितरः॒ शुष्मा॑य॒ नमो॑ वः पितरो जी॒वाय॒ नमो॑ वः पितरः - [  ] \newline

\textbf{Pada Paata} \newline

श॒स्तोक्थ॒स्येति॑ श॒स्त - उ॒क्थ॒स्य॒ । हरि॑वत॒ इति॒ हरि॑ - व॒तः॒ । इन्द्र॑पीत॒स्येतीन्द्र॑ - पी॒त॒स्य॒ । मधु॑मत॒ इति॒ मधु॑ - म॒तः॒ । उप॑हूत॒स्येत्युप॑ - हू॒त॒स्य॒ । उप॑हूत॒ इत्युप॑ - हू॒तः॒ । भ॒क्ष॒या॒मि॒ ॥ आ॒पूर्या॒ इत्या᳚ - पूर्याः᳚ । स्थ॒ । एति॑ । मा॒ । पू॒र॒य॒त॒ । प्र॒जयेति॑ प्र-जया᳚ । च॒ । धने॑न । च॒ ॥ ए॒तत् । ते॒ । त॒त॒ । ये । च॒ । त्वाम् । अन्विति॑ । ए॒तत् । ते॒ । पि॒ता॒म॒ह॒ । प्र॒पि॒ता॒म॒हेति॑ प्र - पि॒ता॒म॒ह॒ । ये । च॒ । त्वाम् । अन्विति॑ । अत्र॑ । पि॒त॒रः॒ । य॒था॒भा॒गमिति॑ यथा-भा॒गम् । म॒न्द॒द्ध्व॒म् । नमः॑ । वः॒ । पि॒त॒रः॒ । रसा॑य । नमः॑ । वः॒ । पि॒त॒रः॒ । शुष्मा॑य । नमः॑ । वः॒ । पि॒त॒रः॒ । जी॒वाय॑ । नमः॑ । वः॒ । पि॒त॒रः॒ ।  \newline




\markright{ TS 3.2.5.6  \hfill https://www.vedavms.in \hfill}

\section{ TS 3.2.5.6 }

\textbf{TS 3.2.5.6 } \newline
\textbf{Samhita Paata} \newline

स्व॒धायै॒ नमो॑ वः पितरो म॒न्यवे॒ नमो॑ वः पितरो घो॒राय॒ पित॑रो॒ नमो॑ वो॒ य ए॒तस्मि॑न् ॅलो॒केस्थ यु॒ष्माꣳस्तेऽनु॒ ये᳚ऽस्मिन् ॅलो॒के मां तेऽनु॒ य ए॒तस्मि॑न् ॅलो॒के स्थ यू॒यं तेषां॒ ॅवसि॑ष्ठा भूयास्त॒ ये᳚ऽस्मिन् ॅलो॒के॑ऽहं तेषां॒ ॅवसि॑ष्ठो भूयासं॒ प्रजा॑पते॒ न त्वदे॒तान्य॒न्यो विश्वा॑ जा॒तानि॒ परि॒ता ब॑भूव । \newline

\textbf{Pada Paata} \newline

स्व॒धाया॒ इति॑ स्व - धायै᳚ । नमः॑ । वः॒ । पि॒त॒रः॒ । म॒न्यवे᳚ । नमः॑ । वः॒ । पि॒त॒रः॒ । घो॒राय॑ । पित॑रः । नमः॑ । वः॒ । ये । ए॒तस्मिन्न्॑ । लो॒के । स्थ । यु॒ष्मान् । ते । अन्विति॑ । ये । अ॒स्मिन्न् । लो॒के । माम् । ते । अन्विति॑ । ये । ए॒तस्मिन्न्॑ । लो॒के । स्थ । यू॒यम् । तेषा᳚म् । वसि॑ष्ठाः । भू॒या॒स्त॒ । ये । अ॒स्मिन्न् । लो॒के । अ॒हम् । तेषा᳚म् । वसि॑ष्ठः । भू॒या॒स॒म् । प्रजा॑पत॒ इति॒ प्रजा᳚-प॒ते॒ । न । त्वत् । ए॒तानि॑ । अ॒न्यः । विश्वा᳚ । जा॒तानि॑ । परीति॑ । ता । ब॒भू॒व॒ ॥  \newline




\markright{ TS 3.2.5.7  \hfill https://www.vedavms.in \hfill}

\section{ TS 3.2.5.7 }

\textbf{TS 3.2.5.7 } \newline
\textbf{Samhita Paata} \newline

यत् का॑मास्ते जुहु॒मस्तन्नो॑ अस्तु व॒यꣳ स्या॑म॒ पत॑यो रयी॒णां ॥ दे॒वकृ॑त॒स्यैन॑सो ऽव॒यज॑नमसि मनु॒ष्य॑कृत॒स्यैन॑सो ऽव॒यज॑नमसि पि॒तृकृ॑त॒स्यैन॑सो ऽव॒यज॑नमस्य॒फ्सु धौ॒तस्य॑ सोम देव ते॒ नृभिः॑ सु॒तस्ये॒ष्ट य॑जुषः स्तु॒तस्तो॑मस्य श॒स्तोक्थ॑स्य॒ यो भ॒क्षोअ॑श्व॒सनि॒र्यो गो॒सनि॒स्तस्य॑ ते पि॒तृभि॑र्भ॒क्षं कृ॑त॒स्यो-प॑हूत॒स्योप॑हूतो भक्षयामि ॥ \newline

\textbf{Pada Paata} \newline

यत्का॑मा॒ इति॒ यत् - का॒माः॒ । ते॒ । जु॒हु॒मः । तत् । नः॒ । अ॒स्तु॒ । व॒यम् । स्या॒म॒ । पत॑यः । र॒यी॒णाम् ॥ दे॒वकृ॑त॒स्येति॑ दे॒व - कृ॒त॒स्य॒ । एन॑सः । अ॒व॒यज॑न॒मित्य॑व - यज॑नम् । अ॒सि॒ । म॒नु॒ष्य॑कृत॒स्येति॑ मनु॒ष्य॑ - कृ॒त॒स्य॒ । एन॑सः । अ॒व॒यज॑न॒मित्य॑व - यज॑नम् । अ॒सि॒ । पि॒तृकृ॑त॒स्येति॑ पि॒तृ - कृ॒त॒स्य॒ । एन॑सः । अ॒व॒यज॑न॒मित्य॑व - यज॑नम् । अ॒सि॒ । अ॒फ्स्वित्य॑प्-सु । धौ॒तस्य॑ । सो॒म॒ । दे॒व॒ । ते॒ । नृभि॒रिति॒ नृ - भिः॒ । सु॒तस्य॑ । इ॒ष्टय॑जुष॒ इती॒ष्ट - य॒जु॒षः॒ । स्तु॒तस्तो॑म॒स्येति॑ स्तु॒त - स्तो॒म॒स्य॒ । श॒स्तोक्थ॒स्येति॑ श॒स्त - उ॒क्थ॒स्य॒ । यः । भ॒क्षः । अ॒श्व॒सनि॒रित्य॑श्व - सनिः॑ । यः । गो॒सनि॒रिति॑ गो - सनिः॑ । तस्य॑ । ते॒ । पि॒तृभि॒रिति॑ पि॒तृ - भिः॒ । भ॒क्षं कृ॑त॒स्येति॑ भ॒क्षं - कृ॒त॒स्य॒ । उप॑हूत॒स्येत्युप॑ - हू॒त॒स्य॒ । उप॑हूत॒ इत्युप॑ - हू॒तः॒ । भ॒क्ष॒या॒मि॒ ॥  \newline




\markright{ TS 3.2.6.1  \hfill https://www.vedavms.in \hfill}

\section{ TS 3.2.6.1 }

\textbf{TS 3.2.6.1 } \newline
\textbf{Samhita Paata} \newline

म॒ही॒नां पयो॑ऽसि॒ विश्वे॑षां दे॒वानां᳚ त॒नूर् ऋ॒द्ध्यास॑म॒द्य पृष॑तीनां॒ ग्रहं॒ पृष॑तीनां॒ ग्रहो॑ऽसि॒ विष्णो॒र्॒.हृद॑यम॒स्येक॑मिष॒ विष्णु॒स्त्वाऽनु॒ विच॑क्रमे भू॒तिर्द॒द्ध्ना घृ॒तेन॑ वर्द्धतां॒ तस्य॑ मे॒ष्टस्य॑ वी॒तस्य॒ द्रवि॑ण॒मा ग॑म्या॒ज्ज्योति॑रसि वैश्वान॒रं पृश्ञि॑यै दु॒ग्धं ॅयाव॑ती॒ द्यावा॑पृथि॒वी म॑हि॒त्वा याव॑च्च स॒प्त सिन्ध॑वो वित॒स्थुः । ताव॑न्तमिन्द्र ते॒ - [  ] \newline

\textbf{Pada Paata} \newline

म॒ही॒नाम् । पयः॑ । अ॒सि॒ । विश्वे॑षाम् । दे॒वाना᳚म् । त॒नूः । ऋ॒द्ध्यास᳚म् । अ॒द्य । पृष॑तीनाम् । ग्रह᳚म् । पृष॑तीनाम् । ग्रहः॑ । अ॒सि॒ । विष्णोः᳚ । हृद॑यम् । अ॒सि॒ । एक᳚म् । इ॒ष॒ । विष्णुः॑ । त्वा॒ । अनु॑ । वीति॑ । च॒क्र॒मे॒ । भू॒तिः । द॒द्ध्ना । घृ॒तेन॑ । व॒र्द्ध॒ता॒म् । तस्य॑ । मा॒ । इ॒ष्टस्य॑ । वी॒तस्य॑ । द्रवि॑णम् । एति॑ । ग॒म्या॒त् । ज्योतिः॑ । अ॒सि॒ । वै॒श्वा॒न॒रम् । पृश्नि॑यै । दु॒ग्धम् । याव॑ती॒ इति॑ । द्यावा॑पृथि॒वी इति॒ द्यावा᳚-पृ॒थि॒वी । म॒हि॒त्वेति॑ महि - त्वा । याव॑त् । च॒ । स॒प्त । सिन्ध॑वः । वि॒त॒स्थुरिति॑ वि - त॒स्थुः ॥ ताव॑न्तम् । इ॒न्द्र॒ । ते॒ ।  \newline




\markright{ TS 3.2.6.2  \hfill https://www.vedavms.in \hfill}

\section{ TS 3.2.6.2 }

\textbf{TS 3.2.6.2 } \newline
\textbf{Samhita Paata} \newline

ग्रहꣳ॑ स॒होर्जा गृ॑ह्णा॒म्यस्तृ॑तं ॥ यत् कृ॑ष्णशकु॒नः पृ॑षदा॒ज्यम॑वमृ॒शेच्छू॒द्रा अ॑स्य प्र॒मायु॑काः स्यु॒र्यच्छ्वा ऽव॑मृ॒शेच्चतु॑ष्पादोऽस्य प॒शवः॑ प्र॒मायु॑काः स्यु॒र्यथ् स्कन्दे॒द्-यज॑मानः प्र॒मायु॑कः स्यात् प॒शवो॒ वै पृ॑षदा॒ज्यं प॒शवो॒ वा ए॒तस्य॑ स्कन्दन्ति॒ यस्य॑ पृषदा॒ज्यꣳ स्कन्द॑ति॒ यत् पृ॑षदा॒ज्यं पुन॑र्गृ॒ह्णाति॑ प॒शूने॒वास्मै॒ पुन॑र्गृह्णाति प्रा॒णो वै पृ॑षदा॒ज्यं प्रा॒णो वा - [  ] \newline

\textbf{Pada Paata} \newline

ग्रह᳚म् । स॒ह । ऊ॒र्जा । गृ॒ह्णा॒मि॒ । अस्तृ॑तम् ॥ यत् । कृ॒ष्ण॒श॒कु॒न इति॑ कृष्ण - श॒कु॒नः । पृ॒ष॒दा॒ज्यमिति॑ पृषत् - आ॒ज्यम् । अ॒व॒मृ॒शेदित्य॑व - मृ॒शेत् । शू॒द्राः । अ॒स्य॒ । प्र॒मायु॑का॒ इति॑ प्र - मायु॑काः । स्युः॒ । यत् । श्वा । अ॒व॒मृ॒शेदित्य॑व - मृ॒शेत् । चतु॑ष्पाद॒ इति॒ चतुः॑ - पा॒दः॒ । अ॒स्य॒ । प॒शवः॑ । प्र॒मायु॑का॒ इति॑ प्र - मायु॑काः । स्युः॒ । यत् । स्कन्दे᳚त् । यज॑मानः । प्र॒मायु॑क॒ इति॑ प्र - मायु॑कः । स्या॒त् । प॒शवः॑ । वै । पृ॒ष॒दा॒ज्यमिति॑ पृषत्-आ॒ज्यम् । प॒शवः॑ । वै । ए॒तस्य॑ । स्क॒न्द॒न्ति॒ । यस्य॑ । पृ॒ष॒दा॒ज्यमिति॑ पृषत् - आ॒ज्यम् । स्कन्द॑ति । यत् । पृ॒ष॒दा॒ज्यमिति॑ पृषत् - आ॒ज्यम् । पुनः॑ । गृ॒ह्णाति॑ । प॒शून् । ए॒व । अ॒स्मै॒ । पुनः॑ । गृ॒ह्णा॒ति॒ । प्रा॒ण इति॑ प्र - अ॒नः । वै । पृ॒ष॒दा॒ज्यमिति॑ पृषत् - आ॒ज्यम् । प्रा॒ण इति॑ प्र - अ॒नः । वै ॥  \newline




\markright{ TS 3.2.6.3  \hfill https://www.vedavms.in \hfill}

\section{ TS 3.2.6.3 }

\textbf{TS 3.2.6.3 } \newline
\textbf{Samhita Paata} \newline

ए॒तस्य॑ स्कन्दति॒ यस्य॑ पृषदा॒ज्यꣳ स्कन्द॑ति॒ यत् पृ॑षदा॒ज्यं पुन॑र्गृ॒ह्णाति॑ प्रा॒णमे॒वास्मै॒ पुन॑र्गृह्णाति॒ हिर॑ण्यमव॒धाय॑ गृह्णात्य॒मृतं॒ ॅवै हिर॑ण्यं प्रा॒णः पृ॑षदा॒ज्यम॒मृत॑मे॒वास्य॑ प्रा॒णे द॑धाति श॒तमा॑नं भवति श॒तायुः॒ पुरु॑षः श॒तेन्द्रि॑य॒ आयु॑ष्ये॒वेन्द्रि॒ये प्रति॑तिष्ठ॒त्यश्व॒मव॑ घ्रापयति प्राजाप॒त्यो वा अश्वः॑ प्राजाप॒त्यः प्रा॒णः स्वादे॒वास्मै॒ योनेः᳚ प्रा॒णं ( ) निर्मि॑मीते॒ वि वा ए॒तस्य॑ य॒ज्ञ्श्छि॑द्यते॒ यस्य॑ पृषदा॒ज्यꣳ स्कन्द॑ति वैष्ण॒व्यर्चा पुन॑र्गृह्णाति य॒ज्ञो वै विष्णु॑र्य॒ज्ञेनै॒व य॒ज्ञ्ꣳ सं त॑नोति ॥ \newline

\textbf{Pada Paata} \newline

ए॒तस्य॑ । स्क॒न्द॒ति॒ । यस्य॑ । पृ॒ष॒दा॒ज्यमिति॑ पृषत् - आ॒ज्यम् । स्कन्द॑ति । यत् । पृ॒ष॒दा॒ज्यमिति॑ पृषत् - आ॒ज्यम् । पुनः॑ । गृ॒ह्णाति॑ । प्रा॒णमिति॑ प्र - अ॒नम् । ए॒व । अ॒स्मै॒ । पुनः॑ । गृ॒ह्णा॒ति॒ । हिर॑ण्यम् । अ॒व॒धायेत्य॑व - धाय॑ । गृ॒ह्णा॒ति॒ । अ॒मृत᳚म् । वै । हिर॑ण्यम् । प्रा॒ण इति॑ प्र - अ॒नः । पृ॒ष॒दा॒ज्यमिति॑ पृषत् - आ॒ज्यम् । अ॒मृत᳚म् । ए॒व । अ॒स्य॒ । प्रा॒ण इति॑ प्र - अ॒ने । द॒धा॒ति॒ । श॒तमा॑न॒मिति॑ श॒त-मा॒न॒म् । भ॒व॒ति॒ । श॒तायु॒रिति॑ श॒त - आ॒युः॒ । पुरु॑षः । श॒तेन्द्रि॑य॒ इति॑ श॒त-इ॒न्द्रि॒यः॒ । आयु॑षि । ए॒व । इ॒न्द्रि॒ये । प्रतीति॑ । ति॒ष्ठ॒ति॒ । अश्व᳚म् । अवेति॑ । घ्रा॒प॒य॒ति॒ । प्रा॒जा॒प॒त्य इति॑ प्राजा - प॒त्यः । वै । अश्वः॑ । प्रा॒जा॒प॒त्य इति॑ प्राजा - प॒त्यः । प्रा॒ण इति॑ प्र-अ॒नः । स्वात् । ए॒व । अ॒स्मै॒ । योनेः᳚ । प्रा॒णमिति॑ प्र - अ॒नम् ( ) । निरिति॑ । मि॒मी॒ते॒ । वीति॑ । वै । ए॒तस्य॑ । य॒ज्ञ्ः । छि॒द्य॒ते॒ । यस्य॑ । पृ॒ष॒दा॒ज्यमिति॑ पृषत् - आ॒ज्यम् । स्कन्द॑ति । वै॒ष्ण॒व्या । ऋ॒चा । पुनः॑ । गृ॒ह्णा॒ति॒ । य॒ज्ञ्ः । वै । विष्णुः॑ । य॒ज्ञेन॑ । ए॒व । य॒ज्ञ्म् । समिति॑ । त॒नो॒ति॒ ॥  \newline




\markright{ TS 3.2.7.1  \hfill https://www.vedavms.in \hfill}

\section{ TS 3.2.7.1 }

\textbf{TS 3.2.7.1 } \newline
\textbf{Samhita Paata} \newline

देव॑ सवितरे॒तत् ते॒ प्राऽऽ*ह॒ तत् प्र च॑ सु॒व प्र च॑ यज॒ बृह॒स्पति॑र्ब्र॒ह्मा ऽऽयु॑ष्मत्या ऋ॒चो मा गा॑त तनू॒पाथ् साम्नः॑ स॒त्या व॑ आ॒शिषः॑ सन्तु स॒त्या आकू॑तय ऋ॒तं च॑ स॒त्यं च॑ वदत स्तु॒त दे॒वस्य॑ सवि॒तुः प्र॑स॒वे स्तु॒तस्य॑ स्तु॒तम॒स्यूर्जं॒ मह्यꣳ॑ स्तु॒तं दु॑हा॒मा मा᳚ स्तु॒तस्य॑ स्तु॒तं ग॑म्याच्छ॒स्त्रस्य॑ श॒स्त्र - [  ] \newline

\textbf{Pada Paata} \newline

देव॑ । स॒वि॒तः॒ । ए॒तत् । ते॒ । प्रेति॑ । आ॒ह॒ । तत् । प्रेति॑ । च॒ । सु॒व । प्रेति॑ । च॒ । य॒ज॒ । बृह॒स्पतिः॑ । ब्र॒ह्मा । आयु॑ष्मत्याः । ऋ॒चः । मा । गा॒त॒ । त॒नू॒पादिति॑ तनू - पात् । साम्नः॑ । स॒त्याः । वः॒ । आ॒शिष॒ इत्या᳚ - शिषः॑ । स॒न्तु॒ । स॒त्याः । आकू॑तय॒ इत्या-कू॒त॒यः॒ । ऋ॒तम् । च॒ । स॒त्यम् । च॒ । व॒द॒त॒ । स्तु॒त । दे॒वस्य॑ । स॒वि॒तुः । प्र॒स॒व इति॑ प्र - स॒वे । स्तु॒तस्य॑ । स्तु॒तम् । अ॒सि॒ । ऊर्ज᳚म् । मह्य᳚म् । स्तु॒तम् । दु॒हा॒म् । एति॑ । मा॒ । स्तु॒तस्य॑ । स्तु॒तम् । ग॒म्या॒त् । श॒स्त्रस्य॑ । श॒स्त्रम् ।  \newline




\markright{ TS 3.2.7.2  \hfill https://www.vedavms.in \hfill}

\section{ TS 3.2.7.2 }

\textbf{TS 3.2.7.2 } \newline
\textbf{Samhita Paata} \newline

म॒स्यूर्जं॒ मह्यꣳ॑ श॒स्त्रं दु॑हा॒मा मा॑ श॒स्त्रस्य॑ श॒स्त्रं ग॑म्या-दिन्द्रि॒याव॑न्तो वनामहे धुक्षी॒महि॑ प्र॒जामिषं᳚ ॥ सा मे॑ स॒त्याऽऽशीर्दे॒वेषु॑ भूयाद्-ब्रह्मवर्च॒सं माऽऽ ग॑म्यात् ॥ य॒ज्ञो ब॑भूव॒ स आ ब॑भूव॒ सप्रज॑ज्ञे॒ स वा॑वृधे । स दे॒वाना॒मधि॑-पतिर्बभूव॒ सो अ॒स्माꣳ अधि॑पतीन् करोतु व॒यꣳ स्या॑म॒ पत॑यो रयी॒णां ॥ य॒ज्ञो वा॒ वै - [  ] \newline

\textbf{Pada Paata} \newline

अ॒सि॒ । ऊर्ज᳚म् । मह्य᳚म् । श॒स्त्रम् । दु॒हा॒म् । एति॑ । मा॒ । श॒स्त्रस्य॑ । श॒स्त्रम् । ग॒म्या॒त् । इ॒न्द्रि॒याव॑न्त॒ इती᳚न्द्रि॒य - व॒न्तः॒ । व॒ना॒म॒हे॒ । धु॒क्षी॒महि॑ । प्र॒जामिति॑ प्र - जाम् । इष᳚म् ॥ सा । मे॒ । स॒त्या । आ॒शीरित्या᳚ - शीः । दे॒वेषु॑ । भू॒या॒त् । ब्र॒ह्म॒व॒र्च॒समिति॑ ब्रह्म - व॒र्च॒सम् । मा॒ । एति॑ । ग॒म्या॒त् ॥ य॒ज्ञ्ः । ब॒भू॒व॒ । सः । एति॑ । ब॒भू॒व॒ । सः । प्रेति॑ । ज॒ज्ञे॒ । सः । वा॒वृ॒धे॒ ॥ सः । दे॒वाना᳚म् । अधि॑पति॒रित्यधि॑ - प॒तिः॒ । ब॒भू॒व॒ । सः । अ॒स्मान् । अधि॑पती॒नित्यधि॑ - प॒ती॒न् । क॒रो॒तु॒ । व॒यम् । स्या॒म॒ । पत॑यः । र॒यी॒णाम् ॥ य॒ज्ञ्ः । वा॒ । वै ।  \newline




\markright{ TS 3.2.7.3  \hfill https://www.vedavms.in \hfill}

\section{ TS 3.2.7.3 }

\textbf{TS 3.2.7.3 } \newline
\textbf{Samhita Paata} \newline

य॒ज्ञ्प॑तिं दु॒हे य॒ज्ञ्प॑तिर्वा य॒ज्ञ्ं दु॑हे॒ स यः स्तु॑तश॒स्त्रयो॒र्दोह॒म वि॑द्वा॒न॒. यज॑ते॒ तं ॅय॒ज्ञो दु॑हे॒ स इ॒ष्ट्वा पापी॑यान् भवति॒ य ए॑नयो॒र्दोहं॑ ॅवि॒द्वान्. यज॑ते॒ स य॒ज्ञ्ं दु॑हे॒ स इ॒ष्ट्वा वसी॑यान् भवति स्तु॒तस्य॑ स्तु॒तम॒स्यूर्जं॒ मह्यꣳ॑ स्तु॒तं दु॑हा॒मा मा᳚ स्तु॒तस्य॑ स्तु॒तं ग॑म्याच्छ॒स्त्रस्य॑ श॒स्त्रम॒स्यूर्जं॒ मह्यꣳ॑ श॒स्त्रं दु॑हा॒ ( ) मा मा॑ श॒स्त्रस्य॑ श॒स्त्रं ग॑म्या॒दित्या॑है॒ष वै स्तु॑तश॒स्त्रयो॒र्दोह॒स्तं ॅय ए॒वं ॅवि॒द्वान्. यज॑ते दु॒ह ए॒व य॒ज्ञ्मि॒ष्ट्वा वसी॑यान् भवति ॥ \newline

\textbf{Pada Paata} \newline

य॒ज्ञ्प॑ति॒मिति॑ य॒ज्ञ् - प॒ति॒म् । दु॒हे । य॒ज्ञ्प॑ति॒रिति॑ य॒ज्ञ्-प॒तिः॒ । वा॒ । य॒ज्ञ्म् । दु॒हे॒ । सः । यः । स्तु॒त॒श॒स्त्रयो॒रिति॑ स्तुत - श॒स्त्रयोः᳚ । दोह᳚म् । अवि॑द्वान् । यज॑ते । तम् । य॒ज्ञ्ः । दु॒हे॒ । सः । इ॒ष्ट्वा । पापी॑यान् । भ॒व॒ति॒ । यः । ए॒न॒योः॒ । दोह᳚म् । वि॒द्वान् । यज॑ते । सः । य॒ज्ञ्म् । दु॒हे॒ । सः । इ॒ष्ट्वा । वसी॑यान् । भ॒व॒ति॒ । स्तु॒तस्य॑ । स्तु॒तम् । अ॒सि॒ । ऊर्ज᳚म् । मह्य᳚म् । स्तु॒तम् । दु॒हा॒म् । एति॑ । मा॒ । स्तु॒तस्य॑ । स्तु॒तम् । ग॒म्या॒त् । श॒स्त्रस्य॑ । श॒स्त्रम् । अ॒सि॒ । ऊर्ज᳚म् । मह्य᳚म् । श॒स्त्रम् । दु॒हा॒म् ( ) । एति॑ । मा॒ । श॒स्त्रस्य॑ । श॒स्त्रम् । ग॒म्या॒त् । इति॑ । आ॒ह॒ । ए॒षः । वै । स्तु॒त॒श॒स्त्रयो॒रिति॑ स्तुत - श॒स्त्रयोः᳚ । दोहः॑ । तम् । यः । ए॒वम् । वि॒द्वान् । यज॑ते । दु॒हे । ए॒व । य॒ज्ञ्म् । इ॒ष्ट्वा । वसी॑यान् । भ॒व॒ति॒ ॥  \newline




\markright{ TS 3.2.8.1  \hfill https://www.vedavms.in \hfill}

\section{ TS 3.2.8.1 }

\textbf{TS 3.2.8.1 } \newline
\textbf{Samhita Paata} \newline

श्ये॒नाय॒ पत्व॑ने॒ स्वाहा॒ वट्थ्स्व॒यम॑भिगूर्ताय॒ नमो॑ विष्ट॒म्भाय॒ धर्म॑णे॒ स्वाहा॒ वट्थ्स्व॒यम॑भिगूर्ताय॒ नमः॑ परि॒धये॑ जन॒प्रथ॑नाय॒ स्वाहा॒ वट्थ्स्व॒यम॑भिगूर्ताय॒ नम॑ ऊ॒र्जे होत्रा॑णाꣳ॒॒ स्वाहा॒ वट्थ्स्व॒यम॑भिगूर्ताय॒ नमः॒ पय॑से॒ होत्रा॑णाꣳ॒॒ स्वाहा॒ वट्थ्स्व॒यम॑भिगूर्ताय॒ नमः॑ प्र॒जाप॑तये॒ मन॑वे॒ स्वाहा॒ वट्थ्स्व॒यम॑भिगूर्ताय॒ नम॑ ऋ॒तमृ॑तपाः सुवर्वा॒ट्थ्स्वाहा॒ वट्थ्स्व॒यम॑भिगूर्ताय॒ नम॑स्तृं॒पन्ताꣳ॒॒ होत्रा॒ मधो᳚र्घृ॒तस्य॑ य॒ज्ञ्प॑ति॒मृष॑य॒ एन॑सा - [  ] \newline

\textbf{Pada Paata} \newline

श्ये॒नाय॑ । पत्व॑ने । स्वाहा᳚ । वट् । स्व॒यम॑भिगूर्ता॒येति॑ स्व॒यं - अ॒भि॒गू॒र्ता॒य॒ । नमः॑ । वि॒ष्ट॒भांयेति॑ वि - स्त॒भांय॑ । धर्म॑णे । स्वाहा᳚ । वट् । स्व॒यम॑भिगूर्ता॒येति॑ स्व॒यं - अ॒भि॒गू॒र्ता॒य॒ । नमः॑ । प॒रि॒धय॒ इति॑ परि - धये᳚ । ज॒न॒प्रथ॑ना॒येति॑ जन - प्रथ॑नाय । स्वाहा᳚ । वट् । स्व॒यम॑भिगूर्ता॒येति॑ स्व॒यं - अ॒भि॒गू॒र्ता॒य॒ । नमः॑ । ऊ॒र्जे । होत्रा॑णाम् । स्वाहा᳚ । वट् । स्व॒यम॑भिगूर्ता॒येति॑ स्व॒यं - अ॒भि॒गू॒र्ता॒य॒ । नमः॑ । पय॑से । होत्रा॑णाम् । स्वाहा᳚ । वट् । स्व॒यम॑भिगूर्ता॒येति॑ स्व॒यं - अ॒भि॒गू॒र्ता॒य॒ । नमः॑ । प्र॒जाप॑तय॒ इति॑ प्र॒जा-प॒त॒ये॒ । मन॑वे । स्वाहा᳚ । वट् । स्व॒यम॑भिगूर्ता॒येति॑ स्व॒यं - अ॒भि॒गू॒र्ता॒य॒ । नमः॑ । ऋ॒तम् । ऋ॒त॒पा॒ इत्यृ॑त - पाः॒ । सु॒व॒र्वा॒डिति॑ सुवः - वा॒ट् । स्वाहा᳚ । वट् । स्व॒यम॑भिगूर्ता॒येति॑ स्व॒यं - अ॒भि॒गू॒र्ता॒य॒ । नमः॑ । तृ॒पंन्ता᳚म् । होत्राः᳚ । मधोः᳚ । घृ॒तस्य॑ । य॒ज्ञ्प॑ति॒मिति॑ य॒ज्ञ् - प॒ति॒म् । ऋष॑यः । एन॑सा ।  \newline




\markright{ TS 3.2.8.2  \hfill https://www.vedavms.in \hfill}

\section{ TS 3.2.8.2 }

\textbf{TS 3.2.8.2 } \newline
\textbf{Samhita Paata} \newline

ऽऽहुः । प्र॒जा निर्भ॑क्ता अनुत॒प्यमा॑ना मध॒व्यौ᳚ स्तो॒कावप॒ तौ र॑राध ॥ सं न॒स्ताभ्याꣳ॑ सृजतुवि॒श्वक॑र्मा घो॒रा ऋष॑यो॒ नमो॑ अस्त्वेभ्यः । चक्षु॑ष एषां॒ मन॑सश्च स॒न्धौ बृह॒स्पत॑ये॒ महि॒ षद् द्यु॒मन्नमः॑ ॥ नमो॑ वि॒श्वक॑र्मणे॒ स उ॑ पात्व॒स्मान॑न॒न्यान्थ्-सो॑म॒पान् मन्य॑मानः । प्रा॒णस्य॑ वि॒द्वान्थ् स॑म॒रे न धीर॒ एन॑श्चकृ॒वान् महि॑ ब॒द्ध ए॑षां ॥ तं ॅवि॑श्वकर्म॒न् - [  ] \newline

\textbf{Pada Paata} \newline

आ॒हुः॒ ॥ प्र॒जा इति॑ प्र - जाः । निर्भ॑क्ता॒ इति॒ निः - भ॒क्ताः॒ । अ॒नु॒त॒प्यमा॑ना॒ इत्य॑नु-त॒प्यमा॑नाः । म॒ध॒व्यौ᳚ । स्तो॒कौ । अपेति॑ । तौ । र॒रा॒ध॒ ॥ समिति॑ । नः॒ । ताभ्या᳚म् । सृ॒ज॒तु॒ । वि॒श्वक॒र्मेति॑ वि॒श्व - क॒र्मा॒ । घो॒राः । ऋष॑यः । नमः॑ । अ॒स्तु॒ । ए॒भ्यः॒ ॥ चक्षु॑षः । ए॒षा॒म् । मन॑सः । च॒ । स॒धांविति॑ सं - धौ । बृह॒स्पत॑ये । महि॑ । सत् । द्यु॒मदिति॑ द्यु - मत् । नमः॑ ॥ नमः॑ । वि॒श्वक॑र्मण॒ इति॑ वि॒श्व - क॒र्म॒णे॒ । सः । उ॒ । पा॒तु॒ । अ॒स्मान् । अ॒न॒न्यान् । सो॒म॒पानिति॑ सोम - पान् । मन्य॑मानः ॥ प्रा॒णस्येति॑ प्र - अ॒नस्य॑ । वि॒द्वान् । स॒म॒र इति॑ सं-अ॒रे । न । धीरः॑ । एनः॑ । च॒कृ॒वान् । महि॑ । ब॒द्धः । ए॒षा॒म् ॥ तम् । वि॒श्व॒क॒र्म॒न्निति॑ विश्व - क॒र्म॒न्न् ।  \newline




\markright{ TS 3.2.8.3  \hfill https://www.vedavms.in \hfill}

\section{ TS 3.2.8.3 }

\textbf{TS 3.2.8.3 } \newline
\textbf{Samhita Paata} \newline

-प्र मु॑ञ्चा स्व॒स्तये॒ ये भ॒क्षय॑न्तो॒ न वसू᳚न्यानृ॒हुः । यान॒ग्नयो॒ऽन्वत॑प्यन्त॒ धिष्णि॑या इ॒यं तेषा॑मव॒या दुरि॑ष्ट्यै॒ स्वि॑ष्टिं न॒स्तां कृ॑णोतु वि॒श्वक॑र्मा ॥ नमः॑ पि॒तृभ्यो॑ अ॒भि ये नो॒ अख्य॑न्. यज्ञ्॒कृतो॑ य॒ज्ञ्का॑माः सुदे॒वा अ॑का॒मा वो॒ दक्षि॑णां॒ न नी॑निम॒ मा न॒स्तस्मा॒ देन॑सः पापयिष्ट । याव॑न्तो॒ वै स॑द॒स्या᳚स्ते सर्वे॑ दक्षि॒ण्या᳚स्तेभ्यो॒ यो दक्षि॑णां॒ न-  [  ] \newline

\textbf{Pada Paata} \newline

प्रेति॑ । मु॒ञ्च॒ । स्व॒स्तये᳚ । ये । भ॒क्षय॑न्तः । न । वसू॑नि । आ॒नृ॒हुः ॥ यान् । अ॒ग्नयः॑ । अ॒न्वत॑प्य॒न्तेत्य॑नु - अत॑प्यन्त । धिष्णि॑याः । इ॒यम् । तेषा᳚म् । अ॒व॒या । दुरि॑ष्ट्या॒ इति॒ दुः - इ॒ष्ट्यै॒ । स्वि॑ष्टि॒मिति॒ सु - इ॒ष्टि॒म् । नः॒ । ताम् । कृ॒णो॒तु॒ । वि॒श्वक॒र्मेति॑ वि॒श्व - क॒र्मा॒ ॥ नमः॑ । पि॒तृभ्य॒ इति॑ पि॒तृ - भ्यः॒ । अ॒भीति॑ । ये । नः॒ । अख्यन्न्॑ । य॒ज्ञ्॒कृत॒ इति॑ यज्ञ् - कृतः॑ । य॒ज्ञ्का॑मा॒ इति॑ य॒ज्ञ् - का॒माः॒ । सु॒दे॒वा इति॑ सु - दे॒वाः । अ॒का॒माः । वः॒ । दक्षि॑णाम् । न । नी॒नि॒म॒ । मा । नः॒ । तस्मा᳚त् । एन॑सः । पा॒प॒यि॒ष्ट॒ ॥ याव॑न्तः । वै । स॒द॒स्याः᳚ । ते । सर्वे᳚ । द॒क्षि॒ण्याः᳚ । तेभ्यः॑ । यः । दक्षि॑णाम् । न ।  \newline




\markright{ TS 3.2.8.4  \hfill https://www.vedavms.in \hfill}

\section{ TS 3.2.8.4 }

\textbf{TS 3.2.8.4 } \newline
\textbf{Samhita Paata} \newline

नये॒दैभ्यो॑ वृश्च्येत॒ यद्-वै᳚श्वकर्म॒णानि॑ जु॒होति॑ सद॒स्या॑ने॒व तत् प्री॑णात्य॒स्मे दे॑वासो॒ वपु॑षे चिकिथ्सत॒ यमा॒शिरा॒ दम्प॑ती वा॒मम॑श्ञु॒तः । पुमा᳚न् पु॒त्रो जा॑यते वि॒न्दते॒ वस्वथ॒ विश्वे॑ अर॒पा ए॑धते गृ॒हः ॥ आ॒शी॒र्दा॒या दम्प॑ती वा॒मम॑श्ञुता॒मरि॑ष्टो॒ रायः॑ सचताꣳ॒॒ समो॑कसा । य आऽसि॑च॒थ् सं दु॑ग्धं कु॒म्भ्या स॒हेष्टेन॒ याम॒न्नम॑तिं जहातु॒ सः ॥ स॒र्पि॒र्ग्री॒वी - [  ] \newline

\textbf{Pada Paata} \newline

नये᳚त् । एति॑ । ए॒भ्यः॒ । वृ॒श्च्ये॒त॒ । यत् । वै॒श्व॒क॒र्म॒णानीति॑ वैश्व - क॒र्म॒णानि॑ । जु॒होति॑ । स॒द॒स्यान्॑ । ए॒व । तत् । प्री॒णा॒ति॒ । अ॒स्मे इति॑ । दे॒वा॒सः॒ । वपु॑षे । चि॒कि॒थ्स॒त॒ । यम् । आ॒शिरा᳚ । दंप॑ती॒ इति॑ । वा॒मम् । अ॒श्नु॒तः ॥ पुमान्॑ । पु॒त्रः । जा॒य॒ते॒ । वि॒न्दते᳚ । वसु॑ । अथ॑ । विश्वे᳚ । अ॒र॒पाः । ए॒ध॒ते॒ । गृ॒हः ॥ आ॒शी॒र्दा॒येत्या॑शीः - दा॒या । दंप॑ती॒ इति॑ । वा॒मम् । अ॒श्नु॒ता॒म् । अरि॑ष्टः । रायः॑ । स॒च॒ता॒म् । समो॑क॒सेति॒ सं - ओ॒क॒सा॒ ॥ यः । एति॑ । असि॑चत् । संदु॑ग्ध॒मिति॒ सं - दु॒ग्ध॒म् । कु॒भ्यां । स॒ह । इ॒ष्टेन॑ । यामन्न्॑ । अम॑तिम् । ज॒हा॒तु॒ । सः ॥ स॒र्पि॒र्ग्री॒वीति॑ सर्पिः - ग्री॒वी ।  \newline




\markright{ TS 3.2.8.5  \hfill https://www.vedavms.in \hfill}

\section{ TS 3.2.8.5 }

\textbf{TS 3.2.8.5 } \newline
\textbf{Samhita Paata} \newline

पीव॑र्यस्य जा॒या पीवा॑नः पु॒त्रा अकृ॑शासो अस्य । स॒हजा॑नि॒र्यः सु॑मख॒स्यमा॑न॒ इन्द्रा॑या॒ऽऽ*शिरꣳ॑ स॒ह कु॒म्भ्याऽदा᳚त् ॥ आ॒शीर्म॒ ऊर्ज॑मु॒त सु॑प्रजा॒स्त्वमिषं॑ दधातु॒ द्रवि॑णꣳ॒॒ सव॑र्चसं । सं॒ जय॒न् क्षेत्रा॑णि॒ सह॑सा॒ऽहमि॑न्द्र कृण्वा॒नो अ॒न्याꣳ अध॑रान्थ्स॒पत्नान्॑ ॥ भू॒तम॑सि भू॒ते मा॑ धा॒ मुख॑मसि॒ मुखं॑ भूयासं॒ द्यावा॑पृथि॒वीभ्यां᳚ त्वा॒ परि॑गृह्णामि॒ विश्वे᳚ त्वा दे॒वा वै᳚श्वान॒राः - [  ] \newline

\textbf{Pada Paata} \newline

पीव॑री । अ॒स्य॒ । जा॒या । पीवा॑नः । पु॒त्राः । अकृ॑शासः । अ॒स्य॒ ॥ स॒हजा॑नि॒रिति॑ स॒ह - जा॒निः॒ । यः । सु॒म॒ख॒स्यमा॑न॒ इति॑ सु - म॒ख॒स्यमा॑नः । इन्द्रा॑य । आ॒शिर᳚म् । स॒ह । कु॒भ्यां । अदा᳚त् ॥ आ॒शीरित्या᳚ - शीः । मे॒ । ऊर्ज᳚म् । उ॒त । सु॒प्र॒जा॒स्त्वमिति॑ सुप्रजाः - त्वम् । इष᳚म् । द॒धा॒तु॒ । द्रवि॑णम् । सव॑र्चस॒मिति॒ स - व॒र्च॒स॒म् ॥ स॒जंय॒न्निति ॑ सं - जयन्न्॑ । क्षेत्रा॑णि । सह॑सा । अ॒हम् । इ॒न्द्र॒ । कृ॒ण्वा॒नः । अ॒न्यान् । अध॑रान् । स॒पत्नान्॑ ॥ भू॒तम् । अ॒सि॒ । भू॒ते । मा॒ । धाः॒ । मुख᳚म् । अ॒सि॒ । मुख᳚म् । भू॒या॒स॒म् । द्यावा॑पृथि॒वीभ्या॒मिति॒ द्यावा᳚ - पृ॒थि॒वीभ्या᳚म् । त्वा॒ । परीति॑ । गृ॒ह्णा॒मि॒ । विश्वे᳚ । त्वा॒ । दे॒वाः । वै॒श्वा॒न॒राः ।  \newline




\markright{ TS 3.2.8.6  \hfill https://www.vedavms.in \hfill}

\section{ TS 3.2.8.6 }

\textbf{TS 3.2.8.6 } \newline
\textbf{Samhita Paata} \newline

प्रच्या॑वयन्तु दि॒वि दे॒वान् दृꣳ॑हा॒न्तरि॑क्षे॒ वयाꣳ॑सि पृथि॒व्यां पार्थि॑वान् ध्रु॒वं ध्रु॒वेण॑ ह॒विषाऽव॒ सोमं॑ नयामसि । यथा॑ नः॒ सर्व॒मिज्जग॑दय॒क्ष्मꣳ सु॒मना॒ अस॑त् । यथा॑ न॒ इन्द्र॒ इद्विशः॒ केव॑लीः॒ सर्वाः॒ सम॑नसः॒ कर॑त् । यथा॑ नः॒ सर्वा॒ इद्दिशो॒ऽस्माकं॒ केव॑ली॒रसन्न्॑ ॥ \newline

\textbf{Pada Paata} \newline

प्रेति॑ । च्या॒व॒य॒न्तु॒ । दि॒वि । दे॒वान् । दृꣳ॒॒ह॒ । अ॒न्तरि॑क्षे । वयाꣳ॑सि । पृ॒थि॒व्याम् । पार्थि॑वान् । ध्रु॒वम् । ध्रु॒वेण॑ । ह॒विषा᳚ । अवेति॑ । सोम᳚म् । न॒या॒म॒सि॒ ॥ यथा᳚ । नः॒ । सर्व᳚म् । इत् । जग॑त् । अ॒य॒क्ष्मम् । सु॒मना॒ इति॑ सु - मनाः᳚ । अस॑त् ॥ यथा᳚ । नः॒ । इन्द्रः॑ । इत् । विशः॑ । केव॑लीः । सर्वाः᳚ । सम॑नस॒ इति॒ स - म॒न॒सः॒ । कर॑त् ॥ यथा᳚ । नः॒ । सर्वाः᳚ । इत् । दिशः॑ । अ॒स्माक᳚म् । केव॑लीः । असन्न्॑ ॥  \newline




\markright{ TS 3.2.9.1  \hfill https://www.vedavms.in \hfill}

\section{ TS 3.2.9.1 }

\textbf{TS 3.2.9.1 } \newline
\textbf{Samhita Paata} \newline

यद्वै होता᳚ऽद्ध्व॒र्युम॑भ्या॒ह्वय॑ते॒ वज्र॑मेनम॒भि प्रव॑र्तय॒त्युक्थ॑शा॒ इत्या॑ह प्रातस्सव॒नं प्र॑ति॒गीर्य॒ त्रीण्ये॒तान्य॒क्षरा॑णि त्रि॒पदा॑ गाय॒त्री गा॑य॒त्रं प्रा॑तस्सव॒नं गा॑यत्रि॒यैव प्रा॑तस्सव॒ने वज्र॑म॒न्तर्द्ध॑त्त उ॒क्थं ॅवा॒चीत्या॑ह॒ माद्ध्य॑दिंनꣳ॒॒ सव॑नं प्रति॒गीर्य॑ च॒त्वार्ये॒तान्य॒-क्षरा॑णि॒ चतु॑ष्पदा त्रि॒ष्टुप् त्रैष्टु॑भं॒ माद्ध्य॑दिंनꣳ॒॒ सव॑नं त्रि॒ष्टुभै॒व माद्ध्य॑न्दिने॒ सव॑ने॒ वज्र॑म॒न्तर्द्ध॑त्त-  [  ] \newline

\textbf{Pada Paata} \newline

यत् । वै । होता᳚ । अ॒द्ध्व॒र्युम् । अ॒भ्या॒ह्वय॑त॒ इत्य॑भि - आ॒ह्वय॑ते । वज्र᳚म् । ए॒न॒म् । अ॒भि । प्रेति॑ । व॒र्त॒य॒ति॒ । उक्थ॑शा॒ इत्युक्थ॑-शाः॒ । इति॑ । आ॒ह॒ । प्रा॒त॒स्स॒व॒नमिति॑ प्रातः - स॒व॒नम् । प्र॒ति॒गीर्येति॑ प्रति - गीर्य॑ । त्रीणि॑ । ए॒तानि॑ । अ॒क्षरा॑णि । त्रि॒पदेति॑ त्रि - पदा᳚ । गा॒य॒त्री । गा॒य॒त्रम् । प्रा॒त॒स्स॒व॒नमिति॑ प्रातः - स॒व॒नम् । गा॒य॒त्रि॒या । ए॒व । प्रा॒त॒स्स॒व॒न इति॑ प्रातः - स॒व॒ने । वज्र᳚म् । अ॒न्तः । ध॒त्ते॒ । उ॒क्थम् । वा॒चि । इति॑ । आ॒ह॒ । माद्ध्य॑न्दिनम् । सव॑नम् । प्र॒ति॒गीर्येति॑ प्रति - गीर्य॑ । च॒त्वारि॑ । ए॒तानि॑ । अ॒क्षरा॑णि । चतु॑ष्प॒देति॒ चतुः॑ - प॒दा॒ । त्रि॒ष्टुप् । त्रैष्टु॑भम् । माद्ध्य॑न्दिनम् । सव॑नम् । त्रि॒ष्टुभा᳚ । ए॒व । माद्ध्य॑न्दिने । सव॑ने । वज्र᳚म् । अ॒न्तः । ध॒त्ते॒ ।  \newline




\markright{ TS 3.2.9.2  \hfill https://www.vedavms.in \hfill}

\section{ TS 3.2.9.2 }

\textbf{TS 3.2.9.2 } \newline
\textbf{Samhita Paata} \newline

उ॒क्थं ॅवा॒चीन्द्रा॒येत्या॑ह तृतीयसव॒नं प्र॑ति॒गीर्य॑ स॒प्तैतान्य॒क्षरा॑णि स॒प्तप॑दा॒ शक्व॑री शाक्व॒रो वज्रो॒ वज्रे॑णै॒व तृ॑तीयसव॒ने वज्र॑म॒न्तर्द्ध॑त्ते ब्रह्मवा॒दिनो॑ वदन्ति॒ स त्वा अ॑द्ध्व॒र्युः स्या॒द्यो य॑थासव॒नं प्र॑तिग॒रे छन्दाꣳ॑सि सम्पा॒दये॒त् तेजः॑ प्रातः सव॒न आ॒त्मन् दधी॑तेन्द्रि॒यं माद्ध्य॑न्दिने॒ सव॑ने प॒शूꣳ स्तृ॑तीयसव॒न इत्युक्थ॑शा॒ इत्या॑ह प्रातस्सव॒नं प्र॑ति॒गीर्य॒ त्रीण्ये॒तान्य॒क्षरा॑णि - [  ] \newline

\textbf{Pada Paata} \newline

उ॒क्थम् । वा॒चि । इन्द्रा॑य । इति॑ । आ॒ह॒ । तृ॒ती॒य॒स॒व॒नमिति॑ तृतीय -स॒व॒नम् । प्र॒ति॒गीर्येति॑ प्रति - गीर्य॑ । स॒प्त । ए॒तानि॑ । अ॒क्षरा॑णि । स॒प्तप॒देति॑ स॒प्त - प॒दा॒ । शक्व॑री । शा॒क्व॒रः । वज्रः॑ । वज्रे॑ण । ए॒व । तृ॒ती॒य॒स॒व॒न इति॑ तृतीय - स॒व॒ने । वज्र᳚म् । अ॒न्तः । ध॒त्ते॒ । ब्र॒ह्म॒वा॒दिन॒ इति॑ ब्रह्म - वा॒दिनः॑ । व॒द॒न्ति॒ । सः । तु । वै । अ॒ध्व॒र्युः । स्या॒त् । यः । य॒था॒स॒व॒नमिति॑ यथा - स॒व॒नम् । प्र॒ति॒ग॒र इति॑ प्रति - ग॒रे । छन्दाꣳ॑सि । स॒पां॒दये॒दिति॑ सं - पा॒दये᳚त् । तेजः॑ । प्रा॒त॒स्स॒व॒न इति॑ प्रातः - स॒व॒ने । आ॒त्मन्न् । दधी॑त । इ॒न्द्रि॒यम् । माध्य॑न्दिने । सव॑ने । प॒शून् । तृ॒ती॒य॒स॒व॒न इति॑ तृतीय - स॒व॒ने । इति॑ । उक्थ॑शा॒ इत्युक्थ॑ - शाः॒ । इति॑ । आ॒ह॒ । प्रा॒त॒स्स॒व॒नमिति॑ प्रातः - स॒व॒नम् । प्र॒ति॒गीर्येति॑ प्रति - गीर्य॑ । त्रीणि॑ । ए॒तानि॑ । अ॒क्षरा॑णि ।  \newline




\markright{ TS 3.2.9.3  \hfill https://www.vedavms.in \hfill}

\section{ TS 3.2.9.3 }

\textbf{TS 3.2.9.3 } \newline
\textbf{Samhita Paata} \newline

त्रि॒पदा॑ गाय॒त्री गा॑य॒त्रं प्रा॑तस्सव॒नं प्रा॑तस्सव॒न ए॒व प्र॑तिग॒रे छन्दाꣳ॑सि॒ संपा॑दय॒त्यथो॒ तेजो॒ वै गा॑य॒त्री तेजः॑ प्रातः सव॒नं तेज॑ ए॒व प्रा॑तस्सव॒न आ॒त्मन् ध॑त्त उ॒क्थं ॅवा॒चीत्या॑ह॒ माद्ध्य॑न्दिनꣳ॒॒ सव॑नं प्रति॒गीर्य॑ च॒त्वार्ये॒तान्य॒क्षरा॑णि॒ चतु॑ष्पदा त्रि॒ष्टुप् त्रैष्टु॑भं॒ माद्ध्य॑न्दिनꣳ॒॒ सव॑नं॒ माद्ध्य॑दिंन ए॒व सव॑ने प्रतिग॒रे छन्दाꣳ॑सि॒ संपा॑दय॒त्यथो॑ इन्द्रि॒यं ॅवै त्रि॒ष्टुगि॑न्द्रि॒यं माद्ध्य॑दिंनꣳ॒॒ सव॑न - [  ] \newline

\textbf{Pada Paata} \newline

त्रि॒पदेति॑ त्रि - पदा᳚ । गा॒य॒त्री । गा॒य॒त्रम् । प्रा॒त॒स्स॒व॒नमिति॑ प्रातः - स॒व॒नम् । प्रा॒त॒स्स॒व॒न इति॑ प्रातः - स॒व॒ने । ए॒व । प्र॒ति॒ग॒र इति॑ प्रति - ग॒रे । छन्दाꣳ॑सि । समिति॑ । पा॒द॒य॒ति॒ । अथो॒ इति॑ । तेजः॑ । वै । गा॒य॒त्री । तेजः॑ । प्रा॒त॒स्स॒व॒नमिति॑ प्रातः - स॒व॒नम् । तेजः॑ । ए॒व । प्रा॒त॒स्स॒व॒न इति॑ प्रातः - स॒व॒ने । आ॒त्मन्न् । ध॒त्ते॒ । उ॒क्थम् । वा॒चि । इति॑ । आ॒ह॒ । माद्ध्य॑न्दिनम् । सव॑नम् । प्र॒ति॒गीर्येति॑ प्रति - गीर्य॑ । च॒त्वारि॑ । ए॒तानि॑ । अ॒क्षरा॑णि । चतु॑ष्प॒देति॒ चतुः॑ - प॒दा॒ । त्रि॒ष्टुप् । त्रैष्टु॑भम् । माद्ध्य॑न्दिनम् । सव॑नम् । माद्ध्य॑न्दिने । ए॒व । सव॑ने । प्र॒ति॒ग॒र इति॑ प्रति - ग॒रे । छन्दाꣳ॑सि । समिति॑ । पा॒द॒य॒ति॒ । अथो॒ इति॑ । इ॒न्द्रि॒यम् । वै । त्रि॒ष्टुक् । इ॒न्द्रि॒यम् । माद्ध्य॑न्दिनम् । सव॑नम् ।  \newline




\markright{ TS 3.2.9.4  \hfill https://www.vedavms.in \hfill}

\section{ TS 3.2.9.4 }

\textbf{TS 3.2.9.4 } \newline
\textbf{Samhita Paata} \newline

-मिन्द्रि॒यमे॒व माद्ध्य॑न्दिने॒ सव॑न आ॒त्मन् ध॑त्त उ॒क्थं ॅवा॒चीन्द्रा॒येत्या॑ह तृतीयसव॒नं प्र॑ति॒गीर्य॑ स॒प्तैतान्य॒क्षरा॑णि स॒प्तप॑दा॒ शक्व॑री शाक्व॒राः प॒शवो॒ जाग॑तं तृतीयसव॒नं तृ॑तीयसव॒न ए॒व प्र॑तिग॒रे छन्दाꣳ॑सि॒ संपा॑दय॒त्यथो॑ प॒शवो॒ वै जग॑ती प॒शव॑स्तृतीयसव॒नं प॒शूने॒व तृ॑तीयसव॒न आ॒त्मन् ध॑त्ते॒ यद्वै होता᳚ऽद्ध्व॒र्युम॑भ्या॒ह्वय॑त आ॒व्य॑मस्मिन् दधाति॒ तद्यन्ना - [  ] \newline

\textbf{Pada Paata} \newline

इ॒न्द्रि॒यम् । ए॒व । माद्ध्य॑न्दिने । सव॑ने । आ॒त्मन्न् । ध॒त्ते॒ । उ॒क्थम् । वा॒चि । इन्द्रा॑य । इति॑ । आ॒ह॒ । तृ॒ती॒य॒स॒व॒नमिति॑ तृतीय - स॒व॒नम् । प्र॒ति॒गीर्येति॑ प्रति - गीर्य॑ । स॒प्त । ए॒तानि॑ । अ॒क्षरा॑णि । स॒प्तप॒देति॑ स॒प्त - प॒दा॒ । शक्व॑री । शा॒क्व॒राः । प॒शवः॑ । जाग॑तम् । तृ॒ती॒य॒स॒व॒नमिति॑ तृतीय - स॒व॒नम् । तृ॒ती॒य॒स॒व॒न इति॑ तृतीय - स॒व॒ने । ए॒व । प्र॒ति॒ग॒र इति॑ प्रति - ग॒रे । छन्दाꣳ॑सि । समिति॑ । पा॒द॒य॒ति॒ । अथो॒ इति॑ । प॒शवः॑ । वै । जग॑ती । प॒शवः॑ । तृ॒ती॒य॒स॒व॒नमिति॑ तृतीय - स॒व॒नम् । प॒शून् । ए॒व । तृ॒ती॒य॒स॒व॒न इति॑ तृतीय - स॒व॒ने । आ॒त्मन्न् । ध॒त्ते॒ । यत् । वै । होता᳚ । अ॒द्ध्व॒र्युम् । अ॒भ्या॒ह्वय॑त॒ इत्य॑भि - आ॒ह्वय॑ते । आ॒व्य᳚म् । अ॒स्मि॒न्न् । द॒धा॒ति॒ । तत् । यत् । न ।  \newline




\markright{ TS 3.2.9.5  \hfill https://www.vedavms.in \hfill}

\section{ TS 3.2.9.5 }

\textbf{TS 3.2.9.5 } \newline
\textbf{Samhita Paata} \newline

प॒हनी॑त पु॒राऽस्य॑ संॅवथ्स॒राद्-गृ॒ह आ वे॑वीर॒ञ्छोꣳसा॒ मोद॑ इ॒वेति॑ प्र॒त्याह्व॑यते॒ तेनै॒व तदप॑ हते॒ यथा॒ वा आय॑तां प्र॒तीक्ष॑त ए॒वम॑द्ध्व॒र्युः प्र॑तिग॒रं प्रती᳚क्षते॒ यद॑भि प्रतिगृणी॒याद्यथा ऽऽय॑तया समृ॒च्छते॑ ता॒दृगे॒व तद्यद॑र्द्ध॒र्चाल्लुप्ये॑त॒ यथा॒ धाव॑द्भ्यो॒ हीय॑ते ता॒दृगे॒व तत् प्र॒बाहु॒ग्वा ऋ॒त्विजा॑मुद्गी॒था उ॑द्गी॒थ ए॒वोद्-गा॑तृ॒णा - [  ] \newline

\textbf{Pada Paata} \newline

अ॒प॒हनी॒तेत्यप॑ - हनी॑त । पु॒रा । अ॒स्य॒ । सं॒ॅव॒थ्स॒रादिति॑ सं - व॒थ्स॒रात् । गृ॒हे । एति॑ । वे॒वी॒र॒न्न् । शोꣳसा᳚ । मोदः॑ । इ॒व॒ । इति॑ । प्र॒त्याह्व॑यत॒ इति॑ प्रति - आह्व॑यते । तेन॑ । ए॒व । तत् । अपेति॑ । ह॒ते॒ । यथा᳚ । वै । आय॑ता॒मित्या - य॒ता॒म् । प्र॒तीक्ष॑त॒ इति॑ प्रति - ईक्ष॑ते । ए॒वम् । अ॒द्ध्व॒र्युः । प्र॒ति॒ग॒रमिति॑ प्रति - ग॒रम् । प्रतीति॑ । ई॒क्ष॒ते॒ । यत् । अ॒भि॒प्र॒ति॒गृ॒णी॒यादित्य॑भि - प्र॒ति॒गृ॒णी॒यात् । यथा᳚ । आय॑त॒येत्या - य॒त॒या॒ । स॒मृ॒च्छत॒ इति॑ सं - ऋ॒च्छते᳚ । ता॒दृक् । ए॒व । तत् । यत् । अ॒र्द्ध॒र्चादित्य॑र्द्ध-ऋ॒चात् । लुप्ये॑त । यथा᳚ । धाव॑द्भ्य॒ इति॒ धाव॑त् - भ्यः॒ । हीय॑ते । ता॒दृक् । ए॒व । तत् । प्र॒बाहु॒गिति॑ प्र - बाहु॑क् । वै । ऋ॒त्विजा᳚म् । उ॒द्गी॒था इत्यु॑त्-गी॒थाः । उ॒द्गी॒थ इत्यु॑त् - गी॒थः । ए॒व । उ॒द्गा॒तृ॒णामित्यु॑त्-गा॒तृ॒णाम् ।  \newline




\markright{ TS 3.2.9.6  \hfill https://www.vedavms.in \hfill}

\section{ TS 3.2.9.6 }

\textbf{TS 3.2.9.6 } \newline
\textbf{Samhita Paata} \newline

मृ॒चः प्र॑ण॒व उ॑क्थशꣳ॒॒सिनां᳚ प्रतिग॒रो᳚ऽद्ध्वर्यू॒णां ॅय ए॒वं ॅवि॒द्वान् प्र॑तिगृ॒णात्य॑न्ना॒द ए॒व भ॑व॒त्याऽस्य॑ प्र॒जायां᳚ ॅवा॒जी जा॑यत इ॒यं ॅवै होता॒ऽसाव॑द्ध्व॒र्युर्यदासी॑नः॒ शꣳ स॑त्य॒स्या ए॒व तद्धोता॒ नैत्यास्त॑ इव॒ हीयमथो॑ इ॒मामे॒व तेन॒ यज॑मानो दुहे॒ यत् तिष्ठ॑न् प्रतिगृ॒णात्य॒मुष्या॑ ए॒व तद॑द्ध्व॒र्युर्नैति॒ - [  ] \newline

\textbf{Pada Paata} \newline

ऋ॒चः । प्र॒ण॒व इति॑ प्र - न॒वः । उ॒क्थ॒शꣳ॒॒सिना॒मित्यु॑क्थ - शꣳ॒॒सिना᳚म् । प्र॒ति॒ग॒र इति॑ प्रति - ग॒रः । अ॒द्ध्व॒र्यू॒णाम् । यः । ए॒वम् । वि॒द्वान् । प्र॒ति॒गृ॒णातीति॑ प्रति-गृ॒णाति॑ । अ॒न्ना॒द इत्य॑न्न - अ॒दः । ए॒व । भ॒व॒ति॒ । एति॑ । अ॒स्य॒ । प्र॒जाया॒मिति॑ प्र - जाया᳚म् । वा॒जी । जा॒य॒ते॒ । इ॒यम् । वै । होता᳚ । अ॒सौ । अ॒द्ध्व॒र्युः । यत् । आसी॑नः । शꣳस॑ति । अ॒स्याः । ए॒व । तत् । होता᳚ । न । ए॒ति॒ । आस्ते᳚ । इ॒व॒ । हि । इ॒यम् । अथो॒ इति॑ । इ॒माम् । ए॒व । तेन॑ । यज॑मानः । दु॒हे॒ । यत् । तिष्ठन्न्॑ । प्र॒ति॒गृ॒णातीति॑ प्रति - गृ॒णाति॑ । अ॒मुष्याः᳚ । ए॒व । तत् । अ॒द्ध्व॒र्युः । न । ए॒ति॒ ।  \newline




\markright{ TS 3.2.9.7  \hfill https://www.vedavms.in \hfill}

\section{ TS 3.2.9.7 }

\textbf{TS 3.2.9.7 } \newline
\textbf{Samhita Paata} \newline

तिष्ठ॑तीव॒ ह्य॑सावथो॑ अ॒मूमे॒व तेन॒ यज॑मानो दुहे॒ यदासी॑नः॒ शꣳस॑ति॒ तस्मा॑दि॒तः प्र॑दानं दे॒वा उप॑ जीवन्ति॒ यत् तिष्ठ॑न् प्रतिगृ॒णाति॒ तस्मा॑द॒मुतः॑ प्रदानं मनु॒ष्या॑ उप॑ जीवन्ति॒ यत् प्राङासी॑नः॒ शꣳस॑ति प्र॒त्यङ् तिष्ठ॑न् प्रतिगृ॒णाति॒ तस्मा᳚त् प्रा॒चीनꣳ॒॒ रेतो॑ धीयते प्र॒तीचीः᳚ प्र॒जा जा॑यन्ते॒ यद्वै होता᳚ऽद्ध्व॒र्युम॑भ्या॒ह्वय॑ते॒ वज्र॑मेनम॒भि प्रव॑र्तयति॒ ( ) परा॒ङा व॑र्तते॒ वज्र॑मे॒व तन्नि क॑रोति ॥ \newline

\textbf{Pada Paata} \newline

तिष्ठ॑ति । इ॒व॒ । हि । अ॒सौ । अथो॒ इति॑ । अ॒मूम् । ए॒व । तेन॑ । यज॑मानः । दु॒हे॒ । यत् । आसी॑नः । शꣳस॑ति । तस्मा᳚त् । इ॒तः प्र॑दान॒मिती॒तः - प्र॒दा॒न॒म् । दे॒वाः । उपेति॑ । जी॒व॒न्ति॒ । यत् । तिष्ठन्न्॑ । प्र॒ति॒गृ॒णातीति॑ प्रति - गृ॒णाति॑ । तस्मा᳚त् । अ॒मुतः॑ प्रदान॒मित्य॒मुतः॑ - प्र॒दा॒न॒म् । म॒नु॒ष्याः᳚ । उपेति॑ । जी॒व॒न्ति॒ । यत् । प्राङ् । आसी॑नः । शꣳस॑ति । प्र॒त्यङ् । तिष्ठन्न्॑ । प्र॒ति॒गृ॒णातीति॑ प्रति - गृ॒णाति॑ । तस्मा᳚त् । प्रा॒चीन᳚म् । रेतः॑ । धी॒य॒ते॒ । प्र॒तीचीः᳚ । प्र॒जा इति॑ प्र - जाः । जा॒य॒न्ते॒ । यत् । वै । होता᳚ । अ॒द्ध्व॒र्युम् । अ॒भ्या॒ह्वय॑त॒ इत्य॑भि - आ॒ह्वय॑ते । वज्र᳚म् । ए॒न॒म् । अ॒भि । प्रेति॑ । व॒र्त॒य॒ति॒ ( ) । पराङ्॑ । एति॑ । व॒र्त॒ते॒ । वज्र᳚म् । ए॒व । तत् । नीति॑ । क॒रो॒ति॒ ॥  \newline




\markright{ TS 3.2.10.1  \hfill https://www.vedavms.in \hfill}

\section{ TS 3.2.10.1 }

\textbf{TS 3.2.10.1 } \newline
\textbf{Samhita Paata} \newline

उ॒प॒या॒मगृ॑हीतोऽसि वाक्ष॒सद॑सि वा॒क्पाभ्यां᳚ त्वा क्रतु॒पाभ्या॑म॒स्य य॒ज्ञ्स्य॑ ध्रु॒वस्याऽद्ध्य॑-क्षाभ्यां गृह्णा-म्युपया॒मगृ॑हीतोऽस्यृत॒सद॑सि चक्षु॒ष्पाभ्यां᳚ त्वा क्रतु॒पाभ्या॑म॒स्य य॒ज्ञ्स्य॑ ध्रु॒वस्याऽद्ध्य॑क्षाभ्यां गृह्णाम्युपया॒मगृ॑हीतोऽसि श्रुत॒सद॑सि श्रोत्र॒पाभ्यां᳚ त्वा क्रतु॒पाभ्या॑म॒स्य य॒ज्ञ्स्य॑ ध्रु॒वस्याऽद्ध्य॑क्षाभ्यां गृह्णामि दे॒वेभ्य॑स्त्वा वि॒श्वदे॑वेभ्यस्त्वा॒ विश्वे᳚भ्यस्त्वा दे॒वेभ्यो॒ विष्ण॑वुरुक्रमै॒ष ते॒ सोम॒स्तꣳ र॑क्षस्व॒ - [  ] \newline

\textbf{Pada Paata} \newline

उ॒प॒या॒मगृ॑हीत॒ इत्यु॑पया॒म - गृ॒ही॒तः॒ । अ॒सि॒ । वा॒क्ष॒सदिति॑ वाक्ष - सत् । अ॒सि॒ । वा॒क्पाभ्या॒मिति॑ वाक् - पाभ्या᳚म् । त्वा॒ । क्र॒तु॒पाभ्या॒मिति॑ क्रतु - पाभ्या᳚म् । अ॒स्य । य॒ज्ञ्स्य॑ । ध्रु॒वस्य॑ । अद्ध्य॑क्षाभ्या॒मित्यधि॑ - अ॒क्षा॒भ्या॒म् । गृ॒ह्णा॒मि॒ । उ॒प॒या॒मगृ॑हीत॒ इत्यु॑पया॒म - गृ॒ही॒तः॒ । अ॒सि॒ । ऋ॒त॒सदित्यृ॑त - सत् । अ॒सि॒ । च॒क्षु॒ष्पाभ्या॒मिति॑ चक्षुः - पाभ्या᳚म् । त्वा॒ । क्र॒तु॒पाभ्या॒मिति॑ क्रतु - पाभ्या᳚म् । अ॒स्य । य॒ज्ञ्स्य॑ । ध्रु॒वस्य॑ । अद्ध्य॑क्षाभ्या॒मित्यधि॑ - अ॒क्षा॒भ्या॒म् । गृ॒ह्णा॒मि॒ । उ॒प॒या॒मगृ॑हीत॒ इत्यु॑पया॒म - गृ॒ही॒तः॒ । अ॒सि॒ । श्रु॒त॒सदिति॑ श्रुत - सत् । अ॒सि॒ । श्रो॒त्र॒पाभ्या॒मिति॑ श्रोत्र - पाभ्या᳚म् । त्वा॒ । क्र॒तु॒पाभ्या॒मिति॑ क्रतु - पाभ्या᳚म् । अ॒स्य । य॒ज्ञ्स्य॑ । ध्रु॒वस्य॑ । अद्ध्य॑क्षाभ्या॒मित्यधि॑ - अ॒क्षा॒भ्या॒म् । गृ॒ह्णा॒मि॒ । दे॒वेभ्यः॑ । त्वा॒ । वि॒श्वदे॑वेभ्य॒ इति॑ वि॒श्व-दे॒वे॒भ्यः॒ । त्वा॒ । विश्वे᳚भ्यः । त्वा॒ । दे॒वेभ्यः॑ । विष्णो᳚ । उ॒रु॒क्र॒मेत्यु॑रु - क्र॒म॒ । ए॒षः । ते॒ । सोमः॑ । तम् । र॒क्ष॒स्व॒ ।  \newline




\markright{ TS 3.2.10.2  \hfill https://www.vedavms.in \hfill}

\section{ TS 3.2.10.2 }

\textbf{TS 3.2.10.2 } \newline
\textbf{Samhita Paata} \newline

तं ते॑ दु॒श्चक्षा॒ माऽव॑ ख्य॒न्मयि॒ वसुः॑ पुरो॒वसु॑र्वा॒क्पा वाचं॑ मे पाहि॒ मयि॒ वसु॑र्वि॒दद्व॑सुश्चक्षु॒ष्पाश्चक्षु॑र् मे पाहि॒ मयि॒ वसुः॑ सं॒ॅयद्व॑सुः श्रोत्र॒पाः श्रोत्रं॑ मे पाहि॒ भूर॑सि॒ श्रेष्ठो॑ रश्मी॒नां प्रा॑ण॒पाः प्रा॒णं मे॑ पाहि॒ धूर॑सि॒ श्रेष्ठो॑ रश्मी॒नाम॑पान॒पा अ॑पा॒नं मे॑ पाहि॒ यो न॑ इन्द्रवायू मित्रावरुणा-वश्विनावभि॒दास॑ति॒ भ्रातृ॑व्य ( ) उ॒त्पिपी॑ते शुभस्पती इ॒दम॒हं तमध॑रं पादयामि॒ यथे᳚न्द्रा॒ऽहमु॑त्त॒मश्चे॒तया॑नि ॥ \newline

\textbf{Pada Paata} \newline

तम् । ते॒ । दु॒श्चक्षा॒ इति॑ दुः - चक्षाः᳚ । मा । अवेति॑ । ख्य॒त् । मयि॑ । वसुः॑ । पु॒रो॒वसु॒रिति॑ पुरः - वसुः॑ । वा॒क्पा इति॑ वाक् - पाः । वाच᳚म् । मे॒ । पा॒हि॒ । मयि॑ । वसुः॑ । वि॒दद्व॑सु॒रिति॑ वि॒दत् - व॒सुः॒ । च॒क्षु॒ष्पा इति॑ चक्षुः - पाः । चक्षुः॑ । मे॒ । पा॒हि॒ । मयि॑ । वसुः॑ । सं॒ॅयद् व॑सु॒रिति॑ सं॒ॅयत् - व॒सुः॒ । श्रो॒त्र॒पा इति॑ श्रोत्र-पाः । श्रोत्र᳚म् । मे॒ । पा॒हि॒ । भूः । अ॒सि॒ । श्रेष्ठः॑ । र॒श्मी॒नाम् । प्रा॒ण॒पा इति॑ प्राण - पाः । प्रा॒णमिति॑ प्र- अ॒नम् । मे॒ । पा॒हि॒ । धूः । अ॒सि॒ । श्रेष्ठः॑ । र॒श्मी॒नाम् । अ॒पा॒न॒पा इत्य॑पान - पाः । अ॒पा॒नमित्य॑प - अ॒नम् । मे॒ । पा॒हि॒ । यः । नः॒ । इ॒न्द्र॒वा॒यू॒ इती᳚न्द्र - वा॒यू॒ । मि॒त्रा॒व॒रु॒णा॒विति॑ मित्रा - व॒रु॒णौ॒ । अ॒श्वि॒नौ॒ । अ॒भि॒दास॒तीत्य॑भि - दास॑ति । भ्रातृ॑व्यः ( ) । उ॒त्पिपी॑त॒ इत्यु॑त् - पिपी॑ते । शु॒भः॒ । प॒ती॒ इति॑ । इ॒दम् । अ॒हम् । तम् । अध॑रम् । पा॒द॒या॒मि॒ । यथा᳚ । इ॒न्द्र॒ । अ॒हम् । उ॒त्त॒म इत्यु॑त् - त॒मः । चे॒तया॑नि ॥  \newline




\markright{ TS 3.2.11.1  \hfill https://www.vedavms.in \hfill}

\section{ TS 3.2.11.1 }

\textbf{TS 3.2.11.1 } \newline
\textbf{Samhita Paata} \newline

प्र सो अ॑ग्ने॒ तवो॒तिभिः॑ सु॒वीरा॑भिस्तरति॒ वाज॑कर्मभिः । यस्य॒ त्वꣳ स॒ख्यमावि॑थ ॥ प्र होत्रे॑ पू॒र्व्यं ॅवचो॒ऽग्नये॑ भरता बृ॒हत् । वि॒पां ज्योतीꣳ॑षि॒ बिभ्र॑ते॒ न वे॒धसे᳚ ॥ अग्ने॒ त्री ते॒ वाजि॑ना॒ त्री ष॒धस्था॑ ति॒स्रस्ते॑ जि॒ह्वा ऋ॑तजात पू॒र्वीः । ति॒स्र उ॑ ते त॒नुवो॑ दे॒ववा॑ता॒स्ताभि॑र्नः पाहि॒ गिरो॒ अप्र॑युच्छन्न् ॥ सं ॅवा॒म् कर्म॑णा॒ समि॒षा - [  ] \newline

\textbf{Pada Paata} \newline

प्रेति॑ । सः । अ॒ग्ने॒ । तव॑ । ऊ॒तिभि॒रित्यू॒ति - भिः॒ । सु॒वीरा॑भि॒रिति॑ सु - वीरा॑भिः । त॒र॒ति॒ । वाज॑कर्मभि॒रिति॒ वाज॑कर्म - भिः॒ ॥ यस्य॑ । त्वम् । स॒ख्यम् । आवि॑थ ॥ प्रेति॑ । होत्रे᳚ । पू॒र्व्यम् । वचः॑ । अ॒ग्नये᳚ । भ॒र॒त॒ । बृ॒हत् ॥ वि॒पामिति॑ वि - पाम् । ज्योतीꣳ॑षि । बिभ्र॑ते । न । वे॒धसे᳚ ॥ अग्ने᳚ । त्री । ते॒ । वाजि॑ना । त्री । स॒धस्थेति॑ स॒ध - स्था॒ । ति॒स्रः । ते॒ । जि॒ह्वाः । ऋ॒त॒जा॒तेत्यृ॑त - जा॒त॒ । पू॒र्वीः ॥ ति॒स्रः । उ॒ । ते॒ । त॒नुवः॑ । दे॒ववा॑ता॒ इति॑ दे॒व - वा॒ताः॒ । ताभिः॑ । नः॒ । पा॒हि॒ । गिरः॑ । अप्र॑युच्छ॒न्नित्यप्र॑ - यु॒च्छ॒न्न् ॥ समिति॑ । वा॒म् । कर्म॑णा । समिति॑ । इ॒षा ।  \newline




\markright{ TS 3.2.11.2  \hfill https://www.vedavms.in \hfill}

\section{ TS 3.2.11.2 }

\textbf{TS 3.2.11.2 } \newline
\textbf{Samhita Paata} \newline

हि॑नो॒मीन्द्रा॑-विष्णू॒ अप॑सस्पा॒रे अ॒स्य । जु॒षेथां᳚ ॅय॒ज्ञ्ं द्रवि॑णं च धत्त॒मरि॑ष्टैर्नः प॒थिभिः॑ पा॒रय॑न्ता ॥ उ॒भा जि॑ग्यथु॒र्न परा॑ जयेथे॒ न परा॑ जिग्ये कत॒रश्च॒नैनोः᳚ । इन्द्र॑श्च विष्णो॒ यदप॑स्पृधेथां त्रे॒धा स॒हस्रं॒ ॅवि तदै॑रयेथां ॥ त्रीण्यायूꣳ॑षि॒ तव॑ जातवेदस्ति॒स्र आ॒जानी॑रु॒षस॑स्ते अग्ने । ताभि॑र्दे॒वाना॒मवो॑ यक्षि वि॒द्वानथा॑ - [  ] \newline

\textbf{Pada Paata} \newline

हि॒नो॒मि॒ । इन्द्रा॑विष्णू॒ इतीन्द्रा᳚ - वि॒ष्णू॒ । अप॑सः । पा॒रे । अ॒स्य ॥ जु॒षेथा᳚म् । य॒ज्ञ्म् । द्रवि॑णम् । च॒ । ध॒त्त॒म् । अरि॑ष्टैः । नः॒ । प॒थिभि॒रिति॑ प॒थि-भिः॒ । पा॒रय॑न्ता ॥ उ॒भा । जि॒ग्य॒थुः॒ । न । परेति॑ । ज॒ये॒थे॒ इति॑ । न । परेति॑ । जि॒ग्ये॒ । क॒त॒रः । च॒न । ए॒नोः॒ ॥ इन्द्रः॑ । च॒ । वि॒ष्णो॒ इति॑ । यत् । अप॑स्पृधेथाम् । त्रे॒धा । स॒हस्र᳚म् । वीति॑ । तत् । ऐ॒र॒ये॒था॒म् ॥ त्रीणि॑ । आयूꣳ॑षि । तव॑ । जा॒त॒वे॒द॒ इति॑ जात - वे॒दः॒ । ति॒स्रः । आ॒जानी॒रित्या᳚ - जानीः᳚ । उ॒षसः॑ । ते॒ । अ॒ग्ने॒ । ताभिः॑ ॥ दे॒वाना᳚म् । अवः॑ । य॒क्षि॒ । वि॒द्वान् । अथ॑ ।  \newline




\markright{ TS 3.2.11.3  \hfill https://www.vedavms.in \hfill}

\section{ TS 3.2.11.3 }

\textbf{TS 3.2.11.3 } \newline
\textbf{Samhita Paata} \newline

-भव॒ यज॑मानाय॒ शंॅयोः ॥ अ॒ग्निस्त्रीणि॑ त्रि॒धातू॒न्या क्षे॑ति वि॒दथा॑ क॒विः । स त्रीꣳरे॑काद॒शाꣳ इ॒ह ॥ यक्ष॑च्च पि॒प्रय॑च्च नो॒ विप्रो॑ दू॒तः परि॑ष्कृतः । नभ॑न्तामन्य॒के स॑मे ॥ इन्द्रा॑विष्णू दृꣳहि॒ताः शम्ब॑रस्य॒ नव॒ पुरो॑ नव॒तिं च॑- श्ञथिष्टं । श॒तं ॅव॒र्चिनः॑ स॒हस्रं॑ च सा॒कꣳ ह॒थो अ॑प्र॒त्यसु॑रस्य वी॒रान् ॥ उ॒त मा॒ता म॑हि॒ष मन्व॑वेनद॒मी त्वा॑ ( ) जहति पुत्र दे॒वाः । अथा᳚ब्रवीद्-वृ॒त्रमिन्द्रो॑ हनि॒ष्यन्थ्-सखे॑ विष्णो वित॒रं ॅविक्र॑मस्व ॥ \newline

\textbf{Pada Paata} \newline

भ॒व॒ । यज॑मानाय । शम् । योः ॥ अ॒ग्निः । त्रीणि॑ । त्रि॒धातू॒नीति॑ त्रि - धातू॑नि । एति॑ । क्षे॒ति॒ । वि॒दथा᳚ । क॒विः ॥ सः । त्रीन् । ए॒का॒द॒शान् । इ॒ह ॥ यक्ष॑त् । च॒ । पि॒प्रय॑त् । च॒ । नः॒ । विप्रः॑ । दू॒तः । परि॑ष्कृतः ॥ नभ॑न्ताम् । अ॒न्य॒के । स॒मे॒ ॥ इन्द्रा॑विष्णू॒ इतीन्द्रा᳚-वि॒ष्णू॒ । दृꣳ॒॒हि॒ताः । शंब॑रस्य । नव॑ । पुरः॑ । न॒व॒तिम् । च॒ । श्न॒थि॒ष्ट॒म् ॥ श॒तम् । व॒र्चिनः॑ । स॒हस्र᳚म् । च॒ । सा॒कम् । ह॒थः । अ॒प्र॒ति । असु॑रस्य । वी॒रान् ॥ उ॒त । मा॒ता । म॒हि॒षम् । अन्विति॑ । अ॒वे॒न॒त् । अ॒मी इति॑ । त्वा॒ ( ) । ज॒ह॒ति॒ । पु॒त्र॒ । दे॒वाः ॥ अथ॑ । अ॒ब्र॒वी॒त् । वृ॒त्रम् । इन्द्रः॑ । ह॒नि॒ष्यन्न् । सखे᳚ । वि॒ष्णो॒ इति॑ । वि॒त॒रमिति॑ वि - त॒रम् । वीति॑ । क्र॒म॒स्व॒ ॥  \newline






\end{document}