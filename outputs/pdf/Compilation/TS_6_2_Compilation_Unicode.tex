\documentclass[17pt]{extarticle}
\usepackage{babel}
\usepackage{fontspec}
\usepackage{polyglossia}
\usepackage{extsizes}

\usepackage{color}   %May be necessary if you want to color links
\usepackage{hyperref}
\hypersetup{
    colorlinks=true, %set true if you want colored links
    linktoc=all,     %set to all if you want both sections and subsections linked
    linkcolor=black,  %choose some color if you want links to stand out
}

\setmainlanguage{sanskrit}
\setotherlanguages{english} %% or other languages
\setlength{\parindent}{0pt}
\pagestyle{myheadings}
\newfontfamily\devanagarifont[Script=Devanagari]{AdishilaVedic}
\renewcommand{\theHsection}{\thepart.section.\thesection}

\newcommand{\VAR}[1]{}
\newcommand{\BLOCK}[1]{}




\begin{document}
\begin{titlepage}
    \begin{center}
 
\begin{sanskrit}
    { \Large
    कृष्ण यजुर्वेदीय तैत्तिरीय संहिता,पद,जटा,घन पाठः 
    }
    \\
    \vspace{2.5cm}
    \mbox{ \Large
    6.2       षष्ठकाण्डे द्वितीयः प्रश्नः - सोममन्त्रब्राह्मणनिरूपणं   }
\end{sanskrit}
\end{center}

\end{titlepage}
\tableofcontents
\phantomsection
\pagebreak

\markright{ TS 6.2.1.1  \hfill https://www.vedavms.in \hfill}

\section{ TS 6.2.1.1 }

\textbf{TS 6.2.1.1 } \newline
\textbf{Samhita Paata} \newline

यदु॒भौ वि॒मुच्या॑ऽऽ*ति॒थ्यं गृ॑ह्णी॒याद्-य॒ज्ञ्ं ॅविच्छि॑न्द्या॒द्-यदु॒भाव-वि॑मुच्य॒ यथाऽना॑गतायाऽऽति॒थ्यं क्रि॒यते॑ ता॒दृगे॒व तद्-विमु॑क्तो॒-ऽन्यो॑ऽन॒ड्वान् भव॒त्य वि॑मुक्तो॒ऽन्यो-ऽथा॑ऽऽ*ति॒थ्यं गृ॑ह्णाति य॒ज्ञ्स्य॒ सन्त॑त्यै॒ पत्न्य॒न्वार॑भते॒ पत्नी॒ हि पारी॑णह्य॒स्येशे॒ पत्नि॑यै॒ वानु॑मतं॒ निर्व॑पति॒ यद्वै पत्नी॑ य॒ज्ञ्स्य॑ क॒रोति॑ मिथु॒नं तदथो॒ पत्नि॑या ए॒वै- [  ] \newline

\textbf{Pada Paata} \newline

यत् । उ॒भौ । वि॒मुच्येति॑ वि-मुच्य॑ । आ॒ति॒थ्यम् । गृ॒ह्णी॒यात् । य॒ज्ञ्म् । वीति॑ । छि॒न्द्या॒त् । यत् । उ॒भौ । अवि॑मु॒च्येत्यवि॑ - मु॒च्य॒ । यथा᳚ । अना॑गता॒येत्यना᳚-ग॒ता॒य॒ । आ॒ति॒थ्यम् । क्रि॒यते᳚ । ता॒दृक् । ए॒व । तत् । विमु॑क्त॒ इति॒ वि - मु॒क्तः॒ । अ॒न्यः । अ॒न॒ड्वान् । भव॑ति । अवि॑मुक्त॒ इत्यवि॑ - मु॒क्तः॒ । अ॒न्यः । अथ॑ । आ॒ति॒थ्यम् । गृ॒ह्णा॒ति॒ । य॒ज्ञ्स्य॑ । सन्त॑त्या॒ इति॒ सं - त॒त्यै॒ । पत्नी᳚ । अ॒न्वार॑भत॒ इत्य॑नु - आर॑भते । पत्नी᳚ । हि । पारी॑णह्य॒स्येति॒ पारि॑ - न॒ह्य॒स्य॒ । ईशे᳚ । पत्नि॑या । ए॒व । अनु॑मत॒मित्यनु॑ - म॒त॒म् । निरिति॑ । व॒प॒ति॒ । यत् । वै । पत्नी᳚ । य॒ज्ञ्स्य॑ । क॒रोति॑ । मि॒थु॒नम् । तत् । अथो॒ इति॑ । पत्नि॑याः । ए॒व ।  \newline


\textbf{Krama Paata} \newline

यदु॒भौ । उ॒भौ वि॒मुच्य॑ । वि॒मुच्या॑ति॒थ्यम् । वि॒मुच्येति॑ वि - मुच्य॑ । आ॒ति॒थ्यम् गृ॑ह्णी॒यात् । गृ॒ह्णी॒याद् य॒ज्ञ्म् । य॒ज्ञ्म् ॅवि । विच्छि॑न्द्यात् । छि॒न्द्या॒द् यत् । यदु॒भौ । उ॒भाववि॑मुच्य । अवि॑मुच्य॒ यथा᳚ । अवि॑मु॒च्येत्यवि॑ - मु॒च्य॒ । यथाऽना॑गताय । अना॑गतायाति॒थ्यम् । अना॑गता॒येत्यना᳚ - ग॒ता॒य॒ । आ॒ति॒थ्यम् क्रि॒यते᳚ । क्रि॒यते॑ ता॒दृक् । ता॒दृगे॒व । ए॒व तत् । तद् विमु॑क्तः । विमु॑क्तो॒ऽन्यः॑ । विमु॑क्त॒ इति॒ वि - मु॒क्तः॒ । अ॒न्यो॑ऽन॒ड्वान् । अ॒न॒ड्वान् भव॑ति । भव॒त्यवि॑मुक्तः । अवि॑मुक्तो॒ऽन्यः । अवि॑मुक्त॒ इत्यवि॑ - मु॒क्तः॒ । अ॒न्योऽथ॑ । अथा॑ति॒थ्यम् । आ॒ति॒थ्यम् गृ॑ह्णाति । गृ॒ह्णा॒ति॒ य॒ज्ञ्स्य॑ । य॒ज्ञ्स्य॒ सन्त॑त्यै । सन्त॑त्यै॒ पत्नी᳚ । सन्त॑त्या॒ इति॒ सम् - त॒त्यै॒ । पत्न्य॒न्वार॑भते । अ॒न्वार॑भते॒ पत्नी᳚ । अ॒न्वार॑भत॒ इत्य॑नु - आर॑भते । पत्नी॒ हि । हि पारी॑णह्यस्य । पारी॑णह्य॒स्येशे᳚ । पारी॑णह्य॒स्येति॒ पारि॑ - न॒ह्य॒स्य॒ । ईशे॒ पत्नि॑या । पत्नि॑यै॒व । ए॒वानु॑मतम् । अनु॑मत॒म् निः । अनु॑मत॒मित्यनु॑ - म॒त॒म् । निर्व॑पति । व॒प॒ति॒ यत् । यद् वै । वै पत्नी᳚ । पत्नी॑ य॒ज्ञ्स्य॑ । य॒ज्ञ्स्य॑ क॒रोति॑ । क॒रोति॑ मिथु॒नम् । मि॒थु॒नम् तत् । तदथो᳚ । अथो॒ पत्नि॑याः । अथो॒ इत्यथो᳚ । पत्नि॑या ए॒व । ए॒वैषः \newline

\textbf{Jatai Paata} \newline

1. यदु॒भा वु॒भौ यद् यदु॒भौ । \newline
2. उ॒भौ वि॒मुच्य॑ वि॒मुच्यो॒भा वु॒भौ वि॒मुच्य॑ । \newline
3. वि॒मुच्या॑ ति॒थ्य मा॑ति॒थ्यम् ॅवि॒मुच्य॑ वि॒मुच्या॑ ति॒थ्यम् । \newline
4. वि॒मुच्येति॑ वि - मुच्य॑ । \newline
5. आ॒ति॒थ्यम् गृ॑ह्णी॒याद् गृ॑ह्णी॒या दा॑ति॒थ्य मा॑ति॒थ्यम् गृ॑ह्णी॒यात् । \newline
6. गृ॒ह्णी॒याद् य॒ज्ञ्ं ॅय॒ज्ञ्म् गृ॑ह्णी॒याद् गृ॑ह्णी॒याद् य॒ज्ञ्म् । \newline
7. य॒ज्ञ्ं ॅवि वि य॒ज्ञ्ं ॅय॒ज्ञ्ं ॅवि । \newline
8. वि च्छि॑न्द्याच् छिन्द्यात्द् वि वि च्छि॑न्द्यात् । \newline
9. छि॒न्द्या॒द् यद् यच् छि॑न्द्याच् छिन्द्या॒द् यत् । \newline
10. यदु॒भा वु॒भौ यद् यदु॒भौ । \newline
11. उ॒भा ववि॑मु॒च्या वि॑मुच्यो॒भा वु॒भा ववि॑मुच्य । \newline
12. अवि॑मुच्य॒ यथा॒ यथा ऽवि॑मु॒च्या वि॑मुच्य॒ यथा᳚ । \newline
13. अवि॑मु॒च्येत्यवि॑ - मु॒च्य॒ । \newline
14. यथा ऽना॑गता॒या ना॑गताय॒ यथा॒ यथा ऽना॑गताय । \newline
15. अना॑गताया ति॒थ्य मा॑ति॒थ्य मना॑गता॒या ना॑गताया ति॒थ्यम् । \newline
16. अना॑गता॒येत्यना᳚ - ग॒ता॒य॒ । \newline
17. आ॒ति॒थ्यम् क्रि॒यते᳚ क्रि॒यत॑ आति॒थ्य मा॑ति॒थ्यम् क्रि॒यते᳚ । \newline
18. क्रि॒यते॑ ता॒दृक् ता॒दृक् क्रि॒यते᳚ क्रि॒यते॑ ता॒दृक् । \newline
19. ता॒दृ गे॒वैव ता॒दृक् ता॒दृ गे॒व । \newline
20. ए॒व तत् तदे॒वैव तत् । \newline
21. तद् विमु॑क्तो॒ विमु॑क्त॒ स्तत् तद् विमु॑क्तः । \newline
22. विमु॑क्तो॒ ऽन्यो᳚ ऽन्यो विमु॑क्तो॒ विमु॑क्तो॒ ऽन्यः । \newline
23. विमु॑क्त॒ इति॒ वि - मु॒क्तः॒ । \newline
24. अ॒न्यो॑ ऽन॒ड्वा न॑न॒ड्वा न॒न्यो᳚(1॒) ऽन्यो॑ ऽन॒ड्वान् । \newline
25. अ॒न॒ड्वान् भव॑ति॒ भव॑ त्यन॒ड्वा न॑न॒ड्वान् भव॑ति । \newline
26. भव॒ त्यवि॑मु॒क्तो ऽवि॑मुक्तो॒ भव॑ति॒ भव॒ त्यवि॑मुक्तः । \newline
27. अवि॑मुक्तो॒ ऽन्यो᳚ ऽन्यो ऽवि॑मु॒क्तो ऽवि॑मुक्तो॒ ऽन्यः । \newline
28. अवि॑मुक्त॒ इत्यवि॑ - मु॒क्तः॒ । \newline
29. अ॒न्यो ऽथाथा॒ न्यो᳚ ऽन्यो ऽथ॑ । \newline
30. अथा॑ति॒थ्य मा॑ति॒थ्य मथाथा॑ ति॒थ्यम् । \newline
31. आ॒ति॒थ्यम् गृ॑ह्णाति गृह्णा त्याति॒थ्य मा॑ति॒थ्यम् गृ॑ह्णाति । \newline
32. गृ॒ह्णा॒ति॒ य॒ज्ञ्स्य॑ य॒ज्ञ्स्य॑ गृह्णाति गृह्णाति य॒ज्ञ्स्य॑ । \newline
33. य॒ज्ञ्स्य॒ सन्त॑त्यै॒ सन्त॑त्यै य॒ज्ञ्स्य॑ य॒ज्ञ्स्य॒ सन्त॑त्यै । \newline
34. सन्त॑त्यै॒ पत्नी॒ पत्नी॒ सन्त॑त्यै॒ सन्त॑त्यै॒ पत्नी᳚ । \newline
35. सन्त॑त्या॒ इति॒ सं - त॒त्यै॒ । \newline
36. पत्न्य॒न्वार॑भते॒ ऽन्वार॑भते॒ पत्नी॒ पत्न्य॒न्वार॑भते । \newline
37. अ॒न्वार॑भते॒ पत्नी॒ पत्न्य॒न्वार॑भते॒ ऽन्वार॑भते॒ पत्नी᳚ । \newline
38. अ॒न्वार॑भत॒ इत्य॑नु - आर॑भते । \newline
39. पत्नी॒ हि हि पत्नी॒ पत्नी॒ हि । \newline
40. हि पारी॑णह्यस्य॒ पारी॑णह्यस्य॒ हि हि पारी॑णह्यस्य । \newline
41. पारी॑णह्य॒स्येश॒ ईशे॒ पारी॑णह्यस्य॒ पारी॑णह्य॒स्येशे᳚ । \newline
42. पारी॑णह्य॒स्येति॒ पारि॑ - न॒ह्य॒स्य॒ । \newline
43. ईशे॒ पत्नि॑या॒ पत्नि॒येश॒ ईशे॒ पत्नि॑या । \newline
44. पत्नि॑ यै॒वैव पत्नि॑या॒ पत्नि॑ यै॒व । \newline
45. ए॒वा नु॑मत॒ मनु॑मत मे॒वैवा नु॑मतम् । \newline
46. अनु॑मत॒न् निर् णिरनु॑मत॒ मनु॑मत॒न् निः । \newline
47. अनु॑मत॒मित्यनु॑ - म॒त॒म् । \newline
48. निर् व॑पति वपति॒ निर् णिर् व॑पति । \newline
49. व॒प॒ति॒ यद् यद् व॑पति वपति॒ यत् । \newline
50. यद् वै वै यद् यद् वै । \newline
51. वै पत्नी॒ पत्नी॒ वै वै पत्नी᳚ । \newline
52. पत्नी॑ य॒ज्ञ्स्य॑ य॒ज्ञ्स्य॒ पत्नी॒ पत्नी॑ य॒ज्ञ्स्य॑ । \newline
53. य॒ज्ञ्स्य॑ क॒रोति॑ क॒रोति॑ य॒ज्ञ्स्य॑ य॒ज्ञ्स्य॑ क॒रोति॑ । \newline
54. क॒रोति॑ मिथु॒नम् मि॑थु॒नम् क॒रोति॑ क॒रोति॑ मिथु॒नम् । \newline
55. मि॒थु॒नम् तत् तन् मि॑थु॒नम् मि॑थु॒नम् तत् । \newline
56. तदथो॒ अथो॒ तत् तदथो᳚ । \newline
57. अथो॒ पत्नि॑याः॒ पत्नि॑या॒ अथो॒ अथो॒ पत्नि॑याः । \newline
58. अथो॒ इत्यथो᳚ । \newline
59. पत्नि॑या ए॒वैव पत्नि॑याः॒ पत्नि॑या ए॒व । \newline
60. ए॒वैष ए॒ष ए॒वैवैषः । \newline

\textbf{Ghana Paata } \newline

1. यदु॒भा वु॒भौ यद् यदु॒भौ वि॒मुच्य॑ वि॒मु च्यो॒भौ यद् यदु॒भौ वि॒मुच्य॑ । \newline
2. उ॒भौ वि॒मुच्य॑ वि॒मु च्यो॒भा वु॒भौ वि॒मु च्या॑ति॒थ्य मा॑ति॒थ्यं ॅवि॒मु च्यो॒भा वु॒भौ वि॒मु च्या॑ति॒थ्यम् । \newline
3. वि॒मु च्या॑ति॒थ्य मा॑ति॒थ्यं ॅवि॒मुच्य॑ वि॒मु च्या॑ति॒थ्यम् गृ॑ह्णी॒याद् गृ॑ह्णी॒या दा॑ति॒थ्यं ॅवि॒मुच्य॑ वि॒मु च्या॑ति॒थ्यम् गृ॑ह्णी॒यात् । \newline
4. वि॒मुच्येति॑ वि - मुच्य॑ । \newline
5. आ॒ति॒थ्यम् गृ॑ह्णी॒याद् गृ॑ह्णी॒या दा॑ति॒थ्य मा॑ति॒थ्यम् गृ॑ह्णी॒याद् य॒ज्ञ्ं ॅय॒ज्ञ्म् गृ॑ह्णी॒या दा॑ति॒थ्य मा॑ति॒थ्यम् गृ॑ह्णी॒याद् य॒ज्ञ्म् । \newline
6. गृ॒ह्णी॒याद् य॒ज्ञ्ं ॅय॒ज्ञ्म् गृ॑ह्णी॒याद् गृ॑ह्णी॒याद् य॒ज्ञ्ं ॅवि वि य॒ज्ञ्म् गृ॑ह्णी॒याद् गृ॑ह्णी॒याद् य॒ज्ञ्ं ॅवि । \newline
7. य॒ज्ञ्ं ॅवि वि य॒ज्ञ्ं ॅय॒ज्ञ्ं ॅवि च्छि॑न्द्याच् छिन्द्या॒द् वि य॒ज्ञ्ं ॅय॒ज्ञ्ं ॅवि च्छि॑न्द्यात् । \newline
8. वि च्छि॑न्द्याच् छिन्द्या॒द् वि वि च्छि॑न्द्या॒द् यद् यच् छि॑न्द्या॒द् वि वि च्छि॑न्द्या॒द् यत् । \newline
9. छि॒न्द्या॒द् यद् यच् छि॑न्द्याच् छिन्द्या॒द् यदु॒भा वु॒भौ यच् छि॑न्द्याच् छिन्द्या॒द् यदु॒भौ । \newline
10. यदु॒भा वु॒भौ यद् यदु॒भा ववि॑मु॒च्या वि॑मुच्यो॒भौ यद् यदु॒भा ववि॑मुच्य । \newline
11. उ॒भा ववि॑मु॒च्या वि॑मुच्यो॒भा वु॒भा ववि॑मुच्य॒ यथा॒ यथा ऽवि॑मुच्यो॒भा वु॒भा ववि॑मुच्य॒ यथा᳚ । \newline
12. अवि॑मुच्य॒ यथा॒ यथा ऽवि॑मु॒च्या वि॑मुच्य॒ यथा ऽना॑गता॒या ना॑गताय॒ यथा ऽवि॑मु॒च्या वि॑मुच्य॒ यथा ऽना॑गताय । \newline
13. अवि॑मु॒च्येत्यवि॑ - मु॒च्य॒ । \newline
14. यथा ऽना॑गता॒या ना॑गताय॒ यथा॒ यथा ऽना॑गता याति॒थ्य मा॑ति॒थ्य मना॑गताय॒ यथा॒ यथा ऽना॑गता याति॒थ्यम् । \newline
15. अना॑गता याति॒थ्य मा॑ति॒थ्य मना॑गता॒ याना॑गता याति॒थ्यम् क्रि॒यते᳚ क्रि॒यत॑ आति॒थ्य मना॑गता॒ याना॑गता याति॒थ्यम् क्रि॒यते᳚ । \newline
16. अना॑गता॒येत्यना᳚ - ग॒ता॒य॒ । \newline
17. आ॒ति॒थ्यम् क्रि॒यते᳚ क्रि॒यत॑ आति॒थ्य मा॑ति॒थ्यम् क्रि॒यते॑ ता॒दृक् ता॒दृक् क्रि॒यत॑ आति॒थ्य मा॑ति॒थ्यम् क्रि॒यते॑ ता॒दृक् । \newline
18. क्रि॒यते॑ ता॒दृक् ता॒दृक् क्रि॒यते᳚ क्रि॒यते॑ ता॒दृ गे॒वैव ता॒दृक् क्रि॒यते᳚ क्रि॒यते॑ ता॒दृ गे॒व । \newline
19. ता॒दृ गे॒वैव ता॒दृक् ता॒दृ गे॒व तत् तदे॒व ता॒दृक् ता॒दृ गे॒व तत् । \newline
20. ए॒व तत् तदे॒ वैव तद् विमु॑क्तो॒ विमु॑क्त॒ स्तदे॒ वैव तद् विमु॑क्तः । \newline
21. तद् विमु॑क्तो॒ विमु॑क्त॒ स्तत् तद् विमु॑क्तो॒ ऽन्यो᳚ ऽन्यो विमु॑क्त॒ स्तत् तद् विमु॑क्तो॒ ऽन्यः । \newline
22. विमु॑क्तो॒ ऽन्यो᳚ ऽन्यो विमु॑क्तो॒ विमु॑क्तो॒ ऽन्यो॑ ऽन॒ड्वा न॑न॒ड्वा न॒न्यो विमु॑क्तो॒ विमु॑क्तो॒ ऽन्यो॑ ऽन॒ड्वान् । \newline
23. विमु॑क्त॒ इति॒ वि - मु॒क्तः॒ । \newline
24. अ॒न्यो॑ ऽन॒ड्वा न॑न॒ड्वा न॒न्यो᳚(1॒) ऽन्यो॑ ऽन॒ड्वान् भव॑ति॒ भव॑ त्यन॒ड्वा न॒न्यो᳚(1॒) ऽन्यो॑ ऽन॒ड्वान् भव॑ति । \newline
25. अ॒न॒ड्वान् भव॑ति॒ भव॑ त्यन॒ड्वा न॑न॒ड्वान् भव॒ त्यवि॑मु॒क्तो ऽवि॑मुक्तो॒ भव॑ त्यन॒ड्वा न॑न॒ड्वान् भव॒ त्यवि॑मुक्तः । \newline
26. भव॒ त्यवि॑मु॒क्तो ऽवि॑मुक्तो॒ भव॑ति॒ भव॒ त्यवि॑मुक्तो॒ ऽन्यो᳚ ऽन्यो ऽवि॑मुक्तो॒ भव॑ति॒ भव॒ त्यवि॑मुक्तो॒ ऽन्यः । \newline
27. अवि॑मुक्तो॒ ऽन्यो᳚ ऽन्यो ऽवि॑मु॒क्तो ऽवि॑मुक्तो॒ ऽन्यो ऽथा था॒न्यो ऽवि॑मु॒क्तो ऽवि॑मुक्तो॒ ऽन्यो ऽथ॑ । \newline
28. अवि॑मुक्त॒ इत्यवि॑ - मु॒क्तः॒ । \newline
29. अ॒न्यो ऽथा था॒न्यो᳚ ऽन्यो ऽथा॑ ति॒थ्य मा॑ति॒थ्य मथा॒न्यो᳚ ऽन्यो ऽथा॑ ति॒थ्यम् । \newline
30. अथा॑ ति॒थ्य मा॑ति॒थ्य मथाथा॑ ति॒थ्यम् गृ॑ह्णाति गृह्णा त्याति॒थ्य मथाथा॑ ति॒थ्यम् गृ॑ह्णाति । \newline
31. आ॒ति॒थ्यम् गृ॑ह्णाति गृह्णा त्याति॒थ्य मा॑ति॒थ्यम् गृ॑ह्णाति य॒ज्ञ्स्य॑ य॒ज्ञ्स्य॑ गृह्णा त्याति॒थ्य मा॑ति॒थ्यम् गृ॑ह्णाति य॒ज्ञ्स्य॑ । \newline
32. गृ॒ह्णा॒ति॒ य॒ज्ञ्स्य॑ य॒ज्ञ्स्य॑ गृह्णाति गृह्णाति य॒ज्ञ्स्य॒ सन्त॑त्यै॒ सन्त॑त्यै य॒ज्ञ्स्य॑ गृह्णाति गृह्णाति य॒ज्ञ्स्य॒ सन्त॑त्यै । \newline
33. य॒ज्ञ्स्य॒ सन्त॑त्यै॒ सन्त॑त्यै य॒ज्ञ्स्य॑ य॒ज्ञ्स्य॒ सन्त॑त्यै॒ पत्नी॒ पत्नी॒ सन्त॑त्यै य॒ज्ञ्स्य॑ य॒ज्ञ्स्य॒ सन्त॑त्यै॒ पत्नी᳚ । \newline
34. सन्त॑त्यै॒ पत्नी॒ पत्नी॒ सन्त॑त्यै॒ सन्त॑त्यै॒ पत्न्य॒ न्वार॑भते॒ ऽन्वार॑भते॒ पत्नी॒ सन्त॑त्यै॒ सन्त॑त्यै॒ पत्न्य॒ न्वार॑भते । \newline
35. सन्त॑त्या॒ इति॒ सं - त॒त्यै॒ । \newline
36. पत्न्य॒ न्वार॑भते॒ ऽन्वार॑भते॒ पत्नी॒ पत्न्य॒ न्वार॑भते॒ पत्नी॒ पत्न्य॒ न्वार॑भते॒ पत्नी॒ पत्न्य॒ न्वार॑भते॒ पत्नी᳚ । \newline
37. अ॒न्वार॑भते॒ पत्नी॒ पत्न्य॒ न्वार॑भते॒ ऽन्वार॑भते॒ पत्नी॒ हि हि पत्न्य॒ न्वार॑भते॒ ऽन्वार॑भते॒ पत्नी॒ हि । \newline
38. अ॒न्वार॑भत॒ इत्य॑नु - आर॑भते । \newline
39. पत्नी॒ हि हि पत्नी॒ पत्नी॒ हि पारी॑णह्यस्य॒ पारी॑णह्यस्य॒ हि पत्नी॒ पत्नी॒ हि पारी॑णह्यस्य । \newline
40. हि पारी॑णह्यस्य॒ पारी॑णह्यस्य॒ हि हि पारी॑णह्य॒ स्येश॒ ईशे॒ पारी॑णह्यस्य॒ हि हि पारी॑णह्य॒स्येशे᳚ । \newline
41. पारी॑णह्य॒ स्येश॒ ईशे॒ पारी॑णह्यस्य॒ पारी॑णह्य॒ स्येशे॒ पत्नि॑या॒ पत्नि॒ येशे॒ पारी॑णह्यस्य॒ पारी॑णह्य॒ स्येशे॒ पत्नि॑या । \newline
42. पारी॑णह्य॒स्येति॒ पारि॑ - न॒ह्य॒स्य॒ । \newline
43. ईशे॒ पत्नि॑या॒ पत्नि॒ येश॒ ईशे॒ पत्नि॑ यै॒वैव पत्नि॒ येश॒ ईशे॒ पत्नि॑यै॒व । \newline
44. पत्नि॑ यै॒वैव पत्नि॑या॒ पत्नि॑यै॒वा नु॑मत॒ मनु॑मत मे॒व पत्नि॑या॒ पत्नि॑यै॒वा नु॑मतम् । \newline
45. ए॒वा नु॑मत॒ मनु॑मत मे॒वैवा नु॑मत॒न् निर् णिरनु॑मत मे॒वैवा नु॑मत॒न् निः । \newline
46. अनु॑मत॒न् निर् णिरनु॑मत॒ मनु॑मत॒न् निर् व॑पति वपति॒ निरनु॑मत॒ मनु॑मत॒न् निर् व॑पति । \newline
47. अनु॑मत॒मित्यनु॑ - म॒त॒म् । \newline
48. निर् व॑पति वपति॒ निर् णिर् व॑पति॒ यद् यद् व॑पति॒ निर् णिर् व॑पति॒ यत् । \newline
49. व॒प॒ति॒ यद् यद् व॑पति वपति॒ यद् वै वै यद् व॑पति वपति॒ यद् वै । \newline
50. यद् वै वै यद् यद् वै पत्नी॒ पत्नी॒ वै यद् यद् वै पत्नी᳚ । \newline
51. वै पत्नी॒ पत्नी॒ वै वै पत्नी॑ य॒ज्ञ्स्य॑ य॒ज्ञ्स्य॒ पत्नी॒ वै वै पत्नी॑ य॒ज्ञ्स्य॑ । \newline
52. पत्नी॑ य॒ज्ञ्स्य॑ य॒ज्ञ्स्य॒ पत्नी॒ पत्नी॑ य॒ज्ञ्स्य॑ क॒रोति॑ क॒रोति॑ य॒ज्ञ्स्य॒ पत्नी॒ पत्नी॑ य॒ज्ञ्स्य॑ क॒रोति॑ । \newline
53. य॒ज्ञ्स्य॑ क॒रोति॑ क॒रोति॑ य॒ज्ञ्स्य॑ य॒ज्ञ्स्य॑ क॒रोति॑ मिथु॒नम् मि॑थु॒नम् क॒रोति॑ य॒ज्ञ्स्य॑ य॒ज्ञ्स्य॑ क॒रोति॑ मिथु॒नम् । \newline
54. क॒रोति॑ मिथु॒नम् मि॑थु॒नम् क॒रोति॑ क॒रोति॑ मिथु॒नम् तत् तन् मि॑थु॒नम् क॒रोति॑ क॒रोति॑ मिथु॒नम् तत् । \newline
55. मि॒थु॒नम् तत् तन् मि॑थु॒नम् मि॑थु॒नम् तदथो॒ अथो॒ तन् मि॑थु॒नम् मि॑थु॒नम् तदथो᳚ । \newline
56. तदथो॒ अथो॒ तत् तदथो॒ पत्नि॑याः॒ पत्नि॑या॒ अथो॒ तत् तदथो॒ पत्नि॑याः । \newline
57. अथो॒ पत्नि॑याः॒ पत्नि॑या॒ अथो॒ अथो॒ पत्नि॑या ए॒वैव पत्नि॑या॒ अथो॒ अथो॒ पत्नि॑या ए॒व । \newline
58. अथो॒ इत्यथो᳚ । \newline
59. पत्नि॑या ए॒वैव पत्नि॑याः॒ पत्नि॑या ए॒वैष ए॒ष ए॒व पत्नि॑याः॒ पत्नि॑या ए॒वैषः । \newline
60. ए॒वैष ए॒ष ए॒वै वैष य॒ज्ञ्स्य॑ य॒ज्ञ् स्यै॒ष ए॒वै वैष य॒ज्ञ्स्य॑ । \newline
\pagebreak
\markright{ TS 6.2.1.2  \hfill https://www.vedavms.in \hfill}

\section{ TS 6.2.1.2 }

\textbf{TS 6.2.1.2 } \newline
\textbf{Samhita Paata} \newline

-ष य॒ज्ञ्स्या᳚न्वार॒भ्ॐ ऽन॑वच्छित्त्यै॒ याव॑द्-भि॒र्वै राजा॑ऽनुच॒रैरा॒गच्छ॑ति॒ सर्वे᳚भ्यो॒ वै तेभ्य॑ आति॒थ्यं क्रि॑यते॒ छन्दाꣳ॑सि॒ खलु॒ वै सोम॑स्य॒ राज्ञो॑ऽनुच॒राण्य॒ग्ने-रा॑ति॒थ्यम॑सि॒ विष्ण॑वे॒ त्वेत्या॑ह गायत्रि॒या ए॒वैतेन॑ करोति॒ सोम॑स्याऽऽ*ति॒थ्यम॑सि॒ विष्ण॑वे॒ त्वेत्या॑ह त्रि॒ष्टुभ॑ ए॒वैतेन॑ करो॒त्यति॑थेराति॒थ्यम॑सि॒ विष्ण॑वे॒ त्वेत्या॑ह॒ जग॑त्या- [  ] \newline

\textbf{Pada Paata} \newline

ए॒षः । य॒ज्ञ्स्य॑ । अ॒न्वा॒र॒म्भ इत्य॑नु -  आ॒र॒म्भः । अन॑वच्छित्त्या॒ इत्यन॑व - छि॒त्त्यै॒ । याव॑द्भि॒रिति॒ याव॑त् - भिः॒ । वै । राजा᳚ । अ॒नु॒च॒रैरित्य॑नु - च॒रैः । आ॒गच्छ॒तीत्या᳚ - गच्छ॑ति । सर्वे᳚भ्यः । वै । तेभ्यः॑ । आ॒ति॒थ्यम् । क्रि॒य॒ते॒ । छन्दाꣳ॑सि । खलु॑ । वै । सोम॑स्य । राज्ञ्ः॑ । अ॒नु॒च॒राणीत्य॑नु - च॒राणि॑ । अ॒ग्नेः । आ॒ति॒थ्यम् । अ॒सि॒ । विष्ण॑वे । त्वा॒ । इति॑ । आ॒ह॒ । गा॒य॒त्रि॒यै । ए॒व । ए॒तेन॑ । क॒रो॒ति॒ । सोम॑स्य । आ॒ति॒थ्यम् । अ॒सि॒ । विष्ण॑वे । त्वा॒ । इति॑ । आ॒ह॒ । त्रि॒ष्टुभे᳚ । ए॒व । ए॒तेन॑ । क॒रो॒ति॒ । अति॑थेः । आ॒ति॒थ्यम् । अ॒सि॒ । विष्ण॑वे । त्वा॒ । इति॑ । आ॒ह॒ । जग॑त्यै ।  \newline


\textbf{Krama Paata} \newline

ए॒ष य॒ज्ञ्स्य॑ । य॒ज्ञ्स्या᳚न्वार॒म्भः । अ॒न्वा॒र॒म्भोऽन॑वच्छित्त्यै । अ॒न्वा॒र॒म्भ इत्य॑नु - आ॒र॒म्भः । अन॑वच्छित्त्यै॒ याव॑द्‌भिः । अन॑वच्छित्त्या॒ इत्यन॑व - छि॒त्त्यै॒ । याव॑द्‌भि॒र् वै । याव॑द्‌भि॒रिति॒ याव॑त् - भिः॒ । वै राजा᳚ । राजा॑ऽनुच॒रैः । अ॒नु॒च॒रैरा॒गच्छ॑ति । अ॒नु॒च॒रैरित्य॑नु - च॒रैः । आ॒गच्छ॑ति॒ सर्वे᳚भ्यः । आ॒गच्छ॒तीत्या᳚ - गच्छ॑ति । सर्वे᳚भ्यो॒ वै । वै तेभ्यः॑ । तेभ्य॑ आति॒थ्यम् । आ॒ति॒थ्यम् क्रि॑यते । क्रि॒य॒ते॒ छन्दाꣳ॑सि । छन्दाꣳ॑सि॒ खलु॑ । खलु॒ वै । वै सोम॑स्य । सोम॑स्य॒ राज्ञ्ः॑ । राज्ञो॑ऽनुच॒राणि॑ । अ॒नु॒च॒राण्य॒ग्नेः । अ॒नु॒च॒राणीत्य॑नु - च॒राणि॑ । अ॒ग्नेरा॑ति॒थ्यम् । आ॒ति॒थ्यम॑सि । अ॒सि॒ विष्ण॑वे । विष्ण॑वे त्वा । त्वेति॑ । इत्या॑ह । आ॒ह॒ गा॒य॒त्रि॒यै । गा॒य॒त्रि॒या ए॒व । ए॒वैतेन॑ । ए॒तेन॑ करोति । क॒रो॒ति॒ सोम॑स्य । सोम॑स्याति॒थ्यम् । आ॒ति॒थ्यम॑सि । अ॒सि॒ विष्ण॑वे । विष्ण॑वे त्वा । त्वेति॑ । इत्या॑ह । आ॒ह॒ त्रि॒ष्टुभे᳚ । त्रि॒ष्टुभ॑ ए॒व । ए॒वैतेन॑ । ए॒तेन॑ करोति । क॒रो॒त्यति॑थेः । अति॑थेराति॒थ्यम् । आ॒ति॒थ्यम॑सि । अ॒सि॒ विष्ण॑वे । विष्ण॑वे त्वा । त्वेति॑ । इत्या॑ह । आ॒ह॒ जग॑त्यै । जग॑त्या ए॒व \newline

\textbf{Jatai Paata} \newline

1. ए॒ष य॒ज्ञ्स्य॑ य॒ज्ञ्स्यै॒ष ए॒ष य॒ज्ञ्स्य॑ । \newline
2. य॒ज्ञ्स्या᳚ न्वार॒म्भो᳚ ऽन्वार॒म्भो य॒ज्ञ्स्य॑ य॒ज्ञ्स्या᳚ न्वार॒म्भः । \newline
3. अ॒न्वा॒र॒म्भो ऽन॑वच्छित्त्या॒ अन॑वच्छित्त्या अन्वार॒म्भो᳚ ऽन्वार॒म्भो ऽन॑वच्छित्त्यै । \newline
4. अ॒न्वा॒र॒म्भ इत्य॑नु - आ॒र॒म्भः । \newline
5. अन॑वच्छित्त्यै॒ याव॑द्भि॒र् याव॑द्भि॒ रन॑वच्छित्त्या॒ अन॑वच्छित्त्यै॒ याव॑द्भिः । \newline
6. अन॑वच्छित्त्या॒ इत्यन॑व - छि॒त्त्यै॒ । \newline
7. याव॑द्भि॒र् वै वै याव॑द्भि॒र् याव॑द्भि॒र् वै । \newline
8. याव॑द्भि॒रिति॒ याव॑त् - भिः॒ । \newline
9. वै राजा॒ राजा॒ वै वै राजा᳚ । \newline
10. राजा॑ ऽनुच॒रै र॑नुच॒रै राजा॒ राजा॑ ऽनुच॒रैः । \newline
11. अ॒नु॒च॒रै रा॒गच्छ॑ त्या॒गच्छ॑ त्यनुच॒रै र॑नुच॒रै रा॒गच्छ॑ति । \newline
12. अ॒नु॒च॒रैरित्य॑नु - च॒रैः । \newline
13. आ॒गच्छ॑ति॒ सर्वे᳚भ्यः॒ सर्वे᳚भ्य आ॒गच्छ॑ त्या॒गच्छ॑ति॒ सर्वे᳚भ्यः । \newline
14. आ॒गच्छ॒तीत्या᳚ - गच्छ॑ति । \newline
15. सर्वे᳚भ्यो॒ वै वै सर्वे᳚भ्यः॒ सर्वे᳚भ्यो॒ वै । \newline
16. वै तेभ्य॒ स्तेभ्यो॒ वै वै तेभ्यः॑ । \newline
17. तेभ्य॑ आति॒थ्य मा॑ति॒थ्यम् तेभ्य॒ स्तेभ्य॑ आति॒थ्यम् । \newline
18. आ॒ति॒थ्यम् क्रि॑यते क्रियत आति॒थ्य मा॑ति॒थ्यम् क्रि॑यते । \newline
19. क्रि॒य॒ते॒ छन्दाꣳ॑सि॒ छन्दाꣳ॑सि क्रियते क्रियते॒ छन्दाꣳ॑सि । \newline
20. छन्दाꣳ॑सि॒ खलु॒ खलु॒ छन्दाꣳ॑सि॒ छन्दाꣳ॑सि॒ खलु॑ । \newline
21. खलु॒ वै वै खलु॒ खलु॒ वै । \newline
22. वै सोम॑स्य॒ सोम॑स्य॒ वै वै सोम॑स्य । \newline
23. सोम॑स्य॒ राज्ञो॒ राज्ञ्ः॒ सोम॑स्य॒ सोम॑स्य॒ राज्ञ्ः॑ । \newline
24. राज्ञो॑ ऽनुच॒रा ण्य॑नुच॒राणि॒ राज्ञो॒ राज्ञो॑ ऽनुच॒राणि॑ । \newline
25. अ॒नु॒च॒रा ण्य॒ग्ने र॒ग्ने र॑नुच॒रा ण्य॑नुच॒रा ण्य॒ग्नेः । \newline
26. अ॒नु॒च॒राणीत्य॑नु - च॒राणि॑ । \newline
27. अ॒ग्ने रा॑ति॒थ्य मा॑ति॒थ्य म॒ग्ने र॒ग्ने रा॑ति॒थ्यम् । \newline
28. आ॒ति॒थ्य म॑स्यस्या ति॒थ्य मा॑ति॒थ्य म॑सि । \newline
29. अ॒सि॒ विष्ण॑वे॒ विष्ण॑वे ऽस्यसि॒ विष्ण॑वे । \newline
30. विष्ण॑वे त्वा त्वा॒ विष्ण॑वे॒ विष्ण॑वे त्वा । \newline
31. त्वेतीति॑ त्वा॒ त्वेति॑ । \newline
32. इत्या॑हा॒हे तीत्या॑ह । \newline
33. आ॒ह॒ गा॒य॒त्रि॒यै गा॑यत्रि॒या आ॑हाह गायत्रि॒यै । \newline
34. गा॒य॒त्रि॒या ए॒वैव गा॑यत्रि॒यै गा॑यत्रि॒या ए॒व । \newline
35. ए॒वैते नै॒ते नै॒वै वैतेन॑ । \newline
36. ए॒तेन॑ करोति करो त्ये॒ते नै॒तेन॑ करोति । \newline
37. क॒रो॒ति॒ सोम॑स्य॒ सोम॑स्य करोति करोति॒ सोम॑स्य । \newline
38. सोम॑स्या ति॒थ्य मा॑ति॒थ्यꣳ सोम॑स्य॒ सोम॑स्या ति॒थ्यम् । \newline
39. आ॒ति॒थ्य म॑स्यस्या ति॒थ्य मा॑ति॒थ्य म॑सि । \newline
40. अ॒सि॒ विष्ण॑वे॒ विष्ण॑वे ऽस्यसि॒ विष्ण॑वे । \newline
41. विष्ण॑वे त्वा त्वा॒ विष्ण॑वे॒ विष्ण॑वे त्वा । \newline
42. त्वेतीति॑ त्वा॒ त्वेति॑ । \newline
43. इत्या॑हा॒हे तीत्या॑ह । \newline
44. आ॒ह॒ त्रि॒ष्टुभे᳚ त्रि॒ष्टुभ॑ आहाह त्रि॒ष्टुभे᳚ । \newline
45. त्रि॒ष्टुभ॑ ए॒वैव त्रि॒ष्टुभे᳚ त्रि॒ष्टुभ॑ ए॒व । \newline
46. ए॒वैते नै॒ते नै॒वै वैतेन॑ । \newline
47. ए॒तेन॑ करोति करो त्ये॒ते नै॒तेन॑ करोति । \newline
48. क॒रो॒ त्यति॑थे॒ रति॑थेः करोति करो॒ त्यति॑थेः । \newline
49. अति॑थे राति॒थ्य मा॑ति॒थ्य मति॑थे॒ रति॑थे राति॒थ्यम् । \newline
50. आ॒ति॒थ्य म॑स्यस्या ति॒थ्य मा॑ति॒थ्य म॑सि । \newline
51. अ॒सि॒ विष्ण॑वे॒ विष्ण॑वे ऽस्यसि॒ विष्ण॑वे । \newline
52. विष्ण॑वे त्वा त्वा॒ विष्ण॑वे॒ विष्ण॑वे त्वा । \newline
53. त्वेतीति॑ त्वा॒ त्वेति॑ । \newline
54. इत्या॑हा॒हे तीत्या॑ह । \newline
55. आ॒ह॒ जग॑त्यै॒ जग॑त्या आहाह॒ जग॑त्यै । \newline
56. जग॑त्या ए॒वैव जग॑त्यै॒ जग॑त्या ए॒व । \newline

\textbf{Ghana Paata } \newline

1. ए॒ष य॒ज्ञ्स्य॑ य॒ज्ञ् स्यै॒ष ए॒ष य॒ज्ञ्स्या᳚ न्वार॒म्भो᳚ ऽन्वार॒म्भो य॒ज्ञ् स्यै॒ष ए॒ष य॒ज्ञ्स्या᳚ न्वार॒म्भः । \newline
2. य॒ज्ञ्स्या᳚ न्वार॒म्भो᳚ ऽन्वार॒म्भो य॒ज्ञ्स्य॑ य॒ज्ञ्स्या᳚ न्वार॒म्भो ऽन॑वच्छित्त्या॒ अन॑वच्छित्त्या अन्वार॒म्भो य॒ज्ञ्स्य॑ य॒ज्ञ्स्या᳚ न्वार॒म्भो ऽन॑वच्छित्त्यै । \newline
3. अ॒न्वा॒र॒म्भो ऽन॑वच्छित्त्या॒ अन॑वच्छित्त्या अन्वार॒म्भो᳚ ऽन्वार॒म्भो ऽन॑वच्छित्त्यै॒ याव॑द्भि॒र् याव॑द्भि॒ रन॑वच्छित्त्या अन्वार॒म्भो᳚ ऽन्वार॒म्भो ऽन॑वच्छित्त्यै॒ याव॑द्भिः । \newline
4. अ॒न्वा॒र॒म्भ इत्य॑नु - आ॒र॒म्भः । \newline
5. अन॑वच्छित्त्यै॒ याव॑द्भि॒र् याव॑द्भि॒ रन॑वच्छित्त्या॒ अन॑वच्छित्त्यै॒ याव॑द्भि॒र् वै वै याव॑द्भि॒ रन॑वच्छित्त्या॒ अन॑वच्छित्त्यै॒ याव॑द्भि॒र् वै । \newline
6. अन॑वच्छित्त्या॒ इत्यन॑व - छि॒त्त्यै॒ । \newline
7. याव॑द्भि॒र् वै वै याव॑द्भि॒र् याव॑द्भि॒र् वै राजा॒ राजा॒ वै याव॑द्भि॒र् याव॑द्भि॒र् वै राजा᳚ । \newline
8. याव॑द्भि॒रिति॒ याव॑त् - भिः॒ । \newline
9. वै राजा॒ राजा॒ वै वै राजा॑ ऽनुच॒रै र॑नुच॒रै राजा॒ वै वै राजा॑ ऽनुच॒रैः । \newline
10. राजा॑ ऽनुच॒रै र॑नुच॒रै राजा॒ राजा॑ ऽनुच॒रै रा॒गच्छ॑ त्या॒गच्छ॑ त्यनुच॒रै राजा॒ राजा॑ ऽनुच॒रै रा॒गच्छ॑ति । \newline
11. अ॒नु॒च॒रै रा॒गच्छ॑ त्या॒गच्छ॑ त्यनुच॒रै र॑नुच॒रै रा॒गच्छ॑ति॒ सर्वे᳚भ्यः॒ सर्वे᳚भ्य आ॒गच्छ॑ त्यनुच॒रै र॑नुच॒रै रा॒गच्छ॑ति॒ सर्वे᳚भ्यः । \newline
12. अ॒नु॒च॒रैरित्य॑नु - च॒रैः । \newline
13. आ॒गच्छ॑ति॒ सर्वे᳚भ्यः॒ सर्वे᳚भ्य आ॒गच्छ॑ त्या॒गच्छ॑ति॒ सर्वे᳚भ्यो॒ वै वै सर्वे᳚भ्य आ॒गच्छ॑ त्या॒गच्छ॑ति॒ सर्वे᳚भ्यो॒ वै । \newline
14. आ॒गच्छ॒तीत्या᳚ - गच्छ॑ति । \newline
15. सर्वे᳚भ्यो॒ वै वै सर्वे᳚भ्यः॒ सर्वे᳚भ्यो॒ वै तेभ्य॒ स्तेभ्यो॒ वै सर्वे᳚भ्यः॒ सर्वे᳚भ्यो॒ वै तेभ्यः॑ । \newline
16. वै तेभ्य॒ स्तेभ्यो॒ वै वै तेभ्य॑ आति॒थ्य मा॑ति॒थ्यम् तेभ्यो॒ वै वै तेभ्य॑ आति॒थ्यम् । \newline
17. तेभ्य॑ आति॒थ्य मा॑ति॒थ्यम् तेभ्य॒ स्तेभ्य॑ आति॒थ्यम् क्रि॑यते क्रियत आति॒थ्यम् तेभ्य॒ स्तेभ्य॑ आति॒थ्यम् क्रि॑यते । \newline
18. आ॒ति॒थ्यम् क्रि॑यते क्रियत आति॒थ्य मा॑ति॒थ्यम् क्रि॑यते॒ छन्दाꣳ॑सि॒ छन्दाꣳ॑सि क्रियत आति॒थ्य मा॑ति॒थ्यम् क्रि॑यते॒ छन्दाꣳ॑सि । \newline
19. क्रि॒य॒ते॒ छन्दाꣳ॑सि॒ छन्दाꣳ॑सि क्रियते क्रियते॒ छन्दाꣳ॑सि॒ खलु॒ खलु॒ छन्दाꣳ॑सि क्रियते क्रियते॒ छन्दाꣳ॑सि॒ खलु॑ । \newline
20. छन्दाꣳ॑सि॒ खलु॒ खलु॒ छन्दाꣳ॑सि॒ छन्दाꣳ॑सि॒ खलु॒ वै वै खलु॒ छन्दाꣳ॑सि॒ छन्दाꣳ॑सि॒ खलु॒ वै । \newline
21. खलु॒ वै वै खलु॒ खलु॒ वै सोम॑स्य॒ सोम॑स्य॒ वै खलु॒ खलु॒ वै सोम॑स्य । \newline
22. वै सोम॑स्य॒ सोम॑स्य॒ वै वै सोम॑स्य॒ राज्ञो॒ राज्ञ्ः॒ सोम॑स्य॒ वै वै सोम॑स्य॒ राज्ञ्ः॑ । \newline
23. सोम॑स्य॒ राज्ञो॒ राज्ञ्ः॒ सोम॑स्य॒ सोम॑स्य॒ राज्ञो॑ ऽनुच॒रा ण्य॑नुच॒राणि॒ राज्ञ्ः॒ सोम॑स्य॒ सोम॑स्य॒ राज्ञो॑ ऽनुच॒राणि॑ । \newline
24. राज्ञो॑ ऽनुच॒रा ण्य॑नुच॒राणि॒ राज्ञो॒ राज्ञो॑ ऽनुच॒रा ण्य॒ग्ने र॒ग्ने र॑नुच॒राणि॒ राज्ञो॒ राज्ञो॑ ऽनुच॒रा ण्य॒ग्नेः । \newline
25. अ॒नु॒च॒रा ण्य॒ग्ने र॒ग्ने र॑नुच॒रा ण्य॑नुच॒रा ण्य॒ग्ने रा॑ति॒थ्य मा॑ति॒थ्य म॒ग्ने र॑नुच॒रा
ण्य॑नुच॒रा ण्य॒ग्ने रा॑ति॒थ्यम् । \newline
26. अ॒नु॒च॒राणीत्य॑नु - च॒राणि॑ । \newline
27. अ॒ग्ने रा॑ति॒थ्य मा॑ति॒थ्य म॒ग्ने र॒ग्ने रा॑ति॒थ्य म॑स्य स्याति॒थ्य म॒ग्ने र॒ग्ने रा॑ति॒थ्य म॑सि । \newline
28. आ॒ति॒थ्य म॑स्य स्याति॒थ्य मा॑ति॒थ्य म॑सि॒ विष्ण॑वे॒ विष्ण॑वे ऽस्याति॒थ्य मा॑ति॒थ्य म॑सि॒ विष्ण॑वे । \newline
29. अ॒सि॒ विष्ण॑वे॒ विष्ण॑वे ऽस्यसि॒ विष्ण॑वे त्वा त्वा॒ विष्ण॑वे ऽस्यसि॒ विष्ण॑वे त्वा । \newline
30. विष्ण॑वे त्वा त्वा॒ विष्ण॑वे॒ विष्ण॑वे॒ त्वेतीति॑ त्वा॒ विष्ण॑वे॒ विष्ण॑वे॒ त्वेति॑ । \newline
31. त्वेतीति॑ त्वा॒ त्वेत्या॑ हा॒हेति॑ त्वा॒ त्वेत्या॑ह । \newline
32. इत्या॑हा॒हे तीत्या॑ह गायत्रि॒यै गा॑यत्रि॒या आ॒हे तीत्या॑ह गायत्रि॒यै । \newline
33. आ॒ह॒ गा॒य॒त्रि॒यै गा॑यत्रि॒या आ॑हाह गायत्रि॒या ए॒वैव गा॑यत्रि॒या आ॑हाह गायत्रि॒या ए॒व । \newline
34. गा॒य॒त्रि॒या ए॒वैव गा॑यत्रि॒यै गा॑यत्रि॒या ए॒वैते नै॒ते नै॒व गा॑यत्रि॒यै गा॑यत्रि॒या ए॒वै तेन॑ । \newline
35. ए॒वैते नै॒ते नै॒वैवै तेन॑ करोति करो त्ये॒ते नै॒वै वैतेन॑ करोति । \newline
36. ए॒तेन॑ करोति करो त्ये॒ते नै॒तेन॑ करोति॒ सोम॑स्य॒ सोम॑स्य करो त्ये॒ते नै॒तेन॑ करोति॒ सोम॑स्य । \newline
37. क॒रो॒ति॒ सोम॑स्य॒ सोम॑स्य करोति करोति॒ सोम॑स्या ति॒थ्य मा॑ति॒थ्यꣳ सोम॑स्य करोति करोति॒ सोम॑स्या ति॒थ्यम् । \newline
38. सोम॑स्या ति॒थ्य मा॑ति॒थ्यꣳ सोम॑स्य॒ सोम॑स्या ति॒थ्य म॑स्य स्याति॒थ्यꣳ सोम॑स्य॒ सोम॑स्या ति॒थ्य म॑सि । \newline
39. आ॒ति॒थ्य म॑स्यस्या ति॒थ्य मा॑ति॒थ्य म॑सि॒ विष्ण॑वे॒ विष्ण॑वे ऽस्या ति॒थ्य मा॑ति॒थ्य म॑सि॒ विष्ण॑वे । \newline
40. अ॒सि॒ विष्ण॑वे॒ विष्ण॑वे ऽस्यसि॒ विष्ण॑वे त्वा त्वा॒ विष्ण॑वे ऽस्यसि॒ विष्ण॑वे त्वा । \newline
41. विष्ण॑वे त्वा त्वा॒ विष्ण॑वे॒ विष्ण॑वे॒ त्वेतीति॑ त्वा॒ विष्ण॑वे॒ विष्ण॑वे॒ त्वेति॑ । \newline
42. त्वेतीति॑ त्वा॒ त्वेत्या॑ हा॒हेति॑ त्वा॒ त्वेत्या॑ह । \newline
43. इत्या॑हा॒हे तीत्या॑ह त्रि॒ष्टुभे᳚ त्रि॒ष्टुभ॑ आ॒हे तीत्या॑ह त्रि॒ष्टुभे᳚ । \newline
44. आ॒ह॒ त्रि॒ष्टुभे᳚ त्रि॒ष्टुभ॑ आहाह त्रि॒ष्टुभ॑ ए॒वैव त्रि॒ष्टुभ॑ आहाह त्रि॒ष्टुभ॑ ए॒व । \newline
45. त्रि॒ष्टुभ॑ ए॒वैव त्रि॒ष्टुभे᳚ त्रि॒ष्टुभ॑ ए॒वैते नै॒ते नै॒व त्रि॒ष्टुभे᳚ त्रि॒ष्टुभ॑ ए॒वैतेन॑ । \newline
46. ए॒वैते नै॒ते नै॒वै वैतेन॑ करोति करो त्ये॒ते नै॒वै वैतेन॑ करोति । \newline
47. ए॒तेन॑ करोति करो त्ये॒ते नै॒तेन॑ करो॒ त्यति॑थे॒ रति॑थेः करो त्ये॒ते नै॒तेन॑ करो॒ त्यति॑थेः । \newline
48. क॒रो॒ त्यति॑थे॒ रति॑थेः करोति करो॒ त्यति॑थे राति॒थ्य मा॑ति॒थ्य मति॑थेः करोति करो॒ त्यति॑थे राति॒थ्यम् । \newline
49. अति॑थे राति॒थ्य मा॑ति॒थ्य मति॑थे॒ रति॑थे राति॒थ्य म॑स्यस्या ति॒थ्य मति॑थे॒ रति॑थे राति॒थ्य म॑सि । \newline
50. आ॒ति॒थ्य म॑स्यस्या ति॒थ्य मा॑ति॒थ्य म॑सि॒ विष्ण॑वे॒ विष्ण॑वे ऽस्याति॒थ्य मा॑ति॒थ्य म॑सि॒ विष्ण॑वे । \newline
51. अ॒सि॒ विष्ण॑वे॒ विष्ण॑वे ऽस्यसि॒ विष्ण॑वे त्वा त्वा॒ विष्ण॑वे ऽस्यसि॒ विष्ण॑वे त्वा । \newline
52. विष्ण॑वे त्वा त्वा॒ विष्ण॑वे॒ विष्ण॑वे॒ त्वेतीति॑ त्वा॒ विष्ण॑वे॒ विष्ण॑वे॒ त्वेति॑ । \newline
53. त्वेतीति॑ त्वा॒ त्वेत्या॑ हा॒हेति॑ त्वा॒ त्वेत्या॑ह । \newline
54. इत्या॑हा॒हे तीत्या॑ह॒ जग॑त्यै॒ जग॑त्या आ॒हे तीत्या॑ह॒ जग॑त्यै । \newline
55. आ॒ह॒ जग॑त्यै॒ जग॑त्या आहाह॒ जग॑त्या ए॒वैव जग॑त्या आहाह॒ जग॑त्या ए॒व । \newline
56. जग॑त्या ए॒वैव जग॑त्यै॒ जग॑त्या ए॒वैते नै॒ते नै॒व जग॑त्यै॒ जग॑त्या ए॒वैतेन॑ । \newline
\pagebreak
\markright{ TS 6.2.1.3  \hfill https://www.vedavms.in \hfill}

\section{ TS 6.2.1.3 }

\textbf{TS 6.2.1.3 } \newline
\textbf{Samhita Paata} \newline

ए॒वैतेन॑ करोत्य॒ग्नये᳚ त्वा रायस्पोष॒दाव्न्ने॒ विष्ण॑वे॒ त्वेत्या॑हानु॒ष्टुभ॑ ए॒वैतेन॑ करोति श्ये॒नाय॑ त्वा सोम॒भृते॒ विष्ण॑वे॒ त्वेत्या॑ह गायत्रि॒या ए॒वैतेन॑ करोति॒ पञ्च॒ कृत्वो॑ गृह्णाति॒ पञ्चा᳚क्षरा प॒ङ्क्तिः पाङ्क्तो॑ य॒ज्ञो य॒ज्ञ्मे॒वाव॑ रुन्धे ब्रह्मवा॒दिनो॑ वदन्ति॒ कस्मा᳚थ् स॒त्याद्-गा॑यत्रि॒या उ॑भ॒यत॑ आति॒थ्यस्य॑ क्रियत॒ इति॒ यदे॒वादः सोम॒मा- [  ] \newline

\textbf{Pada Paata} \newline

ए॒व । ए॒तेन॑ । क॒रो॒ति॒ । अ॒ग्नये᳚ । त्वा॒ । रा॒य॒स्पो॒ष॒दाव्न्न॒ इति॑ रायस्पोष - दाव्न्ने᳚ । विष्ण॑वे । त्वा॒ । इति॑ । आ॒ह॒ । अ॒नु॒ष्टुभ॒ इत्य॑नु - स्तुभे᳚ । ए॒व । ए॒तेन॑ । क॒रो॒ति॒ । श्ये॒नाय॑ । त्वा॒ । सो॒म॒भृत॒ इति॑ सोम - भृते᳚ । विष्ण॑वे । त्वा॒ । इति॑ । आ॒ह॒ । गा॒य॒त्रि॒यै । ए॒व । ए॒तेन॑ । क॒रो॒ति॒ । पञ्च॑ । कृत्वः॑ । गृ॒ह्णा॒ति॒ । पञ्चा᳚क्ष॒रेति॒ पञ्च॑ - अ॒क्ष॒रा॒ । प॒ङ्क्तिः । पाङ्क्तः॑ । य॒ज्ञ्ः । य॒ज्ञ्म् । ए॒व । अवेति॑ । रु॒न्धे॒ । ब्र॒ह्म॒वा॒दिन॒ इति॑ ब्रह्म - वा॒दिनः॑ । व॒द॒न्ति॒ । कस्मा᳚त् । स॒त्यात् । गा॒य॒त्रि॒यै । उ॒भ॒यतः॑ । आ॒ति॒थ्यस्य॑ । क्रि॒य॒ते॒ । इति॑ । यत् । ए॒व । अ॒दः । सोम᳚म् । एति॑ ।  \newline


\textbf{Krama Paata} \newline

ए॒वैतेन॑ । ए॒तेन॑ करोति । क॒रो॒त्य॒ग्नये᳚ । अ॒ग्नये᳚ त्वा । त्वा॒ रा॒य॒स्पो॒ष॒दाव्.न्ने᳚ । रा॒य॒स्पो॒ष॒दाव्.न्ने॒ विष्ण॑वे । रा॒य॒स्पो॒ष॒दाव्.न्न॒ इति॑ रायस्पोष - दाव्.न्ने᳚ । विष्ण॑वे त्वा । त्वेति॑ । इत्या॑ह । आ॒हा॒नु॒ष्टुभे᳚ । अ॒नु॒ष्टुभ॑ ए॒व । अ॒नु॒ष्टुभ॒ इत्य॑नु - स्तुभे᳚ । ए॒वैतेन॑ । ए॒तेन॑ करोति । क॒रो॒ति॒ श्ये॒नाय॑ । श्ये॒नाय॑ त्वा । त्वा॒ सो॒म॒भृते᳚ । सो॒म॒भृते॒ विष्ण॑वे । सो॒म॒भृत॒ इति॑ सोम - भृते᳚ । विष्ण॑वे त्वा । त्वेति॑ । इत्या॑ह । आ॒ह॒ गा॒य॒त्रि॒यै । गा॒य॒त्रि॒या ए॒व । ए॒वैतेन॑ । ए॒तेन॑ करोति । क॒रो॒ति॒ पञ्च॑ । पञ्च॒ कृत्वः॑ । कृत्वो॑ गृह्णाति । गृ॒ह्णा॒ति॒ पञ्चा᳚क्षरा । पञ्चा᳚क्षरा प॒ङ्‍क्तिः । पञ्चा᳚क्ष॒रेति॒ पञ्च॑ - अ॒क्ष॒रा॒ । प॒ङ्‍क्तिः पाङ्‍क्तः॑ । पाङ्‍क्तो॑ य॒ज्ञ्ः । य॒ज्ञो य॒ज्ञ्म् । य॒ज्ञ्मे॒व । ए॒वाव॑ । अव॑ रुन्धे । रु॒न्धे॒ ब्र॒ह्म॒वा॒दिनः॑ । ब्र॒ह्म॒वा॒दिनो॑ वदन्ति । ब्र॒ह्म॒वा॒दिन॒ इति॑ ब्रह्म - वा॒दिनः॑ । व॒द॒न्ति॒ कस्मा᳚त् । कस्मा᳚थ् स॒त्यात् । स॒त्याद् गा॑यत्रि॒यै । गा॒य॒त्रि॒या उ॑भ॒यतः॑ । उ॒भ॒यत॑ आति॒थ्यस्य॑ । आ॒ति॒थ्यस्य॑ क्रियते । क्रि॒य॒त॒ इति॑ । इति॒ यत् । यदे॒व । ए॒वादः । अ॒दः सोम᳚म् । सोम॒मा । आऽह॑रत् \newline

\textbf{Jatai Paata} \newline

1. ए॒वैते नै॒ते नै॒वै वैतेन॑ । \newline
2. ए॒तेन॑ करोति करो त्ये॒ते नै॒तेन॑ करोति । \newline
3. क॒रो॒ त्य॒ग्नये॒ ऽग्नये॑ करोति करो त्य॒ग्नये᳚ । \newline
4. अ॒ग्नये᳚ त्वा त्वा॒ ऽग्नये॒ ऽग्नये᳚ त्वा । \newline
5. त्वा॒ रा॒य॒स्पो॒ष॒दाव्.न्ने॑ रायस्पोष॒दाव्.न्ने᳚ त्वा त्वा रायस्पोष॒दाव्.न्ने᳚ । \newline
6. रा॒य॒स्पो॒ष॒दाव्.न्ने॒ विष्ण॑वे॒ विष्ण॑वे रायस्पोष॒दाव्.न्ने॑ रायस्पोष॒दाव्.न्ने॒ विष्ण॑वे । \newline
7. रा॒य॒स्पो॒ष॒दाव्.न्न॒ इति॑ रायस्पोष - दाव्.न्ने᳚ । \newline
8. विष्ण॑वे त्वा त्वा॒ विष्ण॑वे॒ विष्ण॑वे त्वा । \newline
9. त्वेतीति॑ त्वा॒ त्वेति॑ । \newline
10. इत्या॑हा॒हे तीत्या॑ह । \newline
11. आ॒हा॒ नु॒ष्टुभे॑ ऽनु॒ष्टुभ॑ आहाहा नु॒ष्टुभे᳚ । \newline
12. अ॒नु॒ष्टुभ॑ ए॒वैवा नु॒ष्टुभे॑ ऽनु॒ष्टुभ॑ ए॒व । \newline
13. अ॒नु॒ष्टुभ॒ इत्य॑नु - स्तुभे᳚ । \newline
14. ए॒वैते नै॒ते नै॒वै वैतेन॑ । \newline
15. ए॒तेन॑ करोति करो त्ये॒ते नै॒तेन॑ करोति । \newline
16. क॒रो॒ति॒ श्ये॒नाय॑ श्ये॒नाय॑ करोति करोति श्ये॒नाय॑ । \newline
17. श्ये॒नाय॑ त्वा त्वा श्ये॒नाय॑ श्ये॒नाय॑ त्वा । \newline
18. त्वा॒ सो॒म॒भृते॑ सोम॒भृते᳚ त्वा त्वा सोम॒भृते᳚ । \newline
19. सो॒म॒भृते॒ विष्ण॑वे॒ विष्ण॑वे सोम॒भृते॑ सोम॒भृते॒ विष्ण॑वे । \newline
20. सो॒म॒भृत॒ इति॑ सोम - भृते᳚ । \newline
21. विष्ण॑वे त्वा त्वा॒ विष्ण॑वे॒ विष्ण॑वे त्वा । \newline
22. त्वेतीति॑ त्वा॒ त्वेति॑ । \newline
23. इत्या॑हा॒हे तीत्या॑ह । \newline
24. आ॒ह॒ गा॒य॒त्रि॒यै गा॑यत्रि॒या आ॑हाह गायत्रि॒यै । \newline
25. गा॒य॒त्रि॒या ए॒वैव गा॑यत्रि॒यै गा॑यत्रि॒या ए॒व । \newline
26. ए॒वैते नै॒ते नै॒वै वैतेन॑ । \newline
27. ए॒तेन॑ करोति करो त्ये॒ते नै॒तेन॑ करोति । \newline
28. क॒रो॒ति॒ पञ्च॒ पञ्च॑ करोति करोति॒ पञ्च॑ । \newline
29. पञ्च॒ कृत्वः॒ कृत्वः॒ पञ्च॒ पञ्च॒ कृत्वः॑ । \newline
30. कृत्वो॑ गृह्णाति गृह्णाति॒ कृत्वः॒ कृत्वो॑ गृह्णाति । \newline
31. गृ॒ह्णा॒ति॒ पञ्चा᳚क्षरा॒ पञ्चा᳚क्षरा गृह्णाति गृह्णाति॒ पञ्चा᳚क्षरा । \newline
32. पञ्चा᳚क्षरा प॒ङ्क्तिः प॒ङ्क्तिः पञ्चा᳚क्षरा॒ पञ्चा᳚क्षरा प॒ङ्क्तिः । \newline
33. पञ्चा᳚क्ष॒रेति॒ पञ्च॑ - अ॒क्ष॒रा॒ । \newline
34. प॒ङ्क्तिः पाङ्क्तः॒ पाङ्क्तः॑ प॒ङ्क्तिः प॒ङ्क्तिः पाङ्क्तः॑ । \newline
35. पाङ्क्तो॑ य॒ज्ञो य॒ज्ञ्ः पाङ्क्तः॒ पाङ्क्तो॑ य॒ज्ञ्ः । \newline
36. य॒ज्ञो य॒ज्ञ्ं ॅय॒ज्ञ्ं ॅय॒ज्ञो य॒ज्ञो य॒ज्ञ्म् । \newline
37. य॒ज्ञ् मे॒वैव य॒ज्ञ्ं ॅय॒ज्ञ् मे॒व । \newline
38. ए॒वावा वै॒वै वाव॑ । \newline
39. अव॑ रुन्धे रु॒न्धे ऽवाव॑ रुन्धे । \newline
40. रु॒न्धे॒ ब्र॒ह्म॒वा॒दिनो᳚ ब्रह्मवा॒दिनो॑ रुन्धे रुन्धे ब्रह्मवा॒दिनः॑ । \newline
41. ब्र॒ह्म॒वा॒दिनो॑ वदन्ति वदन्ति ब्रह्मवा॒दिनो᳚ ब्रह्मवा॒दिनो॑ वदन्ति । \newline
42. ब्र॒ह्म॒वा॒दिन॒ इति॑ ब्रह्म - वा॒दिनः॑ । \newline
43. व॒द॒न्ति॒ कस्मा॒त् कस्मा᳚द् वदन्ति वदन्ति॒ कस्मा᳚त् । \newline
44. कस्मा᳚थ् स॒त्याथ् स॒त्यात् कस्मा॒त् कस्मा᳚थ् स॒त्यात् । \newline
45. स॒त्याद् गा॑यत्रि॒यै गा॑यत्रि॒यै स॒त्याथ् स॒त्याद् गा॑यत्रि॒यै । \newline
46. गा॒य॒त्रि॒या उ॑भ॒यत॑ उभ॒यतो॑ गायत्रि॒यै गा॑यत्रि॒या उ॑भ॒यतः॑ । \newline
47. उ॒भ॒यत॑ आति॒थ्यस्या॑ ति॒थ्य स्यो॑भ॒यत॑ उभ॒यत॑ आति॒थ्यस्य॑ । \newline
48. आ॒ति॒थ्यस्य॑ क्रियते क्रियत आति॒थ्यस्या॑ ति॒थ्यस्य॑ क्रियते । \newline
49. क्रि॒य॒त॒ इतीति॑ क्रियते क्रियत॒ इति॑ । \newline
50. इति॒ यद् यदितीति॒ यत् । \newline
51. यदे॒वैव यद् यदे॒व । \newline
52. ए॒वादो॑ ऽद ए॒वैवादः । \newline
53. अ॒दः सोमꣳ॒॒ सोम॑ म॒दो॑ ऽदः सोम᳚म् । \newline
54. सोम॒ मा सोमꣳ॒॒ सोम॒ मा । \newline
55. आ ऽह॑र॒ दह॑र॒ दाऽह॑रत् । \newline

\textbf{Ghana Paata } \newline

1. ए॒वैते नै॒ते नै॒वैवै तेन॑ करोति करो त्ये॒ते नै॒वैवैतेन॑ करोति । \newline
2. ए॒तेन॑ करोति करो त्ये॒ते नै॒तेन॑ करो त्य॒ग्नये॒ ऽग्नये॑ करो त्ये॒ते नै॒तेन॑ करो त्य॒ग्नये᳚ । \newline
3. क॒रो॒ त्य॒ग्नये॒ ऽग्नये॑ करोति करो त्य॒ग्नये᳚ त्वा त्वा॒ ऽग्नये॑ करोति करो त्य॒ग्नये᳚ त्वा । \newline
4. अ॒ग्नये᳚ त्वा त्वा॒ ऽग्नये॒ ऽग्नये᳚ त्वा रायस्पोष॒दाव्.न्ने॑ रायस्पोष॒दाव्.न्ने᳚ त्वा॒ ऽग्नये॒ ऽग्नये᳚ त्वा रायस्पोष॒दाव्.न्ने᳚ । \newline
5. त्वा॒ रा॒य॒स्पो॒ष॒दाव्.न्ने॑ रायस्पोष॒दाव्.न्ने᳚ त्वा त्वा रायस्पोष॒दाव्.न्ने॒ विष्ण॑वे॒ विष्ण॑वे रायस्पोष॒दाव्.न्ने᳚ त्वा त्वा रायस्पोष॒दाव्.न्ने॒ विष्ण॑वे । \newline
6. रा॒य॒स्पो॒ष॒दाव्.न्ने॒ विष्ण॑वे॒ विष्ण॑वे रायस्पोष॒दाव्.न्ने॑ रायस्पोष॒दाव्.न्ने॒ विष्ण॑वे त्वा त्वा॒ विष्ण॑वे रायस्पोष॒दाव्.न्ने॑ रायस्पोष॒दाव्.न्ने॒ विष्ण॑वे त्वा । \newline
7. रा॒य॒स्पो॒ष॒दाव्.न्न॒ इति॑ रायस्पोष - दाव्.न्ने᳚ । \newline
8. विष्ण॑वे त्वा त्वा॒ विष्ण॑वे॒ विष्ण॑वे॒ त्वेतीति॑ त्वा॒ विष्ण॑वे॒ विष्ण॑वे॒ त्वेति॑ । \newline
9. त्वेतीति॑ त्वा॒ त्वेत्या॑ हा॒हेति॑ त्वा॒ त्वेत्या॑ह । \newline
10. इत्या॑हा॒हे तीत्या॑हा नु॒ष्टुभे॑ ऽनु॒ष्टुभ॑ आ॒हे तीत्या॑हा नु॒ष्टुभे᳚ । \newline
11. आ॒हा॒ नु॒ष्टुभे॑ ऽनु॒ष्टुभ॑ आहाहा नु॒ष्टुभ॑ ए॒वैवा नु॒ष्टुभ॑ आहाहा नु॒ष्टुभ॑ ए॒व । \newline
12. अ॒नु॒ष्टुभ॑ ए॒वैवा नु॒ष्टुभे॑ ऽनु॒ष्टुभ॑ ए॒वैते नै॒तेनै॒वा नु॒ष्टुभे॑ ऽनु॒ष्टुभ॑ ए॒वै तेन॑ । \newline
13. अ॒नु॒ष्टुभ॒ इत्य॑नु - स्तुभे᳚ । \newline
14. ए॒वैते नै॒ते नै॒वै वैतेन॑ करोति करो त्ये॒ते नै॒वै वैतेन॑ करोति । \newline
15. ए॒तेन॑ करोति करो त्ये॒ते नै॒तेन॑ करोति श्ये॒नाय॑ श्ये॒नाय॑ करो त्ये॒ते नै॒तेन॑ करोति श्ये॒नाय॑ । \newline
16. क॒रो॒ति॒ श्ये॒नाय॑ श्ये॒नाय॑ करोति करोति श्ये॒नाय॑ त्वा त्वा श्ये॒नाय॑ करोति करोति श्ये॒नाय॑ त्वा । \newline
17. श्ये॒नाय॑ त्वा त्वा श्ये॒नाय॑ श्ये॒नाय॑ त्वा सोम॒भृते॑ सोम॒भृते᳚ त्वा श्ये॒नाय॑ श्ये॒नाय॑ त्वा सोम॒भृते᳚ । \newline
18. त्वा॒ सो॒म॒भृते॑ सोम॒भृते᳚ त्वा त्वा सोम॒भृते॒ विष्ण॑वे॒ विष्ण॑वे सोम॒भृते᳚ त्वा त्वा सोम॒भृते॒ विष्ण॑वे । \newline
19. सो॒म॒भृते॒ विष्ण॑वे॒ विष्ण॑वे सोम॒भृते॑ सोम॒भृते॒ विष्ण॑वे त्वा त्वा॒ विष्ण॑वे सोम॒भृते॑ सोम॒भृते॒ विष्ण॑वे त्वा । \newline
20. सो॒म॒भृत॒ इति॑ सोम - भृते᳚ । \newline
21. विष्ण॑वे त्वा त्वा॒ विष्ण॑वे॒ विष्ण॑वे॒ त्वेतीति॑ त्वा॒ विष्ण॑वे॒ विष्ण॑वे॒ त्वेति॑ । \newline
22. त्वेतीति॑ त्वा॒ त्वेत्या॑ हा॒हेति॑ त्वा॒ त्वेत्या॑ह । \newline
23. इत्या॑हा॒हे तीत्या॑ह गायत्रि॒यै गा॑यत्रि॒या आ॒हे तीत्या॑ह गायत्रि॒यै । \newline
24. आ॒ह॒ गा॒य॒त्रि॒यै गा॑यत्रि॒या आ॑हाह गायत्रि॒या ए॒वैव गा॑यत्रि॒या आ॑हाह गायत्रि॒या ए॒व । \newline
25. गा॒य॒त्रि॒या ए॒वैव गा॑यत्रि॒यै गा॑यत्रि॒या ए॒वैते नै॒तेनै॒व गा॑यत्रि॒यै गा॑यत्रि॒या ए॒वै तेन॑ । \newline
26. ए॒वैते नै॒ते नै॒वैवै तेन॑ करोति करो त्ये॒ते नै॒वैवै तेन॑ करोति । \newline
27. ए॒तेन॑ करोति करो त्ये॒ते नै॒तेन॑ करोति॒ पञ्च॒ पञ्च॑ करो त्ये॒ते नै॒तेन॑ करोति॒ पञ्च॑ । \newline
28. क॒रो॒ति॒ पञ्च॒ पञ्च॑ करोति करोति॒ पञ्च॒ कृत्वः॒ कृत्वः॒ पञ्च॑ करोति करोति॒ पञ्च॒ कृत्वः॑ । \newline
29. पञ्च॒ कृत्वः॒ कृत्वः॒ पञ्च॒ पञ्च॒ कृत्वो॑ गृह्णाति गृह्णाति॒ कृत्वः॒ पञ्च॒ पञ्च॒ कृत्वो॑ गृह्णाति । \newline
30. कृत्वो॑ गृह्णाति गृह्णाति॒ कृत्वः॒ कृत्वो॑ गृह्णाति॒ पञ्चा᳚क्षरा॒ पञ्चा᳚क्षरा गृह्णाति॒ कृत्वः॒ कृत्वो॑ गृह्णाति॒ पञ्चा᳚क्षरा । \newline
31. गृ॒ह्णा॒ति॒ पञ्चा᳚क्षरा॒ पञ्चा᳚क्षरा गृह्णाति गृह्णाति॒ पञ्चा᳚क्षरा प॒ङ्क्तिः प॒ङ्क्तिः पञ्चा᳚क्षरा गृह्णाति गृह्णाति॒ पञ्चा᳚क्षरा प॒ङ्क्तिः । \newline
32. पञ्चा᳚क्षरा प॒ङ्क्तिः प॒ङ्क्तिः पञ्चा᳚क्षरा॒ पञ्चा᳚क्षरा प॒ङ्क्तिः पाङ्क्तः॒ पाङ्क्तः॑ प॒ङ्क्तिः पञ्चा᳚क्षरा॒ पञ्चा᳚क्षरा प॒ङ्क्तिः पाङ्क्तः॑ । \newline
33. पञ्चा᳚क्ष॒रेति॒ पञ्च॑ - अ॒क्ष॒रा॒ । \newline
34. प॒ङ्क्तिः पाङ्क्तः॒ पाङ्क्तः॑ प॒ङ्क्तिः प॒ङ्क्तिः पाङ्क्तो॑ य॒ज्ञो य॒ज्ञ्ः पाङ्क्तः॑ प॒ङ्क्तिः प॒ङ्क्तिः पाङ्क्तो॑ य॒ज्ञ्ः । \newline
35. पाङ्क्तो॑ य॒ज्ञो य॒ज्ञ्ः पाङ्क्तः॒ पाङ्क्तो॑ य॒ज्ञो य॒ज्ञ्ं ॅय॒ज्ञ्ं ॅय॒ज्ञ्ः पाङ्क्तः॒ पाङ्क्तो॑ य॒ज्ञो य॒ज्ञ्म् । \newline
36. य॒ज्ञो य॒ज्ञ्ं ॅय॒ज्ञ्ं ॅय॒ज्ञो य॒ज्ञो य॒ज्ञ् मे॒वैव य॒ज्ञ्ं ॅय॒ज्ञो य॒ज्ञो य॒ज्ञ् मे॒व । \newline
37. य॒ज्ञ् मे॒वैव य॒ज्ञ्ं ॅय॒ज्ञ् मे॒वा वावै॒व य॒ज्ञ्ं ॅय॒ज्ञ् मे॒वाव॑ । \newline
38. ए॒वावा वै॒वै वाव॑ रुन्धे रु॒न्धे ऽवै॒वै वाव॑ रुन्धे । \newline
39. अव॑ रुन्धे रु॒न्धे ऽवाव॑ रुन्धे ब्रह्मवा॒दिनो᳚ ब्रह्मवा॒दिनो॑ रु॒न्धे ऽवाव॑ रुन्धे ब्रह्मवा॒दिनः॑ । \newline
40. रु॒न्धे॒ ब्र॒ह्म॒वा॒दिनो᳚ ब्रह्मवा॒दिनो॑ रुन्धे रुन्धे ब्रह्मवा॒दिनो॑ वदन्ति वदन्ति ब्रह्मवा॒दिनो॑ रुन्धे रुन्धे ब्रह्मवा॒दिनो॑ वदन्ति । \newline
41. ब्र॒ह्म॒वा॒दिनो॑ वदन्ति वदन्ति ब्रह्मवा॒दिनो᳚ ब्रह्मवा॒दिनो॑ वदन्ति॒ कस्मा॒त् कस्मा᳚द् वदन्ति ब्रह्मवा॒दिनो᳚ ब्रह्मवा॒दिनो॑ वदन्ति॒ कस्मा᳚त् । \newline
42. ब्र॒ह्म॒वा॒दिन॒ इति॑ ब्रह्म - वा॒दिनः॑ । \newline
43. व॒द॒न्ति॒ कस्मा॒त् कस्मा᳚द् वदन्ति वदन्ति॒ कस्मा᳚थ् स॒त्याथ् स॒त्यात् कस्मा᳚द् वदन्ति वदन्ति॒ कस्मा᳚थ् स॒त्यात् । \newline
44. कस्मा᳚थ् स॒त्याथ् स॒त्यात् कस्मा॒त् कस्मा᳚थ् स॒त्याद् गा॑यत्रि॒यै गा॑यत्रि॒यै स॒त्यात् कस्मा॒त् कस्मा᳚थ् स॒त्याद् गा॑यत्रि॒यै । \newline
45. स॒त्याद् गा॑यत्रि॒यै गा॑यत्रि॒यै स॒त्याथ् स॒त्याद् गा॑यत्रि॒या उ॑भ॒यत॑ उभ॒यतो॑ गायत्रि॒यै स॒त्याथ् स॒त्याद् गा॑यत्रि॒या उ॑भ॒यतः॑ । \newline
46. गा॒य॒त्रि॒या उ॑भ॒यत॑ उभ॒यतो॑ गायत्रि॒यै गा॑यत्रि॒या उ॑भ॒यत॑ आति॒थ्यस्या॑ ति॒थ्य स्यो॑भ॒यतो॑ गायत्रि॒यै गा॑यत्रि॒या उ॑भ॒यत॑ आति॒थ्यस्य॑ । \newline
47. उ॒भ॒यत॑ आति॒थ्यस्या॑ ति॒थ्य स्यो॑भ॒यत॑ उभ॒यत॑ आति॒थ्यस्य॑ क्रियते क्रियत आति॒थ्य स्यो॑भ॒यत॑ उभ॒यत॑ आति॒थ्यस्य॑ क्रियते । \newline
48. आ॒ति॒थ्यस्य॑ क्रियते क्रियत आति॒थ्यस्या॑ ति॒थ्यस्य॑ क्रियत॒ इतीति॑ क्रियत आति॒थ्यस्या॑ ति॒थ्यस्य॑ क्रियत॒ इति॑ । \newline
49. क्रि॒य॒त॒ इतीति॑ क्रियते क्रियत॒ इति॒ यद् यदिति॑ क्रियते क्रियत॒ इति॒ यत् । \newline
50. इति॒ यद् यदि तीति॒ यदे॒ वैव यदि तीति॒ यदे॒व । \newline
51. यदे॒वैव यद् यदे॒वादो॑ ऽद ए॒व यद् यदे॒वादः । \newline
52. ए॒वादो॑ ऽद ए॒वै वादः सोमꣳ॒॒ सोम॑ म॒द ए॒वै वादः सोम᳚म् । \newline
53. अ॒दः सोमꣳ॒॒ सोम॑ म॒दो॑ ऽदः सोम॒ मा सोम॑ म॒दो॑ ऽदः सोम॒ मा । \newline
54. सोम॒ मा सोमꣳ॒॒ सोम॒ मा ऽह॑र॒ दह॑र॒दा सोमꣳ॒॒ सोम॒ मा ऽह॑रत् । \newline
55. आ ऽह॑र॒ दह॑र॒दा ऽह॑र॒त् तस्मा॒त् तस्मा॒ दह॑र॒दा ऽह॑र॒त् तस्मा᳚त् । \newline
\pagebreak
\markright{ TS 6.2.1.4  \hfill https://www.vedavms.in \hfill}

\section{ TS 6.2.1.4 }

\textbf{TS 6.2.1.4 } \newline
\textbf{Samhita Paata} \newline

ऽह॑र॒त् तस्मा᳚द्-गायत्रि॒या उ॑भ॒यत॑ आति॒थ्यस्य॑ क्रियते पु॒रस्ता᳚च्चो॒ परि॑ष्टाच्च॒ शिरो॒ वा ए॒तद् य॒ज्ञ्स्य॒ यदा॑ति॒थ्यं नव॑कपालः पुरो॒डाशो॑ भवति॒ तस्मा᳚न्नव॒धा शिरो॒ विष्यू॑तं॒ नव॑कपालः पुरो॒डाशो॑ भवति॒ ते त्रय॑स्त्रिकपा॒लास्त्रि॒वृता॒ स्तोमे॑न॒ संमि॑ता॒स्तेज॑स्त्रि॒वृत् तेज॑ ए॒व य॒ज्ञ्स्य॑ शी॒र्॒.षन् द॑धाति॒ नव॑कपालः पुरो॒डाशो॑ भवति॒ ते त्रय॑स्त्रिकपा॒लास्त्रि॒वृता᳚ प्रा॒णेन॒ संमि॑तास्त्रि॒वृद्वै- [  ] \newline

\textbf{Pada Paata} \newline

अह॑रत् । तस्मा᳚त् । गा॒य॒त्रि॒यै । उ॒भ॒यतः॑ । आ॒ति॒थ्यस्य॑ । क्रि॒य॒ते॒ । पु॒रस्ता᳚त् । च॒ । उ॒परि॑ष्टात् । च॒ । शिरः॑ । वै । ए॒तत् । य॒ज्ञ्स्य॑ । यत् । आ॒ति॒थ्यम् । नव॑कपाल॒ इति॒ नव॑ - क॒पा॒लः॒ । पु॒रो॒डाशः॑ । भ॒व॒ति॒ । तस्मा᳚त् । न॒व॒धेति॑ नव - धा । शिरः॑ । विष्यू॑त॒मिति॒ वि - स्यू॒त॒म् । नव॑कपाल॒ इति॒ नव॑ - क॒पा॒लः॒ । पु॒रो॒डाशः॑ । भ॒व॒ति॒ । ते । त्रयः॑ । त्रि॒क॒पा॒ला इति॑ त्रि - क॒पा॒लाः । त्रि॒वृतेति॑ त्रि - वृता᳚ । स्तोमे॑न । संमि॑ता॒ इति॒ सं - मि॒ताः॒ । तेजः॑ । त्रि॒वृदिति॑ त्रि - वृत् । तेजः॑ । ए॒व । य॒ज्ञ्स्य॑ । शी॒र्॒.षन्न् । द॒धा॒ति॒ । नव॑कपाल॒ इति॒ नव॑-क॒पा॒लः॒ । पु॒रो॒डाशः॑ । भ॒व॒ति॒ । ते । त्रयः॑ । त्रि॒क॒पा॒ला इति॑ त्रि - क॒पा॒लाः । त्रि॒वृतेति॑ त्रि-वृता᳚ । प्रा॒णेनेति॑ प्र-अ॒नेन॑ । संमि॑ता॒ इति॒ सं-मि॒ताः॒ । त्रि॒वृदिति॑ त्रि - वृत् । वै ।  \newline


\textbf{Krama Paata} \newline

अह॑र॒त् तस्मा᳚त् । तस्मा᳚द् गायत्रि॒यै । गा॒य॒त्रि॒या उ॑भ॒यतः॑ । उ॒भ॒यत॑ आति॒थ्यस्य॑ । आ॒ति॒थ्यस्य॑ क्रियते । क्रि॒य॒ते॒ पु॒रस्ता᳚त् । पु॒रस्ता᳚च् च । चो॒परि॑ष्टात् । उ॒परि॑ष्टाच् च । च॒ शिरः॑ । शिरो॒ वै । वा ए॒तत् । ए॒तद् य॒ज्ञ्स्य॑ । य॒ज्ञ्स्य॒ यत् । यदा॑ति॒थ्यम् । आ॒ति॒थ्यम् नव॑कपालः । नव॑कपालः पुरो॒डाशः॑ । नव॑कपाल॒ इति॒ नव॑ - क॒पा॒लः॒ । पु॒रो॒डाशो॑ भवति । भ॒व॒ति॒ तस्मा᳚त् । तस्मा᳚न् नव॒धा । न॒व॒धा शिरः॑ । न॒व॒धेति॑ नव - धा । शिरो॒ विष्यू॑तम् । विष्यू॑त॒म् नव॑कपालः । विष्यू॑त॒मिति॒ वि - स्यू॒त॒म् । नव॑कपालः पुरो॒डाशः॑ । नव॑कपाल॒ इति॒ नव॑ - क॒पा॒लः॒ । पु॒रो॒डाशो॑ भवति । भ॒व॒ति॒ ते । ते त्रयः॑ । त्रय॑स्त्रिकपा॒लाः । त्रि॒क॒पा॒लास्त्रि॒वृता᳚ । त्रि॒क॒पा॒ला इति॑ त्रि - क॒पा॒लाः । त्रि॒वृता॒ स्तोमे॑न । त्रि॒वृतेति॑ त्रि - वृता᳚ । स्तोमे॑न॒ सम्मि॑ताः । सम्मि॑ता॒स्तेजः॑ । सम्मि॑ता॒ इति॒ सम् - मि॒ताः॒ । तेज॑ स्त्रि॒वृत् । त्रि॒वृत् तेजः॑ । त्रि॒वृदिति॑ त्रि - वृत् । तेज॑ ए॒व । ए॒व य॒ज्ञ्स्य॑ । य॒ज्ञ्स्य॑ शी॒र्.॒षन्न् । शी॒र्.॒षन् द॑धाति । द॒धा॒ति॒ नव॑कपालः । नव॑कपालः पुरो॒डाशः॑ । नव॑कपाल॒ इति॒ नव॑ - क॒पा॒लः॒ । पु॒रो॒डाशो॑ भवति । भ॒व॒ति॒ ते । ते त्रयः॑ । त्रय॑स्त्रिकपा॒लाः । त्रि॒क॒पा॒लास्त्रि॒वृता᳚ । त्रि॒क॒पा॒ला इति॑ त्रि - क॒पा॒लाः । त्रि॒वृता᳚ प्रा॒णेन॑ । त्रि॒वृतेति॑ त्रि - वृता᳚ । प्रा॒णेन॒ सम्मि॑ताः । प्रा॒णेनेति॑ प्र - अ॒नेन॑ । सम्मि॑तास्त्रि॒वृत् । सम्मि॑ता॒ इति॒ सम् - मि॒ताः॒ । त्रि॒वृद् वै । त्रि॒वृदिति॑ त्रि - वृत् । वै प्रा॒णः \newline

\textbf{Jatai Paata} \newline

1. अह॑र॒त् तस्मा॒त् तस्मा॒ दह॑र॒ दह॑र॒त् तस्मा᳚त् । \newline
2. तस्मा᳚द् गायत्रि॒यै गा॑यत्रि॒यै तस्मा॒त् तस्मा᳚द् गायत्रि॒यै । \newline
3. गा॒य॒त्रि॒या उ॑भ॒यत॑ उभ॒यतो॑ गायत्रि॒यै गा॑यत्रि॒या उ॑भ॒यतः॑ । \newline
4. उ॒भ॒यत॑ आति॒थ्यस्या॑ ति॒थ्यस्यो॑भ॒यत॑ उभ॒यत॑ आति॒थ्यस्य॑ । \newline
5. आ॒ति॒थ्यस्य॑ क्रियते क्रियत आति॒थ्यस्या॑ ति॒थ्यस्य॑ क्रियते । \newline
6. क्रि॒य॒ते॒ पु॒रस्ता᳚त् पु॒रस्ता᳚त् क्रियते क्रियते पु॒रस्ता᳚त् । \newline
7. पु॒रस्ता᳚च् च च पु॒रस्ता᳚त् पु॒रस्ता᳚च् च । \newline
8. चो॒परि॑ष्टा दु॒परि॑ष्टाच् च चो॒परि॑ष्टात् । \newline
9. उ॒परि॑ष्टाच् च चो॒परि॑ष्टा दु॒परि॑ष्टाच् च । \newline
10. च॒ शिरः॒ शिर॑श्च च॒ शिरः॑ । \newline
11. शिरो॒ वै वै शिरः॒ शिरो॒ वै । \newline
12. वा ए॒त दे॒तद् वै वा ए॒तत् । \newline
13. ए॒तद् य॒ज्ञ्स्य॑ य॒ज्ञ् स्यै॒त दे॒तद् य॒ज्ञ्स्य॑ । \newline
14. य॒ज्ञ्स्य॒ यद् यद् य॒ज्ञ्स्य॑ य॒ज्ञ्स्य॒ यत् । \newline
15. यदा॑ति॒थ्य मा॑ति॒थ्यं ॅयद् यदा॑ति॒थ्यम् । \newline
16. आ॒ति॒थ्यम् नव॑कपालो॒ नव॑कपाल आति॒थ्य मा॑ति॒थ्यम् नव॑कपालः । \newline
17. नव॑कपालः पुरो॒डाशः॑ पुरो॒डाशो॒ नव॑कपालो॒ नव॑कपालः पुरो॒डाशः॑ । \newline
18. नव॑कपाल॒ इति॒ नव॑ - क॒पा॒लः॒ । \newline
19. पु॒रो॒डाशो॑ भवति भवति पुरो॒डाशः॑ पुरो॒डाशो॑ भवति । \newline
20. भ॒व॒ति॒ तस्मा॒त् तस्मा᳚द् भवति भवति॒ तस्मा᳚त् । \newline
21. तस्मा᳚न् नव॒धा न॑व॒धा तस्मा॒त् तस्मा᳚न् नव॒धा । \newline
22. न॒व॒धा शिरः॒ शिरो॑ नव॒धा न॑व॒धा शिरः॑ । \newline
23. न॒व॒धेति॑ नव - धा । \newline
24. शिरो॒ विष्यू॑तं॒ ॅविष्यू॑तꣳ॒॒ शिरः॒ शिरो॒ विष्यू॑तम् । \newline
25. विष्यू॑त॒म् नव॑कपालो॒ नव॑कपालो॒ विष्यू॑तं॒ ॅविष्यू॑त॒म् नव॑कपालः । \newline
26. विष्यू॑त॒मिति॒ वि - स्यू॒त॒म् । \newline
27. नव॑कपालः पुरो॒डाशः॑ पुरो॒डाशो॒ नव॑कपालो॒ नव॑कपालः पुरो॒डाशः॑ । \newline
28. नव॑कपाल॒ इति॒ नव॑ - क॒पा॒लः॒ । \newline
29. पु॒रो॒डाशो॑ भवति भवति पुरो॒डाशः॑ पुरो॒डाशो॑ भवति । \newline
30. भ॒व॒ति॒ ते ते भ॑वति भवति॒ ते । \newline
31. ते त्रय॒ स्त्रय॒ स्ते ते त्रयः॑ । \newline
32. त्रय॑ स्त्रिकपा॒ला स्त्रि॑कपा॒ला स्त्रय॒ स्त्रय॑ स्त्रिकपा॒लाः । \newline
33. त्रि॒क॒पा॒ला स्त्रि॒वृता᳚ त्रि॒वृता᳚ त्रिकपा॒ला स्त्रि॑कपा॒ला स्त्रि॒वृता᳚ । \newline
34. त्रि॒क॒पा॒ला इति॑ त्रि - क॒पा॒लाः । \newline
35. त्रि॒वृता॒ स्तोमे॑न॒ स्तोमे॑न त्रि॒वृता᳚ त्रि॒वृता॒ स्तोमे॑न । \newline
36. त्रि॒वृतेति॑ त्रि - वृता᳚ । \newline
37. स्तोमे॑न॒ सम्मि॑ताः॒ सम्मि॑ताः॒ स्तोमे॑न॒ स्तोमे॑न॒ सम्मि॑ताः । \newline
38. सम्मि॑ता॒ स्तेज॒ स्तेजः॒ सम्मि॑ताः॒ सम्मि॑ता॒ स्तेजः॑ । \newline
39. सम्मि॑ता॒ इति॒ सं - मि॒ताः॒ । \newline
40. तेज॑ स्त्रि॒वृत् त्रि॒वृत् तेज॒ स्तेज॑ स्त्रि॒वृत् । \newline
41. त्रि॒वृत् तेज॒ स्तेज॑ स्त्रि॒वृत् त्रि॒वृत् तेजः॑ । \newline
42. त्रि॒वृदिति॑ त्रि - वृत् । \newline
43. तेज॑ ए॒वैव तेज॒ स्तेज॑ ए॒व । \newline
44. ए॒व य॒ज्ञ्स्य॑ य॒ज्ञ् स्यै॒वैव य॒ज्ञ्स्य॑ । \newline
45. य॒ज्ञ्स्य॑ शी॒र्॒.षञ् छी॒र्॒.षन्. य॒ज्ञ्स्य॑ य॒ज्ञ्स्य॑ शी॒र्॒.षन्न् । \newline
46. शी॒र्॒.षन् द॑धाति दधाति शी॒र्॒.षञ् छी॒र्॒.षन् द॑धाति । \newline
47. द॒धा॒ति॒ नव॑कपालो॒ नव॑कपालो दधाति दधाति॒ नव॑कपालः । \newline
48. नव॑कपालः पुरो॒डाशः॑ पुरो॒डाशो॒ नव॑कपालो॒ नव॑कपालः पुरो॒डाशः॑ । \newline
49. नव॑कपाल॒ इति॒ नव॑ - क॒पा॒लः॒ । \newline
50. पु॒रो॒डाशो॑ भवति भवति पुरो॒डाशः॑ पुरो॒डाशो॑ भवति । \newline
51. भ॒व॒ति॒ ते ते भ॑वति भवति॒ ते । \newline
52. ते त्रय॒ स्त्रय॒ स्ते ते त्रयः॑ । \newline
53. त्रय॑ स्त्रिकपा॒ला स्त्रि॑कपा॒ला स्त्रय॒ स्त्रय॑ स्त्रिकपा॒लाः । \newline
54. त्रि॒क॒पा॒ला स्त्रि॒वृता᳚ त्रि॒वृता᳚ त्रिकपा॒ला स्त्रि॑कपा॒ला स्त्रि॒वृता᳚ । \newline
55. त्रि॒क॒पा॒ला इति॑ त्रि - क॒पा॒लाः । \newline
56. त्रि॒वृता᳚ प्रा॒णेन॑ प्रा॒णेन॑ त्रि॒वृता᳚ त्रि॒वृता᳚ प्रा॒णेन॑ । \newline
57. त्रि॒वृतेति॑ त्रि - वृता᳚ । \newline
58. प्रा॒णेन॒ सम्मि॑ताः॒ सम्मि॑ताः प्रा॒णेन॑ प्रा॒णेन॒ सम्मि॑ताः । \newline
59. प्रा॒णेनेति॑ प्र - अ॒नेन॑ । \newline
60. सम्मि॑ता स्त्रि॒वृत् त्रि॒वृथ् सम्मि॑ताः॒ सम्मि॑ता स्त्रि॒वृत् । \newline
61. सम्मि॑ता॒ इति॒ सं - मि॒ताः॒ । \newline
62. त्रि॒वृद् वै वै त्रि॒वृत् त्रि॒वृद् वै । \newline
63. त्रि॒वृदिति॑ त्रि - वृत् । \newline
64. वै प्रा॒णः प्रा॒णो वै वै प्रा॒णः । \newline

\textbf{Ghana Paata } \newline

1. अह॑र॒त् तस्मा॒त् तस्मा॒ दह॑र॒ दह॑र॒त् तस्मा᳚द् गायत्रि॒यै गा॑यत्रि॒यै तस्मा॒ दह॑र॒ दह॑र॒त् तस्मा᳚द् गायत्रि॒यै । \newline
2. तस्मा᳚द् गायत्रि॒यै गा॑यत्रि॒यै तस्मा॒त् तस्मा᳚द् गायत्रि॒या उ॑भ॒यत॑ उभ॒यतो॑ गायत्रि॒यै तस्मा॒त् तस्मा᳚द् गायत्रि॒या उ॑भ॒यतः॑ । \newline
3. गा॒य॒त्रि॒या उ॑भ॒यत॑ उभ॒यतो॑ गायत्रि॒यै गा॑यत्रि॒या उ॑भ॒यत॑ आति॒थ्यस्या॑ ति॒थ्य स्यो॑भ॒यतो॑ गायत्रि॒यै गा॑यत्रि॒या उ॑भ॒यत॑ आति॒थ्यस्य॑ । \newline
4. उ॒भ॒यत॑ आति॒थ्यस्या॑ ति॒थ्य स्यो॑भ॒यत॑ उभ॒यत॑ आति॒थ्यस्य॑ क्रियते क्रियत आति॒थ्यस्यो॑ भ॒यत॑ उभ॒यत॑ आति॒थ्यस्य॑ क्रियते । \newline
5. आ॒ति॒थ्यस्य॑ क्रियते क्रियत आति॒थ्यस्या॑ ति॒थ्यस्य॑ क्रियते पु॒रस्ता᳚त् पु॒रस्ता᳚त् क्रियत आति॒थ्यस्या॑ ति॒थ्यस्य॑ क्रियते पु॒रस्ता᳚त् । \newline
6. क्रि॒य॒ते॒ पु॒रस्ता᳚त् पु॒रस्ता᳚त् क्रियते क्रियते पु॒रस्ता᳚च् च च पु॒रस्ता᳚त् क्रियते क्रियते पु॒रस्ता᳚च् च । \newline
7. पु॒रस्ता᳚च् च च पु॒रस्ता᳚त् पु॒रस्ता᳚च् चो॒परि॑ष्टा दु॒परि॑ष्टाच् च पु॒रस्ता᳚त् पु॒रस्ता᳚च् चो॒परि॑ष्टात् । \newline
8. चो॒परि॑ष्टा दु॒परि॑ष्टाच् च चो॒परि॑ष्टाच् च चो॒परि॑ष्टाच् च चो॒परि॑ष्टाच् च । \newline
9. उ॒परि॑ष्टाच् च चो॒परि॑ष्टा दु॒परि॑ष्टाच् च॒ शिरः॒ शिर॑ श्चो॒परि॑ष्टा दु॒परि॑ष्टाच् च॒ शिरः॑ । \newline
10. च॒ शिरः॒ शिर॑ श्च च॒ शिरो॒ वै वै शिर॑ श्च च॒ शिरो॒ वै । \newline
11. शिरो॒ वै वै शिरः॒ शिरो॒ वा ए॒त दे॒तद् वै शिरः॒ शिरो॒ वा ए॒तत् । \newline
12. वा ए॒त दे॒तद् वै वा ए॒तद् य॒ज्ञ्स्य॑ य॒ज्ञ् स्यै॒तद् वै वा ए॒तद् य॒ज्ञ्स्य॑ । \newline
13. ए॒तद् य॒ज्ञ्स्य॑ य॒ज्ञ् स्यै॒त दे॒तद् य॒ज्ञ्स्य॒ यद् यद् य॒ज्ञ् स्यै॒त दे॒तद् य॒ज्ञ्स्य॒ यत् । \newline
14. य॒ज्ञ्स्य॒ यद् यद् य॒ज्ञ्स्य॑ य॒ज्ञ्स्य॒ यदा॑ति॒थ्य मा॑ति॒थ्यं ॅयद् य॒ज्ञ्स्य॑ य॒ज्ञ्स्य॒ यदा॑ति॒थ्यम् । \newline
15. यदा॑ति॒थ्य मा॑ति॒थ्यं ॅयद् यदा॑ति॒थ्यन् नव॑कपालो॒ नव॑कपाल आति॒थ्यं ॅयद् यदा॑ति॒थ्यन् नव॑कपालः । \newline
16. आ॒ति॒थ्यन् नव॑कपालो॒ नव॑कपाल आति॒थ्य मा॑ति॒थ्यम् नव॑कपालः पुरो॒डाशः॑ पुरो॒डाशो॒ नव॑कपाल आति॒थ्य मा॑ति॒थ्यम् नव॑कपालः पुरो॒डाशः॑ । \newline
17. नव॑कपालः पुरो॒डाशः॑ पुरो॒डाशो॒ नव॑कपालो॒ नव॑कपालः पुरो॒डाशो॑ भवति भवति पुरो॒डाशो॒ नव॑कपालो॒ नव॑कपालः पुरो॒डाशो॑ भवति । \newline
18. नव॑कपाल॒ इति॒ नव॑ - क॒पा॒लः॒ । \newline
19. पु॒रो॒डाशो॑ भवति भवति पुरो॒डाशः॑ पुरो॒डाशो॑ भवति॒ तस्मा॒त् तस्मा᳚द् भवति पुरो॒डाशः॑ पुरो॒डाशो॑ भवति॒ तस्मा᳚त् । \newline
20. भ॒व॒ति॒ तस्मा॒त् तस्मा᳚द् भवति भवति॒ तस्मा᳚न् नव॒धा न॑व॒धा तस्मा᳚द् भवति भवति॒ तस्मा᳚न् नव॒धा । \newline
21. तस्मा᳚न् नव॒धा न॑व॒धा तस्मा॒त् तस्मा᳚न् नव॒धा शिरः॒ शिरो॑ नव॒धा तस्मा॒त् तस्मा᳚न् नव॒धा शिरः॑ । \newline
22. न॒व॒धा शिरः॒ शिरो॑ नव॒धा न॑व॒धा शिरो॒ विष्यू॑तं॒ ॅविष्यू॑तꣳ॒॒ शिरो॑ नव॒धा न॑व॒धा शिरो॒ विष्यू॑तम् । \newline
23. न॒व॒धेति॑ नव - धा । \newline
24. शिरो॒ विष्यू॑तं॒ ॅविष्यू॑तꣳ॒॒ शिरः॒ शिरो॒ विष्यू॑त॒म् नव॑कपालो॒ नव॑कपालो॒ विष्यू॑तꣳ॒॒ शिरः॒ शिरो॒ विष्यू॑त॒म् नव॑कपालः । \newline
25. विष्यू॑त॒म् नव॑कपालो॒ नव॑कपालो॒ विष्यू॑तं॒ ॅविष्यू॑त॒म् नव॑कपालः पुरो॒डाशः॑ पुरो॒डाशो॒ नव॑कपालो॒ विष्यू॑तं॒ ॅविष्यू॑त॒म् नव॑कपालः पुरो॒डाशः॑ । \newline
26. विष्यू॑त॒मिति॒ वि - स्यू॒त॒म् । \newline
27. नव॑कपालः पुरो॒डाशः॑ पुरो॒डाशो॒ नव॑कपालो॒ नव॑कपालः पुरो॒डाशो॑ भवति भवति पुरो॒डाशो॒ नव॑कपालो॒ नव॑कपालः पुरो॒डाशो॑ भवति । \newline
28. नव॑कपाल॒ इति॒ नव॑ - क॒पा॒लः॒ । \newline
29. पु॒रो॒डाशो॑ भवति भवति पुरो॒डाशः॑ पुरो॒डाशो॑ भवति॒ ते ते भ॑वति पुरो॒डाशः॑ पुरो॒डाशो॑ भवति॒ ते । \newline
30. भ॒व॒ति॒ ते ते भ॑वति भवति॒ ते त्रय॒ स्त्रय॒ स्ते भ॑वति भवति॒ ते त्रयः॑ । \newline
31. ते त्रय॒ स्त्रय॒ स्ते ते त्रय॑ स्त्रिकपा॒ला स्त्रि॑कपा॒ला स्त्रय॒ स्ते ते त्रय॑ स्त्रिकपा॒लाः । \newline
32. त्रय॑ स्त्रिकपा॒ला स्त्रि॑कपा॒ला स्त्रय॒ स्त्रय॑ स्त्रिकपा॒ला स्त्रि॒वृता᳚ त्रि॒वृता᳚ त्रिकपा॒ला स्त्रय॒ स्त्रय॑ स्त्रिकपा॒ला स्त्रि॒वृता᳚ । \newline
33. त्रि॒क॒पा॒ला स्त्रि॒वृता᳚ त्रि॒वृता᳚ त्रिकपा॒ला स्त्रि॑कपा॒ला स्त्रि॒वृता॒ स्तोमे॑न॒ स्तोमे॑न त्रि॒वृता᳚ त्रिकपा॒ला स्त्रि॑कपा॒ला स्त्रि॒वृता॒ स्तोमे॑न । \newline
34. त्रि॒क॒पा॒ला इति॑ त्रि - क॒पा॒लाः । \newline
35. त्रि॒वृता॒ स्तोमे॑न॒ स्तोमे॑न त्रि॒वृता᳚ त्रि॒वृता॒ स्तोमे॑न॒ सम्मि॑ताः॒ सम्मि॑ताः॒ स्तोमे॑न त्रि॒वृता᳚ त्रि॒वृता॒ स्तोमे॑न॒ सम्मि॑ताः । \newline
36. त्रि॒वृतेति॑ त्रि - वृता᳚ । \newline
37. स्तोमे॑न॒ सम्मि॑ताः॒ सम्मि॑ताः॒ स्तोमे॑न॒ स्तोमे॑न॒ सम्मि॑ता॒ स्तेज॒ स्तेजः॒ सम्मि॑ताः॒ स्तोमे॑न॒ स्तोमे॑न॒ सम्मि॑ता॒ स्तेजः॑ । \newline
38. सम्मि॑ता॒ स्तेज॒ स्तेजः॒ सम्मि॑ताः॒ सम्मि॑ता॒ स्तेज॑ स्त्रि॒वृत् त्रि॒वृत् तेजः॒ सम्मि॑ताः॒ सम्मि॑ता॒ स्तेज॑ स्त्रि॒वृत् । \newline
39. सम्मि॑ता॒ इति॒ सं - मि॒ताः॒ । \newline
40. तेज॑ स्त्रि॒वृत् त्रि॒वृत् तेज॒ स्तेज॑ स्त्रि॒वृत् तेज॒ स्तेज॑ स्त्रि॒वृत् तेज॒ स्तेज॑ स्त्रि॒वृत् तेजः॑ । \newline
41. त्रि॒वृत् तेज॒ स्तेज॑ स्त्रि॒वृत् त्रि॒वृत् तेज॑ ए॒वैव तेज॑ स्त्रि॒वृत् त्रि॒वृत् तेज॑ ए॒व । \newline
42. त्रि॒वृदिति॑ त्रि - वृत् । \newline
43. तेज॑ ए॒वैव तेज॒ स्तेज॑ ए॒व य॒ज्ञ्स्य॑ य॒ज्ञ्स्यै॒व तेज॒ स्तेज॑ ए॒व य॒ज्ञ्स्य॑ । \newline
44. ए॒व य॒ज्ञ्स्य॑ य॒ज्ञ्स्यै॒वैव य॒ज्ञ्स्य॑ शी॒र्॒.षञ् छी॒र्॒.षन्. य॒ज्ञ्स्यै॒वैव य॒ज्ञ्स्य॑ शी॒र्॒.षन्न् । \newline
45. य॒ज्ञ्स्य॑ शी॒र्॒.षञ् छी॒र्॒.षन्. य॒ज्ञ्स्य॑ य॒ज्ञ्स्य॑ शी॒र्॒.षन् द॑धाति दधाति शी॒र्॒.षन्. य॒ज्ञ्स्य॑ य॒ज्ञ्स्य॑ शी॒र्॒.षन् द॑धाति । \newline
46. शी॒र्॒.षन् द॑धाति दधाति शी॒र्॒.षञ् छी॒र्॒.षन् द॑धाति॒ नव॑कपालो॒ नव॑कपालो दधाति शी॒र्॒.षञ् छी॒र्॒.षन् द॑धाति॒ नव॑कपालः । \newline
47. द॒धा॒ति॒ नव॑कपालो॒ नव॑कपालो दधाति दधाति॒ नव॑कपालः पुरो॒डाशः॑ पुरो॒डाशो॒ नव॑कपालो दधाति दधाति॒ नव॑कपालः पुरो॒डाशः॑ । \newline
48. नव॑कपालः पुरो॒डाशः॑ पुरो॒डाशो॒ नव॑कपालो॒ नव॑कपालः पुरो॒डाशो॑ भवति भवति पुरो॒डाशो॒ नव॑कपालो॒ नव॑कपालः पुरो॒डाशो॑ भवति । \newline
49. नव॑कपाल॒ इति॒ नव॑ - क॒पा॒लः॒ । \newline
50. पु॒रो॒डाशो॑ भवति भवति पुरो॒डाशः॑ पुरो॒डाशो॑ भवति॒ ते ते भ॑वति पुरो॒डाशः॑ पुरो॒डाशो॑ भवति॒ ते । \newline
51. भ॒व॒ति॒ ते ते भ॑वति भवति॒ ते त्रय॒ स्त्रय॒ स्ते भ॑वति भवति॒ ते त्रयः॑ । \newline
52. ते त्रय॒ स्त्रय॒ स्ते ते त्रय॑ स्त्रिकपा॒ला स्त्रि॑कपा॒ला स्त्रय॒ स्ते ते त्रय॑ स्त्रिकपा॒लाः । \newline
53. त्रय॑ स्त्रिकपा॒ला स्त्रि॑कपा॒ला स्त्रय॒ स्त्रय॑ स्त्रिकपा॒ला स्त्रि॒वृता᳚ त्रि॒वृता᳚ त्रिकपा॒ला स्त्रय॒ स्त्रय॑ स्त्रिकपा॒ला स्त्रि॒वृता᳚ । \newline
54. त्रि॒क॒पा॒ला स्त्रि॒वृता᳚ त्रि॒वृता᳚ त्रिकपा॒ला स्त्रि॑कपा॒ला स्त्रि॒वृता᳚ प्रा॒णेन॑ प्रा॒णेन॑ त्रि॒वृता᳚ त्रिकपा॒ला स्त्रि॑कपा॒ला स्त्रि॒वृता᳚ प्रा॒णेन॑ । \newline
55. त्रि॒क॒पा॒ला इति॑ त्रि - क॒पा॒लाः । \newline
56. त्रि॒वृता᳚ प्रा॒णेन॑ प्रा॒णेन॑ त्रि॒वृता᳚ त्रि॒वृता᳚ प्रा॒णेन॒ सम्मि॑ताः॒ सम्मि॑ताः प्रा॒णेन॑ त्रि॒वृता᳚ त्रि॒वृता᳚ प्रा॒णेन॒ सम्मि॑ताः । \newline
57. त्रि॒वृतेति॑ त्रि - वृता᳚ । \newline
58. प्रा॒णेन॒ सम्मि॑ताः॒ सम्मि॑ताः प्रा॒णेन॑ प्रा॒णेन॒ सम्मि॑ता स्त्रि॒वृत् त्रि॒वृथ् सम्मि॑ताः प्रा॒णेन॑ प्रा॒णेन॒ सम्मि॑ता स्त्रि॒वृत् । \newline
59. प्रा॒णेनेति॑ प्र - अ॒नेन॑ । \newline
60. सम्मि॑ता स्त्रि॒वृत् त्रि॒वृथ् सम्मि॑ताः॒ सम्मि॑ता स्त्रि॒वृद् वै वै त्रि॒वृथ् सम्मि॑ताः॒ सम्मि॑ता स्त्रि॒वृद् वै । \newline
61. सम्मि॑ता॒ इति॒ सं - मि॒ताः॒ । \newline
62. त्रि॒वृद् वै वै त्रि॒वृत् त्रि॒वृद् वै प्रा॒णः प्रा॒णो वै त्रि॒वृत् त्रि॒वृद् वै प्रा॒णः । \newline
63. त्रि॒वृदिति॑ त्रि - वृत् । \newline
64. वै प्रा॒णः प्रा॒णो वै वै प्रा॒ण स्त्रि॒वृत॑म् त्रि॒वृत॑म् प्रा॒णो वै वै प्रा॒ण स्त्रि॒वृत᳚म् । \newline
\pagebreak
\markright{ TS 6.2.1.5  \hfill https://www.vedavms.in \hfill}

\section{ TS 6.2.1.5 }

\textbf{TS 6.2.1.5 } \newline
\textbf{Samhita Paata} \newline

प्रा॒ण-स्त्रि॒वृत॑मे॒व प्रा॒णम॑भिपू॒र्वं ॅय॒ज्ञ्स्य॑ शी॒र्॒.षन् द॑धाति प्र॒जाप॑ते॒र्वा ए॒तानि॒ पक्ष्मा॑णि॒ यद॑श्ववा॒ला ऐ᳚क्ष॒वी ति॒रश्ची॒ यदाश्व॑वालः प्रस्त॒रो भव॑त्यैक्ष॒वी ति॒रश्ची᳚ प्र॒जाप॑तेरे॒व तच्चक्षुः॒ संभ॑रति दे॒वा वै या आहु॑ती॒रजु॑हवु॒स्ता असु॑रा नि॒ष्काव॑माद॒न् ते दे॒वाः का᳚र्ष्म॒र्य॑मपश्यन् कर्म॒ण्यो॑ वै कर्मै॑नेन कुर्वी॒तेति॒ ते का᳚र्ष्मर्य॒मया᳚न् परि॒धीन॑- [  ] \newline

\textbf{Pada Paata} \newline

प्रा॒ण इति॑ प्र -अ॒नः । त्रि॒वृत॒मिति॑ त्रि - वृत᳚म् । ए॒व । प्रा॒णमिति॑ प्र -अ॒नम् । अ॒भि॒पू॒र्वमित्य॑भि - पू॒र्वम् । य॒ज्ञ्स्य॑ । शी॒र्॒.षन्न् । द॒धा॒ति॒ । प्र॒जाप॑ते॒रिति॑ प्र॒जा - प॒तेः॒ । वै । ए॒तानि॑ । पक्ष्मा॑णि । यत् । अ॒श्व॒वा॒ला इत्य॑श्व - वा॒लाः । ऐ॒क्ष॒वी इति॑ । ति॒रश्ची॒ इति॑ । यत् । आश्व॑वाल॒ इत्याश्व॑ - वा॒लः॒ । प्र॒स्त॒र इति॑ प्र - स्त॒रः । भव॑ति । ऐ॒क्ष॒वी इति॑ । ति॒रश्ची॒ इति॑ । प्र॒जाप॑ते॒रिति॑ प्र॒जा -प॒तेः॒ । ए॒व । तत् । चक्षुः॑ । समिति॑ । भ॒र॒ति॒ । दे॒वाः । वै । याः । आहु॑ती॒रित्या - हु॒तीः॒ । अजु॑हवुः । ताः । असु॑राः । नि॒ष्काव᳚म् । आ॒द॒न्न् । ते । दे॒वाः । का॒र्ष्म॒र्य᳚म् । अ॒प॒श्य॒न्न् । क॒र्म॒ण्यः॑ । वै । कर्म॑ । ए॒ने॒न॒ । कु॒र्वी॒त॒ । इति॑ । ते । का॒र्ष्म॒र्य॒मया॒निति॑ कार्ष्मर्य - मयान्॑ । प॒रि॒धीनिति॑ परि - धीन् ।  \newline


\textbf{Krama Paata} \newline

प्रा॒णस्त्रि॒वृत᳚म् । प्रा॒ण इति॑ प्र - अ॒नः । त्रि॒वृत॑मे॒व । त्रि॒वृत॒मिति॑ त्रि - वृत᳚म् । ए॒व प्रा॒णम् । प्रा॒णम॑भिपू॒र्वम् । प्रा॒णमिति॑ प्र - अ॒नम् । अ॒भि॒पू॒र्वम् ॅय॒ज्ञ्स्य॑ । अ॒भि॒पू॒र्वमित्य॑भि - पू॒र्वम् । य॒ज्ञ्स्य॑ शी॒र्.॒षन्न् । शी॒र्.॒षन् द॑धाति । द॒धा॒ति॒ प्र॒जाप॑तेः । प्र॒जाप॑ते॒र् वै । प्र॒जाप॑ते॒रिति॑ प्र॒जा - प॒तेः॒ । वा ए॒तानि॑ । ए॒तानि॒ पक्ष्मा॑णि । पक्ष्मा॑णि॒ यत् । यद॑श्ववा॒लाः । अ॒श्व॒वा॒ला ऐ᳚क्ष॒वी । अ॒श्व॒वा॒ला इत्य॑श्व - वा॒लाः । ऐ॒क्ष॒वी ति॒रश्ची᳚ । ऐ॒क्ष॒वी इत्यै᳚क्ष॒वी । ति॒रश्ची॒ यत् । ति॒रश्ची॒ इति॑ ति॒रश्ची᳚ । यदाश्व॑वालः । आश्व॑वालः प्रस्त॒रः । आश्व॑वाल॒ इत्याश्व॑ - वा॒लः॒ । प्र॒स्त॒रो भव॑ति । प्र॒स्त॒र इति॑ प्र - स्त॒रः । भव॑त्यैक्ष॒वी । ऐ॒क्ष॒वी ति॒रश्ची᳚ । ऐ॒क्ष॒वी इत्यै᳚क्ष॒वी । ति॒रश्ची᳚ प्र॒जाप॑तेः । ति॒रश्ची॒ इति॑ - ति॒रश्ची᳚ । प्र॒जाप॑तेरे॒व । प्र॒जाप॑ते॒रिति॑ प्र॒जा - प॒तेः॒ । ए॒व तत् । तच् चक्षुः॑ । चक्षुः॒ सम् । सम् भ॑रति । भ॒र॒ति॒ दे॒वाः । दे॒वा वै । वै याः । या आहु॑तीः । आहु॑ती॒रजु॑हवुः । आहु॑ती॒रित्या - हु॒तीः॒ । अजु॑हवु॒स्ताः । ता असु॑राः । असु॑रा नि॒ष्काव᳚म् । नि॒ष्काव॑मादन्न् । आ॒द॒न् ते । ते दे॒वाः । दे॒वाः का᳚र्ष्म॒र्य᳚म् । का॒र्ष्म॒र्य॑मपश्यन्न् । अ॒प॒श्य॒न् क॒र्म॒ण्यः॑ । क॒र्म॒ण्यो॑ वै । वै कर्म॑ । कर्मै॑नेन । ए॒ने॒न॒ कु॒र्वी॒त॒ । कु॒र्वी॒तेति॑ । इति॒ ते । ते का᳚र्ष्मर्य॒मयान्॑ । का॒र्ष्म॒र्य॒मया᳚न् परि॒धीन् । का॒र्ष्म॒र्य॒मया॒निति॑ कार्ष्मर्य - मयान्॑ । प॒रि॒धीन॑कुर्वत । प॒रि॒धीनिति॑ परि - धीन् \newline

\textbf{Jatai Paata} \newline

1. प्रा॒ण स्त्रि॒वृत॑म् त्रि॒वृत॑म् प्रा॒णः प्रा॒ण स्त्रि॒वृत᳚म् । \newline
2. प्रा॒ण इति॑ प्र - अ॒नः । \newline
3. त्रि॒वृत॑ मे॒वैव त्रि॒वृत॑म् त्रि॒वृत॑ मे॒व । \newline
4. त्रि॒वृत॒मिति॑ त्रि - वृत᳚म् । \newline
5. ए॒व प्रा॒णम् प्रा॒ण मे॒वैव प्रा॒णम् । \newline
6. प्रा॒ण म॑भिपू॒र्व म॑भिपू॒र्वम् प्रा॒णम् प्रा॒ण म॑भिपू॒र्वम् । \newline
7. प्रा॒णमिति॑ प्र - अ॒नम् । \newline
8. अ॒भि॒पू॒र्वं ॅय॒ज्ञ्स्य॑ य॒ज्ञ्स्या॑ भिपू॒र्व म॑भिपू॒र्वं ॅय॒ज्ञ्स्य॑ । \newline
9. अ॒भि॒पू॒र्वमित्य॑भि - पू॒र्वम् । \newline
10. य॒ज्ञ्स्य॑ शी॒र्॒.षञ् छी॒र्॒.षन्. य॒ज्ञ्स्य॑ य॒ज्ञ्स्य॑ शी॒र्॒.षन्न् । \newline
11. शी॒र्॒.षन् द॑धाति दधाति शी॒र्॒.षञ् छी॒र्॒.षन् द॑धाति । \newline
12. द॒धा॒ति॒ प्र॒जाप॑तेः प्र॒जाप॑तेर् दधाति दधाति प्र॒जाप॑तेः । \newline
13. प्र॒जाप॑ते॒र् वै वै प्र॒जाप॑तेः प्र॒जाप॑ते॒र् वै । \newline
14. प्र॒जाप॑ते॒रिति॑ प्र॒जा - प॒तेः॒ । \newline
15. वा ए॒ता न्ये॒तानि॒ वै वा ए॒तानि॑ । \newline
16. ए॒तानि॒ पक्ष्मा॑णि॒ पक्ष्मा᳚ ण्ये॒ता न्ये॒तानि॒ पक्ष्मा॑णि । \newline
17. पक्ष्मा॑णि॒ यद् यत् पक्ष्मा॑णि॒ पक्ष्मा॑णि॒ यत् । \newline
18. यद॑श्ववा॒ला अ॑श्ववा॒ला यद् यद॑श्ववा॒लाः । \newline
19. अ॒श्व॒वा॒ला ऐ᳚क्ष॒वी ऐ᳚क्ष॒वी अ॑श्ववा॒ला अ॑श्ववा॒ला ऐ᳚क्ष॒वी । \newline
20. अ॒श्व॒वा॒ला इत्य॑श्व - वा॒लाः । \newline
21. ऐ॒क्ष॒वी ति॒रश्ची॑ ति॒रश्ची॑ ऐक्ष॒वी ऐ᳚क्ष॒वी ति॒रश्ची᳚ । \newline
22. ऐ॒क्ष॒वी इत्यै᳚क्ष॒वी । \newline
23. ति॒रश्ची॒ यद् यत् ति॒रश्ची॑ ति॒रश्ची॒ यत् । \newline
24. ति॒रश्ची॒ इति॑ ति॒रश्ची᳚ । \newline
25. यदाश्व॑वाल॒ आश्व॑वालो॒ यद् यदाश्व॑वालः । \newline
26. आश्व॑वालः प्रस्त॒रः प्र॑स्त॒र आश्व॑वाल॒ आश्व॑वालः प्रस्त॒रः । \newline
27. आश्व॑वाल॒ इत्याश्व॑ - वा॒लः॒ । \newline
28. प्र॒स्त॒रो भव॑ति॒ भव॑ति प्रस्त॒रः प्र॑स्त॒रो भव॑ति । \newline
29. प्र॒स्त॒र इति॑ प्र - स्त॒रः । \newline
30. भव॑ त्यैक्ष॒वी ऐ᳚क्ष॒वी भव॑ति॒ भव॑ त्यैक्ष॒वी । \newline
31. ऐ॒क्ष॒वी ति॒रश्ची॑ ति॒रश्ची॑ ऐक्ष॒वी ऐ᳚क्ष॒वी ति॒रश्ची᳚ । \newline
32. ऐ॒क्ष॒वी इत्यै᳚क्ष॒वी । \newline
33. ति॒रश्ची᳚ प्र॒जाप॑तेः प्र॒जाप॑ते स्ति॒रश्ची॑ ति॒रश्ची᳚ प्र॒जाप॑तेः । \newline
34. ति॒रश्ची॒ इति॑ ति॒रश्ची᳚ । \newline
35. प्र॒जाप॑ते रे॒वैव प्र॒जाप॑तेः प्र॒जाप॑ते रे॒व । \newline
36. प्र॒जाप॑ते॒रिति॑ प्र॒जा - प॒तेः॒ । \newline
37. ए॒व तत् तदे॒वैव तत् । \newline
38. तच् चक्षु॒ श्चक्षु॒ स्तत् तच् चक्षुः॑ । \newline
39. चक्षुः॒ सꣳ सम् चक्षु॒ श्चक्षुः॒ सम् । \newline
40. सम् भ॑रति भरति॒ सꣳ सम् भ॑रति । \newline
41. भ॒र॒ति॒ दे॒वा दे॒वा भ॑रति भरति दे॒वाः । \newline
42. दे॒वा वै वै दे॒वा दे॒वा वै । \newline
43. वै या या वै वै याः । \newline
44. या आहु॑ती॒ राहु॑ती॒र् या या आहु॑तीः । \newline
45. आहु॑ती॒ रजु॑हवु॒ रजु॑हवु॒ राहु॑ती॒ राहु॑ती॒ रजु॑हवुः । \newline
46. आहु॑ती॒रित्या - हु॒तीः॒ । \newline
47. अजु॑हवु॒ स्ता स्ता अजु॑हवु॒ रजु॑हवु॒ स्ताः । \newline
48. ता असु॑रा॒ असु॑रा॒ स्ता स्ता असु॑राः । \newline
49. असु॑रा नि॒ष्काव॑म् नि॒ष्काव॒ मसु॑रा॒ असु॑रा नि॒ष्काव᳚म् । \newline
50. नि॒ष्काव॑ मादन् नादन् नि॒ष्काव॑म् नि॒ष्काव॑ मादन्न् । \newline
51. आ॒द॒न् ते त आ॑दन् नाद॒न् ते । \newline
52. ते दे॒वा दे॒वा स्ते ते दे॒वाः । \newline
53. दे॒वाः का᳚र्ष्म॒र्य॑म् कार्ष्म॒र्य॑म् दे॒वा दे॒वाः का᳚र्ष्म॒र्य᳚म् । \newline
54. का॒र्ष्म॒र्य॑ मपश्यन् नपश्यन् कार्ष्म॒र्य॑म् कार्ष्म॒र्य॑ मपश्यन्न् । \newline
55. अ॒प॒श्य॒न् क॒र्म॒ण्यः॑ कर्म॒ण्यो॑ ऽपश्यन् नपश्यन् कर्म॒ण्यः॑ । \newline
56. क॒र्म॒ण्यो॑ वै वै क॑र्म॒ण्यः॑ कर्म॒ण्यो॑ वै । \newline
57. वै कर्म॒ कर्म॒ वै वै कर्म॑ । \newline
58. कर्मै॑ नेनैनेन॒ कर्म॒ कर्मै॑नेन । \newline
59. ए॒ने॒न॒ कु॒र्वी॒त॒ कु॒र्वी॒ तै॒ने॒ नै॒ने॒न॒ कु॒र्वी॒त॒ । \newline
60. कु॒र्वी॒ते तीति॑ कुर्वीत कुर्वी॒तेति॑ । \newline
61. इति॒ ते त इतीति॒ ते । \newline
62. ते का᳚र्ष्मर्य॒मया᳚न् कार्ष्मर्य॒मया॒न् ते ते का᳚र्ष्मर्य॒मयान्॑ । \newline
63. का॒र्ष्म॒र्य॒मया᳚न् परि॒धीन् प॑रि॒धीन् का᳚र्ष्मर्य॒मया᳚न् कार्ष्मर्य॒मया᳚न् परि॒धीन् । \newline
64. का॒र्ष्म॒र्य॒मया॒निति॑ कार्ष्मर्य - मयान्॑ । \newline
65. प॒रि॒धी न॑कुर्वता कुर्वत परि॒धीन् प॑रि॒धी न॑कुर्वत । \newline
66. प॒रि॒धीनिति॑ परि - धीन् । \newline

\textbf{Ghana Paata } \newline

1. प्रा॒ण स्त्रि॒वृत॑म् त्रि॒वृत॑म् प्रा॒णः प्रा॒ण स्त्रि॒वृत॑ मे॒वैव त्रि॒वृत॑म् प्रा॒णः प्रा॒ण स्त्रि॒वृत॑ मे॒व । \newline
2. प्रा॒ण इति॑ प्र - अ॒नः । \newline
3. त्रि॒वृत॑ मे॒वैव त्रि॒वृत॑म् त्रि॒वृत॑ मे॒व प्रा॒णम् प्रा॒ण मे॒व त्रि॒वृत॑म् त्रि॒वृत॑ मे॒व प्रा॒णम् । \newline
4. त्रि॒वृत॒मिति॑ त्रि - वृत᳚म् । \newline
5. ए॒व प्रा॒णम् प्रा॒ण मे॒वैव प्रा॒ण म॑भिपू॒र्व म॑भिपू॒र्वम् प्रा॒ण मे॒वैव प्रा॒ण म॑भिपू॒र्वम् । \newline
6. प्रा॒ण म॑भिपू॒र्व म॑भिपू॒र्वम् प्रा॒णम् प्रा॒ण म॑भिपू॒र्वं ॅय॒ज्ञ्स्य॑ य॒ज्ञ्स्या॑ भिपू॒र्वम् प्रा॒णम् प्रा॒ण म॑भिपू॒र्वं ॅय॒ज्ञ्स्य॑ । \newline
7. प्रा॒णमिति॑ प्र - अ॒नम् । \newline
8. अ॒भि॒पू॒र्वं ॅय॒ज्ञ्स्य॑ य॒ज्ञ्स्या॑ भिपू॒र्व म॑भिपू॒र्वं ॅय॒ज्ञ्स्य॑ शी॒र्॒.षञ् छी॒र्॒.षन्. य॒ज्ञ्स्या॑ भिपू॒र्व म॑भिपू॒र्वं ॅय॒ज्ञ्स्य॑ शी॒र्॒.षन्न् । \newline
9. अ॒भि॒पू॒र्वमित्य॑भि - पू॒र्वम् । \newline
10. य॒ज्ञ्स्य॑ शी॒र्॒.षञ् छी॒र्॒.षन्. य॒ज्ञ्स्य॑ य॒ज्ञ्स्य॑ शी॒र्॒.षन् द॑धाति दधाति शी॒र्॒.षन्. य॒ज्ञ्स्य॑ य॒ज्ञ्स्य॑ शी॒र्॒.षन् द॑धाति । \newline
11. शी॒र्॒.षन् द॑धाति दधाति शी॒र्॒.षञ् छी॒र्॒.षन् द॑धाति प्र॒जाप॑तेः प्र॒जाप॑तेर् दधाति शी॒र्॒.षञ् छी॒र्॒.षन् द॑धाति प्र॒जाप॑तेः । \newline
12. द॒धा॒ति॒ प्र॒जाप॑तेः प्र॒जाप॑तेर् दधाति दधाति प्र॒जाप॑ते॒र् वै वै प्र॒जाप॑तेर् दधाति दधाति प्र॒जाप॑ते॒र् वै । \newline
13. प्र॒जाप॑ते॒र् वै वै प्र॒जाप॑तेः प्र॒जाप॑ते॒र् वा ए॒ता न्ये॒तानि॒ वै प्र॒जाप॑तेः प्र॒जाप॑ते॒र् वा ए॒तानि॑ । \newline
14. प्र॒जाप॑ते॒रिति॑ प्र॒जा - प॒तेः॒ । \newline
15. वा ए॒ता न्ये॒तानि॒ वै वा ए॒तानि॒ पक्ष्मा॑णि॒ पक्ष्मा᳚ ण्ये॒तानि॒ वै वा ए॒तानि॒ पक्ष्मा॑णि । \newline
16. ए॒तानि॒ पक्ष्मा॑णि॒ पक्ष्मा᳚ ण्ये॒ता न्ये॒तानि॒ पक्ष्मा॑णि॒ यद् यत् पक्ष्मा᳚ ण्ये॒ता न्ये॒तानि॒ पक्ष्मा॑णि॒ यत् । \newline
17. पक्ष्मा॑णि॒ यद् यत् पक्ष्मा॑णि॒ पक्ष्मा॑णि॒ यद॑श्ववा॒ला अ॑श्ववा॒ला यत् पक्ष्मा॑णि॒ पक्ष्मा॑णि॒ यद॑श्ववा॒लाः । \newline
18. यद॑श्ववा॒ला अ॑श्ववा॒ला यद् यद॑श्ववा॒ला ऐ᳚क्ष॒वी ऐ᳚क्ष॒वी अ॑श्ववा॒ला यद् यद॑श्ववा॒ला ऐ᳚क्ष॒वी । \newline
19. अ॒श्व॒वा॒ला ऐ᳚क्ष॒वी ऐ᳚क्ष॒वी अ॑श्ववा॒ला अ॑श्ववा॒ला ऐ᳚क्ष॒वी ति॒रश्ची॑ ति॒रश्ची॑ ऐक्ष॒वी अ॑श्ववा॒ला अ॑श्ववा॒ला ऐ᳚क्ष॒वी ति॒रश्ची᳚ । \newline
20. अ॒श्व॒वा॒ला इत्य॑श्व - वा॒लाः । \newline
21. ऐ॒क्ष॒वी ति॒रश्ची॑ ति॒रश्ची॑ ऐक्ष॒वी ऐ᳚क्ष॒वी ति॒रश्ची॒ यद् यत् ति॒रश्ची॑ ऐक्ष॒वी ऐ᳚क्ष॒वी ति॒रश्ची॒ यत् । \newline
22. ऐ॒क्ष॒वी इत्यै᳚क्ष॒वी । \newline
23. ति॒रश्ची॒ यद् यत् ति॒रश्ची॑ ति॒रश्ची॒ यदाश्व॑वाल॒ आश्व॑वालो॒ यत् ति॒रश्ची॑ ति॒रश्ची॒ यदाश्व॑वालः । \newline
24. ति॒रश्ची॒ इति॑ ति॒रश्ची᳚ । \newline
25. यदाश्व॑वाल॒ आश्व॑वालो॒ यद् यदाश्व॑वालः प्रस्त॒रः प्र॑स्त॒र आश्व॑वालो॒ यद् यदाश्व॑वालः प्रस्त॒रः । \newline
26. आश्व॑वालः प्रस्त॒रः प्र॑स्त॒र आश्व॑वाल॒ आश्व॑वालः प्रस्त॒रो भव॑ति॒ भव॑ति प्रस्त॒र आश्व॑वाल॒ आश्व॑वालः प्रस्त॒रो भव॑ति । \newline
27. आश्व॑वाल॒ इत्याश्व॑ - वा॒लः॒ । \newline
28. प्र॒स्त॒रो भव॑ति॒ भव॑ति प्रस्त॒रः प्र॑स्त॒रो भव॑ त्यैक्ष॒वी ऐ᳚क्ष॒वी भव॑ति प्रस्त॒रः प्र॑स्त॒रो भव॑ त्यैक्ष॒वी । \newline
29. प्र॒स्त॒र इति॑ प्र - स्त॒रः । \newline
30. भव॑ त्यैक्ष॒वी ऐ᳚क्ष॒वी भव॑ति॒ भव॑ त्यैक्ष॒वी ति॒रश्ची॑ ति॒रश्ची॑ ऐक्ष॒वी भव॑ति॒ भव॑ त्यैक्ष॒वी ति॒रश्ची᳚ । \newline
31. ऐ॒क्ष॒वी ति॒रश्ची॑ ति॒रश्ची॑ ऐक्ष॒वी ऐ᳚क्ष॒वी ति॒रश्ची᳚ प्र॒जाप॑तेः प्र॒जाप॑ते स्ति॒रश्ची॑ ऐक्ष॒वी ऐ᳚क्ष॒वी ति॒रश्ची᳚ प्र॒जाप॑तेः । \newline
32. ऐ॒क्ष॒वी इत्यै᳚क्ष॒वी । \newline
33. ति॒रश्ची᳚ प्र॒जाप॑तेः प्र॒जाप॑ते स्ति॒रश्ची॑ ति॒रश्ची᳚ प्र॒जाप॑ते रे॒वैव प्र॒जाप॑ते स्ति॒रश्ची॑ ति॒रश्ची᳚ प्र॒जाप॑ते रे॒व । \newline
34. ति॒रश्ची॒ इति॑ ति॒रश्ची᳚ । \newline
35. प्र॒जाप॑ते रे॒वैव प्र॒जाप॑तेः प्र॒जाप॑ते रे॒व तत् तदे॒व प्र॒जाप॑तेः प्र॒जाप॑ते रे॒व तत् । \newline
36. प्र॒जाप॑ते॒रिति॑ प्र॒जा - प॒तेः॒ । \newline
37. ए॒व तत् तदे॒ वैव तच् चक्षु॒ श्चक्षु॒ स्तदे॒ वैव तच् चक्षुः॑ । \newline
38. तच् चक्षु॒ श्चक्षु॒ स्तत् तच् चक्षुः॒ सꣳ सम् चक्षु॒ स्तत् तच् चक्षुः॒ सम् । \newline
39. चक्षुः॒ सꣳ सम् चक्षु॒ श्चक्षुः॒ सम् भ॑रति भरति॒ सम् चक्षु॒ श्चक्षुः॒ सम् भ॑रति । \newline
40. सम् भ॑रति भरति॒ सꣳ सम् भ॑रति दे॒वा दे॒वा भ॑रति॒ सꣳ सम् भ॑रति दे॒वाः । \newline
41. भ॒र॒ति॒ दे॒वा दे॒वा भ॑रति भरति दे॒वा वै वै दे॒वा भ॑रति भरति दे॒वा वै । \newline
42. दे॒वा वै वै दे॒वा दे॒वा वै या या वै दे॒वा दे॒वा वै याः । \newline
43. वै या या वै वै या आहु॑ती॒ राहु॑ती॒र् या वै वै या आहु॑तीः । \newline
44. या आहु॑ती॒ राहु॑ती॒र् या या आहु॑ती॒ रजु॑हवु॒ रजु॑हवु॒ राहु॑ती॒र् या या आहु॑ती॒ रजु॑हवुः । \newline
45. आहु॑ती॒ रजु॑हवु॒ रजु॑हवु॒ राहु॑ती॒ राहु॑ती॒ रजु॑हवु॒ स्ता स्ता अजु॑हवु॒ राहु॑ती॒ राहु॑ती॒ रजु॑हवु॒ स्ताः । \newline
46. आहु॑ती॒रित्या - हु॒तीः॒ । \newline
47. अजु॑हवु॒ स्ता स्ता अजु॑हवु॒ रजु॑हवु॒ स्ता असु॑रा॒ असु॑रा॒ स्ता अजु॑हवु॒ रजु॑हवु॒ स्ता असु॑राः । \newline
48. ता असु॑रा॒ असु॑रा॒ स्ता स्ता असु॑रा नि॒ष्काव॑म् नि॒ष्काव॒ मसु॑रा॒ स्ता स्ता असु॑रा नि॒ष्काव᳚म् । \newline
49. असु॑रा नि॒ष्काव॑म् नि॒ष्काव॒ मसु॑रा॒ असु॑रा नि॒ष्काव॑ मादन् नादन् नि॒ष्काव॒ मसु॑रा॒ असु॑रा नि॒ष्काव॑ मादन्न् । \newline
50. नि॒ष्काव॑ मादन् नादन् नि॒ष्काव॑म् नि॒ष्काव॑ माद॒न् ते त आ॑दन् नि॒ष्काव॑म् नि॒ष्काव॑ माद॒न् ते । \newline
51. आ॒द॒न् ते त आ॑दन् नाद॒न् ते दे॒वा दे॒वा स्त आ॑दन् नाद॒न् ते दे॒वाः । \newline
52. ते दे॒वा दे॒वा स्ते ते दे॒वाः का᳚र्ष्म॒र्य॑म् कार्ष्म॒र्य॑म् दे॒वा स्ते ते दे॒वाः का᳚र्ष्म॒र्य᳚म् । \newline
53. दे॒वाः का᳚र्ष्म॒र्य॑म् कार्ष्म॒र्य॑म् दे॒वा दे॒वाः का᳚र्ष्म॒र्य॑ मपश्यन् नपश्यन् कार्ष्म॒र्य॑म् दे॒वा दे॒वाः का᳚र्ष्म॒र्य॑ मपश्यन्न् । \newline
54. का॒र्ष्म॒र्य॑ मपश्यन् नपश्यन् कार्ष्म॒र्य॑म् कार्ष्म॒र्य॑ मपश्यन् कर्म॒ण्यः॑ कर्म॒ण्यो॑ ऽपश्यन् कार्ष्म॒र्य॑म् कार्ष्म॒र्य॑ मपश्यन् कर्म॒ण्यः॑ । \newline
55. अ॒प॒श्य॒न् क॒र्म॒ण्यः॑ कर्म॒ण्यो॑ ऽपश्यन् नपश्यन् कर्म॒ण्यो॑ वै वै क॑र्म॒ण्यो॑ ऽपश्यन् नपश्यन् कर्म॒ण्यो॑ वै । \newline
56. क॒र्म॒ण्यो॑ वै वै क॑र्म॒ण्यः॑ कर्म॒ण्यो॑ वै कर्म॒ कर्म॒ वै क॑र्म॒ण्यः॑ कर्म॒ण्यो॑ वै कर्म॑ । \newline
57. वै कर्म॒ कर्म॒ वै वै कर्मै॑ने नैनेन॒ कर्म॒ वै वै कर्मै॑नेन । \newline
58. कर्मै॑ने नैनेन॒ कर्म॒ कर्मै॑ नेन कुर्वीत कुर्वीतैनेन॒ कर्म॒ कर्मै॑ नेन कुर्वीत । \newline
59. ए॒ने॒न॒ कु॒र्वी॒त॒ कु॒र्वी॒ तै॒ने॒ नै॒ने॒न॒ कु॒र्वी॒ते तीति॑ कुर्वी तैने नैनेन कुर्वी॒तेति॑ । \newline
60. कु॒र्वी॒ते तीति॑ कुर्वीत कुर्वी॒तेति॒ ते त इति॑ कुर्वीत कुर्वी॒तेति॒ ते । \newline
61. इति॒ ते त इतीति॒ ते का᳚र्ष्मर्य॒मया᳚न् कार्ष्मर्य॒मया॒न् त इतीति॒ ते का᳚र्ष्मर्य॒मयान्॑ । \newline
62. ते का᳚र्ष्मर्य॒मया᳚न् कार्ष्मर्य॒मया॒न् ते ते का᳚र्ष्मर्य॒मया᳚न् परि॒धीन् प॑रि॒धीन् का᳚र्ष्मर्य॒मया॒न् ते ते का᳚र्ष्मर्य॒मया᳚न् परि॒धीन् । \newline
63. का॒र्ष्म॒र्य॒मया᳚न् परि॒धीन् प॑रि॒धीन् का᳚र्ष्मर्य॒मया᳚न् कार्ष्मर्य॒मया᳚न् परि॒धी न॑कुर्वता कुर्वत परि॒धीन् का᳚र्ष्मर्य॒मया᳚न् कार्ष्मर्य॒मया᳚न् परि॒धी न॑कुर्वत । \newline
64. का॒र्ष्म॒र्य॒मया॒निति॑ कार्ष्मर्य - मयान्॑ । \newline
65. प॒रि॒धी न॑कुर्वता कुर्वत परि॒धीन् प॑रि॒धी न॑कुर्वत॒ तै स्तै र॑कुर्वत परि॒धीन् प॑रि॒धी न॑कुर्वत॒ तैः । \newline
66. प॒रि॒धीनिति॑ परि - धीन् । \newline
\pagebreak
\markright{ TS 6.2.1.6  \hfill https://www.vedavms.in \hfill}

\section{ TS 6.2.1.6 }

\textbf{TS 6.2.1.6 } \newline
\textbf{Samhita Paata} \newline

-कुर्वत॒ तैर्वै ते रक्षाꣳ॒॒स्यपा᳚घ्नत॒ यत् का᳚र्ष्मर्य॒मयाः᳚ परि॒धयो॒ भव॑न्ति॒ रक्ष॑सा॒मप॑हत्यै॒ सꣳ स्प॑र्.शयति॒ रक्ष॑सा॒मन॑न्व-वचाराय॒ न पु॒रस्ता॒त् परि॑ दधात्यादि॒त्यो ह्ये॑वोद्यन् पु॒रस्ता॒द्-रक्षाꣳ॑स्यप॒हन्त्यू॒र्द्ध्वे स॒मिधा॒वा द॑धात्यु॒परि॑ष्टादे॒व रक्षाꣳ॒॒स्यप॑ हन्ति॒ यजु॑षा॒ऽन्यां तू॒ष्णीम॒न्यां मि॑थुन॒त्वाय॒ द्वे आ द॑धाति द्वि॒पाद्-यज॑मानः॒ प्रति॑ष्ठित्यै ब्रह्मवा॒दिनो॑ वदन्त्य॒- [  ] \newline

\textbf{Pada Paata} \newline

अ॒कु॒र्व॒त॒ । तैः । वै । ते । रक्षाꣳ॑सि । अपेति॑ । अ॒घ्न॒त॒ । यत् । का॒र्ष्म॒र्य॒मया॒ इति॑ कार्ष्मर्य - मयाः᳚ । प॒रि॒धय॒ इति॑ परि - धयः॑ । भव॑न्ति । रक्ष॑साम् । अप॑हत्या॒ इत्यप॑ - ह॒त्यै॒ । समिति॑ । स्प॒र्॒.श॒य॒ति॒ । रक्ष॑साम् । अन॑न्ववचारा॒येत्यन॑नु - अ॒व॒चा॒रा॒य॒ । न । पु॒रस्ता᳚त् । परीति॑ । द॒धा॒ति॒ । आ॒दि॒त्यः । हि । ए॒व । उ॒द्यन्नित्यु॑त्- यन्न् । पु॒रस्ता᳚त् । रक्षाꣳ॑सि । अ॒प॒हन्तीत्य॑प - हन्ति॑ । ऊ॒द्‌र्ध्वे इति॑ । स॒मिधा॒विति॑ सं - इधौ᳚ । एति॑ । द॒धा॒ति॒ । उ॒परि॑ष्टात् । ए॒व । रक्षाꣳ॑सि । अपेति॑ । ह॒न्ति॒ । यजु॑षा । अ॒न्याम् । तू॒ष्णीम् । अ॒न्याम् । मि॒थु॒न॒त्वायेति॑ मिथुन - त्वाय॑ । द्वे इति॑ । एति॑ । द॒धा॒ति॒ । द्वि॒पादिति॑ द्वि - पात् । यज॑मानः । प्रति॑ष्ठित्या॒ इति॒ प्रति॑ - स्थि॒त्यै॒ । ब्र॒ह्म॒वा॒दिन॒ इति॑ ब्रह्म - वा॒दिनः॑ । व॒द॒न्ति॒ ।  \newline


\textbf{Krama Paata} \newline

अ॒कु॒र्व॒त॒ तैः । तैर् वै । वै ते । ते रक्षाꣳ॑सि । रक्षाꣳ॒॒स्यप॑ । अपा᳚घ्नत । अ॒घ्न॒त॒ यत् । यत् का᳚र्ष्मर्य॒मयाः᳚ । का॒र्ष्म॒र्य॒मयाः᳚ परि॒धयः॑ । का॒र्ष्म॒र्य॒मया॒ इति॑ कार्ष्मर्य - मयाः᳚ । प॒रि॒धयो॒ भव॑न्ति । प॒रि॒धय॒ इति॑ परि - धयः॑ । भव॑न्ति॒ रक्ष॑साम् । रक्ष॑सा॒मप॑हत्यै । अप॑हत्यै॒ सम् । अप॑हत्या॒ इत्यप॑ - ह॒त्यै॒ । सꣳ स्प॑र्.शयति । स्व॒र्.॒श॒य॒ति॒ रक्ष॑साम् । रक्ष॑सा॒मन॑न्ववचाराय । अन॑न्ववचाराय॒ न । अन॑न्ववचारा॒येत्यन॑नु - अ॒व॒चा॒रा॒य॒ । न पु॒रस्ता᳚त् । पु॒रस्ता॒त् परि॑ । परि॑ दधाति । द॒धा॒त्या॒दि॒त्यः । आ॒दि॒त्यो हि । ह्ये॑व । ए॒वोद्यन्न् । उ॒द्यन् पु॒रस्ता᳚त् । उ॒द्यन्नित्यु॑त् - यन्न् । पु॒रस्ता॒द् रक्षाꣳ॑सि । रक्षाꣳ॑स्यप॒हन्ति॑ । अ॒प॒हन्त्यू॒र्द्ध्वे । अ॒प॒हन्तीत्य॑प - हन्ति॑ । ऊ॒र्द्ध्वे स॒मिधौ᳚ । ऊ॒र्द्ध्वे इत्यू॒र्द्ध्वे । स॒मिधा॒वा । स॒मिधा॒विति॑ सम् - इधौ᳚ । आ द॑धाति । द॒धा॒त्यु॒परि॑ष्टात् । उ॒परि॑ष्टादे॒व । ए॒व रक्षाꣳ॑सि । रक्षाꣳ॒॒स्यप॑ । अप॑ हन्ति । ह॒न्ति॒ यजु॑षा । यजु॑षा॒ऽन्याम् । अ॒न्याम् तू॒ष्णीम् । तू॒ष्णीम॒न्याम् । अ॒न्याम् मि॑थुन॒त्वाय॑ । मि॒थु॒न॒त्वाय॒ द्वे । मि॒थु॒न॒त्वायेति॑ मिथुन - त्वाय॑ । द्वे आ । द्वे इति॒ द्वे । आ द॑धाति । द॒धा॒ति॒ द्वि॒पात् । द्वि॒पाद् यज॑मानः । द्वि॒पादिति॑ द्वि - पात् । यज॑मानः॒ प्रति॑ष्ठित्यै । प्रति॑ष्ठित्यै ब्रह्मवा॒दिनः॑ । प्रति॑ष्ठित्या॒ इति॒ प्रति॑ - स्थि॒त्यै॒ । ब्र॒ह्म॒वा॒दिनो॑ वदन्ति ( ) । ब्र॒ह्म॒वा॒दिन॒ इति॑ ब्रह्म - वा॒दिनः॑ । व॒द॒न्त्य॒ग्निः \newline

\textbf{Jatai Paata} \newline

1. अ॒कु॒र्व॒त॒ तै स्तै र॑कुर्वता कुर्वत॒ तैः । \newline
2. तैर् वै वै तै स्तैर् वै । \newline
3. वै ते ते वै वै ते । \newline
4. ते रक्षाꣳ॑सि॒ रक्षाꣳ॑सि॒ ते ते रक्षाꣳ॑सि । \newline
5. रक्षाꣳ॒॒ स्यपाप॒ रक्षाꣳ॑सि॒ रक्षाꣳ॒॒ स्यप॑ । \newline
6. अपा᳚ घ्नता घ्न॒ता पापा᳚ घ्नत । \newline
7. अ॒घ्न॒त॒ यद् यद॑घ्नता घ्नत॒ यत् । \newline
8. यत् का᳚र्ष्मर्य॒मयाः᳚ कार्ष्मर्य॒मया॒ यद् यत् का᳚र्ष्मर्य॒मयाः᳚ । \newline
9. का॒र्ष्म॒र्य॒मयाः᳚ परि॒धयः॑ परि॒धयः॑ कार्ष्मर्य॒मयाः᳚ कार्ष्मर्य॒मयाः᳚ परि॒धयः॑ । \newline
10. का॒र्ष्म॒र्य॒मया॒ इति॑ कार्ष्मर्य - मयाः᳚ । \newline
11. प॒रि॒धयो॒ भव॑न्ति॒ भव॑न्ति परि॒धयः॑ परि॒धयो॒ भव॑न्ति । \newline
12. प॒रि॒धय॒ इति॑ परि - धयः॑ । \newline
13. भव॑न्ति॒ रक्ष॑साꣳ॒॒ रक्ष॑सा॒म् भव॑न्ति॒ भव॑न्ति॒ रक्ष॑साम् । \newline
14. रक्ष॑सा॒ मप॑हत्या॒ अप॑हत्यै॒ रक्ष॑साꣳ॒॒ रक्ष॑सा॒ मप॑हत्यै । \newline
15. अप॑हत्यै॒ सꣳ स मप॑हत्या॒ अप॑हत्यै॒ सम् । \newline
16. अप॑हत्या॒ इत्यप॑ - ह॒त्यै॒ । \newline
17. सꣳ स्प॑र्.शयति स्पर्.शयति॒ सꣳ सꣳ स्प॑र्.शयति । \newline
18. स्प॒र्॒.श॒य॒ति॒ रक्ष॑साꣳ॒॒ रक्ष॑साꣳ स्पर्.शयति स्पर्.शयति॒ रक्ष॑साम् । \newline
19. रक्ष॑सा॒ मन॑न्ववचारा॒या न॑न्ववचाराय॒ रक्ष॑साꣳ॒॒ रक्ष॑सा॒ मन॑न्ववचाराय । \newline
20. अन॑न्ववचाराय॒ न नान॑न्ववचारा॒या न॑न्ववचाराय॒ न । \newline
21. अन॑न्ववचारा॒येत्यन॑नु - अ॒व॒चा॒रा॒य॒ । \newline
22. न पु॒रस्ता᳚त् पु॒रस्ता॒न् न न पु॒रस्ता᳚त् । \newline
23. पु॒रस्ता॒त् परि॒ परि॑ पु॒रस्ता᳚त् पु॒रस्ता॒त् परि॑ । \newline
24. परि॑ दधाति दधाति॒ परि॒ परि॑ दधाति । \newline
25. द॒धा॒ त्या॒दि॒त्य आ॑दि॒त्यो द॑धाति दधा त्यादि॒त्यः । \newline
26. आ॒दि॒त्यो हि ह्या॑दि॒त्य आ॑दि॒त्यो हि । \newline
27. ह्ये॑वैव हि ह्ये॑व । \newline
28. ए॒वोद्यन् नु॒द्यन् ने॒वैवोद्यन्न् । \newline
29. उ॒द्यन् पु॒रस्ता᳚त् पु॒रस्ता॑ दु॒द्यन् नु॒द्यन् पु॒रस्ता᳚त् । \newline
30. उ॒द्यन्नित्यु॑त् - यन्न् । \newline
31. पु॒रस्ता॒द् रक्षाꣳ॑सि॒ रक्षाꣳ॑सि पु॒रस्ता᳚त् पु॒रस्ता॒द् रक्षाꣳ॑सि । \newline
32. रक्षाꣳ॑ स्यप॒हन् त्य॑प॒हन्ति॒ रक्षाꣳ॑सि॒ रक्षाꣳ॑ स्यप॒हन्ति॑ । \newline
33. अ॒प॒हन् त्यू॒र्द्ध्वे ऊ॒र्द्ध्वे अ॑प॒हन् त्य॑प॒हन् त्यू॒र्द्ध्वे । \newline
34. अ॒प॒हन्तीत्य॑प - हन्ति॑ । \newline
35. ऊ॒र्द्ध्वे स॒मिधौ॑ स॒मिधा॑ वू॒र्द्ध्वे ऊ॒र्द्ध्वे स॒मिधौ᳚ । \newline
36. ऊ॒र्द्ध्वे इत्यू॒र्द्ध्वे । \newline
37. स॒मिधा॒ वा स॒मिधौ॑ स॒मिधा॒ वा । \newline
38. स॒मिधा॒विति॑ सं - इधौ᳚ । \newline
39. आ द॑धाति दधा॒ त्याद॑धाति । \newline
40. द॒धा॒ त्यु॒परि॑ष्टा दु॒परि॑ष्टाद् दधाति दधा त्यु॒परि॑ष्टात् । \newline
41. उ॒परि॑ष्टा दे॒वैवोपरि॑ष्टा दु॒परि॑ष्टा दे॒व । \newline
42. ए॒व रक्षाꣳ॑सि॒ रक्षाꣳ॑ स्ये॒वैव रक्षाꣳ॑सि । \newline
43. रक्षाꣳ॒॒ स्यपाप॒ रक्षाꣳ॑सि॒ रक्षाꣳ॒॒ स्यप॑ । \newline
44. अप॑ हन्ति ह॒न् त्यपाप॑ हन्ति । \newline
45. ह॒न्ति॒ यजु॑षा॒ यजु॑षा हन्ति हन्ति॒ यजु॑षा । \newline
46. यजु॑षा॒ ऽन्या म॒न्यां ॅयजु॑षा॒ यजु॑षा॒ ऽन्याम् । \newline
47. अ॒न्याम् तू॒ष्णीम् तू॒ष्णी म॒न्या म॒न्याम् तू॒ष्णीम् । \newline
48. तू॒ष्णी म॒न्या म॒न्याम् तू॒ष्णीम् तू॒ष्णी म॒न्याम् । \newline
49. अ॒न्याम् मि॑थुन॒त्वाय॑ मिथुन॒त्वाया॒ न्या म॒न्याम् मि॑थुन॒त्वाय॑ । \newline
50. मि॒थु॒न॒त्वाय॒ द्वे द्वे मि॑थुन॒त्वाय॑ मिथुन॒त्वाय॒ द्वे । \newline
51. मि॒थु॒न॒त्वायेति॑ मिथुन - त्वाय॑ । \newline
52. द्वे आ द्वे द्वे आ । \newline
53. द्वे इति॒ द्वे । \newline
54. आ द॑धाति दधा॒त्या द॑धाति । \newline
55. द॒धा॒ति॒ द्वि॒पाद् द्वि॒पाद् द॑धाति दधाति द्वि॒पात् । \newline
56. द्वि॒पाद् यज॑मानो॒ यज॑मानो द्वि॒पाद् द्वि॒पाद् यज॑मानः । \newline
57. द्वि॒पादिति॑ द्वि - पात् । \newline
58. यज॑मानः॒ प्रति॑ष्ठित्यै॒ प्रति॑ष्ठित्यै॒ यज॑मानो॒ यज॑मानः॒ प्रति॑ष्ठित्यै । \newline
59. प्रति॑ष्ठित्यै ब्रह्मवा॒दिनो᳚ ब्रह्मवा॒दिनः॒ प्रति॑ष्ठित्यै॒ प्रति॑ष्ठित्यै ब्रह्मवा॒दिनः॑ । \newline
60. प्रति॑ष्ठित्या॒ इति॒ प्रति॑ - स्थि॒त्यै॒ । \newline
61. ब्र॒ह्म॒वा॒दिनो॑ वदन्ति वदन्ति ब्रह्मवा॒दिनो᳚ ब्रह्मवा॒दिनो॑ वदन्ति । \newline
62. ब्र॒ह्म॒वा॒दिन॒ इति॑ ब्रह्म - वा॒दिनः॑ । \newline
63. व॒द॒ न्त्य॒ग्नि र॒ग्निर् व॑दन्ति वद न्त्य॒ग्निः । \newline

\textbf{Ghana Paata } \newline

1. अ॒कु॒र्व॒त॒ तै स्तै र॑कुर्वता कुर्वत॒ तैर् वै वै तै र॑कुर्वता कुर्वत॒ तैर् वै । \newline
2. तैर् वै वै तै स्तैर् वै ते ते वै तै स्तैर् वै ते । \newline
3. वै ते ते वै वै ते रक्षाꣳ॑सि॒ रक्षाꣳ॑सि॒ ते वै वै ते रक्षाꣳ॑सि । \newline
4. ते रक्षाꣳ॑सि॒ रक्षाꣳ॑सि॒ ते ते रक्षाꣳ॒॒ स्यपाप॒ रक्षाꣳ॑सि॒ ते ते रक्षाꣳ॒॒ स्यप॑ । \newline
5. रक्षाꣳ॒॒ स्यपाप॒ रक्षाꣳ॑सि॒ रक्षाꣳ॒॒ स्यपा᳚ घ्नता घ्न॒ताप॒ रक्षाꣳ॑सि॒ रक्षाꣳ॒॒ स्यपा᳚ घ्नत । \newline
6. अपा᳚ घ्नता घ्न॒ता पापा᳚ घ्नत॒ यद् यद॑ घ्न॒ता पापा᳚ घ्नत॒ यत् । \newline
7. अ॒घ्न॒त॒ यद् यद॑घ्नता घ्नत॒ यत् का᳚र्ष्मर्य॒मयाः᳚ कार्ष्मर्य॒मया॒ यद॑घ्नता घ्नत॒ यत् 
का᳚र्ष्मर्य॒मयाः᳚ । \newline
8. यत् का᳚र्ष्मर्य॒मयाः᳚ कार्ष्मर्य॒मया॒ यद् यत् का᳚र्ष्मर्य॒मयाः᳚ परि॒धयः॑ परि॒धयः॑ कार्ष्मर्य॒मया॒ यद् यत् का᳚र्ष्मर्य॒मयाः᳚ परि॒धयः॑ । \newline
9. का॒र्ष्म॒र्य॒मयाः᳚ परि॒धयः॑ परि॒धयः॑ कार्ष्मर्य॒मयाः᳚ कार्ष्मर्य॒मयाः᳚ परि॒धयो॒ भव॑न्ति॒ भव॑न्ति परि॒धयः॑ कार्ष्मर्य॒मयाः᳚ कार्ष्मर्य॒मयाः᳚ परि॒धयो॒ भव॑न्ति । \newline
10. का॒र्ष्म॒र्य॒मया॒ इति॑ कार्ष्मर्य - मयाः᳚ । \newline
11. प॒रि॒धयो॒ भव॑न्ति॒ भव॑न्ति परि॒धयः॑ परि॒धयो॒ भव॑न्ति॒ रक्ष॑साꣳ॒॒ रक्ष॑सा॒म् भव॑न्ति परि॒धयः॑ परि॒धयो॒ भव॑न्ति॒ रक्ष॑साम् । \newline
12. प॒रि॒धय॒ इति॑ परि - धयः॑ । \newline
13. भव॑न्ति॒ रक्ष॑साꣳ॒॒ रक्ष॑सा॒म् भव॑न्ति॒ भव॑न्ति॒ रक्ष॑सा॒ मप॑हत्या॒ अप॑हत्यै॒ रक्ष॑सा॒म् भव॑न्ति॒ भव॑न्ति॒ रक्ष॑सा॒ मप॑हत्यै । \newline
14. रक्ष॑सा॒ मप॑हत्या॒ अप॑हत्यै॒ रक्ष॑साꣳ॒॒ रक्ष॑सा॒ मप॑हत्यै॒ सꣳ स मप॑हत्यै॒ रक्ष॑साꣳ॒॒ रक्ष॑सा॒ मप॑हत्यै॒ सम् । \newline
15. अप॑हत्यै॒ सꣳ स मप॑हत्या॒ अप॑हत्यै॒ सꣳ स्प॑र्.शयति स्पर्.शयति॒ स मप॑हत्या॒ अप॑हत्यै॒ सꣳ स्प॑र्.शयति । \newline
16. अप॑हत्या॒ इत्यप॑ - ह॒त्यै॒ । \newline
17. सꣳ स्प॑र्.शयति स्पर्.शयति॒ सꣳ सꣳ स्प॑र्.शयति॒ रक्ष॑साꣳ॒॒ रक्ष॑साꣳ स्पर्.शयति॒ सꣳ सꣳ स्प॑र्.शयति॒ रक्ष॑साम् । \newline
18. स्प॒र्॒.श॒य॒ति॒ रक्ष॑साꣳ॒॒ रक्ष॑साꣳ स्पर्.शयति स्पर्.शयति॒ रक्ष॑सा॒ मन॑न्ववचारा॒या न॑न्ववचाराय॒ रक्ष॑साꣳ स्पर्.शयति स्पर्.शयति॒ रक्ष॑सा॒ मन॑न्ववचाराय । \newline
19. रक्ष॑सा॒ मन॑न्ववचारा॒या न॑न्ववचाराय॒ रक्ष॑साꣳ॒॒ रक्ष॑सा॒ मन॑न्ववचाराय॒ न नान॑ न्ववचाराय॒ रक्ष॑साꣳ॒॒ रक्ष॑सा॒ मन॑न्ववचाराय॒ न । \newline
20. अन॑न्ववचाराय॒ न नान॑न्ववचारा॒या न॑न्ववचाराय॒ न पु॒रस्ता᳚त् पु॒रस्ता॒न् नान॑न्ववचारा॒या न॑न्ववचाराय॒ न पु॒रस्ता᳚त् । \newline
21. अन॑न्ववचारा॒येत्यन॑नु - अ॒व॒चा॒रा॒य॒ । \newline
22. न पु॒रस्ता᳚त् पु॒रस्ता॒न् न न पु॒रस्ता॒त् परि॒ परि॑ पु॒रस्ता॒न् न न पु॒रस्ता॒त् परि॑ । \newline
23. पु॒रस्ता॒त् परि॒ परि॑ पु॒रस्ता᳚त् पु॒रस्ता॒त् परि॑ दधाति दधाति॒ परि॑ पु॒रस्ता᳚त् पु॒रस्ता॒त् परि॑ दधाति । \newline
24. परि॑ दधाति दधाति॒ परि॒ परि॑ दधा त्यादि॒त्य आ॑दि॒त्यो द॑धाति॒ परि॒ परि॑ दधा त्यादि॒त्यः । \newline
25. द॒धा॒ त्या॒दि॒त्य आ॑दि॒त्यो द॑धाति दधा त्यादि॒त्यो हि ह्या॑दि॒त्यो द॑धाति दधा त्यादि॒त्यो हि । \newline
26. आ॒दि॒त्यो हि ह्या॑दि॒त्य आ॑दि॒त्यो ह्ये॑वैव ह्या॑दि॒त्य आ॑दि॒त्यो ह्ये॑व । \newline
27. ह्ये॑वैव हि ह्ये॑वोद्यन् नु॒द्यन् ने॒व हि ह्ये॑वोद्यन्न् । \newline
28. ए॒वोद्यन् नु॒द्यन् ने॒वै वोद्यन् पु॒रस्ता᳚त् पु॒रस्ता॑ दु॒द्यन् ने॒वै वोद्यन् पु॒रस्ता᳚त् । \newline
29. उ॒द्यन् पु॒रस्ता᳚त् पु॒रस्ता॑ दु॒द्यन् नु॒द्यन् पु॒रस्ता॒द् रक्षाꣳ॑सि॒ रक्षाꣳ॑सि पु॒रस्ता॑ दु॒द्यन् नु॒द्यन् पु॒रस्ता॒द् रक्षाꣳ॑सि । \newline
30. उ॒द्यन्नित्यु॑त् - यन्न् । \newline
31. पु॒रस्ता॒द् रक्षाꣳ॑सि॒ रक्षाꣳ॑सि पु॒रस्ता᳚त् पु॒रस्ता॒द् रक्षाꣳ॑ स्यप॒हन् त्य॑प॒हन्ति॒ रक्षाꣳ॑सि पु॒रस्ता᳚त् पु॒रस्ता॒द् रक्षाꣳ॑ स्यप॒हन्ति॑ । \newline
32. रक्षाꣳ॑ स्यप॒हन् त्य॑प॒हन्ति॒ रक्षाꣳ॑सि॒ रक्षाꣳ॑ स्यप॒हन् त्यू॒र्द्ध्वे ऊ॒र्द्ध्वे अ॑प॒हन्ति॒ रक्षाꣳ॑सि॒ रक्षाꣳ॑ स्यप॒हन् त्यू॒र्द्ध्वे । \newline
33. अ॒प॒हन् त्यू॒र्द्ध्वे ऊ॒र्द्ध्वे अ॑प॒हन् त्य॑प॒हन् त्यू॒र्द्ध्वे स॒मिधौ॑ स॒मिधा॑ वू॒र्द्ध्वे अ॑प॒हन् त्य॑प॒हन् त्यू॒र्द्ध्वे स॒मिधौ᳚ । \newline
34. अ॒प॒हन्तीत्य॑प - हन्ति॑ । \newline
35. ऊ॒र्द्ध्वे स॒मिधौ॑ स॒मिधा॑ वू॒र्द्ध्वे ऊ॒र्द्ध्वे स॒मिधा॒ वा स॒मिधा॑ वू॒र्द्ध्वे ऊ॒र्द्ध्वे स॒मिधा॒ वा । \newline
36. ऊ॒र्द्ध्वे इत्यू॒र्द्ध्वे । \newline
37. स॒मिधा॒ वा स॒मिधौ॑ स॒मिधा॒ वा द॑धाति दधा॒त्या स॒मिधौ॑ स॒मिधा॒ वा द॑धाति । \newline
38. स॒मिधा॒विति॑ सं - इधौ᳚ । \newline
39. आ द॑धाति दधा॒त्या द॑धा त्यु॒परि॑ष्टा दु॒परि॑ष्टाद् दधा॒त्या द॑धा त्यु॒परि॑ष्टात् । \newline
40. द॒धा॒ त्यु॒परि॑ष्टा दु॒परि॑ष्टाद् दधाति दधा त्यु॒परि॑ष्टा दे॒वै वोपरि॑ष्टाद् दधाति दधा त्यु॒परि॑ष्टा दे॒व । \newline
41. उ॒परि॑ष्टा दे॒वै वोपरि॑ष्टा दु॒परि॑ष्टा दे॒व रक्षाꣳ॑सि॒ रक्षाꣳ॑ स्ये॒वोपरि॑ष्टा दु॒परि॑ष्टा दे॒व रक्षाꣳ॑सि । \newline
42. ए॒व रक्षाꣳ॑सि॒ रक्षाꣳ॑ स्ये॒वैव रक्षाꣳ॒॒ स्यपाप॒ रक्षाꣳ॑ स्ये॒वैव रक्षाꣳ॒॒ स्यप॑ । \newline
43. रक्षाꣳ॒॒ स्यपाप॒ रक्षाꣳ॑सि॒ रक्षाꣳ॒॒ स्यप॑ हन्ति ह॒न्त्यप॒ रक्षाꣳ॑सि॒ रक्षाꣳ॒॒ स्यप॑ हन्ति । \newline
44. अप॑ हन्ति ह॒न्त्य पाप॑ हन्ति॒ यजु॑षा॒ यजु॑षा ह॒न्त्य पाप॑ हन्ति॒ यजु॑षा । \newline
45. ह॒न्ति॒ यजु॑षा॒ यजु॑षा हन्ति हन्ति॒ यजु॑षा॒ ऽन्या म॒न्यां ॅयजु॑षा हन्ति हन्ति॒ यजु॑षा॒ ऽन्याम् । \newline
46. यजु॑षा॒ ऽन्या म॒न्यां ॅयजु॑षा॒ यजु॑षा॒ ऽन्याम् तू॒ष्णीम् तू॒ष्णी म॒न्यां ॅयजु॑षा॒ यजु॑षा॒ ऽन्याम् तू॒ष्णीम् । \newline
47. अ॒न्याम् तू॒ष्णीम् तू॒ष्णी म॒न्या म॒न्याम् तू॒ष्णी म॒न्या म॒न्याम् तू॒ष्णी म॒न्या म॒न्याम् तू॒ष्णी म॒न्याम् । \newline
48. तू॒ष्णी म॒न्या म॒न्याम् तू॒ष्णीम् तू॒ष्णी म॒न्याम् मि॑थुन॒त्वाय॑ मिथुन॒त्वाया॒ न्याम् तू॒ष्णीम् तू॒ष्णी म॒न्याम् मि॑थुन॒त्वाय॑ । \newline
49. अ॒न्याम् मि॑थुन॒त्वाय॑ मिथुन॒त्वा या॒न्या म॒न्याम् मि॑थुन॒त्वाय॒ द्वे द्वे मि॑थुन॒त्वा या॒न्या म॒न्याम् मि॑थुन॒त्वाय॒ द्वे । \newline
50. मि॒थु॒न॒त्वाय॒ द्वे द्वे मि॑थुन॒त्वाय॑ मिथुन॒त्वाय॒ द्वे आ द्वे मि॑थुन॒त्वाय॑ मिथुन॒त्वाय॒ द्वे आ । \newline
51. मि॒थु॒न॒त्वायेति॑ मिथुन - त्वाय॑ । \newline
52. द्वे आ द्वे द्वे आ द॑धाति दधा॒त्या द्वे द्वे आ द॑धाति । \newline
53. द्वे इति॒ द्वे । \newline
54. आ द॑धाति दधा॒त्या द॑धाति द्वि॒पाद् द्वि॒पाद् द॑धा॒त्या द॑धाति द्वि॒पात् । \newline
55. द॒धा॒ति॒ द्वि॒पाद् द्वि॒पाद् द॑धाति दधाति द्वि॒पाद् यज॑मानो॒ यज॑मानो द्वि॒पाद् द॑धाति दधाति द्वि॒पाद् यज॑मानः । \newline
56. द्वि॒पाद् यज॑मानो॒ यज॑मानो द्वि॒पाद् द्वि॒पाद् यज॑मानः॒ प्रति॑ष्ठित्यै॒ प्रति॑ष्ठित्यै॒ यज॑मानो द्वि॒पाद् द्वि॒पाद् यज॑मानः॒ प्रति॑ष्ठित्यै । \newline
57. द्वि॒पादिति॑ द्वि - पात् । \newline
58. यज॑मानः॒ प्रति॑ष्ठित्यै॒ प्रति॑ष्ठित्यै॒ यज॑मानो॒ यज॑मानः॒ प्रति॑ष्ठित्यै ब्रह्मवा॒दिनो᳚ ब्रह्मवा॒दिनः॒ प्रति॑ष्ठित्यै॒ यज॑मानो॒ यज॑मानः॒ प्रति॑ष्ठित्यै ब्रह्मवा॒दिनः॑ । \newline
59. प्रति॑ष्ठित्यै ब्रह्मवा॒दिनो᳚ ब्रह्मवा॒दिनः॒ प्रति॑ष्ठित्यै॒ प्रति॑ष्ठित्यै ब्रह्मवा॒दिनो॑ वदन्ति वदन्ति ब्रह्मवा॒दिनः॒ प्रति॑ष्ठित्यै॒ प्रति॑ष्ठित्यै ब्रह्मवा॒दिनो॑ वदन्ति । \newline
60. प्रति॑ष्ठित्या॒ इति॒ प्रति॑ - स्थि॒त्यै॒ । \newline
61. ब्र॒ह्म॒वा॒दिनो॑ वदन्ति वदन्ति ब्रह्मवा॒दिनो᳚ ब्रह्मवा॒दिनो॑ वदन् त्य॒ग्नि र॒ग्निर् व॑दन्ति ब्रह्मवा॒दिनो᳚ ब्रह्मवा॒दिनो॑ वदन् त्य॒ग्निः । \newline
62. ब्र॒ह्म॒वा॒दिन॒ इति॑ ब्रह्म - वा॒दिनः॑ । \newline
63. व॒द॒न् त्य॒ग्नि र॒ग्निर् व॑दन्ति वदन् त्य॒ग्नि श्च॑ चा॒ग्निर् व॑दन्ति वदन् त्य॒ग्नि श्च॑ । \newline
\pagebreak
\markright{ TS 6.2.1.7  \hfill https://www.vedavms.in \hfill}

\section{ TS 6.2.1.7 }

\textbf{TS 6.2.1.7 } \newline
\textbf{Samhita Paata} \newline

-ग्निश्च॒ वा ए॒तौ सोम॑श्च क॒था सोमा॑याऽऽ*ति॒थ्यं क्रि॒यते॒ नाग्नय॒ इति॒ यद॒ग्नाव॒ग्निं म॑थि॒त्वा प्र॒हर॑ति॒ तेनै॒वाग्नय॑ आति॒थ्यं क्रि॑य॒ते ऽथो॒ खल्वा॑हुर॒ग्निः सर्वा॑ दे॒वता॒ इति॒ यद्ध॒विरा॒साद्या॒ग्निं मन्थ॑ति ह॒व्यायै॒वाऽऽ*स॑न्नाय॒ सर्वा॑ दे॒वता॑ जनयति ॥ \newline

\textbf{Pada Paata} \newline

अ॒ग्निः । च॒ । वै । ए॒तौ । सोमः॑ । च॒ । क॒था । सोमा॑य । आ॒ति॒थ्यम् । क्रि॒यते᳚ । न । अ॒ग्नये᳚ । इति॑ । यत् । अ॒ग्नौ । अ॒ग्निम् । म॒थि॒त्वा । प्र॒हर॒तीति॑ प्र - हर॑ति । तेन॑ । ए॒व । अ॒ग्नये᳚ । आ॒ति॒थ्यम् । क्रि॒य॒ते॒ । अथो॒ इति॑ । खलु॑ । आ॒हुः॒ । अ॒ग्निः । सर्वाः᳚ । दे॒वताः᳚ । इति॑ । यत् । ह॒विः । आ॒साद्येत्या᳚ - साद्य॑ । अ॒ग्निम् । मन्थ॑ति । ह॒व्याय॑ । ए॒व । आस॑न्ना॒येत्या - स॒न्ना॒य॒ । सर्वाः᳚ । दे॒वताः᳚ । ज॒न॒य॒ति॒ ॥  \newline


\textbf{Krama Paata} \newline

अ॒ग्निश्च॑ । च॒ वै । वा ए॒तौ । ए॒तौ सोमः॑ । सोम॑श्च । च॒ क॒था । क॒था सोमा॑य । सोमा॑याति॒थ्यम् । आ॒ति॒थ्यम् क्रि॒यते᳚ । क्रि॒यते॒ न । नाग्नये᳚ । अ॒ग्नय॒ इति॑ । इति॒ यत् । यद॒ग्नौ । अ॒ग्नाव॒ग्निम् । अ॒ग्निम् म॑थि॒त्वा । म॒थि॒त्वा प्र॒हर॑ति । प्र॒हर॑ति॒ तेन॑ । प्र॒हर॒तीति॑ प्र - हर॑ति । तेनै॒व । ए॒वाग्नये᳚ । अ॒ग्नय॑ आति॒थ्यम् । आ॒ति॒थ्यम् क्रि॑यते । क्रि॒य॒तेऽथो᳚ । अथो॒ खलु॑ । अथो॒ इत्यथो᳚ । खल्वा॑हुः । आ॒हु॒र॒ग्निः । अ॒ग्निः सर्वाः᳚ । सर्वा॑ दे॒वताः᳚ । दे॒वता॒ इति॑ । इति॒ यत् । यद्‍ध॒विः । ह॒विरा॒साद्य॑ । आ॒साद्या॒ग्निम् । आ॒साद्येत्या᳚ - साद्य॑ । अ॒ग्निम् मन्थ॑ति । मन्थ॑ति ह॒व्याय॑ । ह॒व्यायै॒व । ए॒वास॑न्नाय । आस॑न्नाय॒ सर्वाः᳚ । आस॑न्ना॒येत्या - स॒न्ना॒य॒ । सर्वा॑ दे॒वताः᳚ । दे॒वता॑ जनयति । ज॒न॒य॒तीति॑ जनयति । \newline

\textbf{Jatai Paata} \newline

1. अ॒ग्निश्च॑ चा॒ग्नि र॒ग्निश्च॑ । \newline
2. च॒ वै वै च॑ च॒ वै । \newline
3. वा ए॒ता वे॒तौ वै वा ए॒तौ । \newline
4. ए॒तौ सोमः॒ सोम॑ ए॒ता वे॒तौ सोमः॑ । \newline
5. सोम॑श्च च॒ सोमः॒ सोम॑श्च । \newline
6. च॒ क॒था क॒था च॑ च क॒था । \newline
7. क॒था सोमा॑य॒ सोमा॑य क॒था क॒था सोमा॑य । \newline
8. सोमा॑याति॒थ्य मा॑ति॒थ्यꣳ सोमा॑य॒ सोमा॑याति॒थ्यम् । \newline
9. आ॒ति॒थ्यम् क्रि॒यते᳚ क्रि॒यत॑ आति॒थ्य मा॑ति॒थ्यम् क्रि॒यते᳚ । \newline
10. क्रि॒यते॒ न न क्रि॒यते᳚ क्रि॒यते॒ न । \newline
11. नाग्नये॒ ऽग्नये॒ न नाग्नये᳚ । \newline
12. अ॒ग्नय॒ इती त्य॒ग्नये॒ ऽग्नय॒ इति॑ । \newline
13. इति॒ यद् यदितीति॒ यत् । \newline
14. यद॒ग्ना व॒ग्नौ यद् यद॒ग्नौ । \newline
15. अ॒ग्ना व॒ग्नि म॒ग्नि म॒ग्ना व॒ग्ना व॒ग्निम् । \newline
16. अ॒ग्निम् म॑थि॒त्वा म॑थि॒त्वा ऽग्नि म॒ग्निम् म॑थि॒त्वा । \newline
17. म॒थि॒त्वा प्र॒हर॑ति प्र॒हर॑ति मथि॒त्वा म॑थि॒त्वा प्र॒हर॑ति । \newline
18. प्र॒हर॑ति॒ तेन॒ तेन॑ प्र॒हर॑ति प्र॒हर॑ति॒ तेन॑ । \newline
19. प्र॒हर॒तीति॑ प्र - हर॑ति । \newline
20. तेनै॒वैव तेन॒ तेनै॒व । \newline
21. ए॒वाग्नये॒ ऽग्नय॑ ए॒वैवाग्नये᳚ । \newline
22. अ॒ग्नय॑ आति॒थ्य मा॑ति॒थ्य म॒ग्नये॒ ऽग्नय॑ आति॒थ्यम् । \newline
23. आ॒ति॒थ्यम् क्रि॑यते क्रियत आति॒थ्य मा॑ति॒थ्यम् क्रि॑यते । \newline
24. क्रि॒य॒ते ऽथो॒ अथो᳚ क्रियते क्रिय॒ते ऽथो᳚ । \newline
25. अथो॒ खलु॒ खल्वथो॒ अथो॒ खलु॑ । \newline
26. अथो॒ इत्यथो᳚ । \newline
27. खल्वा॑हु राहुः॒ खलु॒ खल्वा॑हुः । \newline
28. आ॒हु॒ र॒ग्नि र॒ग्नि रा॑हु राहु र॒ग्निः । \newline
29. अ॒ग्निः सर्वाः॒ सर्वा॑ अ॒ग्नि र॒ग्निः सर्वाः᳚ । \newline
30. सर्वा॑ दे॒वता॑ दे॒वताः॒ सर्वाः॒ सर्वा॑ दे॒वताः᳚ । \newline
31. दे॒वता॒ इतीति॑ दे॒वता॑ दे॒वता॒ इति॑ । \newline
32. इति॒ यद् यदितीति॒ यत् । \newline
33. यद्ध॒विर्. ह॒विर् यद् यद्ध॒विः । \newline
34. ह॒वि रा॒साद्या॒ साद्य॑ ह॒विर्. ह॒वि रा॒साद्य॑ । \newline
35. आ॒साद्या॒ ग्नि म॒ग्नि मा॒साद्या॒ साद्या॒ ग्निम् । \newline
36. आ॒साद्येत्या᳚ - साद्य॑ । \newline
37. अ॒ग्निम् मन्थ॑ति॒ मन्थ॑ त्य॒ग्नि म॒ग्निम् मन्थ॑ति । \newline
38. मन्थ॑ति ह॒व्याय॑ ह॒व्याय॒ मन्थ॑ति॒ मन्थ॑ति ह॒व्याय॑ । \newline
39. ह॒व्या यै॒वैव ह॒व्याय॑ ह॒व्यायै॒व । \newline
40. ए॒वा स॑न्ना॒या स॑न्ना यै॒वैवास॑न्नाय । \newline
41. आस॑न्नाय॒ सर्वाः॒ सर्वा॒ आस॑न्ना॒या स॑न्नाय॒ सर्वाः᳚ । \newline
42. आस॑न्ना॒येत्या - स॒न्ना॒य॒ । \newline
43. सर्वा॑ दे॒वता॑ दे॒वताः॒ सर्वाः॒ सर्वा॑ दे॒वताः᳚ । \newline
44. दे॒वता॑ जनयति जनयति दे॒वता॑ दे॒वता॑ जनयति । \newline
45. ज॒न॒य॒तीति॑ जनयति । \newline

\textbf{Ghana Paata } \newline

1. अ॒ग्नि श्च॑ चा॒ग्नि र॒ग्नि श्च॒ वै वै चा॒ग्नि र॒ग्नि श्च॒ वै । \newline
2. च॒ वै वै च॑ च॒ वा ए॒ता वे॒तौ वै च॑ च॒ वा ए॒तौ । \newline
3. वा ए॒ता वे॒तौ वै वा ए॒तौ सोमः॒ सोम॑ ए॒तौ वै वा ए॒तौ सोमः॑ । \newline
4. ए॒तौ सोमः॒ सोम॑ ए॒ता वे॒तौ सोम॑ श्च च॒ सोम॑ ए॒ता वे॒तौ सोम॑ श्च । \newline
5. सोम॑ श्च च॒ सोमः॒ सोम॑ श्च क॒था क॒था च॒ सोमः॒ सोम॑ श्च क॒था । \newline
6. च॒ क॒था क॒था च॑ च क॒था सोमा॑य॒ सोमा॑य क॒था च॑ च क॒था सोमा॑य । \newline
7. क॒था सोमा॑य॒ सोमा॑य क॒था क॒था सोमा॑ याति॒थ्य मा॑ति॒थ्यꣳ सोमा॑य क॒था क॒था सोमा॑ याति॒थ्यम् । \newline
8. सोमा॑ याति॒थ्य मा॑ति॒थ्यꣳ सोमा॑य॒ सोमा॑ याति॒थ्यम् क्रि॒यते᳚ क्रि॒यत॑ आति॒थ्यꣳ सोमा॑य॒ सोमा॑ याति॒थ्यम् क्रि॒यते᳚ । \newline
9. आ॒ति॒थ्यम् क्रि॒यते᳚ क्रि॒यत॑ आति॒थ्य मा॑ति॒थ्यम् क्रि॒यते॒ न न क्रि॒यत॑ आति॒थ्य मा॑ति॒थ्यम् क्रि॒यते॒ न । \newline
10. क्रि॒यते॒ न न क्रि॒यते᳚ क्रि॒यते॒ नाग्नये॒ ऽग्नये॒ न क्रि॒यते᳚ क्रि॒यते॒ नाग्नये᳚ । \newline
11. नाग्नये॒ ऽग्नये॒ न नाग्नय॒ इती त्य॒ग्नये॒ न नाग्नय॒ इति॑ । \newline
12. अ॒ग्नय॒ इती त्य॒ग्नये॒ ऽग्नय॒ इति॒ यद् यदि त्य॒ग्नये॒ ऽग्नय॒ इति॒ यत् । \newline
13. इति॒ यद् यदि तीति॒ यद॒ग्ना व॒ग्नौ यदि तीति॒ यद॒ग्नौ । \newline
14. यद॒ग्ना व॒ग्नौ यद् यद॒ग्ना व॒ग्नि म॒ग्नि म॒ग्नौ यद् यद॒ग्ना व॒ग्निम् । \newline
15. अ॒ग्ना व॒ग्नि म॒ग्नि म॒ग्ना व॒ग्ना व॒ग्निम् म॑थि॒त्वा म॑थि॒त्वा ऽग्नि म॒ग्ना व॒ग्ना व॒ग्निम् म॑थि॒त्वा । \newline
16. अ॒ग्निम् म॑थि॒त्वा म॑थि॒त्वा ऽग्नि म॒ग्निम् म॑थि॒त्वा प्र॒हर॑ति प्र॒हर॑ति मथि॒त्वा ऽग्नि म॒ग्निम् म॑थि॒त्वा प्र॒हर॑ति । \newline
17. म॒थि॒त्वा प्र॒हर॑ति प्र॒हर॑ति मथि॒त्वा म॑थि॒त्वा प्र॒हर॑ति॒ तेन॒ तेन॑ प्र॒हर॑ति मथि॒त्वा म॑थि॒त्वा प्र॒हर॑ति॒ तेन॑ । \newline
18. प्र॒हर॑ति॒ तेन॒ तेन॑ प्र॒हर॑ति प्र॒हर॑ति॒ तेनै॒ वैव तेन॑ प्र॒हर॑ति प्र॒हर॑ति॒ तेनै॒व । \newline
19. प्र॒हर॒तीति॑ प्र - हर॑ति । \newline
20. तेनै॒ वैव तेन॒ तेनै॒ वाग्नये॒ ऽग्नय॑ ए॒व तेन॒ तेनै॒ वाग्नये᳚ । \newline
21. ए॒वाग्नये॒ ऽग्नय॑ ए॒वै वाग्नय॑ आति॒थ्य मा॑ति॒थ्य म॒ग्नय॑ ए॒वै वाग्नय॑ आति॒थ्यम् । \newline
22. अ॒ग्नय॑ आति॒थ्य मा॑ति॒थ्य म॒ग्नये॒ ऽग्नय॑ आति॒थ्यम् क्रि॑यते क्रियत आति॒थ्य म॒ग्नये॒ ऽग्नय॑ आति॒थ्यम् क्रि॑यते । \newline
23. आ॒ति॒थ्यम् क्रि॑यते क्रियत आति॒थ्य मा॑ति॒थ्यम् क्रि॑य॒ते ऽथो॒ अथो᳚ क्रियत आति॒थ्य मा॑ति॒थ्यम् क्रि॑य॒ते ऽथो᳚ । \newline
24. क्रि॒य॒ते ऽथो॒ अथो᳚ क्रियते क्रिय॒ते ऽथो॒ खलु॒ खल्वथो᳚ क्रियते क्रिय॒ते ऽथो॒ खलु॑ । \newline
25. अथो॒ खलु॒ खल्वथो॒ अथो॒ खल्वा॑हु राहुः॒ खल्वथो॒ अथो॒ खल्वा॑हुः । \newline
26. अथो॒ इत्यथो᳚ । \newline
27. खल्वा॑हु राहुः॒ खलु॒ खल्वा॑हु र॒ग्नि र॒ग्नि रा॑हुः॒ खलु॒ खल्वा॑हु र॒ग्निः । \newline
28. आ॒हु॒ र॒ग्नि र॒ग्नि रा॑हु राहु र॒ग्निः सर्वाः॒ सर्वा॑ अ॒ग्नि रा॑हु राहु र॒ग्निः सर्वाः᳚ । \newline
29. अ॒ग्निः सर्वाः॒ सर्वा॑ अ॒ग्नि र॒ग्निः सर्वा॑ दे॒वता॑ दे॒वताः॒ सर्वा॑ अ॒ग्नि र॒ग्निः सर्वा॑ दे॒वताः᳚ । \newline
30. सर्वा॑ दे॒वता॑ दे॒वताः॒ सर्वाः॒ सर्वा॑ दे॒वता॒ इतीति॑ दे॒वताः॒ सर्वाः॒ सर्वा॑ दे॒वता॒ इति॑ । \newline
31. दे॒वता॒ इतीति॑ दे॒वता॑ दे॒वता॒ इति॒ यद् यदिति॑ दे॒वता॑ दे॒वता॒ इति॒ यत् । \newline
32. इति॒ यद् यदितीति॒ यद्ध॒विर्. ह॒विर् यदितीति॒ यद्ध॒विः । \newline
33. यद्ध॒विर्. ह॒विर् यद् यद्ध॒वि रा॒साद्या॒ साद्य॑ ह॒विर् यद् यद्ध॒वि रा॒साद्य॑ । \newline
34. ह॒वि रा॒साद्या॒ साद्य॑ ह॒विर्. ह॒वि रा॒साद्या॒ ग्नि म॒ग्नि मा॒साद्य॑ ह॒विर्. ह॒वि रा॒साद्या॒ ग्निम् । \newline
35. आ॒साद्या॒ ग्नि म॒ग्नि मा॒साद्या॒ साद्या॒ ग्निम् मन्थ॑ति॒ मन्थ॑ त्य॒ग्नि मा॒साद्या॒ साद्या॒ ग्निम् मन्थ॑ति । \newline
36. आ॒साद्येत्या᳚ - साद्य॑ । \newline
37. अ॒ग्निम् मन्थ॑ति॒ मन्थ॑ त्य॒ग्नि म॒ग्निम् मन्थ॑ति ह॒व्याय॑ ह॒व्याय॒ मन्थ॑ त्य॒ग्नि म॒ग्निम् मन्थ॑ति ह॒व्याय॑ । \newline
38. मन्थ॑ति ह॒व्याय॑ ह॒व्याय॒ मन्थ॑ति॒ मन्थ॑ति ह॒व्या यै॒वैव ह॒व्याय॒ मन्थ॑ति॒ मन्थ॑ति ह॒व्या यै॒व । \newline
39. ह॒व्या यै॒वैव ह॒व्याय॑ ह॒व्या यै॒वास॑न्ना॒या स॑न्ना यै॒व ह॒व्याय॑ ह॒व्या यै॒वास॑न्नाय । \newline
40. ए॒वा स॑न्ना॒या स॑न्ना यै॒वैवा स॑न्नाय॒ सर्वाः॒ सर्वा॒ आस॑न्ना यै॒वैवा स॑न्नाय॒ सर्वाः᳚ । \newline
41. आस॑न्नाय॒ सर्वाः॒ सर्वा॒ आस॑न्ना॒या स॑न्नाय॒ सर्वा॑ दे॒वता॑ दे॒वताः॒ सर्वा॒ आस॑न्ना॒या स॑न्नाय॒ सर्वा॑ दे॒वताः᳚ । \newline
42. आस॑न्ना॒येत्या - स॒न्ना॒य॒ । \newline
43. सर्वा॑ दे॒वता॑ दे॒वताः॒ सर्वाः॒ सर्वा॑ दे॒वता॑ जनयति जनयति दे॒वताः॒ सर्वाः॒ सर्वा॑ दे॒वता॑ जनयति । \newline
44. दे॒वता॑ जनयति जनयति दे॒वता॑ दे॒वता॑ जनयति । \newline
45. ज॒न॒य॒तीति॑ जनयति । \newline
\pagebreak
\markright{ TS 6.2.2.1  \hfill https://www.vedavms.in \hfill}

\section{ TS 6.2.2.1 }

\textbf{TS 6.2.2.1 } \newline
\textbf{Samhita Paata} \newline

दे॒वा॒सु॒राः संॅय॑त्ता आस॒न् ते दे॒वा मि॒थो विप्रि॑या आस॒न् ते᳚ऽ(1॒)न्यो᳚ऽन्यस्मै॒ ज्यैष्ठ्या॒याति॑ष्ठमानाः पञ्च॒धा व्य॑क्रामन्न॒ग्निर्वसु॑भिः॒ सोमो॑ रु॒द्रैरिन्द्रो॑ म॒रुद्भि॒-र्वरु॑ण आदि॒त्यै-र्बृह॒स्पति॒-र्विश्वै᳚र्दे॒वैस्ते॑ ऽमन्य॒न्तासु॑रेभ्यो॒ वा इ॒दं भ्रातृ॑व्येभ्यो रद्ध्यामो॒ यन्मि॒थो विप्रि॑याः॒ स्मो या न॑ इ॒माः प्रि॒यास्त॒नुव॒स्ताः स॒मव॑द्यामहै॒ ताभ्यः॒ स निर्.ऋ॑च्छा॒द्यो- [  ] \newline

\textbf{Pada Paata} \newline

दे॒वा॒सु॒रा इति॑ देव - अ॒सु॒राः । संॅय॑त्ता॒ इति॒ सं-य॒त्ताः॒ । आ॒स॒न्न् । ते । दे॒वाः । मि॒थः । विप्रि॑या॒ इति॒ वि - प्रि॒याः॒ । आ॒स॒न्न् । ते । अ॒न्यः । अ॒न्यस्मै᳚ । ज्यैष्ठ्या॑य । अति॑ष्ठमानाः । प॒ञ्च॒धेति॑ पञ्च-धा । वीति॑ । अ॒क्रा॒म॒न्न् । अ॒ग्निः । वसु॑भि॒रिति॒ वसु॑ - भिः॒ । सोमः॑ । रु॒द्रैः । इन्द्रः॑ । म॒रुद्भि॒रिति॑ म॒रुत् - भिः॒ । वरु॑णः । आ॒दि॒त्यैः । बृह॒स्पतिः॑ । विश्वैः᳚ । दे॒वैः । ते । अ॒म॒न्य॒न्त॒ । असु॑रेभ्यः । वै । इ॒दम् । भ्रातृ॑व्येभ्यः । र॒द्ध्या॒मः॒ । यत् । मि॒थः । विप्रि॑या॒ इति॒ वि - प्रि॒याः॒ । स्मः । याः । नः॒ । इ॒माः । प्रि॒याः । त॒नुवः॑ । ताः । स॒मव॑द्यामहा॒ इति॑ सं - अव॑द्यामहै । ताभ्यः॑ । सः । निरिति॑ । ऋ॒च्छा॒त् । यः ।  \newline


\textbf{Krama Paata} \newline

दे॒वा॒सु॒राः सम्ॅय॑त्ताः । दे॒वा॒सु॒रा इति॑ देव - अ॒सु॒राः । सम्ॅय॑त्ता आसन्न् । सम्ॅय॑त्ता॒ इति॒ सम् - य॒त्ताः॒ । आ॒स॒न् ते । ते दे॒वाः । दे॒वा मि॒थः । मि॒थो विप्रि॑याः । विप्रि॑या आसन्न् । विप्रि॑या॒ इति॒ वि - प्रि॒याः॒ । आ॒स॒न् ते । ते᳚ऽन्यः । अ॒न्यो᳚ऽन्यस्मै᳚ । अ॒न्यस्मै॒ ज्यैष्ठ्‍या॑य । ज्यैष्ठ्‍या॒याति॑ष्ठमानाः । अति॑ष्ठमानाः पञ्च॒धा । प॒ञ्च॒धा वि । प॒ञ्च॒धेति॑ पञ्च - धा । व्य॑क्रामन्न् । अ॒क्रा॒म॒न्न॒ग्निः । अ॒ग्निर् वसु॑भिः । वसु॑भिः॒ सोमः॑ । वसु॑भि॒रिति॒ वसु॑ - भिः॒ । सोमो॑ रु॒द्रैः । रु॒द्रैरिन्द्रः॑ । इन्द्रो॑ म॒रुद्‌भिः॑ । म॒रुद्‌भि॒र् वरु॑णः । म॒रुद्‌भि॒रिति॑ म॒रुत् - भिः॒ । वरु॑ण आदि॒त्यैः । आ॒दि॒त्यैर् बृह॒स्पतिः॑ । बृह॒स्पति॒र् विश्वैः᳚ । विश्वै᳚र् दे॒वैः । दे॒वैस्ते । ते॑ऽमन्यन्त । अ॒म॒न्य॒न्तासु॑रेभ्यः । असु॑रेभ्यो॒ वै । वा इ॒दम् । इ॒दम् भ्रातृ॑व्येभ्यः । भ्रातृ॑व्येभ्यो रद्ध्यामः । र॒द्ध्या॒मो॒ यत् । यन् मि॒थः । मि॒थो विप्रि॑याः । विप्रि॑याः॒ स्मः । विप्रि॑या॒ इति॒ वि - प्रि॒याः॒ । स्मो याः । या नः॑ । न॒ इ॒माः । इ॒माः प्रि॒याः । प्रि॒यास्त॒नुवः॑ । त॒नुव॒स्ताः । ताः स॒मव॑द्यामहै । स॒मव॑द्यामहै॒ ताभ्यः॑ । स॒मव॑द्यामहा॒ इति॑ सम् - अव॑द्यामहै । ताभ्यः॒ सः । स निः । निर्. ऋ॑च्छात् । ऋ॒च्छा॒द् यः । यो नः॑ \newline

\textbf{Jatai Paata} \newline

1. दे॒वा॒सु॒राः संॅय॑त्ताः॒ संॅय॑त्ता देवासु॒रा दे॑वासु॒राः संॅय॑त्ताः । \newline
2. दे॒वा॒सु॒रा इति॑ देव - अ॒सु॒राः । \newline
3. संॅय॑त्ता आसन् नास॒न् थ्संॅय॑त्ताः॒ संॅय॑त्ता आसन्न् । \newline
4. संॅय॑त्ता॒ इति॒ सं - य॒त्ताः॒ । \newline
5. आ॒स॒न् ते त आ॑सन् नास॒न् ते । \newline
6. ते दे॒वा दे॒वा स्ते ते दे॒वाः । \newline
7. दे॒वा मि॒थो मि॒थो दे॒वा दे॒वा मि॒थः । \newline
8. मि॒थो विप्रि॑या॒ विप्रि॑या मि॒थो मि॒थो विप्रि॑याः । \newline
9. विप्रि॑या आसन् नास॒न्॒. विप्रि॑या॒ विप्रि॑या आसन्न् । \newline
10. विप्रि॑या॒ इति॒ वि - प्रि॒याः॒ । \newline
11. आ॒स॒न् ते त आ॑सन् नास॒न् ते । \newline
12. ते᳚(1॒) ऽन्यो᳚ ऽन्य स्ते ते᳚ ऽन्यः । \newline
13. अ॒न्यो᳚ ऽन्यस्मा॑ अ॒न्यस्मा॑ अ॒न्यो᳚(1॒) ऽन्यो᳚ ऽन्यस्मै᳚ । \newline
14. अ॒न्यस्मै॒ ज्यैष्ठ्या॑य॒ ज्यैष्ठ्या॑या॒ न्यस्मा॑ अ॒न्यस्मै॒ ज्यैष्ठ्या॑य । \newline
15. ज्यैष्ठ्या॒या ति॑ष्ठमाना॒ अति॑ष्ठमाना॒ ज्यैष्ठ्या॑य॒ ज्यैष्ठ्या॒या ति॑ष्ठमानाः । \newline
16. अति॑ष्ठमानाः पञ्च॒धा प॑ञ्च॒धा ऽति॑ष्ठमाना॒ अति॑ष्ठमानाः पञ्च॒धा । \newline
17. प॒ञ्च॒धा वि वि प॑ञ्च॒धा प॑ञ्च॒धा वि । \newline
18. प॒ञ्च॒धेति॑ पञ्च - धा । \newline
19. व्य॑क्रामन् नक्राम॒न्॒. वि व्य॑क्रामन्न् । \newline
20. अ॒क्रा॒म॒न् न॒ग्नि र॒ग्नि र॑क्रामन् नक्रामन् न॒ग्निः । \newline
21. अ॒ग्निर् वसु॑भि॒र् वसु॑भि र॒ग्नि र॒ग्निर् वसु॑भिः । \newline
22. वसु॑भिः॒ सोमः॒ सोमो॒ वसु॑भि॒र् वसु॑भिः॒ सोमः॑ । \newline
23. वसु॑भि॒रिति॒ वसु॑ - भिः॒ । \newline
24. सोमो॑ रु॒द्रै रु॒द्रैः सोमः॒ सोमो॑ रु॒द्रैः । \newline
25. रु॒द्रै रिन्द्र॒ इन्द्रो॑ रु॒द्रै रु॒द्रै रिन्द्रः॑ । \newline
26. इन्द्रो॑ म॒रुद्भि॑र् म॒रुद्भि॒ रिन्द्र॒ इन्द्रो॑ म॒रुद्भिः॑ । \newline
27. म॒रुद्भि॒र् वरु॑णो॒ वरु॑णो म॒रुद्भि॑र् म॒रुद्भि॒र् वरु॑णः । \newline
28. म॒रुद्भि॒रिति॑ म॒रुत् - भिः॒ । \newline
29. वरु॑ण आदि॒त्यै रा॑दि॒त्यैर् वरु॑णो॒ वरु॑ण आदि॒त्यैः । \newline
30. आ॒दि॒त्यैर् बृह॒स्पति॒र् बृह॒स्पति॑ रादि॒त्यै रा॑दि॒त्यैर् बृह॒स्पतिः॑ । \newline
31. बृह॒स्पति॒र् विश्वै॒र् विश्वै॒र् बृह॒स्पति॒र् बृह॒स्पति॒र् विश्वैः᳚ । \newline
32. विश्वै᳚र् दे॒वैर् दे॒वैर् विश्वै॒र् विश्वै᳚र् दे॒वैः । \newline
33. दे॒वै स्ते ते दे॒वैर् दे॒वै स्ते । \newline
34. ते॑ ऽमन्यन्ता मन्यन्त॒ ते ते॑ ऽमन्यन्त । \newline
35. अ॒म॒न्य॒न्ता सु॑रे॒भ्यो ऽसु॑रेभ्यो ऽमन्यन्ता मन्य॒न्ता सु॑रेभ्यः । \newline
36. असु॑रेभ्यो॒ वै वा असु॑रे॒भ्यो ऽसु॑रेभ्यो॒ वै । \newline
37. वा इ॒द मि॒दं ॅवै वा इ॒दम् । \newline
38. इ॒दम् भ्रातृ॑व्येभ्यो॒ भ्रातृ॑व्येभ्य इ॒द मि॒दम् भ्रातृ॑व्येभ्यः । \newline
39. भ्रातृ॑व्येभ्यो रद्ध्यामो रद्ध्यामो॒ भ्रातृ॑व्येभ्यो॒ भ्रातृ॑व्येभ्यो रद्ध्यामः । \newline
40. र॒द्ध्या॒मो॒ यद् यद् र॑द्ध्यामो रद्ध्यामो॒ यत् । \newline
41. यन् मि॒थो मि॒थो यद् यन् मि॒थः । \newline
42. मि॒थो विप्रि॑या॒ विप्रि॑या मि॒थो मि॒थो विप्रि॑याः । \newline
43. विप्रि॑याः॒ स्मः स्मो विप्रि॑या॒ विप्रि॑याः॒ स्मः । \newline
44. विप्रि॑या॒ इति॒ वि - प्रि॒याः॒ । \newline
45. स्मो या याः स्मः स्मो याः । \newline
46. या नो॑ नो॒ या या नः॑ । \newline
47. न॒ इ॒मा इ॒मा नो॑ न इ॒माः । \newline
48. इ॒माः प्रि॒याः प्रि॒या इ॒मा इ॒माः प्रि॒याः । \newline
49. प्रि॒या स्त॒नुव॑ स्त॒नुवः॑ प्रि॒याः प्रि॒या स्त॒नुवः॑ । \newline
50. त॒नुव॒ स्ता स्ता स्त॒नुव॑ स्त॒नुव॒ स्ताः । \newline
51. ताः स॒मव॑द्यामहै स॒मव॑द्यामहै॒ ता स्ताः स॒मव॑द्यामहै । \newline
52. स॒मव॑द्यामहै॒ ताभ्य॒ स्ताभ्यः॑ स॒मव॑द्यामहै स॒मव॑द्यामहै॒ ताभ्यः॑ । \newline
53. स॒मव॑द्यामहा॒ इति॑ सं - अव॑द्यामहै । \newline
54. ताभ्यः॒ स स ताभ्य॒ स्ताभ्यः॒ सः । \newline
55. स निर् णिः स स निः । \newline
56. निर्.ऋ॑च्छा दृच्छा॒न् निर् णिर्. ऋ॑च्छात् । \newline
57. ऋ॒च्छा॒द् यो य ऋ॑च्छा दृच्छा॒द् यः । \newline
58. यो नो॑ नो॒ यो यो नः॑ । \newline

\textbf{Ghana Paata } \newline

1. दे॒वा॒सु॒राः संॅय॑त्ताः॒ संॅय॑त्ता देवासु॒रा दे॑वासु॒राः संॅय॑त्ता आसन् नास॒न् थ्संॅय॑त्ता देवासु॒रा दे॑वासु॒राः संॅय॑त्ता आसन्न् । \newline
2. दे॒वा॒सु॒रा इति॑ देव - अ॒सु॒राः । \newline
3. संॅय॑त्ता आसन् नास॒न् थ्संॅय॑त्ताः॒ संॅय॑त्ता आस॒न् ते त आ॑स॒न् थ्संॅय॑त्ताः॒ संॅय॑त्ता आस॒न् ते । \newline
4. संॅय॑त्ता॒ इति॒ सं - य॒त्ताः॒ । \newline
5. आ॒स॒न् ते त आ॑सन् नास॒न् ते दे॒वा दे॒वा स्त आ॑सन् नास॒न् ते दे॒वाः । \newline
6. ते दे॒वा दे॒वा स्ते ते दे॒वा मि॒थो मि॒थो दे॒वा स्ते ते दे॒वा मि॒थः । \newline
7. दे॒वा मि॒थो मि॒थो दे॒वा दे॒वा मि॒थो विप्रि॑या॒ विप्रि॑या मि॒थो दे॒वा दे॒वा मि॒थो विप्रि॑याः । \newline
8. मि॒थो विप्रि॑या॒ विप्रि॑या मि॒थो मि॒थो विप्रि॑या आसन् नास॒न्॒. विप्रि॑या मि॒थो मि॒थो विप्रि॑या आसन्न् । \newline
9. विप्रि॑या आसन् नास॒न्॒. विप्रि॑या॒ विप्रि॑या आस॒न् ते त आ॑स॒न्॒. विप्रि॑या॒ विप्रि॑या आस॒न् ते । \newline
10. विप्रि॑या॒ इति॒ वि - प्रि॒याः॒ । \newline
11. आ॒स॒न् ते त आ॑सन् नास॒न् ते᳚(1॒) ऽन्यो᳚ ऽन्य स्त आ॑सन् नास॒न् ते᳚ ऽन्यः । \newline
12. ते᳚(1॒) ऽन्यो᳚ ऽन्य स्ते ते᳚(1॒) ऽन्यो᳚ ऽन्यस्मा॑ अ॒न्यस्मा॑ अ॒न्य स्ते ते᳚(1॒) ऽन्यो᳚ ऽन्यस्मै᳚ । \newline
13. अ॒न्यो᳚ ऽन्यस्मा॑ अ॒न्यस्मा॑ अ॒न्यो᳚(1॒) ऽन्यो᳚ ऽन्यस्मै॒ ज्यैष्ठ्या॑य॒ ज्यैष्ठ्या॑या॒ न्यस्मा॑ अ॒न्यो᳚(1॒) ऽन्यो᳚ ऽन्यस्मै॒ ज्यैष्ठ्या॑य । \newline
14. अ॒न्यस्मै॒ ज्यैष्ठ्या॑य॒ ज्यैष्ठ्या॑या॒ न्यस्मा॑ अ॒न्यस्मै॒ ज्यैष्ठ्या॒या ति॑ष्ठमाना॒ अति॑ष्ठमाना॒ ज्यैष्ठ्या॑या॒ न्यस्मा॑ अ॒न्यस्मै॒ ज्यैष्ठ्या॒या ति॑ष्ठमानाः । \newline
15. ज्यैष्ठ्या॒या ति॑ष्ठमाना॒ अति॑ष्ठमाना॒ ज्यैष्ठ्या॑य॒ ज्यैष्ठ्या॒या ति॑ष्ठमानाः पञ्च॒धा प॑ञ्च॒धा ऽति॑ष्ठमाना॒ ज्यैष्ठ्या॑य॒ ज्यैष्ठ्या॒या ति॑ष्ठमानाः पञ्च॒धा । \newline
16. अति॑ष्ठमानाः पञ्च॒धा प॑ञ्च॒धा ऽति॑ष्ठमाना॒ अति॑ष्ठमानाः पञ्च॒धा वि वि प॑ञ्च॒धा ऽति॑ष्ठमाना॒ अति॑ष्ठमानाः पञ्च॒धा वि । \newline
17. प॒ञ्च॒धा वि वि प॑ञ्च॒धा प॑ञ्च॒धा व्य॑क्रामन् नक्राम॒न्॒. वि प॑ञ्च॒धा प॑ञ्च॒धा व्य॑क्रामन्न् । \newline
18. प॒ञ्च॒धेति॑ पञ्च - धा । \newline
19. व्य॑क्रामन् नक्राम॒न्॒. वि व्य॑क्रामन् न॒ग्नि र॒ग्नि र॑क्राम॒न्॒. वि व्य॑क्रामन् न॒ग्निः । \newline
20. अ॒क्रा॒म॒न् न॒ग्नि र॒ग्नि र॑क्रामन् नक्रामन् न॒ग्निर् वसु॑भि॒र् वसु॑भि र॒ग्नि र॑क्रामन् नक्रामन् न॒ग्निर् वसु॑भिः । \newline
21. अ॒ग्निर् वसु॑भि॒र् वसु॑भि र॒ग्नि र॒ग्निर् वसु॑भिः॒ सोमः॒ सोमो॒ वसु॑भि र॒ग्नि र॒ग्निर् वसु॑भिः॒ सोमः॑ । \newline
22. वसु॑भिः॒ सोमः॒ सोमो॒ वसु॑भि॒र् वसु॑भिः॒ सोमो॑ रु॒द्रै रु॒द्रैः सोमो॒ वसु॑भि॒र् वसु॑भिः॒ सोमो॑ रु॒द्रैः । \newline
23. वसु॑भि॒रिति॒ वसु॑ - भिः॒ । \newline
24. सोमो॑ रु॒द्रै रु॒द्रैः सोमः॒ सोमो॑ रु॒द्रै रिन्द्र॒ इन्द्रो॑ रु॒द्रैः सोमः॒ सोमो॑ रु॒द्रै रिन्द्रः॑ । \newline
25. रु॒द्रै रिन्द्र॒ इन्द्रो॑ रु॒द्रै रु॒द्रै रिन्द्रो॑ म॒रुद्भि॑र् म॒रुद्भि॒ रिन्द्रो॑ रु॒द्रै रु॒द्रै रिन्द्रो॑ म॒रुद्भिः॑ । \newline
26. इन्द्रो॑ म॒रुद्भि॑र् म॒रुद्भि॒ रिन्द्र॒ इन्द्रो॑ म॒रुद्भि॒र् वरु॑णो॒ वरु॑णो म॒रुद्भि॒ रिन्द्र॒ इन्द्रो॑ म॒रुद्भि॒र् वरु॑णः । \newline
27. म॒रुद्भि॒र् वरु॑णो॒ वरु॑णो म॒रुद्भि॑र् म॒रुद्भि॒र् वरु॑ण आदि॒त्यै रा॑दि॒त्यैर् वरु॑णो म॒रुद्भि॑र् म॒रुद्भि॒र् वरु॑ण आदि॒त्यैः । \newline
28. म॒रुद्भि॒रिति॑ म॒रुत् - भिः॒ । \newline
29. वरु॑ण आदि॒त्यै रा॑दि॒त्यैर् वरु॑णो॒ वरु॑ण आदि॒त्यैर् बृह॒स्पति॒र् बृह॒स्पति॑ रादि॒त्यैर् वरु॑णो॒ वरु॑ण आदि॒त्यैर् बृह॒स्पतिः॑ । \newline
30. आ॒दि॒त्यैर् बृह॒स्पति॒र् बृह॒स्पति॑ रादि॒त्यै रा॑दि॒त्यैर् बृह॒स्पति॒र् विश्वै॒र् विश्वै॒र् बृह॒स्पति॑ रादि॒त्यै रा॑दि॒त्यैर् बृह॒स्पति॒र् विश्वैः᳚ । \newline
31. बृह॒स्पति॒र् विश्वै॒र् विश्वै॒र् बृह॒स्पति॒र् बृह॒स्पति॒र् विश्वै᳚र् दे॒वैर् दे॒वैर् विश्वै॒र् बृह॒स्पति॒र् बृह॒स्पति॒र् विश्वै᳚र् दे॒वैः । \newline
32. विश्वै᳚र् दे॒वैर् दे॒वैर् विश्वै॒र् विश्वै᳚र् दे॒वै स्ते ते दे॒वैर् विश्वै॒र् विश्वै᳚र् दे॒वै स्ते । \newline
33. दे॒वै स्ते ते दे॒वैर् दे॒वै स्ते॑ ऽमन्यन्ता मन्यन्त॒ ते दे॒वैर् दे॒वै स्ते॑ ऽमन्यन्त । \newline
34. ते॑ ऽमन्यन्ता मन्यन्त॒ ते ते॑ ऽमन्य॒न्ता सु॑रे॒भ्यो ऽसु॑रेभ्यो ऽमन्यन्त॒ ते ते॑ ऽमन्य॒न्ता सु॑रेभ्यः । \newline
35. अ॒म॒न्य॒न्ता सु॑रे॒भ्यो ऽसु॑रेभ्यो ऽमन्यन्ता मन्य॒न्ता सु॑रेभ्यो॒ वै वा असु॑रेभ्यो ऽमन्यन्ता मन्य॒न्ता सु॑रेभ्यो॒ वै । \newline
36. असु॑रेभ्यो॒ वै वा असु॑रे॒भ्यो ऽसु॑रेभ्यो॒ वा इ॒द मि॒दं ॅवा असु॑रे॒भ्यो ऽसु॑रेभ्यो॒ वा इ॒दम् । \newline
37. वा इ॒द मि॒दं ॅवै वा इ॒दम् भ्रातृ॑व्येभ्यो॒ भ्रातृ॑व्येभ्य इ॒दं ॅवै वा इ॒दम् भ्रातृ॑व्येभ्यः । \newline
38. इ॒दम् भ्रातृ॑व्येभ्यो॒ भ्रातृ॑व्येभ्य इ॒द मि॒दम् भ्रातृ॑व्येभ्यो रद्ध्यामो रद्ध्यामो॒ भ्रातृ॑व्येभ्य इ॒द मि॒दम् भ्रातृ॑व्येभ्यो रद्ध्यामः । \newline
39. भ्रातृ॑व्येभ्यो रद्ध्यामो रद्ध्यामो॒ भ्रातृ॑व्येभ्यो॒ भ्रातृ॑व्येभ्यो रद्ध्यामो॒ यद् यद् र॑द्ध्यामो॒ भ्रातृ॑व्येभ्यो॒ भ्रातृ॑व्येभ्यो रद्ध्यामो॒ यत् । \newline
40. र॒द्ध्या॒मो॒ यद् यद् र॑द्ध्यामो रद्ध्यामो॒ यन् मि॒थो मि॒थो यद् र॑द्ध्यामो रद्ध्यामो॒ यन् मि॒थः । \newline
41. यन् मि॒थो मि॒थो यद् यन् मि॒थो विप्रि॑या॒ विप्रि॑या मि॒थो यद् यन् मि॒थो विप्रि॑याः । \newline
42. मि॒थो विप्रि॑या॒ विप्रि॑या मि॒थो मि॒थो विप्रि॑याः॒ स्मः स्मो विप्रि॑या मि॒थो मि॒थो विप्रि॑याः॒ स्मः । \newline
43. विप्रि॑याः॒ स्मः स्मो विप्रि॑या॒ विप्रि॑याः॒ स्मो या याः स्मो विप्रि॑या॒ विप्रि॑याः॒ स्मो याः । \newline
44. विप्रि॑या॒ इति॒ वि - प्रि॒याः॒ । \newline
45. स्मो या याः स्मः स्मो या नो॑ नो॒ याः स्मः स्मो या नः॑ । \newline
46. या नो॑ नो॒ या या न॑ इ॒मा इ॒मा नो॒ या या न॑ इ॒माः । \newline
47. न॒ इ॒मा इ॒मा नो॑ न इ॒माः प्रि॒याः प्रि॒या इ॒मा नो॑ न इ॒माः प्रि॒याः । \newline
48. इ॒माः प्रि॒याः प्रि॒या इ॒मा इ॒माः प्रि॒या स्त॒नुव॑ स्त॒नुवः॑ प्रि॒या इ॒मा इ॒माः प्रि॒या स्त॒नुवः॑ । \newline
49. प्रि॒या स्त॒नुव॑ स्त॒नुवः॑ प्रि॒याः प्रि॒या स्त॒नुव॒ स्ता स्ता स्त॒नुवः॑ प्रि॒याः प्रि॒या स्त॒नुव॒ स्ताः । \newline
50. त॒नुव॒ स्ता स्ता स्त॒नुव॑ स्त॒नुव॒ स्ताः स॒मव॑द्यामहै स॒मव॑द्यामहै॒ ता स्त॒नुव॑ स्त॒नुव॒ स्ताः स॒मव॑द्यामहै । \newline
51. ताः स॒मव॑द्यामहै स॒मव॑द्यामहै॒ ता स्ताः स॒मव॑द्यामहै॒ ताभ्य॒ स्ताभ्यः॑ स॒मव॑द्यामहै॒ ता स्ताः स॒मव॑द्यामहै॒ ताभ्यः॑ । \newline
52. स॒मव॑द्यामहै॒ ताभ्य॒ स्ताभ्यः॑ स॒मव॑द्यामहै स॒मव॑द्यामहै॒ ताभ्यः॒ स स ताभ्यः॑ स॒मव॑द्यामहै स॒मव॑द्यामहै॒ ताभ्यः॒ सः । \newline
53. स॒मव॑द्यामहा॒ इति॑ सं - अव॑द्यामहै । \newline
54. ताभ्यः॒ स स ताभ्य॒ स्ताभ्यः॒ स निर् णिः स ताभ्य॒ स्ताभ्यः॒ स निः । \newline
55. स निर् णिः स स निर्. ऋ॑च्छा दृच्छा॒न् निः स स निर्. ऋ॑च्छात् । \newline
56. निर्. ऋ॑च्छा दृच्छा॒न् निर् णिर्. ऋ॑च्छा॒द् यो य ऋ॑च्छा॒न् निर् णिर्. ऋ॑च्छा॒द् यः । \newline
57. ऋ॒च्छा॒द् यो य ऋ॑च्छा दृच्छा॒द् यो नो॑ नो॒ य ऋ॑च्छा दृच्छा॒द् यो नः॑ । \newline
58. यो नो॑ नो॒ यो यो नः॑ प्रथ॒मः प्र॑थ॒मो नो॒ यो यो नः॑ प्रथ॒मः । \newline
\pagebreak
\markright{ TS 6.2.2.2  \hfill https://www.vedavms.in \hfill}

\section{ TS 6.2.2.2 }

\textbf{TS 6.2.2.2 } \newline
\textbf{Samhita Paata} \newline

नः॑ प्रथ॒मो᳚(1॒)ऽन्यो᳚ऽन्यस्मै॒ द्रुह्या॒दिति॒ तस्मा॒द्यः सता॑नूनप्त्रिणां प्रथ॒मो द्रुह्य॑ति॒ स आर्ति॒मार्च्छ॑ति॒ यत् ता॑नून॒प्त्रꣳ स॑मव॒द्यति॒ भ्रातृ॑व्याभिभूत्यै॒ भव॑त्या॒त्मना॒ परा᳚ऽस्य॒ भ्रातृ॑व्यो भवति॒ पञ्च॒ कृत्वोऽव॑द्यति पञ्च॒धा हि ते तथ् स॑म॒वाद्य॒न्ताथो॒ पञ्चा᳚क्षरा प॒ङ्क्तिः पाङ्क्तो॑ य॒ज्ञो य॒ज्ञ्मे॒वाव॑ रुन्ध॒ आप॑तये त्वा गृह्णा॒मीत्या॑ह प्रा॒णो वा- [  ] \newline

\textbf{Pada Paata} \newline

नः॒ । प्र॒थ॒मः । अ॒न्यः । अ॒न्यस्मै᳚ । द्रुह्या᳚त् । इति॑ । तस्मा᳚त् । यः । सता॑नूनप्त्रिणा॒मिति॒ स - ता॒नू॒न॒प्त्रि॒णा॒म् । प्र॒थ॒मः । द्रुह्य॑ति । सः । आर्ति᳚म् । एति॑ । ऋ॒च्छ॒ति॒ । यत् । ता॒नू॒न॒प्त्रमिति॑ तानू - न॒प्त्रम् । स॒म॒व॒द्यतीति॑ सं-अ॒व॒द्यति॑ । भ्रातृ॑व्याभिभूत्या॒ इति॒ भ्रातृ॑व्य-अ॒भि॒भू॒त्यै॒ । भव॑ति । आ॒त्मना᳚ । परेति॑ । अ॒स्य॒ । भ्रातृ॑व्यः । भ॒व॒ति॒ । पञ्च॑ । कृत्वः॑ । अवेति॑ । द्य॒ति॒ । प॒ञ्च॒धेति॑ पञ्च - धा । हि । ते । तत् । स॒म॒वाद्य॒न्तेति॑ सं - अ॒वाद्य॑न्त । अथो॒ इति॑ । पञ्चा᳚क्ष॒रेति॒ पञ्च॑ - अ॒क्ष॒रा॒ । प॒ङ्क्तिः । पाङ्क्तः॑ । य॒ज्ञ्ः । य॒ज्ञ्म् । ए॒व । अवेति॑ । रु॒न्धे॒ । आप॑तय॒ इत्या - प॒त॒ये॒ । त्वा॒ । गृ॒ह्णा॒मि॒ । इति॑ । आ॒ह॒ । प्रा॒ण इति॑ प्र - अ॒नः । वै ।  \newline


\textbf{Krama Paata} \newline

नः॒ प्र॒थ॒मः । प्र॒थ॒मो᳚ऽन्यः । अ॒न्यो᳚ऽन्यस्मै᳚ । अ॒न्यस्मै॒ द्रुह्या᳚त् । द्रु॒ह्यादिति॑ । इति॒ तस्मा᳚त् । तस्मा॒द् यः । यः सता॑नूनप्त्रिणाम् । सता॑नूनप्त्रिणाम् प्रथ॒मः । सता॑नूनप्त्रिणा॒मिति॒ स - ता॒नू॒न॒प्त्रि॒णा॒म् । प्र॒थ॒मो द्रुह्य॑ति । द्रुह्य॑ति॒ सः । स आर्ति᳚म् । आर्ति॒मा । आर्च्छ॑ति । ऋ॒च्छ॒ति॒ यत् । यत् ता॑नून॒प्त्रम् । ता॒नू॒न॒प्त्रꣳ स॑मव॒द्यति॑ । ता॒नू॒न॒प्त्रमिति॑ तानू - न॒प्त्रम् । स॒म॒व॒द्यति॒ भ्रातृ॑व्याभिभूत्यै । स॒म॒व॒द्यतीति॑ सम् - अ॒व॒द्यति॑ । भ्रातृ॑व्याभिभूत्यै॒ भव॑ति । भ्रातृ॑व्याभिभूत्या॒ इति॒ भ्रातृ॑व्य - अ॒भि॒भू॒त्यै॒ । भव॑त्या॒त्मना᳚ । आ॒त्मना॒ परा᳚ । परा᳚ऽस्य । अ॒स्य॒ भ्रातृ॑व्यः । भ्रातृ॑व्यो भवति । भ॒व॒ति॒ पञ्च॑ । पञ्च॒ कृत्वः॑ । कृत्वोऽव॑ । अव॑ द्यति । द्य॒ति॒ प॒ञ्च॒धा । प॒ञ्च॒धा हि । प॒ञ्च॒धेति॑ पञ्च - धा । हि ते । ते तत् । तथ् स॑म॒वाद्य॑न्त । स॒म॒वाद्य॒न्ताथो᳚ । स॒म॒वाद्य॒न्तेति॑ सम् - अ॒वाद्य॑न्त । अथो॒ पञ्चा᳚क्षरा । अथो॒ इत्यथो᳚ । पञ्चा᳚क्षरा प॒ङ्‍क्तिः । पञ्चा᳚क्ष॒रेति॒ पञ्च॑ - अ॒क्ष॒रा॒ । प॒ङ्‍क्तिः पाङ्‍क्तः॑ । पाङ्‍क्तो॑ य॒ज्ञ्ः । य॒ज्ञो य॒ज्ञ्म् । य॒ज्ञ्मे॒व । ए॒वाव॑ । अव॑ रुन्धे । रु॒न्ध॒ आप॑तये । आप॑तये त्वा । आप॑तय॒ इत्या - प॒त॒ये॒ । त्वा॒ गृ॒ह्णा॒मि॒ । गृ॒ह्णा॒मीति॑ । इत्या॑ह । आ॒ह॒ प्रा॒णः । प्रा॒णो वै । प्रा॒ण इति॑ प्र - अ॒नः । वा आप॑तिः \newline

\textbf{Jatai Paata} \newline

1. नः॒ प्र॒थ॒मः प्र॑थ॒मो नो॑ नः प्रथ॒मः । \newline
2. प्र॒थ॒मो᳚(1॒) ऽन्यो᳚ ऽन्यः प्र॑थ॒मः प्र॑थ॒मो᳚ ऽन्यः । \newline
3. अ॒न्यो᳚ ऽन्यस्मा॑ अ॒न्यस्मा॑ अ॒न्यो᳚(1॒) ऽन्यो᳚ ऽन्यस्मै᳚ । \newline
4. अ॒न्यस्मै॒ द्रुह्या॒द् द्रुह्या॑ द॒न्यस्मा॑ अ॒न्यस्मै॒ द्रुह्या᳚त् । \newline
5. द्रुह्या॒ दितीति॒ द्रुह्या॒द् द्रुह्या॒दिति॑ । \newline
6. इति॒ तस्मा॒त् तस्मा॒ दितीति॒ तस्मा᳚त् । \newline
7. तस्मा॒द् यो य स्तस्मा॒त् तस्मा॒द् यः । \newline
8. यः सता॑नूनप्त्रिणाꣳ॒॒ सता॑नूनप्त्रिणां॒ ॅयो यः सता॑नूनप्त्रिणाम् । \newline
9. सता॑नूनप्त्रिणाम् प्रथ॒मः प्र॑थ॒मः सता॑नूनप्त्रिणाꣳ॒॒ सता॑नूनप्त्रिणाम् प्रथ॒मः । \newline
10. सता॑नूनप्त्रिणा॒मिति॒ स - ता॒नू॒न॒प्त्रि॒णा॒म् । \newline
11. प्र॒थ॒मो द्रुह्य॑ति॒ द्रुह्य॑ति प्रथ॒मः प्र॑थ॒मो द्रुह्य॑ति । \newline
12. द्रुह्य॑ति॒ स स द्रुह्य॑ति॒ द्रुह्य॑ति॒ सः । \newline
13. स आर्ति॒ मार्तिꣳ॒॒ स स आर्ति᳚म् । \newline
14. आर्ति॒ मा ऽऽर्ति॒ मार्ति॒ मा । \newline
15. आर्च्छ॑ त्यृच्छ॒ त्यार्च्छ॑ति । \newline
16. ऋ॒च्छ॒ति॒ यद् यदृ॑च्छ त्यृच्छति॒ यत् । \newline
17. यत् ता॑नून॒प्त्रम् ता॑नून॒प्त्रं ॅयद् यत् ता॑नून॒प्त्रम् । \newline
18. ता॒नू॒न॒प्त्रꣳ स॑मव॒द्यति॑ समव॒द्यति॑ तानून॒प्त्रम् ता॑नून॒प्त्रꣳ स॑मव॒द्यति॑ । \newline
19. ता॒नू॒न॒प्त्रमिति॑ तानू - न॒प्त्रम् । \newline
20. स॒म॒व॒द्यति॒ भ्रातृ॑व्याभिभूत्यै॒ भ्रातृ॑व्याभिभूत्यै समव॒द्यति॑ समव॒द्यति॒ भ्रातृ॑व्याभिभूत्यै । \newline
21. स॒म॒व॒द्यतीति॑ सं - अ॒व॒द्यति॑ । \newline
22. भ्रातृ॑व्याभिभूत्यै॒ भव॑ति॒ भव॑ति॒ भ्रातृ॑व्याभिभूत्यै॒ भ्रातृ॑व्याभिभूत्यै॒ भव॑ति । \newline
23. भ्रातृ॑व्याभिभूत्या॒ इति॒ भ्रातृ॑व्य - अ॒भि॒भू॒त्यै॒ । \newline
24. भव॑ त्या॒त्मना॒ ऽऽत्मना॒ भव॑ति॒ भव॑ त्या॒त्मना᳚ । \newline
25. आ॒त्मना॒ परा॒ परा॒ ऽऽत्मना॒ ऽऽत्मना॒ परा᳚ । \newline
26. परा᳚ ऽस्यास्य॒ परा॒ परा᳚ ऽस्य । \newline
27. अ॒स्य॒ भ्रातृ॑व्यो॒ भ्रातृ॑व्यो ऽस्यास्य॒ भ्रातृ॑व्यः । \newline
28. भ्रातृ॑व्यो भवति भवति॒ भ्रातृ॑व्यो॒ भ्रातृ॑व्यो भवति । \newline
29. भ॒व॒ति॒ पञ्च॒ पञ्च॑ भवति भवति॒ पञ्च॑ । \newline
30. पञ्च॒ कृत्वः॒ कृत्वः॒ पञ्च॒ पञ्च॒ कृत्वः॑ । \newline
31. कृत्वो ऽवाव॒ कृत्वः॒ कृत्वो ऽव॑ । \newline
32. अव॑ द्यति द्य॒ त्यवाव॑ द्यति । \newline
33. द्य॒ति॒ प॒ञ्च॒धा प॑ञ्च॒धा द्य॑ति द्यति पञ्च॒धा । \newline
34. प॒ञ्च॒धा हि हि प॑ञ्च॒धा प॑ञ्च॒धा हि । \newline
35. प॒ञ्च॒धेति॑ पञ्च - धा । \newline
36. हि ते ते हि हि ते । \newline
37. ते तत् तत् ते ते तत् । \newline
38. तथ् स॑म॒वाद्य॑न्त सम॒वाद्य॑न्त॒ तत् तथ् स॑म॒वाद्य॑न्त । \newline
39. स॒म॒वाद्य॒न्ताथो॒ अथो॑ सम॒वाद्य॑न्त सम॒वाद्य॒न्ताथो᳚ । \newline
40. स॒म॒वाद्य॒न्तेति॑ सं - अ॒वाद्य॑न्त । \newline
41. अथो॒ पञ्चा᳚क्षरा॒ पञ्चा᳚क्ष॒रा ऽथो॒ अथो॒ पञ्चा᳚क्षरा । \newline
42. अथो॒ इत्यथो᳚ । \newline
43. पञ्चा᳚क्षरा प॒ङ्क्तिः प॒ङ्क्तिः पञ्चा᳚क्षरा॒ पञ्चा᳚क्षरा प॒ङ्क्तिः । \newline
44. पञ्चा᳚क्ष॒रेति॒ पञ्च॑ - अ॒क्ष॒रा॒ । \newline
45. प॒ङ्क्तिः पाङ्क्तः॒ पाङ्क्तः॑ प॒ङ्क्तिः प॒ङ्क्तिः पाङ्क्तः॑ । \newline
46. पाङ्क्तो॑ य॒ज्ञो य॒ज्ञ्ः पाङ्क्तः॒ पाङ्क्तो॑ य॒ज्ञ्ः । \newline
47. य॒ज्ञो य॒ज्ञ्ं ॅय॒ज्ञ्ं ॅय॒ज्ञो य॒ज्ञो य॒ज्ञ्म् । \newline
48. य॒ज्ञ् मे॒वैव य॒ज्ञ्ं ॅय॒ज्ञ् मे॒व । \newline
49. ए॒वावा वै॒वै वाव॑ । \newline
50. अव॑ रुन्धे रु॒न्धे ऽवाव॑ रुन्धे । \newline
51. रु॒न्ध॒ आप॑तय॒ आप॑तये रुन्धे रुन्ध॒ आप॑तये । \newline
52. आप॑तये त्वा॒ त्वा ऽऽप॑तय॒ आप॑तये त्वा । \newline
53. आप॑तय॒ इत्या - प॒त॒ये॒ । \newline
54. त्वा॒ गृ॒ह्णा॒मि॒ गृ॒ह्णा॒मि॒ त्वा॒ त्वा॒ गृ॒ह्णा॒मि॒ । \newline
55. गृ॒ह्णा॒मीतीति॑ गृह्णामि गृह्णा॒मीति॑ । \newline
56. इत्या॑हा॒हे तीत्या॑ह । \newline
57. आ॒ह॒ प्रा॒णः प्रा॒ण आ॑हाह प्रा॒णः । \newline
58. प्रा॒णो वै वै प्रा॒णः प्रा॒णो वै । \newline
59. प्रा॒ण इति॑ प्र - अ॒नः । \newline
60. वा आप॑ति॒ राप॑ति॒र् वै वा आप॑तिः । \newline

\textbf{Ghana Paata } \newline

1. नः॒ प्र॒थ॒मः प्र॑थ॒मो नो॑ नः प्रथ॒मो᳚(1॒) ऽन्यो᳚ ऽन्यः प्र॑थ॒मो नो॑ नः प्रथ॒मो᳚ ऽन्यः । \newline
2. प्र॒थ॒मो᳚(1॒) ऽन्यो᳚ ऽन्यः प्र॑थ॒मः प्र॑थ॒मो᳚(1॒) ऽन्यो᳚ ऽन्यस्मा॑ अ॒न्यस्मा॑ अ॒न्यः प्र॑थ॒मः प्र॑थ॒मो᳚(1॒) ऽन्यो᳚ ऽन्यस्मै᳚ । \newline
3. अ॒न्यो᳚ ऽन्यस्मा॑ अ॒न्यस्मा॑ अ॒न्यो᳚(1॒) ऽन्यो᳚ ऽन्यस्मै॒ द्रुह्या॒द् द्रुह्या॑ द॒न्यस्मा॑ अ॒न्यो᳚(1॒) ऽन्यो᳚ ऽन्यस्मै॒ द्रुह्या᳚त् । \newline
4. अ॒न्यस्मै॒ द्रुह्या॒द् द्रुह्या॑ द॒न्यस्मा॑ अ॒न्यस्मै॒ द्रुह्या॒ दितीति॒ द्रुह्या॑ द॒न्यस्मा॑ अ॒न्यस्मै॒ द्रुह्या॒दिति॑ । \newline
5. द्रुह्या॒दि तीति॒ द्रुह्या॒द् द्रुह्या॒ दिति॒ तस्मा॒त् तस्मा॒ दिति॒ द्रुह्या॒द् द्रुह्या॒ दिति॒ तस्मा᳚त् । \newline
6. इति॒ तस्मा॒त् तस्मा॒दितीति॒ तस्मा॒द् यो य स्तस्मा॒दितीति॒ तस्मा॒द् यः । \newline
7. तस्मा॒द् यो य स्तस्मा॒त् तस्मा॒द् यः सता॑नूनप्त्रिणाꣳ॒॒ सता॑नूनप्त्रिणां॒ ॅय स्तस्मा॒त् तस्मा॒द् यः सता॑नूनप्त्रिणाम् । \newline
8. यः सता॑नूनप्त्रिणाꣳ॒॒ सता॑नूनप्त्रिणां॒ ॅयो यः सता॑नूनप्त्रिणाम् प्रथ॒मः प्र॑थ॒मः सता॑नूनप्त्रिणां॒ ॅयो यः सता॑नूनप्त्रिणाम् प्रथ॒मः । \newline
9. सता॑नूनप्त्रिणाम् प्रथ॒मः प्र॑थ॒मः सता॑नूनप्त्रिणाꣳ॒॒ सता॑नूनप्त्रिणाम् प्रथ॒मो द्रुह्य॑ति॒ द्रुह्य॑ति प्रथ॒मः सता॑नूनप्त्रिणाꣳ॒॒ सता॑नूनप्त्रिणाम् प्रथ॒मो द्रुह्य॑ति । \newline
10. सता॑नूनप्त्रिणा॒मिति॒ स - ता॒नू॒न॒प्त्रि॒णा॒म् । \newline
11. प्र॒थ॒मो द्रुह्य॑ति॒ द्रुह्य॑ति प्रथ॒मः प्र॑थ॒मो द्रुह्य॑ति॒ स स द्रुह्य॑ति प्रथ॒मः प्र॑थ॒मो द्रुह्य॑ति॒ सः । \newline
12. द्रुह्य॑ति॒ स स द्रुह्य॑ति॒ द्रुह्य॑ति॒ स आर्ति॒ मार्तिꣳ॒॒ स द्रुह्य॑ति॒ द्रुह्य॑ति॒ स आर्ति᳚म् । \newline
13. स आर्ति॒ मार्तिꣳ॒॒ स स आर्ति॒ मा ऽऽर्तिꣳ॒॒ स स आर्ति॒ मा । \newline
14. आर्ति॒ मा ऽऽर्ति॒ मार्ति॒ मार्च्छ॑ त्यृच्छ॒ त्याऽऽर्ति॒ मार्ति॒ मार्च्छ॑ति । \newline
15. आर्च्छ॑ त्यृच्छ॒ त्यार्च्छ॑ति॒ यद् यदृ॑च्छ॒ त्यार्च्छ॑ति॒ यत् । \newline
16. ऋ॒च्छ॒ति॒ यद् यदृ॑च्छ त्यृच्छति॒ यत् ता॑नून॒प्त्रम् ता॑नून॒प्त्रं ॅयदृ॑च्छ त्यृच्छति॒ यत् ता॑नून॒प्त्रम् । \newline
17. यत् ता॑नून॒प्त्रम् ता॑नून॒प्त्रं ॅयद् यत् ता॑नून॒प्त्रꣳ स॑मव॒द्यति॑ समव॒द्यति॑ तानून॒प्त्रं ॅयद् यत् ता॑नून॒प्त्रꣳ स॑मव॒द्यति॑ । \newline
18. ता॒नू॒न॒प्त्रꣳ स॑मव॒द्यति॑ समव॒द्यति॑ तानून॒प्त्रम् ता॑नून॒प्त्रꣳ स॑मव॒द्यति॒ भ्रातृ॑व्याभिभूत्यै॒ भ्रातृ॑व्याभिभूत्यै समव॒द्यति॑ तानून॒प्त्रम् ता॑नून॒प्त्रꣳ स॑मव॒द्यति॒ भ्रातृ॑व्याभिभूत्यै । \newline
19. ता॒नू॒न॒प्त्रमिति॑ तानू - न॒प्त्रम् । \newline
20. स॒म॒व॒द्यति॒ भ्रातृ॑व्याभिभूत्यै॒ भ्रातृ॑व्याभिभूत्यै समव॒द्यति॑ समव॒द्यति॒ भ्रातृ॑व्याभिभूत्यै॒ भव॑ति॒ भव॑ति॒ भ्रातृ॑व्याभिभूत्यै समव॒द्यति॑ समव॒द्यति॒ भ्रातृ॑व्याभिभूत्यै॒ भव॑ति । \newline
21. स॒म॒व॒द्यतीति॑ सं - अ॒व॒द्यति॑ । \newline
22. भ्रातृ॑व्याभिभूत्यै॒ भव॑ति॒ भव॑ति॒ भ्रातृ॑व्याभिभूत्यै॒ भ्रातृ॑व्याभिभूत्यै॒ भव॑ त्या॒त्मना॒ ऽऽत्मना॒ भव॑ति॒ भ्रातृ॑व्याभिभूत्यै॒ भ्रातृ॑व्याभिभूत्यै॒ भव॑ त्या॒त्मना᳚ । \newline
23. भ्रातृ॑व्याभिभूत्या॒ इति॒ भ्रातृ॑व्य - अ॒भि॒भू॒त्यै॒ । \newline
24. भव॑ त्या॒त्मना॒ ऽऽत्मना॒ भव॑ति॒ भव॑ त्या॒त्मना॒ परा॒ परा॒ ऽऽत्मना॒ भव॑ति॒ भव॑ त्या॒त्मना॒ परा᳚ । \newline
25. आ॒त्मना॒ परा॒ परा॒ ऽऽत्मना॒ ऽऽत्मना॒ परा᳚ ऽस्यास्य॒ परा॒ ऽऽत्मना॒ ऽऽत्मना॒ परा᳚ ऽस्य । \newline
26. परा᳚ ऽस्यास्य॒ परा॒ परा᳚ ऽस्य॒ भ्रातृ॑व्यो॒ भ्रातृ॑व्यो ऽस्य॒ परा॒ परा᳚ ऽस्य॒ भ्रातृ॑व्यः । \newline
27. अ॒स्य॒ भ्रातृ॑व्यो॒ भ्रातृ॑व्यो ऽस्यास्य॒ भ्रातृ॑व्यो भवति भवति॒ भ्रातृ॑व्यो ऽस्यास्य॒ भ्रातृ॑व्यो भवति । \newline
28. भ्रातृ॑व्यो भवति भवति॒ भ्रातृ॑व्यो॒ भ्रातृ॑व्यो भवति॒ पञ्च॒ पञ्च॑ भवति॒ भ्रातृ॑व्यो॒ भ्रातृ॑व्यो भवति॒ पञ्च॑ । \newline
29. भ॒व॒ति॒ पञ्च॒ पञ्च॑ भवति भवति॒ पञ्च॒ कृत्वः॒ कृत्वः॒ पञ्च॑ भवति भवति॒ पञ्च॒ कृत्वः॑ । \newline
30. पञ्च॒ कृत्वः॒ कृत्वः॒ पञ्च॒ पञ्च॒ कृत्वो ऽवाव॒ कृत्वः॒ पञ्च॒ पञ्च॒ कृत्वो ऽव॑ । \newline
31. कृत्वो ऽवाव॒ कृत्वः॒ कृत्वो ऽव॑ द्यति द्य॒त्यव॒ कृत्वः॒ कृत्वो ऽव॑ द्यति । \newline
32. अव॑ द्यति द्य॒त्य वाव॑ द्यति पञ्च॒धा प॑ञ्च॒धा द्य॒त्य वाव॑ द्यति पञ्च॒धा । \newline
33. द्य॒ति॒ प॒ञ्च॒धा प॑ञ्च॒धा द्य॑ति द्यति पञ्च॒धा हि हि प॑ञ्च॒धा द्य॑ति द्यति पञ्च॒धा हि । \newline
34. प॒ञ्च॒धा हि हि प॑ञ्च॒धा प॑ञ्च॒धा हि ते ते हि प॑ञ्च॒धा प॑ञ्च॒धा हि ते । \newline
35. प॒ञ्च॒धेति॑ पञ्च - धा । \newline
36. हि ते ते हि हि ते तत् तत् ते हि हि ते तत् । \newline
37. ते तत् तत् ते ते तथ् स॑म॒वाद्य॑न्त सम॒वाद्य॑न्त॒ तत् ते ते तथ् स॑म॒वाद्य॑न्त । \newline
38. तथ् स॑म॒वाद्य॑न्त सम॒वाद्य॑न्त॒ तत् तथ् स॑म॒वाद्य॒न्ताथो॒ अथो॑ सम॒वाद्य॑न्त॒ तत् तथ् स॑म॒वाद्य॒न्ताथो᳚ । \newline
39. स॒म॒वाद्य॒न्ताथो॒ अथो॑ सम॒वाद्य॑न्त सम॒वाद्य॒न्ताथो॒ पञ्चा᳚क्षरा॒ पञ्चा᳚क्ष॒रा ऽथो॑ सम॒वाद्य॑न्त सम॒वाद्य॒न्ताथो॒ पञ्चा᳚क्षरा । \newline
40. स॒म॒वाद्य॒न्तेति॑ सं - अ॒वाद्य॑न्त । \newline
41. अथो॒ पञ्चा᳚क्षरा॒ पञ्चा᳚क्ष॒रा ऽथो॒ अथो॒ पञ्चा᳚क्षरा प॒ङ्क्तिः प॒ङ्क्तिः पञ्चा᳚क्ष॒रा ऽथो॒ अथो॒ पञ्चा᳚क्षरा प॒ङ्क्तिः । \newline
42. अथो॒ इत्यथो᳚ । \newline
43. पञ्चा᳚क्षरा प॒ङ्क्तिः प॒ङ्क्तिः पञ्चा᳚क्षरा॒ पञ्चा᳚क्षरा प॒ङ्क्तिः पाङ्क्तः॒ पाङ्क्तः॑ प॒ङ्क्तिः पञ्चा᳚क्षरा॒ पञ्चा᳚क्षरा प॒ङ्क्तिः पाङ्क्तः॑ । \newline
44. पञ्चा᳚क्ष॒रेति॒ पञ्च॑ - अ॒क्ष॒रा॒ । \newline
45. प॒ङ्क्तिः पाङ्क्तः॒ पाङ्क्तः॑ प॒ङ्क्तिः प॒ङ्क्तिः पाङ्क्तो॑ य॒ज्ञो य॒ज्ञ्ः पाङ्क्तः॑ प॒ङ्क्तिः प॒ङ्क्तिः पाङ्क्तो॑ य॒ज्ञ्ः । \newline
46. पाङ्क्तो॑ य॒ज्ञो य॒ज्ञ्ः पाङ्क्तः॒ पाङ्क्तो॑ य॒ज्ञो य॒ज्ञ्ं ॅय॒ज्ञ्ं ॅय॒ज्ञ्ः पाङ्क्तः॒ पाङ्क्तो॑ य॒ज्ञो य॒ज्ञ्म् । \newline
47. य॒ज्ञो य॒ज्ञ्ं ॅय॒ज्ञ्ं ॅय॒ज्ञो य॒ज्ञो य॒ज्ञ् मे॒वैव य॒ज्ञ्ं ॅय॒ज्ञो य॒ज्ञो य॒ज्ञ् मे॒व । \newline
48. य॒ज्ञ् मे॒वैव य॒ज्ञ्ं ॅय॒ज्ञ् मे॒वावा वै॒व य॒ज्ञ्ं ॅय॒ज्ञ् मे॒वाव॑ । \newline
49. ए॒वावा वै॒वै वाव॑ रुन्धे रु॒न्धे ऽवै॒वै वाव॑ रुन्धे । \newline
50. अव॑ रुन्धे रु॒न्धे ऽवाव॑ रुन्ध॒ आप॑तय॒ आप॑तये रु॒न्धे ऽवाव॑ रुन्ध॒ आप॑तये । \newline
51. रु॒न्ध॒ आप॑तय॒ आप॑तये रुन्धे रुन्ध॒ आप॑तये त्वा॒ त्वा ऽऽप॑तये रुन्धे रुन्ध॒ आप॑तये त्वा । \newline
52. आप॑तये त्वा॒ त्वा ऽऽप॑तय॒ आप॑तये त्वा गृह्णामि गृह्णामि॒ त्वा ऽऽप॑तय॒ आप॑तये त्वा गृह्णामि । \newline
53. आप॑तय॒ इत्या - प॒त॒ये॒ । \newline
54. त्वा॒ गृ॒ह्णा॒मि॒ गृ॒ह्णा॒मि॒ त्वा॒ त्वा॒ गृ॒ह्णा॒मी तीति॑ गृह्णामि त्वा त्वा गृह्णा॒मीति॑ । \newline
55. गृ॒ह्णा॒मी तीति॑ गृह्णामि गृह्णा॒मी त्या॑हा॒हेति॑ गृह्णामि गृह्णा॒ मीत्या॑ह । \newline
56. इत्या॑हा॒हे तीत्या॑ह प्रा॒णः प्रा॒ण आ॒हे तीत्या॑ह प्रा॒णः । \newline
57. आ॒ह॒ प्रा॒णः प्रा॒ण आ॑हाह प्रा॒णो वै वै प्रा॒ण आ॑हाह प्रा॒णो वै । \newline
58. प्रा॒णो वै वै प्रा॒णः प्रा॒णो वा आप॑ति॒ राप॑ति॒र् वै प्रा॒णः प्रा॒णो वा आप॑तिः । \newline
59. प्रा॒ण इति॑ प्र - अ॒नः । \newline
60. वा आप॑ति॒ राप॑ति॒र् वै वा आप॑तिः प्रा॒णम् प्रा॒ण माप॑ति॒र् वै वा आप॑तिः प्रा॒णम् । \newline
\pagebreak
\markright{ TS 6.2.2.3  \hfill https://www.vedavms.in \hfill}

\section{ TS 6.2.2.3 }

\textbf{TS 6.2.2.3 } \newline
\textbf{Samhita Paata} \newline

आप॑तिः प्रा॒णमे॒व प्री॑णाति॒ परि॑पतय॒ इत्या॑ह॒ मनो॒ वै परि॑पति॒र्मन॑ ए॒व प्री॑णाति॒ तनू॒नप्त्र॒ इत्या॑ह त॒नुवो॒ हि ते ताः स॑म॒वाद्य॑न्त शाक्व॒रायेत्या॑ह॒ शक्त्यै॒ हि ते ताः स॑म॒वाद्य॑न्त॒ शक्म॒-न्नोजि॑ष्ठा॒येत्या॒हौजि॑ष्ठꣳ॒॒ हि ते त दा॒त्मनः॑ सम॒वाद्य॒न्ता--ना॑धृष्टमस्यनाधृ॒ष्य-मित्या॒हाना॑धृष्टꣳ॒॒ ह्ये॑तद॑नाधृ॒ष्यं दे॒वाना॒मोज॒- [  ] \newline

\textbf{Pada Paata} \newline

आप॑ति॒रित्या - प॒तिः॒ । प्रा॒णमिति॑ प्र - अ॒नम् । ए॒व । प्री॒णा॒ति॒ । परि॑पतय॒ इति॒ परि॑ - प॒त॒ये॒ । इति॑ । आ॒ह॒ । मनः॑ । वै । परि॑पति॒रिति॒ परि॑ - प॒तिः॒ । मनः॑ । ए॒व । प्री॒णा॒ति॒ । तनू॒नप्त्र॒ इति॒ तनू᳚ - नप्त्रे᳚ । इति॑ । आ॒ह॒ । त॒नुवः॑ । हि । ते । ताः । स॒म॒वाद्य॒न्तेति॑ सं- अ॒वाद्य॑न्त । शा॒क्व॒राय॑ । इति॑ । आ॒ह॒ । शक्त्यै᳚ । हि । ते । ताः । स॒म॒वाद्य॒न्तेति॑ सं-अ॒वाद्य॑न्त । शक्मन्न्॑ । ओजि॑ष्ठाय । इति॑ । आ॒ह॒ । ओजि॑ष्ठम् । हि । ते । तत् । आ॒त्मनः॑ । स॒म॒वाद्य॒न्तेति॑ सं-अ॒वाद्य॑न्त । अना॑धृष्ट॒मित्यना᳚ - धृ॒ष्ट॒म् । अ॒सि॒ । अ॒ना॒धृ॒ष्यमित्य॑ना - धृ॒ष्यम् । इति॑ । आ॒ह॒ । अना॑धृष्ट॒मित्यना᳚ - धृ॒ष्ट॒म् । हि । ए॒तत् । अ॒ना॒धृ॒ष्यमित्य॑ना - धृ॒ष्यम् । दे॒वाना᳚म् । ओजः॑ ।  \newline


\textbf{Krama Paata} \newline

आप॑तिः प्रा॒णम् । आप॑ति॒रित्या - प॒तिः॒ । प्रा॒णमे॒व । प्रा॒णमिति॑ प्र - अ॒नम् । ए॒व प्री॑णाति । प्री॒णा॒ति॒ परि॑पतये । परि॑पतय॒ इति॑ । परि॑पतय॒ इति॒ परि॑ - प॒त॒ये॒ । इत्या॑ह । आ॒ह॒ मनः॑ । मनो॒ वै । वै परि॑पतिः । परि॑पति॒र् मनः॑ । परि॑पति॒रिति॒ परि॑ - प॒तिः॒ । मन॑ ए॒व । ए॒व प्री॑णाति । प्री॒णा॒ति॒ तनू॒नप्त्रे᳚ । तनू॒नप्त्र॒ इति॑ । तनू॒नप्त्र॒ इति॒ तनू᳚ - नप्त्रे᳚ । इत्या॑ह । आ॒ह॒ त॒नुवः॑ । त॒नुवो॒ हि । हि ते । ते ताः । ताः स॑म॒वाद्य॑न्त । स॒म॒वाद्य॑न्त शाक्व॒राय॑ । स॒म॒वाद्य॒न्तेति॑ सम् - अ॒वाद्य॑न्त । शा॒क्व॒रायेति॑ । इत्या॑ह । आ॒ह॒ शक्त्यै᳚ । शक्त्यै॒ हि । हि ते । ते ताः । ताः स॑म॒वाद्य॑न्त । स॒म॒वाद्य॑न्त॒ शक्मन्न्॑ । स॒म॒वाद्य॒न्तेति॑ सम् - अ॒वाद्य॑न्त । शक्म॒न्नोजि॑ष्ठाय । ओजि॑ष्ठा॒येति॑ । इत्या॑ह । आ॒हौजि॑ष्ठम् । ओजि॑ष्ठꣳ॒॒ हि । हि ते । ते तत् । तदा॒त्मनः॑ । आ॒त्मनः॑ सम॒वाद्य॑न्त । स॒म॒वाद्य॒न्ताना॑धृष्टम् । स॒म॒वाद्य॒न्तेति॑ सम् - अ॒वाद्य॑न्त । अना॑धृष्टमसि । अना॑धृष्ट॒मित्यना᳚ - धृ॒ष्ट॒म् । अ॒स्य॒ना॒धृ॒ष्यम् । अ॒ना॒धृ॒ष्यमिति॑ । अ॒ना॒धृ॒ष्यमित्य॑ना - धृ॒ष्यम् । इत्या॑ह । आ॒हाना॑धृष्टम् । अना॑धृष्टꣳ॒॒ हि । अना॑धृष्ट॒मित्यना᳚ - धृ॒ष्ट॒म् । ह्ये॑तत् । ए॒तद॑नाधृ॒ष्यम् । अ॒ना॒धृ॒ष्यम् दे॒वाना᳚म् । अ॒ना॒धृ॒ष्यमित्य॑ना - धृ॒ष्यम् । दे॒वाना॒मोजः॑ । ओज॒ इति॑ \newline

\textbf{Jatai Paata} \newline

1. आप॑तिः प्रा॒णम् प्रा॒ण माप॑ति॒ राप॑तिः प्रा॒णम् । \newline
2. आप॑ति॒रित्या - प॒तिः॒ । \newline
3. प्रा॒ण मे॒वैव प्रा॒णम् प्रा॒ण मे॒व । \newline
4. प्रा॒णमिति॑ प्र - अ॒नम् । \newline
5. ए॒व प्री॑णाति प्रीणा त्ये॒वैव प्री॑णाति । \newline
6. प्री॒णा॒ति॒ परि॑पतये॒ परि॑पतये प्रीणाति प्रीणाति॒ परि॑पतये । \newline
7. परि॑पतय॒ इतीति॒ परि॑पतये॒ परि॑पतय॒ इति॑ । \newline
8. परि॑पतय॒ इति॒ परि॑ - प॒त॒ये॒ । \newline
9. इत्या॑हा॒हे तीत्या॑ह । \newline
10. आ॒ह॒ मनो॒ मन॑ आहाह॒ मनः॑ । \newline
11. मनो॒ वै वै मनो॒ मनो॒ वै । \newline
12. वै परि॑पतिः॒ परि॑पति॒र् वै वै परि॑पतिः । \newline
13. परि॑पति॒र् मनो॒ मनः॒ परि॑पतिः॒ परि॑पति॒र् मनः॑ । \newline
14. परि॑पति॒रिति॒ परि॑ - प॒तिः॒ । \newline
15. मन॑ ए॒वैव मनो॒ मन॑ ए॒व । \newline
16. ए॒व प्री॑णाति प्रीणा त्ये॒वैव प्री॑णाति । \newline
17. प्री॒णा॒ति॒ तनू॒नप्त्रे॒ तनू॒नप्त्रे᳚ प्रीणाति प्रीणाति॒ तनू॒नप्त्रे᳚ । \newline
18. तनू॒नप्त्र॒ इतीति॒ तनू॒नप्त्रे॒ तनू॒नप्त्र॒ इति॑ । \newline
19. तनू॒नप्त्र॒ इति॒ तनू᳚ - नप्त्रे᳚ । \newline
20. इत्या॑हा॒हे तीत्या॑ह । \newline
21. आ॒ह॒ त॒नुव॑ स्त॒नुव॑ आहाह त॒नुवः॑ । \newline
22. त॒नुवो॒ हि हि त॒नुव॑ स्त॒नुवो॒ हि । \newline
23. हि ते ते हि हि ते । \newline
24. ते ता स्ता स्ते ते ताः । \newline
25. ताः स॑म॒वाद्य॑न्त सम॒वाद्य॑न्त॒ ता स्ताः स॑म॒वाद्य॑न्त । \newline
26. स॒म॒वाद्य॑न्त शाक्व॒राय॑ शाक्व॒राय॑ सम॒वाद्य॑न्त सम॒वाद्य॑न्त शाक्व॒राय॑ । \newline
27. स॒म॒वाद्य॒न्तेति॑ सं - अ॒वाद्य॑न्त । \newline
28. शा॒क्व॒राये तीति॑ शाक्व॒राय॑ शाक्व॒रायेति॑ । \newline
29. इत्या॑हा॒हे तीत्या॑ह । \newline
30. आ॒ह॒ शक्त्यै॒ शक्त्या॑ आहाह॒ शक्त्यै᳚ । \newline
31. शक्त्यै॒ हि हि शक्त्यै॒ शक्त्यै॒ हि । \newline
32. हि ते ते हि हि ते । \newline
33. ते ता स्ता स्ते ते ताः । \newline
34. ताः स॑म॒वाद्य॑न्त सम॒वाद्य॑न्त॒ ता स्ताः स॑म॒वाद्य॑न्त । \newline
35. स॒म॒वाद्य॑न्त॒ शक्म॒ञ् छक्मन्᳚ थ्सम॒वाद्य॑न्त सम॒वाद्य॑न्त॒ शक्मन्न्॑ । \newline
36. स॒म॒वाद्य॒न्तेति॑ सं - अ॒वाद्य॑न्त । \newline
37. शक्म॒न् नोजि॑ष्ठा॒ यौजि॑ष्ठाय॒ शक्म॒ञ् छक्म॒न् नोजि॑ष्ठाय । \newline
38. ओजि॑ष्ठा॒ये तीत्योजि॑ष्ठा॒ यौजि॑ष्ठा॒येति॑ । \newline
39. इत्या॑हा॒हे तीत्या॑ह । \newline
40. आ॒हौजि॑ष्ठ॒ मोजि॑ष्ठ माहा॒ हौजि॑ष्ठम् । \newline
41. ओजि॑ष्ठꣳ॒॒ हि ह्योजि॑ष्ठ॒ मोजि॑ष्ठꣳ॒॒ हि । \newline
42. हि ते ते हि हि ते । \newline
43. ते तत् तत् ते ते तत् । \newline
44. तदा॒त्मन॑ आ॒त्मन॒ स्तत् तदा॒त्मनः॑ । \newline
45. आ॒त्मनः॑ सम॒वाद्य॑न्त सम॒वाद्य॑न्ता॒ त्मन॑ आ॒त्मनः॑ सम॒वाद्य॑न्त । \newline
46. स॒म॒वाद्य॒न्ता ना॑धृष्ट॒ मना॑धृष्टꣳ सम॒वाद्य॑न्त सम॒वाद्य॒न्ता ना॑धृष्टम् । \newline
47. स॒म॒वाद्य॒न्तेति॑ सं - अ॒वाद्य॑न्त । \newline
48. अना॑धृष्ट मस्य॒ स्यना॑धृष्ट॒ मना॑धृष्ट मसि । \newline
49. अना॑धृष्ट॒मित्यना᳚ - धृ॒ष्ट॒म् । \newline
50. अ॒स्य॒ना॒धृ॒ष्य म॑नाधृ॒ष्य म॑स्य स्यनाधृ॒ष्यम् । \newline
51. अ॒ना॒धृ॒ष्य मिती त्य॑नाधृ॒ष्य म॑नाधृ॒ष्य मिति॑ । \newline
52. अ॒ना॒धृ॒ष्यमित्य॑ना - धृ॒ष्यम् । \newline
53. इत्या॑हा॒हे तीत्या॑ह । \newline
54. आ॒हाना॑धृष्ट॒ मना॑धृष्ट माहा॒हा ना॑धृष्टम् । \newline
55. अना॑धृष्टꣳ॒॒ हि ह्यना॑धृष्ट॒ मना॑धृष्टꣳ॒॒ हि । \newline
56. अना॑धृष्ट॒मित्यना᳚ - धृ॒ष्ट॒म् । \newline
57. ह्ये॑ तदे॒त द्धि ह्ये॑तत् । \newline
58. ए॒त द॑नाधृ॒ष्य म॑नाधृ॒ष्य मे॒त दे॒त द॑नाधृ॒ष्यम् । \newline
59. अ॒ना॒धृ॒ष्यम् दे॒वाना᳚म् दे॒वाना॑ मनाधृ॒ष्य म॑नाधृ॒ष्यम् दे॒वाना᳚म् । \newline
60. अ॒ना॒धृ॒ष्यमित्य॑ना - धृ॒ष्यम् । \newline
61. दे॒वाना॒ मोज॒ ओजो॑ दे॒वाना᳚म् दे॒वाना॒ मोजः॑ । \newline
62. ओज॒ इती त्योज॒ ओज॒ इति॑ । \newline

\textbf{Ghana Paata } \newline

1. आप॑तिः प्रा॒णम् प्रा॒ण माप॑ति॒ राप॑तिः प्रा॒ण मे॒वैव प्रा॒ण माप॑ति॒ राप॑तिः प्रा॒ण मे॒व । \newline
2. आप॑ति॒रित्या - प॒तिः॒ । \newline
3. प्रा॒ण मे॒वैव प्रा॒णम् प्रा॒ण मे॒व प्री॑णाति प्रीणा त्ये॒व प्रा॒णम् प्रा॒ण मे॒व प्री॑णाति । \newline
4. प्रा॒णमिति॑ प्र - अ॒नम् । \newline
5. ए॒व प्री॑णाति प्रीणा त्ये॒वैव प्री॑णाति॒ परि॑पतये॒ परि॑पतये प्रीणा त्ये॒वैव प्री॑णाति॒ परि॑पतये । \newline
6. प्री॒णा॒ति॒ परि॑पतये॒ परि॑पतये प्रीणाति प्रीणाति॒ परि॑पतय॒ इतीति॒ परि॑पतये प्रीणाति प्रीणाति॒ परि॑पतय॒ इति॑ । \newline
7. परि॑पतय॒ इतीति॒ परि॑पतये॒ परि॑पतय॒ इत्या॑हा॒हेति॒ परि॑पतये॒ परि॑पतय॒ इत्या॑ह । \newline
8. परि॑पतय॒ इति॒ परि॑ - प॒त॒ये॒ । \newline
9. इत्या॑हा॒हे तीत्या॑ह॒ मनो॒ मन॑ आ॒हे तीत्या॑ह॒ मनः॑ । \newline
10. आ॒ह॒ मनो॒ मन॑ आहाह॒ मनो॒ वै वै मन॑ आहाह॒ मनो॒ वै । \newline
11. मनो॒ वै वै मनो॒ मनो॒ वै परि॑पतिः॒ परि॑पति॒र् वै मनो॒ मनो॒ वै परि॑पतिः । \newline
12. वै परि॑पतिः॒ परि॑पति॒र् वै वै परि॑पति॒र् मनो॒ मनः॒ परि॑पति॒र् वै वै परि॑पति॒र् मनः॑ । \newline
13. परि॑पति॒र् मनो॒ मनः॒ परि॑पतिः॒ परि॑पति॒र् मन॑ ए॒वैव मनः॒ परि॑पतिः॒ परि॑पति॒र् मन॑ ए॒व । \newline
14. परि॑पति॒रिति॒ परि॑ - प॒तिः॒ । \newline
15. मन॑ ए॒वैव मनो॒ मन॑ ए॒व प्री॑णाति प्रीणा त्ये॒व मनो॒ मन॑ ए॒व प्री॑णाति । \newline
16. ए॒व प्री॑णाति प्रीणा त्ये॒वैव प्री॑णाति॒ तनू॒नप्त्रे॒ तनू॒नप्त्रे᳚ प्रीणा त्ये॒वैव प्री॑णाति॒ तनू॒नप्त्रे᳚ । \newline
17. प्री॒णा॒ति॒ तनू॒नप्त्रे॒ तनू॒नप्त्रे᳚ प्रीणाति प्रीणाति॒ तनू॒नप्त्र॒ इतीति॒ तनू॒नप्त्रे᳚ प्रीणाति प्रीणाति॒ तनू॒नप्त्र॒ इति॑ । \newline
18. तनू॒नप्त्र॒ इतीति॒ तनू॒नप्त्रे॒ तनू॒नप्त्र॒ इत्या॑हा॒हेति॒ तनू॒नप्त्रे॒ तनू॒नप्त्र॒ इत्या॑ह । \newline
19. तनू॒नप्त्र॒ इति॒ तनू᳚ - नप्त्रे᳚ । \newline
20. इत्या॑हा॒हे तीत्या॑ह त॒नुव॑ स्त॒नुव॑ आ॒हे तीत्या॑ह त॒नुवः॑ । \newline
21. आ॒ह॒ त॒नुव॑ स्त॒नुव॑ आहाह त॒नुवो॒ हि हि त॒नुव॑ आहाह त॒नुवो॒ हि । \newline
22. त॒नुवो॒ हि हि त॒नुव॑ स्त॒नुवो॒ हि ते ते हि त॒नुव॑ स्त॒नुवो॒ हि ते । \newline
23. हि ते ते हि हि ते ता स्ता स्ते हि हि ते ताः । \newline
24. ते ता स्ता स्ते ते ताः स॑म॒वाद्य॑न्त सम॒वाद्य॑न्त॒ ता स्ते ते ताः स॑म॒वाद्य॑न्त । \newline
25. ताः स॑म॒वाद्य॑न्त सम॒वाद्य॑न्त॒ ता स्ताः स॑म॒वाद्य॑न्त शाक्व॒राय॑ शाक्व॒राय॑ सम॒वाद्य॑न्त॒ ता स्ताः स॑म॒वाद्य॑न्त शाक्व॒राय॑ । \newline
26. स॒म॒वाद्य॑न्त शाक्व॒राय॑ शाक्व॒राय॑ सम॒वाद्य॑न्त सम॒वाद्य॑न्त शाक्व॒रायेतीति॑ शाक्व॒राय॑ सम॒वाद्य॑न्त सम॒वाद्य॑न्त शाक्व॒रायेति॑ । \newline
27. स॒म॒वाद्य॒न्तेति॑ सं - अ॒वाद्य॑न्त । \newline
28. शा॒क्व॒राये तीति॑ शाक्व॒राय॑ शाक्व॒राये त्या॑हा॒हेति॑ शाक्व॒राय॑ शाक्व॒राये त्या॑ह । \newline
29. इत्या॑हा॒हे तीत्या॑ह॒ शक्त्यै॒ शक्त्या॑ आ॒हे तीत्या॑ह॒ शक्त्यै᳚ । \newline
30. आ॒ह॒ शक्त्यै॒ शक्त्या॑ आहाह॒ शक्त्यै॒ हि हि शक्त्या॑ आहाह॒ शक्त्यै॒ हि । \newline
31. शक्त्यै॒ हि हि शक्त्यै॒ शक्त्यै॒ हि ते ते हि शक्त्यै॒ शक्त्यै॒ हि ते । \newline
32. हि ते ते हि हि ते ता स्ता स्ते हि हि ते ताः । \newline
33. ते ता स्ता स्ते ते ताः स॑म॒वाद्य॑न्त सम॒वाद्य॑न्त॒ ता स्ते ते ताः स॑म॒वाद्य॑न्त । \newline
34. ताः स॑म॒वाद्य॑न्त सम॒वाद्य॑न्त॒ ता स्ताः स॑म॒वाद्य॑न्त॒ शक्म॒ञ् छक्मन्᳚ थ्सम॒वाद्य॑न्त॒ ता स्ताः स॑म॒वाद्य॑न्त॒ शक्मन्न्॑ । \newline
35. स॒म॒वाद्य॑न्त॒ शक्म॒ञ् छक्मन्᳚ थ्सम॒वाद्य॑न्त सम॒वाद्य॑न्त॒ शक्म॒न् नोजि॑ष्ठा॒ यौजि॑ष्ठाय॒ शक्मन्᳚ थ्सम॒वाद्य॑न्त सम॒वाद्य॑न्त॒ शक्म॒न् नोजि॑ष्ठाय । \newline
36. स॒म॒वाद्य॒न्तेति॑ सं - अ॒वाद्य॑न्त । \newline
37. शक्म॒न् नोजि॑ष्ठा॒ यौजि॑ष्ठाय॒ शक्म॒ञ् छक्म॒न् नोजि॑ष्ठा॒ येतीत्योजि॑ष्ठाय॒ शक्म॒ञ् छक्म॒न् नोजि॑ष्ठा॒येति॑ । \newline
38. ओजि॑ष्ठा॒येती त्योजि॑ष्ठा॒ यौजि॑ष्ठा॒ येत्या॑ हा॒हे त्योजि॑ष्ठा॒ यौजि॑ष्ठा॒येत्या॑ह । \newline
39. इत्या॑हा॒हेती त्या॒हौजि॑ष्ठ॒ मोजि॑ष्ठ मा॒हे तीत्या॒ हौजि॑ष्ठम् । \newline
40. आ॒हौजि॑ष्ठ॒ मोजि॑ष्ठ माहा॒ हौजि॑ष्ठꣳ॒॒ हि ह्योजि॑ष्ठ माहा॒ हौजि॑ष्ठꣳ॒॒ हि । \newline
41. ओजि॑ष्ठꣳ॒॒ हि ह्योजि॑ष्ठ॒ मोजि॑ष्ठꣳ॒॒ हि ते ते ह्योजि॑ष्ठ॒ मोजि॑ष्ठꣳ॒॒ हि ते । \newline
42. हि ते ते हि हि ते तत् तत् ते हि हि ते तत् । \newline
43. ते तत् तत् ते ते तदा॒त्मन॑ आ॒त्मन॒ स्तत् ते ते तदा॒त्मनः॑ । \newline
44. तदा॒त्मन॑ आ॒त्मन॒ स्तत् तदा॒त्मनः॑ सम॒वाद्य॑न्त सम॒वाद्य॑न्ता॒ त्मन॒ स्तत् तदा॒त्मनः॑ सम॒वाद्य॑न्त । \newline
45. आ॒त्मनः॑ सम॒वाद्य॑न्त सम॒वाद्य॑न्ता॒ त्मन॑ आ॒त्मनः॑ सम॒वाद्य॒न्ता ना॑धृष्ट॒ मना॑धृष्टꣳ सम॒वाद्य॑न्ता॒ त्मन॑ आ॒त्मनः॑ सम॒वाद्य॒न्ता ना॑धृष्टम् । \newline
46. स॒म॒वाद्य॒न्ता ना॑धृष्ट॒ मना॑धृष्टꣳ सम॒वाद्य॑न्त सम॒वाद्य॒न्ता ना॑धृष्ट मस्य॒स्य ना॑धृष्टꣳ सम॒वाद्य॑न्त सम॒वाद्य॒न्ता ना॑धृष्ट मसि । \newline
47. स॒म॒वाद्य॒न्तेति॑ सं - अ॒वाद्य॑न्त । \newline
48. अना॑धृष्ट मस्य॒स्य ना॑धृष्ट॒ मना॑धृष्ट मस्य नाधृ॒ष्य म॑नाधृ॒ष्य म॒स्य ना॑धृष्ट॒ मना॑धृष्ट मस्य नाधृ॒ष्यम् । \newline
49. अना॑धृष्ट॒मित्यना᳚ - धृ॒ष्ट॒म् । \newline
50. अ॒स्य॒ ना॒धृ॒ष्य म॑नाधृ॒ष्य म॑स्यस्य नाधृ॒ष्य मिती त्य॑नाधृ॒ष्य म॑स्यस्य नाधृ॒ष्य मिति॑ । \newline
51. अ॒ना॒धृ॒ष्य मिती त्य॑नाधृ॒ष्य म॑नाधृ॒ष्य मित्या॑हा॒हे त्य॑नाधृ॒ष्य म॑नाधृ॒ष्य मित्या॑ह । \newline
52. अ॒ना॒धृ॒ष्यमित्य॑ना - धृ॒ष्यम् । \newline
53. इत्या॑हा॒हे तीत्या॒हा ना॑धृष्ट॒ मना॑धृष्ट मा॒हे तीत्या॒हा ना॑धृष्टम् । \newline
54. आ॒हा ना॑धृष्ट॒ मना॑धृष्ट माहा॒हा ना॑धृष्टꣳ॒॒ हि ह्यना॑धृष्ट माहा॒हा ना॑धृष्टꣳ॒॒ हि । \newline
55. अना॑धृष्टꣳ॒॒ हि ह्यना॑धृष्ट॒ मना॑धृष्टꣳ॒॒ ह्ये॑त दे॒त द्ध्यना॑धृष्ट॒ मना॑धृष्टꣳ॒॒ ह्ये॑तत् । \newline
56. अना॑धृष्ट॒मित्यना᳚ - धृ॒ष्ट॒म् । \newline
57. ह्ये॑ तदे॒तद्धि ह्ये॑त द॑नाधृ॒ष्य म॑नाधृ॒ष्य मे॒तद्धि ह्ये॑त द॑नाधृ॒ष्यम् । \newline
58. ए॒त द॑नाधृ॒ष्य म॑नाधृ॒ष्य मे॒त दे॒त द॑नाधृ॒ष्यम् दे॒वाना᳚म् दे॒वाना॑ मनाधृ॒ष्य मे॒त दे॒त द॑नाधृ॒ष्यम् दे॒वाना᳚म् । \newline
59. अ॒ना॒धृ॒ष्यम् दे॒वाना᳚म् दे॒वाना॑ मनाधृ॒ष्य म॑नाधृ॒ष्यम् दे॒वाना॒ मोज॒ ओजो॑ दे॒वाना॑ मनाधृ॒ष्य म॑नाधृ॒ष्यम् दे॒वाना॒ मोजः॑ । \newline
60. अ॒ना॒धृ॒ष्यमित्य॑ना - धृ॒ष्यम् । \newline
61. दे॒वाना॒ मोज॒ ओजो॑ दे॒वाना᳚म् दे॒वाना॒ मोज॒ इतीत्योजो॑ दे॒वाना᳚म् दे॒वाना॒ मोज॒ इति॑ । \newline
62. ओज॒ इती त्योज॒ ओज॒ इत्या॑हा॒हे त्योज॒ ओज॒ इत्या॑ह । \newline
\pagebreak
\markright{ TS 6.2.2.4  \hfill https://www.vedavms.in \hfill}

\section{ TS 6.2.2.4 }

\textbf{TS 6.2.2.4 } \newline
\textbf{Samhita Paata} \newline

इत्या॑ह दे॒वानाꣳ॒॒ ह्ये॑तदोजो॑ऽभिशस्ति॒पा अ॑नभिशस्ते॒न्यमित्या॑हा-भिशस्ति॒पा ह्ये॑तद॑ -नभिशस्ते॒न्यमनु॑ मे दी॒क्षां दी॒क्षाप॑ति-र्मन्यता॒मित्या॑ह यथाय॒जुरे॒वैतद्-घृ॒तं ॅवै दे॒वा वज्रं॑ कृ॒त्वा सोम॑मघ्न-न्नन्ति॒कमि॑व॒ खलु॒ वा अ॑स्यै॒तच्च॑रन्ति॒ यत् ता॑नून॒प्त्रेण॑ प्र॒च॑रन्त्यꣳ॒॒ शुरꣳ॑ शुस्ते देव सो॒माऽऽ *प्या॑यता॒मित्या॑ह॒ य- [  ] \newline

\textbf{Pada Paata} \newline

इति॑ । आ॒ह॒ । दे॒वाना᳚म् । हि । ए॒तत् । ओजः॑ । अ॒भि॒श॒स्ति॒पा इत्य॑भिशस्ति-पाः । अ॒न॒भि॒श॒स्ते॒न्यमित्य॑नभि - श॒स्ते॒न्यम् । इति॑ । आ॒ह॒ । अ॒भि॒श॒स्ति॒पा इत्य॑भिशस्ति - पाः । हि । ए॒तत् । अ॒न॒भि॒श॒स्ते॒न्यमित्य॑नभि - श॒स्ते॒न्यम् । अन्विति॑ । मे॒ । दी॒क्षाम् । दी॒क्षाप॑ति॒रिति॑ दी॒क्षा - प॒तिः॒ । म॒न्य॒ता॒म् । इति॑ । आ॒ह॒ । य॒था॒य॒जुरिति॑ यथा - य॒जुः । ए॒व । ए॒तत् । घृ॒तम् । वै । दे॒वाः । वज्र᳚म् । कृ॒त्वा । सोम᳚म् । अ॒घ्न॒न्न् । अ॒न्ति॒कम् । इ॒व॒ । खलु॑ । वै । अ॒स्य॒ । ए॒तत् । च॒र॒न्ति॒ । यत् । ता॒नू॒न॒प्त्रेणेति॑ तानू - न॒प्त्रेण॑ । प्र॒चर॒न्तीति॑ प्र - चर॑न्ति । अꣳ॒॒शुरꣳ॑शु॒रित्यꣳ॒॒शुः-अꣳ॒॒शुः॒ । ते॒ । दे॒व॒ । सो॒म॒ । एति॑ । प्या॒य॒ता॒म् । इति॑ । आ॒ह॒ । यत् ।  \newline


\textbf{Krama Paata} \newline

इत्या॑ह । आ॒ह॒ दे॒वाना᳚म् । दे॒वानाꣳ॒॒ हि । ह्ये॑तत् । ए॒तदोजः॑ । ओजो॑ऽभिशस्ति॒पाः । अ॒भि॒श॒स्ति॒पा अ॑नभिशस्ते॒न्यम् । अ॒भि॒श॒स्ति॒पा इत्य॑भिशस्ति - पाः । अ॒न॒भि॒श॒स्ते॒न्यमिति॑ । अ॒न॒भि॒श॒स्ते॒न्यमित्य॑नभि - श॒स्ते॒न्यम् । इत्या॑ह । आ॒हा॒भि॒श॒स्ति॒पाः । अ॒भि॒श॒स्ति॒पा हि । अ॒भि॒श॒स्ति॒पा इत्य॑भिशस्ति - पाः । ह्ये॑तत् । ए॒तद॑नभिशस्ते॒न्यम् । अ॒न॒भि॒श॒स्ते॒न्यमनु॑ । अ॒न॒भि॒श॒स्ते॒न्यमित्य॑नभि - श॒स्ते॒न्यम् । अनु॑ मे । मे॒ दी॒क्षाम् । दी॒क्षाम् दी॒क्षाप॑तिः । दी॒क्षाप॑तिर् मन्यताम् । दी॒क्षाप॑ति॒रिति॑ दी॒क्षा - प॒तिः॒ । म॒न्य॒ता॒मिति॑ । इत्या॑ह । आ॒ह॒ य॒था॒य॒जुः । य॒था॒य॒जुरे॒व । य॒था॒य॒जुरिति॑ यथा - य॒जुः । ए॒वैतत् । ए॒तद् घृ॒तम् । घृ॒तम् ॅवै । वै दे॒वाः । दे॒वा वज्र᳚म् । वज्र॑म् कृ॒त्वा । कृ॒त्वा सोम᳚म् । सोम॑मघ्नन्न् । अ॒घ्न॒न्न॒न्ति॒कम् । अ॒न्ति॒कमि॑व । इ॒व॒ खलु॑ । खलु॒ वै । वा अ॑स्य । अ॒स्यै॒तत् । ए॒तच् च॑रन्ति । च॒र॒न्ति॒ यत् । यत् ता॑नून॒प्त्रेण॑ । ता॒नू॒न॒प्त्रेण॑ प्र॒चर॑न्ति । ता॒नू॒न॒प्त्रेणेति॑ तानू - न॒प्त्रेण॑ । प्र॒चर॑न्त्यꣳ॒॒शुरꣳ॑शुः । प्र॒चर॒न्तीति॑ प्र - चर॑न्ति । अꣳ॒॒शुरꣳ॑शुस्ते । अꣳ॒॒शुरꣳ॑शु॒रित्यꣳ॒॒शुः - अꣳ॒॒शुः॒ । ते॒ दे॒व॒ । दे॒व॒ सो॒म॒ । सो॒मा । आ प्या॑यताम् । प्या॒य॒ता॒मिति॑ । इत्या॑ह । आ॒ह॒ यत् । यदे॒व \newline

\textbf{Jatai Paata} \newline

1. इत्या॑हा॒हे तीत्या॑ह । \newline
2. आ॒ह॒ दे॒वाना᳚म् दे॒वाना॑ माहाह दे॒वाना᳚म् । \newline
3. दे॒वानाꣳ॒॒ हि हि दे॒वाना᳚म् दे॒वानाꣳ॒॒ हि । \newline
4. ह्ये॑त दे॒त द्धि ह्ये॑तत् । \newline
5. ए॒त दोज॒ ओज॑ ए॒त दे॒त दोजः॑ । \newline
6. ओजो॑ ऽभिशस्ति॒पा अ॑भिशस्ति॒पा ओज॒ ओजो॑ ऽभिशस्ति॒पाः । \newline
7. अ॒भि॒श॒स्ति॒पा अ॑नभिशस्ते॒न्य म॑नभिशस्ते॒न्य म॑भिशस्ति॒पा अ॑भिशस्ति॒पा अ॑नभिशस्ते॒न्यम् । \newline
8. अ॒भि॒श॒स्ति॒पा इत्य॑भिशस्ति - पाः । \newline
9. अ॒न॒भि॒श॒स्ते॒न्य मितीत्य॑नभिशस्ते॒न्य म॑नभिशस्ते॒न्य मिति॑ । \newline
10. अ॒न॒भि॒श॒स्ते॒न्यमित्य॑नभि - श॒स्ते॒न्यम् । \newline
11. इत्या॑हा॒हे तीत्या॑ह । \newline
12. आ॒हा॒ भि॒श॒स्ति॒पा अ॑भिशस्ति॒पा आ॑हाहा भिशस्ति॒पाः । \newline
13. अ॒भि॒श॒स्ति॒पा हि ह्य॑भिशस्ति॒पा अ॑भिशस्ति॒पा हि । \newline
14. अ॒भि॒श॒स्ति॒पा इत्य॑भिशस्ति - पाः । \newline
15. ह्ये॑ तदे॒त द्धि ह्ये॑तत् । \newline
16. ए॒त द॑नभिशस्ते॒न्य म॑नभिशस्ते॒न्य मे॒त दे॒त द॑नभिशस्ते॒न्यम् । \newline
17. अ॒न॒भि॒श॒स्ते॒न्य मन् वन् व॑नभिशस्ते॒न्य म॑नभिशस्ते॒न्य मनु॑ । \newline
18. अ॒न॒भि॒श॒स्ते॒न्यमित्य॑नभि - श॒स्ते॒न्यम् । \newline
19. अनु॑ मे॒ मे ऽन्वनु॑ मे । \newline
20. मे॒ दी॒क्षाम् दी॒क्षाम् मे॑ मे दी॒क्षाम् । \newline
21. दी॒क्षाम् दी॒क्षाप॑तिर् दी॒क्षाप॑तिर् दी॒क्षाम् दी॒क्षाम् दी॒क्षाप॑तिः । \newline
22. दी॒क्षाप॑तिर् मन्यताम् मन्यताम् दी॒क्षाप॑तिर् दी॒क्षाप॑तिर् मन्यताम् । \newline
23. दी॒क्षाप॑ति॒रिति॑ दी॒क्षा - प॒तिः॒ । \newline
24. म॒न्य॒ता॒ मितीति॑ मन्यताम् मन्यता॒ मिति॑ । \newline
25. इत्या॑हा॒हे तीत्या॑ह । \newline
26. आ॒ह॒ य॒था॒य॒जुर् य॑थाय॒जु रा॑हाह यथाय॒जुः । \newline
27. य॒था॒य॒जु रे॒वैव य॑थाय॒जुर् य॑थाय॒जु रे॒व । \newline
28. य॒था॒य॒जुरिति॑ यथा - य॒जुः । \newline
29. ए॒वैत दे॒त दे॒वै वैतत् । \newline
30. ए॒तद् घृ॒तम् घृ॒त मे॒त दे॒तद् घृ॒तम् । \newline
31. घृ॒तं ॅवै वै घृ॒तम् घृ॒तं ॅवै । \newline
32. वै दे॒वा दे॒वा वै वै दे॒वाः । \newline
33. दे॒वा वज्रं॒ ॅवज्र॑म् दे॒वा दे॒वा वज्र᳚म् । \newline
34. वज्र॑म् कृ॒त्वा कृ॒त्वा वज्रं॒ ॅवज्र॑म् कृ॒त्वा । \newline
35. कृ॒त्वा सोमꣳ॒॒ सोम॑म् कृ॒त्वा कृ॒त्वा सोम᳚म् । \newline
36. सोम॑ मघ्नन् नघ्न॒न् थ्सोमꣳ॒॒ सोम॑ मघ्नन्न् । \newline
37. अ॒घ्न॒न् न॒न्ति॒क म॑न्ति॒क म॑घ्नन् नघ्नन् नन्ति॒कम् । \newline
38. अ॒न्ति॒क मि॑वे वान्ति॒क म॑न्ति॒क मि॑व । \newline
39. इ॒व॒ खलु॒ खल्वि॑वेव॒ खलु॑ । \newline
40. खलु॒ वै वै खलु॒ खलु॒ वै । \newline
41. वा अ॑स्यास्य॒ वै वा अ॑स्य । \newline
42. अ॒स्यै॒त दे॒त द॑स्या स्यै॒तत् । \newline
43. ए॒तच् च॑रन्ति चर न्त्ये॒त दे॒तच् च॑रन्ति । \newline
44. च॒र॒न्ति॒ यद् यच् च॑रन्ति चरन्ति॒ यत् । \newline
45. यत् ता॑नून॒प्त्रेण॑ तानून॒प्त्रेण॒ यद् यत् ता॑नून॒प्त्रेण॑ । \newline
46. ता॒नू॒न॒प्त्रेण॑ प्र॒चर॑न्ति प्र॒चर॑न्ति तानून॒प्त्रेण॑ तानून॒प्त्रेण॑ प्र॒चर॑न्ति । \newline
47. ता॒नू॒न॒प्त्रेणेति॑ तानू - न॒प्त्रेण॑ । \newline
48. प्र॒चर॑ न्त्यꣳ॒॒शुरꣳ॑शु रꣳ॒॒शुरꣳ॑शुः प्र॒चर॑न्ति प्र॒चर॑ न्त्यꣳ॒॒शुरꣳ॑शुः । \newline
49. प्र॒चर॒न्तीति॑ प्र - चर॑न्ति । \newline
50. अꣳ॒॒शुरꣳ॑शु स्ते ते अꣳ॒॒शुरꣳ॑शु रꣳ॒॒शुरꣳ॑शु स्ते । \newline
51. अꣳ॒॒शुरꣳ॑शु॒रित्यꣳ॒॒शुः - अꣳ॒॒शुः॒ । \newline
52. ते॒ दे॒व॒ दे॒व॒ ते॒ ते॒ दे॒व॒ । \newline
53. दे॒व॒ सो॒म॒ सो॒म॒ दे॒व॒ दे॒व॒ सो॒म॒ । \newline
54. सो॒मा सो॑म सो॒मा । \newline
55. आ प्या॑यताम् प्यायता॒ मा प्या॑यताम् । \newline
56. प्या॒य॒ता॒ मितीति॑ प्यायताम् प्यायता॒ मिति॑ । \newline
57. इत्या॑हा॒हे तीत्या॑ह । \newline
58. आ॒ह॒ यद् यदा॑हाह॒ यत् । \newline
59. यदे॒वैव यद् यदे॒व । \newline

\textbf{Ghana Paata } \newline

1. इत्या॑हा॒हे तीत्या॑ह दे॒वाना᳚म् दे॒वाना॑ मा॒हे तीत्या॑ह दे॒वाना᳚म् । \newline
2. आ॒ह॒ दे॒वाना᳚म् दे॒वाना॑ माहाह दे॒वानाꣳ॒॒ हि हि दे॒वाना॑ माहाह दे॒वानाꣳ॒॒ हि । \newline
3. दे॒वानाꣳ॒॒ हि हि दे॒वाना᳚म् दे॒वानाꣳ॒॒ ह्ये॑त दे॒तद्धि दे॒वाना᳚म् दे॒वानाꣳ॒॒ ह्ये॑तत् । \newline
4. ह्ये॑त दे॒तद्धि ह्ये॑त दोज॒ ओज॑ ए॒तद्धि ह्ये॑त दोजः॑ । \newline
5. ए॒त दोज॒ ओज॑ ए॒त दे॒त दोजो॑ ऽभिशस्ति॒पा अ॑भिशस्ति॒पा ओज॑ ए॒त दे॒त दोजो॑ ऽभिशस्ति॒पाः । \newline
6. ओजो॑ ऽभिशस्ति॒पा अ॑भिशस्ति॒पा ओज॒ ओजो॑ ऽभिशस्ति॒पा अ॑नभिशस्ते॒न्य म॑नभिशस्ते॒न्य म॑भिशस्ति॒पा ओज॒ ओजो॑ ऽभिशस्ति॒पा अ॑नभिशस्ते॒न्यम् । \newline
7. अ॒भि॒श॒स्ति॒पा अ॑नभिशस्ते॒न्य म॑नभिशस्ते॒न्य म॑भिशस्ति॒पा अ॑भिशस्ति॒पा अ॑नभिशस्ते॒न्य मितीत्य॑ नभिशस्ते॒न्य म॑भिशस्ति॒पा अ॑भिशस्ति॒पा अ॑नभिशस्ते॒न्य मिति॑ । \newline
8. अ॒भि॒श॒स्ति॒पा इत्य॑भिशस्ति - पाः । \newline
9. अ॒न॒भि॒श॒स्ते॒न्य मितीत्य॑ नभिशस्ते॒न्य म॑नभिशस्ते॒न्य मित्या॑हा॒हे त्य॑नभिशस्ते॒न्य म॑नभिशस्ते॒न्य मित्या॑ह । \newline
10. अ॒न॒भि॒श॒स्ते॒न्यमित्य॑नभि - श॒स्ते॒न्यम् । \newline
11. इत्या॑हा॒हे तीत्या॑हा भिशस्ति॒पा अ॑भिशस्ति॒पा आ॒हे तीत्या॑हा भिशस्ति॒पाः । \newline
12. आ॒हा॒ भि॒श॒स्ति॒पा अ॑भिशस्ति॒पा आ॑हाहा भिशस्ति॒पा हि ह्य॑भिशस्ति॒पा आ॑हाहा भिशस्ति॒पा हि । \newline
13. अ॒भि॒श॒स्ति॒पा हि ह्य॑भिशस्ति॒पा अ॑भिशस्ति॒पा ह्ये॑त दे॒त द्ध्य॑भिशस्ति॒पा अ॑भिशस्ति॒पा ह्ये॑तत् । \newline
14. अ॒भि॒श॒स्ति॒पा इत्य॑भिशस्ति - पाः । \newline
15. ह्ये॑त दे॒तद्धि ह्ये॑तद॑ नभिशस्ते॒न्य म॑नभिशस्ते॒न्य मे॒तद्धि ह्ये॑त द॑नभिशस्ते॒न्यम् । \newline
16. ए॒त द॑नभिशस्ते॒न्य म॑नभिशस्ते॒न्य मे॒त दे॒त द॑नभिशस्ते॒न्य मन् वन् व॑नभिशस्ते॒न्य मे॒त दे॒त द॑नभिशस्ते॒न्य मनु॑ । \newline
17. अ॒न॒भि॒श॒स्ते॒न्य मन् वन् व॑नभिशस्ते॒न्य म॑नभिशस्ते॒न्य मनु॑ मे॒ मे ऽन्व॑नभिशस्ते॒न्य म॑नभिशस्ते॒न्य मनु॑ मे । \newline
18. अ॒न॒भि॒श॒स्ते॒न्यमित्य॑नभि - श॒स्ते॒न्यम् । \newline
19. अनु॑ मे॒ मे ऽन्वनु॑ मे दी॒क्षाम् दी॒क्षाम् मे ऽन्वनु॑ मे दी॒क्षाम् । \newline
20. मे॒ दी॒क्षाम् दी॒क्षाम् मे॑ मे दी॒क्षाम् दी॒क्षाप॑तिर् दी॒क्षाप॑तिर् दी॒क्षाम् मे॑ मे दी॒क्षाम् दी॒क्षाप॑तिः । \newline
21. दी॒क्षाम् दी॒क्षाप॑तिर् दी॒क्षाप॑तिर् दी॒क्षाम् दी॒क्षाम् दी॒क्षाप॑तिर् मन्यताम् मन्यताम् दी॒क्षाप॑तिर् दी॒क्षाम् दी॒क्षाम् दी॒क्षाप॑तिर् मन्यताम् । \newline
22. दी॒क्षाप॑तिर् मन्यताम् मन्यताम् दी॒क्षाप॑तिर् दी॒क्षाप॑तिर् मन्यता॒ मितीति॑ मन्यताम् दी॒क्षाप॑तिर् दी॒क्षाप॑तिर् मन्यता॒ मिति॑ । \newline
23. दी॒क्षाप॑ति॒रिति॑ दी॒क्षा - प॒तिः॒ । \newline
24. म॒न्य॒ता॒ मितीति॑ मन्यताम् मन्यता॒ मित्या॑ हा॒हेति॑ मन्यताम् मन्यता॒ मित्या॑ह । \newline
25. इत्या॑हा॒हे तीत्या॑ह यथाय॒जुर् य॑थाय॒जु रा॒हे तीत्या॑ह यथाय॒जुः । \newline
26. आ॒ह॒ य॒था॒य॒जुर् य॑थाय॒जु रा॑हाह यथाय॒जु रे॒वैव य॑थाय॒जु रा॑हाह यथाय॒जु रे॒व । \newline
27. य॒था॒य॒जु रे॒वैव य॑थाय॒जुर् य॑थाय॒जु रे॒वैत दे॒त दे॒व य॑थाय॒जुर् य॑थाय॒जु रे॒वैतत् । \newline
28. य॒था॒य॒जुरिति॑ यथा - य॒जुः । \newline
29. ए॒वैत दे॒त दे॒वै वैतद् घृ॒तम् घृ॒त मे॒त दे॒वै वैतद् घृ॒तम् । \newline
30. ए॒तद् घृ॒तम् घृ॒त मे॒त दे॒तद् घृ॒तं ॅवै वै घृ॒त मे॒त दे॒तद् घृ॒तं ॅवै । \newline
31. घृ॒तं ॅवै वै घृ॒तम् घृ॒तं ॅवै दे॒वा दे॒वा वै घृ॒तम् घृ॒तं ॅवै दे॒वाः । \newline
32. वै दे॒वा दे॒वा वै वै दे॒वा वज्रं॒ ॅवज्र॑म् दे॒वा वै वै दे॒वा वज्र᳚म् । \newline
33. दे॒वा वज्रं॒ ॅवज्र॑म् दे॒वा दे॒वा वज्र॑म् कृ॒त्वा कृ॒त्वा वज्र॑म् दे॒वा दे॒वा वज्र॑म् कृ॒त्वा । \newline
34. वज्र॑म् कृ॒त्वा कृ॒त्वा वज्रं॒ ॅवज्र॑म् कृ॒त्वा सोमꣳ॒॒ सोम॑म् कृ॒त्वा वज्रं॒ ॅवज्र॑म् कृ॒त्वा सोम᳚म् । \newline
35. कृ॒त्वा सोमꣳ॒॒ सोम॑म् कृ॒त्वा कृ॒त्वा सोम॑ मघ्नन् नघ्न॒न् थ्सोम॑म् कृ॒त्वा कृ॒त्वा सोम॑ मघ्नन्न् । \newline
36. सोम॑ मघ्नन् नघ्न॒न् थ्सोमꣳ॒॒ सोम॑ मघ्नन् नन्ति॒क म॑न्ति॒क म॑घ्न॒न् थ्सोमꣳ॒॒ सोम॑ मघ्नन् नन्ति॒कम् । \newline
37. अ॒घ्न॒न् न॒न्ति॒क म॑न्ति॒क म॑घ्नन् नघ्नन् नन्ति॒क मि॑वे वान्ति॒क म॑घ्नन् नघ्नन् नन्ति॒क मि॑व । \newline
38. अ॒न्ति॒क मि॑वे वान्ति॒क म॑न्ति॒क मि॑व॒ खलु॒ खल्वि॑ वान्ति॒क म॑न्ति॒क मि॑व॒ खलु॑ । \newline
39. इ॒व॒ खलु॒ खल्वि॑वेव॒ खलु॒ वै वै खल्वि॑वेव॒ खलु॒ वै । \newline
40. खलु॒ वै वै खलु॒ खलु॒ वा अ॑स्यास्य॒ वै खलु॒ खलु॒ वा अ॑स्य । \newline
41. वा अ॑स्यास्य॒ वै वा अ॑स्यै॒त दे॒त द॑स्य॒ वै वा अ॑स्यै॒तत् । \newline
42. अ॒स्यै॒त दे॒त द॑स्या स्यै॒तच् च॑रन्ति चरन् त्ये॒त द॑स्या स्यै॒तच् च॑रन्ति । \newline
43. ए॒तच् च॑रन्ति चरन् त्ये॒त दे॒तच् च॑रन्ति॒ यद् यच् च॑र न्त्ये॒त दे॒तच् च॑रन्ति॒ यत् । \newline
44. च॒र॒न्ति॒ यद् यच् च॑रन्ति चरन्ति॒ यत् ता॑नून॒प्त्रेण॑ तानून॒प्त्रेण॒ यच् च॑रन्ति चरन्ति॒ यत् ता॑नून॒प्त्रेण॑ । \newline
45. यत् ता॑नून॒प्त्रेण॑ तानून॒प्त्रेण॒ यद् यत् ता॑नून॒प्त्रेण॑ प्र॒चर॑न्ति प्र॒चर॑न्ति तानून॒प्त्रेण॒ यद् यत् ता॑नून॒प्त्रेण॑ प्र॒चर॑न्ति । \newline
46. ता॒नू॒न॒प्त्रेण॑ प्र॒चर॑न्ति प्र॒चर॑न्ति तानून॒प्त्रेण॑ तानून॒प्त्रेण॑ प्र॒चर॑ न्त्यꣳ॒॒शुरꣳ॑शु
रꣳ॒॒शुरꣳ॑शुः प्र॒चर॑न्ति तानून॒प्त्रेण॑ तानून॒प्त्रेण॑ प्र॒चर॑ न्त्यꣳ॒॒शुरꣳ॑शुः । \newline
47. ता॒नू॒न॒प्त्रेणेति॑ तानू - न॒प्त्रेण॑ । \newline
48. प्र॒चर॑ न्त्यꣳ॒॒शुरꣳ॑शु रꣳ॒॒शुरꣳ॑शुः प्र॒चर॑न्ति प्र॒चर॑ न्त्यꣳ॒॒शुरꣳ॑शु स्ते ते 
अꣳ॒॒शुरꣳ॑शुः प्र॒चर॑न्ति प्र॒चर॑ न्त्यꣳ॒॒शुरꣳ॑शु स्ते । \newline
49. प्र॒चर॒न्तीति॑ प्र - चर॑न्ति । \newline
50. अꣳ॒॒शुरꣳ॑शु स्ते ते अꣳ॒॒शुरꣳ॑शु रꣳ॒॒शुरꣳ॑शु स्ते देव देव ते अꣳ॒॒शुरꣳ॑शु 
रꣳ॒॒शुरꣳ॑शु स्ते देव । \newline
51. अꣳ॒॒शुरꣳ॑शु॒रित्यꣳ॒॒शुः - अꣳ॒॒शुः॒ । \newline
52. ते॒ दे॒व॒ दे॒व॒ ते॒ ते॒ दे॒व॒ सो॒म॒ सो॒म॒ दे॒व॒ ते॒ ते॒ दे॒व॒ सो॒म॒ । \newline
53. दे॒व॒ सो॒म॒ सो॒म॒ दे॒व॒ दे॒व॒ सो॒मा सो॑म देव देव सो॒मा । \newline
54. सो॒मा सो॑म सो॒मा प्या॑यताम् प्यायता॒ मा सो॑म सो॒मा प्या॑यताम् । \newline
55. आ प्या॑यताम् प्यायता॒ मा प्या॑यता॒ मितीति॑ प्यायता॒ मा प्या॑यता॒ मिति॑ । \newline
56. प्या॒य॒ता॒ मितीति॑ प्यायताम् प्यायता॒ मित्या॑हा॒हेति॑ प्यायताम् प्यायता॒ मित्या॑ह । \newline
57. इत्या॑हा॒हे तीत्या॑ह॒ यद् यदा॒हे तीत्या॑ह॒ यत् । \newline
58. आ॒ह॒ यद् यदा॑ हाह॒ यदे॒ वैव यदा॑ हाह॒ यदे॒व । \newline
59. यदे॒ वैव यद् यदे॒ वास्या᳚ स्यै॒व यद् यदे॒ वास्य॑ । \newline
\pagebreak
\markright{ TS 6.2.2.5  \hfill https://www.vedavms.in \hfill}

\section{ TS 6.2.2.5 }

\textbf{TS 6.2.2.5 } \newline
\textbf{Samhita Paata} \newline

-दे॒वास्या॑पुवा॒यते॒ यन्मीय॑ते॒-तदे॒वास्यै॒तेनाऽऽ *प्या॑यय॒त्या तुभ्य॒मिन्द्रः॑ प्यायता॒मा त्वमिन्द्रा॑य प्याय॒स्वेत्या॑-हो॒भावे॒वेन्द्रं॑ च॒ सोमं॒ चाऽऽ*प्या॑यय॒त्या प्या॑यय॒ सखी᳚न्थ् स॒न्या मे॒धयेत्या॑ह॒र्त्विजो॒ वा अ॑स्य॒ सखा॑य॒स्ताने॒वाऽऽ*प्या॑ययति स्व॒स्ति ते॑ देव सोम सु॒त्याम॑शी॒ये- [  ] \newline

\textbf{Pada Paata} \newline

ए॒व । अ॒स्य॒ । अ॒पु॒वा॒यते᳚ । यत् । मीय॑ते । तत् । ए॒व । अ॒स्य॒ । ए॒तेन॑ । एति॑ । प्या॒य॒य॒ति॒ । एति॑ । तुभ्य᳚म् । इन्द्रः॑ । प्या॒य॒ता॒म् । एति॑ । त्वम् । इन्द्रा॑य । प्या॒य॒स्व॒ । इति॑ । आ॒ह॒ । उ॒भौ । ए॒व । इन्द्र᳚म् । च॒ । सोम᳚म् । च॒ । एति॑ । प्या॒य॒य॒ति॒ । एति॑ । प्या॒य॒य॒ । सखीन्॑ । स॒न्या । मे॒धया᳚ । इति॑ । आ॒ह॒ । ऋ॒त्विजः॑ । वै । अ॒स्य॒ । सखा॑यः । तान् । ए॒व । एति॑ । प्या॒य॒य॒ति॒ । स्व॒स्ति । ते॒ । दे॒व॒ । सो॒म॒ । सु॒त्याम् । अ॒शी॒य॒ ।  \newline


\textbf{Krama Paata} \newline

ए॒वास्य॑ । अ॒स्या॒पु॒वा॒यते᳚ । अ॒पु॒वा॒यते॒ यत् । यन् मीय॑ते । मीय॑ते॒ तत् । तदे॒व । ए॒वास्य॑ । अ॒स्यै॒तेन॑ । ए॒तेना । आ प्या॑ययति । प्या॒य॒य॒त्या । आ तुभ्य᳚म् । तुभ्य॒मिन्द्रः॑ । इन्द्रः॑ प्यायताम् । प्या॒य॒ता॒मा । आ त्वम् । त्वमिन्द्रा॑य । इन्द्रा॑य प्यायस्व । प्या॒य॒स्वेति॑ । इत्या॑ह । आ॒हो॒भौ । उ॒भावे॒व । ए॒वेन्द्र᳚म् । इन्द्र॑म् च । च॒ सोम᳚म् । सोम॑म् च । चा । आ प्या॑ययति । प्या॒य॒य॒त्या । आ प्या॑यय । प्या॒य॒य॒ सखीन्॑ । सखी᳚न्थ् स॒न्या । स॒न्या मे॒धया᳚ । मे॒धयेति॑ । इत्या॑ह । आ॒ह॒र्त्विजः॑ । ऋ॒त्विजो॒ वै । वा अ॑स्य । अ॒स्य॒ सखा॑यः । सखा॑य॒स्तान् । ताने॒व । ए॒वा । आ प्या॑ययति । प्या॒य॒य॒ति॒ स्व॒स्ति । स्व॒स्ति ते᳚ । ते॒ दे॒व॒ । दे॒व॒ सो॒म॒ । सो॒म॒ सु॒त्याम् । सु॒त्याम॑शीय । अ॒शी॒येति॑ \newline

\textbf{Jatai Paata} \newline

1. ए॒वास्या᳚ स्यै॒वै वास्य॑ । \newline
2. अ॒स्या॒ पु॒वा॒यते॑ ऽपुवा॒यते᳚ ऽस्या स्यापुवा॒यते᳚ । \newline
3. अ॒पु॒वा॒यते॒ यद् यद॑पुवा॒यते॑ ऽपुवा॒यते॒ यत् । \newline
4. यन् मीय॑ते॒ मीय॑ते॒ यद् यन् मीय॑ते । \newline
5. मीय॑ते॒ तत् तन् मीय॑ते॒ मीय॑ते॒ तत् । \newline
6. तदे॒वैव तत् तदे॒व । \newline
7. ए॒वास्या᳚ स्यै॒वैवास्य॑ । \newline
8. अ॒स्यै॒ते नै॒तेना᳚ स्या स्यै॒तेन॑ । \newline
9. ए॒ते नैते नै॒तेना । \newline
10. आ प्या॑ययति प्यायय॒त्या प्या॑ययति । \newline
11. प्या॒य॒य॒त्या प्या॑ययति प्यायय॒त्या । \newline
12. आ तुभ्य॒म् तुभ्य॒ मा तुभ्य᳚म् । \newline
13. तुभ्य॒ मिन्द्र॒ इन्द्र॒ स्तुभ्य॒म् तुभ्य॒ मिन्द्रः॑ । \newline
14. इन्द्रः॑ प्यायताम् प्यायता॒ मिन्द्र॒ इन्द्रः॑ प्यायताम् । \newline
15. प्या॒य॒ता॒ मा प्या॑यताम् प्यायता॒ मा । \newline
16. आ त्वम् त्व मा त्वम् । \newline
17. त्व मिन्द्रा॒ येन्द्रा॑य॒ त्वम् त्व मिन्द्रा॑य । \newline
18. इन्द्रा॑य प्यायस्व प्याय॒ स्वेन्द्रा॒ येन्द्रा॑य प्यायस्व । \newline
19. प्या॒य॒स्वे तीति॑ प्यायस्व प्याय॒स्वेति॑ । \newline
20. इत्या॑हा॒हे तीत्या॑ह । \newline
21. आ॒हो॒भा वु॒भा वा॑हाहो॒भौ । \newline
22. उ॒भा वे॒वैवोभा वु॒भा वे॒व । \newline
23. ए॒वेन्द्र॒ मिन्द्र॑ मे॒वैवेन्द्र᳚म् । \newline
24. इन्द्र॑म् च॒ चेन्द्र॒ मिन्द्र॑म् च । \newline
25. च॒ सोमꣳ॒॒ सोम॑म् च च॒ सोम᳚म् । \newline
26. सोम॑म् च च॒ सोमꣳ॒॒ सोम॑म् च । \newline
27. चा च॒ चा । \newline
28. आ प्या॑ययति प्यायय॒त्या प्या॑ययति । \newline
29. प्या॒य॒य॒त्या प्या॑ययति प्यायय॒त्या । \newline
30. आ प्या॑यय प्याय॒या प्या॑यय । \newline
31. प्या॒य॒य॒ सखी॒न् थ्सखी᳚न् प्यायय प्यायय॒ सखीन्॑ । \newline
32. सखी᳚न् थ्स॒न्या स॒न्या सखी॒न् थ्सखी᳚न् थ्स॒न्या । \newline
33. स॒न्या मे॒धया॑ मे॒धया॑ स॒न्या स॒न्या मे॒धया᳚ । \newline
34. मे॒धयेतीति॑ मे॒धया॑ मे॒धयेति॑ । \newline
35. इत्या॑हा॒हे तीत्या॑ह । \newline
36. आ॒ह॒ र्‌त्विज॑ ऋ॒त्विज॑ आहाह॒ र्‌त्विजः॑ । \newline
37. ऋ॒त्विजो॒ वै वा ऋ॒त्विज॑ ऋ॒त्विजो॒ वै । \newline
38. वा अ॑स्यास्य॒ वै वा अ॑स्य । \newline
39. अ॒स्य॒ सखा॑यः॒ सखा॑यो ऽस्यास्य॒ सखा॑यः । \newline
40. सखा॑य॒ स्ताꣳ स्तान् थ्सखा॑यः॒ सखा॑य॒ स्तान् । \newline
41. ता ने॒वैव ताꣳ स्ता ने॒व । \newline
42. ए॒वै वैवा । \newline
43. आ प्या॑ययति प्यायय॒त्या प्या॑ययति । \newline
44. प्या॒य॒य॒ति॒ स्व॒स्ति स्व॒स्ति प्या॑ययति प्याययति स्व॒स्ति । \newline
45. स्व॒स्ति ते॑ ते स्व॒स्ति स्व॒स्ति ते᳚ । \newline
46. ते॒ दे॒व॒ दे॒व॒ ते॒ ते॒ दे॒व॒ । \newline
47. दे॒व॒ सो॒म॒ सो॒म॒ दे॒व॒ दे॒व॒ सो॒म॒ । \newline
48. सो॒म॒ सु॒त्याꣳ सु॒त्याꣳ सो॑म सोम सु॒त्याम् । \newline
49. सु॒त्या म॑शीया शीय सु॒त्याꣳ सु॒त्या म॑शीय । \newline
50. अ॒शी॒येती त्य॑शीया शी॒येति॑ । \newline

\textbf{Ghana Paata } \newline

1. ए॒वा स्या᳚ स्यै॒वैवास्या॑ पुवा॒यते॑ ऽपुवा॒यते᳚ ऽस्यै॒वै वास्या॑ पुवा॒यते᳚ । \newline
2. अ॒स्या॒ पु॒वा॒यते॑ ऽपुवा॒यते᳚ ऽस्यास्या पुवा॒यते॒ यद् यद॑पुवा॒यते᳚ ऽस्यास्या पुवा॒यते॒ यत् । \newline
3. अ॒पु॒वा॒यते॒ यद् यद॑पुवा॒यते॑ ऽपुवा॒यते॒ यन् मीय॑ते॒ मीय॑ते॒ यद॑पुवा॒यते॑ ऽपुवा॒यते॒ यन् मीय॑ते । \newline
4. यन् मीय॑ते॒ मीय॑ते॒ यद् यन् मीय॑ते॒ तत् तन् मीय॑ते॒ यद् यन् मीय॑ते॒ तत् । \newline
5. मीय॑ते॒ तत् तन् मीय॑ते॒ मीय॑ते॒ तदे॒ वैव तन् मीय॑ते॒ मीय॑ते॒ तदे॒व । \newline
6. तदे॒वैव तत् तदे॒ वास्या᳚ स्यै॒व तत् तदे॒ वास्य॑ । \newline
7. ए॒वा स्या᳚ स्यै॒वै वास्यै॒ते नै॒तेना᳚ स्यै॒वै वास्यै॒तेन॑ । \newline
8. अ॒स्यै॒ते नै॒तेना᳚स्या स्यै॒ते नैतेना᳚ स्यास्यै॒ तेना । \newline
9. ए॒ते नैते नै॒तेना प्या॑ययति प्यायय॒ त्यैते नै॒तेना प्या॑ययति । \newline
10. आ प्या॑ययति प्यायय॒त्या प्या॑यय॒त्या प्या॑यय॒त्या प्या॑यय॒त्या । \newline
11. प्या॒य॒य॒त्या प्या॑ययति प्यायय॒त्या तुभ्य॒म् तुभ्य॒ मा प्या॑ययति प्यायय॒त्या तुभ्य᳚म् । \newline
12. आ तुभ्य॒म् तुभ्य॒ मा तुभ्य॒ मिन्द्र॒ इन्द्र॒ स्तुभ्य॒ मा तुभ्य॒ मिन्द्रः॑ । \newline
13. तुभ्य॒ मिन्द्र॒ इन्द्र॒ स्तुभ्य॒म् तुभ्य॒ मिन्द्रः॑ प्यायताम् प्यायता॒ मिन्द्र॒ स्तुभ्य॒म् तुभ्य॒ मिन्द्रः॑ प्यायताम् । \newline
14. इन्द्रः॑ प्यायताम् प्यायता॒ मिन्द्र॒ इन्द्रः॑ प्यायता॒ मा प्या॑यता॒ मिन्द्र॒ इन्द्रः॑ प्यायता॒ मा । \newline
15. प्या॒य॒ता॒ मा प्या॑यताम् प्यायता॒ मा त्वम् त्व मा प्या॑यताम् प्यायता॒ मा त्वम् । \newline
16. आ त्वम् त्व मा त्व मिन्द्रा॒ येन्द्रा॑य॒ त्व मा त्व मिन्द्रा॑य । \newline
17. त्व मिन्द्रा॒ येन्द्रा॑य॒ त्वम् त्व मिन्द्रा॑य प्यायस्व प्याय॒ स्वेन्द्रा॑य॒ त्वम् त्व मिन्द्रा॑य प्यायस्व । \newline
18. इन्द्रा॑य प्यायस्व प्याय॒ स्वेन्द्रा॒ येन्द्रा॑य प्याय॒स्वे तीति॑ प्याय॒स्वेन्द्रा॒ येन्द्रा॑य प्याय॒स्वेति॑ । \newline
19. प्या॒य॒स्वे तीति॑ प्यायस्व प्याय॒स्वे त्या॑हा॒हेति॑ प्यायस्व प्याय॒स्वे त्या॑ह । \newline
20. इत्या॑हा॒हे तीत्या॑ हो॒भा वु॒भा वा॒हे तीत्या॑ हो॒भौ । \newline
21. आ॒हो॒भा वु॒भा वा॑हा हो॒भा वे॒वै वोभा वा॑हा हो॒भा वे॒व । \newline
22. उ॒भा वे॒वै वोभा वु॒भा वे॒वेन्द्र॒ मिन्द्र॑ मे॒वोभा वु॒भा वे॒वेन्द्र᳚म् । \newline
23. ए॒वेन्द्र॒ मिन्द्र॑ मे॒वैवेन्द्र॑म् च॒ चेन्द्र॑ मे॒वै वेन्द्र॑म् च । \newline
24. इन्द्र॑म् च॒ चेन्द्र॒ मिन्द्र॑म् च॒ सोमꣳ॒॒ सोम॒म् चेन्द्र॒ मिन्द्र॑म् च॒ सोम᳚म् । \newline
25. च॒ सोमꣳ॒॒ सोम॑म् च च॒ सोम॑म् च च॒ सोम॑म् च च॒ सोम॑म् च । \newline
26. सोम॑म् च च॒ सोमꣳ॒॒ सोम॒म् चा च॒ सोमꣳ॒॒ सोम॒म् चा । \newline
27. चा च॒ चा प्या॑ययति प्यायय॒त्या च॒ चा प्या॑ययति । \newline
28. आ प्या॑ययति प्यायय॒त्या प्या॑यय॒त्या प्या॑यय॒त्या प्या॑यय॒त्या । \newline
29. प्या॒य॒य॒त्या प्या॑ययति प्यायय॒त्या प्या॑यय प्याय॒या प्या॑ययति प्यायय॒त्या प्या॑यय । \newline
30. आ प्या॑यय प्याय॒या प्या॑यय॒ सखी॒न् थ्सखी᳚न् प्याय॒या प्या॑यय॒ सखीन्॑ । \newline
31. प्या॒य॒य॒ सखी॒न् थ्सखी᳚न् प्यायय प्यायय॒ सखी᳚न् थ्स॒न्या स॒न्या सखी᳚न् प्यायय प्यायय॒ सखी᳚न् थ्स॒न्या । \newline
32. सखी᳚न् थ्स॒न्या स॒न्या सखी॒न् थ्सखी᳚न् थ्स॒न्या मे॒धया॑ मे॒धया॑ स॒न्या सखी॒न् थ्सखी᳚न् थ्स॒न्या मे॒धया᳚ । \newline
33. स॒न्या मे॒धया॑ मे॒धया॑ स॒न्या स॒न्या मे॒ध येतीति॑ मे॒धया॑ स॒न्या स॒न्या मे॒ध येति॑ । \newline
34. मे॒ध येतीति॑ मे॒धया॑ मे॒धये त्या॑हा॒हे ति॑ मे॒धया॑ मे॒धये त्या॑ह । \newline
35. इत्या॑हा॒हे तीत्या॑ह॒ र्‌त्विज॑ ऋ॒त्विज॑ आ॒हे तीत्या॑ह॒ र्‌त्विजः॑ । \newline
36. आ॒ह॒ र्‌त्विज॑ ऋ॒त्विज॑ आहाह॒ र्‌त्विजो॒ वै वा ऋ॒त्विज॑ आहाह॒ र्‌त्विजो॒ वै । \newline
37. ऋ॒त्विजो॒ वै वा ऋ॒त्विज॑ ऋ॒त्विजो॒ वा अ॑स्यास्य॒ वा ऋ॒त्विज॑ ऋ॒त्विजो॒ वा अ॑स्य । \newline
38. वा अ॑स्यास्य॒ वै वा अ॑स्य॒ सखा॑यः॒ सखा॑यो ऽस्य॒ वै वा अ॑स्य॒ सखा॑यः । \newline
39. अ॒स्य॒ सखा॑यः॒ सखा॑यो ऽस्यास्य॒ सखा॑य॒ स्ताꣳ स्तान् थ्सखा॑यो ऽस्यास्य॒ सखा॑य॒ स्तान् । \newline
40. सखा॑य॒ स्ताꣳ स्तान् थ्सखा॑यः॒ सखा॑य॒ स्ताने॒वैव तान् थ्सखा॑यः॒ सखा॑य॒ स्ताने॒व । \newline
41. ताने॒ वैव ताꣳ स्ताने॒ वैव ताꣳ स्ताने॒वा । \newline
42. ए॒वै वैवा प्या॑ययति प्यायय॒ त्यैवैवा प्या॑ययति । \newline
43. आ प्या॑ययति प्यायय॒त्या प्या॑ययति स्व॒स्ति स्व॒स्ति प्या॑यय॒त्या प्या॑ययति स्व॒स्ति । \newline
44. प्या॒य॒य॒ति॒ स्व॒स्ति स्व॒स्ति प्या॑ययति प्याययति स्व॒स्ति ते॑ ते स्व॒स्ति प्या॑ययति प्याययति स्व॒स्ति ते᳚ । \newline
45. स्व॒स्ति ते॑ ते स्व॒स्ति स्व॒स्ति ते॑ देव देव ते स्व॒स्ति स्व॒स्ति ते॑ देव । \newline
46. ते॒ दे॒व॒ दे॒व॒ ते॒ ते॒ दे॒व॒ सो॒म॒ सो॒म॒ दे॒व॒ ते॒ ते॒ दे॒व॒ सो॒म॒ । \newline
47. दे॒व॒ सो॒म॒ सो॒म॒ दे॒व॒ दे॒व॒ सो॒म॒ सु॒त्याꣳ सु॒त्याꣳ सो॑म देव देव सोम सु॒त्याम् । \newline
48. सो॒म॒ सु॒त्याꣳ सु॒त्याꣳ सो॑म सोम सु॒त्या म॑शीया शीय सु॒त्याꣳ सो॑म सोम सु॒त्या म॑शीय । \newline
49. सु॒त्या म॑शीया शीय सु॒त्याꣳ सु॒त्या म॑शी॒येती त्य॑शीय सु॒त्याꣳ सु॒त्या म॑शी॒येति॑ । \newline
50. अ॒शी॒ये तीत्य॑शीया शी॒ये त्या॑हा॒हे त्य॑शीया शी॒ये त्या॑ह । \newline
\pagebreak
\markright{ TS 6.2.2.6  \hfill https://www.vedavms.in \hfill}

\section{ TS 6.2.2.6 }

\textbf{TS 6.2.2.6 } \newline
\textbf{Samhita Paata} \newline

-त्या॑हा॒ ऽऽ*शिष॑मे॒वैतामा शा᳚स्ते॒ प्र वा ए॒ते᳚ऽस्मा-ल्लो॒काच्च्य॑वन्ते॒ ये सोम॑मा-प्या॒यय॑न्त्य-न्तरिक्षदेव॒त्यो॑ हि सोम॒ आप्या॑यित॒ एष्टा॒ रायः॒ प्रेषे भगा॒येत्या॑ह॒ द्यावा॑पृथि॒वीभ्या॑मे॒व न॑म॒स्कृत्या॒स्मिन् ॅलो॒के प्रति॑ तिष्ठन्ति देवासु॒राः संॅय॑त्ता आस॒न् ते दे॒वा बिभ्य॑तो॒ऽग्निं प्रावि॑श॒न् तस्मा॑दाहुर॒ग्निः सर्वा॑ दे॒वता॒ इति॒ ते᳚- [  ] \newline

\textbf{Pada Paata} \newline

इति॑ । आ॒ह॒ । आ॒शिष॒मित्या᳚-शिष᳚म् । ए॒व । ए॒ताम् । एति॑ । शा॒स्ते॒ । प्रेति॑ । वै । ए॒ते । अ॒स्मात् । लो॒कात् । च्य॒व॒न्ते॒ । ये । सोम᳚म् । आ॒प्या॒यय॒न्तीत्या᳚-प्या॒यय॑न्ति । अ॒न्त॒रि॒क्ष॒दे॒व॒त्य॑ इत्य॑न्तरिक्ष-दे॒व॒त्यः॑ । हि । सोमः॑ । आप्या॑यित॒ इत्या - प्या॒यि॒तः॒ । एष्टः॑ । रायः॑ । प्रेति॑ । इ॒षे । भगा॑य । इति॑ । आ॒ह॒ । द्यावा॑पृथि॒वीभ्या॒मिति॒ द्यावा᳚ - पृ॒थि॒वीभ्या᳚म् । ए॒व । न॒म॒स्कृत्येति॑ नमः - कृत्य॑ । अ॒स्मिन्न् । लो॒के । प्रतीति॑ । ति॒ष्ठ॒न्ति॒ । दे॒वा॒सु॒रा इति॑ देव - अ॒सु॒राः । संॅय॑त्ता॒ इति॒ सं - य॒त्ताः॒ । आ॒स॒न्न् । ते । दे॒वाः । बिभ्य॑तः । अ॒ग्निम् । प्रेति॑ । अ॒वि॒श॒न्न् । तस्मा᳚त् । आ॒हुः॒ । अ॒ग्निः । सर्वाः᳚ । दे॒वताः᳚ । इति॑ । ते ।  \newline


\textbf{Krama Paata} \newline

इत्या॑ह । आ॒हा॒शिष᳚म् । आ॒शिष॑मे॒व । आ॒शिष॒मित्या᳚ - शिष᳚म् । ए॒वैताम् । ए॒तामा । आ शा᳚स्ते । शा॒स्ते॒ प्र । प्र वै । वा ए॒ते । ए॒ते᳚ऽस्मात् । अ॒स्माल्लो॒कात् । लो॒काच् च्य॑वन्ते । च्य॒व॒न्ते॒ ये । ये सोम᳚म् । सोम॑माप्या॒यय॑न्ति । आ॒प्या॒यय॑न्त्यन्तरिक्षदेव॒त्यः॑ । आ॒प्या॒यय॒न्तीत्या᳚ - प्या॒यय॑न्ति । अ॒न्त॒रि॒क्ष॒दे॒व॒त्यो॑ हि । अ॒न्त॒रि॒क्ष॒दे॒व॒त्य॑ इत्य॑न्तरिक्ष - दे॒व॒त्यः॑ । हि सोमः॑ । सोम॒ आप्या॑यितः । आप्या॑यित॒ एष्टः॑ । आप्या॑यित॒ इत्या - प्या॒यि॒तः॒ । एष्टा॒ रायः॑ । रायः॒ प्र । प्रेषे । इ॒षे भगा॑य । भगा॒येति॑ । इत्या॑ह । आ॒ह॒ द्यावा॑पृथि॒वीभ्या᳚म् । द्यावा॑पृथि॒वीभ्या॑मे॒व । द्यावा॑पृथि॒वीभ्या॒मिति॒ द्यावा᳚ - पृ॒थि॒वीभ्या᳚म् । ए॒व न॑म॒स्कृत्य॑ । न॒म॒स्कृत्या॒स्मिन्न् । न॒म॒स्कृत्येति॑ नमः - कृत्य॑ । अ॒स्मिन् ॅलो॒के । लो॒के प्रति॑ । प्रति॑ तिष्ठन्ति । ति॒ष्ठ॒न्ति॒ दे॒वा॒सु॒राः । दे॒वा॒सु॒राः सम्ॅय॑त्ताः । दे॒वा॒सु॒रा इति॑ देव - अ॒सु॒राः । सम्ॅय॑त्ता आसन्न् । सम्ॅय॑त्ता॒ इति॒ सम् - य॒त्ताः॒ । आ॒स॒न् ते । ते दे॒वाः । दे॒वा बिभ्य॑तः । बिभ्य॑तो॒ऽग्निम् । अ॒ग्निम् प्र । प्रावि॑शन्न् । अ॒वि॒श॒न् तस्मा᳚त् । तस्मा॑दाहुः । आ॒हु॒र॒ग्निः । अ॒ग्निः सर्वाः᳚ । सर्वा॑ दे॒वताः᳚ । दे॒वता॒ इति॑ । इति॒ ते । ते᳚ऽग्निम् \newline

\textbf{Jatai Paata} \newline

1. इत्या॑हा॒हे तीत्या॑ह । \newline
2. आ॒हा॒शिष॑ मा॒शिष॑ माहा हा॒शिष᳚म् । \newline
3. आ॒शिष॑ मे॒वै वाशिष॑ मा॒शिष॑ मे॒व । \newline
4. आ॒शिष॒मित्या᳚ - शिष᳚म् । \newline
5. ए॒वैता मे॒ता मे॒वै वैताम् । \newline
6. ए॒ता मैता मे॒ता मा । \newline
7. आ शा᳚स्ते शास्त॒ आ शा᳚स्ते । \newline
8. शा॒स्ते॒ प्र प्र शा᳚स्ते शास्ते॒ प्र । \newline
9. प्र वै वै प्र प्र वै । \newline
10. वा ए॒त ए॒ते वै वा ए॒ते । \newline
11. ए॒ते᳚ ऽस्मा द॒स्मा दे॒त ए॒ते᳚ ऽस्मात् । \newline
12. अ॒स्मा ल्लो॒का ल्लो॒का द॒स्मा द॒स्मा ल्लो॒कात् । \newline
13. लो॒काच् च्य॑वन्ते च्यवन्ते लो॒का ल्लो॒काच् च्य॑वन्ते । \newline
14. च्य॒व॒न्ते॒ ये ये च्य॑वन्ते च्यवन्ते॒ ये । \newline
15. ये सोमꣳ॒॒ सोमं॒ ॅये ये सोम᳚म् । \newline
16. सोम॑ माप्या॒यय॑ न्त्याप्या॒यय॑न्ति॒ सोमꣳ॒॒ सोम॑ माप्या॒यय॑न्ति । \newline
17. आ॒प्या॒यय॑ न्त्यन्तरिक्षदेव॒त्यो᳚ ऽन्तरिक्षदेव॒त्य॑ आप्या॒यय॑ न्त्याप्या॒यय॑ न्त्यन्तरिक्षदेव॒त्यः॑ । \newline
18. आ॒प्या॒यय॒न्तीत्या᳚ - प्या॒यय॑न्ति । \newline
19. अ॒न्त॒रि॒क्ष॒दे॒व॒त्यो॑ हि ह्य॑न्तरिक्षदेव॒त्यो᳚ ऽन्तरिक्षदेव॒त्यो॑ हि । \newline
20. अ॒न्त॒रि॒क्ष॒दे॒व॒त्य॑ इत्य॑न्तरिक्ष - दे॒व॒त्यः॑ । \newline
21. हि सोमः॒ सोमो॒ हि हि सोमः॑ । \newline
22. सोम॒ आप्या॑यित॒ आप्या॑यितः॒ सोमः॒ सोम॒ आप्या॑यितः । \newline
23. आप्या॑यित॒ एष्ट॒ रेष्ट॒ राप्या॑यित॒ आप्या॑यित॒ एष्टः॑ । \newline
24. आप्या॑यित॒ इत्या - प्या॒यि॒तः॒ । \newline
25. एष्टा॒ रायो॒ राय॒ एष्ट॒ रेष्टा॒ रायः॑ । \newline
26. रायः॒ प्र प्र रायो॒ रायः॒ प्र । \newline
27. प्रेष इ॒षे प्र प्रेषे । \newline
28. इ॒षे भगा॑य॒ भगा॑ये॒ ष इ॒षे भगा॑य । \newline
29. भगा॒ये तीति॒ भगा॑य॒ भगा॒येति॑ । \newline
30. इत्या॑हा॒हे तीत्या॑ह । \newline
31. आ॒ह॒ द्यावा॑पृथि॒वीभ्या॒म् द्यावा॑पृथि॒वीभ्या॑ माहाह॒ द्यावा॑पृथि॒वीभ्या᳚म् । \newline
32. द्यावा॑पृथि॒वीभ्या॑ मे॒वैव द्यावा॑पृथि॒वीभ्या॒म् द्यावा॑पृथि॒वीभ्या॑ मे॒व । \newline
33. द्यावा॑पृथि॒वीभ्या॒मिति॒ द्यावा᳚ - पृ॒थि॒वीभ्या᳚म् । \newline
34. ए॒व न॑म॒स्कृत्य॑ नम॒स्कृ त्यै॒वैव न॑म॒स्कृत्य॑ । \newline
35. न॒म॒स्कृ त्या॒स्मिन् न॒स्मिन् न॑म॒स्कृत्य॑ नम॒स्कृ त्या॒स्मिन्न् । \newline
36. न॒म॒स्कृत्येति॑ नमः - कृत्य॑ । \newline
37. अ॒स्मिन् ॅलो॒के लो॒के᳚ ऽस्मिन् न॒स्मिन् ॅलो॒के । \newline
38. लो॒के प्रति॒ प्रति॑ लो॒के लो॒के प्रति॑ । \newline
39. प्रति॑ तिष्ठन्ति तिष्ठन्ति॒ प्रति॒ प्रति॑ तिष्ठन्ति । \newline
40. ति॒ष्ठ॒न्ति॒ दे॒वा॒सु॒रा दे॑वासु॒रा स्ति॑ष्ठन्ति तिष्ठन्ति देवासु॒राः । \newline
41. दे॒वा॒सु॒राः संॅय॑त्ताः॒ संॅय॑त्ता देवासु॒रा दे॑वासु॒राः संॅय॑त्ताः । \newline
42. दे॒वा॒सु॒रा इति॑ देव - अ॒सु॒राः । \newline
43. संॅय॑त्ता आसन् नास॒न् थ्संॅय॑त्ताः॒ संॅय॑त्ता आसन्न् । \newline
44. संॅय॑त्ता॒ इति॒ सं - य॒त्ताः॒ । \newline
45. आ॒स॒न् ते त आ॑सन् नास॒न् ते । \newline
46. ते दे॒वा दे॒वा स्ते ते दे॒वाः । \newline
47. दे॒वा बिभ्य॑तो॒ बिभ्य॑तो दे॒वा दे॒वा बिभ्य॑तः । \newline
48. बिभ्य॑तो॒ ऽग्नि म॒ग्निम् बिभ्य॑तो॒ बिभ्य॑तो॒ ऽग्निम् । \newline
49. अ॒ग्निम् प्र प्राग्नि म॒ग्निम् प्र । \newline
50. प्रावि॑शन् नविश॒न् प्र प्रावि॑शन्न् । \newline
51. अ॒वि॒श॒न् तस्मा॒त् तस्मा॑ दविशन् नविश॒न् तस्मा᳚त् । \newline
52. तस्मा॑ दाहु राहु॒ स्तस्मा॒त् तस्मा॑ दाहुः । \newline
53. आ॒हु॒ र॒ग्नि र॒ग्नि रा॑हु राहु र॒ग्निः । \newline
54. अ॒ग्निः सर्वाः॒ सर्वा॑ अ॒ग्नि र॒ग्निः सर्वाः᳚ । \newline
55. सर्वा॑ दे॒वता॑ दे॒वताः॒ सर्वाः॒ सर्वा॑ दे॒वताः᳚ । \newline
56. दे॒वता॒ इतीति॑ दे॒वता॑ दे॒वता॒ इति॑ । \newline
57. इति॒ ते त इतीति॒ ते । \newline
58. ते᳚ ऽग्नि म॒ग्निम् ते ते᳚ ऽग्निम् । \newline

\textbf{Ghana Paata } \newline

1. इत्या॑ हा॒हे तीत्या॑ हा॒शिष॑ मा॒शिष॑ मा॒हे तीत्या॑ हा॒शिष᳚म् । \newline
2. आ॒हा॒ शिष॑ मा॒शिष॑ माहा हा॒शिष॑ मे॒वै वाशिष॑ माहा हा॒शिष॑ मे॒व । \newline
3. आ॒शिष॑ मे॒वै वाशिष॑ मा॒शिष॑ मे॒वैता मे॒ता मे॒वा शिष॑ मा॒शिष॑ मे॒वैताम् । \newline
4. आ॒शिष॒मित्या᳚ - शिष᳚म् । \newline
5. ए॒वैता मे॒ता मे॒वै वैता मैता मे॒वै वैता मा । \newline
6. ए॒ता मैता मे॒ता मा शा᳚स्ते शास्त॒ ऐता मे॒ता मा शा᳚स्ते । \newline
7. आ शा᳚स्ते शास्त॒ आ शा᳚स्ते॒ प्र प्र शा᳚स्त॒ आ शा᳚स्ते॒ प्र । \newline
8. शा॒स्ते॒ प्र प्र शा᳚स्ते शास्ते॒ प्र वै वै प्र शा᳚स्ते शास्ते॒ प्र वै । \newline
9. प्र वै वै प्र प्र वा ए॒त ए॒ते वै प्र प्र वा ए॒ते । \newline
10. वा ए॒त ए॒ते वै वा ए॒ते᳚ ऽस्मा द॒स्मा दे॒ते वै वा ए॒ते᳚ ऽस्मात् । \newline
11. ए॒ते᳚ ऽस्मा द॒स्मा दे॒त ए॒ते᳚ ऽस्मा ल्लो॒का ल्लो॒का द॒स्मा दे॒त ए॒ते᳚ ऽस्मा ल्लो॒कात् । \newline
12. अ॒स्मा ल्लो॒का ल्लो॒का द॒स्मा द॒स्मा ल्लो॒काच् च्य॑वन्ते च्यवन्ते लो॒का द॒स्मा द॒स्मा ल्लो॒काच् च्य॑वन्ते । \newline
13. लो॒काच् च्य॑वन्ते च्यवन्ते लो॒का ल्लो॒काच् च्य॑वन्ते॒ ये ये च्य॑वन्ते लो॒का ल्लो॒काच् च्य॑वन्ते॒ ये । \newline
14. च्य॒व॒न्ते॒ ये ये च्य॑वन्ते च्यवन्ते॒ ये सोमꣳ॒॒ सोमं॒ ॅये च्य॑वन्ते च्यवन्ते॒ ये सोम᳚म् । \newline
15. ये सोमꣳ॒॒ सोमं॒ ॅये ये सोम॑ माप्या॒यय॑ न्त्याप्या॒यय॑न्ति॒ सोमं॒ ॅये ये सोम॑ माप्या॒यय॑न्ति । \newline
16. सोम॑ माप्या॒यय॑ न्त्याप्या॒यय॑न्ति॒ सोमꣳ॒॒ सोम॑ माप्या॒यय॑न् त्यन्तरिक्षदेव॒त्यो᳚ ऽन्तरिक्षदेव॒त्य॑ आप्या॒यय॑न्ति॒ सोमꣳ॒॒ सोम॑ माप्या॒यय॑न् त्यन्तरिक्षदेव॒त्यः॑ । \newline
17. आ॒प्या॒यय॑न् त्यन्तरिक्षदेव॒त्यो᳚ ऽन्तरिक्षदेव॒त्य॑ आप्या॒यय॑न्त्या प्या॒यय॑न् त्यन्तरिक्षदेव॒त्यो॑ हि ह्य॑न्तरिक्षदेव॒त्य॑ आप्या॒यय॑न्त्या प्या॒यय॑न् त्यन्तरिक्षदेव॒त्यो॑ हि । \newline
18. आ॒प्या॒यय॒न्तीत्या᳚ - प्या॒यय॑न्ति । \newline
19. अ॒न्त॒रि॒क्ष॒दे॒व॒त्यो॑ हि ह्य॑न्तरिक्षदेव॒त्यो᳚ ऽन्तरिक्षदेव॒त्यो॑ हि सोमः॒ सोमो॒ ह्य॑न्तरिक्षदेव॒त्यो᳚ ऽन्तरिक्षदेव॒त्यो॑ हि सोमः॑ । \newline
20. अ॒न्त॒रि॒क्ष॒दे॒व॒त्य॑ इत्य॑न्तरिक्ष - दे॒व॒त्यः॑ । \newline
21. हि सोमः॒ सोमो॒ हि हि सोम॒ आप्या॑यित॒ आप्या॑यितः॒ सोमो॒ हि हि सोम॒ आप्या॑यितः । \newline
22. सोम॒ आप्या॑यित॒ आप्या॑यितः॒ सोमः॒ सोम॒ आप्या॑यित॒ एष्ट॒ रेष्ट॒ राप्या॑यितः॒ सोमः॒ सोम॒ आप्या॑यित॒ एष्टः॑ । \newline
23. आप्या॑यित॒ एष्ट॒ रेष्ट॒ राप्या॑यित॒ आप्या॑यित॒ एष्टा॒ रायो॒ राय॒ एष्ट॒ राप्या॑यित॒ आप्या॑यित॒ एष्टा॒ रायः॑ । \newline
24. आप्या॑यित॒ इत्या - प्या॒यि॒तः॒ । \newline
25. एष्टा॒ रायो॒ राय॒ एष्ट॒ रेष्टा॒ रायः॒ प्र प्र राय॒ एष्ट॒ रेष्टा॒ रायः॒ प्र । \newline
26. रायः॒ प्र प्र रायो॒ रायः॒ प्रेष इ॒षे प्र रायो॒ रायः॒ प्रेषे । \newline
27. प्रेष इ॒षे प्र प्रेषे भगा॑य॒ भगा॑ये॒षे प्र प्रेषे भगा॑य । \newline
28. इ॒षे भगा॑य॒ भगा॑ ये॒ष इ॒षे भगा॒ये तीति॒ भगा॑ ये॒ष इ॒षे भगा॒येति॑ । \newline
29. भगा॒ये तीति॒ भगा॑य॒ भगा॒ये त्या॑हा॒हेति॒ भगा॑य॒ भगा॒ये त्या॑ह । \newline
30. इत्या॑हा॒हे तीत्या॑ह॒ द्यावा॑पृथि॒वीभ्या॒म् द्यावा॑पृथि॒वीभ्या॑ मा॒हे तीत्या॑ह॒ द्यावा॑पृथि॒वीभ्या᳚म् । \newline
31. आ॒ह॒ द्यावा॑पृथि॒वीभ्या॒म् द्यावा॑पृथि॒वीभ्या॑ माहाह॒ द्यावा॑पृथि॒वीभ्या॑ मे॒वैव द्यावा॑पृथि॒वीभ्या॑ माहाह॒ द्यावा॑पृथि॒वीभ्या॑ मे॒व । \newline
32. द्यावा॑पृथि॒वीभ्या॑ मे॒वैव द्यावा॑पृथि॒वीभ्या॒म् द्यावा॑पृथि॒वीभ्या॑ मे॒व न॑म॒स्कृत्य॑ नम॒स्कृत्यै॒व द्यावा॑पृथि॒वीभ्या॒म् द्यावा॑पृथि॒वीभ्या॑ मे॒व न॑म॒स्कृत्य॑ । \newline
33. द्यावा॑पृथि॒वीभ्या॒मिति॒ द्यावा᳚ - पृ॒थि॒वीभ्या᳚म् । \newline
34. ए॒व न॑म॒स्कृत्य॑ नम॒स्कृत्यै॒ वैव न॑म॒स्कृ त्या॒स्मिन् न॒स्मिन् न॑म॒स्कृ त्यै॒वैव न॑म॒स्कृ त्या॒स्मिन्न् । \newline
35. न॒म॒स्कृ त्या॒स्मिन् न॒स्मिन् न॑म॒स्कृत्य॑ नम॒स्कृ त्या॒स्मिन् ॅलो॒के लो॒के᳚ ऽस्मिन् न॑म॒स्कृत्य॑ नम॒स्कृ त्या॒स्मिन् ॅलो॒के । \newline
36. न॒म॒स्कृत्येति॑ नमः - कृत्य॑ । \newline
37. अ॒स्मिन् ॅलो॒के लो॒के᳚ ऽस्मिन् न॒स्मिन् ॅलो॒के प्रति॒ प्रति॑ लो॒के᳚ ऽस्मिन् न॒स्मिन् ॅलो॒के प्रति॑ । \newline
38. लो॒के प्रति॒ प्रति॑ लो॒के लो॒के प्रति॑ तिष्ठन्ति तिष्ठन्ति॒ प्रति॑ लो॒के लो॒के प्रति॑ तिष्ठन्ति । \newline
39. प्रति॑ तिष्ठन्ति तिष्ठन्ति॒ प्रति॒ प्रति॑ तिष्ठन्ति देवासु॒रा दे॑वासु॒रा स्ति॑ष्ठन्ति॒ प्रति॒ प्रति॑ तिष्ठन्ति देवासु॒राः । \newline
40. ति॒ष्ठ॒न्ति॒ दे॒वा॒सु॒रा दे॑वासु॒रा स्ति॑ष्ठन्ति तिष्ठन्ति देवासु॒राः संॅय॑त्ताः॒ संॅय॑त्ता देवासु॒रा स्ति॑ष्ठन्ति तिष्ठन्ति देवासु॒राः संॅय॑त्ताः । \newline
41. दे॒वा॒सु॒राः संॅय॑त्ताः॒ संॅय॑त्ता देवासु॒रा दे॑वासु॒राः संॅय॑त्ता आसन् नास॒न् थ्संॅय॑त्ता देवासु॒रा दे॑वासु॒राः संॅय॑त्ता आसन्न् । \newline
42. दे॒वा॒सु॒रा इति॑ देव - अ॒सु॒राः । \newline
43. संॅय॑त्ता आसन् नास॒न् थ्संॅय॑त्ताः॒ संॅय॑त्ता आस॒न् ते त आ॑स॒न् थ्संॅय॑त्ताः॒ संॅय॑त्ता आस॒न् ते । \newline
44. संॅय॑त्ता॒ इति॒ सं - य॒त्ताः॒ । \newline
45. आ॒स॒न् ते त आ॑सन् नास॒न् ते दे॒वा दे॒वा स्त आ॑सन् नास॒न् ते दे॒वाः । \newline
46. ते दे॒वा दे॒वा स्ते ते दे॒वा बिभ्य॑तो॒ बिभ्य॑तो दे॒वा स्ते ते दे॒वा बिभ्य॑तः । \newline
47. दे॒वा बिभ्य॑तो॒ बिभ्य॑तो दे॒वा दे॒वा बिभ्य॑तो॒ ऽग्नि म॒ग्निम् बिभ्य॑तो दे॒वा दे॒वा बिभ्य॑तो॒ ऽग्निम् । \newline
48. बिभ्य॑तो॒ ऽग्नि म॒ग्निम् बिभ्य॑तो॒ बिभ्य॑तो॒ ऽग्निम् प्र प्राग्निम् बिभ्य॑तो॒ बिभ्य॑तो॒ ऽग्निम् प्र । \newline
49. अ॒ग्निम् प्र प्राग्नि म॒ग्निम् प्रावि॑शन् नविश॒न् प्राग्नि म॒ग्निम् प्रावि॑शन्न् । \newline
50. प्रावि॑शन् नविश॒न् प्र प्रावि॑श॒न् तस्मा॒त् तस्मा॑ दविश॒न् प्र प्रावि॑श॒न् तस्मा᳚त् । \newline
51. अ॒वि॒श॒न् तस्मा॒त् तस्मा॑ दविशन् नविश॒न् तस्मा॑ दाहु राहु॒ स्तस्मा॑ दविशन् नविश॒न् तस्मा॑ दाहुः । \newline
52. तस्मा॑ दाहु राहु॒ स्तस्मा॒त् तस्मा॑ दाहु र॒ग्नि र॒ग्नि रा॑हु॒ स्तस्मा॒त् तस्मा॑ दाहु र॒ग्निः । \newline
53. आ॒हु॒ र॒ग्नि र॒ग्नि रा॑हु राहु र॒ग्निः सर्वाः॒ सर्वा॑ अ॒ग्नि रा॑हु राहु र॒ग्निः सर्वाः᳚ । \newline
54. अ॒ग्निः सर्वाः॒ सर्वा॑ अ॒ग्नि र॒ग्निः सर्वा॑ दे॒वता॑ दे॒वताः॒ सर्वा॑ अ॒ग्नि र॒ग्निः सर्वा॑ दे॒वताः᳚ । \newline
55. सर्वा॑ दे॒वता॑ दे॒वताः॒ सर्वाः॒ सर्वा॑ दे॒वता॒ इतीति॑ दे॒वताः॒ सर्वाः॒ सर्वा॑ दे॒वता॒ इति॑ । \newline
56. दे॒वता॒ इतीति॑ दे॒वता॑ दे॒वता॒ इति॒ ते त इति॑ दे॒वता॑ दे॒वता॒ इति॒ ते । \newline
57. इति॒ ते त इतीति॒ ते᳚ ऽग्नि म॒ग्निम् त इतीति॒ ते᳚ ऽग्निम् । \newline
58. ते᳚ ऽग्नि म॒ग्निम् ते ते᳚ ऽग्नि मे॒वै वाग्निम् ते ते᳚ ऽग्नि मे॒व । \newline
\pagebreak
\markright{ TS 6.2.2.7  \hfill https://www.vedavms.in \hfill}

\section{ TS 6.2.2.7 }

\textbf{TS 6.2.2.7 } \newline
\textbf{Samhita Paata} \newline

ऽग्निमे॒व वरू॑थं कृ॒त्वा ऽसु॑रान॒भ्य॑भव-न्न॒ग्निमि॑व॒ खलु॒ वा ए॒ष प्रवि॑शति॒ यो॑ऽवान्तरदी॒क्षामु॒पैति॒ भ्रातृ॑व्याभिभूत्यै॒ भव॑त्या॒त्मना॒ परा᳚ऽस्य॒ भ्रातृ॑व्यो भवत्या॒त्मान॑मे॒व दी॒क्षया॑ पाति प्र॒जाम॑वान्तरदी॒क्षया॑ सन्त॒रां मेख॑लाꣳ स॒माय॑च्छते प्र॒जा ह्या᳚त्मनोऽन्त॑रतरा त॒प्तव्र॑तो भवति॒ मद॑न्तीभिर्मार्जयते॒ निर्ह्य॑ग्निः शी॒तेन॒ वाय॑ति॒ समि॑द्ध्यै॒ या ते॑ अग्ने॒ ( ) रुद्रि॑या त॒नूरित्या॑ह॒ स्वयै॒वैन॑द् दे॒वत॑या व्रतयति सयोनि॒त्वाय॒ शान्त्यै᳚ ॥ \newline

\textbf{Pada Paata} \newline

अ॒ग्निम् । ए॒व । वरू॑थम् । कृ॒त्वा । असु॑रान् । अ॒भीति॑ । अ॒भ॒व॒न्न् । अ॒ग्निम् । इ॒व॒ । खलु॑ । वै । ए॒षः । प्रेति॑ । वि॒श॒ति॒ । यः । अ॒वा॒न्त॒र॒दी॒क्षामित्य॑वान्तर - दी॒क्षाम् । उ॒पैतीत्यु॑प - एति॑ । भ्रातृ॑व्याभिभूत्या॒ इति॒ भ्रातृ॑व्य - अ॒भि॒भू॒त्यै॒ । भव॑ति । आ॒त्मना᳚ । परेति॑ । अ॒स्य॒ । भ्रातृ॑व्यः । भ॒व॒ति॒ । आ॒त्मान᳚म् । ए॒व । दी॒क्षया᳚ । पा॒ति॒ । प्र॒जामिति॑ प्र - जाम् । अ॒वा॒न्त॒र॒दी॒क्षयेत्य॑वान्तर - दी॒क्षया᳚ । स॒न्त॒रामिति॑ सं - त॒राम् । मेख॑लाम् । स॒माय॑च्छत॒ इति॑ सं - आय॑च्छते । प्रजेति॑ प्र - जा । हि । आ॒त्मनः॑ । अन्त॑रत॒रेत्यन्त॑र - त॒रा॒ । त॒प्तव्र॑त॒ इति॑ त॒प्त - व्र॒तः॒ । भ॒व॒ति॒ । मद॑न्तीभिः । मा॒र्ज॒य॒ते॒ । निरिति॑ । हि । अ॒ग्निः । शी॒तेन॑ । वाय॑ति । समि॑द्ध्या॒ इति॒ सं-इ॒द्ध्यै॒ । या । ते॒ । अ॒ग्ने॒ ( ) । रुद्रि॑या । त॒नूः । इति॑ । आ॒ह॒ । स्वया᳚ । ए॒व । ए॒न॒त् । दे॒वत॑या । व्र॒त॒य॒ति॒ । स॒यो॒नि॒त्वायेति॑ सयोनि - त्वाय॑ । शान्त्यै᳚ ॥  \newline


\textbf{Krama Paata} \newline

अ॒ग्निमे॒व । ए॒व वरू॑थम् । वरू॑थम् कृ॒त्वा । कृ॒त्वाऽसु॑रान् । असु॑रान॒भि । अ॒भ्य॑भवन्न् । अ॒भ॒व॒न्न॒ग्निम् । अ॒ग्निमि॑व । इ॒व॒ खलु॑ । खलु॒ वै । वा ए॒षः । ए॒ष प्र । प्र वि॑शति । वि॒श॒ति॒ यः । यो॑ऽवान्तरदी॒क्षाम् । अ॒वा॒न्त॒र॒दी॒क्षामु॒पैति॑ । अ॒वा॒न्त॒र॒दी॒क्षामित्य॑वान्तर - दी॒क्षाम् । उ॒पैति॒ भ्रातृ॑व्याभिभूत्यै । उ॒पैतीत्यु॑प - एति॑ । भातृ॑व्याभिभूत्यै॒ भव॑ति । भातृ॑व्याभिभूत्या॒ इति॒ भ्रातृ॑व्य - अ॒भि॒भू॒त्यै॒ । भव॑त्या॒त्मना᳚ । आ॒त्मना॒ परा᳚ । परा᳚ऽस्य । अ॒स्य॒ भ्रातृ॑व्यः । भ्रातृ॑व्यो भवति । भ॒व॒त्या॒त्मान᳚म् । आ॒त्मान॑मे॒व । ए॒व दी॒क्षया᳚ । दी॒क्षया॑ पाति । पा॒ति॒ प्र॒जाम् । प्र॒जाम॑वान्तरदी॒क्षया᳚ । प्र॒जामिति॑ प्र - जाम् । अ॒वा॒न्त॒र॒दी॒क्षया॑ सन्त॒राम् । अ॒वा॒न्त॒र॒दी॒क्षयेत्य॑वान्तर - दी॒क्षया᳚ । स॒न्त॒राम् मेख॑लाम् । स॒न्त॒रामिति॑ सम् - त॒राम् । मेख॑लाꣳ स॒माय॑च्छते । स॒माय॑च्छते प्र॒जा । स॒माय॑च्छत॒ इति॑ सम् - आय॑च्छते । प्र॒जा हि । प्र॒जेति॑ प्र - जा । ह्या᳚त्मनः॑ । आ॒त्मनोऽन्त॑रतरा । अन्त॑रतरा त॒प्तव्र॑तः । अन्त॑रत॒रेत्यन्त॑र - त॒रा॒ । त॒प्तव्र॑तो भवति । त॒प्तव्र॑त॒ इति॑ त॒प्त - व्र॒तः॒ । भ॒व॒ति॒ मद॑न्तीभिः । मद॑न्तीभिर् मार्जयते । मा॒र्ज॒य॒ते॒ निः । निर्. हि । ह्य॑ग्निः । अ॒ग्निः शी॒तेन॑ । शी॒तेन॒ वाय॑ति । वाय॑ति॒ समि॑द्ध्यै । समि॑द्ध्यै॒ या । समि॑द्ध्या॒ इति॒ सम् - इ॒द्ध्यै॒ । या ते᳚ । ते॒ अ॒ग्ने॒ ( ) । अ॒ग्ने॒ रुद्रि॑या । रुद्रि॑या त॒नूः । त॒नूरिति॑ । इत्या॑ह । आ॒ह॒ स्वया᳚ । स्वयै॒व । ए॒वैन॑त् । ए॒न॒द् दे॒वत॑या । दे॒वत॑या व्रतयति । व्र॒त॒य॒ति॒ स॒यो॒नि॒त्वाय॑ । स॒यो॒नि॒त्वाय॒ शान्त्यै᳚ । स॒यो॒नि॒त्वायेति॑ सयोनि - त्वाय॑ । शान्त्या॒ इति॒ शान्त्यै᳚ । \newline

\textbf{Jatai Paata} \newline

1. अ॒ग्नि मे॒वैवाग्नि म॒ग्नि मे॒व । \newline
2. ए॒व वरू॑थं॒ ॅवरू॑थ मे॒वैव वरू॑थम् । \newline
3. वरू॑थम् कृ॒त्वा कृ॒त्वा वरू॑थं॒ ॅवरू॑थम् कृ॒त्वा । \newline
4. कृ॒त्वा ऽसु॑रा॒ नसु॑रान् कृ॒त्वा कृ॒त्वा ऽसु॑रान् । \newline
5. असु॑रा न॒भ्य॑भ्यसु॑रा॒ नसु॑रा न॒भि । \newline
6. अ॒भ्य॑भवन् नभवन् न॒भ्या᳚(1॒)भ्य॑भवन्न् । \newline
7. अ॒भ॒व॒न् न॒ग्नि म॒ग्नि म॑भवन् नभवन् न॒ग्निम् । \newline
8. अ॒ग्नि मि॑वे वा॒ग्नि म॒ग्नि मि॑व । \newline
9. इ॒व॒ खलु॒ खल्वि॑वे व॒ खलु॑ । \newline
10. खलु॒ वै वै खलु॒ खलु॒ वै । \newline
11. वा ए॒ष ए॒ष वै वा ए॒षः । \newline
12. ए॒ष प्र प्रैष ए॒ष प्र । \newline
13. प्र वि॑शति विशति॒ प्र प्र वि॑शति । \newline
14. वि॒श॒ति॒ यो यो वि॑शति विशति॒ यः । \newline
15. यो॑ ऽवान्तरदी॒क्षा म॑वान्तरदी॒क्षां ॅयो यो॑ ऽवान्तरदी॒क्षाम् । \newline
16. अ॒वा॒न्त॒र॒दी॒क्षा मु॒पै त्यु॒पैत्य॑वान्तरदी॒क्षा म॑वान्तरदी॒क्षा मु॒पैति॑ । \newline
17. अ॒वा॒न्त॒र॒दी॒क्षामित्य॑वान्तर - दी॒क्षाम् । \newline
18. उ॒पैति॒ भ्रातृ॑व्याभिभूत्यै॒ भ्रातृ॑व्याभिभूत्या उ॒पै त्यु॒पैति॒ भ्रातृ॑व्याभिभूत्यै । \newline
19. उ॒पैतीत्यु॑प - एति॑ । \newline
20. भ्रातृ॑व्याभिभूत्यै॒ भव॑ति॒ भव॑ति॒ भ्रातृ॑व्याभिभूत्यै॒ भ्रातृ॑व्याभिभूत्यै॒ भव॑ति । \newline
21. भ्रातृ॑व्याभिभूत्या॒ इति॒ भ्रातृ॑व्य - अ॒भि॒भू॒त्यै॒ । \newline
22. भव॑ त्या॒त्मना॒ ऽऽत्मना॒ भव॑ति॒ भव॑ त्या॒त्मना᳚ । \newline
23. आ॒त्मना॒ परा॒ परा॒ ऽऽत्मना॒ ऽऽत्मना॒ परा᳚ । \newline
24. परा᳚ ऽस्यास्य॒ परा॒ परा᳚ ऽस्य । \newline
25. अ॒स्य॒ भ्रातृ॑व्यो॒ भ्रातृ॑व्यो ऽस्यास्य॒ भ्रातृ॑व्यः । \newline
26. भ्रातृ॑व्यो भवति भवति॒ भ्रातृ॑व्यो॒ भ्रातृ॑व्यो भवति । \newline
27. भ॒व॒ त्या॒त्मान॑ मा॒त्मान॑म् भवति भव त्या॒त्मान᳚म् । \newline
28. आ॒त्मान॑ मे॒वैवात्मान॑ मा॒त्मान॑ मे॒व । \newline
29. ए॒व दी॒क्षया॑ दी॒क्ष यै॒वैव दी॒क्षया᳚ । \newline
30. दी॒क्षया॑ पाति पाति दी॒क्षया॑ दी॒क्षया॑ पाति । \newline
31. पा॒ति॒ प्र॒जाम् प्र॒जाम् पा॑ति पाति प्र॒जाम् । \newline
32. प्र॒जा म॑वान्तरदी॒क्षया॑ ऽवान्तरदी॒क्षया᳚ प्र॒जाम् प्र॒जा म॑वान्तरदी॒क्षया᳚ । \newline
33. प्र॒जामिति॑ प्र - जाम् । \newline
34. अ॒वा॒न्त॒र॒दी॒क्षया॑ सन्त॒राꣳ स॑न्त॒रा म॑वान्तरदी॒क्षया॑ ऽवान्तरदी॒क्षया॑ सन्त॒राम् । \newline
35. अ॒वा॒न्त॒र॒दी॒क्षयेत्य॑वान्तर - दी॒क्षया᳚ । \newline
36. स॒न्त॒राम् मेख॑ला॒म् मेख॑लाꣳ सन्त॒राꣳ स॑न्त॒राम् मेख॑लाम् । \newline
37. स॒न्त॒रामिति॑ सं - त॒राम् । \newline
38. मेख॑लाꣳ स॒माय॑च्छते स॒माय॑च्छते॒ मेख॑ला॒म् मेख॑लाꣳ स॒माय॑च्छते । \newline
39. स॒माय॑च्छते प्र॒जा प्र॒जा स॒माय॑च्छते स॒माय॑च्छते प्र॒जा । \newline
40. स॒माय॑च्छत॒ इति॑ सं - आय॑च्छते । \newline
41. प्र॒जा हि हि प्र॒जा प्र॒जा हि । \newline
42. प्र॒जेति॑ प्र - जा । \newline
43. ह्या᳚त्मन॑ आ॒त्मनो॒ हि ह्या᳚त्मनः॑ । \newline
44. आ॒त्मनो ऽन्त॑रत॒रा ऽन्त॑रतरा॒ ऽऽत्मन॑ आ॒त्मनो ऽन्त॑रतरा । \newline
45. अन्त॑रतरा त॒प्तव्र॑त स्त॒प्तव्र॒तो ऽन्त॑रत॒रा ऽन्त॑रतरा त॒प्तव्र॑तः । \newline
46. अन्त॑रत॒रेत्यन्त॑र - त॒रा॒ । \newline
47. त॒प्तव्र॑तो भवति भवति त॒प्तव्र॑त स्त॒प्तव्र॑तो भवति । \newline
48. त॒प्तव्र॑त॒ इति॑ त॒प्त - व्र॒तः॒ । \newline
49. भ॒व॒ति॒ मद॑न्तीभि॒र् मद॑न्तीभिर् भवति भवति॒ मद॑न्तीभिः । \newline
50. मद॑न्तीभिर् मार्जयते मार्जयते॒ मद॑न्तीभि॒र् मद॑न्तीभिर् मार्जयते । \newline
51. मा॒र्ज॒य॒ते॒ निर् णिर् मा᳚र्जयते मार्जयते॒ निः । \newline
52. निर्. हि हि निर् णिर्. हि । \newline
53. ह्य॑ग्नि र॒ग्निर्. हि ह्य॑ग्निः । \newline
54. अ॒ग्निः शी॒तेन॑ शी॒तेना॒ग्नि र॒ग्निः शी॒तेन॑ । \newline
55. शी॒तेन॒ वाय॑ति॒ वाय॑ति शी॒तेन॑ शी॒तेन॒ वाय॑ति । \newline
56. वाय॑ति॒ समि॑द्ध्यै॒ समि॑द्ध्यै॒ वाय॑ति॒ वाय॑ति॒ समि॑द्ध्यै । \newline
57. समि॑द्ध्यै॒ या या समि॑द्ध्यै॒ समि॑द्ध्यै॒ या । \newline
58. समि॑द्ध्या॒ इति॒ सं - इ॒द्ध्यै॒ । \newline
59. या ते॑ ते॒ या या ते᳚ । \newline
60. ते॒ अ॒ग्ने॒ ऽग्ने॒ ते॒ ते॒ अ॒ग्ने॒ । \newline
61. अ॒ग्ने॒ रुद्रि॑या॒ रुद्रि॑या ऽग्ने ऽग्ने॒ रुद्रि॑या । \newline
62. रुद्रि॑या त॒नू स्त॒नू रुद्रि॑या॒ रुद्रि॑या त॒नूः । \newline
63. त॒नू रितीति॑ त॒नू स्त॒नू रिति॑ । \newline
64. इत्या॑हा॒हे तीत्या॑ह । \newline
65. आ॒ह॒ स्वया॒ स्वया॑ ऽऽहाह॒ स्वया᳚ । \newline
66. स्वयै॒वैव स्वया॒ स्वयै॒व । \newline
67. ए॒वैन॑ देन दे॒वै वैन॑त् । \newline
68. ए॒न॒द् दे॒वत॑या दे॒वत॑ यैन देनद् दे॒वत॑या । \newline
69. दे॒वत॑या व्रतयति व्रतयति दे॒वत॑या दे॒वत॑या व्रतयति । \newline
70. व्र॒त॒य॒ति॒ स॒यो॒नि॒त्वाय॑ सयोनि॒त्वाय॑ व्रतयति व्रतयति सयोनि॒त्वाय॑ । \newline
71. स॒यो॒नि॒त्वाय॒ शान्त्यै॒ शान्त्यै॑ सयोनि॒त्वाय॑ सयोनि॒त्वाय॒ शान्त्यै᳚ । \newline
72. स॒यो॒नि॒त्वायेति॑ सयोनि - त्वाय॑ । \newline
73. शान्त्या॒ इति॒ शान्त्यै᳚ । \newline

\textbf{Ghana Paata } \newline

1. अ॒ग्नि मे॒वैवाग्नि म॒ग्नि मे॒व वरू॑थं॒ ॅवरू॑थ मे॒वाग्नि म॒ग्नि मे॒व वरू॑थम् । \newline
2. ए॒व वरू॑थं॒ ॅवरू॑थ मे॒वैव वरू॑थम् कृ॒त्वा कृ॒त्वा वरू॑थ मे॒वैव वरू॑थम् कृ॒त्वा । \newline
3. वरू॑थम् कृ॒त्वा कृ॒त्वा वरू॑थं॒ ॅवरू॑थम् कृ॒त्वा ऽसु॑रा॒ नसु॑रान् कृ॒त्वा वरू॑थं॒ ॅवरू॑थम् कृ॒त्वा ऽसु॑रान् । \newline
4. कृ॒त्वा ऽसु॑रा॒ नसु॑रान् कृ॒त्वा कृ॒त्वा ऽसु॑रा न॒भ्य॑ भ्यसु॑रान् कृ॒त्वा कृ॒त्वा ऽसु॑रा न॒भि । \newline
5. असु॑रा न॒भ्य॑ भ्यसु॑रा॒ नसु॑रा न॒भ्य॑ भवन् नभवन् न॒भ्यसु॑रा॒ नसु॑रा न॒भ्य॑ भवन्न् । \newline
6. अ॒भ्य॑भवन् नभवन् न॒भ्या᳚(1॒) भ्य॑भवन् न॒ग्नि म॒ग्नि म॑भवन् न॒भ्या᳚(1॒)भ्य॑भवन् न॒ग्निम् । \newline
7. अ॒भ॒व॒न् न॒ग्नि म॒ग्नि म॑भवन् नभवन् न॒ग्नि मि॑वे वा॒ग्नि म॑भवन् नभवन् न॒ग्नि मि॑व । \newline
8. अ॒ग्नि मि॑वे वा॒ग्नि म॒ग्नि मि॑व॒ खलु॒ खल्वि॑ वा॒ग्नि म॒ग्नि मि॑व॒ खलु॑ । \newline
9. इ॒व॒ खलु॒ खल्वि॑वेव॒ खलु॒ वै वै खल्वि॑वेव॒ खलु॒ वै । \newline
10. खलु॒ वै वै खलु॒ खलु॒ वा ए॒ष ए॒ष वै खलु॒ खलु॒ वा ए॒षः । \newline
11. वा ए॒ष ए॒ष वै वा ए॒ष प्र प्रैष वै वा ए॒ष प्र । \newline
12. ए॒ष प्र प्रैष ए॒ष प्र वि॑शति विशति॒ प्रैष ए॒ष प्र वि॑शति । \newline
13. प्र वि॑शति विशति॒ प्र प्र वि॑शति॒ यो यो वि॑शति॒ प्र प्र वि॑शति॒ यः । \newline
14. वि॒श॒ति॒ यो यो वि॑शति विशति॒ यो॑ ऽवान्तरदी॒क्षा म॑वान्तरदी॒क्षां ॅयो वि॑शति विशति॒ यो॑ ऽवान्तरदी॒क्षाम् । \newline
15. यो॑ ऽवान्तरदी॒क्षा म॑वान्तरदी॒क्षां ॅयो यो॑ ऽवान्तरदी॒क्षा मु॒पै त्यु॒पै त्य॑वान्तरदी॒क्षां ॅयो यो॑ ऽवान्तरदी॒क्षा मु॒पैति॑ । \newline
16. अ॒वा॒न्त॒र॒दी॒क्षा मु॒पै त्यु॒पै त्य॑वान्तरदी॒क्षा म॑वान्तरदी॒क्षा मु॒पैति॒ भ्रातृ॑व्याभिभूत्यै॒ भ्रातृ॑व्याभिभूत्या उ॒पै त्य॑वान्तरदी॒क्षा म॑वान्तरदी॒क्षा मु॒पैति॒ भ्रातृ॑व्याभिभूत्यै । \newline
17. अ॒वा॒न्त॒र॒दी॒क्षामित्य॑वान्तर - दी॒क्षाम् । \newline
18. उ॒पैति॒ भ्रातृ॑व्याभिभूत्यै॒ भ्रातृ॑व्याभिभूत्या उ॒पै त्यु॒पैति॒ भ्रातृ॑व्याभिभूत्यै॒ भव॑ति॒ भव॑ति॒ भ्रातृ॑व्याभिभूत्या उ॒पै त्यु॒पैति॒ भ्रातृ॑व्याभिभूत्यै॒ भव॑ति । \newline
19. उ॒पैतीत्यु॑प - एति॑ । \newline
20. भ्रातृ॑व्याभिभूत्यै॒ भव॑ति॒ भव॑ति॒ भ्रातृ॑व्याभिभूत्यै॒ भ्रातृ॑व्याभिभूत्यै॒ भव॑ त्या॒त्मना॒ ऽऽत्मना॒ भव॑ति॒ भ्रातृ॑व्याभिभूत्यै॒ भ्रातृ॑व्याभिभूत्यै॒ भव॑ त्या॒त्मना᳚ । \newline
21. भ्रातृ॑व्याभिभूत्या॒ इति॒ भ्रातृ॑व्य - अ॒भि॒भू॒त्यै॒ । \newline
22. भव॑ त्या॒त्मना॒ ऽऽत्मना॒ भव॑ति॒ भव॑ त्या॒त्मना॒ परा॒ परा॒ ऽऽत्मना॒ भव॑ति॒ भव॑ त्या॒त्मना॒ परा᳚ । \newline
23. आ॒त्मना॒ परा॒ परा॒ ऽऽत्मना॒ ऽऽत्मना॒ परा᳚ ऽस्यास्य॒ परा॒ ऽऽत्मना॒ ऽऽत्मना॒ परा᳚ ऽस्य । \newline
24. परा᳚ ऽस्यास्य॒ परा॒ परा᳚ ऽस्य॒ भ्रातृ॑व्यो॒ भ्रातृ॑व्यो ऽस्य॒ परा॒ परा᳚ ऽस्य॒ भ्रातृ॑व्यः । \newline
25. अ॒स्य॒ भ्रातृ॑व्यो॒ भ्रातृ॑व्यो ऽस्यास्य॒ भ्रातृ॑व्यो भवति भवति॒ भ्रातृ॑व्यो ऽस्यास्य॒ भ्रातृ॑व्यो भवति । \newline
26. भ्रातृ॑व्यो भवति भवति॒ भ्रातृ॑व्यो॒ भ्रातृ॑व्यो भव त्या॒त्मान॑ मा॒त्मान॑म् भवति॒ भ्रातृ॑व्यो॒ भ्रातृ॑व्यो भव त्या॒त्मान᳚म् । \newline
27. भ॒व॒ त्या॒त्मान॑ मा॒त्मान॑म् भवति भव त्या॒त्मान॑ मे॒वै वात्मान॑म् भवति भव त्या॒त्मान॑ मे॒व । \newline
28. आ॒त्मान॑ मे॒वै वात्मान॑ मा॒त्मान॑ मे॒व दी॒क्षया॑ दी॒क्ष यै॒वात्मान॑ मा॒त्मान॑ मे॒व दी॒क्षया᳚ । \newline
29. ए॒व दी॒क्षया॑ दी॒क्ष यै॒वैव दी॒क्षया॑ पाति पाति दी॒क्ष यै॒वैव दी॒क्षया॑ पाति । \newline
30. दी॒क्षया॑ पाति पाति दी॒क्षया॑ दी॒क्षया॑ पाति प्र॒जाम् प्र॒जाम् पा॑ति दी॒क्षया॑ दी॒क्षया॑ पाति प्र॒जाम् । \newline
31. पा॒ति॒ प्र॒जाम् प्र॒जाम् पा॑ति पाति प्र॒जा म॑वान्तरदी॒क्षया॑ ऽवान्तरदी॒क्षया᳚ प्र॒जाम् पा॑ति पाति प्र॒जा म॑वान्तरदी॒क्षया᳚ । \newline
32. प्र॒जा म॑वान्तरदी॒क्षया॑ ऽवान्तरदी॒क्षया᳚ प्र॒जाम् प्र॒जा म॑वान्तरदी॒क्षया॑ सन्त॒राꣳ स॑न्त॒रा म॑वान्तरदी॒क्षया᳚ प्र॒जाम् प्र॒जा म॑वान्तरदी॒क्षया॑ सन्त॒राम् । \newline
33. प्र॒जामिति॑ प्र - जाम् । \newline
34. अ॒वा॒न्त॒र॒दी॒क्षया॑ सन्त॒राꣳ स॑न्त॒रा म॑वान्तरदी॒क्षया॑ ऽवान्तरदी॒क्षया॑ सन्त॒राम् मेख॑ला॒म् मेख॑लाꣳ सन्त॒रा म॑वान्तरदी॒क्षया॑ ऽवान्तरदी॒क्षया॑ सन्त॒राम् मेख॑लाम् । \newline
35. अ॒वा॒न्त॒र॒दी॒क्षयेत्य॑वान्तर - दी॒क्षया᳚ । \newline
36. स॒न्त॒राम् मेख॑ला॒म् मेख॑लाꣳ सन्त॒राꣳ स॑न्त॒राम् मेख॑लाꣳ स॒माय॑च्छते स॒माय॑च्छते॒ मेख॑लाꣳ सन्त॒राꣳ स॑न्त॒राम् मेख॑लाꣳ स॒माय॑च्छते । \newline
37. स॒न्त॒रामिति॑ सं - त॒राम् । \newline
38. मेख॑लाꣳ स॒माय॑च्छते स॒माय॑च्छते॒ मेख॑ला॒म् मेख॑लाꣳ स॒माय॑च्छते प्र॒जा प्र॒जा स॒माय॑च्छते॒ मेख॑ला॒म् मेख॑लाꣳ स॒माय॑च्छते प्र॒जा । \newline
39. स॒माय॑च्छते प्र॒जा प्र॒जा स॒माय॑च्छते स॒माय॑च्छते प्र॒जा हि हि प्र॒जा स॒माय॑च्छते स॒माय॑च्छते प्र॒जा हि । \newline
40. स॒माय॑च्छत॒ इति॑ सं - आय॑च्छते । \newline
41. प्र॒जा हि हि प्र॒जा प्र॒जा ह्या᳚त्मन॑ आ॒त्मनो॒ हि प्र॒जा प्र॒जा ह्या᳚त्मनः॑ । \newline
42. प्र॒जेति॑ प्र - जा । \newline
43. ह्या᳚त्मन॑ आ॒त्मनो॒ हि ह्या᳚त्मनो ऽन्त॑रत॒रा ऽन्त॑रतरा॒ ऽऽत्मनो॒ हि ह्या᳚त्मनो ऽन्त॑रतरा । \newline
44. आ॒त्मनो ऽन्त॑रत॒रा ऽन्त॑रतरा॒ ऽऽत्मन॑ आ॒त्मनो ऽन्त॑रतरा त॒प्तव्र॑त स्त॒प्तव्र॒तो ऽन्त॑रतरा॒ ऽऽत्मन॑ आ॒त्मनो ऽन्त॑रतरा त॒प्तव्र॑तः । \newline
45. अन्त॑रतरा त॒प्तव्र॑त स्त॒प्तव्र॒तो ऽन्त॑रत॒रा ऽन्त॑रतरा त॒प्तव्र॑तो भवति भवति त॒प्तव्र॒तो ऽन्त॑रत॒रा ऽन्त॑रतरा त॒प्तव्र॑तो भवति । \newline
46. अन्त॑रत॒रेत्यन्त॑र - त॒रा॒ । \newline
47. त॒प्तव्र॑तो भवति भवति त॒प्तव्र॑त स्त॒प्तव्र॑तो भवति॒ मद॑न्तीभि॒र् मद॑न्तीभिर् भवति त॒प्तव्र॑त स्त॒प्तव्र॑तो भवति॒ मद॑न्तीभिः । \newline
48. त॒प्तव्र॑त॒ इति॑ त॒प्त - व्र॒तः॒ । \newline
49. भ॒व॒ति॒ मद॑न्तीभि॒र् मद॑न्तीभिर् भवति भवति॒ मद॑न्तीभिर् मार्जयते मार्जयते॒ मद॑न्तीभिर् भवति भवति॒ मद॑न्तीभिर् मार्जयते । \newline
50. मद॑न्तीभिर् मार्जयते मार्जयते॒ मद॑न्तीभि॒र् मद॑न्तीभिर् मार्जयते॒ निर् णिर् मा᳚र्जयते॒ मद॑न्तीभि॒र् मद॑न्तीभिर् मार्जयते॒ निः । \newline
51. मा॒र्ज॒य॒ते॒ निर् णिर् मा᳚र्जयते मार्जयते॒ निर्. हि हि निर् मा᳚र्जयते मार्जयते॒ निर्. हि । \newline
52. निर्. हि हि निर् णिर् ह्य॑ग्नि र॒ग्निर्. हि निर् णिर् ह्य॑ग्निः । \newline
53. ह्य॑ग्नि र॒ग्निर्. हि ह्य॑ग्निः शी॒तेन॑ शी॒तेना॒ ग्निर्. हि ह्य॑ग्निः शी॒तेन॑ । \newline
54. अ॒ग्निः शी॒तेन॑ शी॒तेना॒ ग्नि र॒ग्निः शी॒तेन॒ वाय॑ति॒ वाय॑ति शी॒तेना॒ ग्नि र॒ग्निः शी॒तेन॒ वाय॑ति । \newline
55. शी॒तेन॒ वाय॑ति॒ वाय॑ति शी॒तेन॑ शी॒तेन॒ वाय॑ति॒ समि॑द्ध्यै॒ समि॑द्ध्यै॒ वाय॑ति शी॒तेन॑ शी॒तेन॒ वाय॑ति॒ समि॑द्ध्यै । \newline
56. वाय॑ति॒ समि॑द्ध्यै॒ समि॑द्ध्यै॒ वाय॑ति॒ वाय॑ति॒ समि॑द्ध्यै॒ या या समि॑द्ध्यै॒ वाय॑ति॒ वाय॑ति॒ समि॑द्ध्यै॒ या । \newline
57. समि॑द्ध्यै॒ या या समि॑द्ध्यै॒ समि॑द्ध्यै॒ या ते॑ ते॒ या समि॑द्ध्यै॒ समि॑द्ध्यै॒ या ते᳚ । \newline
58. समि॑द्ध्या॒ इति॒ सं - इ॒द्ध्यै॒ । \newline
59. या ते॑ ते॒ या या ते॑ अग्ने ऽग्ने ते॒ या या ते॑ अग्ने । \newline
60. ते॒ अ॒ग्ने॒ ऽग्ने॒ ते॒ ते॒ अ॒ग्ने॒ रुद्रि॑या॒ रुद्रि॑या ऽग्ने ते ते अग्ने॒ रुद्रि॑या । \newline
61. अ॒ग्ने॒ रुद्रि॑या॒ रुद्रि॑या ऽग्ने ऽग्ने॒ रुद्रि॑या त॒नू स्त॒नू रुद्रि॑या ऽग्ने ऽग्ने॒ रुद्रि॑या त॒नूः । \newline
62. रुद्रि॑या त॒नू स्त॒नू रुद्रि॑या॒ रुद्रि॑या त॒नू रितीति॑ त॒नू रुद्रि॑या॒ रुद्रि॑या त॒नू रिति॑ । \newline
63. त॒नू रितीति॑ त॒नू स्त॒नू रित्या॑हा॒ हेति॑ त॒नू स्त॒नू रित्या॑ह । \newline
64. इत्या॑हा॒हे तीत्या॑ह॒ स्वया॒ स्वया॒ ऽऽहे तीत्या॑ह॒ स्वया᳚ । \newline
65. आ॒ह॒ स्वया॒ स्वया॑ ऽऽहाह॒ स्वयै॒वैव स्वया॑ ऽऽहाह॒ स्वयै॒व । \newline
66. स्वयै॒वैव स्वया॒ स्वयै॒वैन॑ देन दे॒व स्वया॒ स्वयै॒वैन॑त् । \newline
67. ए॒वैन॑ देन दे॒वै वैन॑द् दे॒वत॑या दे॒वत॑ यैन दे॒वै वैन॑द् दे॒वत॑या । \newline
68. ए॒न॒द् दे॒वत॑या दे॒वत॑ यैन देनद् दे॒वत॑या व्रतयति व्रतयति दे॒वत॑ यैन देनद् दे॒वत॑या व्रतयति । \newline
69. दे॒वत॑या व्रतयति व्रतयति दे॒वत॑या दे॒वत॑या व्रतयति सयोनि॒त्वाय॑ सयोनि॒त्वाय॑ व्रतयति दे॒वत॑या दे॒वत॑या व्रतयति सयोनि॒त्वाय॑ । \newline
70. व्र॒त॒य॒ति॒ स॒यो॒नि॒त्वाय॑ सयोनि॒त्वाय॑ व्रतयति व्रतयति सयोनि॒त्वाय॒ शान्त्यै॒ शान्त्यै॑ सयोनि॒त्वाय॑ व्रतयति व्रतयति सयोनि॒त्वाय॒ शान्त्यै᳚ । \newline
71. स॒यो॒नि॒त्वाय॒ शान्त्यै॒ शान्त्यै॑ सयोनि॒त्वाय॑ सयोनि॒त्वाय॒ शान्त्यै᳚ । \newline
72. स॒यो॒नि॒त्वायेति॑ सयोनि - त्वाय॑ । \newline
73. शान्त्या॒ इति॒ शान्त्यै᳚ । \newline
\pagebreak
\markright{ TS 6.2.3.1  \hfill https://www.vedavms.in \hfill}

\section{ TS 6.2.3.1 }

\textbf{TS 6.2.3.1 } \newline
\textbf{Samhita Paata} \newline

तेषा॒मसु॑राणां ति॒स्रः पुर॑ आस-न्नय॒स्म-य्य॑व॒माऽथ॑ रज॒ताऽथ॒ हरि॑णी॒ ता दे॒वा जेतुं॒ नाश॑क्नुव॒न् ता उ॑प॒सदै॒वाजि॑गीष॒न् तस्मा॑दाहु॒र्यश्चै॒वं ॅवेद॒ यश्च॒ नोप॒सदा॒ वै म॑हापु॒रं ज॑य॒न्तीति॒ त इषुꣳ॒॒ सम॑स्कुर्वता॒- ग्निमनी॑कꣳ॒॒ सोमꣳ॑ श॒ल्यं ॅविष्णुं॒ तेज॑नं॒ ते᳚ऽब्रुव॒न् क इ॒माम॑सिष्य॒तीति॑- [  ] \newline

\textbf{Pada Paata} \newline

तेषा᳚म् । असु॑राणाम् । ति॒स्रः । पुरः॑ । आ॒स॒न्न् । अ॒य॒स्मयी᳚ । अ॒व॒मा । अथ॑ । र॒ज॒ता । अथ॑ । हरि॑णी । ताः । दे॒वाः । जेतु᳚म् । न । अ॒श॒क्नु॒व॒न्न् । ताः । उ॒प॒सदेत्यु॑प - सदा᳚ । ए॒व । अ॒जि॒गी॒ष॒न्न् । तस्मा᳚त् । आ॒हुः॒ । यः । च॒ । ए॒वम् । वेद॑ । यः । च॒ । न । उ॒प॒सदेत्यु॑प - सदा᳚ । वै । म॒हा॒पु॒रमिति॑ महा - पु॒रम् । ज॒य॒न्ति॒ । इति॑ । ते । इषु᳚म् । समिति॑ । अ॒कु॒र्व॒त॒ । अ॒ग्निम् । अनी॑कम् । सोम᳚म् । श॒ल्यम् । विष्णु᳚म् । तेज॑नम् । ते । अ॒ब्रु॒व॒न्न् । कः । इ॒माम् । अ॒सि॒ष्य॒ति॒ । इति॑ ।  \newline


\textbf{Krama Paata} \newline

तेषा॒मसु॑राणाम् । असु॑राणाम् ति॒स्रः । ति॒स्रः पुरः॑ । पुर॑ आसन्न् । आ॒स॒न्न॒य॒स्मयी᳚ । अ॒य॒स्मय्य॑व॒मा । अ॒व॒माऽथ॑ । अथ॑ रज॒ता । र॒ज॒ताऽथ॑ । अथ॒ हरि॑णी । हरि॑णी॒ ताः । ता दे॒वाः । दे॒वा जेतु᳚म् । जेतु॒म् न । नाश॑क्नुवन्न् । अ॒श॒क्नु॒व॒न् ताः । ता उ॑प॒सदा᳚ । उ॒प॒सदै॒व । उ॒प॒सदेत्यु॑प - सदा᳚ । ए॒वाजि॑गीषन्न् । अ॒जि॒गी॒ष॒न् तस्मा᳚त् । तस्मा॑दाहुः । आ॒हु॒र् यः । यश्च॑ । चै॒वम् । ए॒वम् ॅवेद॑ । वेद॒ यः । यश्च॑ । च॒ न । नोप॒सदा᳚ । उ॒प॒सदा॒ वै । उ॒प॒सदेत्यु॑प - सदा᳚ । वै म॑हापु॒रम् । म॒हा॒पु॒रम् ज॑यन्ति । म॒हा॒पु॒रमिति॑ महा - पु॒रम् । ज॒य॒न्तीति॑ । इति॒ ते । त इषु᳚म् । इषुꣳ॒॒ सम् । सम॑स्कुर्वत । अ॒कु॒र्व॒ता॒ग्निम् । अ॒ग्निमनी॑कम् । अनी॑कꣳ॒॒ सोम᳚म् । सोमꣳ॑ श॒ल्यम् । श॒ल्यम् ॅविष्णु᳚म् । विष्णु॒म् तेज॑नम् । तेज॑न॒म् ते । ते᳚ऽब्रुवन्न् । अ॒ब्रु॒व॒न् कः । क इ॒माम् । इ॒माम॑सिष्यति । अ॒सि॒ष्य॒तीति॑ । इति॑ रु॒द्रः \newline

\textbf{Jatai Paata} \newline

1. तेषा॒ मसु॑राणा॒ मसु॑राणा॒म् तेषा॒म् तेषा॒ मसु॑राणाम् । \newline
2. असु॑राणाम् ति॒स्र स्ति॒स्रो ऽसु॑राणा॒ मसु॑राणाम् ति॒स्रः । \newline
3. ति॒स्रः पुरः॒ पुर॑ स्ति॒स्र स्ति॒स्रः पुरः॑ । \newline
4. पुर॑ आसन् नास॒न् पुरः॒ पुर॑ आसन्न् । \newline
5. आ॒स॒न् न॒य॒स्म य्य॑य॒स्म य्या॑सन् नासन् नय॒स्मयी᳚ । \newline
6. अ॒य॒स्म य्य॑व॒मा ऽव॒मा ऽय॒स्म य्य॑य॒स्म य्य॑व॒मा । \newline
7. अ॒व॒मा ऽथाथा॑ व॒मा ऽव॒मा ऽथ॑ । \newline
8. अथ॑ रज॒ता र॑ज॒ता ऽथाथ॑ रज॒ता । \newline
9. र॒ज॒ता ऽथाथ॑ रज॒ता र॑ज॒ता ऽथ॑ । \newline
10. अथ॒ हरि॑णी॒ हरि॒ ण्यथाथ॒ हरि॑णी । \newline
11. हरि॑णी॒ ता स्ता हरि॑णी॒ हरि॑णी॒ ताः । \newline
12. ता दे॒वा दे॒वा स्ता स्ता दे॒वाः । \newline
13. दे॒वा जेतु॒म् जेतु॑म् दे॒वा दे॒वा जेतु᳚म् । \newline
14. जेतु॒न्न न जेतु॒म् जेतु॒न्न । \newline
15. नाश॑क्नुवन् नशक्नुव॒न् न नाश॑क्नुवन्न् । \newline
16. अ॒श॒क्नु॒व॒न् ता स्ता अ॑शक्नुवन् नशक्नुव॒न् ताः । \newline
17. ता उ॑प॒स दो॑प॒सदा॒ ता स्ता उ॑प॒सदा᳚ । \newline
18. उ॒प॒स दै॒वै वोप॒स दो॑प॒स दै॒व । \newline
19. उ॒प॒सदेत्यु॑प - सदा᳚ । \newline
20. ए॒वाजि॑गीषन् नजिगीषन् ने॒वैवाजि॑गीषन्न् । \newline
21. अ॒जि॒गी॒ष॒न् तस्मा॒त् तस्मा॑ दजिगीषन् नजिगीष॒न् तस्मा᳚त् । \newline
22. तस्मा॑ दाहु राहु॒ स्तस्मा॒त् तस्मा॑ दाहुः । \newline
23. आ॒हु॒र् यो य आ॑हु राहु॒र् यः । \newline
24. यश्च॑ च॒ यो यश्च॑ । \newline
25. चै॒व मे॒वम् च॑ चै॒वम् । \newline
26. ए॒वं ॅवेद॒ वेदै॒व मे॒वं ॅवेद॑ । \newline
27. वेद॒ यो यो वेद॒ वेद॒ यः । \newline
28. यश्च॑ च॒ यो यश्च॑ । \newline
29. च॒ न न च॑ च॒ न । \newline
30. नोप॒स दो॑प॒सदा॒ न नोप॒सदा᳚ । \newline
31. उ॒प॒सदा॒ वै वा उ॑प॒स दो॑प॒सदा॒ वै । \newline
32. उ॒प॒सदेत्यु॑प - सदा᳚ । \newline
33. वै म॑हापु॒रम् म॑हापु॒रं ॅवै वै म॑हापु॒रम् । \newline
34. म॒हा॒पु॒रम् ज॑यन्ति जयन्ति महापु॒रम् म॑हापु॒रम् ज॑यन्ति । \newline
35. म॒हा॒पु॒रमिति॑ महा - पु॒रम् । \newline
36. ज॒य॒न्तीतीति॑ जयन्ति जय॒न्तीति॑ । \newline
37. इति॒ ते त इतीति॒ ते । \newline
38. त इषु॒ मिषु॒म् ते त इषु᳚म् । \newline
39. इषुꣳ॒॒ सꣳ स मिषु॒ मिषुꣳ॒॒ सम् । \newline
40. स म॑स्कुर्वता कुर्वत॒ सꣳ स म॑स्कुर्वत । \newline
41. अ॒कु॒र्व॒ता॒ग्नि म॒ग्नि म॑कुर्वता कुर्वता॒ग्निम् । \newline
42. अ॒ग्नि मनी॑क॒ मनी॑क म॒ग्नि म॒ग्नि मनी॑कम् । \newline
43. अनी॑कꣳ॒॒ सोमꣳ॒॒ सोम॒ मनी॑क॒ मनी॑कꣳ॒॒ सोम᳚म् । \newline
44. सोमꣳ॑ श॒ल्यꣳ श॒ल्यꣳ सोमꣳ॒॒ सोमꣳ॑ श॒ल्यम् । \newline
45. श॒ल्यं ॅविष्णुं॒ ॅविष्णुꣳ॑ श॒ल्यꣳ श॒ल्यं ॅविष्णु᳚म् । \newline
46. विष्णु॒म् तेज॑न॒म् तेज॑नं॒ ॅविष्णुं॒ ॅविष्णु॒म् तेज॑नम् । \newline
47. तेज॑न॒म् ते ते तेज॑न॒म् तेज॑न॒म् ते । \newline
48. ते᳚ ऽब्रुवन् नब्रुव॒न् ते ते᳚ ऽब्रुवन्न् । \newline
49. अ॒ब्रु॒व॒न् कः को᳚ ऽब्रुवन् नब्रुव॒न् कः । \newline
50. क इ॒मा मि॒माम् कः क इ॒माम् । \newline
51. इ॒मा म॑सिष्य त्यसिष्यती॒मा मि॒मा म॑सिष्यति । \newline
52. अ॒सि॒ष्य॒तीती त्य॑सिष्य त्यसिष्य॒तीति॑ । \newline
53. इति॑ रु॒द्रो रु॒द्र इतीति॑ रु॒द्रः । \newline

\textbf{Ghana Paata } \newline

1. तेषा॒ मसु॑राणा॒ मसु॑राणा॒म् तेषा॒म् तेषा॒ मसु॑राणाम् ति॒स्र स्ति॒स्रो ऽसु॑राणा॒म् तेषा॒म् तेषा॒ मसु॑राणाम् ति॒स्रः । \newline
2. असु॑राणाम् ति॒स्र स्ति॒स्रो ऽसु॑राणा॒ मसु॑राणाम् ति॒स्रः पुरः॒ पुर॑ स्ति॒स्रो ऽसु॑राणा॒ मसु॑राणाम् ति॒स्रः पुरः॑ । \newline
3. ति॒स्रः पुरः॒ पुर॑ स्ति॒स्र स्ति॒स्रः पुर॑ आसन् नास॒न् पुर॑ स्ति॒स्र स्ति॒स्रः पुर॑ आसन्न् । \newline
4. पुर॑ आसन् नास॒न् पुरः॒ पुर॑ आसन् नय॒स्मय्य॑ य॒स्मय्या॑स॒न् पुरः॒ पुर॑ आसन् नय॒स्मयी᳚ । \newline
5. आ॒स॒न् न॒य॒स्मय्य॑ य॒स्मय्या॑सन् नासन् नय॒स्म य्य॑व॒मा ऽव॒मा ऽय॒स्मय्या॑ सन् नासन् 
नय॒स्मय्य॑व॒मा । \newline
6. अ॒य॒स्मय्य॑व॒मा ऽव॒मा ऽय॒स्मय्य॑ य॒स्मय्य॑ व॒मा ऽथाथा॑ व॒मा ऽय॒स्मय्य॑ य॒स्मय्य॑ व॒मा ऽथ॑ । \newline
7. अ॒व॒मा ऽथाथा॑ व॒मा ऽव॒मा ऽथ॑ रज॒ता र॑ज॒ता ऽथा॑ व॒मा ऽव॒मा ऽथ॑ रज॒ता । \newline
8. अथ॑ रज॒ता र॑ज॒ता ऽथाथ॑ रज॒ता ऽथाथ॑ रज॒ता ऽथाथ॑ रज॒ता ऽथ॑ । \newline
9. र॒ज॒ता ऽथाथ॑ रज॒ता र॑ज॒ता ऽथ॒ हरि॑णी॒ हरि॒ण्यथ॑ रज॒ता र॑ज॒ता ऽथ॒ हरि॑णी । \newline
10. अथ॒ हरि॑णी॒ हरि॒ण्य थाथ॒ हरि॑णी॒ ता स्ता हरि॒ण्य थाथ॒ हरि॑णी॒ ताः । \newline
11. हरि॑णी॒ ता स्ता हरि॑णी॒ हरि॑णी॒ ता दे॒वा दे॒वा स्ता हरि॑णी॒ हरि॑णी॒ ता दे॒वाः । \newline
12. ता दे॒वा दे॒वा स्ता स्ता दे॒वा जेतु॒म् जेतु॑म् दे॒वा स्ता स्ता दे॒वा जेतु᳚म् । \newline
13. दे॒वा जेतु॒म् जेतु॑म् दे॒वा दे॒वा जेतु॒न् न न जेतु॑म् दे॒वा दे॒वा जेतु॒न् न । \newline
14. जेतु॒न् न न जेतु॒म् जेतु॒न् नाश॑क्नुवन् नशक्नुव॒न् न जेतु॒म् जेतु॒न् नाश॑क्नुवन्न् । \newline
15. नाश॑क्नुवन् नशक्नुव॒न् न नाश॑क्नुव॒न् ता स्ता अ॑शक्नुव॒न् न नाश॑क्नुव॒न् ताः । \newline
16. अ॒श॒क्नु॒व॒न् ता स्ता अ॑शक्नुवन् नशक्नुव॒न् ता उ॑प॒सदो॑ प॒सदा॒ ता अ॑शक्नुवन् नशक्नुव॒न् ता उ॑प॒सदा᳚ । \newline
17. ता उ॑प॒स दो॑प॒सदा॒ ता स्ता उ॑प॒सदै॒ वैवोप॒सदा॒ ता स्ता उ॑प॒स दै॒व । \newline
18. उ॒प॒सदै॒ वैवोप॒स दो॑प॒सदै॒वा जि॑गीषन् नजिगीषन् ने॒वोप॒स दो॑प॒सदै॒वा जि॑गीषन्न् । \newline
19. उ॒प॒सदेत्यु॑प - सदा᳚ । \newline
20. ए॒वाजि॑गीषन् नजिगीषन् ने॒वै वाजि॑गीष॒न् तस्मा॒त् तस्मा॑ दजिगीषन् ने॒वैवा जि॑गीष॒न् तस्मा᳚त् । \newline
21. अ॒जि॒गी॒ष॒न् तस्मा॒त् तस्मा॑ दजिगीषन् नजिगीष॒न् तस्मा॑ दाहु राहु॒ स्तस्मा॑ दजिगीषन् नजिगीष॒न् तस्मा॑ दाहुः । \newline
22. तस्मा॑ दाहु राहु॒ स्तस्मा॒त् तस्मा॑ दाहु॒र् यो य आ॑हु॒ स्तस्मा॒त् तस्मा॑ दाहु॒र् यः । \newline
23. आ॒हु॒र् यो य आ॑हु राहु॒र् यश्च॑ च॒ य आ॑हु राहु॒र् यश्च॑ । \newline
24. यश्च॑ च॒ यो यश्चै॒व मे॒वम् च॒ यो यश्चै॒वम् । \newline
25. चै॒व मे॒वम् च॑ चै॒वं ॅवेद॒ वेदै॒वम् च॑ चै॒वं ॅवेद॑ । \newline
26. ए॒वं ॅवेद॒ वेदै॒व मे॒वं ॅवेद॒ यो यो वेदै॒व मे॒वं ॅवेद॒ यः । \newline
27. वेद॒ यो यो वेद॒ वेद॒ यश्च॑ च॒ यो वेद॒ वेद॒ यश्च॑ । \newline
28. यश्च॑ च॒ यो यश्च॒ न न च॒ यो यश्च॒ न । \newline
29. च॒ न न च॑ च॒ नोप॒सदो॑ प॒सदा॒ न च॑ च॒ नोप॒सदा᳚ । \newline
30. नोप॒सदो॑ प॒सदा॒ न नोप॒सदा॒ वै वा उ॑प॒सदा॒ न नोप॒सदा॒ वै । \newline
31. उ॒प॒सदा॒ वै वा उ॑प॒सदो॑ प॒सदा॒ वै म॑हापु॒रम् म॑हापु॒रं ॅवा उ॑प॒सदो॑ प॒सदा॒ वै म॑हापु॒रम् । \newline
32. उ॒प॒सदेत्यु॑प - सदा᳚ । \newline
33. वै म॑हापु॒रम् म॑हापु॒रं ॅवै वै म॑हापु॒रम् ज॑यन्ति जयन्ति महापु॒रं ॅवै वै म॑हापु॒रम् ज॑यन्ति । \newline
34. म॒हा॒पु॒रम् ज॑यन्ति जयन्ति महापु॒रम् म॑हापु॒रम् ज॑य॒न्ती तीति॑ जयन्ति महापु॒रम् म॑हापु॒रम् ज॑य॒न्तीति॑ । \newline
35. म॒हा॒पु॒रमिति॑ महा - पु॒रम् । \newline
36. ज॒य॒न्ती तीति॑ जयन्ति जय॒न्तीति॒ ते त इति॑ जयन्ति जय॒न्तीति॒ ते । \newline
37. इति॒ ते त इतीति॒ त इषु॒ मिषु॒म् त इतीति॒ त इषु᳚म् । \newline
38. त इषु॒ मिषु॒म् ते त इषुꣳ॒॒ सꣳ स मिषु॒म् ते त इषुꣳ॒॒ सम् । \newline
39. इषुꣳ॒॒ सꣳ स मिषु॒ मिषुꣳ॒॒ स म॑स्कुर्वता कुर्वत॒ स मिषु॒ मिषुꣳ॒॒ स म॑स्कुर्वत । \newline
40. स म॑स्कुर्वता कुर्वत॒ सꣳ स म॑स्कुर्व ता॒ग्नि म॒ग्नि म॑कुर्वत॒ सꣳ स म॑स्कुर्व ता॒ग्निम् । \newline
41. अ॒कु॒र्व॒ ता॒ग्नि म॒ग्नि म॑कुर्वता कुर्वता॒ग्नि मनी॑क॒ मनी॑क म॒ग्नि म॑कुर्वता कुर्व ता॒ग्नि मनी॑कम् । \newline
42. अ॒ग्नि मनी॑क॒ मनी॑क म॒ग्नि म॒ग्नि मनी॑कꣳ॒॒ सोमꣳ॒॒ सोम॒ मनी॑क म॒ग्नि म॒ग्नि मनी॑कꣳ॒॒ सोम᳚म् । \newline
43. अनी॑कꣳ॒॒ सोमꣳ॒॒ सोम॒ मनी॑क॒ मनी॑कꣳ॒॒ सोमꣳ॑ श॒ल्यꣳ श॒ल्यꣳ सोम॒ मनी॑क॒ मनी॑कꣳ॒॒ सोमꣳ॑ श॒ल्यम् । \newline
44. सोमꣳ॑ श॒ल्यꣳ श॒ल्यꣳ सोमꣳ॒॒ सोमꣳ॑ श॒ल्यं ॅविष्णुं॒ ॅविष्णुꣳ॑ श॒ल्यꣳ सोमꣳ॒॒ सोमꣳ॑ श॒ल्यं ॅविष्णु᳚म् । \newline
45. श॒ल्यं ॅविष्णुं॒ ॅविष्णुꣳ॑ श॒ल्यꣳ श॒ल्यं ॅविष्णु॒म् तेज॑न॒म् तेज॑नं॒ ॅविष्णुꣳ॑ श॒ल्यꣳ श॒ल्यं ॅविष्णु॒म् तेज॑नम् । \newline
46. विष्णु॒म् तेज॑न॒म् तेज॑नं॒ ॅविष्णुं॒ ॅविष्णु॒म् तेज॑न॒म् ते ते तेज॑नं॒ ॅविष्णुं॒ ॅविष्णु॒म् तेज॑न॒म् ते । \newline
47. तेज॑न॒म् ते ते तेज॑न॒म् तेज॑न॒म् ते᳚ ऽब्रुवन् नब्रुव॒न् ते तेज॑न॒म् तेज॑न॒म् ते᳚ ऽब्रुवन्न् । \newline
48. ते᳚ ऽब्रुवन् नब्रुव॒न् ते ते᳚ ऽब्रुव॒न् कः को᳚ ऽब्रुव॒न् ते ते᳚ ऽब्रुव॒न् कः । \newline
49. अ॒ब्रु॒व॒न् कः को᳚ ऽब्रुवन् नब्रुव॒न् क इ॒मा मि॒माम् को᳚ ऽब्रुवन् नब्रुव॒न् क इ॒माम् । \newline
50. क इ॒मा मि॒माम् कः क इ॒मा म॑सिष्य त्यसिष्यती॒ माम् कः क इ॒मा म॑सिष्यति । \newline
51. इ॒मा म॑सिष्य त्यसिष्य ती॒मा मि॒मा म॑सिष्य॒ तीती त्य॑सिष्य ती॒मा मि॒मा म॑सिष्य॒ तीति॑ । \newline
52. अ॒सि॒ष्य॒ तीती त्य॑सिष्य त्यसिष्य॒तीति॑ रु॒द्रो रु॒द्र इत्य॑सिष्य त्यसिष्य॒ तीति॑ रु॒द्रः । \newline
53. इति॑ रु॒द्रो रु॒द्र इतीति॑ रु॒द्र इतीति॑ रु॒द्र इतीति॑ रु॒द्र इति॑ । \newline
\pagebreak
\markright{ TS 6.2.3.2  \hfill https://www.vedavms.in \hfill}

\section{ TS 6.2.3.2 }

\textbf{TS 6.2.3.2 } \newline
\textbf{Samhita Paata} \newline

रु॒द्र इत्य॑ब्रुवन् रु॒द्रो वै क्रू॒रः सो᳚ऽस्य॒त्विति॒ सो᳚ऽब्रवी॒द् वरं॑ ॅवृणा अ॒हमे॒व प॑शू॒ना-मधि॑पतिरसा॒नीति॒ तस्मा᳚द् रु॒द्रः प॑शू॒ना-मधि॑पति॒स्ताꣳ रु॒द्रोऽवा॑सृज॒थ् स ति॒स्रः पुरो॑ भि॒त्त्वैभ्यो लो॒केभ्यो- ऽसु॑रा॒न् प्राणु॑दत॒ यदु॑प॒सद॑ उपस॒द्यन्ते॒ भ्रातृ॑व्यपराणुत्यै॒ नान्यामाहु॑तिं पु॒रस्ता᳚-ज्जुहुया॒द्-यद॒न्यामाहु॑तिं पु॒रस्ता᳚ज्जुहु॒या- [  ] \newline

\textbf{Pada Paata} \newline

रु॒द्रः । इति॑ । अ॒ब्रु॒व॒न्न् । रु॒द्रः । वै । क्रू॒रः । सः । अ॒स्य॒तु॒ । इति॑ । सः । अ॒ब्र॒वी॒त् । वर᳚म् । वृ॒णै॒ । अ॒हम् । ए॒व । प॒शू॒नाम् । अधि॑पति॒रित्यधि॑ - प॒तिः॒ । अ॒सा॒नि॒ । इति॑ । तस्मा᳚त् । रु॒द्रः । प॒शू॒नाम् । अधि॑पति॒रित्यधि॑ - प॒तिः॒ । ताम् । रु॒द्रः । अवेति॑ । अ॒सृ॒ज॒त् । सः । ति॒स्रः । पुरः॑ । भि॒त्वा । ए॒भ्यः । लो॒केभ्यः॑ । असु॑रान् । प्रेति॑ । अ॒नु॒द॒त॒ । यत् । उ॒प॒सद॒ इत्यु॑प - सदः॑ । उ॒प॒स॒द्यन्त॒ इत्यु॑प-स॒द्यन्ते᳚ । भ्रातृ॑व्यपराणुत्या॒ इति॒ भ्रातृ॑व्य-प॒रा॒णु॒त्यै॒ । न । अ॒न्याम् । आहु॑ति॒मित्या - हु॒ति॒म् । पु॒रस्ता᳚त् । जु॒हु॒या॒त् । यत् । अ॒न्याम् । आहु॑ति॒मित्या - हु॒ति॒म् । पु॒रस्ता᳚त् । जु॒हु॒यात् ।  \newline


\textbf{Krama Paata} \newline

रु॒द्र इति॑ । इत्य॑ब्रुवन्न् । अ॒ब्रु॒व॒न् रु॒द्रः । रु॒द्रो वै । वै क्रू॒रः । क्रू॒रः सः । सो᳚ऽस्यतु । अ॒स्य॒त्विति॑ । इति॒ सः । सो᳚ऽब्रवीत् । अ॒ब्र॒वी॒द् वर᳚म् । वर॑म् ॅवृणै । वृ॒णा॒ अ॒हम् । अ॒हमे॒व । ए॒व प॑शू॒नाम् । प॒शू॒नामधि॑पतिः । अधि॑पतिरसानि । अधि॑पति॒रित्यधि॑ - प॒तिः॒ । अ॒सा॒नीति॑ । इति॒ तस्मा᳚त् । तस्मा᳚द् रु॒द्रः । रु॒द्रः प॑शू॒नाम् । प॒शू॒नामधि॑पतिः । अधि॑पति॒स्ताम् । अधि॑पति॒रित्यधि॑ - प॒तिः॒ । ताꣳ रु॒द्रः । रु॒द्रोऽव॑ । अवा॑सृजत् । अ॒सृ॒ज॒थ् सः । स ति॒स्रः । ति॒स्रः पुरः॑ । पुरो॑ भि॒त्वा । भि॒त्वैभ्यः । ए॒भ्यो लो॒केभ्यः॑ । लो॒केभ्योऽसु॑रान् । असु॑रा॒न् प्र । प्राणु॑दत । अ॒नु॒द॒त॒ यत् । यदु॑प॒सदः॑ । उ॒प॒सद॑ उपस॒द्यन्ते᳚ । उ॒प॒सद॒ इत्यु॑प - सदः॑ । उ॒प॒स॒द्यन्ते॒ भ्रातृ॑व्यपराणुत्यै । उ॒प॒स॒द्यन्त॒ इत्यु॑प - स॒द्यन्ते᳚ । भ्रातृ॑व्यपराणुत्यै॒ न । भ्रातृ॑व्यपराणुत्या॒ इति॒ भ्रातृ॑व्य - प॒रा॒णु॒त्यै॒ । नान्याम् । अ॒न्यामाहु॑तिम् । आहु॑तिम् पु॒रस्ता᳚त् । आहु॑ति॒मित्या - हु॒ति॒म् । पु॒रस्ता᳚ज् जुहुयात् । जु॒हु॒या॒द् यत् । यद॒न्याम् । अ॒न्यामाहु॑तिम् । आहु॑तिम् पु॒रस्ता᳚त् । आहु॑ति॒मित्या - हु॒ति॒म् । पु॒रस्ता᳚ज् जुहु॒यात् । जु॒हु॒याद॒न्यत् \newline

\textbf{Jatai Paata} \newline

1. रु॒द्र इतीति॑ रु॒द्रो रु॒द्र इति॑ । \newline
2. इत्य॑ब्रुवन् नब्रुव॒न् निती त्य॑ब्रुवन्न् । \newline
3. अ॒ब्रु॒व॒न् रु॒द्रो रु॒द्रो᳚ ऽब्रुवन् नब्रुवन् रु॒द्रः । \newline
4. रु॒द्रो वै वै रु॒द्रो रु॒द्रो वै । \newline
5. वै क्रू॒रः क्रू॒रो वै वै क्रू॒रः । \newline
6. क्रू॒रः स स क्रू॒रः क्रू॒रः सः । \newline
7. सो᳚ ऽस्य त्वस्यतु॒ स सो᳚ ऽस्यतु । \newline
8. अ॒स्य॒ त्विती त्य॑स्य त्वस्य॒त्विति॑ । \newline
9. इति॒ स स इतीति॒ सः । \newline
10. सो᳚ ऽब्रवी दब्रवी॒थ् स सो᳚ ऽब्रवीत् । \newline
11. अ॒ब्र॒वी॒द् वरं॒ ॅवर॑ मब्रवी दब्रवी॒द् वर᳚म् । \newline
12. वरं॑ ॅवृणै वृणै॒ वरं॒ ॅवरं॑ ॅवृणै । \newline
13. वृ॒णा॒ अ॒ह म॒हं ॅवृ॑णै वृणा अ॒हम् । \newline
14. अ॒ह मे॒वैवाह म॒ह मे॒व । \newline
15. ए॒व प॑शू॒नाम् प॑शू॒ना मे॒वैव प॑शू॒नाम् । \newline
16. प॒शू॒ना मधि॑पति॒ रधि॑पतिः पशू॒नाम् प॑शू॒ना मधि॑पतिः । \newline
17. अधि॑पति रसा न्यसा॒ न्यधि॑पति॒ रधि॑पति रसानि । \newline
18. अधि॑पति॒रित्यधि॑ - प॒तिः॒ । \newline
19. अ॒सा॒नीती त्य॑सा न्यसा॒नीति॑ । \newline
20. इति॒ तस्मा॒त् तस्मा॒ दितीति॒ तस्मा᳚त् । \newline
21. तस्मा᳚द् रु॒द्रो रु॒द्र स्तस्मा॒त् तस्मा᳚द् रु॒द्रः । \newline
22. रु॒द्रः प॑शू॒नाम् प॑शू॒नाꣳ रु॒द्रो रु॒द्रः प॑शू॒नाम् । \newline
23. प॒शू॒ना मधि॑पति॒ रधि॑पतिः पशू॒नाम् प॑शू॒ना मधि॑पतिः । \newline
24. अधि॑पति॒ स्ताम् ता मधि॑पति॒ रधि॑पति॒ स्ताम् । \newline
25. अधि॑पति॒रित्यधि॑ - प॒तिः॒ । \newline
26. ताꣳ रु॒द्रो रु॒द्र स्ताम् ताꣳ रु॒द्रः । \newline
27. रु॒द्रो ऽवाव॑ रु॒द्रो रु॒द्रो ऽव॑ । \newline
28. अवा॑सृज दसृज॒ दवावा॑ सृजत् । \newline
29. अ॒सृ॒ज॒थ् स सो॑ ऽसृज दसृज॒थ् सः । \newline
30. स ति॒स्र स्ति॒स्रः स स ति॒स्रः । \newline
31. ति॒स्रः पुरः॒ पुर॑ स्ति॒स्र स्ति॒स्रः पुरः॑ । \newline
32. पुरो॑ भि॒त्वा भि॒त्वा पुरः॒ पुरो॑ भि॒त्वा । \newline
33. भि॒त्वैभ्य ए॒भ्यो भि॒त्वा भि॒त्वैभ्यः । \newline
34. ए॒भ्यो लो॒केभ्यो॑ लो॒केभ्य॑ ए॒भ्य ए॒भ्यो लो॒केभ्यः॑ । \newline
35. लो॒केभ्यो ऽसु॑रा॒ नसु॑रान् ॅलो॒केभ्यो॑ लो॒केभ्यो ऽसु॑रान् । \newline
36. असु॑रा॒न् प्र प्रासु॑रा॒ नसु॑रा॒न् प्र । \newline
37. प्राणु॑दता नुदत॒ प्र प्राणु॑दत । \newline
38. अ॒नु॒द॒त॒ यद् यद॑नुदता नुदत॒ यत् । \newline
39. यदु॑प॒सद॑ उप॒सदो॒ यद् यदु॑प॒सदः॑ । \newline
40. उ॒प॒सद॑ उपस॒द्यन्त॑ उपस॒द्यन्त॑ उप॒सद॑ उप॒सद॑ उपस॒द्यन्ते᳚ । \newline
41. उ॒प॒सद॒ इत्यु॑प - सदः॑ । \newline
42. उ॒प॒स॒द्यन्ते॒ भ्रातृ॑व्यपराणुत्यै॒ भ्रातृ॑व्यपराणुत्या उपस॒द्यन्त॑ उपस॒द्यन्ते॒ भ्रातृ॑व्यपराणुत्यै । \newline
43. उ॒प॒स॒द्यन्त॒ इत्यु॑प - स॒द्यन्ते᳚ । \newline
44. भ्रातृ॑व्यपराणुत्यै॒ न न भ्रातृ॑व्यपराणुत्यै॒ भ्रातृ॑व्यपराणुत्यै॒ न । \newline
45. भ्रातृ॑व्यपराणुत्या॒ इति॒ भ्रातृ॑व्य - प॒रा॒णु॒त्यै॒ । \newline
46. नान्या म॒न्यान्न नान्याम् । \newline
47. अ॒न्या माहु॑ति॒ माहु॑ति म॒न्या म॒न्या माहु॑तिम् । \newline
48. आहु॑तिम् पु॒रस्ता᳚त् पु॒रस्ता॒ दाहु॑ति॒ माहु॑तिम् पु॒रस्ता᳚त् । \newline
49. आहु॑ति॒मित्या - हु॒ति॒म् । \newline
50. पु॒रस्ता᳚ज् जुहुयाज् जुहुयात् पु॒रस्ता᳚त् पु॒रस्ता᳚ज् जुहुयात् । \newline
51. जु॒हु॒या॒द् यद् यज् जु॑हुयाज् जुहुया॒द् यत् । \newline
52. यद॒न्या म॒न्यां ॅयद् यद॒न्याम् । \newline
53. अ॒न्या माहु॑ति॒ माहु॑ति म॒न्या म॒न्या माहु॑तिम् । \newline
54. आहु॑तिम् पु॒रस्ता᳚त् पु॒रस्ता॒ दाहु॑ति॒ माहु॑तिम् पु॒रस्ता᳚त् । \newline
55. आहु॑ति॒मित्या - हु॒ति॒म् । \newline
56. पु॒रस्ता᳚ज् जुहु॒याज् जु॑हु॒यात् पु॒रस्ता᳚त् पु॒रस्ता᳚ज् जुहु॒यात् । \newline
57. जु॒हु॒या द॒न्य द॒न्यज् जु॑हु॒याज् जु॑हु॒या द॒न्यत् । \newline

\textbf{Ghana Paata } \newline

1. रु॒द्र इतीति॑ रु॒द्रो रु॒द्र इत्य॑ब्रुवन् नब्रुव॒न्निति॑ रु॒द्रो रु॒द्र इत्य॑ब्रुवन्न् । \newline
2. इत्य॑ब्रुवन् नब्रुव॒न्निती त्य॑ब्रुवन् रु॒द्रो रु॒द्रो᳚ ऽब्रुव॒न्निती त्य॑ब्रुवन् रु॒द्रः । \newline
3. अ॒ब्रु॒व॒न् रु॒द्रो रु॒द्रो᳚ ऽब्रुवन् नब्रुवन् रु॒द्रो वै वै रु॒द्रो᳚ ऽब्रुवन् नब्रुवन् रु॒द्रो वै । \newline
4. रु॒द्रो वै वै रु॒द्रो रु॒द्रो वै क्रू॒रः क्रू॒रो वै रु॒द्रो रु॒द्रो वै क्रू॒रः । \newline
5. वै क्रू॒रः क्रू॒रो वै वै क्रू॒रः स स क्रू॒रो वै वै क्रू॒रः सः । \newline
6. क्रू॒रः स स क्रू॒रः क्रू॒रः सो᳚ ऽस्यत्व स्यतु॒ स क्रू॒रः क्रू॒रः सो᳚ ऽस्यतु । \newline
7. सो᳚ ऽस्यत्व स्यतु॒ स सो᳚ ऽस्य॒त्वि तीत्य॑ स्यतु॒ स सो᳚ ऽस्य॒त्विति॑ । \newline
8. अ॒स्य॒त्वि तीत्य॑स्यत्व स्य॒त्विति॒ स स इत्य॑स्य त्वस्य॒त्विति॒ सः । \newline
9. इति॒ स स इतीति॒ सो᳚ ऽब्रवी दब्रवी॒थ् स इतीति॒ सो᳚ ऽब्रवीत् । \newline
10. सो᳚ ऽब्रवी दब्रवी॒थ् स सो᳚ ऽब्रवी॒द् वरं॒ ॅवर॑ मब्रवी॒थ् स सो᳚ ऽब्रवी॒द् वर᳚म् । \newline
11. अ॒ब्र॒वी॒द् वरं॒ ॅवर॑ मब्रवी दब्रवी॒द् वरं॑ ॅवृणै वृणै॒ वर॑ मब्रवी दब्रवी॒द् वरं॑ ॅवृणै । \newline
12. वरं॑ ॅवृणै वृणै॒ वरं॒ ॅवरं॑ ॅवृणा अ॒ह म॒हं ॅवृ॑णै॒ वरं॒ ॅवरं॑ ॅवृणा अ॒हम् । \newline
13. वृ॒णा॒ अ॒ह म॒हं ॅवृ॑णै वृणा अ॒ह मे॒वै वाहं ॅवृ॑णै वृणा अ॒ह मे॒व । \newline
14. अ॒ह मे॒वै वाह म॒ह मे॒व प॑शू॒नाम् प॑शू॒ना मे॒वाह म॒ह मे॒व प॑शू॒नाम् । \newline
15. ए॒व प॑शू॒नाम् प॑शू॒ना मे॒वैव प॑शू॒ना मधि॑पति॒ रधि॑पतिः पशू॒ना मे॒वैव प॑शू॒ना मधि॑पतिः । \newline
16. प॒शू॒ना मधि॑पति॒ रधि॑पतिः पशू॒नाम् प॑शू॒ना मधि॑पति रसान्य सा॒न्यधि॑पतिः पशू॒नाम् प॑शू॒ना मधि॑पति रसानि । \newline
17. अधि॑पति रसान्य सा॒न्यधि॑पति॒ रधि॑पति रसा॒नी तीत्य॑सा॒ न्यधि॑पति॒ रधि॑पति रसा॒ नीति॑ । \newline
18. अधि॑पति॒रित्यधि॑ - प॒तिः॒ । \newline
19. अ॒सा॒नी तीत्य॑सा न्यसा॒ नीति॒ तस्मा॒त् तस्मा॒ दित्य॑सा न्यसा॒ नीति॒ तस्मा᳚त् । \newline
20. इति॒ तस्मा॒त् तस्मा॒दि तीति॒ तस्मा᳚द् रु॒द्रो रु॒द्र स्तस्मा॒दि तीति॒ तस्मा᳚द् रु॒द्रः । \newline
21. तस्मा᳚द् रु॒द्रो रु॒द्र स्तस्मा॒त् तस्मा᳚द् रु॒द्रः प॑शू॒नाम् प॑शू॒नाꣳ रु॒द्र स्तस्मा॒त् तस्मा᳚द् रु॒द्रः प॑शू॒नाम् । \newline
22. रु॒द्रः प॑शू॒नाम् प॑शू॒नाꣳ रु॒द्रो रु॒द्रः प॑शू॒ना मधि॑पति॒ रधि॑पतिः पशू॒नाꣳ रु॒द्रो रु॒द्रः प॑शू॒ना मधि॑पतिः । \newline
23. प॒शू॒ना मधि॑पति॒ रधि॑पतिः पशू॒नाम् प॑शू॒ना मधि॑पति॒ स्ताम् ता मधि॑पतिः पशू॒नाम् प॑शू॒ना मधि॑पति॒ स्ताम् । \newline
24. अधि॑पति॒ स्ताम् ता मधि॑पति॒ रधि॑पति॒ स्ताꣳ रु॒द्रो रु॒द्र स्ता मधि॑पति॒ रधि॑पति॒ स्ताꣳ रु॒द्रः । \newline
25. अधि॑पति॒रित्यधि॑ - प॒तिः॒ । \newline
26. ताꣳ रु॒द्रो रु॒द्र स्ताम् ताꣳ रु॒द्रो ऽवाव॑ रु॒द्र स्ताम् ताꣳ रु॒द्रो ऽव॑ । \newline
27. रु॒द्रो ऽवाव॑ रु॒द्रो रु॒द्रो ऽवा॑ सृज दसृज॒ दव॑ रु॒द्रो रु॒द्रो ऽवा॑ सृजत् । \newline
28. अवा॑ सृज दसृज॒ दवावा॑ सृज॒थ् स सो॑ ऽसृज॒ दवावा॑ सृज॒थ् सः । \newline
29. अ॒सृ॒ज॒थ् स सो॑ ऽसृज दसृज॒थ् स ति॒स्र स्ति॒स्रः सो॑ ऽसृज दसृज॒थ् स ति॒स्रः । \newline
30. स ति॒स्र स्ति॒स्रः स स ति॒स्रः पुरः॒ पुर॑ स्ति॒स्रः स स ति॒स्रः पुरः॑ । \newline
31. ति॒स्रः पुरः॒ पुर॑ स्ति॒स्र स्ति॒स्रः पुरो॑ भि॒त्वा भि॒त्वा पुर॑ स्ति॒स्र स्ति॒स्रः पुरो॑ भि॒त्वा । \newline
32. पुरो॑ भि॒त्वा भि॒त्वा पुरः॒ पुरो॑ भि॒त्वैभ्य ए॒भ्यो भि॒त्वा पुरः॒ पुरो॑ भि॒त्वैभ्यः । \newline
33. भि॒त्वैभ्य ए॒भ्यो भि॒त्वा भि॒त्वैभ्यो लो॒केभ्यो॑ लो॒केभ्य॑ ए॒भ्यो भि॒त्वा भि॒त्वैभ्यो लो॒केभ्यः॑ । \newline
34. ए॒भ्यो लो॒केभ्यो॑ लो॒केभ्य॑ ए॒भ्य ए॒भ्यो लो॒केभ्यो ऽसु॑रा॒ नसु॑रान् ॅलो॒केभ्य॑ ए॒भ्य ए॒भ्यो लो॒केभ्यो ऽसु॑रान् । \newline
35. लो॒केभ्यो ऽसु॑रा॒ नसु॑रान् ॅलो॒केभ्यो॑ लो॒केभ्यो ऽसु॑रा॒न् प्र प्रासु॑रान् ॅलो॒केभ्यो॑ लो॒केभ्यो ऽसु॑रा॒न् प्र । \newline
36. असु॑रा॒न् प्र प्रासु॑रा॒ नसु॑रा॒न् प्राणु॑दता नुदत॒ प्रासु॑रा॒ नसु॑रा॒न् प्राणु॑दत । \newline
37. प्राणु॑दता नुदत॒ प्र प्राणु॑दत॒ यद् यद॑नुदत॒ प्र प्राणु॑दत॒ यत् । \newline
38. अ॒नु॒द॒त॒ यद् यद॑नुदता नुदत॒ यदु॑प॒सद॑ उप॒सदो॒ यद॑नुदता नुदत॒ यदु॑प॒सदः॑ । \newline
39. यदु॑प॒सद॑ उप॒सदो॒ यद् यदु॑प॒सद॑ उपस॒द्यन्त॑ उपस॒द्यन्त॑ उप॒सदो॒ यद् यदु॑प॒सद॑ उपस॒द्यन्ते᳚ । \newline
40. उ॒प॒सद॑ उपस॒द्यन्त॑ उपस॒द्यन्त॑ उप॒सद॑ उप॒सद॑ उपस॒द्यन्ते॒ भ्रातृ॑व्यपराणुत्यै॒ भ्रातृ॑व्यपराणुत्या उपस॒द्यन्त॑ उप॒सद॑ उप॒सद॑ उपस॒द्यन्ते॒ भ्रातृ॑व्यपराणुत्यै । \newline
41. उ॒प॒सद॒ इत्यु॑प - सदः॑ । \newline
42. उ॒प॒स॒द्यन्ते॒ भ्रातृ॑व्यपराणुत्यै॒ भ्रातृ॑व्यपराणुत्या उपस॒द्यन्त॑ उपस॒द्यन्ते॒ भ्रातृ॑व्यपराणुत्यै॒ न न भ्रातृ॑व्यपराणुत्या उपस॒द्यन्त॑ उपस॒द्यन्ते॒ भ्रातृ॑व्यपराणुत्यै॒ न । \newline
43. उ॒प॒स॒द्यन्त॒ इत्यु॑प - स॒द्यन्ते᳚ । \newline
44. भ्रातृ॑व्यपराणुत्यै॒ न न भ्रातृ॑व्यपराणुत्यै॒ भ्रातृ॑व्यपराणुत्यै॒ नान्या म॒न्यान् न भ्रातृ॑व्यपराणुत्यै॒ भ्रातृ॑व्यपराणुत्यै॒ नान्याम् । \newline
45. भ्रातृ॑व्यपराणुत्या॒ इति॒ भ्रातृ॑व्य - प॒रा॒णु॒त्यै॒ । \newline
46. नान्या म॒न्यान् न नान्या माहु॑ति॒ माहु॑ति म॒न्यान् न नान्या माहु॑तिम् । \newline
47. अ॒न्या माहु॑ति॒ माहु॑ति म॒न्या म॒न्या माहु॑तिम् पु॒रस्ता᳚त् पु॒रस्ता॒ दाहु॑ति म॒न्या म॒न्या माहु॑तिम् पु॒रस्ता᳚त् । \newline
48. आहु॑तिम् पु॒रस्ता᳚त् पु॒रस्ता॒ दाहु॑ति॒ माहु॑तिम् पु॒रस्ता᳚ज् जुहुयाज् जुहुयात् पु॒रस्ता॒ दाहु॑ति॒ माहु॑तिम् पु॒रस्ता᳚ज् जुहुयात् । \newline
49. आहु॑ति॒मित्या - हु॒ति॒म् । \newline
50. पु॒रस्ता᳚ज् जुहुयाज् जुहुयात् पु॒रस्ता᳚त् पु॒रस्ता᳚ज् जुहुया॒द् यद् यज् जु॑हुयात् पु॒रस्ता᳚त् पु॒रस्ता᳚ज् जुहुया॒द् यत् । \newline
51. जु॒हु॒या॒द् यद् यज् जु॑हुयाज् जुहुया॒द् यद॒न्या म॒न्यां ॅयज् जु॑हुयाज् जुहुया॒द् यद॒न्याम् । \newline
52. यद॒न्या म॒न्यां ॅयद् यद॒न्या माहु॑ति॒ माहु॑ति म॒न्यां ॅयद् यद॒न्या माहु॑तिम् । \newline
53. अ॒न्या माहु॑ति॒ माहु॑ति म॒न्या म॒न्या माहु॑तिम् पु॒रस्ता᳚त् पु॒रस्ता॒ दाहु॑ति म॒न्या म॒न्या माहु॑तिम् पु॒रस्ता᳚त् । \newline
54. आहु॑तिम् पु॒रस्ता᳚त् पु॒रस्ता॒ दाहु॑ति॒ माहु॑तिम् पु॒रस्ता᳚ज् जुहु॒याज् जु॑हु॒यात् पु॒रस्ता॒ दाहु॑ति॒ माहु॑तिम् पु॒रस्ता᳚ज् जुहु॒यात् । \newline
55. आहु॑ति॒मित्या - हु॒ति॒म् । \newline
56. पु॒रस्ता᳚ज् जुहु॒याज् जु॑हु॒यात् पु॒रस्ता᳚त् पु॒रस्ता᳚ज् जुहु॒या द॒न्य द॒न्यज् जु॑हु॒यात् पु॒रस्ता᳚त् पु॒रस्ता᳚ज् जुहु॒या द॒न्यत् । \newline
57. जु॒हु॒या द॒न्य द॒न्यज् जु॑हु॒याज् जु॑हु॒या द॒न्यन् मुख॒म् मुख॑ म॒न्यज् जु॑हु॒याज् जु॑हु॒या द॒न्यन् मुख᳚म् । \newline
\pagebreak
\markright{ TS 6.2.3.3  \hfill https://www.vedavms.in \hfill}

\section{ TS 6.2.3.3 }

\textbf{TS 6.2.3.3 } \newline
\textbf{Samhita Paata} \newline

द॒न्यन्मुखं॑ कुर्याथ् स्रु॒वेणा॑ऽघा॒रमा घा॑रयति य॒ज्ञ्स्य॒ प्रज्ञा᳚त्यै॒ परा॑ङति॒क्रम्य॑ जुहोति॒ परा॑च ए॒वैभ्यो लो॒केभ्यो॒ यज॑मानो॒ भ्रातृ॑व्या॒न् प्र णु॑दते॒ पुन॑रत्या॒क्रम्यो॑प॒सदं॑ जुहोति प्र॒णुद्यै॒वैभ्यो लो॒केभ्यो॒ भ्रातृ॑व्याञ्जि॒त्वा भ्रा॑तृव्यलो॒क-म॒भ्यारो॑हति दे॒वा वै याः प्रा॒तरु॑प॒सद॑ उ॒पासी॑द॒-न्नह्न॒स्ताभि॒रसु॑रा॒न् प्राणु॑दन्त॒ याः सा॒यꣳ रात्रि॑यै॒ ताभि॒र्यथ् सा॒यं प्रा॑तरुप॒सद॑- [  ] \newline

\textbf{Pada Paata} \newline

अ॒न्यत् । मुख᳚म् । कु॒र्या॒त् । स्रु॒वेण॑ । आ॒घा॒रमित्या᳚ - घा॒रम् । एति॑ । घा॒र॒य॒ति॒ । य॒ज्ञ्स्य॑ । प्रज्ञा᳚त्या॒ इति॒ प्र - ज्ञा॒त्यै॒ । पराङ्॑ । अ॒ति॒क्रम्येत्य॑ति - क्रम्य॑ । जु॒हो॒ति॒ । परा॑चः । ए॒व । ए॒भ्यः । लो॒केभ्यः॑ । यज॑मानः । भ्रातृ॑व्यान् । प्रेति॑ । नु॒द॒ते॒ । पुनः॑ । अ॒त्या॒क्रम्येत्य॑ति - आ॒क्रम्य॑ । उ॒प॒सद॒मित्यु॑प - सद᳚म् । जु॒हो॒ति॒ । प्र॒णुद्येति॑ प्र - नुद्य॑ । ए॒व । ए॒भ्यः । लो॒केभ्यः॑ । भ्रातृ॑व्यान् । जि॒त्वा । भ्रा॒तृ॒व्य॒लो॒कमिति॑ भ्रातृव्य - लो॒कम् । अ॒भ्यारो॑ह॒तीत्य॑भि - आरो॑हति । दे॒वाः । वै । याः । प्रा॒तः । उ॒प॒सद॒ इत्यु॑प - सदः॑ । उ॒पासी॑द॒न्नित्यु॑प - असी॑दन्न् । अह्नः॑ । ताभिः॑ । असु॑रान् । प्रेति॑ । अ॒नु॒द॒न्त॒ । याः । सा॒यम् । रात्रि॑यै । ताभिः॑ । यत् । सा॒यम्प्रा॑त॒रिति॑ सा॒यम् - प्रा॒तः॒ । उ॒प॒सद॒ इत्यु॑प - सदः॑ ।  \newline


\textbf{Krama Paata} \newline

अ॒न्यन् मुख᳚म् । मुख॑म् कुर्यात् । कु॒र्या॒थ् स्रु॒वेण॑ । स्रु॒वेणा॑घा॒रम् । आ॒घा॒रमा । आ॒घा॒रमित्या᳚ - घा॒रम् । 
आ घा॑रयति । घा॒र॒य॒ति॒ य॒ज्ञ्स्य॑ । य॒ज्ञ्स्य॒ प्रज्ञा᳚त्यै । प्रज्ञा᳚त्यै॒ पराङ्‍॑ । प्रज्ञा᳚त्या॒ इति॒ प्र - ज्ञा॒त्यै॒ । परा॑ङति॒क्रम्य॑ । अ॒ति॒क्रम्य॑ जुहोति । अ॒ति॒क्रम्येत्य॑ति - क्रम्य॑ । जु॒हो॒ति॒ परा॑चः । परा॑च ए॒व । ए॒वैभ्यः । ए॒भ्यो लो॒केभ्यः॑ । लो॒केभ्यो॒ यज॑मानः । यज॑मानो॒ भ्रातृ॑व्यान् । भ्रातृ॑व्या॒न् प्र । प्र णु॑दते । नु॒द॒ते॒ पुनः॑ । पुन॑रत्या॒क्रम्य॑ । अ॒त्या॒क्रम्यो॑प॒सद᳚म् । अ॒त्या॒क्रम्येत्य॑ति - आ॒क्रम्य॑ । उ॒प॒सद॑म् जुहोति । उ॒प॒सद॒मित्यु॑प - सद᳚म् । जु॒हो॒ति॒ प्र॒णुद्य॑ । प्र॒णुद्यै॒व । प्र॒णुद्येति॑ प्र - नुद्य॑ । ए॒वैभ्यः । ए॒भ्यो लो॒केभ्यः॑ । लो॒केभ्यो॒ भ्रातृ॑व्यान् । भ्रातृ॑व्यान् जि॒त्वा । जि॒त्वा भ्रा॑तृव्यलो॒कम् । भ्रा॒तृ॒व्य॒लो॒कम॒भ्यारो॑हति । भ्रा॒तृ॒व्य॒लो॒कमिति॑ भ्रातृव्य - लो॒कम् । अ॒भ्यारो॑हति दे॒वाः । अ॒भ्यारो॑ह॒तीत्य॑भि - आरो॑हति । दे॒वा वै । वै याः । याः प्रा॒तः । प्रा॒तरु॑प॒सदः॑ । उ॒प॒सद॑ उ॒पासी॑दन्न् । उ॒प॒सद॒ इत्यु॑प - सदः॑ । उ॒पासी॑द॒न्नह्नः॑ । उ॒पासी॑द॒न्नित्यु॑प - असी॑दन्न् । अह्न॒स्ताभिः॑ । ताभि॒रसु॑रान् । असु॑रा॒न् प्र । प्राणु॑दन्त । अ॒नु॒द॒न्त॒ याः । याः सा॒यम् । सा॒यꣳ रात्रि॑यै । रात्रि॑यै॒ ताभिः॑ । ताभि॒र् यत् । यथ् सा॒यम्प्रा॑तः । सा॒यम्प्रा॑तरुप॒सदः॑ । सा॒यम्प्रा॑त॒रिति॑ सा॒यम् - प्रा॒तः॒ । उ॒प॒सद॑ उपस॒द्यन्ते᳚ । उ॒प॒सद॒ इत्यु॑प - सदः॑ \newline

\textbf{Jatai Paata} \newline

1. अ॒न्यन् मुख॒म् मुख॑ म॒न्य द॒न्यन् मुख᳚म् । \newline
2. मुख॑म् कुर्यात् कुर्या॒न् मुख॒म् मुख॑म् कुर्यात् । \newline
3. कु॒र्या॒थ् स्रु॒वेण॑ स्रु॒वेण॑ कुर्यात् कुर्याथ् स्रु॒वेण॑ । \newline
4. स्रु॒वेणा॑घा॒र मा॑घा॒रꣳ स्रु॒वेण॑ स्रु॒वेणा॑घा॒रम् । \newline
5. आ॒घा॒र मा ऽऽघा॒र मा॑घा॒र मा । \newline
6. आ॒घा॒रमित्या᳚ - घा॒रम् । \newline
7. आ घा॑रयति घारय॒त्या घा॑रयति । \newline
8. घा॒र॒य॒ति॒ य॒ज्ञ्स्य॑ य॒ज्ञ्स्य॑ घारयति घारयति य॒ज्ञ्स्य॑ । \newline
9. य॒ज्ञ्स्य॒ प्रज्ञा᳚त्यै॒ प्रज्ञा᳚त्यै य॒ज्ञ्स्य॑ य॒ज्ञ्स्य॒ प्रज्ञा᳚त्यै । \newline
10. प्रज्ञा᳚त्यै॒ परा॒ङ् परा॒ङ् प्रज्ञा᳚त्यै॒ प्रज्ञा᳚त्यै॒ पराङ्॑ । \newline
11. प्रज्ञा᳚त्या॒ इति॒ प्र - ज्ञा॒त्यै॒ । \newline
12. परा॑ ङति॒क्रम्या॑ ति॒क्रम्य॒ परा॒ङ् परा॑ ङति॒क्रम्य॑ । \newline
13. अ॒ति॒क्रम्य॑ जुहोति जुहो त्यति॒क्रम्या॑ ति॒क्रम्य॑ जुहोति । \newline
14. अ॒ति॒क्रम्येत्य॑ति - क्रम्य॑ । \newline
15. जु॒हो॒ति॒ परा॑चः॒ परा॑चो जुहोति जुहोति॒ परा॑चः । \newline
16. परा॑च ए॒वैव परा॑चः॒ परा॑च ए॒व । \newline
17. ए॒वैभ्य ए॒भ्य ए॒वैवैभ्यः । \newline
18. ए॒भ्यो लो॒केभ्यो॑ लो॒केभ्य॑ ए॒भ्य ए॒भ्यो लो॒केभ्यः॑ । \newline
19. लो॒केभ्यो॒ यज॑मानो॒ यज॑मानो लो॒केभ्यो॑ लो॒केभ्यो॒ यज॑मानः । \newline
20. यज॑मानो॒ भ्रातृ॑व्या॒न् भ्रातृ॑व्या॒न्॒. यज॑मानो॒ यज॑मानो॒ भ्रातृ॑व्यान् । \newline
21. भ्रातृ॑व्या॒न् प्र प्र भ्रातृ॑व्या॒न् भ्रातृ॑व्या॒न् प्र । \newline
22. प्र णु॑दते नुदते॒ प्र प्र णु॑दते । \newline
23. नु॒द॒ते॒ पुनः॒ पुन॑र् नुदते नुदते॒ पुनः॑ । \newline
24. पुन॑ रत्या॒क्रम्या᳚ त्या॒क्रम्य॒ पुनः॒ पुन॑ रत्या॒क्रम्य॑ । \newline
25. अ॒त्या॒क्र म्यो॑प॒सद॑ मुप॒सद॑ मत्या॒क्र म्या᳚त्या॒क्र म्यो॑प॒सद᳚म् । \newline
26. अ॒त्या॒क्रम्येत्य॑ति - आ॒क्रम्य॑ । \newline
27. उ॒प॒सद॑म् जुहोति जुहो त्युप॒सद॑ मुप॒सद॑म् जुहोति । \newline
28. उ॒प॒सद॒मित्यु॑प - सद᳚म् । \newline
29. जु॒हो॒ति॒ प्र॒णुद्य॑ प्र॒णुद्य॑ जुहोति जुहोति प्र॒णुद्य॑ । \newline
30. प्र॒णुद्यै॒वैव प्र॒णुद्य॑ प्र॒णुद्यै॒व । \newline
31. प्र॒णुद्येति॑ प्र - नुद्य॑ । \newline
32. ए॒वैभ्य ए॒भ्य ए॒वैवैभ्यः । \newline
33. ए॒भ्यो लो॒केभ्यो॑ लो॒केभ्य॑ ए॒भ्य ए॒भ्यो लो॒केभ्यः॑ । \newline
34. लो॒केभ्यो॒ भ्रातृ॑व्या॒न् भ्रातृ॑व्यान् ॅलो॒केभ्यो॑ लो॒केभ्यो॒ भ्रातृ॑व्यान् । \newline
35. भ्रातृ॑व्यान् जि॒त्वा जि॒त्वा भ्रातृ॑व्या॒न् भ्रातृ॑व्यान् जि॒त्वा । \newline
36. जि॒त्वा भ्रा॑तृव्यलो॒कम् भ्रा॑तृव्यलो॒कम् जि॒त्वा जि॒त्वा भ्रा॑तृव्यलो॒कम् । \newline
37. भ्रा॒तृ॒व्य॒लो॒क म॒भ्यारो॑ह त्य॒भ्यारो॑हति भ्रातृव्यलो॒कम् भ्रा॑तृव्यलो॒क म॒भ्यारो॑हति । \newline
38. भ्रा॒तृ॒व्य॒लो॒कमिति॑ भ्रातृव्य - लो॒कम् । \newline
39. अ॒भ्यारो॑हति दे॒वा दे॒वा अ॒भ्यारो॑ह त्य॒भ्यारो॑हति दे॒वाः । \newline
40. अ॒भ्यारो॑ह॒तीत्य॑भि - आरो॑हति । \newline
41. दे॒वा वै वै दे॒वा दे॒वा वै । \newline
42. वै या या वै वै याः । \newline
43. याः प्रा॒तः प्रा॒तर् या याः प्रा॒तः । \newline
44. प्रा॒त रु॑प॒सद॑ उप॒सदः॑ प्रा॒तः प्रा॒त रु॑प॒सदः॑ । \newline
45. उ॒प॒सद॑ उ॒पासी॑दन् नु॒पासी॑दन् नुप॒सद॑ उप॒सद॑ उ॒पासी॑दन्न् । \newline
46. उ॒प॒सद॒ इत्यु॑प - सदः॑ । \newline
47. उ॒पासी॑द॒न् नह्नो ऽह्न॑ उ॒पासी॑दन् नु॒पासी॑द॒न् नह्नः॑ । \newline
48. उ॒पासी॑द॒न्नित्यु॑प - असी॑दन्न् । \newline
49. अह्न॒ स्ताभि॒ स्ताभि॒ रह्नो ऽह्न॒ स्ताभिः॑ । \newline
50. ताभि॒ रसु॑रा॒ नसु॑रा॒न् ताभि॒ स्ताभि॒ रसु॑रान् । \newline
51. असु॑रा॒न् प्र प्रासु॑रा॒ नसु॑रा॒न् प्र । \newline
52. प्राणु॑दन्ता नुदन्त॒ प्र प्राणु॑दन्त । \newline
53. अ॒नु॒द॒न्त॒ या या अ॑नुदन्ता नुदन्त॒ याः । \newline
54. याः सा॒यꣳ सा॒यं ॅया याः सा॒यम् । \newline
55. सा॒यꣳ रात्रि॑यै॒ रात्रि॑यै सा॒यꣳ सा॒यꣳ रात्रि॑यै । \newline
56. रात्रि॑यै॒ ताभि॒ स्ताभी॒ रात्रि॑यै॒ रात्रि॑यै॒ ताभिः॑ । \newline
57. ताभि॒र् यद् यत् ताभि॒ स्ताभि॒र् यत् । \newline
58. यथ् सा॒यम्प्रा॑तः सा॒यम्प्रा॑त॒र् यद् यथ् सा॒यम्प्रा॑तः । \newline
59. सा॒यम्प्रा॑त रुप॒सद॑ उप॒सदः॑ सा॒यम्प्रा॑तः सा॒यम्प्रा॑त रुप॒सदः॑ । \newline
60. सा॒यम्प्रा॑त॒रिति॑ सा॒यम् - प्रा॒तः॒ । \newline
61. उ॒प॒सद॑ उपस॒द्यन्त॑ उपस॒द्यन्त॑ उप॒सद॑ उप॒सद॑ उपस॒द्यन्ते᳚ । \newline
62. उ॒प॒सद॒ इत्यु॑प - सदः॑ । \newline

\textbf{Ghana Paata } \newline

1. अ॒न्यन् मुख॒म् मुख॑ म॒न्य द॒न्यन् मुख॑म् कुर्यात् कुर्या॒न् मुख॑ म॒न्य द॒न्यन् मुख॑म् कुर्यात् । \newline
2. मुख॑म् कुर्यात् कुर्या॒न् मुख॒म् मुख॑म् कुर्याथ् स्रु॒वेण॑ स्रु॒वेण॑ कुर्या॒न् मुख॒म् मुख॑म् कुर्याथ् स्रु॒वेण॑ । \newline
3. कु॒र्या॒थ् स्रु॒वेण॑ स्रु॒वेण॑ कुर्यात् कुर्याथ् स्रु॒वेणा॑ घा॒र मा॑घा॒रꣳ स्रु॒वेण॑ कुर्यात् कुर्याथ् स्रु॒वेणा॑ घा॒रम् । \newline
4. स्रु॒वेणा॑ घा॒र मा॑घा॒रꣳ स्रु॒वेण॑ स्रु॒वेणा॑ घा॒र मा ऽऽघा॒रꣳ स्रु॒वेण॑ स्रु॒वेणा॑ घा॒र मा । \newline
5. आ॒घा॒र मा ऽऽघा॒र मा॑घा॒र मा घा॑रयति घारय॒त्या ऽऽघा॒र मा॑घा॒र मा घा॑रयति । \newline
6. आ॒घा॒रमित्या᳚ - घा॒रम् । \newline
7. आ घा॑रयति घारय॒त्या घा॑रयति य॒ज्ञ्स्य॑ य॒ज्ञ्स्य॑ घारय॒त्या घा॑रयति य॒ज्ञ्स्य॑ । \newline
8. घा॒र॒य॒ति॒ य॒ज्ञ्स्य॑ य॒ज्ञ्स्य॑ घारयति घारयति य॒ज्ञ्स्य॒ प्रज्ञा᳚त्यै॒ प्रज्ञा᳚त्यै य॒ज्ञ्स्य॑ घारयति घारयति य॒ज्ञ्स्य॒ प्रज्ञा᳚त्यै । \newline
9. य॒ज्ञ्स्य॒ प्रज्ञा᳚त्यै॒ प्रज्ञा᳚त्यै य॒ज्ञ्स्य॑ य॒ज्ञ्स्य॒ प्रज्ञा᳚त्यै॒ परा॒ङ् परा॒ङ् प्रज्ञा᳚त्यै य॒ज्ञ्स्य॑ य॒ज्ञ्स्य॒ प्रज्ञा᳚त्यै॒ पराङ्॑ । \newline
10. प्रज्ञा᳚त्यै॒ परा॒ङ् परा॒ङ् प्रज्ञा᳚त्यै॒ प्रज्ञा᳚त्यै॒ परा॑ ङति॒क्रम्या॑ ति॒क्रम्य॒ परा॒ङ् प्रज्ञा᳚त्यै॒ प्रज्ञा᳚त्यै॒ परा॑ ङति॒क्रम्य॑ । \newline
11. प्रज्ञा᳚त्या॒ इति॒ प्र - ज्ञा॒त्यै॒ । \newline
12. परा॑ ङति॒क्रम्या॑ ति॒क्रम्य॒ परा॒ङ् परा॑ ङति॒क्रम्य॑ जुहोति जुहो त्यति॒क्रम्य॒ परा॒ङ् परा॑ ङति॒क्रम्य॑ जुहोति । \newline
13. अ॒ति॒क्रम्य॑ जुहोति जुहो त्यति॒क्रम्या॑ ति॒क्रम्य॑ जुहोति॒ परा॑चः॒ परा॑चो जुहो त्यति॒क्रम्या॑ ति॒क्रम्य॑ जुहोति॒ परा॑चः । \newline
14. अ॒ति॒क्रम्येत्य॑ति - क्रम्य॑ । \newline
15. जु॒हो॒ति॒ परा॑चः॒ परा॑चो जुहोति जुहोति॒ परा॑च ए॒वैव परा॑चो जुहोति जुहोति॒ परा॑च ए॒व । \newline
16. परा॑च ए॒वैव परा॑चः॒ परा॑च ए॒वैभ्य ए॒भ्य ए॒व परा॑चः॒ परा॑च ए॒वैभ्यः । \newline
17. ए॒वैभ्य ए॒भ्य ए॒वै वैभ्यो लो॒केभ्यो॑ लो॒केभ्य॑ ए॒भ्य ए॒वै वैभ्यो लो॒केभ्यः॑ । \newline
18. ए॒भ्यो लो॒केभ्यो॑ लो॒केभ्य॑ ए॒भ्य ए॒भ्यो लो॒केभ्यो॒ यज॑मानो॒ यज॑मानो लो॒केभ्य॑ ए॒भ्य ए॒भ्यो लो॒केभ्यो॒ यज॑मानः । \newline
19. लो॒केभ्यो॒ यज॑मानो॒ यज॑मानो लो॒केभ्यो॑ लो॒केभ्यो॒ यज॑मानो॒ भ्रातृ॑व्या॒न् भ्रातृ॑व्या॒न्॒. यज॑मानो लो॒केभ्यो॑ लो॒केभ्यो॒ यज॑मानो॒ भ्रातृ॑व्यान् । \newline
20. यज॑मानो॒ भ्रातृ॑व्या॒न् भ्रातृ॑व्या॒न्॒. यज॑मानो॒ यज॑मानो॒ भ्रातृ॑व्या॒न् प्र प्र भ्रातृ॑व्या॒न्॒. यज॑मानो॒ यज॑मानो॒ भ्रातृ॑व्या॒न् प्र । \newline
21. भ्रातृ॑व्या॒न् प्र प्र भ्रातृ॑व्या॒न् भ्रातृ॑व्या॒न् प्र णु॑दते नुदते॒ प्र भ्रातृ॑व्या॒न् भ्रातृ॑व्या॒न् प्र णु॑दते । \newline
22. प्र णु॑दते नुदते॒ प्र प्र णु॑दते॒ पुनः॒ पुन॑र् नुदते॒ प्र प्र णु॑दते॒ पुनः॑ । \newline
23. नु॒द॒ते॒ पुनः॒ पुन॑र् नुदते नुदते॒ पुन॑ रत्या॒क्रम्या᳚ त्या॒क्रम्य॒ पुन॑र् नुदते नुदते॒ पुन॑ रत्या॒क्रम्य॑ । \newline
24. पुन॑ रत्या॒क्रम्या᳚ त्या॒क्रम्य॒ पुनः॒ पुन॑ रत्या॒क्र म्यो॑प॒सद॑ मुप॒सद॑ मत्या॒क्रम्य॒ पुनः॒ पुन॑ रत्या॒क्र म्यो॑प॒सद᳚म् । \newline
25. अ॒त्या॒क्र म्यो॑प॒सद॑ मुप॒सद॑ मत्या॒क्रम्या᳚ त्या॒क्रम्यो॑ प॒सद॑म् जुहोति जुहो त्युप॒सद॑ मत्या॒क्रम्या᳚ त्या॒क्र म्यो॑प॒सद॑म् जुहोति । \newline
26. अ॒त्या॒क्रम्येत्य॑ति - आ॒क्रम्य॑ । \newline
27. उ॒प॒सद॑म् जुहोति जुहो त्युप॒सद॑ मुप॒सद॑म् जुहोति प्र॒णुद्य॑ प्र॒णुद्य॑ जुहो त्युप॒सद॑ मुप॒सद॑म् जुहोति प्र॒णुद्य॑ । \newline
28. उ॒प॒सद॒मित्यु॑प - सद᳚म् । \newline
29. जु॒हो॒ति॒ प्र॒णुद्य॑ प्र॒णुद्य॑ जुहोति जुहोति प्र॒णुद्यै॒वैव प्र॒णुद्य॑ जुहोति जुहोति प्र॒णुद्यै॒व । \newline
30. प्र॒णुद्यै॒ वैव प्र॒णुद्य॑ प्र॒णुद्यै॒ वैभ्य ए॒भ्य ए॒व प्र॒णुद्य॑ प्र॒णुद्यै॒ वैभ्यः । \newline
31. प्र॒णुद्येति॑ प्र - नुद्य॑ । \newline
32. ए॒वैभ्य ए॒भ्य ए॒वै वैभ्यो लो॒केभ्यो॑ लो॒केभ्य॑ ए॒भ्य ए॒वै वैभ्यो लो॒केभ्यः॑ । \newline
33. ए॒भ्यो लो॒केभ्यो॑ लो॒केभ्य॑ ए॒भ्य ए॒भ्यो लो॒केभ्यो॒ भ्रातृ॑व्या॒न् भ्रातृ॑व्यान् ॅलो॒केभ्य॑ ए॒भ्य ए॒भ्यो लो॒केभ्यो॒ भ्रातृ॑व्यान् । \newline
34. लो॒केभ्यो॒ भ्रातृ॑व्या॒न् भ्रातृ॑व्यान् ॅलो॒केभ्यो॑ लो॒केभ्यो॒ भ्रातृ॑व्यान् जि॒त्वा जि॒त्वा भ्रातृ॑व्यान् ॅलो॒केभ्यो॑ लो॒केभ्यो॒ भ्रातृ॑व्यान् जि॒त्वा । \newline
35. भ्रातृ॑व्यान् जि॒त्वा जि॒त्वा भ्रातृ॑व्या॒न् भ्रातृ॑व्यान् जि॒त्वा भ्रा॑तृव्यलो॒कम् भ्रा॑तृव्यलो॒कम् जि॒त्वा भ्रातृ॑व्या॒न् भ्रातृ॑व्यान् जि॒त्वा भ्रा॑तृव्यलो॒कम् । \newline
36. जि॒त्वा भ्रा॑तृव्यलो॒कम् भ्रा॑तृव्यलो॒कम् जि॒त्वा जि॒त्वा भ्रा॑तृव्यलो॒क म॒भ्यारो॑ह त्य॒भ्यारो॑हति भ्रातृव्यलो॒कम् जि॒त्वा जि॒त्वा भ्रा॑तृव्यलो॒क म॒भ्यारो॑हति । \newline
37. भ्रा॒तृ॒व्य॒लो॒क म॒भ्यारो॑ह त्य॒भ्यारो॑हति भ्रातृव्यलो॒कम् भ्रा॑तृव्यलो॒क म॒भ्यारो॑हति दे॒वा दे॒वा अ॒भ्यारो॑हति भ्रातृव्यलो॒कम् भ्रा॑तृव्यलो॒क म॒भ्यारो॑हति दे॒वाः । \newline
38. भ्रा॒तृ॒व्य॒लो॒कमिति॑ भ्रातृव्य - लो॒कम् । \newline
39. अ॒भ्यारो॑हति दे॒वा दे॒वा अ॒भ्यारो॑ह त्य॒भ्यारो॑हति दे॒वा वै वै दे॒वा अ॒भ्यारो॑ह त्य॒भ्यारो॑हति दे॒वा वै । \newline
40. अ॒भ्यारो॑ह॒तीत्य॑भि - आरो॑हति । \newline
41. दे॒वा वै वै दे॒वा दे॒वा वै या या वै दे॒वा दे॒वा वै याः । \newline
42. वै या या वै वै याः प्रा॒तः प्रा॒तर् या वै वै याः प्रा॒तः । \newline
43. याः प्रा॒तः प्रा॒तर् या याः प्रा॒त रु॑प॒सद॑ उप॒सदः॑ प्रा॒तर् या याः प्रा॒त रु॑प॒सदः॑ । \newline
44. प्रा॒त रु॑प॒सद॑ उप॒सदः॑ प्रा॒तः प्रा॒त रु॑प॒सद॑ उ॒पासी॑दन् नु॒पासी॑दन् नुप॒सदः॑ प्रा॒तः प्रा॒त रु॑प॒सद॑ उ॒पासी॑दन्न् । \newline
45. उ॒प॒सद॑ उ॒पासी॑दन् नु॒पासी॑दन् नुप॒सद॑ उप॒सद॑ उ॒पासी॑द॒न् नह्नो ऽह्न॑ उ॒पासी॑दन् नुप॒सद॑ उप॒सद॑ उ॒पासी॑द॒न् नह्नः॑ । \newline
46. उ॒प॒सद॒ इत्यु॑प - सदः॑ । \newline
47. उ॒पासी॑द॒न् नह्नो ऽह्न॑ उ॒पासी॑दन् नु॒पासी॑द॒न् नह्न॒ स्ताभि॒ स्ताभि॒ रह्न॑ उ॒पासी॑दन् नु॒पासी॑द॒न् नह्न॒ स्ताभिः॑ । \newline
48. उ॒पासी॑द॒न्नित्यु॑प - असी॑दन्न् । \newline
49. अह्न॒ स्ताभि॒ स्ताभि॒ रह्नो ऽह्न॒ स्ताभि॒ रसु॑रा॒ नसु॑रा॒न् ताभि॒ रह्नो ऽह्न॒ स्ताभि॒ रसु॑रान् । \newline
50. ताभि॒ रसु॑रा॒ नसु॑रा॒न् ताभि॒ स्ताभि॒ रसु॑रा॒न् प्र प्रासु॑रा॒न् ताभि॒ स्ताभि॒ रसु॑रा॒न् प्र । \newline
51. असु॑रा॒न् प्र प्रासु॑रा॒ नसु॑रा॒न् प्राणु॑दन्ता नुदन्त॒ प्रासु॑रा॒ नसु॑रा॒न् प्राणु॑दन्त । \newline
52. प्राणु॑दन्ता नुदन्त॒ प्र प्राणु॑दन्त॒ या या अ॑नुदन्त॒ प्र प्राणु॑दन्त॒ याः । \newline
53. अ॒नु॒द॒न्त॒ या या अ॑नुदन्ता नुदन्त॒ याः सा॒यꣳ सा॒यं ॅया अ॑नुदन्ता नुदन्त॒ याः सा॒यम् । \newline
54. याः सा॒यꣳ सा॒यं ॅया याः सा॒यꣳ रात्रि॑यै॒ रात्रि॑यै सा॒यं ॅया याः सा॒यꣳ रात्रि॑यै । \newline
55. सा॒यꣳ रात्रि॑यै॒ रात्रि॑यै सा॒यꣳ सा॒यꣳ रात्रि॑यै॒ ताभि॒ स्ताभी॒ रात्रि॑यै सा॒यꣳ सा॒यꣳ रात्रि॑यै॒ ताभिः॑ । \newline
56. रात्रि॑यै॒ ताभि॒ स्ताभी॒ रात्रि॑यै॒ रात्रि॑यै॒ ताभि॒र् यद् यत् ताभी॒ रात्रि॑यै॒ रात्रि॑यै॒ ताभि॒र् यत् । \newline
57. ताभि॒र् यद् यत् ताभि॒ स्ताभि॒र् यथ् सा॒यम्प्रा॑तः सा॒यम्प्रा॑त॒र् यत् ताभि॒ स्ताभि॒र् यथ् सा॒यम्प्रा॑तः । \newline
58. यथ् सा॒यम्प्रा॑तः सा॒यम्प्रा॑त॒र् यद् यथ् सा॒यम्प्रा॑त रुप॒सद॑ उप॒सदः॑ सा॒यम्प्रा॑त॒र् यद् यथ् सा॒यम्प्रा॑त रुप॒सदः॑ । \newline
59. सा॒यम्प्रा॑त रुप॒सद॑ उप॒सदः॑ सा॒यम्प्रा॑तः सा॒यम्प्रा॑त रुप॒सद॑ उपस॒द्यन्त॑ उपस॒द्यन्त॑ उप॒सदः॑ सा॒यम्प्रा॑तः सा॒यम्प्रा॑त रुप॒सद॑ उपस॒द्यन्ते᳚ । \newline
60. सा॒यम्प्रा॑त॒रिति॑ सा॒यम् - प्रा॒तः॒ । \newline
61. उ॒प॒सद॑ उपस॒द्यन्त॑ उपस॒द्यन्त॑ उप॒सद॑ उप॒सद॑ उपस॒द्यन्ते॑ ऽहोरा॒त्राभ्या॑ महोरा॒त्राभ्या॑ मुपस॒द्यन्त॑ उप॒सद॑ उप॒सद॑ उपस॒द्यन्ते॑ ऽहोरा॒त्राभ्या᳚म् । \newline
62. उ॒प॒सद॒ इत्यु॑प - सदः॑ । \newline
\pagebreak
\markright{ TS 6.2.3.4  \hfill https://www.vedavms.in \hfill}

\section{ TS 6.2.3.4 }

\textbf{TS 6.2.3.4 } \newline
\textbf{Samhita Paata} \newline

उपस॒द्यन्ते॑ ऽहोरा॒त्राभ्या॑मे॒व तद्-यज॑मानो॒ भ्रातृ॑व्या॒न् प्र णु॑दते॒ याः प्रा॒तर्या॒ज्याः᳚ स्युस्ताः सा॒यं पु॑रोऽनुवा॒क्याः᳚ कुर्या॒दया॑तयामत्वाय ति॒स्र उ॑प॒सद॒ उपै॑ति॒ त्रय॑ इ॒मे लो॒का इ॒माने॒व लो॒कान् प्री॑णाति॒ षट्थ् सं प॑द्यन्ते॒ षड्वा ऋ॒तव॑ ऋ॒तूने॒व प्री॑णाति॒ द्वाद॑शा॒हीने॒ सोम॒ उपै॑ति॒ द्वाद॑श॒ मासाः᳚ संॅवथ्स॒रः स॑वंथ्स॒रमे॒व प्री॑णाति॒ चतु॑र्विꣳशतिः॒ सं- [  ] \newline

\textbf{Pada Paata} \newline

उ॒प॒स॒द्यन्त॒ इत्यु॑प - स॒द्यन्ते᳚ । अ॒हो॒रा॒त्राभ्या॒मित्य॑हः - रा॒त्राभ्या᳚म् । ए॒व । तत् । यज॑मानः । भ्रातृ॑व्यान् । प्रेति॑ । नु॒द॒ते॒ । याः । प्रा॒तः । या॒ज्याः᳚ । स्युः । ताः । सा॒यम् । पु॒रो॒नु॒वा॒क्या॑ इति॑ पुरः - अ॒नु॒वा॒क्याः᳚ । कु॒र्या॒त् । अया॑तयामत्वा॒येत्यया॑तयाम - त्वा॒य॒ । ति॒स्रः । उ॒प॒सद॒ इत्यु॑प - सदः॑ । उपेति॑ । ए॒ति॒ । त्रयः॑ । इ॒मे । लो॒काः । इ॒मान् । ए॒व । लो॒कान् । प्री॒णा॒ति॒ । षट् । समिति॑ । प॒द्य॒न्ते॒ । षट् । वै । ऋ॒तवः॑ । ऋ॒तून् । ए॒व । प्री॒णा॒ति॒ । द्वाद॑श । अ॒हीने᳚ । सोमे᳚ । उपेति॑ । ए॒ति॒ । द्वाद॑श । मासाः᳚ । सं॒ॅव॒थ्स॒र इति॑ सं - व॒थ्स॒रः । सं॒ॅव॒थ्स॒रमिति॑ सं - व॒थ्स॒रम् । ए॒व । प्री॒णा॒ति॒ । चतु॑र्विꣳशति॒रिति॒ चतुः॑ - विꣳ॒॒श॒तिः॒ । समिति॑ ।  \newline


\textbf{Krama Paata} \newline

उ॒प॒स॒द्यन्ते॑ऽहोरा॒त्राभ्या᳚म् । उ॒प॒स॒द्यन्त॒ इत्यु॑प - स॒द्यन्ते᳚ । अ॒हो॒रा॒त्राभ्या॑मे॒व । अ॒हो॒रा॒त्राभ्या॒मित्य॑हः - रा॒त्राभ्या᳚म् । ए॒व तत् । तद् यज॑मानः । यज॑मानो॒ भ्रातृ॑व्यान् । भ्रातृ॑व्या॒न् प्र । प्र णु॑दते । नु॒द॒ते॒ याः । याः प्रा॒तः । प्रा॒तर् या॒ज्याः᳚ । या॒ज्याः᳚ स्युः । स्युस्ताः । ताः सा॒यम् । सा॒यम् पु॑रोनुवा॒क्याः᳚ । पु॒रो॒नु॒वा॒क्याः᳚ कुर्यात् । पु॒रो॒नु॒वा॒क्या॑ इति॑ पुरः - अ॒नु॒वा॒क्याः᳚ । कु॒र्या॒दया॑तयामत्वाय । अया॑तयामत्वाय ति॒स्रः । अया॑तयामत्वा॒येत्यया॑तयाम - त्वा॒य॒ । ति॒स्र उ॑प॒सदः॑ । उ॒प॒सद॒ उप॑ । उ॒प॒सद॒ इत्यु॑प - सदः॑ । उपै॑ति । ए॒ति॒ त्रयः॑ । त्रय॑ इ॒मे । इ॒मे लो॒काः । लो॒का इ॒मान् । इ॒माने॒व । ए॒व लो॒कान् । लो॒कान् प्री॑णाति । प्री॒णा॒ति॒ षट् । षट्थ् सम् । सम् प॑द्यन्ते । प॒द्य॒न्ते॒ षट् । षड् वै । वा ऋ॒तवः॑ । ऋ॒तव॑ ऋ॒तून् । ऋ॒तूने॒व । ए॒व प्री॑णाति । प्री॒णा॒ति॒ द्वाद॑श । द्वाद॑शा॒हीने᳚ । अ॒हीने॒ सोमे᳚ । सोम॒ उप॑ । उपै॑ति । ए॒ति॒ द्वाद॑श । द्वाद॑श॒ मासाः᳚ । मासाः᳚ सम्ॅवथ्स॒रः । स॒म्ॅव॒थ्स॒रः स॑म्ॅवथ्स॒रम् । स॒म्ॅव॒थ्स॒र इति॑ सम् - व॒थ्स॒रः । स॒म्ॅव॒थ्स॒रमे॒व । स॒म्ॅव॒थ्स॒रमिति॑ सम् - व॒थ्स॒रम् । ए॒व प्री॑णाति । प्री॒णा॒ति॒ चतु॑र्विꣳशतिः । चतु॑र्विꣳशतिः॒ सम् । चतु॑र्विꣳशति॒रिति॒ चतुः॑ - विꣳ॒॒श॒तिः॒ । सम्प॑द्यन्ते \newline

\textbf{Jatai Paata} \newline

1. उ॒प॒स॒द्यन्ते॑ ऽहोरा॒त्राभ्या॑ महोरा॒त्राभ्या॑ मुपस॒द्यन्त॑ उपस॒द्यन्ते॑ ऽहोरा॒त्राभ्या᳚म् । \newline
2. उ॒प॒स॒द्यन्त॒ इत्यु॑प - स॒द्यन्ते᳚ । \newline
3. अ॒हो॒रा॒त्राभ्या॑ मे॒वैवाहो॑रा॒त्राभ्या॑ महोरा॒त्राभ्या॑ मे॒व । \newline
4. अ॒हो॒रा॒त्राभ्या॒मित्य॑हः - रा॒त्राभ्या᳚म् । \newline
5. ए॒व तत् तदे॒वैव तत् । \newline
6. तद् यज॑मानो॒ यज॑मान॒ स्तत् तद् यज॑मानः । \newline
7. यज॑मानो॒ भ्रातृ॑व्या॒न् भ्रातृ॑व्या॒न्॒. यज॑मानो॒ यज॑मानो॒ भ्रातृ॑व्यान् । \newline
8. भ्रातृ॑व्या॒न् प्र प्र भ्रातृ॑व्या॒न् भ्रातृ॑व्या॒न् प्र । \newline
9. प्र णु॑दते नुदते॒ प्र प्र णु॑दते । \newline
10. नु॒द॒ते॒ या या नु॑दते नुदते॒ याः । \newline
11. याः प्रा॒तः प्रा॒तर् या याः प्रा॒तः । \newline
12. प्रा॒तर् या॒ज्या॑ या॒ज्याः᳚ प्रा॒तः प्रा॒तर् या॒ज्याः᳚ । \newline
13. या॒ज्याः᳚ स्युः स्युर् या॒ज्या॑ या॒ज्याः᳚ स्युः । \newline
14. स्यु स्ता स्ताः स्युः स्युस्ताः । \newline
15. ताः सा॒यꣳ सा॒यम् ता स्ताः सा॒यम् । \newline
16. सा॒यम् पु॑रोनुवा॒क्याः᳚ पुरोनुवा॒क्याः᳚ सा॒यꣳ सा॒यम् पु॑रोनुवा॒क्याः᳚ । \newline
17. पु॒रो॒नु॒वा॒क्याः᳚ कुर्यात् कुर्यात् पुरोनुवा॒क्याः᳚ पुरोनुवा॒क्याः᳚ कुर्यात् । \newline
18. पु॒रो॒नु॒वा॒क्या॑ इति॑ पुरः - अ॒नु॒वा॒क्याः᳚ । \newline
19. कु॒र्या॒ दया॑तयामत्वा॒या या॑तयामत्वाय कुर्यात् कुर्या॒ दया॑तयामत्वाय । \newline
20. अया॑तयामत्वाय ति॒स्र स्ति॒स्रो ऽया॑तयामत्वा॒या या॑तयामत्वाय ति॒स्रः । \newline
21. अया॑तयामत्वा॒येत्यया॑तयाम - त्वा॒य॒ । \newline
22. ति॒स्र उ॑प॒सद॑ उप॒सद॑ स्ति॒स्र स्ति॒स्र उ॑प॒सदः॑ । \newline
23. उ॒प॒सद॒ उपोपो॑प॒सद॑ उप॒सद॒ उप॑ । \newline
24. उ॒प॒सद॒ इत्यु॑प - सदः॑ । \newline
25. उपै᳚ त्ये॒ त्युपोपै॑ति । \newline
26. ए॒ति॒ त्रय॒ स्त्रय॑ एत्येति॒ त्रयः॑ । \newline
27. त्रय॑ इ॒म इ॒मे त्रय॒ स्त्रय॑ इ॒मे । \newline
28. इ॒मे लो॒का लो॒का इ॒म इ॒मे लो॒काः । \newline
29. लो॒का इ॒मा नि॒मान् ॅलो॒का लो॒का इ॒मान् । \newline
30. इ॒मा ने॒वैवे मा नि॒मा ने॒व । \newline
31. ए॒व लो॒कान् ॅलो॒का ने॒वैव लो॒कान् । \newline
32. लो॒कान् प्री॑णाति प्रीणाति लो॒कान् ॅलो॒कान् प्री॑णाति । \newline
33. प्री॒णा॒ति॒ षट् थ्षट् प्री॑णाति प्रीणाति॒ षट् । \newline
34. षट् थ्सꣳ सꣳ षट् थ्षट् थ्सम् । \newline
35. सम् प॑द्यन्ते पद्यन्ते॒ सꣳ सम् प॑द्यन्ते । \newline
36. प॒द्य॒न्ते॒ षट् थ्षट् प॑द्यन्ते पद्यन्ते॒ षट् । \newline
37. षड् वै वै षट् थ्षड् वै । \newline
38. वा ऋ॒तव॑ ऋ॒तवो॒ वै वा ऋ॒तवः॑ । \newline
39. ऋ॒तव॑ ऋ॒तू नृ॒तू नृ॒तव॑ ऋ॒तव॑ ऋ॒तून् । \newline
40. ऋ॒तू ने॒वैव र्‌तू नृ॒तू ने॒व । \newline
41. ए॒व प्री॑णाति प्रीणा त्ये॒वैव प्री॑णाति । \newline
42. प्री॒णा॒ति॒ द्वाद॑श॒ द्वाद॑श प्रीणाति प्रीणाति॒ द्वाद॑श । \newline
43. द्वाद॑शा॒हीने॒ ऽहीने॒ द्वाद॑श॒ द्वाद॑शा॒हीने᳚ । \newline
44. अ॒हीने॒ सोमे॒ सोमे॒ ऽहीने॒ ऽहीने॒ सोमे᳚ । \newline
45. सोम॒ उपोप॒ सोमे॒ सोम॒ उप॑ । \newline
46. उपै᳚त्ये॒ त्युपोपै॑ति । \newline
47. ए॒ति॒ द्वाद॑श॒ द्वाद॑ शैत्येति॒ द्वाद॑श । \newline
48. द्वाद॑श॒ मासा॒ मासा॒ द्वाद॑श॒ द्वाद॑श॒ मासाः᳚ । \newline
49. मासाः᳚ संॅवथ्स॒रः सं॑ॅवथ्स॒रो मासा॒ मासाः᳚ संॅवथ्स॒रः । \newline
50. सं॒ॅव॒थ्स॒रः सं॑ॅवथ्स॒रꣳ सं॑ॅवथ्स॒रꣳ सं॑ॅवथ्स॒रः सं॑ॅवथ्स॒रः सं॑ॅवथ्स॒रम् । \newline
51. सं॒ॅव॒थ्स॒र इति॑ सं - व॒थ्स॒रः । \newline
52. सं॒ॅव॒थ्स॒र मे॒वैव सं॑ॅवथ्स॒रꣳ सं॑ॅवथ्स॒र मे॒व । \newline
53. सं॒ॅव॒थ्स॒रमिति॑ सं - व॒थ्स॒रम् । \newline
54. ए॒व प्री॑णाति प्रीणा त्ये॒वैव प्री॑णाति । \newline
55. प्री॒णा॒ति॒ चतु॑र्विꣳशति॒ श्चतु॑र्विꣳशतिः प्रीणाति प्रीणाति॒ चतु॑र्विꣳशतिः । \newline
56. चतु॑र्विꣳशतिः॒ सꣳ सम् चतु॑र्विꣳशति॒ श्चतु॑र्विꣳशतिः॒ सम् । \newline
57. चतु॑र्विꣳशति॒रिति॒ चतुः॑ - विꣳ॒॒श॒तिः॒ । \newline
58. सम् प॑द्यन्ते पद्यन्ते॒ सꣳ सम् प॑द्यन्ते । \newline

\textbf{Ghana Paata } \newline

1. उ॒प॒स॒द्यन्ते॑ ऽहोरा॒त्राभ्या॑ महोरा॒त्राभ्या॑ मुपस॒द्यन्त॑ उपस॒द्यन्ते॑ ऽहोरा॒त्राभ्या॑ मे॒वै वाहो॑रा॒त्राभ्या॑ मुपस॒द्यन्त॑ उपस॒द्यन्ते॑ ऽहोरा॒त्राभ्या॑ मे॒व । \newline
2. उ॒प॒स॒द्यन्त॒ इत्यु॑प - स॒द्यन्ते᳚ । \newline
3. अ॒हो॒रा॒त्राभ्या॑ मे॒वै वाहो॑रा॒त्राभ्या॑ महोरा॒त्राभ्या॑ मे॒व तत् तदे॒ वाहो॑रा॒त्राभ्या॑ महोरा॒त्राभ्या॑ मे॒व तत् । \newline
4. अ॒हो॒रा॒त्राभ्या॒मित्य॑हः - रा॒त्राभ्या᳚म् । \newline
5. ए॒व तत् तदे॒ वैव तद् यज॑मानो॒ यज॑मान॒ स्तदे॒ वैव तद् यज॑मानः । \newline
6. तद् यज॑मानो॒ यज॑मान॒ स्तत् तद् यज॑मानो॒ भ्रातृ॑व्या॒न् भ्रातृ॑व्या॒न्॒. यज॑मान॒ स्तत् तद् यज॑मानो॒ भ्रातृ॑व्यान् । \newline
7. यज॑मानो॒ भ्रातृ॑व्या॒न् भ्रातृ॑व्या॒न्॒. यज॑मानो॒ यज॑मानो॒ भ्रातृ॑व्या॒न् प्र प्र भ्रातृ॑व्या॒न्॒. यज॑मानो॒ यज॑मानो॒ भ्रातृ॑व्या॒न् प्र । \newline
8. भ्रातृ॑व्या॒न् प्र प्र भ्रातृ॑व्या॒न् भ्रातृ॑व्या॒न् प्र णु॑दते नुदते॒ प्र भ्रातृ॑व्या॒न् भ्रातृ॑व्या॒न् प्र णु॑दते । \newline
9. प्र णु॑दते नुदते॒ प्र प्र णु॑दते॒ या या नु॑दते॒ प्र प्र णु॑दते॒ याः । \newline
10. नु॒द॒ते॒ या या नु॑दते नुदते॒ याः प्रा॒तः प्रा॒तर् या नु॑दते नुदते॒ याः प्रा॒तः । \newline
11. याः प्रा॒तः प्रा॒तर् या याः प्रा॒तर् या॒ज्या॑ या॒ज्याः᳚ प्रा॒तर् या याः प्रा॒तर् या॒ज्याः᳚ । \newline
12. प्रा॒तर् या॒ज्या॑ या॒ज्याः᳚ प्रा॒तः प्रा॒तर् या॒ज्याः᳚ स्युः स्युर् या॒ज्याः᳚ प्रा॒तः प्रा॒तर् या॒ज्याः᳚ स्युः । \newline
13. या॒ज्याः᳚ स्युः स्युर् या॒ज्या॑ या॒ज्याः᳚ स्यु स्ता स्ताः स्युर् या॒ज्या॑ या॒ज्याः᳚ स्यु स्ताः । \newline
14. स्यु स्ता स्ताः स्युः स्यु स्ताः सा॒यꣳ सा॒यम् ताः स्युः स्यु स्ताः सा॒यम् । \newline
15. ताः सा॒यꣳ सा॒यम् ता स्ताः सा॒यम् पु॑रोनुवा॒क्याः᳚ पुरोनुवा॒क्याः᳚ सा॒यम् ता स्ताः सा॒यम् पु॑रोनुवा॒क्याः᳚ । \newline
16. सा॒यम् पु॑रोनुवा॒क्याः᳚ पुरोनुवा॒क्याः᳚ सा॒यꣳ सा॒यम् पु॑रोनुवा॒क्याः᳚ कुर्यात् कुर्यात् पुरोनुवा॒क्याः᳚ सा॒यꣳ सा॒यम् पु॑रोनुवा॒क्याः᳚ कुर्यात् । \newline
17. पु॒रो॒नु॒वा॒क्याः᳚ कुर्यात् कुर्यात् पुरोनुवा॒क्याः᳚ पुरोनुवा॒क्याः᳚ कुर्या॒ दया॑तयामत्वा॒या या॑तयामत्वाय कुर्यात् पुरोनुवा॒क्याः᳚ पुरोनुवा॒क्याः᳚ कुर्या॒ दया॑तयामत्वाय । \newline
18. पु॒रो॒नु॒वा॒क्या॑ इति॑ पुरः - अ॒नु॒वा॒क्याः᳚ । \newline
19. कु॒र्या॒ दया॑तयामत्वा॒या या॑तयामत्वाय कुर्यात् कुर्या॒ दया॑तयामत्वाय ति॒स्र स्ति॒स्रो ऽया॑तयामत्वाय कुर्यात् कुर्या॒ दया॑तयामत्वाय ति॒स्रः । \newline
20. अया॑तयामत्वाय ति॒स्र स्ति॒स्रो ऽया॑तयामत्वा॒या या॑तयामत्वाय ति॒स्र उ॑प॒सद॑ उप॒सद॑ स्ति॒स्रो ऽया॑तयामत्वा॒या या॑तयामत्वाय ति॒स्र उ॑प॒सदः॑ । \newline
21. अया॑तयामत्वा॒येत्यया॑तयाम - त्वा॒य॒ । \newline
22. ति॒स्र उ॑प॒सद॑ उप॒सद॑ स्ति॒स्र स्ति॒स्र उ॑प॒सद॒ उपोपो॑ प॒सद॑ स्ति॒स्र स्ति॒स्र उ॑प॒सद॒ उप॑ । \newline
23. उ॒प॒सद॒ उपोपो॑प॒सद॑ उप॒सद॒ उपै᳚त्ये॒ त्युपो॑प॒सद॑ उप॒सद॒ उपै॑ति । \newline
24. उ॒प॒सद॒ इत्यु॑प - सदः॑ । \newline
25. उपै᳚त्ये॒ त्युपोपै॑ति॒ त्रय॒ स्त्रय॑ ए॒त्युपोपै॑ति॒ त्रयः॑ । \newline
26. ए॒ति॒ त्रय॒ स्त्रय॑ एत्येति॒ त्रय॑ इ॒म इ॒मे त्रय॑ एत्येति॒ त्रय॑ इ॒मे । \newline
27. त्रय॑ इ॒म इ॒मे त्रय॒ स्त्रय॑ इ॒मे लो॒का लो॒का इ॒मे त्रय॒ स्त्रय॑ इ॒मे लो॒काः । \newline
28. इ॒मे लो॒का लो॒का इ॒म इ॒मे लो॒का इ॒मा नि॒मान् ॅलो॒का इ॒म इ॒मे लो॒का इ॒मान् । \newline
29. लो॒का इ॒मा नि॒मान् ॅलो॒का लो॒का इ॒माने॒ वैवेमान् ॅलो॒का लो॒का इ॒मा ने॒व । \newline
30. इ॒माने॒ वैवेमा नि॒माने॒व लो॒कान् ॅलो॒काने॒वेमा नि॒मा ने॒व लो॒कान् । \newline
31. ए॒व लो॒कान् ॅलो॒काने॒ वैव लो॒कान् प्री॑णाति प्रीणाति लो॒काने॒ वैव लो॒कान् प्री॑णाति । \newline
32. लो॒कान् प्री॑णाति प्रीणाति लो॒कान् ॅलो॒कान् प्री॑णाति॒ षट् थ्षट् प्री॑णाति लो॒कान् ॅलो॒कान् प्री॑णाति॒ षट् । \newline
33. प्री॒णा॒ति॒ षट् थ्षट् प्री॑णाति प्रीणाति॒ षट् थ्सꣳ सꣳ षट् प्री॑णाति प्रीणाति॒ षट् थ्सम् । \newline
34. षट् थ्सꣳ सꣳ षट् थ्षट् थ्सम् प॑द्यन्ते पद्यन्ते॒ सꣳ षट् थ्षट् थ्सम् प॑द्यन्ते । \newline
35. सम् प॑द्यन्ते पद्यन्ते॒ सꣳ सम् प॑द्यन्ते॒ षट् थ्षट् प॑द्यन्ते॒ सꣳ सम् प॑द्यन्ते॒ षट् । \newline
36. प॒द्य॒न्ते॒ षट् थ्षट् प॑द्यन्ते पद्यन्ते॒ षड् वै वै षट् प॑द्यन्ते पद्यन्ते॒ षड् वै । \newline
37. षड् वै वै षट् थ्षड् वा ऋ॒तव॑ ऋ॒तवो॒ वै षट् थ्षड् वा ऋ॒तवः॑ । \newline
38. वा ऋ॒तव॑ ऋ॒तवो॒ वै वा ऋ॒तव॑ ऋ॒तू नृ॒तू नृ॒तवो॒ वै वा ऋ॒तव॑ ऋ॒तून् । \newline
39. ऋ॒तव॑ ऋ॒तू नृ॒तू नृ॒तव॑ ऋ॒तव॑ ऋ॒तूने॒ वैव र्‌तू नृ॒तव॑ ऋ॒तव॑ ऋ॒तूने॒व । \newline
40. ऋ॒तूने॒ वैव र्‌तू नृ॒तू ने॒व प्री॑णाति प्रीणात्ये॒व र्‌तू नृ॒तू ने॒व प्री॑णाति । \newline
41. ए॒व प्री॑णाति प्रीणा त्ये॒वैव प्री॑णाति॒ द्वाद॑श॒ द्वाद॑श प्रीणा त्ये॒वैव प्री॑णाति॒ द्वाद॑श । \newline
42. प्री॒णा॒ति॒ द्वाद॑श॒ द्वाद॑श प्रीणाति प्रीणाति॒ द्वाद॑शा॒ हीने॒ ऽहीने॒ द्वाद॑श प्रीणाति प्रीणाति॒ द्वाद॑शा॒ हीने᳚ । \newline
43. द्वाद॑शा॒ हीने॒ ऽहीने॒ द्वाद॑श॒ द्वाद॑शा॒ हीने॒ सोमे॒ सोमे॒ ऽहीने॒ द्वाद॑श॒ द्वाद॑शा॒ हीने॒ सोमे᳚ । \newline
44. अ॒हीने॒ सोमे॒ सोमे॒ ऽहीने॒ ऽहीने॒ सोम॒ उपोप॒ सोमे॒ ऽहीने॒ ऽहीने॒ सोम॒ उप॑ । \newline
45. सोम॒ उपोप॒ सोमे॒ सोम॒ उपै᳚त्ये॒ त्युप॒ सोमे॒ सोम॒ उपै॑ति । \newline
46. उपै᳚त्ये॒ त्युपोपै॑ति॒ द्वाद॑श॒ द्वाद॑शै॒ त्युपो पै॑ति॒ द्वाद॑श । \newline
47. ए॒ति॒ द्वाद॑श॒ द्वाद॑ शैत्येति॒ द्वाद॑श॒ मासा॒ मासा॒ द्वाद॑ शैत्येति॒ द्वाद॑श॒ मासाः᳚ । \newline
48. द्वाद॑श॒ मासा॒ मासा॒ द्वाद॑श॒ द्वाद॑श॒ मासाः᳚ संॅवथ्स॒रः सं॑ॅवथ्स॒रो मासा॒ द्वाद॑श॒ द्वाद॑श॒ मासाः᳚ संॅवथ्स॒रः । \newline
49. मासाः᳚ संॅवथ्स॒रः सं॑ॅवथ्स॒रो मासा॒ मासाः᳚ संॅवथ्स॒रः सं॑ॅवथ्स॒रꣳ सं॑ॅवथ्स॒रꣳ सं॑ॅवथ्स॒रो मासा॒ मासाः᳚ संॅवथ्स॒रः सं॑ॅवथ्स॒रम् । \newline
50. सं॒ॅव॒थ्स॒रः सं॑ॅवथ्स॒रꣳ सं॑ॅवथ्स॒रꣳ सं॑ॅवथ्स॒रः सं॑ॅवथ्स॒रः सं॑ॅवथ्स॒र मे॒वैव सं॑ॅवथ्स॒रꣳ सं॑ॅवथ्स॒रः सं॑ॅवथ्स॒रः सं॑ॅवथ्स॒र मे॒व । \newline
51. सं॒ॅव॒थ्स॒र इति॑ सं - व॒थ्स॒रः । \newline
52. सं॒ॅव॒थ्स॒र मे॒वैव सं॑ॅवथ्स॒रꣳ सं॑ॅवथ्स॒र मे॒व प्री॑णाति प्रीणा त्ये॒व सं॑ॅवथ्स॒रꣳ सं॑ॅवथ्स॒र मे॒व प्री॑णाति । \newline
53. सं॒ॅव॒थ्स॒रमिति॑ सं - व॒थ्स॒रम् । \newline
54. ए॒व प्री॑णाति प्रीणा त्ये॒वैव प्री॑णाति॒ चतु॑र्विꣳशति॒ श्चतु॑र्विꣳशतिः प्रीणा त्ये॒वैव प्री॑णाति॒ चतु॑र्विꣳशतिः । \newline
55. प्री॒णा॒ति॒ चतु॑र्विꣳशति॒ श्चतु॑र्विꣳशतिः प्रीणाति प्रीणाति॒ चतु॑र्विꣳशतिः॒ सꣳ सम् चतु॑र्विꣳशतिः प्रीणाति प्रीणाति॒ चतु॑र्विꣳशतिः॒ सम् । \newline
56. चतु॑र्विꣳशतिः॒ सꣳ सम् चतु॑र्विꣳशति॒ श्चतु॑र्विꣳशतिः॒ सम् प॑द्यन्ते पद्यन्ते॒ सम् चतु॑र्विꣳशति॒ श्चतु॑र्विꣳशतिः॒ सम् प॑द्यन्ते । \newline
57. चतु॑र्विꣳशति॒रिति॒ चतुः॑ - विꣳ॒॒श॒तिः॒ । \newline
58. सम् प॑द्यन्ते पद्यन्ते॒ सꣳ सम् प॑द्यन्ते॒ चतु॑र्विꣳशति॒ श्चतु॑र्विꣳशतिः पद्यन्ते॒ सꣳ सम् प॑द्यन्ते॒ चतु॑र्विꣳशतिः । \newline
\pagebreak
\markright{ TS 6.2.3.5  \hfill https://www.vedavms.in \hfill}

\section{ TS 6.2.3.5 }

\textbf{TS 6.2.3.5 } \newline
\textbf{Samhita Paata} \newline

प॑द्यन्ते॒ चतु॑र्विꣳशति-रर्द्धमा॒सा अ॑र्द्धमा॒साने॒व प्री॑णा॒त्यारा᳚ग्रा-मवान्तरदी॒क्षा-मुपे॑या॒द्यः का॒मये॑ता॒ऽस्मिन् मे॑ लो॒केऽर्द्धु॑कꣳ स्या॒दित्येक॒मग्रेऽथ॒ द्वावथ॒ त्रीनथ॑ च॒तुर॑ ए॒षा वा आरा᳚ग्रा ऽवान्तरदी॒क्षा ऽस्मिन्ने॒वास्मै॑ लो॒केऽर्द्धु॑कं भवति प॒रोव॑रीयसी-मवान्तरदी॒क्षा-मुपे॑या॒द्यः का॒मये॑ता॒मुष्मि॑न् मे लो॒केऽर्द्धु॑कꣳ स्या॒दिति॑ च॒तुरोऽग्रे ( ) ऽथ॒ त्रीनथ॒ द्वावथैक॑मे॒षा वै प॒रोव॑रीयस्य-वान्तरदी॒क्षा ऽमुष्मि॑न्ने॒वास्मै॑ लो॒केऽर्द्धु॑कं भवति ॥ \newline

\textbf{Pada Paata} \newline

प॒द्य॒न्ते॒ । चतु॑र्विꣳशति॒रिति॒ चतुः॑ - विꣳ॒॒श॒तिः॒ । अ॒द्‌र्ध॒मा॒सा इत्य॑द्‌र्ध - मा॒साः । अ॒द्‌र्ध॒मा॒सानित्य॑र्ध - मा॒सान् । ए॒व । प्री॒णा॒ति॒ । आरा᳚ग्रा॒मित्यारा᳚ - अ॒ग्रा॒म् । अ॒वा॒न्त॒र॒दी॒क्षामित्य॑वान्तर - दी॒क्षाम् । उपेति॑ । इ॒या॒त् । यः । का॒मये॑त । अ॒स्मिन्न् । मे॒ । लो॒के । अद्‌र्धु॑कम् । स्या॒त् । इति॑ । एक᳚म् । अग्रे᳚ । अथ॑ । द्वौ । अथ॑ । त्रीन् । अथ॑ । च॒तुरः॑ । ए॒षा । वै । आरा॒ग्रेत्यारा᳚ - अ॒ग्रा॒ । अ॒वा॒न्त॒र॒दी॒क्षेत्य॑वान्तर - दी॒क्षा । अ॒स्मिन्न् । ए॒व । अ॒स्मै॒ । लो॒के । अद्‌र्धु॑कम् । भ॒व॒ति॒ । प॒रोव॑रीयसी॒मिति॑ प॒रः - व॒री॒य॒सी॒म् । अ॒वा॒न्त॒र॒दी॒क्षामित्य॑वान्तर - दी॒क्षाम् । उपेति॑ । इ॒या॒त् । यः । का॒मये॑त । अ॒मुष्मिन्न्॑ । मे॒ । लो॒के । अद्‌र्धु॑कम् । स्या॒त् । इति॑ । च॒तुरः॑ । अग्रे᳚ ( ) । अथ॑ । त्रीन् । अथ॑ । द्वौ । अथ॑ । एक᳚म् । ए॒षा । वै । प॒रोव॑रीय॒सीति॑ प॒रः - व॒री॒य॒सि॒ । अ॒वा॒न्त॒र॒दी॒क्षेत्य॑वान्तर - दी॒क्षा । अ॒मुष्मिन्न्॑ । ए॒व । अ॒स्मै॒ । लो॒के । अद्‌र्धु॑कम् । भ॒व॒ति॒ ॥  \newline


\textbf{Krama Paata} \newline

प॒द्य॒न्ते॒ चतु॑र्विꣳशतिः । चतु॑र्विꣳशतिरर्द्धमा॒साः । चतु॑र्विꣳशति॒रिति॒ चतुः॑ - विꣳ॒॒श॒तिः॒ । अ॒र्द्ध॒मा॒सा अ॑र्द्धमा॒सान् । अ॒र्द्ध॒मा॒सा इत्य॑र्द्ध - मा॒साः । अ॒र्द्ध॒मा॒साने॒व । अ॒र्द्ध॒मा॒सानित्य॑र्द्ध - मा॒सान् । ए॒व प्री॑णाति । प्री॒णा॒त्यारा᳚ग्राम् । आरा᳚ग्रामवान्तरदी॒क्षाम् । आरा᳚ग्रा॒मित्यारा᳚ - अ॒ग्रा॒म् । अ॒वा॒न्त॒र॒दी॒क्षामुप॑ । अ॒वा॒न्त॒र॒दी॒क्षामित्य॑वान्तर - दी॒क्षाम् । उपे॑यात् । इ॒या॒द् यः । यः का॒मये॑त । का॒मये॑ता॒स्मिन्न् । अ॒स्मिन् मे᳚ । मे॒ लो॒के । लो॒केऽर्द्धु॑कम् । अर्द्धु॑कꣳ स्यात् । स्या॒दिति॑ । इत्येक᳚म् । एक॒मग्रे᳚ । अग्रेऽथ॑ । अथ॒ द्वौ । द्वावथ॑ । अथ॒ त्रीन् । त्रीनथ॑ । अथ॑ च॒तुरः॑ । च॒तुर॑ ए॒षा । ए॒षा वै । वा आरा᳚ग्रा । आरा᳚ग्राऽवान्तरदी॒क्षा । आरा॒ग्रेत्यारा᳚ - अ॒ग्रा॒ । अ॒वा॒न्त॒र॒दी॒क्षाऽस्मिन्न् । अ॒वा॒न्त॒र॒दी॒क्षेत्य॑वान्तर - दी॒क्षा । अ॒स्मिन्ने॒व । ए॒वास्मै᳚ । अ॒स्मै॒ लो॒के । लो॒केऽर्द्धु॑कम् । अर्द्धु॑कम् भवति । भ॒व॒ति॒ प॒रोव॑रीयसीम् । प॒रोव॑रीयसीमवान्तरदी॒क्षाम् । प॒रोव॑रीयसी॒मिति॑ प॒रः - व॒री॒य॒सी॒म् । अ॒वा॒न्त॒र॒दी॒क्षामुप॑ । अ॒वा॒न्त॒र॒दी॒क्षामित्य॑वान्तर - दी॒क्षाम् । उपे॑यात् । इ॒या॒द् यः । यः का॒मये॑त । का॒मये॑ता॒मुष्मिन्न्॑ । अ॒मुष्मि॑न् मे । मे॒ लो॒के । लो॒केऽर्द्धु॑कम् । अर्द्धु॑कꣳ स्यात् । स्या॒दिति॑ । इति॑ च॒तुरः॑ । च॒तुरोऽग्रे᳚ ( ) । अग्रेऽथ॑ । अथ॒ त्रीन् । त्रीनथ॑ । अथ॒ द्वौ । द्वावथ॑ । अथैक᳚म् । एक॑मे॒षा । ए॒षा वै । वै प॒रोव॑रीयसी । प॒रोव॑रीयस्यवान्तरदी॒क्षा । प॒रोव॑रीय॒सीति॑ प॒रः - व॒री॒य॒सी॒ । अ॒वा॒न्त॒र॒दी॒क्षाऽमुष्मिन्न्॑ । अ॒वा॒न्त॒र॒दी॒क्षेत्य॑वान्तर - दी॒क्षा । अ॒मुष्मि॑न्ने॒व । ए॒वास्मै᳚ । अ॒स्मै॒ लो॒के । लो॒केऽर्द्धु॑कम् । अर्द्धु॑कम् भवति । भ॒व॒तीति॑ भवति । \newline

\textbf{Jatai Paata} \newline

1. प॒द्य॒न्ते॒ चतु॑र्विꣳशति॒ श्चतु॑र्विꣳशतिः पद्यन्ते पद्यन्ते॒ चतु॑र्विꣳशतिः । \newline
2. चतु॑र्विꣳशति रर्द्धमा॒सा अ॑र्द्धमा॒सा श्चतु॑र्विꣳशति॒ श्चतु॑र्विꣳशति रर्द्धमा॒साः । \newline
3. चतु॑र्विꣳशति॒रिति॒ चतुः॑ - विꣳ॒॒श॒तिः॒ । \newline
4. अ॒र्द्ध॒मा॒सा अ॑र्द्धमा॒सा न॑र्द्धमा॒सा न॑र्द्धमा॒सा अ॑र्द्धमा॒सा अ॑र्द्धमा॒सान् । \newline
5. अ॒र्द्ध॒मा॒सा इत्य॑र्द्ध - मा॒साः । \newline
6. अ॒र्द्ध॒मा॒सा ने॒वैवार्द्ध॑मा॒सा न॑र्द्धमा॒सा ने॒व । \newline
7. अ॒र्द्ध॒मा॒सानित्य॑र्द्ध - मा॒सान् । \newline
8. ए॒व प्री॑णाति प्रीणा त्ये॒वैव प्री॑णाति । \newline
9. प्री॒णा॒ त्यारा᳚ग्रा॒ मारा᳚ग्राम् प्रीणाति प्रीणा॒ त्यारा᳚ग्राम् । \newline
10. आरा᳚ग्रा मवान्तरदी॒क्षा म॑वान्तरदी॒क्षा मारा᳚ग्रा॒ मारा᳚ग्रा मवान्तरदी॒क्षाम् । \newline
11. आरा᳚ग्रा॒मित्यारा᳚ - अ॒ग्रा॒म् । \newline
12. अ॒वा॒न्त॒र॒दी॒क्षा मुपोपा॑वान्तरदी॒क्षा म॑वान्तरदी॒क्षा मुप॑ । \newline
13. अ॒वा॒न्त॒र॒दी॒क्षामित्य॑वान्तर - दी॒क्षाम् । \newline
14. उपे॑या दिया॒ दुपोपे॑यात् । \newline
15. इ॒या॒द् यो य इ॑या दिया॒द् यः । \newline
16. यः का॒मये॑त का॒मये॑त॒ यो यः का॒मये॑त । \newline
17. का॒मये॑ता॒ स्मिन् न॒स्मिन् का॒मये॑त का॒मये॑ता॒ स्मिन्न् । \newline
18. अ॒स्मिन् मे॑ मे॒ ऽस्मिन् न॒स्मिन् मे᳚ । \newline
19. मे॒ लो॒के लो॒के मे॑ मे लो॒के । \newline
20. लो॒के ऽर्द्धु॑क॒ मर्द्धु॑कम् ॅलो॒के लो॒के ऽर्द्धु॑कम् । \newline
21. अर्द्धु॑कꣳ स्याथ् स्या॒ दर्द्धु॑क॒ मर्द्धु॑कꣳ स्यात् । \newline
22. स्या॒ दितीति॑ स्याथ् स्या॒ दिति॑ । \newline
23. इत्येक॒ मेक॒ मिती त्येक᳚म् । \newline
24. एक॒ मग्रे ऽग्र॒ एक॒ मेक॒ मग्रे᳚ । \newline
25. अग्रे ऽथाथाग्रे ऽग्रे ऽथ॑ । \newline
26. अथ॒ द्वौ द्वा वथाथ॒ द्वौ । \newline
27. द्वा वथाथ॒ द्वौ द्वा वथ॑ । \newline
28. अथ॒ त्रीꣳ स्त्री नथाथ॒ त्रीन् । \newline
29. त्री नथाथ॒ त्रीꣳ स्त्री नथ॑ । \newline
30. अथ॑ च॒तुर॑ श्च॒तुरो ऽथाथ॑ च॒तुरः॑ । \newline
31. च॒तुर॑ ए॒षैषा च॒तुर॑ श्च॒तुर॑ ए॒षा । \newline
32. ए॒षा वै वा ए॒षैषा वै । \newline
33. वा आरा॒ग्रा ऽऽरा᳚ग्रा॒ वै वा आरा᳚ग्रा । \newline
34. आरा᳚ग्रा ऽवान्तरदी॒क्षा ऽवा᳚न्तरदी॒क्षा ऽऽरा॒ग्रा ऽऽरा᳚ग्रा ऽवान्तरदी॒क्षा । \newline
35. आरा॒ग्रेत्यारा᳚ - अ॒ग्रा॒ । \newline
36. अ॒वा॒न्त॒र॒दी॒क्षा ऽस्मिन् न॒स्मिन् न॑वान्तरदी॒क्षा ऽवा᳚न्तरदी॒क्षा ऽस्मिन्न् । \newline
37. अ॒वा॒न्त॒र॒दी॒क्षेत्य॑वान्तर - दी॒क्षा । \newline
38. अ॒स्मिन् ने॒वैवास्मिन् न॒स्मिन् ने॒व । \newline
39. ए॒वास्मा॑ अस्मा ए॒वैवास्मै᳚ । \newline
40. अ॒स्मै॒ लो॒के लो॒के᳚ ऽस्मा अस्मै लो॒के । \newline
41. लो॒के ऽर्द्धु॑क॒ मर्द्धु॑कम् ॅलो॒के लो॒के ऽर्द्धु॑कम् । \newline
42. अर्द्धु॑कम् भवति भव॒ त्यर्द्धु॑क॒ मर्द्धु॑कम् भवति । \newline
43. भ॒व॒ति॒ प॒रोव॑रीयसीम् प॒रोव॑रीयसीम् भवति भवति प॒रोव॑रीयसीम् । \newline
44. प॒रोव॑रीयसी मवान्तरदी॒क्षा म॑वान्तरदी॒क्षाम् प॒रोव॑रीयसीम् प॒रोव॑रीयसी मवान्तरदी॒क्षाम् । \newline
45. प॒रोव॑रीयसी॒मिति॑ प॒रः - व॒री॒य॒सी॒म् । \newline
46. अ॒वा॒न्त॒र॒दी॒क्षा मुपोपा॑वान्तरदी॒क्षा म॑वान्तरदी॒क्षा मुप॑ । \newline
47. अ॒वा॒न्त॒र॒दी॒क्षामित्य॑वान्तर - दी॒क्षाम् । \newline
48. उपे॑या दिया॒ दुपोपे॑यात् । \newline
49. इ॒या॒द् यो य इ॑या दिया॒द् यः । \newline
50. यः का॒मये॑त का॒मये॑त॒ यो यः का॒मये॑त । \newline
51. का॒मये॑ता॒ मुष्मि॑न् न॒मुष्मि॑न् का॒मये॑त का॒मये॑ता॒ मुष्मिन्न्॑ । \newline
52. अ॒मुष्मि॑न् मे मे॒ ऽमुष्मि॑न् न॒मुष्मि॑न् मे । \newline
53. मे॒ लो॒के लो॒के मे॑ मे लो॒के । \newline
54. लो॒के ऽर्द्धु॑क॒ मर्द्धु॑कम् ॅलो॒के लो॒के ऽर्द्धु॑कम् । \newline
55. अर्द्धु॑कꣳ स्याथ् स्या॒ दर्द्धु॑क॒ मर्द्धु॑कꣳ स्यात् । \newline
56. स्या॒ दितीति॑ स्याथ् स्या॒ दिति॑ । \newline
57. इति॑ च॒तुर॑ श्च॒तुर॒ इतीति॑ च॒तुरः॑ । \newline
58. च॒तुरो ऽग्रे ऽग्रे॑ च॒तुर॑ श्च॒तुरो ऽग्रे᳚ । \newline
59. अग्रे ऽथाथाग्रे ऽग्रे ऽथ॑ । \newline
60. अथ॒ त्रीꣳ स्त्री नथाथ॒ त्रीन् । \newline
61. त्री नथाथ॒ त्रीꣳ स्त्री नथ॑ । \newline
62. अथ॒ द्वौ द्वा वथाथ॒ द्वौ । \newline
63. द्वा वथाथ॒ द्वौ द्वा वथ॑ । \newline
64. अथैक॒ मेक॒ मथाथैक᳚म् । \newline
65. एक॑ मे॒षैषैक॒ मेक॑ मे॒षा । \newline
66. ए॒षा वै वा ए॒षैषा वै । \newline
67. वै प॒रोव॑रीयसी प॒रोव॑रीयसी॒ वै वै प॒रोव॑रीयसी । \newline
68. प॒रोव॑रीय स्यवान्तरदी॒क्षा ऽवा᳚न्तरदी॒क्षा प॒रोव॑रीयसी प॒रोव॑रीय स्यवान्तरदी॒क्षा । \newline
69. प॒रोव॑रीय॒सीति॑ प॒रः - व॒री॒य॒सी॒ । \newline
70. अ॒वा॒न्त॒र॒दी॒क्षा ऽमुष्मि॑न् न॒मुष्मि॑न् नवान्तरदी॒क्षा ऽवा᳚न्तरदी॒क्षा ऽमुष्मिन्न्॑ । \newline
71. अ॒वा॒न्त॒र॒दी॒क्षेत्य॑वान्तर - दी॒क्षा । \newline
72. अ॒मुष्मि॑न् ने॒वैवामुष्मि॑न् न॒मुष्मि॑न् ने॒व । \newline
73. ए॒वास्मा॑ अस्मा ए॒वैवास्मै᳚ । \newline
74. अ॒स्मै॒ लो॒के लो॒के᳚ ऽस्मा अस्मै लो॒के । \newline
75. लो॒के ऽर्द्धु॑क॒ मर्द्धु॑कम् ॅलो॒के लो॒के ऽर्द्धु॑कम् । \newline
76. अर्द्धु॑कम् भवति भव॒ त्यर्द्धु॑क॒ मर्द्धु॑कम् भवति । \newline
77. भ॒व॒तीति॑ भवति । \newline

\textbf{Ghana Paata } \newline

1. प॒द्य॒न्ते॒ चतु॑र्विꣳशति॒ श्चतु॑र्विꣳशतिः पद्यन्ते पद्यन्ते॒ चतु॑र्विꣳशति रर्द्धमा॒सा अ॑र्द्धमा॒सा श्चतु॑र्विꣳशतिः पद्यन्ते पद्यन्ते॒ चतु॑र्विꣳशति रर्द्धमा॒साः । \newline
2. चतु॑र्विꣳशति रर्द्धमा॒सा अ॑र्द्धमा॒सा श्चतु॑र्विꣳशति॒ श्चतु॑र्विꣳशति रर्द्धमा॒सा अ॑र्द्धमा॒सा न॑र्द्धमा॒सा न॑र्द्धमा॒सा श्चतु॑र्विꣳशति॒ श्चतु॑र्विꣳशति रर्द्धमा॒सा अ॑र्द्धमा॒सान् । \newline
3. चतु॑र्विꣳशति॒रिति॒ चतुः॑ - विꣳ॒॒श॒तिः॒ । \newline
4. अ॒र्द्ध॒मा॒सा अ॑र्द्धमा॒सा न॑र्द्धमा॒सा न॑र्द्धमा॒सा अ॑र्द्धमा॒सा अ॑र्द्धमा॒सा ने॒वैवार्द्ध॑मा॒सा न॑र्द्धमा॒सा अ॑र्द्धमा॒सा अ॑र्द्धमा॒साने॒व । \newline
5. अ॒र्द्ध॒मा॒सा इत्य॑र्द्ध - मा॒साः । \newline
6. अ॒र्द्ध॒मा॒साने॒ वैवार्द्ध॑मा॒सा न॑र्द्धमा॒साने॒व प्री॑णाति प्रीणा त्ये॒वार्द्ध॑मा॒सा न॑र्द्धमा॒साने॒व प्री॑णाति । \newline
7. अ॒र्द्ध॒मा॒सानित्य॑र्ध - मा॒सान् । \newline
8. ए॒व प्री॑णाति प्रीणा त्ये॒वैव प्री॑णा॒ त्यारा᳚ग्रा॒ मारा᳚ग्राम् प्रीणा त्ये॒वैव प्री॑णा॒ त्यारा᳚ग्राम् । \newline
9. प्री॒णा॒ त्यारा᳚ग्रा॒ मारा᳚ग्राम् प्रीणाति प्रीणा॒ त्यारा᳚ग्रा मवान्तरदी॒क्षा म॑वान्तरदी॒क्षा मारा᳚ग्राम् प्रीणाति प्रीणा॒ त्यारा᳚ग्रा मवान्तरदी॒क्षाम् । \newline
10. आरा᳚ग्रा मवान्तरदी॒क्षा म॑वान्तरदी॒क्षा मारा᳚ग्रा॒ मारा᳚ग्रा मवान्तरदी॒क्षा मुपोपा॑ वान्तरदी॒क्षा मारा᳚ग्रा॒ मारा᳚ग्रा मवान्तरदी॒क्षा मुप॑ । \newline
11. आरा᳚ग्रा॒मित्यारा᳚ - अ॒ग्रा॒म् । \newline
12. अ॒वा॒न्त॒र॒दी॒क्षा मुपोपा॑ वान्तरदी॒क्षा म॑वान्तरदी॒क्षा मुपे॑ यादिया॒ दुपा॑ वान्तरदी॒क्षा म॑वान्तरदी॒क्षा मुपे॑ यात् । \newline
13. अ॒वा॒न्त॒र॒दी॒क्षामित्य॑वान्तर - दी॒क्षाम् । \newline
14. उपे॑या दिया॒ दुपो पे॑या॒द् यो य इ॑या॒ दुपो पे॑या॒द् यः । \newline
15. इ॒या॒द् यो य इ॑या दिया॒द् यः का॒मये॑त का॒मये॑त॒ य इ॑या दिया॒द् यः का॒मये॑त । \newline
16. यः का॒मये॑त का॒मये॑त॒ यो यः का॒मये॑ ता॒स्मिन् न॒स्मिन् का॒मये॑त॒ यो यः का॒मये॑ता॒स्मिन्न् । \newline
17. का॒मये॑ता॒स्मिन् न॒स्मिन् का॒मये॑त का॒मये॑ ता॒स्मिन् मे॑ मे॒ ऽस्मिन् का॒मये॑त का॒मये॑ ता॒स्मिन् मे᳚ । \newline
18. अ॒स्मिन् मे॑ मे॒ ऽस्मिन् न॒स्मिन् मे॑ लो॒के लो॒के मे॒ ऽस्मिन् न॒स्मिन् मे॑ लो॒के । \newline
19. मे॒ लो॒के लो॒के मे॑ मे लो॒के ऽर्द्धु॑क॒ मर्द्धु॑कम् ॅलो॒के मे॑ मे लो॒के ऽर्द्धु॑कम् । \newline
20. लो॒के ऽर्द्धु॑क॒ मर्द्धु॑कम् ॅलो॒के लो॒के ऽर्द्धु॑कꣳ स्याथ् स्या॒ दर्द्धु॑कम् ॅलो॒के लो॒के ऽर्द्धु॑कꣳ स्यात् । \newline
21. अर्द्धु॑कꣳ स्याथ् स्या॒ दर्द्धु॑क॒ मर्द्धु॑कꣳ स्या॒दि तीति॑ स्या॒ दर्द्धु॑क॒ मर्द्धु॑कꣳ स्या॒ दिति॑ । \newline
22. स्या॒ दितीति॑ स्याथ् स्या॒दि त्येक॒ मेक॒ मिति॑ स्याथ् स्या॒ दित्येक᳚म् । \newline
23. इत्येक॒ मेक॒ मितीत्येक॒ मग्रे ऽग्र॒ एक॒ मितीत्येक॒ मग्रे᳚ । \newline
24. एक॒ मग्रे ऽग्र॒ एक॒ मेक॒ मग्रे ऽथा थाग्र॒ एक॒ मेक॒ मग्रे ऽथ॑ । \newline
25. अग्रे ऽथा थाग्रे ऽग्रे ऽथ॒ द्वौ द्वा वथाग्रे ऽग्रे ऽथ॒ द्वौ । \newline
26. अथ॒ द्वौ द्वा वथाथ॒ द्वा वथाथ॒ द्वा वथाथ॒ द्वा वथ॑ । \newline
27. द्वा वथाथ॒ द्वौ द्वा वथ॒ त्रीꣳ स्त्रीनथ॒ द्वौ द्वा वथ॒ त्रीन् । \newline
28. अथ॒ त्रीꣳ स्त्रीन थाथ॒ त्रीन थाथ॒ त्रीन थाथ॒ त्रीनथ॑ । \newline
29. त्रीन थाथ॒ त्रीꣳ स्त्रीनथ॑ च॒तुर॑ श्च॒तुरो ऽथ॒ त्रीꣳ स्त्रीनथ॑ च॒तुरः॑ । \newline
30. अथ॑ च॒तुर॑ श्च॒तुरो ऽथाथ॑ च॒तुर॑ ए॒षैषा च॒तुरो ऽथाथ॑ च॒तुर॑ ए॒षा । \newline
31. च॒तुर॑ ए॒षैषा च॒तुर॑ श्च॒तुर॑ ए॒षा वै वा ए॒षा च॒तुर॑ श्च॒तुर॑ ए॒षा वै । \newline
32. ए॒षा वै वा ए॒षैषा वा आरा॒ग्रा ऽऽरा᳚ग्रा॒ वा ए॒षैषा वा आरा᳚ग्रा । \newline
33. वा आरा॒ग्रा ऽऽरा᳚ग्रा॒ वै वा आरा᳚ग्रा ऽवान्तरदी॒क्षा ऽवा᳚न्तरदी॒क्षा ऽऽरा᳚ग्रा॒ वै वा आरा᳚ग्रा ऽवान्तरदी॒क्षा । \newline
34. आरा᳚ग्रा ऽवान्तरदी॒क्षा ऽवा᳚न्तरदी॒क्षा ऽऽरा॒ग्रा ऽऽरा᳚ग्रा ऽवान्तरदी॒क्षा ऽस्मिन् न॒स्मिन् न॑वान्तरदी॒क्षा ऽऽरा॒ग्रा ऽऽरा᳚ग्रा ऽवान्तरदी॒क्षा ऽस्मिन्न् । \newline
35. आरा॒ग्रेत्यारा᳚ - अ॒ग्रा॒ । \newline
36. अ॒वा॒न्त॒र॒दी॒क्षा ऽस्मिन् न॒स्मिन् न॑वान्तरदी॒क्षा ऽवा᳚न्तरदी॒क्षा ऽस्मिन्ने॒ वैवास्मिन् न॑वान्तरदी॒क्षा ऽवा᳚न्तरदी॒क्षा ऽस्मिन्ने॒व । \newline
37. अ॒वा॒न्त॒र॒दी॒क्षेत्य॑वान्तर - दी॒क्षा । \newline
38. अ॒स्मिन् ने॒वैवास्मिन् न॒स्मिन् ने॒वास्मा॑ अस्मा ए॒वास्मिन् न॒स्मिन् ने॒वास्मै᳚ । \newline
39. ए॒वास्मा॑ अस्मा ए॒वै वास्मै॑ लो॒के लो॒के᳚ ऽस्मा ए॒वै वास्मै॑ लो॒के । \newline
40. अ॒स्मै॒ लो॒के लो॒के᳚ ऽस्मा अस्मै लो॒के ऽर्द्धु॑क॒ मर्द्धु॑कम् ॅलो॒के᳚ ऽस्मा अस्मै लो॒के ऽर्द्धु॑कम् । \newline
41. लो॒के ऽर्द्धु॑क॒ मर्द्धु॑कम् ॅलो॒के लो॒के ऽर्द्धु॑कम् भवति भव॒ त्यर्द्धु॑कम् ॅलो॒के लो॒के ऽर्द्धु॑कम् भवति । \newline
42. अर्द्धु॑कम् भवति भव॒ त्यर्द्धु॑क॒ मर्द्धु॑कम् भवति प॒रोव॑रीयसीम् प॒रोव॑रीयसीम् भव॒ त्यर्द्धु॑क॒ मर्द्धु॑कम् भवति प॒रोव॑रीयसीम् । \newline
43. भ॒व॒ति॒ प॒रोव॑रीयसीम् प॒रोव॑रीयसीम् भवति भवति प॒रोव॑रीयसी मवान्तरदी॒क्षा म॑वान्तरदी॒क्षाम् प॒रोव॑रीयसीम् भवति भवति प॒रोव॑रीयसी मवान्तरदी॒क्षाम् । \newline
44. प॒रोव॑रीयसी मवान्तरदी॒क्षा म॑वान्तरदी॒क्षाम् प॒रोव॑रीयसीम् प॒रोव॑रीयसी मवान्तरदी॒क्षा मुपोपा॑ वान्तरदी॒क्षाम् प॒रोव॑रीयसीम् प॒रोव॑रीयसी मवान्तरदी॒क्षा मुप॑ । \newline
45. प॒रोव॑रीयसी॒मिति॑ प॒रः - व॒री॒य॒सी॒म् । \newline
46. अ॒वा॒न्त॒र॒दी॒क्षा मुपोपा॑ वान्तरदी॒क्षा म॑वान्तरदी॒क्षा मुपे॑ यादिया॒ दुपा॑ वान्तरदी॒क्षा म॑वान्तरदी॒क्षा मुपे॑ यात् । \newline
47. अ॒वा॒न्त॒र॒दी॒क्षामित्य॑वान्तर - दी॒क्षाम् । \newline
48. उपे॑ यादिया॒ दुपोपे॑या॒द् यो य इ॑या॒ दुपोपे॑या॒द् यः । \newline
49. इ॒या॒द् यो य इ॑या दिया॒द् यः का॒मये॑त का॒मये॑त॒ य इ॑या दिया॒द् यः का॒मये॑त । \newline
50. यः का॒मये॑त का॒मये॑त॒ यो यः का॒मये॑ता॒ मुष्मि॑न् न॒मुष्मि॑न् का॒मये॑त॒ यो यः का॒मये॑ता॒ मुष्मिन्न्॑ । \newline
51. का॒मये॑ता॒ मुष्मि॑न्न॒ मुष्मि॑न् का॒मये॑त का॒मये॑ता॒ मुष्मि॑न् मे मे॒ ऽमुष्मि॑न् का॒मये॑त का॒मये॑ता॒ मुष्मि॑न् मे । \newline
52. अ॒मुष्मि॑न् मे मे॒ ऽमुष्मि॑न्न॒ मुष्मि॑न् मे लो॒के लो॒के मे॒ ऽमुष्मि॑न्न॒ मुष्मि॑न् मे लो॒के । \newline
53. मे॒ लो॒के लो॒के मे॑ मे लो॒के ऽर्द्धु॑क॒ मर्द्धु॑कम् ॅलो॒के मे॑ मे लो॒के ऽर्द्धु॑कम् । \newline
54. लो॒के ऽर्द्धु॑क॒ मर्द्धु॑कम् ॅलो॒के लो॒के ऽर्द्धु॑कꣳ स्याथ् स्या॒ दर्द्धु॑कम् ॅलो॒के लो॒के ऽर्द्धु॑कꣳ स्यात् । \newline
55. अर्द्धु॑कꣳ स्याथ् स्या॒ दर्द्धु॑क॒ मर्द्धु॑कꣳ स्या॒दि तीति॑ स्या॒ दर्द्धु॑क॒ मर्द्धु॑कꣳ स्या॒ दिति॑ । \newline
56. स्या॒दि तीति॑ स्याथ् स्या॒दिति॑ च॒तुर॑ श्च॒तुर॒ इति॑ स्याथ् स्या॒दिति॑ च॒तुरः॑ । \newline
57. इति॑ च॒तुर॑ श्च॒तुर॒ इतीति॑ च॒तुरो ऽग्रे ऽग्रे॑ च॒तुर॒ इतीति॑ च॒तुरो ऽग्रे᳚ । \newline
58. च॒तुरो ऽग्रे ऽग्रे॑ च॒तुर॑ श्च॒तुरो ऽग्रे ऽथा थाग्रे॑ च॒तुर॑ श्च॒तुरो ऽग्रे ऽथ॑ । \newline
59. अग्रे ऽथा थाग्रे ऽग्रे ऽथ॒ त्रीꣳ स्त्रीन थाग्रे ऽग्रे ऽथ॒ त्रीन् । \newline
60. अथ॒ त्रीꣳ स्त्री नथाथ॒ त्रीन थाथ॒ त्रीन थाथ॒ त्रीनथ॑ । \newline
61. त्रीनथाथ॒ त्रीꣳ स्त्रीनथ॒ द्वौ द्वा वथ॒ त्रीꣳ स्त्रीनथ॒ द्वौ । \newline
62. अथ॒ द्वौ द्वा वथाथ॒ द्वा वथाथ॒ द्वा वथाथ॒ द्वा वथ॑ । \newline
63. द्वा वथाथ॒ द्वौ द्वा वथैक॒ मेक॒ मथ॒ द्वौ द्वा वथैक᳚म् । \newline
64. अथैक॒ मेक॒ मथा थैक॑ मे॒षै षैक॒ मथा थैक॑ मे॒षा । \newline
65. एक॑ मे॒षै षैक॒ मेक॑ मे॒षा वै वा ए॒षैक॒ मेक॑ मे॒षा वै । \newline
66. ए॒षा वै वा ए॒षैषा वै प॒रोव॑रीयसी प॒रोव॑रीयसी॒ वा ए॒षैषा वै प॒रोव॑रीयसी । \newline
67. वै प॒रोव॑रीयसी प॒रोव॑रीयसी॒ वै वै प॒रोव॑रीय स्यवान्तरदी॒क्षा ऽवा᳚न्तरदी॒क्षा प॒रोव॑रीयसी॒ वै वै प॒रोव॑रीय स्यवान्तरदी॒क्षा । \newline
68. प॒रोव॑रीय स्यवान्तरदी॒क्षा ऽवा᳚न्तरदी॒क्षा प॒रोव॑रीयसी प॒रोव॑रीय स्यवान्तरदी॒क्षा ऽमुष्मि॑न् न॒मुष्मि॑न् नवान्तरदी॒क्षा प॒रोव॑रीयसी प॒रोव॑रीय स्यवान्तरदी॒क्षा ऽमुष्मिन्न्॑ । \newline
69. प॒रोव॑रीय॒सीति॑ प॒रः - व॒री॒य॒सी॒ । \newline
70. अ॒वा॒न्त॒र॒दी॒क्षा ऽमुष्मि॑न् न॒मुष्मि॑न् नवान्तरदी॒क्षा ऽवा᳚न्तरदी॒क्षा ऽमुष्मि॑न् ने॒वैवा मुष्मि॑न् नवान्तरदी॒क्षा ऽवा᳚न्तरदी॒क्षा ऽमुष्मि॑न् ने॒व । \newline
71. अ॒वा॒न्त॒र॒दी॒क्षेत्य॑वान्तर - दी॒क्षा । \newline
72. अ॒मुष्मि॑न् ने॒वै वामुष्मि॑न् न॒मुष्मि॑न् ने॒वास्मा॑ अस्मा ए॒वामुष्मि॑न् न॒मुष्मि॑न्ने॒ वास्मै᳚ । \newline
73. ए॒वास्मा॑ अस्मा ए॒वैवास्मै॑ लो॒के लो॒के᳚ ऽस्मा ए॒वैवास्मै॑ लो॒के । \newline
74. अ॒स्मै॒ लो॒के लो॒के᳚ ऽस्मा अस्मै लो॒के ऽर्द्धु॑क॒ मर्द्धु॑कम् ॅलो॒के᳚ ऽस्मा अस्मै लो॒के ऽर्द्धु॑कम् । \newline
75. लो॒के ऽर्द्धु॑क॒ मर्द्धु॑कम् ॅलो॒के लो॒के ऽर्द्धु॑कम् भवति भव॒ त्यर्द्धु॑कम् ॅलो॒के लो॒के ऽर्द्धु॑कम् भवति । \newline
76. अर्द्धु॑कम् भवति भव॒ त्यर्द्धु॑क॒ मर्द्धु॑कम् भवति । \newline
77. भ॒व॒तीति॑ भवति । \newline
\pagebreak
\markright{ TS 6.2.4.1  \hfill https://www.vedavms.in \hfill}

\section{ TS 6.2.4.1 }

\textbf{TS 6.2.4.1 } \newline
\textbf{Samhita Paata} \newline

सु॒व॒र्गं ॅवा ए॒ते लो॒कं ॅय॑न्ति॒ य उ॑प॒सद॑ उप॒यन्ति॒ तेषां॒ ॅय उ॒न्नय॑ते॒ हीय॑त ए॒व स नोद॑ने॒षीति॒ सू᳚न्नीयमिव॒ यो वै स्वा॒र्थेतां᳚ ॅय॒ताꣳ श्रा॒न्तो हीय॑त उ॒त स नि॒ष्ट्याय॑ स॒ह व॑सति॒ तस्मा᳚थ् स॒कृदु॒न्नीय॒ नाप॑र॒मुन्न॑येत द॒द्ध्नोन्न॑येतै॒तद्वै प॑शू॒नाꣳ रू॒पꣳ रू॒पेणै॒व प॒शूनव॑ रुन्धे- [  ] \newline

\textbf{Pada Paata} \newline

सु॒व॒र्गमिति॑ सुवः - गम् । वै । ए॒ते । लो॒कम् । य॒न्ति॒ । ये । उ॒प॒सद॒ इत्यु॑प - सदः॑ । उ॒प॒यन्तीयु॑प - यन्ति॑ । तेषा᳚म् । यः । उ॒न्नय॑त॒ इत्यु॑त् - नय॑ते । हीय॑ते । ए॒व । सः । न । उदिति॑ । अ॒ने॒षि॒ । इति॑ । सू᳚न्नीय॒मिति॒ सु - उ॒न्नी॒य॒म् । इ॒व॒ । यः । वै । स्वा॒र्थेता॒मिति॑ स्वार्थ-इता᳚म् । य॒ताम् । श्रा॒न्तः । हीय॑ते । उ॒त । सः । नि॒ष्ट्याय॑ । स॒ह । व॒स॒ति॒ । तस्मा᳚त् । स॒कृत् । उ॒न्नीयेत्यु॑त्-नीय॑ । न । अप॑रम् । उदिति॑ । न॒ये॒त॒ । द॒द्ध्ना । उदिति॑ । न॒ये॒त॒ । ए॒तत् । वै । प॒शू॒नाम् । रू॒पम् । रू॒पेण॑ । ए॒व । प॒शून् । अवेति॑ । रु॒न्धे॒ ।  \newline


\textbf{Krama Paata} \newline

सु॒व॒र्गम् ॅवै । सु॒व॒र्गमिति॑ सुवः - गम् । वा ए॒ते । ए॒ते लो॒कम् । लो॒कम् ॅय॑न्ति । य॒न्ति॒ ये । य उ॑प॒सदः॑ । उ॒प॒सद॑ उप॒यन्ति॑ । उ॒प॒सद॒ इत्यु॑प - सदः॑ । उ॒प॒यन्ति॒ तेषा᳚म् । उ॒प॒यन्तीत्यु॑प - यन्ति॑ । तेषा॒म् ॅयः । य उ॒न्नय॑ते । उ॒न्नय॑ते॒ हीय॑ते । उ॒न्नय॑त॒ इत्यु॑त् - नय॑ते । हीय॑त ए॒व । ए॒व सः । स न । नोत् । उद॑नेषि । अ॒ने॒षीति॑ । इति॒ सू᳚न्नीयम् । सू᳚न्नीयमिव । सू᳚न्नीय॒मिति॒ सु - उ॒न्नी॒य॒म् । इ॒व॒ यः । यो वै । वै स्वा॒र्थेता᳚म् । स्वा॒र्थेता᳚म् ॅय॒ताम् । स्वा॒र्थेता॒मिति॑ स्वार्थ - इता᳚म् । य॒ताꣳ श्रा॒न्तः । श्रा॒न्तो हीय॑ते । हीय॑त उ॒त । उ॒त सः । स नि॒ष्ट्‍याय॑ । नि॒ष्ट्‍याय॑ स॒ह । स॒ह व॑सति । व॒स॒ति॒ तस्मा᳚त् । तस्मा᳚थ् स॒कृत् । स॒कृदु॒न्नीय॑ । उ॒न्नीय॒ न । उ॒न्नीयेत्यु॑त् - नीय॑ । नाप॑रम् । अप॑र॒मुत् । उन् न॑येत । न॒ये॒त॒ द॒द्ध्ना । द॒द्ध्नोत् । उन् न॑येत । न॒ये॒तै॒तत् । ए॒तद् वै । वै प॑शू॒नाम् । प॒शू॒नाꣳ रू॒पम् । रू॒पꣳ रू॒पेण॑ । रू॒पेणै॒व । ए॒व प॒शून् । प॒शूनव॑ । अव॑ रुन्धे । रु॒न्धे॒ य॒ज्ञ्ः \newline

\textbf{Jatai Paata} \newline

1. सु॒व॒र्गं ॅवै वै सु॑व॒र्गꣳ सु॑व॒र्गं ॅवै । \newline
2. सु॒व॒र्गमिति॑ सुवः - गम् । \newline
3. वा ए॒त ए॒ते वै वा ए॒ते । \newline
4. ए॒ते लो॒कम् ॅलो॒क मे॒त ए॒ते लो॒कम् । \newline
5. लो॒कं ॅय॑न्ति यन्ति लो॒कम् ॅलो॒कं ॅय॑न्ति । \newline
6. य॒न्ति॒ ये ये य॑न्ति यन्ति॒ ये । \newline
7. य उ॑प॒सद॑ उप॒सदो॒ ये य उ॑प॒सदः॑ । \newline
8. उ॒प॒सद॑ उप॒य न्त्यु॑प॒य न्त्यु॑प॒सद॑ उप॒सद॑ उप॒यन्ति॑ । \newline
9. उ॒प॒सद॒ इत्यु॑प - सदः॑ । \newline
10. उ॒प॒यन्ति॒ तेषा॒म् तेषा॑ मुप॒य न्त्यु॑प॒यन्ति॒ तेषा᳚म् । \newline
11. उ॒प॒यन्तीयु॑प - यन्ति॑ । \newline
12. तेषां॒ ॅयो य स्तेषा॒म् तेषां॒ ॅयः । \newline
13. य उ॒न्नय॑त उ॒न्नय॑ते॒ यो य उ॒न्नय॑ते । \newline
14. उ॒न्नय॑ते॒ हीय॑ते॒ हीय॑त उ॒न्नय॑त उ॒न्नय॑ते॒ हीय॑ते । \newline
15. उ॒न्नय॑त॒ इत्यु॑त् - नय॑ते । \newline
16. हीय॑त ए॒वैव हीय॑ते॒ हीय॑त ए॒व । \newline
17. ए॒व स स ए॒वैव सः । \newline
18. स न न स स न । \newline
19. नोदुन् न नोत् । \newline
20. उद॑नेष्य ने॒ष्यु दुद॑नेषि । \newline
21. अ॒ने॒षीत् ईत्य॑ नेष्य ने॒षीति॑ । \newline
22. इती॒ सू᳚न्नीयꣳ॒॒ सू᳚न्नीय॒ मितीती॒ सू᳚न्नीयम् । \newline
23. सू᳚न्नीय मिवे॒व सू᳚न्नीयꣳ॒॒ सू᳚न्नीय मिव । \newline
24. सू᳚न्नीय॒मिति॒ सु - उ॒न्नी॒य॒म् । \newline
25. इ॒व॒ यो य इ॑वेव॒ यः । \newline
26. यो वै वै यो यो वै । \newline
27. वै स्वा॒र्थेताꣳ॑ स्वा॒र्थेतां॒ ॅवै वै स्वा॒र्थेता᳚म् । \newline
28. स्वा॒र्थेतां᳚ ॅय॒तां ॅय॒ताꣳ स्वा॒र्थेताꣳ॑ स्वा॒र्थेतां᳚ ॅय॒ताम् । \newline
29. स्वा॒र्थेता॒मिति॑ स्वार्थ - इता᳚म् । \newline
30. य॒ताꣳ श्रा॒न्तः श्रा॒न्तो य॒तां ॅय॒ताꣳ श्रा॒न्तः । \newline
31. श्रा॒न्तो हीय॑ते॒ हीय॑ते श्रा॒न्तः श्रा॒न्तो हीय॑ते । \newline
32. हीय॑त उ॒तोत हीय॑ते॒ हीय॑त उ॒त । \newline
33. उ॒त स स उ॒तोत सः । \newline
34. स नि॒ष्ट्याय॑ नि॒ष्ट्याय॒ स स नि॒ष्ट्याय॑ । \newline
35. नि॒ष्ट्याय॑ स॒ह स॒ह नि॒ष्ट्याय॑ नि॒ष्ट्याय॑ स॒ह । \newline
36. स॒ह व॑सति वसति स॒ह स॒ह व॑सति । \newline
37. व॒स॒ति॒ तस्मा॒त् तस्मा᳚द् वसति वसति॒ तस्मा᳚त् । \newline
38. तस्मा᳚थ् स॒कृथ् स॒कृत् तस्मा॒त् तस्मा᳚थ् स॒कृत् । \newline
39. स॒कृ दु॒न्नी यो॒न्नीय॑ स॒कृथ् स॒कृ दु॒न्नीय॑ । \newline
40. उ॒न्नीय॒ न नोन्नी यो॒न्नीय॒ न । \newline
41. उ॒न्नीयेत्यु॑त् - नीय॑ । \newline
42. नाप॑र॒ मप॑र॒न्न नाप॑रम् । \newline
43. अप॑र॒ मुदु दप॑र॒ मप॑र॒ मुत् । \newline
44. उन् न॑येत नये॒तोदुन् न॑येत । \newline
45. न॒ये॒त॒ द॒द्ध्ना द॒द्ध्ना न॑येत नयेत द॒द्ध्ना । \newline
46. द॒द्ध्नोदुद् द॒द्ध्ना द॒द्ध्नोत् । \newline
47. उन् न॑येत नये॒तोदुन् न॑येत । \newline
48. न॒ये॒ तै॒त दे॒तन् न॑येत नयेतै॒तत् । \newline
49. ए॒तद् वै वा ए॒त दे॒तद् वै । \newline
50. वै प॑शू॒नाम् प॑शू॒नां ॅवै वै प॑शू॒नाम् । \newline
51. प॒शू॒नाꣳ रू॒पꣳ रू॒पम् प॑शू॒नाम् प॑शू॒नाꣳ रू॒पम् । \newline
52. रू॒पꣳ रू॒पेण॑ रू॒पेण॑ रू॒पꣳ रू॒पꣳ रू॒पेण॑ । \newline
53. रू॒पेणै॒वैव रू॒पेण॑ रू॒पेणै॒व । \newline
54. ए॒व प॒शून् प॒शू ने॒वैव प॒शून् । \newline
55. प॒शू नवाव॑ प॒शून् प॒शू नव॑ । \newline
56. अव॑ रुन्धे रु॒न्धे ऽवाव॑ रुन्धे । \newline
57. रु॒न्धे॒ य॒ज्ञो य॒ज्ञो रु॑न्धे रुन्धे य॒ज्ञ्ः । \newline

\textbf{Ghana Paata } \newline

1. सु॒व॒र्गं ॅवै वै सु॑व॒र्गꣳ सु॑व॒र्गं ॅवा ए॒त ए॒ते वै सु॑व॒र्गꣳ सु॑व॒र्गं ॅवा ए॒ते । \newline
2. सु॒व॒र्गमिति॑ सुवः - गम् । \newline
3. वा ए॒त ए॒ते वै वा ए॒ते लो॒कम् ॅलो॒क मे॒ते वै वा ए॒ते लो॒कम् । \newline
4. ए॒ते लो॒कम् ॅलो॒क मे॒त ए॒ते लो॒कं ॅय॑न्ति यन्ति लो॒क मे॒त ए॒ते लो॒कं ॅय॑न्ति । \newline
5. लो॒कं ॅय॑न्ति यन्ति लो॒कम् ॅलो॒कं ॅय॑न्ति॒ ये ये य॑न्ति लो॒कम् ॅलो॒कं ॅय॑न्ति॒ ये । \newline
6. य॒न्ति॒ ये ये य॑न्ति यन्ति॒ य उ॑प॒सद॑ उप॒सदो॒ ये य॑न्ति यन्ति॒ य उ॑प॒सदः॑ । \newline
7. य उ॑प॒सद॑ उप॒सदो॒ ये य उ॑प॒सद॑ उप॒यन् त्यु॑प॒यन् त्यु॑प॒सदो॒ ये य उ॑प॒सद॑ उप॒यन्ति॑ । \newline
8. उ॒प॒सद॑ उप॒यन् त्यु॑प॒यन् त्यु॑प॒सद॑ उप॒सद॑ उप॒यन्ति॒ तेषा॒म् तेषा॑ मुप॒यन् त्यु॑प॒सद॑ उप॒सद॑ उप॒यन्ति॒ तेषा᳚म् । \newline
9. उ॒प॒सद॒ इत्यु॑प - सदः॑ । \newline
10. उ॒प॒यन्ति॒ तेषा॒म् तेषा॑ मुप॒यन् त्यु॑प॒यन्ति॒ तेषां॒ ॅयो यस्तेषा॑ मुप॒यन् त्यु॑प॒यन्ति॒ तेषां॒ ॅयः । \newline
11. उ॒प॒यन्तीयु॑प - यन्ति॑ । \newline
12. तेषां॒ ॅयो यस्तेषा॒म् तेषां॒ ॅय उ॒न्नय॑त उ॒न्नय॑ते॒ यस्तेषा॒म् तेषां॒ ॅय उ॒न्नय॑ते । \newline
13. य उ॒न्नय॑त उ॒न्नय॑ते॒ यो य उ॒न्नय॑ते॒ हीय॑ते॒ हीय॑त उ॒न्नय॑ते॒ यो य उ॒न्नय॑ते॒ हीय॑ते । \newline
14. उ॒न्नय॑ते॒ हीय॑ते॒ हीय॑त उ॒न्नय॑त उ॒न्नय॑ते॒ हीय॑त ए॒वैव हीय॑त उ॒न्नय॑त उ॒न्नय॑ते॒ हीय॑त ए॒व । \newline
15. उ॒न्नय॑त॒ इत्यु॑त् - नय॑ते । \newline
16. हीय॑त ए॒वैव हीय॑ते॒ हीय॑त ए॒व स स ए॒व हीय॑ते॒ हीय॑त ए॒व सः । \newline
17. ए॒व स स ए॒वैव स न न स ए॒वैव स न । \newline
18. स न न स स नोदुन् न स स नोत् । \newline
19. नोदुन् न नो द॑नेष्य ने॒ष्युन् न नोद॑नेषि । \newline
20. उद॑ने ष्यने॒ ष्युदु द॑ने॒षीती त्य॑ने॒ष्युदु द॑ने॒षीति॑ । \newline
21. अ॒ने॒षीती त्य॑ने ष्यने॒षीती॒ सू᳚न्नीयꣳ॒॒ सू᳚न्नीय॒ मित्य॑ने ष्यने॒षीती॒ सू᳚न्नीयम् । \newline
22. इती॒ सू᳚न्नीयꣳ॒॒ सू᳚न्नीय॒ मितीती॒ सू᳚न्नीय मिवे॒व सू᳚न्नीय॒ मितीती॒ सू᳚न्नीय मिव । \newline
23. सू᳚न्नीय मिवे॒व सू᳚न्नीयꣳ॒॒ सू᳚न्नीय मिव॒ यो य इ॒व सू᳚न्नीयꣳ॒॒ सू᳚न्नीय मिव॒ यः । \newline
24. सू᳚न्नीय॒मिति॒ सु - उ॒न्नी॒य॒म् । \newline
25. इ॒व॒ यो य इ॑वेव॒ यो वै वै य इ॑वेव॒ यो वै । \newline
26. यो वै वै यो यो वै स्वा॒र्थेताꣳ॑ स्वा॒र्थेतां॒ ॅवै यो यो वै स्वा॒र्थेता᳚म् । \newline
27. वै स्वा॒र्थेताꣳ॑ स्वा॒र्थेतां॒ ॅवै वै स्वा॒र्थेतां᳚ ॅय॒तां ॅय॒ताꣳ स्वा॒र्थेतां॒ ॅवै वै स्वा॒र्थेतां᳚ ॅय॒ताम् । \newline
28. स्वा॒र्थेतां᳚ ॅय॒तां ॅय॒ताꣳ स्वा॒र्थेताꣳ॑ स्वा॒र्थेतां᳚ ॅय॒ताꣳ श्रा॒न्तः श्रा॒न्तो य॒ताꣳ स्वा॒र्थेताꣳ॑ स्वा॒र्थेतां᳚ ॅय॒ताꣳ श्रा॒न्तः । \newline
29. स्वा॒र्थेता॒मिति॑ स्वार्थ - इता᳚म् । \newline
30. य॒ताꣳ श्रा॒न्तः श्रा॒न्तो य॒तां ॅय॒ताꣳ श्रा॒न्तो हीय॑ते॒ हीय॑ते श्रा॒न्तो य॒तां ॅय॒ताꣳ श्रा॒न्तो हीय॑ते । \newline
31. श्रा॒न्तो हीय॑ते॒ हीय॑ते श्रा॒न्तः श्रा॒न्तो हीय॑त उ॒तोत हीय॑ते श्रा॒न्तः श्रा॒न्तो हीय॑त उ॒त । \newline
32. हीय॑त उ॒तोत हीय॑ते॒ हीय॑त उ॒त स स उ॒त हीय॑ते॒ हीय॑त उ॒त सः । \newline
33. उ॒त स स उ॒तोत स नि॒ष्ट्याय॑ नि॒ष्ट्याय॒ स उ॒तोत स नि॒ष्ट्याय॑ । \newline
34. स नि॒ष्ट्याय॑ नि॒ष्ट्याय॒ स स नि॒ष्ट्याय॑ स॒ह स॒ह नि॒ष्ट्याय॒ स स नि॒ष्ट्याय॑ स॒ह । \newline
35. नि॒ष्ट्याय॑ स॒ह स॒ह नि॒ष्ट्याय॑ नि॒ष्ट्याय॑ स॒ह व॑सति वसति स॒ह नि॒ष्ट्याय॑ नि॒ष्ट्याय॑ स॒ह व॑सति । \newline
36. स॒ह व॑सति वसति स॒ह स॒ह व॑सति॒ तस्मा॒त् तस्मा᳚द् वसति स॒ह स॒ह व॑सति॒ तस्मा᳚त् । \newline
37. व॒स॒ति॒ तस्मा॒त् तस्मा᳚द् वसति वसति॒ तस्मा᳚थ् स॒कृथ् स॒कृत् तस्मा᳚द् वसति वसति॒ तस्मा᳚थ् स॒कृत् । \newline
38. तस्मा᳚थ् स॒कृथ् स॒कृत् तस्मा॒त् तस्मा᳚थ् स॒कृ दु॒न्नी यो॒न्नीय॑ स॒कृत् तस्मा॒त् तस्मा᳚थ् स॒कृ दु॒न्नीय॑ । \newline
39. स॒कृ दु॒न्नी यो॒न्नीय॑ स॒कृथ् स॒कृ दु॒न्नीय॒ न नोन्नीय॑ स॒कृथ् स॒कृ दु॒न्नीय॒ न । \newline
40. उ॒न्नीय॒ न नोन्नी यो॒न्नीय॒ नाप॑र॒ मप॑र॒न् नोन्नी यो॒न्नीय॒ नाप॑रम् । \newline
41. उ॒न्नीयेत्यु॑त् - नीय॑ । \newline
42. नाप॑र॒ मप॑र॒म् न नाप॑र॒ मुदु दप॑र॒म् न नाप॑र॒ मुत् । \newline
43. अप॑र॒ मुदु दप॑र॒ मप॑र॒ मुन् न॑येत नये॒तोदप॑र॒ मप॑र॒ मुन् न॑येत । \newline
44. उन् न॑येत नये॒तोदुन् न॑येत द॒द्ध्ना द॒द्ध्ना न॑ये॒ तोदुन् न॑येत द॒द्ध्ना । \newline
45. न॒ये॒त॒ द॒द्ध्ना द॒द्ध्ना न॑येत नयेत द॒द्ध्नोदुद् द॒द्ध्ना न॑येत नयेत द॒द्ध्नोत् । \newline
46. द॒द्ध्नोदुद् द॒द्ध्ना द॒द्ध्नोन् न॑येत नये॒तोद् द॒द्ध्ना द॒द्ध्नोन् न॑येत । \newline
47. उन् न॑येत नये॒तोदुन् न॑ये तै॒त दे॒तन् न॑ये॒तोदुन् न॑येतै॒तत् । \newline
48. न॒ये॒ तै॒त दे॒तन् न॑येत नयेतै॒तद् वै वा ए॒तन् न॑येत नयेतै॒तद् वै । \newline
49. ए॒तद् वै वा ए॒त दे॒तद् वै प॑शू॒नाम् प॑शू॒नां ॅवा ए॒त दे॒तद् वै प॑शू॒नाम् । \newline
50. वै प॑शू॒नाम् प॑शू॒नां ॅवै वै प॑शू॒नाꣳ रू॒पꣳ रू॒पम् प॑शू॒नां ॅवै वै प॑शू॒नाꣳ रू॒पम् । \newline
51. प॒शू॒नाꣳ रू॒पꣳ रू॒पम् प॑शू॒नाम् प॑शू॒नाꣳ रू॒पꣳ रू॒पेण॑ रू॒पेण॑ रू॒पम् प॑शू॒नाम् प॑शू॒नाꣳ रू॒पꣳ रू॒पेण॑ । \newline
52. रू॒पꣳ रू॒पेण॑ रू॒पेण॑ रू॒पꣳ रू॒पꣳ रू॒पेणै॒वैव रू॒पेण॑ रू॒पꣳ रू॒पꣳ रू॒पेणै॒व । \newline
53. रू॒पेणै॒ वैव रू॒पेण॑ रू॒पेणै॒व प॒शून् प॒शूने॒व रू॒पेण॑ रू॒पेणै॒व प॒शून् । \newline
54. ए॒व प॒शून् प॒शूने॒ वैव प॒शून वाव॑ प॒शूने॒ वैव प॒शूनव॑ । \newline
55. प॒शून वाव॑ प॒शून् प॒शूनव॑ रुन्धे रु॒न्धे ऽव॑ प॒शून् प॒शूनव॑ रुन्धे । \newline
56. अव॑ रुन्धे रु॒न्धे ऽवाव॑ रुन्धे य॒ज्ञो य॒ज्ञो रु॒न्धे ऽवाव॑ रुन्धे य॒ज्ञ्ः । \newline
57. रु॒न्धे॒ य॒ज्ञो य॒ज्ञो रु॑न्धे रुन्धे य॒ज्ञो दे॒वेभ्यो॑ दे॒वेभ्यो॑ य॒ज्ञो रु॑न्धे रुन्धे य॒ज्ञो दे॒वेभ्यः॑ । \newline
\pagebreak
\markright{ TS 6.2.4.2  \hfill https://www.vedavms.in \hfill}

\section{ TS 6.2.4.2 }

\textbf{TS 6.2.4.2 } \newline
\textbf{Samhita Paata} \newline

य॒ज्ञो दे॒वेभ्यो॒ निला॑यत॒ विष्णू॑ रू॒पं कृ॒त्वा स पृ॑थि॒वीं प्रावि॑श॒त् तं दे॒वा हस्ता᳚न्थ् सꣳ॒॒ रभ्यै᳚च्छ॒न् तमिन्द्र॑ उ॒पर्यु॑प॒र्यत्य॑क्राम॒थ् सो᳚ऽब्रवी॒त् को मा॒ऽयमु॒पर्यु॑प॒र्यत्य॑क्रमी॒-दित्य॒हं दु॒र्गे हन्तेत्यथ॒ कस्त्वमित्य॒हं दु॒र्गादाह॒र्तेति॒ सो᳚ऽब्रवीद् दु॒र्गे वै हन्ता॑ऽवोचथा वरा॒हो॑ऽयं ॅवा॑ममो॒षः- [  ] \newline

\textbf{Pada Paata} \newline

य॒ज्ञ्ः । दे॒वेभ्यः॑ । निला॑यत । विष्णुः॑ । रू॒पम् । कृ॒त्वा । सः । पृ॒थि॒वीम् । प्रेति॑ । अ॒वि॒श॒त् । तम् । दे॒वाः । हस्तान्॑ । सꣳ॒॒रभ्येति॑ सं - रभ्य॑ । ऐ॒च्छ॒न्न् । तम् । इन्द्रः॑ । उ॒पर्यु॑प॒रीत्यु॒परि॑ - उ॒प॒रि॒ । अतीति॑ । अ॒क्रा॒म॒त् । सः । अ॒ब्र॒वी॒त् । कः । मा॒ । अ॒यम् । उ॒पर्यु॑प॒रीत्यु॒परि॑ - उ॒प॒रि॒ । अतीति॑ । अ॒क्र॒मी॒त् । इति॑ । अ॒हम् । दु॒र्ग इति॑ दुः - गे । हन्ता᳚ । इति॑ । अथ॑ । कः । त्वम् । इति॑ । अ॒हम् । दु॒र्गादिति॑ दुः - गात् । आह॒र्तेत्या - ह॒र्ता॒ । इति॑ । सः । अ॒ब्र॒वी॒त् । दु॒र्ग इति॑ दुः - गे । वै । हन्ता᳚ । अ॒वो॒च॒थाः॒ । व॒रा॒हः । अ॒यम् । वा॒म॒मो॒ष इति॑ वाम - मो॒षः ।  \newline


\textbf{Krama Paata} \newline

य॒ज्ञो दे॒वेभ्यः॑ । दे॒वेभ्यो॒ निला॑यत । निला॑यत॒ विष्णुः॑ । विष्णू॑ रू॒पम् । रू॒पम् कृ॒त्वा । कृ॒त्वा सः । स पृ॑थि॒वीम् । पृ॒थि॒वीम् प्र । प्रावि॑शत् । अ॒वि॒श॒त् तम् । तम् दे॒वाः । दे॒वा हस्तान्॑ । हस्ता᳚न्थ् सꣳ॒॒रभ्य॑ । सꣳ॒॒रभ्यै᳚च्छन्न् । सꣳ॒॒रभ्येति॑ सम् - रभ्य॑ । ऐ॒च्छ॒न् तम् । तमिन्द्रः॑ । इन्द्र॑ उ॒पर्यु॑परि । उ॒पर्यु॑प॒र्यति॑ । उ॒पर्यु॑प॒रीत्यु॒परि॑ - उ॒प॒रि॒ । अत्य॑क्रामत् । अ॒क्रा॒म॒थ् सः । सो᳚ऽब्रवीत् । अ॒ब्र॒वी॒त् कः । को मा᳚ । मा॒ऽयम् । अ॒यमु॒पर्यु॑परि । उ॒पर्यु॑प॒र्यति॑ । उ॒पर्यु॑प॒रीत्यु॒परि॑ - उ॒प॒रि॒ । अत्य॑क्रमीत् । अ॒क्र॒मी॒दिति॑ । इत्य॒हम् । अ॒हम् दु॒र्गे । दु॒र्गे हन्ता᳚ । दु॒र्ग इति॑ दुः - गे । हन्तेति॑ । इत्यथ॑ । अथ॒ कः । कस्त्वम् । त्वमिति॑ । इत्य॒हम् । अ॒हम् दु॒र्गात् । दु॒र्गादाह॑र्ता । दु॒र्गादिति॑ दुः - गात् । आह॒र्तेति॑ । आह॒र्तेत्या - ह॒र्ता॒ । इति॒ सः । सो᳚ऽब्रवीत् । अ॒ब्र॒वी॒द् दु॒र्गे । दु॒र्गे वै । दु॒र्ग इति॑ दुः - गे । वै हन्ता᳚ । हन्ता॑ऽवोचथाः । अ॒वो॒च॒था॒ व॒रा॒हः । व॒रा॒हो॑ऽयम् । अ॒यम् ॅवा॑ममो॒षः । वा॒म॒मो॒षः स॑प्ता॒नाम् । वा॒म॒मो॒ष इति॑ वाम - मो॒षः \newline

\textbf{Jatai Paata} \newline

1. य॒ज्ञो दे॒वेभ्यो॑ दे॒वेभ्यो॑ य॒ज्ञो य॒ज्ञो दे॒वेभ्यः॑ । \newline
2. दे॒वेभ्यो॒ निला॑यत॒ निला॑यत दे॒वेभ्यो॑ दे॒वेभ्यो॒ निला॑यत । \newline
3. निला॑यत॒ विष्णु॒र् विष्णु॒र् निला॑यत॒ निला॑यत॒ विष्णुः॑ । \newline
4. विष्णू॑ रू॒पꣳ रू॒पं ॅविष्णु॒र् विष्णू॑ रू॒पम् । \newline
5. रू॒पम् कृ॒त्वा कृ॒त्वा रू॒पꣳ रू॒पम् कृ॒त्वा । \newline
6. कृ॒त्वा स स कृ॒त्वा कृ॒त्वा सः । \newline
7. स पृ॑थि॒वीम् पृ॑थि॒वीꣳ स स पृ॑थि॒वीम् । \newline
8. पृ॒थि॒वीम् प्र प्र पृ॑थि॒वीम् पृ॑थि॒वीम् प्र । \newline
9. प्रावि॑श दविश॒त् प्र प्रावि॑शत् । \newline
10. अ॒वि॒श॒त् तम् त म॑विश दविश॒त् तम् । \newline
11. तम् दे॒वा दे॒वा स्तम् तम् दे॒वाः । \newline
12. दे॒वा हस्ता॒न्॒. हस्ता᳚न् दे॒वा दे॒वा हस्तान्॑ । \newline
13. हस्ता᳚न् थ्सꣳ॒॒रभ्य॑ सꣳ॒॒रभ्य॒ हस्ता॒न्॒. हस्ता᳚न् थ्सꣳ॒॒रभ्य॑ । \newline
14. सꣳ॒॒रभ्यै᳚च्छन् नैच्छन् थ्सꣳ॒॒रभ्य॑ सꣳ॒॒रभ्यै᳚च्छन्न् । \newline
15. सꣳ॒॒रभ्येति॑ सं - रभ्य॑ । \newline
16. ऐ॒च्छ॒न् तम् त मै᳚च्छन् नैच्छ॒न् तम् । \newline
17. त मिन्द्र॒ इन्द्र॒ स्तम् त मिन्द्रः॑ । \newline
18. इन्द्र॑ उ॒पर्यु॑पर् यु॒पर्यु॑प॒ रीन्द्र॒ इन्द्र॑ उ॒पर्यु॑परि । \newline
19. उ॒पर्यु॑प॒र् यत्य त्यु॒पर्यु॑पर् यु॒पर्यु॑प॒र् यति॑ । \newline
20. उ॒पर्यु॑प॒रीत्यु॒परि॑ - उ॒प॒रि॒ । \newline
21. अत्य॑क्राम दक्राम॒ दत्य त्य॑क्रामत् । \newline
22. अ॒क्रा॒म॒थ् स सो᳚ ऽक्राम दक्राम॒थ् सः । \newline
23. सो᳚ ऽब्रवी दब्रवी॒थ् स सो᳚ ऽब्रवीत् । \newline
24. अ॒ब्र॒वी॒त् कः को᳚ ऽब्रवी दब्रवी॒त् कः । \newline
25. को मा॑ मा॒ कः को मा᳚ । \newline
26. मा॒ ऽय म॒यम् मा॑ मा॒ ऽयम् । \newline
27. अ॒य मु॒पर्यु॑पर् यु॒पर्यु॑पर् य॒य म॒य मु॒पर्यु॑परि । \newline
28. उ॒पर्यु॑प॒र् यत्य त्यु॒पर्यु॑पर् यु॒पर्यु॑प॒र् यति॑ । \newline
29. उ॒पर्यु॑प॒रीत्यु॒परि॑ - उ॒प॒रि॒ । \newline
30. अत्य॑क्रमी दक्रमी॒ दत्य त्य॑क्रमीत् । \newline
31. अ॒क्र॒मी॒ दिती त्य॑क्रमी दक्रमी॒ दिति॑ । \newline
32. इत्य॒ह म॒ह मिती त्य॒हम् । \newline
33. अ॒हम् दु॒र्गे दु॒र्गे॑ ऽह म॒हम् दु॒र्गे । \newline
34. दु॒र्गे हन्ता॒ हन्ता॑ दु॒र्गे दु॒र्गे हन्ता᳚ । \newline
35. दु॒र्ग इति॑ दुः - गे । \newline
36. हन्तेतीति॒ हन्ता॒ हन्तेति॑ । \newline
37. इत्यथाथे तीत्यथ॑ । \newline
38. अथ॒ कः को ऽथाथ॒ कः । \newline
39. क स्त्वम् त्वम् कः क स्त्वम् । \newline
40. त्व मितीति॒ त्वम् त्व मिति॑ । \newline
41. इत्य॒ह म॒ह मिती त्य॒हम् । \newline
42. अ॒हम् दु॒र्गाद् दु॒र्गा द॒ह म॒हम् दु॒र्गात् । \newline
43. दु॒र्गा दाह॒र्ता ऽऽह॑र्ता दु॒र्गाद् दु॒र्गा दाह॑र्ता । \newline
44. दु॒र्गादिति॑ दुः - गात् । \newline
45. आह॒र्तेती त्याह॒र्ता ऽऽह॒र्तेति॑ । \newline
46. आह॒र्तेत्या - ह॒र्ता॒ । \newline
47. इति॒ स स इतीति॒ सः । \newline
48. सो᳚ ऽब्रवी दब्रवी॒थ् स सो᳚ ऽब्रवीत् । \newline
49. अ॒ब्र॒वी॒द् दु॒र्गे दु॒र्गे᳚ ऽब्रवी दब्रवीद् दु॒र्गे । \newline
50. दु॒र्गे वै वै दु॒र्गे दु॒र्गे वै । \newline
51. दु॒र्ग इति॑ दुः - गे । \newline
52. वै हन्ता॒ हन्ता॒ वै वै हन्ता᳚ । \newline
53. हन्ता॑ ऽवोचथा अवोचथा॒ हन्ता॒ हन्ता॑ ऽवोचथाः । \newline
54. अ॒वो॒च॒था॒ व॒रा॒हो व॑रा॒हो॑ ऽवोचथा अवोचथा वरा॒हः । \newline
55. व॒रा॒हो॑ ऽय म॒यं ॅव॑रा॒हो व॑रा॒हो॑ ऽयम् । \newline
56. अ॒यं ॅवा॑ममो॒षो वा॑ममो॒षो॑ ऽय म॒यं ॅवा॑ममो॒षः । \newline
57. वा॒म॒मो॒षः स॑प्ता॒नाꣳ स॑प्ता॒नां ॅवा॑ममो॒षो वा॑ममो॒षः स॑प्ता॒नाम् । \newline
58. वा॒म॒मो॒ष इति॑ वाम - मो॒षः । \newline

\textbf{Ghana Paata } \newline

1. य॒ज्ञो दे॒वेभ्यो॑ दे॒वेभ्यो॑ य॒ज्ञो य॒ज्ञो दे॒वेभ्यो॒ निला॑यत॒ निला॑यत दे॒वेभ्यो॑ य॒ज्ञो य॒ज्ञो दे॒वेभ्यो॒ निला॑यत । \newline
2. दे॒वेभ्यो॒ निला॑यत॒ निला॑यत दे॒वेभ्यो॑ दे॒वेभ्यो॒ निला॑यत॒ विष्णु॒र् विष्णु॒र् निला॑यत दे॒वेभ्यो॑ दे॒वेभ्यो॒ निला॑यत॒ विष्णुः॑ । \newline
3. निला॑यत॒ विष्णु॒र् विष्णु॒र् निला॑यत॒ निला॑यत॒ विष्णू॑ रू॒पꣳ रू॒पं ॅविष्णु॒र् निला॑यत॒ निला॑यत॒ विष्णू॑ रू॒पम् । \newline
4. विष्णू॑ रू॒पꣳ रू॒पं ॅविष्णु॒र् विष्णू॑ रू॒पम् कृ॒त्वा कृ॒त्वा रू॒पं ॅविष्णु॒र् विष्णू॑ रू॒पम् कृ॒त्वा । \newline
5. रू॒पम् कृ॒त्वा कृ॒त्वा रू॒पꣳ रू॒पम् कृ॒त्वा स स कृ॒त्वा रू॒पꣳ रू॒पम् कृ॒त्वा सः । \newline
6. कृ॒त्वा स स कृ॒त्वा कृ॒त्वा स पृ॑थि॒वीम् पृ॑थि॒वीꣳ स कृ॒त्वा कृ॒त्वा स पृ॑थि॒वीम् । \newline
7. स पृ॑थि॒वीम् पृ॑थि॒वीꣳ स स पृ॑थि॒वीम् प्र प्र पृ॑थि॒वीꣳ स स पृ॑थि॒वीम् प्र । \newline
8. पृ॒थि॒वीम् प्र प्र पृ॑थि॒वीम् पृ॑थि॒वीम् प्रावि॑श दविश॒त् प्र पृ॑थि॒वीम् पृ॑थि॒वीम् प्रावि॑शत् । \newline
9. प्रावि॑श दविश॒त् प्र प्रावि॑श॒त् तम् त म॑विश॒त् प्र प्रावि॑श॒त् तम् । \newline
10. अ॒वि॒श॒त् तम् त म॑विश दविश॒त् तम् दे॒वा दे॒वा स्त म॑विश दविश॒त् तम् दे॒वाः । \newline
11. तम् दे॒वा दे॒वा स्तम् तम् दे॒वा हस्ता॒न्॒. हस्ता᳚न् दे॒वा स्तम् तम् दे॒वा हस्तान्॑ । \newline
12. दे॒वा हस्ता॒न्॒. हस्ता᳚न् दे॒वा दे॒वा हस्ता᳚न् थ्सꣳ॒॒रभ्य॑ सꣳ॒॒रभ्य॒ हस्ता᳚न् दे॒वा दे॒वा हस्ता᳚न् थ्सꣳ॒॒रभ्य॑ । \newline
13. हस्ता᳚न् थ्सꣳ॒॒रभ्य॑ सꣳ॒॒रभ्य॒ हस्ता॒न्॒. हस्ता᳚न् थ्सꣳ॒॒र भ्यै᳚च्छन् नैच्छन् थ्सꣳ॒॒रभ्य॒ हस्ता॒न्॒. हस्ता᳚न् थ्सꣳ॒॒र भ्यै᳚च्छन्न् । \newline
14. सꣳ॒॒रभ्यै᳚च्छन् नैच्छन् थ्सꣳ॒॒रभ्य॑ सꣳ॒॒रभ्यै᳚च्छ॒न् तम् त मै᳚च्छन् थ्सꣳ॒॒रभ्य॑ सꣳ॒॒रभ्यै᳚च्छ॒न् तम् । \newline
15. सꣳ॒॒रभ्येति॑ सं - रभ्य॑ । \newline
16. ऐ॒च्छ॒न् तम् त मै᳚च्छन् नैच्छ॒न् त मिन्द्र॒ इन्द्र॒ स्त मै᳚च्छन् नैच्छ॒न् त मिन्द्रः॑ । \newline
17. त मिन्द्र॒ इन्द्र॒ स्तम् त मिन्द्र॑ उ॒पर्यु॑पर् यु॒पर्यु॑प॒ रीन्द्र॒स्तम् त मिन्द्र॑ उ॒पर्यु॑परि । \newline
18. इन्द्र॑ उ॒पर्यु॑पर् यु॒पर्यु॑प॒रीन्द्र॒ इन्द्र॑ उ॒पर्यु॑प॒र् यत्य त्यु॒पर्यु॑प॒रीन्द्र॒ इन्द्र॑ उ॒पर्यु॑प॒र् यति॑ । \newline
19. उ॒पर्यु॑प॒र् यत्य त्यु॒पर्यु॑पर् यु॒पर्यु॑प॒र् यत्य॑क्राम दक्राम॒ दत्यु॒पर्यु॑पर् यु॒पर्यु॑प॒र् यत्य॑क्रामत् । \newline
20. उ॒पर्यु॑प॒रीत्यु॒परि॑ - उ॒प॒रि॒ । \newline
21. अत्य॑क्राम दक्राम॒ दत्य त्य॑क्राम॒थ् स सो᳚ ऽक्राम॒ दत्य त्य॑क्राम॒थ् सः । \newline
22. अ॒क्रा॒म॒थ् स सो᳚ ऽक्राम दक्राम॒थ् सो᳚ ऽब्रवी दब्रवी॒थ् सो᳚ ऽक्राम दक्राम॒थ् सो᳚ ऽब्रवीत् । \newline
23. सो᳚ ऽब्रवी दब्रवी॒थ् स सो᳚ ऽब्रवी॒त् कः को᳚ ऽब्रवी॒थ् स सो᳚ ऽब्रवी॒त् कः । \newline
24. अ॒ब्र॒वी॒त् कः को᳚ ऽब्रवी दब्रवी॒त् को मा॑ मा॒ को᳚ ऽब्रवी दब्रवी॒त् को मा᳚ । \newline
25. को मा॑ मा॒ कः को मा॒ ऽय म॒यम् मा॒ कः को मा॒ ऽयम् । \newline
26. मा॒ ऽय म॒यम् मा॑ मा॒ ऽय मु॒पर्यु॑पर् यु॒पर्यु॑पर् य॒यम् मा॑ मा॒ ऽय मु॒पर्यु॑परि । \newline
27. अ॒य मु॒पर्यु॑पर् यु॒पर्यु॑पर् य॒य म॒य मु॒पर्यु॑प॒र् यत्य त्यु॒पर्यु॑पर् य॒य म॒य मु॒पर्यु॑प॒र्यति॑ । \newline
28. उ॒पर्यु॑प॒र् यत्य त्यु॒पर्यु॑पर् यु॒पर्यु॑प॒र् यत्य॑क्रमी दक्रमी॒ दत्यु॒पर्यु॑पर् यु॒पर्यु॑प॒र् यत्य॑क्रमीत् । \newline
29. उ॒पर्यु॑प॒रीत्यु॒परि॑ - उ॒प॒रि॒ । \newline
30. अत्य॑ क्रमी दक्रमी॒ दत्य त्य॑क्रमी॒दि तीत्य॑क्रमी॒ दत्य त्य॑क्रमी॒ दिति॑ । \newline
31. अ॒क्र॒मी॒ दिती त्य॑क्रमी दक्रमी॒ दित्य॒ह म॒ह मित्य॑क्रमी दक्रमी॒ दित्य॒हम् । \newline
32. इत्य॒ह म॒ह मिती त्य॒हम् दु॒र्गे दु॒र्गे॑ ऽह मिती त्य॒हम् दु॒र्गे । \newline
33. अ॒हम् दु॒र्गे दु॒र्गे॑ ऽह म॒हम् दु॒र्गे हन्ता॒ हन्ता॑ दु॒र्गे॑ ऽह म॒हम् दु॒र्गे हन्ता᳚ । \newline
34. दु॒र्गे हन्ता॒ हन्ता॑ दु॒र्गे दु॒र्गे हन्तेतीति॒ हन्ता॑ दु॒र्गे दु॒र्गे हन्तेति॑ । \newline
35. दु॒र्ग इति॑ दुः - गे । \newline
36. हन्ते तीति॒ हन्ता॒ हन्ते त्यथाथेति॒ हन्ता॒ हन्ते त्यथ॑ । \newline
37. इत्यथाथे तीत्यथ॒ कः को ऽथे तीत्यथ॒ कः । \newline
38. अथ॒ कः को ऽथाथ॒ क स्त्वम् त्वम् को ऽथाथ॒ क स्त्वम् । \newline
39. क स्त्वम् त्वम् कः क स्त्व मितीति॒ त्वम् कः क स्त्व मिति॑ । \newline
40. त्व मितीति॒ त्वम् त्व मित्य॒ह म॒ह मिति॒ त्वम् त्व मित्य॒हम् । \newline
41. इत्य॒ह म॒ह मिती त्य॒हम् दु॒र्गाद् दु॒र्गा द॒ह मिती त्य॒हम् दु॒र्गात् । \newline
42. अ॒हम् दु॒र्गाद् दु॒र्गा द॒ह म॒हम् दु॒र्गा दाह॒र्ता ऽऽह॑र्ता दु॒र्गा द॒ह म॒हम् दु॒र्गा दाह॑र्ता । \newline
43. दु॒र्गा दाह॒र्ता ऽऽह॑र्ता दु॒र्गाद् दु॒र्गा दाह॒र्तेती त्याह॑र्ता दु॒र्गाद् दु॒र्गा दाह॒र्तेति॑ । \newline
44. दु॒र्गादिति॑ दुः - गात् । \newline
45. आह॒र्ते तीत्या ह॒र्ता ऽऽह॒र्तेति॒ स स इत्या ह॒र्ता ऽऽह॒र्तेति॒ सः । \newline
46. आह॒र्तेत्या - ह॒र्ता॒ । \newline
47. इति॒ स स इतीति॒ सो᳚ ऽब्रवी दब्रवी॒थ् स इतीति॒ सो᳚ ऽब्रवीत् । \newline
48. सो᳚ ऽब्रवी दब्रवी॒थ् स सो᳚ ऽब्रवीद् दु॒र्गे दु॒र्गे᳚ ऽब्रवी॒थ् स सो᳚ ऽब्रवीद् दु॒र्गे । \newline
49. अ॒ब्र॒वी॒द् दु॒र्गे दु॒र्गे᳚ ऽब्रवी दब्रवीद् दु॒र्गे वै वै दु॒र्गे᳚ ऽब्रवी दब्रवीद् दु॒र्गे वै । \newline
50. दु॒र्गे वै वै दु॒र्गे दु॒र्गे वै हन्ता॒ हन्ता॒ वै दु॒र्गे दु॒र्गे वै हन्ता᳚ । \newline
51. दु॒र्ग इति॑ दुः - गे । \newline
52. वै हन्ता॒ हन्ता॒ वै वै हन्ता॑ ऽवोचथा अवोचथा॒ हन्ता॒ वै वै हन्ता॑ ऽवोचथाः । \newline
53. हन्ता॑ ऽवोचथा अवोचथा॒ हन्ता॒ हन्ता॑ ऽवोचथा वरा॒हो व॑रा॒हो॑ ऽवोचथा॒ हन्ता॒ हन्ता॑ ऽवोचथा वरा॒हः । \newline
54. अ॒वो॒च॒था॒ व॒रा॒हो व॑रा॒हो॑ ऽवोचथा अवोचथा वरा॒हो॑ ऽय म॒यं ॅव॑रा॒हो॑ ऽवोचथा अवोचथा वरा॒हो॑ ऽयम् । \newline
55. व॒रा॒हो॑ ऽय म॒यं ॅव॑रा॒हो व॑रा॒हो॑ ऽयं ॅवा॑ममो॒षो वा॑ममो॒षो॑ ऽयं ॅव॑रा॒हो व॑रा॒हो॑ ऽयं ॅवा॑ममो॒षः । \newline
56. अ॒यं ॅवा॑ममो॒षो वा॑ममो॒षो॑ ऽय म॒यं ॅवा॑ममो॒षः स॑प्ता॒नाꣳ स॑प्ता॒नां ॅवा॑ममो॒षो॑ ऽय म॒यं ॅवा॑ममो॒षः स॑प्ता॒नाम् । \newline
57. वा॒म॒मो॒षः स॑प्ता॒नाꣳ स॑प्ता॒नां ॅवा॑ममो॒षो वा॑ममो॒षः स॑प्ता॒नाम् गि॑री॒णाम् गि॑री॒णाꣳ स॑प्ता॒नां ॅवा॑ममो॒षो वा॑ममो॒षः स॑प्ता॒नाम् गि॑री॒णाम् । \newline
58. वा॒म॒मो॒ष इति॑ वाम - मो॒षः । \newline
\pagebreak
\markright{ TS 6.2.4.3  \hfill https://www.vedavms.in \hfill}

\section{ TS 6.2.4.3 }

\textbf{TS 6.2.4.3 } \newline
\textbf{Samhita Paata} \newline

स॑प्ता॒नां गि॑री॒णां प॒रस्ता᳚द्वि॒त्तं ॅवेद्य॒मसु॑राणां बिभर्ति॒ तं ज॑हि॒ यदि॑ दु॒र्गे हन्ताऽसीति॒ स द॑र्भपुञ्जी॒लमु॒द्-वृह्य॑ स॒प्त गि॒रीन् भि॒त्त्वा तम॑ह॒न्थ् सो᳚ऽब्रवीद् दु॒र्गाद्वा आह॑र्तावोचथा ए॒तमा ह॒रेति॒ तमे᳚भ्यो य॒ज्ञ् ए॒व य॒ज्ञ्माऽह॑र॒द्यत् तद्वि॒त्तं ॅवेद्य॒मसु॑राणा॒-मवि॑न्दन्त॒ तदेकं॒ ॅवेद्यै॑ वेदि॒त्वमसु॑राणां॒- [  ] \newline

\textbf{Pada Paata} \newline

स॒प्ता॒नाम् । गि॒री॒णाम् । प॒रस्ता᳚त् । वि॒त्तम् । वेद्य᳚म् । असु॑राणाम् । बि॒भ॒र्ति॒ । तम् । ज॒हि॒ । यदि॑ । दु॒र्ग इति॑ दुः - गे । हन्ता᳚ । असि॑ । इति॑ । सः । द॒र्भ॒पु॒ञ्जी॒लमिति॑ दर्भ - पु॒ञ्जी॒लम् । उ॒द्वृह्येत्यु॑त् - वृह्य॑ । स॒प्त । गि॒रीन् । भि॒त्वा । तम् । अ॒ह॒न्न् । सः । अ॒ब्र॒वी॒त् । दु॒र्गादिति॑ दुः - गात् । वै । आह॒र्तेत्या - ह॒र्ता॒ । अ॒वो॒च॒थाः॒ । ए॒तम् । एति॑ । ह॒र॒ । इति॑ । तम् । ए॒भ्यः॒ । य॒ज्ञ्ः । ए॒व । य॒ज्ञ्म् । एति॑ । अ॒ह॒र॒त् । यत् । तत् । वि॒त्तम् । वेद्य᳚म् । असु॑राणाम् । अवि॑न्दन्त । तत् । एक᳚म् । वेद्यै᳚ । वे॒दि॒त्वमिति॑ वेदि - त्वम् । असु॑राणाम् ।  \newline


\textbf{Krama Paata} \newline

स॒प्ता॒नाम् गि॑री॒णाम् । गि॒री॒णाम् प॒रस्ता᳚त् । प॒रस्ता᳚द् वि॒त्तम् । वि॒त्तम् ॅवेद्य᳚म् । वेद्य॒मसु॑राणाम् । असु॑राणाम् बिभर्ति । बि॒भ॒र्ति॒ तम् । तम् ज॑हि । ज॒हि॒ यदि॑ । यदि॑ दु॒र्गे । दु॒र्गे हन्ता᳚ । दु॒र्ग इति॑ दुः - गे । हन्ताऽसि॑ । असीति॑ । इति॒ सः । स द॑र्भपुञ्जी॒लम् । द॒र्भ॒पु॒ञ्जी॒लमु॒द्‍वृह्य॑ । द॒र्भ॒पु॒ञ्जी॒लमिति॑ दर्भ - पु॒ञ्जी॒लम् । उ॒द्‍वृह्य॑ स॒प्त । उ॒द्‍वृह्येत्यु॑त् - वृह्य॑ । स॒प्त गि॒रीन् । गि॒रीन् भि॒त्वा । भि॒त्वा तम् । तम॑हन्न् । अ॒ह॒न्थ् सः । सो᳚ऽब्रवीत् । अ॒ब्र॒वी॒द् दु॒र्गात् । दु॒र्गाद् वै । दु॒र्गादिति॑ दुः - गात् । वा आह॑र्ता । आह॑र्ताऽवोचथाः । आह॒र्तेत्या - ह॒र्ता॒ । अ॒वो॒च॒था॒ ए॒तम् । ए॒तमा । आ ह॑र । ह॒रेति॑ । इति॒ तम् । तमे᳚भ्यः । ए॒भ्यो॒ य॒ज्ञ्ः । य॒ज्ञ् ए॒व । ए॒व य॒ज्ञ्म् । य॒ज्ञ्मा । आऽह॑रत् । अ॒ह॒र॒द् यत् । यत् तत् । तद् वि॒त्तम् । वि॒त्तम् ॅवेद्य᳚म् । वेद्य॒मसु॑राणाम् । असु॑राणा॒मवि॑न्दन्त । अवि॑न्दन्त॒ तत् । तदेक᳚म् । एक॒म् ॅवेद्यै᳚ । वेद्यै॑ वेदि॒त्वम् । वे॒दि॒त्वमसु॑राणाम् । वे॒दि॒त्वमिति॑ वेदि - त्वम् । असु॑राणा॒म् ॅवै \newline

\textbf{Jatai Paata} \newline

1. स॒प्ता॒नाम् गि॑री॒णाम् गि॑री॒णाꣳ स॑प्ता॒नाꣳ स॑प्ता॒नाम् गि॑री॒णाम् । \newline
2. गि॒री॒णाम् प॒रस्ता᳚त् प॒रस्ता᳚द् गिरी॒णाम् गि॑री॒णाम् प॒रस्ता᳚त् । \newline
3. प॒रस्ता᳚द् वि॒त्तं ॅवि॒त्तम् प॒रस्ता᳚त् प॒रस्ता᳚द् वि॒त्तम् । \newline
4. वि॒त्तं ॅवेद्यं॒ ॅवेद्यं॑ ॅवि॒त्तं ॅवि॒त्तं ॅवेद्य᳚म् । \newline
5. वेद्य॒ मसु॑राणा॒ मसु॑राणां॒ ॅवेद्यं॒ ॅवेद्य॒ मसु॑राणाम् । \newline
6. असु॑राणाम् बिभर्ति बिभ॒र् त्यसु॑राणा॒ मसु॑राणाम् बिभर्ति । \newline
7. बि॒भ॒र्ति॒ तम् तम् बि॑भर्ति बिभर्ति॒ तम् । \newline
8. तम् ज॑हि जहि॒ तम् तम् ज॑हि । \newline
9. ज॒हि॒ यदि॒ यदि॑ जहि जहि॒ यदि॑ । \newline
10. यदि॑ दु॒र्गे दु॒र्गे यदि॒ यदि॑ दु॒र्गे । \newline
11. दु॒र्गे हन्ता॒ हन्ता॑ दु॒र्गे दु॒र्गे हन्ता᳚ । \newline
12. दु॒र्ग इति॑ दुः - गे । \newline
13. हन्ता ऽस्यसि॒ हन्ता॒ हन्ता ऽसि॑ । \newline
14. असीती त्यस्यसीति॑ । \newline
15. इति॒ स स इतीति॒ सः । \newline
16. स द॑र्भपुञ्जी॒लम् द॑र्भपुञ्जी॒लꣳ स स द॑र्भपुञ्जी॒लम् । \newline
17. द॒र्भ॒पु॒ञ्जी॒ल मु॒द्‍वृ ह्यो॒द्‍वृह्य॑ दर्भपुञ्जी॒लम् द॑र्भपुञ्जी॒ल मु॒द्‌वृह्य॑ । \newline
18. द॒र्भ॒पु॒ञ्जी॒लमिति॑ दर्भ - पु॒ञ्जी॒लम् । \newline
19. उ॒द्‌वृह्य॑ स॒प्त स॒प्तोद्‌वृ ह्यो॒द्‌वृह्य॑ स॒प्त । \newline
20. उ॒द्‌वृह्येत्यु॑त् - वृह्य॑ । \newline
21. स॒प्त गि॒रीन् गि॒रीन् थ्स॒प्त स॒प्त गि॒रीन् । \newline
22. गि॒रीन् भि॒त्वा भि॒त्वा गि॒रीन् गि॒रीन् भि॒त्वा । \newline
23. भि॒त्वा तम् तम् भि॒त्वा भि॒त्वा तम् । \newline
24. त म॑हन् नह॒न् तम् त म॑हन्न् । \newline
25. अ॒ह॒न् थ्स सो॑ ऽहन् नह॒न् थ्सः । \newline
26. सो᳚ ऽब्रवी दब्रवी॒थ् स सो᳚ ऽब्रवीत् । \newline
27. अ॒ब्र॒वी॒द् दु॒र्गाद् दु॒र्गा द॑ब्रवी दब्रवीद् दु॒र्गात् । \newline
28. दु॒र्गाद् वै वै दु॒र्गाद् दु॒र्गाद् वै । \newline
29. दु॒र्गादिति॑ दुः - गात् । \newline
30. वा आह॒र्ता ऽऽह॑र्ता॒ वै वा आह॑र्ता । \newline
31. आह॑र्ता ऽवोचथा अवोचथा॒ आह॒र्ता ऽऽह॑र्ता ऽवोचथाः । \newline
32. आह॒र्तेत्या - ह॒र्ता॒ । \newline
33. अ॒वो॒च॒था॒ ए॒त मे॒त म॑वोचथा अवोचथा ए॒तम् । \newline
34. ए॒त मैत मे॒त मा । \newline
35. आ ह॑र ह॒रा ह॑र । \newline
36. ह॒रे तीति॑ हर ह॒रेति॑ । \newline
37. इति॒ तम् त मितीति॒ तम् । \newline
38. त मे᳚भ्य एभ्य॒ स्तम् त मे᳚भ्यः । \newline
39. ए॒भ्यो॒ य॒ज्ञो य॒ज्ञ् ए᳚भ्य एभ्यो य॒ज्ञ्ः । \newline
40. य॒ज्ञ् ए॒वैव य॒ज्ञो य॒ज्ञ् ए॒व । \newline
41. ए॒व य॒ज्ञ्ं ॅय॒ज्ञ् मे॒वैव य॒ज्ञ्म् । \newline
42. य॒ज्ञ् मा य॒ज्ञ्ं ॅय॒ज्ञ् मा । \newline
43. आ ऽह॑र दहर॒दा ऽह॑रत् । \newline
44. अ॒ह॒र॒द् यद् यद॑हर दहर॒द् यत् । \newline
45. यत् तत् तद् यद् यत् तत् । \newline
46. तद् वि॒त्तं ॅवि॒त्तम् तत् तद् वि॒त्तम् । \newline
47. वि॒त्तं ॅवेद्यं॒ ॅवेद्यं॑ ॅवि॒त्तं ॅवि॒त्तं ॅवेद्य᳚म् । \newline
48. वेद्य॒ मसु॑राणा॒ मसु॑राणां॒ ॅवेद्यं॒ ॅवेद्य॒ मसु॑राणाम् । \newline
49. असु॑राणा॒ मवि॑न्द॒न्ता वि॑न्द॒न्ता सु॑राणा॒ मसु॑राणा॒ मवि॑न्दन्त । \newline
50. अवि॑न्दन्त॒ तत् तदवि॑न्द॒न्ता वि॑न्दन्त॒ तत् । \newline
51. तदेक॒ मेक॒म् तत् तदेक᳚म् । \newline
52. एकं॒ ॅवेद्यै॒ वेद्या॒ एक॒ मेकं॒ ॅवेद्यै᳚ । \newline
53. वेद्यै॑ वेदि॒त्वं ॅवे॑दि॒त्वं ॅवेद्यै॒ वेद्यै॑ वेदि॒त्वम् । \newline
54. वे॒दि॒त्व मसु॑राणा॒ मसु॑राणां ॅवेदि॒त्वं ॅवे॑दि॒त्व मसु॑राणाम् । \newline
55. वे॒दि॒त्वमिति॑ वेदि - त्वम् । \newline
56. असु॑राणां॒ ॅवै वा असु॑राणा॒ मसु॑राणां॒ ॅवै । \newline

\textbf{Ghana Paata } \newline

1. स॒प्ता॒नाम् गि॑री॒णाम् गि॑री॒णाꣳ स॑प्ता॒नाꣳ स॑प्ता॒नाम् गि॑री॒णाम् प॒रस्ता᳚त् प॒रस्ता᳚द् गिरी॒णाꣳ स॑प्ता॒नाꣳ स॑प्ता॒नाम् गि॑री॒णाम् प॒रस्ता᳚त् । \newline
2. गि॒री॒णाम् प॒रस्ता᳚त् प॒रस्ता᳚द् गिरी॒णाम् गि॑री॒णाम् प॒रस्ता᳚द् वि॒त्तं ॅवि॒त्तम् प॒रस्ता᳚द् गिरी॒णाम् गि॑री॒णाम् प॒रस्ता᳚द् वि॒त्तम् । \newline
3. प॒रस्ता᳚द् वि॒त्तं ॅवि॒त्तम् प॒रस्ता᳚त् प॒रस्ता᳚द् वि॒त्तं ॅवेद्यं॒ ॅवेद्यं॑ ॅवि॒त्तम् प॒रस्ता᳚त् प॒रस्ता᳚द् वि॒त्तं ॅवेद्य᳚म् । \newline
4. वि॒त्तं ॅवेद्यं॒ ॅवेद्यं॑ ॅवि॒त्तं ॅवि॒त्तं ॅवेद्य॒ मसु॑राणा॒ मसु॑राणां॒ ॅवेद्यं॑ ॅवि॒त्तं ॅवि॒त्तं ॅवेद्य॒ मसु॑राणाम् । \newline
5. वेद्य॒ मसु॑राणा॒ मसु॑राणां॒ ॅवेद्यं॒ ॅवेद्य॒ मसु॑राणाम् बिभर्ति बिभ॒र् त्यसु॑राणां॒ ॅवेद्यं॒ ॅवेद्य॒ मसु॑राणाम् बिभर्ति । \newline
6. असु॑राणाम् बिभर्ति बिभ॒र् त्यसु॑राणा॒ मसु॑राणाम् बिभर्ति॒ तम् तम् बि॑भ॒र्त्य सु॑राणा॒ मसु॑राणाम् बिभर्ति॒ तम् । \newline
7. बि॒भ॒र्ति॒ तम् तम् बि॑भर्ति बिभर्ति॒ तम् ज॑हि जहि॒ तम् बि॑भर्ति बिभर्ति॒ तम् ज॑हि । \newline
8. तम् ज॑हि जहि॒ तम् तम् ज॑हि॒ यदि॒ यदि॑ जहि॒ तम् तम् ज॑हि॒ यदि॑ । \newline
9. ज॒हि॒ यदि॒ यदि॑ जहि जहि॒ यदि॑ दु॒र्गे दु॒र्गे यदि॑ जहि जहि॒ यदि॑ दु॒र्गे । \newline
10. यदि॑ दु॒र्गे दु॒र्गे यदि॒ यदि॑ दु॒र्गे हन्ता॒ हन्ता॑ दु॒र्गे यदि॒ यदि॑ दु॒र्गे हन्ता᳚ । \newline
11. दु॒र्गे हन्ता॒ हन्ता॑ दु॒र्गे दु॒र्गे हन्ता ऽस्यसि॒ हन्ता॑ दु॒र्गे दु॒र्गे हन्ता ऽसि॑ । \newline
12. दु॒र्ग इति॑ दुः - गे । \newline
13. हन्ता ऽस्यसि॒ हन्ता॒ हन्ता ऽसीती त्यसि॒ हन्ता॒ हन्ता ऽसीति॑ । \newline
14. असीती त्यस्य सीति॒ स स इत्य स्यसीति॒ सः । \newline
15. इति॒ स स इतीति॒ स द॑र्भपुञ्जी॒लम् द॑र्भपुञ्जी॒लꣳ स इतीति॒ स द॑र्भपुञ्जी॒लम् । \newline
16. स द॑र्भपुञ्जी॒लम् द॑र्भपुञ्जी॒लꣳ स स द॑र्भपुञ्जी॒ल मु॒द्‌वृह्यो॒द्‌वृह्य॑ दर्भपुञ्जी॒लꣳ स स द॑र्भपुञ्जी॒ल मु॒द्‌वृह्य॑ । \newline
17. द॒र्भ॒पु॒ञ्जी॒ल मु॒द्‌वृह्यो॒द्‌वृह्य॑ दर्भपुञ्जी॒लम् द॑र्भपुञ्जी॒ल मु॒द्‌वृह्य॑ स॒प्त स॒प्तोद्‌वृह्य॑ दर्भपुञ्जी॒लम् द॑र्भपुञ्जी॒ल मु॒द्‌वृह्य॑ स॒प्त । \newline
18. द॒र्भ॒पु॒ञ्जी॒लमिति॑ दर्भ - पु॒ञ्जी॒लम् । \newline
19. उ॒द्‌वृह्य॑ स॒प्त स॒प्तोद्‌वृह्यो॒द्‌वृह्य॑ स॒प्त गि॒रीन् गि॒रीन् थ्स॒प्तोद्‌वृह्यो॒द्‌वृह्य॑ स॒प्त गि॒रीन् । \newline
20. उ॒द्‌वृह्येत्यु॑त् - वृह्य॑ । \newline
21. स॒प्त गि॒रीन् गि॒रीन् थ्स॒प्त स॒प्त गि॒रीन् भि॒त्वा भि॒त्वा गि॒रीन् थ्स॒प्त स॒प्त गि॒रीन् भि॒त्वा । \newline
22. गि॒रीन् भि॒त्वा भि॒त्वा गि॒रीन् गि॒रीन् भि॒त्वा तम् तम् भि॒त्वा गि॒रीन् गि॒रीन् भि॒त्वा तम् । \newline
23. भि॒त्वा तम् तम् भि॒त्वा भि॒त्वा त म॑हन् नह॒न् तम् भि॒त्वा भि॒त्वा त म॑हन्न् । \newline
24. त म॑हन् नह॒न् तम् त म॑ह॒न् थ्स सो॑ ऽह॒न् तम् त म॑ह॒न् थ्सः । \newline
25. अ॒ह॒न् थ्स सो॑ ऽहन् नह॒न् थ्सो᳚ ऽब्रवी दब्रवी॒थ् सो॑ ऽहन् नह॒न् थ्सो᳚ ऽब्रवीत् । \newline
26. सो᳚ ऽब्रवी दब्रवी॒थ् स सो᳚ ऽब्रवीद् दु॒र्गाद् दु॒र्गा द॑ब्रवी॒थ् स सो᳚ ऽब्रवीद् दु॒र्गात् । \newline
27. अ॒ब्र॒वी॒द् दु॒र्गाद् दु॒र्गा द॑ब्रवी दब्रवीद् दु॒र्गाद् वै वै दु॒र्गा द॑ब्रवी दब्रवीद् दु॒र्गाद् वै । \newline
28. दु॒र्गाद् वै वै दु॒र्गाद् दु॒र्गाद् वा आह॒र्ता ऽऽह॑र्ता॒ वै दु॒र्गाद् दु॒र्गाद् वा आह॑र्ता । \newline
29. दु॒र्गादिति॑ दुः - गात् । \newline
30. वा आह॒र्ता ऽऽह॑र्ता॒ वै वा आह॑र्ता ऽवोचथा अवोचथा॒ आह॑र्ता॒ वै वा आह॑र्ता ऽवोचथाः । \newline
31. आह॑र्ता ऽवोचथा अवोचथा॒ आह॒र्ता ऽऽह॑र्ता ऽवोचथा ए॒त मे॒त म॑वोचथा॒ आह॒र्ता ऽऽह॑र्ता ऽवोचथा ए॒तम् । \newline
32. आह॒र्तेत्या - ह॒र्ता॒ । \newline
33. अ॒वो॒च॒था॒ ए॒त मे॒त म॑वोचथा अवोचथा ए॒त मैत म॑वोचथा अवोचथा ए॒त मा । \newline
34. ए॒त मैत मे॒त मा ह॑र ह॒रैत मे॒त मा ह॑र । \newline
35. आ ह॑र ह॒रा ह॒रे तीति॑ ह॒रा ह॒रेति॑ । \newline
36. ह॒रे तीति॑ हर ह॒रेति॒ तम् त मिति॑ हर ह॒रेति॒ तम् । \newline
37. इति॒ तम् त मितीति॒ त मे᳚भ्य एभ्य् अ॒स्त मितीति॒ त मे᳚भ्यः । \newline
38. त मे᳚भ्य एभ्य॒ स्तम् त मे᳚भ्यो य॒ज्ञो य॒ज्ञ् ए᳚भ्य॒ स्तम् त मे᳚भ्यो य॒ज्ञ्ः । \newline
39. ए॒भ्यो॒ य॒ज्ञो य॒ज्ञ् ए᳚भ्य एभ्यो य॒ज्ञ् ए॒वैव य॒ज्ञ् ए᳚भ्य एभ्यो य॒ज्ञ् ए॒व । \newline
40. य॒ज्ञ् ए॒वैव य॒ज्ञो य॒ज्ञ् ए॒व य॒ज्ञ्ं ॅय॒ज्ञ् मे॒व य॒ज्ञो य॒ज्ञ् ए॒व य॒ज्ञ्म् । \newline
41. ए॒व य॒ज्ञ्ं ॅय॒ज्ञ् मे॒वैव य॒ज्ञ् मा य॒ज्ञ् मे॒वैव य॒ज्ञ् मा । \newline
42. य॒ज्ञ् मा य॒ज्ञ्ं ॅय॒ज्ञ् मा ऽह॑र दहर॒दा य॒ज्ञ्ं ॅय॒ज्ञ् मा ऽह॑रत् । \newline
43. आ ऽह॑र दहर॒दा ऽह॑र॒द् यद् यद॑ हर॒दा ऽह॑र॒द् यत् । \newline
44. अ॒ह॒र॒द् यद् यद॑हर दहर॒द् यत् तत् तद् यद॑हर दहर॒द् यत् तत् । \newline
45. यत् तत् तद् यद् यत् तद् वि॒त्तं ॅवि॒त्तम् तद् यद् यत् तद् वि॒त्तम् । \newline
46. तद् वि॒त्तं ॅवि॒त्तम् तत् तद् वि॒त्तं ॅवेद्यं॒ ॅवेद्यं॑ ॅवि॒त्तम् तत् तद् वि॒त्तं ॅवेद्य᳚म् । \newline
47. वि॒त्तं ॅवेद्यं॒ ॅवेद्यं॑ ॅवि॒त्तं ॅवि॒त्तं ॅवेद्य॒ मसु॑राणा॒ मसु॑राणां॒ ॅवेद्यं॑ ॅवि॒त्तं ॅवि॒त्तं ॅवेद्य॒ मसु॑राणाम् । \newline
48. वेद्य॒ मसु॑राणा॒ मसु॑राणां॒ ॅवेद्यं॒ ॅवेद्य॒ मसु॑राणा॒ मवि॑न्द॒न्ता वि॑न्द॒न्ता सु॑राणां॒ ॅवेद्यं॒ ॅवेद्य॒ मसु॑राणा॒ मवि॑न्दन्त । \newline
49. असु॑राणा॒ मवि॑न्द॒न्ता वि॑न्द॒न्ता सु॑राणा॒ मसु॑राणा॒ मवि॑न्दन्त॒ तत् तदवि॑न्द॒न्ता सु॑राणा॒ मसु॑राणा॒ मवि॑न्दन्त॒ तत् । \newline
50. अवि॑न्दन्त॒ तत् तदवि॑न्द॒न्ता वि॑न्दन्त॒ तदेक॒ मेक॒म् तदवि॑न्द॒न्ता वि॑न्दन्त॒ तदेक᳚म् । \newline
51. तदेक॒ मेक॒म् तत् तदेकं॒ ॅवेद्यै॒ वेद्या॒ एक॒म् तत् तदेकं॒ ॅवेद्यै᳚ । \newline
52. एकं॒ ॅवेद्यै॒ वेद्या॒ एक॒ मेकं॒ ॅवेद्यै॑ वेदि॒त्वं ॅवे॑दि॒त्वं ॅवेद्या॒ एक॒ मेकं॒ ॅवेद्यै॑ वेदि॒त्वम् । \newline
53. वेद्यै॑ वेदि॒त्वं ॅवे॑दि॒त्वं ॅवेद्यै॒ वेद्यै॑ वेदि॒त्व मसु॑राणा॒ मसु॑राणां ॅवेदि॒त्वं ॅवेद्यै॒ वेद्यै॑ वेदि॒त्व मसु॑राणाम् । \newline
54. वे॒दि॒त्व मसु॑राणा॒ मसु॑राणां ॅवेदि॒त्वं ॅवे॑दि॒त्व मसु॑राणां॒ ॅवै वा असु॑राणां ॅवेदि॒त्वं ॅवे॑दि॒त्व मसु॑राणां॒ ॅवै । \newline
55. वे॒दि॒त्वमिति॑ वेदि - त्वम् । \newline
56. असु॑राणां॒ ॅवै वा असु॑राणा॒ मसु॑राणां॒ ॅवा इ॒य मि॒यं ॅवा असु॑राणा॒ मसु॑राणां॒ ॅवा इ॒यम् । \newline
\pagebreak
\markright{ TS 6.2.4.4  \hfill https://www.vedavms.in \hfill}

\section{ TS 6.2.4.4 }

\textbf{TS 6.2.4.4 } \newline
\textbf{Samhita Paata} \newline

ॅवा इ॒यमग्र॑ आसी॒द् याव॒दासी॑नः परा॒पश्य॑ति॒ ताव॑द् दे॒वानां॒ ते दे॒वा अ॑ब्रुव॒न्नस्त्वे॒व नो॒ऽस्यामपीति॒ किय॑द्वो दास्याम॒ इति॒ याव॑दि॒यꣳ स॑लावृ॒की त्रिः प॑रि॒क्राम॑ति॒ ताव॑न्नो द॒त्तेति॒ स इन्द्रः॑ सलावृ॒की रू॒पं कृ॒त्वेमां त्रिः स॒र्वतः॒ पर्य॑क्राम॒त् तदि॒माम॑विन्दन्त॒ यदि॒मामवि॑न्दन्त॒ तद् वेद्यै॑ वेदि॒त्वꣳ- [  ] \newline

\textbf{Pada Paata} \newline

वै । इ॒यम् । अग्रे᳚ । आ॒सी॒त् । याव॑त् । आसी॑नः । प॒रा॒पश्य॒तीति॑ परा - पश्य॑ति । ताव॑त् । दे॒वाना᳚म् । ते । दे॒वाः । अ॒ब्रु॒व॒न्न् । अस्तु॑ । ए॒व । नः॒ । अ॒स्याम् । अपीति॑ । इति॑ । किय॑त् । वः॒ । दा॒स्या॒मः॒ । इति॑ । याव॑त् । इ॒यम् । स॒ला॒वृ॒की । त्रिः । प॒रि॒क्राम॒तीति॑ परि-क्राम॑ति । ताव॑त् । नः॒ । द॒त्त॒ । इति॑ । सः । इन्द्रः॑ । स॒ला॒वृ॒की । रू॒पम् । कृ॒त्वा । इ॒माम् । त्रिः । स॒र्वतः॑ । परीति॑ । अ॒क्रा॒म॒त् । तत् । इ॒माम् । अ॒वि॒न्द॒न्त॒ । यत् । इ॒माम् । अवि॑न्दन्त । तत् । वेद्यै᳚ । वे॒दि॒त्वमिति॑ वेदि - त्वम् ।  \newline


\textbf{Krama Paata} \newline

वा इ॒यम् । इ॒यमग्रे᳚ । अग्र॑ आसीत् । आ॒सी॒द् याव॑त् । याव॒दासी॑नः । आसी॑नः परा॒पश्य॑ति । प॒रा॒पश्य॑ति॒ ताव॑त् । प॒रा॒पश्य॒तीति॑ परा - पश्य॑ति । ताव॑द् दे॒वाना᳚म् । दे॒वाना॒म् ते । ते दे॒वाः । दे॒वा अ॑ब्रुवन्न् । अ॒ब्रु॒व॒न्नस्तु॑ । अस्त्वे॒व । ए॒व नः॑ । नो॒ऽस्याम् । अ॒स्यामपि॑ । अपीति॑ । इति॒ किय॑त् । किय॑द् वः । वो॒ दा॒स्या॒मः॒ । दा॒स्या॒म॒ इति॑ । इति॒ याव॑त् । याव॑दि॒यम् । इ॒यꣳ स॑लावृ॒की । स॒ला॒वृ॒की त्रिः । त्रिः प॑रि॒क्राम॑ति । प॒रि॒क्राम॑ति॒ ताव॑त् । प॒रि॒क्राम॒तीति॑ परि - क्राम॑ति । ताव॑न् नः । नो॒ द॒त्त॒ । द॒त्तेति॑ । इति॒ सः । स इन्द्रः॑ । इन्द्रः॑ सलावृ॒की । स॒ला॒वृ॒की रू॒पम् । रू॒पम् कृ॒त्वा । कृ॒त्वेमाम् । इ॒माम् त्रिः । त्रिः स॒र्वतः॑ । स॒र्वतः॒ परि॑ । पर्य॑क्रामत् । अ॒क्रा॒म॒त् तत् । तदि॒माम् । इ॒माम॑विन्दन्त । अ॒वि॒न्द॒न्त॒ यत् । यदि॒माम् । इ॒मामवि॑न्दन्त । अवि॑न्दन्त॒ तत् । तद् वेद्यै᳚ । वेद्यै॑ वेदि॒त्वम् । वे॒दि॒त्वꣳ सा । वे॒दि॒त्वमिति॑ वेदि - त्वम् \newline

\textbf{Jatai Paata} \newline

1. वा इ॒य मि॒यं ॅवै वा इ॒यम् । \newline
2. इ॒य मग्रे ऽग्र॑ इ॒य मि॒य मग्रे᳚ । \newline
3. अग्र॑ आसी दासी॒ दग्रे ऽग्र॑ आसीत् । \newline
4. आ॒सी॒द् याव॒द् याव॑ दासी दासी॒द् याव॑त् । \newline
5. याव॒ दासी॑न॒ आसी॑नो॒ याव॒द् याव॒ दासी॑नः । \newline
6. आसी॑नः परा॒पश्य॑ति परा॒पश्य॒ त्यासी॑न॒ आसी॑नः परा॒पश्य॑ति । \newline
7. प॒रा॒पश्य॑ति॒ ताव॒त् ताव॑त् परा॒पश्य॑ति परा॒पश्य॑ति॒ ताव॑त् । \newline
8. प॒रा॒पश्य॒तीति॑ परा - पश्य॑ति । \newline
9. ताव॑द् दे॒वाना᳚म् दे॒वाना॒म् ताव॒त् ताव॑द् दे॒वाना᳚म् । \newline
10. दे॒वाना॒म् ते ते दे॒वाना᳚म् दे॒वाना॒म् ते । \newline
11. ते दे॒वा दे॒वा स्ते ते दे॒वाः । \newline
12. दे॒वा अ॑ब्रुवन् नब्रुवन् दे॒वा दे॒वा अ॑ब्रुवन्न् । \newline
13. अ॒ब्रु॒व॒न् नस्त्व स्त्व॑ब्रुवन् नब्रुव॒न् नस्तु॑ । \newline
14. अस्त्वे॒ वैवास्त्व स्त्वे॒व । \newline
15. ए॒व नो॑ न ए॒वैव नः॑ । \newline
16. नो॒ ऽस्या म॒स्यान्नो॑ नो॒ ऽस्याम् । \newline
17. अ॒स्या मप्य प्य॒स्या म॒स्या मपि॑ । \newline
18. अपीती त्यप्यपीति॑ । \newline
19. इति॒ किय॒त् किय॒ दितीति॒ किय॑त् । \newline
20. किय॑द् वो वः॒ किय॒त् किय॑द् वः । \newline
21. वो॒ दा॒स्या॒मो॒ दा॒स्या॒मो॒ वो॒ वो॒ दा॒स्या॒मः॒ । \newline
22. दा॒स्या॒म॒ इतीति॑ दास्यामो दास्याम॒ इति॑ । \newline
23. इति॒ याव॒द् याव॒ दितीति॒ याव॑त् । \newline
24. याव॑दि॒य मि॒यं ॅयाव॒द् याव॑ दि॒यम् । \newline
25. इ॒यꣳ स॑लावृ॒की स॑लावृ॒कीय मि॒यꣳ स॑लावृ॒की । \newline
26. स॒ला॒वृ॒की त्रि स्त्रिः स॑लावृ॒की स॑लावृ॒की त्रिः । \newline
27. त्रिः प॑रि॒क्राम॑ति परि॒क्राम॑ति॒ त्रि स्त्रिः प॑रि॒क्राम॑ति । \newline
28. प॒रि॒क्राम॑ति॒ ताव॒त् ताव॑त् परि॒क्राम॑ति परि॒क्राम॑ति॒ ताव॑त् । \newline
29. प॒रि॒क्राम॒तीति॑ परि - क्राम॑ति । \newline
30. ताव॑न् नो न॒ स्ताव॒त् ताव॑न् नः । \newline
31. नो॒ द॒त्त॒ द॒त्त॒ नो॒ नो॒ द॒त्त॒ । \newline
32. द॒त्ते तीति॑ दत्त द॒त्तेति॑ । \newline
33. इति॒ स स इतीति॒ सः । \newline
34. स इन्द्र॒ इन्द्रः॒ स स इन्द्रः॑ । \newline
35. इन्द्रः॑ सलावृ॒की स॑लावृ॒कीन्द्र॒ इन्द्रः॑ सलावृ॒की । \newline
36. स॒ला॒वृ॒की रू॒पꣳ रू॒पꣳ स॑लावृ॒की स॑लावृ॒की रू॒पम् । \newline
37. रू॒पम् कृ॒त्वा कृ॒त्वा रू॒पꣳ रू॒पम् कृ॒त्वा । \newline
38. कृ॒त्वेमा मि॒माम् कृ॒त्वा कृ॒त्वेमाम् । \newline
39. इ॒माम् त्रि स्त्रि रि॒मा मि॒माम् त्रिः । \newline
40. त्रिः स॒र्वतः॑ स॒र्वत॒ स्त्रि स्त्रिः स॒र्वतः॑ । \newline
41. स॒र्वतः॒ परि॒ परि॑ स॒र्वतः॑ स॒र्वतः॒ परि॑ । \newline
42. पर्य॑क्राम दक्राम॒त् परि॒ पर्य॑क्रामत् । \newline
43. अ॒क्रा॒म॒त् तत् तद॑क्राम दक्राम॒त् तत् । \newline
44. तदि॒मा मि॒माम् तत् तदि॒माम् । \newline
45. इ॒मा म॑विन्दन्ता विन्दन्ते॒ मा मि॒मा म॑विन्दन्त । \newline
46. अ॒वि॒न्द॒न्त॒ यद् यद॑विन्दन्ता विन्दन्त॒ यत् । \newline
47. यदि॒मा मि॒मां ॅयद् यदि॒माम् । \newline
48. इ॒मा मवि॑न्द॒न्ता वि॑न्दन्ते॒ मा मि॒मा मवि॑न्दन्त । \newline
49. अवि॑न्दन्त॒ तत् तदवि॑न्द॒न्ता वि॑न्दन्त॒ तत् । \newline
50. तद् वेद्यै॒ वेद्यै॒ तत् तद् वेद्यै᳚ । \newline
51. वेद्यै॑ वेदि॒त्वं ॅवे॑दि॒त्वं ॅवेद्यै॒ वेद्यै॑ वेदि॒त्वम् । \newline
52. वे॒दि॒त्वꣳ सा सा वे॑दि॒त्वं ॅवे॑दि॒त्वꣳ सा । \newline
53. वे॒दि॒त्वमिति॑ वेदि - त्वम् । \newline

\textbf{Ghana Paata } \newline

1. वा इ॒य मि॒यं ॅवै वा इ॒य मग्रे ऽग्र॑ इ॒यं ॅवै वा इ॒य मग्रे᳚ । \newline
2. इ॒य मग्रे ऽग्र॑ इ॒य मि॒य मग्र॑ आसी दासी॒ दग्र॑ इ॒य मि॒य मग्र॑ आसीत् । \newline
3. अग्र॑ आसी दासी॒ दग्रे ऽग्र॑ आसी॒द् याव॒द् याव॑ दासी॒ दग्रे ऽग्र॑ आसी॒द् याव॑त् । \newline
4. आ॒सी॒द् याव॒द् याव॑ दासी दासी॒द् याव॒ दासी॑न॒ आसी॑नो॒ याव॑ दासी दासी॒द् याव॒ दासी॑नः । \newline
5. याव॒ दासी॑न॒ आसी॑नो॒ याव॒द् याव॒ दासी॑नः परा॒पश्य॑ति परा॒पश्य॒ त्यासी॑नो॒ याव॒द् याव॒ दासी॑नः परा॒पश्य॑ति । \newline
6. आसी॑नः परा॒पश्य॑ति परा॒पश्य॒ त्यासी॑न॒ आसी॑नः परा॒पश्य॑ति॒ ताव॒त् ताव॑त् परा॒पश्य॒ त्यासी॑न॒ आसी॑नः परा॒पश्य॑ति॒ ताव॑त् । \newline
7. प॒रा॒पश्य॑ति॒ ताव॒त् ताव॑त् परा॒पश्य॑ति परा॒पश्य॑ति॒ ताव॑द् दे॒वाना᳚म् दे॒वाना॒म् ताव॑त् परा॒पश्य॑ति परा॒पश्य॑ति॒ ताव॑द् दे॒वाना᳚म् । \newline
8. प॒रा॒पश्य॒तीति॑ परा - पश्य॑ति । \newline
9. ताव॑द् दे॒वाना᳚म् दे॒वाना॒म् ताव॒त् ताव॑द् दे॒वाना॒म् ते ते दे॒वाना॒म् ताव॒त् ताव॑द् दे॒वाना॒म् ते । \newline
10. दे॒वाना॒म् ते ते दे॒वाना᳚म् दे॒वाना॒म् ते दे॒वा दे॒वा स्ते दे॒वाना᳚म् दे॒वाना॒म् ते दे॒वाः । \newline
11. ते दे॒वा दे॒वा स्ते ते दे॒वा अ॑ब्रुवन् नब्रुवन् दे॒वा स्ते ते दे॒वा अ॑ब्रुवन्न् । \newline
12. दे॒वा अ॑ब्रुवन् नब्रुवन् दे॒वा दे॒वा अ॑ब्रुव॒न् नस्त्व स्त्व॑ब्रुवन् दे॒वा दे॒वा अ॑ब्रुव॒न् नस्तु॑ । \newline
13. अ॒ब्रु॒व॒न् नस्त्व स्त्व॑ब्रुवन् नब्रुव॒न् नस्त्वे॒ वैवास्त्व॑ ब्रुवन् नब्रुव॒न्न स्त्वे॒व । \newline
14. अस्त्वे॒ वैवास्त्व स्त्वे॒व नो॑ न ए॒वास्त्व स्त्वे॒व नः॑ । \newline
15. ए॒व नो॑ न ए॒वैव नो॒ ऽस्या म॒स्यान्न॑ ए॒वैव नो॒ ऽस्याम् । \newline
16. नो॒ ऽस्या म॒स्यान् नो॑ नो॒ ऽस्या मप्यप्य॒ स्यान्नो॑ नो॒ ऽस्या मपि॑ । \newline
17. अ॒स्या मप्य प्य॒स्या म॒स्या मपी तीत्य प्य॒स्या म॒स्या मपीति॑ । \newline
18. अपीती त्यप्य पीति॒ किय॒त् किय॒ दित्यप्य पीति॒ किय॑त् । \newline
19. इति॒ किय॒त् किय॒ दितीति॒ किय॑द् वो वः॒ किय॒ दितीति॒ किय॑द् वः । \newline
20. किय॑द् वो वः॒ किय॒त् किय॑द् वो दास्यामो दास्यामो वः॒ किय॒त् किय॑द् वो दास्यामः । \newline
21. वो॒ दा॒स्या॒मो॒ दा॒स्या॒मो॒ वो॒ वो॒ दा॒स्या॒म॒ इतीति॑ दास्यामो वो वो दास्याम॒ इति॑ । \newline
22. दा॒स्या॒म॒ इतीति॑ दास्यामो दास्याम॒ इति॒ याव॒द् याव॒ दिति॑ दास्यामो दास्याम॒ इति॒ याव॑त् । \newline
23. इति॒ याव॒द् याव॒ दितीति॒ याव॑ दि॒य मि॒यं ॅयाव॒दि तीति॒ याव॑ दि॒यम् । \newline
24. याव॑ दि॒य मि॒यं ॅयाव॒द् याव॑ दि॒यꣳ स॑लावृ॒की स॑लावृ॒कीयं ॅयाव॒द् याव॑ दि॒यꣳ स॑लावृ॒की । \newline
25. इ॒यꣳ स॑लावृ॒की स॑लावृ॒कीय मि॒यꣳ स॑लावृ॒की त्रि स्त्रिः स॑लावृ॒कीय मि॒यꣳ स॑लावृ॒की त्रिः । \newline
26. स॒ला॒वृ॒की त्रि स्त्रिः स॑लावृ॒की स॑लावृ॒की त्रिः प॑रि॒क्राम॑ति परि॒क्राम॑ति॒ त्रिः स॑लावृ॒की स॑लावृ॒की त्रिः प॑रि॒क्राम॑ति । \newline
27. त्रिः प॑रि॒क्राम॑ति परि॒क्राम॑ति॒ त्रि स्त्रिः प॑रि॒क्राम॑ति॒ ताव॒त् ताव॑त् परि॒क्राम॑ति॒ त्रि स्त्रिः प॑रि॒क्राम॑ति॒ ताव॑त् । \newline
28. प॒रि॒क्राम॑ति॒ ताव॒त् ताव॑त् परि॒क्राम॑ति परि॒क्राम॑ति॒ ताव॑न् नो न॒ स्ताव॑त् परि॒क्राम॑ति परि॒क्राम॑ति॒ ताव॑न् नः । \newline
29. प॒रि॒क्राम॒तीति॑ परि - क्राम॑ति । \newline
30. ताव॑न् नो न॒ स्ताव॒त् ताव॑न् नो दत्त दत्त न॒ स्ताव॒त् ताव॑न् नो दत्त । \newline
31. नो॒ द॒त्त॒ द॒त्त॒ नो॒ नो॒ द॒त्ते तीति॑ दत्त नो नो द॒त्तेति॑ । \newline
32. द॒त्ते तीति॑ दत्त द॒त्तेति॒ स स इति॑ दत्त द॒त्तेति॒ सः । \newline
33. इति॒ स स इतीति॒ स इन्द्र॒ इन्द्रः॒ स इतीति॒ स इन्द्रः॑ । \newline
34. स इन्द्र॒ इन्द्रः॒ स स इन्द्रः॑ सलावृ॒की स॑लावृ॒ कीन्द्रः॒ स स इन्द्रः॑ सलावृ॒की । \newline
35. इन्द्रः॑ सलावृ॒की स॑लावृ॒कीन्द्र॒ इन्द्रः॑ सलावृ॒की रू॒पꣳ रू॒पꣳ स॑लावृ॒कीन्द्र॒ इन्द्रः॑ सलावृ॒की रू॒पम् । \newline
36. स॒ला॒वृ॒की रू॒पꣳ रू॒पꣳ स॑लावृ॒की स॑लावृ॒की रू॒पम् कृ॒त्वा कृ॒त्वा रू॒पꣳ स॑लावृ॒की स॑लावृ॒की रू॒पम् कृ॒त्वा । \newline
37. रू॒पम् कृ॒त्वा कृ॒त्वा रू॒पꣳ रू॒पम् कृ॒त्वेमा मि॒माम् कृ॒त्वा रू॒पꣳ रू॒पम् कृ॒त्वेमाम् । \newline
38. कृ॒त्वेमा मि॒माम् कृ॒त्वा कृ॒त्वेमाम् त्रि स्त्रि रि॒माम् कृ॒त्वा कृ॒त्वेमाम् त्रिः । \newline
39. इ॒माम् त्रि स्त्रि रि॒मा मि॒माम् त्रिः स॒र्वतः॑ स॒र्वत॒ स्त्रि रि॒मा मि॒माम् त्रिः स॒र्वतः॑ । \newline
40. त्रिः स॒र्वतः॑ स॒र्वत॒ स्त्रि स्त्रिः स॒र्वतः॒ परि॒ परि॑ स॒र्वत॒ स्त्रि स्त्रिः स॒र्वतः॒ परि॑ । \newline
41. स॒र्वतः॒ परि॒ परि॑ स॒र्वतः॑ स॒र्वतः॒ पर्य॑क्राम दक्राम॒त् परि॑ स॒र्वतः॑ स॒र्वतः॒ पर्य॑क्रामत् । \newline
42. पर्य॑क्राम दक्राम॒त् परि॒ पर्य॑क्राम॒त् तत् तद॑क्राम॒त् परि॒ पर्य॑क्राम॒त् तत् । \newline
43. अ॒क्रा॒म॒त् तत् तद॑क्राम दक्राम॒त् तदि॒मा मि॒माम् तद॑क्राम दक्राम॒त् तदि॒माम् । \newline
44. तदि॒मा मि॒माम् तत् तदि॒मा म॑विन्दन्ता विन्दन्ते॒माम् तत् तदि॒मा म॑विन्दन्त । \newline
45. इ॒मा म॑विन्दन्ता विन्दन्ते॒मा मि॒मा म॑विन्दन्त॒ यद् यद॑विन्दन्ते॒मा मि॒मा म॑विन्दन्त॒ यत् । \newline
46. अ॒वि॒न्द॒न्त॒ यद् यद॑विन्दन्ता विन्दन्त॒ यदि॒मा मि॒मां ॅयद॑विन्दन्ता विन्दन्त॒ यदि॒माम् । \newline
47. यदि॒मा मि॒मां ॅयद् यदि॒मा मवि॑न्द॒न्ता वि॑न्दन्ते॒मां ॅयद् यदि॒मा मवि॑न्दन्त । \newline
48. इ॒मा मवि॑न्द॒न्ता वि॑न्दन्ते॒मा मि॒मा मवि॑न्दन्त॒ तत् तदवि॑न्दन्ते॒मा मि॒मा मवि॑न्दन्त॒ तत् । \newline
49. अवि॑न्दन्त॒ तत् तदवि॑न्द॒न्ता वि॑न्दन्त॒ तद् वेद्यै॒ वेद्यै॒ तदवि॑न्द॒न्ता वि॑न्दन्त॒ तद् वेद्यै᳚ । \newline
50. तद् वेद्यै॒ वेद्यै॒ तत् तद् वेद्यै॑ वेदि॒त्वं ॅवे॑दि॒त्वं ॅवेद्यै॒ तत् तद् वेद्यै॑ वेदि॒त्वम् । \newline
51. वेद्यै॑ वेदि॒त्वं ॅवे॑दि॒त्वं ॅवेद्यै॒ वेद्यै॑ वेदि॒त्वꣳ सा सा वे॑दि॒त्वं ॅवेद्यै॒ वेद्यै॑ वेदि॒त्वꣳ सा । \newline
52. वे॒दि॒त्वꣳ सा सा वे॑दि॒त्वं ॅवे॑दि॒त्वꣳ सा वै वै सा वे॑दि॒त्वं ॅवे॑दि॒त्वꣳ सा वै । \newline
53. वे॒दि॒त्वमिति॑ वेदि - त्वम् । \newline
\pagebreak
\markright{ TS 6.2.4.5  \hfill https://www.vedavms.in \hfill}

\section{ TS 6.2.4.5 }

\textbf{TS 6.2.4.5 } \newline
\textbf{Samhita Paata} \newline

सा वा इ॒यꣳ सर्वै॒व वेदि॒रिय॑ति शक्ष्या॒मीति॒ त्वा अ॑व॒माय॑ यजन्ते त्रिꣳ॒॒शत् प॒दानि॑ प॒श्चात् ति॒रश्ची॑ भवति॒ षट्त्रिꣳ॑श॒त् प्राची॒ चतु॑र्विꣳशतिः पु॒रस्ता᳚त् ति॒रश्ची॒ दश॑दश॒ संप॑द्यन्ते॒ दशा᳚क्षरा वि॒राडन्नं॑ ॅवि॒राड् वि॒राजै॒वान्नाद्य॒मव॑ रुन्ध॒ उद्ध॑न्ति॒ यदे॒वास्या॑ अमे॒द्ध्यं तदप॑ ह॒न्त्युद्ध॑न्ति॒ तस्मा॒दोष॑धयः॒ परा॑ भवन्ति ( ) ब॒र्॒.हिः स्तृ॑णाति॒ तस्मा॒दोष॑धयः॒ पुन॒रा भ॑व॒न्त्युत्त॑रं ब॒र्॒.हिष॑ उत्तरब॒र्॒.हिः स्तृ॑णाति प्र॒जा वै ब॒र्॒.हिर्यज॑मान उत्तर ब॒र्॒.हि र्यज॑मान-मे॒वा-य॑जमाना॒दुत्त॑रं करोति॒ तस्मा॒द्-यज॑मा॒नो ऽय॑जमाना॒दुत्त॑रः ॥ \newline

\textbf{Pada Paata} \newline

सा । वै । इ॒यम् । सर्वा᳚ । ए॒व । वेदिः॑ । इय॑ति । श॒क्ष्या॒मि॒ । इति॑ । तु । वै । अ॒व॒मायेत्य॑व - माय॑ । य॒ज॒न्ते॒ । त्रिꣳ॒॒शत् । प॒दानि॑ । प॒श्चात् । ति॒रश्ची᳚ । भ॒व॒ति॒ । षट्त्रिꣳ॑श॒दिति॒ षट् - त्रिꣳ॒॒श॒त् । प्राची᳚ । चतु॑र्विꣳशति॒रिति॒ चतुः॑ - विꣳ॒॒श॒तिः॒ । पु॒रस्ता᳚त् । ति॒रश्ची᳚ । दश॑द॒शेति॒ दश॑-द॒श॒ । समिति॑ । प॒द्य॒न्ते॒ । दशा᳚क्ष॒रेति॒ दश॑ - अ॒क्ष॒रा॒ । वि॒राडिति॑ वि - राट् । अन्न᳚म् । वि॒राडिति॑ वि - राट् । वि॒राजेति॑ वि - राजा᳚ । ए॒व । अ॒न्नाद्य॒मित्य॑न्न - अद्य᳚म् । अवेति॑ । रु॒न्धे॒ । उदिति॑ । ह॒न्ति॒ । यत् । ए॒व । अ॒स्याः॒ । अ॒मे॒द्ध्यम् । तत् । अपेति॑ । ह॒न्ति॒ । उदिति॑ । ह॒न्ति॒ । तस्मा᳚त् । ओष॑धयः । परेति॑ । भ॒व॒न्ति॒ ( ) । ब॒र्॒.हिः । स्तृ॒णा॒ति॒ । तस्मा᳚त् । ओष॑धयः । पुनः॑ । एति॑ । भ॒व॒न्ति॒ । उत्त॑र॒मित्युत् - त॒र॒म् । ब॒र्॒.हिषः॑ । उ॒त्त॒र॒ब॒र्॒.हिरित्यु॑त्तर - ब॒र्॒.हिः । स्तृ॒णा॒ति॒ । प्र॒जा इति॑ प्र - जाः । वै । ब॒र्॒.हिः । यज॑मानः । उ॒त्त॒र॒ब॒र॒.हिरित्यु॑त्तर - ब॒र॒.हिः । यज॑मानम् । ए॒व । अय॑जमानात् । उत्त॑र॒मित्युत् - त॒र॒म् । क॒रो॒ति॒ । तस्मा᳚त् । यज॑मानः । अय॑जमानात् । उत्त॑र॒ इत्युत् - त॒रः॒ ॥  \newline


\textbf{Krama Paata} \newline

सा वै । वा इ॒यम् । इ॒यꣳ सर्वा᳚ । सर्वै॒व । ए॒व वेदिः॑ । वेदि॒रिय॑ति । इय॑ति शक्ष्यामि । श॒क्ष्या॒मीति॑ । इति॒ तु । त्वै । वा अ॑व॒माय॑ । अ॒व॒माय॑ यजन्ते । अ॒व॒मायेत्य॑व - माय॑ । य॒ज॒न्ते॒ त्रिꣳ॒॒शत् । त्रिꣳ॒॒शत् प॒दानि॑ । प॒दानि॑ प॒श्चात् । प॒श्चात् ति॒रश्ची᳚ । ति॒रश्ची॑ भवति । भ॒व॒ति॒ षट्त्रिꣳ॑शत् । षट्त्रिꣳ॑श॒त् प्राची᳚ । षट्त्रिꣳ॑श॒दिति॒ षट् - त्रिꣳ॒॒श॒त्॒ । प्राची॒ चतु॑र्विꣳशतिः । चतु॑र्विꣳशतिः पु॒रस्ता᳚त् । चतु॑र्विꣳशति॒रिति॒ चतुः॑ - विꣳ॒॒श॒तिः॒ । पु॒रस्ता᳚त् ति॒रश्ची᳚ । ति॒रश्ची॒ दश॑दश । दश॑दश॒ सम् । दश॑द॒शेति॒ दश॑ - द॒श॒ । सम् प॑द्यन्ते । प॒द्य॒न्ते॒ दशा᳚क्षरा । दशा᳚क्षरा वि॒राट् । दशा᳚क्ष॒रेति॒ दश॑ - अ॒क्ष॒रा॒ । वि॒राडन्न᳚म् । वि॒राडिति॑ वि - राट् । अन्न॑म् ॅवि॒राट् । वि॒राड् वि॒राजा᳚ । वि॒राडिति॑ वि - राट् । वि॒राजै॒व । वि॒राजेति॑ वि - राजा᳚ । ए॒वान्नाद्य᳚म् । अ॒न्नाद्य॒मव॑ । अ॒न्नाद्य॒मित्य॑न्न - अद्य᳚म् । अव॑ रुन्धे । रु॒न्ध॒ उत् । उद्‍ध॑न्ति । ह॒न्ति॒ यत् । यदे॒व । ए॒वास्याः᳚ । अ॒स्या॒ अ॒मे॒द्ध्यम् । अ॒मे॒द्ध्यम् तत् । तदप॑ । अप॑ हन्ति । ह॒न्त्युत् । उद्‍ध॑न्ति । ह॒न्ति॒ तस्मा᳚त् । तस्मा॒दोष॑धयः । ओष॑धयः॒ परा᳚ । परा॑ भवन्ति ( ) । भ॒व॒न्ति॒ ब॒र्.॒हिः । ब॒र्.॒हिः स्तृ॑णाति । स्तृ॒णा॒ति॒ तस्मा᳚त् । तस्मा॒दोष॑धयः । ओष॑धयः॒ पुनः॑ । पुन॒रा । आ भ॑वन्ति । भ॒व॒न्त्युत्त॑रम् । उत्त॑रम् ब॒र्.॒हिषः॑ । उत्त॑र॒मित्युत् - त॒र॒म् । ब॒र्.॒हिष॑ उत्तरब॒र्.॒हिः । उ॒त्त॒र॒ब॒र्.॒हिः स्तृ॑णाति । ॒त्त॒र॒ब॒र्.॒हिरित्यु॑त्तर - ब॒र्.॒हिः । स्तृ॒णा॒ति॒ प्र॒जाः । प्र॒जा वै । प्र॒जा इति॑ प्र - जाः । वै ब॒र्.॒हिः । ब॒र्.॒हिर् यज॑मानः । यज॑मान उत्तरब॒र्.॒हिः । उ॒त्त॒र॒ब॒र्.॒हिर् यज॑मानम् । उ॒त्त॒र॒ब॒र्.॒हिरित्यु॑त्तर - ब॒र्.॒हिः । यज॑मानमे॒व । ए॒वाय॑जमानात् । अय॑जमाना॒दुत्त॑रम् । उत्त॑रम् करोति । उत्त॑र॒मित्युत् - त॒र॒म् । क॒रो॒ति॒ तस्मा᳚त् । तस्मा॒द् यज॑मानः । यज॑मा॒नोऽय॑जमानात् । अय॑जमाना॒दुत्त॑रः । उत्त॑र॒ इत्युत् - त॒रः॒ । \newline

\textbf{Jatai Paata} \newline

1. सा वै वै सा सा वै । \newline
2. वा इ॒य मि॒यं ॅवै वा इ॒यम् । \newline
3. इ॒यꣳ सर्वा॒ सर्वे॒य मि॒यꣳ सर्वा᳚ । \newline
4. सर्वै॒वैव सर्वा॒ सर्वै॒व । \newline
5. ए॒व वेदि॒र् वेदि॑ रे॒वैव वेदिः॑ । \newline
6. वेदि॒ रिय॒तीय॑ति॒ वेदि॒र् वेदि॒ रिय॑ति । \newline
7. इय॑ति शक्ष्यामि शक्ष्या॒मी य॒ती य॑ति शक्ष्यामि । \newline
8. श॒क्ष्या॒मी तीति॑ शक्ष्यामि शक्ष्या॒मीति॑ । \newline
9. इति॒ तु त्वितीति॒ तु । \newline
10. त्वै वै तुत् वै । \newline
11. वा अ॑व॒माया॑ व॒माय॒ वै वा अ॑व॒माय॑ । \newline
12. अ॒व॒माय॑ यजन्ते यजन्ते ऽव॒माया॑ व॒माय॑ यजन्ते । \newline
13. अ॒व॒मायेत्य॑व - माय॑ । \newline
14. य॒ज॒न्ते॒ त्रिꣳ॒॒शत् त्रिꣳ॒॒शद् य॑जन्ते यजन्ते त्रिꣳ॒॒शत् । \newline
15. त्रिꣳ॒॒शत् प॒दानि॑ प॒दानि॑ त्रिꣳ॒॒शत् त्रिꣳ॒॒शत् प॒दानि॑ । \newline
16. प॒दानि॑ प॒श्चात् प॒श्चात् प॒दानि॑ प॒दानि॑ प॒श्चात् । \newline
17. प॒श्चात् ति॒रश्ची॑ ति॒रश्ची॑ प॒श्चात् प॒श्चात् ति॒रश्ची᳚ । \newline
18. ति॒रश्ची॑ भवति भवति ति॒रश्ची॑ ति॒रश्ची॑ भवति । \newline
19. भ॒व॒ति॒ षट्त्रिꣳ॑श॒थ् षट्त्रिꣳ॑शद् भवति भवति॒ षट्त्रिꣳ॑शत् । \newline
20. षट्त्रिꣳ॑श॒त् प्राची॒ प्राची॒ षट्त्रिꣳ॑श॒थ् षट्त्रिꣳ॑श॒त् प्राची᳚ । \newline
21. षट्त्रिꣳ॑श॒दिति॒ षट् - त्रिꣳ॒॒श॒त् । \newline
22. प्राची॒ चतु॑र्विꣳशति॒ श्चतु॑र्विꣳशतिः॒ प्राची॒ प्राची॒ चतु॑र्विꣳशतिः । \newline
23. चतु॑र्विꣳशतिः पु॒रस्ता᳚त् पु॒रस्ता॒च् चतु॑र्विꣳशति॒ श्चतु॑र्विꣳशतिः पु॒रस्ता᳚त् । \newline
24. चतु॑र्विꣳशति॒रिति॒ चतुः॑ - विꣳ॒॒श॒तिः॒ । \newline
25. पु॒रस्ता᳚त् ति॒रश्ची॑ ति॒रश्ची॑ पु॒रस्ता᳚त् पु॒रस्ता᳚त् ति॒रश्ची᳚ । \newline
26. ति॒रश्ची॒ दश॑दश॒ दश॑दश ति॒रश्ची॑ ति॒रश्ची॒ दश॑दश । \newline
27. दश॑दश॒ सꣳ सम् दश॑दश॒ दश॑दश॒ सम् । \newline
28. दश॑द॒शेति॒ दश॑ - द॒श॒ । \newline
29. सम् प॑द्यन्ते पद्यन्ते॒ सꣳ सम् प॑द्यन्ते । \newline
30. प॒द्य॒न्ते॒ दशा᳚क्षरा॒ दशा᳚क्षरा पद्यन्ते पद्यन्ते॒ दशा᳚क्षरा । \newline
31. दशा᳚क्षरा वि॒राड् वि॒राड् दशा᳚क्षरा॒ दशा᳚क्षरा वि॒राट् । \newline
32. दशा᳚क्ष॒रेति॒ दश॑ - अ॒क्ष॒रा॒ । \newline
33. वि॒रा डन्न॒ मन्नं॑ ॅवि॒राड् वि॒रा डन्न᳚म् । \newline
34. वि॒राडिति॑ वि - राट् । \newline
35. अन्नं॑ ॅवि॒राड् वि॒रा डन्न॒ मन्नं॑ ॅवि॒राट् । \newline
36. वि॒राड् वि॒राजा॑ वि॒राजा॑ वि॒राड् वि॒राड् वि॒राजा᳚ । \newline
37. वि॒राडिति॑ वि - राट् । \newline
38. वि॒राजै॒वैव वि॒राजा॑ वि॒राजै॒व । \newline
39. वि॒राजेति॑ वि - राजा᳚ । \newline
40. ए॒वा न्नाद्य॑ म॒न्नाद्य॑ मे॒वैवा न्नाद्य᳚म् । \newline
41. अ॒न्नाद्य॒ मवावा॒ न्नाद्य॑ म॒न्नाद्य॒ मव॑ । \newline
42. अ॒न्नाद्य॒मित्य॑न्न - अद्य᳚म् । \newline
43. अव॑ रुन्धे रु॒न्धे ऽवाव॑ रुन्धे । \newline
44. रु॒न्ध॒ उदुद् रु॑न्धे रुन्ध॒ उत् । \newline
45. उद्ध॑न्ति ह॒न्त्युदु द्ध॑न्ति । \newline
46. ह॒न्ति॒ यद् यद्ध॑न्ति हन्ति॒ यत् । \newline
47. यदे॒वैव यद् यदे॒व । \newline
48. ए॒वास्या॑ अस्या ए॒वैवास्याः᳚ । \newline
49. अ॒स्या॒ अ॒मे॒द्ध्य म॑मे॒द्ध्य म॑स्या अस्या अमे॒द्ध्यम् । \newline
50. अ॒मे॒द्ध्यम् तत् तद॑मे॒द्ध्य म॑मे॒द्ध्यम् तत् । \newline
51. तदपाप॒ तत् तदप॑ । \newline
52. अप॑ हन्ति ह॒न्त्यपाप॑ हन्ति । \newline
53. ह॒न्त्युदु द्ध॑न्ति ह॒न्त्युत् । \newline
54. उद्ध॑न्ति ह॒न्त्युदु द्ध॑न्ति । \newline
55. ह॒न्ति॒ तस्मा॒त् तस्मा᳚ द्धन्ति हन्ति॒ तस्मा᳚त् । \newline
56. तस्मा॒ दोष॑धय॒ ओष॑धय॒ स्तस्मा॒त् तस्मा॒ दोष॑धयः । \newline
57. ओष॑धयः॒ परा॒ परौष॑धय॒ ओष॑धयः॒ परा᳚ । \newline
58. परा॑ भवन्ति भवन्ति॒ परा॒ परा॑ भवन्ति । \newline
59. भ॒व॒न्ति॒ ब॒र्॒.हिर् ब॒र्॒.हिर् भ॑वन्ति भवन्ति ब॒र्॒.हिः । \newline
60. ब॒र्॒.हिः स्तृ॑णाति स्तृणाति ब॒र्॒.हिर् ब॒र्॒.हिः स्तृ॑णाति । \newline
61. स्तृ॒णा॒ति॒ तस्मा॒त् तस्मा᳚थ् स्तृणाति स्तृणाति॒ तस्मा᳚त् । \newline
62. तस्मा॒ दोष॑धय॒ ओष॑धय॒ स्तस्मा॒त् तस्मा॒ दोष॑धयः । \newline
63. ओष॑धयः॒ पुनः॒ पुन॒ रोष॑धय॒ ओष॑धयः॒ पुनः॑ । \newline
64. पुन॒ रा पुनः॒ पुन॒ रा । \newline
65. आ भ॑वन्ति भव॒न्त्या भ॑वन्ति । \newline
66. भ॒व॒ न्त्युत्त॑र॒ मुत्त॑रम् भवन्ति भव॒ न्त्युत्त॑रम् । \newline
67. उत्त॑रम् ब॒र्॒.हिषो॑ ब॒र्॒.हिष॒ उत्त॑र॒ मुत्त॑रम् ब॒र्॒.हिषः॑ । \newline
68. उत्त॑र॒मित्युत् - त॒र॒म् । \newline
69. ब॒र्॒.हिष॑ उत्तरब॒र्॒.हि रु॑त्तरब॒र्॒.हिर् ब॒र्॒.हिषो॑ ब॒र्॒.हिष॑ उत्तरब॒र्॒.हिः । \newline
70. उ॒त्त॒र॒ब॒र्॒.हिः स्तृ॑णाति स्तृणा त्युत्तरब॒र्॒.हि रु॑त्तरब॒र्॒.हिः स्तृ॑णाति । \newline
71. उ॒त्त॒र॒ब॒र्॒.हिरित्यु॑त्तर - ब॒र्॒.हिः । \newline
72. स्तृ॒णा॒ति॒ प्र॒जाः प्र॒जाः स्तृ॑णाति स्तृणाति प्र॒जाः । \newline
73. प्र॒जा वै वै प्र॒जाः प्र॒जा वै । \newline
74. प्र॒जा इति॑ प्र - जाः । \newline
75. वै ब॒र्॒.हिर् ब॒र्॒.हिर् वै वै ब॒र्॒.हिः । \newline
76. ब॒र्॒.हिर् यज॑मानो॒ यज॑मानो ब॒र्॒.हिर् ब॒र्॒.हिर् यज॑मानः । \newline
77. यज॑मान उत्तरब॒र्॒.हि रु॑त्तरब॒र्॒.हिर् यज॑मानो॒ यज॑मान उत्तरब॒र्॒.हिः । \newline
78. उ॒त्त॒र॒ब॒र्॒.हिर् यज॑मानं॒ ॅयज॑मान मुत्तरब॒र्॒.हि रु॑त्तरब॒र्॒.हिर् यज॑मानम् । \newline
79. उ॒त्त॒र॒ब॒र॒.हिरित्यु॑त्तर - ब॒र॒.हिः । \newline
80. यज॑मान मे॒वैव यज॑मानं॒ ॅयज॑मान मे॒व । \newline
81. ए॒वा य॑जमाना॒ दय॑जमाना दे॒वैवा य॑जमानात् । \newline
82. अय॑जमाना॒ दुत्त॑र॒ मुत्त॑र॒ मय॑जमाना॒ दय॑जमाना॒ दुत्त॑रम् । \newline
83. उत्त॑रम् करोति करो॒ त्युत्त॑र॒ मुत्त॑रम् करोति । \newline
84. उत्त॑र॒मित्युत् - त॒र॒म् । \newline
85. क॒रो॒ति॒ तस्मा॒त् तस्मा᳚त् करोति करोति॒ तस्मा᳚त् । \newline
86. तस्मा॒द् यज॑मानो॒ यज॑मान॒ स्तस्मा॒त् तस्मा॒द् यज॑मानः । \newline
87. यज॑मा॒नो ऽय॑जमाना॒ दय॑जमाना॒द् यज॑मानो॒ यज॑मा॒नो ऽय॑जमानात् । \newline
88. अय॑जमाना॒ दुत्त॑र॒ उत्त॒रो ऽय॑जमाना॒ दय॑जमाना॒ दुत्त॑रः । \newline
89. उत्त॑र॒ इत्युत् - त॒रः॒ । \newline

\textbf{Ghana Paata } \newline

1. सा वै वै सा सा वा इ॒य मि॒यं ॅवै सा सा वा इ॒यम् । \newline
2. वा इ॒य मि॒यं ॅवै वा इ॒यꣳ सर्वा॒ सर्वे॒यं ॅवै वा इ॒यꣳ सर्वा᳚ । \newline
3. इ॒यꣳ सर्वा॒ सर्वे॒य मि॒यꣳ सर्वै॒ वैव सर्वे॒य मि॒यꣳ सर्वै॒व । \newline
4. सर्वै॒ वैव सर्वा॒ सर्वै॒व वेदि॒र् वेदि॑ रे॒व सर्वा॒ सर्वै॒व वेदिः॑ । \newline
5. ए॒व वेदि॒र् वेदि॑ रे॒वैव वेदि॒ रिय॒ती य॑ति॒ वेदि॑ रे॒वैव वेदि॒ रिय॑ति । \newline
6. वेदि॒ रिय॒ती य॑ति॒ वेदि॒र् वेदि॒ रिय॑ति शक्ष्यामि शक्ष्या॒ मीय॑ति॒ वेदि॒र् वेदि॒ रिय॑ति शक्ष्यामि । \newline
7. इय॑ति शक्ष्यामि शक्ष्या॒मी य॒ती य॑ति शक्ष्या॒ मीतीति॑ शक्ष्या॒ मीय॒ती य॑ति शक्ष्या॒ मीति॑ । \newline
8. श॒क्ष्या॒मी तीति॑ शक्ष्यामि शक्ष्या॒मीति॒ तु त्विति॑ शक्ष्यामि शक्ष्या॒मीति॒ तु । \newline
9. इति॒ तु त्वितीति॒ त्वै वै त्वितीति॒ त्वै । \newline
10. त्वै वै तु त्वा अ॑व॒माया॑ व॒माय॒ वै तु त्वा अ॑व॒माय॑ । \newline
11. वा अ॑व॒माया॑ व॒माय॒ वै वा अ॑व॒माय॑ यजन्ते यजन्ते ऽव॒माय॒ वै वा अ॑व॒माय॑ यजन्ते । \newline
12. अ॒व॒माय॑ यजन्ते यजन्ते ऽव॒माया॑ व॒माय॑ यजन्ते त्रिꣳ॒॒शत् त्रिꣳ॒॒शद् य॑जन्ते ऽव॒माया॑ व॒माय॑ यजन्ते त्रिꣳ॒॒शत् । \newline
13. अ॒व॒मायेत्य॑व - माय॑ । \newline
14. य॒ज॒न्ते॒ त्रिꣳ॒॒शत् त्रिꣳ॒॒शद् य॑जन्ते यजन्ते त्रिꣳ॒॒शत् प॒दानि॑ प॒दानि॑ त्रिꣳ॒॒शद् य॑जन्ते यजन्ते त्रिꣳ॒॒शत् प॒दानि॑ । \newline
15. त्रिꣳ॒॒शत् प॒दानि॑ प॒दानि॑ त्रिꣳ॒॒शत् त्रिꣳ॒॒शत् प॒दानि॑ प॒श्चात् प॒श्चात् प॒दानि॑ त्रिꣳ॒॒शत् त्रिꣳ॒॒शत् प॒दानि॑ प॒श्चात् । \newline
16. प॒दानि॑ प॒श्चात् प॒श्चात् प॒दानि॑ प॒दानि॑ प॒श्चात् ति॒रश्ची॑ ति॒रश्ची॑ प॒श्चात् प॒दानि॑ प॒दानि॑ प॒श्चात् ति॒रश्ची᳚ । \newline
17. प॒श्चात् ति॒रश्ची॑ ति॒रश्ची॑ प॒श्चात् प॒श्चात् ति॒रश्ची॑ भवति भवति ति॒रश्ची॑ प॒श्चात् प॒श्चात् ति॒रश्ची॑ भवति । \newline
18. ति॒रश्ची॑ भवति भवति ति॒रश्ची॑ ति॒रश्ची॑ भवति॒ षट्त्रिꣳ॑श॒ थ्षट्त्रिꣳ॑शद् भवति ति॒रश्ची॑ ति॒रश्ची॑ भवति॒ षट्त्रिꣳ॑शत् । \newline
19. भ॒व॒ति॒ षट्त्रिꣳ॑श॒ थ्षट्त्रिꣳ॑शद् भवति भवति॒ षट्त्रिꣳ॑श॒त् प्राची॒ प्राची॒ षट्त्रिꣳ॑शद् भवति भवति॒ षट्त्रिꣳ॑श॒त् प्राची᳚ । \newline
20. षट्त्रिꣳ॑श॒त् प्राची॒ प्राची॒ षट्त्रिꣳ॑श॒ थ्षट्त्रिꣳ॑श॒त् प्राची॒ चतु॑र्विꣳशति॒ श्चतु॑र्विꣳशतिः॒ प्राची॒ षट्त्रिꣳ॑श॒ थ्षट्त्रिꣳ॑श॒त् प्राची॒ चतु॑र्विꣳशतिः । \newline
21. षट्त्रिꣳ॑श॒दिति॒ षट् - त्रिꣳ॒॒श॒त् । \newline
22. प्राची॒ चतु॑र्विꣳशति॒ श्चतु॑र्विꣳशतिः॒ प्राची॒ प्राची॒ चतु॑र्विꣳशतिः पु॒रस्ता᳚त् पु॒रस्ता॒च् चतु॑र्विꣳशतिः॒ प्राची॒ प्राची॒ चतु॑र्विꣳशतिः पु॒रस्ता᳚त् । \newline
23. चतु॑र्विꣳशतिः पु॒रस्ता᳚त् पु॒रस्ता॒च् चतु॑र्विꣳशति॒ श्चतु॑र्विꣳशतिः पु॒रस्ता᳚त् ति॒रश्ची॑ ति॒रश्ची॑ पु॒रस्ता॒च् चतु॑र्विꣳशति॒ श्चतु॑र्विꣳशतिः पु॒रस्ता᳚त् ति॒रश्ची᳚ । \newline
24. चतु॑र्विꣳशति॒रिति॒ चतुः॑ - विꣳ॒॒श॒तिः॒ । \newline
25. पु॒रस्ता᳚त् ति॒रश्ची॑ ति॒रश्ची॑ पु॒रस्ता᳚त् पु॒रस्ता᳚त् ति॒रश्ची॒ दश॑दश॒ दश॑दश ति॒रश्ची॑ पु॒रस्ता᳚त् पु॒रस्ता᳚त् ति॒रश्ची॒ दश॑दश । \newline
26. ति॒रश्ची॒ दश॑दश॒ दश॑दश ति॒रश्ची॑ ति॒रश्ची॒ दश॑दश॒ सꣳ सम् दश॑दश ति॒रश्ची॑ ति॒रश्ची॒ दश॑दश॒ सम् । \newline
27. दश॑दश॒ सꣳ सम् दश॑दश॒ दश॑दश॒ सम् प॑द्यन्ते पद्यन्ते॒ सम् दश॑दश॒ दश॑दश॒ सम् प॑द्यन्ते । \newline
28. दश॑द॒शेति॒ दश॑ - द॒श॒ । \newline
29. सम् प॑द्यन्ते पद्यन्ते॒ सꣳ सम् प॑द्यन्ते॒ दशा᳚क्षरा॒ दशा᳚क्षरा पद्यन्ते॒ सꣳ सम् प॑द्यन्ते॒ दशा᳚क्षरा । \newline
30. प॒द्य॒न्ते॒ दशा᳚क्षरा॒ दशा᳚क्षरा पद्यन्ते पद्यन्ते॒ दशा᳚क्षरा वि॒राड् वि॒राड् दशा᳚क्षरा पद्यन्ते पद्यन्ते॒ दशा᳚क्षरा वि॒राट् । \newline
31. दशा᳚क्षरा वि॒राड् वि॒राड् दशा᳚क्षरा॒ दशा᳚क्षरा वि॒रा डन्न॒ मन्नं॑ ॅवि॒राड् दशा᳚क्षरा॒ दशा᳚क्षरा वि॒रा डन्न᳚म् । \newline
32. दशा᳚क्ष॒रेति॒ दश॑ - अ॒क्ष॒रा॒ । \newline
33. वि॒रा डन्न॒ मन्नं॑ ॅवि॒राड् वि॒रा डन्नं॑ ॅवि॒राड् वि॒रा डन्नं॑ ॅवि॒राड् वि॒रा डन्नं॑ ॅवि॒राट् । \newline
34. वि॒राडिति॑ वि - राट् । \newline
35. अन्नं॑ ॅवि॒राड् वि॒रा डन्न॒ मन्नं॑ ॅवि॒राड् वि॒राजा॑ वि॒राजा॑ वि॒रा डन्न॒ मन्नं॑ ॅवि॒राड् वि॒राजा᳚ । \newline
36. वि॒राड् वि॒राजा॑ वि॒राजा॑ वि॒राड् वि॒राड् वि॒राजै॒वैव वि॒राजा॑ वि॒राड् वि॒राड् वि॒राजै॒व । \newline
37. वि॒राडिति॑ वि - राट् । \newline
38. वि॒राजै॒वैव वि॒राजा॑ वि॒रा जै॒वान्नाद्य॑ म॒न्नाद्य॑ मे॒व वि॒राजा॑ वि॒रा जै॒वान्नाद्य᳚म् । \newline
39. वि॒राजेति॑ वि - राजा᳚ । \newline
40. ए॒वान्नाद्य॑ म॒न्नाद्य॑ मे॒वै वान्नाद्य॒ मवा वा॒न्नाद्य॑ मे॒वै वान्नाद्य॒ मव॑ । \newline
41. अ॒न्नाद्य॒ मवा वा॒न्नाद्य॑ म॒न्नाद्य॒ मव॑ रुन्धे रु॒न्धे ऽवा॒न्नाद्य॑ म॒न्नाद्य॒ मव॑ रुन्धे । \newline
42. अ॒न्नाद्य॒मित्य॑न्न - अद्य᳚म् । \newline
43. अव॑ रुन्धे रु॒न्धे ऽवाव॑ रुन्ध॒ उदुद् रु॒न्धे ऽवाव॑ रुन्ध॒ उत् । \newline
44. रु॒न्ध॒ उदुद् रु॑न्धे रुन्ध॒ उद्ध॑न्ति ह॒न् त्युद् रु॑न्धे रुन्ध॒ उद्ध॑न्ति । \newline
45. उद्ध॑न्ति ह॒न् त्युदु द्ध॑न्ति॒ यद् यद्ध॒न् त्युदु द्ध॑न्ति॒ यत् । \newline
46. ह॒न्ति॒ यद् यद्ध॑न्ति हन्ति॒ यदे॒ वैव यद्ध॑न्ति हन्ति॒ यदे॒व । \newline
47. यदे॒ वैव यद् यदे॒ वास्या॑ अस्या ए॒व यद् यदे॒ वास्याः᳚ । \newline
48. ए॒वास्या॑ अस्या ए॒वै वास्या॑ अमे॒द्ध्य म॑मे॒द्ध्य म॑स्या ए॒वै वास्या॑ अमे॒द्ध्यम् । \newline
49. अ॒स्या॒ अ॒मे॒द्ध्य म॑मे॒द्ध्य म॑स्या अस्या अमे॒द्ध्यम् तत् तद॑मे॒द्ध्य म॑स्या अस्या अमे॒द्ध्यम् तत् । \newline
50. अ॒मे॒द्ध्यम् तत् तद॑मे॒द्ध्य म॑मे॒द्ध्यम् तदपाप॒ तद॑मे॒द्ध्य म॑मे॒द्ध्यम् तदप॑ । \newline
51. तदपाप॒ तत् तदप॑ हन्ति ह॒न् त्यप॒ तत् तदप॑ हन्ति । \newline
52. अप॑ हन्ति ह॒न् त्यपाप॑ ह॒न्त्यु दुद्ध॒न् त्यपाप॑ ह॒न् त्युत् । \newline
53. ह॒न्त्यु दुद्ध॑न्ति ह॒न् त्युद्ध॑न्ति ह॒न् त्युद्ध॑न्ति ह॒न् त्युद्ध॑न्ति । \newline
54. उद्ध॑न्ति ह॒न् त्युदु द्ध॑न्ति॒ तस्मा॒त् तस्मा᳚ द्ध॒न् त्युदु द्ध॑न्ति॒ तस्मा᳚त् । \newline
55. ह॒न्ति॒ तस्मा॒त् तस्मा᳚ द्धन्ति हन्ति॒ तस्मा॒ दोष॑धय॒ ओष॑धय॒ स्तस्मा᳚ द्धन्ति हन्ति॒ तस्मा॒ दोष॑धयः । \newline
56. तस्मा॒ दोष॑धय॒ ओष॑धय॒ स्तस्मा॒त् तस्मा॒ दोष॑धयः॒ परा॒ परौष॑धय॒ स्तस्मा॒त् तस्मा॒ दोष॑धयः॒ परा᳚ । \newline
57. ओष॑धयः॒ परा॒ परौष॑धय॒ ओष॑धयः॒ परा॑ भवन्ति भवन्ति॒ परौष॑धय॒ ओष॑धयः॒ परा॑ भवन्ति । \newline
58. परा॑ भवन्ति भवन्ति॒ परा॒ परा॑ भवन्ति ब॒र्॒.हिर् ब॒र्॒.हिर् भ॑वन्ति॒ परा॒ परा॑ भवन्ति ब॒र्॒.हिः । \newline
59. भ॒व॒न्ति॒ ब॒र्॒.हिर् ब॒र्॒.हिर् भ॑वन्ति भवन्ति ब॒र्॒.हिः स्तृ॑णाति स्तृणाति ब॒र्॒.हिर् भ॑वन्ति भवन्ति ब॒र्॒.हिः स्तृ॑णाति । \newline
60. ब॒र्॒.हिः स्तृ॑णाति स्तृणाति ब॒र्॒.हिर् ब॒र्॒.हिः स्तृ॑णाति॒ तस्मा॒त् तस्मा᳚थ् स्तृणाति ब॒र्॒.हिर् ब॒र्॒.हिः स्तृ॑णाति॒ तस्मा᳚त् । \newline
61. स्तृ॒णा॒ति॒ तस्मा॒त् तस्मा᳚थ् स्तृणाति स्तृणाति॒ तस्मा॒ दोष॑धय॒ ओष॑धय॒ स्तस्मा᳚थ् स्तृणाति स्तृणाति॒ तस्मा॒ दोष॑धयः । \newline
62. तस्मा॒ दोष॑धय॒ ओष॑धय॒ स्तस्मा॒त् तस्मा॒ दोष॑धयः॒ पुनः॒ पुन॒ रोष॑धय॒ स्तस्मा॒त् तस्मा॒ दोष॑धयः॒ पुनः॑ । \newline
63. ओष॑धयः॒ पुनः॒ पुन॒ रोष॑धय॒ ओष॑धयः॒ पुन॒ रा पुन॒ रोष॑धय॒ ओष॑धयः॒ पुन॒ रा । \newline
64. पुन॒ रा पुनः॒ पुन॒ रा भ॑वन्ति भव॒न् त्या पुनः॒ पुन॒ रा भ॑वन्ति । \newline
65. आ भ॑वन्ति भव॒न्त्या भ॑व॒न् त्युत्त॑र॒ मुत्त॑रम् भव॒न्त्या भ॑व॒न् त्युत्त॑रम् । \newline
66. भ॒व॒न् त्युत्त॑र॒ मुत्त॑रम् भवन्ति भव॒न् त्युत्त॑रम् ब॒र्॒.हिषो॑ ब॒र्॒.हिष॒ उत्त॑रम् भवन्ति भव॒न् त्युत्त॑रम् ब॒र्॒.हिषः॑ । \newline
67. उत्त॑रम् ब॒र्॒.हिषो॑ ब॒र्॒.हिष॒ उत्त॑र॒ मुत्त॑रम् ब॒र्॒.हिष॑ उत्तरब॒र्॒.हि रु॑त्तरब॒र्॒.हिर् ब॒र्॒.हिष॒ उत्त॑र॒ मुत्त॑रम् ब॒र्॒.हिष॑ उत्तरब॒र्॒.हिः । \newline
68. उत्त॑र॒मित्युत् - त॒र॒म् । \newline
69. ब॒र्॒.हिष॑ उत्तरब॒र्॒.हि रु॑त्तरब॒र्॒.हिर् ब॒र्॒.हिषो॑ ब॒र्॒.हिष॑ उत्तरब॒र्॒.हिः स्तृ॑णाति स्तृणा त्युत्तरब॒र्॒.हिर् ब॒र्॒.हिषो॑ ब॒र्॒.हिष॑ उत्तरब॒र्॒.हिः स्तृ॑णाति । \newline
70. उ॒त्त॒र॒ब॒र्॒.हिः स्तृ॑णाति स्तृणा त्युत्तरब॒र्॒.हि रु॑त्तरब॒र्॒.हिः स्तृ॑णाति प्र॒जाः प्र॒जाः स्तृ॑णा त्युत्तरब॒र्॒.हि रु॑त्तरब॒र्॒.हिः स्तृ॑णाति प्र॒जाः । \newline
71. उ॒त्त॒र॒ब॒र्॒.हिरित्यु॑त्तर - ब॒र्॒.हिः । \newline
72. स्तृ॒णा॒ति॒ प्र॒जाः प्र॒जाः स्तृ॑णाति स्तृणाति प्र॒जा वै वै प्र॒जाः स्तृ॑णाति स्तृणाति प्र॒जा वै । \newline
73. प्र॒जा वै वै प्र॒जाः प्र॒जा वै ब॒र्॒.हिर् ब॒र्॒.हिर् वै प्र॒जाः प्र॒जा वै ब॒र्॒.हिः । \newline
74. प्र॒जा इति॑ प्र - जाः । \newline
75. वै ब॒र्॒.हिर् ब॒र्॒.हिर् वै वै ब॒र्॒.हिर् यज॑मानो॒ यज॑मानो ब॒र्॒.हिर् वै वै ब॒र्॒.हिर् यज॑मानः । \newline
76. ब॒र्॒.हिर् यज॑मानो॒ यज॑मानो ब॒र्॒.हिर् ब॒र्॒.हिर् यज॑मान उत्तरब॒र्॒.हि रु॑त्तरब॒र्॒.हिर् यज॑मानो ब॒र्॒.हिर् ब॒र्॒.हिर् यज॑मान उत्तरब॒र्॒.हिः । \newline
77. यज॑मान उत्तरब॒र्॒.हि रु॑त्तरब॒र्॒.हिर् यज॑मानो॒ यज॑मान उत्तरब॒र्॒.हिर् यज॑मानं॒ ॅयज॑मान मुत्तरब॒र्॒.हिर् यज॑मानो॒ यज॑मान उत्तरब॒र्॒.हिर् यज॑मानम् । \newline
78. उ॒त्त॒र॒ब॒र्॒.हिर् यज॑मानं॒ ॅयज॑मान मुत्तरब॒र्॒.हि रु॑त्तरब॒र्॒.हिर् यज॑मान मे॒वैव यज॑मान मुत्तरब॒र्॒.हि रु॑त्तरब॒र्॒.हिर् यज॑मान मे॒व । \newline
79. उ॒त्त॒र॒ब॒र॒.हिरित्यु॑त्तर - ब॒र॒.हिः । \newline
80. यज॑मान मे॒वैव यज॑मानं॒ ॅयज॑मान मे॒वा य॑जमाना॒ दय॑जमाना दे॒व यज॑मानं॒ ॅयज॑मान मे॒वा य॑जमानात् । \newline
81. ए॒वा य॑जमाना॒ दय॑जमाना दे॒वैवा य॑जमाना॒ दुत्त॑र॒ मुत्त॑र॒ मय॑जमाना दे॒वैवा य॑जमाना॒ दुत्त॑रम् । \newline
82. अय॑जमाना॒ दुत्त॑र॒ मुत्त॑र॒ मय॑जमाना॒ दय॑जमाना॒ दुत्त॑रम् करोति करो॒ त्युत्त॑र॒ मय॑जमाना॒ दय॑जमाना॒ दुत्त॑रम् करोति । \newline
83. उत्त॑रम् करोति करो॒ त्युत्त॑र॒ मुत्त॑रम् करोति॒ तस्मा॒त् तस्मा᳚त् करो॒ त्युत्त॑र॒ मुत्त॑रम् करोति॒ तस्मा᳚त् । \newline
84. उत्त॑र॒मित्युत् - त॒र॒म् । \newline
85. क॒रो॒ति॒ तस्मा॒त् तस्मा᳚त् करोति करोति॒ तस्मा॒द् यज॑मानो॒ यज॑मान॒ स्तस्मा᳚त् करोति करोति॒ तस्मा॒द् यज॑मानः । \newline
86. तस्मा॒द् यज॑मानो॒ यज॑मान॒ स्तस्मा॒त् तस्मा॒द् यज॑मा॒नो ऽय॑जमाना॒ दय॑जमाना॒द् यज॑मान॒ स्तस्मा॒त् तस्मा॒द् यज॑मा॒नो ऽय॑जमानात् । \newline
87. यज॑मा॒नो ऽय॑जमाना॒ दय॑जमाना॒द् यज॑मानो॒ यज॑मा॒नो ऽय॑जमाना॒ दुत्त॑र॒ उत्त॒रो ऽय॑जमाना॒द् यज॑मानो॒ यज॑मा॒नो ऽय॑जमाना॒ दुत्त॑रः । \newline
88. अय॑जमाना॒ दुत्त॑र॒ उत्त॒रो ऽय॑जमाना॒ दय॑जमाना॒ दुत्त॑रः । \newline
89. उत्त॑र॒ इत्युत् - त॒रः॒ । \newline
\pagebreak
\markright{ TS 6.2.5.1  \hfill https://www.vedavms.in \hfill}

\section{ TS 6.2.5.1 }

\textbf{TS 6.2.5.1 } \newline
\textbf{Samhita Paata} \newline

यद्वा अनी॑शानो भा॒रमा॑द॒त्ते वि वै स लि॑शते॒ यद् द्वाद॑श सा॒ह्नस्यो॑प॒सदः॒ स्युस्ति॒स्त्रो॑ऽहीन॑स्य य॒ज्ञ्स्य॒ विलो॑म क्रियेत ति॒स्र ए॒व सा॒ह्नस्यो॑प॒सदो॒ द्वाद॑शा॒हीन॑स्य य॒ज्ञ्स्य॑ सवीर्य॒त्वायाथो॒ सलो॑म क्रियते व॒थ्सस्यैकः॒ स्तनो॑ भा॒गी हि सोऽथैकꣳ॒॒ स्तनं॑ ॅव्र॒तमुपै॒त्यथ॒ द्वावथ॒ त्रीनथ॑ च॒तुर॑ एतद्वै- [  ] \newline

\textbf{Pada Paata} \newline

यत् । वै । अनी॑शानः । भा॒रम् । आ॒द॒त्त इत्या᳚ - द॒त्ते । वीति॑ । वै । सः । लि॒श॒ते॒ । यत् । द्वाद॑श । सा॒ह्नस्येति॑ स - अ॒ह्नस्य॑ । उ॒प॒सद॒ इत्यु॑प - सदः॑ । स्युः । ति॒स्त्रः । अ॒हीन॑स्य । य॒ज्ञ्स्य॑ । विलो॒मेति॒ वि - लो॒म॒ । क्रि॒ये॒त॒ । ति॒स्रः । ए॒व । सा॒ह्नस्येति॑ स - अ॒ह्नस्य॑ । उ॒प॒सद॒ इत्यु॑प - सदः॑ । द्वाद॑श । अ॒हीन॑स्य । य॒ज्ञ्स्य॑ । स॒वी॒र्य॒त्वायेति॑ सवीर्य - त्वाय॑ । अथो॒ इति॑ । सलो॒मेति॒ स-लो॒म॒ । क्रि॒य॒ते॒ । व॒थ्सस्य॑ । एकः॑ । स्तनः॑ । भा॒गी । हि । सः । अथ॑ । एक᳚म् । स्तन᳚म् । व्र॒तम् । उपेति॑ । ए॒ति॒ । अथ॑ । द्वौ । अथ॑ । त्रीन् । अथ॑ । च॒तुरः॑ । ए॒तत् । वै ।  \newline


\textbf{Krama Paata} \newline

यद् वै । वा अनी॑शानः । अनी॑शानो भा॒रम् । भा॒रमा॑द॒त्ते । आ॒द॒त्ते वि । आ॒द॒त्त इत्या᳚ - द॒त्ते । वि वै । वै सः । स लि॑शते । लि॒श॒ते॒ यत् । यद् द्वाद॑श । द्वाद॑श सा॒ह्नस्य॑ । सा॒ह्नस्यो॑प॒सदः॑ । सा॒ह्नस्येति॑ स - अ॒ह्नस्य॑ । उ॒प॒सदः॒ स्युः । उ॒प॒सद॒ इत्यु॑प - सदः॑ । स्युस्ति॒स्रः । ति॒स्रो॑ऽहीन॑स्य । अ॒हीन॑स्य य॒ज्ञ्स्य॑ । य॒ज्ञ्स्य॒ विलो॑म । विलो॑म क्रियेत । विलो॒मेति॒ वि - लो॒म॒ । क्रि॒ये॒त॒ ति॒स्रः । ति॒स्र ए॒व । ए॒व सा॒ह्नस्य॑ । सा॒ह्नस्यो॑प॒सदः॑ । सा॒ह्नस्येति॑ स - अ॒ह्नस्य॑ । उ॒प॒सदो॒ द्वाद॑श । उ॒प॒सद॒ इत्यु॑प - सदः॑ । द्वाद॑शा॒हीन॑स्य । अ॒हीन॑स्य य॒ज्ञ्स्य॑ । य॒ज्ञ्स्य॑ सवीर्य॒त्वाय॑ । स॒वी॒र्य॒त्वायाथो᳚ । स॒वी॒र्य॒त्वायेति॑ सवीर्य - त्वाय॑ । अथो॒ सलो॑म । अथो॒ इत्यथो᳚ । सलो॑म क्रियते । सलो॒मेति॒ स - लो॒म॒ । क्रि॒य॒ते॒ व॒थ्सस्य॑ । व॒थ्सस्यैकः॑ । एकः॒ स्तनः॑ । स्तनो॑ भा॒गी । भा॒गी हि । हि सः । सोऽथ॑ । अथैक᳚म् । एकꣳ॒॒ स्तन᳚म् । स्तन॑म् ॅव्र॒तम् । व्र॒तमुप॑ । उपै॑ति । ए॒त्यथ॑ । अथ॒ द्वौ । द्वावथ॑ । अथ॒ त्रीन् । त्रीनथ॑ । अथ॑ च॒तुरः॑ । च॒तुर॑ ए॒तत् । ए॒तद् वै । वै क्षु॒रप॑वि \newline

\textbf{Jatai Paata} \newline

1. यद् वै वै यद् यद् वै । \newline
2. वा अनी॑शा॒नो ऽनी॑शानो॒ वै वा अनी॑शानः । \newline
3. अनी॑शानो भा॒रम् भा॒र मनी॑शा॒नो ऽनी॑शानो भा॒रम् । \newline
4. भा॒र मा॑द॒त्त आ॑द॒त्ते भा॒रम् भा॒र मा॑द॒त्ते । \newline
5. आ॒द॒त्ते वि व्या॑द॒त्त आ॑द॒त्ते वि । \newline
6. आ॒द॒त्त इत्या᳚ - द॒त्ते । \newline
7. वि वै वै वि वि वै । \newline
8. वै स स वै वै सः । \newline
9. स लि॑शते लिशते॒ स स लि॑शते । \newline
10. लि॒श॒ते॒ यद् यल्लि॑शते लिशते॒ यत् । \newline
11. यद् द्वाद॑श॒ द्वाद॑श॒ यद् यद् द्वाद॑श । \newline
12. द्वाद॑श सा॒ह्नस्य॑ सा॒ह्नस्य॒ द्वाद॑श॒ द्वाद॑श सा॒ह्नस्य॑ । \newline
13. सा॒ह्न स्यो॑प॒सद॑ उप॒सदः॑ सा॒ह्नस्य॑ सा॒ह्न स्यो॑प॒सदः॑ । \newline
14. सा॒ह्नस्येति॑ स - अ॒ह्नस्य॑ । \newline
15. उ॒प॒सदः॒ स्युः स्यु रु॑प॒सद॑ उप॒सदः॒ स्युः । \newline
16. उ॒प॒सद॒ इत्यु॑प - सदः॑ । \newline
17. स्यु स्ति॒स्र स्ति॒स्रः स्युः स्यु स्ति॒स्रः । \newline
18. ति॒स्रो॑ ऽहीन॑स्या॒ हीन॑स्य ति॒स्र स्ति॒स्रो॑ ऽहीन॑स्य । \newline
19. अ॒हीन॑स्य य॒ज्ञ्स्य॑ य॒ज्ञ्स्या॒ हीन॑स्या॒ हीन॑स्य य॒ज्ञ्स्य॑ । \newline
20. य॒ज्ञ्स्य॒ विलो॑म॒ विलो॑म य॒ज्ञ्स्य॑ य॒ज्ञ्स्य॒ विलो॑म । \newline
21. विलो॑म क्रियेत क्रियेत॒ विलो॑म॒ विलो॑म क्रियेत । \newline
22. विलो॒मेति॒ वि - लो॒म॒ । \newline
23. क्रि॒ये॒त॒ ति॒स्र स्ति॒स्रः क्रि॑येत क्रियेत ति॒स्रः । \newline
24. ति॒स्र ए॒वैव ति॒स्र स्ति॒स्र ए॒व । \newline
25. ए॒व सा॒ह्नस्य॑ सा॒ह्न स्यै॒वैव सा॒ह्नस्य॑ । \newline
26. सा॒ह्न स्यो॑प॒सद॑ उप॒सदः॑ सा॒ह्नस्य॑ सा॒ह्न स्यो॑प॒सदः॑ । \newline
27. सा॒ह्नस्येति॑ स - अ॒ह्नस्य॑ । \newline
28. उ॒प॒सदो॒ द्वाद॑श॒ द्वाद॑शोप॒सद॑ उप॒सदो॒ द्वाद॑श । \newline
29. उ॒प॒सद॒ इत्यु॑प - सदः॑ । \newline
30. द्वाद॑शा॒ हीन॑स्या॒ हीन॑स्य॒ द्वाद॑श॒ द्वाद॑शा॒ हीन॑स्य । \newline
31. अ॒हीन॑स्य य॒ज्ञ्स्य॑ य॒ज्ञ्स्या॒ हीन॑स्या॒ हीन॑स्य य॒ज्ञ्स्य॑ । \newline
32. य॒ज्ञ्स्य॑ सवीर्य॒त्वाय॑ सवीर्य॒त्वाय॑ य॒ज्ञ्स्य॑ य॒ज्ञ्स्य॑ सवीर्य॒त्वाय॑ । \newline
33. स॒वी॒र्य॒त्वायाथो॒ अथो॑ सवीर्य॒त्वाय॑ सवीर्य॒त्वायाथो᳚ । \newline
34. स॒वी॒र्य॒त्वायेति॑ सवीर्य - त्वाय॑ । \newline
35. अथो॒ सलो॑म॒ सलो॒माथो॒ अथो॒ सलो॑म । \newline
36. अथो॒ इत्यथो᳚ । \newline
37. सलो॑म क्रियते क्रियते॒ सलो॑म॒ सलो॑म क्रियते । \newline
38. सलो॒मेति॒ स - लो॒म॒ । \newline
39. क्रि॒य॒ते॒ व॒थ्सस्य॑ व॒थ्सस्य॑ क्रियते क्रियते व॒थ्सस्य॑ । \newline
40. व॒थ्सस्यैक॒ एको॑ व॒थ्सस्य॑ व॒थ्सस्यैकः॑ । \newline
41. एकः॒ स्तनः॒ स्तन॒ एक॒ एकः॒ स्तनः॑ । \newline
42. स्तनो॑ भा॒गी भा॒गी स्तनः॒ स्तनो॑ भा॒गी । \newline
43. भा॒गी हि हि भा॒गी भा॒गी हि । \newline
44. हि स स हि हि सः । \newline
45. सो ऽथाथ॒ स सो ऽथ॑ । \newline
46. अथैक॒ मेक॒ मथाथैक᳚म् । \newline
47. एकꣳ॒॒ स्तनꣳ॒॒ स्तन॒ मेक॒ मेकꣳ॒॒ स्तन᳚म् । \newline
48. स्तनं॑ ॅव्र॒तं ॅव्र॒तꣳ स्तनꣳ॒॒ स्तनं॑ ॅव्र॒तम् । \newline
49. व्र॒त मुपोप॑ व्र॒तं ॅव्र॒त मुप॑ । \newline
50. उपै᳚त्ये॒ त्युपोपै॑ति । \newline
51. ए॒त्यथा थै᳚त्ये॒ त्यथ॑ । \newline
52. अथ॒ द्वौ द्वा वथाथ॒ द्वौ । \newline
53. द्वा वथाथ॒ द्वौ द्वा वथ॑ । \newline
54. अथ॒ त्रीꣳ स्त्री नथाथ॒ त्रीन् । \newline
55. त्री नथाथ॒ त्रीꣳ स्त्री नथ॑ । \newline
56. अथ॑ च॒तुर॑ श्च॒तुरो ऽथाथ॑ च॒तुरः॑ । \newline
57. च॒तुर॑ ए॒त दे॒तच् च॒तुर॑ श्च॒तुर॑ ए॒तत् । \newline
58. ए॒तद् वै वा ए॒त दे॒तद् वै । \newline
59. वै क्षु॒रप॑वि क्षु॒रप॑वि॒ वै वै क्षु॒रप॑वि । \newline

\textbf{Ghana Paata } \newline

1. यद् वै वै यद् यद् वा अनी॑शा॒नो ऽनी॑शानो॒ वै यद् यद् वा अनी॑शानः । \newline
2. वा अनी॑शा॒नो ऽनी॑शानो॒ वै वा अनी॑शानो भा॒रम् भा॒र मनी॑शानो॒ वै वा अनी॑शानो भा॒रम् । \newline
3. अनी॑शानो भा॒रम् भा॒र मनी॑शा॒नो ऽनी॑शानो भा॒र मा॑द॒त्त आ॑द॒त्ते भा॒र मनी॑शा॒नो ऽनी॑शानो भा॒र मा॑द॒त्ते । \newline
4. भा॒र मा॑द॒त्त आ॑द॒त्ते भा॒रम् भा॒र मा॑द॒त्ते वि व्या॑द॒त्ते भा॒रम् भा॒र मा॑द॒त्ते वि । \newline
5. आ॒द॒त्ते वि व्या॑द॒त्त आ॑द॒त्ते वि वै वै व्या॑द॒त्त आ॑द॒त्ते वि वै । \newline
6. आ॒द॒त्त इत्या᳚ - द॒त्ते । \newline
7. वि वै वै वि वि वै स स वै वि वि वै सः । \newline
8. वै स स वै वै स लि॑शते लिशते॒ स वै वै स लि॑शते । \newline
9. स लि॑शते लिशते॒ स स लि॑शते॒ यद् यल्लि॑शते॒ स स लि॑शते॒ यत् । \newline
10. लि॒श॒ते॒ यद् यल्लि॑शते लिशते॒ यद् द्वाद॑श॒ द्वाद॑श॒ यल्लि॑शते लिशते॒ यद् द्वाद॑श । \newline
11. यद् द्वाद॑श॒ द्वाद॑श॒ यद् यद् द्वाद॑श सा॒ह्नस्य॑ सा॒ह्नस्य॒ द्वाद॑श॒ यद् यद् द्वाद॑श सा॒ह्नस्य॑ । \newline
12. द्वाद॑श सा॒ह्नस्य॑ सा॒ह्नस्य॒ द्वाद॑श॒ द्वाद॑श सा॒ह्नस्यो॑ प॒सद॑ उप॒सदः॑ सा॒ह्नस्य॒ द्वाद॑श॒ द्वाद॑श सा॒ह्नस्यो॑ प॒सदः॑ । \newline
13. सा॒ह्नस्यो॑ प॒सद॑ उप॒सदः॑ सा॒ह्नस्य॑ सा॒ह्नस्यो॑ प॒सदः॒ स्युः स्युरु॑प॒सदः॑ सा॒ह्नस्य॑ सा॒ह्नस्यो॑ प॒सदः॒ स्युः । \newline
14. सा॒ह्नस्येति॑ स - अ॒ह्नस्य॑ । \newline
15. उ॒प॒सदः॒ स्युः स्यु रु॑प॒सद॑ उप॒सदः॒ स्यु स्ति॒स्र स्ति॒स्रः स्यु रु॑प॒सद॑ उप॒सदः॒ स्यु स्ति॒स्रः । \newline
16. उ॒प॒सद॒ इत्यु॑प - सदः॑ । \newline
17. स्यु स्ति॒स्र स्ति॒स्रः स्युः स्यु स्ति॒स्रो॑ ऽहीन॑स्या॒ हीन॑स्य ति॒स्रः स्युः स्यु स्ति॒स्रो॑ ऽहीन॑स्य । \newline
18. ति॒स्रो॑ ऽहीन॑स्या॒ हीन॑स्य ति॒स्र स्ति॒स्रो॑ ऽहीन॑स्य य॒ज्ञ्स्य॑ य॒ज्ञ्स्या॒ हीन॑स्य ति॒स्र स्ति॒स्रो॑ ऽहीन॑स्य य॒ज्ञ्स्य॑ । \newline
19. अ॒हीन॑स्य य॒ज्ञ्स्य॑ य॒ज्ञ्स्या॒ हीन॑स्या॒ हीन॑स्य य॒ज्ञ्स्य॒ विलो॑म॒ विलो॑म य॒ज्ञ्स्या॒ हीन॑स्या॒ हीन॑स्य य॒ज्ञ्स्य॒ विलो॑म । \newline
20. य॒ज्ञ्स्य॒ विलो॑म॒ विलो॑म य॒ज्ञ्स्य॑ य॒ज्ञ्स्य॒ विलो॑म क्रियेत क्रियेत॒ विलो॑म य॒ज्ञ्स्य॑ य॒ज्ञ्स्य॒ विलो॑म क्रियेत । \newline
21. विलो॑म क्रियेत क्रियेत॒ विलो॑म॒ विलो॑म क्रियेत ति॒स्र स्ति॒स्रः क्रि॑येत॒ विलो॑म॒ विलो॑म क्रियेत ति॒स्रः । \newline
22. विलो॒मेति॒ वि - लो॒म॒ । \newline
23. क्रि॒ये॒त॒ ति॒स्र स्ति॒स्रः क्रि॑येत क्रियेत ति॒स्र ए॒वैव ति॒स्रः क्रि॑येत क्रियेत ति॒स्र ए॒व । \newline
24. ति॒स्र ए॒वैव ति॒स्र स्ति॒स्र ए॒व सा॒ह्नस्य॑ सा॒ह्न स्यै॒व ति॒स्र स्ति॒स्र ए॒व सा॒ह्नस्य॑ । \newline
25. ए॒व सा॒ह्नस्य॑ सा॒ह्नस्यै॒वैव सा॒ह्न स्यो॑प॒सद॑ उप॒सदः॑ सा॒ह्नस्यै॒वैव सा॒ह्न स्यो॑प॒सदः॑ । \newline
26. सा॒ह्न स्यो॑प॒सद॑ उप॒सदः॑ सा॒ह्नस्य॑ सा॒ह्न स्यो॑प॒सदो॒ द्वाद॑श॒ द्वाद॑ शोप॒सदः॑ सा॒ह्नस्य॑ सा॒ह्न स्यो॑प॒सदो॒ द्वाद॑श । \newline
27. सा॒ह्नस्येति॑ स - अ॒ह्नस्य॑ । \newline
28. उ॒प॒सदो॒ द्वाद॑श॒ द्वाद॑शोप॒सद॑ उप॒सदो॒ द्वाद॑शा॒ हीन॑स्या॒ हीन॑स्य॒ द्वाद॑शोप॒सद॑ उप॒सदो॒ द्वाद॑शा॒ हीन॑स्य । \newline
29. उ॒प॒सद॒ इत्यु॑प - सदः॑ । \newline
30. द्वाद॑शा॒ हीन॑स्या॒ हीन॑स्य॒ द्वाद॑श॒ द्वाद॑शा॒ हीन॑स्य य॒ज्ञ्स्य॑ य॒ज्ञ्स्या॒ हीन॑स्य॒ द्वाद॑श॒ द्वाद॑शा॒ हीन॑स्य य॒ज्ञ्स्य॑ । \newline
31. अ॒हीन॑स्य य॒ज्ञ्स्य॑ य॒ज्ञ्स्या॒ हीन॑स्या॒ हीन॑स्य य॒ज्ञ्स्य॑ सवीर्य॒त्वाय॑ सवीर्य॒त्वाय॑ य॒ज्ञ्स्या॒
हीन॑स्या॒ही न॑स्य य॒ज्ञ्स्य॑ सवीर्य॒त्वाय॑ । \newline
32. य॒ज्ञ्स्य॑ सवीर्य॒त्वाय॑ सवीर्य॒त्वाय॑ य॒ज्ञ्स्य॑ य॒ज्ञ्स्य॑ सवीर्य॒त्वा याथो॒ अथो॑ सवीर्य॒त्वाय॑ य॒ज्ञ्स्य॑ य॒ज्ञ्स्य॑ सवीर्य॒त्वा याथो᳚ । \newline
33. स॒वी॒र्य॒त्वायाथो॒ अथो॑ सवीर्य॒त्वाय॑ सवीर्य॒त्वायाथो॒ सलो॑म॒ सलो॒माथो॑ सवीर्य॒त्वाय॑ सवीर्य॒त्वायाथो॒ सलो॑म । \newline
34. स॒वी॒र्य॒त्वायेति॑ सवीर्य - त्वाय॑ । \newline
35. अथो॒ सलो॑म॒ सलो॒ माथो॒ अथो॒ सलो॑म क्रियते क्रियते॒ सलो॒ माथो॒ अथो॒ सलो॑म क्रियते । \newline
36. अथो॒ इत्यथो᳚ । \newline
37. सलो॑म क्रियते क्रियते॒ सलो॑म॒ सलो॑म क्रियते व॒थ्सस्य॑ व॒थ्सस्य॑ क्रियते॒ सलो॑म॒ सलो॑म क्रियते व॒थ्सस्य॑ । \newline
38. सलो॒मेति॒ स - लो॒म॒ । \newline
39. क्रि॒य॒ते॒ व॒थ्सस्य॑ व॒थ्सस्य॑ क्रियते क्रियते व॒थ्सस्यैक॒ एको॑ व॒थ्सस्य॑ क्रियते क्रियते व॒थ्स स्यैकः॑ । \newline
40. व॒थ्स स्यैक॒ एको॑ व॒थ्सस्य॑ व॒थ्स स्यैकः॒ स्तनः॒ स्तन॒ एको॑ व॒थ्सस्य॑ व॒थ्स स्यैकः॒ स्तनः॑ । \newline
41. एकः॒ स्तनः॒ स्तन॒ एक॒ एकः॒ स्तनो॑ भा॒गी भा॒गी स्तन॒ एक॒ एकः॒ स्तनो॑ भा॒गी । \newline
42. स्तनो॑ भा॒गी भा॒गी स्तनः॒ स्तनो॑ भा॒गी हि हि भा॒गी स्तनः॒ स्तनो॑ भा॒गी हि । \newline
43. भा॒गी हि हि भा॒गी भा॒गी हि स स हि भा॒गी भा॒गी हि सः । \newline
44. हि स स हि हि सो ऽथाथ॒ स हि हि सो ऽथ॑ । \newline
45. सो ऽथाथ॒ स सो ऽथैक॒ मेक॒ मथ॒ स सो ऽथैक᳚म् । \newline
46. अथैक॒ मेक॒ मथा थैकꣳ॒॒ स्तनꣳ॒॒ स्तन॒ मेक॒ मथा थैकꣳ॒॒ स्तन᳚म् । \newline
47. एकꣳ॒॒ स्तनꣳ॒॒ स्तन॒ मेक॒ मेकꣳ॒॒ स्तनं॑ ॅव्र॒तं ॅव्र॒तꣳ स्तन॒ मेक॒ मेकꣳ॒॒ स्तनं॑ ॅव्र॒तम् । \newline
48. स्तनं॑ ॅव्र॒तं ॅव्र॒तꣳ स्तनꣳ॒॒ स्तनं॑ ॅव्र॒त मुपोप॑ व्र॒तꣳ स्तनꣳ॒॒ स्तनं॑ ॅव्र॒त मुप॑ । \newline
49. व्र॒त मुपोप॑ व्र॒तं ॅव्र॒त मुपै᳚ त्ये॒त्युप॑ व्र॒तं ॅव्र॒त मुपै॑ति । \newline
50. उपै᳚ त्ये॒त्युपो पै॒त्यथा थै॒त्युपो पै॒त्यथ॑ । \newline
51. ए॒त्यथा थै᳚त्ये॒त्यथ॒ द्वौ द्वा वथै᳚ त्ये॒त्यथ॒ द्वौ । \newline
52. अथ॒ द्वौ द्वा वथाथ॒ द्वा वथाथ॒ द्वा वथाथ॒ द्वा वथ॑ । \newline
53. द्वा वथाथ॒ द्वौ द्वा वथ॒ त्रीꣳ स्त्रीनथ॒ द्वौ द्वा वथ॒ त्रीन् । \newline
54. अथ॒ त्रीꣳ स्त्रीनथाथ॒ त्रीन थाथ॒ त्रीनथाथ॒ त्रीनथ॑ । \newline
55. त्रीनथाथ॒ त्रीꣳ स्त्रीनथ॑ च॒तुर॑ श्च॒तुरो ऽथ॒ त्रीꣳ स्त्रीनथ॑ च॒तुरः॑ । \newline
56. अथ॑ च॒तुर॑ श्च॒तुरो ऽथाथ॑ च॒तुर॑ ए॒त दे॒तच् च॒तुरो ऽथाथ॑ च॒तुर॑ ए॒तत् । \newline
57. च॒तुर॑ ए॒त दे॒तच् च॒तुर॑ श्च॒तुर॑ ए॒तद् वै वा ए॒तच् च॒तुर॑ श्च॒तुर॑ ए॒तद् वै । \newline
58. ए॒तद् वै वा ए॒त दे॒तद् वै क्षु॒रप॑वि क्षु॒रप॑वि॒ वा ए॒त दे॒तद् वै क्षु॒रप॑वि । \newline
59. वै क्षु॒रप॑वि क्षु॒रप॑वि॒ वै वै क्षु॒रप॑वि॒ नाम॒ नाम॑ क्षु॒रप॑वि॒ वै वै क्षु॒रप॑वि॒ नाम॑ । \newline
\pagebreak
\markright{ TS 6.2.5.2  \hfill https://www.vedavms.in \hfill}

\section{ TS 6.2.5.2 }

\textbf{TS 6.2.5.2 } \newline
\textbf{Samhita Paata} \newline

क्षु॒रप॑वि॒ नाम॑ व्र॒तं ॅयेन॒ प्र जा॒तान् भ्रातृ॑व्यान् नु॒दते॒ प्रति॑ जनि॒ष्यमा॑णा॒नथो॒ कनी॑यसै॒व भूय॒ उपै॑ति च॒तुरोऽग्रे॒ स्तना᳚न् व्र॒तमुपै॒त्यथ॒ त्रीनथ॒ द्वावथैक॑मे॒तद्वै सु॑जघ॒नं नाम॑ व्र॒तं त॑प॒स्यꣳ॑ सुव॒र्ग्य॑मथो॒ प्रैव जा॑यते प्र॒जया॑ प॒शुभि॑र्यवा॒गू रा॑ज॒न्य॑स्य व्र॒तं क्रू॒रेव॒ वै य॑वा॒गूः क्रू॒र इ॑व- [  ] \newline

\textbf{Pada Paata} \newline

क्षु॒रप॒वीति॑ क्षु॒र - प॒वि॒ । नाम॑ । व्र॒तम् । येन॑ । प्रेति॑ । जा॒तान् । भ्रातृ॑व्यान् । नु॒दते᳚ । प्रतीति॑ । ज॒नि॒ष्यमा॑णान् । अथो॒ इति॑ । कनी॑यसा । ए॒व । भूयः॑ । उपेति॑ । ए॒ति॒ । च॒तुरः॑ । अग्रे᳚ । स्तनान्॑ । व्र॒तम् । उपेति॑ । ए॒ति॒ । अथ॑ । त्रीन् । अथ॑ । द्वौ । अथ॑ । एक᳚म् । ए॒तत् । वै । सु॒ज॒घ॒नमिति॑ सु - ज॒घ॒नम् । नाम॑ । व्र॒तम् । त॒प॒स्य᳚म् । सु॒व॒र्ग्य॑मिति॑ सुवः - ग्य᳚म् । अथो॒ इति॑ । प्रेति॑ । ए॒व । जा॒य॒ते॒ । प्र॒जयेति॑ प्र - जया᳚ । प॒शुभि॒रिति॑ प॒शु - भिः॒ । य॒वा॒गूः । रा॒ज॒न्य॑स्य । व्र॒तम् । क्रू॒रा । इ॒व॒ । वै । य॒वा॒गूः । क्रू॒रः । इ॒व॒ ।  \newline


\textbf{Krama Paata} \newline

क्षु॒रप॑वि॒ नाम॑ । क्षु॒रप॒वीति॑ क्षु॒र - प॒वि॒ । नाम॑ व्र॒तम् । व्र॒तम् ॅयेन॑ । येन॒ प्र । प्र जा॒तान् । 
जा॒तान् भ्रातृ॑व्यान् । भ्रातृ॑व्यान् नु॒दते᳚ । नु॒दते॒ प्रति॑ । प्रति॑ जनि॒ष्यमा॑णान् । ज॒नि॒ष्यमा॑णा॒नथो᳚ । अथो॒ कनी॑यसा । अथो॒ इत्यथो᳚ । कनी॑यसै॒व । ए॒व भूयः॑ । भूय॒ उप॑ । उपै॑ति । ए॒ति॒ च॒तुरः॑ । च॒तुरोऽग्रे᳚ । अग्रे॒ स्तनान्॑ । स्तना᳚न् व्र॒तम् । व्र॒तमुप॑ । उपै॑ति । ए॒त्यथ॑ । अथ॒ त्रीन् । त्रीनथ॑ । अथ॒ द्वौ । द्वावथ॑ । अथैक᳚म् । एक॑मे॒तत् । ए॒तद् वै । वै सु॑जघ॒नम् । सु॒ज॒घ॒नम् नाम॑ । सु॒ज॒घ॒नमिति॑ सु - ज॒घ॒नम् । नाम॑ व्र॒तम् । व्र॒तम् त॑प॒स्य᳚म् । त॒प॒स्यꣳ॑ सुव॒र्ग्य᳚म् । सु॒व॒र्ग्य॑मथो᳚ । सु॒व॒र्ग्य॑मिति॑ सुवः - ग्य᳚म् । अथो॒ प्र । अथो॒ इत्यथो᳚ । प्रैव । ए॒व जा॑यते । जा॒य॒ते॒ प्र॒जया᳚ । प्र॒जया॑ प॒शुभिः॑ । प्र॒जयेति॑ प्र - जया᳚ । प॒शुभि॑र् यवा॒गूः । प॒शुभि॒रिति॑ प॒शु - भिः॒ । य॒वा॒गू रा॑ज॒न्य॑स्य । रा॒ज॒न्य॑स्य व्र॒तम् । व्र॒तम् क्रू॒रा । क्रू॒रेव॑ । इ॒व॒ वै । वै य॑वा॒गूः । य॒वा॒गूः क्रू॒रः । क्रू॒र इ॑व । इ॒व॒ रा॒ज॒न्यः॑ \newline

\textbf{Jatai Paata} \newline

1. क्षु॒रप॑वि॒ नाम॒ नाम॑ क्षु॒रप॑वि क्षु॒रप॑वि॒ नाम॑ । \newline
2. क्षु॒रप॒वीति॑ क्षु॒र - प॒वि॒ । \newline
3. नाम॑ व्र॒तं ॅव्र॒तम् नाम॒ नाम॑ व्र॒तम् । \newline
4. व्र॒तं ॅयेन॒ येन॑ व्र॒तं ॅव्र॒तं ॅयेन॑ । \newline
5. येन॒ प्र प्र येन॒ येन॒ प्र । \newline
6. प्र जा॒तान् जा॒तान् प्र प्र जा॒तान् । \newline
7. जा॒तान् भ्रातृ॑व्या॒न् भ्रातृ॑व्यान् जा॒तान् जा॒तान् भ्रातृ॑व्यान् । \newline
8. भ्रातृ॑व्यान् नु॒दते॑ नु॒दते॒ भ्रातृ॑व्या॒न् भ्रातृ॑व्यान् नु॒दते᳚ । \newline
9. नु॒दते॒ प्रति॒ प्रति॑ नु॒दते॑ नु॒दते॒ प्रति॑ । \newline
10. प्रति॑ जनि॒ष्यमा॑णान् जनि॒ष्यमा॑णा॒न् प्रति॒ प्रति॑ जनि॒ष्यमा॑णान् । \newline
11. ज॒नि॒ष्यमा॑णा॒ नथो॒ अथो॑ जनि॒ष्यमा॑णान् जनि॒ष्यमा॑णा॒ नथो᳚ । \newline
12. अथो॒ कनी॑यसा॒ कनी॑य॒सा ऽथो॒ अथो॒ कनी॑यसा । \newline
13. अथो॒ इत्यथो᳚ । \newline
14. कनी॑य सै॒वैव कनी॑यसा॒ कनी॑य सै॒व । \newline
15. ए॒व भूयो॒ भूय॑ ए॒वैव भूयः॑ । \newline
16. भूय॒ उपोप॒ भूयो॒ भूय॒ उप॑ । \newline
17. उपै᳚त्ये॒ त्युपोपै॑ति । \newline
18. ए॒ति॒ च॒तुर॑ श्च॒तुर॑ एत्येति च॒तुरः॑ । \newline
19. च॒तुरो ऽग्रे ऽग्रे॑ च॒तुर॑ श्च॒तुरो ऽग्रे᳚ । \newline
20. अग्रे॒ स्तना॒न् थ्स्तना॒ नग्रे ऽग्रे॒ स्तनान्॑ । \newline
21. स्तना᳚न् व्र॒तं ॅव्र॒तꣳ स्तना॒न् थ्स्तना᳚न् व्र॒तम् । \newline
22. व्र॒त मुपोप॑ व्र॒तं ॅव्र॒त मुप॑ । \newline
23. उपै᳚त्ये॒ त्युपोपै॑ति । \newline
24. ए॒त्यथा थै᳚त्ये॒ त्यथ॑ । \newline
25. अथ॒ त्रीꣳ स्त्री नथाथ॒ त्रीन् । \newline
26. त्री नथाथ॒ त्रीꣳ स्त्री नथ॑ । \newline
27. अथ॒ द्वौ द्वा वथाथ॒ द्वौ । \newline
28. द्वा वथाथ॒ द्वौ द्वा वथ॑ । \newline
29. अथैक॒ मेक॒ मथाथैक᳚म् । \newline
30. एक॑ मे॒त दे॒त देक॒ मेक॑ मे॒तत् । \newline
31. ए॒तद् वै वा ए॒त दे॒तद् वै । \newline
32. वै सु॑जघ॒नꣳ सु॑जघ॒नं ॅवै वै सु॑जघ॒नम् । \newline
33. सु॒ज॒घ॒नम् नाम॒ नाम॑ सुजघ॒नꣳ सु॑जघ॒नम् नाम॑ । \newline
34. सु॒ज॒घ॒नमिति॑ सु - ज॒घ॒नम् । \newline
35. नाम॑ व्र॒तं ॅव्र॒तम् नाम॒ नाम॑ व्र॒तम् । \newline
36. व्र॒तम् त॑प॒स्य॑म् तप॒स्यं॑ ॅव्र॒तं ॅव्र॒तम् त॑प॒स्य᳚म् । \newline
37. त॒प॒स्यꣳ॑ सुव॒र्ग्यꣳ॑ सुव॒र्ग्य॑म् तप॒स्य॑म् तप॒स्यꣳ॑ सुव॒र्ग्य᳚म् । \newline
38. सु॒व॒र्ग्य॑ मथो॒ अथो॑ सुव॒र्ग्यꣳ॑ सुव॒र्ग्य॑ मथो᳚ । \newline
39. सु॒व॒र्ग्य॑मिति॑ सुवः - ग्य᳚म् । \newline
40. अथो॒ प्र प्राथो॒ अथो॒ प्र । \newline
41. अथो॒ इत्यथो᳚ । \newline
42. प्रैवैव प्र प्रैव । \newline
43. ए॒व जा॑यते जायत ए॒वैव जा॑यते । \newline
44. जा॒य॒ते॒ प्र॒जया᳚ प्र॒जया॑ जायते जायते प्र॒जया᳚ । \newline
45. प्र॒जया॑ प॒शुभिः॑ प॒शुभिः॑ प्र॒जया᳚ प्र॒जया॑ प॒शुभिः॑ । \newline
46. प्र॒जयेति॑ प्र - जया᳚ । \newline
47. प॒शुभि॑र् यवा॒गूर् य॑वा॒गूः प॒शुभिः॑ प॒शुभि॑र् यवा॒गूः । \newline
48. प॒शुभि॒रिति॑ प॒शु - भिः॒ । \newline
49. य॒वा॒गू रा॑ज॒न्य॑स्य राज॒न्य॑स्य यवा॒गूर् य॑वा॒गू रा॑ज॒न्य॑स्य । \newline
50. रा॒ज॒न्य॑स्य व्र॒तं ॅव्र॒तꣳ रा॑ज॒न्य॑स्य राज॒न्य॑स्य व्र॒तम् । \newline
51. व्र॒तम् क्रू॒रा क्रू॒रा व्र॒तं ॅव्र॒तम् क्रू॒रा । \newline
52. क्रू॒रेवे॑व क्रू॒रा क्रू॒रेव॑ । \newline
53. इ॒व॒ वै वा इ॑वेव॒ वै । \newline
54. वै य॑वा॒गूर् य॑वा॒गूर् वै वै य॑वा॒गूः । \newline
55. य॒वा॒गूः क्रू॒रः क्रू॒रो य॑वा॒गूर् य॑वा॒गूः क्रू॒रः । \newline
56. क्रू॒र इ॑वेव क्रू॒रः क्रू॒र इ॑व । \newline
57. इ॒व॒ रा॒ज॒न्यो॑ राज॒न्य॑ इवेव राज॒न्यः॑ । \newline

\textbf{Ghana Paata } \newline

1. क्षु॒रप॑वि॒ नाम॒ नाम॑ क्षु॒रप॑वि क्षु॒रप॑वि॒ नाम॑ व्र॒तं ॅव्र॒तम् नाम॑ क्षु॒रप॑वि क्षु॒रप॑वि॒ नाम॑ व्र॒तम् । \newline
2. क्षु॒रप॒वीति॑ क्षु॒र - प॒वि॒ । \newline
3. नाम॑ व्र॒तं ॅव्र॒तम् नाम॒ नाम॑ व्र॒तं ॅयेन॒ येन॑ व्र॒तम् नाम॒ नाम॑ व्र॒तं ॅयेन॑ । \newline
4. व्र॒तं ॅयेन॒ येन॑ व्र॒तं ॅव्र॒तं ॅयेन॒ प्र प्र येन॑ व्र॒तं ॅव्र॒तं ॅयेन॒ प्र । \newline
5. येन॒ प्र प्र येन॒ येन॒ प्र जा॒तान् जा॒तान् प्र येन॒ येन॒ प्र जा॒तान् । \newline
6. प्र जा॒तान् जा॒तान् प्र प्र जा॒तान् भ्रातृ॑व्या॒न् भ्रातृ॑व्यान् जा॒तान् प्र प्र जा॒तान् भ्रातृ॑व्यान् । \newline
7. जा॒तान् भ्रातृ॑व्या॒न् भ्रातृ॑व्यान् जा॒तान् जा॒तान् भ्रातृ॑व्यान् नु॒दते॑ नु॒दते॒ भ्रातृ॑व्यान् जा॒तान् जा॒तान् भ्रातृ॑व्यान् नु॒दते᳚ । \newline
8. भ्रातृ॑व्यान् नु॒दते॑ नु॒दते॒ भ्रातृ॑व्या॒न् भ्रातृ॑व्यान् नु॒दते॒ प्रति॒ प्रति॑ नु॒दते॒ भ्रातृ॑व्या॒न् भ्रातृ॑व्यान् नु॒दते॒ प्रति॑ । \newline
9. नु॒दते॒ प्रति॒ प्रति॑ नु॒दते॑ नु॒दते॒ प्रति॑ जनि॒ष्यमा॑णान् जनि॒ष्यमा॑णा॒न् प्रति॑ नु॒दते॑ नु॒दते॒ प्रति॑ जनि॒ष्यमा॑णान् । \newline
10. प्रति॑ जनि॒ष्यमा॑णान् जनि॒ष्यमा॑णा॒न् प्रति॒ प्रति॑ जनि॒ष्यमा॑णा॒ नथो॒ अथो॑ जनि॒ष्यमा॑णा॒न् प्रति॒ प्रति॑ जनि॒ष्यमा॑णा॒ नथो᳚ । \newline
11. ज॒नि॒ष्यमा॑णा॒ नथो॒ अथो॑ जनि॒ष्यमा॑णान् जनि॒ष्यमा॑णा॒ नथो॒ कनी॑यसा॒ कनी॑य॒सा ऽथो॑ जनि॒ष्यमा॑णान् जनि॒ष्यमा॑णा॒ नथो॒ कनी॑यसा । \newline
12. अथो॒ कनी॑यसा॒ कनी॑य॒सा ऽथो॒ अथो॒ कनी॑यसै॒ वैव कनी॑य॒सा ऽथो॒ अथो॒ कनी॑य सै॒व । \newline
13. अथो॒ इत्यथो᳚ । \newline
14. कनी॑यसै॒ वैव कनी॑यसा॒ कनी॑य सै॒व भूयो॒ भूय॑ ए॒व कनी॑यसा॒ कनी॑य सै॒व भूयः॑ । \newline
15. ए॒व भूयो॒ भूय॑ ए॒वैव भूय॒ उपोप॒ भूय॑ ए॒वैव भूय॒ उप॑ । \newline
16. भूय॒ उपोप॒ भूयो॒ भूय॒ उपै᳚त्ये॒ त्युप॒ भूयो॒ भूय॒ उपै॑ति । \newline
17. उपै᳚त्ये॒ त्युपोपै॑ति च॒तुर॑ श्च॒तुर॑ ए॒त्युपोपै॑ति च॒तुरः॑ । \newline
18. ए॒ति॒ च॒तुर॑ श्च॒तुर॑ एत्येति च॒तुरो ऽग्रे ऽग्रे॑ च॒तुर॑ एत्येति च॒तुरो ऽग्रे᳚ । \newline
19. च॒तुरो ऽग्रे ऽग्रे॑ च॒तुर॑ श्च॒तुरो ऽग्रे॒ स्तना॒न् थ्स्तना॒ नग्रे॑ च॒तुर॑ श्च॒तुरो ऽग्रे॒ स्तनान्॑ । \newline
20. अग्रे॒ स्तना॒न् थ्स्तना॒ नग्रे ऽग्रे॒ स्तना᳚न् व्र॒तं ॅव्र॒तꣳ स्तना॒ नग्रे ऽग्रे॒ स्तना᳚न् व्र॒तम् । \newline
21. स्तना᳚न् व्र॒तं ॅव्र॒तꣳ स्तना॒न् थ्स्तना᳚न् व्र॒त मुपोप॑ व्र॒तꣳ स्तना॒न् थ्स्तना᳚न् व्र॒त मुप॑ । \newline
22. व्र॒त मुपोप॑ व्र॒तं ॅव्र॒त मुपै᳚त्ये॒ त्युप॑ व्र॒तं ॅव्र॒त मुपै॑ति । \newline
23. उपै᳚त्ये॒ त्युपोपै॒ त्यथा थै॒त्युपोपै॒ त्यथ॑ । \newline
24. ए॒त्य थाथै᳚ त्ये॒त्यथ॒ त्रीꣳ स्त्रीनथै᳚ त्ये॒त्यथ॒ त्रीन् । \newline
25. अथ॒ त्रीꣳ स्त्रीन थाथ॒ त्रीन थाथ॒ त्रीनथाथ॒ त्रीनथ॑ । \newline
26. त्रीन थाथ॒ त्रीꣳ स्त्रीनथ॒ द्वौ द्वा वथ॒ त्रीꣳ स्त्रीनथ॒ द्वौ । \newline
27. अथ॒ द्वौ द्वा वथाथ॒ द्वा वथाथ॒ द्वा वथाथ॒ द्वा वथ॑ । \newline
28. द्वा वथाथ॒ द्वौ द्वा वथैक॒ मेक॒ मथ॒ द्वौ द्वा वथैक᳚म् । \newline
29. अथैक॒ मेक॒ मथा थैक॑ मे॒त दे॒त देक॒ मथा थैक॑ मे॒तत् । \newline
30. एक॑ मे॒त दे॒त देक॒ मेक॑ मे॒तद् वै वा ए॒त देक॒ मेक॑ मे॒तद् वै । \newline
31. ए॒तद् वै वा ए॒त दे॒तद् वै सु॑जघ॒नꣳ सु॑जघ॒नं ॅवा ए॒त दे॒तद् वै सु॑जघ॒नम् । \newline
32. वै सु॑जघ॒नꣳ सु॑जघ॒नं ॅवै वै सु॑जघ॒नम् नाम॒ नाम॑ सुजघ॒नं ॅवै वै सु॑जघ॒नम् नाम॑ । \newline
33. सु॒ज॒घ॒नम् नाम॒ नाम॑ सुजघ॒नꣳ सु॑जघ॒नम् नाम॑ व्र॒तं ॅव्र॒तन् नाम॑ सुजघ॒नꣳ सु॑जघ॒नम् नाम॑ व्र॒तम् । \newline
34. सु॒ज॒घ॒नमिति॑ सु - ज॒घ॒नम् । \newline
35. नाम॑ व्र॒तं ॅव्र॒तन् नाम॒ नाम॑ व्र॒तम् त॑प॒स्य॑म् तप॒स्यं॑ ॅव्र॒तन् नाम॒ नाम॑ व्र॒तम् त॑प॒स्य᳚म् । \newline
36. व्र॒तम् त॑प॒स्य॑म् तप॒स्यं॑ ॅव्र॒तं ॅव्र॒तम् त॑प॒स्यꣳ॑ सुव॒र्ग्यꣳ॑ सुव॒र्ग्य॑म् तप॒स्यं॑ ॅव्र॒तं ॅव्र॒तम् त॑प॒स्यꣳ॑ सुव॒र्ग्य᳚म् । \newline
37. त॒प॒स्यꣳ॑ सुव॒र्ग्यꣳ॑ सुव॒र्ग्य॑म् तप॒स्य॑म् तप॒स्यꣳ॑ सुव॒र्ग्य॑ मथो॒ अथो॑ सुव॒र्ग्य॑म् तप॒स्य॑म् तप॒स्यꣳ॑ सुव॒र्ग्य॑ मथो᳚ । \newline
38. सु॒व॒र्ग्य॑ मथो॒ अथो॑ सुव॒र्ग्यꣳ॑ सुव॒र्ग्य॑ मथो॒ प्र प्राथो॑ सुव॒र्ग्यꣳ॑ सुव॒र्ग्य॑ मथो॒ प्र । \newline
39. सु॒व॒र्ग्य॑मिति॑ सुवः - ग्य᳚म् । \newline
40. अथो॒ प्र प्राथो॒ अथो॒ प्रैवैव प्राथो॒ अथो॒ प्रैव । \newline
41. अथो॒ इत्यथो᳚ । \newline
42. प्रैवैव प्र प्रैव जा॑यते जायत ए॒व प्र प्रैव जा॑यते । \newline
43. ए॒व जा॑यते जायत ए॒वैव जा॑यते प्र॒जया᳚ प्र॒जया॑ जायत ए॒वैव जा॑यते प्र॒जया᳚ । \newline
44. जा॒य॒ते॒ प्र॒जया᳚ प्र॒जया॑ जायते जायते प्र॒जया॑ प॒शुभिः॑ प॒शुभिः॑ प्र॒जया॑ जायते जायते प्र॒जया॑ प॒शुभिः॑ । \newline
45. प्र॒जया॑ प॒शुभिः॑ प॒शुभिः॑ प्र॒जया᳚ प्र॒जया॑ प॒शुभि॑र् यवा॒गूर् य॑वा॒गूः प॒शुभिः॑ प्र॒जया᳚ प्र॒जया॑ प॒शुभि॑र् यवा॒गूः । \newline
46. प्र॒जयेति॑ प्र - जया᳚ । \newline
47. प॒शुभि॑र् यवा॒गूर् य॑वा॒गूः प॒शुभिः॑ प॒शुभि॑र् यवा॒गू रा॑ज॒न्य॑स्य राज॒न्य॑स्य यवा॒गूः प॒शुभिः॑ प॒शुभि॑र् यवा॒गू रा॑ज॒न्य॑स्य । \newline
48. प॒शुभि॒रिति॑ प॒शु - भिः॒ । \newline
49. य॒वा॒गू रा॑ज॒न्य॑स्य राज॒न्य॑स्य यवा॒गूर् य॑वा॒गू रा॑ज॒न्य॑स्य व्र॒तं ॅव्र॒तꣳ रा॑ज॒न्य॑स्य यवा॒गूर् य॑वा॒गू रा॑ज॒न्य॑स्य व्र॒तम् । \newline
50. रा॒ज॒न्य॑स्य व्र॒तं ॅव्र॒तꣳ रा॑ज॒न्य॑स्य राज॒न्य॑स्य व्र॒तम् क्रू॒रा क्रू॒रा व्र॒तꣳ रा॑ज॒न्य॑स्य राज॒न्य॑स्य व्र॒तम् क्रू॒रा । \newline
51. व्र॒तम् क्रू॒रा क्रू॒रा व्र॒तं ॅव्र॒तम् क्रू॒रेवे॑व क्रू॒रा व्र॒तं ॅव्र॒तम् क्रू॒रेव॑ । \newline
52. क्रू॒रेवे॑व क्रू॒रा क्रू॒रेव॒ वै वा इ॑व क्रू॒रा क्रू॒रेव॒ वै । \newline
53. इ॒व॒ वै वा इ॑वेव॒ वै य॑वा॒गूर् य॑वा॒गूर् वा इ॑वेव॒ वै य॑वा॒गूः । \newline
54. वै य॑वा॒गूर् य॑वा॒गूर् वै वै य॑वा॒गूः क्रू॒रः क्रू॒रो य॑वा॒गूर् वै वै य॑वा॒गूः क्रू॒रः । \newline
55. य॒वा॒गूः क्रू॒रः क्रू॒रो य॑वा॒गूर् य॑वा॒गूः क्रू॒र इ॑वेव क्रू॒रो य॑वा॒गूर् य॑वा॒गूः क्रू॒र इ॑व । \newline
56. क्रू॒र इ॑वेव क्रू॒रः क्रू॒र इ॑व राज॒न्यो॑ राज॒न्य॑ इव क्रू॒रः क्रू॒र इ॑व राज॒न्यः॑ । \newline
57. इ॒व॒ रा॒ज॒न्यो॑ राज॒न्य॑ इवेव राज॒न्यो॑ वज्र॑स्य॒ वज्र॑स्य राज॒न्य॑ इवेव राज॒न्यो॑ वज्र॑स्य । \newline
\pagebreak
\markright{ TS 6.2.5.3  \hfill https://www.vedavms.in \hfill}

\section{ TS 6.2.5.3 }

\textbf{TS 6.2.5.3 } \newline
\textbf{Samhita Paata} \newline

राज॒न्यो॑ वज्र॑स्य रू॒पꣳ समृ॑द्ध्या आ॒मिक्षा॒ वैश्य॑स्य पाकय॒ज्ञ्स्य॑ रू॒पं पुष्ट्यै॒ पयो᳚ ब्राह्म॒णस्य॒ तेजो॒ वै ब्रा᳚ह्म॒णस्तेजः॒ पय॒स्तेज॑सै॒व तेजः॒ पय॑ आ॒त्मन् ध॒त्ते ऽथो॒ पय॑सा॒ वै गर्भा॑ वर्द्धन्ते॒ गर्भ॑ इव॒ खलु॒ वा ए॒ष यद् दी᳚क्षि॒तो यद॑स्य॒ पयो᳚ व्र॒तं भव॑त्या॒त्मान॑मे॒व तद् व॑र्द्धयति॒ त्रिव्र॑तो॒ वै मनु॑रासी॒द् द्विव्र॑ता॒ असु॑रा॒ एक॑व्रता- [  ] \newline

\textbf{Pada Paata} \newline

रा॒ज॒न्यः॑ । वज्र॑स्य । रू॒पम् । समृ॑द्ध्या॒ इति॒ सं-ऋ॒द्ध्यै॒ । आ॒मिक्षा᳚ । वैश्य॑स्य । पा॒क॒य॒ज्ञ्स्येति॑ पाक - य॒ज्ञ्स्य॑ । रू॒पम् । पुष्ट्यै᳚ । पयः॑ । ब्रा॒ह्म॒णस्य॑ । तेजः॑ । वै । ब्रा॒ह्म॒णः । तेजः॑ । पयः॑ । तेज॑सा । ए॒व । तेजः॑ । पयः॑ । आ॒त्मन्न् । ध॒त्ते॒ । अथो॒ इति॑ । पय॑सा । वै । गर्भाः᳚ । व॒द्‌र्ध॒न्ते॒ । गर्भः॑ । इ॒व॒ । खलु॑ । वै । ए॒षः । यत् । दी॒क्षि॒तः । यत् । अ॒स्य॒ । पयः॑ । व्र॒तम् । भव॑ति । आ॒त्मान᳚म् । ए॒व । तत् । व॒द्‌र्ध॒य॒ति॒ । त्रिव्र॑त॒ इति॒ त्रि - व्र॒तः॒ । वै । मनुः॑ । आ॒सी॒त् । द्विव्र॑ता॒ इति॒ द्वि - व्र॒ताः॒ । असु॑राः । एक॑व्रता॒ इत्येक॑ - व्र॒ताः॒ ।  \newline


\textbf{Krama Paata} \newline

रा॒ज॒न्यो॑ वज्र॑स्य । वज्र॑स्य रू॒पम् । रू॒पꣳ समृ॑द्ध्यै । समृ॑द्ध्या आ॒मिक्षा᳚ । समृ॑द्ध्या॒ इति॒ सम् - ऋ॒द्ध्यै॒ । आ॒मिक्षा॒ वैश्य॑स्य । वैश्य॑स्य पाकय॒ज्ञ्स्य॑ । पा॒क॒य॒ज्ञ्स्य॑ रू॒पम् । पा॒क॒य॒ज्ञ्स्येति॑ पाक - य॒ज्ञ्स्य॑ । रू॒पम् पुष्ट्‍यै᳚ । पुष्ट्‍यै॒ पयः॑ । पयो᳚ ब्राह्म॒णस्य॑ । ब्रा॒ह्म॒णस्य॒ तेजः॑ । तेजो॒ वै । वै ब्रा᳚ह्म॒णः । ब्रा॒ह्म॒णस्तेजः॑ । तेजः॒ पयः॑ । पय॒स्तेज॑सा । तेज॑सै॒व । ए॒व तेजः॑ । तेजः॒ पयः॑ । पय॑ आ॒त्मन्न् । आ॒त्मन् ध॑त्ते । ध॒त्तेऽथो᳚ । अथो॒ पय॑सा । अथो॒ इत्यथो᳚ । पय॑सा॒ वै । वै गर्भाः᳚ । गर्भा॑ वर्द्धन्ते । व॒र्द्ध॒न्ते॒ गर्भः॑ । गर्भ॑ इव । इ॒व॒ खलु॑ । खलु॒ वै । वा ए॒षः । ए॒ष यत् । यद् दी᳚क्षि॒तः । दी॒क्षि॒तो यत् । यद॑स्य । अ॒स्य॒ पयः॑ । पयो᳚ व्र॒तम् । व्र॒तम् भव॑ति । भव॑त्या॒त्मान᳚म् । आ॒त्मान॑मे॒व । ए॒व तत् । तद् व॑र्द्धयति । व॒र्द्ध॒य॒ति॒ त्रिव्र॑तः । त्रिव्र॑तो॒ वै । त्रिव्र॑त॒ इति॒ त्रि - व्र॒तः॒ । वै मनुः॑ । मनु॑रासीत् । आ॒सी॒द् द्विव्र॑ताः । द्विव्र॑ता॒ असु॑राः । द्विव्र॑ता॒ इति॒ द्वि - व्र॒ताः॒ । असु॑रा॒ एक॑व्रताः । एक॑व्रता दे॒वाः । एक॑व्रता॒ इत्येक॑ - व्र॒ताः॒ \newline

\textbf{Jatai Paata} \newline

1. रा॒ज॒न्यो॑ वज्र॑स्य॒ वज्र॑स्य राज॒न्यो॑ राज॒न्यो॑ वज्र॑स्य । \newline
2. वज्र॑स्य रू॒पꣳ रू॒पं ॅवज्र॑स्य॒ वज्र॑स्य रू॒पम् । \newline
3. रू॒पꣳ समृ॑द्ध्यै॒ समृ॑द्ध्यै रू॒पꣳ रू॒पꣳ समृ॑द्ध्यै । \newline
4. समृ॑द्ध्या आ॒मिक्षा॒ ऽऽमिक्षा॒ समृ॑द्ध्यै॒ समृ॑द्ध्या आ॒मिक्षा᳚ । \newline
5. समृ॑द्ध्या॒ इति॒ सं - ऋ॒द्ध्यै॒ । \newline
6. आ॒मिक्षा॒ वैश्य॑स्य॒ वैश्य॑स्या॒ मिक्षा॒ ऽऽमिक्षा॒ वैश्य॑स्य । \newline
7. वैश्य॑स्य पाकय॒ज्ञ्स्य॑ पाकय॒ज्ञ्स्य॒ वैश्य॑स्य॒ वैश्य॑स्य पाकय॒ज्ञ्स्य॑ । \newline
8. पा॒क॒य॒ज्ञ्स्य॑ रू॒पꣳ रू॒पम् पा॑कय॒ज्ञ्स्य॑ पाकय॒ज्ञ्स्य॑ रू॒पम् । \newline
9. पा॒क॒य॒ज्ञ्स्येति॑ पाक - य॒ज्ञ्स्य॑ । \newline
10. रू॒पम् पुष्ट्यै॒ पुष्ट्यै॑ रू॒पꣳ रू॒पम् पुष्ट्यै᳚ । \newline
11. पुष्ट्यै॒ पयः॒ पयः॒ पुष्ट्यै॒ पुष्ट्यै॒ पयः॑ । \newline
12. पयो᳚ ब्राह्म॒णस्य॑ ब्राह्म॒णस्य॒ पयः॒ पयो᳚ ब्राह्म॒णस्य॑ । \newline
13. ब्रा॒ह्म॒णस्य॒ तेज॒ स्तेजो᳚ ब्राह्म॒णस्य॑ ब्राह्म॒णस्य॒ तेजः॑ । \newline
14. तेजो॒ वै वै तेज॒ स्तेजो॒ वै । \newline
15. वै ब्रा᳚ह्म॒णो ब्रा᳚ह्म॒णो वै वै ब्रा᳚ह्म॒णः । \newline
16. ब्रा॒ह्म॒ण स्तेज॒ स्तेजो᳚ ब्राह्म॒णो ब्रा᳚ह्म॒ण स्तेजः॑ । \newline
17. तेजः॒ पयः॒ पय॒ स्तेज॒ स्तेजः॒ पयः॑ । \newline
18. पय॒ स्तेज॑सा॒ तेज॑सा॒ पयः॒ पय॒ स्तेज॑सा । \newline
19. तेज॑सै॒वैव तेज॑सा॒ तेज॑सै॒व । \newline
20. ए॒व तेज॒ स्तेज॑ ए॒वैव तेजः॑ । \newline
21. तेजः॒ पयः॒ पय॒ स्तेज॒ स्तेजः॒ पयः॑ । \newline
22. पय॑ आ॒त्मन् ना॒त्मन् पयः॒ पय॑ आ॒त्मन्न् । \newline
23. आ॒त्मन् ध॑त्ते धत्त आ॒त्मन् ना॒त्मन् ध॑त्ते । \newline
24. ध॒त्ते ऽथो॒ अथो॑ धत्ते ध॒त्ते ऽथो᳚ । \newline
25. अथो॒ पय॑सा॒ पय॒सा ऽथो॒ अथो॒ पय॑सा । \newline
26. अथो॒ इत्यथो᳚ । \newline
27. पय॑सा॒ वै वै पय॑सा॒ पय॑सा॒ वै । \newline
28. वै गर्भा॒ गर्भा॒ वै वै गर्भाः᳚ । \newline
29. गर्भा॑ वर्द्धन्ते वर्द्धन्ते॒ गर्भा॒ गर्भा॑ वर्द्धन्ते । \newline
30. व॒र्द्ध॒न्ते॒ गर्भो॒ गर्भो॑ वर्द्धन्ते वर्द्धन्ते॒ गर्भः॑ । \newline
31. गर्भ॑ इवेव॒ गर्भो॒ गर्भ॑ इव । \newline
32. इ॒व॒ खलु॒ खल्वि॑वेव॒ खलु॑ । \newline
33. खलु॒ वै वै खलु॒ खलु॒ वै । \newline
34. वा ए॒ष ए॒ष वै वा ए॒षः । \newline
35. ए॒ष यद् यदे॒ष ए॒ष यत् । \newline
36. यद् दी᳚क्षि॒तो दी᳚क्षि॒तो यद् यद् दी᳚क्षि॒तः । \newline
37. दी॒क्षि॒तो यद् यद् दी᳚क्षि॒तो दी᳚क्षि॒तो यत् । \newline
38. यद॑स्यास्य॒ यद् यद॑स्य । \newline
39. अ॒स्य॒ पयः॒ पयो᳚ ऽस्यास्य॒ पयः॑ । \newline
40. पयो᳚ व्र॒तं ॅव्र॒तम् पयः॒ पयो᳚ व्र॒तम् । \newline
41. व्र॒तम् भव॑ति॒ भव॑ति व्र॒तं ॅव्र॒तम् भव॑ति । \newline
42. भव॑ त्या॒त्मान॑ मा॒त्मान॒म् भव॑ति॒ भव॑ त्या॒त्मान᳚म् । \newline
43. आ॒त्मान॑ मे॒वैवात्मान॑ मा॒त्मान॑ मे॒व । \newline
44. ए॒व तत् तदे॒वैव तत् । \newline
45. तद् व॑र्द्धयति वर्द्धयति॒ तत् तद् व॑र्द्धयति । \newline
46. व॒र्द्ध॒य॒ति॒ त्रिव्र॑त॒ स्त्रिव्र॑तो वर्द्धयति वर्द्धयति॒ त्रिव्र॑तः । \newline
47. त्रिव्र॑तो॒ वै वै त्रिव्र॑त॒ स्त्रिव्र॑तो॒ वै । \newline
48. त्रिव्र॑त॒ इति॒ त्रि - व्र॒तः॒ । \newline
49. वै मनु॒र् मनु॒र् वै वै मनुः॑ । \newline
50. मनु॑ रासी दासी॒न् मनु॒र् मनु॑ रासीत् । \newline
51. आ॒सी॒द् द्विव्र॑ता॒ द्विव्र॑ता आसी दासी॒द् द्विव्र॑ताः । \newline
52. द्विव्र॑ता॒ असु॑रा॒ असु॑रा॒ द्विव्र॑ता॒ द्विव्र॑ता॒ असु॑राः । \newline
53. द्विव्र॑ता॒ इति॒ द्वि - व्र॒ताः॒ । \newline
54. असु॑रा॒ एक॑व्रता॒ एक॑व्रता॒ असु॑रा॒ असु॑रा॒ एक॑व्रताः । \newline
55. एक॑व्रता दे॒वा दे॒वा एक॑व्रता॒ एक॑व्रता दे॒वाः । \newline
56. एक॑व्रता॒ इत्येक॑ - व्र॒ताः॒ । \newline

\textbf{Ghana Paata } \newline

1. रा॒ज॒न्यो॑ वज्र॑स्य॒ वज्र॑स्य राज॒न्यो॑ राज॒न्यो॑ वज्र॑स्य रू॒पꣳ रू॒पं ॅवज्र॑स्य राज॒न्यो॑ राज॒न्यो॑ वज्र॑स्य रू॒पम् । \newline
2. वज्र॑स्य रू॒पꣳ रू॒पं ॅवज्र॑स्य॒ वज्र॑स्य रू॒पꣳ समृ॑द्ध्यै॒ समृ॑द्ध्यै रू॒पं ॅवज्र॑स्य॒ वज्र॑स्य रू॒पꣳ समृ॑द्ध्यै । \newline
3. रू॒पꣳ समृ॑द्ध्यै॒ समृ॑द्ध्यै रू॒पꣳ रू॒पꣳ समृ॑द्ध्या आ॒मिक्षा॒ ऽऽमिक्षा॒ समृ॑द्ध्यै रू॒पꣳ रू॒पꣳ समृ॑द्ध्या आ॒मिक्षा᳚ । \newline
4. समृ॑द्ध्या आ॒मिक्षा॒ ऽऽमिक्षा॒ समृ॑द्ध्यै॒ समृ॑द्ध्या आ॒मिक्षा॒ वैश्य॑स्य॒ वैश्य॑स्या॒ मिक्षा॒ समृ॑द्ध्यै॒ समृ॑द्ध्या आ॒मिक्षा॒ वैश्य॑स्य । \newline
5. समृ॑द्ध्या॒ इति॒ सं - ऋ॒द्ध्यै॒ । \newline
6. आ॒मिक्षा॒ वैश्य॑स्य॒ वैश्य॑स्या॒ मिक्षा॒ ऽऽमिक्षा॒ वैश्य॑स्य पाकय॒ज्ञ्स्य॑ पाकय॒ज्ञ्स्य॒ वैश्य॑स्या॒ मिक्षा॒ ऽऽमिक्षा॒ वैश्य॑स्य पाकय॒ज्ञ्स्य॑ । \newline
7. वैश्य॑स्य पाकय॒ज्ञ्स्य॑ पाकय॒ज्ञ्स्य॒ वैश्य॑स्य॒ वैश्य॑स्य पाकय॒ज्ञ्स्य॑ रू॒पꣳ रू॒पम् पा॑कय॒ज्ञ्स्य॒ वैश्य॑स्य॒ वैश्य॑स्य पाकय॒ज्ञ्स्य॑ रू॒पम् । \newline
8. पा॒क॒य॒ज्ञ्स्य॑ रू॒पꣳ रू॒पम् पा॑कय॒ज्ञ्स्य॑ पाकय॒ज्ञ्स्य॑ रू॒पम् पुष्ट्यै॒ पुष्ट्यै॑ रू॒पम् पा॑कय॒ज्ञ्स्य॑ पाकय॒ज्ञ्स्य॑ रू॒पम् पुष्ट्यै᳚ । \newline
9. पा॒क॒य॒ज्ञ्स्येति॑ पाक - य॒ज्ञ्स्य॑ । \newline
10. रू॒पम् पुष्ट्यै॒ पुष्ट्यै॑ रू॒पꣳ रू॒पम् पुष्ट्यै॒ पयः॒ पयः॒ पुष्ट्यै॑ रू॒पꣳ रू॒पम् पुष्ट्यै॒ पयः॑ । \newline
11. पुष्ट्यै॒ पयः॒ पयः॒ पुष्ट्यै॒ पुष्ट्यै॒ पयो᳚ ब्राह्म॒णस्य॑ ब्राह्म॒णस्य॒ पयः॒ पुष्ट्यै॒ पुष्ट्यै॒ पयो᳚ ब्राह्म॒णस्य॑ । \newline
12. पयो᳚ ब्राह्म॒णस्य॑ ब्राह्म॒णस्य॒ पयः॒ पयो᳚ ब्राह्म॒णस्य॒ तेज॒ स्तेजो᳚ ब्राह्म॒णस्य॒ पयः॒ पयो᳚ ब्राह्म॒णस्य॒ तेजः॑ । \newline
13. ब्रा॒ह्म॒णस्य॒ तेज॒ स्तेजो᳚ ब्राह्म॒णस्य॑ ब्राह्म॒णस्य॒ तेजो॒ वै वै तेजो᳚ ब्राह्म॒णस्य॑ ब्राह्म॒णस्य॒ तेजो॒ वै । \newline
14. तेजो॒ वै वै तेज॒ स्तेजो॒ वै ब्रा᳚ह्म॒णो ब्रा᳚ह्म॒णो वै तेज॒ स्तेजो॒ वै ब्रा᳚ह्म॒णः । \newline
15. वै ब्रा᳚ह्म॒णो ब्रा᳚ह्म॒णो वै वै ब्रा᳚ह्म॒ण स्तेज॒ स्तेजो᳚ ब्राह्म॒णो वै वै ब्रा᳚ह्म॒ण स्तेजः॑ । \newline
16. ब्रा॒ह्म॒ण स्तेज॒ स्तेजो᳚ ब्राह्म॒णो ब्रा᳚ह्म॒ण स्तेजः॒ पयः॒ पय॒ स्तेजो᳚ ब्राह्म॒णो ब्रा᳚ह्म॒ण स्तेजः॒ पयः॑ । \newline
17. तेजः॒ पयः॒ पय॒ स्तेज॒ स्तेजः॒ पय॒ स्तेज॑सा॒ तेज॑सा॒ पय॒ स्तेज॒ स्तेजः॒ पय॒ स्तेज॑सा । \newline
18. पय॒ स्तेज॑सा॒ तेज॑सा॒ पयः॒ पय॒ स्तेज॑ सै॒वैव तेज॑सा॒ पयः॒ पय॒ स्तेज॑सै॒व । \newline
19. तेज॑सै॒वैव तेज॑सा॒ तेज॑सै॒व तेज॒ स्तेज॑ ए॒व तेज॑सा॒ तेज॑सै॒व तेजः॑ । \newline
20. ए॒व तेज॒ स्तेज॑ ए॒वैव तेजः॒ पयः॒ पय॒ स्तेज॑ ए॒वैव तेजः॒ पयः॑ । \newline
21. तेजः॒ पयः॒ पय॒ स्तेज॒ स्तेजः॒ पय॑ आ॒त्मन् ना॒त्मन् पय॒ स्तेज॒ स्तेजः॒ पय॑ आ॒त्मन्न् । \newline
22. पय॑ आ॒त्मन् ना॒त्मन् पयः॒ पय॑ आ॒त्मन् ध॑त्ते धत्त आ॒त्मन् पयः॒ पय॑ आ॒त्मन् ध॑त्ते । \newline
23. आ॒त्मन् ध॑त्ते धत्त आ॒त्मन् ना॒त्मन् ध॒त्ते ऽथो॒ अथो॑ धत्त आ॒त्मन् ना॒त्मन् ध॒त्ते ऽथो᳚ । \newline
24. ध॒त्ते ऽथो॒ अथो॑ धत्ते ध॒त्ते ऽथो॒ पय॑सा॒ पय॒सा ऽथो॑ धत्ते ध॒त्ते ऽथो॒ पय॑सा । \newline
25. अथो॒ पय॑सा॒ पय॒सा ऽथो॒ अथो॒ पय॑सा॒ वै वै पय॒सा ऽथो॒ अथो॒ पय॑सा॒ वै । \newline
26. अथो॒ इत्यथो᳚ । \newline
27. पय॑सा॒ वै वै पय॑सा॒ पय॑सा॒ वै गर्भा॒ गर्भा॒ वै पय॑सा॒ पय॑सा॒ वै गर्भाः᳚ । \newline
28. वै गर्भा॒ गर्भा॒ वै वै गर्भा॑ वर्द्धन्ते वर्द्धन्ते॒ गर्भा॒ वै वै गर्भा॑ वर्द्धन्ते । \newline
29. गर्भा॑ वर्द्धन्ते वर्द्धन्ते॒ गर्भा॒ गर्भा॑ वर्द्धन्ते॒ गर्भो॒ गर्भो॑ वर्द्धन्ते॒ गर्भा॒ गर्भा॑ वर्द्धन्ते॒ गर्भः॑ । \newline
30. व॒र्द्ध॒न्ते॒ गर्भो॒ गर्भो॑ वर्द्धन्ते वर्द्धन्ते॒ गर्भ॑ इवेव॒ गर्भो॑ वर्द्धन्ते वर्द्धन्ते॒ गर्भ॑ इव । \newline
31. गर्भ॑ इवेव॒ गर्भो॒ गर्भ॑ इव॒ खलु॒ खल्वि॑व॒ गर्भो॒ गर्भ॑ इव॒ खलु॑ । \newline
32. इ॒व॒ खलु॒ खल्वि॑वेव॒ खलु॒ वै वै खल्वि॑वेव॒ खलु॒ वै । \newline
33. खलु॒ वै वै खलु॒ खलु॒ वा ए॒ष ए॒ष वै खलु॒ खलु॒ वा ए॒षः । \newline
34. वा ए॒ष ए॒ष वै वा ए॒ष यद् यदे॒ष वै वा ए॒ष यत् । \newline
35. ए॒ष यद् यदे॒ष ए॒ष यद् दी᳚क्षि॒तो दी᳚क्षि॒तो यदे॒ष ए॒ष यद् दी᳚क्षि॒तः । \newline
36. यद् दी᳚क्षि॒तो दी᳚क्षि॒तो यद् यद् दी᳚क्षि॒तो यद् यद् दी᳚क्षि॒तो यद् यद् दी᳚क्षि॒तो यत् । \newline
37. दी॒क्षि॒तो यद् यद् दी᳚क्षि॒तो दी᳚क्षि॒तो यद॑स्यास्य॒ यद् दी᳚क्षि॒तो दी᳚क्षि॒तो यद॑स्य । \newline
38. यद॑स्यास्य॒ यद् यद॑स्य॒ पयः॒ पयो᳚ ऽस्य॒ यद् यद॑स्य॒ पयः॑ । \newline
39. अ॒स्य॒ पयः॒ पयो᳚ ऽस्यास्य॒ पयो᳚ व्र॒तं ॅव्र॒तम् पयो᳚ ऽस्यास्य॒ पयो᳚ व्र॒तम् । \newline
40. पयो᳚ व्र॒तं ॅव्र॒तम् पयः॒ पयो᳚ व्र॒तम् भव॑ति॒ भव॑ति व्र॒तम् पयः॒ पयो᳚ व्र॒तम् भव॑ति । \newline
41. व्र॒तम् भव॑ति॒ भव॑ति व्र॒तं ॅव्र॒तम् भव॑ त्या॒त्मान॑ मा॒त्मान॒म् भव॑ति व्र॒तं ॅव्र॒तम् भव॑ त्या॒त्मान᳚म् । \newline
42. भव॑ त्या॒त्मान॑ मा॒त्मान॒म् भव॑ति॒ भव॑ त्या॒त्मान॑ मे॒वैवात्मान॒म् भव॑ति॒ भव॑ त्या॒त्मान॑ मे॒व । \newline
43. आ॒त्मान॑ मे॒वै वात्मान॑ मा॒त्मान॑ मे॒व तत् तदे॒ वात्मान॑ मा॒त्मान॑ मे॒व तत् । \newline
44. ए॒व तत् तदे॒ वैव तद् व॑र्द्धयति वर्द्धयति॒ तदे॒ वैव तद् व॑र्द्धयति । \newline
45. तद् व॑र्द्धयति वर्द्धयति॒ तत् तद् व॑र्द्धयति॒ त्रिव्र॑त॒ स्त्रिव्र॑तो वर्द्धयति॒ तत् तद् व॑र्द्धयति॒ त्रिव्र॑तः । \newline
46. व॒र्द्ध॒य॒ति॒ त्रिव्र॑त॒ स्त्रिव्र॑तो वर्द्धयति वर्द्धयति॒ त्रिव्र॑तो॒ वै वै त्रिव्र॑तो वर्द्धयति वर्द्धयति॒ त्रिव्र॑तो॒ वै । \newline
47. त्रिव्र॑तो॒ वै वै त्रिव्र॑त॒ स्त्रिव्र॑तो॒ वै मनु॒र् मनु॒र् वै त्रिव्र॑त॒ स्त्रिव्र॑तो॒ वै मनुः॑ । \newline
48. त्रिव्र॑त॒ इति॒ त्रि - व्र॒तः॒ । \newline
49. वै मनु॒र् मनु॒र् वै वै मनु॑ रासी दासी॒न् मनु॒र् वै वै मनु॑ रासीत् । \newline
50. मनु॑ रासी दासी॒न् मनु॒र् मनु॑ रासी॒द् द्विव्र॑ता॒ द्विव्र॑ता आसी॒न् मनु॒र् मनु॑ रासी॒द् द्विव्र॑ताः । \newline
51. आ॒सी॒द् द्विव्र॑ता॒ द्विव्र॑ता आसी दासी॒द् द्विव्र॑ता॒ असु॑रा॒ असु॑रा॒ द्विव्र॑ता आसी दासी॒द् द्विव्र॑ता॒ असु॑राः । \newline
52. द्विव्र॑ता॒ असु॑रा॒ असु॑रा॒ द्विव्र॑ता॒ द्विव्र॑ता॒ असु॑रा॒ एक॑व्रता॒ एक॑व्रता॒ असु॑रा॒ द्विव्र॑ता॒ द्विव्र॑ता॒ असु॑रा॒ एक॑व्रताः । \newline
53. द्विव्र॑ता॒ इति॒ द्वि - व्र॒ताः॒ । \newline
54. असु॑रा॒ एक॑व्रता॒ एक॑व्रता॒ असु॑रा॒ असु॑रा॒ एक॑व्रता दे॒वा दे॒वा एक॑व्रता॒ असु॑रा॒ असु॑रा॒ एक॑व्रता दे॒वाः । \newline
55. एक॑व्रता दे॒वा दे॒वा एक॑व्रता॒ एक॑व्रता दे॒वाः प्रा॒तः प्रा॒तर् दे॒वा एक॑व्रता॒ एक॑व्रता दे॒वाः प्रा॒तः । \newline
56. एक॑व्रता॒ इत्येक॑ - व्र॒ताः॒ । \newline
\pagebreak
\markright{ TS 6.2.5.4  \hfill https://www.vedavms.in \hfill}

\section{ TS 6.2.5.4 }

\textbf{TS 6.2.5.4 } \newline
\textbf{Samhita Paata} \newline

दे॒वाः प्रा॒तर्म॒द्ध्यंदि॑ने सा॒यं तन् मनो᳚र्व्र॒तमा॑सीत् पाकय॒ज्ञ्स्य॑ रू॒पं पुष्ट्यै᳚ प्रा॒तश्च॑ सा॒यं चासु॑राणां निर्म॒द्ध्यं क्षु॒धो रू॒पं तत॒स्ते परा॑ऽभवन् म॒द्ध्यंदि॑ने मद्ध्यरा॒त्रे दे॒वानां॒ तत॒स्ते॑ऽभवन्थ् सुव॒र्गं ॅलो॒कमा॑य॒न्॒. यद॑स्य म॒द्ध्यंदि॑ने मद्ध्यरा॒त्रे व्र॒तं भव॑ति मद्ध्य॒तो वा अन्ने॑न भुञ्जते मद्ध्य॒त ए॒व तदूर्जं॑ धत्ते॒ भ्रातृ॑व्याभिभूत्यै॒ भव॑त्या॒त्मना॒- [  ] \newline

\textbf{Pada Paata} \newline

दे॒वाः । प्रा॒तः । म॒द्ध्यन्दि॑ने । सा॒यम् । तत् । मनोः᳚ । व्र॒तम् । आ॒सी॒त् । पा॒क॒य॒ज्ञ्स्येति॑ पाक - य॒ज्ञ्स्य॑ । रू॒पम् । पुष्ट्यै᳚ । प्रा॒तः । च॒ । सा॒यम् । च॒ । असु॑राणाम् । नि॒र्म॒द्ध्यमिति॑ निः-म॒द्ध्यम् । क्षु॒धः । रू॒पम् । ततः॑ । ते । परेति॑ । अ॒भ॒व॒न्न् । म॒द्ध्यन्दि॑ने । म॒द्ध्य॒रा॒त्र इति॑ मद्ध्य - रा॒त्रे । दे॒वाना᳚म् । ततः॑ । ते । अ॒भ॒व॒न्न् । सु॒व॒र्गमिति॑ सुवः-गम् । लो॒कम् । आ॒य॒न्न् । यत् । अ॒स्य॒ । म॒द्ध्यन्दि॑ने । म॒द्ध्य॒रा॒त्र इति॑ मद्ध्य-रा॒त्रे । व्र॒तम् । भव॑ति । म॒द्ध्य॒तः । वै । अन्ने॑न । भु॒ञ्ज॒ते॒ । म॒द्ध्य॒तः । ए॒व । तत् । ऊर्ज᳚म् । ध॒त्ते॒ । भ्रातृ॑व्याभिभूत्या॒ इति॒ भ्रातृ॑व्य - अ॒भि॒भू॒त्यै॒ । भव॑ति । आ॒त्मना᳚ ।  \newline


\textbf{Krama Paata} \newline

दे॒वाः प्रा॒तः । प्रा॒तर् म॒द्ध्यन्दि॑ने । म॒द्ध्यन्दि॑ने सा॒यम् । सा॒यम् तत् । तन् मनोः᳚ । मनो᳚र् व्र॒तम् । व्र॒तमा॑सीत् । आ॒सी॒त् पा॒क॒य॒ज्ञ्स्य॑ । पा॒क॒य॒ज्ञ्स्य॑ रू॒पम् । पा॒क॒य॒ज्ञ्स्येति॑ पाक - य॒ज्ञ्स्य॑ । रू॒पम् पुष्ट्‍यै᳚ । पुष्ट्‍यै᳚ प्रा॒तः । प्रा॒तश्च॑ । च॒ सा॒यम् । सा॒यम् च॑ । चासु॑राणाम् । असु॑राणाम् निर्म॒द्ध्यम् । नि॒र्म॒द्ध्यम् क्षु॒धः । नि॒र्म॒द्ध्यमिति॑ निः - म॒द्ध्यम् । क्षु॒धो रू॒पम् । रू॒पम् ततः॑ । तत॒स्ते । ते परा᳚ । परा॑ऽभवन्न् । अ॒भ॒व॒न् म॒द्ध्यन्दि॑ने । म॒द्ध्यन्दि॑ने मद्ध्यरा॒त्रे । म॒द्ध्य॒रा॒त्रे दे॒वाना᳚म् । म॒द्ध्य॒रा॒त्र इति॑ मद्ध्य - रा॒त्रे । दे॒वाना॒म् ततः॑ । तत॒स्ते । ते॑ऽभवन्न् । अ॒भ॒व॒न्थ् सु॒व॒र्गम् । सु॒व॒र्गम् ॅलो॒कम् । सु॒व॒र्गमिति॑ सुवः - गम् । लो॒कमा॑यन्न् । आ॒य॒न्.॒ यत् । यद॑स्य । अ॒स्य॒ म॒द्ध्यन्दि॑ने । म॒द्ध्यन्दि॑ने मद्ध्यरा॒त्रे । म॒द्ध्य॒रा॒त्रे व्र॒तम् । म॒द्ध्य॒रा॒त्र इति॑ मद्ध्य - रा॒त्रे । व्र॒तम् भव॑ति । भव॑ति मद्ध्य॒तः । म॒द्ध्य॒तो वै । वा अन्ने॑न । अन्ने॑न भुञ्जते । भु॒ञ्ज॒ते॒ म॒द्ध्य॒तः । म॒द्ध्य॒त ए॒व । ए॒व तत् । तदूर्ज᳚म् । ऊर्ज॑म् धत्ते । ध॒त्ते॒ भ्रातृ॑व्याभिभूत्यै । भ्रातृ॑व्याभिभूत्यै॒ भव॑ति । भ्रात॑व्याभिभूत्या॒ इति॒ भ्रातृ॑व्य - अ॒भि॒भू॒त्यै॒ । भव॑त्या॒त्मना᳚ । आ॒त्मना॒ परा᳚ \newline

\textbf{Jatai Paata} \newline

1. दे॒वाः प्रा॒तः प्रा॒तर् दे॒वा दे॒वाः प्रा॒तः । \newline
2. प्रा॒तर् म॒द्ध्यन्दि॑ने म॒द्ध्यन्दि॑ने प्रा॒तः प्रा॒तर् म॒द्ध्यन्दि॑ने । \newline
3. म॒द्ध्यन्दि॑ने सा॒यꣳ सा॒यम् म॒द्ध्यन्दि॑ने म॒द्ध्यन्दि॑ने सा॒यम् । \newline
4. सा॒यम् तत् तथ् सा॒यꣳ सा॒यम् तत् । \newline
5. तन् मनो॒र् मनो॒ स्तत् तन् मनोः᳚ । \newline
6. मनो᳚र् व्र॒तं ॅव्र॒तम् मनो॒र् मनो᳚र् व्र॒तम् । \newline
7. व्र॒त मा॑सी दासीद् व्र॒तं ॅव्र॒त मा॑सीत् । \newline
8. आ॒सी॒त् पा॒क॒य॒ज्ञ्स्य॑ पाकय॒ज्ञ्स्या॑सी दासीत् पाकय॒ज्ञ्स्य॑ । \newline
9. पा॒क॒य॒ज्ञ्स्य॑ रू॒पꣳ रू॒पम् पा॑कय॒ज्ञ्स्य॑ पाकय॒ज्ञ्स्य॑ रू॒पम् । \newline
10. पा॒क॒य॒ज्ञ्स्येति॑ पाक - य॒ज्ञ्स्य॑ । \newline
11. रू॒पम् पुष्ट्यै॒ पुष्ट्यै॑ रू॒पꣳ रू॒पम् पुष्ट्यै᳚ । \newline
12. पुष्ट्यै᳚ प्रा॒तः प्रा॒तः पुष्ट्यै॒ पुष्ट्यै᳚ प्रा॒तः । \newline
13. प्रा॒तश्च॑ च प्रा॒तः प्रा॒तश्च॑ । \newline
14. च॒ सा॒यꣳ सा॒यम् च॑ च सा॒यम् । \newline
15. सा॒यम् च॑ च सा॒यꣳ सा॒यम् च॑ । \newline
16. चासु॑राणा॒ मसु॑राणाम् च॒ चासु॑राणाम् । \newline
17. असु॑राणाम् निर्म॒द्ध्यम् नि॑र्म॒द्ध्य मसु॑राणा॒ मसु॑राणाम् निर्म॒द्ध्यम् । \newline
18. नि॒र्म॒द्ध्यम् क्षु॒धः क्षु॒धो नि॑र्म॒द्ध्यम् नि॑र्म॒द्ध्यम् क्षु॒धः । \newline
19. नि॒र्म॒द्ध्यमिति॑ निः - म॒द्ध्यम् । \newline
20. क्षु॒धो रू॒पꣳ रू॒पम् क्षु॒धः क्षु॒धो रू॒पम् । \newline
21. रू॒पम् तत॒ स्ततो॑ रू॒पꣳ रू॒पम् ततः॑ । \newline
22. तत॒स्ते ते तत॒ स्तत॒ स्ते । \newline
23. ते परा॒ परा॒ ते ते परा᳚ । \newline
24. परा॑ ऽभवन् नभव॒न् परा॒ परा॑ ऽभवन्न् । \newline
25. अ॒भ॒व॒न् म॒द्ध्यन्दि॑ने म॒द्ध्यन्दि॑ने ऽभवन् नभवन् म॒द्ध्यन्दि॑ने । \newline
26. म॒द्ध्यन्दि॑ने मद्ध्यरा॒त्रे म॑द्ध्यरा॒त्रे म॒द्ध्यन्दि॑ने म॒द्ध्यन्दि॑ने मद्ध्यरा॒त्रे । \newline
27. म॒द्ध्य॒रा॒त्रे दे॒वाना᳚म् दे॒वाना᳚म् मद्ध्यरा॒त्रे म॑द्ध्यरा॒त्रे दे॒वाना᳚म् । \newline
28. म॒द्ध्य॒रा॒त्र इति॑ मद्ध्य - रा॒त्रे । \newline
29. दे॒वाना॒म् तत॒ स्ततो॑ दे॒वाना᳚म् दे॒वाना॒म् ततः॑ । \newline
30. तत॒ स्ते ते तत॒ स्तत॒ स्ते । \newline
31. ते॑ ऽभवन् नभव॒न् ते ते॑ ऽभवन्न् । \newline
32. अ॒भ॒व॒न् थ्सु॒व॒र्गꣳ सु॑व॒र्ग म॑भवन् नभवन् थ्सुव॒र्गम् । \newline
33. सु॒व॒र्गम् ॅलो॒कम् ॅलो॒कꣳ सु॑व॒र्गꣳ सु॑व॒र्गम् ॅलो॒कम् । \newline
34. सु॒व॒र्गमिति॑ सुवः - गम् । \newline
35. लो॒क मा॑यन् नायन् ॅलो॒कम् ॅलो॒क मा॑यन्न् । \newline
36. आ॒य॒न्॒. यद् यदा॑यन् नाय॒न्॒. यत् । \newline
37. यद॑स्यास्य॒ यद् यद॑स्य । \newline
38. अ॒स्य॒ म॒द्ध्यन्दि॑ने म॒द्ध्यन्दि॑ने ऽस्यास्य म॒द्ध्यन्दि॑ने । \newline
39. म॒द्ध्यन्दि॑ने मद्ध्यरा॒त्रे म॑द्ध्यरा॒त्रे म॒द्ध्यन्दि॑ने म॒द्ध्यन्दि॑ने मद्ध्यरा॒त्रे । \newline
40. म॒द्ध्य॒रा॒त्रे व्र॒तं ॅव्र॒तम् म॑द्ध्यरा॒त्रे म॑द्ध्यरा॒त्रे व्र॒तम् । \newline
41. म॒द्ध्य॒रा॒त्र इति॑ मद्ध्य - रा॒त्रे । \newline
42. व्र॒तम् भव॑ति॒ भव॑ति व्र॒तं ॅव्र॒तम् भव॑ति । \newline
43. भव॑ति मद्ध्य॒तो म॑द्ध्य॒तो भव॑ति॒ भव॑ति मद्ध्य॒तः । \newline
44. म॒द्ध्य॒तो वै वै म॑द्ध्य॒तो म॑द्ध्य॒तो वै । \newline
45. वा अन्ने॒ना न्ने॑न॒ वै वा अन्ने॑न । \newline
46. अन्ने॑न भुञ्जते भुञ्ज॒ते ऽन्ने॒ना न्ने॑न भुञ्जते । \newline
47. भु॒ञ्ज॒ते॒ म॒द्ध्य॒तो म॑द्ध्य॒तो भु॑ञ्जते भुञ्जते मद्ध्य॒तः । \newline
48. म॒द्ध्य॒त ए॒वैव म॑द्ध्य॒तो म॑द्ध्य॒त ए॒व । \newline
49. ए॒व तत् तदे॒वैव तत् । \newline
50. तदूर्ज॒ मूर्ज॒म् तत् तदूर्ज᳚म् । \newline
51. ऊर्ज॑म् धत्ते धत्त॒ ऊर्ज॒ मूर्ज॑म् धत्ते । \newline
52. ध॒त्ते॒ भ्रातृ॑व्याभिभूत्यै॒ भ्रातृ॑व्याभिभूत्यै धत्ते धत्ते॒ भ्रातृ॑व्याभिभूत्यै । \newline
53. भ्रातृ॑व्याभिभूत्यै॒ भव॑ति॒ भव॑ति॒ भ्रातृ॑व्याभिभूत्यै॒ भ्रातृ॑व्याभिभूत्यै॒ भव॑ति । \newline
54. भ्रातृ॑व्याभिभूत्या॒ इति॒ भ्रातृ॑व्य - अ॒भि॒भू॒त्यै॒ । \newline
55. भव॑ त्या॒त्मना॒ ऽऽत्मना॒ भव॑ति॒ भव॑ त्या॒त्मना᳚ । \newline
56. आ॒त्मना॒ परा॒ परा॒ ऽऽत्मना॒ ऽऽत्मना॒ परा᳚ । \newline

\textbf{Ghana Paata } \newline

1. दे॒वाः प्रा॒तः प्रा॒तर् दे॒वा दे॒वाः प्रा॒तर् म॒द्ध्यन्दि॑ने म॒द्ध्यन्दि॑ने प्रा॒तर् दे॒वा दे॒वाः प्रा॒तर् म॒द्ध्यन्दि॑ने । \newline
2. प्रा॒तर् म॒द्ध्यन्दि॑ने म॒द्ध्यन्दि॑ने प्रा॒तः प्रा॒तर् म॒द्ध्यन्दि॑ने सा॒यꣳ सा॒यम् म॒द्ध्यन्दि॑ने प्रा॒तः प्रा॒तर् म॒द्ध्यन्दि॑ने सा॒यम् । \newline
3. म॒द्ध्यन्दि॑ने सा॒यꣳ सा॒यम् म॒द्ध्यन्दि॑ने म॒द्ध्यन्दि॑ने सा॒यम् तत् तथ् सा॒यम् म॒द्ध्यन्दि॑ने म॒द्ध्यन्दि॑ने सा॒यम् तत् । \newline
4. सा॒यम् तत् तथ् सा॒यꣳ सा॒यम् तन् मनो॒र् मनो॒ स्तथ् सा॒यꣳ सा॒यम् तन् मनोः᳚ । \newline
5. तन् मनो॒र् मनो॒ स्तत् तन् मनो᳚र् व्र॒तं ॅव्र॒तम् मनो॒ स्तत् तन् मनो᳚र् व्र॒तम् । \newline
6. मनो᳚र् व्र॒तं ॅव्र॒तम् मनो॒र् मनो᳚र् व्र॒त मा॑सी दासीद् व्र॒तम् मनो॒र् मनो᳚र् व्र॒त मा॑सीत् । \newline
7. व्र॒त मा॑सी दासीद् व्र॒तं ॅव्र॒त मा॑सीत् पाकय॒ज्ञ्स्य॑ पाकय॒ज्ञ्स्या॑सीद् व्र॒तं ॅव्र॒त मा॑सीत् पाकय॒ज्ञ्स्य॑ । \newline
8. आ॒सी॒त् पा॒क॒य॒ज्ञ्स्य॑ पाकय॒ज्ञ्स्या॑सी दासीत् पाकय॒ज्ञ्स्य॑ रू॒पꣳ रू॒पम् पा॑कय॒ज्ञ्स्या॑सी दासीत् पाकय॒ज्ञ्स्य॑ रू॒पम् । \newline
9. पा॒क॒य॒ज्ञ्स्य॑ रू॒पꣳ रू॒पम् पा॑कय॒ज्ञ्स्य॑ पाकय॒ज्ञ्स्य॑ रू॒पम् पुष्ट्यै॒ पुष्ट्यै॑ रू॒पम् पा॑कय॒ज्ञ्स्य॑ पाकय॒ज्ञ्स्य॑ रू॒पम् पुष्ट्यै᳚ । \newline
10. पा॒क॒य॒ज्ञ्स्येति॑ पाक - य॒ज्ञ्स्य॑ । \newline
11. रू॒पम् पुष्ट्यै॒ पुष्ट्यै॑ रू॒पꣳ रू॒पम् पुष्ट्यै᳚ प्रा॒तः प्रा॒तः पुष्ट्यै॑ रू॒पꣳ रू॒पम् पुष्ट्यै᳚ प्रा॒तः । \newline
12. पुष्ट्यै᳚ प्रा॒तः प्रा॒तः पुष्ट्यै॒ पुष्ट्यै᳚ प्रा॒तश्च॑ च प्रा॒तः पुष्ट्यै॒ पुष्ट्यै᳚ प्रा॒तश्च॑ । \newline
13. प्रा॒तश्च॑ च प्रा॒तः प्रा॒तश्च॑ सा॒यꣳ सा॒यम् च॑ प्रा॒तः प्रा॒तश्च॑ सा॒यम् । \newline
14. च॒ सा॒यꣳ सा॒यम् च॑ च सा॒यम् च॑ च सा॒यम् च॑ च सा॒यम् च॑ । \newline
15. सा॒यम् च॑ च सा॒यꣳ सा॒यम् चासु॑राणा॒ मसु॑राणाम् च सा॒यꣳ सा॒यम् चासु॑राणाम् । \newline
16. चासु॑राणा॒ मसु॑राणाम् च॒ चासु॑राणान् निर्म॒द्ध्यम् नि॑र्म॒द्ध्य मसु॑राणाम् च॒ चासु॑राणाम् निर्म॒द्ध्यम् । \newline
17. असु॑राणाम् निर्म॒द्ध्यम् नि॑र्म॒द्ध्य मसु॑राणा॒ मसु॑राणाम् निर्म॒द्ध्यम् क्षु॒धः क्षु॒धो नि॑र्म॒द्ध्य मसु॑राणा॒ मसु॑राणाम् निर्म॒द्ध्यम् क्षु॒धः । \newline
18. नि॒र्म॒द्ध्यम् क्षु॒धः क्षु॒धो नि॑र्म॒द्ध्यम् नि॑र्म॒द्ध्यम् क्षु॒धो रू॒पꣳ रू॒पम् क्षु॒धो नि॑र्म॒द्ध्यम् नि॑र्म॒द्ध्यम् क्षु॒धो रू॒पम् । \newline
19. नि॒र्म॒द्ध्यमिति॑ निः - म॒द्ध्यम् । \newline
20. क्षु॒धो रू॒पꣳ रू॒पम् क्षु॒धः क्षु॒धो रू॒पम् तत॒ स्ततो॑ रू॒पम् क्षु॒धः क्षु॒धो रू॒पम् ततः॑ । \newline
21. रू॒पम् तत॒ स्ततो॑ रू॒पꣳ रू॒पम् तत॒ स्ते ते ततो॑ रू॒पꣳ रू॒पम् तत॒ स्ते । \newline
22. तत॒ स्ते ते तत॒ स्तत॒ स्ते परा॒ परा॒ ते तत॒ स्तत॒ स्ते परा᳚ । \newline
23. ते परा॒ परा॒ ते ते परा॑ ऽभवन् नभव॒न् परा॒ ते ते परा॑ ऽभवन्न् । \newline
24. परा॑ ऽभवन् नभव॒न् परा॒ परा॑ ऽभवन् म॒द्ध्यन्दि॑ने म॒द्ध्यन्दि॑ने ऽभव॒न् परा॒ परा॑ ऽभवन् म॒द्ध्यन्दि॑ने । \newline
25. अ॒भ॒व॒न् म॒द्ध्यन्दि॑ने म॒द्ध्यन्दि॑ने ऽभवन् नभवन् म॒द्ध्यन्दि॑ने मद्ध्यरा॒त्रे म॑द्ध्यरा॒त्रे म॒द्ध्यन्दि॑ने ऽभवन् नभवन् म॒द्ध्यन्दि॑ने मद्ध्यरा॒त्रे । \newline
26. म॒द्ध्यन्दि॑ने मद्ध्यरा॒त्रे म॑द्ध्यरा॒त्रे म॒द्ध्यन्दि॑ने म॒द्ध्यन्दि॑ने मद्ध्यरा॒त्रे दे॒वाना᳚म् दे॒वाना᳚म् मद्ध्यरा॒त्रे म॒द्ध्यन्दि॑ने म॒द्ध्यन्दि॑ने मद्ध्यरा॒त्रे दे॒वाना᳚म् । \newline
27. म॒द्ध्य॒रा॒त्रे दे॒वाना᳚म् दे॒वाना᳚म् मद्ध्यरा॒त्रे म॑द्ध्यरा॒त्रे दे॒वाना॒म् तत॒ स्ततो॑ दे॒वाना᳚म् मद्ध्यरा॒त्रे म॑द्ध्यरा॒त्रे दे॒वाना॒म् ततः॑ । \newline
28. म॒द्ध्य॒रा॒त्र इति॑ मद्ध्य - रा॒त्रे । \newline
29. दे॒वाना॒म् तत॒ स्ततो॑ दे॒वाना᳚म् दे॒वाना॒म् तत॒ स्ते ते ततो॑ दे॒वाना᳚म् दे॒वाना॒म् तत॒ स्ते । \newline
30. तत॒ स्ते ते तत॒ स्तत॒ स्ते॑ ऽभवन् नभव॒न् ते तत॒ स्तत॒ स्ते॑ ऽभवन्न् । \newline
31. ते॑ ऽभवन् नभव॒न् ते ते॑ ऽभवन् थ्सुव॒र्गꣳ सु॑व॒र्ग म॑भव॒न् ते ते॑ ऽभवन् थ्सुव॒र्गम् । \newline
32. अ॒भ॒व॒न् थ्सु॒व॒र्गꣳ सु॑व॒र्ग म॑भवन् नभवन् थ्सुव॒र्गम् ॅलो॒कम् ॅलो॒कꣳ सु॑व॒र्ग म॑भवन् नभवन् थ्सुव॒र्गम् ॅलो॒कम् । \newline
33. सु॒व॒र्गम् ॅलो॒कम् ॅलो॒कꣳ सु॑व॒र्गꣳ सु॑व॒र्गम् ॅलो॒क मा॑यन् नायन् ॅलो॒कꣳ सु॑व॒र्गꣳ सु॑व॒र्गम् ॅलो॒क मा॑यन्न् । \newline
34. सु॒व॒र्गमिति॑ सुवः - गम् । \newline
35. लो॒क मा॑यन् नायन् ॅलो॒कम् ॅलो॒क मा॑य॒न्॒. यद् यदा॑यन् ॅलो॒कम् ॅलो॒क मा॑य॒न्॒. यत् । \newline
36. आ॒य॒न्॒. यद् यदा॑यन् नाय॒न्॒. यद॑ स्यास्य॒ यदा॑यन् नाय॒न्॒. यद॑स्य । \newline
37. यद॑ स्यास्य॒ यद् यद॑स्य म॒द्ध्यन्दि॑ने म॒द्ध्यन्दि॑ने ऽस्य॒ यद् यद॑स्य म॒द्ध्यन्दि॑ने । \newline
38. अ॒स्य॒ म॒द्ध्यन्दि॑ने म॒द्ध्यन्दि॑ने ऽस्यास्य म॒द्ध्यन्दि॑ने मद्ध्यरा॒त्रे म॑द्ध्यरा॒त्रे म॒द्ध्यन्दि॑ने ऽस्यास्य म॒द्ध्यन्दि॑ने मद्ध्यरा॒त्रे । \newline
39. म॒द्ध्यन्दि॑ने मद्ध्यरा॒त्रे म॑द्ध्यरा॒त्रे म॒द्ध्यन्दि॑ने म॒द्ध्यन्दि॑ने मद्ध्यरा॒त्रे व्र॒तं ॅव्र॒तम् म॑द्ध्यरा॒त्रे म॒द्ध्यन्दि॑ने म॒द्ध्यन्दि॑ने मद्ध्यरा॒त्रे व्र॒तम् । \newline
40. म॒द्ध्य॒रा॒त्रे व्र॒तं ॅव्र॒तम् म॑द्ध्यरा॒त्रे म॑द्ध्यरा॒त्रे व्र॒तम् भव॑ति॒ भव॑ति व्र॒तम् म॑द्ध्यरा॒त्रे म॑द्ध्यरा॒त्रे व्र॒तम् भव॑ति । \newline
41. म॒द्ध्य॒रा॒त्र इति॑ मद्ध्य - रा॒त्रे । \newline
42. व्र॒तम् भव॑ति॒ भव॑ति व्र॒तं ॅव्र॒तम् भव॑ति मद्ध्य॒तो म॑द्ध्य॒तो भव॑ति व्र॒तं ॅव्र॒तम् भव॑ति मद्ध्य॒तः । \newline
43. भव॑ति मद्ध्य॒तो म॑द्ध्य॒तो भव॑ति॒ भव॑ति मद्ध्य॒तो वै वै म॑द्ध्य॒तो भव॑ति॒ भव॑ति मद्ध्य॒तो वै । \newline
44. म॒द्ध्य॒तो वै वै म॑द्ध्य॒तो म॑द्ध्य॒तो वा अन्ने॒ना न्ने॑न॒ वै म॑द्ध्य॒तो म॑द्ध्य॒तो वा अन्ने॑न । \newline
45. वा अन्ने॒ना न्ने॑न॒ वै वा अन्ने॑न भुञ्जते भुञ्ज॒ते ऽन्ने॑न॒ वै वा अन्ने॑न भुञ्जते । \newline
46. अन्ने॑न भुञ्जते भुञ्ज॒ते ऽन्ने॒ना न्ने॑न भुञ्जते मद्ध्य॒तो म॑द्ध्य॒तो भु॑ञ्ज॒ते ऽन्ने॒ना न्ने॑न भुञ्जते मद्ध्य॒तः । \newline
47. भु॒ञ्ज॒ते॒ म॒द्ध्य॒तो म॑द्ध्य॒तो भु॑ञ्जते भुञ्जते मद्ध्य॒त ए॒वैव म॑द्ध्य॒तो भु॑ञ्जते भुञ्जते मद्ध्य॒त ए॒व । \newline
48. म॒द्ध्य॒त ए॒वैव म॑द्ध्य॒तो म॑द्ध्य॒त ए॒व तत् तदे॒व म॑द्ध्य॒तो म॑द्ध्य॒त ए॒व तत् । \newline
49. ए॒व तत् तदे॒ वैव तदूर्ज॒ मूर्ज॒म् तदे॒ वैव तदूर्ज᳚म् । \newline
50. तदूर्ज॒ मूर्ज॒म् तत् तदूर्ज॑म् धत्ते धत्त॒ ऊर्ज॒म् तत् तदूर्ज॑म् धत्ते । \newline
51. ऊर्ज॑म् धत्ते धत्त॒ ऊर्ज॒ मूर्ज॑म् धत्ते॒ भ्रातृ॑व्याभिभूत्यै॒ भ्रातृ॑व्याभिभूत्यै धत्त॒ ऊर्ज॒ मूर्ज॑म् धत्ते॒ भ्रातृ॑व्याभिभूत्यै । \newline
52. ध॒त्ते॒ भ्रातृ॑व्याभिभूत्यै॒ भ्रातृ॑व्याभिभूत्यै धत्ते धत्ते॒ भ्रातृ॑व्याभिभूत्यै॒ भव॑ति॒ भव॑ति॒ भ्रातृ॑व्याभिभूत्यै धत्ते धत्ते॒ भ्रातृ॑व्याभिभूत्यै॒ भव॑ति । \newline
53. भ्रातृ॑व्याभिभूत्यै॒ भव॑ति॒ भव॑ति॒ भ्रातृ॑व्याभिभूत्यै॒ भ्रातृ॑व्याभिभूत्यै॒ भव॑त्या॒त्मना॒ ऽऽत्मना॒ भव॑ति॒ भ्रातृ॑व्याभिभूत्यै॒ भ्रातृ॑व्याभिभूत्यै॒ भव॑त्या॒त्मना᳚ । \newline
54. भ्रातृ॑व्याभिभूत्या॒ इति॒ भ्रातृ॑व्य - अ॒भि॒भू॒त्यै॒ । \newline
55. भव॑ त्या॒त्मना॒ ऽऽत्मना॒ भव॑ति॒ भव॑ त्या॒त्मना॒ परा॒ परा॒ ऽऽत्मना॒ भव॑ति॒ भव॑ त्या॒त्मना॒ परा᳚ । \newline
56. आ॒त्मना॒ परा॒ परा॒ ऽऽत्मना॒ ऽऽत्मना॒ परा᳚ ऽस्यास्य॒ परा॒ ऽऽत्मना॒ ऽऽत्मना॒ परा᳚ ऽस्य । \newline
\pagebreak
\markright{ TS 6.2.5.5  \hfill https://www.vedavms.in \hfill}

\section{ TS 6.2.5.5 }

\textbf{TS 6.2.5.5 } \newline
\textbf{Samhita Paata} \newline

परा᳚ऽस्य॒ भ्रातृ॑व्यो भवति॒ गर्भो॒ वा ए॒ष यद् दी᳚क्षि॒तो योनि॑ र्दीक्षितविमि॒तं ॅयद् दी᳚क्षि॒तो दी᳚क्षितविमि॒तात् प्र॒वसे॒द् यथा॒ योने॒र्गर्भः॒ स्कन्द॑ति ता॒दृगे॒व तन्न प्र॑वस्त॒व्य॑मा॒त्मनो॑ गोपी॒थायै॒ष वै व्या॒घ्रः कु॑लगो॒पो यद॒ग्निस्तस्मा॒द् यद् दी᳚क्षि॒तः प्र॒वसे॒थ् स ए॑नमीश्व॒रो॑ऽनू॒त्थाय॒ हन्तो॒र्न प्र॑वस्त॒व्य॑मा॒त्मनो॒ गुप्त्यै॑ दक्षिण॒तः श॑य ए॒तद्वै यज॑मानस्या॒ ( ) ऽऽ*यत॑नꣳ॒॒स्व ए॒वाऽऽ*यत॑ने शये॒ ऽग्निम॑भ्या॒वृत्य॑ शये दे॒वता॑ ए॒व य॒ज्ञ्म॑भ्या॒वृत्य॑ शये ॥ \newline

\textbf{Pada Paata} \newline

परेति॑ । अ॒स्य॒ । भ्रातृ॑व्यः । भ॒व॒ति॒ । गर्भः॑ । वै । ए॒षः । यत् । दी॒क्षि॒तः । योनिः॑ । दी॒क्षि॒त॒वि॒मि॒तमिति॑ दीक्षित - वि॒मि॒तम् । यत् । दी॒क्षि॒तः । दी॒क्षि॒त॒वि॒मि॒तादिति॑ दीक्षित - वि॒मि॒तात् । प्र॒वसे॒दिति॑ प्र - वसे᳚त् । यथा᳚ । योनेः᳚ । गर्भः॑ । स्कन्द॑ति । ता॒दृक् । ए॒व । तत् । न । प्र॒व॒स्त॒व्य॑मिति॑ प्र - व॒स्त॒व्य᳚म् । आ॒त्मनः॑ । गो॒पी॒थाय॑ । ए॒षः । वै । व्या॒घ्रः । कु॒ल॒गो॒प इति॑ कुल - गो॒पः । यत् । अ॒ग्निः । तस्मा᳚त् । यत् । दी॒क्षि॒तः । प्र॒वसे॒दिति॑ प्र - वसे᳚त् । सः । ए॒न॒म् । ई॒श्व॒रः । अ॒नू॒त्थायेत्य॑नु - उ॒त्थाय॑ । हन्तोः᳚ । न । प्र॒व॒स्त॒व्य॑मिति॑ प्र - व॒स्त॒व्य᳚म् । आ॒त्मनः॑ । गुप्त्यै᳚ । द॒क्षि॒ण॒तः । श॒ये॒ । ए॒तत् । वै । यज॑मानस्य ( ) । आ॒यत॑न॒मित्या᳚ - यत॑नम् । स्वे । ए॒व । आ॒यत॑न॒ इत्या᳚-यत॑ने । श॒ये॒ । अ॒ग्निम् । अ॒भ्या॒वृत्येत्य॑भि-आ॒वृत्य॑ । श॒ये॒ । दे॒वताः᳚ । ए॒व । य॒ज्ञ्म् । अ॒भ्या॒वृत्येत्य॑भि -आ॒वृत्य॑ । श॒ये॒ ॥  \newline


\textbf{Krama Paata} \newline

परा᳚स्य । अ॒स्य॒ भ्रातृ॑व्यः । भ्रातृ॑व्यो भवति । भ॒व॒ति॒ गर्भः॑ । गर्भो॒ वै । वा ए॒षः । ए॒ष यत् । यद् दी᳚क्षि॒तः । दी॒क्षि॒तो योनिः॑ । योनि॑र् दीक्षितविमि॒तम् । दी॒क्षि॒त॒वि॒मि॒तम् ॅयत् । दी॒क्षि॒त॒वि॒मि॒तमिति॑ दीक्षित - वि॒मि॒तम् । यद् दी᳚क्षि॒तः । दी॒क्षि॒तो दी᳚क्षितविमि॒तात् । दी॒क्षि॒त॒वि॒मि॒तात् प्र॒वसे᳚त् । दी॒क्षि॒त॒वि॒मि॒तादिति॑ दीक्षित - वि॒मि॒तात् । प्र॒वसे॒द् यथा᳚ । प्र॒वसे॒दिति॑ प्र - वसे᳚त् । यथा॒ योनेः᳚ । योने॒र् गर्भः॑ । गर्भः॒ स्कन्द॑ति । स्कन्द॑ति ता॒दृक् । ता॒दृगे॒व । ए॒व तत् । तन् न । न प्र॑वस्त॒व्य᳚म् । प्र॒व॒स्त॒व्य॑मा॒त्मनः॑ । प्र॒व॒स्त॒व्य॑मिति॑ प्र - व॒स्त॒व्य᳚म् । आ॒त्मनो॑ गोपी॒थाय॑ । गो॒पी॒थायै॒षः । ए॒ष वै । वै व्या॒घ्रः । व्या॒घ्रः कु॑लगो॒पः । कु॒ल॒गो॒पो यत् । कु॒ल॒गो॒प इति॑ कुल - गो॒पः । यद॒ग्निः । अ॒ग्निस्तस्मा᳚त् । तस्मा॒द् यत् । यद् दी᳚क्षि॒तः । दी॒क्षि॒तः प्र॒वसे᳚त् । प्र॒वसे॒थ् सः । प्र॒वसे॒दिति॑ प्र - वसे᳚त् । स ए॑नम् । ए॒न॒मी॒श्व॒रः । ई॒श्व॒रो॑ऽनू॒त्थाय॑ । अ॒नू॒त्थाय॒ हन्तोः᳚ । अ॒नू॒त्थायेत्य॑नु - उ॒त्थाय॑ । हन्तो॒र् न । न प्र॑वस्त॒व्य᳚म् । प्र॒व॒स्त॒व्य॑मा॒त्मनः॑ । प्र॒व॒स्त॒व्य॑मिति॑ प्र - व॒स्त॒व्य᳚म् । आ॒त्मनो॒ गुप्त्यै᳚ । गुप्त्यै॑ दक्षिण॒तः । द॒क्षि॒ण॒तः श॑ये । श॒य॒ ए॒तत् । ए॒तद् वै । वै यज॑मानस्य ( ) । यज॑मानस्या॒यत॑नम् । आ॒यत॑नꣳ॒॒ स्वे । आ॒यत॑न॒मित्या᳚ - यत॑नम् । स्व ए॒व । ए॒वायत॑ने । आ॒यत॑ने शये । आ॒यत॑न॒ इत्या᳚ - यत॑ने । श॒ये॒ऽग्निम् । अ॒ग्निम॑भ्या॒वृत्य॑ । अ॒भ्या॒वृत्य॑ शये । अ॒भ्या॒वृत्येत्य॑भि - आ॒वृत्य॑ । श॒ये॒ दे॒वताः᳚ । दे॒वता॑ ए॒व । ए॒व य॒ज्ञ्म् । य॒ज्ञ्म॑भ्या॒वृत्य॑ । अ॒भ्या॒वृत्य॑ शये । अ॒भ्या॒वृत्येत्य॑भि - आ॒वृत्य॑ । श॒य॒ इति॑ शये । \newline

\textbf{Jatai Paata} \newline

1. परा᳚ ऽस्यास्य॒ परा॒ परा᳚ ऽस्य । \newline
2. अ॒स्य॒ भ्रातृ॑व्यो॒ भ्रातृ॑व्यो ऽस्यास्य॒ भ्रातृ॑व्यः । \newline
3. भ्रातृ॑व्यो भवति भवति॒ भ्रातृ॑व्यो॒ भ्रातृ॑व्यो भवति । \newline
4. भ॒व॒ति॒ गर्भो॒ गर्भो॑ भवति भवति॒ गर्भः॑ । \newline
5. गर्भो॒ वै वै गर्भो॒ गर्भो॒ वै । \newline
6. वा ए॒ष ए॒ष वै वा ए॒षः । \newline
7. ए॒ष यद् यदे॒ष ए॒ष यत् । \newline
8. यद् दी᳚क्षि॒तो दी᳚क्षि॒तो यद् यद् दी᳚क्षि॒तः । \newline
9. दी॒क्षि॒तो योनि॒र् योनि॑र् दीक्षि॒तो दी᳚क्षि॒तो योनिः॑ । \newline
10. योनि॑र् दीक्षितविमि॒तम् दी᳚क्षितविमि॒तं ॅयोनि॒र् योनि॑र् दीक्षितविमि॒तम् । \newline
11. दी॒क्षि॒त॒वि॒मि॒तं ॅयद् यद् दी᳚क्षितविमि॒तम् दी᳚क्षितविमि॒तं ॅयत् । \newline
12. दी॒क्षि॒त॒वि॒मि॒तमिति॑ दीक्षित - वि॒मि॒तम् । \newline
13. यद् दी᳚क्षि॒तो दी᳚क्षि॒तो यद् यद् दी᳚क्षि॒तः । \newline
14. दी॒क्षि॒तो दी᳚क्षितविमि॒ताद् दी᳚क्षितविमि॒ताद् दी᳚क्षि॒तो दी᳚क्षि॒तो दी᳚क्षितविमि॒तात् । \newline
15. दी॒क्षि॒त॒वि॒मि॒तात् प्र॒वसे᳚त् प्र॒वसे᳚द् दीक्षितविमि॒ताद् दी᳚क्षितविमि॒तात् प्र॒वसे᳚त् । \newline
16. दी॒क्षि॒त॒वि॒मि॒तादिति॑ दीक्षित - वि॒मि॒तात् । \newline
17. प्र॒वसे॒द् यथा॒ यथा᳚ प्र॒वसे᳚त् प्र॒वसे॒द् यथा᳚ । \newline
18. प्र॒वसे॒दिति॑ प्र - वसे᳚त् । \newline
19. यथा॒ योने॒र् योने॒र् यथा॒ यथा॒ योनेः᳚ । \newline
20. योने॒र् गर्भो॒ गर्भो॒ योने॒र् योने॒र् गर्भः॑ । \newline
21. गर्भः॒ स्कन्द॑ति॒ स्कन्द॑ति॒ गर्भो॒ गर्भः॒ स्कन्द॑ति । \newline
22. स्कन्द॑ति ता॒दृक् ता॒दृख् स्कन्द॑ति॒ स्कन्द॑ति ता॒दृक् । \newline
23. ता॒दृ गे॒वैव ता॒दृक् ता॒दृ गे॒व । \newline
24. ए॒व तत् तदे॒वैव तत् । \newline
25. तन् न न तत् तन् न । \newline
26. न प्र॑वस्त॒व्य॑म् प्रवस्त॒व्य॑म् न न प्र॑वस्त॒व्य᳚म् । \newline
27. प्र॒व॒स्त॒व्य॑ मा॒त्मन॑ आ॒त्मनः॑ प्रवस्त॒व्य॑म् प्रवस्त॒व्य॑ मा॒त्मनः॑ । \newline
28. प्र॒व॒स्त॒व्य॑मिति॑ प्र - व॒स्त॒व्य᳚म् । \newline
29. आ॒त्मनो॑ गोपी॒थाय॑ गोपी॒थाया॒ त्मन॑ आ॒त्मनो॑ गोपी॒थाय॑ । \newline
30. गो॒पी॒था यै॒ष ए॒ष गो॑पी॒थाय॑ गोपी॒था यै॒षः । \newline
31. ए॒ष वै वा ए॒ष ए॒ष वै । \newline
32. वै व्या॒घ्रो व्या॒घ्रो वै वै व्या॒घ्रः । \newline
33. व्या॒घ्रः कु॑लगो॒पः कु॑लगो॒पो व्या॒घ्रो व्या॒घ्रः कु॑लगो॒पः । \newline
34. कु॒ल॒गो॒पो यद् यत् कु॑लगो॒पः कु॑लगो॒पो यत् । \newline
35. कु॒ल॒गो॒प इति॑ कुल - गो॒पः । \newline
36. यद॒ग्नि र॒ग्निर् यद् यद॒ग्निः । \newline
37. अ॒ग्नि स्तस्मा॒त् तस्मा॑ द॒ग्नि र॒ग्नि स्तस्मा᳚त् । \newline
38. तस्मा॒द् यद् यत् तस्मा॒त् तस्मा॒द् यत् । \newline
39. यद् दी᳚क्षि॒तो दी᳚क्षि॒तो यद् यद् दी᳚क्षि॒तः । \newline
40. दी॒क्षि॒तः प्र॒वसे᳚त् प्र॒वसे᳚द् दीक्षि॒तो दी᳚क्षि॒तः प्र॒वसे᳚त् । \newline
41. प्र॒वसे॒थ् स स प्र॒वसे᳚त् प्र॒वसे॒थ् सः । \newline
42. प्र॒वसे॒दिति॑ प्र - वसे᳚त् । \newline
43. स ए॑न मेनꣳ॒॒ स स ए॑नम् । \newline
44. ए॒न॒ मी॒श्व॒र ई᳚श्व॒र ए॑न मेन मीश्व॒रः । \newline
45. ई॒श्व॒रो॑ ऽनू॒त्थाया॑ नू॒त्था ये᳚श्व॒र ई᳚श्व॒रो॑ ऽनू॒त्थाय॑ । \newline
46. अ॒नू॒त्थाय॒ हन्तो॒र्॒. हन्तो॑ रनू॒त्थाया॑ नू॒त्थाय॒ हन्तोः᳚ । \newline
47. अ॒नू॒त्थायेत्य॑नु - उ॒त्थाय॑ । \newline
48. हन्तो॒र् न न हन्तो॒र्॒. हन्तो॒र् न । \newline
49. न प्र॑वस्त॒व्य॑म् प्रवस्त॒व्य॑म् न न प्र॑वस्त॒व्य᳚म् । \newline
50. प्र॒व॒स्त॒व्य॑ मा॒त्मन॑ आ॒त्मनः॑ प्रवस्त॒व्य॑म् प्रवस्त॒व्य॑ मा॒त्मनः॑ । \newline
51. प्र॒व॒स्त॒व्य॑मिति॑ प्र - व॒स्त॒व्य᳚म् । \newline
52. आ॒त्मनो॒ गुप्त्यै॒ गुप्त्या॑ आ॒त्मन॑ आ॒त्मनो॒ गुप्त्यै᳚ । \newline
53. गुप्त्यै॑ दक्षिण॒तो द॑क्षिण॒तो गुप्त्यै॒ गुप्त्यै॑ दक्षिण॒तः । \newline
54. द॒क्षि॒ण॒तः श॑ये शये दक्षिण॒तो द॑क्षिण॒तः श॑ये । \newline
55. श॒य॒ ए॒त दे॒त च्छ॑ये शय ए॒तत् । \newline
56. ए॒तद् वै वा ए॒त दे॒तद् वै । \newline
57. वै यज॑मानस्य॒ यज॑मानस्य॒ वै वै यज॑मानस्य । \newline
58. यज॑मानस्या॒ यत॑न मा॒यत॑नं॒ ॅयज॑मानस्य॒ यज॑मानस्या॒ यत॑नम् । \newline
59. आ॒यत॑नꣳ॒॒ स्वे स्व आ॒यत॑न मा॒यत॑नꣳ॒॒ स्वे । \newline
60. आ॒यत॑न॒मित्या᳚ - यत॑नम् । \newline
61. स्व ए॒वैव स्वे स्व ए॒व । \newline
62. ए॒वायत॑न आ॒यत॑न ए॒वैवायत॑ने । \newline
63. आ॒यत॑ने शये शय आ॒यत॑न आ॒यत॑ने शये । \newline
64. आ॒यत॑न॒ इत्या᳚ - यत॑ने । \newline
65. श॒ये॒ ऽग्नि म॒ग्निꣳ श॑ये शये॒ ऽग्निम् । \newline
66. अ॒ग्नि म॑भ्या॒वृत्या᳚ भ्या॒वृत्या॒ग्नि म॒ग्नि म॑भ्या॒वृत्य॑ । \newline
67. अ॒भ्या॒वृत्य॑ शये शये ऽभ्या॒वृत्या᳚ भ्या॒वृत्य॑ शये । \newline
68. अ॒भ्या॒वृत्येत्य॑भि - आ॒वृत्य॑ । \newline
69. श॒ये॒ दे॒वता॑ दे॒वताः᳚ शये शये दे॒वताः᳚ । \newline
70. दे॒वता॑ ए॒वैव दे॒वता॑ दे॒वता॑ ए॒व । \newline
71. ए॒व य॒ज्ञ्ं ॅय॒ज्ञ् मे॒वैव य॒ज्ञ्म् । \newline
72. य॒ज्ञ् म॑भ्या॒वृत्या᳚ भ्या॒वृत्य॑ य॒ज्ञ्ं ॅय॒ज्ञ् म॑भ्या॒वृत्य॑ । \newline
73. अ॒भ्या॒वृत्य॑ शये शये ऽभ्या॒वृत्या᳚ भ्या॒वृत्य॑ शये । \newline
74. अ॒भ्या॒वृत्येत्य॑भि - आ॒वृत्य॑ । \newline
75. श॒य॒ इति॑ शये । \newline

\textbf{Ghana Paata } \newline

1. परा᳚ ऽस्यास्य॒ परा॒ परा᳚ ऽस्य॒ भ्रातृ॑व्यो॒ भ्रातृ॑व्यो ऽस्य॒ परा॒ परा᳚ ऽस्य॒ भ्रातृ॑व्यः । \newline
2. अ॒स्य॒ भ्रातृ॑व्यो॒ भ्रातृ॑व्यो ऽस्यास्य॒ भ्रातृ॑व्यो भवति भवति॒ भ्रातृ॑व्यो ऽस्यास्य॒ भ्रातृ॑व्यो भवति । \newline
3. भ्रातृ॑व्यो भवति भवति॒ भ्रातृ॑व्यो॒ भ्रातृ॑व्यो भवति॒ गर्भो॒ गर्भो॑ भवति॒ भ्रातृ॑व्यो॒ भ्रातृ॑व्यो भवति॒ गर्भः॑ । \newline
4. भ॒व॒ति॒ गर्भो॒ गर्भो॑ भवति भवति॒ गर्भो॒ वै वै गर्भो॑ भवति भवति॒ गर्भो॒ वै । \newline
5. गर्भो॒ वै वै गर्भो॒ गर्भो॒ वा ए॒ष ए॒ष वै गर्भो॒ गर्भो॒ वा ए॒षः । \newline
6. वा ए॒ष ए॒ष वै वा ए॒ष यद् यदे॒ष वै वा ए॒ष यत् । \newline
7. ए॒ष यद् यदे॒ष ए॒ष यद् दी᳚क्षि॒तो दी᳚क्षि॒तो यदे॒ष ए॒ष यद् दी᳚क्षि॒तः । \newline
8. यद् दी᳚क्षि॒तो दी᳚क्षि॒तो यद् यद् दी᳚क्षि॒तो योनि॒र् योनि॑र् दीक्षि॒तो यद् यद् दी᳚क्षि॒तो योनिः॑ । \newline
9. दी॒क्षि॒तो योनि॒र् योनि॑र् दीक्षि॒तो दी᳚क्षि॒तो योनि॑र् दीक्षितविमि॒तम् दी᳚क्षितविमि॒तं ॅयोनि॑र् दीक्षि॒तो दी᳚क्षि॒तो योनि॑र् दीक्षितविमि॒तम् । \newline
10. योनि॑र् दीक्षितविमि॒तम् दी᳚क्षितविमि॒तं ॅयोनि॒र् योनि॑र् दीक्षितविमि॒तं ॅयद् यद् दी᳚क्षितविमि॒तं ॅयोनि॒र् योनि॑र् दीक्षितविमि॒तं ॅयत् । \newline
11. दी॒क्षि॒त॒वि॒मि॒तं ॅयद् यद् दी᳚क्षितविमि॒तम् दी᳚क्षितविमि॒तं ॅयद् दी᳚क्षि॒तो दी᳚क्षि॒तो यद् दी᳚क्षितविमि॒तम् दी᳚क्षितविमि॒तं ॅयद् दी᳚क्षि॒तः । \newline
12. दी॒क्षि॒त॒वि॒मि॒तमिति॑ दीक्षित - वि॒मि॒तम् । \newline
13. यद् दी᳚क्षि॒तो दी᳚क्षि॒तो यद् यद् दी᳚क्षि॒तो दी᳚क्षितविमि॒ताद् दी᳚क्षितविमि॒ताद् दी᳚क्षि॒तो यद् यद् दी᳚क्षि॒तो दी᳚क्षितविमि॒तात् । \newline
14. दी॒क्षि॒तो दी᳚क्षितविमि॒ताद् दी᳚क्षितविमि॒ताद् दी᳚क्षि॒तो दी᳚क्षि॒तो दी᳚क्षितविमि॒तात् प्र॒वसे᳚त् प्र॒वसे᳚द् दीक्षितविमि॒ताद् दी᳚क्षि॒तो दी᳚क्षि॒तो दी᳚क्षितविमि॒तात् प्र॒वसे᳚त् । \newline
15. दी॒क्षि॒त॒वि॒मि॒तात् प्र॒वसे᳚त् प्र॒वसे᳚द् दीक्षितविमि॒ताद् दी᳚क्षितविमि॒तात् प्र॒वसे॒द् यथा॒ यथा᳚ प्र॒वसे᳚द् दीक्षितविमि॒ताद् दी᳚क्षितविमि॒तात् प्र॒वसे॒द् यथा᳚ । \newline
16. दी॒क्षि॒त॒वि॒मि॒तादिति॑ दीक्षित - वि॒मि॒तात् । \newline
17. प्र॒वसे॒द् यथा॒ यथा᳚ प्र॒वसे᳚त् प्र॒वसे॒द् यथा॒ योने॒र् योने॒र् यथा᳚ प्र॒वसे᳚त् प्र॒वसे॒द् यथा॒ योनेः᳚ । \newline
18. प्र॒वसे॒दिति॑ प्र - वसे᳚त् । \newline
19. यथा॒ योने॒र् योने॒र् यथा॒ यथा॒ योने॒र् गर्भो॒ गर्भो॒ योने॒र् यथा॒ यथा॒ योने॒र् गर्भः॑ । \newline
20. योने॒र् गर्भो॒ गर्भो॒ योने॒र् योने॒र् गर्भः॒ स्कन्द॑ति॒ स्कन्द॑ति॒ गर्भो॒ योने॒र् योने॒र् गर्भः॒ स्कन्द॑ति । \newline
21. गर्भः॒ स्कन्द॑ति॒ स्कन्द॑ति॒ गर्भो॒ गर्भः॒ स्कन्द॑ति ता॒दृक् ता॒दृख् स्कन्द॑ति॒ गर्भो॒ गर्भः॒ स्कन्द॑ति ता॒दृक् । \newline
22. स्कन्द॑ति ता॒दृक् ता॒दृख् स्कन्द॑ति॒ स्कन्द॑ति ता॒दृ गे॒वैव ता॒दृख् स्कन्द॑ति॒ स्कन्द॑ति ता॒दृ गे॒व । \newline
23. ता॒दृ गे॒वैव ता॒दृक् ता॒दृ गे॒व तत् तदे॒व ता॒दृक् ता॒दृ गे॒व तत् । \newline
24. ए॒व तत् तदे॒ वैव तन् न न तदे॒ वैव तन् न । \newline
25. तन् न न तत् तन् न प्र॑वस्त॒व्य॑म् प्रवस्त॒व्य॑न् न तत् तन् न प्र॑वस्त॒व्य᳚म् । \newline
26. न प्र॑वस्त॒व्य॑म् प्रवस्त॒व्य॑न् न न प्र॑वस्त॒व्य॑ मा॒त्मन॑ आ॒त्मनः॑ प्रवस्त॒व्य॑न् न न प्र॑वस्त॒व्य॑ मा॒त्मनः॑ । \newline
27. प्र॒व॒स्त॒व्य॑ मा॒त्मन॑ आ॒त्मनः॑ प्रवस्त॒व्य॑म् प्रवस्त॒व्य॑ मा॒त्मनो॑ गोपी॒थाय॑ गोपी॒था या॒त्मनः॑ प्रवस्त॒व्य॑म् प्रवस्त॒व्य॑ मा॒त्मनो॑ गोपी॒थाय॑ । \newline
28. प्र॒व॒स्त॒व्य॑मिति॑ प्र - व॒स्त॒व्य᳚म् । \newline
29. आ॒त्मनो॑ गोपी॒थाय॑ गोपी॒था या॒त्मन॑ आ॒त्मनो॑ गोपी॒था यै॒ष ए॒ष गो॑पी॒था या॒त्मन॑ आ॒त्मनो॑ गोपी॒था यै॒षः । \newline
30. गो॒पी॒था यै॒ष ए॒ष गो॑पी॒थाय॑ गोपी॒था यै॒ष वै वा ए॒ष गो॑पी॒थाय॑ गोपी॒था यै॒ष वै । \newline
31. ए॒ष वै वा ए॒ष ए॒ष वै व्या॒घ्रो व्या॒घ्रो वा ए॒ष ए॒ष वै व्या॒घ्रः । \newline
32. वै व्या॒घ्रो व्या॒घ्रो वै वै व्या॒घ्रः कु॑लगो॒पः कु॑लगो॒पो व्या॒घ्रो वै वै व्या॒घ्रः कु॑लगो॒पः । \newline
33. व्या॒घ्रः कु॑लगो॒पः कु॑लगो॒पो व्या॒घ्रो व्या॒घ्रः कु॑लगो॒पो यद् यत् कु॑लगो॒पो व्या॒घ्रो व्या॒घ्रः कु॑लगो॒पो यत् । \newline
34. कु॒ल॒गो॒पो यद् यत् कु॑लगो॒पः कु॑लगो॒पो यद॒ग्नि र॒ग्निर् यत् कु॑लगो॒पः कु॑लगो॒पो यद॒ग्निः । \newline
35. कु॒ल॒गो॒प इति॑ कुल - गो॒पः । \newline
36. यद॒ग्नि र॒ग्निर् यद् यद॒ग्नि स्तस्मा॒त् तस्मा॑ द॒ग्निर् यद् यद॒ग्नि स्तस्मा᳚त् । \newline
37. अ॒ग्नि स्तस्मा॒त् तस्मा॑ द॒ग्नि र॒ग्नि स्तस्मा॒द् यद् यत् तस्मा॑ द॒ग्नि र॒ग्नि स्तस्मा॒द् यत् । \newline
38. तस्मा॒द् यद् यत् तस्मा॒त् तस्मा॒द् यद् दी᳚क्षि॒तो दी᳚क्षि॒तो यत् तस्मा॒त् तस्मा॒द् यद् दी᳚क्षि॒तः । \newline
39. यद् दी᳚क्षि॒तो दी᳚क्षि॒तो यद् यद् दी᳚क्षि॒तः प्र॒वसे᳚त् प्र॒वसे᳚द् दीक्षि॒तो यद् यद् दी᳚क्षि॒तः प्र॒वसे᳚त् । \newline
40. दी॒क्षि॒तः प्र॒वसे᳚त् प्र॒वसे᳚द् दीक्षि॒तो दी᳚क्षि॒तः प्र॒वसे॒थ् स स प्र॒वसे᳚द् दीक्षि॒तो दी᳚क्षि॒तः प्र॒वसे॒थ् सः । \newline
41. प्र॒वसे॒थ् स स प्र॒वसे᳚त् प्र॒वसे॒थ् स ए॑न मेनꣳ॒॒ स प्र॒वसे᳚त् प्र॒वसे॒थ् स ए॑नम् । \newline
42. प्र॒वसे॒दिति॑ प्र - वसे᳚त् । \newline
43. स ए॑न मेनꣳ॒॒ स स ए॑न मीश्व॒र ई᳚श्व॒र ए॑नꣳ॒॒ स स ए॑न मीश्व॒रः । \newline
44. ए॒न॒ मी॒श्व॒र ई᳚श्व॒र ए॑न मेन मीश्व॒रो॑ ऽनू॒त्थाया॑ नू॒त्था ये᳚श्व॒र ए॑न मेन मीश्व॒रो॑ ऽनू॒त्थाय॑ । \newline
45. ई॒श्व॒रो॑ ऽनू॒त्था या॑नू॒त्था ये᳚श्व॒र ई᳚श्व॒रो॑ ऽनू॒त्थाय॒ हन्तो॒र्॒. हन्तो॑ रनू॒त्था ये᳚श्व॒र ई᳚श्व॒रो॑ ऽनू॒त्थाय॒ हन्तोः᳚ । \newline
46. अ॒नू॒त्थाय॒ हन्तो॒र्॒. हन्तो॑ रनू॒त्था या॑नू॒त्थाय॒ हन्तो॒र् न न हन्तो॑ रनू॒त्था या॑नू॒त्थाय॒ हन्तो॒र् न । \newline
47. अ॒नू॒त्थायेत्य॑नु - उ॒त्थाय॑ । \newline
48. हन्तो॒र् न न हन्तो॒र्॒. हन्तो॒र् न प्र॑वस्त॒व्य॑म् प्रवस्त॒व्य॑न्न हन्तो॒र्॒. हन्तो॒र् न प्र॑वस्त॒व्य᳚म् । \newline
49. न प्र॑वस्त॒व्य॑म् प्रवस्त॒व्य॑न् न न प्र॑वस्त॒व्य॑ मा॒त्मन॑ आ॒त्मनः॑ प्रवस्त॒व्य॑न् न न प्र॑वस्त॒व्य॑ मा॒त्मनः॑ । \newline
50. प्र॒व॒स्त॒व्य॑ मा॒त्मन॑ आ॒त्मनः॑ प्रवस्त॒व्य॑म् प्रवस्त॒व्य॑ मा॒त्मनो॒ गुप्त्यै॒ गुप्त्या॑ आ॒त्मनः॑ प्रवस्त॒व्य॑म् प्रवस्त॒व्य॑ मा॒त्मनो॒ गुप्त्यै᳚ । \newline
51. प्र॒व॒स्त॒व्य॑मिति॑ प्र - व॒स्त॒व्य᳚म् । \newline
52. आ॒त्मनो॒ गुप्त्यै॒ गुप्त्या॑ आ॒त्मन॑ आ॒त्मनो॒ गुप्त्यै॑ दक्षिण॒तो द॑क्षिण॒तो गुप्त्या॑ आ॒त्मन॑ आ॒त्मनो॒ गुप्त्यै॑ दक्षिण॒तः । \newline
53. गुप्त्यै॑ दक्षिण॒तो द॑क्षिण॒तो गुप्त्यै॒ गुप्त्यै॑ दक्षिण॒तः श॑ये शये दक्षिण॒तो गुप्त्यै॒ गुप्त्यै॑ दक्षिण॒तः श॑ये । \newline
54. द॒क्षि॒ण॒तः श॑ये शये दक्षिण॒तो द॑क्षिण॒तः श॑य ए॒त दे॒त च्छ॑ये दक्षिण॒तो द॑क्षिण॒तः श॑य ए॒तत् । \newline
55. श॒य॒ ए॒त दे॒त च्छ॑ये शय ए॒तद् वै वा ए॒त च्छ॑ये शय ए॒तद् वै । \newline
56. ए॒तद् वै वा ए॒त दे॒तद् वै यज॑मानस्य॒ यज॑मानस्य॒ वा ए॒त दे॒तद् वै यज॑मानस्य । \newline
57. वै यज॑मानस्य॒ यज॑मानस्य॒ वै वै यज॑मानस्या॒ यत॑न मा॒यत॑नं॒ ॅयज॑मानस्य॒ वै वै यज॑मान स्या॒यत॑नम् । \newline
58. यज॑मानस्या॒ यत॑न मा॒यत॑नं॒ ॅयज॑मानस्य॒ यज॑मान स्या॒यत॑नꣳ॒॒ स्वे स्व आ॒यत॑नं॒ ॅयज॑मानस्य॒ यज॑मान स्या॒यत॑नꣳ॒॒ स्वे । \newline
59. आ॒यत॑नꣳ॒॒ स्वे स्व आ॒यत॑न मा॒यत॑नꣳ॒॒ स्व ए॒वैव स्व आ॒यत॑न मा॒यत॑नꣳ॒॒ स्व ए॒व । \newline
60. आ॒यत॑न॒मित्या᳚ - यत॑नम् । \newline
61. स्व ए॒वैव स्वे स्व ए॒वा यत॑न आ॒यत॑न ए॒व स्वे स्व ए॒वाय त॑ने । \newline
62. ए॒वा यत॑न आ॒यत॑न ए॒वै वायत॑ने शये शय आ॒यत॑न ए॒वैवा यत॑ने शये । \newline
63. आ॒यत॑ने शये शय आ॒यत॑न आ॒यत॑ने शये॒ ऽग्नि म॒ग्निꣳ श॑य आ॒यत॑न आ॒यत॑ने शये॒ ऽग्निम् । \newline
64. आ॒यत॑न॒ इत्या᳚ - यत॑ने । \newline
65. श॒ये॒ ऽग्नि म॒ग्निꣳ श॑ये शये॒ ऽग्नि म॑भ्या॒वृत्या᳚ भ्या॒वृत्या॒ ग्निꣳ श॑ये शये॒ ऽग्नि म॑भ्या॒वृत्य॑ । \newline
66. अ॒ग्नि म॑भ्या॒वृत्या᳚ भ्या॒वृत्या॒ग्नि म॒ग्नि म॑भ्या॒वृत्य॑ शये शये ऽभ्या॒वृत्या॒ग्नि म॒ग्नि म॑भ्या॒वृत्य॑ शये । \newline
67. अ॒भ्या॒वृत्य॑ शये शये ऽभ्या॒वृत्या᳚ भ्या॒वृत्य॑ शये दे॒वता॑ दे॒वताः᳚ शये ऽभ्या॒वृत्या᳚ भ्या॒वृत्य॑ शये दे॒वताः᳚ । \newline
68. अ॒भ्या॒वृत्येत्य॑भि - आ॒वृत्य॑ । \newline
69. श॒ये॒ दे॒वता॑ दे॒वताः᳚ शये शये दे॒वता॑ ए॒वैव दे॒वताः᳚ शये शये दे॒वता॑ ए॒व । \newline
70. दे॒वता॑ ए॒वैव दे॒वता॑ दे॒वता॑ ए॒व य॒ज्ञ्ं ॅय॒ज्ञ् मे॒व दे॒वता॑ दे॒वता॑ ए॒व य॒ज्ञ्म् । \newline
71. ए॒व य॒ज्ञ्ं ॅय॒ज्ञ् मे॒वैव य॒ज्ञ् म॑भ्या॒वृत्या᳚ भ्या॒वृत्य॑ य॒ज्ञ् मे॒वैव य॒ज्ञ् म॑भ्या॒वृत्य॑ । \newline
72. य॒ज्ञ् म॑भ्या॒वृत्या᳚ भ्या॒वृत्य॑ य॒ज्ञ्ं ॅय॒ज्ञ् म॑भ्या॒वृत्य॑ शये शये ऽभ्या॒वृत्य॑ य॒ज्ञ्ं ॅय॒ज्ञ् म॑भ्या॒वृत्य॑ शये । \newline
73. अ॒भ्या॒वृत्य॑ शये शये ऽभ्या॒वृत्या᳚ भ्या॒वृत्य॑ शये । \newline
74. अ॒भ्या॒वृत्येत्य॑भि - आ॒वृत्य॑ । \newline
75. श॒य॒ इति॑ शये । \newline
\pagebreak
\markright{ TS 6.2.6.1  \hfill https://www.vedavms.in \hfill}

\section{ TS 6.2.6.1 }

\textbf{TS 6.2.6.1 } \newline
\textbf{Samhita Paata} \newline

पु॒रोह॑विषि देव॒यज॑ने याजये॒द्यं का॒मये॒तोपै॑न॒मुत्त॑रो य॒ज्ञो न॑मेद॒भि सु॑व॒र्गं ॅलो॒कं ज॑ये॒दित्ये॒तद्वै पु॒रोह॑विर्देव॒यज॑नं॒ ॅयस्य॒ होता᳚ प्रातरनुवा॒क -म॑नुब्रु॒व-न्न॒ग्निम॒प आ॑दि॒त्यम॒भि वि॒पश्य॒त्युपै॑न॒मुत्त॑रो य॒ज्ञो न॑मत्य॒भि सु॑व॒र्गं ॅलो॒कं ज॑यत्या॒प्ते दे॑व॒यज॑ने याजये॒द् -भ्रातृ॑व्यवन्तं॒ पन्थां᳚ ॅवाऽधिस्प॒र्॒.शये᳚त् क॒र्तं ॅवा॒ याव॒न्नान॑से॒ यात॒वै- [  ] \newline

\textbf{Pada Paata} \newline

पु॒रोह॑वि॒षीति॑ पु॒रः-ह॒वि॒षि॒ । दे॒व॒यज॑न॒ इति॑ देव-यज॑ने । या॒ज॒ये॒त् । यम् । का॒मये॑त । उपेति॑ । ए॒न॒म् । उत्त॑र॒ इत्युत् - त॒रः॒ । य॒ज्ञ्ः । न॒मे॒त् । अ॒भीति॑ । सु॒व॒र्गमिति॑ सुवः-गम् । लो॒कम् । ज॒ये॒त् । इति॑ । ए॒तत् । वै । पु॒रोह॑वि॒रिति॑ पु॒रः-ह॒विः॒ । दे॒व॒यज॑न॒मिति॑ देव-यज॑नम् । यस्य॑ । होता᳚ । प्रा॒त॒र॒नु॒वा॒कमिति॑ प्रातः-अ॒नु॒वा॒कम् । अ॒नु॒ब्रु॒वन्नित्य॑नु - ब्रु॒वन्न् । अ॒ग्निम् । अ॒पः । आ॒दि॒त्यम् । अ॒भीति॑ । वि॒पश्य॒तीति॑ वि-पश्य॑ति । उपेति॑ । ए॒न॒म् । उत्त॑र॒ इत्युत् - त॒रः॒ । य॒ज्ञ्ः । न॒म॒ति॒ । अ॒भीति॑ । सु॒व॒र्गमिति॑ सुवः - गम् । लो॒कम् । ज॒य॒ति॒ । आ॒प्ते । दे॒व॒यज॑न॒ इति॑ देव-यज॑ने । या॒ज॒ये॒त् । भ्रातृ॑व्यवन्त॒मिति॒ भ्रातृ॑व्य - व॒न्त॒म् । पन्था᳚म् । वा॒ । अ॒धि॒स्प॒र्॒.॒शये॒दित्य॑धि-स्प॒र्॒.शये᳚त् । क॒र्तम् । वा॒ । याव॑त् । न । अन॑से । यात॒वै ।  \newline


\textbf{Krama Paata} \newline

पु॒रोह॑विषि देव॒यज॑ने । पु॒रोह॑वि॒षीति॑ पु॒रः - ह॒वि॒षि॒ । दे॒व॒यज॑ने याजयेत् । दे॒व॒यज॑न॒ इति॑ देव - यज॑ने । या॒ज॒ये॒द् यम् । यम् का॒मये॑त । का॒मये॒तोप॑ । उपै॑नम् । ए॒न॒मुत्त॑रः । उत्त॑रो य॒ज्ञ्ः । उत्त॑र॒ इत्युत् - त॒रः॒ । य॒ज्ञो न॑मेत् । न॒मे॒द॒भि । अ॒भि सु॑व॒र्गम् । सु॒व॒र्गम् ॅलो॒कम् । सु॒व॒र्गमिति॑ सुवः - गम् । लो॒कम् ज॑येत् । ज॒ये॒दिति॑ । इत्ये॒तत् । ए॒तद् वै । वै पु॒रोह॑विः । पु॒रोह॑विर् देव॒यज॑नम् । पु॒रोह॑वि॒रिति॑ पु॒रः - ह॒विः॒ । दे॒व॒यज॑न॒म् ॅयस्य॑ । दे॒व॒यज॑न॒मिति॑ देव - यज॑नम् । यस्य॒ होता᳚ । होता᳚ प्रातरनुवा॒कम् । प्रा॒त॒र॒नु॒वा॒कम॑नुब्रु॒वन्न् । प्रा॒त॒र॒नु॒वा॒कमिति॑ प्रातः - अ॒नु॒वा॒कम् । अ॒नु॒ब्रु॒वन्न॒ग्निम् । अ॒नु॒ब्रु॒वन्नित्य॑नु - ब्रु॒वन्न् । अ॒ग्निम॒पः । अ॒प आ॑दि॒त्यम् । आ॒दि॒त्यम॒भि । अ॒भि वि॒पश्य॑ति । वि॒पश्य॒त्युप॑ । वि॒पश्य॒तीति॑ वि - पश्य॑ति । उपै॑नम् । ए॒न॒मुत्त॑रः । उत्त॑रो य॒ज्ञ्ः । उत्त॑र॒ इत्युत् - त॒रः॒ । य॒ज्ञो न॑मति । न॒म॒त्य॒भि । अ॒भि सु॑व॒र्गम् । सु॒व॒र्गम् ॅलो॒कम् । सु॒व॒र्गमिति॑ सुवः - गम् । लो॒कम् ज॑यति । ज॒य॒त्या॒प्ते । आ॒प्ते दे॑व॒यज॑ने । दे॒व॒यज॑ने याजयेत् । दे॒व॒यज॑न॒ इति॑ देव - यज॑ने । या॒ज॒ये॒द् भ्रातृ॑व्यवन्तम् । भ्रातृ॑व्यवन्त॒म् पन्था᳚म् । भ्रातृ॑व्यवन्त॒मिति॒ भ्रातृ॑व्य - व॒न्त॒म् । पन्था᳚म् ॅवा । वा॒ऽधि॒स्प॒र्.॒शये᳚त् । अ॒धि॒स्प॒र्.॒शये᳚त् क॒र्तम् । अ॒धि॒स्प॒र्.॒शये॒दित्य॑धि - स्प॒र्.॒शये᳚त् । क॒र्तम् ॅवा᳚ । वा॒ याव॑त् । याव॒न् न । नान॑से । अन॑से॒ यात॒वै । यात॒वै न \newline

\textbf{Jatai Paata} \newline

1. पु॒रोह॑विषि देव॒यज॑ने देव॒यज॑ने पु॒रोह॑विषि पु॒रोह॑विषि देव॒यज॑ने । \newline
2. पु॒रोह॑वि॒षीति॑ पु॒रः - ह॒वि॒षि॒ । \newline
3. दे॒व॒यज॑ने याजयेद् याजयेद् देव॒यज॑ने देव॒यज॑ने याजयेत् । \newline
4. दे॒व॒यज॑न॒ इति॑ देव - यज॑ने । \newline
5. या॒ज॒ये॒द् यं ॅयं ॅया॑जयेद् याजये॒द् यम् । \newline
6. यम् का॒मये॑त का॒मये॑त॒ यं ॅयम् का॒मये॑त । \newline
7. का॒मये॒तोपोप॑ का॒मये॑त का॒मये॒तोप॑ । \newline
8. उपै॑न मेन॒ मुपोपै॑नम् । \newline
9. ए॒न॒ मुत्त॑र॒ उत्त॑र एन मेन॒ मुत्त॑रः । \newline
10. उत्त॑रो य॒ज्ञो य॒ज्ञ् उत्त॑र॒ उत्त॑रो य॒ज्ञ्ः । \newline
11. उत्त॑र॒ इत्युत् - त॒रः॒ । \newline
12. य॒ज्ञो न॑मेन् नमेद् य॒ज्ञो य॒ज्ञो न॑मेत् । \newline
13. न॒मे॒ द॒भ्य॑भि न॑मेन् नमे द॒भि । \newline
14. अ॒भि सु॑व॒र्गꣳ सु॑व॒र्ग म॒भ्य॑भि सु॑व॒र्गम् । \newline
15. सु॒व॒र्गम् ॅलो॒कम् ॅलो॒कꣳ सु॑व॒र्गꣳ सु॑व॒र्गम् ॅलो॒कम् । \newline
16. सु॒व॒र्गमिति॑ सुवः - गम् । \newline
17. लो॒कम् ज॑येज् जये ल्लो॒कम् ॅलो॒कम् ज॑येत् । \newline
18. ज॒ये॒ दितीति॑ जयेज् जये॒ दिति॑ । \newline
19. इत्ये॒त दे॒त दितीत्ये॒तत् । \newline
20. ए॒तद् वै वा ए॒त दे॒तद् वै । \newline
21. वै पु॒रोह॑विः पु॒रोह॑वि॒र् वै वै पु॒रोह॑विः । \newline
22. पु॒रोह॑विर् देव॒यज॑नम् देव॒यज॑नम् पु॒रोह॑विः पु॒रोह॑विर् देव॒यज॑नम् । \newline
23. पु॒रोह॑वि॒रिति॑ पु॒रः - ह॒विः॒ । \newline
24. दे॒व॒यज॑नं॒ ॅयस्य॒ यस्य॑ देव॒यज॑नम् देव॒यज॑नं॒ ॅयस्य॑ । \newline
25. दे॒व॒यज॑न॒मिति॑ देव - यज॑नम् । \newline
26. यस्य॒ होता॒ होता॒ यस्य॒ यस्य॒ होता᳚ । \newline
27. होता᳚ प्रातरनुवा॒कम् प्रा॑तरनुवा॒कꣳ होता॒ होता᳚ प्रातरनुवा॒कम् । \newline
28. प्रा॒त॒र॒नु॒वा॒क म॑नुब्रु॒वन् न॑नुब्रु॒वन् प्रा॑तरनुवा॒कम् प्रा॑तरनुवा॒क म॑नुब्रु॒वन्न् । \newline
29. प्रा॒त॒र॒नु॒वा॒कमिति॑ प्रातः - अ॒नु॒वा॒कम् । \newline
30. अ॒नु॒ब्रु॒वन् न॒ग्नि म॒ग्नि म॑नुब्रु॒वन् न॑नुब्रु॒वन् न॒ग्निम् । \newline
31. अ॒नु॒ब्रु॒वन्नित्य॑नु - ब्रु॒वन्न् । \newline
32. अ॒ग्नि म॒पो᳚(1॒) ऽपो᳚ ऽग्नि म॒ग्नि म॒पः । \newline
33. अ॒प आ॑दि॒त्य मा॑दि॒त्य म॒पो॑ ऽप आ॑दि॒त्यम् । \newline
34. आ॒दि॒त्य म॒भ्या᳚(1॒)भ्या॑दि॒त्य मा॑दि॒त्य म॒भि । \newline
35. अ॒भि वि॒पश्य॑ति वि॒पश्य॑ त्य॒भ्य॑भि वि॒पश्य॑ति । \newline
36. वि॒पश्य॒ त्युपोप॑ वि॒पश्य॑ति वि॒पश्य॒ त्युप॑ । \newline
37. वि॒पश्य॒तीति॑ वि - पश्य॑ति । \newline
38. उपै॑न मेन॒ मुपोपै॑नम् । \newline
39. ए॒न॒ मुत्त॑र॒ उत्त॑र एन मेन॒ मुत्त॑रः । \newline
40. उत्त॑रो य॒ज्ञो य॒ज्ञ् उत्त॑र॒ उत्त॑रो य॒ज्ञ्ः । \newline
41. उत्त॑र॒ इत्युत् - त॒रः॒ । \newline
42. य॒ज्ञो न॑मति नमति य॒ज्ञो य॒ज्ञो न॑मति । \newline
43. न॒म॒ त्य॒भ्य॑भि न॑मति नम त्य॒भि । \newline
44. अ॒भि सु॑व॒र्गꣳ सु॑व॒र्ग म॒भ्य॑भि सु॑व॒र्गम् । \newline
45. सु॒व॒र्गम् ॅलो॒कम् ॅलो॒कꣳ सु॑व॒र्गꣳ सु॑व॒र्गम् ॅलो॒कम् । \newline
46. सु॒व॒र्गमिति॑ सुवः - गम् । \newline
47. लो॒कम् ज॑यति जयति लो॒कम् ॅलो॒कम् ज॑यति । \newline
48. ज॒य॒ त्या॒प्त आ॒प्ते ज॑यति जय त्या॒प्ते । \newline
49. आ॒प्ते दे॑व॒यज॑ने देव॒यज॑न आ॒प्त आ॒प्ते दे॑व॒यज॑ने । \newline
50. दे॒व॒यज॑ने याजयेद् याजयेद् देव॒यज॑ने देव॒यज॑ने याजयेत् । \newline
51. दे॒व॒यज॑न॒ इति॑ देव - यज॑ने । \newline
52. या॒ज॒ये॒द् भ्रातृ॑व्यवन्त॒म् भ्रातृ॑व्यवन्तं ॅयाजयेद् याजये॒द् भ्रातृ॑व्यवन्तम् । \newline
53. भ्रातृ॑व्यवन्त॒म् पन्था॒म् पन्था॒म् भ्रातृ॑व्यवन्त॒म् भ्रातृ॑व्यवन्त॒म् पन्था᳚म् । \newline
54. भ्रातृ॑व्यवन्त॒मिति॒ भ्रातृ॑व्य - व॒न्त॒म् । \newline
55. पन्थां᳚ ॅवा वा॒ पन्था॒म् पन्थां᳚ ॅवा । \newline
56. वा॒ ऽधि॒स्प॒र्॒.शये॑ दधिस्प॒र्॒.शये᳚द् वा वा ऽधिस्प॒र्॒.शये᳚त् । \newline
57. अ॒धि॒स्प॒र्॒.शये᳚त् क॒र्तम् क॒र्त म॑धिस्प॒र्॒.शये॑ दधिस्प॒र्॒.शये᳚त् क॒र्तम् । \newline
58. अ॒धि॒स्प॒र्॒.॒शये॒दित्य॑धि - स्प॒र्॒.शये᳚त् । \newline
59. क॒र्तं ॅवा॑ वा क॒र्तम् क॒र्तं ॅवा᳚ । \newline
60. वा॒ याव॒द् याव॑द् वा वा॒ याव॑त् । \newline
61. याव॒न् न न याव॒द् याव॒न् न । \newline
62. ना न॒से ऽन॑से॒ न ना न॑से । \newline
63. अन॑से॒ यात॒वै यात॒वा अन॒से ऽन॑से॒ यात॒वै । \newline
64. यात॒वै न न यात॒वै यात॒वै न । \newline

\textbf{Ghana Paata } \newline

1. पु॒रोह॑विषि देव॒यज॑ने देव॒यज॑ने पु॒रोह॑विषि पु॒रोह॑विषि देव॒यज॑ने याजयेद् याजयेद् देव॒यज॑ने पु॒रोह॑विषि पु॒रोह॑विषि देव॒यज॑ने याजयेत् । \newline
2. पु॒रोह॑वि॒षीति॑ पु॒रः - ह॒वि॒षि॒ । \newline
3. दे॒व॒यज॑ने याजयेद् याजयेद् देव॒यज॑ने देव॒यज॑ने याजये॒द् यं ॅयं ॅया॑जयेद् देव॒यज॑ने देव॒यज॑ने याजये॒द् यम् । \newline
4. दे॒व॒यज॑न॒ इति॑ देव - यज॑ने । \newline
5. या॒ज॒ये॒द् यं ॅयं ॅया॑जयेद् याजये॒द् यम् का॒मये॑त का॒मये॑त॒ यं ॅया॑जयेद् याजये॒द् यम् का॒मये॑त । \newline
6. यम् का॒मये॑त का॒मये॑त॒ यं ॅयम् का॒मये॒तो पोप॑ का॒मये॑त॒ यं ॅयम् का॒मये॒तोप॑ । \newline
7. का॒मये॒तो पोप॑ का॒मये॑त का॒मये॒ तोपै॑न मेन॒ मुप॑ का॒मये॑त का॒मये॒ तोपै॑नम् । \newline
8. उपै॑न मेन॒ मुपो पै॑न॒ मुत्त॑र॒ उत्त॑र एन॒ मुपो पै॑न॒ मुत्त॑रः । \newline
9. ए॒न॒ मुत्त॑र॒ उत्त॑र एन मेन॒ मुत्त॑रो य॒ज्ञो य॒ज्ञ् उत्त॑र एन मेन॒ मुत्त॑रो य॒ज्ञ्ः । \newline
10. उत्त॑रो य॒ज्ञो य॒ज्ञ् उत्त॑र॒ उत्त॑रो य॒ज्ञो न॑मेन् नमेद् य॒ज्ञ् उत्त॑र॒ उत्त॑रो य॒ज्ञो न॑मेत् । \newline
11. उत्त॑र॒ इत्युत् - त॒रः॒ । \newline
12. य॒ज्ञो न॑मेन् नमेद् य॒ज्ञो य॒ज्ञो न॑मे द॒भ्य॑भि न॑मेद् य॒ज्ञो य॒ज्ञो न॑मेद॒भि । \newline
13. न॒मे॒ द॒भ्य॑भि न॑मेन् नमेद॒भि सु॑व॒र्गꣳ सु॑व॒र्ग म॒भि न॑मेन् नमेद॒भि सु॑व॒र्गम् । \newline
14. अ॒भि सु॑व॒र्गꣳ सु॑व॒र्ग म॒भ्य॑भि सु॑व॒र्गम् ॅलो॒कम् ॅलो॒कꣳ सु॑व॒र्ग म॒भ्य॑भि सु॑व॒र्गम् ॅलो॒कम् । \newline
15. सु॒व॒र्गम् ॅलो॒कम् ॅलो॒कꣳ सु॑व॒र्गꣳ सु॑व॒र्गम् ॅलो॒कम् ज॑येज् जये ल्लो॒कꣳ सु॑व॒र्गꣳ सु॑व॒र्गम् ॅलो॒कम् ज॑येत् । \newline
16. सु॒व॒र्गमिति॑ सुवः - गम् । \newline
17. लो॒कम् ज॑येज् जये ल्लो॒कम् ॅलो॒कम् ज॑ये॒दि तीति॑ जये ल्लो॒कम् ॅलो॒कम् ज॑ये॒ दिति॑ । \newline
18. ज॒ये॒ दितीति॑ जयेज् जये॒ दित्ये॒त दे॒त दिति॑ जयेज् जये॒ दित्ये॒तत् । \newline
19. इत्ये॒त दे॒त दिती त्ये॒तद् वै वा ए॒तदिती त्ये॒तद् वै । \newline
20. ए॒तद् वै वा ए॒त दे॒तद् वै पु॒रोह॑विः पु॒रोह॑वि॒र् वा ए॒त दे॒तद् वै पु॒रोह॑विः । \newline
21. वै पु॒रोह॑विः पु॒रोह॑वि॒र् वै वै पु॒रोह॑विर् देव॒यज॑नम् देव॒यज॑नम् पु॒रोह॑वि॒र् वै वै पु॒रोह॑विर् देव॒यज॑नम् । \newline
22. पु॒रोह॑विर् देव॒यज॑नम् देव॒यज॑नम् पु॒रोह॑विः पु॒रोह॑विर् देव॒यज॑नं॒ ॅयस्य॒ यस्य॑ देव॒यज॑नम् पु॒रोह॑विः पु॒रोह॑विर् देव॒यज॑नं॒ ॅयस्य॑ । \newline
23. पु॒रोह॑वि॒रिति॑ पु॒रः - ह॒विः॒ । \newline
24. दे॒व॒यज॑नं॒ ॅयस्य॒ यस्य॑ देव॒यज॑नम् देव॒यज॑नं॒ ॅयस्य॒ होता॒ होता॒ यस्य॑ देव॒यज॑नम् देव॒यज॑नं॒ ॅयस्य॒ होता᳚ । \newline
25. दे॒व॒यज॑न॒मिति॑ देव - यज॑नम् । \newline
26. यस्य॒ होता॒ होता॒ यस्य॒ यस्य॒ होता᳚ प्रातरनुवा॒कम् प्रा॑तरनुवा॒कꣳ होता॒ यस्य॒ यस्य॒ होता᳚ प्रातरनुवा॒कम् । \newline
27. होता᳚ प्रातरनुवा॒कम् प्रा॑तरनुवा॒कꣳ होता॒ होता᳚ प्रातरनुवा॒क म॑नुब्रु॒वन् न॑नुब्रु॒वन् प्रा॑तरनुवा॒कꣳ होता॒ होता᳚ प्रातरनुवा॒क म॑नुब्रु॒वन्न् । \newline
28. प्रा॒त॒र॒नु॒वा॒क म॑नुब्रु॒वन् न॑नुब्रु॒वन् प्रा॑तरनुवा॒कम् प्रा॑तरनुवा॒क म॑नुब्रु॒वन् न॒ग्नि म॒ग्नि म॑नुब्रु॒वन् प्रा॑तरनुवा॒कम् प्रा॑तरनुवा॒क म॑नुब्रु॒वन् न॒ग्निम् । \newline
29. प्रा॒त॒र॒नु॒वा॒कमिति॑ प्रातः - अ॒नु॒वा॒कम् । \newline
30. अ॒नु॒ब्रु॒वन् न॒ग्नि म॒ग्नि म॑नुब्रु॒वन् न॑नुब्रु॒वन् न॒ग्नि म॒पो᳚(1॒) ऽपो᳚ ऽग्नि म॑नुब्रु॒वन् न॑नुब्रु॒वन् न॒ग्नि म॒पः । \newline
31. अ॒नु॒ब्रु॒वन्नित्य॑नु - ब्रु॒वन्न् । \newline
32. अ॒ग्नि म॒पो᳚(1॒) ऽपो᳚ ऽग्नि म॒ग्नि म॒प आ॑दि॒त्य मा॑दि॒त्य म॒पो᳚ ऽग्नि म॒ग्नि म॒प आ॑दि॒त्यम् । \newline
33. अ॒प आ॑दि॒त्य मा॑दि॒त्य म॒पो॑ ऽप आ॑दि॒त्य म॒भ्या᳚(1॒)भ्या॑दि॒त्य म॒पो॑ ऽप आ॑दि॒त्य म॒भि । \newline
34. आ॒दि॒त्य म॒भ्या᳚(1॒)भ्या॑दि॒त्य मा॑दि॒त्य म॒भि वि॒पश्य॑ति वि॒पश्य॑ त्य॒भ्या॑दि॒त्य मा॑दि॒त्य म॒भि वि॒पश्य॑ति । \newline
35. अ॒भि वि॒पश्य॑ति वि॒पश्य॑ त्य॒भ्य॑भि वि॒पश्य॒ त्युपोप॑ वि॒पश्य॑ त्य॒भ्य॑भि वि॒पश्य॒त्युप॑ । \newline
36. वि॒पश्य॒ त्युपोप॑ वि॒पश्य॑ति वि॒पश्य॒ त्युपै॑न मेन॒ मुप॑ वि॒पश्य॑ति वि॒पश्य॒ त्युपै॑नम् । \newline
37. वि॒पश्य॒तीति॑ वि - पश्य॑ति । \newline
38. उपै॑न मेन॒ मुपो पै॑न॒ मुत्त॑र॒ उत्त॑र एन॒ मुपो पै॑न॒ मुत्त॑रः । \newline
39. ए॒न॒ मुत्त॑र॒ उत्त॑र एन मेन॒ मुत्त॑रो य॒ज्ञो य॒ज्ञ् उत्त॑र एन मेन॒ मुत्त॑रो य॒ज्ञ्ः । \newline
40. उत्त॑रो य॒ज्ञो य॒ज्ञ् उत्त॑र॒ उत्त॑रो य॒ज्ञो न॑मति नमति य॒ज्ञ् उत्त॑र॒ उत्त॑रो य॒ज्ञो न॑मति । \newline
41. उत्त॑र॒ इत्युत् - त॒रः॒ । \newline
42. य॒ज्ञो न॑मति नमति य॒ज्ञो य॒ज्ञो न॑म त्य॒भ्य॑भि न॑मति य॒ज्ञो य॒ज्ञो न॑मत्य॒भि । \newline
43. न॒म॒ त्य॒भ्य॑भि न॑मति नम त्य॒भि सु॑व॒र्गꣳ सु॑व॒र्ग म॒भि न॑मति नम त्य॒भि सु॑व॒र्गम् । \newline
44. अ॒भि सु॑व॒र्गꣳ सु॑व॒र्ग म॒भ्य॑भि सु॑व॒र्गम् ॅलो॒कम् ॅलो॒कꣳ सु॑व॒र्ग म॒भ्य॑भि सु॑व॒र्गम् ॅलो॒कम् । \newline
45. सु॒व॒र्गम् ॅलो॒कम् ॅलो॒कꣳ सु॑व॒र्गꣳ सु॑व॒र्गम् ॅलो॒कम् ज॑यति जयति लो॒कꣳ सु॑व॒र्गꣳ सु॑व॒र्गम् ॅलो॒कम् ज॑यति । \newline
46. सु॒व॒र्गमिति॑ सुवः - गम् । \newline
47. लो॒कम् ज॑यति जयति लो॒कम् ॅलो॒कम् ज॑य त्या॒प्त आ॒प्ते ज॑यति लो॒कम् ॅलो॒कम् ज॑य त्या॒प्ते । \newline
48. ज॒य॒ त्या॒प्त आ॒प्ते ज॑यति जयत्या॒प्ते दे॑व॒यज॑ने देव॒यज॑न आ॒प्ते ज॑यति जय त्या॒प्ते दे॑व॒यज॑ने । \newline
49. आ॒प्ते दे॑व॒यज॑ने देव॒यज॑न आ॒प्त आ॒प्ते दे॑व॒यज॑ने याजयेद् याजयेद् देव॒यज॑न आ॒प्त आ॒प्ते दे॑व॒यज॑ने याजयेत् । \newline
50. दे॒व॒यज॑ने याजयेद् याजयेद् देव॒यज॑ने देव॒यज॑ने याजये॒द् भ्रातृ॑व्यवन्त॒म् भ्रातृ॑व्यवन्तं ॅयाजयेद् देव॒यज॑ने देव॒यज॑ने याजये॒द् भ्रातृ॑व्यवन्तम् । \newline
51. दे॒व॒यज॑न॒ इति॑ देव - यज॑ने । \newline
52. या॒ज॒ये॒द् भ्रातृ॑व्यवन्त॒म् भ्रातृ॑व्यवन्तं ॅयाजयेद् याजये॒द् भ्रातृ॑व्यवन्त॒म् पन्था॒म् पन्था॒म् भ्रातृ॑व्यवन्तं ॅयाजयेद् याजये॒द् भ्रातृ॑व्यवन्त॒म् पन्था᳚म् । \newline
53. भ्रातृ॑व्यवन्त॒म् पन्था॒म् पन्था॒म् भ्रातृ॑व्यवन्त॒म् भ्रातृ॑व्यवन्त॒म् पन्थां᳚ ॅवा वा॒ पन्था॒म् भ्रातृ॑व्यवन्त॒म् भ्रातृ॑व्यवन्त॒म् पन्थां᳚ ॅवा । \newline
54. भ्रातृ॑व्यवन्त॒मिति॒ भ्रातृ॑व्य - व॒न्त॒म् । \newline
55. पन्थां᳚ ॅवा वा॒ पन्था॒म् पन्थां᳚ ॅवा ऽधिस्प॒र्॒.शये॑ दधिस्प॒र्॒.शये᳚द् वा॒ पन्था॒म् पन्थां᳚ ॅवा ऽधिस्प॒र्॒.शये᳚त् । \newline
56. वा॒ ऽधि॒स्प॒र्॒.शये॑ दधिस्प॒र्॒.शये᳚द् वा वा ऽधिस्प॒र्॒.शये᳚त् क॒र्तम् क॒र्त म॑धिस्प॒र्॒.शये᳚द् वा वा ऽधिस्प॒र्॒.शये᳚त् क॒र्तम् । \newline
57. अ॒धि॒स्प॒र्॒.शये᳚त् क॒र्तम् क॒र्त म॑धिस्प॒र्॒.शये॑ दधिस्प॒र्॒.शये᳚त् क॒र्तं ॅवा॑ वा क॒र्त म॑धिस्प॒र्॒.शये॑ दधिस्प॒र्॒.शये᳚त् क॒र्तं ॅवा᳚ । \newline
58. अ॒धि॒स्प॒र्॒.॒शये॒दित्य॑धि - स्प॒र्॒.शये᳚त् । \newline
59. क॒र्तं ॅवा॑ वा क॒र्तम् क॒र्तं ॅवा॒ याव॒द् याव॑द् वा क॒र्तम् क॒र्तं ॅवा॒ याव॑त् । \newline
60. वा॒ याव॒द् याव॑द् वा वा॒ याव॒न् न न याव॑द् वा वा॒ याव॒न् न । \newline
61. याव॒न् न न याव॒द् याव॒न् नान॒से ऽन॑से॒ न याव॒द् याव॒न् नान॑से । \newline
62. नान॒से ऽन॑से॒ न नान॑से॒ यात॒वै यात॒वा अन॑से॒ न नान॑से॒ यात॒वै । \newline
63. अन॑से॒ यात॒वै यात॒वा अन॒से ऽन॑से॒ यात॒वै न न यात॒वा अन॒से ऽन॑से॒ यात॒वै न । \newline
64. यात॒वै न न यात॒वै यात॒वै न रथा॑य॒ रथा॑य॒ न यात॒वै यात॒वै न रथा॑य । \newline
\pagebreak
\markright{ TS 6.2.6.2  \hfill https://www.vedavms.in \hfill}

\section{ TS 6.2.6.2 }

\textbf{TS 6.2.6.2 } \newline
\textbf{Samhita Paata} \newline

न रथा॑यै॒तद्वा आ॒प्तं दे॑व॒यज॑नमा॒प्नोत्ये॒व भ्रातृ॑व्यं॒ नैनं॒ भ्रातृ॑व्य आप्नो॒त्येको᳚न्नते देव॒य॑जने याजयेत् प॒शुका॑म॒-मेको᳚न्नता॒द्वै दे॑व॒यज॑ना॒दङ्गि॑रसः प॒शून॑सृजन्तान्त॒रा स॑दोहविर्द्धा॒ने उ॑न्न॒तꣳ स्या॑दे॒तद्वा एको᳚न्नतं देव॒यज॑नं पशु॒माने॒व भ॑वति॒ त्र्यु॑न्नते देव॒यज॑ने याजयेथ् सुव॒र्गका॑मं॒ त्र्यु॑न्नता॒द्वै दे॑व॒यज॑ना॒दङ्गि॑रसः सुव॒र्गं ॅलो॒कमा॑य-न्नन्त॒रा ऽऽह॑व॒नीयं च हवि॒र्द्धानं॑ चो- [  ] \newline

\textbf{Pada Paata} \newline

न । रथा॑य । ए॒तत् । वै । आ॒प्तम् । दे॒व॒यज॑न॒मिति॑ देव - यज॑नम् । आ॒प्नोति॑ । ए॒व । भ्रातृ॑व्यम् । न । ए॒न॒म् । भ्रातृ॑व्यः । आ॒प्नो॒ति॒ । एको᳚न्नत॒ इत्येक॑-उ॒न्न॒ते॒ । दे॒व॒यज॑न॒ इति॑ देव - यज॑ने । या॒ज॒ये॒त् । प॒शुका॑म॒मिति॑ प॒शु - का॒म॒म् । एको᳚न्नता॒दित्येक॑ - उ॒न्न॒ता॒त् । वै । दे॒व॒यज॑ना॒दिति॑ देव - यज॑नात् । अङ्गि॑रसः । प॒शून् । अ॒सृ॒ज॒न्त॒ । अ॒न्त॒रा । स॒दो॒ह॒वि॒द्‌र्धा॒ने इति॑ सदः-ह॒वि॒द्‌र्धा॒ने । उ॒न्न॒तमित्यु॑त् - न॒तम् । स्या॒त् । ए॒तत् । वै । एको᳚न्नत॒मित्येक॑ - उ॒न्न॒त॒म् । दे॒व॒यज॑न॒मिति॑ देव - यज॑नम् । प॒शु॒मानिति॑ पशु - मान् । ए॒व । भ॒व॒ति॒ । त्र्यु॑न्नत॒ इति॒ त्रि - उ॒न्न॒ते॒ । दे॒व॒यज॑न॒ इति॑ देव - यज॑ने । या॒ज॒ये॒त् । सु॒व॒र्गका॑म॒मिति॑ सुव॒र्ग - का॒म॒म् । त्र्यु॑न्नता॒दिति॒ त्रि - उ॒न्न॒ता॒त् । वै । दे॒व॒यज॑ना॒दिति॑ देव - यज॑नात् । अङ्गि॑रसः । सु॒व॒र्गमिति॑ सुवः - गम् । लो॒कम् । आ॒य॒न्न् । अ॒न्त॒रा । आ॒ह॒व॒नीय॒मित्या᳚-ह॒व॒नीय᳚म् । च॒ । ह॒वि॒द्‌र्धान॒मिति॑ हविः - धान᳚म् । च॒ ।  \newline


\textbf{Krama Paata} \newline

न रथा॑य । रथा॑यै॒तत् । ए॒तद् वै । वा आ॒प्तम् । आ॒प्तम् दे॑व॒यज॑नम् । दे॒व॒यज॑नमा॒प्नोति॑ । दे॒व॒यज॑न॒मिति॑ देव - यज॑नम् । आ॒प्नोत्ये॒व । ए॒व भ्रातृ॑व्यम् । भ्रातृ॑व्य॒म् न । नैन᳚म् । ए॒न॒म् भ्रातृ॑व्यः । भ्रातृ॑व्य आप्नोति । आ॒प्नो॒त्येको᳚न्नते । एको᳚न्नते देव॒यज॑ने । एको᳚न्नत॒ इत्येक॑ - उ॒न्न॒ते॒ । दे॒व॒यज॑ने याजयेत् । दे॒व॒यज॑न॒ इति॑ देव - यज॑ने । या॒ज॒ये॒त् प॒शुका॑मम् । प॒शुका॑म॒मेको᳚न्नतात् । प॒शुका॑म॒मिति॑ प॒शु - का॒म॒म् । एको᳚न्नता॒द् वै । एको᳚न्नता॒दित्येक॑ - उ॒न्न॒ता॒त्॒ । वै दे॑व॒यज॑नात् । दे॒व॒यज॑ना॒दङ्‍गि॑रसः । दे॒व॒यज॑ना॒दिति॑ देव - यज॑नात् । अङ्‍गि॑रसः प॒शून् । प॒शून॑सृजन्त । अ॒सृ॒ज॒न्ता॒न्त॒रा । अ॒न्त॒रा स॑दोहविर्द्धा॒ने । स॒दो॒ह॒वि॒र्द्धा॒ने उ॑न्न॒तम् । स॒दो॒ह॒वि॒र्द्धा॒ने इति॑ सदः - ह॒वि॒र्द्धा॒ने । उ॒न्न॒तꣳ स्या᳚त् । उ॒न्न॒तमित्यु॑त् - न॒तम् । स्या॒दे॒तत् । ए॒तद् वै । वा एको᳚न्नतम् । एको᳚न्नतम् देव॒यज॑नम् । एको᳚न्नत॒मित्येक॑ - उ॒न्न॒त॒म् । दे॒व॒यज॑नम् पशु॒मान् । दे॒व॒यज॑न॒मिति॑ देव - यज॑नम् । प॒शु॒माने॒व । प॒शु॒मानिति॑ पशु - मान् । ए॒व भ॑वति । भ॒व॒ति॒ त्र्यु॑न्नते । त्र्यु॑न्नते देव॒यज॑ने । त्र्यु॑न्नत॒ इति॒ त्रि - उ॒न्न॒ते॒ । दे॒व॒यज॑ने याजयेत् । दे॒व॒यज॑न॒ इति॑ देव - यज॑ने । या॒ज॒ये॒थ् सु॒व॒र्गका॑मम् । सु॒व॒र्गका॑म॒म् त्र्यु॑न्नतात् । सु॒व॒र्गका॑म॒मिति॑ सुव॒र्ग - का॒म॒म् । त्र्यु॑न्नता॒द् वै । त्र्यु॑न्नता॒दिति॒ त्रि - उ॒न्न॒ता॒त्॒ । वै दे॑व॒यज॑नात् । दे॒व॒यज॑ना॒दङ्‍गि॑रसः । दे॒व॒यज॑ना॒दिति॑ देव - यज॑नात् । अङ्‍गि॑रसः सुव॒र्गम् । सु॒व॒र्गम् ॅलो॒कम् । सु॒व॒र्गमिति॑ सुवः - गम् । लो॒कमा॑यन्न् । आ॒य॒न्न॒न्त॒रा । अ॒न्त॒राऽऽह॑व॒नीय᳚म् । आ॒ह॒व॒नीय॑म् च । आ॒ह॒व॒नीय॒मित्या᳚ - ह॒व॒नीय᳚म् । च॒ ह॒वि॒र्द्धान᳚म् । ह॒वि॒र्द्धान॑म् च । ह॒वि॒र्द्धान॒मिति॑ हविः - धान᳚म् । चो॒न्न॒तम् \newline

\textbf{Jatai Paata} \newline

1. न रथा॑य॒ रथा॑य॒ न न रथा॑य । \newline
2. रथा॑ यै॒त दे॒तद् रथा॑य॒ रथा॑ यै॒तत् । \newline
3. ए॒तद् वै वा ए॒त दे॒तद् वै । \newline
4. वा आ॒प्त मा॒प्तं ॅवै वा आ॒प्तम् । \newline
5. आ॒प्तम् दे॑व॒यज॑नम् देव॒यज॑न मा॒प्त मा॒प्तम् दे॑व॒यज॑नम् । \newline
6. दे॒व॒यज॑न मा॒प्नो त्या॒प्नोति॑ देव॒यज॑नम् देव॒यज॑न मा॒प्नोति॑ । \newline
7. दे॒व॒यज॑न॒मिति॑ देव - यज॑नम् । \newline
8. आ॒प्नो त्ये॒वैवा प्नो त्या॒प्नो त्ये॒व । \newline
9. ए॒व भ्रातृ॑व्य॒म् भ्रातृ॑व्य मे॒वैव भ्रातृ॑व्यम् । \newline
10. भ्रातृ॑व्य॒न्न न भ्रातृ॑व्य॒म् भ्रातृ॑व्य॒न्न । \newline
11. नैन॑ मेन॒न्न नैन᳚म् । \newline
12. ए॒न॒म् भ्रातृ॑व्यो॒ भ्रातृ॑व्य एन मेन॒म् भ्रातृ॑व्यः । \newline
13. भ्रातृ॑व्य आप्नो त्याप्नोति॒ भ्रातृ॑व्यो॒ भ्रातृ॑व्य आप्नोति । \newline
14. आ॒प्नो॒ त्येको᳚न्नत॒ एको᳚न्नत आप्नो त्याप्नो॒ त्येको᳚न्नते । \newline
15. एको᳚न्नते देव॒यज॑ने देव॒यज॑न॒ एको᳚न्नत॒ एको᳚न्नते देव॒यज॑ने । \newline
16. एको᳚न्नत॒ इत्येक॑ - उ॒न्न॒ते॒ । \newline
17. दे॒व॒यज॑ने याजयेद् याजयेद् देव॒यज॑ने देव॒यज॑ने याजयेत् । \newline
18. दे॒व॒यज॑न॒ इति॑ देव - यज॑ने । \newline
19. या॒ज॒ये॒त् प॒शुका॑मम् प॒शुका॑मं ॅयाजयेद् याजयेत् प॒शुका॑मम् । \newline
20. प॒शुका॑म॒ मेको᳚न्नता॒देको᳚न्नतात् प॒शुका॑मम् प॒शुका॑म॒ मेको᳚न्नतात् । \newline
21. प॒शुका॑म॒मिति॑ प॒शु - का॒म॒म् । \newline
22. एको᳚न्नता॒द् वै वा एको᳚न्नता॒ देको᳚न्नता॒द् वै । \newline
23. एको᳚न्नता॒दित्येक॑ - उ॒न्न॒ता॒त् । \newline
24. वै दे॑व॒यज॑नाद् देव॒यज॑ना॒द् वै वै दे॑व॒यज॑नात् । \newline
25. दे॒व॒यज॑ना॒ दङ्गि॑र॒सो ऽङ्गि॑रसो देव॒यज॑नाद् देव॒यज॑ना॒ दङ्गि॑रसः । \newline
26. दे॒व॒यज॑ना॒दिति॑ देव - यज॑नात् । \newline
27. अङ्गि॑रसः प॒शून् प॒शू नङ्गि॑र॒सो ऽङ्गि॑रसः प॒शून् । \newline
28. प॒शू न॑सृजन्ता सृजन्त प॒शून् प॒शू न॑सृजन्त । \newline
29. अ॒सृ॒ज॒न्ता॒ न्त॒रा ऽन्त॒रा ऽसृ॑जन्ता सृजन्ता न्त॒रा । \newline
30. अ॒न्त॒रा स॑दोहविर्द्धा॒ने स॑दोहविर्द्धा॒ने अ॑न्त॒रा ऽन्त॒रा स॑दोहविर्द्धा॒ने । \newline
31. स॒दो॒ह॒वि॒र्द्धा॒ने उ॑न्न॒त मु॑न्न॒तꣳ स॑दोहविर्द्धा॒ने स॑दोहविर्द्धा॒ने उ॑न्न॒तम् । \newline
32. स॒दो॒ह॒वि॒र्द्धा॒ने इति॑ सदः - ह॒वि॒र्द्धा॒ने । \newline
33. उ॒न्न॒तꣳ स्या᳚थ् स्या दुन्न॒त मु॑न्न॒तꣳ स्या᳚त् । \newline
34. उ॒न्न॒तमित्यु॑त् - न॒तम् । \newline
35. स्या॒ दे॒त दे॒तथ् स्या᳚थ् स्या दे॒तत् । \newline
36. ए॒तद् वै वा ए॒त दे॒तद् वै । \newline
37. वा एको᳚न्नत॒ मेको᳚न्नतं॒ ॅवै वा एको᳚न्नतम् । \newline
38. एको᳚न्नतम् देव॒यज॑नम् देव॒यज॑न॒ मेको᳚न्नत॒ मेको᳚न्नतम् देव॒यज॑नम् । \newline
39. एको᳚न्नत॒मित्येक॑ - उ॒न्न॒त॒म् । \newline
40. दे॒व॒यज॑नम् पशु॒मान् प॑शु॒मान् दे॑व॒यज॑नम् देव॒यज॑नम् पशु॒मान् । \newline
41. दे॒व॒यज॑न॒मिति॑ देव - यज॑नम् । \newline
42. प॒शु॒मा ने॒वैव प॑शु॒मान् प॑शु॒मा ने॒व । \newline
43. प॒शु॒मानिति॑ पशु - मान् । \newline
44. ए॒व भ॑वति भव त्ये॒वैव भ॑वति । \newline
45. भ॒व॒ति॒ त्र्यु॑न्नते॒ त्र्यु॑न्नते भवति भवति॒ त्र्यु॑न्नते । \newline
46. त्र्यु॑न्नते देव॒यज॑ने देव॒यज॑ने॒ त्र्यु॑न्नते॒ त्र्यु॑न्नते देव॒यज॑ने । \newline
47. त्र्यु॑न्नत॒ इति॒ त्रि - उ॒न्न॒ते॒ । \newline
48. दे॒व॒यज॑ने याजयेद् याजयेद् देव॒यज॑ने देव॒यज॑ने याजयेत् । \newline
49. दे॒व॒यज॑न॒ इति॑ देव - यज॑ने । \newline
50. या॒ज॒ये॒थ् सु॒व॒र्गका॑मꣳ सुव॒र्गका॑मं ॅयाजयेद् याजयेथ् सुव॒र्गका॑मम् । \newline
51. सु॒व॒र्गका॑म॒म् त्र्यु॑न्नता॒त् त्र्यु॑न्नताथ् सुव॒र्गका॑मꣳ सुव॒र्गका॑म॒म् त्र्यु॑न्नतात् । \newline
52. सु॒व॒र्गका॑म॒मिति॑ सुव॒र्ग - का॒म॒म् । \newline
53. त्र्यु॑न्नता॒द् वै वै त्र्यु॑न्नता॒त् त्र्यु॑न्नता॒द् वै । \newline
54. त्र्यु॑न्नता॒दिति॒ त्रि - उ॒न्न॒ता॒त् । \newline
55. वै दे॑व॒यज॑नाद् देव॒यज॑ना॒द् वै वै दे॑व॒यज॑नात् । \newline
56. दे॒व॒यज॑ना॒ दङ्गि॑र॒सो ऽङ्गि॑रसो देव॒यज॑नाद् देव॒यज॑ना॒ दङ्गि॑रसः । \newline
57. दे॒व॒यज॑ना॒दिति॑ देव - यज॑नात् । \newline
58. अङ्गि॑रसः सुव॒र्गꣳ सु॑व॒र्ग मङ्गि॑र॒सो ऽङ्गि॑रसः सुव॒र्गम् । \newline
59. सु॒व॒र्गम् ॅलो॒कम् ॅलो॒कꣳ सु॑व॒र्गꣳ सु॑व॒र्गम् ॅलो॒कम् । \newline
60. सु॒व॒र्गमिति॑ सुवः - गम् । \newline
61. लो॒क मा॑यन् नायन् ॅलो॒कम् ॅलो॒क मा॑यन्न् । \newline
62. आ॒य॒न् न॒न्त॒रा ऽन्त॒रा ऽऽय॑न् नायन् नन्त॒रा । \newline
63. अ॒न्त॒रा ऽऽह॑व॒नीय॑ माहव॒नीय॑ मन्त॒रा ऽन्त॒रा ऽऽह॑व॒नीय᳚म् । \newline
64. आ॒ह॒व॒नीय॑म् च चाहव॒नीय॑ माहव॒नीय॑म् च । \newline
65. आ॒ह॒व॒नीय॒मित्या᳚ - ह॒व॒नीय᳚म् । \newline
66. च॒ ह॒वि॒र्द्धानꣳ॑ हवि॒र्द्धान॑म् च च हवि॒र्द्धान᳚म् । \newline
67. ह॒वि॒र्द्धान॑म् च च हवि॒र्द्धानꣳ॑ हवि॒र्द्धान॑म् च । \newline
68. ह॒वि॒र्द्धान॒मिति॑ हविः - धान᳚म् । \newline
69. चो॒न्न॒त मु॑न्न॒तम् च॑ चोन्न॒तम् । \newline

\textbf{Ghana Paata } \newline

1. न रथा॑य॒ रथा॑य॒ न न रथा॑यै॒त दे॒तद् रथा॑य॒ न न रथा॑ यै॒तत् । \newline
2. रथा॑ यै॒त दे॒तद् रथा॑य॒ रथा॑ यै॒तद् वै वा ए॒तद् रथा॑य॒ रथा॑ यै॒तद् वै । \newline
3. ए॒तद् वै वा ए॒त दे॒तद् वा आ॒प्त मा॒प्तं ॅवा ए॒त दे॒तद् वा आ॒प्तम् । \newline
4. वा आ॒प्त मा॒प्तं ॅवै वा आ॒प्तम् दे॑व॒यज॑नम् देव॒यज॑न मा॒प्तं ॅवै वा आ॒प्तम् दे॑व॒यज॑नम् । \newline
5. आ॒प्तम् दे॑व॒यज॑नम् देव॒यज॑न मा॒प्त मा॒प्तम् दे॑व॒यज॑न मा॒प्नो त्या॒प्नोति॑ देव॒यज॑न मा॒प्त मा॒प्तम् दे॑व॒यज॑न मा॒प्नोति॑ । \newline
6. दे॒व॒यज॑न मा॒प्नो त्या॒प्नोति॑ देव॒यज॑नम् देव॒यज॑न मा॒प्नो त्ये॒वै वाप्नोति॑ देव॒यज॑नम् देव॒यज॑न मा॒प्नो त्ये॒व । \newline
7. दे॒व॒यज॑न॒मिति॑ देव - यज॑नम् । \newline
8. आ॒प्नो त्ये॒वै वाप्नो त्या॒प्नो त्ये॒व भ्रातृ॑व्य॒म् भ्रातृ॑व्य मे॒वाप्नो त्या॒प्नो त्ये॒व भ्रातृ॑व्यम् । \newline
9. ए॒व भ्रातृ॑व्य॒म् भ्रातृ॑व्य मे॒वैव भ्रातृ॑व्य॒न् न न भ्रातृ॑व्य मे॒वैव भ्रातृ॑व्य॒न् न । \newline
10. भ्रातृ॑व्य॒न् न न भ्रातृ॑व्य॒म् भ्रातृ॑व्य॒न् नैन॑ मेन॒न् न भ्रातृ॑व्य॒म् भ्रातृ॑व्य॒म् नैन᳚म् । \newline
11. नैन॑ मेन॒न् न नैन॒म् भ्रातृ॑व्यो॒ भ्रातृ॑व्य एन॒न् न नैन॒म् भ्रातृ॑व्यः । \newline
12. ए॒न॒म् भ्रातृ॑व्यो॒ भ्रातृ॑व्य एन मेन॒म् भ्रातृ॑व्य आप्नो त्याप्नोति॒ भ्रातृ॑व्य एन मेन॒म् भ्रातृ॑व्य आप्नोति । \newline
13. भ्रातृ॑व्य आप्नो त्याप्नोति॒ भ्रातृ॑व्यो॒ भ्रातृ॑व्य आप्नो॒ त्येको᳚न्नत॒ एको᳚न्नत आप्नोति॒ भ्रातृ॑व्यो॒ भ्रातृ॑व्य आप्नो॒ त्येको᳚न्नते । \newline
14. आ॒प्नो॒ त्येको᳚न्नत॒ एको᳚न्नत आप्नो त्याप्नो॒ त्येको᳚न्नते देव॒यज॑ने देव॒यज॑न॒ एको᳚न्नत आप्नो त्याप्नो॒ त्येको᳚न्नते देव॒यज॑ने । \newline
15. एको᳚न्नते देव॒यज॑ने देव॒यज॑न॒ एको᳚न्नत॒ एको᳚न्नते देव॒यज॑ने याजयेद् याजयेद् देव॒यज॑न॒ एको᳚न्नत॒ एको᳚न्नते देव॒यज॑ने याजयेत् । \newline
16. एको᳚न्नत॒ इत्येक॑ - उ॒न्न॒ते॒ । \newline
17. दे॒व॒यज॑ने याजयेद् याजयेद् देव॒यज॑ने देव॒यज॑ने याजयेत् प॒शुका॑मम् प॒शुका॑मं ॅयाजयेद् देव॒यज॑ने देव॒यज॑ने याजयेत् प॒शुका॑मम् । \newline
18. दे॒व॒यज॑न॒ इति॑ देव - यज॑ने । \newline
19. या॒ज॒ये॒त् प॒शुका॑मम् प॒शुका॑मं ॅयाजयेद् याजयेत् प॒शुका॑म॒ मेको᳚न्नता॒ देको᳚न्नतात् प॒शुका॑मं ॅयाजयेद् याजयेत् प॒शुका॑म॒ मेको᳚न्नतात् । \newline
20. प॒शुका॑म॒ मेको᳚न्नता॒ देको᳚न्नतात् प॒शुका॑मम् प॒शुका॑म॒ मेको᳚न्नता॒द् वै वा एको᳚न्नतात् प॒शुका॑मम् प॒शुका॑म॒ मेको᳚न्नता॒द् वै । \newline
21. प॒शुका॑म॒मिति॑ प॒शु - का॒म॒म् । \newline
22. एको᳚न्नता॒द् वै वा एको᳚न्नता॒ देको᳚न्नता॒द् वै दे॑व॒यज॑नाद् देव॒यज॑ना॒द् वा एको᳚न्नता॒ देको᳚न्नता॒द् वै दे॑व॒यज॑नात् । \newline
23. एको᳚न्नता॒दित्येक॑ - उ॒न्न॒ता॒त् । \newline
24. वै दे॑व॒यज॑नाद् देव॒यज॑ना॒द् वै वै दे॑व॒यज॑ना॒ दङ्गि॑र॒सो ऽङ्गि॑रसो देव॒यज॑ना॒द् वै वै दे॑व॒यज॑ना॒ दङ्गि॑रसः । \newline
25. दे॒व॒यज॑ना॒ दङ्गि॑र॒सो ऽङ्गि॑रसो देव॒यज॑नाद् देव॒यज॑ना॒ दङ्गि॑रसः प॒शून् प॒शू नङ्गि॑रसो देव॒यज॑नाद् देव॒यज॑ना॒ दङ्गि॑रसः प॒शून् । \newline
26. दे॒व॒यज॑ना॒दिति॑ देव - यज॑नात् । \newline
27. अङ्गि॑रसः प॒शून् प॒शू नङ्गि॑र॒सो ऽङ्गि॑रसः प॒शू न॑सृजन्ता सृजन्त प॒शू नङ्गि॑र॒सो ऽङ्गि॑रसः प॒शू न॑सृजन्त । \newline
28. प॒शू न॑सृजन्ता सृजन्त प॒शून् प॒शू न॑सृजन्ता न्त॒रा ऽन्त॒रा ऽसृ॑जन्त प॒शून् प॒शू न॑सृजन्ता न्त॒रा । \newline
29. अ॒सृ॒ज॒न्ता॒ न्त॒रा ऽन्त॒रा ऽसृ॑जन्ता सृजन्ता न्त॒रा स॑दोहविर्द्धा॒ने स॑दोहविर्द्धा॒ने अ॑न्त॒रा ऽसृ॑जन्ता सृजन्ता न्त॒रा स॑दोहविर्द्धा॒ने । \newline
30. अ॒न्त॒रा स॑दोहविर्द्धा॒ने स॑दोहविर्द्धा॒ने अ॑न्त॒रा ऽन्त॒रा स॑दोहविर्द्धा॒ने उ॑न्न॒त मु॑न्न॒तꣳ स॑दोहविर्द्धा॒ने अ॑न्त॒रा ऽन्त॒रा स॑दोहविर्द्धा॒ने उ॑न्न॒तम् । \newline
31. स॒दो॒ह॒वि॒र्द्धा॒ने उ॑न्न॒त मु॑न्न॒तꣳ स॑दोहविर्द्धा॒ने स॑दोहविर्द्धा॒ने उ॑न्न॒तꣳ स्या᳚थ् स्या दुन्न॒तꣳ स॑दोहविर्द्धा॒ने स॑दोहविर्द्धा॒ने उ॑न्न॒तꣳ स्या᳚त् । \newline
32. स॒दो॒ह॒वि॒र्द्धा॒ने इति॑ सदः - ह॒वि॒र्द्धा॒ने । \newline
33. उ॒न्न॒तꣳ स्या᳚थ् स्या दुन्न॒त मु॑न्न॒तꣳ स्या॑ दे॒त दे॒तथ् स्या॑ दुन्न॒त मु॑न्न॒तꣳ स्या॑ दे॒तत् । \newline
34. उ॒न्न॒तमित्यु॑त् - न॒तम् । \newline
35. स्या॒ दे॒त दे॒तथ् स्या᳚थ् स्या दे॒तद् वै वा ए॒तथ् स्या᳚थ् स्या दे॒तद् वै । \newline
36. ए॒तद् वै वा ए॒त दे॒तद् वा एको᳚न्नत॒ मेको᳚न्नतं॒ ॅवा ए॒त दे॒तद् वा एको᳚न्नतम् । \newline
37. वा एको᳚न्नत॒ मेको᳚न्नतं॒ ॅवै वा एको᳚न्नतम् देव॒यज॑नम् देव॒यज॑न॒ मेको᳚न्नतं॒ ॅवै वा एको᳚न्नतम् देव॒यज॑नम् । \newline
38. एको᳚न्नतम् देव॒यज॑नम् देव॒यज॑न॒ मेको᳚न्नत॒ मेको᳚न्नतम् देव॒यज॑नम् पशु॒मान् प॑शु॒मान् दे॑व॒यज॑न॒ मेको᳚न्नत॒ मेको᳚न्नतम् देव॒यज॑नम् पशु॒मान् । \newline
39. एको᳚न्नत॒मित्येक॑ - उ॒न्न॒त॒म् । \newline
40. दे॒व॒यज॑नम् पशु॒मान् प॑शु॒मान् दे॑व॒यज॑नम् देव॒यज॑नम् पशु॒मा ने॒वैव प॑शु॒मान् दे॑व॒यज॑नम् देव॒यज॑नम् पशु॒मा ने॒व । \newline
41. दे॒व॒यज॑न॒मिति॑ देव - यज॑नम् । \newline
42. प॒शु॒मा ने॒वैव प॑शु॒मान् प॑शु॒मा ने॒व भ॑वति भव त्ये॒व प॑शु॒मान् प॑शु॒मा ने॒व भ॑वति । \newline
43. प॒शु॒मानिति॑ पशु - मान् । \newline
44. ए॒व भ॑वति भव त्ये॒वैव भ॑वति॒ त्र्यु॑न्नते॒ त्र्यु॑न्नते भव त्ये॒वैव भ॑वति॒ त्र्यु॑न्नते । \newline
45. भ॒व॒ति॒ त्र्यु॑न्नते॒ त्र्यु॑न्नते भवति भवति॒ त्र्यु॑न्नते देव॒यज॑ने देव॒यज॑ने॒ त्र्यु॑न्नते भवति भवति॒ त्र्यु॑न्नते देव॒यज॑ने । \newline
46. त्र्यु॑न्नते देव॒यज॑ने देव॒यज॑ने॒ त्र्यु॑न्नते॒ त्र्यु॑न्नते देव॒यज॑ने याजयेद् याजयेद् देव॒यज॑ने॒ त्र्यु॑न्नते॒ त्र्यु॑न्नते देव॒यज॑ने याजयेत् । \newline
47. त्र्यु॑न्नत॒ इति॒ त्रि - उ॒न्न॒ते॒ । \newline
48. दे॒व॒यज॑ने याजयेद् याजयेद् देव॒यज॑ने देव॒यज॑ने याजयेथ् सुव॒र्गका॑मꣳ सुव॒र्गका॑मं ॅयाजयेद् देव॒यज॑ने देव॒यज॑ने याजयेथ् सुव॒र्गका॑मम् । \newline
49. दे॒व॒यज॑न॒ इति॑ देव - यज॑ने । \newline
50. या॒ज॒ये॒थ् सु॒व॒र्गका॑मꣳ सुव॒र्गका॑मं ॅयाजयेद् याजयेथ् सुव॒र्गका॑म॒म् त्र्यु॑न्नता॒त् त्र्यु॑न्नताथ् सुव॒र्गका॑मं ॅयाजयेद् याजयेथ् सुव॒र्गका॑म॒म् त्र्यु॑न्नतात् । \newline
51. सु॒व॒र्गका॑म॒म् त्र्यु॑न्नता॒त् त्र्यु॑न्नताथ् सुव॒र्गका॑मꣳ सुव॒र्गका॑म॒म् त्र्यु॑न्नता॒द् वै वै त्र्यु॑न्नताथ् सुव॒र्गका॑मꣳ सुव॒र्गका॑म॒म् त्र्यु॑न्नता॒द् वै । \newline
52. सु॒व॒र्गका॑म॒मिति॑ सुव॒र्ग - का॒म॒म् । \newline
53. त्र्यु॑न्नता॒द् वै वै त्र्यु॑न्नता॒त् त्र्यु॑न्नता॒द् वै दे॑व॒यज॑नाद् देव॒यज॑ना॒द् वै त्र्यु॑न्नता॒त् त्र्यु॑न्नता॒द् वै दे॑व॒यज॑नात् । \newline
54. त्र्यु॑न्नता॒दिति॒ त्रि - उ॒न्न॒ता॒त् । \newline
55. वै दे॑व॒यज॑नाद् देव॒यज॑ना॒द् वै वै दे॑व॒यज॑ना॒ दङ्गि॑र॒सो ऽङ्गि॑रसो देव॒यज॑ना॒द् वै वै दे॑व॒यज॑ना॒ दङ्गि॑रसः । \newline
56. दे॒व॒यज॑ना॒ दङ्गि॑र॒सो ऽङ्गि॑रसो देव॒यज॑नाद् देव॒यज॑ना॒ दङ्गि॑रसः सुव॒र्गꣳ सु॑व॒र्ग मङ्गि॑रसो देव॒यज॑नाद् देव॒यज॑ना॒ दङ्गि॑रसः सुव॒र्गम् । \newline
57. दे॒व॒यज॑ना॒दिति॑ देव - यज॑नात् । \newline
58. अङ्गि॑रसः सुव॒र्गꣳ सु॑व॒र्ग मङ्गि॑र॒सो ऽङ्गि॑रसः सुव॒र्गम् ॅलो॒कम् ॅलो॒कꣳ सु॑व॒र्ग मङ्गि॑र॒सो ऽङ्गि॑रसः सुव॒र्गम् ॅलो॒कम् । \newline
59. सु॒व॒र्गम् ॅलो॒कम् ॅलो॒कꣳ सु॑व॒र्गꣳ सु॑व॒र्गम् ॅलो॒क मा॑यन् नायन् ॅलो॒कꣳ सु॑व॒र्गꣳ सु॑व॒र्गम् ॅलो॒क मा॑यन्न् । \newline
60. सु॒व॒र्गमिति॑ सुवः - गम् । \newline
61. लो॒क मा॑यन् नायन् ॅलो॒कम् ॅलो॒क मा॑यन् नन्त॒रा ऽन्त॒रा ऽऽय॑न् ॅलो॒कम् ॅलो॒क मा॑यन् नन्त॒रा । \newline
62. आ॒य॒न् न॒न्त॒रा ऽन्त॒रा ऽऽय॑न् नायन् नन्त॒रा ऽऽह॑व॒नीय॑ माहव॒नीय॑ मन्त॒रा ऽऽय॑न् नायन् नन्त॒रा ऽऽह॑व॒नीय᳚म् । \newline
63. अ॒न्त॒रा ऽऽह॑व॒नीय॑ माहव॒नीय॑ मन्त॒रा ऽन्त॒रा ऽऽह॑व॒नीय॑म् च चाहव॒नीय॑ मन्त॒रा ऽन्त॒रा ऽऽह॑व॒नीय॑म् च । \newline
64. आ॒ह॒व॒नीय॑म् च चाहव॒नीय॑ माहव॒नीय॑म् च हवि॒र्द्धानꣳ॑ हवि॒र्द्धान॑म् चाहव॒नीय॑ माहव॒नीय॑म् च हवि॒र्द्धान᳚म् । \newline
65. आ॒ह॒व॒नीय॒मित्या᳚ - ह॒व॒नीय᳚म् । \newline
66. च॒ ह॒वि॒र्द्धानꣳ॑ हवि॒र्द्धान॑म् च च हवि॒र्द्धान॑म् च च हवि॒र्द्धान॑म् च च हवि॒र्द्धान॑म् च । \newline
67. ह॒वि॒र्द्धान॑म् च च हवि॒र्द्धानꣳ॑ हवि॒र्द्धान॑म् चोन्न॒त मु॑न्न॒तम् च॑ हवि॒र्द्धानꣳ॑ हवि॒र्द्धान॑म् चोन्न॒तम् । \newline
68. ह॒वि॒र्द्धान॒मिति॑ हविः - धान᳚म् । \newline
69. चो॒न्न॒त मु॑न्न॒तम् च॑ चोन्न॒तꣳ स्या᳚थ् स्या दुन्न॒तम् च॑ चोन्न॒तꣳ स्या᳚त् । \newline
\pagebreak
\markright{ TS 6.2.6.3  \hfill https://www.vedavms.in \hfill}

\section{ TS 6.2.6.3 }

\textbf{TS 6.2.6.3 } \newline
\textbf{Samhita Paata} \newline

-न्न॒तꣳ स्या॑दन्त॒रा ह॑वि॒र्द्धानं॑ च॒ सद॑श्चान्त॒रा सद॑श्च॒ गार्.ह॑पत्यं चै॒तद्वै त्र्यु॑न्नतं देव॒यज॑नꣳ सुव॒र्गमे॒व लो॒कमे॑ति॒ प्रति॑ष्ठिते देव॒यज॑ने याजयेत् प्रति॒ष्ठाका॑ममे॒तद्वै प्रति॑ष्ठितं देव॒यज॑नं॒ ॅयथ् स॒र्वतः॑ स॒मं प्रत्ये॒व ति॑ष्ठति॒ यत्रा॒न्या अ॑न्या॒ ओष॑धयो॒ व्यति॑षक्ताः॒ स्युस्तद्-या॑जयेत् प॒शुका॑ममे॒तद्वै प॑शू॒नाꣳ रू॒पꣳ रू॒पेणै॒वास्मै॑ प॒शू- [  ] \newline

\textbf{Pada Paata} \newline

उ॒न्न॒तमित्यु॑त् - न॒तम् । स्या॒त् । अ॒न्त॒रा । ह॒वि॒द्‌र्धान॒मिति॑ हविः - धान᳚म् । च॒ । सदः॑ । च॒ । अ॒न्त॒रा । सदः॑ । च॒ । गार्.ह॑पत्य॒मिति॒ गार्.ह॑ - प॒त्य॒म् । च॒ । ए॒तत् । वै । त्र्यु॑न्नत॒मिति॒ त्रि - उ॒न्न॒त॒म् । दे॒व॒यज॑न॒मिति॑ देव - यज॑नम् । सु॒व॒र्गमिति॑ सुवः - गम् । ए॒व । लो॒कम् । ए॒ति॒ । प्रति॑ष्ठित॒ इति॒ प्रति॑ - स्थि॒ते॒ । दे॒व॒यज॑न॒ इति॑ देव - यज॑ने । या॒ज॒ये॒त् । प्र॒ति॒ष्ठाका॑म॒मिति॑ प्रति॒ष्ठा - का॒म॒म् । ए॒तत् । वै । प्रति॑ष्ठित॒मिति॒ प्रति॑ - स्थि॒तम्॒ । दे॒व॒यज॑न॒मिति॑ देव - यज॑नम् । यत् । स॒र्वतः॑ । स॒मम् । प्रतीति॑ । ए॒व । ति॒ष्ठ॒ति॒ । यत्र॑ । अ॒न्या‌अ॑न्या॒ इत्य॒न्याः - अ॒न्याः॒ । ओष॑धयः । व्यति॑षक्ता॒ इति॑ वि - अति॑षक्ताः । स्युः । तत् । या॒ज॒ये॒त् । प॒शुका॑म॒मिति॑ प॒शु - का॒म॒म् । ए॒तत् । वै । प॒शू॒नाम् । रू॒पम् । रू॒पेण॑ । ए॒व । अ॒स्मै॒ । प॒शून् ।  \newline


\textbf{Krama Paata} \newline

उ॒न्न॒तꣳ स्या᳚त् । उ॒न्न॒तमित्यु॑त् - न॒तम् । स्या॒द॒न्त॒रा । अ॒न्त॒रा ह॑वि॒र्द्धान᳚म् । ह॒वि॒र्द्धान॑म् च । ह॒वि॒र्द्धान॒मिति॑ हविः - धान᳚म् । च॒ सदः॑ । सद॑श्च । चा॒न्त॒रा । अ॒न्त॒रा सदः॑ । सद॑श्च । च॒ गार्.ह॑पत्यम् । गार्.ह॑पत्यम् च । गार्.ह॑पत्य॒मिति॒ गार्.ह॑ - प॒त्य॒म् । चै॒तत् । ए॒तद् वै । वै त्र्यु॑न्नतम् । त्र्यु॑न्नतम् देव॒यज॑नम् । त्र्यु॑न्नत॒मिति॒ त्रि - उ॒न्न॒त॒म् । दे॒व॒यज॑नꣳ सुव॒र्गम् । दे॒व॒यज॑न॒मिति॑ देव - यज॑नम् । सु॒व॒र्गमे॒व । सु॒व॒र्गमिति॑ सुवः - गम् । ए॒व लो॒कम् । लो॒कमे॑ति । ए॒ति॒ प्रति॑ष्ठिते । प्रति॑ष्ठिते देव॒यज॑ने । प्रति॑ष्ठित॒ इति॒ प्रति॑ - स्थि॒ते॒ । दे॒व॒यज॑ने याजयेत् । दे॒व॒यज॑न॒ इति॑ देव - यज॑ने । या॒ज॒ये॒त् प्र॒ति॒ष्ठाका॑मम् । प्र॒ति॒ष्ठाका॑ममे॒तत् । प्र॒ति॒ष्ठाका॑म॒मिति॑ प्रति॒ष्ठा - का॒म॒म् । ए॒तद् वै । वै प्रति॑ष्ठितम् । प्रति॑ष्ठितम् देव॒यज॑नम् । प्रति॑ष्ठित॒मिति॒ प्रति॑ - स्थि॒त॒म् । दे॒व॒यज॑न॒म् ॅयत् । दे॒व॒यज॑न॒मिति॑ देव - यज॑नम् । यथ् स॒र्वतः॑ । स॒र्वतः॑ स॒मम् । स॒मम् प्रति॑ । प्रत्ये॒व । ए॒व ति॑ष्ठति । ति॒ष्ठ॒ति॒ यत्र॑ । यत्रा॒न्याअ॑न्याः । अ॒न्याअ॑न्या॒ ओष॑धयः । अ॒न्याअ॑न्या॒ इत्य॒न्याः - अ॒न्याः॒ । ओष॑धयो॒ व्यति॑षक्ताः । व्यति॑षक्ताः॒ स्युः । व्यति॑षक्ता॒ इति॑ वि - अति॑षक्ताः । स्युस्तत् । तद् या॑जयेत् । या॒ज॒ये॒त् प॒शुका॑मम् । प॒शुका॑ममे॒तत् । प॒शुका॑म॒मिति॑ प॒शु - का॒म॒म् । ए॒तद् वै । वै प॑शू॒नाम् । प॒शू॒नाꣳ रू॒पम् । रू॒पꣳ रू॒पेण॑ । रू॒पेणै॒व । ए॒वास्मै᳚ । अ॒स्मै॒ प॒शून् । प॒शूनव॑ \newline

\textbf{Jatai Paata} \newline

1. उ॒न्न॒तꣳ स्या᳚थ् स्या दुन्न॒त मु॑न्न॒तꣳ स्या᳚त् । \newline
2. उ॒न्न॒तमित्यु॑त् - न॒तम् । \newline
3. स्या॒ द॒न्त॒रा ऽन्त॒रा स्या᳚थ् स्या दन्त॒रा । \newline
4. अ॒न्त॒रा ह॑वि॒र्द्धानꣳ॑ हवि॒र्द्धान॑ मन्त॒रा ऽन्त॒रा ह॑वि॒र्द्धान᳚म् । \newline
5. ह॒वि॒र्द्धान॑म् च च हवि॒र्द्धानꣳ॑ हवि॒र्द्धान॑म् च । \newline
6. ह॒वि॒र्द्धान॒मिति॑ हविः - धान᳚म् । \newline
7. च॒ सदः॒ सद॑श्च च॒ सदः॑ । \newline
8. सद॑श्च च॒ सदः॒ सद॑श्च । \newline
9. चा॒न्त॒रा ऽन्त॒रा च॑ चान्त॒रा । \newline
10. अ॒न्त॒रा सदः॒ सदो᳚ ऽन्त॒रा ऽन्त॒रा सदः॑ । \newline
11. सद॑श्च च॒ सदः॒ सद॑श्च । \newline
12. च॒ गार्.ह॑पत्य॒म् गार्.ह॑पत्यम् च च॒ गार्.ह॑पत्यम् । \newline
13. गार्.ह॑पत्यम् च च॒ गार्.ह॑पत्य॒म् गार्.ह॑पत्यम् च । \newline
14. गार्.ह॑पत्य॒मिति॒ गार्.ह॑ - प॒त्य॒म् । \newline
15. चै॒त दे॒तच् च॑ चै॒तत् । \newline
16. ए॒तद् वै वा ए॒त दे॒तद् वै । \newline
17. वै त्र्यु॑न्नत॒म् त्र्यु॑न्नतं॒ ॅवै वै त्र्यु॑न्नतम् । \newline
18. त्र्यु॑न्नतम् देव॒यज॑नम् देव॒यज॑न॒म् त्र्यु॑न्नत॒म् त्र्यु॑न्नतम् देव॒यज॑नम् । \newline
19. त्र्यु॑न्नत॒मिति॒ त्रि - उ॒न्न॒त॒म् । \newline
20. दे॒व॒यज॑नꣳ सुव॒र्गꣳ सु॑व॒र्गम् दे॑व॒यज॑नम् देव॒यज॑नꣳ सुव॒र्गम् । \newline
21. दे॒व॒यज॑न॒मिति॑ देव - यज॑नम् । \newline
22. सु॒व॒र्ग मे॒वैव सु॑व॒र्गꣳ सु॑व॒र्ग मे॒व । \newline
23. सु॒व॒र्गमिति॑ सुवः - गम् । \newline
24. ए॒व लो॒कम् ॅलो॒क मे॒वैव लो॒कम् । \newline
25. लो॒क मे᳚त्येति लो॒कम् ॅलो॒क मे॑ति । \newline
26. ए॒ति॒ प्रति॑ष्ठिते॒ प्रति॑ष्ठित एत्येति॒ प्रति॑ष्ठिते । \newline
27. प्रति॑ष्ठिते देव॒यज॑ने देव॒यज॑ने॒ प्रति॑ष्ठिते॒ प्रति॑ष्ठिते देव॒यज॑ने । \newline
28. प्रति॑ष्ठित॒ इति॒ प्रति॑ - स्थि॒ते॒ । \newline
29. दे॒व॒यज॑ने याजयेद् याजयेद् देव॒यज॑ने देव॒यज॑ने याजयेत् । \newline
30. दे॒व॒यज॑न॒ इति॑ देव - यज॑ने । \newline
31. या॒ज॒ये॒त् प्र॒ति॒ष्ठाका॑मम् प्रति॒ष्ठाका॑मं ॅयाजयेद् याजयेत् प्रति॒ष्ठाका॑मम् । \newline
32. प्र॒ति॒ष्ठाका॑म मे॒त दे॒तत् प्र॑ति॒ष्ठाका॑मम् प्रति॒ष्ठाका॑म मे॒तत् । \newline
33. प्र॒ति॒ष्ठाका॑म॒मिति॑ प्रति॒ष्ठा - का॒म॒म् । \newline
34. ए॒तद् वै वा ए॒त दे॒तद् वै । \newline
35. वै प्रति॑ष्ठित॒म् प्रति॑ष्ठितं॒ ॅवै वै प्रति॑ष्ठितम् । \newline
36. प्रति॑ष्ठितम् देव॒यज॑नम् देव॒यज॑न॒म् प्रति॑ष्ठित॒म् प्रति॑ष्ठितम् देव॒यज॑नम् । \newline
37. प्रति॑ष्ठित॒मिति॒ प्रति॑ - स्थि॒त॒म् । \newline
38. दे॒व॒यज॑नं॒ ॅयद् यद् दे॑व॒यज॑नम् देव॒यज॑नं॒ ॅयत् । \newline
39. दे॒व॒यज॑न॒मिति॑ देव - यज॑नम् । \newline
40. यथ् स॒र्वतः॑ स॒र्वतो॒ यद् यथ् स॒र्वतः॑ । \newline
41. स॒र्वतः॑ स॒मꣳ स॒मꣳ स॒र्वतः॑ स॒र्वतः॑ स॒मम् । \newline
42. स॒मम् प्रति॒ प्रति॑ स॒मꣳ स॒मम् प्रति॑ । \newline
43. प्रत्ये॒वैव प्रति॒ प्रत्ये॒व । \newline
44. ए॒व ति॑ष्ठति तिष्ठ त्ये॒वैव ति॑ष्ठति । \newline
45. ति॒ष्ठ॒ति॒ यत्र॒ यत्र॑ तिष्ठति तिष्ठति॒ यत्र॑ । \newline
46. यत्रा॒ न्या‌अ॑न्या अ॒न्या‌अ॑न्या॒ यत्र॒ यत्रा॒ न्या‌अ॑न्याः । \newline
47. अ॒न्या‌अ॑न्या॒ ओष॑धय॒ ओष॑धयो॒ ऽन्या‌अ॑न्या अ॒न्या‌अ॑न्या॒ ओष॑धयः । \newline
48. अ॒न्या‌अ॑न्या॒ इत्य॒न्याः - अ॒न्याः॒ । \newline
49. ओष॑धयो॒ व्यति॑षक्ता॒ व्यति॑षक्ता॒ ओष॑धय॒ ओष॑धयो॒ व्यति॑षक्ताः । \newline
50. व्यति॑षक्ताः॒ स्युः स्युर् व्यति॑षक्ता॒ व्यति॑षक्ताः॒ स्युः । \newline
51. व्यति॑षक्ता॒ इति॑ वि - अति॑षक्ताः । \newline
52. स्यु स्तत् तथ् स्युः स्यु स्तत् । \newline
53. तद् या॑जयेद् याजये॒त् तत् तद् या॑जयेत् । \newline
54. या॒ज॒ये॒त् प॒शुका॑मम् प॒शुका॑मं ॅयाजयेद् याजयेत् प॒शुका॑मम् । \newline
55. प॒शुका॑म मे॒त दे॒तत् प॒शुका॑मम् प॒शुका॑म मे॒तत् । \newline
56. प॒शुका॑म॒मिति॑ प॒शु - का॒म॒म् । \newline
57. ए॒तद् वै वा ए॒त दे॒तद् वै । \newline
58. वै प॑शू॒नाम् प॑शू॒नां ॅवै वै प॑शू॒नाम् । \newline
59. प॒शू॒नाꣳ रू॒पꣳ रू॒पम् प॑शू॒नाम् प॑शू॒नाꣳ रू॒पम् । \newline
60. रू॒पꣳ रू॒पेण॑ रू॒पेण॑ रू॒पꣳ रू॒पꣳ रू॒पेण॑ । \newline
61. रू॒पे णै॒वैव रू॒पेण॑ रू॒पेणै॒व । \newline
62. ए॒वास्मा॑ अस्मा ए॒वैवास्मै᳚ । \newline
63. अ॒स्मै॒ प॒शून् प॒शू न॑स्मा अस्मै प॒शून् । \newline
64. प॒शू नवाव॑ प॒शून् प॒शू नव॑ । \newline

\textbf{Ghana Paata } \newline

1. उ॒न्न॒तꣳ स्या᳚थ् स्या दुन्न॒त मु॑न्न॒तꣳ स्या॑ दन्त॒रा ऽन्त॒रा स्या॑ दुन्न॒त मु॑न्न॒तꣳ स्या॑ दन्त॒रा । \newline
2. उ॒न्न॒तमित्यु॑त् - न॒तम् । \newline
3. स्या॒ द॒न्त॒रा ऽन्त॒रा स्या᳚थ् स्या दन्त॒रा ह॑वि॒र्द्धानꣳ॑ हवि॒र्द्धान॑ मन्त॒रा स्या᳚थ् स्या दन्त॒रा ह॑वि॒र्द्धान᳚म् । \newline
4. अ॒न्त॒रा ह॑वि॒र्द्धानꣳ॑ हवि॒र्द्धान॑ मन्त॒रा ऽन्त॒रा ह॑वि॒र्द्धान॑म् च च हवि॒र्द्धान॑ मन्त॒रा ऽन्त॒रा ह॑वि॒र्द्धान॑म् च । \newline
5. ह॒वि॒र्द्धान॑म् च च हवि॒र्द्धानꣳ॑ हवि॒र्द्धान॑म् च॒ सदः॒ सद॑श्च हवि॒र्द्धानꣳ॑ हवि॒र्द्धान॑म् च॒ सदः॑ । \newline
6. ह॒वि॒र्द्धान॒मिति॑ हविः - धान᳚म् । \newline
7. च॒ सदः॒ सद॑श्च च॒ सद॑श्च च॒ सद॑श्च च॒ सद॑श्च । \newline
8. सद॑श्च च॒ सदः॒ सद॑ श्चान्त॒रा ऽन्त॒रा च॒ सदः॒ सद॑ श्चान्त॒रा । \newline
9. चा॒न्त॒रा ऽन्त॒रा च॑ चान्त॒रा सदः॒ सदो᳚ ऽन्त॒रा च॑ चान्त॒रा सदः॑ । \newline
10. अ॒न्त॒रा सदः॒ सदो᳚ ऽन्त॒रा ऽन्त॒रा सद॑श्च च॒ सदो᳚ ऽन्त॒रा ऽन्त॒रा सद॑श्च । \newline
11. सद॑श्च च॒ सदः॒ सद॑श्च॒ गार्.ह॑पत्य॒म् गार्.ह॑पत्यम् च॒ सदः॒ सद॑श्च॒ गार्.ह॑पत्यम् । \newline
12. च॒ गार्.ह॑पत्य॒म् गार्.ह॑पत्यम् च च॒ गार्.ह॑पत्यम् च च॒ गार्.ह॑पत्यम् च च॒ गार्.ह॑पत्यम् च । \newline
13. गार्.ह॑पत्यम् च च॒ गार्.ह॑पत्य॒म् गार्.ह॑पत्यम् चै॒त दे॒तच् च॒ गार्.ह॑पत्य॒म् गार्.ह॑पत्यम् चै॒तत् । \newline
14. गार्.ह॑पत्य॒मिति॒ गार्.ह॑ - प॒त्य॒म् । \newline
15. चै॒त दे॒तच् च॑ चै॒तद् वै वा ए॒तच् च॑ चै॒तद् वै । \newline
16. ए॒तद् वै वा ए॒त दे॒तद् वै त्र्यु॑न्नत॒म् त्र्यु॑न्नतं॒ ॅवा ए॒त दे॒तद् वै त्र्यु॑न्नतम् । \newline
17. वै त्र्यु॑न्नत॒म् त्र्यु॑न्नतं॒ ॅवै वै त्र्यु॑न्नतम् देव॒यज॑नम् देव॒यज॑न॒म् त्र्यु॑न्नतं॒ ॅवै वै त्र्यु॑न्नतम् देव॒यज॑नम् । \newline
18. त्र्यु॑न्नतम् देव॒यज॑नम् देव॒यज॑न॒म् त्र्यु॑न्नत॒म् त्र्यु॑न्नतम् देव॒यज॑नꣳ सुव॒र्गꣳ सु॑व॒र्गम् दे॑व॒यज॑न॒म् त्र्यु॑न्नत॒म् त्र्यु॑न्नतम् देव॒यज॑नꣳ सुव॒र्गम् । \newline
19. त्र्यु॑न्नत॒मिति॒ त्रि - उ॒न्न॒त॒म् । \newline
20. दे॒व॒यज॑नꣳ सुव॒र्गꣳ सु॑व॒र्गम् दे॑व॒यज॑नम् देव॒यज॑नꣳ सुव॒र्ग मे॒वैव सु॑व॒र्गम् दे॑व॒यज॑नम् देव॒यज॑नꣳ सुव॒र्ग मे॒व । \newline
21. दे॒व॒यज॑न॒मिति॑ देव - यज॑नम् । \newline
22. सु॒व॒र्ग मे॒वैव सु॑व॒र्गꣳ सु॑व॒र्ग मे॒व लो॒कम् ॅलो॒क मे॒व सु॑व॒र्गꣳ सु॑व॒र्ग मे॒व लो॒कम् । \newline
23. सु॒व॒र्गमिति॑ सुवः - गम् । \newline
24. ए॒व लो॒कम् ॅलो॒क मे॒वैव लो॒क मे᳚त्येति लो॒क मे॒वैव लो॒क मे॑ति । \newline
25. लो॒क मे᳚त्येति लो॒कम् ॅलो॒क मे॑ति॒ प्रति॑ष्ठिते॒ प्रति॑ष्ठित एति लो॒कम् ॅलो॒क मे॑ति॒ प्रति॑ष्ठिते । \newline
26. ए॒ति॒ प्रति॑ष्ठिते॒ प्रति॑ष्ठित एत्येति॒ प्रति॑ष्ठिते देव॒यज॑ने देव॒यज॑ने॒ प्रति॑ष्ठित एत्येति॒ प्रति॑ष्ठिते देव॒यज॑ने । \newline
27. प्रति॑ष्ठिते देव॒यज॑ने देव॒यज॑ने॒ प्रति॑ष्ठिते॒ प्रति॑ष्ठिते देव॒यज॑ने याजयेद् याजयेद् देव॒यज॑ने॒ प्रति॑ष्ठिते॒ प्रति॑ष्ठिते देव॒यज॑ने याजयेत् । \newline
28. प्रति॑ष्ठित॒ इति॒ प्रति॑ - स्थि॒ते॒ । \newline
29. दे॒व॒यज॑ने याजयेद् याजयेद् देव॒यज॑ने देव॒यज॑ने याजयेत् प्रति॒ष्ठाका॑मम् प्रति॒ष्ठाका॑मं ॅयाजयेद् देव॒यज॑ने देव॒यज॑ने याजयेत् प्रति॒ष्ठाका॑मम् । \newline
30. दे॒व॒यज॑न॒ इति॑ देव - यज॑ने । \newline
31. या॒ज॒ये॒त् प्र॒ति॒ष्ठाका॑मम् प्रति॒ष्ठाका॑मं ॅयाजयेद् याजयेत् प्रति॒ष्ठाका॑म मे॒त दे॒तत् प्र॑ति॒ष्ठाका॑मं ॅयाजयेद् याजयेत् प्रति॒ष्ठाका॑म मे॒तत् । \newline
32. प्र॒ति॒ष्ठाका॑म मे॒त दे॒तत् प्र॑ति॒ष्ठाका॑मम् प्रति॒ष्ठाका॑म मे॒तद् वै वा ए॒तत् प्र॑ति॒ष्ठाका॑मम् प्रति॒ष्ठाका॑म मे॒तद् वै । \newline
33. प्र॒ति॒ष्ठाका॑म॒मिति॑ प्रति॒ष्ठा - का॒म॒म् । \newline
34. ए॒तद् वै वा ए॒त दे॒तद् वै प्रति॑ष्ठित॒म् प्रति॑ष्ठितं॒ ॅवा ए॒त दे॒तद् वै प्रति॑ष्ठितम् । \newline
35. वै प्रति॑ष्ठित॒म् प्रति॑ष्ठितं॒ ॅवै वै प्रति॑ष्ठितम् देव॒यज॑नम् देव॒यज॑न॒म् प्रति॑ष्ठितं॒ ॅवै वै प्रति॑ष्ठितम् देव॒यज॑नम् । \newline
36. प्रति॑ष्ठितम् देव॒यज॑नम् देव॒यज॑न॒म् प्रति॑ष्ठित॒म् प्रति॑ष्ठितम् देव॒यज॑नं॒ ॅयद् यद् दे॑व॒यज॑न॒म् प्रति॑ष्ठित॒म् प्रति॑ष्ठितम् देव॒यज॑नं॒ ॅयत् । \newline
37. प्रति॑ष्ठित॒मिति॒ प्रति॑ - स्थि॒त॒म् । \newline
38. दे॒व॒यज॑नं॒ ॅयद् यद् दे॑व॒यज॑नम् देव॒यज॑नं॒ ॅयथ् स॒र्वतः॑ स॒र्वतो॒ यद् दे॑व॒यज॑नम् देव॒यज॑नं॒ ॅयथ् स॒र्वतः॑ । \newline
39. दे॒व॒यज॑न॒मिति॑ देव - यज॑नम् । \newline
40. यथ् स॒र्वतः॑ स॒र्वतो॒ यद् यथ् स॒र्वतः॑ स॒मꣳ स॒मꣳ स॒र्वतो॒ यद् यथ् स॒र्वतः॑ स॒मम् । \newline
41. स॒र्वतः॑ स॒मꣳ स॒मꣳ स॒र्वतः॑ स॒र्वतः॑ स॒मम् प्रति॒ प्रति॑ स॒मꣳ स॒र्वतः॑ स॒र्वतः॑ स॒मम् प्रति॑ । \newline
42. स॒मम् प्रति॒ प्रति॑ स॒मꣳ स॒मम् प्रत्ये॒ वैव प्रति॑ स॒मꣳ स॒मम् प्रत्ये॒व । \newline
43. प्रत्ये॒ वैव प्रति॒ प्रत्ये॒व ति॑ष्ठति तिष्ठ त्ये॒व प्रति॒ प्रत्ये॒व ति॑ष्ठति । \newline
44. ए॒व ति॑ष्ठति तिष्ठ त्ये॒वैव ति॑ष्ठति॒ यत्र॒ यत्र॑ तिष्ठ त्ये॒वैव ति॑ष्ठति॒ यत्र॑ । \newline
45. ति॒ष्ठ॒ति॒ यत्र॒ यत्र॑ तिष्ठति तिष्ठति॒ यत्रा॒ न्या‌अ॑न्या अ॒न्या‌अ॑न्या॒ यत्र॑ तिष्ठति तिष्ठति॒ यत्रा॒ न्या‌अ॑न्याः । \newline
46. यत्रा॒ न्या‌अ॑न्या अ॒न्या‌अ॑न्या॒ यत्र॒ यत्रा॒ न्या‌अ॑न्या॒ ओष॑धय॒ ओष॑धयो॒ ऽन्या‌अ॑न्या॒ यत्र॒ यत्रा॒ न्या‌अ॑न्या॒ ओष॑धयः । \newline
47. अ॒न्या‌अ॑न्या॒ ओष॑धय॒ ओष॑धयो॒ ऽन्या‌अ॑न्या अ॒न्या‌अ॑न्या॒ ओष॑धयो॒ व्यति॑षक्ता॒ व्यति॑षक्ता॒ ओष॑धयो॒ ऽन्या‌अ॑न्या अ॒न्या‌अ॑न्या॒ ओष॑धयो॒ व्यति॑षक्ताः । \newline
48. अ॒न्या‌अ॑न्या॒ इत्य॒न्याः - अ॒न्याः॒ । \newline
49. ओष॑धयो॒ व्यति॑षक्ता॒ व्यति॑षक्ता॒ ओष॑धय॒ ओष॑धयो॒ व्यति॑षक्ताः॒ स्युः स्युर् व्यति॑षक्ता॒ ओष॑धय॒ ओष॑धयो॒ व्यति॑षक्ताः॒ स्युः । \newline
50. व्यति॑षक्ताः॒ स्युः स्युर् व्यति॑षक्ता॒ व्यति॑षक्ताः॒ स्यु स्तत् तथ् स्युर् व्यति॑षक्ता॒ व्यति॑षक्ताः॒ स्यु स्तत् । \newline
51. व्यति॑षक्ता॒ इति॑ वि - अति॑षक्ताः । \newline
52. स्यु स्तत् तथ् स्युः स्यु स्तद् या॑जयेद् याजये॒त् तथ् स्युः स्यु स्तद् या॑जयेत् । \newline
53. तद् या॑जयेद् याजये॒त् तत् तद् या॑जयेत् प॒शुका॑मम् प॒शुका॑मं ॅयाजये॒त् तत् तद् या॑जयेत् प॒शुका॑मम् । \newline
54. या॒ज॒ये॒त् प॒शुका॑मम् प॒शुका॑मं ॅयाजयेद् याजयेत् प॒शुका॑म मे॒त दे॒तत् प॒शुका॑मं ॅयाजयेद् याजयेत् प॒शुका॑म मे॒तत् । \newline
55. प॒शुका॑म मे॒त दे॒तत् प॒शुका॑मम् प॒शुका॑म मे॒तद् वै वा ए॒तत् प॒शुका॑मम् प॒शुका॑म मे॒तद् वै । \newline
56. प॒शुका॑म॒मिति॑ प॒शु - का॒म॒म् । \newline
57. ए॒तद् वै वा ए॒त दे॒तद् वै प॑शू॒नाम् प॑शू॒नां ॅवा ए॒त दे॒तद् वै प॑शू॒नाम् । \newline
58. वै प॑शू॒नाम् प॑शू॒नां ॅवै वै प॑शू॒नाꣳ रू॒पꣳ रू॒पम् प॑शू॒नां ॅवै वै प॑शू॒नाꣳ रू॒पम् । \newline
59. प॒शू॒नाꣳ रू॒पꣳ रू॒पम् प॑शू॒नाम् प॑शू॒नाꣳ रू॒पꣳ रू॒पेण॑ रू॒पेण॑ रू॒पम् प॑शू॒नाम् प॑शू॒नाꣳ रू॒पꣳ रू॒पेण॑ । \newline
60. रू॒पꣳ रू॒पेण॑ रू॒पेण॑ रू॒पꣳ रू॒पꣳ रू॒पे णै॒वैव रू॒पेण॑ रू॒पꣳ रू॒पꣳ रू॒पेणै॒व । \newline
61. रू॒पे णै॒वैव रू॒पेण॑ रू॒पे णै॒वास्मा॑ अस्मा ए॒व रू॒पेण॑ रू॒पे णै॒वास्मै᳚ । \newline
62. ए॒वास्मा॑ अस्मा ए॒वै वास्मै॑ प॒शून् प॒शू न॑स्मा ए॒वै वास्मै॑ प॒शून् । \newline
63. अ॒स्मै॒ प॒शून् प॒शू न॑स्मा अस्मै प॒शू नवाव॑ प॒शू न॑स्मा अस्मै प॒शू नव॑ । \newline
64. प॒शू नवाव॑ प॒शून् प॒शू नव॑ रुन्धे रु॒न्धे ऽव॑ प॒शून् प॒शू नव॑ रुन्धे । \newline
\pagebreak
\markright{ TS 6.2.6.4  \hfill https://www.vedavms.in \hfill}

\section{ TS 6.2.6.4 }

\textbf{TS 6.2.6.4 } \newline
\textbf{Samhita Paata} \newline

नव॑ रुन्धे पशु॒माने॒व भ॑वति॒ निर्.ऋ॑तिगृहीते देव॒यज॑ने याजये॒द्यं का॒मये॑त॒ निर्.ऋ॑त्यास्य य॒ज्ञ्ं ग्रा॑हयेय॒मित्ये॒तद्वै निर्.ऋ॑तिगृहीतं देव॒यज॑नं॒ ॅयथ् स॒दृश्यै॑ स॒त्या॑ ऋ॒क्षं निर्.ऋ॑त्यै॒वास्य॑ य॒ज्ञ्ं ग्रा॑हयति॒ व्यावृ॑त्ते देव॒यज॑ने याजयेद् व्या॒वृत्का॑मं॒ ॅयं पात्रे॑ वा॒ तल्पे॑ वा॒ मीमाꣳ॑सेरन् प्रा॒चीन॑माहव॒नीया᳚त् प्रव॒णꣳ स्या᳚त् प्रती॒चीनं॒ गार्.ह॑पत्यादे॒तद्वै व्यावृ॑त्तं देव॒यज॑नं॒ ॅवि पा॒प्मना॒ ( ) भ्रातृ॑व्ये॒णा- ऽऽ*व॑र्तते॒ नैनं॒ पात्रे॒ न तल्पे॑ मीमाꣳ सन्ते का॒र्ये॑ देव॒यज॑ने याजये॒द्-भूति॑कामं का॒या॑ वै पुरु॑षो॒ भव॑त्ये॒व ॥ \newline

\textbf{Pada Paata} \newline

अवेति॑ । रु॒न्धे॒ । प॒शु॒मानिति॑ पशु - मान् । ए॒व । भ॒व॒ति॒ । निर्.ऋ॑तिगृहीत॒ इति॒ निर्.ऋ॑ति - गृ॒ही॒ते॒ । दे॒व॒यज॑न॒ इति॑ देव-यज॑ने । या॒ज॒ये॒त् । यम् । का॒मये॑त । निर्.ऋ॒त्येति॒ निः - ऋ॒त्या॒ । अ॒स्य॒ । य॒ज्ञ्म् । ग्रा॒ह॒ये॒य॒म् । इति॑ । ए॒तत् । वै । निर्.ऋ॑तिगृहीत॒मिति॒ निर्.ऋ॑ति - गृ॒ही॒त॒म् । दे॒व॒यज॑न॒मिति॑ देव - यज॑नम् । यत् । स॒दृश्यै᳚ । स॒त्याः᳚ । ऋ॒क्षम् । निर्.ऋ॒त्येति॒ निः - ऋ॒त्या॒ । ए॒व । अ॒स्य॒ । य॒ज्ञ्म् । ग्रा॒ह॒य॒ति॒ । व्यावृ॑त्त॒ इति॑ वि-आवृ॑त्ते । दे॒व॒यज॑न॒ इति॑ देव - यज॑ने । या॒ज॒ये॒त् । व्या॒वृत्का॑म॒मिति॑ व्या॒वृत्-का॒म॒म् । यम् । पात्रे᳚ । वा॒ । तल्पे᳚ । वा॒ । मीमाꣳ॑सेरन्न् । प्रा॒चीन᳚म् । आ॒ह॒व॒नीया॒दित्या᳚ -ह॒व॒नीया᳚त् । प्र॒व॒णमिति॑ प्र - व॒नम् । स्या॒त् । प्र॒ती॒चीन᳚म् । गार्.ह॑पत्या॒दिति॒ गार्.ह॑ - प॒त्या॒त् । ए॒तत् । वै । व्यावृ॑त्त॒मिति॑ वि - आवृ॑त्तम् । दे॒व॒यज॑न॒मिति॑ देव - यज॑नम् । वीति॑ । पा॒प्मना᳚ ( ) । भ्रातृ॑व्येण । एति॑ । व॒र्त॒ते॒ । न । ए॒न॒म् । पात्रे᳚ । न । तल्पे᳚ । मी॒माꣳ॒॒स॒न्ते॒ । का॒र्ये᳚ । दे॒व॒यज॑न॒ इति॑ देव-यज॑ने । या॒ज॒ये॒त् । भूति॑काम॒मिति॒ भूति॑ - का॒म॒म् । का॒र्यः॑ । वै । पुरु॑षः । भव॑ति । ए॒व ॥  \newline


\textbf{Krama Paata} \newline

अव॑ रुन्धे । रु॒न्धे॒ प॒शु॒मान् । प॒शु॒माने॒व । प॒शु॒मानिति॑ पशु - मान् । ए॒व भ॑वति । भ॒व॒ति॒ निर्.ऋ॑तिगृहीते । निर्.ऋ॑तिगृहीते देव॒यज॑ने । निर्.ऋ॑तिगृहीत॒ इति॒ निर्.ऋ॑ति - गृ॒ही॒ते॒ । दे॒व॒यज॑ने याजयेत् । दे॒व॒यज॑न॒ इति॑ देव - यज॑ने । या॒ज॒ये॒द् यम् । यम् का॒मये॑त । का॒मये॑त॒ निर्.ऋ॑त्या । निर्.ऋ॑त्याऽस्य । निर्.ऋ॒त्येति॒ निः - ऋ॒त्या॒ । अ॒स्य॒ य॒ज्ञ्म् । य॒ज्ञ्म् ग्रा॑हयेयम् । ग्रा॒ह॒ये॒य॒मिति॑ । इत्ये॒तत् । ए॒तद् वै । वै निर्.ऋ॑तिगृहीतम् । निर्.ऋ॑तिगृहीतम् देव॒यज॑नम् । निर्.ऋ॑तिगृहीत॒मिति॒ निर्.ऋ॑ति - गृ॒ही॒त॒म् । दे॒व॒यज॑न॒म् ॅयत् । दे॒व॒यज॑न॒मिति॑ देव - यज॑नम् । यथ् स॒दृश्यै᳚ । स॒दृश्यै॑ स॒त्याः᳚ । स॒त्या॑ ऋ॒क्षम् । ऋ॒क्षम् निर्.ऋ॑त्या । निर्.ऋ॑त्यै॒व । निर्.ऋ॒त्येति॒ निः - ऋ॒त्या॒ । ए॒वास्य॑ । अ॒स्य॒ य॒ज्ञ्म् । य॒ज्ञ्म् ग्रा॑हयति । ग्रा॒ह॒य॒ति॒ व्यावृ॑त्ते । व्यावृ॑त्ते देव॒यज॑ने । व्यावृ॑त्त॒ इति॑ वि - आवृ॑त्ते । दे॒व॒यज॑ने याजयेत् । दे॒व॒यज॑न॒ इति॑ देव - यज॑ने । या॒ज॒ये॒द् व्या॒वृत्का॑मम् । व्या॒वृत्का॑म॒म् ॅयम् । व्या॒वृत्का॑म॒मिति॑ व्या॒वृत् - का॒म॒म् । यम् पात्रे᳚ । पात्रे॑ वा । वा॒ तल्पे᳚ । तल्पे॑ वा । वा॒ मीमाꣳ॑सेरन्न् । मीमाꣳ॑सेरन् प्रा॒चीन᳚म् । प्रा॒चीन॑माहव॒नीया᳚त् । आ॒ह॒व॒नीया᳚त् प्रव॒णम् । आ॒ह॒व॒नीया॒दित्या᳚ - ह॒व॒नीया᳚त् । प्र॒व॒णꣳस्या᳚त् । प्र॒व॒णमिति॑ प्र - व॒नम् । स्या॒त् प्र॒ती॒चीन᳚म् । प्र॒ती॒चीन॒म् गार्.ह॑पत्यात् । गार्.ह॑पत्यादे॒तत् । गार्.ह॑पत्या॒दिति॒ गार्.ह॑ - प॒त्या॒त् । ए॒तद् वै । वै व्यावृ॑त्तम् । व्यावृ॑त्तम् देव॒यज॑नम् । व्यावृ॑त्त॒मिति॑ वि - आवृ॑त्तम् । दे॒व॒यज॑न॒म् ॅवि । दे॒व॒यज॑न॒मिति॑ देव - यज॑नम् । वि पा॒प्मना᳚ ( ) । पा॒प्मना॒ भ्रातृ॑व्येण । भ्रातृ॑व्ये॒णा । आ व॑र्तते । व॒र्त॒ते॒ न । नैन᳚म् । ए॒न॒म् पात्रे᳚ । पात्रे॒ न । न तल्पे᳚ । तल्पे॑ मीमाꣳसन्ते । मी॒माꣳ॒॒स॒न्ते॒ का॒र्ये᳚ । का॒र्ये॑ देव॒यज॑ने । दे॒व॒यज॑ने याजयेत् । दे॒व॒यज॑न॒ इति॑ देव - यज॑ने । या॒ज॒ये॒द् भूति॑कामम् । भूति॑कामम् का॒र्यः॑ । भूति॑काम॒मिति॒ भूति॑ - का॒म॒म् । का॒र्यो॑ वै । वै पुरु॑षः । पुरु॑षो॒ भव॑ति । भव॑त्ये॒व । ए॒वेत्ये॒व । \newline

\textbf{Jatai Paata} \newline

1. अव॑ रुन्धे रु॒न्धे ऽवाव॑ रुन्धे । \newline
2. रु॒न्धे॒ प॒शु॒मान् प॑शु॒मान् रु॑न्धे रुन्धे पशु॒मान् । \newline
3. प॒शु॒मा ने॒वैव प॑शु॒मान् प॑शु॒मा ने॒व । \newline
4. प॒शु॒मानिति॑ पशु - मान् । \newline
5. ए॒व भ॑वति भव त्ये॒वैव भ॑वति । \newline
6. भ॒व॒ति॒ निर्.ऋ॑तिगृहीते॒ निर्.ऋ॑तिगृहीते भवति भवति॒ निर्.ऋ॑तिगृहीते । \newline
7. निर्.ऋ॑तिगृहीते देव॒यज॑ने देव॒यज॑ने॒ निर्.ऋ॑तिगृहीते॒ निर्.ऋ॑तिगृहीते देव॒यज॑ने । \newline
8. निर्.ऋ॑तिगृहीत॒ इति॒ निर्.ऋ॑ति - गृ॒ही॒ते॒ । \newline
9. दे॒व॒यज॑ने याजयेद् याजयेद् देव॒यज॑ने देव॒यज॑ने याजयेत् । \newline
10. दे॒व॒यज॑न॒ इति॑ देव - यज॑ने । \newline
11. या॒ज॒ये॒द् यं ॅयं ॅया॑जयेद् याजये॒द् यम् । \newline
12. यम् का॒मये॑त का॒मये॑त॒ यं ॅयम् का॒मये॑त । \newline
13. का॒मये॑त॒ निर्.ऋ॑त्या॒ निर्.ऋ॑त्या का॒मये॑त का॒मये॑त॒ निर्.ऋ॑त्या । \newline
14. निर्.ऋ॑त्या ऽस्यास्य॒ निर्.ऋ॑त्या॒ निर्.ऋ॑त्या ऽस्य । \newline
15. निर्.ऋ॒त्येति॒ निः - ऋ॒त्या॒ । \newline
16. अ॒स्य॒ य॒ज्ञ्ं ॅय॒ज्ञ् म॑स्यास्य य॒ज्ञ्म् । \newline
17. य॒ज्ञ्म् ग्रा॑हयेयम् ग्राहयेयं ॅय॒ज्ञ्ं ॅय॒ज्ञ्म् ग्रा॑हयेयम् । \newline
18. ग्रा॒ह॒ये॒य॒ मितीति॑ ग्राहयेयम् ग्राहयेय॒ मिति॑ । \newline
19. इत्ये॒त दे॒त दिती त्ये॒तत् । \newline
20. ए॒तद् वै वा ए॒त दे॒तद् वै । \newline
21. वै निर्.ऋ॑तिगृहीत॒म् निर्.ऋ॑तिगृहीतं॒ ॅवै वै निर्.ऋ॑तिगृहीतम् । \newline
22. निर्.ऋ॑तिगृहीतम् देव॒यज॑नम् देव॒यज॑न॒म् निर्.ऋ॑तिगृहीत॒म् निर्.ऋ॑तिगृहीतम् देव॒यज॑नम् । \newline
23. निर्.ऋ॑तिगृहीत॒मिति॒ निर्.ऋ॑ति - गृ॒ही॒त॒म् । \newline
24. दे॒व॒यज॑नं॒ ॅयद् यद् दे॑व॒यज॑नम् देव॒यज॑नं॒ ॅयत् । \newline
25. दे॒व॒यज॑न॒मिति॑ देव - यज॑नम् । \newline
26. यथ् स॒दृश्यै॑ स॒दृश्यै॒ यद् यथ् स॒दृश्यै᳚ । \newline
27. स॒दृश्यै॑ स॒त्याः᳚ स॒त्याः᳚ स॒दृश्यै॑ स॒दृश्यै॑ स॒त्याः᳚ । \newline
28. स॒त्या॑ ऋ॒क्ष मृ॒क्षꣳ स॒त्याः᳚ स॒त्या॑ ऋ॒क्षम् । \newline
29. ऋ॒क्षम् निर्.ऋ॑त्या॒ निर्.ऋ॑त्य॒ र्‌क्ष मृ॒क्षम् निर्.ऋ॑त्या । \newline
30. निर्.ऋ॑ त्यै॒वैव निर्.ऋ॑त्या॒ निर्.ऋ॑ त्यै॒व । \newline
31. निर्.ऋ॒त्येति॒ निः - ऋ॒त्या॒ । \newline
32. ए॒वास्या᳚ स्यै॒वै वास्य॑ । \newline
33. अ॒स्य॒ य॒ज्ञ्ं ॅय॒ज्ञ् म॑स्यास्य य॒ज्ञ्म् । \newline
34. य॒ज्ञ्म् ग्रा॑हयति ग्राहयति य॒ज्ञ्ं ॅय॒ज्ञ्म् ग्रा॑हयति । \newline
35. ग्रा॒ह॒य॒ति॒ व्यावृ॑त्ते॒ व्यावृ॑त्ते ग्राहयति ग्राहयति॒ व्यावृ॑त्ते । \newline
36. व्यावृ॑त्ते देव॒यज॑ने देव॒यज॑ने॒ व्यावृ॑त्ते॒ व्यावृ॑त्ते देव॒यज॑ने । \newline
37. व्यावृ॑त्त॒ इति॑ वि - आवृ॑त्ते । \newline
38. दे॒व॒यज॑ने याजयेद् याजयेद् देव॒यज॑ने देव॒यज॑ने याजयेत् । \newline
39. दे॒व॒यज॑न॒ इति॑ देव - यज॑ने । \newline
40. या॒ज॒ये॒द् व्या॒वृत्का॑मं ॅव्या॒वृत्का॑मं ॅयाजयेद् याजयेद् व्या॒वृत्का॑मम् । \newline
41. व्या॒वृत्का॑मं॒ ॅयं ॅयं ॅव्या॒वृत्का॑मं ॅव्या॒वृत्का॑मं॒ ॅयम् । \newline
42. व्या॒वृत्का॑म॒मिति॑ व्या॒वृत् - का॒म॒म् । \newline
43. यम् पात्रे॒ पात्रे॒ यं ॅयम् पात्रे᳚ । \newline
44. पात्रे॑ वा वा॒ पात्रे॒ पात्रे॑ वा । \newline
45. वा॒ तल्पे॒ तल्पे॑ वा वा॒ तल्पे᳚ । \newline
46. तल्पे॑ वा वा॒ तल्पे॒ तल्पे॑ वा । \newline
47. वा॒ मीमाꣳ॑सेर॒न् मीमाꣳ॑सेरन्. वा वा॒ मीमाꣳ॑सेरन्न् । \newline
48. मीमाꣳ॑सेरन् प्रा॒चीन॑म् प्रा॒चीन॒म् मीमाꣳ॑सेर॒न् मीमाꣳ॑सेरन् प्रा॒चीन᳚म् । \newline
49. प्रा॒चीन॑ माहव॒नीया॑ दाहव॒नीया᳚त् प्रा॒चीन॑म् प्रा॒चीन॑ माहव॒नीया᳚त् । \newline
50. आ॒ह॒व॒नीया᳚त् प्रव॒णम् प्र॑व॒ण मा॑हव॒नीया॑ दाहव॒नीया᳚त् प्रव॒णम् । \newline
51. आ॒ह॒व॒नीया॒दित्या᳚ - ह॒व॒नीया᳚त् । \newline
52. प्र॒व॒णꣳ स्या᳚थ् स्यात् प्रव॒णम् प्र॑व॒णꣳ स्या᳚त् । \newline
53. प्र॒व॒णमिति॑ प्र - व॒नम् । \newline
54. स्या॒त् प्र॒ती॒चीन॑म् प्रती॒चीनꣳ॑ स्याथ् स्यात् प्रती॒चीन᳚म् । \newline
55. प्र॒ती॒चीन॒म् गार्.ह॑पत्या॒द् गार्.ह॑पत्यात् प्रती॒चीन॑म् प्रती॒चीन॒म् गार्.ह॑पत्यात् । \newline
56. गार्.ह॑पत्या दे॒त दे॒तद् गार्.ह॑पत्या॒द् गार्.ह॑पत्या दे॒तत् । \newline
57. गार्.ह॑पत्या॒दिति॒ गार्.ह॑ - प॒त्या॒त् । \newline
58. ए॒तद् वै वा ए॒त दे॒तद् वै । \newline
59. वै व्यावृ॑त्तं॒ ॅव्यावृ॑त्तं॒ ॅवै वै व्यावृ॑त्तम् । \newline
60. व्यावृ॑त्तम् देव॒यज॑नम् देव॒यज॑नं॒ ॅव्यावृ॑त्तं॒ ॅव्यावृ॑त्तम् देव॒यज॑नम् । \newline
61. व्यावृ॑त्त॒मिति॑ वि - आवृ॑त्तम् । \newline
62. दे॒व॒यज॑नं॒ ॅवि वि दे॑व॒यज॑नम् देव॒यज॑नं॒ ॅवि । \newline
63. दे॒व॒यज॑न॒मिति॑ देव - यज॑नम् । \newline
64. वि पा॒प्मना॑ पा॒प्मना॒ वि वि पा॒प्मना᳚ । \newline
65. पा॒प्मना॒ भ्रातृ॑व्येण॒ भ्रातृ॑व्येण पा॒प्मना॑ पा॒प्मना॒ भ्रातृ॑व्येण । \newline
66. भ्रातृ॑व्ये॒णा भ्रातृ॑व्येण॒ भ्रातृ॑व्ये॒णा । \newline
67. आ व॑र्तते वर्तत॒ आ व॑र्तते । \newline
68. व॒र्त॒ते॒ न न व॑र्तते वर्तते॒ न । \newline
69. नैन॑ मेन॒न्न नैन᳚म् । \newline
70. ए॒न॒म् पात्रे॒ पात्र॑ एन मेन॒म् पात्रे᳚ । \newline
71. पात्रे॒ न न पात्रे॒ पात्रे॒ न । \newline
72. न तल्पे॒ तल्पे॒ न न तल्पे᳚ । \newline
73. तल्पे॑ मीमाꣳसन्ते मीमाꣳसन्ते॒ तल्पे॒ तल्पे॑ मीमाꣳसन्ते । \newline
74. मी॒माꣳ॒॒स॒न्ते॒ का॒र्ये॑ का॒र्ये॑ मीमाꣳसन्ते मीमाꣳसन्ते का॒र्ये᳚ । \newline
75. का॒र्ये॑ देव॒यज॑ने देव॒यज॑ने का॒र्ये॑ का॒र्ये॑ देव॒यज॑ने । \newline
76. दे॒व॒यज॑ने याजयेद् याजयेद् देव॒यज॑ने देव॒यज॑ने याजयेत् । \newline
77. दे॒व॒यज॑न॒ इति॑ देव - यज॑ने । \newline
78. या॒ज॒ये॒द् भूति॑काम॒म् भूति॑कामं ॅयाजयेद् याजये॒द् भूति॑कामम् । \newline
79. भूति॑कामम् का॒र्यः॑ का॒र्यो॑ भूति॑काम॒म् भूति॑कामम् का॒र्यः॑ । \newline
80. भूति॑काम॒मिति॒ भूति॑ - का॒म॒म् । \newline
81. का॒र्यो॑ वै वै का॒र्यः॑ का॒र्यो॑ वै । \newline
82. वै पुरु॑षः॒ पुरु॑षो॒ वै वै पुरु॑षः । \newline
83. पुरु॑षो॒ भव॑ति॒ भव॑ति॒ पुरु॑षः॒ पुरु॑षो॒ भव॑ति । \newline
84. भव॑ त्ये॒वैव भव॑ति॒ भव॑ त्ये॒व । \newline
85. ए॒वेत्ये॒व । \newline

\textbf{Ghana Paata } \newline

1. अव॑ रुन्धे रु॒न्धे ऽवाव॑ रुन्धे पशु॒मान् प॑शु॒मान् रु॒न्धे ऽवाव॑ रुन्धे पशु॒मान् । \newline
2. रु॒न्धे॒ प॒शु॒मान् प॑शु॒मान् रु॑न्धे रुन्धे पशु॒मा ने॒वैव प॑शु॒मान् रु॑न्धे रुन्धे पशु॒मा ने॒व । \newline
3. प॒शु॒मा ने॒वैव प॑शु॒मान् प॑शु॒मा ने॒व भ॑वति भव त्ये॒व प॑शु॒मान् प॑शु॒मा ने॒व भ॑वति । \newline
4. प॒शु॒मानिति॑ पशु - मान् । \newline
5. ए॒व भ॑वति भव त्ये॒वैव भ॑वति॒ निर्.ऋ॑तिगृहीते॒ निर्.ऋ॑तिगृहीते भव त्ये॒वैव भ॑वति॒ निर्.ऋ॑तिगृहीते । \newline
6. भ॒व॒ति॒ निर्.ऋ॑तिगृहीते॒ निर्.ऋ॑तिगृहीते भवति भवति॒ निर्.ऋ॑तिगृहीते देव॒यज॑ने देव॒यज॑ने॒ निर्.ऋ॑तिगृहीते भवति भवति॒ निर्.ऋ॑तिगृहीते देव॒यज॑ने । \newline
7. निर्.ऋ॑तिगृहीते देव॒यज॑ने देव॒यज॑ने॒ निर्.ऋ॑तिगृहीते॒ निर्.ऋ॑तिगृहीते देव॒यज॑ने याजयेद् याजयेद् देव॒यज॑ने॒ निर्.ऋ॑तिगृहीते॒ निर्.ऋ॑तिगृहीते देव॒यज॑ने याजयेत् । \newline
8. निर्.ऋ॑तिगृहीत॒ इति॒ निर्.ऋ॑ति - गृ॒ही॒ते॒ । \newline
9. दे॒व॒यज॑ने याजयेद् याजयेद् देव॒यज॑ने देव॒यज॑ने याजये॒द् यं ॅयं ॅया॑जयेद् देव॒यज॑ने देव॒यज॑ने याजये॒द् यम् । \newline
10. दे॒व॒यज॑न॒ इति॑ देव - यज॑ने । \newline
11. या॒ज॒ये॒द् यं ॅयं ॅया॑जयेद् याजये॒द् यम् का॒मये॑त का॒मये॑त॒ यं ॅया॑जयेद् याजये॒द् यम् का॒मये॑त । \newline
12. यम् का॒मये॑त का॒मये॑त॒ यं ॅयम् का॒मये॑त॒ निर्.ऋ॑त्या॒ निर्.ऋ॑त्या का॒मये॑त॒ यं ॅयम् का॒मये॑त॒ निर्.ऋ॑त्या । \newline
13. का॒मये॑त॒ निर्.ऋ॑त्या॒ निर्.ऋ॑त्या का॒मये॑त का॒मये॑त॒ निर्.ऋ॑त्या ऽस्यास्य॒ निर्.ऋ॑त्या का॒मये॑त का॒मये॑त॒ निर्.ऋ॑त्या ऽस्य । \newline
14. निर्.ऋ॑त्या ऽस्यास्य॒ निर्.ऋ॑त्या॒ निर्.ऋ॑त्या ऽस्य य॒ज्ञ्ं ॅय॒ज्ञ् म॑स्य॒ निर्.ऋ॑त्या॒ निर्.ऋ॑त्या ऽस्य य॒ज्ञ्म् । \newline
15. निर्.ऋ॒त्येति॒ निः - ऋ॒त्या॒ । \newline
16. अ॒स्य॒ य॒ज्ञ्ं ॅय॒ज्ञ् म॑स्यास्य य॒ज्ञ्म् ग्रा॑हयेयम् ग्राहयेयं ॅय॒ज्ञ् म॑स्यास्य य॒ज्ञ्म् ग्रा॑हयेयम् । \newline
17. य॒ज्ञ्म् ग्रा॑हयेयम् ग्राहयेयं ॅय॒ज्ञ्ं ॅय॒ज्ञ्म् ग्रा॑हयेय॒ मितीति॑ ग्राहयेयं ॅय॒ज्ञ्ं ॅय॒ज्ञ्म् ग्रा॑हयेय॒ मिति॑ । \newline
18. ग्रा॒ह॒ये॒य॒ मितीति॑ ग्राहयेयम् ग्राहयेय॒ मित्ये॒त दे॒त दिति॑ ग्राहयेयम् ग्राहयेय॒ मित्ये॒तत् । \newline
19. इत्ये॒त दे॒त दिती त्ये॒तद् वै वा ए॒त दिती त्ये॒तद् वै । \newline
20. ए॒तद् वै वा ए॒त दे॒तद् वै निर्.ऋ॑तिगृहीत॒म् निर्.ऋ॑तिगृहीतं॒ ॅवा ए॒त दे॒तद् वै निर्.ऋ॑तिगृहीतम् । \newline
21. वै निर्.ऋ॑तिगृहीत॒म् निर्.ऋ॑तिगृहीतं॒ ॅवै वै निर्.ऋ॑तिगृहीतम् देव॒यज॑नम् देव॒यज॑न॒म् निर्.ऋ॑तिगृहीतं॒ ॅवै वै निर्.ऋ॑तिगृहीतम् देव॒यज॑नम् । \newline
22. निर्.ऋ॑तिगृहीतम् देव॒यज॑नम् देव॒यज॑न॒म् निर्.ऋ॑तिगृहीत॒म् निर्.ऋ॑तिगृहीतम् देव॒यज॑नं॒ ॅयद् यद् दे॑व॒यज॑न॒म् निर्.ऋ॑तिगृहीत॒म् निर्.ऋ॑तिगृहीतम् देव॒यज॑नं॒ ॅयत् । \newline
23. निर्.ऋ॑तिगृहीत॒मिति॒ निर्.ऋ॑ति - गृ॒ही॒त॒म् । \newline
24. दे॒व॒यज॑नं॒ ॅयद् यद् दे॑व॒यज॑नम् देव॒यज॑नं॒ ॅयथ् स॒दृश्यै॑ स॒दृश्यै॒ यद् दे॑व॒यज॑नम् देव॒यज॑नं॒ ॅयथ् स॒दृश्यै᳚ । \newline
25. दे॒व॒यज॑न॒मिति॑ देव - यज॑नम् । \newline
26. यथ् स॒दृश्यै॑ स॒दृश्यै॒ यद् यथ् स॒दृश्यै॑ स॒त्याः᳚ स॒त्याः᳚ स॒दृश्यै॒ यद् यथ् स॒दृश्यै॑ स॒त्याः᳚ । \newline
27. स॒दृश्यै॑ स॒त्याः᳚ स॒त्याः᳚ स॒दृश्यै॑ स॒दृश्यै॑ स॒त्या॑ ऋ॒क्ष मृ॒क्षꣳ स॒त्याः᳚ स॒दृश्यै॑ स॒दृश्यै॑ स॒त्या॑ ऋ॒क्षम् । \newline
28. स॒त्या॑ ऋ॒क्ष मृ॒क्षꣳ स॒त्याः᳚ स॒त्या॑ ऋ॒क्षम् निर्.ऋ॑त्या॒ निर्.ऋ॑त्य॒ र्‌क्षꣳ स॒त्याः᳚ स॒त्या॑ ऋ॒क्षम् निर्.ऋ॑त्या । \newline
29. ऋ॒क्षम् निर्.ऋ॑त्या॒ निर्.ऋ॑त्य॒ र्‌क्ष मृ॒क्षम् निर्.ऋ॑ त्यै॒वैव निर्.ऋ॑त्य॒ र्‌क्ष मृ॒क्षम् निर्.ऋ॑त्यै॒व । \newline
30. निर्.ऋ॑त्यै॒ वैव निर्.ऋ॑त्या॒ निर्.ऋ॑त्यै॒ वास्या᳚ स्यै॒व निर्.ऋ॑त्या॒ निर्.ऋ॑त्यै॒वास्य॑ । \newline
31. निर्.ऋ॒त्येति॒ निः - ऋ॒त्या॒ । \newline
32. ए॒वास्या᳚ स्यै॒वै वास्य॑ य॒ज्ञ्ं ॅय॒ज्ञ् म॑स्यै॒वै वास्य॑ य॒ज्ञ्म् । \newline
33. अ॒स्य॒ य॒ज्ञ्ं ॅय॒ज्ञ् म॑स्यास्य य॒ज्ञ्म् ग्रा॑हयति ग्राहयति य॒ज्ञ् म॑स्यास्य य॒ज्ञ्म् ग्रा॑हयति । \newline
34. य॒ज्ञ्म् ग्रा॑हयति ग्राहयति य॒ज्ञ्ं ॅय॒ज्ञ्म् ग्रा॑हयति॒ व्यावृ॑त्ते॒ व्यावृ॑त्ते ग्राहयति य॒ज्ञ्ं ॅय॒ज्ञ्म् ग्रा॑हयति॒ व्यावृ॑त्ते । \newline
35. ग्रा॒ह॒य॒ति॒ व्यावृ॑त्ते॒ व्यावृ॑त्ते ग्राहयति ग्राहयति॒ व्यावृ॑त्ते देव॒यज॑ने देव॒यज॑ने॒ व्यावृ॑त्ते ग्राहयति ग्राहयति॒ व्यावृ॑त्ते देव॒यज॑ने । \newline
36. व्यावृ॑त्ते देव॒यज॑ने देव॒यज॑ने॒ व्यावृ॑त्ते॒ व्यावृ॑त्ते देव॒यज॑ने याजयेद् याजयेद् देव॒यज॑ने॒ व्यावृ॑त्ते॒ व्यावृ॑त्ते देव॒यज॑ने याजयेत् । \newline
37. व्यावृ॑त्त॒ इति॑ वि - आवृ॑त्ते । \newline
38. दे॒व॒यज॑ने याजयेद् याजयेद् देव॒यज॑ने देव॒यज॑ने याजयेद् व्या॒वृत्का॑मं ॅव्या॒वृत्का॑मं ॅयाजयेद् देव॒यज॑ने देव॒यज॑ने याजयेद् व्या॒वृत्का॑मम् । \newline
39. दे॒व॒यज॑न॒ इति॑ देव - यज॑ने । \newline
40. या॒ज॒ये॒द् व्या॒वृत्का॑मं ॅव्या॒वृत्का॑मं ॅयाजयेद् याजयेद् व्या॒वृत्का॑मं॒ ॅयं ॅयं ॅव्या॒वृत्का॑मं ॅयाजयेद् याजयेद् व्या॒वृत्का॑मं॒ ॅयम् । \newline
41. व्या॒वृत्का॑मं॒ ॅयं ॅयं ॅव्या॒वृत्का॑मं ॅव्या॒वृत्का॑मं॒ ॅयम् पात्रे॒ पात्रे॒ यं ॅव्या॒वृत्का॑मं ॅव्या॒वृत्का॑मं॒ ॅयम् पात्रे᳚ । \newline
42. व्या॒वृत्का॑म॒मिति॑ व्या॒वृत् - का॒म॒म् । \newline
43. यम् पात्रे॒ पात्रे॒ यं ॅयम् पात्रे॑ वा वा॒ पात्रे॒ यं ॅयम् पात्रे॑ वा । \newline
44. पात्रे॑ वा वा॒ पात्रे॒ पात्रे॑ वा॒ तल्पे॒ तल्पे॑ वा॒ पात्रे॒ पात्रे॑ वा॒ तल्पे᳚ । \newline
45. वा॒ तल्पे॒ तल्पे॑ वा वा॒ तल्पे॑ वा वा॒ तल्पे॑ वा वा॒ तल्पे॑ वा । \newline
46. तल्पे॑ वा वा॒ तल्पे॒ तल्पे॑ वा॒ मीमाꣳ॑सेर॒न् मीमाꣳ॑सेरन्. वा॒ तल्पे॒ तल्पे॑ वा॒ मीमाꣳ॑सेरन्न् । \newline
47. वा॒ मीमाꣳ॑सेर॒न् मीमाꣳ॑सेरन्. वा वा॒ मीमाꣳ॑सेरन् प्रा॒चीन॑म् प्रा॒चीन॒म् मीमाꣳ॑सेरन्. वा वा॒ मीमाꣳ॑सेरन् प्रा॒चीन᳚म् । \newline
48. मीमाꣳ॑सेरन् प्रा॒चीन॑म् प्रा॒चीन॒म् मीमाꣳ॑सेर॒न् मीमाꣳ॑सेरन् प्रा॒चीन॑ माहव॒नीया॑ दाहव॒नीया᳚त् प्रा॒चीन॒म् मीमाꣳ॑सेर॒न् मीमाꣳ॑सेरन् प्रा॒चीन॑ माहव॒नीया᳚त् । \newline
49. प्रा॒चीन॑ माहव॒नीया॑ दाहव॒नीया᳚त् प्रा॒चीन॑म् प्रा॒चीन॑ माहव॒नीया᳚त् प्रव॒णम् प्र॑व॒ण मा॑हव॒नीया᳚त् प्रा॒चीन॑म् प्रा॒चीन॑ माहव॒नीया᳚त् प्रव॒णम् । \newline
50. आ॒ह॒व॒नीया᳚त् प्रव॒णम् प्र॑व॒ण मा॑हव॒नीया॑ दाहव॒नीया᳚त् प्रव॒णꣳ स्या᳚थ् स्यात् प्रव॒ण मा॑हव॒नीया॑ दाहव॒नीया᳚त् प्रव॒णꣳ स्या᳚त् । \newline
51. आ॒ह॒व॒नीया॒दित्या᳚ - ह॒व॒नीया᳚त् । \newline
52. प्र॒व॒णꣳ स्या᳚थ् स्यात् प्रव॒णम् प्र॑व॒णꣳ स्या᳚त् प्रती॒चीन॑म् प्रती॒चीनꣳ॑ स्यात् प्रव॒णम् प्र॑व॒णꣳ स्या᳚त् प्रती॒चीन᳚म् । \newline
53. प्र॒व॒णमिति॑ प्र - व॒नम् । \newline
54. स्या॒त् प्र॒ती॒चीन॑म् प्रती॒चीनꣳ॑ स्याथ् स्यात् प्रती॒चीन॒म् गार्.ह॑पत्या॒द् गार्.ह॑पत्यात् प्रती॒चीनꣳ॑ स्याथ् स्यात् प्रती॒चीन॒म् गार्.ह॑पत्यात् । \newline
55. प्र॒ती॒चीन॒म् गार्.ह॑पत्या॒द् गार्.ह॑पत्यात् प्रती॒चीन॑म् प्रती॒चीन॒म् गार्.ह॑पत्या दे॒त दे॒तद् गार्.ह॑पत्यात् प्रती॒चीन॑म् प्रती॒चीन॒म् गार्.ह॑पत्या दे॒तत् । \newline
56. गार्.ह॑पत्या दे॒त दे॒तद् गार्.ह॑पत्या॒द् गार्.ह॑पत्या दे॒तद् वै वा ए॒तद् गार्.ह॑पत्या॒द् गार्.ह॑पत्या दे॒तद् वै । \newline
57. गार्.ह॑पत्या॒दिति॒ गार्.ह॑ - प॒त्या॒त् । \newline
58. ए॒तद् वै वा ए॒त दे॒तद् वै व्यावृ॑त्तं॒ ॅव्यावृ॑त्तं॒ ॅवा ए॒त दे॒तद् वै व्यावृ॑त्तम् । \newline
59. वै व्यावृ॑त्तं॒ ॅव्यावृ॑त्तं॒ ॅवै वै व्यावृ॑त्तम् देव॒यज॑नम् देव॒यज॑नं॒ ॅव्यावृ॑त्तं॒ ॅवै वै व्यावृ॑त्तम् देव॒यज॑नम् । \newline
60. व्यावृ॑त्तम् देव॒यज॑नम् देव॒यज॑नं॒ ॅव्यावृ॑त्तं॒ ॅव्यावृ॑त्तम् देव॒यज॑नं॒ ॅवि वि दे॑व॒यज॑नं॒ ॅव्यावृ॑त्तं॒ ॅव्यावृ॑त्तम् देव॒यज॑नं॒ ॅवि । \newline
61. व्यावृ॑त्त॒मिति॑ वि - आवृ॑त्तम् । \newline
62. दे॒व॒यज॑नं॒ ॅवि वि दे॑व॒यज॑नम् देव॒यज॑नं॒ ॅवि पा॒प्मना॑ पा॒प्मना॒ वि दे॑व॒यज॑नम् देव॒यज॑नं॒ ॅवि पा॒प्मना᳚ । \newline
63. दे॒व॒यज॑न॒मिति॑ देव - यज॑नम् । \newline
64. वि पा॒प्मना॑ पा॒प्मना॒ वि वि पा॒प्मना॒ भ्रातृ॑व्येण॒ भ्रातृ॑व्येण पा॒प्मना॒ वि वि पा॒प्मना॒ भ्रातृ॑व्येण । \newline
65. पा॒प्मना॒ भ्रातृ॑व्येण॒ भ्रातृ॑व्येण पा॒प्मना॑ पा॒प्मना॒ भ्रातृ॑व्ये॒णा भ्रातृ॑व्येण पा॒प्मना॑ पा॒प्मना॒ भ्रातृ॑व्ये॒णा । \newline
66. भ्रातृ॑व्ये॒णा भ्रातृ॑व्येण॒ भ्रातृ॑व्ये॒णा व॑र्तते वर्तत॒ आ भ्रातृ॑व्येण॒ भ्रातृ॑व्ये॒णा व॑र्तते । \newline
67. आ व॑र्तते वर्तत॒ आ व॑र्तते॒ न न व॑र्तत॒ आ व॑र्तते॒ न । \newline
68. व॒र्त॒ते॒ न न व॑र्तते वर्तते॒ नैन॑ मेन॒न् न व॑र्तते वर्तते॒ नैन᳚म् । \newline
69. नैन॑ मेन॒न् न नैन॒म् पात्रे॒ पात्र॑ एन॒न् न नैन॒म् पात्रे᳚ । \newline
70. ए॒न॒म् पात्रे॒ पात्र॑ एन मेन॒म् पात्रे॒ न न पात्र॑ एन मेन॒म् पात्रे॒ न । \newline
71. पात्रे॒ न न पात्रे॒ पात्रे॒ न तल्पे॒ तल्पे॒ न पात्रे॒ पात्रे॒ न तल्पे᳚ । \newline
72. न तल्पे॒ तल्पे॒ न न तल्पे॑ मीमाꣳसन्ते मीमाꣳसन्ते॒ तल्पे॒ न न तल्पे॑ मीमाꣳसन्ते । \newline
73. तल्पे॑ मीमाꣳसन्ते मीमाꣳसन्ते॒ तल्पे॒ तल्पे॑ मीमाꣳसन्ते का॒र्ये॑ का॒र्ये॑ मीमाꣳसन्ते॒ तल्पे॒ तल्पे॑ मीमाꣳसन्ते का॒र्ये᳚ । \newline
74. मी॒माꣳ॒॒स॒न्ते॒ का॒र्ये॑ का॒र्ये॑ मीमाꣳसन्ते मीमाꣳसन्ते का॒र्ये॑ देव॒यज॑ने देव॒यज॑ने का॒र्ये॑ मीमाꣳसन्ते मीमाꣳसन्ते का॒र्ये॑ देव॒यज॑ने । \newline
75. का॒र्ये॑ देव॒यज॑ने देव॒यज॑ने का॒र्ये॑ का॒र्ये॑ देव॒यज॑ने याजयेद् याजयेद् देव॒यज॑ने का॒र्ये॑ का॒र्ये॑ देव॒यज॑ने याजयेत् । \newline
76. दे॒व॒यज॑ने याजयेद् याजयेद् देव॒यज॑ने देव॒यज॑ने याजये॒द् भूति॑काम॒म् भूति॑कामं ॅयाजयेद् देव॒यज॑ने देव॒यज॑ने याजये॒द् भूति॑कामम् । \newline
77. दे॒व॒यज॑न॒ इति॑ देव - यज॑ने । \newline
78. या॒ज॒ये॒द् भूति॑काम॒म् भूति॑कामं ॅयाजयेद् याजये॒द् भूति॑कामम् का॒र्यः॑ का॒र्यो॑ भूति॑कामं ॅयाजयेद् याजये॒द् भूति॑कामम् का॒र्यः॑ । \newline
79. भूति॑कामम् का॒र्यः॑ का॒र्यो॑ भूति॑काम॒म् भूति॑कामम् का॒र्यो॑ वै वै का॒र्यो॑ भूति॑काम॒म् भूति॑कामम् का॒र्यो॑ वै । \newline
80. भूति॑काम॒मिति॒ भूति॑ - का॒म॒म् । \newline
81. का॒र्यो॑ वै वै का॒र्यः॑ का॒र्यो॑ वै पुरु॑षः॒ पुरु॑षो॒ वै का॒र्यः॑ का॒र्यो॑ वै पुरु॑षः । \newline
82. वै पुरु॑षः॒ पुरु॑षो॒ वै वै पुरु॑षो॒ भव॑ति॒ भव॑ति॒ पुरु॑षो॒ वै वै पुरु॑षो॒ भव॑ति । \newline
83. पुरु॑षो॒ भव॑ति॒ भव॑ति॒ पुरु॑षः॒ पुरु॑षो॒ भव॑ त्ये॒वैव भव॑ति॒ पुरु॑षः॒ पुरु॑षो॒ भव॑ त्ये॒व । \newline
84. भव॑ त्ये॒वैव भव॑ति॒ भव॑ त्ये॒व । \newline
85. ए॒वेत्ये॒व । \newline
\pagebreak
\markright{ TS 6.2.7.1  \hfill https://www.vedavms.in \hfill}

\section{ TS 6.2.7.1 }

\textbf{TS 6.2.7.1 } \newline
\textbf{Samhita Paata} \newline

तेभ्य॑ उत्तरवे॒दिः सिꣳ॒॒॒ही रू॒पं कृ॒त्वोभया॑-नन्त॒राऽप॒क्रम्या॑तिष्ठ॒त् ते दे॒वा अ॑मन्यन्त यत॒रान्. वा इ॒यमु॑पाव॒र्थ्स्यति॒ त इ॒दं भ॑विष्य॒न्तीति॒ तामुपा॑मन्त्रयन्त॒ साऽब्र॑वी॒द् वरं॑ ॅवृणै॒ सर्वा॒न् मया॒ कामा॒न् व्य॑श्नवथ॒ पूर्वां॒ तु मा॒ऽग्नेराहु॑तिरश्नवता॒ इति॒ तस्मा॑दुत्तरवे॒दिं पूर्वा॑म॒ग्ने- र्व्याघा॑रयन्ति॒ वारे॑वृतꣳ॒॒ ह्य॑स्यै॒ शम्य॑या॒ परि॑ मिमीते॒- [  ] \newline

\textbf{Pada Paata} \newline

तेभ्यः॑ । उ॒त्त॒र॒वे॒दिरित्यु॑त्तर - वे॒दिः । सिꣳ॒॒हीः । रू॒पम् । कृ॒त्वा । उ॒भयान्॑ । अ॒न्त॒रा । अ॒प॒क्रम्येत्य॑प - क्रम्य॑ । अ॒ति॒ष्ठ॒त् । ते । दे॒वाः । अ॒म॒न्य॒न्त॒ । य॒त॒रान् । वै । इ॒यम् । उ॒पा॒व॒र्थ्स्यतीत्यु॑प-आ॒व॒र्थ्स्यति॑ । ते । इ॒दम् । भ॒वि॒ष्य॒न्ति॒ । इति॑ । ताम् । उपेति॑ । अ॒म॒न्त्र॒य॒न्त॒ । सा । अ॒ब्र॒वी॒त् । वर᳚म् । वृ॒णै॒ । सर्वान्॑ । मया᳚ । कामान्॑ । वीति॑ । अ॒श्न॒व॒थ॒ । पूर्वा᳚म् । तु । मा॒ । अ॒ग्नेः । आहु॑ति॒रित्या - हु॒तिः॒ । अ॒श्न॒व॒तै॒ । इति॑ । तस्मा᳚त् । उ॒त्त॒र॒वे॒दिमित्यु॑त्तर - वे॒दिम् । पूर्वा᳚म् । अ॒ग्नेः । व्याघा॑रय॒न्तीति॑ वि-आघा॑रयन्ति । वारे॑वृत॒मिति॒ वारे᳚-वृ॒त॒म् । हि । अ॒स्यै॒ । शम्य॑या । परीति॑ । मि॒मी॒ते॒ ।  \newline


\textbf{Krama Paata} \newline

तेभ्य॑ उत्तरवे॒दिः । उ॒त्त॒र॒वे॒दिः सिꣳ॒॒हीः । उ॒त्त॒र॒वे॒दिरित्यु॑त्तर - वे॒दिः । सिꣳ॒॒ही रू॒पम् । रू॒पम् कृ॒त्वा । कृ॒त्वोभयान्॑ । उ॒भया॑नन्त॒रा । अ॒न्त॒राऽप॒क्रम्य॑ । अ॒प॒क्रम्या॑तिष्ठत् । अ॒प॒क्रम्येत्य॑प - क्रम्य॑ । अ॒ति॒ष्ठ॒त् ते । ते दे॒वाः । दे॒वा अ॑मन्यन्त । अ॒म॒न्य॒न्त॒ य॒त॒रान् । य॒त॒रान्. वै । वा इ॒यम् । इ॒यमु॑पाव॒र्थ्स्यति॑ । उ॒पा॒व॒र्थ्स्यति॒ ते । उ॒पा॒व॒र्त्स्यतीत्यु॑प - आ॒व॒र्थ्स्यति॑ । त इ॒दम् । इ॒दम् भ॑विष्यन्ति । भ॒वि॒ष्य॒न्तीति॑ । इति॒ ताम् । तामुप॑ । उपा॑मन्त्रयन्त । अ॒म॒न्त्र॒य॒न्त॒ सा । साऽब्र॑वीत् । अ॒ब्र॒वी॒द् वर᳚म् । वर॑म् ॅवृणै । वृ॒णै॒ सर्वान्॑ । सर्वा॒न् मया᳚ । मया॒ कामान्॑ । कामा॒न्.॒ वि । व्य॑श्ञवथ । अ॒श्ञ॒व॒थ॒ पूर्वा᳚म् । पूर्वा॒म् तु । तु मा᳚ । मा॒ऽग्नेः । अ॒ग्नेराहु॑तिः । आहु॑तिरश्ञवतै । आहु॑ति॒रित्या - हु॒तिः॒ । अ॒श्ञ॒व॒ता॒ इति॑ । इति॒ तस्मा᳚त् । तस्मा॑दुत्तरवे॒दिम् । उ॒त्त॒र॒वे॒दिम् पूर्वा᳚म् । उ॒त्त॒र॒वे॒दिमित्यु॑त्तर - वे॒दिम् । पूर्वा॑म॒ग्नेः । अ॒ग्नेर् व्याघा॑रयन्ति । व्याघा॑रयन्ति॒ वारे॑वृतम् । व्याघा॑रय॒न्तीति॑ वि - आघा॑रयन्ति । वारे॑वृतꣳ॒॒ हि । वारे॑वृत॒मिति॒ वारे᳚ - वृ॒त॒म् । ह्य॑स्यै । अ॒स्यै॒ शम्य॑या । शम्य॑या॒ परि॑ । परि॑ मिमीते । मि॒मी॒ते॒ मात्रा᳚ \newline

\textbf{Jatai Paata} \newline

1. तेभ्य॑ उत्तरवे॒दि रु॑त्तरवे॒दि स्तेभ्य॒ स्तेभ्य॑ उत्तरवे॒दिः । \newline
2. उ॒त्त॒र॒वे॒दिः सिꣳ॒॒हीः सिꣳ॒॒ही रु॑त्तरवे॒दि रु॑त्तरवे॒दिः सिꣳ॒॒हीः । \newline
3. उ॒त्त॒र॒वे॒दिरित्यु॑त्तर - वे॒दिः । \newline
4. सिꣳ॒॒ही रू॒पꣳ रू॒पꣳ सिꣳ॒॒हीः सिꣳ॒॒ही रू॒पम् । \newline
5. रू॒पम् कृ॒त्वा कृ॒त्वा रू॒पꣳ रू॒पम् कृ॒त्वा । \newline
6. कृ॒त्वोभया॑ नु॒भया᳚न् कृ॒त्वा कृ॒त्वोभयान्॑ । \newline
7. उ॒भया॑ नन्त॒रा ऽन्त॒रोभया॑ नु॒भया॑ नन्त॒रा । \newline
8. अ॒न्त॒रा ऽप॒क्रम्या॑ प॒क्रम्या᳚ न्त॒रा ऽन्त॒रा ऽप॒क्रम्य॑ । \newline
9. अ॒प॒क्रम्या॑ तिष्ठ दतिष्ठ दप॒क्रम्या॑ प॒क्रम्या॑ति ष्ठत् । \newline
10. अ॒प॒क्रम्येत्य॑प - क्रम्य॑ । \newline
11. अ॒ति॒ष्ठ॒त् ते ते॑ ऽतिष्ठ दतिष्ठ॒त् ते । \newline
12. ते दे॒वा दे॒वा स्ते ते दे॒वाः । \newline
13. दे॒वा अ॑मन्यन्ता मन्यन्त दे॒वा दे॒वा अ॑मन्यन्त । \newline
14. अ॒म॒न्य॒न्त॒ य॒त॒रान्. य॑त॒रा न॑मन्यन्ता मन्यन्त यत॒रान् । \newline
15. य॒त॒रान्. वै वै य॑त॒रान्. य॑त॒रान्. वै । \newline
16. वा इ॒य मि॒यं ॅवै वा इ॒यम् । \newline
17. इ॒य मु॑पाव॒र्थ्स्य त्यु॑पाव॒र्थ्स्यती॒य मि॒य मु॑पाव॒र्थ्स्यति॑ । \newline
18. उ॒पा॒व॒र्थ्स्यति॒ ते त उ॑पाव॒र्थ्स्य त्यु॑पाव॒र्थ्स्यति॒ ते । \newline
19. उ॒पा॒व॒र्थ्स्यतीत्यु॑प - आ॒व॒र्थ्स्यति॑ । \newline
20. त इ॒द मि॒दम् ते त इ॒दम् । \newline
21. इ॒दम् भ॑विष्यन्ति भविष्यन्ती॒द मि॒दम् भ॑विष्यन्ति । \newline
22. भ॒वि॒ष्य॒न्तीतीति॑ भविष्यन्ति भविष्य॒न्तीति॑ । \newline
23. इति॒ ताम् ता मितीति॒ ताम् । \newline
24. ता मुपोप॒ ताम् ता मुप॑ । \newline
25. उपा॑ मन्त्रयन्ता मन्त्रय॒ न्तोपोपा॑ मन्त्रयन्त । \newline
26. अ॒म॒न्त्र॒य॒न्त॒ सा सा ऽम॑न्त्रयन्ता मन्त्रयन्त॒ सा । \newline
27. सा ऽब्र॑वी दब्रवी॒थ् सा सा ऽब्र॑वीत् । \newline
28. अ॒ब्र॒वी॒द् वरं॒ ॅवर॑ मब्रवी दब्रवी॒द् वर᳚म् । \newline
29. वरं॑ ॅवृणै वृणै॒ वरं॒ ॅवरं॑ ॅवृणै । \newline
30. वृ॒णै॒ सर्वा॒न् थ्सर्वा᳚न् वृणै वृणै॒ सर्वान्॑ । \newline
31. सर्वा॒न् मया॒ मया॒ सर्वा॒न् थ्सर्वा॒न् मया᳚ । \newline
32. मया॒ कामा॒न् कामा॒न् मया॒ मया॒ कामान्॑ । \newline
33. कामा॒न्॒. वि वि कामा॒न् कामा॒न्॒. वि । \newline
34. व्य॑श्ञवथा श्ञवथ॒ वि व्य॑श्ञवथ । \newline
35. अ॒श्ञ॒व॒थ॒ पूर्वा॒म् पूर्वा॑ मश्ञवथा श्ञवथ॒ पूर्वा᳚म् । \newline
36. पूर्वा॒म् तु तु पूर्वा॒म् पूर्वा॒म् तु । \newline
37. तु मा॑ मा॒ तु तु मा᳚ । \newline
38. मा॒ ऽग्ने र॒ग्नेर् मा॑ मा॒ ऽग्नेः । \newline
39. अ॒ग्ने राहु॑ति॒ राहु॑ति र॒ग्ने र॒ग्ने राहु॑तिः । \newline
40. आहु॑ति रश्ञवता अश्ञवता॒ आहु॑ति॒ राहु॑ति रश्ञवतै । \newline
41. आहु॑ति॒रित्या - हु॒तिः॒ । \newline
42. अ॒श्ञ॒व॒ता॒ इती त्य॑श्ञवता अश्ञवता॒ इति॑ । \newline
43. इति॒ तस्मा॒त् तस्मा॒ दितीति॒ तस्मा᳚त् । \newline
44. तस्मा॑ दुत्तरवे॒दि मु॑त्तरवे॒दिम् तस्मा॒त् तस्मा॑ दुत्तरवे॒दिम् । \newline
45. उ॒त्त॒र॒वे॒दिम् पूर्वा॒म् पूर्वा॑ मुत्तरवे॒दि मु॑त्तरवे॒दिम् पूर्वा᳚म् । \newline
46. उ॒त्त॒र॒वे॒दिमित्यु॑त्तर - वे॒दिम् । \newline
47. पूर्वा॑ म॒ग्ने र॒ग्नेः पूर्वा॒म् पूर्वा॑ म॒ग्नेः । \newline
48. अ॒ग्नेर् व्याघा॑रयन्ति॒ व्याघा॑रय न्त्य॒ग्ने र॒ग्नेर् व्याघा॑रयन्ति । \newline
49. व्याघा॑रयन्ति॒ वारे॑वृतं॒ ॅवारे॑वृतं॒ ॅव्याघा॑रयन्ति॒ व्याघा॑रयन्ति॒ वारे॑वृतम् । \newline
50. व्याघा॑रय॒न्तीति॑ वि - आघा॑रयन्ति । \newline
51. वारे॑वृतꣳ॒॒ हि हि वारे॑वृतं॒ ॅवारे॑वृतꣳ॒॒ हि । \newline
52. वारे॑वृत॒मिति॒ वारे᳚ - वृ॒त॒म् । \newline
53. ह्य॑स्या अस्यै॒ हि ह्य॑स्यै । \newline
54. अ॒स्यै॒ शम्य॑या॒ शम्य॑या ऽस्या अस्यै॒ शम्य॑या । \newline
55. शम्य॑या॒ परि॒ परि॒ शम्य॑या॒ शम्य॑या॒ परि॑ । \newline
56. परि॑ मिमीते मिमीते॒ परि॒ परि॑ मिमीते । \newline
57. मि॒मी॒ते॒ मात्रा॒ मात्रा॑ मिमीते मिमीते॒ मात्रा᳚ । \newline

\textbf{Ghana Paata } \newline

1. तेभ्य॑ उत्तरवे॒दि रु॑त्तरवे॒दि स्तेभ्य॒ स्तेभ्य॑ उत्तरवे॒दिः सिꣳ॒॒हीः सिꣳ॒॒ही रु॑त्तरवे॒दि स्तेभ्य॒ स्तेभ्य॑ उत्तरवे॒दिः सिꣳ॒॒हीः । \newline
2. उ॒त्त॒र॒वे॒दिः सिꣳ॒॒हीः सिꣳ॒॒ही रु॑त्तरवे॒दि रु॑त्तरवे॒दिः सिꣳ॒॒ही रू॒पꣳ रू॒पꣳ सिꣳ॒॒ही रु॑त्तरवे॒दि रु॑त्तरवे॒दिः सिꣳ॒॒ही रू॒पम् । \newline
3. उ॒त्त॒र॒वे॒दिरित्यु॑त्तर - वे॒दिः । \newline
4. सिꣳ॒॒ही रू॒पꣳ रू॒पꣳ सिꣳ॒॒हीः सिꣳ॒॒ही रू॒पम् कृ॒त्वा कृ॒त्वा रू॒पꣳ सिꣳ॒॒हीः सिꣳ॒॒ही रू॒पम् कृ॒त्वा । \newline
5. रू॒पम् कृ॒त्वा कृ॒त्वा रू॒पꣳ रू॒पम् कृ॒त्वोभया॑ नु॒भया᳚न् कृ॒त्वा रू॒पꣳ रू॒पम् कृ॒त्वोभयान्॑ । \newline
6. कृ॒त्वोभया॑ नु॒भया᳚न् कृ॒त्वा कृ॒त्वोभया॑ नन्त॒रा ऽन्त॒रोभया᳚न् कृ॒त्वा कृ॒त्वोभया॑ नन्त॒रा । \newline
7. उ॒भया॑ नन्त॒रा ऽन्त॒रोभया॑ नु॒भया॑ नन्त॒रा ऽप॒क्रम्या॑ प॒क्रम्या᳚ न्त॒रोभया॑ नु॒भया॑ नन्त॒रा ऽप॒क्रम्य॑ । \newline
8. अ॒न्त॒रा ऽप॒क्रम्या॑ प॒क्रम्या᳚ न्त॒रा ऽन्त॒रा ऽप॒क्रम्या॑ तिष्ठ दतिष्ठ दप॒क्रम्या᳚ न्त॒रा ऽन्त॒रा ऽप॒क्रम्या॑ तिष्ठत् । \newline
9. अ॒प॒क्रम्या॑ तिष्ठ दतिष्ठ दप॒क्रम्या॑ प॒क्रम्या॑ तिष्ठ॒त् ते ते॑ ऽतिष्ठ दप॒क्रम्या॑ प॒क्रम्या॑ तिष्ठ॒त् ते । \newline
10. अ॒प॒क्रम्येत्य॑प - क्रम्य॑ । \newline
11. अ॒ति॒ष्ठ॒त् ते ते॑ ऽतिष्ठ दतिष्ठ॒त् ते दे॒वा दे॒वा स्ते॑ ऽतिष्ठ दतिष्ठ॒त् ते दे॒वाः । \newline
12. ते दे॒वा दे॒वा स्ते ते दे॒वा अ॑मन्यन्ता मन्यन्त दे॒वा स्ते ते दे॒वा अ॑मन्यन्त । \newline
13. दे॒वा अ॑मन्यन्ता मन्यन्त दे॒वा दे॒वा अ॑मन्यन्त यत॒रान्. य॑त॒रा न॑मन्यन्त दे॒वा दे॒वा अ॑मन्यन्त यत॒रान् । \newline
14. अ॒म॒न्य॒न्त॒ य॒त॒रान्. य॑त॒रा न॑मन्यन्ता मन्यन्त यत॒रान्. वै वै य॑त॒रा न॑मन्यन्ता मन्यन्त यत॒रान्. वै । \newline
15. य॒त॒रान्. वै वै य॑त॒रान्. य॑त॒रान्. वा इ॒य मि॒यं ॅवै य॑त॒रान्. य॑त॒रान्. वा इ॒यम् । \newline
16. वा इ॒य मि॒यं ॅवै वा इ॒य मु॑पाव॒र्थ्स्य त्यु॑पाव॒र्थ्स्य ती॒यं ॅवै वा इ॒य मु॑पाव॒र्थ्स्यति॑ । \newline
17. इ॒य मु॑पाव॒र्थ्स्य त्यु॑पाव॒र्थ्स्य ती॒य मि॒य मु॑पाव॒र्थ्स्यति॒ ते त उ॑पाव॒र्थ्स्य ती॒य मि॒य मु॑पाव॒र्थ्स्यति॒ ते । \newline
18. उ॒पा॒व॒र्थ्स्यति॒ ते त उ॑पाव॒र्थ्स्य त्यु॑पाव॒र्थ्स्यति॒ त इ॒द मि॒दम् त उ॑पाव॒र्थ्स्य त्यु॑पाव॒र्थ्स्यति॒ त इ॒दम् । \newline
19. उ॒पा॒व॒र्थ्स्यतीत्यु॑प - आ॒व॒र्थ्स्यति॑ । \newline
20. त इ॒द मि॒दम् ते त इ॒दम् भ॑विष्यन्ति भविष्यन् ती॒दम् ते त इ॒दम् भ॑विष्यन्ति । \newline
21. इ॒दम् भ॑विष्यन्ति भविष्यन् ती॒द मि॒दम् भ॑विष्य॒न्ती तीति॑ भविष्यन् ती॒द मि॒दम् भ॑विष्य॒न्तीति॑ । \newline
22. भ॒वि॒ष्य॒न्ती तीति॑ भविष्यन्ति भविष्य॒न्तीति॒ ताम् ता मिति॑ भविष्यन्ति भविष्य॒न्तीति॒ ताम् । \newline
23. इति॒ ताम् ता मितीति॒ ता मुपोप॒ ता मितीति॒ ता मुप॑ । \newline
24. ता मुपोप॒ ताम् ता मुपा॑मन्त्रयन्ता मन्त्रय॒न्तोप॒ ताम् ता मुपा॑मन्त्रयन्त । \newline
25. उपा॑मन्त्रयन्ता मन्त्रय॒न्तो पोपा॑ मन्त्रयन्त॒ सा सा ऽम॑न्त्रय॒न्तो पोपा॑ मन्त्रयन्त॒ सा । \newline
26. अ॒म॒न्त्र॒य॒न्त॒ सा सा ऽम॑न्त्रयन्ता मन्त्रयन्त॒ सा ऽब्र॑वी दब्रवी॒थ् सा ऽम॑न्त्रयन्ता मन्त्रयन्त॒ सा ऽब्र॑वीत् । \newline
27. सा ऽब्र॑वी दब्रवी॒थ् सा सा ऽब्र॑वी॒द् वरं॒ ॅवर॑ मब्रवी॒थ् सा सा ऽब्र॑वी॒द् वर᳚म् । \newline
28. अ॒ब्र॒वी॒द् वरं॒ ॅवर॑ मब्रवी दब्रवी॒द् वरं॑ ॅवृणै वृणै॒ वर॑ मब्रवी दब्रवी॒द् वरं॑ ॅवृणै । \newline
29. वरं॑ ॅवृणै वृणै॒ वरं॒ ॅवरं॑ ॅवृणै॒ सर्वा॒न् थ्सर्वा᳚न् वृणै॒ वरं॒ ॅवरं॑ ॅवृणै॒ सर्वान्॑ । \newline
30. वृ॒णै॒ सर्वा॒न् थ्सर्वा᳚न् वृणै वृणै॒ सर्वा॒न् मया॒ मया॒ सर्वा᳚न् वृणै वृणै॒ सर्वा॒न् मया᳚ । \newline
31. सर्वा॒न् मया॒ मया॒ सर्वा॒न् थ्सर्वा॒न् मया॒ कामा॒न् कामा॒न् मया॒ सर्वा॒न् थ्सर्वा॒न् मया॒ कामान्॑ । \newline
32. मया॒ कामा॒न् कामा॒न् मया॒ मया॒ कामा॒न्॒. वि वि कामा॒न् मया॒ मया॒ कामा॒न्॒. वि । \newline
33. कामा॒न्॒. वि वि कामा॒न् कामा॒न् व्य॑श्ञवथा श्ञवथ॒ वि कामा॒न् कामा॒न् व्य॑श्ञवथ । \newline
34. व्य॑श्ञवथा श्ञवथ॒ वि व्य॑श्ञवथ॒ पूर्वा॒म् पूर्वा॑ मश्ञवथ॒ वि व्य॑श्ञवथ॒ पूर्वा᳚म् । \newline
35. अ॒श्ञ॒व॒थ॒ पूर्वा॒म् पूर्वा॑ मश्ञवथा श्ञवथ॒ पूर्वा॒म् तु तु पूर्वा॑ मश्ञवथा श्ञवथ॒ पूर्वा॒म् तु । \newline
36. पूर्वा॒म् तु तु पूर्वा॒म् पूर्वा॒म् तु मा॑ मा॒ तु पूर्वा॒म् पूर्वा॒म् तु मा᳚ । \newline
37. तु मा॑ मा॒ तु तु मा॒ ऽग्ने र॒ग्नेर् मा॒ तु तु मा॒ ऽग्नेः । \newline
38. मा॒ ऽग्ने र॒ग्नेर् मा॑ मा॒ ऽग्ने राहु॑ति॒ राहु॑ति र॒ग्नेर् मा॑ मा॒ ऽग्ने राहु॑तिः । \newline
39. अ॒ग्ने राहु॑ति॒ राहु॑ति र॒ग्ने र॒ग्ने राहु॑ति रश्ञवता अश्ञवता॒ आहु॑ति र॒ग्ने र॒ग्ने राहु॑ति रश्ञवतै । \newline
40. आहु॑ति रश्ञवता अश्ञवता॒ आहु॑ति॒ राहु॑ति रश्ञवता॒ इती त्य॑श्ञवता॒ आहु॑ति॒ राहु॑ति रश्ञवता॒ इति॑ । \newline
41. आहु॑ति॒रित्या - हु॒तिः॒ । \newline
42. अ॒श्ञ॒व॒ता॒ इतीत्य॑ श्ञवता अश्ञवता॒ इति॒ तस्मा॒त् तस्मा॒दि त्य॑श्ञवता अश्ञवता॒ इति॒ तस्मा᳚त् । \newline
43. इति॒ तस्मा॒त् तस्मा॒दि तीति॒ तस्मा॑ दुत्तरवे॒दि मु॑त्तरवे॒दिम् तस्मा॒दि तीति॒ तस्मा॑ दुत्तरवे॒दिम् । \newline
44. तस्मा॑ दुत्तरवे॒दि मु॑त्तरवे॒दिम् तस्मा॒त् तस्मा॑ दुत्तरवे॒दिम् पूर्वा॒म् पूर्वा॑ मुत्तरवे॒दिम् तस्मा॒त् तस्मा॑ दुत्तरवे॒दिम् पूर्वा᳚म् । \newline
45. उ॒त्त॒र॒वे॒दिम् पूर्वा॒म् पूर्वा॑ मुत्तरवे॒दि मु॑त्तरवे॒दिम् पूर्वा॑ म॒ग्ने र॒ग्नेः पूर्वा॑ मुत्तरवे॒दि मु॑त्तरवे॒दिम् पूर्वा॑ म॒ग्नेः । \newline
46. उ॒त्त॒र॒वे॒दिमित्यु॑त्तर - वे॒दिम् । \newline
47. पूर्वा॑ म॒ग्ने र॒ग्नेः पूर्वा॒म् पूर्वा॑ म॒ग्नेर् व्याघा॑रयन्ति॒ व्याघा॑रयन् त्य॒ग्नेः पूर्वा॒म् पूर्वा॑ म॒ग्नेर् व्याघा॑रयन्ति । \newline
48. अ॒ग्नेर् व्याघा॑रयन्ति॒ व्याघा॑रयन् त्य॒ग्ने र॒ग्नेर् व्याघा॑रयन्ति॒ वारे॑वृतं॒ ॅवारे॑वृतं॒ ॅव्याघा॑रयन् त्य॒ग्ने र॒ग्नेर् व्याघा॑रयन्ति॒ वारे॑वृतम् । \newline
49. व्याघा॑रयन्ति॒ वारे॑वृतं॒ ॅवारे॑वृतं॒ ॅव्याघा॑रयन्ति॒ व्याघा॑रयन्ति॒ वारे॑वृतꣳ॒॒ हि हि वारे॑वृतं॒ ॅव्याघा॑रयन्ति॒ व्याघा॑रयन्ति॒ वारे॑वृतꣳ॒॒ हि । \newline
50. व्याघा॑रय॒न्तीति॑ वि - आघा॑रयन्ति । \newline
51. वारे॑वृतꣳ॒॒ हि हि वारे॑वृतं॒ ॅवारे॑वृतꣳ॒॒ ह्य॑स्या अस्यै॒ हि वारे॑वृतं॒ ॅवारे॑वृतꣳ॒॒ ह्य॑स्यै । \newline
52. वारे॑वृत॒मिति॒ वारे᳚ - वृ॒त॒म् । \newline
53. ह्य॑स्या अस्यै॒ हि ह्य॑स्यै॒ शम्य॑या॒ शम्य॑या ऽस्यै॒ हि ह्य॑स्यै॒ शम्य॑या । \newline
54. अ॒स्यै॒ शम्य॑या॒ शम्य॑या ऽस्या अस्यै॒ शम्य॑या॒ परि॒ परि॒ शम्य॑या ऽस्या अस्यै॒ शम्य॑या॒ परि॑ । \newline
55. शम्य॑या॒ परि॒ परि॒ शम्य॑या॒ शम्य॑या॒ परि॑ मिमीते मिमीते॒ परि॒ शम्य॑या॒ शम्य॑या॒ परि॑ मिमीते । \newline
56. परि॑ मिमीते मिमीते॒ परि॒ परि॑ मिमीते॒ मात्रा॒ मात्रा॑ मिमीते॒ परि॒ परि॑ मिमीते॒ मात्रा᳚ । \newline
57. मि॒मी॒ते॒ मात्रा॒ मात्रा॑ मिमीते मिमीते॒ मात्रै॒ वैव मात्रा॑ मिमीते मिमीते॒ मात्रै॒व । \newline
\pagebreak
\markright{ TS 6.2.7.2  \hfill https://www.vedavms.in \hfill}

\section{ TS 6.2.7.2 }

\textbf{TS 6.2.7.2 } \newline
\textbf{Samhita Paata} \newline

मात्रै॒वास्यै॒ साथो॑ यु॒क्तेनै॒व यु॒क्तमव॑ रुन्धे वि॒त्ताय॑नी मे॒ऽसीत्या॑ह वि॒त्ता ह्ये॑ना॒नाव॑त् ति॒क्ताय॑नी मे॒ऽसीत्या॑ह ति॒क्तान्. ह्ये॑ना॒नाव॒दव॑तान्मा नाथि॒तमित्या॑ह नाथि॒तान्. ह्ये॑ना॒नाव॒दव॑तान्मा व्यथि॒तमित्या॑ह व्यथि॒तान्. ह्ये॑ना॒नाव॑द्-वि॒देर॒ग्निर्नभो॒ नामा- [  ] \newline

\textbf{Pada Paata} \newline

मात्रा᳚ । ए॒व । अ॒स्यै॒ । सा । अथो॒ इति॑ । यु॒क्तेन॑ । ए॒व । यु॒क्तम् । अवेति॑ । रु॒न्धे॒ । वि॒त्ताय॒नीति॑ वित्त - अय॑नी । मे॒ । अ॒सि॒ । इति॑ । आ॒ह॒ । वि॒त्ता । हि । ए॒ना॒न् । आव॑त् । ति॒क्ताय॒नीति॑ तिक्त - अय॑नी । मे॒ । अ॒सि॒ । इति॑ । आ॒ह॒ । ति॒क्तान् । हि । ए॒ना॒न् । आव॑त् । अव॑तात् । मा॒ । ना॒थि॒तम् । इति॑ । आ॒ह॒ । ना॒थि॒तान् । हि । ए॒ना॒न् । आव॑त् । अव॑तात् । मा॒ । व्य॒थि॒तम् । इति॑ । आ॒ह॒ । व्य॒थि॒तान् । हि । ए॒ना॒न् । आव॑त् । वि॒देः । अ॒ग्निः । नभः॑ । नाम॑ ।  \newline


\textbf{Krama Paata} \newline

मात्रै॒व । ए॒वास्यै᳚ । अ॒स्यै॒ सा । साऽथो᳚ । अथो॑ यु॒क्तेन॑ । अथो॒ इत्यथो᳚ । यु॒क्तेनै॒व । ए॒व यु॒क्तम् । यु॒क्तमव॑ । अव॑ रुन्धे । रु॒न्धे॒ वि॒त्ताय॑नी । वि॒त्ताय॑नी मे । वि॒त्ताय॒नीति॑ वित्त - अय॑नी । मे॒ऽसि॒ । अ॒सीति॑ । इत्या॑ह । आ॒ह॒ वि॒त्ता । वि॒त्ता हि । ह्ये॑नान् । ए॒ना॒नाव॑त् । आव॑त् ति॒क्ताय॑नी । ति॒क्ताय॑नी मे । ति॒क्ताय॒नीति॑ तिक्त - अय॑नी । मे॒ऽसि॒ । अ॒सीति॑ । इत्या॑ह । आ॒ह॒ ति॒क्तान् । ति॒क्तान्. हि । ह्ये॑नान् । ए॒ना॒नाव॑त् । आव॒दव॑तात् । अव॑तान् मा । मा॒ ना॒थि॒तम् । ना॒थि॒तमिति॑ । इत्या॑ह । आ॒ह॒ ना॒थि॒तान् । ना॒थि॒तान्. हि । ह्ये॑नान् । ए॒ना॒नाव॑त् । आव॒दव॑तात् । अव॑तान् मा । मा॒ व्य॒थि॒तम् । व्य॒थि॒तमिति॑ । इत्या॑ह । आ॒ह॒ व्य॒थि॒तान् । व्य॒थि॒तान्. हि । ह्ये॑नान् । ए॒ना॒नाव॑त् । आव॑द् वि॒देः । वि॒देर॒ग्निः । 
अ॒ग्निर् नभः॑ । नभो॒ नाम॑ । नामाग्ने᳚ \newline

\textbf{Jatai Paata} \newline

1. मात्रै॒वैव मात्रा॒ मात्रै॒व । \newline
2. ए॒वास्या॑ अस्या ए॒वैवास्यै᳚ । \newline
3. अ॒स्यै॒ सा सा ऽस्या॑ अस्यै॒ सा । \newline
4. सा ऽथो॒ अथो॒ सा सा ऽथो᳚ । \newline
5. अथो॑ यु॒क्तेन॑ यु॒क्तेनाथो॒ अथो॑ यु॒क्तेन॑ । \newline
6. अथो॒ इत्यथो᳚ । \newline
7. यु॒क्ते नै॒वैव यु॒क्तेन॑ यु॒क्ते नै॒व । \newline
8. ए॒व यु॒क्तं ॅयु॒क्त मे॒वैव यु॒क्तम् । \newline
9. यु॒क्त मवाव॑ यु॒क्तं ॅयु॒क्त मव॑ । \newline
10. अव॑ रुन्धे रु॒न्धे ऽवाव॑ रुन्धे । \newline
11. रु॒न्धे॒ वि॒त्ताय॑नी वि॒त्ताय॑नी रुन्धे रुन्धे वि॒त्ताय॑नी । \newline
12. वि॒त्ताय॑नी मे मे वि॒त्ताय॑नी वि॒त्ताय॑नी मे । \newline
13. वि॒त्ताय॒नीति॑ वित्त - अय॑नी । \newline
14. मे॒ ऽस्य॒सि॒ मे॒ मे॒ ऽसि॒ । \newline
15. अ॒सीती त्य॑स्य॒सीति॑ । \newline
16. इत्या॑हा॒हे तीत्या॑ह । \newline
17. आ॒ह॒ वि॒त्ता वि॒त्ता ऽऽहा॑ह वि॒त्ता । \newline
18. वि॒त्ता हि हि वि॒त्ता वि॒त्ता हि । \newline
19. ह्ये॑ना नेना॒न्॒. हि ह्ये॑नान् । \newline
20. ए॒ना॒ नाव॒ दाव॑ देना नेना॒ नाव॑त् । \newline
21. आव॑त् ति॒क्ताय॑नी ति॒क्ताय॒ न्याव॒ दाव॑त् ति॒क्ताय॑नी । \newline
22. ति॒क्ताय॑नी मे मे ति॒क्ताय॑नी ति॒क्ताय॑नी मे । \newline
23. ति॒क्ताय॒नीति॑ तिक्त - अय॑नी । \newline
24. मे॒ ऽस्य॒सि॒ मे॒ मे॒ ऽसि॒ । \newline
25. अ॒सीती त्य॑स्य॒सीति॑ । \newline
26. इत्या॑हा॒हे तीत्या॑ह । \newline
27. आ॒ह॒ ति॒क्तान् ति॒क्ता ना॑हाह ति॒क्तान् । \newline
28. ति॒क्तान्. हि हि ति॒क्तान् ति॒क्तान्. हि । \newline
29. ह्ये॑ना नेना॒न्॒. हि ह्ये॑नान् । \newline
30. ए॒ना॒ नाव॒ दाव॑ देना नेना॒ नाव॑त् । \newline
31. आव॒ दव॑ता॒ दव॑ता॒ दाव॒ दाव॒ दव॑तात् । \newline
32. अव॑तान् मा॒ मा ऽव॑ता॒ दव॑तान् मा । \newline
33. मा॒ ना॒थि॒तम् ना॑थि॒तम् मा॑ मा नाथि॒तम् । \newline
34. ना॒थि॒त मितीति॑ नाथि॒तम् ना॑थि॒त मिति॑ । \newline
35. इत्या॑हा॒हे तीत्या॑ह । \newline
36. आ॒ह॒ ना॒थि॒तान् ना॑थि॒ता ना॑हाह नाथि॒तान् । \newline
37. ना॒थि॒तान्. हि हि ना॑थि॒तान् ना॑थि॒तान्. हि । \newline
38. ह्ये॑ना नेना॒न्॒. हि ह्ये॑नान् । \newline
39. ए॒ना॒ नाव॒ दाव॑ देना नेना॒ नाव॑त् । \newline
40. आव॒ दव॑ता॒ दव॑ता॒ दाव॒ दाव॒ दव॑तात् । \newline
41. अव॑तान् मा॒ मा ऽव॑ता॒ दव॑तान् मा । \newline
42. मा॒ व्य॒थि॒तं ॅव्य॑थि॒तम् मा॑ मा व्यथि॒तम् । \newline
43. व्य॒थि॒त मितीति॑ व्यथि॒तं ॅव्य॑थि॒त मिति॑ । \newline
44. इत्या॑हा॒हे तीत्या॑ह । \newline
45. आ॒ह॒ व्य॒थि॒तान् व्य॑थि॒ता ना॑हाह व्यथि॒तान् । \newline
46. व्य॒थि॒तान्. हि हि व्य॑थि॒तान् व्य॑थि॒तान्. हि । \newline
47. ह्ये॑ना नेना॒न्॒. हि ह्ये॑नान् । \newline
48. ए॒ना॒ नाव॒ दाव॑ देना नेना॒ नाव॑त् । \newline
49. आव॑द् वि॒देर् वि॒दे राव॒ दाव॑द् वि॒देः । \newline
50. वि॒दे र॒ग्नि र॒ग्निर् वि॒देर् वि॒दे र॒ग्निः । \newline
51. अ॒ग्निर् नभो॒ नभो॒ ऽग्नि र॒ग्निर् नभः॑ । \newline
52. नभो॒ नाम॒ नाम॒ नभो॒ नभो॒ नाम॑ । \newline
53. नामाग्ने ऽग्ने॒ नाम॒ नामाग्ने᳚ । \newline

\textbf{Ghana Paata } \newline

1. मात्रै॒ वैव मात्रा॒ मात्रै॒ वास्या॑ अस्या ए॒व मात्रा॒ मात्रै॒ वास्यै᳚ । \newline
2. ए॒वास्या॑ अस्या ए॒वै वास्यै॒ सा सा ऽस्या॑ ए॒वै वास्यै॒ सा । \newline
3. अ॒स्यै॒ सा सा ऽस्या॑ अस्यै॒ सा ऽथो॒ अथो॒ सा ऽस्या॑ अस्यै॒ सा ऽथो᳚ । \newline
4. सा ऽथो॒ अथो॒ सा सा ऽथो॑ यु॒क्तेन॑ यु॒क्ते नाथो॒ सा सा ऽथो॑ यु॒क्तेन॑ । \newline
5. अथो॑ यु॒क्तेन॑ यु॒क्ते नाथो॒ अथो॑ यु॒क्तेनै॒ वैव यु॒क्ते नाथो॒ अथो॑ यु॒क्तेनै॒व । \newline
6. अथो॒ इत्यथो᳚ । \newline
7. यु॒क्तेनै॒ वैव यु॒क्तेन॑ यु॒क्तेनै॒व यु॒क्तं ॅयु॒क्त मे॒व यु॒क्तेन॑ यु॒क्तेनै॒व यु॒क्तम् । \newline
8. ए॒व यु॒क्तं ॅयु॒क्त मे॒वैव यु॒क्त मवाव॑ यु॒क्त मे॒वैव यु॒क्त मव॑ । \newline
9. यु॒क्त मवाव॑ यु॒क्तं ॅयु॒क्त मव॑ रुन्धे रु॒न्धे ऽव॑ यु॒क्तं ॅयु॒क्त मव॑ रुन्धे । \newline
10. अव॑ रुन्धे रु॒न्धे ऽवाव॑ रुन्धे वि॒त्ताय॑नी वि॒त्ताय॑नी रु॒न्धे ऽवाव॑ रुन्धे वि॒त्ताय॑नी । \newline
11. रु॒न्धे॒ वि॒त्ताय॑नी वि॒त्ताय॑नी रुन्धे रुन्धे वि॒त्ताय॑नी मे मे वि॒त्ताय॑नी रुन्धे रुन्धे वि॒त्ताय॑नी मे । \newline
12. वि॒त्ताय॑नी मे मे वि॒त्ताय॑नी वि॒त्ताय॑नी मे ऽस्यसि मे वि॒त्ताय॑नी वि॒त्ताय॑नी मे ऽसि । \newline
13. वि॒त्ताय॒नीति॑ वित्त - अय॑नी । \newline
14. मे॒ ऽस्य॒सि॒ मे॒ मे॒ ऽसीती त्य॑सि मे मे॒ ऽसीति॑ । \newline
15. अ॒सीती त्य॑स्य॒ सीत्या॑ हा॒हे त्य॑स्य॒ सीत्या॑ह । \newline
16. इत्या॑हा॒हे तीत्या॑ह वि॒त्ता वि॒त्ता ऽऽहे तीत्या॑ह वि॒त्ता । \newline
17. आ॒ह॒ वि॒त्ता वि॒त्ता ऽऽहा॑ह वि॒त्ता हि हि वि॒त्ता ऽऽहा॑ह वि॒त्ता हि । \newline
18. वि॒त्ता हि हि वि॒त्ता वि॒त्ता ह्ये॑ना नेना॒न्॒. हि वि॒त्ता वि॒त्ता ह्ये॑नान् । \newline
19. ह्ये॑ना नेना॒न्॒. हि ह्ये॑ना॒ नाव॒ दाव॑ देना॒न्॒. हि ह्ये॑ना॒ नाव॑त् । \newline
20. ए॒ना॒ नाव॒ दाव॑ देना नेना॒ नाव॑त् ति॒क्ताय॑नी ति॒क्ताय॒ न्याव॑ देना नेना॒ नाव॑त् ति॒क्ताय॑नी । \newline
21. आव॑त् ति॒क्ताय॑नी ति॒क्ताय॒ न्याव॒ दाव॑त् ति॒क्ताय॑नी मे मे ति॒क्ताय॒ न्याव॒ दाव॑त् ति॒क्ताय॑नी मे । \newline
22. ति॒क्ताय॑नी मे मे ति॒क्ताय॑नी ति॒क्ताय॑नी मे ऽस्यसि मे ति॒क्ताय॑नी ति॒क्ताय॑नी मे ऽसि । \newline
23. ति॒क्ताय॒नीति॑ तिक्त - अय॑नी । \newline
24. मे॒ ऽस्य॒सि॒ मे॒ मे॒ ऽसीती त्य॑सि मे मे॒ ऽसीति॑ । \newline
25. अ॒सी तीत्य॑ स्य॒सीत्या॑ हा॒हे त्य॑स्य॒ सीत्या॑ह । \newline
26. इत्या॑हा॒हे तीत्या॑ह ति॒क्तान् ति॒क्ता ना॒हे तीत्या॑ह ति॒क्तान् । \newline
27. आ॒ह॒ ति॒क्तान् ति॒क्ता ना॑हाह ति॒क्तान्. हि हि ति॒क्ता ना॑हाह ति॒क्तान्. हि । \newline
28. ति॒क्तान्. हि हि ति॒क्तान् ति॒क्तान् ह्ये॑ना नेना॒न्॒. हि ति॒क्तान् ति॒क्तान् ह्ये॑नान् । \newline
29. ह्ये॑ना नेना॒न्॒. हि ह्ये॑ना॒ नाव॒ दाव॑ देना॒न्॒. हि ह्ये॑ना॒ नाव॑त् । \newline
30. ए॒ना॒ नाव॒ दाव॑ देना नेना॒ नाव॒ दव॑ता॒ दव॑ता॒ दाव॑ देना नेना॒ नाव॒ दव॑तात् । \newline
31. आव॒ दव॑ता॒ दव॑ता॒ दाव॒ दाव॒ दव॑तान् मा॒ मा ऽव॑ता॒ दाव॒ दाव॒ दव॑तान् मा । \newline
32. अव॑तान् मा॒ मा ऽव॑ता॒ दव॑तान् मा नाथि॒तम् ना॑थि॒तम् मा ऽव॑ता॒ दव॑तान् मा नाथि॒तम् । \newline
33. मा॒ ना॒थि॒तम् ना॑थि॒तम् मा॑ मा नाथि॒त मितीति॑ नाथि॒तम् मा॑ मा नाथि॒त मिति॑ । \newline
34. ना॒थि॒त मितीति॑ नाथि॒तम् ना॑थि॒त मित्या॑हा॒हेति॑ नाथि॒तम् ना॑थि॒त मित्या॑ह । \newline
35. इत्या॑हा॒हे तीत्या॑ह नाथि॒तान् ना॑थि॒ता ना॒हे तीत्या॑ह नाथि॒तान् । \newline
36. आ॒ह॒ ना॒थि॒तान् ना॑थि॒ता ना॑हाह नाथि॒तान्. हि हि ना॑थि॒ता ना॑हाह नाथि॒तान्. हि । \newline
37. ना॒थि॒तान्. हि हि ना॑थि॒तान् ना॑थि॒तान् ह्ये॑ना नेना॒न्॒. हि ना॑थि॒तान् ना॑थि॒तान् ह्ये॑नान् । \newline
38. ह्ये॑ना नेना॒न्॒. हि ह्ये॑ना॒ नाव॒ दाव॑ देना॒न्॒. हि ह्ये॑ना॒ नाव॑त् । \newline
39. ए॒ना॒ नाव॒ दाव॑ देना नेना॒ नाव॒ दव॑ता॒ दव॑ता॒ दाव॑ देना नेना॒ नाव॒ दव॑तात् । \newline
40. आव॒ दव॑ता॒ दव॑ता॒ दाव॒ दाव॒ दव॑तान् मा॒ मा ऽव॑ता॒ दाव॒ दाव॒ दव॑तान् मा । \newline
41. अव॑तान् मा॒ मा ऽव॑ता॒ दव॑तान् मा व्यथि॒तं ॅव्य॑थि॒तम् मा ऽव॑ता॒ दव॑तान् मा व्यथि॒तम् । \newline
42. मा॒ व्य॒थि॒तं ॅव्य॑थि॒तम् मा॑ मा व्यथि॒त मितीति॑ व्यथि॒तम् मा॑ मा व्यथि॒त मिति॑ । \newline
43. व्य॒थि॒त मितीति॑ व्यथि॒तं ॅव्य॑थि॒त मित्या॑हा॒हेति॑ व्यथि॒तं ॅव्य॑थि॒त मित्या॑ह । \newline
44. इत्या॑हा॒हे तीत्या॑ह व्यथि॒तान् व्य॑थि॒ता ना॒हे तीत्या॑ह व्यथि॒तान् । \newline
45. आ॒ह॒ व्य॒थि॒तान् व्य॑थि॒ता ना॑हाह व्यथि॒तान्. हि हि व्य॑थि॒ता ना॑हाह व्यथि॒तान्. हि । \newline
46. व्य॒थि॒तान्. हि हि व्य॑थि॒तान् व्य॑थि॒तान् ह्ये॑ना नेना॒न्॒. हि व्य॑थि॒तान् व्य॑थि॒तान् ह्ये॑नान् । \newline
47. ह्ये॑ना नेना॒न्॒. हि ह्ये॑ना॒ नाव॒ दाव॑ देना॒न्॒. हि ह्ये॑ना॒ नाव॑त् । \newline
48. ए॒ना॒ नाव॒ दाव॑ देना नेना॒ नाव॑द् वि॒देर् वि॒दे राव॑ देना नेना॒ नाव॑द् वि॒देः । \newline
49. आव॑द् वि॒देर् वि॒दे राव॒ दाव॑द् वि॒दे र॒ग्नि र॒ग्निर् वि॒दे राव॒ दाव॑द् वि॒दे र॒ग्निः । \newline
50. वि॒दे र॒ग्नि र॒ग्निर् वि॒देर् वि॒दे र॒ग्निर् नभो॒ नभो॒ ऽग्निर् वि॒देर् वि॒दे र॒ग्निर् नभः॑ । \newline
51. अ॒ग्निर् नभो॒ नभो॒ ऽग्नि र॒ग्निर् नभो॒ नाम॒ नाम॒ नभो॒ ऽग्नि र॒ग्निर् नभो॒ नाम॑ । \newline
52. नभो॒ नाम॒ नाम॒ नभो॒ नभो॒ नामाग्ने ऽग्ने॒ नाम॒ नभो॒ नभो॒ नामाग्ने᳚ । \newline
53. नामाग्ने ऽग्ने॒ नाम॒ नामाग्ने॑ अङ्गिरो अङ्गि॒रो ऽग्ने॒ नाम॒ नामाग्ने॑ अङ्गिरः । \newline
\pagebreak
\markright{ TS 6.2.7.3  \hfill https://www.vedavms.in \hfill}

\section{ TS 6.2.7.3 }

\textbf{TS 6.2.7.3 } \newline
\textbf{Samhita Paata} \newline

ग्ने॑ अङ्गिर॒ इति॒ त्रिर्.ह॑रति॒ य ए॒वैषु लो॒केष्व॒ग्नय॒-स्ताने॒वाव॑ रुन्धे तू॒ष्णीं च॑तु॒र्त्थꣳ ह॑र॒त्यनि॑-रुक्तमे॒वाव॑ रुन्धे सिꣳ॒॒हीर॑सि महि॒षीर॒सीत्या॑ह सिꣳ॒॒हीर्ह्ये॑षा रू॒पं कृ॒त्वोभया॑-नन्त॒रा ऽप॒क्रम्याति॑ष्ठदु॒रु प्र॑थस्वो॒रु ते॑ य॒ज्ञ्प॑तिः प्रथता॒मित्या॑ह॒ यज॑मानमे॒व प्र॒जया॑ प॒शुभिः॑ प्रथयति ध्रु॒वा- [  ] \newline

\textbf{Pada Paata} \newline

अग्ने᳚ । अ॒ङ्गि॒रः॒ । इति॑ । त्रिः । ह॒र॒ति॒ । ये । ए॒व । ए॒षु । लो॒केषु॑ । अ॒ग्नयः॑ । तान् । ए॒व । अवेति॑ । रु॒न्धे॒ । तू॒ष्णीम् । च॒तु॒र्थम् । ह॒र॒ति॒ । अनि॑रुक्त॒मित्यनिः॑ - उ॒क्त॒म् । ए॒व । अवेति॑ । रु॒न्धे॒ । सिꣳ॒॒हीः । अ॒सि॒ । म॒हि॒षीः । अ॒सि॒ । इति॑ । आ॒ह॒ । सिꣳ॒॒हीः । हि । ए॒षा । रू॒पम् । कृ॒त्वा । उ॒भयान्॑ । अ॒न्त॒रा । अ॒प॒क्रम्येत्य॑प - क्रम्य॑ । अति॑ष्ठत् । उ॒रु । प्र॒थ॒स्व॒ । उ॒रु । ते॒ । य॒ज्ञ्प॑ति॒रिति॑ य॒ज्ञ् - प॒तिः॒ । प्र॒थ॒ता॒म् । इति॑ । आ॒ह॒ । यज॑मानम् । ए॒व । प्र॒जयेति॑ प्र - जया᳚ । प॒शुभि॒रिति॑ प॒शु - भिः॒ । प्र॒थ॒य॒ति॒ । ध्रु॒वा ।  \newline


\textbf{Krama Paata} \newline

अग्ने॑ अङ्‍गिरः । अ॒ङ्‍गि॒र॒ इति॑ । इति॒ त्रिः । त्रिर्. ह॑रति । ह॒र॒ति॒ ये । य ए॒व । ए॒वैषु । ए॒षु लो॒केषु॑ । लो॒केष्व॒ग्नयः॑ । अ॒ग्नय॒स्तान् । ताने॒व । ए॒वाव॑ । अव॑ रुन्धे । रु॒न्धे॒ तू॒ष्णीम् । तू॒ष्णीम् च॑तु॒र्थम् । च॒तु॒र्थꣳ ह॑रति । ह॒र॒त्यनि॑रुक्तम् । अनि॑रुक्तमे॒व । अनि॑रुक्त॒मित्यनिः॑ - उ॒क्त॒म् । ए॒वाव॑ । अव॑ रुन्धे । रु॒न्धे॒ सिꣳ॒॒हीः । सिꣳ॒॒हीर॑सि । अ॒सि॒ म॒हि॒षीः । म॒हि॒षीर॑सि । अ॒सीति॑ । इत्या॑ह । आ॒ह॒ सिꣳ॒॒हीः । सिꣳ॒॒हीर्. हि । ह्ये॑षा । ए॒षा रू॒पम् । रू॒पम् कृ॒त्वा । कृ॒त्वोभयान्॑ । उ॒भया॑नन्त॒रा । अ॒न्त॒राऽप॒क्रम्य॑ । अ॒प॒क्रम्याति॑ष्ठत् । अ॒प॒क्रम्येत्य॑प - क्रम्य॑ । अति॑ष्ठदु॒रु । उ॒रु प्र॑थस्व । प्र॒थ॒स्वो॒रु । उ॒रु ते᳚ । ते॒ य॒ज्ञ्प॑तिः । य॒ज्ञ्प॑तिः प्रथताम् । य॒ज्ञ्प॑ति॒रिति॑ य॒ज्ञ् - प॒तिः॒ । प्र॒थ॒ता॒मिति॑ । इत्या॑ह । आ॒ह॒ यज॑मानम् । यज॑मानमे॒व । ए॒व प्र॒जया᳚ । प्र॒जया॑ प॒शुभिः॑ । प्र॒जयेति॑ प्र - जया᳚ । प॒शुभिः॑ प्रथयति । प॒शुभि॒रिति॑ प॒शु - भिः॒ । प्र॒थ॒य॒ति॒ ध्रु॒वा । ध्रु॒वाऽसि॑ \newline

\textbf{Jatai Paata} \newline

1. अग्ने॑ अङ्गिरो अङ्गि॒रो ऽग्ने ऽग्ने॑ अङ्गिरः । \newline
2. अ॒ङ्गि॒र॒ इती त्य॑ङ्गिरो अङ्गिर॒ इति॑ । \newline
3. इति॒ त्रि स्त्रि रितीति॒ त्रिः । \newline
4. त्रिर्. ह॑रति हरति॒ त्रि स्त्रिर्. ह॑रति । \newline
5. ह॒र॒ति॒ ये ये ह॑रति हरति॒ ये । \newline
6. य ए॒वैव ये य ए॒व । \newline
7. ए॒वै ष्वे᳚(1॒)ष्वे॑ वैवैषु । \newline
8. ए॒षु लो॒केषु॑ लो॒के ष्वे॒ष्वे॑षु लो॒केषु॑ । \newline
9. लो॒के ष्व॒ग्नयो॒ ऽग्नयो॑ लो॒केषु॑ लो॒के ष्व॒ग्नयः॑ । \newline
10. अ॒ग्नय॒ स्ताꣳ स्ता न॒ग्नयो॒ ऽग्नय॒ स्तान् । \newline
11. ता ने॒वैव ताꣳ स्ता ने॒व । \newline
12. ए॒वावा वै॒वै वाव॑ । \newline
13. अव॑ रुन्धे रु॒न्धे ऽवाव॑ रुन्धे । \newline
14. रु॒न्धे॒ तू॒ष्णीम् तू॒ष्णीꣳ रु॑न्धे रुन्धे तू॒ष्णीम् । \newline
15. तू॒ष्णीम् च॑तु॒र्थम् च॑तु॒र्थम् तू॒ष्णीम् तू॒ष्णीम् च॑तु॒र्थम् । \newline
16. च॒तु॒र्थꣳ ह॑रति हरति चतु॒र्थम् च॑तु॒र्थꣳ ह॑रति । \newline
17. ह॒र॒ त्यनि॑रुक्त॒ मनि॑रुक्तꣳ हरति हर॒ त्यनि॑रुक्तम् । \newline
18. अनि॑रुक्त मे॒वैवा नि॑रुक्त॒ मनि॑रुक्त मे॒व । \newline
19. अनि॑रुक्त॒मित्यनिः॑ - उ॒क्त॒म् । \newline
20. ए॒वावा वै॒वै वाव॑ । \newline
21. अव॑ रुन्धे रु॒न्धे ऽवाव॑ रुन्धे । \newline
22. रु॒न्धे॒ सिꣳ॒॒हीः सिꣳ॒॒ही रु॑न्धे रुन्धे सिꣳ॒॒हीः । \newline
23. सिꣳ॒॒ही र॑स्यसि सिꣳ॒॒हीः सिꣳ॒॒ही र॑सि । \newline
24. अ॒सि॒ म॒हि॒षीर् म॑हि॒षी र॑स्यसि महि॒षीः । \newline
25. म॒हि॒षी र॑स्यसि महि॒षीर् म॑हि॒षी र॑सि । \newline
26. अ॒सीती त्य॑स्य॒सीति॑ । \newline
27. इत्या॑हा॒हे तीत्या॑ह । \newline
28. आ॒ह॒ सिꣳ॒॒हीः सिꣳ॒॒ही रा॑हाह सिꣳ॒॒हीः । \newline
29. सिꣳ॒॒हीर्. हि हि सिꣳ॒॒हीः सिꣳ॒॒हीर्. हि । \newline
30. ह्ये॑षैषा हि ह्ये॑षा । \newline
31. ए॒षा रू॒पꣳ रू॒प मे॒षैषा रू॒पम् । \newline
32. रू॒पम् कृ॒त्वा कृ॒त्वा रू॒पꣳ रू॒पम् कृ॒त्वा । \newline
33. कृ॒त्वोभया॑ नु॒भया᳚न् कृ॒त्वा कृ॒त्वोभयान्॑ । \newline
34. उ॒भया॑ नन्त॒रा ऽन्त॒रोभया॑ नु॒भया॑ नन्त॒रा । \newline
35. अ॒न्त॒रा ऽप॒क्रम्या॑ प॒क्रम्या᳚ न्त॒रा ऽन्त॒रा ऽप॒क्रम्य॑ । \newline
36. अ॒प॒क्रम्या ति॑ष्ठ॒ दति॑ष्ठ दप॒क्रम्या॑प् अ॒क्रम्या ति॑ष्ठत् । \newline
37. अ॒प॒क्रम्येत्य॑प - क्रम्य॑ । \newline
38. अति॑ष्ठ दु॒रू᳚र्व ति॑ष्ठ॒ दति॑ष्ठ दु॒रु । \newline
39. उ॒रु प्र॑थस्व प्रथस्वो॒रू॑रु प्र॑थस्व । \newline
40. प्र॒थ॒स्वो॒रू॑रु प्र॑थस्व प्रथस्वो॒रु । \newline
41. उ॒रु ते॑ त उ॒रू॑रु ते᳚ । \newline
42. ते॒ य॒ज्ञ्प॑तिर् य॒ज्ञ्प॑ति स्ते ते य॒ज्ञ्प॑तिः । \newline
43. य॒ज्ञ्प॑तिः प्रथताम् प्रथतां ॅय॒ज्ञ्प॑तिर् य॒ज्ञ्प॑तिः प्रथताम् । \newline
44. य॒ज्ञ्प॑ति॒रिति॑ य॒ज्ञ् - प॒तिः॒ । \newline
45. प्र॒थ॒ता॒ मितीति॑ प्रथताम् प्रथता॒ मिति॑ । \newline
46. इत्या॑हा॒हे तीत्या॑ह । \newline
47. आ॒ह॒ यज॑मानं॒ ॅयज॑मान माहाह॒ यज॑मानम् । \newline
48. यज॑मान मे॒वैव यज॑मानं॒ ॅयज॑मान मे॒व । \newline
49. ए॒व प्र॒जया᳚ प्र॒ज यै॒वैव प्र॒जया᳚ । \newline
50. प्र॒जया॑ प॒शुभिः॑ प॒शुभिः॑ प्र॒जया᳚ प्र॒जया॑ प॒शुभिः॑ । \newline
51. प्र॒जयेति॑ प्र - जया᳚ । \newline
52. प॒शुभिः॑ प्रथयति प्रथयति प॒शुभिः॑ प॒शुभिः॑ प्रथयति । \newline
53. प॒शुभि॒रिति॑ प॒शु - भिः॒ । \newline
54. प्र॒थ॒य॒ति॒ ध्रु॒वा ध्रु॒वा प्र॑थयति प्रथयति ध्रु॒वा । \newline
55. ध्रु॒वा ऽस्य॑सि ध्रु॒वा ध्रु॒वा ऽसि॑ । \newline

\textbf{Ghana Paata } \newline

1. अग्ने॑ अङ्गिरो अङ्गि॒रो ऽग्ने ऽग्ने॑ अङ्गिर॒ इती त्य॑ङ्गि॒रो ऽग्ने ऽग्ने॑ अङ्गिर॒ इति॑ । \newline
2. अ॒ङ्गि॒र॒ इती त्य॑ङ्गिरो अङ्गिर॒ इति॒ त्रि स्त्रि रित्य॑ङ्गिरो अङ्गिर॒ इति॒ त्रिः । \newline
3. इति॒ त्रि स्त्रि रितीति॒ त्रिर्. ह॑रति हरति॒ त्रि रितीति॒ त्रिर्. ह॑रति । \newline
4. त्रिर्. ह॑रति हरति॒ त्रि स्त्रिर्. ह॑रति॒ ये ये ह॑रति॒ त्रि स्त्रिर्. ह॑रति॒ ये । \newline
5. ह॒र॒ति॒ ये ये ह॑रति हरति॒ य ए॒वैव ये ह॑रति हरति॒ य ए॒व । \newline
6. य ए॒वैव ये य ए॒वैष्वे᳚(1॒)ष्वे॑व ये य ए॒वैषु । \newline
7. ए॒वैष्वे᳚(1॒)ष्वे॑ वैवैषु लो॒केषु॑ लो॒के ष्वे॒ ष्वे॑ वैवैषु लो॒केषु॑ । \newline
8. ए॒षु लो॒केषु॑ लो॒केष्वे॒ ष्वे॑षु लो॒के ष्व॒ग्नयो॒ ऽग्नयो॑ लो॒केष्वे॒ ष्वे॑षु लो॒के ष्व॒ग्नयः॑ । \newline
9. लो॒के ष्व॒ग्नयो॒ ऽग्नयो॑ लो॒केषु॑ लो॒के ष्व॒ग्नय॒ स्ताꣳ स्ता न॒ग्नयो॑ लो॒केषु॑ लो॒के ष्व॒ग्नय॒ स्तान् । \newline
10. अ॒ग्नय॒ स्ताꣳ स्ता न॒ग्नयो॒ ऽग्नय॒ स्ता ने॒वैव ता न॒ग्नयो॒ ऽग्नय॒ स्ताने॒व । \newline
11. ताने॒ वैव ताꣳ स्ता ने॒वावा वै॒व ताꣳ स्ता ने॒वाव॑ । \newline
12. ए॒वावा वै॒वै वाव॑ रुन्धे रु॒न्धे ऽवै॒वै वाव॑ रुन्धे । \newline
13. अव॑ रुन्धे रु॒न्धे ऽवाव॑ रुन्धे तू॒ष्णीम् तू॒ष्णीꣳ रु॒न्धे ऽवाव॑ रुन्धे तू॒ष्णीम् । \newline
14. रु॒न्धे॒ तू॒ष्णीम् तू॒ष्णीꣳ रु॑न्धे रुन्धे तू॒ष्णीम् च॑तु॒र्थम् च॑तु॒र्थम् तू॒ष्णीꣳ रु॑न्धे रुन्धे तू॒ष्णीम् च॑तु॒र्थम् । \newline
15. तू॒ष्णीम् च॑तु॒र्थम् च॑तु॒र्थम् तू॒ष्णीम् तू॒ष्णीम् च॑तु॒र्थꣳ ह॑रति हरति चतु॒र्थम् तू॒ष्णीम् तू॒ष्णीम् च॑तु॒र्थꣳ ह॑रति । \newline
16. च॒तु॒र्थꣳ ह॑रति हरति चतु॒र्थम् च॑तु॒र्थꣳ ह॑र॒ त्यनि॑रुक्त॒ मनि॑रुक्तꣳ हरति चतु॒र्थम् च॑तु॒र्थꣳ ह॑र॒ त्यनि॑रुक्तम् । \newline
17. ह॒र॒ त्यनि॑रुक्त॒ मनि॑रुक्तꣳ हरति हर॒ त्यनि॑रुक्त मे॒वै वानि॑रुक्तꣳ हरति हर॒ त्यनि॑रुक्त मे॒व । \newline
18. अनि॑रुक्त मे॒वै वानि॑रुक्त॒ मनि॑रुक्त मे॒वा वावै॒ वानि॑रुक्त॒ मनि॑रुक्त मे॒वाव॑ । \newline
19. अनि॑रुक्त॒मित्यनिः॑ - उ॒क्त॒म् । \newline
20. ए॒वावा वै॒वै वाव॑ रुन्धे रु॒न्धे ऽवै॒वै वाव॑ रुन्धे । \newline
21. अव॑ रुन्धे रु॒न्धे ऽवाव॑ रुन्धे सिꣳ॒॒हीः सिꣳ॒॒ही रु॒न्धे ऽवाव॑ रुन्धे सिꣳ॒॒हीः । \newline
22. रु॒न्धे॒ सिꣳ॒॒हीः सिꣳ॒॒ही रु॑न्धे रुन्धे सिꣳ॒॒ही र॑स्यसि सिꣳ॒॒ही रु॑न्धे रुन्धे सिꣳ॒॒ही र॑सि । \newline
23. सिꣳ॒॒ही र॑स्यसि सिꣳ॒॒हीः सिꣳ॒॒ही र॑सि महि॒षीर् म॑हि॒षी र॑सि सिꣳ॒॒हीः सिꣳ॒॒ही र॑सि महि॒षीः । \newline
24. अ॒सि॒ म॒हि॒षीर् म॑हि॒षी र॑स्यसि महि॒षी र॑स्यसि महि॒षी र॑स्यसि महि॒षी र॑सि । \newline
25. म॒हि॒षी र॑स्यसि महि॒षीर् म॑हि॒षी र॒सीती त्य॑सि महि॒षीर् म॑हि॒षी र॒सीति॑ । \newline
26. अ॒सीती त्य॑स्य॒ सीत्या॑हा॒हे त्य॑स्य॒ सीत्या॑ह । \newline
27. इत्या॑हा॒हे तीत्या॑ह सिꣳ॒॒हीः सिꣳ॒॒ही रा॒हे तीत्या॑ह सिꣳ॒॒हीः । \newline
28. आ॒ह॒ सिꣳ॒॒हीः सिꣳ॒॒ही रा॑हाह सिꣳ॒॒हीर्. हि हि सिꣳ॒॒ही रा॑हाह सिꣳ॒॒हीर्. हि । \newline
29. सिꣳ॒॒हीर्. हि हि सिꣳ॒॒हीः सिꣳ॒॒हीर् ह्ये॑षैषा हि सिꣳ॒॒हीः सिꣳ॒॒हीर् ह्ये॑षा । \newline
30. ह्ये॑षैषा हि ह्ये॑षा रू॒पꣳ रू॒प मे॒षा हि ह्ये॑षा रू॒पम् । \newline
31. ए॒षा रू॒पꣳ रू॒प मे॒षैषा रू॒पम् कृ॒त्वा कृ॒त्वा रू॒प मे॒षैषा रू॒पम् कृ॒त्वा । \newline
32. रू॒पम् कृ॒त्वा कृ॒त्वा रू॒पꣳ रू॒पम् कृ॒त्वो भया॑ नु॒भया᳚न् कृ॒त्वा रू॒पꣳ रू॒पम् कृ॒त्वोभयान्॑ । \newline
33. कृ॒त्वोभया॑ नु॒भया᳚न् कृ॒त्वा कृ॒त्वोभया॑ नन्त॒रा ऽन्त॒रो भया᳚न् कृ॒त्वा कृ॒त्वोभया॑ नन्त॒रा । \newline
34. उ॒भया॑ नन्त॒रा ऽन्त॒रो भया॑ नु॒भया॑ नन्त॒रा ऽप॒क्रम्या॑ प॒क्रम्या᳚ न्त॒रोभया॑ नु॒भया॑ नन्त॒रा ऽप॒क्रम्य॑ । \newline
35. अ॒न्त॒रा ऽप॒क्रम्या॑ प॒क्रम्या᳚ न्त॒रा ऽन्त॒रा ऽप॒क्रम्या ति॑ष्ठ॒ दति॑ष्ठ दप॒क्रम्या᳚न्त॒रा ऽन्त॒रा ऽप॒क्रम्या ति॑ष्ठत् । \newline
36. अ॒प॒क्रम्या ति॑ष्ठ॒ दति॑ष्ठ दप॒क्रम्या॑ प॒क्रम्या ति॑ष्ठ दु॒रू᳚र्व ति॑ष्ठ दप॒क्रम्या॑ प॒क्रम्या ति॑ष्ठ दु॒रु । \newline
37. अ॒प॒क्रम्येत्य॑प - क्रम्य॑ । \newline
38. अति॑ष्ठ दु॒रू᳚र्व ति॑ष्ठ॒ दति॑ष्ठ दु॒रु प्र॑थस्व प्रथ स्वो॒र्वति॑ष्ठ॒ दति॑ष्ठ दु॒रु प्र॑थस्व । \newline
39. उ॒रु प्र॑थस्व प्रथ स्वो॒रू॑रु प्र॑थ स्वो॒रू॑रु प्र॑थ स्वो॒रू॑रु प्र॑थ स्वो॒रु । \newline
40. प्र॒थ॒ स्वो॒रू॑रु प्र॑थस्व प्रथ स्वो॒रु ते॑ त उ॒रु प्र॑थस्व प्रथ स्वो॒रु ते᳚ । \newline
41. उ॒रु ते॑ त उ॒रू॑रु ते॑ य॒ज्ञ्प॑तिर् य॒ज्ञ्प॑ति स्त उ॒रू॑रु ते॑ य॒ज्ञ्प॑तिः । \newline
42. ते॒ य॒ज्ञ्प॑तिर् य॒ज्ञ्प॑ति स्ते ते य॒ज्ञ्प॑तिः प्रथताम् प्रथतां ॅय॒ज्ञ्प॑ति स्ते ते य॒ज्ञ्प॑तिः प्रथताम् । \newline
43. य॒ज्ञ्प॑तिः प्रथताम् प्रथतां ॅय॒ज्ञ्प॑तिर् य॒ज्ञ्प॑तिः प्रथता॒ मितीति॑ प्रथतां ॅय॒ज्ञ्प॑तिर् य॒ज्ञ्प॑तिः प्रथता॒ मिति॑ । \newline
44. य॒ज्ञ्प॑ति॒रिति॑ य॒ज्ञ् - प॒तिः॒ । \newline
45. प्र॒थ॒ता॒ मितीति॑ प्रथताम् प्रथता॒ मित्या॑ हा॒हेति॑ प्रथताम् प्रथता॒ मित्या॑ह । \newline
46. इत्या॑हा॒हे तीत्या॑ह॒ यज॑मानं॒ ॅयज॑मान मा॒हे तीत्या॑ह॒ यज॑मानम् । \newline
47. आ॒ह॒ यज॑मानं॒ ॅयज॑मान माहाह॒ यज॑मान मे॒वैव यज॑मान माहाह॒ यज॑मान मे॒व । \newline
48. यज॑मान मे॒वैव यज॑मानं॒ ॅयज॑मान मे॒व प्र॒जया᳚ प्र॒जयै॒व यज॑मानं॒ ॅयज॑मान मे॒व प्र॒जया᳚ । \newline
49. ए॒व प्र॒जया᳚ प्र॒जयै॒ वैव प्र॒जया॑ प॒शुभिः॑ प॒शुभिः॑ प्र॒जयै॒ वैव प्र॒जया॑ प॒शुभिः॑ । \newline
50. प्र॒जया॑ प॒शुभिः॑ प॒शुभिः॑ प्र॒जया᳚ प्र॒जया॑ प॒शुभिः॑ प्रथयति प्रथयति प॒शुभिः॑ प्र॒जया᳚ प्र॒जया॑ प॒शुभिः॑ प्रथयति । \newline
51. प्र॒जयेति॑ प्र - जया᳚ । \newline
52. प॒शुभिः॑ प्रथयति प्रथयति प॒शुभिः॑ प॒शुभिः॑ प्रथयति ध्रु॒वा ध्रु॒वा प्र॑थयति प॒शुभिः॑ प॒शुभिः॑ प्रथयति ध्रु॒वा । \newline
53. प॒शुभि॒रिति॑ प॒शु - भिः॒ । \newline
54. प्र॒थ॒य॒ति॒ ध्रु॒वा ध्रु॒वा प्र॑थयति प्रथयति ध्रु॒वा ऽस्य॑सि ध्रु॒वा प्र॑थयति प्रथयति ध्रु॒वा ऽसि॑ । \newline
55. ध्रु॒वा ऽस्य॑सि ध्रु॒वा ध्रु॒वा ऽसीती त्य॑सि ध्रु॒वा ध्रु॒वा ऽसीति॑ । \newline
\pagebreak
\markright{ TS 6.2.7.4  \hfill https://www.vedavms.in \hfill}

\section{ TS 6.2.7.4 }

\textbf{TS 6.2.7.4 } \newline
\textbf{Samhita Paata} \newline

ऽसीति॒ सꣳ ह॑न्ति॒ धृत्यै॑ दे॒वेभ्यः॑ शुन्धस्व दे॒वेभ्यः॑ शुंभ॒स्वेत्यव॑ चो॒क्षति॒ प्र च॑ किरति॒ शुद्ध्या॑ इन्द्रघो॒षस्त्वा॒ वसु॑भिः पु॒रस्ता᳚त् पा॒त्वित्या॑ह दि॒ग्भ्य ए॒वैनां॒ प्रोक्ष॑ति दे॒वाꣳश्चेदु॑-त्तरवे॒दिरु॒पाव॑वर्ती॒हैव वि ज॑यामहा॒ इत्यसु॑रा॒ वज्र॑मु॒द्यत्य॑ दे॒वान॒भ्या॑यन्त॒ तानि॑न्द्रघो॒षो वसु॑भिः पु॒रस्ता॒दपा॑- [  ] \newline

\textbf{Pada Paata} \newline

अ॒सि॒ । इति॑ । समिति॑ । ह॒न्ति॒ । धृत्यै᳚ । दे॒वेभ्यः॑ । शु॒न्ध॒स्व॒ । दे॒वेभ्यः॑ । शु॒म्भ॒स्व॒ । इति॑ । अवेति॑ । च॒ । उ॒क्षति॑ । प्रेति॑ । च॒ । कि॒र॒ति॒ । शुद्ध्यै᳚ । इ॒न्द्र॒घो॒ष इती᳚न्द्र - घो॒षः । त्वा॒ । वसु॑भि॒रिति॒ वसु॑ - भिः॒ । पु॒रस्ता᳚त् । पा॒तु॒ । इति॑ । आ॒ह॒ । दि॒ग्भ्य इति॑ दिक् - भ्यः । ए॒व । ए॒ना॒म् । प्रेति॑ । उ॒क्ष॒ति॒ । दे॒वान् । च॒ । इत् । उ॒त्त॒र॒वे॒दिरित्यु॑त्तर - वे॒दिः । उ॒पाव॑व॒र्तीत्यु॑प - आव॑वर्ति । इ॒ह । ए॒व । वीति॑ । ज॒या॒म॒है॒ । इति॑ । असु॑राः । वज्र᳚म् । उ॒द्यत्येत्यु॑त् - यत्य॑ । दे॒वान् । अ॒भीति॑ । आ॒य॒न्त॒ । तान् । इ॒न्द्र॒घो॒ष इती᳚न्द्र - घो॒षः । वसु॑भि॒रिति॒ वसु॑ - भिः॒ । पु॒रस्ता᳚त् । अपेति॑ ।  \newline


\textbf{Krama Paata} \newline

अ॒सीति॑ । इति॒ सम् । सꣳ ह॑न्ति । ह॒न्ति॒ धृत्यै᳚ । धृत्यै॑ दे॒वेभ्यः॑ । दे॒वेभ्यः॑ शुन्धस्व । शु॒न्ध॒स्व॒ दे॒वेभ्यः॑ । दे॒वेभ्यः॑ शुम्भस्व । शु॒म्भ॒स्वेति॑ । इत्यव॑ । अव॑ च । चो॒क्षति॑ । उ॒क्षति॒ प्र । प्र च॑ । च॒ कि॒र॒ति॒ । कि॒र॒ति॒ शुद्ध्यै᳚ । शुद्ध्या॑ इन्द्रघो॒षः । इ॒न्द्र॒घो॒ष स्त्वा᳚ । इ॒न्द्र॒घो॒ष इती᳚न्द्र - घो॒षः । त्वा॒ वसु॑भिः । वसु॑भिः पु॒रस्ता᳚त् । वसु॑भि॒रिति॒ वसु॑ - भिः॒ । पु॒रस्ता᳚त् पातु । पा॒त्विति॑ । इत्या॑ह । आ॒ह॒ दि॒ग्भ्यः । दि॒ग्भ्य ए॒व । दि॒ग्भ्य इति॑ दिक् - भ्यः । ए॒वैना᳚म् । ए॒ना॒म् प्र । प्रोक्ष॑ति । उ॒क्ष॒ति॒ दे॒वान् । दे॒वाꣳश्च॑ । चेत् । इदु॑त्तरवे॒दिः । उ॒त्त॒र॒वे॒दिरु॒पाव॑वर्ति । उ॒त्त॒र॒वे॒दिरित्यु॑त्तर - वे॒दिः । उ॒पाव॑वर्ती॒ह । उ॒पाव॑व॒र्तीत्यु॑प - आव॑वर्ति । इ॒हैव । ए॒व वि । वि ज॑यामहै । ज॒या॒म॒हा॒ इति॑ । इत्यसु॑राः । असु॑रा॒ वज्र᳚म् । वज्र॑मु॒द्यत्य॑ । उ॒द्यत्य॑ दे॒वान् । उ॒द्यत्येत्यु॑त् - यत्य॑ । दे॒वान॒भि । अ॒भ्या॑यन्त । आ॒य॒न्त॒ तान् । तानि॑न्द्रघो॒षः । इ॒न्द्र॒घो॒षो वसु॑भिः । इ॒न्द्र॒घो॒ष इती᳚न्द्र - घो॒षः । वसु॑भिः पु॒रस्ता᳚त् । वसु॑भि॒रिति॒ वसु॑ - भिः॒ । पु॒रस्ता॒दप॑ । अपा॑नुदत \newline

\textbf{Jatai Paata} \newline

1. अ॒सीती त्य॑स्य॒सीति॑ । \newline
2. इति॒ सꣳ स मितीति॒ सम् । \newline
3. सꣳ ह॑न्ति हन्ति॒ सꣳ सꣳ ह॑न्ति । \newline
4. ह॒न्ति॒ धृत्यै॒ धृत्यै॑ हन्ति हन्ति॒ धृत्यै᳚ । \newline
5. धृत्यै॑ दे॒वेभ्यो॑ दे॒वेभ्यो॒ धृत्यै॒ धृत्यै॑ दे॒वेभ्यः॑ । \newline
6. दे॒वेभ्यः॑ शुन्धस्व शुन्धस्व दे॒वेभ्यो॑ दे॒वेभ्यः॑ शुन्धस्व । \newline
7. शु॒न्ध॒स्व॒ दे॒वेभ्यो॑ दे॒वेभ्यः॑ शुन्धस्व शुन्धस्व दे॒वेभ्यः॑ । \newline
8. दे॒वेभ्यः॑ शुम्भस्व शुम्भस्व दे॒वेभ्यो॑ दे॒वेभ्यः॑ शुम्भस्व । \newline
9. शु॒म्भ॒स्वे तीति॑ शुम्भस्व शुम्भ॒स्वेति॑ । \newline
10. इत्यवावे तीत्यव॑ । \newline
11. अव॑ च॒ चावाव॑ च । \newline
12. चो॒क्ष त्यु॒क्षति॑ च चो॒क्षति॑ । \newline
13. उ॒क्षति॒ प्र प्रोक्ष त्यु॒क्षति॒ प्र । \newline
14. प्र च॑ च॒ प्र प्र च॑ । \newline
15. च॒ कि॒र॒ति॒ कि॒र॒ति॒ च॒ च॒ कि॒र॒ति॒ । \newline
16. कि॒र॒ति॒ शुद्ध्यै॒ शुद्ध्यै॑ किरति किरति॒ शुद्ध्यै᳚ । \newline
17. शुद्ध्या॑ इन्द्रघो॒ष इ॑न्द्रघो॒षः शुद्ध्यै॒ शुद्ध्या॑ इन्द्रघो॒षः । \newline
18. इ॒न्द्र॒घो॒ष स्त्वा᳚ त्वेन्द्रघो॒ष इ॑न्द्रघो॒ष स्त्वा᳚ । \newline
19. इ॒न्द्र॒घो॒ष इती᳚न्द्र - घो॒षः । \newline
20. त्वा॒ वसु॑भि॒र् वसु॑भि स्त्वा त्वा॒ वसु॑भिः । \newline
21. वसु॑भिः पु॒रस्ता᳚त् पु॒रस्ता॒द् वसु॑भि॒र् वसु॑भिः पु॒रस्ता᳚त् । \newline
22. वसु॑भि॒रिति॒ वसु॑ - भिः॒ । \newline
23. पु॒रस्ता᳚त् पातु पातु पु॒रस्ता᳚त् पु॒रस्ता᳚त् पातु । \newline
24. पा॒त्वितीति॑ पातु पा॒त्विति॑ । \newline
25. इत्या॑हा॒हे तीत्या॑ह । \newline
26. आ॒ह॒ दि॒ग्भ्यो दि॒ग्भ्य आ॑हाह दि॒ग्भ्यः । \newline
27. दि॒ग्भ्य ए॒वैव दि॒ग्भ्यो दि॒ग्भ्य ए॒व । \newline
28. दि॒ग्भ्य इति॑ दिक् - भ्यः । \newline
29. ए॒वैना॑ मेना मे॒वैवैना᳚म् । \newline
30. ए॒ना॒म् प्र प्रैना॑ मेना॒म् प्र । \newline
31. प्रोक्ष॑ त्युक्षति॒ प्र प्रोक्ष॑ति । \newline
32. उ॒क्ष॒ति॒ दे॒वान् दे॒वा नु॑क्ष त्युक्षति दे॒वान् । \newline
33. दे॒वाꣳश्च॑ च दे॒वान् दे॒वाꣳश्च॑ । \newline
34. चे दिच् च॒ चे त् । \newline
35. इदु॑त्तरवे॒दि रु॑त्तरवे॒दि रिदि दु॑त्तरवे॒दिः । \newline
36. उ॒त्त॒र॒वे॒दि रु॒पाव॑वर् त्यु॒पाव॑वर् त्युत्तरवे॒दि रु॑त्तरवे॒दि रु॒पाव॑वर्ति । \newline
37. उ॒त्त॒र॒वे॒दिरित्यु॑त्तर - वे॒दिः । \newline
38. उ॒पाव॑वर्ती॒हे होपाव॑वर् त्यु॒पाव॑वर्ती॒ह । \newline
39. उ॒पाव॑व॒र्तीत्यु॑प - आव॑वर्ति । \newline
40. इ॒हैवैवे हेहैव । \newline
41. ए॒व वि व्ये॑वैव वि । \newline
42. वि ज॑यामहै जयामहै॒ वि वि ज॑यामहै । \newline
43. ज॒या॒म॒हा॒ इतीति॑ जयामहै जयामहा॒ इति॑ । \newline
44. इत्यसु॑रा॒ असु॑रा॒ इती त्यसु॑राः । \newline
45. असु॑रा॒ वज्रं॒ ॅवज्र॒ मसु॑रा॒ असु॑रा॒ वज्र᳚म् । \newline
46. वज्र॑ मु॒द्य त्यो॒द्यत्य॒ वज्रं॒ ॅवज्र॑ मु॒द्यत्य॑ । \newline
47. उ॒द्यत्य॑ दे॒वान् दे॒वा नु॒द्य त्यो॒द्यत्य॑ दे॒वान् । \newline
48. उ॒द्यत्येत्यु॑त् - यत्य॑ । \newline
49. दे॒वा न॒भ्य॑भि दे॒वान् दे॒वा न॒भि । \newline
50. अ॒भ्या॑ यन्ता यन्ता॒ भ्या᳚(1॒)भ्या॑ यन्त । \newline
51. आ॒य॒न्त॒ ताꣳ स्ता ना॑यन्ता यन्त॒ तान् । \newline
52. ता नि॑न्द्रघो॒ष इ॑न्द्रघो॒ष स्ताꣳ स्ता नि॑न्द्रघो॒षः । \newline
53. इ॒न्द्र॒घो॒षो वसु॑भि॒र् वसु॑भि रिन्द्रघो॒ष इ॑न्द्रघो॒षो वसु॑भिः । \newline
54. इ॒न्द्र॒घो॒ष इती᳚न्द्र - घो॒षः । \newline
55. वसु॑भिः पु॒रस्ता᳚त् पु॒रस्ता॒द् वसु॑भि॒र् वसु॑भिः पु॒रस्ता᳚त् । \newline
56. वसु॑भि॒रिति॒ वसु॑ - भिः॒ । \newline
57. पु॒रस्ता॒ दपाप॑ पु॒रस्ता᳚त् पु॒रस्ता॒ दप॑ । \newline
58. अपा॑ नुदता नुद॒ता पापा॑ नुदत । \newline

\textbf{Ghana Paata } \newline

1. अ॒सीती त्य॑स्य॒ सीति॒ सꣳ स मित्य॑स्य॒ सीति॒ सम् । \newline
2. इति॒ सꣳ स मितीति॒ सꣳ ह॑न्ति हन्ति॒ स मितीति॒ सꣳ ह॑न्ति । \newline
3. सꣳ ह॑न्ति हन्ति॒ सꣳ सꣳ ह॑न्ति॒ धृत्यै॒ धृत्यै॑ हन्ति॒ सꣳ सꣳ ह॑न्ति॒ धृत्यै᳚ । \newline
4. ह॒न्ति॒ धृत्यै॒ धृत्यै॑ हन्ति हन्ति॒ धृत्यै॑ दे॒वेभ्यो॑ दे॒वेभ्यो॒ धृत्यै॑ हन्ति हन्ति॒ धृत्यै॑ दे॒वेभ्यः॑ । \newline
5. धृत्यै॑ दे॒वेभ्यो॑ दे॒वेभ्यो॒ धृत्यै॒ धृत्यै॑ दे॒वेभ्यः॑ शुन्धस्व शुन्धस्व दे॒वेभ्यो॒ धृत्यै॒ धृत्यै॑ दे॒वेभ्यः॑ शुन्धस्व । \newline
6. दे॒वेभ्यः॑ शुन्धस्व शुन्धस्व दे॒वेभ्यो॑ दे॒वेभ्यः॑ शुन्धस्व दे॒वेभ्यो॑ दे॒वेभ्यः॑ शुन्धस्व दे॒वेभ्यो॑ दे॒वेभ्यः॑ शुन्धस्व दे॒वेभ्यः॑ । \newline
7. शु॒न्ध॒स्व॒ दे॒वेभ्यो॑ दे॒वेभ्यः॑ शुन्धस्व शुन्धस्व दे॒वेभ्यः॑ शुम्भस्व शुम्भस्व दे॒वेभ्यः॑ शुन्धस्व शुन्धस्व दे॒वेभ्यः॑ शुम्भस्व । \newline
8. दे॒वेभ्यः॑ शुम्भस्व शुम्भस्व दे॒वेभ्यो॑ दे॒वेभ्यः॑ शुम्भ॒स्वेतीति॑ शुम्भस्व दे॒वेभ्यो॑ दे॒वेभ्यः॑ शुम्भ॒स्वेति॑ । \newline
9. शु॒म्भ॒स्वेतीति॑ शुम्भस्व शुम्भ॒ स्वेत्यवावेति॑ शुम्भस्व शुम्भ॒ स्वेत्यव॑ । \newline
10. इत्यवावे तीत्यव॑ च॒ चावे तीत्यव॑ च । \newline
11. अव॑ च॒ चावाव॑ चो॒क्ष त्यु॒क्षति॒ चावाव॑ चो॒क्षति॑ । \newline
12. चो॒क्ष त्यु॒क्षति॑ च चो॒क्षति॒ प्र प्रोक्षति॑ च चो॒क्षति॒ प्र । \newline
13. उ॒क्षति॒ प्र प्रोक्ष त्यु॒क्षति॒ प्र च॑ च॒ प्रोक्ष त्यु॒क्षति॒ प्र च॑ । \newline
14. प्र च॑ च॒ प्र प्र च॑ किरति किरति च॒ प्र प्र च॑ किरति । \newline
15. च॒ कि॒र॒ति॒ कि॒र॒ति॒ च॒ च॒ कि॒र॒ति॒ शुद्ध्यै॒ शुद्ध्यै॑ किरति च च किरति॒ शुद्ध्यै᳚ । \newline
16. कि॒र॒ति॒ शुद्ध्यै॒ शुद्ध्यै॑ किरति किरति॒ शुद्ध्या॑ इन्द्रघो॒ष इ॑न्द्रघो॒षः शुद्ध्यै॑ किरति किरति॒ शुद्ध्या॑ इन्द्रघो॒षः । \newline
17. शुद्ध्या॑ इन्द्रघो॒ष इ॑न्द्रघो॒षः शुद्ध्यै॒ शुद्ध्या॑ इन्द्रघो॒ष स्त्वा᳚ त्वेन्द्रघो॒षः शुद्ध्यै॒ शुद्ध्या॑ इन्द्रघो॒ष स्त्वा᳚ । \newline
18. इ॒न्द्र॒घो॒ष स्त्वा᳚ त्वेन्द्रघो॒ष इ॑न्द्रघो॒ष स्त्वा॒ वसु॑भि॒र् वसु॑भि स्त्वेन्द्रघो॒ष इ॑न्द्रघो॒ष स्त्वा॒ वसु॑भिः । \newline
19. इ॒न्द्र॒घो॒ष इती᳚न्द्र - घो॒षः । \newline
20. त्वा॒ वसु॑भि॒र् वसु॑भि स्त्वा त्वा॒ वसु॑भिः पु॒रस्ता᳚त् पु॒रस्ता॒द् वसु॑भि स्त्वा त्वा॒ वसु॑भिः पु॒रस्ता᳚त् । \newline
21. वसु॑भिः पु॒रस्ता᳚त् पु॒रस्ता॒द् वसु॑भि॒र् वसु॑भिः पु॒रस्ता᳚त् पातु पातु पु॒रस्ता॒द् वसु॑भि॒र् वसु॑भिः पु॒रस्ता᳚त् पातु । \newline
22. वसु॑भि॒रिति॒ वसु॑ - भिः॒ । \newline
23. पु॒रस्ता᳚त् पातु पातु पु॒रस्ता᳚त् पु॒रस्ता᳚त् पा॒त्वि तीति॑ पातु पु॒रस्ता᳚त् पु॒रस्ता᳚त् पा॒त्विति॑ । \newline
24. पा॒त्वि तीति॑ पातु पा॒त्वित्या॑ हा॒हेति॑ पातु पा॒त्वि त्या॑ह । \newline
25. इत्या॑हा॒हे तीत्या॑ह दि॒ग्भ्यो दि॒ग्भ्य आ॒हे तीत्या॑ह दि॒ग्भ्यः । \newline
26. आ॒ह॒ दि॒ग्भ्यो दि॒ग्भ्य आ॑हाह दि॒ग्भ्य ए॒वैव दि॒ग्भ्य आ॑हाह दि॒ग्भ्य ए॒व । \newline
27. दि॒ग्भ्य ए॒वैव दि॒ग्भ्यो दि॒ग्भ्य ए॒वैना॑ मेना मे॒व दि॒ग्भ्यो दि॒ग्भ्य ए॒वैना᳚म् । \newline
28. दि॒ग्भ्य इति॑ दिक् - भ्यः । \newline
29. ए॒वैना॑ मेना मे॒वै वैना॒म् प्र प्रैना॑ मे॒वै वैना॒म् प्र । \newline
30. ए॒ना॒म् प्र प्रैना॑ मेना॒म् प्रोक्ष॑ त्युक्षति॒ प्रैना॑ मेना॒म् प्रोक्ष॑ति । \newline
31. प्रोक्ष॑ त्युक्षति॒ प्र प्रोक्ष॑ति दे॒वान् दे॒वा नु॑क्षति॒ प्र प्रोक्ष॑ति दे॒वान् । \newline
32. उ॒क्ष॒ति॒ दे॒वान् दे॒वा नु॑क्ष त्युक्षति दे॒वाꣳश्च॑ च दे॒वा नु॑क्ष त्युक्षति दे॒वाꣳश्च॑ । \newline
33. दे॒वाꣳश्च॑ च दे॒वान् दे॒वाꣳश्चे दिच् च॑ दे॒वान् दे॒वाꣳश्चेत् । \newline
34. चेदिच् च॒ चे दु॑त्तरवे॒दि रु॑त्तरवे॒दि रिच् च॒ चे दु॑त्तरवे॒दिः । \newline
35. इदु॑त्तरवे॒दि रु॑त्तरवे॒दिरिदि दु॑त्तरवे॒दि रु॒पाव॑वर् त्यु॒पाव॑वर् त्युत्तरवे॒दि रिदि दु॑त्तरवे॒दि रु॒पाव॑वर्ति । \newline
36. उ॒त्त॒र॒वे॒दि रु॒पाव॑वर् त्यु॒पाव॑वर् त्युत्तरवे॒दि रु॑त्तरवे॒दि रु॒पाव॑वर् ती॒हे होपाव॑वर् त्युत्तरवे॒दि रु॑त्तरवे॒दि रु॒पाव॑वर् ती॒ह । \newline
37. उ॒त्त॒र॒वे॒दिरित्यु॑त्तर - वे॒दिः । \newline
38. उ॒पाव॑वर् ती॒हे होपाव॑वर् त्यु॒पाव॑वर् ती॒है वैवे होपाव॑वर् त्यु॒पाव॑वर् ती॒हैव । \newline
39. उ॒पाव॑व॒र्तीत्यु॑प - आव॑वर्ति । \newline
40. इ॒है वैवे हेहैव वि व्ये॑वे हेहैव वि । \newline
41. ए॒व वि व्ये॑वैव वि ज॑यामहै जयामहै॒ व्ये॑वैव वि ज॑यामहै । \newline
42. वि ज॑यामहै जयामहै॒ वि वि ज॑यामहा॒ इतीति॑ जयामहै॒ वि वि ज॑यामहा॒ इति॑ । \newline
43. ज॒या॒म॒हा॒ इतीति॑ जयामहै जयामहा॒ इत्यसु॑रा॒ असु॑रा॒ इति॑ जयामहै जयामहा॒ इत्यसु॑राः । \newline
44. इत्यसु॑रा॒ असु॑रा॒ इती त्यसु॑रा॒ वज्रं॒ ॅवज्र॒ मसु॑रा॒ इती त्यसु॑रा॒ वज्र᳚म् । \newline
45. असु॑रा॒ वज्रं॒ ॅवज्र॒ मसु॑रा॒ असु॑रा॒ वज्र॑ मु॒द्य त्यो॒द्यत्य॒ वज्र॒ मसु॑रा॒ असु॑रा॒ वज्र॑ मु॒द्यत्य॑ । \newline
46. वज्र॑ मु॒द्य त्यो॒द्यत्य॒ वज्रं॒ ॅवज्र॑ मु॒द्यत्य॑ दे॒वान् दे॒वा नु॒द्यत्य॒ वज्रं॒ ॅवज्र॑ मु॒द्यत्य॑ दे॒वान् । \newline
47. उ॒द्यत्य॑ दे॒वान् दे॒वा नु॒द्य त्यो॒द्यत्य॑ दे॒वा न॒भ्य॑भि दे॒वा नु॒द्य त्यो॒द्यत्य॑ दे॒वा न॒भि । \newline
48. उ॒द्यत्येत्यु॑त् - यत्य॑ । \newline
49. दे॒वा न॒भ्य॑भि दे॒वान् दे॒वा न॒भ्या॑ यन्ता यन्ता॒भि दे॒वान् दे॒वा न॒भ्या॑ यन्त । \newline
50. अ॒भ्या॑ यन्ता यन्ता॒भ्या᳚(1॒)भ्या॑यन्त॒ ताꣳ स्ताना॑ यन्ता॒भ्या᳚(1॒)भ्या॑यन्त॒ तान् । \newline
51. आ॒य॒न्त॒ ताꣳ स्ताना॑ यन्ता यन्त॒ तानि॑न्द्रघो॒ष इ॑न्द्रघो॒ष स्ता ना॑यन्ता यन्त॒ तानि॑न्द्रघो॒षः । \newline
52. तानि॑न्द्रघो॒ष इ॑न्द्रघो॒ष स्ताꣳ स्ता नि॑न्द्रघो॒षो वसु॑भि॒र् वसु॑भि रिन्द्रघो॒ष स्ताꣳ स्ता नि॑न्द्रघो॒षो वसु॑भिः । \newline
53. इ॒न्द्र॒घो॒षो वसु॑भि॒र् वसु॑भि रिन्द्रघो॒ष इ॑न्द्रघो॒षो वसु॑भिः पु॒रस्ता᳚त् पु॒रस्ता॒द् वसु॑भि रिन्द्रघो॒ष इ॑न्द्रघो॒षो वसु॑भिः पु॒रस्ता᳚त् । \newline
54. इ॒न्द्र॒घो॒ष इती᳚न्द्र - घो॒षः । \newline
55. वसु॑भिः पु॒रस्ता᳚त् पु॒रस्ता॒द् वसु॑भि॒र् वसु॑भिः पु॒रस्ता॒ दपाप॑ पु॒रस्ता॒द् वसु॑भि॒र् वसु॑भिः पु॒रस्ता॒ दप॑ । \newline
56. वसु॑भि॒रिति॒ वसु॑ - भिः॒ । \newline
57. पु॒रस्ता॒ दपाप॑ पु॒रस्ता᳚त् पु॒रस्ता॒ दपा॑ नुदता नुद॒ताप॑ पु॒रस्ता᳚त् पु॒रस्ता॒ दपा॑ नुदत । \newline
58. अपा॑ नुदता नुद॒ता पापा॑ नुदत॒ मनो॑जवा॒ मनो॑जवा अनुद॒ता पापा॑ नुदत॒ मनो॑जवाः । \newline
\pagebreak
\markright{ TS 6.2.7.5  \hfill https://www.vedavms.in \hfill}

\section{ TS 6.2.7.5 }

\textbf{TS 6.2.7.5 } \newline
\textbf{Samhita Paata} \newline

-नुदत॒ मनो॑जवाः पि॒तृभि॑ र्दक्षिण॒तः प्रचे॑ता रु॒द्रैः प॒श्चाद्-वि॒श्वक॑र्माऽऽदि॒त्यैरु॑त्तर॒तो यदे॒वमु॑त्तरवे॒दिं प्रो॒क्षति॑ दि॒ग्भ्य ए॒व तद्-यज॑मानो॒ भ्रातृ॑व्या॒न् प्र णु॑दत॒ इन्द्रो॒ यती᳚न्थ् सालावृ॒केभ्यः॒ प्राय॑च्छ॒त् तान् द॑क्षिण॒त उ॑त्तरवे॒द्या आ॑द॒न्॒ यत् प्रोक्ष॑णीना-मु॒च्छिष्ये॑त॒ तद् द॑क्षिण॒त उ॑त्तरवे॒द्यै नि न॑ये॒द्-यदे॒व तत्र॑ क्रू॒रं तत् तेन॑ शमयति॒ यं द्वि॒ष्यात् तं ध्या॑येच्छु॒चै ( ) वैन॑मर्पयति ॥(मि॒मी॒ते॒ - नाम॑ - ध्रु॒वा - ऽप॑ - शु॒चा - त्रीणि॑ च) ( आ7) \newline

\textbf{Pada Paata} \newline

अ॒नु॒द॒त॒ । मनो॑जवा॒ इति॒ मनः॑ - ज॒वाः॒ । पि॒तृभि॒रिति॑ पि॒तृ - भिः॒ । द॒क्षि॒ण॒तः । प्रचे॑ता॒ इति॒ प्र - चे॒ताः॒ । रु॒द्रैः । प॒श्चात् । वि॒श्वक॒र्मेति॑ वि॒श्व - क॒र्मा॒ । आ॒दि॒त्यैः । उ॒त्त॒र॒त इत्यु॑त् - त॒र॒तः । यत् । ए॒वम् । उ॒त्त॒र॒वे॒दिमित्यु॑त्तर - वे॒दिम् । प्रो॒क्षतीति॑ प्र - उ॒क्षति॑ । दि॒ग्भ्य इति॑ दिक् - भ्यः । ए॒व । तत् । यज॑मानः । भ्रातृ॑व्यान् । प्रेति॑ । नु॒द॒ते॒ । इन्द्रः॑ । यतीन्॑ । सा॒ला॒वृ॒केभ्यः॑ । प्रेति॑ । अ॒य॒च्छ॒त् । तान् । द॒क्षि॒ण॒तः । उ॒त्त॒र॒वे॒द्या इत्यु॑त्तर - वे॒द्याः । आ॒द॒न्न् । यत् । प्रोक्ष॑णीना॒मिति॑ प्र - उक्ष॑णीनाम् । उ॒च्छिष्ये॒तेत्यु॑त् - शिष्ये॑त । तत् । द॒क्षि॒ण॒तः । उ॒त्त॒र॒वे॒द्या इत्यु॑त्तर - वे॒द्यै । नीति॑ । न॒ये॒त् । यत् । ए॒व । तत्र॑ । क्रू॒रम् । तत् । तेन॑ । श॒म॒य॒ति॒ । यम् । द्वि॒ष्यात् । तम् । ध्या॒ये॒त् । शु॒चा ( ) । ए॒व । ए॒न॒म् । अ॒र्प॒य॒ति॒ ॥(मि॒मी॒ते॒ - नाम॑ - ध्रु॒वा - ऽप॑ - शु॒चा - त्रीणि॑ च) ( आ7)  \newline


\textbf{Krama Paata} \newline

अ॒नु॒द॒त॒ मनो॑जवाः । मनो॑जवाः पि॒तृभिः॑ । मनो॑जवा॒ इति॒ मनः॑ - ज॒वाः॒ । पि॒तृभि॑र् दक्षिण॒तः । पि॒तृभि॒रिति॑ पि॒तृ - भिः॒ । द॒क्षि॒ण॒तः प्रचे॑ताः । प्रचे॑ता रु॒द्रैः । प्रचे॑ता॒ इति॒ प्र - चे॒ताः॒ । रु॒द्रैः प॒श्चात् । प॒श्चाद् वि॒श्वक॑र्मा । वि॒श्वक॑र्माऽऽदि॒त्यैः । वि॒श्वक॒र्मेति॑ वि॒श्व - क॒र्मा॒ । आ॒दि॒त्यैरु॑त्तर॒तः । उ॒त्त॒र॒तो यत् । उ॒त्त॒र॒त इत्यु॑त् - त॒र॒तः । यदे॒वम् । ए॒वमु॑त्तरवे॒दिम् । उ॒त्त॒र॒वे॒दिम् प्रो॒क्षति॑ । उ॒त्त॒र॒वे॒दिमित्यु॑त्तर - वे॒दिम् । प्रो॒क्षति॑ दि॒ग्भ्यः । प्रो॒क्षतीति॑ प्र - उ॒क्षति॑ । दि॒ग्भ्य ए॒व । दि॒ग्भ्य इति॑ दिक् - भ्यः । ए॒व तत् । तद् यज॑मानः । यज॑मानो॒ भ्रातृ॑व्यान् । भ्रातृ॑व्या॒न् प्र । प्र णु॑दते । नु॒द॒त॒ इन्द्रः॑ । इन्द्रो॒ यतीन्॑ । यती᳚न्थ् सालावृ॒केभ्यः॑ । सा॒ला॒वृ॒केभ्यः॒ प्र । प्राय॑च्छत् । अ॒य॒च्छ॒त् तान् । तान् द॑क्षिण॒तः । द॒क्षि॒ण॒त उ॑त्तरवे॒द्याः । उ॒त्त॒र॒वे॒द्या आ॑दन्न् । उ॒त्त॒र॒वे॒द्या इत्यु॑त्तर - वे॒द्याः । आ॒द॒न्न्॒. यत् । यत् प्रोक्ष॑णीनाम् । प्रोक्ष॑णीनामु॒च्छिष्ये॑त । प्रोक्ष॑णीना॒मिति॑ प्र - उक्ष॑णीनाम् । उ॒च्छिष्ये॑त॒ तत् । उ॒च्छिष्ये॒तेत्यु॑त् - शिष्ये॑त । तद् द॑क्षिण॒तः । द॒क्षि॒ण॒त उ॑त्तरवे॒द्यै । उ॒त्त॒र॒वे॒द्यै नि । उ॒त्त॒र॒वेद्या इत्यु॑त्तर - वे॒द्यै । नि न॑येत् । न॒ये॒द् यत् । यदे॒व । ए॒व तत्र॑ । तत्र॑ क्रू॒रम् । क्रू॒रम् तत् । तत् तेन॑ । तेन॑ शमयति । श॒म॒य॒ति॒ यम् । यम् द्वि॒ष्यात् । द्वि॒ष्यात् तम् । तम् ध्या॑येत् । ध्या॒ये॒च्छु॒चा ( ) । शु॒चैव । ए॒वैन᳚म् । ए॒न॒म॒र्प॒य॒ति॒ । अ॒र्प॒य॒तीत्य॑र्पयति । \newline

\textbf{Jatai Paata} \newline

1. अ॒नु॒द॒त॒ मनो॑जवा॒ मनो॑जवा अनुदता नुदत॒ मनो॑जवाः । \newline
2. मनो॑जवाः पि॒तृभिः॑ पि॒तृभि॒र् मनो॑जवा॒ मनो॑जवाः पि॒तृभिः॑ । \newline
3. मनो॑जवा॒ इति॒ मनः॑ - ज॒वाः॒ । \newline
4. पि॒तृभि॑र् दक्षिण॒तो द॑क्षिण॒तः पि॒तृभिः॑ पि॒तृभि॑र् दक्षिण॒तः । \newline
5. पि॒तृभि॒रिति॑ पि॒तृ - भिः॒ । \newline
6. द॒क्षि॒ण॒तः प्रचे॑ताः॒ प्रचे॑ता दक्षिण॒तो द॑क्षिण॒तः प्रचे॑ताः । \newline
7. प्रचे॑ता रु॒द्रै रु॒द्रैः प्रचे॑ताः॒ प्रचे॑ता रु॒द्रैः । \newline
8. प्रचे॑ता॒ इति॒ प्र - चे॒ताः॒ । \newline
9. रु॒द्रैः प॒श्चात् प॒श्चाद् रु॒द्रै रु॒द्रैः प॒श्चात् । \newline
10. प॒श्चाद् वि॒श्वक॑र्मा वि॒श्वक॑र्मा प॒श्चात् प॒श्चाद् वि॒श्वक॑र्मा । \newline
11. वि॒श्वक॑र्मा ऽऽदि॒त्यै रा॑दि॒त्यैर् वि॒श्वक॑र्मा वि॒श्वक॑र्मा ऽऽदि॒त्यैः । \newline
12. वि॒श्वक॒र्मेति॑ वि॒श्व - क॒र्मा॒ । \newline
13. आ॒दि॒त्यै रु॑त्तर॒त उ॑त्तर॒त आ॑दि॒त्यै रा॑दि॒त्यै रु॑त्तर॒तः । \newline
14. उ॒त्त॒र॒तो यद् यदु॑त्तर॒त उ॑त्तर॒तो यत् । \newline
15. उ॒त्त॒र॒त इत्यु॑त् - त॒र॒तः । \newline
16. यदे॒व मे॒वं ॅयद् यदे॒वम् । \newline
17. ए॒व मु॑त्तरवे॒दि मु॑त्तरवे॒दि मे॒व मे॒व मु॑त्तरवे॒दिम् । \newline
18. उ॒त्त॒र॒वे॒दिम् प्रो॒क्षति॑ प्रो॒क्ष त्यु॑त्तरवे॒दि मु॑त्तरवे॒दिम् प्रो॒क्षति॑ । \newline
19. उ॒त्त॒र॒वे॒दिमित्यु॑त्तर - वे॒दिम् । \newline
20. प्रो॒क्षति॑ दि॒ग्भ्यो दि॒ग्भ्यः प्रो॒क्षति॑ प्रो॒क्षति॑ दि॒ग्भ्यः । \newline
21. प्रो॒क्षतीति॑ प्र - उ॒क्षति॑ । \newline
22. दि॒ग्भ्य ए॒वैव दि॒ग्भ्यो दि॒ग्भ्य ए॒व । \newline
23. दि॒ग्भ्य इति॑ दिक् - भ्यः । \newline
24. ए॒व तत् तदे॒वैव तत् । \newline
25. तद् यज॑मानो॒ यज॑मान॒ स्तत् तद् यज॑मानः । \newline
26. यज॑मानो॒ भ्रातृ॑व्या॒न् भ्रातृ॑व्या॒न्॒. यज॑मानो॒ यज॑मानो॒ भ्रातृ॑व्यान् । \newline
27. भ्रातृ॑व्या॒न् प्र प्र भ्रातृ॑व्या॒न् भ्रातृ॑व्या॒न् प्र । \newline
28. प्र णु॑दते नुदते॒ प्र प्र णु॑दते । \newline
29. नु॒द॒त॒ इन्द्र॒ इन्द्रो॑ नुदते नुदत॒ इन्द्रः॑ । \newline
30. इन्द्रो॒ यती॒न्॒. यती॒ निन्द्र॒ इन्द्रो॒ यतीन्॑ । \newline
31. यती᳚न् थ्सालावृ॒केभ्यः॑ सालावृ॒केभ्यो॒ यती॒न्॒. यती᳚न् थ्सालावृ॒केभ्यः॑ । \newline
32. सा॒ला॒वृ॒केभ्यः॒ प्र प्र सा॑लावृ॒केभ्यः॑ सालावृ॒केभ्यः॒ प्र । \newline
33. प्राय॑च्छ दयच्छ॒त् प्र प्राय॑च्छत् । \newline
34. अ॒य॒च्छ॒त् ताꣳ स्ता न॑यच्छ दयच्छ॒त् तान् । \newline
35. तान् द॑क्षिण॒तो द॑क्षिण॒त स्ताꣳ स्तान् द॑क्षिण॒तः । \newline
36. द॒क्षि॒ण॒त उ॑त्तरवे॒द्या उ॑त्तरवे॒द्या द॑क्षिण॒तो द॑क्षिण॒त उ॑त्तरवे॒द्याः । \newline
37. उ॒त्त॒र॒वे॒द्या आ॑दन् नादन् नुत्तरवे॒द्या उ॑त्तरवे॒द्या आ॑दन्न् । \newline
38. उ॒त्त॒र॒वे॒द्या इत्यु॑त्तर - वे॒द्याः । \newline
39. आ॒द॒न्॒. यद् यदा॑दन् नाद॒न्॒. यत् । \newline
40. यत् प्रोक्ष॑णीना॒म् प्रोक्ष॑णीनां॒ ॅयद् यत् प्रोक्ष॑णीनाम् । \newline
41. प्रोक्ष॑णीना मु॒च्छिष्ये॑ तो॒च्छिष्ये॑त॒ प्रोक्ष॑णीना॒म् प्रोक्ष॑णीना मु॒च्छिष्ये॑त । \newline
42. प्रोक्ष॑णीना॒मिति॑ प्र - उक्ष॑णीनाम् । \newline
43. उ॒च्छिष्ये॑त॒ तत् तदु॒च्छिष्ये॑ तो॒च्छिष्ये॑त॒ तत् । \newline
44. उ॒च्छिष्ये॒तेत्यु॑त् - शिष्ये॑त । \newline
45. तद् द॑क्षिण॒तो द॑क्षिण॒त स्तत् तद् द॑क्षिण॒तः । \newline
46. द॒क्षि॒ण॒त उ॑त्तरवे॒द्या उ॑त्तरवे॒द्यै द॑क्षिण॒तो द॑क्षिण॒त उ॑त्तरवे॒द्यै । \newline
47. उ॒त्त॒र॒वे॒द्यै नि न्यु॑त्तरवे॒द्या उ॑त्तरवे॒द्यै नि । \newline
48. उ॒त्त॒र॒वे॒द्या इत्यु॑त्तर - वे॒द्यै । \newline
49. नि न॑येन् नये॒न् नि नि न॑येत् । \newline
50. न॒ये॒द् यद् यन् न॑येन् नये॒द् यत् । \newline
51. यदे॒वैव यद् यदे॒व । \newline
52. ए॒व तत्र॒ तत्रै॒वैव तत्र॑ । \newline
53. तत्र॑ क्रू॒रम् क्रू॒रम् तत्र॒ तत्र॑ क्रू॒रम् । \newline
54. क्रू॒रम् तत् तत् क्रू॒रम् क्रू॒रम् तत् । \newline
55. तत् तेन॒ तेन॒ तत् तत् तेन॑ । \newline
56. तेन॑ शमयति शमयति॒ तेन॒ तेन॑ शमयति । \newline
57. श॒म॒य॒ति॒ यं ॅयꣳ श॑मयति शमयति॒ यम् । \newline
58. यम् द्वि॒ष्याद् द्वि॒ष्याद् यं ॅयम् द्वि॒ष्यात् । \newline
59. द्वि॒ष्यात् तम् तम् द्वि॒ष्याद् द्वि॒ष्यात् तम् । \newline
60. तम् ध्या॑येद् ध्याये॒त् तम् तम् ध्या॑येत् । \newline
61. ध्या॒ये॒च्छु॒चा शु॒चा ध्या॑येद् ध्याये च्छु॒चा । \newline
62. शु॒चैवैव शु॒चा शु॒चैव । \newline
63. ए॒वैन॑ मेन मे॒वैवैन᳚म् । \newline
64. ए॒न॒ म॒र्प॒य॒ त्य॒र्प॒य॒ त्ये॒न॒ मे॒न॒ म॒र्प॒य॒ति॒ । \newline
65. अ॒र्प॒य॒तीत्य॑र्पयति । \newline

\textbf{Ghana Paata } \newline

1. अ॒नु॒द॒त॒ मनो॑जवा॒ मनो॑जवा अनुदता नुदत॒ मनो॑जवाः पि॒तृभिः॑ पि॒तृभि॒र् मनो॑जवा अनुदता नुदत॒ मनो॑जवाः पि॒तृभिः॑ । \newline
2. मनो॑जवाः पि॒तृभिः॑ पि॒तृभि॒र् मनो॑जवा॒ मनो॑जवाः पि॒तृभि॑र् दक्षिण॒तो द॑क्षिण॒तः पि॒तृभि॒र् मनो॑जवा॒ मनो॑जवाः पि॒तृभि॑र् दक्षिण॒तः । \newline
3. मनो॑जवा॒ इति॒ मनः॑ - ज॒वाः॒ । \newline
4. पि॒तृभि॑र् दक्षिण॒तो द॑क्षिण॒तः पि॒तृभिः॑ पि॒तृभि॑र् दक्षिण॒तः प्रचे॑ताः॒ प्रचे॑ता दक्षिण॒तः 
पि॒तृभिः॑ पि॒तृभि॑र् दक्षिण॒तः प्रचे॑ताः । \newline
5. पि॒तृभि॒रिति॑ पि॒तृ - भिः॒ । \newline
6. द॒क्षि॒ण॒तः प्रचे॑ताः॒ प्रचे॑ता दक्षिण॒तो द॑क्षिण॒तः प्रचे॑ता रु॒द्रै रु॒द्रैः प्रचे॑ता दक्षिण॒तो द॑क्षिण॒तः प्रचे॑ता रु॒द्रैः । \newline
7. प्रचे॑ता रु॒द्रै रु॒द्रैः प्रचे॑ताः॒ प्रचे॑ता रु॒द्रैः प॒श्चात् प॒श्चाद् रु॒द्रैः प्रचे॑ताः॒ प्रचे॑ता रु॒द्रैः प॒श्चात् । \newline
8. प्रचे॑ता॒ इति॒ प्र - चे॒ताः॒ । \newline
9. रु॒द्रैः प॒श्चात् प॒श्चाद् रु॒द्रै रु॒द्रैः प॒श्चाद् वि॒श्वक॑र्मा वि॒श्वक॑र्मा प॒श्चाद् रु॒द्रै रु॒द्रैः प॒श्चाद् वि॒श्वक॑र्मा । \newline
10. प॒श्चाद् वि॒श्वक॑र्मा वि॒श्वक॑र्मा प॒श्चात् प॒श्चाद् वि॒श्वक॑र्मा ऽऽदि॒त्यै रा॑दि॒त्यैर् वि॒श्वक॑र्मा प॒श्चात् प॒श्चाद् वि॒श्वक॑र्मा ऽऽदि॒त्यैः । \newline
11. वि॒श्वक॑र्मा ऽऽदि॒त्यै रा॑दि॒त्यैर् वि॒श्वक॑र्मा वि॒श्वक॑र्मा ऽऽदि॒त्यै रु॑त्तर॒त उ॑त्तर॒त आ॑दि॒त्यैर् वि॒श्वक॑र्मा वि॒श्वक॑र्मा ऽऽदि॒त्यै रु॑त्तर॒तः । \newline
12. वि॒श्वक॒र्मेति॑ वि॒श्व - क॒र्मा॒ । \newline
13. आ॒दि॒त्यै रु॑त्तर॒त उ॑त्तर॒त आ॑दि॒त्यै रा॑दि॒त्यै रु॑त्तर॒तो यद् यदु॑त्तर॒त आ॑दि॒त्यै रा॑दि॒त्यै रु॑त्तर॒तो यत् । \newline
14. उ॒त्त॒र॒तो यद् यदु॑त्तर॒त उ॑त्तर॒तो यदे॒व मे॒वं ॅयदु॑त्तर॒त उ॑त्तर॒तो यदे॒वम् । \newline
15. उ॒त्त॒र॒त इत्यु॑त् - त॒र॒तः । \newline
16. यदे॒व मे॒वं ॅयद् यदे॒व मु॑त्तरवे॒दि मु॑त्तरवे॒दि मे॒वं ॅयद् यदे॒व मु॑त्तरवे॒दिम् । \newline
17. ए॒व मु॑त्तरवे॒दि मु॑त्तरवे॒दि मे॒व मे॒व मु॑त्तरवे॒दिम् प्रो॒क्षति॑ प्रो॒क्ष त्यु॑त्तरवे॒दि मे॒व मे॒व मु॑त्तरवे॒दिम् प्रो॒क्षति॑ । \newline
18. उ॒त्त॒र॒वे॒दिम् प्रो॒क्षति॑ प्रो॒क्ष त्यु॑त्तरवे॒दि मु॑त्तरवे॒दिम् प्रो॒क्षति॑ दि॒ग्भ्यो दि॒ग्भ्यः प्रो॒क्ष त्यु॑त्तरवे॒दि मु॑त्तरवे॒दिम् प्रो॒क्षति॑ दि॒ग्भ्यः । \newline
19. उ॒त्त॒र॒वे॒दिमित्यु॑त्तर - वे॒दिम् । \newline
20. प्रो॒क्षति॑ दि॒ग्भ्यो दि॒ग्भ्यः प्रो॒क्षति॑ प्रो॒क्षति॑ दि॒ग्भ्य ए॒वैव दि॒ग्भ्यः प्रो॒क्षति॑ प्रो॒क्षति॑ दि॒ग्भ्य ए॒व । \newline
21. प्रो॒क्षतीति॑ प्र - उ॒क्षति॑ । \newline
22. दि॒ग्भ्य ए॒वैव दि॒ग्भ्यो दि॒ग्भ्य ए॒व तत् तदे॒व दि॒ग्भ्यो दि॒ग्भ्य ए॒व तत् । \newline
23. दि॒ग्भ्य इति॑ दिक् - भ्यः । \newline
24. ए॒व तत् तदे॒वैव तद् यज॑मानो॒ यज॑मान॒ स्त दे॒वैव तद् यज॑मानः । \newline
25. तद् यज॑मानो॒ यज॑मान॒ स्तत् तद् यज॑मानो॒ भ्रातृ॑व्या॒न् भ्रातृ॑व्या॒न्॒. यज॑मान॒ स्तत् तद् यज॑मानो॒ भ्रातृ॑व्यान् । \newline
26. यज॑मानो॒ भ्रातृ॑व्या॒न् भ्रातृ॑व्या॒न्॒. यज॑मानो॒ यज॑मानो॒ भ्रातृ॑व्या॒न् प्र प्र भ्रातृ॑व्या॒न्॒. यज॑मानो॒ यज॑मानो॒ भ्रातृ॑व्या॒न् प्र । \newline
27. भ्रातृ॑व्या॒न् प्र प्र भ्रातृ॑व्या॒न् भ्रातृ॑व्या॒न् प्र णु॑दते नुदते॒ प्र भ्रातृ॑व्या॒न् भ्रातृ॑व्या॒न् प्र णु॑दते । \newline
28. प्र णु॑दते नुदते॒ प्र प्र णु॑दत॒ इन्द्र॒ इन्द्रो॑ नुदते॒ प्र प्र णु॑दत॒ इन्द्रः॑ । \newline
29. नु॒द॒त॒ इन्द्र॒ इन्द्रो॑ नुदते नुदत॒ इन्द्रो॒ यती॒न्॒. यती॒ निन्द्रो॑ नुदते नुदत॒ इन्द्रो॒ यतीन्॑ । \newline
30. इन्द्रो॒ यती॒न्॒. यती॒ निन्द्र॒ इन्द्रो॒ यती᳚न् थ्सालावृ॒केभ्यः॑ सालावृ॒केभ्यो॒ यती॒ निन्द्र॒ इन्द्रो॒ यती᳚न् थ्सालावृ॒केभ्यः॑ । \newline
31. यती᳚न् थ्सालावृ॒केभ्यः॑ सालावृ॒केभ्यो॒ यती॒न्॒. यती᳚न् थ्सालावृ॒केभ्यः॒ प्र प्र सा॑लावृ॒केभ्यो॒ यती॒न्॒. यती᳚न् थ्सालावृ॒केभ्यः॒ प्र । \newline
32. सा॒ला॒वृ॒केभ्यः॒ प्र प्र सा॑लावृ॒केभ्यः॑ सालावृ॒केभ्यः॒ प्राय॑च्छ दयच्छ॒त् प्र सा॑लावृ॒केभ्यः॑ सालावृ॒केभ्यः॒ प्राय॑च्छत् । \newline
33. प्राय॑च्छ दयच्छ॒त् प्र प्राय॑च्छ॒त् ताꣳ स्ता न॑यच्छ॒त् प्र प्राय॑च्छ॒त् तान् । \newline
34. अ॒य॒च्छ॒त् ताꣳ स्ता न॑यच्छ दयच्छ॒त् तान् द॑क्षिण॒तो द॑क्षिण॒त स्ता न॑यच्छ दयच्छ॒त् तान् द॑क्षिण॒तः । \newline
35. तान् द॑क्षिण॒तो द॑क्षिण॒त स्ताꣳ स्तान् द॑क्षिण॒त उ॑त्तरवे॒द्या उ॑त्तरवे॒द्या द॑क्षिण॒त स्ताꣳ स्तान् द॑क्षिण॒त उ॑त्तरवे॒द्याः । \newline
36. द॒क्षि॒ण॒त उ॑त्तरवे॒द्या उ॑त्तरवे॒द्या द॑क्षिण॒तो द॑क्षिण॒त उ॑त्तरवे॒द्या आ॑दन् नादन् नुत्तरवे॒द्या द॑क्षिण॒तो द॑क्षिण॒त उ॑त्तरवे॒द्या आ॑दन्न् । \newline
37. उ॒त्त॒र॒वे॒द्या आ॑दन् नादन् नुत्तरवे॒द्या उ॑त्तरवे॒द्या आ॑द॒न्॒. यद् यदा॑दन् नुत्तरवे॒द्या उ॑त्तरवे॒द्या आ॑द॒न्॒. यत् । \newline
38. उ॒त्त॒र॒वे॒द्या इत्यु॑त्तर - वे॒द्याः । \newline
39. आ॒द॒न्॒. यद् यदा॑दन् नाद॒न्॒. यत् प्रोक्ष॑णीना॒म् प्रोक्ष॑णीनां॒ ॅयदा॑दन् नाद॒न्॒. यत् प्रोक्ष॑णीनाम् । \newline
40. यत् प्रोक्ष॑णीना॒म् प्रोक्ष॑णीनां॒ ॅयद् यत् प्रोक्ष॑णीना मु॒च्छिष्ये॑ तो॒च्छिष्ये॑त॒ प्रोक्ष॑णीनां॒ ॅयद् यत् प्रोक्ष॑णीना मु॒च्छिष्ये॑त । \newline
41. प्रोक्ष॑णीना मु॒च्छिष्ये॑ तो॒च्छिष्ये॑त॒ प्रोक्ष॑णीना॒म् प्रोक्ष॑णीना मु॒च्छिष्ये॑त॒ तत् तदु॒च्छिष्ये॑त॒ प्रोक्ष॑णीना॒म् प्रोक्ष॑णीना मु॒च्छिष्ये॑त॒ तत् । \newline
42. प्रोक्ष॑णीना॒मिति॑ प्र - उक्ष॑णीनाम् । \newline
43. उ॒च्छिष्ये॑त॒ तत् तदु॒च्छिष्ये॑ तो॒च्छिष्ये॑त॒ तद् द॑क्षिण॒तो द॑क्षिण॒त स्तदु॒च्छिष्ये॑ तो॒च्छिष्ये॑त॒ तद् द॑क्षिण॒तः । \newline
44. उ॒च्छिष्ये॒तेत्यु॑त् - शिष्ये॑त । \newline
45. तद् द॑क्षिण॒तो द॑क्षिण॒त स्तत् तद् द॑क्षिण॒त उ॑त्तरवे॒द्या उ॑त्तरवे॒द्यै द॑क्षिण॒त स्तत् तद् द॑क्षिण॒त उ॑त्तरवे॒द्यै । \newline
46. द॒क्षि॒ण॒त उ॑त्तरवे॒द्या उ॑त्तरवे॒द्यै द॑क्षिण॒तो द॑क्षिण॒त उ॑त्तरवे॒द्यै नि न्यु॑त्तरवे॒द्यै द॑क्षिण॒तो द॑क्षिण॒त उ॑त्तरवे॒द्यै नि । \newline
47. उ॒त्त॒र॒वे॒द्यै नि न्यु॑त्तरवे॒द्या उ॑त्तरवे॒द्यै नि न॑येन् नये॒न् न्यु॑त्तरवे॒द्या उ॑त्तरवे॒द्यै नि न॑येत् । \newline
48. उ॒त्त॒र॒वे॒द्या इत्यु॑त्तर - वे॒द्यै । \newline
49. नि न॑येन् नये॒न् नि नि न॑ये॒द् यद् यन् न॑ये॒न् नि नि न॑ये॒द् यत् । \newline
50. न॒ये॒द् यद् यन् न॑येन् नये॒द् यदे॒ वैव यन् न॑येन् नये॒द् यदे॒व । \newline
51. यदे॒वैव यद् यदे॒व तत्र॒ तत्रै॒व यद् यदे॒व तत्र॑ । \newline
52. ए॒व तत्र॒ तत्रै॒ वैव तत्र॑ क्रू॒रम् क्रू॒रम् तत्रै॒ वैव तत्र॑ क्रू॒रम् । \newline
53. तत्र॑ क्रू॒रम् क्रू॒रम् तत्र॒ तत्र॑ क्रू॒रम् तत् तत् क्रू॒रम् तत्र॒ तत्र॑ क्रू॒रम् तत् । \newline
54. क्रू॒रम् तत् तत् क्रू॒रम् क्रू॒रम् तत् तेन॒ तेन॒ तत् क्रू॒रम् क्रू॒रम् तत् तेन॑ । \newline
55. तत् तेन॒ तेन॒ तत् तत् तेन॑ शमयति शमयति॒ तेन॒ तत् तत् तेन॑ शमयति । \newline
56. तेन॑ शमयति शमयति॒ तेन॒ तेन॑ शमयति॒ यं ॅयꣳ श॑मयति॒ तेन॒ तेन॑ शमयति॒ यम् । \newline
57. श॒म॒य॒ति॒ यं ॅयꣳ श॑मयति शमयति॒ यम् द्वि॒ष्याद् द्वि॒ष्याद् यꣳ श॑मयति शमयति॒ यम् द्वि॒ष्यात् । \newline
58. यम् द्वि॒ष्याद् द्वि॒ष्याद् यं ॅयम् द्वि॒ष्यात् तम् तम् द्वि॒ष्याद् यं ॅयम् द्वि॒ष्यात् तम् । \newline
59. द्वि॒ष्यात् तम् तम् द्वि॒ष्याद् द्वि॒ष्यात् तम् ध्या॑येद् ध्याये॒त् तम् द्वि॒ष्याद् द्वि॒ष्यात् तम् ध्या॑येत् । \newline
60. तम् ध्या॑येद् ध्याये॒त् तम् तम् ध्या॑ये च्छु॒चा शु॒चा ध्या॑ये॒त् तम् तम् ध्या॑ये च्छु॒चा । \newline
61. ध्या॒ये॒ च्छु॒चा शु॒चा ध्या॑येद् ध्याये च्छु॒चै वैव शु॒चा ध्या॑येद् ध्याये च्छु॒चैव । \newline
62. शु॒चै वैव शु॒चा शु॒चैवैन॑ मेन मे॒व शु॒चा शु॒चैवैन᳚म् । \newline
63. ए॒वैन॑ मेन मे॒वै वैन॑ मर्पय त्यर्पय त्येन मे॒वै वैन॑ मर्पयति । \newline
64. ए॒न॒ म॒र्प॒य॒ त्य॒र्प॒य॒ त्ये॒न॒ मे॒न॒ म॒र्प॒य॒ति॒ । \newline
65. अ॒र्प॒य॒तीत्य॑र्पयति । \newline
\pagebreak
\markright{ TS 6.2.8.1  \hfill https://www.vedavms.in \hfill}

\section{ TS 6.2.8.1 }

\textbf{TS 6.2.8.1 } \newline
\textbf{Samhita Paata} \newline

सोत्त॑रवे॒दिर॑ब्रवी॒थ् सर्वा॒न् मया॒ कामा॒न् व्य॑श्नव॒थेति॒ ते दे॒वा अ॑कामय॒न्तासु॑रा॒न् भ्रातृ॑व्यान॒भि भ॑वे॒मेति॒ ते॑ऽजुहवुः सिꣳ॒॒हीर॑सि सपत्नसा॒ही स्वाहेति॒ तेऽसु॑रा॒न् भ्रातृ॑व्या-न॒भ्य॑भव॒न् तेऽसु॑रा॒न् भ्रातृ॑व्या-नभि॒भूया॑कामयन्त प्र॒जां ॅवि॑न्देम॒हीति॒ ते॑ऽजुहवुः सिꣳ॒॒हीर॑सि सुप्रजा॒वनिः॒ स्वाहेति॒ ते प्र॒जाम॑विन्दन्त॒ ते प्र॒जां ॅवि॒त्त्वा- [  ] \newline

\textbf{Pada Paata} \newline

सा । उ॒त्त॒र॒वे॒दिरित्यु॑त्तर-वे॒दिः । अ॒ब्र॒वी॒त् । सर्वान्॑ । मया᳚ । कामान्॑ । वीति॑ । अ॒श्न॒व॒थ॒ । इति॑ । ते । दे॒वाः । अ॒का॒म॒य॒न्त॒ । असु॑रान् । भ्रातृ॑व्यान् । अ॒भीति॑ । भ॒वे॒म॒ । इति॑ । ते । अ॒जु॒ह॒वुः॒ । सिꣳ॒॒हीः । अ॒सि॒ । स॒प॒त्न॒सा॒हीति॑ सपत्न - सा॒ही । स्वाहा᳚ । इति॑ । ते । असु॑रान् । भ्रातृ॑व्यान् । अ॒भीति॑ । अ॒भ॒व॒न्न् । ते । असु॑रान् । भ्रातृ॑व्यान् । अ॒भि॒भूयेत्य॑भि - भूय॑ । अ॒का॒म॒य॒न्त॒ । प्र॒जामिति॑ प्र-जाम् । वि॒न्दे॒म॒हि॒ । इति॑ । ते । अ॒जु॒ह॒वुः॒ । सिꣳ॒॒हीः । अ॒सि॒ । सु॒प्र॒जा॒वनि॒रिति॑ सुप्रजा - वनिः॑ । स्वाहा᳚ । इति॑ । ते । प्र॒जामिति॑ प्र-जाम् । अ॒वि॒न्द॒न्त॒ । ते । प्र॒जामिति॑ प्र - जाम् । वि॒त्त्वा ।  \newline


\textbf{Krama Paata} \newline

सोत्त॑रवे॒दिः । उ॒त्त॒र॒वे॒दिर॑ब्रवीत् । उ॒त्त॒र॒वे॒दिरित्यु॑त्तर - वे॒दिः । अ॒ब्र॒वी॒थ् सर्वान्॑ । सर्वा॒न् मया᳚ । मया॒ कामान्॑ । कामा॒न्.॒ वि । व्य॑श्ञवथ । अ॒श्ञ॒व॒थेति॑ । इति॒ ते । ते दे॒वाः । दे॒वा अ॑कामयन्त । अ॒का॒म॒य॒न्तासु॑रान् । असु॑रा॒न् भ्रातृ॑व्यान् । भ्रातृ॑व्यान॒भि । अ॒भि भ॑वेम । भ॒वे॒मेति॑ । इति॒ ते । ते॑ऽजुहवुः । अ॒जु॒ह॒वुः॒ सिꣳ॒॒हीः । सिꣳ॒॒हीर॑सि । अ॒सि॒ स॒प॒त्न॒सा॒ही । स॒प॒त्न॒सा॒ही स्वाहा᳚ । स॒प॒त्न॒सा॒हीति॑ सपत्न - सा॒ही । स्वाहेति॑ । इति॒ ते । तेऽसु॑रान् । असु॑रा॒न् भ्रातृ॑व्यान् । भ्रातृ॑व्यान॒भि । अ॒भ्य॑भवन्न् । अ॒भ॒व॒न् ते । तेऽसु॑रान् । असु॑रा॒न् भ्रातृ॑व्यान् । भ्रातृ॑व्यानभि॒भूय॑ । अ॒भि॒भूया॑कामयन्त । अ॒भि॒भूयेत्य॑भि - भूय॑ । अ॒का॒म॒य॒न्त॒ प्र॒जाम् । प्र॒जाम् ॅवि॑न्देमहि । प्र॒जामिति॑ प्र - जाम् । वि॒न्दे॒म॒हीति॑ । इति॒ ते । ते॑ऽजुहवुः । अ॒जु॒ह॒वुः॒ सिꣳ॒॒हीः । सिꣳ॒॒हीर॑सि । अ॒सि॒ सु॒प्र॒जा॒वनिः॑ । सु॒प्र॒जा॒वनिः॒ स्वाहा᳚ । सु॒प्र॒जा॒वनि॒रिति॑ सुप्रजा - वनिः॑ । स्वाहेति॑ । इति॒ ते । ते प्र॒जाम् । प्र॒जाम॑विन्दन्त । प्र॒जामिति॑ प्र - जाम् । अ॒वि॒न्द॒न्त॒ ते । ते प्र॒जाम् । प्र॒जाम् ॅवि॒त्वा । प्र॒जामिति॑ प्र - जाम् । वि॒त्वाऽका॑मयन्त \newline

\textbf{Jatai Paata} \newline

1. सोत्त॑रवे॒दि रु॑त्तरवे॒दिः सा सोत्त॑रवे॒दिः । \newline
2. उ॒त्त॒र॒वे॒दि र॑ब्रवीदब्रवी दुत्तरवे॒दि रु॑त्तरवे॒दि र॑ब्रवीत् । \newline
3. उ॒त्त॒र॒वे॒दिरित्यु॑त्तर - वे॒दिः । \newline
4. अ॒ब्र॒वी॒थ् सर्वा॒न् थ्सर्वा॑ नब्रवी दब्रवी॒थ् सर्वान्॑ । \newline
5. सर्वा॒न् मया॒ मया॒ सर्वा॒न् थ्सर्वा॒न् मया᳚ । \newline
6. मया॒ कामा॒न् कामा॒न् मया॒ मया॒ कामान्॑ । \newline
7. कामा॒न्॒. वि वि कामा॒न् कामा॒न्॒. वि । \newline
8. व्य॑श्ञवथा श्ञवथ॒ वि व्य॑श्ञवथ । \newline
9. अ॒श्ञ॒व॒थे तीत्य॑श्ञवथा श्ञव॒थेति॑ । \newline
10. इति॒ ते त इतीति॒ ते । \newline
11. ते दे॒वा दे॒वा स्ते ते दे॒वाः । \newline
12. दे॒वा अ॑कामयन्ता कामयन्त दे॒वा दे॒वा अ॑कामयन्त । \newline
13. अ॒का॒म॒य॒न्ता सु॑रा॒ नसु॑रा नकामयन्ता कामय॒न्ता सु॑रान् । \newline
14. असु॑रा॒न् भ्रातृ॑व्या॒न् भ्रातृ॑व्या॒ नसु॑रा॒ नसु॑रा॒न् भ्रातृ॑व्यान् । \newline
15. भ्रातृ॑व्या न॒भ्य॑भि भ्रातृ॑व्या॒न् भ्रातृ॑व्या न॒भि । \newline
16. अ॒भि भ॑वेम भवेमा॒भ्य॑भि भ॑वेम । \newline
17. भ॒वे॒मे तीति॑ भवेम भवे॒मेति॑ । \newline
18. इति॒ ते त इतीति॒ ते । \newline
19. ते॑ ऽजुहवु रजुहवु॒ स्ते ते॑ ऽजुहवुः । \newline
20. अ॒जु॒ह॒वुः॒ सिꣳ॒॒हीः सिꣳ॒॒ही र॑जुहवु रजुहवुः सिꣳ॒॒हीः । \newline
21. सिꣳ॒॒ही र॑स्यसि सिꣳ॒॒हीः सिꣳ॒॒ही र॑सि । \newline
22. अ॒सि॒ स॒प॒त्न॒सा॒ही स॑पत्नसा॒ह्य॑स्यसि सपत्नसा॒ही । \newline
23. स॒प॒त्न॒सा॒ही स्वाहा॒ स्वाहा॑ सपत्नसा॒ही स॑पत्नसा॒ही स्वाहा᳚ । \newline
24. स॒प॒त्न॒सा॒हीति॑ सपत्न - सा॒ही । \newline
25. स्वाहेतीति॒ स्वाहा॒ स्वाहेति॑ । \newline
26. इति॒ ते त इतीति॒ ते । \newline
27. ते ऽसु॑रा॒ नसु॑रा॒न् ते ते ऽसु॑रान् । \newline
28. असु॑रा॒न् भ्रातृ॑व्या॒न् भ्रातृ॑व्या॒ नसु॑रा॒ नसु॑रा॒न् भ्रातृ॑व्यान् । \newline
29. भ्रातृ॑व्या न॒भ्य॑भि भ्रातृ॑व्या॒न् भ्रातृ॑व्या न॒भि । \newline
30. अ॒भ्य॑भवन् नभवन् न॒भ्या᳚(1॒)भ्य॑भवन्न् । \newline
31. अ॒भ॒व॒न् ते ते॑ ऽभवन् नभव॒न् ते । \newline
32. ते ऽसु॑रा॒ नसु॑रा॒न् ते ते ऽसु॑रान् । \newline
33. असु॑रा॒न् भ्रातृ॑व्या॒न् भ्रातृ॑व्या॒ नसु॑रा॒ नसु॑रा॒न् भ्रातृ॑व्यान् । \newline
34. भ्रातृ॑व्या नभि॒भूया॑ भि॒भूय॒ भ्रातृ॑व्या॒न् भ्रातृ॑व्या नभि॒भूय॑ । \newline
35. अ॒भि॒भूया॑ कामयन्ता कामयन्ता भि॒भूया॑ भि॒भूया॑ कामयन्त । \newline
36. अ॒भि॒भूयेत्य॑भि - भूय॑ । \newline
37. अ॒का॒म॒य॒न्त॒ प्र॒जाम् प्र॒जा म॑कामयन्ता कामयन्त प्र॒जाम् । \newline
38. प्र॒जां ॅवि॑न्देमहि विन्देमहि प्र॒जाम् प्र॒जां ॅवि॑न्देमहि । \newline
39. प्र॒जामिति॑ प्र - जाम् । \newline
40. वि॒न्दे॒म॒हीतीति॑ विन्देमहि विन्देम॒हीति॑ । \newline
41. इति॒ ते त इतीति॒ ते । \newline
42. ते॑ ऽजुहवु रजुहवु॒ स्ते ते॑ ऽजुहवुः । \newline
43. अ॒जु॒ह॒वुः॒ सिꣳ॒॒हीः सिꣳ॒॒ही र॑जुहवु रजुहवुः सिꣳ॒॒हीः । \newline
44. सिꣳ॒॒ही र॑स्यसि सिꣳ॒॒हीः सिꣳ॒॒ही र॑सि । \newline
45. अ॒सि॒ सु॒प्र॒जा॒वनिः॑ सुप्रजा॒वनि॑ रस्यसि सुप्रजा॒वनिः॑ । \newline
46. सु॒प्र॒जा॒वनिः॒ स्वाहा॒ स्वाहा॑ सुप्रजा॒वनिः॑ सुप्रजा॒वनिः॒ स्वाहा᳚ । \newline
47. सु॒प्र॒जा॒वनि॒रिति॑ सुप्रजा - वनिः॑ । \newline
48. स्वाहेतीति॒ स्वाहा॒ स्वाहेति॑ । \newline
49. इति॒ ते त इतीति॒ ते । \newline
50. ते प्र॒जाम् प्र॒जाम् ते ते प्र॒जाम् । \newline
51. प्र॒जा म॑विन्दन्ता विन्दन्त प्र॒जाम् प्र॒जा म॑विन्दन्त । \newline
52. प्र॒जामिति॑ प्र - जाम् । \newline
53. अ॒वि॒न्द॒न्त॒ ते ते॑ ऽविन्दन्ता विन्दन्त॒ ते । \newline
54. ते प्र॒जाम् प्र॒जाम् ते ते प्र॒जाम् । \newline
55. प्र॒जां ॅवि॒त्त्वा वि॒त्त्वा प्र॒जाम् प्र॒जां ॅवि॒त्त्वा । \newline
56. प्र॒जामिति॑ प्र - जाम् । \newline
57. वि॒त्त्वा ऽका॑मयन्ता कामयन्त वि॒त्त्वा वि॒त्त्वा ऽका॑मयन्त । \newline

\textbf{Ghana Paata } \newline

1. सोत्त॑रवे॒दि रु॑त्तरवे॒दिः सा सोत्त॑रवे॒दि र॑ब्रवी दब्रवी दुत्तरवे॒दिः सा सोत्त॑रवे॒दि र॑ब्रवीत् । \newline
2. उ॒त्त॒र॒वे॒दि र॑ब्रवी दब्रवी दुत्तरवे॒दि रु॑त्तरवे॒दि र॑ब्रवी॒थ् सर्वा॒न् थ्सर्वा॑ नब्रवी दुत्तरवे॒दि रु॑त्तरवे॒दि र॑ब्रवी॒थ् सर्वान्॑ । \newline
3. उ॒त्त॒र॒वे॒दिरित्यु॑त्तर - वे॒दिः । \newline
4. अ॒ब्र॒वी॒थ् सर्वा॒न् थ्सर्वा॑ नब्रवी दब्रवी॒थ् सर्वा॒न् मया॒ मया॒ सर्वा॑ नब्रवी दब्रवी॒थ् सर्वा॒न् मया᳚ । \newline
5. सर्वा॒न् मया॒ मया॒ सर्वा॒न् थ्सर्वा॒न् मया॒ कामा॒न् कामा॒न् मया॒ सर्वा॒न् थ्सर्वा॒न् मया॒ कामान्॑ । \newline
6. मया॒ कामा॒न् कामा॒न् मया॒ मया॒ कामा॒न्॒. वि वि कामा॒न् मया॒ मया॒ कामा॒न्॒. वि । \newline
7. कामा॒न्॒. वि वि कामा॒न् कामा॒न् व्य॑श्ञवथा श्ञवथ॒ वि कामा॒न् कामा॒न् व्य॑श्ञवथ । \newline
8. व्य॑श्ञवथा श्ञवथ॒ वि व्य॑श्ञव॒थे तीत्य॑श्ञवथ॒ वि व्य॑श्ञव॒थेति॑ । \newline
9. अ॒श्ञ॒व॒थे तीत्य॑श्ञवथा श्ञव॒थेति॒ ते त इत्य॑श्ञवथा श्ञव॒थेति॒ ते । \newline
10. इति॒ ते त इतीति॒ ते दे॒वा दे॒वा स्त इतीति॒ ते दे॒वाः । \newline
11. ते दे॒वा दे॒वा स्ते ते दे॒वा अ॑कामयन्ता कामयन्त दे॒वा स्ते ते दे॒वा अ॑कामयन्त । \newline
12. दे॒वा अ॑कामयन्ता कामयन्त दे॒वा दे॒वा अ॑कामय॒न्ता सु॑रा॒ नसु॑रा नकामयन्त दे॒वा दे॒वा अ॑कामय॒न्ता सु॑रान् । \newline
13. अ॒का॒म॒य॒न्ता सु॑रा॒ नसु॑रा नकामयन्ता कामय॒न्ता सु॑रा॒न् भ्रातृ॑व्या॒न् भ्रातृ॑व्या॒ नसु॑रा नकामयन्ता कामय॒न्ता सु॑रा॒न् भ्रातृ॑व्यान् । \newline
14. असु॑रा॒न् भ्रातृ॑व्या॒न् भ्रातृ॑व्या॒ नसु॑रा॒ नसु॑रा॒न् भ्रातृ॑व्या न॒भ्य॑भि भ्रातृ॑व्या॒ नसु॑रा॒ नसु॑रा॒न् भ्रातृ॑व्या न॒भि । \newline
15. भ्रातृ॑व्या न॒भ्य॑भि भ्रातृ॑व्या॒न् भ्रातृ॑व्या न॒भि भ॑वेम भवेमा॒भि भ्रातृ॑व्या॒न् भ्रातृ॑व्या न॒भि भ॑वेम । \newline
16. अ॒भि भ॑वेम भवेमा॒ भ्य॑भि भ॑वे॒मे तीति॑ भवेमा॒ भ्य॑भि भ॑वे॒मेति॑ । \newline
17. भ॒वे॒मे तीति॑ भवेम भवे॒मेति॒ ते त इति॑ भवेम भवे॒मेति॒ ते । \newline
18. इति॒ ते त इतीति॒ ते॑ ऽजुहवु रजुहवु॒ स्त इतीति॒ ते॑ ऽजुहवुः । \newline
19. ते॑ ऽजुहवु रजुहवु॒ स्ते ते॑ ऽजुहवुः सिꣳ॒॒हीः सिꣳ॒॒ही र॑जुहवु॒ स्ते ते॑ ऽजुहवुः सिꣳ॒॒हीः । \newline
20. अ॒जु॒ह॒वुः॒ सिꣳ॒॒हीः सिꣳ॒॒ही र॑जुहवु रजुहवुः सिꣳ॒॒ही र॑स्यसि सिꣳ॒॒ही र॑जुहवु रजुहवुः सिꣳ॒॒हीर॑सि । \newline
21. सिꣳ॒॒ही र॑स्यसि सिꣳ॒॒हीः सिꣳ॒॒हीर॑सि सपत्नसा॒ही स॑पत्नसा॒ ह्य॑सि सिꣳ॒॒हीः सिꣳ॒॒हीर॑सि सपत्नसा॒ही । \newline
22. अ॒सि॒ स॒प॒त्न॒सा॒ही स॑पत्नसा॒ ह्य॑स्यसि सपत्नसा॒ही स्वाहा॒ स्वाहा॑ सपत्नसा॒ ह्य॑स्यसि सपत्नसा॒ही स्वाहा᳚ । \newline
23. स॒प॒त्न॒सा॒ही स्वाहा॒ स्वाहा॑ सपत्नसा॒ही स॑पत्नसा॒ही स्वाहे तीति॒ स्वाहा॑ सपत्नसा॒ही स॑पत्नसा॒ही स्वाहेति॑ । \newline
24. स॒प॒त्न॒सा॒हीति॑ सपत्न - सा॒ही । \newline
25. स्वाहेतीति॒ स्वाहा॒ स्वाहेति॒ ते त इति॒ स्वाहा॒ स्वाहेति॒ ते । \newline
26. इति॒ ते त इतीति॒ ते ऽसु॑रा॒ नसु॑रा॒न् त इतीति॒ ते ऽसु॑रान् । \newline
27. ते ऽसु॑रा॒ नसु॑रा॒न् ते ते ऽसु॑रा॒न् भ्रातृ॑व्या॒न् भ्रातृ॑व्या॒ नसु॑रा॒न् ते ते ऽसु॑रा॒न् भ्रातृ॑व्यान् । \newline
28. असु॑रा॒न् भ्रातृ॑व्या॒न् भ्रातृ॑व्या॒ नसु॑रा॒ नसु॑रा॒न् भ्रातृ॑व्या न॒भ्य॑भि भ्रातृ॑व्या॒ -असु॑रा॒ नसु॑रा॒न् भ्रातृ॑व्या न॒भि । \newline
29. भ्रातृ॑व्या न॒भ्य॑भि भ्रातृ॑व्या॒न् भ्रातृ॑व्या न॒भ्य॑भवन् नभवन् न॒भि भ्रातृ॑व्या॒न् भ्रातृ॑व्या न॒भ्य॑भवन्न् । \newline
30. अ॒भ्य॑भवन् नभवन् न॒भ्या᳚(1॒)भ्य॑भव॒न् ते ते॑ ऽभवन् न॒भ्या᳚(1॒)भ्य॑भव॒न् ते । \newline
31. अ॒भ॒व॒न् ते ते॑ ऽभवन् नभव॒न् ते ऽसु॑रा॒ नसु॑रा॒न् ते॑ ऽभवन् नभव॒न् ते ऽसु॑रान् । \newline
32. ते ऽसु॑रा॒ नसु॑रा॒न् ते ते ऽसु॑रा॒न् भ्रातृ॑व्या॒न् भ्रातृ॑व्या॒ नसु॑रा॒न् ते ते ऽसु॑रा॒न् भ्रातृ॑व्यान् । \newline
33. असु॑रा॒न् भ्रातृ॑व्या॒न् भ्रातृ॑व्या॒ नसु॑रा॒ नसु॑रा॒न् भ्रातृ॑व्या नभि॒भूया॑ भि॒भूय॒ भ्रातृ॑व्या॒ नसु॑रा॒ नसु॑रा॒न् भ्रातृ॑व्या नभि॒भूय॑ । \newline
34. भ्रातृ॑व्या नभि॒भूया॑ भि॒भूय॒ भ्रातृ॑व्या॒न् भ्रातृ॑व्या नभि॒भूया॑ कामयन्ता कामयन्ता भि॒भूय॒ भ्रातृ॑व्या॒न् भ्रातृ॑व्या नभि॒भूया॑ कामयन्त । \newline
35. अ॒भि॒भूया॑ कामयन्ता कामयन्ता भि॒भूया॑ भि॒भूया॑ कामयन्त प्र॒जाम् प्र॒जा म॑कामयन्ता भि॒भूया॑ भि॒भूया॑ कामयन्त प्र॒जाम् । \newline
36. अ॒भि॒भूयेत्य॑भि - भूय॑ । \newline
37. अ॒का॒म॒य॒न्त॒ प्र॒जाम् प्र॒जा म॑कामयन्ता कामयन्त प्र॒जां ॅवि॑न्देमहि विन्देमहि प्र॒जा म॑कामयन्ता कामयन्त प्र॒जां ॅवि॑न्देमहि । \newline
38. प्र॒जां ॅवि॑न्देमहि विन्देमहि प्र॒जाम् प्र॒जां ॅवि॑न्देम॒ही तीति॑ विन्देमहि प्र॒जाम् प्र॒जां ॅवि॑न्देम॒हीति॑ । \newline
39. प्र॒जामिति॑ प्र - जाम् । \newline
40. वि॒न्दे॒म॒ही तीति॑ विन्देमहि विन्देम॒हीति॒ ते त इति॑ विन्देमहि विन्देम॒हीति॒ ते । \newline
41. इति॒ ते त इतीति॒ ते॑ ऽजुहवु रजुहवु॒ स्त इतीति॒ ते॑ ऽजुहवुः । \newline
42. ते॑ ऽजुहवु रजुहवु॒ स्ते ते॑ ऽजुहवुः सिꣳ॒॒हीः सिꣳ॒॒ही र॑जुहवु॒ स्ते ते॑ ऽजुहवुः सिꣳ॒॒हीः । \newline
43. अ॒जु॒ह॒वुः॒ सिꣳ॒॒हीः सिꣳ॒॒ही र॑जुहवु रजुहवुः सिꣳ॒॒ही र॑स्यसि सिꣳ॒॒ही र॑जुहवु रजुहवुः सिꣳ॒॒हीर॑सि । \newline
44. सिꣳ॒॒ही र॑स्यसि सिꣳ॒॒हीः सिꣳ॒॒हीर॑सि सुप्रजा॒वनिः॑ सुप्रजा॒वनि॑ रसि सिꣳ॒॒हीः सिꣳ॒॒ही र॑सि सुप्रजा॒वनिः॑ । \newline
45. अ॒सि॒ सु॒प्र॒जा॒वनिः॑ सुप्रजा॒वनि॑ रस्यसि सुप्रजा॒वनिः॒ स्वाहा॒ स्वाहा॑ सुप्रजा॒वनि॑ रस्यसि सुप्रजा॒वनिः॒ स्वाहा᳚ । \newline
46. सु॒प्र॒जा॒वनिः॒ स्वाहा॒ स्वाहा॑ सुप्रजा॒वनिः॑ सुप्रजा॒वनिः॒ स्वाहे तीति॒ स्वाहा॑ सुप्रजा॒वनिः॑ सुप्रजा॒वनिः॒ स्वाहेति॑ । \newline
47. सु॒प्र॒जा॒वनि॒रिति॑ सुप्रजा - वनिः॑ । \newline
48. स्वाहे तीति॒ स्वाहा॒ स्वाहेति॒ ते त इति॒ स्वाहा॒ स्वाहेति॒ ते । \newline
49. इति॒ ते त इतीति॒ ते प्र॒जाम् प्र॒जाम् त इतीति॒ ते प्र॒जाम् । \newline
50. ते प्र॒जाम् प्र॒जाम् ते ते प्र॒जा म॑विन्दन्ता विन्दन्त प्र॒जाम् ते ते प्र॒जा म॑विन्दन्त । \newline
51. प्र॒जा म॑विन्दन्ता विन्दन्त प्र॒जाम् प्र॒जा म॑विन्दन्त॒ ते ते॑ ऽविन्दन्त प्र॒जाम् प्र॒जा म॑विन्दन्त॒ ते । \newline
52. प्र॒जामिति॑ प्र - जाम् । \newline
53. अ॒वि॒न्द॒न्त॒ ते ते॑ ऽविन्दन्ता विन्दन्त॒ ते प्र॒जाम् प्र॒जाम् ते॑ ऽविन्दन्ता विन्दन्त॒ ते प्र॒जाम् । \newline
54. ते प्र॒जाम् प्र॒जाम् ते ते प्र॒जां ॅवि॒त्त्वा वि॒त्त्वा प्र॒जाम् ते ते प्र॒जां ॅवि॒त्त्वा । \newline
55. प्र॒जां ॅवि॒त्त्वा वि॒त्त्वा प्र॒जाम् प्र॒जां ॅवि॒त्त्वा ऽका॑मयन्ता कामयन्त वि॒त्त्वा प्र॒जाम् प्र॒जां ॅवि॒त्त्वा ऽका॑मयन्त । \newline
56. प्र॒जामिति॑ प्र - जाम् । \newline
57. वि॒त्त्वा ऽका॑मयन्ता कामयन्त वि॒त्त्वा वि॒त्त्वा ऽका॑मयन्त प॒शून् प॒शू न॑कामयन्त वि॒त्त्वा वि॒त्त्वा ऽका॑मयन्त प॒शून् । \newline
\pagebreak
\markright{ TS 6.2.8.2  \hfill https://www.vedavms.in \hfill}

\section{ TS 6.2.8.2 }

\textbf{TS 6.2.8.2 } \newline
\textbf{Samhita Paata} \newline

ऽका॑मयन्त प॒शून्. वि॑न्देम॒हीति॒ ते॑ऽजुहवुः सिꣳ॒॒हीर॑सि रायस्पोष॒वनिः॒ स्वाहेति॒ ते प॒शून॑विन्दन्त॒ ते प॒शून्. वि॒त्त्वाऽका॑मयन्त प्रति॒ष्ठां ॅवि॑न्देम॒हीति॒ ते॑ऽजुहवुः सिꣳ॒॒ही-र॑स्यादित्य॒वनिः॒ स्वाहेति॒ त इ॒मां प्र॑ति॒ष्ठाम॑विन्दन्त॒ त इ॒मां प्र॑ति॒ष्ठां ॅवि॒त्त्वाऽका॑मयन्त दे॒वता॑ आ॒शिष॒ उपे॑या॒मेति॒ ते॑ऽजुहवुः सिꣳ॒॒हीर॒स्या व॑ह दे॒वान् दे॑वय॒ते- [  ] \newline

\textbf{Pada Paata} \newline

अ॒का॒म॒य॒न्त॒ । प॒शून् । वि॒न्दे॒म॒हि॒ । इति॑ । ते । अ॒जु॒ह॒वुः॒ । सिꣳ॒॒हीः । अ॒सि॒ । रा॒य॒स्पो॒ष॒वनि॒रिति॑ रायस्पोष - वनिः॑ । स्वाहा᳚ । इति॑ । ते । प॒शून् । अ॒वि॒न्द॒न्त॒ । ते । प॒शून् । वि॒त्त्वा । अ॒का॒म॒य॒न्त॒ । प्र॒ति॒ष्ठामिति॑ प्रति - स्थाम् । वि॒न्दे॒म॒हि॒ । इति॑ । ते । अ॒जु॒ह॒वुः॒ । सिꣳ॒॒हीः । अ॒सि॒ । आ॒दि॒त्य॒वनि॒रित्या॑दित्य - वनिः॑ । स्वाहा᳚ । इति॑ । ते । इ॒माम् । प्र॒ति॒ष्ठामिति॑ प्रति - स्थाम् । अ॒वि॒न्द॒न्त॒ । ते । इ॒माम् । प्र॒ति॒ष्ठामिति॑ प्रति - स्थाम् । वि॒त्त्वा । अ॒का॒म॒य॒न्त॒ । दे॒वताः᳚ । आ॒शिष॒ इत्या᳚ - शिषः॑ । उपेति॑ । इ॒या॒म॒ । इति॑ । ते । अ॒जु॒ह॒वुः॒ । सिꣳ॒॒हीः । अ॒सि॒ । एति॑ । व॒ह॒ । दे॒वान् । दे॒व॒य॒त इति॑ देव - य॒ते ।  \newline


\textbf{Krama Paata} \newline

अ॒का॒म॒य॒न्त॒ प॒शून् । प॒शून्. वि॑न्देमहि । वि॒न्दे॒म॒हीति॑ । इति॒ ते । ते॑ऽजुहवुः । अ॒जु॒ह॒वुः॒ सिꣳ॒॒हीः । 
सिꣳ॒॒हीर॑सि । अ॒सि॒ रा॒य॒स्पो॒ष॒वनिः॑ । रा॒य॒स्पो॒ष॒वनिः॒ स्वाहा᳚ । रा॒य॒स्पो॒ष॒वनि॒रिति॑ रायस्पोष - वनिः॑ । स्वाहेति॑ । इति॒ ते । ते प॒शून् । प॒शून॑विन्दन्त । अ॒वि॒न्द॒न्त॒ ते । ते प॒शून् । प॒शून्. वि॒त्वा । वि॒त्वाऽका॑मयन्त । अ॒का॒म॒य॒न्त॒ प्र॒ति॒ष्ठाम् । प्र॒ति॒ष्ठाम् ॅवि॑न्देमहि । प्र॒ति॒ष्ठामिति॑ प्रति - स्थाम् । वि॒न्दे॒म॒हीति॑ । इति॒ ते । ते॑ऽजुहवुः । अ॒जु॒ह॒वुः॒ सिꣳ॒॒हीः । सिꣳ॒॒हीर॑सि । अ॒स्या॒दि॒त्य॒वनिः॑ । आ॒दि॒त्य॒वनिः॒ स्वाहा᳚ । आ॒दि॒त्य॒वनि॒रित्या॑दित्य - वनिः॑ । स्वाहेति॑ । इति॒ ते । त इ॒माम् । इ॒माम् प्र॑ति॒ष्ठाम् । प्र॒ति॒ष्ठाम॑विन्दन्त । प्र॒ति॒ष्ठामिति॑ प्रति - स्थाम् । अ॒वि॒न्द॒न्त॒ ते । त इ॒माम् । इ॒माम् प्र॑ति॒ष्ठाम् । प्र॒ति॒ष्ठाम् ॅवि॒त्वा । प्र॒ति॒ष्ठामिति॑ प्रति - स्थाम् । वि॒त्वाऽका॑मयन्त । अ॒का॒म॒य॒न्त॒ दे॒वताः᳚ । दे॒वता॑ आ॒शिषः॑ । आ॒शिष॒ उप॑ । आ॒शिष॒ इत्या᳚ - शिषः॑ । उपे॑याम । इ॒या॒मेति॑ । इति॒ ते । ते॑ऽजुहवुः । अ॒जु॒ह॒वुः॒ सिꣳ॒॒हीः । सिꣳ॒॒हीर॑सि । अ॒स्या । आ व॑ह । व॒ह॒ दे॒वान् । दे॒वान् दे॑वय॒ते । दे॒व॒य॒ते यज॑मानाय । दे॒व॒य॒त इति॑ देव - य॒ते \newline

\textbf{Jatai Paata} \newline

1. अ॒का॒म॒य॒न्त॒ प॒शून् प॒शू न॑कामयन्ता कामयन्त प॒शून् । \newline
2. प॒शून्. वि॑न्देमहि विन्देमहि प॒शून् प॒शून्. वि॑न्देमहि । \newline
3. वि॒न्दे॒म॒हीतीति॑ विन्देमहि विन्देम॒हीति॑ । \newline
4. इति॒ ते त इतीति॒ ते । \newline
5. ते॑ ऽजुहवु रजुहवु॒ स्ते ते॑ ऽजुहवुः । \newline
6. अ॒जु॒ह॒वुः॒ सिꣳ॒॒हीः सिꣳ॒॒ही र॑जुहवु रजुहवुः सिꣳ॒॒हीः । \newline
7. सिꣳ॒॒ही र॑स्यसि सिꣳ॒॒हीः सिꣳ॒॒ही र॑सि । \newline
8. अ॒सि॒ रा॒य॒स्पो॒ष॒वनी॑ रायस्पोष॒वनि॑ रस्यसि रायस्पोष॒वनिः॑ । \newline
9. रा॒य॒स्पो॒ष॒वनिः॒ स्वाहा॒ स्वाहा॑ रायस्पोष॒वनी॑ रायस्पोष॒वनिः॒ स्वाहा᳚ । \newline
10. रा॒य॒स्पो॒ष॒वनि॒रिति॑ रायस्पोष - वनिः॑ । \newline
11. स्वाहेतीति॒ स्वाहा॒ स्वाहेति॑ । \newline
12. इति॒ ते त इतीति॒ ते । \newline
13. ते प॒शून् प॒शून् ते ते प॒शून् । \newline
14. प॒शू न॑विन्दन्ता विन्दन्त प॒शून् प॒शू न॑विन्दन्त । \newline
15. अ॒वि॒न्द॒न्त॒ ते ते॑ ऽविन्दन्ता विन्दन्त॒ ते । \newline
16. ते प॒शून् प॒शून् ते ते प॒शून् । \newline
17. प॒शून्. वि॒त्त्वा वि॒त्त्वा प॒शून् प॒शून्. वि॒त्त्वा । \newline
18. वि॒त्त्वा ऽका॑मयन्ता कामयन्त वि॒त्त्वा वि॒त्त्वा ऽका॑मयन्त । \newline
19. अ॒का॒म॒य॒न्त॒ प्र॒ति॒ष्ठाम् प्र॑ति॒ष्ठा म॑कामयन्ता कामयन्त प्रति॒ष्ठाम् । \newline
20. प्र॒ति॒ष्ठां ॅवि॑न्देमहि विन्देमहि प्रति॒ष्ठाम् प्र॑ति॒ष्ठां ॅवि॑न्देमहि । \newline
21. प्र॒ति॒ष्ठामिति॑ प्रति - स्थाम् । \newline
22. वि॒न्दे॒म॒हीतीति॑ विन्देमहि विन्देम॒हीति॑ । \newline
23. इति॒ ते त इतीति॒ ते । \newline
24. ते॑ ऽजुहवु रजुहवु॒ स्ते ते॑ ऽजुहवुः । \newline
25. अ॒जु॒ह॒वुः॒ सिꣳ॒॒हीः सिꣳ॒॒ही र॑जुहवु रजुहवुः सिꣳ॒॒हीः । \newline
26. सिꣳ॒॒ही र॑स्यसि सिꣳ॒॒हीः सिꣳ॒॒ही र॑सि । \newline
27. अ॒स्या॒ दि॒त्य॒वनि॑ रादित्य॒वनि॑ रस्यस्या दित्य॒वनिः॑ । \newline
28. आ॒दि॒त्य॒वनिः॒ स्वाहा॒ स्वाहा॑ ऽऽदित्य॒वनि॑ रादित्य॒वनिः॒ स्वाहा᳚ । \newline
29. आ॒दि॒त्य॒वनि॒रित्या॑दित्य - वनिः॑ । \newline
30. स्वाहेतीति॒ स्वाहा॒ स्वाहेति॑ । \newline
31. इति॒ ते त इतीति॒ ते । \newline
32. त इ॒मा मि॒माम् ते त इ॒माम् । \newline
33. इ॒माम् प्र॑ति॒ष्ठाम् प्र॑ति॒ष्ठा मि॒मा मि॒माम् प्र॑ति॒ष्ठाम् । \newline
34. प्र॒ति॒ष्ठा म॑विन्दन्ता विन्दन्त प्रति॒ष्ठाम् प्र॑ति॒ष्ठा म॑विन्दन्त । \newline
35. प्र॒ति॒ष्ठामिति॑ प्रति - स्थाम् । \newline
36. अ॒वि॒न्द॒न्त॒ ते ते॑ ऽविन्दन्ता विन्दन्त॒ ते । \newline
37. त इ॒मा मि॒माम् ते त इ॒माम् । \newline
38. इ॒माम् प्र॑ति॒ष्ठाम् प्र॑ति॒ष्ठा मि॒मा मि॒माम् प्र॑ति॒ष्ठाम् । \newline
39. प्र॒ति॒ष्ठां ॅवि॒त्त्वा वि॒त्त्वा प्र॑ति॒ष्ठाम् प्र॑ति॒ष्ठां ॅवि॒त्त्वा । \newline
40. प्र॒ति॒ष्ठामिति॑ प्रति - स्थाम् । \newline
41. वि॒त्त्वा ऽका॑मयन्ता कामयन्त वि॒त्त्वा वि॒त्त्वा ऽका॑मयन्त । \newline
42. अ॒का॒म॒य॒न्त॒ दे॒वता॑ दे॒वता॑ अकामयन्ता कामयन्त दे॒वताः᳚ । \newline
43. दे॒वता॑ आ॒शिष॑ आ॒शिषो॑ दे॒वता॑ दे॒वता॑ आ॒शिषः॑ । \newline
44. आ॒शिष॒ उपोपा॒शिष॑ आ॒शिष॒ उप॑ । \newline
45. आ॒शिष॒ इत्या᳚ - शिषः॑ । \newline
46. उपे॑ यामे या॒मोपोपे॑ याम । \newline
47. इ॒या॒मे तीती॑यामे या॒मेति॑ । \newline
48. इति॒ ते त इतीति॒ ते । \newline
49. ते॑ ऽजुहवु रजुहवु॒ स्ते ते॑ ऽजुहवुः । \newline
50. अ॒जु॒ह॒वुः॒ सिꣳ॒॒हीः सिꣳ॒॒ही र॑जुहवु रजुहवुः सिꣳ॒॒हीः । \newline
51. सिꣳ॒॒ही र॑स्यसि सिꣳ॒॒हीः सिꣳ॒॒ही र॑सि । \newline
52. अ॒स्या ऽस्य॒स्या । \newline
53. आ व॑ह व॒हा व॑ह । \newline
54. व॒ह॒ दे॒वान् दे॒वान्. व॑ह वह दे॒वान् । \newline
55. दे॒वान् दे॑वय॒ते दे॑वय॒ते दे॒वान् दे॒वान् दे॑वय॒ते । \newline
56. दे॒व॒य॒ते यज॑मानाय॒ यज॑मानाय देवय॒ते दे॑वय॒ते यज॑मानाय । \newline
57. दे॒व॒य॒त इति॑ देव - य॒ते । \newline

\textbf{Ghana Paata } \newline

1. अ॒का॒म॒य॒न्त॒ प॒शून् प॒शू न॑कामयन्ता कामयन्त प॒शून्. वि॑न्देमहि विन्देमहि प॒शू न॑कामयन्ता कामयन्त प॒शून्. वि॑न्देमहि । \newline
2. प॒शून्. वि॑न्देमहि विन्देमहि प॒शून् प॒शून्. वि॑न्देम॒ही तीति॑ विन्देमहि प॒शून् प॒शून्. वि॑न्देम॒हीति॑ । \newline
3. वि॒न्दे॒म॒ही तीति॑ विन्देमहि विन्देम॒हीति॒ ते त इति॑ विन्देमहि विन्देम॒हीति॒ ते । \newline
4. इति॒ ते त इतीति॒ ते॑ ऽजुहवु रजुहवु॒ स्त इतीति॒ ते॑ ऽजुहवुः । \newline
5. ते॑ ऽजुहवु रजुहवु॒ स्ते ते॑ ऽजुहवुः सिꣳ॒॒हीः सिꣳ॒॒ही र॑जुहवु॒ स्ते ते॑ ऽजुहवुः सिꣳ॒॒हीः । \newline
6. अ॒जु॒ह॒वुः॒ सिꣳ॒॒हीः सिꣳ॒॒ही र॑जुहवु रजुहवुः सिꣳ॒॒ही र॑स्यसि सिꣳ॒॒ही र॑जुहवु रजुहवुः सिꣳ॒॒हीर॑सि । \newline
7. सिꣳ॒॒ही र॑स्यसि सिꣳ॒॒हीः सिꣳ॒॒ही र॑सि रायस्पोष॒वनी॑ रायस्पोष॒वनि॑ रसि सिꣳ॒॒हीः सिꣳ॒॒ही र॑सि रायस्पोष॒वनिः॑ । \newline
8. अ॒सि॒ रा॒य॒स्पो॒ष॒वनी॑ रायस्पोष॒वनि॑ रस्यसि रायस्पोष॒वनिः॒ स्वाहा॒ स्वाहा॑ रायस्पोष॒वनि॑ रस्यसि रायस्पोष॒वनिः॒ स्वाहा᳚ । \newline
9. रा॒य॒स्पो॒ष॒वनिः॒ स्वाहा॒ स्वाहा॑ रायस्पोष॒वनी॑ रायस्पोष॒वनिः॒ स्वाहे तीति॒ स्वाहा॑ रायस्पोष॒वनी॑ रायस्पोष॒वनिः॒ स्वाहेति॑ । \newline
10. रा॒य॒स्पो॒ष॒वनि॒रिति॑ रायस्पोष - वनिः॑ । \newline
11. स्वाहेतीति॒ स्वाहा॒ स्वाहेति॒ ते त इति॒ स्वाहा॒ स्वाहेति॒ ते । \newline
12. इति॒ ते त इतीति॒ ते प॒शून् प॒शून् त इतीति॒ ते प॒शून् । \newline
13. ते प॒शून् प॒शून् ते ते प॒शू न॑विन्दन्ता विन्दन्त प॒शून् ते ते प॒शू न॑विन्दन्त । \newline
14. प॒शू न॑विन्दन्ता विन्दन्त प॒शून् प॒शून॑ विन्दन्त॒ ते ते॑ ऽविन्दन्त प॒शून् प॒शू न॑विन्दन्त॒ ते । \newline
15. अ॒वि॒न्द॒न्त॒ ते ते॑ ऽविन्दन्ता विन्दन्त॒ ते प॒शून् प॒शून् ते॑ ऽविन्दन्ता विन्दन्त॒ ते प॒शून् । \newline
16. ते प॒शून् प॒शून् ते ते प॒शून्. वि॒त्त्वा वि॒त्त्वा प॒शून् ते ते प॒शून्. वि॒त्त्वा । \newline
17. प॒शून्. वि॒त्त्वा वि॒त्त्वा प॒शून् प॒शून्. वि॒त्त्वा ऽका॑मयन्ता कामयन्त वि॒त्त्वा प॒शून् प॒शून्. वि॒त्त्वा ऽका॑मयन्त । \newline
18. वि॒त्त्वा ऽका॑मयन्ता कामयन्त वि॒त्त्वा वि॒त्त्वा ऽका॑मयन्त प्रति॒ष्ठाम् प्र॑ति॒ष्ठा म॑कामयन्त वि॒त्त्वा वि॒त्त्वा ऽका॑मयन्त प्रति॒ष्ठाम् । \newline
19. अ॒का॒म॒य॒न्त॒ प्र॒ति॒ष्ठाम् प्र॑ति॒ष्ठा म॑कामयन्ता कामयन्त प्रति॒ष्ठां ॅवि॑न्देमहि विन्देमहि प्रति॒ष्ठा म॑कामयन्ता कामयन्त प्रति॒ष्ठां ॅवि॑न्देमहि । \newline
20. प्र॒ति॒ष्ठां ॅवि॑न्देमहि विन्देमहि प्रति॒ष्ठाम् प्र॑ति॒ष्ठां ॅवि॑न्देम॒ही तीति॑ विन्देमहि प्रति॒ष्ठाम् प्र॑ति॒ष्ठां ॅवि॑न्देम॒हीति॑ । \newline
21. प्र॒ति॒ष्ठामिति॑ प्रति - स्थाम् । \newline
22. वि॒न्दे॒म॒ही तीति॑ विन्देमहि विन्देम॒हीति॒ ते त इति॑ विन्देमहि विन्देम॒हीति॒ ते । \newline
23. इति॒ ते त इतीति॒ ते॑ ऽजुहवु रजुहवु॒ स्त इतीति॒ ते॑ ऽजुहवुः । \newline
24. ते॑ ऽजुहवु रजुहवु॒ स्ते ते॑ ऽजुहवुः सिꣳ॒॒हीः सिꣳ॒॒ही र॑जुहवु॒ स्ते ते॑ ऽजुहवुः सिꣳ॒॒हीः । \newline
25. अ॒जु॒ह॒वुः॒ सिꣳ॒॒हीः सिꣳ॒॒ही र॑जुहवु रजुहवुः सिꣳ॒॒ही र॑स्यसि सिꣳ॒॒ही र॑जुहवु रजुहवुः सिꣳ॒॒हीर॑सि । \newline
26. सिꣳ॒॒ही र॑स्यसि सिꣳ॒॒हीः सिꣳ॒॒ही र॑स्यादित्य॒वनि॑ रादित्य॒वनि॑ रसि सिꣳ॒॒हीः सिꣳ॒॒ही र॑स्यादित्य॒वनिः॑ । \newline
27. अ॒स्या॒दि॒त्य॒वनि॑ रादित्य॒वनि॑ रस्य स्यादित्य॒वनिः॒ स्वाहा॒ स्वाहा॑ ऽऽदित्य॒वनि॑ रस्य स्यादित्य॒वनिः॒ स्वाहा᳚ । \newline
28. आ॒दि॒त्य॒वनिः॒ स्वाहा॒ स्वाहा॑ ऽऽदित्य॒वनि॑ रादित्य॒वनिः॒ स्वाहे तीति॒ स्वाहा॑ ऽऽदित्य॒वनि॑ रादित्य॒वनिः॒ स्वाहेति॑ । \newline
29. आ॒दि॒त्य॒वनि॒रित्या॑दित्य - वनिः॑ । \newline
30. स्वाहेतीति॒ स्वाहा॒ स्वाहेति॒ ते त इति॒ स्वाहा॒ स्वाहेति॒ ते । \newline
31. इति॒ ते त इतीति॒ त इ॒मा मि॒माम् त इतीति॒ त इ॒माम् । \newline
32. त इ॒मा मि॒माम् ते त इ॒माम् प्र॑ति॒ष्ठाम् प्र॑ति॒ष्ठा मि॒माम् ते त इ॒माम् प्र॑ति॒ष्ठाम् । \newline
33. इ॒माम् प्र॑ति॒ष्ठाम् प्र॑ति॒ष्ठा मि॒मा मि॒माम् प्र॑ति॒ष्ठा म॑विन्दन्ता विन्दन्त प्रति॒ष्ठा मि॒मा मि॒माम् प्र॑ति॒ष्ठा म॑विन्दन्त । \newline
34. प्र॒ति॒ष्ठा म॑विन्दन्ता विन्दन्त प्रति॒ष्ठाम् प्र॑ति॒ष्ठा म॑विन्दन्त॒ ते ते॑ ऽविन्दन्त प्रति॒ष्ठाम् प्र॑ति॒ष्ठा म॑विन्दन्त॒ ते । \newline
35. प्र॒ति॒ष्ठामिति॑ प्रति - स्थाम् । \newline
36. अ॒वि॒न्द॒न्त॒ ते ते॑ ऽविन्दन्ता विन्दन्त॒ त इ॒मा मि॒माम् ते॑ ऽविन्दन्ता विन्दन्त॒ त इ॒माम् । \newline
37. त इ॒मा मि॒माम् ते त इ॒माम् प्र॑ति॒ष्ठाम् प्र॑ति॒ष्ठा मि॒माम् ते त इ॒माम् प्र॑ति॒ष्ठाम् । \newline
38. इ॒माम् प्र॑ति॒ष्ठाम् प्र॑ति॒ष्ठा मि॒मा मि॒माम् प्र॑ति॒ष्ठां ॅवि॒त्त्वा वि॒त्त्वा प्र॑ति॒ष्ठा मि॒मा मि॒माम् प्र॑ति॒ष्ठां ॅवि॒त्त्वा । \newline
39. प्र॒ति॒ष्ठां ॅवि॒त्त्वा वि॒त्त्वा प्र॑ति॒ष्ठाम् प्र॑ति॒ष्ठां ॅवि॒त्त्वा ऽका॑मयन्ता कामयन्त वि॒त्त्वा प्र॑ति॒ष्ठाम् प्र॑ति॒ष्ठां ॅवि॒त्त्वा ऽका॑मयन्त । \newline
40. प्र॒ति॒ष्ठामिति॑ प्रति - स्थाम् । \newline
41. वि॒त्त्वा ऽका॑मयन्ता कामयन्त वि॒त्त्वा वि॒त्त्वा ऽका॑मयन्त दे॒वता॑ दे॒वता॑ अकामयन्त वि॒त्त्वा वि॒त्त्वा ऽका॑मयन्त दे॒वताः᳚ । \newline
42. अ॒का॒म॒य॒न्त॒ दे॒वता॑ दे॒वता॑ अकामयन्ता कामयन्त दे॒वता॑ आ॒शिष॑ आ॒शिषो॑ दे॒वता॑ अकामयन्ता कामयन्त दे॒वता॑ आ॒शिषः॑ । \newline
43. दे॒वता॑ आ॒शिष॑ आ॒शिषो॑ दे॒वता॑ दे॒वता॑ आ॒शिष॒ उपोपा॒शिषो॑ दे॒वता॑ दे॒वता॑ आ॒शिष॒ उप॑ । \newline
44. आ॒शिष॒ उपोपा॒शिष॑ आ॒शिष॒ उपे॑ यामे या॒मोपा॒शिष॑ आ॒शिष॒ उपे॑ याम । \newline
45. आ॒शिष॒ इत्या᳚ - शिषः॑ । \newline
46. उपे॑ यामे या॒मोपोपे॑ या॒मे तीती॑या॒ मोपोपे॑ या॒मेति॑ । \newline
47. इ॒या॒मे तीती॑यामे या॒मेति॒ ते त इती॑यामे या॒मेति॒ ते । \newline
48. इति॒ ते त इतीति॒ ते॑ ऽजुहवु रजुहवु॒ स्त इतीति॒ ते॑ ऽजुहवुः । \newline
49. ते॑ ऽजुहवु रजुहवु॒ स्ते ते॑ ऽजुहवुः सिꣳ॒॒हीः सिꣳ॒॒ही र॑जुहवु॒ स्ते ते॑ ऽजुहवुः सिꣳ॒॒हीः । \newline
50. अ॒जु॒ह॒वुः॒ सिꣳ॒॒हीः सिꣳ॒॒ही र॑जुहवु रजुहवुः सिꣳ॒॒ही र॑स्यसि सिꣳ॒॒ही र॑जुहवु रजुहवुः सिꣳ॒॒हीर॑सि । \newline
51. सिꣳ॒॒ही र॑स्यसि सिꣳ॒॒हीः सिꣳ॒॒ही र॒स्या ऽसि॑ सिꣳ॒॒हीः सिꣳ॒॒ही र॒स्या । \newline
52. अ॒स्या ऽस्य॒स्या व॑ह व॒हा ऽस्य॒स्या व॑ह । \newline
53. आ व॑ह व॒हा व॑ह दे॒वान् दे॒वान्. व॒हा व॑ह दे॒वान् । \newline
54. व॒ह॒ दे॒वान् दे॒वान्. व॑ह वह दे॒वान् दे॑वय॒ते दे॑वय॒ते दे॒वान्. व॑ह वह दे॒वान् दे॑वय॒ते । \newline
55. दे॒वान् दे॑वय॒ते दे॑वय॒ते दे॒वान् दे॒वान् दे॑वय॒ते यज॑मानाय॒ यज॑मानाय देवय॒ते दे॒वान् दे॒वान् दे॑वय॒ते यज॑मानाय । \newline
56. दे॒व॒य॒ते यज॑मानाय॒ यज॑मानाय देवय॒ते दे॑वय॒ते यज॑मानाय॒ स्वाहा॒ स्वाहा॒ यज॑मानाय देवय॒ते दे॑वय॒ते यज॑मानाय॒ स्वाहा᳚ । \newline
57. दे॒व॒य॒त इति॑ देव - य॒ते । \newline
\pagebreak
\markright{ TS 6.2.8.3  \hfill https://www.vedavms.in \hfill}

\section{ TS 6.2.8.3 }

\textbf{TS 6.2.8.3 } \newline
\textbf{Samhita Paata} \newline

यज॑मानाय॒ स्वाहेति॒ ते दे॒वता॑ आ॒शिष॒ उपा॑य॒न् पञ्च॒ कृत्वो॒ व्याघा॑रयति॒ पञ्चा᳚क्षरा प॒ङ्क्तिः पाङ्क्तो॑ य॒ज्ञो य॒ज्ञ्मे॒वाव॑ रुन्धे ऽक्ष्ण॒या व्याघा॑रयति॒ तस्मा॑दक्ष्ण॒या प॒शवोऽङ्गा॑नि॒ प्रह॑रन्ति॒ प्रति॑ष्ठित्यै भू॒तेभ्य॒स्त्वेति॒ स्रुच॒मुद्गृ॑ह्णाति॒ य ए॒व दे॒वा भू॒तास्तेषां॒ तद्-भा॑ग॒धेयं॒ ताने॒व तेन॑ प्रीणाति॒ पौतु॑द्रवान् परि॒धीन् परि॑ दधात्ये॒षां- [  ] \newline

\textbf{Pada Paata} \newline

यज॑मानाय । स्वाहा᳚ । इति॑ । ते । दे॒वताः᳚ । आ॒शिष॒ इत्या᳚ - शिषः॑ । उपेति॑ । आ॒य॒न्न् । पञ्च॑ । कृत्वः॑ । व्याघा॑रय॒तीति॑ वि - आघा॑रयति । पञ्चा᳚क्ष॒रेति॒ पञ्च॑-अ॒क्ष॒रा॒ । प॒ङ्क्तिः । पाङ्क्तः॑ । य॒ज्ञ्ः । य॒ज्ञ्म् । ए॒व । अवेति॑ । रु॒न्धे॒ । अ॒क्ष्ण॒या । व्याघा॑रय॒तीति॑ वि-आघा॑रयति । तस्मा᳚त् । अ॒क्ष्ण॒या । प॒शवः॑ । अङ्गा॑नि । प्रेति॑ । ह॒र॒न्ति॒ । प्रति॑ष्ठित्या॒ इति॒ प्रति॑ - स्थि॒त्यै॒ । भू॒तेभ्यः॑ । त्वा॒ । इति॑ । स्रुच᳚म् । उदिति॑ । गृ॒ह्णा॒ति॒ । ये । ए॒व । दे॒वाः । भू॒ताः । तेषा᳚म् । तत् । भा॒ग॒धेय॒मिति॑ भाग-धेय᳚म् । तान् । ए॒व । तेन॑ । प्री॒णा॒ति॒ । पौतु॑द्रवान् । प॒रि॒धीनिति॑ परि - धीन् । परीति॑ । द॒धा॒ति॒ । ए॒षाम् ।  \newline


\textbf{Krama Paata} \newline

यज॑मानाय॒ स्वाहा᳚ । स्वाहेति॑ । इति॒ ते । ते दे॒वताः᳚ । दे॒वता॑ आ॒शिषः॑ । आ॒शिष॒ उप॑ । आ॒शिष॒ इत्या᳚ - शिषः॑ । उपा॑यन्न् । आ॒य॒न् पञ्च॑ । पञ्च॒ कृत्वः॑ । कृत्वो॒ व्याघा॑रयति । व्याघा॑रयति॒ पञ्चा᳚क्षरा । व्याघा॑रय॒तीति॑ वि - आघा॑रयति । पञ्चा᳚क्षरा प॒ङ्‍क्तिः । पञ्चा᳚क्ष॒रेति॒ पञ्च॑ - अ॒क्ष॒रा॒ । प॒ङ्‍क्तिः पाङ्‍क्तः॑ । पाङ्‍क्तो॑ य॒ज्ञ्ः । य॒ज्ञो य॒ज्ञ्म् । य॒ज्ञ्मे॒व । ए॒वाव॑ । अव॑ रुन्धे । रु॒न्धे॒ऽक्ष्ण॒या । अ॒क्ष्ण॒या व्याघा॑रयति । व्याघा॑रयति॒ तस्मा᳚त् । व्याघा॑रय॒तीति॑ वि - आघा॑रयति । तस्मा॑दक्ष्ण॒या । अ॒क्ष्ण॒या प॒शवः॑ । प॒शवोऽङ्‍गा॑नि । अङ्‍गा॑नि॒ प्र । प्र ह॑रन्ति । ह॒र॒न्ति॒ प्रति॑ष्ठित्यै । प्रति॑ष्ठित्यै भू॒तेभ्यः॑ । प्रति॑ष्ठित्या॒ इति॒ प्रति॑ - स्थि॒त्यै॒ । भू॒तेभ्य॑स्त्वा । त्वेति॑ । इति॒ स्रुच᳚म् । स्रुच॒मुत् । उद् गृ॑ह्णाति । गृ॒ह्णा॒ति॒ ये । य ए॒व । ए॒व दे॒वाः । दे॒वा भू॒ताः । भू॒ता स्तेषा᳚म् । तेषा॒म् तत् । तद् भा॑ग॒धेय᳚म् । भा॒ग॒धेय॒म् तान् । भा॒ग॒धेय॒मिति॑ भाग - धेय᳚म् । ताने॒व । ए॒व तेन॑ । तेन॑ प्रीणाति । प्री॒णा॒ति॒ पौतु॑द्रवान् । पौतु॑द्रवान् परि॒धीन् । प॒रि॒धीन् परि॑ । प॒रि॒धीनिति॑ परि - धीन् । परि॑ दधाति । द॒धा॒त्ये॒षाम् । ए॒षाम् ॅलो॒काना᳚म् \newline

\textbf{Jatai Paata} \newline

1. यज॑मानाय॒ स्वाहा॒ स्वाहा॒ यज॑मानाय॒ यज॑मानाय॒ स्वाहा᳚ । \newline
2. स्वाहेतीति॒ स्वाहा॒ स्वाहेति॑ । \newline
3. इति॒ ते त इतीति॒ ते । \newline
4. ते दे॒वता॑ दे॒वता॒ स्ते ते दे॒वताः᳚ । \newline
5. दे॒वता॑ आ॒शिष॑ आ॒शिषो॑ दे॒वता॑ दे॒वता॑ आ॒शिषः॑ । \newline
6. आ॒शिष॒ उपोपा॒शिष॑ आ॒शिष॒ उप॑ । \newline
7. आ॒शिष॒ इत्या᳚ - शिषः॑ । \newline
8. उपा॑यन् नाय॒न् नुपोपा॑यन्न् । \newline
9. आ॒य॒न् पञ्च॒ पञ्चा॑यन् नाय॒न् पञ्च॑ । \newline
10. पञ्च॒ कृत्वः॒ कृत्वः॒ पञ्च॒ पञ्च॒ कृत्वः॑ । \newline
11. कृत्वो॒ व्याघा॑रयति॒ व्याघा॑रयति॒ कृत्वः॒ कृत्वो॒ व्याघा॑रयति । \newline
12. व्याघा॑रयति॒ पञ्चा᳚क्षरा॒ पञ्चा᳚क्षरा॒ व्याघा॑रयति॒ व्याघा॑रयति॒ पञ्चा᳚क्षरा । \newline
13. व्याघा॑रय॒तीति॑ वि - आघा॑रयति । \newline
14. पञ्चा᳚क्षरा प॒ङ्क्तिः प॒ङ्क्तिः पञ्चा᳚क्षरा॒ पञ्चा᳚क्षरा प॒ङ्क्तिः । \newline
15. पञ्चा᳚क्ष॒रेति॒ पञ्च॑ - अ॒क्ष॒रा॒ । \newline
16. प॒ङ्क्तिः पाङ्क्तः॒ पाङ्क्तः॑ प॒ङ्क्तिः प॒ङ्क्तिः पाङ्क्तः॑ । \newline
17. पाङ्क्तो॑ य॒ज्ञो य॒ज्ञ्ः पाङ्क्तः॒ पाङ्क्तो॑ य॒ज्ञ्ः । \newline
18. य॒ज्ञो य॒ज्ञ्ं ॅय॒ज्ञ्ं ॅय॒ज्ञो य॒ज्ञो य॒ज्ञ्म् । \newline
19. य॒ज्ञ् मे॒वैव य॒ज्ञ्ं ॅय॒ज्ञ् मे॒व । \newline
20. ए॒वावा वै॒वै वाव॑ । \newline
21. अव॑ रुन्धे रु॒न्धे ऽवाव॑ रुन्धे । \newline
22. रु॒न्धे॒ ऽक्ष्ण॒या ऽक्ष्ण॒या रु॑न्धे रुन्धे ऽक्ष्ण॒या । \newline
23. अ॒क्ष्ण॒या व्याघा॑रयति॒ व्याघा॑रय त्यक्ष्ण॒या ऽक्ष्ण॒या व्याघा॑रयति । \newline
24. व्याघा॑रयति॒ तस्मा॒त् तस्मा॒द् व्याघा॑रयति॒ व्याघा॑रयति॒ तस्मा᳚त् । \newline
25. व्याघा॑रय॒तीति॑ वि - आघा॑रयति । \newline
26. तस्मा॑ दक्ष्ण॒या ऽक्ष्ण॒या तस्मा॒त् तस्मा॑ दक्ष्ण॒या । \newline
27. अ॒क्ष्ण॒या प॒शवः॑ प॒शवो᳚ ऽक्ष्ण॒या ऽक्ष्ण॒या प॒शवः॑ । \newline
28. प॒शवो ऽङ्गा॒ न्यङ्गा॑नि प॒शवः॑ प॒शवो ऽङ्गा॑नि । \newline
29. अङ्गा॑नि॒ प्र प्राङ्गा॒ न्यङ्गा॑नि॒ प्र । \newline
30. प्र ह॑रन्ति हरन्ति॒ प्र प्र ह॑रन्ति । \newline
31. ह॒र॒न्ति॒ प्रति॑ष्ठित्यै॒ प्रति॑ष्ठित्यै हरन्ति हरन्ति॒ प्रति॑ष्ठित्यै । \newline
32. प्रति॑ष्ठित्यै भू॒तेभ्यो॑ भू॒तेभ्यः॒ प्रति॑ष्ठित्यै॒ प्रति॑ष्ठित्यै भू॒तेभ्यः॑ । \newline
33. प्रति॑ष्ठित्या॒ इति॒ प्रति॑ - स्थि॒त्यै॒ । \newline
34. भू॒तेभ्य॑ स्त्वा त्वा भू॒तेभ्यो॑ भू॒तेभ्य॑ स्त्वा । \newline
35. त्वेतीति॑ त्वा॒ त्वेति॑ । \newline
36. इति॒ स्रुचꣳ॒॒ स्रुच॒ मितीति॒ स्रुच᳚म् । \newline
37. स्रुच॒ मुदुथ् स्रुचꣳ॒॒ स्रुच॒ मुत् । \newline
38. उद् गृ॑ह्णाति गृह्णा॒ त्युदुद् गृ॑ह्णाति । \newline
39. गृ॒ह्णा॒ति॒ ये ये गृ॑ह्णाति गृह्णाति॒ ये । \newline
40. य ए॒वैव ये य ए॒व । \newline
41. ए॒व दे॒वा दे॒वा ए॒वैव दे॒वाः । \newline
42. दे॒वा भू॒ता भू॒ता दे॒वा दे॒वा भू॒ताः । \newline
43. भू॒ता स्तेषा॒म् तेषा᳚म् भू॒ता भू॒ता स्तेषा᳚म् । \newline
44. तेषा॒म् तत् तत् तेषा॒म् तेषा॒म् तत् । \newline
45. तद् भा॑ग॒धेय॑म् भाग॒धेय॒म् तत् तद् भा॑ग॒धेय᳚म् । \newline
46. भा॒ग॒धेय॒म् ताꣳ स्तान् भा॑ग॒धेय॑म् भाग॒धेय॒म् तान् । \newline
47. भा॒ग॒धेय॒मिति॑ भाग - धेय᳚म् । \newline
48. ता ने॒वैव ताꣳ स्ता ने॒व । \newline
49. ए॒व तेन॒ तेनै॒वैव तेन॑ । \newline
50. तेन॑ प्रीणाति प्रीणाति॒ तेन॒ तेन॑ प्रीणाति । \newline
51. प्री॒णा॒ति॒ पौतु॑द्रवा॒न् पौतु॑द्रवान् प्रीणाति प्रीणाति॒ पौतु॑द्रवान् । \newline
52. पौतु॑द्रवान् परि॒धीन् प॑रि॒धीन् पौतु॑द्रवा॒न् पौतु॑द्रवान् परि॒धीन् । \newline
53. प॒रि॒धीन् परि॒ परि॑ परि॒धीन् प॑रि॒धीन् परि॑ । \newline
54. प॒रि॒धीनिति॑ परि - धीन् । \newline
55. परि॑ दधाति दधाति॒ परि॒ परि॑ दधाति । \newline
56. द॒धा॒ त्ये॒षा मे॒षाम् द॑धाति दधा त्ये॒षाम् । \newline
57. ए॒षाम् ॅलो॒काना᳚म् ॅलो॒काना॑ मे॒षा मे॒षाम् ॅलो॒काना᳚म् । \newline

\textbf{Ghana Paata } \newline

1. यज॑मानाय॒ स्वाहा॒ स्वाहा॒ यज॑मानाय॒ यज॑मानाय॒ स्वाहे तीति॒ स्वाहा॒ यज॑मानाय॒ यज॑मानाय॒ स्वाहेति॑ । \newline
2. स्वाहेतीति॒ स्वाहा॒ स्वाहेति॒ ते त इति॒ स्वाहा॒ स्वाहेति॒ ते । \newline
3. इति॒ ते त इतीति॒ ते दे॒वता॑ दे॒वता॒ स्त इतीति॒ ते दे॒वताः᳚ । \newline
4. ते दे॒वता॑ दे॒वता॒ स्ते ते दे॒वता॑ आ॒शिष॑ आ॒शिषो॑ दे॒वता॒ स्ते ते दे॒वता॑ आ॒शिषः॑ । \newline
5. दे॒वता॑ आ॒शिष॑ आ॒शिषो॑ दे॒वता॑ दे॒वता॑ आ॒शिष॒ उपोपा॒ शिषो॑ दे॒वता॑ दे॒वता॑ आ॒शिष॒ उप॑ । \newline
6. आ॒शिष॒ उपोपा॒ शिष॑ आ॒शिष॒ उपा॑यन् नाय॒न् नुपा॒शिष॑ आ॒शिष॒ उपा॑यन्न् । \newline
7. आ॒शिष॒ इत्या᳚ - शिषः॑ । \newline
8. उपा॑यन् नाय॒न् नुपो पा॑य॒न् पञ्च॒ पञ्चा॑य॒न् नुपो पा॑य॒न् पञ्च॑ । \newline
9. आ॒य॒न् पञ्च॒ पञ्चा॑यन् नाय॒न् पञ्च॒ कृत्वः॒ कृत्वः॒ पञ्चा॑यन् नाय॒न् पञ्च॒ कृत्वः॑ । \newline
10. पञ्च॒ कृत्वः॒ कृत्वः॒ पञ्च॒ पञ्च॒ कृत्वो॒ व्याघा॑रयति॒ व्याघा॑रयति॒ कृत्वः॒ पञ्च॒ पञ्च॒ कृत्वो॒ व्याघा॑रयति । \newline
11. कृत्वो॒ व्याघा॑रयति॒ व्याघा॑रयति॒ कृत्वः॒ कृत्वो॒ व्याघा॑रयति॒ पञ्चा᳚क्षरा॒ पञ्चा᳚क्षरा॒ व्याघा॑रयति॒ कृत्वः॒ कृत्वो॒ व्याघा॑रयति॒ पञ्चा᳚क्षरा । \newline
12. व्याघा॑रयति॒ पञ्चा᳚क्षरा॒ पञ्चा᳚क्षरा॒ व्याघा॑रयति॒ व्याघा॑रयति॒ पञ्चा᳚क्षरा प॒ङ्क्तिः प॒ङ्क्तिः पञ्चा᳚क्षरा॒ व्याघा॑रयति॒ व्याघा॑रयति॒ पञ्चा᳚क्षरा प॒ङ्क्तिः । \newline
13. व्याघा॑रय॒तीति॑ वि - आघा॑रयति । \newline
14. पञ्चा᳚क्षरा प॒ङ्क्तिः प॒ङ्क्तिः पञ्चा᳚क्षरा॒ पञ्चा᳚क्षरा प॒ङ्क्तिः पाङ्क्तः॒ पाङ्क्तः॑ प॒ङ्क्तिः पञ्चा᳚क्षरा॒ पञ्चा᳚क्षरा प॒ङ्क्तिः पाङ्क्तः॑ । \newline
15. पञ्चा᳚क्ष॒रेति॒ पञ्च॑ - अ॒क्ष॒रा॒ । \newline
16. प॒ङ्क्तिः पाङ्क्तः॒ पाङ्क्तः॑ प॒ङ्क्तिः प॒ङ्क्तिः पाङ्क्तो॑ य॒ज्ञो य॒ज्ञ्ः पाङ्क्तः॑ प॒ङ्क्तिः प॒ङ्क्तिः पाङ्क्तो॑ य॒ज्ञ्ः । \newline
17. पाङ्क्तो॑ य॒ज्ञो य॒ज्ञ्ः पाङ्क्तः॒ पाङ्क्तो॑ य॒ज्ञो य॒ज्ञ्ं ॅय॒ज्ञ्ं ॅय॒ज्ञ्ः पाङ्क्तः॒ पाङ्क्तो॑ य॒ज्ञो य॒ज्ञ्म् । \newline
18. य॒ज्ञो य॒ज्ञ्ं ॅय॒ज्ञ्ं ॅय॒ज्ञो य॒ज्ञो य॒ज्ञ् मे॒वैव य॒ज्ञ्ं ॅय॒ज्ञो य॒ज्ञो य॒ज्ञ् मे॒व । \newline
19. य॒ज्ञ् मे॒वैव य॒ज्ञ्ं ॅय॒ज्ञ् मे॒वावा वै॒व य॒ज्ञ्ं ॅय॒ज्ञ् मे॒वाव॑ । \newline
20. ए॒वावा वै॒वै वाव॑ रुन्धे रु॒न्धे ऽवै॒वै वाव॑ रुन्धे । \newline
21. अव॑ रुन्धे रु॒न्धे ऽवाव॑ रुन्धे ऽक्ष्ण॒या ऽक्ष्ण॒या रु॒न्धे ऽवाव॑ रुन्धे ऽक्ष्ण॒या । \newline
22. रु॒न्धे॒ ऽक्ष्ण॒या ऽक्ष्ण॒या रु॑न्धे रुन्धे ऽक्ष्ण॒या व्याघा॑रयति॒ व्याघा॑रय त्यक्ष्ण॒या रु॑न्धे रुन्धे ऽक्ष्ण॒या व्याघा॑रयति । \newline
23. अ॒क्ष्ण॒या व्याघा॑रयति॒ व्याघा॑रय त्यक्ष्ण॒या ऽक्ष्ण॒या व्याघा॑रयति॒ तस्मा॒त् तस्मा॒द् व्याघा॑रय त्यक्ष्ण॒या ऽक्ष्ण॒या व्याघा॑रयति॒ तस्मा᳚त् । \newline
24. व्याघा॑रयति॒ तस्मा॒त् तस्मा॒द् व्याघा॑रयति॒ व्याघा॑रयति॒ तस्मा॑ दक्ष्ण॒या ऽक्ष्ण॒या तस्मा॒द् व्याघा॑रयति॒ व्याघा॑रयति॒ तस्मा॑ दक्ष्ण॒या । \newline
25. व्याघा॑रय॒तीति॑ वि - आघा॑रयति । \newline
26. तस्मा॑ दक्ष्ण॒या ऽक्ष्ण॒या तस्मा॒त् तस्मा॑ दक्ष्ण॒या प॒शवः॑ प॒शवो᳚ ऽक्ष्ण॒या तस्मा॒त् तस्मा॑ दक्ष्ण॒या प॒शवः॑ । \newline
27. अ॒क्ष्ण॒या प॒शवः॑ प॒शवो᳚ ऽक्ष्ण॒या ऽक्ष्ण॒या प॒शवो ऽङ्गा॒ न्यङ्गा॑नि प॒शवो᳚ ऽक्ष्ण॒या ऽक्ष्ण॒या प॒शवो ऽङ्गा॑नि । \newline
28. प॒शवो ऽङ्गा॒ न्यङ्गा॑नि प॒शवः॑ प॒शवो ऽङ्गा॑नि॒ प्र प्राङ्गा॑नि प॒शवः॑ प॒शवो ऽङ्गा॑नि॒ प्र । \newline
29. अङ्गा॑नि॒ प्र प्राङ्गा॒ न्यङ्गा॑नि॒ प्र ह॑रन्ति हरन्ति॒ प्राङ्गा॒ न्यङ्गा॑नि॒ प्र ह॑रन्ति । \newline
30. प्र ह॑रन्ति हरन्ति॒ प्र प्र ह॑रन्ति॒ प्रति॑ष्ठित्यै॒ प्रति॑ष्ठित्यै हरन्ति॒ प्र प्र ह॑रन्ति॒ प्रति॑ष्ठित्यै । \newline
31. ह॒र॒न्ति॒ प्रति॑ष्ठित्यै॒ प्रति॑ष्ठित्यै हरन्ति हरन्ति॒ प्रति॑ष्ठित्यै भू॒तेभ्यो॑ भू॒तेभ्यः॒ प्रति॑ष्ठित्यै हरन्ति हरन्ति॒ प्रति॑ष्ठित्यै भू॒तेभ्यः॑ । \newline
32. प्रति॑ष्ठित्यै भू॒तेभ्यो॑ भू॒तेभ्यः॒ प्रति॑ष्ठित्यै॒ प्रति॑ष्ठित्यै भू॒तेभ्य॑ स्त्वा त्वा भू॒तेभ्यः॒ प्रति॑ष्ठित्यै॒ प्रति॑ष्ठित्यै भू॒तेभ्य॑ स्त्वा । \newline
33. प्रति॑ष्ठित्या॒ इति॒ प्रति॑ - स्थि॒त्यै॒ । \newline
34. भू॒तेभ्य॑ स्त्वा त्वा भू॒तेभ्यो॑ भू॒तेभ्य॒ स्त्वे तीति॑ त्वा भू॒तेभ्यो॑ भू॒तेभ्य॒ स्त्वेति॑ । \newline
35. त्वेतीति॑ त्वा॒ त्वेति॒ स्रुचꣳ॒॒ स्रुच॒ मिति॑ त्वा॒ त्वेति॒ स्रुच᳚म् । \newline
36. इति॒ स्रुचꣳ॒॒ स्रुच॒ मितीति॒ स्रुच॒ मुदुथ् स्रुच॒ मितीति॒ स्रुच॒ मुत् । \newline
37. स्रुच॒ मुदुथ् स्रुचꣳ॒॒ स्रुच॒ मुद् गृ॑ह्णाति गृह्णा॒त्युथ् स्रुचꣳ॒॒ स्रुच॒ मुद् गृ॑ह्णाति । \newline
38. उद् गृ॑ह्णाति गृह्णा॒ त्युदुद् गृ॑ह्णाति॒ ये ये गृ॑ह्णा॒ त्युदुद् गृ॑ह्णाति॒ ये । \newline
39. गृ॒ह्णा॒ति॒ ये ये गृ॑ह्णाति गृह्णाति॒ य ए॒वैव ये गृ॑ह्णाति गृह्णाति॒ य ए॒व । \newline
40. य ए॒वैव ये य ए॒व दे॒वा दे॒वा ए॒व ये य ए॒व दे॒वाः । \newline
41. ए॒व दे॒वा दे॒वा ए॒वैव दे॒वा भू॒ता भू॒ता दे॒वा ए॒वैव दे॒वा भू॒ताः । \newline
42. दे॒वा भू॒ता भू॒ता दे॒वा दे॒वा भू॒ता स्तेषा॒म् तेषा᳚म् भू॒ता दे॒वा दे॒वा भू॒ता स्तेषा᳚म् । \newline
43. भू॒ता स्तेषा॒म् तेषा᳚म् भू॒ता भू॒ता स्तेषा॒म् तत् तत् तेषा᳚म् भू॒ता भू॒ता स्तेषा॒म् तत् । \newline
44. तेषा॒म् तत् तत् तेषा॒म् तेषा॒म् तद् भा॑ग॒धेय॑म् भाग॒धेय॒म् तत् तेषा॒म् तेषा॒म् तद् भा॑ग॒धेय᳚म् । \newline
45. तद् भा॑ग॒धेय॑म् भाग॒धेय॒म् तत् तद् भा॑ग॒धेय॒म् ताꣳ स्तान् भा॑ग॒धेय॒म् तत् तद् भा॑ग॒धेय॒म् तान् । \newline
46. भा॒ग॒धेय॒म् ताꣳ स्तान् भा॑ग॒धेय॑म् भाग॒धेय॒म् ताने॒वैव तान् भा॑ग॒धेय॑म् भाग॒धेय॒म् ताने॒व । \newline
47. भा॒ग॒धेय॒मिति॑ भाग - धेय᳚म् । \newline
48. ताने॒ वैव ताꣳ स्ताने॒व तेन॒ तेनै॒व ताꣳ स्ता ने॒व तेन॑ । \newline
49. ए॒व तेन॒ तेनै॒वैव तेन॑ प्रीणाति प्रीणाति॒ तेनै॒ वैव तेन॑ प्रीणाति । \newline
50. तेन॑ प्रीणाति प्रीणाति॒ तेन॒ तेन॑ प्रीणाति॒ पौतु॑द्रवा॒न् पौतु॑द्रवान् प्रीणाति॒ तेन॒ तेन॑ प्रीणाति॒ पौतु॑द्रवान् । \newline
51. प्री॒णा॒ति॒ पौतु॑द्रवा॒न् पौतु॑द्रवान् प्रीणाति प्रीणाति॒ पौतु॑द्रवान् परि॒धीन् प॑रि॒धीन् पौतु॑द्रवान् प्रीणाति प्रीणाति॒ पौतु॑द्रवान् परि॒धीन् । \newline
52. पौतु॑द्रवान् परि॒धीन् प॑रि॒धीन् पौतु॑द्रवा॒न् पौतु॑द्रवान् परि॒धीन् परि॒ परि॑ परि॒धीन् पौतु॑द्रवा॒न् पौतु॑द्रवान् परि॒धीन् परि॑ । \newline
53. प॒रि॒धीन् परि॒ परि॑ परि॒धीन् प॑रि॒धीन् परि॑ दधाति दधाति॒ परि॑ परि॒धीन् प॑रि॒धीन् परि॑ दधाति । \newline
54. प॒रि॒धीनिति॑ परि - धीन् । \newline
55. परि॑ दधाति दधाति॒ परि॒ परि॑ दधा त्ये॒षा मे॒षाम् द॑धाति॒ परि॒ परि॑ दधा त्ये॒षाम् । \newline
56. द॒धा॒ त्ये॒षा मे॒षाम् द॑धाति दधा त्ये॒षाम् ॅलो॒काना᳚म् ॅलो॒काना॑ मे॒षाम् द॑धाति दधा त्ये॒षाम् ॅलो॒काना᳚म् । \newline
57. ए॒षाम् ॅलो॒काना᳚म् ॅलो॒काना॑ मे॒षा मे॒षाम् ॅलो॒कानां॒ ॅविधृ॑त्यै॒ विधृ॑त्यै लो॒काना॑ मे॒षा मे॒षाम् ॅलो॒कानां॒ ॅविधृ॑त्यै । \newline
\pagebreak
\markright{ TS 6.2.8.4  \hfill https://www.vedavms.in \hfill}

\section{ TS 6.2.8.4 }

\textbf{TS 6.2.8.4 } \newline
\textbf{Samhita Paata} \newline

ॅलो॒कानां॒ ॅविधृ॑त्या अ॒ग्नेस्त्रयो॒ ज्यायाꣳ॑सो॒ भ्रात॑र आस॒न् ते दे॒वेभ्यो॑ ह॒व्यं ॅवह॑न्तः॒ प्रामी॑यन्त॒ सो᳚ऽग्निर॑बिभेदि॒त्थं ॅवाव स्य आर्ति॒माऽरि॑ष्य॒तीति॒ स निला॑यत॒ स यां ॅवन॒स्पति॒ष्वव॑स॒त्तां पूतु॑द्रौ॒ यामोष॑धीषु॒ ताꣳ सु॑गन्धि॒तेज॑ने॒ यां प॒शुषु॒ तां पेत्व॑स्यान्त॒रा शृङ्गे॒ तं दे॒वताः॒ प्रैष॑मैच्छ॒न् तमन्व॑विन्द॒न् तम॑ब्रुव॒- [  ] \newline

\textbf{Pada Paata} \newline

लो॒काना᳚म् । विधृ॑त्या॒ इति॒ वि-धृ॒त्यै॒ । अ॒ग्नेः । त्रयः॑ । ज्यायाꣳ॑सः । भ्रात॑रः । आ॒स॒न्न् । ते । दे॒वेभ्यः॑ । ह॒व्यम् । वह॑न्तः । प्रेति॑ । अ॒मी॒य॒न्त॒ । सः । अ॒ग्निः । अ॒बि॒भे॒त् । इ॒त्थम् । वाव । स्यः । आर्ति᳚म् । एति॑ । अ॒रि॒ष्य॒ति॒ । इति॑ । सः । निला॑यत । सः । याम् । वन॒स्पति॑षु । अव॑सत् । ताम् । पूतु॑द्रौ । याम् । ओष॑धीषु । ताम् । सु॒ग॒न्धि॒तेज॑न॒ इति॑ सुगन्धि - तेज॑ने । याम् । प॒शुषु॑ । ताम् । पेत्व॑स्य । अ॒न्त॒रा । शृङ्गे॒ इति॑ । तम् । दे॒वताः᳚ । प्रैष॒मिति॑ प्र -एष᳚म् । ऐ॒च्छ॒न्न् । तम् । अन्विति॑ । अ॒वि॒न्द॒न्न् । तम् । अ॒ब्रु॒व॒न्न् ।  \newline


\textbf{Krama Paata} \newline

लो॒काना॒म् ॅविधृ॑त्यै । विधृ॑त्या अ॒ग्नेः । विधृ॑त्या॒ इति॒ वि - धृ॒त्यै॒ । अ॒ग्नेस्त्रयः॑ । त्रयो॒ ज्यायाꣳ॑सः । ज्यायाꣳ॑सो॒ भ्रात॑रः । भ्रात॑र आसन्न् । आ॒स॒न् ते । ते दे॒वेभ्यः॑ । दे॒वेभ्यो॑ ह॒व्यम् । ह॒व्यम् ॅवह॑न्तः । वह॑न्तः॒ प्र । प्रामी॑यन्त । अ॒मी॒य॒न्त॒ सः । सो᳚ऽग्निः । अ॒ग्निर॑बिभेत् । अ॒बि॒भे॒दि॒त्थम् । इ॒त्थम् ॅवाव । वाव स्यः । स्य आर्ति᳚म् । आर्ति॒मा । आऽरि॑ष्यति । अ॒रि॒ष्य॒तीति॑ । इति॒ सः । स निला॑यत । निला॑यत॒ सः । स याम् । याम् ॅवन॒स्पति॑षु । वन॒स्पति॒ष्वव॑सत् । अव॑स॒त् ताम् । ताम् पूतु॑द्रौ । पूतु॑द्रौ॒ याम् । यामोष॑धीषु । ओष॑धीषु॒ ताम् । ताꣳ सु॑गन्धि॒तेज॑ने । सु॒ग॒न्धि॒तेज॑ने॒ याम् । सु॒ग॒न्धि॒तेज॑न॒ इति॑ सुगन्धि - तेज॑ने । याम् प॒शुषु॑ । प॒शुषु॒ ताम् । ताम् पेत्व॑स्य । पेत्व॑स्यान्त॒रा । अ॒न्त॒रा शृङ्‍गे᳚ । शृङ्‍गे॒ तम् । शृङ्‍गे॒ इति॒ शृङ्‍गे᳚ । तम् दे॒वताः᳚ । दे॒वता॒ प्रैष᳚म् । प्रैष॑मैच्छन्न् । प्रैष॒मिति॑ प्र - एष᳚म् । ऐ॒च्छ॒न् तम् । तमनु॑ । अन्व॑विन्दन्न् । अ॒वि॒न्द॒न् तम् । तम॑ब्रुवन्न् । अ॒ब्रु॒व॒न्नुप॑ \newline

\textbf{Jatai Paata} \newline

1. लो॒कानां॒ ॅविधृ॑त्यै॒ विधृ॑त्यै लो॒काना᳚म् ॅलो॒कानां॒ ॅविधृ॑त्यै । \newline
2. विधृ॑त्या अ॒ग्ने र॒ग्नेर् विधृ॑त्यै॒ विधृ॑त्या अ॒ग्नेः । \newline
3. विधृ॑त्या॒ इति॒ वि - धृ॒त्यै॒ । \newline
4. अ॒ग्ने स्त्रय॒ स्त्रयो॒ ऽग्ने र॒ग्ने स्त्रयः॑ । \newline
5. त्रयो॒ ज्यायाꣳ॑सो॒ ज्यायाꣳ॑स॒ स्त्रय॒ स्त्रयो॒ ज्यायाꣳ॑सः । \newline
6. ज्यायाꣳ॑सो॒ भ्रात॑रो॒ भ्रात॑रो॒ ज्यायाꣳ॑सो॒ ज्यायाꣳ॑सो॒ भ्रात॑रः । \newline
7. भ्रात॑र आसन् नास॒न् भ्रात॑रो॒ भ्रात॑र आसन्न् । \newline
8. आ॒स॒न् ते त आ॑सन् नास॒न् ते । \newline
9. ते दे॒वेभ्यो॑ दे॒वेभ्य॒ स्ते ते दे॒वेभ्यः॑ । \newline
10. दे॒वेभ्यो॑ ह॒व्यꣳ ह॒व्यम् दे॒वेभ्यो॑ दे॒वेभ्यो॑ ह॒व्यम् । \newline
11. ह॒व्यं ॅवह॑न्तो॒ वह॑न्तो ह॒व्यꣳ ह॒व्यं ॅवह॑न्तः । \newline
12. वह॑न्तः॒ प्र प्र वह॑न्तो॒ वह॑न्तः॒ प्र । \newline
13. प्रामी॑यन्ता मीयन्त॒ प्र प्रामी॑यन्त । \newline
14. अ॒मी॒य॒न्त॒ स सो॑ ऽमीयन्ता मीयन्त॒ सः । \newline
15. सो᳚ ऽग्नि र॒ग्निः स सो᳚ ऽग्निः । \newline
16. अ॒ग्नि र॑बिभे दबिभे द॒ग्नि र॒ग्नि र॑बिभेत् । \newline
17. अ॒बि॒भे॒ दि॒त्थ मि॒त्थ म॑बिभे दबिभे दि॒त्थम् । \newline
18. इ॒त्थं ॅवाव वावे त्थ मि॒त्थं ॅवाव । \newline
19. वाव स्य स्य वाव वाव स्यः । \newline
20. स्य आर्ति॒ मार्तिꣳ॒॒ स्य स्य आर्ति᳚म् । \newline
21. आर्ति॒ मा ऽऽर्ति॒ मार्ति॒ मा । \newline
22. आ ऽरि॑ष्य त्यरिष्य॒त्या ऽरि॑ष्यति । \newline
23. अ॒रि॒ष्य॒तीती त्य॑रिष्य त्यरिष्य॒तीति॑ । \newline
24. इति॒ स स इतीति॒ सः । \newline
25. स निला॑यत॒ निला॑यत॒ स स निला॑यत । \newline
26. निला॑यत॒ स स निला॑यत॒ निला॑यत॒ सः । \newline
27. स यां ॅयाꣳ स स याम् । \newline
28. यां ॅवन॒स्पति॑षु॒ वन॒स्पति॑षु॒ यां ॅयां ॅवन॒स्पति॑षु । \newline
29. वन॒स्पति॒ ष्वव॑स॒ दव॑स॒द् वन॒स्पति॑षु॒ वन॒स्पति॒ ष्वव॑सत् । \newline
30. अव॑स॒त् ताम् ता मव॑स॒ दव॑स॒त् ताम् । \newline
31. ताम् पूतु॑द्रौ॒ पूतु॑द्रौ॒ ताम् ताम् पूतु॑द्रौ । \newline
32. पूतु॑द्रौ॒ यां ॅयाम् पूतु॑द्रौ॒ पूतु॑द्रौ॒ याम् । \newline
33. या मोष॑धी॒ ष्वोष॑धीषु॒ यां ॅया मोष॑धीषु । \newline
34. ओष॑धीषु॒ ताम् ता मोष॑धी॒ ष्वोष॑धीषु॒ ताम् । \newline
35. ताꣳ सु॑गन्धि॒तेज॑ने सुगन्धि॒तेज॑ने॒ ताम् ताꣳ सु॑गन्धि॒तेज॑ने । \newline
36. सु॒ग॒न्धि॒तेज॑ने॒ यां ॅयाꣳ सु॑गन्धि॒तेज॑ने सुगन्धि॒तेज॑ने॒ याम् । \newline
37. सु॒ग॒न्धि॒तेज॑न॒ इति॑ सुगन्धि - तेज॑ने । \newline
38. याम् प॒शुषु॑ प॒शुषु॒ यां ॅयाम् प॒शुषु॑ । \newline
39. प॒शुषु॒ ताम् ताम् प॒शुषु॑ प॒शुषु॒ ताम् । \newline
40. ताम् पेत्व॑स्य॒ पेत्व॑स्य॒ ताम् ताम् पेत्व॑स्य । \newline
41. पेत्व॑स्या न्त॒रा ऽन्त॒रा पेत्व॑स्य॒ पेत्व॑स्यान्त॒रा । \newline
42. अ॒न्त॒रा शृङ्गे॒ शृङ्गे॑ अन्त॒रा ऽन्त॒रा शृङ्गे᳚ । \newline
43. शृङ्गे॒ तम् तꣳ शृङ्गे॒ शृङ्गे॒ तम् । \newline
44. शृङ्गे॒ इति॒ शृङ्गे᳚ । \newline
45. तम् दे॒वता॑ दे॒वता॒ स्तम् तम् दे॒वताः᳚ । \newline
46. दे॒वताः॒ प्रैष॒म् प्रैष॑म् दे॒वता॑ दे॒वताः॒ प्रैष᳚म् । \newline
47. प्रैष॑ मैच्छन् नैच्छ॒न् प्रैष॒म् प्रैष॑ मैच्छन्न् । \newline
48. प्रैष॒मिति॑ प्र - एष᳚म् । \newline
49. ऐ॒च्छ॒न् तम् त मै᳚च्छन् नैच्छ॒न् तम् । \newline
50. त मन्वनु॒ तम् त मनु॑ । \newline
51. अन्व॑विन्दन् नविन्द॒न् नन् वन् व॑विन्दन्न् । \newline
52. अ॒वि॒न्द॒न् तम् त म॑विन्दन् नविन्द॒न् तम् । \newline
53. त म॑ब्रुवन् नब्रुव॒न् तम् त म॑ब्रुवन्न् । \newline
54. अ॒ब्रु॒व॒न् नुपोपा᳚ब्रुवन् नब्रुव॒न् नुप॑ । \newline

\textbf{Ghana Paata } \newline

1. लो॒कानां॒ ॅविधृ॑त्यै॒ विधृ॑त्यै लो॒काना᳚म् ॅलो॒कानां॒ ॅविधृ॑त्या अ॒ग्ने र॒ग्नेर् विधृ॑त्यै लो॒काना᳚म् ॅलो॒कानां॒ ॅविधृ॑त्या अ॒ग्नेः । \newline
2. विधृ॑त्या अ॒ग्ने र॒ग्नेर् विधृ॑त्यै॒ विधृ॑त्या अ॒ग्ने स्त्रय॒ स्त्रयो॒ ऽग्नेर् विधृ॑त्यै॒ विधृ॑त्या अ॒ग्ने स्त्रयः॑ । \newline
3. विधृ॑त्या॒ इति॒ वि - धृ॒त्यै॒ । \newline
4. अ॒ग्ने स्त्रय॒ स्त्रयो॒ ऽग्ने र॒ग्ने स्त्रयो॒ ज्यायाꣳ॑सो॒ ज्यायाꣳ॑स॒ स्त्रयो॒ ऽग्ने र॒ग्ने स्त्रयो॒ ज्यायाꣳ॑सः । \newline
5. त्रयो॒ ज्यायाꣳ॑सो॒ ज्यायाꣳ॑स॒ स्त्रय॒ स्त्रयो॒ ज्यायाꣳ॑सो॒ भ्रात॑रो॒ भ्रात॑रो॒ ज्यायाꣳ॑स॒ स्त्रय॒ स्त्रयो॒ ज्यायाꣳ॑सो॒ भ्रात॑रः । \newline
6. ज्यायाꣳ॑सो॒ भ्रात॑रो॒ भ्रात॑रो॒ ज्यायाꣳ॑सो॒ ज्यायाꣳ॑सो॒ भ्रात॑र आसन् नास॒न् भ्रात॑रो॒ ज्यायाꣳ॑सो॒ ज्यायाꣳ॑सो॒ भ्रात॑र आसन्न् । \newline
7. भ्रात॑र आसन् नास॒न् भ्रात॑रो॒ भ्रात॑र आस॒न् ते त आ॑स॒न् भ्रात॑रो॒ भ्रात॑र आस॒न् ते । \newline
8. आ॒स॒न् ते त आ॑सन् नास॒न् ते दे॒वेभ्यो॑ दे॒वेभ्य॒ स्त आ॑सन् नास॒न् ते दे॒वेभ्यः॑ । \newline
9. ते दे॒वेभ्यो॑ दे॒वेभ्य॒ स्ते ते दे॒वेभ्यो॑ ह॒व्यꣳ ह॒व्यम् दे॒वेभ्य॒ स्ते ते दे॒वेभ्यो॑ ह॒व्यम् । \newline
10. दे॒वेभ्यो॑ ह॒व्यꣳ ह॒व्यम् दे॒वेभ्यो॑ दे॒वेभ्यो॑ ह॒व्यं ॅवह॑न्तो॒ वह॑न्तो ह॒व्यम् दे॒वेभ्यो॑ दे॒वेभ्यो॑ ह॒व्यं ॅवह॑न्तः । \newline
11. ह॒व्यं ॅवह॑न्तो॒ वह॑न्तो ह॒व्यꣳ ह॒व्यं ॅवह॑न्तः॒ प्र प्र वह॑न्तो ह॒व्यꣳ ह॒व्यं ॅवह॑न्तः॒ प्र । \newline
12. वह॑न्तः॒ प्र प्र वह॑न्तो॒ वह॑न्तः॒ प्रामी॑यन्ता मीयन्त॒ प्र वह॑न्तो॒ वह॑न्तः॒ प्रामी॑यन्त । \newline
13. प्रामी॑यन्ता मीयन्त॒ प्र प्रामी॑यन्त॒ स सो॑ ऽमीयन्त॒ प्र प्रामी॑यन्त॒ सः । \newline
14. अ॒मी॒य॒न्त॒ स सो॑ ऽमीयन्ता मीयन्त॒ सो᳚ ऽग्नि र॒ग्निः सो॑ ऽमीयन्ता मीयन्त॒ सो᳚ ऽग्निः । \newline
15. सो᳚ ऽग्नि र॒ग्निः स सो᳚ ऽग्नि र॑बिभे दबिभे द॒ग्निः स सो᳚ ऽग्नि र॑बिभेत् । \newline
16. अ॒ग्नि र॑बिभे दबिभे द॒ग्नि र॒ग्नि र॑बिभे दि॒त्थ मि॒त्थ म॑बिभे द॒ग्नि र॒ग्नि र॑बिभे दि॒त्थम् । \newline
17. अ॒बि॒भे॒ दि॒त्थ मि॒त्थ म॑बिभे दबिभे दि॒त्थं ॅवाव वावे त्थ म॑बिभे दबिभे दि॒त्थं ॅवाव । \newline
18. इ॒त्थं ॅवाव वावे त्थ मि॒त्थं ॅवाव स्य स्य वावे त्थ मि॒त्थं ॅवाव स्यः । \newline
19. वाव स्य स्य वाव वाव स्य आर्ति॒ मार्तिꣳ॒॒ स्य वाव वाव स्य आर्ति᳚म् । \newline
20. स्य आर्ति॒ मार्तिꣳ॒॒ स्य स्य आर्ति॒ मा ऽऽर्तिꣳ॒॒ स्य स्य आर्ति॒ मा । \newline
21. आर्ति॒ मा ऽऽर्ति॒ मार्ति॒ मा ऽरि॑ष्य त्यरिष्य॒ त्याऽऽर्ति॒ मार्ति॒ मा ऽरि॑ष्यति । \newline
22. आ ऽरि॑ष्य त्यरिष्य॒ त्याऽरि॑ष्य॒ तीती त्य॑रिष्य॒ त्याऽरि॑ष्य॒ तीति॑ । \newline
23. अ॒रि॒ष्य॒ तीती त्य॑रिष्य त्यरिष्य॒ तीति॒ स स इत्य॑रिष्य त्यरिष्य॒ तीति॒ सः । \newline
24. इति॒ स स इतीति॒ स निला॑यत॒ निला॑यत॒ स इतीति॒ स निला॑यत । \newline
25. स निला॑यत॒ निला॑यत॒ स स निला॑यत॒ स स निला॑यत॒ स स निला॑यत॒ सः । \newline
26. निला॑यत॒ स स निला॑यत॒ निला॑यत॒ स यां ॅयाꣳ स निला॑यत॒ निला॑यत॒ स याम् । \newline
27. स यां ॅयाꣳ स स यां ॅवन॒स्पति॑षु॒ वन॒स्पति॑षु॒ याꣳ स स यां ॅवन॒स्पति॑षु । \newline
28. यां ॅवन॒स्पति॑षु॒ वन॒स्पति॑षु॒ यां ॅयां ॅवन॒स्पति॒ ष्वव॑स॒ दव॑स॒द् वन॒स्पति॑षु॒ यां ॅयां ॅवन॒स्पति॒ ष्वव॑सत् । \newline
29. वन॒स्पति॒ ष्वव॑स॒ दव॑स॒द् वन॒स्पति॑षु॒ वन॒स्पति॒ ष्वव॑स॒त् ताम् ता मव॑स॒द् वन॒स्पति॑षु॒ वन॒स्पति॒ ष्वव॑स॒त् ताम् । \newline
30. अव॑स॒त् ताम् ता मव॑स॒ दव॑स॒त् ताम् पूतु॑द्रौ॒ पूतु॑द्रौ॒ ता मव॑स॒ दव॑स॒त् ताम् पूतु॑द्रौ । \newline
31. ताम् पूतु॑द्रौ॒ पूतु॑द्रौ॒ ताम् ताम् पूतु॑द्रौ॒ यां ॅयाम् पूतु॑द्रौ॒ ताम् ताम् पूतु॑द्रौ॒ याम् । \newline
32. पूतु॑द्रौ॒ यां ॅयाम् पूतु॑द्रौ॒ पूतु॑द्रौ॒ या मोष॑धी॒ ष्वोष॑धीषु॒ याम् पूतु॑द्रौ॒ पूतु॑द्रौ॒ या मोष॑धीषु । \newline
33. या मोष॑धी॒ ष्वोष॑धीषु॒ यां ॅया मोष॑धीषु॒ ताम् ता मोष॑धीषु॒ यां ॅया मोष॑धीषु॒ ताम् । \newline
34. ओष॑धीषु॒ ताम् ता मोष॑धी॒ ष्वोष॑धीषु॒ ताꣳ सु॑गन्धि॒तेज॑ने सुगन्धि॒तेज॑ने॒ ता मोष॑धी॒
ष्वोष॑धीषु॒ ताꣳ सु॑गन्धि॒तेज॑ने । \newline
35. ताꣳ सु॑गन्धि॒तेज॑ने सुगन्धि॒तेज॑ने॒ ताम् ताꣳ सु॑गन्धि॒तेज॑ने॒ यां ॅयाꣳ सु॑गन्धि॒तेज॑ने॒ ताम् ताꣳ सु॑गन्धि॒तेज॑ने॒ याम् । \newline
36. सु॒ग॒न्धि॒तेज॑ने॒ यां ॅयाꣳ सु॑गन्धि॒तेज॑ने सुगन्धि॒तेज॑ने॒ याम् प॒शुषु॑ प॒शुषु॒ याꣳ सु॑गन्धि॒तेज॑ने सुगन्धि॒तेज॑ने॒ याम् प॒शुषु॑ । \newline
37. सु॒ग॒न्धि॒तेज॑न॒ इति॑ सुगन्धि - तेज॑ने । \newline
38. याम् प॒शुषु॑ प॒शुषु॒ यां ॅयाम् प॒शुषु॒ ताम् ताम् प॒शुषु॒ यां ॅयाम् प॒शुषु॒ ताम् । \newline
39. प॒शुषु॒ ताम् ताम् प॒शुषु॑ प॒शुषु॒ ताम् पेत्व॑स्य॒ पेत्व॑स्य॒ ताम् प॒शुषु॑ प॒शुषु॒ ताम् पेत्व॑स्य । \newline
40. ताम् पेत्व॑स्य॒ पेत्व॑स्य॒ ताम् ताम् पेत्व॑स्यान्त॒रा ऽन्त॒रा पेत्व॑स्य॒ ताम् ताम् पेत्व॑स्यान्त॒रा । \newline
41. पेत्व॑स्या न्त॒रा ऽन्त॒रा पेत्व॑स्य॒ पेत्व॑स्या न्त॒रा शृङ्गे॒ शृङ्गे॑ अन्त॒रा पेत्व॑स्य॒ पेत्व॑स्या न्त॒रा शृङ्गे᳚ । \newline
42. अ॒न्त॒रा शृङ्गे॒ शृङ्गे॑ अन्त॒रा ऽन्त॒रा शृङ्गे॒ तम् तꣳ शृङ्गे॑ अन्त॒रा ऽन्त॒रा शृङ्गे॒ तम् । \newline
43. शृङ्गे॒ तम् तꣳ शृङ्गे॒ शृङ्गे॒ तम् दे॒वता॑ दे॒वता॒ स्तꣳ शृङ्गे॒ शृङ्गे॒ तम् दे॒वताः᳚ । \newline
44. शृङ्गे॒ इति॒ शृङ्गे᳚ । \newline
45. तम् दे॒वता॑ दे॒वता॒ स्तम् तम् दे॒वताः॒ प्रैष॒म् प्रैष॑म् दे॒वता॒ स्तम् तम् दे॒वताः॒ प्रैष᳚म् । \newline
46. दे॒वताः॒ प्रैष॒म् प्रैष॑म् दे॒वता॑ दे॒वताः॒ प्रैष॑ मैच्छन् नैच्छ॒न् प्रैष॑म् दे॒वता॑ दे॒वताः॒ प्रैष॑ मैच्छन्न् । \newline
47. प्रैष॑ मैच्छन् नैच्छ॒न् प्रैष॒म् प्रैष॑ मैच्छ॒न् तम् त मै᳚च्छ॒न् प्रैष॒म् प्रैष॑ मैच्छ॒न् तम् । \newline
48. प्रैष॒मिति॑ प्र - एष᳚म् । \newline
49. ऐ॒च्छ॒न् तम् त मै᳚च्छन् नैच्छ॒न् त मन् वनु॒ त मै᳚च्छन् नैच्छ॒न् त मनु॑ । \newline
50. त मन् वनु॒ तम् त मन् व॑विन्दन् नविन्द॒न् ननु॒ तम् त मन् व॑विन्दन्न् । \newline
51. अन्व॑विन्दन् नविन्द॒न् नन् वन् व॑विन्द॒न् तम् त म॑विन्द॒न् नन् वन् व॑विन्द॒न् तम् । \newline
52. अ॒वि॒न्द॒न् तम् त म॑विन्दन् नविन्द॒न् त म॑ब्रुवन् नब्रुव॒न् त म॑विन्दन् नविन्द॒न् त म॑ब्रुवन्न् । \newline
53. त म॑ब्रुवन् नब्रुव॒न् तम् त म॑ब्रुव॒न् नुपोपा᳚ ब्रुव॒न् तम् त म॑ब्रुव॒न् नुप॑ । \newline
54. अ॒ब्रु॒व॒न् नुपोपा᳚ ब्रुवन् नब्रुव॒न् नुप॑ नो न॒ उपा᳚ब्रुवन् नब्रुव॒न् नुप॑ नः । \newline
\pagebreak
\markright{ TS 6.2.8.5  \hfill https://www.vedavms.in \hfill}

\section{ TS 6.2.8.5 }

\textbf{TS 6.2.8.5 } \newline
\textbf{Samhita Paata} \newline

-न्नुप॑ न॒ आ व॑र्तस्व ह॒व्यं नो॑ व॒हेति॒ सो᳚ऽब्रवी॒द् वरं॑ ॅवृणै॒ यदे॒व गृ॑ही॒तस्याहु॑तस्य बहिःपरि॒धि स्कन्दा॒त् तन्मे॒ भ्रातृ॑णां भाग॒धेय॑मस॒दिति॒ तस्मा॒द्-यद्-गृ॑ही॒तस्याहु॑तस्य बहिःपरि॒धि स्कन्द॑ति॒ तेषां॒ तद् भा॑ग॒धेयं॒ ताने॒व तेन॑ प्रीणाति॒ सो॑ऽमन्यतास्थ॒न्वन्तो॑ मे॒ पूर्वे॒ भ्रात॑रः॒ प्रामे॑षता॒स्थानि॑ शातया॒ इति॒ स यान्य॒- [  ] \newline

\textbf{Pada Paata} \newline

उपेति॑ । नः॒ । एति॑ । व॒र्त॒स्व॒ । ह॒व्यम् । नः॒ । व॒ह॒ । इति॑ । सः । अ॒ब्र॒वी॒त् । वर᳚म् । वृ॒णै॒ । यत् । ए॒व । गृ॒ही॒तस्य॑ । अहु॑तस्य । ब॒हिः॒प॒रि॒धीति॑ बहिः - प॒रि॒धि । स्कन्दा᳚त् । तत् । मे॒ । भ्रातृ॑णाम् । भा॒ग॒धेय॒मिति॑ भाग - धेय᳚म् । अ॒स॒त् । इति॑ । तस्मा᳚त् । यत् । गृ॒ही॒तस्य॑ । अहु॑तस्य । ब॒हिः॒प॒रि॒धीति॑ बहिः - प॒रि॒धि । स्कन्द॑ति । तेषा᳚म् । तत् । भा॒ग॒धेय॒मिति॑ भाग - धेय᳚म् । तान् । ए॒व । तेन॑ । प्री॒णा॒ति॒ । सः । अ॒म॒न्य॒त॒ । अ॒स्थ॒न्वन्त॒ इत्य॑स्थन्न् - वन्तः॑ । मे॒ । पूर्वे᳚ । भ्रात॑रः । प्रेति॑ । अ॒मे॒ष॒त॒ । अ॒स्थानि॑ । शा॒त॒यै॒ । इति॑ । सः । यानि॑ ।  \newline


\textbf{Krama Paata} \newline

उप॑ नः । न॒ आ । आ व॑र्तस्व । व॒र्त॒स्व॒ ह॒व्यम् । ह॒व्यम् नः॑ । नो॒ व॒ह॒ । व॒हेति॑ । इति॒ सः । सो᳚ऽब्रवीत् । अ॒ब्र॒वी॒द् वर᳚म् । वर॑म् ॅवृणै । वृ॒णै॒ यत् । यदे॒व । ए॒व गृ॑ही॒तस्य॑ । गृ॒ही॒तस्याहु॑तस्य । अहु॑तस्य बहिःपरि॒धि । ब॒हिः॒प॒रि॒धि स्कन्दा᳚त् । ब॒हिः॒प॒रि॒धीति॑ बहिः - प॒रि॒धि । स्कन्दा॒त् तत् । तन् मे᳚ । मे॒ भ्रातृ॑णाम् । भ्रातृ॑णाम् भाग॒धेय᳚म् । भा॒ग॒धेय॑मसत् । भा॒ग॒धेय॒मिति॑ भाग - धेय᳚म् । अ॒स॒दिति॑ । इति॒ तस्मा᳚त् । तस्मा॒द् यत् । यद् गृ॑ही॒तस्य॑ । गृ॒ही॒तस्याहु॑तस्य । अहु॑तस्य बहिःपरि॒धि । ब॒हिः॒प॒रि॒धि स्कन्द॑ति । ब॒हिः॒प॒रि॒धीति॑ बहिः - प॒रि॒धि । स्कन्द॑ति॒ तेषा᳚म् । तेषा॒म् तत् । तद् भा॑ग॒धेय᳚म् । भा॒ग॒धेय॒म् तान् । भा॒ग॒धेय॒मिति॑ भाग - धेय᳚म् । ताने॒व । ए॒व तेन॑ । तेन॑ प्रीणाति । प्री॒णा॒ति॒ सः । सो॑ऽमन्यत । अ॒म॒न्य॒ता॒स्थ॒न्वन्तः॑ । अ॒स्थ॒न्वन्तो॑ मे । अ॒स्थ॒न्वन्त॒ इत्य॑स्थन्न् - वन्तः॑ । मे॒ पूर्वे᳚ । पूर्वे॒ भ्रात॑रः । भ्रात॑रः॒ प्र । प्रामे॑षत । अ॒मे॒ष॒ता॒स्थानि॑ । अ॒स्थानि॑ शातयै । शा॒त॒या॒ इति॑ । इति॒ सः । स यानि॑ ( ) । यान्य॒स्थानि॑ \newline

\textbf{Jatai Paata} \newline

1. उप॑ नो न॒ उपोप॑ नः । \newline
2. न॒ आ नो॑ न॒ आ । \newline
3. आ व॑र्तस्व वर्त॒स्वा व॑र्तस्व । \newline
4. व॒र्त॒स्व॒ ह॒व्यꣳ ह॒व्यं ॅव॑र्तस्व वर्तस्व ह॒व्यम् । \newline
5. ह॒व्यन्नो॑ नो ह॒व्यꣳ ह॒व्यन्नः॑ । \newline
6. नो॒ व॒ह॒ व॒ह॒ नो॒ नो॒ व॒ह॒ । \newline
7. व॒हे तीति॑ वह व॒हेति॑ । \newline
8. इति॒ स स इतीति॒ सः । \newline
9. सो᳚ ऽब्रवी दब्रवी॒थ् स सो᳚ ऽब्रवीत् । \newline
10. अ॒ब्र॒वी॒द् वरं॒ ॅवर॑ मब्रवी दब्रवी॒द् वर᳚म् । \newline
11. वरं॑ ॅवृणै वृणै॒ वरं॒ ॅवरं॑ ॅवृणै । \newline
12. वृ॒णै॒ यद् यद् वृ॑णै वृणै॒ यत् । \newline
13. यदे॒वैव यद् यदे॒व । \newline
14. ए॒व गृ॑ही॒तस्य॑ गृही॒त स्यै॒वैव गृ॑ही॒तस्य॑ । \newline
15. गृ॒ही॒तस्या हु॑त॒स्या हु॑तस्य गृही॒तस्य॑ गृही॒तस्या हु॑तस्य । \newline
16. अहु॑तस्य बहिःपरि॒धि ब॑हिःपरि॒ ध्यहु॑त॒स्या हु॑तस्य बहिःपरि॒धि । \newline
17. ब॒हिः॒प॒रि॒धि स्कन्दा॒थ् स्कन्दा᳚द् बहिःपरि॒धि ब॑हिःपरि॒धि स्कन्दा᳚त् । \newline
18. ब॒हिः॒प॒रि॒धीति॑ बहिः - प॒रि॒धि । \newline
19. स्कन्दा॒त् तत् तथ् स्कन्दा॒थ् स्कन्दा॒त् तत् । \newline
20. तन् मे॑ मे॒ तत् तन् मे᳚ । \newline
21. मे॒ भ्रातृ॑णा॒म् भ्रातृ॑णाम् मे मे॒ भ्रातृ॑णाम् । \newline
22. भ्रातृ॑णाम् भाग॒धेय॑म् भाग॒धेय॒म् भ्रातृ॑णा॒म् भ्रातृ॑णाम् भाग॒धेय᳚म् । \newline
23. भा॒ग॒धेय॑ मस दसद् भाग॒धेय॑म् भाग॒धेय॑ मसत् । \newline
24. भा॒ग॒धेय॒मिति॑ भाग - धेय᳚म् । \newline
25. अ॒स॒दिती त्य॑स दस॒ दिति॑ । \newline
26. इति॒ तस्मा॒त् तस्मा॒ दितीति॒ तस्मा᳚त् । \newline
27. तस्मा॒द् यद् यत् तस्मा॒त् तस्मा॒द् यत् । \newline
28. यद् गृ॑ही॒तस्य॑ गृही॒तस्य॒ यद् यद् गृ॑ही॒तस्य॑ । \newline
29. गृ॒ही॒तस्या हु॑त॒स्या हु॑तस्य गृही॒तस्य॑ गृही॒तस्या हु॑तस्य । \newline
30. अहु॑तस्य बहिःपरि॒धि ब॑हिःपरि॒ ध्यहु॑त॒स्या हु॑तस्य बहिःपरि॒धि । \newline
31. ब॒हिः॒प॒रि॒धि स्कन्द॑ति॒ स्कन्द॑ति बहिःपरि॒धि ब॑हिःपरि॒धि स्कन्द॑ति । \newline
32. ब॒हिः॒प॒रि॒धीति॑ बहिः - प॒रि॒धि । \newline
33. स्कन्द॑ति॒ तेषा॒म् तेषाꣳ॒॒ स्कन्द॑ति॒ स्कन्द॑ति॒ तेषा᳚म् । \newline
34. तेषा॒म् तत् तत् तेषा॒म् तेषा॒म् तत् । \newline
35. तद् भा॑ग॒धेय॑म् भाग॒धेय॒म् तत् तद् भा॑ग॒धेय᳚म् । \newline
36. भा॒ग॒धेय॒म् ताꣳ स्तान् भा॑ग॒धेय॑म् भाग॒धेय॒म् तान् । \newline
37. भा॒ग॒धेय॒मिति॑ भाग - धेय᳚म् । \newline
38. ता ने॒वैव ताꣳ स्ता ने॒व । \newline
39. ए॒व तेन॒ तेनै॒वैव तेन॑ । \newline
40. तेन॑ प्रीणाति प्रीणाति॒ तेन॒ तेन॑ प्रीणाति । \newline
41. प्री॒णा॒ति॒ स स प्री॑णाति प्रीणाति॒ सः । \newline
42. सो॑ ऽमन्यता मन्यत॒ स सो॑ ऽमन्यत । \newline
43. अ॒म॒न्य॒ता॒ स्थ॒न्वन्तो᳚ ऽस्थ॒न्वन्तो॑ ऽमन्यता मन्यता स्थ॒न्वन्तः॑ । \newline
44. अ॒स्थ॒न्वन्तो॑ मे मे ऽस्थ॒न्वन्तो᳚ ऽस्थ॒न्वन्तो॑ मे । \newline
45. अ॒स्थ॒न्वन्त॒ इत्य॑स्थन्न् - वन्तः॑ । \newline
46. मे॒ पूर्वे॒ पूर्वे॑ मे मे॒ पूर्वे᳚ । \newline
47. पूर्वे॒ भ्रात॑रो॒ भ्रात॑रः॒ पूर्वे॒ पूर्वे॒ भ्रात॑रः । \newline
48. भ्रात॑रः॒ प्र प्र भ्रात॑रो॒ भ्रात॑रः॒ प्र । \newline
49. प्रामे॑षता मेषत॒ प्र प्रामे॑षत । \newline
50. अ॒मे॒ष॒ता॒ स्था न्य॒स्था न्य॑मेषता मेषता॒ स्थानि॑ । \newline
51. अ॒स्थानि॑ शातयै शातया अ॒स्था न्य॒स्थानि॑ शातयै । \newline
52. शा॒त॒या॒ इतीति॑ शातयै शातया॒ इति॑ । \newline
53. इति॒ स स इतीति॒ सः । \newline
54. स यानि॒ यानि॒ स स यानि॑ । \newline
55. यान्य॒स्था न्य॒स्थानि॒ यानि॒ यान्य॒स्थानि॑ । \newline

\textbf{Ghana Paata } \newline

1. उप॑ नो न॒ उपोप॑ न॒ आ न॒ उपोप॑ न॒ आ । \newline
2. न॒ आ नो॑ न॒ आ व॑र्तस्व वर्त॒स्वा नो॑ न॒ आ व॑र्तस्व । \newline
3. आ व॑र्तस्व वर्त॒स्वा व॑र्तस्व ह॒व्यꣳ ह॒व्यं ॅव॑र्त॒स्वा व॑र्तस्व ह॒व्यम् । \newline
4. व॒र्त॒स्व॒ ह॒व्यꣳ ह॒व्यं ॅव॑र्तस्व वर्तस्व ह॒व्यन् नो॑ नो ह॒व्यं ॅव॑र्तस्व वर्तस्व ह॒व्यन् नः॑ । \newline
5. ह॒व्यन् नो॑ नो ह॒व्यꣳ ह॒व्यन् नो॑ वह वह नो ह॒व्यꣳ ह॒व्यन् नो॑ वह । \newline
6. नो॒ व॒ह॒ व॒ह॒ नो॒ नो॒ व॒हे तीति॑ वह नो नो व॒हेति॑ । \newline
7. व॒हे तीति॑ वह व॒हेति॒ स स इति॑ वह व॒हेति॒ सः । \newline
8. इति॒ स स इतीति॒ सो᳚ ऽब्रवी दब्रवी॒थ् स इतीति॒ सो᳚ ऽब्रवीत् । \newline
9. सो᳚ ऽब्रवी दब्रवी॒थ् स सो᳚ ऽब्रवी॒द् वरं॒ ॅवर॑ मब्रवी॒थ् स सो᳚ ऽब्रवी॒द् वर᳚म् । \newline
10. अ॒ब्र॒वी॒द् वरं॒ ॅवर॑ मब्रवी दब्रवी॒द् वरं॑ ॅवृणै वृणै॒ वर॑ मब्रवी दब्रवी॒द् वरं॑ ॅवृणै । \newline
11. वरं॑ ॅवृणै वृणै॒ वरं॒ ॅवरं॑ ॅवृणै॒ यद् यद् वृ॑णै॒ वरं॒ ॅवरं॑ ॅवृणै॒ यत् । \newline
12. वृ॒णै॒ यद् यद् वृ॑णै वृणै॒ यदे॒ वैव यद् वृ॑णै वृणै॒ यदे॒व । \newline
13. यदे॒ वैव यद् यदे॒व गृ॑ही॒तस्य॑ गृही॒तस्यै॒व यद् यदे॒व गृ॑ही॒तस्य॑ । \newline
14. ए॒व गृ॑ही॒तस्य॑ गृही॒त स्यै॒वैव गृ॑ही॒तस्या हु॑त॒स्या हु॑तस्य गृही॒त स्यै॒वैव गृ॑ही॒तस्या हु॑तस्य । \newline
15. गृ॒ही॒तस्या हु॑त॒स्या हु॑तस्य गृही॒तस्य॑ गृही॒तस्या हु॑तस्य बहिःपरि॒धि ब॑हिःपरि॒ध्य हु॑तस्य गृही॒तस्य॑ गृही॒तस्या हु॑तस्य बहिःपरि॒धि । \newline
16. अहु॑तस्य बहिःपरि॒धि ब॑हिःपरि॒ध्य हु॑त॒स्या हु॑तस्य बहिःपरि॒धि स्कन्दा॒थ् स्कन्दा᳚द् बहिःपरि॒ध्य हु॑त॒स्या हु॑तस्य बहिःपरि॒धि स्कन्दा᳚त् । \newline
17. ब॒हिः॒प॒रि॒धि स्कन्दा॒थ् स्कन्दा᳚द् बहिःपरि॒धि ब॑हिःपरि॒धि स्कन्दा॒त् तत् तथ् स्कन्दा᳚द् बहिःपरि॒धि ब॑हिःपरि॒धि स्कन्दा॒त् तत् । \newline
18. ब॒हिः॒प॒रि॒धीति॑ बहिः - प॒रि॒धि । \newline
19. स्कन्दा॒त् तत् तथ् स्कन्दा॒थ् स्कन्दा॒त् तन् मे॑ मे॒ तथ् स्कन्दा॒थ् स्कन्दा॒त् तन् मे᳚ । \newline
20. तन् मे॑ मे॒ तत् तन् मे॒ भ्रातृ॑णा॒म् भ्रातृ॑णाम् मे॒ तत् तन् मे॒ भ्रातृ॑णाम् । \newline
21. मे॒ भ्रातृ॑णा॒म् भ्रातृ॑णाम् मे मे॒ भ्रातृ॑णाम् भाग॒धेय॑म् भाग॒धेय॒म् भ्रातृ॑णाम् मे मे॒ भ्रातृ॑णाम् भाग॒धेय᳚म् । \newline
22. भ्रातृ॑णाम् भाग॒धेय॑म् भाग॒धेय॒म् भ्रातृ॑णा॒म् भ्रातृ॑णाम् भाग॒धेय॑ मस दसद् भाग॒धेय॒म् भ्रातृ॑णा॒म् भ्रातृ॑णाम् भाग॒धेय॑ मसत् । \newline
23. भा॒ग॒धेय॑ मस दसद् भाग॒धेय॑म् भाग॒धेय॑ मस॒ दिती त्य॑सद् भाग॒धेय॑म् भाग॒धेय॑ मस॒ दिति॑ । \newline
24. भा॒ग॒धेय॒मिति॑ भाग - धेय᳚म् । \newline
25. अ॒स॒दिती त्य॑स दस॒दिति॒ तस्मा॒त् तस्मा॒ दित्य॑स दस॒दिति॒ तस्मा᳚त् । \newline
26. इति॒ तस्मा॒त् तस्मा॒दि तीति॒ तस्मा॒द् यद् यत् तस्मा॒दि तीति॒ तस्मा॒द् यत् । \newline
27. तस्मा॒द् यद् यत् तस्मा॒त् तस्मा॒द् यद् गृ॑ही॒तस्य॑ गृही॒तस्य॒ यत् तस्मा॒त् तस्मा॒द् यद् गृ॑ही॒तस्य॑ । \newline
28. यद् गृ॑ही॒तस्य॑ गृही॒तस्य॒ यद् यद् गृ॑ही॒तस्या हु॑त॒स्या हु॑तस्य गृही॒तस्य॒ यद् यद् गृ॑ही॒तस्या हु॑तस्य । \newline
29. गृ॒ही॒तस्या हु॑त॒स्या हु॑तस्य गृही॒तस्य॑ गृही॒तस्या हु॑तस्य बहिःपरि॒धि ब॑हिःपरि॒ ध्यहु॑तस्य गृही॒तस्य॑ गृही॒तस्या हु॑तस्य बहिःपरि॒धि । \newline
30. अहु॑तस्य बहिःपरि॒धि ब॑हिःपरि॒ ध्यहु॑त॒स्या हु॑तस्य बहिःपरि॒धि स्कन्द॑ति॒ स्कन्द॑ति बहिःपरि॒ 
ध्यहु॑त॒स्या हु॑तस्य बहिःपरि॒धि स्कन्द॑ति । \newline
31. ब॒हिः॒प॒रि॒धि स्कन्द॑ति॒ स्कन्द॑ति बहिःपरि॒धि ब॑हिःपरि॒धि स्कन्द॑ति॒ तेषा॒म् तेषाꣳ॒॒ स्कन्द॑ति बहिःपरि॒धि ब॑हिःपरि॒धि स्कन्द॑ति॒ तेषा᳚म् । \newline
32. ब॒हिः॒प॒रि॒धीति॑ बहिः - प॒रि॒धि । \newline
33. स्कन्द॑ति॒ तेषा॒म् तेषाꣳ॒॒ स्कन्द॑ति॒ स्कन्द॑ति॒ तेषा॒म् तत् तत् तेषाꣳ॒॒ स्कन्द॑ति॒ स्कन्द॑ति॒ तेषा॒म् तत् । \newline
34. तेषा॒म् तत् तत् तेषा॒म् तेषा॒म् तद् भा॑ग॒धेय॑म् भाग॒धेय॒म् तत् तेषा॒म् तेषा॒म् तद् भा॑ग॒धेय᳚म् । \newline
35. तद् भा॑ग॒धेय॑म् भाग॒धेय॒म् तत् तद् भा॑ग॒धेय॒म् ताꣳ स्तान् भा॑ग॒धेय॒म् तत् तद् भा॑ग॒धेय॒म् तान् । \newline
36. भा॒ग॒धेय॒म् ताꣳ स्तान् भा॑ग॒धेय॑म् भाग॒धेय॒म् ताने॒ वैव तान् भा॑ग॒धेय॑म् भाग॒धेय॒म् ताने॒व । \newline
37. भा॒ग॒धेय॒मिति॑ भाग - धेय᳚म् । \newline
38. ताने॒ वैव ताꣳ स्ता ने॒व तेन॒ तेनै॒व ताꣳ स्ता ने॒व तेन॑ । \newline
39. ए॒व तेन॒ तेनै॒ वैव तेन॑ प्रीणाति प्रीणाति॒ तेनै॒वैव तेन॑ प्रीणाति । \newline
40. तेन॑ प्रीणाति प्रीणाति॒ तेन॒ तेन॑ प्रीणाति॒ स स प्री॑णाति॒ तेन॒ तेन॑ प्रीणाति॒ सः । \newline
41. प्री॒णा॒ति॒ स स प्री॑णाति प्रीणाति॒ सो॑ ऽमन्यता मन्यत॒ स प्री॑णाति प्रीणाति॒ सो॑ ऽमन्यत । \newline
42. सो॑ ऽमन्यता मन्यत॒ स सो॑ ऽमन्यता स्थ॒न्वन्तो᳚ ऽस्थ॒न्वन्तो॑ ऽमन्यत॒ स सो॑ ऽमन्यता स्थ॒न्वन्तः॑ । \newline
43. अ॒म॒न्य॒ता॒ स्थ॒न्वन्तो᳚ ऽस्थ॒न्वन्तो॑ ऽमन्यता मन्यता स्थ॒न्वन्तो॑ मे मे ऽस्थ॒न्वन्तो॑ ऽमन्यता मन्यता स्थ॒न्वन्तो॑ मे । \newline
44. अ॒स्थ॒न्वन्तो॑ मे मे ऽस्थ॒न्वन्तो᳚ ऽस्थ॒न्वन्तो॑ मे॒ पूर्वे॒ पूर्वे॑ मे ऽस्थ॒न्वन्तो᳚ ऽस्थ॒न्वन्तो॑ मे॒ पूर्वे᳚ । \newline
45. अ॒स्थ॒न्वन्त॒ इत्य॑स्थन्न् - वन्तः॑ । \newline
46. मे॒ पूर्वे॒ पूर्वे॑ मे मे॒ पूर्वे॒ भ्रात॑रो॒ भ्रात॑रः॒ पूर्वे॑ मे मे॒ पूर्वे॒ भ्रात॑रः । \newline
47. पूर्वे॒ भ्रात॑रो॒ भ्रात॑रः॒ पूर्वे॒ पूर्वे॒ भ्रात॑रः॒ प्र प्र भ्रात॑रः॒ पूर्वे॒ पूर्वे॒ भ्रात॑रः॒ प्र । \newline
48. भ्रात॑रः॒ प्र प्र भ्रात॑रो॒ भ्रात॑रः॒ प्रामे॑षता मेषत॒ प्र भ्रात॑रो॒ भ्रात॑रः॒ प्रामे॑षत । \newline
49. प्रामे॑षता मेषत॒ प्र प्रामे॑षता॒ स्था न्य॒स्था न्य॑मेषत॒ प्र प्रामे॑षता॒ स्थानि॑ । \newline
50. अ॒मे॒ष॒ता॒ स्था न्य॒स्था न्य॑मेषता मेषता॒ स्थानि॑ शातयै शातया अ॒स्था न्य॑मेषता मेषता॒ स्थानि॑ शातयै । \newline
51. अ॒स्थानि॑ शातयै शातया अ॒स्था न्य॒स्थानि॑ शातया॒ इतीति॑ शातया अ॒स्था न्य॒स्थानि॑ शातया॒ इति॑ । \newline
52. शा॒त॒या॒ इतीति॑ शातयै शातया॒ इति॒ स स इति॑ शातयै शातया॒ इति॒ सः । \newline
53. इति॒ स स इतीति॒ स यानि॒ यानि॒ स इतीति॒ स यानि॑ । \newline
54. स यानि॒ यानि॒ स स यान्य॒ स्था न्य॒स्थानि॒ यानि॒ स स या न्य॒स्थानि॑ । \newline
55. यान्य॒ स्था न्य॒स्थानि॒ यानि॒ या न्य॒स्था न्यशा॑तय॒ता शा॑तयता॒ स्थानि॒ यानि॒ या न्य॒स्था न्यशा॑तयत । \newline
\pagebreak
\markright{ TS 6.2.8.6  \hfill https://www.vedavms.in \hfill}

\section{ TS 6.2.8.6 }

\textbf{TS 6.2.8.6 } \newline
\textbf{Samhita Paata} \newline

-स्थान्यशा॑तयत॒ तत् पूतु॑द्र्व-भव॒द्-यन्माꣳ॒॒ समुप॑मृतं॒ तद्-गुल्गु॑लु॒ यदे॒तान्थ् स॑भां॒रान्थ् सं॒ भर॑त्य॒ग्निमे॒व तथ् संभ॑रत्य॒ग्नेः पुरी॑ष-म॒सीत्या॑हा॒ग्नेर्ह्ये॑तत् पुरी॑षं॒ ॅयथ् स॑भां॒रा अथो॒ खल्वा॑हुरे॒ते वावैनं॒ ते भ्रात॑रः॒ परि॑ शेरे॒ यत् पौतु॑द्रवाः परि॒धय॒ इति॑ ॥ \newline

\textbf{Pada Paata} \newline

अ॒स्थानि॑ । अशा॑तयत । तत् । पूतु॑द्रु । अ॒भ॒व॒त् । यत् । माꣳ॒॒सम् । उप॑मृत॒मित्युप॑ - मृ॒त॒म् । तत् । गुल्गु॑लु । यत् । ए॒तान् । स॒भां॒रानिति॑ सं - भा॒रान् । स॒म्भर॒तीति॑ सं-भर॑ति । अ॒ग्निम् । ए॒व । तत् । समिति॑ । भ॒र॒ति॒ । अ॒ग्नेः । पुरी॑षम् । अ॒सि॒ । इति॑ । आ॒ह॒ । अ॒ग्नेः । हि । ए॒तत् । पुरी॑षम् । यत् । स॒भां॒रा इति॑ सं - भा॒राः । अथो॒ इति॑ । खलु॑ । आ॒हुः॒ । ए॒ते । वाव । ए॒न॒म् । ते । भ्रात॑रः । परीति॑ । शे॒रे॒ । यत् । पौतु॑द्रवाः । प॒रि॒धय॒ इति॑ परि-धयः॑ । इति॑ ॥  \newline


\textbf{Krama Paata} \newline

अ॒स्थान्यशा॑तयत । अशा॑तयत॒ तत् । तत् पूतु॑द्रु । पूतु॑द्र्वभवत् । अ॒भ॒व॒द् यत् । यन् माꣳ॒॒सम् । माꣳ॒॒समुप॑मृतम् । उप॑मृत॒म् तत् । उप॑मृत॒मित्युप॑ - मृ॒त॒म् । तद् गुल्गु॑लु । गुल्गु॑लु॒ यत् । यदे॒तान् । ए॒तान्थ् 
स॑म्भा॒रान् । स॒म्भा॒रान्थ् स॒म्भर॑ति । स॒म्भा॒रानिति॑ सम् - भा॒रान् । स॒म्भर॑त्य॒ग्निम् । स॒म्भर॒तीति॑ सम् - भर॑ति । अ॒ग्निमे॒व । ए॒व तत् । तथ् सम् । सम् भ॑रति । भ॒र॒त्य॒ग्नेः । अ॒ग्नेः पुरी॑षम् । पुरी॑षमसि । अ॒सीति॑ । इत्या॑ह । आ॒हा॒ग्नेः । अ॒ग्नेर्. हि । ह्ये॑तत् । ए॒तत् पुरी॑षम् । पुरी॑ष॒म् ॅयत् । यथ् स॑म्भा॒राः । स॒म्भा॒रा अथो᳚ । स॒म्भा॒रा इति॑ सम् - भा॒राः । अथो॒ खलु॑ । अथो॒ इत्यथो᳚ । खल्वा॑हुः । आ॒हु॒रे॒ते । ए॒ते वाव । वावैन᳚म् । ए॒न॒म् ते । ते भ्रात॑रः । भ्रात॑रः॒ परि॑ । परि॑ शेरे । शे॒रे॒ यत् । यत् पौतु॑द्रवाः । पौतु॑द्रवाः परि॒धयः॑ । प॒रि॒धय॒ इति॑ । प॒रि॒धय॒ इति॑ परि - धयः॑ । इतीतीति॑ । \newline

\textbf{Jatai Paata} \newline

1. अ॒स्था न्यशा॑तय॒ता शा॑तयता॒ स्था न्य॒स्था न्यशा॑तयत । \newline
2. अशा॑तयत॒ तत् तदशा॑तय॒ता शा॑तयत॒ तत् । \newline
3. तत् पूतु॑द्रु॒ पूतु॑द्रु॒ तत् तत् पूतु॑द्रु । \newline
4. पूतु॑द्‍र्वभवब्दभव॒त् पूतु॑द्रु॒ पूतु॑द्‍र्वभवत् । \newline
5. अ॒भ॒व॒द् यद् यद॑भव दभव॒द् यत् । \newline
6. यन् माꣳ॒॒सम् माꣳ॒॒सं ॅयद् यन् माꣳ॒॒सम् । \newline
7. माꣳ॒॒स मुप॑मृत॒ मुप॑मृतम् माꣳ॒॒सम् माꣳ॒॒स मुप॑मृतम् । \newline
8. उप॑मृत॒म् तत् तदुप॑मृत॒ मुप॑मृत॒म् तत् । \newline
9. उप॑मृत॒मित्युप॑ - मृ॒त॒म् । \newline
10. तद् गुल्गु॑लु॒ गुल्गु॑लु॒ तत् तद् गुल्गु॑लु । \newline
11. गुल्गु॑लु॒ यद् यद् गुल्गु॑लु॒ गुल्गु॑लु॒ यत् । \newline
12. यदे॒ता ने॒तान्. यद् यदे॒तान् । \newline
13. ए॒तान् थ्स॑म्भा॒रान् थ्स॑म्भा॒रा ने॒ता ने॒तान् थ्स॑म्भा॒रान् । \newline
14. स॒म्भा॒रान् थ्स॒म्भर॑ति स॒म्भर॑ति सम्भा॒रान् थ्स॑म्भा॒रान् थ्स॒म्भर॑ति । \newline
15. स॒म्भा॒रानिति॑ सम् - भा॒रान् । \newline
16. स॒म्भर॑ त्य॒ग्नि म॒ग्निꣳ स॒म्भर॑ति स॒म्भर॑ त्य॒ग्निम् । \newline
17. स॒म्भर॒तीति॑ सं - भर॑ति । \newline
18. अ॒ग्नि मे॒वैवाग्नि म॒ग्नि मे॒व । \newline
19. ए॒व तत् तदे॒वैव तत् । \newline
20. तथ् सꣳ सम् तत् तथ् सम् । \newline
21. सम् भ॑रति भरति॒ सꣳ सम् भ॑रति । \newline
22. भ॒र॒ त्य॒ग्ने र॒ग्नेर् भ॑रति भर त्य॒ग्नेः । \newline
23. अ॒ग्नेः पुरी॑ष॒म् पुरी॑ष म॒ग्ने र॒ग्नेः पुरी॑षम् । \newline
24. पुरी॑ष मस्यसि॒ पुरी॑ष॒म् पुरी॑ष मसि । \newline
25. अ॒सीती त्य॑स्य॒सीति॑ । \newline
26. इत्या॑हा॒हे तीत्या॑ह । \newline
27. आ॒हा॒ग्ने र॒ग्ने रा॑हाहा॒ग्नेः । \newline
28. अ॒ग्नेर्. हि ह्य॑ग्ने र॒ग्नेर्. हि । \newline
29. ह्ये॑त दे॒त द्धि ह्ये॑तत् । \newline
30. ए॒तत् पुरी॑ष॒म् पुरी॑ष मे॒त दे॒तत् पुरी॑षम् । \newline
31. पुरी॑षं॒ ॅयद् यत् पुरी॑ष॒म् पुरी॑षं॒ ॅयत् । \newline
32. यथ् सं॑भा॒राः सं॑भा॒रा यद् यथ् सं॑भा॒राः । \newline
33. सं॒भा॒रा अथो॒ अथो॑ संभा॒राः सं॑भा॒रा अथो᳚ । \newline
34. सं॒भा॒रा इति॑ सं - भा॒राः । \newline
35. अथो॒ खलु॒ खल्वथो॒ अथो॒ खलु॑ । \newline
36. अथो॒ इत्यथो᳚ । \newline
37. खल्वा॑हु राहुः॒ खलु॒ खल्वा॑हुः । \newline
38. आ॒हु॒ रे॒त ए॒त आ॑हु राहु रे॒ते । \newline
39. ए॒ते वाव वावैत ए॒ते वाव । \newline
40. वावैन॑ मेनं॒ ॅवाव वावैन᳚म् । \newline
41. ए॒न॒म् ते त ए॑न मेन॒म् ते । \newline
42. ते भ्रात॑रो॒ भ्रात॑र॒ स्ते ते भ्रात॑रः । \newline
43. भ्रात॑रः॒ परि॒ परि॒ भ्रात॑रो॒ भ्रात॑रः॒ परि॑ । \newline
44. परि॑ शेरे शेरे॒ परि॒ परि॑ शेरे । \newline
45. शे॒रे॒ यद् यच्छे॑रे शेरे॒ यत् । \newline
46. यत् पौतु॑द्रवाः॒ पौतु॑द्रवा॒ यद् यत् पौतु॑द्रवाः । \newline
47. पौतु॑द्रवाः परि॒धयः॑ परि॒धयः॒ पौतु॑द्रवाः॒ पौतु॑द्रवाः परि॒धयः॑ । \newline
48. प॒रि॒धय॒ इतीति॑ परि॒धयः॑ परि॒धय॒ इति॑ । \newline
49. प॒रि॒धय॒ इति॑ परि - धयः॑ । \newline
50. इतीतीति॑ । \newline

\textbf{Ghana Paata } \newline

1. अ॒स्था न्यशा॑तय॒ता शा॑तयता॒ स्था न्य॒स्था न्यशा॑तयत॒ तत् तदशा॑तयता॒ स्था न्य॒स्था न्यशा॑तयत॒ तत् । \newline
2. अशा॑तयत॒ तत् तदशा॑तय॒ता शा॑तयत॒ तत् पूतु॑द्रु॒ पूतु॑द्रु॒ तदशा॑तय॒ता शा॑तयत॒ तत् पूतु॑द्रु । \newline
3. तत् पूतु॑द्रु॒ पूतु॑द्रु॒ तत् तत् पूतु॑द्र्‍व भव दभव॒त् पूतु॑द्रु॒ तत् तत् पूतु॑द्र्‍व भवत् । \newline
4. पूतु॑द्र्‍व भवद भव॒त् पूतु॑द्रु॒ पूतु॑द्र्‍व भव॒द् यद् यद॑ भव॒त् पूतु॑द्रु॒ पूतु॑द्र्‍व भव॒द् यत् । \newline
5. अ॒भ॒व॒द् यद् यद॑भव दभव॒द् यन् माꣳ॒॒सम् माꣳ॒॒सं ॅयद॑भव दभव॒द् यन् माꣳ॒॒सम् । \newline
6. यन् माꣳ॒॒सम् माꣳ॒॒सं ॅयद् यन् माꣳ॒॒स मुप॑मृत॒ मुप॑मृतम् माꣳ॒॒सं ॅयद् यन् माꣳ॒॒स मुप॑मृतम् । \newline
7. माꣳ॒॒स मुप॑मृत॒ मुप॑मृतम् माꣳ॒॒सम् माꣳ॒॒स मुप॑मृत॒म् तत् तदुप॑मृतम् माꣳ॒॒सम् माꣳ॒॒स मुप॑मृत॒म् तत् । \newline
8. उप॑मृत॒म् तत् तदुप॑मृत॒ मुप॑मृत॒म् तद् गुल्गु॑लु॒ गुल्गु॑लु॒ तदुप॑मृत॒ मुप॑मृत॒म् तद् गुल्गु॑लु । \newline
9. उप॑मृत॒मित्युप॑ - मृ॒त॒म् । \newline
10. तद् गुल्गु॑लु॒ गुल्गु॑लु॒ तत् तद् गुल्गु॑लु॒ यद् यद् गुल्गु॑लु॒ तत् तद् गुल्गु॑लु॒ यत् । \newline
11. गुल्गु॑लु॒ यद् यद् गुल्गु॑लु॒ गुल्गु॑लु॒ यदे॒ता ने॒तान्. यद् गुल्गु॑लु॒ गुल्गु॑लु॒ यदे॒तान् । \newline
12. यदे॒ता ने॒तान्. यद् यदे॒तान् थ्सं॑भा॒रान् थ्सं॑भा॒रा ने॒तान्. यद् यदे॒तान् थ्सं॑भा॒रान् । \newline
13. ए॒तान् थ्सं॑भा॒रान् थ्सं॑भा॒रा ने॒ता ने॒तान् थ्सं॑भा॒रान् थ्स॒म्भर॑ति स॒म्भर॑ति संभा॒रा ने॒ता ने॒तान् थ्सं॑भा॒रान् थ्स॒म्भर॑ति । \newline
14. सं॒भा॒रान् थ्स॒म्भर॑ति स॒म्भर॑ति संभा॒रान् थ्सं॑भा॒रान् थ्स॒म्भर॑ त्य॒ग्नि म॒ग्निꣳ स॒म्भर॑ति संभा॒रान् थ्सं॑भा॒रान् थ्स॒म्भर॑ त्य॒ग्निम् । \newline
15. सं॒भा॒रानिति॑ सं - भा॒रान् । \newline
16. स॒म्भर॑ त्य॒ग्नि म॒ग्निꣳ स॒म्भर॑ति स॒म्भर॑ त्य॒ग्नि मे॒वैवाग्निꣳ स॒म्भर॑ति स॒म्भर॑ त्य॒ग्नि मे॒व । \newline
17. स॒म्भर॒तीति॑ सं - भर॑ति । \newline
18. अ॒ग्नि मे॒वै वाग्नि म॒ग्नि मे॒व तत् तदे॒ वाग्नि म॒ग्नि मे॒व तत् । \newline
19. ए॒व तत् तदे॒ वैव तथ् सꣳ सम् तदे॒ वैव तथ् सम् । \newline
20. तथ् सꣳ सम् तत् तथ् सम् भ॑रति भरति॒ सम् तत् तथ् सम् भ॑रति । \newline
21. सम् भ॑रति भरति॒ सꣳ सम् भ॑र त्य॒ग्ने र॒ग्नेर् भ॑रति॒ सꣳ सम् भ॑र त्य॒ग्नेः । \newline
22. भ॒र॒ त्य॒ग्ने र॒ग्नेर् भ॑रति भर त्य॒ग्नेः पुरी॑ष॒म् पुरी॑ष म॒ग्नेर् भ॑रति भर त्य॒ग्नेः पुरी॑षम् । \newline
23. अ॒ग्नेः पुरी॑ष॒म् पुरी॑ष म॒ग्ने र॒ग्नेः पुरी॑ष मस्यसि॒ पुरी॑ष म॒ग्ने र॒ग्नेः पुरी॑ष मसि । \newline
24. पुरी॑ष मस्यसि॒ पुरी॑ष॒म् पुरी॑ष म॒सीती त्य॑सि॒ पुरी॑ष॒म् पुरी॑ष म॒सीति॑ । \newline
25. अ॒सी तीत्य॑स्य॒ सीत्या॑हा॒हे त्य॑स्य॒ सीत्या॑ह । \newline
26. इत्या॑हा॒हे तीत्या॑हा॒ग्ने र॒ग्ने रा॒हे तीत्या॑ हा॒ग्नेः । \newline
27. आ॒हा॒ग्ने र॒ग्ने रा॑हा हा॒ग्नेर्. हि ह्य॑ग्ने रा॑हा हा॒ग्नेर्. हि । \newline
28. अ॒ग्नेर्. हि ह्य॑ग्ने र॒ग्नेर् ह्ये॑त दे॒त द्ध्य॑ग्ने र॒ग्नेर् ह्ये॑तत् । \newline
29. ह्ये॑त दे॒तद्धि ह्ये॑तत् पुरी॑ष॒म् पुरी॑ष मे॒तद्धि ह्ये॑तत् पुरी॑षम् । \newline
30. ए॒तत् पुरी॑ष॒म् पुरी॑ष मे॒त दे॒तत् पुरी॑षं॒ ॅयद् यत् पुरी॑ष मे॒त दे॒तत् पुरी॑षं॒ ॅयत् । \newline
31. पुरी॑षं॒ ॅयद् यत् पुरी॑ष॒म् पुरी॑षं॒ ॅयथ् सं॑भा॒राः सं॑भा॒रा यत् पुरी॑ष॒म् पुरी॑षं॒ ॅयथ् सं॑भा॒राः । \newline
32. यथ् सं॑भा॒राः सं॑भा॒रा यद् यथ् सं॑भा॒रा अथो॒ अथो॑ संभा॒रा यद् यथ् सं॑भा॒रा अथो᳚ । \newline
33. सं॒भा॒रा अथो॒ अथो॑ संभा॒राः सं॑भा॒रा अथो॒ खलु॒ खल्वथो॑ संभा॒राः सं॑भा॒रा अथो॒ खलु॑ । \newline
34. सं॒भा॒रा इति॑ सं - भा॒राः । \newline
35. अथो॒ खलु॒ खल्वथो॒ अथो॒ खल्वा॑हु राहुः॒ खल्वथो॒ अथो॒ खल्वा॑हुः । \newline
36. अथो॒ इत्यथो᳚ । \newline
37. खल्वा॑हु राहुः॒ खलु॒ खल्वा॑हु रे॒त ए॒त आ॑हुः॒ खलु॒ खल्वा॑हु रे॒ते । \newline
38. आ॒हु॒ रे॒त ए॒त आ॑हु राहु रे॒ते वाव वावैत आ॑हु राहु रे॒ते वाव । \newline
39. ए॒ते वाव वावैत ए॒ते वावैन॑ मेनं॒ ॅवावैत ए॒ते वावैन᳚म् । \newline
40. वावैन॑ मेनं॒ ॅवाव वावैन॒म् ते त ए॑नं॒ ॅवाव वावैन॒म् ते । \newline
41. ए॒न॒म् ते त ए॑न मेन॒म् ते भ्रात॑रो॒ भ्रात॑र॒ स्त ए॑न मेन॒म् ते भ्रात॑रः । \newline
42. ते भ्रात॑रो॒ भ्रात॑र॒ स्ते ते भ्रात॑रः॒ परि॒ परि॒ भ्रात॑र॒ स्ते ते भ्रात॑रः॒ परि॑ । \newline
43. भ्रात॑रः॒ परि॒ परि॒ भ्रात॑रो॒ भ्रात॑रः॒ परि॑ शेरे शेरे॒ परि॒ भ्रात॑रो॒ भ्रात॑रः॒ परि॑ शेरे । \newline
44. परि॑ शेरे शेरे॒ परि॒ परि॑ शेरे॒ यद् यच्छे॑रे॒ परि॒ परि॑ शेरे॒ यत् । \newline
45. शे॒रे॒ यद् यच्छे॑रे शेरे॒ यत् पौतु॑द्रवाः॒ पौतु॑द्रवा॒ यच्छे॑रे शेरे॒ यत् पौतु॑द्रवाः । \newline
46. यत् पौतु॑द्रवाः॒ पौतु॑द्रवा॒ यद् यत् पौतु॑द्रवाः परि॒धयः॑ परि॒धयः॒ पौतु॑द्रवा॒ यद् यत् पौतु॑द्रवाः परि॒धयः॑ । \newline
47. पौतु॑द्रवाः परि॒धयः॑ परि॒धयः॒ पौतु॑द्रवाः॒ पौतु॑द्रवाः परि॒धय॒ इतीति॑ परि॒धयः॒ पौतु॑द्रवाः॒ पौतु॑द्रवाः परि॒धय॒ इति॑ । \newline
48. प॒रि॒धय॒ इतीति॑ परि॒धयः॑ परि॒धय॒ इति॑ । \newline
49. प॒रि॒धय॒ इति॑ परि - धयः॑ । \newline
50. इतीतीति॑ । \newline
\pagebreak
\markright{ TS 6.2.9.1  \hfill https://www.vedavms.in \hfill}

\section{ TS 6.2.9.1 }

\textbf{TS 6.2.9.1 } \newline
\textbf{Samhita Paata} \newline

ब॒द्धमव॑ स्यति वरुणपा॒शादे॒वैने॑ मुञ्चति॒ प्रणे॑नेक्ति॒ मेद्ध्ये॑ ए॒वैने॑ करोति सावित्रि॒यर्चा हु॒त्वा ह॑वि॒र्द्धाने॒ प्र व॑र्तयति सवि॒तृप्र॑सूत ए॒वैने॒ प्र व॑र्तयति॒ वरु॑णो॒ वा ए॒ष दु॒र्वागु॑भ॒यतो॑ ब॒द्धो यदक्षः॒ स यदु॒थ् सर्जे॒द्-यज॑मानस्य गृ॒हा-न॒भ्युथ्स॑र्जेथ् सु॒वाग्दे॑व॒ दुर्याꣳ॒॒ आ व॒देत्या॑ह गृ॒हा वै दुर्याः॒ शान्त्यै॒ पत्न्यु- [  ] \newline

\textbf{Pada Paata} \newline

ब॒द्धम् । अवेति॑ । स्य॒ति॒ । व॒रु॒ण॒पा॒शादिति॑ वरुण - पा॒शात् । ए॒व । ए॒ने॒ इति॑ । मु॒ञ्च॒ति॒ । प्रेति॑ । ने॒ने॒क्ति॒ । मेद्ध्ये॒ इति॑ । ए॒व । ए॒ने॒ इति॑ । क॒रो॒ति॒ । सा॒वि॒त्रि॒या । ऋ॒चा । हु॒त्वा । ह॒वि॒द्‌र्धाने॒ इति॑ हविः-धाने᳚ । प्रेति॑ । व॒र्त॒य॒ति॒ । स॒वि॒तृप्र॑सूत॒ इति॑ सवि॒तृ - प्र॒सू॒तः॒ । ए॒व । ए॒ने॒ इति॑ । प्रेति॑ । व॒र्त॒य॒ति॒ । वरु॑णः । वै । ए॒षः । दु॒र्वागिति॑ दुः - वाक् । उ॒भ॒यतः॑ । ब॒द्धः । यत् । अक्षः॑ । सः । यत् । उ॒थ्सर्जे॒दित्यु॑त् - सर्जे᳚त् । यज॑मानस्य । गृ॒हान् । अ॒भ्युथ्स॑र्जे॒दित्य॑भि - उथ्स॑र्जेत् । सु॒वागिति॑ सु - वाक् । दे॒व॒ । दुर्यान्॑ । एति॑ । व॒द॒ । इति॑ । आ॒ह॒ । गृ॒हाः । वै । दुर्याः᳚ । शान्त्यै᳚ । पत्नी᳚ ।  \newline


\textbf{Krama Paata} \newline

ब॒द्धमव॑ । अव॑ स्यति । स्य॒ति॒ व॒रु॒ण॒पा॒शात् । व॒रु॒ण॒पा॒शादे॒व । व॒रु॒ण॒पा॒शादिति॑ वरुण - पा॒शात् । ए॒वैने᳚ । ए॒ने॒ मु॒ञ्च॒ति॒ । ए॒ने॒ इत्ये॑ने । मु॒ञ्च॒ति॒ प्र । प्र णे॑नेक्ति । ने॒ने॒क्ति॒ मेद्ध्ये᳚ । मेद्ध्ये॑ ए॒व । मेद्ध्ये॒ इति॒ मेद्ध्ये᳚ । ए॒वैने᳚ । ए॒ने॒ क॒रो॒ति॒ । ए॒ने॒ इत्ये॑ने । क॒रो॒ति॒ सा॒वि॒त्रि॒या । सा॒वि॒त्रि॒यर्चा । ऋ॒चा हु॒त्वा । हु॒त्वा ह॑वि॒र्द्धाने᳚ । ह॒वि॒र्द्धाने॒ प्र । ह॒वि॒र्द्धाने॒ इति॑ हविः - धाने᳚ । प्र व॑र्तयति । व॒र्त॒य॒ति॒ स॒वि॒तृप्र॑सूतः । स॒वि॒तृप्र॑सूत ए॒व । स॒वि॒तृप्र॑सूत॒ इति॑ सवि॒तृ - प्र॒सू॒तः॒ । ए॒वैने᳚ । ए॒ने॒ प्र । ए॒ने॒ इत्ये॑ने । प्र व॑र्तयति । व॒र्त॒य॒ति॒ वरु॑णः । वरु॑णो॒ वै । वा ए॒षः । ए॒ष दु॒र्वाक् । दु॒र्वागु॑भ॒यतः॑ । दु॒र्वागिति॑ दुः - वाक् । उ॒भ॒यतो॑ ब॒द्धः । ब॒द्धो यत् । यदक्षः॑ । अक्षः॒ सः । स यत् । यदु॒थ्सर्जे᳚त् । उ॒थ्सर्जे॒द् यज॑मानस्य । उ॒थ्सर्जे॒दित्यु॑त् - सर्जे᳚त् । यज॑मानस्य गृ॒हान् । गृ॒हान॒भ्युथ्स॑र्जेत् । अ॒भ्युथ्स॑र्जेथ् सु॒वाक् । अ॒भ्युथ्स॑र्जे॒दित्य॑भि - उथ्स॑र्जेत् । सु॒वाग् दे॑व । सु॒वागिति॑ सु - वाक् । दे॒व॒ दुर्यान्॑ । दुर्याꣳ॒॒ आ । आ व॑द । व॒देति॑ । इत्या॑ह । आ॒ह॒ गृ॒हाः । गृ॒हा वै । वै दुर्याः᳚ । दुर्याः॒ शान्त्यै᳚ । शान्त्यै॒ पत्नी᳚ । पत्न्युप॑ \newline

\textbf{Jatai Paata} \newline

1. ब॒द्ध मवाव॑ ब॒द्धम् ब॒द्ध मव॑ । \newline
2. अव॑ स्यति स्य॒ त्यवाव॑ स्यति । \newline
3. स्य॒ति॒ व॒रु॒ण॒पा॒शाद् व॑रुणपा॒शाथ् स्य॑ति स्यति वरुणपा॒शात् । \newline
4. व॒रु॒ण॒पा॒शा दे॒वैव व॑रुणपा॒शाद् व॑रुणपा॒शा दे॒व । \newline
5. व॒रु॒ण॒पा॒शादिति॑ वरुण - पा॒शात् । \newline
6. ए॒वैने॑ एने ए॒वैवैने᳚ । \newline
7. ए॒ने॒ मु॒ञ्च॒ति॒ मु॒ञ्च॒ त्ये॒ने॒ ए॒ने॒ मु॒ञ्च॒ति॒ । \newline
8. ए॒ने॒ इत्ये॑ने । \newline
9. मु॒ञ्च॒ति॒ प्र प्र मु॑ञ्चति मुञ्चति॒ प्र । \newline
10. प्र णे॑नेक्ति नेनेक्ति॒ प्र प्र णे॑नेक्ति । \newline
11. ने॒ने॒क्ति॒ मेद्ध्ये॒ मेद्ध्ये॑ नेनेक्ति नेनेक्ति॒ मेद्ध्ये᳚ । \newline
12. मेद्ध्ये॑ ए॒वैव मेद्ध्ये॒ मेद्ध्ये॑ ए॒व । \newline
13. मेद्ध्ये॒ इति॒ मेद्ध्ये᳚ । \newline
14. ए॒वैने॑ एने ए॒वैवैने᳚ । \newline
15. ए॒ने॒ क॒रो॒ति॒ क॒रो॒ त्ये॒ने॒ ए॒ने॒ क॒रो॒ति॒ । \newline
16. ए॒ने॒ इत्ये॑ने । \newline
17. क॒रो॒ति॒ सा॒वि॒त्रि॒या सा॑वित्रि॒या क॑रोति करोति सावित्रि॒या । \newline
18. सा॒वि॒त्रि॒य र्‌च र्‌चा सा॑वित्रि॒या सा॑वित्रि॒य र्‌चा । \newline
19. ऋ॒चा हु॒त्वा हु॒त्व र्‌च र्‌चा हु॒त्वा । \newline
20. हु॒त्वा ह॑वि॒र्द्धाने॑ हवि॒र्द्धाने॑ हु॒त्वा हु॒त्वा ह॑वि॒र्द्धाने᳚ । \newline
21. ह॒वि॒र्द्धाने॒ प्र प्र ह॑वि॒र्द्धाने॑ हवि॒र्द्धाने॒ प्र । \newline
22. ह॒वि॒र्द्धाने॒ इति॑ हविः - धाने᳚ । \newline
23. प्र व॑र्तयति वर्तयति॒ प्र प्र व॑र्तयति । \newline
24. व॒र्त॒य॒ति॒ स॒वि॒तृप्र॑सूतः सवि॒तृप्र॑सूतो वर्तयति वर्तयति सवि॒तृप्र॑सूतः । \newline
25. स॒वि॒तृप्र॑सूत ए॒वैव स॑वि॒तृप्र॑सूतः सवि॒तृप्र॑सूत ए॒व । \newline
26. स॒वि॒तृप्र॑सूत॒ इति॑ सवि॒तृ - प्र॒सू॒तः॒ । \newline
27. ए॒वैने॑ एने ए॒वैवैने᳚ । \newline
28. ए॒ने॒ प्र प्रैने॑ एने॒ प्र । \newline
29. ए॒ने॒ इत्ये॑ने । \newline
30. प्र व॑र्तयति वर्तयति॒ प्र प्र व॑र्तयति । \newline
31. व॒र्त॒य॒ति॒ वरु॑णो॒ वरु॑णो वर्तयति वर्तयति॒ वरु॑णः । \newline
32. वरु॑णो॒ वै वै वरु॑णो॒ वरु॑णो॒ वै । \newline
33. वा ए॒ष ए॒ष वै वा ए॒षः । \newline
34. ए॒ष दु॒र्वाग् दु॒र्वा गे॒ष ए॒ष दु॒र्वाक् । \newline
35. दु॒र्वा गु॑भ॒यत॑ उभ॒यतो॑ दु॒र्वाग् दु॒र्वा गु॑भ॒यतः॑ । \newline
36. दु॒र्वागिति॑ दुः - वाक् । \newline
37. उ॒भ॒यतो॑ ब॒द्धो ब॒द्ध उ॑भ॒यत॑ उभ॒यतो॑ ब॒द्धः । \newline
38. ब॒द्धो यद् यद् ब॒द्धो ब॒द्धो यत् । \newline
39. यदक्षो ऽक्षो॒ यद् यदक्षः॑ । \newline
40. अक्षः॒ स सो ऽक्षो ऽक्षः॒ सः । \newline
41. स यद् यथ् स स यत् । \newline
42. यदु॒थ्सर्जे॑ दु॒थ्सर्जे॒द् यद् यदु॒थ्सर्जे᳚त् । \newline
43. उ॒थ्सर्जे॒द् यज॑मानस्य॒ यज॑मान स्यो॒थ्सर्जे॑ दु॒थ्सर्जे॒द् यज॑मानस्य । \newline
44. उ॒थ्सर्जे॒दित्यु॑त् - सर्जे᳚त् । \newline
45. यज॑मानस्य गृ॒हान् गृ॒हान्. यज॑मानस्य॒ यज॑मानस्य गृ॒हान् । \newline
46. गृ॒हा न॒भ्युथ्स॑र्जे द॒भ्युथ्स॑र्जेद् गृ॒हान् गृ॒हा न॒भ्युथ्स॑र्जेत् । \newline
47. अ॒भ्युथ्स॑र्जेथ् सु॒वाख् सु॒वा ग॒भ्युथ्स॑र्जे द॒भ्युथ्स॑र्जेथ् सु॒वाक् । \newline
48. अ॒भ्युथ्स॑र्जे॒दित्य॑भि - उथ्स॑र्जेत् । \newline
49. सु॒वाग् दे॑व देव सु॒वाख् सु॒वाग् दे॑व । \newline
50. सु॒वागिति॑ सु - वाक् । \newline
51. दे॒व॒ दुर्या॒न् दुर्या᳚न् देव देव॒ दुर्यान्॑ । \newline
52. दुर्याꣳ॒॒ आ दुर्या॒न् दुर्याꣳ॒॒ आ । \newline
53. आ व॑द व॒दा व॑द । \newline
54. व॒दे तीति॑ वद व॒देति॑ । \newline
55. इत्या॑हा॒हे तीत्या॑ह । \newline
56. आ॒ह॒ गृ॒हा गृ॒हा आ॑हाह गृ॒हाः । \newline
57. गृ॒हा वै वै गृ॒हा गृ॒हा वै । \newline
58. वै दुर्या॒ दुर्या॒ वै वै दुर्याः᳚ । \newline
59. दुर्याः॒ शान्त्यै॒ शान्त्यै॒ दुर्या॒ दुर्याः॒ शान्त्यै᳚ । \newline
60. शान्त्यै॒ पत्नी॒ पत्नी॒ शान्त्यै॒ शान्त्यै॒ पत्नी᳚ । \newline
61. पत्न्युपोप॒ पत्नी॒ पत्न्युप॑ । \newline

\textbf{Ghana Paata } \newline

1. ब॒द्ध मवाव॑ ब॒द्धम् ब॒द्ध मव॑ स्यति स्य॒त्यव॑ ब॒द्धम् ब॒द्ध मव॑ स्यति । \newline
2. अव॑ स्यति स्य॒त्य वाव॑ स्यति वरुणपा॒शाद् व॑रुणपा॒शाथ् स्य॒त्य वाव॑ स्यति वरुणपा॒शात् । \newline
3. स्य॒ति॒ व॒रु॒ण॒पा॒शाद् व॑रुणपा॒शाथ् स्य॑ति स्यति वरुणपा॒शा दे॒वैव व॑रुणपा॒शाथ् स्य॑ति स्यति वरुणपा॒शा दे॒व । \newline
4. व॒रु॒ण॒पा॒शा दे॒वैव व॑रुणपा॒शाद् व॑रुणपा॒शा दे॒वैने॑ एने ए॒व व॑रुणपा॒शाद् व॑रुणपा॒शा दे॒वैने᳚ । \newline
5. व॒रु॒ण॒पा॒शादिति॑ वरुण - पा॒शात् । \newline
6. ए॒वैने॑ एने ए॒वै वैने॑ मुञ्चति मुञ्च त्येने ए॒वै वैने॑ मुञ्चति । \newline
7. ए॒ने॒ मु॒ञ्च॒ति॒ मु॒ञ्च॒ त्ये॒ने॒ ए॒ने॒ मु॒ञ्च॒ति॒ प्र प्र मु॑ञ्च त्येने एने मुञ्चति॒ प्र । \newline
8. ए॒ने॒ इत्ये॑ने । \newline
9. मु॒ञ्च॒ति॒ प्र प्र मु॑ञ्चति मुञ्चति॒ प्र णे॑नेक्ति नेनेक्ति॒ प्र मु॑ञ्चति मुञ्चति॒ प्र णे॑नेक्ति । \newline
10. प्र णे॑नेक्ति नेनेक्ति॒ प्र प्र णे॑नेक्ति॒ मेद्ध्ये॒ मेद्ध्ये॑ नेनेक्ति॒ प्र प्र णे॑नेक्ति॒ मेद्ध्ये᳚ । \newline
11. ने॒ने॒क्ति॒ मेद्ध्ये॒ मेद्ध्ये॑ नेनेक्ति नेनेक्ति॒ मेद्ध्ये॑ ए॒वैव मेद्ध्ये॑ नेनेक्ति नेनेक्ति॒ मेद्ध्ये॑ ए॒व । \newline
12. मेद्ध्ये॑ ए॒वैव मेद्ध्ये॒ मेद्ध्ये॑ ए॒वैने॑ एने ए॒व मेद्ध्ये॒ मेद्ध्ये॑ ए॒वैने᳚ । \newline
13. मेद्ध्ये॒ इति॒ मेद्ध्ये᳚ । \newline
14. ए॒वैने॑ एने ए॒वै वैने॑ करोति करो त्येने ए॒वै वैने॑ करोति । \newline
15. ए॒ने॒ क॒रो॒ति॒ क॒रो॒ त्ये॒ने॒ ए॒ने॒ क॒रो॒ति॒ सा॒वि॒त्रि॒या सा॑वित्रि॒या क॑रो त्येने एने करोति सावित्रि॒या । \newline
16. ए॒ने॒ इत्ये॑ने । \newline
17. क॒रो॒ति॒ सा॒वि॒त्रि॒या सा॑वित्रि॒या क॑रोति करोति सावित्रि॒य र्‌च र्‌चा सा॑वित्रि॒या क॑रोति करोति सावित्रि॒य र्‌चा । \newline
18. सा॒वि॒त्रि॒य र्‌च र्‌चा सा॑वित्रि॒या सा॑वित्रि॒य र्‌चा हु॒त्वा हु॒त्वर्चा सा॑वित्रि॒या सा॑वित्रि॒य र्‌चा हु॒त्वा । \newline
19. ऋ॒चा हु॒त्वा हु॒त्व र्‌च र्‌चा हु॒त्वा ह॑वि॒र्द्धाने॑ हवि॒र्द्धाने॑ हु॒त्व र्‌च र्‌चा हु॒त्वा ह॑वि॒र्द्धाने᳚ । \newline
20. हु॒त्वा ह॑वि॒र्द्धाने॑ हवि॒र्द्धाने॑ हु॒त्वा हु॒त्वा ह॑वि॒र्द्धाने॒ प्र प्र ह॑वि॒र्द्धाने॑ हु॒त्वा हु॒त्वा ह॑वि॒र्द्धाने॒ प्र । \newline
21. ह॒वि॒र्द्धाने॒ प्र प्र ह॑वि॒र्द्धाने॑ हवि॒र्द्धाने॒ प्र व॑र्तयति वर्तयति॒ प्र ह॑वि॒र्द्धाने॑ हवि॒र्द्धाने॒ प्र व॑र्तयति । \newline
22. ह॒वि॒र्द्धाने॒ इति॑ हविः - धाने᳚ । \newline
23. प्र व॑र्तयति वर्तयति॒ प्र प्र व॑र्तयति सवि॒तृप्र॑सूतः सवि॒तृप्र॑सूतो वर्तयति॒ प्र प्र व॑र्तयति सवि॒तृप्र॑सूतः । \newline
24. व॒र्त॒य॒ति॒ स॒वि॒तृप्र॑सूतः सवि॒तृप्र॑सूतो वर्तयति वर्तयति सवि॒तृप्र॑सूत ए॒वैव स॑वि॒तृप्र॑सूतो वर्तयति वर्तयति सवि॒तृप्र॑सूत ए॒व । \newline
25. स॒वि॒तृप्र॑सूत ए॒वैव स॑वि॒तृप्र॑सूतः सवि॒तृप्र॑सूत ए॒वैने॑ एने ए॒व स॑वि॒तृप्र॑सूतः सवि॒तृप्र॑सूत ए॒वैने᳚ । \newline
26. स॒वि॒तृप्र॑सूत॒ इति॑ सवि॒तृ - प्र॒सू॒तः॒ । \newline
27. ए॒वैने॑ एने ए॒वै वैने॒ प्र प्रैने॑ ए॒वै वैने॒ प्र । \newline
28. ए॒ने॒ प्र प्रैने॑ एने॒ प्र व॑र्तयति वर्तयति॒ प्रैने॑ एने॒ प्र व॑र्तयति । \newline
29. ए॒ने॒ इत्ये॑ने । \newline
30. प्र व॑र्तयति वर्तयति॒ प्र प्र व॑र्तयति॒ वरु॑णो॒ वरु॑णो वर्तयति॒ प्र प्र व॑र्तयति॒ वरु॑णः । \newline
31. व॒र्त॒य॒ति॒ वरु॑णो॒ वरु॑णो वर्तयति वर्तयति॒ वरु॑णो॒ वै वै वरु॑णो वर्तयति वर्तयति॒ वरु॑णो॒ वै । \newline
32. वरु॑णो॒ वै वै वरु॑णो॒ वरु॑णो॒ वा ए॒ष ए॒ष वै वरु॑णो॒ वरु॑णो॒ वा ए॒षः । \newline
33. वा ए॒ष ए॒ष वै वा ए॒ष दु॒र्वाग् दु॒र्वा गे॒ष वै वा ए॒ष दु॒र्वाक् । \newline
34. ए॒ष दु॒र्वाग् दु॒र्वा गे॒ष ए॒ष दु॒र्वा गु॑भ॒यत॑ उभ॒यतो॑ दु॒र्वा गे॒ष ए॒ष दु॒र्वा गु॑भ॒यतः॑ । \newline
35. दु॒र्वा गु॑भ॒यत॑ उभ॒यतो॑ दु॒र्वाग् दु॒र्वा गु॑भ॒यतो॑ ब॒द्धो ब॒द्ध उ॑भ॒यतो॑ दु॒र्वाग् दु॒र्वा गु॑भ॒यतो॑ ब॒द्धः । \newline
36. दु॒र्वागिति॑ दुः - वाक् । \newline
37. उ॒भ॒यतो॑ ब॒द्धो ब॒द्ध उ॑भ॒यत॑ उभ॒यतो॑ ब॒द्धो यद् यद् ब॒द्ध उ॑भ॒यत॑ उभ॒यतो॑ ब॒द्धो यत् । \newline
38. ब॒द्धो यद् यद् ब॒द्धो ब॒द्धो यदक्षो ऽक्षो॒ यद् ब॒द्धो ब॒द्धो यदक्षः॑ । \newline
39. यदक्षो ऽक्षो॒ यद् यदक्षः॒ स सो ऽक्षो॒ यद् यदक्षः॒ सः । \newline
40. अक्षः॒ स सो ऽक्षो ऽक्षः॒ स यद् यथ् सो ऽक्षो ऽक्षः॒ स यत् । \newline
41. स यद् यथ् स स यदु॒थ्सर्जे॑ दु॒थ्सर्जे॒द् यथ् स स यदु॒थ्सर्जे᳚त् । \newline
42. यदु॒थ्सर्जे॑ दु॒थ्सर्जे॒द् यद् यदु॒थ्सर्जे॒द् यज॑मानस्य॒ यज॑मान स्यो॒थ्सर्जे॒द् यद् यदु॒थ्सर्जे॒द् यज॑मानस्य । \newline
43. उ॒थ्सर्जे॒द् यज॑मानस्य॒ यज॑मान स्यो॒थ्सर्जे॑ दु॒थ्सर्जे॒द् यज॑मानस्य गृ॒हान् गृ॒हान्. यज॑मान स्यो॒थ्सर्जे॑ दु॒थ्सर्जे॒द् यज॑मानस्य गृ॒हान् । \newline
44. उ॒थ्सर्जे॒दित्यु॑त् - सर्जे᳚त् । \newline
45. यज॑मानस्य गृ॒हान् गृ॒हान्. यज॑मानस्य॒ यज॑मानस्य गृ॒हा न॒भ्युथ्स॑र्जे द॒भ्युथ्स॑र्जेद् गृ॒हान्. यज॑मानस्य॒ यज॑मानस्य गृ॒हा न॒भ्युथ्स॑र्जेत् । \newline
46. गृ॒हा न॒भ्युथ्स॑र्जे द॒भ्युथ्स॑र्जेद् गृ॒हान् गृ॒हा न॒भ्युथ्स॑र्जेथ् सु॒वाख् सु॒वा ग॒भ्युथ्स॑र्जेद् गृ॒हान् गृ॒हा न॒भ्युथ्स॑र्जेथ् सु॒वाक् । \newline
47. अ॒भ्युथ्स॑र्जेथ् सु॒वाख् सु॒वा ग॒भ्युथ्स॑र्जे द॒भ्युथ्स॑र्जेथ् सु॒वाग् दे॑व देव सु॒वा ग॒भ्युथ्स॑र्जे द॒भ्युथ्स॑र्जेथ् सु॒वाग् दे॑व । \newline
48. अ॒भ्युथ्स॑र्जे॒दित्य॑भि - उथ्स॑र्जेत् । \newline
49. सु॒वाग् दे॑व देव सु॒वाख् सु॒वाग् दे॑व॒ दुर्या॒न् दुर्या᳚न् देव सु॒वाख् सु॒वाग् दे॑व॒ दुर्यान्॑ । \newline
50. सु॒वागिति॑ सु - वाक् । \newline
51. दे॒व॒ दुर्या॒न् दुर्या᳚न् देव देव॒ दुर्याꣳ॒॒ आ दुर्या᳚न् देव देव॒ दुर्याꣳ॒॒ आ । \newline
52. दुर्याꣳ॒॒ आ दुर्या॒न् दुर्याꣳ॒॒ आ व॑द व॒दा दुर्या॒न् दुर्याꣳ॒॒ आ व॑द । \newline
53. आ व॑द व॒दा व॒देतीति॑ व॒दा व॒देति॑ । \newline
54. व॒दे तीति॑ वद व॒दे त्या॑हा॒हेति॑ वद व॒दे त्या॑ह । \newline
55. इत्या॑हा॒हे तीत्या॑ह गृ॒हा गृ॒हा आ॒हे तीत्या॑ह गृ॒हाः । \newline
56. आ॒ह॒ गृ॒हा गृ॒हा आ॑हाह गृ॒हा वै वै गृ॒हा आ॑हाह गृ॒हा वै । \newline
57. गृ॒हा वै वै गृ॒हा गृ॒हा वै दुर्या॒ दुर्या॒ वै गृ॒हा गृ॒हा वै दुर्याः᳚ । \newline
58. वै दुर्या॒ दुर्या॒ वै वै दुर्याः॒ शान्त्यै॒ शान्त्यै॒ दुर्या॒ वै वै दुर्याः॒ शान्त्यै᳚ । \newline
59. दुर्याः॒ शान्त्यै॒ शान्त्यै॒ दुर्या॒ दुर्याः॒ शान्त्यै॒ पत्नी॒ पत्नी॒ शान्त्यै॒ दुर्या॒ दुर्याः॒ शान्त्यै॒ पत्नी᳚ । \newline
60. शान्त्यै॒ पत्नी॒ पत्नी॒ शान्त्यै॒ शान्त्यै॒ पत्न्यु पोप॒ पत्नी॒ शान्त्यै॒ शान्त्यै॒ पत्न्युप॑ । \newline
61. पत्न्यु पोप॒ पत्नी॒ पत्न्युपा॑ नक्त्य न॒क्त्युप॒ पत्नी॒ पत्न्युपा॑ नक्ति । \newline
\pagebreak
\markright{ TS 6.2.9.2  \hfill https://www.vedavms.in \hfill}

\section{ TS 6.2.9.2 }

\textbf{TS 6.2.9.2 } \newline
\textbf{Samhita Paata} \newline

पा॑नक्ति॒ पत्नी॒ हि सर्व॑स्य मि॒त्रं मि॑त्र॒त्वाय॒ यद्वै पत्नी॑ य॒ज्ञ्स्य॑ क॒रोति॑ मिथु॒नं तदथो॒ पत्नि॑या ए॒वैष य॒ज्ञ्स्या᳚-न्वार॒भ्ॐऽन॑वच्छित्त्यै॒ वर्त्म॑ना॒ वा अ॒न्वित्य॑ य॒ज्ञ्ꣳ रक्षाꣳ॑सि जिघाꣳसन्ति वैष्ण॒वीभ्या॑मृ॒ग्भ्यां ॅवर्त्म॑नो र्जुहोति य॒ज्ञो वै विष्णु॑र्य॒ज्ञादे॒व रक्षाꣳ॒॒स्यप॑ हन्ति॒ यद॑द्ध्व॒र्यु-र॑न॒ग्ना-वाहु॑तिं जुहु॒या-द॒न्धो᳚ऽद्ध्व॒र्युः स्या॒द्-रक्षाꣳ॑सि य॒ज्ञ्ꣳ ह॑न्यु॒र्- [  ] \newline

\textbf{Pada Paata} \newline

उपेति॑ । अ॒न॒क्ति॒ । पत्नी᳚ । हि । सर्व॑स्य । मि॒त्रम् । मि॒त्र॒त्वायेति॑ मित्र - त्वाय॑ । यत् । वै । पत्नी᳚ । य॒ज्ञ्स्य॑ । क॒रोति॑ । मि॒थु॒नम् । तत् । अथो॒ इति॑ । पत्नि॑याः । ए॒व । ए॒षः । य॒ज्ञ्स्य॑ । अ॒न्वा॒र॒भं इत्य॑नु - आ॒र॒भंः । अन॑वच्छित्त्या॒ इत्यन॑व - छि॒त्त्यै॒ । वर्त्म॑ना । वै । अ॒न्वित्येत्य॑नु - इत्य॑ । य॒ज्ञ्म् । रक्षाꣳ॑सि । जि॒घाꣳ॒॒स॒न्ति॒ । वै॒ष्ण॒वीभ्या᳚म् । ऋ॒ग्भ्यामित्यृ॑क्-भ्याम् । वर्त्म॑नोः । जु॒हो॒ति॒ । य॒ज्ञ्ः । वै । विष्णुः॑ । य॒ज्ञात् । ए॒व । रक्षाꣳ॑सि । अपेति॑ । ह॒न्ति॒ । यत् । अ॒द्ध्व॒र्युः । अ॒न॒ग्नौ । आहु॑ति॒मित्या - हु॒ति॒म् । जु॒हु॒यात् । अ॒न्धः । अ॒द्ध्व॒र्युः । स्या॒त् । रक्षाꣳ॑सि । य॒ज्ञ्म् । ह॒न्युः॒ ।  \newline


\textbf{Krama Paata} \newline

उपा॑नक्ति । अ॒न॒क्ति॒ पत्नी᳚ । पत्नी॒ हि । हि सर्व॑स्य । सर्व॑स्य मि॒त्रम् । मि॒त्रम् मि॑त्र॒त्वाय॑ । मि॒त्र॒त्वाय॒ यत् । मि॒त्र॒त्वायेति॑ मित्र - त्वाय॑ । यद् वै । वै पत्नी᳚ । पत्नी॑ य॒ज्ञ्स्य॑ । य॒ज्ञ्स्य॑ क॒रोति॑ । क॒रोति॑ मिथु॒नम् । मि॒थु॒नम् तत् । तदथो᳚ । अथो॒ पत्नि॑याः । अथो॒ इत्यथो᳚ । पत्नि॑या ए॒व । ए॒वैषः । ए॒ष य॒ज्ञ्स्य॑ । य॒ज्ञ्स्या᳚न्वार॒म्भः । अ॒न्वा॒र॒म्भोऽन॑वच्छित्त्यै । अ॒न्वा॒र॒म्भ इत्य॑नु - आ॒र॒म्भः । अन॑वच्छित्त्यै॒ वर्त्म॑ना । अन॑वच्छित्त्या॒ इत्यन॑व - छि॒त्त्यै॒ । वर्त्म॑ना॒ वै । वा अ॒न्वित्य॑ । अ॒न्वित्य॑ य॒ज्ञ्म् । अ॒न्वित्येत्य॑नु - इत्य॑ । य॒ज्ञ्ꣳ रक्षाꣳ॑सि । रक्षाꣳ॑सि जिघाꣳसन्ति । जि॒घाꣳ॒॒स॒न्ति॒ वै॒ष्ण॒वीभ्या᳚म् । वै॒ष्ण॒वीभ्या॑मृ॒ग्भ्याम् । ऋ॒ग्भ्याम् ॅवर्त्म॑नोः । ऋ॒ग्भ्यामित्यृ॑क् - भ्याम् । वर्त्म॑नोर् जुहोति । जु॒हो॒ति॒ य॒ज्ञ्ः । य॒ज्ञो वै । वै विष्णुः॑ । विष्णु॑र् य॒ज्ञात् । य॒ज्ञादे॒व । ए॒व रक्षाꣳ॑सि । रक्षाꣳ॒॒स्यप॑ । अप॑ हन्ति । ह॒न्ति॒ यत् । यद॑द्ध्व॒र्युः । अ॒द्ध्व॒र्युर॑न॒ग्नौ । अ॒न॒ग्नावाहु॑तिम् । आहु॑तिम् जुहु॒यात् । आहु॑ति॒मित्या - हु॒ति॒म् । जु॒हु॒याद॒न्धः । अ॒न्धो᳚ऽद्ध्व॒र्युः । अ॒द्ध्व॒र्युः स्या᳚त् । स्या॒द् रक्षाꣳ॑सि । रक्षाꣳ॑सि य॒ज्ञ्म् । य॒ज्ञ्ꣳ ह॑न्युः । ह॒न्यु॒र्.॒ हिर॑ण्यम् \newline

\textbf{Jatai Paata} \newline

1. उपा॑ नक्त्य न॒क्त्युपोपा॑ नक्ति । \newline
2. अ॒न॒क्ति॒ पत्नी॒ पत्न्य॑ नक्त्यनक्ति॒ पत्नी᳚ । \newline
3. पत्नी॒ हि हि पत्नी॒ पत्नी॒ हि । \newline
4. हि सर्व॑स्य॒ सर्व॑स्य॒ हि हि सर्व॑स्य । \newline
5. सर्व॑स्य मि॒त्रम् मि॒त्रꣳ सर्व॑स्य॒ सर्व॑स्य मि॒त्रम् । \newline
6. मि॒त्रम् मि॑त्र॒त्वाय॑ मित्र॒त्वाय॑ मि॒त्रम् मि॒त्रम् मि॑त्र॒त्वाय॑ । \newline
7. मि॒त्र॒त्वाय॒ यद् यन् मि॑त्र॒त्वाय॑ मित्र॒त्वाय॒ यत् । \newline
8. मि॒त्र॒त्वायेति॑ मित्र - त्वाय॑ । \newline
9. यद् वै वै यद् यद् वै । \newline
10. वै पत्नी॒ पत्नी॒ वै वै पत्नी᳚ । \newline
11. पत्नी॑ य॒ज्ञ्स्य॑ य॒ज्ञ्स्य॒ पत्नी॒ पत्नी॑ य॒ज्ञ्स्य॑ । \newline
12. य॒ज्ञ्स्य॑ क॒रोति॑ क॒रोति॑ य॒ज्ञ्स्य॑ य॒ज्ञ्स्य॑ क॒रोति॑ । \newline
13. क॒रोति॑ मिथु॒नम् मि॑थु॒नम् क॒रोति॑ क॒रोति॑ मिथु॒नम् । \newline
14. मि॒थु॒नम् तत् तन् मि॑थु॒नम् मि॑थु॒नम् तत् । \newline
15. तदथो॒ अथो॒ तत् तदथो᳚ । \newline
16. अथो॒ पत्नि॑याः॒ पत्नि॑या॒ अथो॒ अथो॒ पत्नि॑याः । \newline
17. अथो॒ इत्यथो᳚ । \newline
18. पत्नि॑या ए॒वैव पत्नि॑याः॒ पत्नि॑या ए॒व । \newline
19. ए॒वैष ए॒ष ए॒वैवैषः । \newline
20. ए॒ष य॒ज्ञ्स्य॑ य॒ज्ञ्स्यै॒ष ए॒ष य॒ज्ञ्स्य॑ । \newline
21. य॒ज्ञ्स्या᳚ न्वारं॒भो᳚ ऽन्वारं॒भो य॒ज्ञ्स्य॑ य॒ज्ञ्स्या᳚ न्वारं॒भः । \newline
22. अ॒न्वा॒रं॒भो ऽन॑वच्छित्त्या॒ अन॑वच्छित्त्या अन्वारं॒भो᳚ ऽन्वारं॒भो ऽन॑वच्छित्त्यै । \newline
23. अ॒न्वा॒रं॒भ इत्य॑नु - आ॒रं॒भः । \newline
24. अन॑वच्छित्त्यै॒ वर्त्म॑ना॒ वर्त्म॒ना ऽन॑वच्छित्त्या॒ अन॑वच्छित्त्यै॒ वर्त्म॑ना । \newline
25. अन॑वच्छित्त्या॒ इत्यन॑व - छि॒त्त्यै॒ । \newline
26. वर्त्म॑ना॒ वै वै वर्त्म॑ना॒ वर्त्म॑ना॒ वै । \newline
27. वा अ॒न्वित्या॒न् वित्य॒ वै वा अ॒न्वित्य॑ । \newline
28. अ॒न्वित्य॑ य॒ज्ञ्ं ॅय॒ज्ञ् म॒न्वित्या॒न् वित्य॑ य॒ज्ञ्म् । \newline
29. अ॒न्वित्येत्य॑नु - इत्य॑ । \newline
30. य॒ज्ञ्ꣳ रक्षाꣳ॑सि॒ रक्षाꣳ॑सि य॒ज्ञ्ं ॅय॒ज्ञ्ꣳ रक्षाꣳ॑सि । \newline
31. रक्षाꣳ॑सि जिघाꣳसन्ति जिघाꣳसन्ति॒ रक्षाꣳ॑सि॒ रक्षाꣳ॑सि जिघाꣳसन्ति । \newline
32. जि॒घाꣳ॒॒स॒न्ति॒ वै॒ष्ण॒वीभ्यां᳚ ॅवैष्ण॒वीभ्या᳚म् जिघाꣳसन्ति जिघाꣳसन्ति वैष्ण॒वीभ्या᳚म् । \newline
33. वै॒ष्ण॒वीभ्या॑ मृ॒ग्भ्या मृ॒ग्भ्यां ॅवै᳚ष्ण॒वीभ्यां᳚ ॅवैष्ण॒वीभ्या॑ मृ॒ग्भ्याम् । \newline
34. ऋ॒ग्भ्यां ॅवर्त्म॑नो॒र् वर्त्म॑नोर्. ऋ॒ग्भ्या मृ॒ग्भ्यां ॅवर्त्म॑नोः । \newline
35. ऋ॒ग्भ्यामित्यृ॑क् - भ्याम् । \newline
36. वर्त्म॑नोर् जुहोति जुहोति॒ वर्त्म॑नो॒र् वर्त्म॑नोर् जुहोति । \newline
37. जु॒हो॒ति॒ य॒ज्ञो य॒ज्ञो जु॑होति जुहोति य॒ज्ञ्ः । \newline
38. य॒ज्ञो वै वै य॒ज्ञो य॒ज्ञो वै । \newline
39. वै विष्णु॒र् विष्णु॒र् वै वै विष्णुः॑ । \newline
40. विष्णु॑र् य॒ज्ञाद् य॒ज्ञाद् विष्णु॒र् विष्णु॑र् य॒ज्ञात् । \newline
41. य॒ज्ञा दे॒वैव य॒ज्ञाद् य॒ज्ञा दे॒व । \newline
42. ए॒व रक्षाꣳ॑सि॒ रक्षाꣳ॑स्ये॒वैव रक्षाꣳ॑सि । \newline
43. रक्षाꣳ॒॒ स्यपाप॒ रक्षाꣳ॑सि॒ रक्षाꣳ॒॒ स्यप॑ । \newline
44. अप॑ हन्ति ह॒न्त्यपाप॑ हन्ति । \newline
45. ह॒न्ति॒ यद् यद्ध॑न्ति हन्ति॒ यत् । \newline
46. यद॑द्ध्व॒र्यु र॑द्ध्व॒र्युर् यद् यद॑द्ध्व॒र्युः । \newline
47. अ॒द्ध्व॒र्यु र॑न॒ग्ना व॑न॒ग्ना व॑द्ध्व॒र्यु र॑द्ध्व॒र्यु र॑न॒ग्नौ । \newline
48. अ॒न॒ग्ना वाहु॑ति॒ माहु॑ति मन॒ग्ना व॑न॒ग्ना वाहु॑तिम् । \newline
49. आहु॑तिम् जुहु॒याज् जु॑हु॒या दाहु॑ति॒ माहु॑तिम् जुहु॒यात् । \newline
50. आहु॑ति॒मित्या - हु॒ति॒म् । \newline
51. जु॒हु॒या द॒न्धो᳚ ऽन्धो जु॑हु॒याज् जु॑हु॒या द॒न्धः । \newline
52. अ॒न्धो᳚ ऽद्ध्व॒र्यु र॑द्ध्व॒र्यु र॒न्धो᳚(1॒) ऽन्धो᳚ ऽद्ध्व॒र्युः । \newline
53. अ॒द्ध्व॒र्युः स्या᳚थ् स्या दद्ध्व॒र्यु र॑द्ध्व॒र्युः स्या᳚त् । \newline
54. स्या॒द् रक्षाꣳ॑सि॒ रक्षाꣳ॑सि स्याथ् स्या॒द् रक्षाꣳ॑सि । \newline
55. रक्षाꣳ॑सि य॒ज्ञ्ं ॅय॒ज्ञ्ꣳ रक्षाꣳ॑सि॒ रक्षाꣳ॑सि य॒ज्ञ्म् । \newline
56. य॒ज्ञ्ꣳ ह॑न्युर्. हन्युर् य॒ज्ञ्ं ॅय॒ज्ञ्ꣳ ह॑न्युः । \newline
57. ह॒न्यु॒र्॒. हिर॑ण्यꣳ॒॒ हिर॑ण्यꣳ हन्युर्. हन्यु॒र्॒. हिर॑ण्यम् । \newline

\textbf{Ghana Paata } \newline

1. उपा॑नक् त्यन॒क् त्युपोपा॑ नक्ति॒ पत्नी॒ पत्न्य॑ न॒क्त्युपोपा॑ नक्ति॒ पत्नी᳚ । \newline
2. अ॒न॒क्ति॒ पत्नी॒ पत्न्य॑नक् त्यनक्ति॒ पत्नी॒ हि हि पत्न्य॑नक् त्यनक्ति॒ पत्नी॒ हि । \newline
3. पत्नी॒ हि हि पत्नी॒ पत्नी॒ हि सर्व॑स्य॒ सर्व॑स्य॒ हि पत्नी॒ पत्नी॒ हि सर्व॑स्य । \newline
4. हि सर्व॑स्य॒ सर्व॑स्य॒ हि हि सर्व॑स्य मि॒त्रम् मि॒त्रꣳ सर्व॑स्य॒ हि हि सर्व॑स्य मि॒त्रम् । \newline
5. सर्व॑स्य मि॒त्रम् मि॒त्रꣳ सर्व॑स्य॒ सर्व॑स्य मि॒त्रम् मि॑त्र॒त्वाय॑ मित्र॒त्वाय॑ मि॒त्रꣳ सर्व॑स्य॒ सर्व॑स्य मि॒त्रम् मि॑त्र॒त्वाय॑ । \newline
6. मि॒त्रम् मि॑त्र॒त्वाय॑ मित्र॒त्वाय॑ मि॒त्रम् मि॒त्रम् मि॑त्र॒त्वाय॒ यद् यन् मि॑त्र॒त्वाय॑ मि॒त्रम् मि॒त्रम् मि॑त्र॒त्वाय॒ यत् । \newline
7. मि॒त्र॒त्वाय॒ यद् यन् मि॑त्र॒त्वाय॑ मित्र॒त्वाय॒ यद् वै वै यन् मि॑त्र॒त्वाय॑ मित्र॒त्वाय॒ यद् वै । \newline
8. मि॒त्र॒त्वायेति॑ मित्र - त्वाय॑ । \newline
9. यद् वै वै यद् यद् वै पत्नी॒ पत्नी॒ वै यद् यद् वै पत्नी᳚ । \newline
10. वै पत्नी॒ पत्नी॒ वै वै पत्नी॑ य॒ज्ञ्स्य॑ य॒ज्ञ्स्य॒ पत्नी॒ वै वै पत्नी॑ य॒ज्ञ्स्य॑ । \newline
11. पत्नी॑ य॒ज्ञ्स्य॑ य॒ज्ञ्स्य॒ पत्नी॒ पत्नी॑ य॒ज्ञ्स्य॑ क॒रोति॑ क॒रोति॑ य॒ज्ञ्स्य॒ पत्नी॒ पत्नी॑ य॒ज्ञ्स्य॑ क॒रोति॑ । \newline
12. य॒ज्ञ्स्य॑ क॒रोति॑ क॒रोति॑ य॒ज्ञ्स्य॑ य॒ज्ञ्स्य॑ क॒रोति॑ मिथु॒नम् मि॑थु॒नम् क॒रोति॑ य॒ज्ञ्स्य॑ य॒ज्ञ्स्य॑ क॒रोति॑ मिथु॒नम् । \newline
13. क॒रोति॑ मिथु॒नम् मि॑थु॒नम् क॒रोति॑ क॒रोति॑ मिथु॒नम् तत् तन् मि॑थु॒नम् क॒रोति॑ क॒रोति॑ मिथु॒नम् तत् । \newline
14. मि॒थु॒नम् तत् तन् मि॑थु॒नम् मि॑थु॒नम् तदथो॒ अथो॒ तन् मि॑थु॒नम् मि॑थु॒नम् तदथो᳚ । \newline
15. तदथो॒ अथो॒ तत् तदथो॒ पत्नि॑याः॒ पत्नि॑या॒ अथो॒ तत् तदथो॒ पत्नि॑याः । \newline
16. अथो॒ पत्नि॑याः॒ पत्नि॑या॒ अथो॒ अथो॒ पत्नि॑या ए॒वैव पत्नि॑या॒ अथो॒ अथो॒ पत्नि॑या ए॒व । \newline
17. अथो॒ इत्यथो᳚ । \newline
18. पत्नि॑या ए॒वैव पत्नि॑याः॒ पत्नि॑या ए॒वैष ए॒ष ए॒व पत्नि॑याः॒ पत्नि॑या ए॒वैषः । \newline
19. ए॒वैष ए॒ष ए॒वै वैष य॒ज्ञ्स्य॑ य॒ज्ञ्स्यै॒ष ए॒वै वैष य॒ज्ञ्स्य॑ । \newline
20. ए॒ष य॒ज्ञ्स्य॑ य॒ज्ञ्स्यै॒ष ए॒ष य॒ज्ञ्स्या᳚ न्वारं॒भो᳚ ऽन्वारं॒भो य॒ज्ञ्स्यै॒ष ए॒ष य॒ज्ञ्स्या᳚ न्वारं॒भः । \newline
21. य॒ज्ञ्स्या᳚ न्वारं॒भो᳚ ऽन्वारं॒भो य॒ज्ञ्स्य॑ य॒ज्ञ्स्या᳚ न्वारं॒भो ऽन॑वच्छित्त्या॒ अन॑वच्छित्त्या अन्वारं॒भो य॒ज्ञ्स्य॑ य॒ज्ञ्स्या᳚ न्वारं॒भो ऽन॑वच्छित्त्यै । \newline
22. अ॒न्वा॒रं॒भो ऽन॑वच्छित्त्या॒ अन॑वच्छित्त्या अन्वारं॒भो᳚ ऽन्वारं॒भो ऽन॑वच्छित्त्यै॒ वर्त्म॑ना॒ वर्त्म॒ना ऽन॑वच्छित्त्या अन्वारं॒भो᳚ ऽन्वारं॒भो ऽन॑वच्छित्त्यै॒ वर्त्म॑ना । \newline
23. अ॒न्वा॒रं॒भ इत्य॑नु - आ॒रं॒भः । \newline
24. अन॑वच्छित्त्यै॒ वर्त्म॑ना॒ वर्त्म॒ना ऽन॑वच्छित्त्या॒ अन॑वच्छित्त्यै॒ वर्त्म॑ना॒ वै वै वर्त्म॒ना ऽन॑वच्छित्त्या॒ अन॑वच्छित्त्यै॒ वर्त्म॑ना॒ वै । \newline
25. अन॑वच्छित्त्या॒ इत्यन॑व - छि॒त्त्यै॒ । \newline
26. वर्त्म॑ना॒ वै वै वर्त्म॑ना॒ वर्त्म॑ना॒ वा अ॒न्वित्या॒ न्वित्य॒ वै वर्त्म॑ना॒ वर्त्म॑ना॒ वा अ॒न्वित्य॑ । \newline
27. वा अ॒न्वित्या॒ न्वित्य॒ वै वा अ॒न्वित्य॑ य॒ज्ञ्ं ॅय॒ज्ञ् म॒न्वित्य॒ वै वा अ॒न्वित्य॑ य॒ज्ञ्म् । \newline
28. अ॒न्वित्य॑ य॒ज्ञ्ं ॅय॒ज्ञ् म॒न्वित्या॒ न्वित्य॑ य॒ज्ञ्ꣳ रक्षाꣳ॑सि॒ रक्षाꣳ॑सि य॒ज्ञ् म॒न्वित्या॒ न्वित्य॑ य॒ज्ञ्ꣳ रक्षाꣳ॑सि । \newline
29. अ॒न्वित्येत्य॑नु - इत्य॑ । \newline
30. य॒ज्ञ्ꣳ रक्षाꣳ॑सि॒ रक्षाꣳ॑सि य॒ज्ञ्ं ॅय॒ज्ञ्ꣳ रक्षाꣳ॑सि जिघाꣳसन्ति जिघाꣳसन्ति॒ रक्षाꣳ॑सि य॒ज्ञ्ं ॅय॒ज्ञ्ꣳ रक्षाꣳ॑सि जिघाꣳसन्ति । \newline
31. रक्षाꣳ॑सि जिघाꣳसन्ति जिघाꣳसन्ति॒ रक्षाꣳ॑सि॒ रक्षाꣳ॑सि जिघाꣳसन्ति वैष्ण॒वीभ्यां᳚ ॅवैष्ण॒वीभ्या᳚म् जिघाꣳसन्ति॒ रक्षाꣳ॑सि॒ रक्षाꣳ॑सि जिघाꣳसन्ति वैष्ण॒वीभ्या᳚म् । \newline
32. जि॒घाꣳ॒॒स॒न्ति॒ वै॒ष्ण॒वीभ्यां᳚ ॅवैष्ण॒वीभ्या᳚म् जिघाꣳसन्ति जिघाꣳसन्ति वैष्ण॒वीभ्या॑ मृ॒ग्भ्या मृ॒ग्भ्यां ॅवै᳚ष्ण॒वीभ्या᳚म् जिघाꣳसन्ति जिघाꣳसन्ति वैष्ण॒वीभ्या॑ मृ॒ग्भ्याम् । \newline
33. वै॒ष्ण॒वीभ्या॑ मृ॒ग्भ्या मृ॒ग्भ्यां ॅवै᳚ष्ण॒वीभ्यां᳚ ॅवैष्ण॒वीभ्या॑ मृ॒ग्भ्यां ॅवर्त्म॑नो॒र् वर्त्म॑नोर्. ऋ॒ग्भ्यां ॅवै᳚ष्ण॒वीभ्यां᳚ ॅवैष्ण॒वीभ्या॑ मृ॒ग्भ्यां ॅवर्त्म॑नोः । \newline
34. ऋ॒ग्भ्यां ॅवर्त्म॑नो॒र् वर्त्म॑नोर्. ऋ॒ग्भ्या मृ॒ग्भ्यां ॅवर्त्म॑नोर् जुहोति जुहोति॒ वर्त्म॑नोर्. ऋ॒ग्भ्या मृ॒ग्भ्यां ॅवर्त्म॑नोर् जुहोति । \newline
35. ऋ॒ग्भ्यामित्यृ॑क् - भ्याम् । \newline
36. वर्त्म॑नोर् जुहोति जुहोति॒ वर्त्म॑नो॒र् वर्त्म॑नोर् जुहोति य॒ज्ञो य॒ज्ञो जु॑होति॒ वर्त्म॑नो॒र् वर्त्म॑नोर् जुहोति य॒ज्ञ्ः । \newline
37. जु॒हो॒ति॒ य॒ज्ञो य॒ज्ञो जु॑होति जुहोति य॒ज्ञो वै वै य॒ज्ञो जु॑होति जुहोति य॒ज्ञो वै । \newline
38. य॒ज्ञो वै वै य॒ज्ञो य॒ज्ञो वै विष्णु॒र् विष्णु॒र् वै य॒ज्ञो य॒ज्ञो वै विष्णुः॑ । \newline
39. वै विष्णु॒र् विष्णु॒र् वै वै विष्णु॑र् य॒ज्ञाद् य॒ज्ञाद् विष्णु॒र् वै वै विष्णु॑र् य॒ज्ञात् । \newline
40. विष्णु॑र् य॒ज्ञाद् य॒ज्ञाद् विष्णु॒र् विष्णु॑र् य॒ज्ञा दे॒वैव य॒ज्ञाद् विष्णु॒र् विष्णु॑र् य॒ज्ञा दे॒व । \newline
41. य॒ज्ञा दे॒वैव य॒ज्ञाद् य॒ज्ञा दे॒व रक्षाꣳ॑सि॒ रक्षाꣳ॑ स्ये॒व य॒ज्ञाद् य॒ज्ञादे॒व रक्षाꣳ॑सि । \newline
42. ए॒व रक्षाꣳ॑सि॒ रक्षाꣳ॑ स्ये॒वैव रक्षाꣳ॒॒ स्यपाप॒ रक्षाꣳ॑ स्ये॒वैव रक्षाꣳ॒॒ स्यप॑ । \newline
43. रक्षाꣳ॒॒ स्यपाप॒ रक्षाꣳ॑सि॒ रक्षाꣳ॒॒ स्यप॑ हन्ति ह॒न्त्यप॒ रक्षाꣳ॑सि॒ रक्षाꣳ॒॒ स्यप॑ हन्ति । \newline
44. अप॑ हन्ति ह॒न्त्य पाप॑ हन्ति॒ यद् य द्ध॒न्त्य पाप॑ हन्ति॒ यत् । \newline
45. ह॒न्ति॒ यद् यद्ध॑न्ति हन्ति॒ यद॑द्ध्व॒र्यु र॑द्ध्व॒र्युर् यद्ध॑न्ति हन्ति॒ यद॑द्ध्व॒र्युः । \newline
46. यद॑द्ध्व॒र्यु र॑द्ध्व॒र्युर् यद् यद॑द्ध्व॒र्यु र॑न॒ग्ना व॑न॒ग्ना व॑द्ध्व॒र्युर् यद् यद॑द्ध्व॒र्यु र॑न॒ग्नौ । \newline
47. अ॒द्ध्व॒र्यु र॑न॒ग्ना व॑न॒ग्ना व॑द्ध्व॒र्यु र॑द्ध्व॒र्यु र॑न॒ग्ना वाहु॑ति॒ माहु॑ति मन॒ग्ना व॑द्ध्व॒र्यु र॑द्ध्व॒र्यु र॑न॒ग्ना वाहु॑तिम् । \newline
48. अ॒न॒ग्ना वाहु॑ति॒ माहु॑ति मन॒ग्ना व॑न॒ग्ना वाहु॑तिम् जुहु॒याज् जु॑हु॒या दाहु॑ति मन॒ग्ना व॑न॒ग्ना वाहु॑तिम् जुहु॒यात् । \newline
49. आहु॑तिम् जुहु॒याज् जु॑हु॒या दाहु॑ति॒ माहु॑तिम् जुहु॒या द॒न्धो᳚ ऽन्धो जु॑हु॒या दाहु॑ति॒ माहु॑तिम् जुहु॒या द॒न्धः । \newline
50. आहु॑ति॒मित्या - हु॒ति॒म् । \newline
51. जु॒हु॒या द॒न्धो᳚ ऽन्धो जु॑हु॒याज् जु॑हु॒या द॒न्धो᳚ ऽद्ध्व॒र्यु र॑द्ध्व॒र्यु र॒न्धो जु॑हु॒याज् जु॑हु॒या द॒न्धो᳚ ऽद्ध्व॒र्युः । \newline
52. अ॒न्धो᳚ ऽद्ध्व॒र्यु र॑द्ध्व॒र्यु र॒न्धो᳚(1॒) ऽन्धो᳚ ऽद्ध्व॒र्युः स्या᳚थ् स्या दद्ध्व॒र्यु र॒न्धो᳚(1॒) ऽन्धो᳚ ऽद्ध्व॒र्युः स्या᳚त् । \newline
53. अ॒द्ध्व॒र्युः स्या᳚थ् स्या दद्ध्व॒र्यु र॑द्ध्व॒र्युः स्या॒द् रक्षाꣳ॑सि॒ रक्षाꣳ॑सि स्या दद्ध्व॒र्यु र॑द्ध्व॒र्युः स्या॒द् रक्षाꣳ॑सि । \newline
54. स्या॒द् रक्षाꣳ॑सि॒ रक्षाꣳ॑सि स्याथ् स्या॒द् रक्षाꣳ॑सि य॒ज्ञ्ं ॅय॒ज्ञ्ꣳ रक्षाꣳ॑सि स्याथ् स्या॒द् रक्षाꣳ॑सि य॒ज्ञ्म् । \newline
55. रक्षाꣳ॑सि य॒ज्ञ्ं ॅय॒ज्ञ्ꣳ रक्षाꣳ॑सि॒ रक्षाꣳ॑सि य॒ज्ञ्ꣳ ह॑न्युर्. हन्युर् य॒ज्ञ्ꣳ रक्षाꣳ॑सि॒ रक्षाꣳ॑सि य॒ज्ञ्ꣳ ह॑न्युः । \newline
56. य॒ज्ञ्ꣳ ह॑न्युर्. हन्युर् य॒ज्ञ्ं ॅय॒ज्ञ्ꣳ ह॑न्यु॒र्॒. हिर॑ण्यꣳ॒॒ हिर॑ण्यꣳ हन्युर् य॒ज्ञ्ं ॅय॒ज्ञ्ꣳ ह॑न्यु॒र्॒. हिर॑ण्यम् । \newline
57. ह॒न्यु॒र्॒. हिर॑ण्यꣳ॒॒ हिर॑ण्यꣳ हन्युर्. हन्यु॒र्॒. हिर॑ण्य मु॒पास्यो॒पास्य॒ हिर॑ण्यꣳ हन्युर्. हन्यु॒र्॒. हिर॑ण्य मु॒पास्य॑ । \newline
\pagebreak
\markright{ TS 6.2.9.3  \hfill https://www.vedavms.in \hfill}

\section{ TS 6.2.9.3 }

\textbf{TS 6.2.9.3 } \newline
\textbf{Samhita Paata} \newline

-हिर॑ण्यमु॒पास्य॑ जुहोत्यग्नि॒वत्ये॒व जु॑होति॒ नान्धो᳚ऽद्ध्व॒र्युर्भव॑ति॒ न य॒ज्ञ्ꣳ रक्षाꣳ॑सि घ्नन्ति॒ प्राची॒ प्रेत॑मद्ध्व॒रं क॒ल्प॑यन्ती॒ इत्या॑ह सुव॒र्गमे॒वैने॑ लो॒कं ग॑मय॒त्यत्र॑ रमेथां॒ ॅवर्ष्म॑न् पृथि॒व्या इत्या॑ह॒ वर्ष्म॒ ह्ये॑तत् पृ॑थि॒व्या यद् दे॑व॒यज॑नꣳ॒॒ शिरो॒ वा ए॒तद्-य॒ज्ञ्स्य॒ यद्ध॑वि॒र्द्धानं॑ दि॒वो वा॑ विष्णवु॒त वा॑ पृथि॒व्या- [  ] \newline

\textbf{Pada Paata} \newline

हिर॑ण्यम् । उ॒पास्येत्यु॑प - अस्य॑ । जु॒हो॒ति॒ । अ॒ग्नि॒वतीत्य॑ग्नि-वति॑ । ए॒व । जु॒हो॒ति॒ । न । अ॒न्धः । अ॒द्ध्व॒र्युः । भव॑ति । न । य॒ज्ञ्म् । रक्षाꣳ॑सि । घ्न॒न्ति॒ । प्राची॒ इति॑ । प्रेति॑ । इ॒त॒म् । अ॒द्ध्व॒रम् । क॒ल्प॑यन्ती॒ इति॑ । इति॑ । आ॒ह॒ । सु॒व॒र्गमिति॑ सुवः - गम् । ए॒व । ए॒ने॒ इति॑ । लो॒कम् । ग॒म॒य॒ति॒ । अत्र॑ । र॒मे॒था॒म् । वर्ष्मन्न्॑ । पृ॒थि॒व्याः । इति॑ । आ॒ह॒ । वर्ष्म॑ । हि । ए॒तत् । पृ॒थि॒व्याः । यत् । दे॒व॒यज॑न॒मिति॑ देव - यज॑नम् । शिरः॑ । वै । ए॒तत् । य॒ज्ञ्स्य॑ । यत् । ह॒वि॒द्‌र्धान॒मिति॑ हविः-धान᳚म् । दि॒वः । वा॒ । वि॒ष्णो॒ । उ॒त । वा॒ । पृ॒थि॒व्याः ।  \newline


\textbf{Krama Paata} \newline

हिर॑ण्यमु॒पास्य॑ । उ॒पास्य॑ जुहोति । उ॒पासेत्यु॑प - अस्य॑ । जु॒हो॒त्य॒ग्नि॒वति॑ । अ॒ग्नि॒वत्ये॒व । अ॒ग्नि॒वतीत्य॑ग्नि - वति॑ । ए॒व जु॑होति । जु॒हो॒ति॒ न । नान्धः । अ॒न्धो᳚ऽद्ध्व॒र्युः । अ॒द्ध्व॒र्युर् भव॑ति । भव॑ति॒ न । न य॒ज्ञ्म् । य॒ज्ञ्ꣳ रक्षाꣳ॑सि । रक्षाꣳ॑सि घ्नन्ति । घ्न॒न्ति॒ प्राची᳚ । प्राची॒ प्र । प्राची॒ इति॒ प्राची᳚ । प्रेत᳚म् । इ॒त॒म॒द्ध्व॒रम् । अ॒द्ध्व॒रम् क॒ल्पय॑न्ती । क॒ल्पय॑न्ती॒ इति॑ । क॒ल्पय॑न्ती॒ इति॑ क॒ल्पय॑न्ती । इत्या॑ह । आ॒ह॒ सु॒व॒र्गम् । सु॒व॒र्गमे॒व । सु॒व॒र्गमिति॑ सुवः - गम् । ए॒वैने᳚ । ए॒ने॒ लो॒कम् । ए॒ने॒ इत्ये॑ने । लो॒कम् ग॑मयति । ग॒म॒य॒त्यत्र॑ । अत्र॑ रमेथाम् । र॒मे॒था॒म् ॅवर्ष्मन्न्॑ । वर्ष्म॑न् पृथि॒व्याः । पृ॒थि॒व्या इति॑ । इत्या॑ह । आ॒ह॒ वर्ष्म॑ । वर्ष्म॒ हि । ह्ये॑तत् । ए॒तत् पृ॑थि॒व्याः । पृ॒थि॒व्या यत् । यद् दे॑व॒यज॑नम् । दे॒व॒यज॑नꣳ॒॒ शिरः॑ । दे॒व॒यज॑न॒मिति॑ देव - यज॑नम् । शिरो॒ वै । वा ए॒तत् । ए॒तद् य॒ज्ञ्स्य॑ । य॒ज्ञ्स्य॒ यत् । यद्‍ध॑वि॒र्धान᳚म् । ह॒वि॒र्द्धान॑म् दि॒वः । ह॒वि॒र्द्धान॒मिति॑ हविः - धान᳚म् । दि॒वो वा᳚ । वा॒ वि॒ष्णो॒ । वि॒ष्ण॒वु॒त । उ॒त वा᳚ । वा॒ पृ॒थि॒व्याः । पृ॒थि॒व्या इति॑ \newline

\textbf{Jatai Paata} \newline

1. हिर॑ण्य मु॒पा स्यो॒पास्य॒ हिर॑ण्यꣳ॒॒ हिर॑ण्य मु॒पास्य॑ । \newline
2. उ॒पास्य॑ जुहोति जुहो त्यु॒पा स्यो॒पास्य॑ जुहोति । \newline
3. उ॒पास्येत्यु॑प - अस्य॑ । \newline
4. जु॒हो॒ त्य॒ग्नि॒व त्य॑ग्नि॒वति॑ जुहोति जुहो त्यग्नि॒वति॑ । \newline
5. अ॒ग्नि॒व त्ये॒वैवा ग्नि॒व त्य॑ग्नि॒व त्ये॒व । \newline
6. अ॒ग्नि॒वतीत्य॑ग्नि - वति॑ । \newline
7. ए॒व जु॑होति जुहो त्ये॒वैव जु॑होति । \newline
8. जु॒हो॒ति॒ न न जु॑होति जुहोति॒ न । \newline
9. नान्धो᳚ ऽन्धो न नान्धः । \newline
10. अ॒न्धो᳚ ऽद्ध्व॒र्यु र॑द्ध्व॒र्यु र॒न्धो᳚(1॒) ऽन्धो᳚ ऽद्ध्व॒र्युः । \newline
11. अ॒द्ध्व॒र्युर् भव॑ति॒ भव॑ त्यद्ध्व॒र्यु र॑द्ध्व॒र्युर् भव॑ति । \newline
12. भव॑ति॒ न न भव॑ति॒ भव॑ति॒ न । \newline
13. न य॒ज्ञ्ं ॅय॒ज्ञ्न्न न य॒ज्ञ्म् । \newline
14. य॒ज्ञ्ꣳ रक्षाꣳ॑सि॒ रक्षाꣳ॑सि य॒ज्ञ्ं ॅय॒ज्ञ्ꣳ रक्षाꣳ॑सि । \newline
15. रक्षाꣳ॑सि घ्नन्ति घ्नन्ति॒ रक्षाꣳ॑सि॒ रक्षाꣳ॑सि घ्नन्ति । \newline
16. घ्न॒न्ति॒ प्राची॒ प्राची᳚ घ्नन्ति घ्नन्ति॒ प्राची᳚ । \newline
17. प्राची॒ प्र प्र प्राची॒ प्राची॒ प्र । \newline
18. प्राची॒ इति॒ प्राची᳚ । \newline
19. प्रेत॑ मित॒म् प्र प्रेत᳚म् । \newline
20. इ॒त॒ म॒द्ध्व॒र म॑द्ध्व॒र मि॑त मित मद्ध्व॒रम् । \newline
21. अ॒द्ध्व॒रम् क॒ल्पय॑न्ती क॒ल्पय॑न्ती अद्ध्व॒र म॑द्ध्व॒रम् क॒ल्पय॑न्ती । \newline
22. क॒ल्पय॑न्ती॒ इतीति॑ क॒ल्पय॑न्ती क॒ल्पय॑न्ती॒ इति॑ । \newline
23. क॒ल्पय॑न्ती॒ इति॑ क॒ल्पय॑न्ती । \newline
24. इत्या॑हा॒हे तीत्या॑ह । \newline
25. आ॒ह॒ सु॒व॒र्गꣳ सु॑व॒र्ग मा॑हाह सुव॒र्गम् । \newline
26. सु॒व॒र्ग मे॒वैव सु॑व॒र्गꣳ सु॑व॒र्ग मे॒व । \newline
27. सु॒व॒र्गमिति॑ सुवः - गम् । \newline
28. ए॒वैने॑ एने ए॒वैवैने᳚ । \newline
29. ए॒ने॒ लो॒कम् ॅलो॒क मे॑ने एने लो॒कम् । \newline
30. ए॒ने॒ इत्ये॑ने । \newline
31. लो॒कम् ग॑मयति गमयति लो॒कम् ॅलो॒कम् ग॑मयति । \newline
32. ग॒म॒य॒ त्यत्रात्र॑ गमयति गमय॒ त्यत्र॑ । \newline
33. अत्र॑ रमेथाꣳ रमेथा॒ मत्रात्र॑ रमेथाम् । \newline
34. र॒मे॒थां॒ ॅवर्ष्म॒न्॒. वर्ष्म॑न् रमेथाꣳ रमेथां॒ ॅवर्ष्मन्न्॑ । \newline
35. वर्ष्म॑न् पृथि॒व्याः पृ॑थि॒व्या वर्ष्म॒न्॒. वर्ष्म॑न् पृथि॒व्याः । \newline
36. पृ॒थि॒व्या इतीति॑ पृथि॒व्याः पृ॑थि॒व्या इति॑ । \newline
37. इत्या॑हा॒हे तीत्या॑ह । \newline
38. आ॒ह॒ वर्ष्म॒ वर्ष्मा॑हाह॒ वर्ष्म॑ । \newline
39. वर्ष्म॒ हि हि वर्ष्म॒ वर्ष्म॒ हि । \newline
40. ह्ये॑त दे॒त द्धि ह्ये॑तत् । \newline
41. ए॒तत् पृ॑थि॒व्याः पृ॑थि॒व्या ए॒त दे॒तत् पृ॑थि॒व्याः । \newline
42. पृ॒थि॒व्या यद् यत् पृ॑थि॒व्याः पृ॑थि॒व्या यत् । \newline
43. यद् दे॑व॒यज॑नम् देव॒यज॑नं॒ ॅयद् यद् दे॑व॒यज॑नम् । \newline
44. दे॒व॒यज॑नꣳ॒॒ शिरः॒ शिरो॑ देव॒यज॑नम् देव॒यज॑नꣳ॒॒ शिरः॑ । \newline
45. दे॒व॒यज॑न॒मिति॑ देव - यज॑नम् । \newline
46. शिरो॒ वै वै शिरः॒ शिरो॒ वै । \newline
47. वा ए॒त दे॒तद् वै वा ए॒तत् । \newline
48. ए॒तद् य॒ज्ञ्स्य॑ य॒ज्ञ् स्यै॒त दे॒तद् य॒ज्ञ्स्य॑ । \newline
49. य॒ज्ञ्स्य॒ यद् यद् य॒ज्ञ्स्य॑ य॒ज्ञ्स्य॒ यत् । \newline
50. यद्ध॑वि॒र्द्धानꣳ॑ हवि॒र्द्धानं॒ ॅयद् यद्ध॑वि॒र्द्धान᳚म् । \newline
51. ह॒वि॒र्द्धान॑म् दि॒वो दि॒वो ह॑वि॒र्द्धानꣳ॑ हवि॒र्द्धान॑म् दि॒वः । \newline
52. ह॒वि॒र्द्धान॒मिति॑ हविः - धान᳚म् । \newline
53. दि॒वो वा॑ वा दि॒वो दि॒वो वा᳚ । \newline
54. वा॒ वि॒ष्णो॒ वि॒ष्णो॒ वा॒ वा॒ वि॒ष्णो॒ । \newline
55. वि॒ष्ण॒ वु॒तोत वि॑ष्णो विष्ण वु॒त । \newline
56. उ॒त वा॑ वो॒तोत वा᳚ । \newline
57. वा॒ पृ॒थि॒व्याः पृ॑थि॒व्या वा॑ वा पृथि॒व्याः । \newline
58. पृ॒थि॒व्या इतीति॑ पृथि॒व्याः पृ॑थि॒व्या इति॑ । \newline

\textbf{Ghana Paata } \newline

1. हिर॑ण्य मु॒पास्यो॒पास्य॒ हिर॑ण्यꣳ॒॒ हिर॑ण्य मु॒पास्य॑ जुहोति जुहो त्यु॒पास्य॒ हिर॑ण्यꣳ॒॒ हिर॑ण्य मु॒पास्य॑ जुहोति । \newline
2. उ॒पास्य॑ जुहोति जुहो त्यु॒पा स्यो॒पास्य॑ जुहो त्यग्नि॒व त्य॑ग्नि॒वति॑ जुहो त्यु॒पा स्यो॒पास्य॑ जुहो त्यग्नि॒वति॑ । \newline
3. उ॒पास्येत्यु॑प - अस्य॑ । \newline
4. जु॒हो॒ त्य॒ग्नि॒व त्य॑ग्नि॒वति॑ जुहोति जुहो त्यग्नि॒व त्ये॒वै वाग्नि॒वति॑ जुहोति जुहो त्यग्नि॒ वत्ये॒व । \newline
5. अ॒ग्नि॒व त्ये॒वै वाग्नि॒व त्य॑ग्नि॒व त्ये॒व जु॑होति जुहो त्ये॒वाग्नि॒व त्य॑ग्नि॒व त्ये॒व जु॑होति । \newline
6. अ॒ग्नि॒वतीत्य॑ग्नि - वति॑ । \newline
7. ए॒व जु॑होति जुहो त्ये॒वैव जु॑होति॒ न न जु॑हो त्ये॒वैव जु॑होति॒ न । \newline
8. जु॒हो॒ति॒ न न जु॑होति जुहोति॒ नान्धो᳚ ऽन्धो न जु॑होति जुहोति॒ नान्धः । \newline
9. नान्धो᳚ ऽन्धो न नान्धो᳚ ऽद्ध्व॒र्यु र॑द्ध्व॒र्यु र॒न्धो न नान्धो᳚ ऽद्ध्व॒र्युः । \newline
10. अ॒न्धो᳚ ऽद्ध्व॒र्यु र॑द्ध्व॒र्यु र॒न्धो᳚(1॒) ऽन्धो᳚ ऽद्ध्व॒र्युर् भव॑ति॒ भव॑ त्यद्ध्व॒र्यु र॒न्धो᳚(1॒) ऽन्धो᳚ ऽद्ध्व॒र्युर् भव॑ति । \newline
11. अ॒द्ध्व॒र्युर् भव॑ति॒ भव॑ त्यद्ध्व॒र्यु र॑द्ध्व॒र्युर् भव॑ति॒ न न भव॑ त्यद्ध्व॒र्यु र॑द्ध्व॒र्युर् भव॑ति॒ न । \newline
12. भव॑ति॒ न न भव॑ति॒ भव॑ति॒ न य॒ज्ञ्ं ॅय॒ज्ञ्न् न भव॑ति॒ भव॑ति॒ न य॒ज्ञ्म् । \newline
13. न य॒ज्ञ्ं ॅय॒ज्ञ्न् न न य॒ज्ञ्ꣳ रक्षाꣳ॑सि॒ रक्षाꣳ॑सि य॒ज्ञ्न् न न य॒ज्ञ्ꣳ रक्षाꣳ॑सि । \newline
14. य॒ज्ञ्ꣳ रक्षाꣳ॑सि॒ रक्षाꣳ॑सि य॒ज्ञ्ं ॅय॒ज्ञ्ꣳ रक्षाꣳ॑सि घ्नन्ति घ्नन्ति॒ रक्षाꣳ॑सि य॒ज्ञ्ं ॅय॒ज्ञ्ꣳ रक्षाꣳ॑सि घ्नन्ति । \newline
15. रक्षाꣳ॑सि घ्नन्ति घ्नन्ति॒ रक्षाꣳ॑सि॒ रक्षाꣳ॑सि घ्नन्ति॒ प्राची॒ प्राची᳚ घ्नन्ति॒ रक्षाꣳ॑सि॒ रक्षाꣳ॑सि घ्नन्ति॒ प्राची᳚ । \newline
16. घ्न॒न्ति॒ प्राची॒ प्राची᳚ घ्नन्ति घ्नन्ति॒ प्राची॒ प्र प्र प्राची᳚ घ्नन्ति घ्नन्ति॒ प्राची॒ प्र । \newline
17. प्राची॒ प्र प्र प्राची॒ प्राची॒ प्रेत॑ मित॒म् प्र प्राची॒ प्राची॒ प्रेत᳚म् । \newline
18. प्राची॒ इति॒ प्राची᳚ । \newline
19. प्रेत॑ मित॒म् प्र प्रेत॑ मद्ध्व॒र म॑द्ध्व॒र मि॑त॒म् प्र प्रेत॑ मद्ध्व॒रम् । \newline
20. इ॒त॒ म॒द्ध्व॒र म॑द्ध्व॒र मि॑त मित मद्ध्व॒रम् क॒ल्पय॑न्ती क॒ल्पय॑न्ती अद्ध्व॒र मि॑त मित मद्ध्व॒रम् क॒ल्पय॑न्ती । \newline
21. अ॒द्ध्व॒रम् क॒ल्पय॑न्ती क॒ल्पय॑न्ती अद्ध्व॒र म॑द्ध्व॒रम् क॒ल्पय॑न्ती॒ इतीति॑ क॒ल्पय॑न्ती अद्ध्व॒र म॑द्ध्व॒रम् क॒ल्पय॑न्ती॒ इति॑ । \newline
22. क॒ल्पय॑न्ती॒ इतीति॑ क॒ल्पय॑न्ती क॒ल्पय॑न्ती॒ इत्या॑हा॒हेति॑ क॒ल्पय॑न्ती क॒ल्पय॑न्ती॒ इत्या॑ह । \newline
23. क॒ल्पय॑न्ती॒ इति॑ क॒ल्पय॑न्ती । \newline
24. इत्या॑हा॒हे तीत्या॑ह सुव॒र्गꣳ सु॑व॒र्ग मा॒हे तीत्या॑ह सुव॒र्गम् । \newline
25. आ॒ह॒ सु॒व॒र्गꣳ सु॑व॒र्ग मा॑हाह सुव॒र्ग मे॒वैव सु॑व॒र्ग मा॑हाह सुव॒र्ग मे॒व । \newline
26. सु॒व॒र्ग मे॒वैव सु॑व॒र्गꣳ सु॑व॒र्ग मे॒वैने॑ एने ए॒व सु॑व॒र्गꣳ सु॑व॒र्ग मे॒वैने᳚ । \newline
27. सु॒व॒र्गमिति॑ सुवः - गम् । \newline
28. ए॒वैने॑ एने ए॒वै वैने॑ लो॒कम् ॅलो॒क मे॑ने ए॒वै वैने॑ लो॒कम् । \newline
29. ए॒ने॒ लो॒कम् ॅलो॒क मे॑ने एने लो॒कम् ग॑मयति गमयति लो॒क मे॑ने एने लो॒कम् ग॑मयति । \newline
30. ए॒ने॒ इत्ये॑ने । \newline
31. लो॒कम् ग॑मयति गमयति लो॒कम् ॅलो॒कम् ग॑मय॒ त्यत्रात्र॑ गमयति लो॒कम् ॅलो॒कम् ग॑मय॒ त्यत्र॑ । \newline
32. ग॒म॒य॒ त्यत्रात्र॑ गमयति गमय॒ त्यत्र॑ रमेथाꣳ रमेथा॒ मत्र॑ गमयति गमय॒ त्यत्र॑ रमेथाम् । \newline
33. अत्र॑ रमेथाꣳ रमेथा॒ मत्रात्र॑ रमेथां॒ ॅवर्ष्म॒न्॒. वर्ष्म॑न् रमेथा॒ मत्रात्र॑ रमेथां॒ ॅवर्ष्मन्न्॑ । \newline
34. र॒मे॒थां॒ ॅवर्ष्म॒न्॒. वर्ष्म॑न् रमेथाꣳ रमेथां॒ ॅवर्ष्म॑न् पृथि॒व्याः पृ॑थि॒व्या वर्ष्म॑न् रमेथाꣳ रमेथां॒ ॅवर्ष्म॑न् पृथि॒व्याः । \newline
35. वर्ष्म॑न् पृथि॒व्याः पृ॑थि॒व्या वर्ष्म॒न्॒. वर्ष्म॑न् पृथि॒व्या इतीति॑ पृथि॒व्या वर्ष्म॒न्॒. वर्ष्म॑न् पृथि॒व्या इति॑ । \newline
36. पृ॒थि॒व्या इतीति॑ पृथि॒व्याः पृ॑थि॒व्या इत्या॑हा॒ हेति॑ पृथि॒व्याः पृ॑थि॒व्या इत्या॑ह । \newline
37. इत्या॑हा॒हे तीत्या॑ह॒ वर्ष्म॒ वर्ष्मा॒हे तीत्या॑ह॒ वर्ष्म॑ । \newline
38. आ॒ह॒ वर्ष्म॒ वर्ष्मा॑ हाह॒ वर्ष्म॒ हि हि वर्ष्मा॑ हाह॒ वर्ष्म॒ हि । \newline
39. वर्ष्म॒ हि हि वर्ष्म॒ वर्ष्म॒ ह्ये॑त दे॒त द्धि वर्ष्म॒ वर्ष्म॒ ह्ये॑तत् । \newline
40. ह्ये॑त दे॒त द्धि ह्ये॑तत् पृ॑थि॒व्याः पृ॑थि॒व्या ए॒त द्धि ह्ये॑तत् पृ॑थि॒व्याः । \newline
41. ए॒तत् पृ॑थि॒व्याः पृ॑थि॒व्या ए॒त दे॒तत् पृ॑थि॒व्या यद् यत् पृ॑थि॒व्या ए॒त दे॒तत् पृ॑थि॒व्या यत् । \newline
42. पृ॒थि॒व्या यद् यत् पृ॑थि॒व्याः पृ॑थि॒व्या यद् दे॑व॒यज॑नम् देव॒यज॑नं॒ ॅयत् पृ॑थि॒व्याः पृ॑थि॒व्या यद् दे॑व॒यज॑नम् । \newline
43. यद् दे॑व॒यज॑नम् देव॒यज॑नं॒ ॅयद् यद् दे॑व॒यज॑नꣳ॒॒ शिरः॒ शिरो॑ देव॒यज॑नं॒ ॅयद् यद् दे॑व॒यज॑नꣳ॒॒ शिरः॑ । \newline
44. दे॒व॒यज॑नꣳ॒॒ शिरः॒ शिरो॑ देव॒यज॑नम् देव॒यज॑नꣳ॒॒ शिरो॒ वै वै शिरो॑ देव॒यज॑नम् देव॒यज॑नꣳ॒॒ शिरो॒ वै । \newline
45. दे॒व॒यज॑न॒मिति॑ देव - यज॑नम् । \newline
46. शिरो॒ वै वै शिरः॒ शिरो॒ वा ए॒त दे॒तद् वै शिरः॒ शिरो॒ वा ए॒तत् । \newline
47. वा ए॒त दे॒तद् वै वा ए॒तद् य॒ज्ञ्स्य॑ य॒ज्ञ् स्यै॒तद् वै वा ए॒तद् य॒ज्ञ्स्य॑ । \newline
48. ए॒तद् य॒ज्ञ्स्य॑ य॒ज्ञ् स्यै॒त दे॒तद् य॒ज्ञ्स्य॒ यद् यद् य॒ज्ञ् स्यै॒त दे॒तद् य॒ज्ञ्स्य॒ यत् । \newline
49. य॒ज्ञ्स्य॒ यद् यद् य॒ज्ञ्स्य॑ य॒ज्ञ्स्य॒ यद्ध॑वि॒र्द्धानꣳ॑ हवि॒र्द्धानं॒ ॅयद् य॒ज्ञ्स्य॑ य॒ज्ञ्स्य॒ यद्ध॑वि॒र्द्धान᳚म् । \newline
50. यद्ध॑वि॒र्द्धानꣳ॑ हवि॒र्द्धानं॒ ॅयद् यद्ध॑वि॒र्द्धान॑म् दि॒वो दि॒वो ह॑वि॒र्द्धानं॒ ॅयद् 
यद्ध॑वि॒र्द्धान॑म् दि॒वः । \newline
51. ह॒वि॒र्द्धान॑म् दि॒वो दि॒वो ह॑वि॒र्द्धानꣳ॑ हवि॒र्द्धान॑म् दि॒वो वा॑ वा दि॒वो ह॑वि॒र्द्धानꣳ॑ हवि॒र्द्धान॑म् दि॒वो वा᳚ । \newline
52. ह॒वि॒र्द्धान॒मिति॑ हविः - धान᳚म् । \newline
53. दि॒वो वा॑ वा दि॒वो दि॒वो वा॑ विष्णो विष्णो वा दि॒वो दि॒वो वा॑ विष्णो । \newline
54. वा॒ वि॒ष्णो॒ वि॒ष्णो॒ वा॒ वा॒ वि॒ष्ण॒ वु॒तोत वि॑ष्णो वा वा विष्णवु॒त । \newline
55. वि॒ष्ण॒ वु॒तोत वि॑ष्णो विष्णवु॒त वा॑ वो॒त वि॑ष्णो विष्णवु॒त वा᳚ । \newline
56. उ॒त वा॑ वो॒तोत वा॑ पृथि॒व्याः पृ॑थि॒व्या वो॒तोत वा॑ पृथि॒व्याः । \newline
57. वा॒ पृ॒थि॒व्याः पृ॑थि॒व्या वा॑ वा पृथि॒व्या इतीति॑ पृथि॒व्या वा॑ वा पृथि॒व्या इति॑ । \newline
58. पृ॒थि॒व्या इतीति॑ पृथि॒व्याः पृ॑थि॒व्या इत्या॒ शीर्प॑दया॒ ऽऽशीर्प॑द॒येति॑ पृथि॒व्याः पृ॑थि॒व्या इत्या॒ शीर्प॑दया । \newline
\pagebreak
\markright{ TS 6.2.9.4  \hfill https://www.vedavms.in \hfill}

\section{ TS 6.2.9.4 }

\textbf{TS 6.2.9.4 } \newline
\textbf{Samhita Paata} \newline

इत्या॒शीर्प॑दय॒र्चा दक्षि॑णस्य हवि॒र्द्धान॑स्य मे॒थीं नि ह॑न्ति शीर्.ष॒त ए॒व य॒ज्ञ्स्य॒ यज॑मान आ॒शिषोऽव॑ रुन्धे द॒ण्डो वा औ॑प॒रस्तृ॒तीय॑स्य हवि॒र्द्धान॑स्य वषट्का॒रे-णाक्ष॑-मच्छिन॒द्-यत्-तृ॒तीयं॑ छ॒दिर्. ह॑वि॒र्द्धान॑यो-रुदाह्रि॒यते॑ तृ॒तीय॑स्य हवि॒र्द्धान॒स्याव॑रुद्ध्यै॒ शिरो॒ वा ए॒तद्-य॒ज्ञ्स्य॒ यद्ध॑वि॒र्द्धानं॒ ॅविष्णो॑ र॒राट॑मसि॒ विष्णोः᳚ पृ॒ष्ठम॒सीत्या॑ह॒ तस्मा॑देताव॒द्धा शिरो॒ विष्यू॑तं॒ ॅविष्णोः॒ ( ) स्यूर॑सि॒ विष्णो᳚र्द्ध्रु॒वम॒सीत्या॑ह वैष्ण॒वꣳ हि दे॒वत॑या हवि॒र्द्धानं॒ ॅयं प्र॑थ॒मं ग्र॒न्थिं ग्र॑थ्नी॒याद्यत् तं न वि॑स्रꣳ॒॒ सये॒दमे॑हेनाद्ध्व॒र्युः प्रमी॑येत॒ तस्मा॒थ् स वि॒स्रस्यः॑ ॥ \newline

\textbf{Pada Paata} \newline

इति॑ । आ॒शीर्प॑द॒येत्या॒शीः - प॒द॒या॒ । ऋ॒चा । दक्षि॑णस्य । ह॒वि॒द्‌र्धान॒स्येति॑ हविः - धान॑स्य । मे॒थीम् । नीति॑ । ह॒न्ति॒ । शी॒र्॒.ष॒तः । ए॒व । य॒ज्ञ्स्य॑ । यज॑मानः । आ॒शिष॒ इत्या᳚ - शिषः॑ । अवेति॑ । रु॒न्धे॒ । द॒ण्डः । वै । औ॒प॒रः । तृ॒तीय॑स्य । ह॒वि॒द्‌र्धान॒स्येति॑ हविः - धान॑स्य । व॒ष॒ट्का॒रेणेति॑ वषट् - का॒रेण॑ । अक्ष᳚म् । अ॒च्छि॒न॒त् । यत् । तृ॒तीय᳚म् । छ॒दिः । ह॒वि॒द्‌र्धान॒योरिति॑ हविः-धान॑योः । उ॒दा॒ह्रि॒यत॒ इत्यु॑त् - आ॒ह्रि॒यते᳚ । तृ॒तीय॑स्य । ह॒वि॒द्‌र्धान॒स्येति॑ हविः - धान॑स्य । अव॑रुद्ध्या॒ इत्यव॑ - रु॒द्ध्यै॒ । शिरः॑ । वै । ए॒तत् । य॒ज्ञ्स्य॑ । यत् । ह॒वि॒द्‌र्धान॒मिति॑ हविः-धान᳚म् । विष्णोः᳚ । र॒राट᳚म् । अ॒सि॒ । विष्णोः᳚ । पृ॒ष्ठम् । अ॒सि॒ । इति॑ । आ॒ह॒ । तस्मा᳚त् । ए॒ता॒व॒द्धेत्ये॑तावत् - धा । शिरः॑ । विष्यू॑त॒मिति॒ वि - स्यू॒त॒म् । विष्णोः᳚ ( ) । स्यूः । अ॒सि॒ । विष्णोः᳚ । ध्रु॒वम् । अ॒सि॒ । इति॑ । आ॒ह॒ । वै॒ष्ण॒वम् । हि । दे॒वत॑या । ह॒वि॒द्‌र्धान॒मिति॑ हविः - धान᳚म् । यम् । प्र॒थ॒मम् । ग्र॒न्थिम् । ग्र॒थ्नी॒यात् । यत् । तम् । न । वि॒स्रꣳ॒॒सये॒दिति॑ वि - स्रꣳ॒॒सये᳚त् । अमे॑हेन । अ॒द्ध्व॒र्युः । प्रेति॑ । मी॒ये॒त॒ । तस्मा᳚त् । सः । वि॒स्रस्य॒ इति॑ वि-स्रस्यः॑ ॥  \newline


\textbf{Krama Paata} \newline

इत्या॒शीर्प॑दया । आ॒शीर्प॑दय॒र्चा । आ॒शीर्प॑द॒येत्या॒शीः - प॒द॒या॒ । ऋ॒चा दक्षि॑णस्य । दक्षि॑णस्य हवि॒र्द्धान॑स्य । ह॒वि॒र्द्धान॑स्य मे॒थीम् । ह॒वि॒र्द्धान॒स्येति॑ हविः - धान॑स्य । मे॒थीम् नि । नि ह॑न्ति । ह॒न्ति॒ शी॒र्॒.ष॒तः । शी॒र्॒.ष॒त ए॒व । ए॒व य॒ज्ञ्स्य॑ । य॒ज्ञ्स्य॒ यज॑मानः । यज॑मान आ॒शिषः॑ । आ॒शिषोऽव॑ । आ॒शिष॒ इत्या᳚ - शिषः॑ । अव॑ रुन्धे । रु॒न्धे॒ द॒ण्डः । द॒ण्डो वै । वा औ॑प॒रः । औ॒प॒रस्तृ॒तीय॑स्य । तृ॒तीय॑स्य हवि॒र्द्धान॑स्य । ह॒वि॒र्द्धान॑स्य वषट्का॒रेण॑ । ह॒वि॒र्द्धान॒स्येति॑ हविः - धान॑स्य । व॒ष॒ट्का॒रेणाक्ष᳚म् । व॒ष॒ट्का॒रेणेति॑ वषट् - का॒रेण॑ । अक्ष॑मच्छिनत् । अ॒च्छि॒न॒द् यत् । यत् तृ॒तीय᳚म् । तृ॒तीय॑म् छ॒दिः । छ॒दिर्. ह॑वि॒र्द्धा॑नयोः । ह॒वि॒र्द्धान॑योरुदाह्रि॒यते᳚ । ह॒वि॒र्द्धान॑यो॒रिति॑ हविः - धान॑योः । उ॒दा॒ह्रि॒यते॑ तृ॒तीय॑स्य । उ॒दा॒ह्रि॒यत॒ इत्यु॑त् - आ॒ह्रि॒यते᳚ । तृ॒तीय॑स्य हवि॒र्द्धान॑स्य । ह॒वि॒र्द्धान॒स्याव॑रुद्ध्यै । ह॒वि॒र्द्धान॒स्येति॑ हविः - धान॑स्य । अव॑रुद्ध्यै॒ शिरः॑ । अव॑रुद्ध्या॒ इत्यव॑ - रु॒द्ध्यै॒ । शिरो॒ वै । वा ए॒तत् । ए॒तद् य॒ज्ञ्स्य॑ । य॒ज्ञ्स्य॒ यत् । यद्‍ध॑वि॒र्द्धान᳚म् । ह॒वि॒र्द्धान॒म् ॅविष्णोः᳚ । ह॒वि॒र्द्धान॒मिति॑ हविः - धान᳚म् । विष्णो॑ र॒राट᳚म् । र॒राट॑मसि । अ॒सि॒ विष्णोः᳚ । विष्णोः᳚ पृ॒ष्ठम् । पृ॒ष्ठम॑सि । अ॒सीति॑ । इत्या॑ह । आ॒ह॒ तस्मा᳚त् । तस्मा॑देताव॒द्धा । ए॒ता॒व॒द्धा शिरः॑ । ए॒ता॒व॒द्धेत्ये॑तावत् - धा । शिरो॒ विष्यू॑तम् । विष्यू॑त॒म् ॅविष्णोः᳚ ( ) । विष्यू॑त॒मिति॒ वि - स्यू॒त॒म् । विष्णोः॒ स्यूः । स्यूर॑सि । अ॒सि॒ विष्णोः᳚ । विष्णो᳚र् ध्रु॒वम् । ध्रु॒वम॑सि । अ॒सीति॑ । इत्या॑ह । आ॒ह॒ वै॒ष्ण॒वम् । वै॒ष्ण॒वꣳ हि । हि दे॒वत॑या । दे॒वत॑या हवि॒र्द्धान᳚म् । ह॒वि॒र्द्धान॒म् ॅयम् । ह॒वि॒र्द्धान॒मिति॑ हविः - धान᳚म् । यम् प्र॑थ॒मम् । प्र॒थ॒मम् ग्र॒न्थिम् । ग्र॒न्थिम् ग्र॑थ्नी॒यात् । ग्र॒थ्नी॒याद् यत् । यत् तम् । तम् न । न वि॑स्रꣳ॒॒सये᳚त् । वि॒स्रꣳ॒॒सये॒दमे॑हेन । वि॒स्रꣳ॒॒सये॒दिति॑ वि - स्रꣳ॒॒सये᳚त् । अमे॑हेनाद्ध्व॒र्युः । अ॒द्ध्व॒र्युः प्र । प्र मी॑येत । मी॒ये॒त॒ तस्मा᳚त् । तस्मा॒थ् सः । स वि॒स्रस्यः॑ । वि॒स्रस्य॒ इति॑ वि - स्रस्यः॑ । \newline

\textbf{Jatai Paata} \newline

1. इत्या॒शीर्प॑दया॒ ऽऽशीर्प॑द॒येती त्या॒शीर्प॑दया । \newline
2. आ॒शीर्प॑दय॒ र्‌च र्‌चा ऽऽशीर्प॑दया॒ ऽऽशीर्प॑दय॒ र्‌चा । \newline
3. आ॒शीर्प॑द॒येत्या॒शीः - प॒द॒या॒ । \newline
4. ऋ॒चा दक्षि॑णस्य॒ दक्षि॑णस्य॒ र्‌च र्‌चा दक्षि॑णस्य । \newline
5. दक्षि॑णस्य हवि॒र्द्धान॑स्य हवि॒र्द्धान॑स्य॒ दक्षि॑णस्य॒ दक्षि॑णस्य हवि॒र्द्धान॑स्य । \newline
6. ह॒वि॒र्द्धान॑स्य मे॒थीम् मे॒थीꣳ ह॑वि॒र्द्धान॑स्य हवि॒र्द्धान॑स्य मे॒थीम् । \newline
7. ह॒वि॒र्द्धान॒स्येति॑ हविः - धान॑स्य । \newline
8. मे॒थीन्नि नि मे॒थीम् मे॒थीन्नि । \newline
9. नि ह॑न्ति हन्ति॒ नि नि ह॑न्ति । \newline
10. ह॒न्ति॒ शी॒र्॒.ष॒तः शी॑र्.ष॒तो ह॑न्ति हन्ति शीर्.ष॒तः । \newline
11. शी॒र्॒.ष॒त ए॒वैव शी॑र्.ष॒तः शी॑र्.ष॒त ए॒व । \newline
12. ए॒व य॒ज्ञ्स्य॑ य॒ज्ञ् स्यै॒वैव य॒ज्ञ्स्य॑ । \newline
13. य॒ज्ञ्स्य॒ यज॑मानो॒ यज॑मानो य॒ज्ञ्स्य॑ य॒ज्ञ्स्य॒ यज॑मानः । \newline
14. यज॑मान आ॒शिष॑ आ॒शिषो॒ यज॑मानो॒ यज॑मान आ॒शिषः॑ । \newline
15. आ॒शिषो ऽवावा॒ शिष॑ आ॒शिषो ऽव॑ । \newline
16. आ॒शिष॒ इत्या᳚ - शिषः॑ । \newline
17. अव॑ रुन्धे रु॒न्धे ऽवाव॑ रुन्धे । \newline
18. रु॒न्धे॒ द॒ण्डो द॒ण्डो रु॑न्धे रुन्धे द॒ण्डः । \newline
19. द॒ण्डो वै वै द॒ण्डो द॒ण्डो वै । \newline
20. वा औ॑प॒र औ॑प॒रो वै वा औ॑प॒रः । \newline
21. औ॒प॒र स्तृ॒तीय॑स्य तृ॒तीय॑ स्यौप॒र औ॑प॒र स्तृ॒तीय॑स्य । \newline
22. तृ॒तीय॑स्य हवि॒र्द्धान॑स्य हवि॒र्द्धान॑स्य तृ॒तीय॑स्य तृ॒तीय॑स्य हवि॒र्द्धान॑स्य । \newline
23. ह॒वि॒र्द्धान॑स्य वषट्का॒रेण॑ वषट्का॒रेण॑ हवि॒र्द्धान॑स्य हवि॒र्द्धान॑स्य वषट्का॒रेण॑ । \newline
24. ह॒वि॒र्द्धान॒स्येति॑ हविः - धान॑स्य । \newline
25. व॒ष॒ट्का॒रेणा क्ष॒ मक्षं॑ ॅवषट्का॒रेण॑ वषट्का॒रेणा क्ष᳚म् । \newline
26. व॒ष॒ट्का॒रेणेति॑ वषट् - का॒रेण॑ । \newline
27. अक्ष॑ मच्छिन दच्छिन॒ दक्ष॒ मक्ष॑ मच्छिनत् । \newline
28. अ॒च्छि॒न॒द् यद् यद॑च्छिन दच्छिन॒द् यत् । \newline
29. यत् तृ॒तीय॑म् तृ॒तीयं॒ ॅयद् यत् तृ॒तीय᳚म् । \newline
30. तृ॒तीय॑म् छ॒दि श्छ॒दि स्तृ॒तीय॑म् तृ॒तीय॑म् छ॒दिः । \newline
31. छ॒दिर्. ह॑वि॒र्द्धान॑योर्. हवि॒र्द्धान॑यो श्छ॒दि श्छ॒दिर्. ह॑वि॒र्द्धान॑योः । \newline
32. ह॒वि॒र्द्धान॑यो रुदाह्रि॒यत॑ उदाह्रि॒यते॑ हवि॒र्द्धान॑योर्. हवि॒र्द्धान॑यो रुदाह्रि॒यते᳚ । \newline
33. ह॒वि॒र्द्धान॑यो॒रिति॑ हविः - धान॑योः । \newline
34. उ॒दा॒ह्रि॒यते॑ तृ॒तीय॑स्य तृ॒तीय॑ स्योदाह्रि॒यत॑ उदाह्रि॒यते॑ तृ॒तीय॑स्य । \newline
35. उ॒दा॒ह्रि॒यत॒ इत्यु॑त् - आ॒ह्रि॒यते᳚ । \newline
36. तृ॒तीय॑स्य हवि॒र्द्धान॑स्य हवि॒र्द्धान॑स्य तृ॒तीय॑स्य तृ॒तीय॑स्य हवि॒र्द्धान॑स्य । \newline
37. ह॒वि॒र्द्धान॒स्या व॑रुद्ध्या॒ अव॑रुद्ध्यै हवि॒र्द्धान॑स्य हवि॒र्द्धान॒स्या व॑रुद्ध्यै । \newline
38. ह॒वि॒र्द्धान॒स्येति॑ हविः - धान॑स्य । \newline
39. अव॑रुद्ध्यै॒ शिरः॒ शिरो ऽव॑रुद्ध्या॒ अव॑रुद्ध्यै॒ शिरः॑ । \newline
40. अव॑रुद्ध्या॒ इत्यव॑ - रु॒द्ध्यै॒ । \newline
41. शिरो॒ वै वै शिरः॒ शिरो॒ वै । \newline
42. वा ए॒त दे॒तद् वै वा ए॒तत् । \newline
43. ए॒तद् य॒ज्ञ्स्य॑ य॒ज्ञ् स्यै॒त दे॒तद् य॒ज्ञ्स्य॑ । \newline
44. य॒ज्ञ्स्य॒ यद् यद् य॒ज्ञ्स्य॑ य॒ज्ञ्स्य॒ यत् । \newline
45. यद्ध॑वि॒र्द्धानꣳ॑ हवि॒र्द्धानं॒ ॅयद् यद्ध॑वि॒र्द्धान᳚म् । \newline
46. ह॒वि॒र्द्धानं॒ ॅविष्णो॒र् विष्णोर्॑. हवि॒र्द्धानꣳ॑ हवि॒र्द्धानं॒ ॅविष्णोः᳚ । \newline
47. ह॒वि॒र्द्धान॒मिति॑ हविः - धान᳚म् । \newline
48. विष्णो॑ र॒राटꣳ॑ र॒राटं॒ ॅविष्णो॒र् विष्णो॑ र॒राट᳚म् । \newline
49. र॒राट॑ मस्यसि र॒राटꣳ॑ र॒राट॑ मसि । \newline
50. अ॒सि॒ विष्णो॒र् विष्णो॑ रस्यसि॒ विष्णोः᳚ । \newline
51. विष्णोः᳚ पृ॒ष्ठम् पृ॒ष्ठं ॅविष्णो॒र् विष्णोः᳚ पृ॒ष्ठम् । \newline
52. पृ॒ष्ठ म॑स्यसि पृ॒ष्ठम् पृ॒ष्ठ म॑सि । \newline
53. अ॒सीती त्य॑स्य॒सीति॑ । \newline
54. इत्या॑हा॒हे तीत्या॑ह । \newline
55. आ॒ह॒ तस्मा॒त् तस्मा॑ दाहाह॒ तस्मा᳚त् । \newline
56. तस्मा॑ देताव॒ द्धैता॑व॒द्धा तस्मा॒त् तस्मा॑ देताव॒द्धा । \newline
57. ए॒ता॒व॒द्धा शिरः॒ शिर॑ एताव॒ द्धैता॑व॒द्धा शिरः॑ । \newline
58. ए॒ता॒व॒द्धेत्ये॑तावत् - धा । \newline
59. शिरो॒ विष्यू॑तं॒ ॅविष्यू॑तꣳ॒॒ शिरः॒ शिरो॒ विष्यू॑तम् । \newline
60. विष्यू॑तं॒ ॅविष्णो॒र् विष्णो॒र् विष्यू॑तं॒ ॅविष्यू॑तं॒ ॅविष्णोः᳚ । \newline
61. विष्यू॑त॒मिति॒ वि - स्यू॒त॒म् । \newline
62. विष्णोः॒ स्यूः स्यूर् विष्णो॒र् विष्णोः॒ स्यूः । \newline
63. स्यू र॑स्यसि॒ स्यूः स्यू र॑सि । \newline
64. अ॒सि॒ विष्णो॒र् विष्णो॑ रस्यसि॒ विष्णोः᳚ । \newline
65. विष्णो᳚र् ध्रु॒वम् ध्रु॒वं ॅविष्णो॒र् विष्णो᳚र् ध्रु॒वम् । \newline
66. ध्रु॒व म॑स्यसि ध्रु॒वम् ध्रु॒व म॑सि । \newline
67. अ॒सीती त्य॑स्य॒सीति॑ । \newline
68. इत्या॑हा॒हे तीत्या॑ह । \newline
69. आ॒ह॒ वै॒ष्ण॒वं ॅवै᳚ष्ण॒व मा॑हाह वैष्ण॒वम् । \newline
70. वै॒ष्ण॒वꣳ हि हि वै᳚ष्ण॒वं ॅवै᳚ष्ण॒वꣳ हि । \newline
71. हि दे॒वत॑या दे॒वत॑या॒ हि हि दे॒वत॑या । \newline
72. दे॒वत॑या हवि॒र्द्धानꣳ॑ हवि॒र्द्धान॑म् दे॒वत॑या दे॒वत॑या हवि॒र्द्धान᳚म् । \newline
73. ह॒वि॒र्द्धानं॒ ॅयं ॅयꣳ ह॑वि॒र्द्धानꣳ॑ हवि॒र्द्धानं॒ ॅयम् । \newline
74. ह॒वि॒र्द्धान॒मिति॑ हविः - धान᳚म् । \newline
75. यम् प्र॑थ॒मम् प्र॑थ॒मं ॅयं ॅयम् प्र॑थ॒मम् । \newline
76. प्र॒थ॒मम् ग्र॒न्थिम् ग्र॒न्थिम् प्र॑थ॒मम् प्र॑थ॒मम् ग्र॒न्थिम् । \newline
77. ग्र॒न्थिम् ग्र॑थ्नी॒याद् ग्र॑थ्नी॒याद् ग्र॒न्थिम् ग्र॒न्थिम् ग्र॑थ्नी॒यात् । \newline
78. ग्र॒थ्नी॒याद् यद् यद् ग्र॑थ्नी॒याद् ग्र॑थ्नी॒याद् यत् । \newline
79. यत् तम् तं ॅयद् यत् तम् । \newline
80. तन्न न तम् तन्न । \newline
81. न वि॑स्रꣳ॒॒सये᳚द् विस्रꣳ॒॒सये॒न् न न वि॑स्रꣳ॒॒सये᳚त् । \newline
82. वि॒स्रꣳ॒॒सये॒ दमे॑हे॒ना मे॑हेन विस्रꣳ॒॒सये᳚द् विस्रꣳ॒॒सये॒ दमे॑हेन । \newline
83. वि॒स्रꣳ॒॒सये॒दिति॑ वि - स्रꣳ॒॒सये᳚त् । \newline
84. अमे॑हेना द्ध्व॒र्यु र॑द्ध्व॒र्यु रमे॑हे॒ना मे॑हेना द्ध्व॒र्युः । \newline
85. अ॒द्ध्व॒र्युः प्र प्राद्ध्व॒र्यु र॑द्ध्व॒र्युः प्र । \newline
86. प्र मी॑येत मीयेत॒ प्र प्र मी॑येत । \newline
87. मी॒ये॒त॒ तस्मा॒त् तस्मा᳚न् मीयेत मीयेत॒ तस्मा᳚त् । \newline
88. तस्मा॒थ् स स तस्मा॒त् तस्मा॒थ् सः । \newline
89. स वि॒स्रस्यो॑ वि॒स्रस्यः॒ स स वि॒स्रस्यः॑ । \newline
90. वि॒स्रस्य॒ इति॑ वि - स्रस्यः॑ । \newline

\textbf{Ghana Paata } \newline

1. इत्या॒ शीर्प॑दया॒ ऽऽशीर्प॑द॒येतीत्या॒ शीर्प॑दय॒ र्‌च र्‌चा ऽऽशीर्प॑द॒ येतीत्या॒ शीर्प॑दय॒ र्‌चा । \newline
2. आ॒शीर्प॑दय॒ र्‌च र्‌चा ऽऽशीर्प॑दया॒ ऽऽशीर्प॑दय॒ र्‌चा दक्षि॑णस्य॒ दक्षि॑णस्य॒ र्‌चा ऽऽशीर्प॑दया॒ ऽऽशीर्प॑दय॒ र्‌चा दक्षि॑णस्य । \newline
3. आ॒शीर्प॑द॒येत्या॒शीः - प॒द॒या॒ । \newline
4. ऋ॒चा दक्षि॑णस्य॒ दक्षि॑णस्य॒ र्‌च र्‌चा दक्षि॑णस्य हवि॒र्द्धान॑स्य हवि॒र्द्धान॑स्य॒ दक्षि॑णस्य॒ र्‌च र्‌चा दक्षि॑णस्य हवि॒र्द्धान॑स्य । \newline
5. दक्षि॑णस्य हवि॒र्द्धान॑स्य हवि॒र्द्धान॑स्य॒ दक्षि॑णस्य॒ दक्षि॑णस्य हवि॒र्द्धान॑स्य मे॒थीम् मे॒थीꣳ ह॑वि॒र्द्धान॑स्य॒ दक्षि॑णस्य॒ दक्षि॑णस्य हवि॒र्द्धान॑स्य मे॒थीम् । \newline
6. ह॒वि॒र्द्धान॑स्य मे॒थीम् मे॒थीꣳ ह॑वि॒र्द्धान॑स्य हवि॒र्द्धान॑स्य मे॒थीन् नि नि मे॒थीꣳ ह॑वि॒र्द्धान॑स्य हवि॒र्द्धान॑स्य मे॒थीन् नि । \newline
7. ह॒वि॒र्द्धान॒स्येति॑ हविः - धान॑स्य । \newline
8. मे॒थीन् नि नि मे॒थीम् मे॒थीन् नि ह॑न्ति हन्ति॒ नि मे॒थीम् मे॒थीन् नि ह॑न्ति । \newline
9. नि ह॑न्ति हन्ति॒ नि नि ह॑न्ति शीर्.ष॒तः शी॑र्.ष॒तो ह॑न्ति॒ नि नि ह॑न्ति शीर्.ष॒तः । \newline
10. ह॒न्ति॒ शी॒र्॒.ष॒तः शी॑र्.ष॒तो ह॑न्ति हन्ति शीर्.ष॒त ए॒वैव शी॑र्.ष॒तो ह॑न्ति हन्ति शीर्.ष॒त ए॒व । \newline
11. शी॒र्॒.ष॒त ए॒वैव शी॑र्.ष॒तः शी॑र्.ष॒त ए॒व य॒ज्ञ्स्य॑ य॒ज्ञ्स्यै॒व शी॑र्.ष॒तः शी॑र्.ष॒त ए॒व य॒ज्ञ्स्य॑ । \newline
12. ए॒व य॒ज्ञ्स्य॑ य॒ज्ञ्स्यै॒ वैव य॒ज्ञ्स्य॒ यज॑मानो॒ यज॑मानो य॒ज्ञ्स्यै॒ वैव य॒ज्ञ्स्य॒ यज॑मानः । \newline
13. य॒ज्ञ्स्य॒ यज॑मानो॒ यज॑मानो य॒ज्ञ्स्य॑ य॒ज्ञ्स्य॒ यज॑मान आ॒शिष॑ आ॒शिषो॒ यज॑मानो य॒ज्ञ्स्य॑ य॒ज्ञ्स्य॒ यज॑मान आ॒शिषः॑ । \newline
14. यज॑मान आ॒शिष॑ आ॒शिषो॒ यज॑मानो॒ यज॑मान आ॒शिषो ऽवावा॒ शिषो॒ यज॑मानो॒ यज॑मान आ॒शिषो ऽव॑ । \newline
15. आ॒शिषो ऽवावा॒ शिष॑ आ॒शिषो ऽव॑ रुन्धे रु॒न्धे ऽवा॒शिष॑ आ॒शिषो ऽव॑ रुन्धे । \newline
16. आ॒शिष॒ इत्या᳚ - शिषः॑ । \newline
17. अव॑ रुन्धे रु॒न्धे ऽवाव॑ रुन्धे द॒ण्डो द॒ण्डो रु॒न्धे ऽवाव॑ रुन्धे द॒ण्डः । \newline
18. रु॒न्धे॒ द॒ण्डो द॒ण्डो रु॑न्धे रुन्धे द॒ण्डो वै वै द॒ण्डो रु॑न्धे रुन्धे द॒ण्डो वै । \newline
19. द॒ण्डो वै वै द॒ण्डो द॒ण्डो वा औ॑प॒र औ॑प॒रो वै द॒ण्डो द॒ण्डो वा औ॑प॒रः । \newline
20. वा औ॑प॒र औ॑प॒रो वै वा औ॑प॒र स्तृ॒तीय॑स्य तृ॒तीय॑ स्यौप॒रो वै वा औ॑प॒र स्तृ॒तीय॑स्य । \newline
21. औ॒प॒र स्तृ॒तीय॑स्य तृ॒तीय॑ स्यौप॒र औ॑प॒र स्तृ॒तीय॑स्य हवि॒र्द्धान॑स्य हवि॒र्द्धान॑स्य तृ॒तीय॑ स्यौप॒र औ॑प॒र स्तृ॒तीय॑स्य हवि॒र्द्धान॑स्य । \newline
22. तृ॒तीय॑स्य हवि॒र्द्धान॑स्य हवि॒र्द्धान॑स्य तृ॒तीय॑स्य तृ॒तीय॑स्य हवि॒र्द्धान॑स्य वषट्का॒रेण॑ वषट्का॒रेण॑ हवि॒र्द्धान॑स्य तृ॒तीय॑स्य तृ॒तीय॑स्य हवि॒र्द्धान॑स्य वषट्का॒रेण॑ । \newline
23. ह॒वि॒र्द्धान॑स्य वषट्का॒रेण॑ वषट्का॒रेण॑ हवि॒र्द्धान॑स्य हवि॒र्द्धान॑स्य वषट्का॒रेणा
क्ष॒ मक्षं॑ ॅवषट्का॒रेण॑ हवि॒र्द्धान॑स्य हवि॒र्द्धान॑स्य वषट्का॒रेणाक्ष᳚म् । \newline
24. ह॒वि॒र्द्धान॒स्येति॑ हविः - धान॑स्य । \newline
25. व॒ष॒ट्का॒रेणा क्ष॒ मक्षं॑ ॅवषट्का॒रेण॑ वषट्का॒रेणा क्ष॑ मच्छिन दच्छिन॒ दक्षं॑ ॅवषट्का॒रेण॑ वषट्का॒रेणा क्ष॑ मच्छिनत् । \newline
26. व॒ष॒ट्का॒रेणेति॑ वषट् - का॒रेण॑ । \newline
27. अक्ष॑ मच्छिन दच्छिन॒ दक्ष॒ मक्ष॑ मच्छिन॒द् यद् यद॑च्छिन॒ दक्ष॒ मक्ष॑ मच्छिन॒द् यत् । \newline
28. अ॒च्छि॒न॒द् यद् यद॑च्छिन दच्छिन॒द् यत् तृ॒तीय॑म् तृ॒तीयं॒ ॅयद॑च्छिन दच्छिन॒द् यत् तृ॒तीय᳚म् । \newline
29. यत् तृ॒तीय॑म् तृ॒तीयं॒ ॅयद् यत् तृ॒तीय॑म् छ॒दि श्छ॒दि स्तृ॒तीयं॒ ॅयद् यत् तृ॒तीय॑म् छ॒दिः । \newline
30. तृ॒तीय॑म् छ॒दि श्छ॒दि स्तृ॒तीय॑म् तृ॒तीय॑म् छ॒दिर्. ह॑वि॒र्द्धान॑योर्. हवि॒र्द्धान॑यो श्छ॒दि स्तृ॒तीय॑म् तृ॒तीय॑म् छ॒दिर्. ह॑वि॒र्द्धान॑योः । \newline
31. छ॒दिर्. ह॑वि॒र्द्धान॑योर्. हवि॒र्द्धान॑यो श्छ॒दि श्छ॒दिर्. ह॑वि॒र्द्धान॑यो रुदाह्रि॒यत॑ उदाह्रि॒यते॑ हवि॒र्द्धान॑यो श्छ॒दि श्छ॒दिर्. ह॑वि॒र्द्धान॑यो रुदाह्रि॒यते᳚ । \newline
32. ह॒वि॒र्द्धान॑यो रुदाह्रि॒यत॑ उदाह्रि॒यते॑ हवि॒र्द्धान॑योर्. हवि॒र्द्धान॑यो रुदाह्रि॒यते॑ तृ॒तीय॑स्य 
तृ॒तीय॑ स्योदाह्रि॒यते॑ हवि॒र्द्धान॑योर्. हवि॒र्द्धान॑यो रुदाह्रि॒यते॑ तृ॒तीय॑स्य । \newline
33. ह॒वि॒र्द्धान॑यो॒रिति॑ हविः - धान॑योः । \newline
34. उ॒दा॒ह्रि॒यते॑ तृ॒तीय॑स्य तृ॒तीय॑ स्योदाह्रि॒यत॑ उदाह्रि॒यते॑ तृ॒तीय॑स्य हवि॒र्द्धान॑स्य हवि॒र्द्धान॑स्य तृ॒तीय॑ स्योदाह्रि॒यत॑ उदाह्रि॒यते॑ तृ॒तीय॑स्य हवि॒र्द्धान॑स्य । \newline
35. उ॒दा॒ह्रि॒यत॒ इत्यु॑त् - आ॒ह्रि॒यते᳚ । \newline
36. तृ॒तीय॑स्य हवि॒र्द्धान॑स्य हवि॒र्द्धान॑स्य तृ॒तीय॑स्य तृ॒तीय॑स्य हवि॒र्द्धान॒स्या व॑रुद्ध्या॒ अव॑रुद्ध्यै हवि॒र्द्धान॑स्य तृ॒तीय॑स्य तृ॒तीय॑स्य हवि॒र्द्धान॒स्या व॑रुद्ध्यै । \newline
37. ह॒वि॒र्द्धान॒स्या व॑रुद्ध्या॒ अव॑रुद्ध्यै हवि॒र्द्धान॑स्य हवि॒र्द्धान॒स्या व॑रुद्ध्यै॒ शिरः॒ शिरो ऽव॑रुद्ध्यै हवि॒र्द्धान॑स्य हवि॒र्द्धान॒स्या व॑रुद्ध्यै॒ शिरः॑ । \newline
38. ह॒वि॒र्द्धान॒स्येति॑ हविः - धान॑स्य । \newline
39. अव॑रुद्ध्यै॒ शिरः॒ शिरो ऽव॑रुद्ध्या॒ अव॑रुद्ध्यै॒ शिरो॒ वै वै शिरो ऽव॑रुद्ध्या॒ अव॑रुद्ध्यै॒ शिरो॒ वै । \newline
40. अव॑रुद्ध्या॒ इत्यव॑ - रु॒द्ध्यै॒ । \newline
41. शिरो॒ वै वै शिरः॒ शिरो॒ वा ए॒त दे॒तद् वै शिरः॒ शिरो॒ वा ए॒तत् । \newline
42. वा ए॒त दे॒तद् वै वा ए॒तद् य॒ज्ञ्स्य॑ य॒ज्ञ् स्यै॒तद् वै वा ए॒तद् य॒ज्ञ्स्य॑ । \newline
43. ए॒तद् य॒ज्ञ्स्य॑ य॒ज्ञ् स्यै॒त दे॒तद् य॒ज्ञ्स्य॒ यद् यद् य॒ज्ञ् स्यै॒त दे॒तद् य॒ज्ञ्स्य॒ यत् । \newline
44. य॒ज्ञ्स्य॒ यद् यद् य॒ज्ञ्स्य॑ य॒ज्ञ्स्य॒ यद्ध॑वि॒र्द्धानꣳ॑ हवि॒र्द्धानं॒ ॅयद् य॒ज्ञ्स्य॑ य॒ज्ञ्स्य॒ यद्ध॑वि॒र्द्धान᳚म् । \newline
45. यद्ध॑वि॒र्द्धानꣳ॑ हवि॒र्द्धानं॒ ॅयद् यद्ध॑वि॒र्द्धानं॒ ॅविष्णो॒र् विष्णोर्॑. हवि॒र्द्धानं॒ ॅयद् यद्ध॑वि॒र्द्धानं॒ ॅविष्णोः᳚ । \newline
46. ह॒वि॒र्द्धानं॒ ॅविष्णो॒र् विष्णोर्॑. हवि॒र्द्धानꣳ॑ हवि॒र्द्धानं॒ ॅविष्णो॑ र॒राटꣳ॑ र॒राटं॒ ॅविष्णोर्॑. हवि॒र्द्धानꣳ॑ हवि॒र्द्धानं॒ ॅविष्णो॑ र॒राट᳚म् । \newline
47. ह॒वि॒र्द्धान॒मिति॑ हविः - धान᳚म् । \newline
48. विष्णो॑ र॒राटꣳ॑ र॒राटं॒ ॅविष्णो॒र् विष्णो॑ र॒राट॑ मस्यसि र॒राटं॒ ॅविष्णो॒र् विष्णो॑ र॒राट॑ मसि । \newline
49. र॒राट॑ मस्यसि र॒राटꣳ॑ र॒राट॑ मसि॒ विष्णो॒र् विष्णो॑ रसि र॒राटꣳ॑ र॒राट॑ मसि॒ विष्णोः᳚ । \newline
50. अ॒सि॒ विष्णो॒र् विष्णो॑ रस्यसि॒ विष्णोः᳚ पृ॒ष्ठम् पृ॒ष्ठं ॅविष्णो॑ रस्यसि॒ विष्णोः᳚ पृ॒ष्ठम् । \newline
51. विष्णोः᳚ पृ॒ष्ठम् पृ॒ष्ठं ॅविष्णो॒र् विष्णोः᳚ पृ॒ष्ठ म॑स्यसि पृ॒ष्ठं ॅविष्णो॒र् विष्णोः᳚ पृ॒ष्ठ म॑सि । \newline
52. पृ॒ष्ठ म॑स्यसि पृ॒ष्ठम् पृ॒ष्ठ म॒सीती त्य॑सि पृ॒ष्ठम् पृ॒ष्ठ म॒सीति॑ । \newline
53. अ॒सीती त्य॑स्य॒ सीत्या॑ हा॒हे त्य॑स्य॒ सीत्या॑ह । \newline
54. इत्या॑हा॒हे तीत्या॑ह॒ तस्मा॒त् तस्मा॑ दा॒हे तीत्या॑ह॒ तस्मा᳚त् । \newline
55. आ॒ह॒ तस्मा॒त् तस्मा॑ दाहाह॒ तस्मा॑ देताव॒ द्धैता॑व॒द्धा तस्मा॑ दाहाह॒ तस्मा॑ देताव॒द्धा । \newline
56. तस्मा॑ देताव॒ द्धैता॑व॒द्धा तस्मा॒त् तस्मा॑ देताव॒द्धा शिरः॒ शिर॑ एताव॒द्धा तस्मा॒त् तस्मा॑ देताव॒द्धा शिरः॑ । \newline
57. ए॒ता॒व॒द्धा शिरः॒ शिर॑ एताव॒ द्धैता॑व॒द्धा शिरो॒ विष्यू॑तं॒ ॅविष्यू॑तꣳ॒॒ शिर॑ एताव॒ द्धैता॑व॒द्धा शिरो॒ विष्यू॑तम् । \newline
58. ए॒ता॒व॒द्धेत्ये॑तावत् - धा । \newline
59. शिरो॒ विष्यू॑तं॒ ॅविष्यू॑तꣳ॒॒ शिरः॒ शिरो॒ विष्यू॑तं॒ ॅविष्णो॒र् विष्णो॒र् विष्यू॑तꣳ॒॒ शिरः॒ शिरो॒ विष्यू॑तं॒ ॅविष्णोः᳚ । \newline
60. विष्यू॑तं॒ ॅविष्णो॒र् विष्णो॒र् विष्यू॑तं॒ ॅविष्यू॑तं॒ ॅविष्णोः॒ स्यूः स्यूर् विष्णो॒र् विष्यू॑तं॒ ॅविष्यू॑तं॒ ॅविष्णोः॒ स्यूः । \newline
61. विष्यू॑त॒मिति॒ वि - स्यू॒त॒म् । \newline
62. विष्णोः॒ स्यूः स्यूर् विष्णो॒र् विष्णोः॒ स्यू र॑स्यसि॒ स्यूर् विष्णो॒र् विष्णोः॒ स्यूर॑सि । \newline
63. स्यूर॑ स्यसि॒ स्यूः स्यूर॑सि॒ विष्णो॒र् विष्णो॑ रसि॒ स्यूः स्यूर॑सि॒ विष्णोः᳚ । \newline
64. अ॒सि॒ विष्णो॒र् विष्णो॑ रस्यसि॒ विष्णो᳚र् ध्रु॒वम् ध्रु॒वं ॅविष्णो॑ रस्यसि॒ विष्णो᳚र् ध्रु॒वम् । \newline
65. विष्णो᳚र् ध्रु॒वम् ध्रु॒वं ॅविष्णो॒र् विष्णो᳚र् ध्रु॒व म॑स्यसि ध्रु॒वं ॅविष्णो॒र् विष्णो᳚र् ध्रु॒व म॑सि । \newline
66. ध्रु॒व म॑स्यसि ध्रु॒वम् ध्रु॒व म॒सीती त्य॑सि ध्रु॒वम् ध्रु॒व म॒सीति॑ । \newline
67. अ॒सीती त्य॑स्य॒ सीत्या॑ हा॒हे त्य॑स्य॒ सीत्या॑ह । \newline
68. इत्या॑हा॒हे तीत्या॑ह वैष्ण॒वं ॅवै᳚ष्ण॒व मा॒हे तीत्या॑ह वैष्ण॒वम् । \newline
69. आ॒ह॒ वै॒ष्ण॒वं ॅवै᳚ष्ण॒व मा॑हाह वैष्ण॒वꣳ हि हि वै᳚ष्ण॒व मा॑हाह वैष्ण॒वꣳ हि । \newline
70. वै॒ष्ण॒वꣳ हि हि वै᳚ष्ण॒वं ॅवै᳚ष्ण॒वꣳ हि दे॒वत॑या दे॒वत॑या॒ हि वै᳚ष्ण॒वं ॅवै᳚ष्ण॒वꣳ हि दे॒वत॑या । \newline
71. हि दे॒वत॑या दे॒वत॑या॒ हि हि दे॒वत॑या हवि॒र्द्धानꣳ॑ हवि॒र्द्धान॑म् दे॒वत॑या॒ हि हि दे॒वत॑या हवि॒र्द्धान᳚म् । \newline
72. दे॒वत॑या हवि॒र्द्धानꣳ॑ हवि॒र्द्धान॑म् दे॒वत॑या दे॒वत॑या हवि॒र्द्धानं॒ ॅयं ॅयꣳ ह॑वि॒र्द्धान॑म् दे॒वत॑या दे॒वत॑या हवि॒र्द्धानं॒ ॅयम् । \newline
73. ह॒वि॒र्द्धानं॒ ॅयं ॅयꣳ ह॑वि॒र्द्धानꣳ॑ हवि॒र्द्धानं॒ ॅयम् प्र॑थ॒मम् प्र॑थ॒मं ॅयꣳ ह॑वि॒र्द्धानꣳ॑ हवि॒र्द्धानं॒ ॅयम् प्र॑थ॒मम् । \newline
74. ह॒वि॒र्द्धान॒मिति॑ हविः - धान᳚म् । \newline
75. यम् प्र॑थ॒मम् प्र॑थ॒मं ॅयं ॅयम् प्र॑थ॒मम् ग्र॒न्थिम् ग्र॒न्थिम् प्र॑थ॒मं ॅयं ॅयम् प्र॑थ॒मम् ग्र॒न्थिम् । \newline
76. प्र॒थ॒मम् ग्र॒न्थिम् ग्र॒न्थिम् प्र॑थ॒मम् प्र॑थ॒मम् ग्र॒न्थिम् ग्र॑थ्नी॒याद् ग्र॑थ्नी॒याद् ग्र॒न्थिम् प्र॑थ॒मम् प्र॑थ॒मम् ग्र॒न्थिम् ग्र॑थ्नी॒यात् । \newline
77. ग्र॒न्थिम् ग्र॑थ्नी॒याद् ग्र॑थ्नी॒याद् ग्र॒न्थिम् ग्र॒न्थिम् ग्र॑थ्नी॒याद् यद् यद् ग्र॑थ्नी॒याद् ग्र॒न्थिम् ग्र॒न्थिम् ग्र॑थ्नी॒याद् यत् । \newline
78. ग्र॒थ्नी॒याद् यद् यद् ग्र॑थ्नी॒याद् ग्र॑थ्नी॒याद् यत् तम् तं ॅयद् ग्र॑थ्नी॒याद् ग्र॑थ्नी॒याद् यत् तम् । \newline
79. यत् तम् तं ॅयद् यत् तन्न न तं ॅयद् यत् तन्न । \newline
80. तन् न न तम् तन् न वि॑स्रꣳ॒॒सये᳚द् विस्रꣳ॒॒सये॒न् न तम् तन् न वि॑स्रꣳ॒॒सये᳚त् । \newline
81. न वि॑स्रꣳ॒॒सये᳚द् विस्रꣳ॒॒सये॒न् न न वि॑स्रꣳ॒॒सये॒ दमे॑हे॒ना मे॑हेन विस्रꣳ॒॒सये॒न् न न वि॑स्रꣳ॒॒सये॒ दमे॑हेन । \newline
82. वि॒स्रꣳ॒॒सये॒ दमे॑हे॒ना मे॑हेन विस्रꣳ॒॒सये᳚द् विस्रꣳ॒॒सये॒ दमे॑हेना द्ध्व॒र्यु र॑द्ध्व॒र्यु रमे॑हेन विस्रꣳ॒॒सये᳚द् विस्रꣳ॒॒सये॒ दमे॑हेना द्ध्व॒र्युः । \newline
83. वि॒स्रꣳ॒॒सये॒दिति॑ वि - स्रꣳ॒॒सये᳚त् । \newline
84. अमे॑हेना द्ध्व॒र्यु र॑द्ध्व॒र्यु रमे॑हे॒ना मे॑हेना द्ध्व॒र्युः प्र प्राद्ध्व॒र्यु रमे॑हे॒ना मे॑हेना द्ध्व॒र्युः प्र । \newline
85. अ॒द्ध्व॒र्युः प्र प्राद्ध्व॒र्यु र॑द्ध्व॒र्युः प्र मी॑येत मीयेत॒ प्राद्ध्व॒र्यु र॑द्ध्व॒र्युः प्र मी॑येत । \newline
86. प्र मी॑येत मीयेत॒ प्र प्र मी॑येत॒ तस्मा॒त् तस्मा᳚न् मीयेत॒ प्र प्र मी॑येत॒ तस्मा᳚त् । \newline
87. मी॒ये॒त॒ तस्मा॒त् तस्मा᳚न् मीयेत मीयेत॒ तस्मा॒थ् स स तस्मा᳚न् मीयेत मीयेत॒ तस्मा॒थ् सः । \newline
88. तस्मा॒थ् स स तस्मा॒त् तस्मा॒थ् स वि॒स्रस्यो॑ वि॒स्रस्यः॒ स तस्मा॒त् तस्मा॒थ् स वि॒स्रस्यः॑ । \newline
89. स वि॒स्रस्यो॑ वि॒स्रस्यः॒ स स वि॒स्रस्यः॑ । \newline
90. वि॒स्रस्य॒ इति॑ वि - स्रस्यः॑ । \newline
\pagebreak
\markright{ TS 6.2.10.1  \hfill https://www.vedavms.in \hfill}

\section{ TS 6.2.10.1 }

\textbf{TS 6.2.10.1 } \newline
\textbf{Samhita Paata} \newline

दे॒वस्य॑ त्वा सवि॒तुः प्र॑स॒व इत्यभ्रि॒मा द॑त्ते॒ प्रसू᳚त्या अ॒श्विनो᳚ र्बा॒हुभ्या॒मित्या॑हा॒श्विनौ॒ हि दे॒वाना॑मद्ध्व॒र्यू आस्तां᳚ पू॒ष्णो हस्ता᳚भ्या॒मित्या॑ह॒ यत्यै॒ वज्र॑ इव॒ वा ए॒षा यदभ्रि॒रभ्रि॑रसि॒ नारि॑र॒सीत्या॑ह॒ शान्त्यै॒ काण्डे॑ काण्डे॒ वै क्रि॒यमा॑णे य॒ज्ञ्ꣳ रक्षाꣳ॑सि जिघाꣳसन्ति॒ परि॑लिखितꣳ॒॒ रक्षः॒ परि॑लिखिता॒ अरा॑तय॒ इत्या॑ह॒ रक्ष॑सा॒मप॑हत्या- [  ] \newline

\textbf{Pada Paata} \newline

दे॒वस्य॑ । त्वा॒ । स॒वि॒तुः । प्र॒स॒व इति॑ प्र -स॒वे । इति॑ । अभ्रि᳚म् । एति॑ । द॒त्ते॒ । प्रसू᳚त्या॒ इति॒ प्र - सू॒त्यै॒ । अ॒श्विनोः᳚ । बा॒हुभ्या॒मिति॑ बा॒हु-भ्या॒म् । इति॑ । आ॒ह॒ । अ॒श्विनौ᳚ । हि । दे॒वाना᳚म् । अ॒द्ध्व॒र्यू इति॑ । आस्ता᳚म् । पू॒ष्णः । हस्ता᳚भ्याम् । इति॑ । आ॒ह॒ । यत्यै᳚ । वज्रः॑ । इ॒व॒ । वै । ए॒षा । यत् । अभ्रिः॑ । अभ्रिः॑ । अ॒सि॒ । नारिः॑ । अ॒सि॒ । इति॑ । आ॒ह॒ । शान्त्यै᳚ । काण्डे॑काण्ड॒ इति॒ काण्डे᳚ - का॒ण्डे॒ । वै । क्रि॒यमा॑णे । य॒ज्ञ्म् । रक्षाꣳ॑सि । जि॒घाꣳ॒॒स॒न्ति॒ । परि॑लिखित॒मिति॒ परि॑ - लि॒खि॒त॒म् । रक्षः॑ । परि॑लिखिता॒ इति॒ परि॑ - लि॒खि॒ताः॒ । अरा॑तयः । इति॑ । आ॒ह॒ । रक्ष॑साम् । अप॑हत्या॒ इत्यप॑ - ह॒त्यै॒ ।  \newline


\textbf{Krama Paata} \newline

दे॒वस्य॑ त्वा । त्वा॒ स॒वि॒तुः । स॒वि॒तुः प्र॑स॒वे । प्र॒स॒व इति॑ । प्र॒स॒व इति॑ प्र - स॒वे । इत्यभ्रि᳚म् । अभ्रि॒मा । आ द॑त्ते । द॒त्ते॒ प्रसू᳚त्यै । प्रसू᳚त्या अ॒श्विनोः᳚ । प्रसू᳚त्या॒ इति॒ प्र - सू॒त्यै॒ । अ॒श्विनो᳚र् बा॒हुभ्या᳚म् । बा॒हुभ्या॒मिति॑ । बा॒हुभ्या॒मिति॑ बा॒हु - भ्या॒म् । इत्या॑ह । आ॒हा॒श्विनौ᳚ । अ॒श्विनौ॒ हि । हि दे॒वाना᳚म् । दे॒वाना॑मद्ध्व॒र्यू । अ॒द्ध्व॒र्यू आस्ता᳚म् । अ॒द्ध्व॒र्यू इत्य॑द्ध्व॒र्यू । आस्ता᳚म् पू॒ष्णः । पू॒ष्णो हस्ता᳚भ्याम् । हस्ता᳚भ्या॒मिति॑ । इत्या॑ह । आ॒ह॒ यत्यै᳚ । यत्यै॒ वज्रः॑ । वज्र॑ इव । इ॒व॒ वै । वा ए॒षा । ए॒षा यत् । यदभ्रिः॑ । अभ्रि॒रभ्रिः॑ । अभ्रि॑रसि । अ॒सि॒ नारिः॑ । नारि॑रसि । अ॒सीति॑ । इत्या॑ह । आ॒ह॒ शान्त्यै᳚ । शान्त्यै॒ काण्डे॑काण्डे । काण्डे॑काण्डे॒ वै । काण्डे॑काण्ड॒ इति॒ काण्डे᳚ - का॒ण्डे॒ । वै क्रि॒यमा॑णे । क्रि॒यमा॑णे य॒ज्ञ्म् । य॒ज्ञ्ꣳ रक्षाꣳ॑सि । रक्षाꣳ॑सि जिघाꣳसन्ति । जि॒घाꣳ॒॒स॒न्ति॒ परि॑लिखितम् । परि॑लिखितꣳ॒॒ रक्षः॑ । परि॑लिखित॒मिति॒ परि॑ - लि॒खि॒त॒म् । रक्षः॒ परि॑लिखिताः । परि॑लिखिता॒ अरा॑तयः । परि॑लिखिता॒ इति॒ परि॑ - लि॒खि॒ताः॒ । अरा॑तय॒ इति॑ । इत्या॑ह । आ॒ह॒ रक्ष॑साम् । रक्ष॑सा॒मप॑हत्यै । अप॑हत्या इ॒दम् । अप॑हत्या॒ इत्यप॑ - ह॒त्यै॒ \newline

\textbf{Jatai Paata} \newline

1. दे॒वस्य॑ त्वा त्वा दे॒वस्य॑ दे॒वस्य॑ त्वा । \newline
2. त्वा॒ स॒वि॒तुः स॑वि॒तु स्त्वा᳚ त्वा सवि॒तुः । \newline
3. स॒वि॒तुः प्र॑स॒वे प्र॑स॒वे स॑वि॒तुः स॑वि॒तुः प्र॑स॒वे । \newline
4. प्र॒स॒व इतीति॑ प्रस॒वे प्र॑स॒व इति॑ । \newline
5. प्र॒स॒व इति॑ प्र - स॒वे । \newline
6. इत्यभ्रि॒ मभ्रि॒ मिती त्यभ्रि᳚म् । \newline
7. अभ्रि॒ मा ऽभ्रि॒ मभ्रि॒ मा । \newline
8. आ द॑त्ते दत्त॒ आ द॑त्ते । \newline
9. द॒त्ते॒ प्रसू᳚त्यै॒ प्रसू᳚त्यै दत्ते दत्ते॒ प्रसू᳚त्यै । \newline
10. प्रसू᳚त्या अ॒श्विनो॑ र॒श्विनोः॒ प्रसू᳚त्यै॒ प्रसू᳚त्या अ॒श्विनोः᳚ । \newline
11. प्रसू᳚त्या॒ इति॒ प्र - सू॒त्यै॒ । \newline
12. अ॒श्विनो᳚र् बा॒हुभ्या᳚म् बा॒हुभ्या॑ म॒श्विनो॑ र॒श्विनो᳚र् बा॒हुभ्या᳚म् । \newline
13. बा॒हुभ्या॒ मितीति॑ बा॒हुभ्या᳚म् बा॒हुभ्या॒ मिति॑ । \newline
14. बा॒हुभ्या॒मिति॑ बा॒हु - भ्या॒म् । \newline
15. इत्या॑हा॒हे तीत्या॑ह । \newline
16. आ॒हा॒श्विना॑ व॒श्विना॑ वाहाहा॒ श्विनौ᳚ । \newline
17. अ॒श्विनौ॒ हि ह्य॑श्विना॑ व॒श्विनौ॒ हि । \newline
18. हि दे॒वाना᳚म् दे॒वानाꣳ॒॒ हि हि दे॒वाना᳚म् । \newline
19. दे॒वाना॑ मद्ध्व॒र्यू अ॑द्ध्व॒र्यू दे॒वाना᳚म् दे॒वाना॑ मद्ध्व॒र्यू । \newline
20. अ॒द्ध्व॒र्यू आस्ता॒ मास्ता॑ मद्ध्व॒र्यू अ॑द्ध्व॒र्यू आस्ता᳚म् । \newline
21. अ॒द्ध्व॒र्यू इति॑ अ॒द्ध्व॒र्यू । \newline
22. आस्ता᳚म् पू॒ष्णः पू॒ष्ण आस्ता॒ मास्ता᳚म् पू॒ष्णः । \newline
23. पू॒ष्णो हस्ता᳚भ्याꣳ॒॒ हस्ता᳚भ्याम् पू॒ष्णः पू॒ष्णो हस्ता᳚भ्याम् । \newline
24. हस्ता᳚भ्या॒ मितीति॒ हस्ता᳚भ्याꣳ॒॒ हस्ता᳚भ्या॒ मिति॑ । \newline
25. इत्या॑हा॒हे तीत्या॑ह । \newline
26. आ॒ह॒ यत्यै॒ यत्या॑ आहाह॒ यत्यै᳚ । \newline
27. यत्यै॒ वज्रो॒ वज्रो॒ यत्यै॒ यत्यै॒ वज्रः॑ । \newline
28. वज्र॑ इवेव॒ वज्रो॒ वज्र॑ इव । \newline
29. इ॒व॒ वै वा इ॑वेव॒ वै । \newline
30. वा ए॒षैषा वै वा ए॒षा । \newline
31. ए॒षा यद् यदे॒षैषा यत् । \newline
32. यदभ्रि॒ रभ्रि॒र् यद् यदभ्रिः॑ । \newline
33. अभ्रि॒ रभ्रिः॑ । \newline
34. अभ्रि॑ रस्य॒ स्यभ्रि॒ रभ्रि॑ रसि । \newline
35. अ॒सि॒ नारि॒र् नारि॑ रस्यसि॒ नारिः॑ । \newline
36. नारि॑ रस्यसि॒ नारि॒र् नारि॑ रसि । \newline
37. अ॒सीती त्य॑स्य॒सीति॑ । \newline
38. इत्या॑हा॒हे तीत्या॑ह । \newline
39. आ॒ह॒ शान्त्यै॒ शान्त्या॑ आहाह॒ शान्त्यै᳚ । \newline
40. शान्त्यै॒ काण्डे॑काण्डे॒ काण्डे॑काण्डे॒ शान्त्यै॒ शान्त्यै॒ काण्डे॑काण्डे । \newline
41. काण्डे॑काण्डे॒ वै वै काण्डे॑काण्डे॒ काण्डे॑काण्डे॒ वै । \newline
42. काण्डे॑काण्ड॒ इति॒ काण्डे᳚ - का॒ण्डे॒ । \newline
43. वै क्रि॒यमा॑णे क्रि॒यमा॑णे॒ वै वै क्रि॒यमा॑णे । \newline
44. क्रि॒यमा॑णे य॒ज्ञ्ं ॅय॒ज्ञ्म् क्रि॒यमा॑णे क्रि॒यमा॑णे य॒ज्ञ्म् । \newline
45. य॒ज्ञ्ꣳ रक्षाꣳ॑सि॒ रक्षाꣳ॑सि य॒ज्ञ्ं ॅय॒ज्ञ्ꣳ रक्षाꣳ॑सि । \newline
46. रक्षाꣳ॑सि जिघाꣳसन्ति जिघाꣳसन्ति॒ रक्षाꣳ॑सि॒ रक्षाꣳ॑सि जिघाꣳसन्ति । \newline
47. जि॒घाꣳ॒॒स॒न्ति॒ परि॑लिखित॒म् परि॑लिखितम् जिघाꣳसन्ति जिघाꣳसन्ति॒ परि॑लिखितम् । \newline
48. परि॑लिखितꣳ॒॒ रक्षो॒ रक्षः॒ परि॑लिखित॒म् परि॑लिखितꣳ॒॒ रक्षः॑ । \newline
49. परि॑लिखित॒मिति॒ परि॑ - लि॒खि॒त॒म् । \newline
50. रक्षः॒ परि॑लिखिताः॒ परि॑लिखिता॒ रक्षो॒ रक्षः॒ परि॑लिखिताः । \newline
51. परि॑लिखिता॒ अरा॑त॒यो ऽरा॑तयः॒ परि॑लिखिताः॒ परि॑लिखिता॒ अरा॑तयः । \newline
52. परि॑लिखिता॒ इति॒ परि॑ - लि॒खि॒ताः॒ । \newline
53. अरा॑तय॒ इती त्यरा॑त॒यो ऽरा॑तय॒ इति॑ । \newline
54. इत्या॑हा॒हे तीत्या॑ह । \newline
55. आ॒ह॒ रक्ष॑साꣳ॒॒ रक्ष॑सा माहाह॒ रक्ष॑साम् । \newline
56. रक्ष॑सा॒ मप॑हत्या॒ अप॑हत्यै॒ रक्ष॑साꣳ॒॒ रक्ष॑सा॒ मप॑हत्यै । \newline
57. अप॑हत्या इ॒द मि॒द मप॑हत्या॒ अप॑हत्या इ॒दम् । \newline
58. अप॑हत्या॒ इत्यप॑ - ह॒त्यै॒ । \newline

\textbf{Ghana Paata } \newline

1. दे॒वस्य॑ त्वा त्वा दे॒वस्य॑ दे॒वस्य॑ त्वा सवि॒तुः स॑वि॒तु स्त्वा॑ दे॒वस्य॑ दे॒वस्य॑ त्वा सवि॒तुः । \newline
2. त्वा॒ स॒वि॒तुः स॑वि॒तु स्त्वा᳚ त्वा सवि॒तुः प्र॑स॒वे प्र॑स॒वे स॑वि॒तु स्त्वा᳚ त्वा सवि॒तुः प्र॑स॒वे । \newline
3. स॒वि॒तुः प्र॑स॒वे प्र॑स॒वे स॑वि॒तुः स॑वि॒तुः प्र॑स॒व इतीति॑ प्रस॒वे स॑वि॒तुः स॑वि॒तुः प्र॑स॒व इति॑ । \newline
4. प्र॒स॒व इतीति॑ प्रस॒वे प्र॑स॒व इत्यभ्रि॒ मभ्रि॒ मिति॑ प्रस॒वे प्र॑स॒व इत्यभ्रि᳚म् । \newline
5. प्र॒स॒व इति॑ प्र - स॒वे । \newline
6. इत्यभ्रि॒ मभ्रि॒ मिती त्यभ्रि॒ मा ऽभ्रि॒ मिती त्यभ्रि॒ मा । \newline
7. अभ्रि॒ मा ऽभ्रि॒ मभ्रि॒ मा द॑त्ते दत्त॒ आ ऽभ्रि॒ मभ्रि॒ मा द॑त्ते । \newline
8. आ द॑त्ते दत्त॒ आ द॑त्ते॒ प्रसू᳚त्यै॒ प्रसू᳚त्यै दत्त॒ आ द॑त्ते॒ प्रसू᳚त्यै । \newline
9. द॒त्ते॒ प्रसू᳚त्यै॒ प्रसू᳚त्यै दत्ते दत्ते॒ प्रसू᳚त्या अ॒श्विनो॑ र॒श्विनोः॒ प्रसू᳚त्यै दत्ते दत्ते॒ प्रसू᳚त्या अ॒श्विनोः᳚ । \newline
10. प्रसू᳚त्या अ॒श्विनो॑ र॒श्विनोः॒ प्रसू᳚त्यै॒ प्रसू᳚त्या अ॒श्विनो᳚र् बा॒हुभ्या᳚म् बा॒हुभ्या॑ म॒श्विनोः॒ प्रसू᳚त्यै॒ प्रसू᳚त्या अ॒श्विनो᳚र् बा॒हुभ्या᳚म् । \newline
11. प्रसू᳚त्या॒ इति॒ प्र - सू॒त्यै॒ । \newline
12. अ॒श्विनो᳚र् बा॒हुभ्या᳚म् बा॒हुभ्या॑ म॒श्विनो॑ र॒श्विनो᳚र् बा॒हुभ्या॒ मितीति॑ बा॒हुभ्या॑ म॒श्विनो॑ र॒श्विनो᳚र् बा॒हुभ्या॒ मिति॑ । \newline
13. बा॒हुभ्या॒ मितीति॑ बा॒हुभ्या᳚म् बा॒हुभ्या॒ मित्या॑हा॒हेति॑ बा॒हुभ्या᳚म् बा॒हुभ्या॒ मित्या॑ह । \newline
14. बा॒हुभ्या॒मिति॑ बा॒हु - भ्या॒म् । \newline
15. इत्या॑हा॒हे तीत्या॑ हा॒श्विना॑ व॒श्विना॑ वा॒हे तीत्या॑ हा॒श्विनौ᳚ । \newline
16. आ॒हा॒श्विना॑ व॒श्विना॑ वाहा हा॒श्विनौ॒ हि ह्य॑श्विना॑ वाहा हा॒श्विनौ॒ हि । \newline
17. अ॒श्विनौ॒ हि ह्य॑श्विना॑ व॒श्विनौ॒ हि दे॒वाना᳚म् दे॒वानाꣳ॒॒ ह्य॑श्विना॑ व॒श्विनौ॒ हि दे॒वाना᳚म् । \newline
18. हि दे॒वाना᳚म् दे॒वानाꣳ॒॒ हि हि दे॒वाना॑ मद्ध्व॒र्यू अ॑द्ध्व॒र्यू दे॒वानाꣳ॒॒ हि हि दे॒वाना॑ मद्ध्व॒र्यू । \newline
19. दे॒वाना॑ मद्ध्व॒र्यू अ॑द्ध्व॒र्यू दे॒वाना᳚म् दे॒वाना॑ मद्ध्व॒र्यू आस्ता॒ मास्ता॑ मद्ध्व॒र्यू दे॒वाना᳚म् दे॒वाना॑ मद्ध्व॒र्यू आस्ता᳚म् । \newline
20. अ॒द्ध्व॒र्यू आस्ता॒ मास्ता॑ मद्ध्व॒र्यू अ॑द्ध्व॒र्यू आस्ता᳚म् पू॒ष्णः पू॒ष्ण आस्ता॑ मद्ध्व॒र्यू अ॑द्ध्व॒र्यू आस्ता᳚म् पू॒ष्णः । \newline
21. अ॒द्ध्व॒र्यू इति॑ अ॒द्ध्व॒र्यू । \newline
22. आस्ता᳚म् पू॒ष्णः पू॒ष्ण आस्ता॒ मास्ता᳚म् पू॒ष्णो हस्ता᳚भ्याꣳ॒॒ हस्ता᳚भ्याम् पू॒ष्ण आस्ता॒ मास्ता᳚म् पू॒ष्णो हस्ता᳚भ्याम् । \newline
23. पू॒ष्णो हस्ता᳚भ्याꣳ॒॒ हस्ता᳚भ्याम् पू॒ष्णः पू॒ष्णो हस्ता᳚भ्या॒ मितीति॒ हस्ता᳚भ्याम् पू॒ष्णः पू॒ष्णो हस्ता᳚भ्या॒ मिति॑ । \newline
24. हस्ता᳚भ्या॒ मितीति॒ हस्ता᳚भ्याꣳ॒॒ हस्ता᳚भ्या॒ मित्या॑हा॒हेति॒ हस्ता᳚भ्याꣳ॒॒ हस्ता᳚भ्या॒ मित्या॑ह । \newline
25. इत्या॑हा॒हे तीत्या॑ह॒ यत्यै॒ यत्या॑ आ॒हे तीत्या॑ह॒ यत्यै᳚ । \newline
26. आ॒ह॒ यत्यै॒ यत्या॑ आहाह॒ यत्यै॒ वज्रो॒ वज्रो॒ यत्या॑ आहाह॒ यत्यै॒ वज्रः॑ । \newline
27. यत्यै॒ वज्रो॒ वज्रो॒ यत्यै॒ यत्यै॒ वज्र॑ इवेव॒ वज्रो॒ यत्यै॒ यत्यै॒ वज्र॑ इव । \newline
28. वज्र॑ इवेव॒ वज्रो॒ वज्र॑ इव॒ वै वा इ॑व॒ वज्रो॒ वज्र॑ इव॒ वै । \newline
29. इ॒व॒ वै वा इ॑वे व॒ वा ए॒षैषा वा इ॑वे व॒ वा ए॒षा । \newline
30. वा ए॒षैषा वै वा ए॒षा यद् यदे॒षा वै वा ए॒षा यत् । \newline
31. ए॒षा यद् यदे॒ षैषा यदभ्रि॒ रभ्रि॒र् यदे॒ षैषा यदभ्रिः॑ । \newline
32. यदभ्रि॒ रभ्रि॒र् यद् यदभ्रिः॑ । \newline
33. अभ्रि॒रभ्रिः॑ । \newline
34. अभ्रि॑ रस्य॒ स्यभ्रि॒ रभ्रि॑ रसि॒ नारि॒र् नारि॑ र॒स्यभ्रि॒ रभ्रि॑ रसि॒ नारिः॑ । \newline
35. अ॒सि॒ नारि॒र् नारि॑ रस्यसि॒ नारि॑ रस्यसि॒ नारि॑ रस्यसि॒ नारि॑रसि । \newline
36. नारि॑ रस्यसि॒ नारि॒र् नारि॑ र॒सी तीत्य॑सि॒ नारि॒र् नारि॑ र॒सीति॑ । \newline
37. अ॒सीती त्य॑स्य॒ सीत्या॑हा॒हे त्य॑स्य॒ सीत्या॑ह । \newline
38. इत्या॑हा॒हे तीत्या॑ह॒ शान्त्यै॒ शान्त्या॑ आ॒हे तीत्या॑ह॒ शान्त्यै᳚ । \newline
39. आ॒ह॒ शान्त्यै॒ शान्त्या॑ आहाह॒ शान्त्यै॒ काण्डे॑काण्डे॒ काण्डे॑काण्डे॒ शान्त्या॑ आहाह॒ शान्त्यै॒ काण्डे॑काण्डे । \newline
40. शान्त्यै॒ काण्डे॑काण्डे॒ काण्डे॑काण्डे॒ शान्त्यै॒ शान्त्यै॒ काण्डे॑काण्डे॒ वै वै काण्डे॑काण्डे॒ शान्त्यै॒ शान्त्यै॒ काण्डे॑काण्डे॒ वै । \newline
41. काण्डे॑काण्डे॒ वै वै काण्डे॑काण्डे॒ काण्डे॑काण्डे॒ वै क्रि॒यमा॑णे क्रि॒यमा॑णे॒ वै काण्डे॑काण्डे॒ काण्डे॑काण्डे॒ वै क्रि॒यमा॑णे । \newline
42. काण्डे॑काण्ड॒ इति॒ काण्डे᳚ - का॒ण्डे॒ । \newline
43. वै क्रि॒यमा॑णे क्रि॒यमा॑णे॒ वै वै क्रि॒यमा॑णे य॒ज्ञ्ं ॅय॒ज्ञ्म् क्रि॒यमा॑णे॒ वै वै क्रि॒यमा॑णे य॒ज्ञ्म् । \newline
44. क्रि॒यमा॑णे य॒ज्ञ्ं ॅय॒ज्ञ्म् क्रि॒यमा॑णे क्रि॒यमा॑णे य॒ज्ञ्ꣳ रक्षाꣳ॑सि॒ रक्षाꣳ॑सि य॒ज्ञ्म् क्रि॒यमा॑णे क्रि॒यमा॑णे य॒ज्ञ्ꣳ रक्षाꣳ॑सि । \newline
45. य॒ज्ञ्ꣳ रक्षाꣳ॑सि॒ रक्षाꣳ॑सि य॒ज्ञ्ं ॅय॒ज्ञ्ꣳ रक्षाꣳ॑सि जिघाꣳसन्ति जिघाꣳसन्ति॒ रक्षाꣳ॑सि य॒ज्ञ्ं ॅय॒ज्ञ्ꣳ रक्षाꣳ॑सि जिघाꣳसन्ति । \newline
46. रक्षाꣳ॑सि जिघाꣳसन्ति जिघाꣳसन्ति॒ रक्षाꣳ॑सि॒ रक्षाꣳ॑सि जिघाꣳसन्ति॒ परि॑लिखित॒म् परि॑लिखितम् जिघाꣳसन्ति॒ रक्षाꣳ॑सि॒ रक्षाꣳ॑सि जिघाꣳसन्ति॒ परि॑लिखितम् । \newline
47. जि॒घाꣳ॒॒स॒न्ति॒ परि॑लिखित॒म् परि॑लिखितम् जिघाꣳसन्ति जिघाꣳसन्ति॒ परि॑लिखितꣳ॒॒ रक्षो॒ रक्षः॒ परि॑लिखितम् जिघाꣳसन्ति जिघाꣳसन्ति॒ परि॑लिखितꣳ॒॒ रक्षः॑ । \newline
48. परि॑लिखितꣳ॒॒ रक्षो॒ रक्षः॒ परि॑लिखित॒म् परि॑लिखितꣳ॒॒ रक्षः॒ परि॑लिखिताः॒ परि॑लिखिता॒ रक्षः॒ परि॑लिखित॒म् परि॑लिखितꣳ॒॒ रक्षः॒ परि॑लिखिताः । \newline
49. परि॑लिखित॒मिति॒ परि॑ - लि॒खि॒त॒म् । \newline
50. रक्षः॒ परि॑लिखिताः॒ परि॑लिखिता॒ रक्षो॒ रक्षः॒ परि॑लिखिता॒ अरा॑त॒यो ऽरा॑तयः॒ परि॑लिखिता॒ रक्षो॒ रक्षः॒ परि॑लिखिता॒ अरा॑तयः । \newline
51. परि॑लिखिता॒ अरा॑त॒यो ऽरा॑तयः॒ परि॑लिखिताः॒ परि॑लिखिता॒ अरा॑तय॒ इती त्यरा॑तयः॒ परि॑लिखिताः॒ परि॑लिखिता॒ अरा॑तय॒ इति॑ । \newline
52. परि॑लिखिता॒ इति॒ परि॑ - लि॒खि॒ताः॒ । \newline
53. अरा॑तय॒ इती त्यरा॑त॒यो ऽरा॑तय॒ इत्या॑हा॒हे त्यरा॑त॒यो ऽरा॑तय॒ इत्या॑ह । \newline
54. इत्या॑हा॒हे तीत्या॑ह॒ रक्ष॑साꣳ॒॒ रक्ष॑सा मा॒हे तीत्या॑ह॒ रक्ष॑साम् । \newline
55. आ॒ह॒ रक्ष॑साꣳ॒॒ रक्ष॑सा माहाह॒ रक्ष॑सा॒ मप॑हत्या॒ अप॑हत्यै॒ रक्ष॑सा माहाह॒ रक्ष॑सा॒ मप॑हत्यै । \newline
56. रक्ष॑सा॒ मप॑हत्या॒ अप॑हत्यै॒ रक्ष॑साꣳ॒॒ रक्ष॑सा॒ मप॑हत्या इ॒द मि॒द मप॑हत्यै॒ रक्ष॑साꣳ॒॒ रक्ष॑सा॒ मप॑हत्या इ॒दम् । \newline
57. अप॑हत्या इ॒द मि॒द मप॑हत्या॒ अप॑हत्या इ॒द म॒ह म॒ह मि॒द मप॑हत्या॒ अप॑हत्या इ॒द म॒हम् । \newline
58. अप॑हत्या॒ इत्यप॑ - ह॒त्यै॒ । \newline
\pagebreak
\markright{ TS 6.2.10.2  \hfill https://www.vedavms.in \hfill}

\section{ TS 6.2.10.2 }

\textbf{TS 6.2.10.2 } \newline
\textbf{Samhita Paata} \newline

इ॒दम॒हꣳ रक्ष॑सो ग्री॒वा अपि॑ कृन्तामि॒ यो᳚ऽस्मान् द्वेष्टि॒ यं च॑ व॒यं द्वि॒ष्म इत्या॑ह॒ द्वौ वाव पुरु॑षौ॒ यं चै॒व द्वेष्टि॒ यश्चै॑नं॒ द्वेष्टि॒ तयो॑रे॒वान॑न्तरायं ग्री॒वाः कृ॑न्तति दि॒वे त्वा॒ऽन्तरि॑क्षाय त्वा पृथि॒व्यै त्वेत्या॑है॒भ्य ए॒वैनां᳚ ॅलो॒केभ्यः॒ प्रोक्ष॑ति प॒रस्ता॑द॒र्वाचीं॒ प्रोक्ष॑ति॒ तस्मा᳚त्- [  ] \newline

\textbf{Pada Paata} \newline

इ॒दम् । अ॒हम् । रक्ष॑सः । ग्री॒वाः । अपीति॑ । कृ॒न्ता॒मि॒ । यः । अ॒स्मान् । द्वेष्टि॑ । यम् । च॒ । व॒यम् । द्वि॒ष्मः । इति॑ । आ॒ह॒ । द्वौ । वाव । पुरु॑षौ । यम् । च॒ । ए॒व । द्वेष्टि॑ । यः । च॒ । ए॒न॒म् । द्वेष्टि॑ । तयोः᳚ । ए॒व । अन॑न्तराय॒मित्यन॑न्तः - आ॒य॒म् । ग्री॒वाः । कृ॒न्त॒ति॒ । दि॒वे । त्वा॒ । अ॒न्तरि॑क्षाय । त्वा॒ । पृ॒थि॒व्यै । त्वा॒ । इति॑ । आ॒ह॒ । ए॒भ्यः । ए॒व । ए॒ना॒म् । लो॒केभ्यः॑ । प्रेति॑ । उ॒क्ष॒ति॒ । प॒रस्ता᳚त् । अ॒र्वाची᳚म् । प्रेति॑ । उ॒क्ष॒ति॒ । तस्मा᳚त् ।  \newline


\textbf{Krama Paata} \newline

इ॒दम॒हम् । अ॒हꣳ रक्ष॑सः । रक्ष॑सो ग्री॒वाः । ग्री॒वा अपि॑ । अपि॑ कृन्तामि । कृ॒न्ता॒मि॒ यः । यो᳚ऽस्मान् । अ॒स्मान् द्वेष्टि॑ । द्वेष्टि॒ यम् । यम् च॑ । च॒ व॒यम् । व॒यम् द्वि॒ष्मः । द्वि॒ष्म इति॑ । इत्या॑ह । आ॒ह॒ द्वौ । द्वौ वाव । वाव पुरु॑षौ । पुरु॑षौ॒ यम् । यम् च॑ । चै॒ व । ए॒व द्वेष्टि॑ । द्वेष्टि॒ यः । यश्च॑ । चै॒न॒म् । ए॒न॒म् द्वेष्टि॑ । द्वेष्टि॒ तयोः᳚ । तयो॑रे॒व । ए॒वान॑न्तरायम् । अन॑न्तरायम् ग्री॒वाः । अन॑न्तराय॒मित्यन॑न्तः - आ॒य॒म् । ग्री॒वाः कृ॑न्तति । कृ॒न्त॒ति॒ दि॒वे । दि॒वे त्वा᳚ । त्वा॒ऽन्तरि॑क्षाय । अ॒न्तरि॑क्षाय त्वा । त्वा॒ पृ॒थि॒व्यै । पृ॒थि॒व्यै त्वा᳚ । त्वेति॑ । इत्या॑ह । आ॒है॒भ्यः । ए॒भ्य ए॒व । ए॒वैना᳚म् । ए॒ना॒म् ॅलो॒केभ्यः॑ । लो॒केभ्यः॒ प्र । प्रोक्ष॑ति । उ॒क्ष॒ति॒ प॒रस्ता᳚त् । प॒रस्ता॑द॒र्वाची᳚म् । अ॒र्वाची॒म् प्र । प्रोक्ष॑ति । उ॒क्ष॒ति॒ तस्मा᳚त् । तस्मा᳚त् प॒रस्ता᳚त् \newline

\textbf{Jatai Paata} \newline

1. इ॒द म॒ह म॒ह मि॒द मि॒द म॒हम् । \newline
2. अ॒हꣳ रक्ष॑सो॒ रक्ष॑सो॒ ऽह म॒हꣳ रक्ष॑सः । \newline
3. रक्ष॑सो ग्री॒वा ग्री॒वा रक्ष॑सो॒ रक्ष॑सो ग्री॒वाः । \newline
4. ग्री॒वा अप्यपि॑ ग्री॒वा ग्री॒वा अपि॑ । \newline
5. अपि॑ कृन्तामि कृन्ता॒ म्यप्यपि॑ कृन्तामि । \newline
6. कृ॒न्ता॒मि॒ यो यः कृ॑न्तामि कृन्तामि॒ यः । \newline
7. यो᳚ ऽस्मा न॒स्मान्. यो यो᳚ ऽस्मान् । \newline
8. अ॒स्मान् द्वेष्टि॒ द्वेष्ट्य॒स्मा न॒स्मान् द्वेष्टि॑ । \newline
9. द्वेष्टि॒ यं ॅयम् द्वेष्टि॒ द्वेष्टि॒ यम् । \newline
10. यम् च॑ च॒ यं ॅयम् च॑ । \newline
11. च॒ व॒यं ॅव॒यम् च॑ च व॒यम् । \newline
12. व॒यम् द्वि॒ष्मो द्वि॒ष्मो व॒यं ॅव॒यम् द्वि॒ष्मः । \newline
13. द्वि॒ष्म इतीति॑ द्वि॒ष्मो द्वि॒ष्म इति॑ । \newline
14. इत्या॑हा॒हे तीत्या॑ह । \newline
15. आ॒ह॒ द्वौ द्वा वा॑हाह॒ द्वौ । \newline
16. द्वौ वाव वाव द्वौ द्वौ वाव । \newline
17. वाव पुरु॑षौ॒ पुरु॑षौ॒ वाव वाव पुरु॑षौ । \newline
18. पुरु॑षौ॒ यं ॅयम् पुरु॑षौ॒ पुरु॑षौ॒ यम् । \newline
19. यम् च॑ च॒ यं ॅयम् च॑ । \newline
20. चै॒वैव च॑ चै॒व । \newline
21. ए॒व द्वेष्टि॒ द्वेष्ट्ये॒वैव द्वेष्टि॑ । \newline
22. द्वेष्टि॒ यो यो द्वेष्टि॒ द्वेष्टि॒ यः । \newline
23. यश्च॑ च॒ यो यश्च॑ । \newline
24. चै॒न॒ मे॒न॒म् च॒ चै॒न॒म् । \newline
25. ए॒न॒म् द्वेष्टि॒ द्वेष्ट्ये॑न मेन॒म् द्वेष्टि॑ । \newline
26. द्वेष्टि॒ तयो॒ स्तयो॒र् द्वेष्टि॒ द्वेष्टि॒ तयोः᳚ । \newline
27. तयो॑ रे॒वैव तयो॒ स्तयो॑ रे॒व । \newline
28. ए॒वा न॑न्तराय॒ मन॑न्तराय मे॒वैवा न॑न्तरायम् । \newline
29. अन॑न्तरायम् ग्री॒वा ग्री॒वा अन॑न्तराय॒ मन॑न्तरायम् ग्री॒वाः । \newline
30. अन॑न्तराय॒मित्यन॑न्तः - आ॒य॒म् । \newline
31. ग्री॒वाः कृ॑न्तति कृन्तति ग्री॒वा ग्री॒वाः कृ॑न्तति । \newline
32. कृ॒न्त॒ति॒ दि॒वे दि॒वे कृ॑न्तति कृन्तति दि॒वे । \newline
33. दि॒वे त्वा᳚ त्वा दि॒वे दि॒वे त्वा᳚ । \newline
34. त्वा॒ ऽन्तरि॑क्षाया॒ न्तरि॑क्षाय त्वा त्वा॒ ऽन्तरि॑क्षाय । \newline
35. अ॒न्तरि॑क्षाय त्वा त्वा॒ ऽन्तरि॑क्षाया॒ न्तरि॑क्षाय त्वा । \newline
36. त्वा॒ पृ॒थि॒व्यै पृ॑थि॒व्यै त्वा᳚ त्वा पृथि॒व्यै । \newline
37. पृ॒थि॒व्यै त्वा᳚ त्वा पृथि॒व्यै पृ॑थि॒व्यै त्वा᳚ । \newline
38. त्वेतीति॑ त्वा॒ त्वेति॑ । \newline
39. इत्या॑हा॒हे तीत्या॑ह । \newline
40. आ॒है॒भ्य ए॒भ्य आ॑हा है॒भ्यः । \newline
41. ए॒भ्य ए॒वैवैभ्य ए॒भ्य ए॒व । \newline
42. ए॒वैना॑ मेना मे॒वैवैना᳚म् । \newline
43. ए॒ना॒म् ॅलो॒केभ्यो॑ लो॒केभ्य॑ एना मेनाम् ॅलो॒केभ्यः॑ । \newline
44. लो॒केभ्यः॒ प्र प्र लो॒केभ्यो॑ लो॒केभ्यः॒ प्र । \newline
45. प्रोक्ष॑ त्युक्षति॒ प्र प्रोक्ष॑ति । \newline
46. उ॒क्ष॒ति॒ प॒रस्ता᳚त् प॒रस्ता॑ दुक्ष त्युक्षति प॒रस्ता᳚त् । \newline
47. प॒रस्ता॑ द॒र्वाची॑ म॒र्वाची᳚म् प॒रस्ता᳚त् प॒रस्ता॑ द॒र्वाची᳚म् । \newline
48. अ॒र्वाची॒म् प्र प्रार्वाची॑ म॒र्वाची॒म् प्र । \newline
49. प्रोक्ष॑ त्युक्षति॒ प्र प्रोक्ष॑ति । \newline
50. उ॒क्ष॒ति॒ तस्मा॒त् तस्मा॑ दुक्ष त्युक्षति॒ तस्मा᳚त् । \newline
51. तस्मा᳚त् प॒रस्ता᳚त् प॒रस्ता॒त् तस्मा॒त् तस्मा᳚त् प॒रस्ता᳚त् । \newline

\textbf{Ghana Paata } \newline

1. इ॒द म॒ह म॒ह मि॒द मि॒द म॒हꣳ रक्ष॑सो॒ रक्ष॑सो॒ ऽह मि॒द मि॒द म॒हꣳ रक्ष॑सः । \newline
2. अ॒हꣳ रक्ष॑सो॒ रक्ष॑सो॒ ऽह म॒हꣳ रक्ष॑सो ग्री॒वा ग्री॒वा रक्ष॑सो॒ ऽह म॒हꣳ रक्ष॑सो ग्री॒वाः । \newline
3. रक्ष॑सो ग्री॒वा ग्री॒वा रक्ष॑सो॒ रक्ष॑सो ग्री॒वा अप्यपि॑ ग्री॒वा रक्ष॑सो॒ रक्ष॑सो ग्री॒वा अपि॑ । \newline
4. ग्री॒वा अप्यपि॑ ग्री॒वा ग्री॒वा अपि॑ कृन्तामि कृन्ता॒ म्यपि॑ ग्री॒वा ग्री॒वा अपि॑ कृन्तामि । \newline
5. अपि॑ कृन्तामि कृन्ता॒ म्यप्यपि॑ कृन्तामि॒ यो यः कृ॑न्ता॒ म्यप्यपि॑ कृन्तामि॒ यः । \newline
6. कृ॒न्ता॒मि॒ यो यः कृ॑न्तामि कृन्तामि॒ यो᳚ ऽस्मा न॒स्मान्. यः कृ॑न्तामि कृन्तामि॒ यो᳚ ऽस्मान् । \newline
7. यो᳚ ऽस्मा न॒स्मान्. यो यो᳚ ऽस्मान् द्वेष्टि॒ द्वेष्ट्य॒स्मान्. यो यो᳚ ऽस्मान् द्वेष्टि॑ । \newline
8. अ॒स्मान् द्वेष्टि॒ द्वेष्ट्य॒स्मा न॒स्मान् द्वेष्टि॒ यं ॅयम् द्वेष्ट्य॒स्मा न॒स्मान् द्वेष्टि॒ यम् । \newline
9. द्वेष्टि॒ यं ॅयम् द्वेष्टि॒ द्वेष्टि॒ यम् च॑ च॒ यम् द्वेष्टि॒ द्वेष्टि॒ यम् च॑ । \newline
10. यम् च॑ च॒ यं ॅयम् च॑ व॒यं ॅव॒यम् च॒ यं ॅयम् च॑ व॒यम् । \newline
11. च॒ व॒यं ॅव॒यम् च॑ च व॒यम् द्वि॒ष्मो द्वि॒ष्मो व॒यम् च॑ च व॒यम् द्वि॒ष्मः । \newline
12. व॒यम् द्वि॒ष्मो द्वि॒ष्मो व॒यं ॅव॒यम् द्वि॒ष्म इतीति॑ द्वि॒ष्मो व॒यं ॅव॒यम् द्वि॒ष्म इति॑ । \newline
13. द्वि॒ष्म इतीति॑ द्वि॒ष्मो द्वि॒ष्म इत्या॑हा॒ हेति॑ द्वि॒ष्मो द्वि॒ष्म इत्या॑ह । \newline
14. इत्या॑हा॒हे तीत्या॑ह॒ द्वौ द्वा वा॒हे तीत्या॑ह॒ द्वौ । \newline
15. आ॒ह॒ द्वौ द्वा वा॑हाह॒ द्वौ वाव वाव द्वा वा॑हाह॒ द्वौ वाव । \newline
16. द्वौ वाव वाव द्वौ द्वौ वाव पुरु॑षौ॒ पुरु॑षौ॒ वाव द्वौ द्वौ वाव पुरु॑षौ । \newline
17. वाव पुरु॑षौ॒ पुरु॑षौ॒ वाव वाव पुरु॑षौ॒ यं ॅयम् पुरु॑षौ॒ वाव वाव पुरु॑षौ॒ यम् । \newline
18. पुरु॑षौ॒ यं ॅयम् पुरु॑षौ॒ पुरु॑षौ॒ यम् च॑ च॒ यम् पुरु॑षौ॒ पुरु॑षौ॒ यम् च॑ । \newline
19. यम् च॑ च॒ यं ॅयम् चै॒वैव च॒ यं ॅयम् चै॒व । \newline
20. चै॒वैव च॑ चै॒व द्वेष्टि॒ द्वेष्ट्ये॒व च॑ चै॒व द्वेष्टि॑ । \newline
21. ए॒व द्वेष्टि॒ द्वेष्ट्ये॒ वैव द्वेष्टि॒ यो यो द्वेष्ट्ये॒ वैव द्वेष्टि॒ यः । \newline
22. द्वेष्टि॒ यो यो द्वेष्टि॒ द्वेष्टि॒ यश्च॑ च॒ यो द्वेष्टि॒ द्वेष्टि॒ यश्च॑ । \newline
23. यश्च॑ च॒ यो यश्चै॑न मेनम् च॒ यो यश्चै॑नम् । \newline
24. चै॒न॒ मे॒न॒म् च॒ चै॒न॒म् द्वेष्टि॒ द्वेष्ट्ये॑नम् च चैन॒म् द्वेष्टि॑ । \newline
25. ए॒न॒म् द्वेष्टि॒ द्वेष्ट्ये॑न मेन॒म् द्वेष्टि॒ तयो॒ स्तयो॒र् द्वेष्ट्ये॑न मेन॒म् द्वेष्टि॒ तयोः᳚ । \newline
26. द्वेष्टि॒ तयो॒ स्तयो॒र् द्वेष्टि॒ द्वेष्टि॒ तयो॑ रे॒वैव तयो॒र् द्वेष्टि॒ द्वेष्टि॒ तयो॑ रे॒व । \newline
27. तयो॑ रे॒वैव तयो॒ स्तयो॑ रे॒वान॑न्तराय॒ मन॑न्तराय मे॒व तयो॒ स्तयो॑ रे॒वा न॑न्तरायम् । \newline
28. ए॒वान॑न्तराय॒ मन॑न्तराय मे॒वै वान॑न्तरायम् ग्री॒वा ग्री॒वा अन॑न्तराय मे॒वै वान॑न्तरायम् ग्री॒वाः । \newline
29. अन॑न्तरायम् ग्री॒वा ग्री॒वा अन॑न्तराय॒ मन॑न्तरायम् ग्री॒वाः कृ॑न्तति कृन्तति ग्री॒वा अन॑न्तराय॒ मन॑न्तरायम् ग्री॒वाः कृ॑न्तति । \newline
30. अन॑न्तराय॒मित्यन॑न्तः - आ॒य॒म् । \newline
31. ग्री॒वाः कृ॑न्तति कृन्तति ग्री॒वा ग्री॒वाः कृ॑न्तति दि॒वे दि॒वे कृ॑न्तति ग्री॒वा ग्री॒वाः कृ॑न्तति दि॒वे । \newline
32. कृ॒न्त॒ति॒ दि॒वे दि॒वे कृ॑न्तति कृन्तति दि॒वे त्वा᳚ त्वा दि॒वे कृ॑न्तति कृन्तति दि॒वे त्वा᳚ । \newline
33. दि॒वे त्वा᳚ त्वा दि॒वे दि॒वे त्वा॒ ऽन्तरि॑क्षाया॒ न्तरि॑क्षाय त्वा दि॒वे दि॒वे त्वा॒ ऽन्तरि॑क्षाय । \newline
34. त्वा॒ ऽन्तरि॑क्षाया॒ न्तरि॑क्षाय त्वा त्वा॒ ऽन्तरि॑क्षाय त्वा त्वा॒ ऽन्तरि॑क्षाय त्वा त्वा॒ ऽन्तरि॑क्षाय त्वा । \newline
35. अ॒न्तरि॑क्षाय त्वा त्वा॒ ऽन्तरि॑क्षाया॒ न्तरि॑क्षाय त्वा पृथि॒व्यै पृ॑थि॒व्यै त्वा॒ ऽन्तरि॑क्षाया॒ न्तरि॑क्षाय त्वा पृथि॒व्यै । \newline
36. त्वा॒ पृ॒थि॒व्यै पृ॑थि॒व्यै त्वा᳚ त्वा पृथि॒व्यै त्वा᳚ त्वा पृथि॒व्यै त्वा᳚ त्वा पृथि॒व्यै त्वा᳚ । \newline
37. पृ॒थि॒व्यै त्वा᳚ त्वा पृथि॒व्यै पृ॑थि॒व्यै त्वेतीति॑ त्वा पृथि॒व्यै पृ॑थि॒व्यै त्वेति॑ । \newline
38. त्वेतीति॑ त्वा॒ त्वेत्या॑हा॒ हेति॑ त्वा॒ त्वेत्या॑ह । \newline
39. इत्या॑हा॒हे तीत्या॑ है॒भ्य ए॒भ्य आ॒हे तीत्या॑ है॒भ्यः । \newline
40. आ॒है॒भ्य ए॒भ्य आ॑हा है॒भ्य ए॒वै वैभ्य आ॑हा है॒भ्य ए॒व । \newline
41. ए॒भ्य ए॒वै वैभ्य ए॒भ्य ए॒वैना॑ मेना मे॒वैभ्य ए॒भ्य ए॒वैना᳚म् । \newline
42. ए॒वैना॑ मेना मे॒वै वैना᳚म् ॅलो॒केभ्यो॑ लो॒केभ्य॑ एना मे॒वै वैना᳚म् ॅलो॒केभ्यः॑ । \newline
43. ए॒ना॒म् ॅलो॒केभ्यो॑ लो॒केभ्य॑ एना मेनाम् ॅलो॒केभ्यः॒ प्र प्र लो॒केभ्य॑ एना मेनाम् ॅलो॒केभ्यः॒ प्र । \newline
44. लो॒केभ्यः॒ प्र प्र लो॒केभ्यो॑ लो॒केभ्यः॒ प्रोक्ष॑ त्युक्षति॒ प्र लो॒केभ्यो॑ लो॒केभ्यः॒ प्रोक्ष॑ति । \newline
45. प्रोक्ष॑ त्युक्षति॒ प्र प्रोक्ष॑ति प॒रस्ता᳚त् प॒रस्ता॑ दुक्षति॒ प्र प्रोक्ष॑ति प॒रस्ता᳚त् । \newline
46. उ॒क्ष॒ति॒ प॒रस्ता᳚त् प॒रस्ता॑ दुक्ष त्युक्षति प॒रस्ता॑ द॒र्वाची॑ म॒र्वाची᳚म् प॒रस्ता॑ दुक्ष त्युक्षति प॒रस्ता॑ द॒र्वाची᳚म् । \newline
47. प॒रस्ता॑ द॒र्वाची॑ म॒र्वाची᳚म् प॒रस्ता᳚त् प॒रस्ता॑ द॒र्वाची॒म् प्र प्रार्वाची᳚म् प॒रस्ता᳚त् प॒रस्ता॑ द॒र्वाची॒म् प्र । \newline
48. अ॒र्वाची॒म् प्र प्रार्वाची॑ म॒र्वाची॒म् प्रोक्ष॑ त्युक्षति॒ प्रार्वाची॑ म॒र्वाची॒म् प्रोक्ष॑ति । \newline
49. प्रोक्ष॑ त्युक्षति॒ प्र प्रोक्ष॑ति॒ तस्मा॒त् तस्मा॑ दुक्षति॒ प्र प्रोक्ष॑ति॒ तस्मा᳚त् । \newline
50. उ॒क्ष॒ति॒ तस्मा॒त् तस्मा॑ दुक्ष त्युक्षति॒ तस्मा᳚त् प॒रस्ता᳚त् प॒रस्ता॒त् तस्मा॑ दुक्ष त्युक्षति॒ तस्मा᳚त् प॒रस्ता᳚त् । \newline
51. तस्मा᳚त् प॒रस्ता᳚त् प॒रस्ता॒त् तस्मा॒त् तस्मा᳚त् प॒रस्ता॑ द॒र्वाची॑ म॒र्वाची᳚म् प॒रस्ता॒त् तस्मा॒त् तस्मा᳚त् प॒रस्ता॑ द॒र्वाची᳚म् । \newline
\pagebreak
\markright{ TS 6.2.10.3  \hfill https://www.vedavms.in \hfill}

\section{ TS 6.2.10.3 }

\textbf{TS 6.2.10.3 } \newline
\textbf{Samhita Paata} \newline

प॒रस्ता॑द॒र्वाचीं᳚ मनु॒ष्या॑ ऊर्ज॒मुप॑ जीवन्ति क्रू॒रमि॑व॒ वा ए॒तत् क॑रोति॒ यत् खन॑त्य॒पोऽव॑ नयति॒ शान्त्यै॒ यव॑मती॒रव॑ नय॒त्यूर्ग्वै॑ यव॒ ऊर्गु॑दु॒बंर॑ ऊ॒र्जैवोर्जꣳ॒॒ सम॑र्द्धयति॒ यज॑मानेन॒ सम्मि॒तौदु॑बंरी भवति॒ यावा॑ने॒व यज॑मान॒स्ताव॑ती-मे॒वास्मि॒-न्नूर्जं॑ दधाति पितृ॒णाꣳ सद॑नम॒सीति॑ ब॒र्॒.हिरव॑ स्तृणाति पितृदेव॒त्या᳚(1॒)ꣳ॒- [  ] \newline

\textbf{Pada Paata} \newline

प॒रस्ता᳚त् । अ॒र्वाची᳚म् । म॒नु॒ष्याः᳚ । ऊर्ज᳚म् । उपेति॑ । जी॒व॒न्ति॒ । क्रू॒रम् । इ॒व॒ । वै । ए॒तत् । क॒रो॒ति॒ । यत् । खन॑ति । अ॒पः । अवेति॑ । न॒य॒ति॒ । शान्त्यै᳚ । यव॑मती॒रिति॒ यव॑ - म॒तीः॒ । अवेति॑ । न॒य॒ति॒ । ऊर्क् । वै । यवः॑ । ऊर्क् । उ॒दु॒म्बरः॑ । ऊ॒र्जा । ए॒व । ऊर्ज᳚म् । समिति॑ । अ॒द्‌र्ध॒य॒ति॒ । यज॑मानेन । सम्मि॒तेति॒ सं - मि॒ता॒ । औदु॑बंरी । भ॒व॒ति॒ । यावान्॑ । ए॒व । यज॑मानः । ताव॑तीम् । ए॒व । अ॒स्मि॒न्न् । ऊर्ज᳚म् । द॒धा॒ति॒ । पि॒तृ॒णाम् । सद॑नम् । अ॒सि॒ । इति॑ । ब॒र्॒.हिः । अवेति॑ । स्तृ॒णा॒ति॒ । पि॒तृ॒दे॒व॒त्य॑मिति॑ पितृ - दे॒व॒त्य᳚म् ।  \newline


\textbf{Krama Paata} \newline

प॒रस्ता॑द॒र्वाची᳚म् । अ॒र्वाची᳚म् मनु॒ष्याः᳚ । म॒नु॒ष्या॑ ऊर्ज᳚म् । ऊर्ज॒मुप॑ । उप॑ जीवन्ति । जी॒व॒न्ति॒ क्रू॒रम् । क्रू॒रमि॑व । इ॒व॒ वै । वा ए॒तत् । ए॒तत् क॑रोति । क॒रो॒ति॒ यत् । यत् खन॑ति । खन॑त्य॒पः । अ॒पोऽव॑ । अव॑ नयति । न॒य॒ति॒ शान्त्यै᳚ । शान्त्यै॒ यव॑मतीः । यव॑मती॒रव॑ । यव॑मती॒रिति॒ यव॑ - म॒तीः॒ । अव॑ नयति । न॒य॒त्यूर्क् । ऊर्ग् वै । वै यवः॑ । यव॒ ऊर्क् । ऊर्गु॑दु॒म्बरः॑ । उ॒दु॒म्बर॑ ऊ॒र्जा । ऊ॒र्जैव । ए॒वोर्ज᳚म् । ऊर्जꣳ॒॒ सम् । सम॑र्द्धयति । अ॒र्द्ध॒य॒ति॒ यज॑मानेन । यज॑मानेन॒ सम्मि॑ता । सम्मि॒तौदु॑म्बरी । सम्मि॒तेति॒ सम् - मि॒ता॒ । औदु॑म्बरी भवति । भ॒व॒ति॒ यावान्॑ । यावा॑ने॒व । ए॒व यज॑मानः । यज॑मान॒स्ताव॑तीम् । ताव॑तीमे॒व । ए॒वास्मिन्न्॑ । अ॒स्मि॒न्नूर्ज᳚म् । ऊर्ज॑म् दधाति । द॒धा॒ति॒ पि॒तृ॒णाम् । पि॒तृ॒णाꣳ सद॑नम् । सद॑नमसि । अ॒सीति॑ । इति॑ ब॒र्.॒हिः । ब॒र्.॒हिरव॑ । अव॑ स्तृणाति । स्तृ॒णा॒ति॒ पि॒तृ॒दे॒व॒त्य᳚म् । पि॒तृ॒दे॒व॒त्यꣳ॑ हि । पि॒तृ॒दे॒व॒त्य॑मिति॑ पितृ - दे॒व॒त्य᳚म् \newline

\textbf{Jatai Paata} \newline

1. प॒रस्ता॑ द॒र्वाची॑ म॒र्वाची᳚म् प॒रस्ता᳚त् प॒रस्ता॑ द॒र्वाची᳚म् । \newline
2. अ॒र्वाची᳚म् मनु॒ष्या॑ मनु॒ष्या॑ अ॒र्वाची॑ म॒र्वाची᳚म् मनु॒ष्याः᳚ । \newline
3. म॒नु॒ष्या॑ ऊर्ज॒ मूर्ज॑म् मनु॒ष्या॑ मनु॒ष्या॑ ऊर्ज᳚म् । \newline
4. ऊर्ज॒ मुपोपोर्ज॒ मूर्ज॒ मुप॑ । \newline
5. उप॑ जीवन्ति जीव॒ न्त्युपोप॑ जीवन्ति । \newline
6. जी॒व॒न्ति॒ क्रू॒रम् क्रू॒रम् जी॑वन्ति जीवन्ति क्रू॒रम् । \newline
7. क्रू॒र मि॑वेव क्रू॒रम् क्रू॒र मि॑व । \newline
8. इ॒व॒ वै वा इ॑वेव॒ वै । \newline
9. वा ए॒त दे॒तद् वै वा ए॒तत् । \newline
10. ए॒तत् क॑रोति करो त्ये॒त दे॒तत् क॑रोति । \newline
11. क॒रो॒ति॒ यद् यत् क॑रोति करोति॒ यत् । \newline
12. यत् खन॑ति॒ खन॑ति॒ यद् यत् खन॑ति । \newline
13. खन॑ त्य॒पो॑ ऽपः खन॑ति॒ खन॑ त्य॒पः । \newline
14. अ॒पो ऽवावा॒पो॑ ऽपो ऽव॑ । \newline
15. अव॑ नयति नय॒ त्यवाव॑ नयति । \newline
16. न॒य॒ति॒ शान्त्यै॒ शान्त्यै॑ नयति नयति॒ शान्त्यै᳚ । \newline
17. शान्त्यै॒ यव॑मती॒र् यव॑मतीः॒ शान्त्यै॒ शान्त्यै॒ यव॑मतीः । \newline
18. यव॑मती॒ रवाव॒ यव॑मती॒र् यव॑मती॒ रव॑ । \newline
19. यव॑मती॒रिति॒ यव॑ - म॒तीः॒ । \newline
20. अव॑ नयति नय॒ त्यवाव॑ नयति । \newline
21. न॒य॒ त्यूर् गूर्ङ् न॑यति नय॒ त्यूर्क् । \newline
22. ऊर्ग् वै वा ऊर् गूर्ग् वै । \newline
23. वै यवो॒ यवो॒ वै वै यवः॑ । \newline
24. यव॒ ऊर् गूर्ग् यवो॒ यव॒ ऊर्क् । \newline
25. ऊर् गु॑दु॒म्बर॑ उदु॒म्बर॒ ऊर् गूर् गु॑दु॒म्बरः॑ । \newline
26. उ॒दु॒म्बर॑ ऊ॒र्जोर्जो दु॒म्बर॑ उदु॒म्बर॑ ऊ॒र्जा । \newline
27. ऊ॒र्जै वैवोर्जोर् जैव । \newline
28. ए॒वोर्ज॒ मूर्ज॑ मे॒वैवोर्ज᳚म् । \newline
29. ऊर्जꣳ॒॒ सꣳ स मूर्ज॒ मूर्जꣳ॒॒ सम् । \newline
30. स म॑र्द्धय त्यर्द्धयति॒ सꣳ स म॑र्द्धयति । \newline
31. अ॒र्द्ध॒य॒ति॒ यज॑मानेन॒ यज॑मानेना र्द्धय त्यर्द्धयति॒ यज॑मानेन । \newline
32. यज॑मानेन॒ सम्मि॑ता॒ सम्मि॑ता॒ यज॑मानेन॒ यज॑मानेन॒ सम्मि॑ता । \newline
33. सम्मि॒ तौदुं॑ब॒र् यौदुं॑बरी॒ सम्मि॑ता॒ सम्मि॒ तौदुं॑बरी । \newline
34. सम्मि॒तेति॒ सं - मि॒ता॒ । \newline
35. औदुं॑बरी भवति भव॒ त्यौदुं॑ब॒र् यौदुं॑बरी भवति । \newline
36. भ॒व॒ति॒ यावा॒न्॒. यावा᳚न् भवति भवति॒ यावान्॑ । \newline
37. यावा॑ ने॒वैव यावा॒न्॒. यावा॑ ने॒व । \newline
38. ए॒व यज॑मानो॒ यज॑मान ए॒वैव यज॑मानः । \newline
39. यज॑मान॒ स्ताव॑ती॒म् ताव॑तीं॒ ॅयज॑मानो॒ यज॑मान॒ स्ताव॑तीम् । \newline
40. ताव॑ती मे॒वैव ताव॑ती॒म् ताव॑ती मे॒व । \newline
41. ए॒वास्मि॑न् नस्मिन् ने॒वैवास्मिन्न्॑ । \newline
42. अ॒स्मि॒न् नूर्ज॒ मूर्ज॑ मस्मिन् नस्मि॒न् नूर्ज᳚म् । \newline
43. ऊर्ज॑म् दधाति दधा॒ त्यूर्ज॒ मूर्ज॑म् दधाति । \newline
44. द॒धा॒ति॒ पि॒तृ॒णाम् पि॑तृ॒णाम् द॑धाति दधाति पितृ॒णाम् । \newline
45. पि॒तृ॒णाꣳ सद॑नꣳ॒॒ सद॑नम् पितृ॒णाम् पि॑तृ॒णाꣳ सद॑नम् । \newline
46. सद॑न मस्यसि॒ सद॑नꣳ॒॒ सद॑न मसि । \newline
47. अ॒सीती त्य॑स्य॒सीति॑ । \newline
48. इति॑ ब॒र्॒.हिर् ब॒र्॒.हि रितीति॑ ब॒र्॒.हिः । \newline
49. ब॒र्॒.हि रवाव॑ ब॒र्॒.हिर् ब॒र्॒.हि रव॑ । \newline
50. अव॑ स्तृणाति स्तृणा॒ त्यवाव॑ स्तृणाति । \newline
51. स्तृ॒णा॒ति॒ पि॒तृ॒दे॒व॒त्य॑म् पितृदेव॒त्यꣳ॑ स्तृणाति स्तृणाति पितृदेव॒त्य᳚म् । \newline
52. पि॒तृ॒दे॒व॒त्यꣳ॑ हि हि पि॑तृदेव॒त्य॑म् पितृदेव॒त्यꣳ॑ हि । \newline
53. पि॒तृ॒दे॒व॒त्य॑मिति॑ पितृ - दे॒व॒त्य᳚म् । \newline

\textbf{Ghana Paata } \newline

1. प॒रस्ता॑ द॒र्वाची॑ म॒र्वाची᳚म् प॒रस्ता᳚त् प॒रस्ता॑ द॒र्वाची᳚म् मनु॒ष्या॑ मनु॒ष्या॑ अ॒र्वाची᳚म् प॒रस्ता᳚त् प॒रस्ता॑ द॒र्वाची᳚म् मनु॒ष्याः᳚ । \newline
2. अ॒र्वाची᳚म् मनु॒ष्या॑ मनु॒ष्या॑ अ॒र्वाची॑ म॒र्वाची᳚म् मनु॒ष्या॑ ऊर्ज॒ मूर्ज॑म् मनु॒ष्या॑ अ॒र्वाची॑ म॒र्वाची᳚म् मनु॒ष्या॑ ऊर्ज᳚म् । \newline
3. म॒नु॒ष्या॑ ऊर्ज॒ मूर्ज॑म् मनु॒ष्या॑ मनु॒ष्या॑ ऊर्ज॒ मुपो पोर्ज॑म् मनु॒ष्या॑ मनु॒ष्या॑ ऊर्ज॒ मुप॑ । \newline
4. ऊर्ज॒ मुपो पोर्ज॒ मूर्ज॒ मुप॑ जीवन्ति जीव॒ न्त्युपोर्ज॒ मूर्ज॒ मुप॑ जीवन्ति । \newline
5. उप॑ जीवन्ति जीव॒ न्त्युपोप॑ जीवन्ति क्रू॒रम् क्रू॒रम् जी॑व॒ न्त्युपोप॑ जीवन्ति क्रू॒रम् । \newline
6. जी॒व॒न्ति॒ क्रू॒रम् क्रू॒रम् जी॑वन्ति जीवन्ति क्रू॒र मि॑वेव क्रू॒रम् जी॑वन्ति जीवन्ति क्रू॒र मि॑व । \newline
7. क्रू॒र मि॑वेव क्रू॒रम् क्रू॒र मि॑व॒ वै वा इ॑व क्रू॒रम् क्रू॒र मि॑व॒ वै । \newline
8. इ॒व॒ वै वा इ॑वेव॒ वा ए॒त दे॒तद् वा इ॑वेव॒ वा ए॒तत् । \newline
9. वा ए॒त दे॒तद् वै वा ए॒तत् क॑रोति करो त्ये॒तद् वै वा ए॒तत् क॑रोति । \newline
10. ए॒तत् क॑रोति करो त्ये॒त दे॒तत् क॑रोति॒ यद् यत् क॑रो त्ये॒त दे॒तत् क॑रोति॒ यत् । \newline
11. क॒रो॒ति॒ यद् यत् क॑रोति करोति॒ यत् खन॑ति॒ खन॑ति॒ यत् क॑रोति करोति॒ यत् खन॑ति । \newline
12. यत् खन॑ति॒ खन॑ति॒ यद् यत् खन॑ त्य॒पो॑ ऽपः खन॑ति॒ यद् यत् खन॑ त्य॒पः । \newline
13. खन॑ त्य॒पो॑ ऽपः खन॑ति॒ खन॑ त्य॒पो ऽवावा॒पः खन॑ति॒ खन॑ त्य॒पो ऽव॑ । \newline
14. अ॒पो ऽवावा॒पो॑ ऽपो ऽव॑ नयति नय॒ त्यवा॒पो॑ ऽपो ऽव॑ नयति । \newline
15. अव॑ नयति नय॒ त्यवाव॑ नयति॒ शान्त्यै॒ शान्त्यै॑ नय॒ त्यवाव॑ नयति॒ शान्त्यै᳚ । \newline
16. न॒य॒ति॒ शान्त्यै॒ शान्त्यै॑ नयति नयति॒ शान्त्यै॒ यव॑मती॒र् यव॑मतीः॒ शान्त्यै॑ नयति नयति॒ शान्त्यै॒ यव॑मतीः । \newline
17. शान्त्यै॒ यव॑मती॒र् यव॑मतीः॒ शान्त्यै॒ शान्त्यै॒ यव॑मती॒ रवाव॒ यव॑मतीः॒ शान्त्यै॒ शान्त्यै॒ यव॑मती॒ रव॑ । \newline
18. यव॑मती॒ रवाव॒ यव॑मती॒र् यव॑मती॒ रव॑ नयति नय॒ त्यव॒ यव॑मती॒र् यव॑मती॒ रव॑ नयति । \newline
19. यव॑मती॒रिति॒ यव॑ - म॒तीः॒ । \newline
20. अव॑ नयति नय॒ त्यवाव॑ नय॒त्यूर् गूर्ङ् न॑य॒ त्यवाव॑ नय॒त्यूर्क् । \newline
21. न॒य॒त्यूर् गूर्ङ् न॑यति नय॒त्यूर्ग् वै वा ऊर्ङ् न॑यति नय॒त्यूर्ग् वै । \newline
22. ऊर्ग् वै वा ऊर् गूर्ग् वै यवो॒ यवो॒ वा ऊर् गूर्ग् वै यवः॑ । \newline
23. वै यवो॒ यवो॒ वै वै यव॒ ऊर् गूर्ग् यवो॒ वै वै यव॒ ऊर्क् । \newline
24. यव॒ ऊर् गूर्ग् यवो॒ यव॒ ऊर् गु॑दु॒म्बर॑ उदु॒म्बर॒ ऊर्ग् यवो॒ यव॒ ऊर् गु॑दु॒म्बरः॑ । \newline
25. ऊर्गु॑दु॒म्बर॑ उदु॒म्बर॒ ऊर् गूर्गु॑दु॒म्बर॑ ऊ॒र्जोर्जो दु॒म्बर॒ ऊर् गूर्गु॑दु॒म्बर॑ ऊ॒र्जा । \newline
26. उ॒दु॒म्बर॑ ऊ॒र्जोर्जो दु॒म्बर॑ उदु॒म्बर॑ ऊ॒र्जैवै वोर्जो दु॒म्बर॑ उदु॒म्बर॑ ऊ॒र्जैव । \newline
27. ऊ॒र्जैवै वोर्जोर् जैवोर्ज॒ मूर्ज॑ मे॒वोर् जोर्जै वोर्ज᳚म् । \newline
28. ए॒वोर्ज॒ मूर्ज॑ मे॒वै वोर्जꣳ॒॒ सꣳ स मूर्ज॑ मे॒वै वोर्जꣳ॒॒ सम् । \newline
29. ऊर्जꣳ॒॒ सꣳ स मूर्ज॒ मूर्जꣳ॒॒ स म॑र्द्धय त्यर्द्धयति॒ स मूर्ज॒ मूर्जꣳ॒॒ स म॑र्द्धयति । \newline
30. स म॑र्द्धय त्यर्द्धयति॒ सꣳ स म॑र्द्धयति॒ यज॑मानेन॒ यज॑मानेना र्द्धयति॒ सꣳ स म॑र्द्धयति॒ यज॑मानेन । \newline
31. अ॒र्द्ध॒य॒ति॒ यज॑मानेन॒ यज॑मानेना र्द्धय त्यर्द्धयति॒ यज॑मानेन॒ सम्मि॑ता॒ सम्मि॑ता॒ 
यज॑मानेना र्द्धय त्यर्द्धयति॒ यज॑मानेन॒ सम्मि॑ता । \newline
32. यज॑मानेन॒ सम्मि॑ता॒ सम्मि॑ता॒ यज॑मानेन॒ यज॑मानेन॒ सम्मि॒ तौदुं॑ब॒र् यौदुं॑बरी॒ सम्मि॑ता॒ यज॑मानेन॒ यज॑मानेन॒ सम्मि॒ तौदुं॑बरी । \newline
33. सम्मि॒ तौदुं॑ब॒र् यौदुं॑बरी॒ सम्मि॑ता॒ सम्मि॒ तौदुं॑बरी भवति भव॒ त्यौदुं॑बरी॒ सम्मि॑ता॒ 
सम्मि॒ तौदुं॑बरी भवति । \newline
34. सम्मि॒तेति॒ सं - मि॒ता॒ । \newline
35. औदुं॑बरी भवति भव॒ त्यौदुं॑ब॒र् यौदुं॑बरी भवति॒ यावा॒न्॒. यावा᳚न् भव॒ त्यौदुं॑ब॒र् यौदुं॑बरी भवति॒ यावान्॑ । \newline
36. भ॒व॒ति॒ यावा॒न्॒. यावा᳚न् भवति भवति॒ यावा॑ ने॒वैव यावा᳚न् भवति भवति॒ यावा॑ने॒व । \newline
37. यावा॑ने॒ वैव यावा॒न्॒. यावा॑ने॒व यज॑मानो॒ यज॑मान ए॒व यावा॒न्॒. यावा॑ने॒व यज॑मानः । \newline
38. ए॒व यज॑मानो॒ यज॑मान ए॒वैव यज॑मान॒ स्ताव॑ती॒म् ताव॑तीं॒ ॅयज॑मान ए॒वैव यज॑मान॒ स्ताव॑तीम् । \newline
39. यज॑मान॒ स्ताव॑ती॒म् ताव॑तीं॒ ॅयज॑मानो॒ यज॑मान॒ स्ताव॑ती मे॒वैव ताव॑तीं॒ ॅयज॑मानो॒ यज॑मान॒ स्ताव॑ती मे॒व । \newline
40. ताव॑ती मे॒वैव ताव॑ती॒म् ताव॑ती मे॒वास्मि॑न् नस्मिन्ने॒व ताव॑ती॒म् ताव॑ती मे॒वास्मिन्न्॑ । \newline
41. ए॒वास्मि॑न् नस्मिन् ने॒वै वास्मि॒न् नूर्ज॒ मूर्ज॑ मस्मिन् ने॒वै वास्मि॒न् नूर्ज᳚म् । \newline
42. अ॒स्मि॒न् नूर्ज॒ मूर्ज॑ मस्मिन् नस्मि॒न् नूर्ज॑म् दधाति दधा॒ त्यूर्ज॑ मस्मिन् नस्मि॒न् नूर्ज॑म् दधाति । \newline
43. ऊर्ज॑म् दधाति दधा॒ त्यूर्ज॒ मूर्ज॑म् दधाति पितृ॒णाम् पि॑तृ॒णाम् द॑धा॒ त्यूर्ज॒ मूर्ज॑म् दधाति पितृ॒णाम् । \newline
44. द॒धा॒ति॒ पि॒तृ॒णाम् पि॑तृ॒णाम् द॑धाति दधाति पितृ॒णाꣳ सद॑नꣳ॒॒ सद॑नम् पितृ॒णाम् द॑धाति दधाति पितृ॒णाꣳ सद॑नम् । \newline
45. पि॒तृ॒णाꣳ सद॑नꣳ॒॒ सद॑नम् पितृ॒णाम् पि॑तृ॒णाꣳ सद॑न मस्यसि॒ सद॑नम् पितृ॒णाम् पि॑तृ॒णाꣳ सद॑न मसि । \newline
46. सद॑न मस्यसि॒ सद॑नꣳ॒॒ सद॑न म॒सी तीत्य॑सि॒ सद॑नꣳ॒॒ सद॑न म॒सीति॑ । \newline
47. अ॒सीती त्य॑स्य॒ सीति॑ ब॒र्॒.हिर् ब॒र्॒.हि रित्य॑ स्य॒सीति॑ ब॒र्॒.हिः । \newline
48. इति॑ ब॒र्॒.हिर् ब॒र्॒.हि रितीति॑ ब॒र्॒.हि रवाव॑ ब॒र्॒.हि रितीति॑ ब॒र्॒.हि रव॑ । \newline
49. ब॒र्॒.हि रवाव॑ ब॒र्॒.हिर् ब॒र्॒.हि रव॑ स्तृणाति स्तृणा॒ त्यव॑ ब॒र्॒.हिर् ब॒र्॒.हिरव॑ स्तृणाति । \newline
50. अव॑ स्तृणाति स्तृणा॒ त्यवाव॑ स्तृणाति पितृदेव॒त्य॑म् पितृदेव॒त्यꣳ॑ स्तृणा॒ त्यवाव॑ स्तृणाति पितृदेव॒त्य᳚म् । \newline
51. स्तृ॒णा॒ति॒ पि॒तृ॒दे॒व॒त्य॑म् पितृदेव॒त्यꣳ॑ स्तृणाति स्तृणाति पितृदेव॒त्यꣳ॑ हि हि पि॑तृदेव॒त्यꣳ॑ स्तृणाति स्तृणाति पितृदेव॒त्यꣳ॑ हि । \newline
52. पि॒तृ॒दे॒व॒त्यꣳ॑ हि हि पि॑तृदेव॒त्य॑म् पितृदेव॒त्या᳚ [(अ1॒) ꣳ॒] ह्ये॑त दे॒त द्धि पि॑तृदेव॒त्य॑म् 
पितृदेव॒त्या᳚ [अ(1॒) ꣳ॒] ह्ये॑तत् । \newline
53. पि॒तृ॒दे॒व॒त्य॑मिति॑ पितृ - दे॒व॒त्य᳚म् । \newline
\pagebreak
\markright{ TS 6.2.10.4  \hfill https://www.vedavms.in \hfill}

\section{ TS 6.2.10.4 }

\textbf{TS 6.2.10.4 } \newline
\textbf{Samhita Paata} \newline

ह्ये॑तद्-यन्निखा॑तं॒ ॅयद्-ब॒र्॒.हिरन॑वस्तीर्य मिनु॒यात् पि॑तृदेव॒त्या॑ निखा॑ता स्याद्-ब॒र्॒.हिर॑व॒स्तीर्य॑ मिनोत्य॒स्यामे॒वैनां᳚ मिनो॒त्यथो᳚ स्वा॒रुह॑मे॒वैनां᳚ करो॒त्युद्-दिवꣳ॑ स्तभा॒नाऽऽन्तरि॑क्षं पृ॒णेत्या॑है॒षां ॅलो॒कानां॒ ॅविधृ॑त्यै द्युता॒नस्त्वा॑ मारु॒तो मि॑नो॒त्वित्या॑ह द्युता॒नो ह॑ स्म॒ वै मा॑रु॒तो दे॒वाना॒मौदु॑बंरीं मिनोति॒ तेनै॒वै- [  ] \newline

\textbf{Pada Paata} \newline

हि । ए॒तत् । यत् । निखा॑त॒मिति॒ नि - खा॒त॒म् । यत् । ब॒र्॒.हिः । अन॑वस्ती॒र्येत्यन॑व-स्ती॒र्य॒ । मि॒नु॒यात् । पि॒तृ॒दे॒व॒त्येति॑ पितृ - दे॒व॒त्या᳚ । निखा॒तेति॒ नि - खा॒ता॒ । स्या॒त् । ब॒र॒.हिः । अ॒व॒स्तीर्येत्य॑व- स्तीर्य॑ । मि॒नो॒ति॒ । अ॒स्याम् । ए॒व । ए॒ना॒म् । मि॒नो॒ति॒ । अथो॒ इति॑ । स्वा॒रुह॒मिति॑ स्व-रुह᳚म् । ए॒व । ए॒ना॒म् । क॒रो॒ति॒ । उदिति॑ । दिव᳚म् । स्त॒भा॒न॒ । एति॑ । अ॒न्तरि॑क्षम् । पृ॒ण॒ । इति॑ । आ॒ह॒ । ए॒षाम् । लो॒काना᳚म् । विधृ॑त्या॒ इति॒ वि - धृ॒त्यै॒ । द्यु॒ता॒नः । त्वा॒ । मा॒रु॒तः । मि॒नो॒तु॒ । इति॑ । आ॒ह॒ । द्यु॒ता॒नः । ह॒ । स्म॒ । वै । मा॒रु॒तः । दे॒वाना᳚म् । औदु॑बंरीम् । मि॒नो॒ति॒ । तेन॑ । ए॒व ।  \newline


\textbf{Krama Paata} \newline

ह्ये॑तत् । ए॒तद् यत् । यन् निखा॑तम् । निखा॑त॒म् ॅयत् । निखा॑त॒मिति॒ नि - खा॒त॒म् । यद् ब॒र्.॒हिः । ब॒र्.॒हिरन॑वस्तीर्य । अन॑वस्तीर्य मिनु॒यात् । अन॑वस्ती॒र्येत्यन॑व - स्ती॒र्य॒ । मि॒नु॒यात् पि॑तृदेव॒त्या᳚ । पि॒तृ॒दे॒व॒त्या॑ निखा॑ता । पि॒तृ॒दे॒व॒त्येति॑ पितृ - दे॒व॒त्या᳚ । निखा॑ता स्यात् । निखा॒तेति॒ नि - खा॒ता॒ । स्या॒द् ब॒र्.॒हिः । ब॒र्.॒हिर॑व॒स्तीर्य॑ । अ॒व॒स्तीर्य॑ मिनोति । अ॒व॒स्तीर्येत्य॑व - स्तीर्य॑ । मि॒नो॒त्य॒स्याम् । अ॒स्यामे॒व । ए॒वैना᳚म् । ए॒ना॒म् मि॒नो॒ति॒ । मि॒नो॒त्यथो᳚ । अथो᳚ स्वा॒रुह᳚म् । अथो॒ इत्यथो᳚ । स्वा॒रुह॑मे॒व । स्वा॒रुह॒मिति॑ स्व - रुह᳚म् । ए॒वैना᳚म् । ए॒ना॒म् क॒रो॒ति॒ । क॒रो॒त्युत् । उद् दिव᳚म् । दिवꣳ॑ स्तभा॒न । स्त॒भा॒ना । आऽन्तरि॑क्षम् । अ॒न्तरि॑क्षम् पृण । पृ॒णेति॑ । इत्या॑ह । आ॒है॒षाम् । ए॒षाम् ॅलो॒काना᳚म् । लो॒काना॒म् ॅविधृ॑त्यै । विधृ॑त्यै द्युता॒नः । विधृ॑त्या॒ इति॒ वि - धृ॒त्यै॒ । द्यु॒ता॒नस्त्वा᳚ । त्वा॒ मा॒रु॒तः । मा॒रु॒तो मि॑नोतु । मि॒नो॒त्विति॑ । इत्या॑ह । आ॒ह॒ द्यु॒ता॒नः । द्यु॒ता॒नो ह॑ । ह॒ स्म॒ । स्म॒ वै । वै मा॑रु॒तः । मा॒रु॒तो दे॒वाना᳚म् । दे॒वाना॒मौदु॑म्बरीम् । औदु॑म्बरीम् मिनोति । मि॒नो॒ति॒ तेन॑ । तेनै॒व । ए॒वैना᳚म् \newline

\textbf{Jatai Paata} \newline

1. ह्ये॑त दे॒त द्धि ह्ये॑तत् । \newline
2. ए॒तद् यद् यदे॒त दे॒तद् यत् । \newline
3. यन् निखा॑त॒म् निखा॑तं॒ ॅयद् यन् निखा॑तम् । \newline
4. निखा॑तं॒ ॅयद् यन् निखा॑त॒म् निखा॑तं॒ ॅयत् । \newline
5. निखा॑त॒मिति॒ नि - खा॒त॒म् । \newline
6. यद् ब॒र्॒.हिर् ब॒र्॒.हिर् यद् यद् ब॒र्॒.हिः । \newline
7. ब॒र्॒.हि रन॑वस्ती॒र्या न॑वस्तीर्य ब॒र्॒.हिर् ब॒र्॒.हि रन॑वस्तीर्य । \newline
8. अन॑वस्तीर्य मिनु॒यान् मि॑नु॒या दन॑वस्ती॒र्या न॑वस्तीर्य मिनु॒यात् । \newline
9. अन॑वस्ती॒र्येत्यन॑व - स्ती॒र्य॒ । \newline
10. मि॒नु॒यात् पि॑तृदेव॒त्या॑ पितृदेव॒त्या॑ मिनु॒यान् मि॑नु॒यात् पि॑तृदेव॒त्या᳚ । \newline
11. पि॒तृ॒दे॒व॒त्या॑ निखा॑ता॒ निखा॑ता पितृदेव॒त्या॑ पितृदेव॒त्या॑ निखा॑ता । \newline
12. पि॒तृ॒दे॒व॒त्येति॑ पितृ - दे॒व॒त्या᳚ । \newline
13. निखा॑ता स्याथ् स्या॒न् निखा॑ता॒ निखा॑ता स्यात् । \newline
14. निखा॒तेति॒ नि - खा॒ता॒ । \newline
15. स्या॒द् ब॒र्॒.हिर् ब॒र्॒.हिः स्या᳚थ् स्याद् ब॒र्॒.हिः । \newline
16. ब॒र्॒.हि र॑व॒स्तीर्या॑ व॒स्तीर्य॑ ब॒र्॒.हिर् ब॒र्॒.हि र॑व॒स्तीर्य॑ । \newline
17. अ॒व॒स्तीर्य॑ मिनोति मिनो त्यव॒स्तीर्या॑ व॒स्तीर्य॑ मिनोति । \newline
18. अ॒व॒स्तीर्येत्य॑व - स्तीर्य॑ । \newline
19. मि॒नो॒ त्य॒स्या म॒स्याम् मि॑नोति मिनो त्य॒स्याम् । \newline
20. अ॒स्या मे॒वैवास्या म॒स्या मे॒व । \newline
21. ए॒वैना॑ मेना मे॒वैवैना᳚म् । \newline
22. ए॒ना॒म् मि॒नो॒ति॒ मि॒नो॒ त्ये॒ना॒ मे॒ना॒म् मि॒नो॒ति॒ । \newline
23. मि॒नो॒ त्यथो॒ अथो॑ मिनोति मिनो॒ त्यथो᳚ । \newline
24. अथो᳚ स्वा॒रुहꣳ॑ स्वा॒रुह॒ मथो॒ अथो᳚ स्वा॒रुह᳚म् । \newline
25. अथो॒ इत्यथो᳚ । \newline
26. स्वा॒रुह॑ मे॒वैव स्वा॒रुहꣳ॑ स्वा॒रुह॑ मे॒व । \newline
27. स्वा॒रुह॒मिति॑ स्व - रुह᳚म् । \newline
28. ए॒वैना॑ मेना मे॒वै वैना᳚म् । \newline
29. ए॒ना॒म् क॒रो॒ति॒ क॒रो॒ त्ये॒ना॒ मे॒ना॒म् क॒रो॒ति॒ । \newline
30. क॒रो॒ त्युदुत् क॑रोति करो॒ त्युत् । \newline
31. उद् दिव॒म् दिव॒ मुदुद् दिव᳚म् । \newline
32. दिवꣳ॑ स्तभान स्तभान॒ दिव॒म् दिवꣳ॑ स्तभान । \newline
33. स्त॒भा॒ना स्त॑भान स्तभा॒ना । \newline
34. आ ऽन्तरि॑क्ष म॒न्तरि॑क्ष॒ मा ऽन्तरि॑क्षम् । \newline
35. अ॒न्तरि॑क्षम् पृण पृणा॒न्तरि॑क्ष म॒न्तरि॑क्षम् पृण । \newline
36. पृ॒णे तीति॑ पृण पृ॒णेति॑ । \newline
37. इत्या॑हा॒हे तीत्या॑ह । \newline
38. आ॒है॒षा मे॒षा मा॑हाहै॒षाम् । \newline
39. ए॒षाम् ॅलो॒काना᳚म् ॅलो॒काना॑ मे॒षा मे॒षाम् ॅलो॒काना᳚म् । \newline
40. लो॒कानां॒ ॅविधृ॑त्यै॒ विधृ॑त्यै लो॒काना᳚म् ॅलो॒कानां॒ ॅविधृ॑त्यै । \newline
41. विधृ॑त्यै द्युता॒नो द्यु॑ता॒नो विधृ॑त्यै॒ विधृ॑त्यै द्युता॒नः । \newline
42. विधृ॑त्या॒ इति॒ वि - धृ॒त्यै॒ । \newline
43. द्यु॒ता॒न स्त्वा᳚ त्वा द्युता॒नो द्यु॑ता॒न स्त्वा᳚ । \newline
44. त्वा॒ मा॒रु॒तो मा॑रु॒त स्त्वा᳚ त्वा मारु॒तः । \newline
45. मा॒रु॒तो मि॑नोतु मिनोतु मारु॒तो मा॑रु॒तो मि॑नोतु । \newline
46. मि॒नो॒ त्वितीति॑ मिनोतु मिनो॒ त्विति॑ । \newline
47. इत्या॑हा॒हे तीत्या॑ह । \newline
48. आ॒ह॒ द्यु॒ता॒नो द्यु॑ता॒न आ॑हाह द्युता॒नः । \newline
49. द्यु॒ता॒नो ह॑ ह द्युता॒नो द्यु॑ता॒नो ह॑ । \newline
50. ह॒ स्म॒ स्म॒ ह॒ ह॒ स्म॒ । \newline
51. स्म॒ वै वै स्म॑ स्म॒ वै । \newline
52. वै मा॑रु॒तो मा॑रु॒तो वै वै मा॑रु॒तः । \newline
53. मा॒रु॒तो दे॒वाना᳚म् दे॒वाना᳚म् मारु॒तो मा॑रु॒तो दे॒वाना᳚म् । \newline
54. दे॒वाना॒ मौदुं॑बरी॒ मौदुं॑बरीम् दे॒वाना᳚म् दे॒वाना॒ मौदुं॑बरीम् । \newline
55. औदुं॑बरीम् मिनोति मिनो॒ त्यौदुं॑बरी॒ मौदुं॑बरीम् मिनोति । \newline
56. मि॒नो॒ति॒ तेन॒ तेन॑ मिनोति मिनोति॒ तेन॑ । \newline
57. तेनै॒वैव तेन॒ तेनै॒व । \newline
58. ए॒वैना॑ मेना मे॒वैवैना᳚म् । \newline

\textbf{Ghana Paata } \newline

1. ह्ये॑त दे॒तद्धि ह्ये॑तद् यद् यदे॒त द्धि ह्ये॑तद् यत् । \newline
2. ए॒तद् यद् यदे॒त दे॒तद् यन् निखा॑त॒म् निखा॑तं॒ ॅयदे॒त दे॒तद् यन् निखा॑तम् । \newline
3. यन् निखा॑त॒म् निखा॑तं॒ ॅयद् यन् निखा॑तं॒ ॅयद् यन् निखा॑तं॒ ॅयद् यन् निखा॑तं॒ ॅयत् । \newline
4. निखा॑तं॒ ॅयद् यन् निखा॑त॒म् निखा॑तं॒ ॅयद् ब॒र्॒.हिर् ब॒र्॒.हिर् यन् निखा॑त॒म् निखा॑तं॒ ॅयद् ब॒र्॒.हिः । \newline
5. निखा॑त॒मिति॒ नि - खा॒त॒म् । \newline
6. यद् ब॒र्॒.हिर् ब॒र्॒.हिर् यद् यद् ब॒र्॒.हि रन॑वस्ती॒र्या न॑वस्तीर्य ब॒र्॒.हिर् यद् यद् ब॒र्॒.हि रन॑वस्तीर्य । \newline
7. ब॒र्॒.हि रन॑वस्ती॒र्या न॑वस्तीर्य ब॒र्॒.हिर् ब॒र्॒.हि रन॑वस्तीर्य मिनु॒यान् मि॑नु॒या दन॑वस्तीर्य ब॒र्॒.हिर् ब॒र्॒.हि रन॑वस्तीर्य मिनु॒यात् । \newline
8. अन॑वस्तीर्य मिनु॒यान् मि॑नु॒या दन॑वस्ती॒र्या न॑वस्तीर्य मिनु॒यात् पि॑तृदेव॒त्या॑ पितृदेव॒त्या॑ मिनु॒या दन॑वस्ती॒र्या न॑वस्तीर्य मिनु॒यात् पि॑तृदेव॒त्या᳚ । \newline
9. अन॑वस्ती॒र्येत्यन॑व - स्ती॒र्य॒ । \newline
10. मि॒नु॒यात् पि॑तृदेव॒त्या॑ पितृदेव॒त्या॑ मिनु॒यान् मि॑नु॒यात् पि॑तृदेव॒त्या॑ निखा॑ता॒ निखा॑ता पितृदेव॒त्या॑ मिनु॒यान् मि॑नु॒यात् पि॑तृदेव॒त्या॑ निखा॑ता । \newline
11. पि॒तृ॒दे॒व॒त्या॑ निखा॑ता॒ निखा॑ता पितृदेव॒त्या॑ पितृदेव॒त्या॑ निखा॑ता स्याथ् स्या॒न् निखा॑ता पितृदेव॒त्या॑ पितृदेव॒त्या॑ निखा॑ता स्यात् । \newline
12. पि॒तृ॒दे॒व॒त्येति॑ पितृ - दे॒व॒त्या᳚ । \newline
13. निखा॑ता स्याथ् स्या॒न् निखा॑ता॒ निखा॑ता स्याद् ब॒र्॒.हिर् ब॒र्॒.हिः स्या॒न् निखा॑ता॒ निखा॑ता स्याद् ब॒र्॒.हिः । \newline
14. निखा॒तेति॒ नि - खा॒ता॒ । \newline
15. स्या॒द् ब॒र्॒.हिर् ब॒र्॒.हिः स्या᳚थ् स्याद् ब॒र्॒.हि र॑व॒स्तीर्या॑ व॒स्तीर्य॑ ब॒र्॒.हिः स्या᳚थ् स्याद् ब॒र्॒.हि र॑व॒स्तीर्य॑ । \newline
16. ब॒र्॒.हि र॑व॒स्तीर्या॑ व॒स्तीर्य॑ ब॒र्॒.हिर् ब॒र्॒.हि र॑व॒स्तीर्य॑ मिनोति मिनो त्यव॒स्तीर्य॑ ब॒र्॒.हिर् ब॒र्॒.हि र॑व॒स्तीर्य॑ मिनोति । \newline
17. अ॒व॒स्तीर्य॑ मिनोति मिनो त्यव॒स्तीर्या॑ व॒स्तीर्य॑ मिनो त्य॒स्या म॒स्याम् मि॑नो त्यव॒स्तीर्या॑ व॒स्तीर्य॑ मिनो त्य॒स्याम् । \newline
18. अ॒व॒स्तीर्येत्य॑व - स्तीर्य॑ । \newline
19. मि॒नो॒ त्य॒स्या म॒स्याम् मि॑नोति मिनो त्य॒स्या मे॒वै वास्याम् मि॑नोति मिनो त्य॒स्या मे॒व । \newline
20. अ॒स्या मे॒वै वास्या म॒स्या मे॒वैना॑ मेना मे॒वास्या म॒स्या मे॒वैना᳚म् । \newline
21. ए॒वैना॑ मेना मे॒वै वैना᳚म् मिनोति मिनो त्येना मे॒वै वैना᳚म् मिनोति । \newline
22. ए॒ना॒म् मि॒नो॒ति॒ मि॒नो॒ त्ये॒ना॒ मे॒ना॒म् मि॒नो॒ त्यथो॒ अथो॑ मिनो त्येना मेनाम् मिनो॒ त्यथो᳚ । \newline
23. मि॒नो॒ त्यथो॒ अथो॑ मिनोति मिनो॒ त्यथो᳚ स्वा॒रुहꣳ॑ स्वा॒रुह॒ मथो॑ मिनोति मिनो॒ त्यथो᳚ स्वा॒रुह᳚म् । \newline
24. अथो᳚ स्वा॒रुहꣳ॑ स्वा॒रुह॒ मथो॒ अथो᳚ स्वा॒रुह॑ मे॒वैव स्वा॒रुह॒ मथो॒ अथो᳚ स्वा॒रुह॑ मे॒व । \newline
25. अथो॒ इत्यथो᳚ । \newline
26. स्वा॒रुह॑ मे॒वैव स्वा॒रुहꣳ॑ स्वा॒रुह॑ मे॒वैना॑ मेना मे॒व स्वा॒रुहꣳ॑ स्वा॒रुह॑ मे॒वैना᳚म् । \newline
27. स्वा॒रुह॒मिति॑ स्व - रुह᳚म् । \newline
28. ए॒वैना॑ मेना मे॒वै वैना᳚म् करोति करो त्येना मे॒वै वैना᳚म् करोति । \newline
29. ए॒ना॒म् क॒रो॒ति॒ क॒रो॒ त्ये॒ना॒ मे॒ना॒म् क॒रो॒ त्युदुत् क॑रो त्येना मेनाम् करो॒ त्युत् । \newline
30. क॒रो॒ त्युदुत् क॑रोति करो॒ त्युद् दिव॒म् दिव॒ मुत् क॑रोति करो॒ त्युद् दिव᳚म् । \newline
31. उद् दिव॒म् दिव॒ मुदुद् दिवꣳ॑ स्तभान स्तभान॒ दिव॒ मुदुद् दिवꣳ॑ स्तभान । \newline
32. दिवꣳ॑ स्तभान स्तभान॒ दिव॒म् दिवꣳ॑ स्तभा॒ना स्त॑भान॒ दिव॒म् दिवꣳ॑ स्तभा॒ना । \newline
33. स्त॒भा॒ना स्त॑भान स्तभा॒ना ऽन्तरि॑क्ष म॒न्तरि॑क्ष॒ मा स्त॑भान स्तभा॒ना ऽन्तरि॑क्षम् । \newline
34. आ ऽन्तरि॑क्ष म॒न्तरि॑क्ष॒ मा ऽन्तरि॑क्षम् पृण पृणा॒ न्तरि॑क्ष॒ मा ऽन्तरि॑क्षम् पृण । \newline
35. अ॒न्तरि॑क्षम् पृण पृणा॒न्तरि॑क्ष म॒न्तरि॑क्षम् पृ॒णे तीति॑ पृणा॒न्तरि॑क्ष म॒न्तरि॑क्षम् पृ॒णेति॑ । \newline
36. पृ॒णे तीति॑ पृण पृ॒णे त्या॑हा॒ हेति॑ पृण पृ॒णे त्या॑ह । \newline
37. इत्या॑हा॒हे तीत्या॑ है॒षा मे॒षा मा॒हे तीत्या॑ है॒षाम् । \newline
38. आ॒है॒षा मे॒षा मा॑हा है॒षाम् ॅलो॒काना᳚म् ॅलो॒काना॑ मे॒षा मा॑हा है॒षाम् ॅलो॒काना᳚म् । \newline
39. ए॒षाम् ॅलो॒काना᳚म् ॅलो॒काना॑ मे॒षा मे॒षाम् ॅलो॒कानां॒ ॅविधृ॑त्यै॒ विधृ॑त्यै लो॒काना॑ मे॒षा मे॒षाम् ॅलो॒कानां॒ ॅविधृ॑त्यै । \newline
40. लो॒कानां॒ ॅविधृ॑त्यै॒ विधृ॑त्यै लो॒काना᳚म् ॅलो॒कानां॒ ॅविधृ॑त्यै द्युता॒नो द्यु॑ता॒नो विधृ॑त्यै लो॒काना᳚म् ॅलो॒कानां॒ ॅविधृ॑त्यै द्युता॒नः । \newline
41. विधृ॑त्यै द्युता॒नो द्यु॑ता॒नो विधृ॑त्यै॒ विधृ॑त्यै द्युता॒न स्त्वा᳚ त्वा द्युता॒नो विधृ॑त्यै॒ विधृ॑त्यै द्युता॒न स्त्वा᳚ । \newline
42. विधृ॑त्या॒ इति॒ वि - धृ॒त्यै॒ । \newline
43. द्यु॒ता॒न स्त्वा᳚ त्वा द्युता॒नो द्यु॑ता॒न स्त्वा॑ मारु॒तो मा॑रु॒त स्त्वा᳚ द्युता॒नो द्यु॑ता॒न स्त्वा॑ मारु॒तः । \newline
44. त्वा॒ मा॒रु॒तो मा॑रु॒त स्त्वा᳚ त्वा मारु॒तो मि॑नोतु मिनोतु मारु॒त स्त्वा᳚ त्वा मारु॒तो मि॑नोतु । \newline
45. मा॒रु॒तो मि॑नोतु मिनोतु मारु॒तो मा॑रु॒तो मि॑नो॒ त्वितीति॑ मिनोतु मारु॒तो मा॑रु॒तो मि॑नो॒ त्विति॑ । \newline
46. मि॒नो॒ त्वितीति॑ मिनोतु मिनो॒ त्वित्या॑ हा॒हेति॑ मिनोतु मिनो॒ त्वित्या॑ह । \newline
47. इत्या॑हा॒हे तीत्या॑ह द्युता॒नो द्यु॑ता॒न आ॒हे तीत्या॑ह द्युता॒नः । \newline
48. आ॒ह॒ द्यु॒ता॒नो द्यु॑ता॒न आ॑हाह द्युता॒नो ह॑ ह द्युता॒न आ॑हाह द्युता॒नो ह॑ । \newline
49. द्यु॒ता॒नो ह॑ ह द्युता॒नो द्यु॑ता॒नो ह॑ स्म स्म ह द्युता॒नो द्यु॑ता॒नो ह॑ स्म । \newline
50. ह॒ स्म॒ स्म॒ ह॒ ह॒ स्म॒ वै वै स्म॑ ह ह स्म॒ वै । \newline
51. स्म॒ वै वै स्म॑ स्म॒ वै मा॑रु॒तो मा॑रु॒तो वै स्म॑ स्म॒ वै मा॑रु॒तः । \newline
52. वै मा॑रु॒तो मा॑रु॒तो वै वै मा॑रु॒तो दे॒वाना᳚म् दे॒वाना᳚म् मारु॒तो वै वै मा॑रु॒तो दे॒वाना᳚म् । \newline
53. मा॒रु॒तो दे॒वाना᳚म् दे॒वाना᳚म् मारु॒तो मा॑रु॒तो दे॒वाना॒ मौदुं॑बरी॒ मौदुं॑बरीम् दे॒वाना᳚म् मारु॒तो मा॑रु॒तो दे॒वाना॒ मौदुं॑बरीम् । \newline
54. दे॒वाना॒ मौदुं॑बरी॒ मौदुं॑बरीम् दे॒वाना᳚म् दे॒वाना॒ मौदुं॑बरीम् मिनोति मिनो॒ त्यौदुं॑बरीम् दे॒वाना᳚म् दे॒वाना॒ मौदुं॑बरीम् मिनोति । \newline
55. औदुं॑बरीम् मिनोति मिनो॒ त्यौदुं॑बरी॒ मौदुं॑बरीम् मिनोति॒ तेन॒ तेन॑ मिनो॒ त्यौदुं॑बरी॒ मौदुं॑बरीम् मिनोति॒ तेन॑ । \newline
56. मि॒नो॒ति॒ तेन॒ तेन॑ मिनोति मिनोति॒ तेनै॒ वैव तेन॑ मिनोति मिनोति॒ तेनै॒व । \newline
57. तेनै॒ वैव तेन॒ तेनै॒ वैना॑ मेना मे॒व तेन॒ तेनै॒ वैना᳚म् । \newline
58. ए॒वैना॑ मेना मे॒वै वैना᳚म् मिनोति मिनो त्येना मे॒वै वैना᳚म् मिनोति । \newline
\pagebreak
\markright{ TS 6.2.10.5  \hfill https://www.vedavms.in \hfill}

\section{ TS 6.2.10.5 }

\textbf{TS 6.2.10.5 } \newline
\textbf{Samhita Paata} \newline

-नां᳚ मिनोति ब्रह्म॒वनिं॑ त्वा क्षत्र॒वनि॒मित्या॑ह यथाय॒जुरे॒वैतद् घृ॒तेन॑ द्यावापृथिवी॒ आ पृ॑णेथा॒मित्यौदु॑बंर्यां जुहोति॒ द्यावा॑पृथि॒वी ए॒व रसे॑नानक्त्या॒-न्तम॒न्व-व॑स्रावयत्या॒न्तमे॒व यज॑मानं॒ तेज॑सा-नक्त्यै॒न्द्रम॒सीति॑ छ॒दिरधि॒ नि द॑धात्यै॒न्द्रꣳ हि दे॒वत॑या॒ सदो॑ विश्वज॒नस्य॑ छा॒येत्या॑ह विश्वज॒नस्य॒ ह्ये॑षा छा॒या यथ् सदो॒ नव॑छदि॒- [  ] \newline

\textbf{Pada Paata} \newline

ए॒ना॒म् । मि॒नो॒ति॒ । ब्र॒ह्म॒वनि॒मिति॑ ब्रह्म - वनि᳚म् । त्वा॒ । क्ष॒त्र॒वनि॒मिति॑ क्षत्र - वनि᳚म् । इति॑ । आ॒ह॒ । य॒था॒य॒जु॒रिति॑ यथा - य॒जुः । ए॒व । ए॒तत् । घृ॒तेन॑ । द्या॒वा॒पृ॒थि॒वी॒ इति॑ द्यावा-पृ॒थि॒वी॒ । एति॑ । पृ॒णे॒था॒म् । इति॑ । औदु॑बंर्याम् । जु॒हो॒ति॒ । द्यावा॑पृथि॒वी इति॒ द्यावा᳚ - पृ॒थि॒वी । ए॒व । रसे॑न । अ॒न॒क्ति॒ । आ॒न्तमित्या᳚ - अ॒न्तम् । अ॒न्वव॑स्रावय॒तीय॑नु -अव॑स्रावयति । आ॒न्तमित्या᳚-अ॒न्तम् । ए॒व । यज॑मानम् । तेज॑सा । अ॒न॒क्ति॒ । ऐ॒न्द्रम् । अ॒सि॒ । इति॑ । छ॒दिः । अधि॑ । नीति॑ । द॒धा॒ति॒ । ऐ॒न्द्रम् । हि । दे॒वत॑या । सदः॑ । वि॒श्व॒ज॒नस्येति॑ विश्व - ज॒नस्य॑ । छा॒या । इति॑ । आ॒ह॒ । वि॒श्व॒ज॒नस्येति॑ विश्व - ज॒नस्य॑ । हि । ए॒षा । छा॒या । यत् । सदः॑ । नव॑छ॒दीति॒ नव॑ - छ॒दि॒ ।  \newline


\textbf{Krama Paata} \newline

ए॒ना॒म् मि॒नो॒ति॒ । मि॒नो॒ति॒ ब्र॒ह्म॒वनि᳚म् । ब्र॒ह्म॒वनि॑म् त्वा । ब्र॒ह्म॒वनि॒मिति॑ ब्रह्म - वनि᳚म् । त्वा॒ क्ष॒त्र॒वनि᳚म् । क्ष॒त्र॒वनि॒मिति॑ । क्ष॒त्र॒वनि॒मिति॑ क्षत्र - वनि᳚म् । इत्या॑ह । आ॒ह॒ य॒था॒य॒जुः । य॒था॒य॒जुरे॒व । य॒था॒य॒जुरिति॑ यथा - य॒जुः । ए॒वैतत् । ए॒तद् घृ॒तेन॑ । घृ॒तेन॑ द्यावापृथिवी । द्या॒वा॒पृ॒थि॒वी॒ आ । द्या॒वा॒पृ॒थि॒वी॒ इति॑ द्यावा - पृ॒थि॒वी॒ । आ पृ॑णेथाम् । पृ॒णे॒था॒मिति॑ । इत्यौदु॑म्बर्याम् । औदु॑म्बर्याम् जुहोति । जु॒हो॒ति॒ द्यावा॑पृथि॒वी । द्यावा॑पृथि॒वी ए॒व । द्यावा॑पृथि॒वी इति॒ द्यावा᳚ - पृ॒थि॒वी । ए॒व रसे॑न । रसे॑नानक्ति । अ॒न॒क्त्या॒न्तम् । आ॒न्तम॒न्वव॑स्रावयति । आ॒न्तमित्या᳚ - अ॒न्तम् । अ॒न्वव॑स्रावयत्या॒न्तम् । अ॒न्वव॑स्रावय॒तीत्य॑नु - अव॑स्रावयति । आ॒न्तमे॒व । आ॒न्तमित्या᳚ - अ॒न्तम् । ए॒व यज॑मानम् । यज॑मान॒म् तेज॑सा । तेज॑साऽनक्ति । अ॒न॒क्त्यै॒न्द्रम् । ऐ॒न्द्रम॑सि । अ॒सीति॑ । इति॑ छ॒दिः । छ॒दिरधि॑ । अधि॒ नि । नि द॑धाति । द॒धा॒त्यै॒न्द्रम् । ऐ॒न्द्रꣳ हि । हि दे॒वत॑या । दे॒वत॑या॒ सदः॑ । सदो॑ विश्वज॒नस्य॑ । वि॒श्व॒ज॒नस्य॑ छा॒या । वि॒श्व॒ज॒नस्येति॑ विश्व - ज॒नस्य॑ । छा॒येति॑ । इत्या॑ह । आ॒ह॒ वि॒श्व॒ज॒नस्य॑ । वि॒श्व॒ज॒नस्य॒ हि । वि॒श्व॒ज॒नस्येति॑ विश्व - ज॒नस्य॑ । ह्ये॑षा । ए॒षा छा॒या । छा॒या यत् । यथ् सदः॑ । सदो॒ नव॑छदि । नव॑छदि॒ तेज॑स्कामस्य । नव॑छ॒दीति॒ नव॑ - छ॒दि॒ \newline

\textbf{Jatai Paata} \newline

1. ए॒ना॒म् मि॒नो॒ति॒ मि॒नो॒ त्ये॒ना॒ मे॒ना॒म् मि॒नो॒ति॒ । \newline
2. मि॒नो॒ति॒ ब्र॒ह्म॒वनि॑म् ब्रह्म॒वनि॑म् मिनोति मिनोति ब्रह्म॒वनि᳚म् । \newline
3. ब्र॒ह्म॒वनि॑म् त्वा त्वा ब्रह्म॒वनि॑म् ब्रह्म॒वनि॑म् त्वा । \newline
4. ब्र॒ह्म॒वनि॒मिति॑ ब्रह्म - वनि᳚म् । \newline
5. त्वा॒ क्ष॒त्र॒वनि॑म् क्षत्र॒वनि॑म् त्वा त्वा क्षत्र॒वनि᳚म् । \newline
6. क्ष॒त्र॒वनि॒ मितीति॑ क्षत्र॒वनि॑म् क्षत्र॒वनि॒ मिति॑ । \newline
7. क्ष॒त्र॒वनि॒मिति॑ क्षत्र - वनि᳚म् । \newline
8. इत्या॑हा॒हे तीत्या॑ह । \newline
9. आ॒ह॒ य॒था॒य॒जुर् य॑थाय॒जु रा॑हाह यथाय॒जुः । \newline
10. य॒था॒य॒जु रे॒वैव य॑थाय॒जुर् य॑थाय॒जु रे॒व । \newline
11. य॒था॒य॒जुरिति॑ यथा - य॒जुः । \newline
12. ए॒वैत दे॒त दे॒वै वैतत् । \newline
13. ए॒तद् घृ॒तेन॑ घृ॒ते नै॒त दे॒तद् घृ॒तेन॑ । \newline
14. घृ॒तेन॑ द्यावापृथिवी द्यावापृथिवी घृ॒तेन॑ घृ॒तेन॑ द्यावापृथिवी । \newline
15. द्या॒वा॒पृ॒थि॒वी॒ आ द्या॑वापृथिवी द्यावापृथिवी॒ आ । \newline
16. द्या॒वा॒पृ॒थि॒वी॒ इति॑ द्यावा - पृ॒थि॒वी॒ । \newline
17. आ पृ॑णेथाम् पृणेथा॒ मा पृ॑णेथाम् । \newline
18. पृ॒णे॒था॒ मितीति॑ पृणेथाम् पृणेथा॒ मिति॑ । \newline
19. इत्यौदुं॑बर्या॒ मौदुं॑बर्या॒ मिती त्यौदुं॑बर्याम् । \newline
20. औदुं॑बर्याम् जुहोति जुहो॒ त्यौदुं॑बर्या॒ मौदुं॑बर्याम् जुहोति । \newline
21. जु॒हो॒ति॒ द्यावा॑पृथि॒वी द्यावा॑पृथि॒वी जु॑होति जुहोति॒ द्यावा॑पृथि॒वी । \newline
22. द्यावा॑पृथि॒वी ए॒वैव द्यावा॑पृथि॒वी द्यावा॑पृथि॒वी ए॒व । \newline
23. द्यावा॑पृथि॒वी इति॒ द्यावा᳚ - पृ॒थि॒वी । \newline
24. ए॒व रसे॑न॒ रसे॑ नै॒वैव रसे॑न । \newline
25. रसे॑ नानक्त्यनक्ति॒ रसे॑न॒ रसे॑ना नक्ति । \newline
26. अ॒न॒क्त्या॒न्त मा॒न्त म॑नक्त्य नक्त्या॒न्तम् । \newline
27. आ॒न्त म॒न्वव॑स्रावय त्य॒न्वव॑स्रावयत्या॒न्त मा॒न्त म॒न्वव॑स्रावयति । \newline
28. आ॒न्तमित्या᳚ - अ॒न्तम् । \newline
29. अ॒न्वव॑स्रावयत्या॒न्त मा॒न्त म॒न्वव॑स्रावय त्य॒न्वव॑स्रावयत्या॒न्तम् । \newline
30. अ॒न्वव॑स्रावय॒तीत्य॑नु - अव॑स्रावयति । \newline
31. आ॒न्त मे॒वैवान्त मा॒न्त मे॒व । \newline
32. आ॒न्तमित्या᳚ - अ॒न्तम् । \newline
33. ए॒व यज॑मानं॒ ॅयज॑मान मे॒वैव यज॑मानम् । \newline
34. यज॑मान॒म् तेज॑सा॒ तेज॑सा॒ यज॑मानं॒ ॅयज॑मान॒म् तेज॑सा । \newline
35. तेज॑सा ऽनक्त्यनक्ति॒ तेज॑सा॒ तेज॑सा ऽनक्ति । \newline
36. अ॒न॒क्त्यै॒न्द्र मै॒न्द्र म॑नक् त्यनक् त्यै॒न्द्रम् । \newline
37. ऐ॒न्द्र म॑स्य स्यै॒न्द्र मै॒न्द्र म॑सि । \newline
38. अ॒सीती त्य॑स्य॒सीति॑ । \newline
39. इति॑ छ॒दि श्छ॒दि रितीति॑ छ॒दिः । \newline
40. छ॒दि रध्यधि॑ च्छ॒दि श्छ॒दि रधि॑ । \newline
41. अधि॒ नि न्यध्य धि॒ नि । \newline
42. नि द॑धाति दधाति॒ नि नि द॑धाति । \newline
43. द॒धा॒ त्यै॒न्द्र मै॒न्द्रम् द॑धाति दधा त्यै॒न्द्रम् । \newline
44. ऐ॒न्द्रꣳ हि ह्यै᳚न्द्र मै॒न्द्रꣳ हि । \newline
45. हि दे॒वत॑या दे॒वत॑या॒ हि हि दे॒वत॑या । \newline
46. दे॒वत॑या॒ सदः॒ सदो॑ दे॒वत॑या दे॒वत॑या॒ सदः॑ । \newline
47. सदो॑ विश्वज॒नस्य॑ विश्वज॒नस्य॒ सदः॒ सदो॑ विश्वज॒नस्य॑ । \newline
48. वि॒श्व॒ज॒नस्य॑ छा॒या छा॒या वि॑श्वज॒नस्य॑ विश्वज॒नस्य॑ छा॒या । \newline
49. वि॒श्व॒ज॒नस्येति॑ विश्व - ज॒नस्य॑ । \newline
50. छा॒येतीति॑ छा॒या छा॒येति॑ । \newline
51. इत्या॑हा॒हे तीत्या॑ह । \newline
52. आ॒ह॒ वि॒श्व॒ज॒नस्य॑ विश्वज॒नस्या॑ हाह विश्वज॒नस्य॑ । \newline
53. वि॒श्व॒ज॒नस्य॒ हि हि वि॑श्वज॒नस्य॑ विश्वज॒नस्य॒ हि । \newline
54. वि॒श्व॒ज॒नस्येति॑ विश्व - ज॒नस्य॑ । \newline
55. ह्ये॑षैषा हि ह्ये॑षा । \newline
56. ए॒षा छा॒या छा॒यैषैषा छा॒या । \newline
57. छा॒या यद् यच् छा॒या छा॒या यत् । \newline
58. यथ् सदः॒ सदो॒ यद् यथ् सदः॑ । \newline
59. सदो॒ नव॑छदि॒ नव॑छदि॒ सदः॒ सदो॒ नव॑छदि । \newline
60. नव॑छदि॒ तेज॑स्कामस्य॒ तेज॑स्कामस्य॒ नव॑छदि॒ नव॑छदि॒ तेज॑स्कामस्य । \newline
61. नव॑छ॒दीति॒ नव॑ - छ॒दि॒ । \newline

\textbf{Ghana Paata } \newline

1. ए॒ना॒म् मि॒नो॒ति॒ मि॒नो॒ त्ये॒ना॒ मे॒ना॒म् मि॒नो॒ति॒ ब्र॒ह्म॒वनि॑म् ब्रह्म॒वनि॑म् मिनो त्येना मेनाम् मिनोति ब्रह्म॒वनि᳚म् । \newline
2. मि॒नो॒ति॒ ब्र॒ह्म॒वनि॑म् ब्रह्म॒वनि॑म् मिनोति मिनोति ब्रह्म॒वनि॑म् त्वा त्वा ब्रह्म॒वनि॑म् मिनोति मिनोति ब्रह्म॒वनि॑म् त्वा । \newline
3. ब्र॒ह्म॒वनि॑म् त्वा त्वा ब्रह्म॒वनि॑म् ब्रह्म॒वनि॑म् त्वा क्षत्र॒वनि॑म् क्षत्र॒वनि॑म् त्वा ब्रह्म॒वनि॑म् ब्रह्म॒वनि॑म् त्वा क्षत्र॒वनि᳚म् । \newline
4. ब्र॒ह्म॒वनि॒मिति॑ ब्रह्म - वनि᳚म् । \newline
5. त्वा॒ क्ष॒त्र॒वनि॑म् क्षत्र॒वनि॑म् त्वा त्वा क्षत्र॒वनि॒ मितीति॑ क्षत्र॒वनि॑म् त्वा त्वा क्षत्र॒वनि॒ मिति॑ । \newline
6. क्ष॒त्र॒वनि॒ मितीति॑ क्षत्र॒वनि॑म् क्षत्र॒वनि॒ मित्या॑हा॒ हेति॑ क्षत्र॒वनि॑म् क्षत्र॒वनि॒ मित्या॑ह । \newline
7. क्ष॒त्र॒वनि॒मिति॑ क्षत्र - वनि᳚म् । \newline
8. इत्या॑हा॒हे तीत्या॑ह यथाय॒जुर् य॑थाय॒जु रा॒हे तीत्या॑ह यथाय॒जुः । \newline
9. आ॒ह॒ य॒था॒य॒जुर् य॑थाय॒जु रा॑हाह यथाय॒जु रे॒वैव य॑थाय॒जु रा॑हाह यथाय॒जु रे॒व । \newline
10. य॒था॒य॒जु रे॒वैव य॑थाय॒जुर् य॑थाय॒जु रे॒वैत दे॒त दे॒व य॑थाय॒जुर् य॑थाय॒जु रे॒वैतत् । \newline
11. य॒था॒य॒जुरिति॑ यथा - य॒जुः । \newline
12. ए॒वैत दे॒त दे॒वै वैतद् घृ॒तेन॑ घृ॒तेनै॒त दे॒वैवैतद् घृ॒तेन॑ । \newline
13. ए॒तद् घृ॒तेन॑ घृ॒तेनै॒त दे॒तद् घृ॒तेन॑ द्यावापृथिवी द्यावापृथिवी घृ॒तेनै॒त दे॒तद् घृ॒तेन॑ द्यावापृथिवी । \newline
14. घृ॒तेन॑ द्यावापृथिवी द्यावापृथिवी घृ॒तेन॑ घृ॒तेन॑ द्यावापृथिवी॒ आ द्या॑वापृथिवी घृ॒तेन॑ घृ॒तेन॑ द्यावापृथिवी॒ आ । \newline
15. द्या॒वा॒पृ॒थि॒वी॒ आ द्या॑वापृथिवी द्यावापृथिवी॒ आ पृ॑णेथाम् पृणेथा॒ मा द्या॑वापृथिवी द्यावापृथिवी॒ आ पृ॑णेथाम् । \newline
16. द्या॒वा॒पृ॒थि॒वी॒ इति॑ द्यावा - पृ॒थि॒वी॒ । \newline
17. आ पृ॑णेथाम् पृणेथा॒ मा पृ॑णेथा॒ मितीति॑ पृणेथा॒ मा पृ॑णेथा॒ मिति॑ । \newline
18. पृ॒णे॒था॒ मितीति॑ पृणेथाम् पृणेथा॒ मित्यौ दुं॑बर्या॒ मौदुं॑बर्या॒ मिति॑ पृणेथाम् पृणेथा॒ मित्यौ दुं॑बर्याम् । \newline
19. इत्यौदुं॑बर्या॒ मौदुं॑बर्या॒ मिती त्यौदुं॑बर्याम् जुहोति जुहो॒ त्यौदुं॑बर्या॒ मिती त्यौदुं॑बर्याम् जुहोति । \newline
20. औदुं॑बर्याम् जुहोति जुहो॒ त्यौदुं॑बर्या॒ मौदुं॑बर्याम् जुहोति॒ द्यावा॑पृथि॒वी द्यावा॑पृथि॒वी जु॑हो॒
त्यौदुं॑बर्या॒ मौदुं॑बर्याम् जुहोति॒ द्यावा॑पृथि॒वी । \newline
21. जु॒हो॒ति॒ द्यावा॑पृथि॒वी द्यावा॑पृथि॒वी जु॑होति जुहोति॒ द्यावा॑पृथि॒वी ए॒वैव द्यावा॑पृथि॒वी जु॑होति जुहोति॒ द्यावा॑पृथि॒वी ए॒व । \newline
22. द्यावा॑पृथि॒वी ए॒वैव द्यावा॑पृथि॒वी द्यावा॑पृथि॒वी ए॒व रसे॑न॒ रसे॑नै॒व द्यावा॑पृथि॒वी द्यावा॑पृथि॒वी ए॒व रसे॑न । \newline
23. द्यावा॑पृथि॒वी इति॒ द्यावा᳚ - पृ॒थि॒वी । \newline
24. ए॒व रसे॑न॒ रसे॑नै॒ वैव रसे॑ना नक्त्यनक्ति॒ रसे॑नै॒ वैव रसे॑नानक्ति । \newline
25. रसे॑नानक् त्यनक्ति॒ रसे॑न॒ रसे॑ना नक्त्या॒न्त मा॒न्त म॑नक्ति॒ रसे॑न॒ रसे॑ना नक्त्या॒न्तम् । \newline
26. अ॒न॒ क्त्या॒न्त मा॒न्त म॑नक् त्यनक्त्या॒न्त म॒न्वव॑स्रावय त्य॒न्वव॑स्रावय त्या॒न्त म॑नक् त्यनक्त्या॒न्त म॒न्वव॑स्रावयति । \newline
27. आ॒न्त म॒न्वव॑स्रावय त्य॒न्वव॑स्रावय त्या॒न्त मा॒न्त म॒न्वव॑स्रावय त्या॒न्त मा॒न्त म॒न्वव॑स्रावय त्या॒न्त मा॒न्त म॒न्वव॑स्रावय त्या॒न्तम् । \newline
28. आ॒न्तमित्या᳚ - अ॒न्तम् । \newline
29. अ॒न्वव॑स्रावय त्या॒न्त मा॒न्त म॒न्वव॑स्रावय त्य॒न्वव॑स्रावय त्या॒न्त मे॒वैवान्त म॒न्वव॑स्रावय त्य॒न्वव॑स्रावय त्या॒न्त मे॒व । \newline
30. अ॒न्वव॑स्रावय॒तीत्य॑नु - अव॑स्रावयति । \newline
31. आ॒न्त मे॒वै वान्त मा॒न्त मे॒व यज॑मानं॒ ॅयज॑मान मे॒वान्त मा॒न्त मे॒व यज॑मानम् । \newline
32. आ॒न्तमित्या᳚ - अ॒न्तम् । \newline
33. ए॒व यज॑मानं॒ ॅयज॑मान मे॒वैव यज॑मान॒म् तेज॑सा॒ तेज॑सा॒ यज॑मान मे॒वैव यज॑मान॒म् तेज॑सा । \newline
34. यज॑मान॒म् तेज॑सा॒ तेज॑सा॒ यज॑मानं॒ ॅयज॑मान॒म् तेज॑सा ऽनक्त्यनक्ति॒ तेज॑सा॒ यज॑मानं॒ ॅयज॑मान॒म् तेज॑सा ऽनक्ति । \newline
35. तेज॑सा ऽनक्त्य नक्ति॒ तेज॑सा॒ तेज॑सा ऽनक्त्यै॒न्द्र मै॒न्द्र म॑नक्ति॒ तेज॑सा॒ तेज॑सा ऽनक्त्यै॒न्द्रम् । \newline
36. अ॒न॒क्त्यै॒न्द्र मै॒न्द्र म॑नक्त्य नक्त्यै॒न्द्र म॑स्य स्यै॒न्द्र म॑नक्त्य नक्त्यै॒न्द्र म॑सि । \newline
37. ऐ॒न्द्र म॑स्य स्यै॒न्द्र मै॒न्द्र म॒सीती त्य॑स्यै॒न्द्र मै॒न्द्र म॒सीति॑ । \newline
38. अ॒सीती त्य॑स्य॒ सीति॑ छ॒दि श्छ॒दि रित्य॑स्य॒ सीति॑ छ॒दिः । \newline
39. इति॑ छ॒दि श्छ॒दि रितीति॑ छ॒दि रध्यधि॑ च्छ॒दि रितीति॑ छ॒दि रधि॑ । \newline
40. छ॒दि रध्यधि॑ च्छ॒दि श्छ॒दि रधि॒ नि न्यधि॑ च्छ॒दि श्छ॒दि रधि॒ नि । \newline
41. अधि॒ नि न्यध्यधि॒ नि द॑धाति दधाति॒ न्यध्यधि॒ नि द॑धाति । \newline
42. नि द॑धाति दधाति॒ नि नि द॑धा त्यै॒न्द्र मै॒न्द्रम् द॑धाति॒ नि नि द॑धा त्यै॒न्द्रम् । \newline
43. द॒धा॒ त्यै॒न्द्र मै॒न्द्रम् द॑धाति दधा त्यै॒न्द्रꣳ हि ह्यै᳚न्द्रम् द॑धाति दधा त्यै॒न्द्रꣳ हि । \newline
44. ऐ॒न्द्रꣳ हि ह्यै᳚न्द्र मै॒न्द्रꣳ हि दे॒वत॑या दे॒वत॑या॒ ह्यै᳚न्द्र मै॒न्द्रꣳ हि दे॒वत॑या । \newline
45. हि दे॒वत॑या दे॒वत॑या॒ हि हि दे॒वत॑या॒ सदः॒ सदो॑ दे॒वत॑या॒ हि हि दे॒वत॑या॒ सदः॑ । \newline
46. दे॒वत॑या॒ सदः॒ सदो॑ दे॒वत॑या दे॒वत॑या॒ सदो॑ विश्वज॒नस्य॑ विश्वज॒नस्य॒ सदो॑ दे॒वत॑या दे॒वत॑या॒ सदो॑ विश्वज॒नस्य॑ । \newline
47. सदो॑ विश्वज॒नस्य॑ विश्वज॒नस्य॒ सदः॒ सदो॑ विश्वज॒नस्य॑ छा॒या छा॒या वि॑श्वज॒नस्य॒ सदः॒ सदो॑ विश्वज॒नस्य॑ छा॒या । \newline
48. वि॒श्व॒ज॒नस्य॑ छा॒या छा॒या वि॑श्वज॒नस्य॑ विश्वज॒नस्य॑ छा॒ये तीति॑ छा॒या वि॑श्वज॒नस्य॑ विश्वज॒नस्य॑ छा॒येति॑ । \newline
49. वि॒श्व॒ज॒नस्येति॑ विश्व - ज॒नस्य॑ । \newline
50. छा॒ये तीति॑ छा॒या छा॒येत्या॑ हा॒हेति॑ छा॒या छा॒ये त्या॑ह । \newline
51. इत्या॑हा॒हे तीत्या॑ह विश्वज॒नस्य॑ विश्वज॒नस्या॒हे तीत्या॑ह विश्वज॒नस्य॑ । \newline
52. आ॒ह॒ वि॒श्व॒ज॒नस्य॑ विश्वज॒नस्या॑हाह विश्वज॒नस्य॒ हि हि वि॑श्वज॒नस्या॑हाह विश्वज॒नस्य॒ हि । \newline
53. वि॒श्व॒ज॒नस्य॒ हि हि वि॑श्वज॒नस्य॑ विश्वज॒नस्य॒ ह्ये॑षैषा हि वि॑श्वज॒नस्य॑ विश्वज॒नस्य॒ ह्ये॑षा । \newline
54. वि॒श्व॒ज॒नस्येति॑ विश्व - ज॒नस्य॑ । \newline
55. ह्ये॑षैषा हि ह्ये॑षा छा॒या छा॒यैषा हि ह्ये॑षा छा॒या । \newline
56. ए॒षा छा॒या छा॒यै षैषा छा॒या यद् यच् छा॒यै षैषा छा॒या यत् । \newline
57. छा॒या यद् यच् छा॒या छा॒या यथ् सदः॒ सदो॒ यच् छा॒या छा॒या यथ् सदः॑ । \newline
58. यथ् सदः॒ सदो॒ यद् यथ् सदो॒ नव॑छदि॒ नव॑छदि॒ सदो॒ यद् यथ् सदो॒ नव॑छदि । \newline
59. सदो॒ नव॑छदि॒ नव॑छदि॒ सदः॒ सदो॒ नव॑छदि॒ तेज॑स्कामस्य॒ तेज॑स्कामस्य॒ नव॑छदि॒ सदः॒ सदो॒ नव॑छदि॒ तेज॑स्कामस्य । \newline
60. नव॑छदि॒ तेज॑स्कामस्य॒ तेज॑स्कामस्य॒ नव॑छदि॒ नव॑छदि॒ तेज॑स्कामस्य मिनुयान् मिनुया॒त् तेज॑स्कामस्य॒ नव॑छदि॒ नव॑छदि॒ तेज॑स्कामस्य मिनुयात् । \newline
61. नव॑छ॒दीति॒ नव॑ - छ॒दि॒ । \newline
\pagebreak
\markright{ TS 6.2.10.6  \hfill https://www.vedavms.in \hfill}

\section{ TS 6.2.10.6 }

\textbf{TS 6.2.10.6 } \newline
\textbf{Samhita Paata} \newline

तेज॑स्कामस्य मिनुयात् त्रि॒वृता॒ स्तोमे॑न॒ सम्मि॑तं॒ तेज॑स्त्रि॒वृत् ते॑ज॒स्व्ये॑व भ॑व॒-त्येका॑दश-छदीन्द्रि॒यका॑म॒-स्यैका॑दशाक्षरा त्रि॒ष्टुगि॑न्द्रि॒यं त्रि॒ष्टुगि॑न्द्रिया॒व्ये॑व भ॑वति॒ पञ्च॑दशछदि॒ भ्रातृ॑व्यवतः पञ्चद॒शो वज्रो॒ भ्रातृ॑व्याभिभूत्यै स॒प्तद॑शछदि प्र॒जाका॑मस्य सप्तद॒शः प्र॒जाप॑तिः प्र॒जाप॑ते॒राप्त्या॒ एक॑विꣳशतिछदि प्रति॒ष्ठाका॑म-स्यैकविꣳ॒॒शः स्तोमा॑नां प्रति॒ष्ठा प्रति॑ष्ठित्या उ॒दरं॒ ॅवै सद॒ ऊर्गु॑दु॒बंरो॑ मद्ध्य॒त औदु॑बंरीं मिनोति मद्ध्य॒त ए॒व प्र॒जाना॒मूर्जं॑ दधाति॒ तस्मा᳚न्- [  ] \newline

\textbf{Pada Paata} \newline

तेज॑स्काम॒स्येति॒ तेजः॑ - का॒म॒स्य॒ । मि॒नु॒या॒त् । त्रि॒वृतेति॑ त्रि-वृता᳚ । स्तोमे॑न । सम्मि॑त॒मिति॒ सं - मि॒त॒म् । तेजः॑ । त्रि॒वृदिति॑ त्रि -वृत् । ते॒ज॒स्वी । ए॒व । भ॒व॒ति॒ । एका॑दशछ॒दीत्येका॑दश - छ॒दि॒ । इ॒न्द्रि॒यका॑म॒स्येती᳚न्द्रि॒य - का॒म॒स्य॒ । एका॑दशाक्ष॒रेत्येका॑दश-अ॒क्ष॒रा॒ । त्रि॒ष्टुक् । इ॒न्द्रि॒यम् । त्रि॒ष्टुक् । इ॒न्द्रि॒या॒वी । ए॒व । भ॒व॒ति॒ । पञ्च॑दशछ॒दीति॒ पञ्च॑दश-छ॒दि॒ । भ्रातृ॑व्यवत॒ इति॒ भ्रातृ॑व्य - व॒तः॒ । प॒ञ्च॒द॒श इति॑ पञ्च - द॒शः । वज्रः॑ । भ्रातृ॑व्याभिभूत्या॒ इति॒ भ्रातृ॑व्य - अ॒भि॒भू॒त्यै॒ । स॒प्तद॑शछ॒दीति॑ स॒प्तद॑श - छ॒दि॒ । प्र॒जाका॑म॒स्येति॑ प्र॒जा - का॒म॒स्य॒ । स॒प्त॒द॒श इति॑ सप्त - द॒शः । प्र॒जाप॑ति॒रिति॑ प्र॒जा - प॒तिः॒ । प्र॒जाप॑ते॒रिति॑ प्र॒जा - प॒तेः॒ । आप्त्यै᳚ । एक॑विꣳशतिछ॒दीत्येक॑विꣳशति - छ॒दि॒ । प्र॒ति॒ष्ठाका॑म॒स्येति॑ प्रति॒ष्ठा - का॒म॒स्य॒ । ए॒क॒विꣳ॒॒श इत्ये॑क-विꣳ॒॒शः । स्तोमा॑नां । प्र॒ति॒ष्ठेति॑ प्रति - स्था । प्रति॑ष्ठित्या॒ इति॒ प्रति॑ - स्थि॒त्यै॒ । उ॒दर᳚म् । वै । सदः॑ । ऊर्क् । उ॒दु॒बंरः॑ । म॒द्ध्य॒तः । औदु॑बंरीम् । मि॒नो॒ति॒ । म॒द्ध्य॒तः । ए॒व । प्र॒जाना॒मिति॑ प्र - जाना᳚म् । ऊर्ज᳚म् । द॒धा॒ति॒ । तस्मा᳚त् ।  \newline


\textbf{Krama Paata} \newline

तेज॑स्कामस्य मिनुयात् । तेज॑स्काम॒स्येति॒ तेजः॑ - का॒म॒स्य॒ । मि॒नु॒या॒त् त्रि॒वृता᳚ । त्रि॒वृता॒ स्तोमे॑न । त्रि॒वृतेति॑ त्रि - वृता᳚ । स्तोमे॑न॒ सम्मि॑तम् । सम्मि॑त॒म् तेजः॑ । सम्मि॑त॒मिति॒ सम् - मि॒त॒म् । तेज॑स्त्रि॒वृत् । त्रि॒वृत् ते॑ज॒स्वी । त्रि॒वृदिति॑ त्रि - वृत् । ते॒ज॒स्व्ये॑व । ए॒व भ॑वति । भ॒व॒त्येका॑दशछदि । एका॑दशछदीन्द्रि॒यका॑मस्य । एका॑दशछ॒दीत्येका॑दश - छ॒दि॒ । इ॒न्द्रि॒यका॑म॒स्यैका॑दशाक्षरा । इ॒न्द्रि॒यका॑म॒स्येती᳚न्द्रि॒य - का॒म॒स्य॒ । एका॑दशाक्षरा त्रि॒ष्टुक् । एका॑दशाक्ष॒रेत्येका॑दश - अ॒क्ष॒रा॒ । त्रि॒ष्टुगि॑न्द्रि॒यम् । इ॒न्द्रि॒यम् त्रि॒ष्टुक् । त्रि॒ष्टुगि॑न्द्रिया॒वी । इ॒न्द्रि॒या॒व्ये॑व । ए॒व भ॑वति । भ॒व॒ति॒ पञ्च॑दशछदि । पञ्च॑दशछदि॒ भ्रातृ॑व्यवतः । पञ्च॑दशछ॒दीति॒ पञ्च॑दश - छ॒दि॒ । भ्रातृ॑व्यवतः पञ्चद॒शः । भ्रातृ॑व्यवत॒ इति॒ भ्रातृ॑व्य - व॒तः॒ । प॒ञ्च॒द॒शो वज्रः॑ । प॒ञ्च॒द॒श इति॑ पञ्च - द॒शः । वज्रो॒ भ्रातृ॑व्याभिभूत्यै । भ्रातृ॑व्याभिभूत्यै स॒प्तद॑शछदि । भातृ॑व्याभिभू॒त्या इति॒ भ्रातृ॑व्य - अ॒भि॒भू॒त्यै॒ । स॒प्तद॑शछदि प्र॒जाका॑मस्य । स॒प्तद॑शछ॒दीति॑ स॒प्तद॑श - छ॒दि॒ । प्र॒जाका॑मस्य सप्तद॒शः । प्र॒जाका॑म॒स्येति॑ प्र॒जा - का॒म॒स्य॒ । स॒प्त॒द॒शः प्र॒जाप॑तिः । स॒प्त॒द॒श इति॑ सप्त - द॒शः । प्र॒जाप॑तिः प्र॒जाप॑तेः । प्र॒जाप॑ति॒रिति॑ प्र॒जा - प॒तिः॒ । प्र॒जाप॑ते॒राप्त्यै᳚ । प्र॒जाप॑ते॒रिति॑ प्र॒जा - प॒तेः॒ । आप्त्या॒ एक॑विꣳशतिछदि । एक॑विꣳशतिछदि प्रति॒ष्ठाका॑मस्य । एक॑विꣳशतिछ॒दीत्येक॑विꣳशति - छ॒दि॒ । प्र॒ति॒ष्ठाका॑मस्यैकविꣳ॒॒शः । प्र॒ति॒ष्ठाका॑म॒स्येति॑ प्रति॒ष्ठा - का॒म॒स्य॒ । ए॒क॒विꣳ॒॒शः स्तोमा॑नाम् । ए॒क॒विꣳ॒॒श इत्ये॑क - विꣳ॒॒शः । स्तोमा॑नाम् प्रति॒ष्ठा । प्र॒ति॒ष्ठा प्रति॑ष्ठित्यै । प्र॒ति॒ष्ठेति॑ प्रति - स्था । प्रति॑ष्ठित्या उ॒दर᳚म् । प्रति॑ष्ठित्या॒ इति॒ प्रति॑ - स्थि॒त्यै॒ । उ॒दर॒म् ॅवै । वै सदः॑ । सद॒ ऊर्क् । ऊर्गु॑दु॒म्बरः॑ । उ॒दु॒म्बरो॑ मद्ध्य॒तः । म॒द्ध्य॒त औदु॑म्बरीम् । औदु॑म्बरीम् मिनोति । मि॒नो॒ति॒ म॒द्ध्य॒तः । म॒द्ध्य॒त ए॒व । ए॒व प्र॒जाना᳚म् । प्र॒जाना॒मूर्ज᳚म् । प्र॒जाना॒मिति॑ प्र - जाना᳚म् । ऊर्ज॑म् दधाति । द॒धा॒ति॒ तस्मा᳚त् । तस्मा᳚न् मद्ध्य॒तः \newline

\textbf{Jatai Paata} \newline

1. तेज॑स्कामस्य मिनुयान् मिनुया॒त् तेज॑स्कामस्य॒ तेज॑स्कामस्य मिनुयात् । \newline
2. तेज॑स्काम॒स्येति॒ तेजः॑ - का॒म॒स्य॒ । \newline
3. मि॒नु॒या॒त् त्रि॒वृता᳚ त्रि॒वृता॑ मिनुयान् मिनुयात् त्रि॒वृता᳚ । \newline
4. त्रि॒वृता॒ स्तोमे॑न॒ स्तोमे॑न त्रि॒वृता᳚ त्रि॒वृता॒ स्तोमे॑न । \newline
5. त्रि॒वृतेति॑ त्रि - वृता᳚ । \newline
6. स्तोमे॑न॒ सम्मि॑तꣳ॒॒ सम्मि॑तꣳ॒॒ स्तोमे॑न॒ स्तोमे॑न॒ सम्मि॑तम् । \newline
7. सम्मि॑त॒म् तेज॒ स्तेजः॒ सम्मि॑तꣳ॒॒ सम्मि॑त॒म् तेजः॑ । \newline
8. सम्मि॑त॒मिति॒ सं - मि॒त॒म् । \newline
9. तेज॑ स्त्रि॒वृत् त्रि॒वृत् तेज॒ स्तेज॑ स्त्रि॒वृत् । \newline
10. त्रि॒वृत् ते॑ज॒स्वी ते॑ज॒स्वी त्रि॒वृत् त्रि॒वृत् ते॑ज॒स्वी । \newline
11. त्रि॒वृदिति॑ त्रि - वृत् । \newline
12. ते॒ज॒ स्व्ये॑वैव ते॑ज॒स्वी ते॑ज॒ स्व्ये॑व । \newline
13. ए॒व भ॑वति भव त्ये॒वैव भ॑वति । \newline
14. भ॒व॒ त्येका॑दशछ॒ द्येका॑दशछदि भवति भव॒ त्येका॑दशछदि । \newline
15. एका॑दशछ दीन्द्रि॒यका॑म स्येन्द्रि॒यका॑म॒ स्यैका॑दशछ॒ द्येका॑दशछ दीन्द्रि॒यका॑मस्य । \newline
16. एका॑दशछ॒दीत्येका॑दश - छ॒दि॒ । \newline
17. इ॒न्द्रि॒यका॑म॒ स्यैका॑दशाक्ष॒ रैका॑दशाक्ष रेन्द्रि॒यका॑म स्येन्द्रि॒यका॑म॒ स्यैका॑दशाक्षरा । \newline
18. इ॒न्द्रि॒यका॑म॒स्येती᳚न्द्रि॒य - का॒म॒स्य॒ । \newline
19. एका॑दशाक्षरा त्रि॒ष्टुक् त्रि॒ष्टुगेका॑दशाक्ष॒ रैका॑दशाक्षरा त्रि॒ष्टुक् । \newline
20. एका॑दशाक्ष॒रेत्येका॑दश - अ॒क्ष॒रा॒ । \newline
21. त्रि॒ष्टु गि॑न्द्रि॒य मि॑न्द्रि॒यम् त्रि॒ष्टुक् त्रि॒ष्टु गि॑न्द्रि॒यम् । \newline
22. इ॒न्द्रि॒यम् त्रि॒ष्टुक् त्रि॒ष्टु गि॑न्द्रि॒य मि॑न्द्रि॒यम् त्रि॒ष्टुक् । \newline
23. त्रि॒ष्टु गि॑न्द्रिया॒ वीन्द्रि॑या॒वी त्रि॒ष्टुक् त्रि॒ष्टु गि॑न्द्रिया॒वी । \newline
24. इ॒न्द्रि॒या॒ व्ये॑वै वेन्द्रि॑या॒वी न्द्रि॑या॒ व्ये॑व । \newline
25. ए॒व भ॑वति भव त्ये॒वैव भ॑वति । \newline
26. भ॒व॒ति॒ पञ्च॑दशछदि॒ पञ्च॑दशछदि भवति भवति॒ पञ्च॑दशछदि । \newline
27. पञ्च॑दशछदि॒ भ्रातृ॑व्यवतो॒ भ्रातृ॑व्यवतः॒ पञ्च॑दशछदि॒ पञ्च॑दशछदि॒ भ्रातृ॑व्यवतः । \newline
28. पञ्च॑दशछ॒दीति॒ पञ्च॑दश - छ॒दि॒ । \newline
29. भ्रातृ॑व्यवतः पञ्चद॒शः प॑ञ्चद॒शो भ्रातृ॑व्यवतो॒ भ्रातृ॑व्यवतः पञ्चद॒शः । \newline
30. भ्रातृ॑व्यवत॒ इति॒ भ्रातृ॑व्य - व॒तः॒ । \newline
31. प॒ञ्च॒द॒शो वज्रो॒ वज्रः॑ पञ्चद॒शः प॑ञ्चद॒शो वज्रः॑ । \newline
32. प॒ञ्च॒द॒श इति॑ पञ्च - द॒शः । \newline
33. वज्रो॒ भ्रातृ॑व्याभिभूत्यै॒ भ्रातृ॑व्याभिभूत्यै॒ वज्रो॒ वज्रो॒ भ्रातृ॑व्याभिभूत्यै । \newline
34. भ्रातृ॑व्याभिभूत्यै स॒प्तद॑शछदि स॒प्तद॑शछदि॒ भ्रातृ॑व्याभिभूत्यै॒ भ्रातृ॑व्याभिभूत्यै स॒प्तद॑शछदि । \newline
35. भ्रातृ॑व्याभिभूत्या॒ इति॒ भ्रातृ॑व्य - अ॒भि॒भू॒त्यै॒ । \newline
36. स॒प्तद॑शछदि प्र॒जाका॑मस्य प्र॒जाका॑मस्य स॒प्तद॑शछदि स॒प्तद॑शछदि प्र॒जाका॑मस्य । \newline
37. स॒प्तद॑शछ॒दीति॑ स॒प्तद॑श - छ॒दि॒ । \newline
38. प्र॒जाका॑मस्य सप्तद॒शः स॑प्तद॒शः प्र॒जाका॑मस्य प्र॒जाका॑मस्य सप्तद॒शः । \newline
39. प्र॒जाका॑म॒स्येति॑ प्र॒जा - का॒म॒स्य॒ । \newline
40. स॒प्त॒द॒शः प्र॒जाप॑तिः प्र॒जाप॑तिः सप्तद॒शः स॑प्तद॒शः प्र॒जाप॑तिः । \newline
41. स॒प्त॒द॒श इति॑ सप्त - द॒शः । \newline
42. प्र॒जाप॑तिः प्र॒जाप॑तेः प्र॒जाप॑तेः प्र॒जाप॑तिः प्र॒जाप॑तिः प्र॒जाप॑तेः । \newline
43. प्र॒जाप॑ति॒रिति॑ प्र॒जा - प॒तिः॒ । \newline
44. प्र॒जाप॑ते॒ राप्त्या॒ आप्त्यै᳚ प्र॒जाप॑तेः प्र॒जाप॑ते॒ राप्त्यै᳚ । \newline
45. प्र॒जाप॑ते॒रिति॑ प्र॒जा - प॒तेः॒ । \newline
46. आप्त्या॒ एक॑विꣳशतिछ॒ द्येक॑विꣳशतिछ॒ द्याप्त्या॒ आप्त्या॒ एक॑विꣳशतिछदि । \newline
47. एक॑विꣳशतिछदि प्रति॒ष्ठाका॑मस्य प्रति॒ष्ठाका॑म॒ स्यैक॑विꣳशतिछ॒ द्येक॑विꣳशतिछदि प्रति॒ष्ठाका॑मस्य । \newline
48. एक॑विꣳशतिछ॒दीत्येक॑विꣳशति - छ॒दि॒ । \newline
49. प्र॒ति॒ष्ठाका॑म स्यैकविꣳ॒॒श ए॑कविꣳ॒॒शः प्र॑ति॒ष्ठाका॑मस्य प्रति॒ष्ठाका॑म स्यैकविꣳ॒॒शः । \newline
50. प्र॒ति॒ष्ठाका॑म॒स्येति॑ प्रति॒ष्ठा - का॒म॒स्य॒ । \newline
51. ए॒क॒विꣳ॒॒शः स्तोमा॑नाꣳ॒॒ स्तोमा॑ना मेकविꣳ॒॒श ए॑कविꣳ॒॒शः स्तोमा॑नां । \newline
52. ए॒क॒विꣳ॒॒श इत्ये॑क - विꣳ॒॒शः । \newline
53. स्तोमा॑नां प्रति॒ष्ठा प्र॑ति॒ष्ठा स्तोमा॑नाꣳ॒॒ स्तोमा॑नां प्रति॒ष्ठा । \newline
54. प्र॒ति॒ष्ठा प्रति॑ष्ठित्यै॒ प्रति॑ष्ठित्यै प्रति॒ष्ठा प्र॑ति॒ष्ठा प्रति॑ष्ठित्यै । \newline
55. प्र॒ति॒ष्ठेति॑ प्रति - स्था । \newline
56. प्रति॑ष्ठित्या उ॒दर॑ मु॒दर॒म् प्रति॑ष्ठित्यै॒ प्रति॑ष्ठित्या उ॒दर᳚म् । \newline
57. प्रति॑ष्ठित्या॒ इति॒ प्रति॑ - स्थि॒त्यै॒ । \newline
58. उ॒दरं॒ ॅवै वा उ॒दर॑ मु॒दरं॒ ॅवै । \newline
59. वै सदः॒ सदो॒ वै वै सदः॑ । \newline
60. सद॒ ऊर् गूर्ख् सदः॒ सद॒ ऊर्क् । \newline
61. ऊर् गु॒दुंबर॑ उ॒दुंबर॒ ऊर् गूर् गु॒दुंबरः॑ । \newline
62. उ॒दुंबरो॑ मद्ध्य॒तो म॑द्ध्य॒त उ॒दुंबर॑ उ॒दुंबरो॑ मद्ध्य॒तः । \newline
63. म॒द्ध्य॒त औदुं॑बरी॒ मौदुं॑बरीम् मद्ध्य॒तो म॑द्ध्य॒त औदुं॑बरीम् । \newline
64. औदुं॑बरीम् मिनोति मिनो॒ त्यौदुं॑बरी॒ मौदुं॑बरीम् मिनोति । \newline
65. मि॒नो॒ति॒ म॒द्ध्य॒तो म॑द्ध्य॒तो मि॑नोति मिनोति मद्ध्य॒तः । \newline
66. म॒द्ध्य॒त ए॒वैव म॑द्ध्य॒तो म॑द्ध्य॒त ए॒व । \newline
67. ए॒व प्र॒जाना᳚म् प्र॒जाना॑ मे॒वैव प्र॒जाना᳚म् । \newline
68. प्र॒जाना॒ मूर्ज॒ मूर्ज॑म् प्र॒जाना᳚म् प्र॒जाना॒ मूर्ज᳚म् । \newline
69. प्र॒जाना॒मिति॑ प्र - जाना᳚म् । \newline
70. ऊर्ज॑म् दधाति दधा॒ त्यूर्ज॒ मूर्ज॑म् दधाति । \newline
71. द॒धा॒ति॒ तस्मा॒त् तस्मा᳚द् दधाति दधाति॒ तस्मा᳚त् । \newline
72. तस्मा᳚न् मद्ध्य॒तो म॑द्ध्य॒त स्तस्मा॒त् तस्मा᳚न् मद्ध्य॒तः । \newline

\textbf{Ghana Paata } \newline

1. तेज॑स्कामस्य मिनुयान् मिनुया॒त् तेज॑स्कामस्य॒ तेज॑स्कामस्य मिनुयात् त्रि॒वृता᳚ त्रि॒वृता॑ मिनुया॒त् तेज॑स्कामस्य॒ तेज॑स्कामस्य मिनुयात् त्रि॒वृता᳚ । \newline
2. तेज॑स्काम॒स्येति॒ तेजः॑ - का॒म॒स्य॒ । \newline
3. मि॒नु॒या॒त् त्रि॒वृता᳚ त्रि॒वृता॑ मिनुयान् मिनुयात् त्रि॒वृता॒ स्तोमे॑न॒ स्तोमे॑न त्रि॒वृता॑ मिनुयान् मिनुयात् त्रि॒वृता॒ स्तोमे॑न । \newline
4. त्रि॒वृता॒ स्तोमे॑न॒ स्तोमे॑न त्रि॒वृता᳚ त्रि॒वृता॒ स्तोमे॑न॒ सम्मि॑तꣳ॒॒ सम्मि॑तꣳ॒॒ स्तोमे॑न त्रि॒वृता᳚ त्रि॒वृता॒ स्तोमे॑न॒ सम्मि॑तम् । \newline
5. त्रि॒वृतेति॑ त्रि - वृता᳚ । \newline
6. स्तोमे॑न॒ सम्मि॑तꣳ॒॒ सम्मि॑तꣳ॒॒ स्तोमे॑न॒ स्तोमे॑न॒ सम्मि॑त॒म् तेज॒ स्तेजः॒ सम्मि॑तꣳ॒॒ स्तोमे॑न॒ स्तोमे॑न॒ सम्मि॑त॒म् तेजः॑ । \newline
7. सम्मि॑त॒म् तेज॒ स्तेजः॒ सम्मि॑तꣳ॒॒ सम्मि॑त॒म् तेज॑ स्त्रि॒वृत् त्रि॒वृत् तेजः॒ सम्मि॑तꣳ॒॒ सम्मि॑त॒म् तेज॑ स्त्रि॒वृत् । \newline
8. सम्मि॑त॒मिति॒ सं - मि॒त॒म् । \newline
9. तेज॑ स्त्रि॒वृत् त्रि॒वृत् तेज॒ स्तेज॑ स्त्रि॒वृत् ते॑ज॒स्वी ते॑ज॒स्वी त्रि॒वृत् तेज॒ स्तेज॑ स्त्रि॒वृत् ते॑ज॒स्वी । \newline
10. त्रि॒वृत् ते॑ज॒स्वी ते॑ज॒स्वी त्रि॒वृत् त्रि॒वृत् ते॑ज॒ स्व्ये॑वैव ते॑ज॒स्वी त्रि॒वृत् त्रि॒वृत् ते॑ज॒ स्व्ये॑व । \newline
11. त्रि॒वृदिति॑ त्रि - वृत् । \newline
12. ते॒ज॒ स्व्ये॑वैव ते॑ज॒स्वी ते॑ज॒ स्व्ये॑व भ॑वति भव त्ये॒व ते॑ज॒स्वी ते॑ज॒ स्व्ये॑व भ॑वति । \newline
13. ए॒व भ॑वति भव त्ये॒वैव भ॑व॒ त्येका॑दशछ॒ द्येका॑दशछदि भव त्ये॒वैव भ॑व॒ 
त्येका॑दशछदि । \newline
14. भ॒व॒ त्येका॑दशछ॒ द्येका॑दशछदि भवति भव॒ त्येका॑दशछ दीन्द्रि॒यका॑म स्येन्द्रि॒यका॑म॒ स्यैका॑दशछदि भवति भव॒ त्येका॑दशछ दीन्द्रि॒यका॑मस्य । \newline
15. एका॑दशछ दीन्द्रि॒यका॑म स्येन्द्रि॒यका॑म॒ स्यैका॑दशछ॒ द्येका॑दशछ दीन्द्रि॒यका॑म॒ स्यैका॑दशाक्ष॒ रैका॑दशाक्ष रेन्द्रि॒यका॑म॒ स्यैका॑दशछ॒ द्येका॑दशछ दीन्द्रि॒यका॑म॒ स्यैका॑दशाक्षरा । \newline
16. एका॑दशछ॒दीत्येका॑दश - छ॒दि॒ । \newline
17. इ॒न्द्रि॒यका॑म॒ स्यैका॑दशाक्ष॒ रैका॑दशाक्ष रेन्द्रि॒यका॑म स्येन्द्रि॒यका॑म॒ स्यैका॑दशाक्षरा त्रि॒ष्टुक् त्रि॒ष्टु गेका॑दशाक्ष रेन्द्रि॒यका॑म स्येन्द्रि॒यका॑म॒ स्यैका॑दशाक्षरा त्रि॒ष्टुक् । \newline
18. इ॒न्द्रि॒यका॑म॒स्येती᳚न्द्रि॒य - का॒म॒स्य॒ । \newline
19. एका॑दशाक्षरा त्रि॒ष्टुक् त्रि॒ष्टु गेका॑दशाक्ष॒ रैका॑दशाक्षरा त्रि॒ष्टु गि॑न्द्रि॒य मि॑न्द्रि॒यम् त्रि॒ष्टु गेका॑दशाक्ष॒ रैका॑दशाक्षरा त्रि॒ष्टु गि॑न्द्रि॒यम् । \newline
20. एका॑दशाक्ष॒रेत्येका॑दश - अ॒क्ष॒रा॒ । \newline
21. त्रि॒ष्टु गि॑न्द्रि॒य मि॑न्द्रि॒यम् त्रि॒ष्टुक् त्रि॒ष्टु गि॑न्द्रि॒यम् त्रि॒ष्टुक् त्रि॒ष्टु गि॑न्द्रि॒यम् त्रि॒ष्टुक् त्रि॒ष्टु गि॑न्द्रि॒यम् त्रि॒ष्टुक् । \newline
22. इ॒न्द्रि॒यम् त्रि॒ष्टुक् त्रि॒ष्टु गि॑न्द्रि॒य मि॑न्द्रि॒यम् त्रि॒ष्टु गि॑न्द्रिया॒वी न्द्रि॑या॒वी त्रि॒ष्टु गि॑न्द्रि॒य मि॑न्द्रि॒यम् त्रि॒ष्टुगि॑न्द्रिया॒वी । \newline
23. त्रि॒ष्टु गि॑न्द्रिया॒वी न्द्रि॑या॒वी त्रि॒ष्टुक् त्रि॒ष्टु गि॑न्द्रिया॒ व्ये॑वै वेन्द्रि॑या॒वी त्रि॒ष्टुक् त्रि॒ष्टु गि॑न्द्रिया॒ व्ये॑व । \newline
24. इ॒न्द्रि॒या॒ व्ये॑वै वेन्द्रि॑या॒वीन्द्रि॑या॒ व्ये॑व भ॑वति भव त्ये॒वेन्द्रि॑या॒ वीन्द्रि॑या॒ व्ये॑व भ॑वति । \newline
25. ए॒व भ॑वति भव त्ये॒वैव भ॑वति॒ पञ्च॑दशछदि॒ पञ्च॑दशछदि भव त्ये॒वैव भ॑वति॒ पञ्च॑दशछदि । \newline
26. भ॒व॒ति॒ पञ्च॑दशछदि॒ पञ्च॑दशछदि भवति भवति॒ पञ्च॑दशछदि॒ भ्रातृ॑व्यवतो॒ भ्रातृ॑व्यवतः॒ पञ्च॑दशछदि भवति भवति॒ पञ्च॑दशछदि॒ भ्रातृ॑व्यवतः । \newline
27. पञ्च॑दशछदि॒ भ्रातृ॑व्यवतो॒ भ्रातृ॑व्यवतः॒ पञ्च॑दशछदि॒ पञ्च॑दशछदि॒ भ्रातृ॑व्यवतः पञ्चद॒शः प॑ञ्चद॒शो भ्रातृ॑व्यवतः॒ पञ्च॑दशछदि॒ पञ्च॑दशछदि॒ भ्रातृ॑व्यवतः पञ्चद॒शः । \newline
28. पञ्च॑दशछ॒दीति॒ पञ्च॑दश - छ॒दि॒ । \newline
29. भ्रातृ॑व्यवतः पञ्चद॒शः प॑ञ्चद॒शो भ्रातृ॑व्यवतो॒ भ्रातृ॑व्यवतः पञ्चद॒शो वज्रो॒ वज्रः॑ पञ्चद॒शो भ्रातृ॑व्यवतो॒ भ्रातृ॑व्यवतः पञ्चद॒शो वज्रः॑ । \newline
30. भ्रातृ॑व्यवत॒ इति॒ भ्रातृ॑व्य - व॒तः॒ । \newline
31. प॒ञ्च॒द॒शो वज्रो॒ वज्रः॑ पञ्चद॒शः प॑ञ्चद॒शो वज्रो॒ भ्रातृ॑व्याभिभूत्यै॒ भ्रातृ॑व्याभिभूत्यै॒ वज्रः॑ पञ्चद॒शः प॑ञ्चद॒शो वज्रो॒ भ्रातृ॑व्याभिभूत्यै । \newline
32. प॒ञ्च॒द॒श इति॑ पञ्च - द॒शः । \newline
33. वज्रो॒ भ्रातृ॑व्याभिभूत्यै॒ भ्रातृ॑व्याभिभूत्यै॒ वज्रो॒ वज्रो॒ भ्रातृ॑व्याभिभूत्यै स॒प्तद॑शछदि स॒प्तद॑शछदि॒ भ्रातृ॑व्याभिभूत्यै॒ वज्रो॒ वज्रो॒ भ्रातृ॑व्याभिभूत्यै स॒प्तद॑शछदि । \newline
34. भ्रातृ॑व्याभिभूत्यै स॒प्तद॑शछदि स॒प्तद॑शछदि॒ भ्रातृ॑व्याभिभूत्यै॒ भ्रातृ॑व्याभिभूत्यै स॒प्तद॑शछदि प्र॒जाका॑मस्य प्र॒जाका॑मस्य स॒प्तद॑शछदि॒ भ्रातृ॑व्याभिभूत्यै॒ भ्रातृ॑व्याभिभूत्यै स॒प्तद॑शछदि प्र॒जाका॑मस्य । \newline
35. भ्रातृ॑व्याभिभूत्या॒ इति॒ भ्रातृ॑व्य - अ॒भि॒भू॒त्यै॒ । \newline
36. स॒प्तद॑शछदि प्र॒जाका॑मस्य प्र॒जाका॑मस्य स॒प्तद॑शछदि स॒प्तद॑शछदि प्र॒जाका॑मस्य सप्तद॒शः स॑प्तद॒शः प्र॒जाका॑मस्य स॒प्तद॑शछदि स॒प्तद॑शछदि प्र॒जाका॑मस्य सप्तद॒शः । \newline
37. स॒प्तद॑शछ॒दीति॑ स॒प्तद॑श - छ॒दि॒ । \newline
38. प्र॒जाका॑मस्य सप्तद॒शः स॑प्तद॒शः प्र॒जाका॑मस्य प्र॒जाका॑मस्य सप्तद॒शः प्र॒जाप॑तिः प्र॒जाप॑तिः सप्तद॒शः प्र॒जाका॑मस्य प्र॒जाका॑मस्य सप्तद॒शः प्र॒जाप॑तिः । \newline
39. प्र॒जाका॑म॒स्येति॑ प्र॒जा - का॒म॒स्य॒ । \newline
40. स॒प्त॒द॒शः प्र॒जाप॑तिः प्र॒जाप॑तिः सप्तद॒शः स॑प्तद॒शः प्र॒जाप॑तिः प्र॒जाप॑तेः प्र॒जाप॑तेः प्र॒जाप॑तिः सप्तद॒शः स॑प्तद॒शः प्र॒जाप॑तिः प्र॒जाप॑तेः । \newline
41. स॒प्त॒द॒श इति॑ सप्त - द॒शः । \newline
42. प्र॒जाप॑तिः प्र॒जाप॑तेः प्र॒जाप॑तेः प्र॒जाप॑तिः प्र॒जाप॑तिः प्र॒जाप॑ते॒ राप्त्या॒ आप्त्यै᳚ प्र॒जाप॑तेः प्र॒जाप॑तिः प्र॒जाप॑तिः प्र॒जाप॑ते॒ राप्त्यै᳚ । \newline
43. प्र॒जाप॑ति॒रिति॑ प्र॒जा - प॒तिः॒ । \newline
44. प्र॒जाप॑ते॒ राप्त्या॒ आप्त्यै᳚ प्र॒जाप॑तेः प्र॒जाप॑ते॒ राप्त्या॒ एक॑विꣳशतिछ॒ द्येक॑विꣳशतिछ॒ द्याप्त्यै᳚ प्र॒जाप॑तेः प्र॒जाप॑ते॒ राप्त्या॒ एक॑विꣳशतिछदि । \newline
45. प्र॒जाप॑ते॒रिति॑ प्र॒जा - प॒तेः॒ । \newline
46. आप्त्या॒ एक॑विꣳशतिछ॒ द्येक॑विꣳशतिछ॒ द्याप्त्या॒ आप्त्या॒ एक॑विꣳशतिछदि प्रति॒ष्ठाका॑मस्य प्रति॒ष्ठाका॑म॒ स्यैक॑विꣳशतिछ॒ द्याप्त्या॒ आप्त्या॒ एक॑विꣳशतिछदि प्रति॒ष्ठाका॑मस्य । \newline
47. एक॑विꣳशतिछदि प्रति॒ष्ठाका॑मस्य प्रति॒ष्ठाका॑म॒ स्यैक॑विꣳशतिछ॒ द्येक॑विꣳशतिछदि प्रति॒ष्ठाका॑म स्यैकविꣳ॒॒श ए॑कविꣳ॒॒शः प्र॑ति॒ष्ठाका॑म॒ स्यैक॑विꣳशतिछ॒ द्येक॑विꣳशतिछदि प्रति॒ष्ठाका॑म स्यैकविꣳ॒॒शः । \newline
48. एक॑विꣳशतिछ॒दीत्येक॑विꣳशति - छ॒दि॒ । \newline
49. प्र॒ति॒ष्ठाका॑म स्यैकविꣳ॒॒श ए॑कविꣳ॒॒शः प्र॑ति॒ष्ठाका॑मस्य प्रति॒ष्ठाका॑म स्यैकविꣳ॒॒शः स्तोमा॑नाꣳ॒॒ स्तोमा॑ना मेकविꣳ॒॒शः प्र॑ति॒ष्ठाका॑मस्य प्रति॒ष्ठाका॑म स्यैकविꣳ॒॒शः स्तोमा॑नां । \newline
50. प्र॒ति॒ष्ठाका॑म॒स्येति॑ प्रति॒ष्ठा - का॒म॒स्य॒ । \newline
51. ए॒क॒विꣳ॒॒शः स्तोमा॑नाꣳ॒॒ स्तोमा॑ना मेकविꣳ॒॒श ए॑कविꣳ॒॒शः स्तोमा॑नां प्रति॒ष्ठा प्र॑ति॒ष्ठा स्तोमा॑ना मेकविꣳ॒॒श ए॑कविꣳ॒॒शः स्तोमा॑नां प्रति॒ष्ठा । \newline
52. ए॒क॒विꣳ॒॒श इत्ये॑क - विꣳ॒॒शः । \newline
53. स्तोमा॑नां प्रति॒ष्ठा प्र॑ति॒ष्ठा स्तोमा॑नाꣳ॒॒ स्तोमा॑नां प्रति॒ष्ठा प्रति॑ष्ठित्यै॒ प्रति॑ष्ठित्यै प्रति॒ष्ठा स्तोमा॑नाꣳ॒॒ स्तोमा॑नां प्रति॒ष्ठा प्रति॑ष्ठित्यै । \newline
54. प्र॒ति॒ष्ठा प्रति॑ष्ठित्यै॒ प्रति॑ष्ठित्यै प्रति॒ष्ठा प्र॑ति॒ष्ठा प्रति॑ष्ठित्या उ॒दर॑ मु॒दर॒म् प्रति॑ष्ठित्यै प्रति॒ष्ठा प्र॑ति॒ष्ठा प्रति॑ष्ठित्या उ॒दर᳚म् । \newline
55. प्र॒ति॒ष्ठेति॑ प्रति - स्था । \newline
56. प्रति॑ष्ठित्या उ॒दर॑ मु॒दर॒म् प्रति॑ष्ठित्यै॒ प्रति॑ष्ठित्या उ॒दरं॒ ॅवै वा उ॒दर॒म् प्रति॑ष्ठित्यै॒ प्रति॑ष्ठित्या उ॒दरं॒ ॅवै । \newline
57. प्रति॑ष्ठित्या॒ इति॒ प्रति॑ - स्थि॒त्यै॒ । \newline
58. उ॒दरं॒ ॅवै वा उ॒दर॑ मु॒दरं॒ ॅवै सदः॒ सदो॒ वा उ॒दर॑ मु॒दरं॒ ॅवै सदः॑ । \newline
59. वै सदः॒ सदो॒ वै वै सद॒ ऊर् गूर्ख् सदो॒ वै वै सद॒ ऊर्क् । \newline
60. सद॒ ऊर् गूर्ख् सदः॒ सद॒ ऊर्गु॒दुंबर॑ उ॒दुंबर॒ ऊर्ख् सदः॒ सद॒ ऊर्गु॒दुंबरः॑ । \newline
61. ऊर्गु॒दुंबर॑ उ॒दुंबर॒ ऊर् गूर् गु॒दुंबरो॑ मद्ध्य॒तो म॑द्ध्य॒त उ॒दुंबर॒ ऊर् गूर् गु॒दुंबरो॑ मद्ध्य॒तः । \newline
62. उ॒दुंबरो॑ मद्ध्य॒तो म॑द्ध्य॒त उ॒दुंबर॑ उ॒दुंबरो॑ मद्ध्य॒त औदुं॑बरी॒ मौदुं॑बरीम् मद्ध्य॒त उ॒दुंबर॑ उ॒दुंबरो॑ मद्ध्य॒त औदुं॑बरीम् । \newline
63. म॒द्ध्य॒त औदुं॑बरी॒ मौदुं॑बरीम् मद्ध्य॒तो म॑द्ध्य॒त औदुं॑बरीम् मिनोति मिनो॒ त्यौदुं॑बरीम् मद्ध्य॒तो म॑द्ध्य॒त औदुं॑बरीम् मिनोति । \newline
64. औदुं॑बरीम् मिनोति मिनो॒ त्यौदुं॑बरी॒ मौदुं॑बरीम् मिनोति मद्ध्य॒तो म॑द्ध्य॒तो मि॑नो॒ त्यौदुं॑बरी॒ मौदुं॑बरीम् मिनोति मद्ध्य॒तः । \newline
65. मि॒नो॒ति॒ म॒द्ध्य॒तो म॑द्ध्य॒तो मि॑नोति मिनोति मद्ध्य॒त ए॒वैव म॑द्ध्य॒तो मि॑नोति मिनोति मद्ध्य॒त ए॒व । \newline
66. म॒द्ध्य॒त ए॒वैव म॑द्ध्य॒तो म॑द्ध्य॒त ए॒व प्र॒जाना᳚म् प्र॒जाना॑ मे॒व म॑द्ध्य॒तो म॑द्ध्य॒त ए॒व प्र॒जाना᳚म् । \newline
67. ए॒व प्र॒जाना᳚म् प्र॒जाना॑ मे॒वैव प्र॒जाना॒ मूर्ज॒ मूर्ज॑म् प्र॒जाना॑ मे॒वैव प्र॒जाना॒ मूर्ज᳚म् । \newline
68. प्र॒जाना॒ मूर्ज॒ मूर्ज॑म् प्र॒जाना᳚म् प्र॒जाना॒ मूर्ज॑म् दधाति दधा॒ त्यूर्ज॑म् प्र॒जाना᳚म् प्र॒जाना॒ मूर्ज॑म् दधाति । \newline
69. प्र॒जाना॒मिति॑ प्र - जाना᳚म् । \newline
70. ऊर्ज॑म् दधाति दधा॒ त्यूर्ज॒ मूर्ज॑म् दधाति॒ तस्मा॒त् तस्मा᳚द् दधा॒ त्यूर्ज॒ मूर्ज॑म् दधाति॒ तस्मा᳚त् । \newline
71. द॒धा॒ति॒ तस्मा॒त् तस्मा᳚द् दधाति दधाति॒ तस्मा᳚न् मद्ध्य॒तो म॑द्ध्य॒त स्तस्मा᳚द् दधाति दधाति॒ तस्मा᳚न् मद्ध्य॒तः । \newline
72. तस्मा᳚न् मद्ध्य॒तो म॑द्ध्य॒त स्तस्मा॒त् तस्मा᳚न् मद्ध्य॒त ऊ॒र्जोर्जा म॑द्ध्य॒त स्तस्मा॒त् तस्मा᳚न् मद्ध्य॒त ऊ॒र्जा । \newline
\pagebreak
\markright{ TS 6.2.10.7  \hfill https://www.vedavms.in \hfill}

\section{ TS 6.2.10.7 }

\textbf{TS 6.2.10.7 } \newline
\textbf{Samhita Paata} \newline

मद्ध्य॒त ऊ॒र्जा भु॑ञ्जते यजमानलो॒के वै दक्षि॑णानि छ॒दीꣳषि॑ भ्रातृव्यलो॒क उत्त॑राणि॒ दक्षि॑णा॒न्युत्त॑राणि करोति॒ यज॑मान-मे॒वा-य॑जमाना॒दुत्त॑रं करोति॒ तस्मा॒द् यज॑मा॒नोऽय॑जमाना॒दुत्त॑रो ऽन्तर्व॒र्तान् क॑रोति॒ व्यावृ॑त्त्यै॒ तस्मा॒दर॑ण्यं प्र॒जा उप॑ जीवन्ति॒ परि॑ त्वा गिर्वणो॒ गिर॒ इत्या॑ह यथाय॒जुरे॒वैतदिन्द्र॑स्य॒ स्यूर॒सीन्द्र॑स्य ध्रु॒वम॒सीत्या॑है॒न्द्रꣳ हि दे॒वत॑या॒ सदो॒ ( ) यं प्र॑थ॒मं ग्र॒न्थिं ग्र॑थ्नी॒याद्यत् तं न वि॑स्रꣳ॒॒ सये॒दमे॑हेनाद्ध्व॒र्युः प्रमी॑येत॒ तस्मा॒थ् स वि॒स्रस्यः॑ ॥ \newline

\textbf{Pada Paata} \newline

म॒द्ध्य॒तः । ऊ॒र्जा । भु॒ञ्ज॒ते॒ । य॒ज॒मा॒न॒लो॒क इति॑ यजमान - लो॒के । वै । दक्षि॑णानि । छ॒दीꣳषि॑ । भ्रा॒तृ॒व्य॒लो॒क इति॑ भ्रातृव्य - लो॒के । उत्त॑रा॒णीत्युत् - त॒रा॒णि॒ । दक्षि॑णानि । उत्त॑रा॒णीत्युत् - त॒रा॒णि॒ । क॒रो॒ति॒ । यज॑मानम् । ए॒व । अय॑जमानात् । उत्त॑र॒मित्युत् - त॒र॒म् । क॒रो॒ति॒ । तस्मा᳚त् । यज॑मानः । अय॑जमानात् । उत्त॑र॒ इत्युत् - त॒रः॒ । अ॒न्त॒र्व॒र्तानित्य॑न्तः - व॒र्तान् । क॒रो॒ति॒ । व्यावृ॑त्त्या॒ इति॑ वि-आवृ॑त्त्यै । तस्मा᳚त् । अर॑ण्यम् । प्र॒जा इति॑ प्र - जाः । उपेति॑ । जी॒व॒न्ति॒ । परीति॑ । त्वा॒ । गि॒र्व॒णः॒ । गिरः॑ । इति॑ । आ॒ह॒ । य॒था॒य॒जुरिति॑ यथा - य॒जुः । ए॒व । ए॒तत् । इन्द्र॑स्य । स्यूः । अ॒सि॒ । इन्द्र॑स्य । ध्रु॒वम् । अ॒सि॒ । इति॑ । आ॒ह॒ । ऐ॒न्द्रम् । हि । दे॒वत॑या । सदः॑ ( ) । यम् । प्र॒थ॒मम् । ग्र॒न्थिम् । ग्र॒थ्नी॒यात् । यत् । तम् । न । वि॒स्रꣳ॒॒सये॒दिति॑ वि - स्रꣳ॒॒सये᳚त् । अमे॑हेन् । अ॒द्ध्व॒र्युः । प्रेति॑ । मी॒ये॒त॒ । तस्मा᳚त् । सः । वि॒स्रस्य॒ इति॑ वि - स्रस्यः॑ ॥  \newline


\textbf{Krama Paata} \newline

म॒द्ध्य॒त ऊ॒र्जा । ऊ॒र्जा भु॑ञ्जते । भु॒ञ्ज॒ते॒ य॒ज॒मा॒न॒लो॒के । य॒ज॒मा॒न॒लो॒के वै । य॒ज॒मा॒न॒लो॒क इति॑ यजमान - लो॒के । वै दक्षि॑णानि । दक्षि॑णानि छ॒दीꣳषि॑ । छ॒दीꣳषि॑ भ्रातृव्यलो॒के । भ्रा॒तृ॒व्य॒लो॒क उत्त॑राणि । भ्रा॒तृ॒व्य॒लो॒क इति॑ भ्रातृव्य - लो॒के । उत्त॑राणि॒ दक्षि॑णानि । उत्त॑रा॒णीत्युत् - त॒रा॒णि॒ । दक्षि॑णा॒न्युत्त॑राणि । उत्त॑राणि करोति । उत्ता॑रा॒णीत्युत् - त॒रा॒णि॒ । क॒रो॒ति॒ यज॑मानम् । यज॑मानमे॒व । ए॒वाय॑जमानात् । अय॑जमाना॒दुत्त॑रम् । उत्त॑रम् करोति । उत्त॑र॒मित्युत् - त॒र॒म् । क॒रो॒ति॒ तस्मा᳚त् । तस्मा॒द् यज॑मानः । यज॑मा॒नोऽय॑जमानात् । अय॑जमाना॒दुत्त॑रः । उत्त॑रोऽन्तर्व॒र्तान् । उत्त॑र॒ इत्युत् - त॒रः॒ । अ॒न्त॒र्व॒र्तान् क॑रोति । अ॒न्त॒र्व॒र्तानित्य॑न्तः - व॒र्तान् । क॒रो॒ति॒ व्यावृ॑त्त्यै । व्यावृ॑त्त्यै॒ तस्मा᳚त् । व्यावृ॑त्त्या॒ इति॑ वि - आवृ॑त्त्यै । तस्मा॒दर॑ण्यम् । अर॑ण्यम् प्र॒जाः । प्र॒जा उप॑ । प्र॒जा इति॑ प्र - जाः । उप॑ जीवन्ति । जी॒व॒न्ति॒ परि॑ । परि॑ त्वा । त्वा॒ गि॒र्व॒णः॒ । गि॒र्व॒णो॒ गिरः॑ । गिर॒ इति॑ । इत्या॑ह । आ॒ह॒ य॒था॒य॒जुः । य॒था॒य॒जुरे॒व । य॒था॒य॒जुरिति॑ यथा - य॒जुः । ए॒वैतत् । ए॒तदिन्द्र॑स्य । इन्द्र॑स्य॒ स्यूः । स्यूर॑सि । अ॒सीन्द्र॑स्य । इन्द्र॑स्य ध्रु॒वम् । ध्रु॒वम॑सि । अ॒सीति॑ । इत्या॑ह । आ॒है॒न्द्रम् । ऐ॒न्द्रꣳ हि । हि दे॒वत॑या । दे॒वत॑या॒ सदः॑ ( ) । सदो॒ यम् । यम् प्र॑थ॒मम् । प्र॒थ॒मम् ग्र॒न्थिम् । ग्र॒न्थिम् ग्र॑थ्नी॒यात् । ग्र॒थ्नी॒याद् यत् । यत् तम् । तम् न । न वि॑स्रꣳ॒॒सये᳚त् । वि॒स्रꣳ॒॒सये॒दमे॑हेन । वि॒स्रꣳ॒॒सये॒दिति॑ वि - सꣳ॒॒स्रये᳚त् । अमे॑हेनाद्ध्व॒र्युः । अ॒द्ध्व॒र्युः प्र । प्र मी॑येत । मी॒ये॒त॒ तस्मा᳚त् । तस्मा॒थ् सः । स वि॒स्रस्यः॑ । वि॒स्रस्य॒ इति॑ वि - स्रस्यः॑ । \newline

\textbf{Jatai Paata} \newline

1. म॒द्ध्य॒त ऊ॒र्जोर्जा म॑द्ध्य॒तो म॑द्ध्य॒त ऊ॒र्जा । \newline
2. ऊ॒र्जा भु॑ञ्जते भुञ्जत ऊ॒र्जोर्जा भु॑ञ्जते । \newline
3. भु॒ञ्ज॒ते॒ य॒ज॒मा॒न॒लो॒के य॑जमानलो॒के भु॑ञ्जते भुञ्जते यजमानलो॒के । \newline
4. य॒ज॒मा॒न॒लो॒के वै वै य॑जमानलो॒के य॑जमानलो॒के वै । \newline
5. य॒ज॒मा॒न॒लो॒क इति॑ यजमान - लो॒के । \newline
6. वै दक्षि॑णानि॒ दक्षि॑णानि॒ वै वै दक्षि॑णानि । \newline
7. दक्षि॑णानि छ॒दीꣳषि॑ छ॒दीꣳषि॒ दक्षि॑णानि॒ दक्षि॑णानि छ॒दीꣳषि॑ । \newline
8. छ॒दीꣳषि॑ भ्रातृव्यलो॒के भ्रा॑तृव्यलो॒के छ॒दीꣳषि॑ छ॒दीꣳषि॑ भ्रातृव्यलो॒के । \newline
9. भ्रा॒तृ॒व्य॒लो॒क उत्त॑रा॒ ण्युत्त॑राणि भ्रातृव्यलो॒के भ्रा॑तृव्यलो॒क उत्त॑राणि । \newline
10. भ्रा॒तृ॒व्य॒लो॒क इति॑ भ्रातृव्य - लो॒के । \newline
11. उत्त॑राणि॒ दक्षि॑णानि॒ दक्षि॑णा॒ न्युत्त॑रा॒ ण्युत्त॑राणि॒ दक्षि॑णनि । \newline
12. उत्त॑रा॒णीत्युत् - त॒रा॒णि॒ । \newline
13. दक्षि॑णा॒ न्युत्त॑रा॒ ण्युत्त॑राणि॒ दक्षि॑णानि॒ दक्षि॑णा॒ न्युत्त॑राणि । \newline
14. उत्त॑राणि करोति करो॒ त्युत्त॑रा॒ ण्युत्त॑राणि करोति । \newline
15. उत्त॑रा॒णीत्युत् - त॒रा॒णि॒ । \newline
16. क॒रो॒ति॒ यज॑मानं॒ ॅयज॑मानम् करोति करोति॒ यज॑मानम् । \newline
17. यज॑मान मे॒वैव यज॑मानं॒ ॅयज॑मान मे॒व । \newline
18. ए॒वा य॑जमाना॒ दय॑जमाना दे॒वैवा य॑जमानात् । \newline
19. अय॑जमाना॒ दुत्त॑र॒ मुत्त॑र॒ मय॑जमाना॒ दय॑जमाना॒ दुत्त॑रम् । \newline
20. उत्त॑रम् करोति करो॒ त्युत्त॑र॒ मुत्त॑रम् करोति । \newline
21. उत्त॑र॒मित्युत् - त॒र॒म् । \newline
22. क॒रो॒ति॒ तस्मा॒त् तस्मा᳚त् करोति करोति॒ तस्मा᳚त् । \newline
23. तस्मा॒द् यज॑मानो॒ यज॑मान॒ स्तस्मा॒त् तस्मा॒द् यज॑मानः । \newline
24. यज॑मा॒नो ऽय॑जमाना॒ दय॑जमाना॒द् यज॑मानो॒ यज॑मा॒नो ऽय॑जमानात् । \newline
25. अय॑जमाना॒ दुत्त॑र॒ उत्त॒रो ऽय॑जमाना॒ दय॑जमाना॒ दुत्त॑रः । \newline
26. उत्त॑रो ऽन्तर्व॒र्ता न॑न्तर्व॒र्ता नुत्त॑र॒ उत्त॑रो ऽन्तर्व॒र्तान् । \newline
27. उत्त॑र॒ इत्युत् - त॒रः॒ । \newline
28. अ॒न्त॒र्व॒र्तान् क॑रोति करो त्यन्तर्व॒र्ता न॑न्तर्व॒र्तान् क॑रोति । \newline
29. अ॒न्त॒र्व॒र्तानित्य॑न्तः - व॒र्तान् । \newline
30. क॒रो॒ति॒ व्यावृ॑त्त्यै॒ व्यावृ॑त्त्यै करोति करोति॒ व्यावृ॑त्त्यै । \newline
31. व्यावृ॑त्त्यै॒ तस्मा॒त् तस्मा॒द् व्यावृ॑त्त्यै॒ व्यावृ॑त्त्यै॒ तस्मा᳚त् । \newline
32. व्यावृ॑त्त्या॒ इति॑ वि - आवृ॑त्त्यै । \newline
33. तस्मा॒ दर॑ण्य॒ मर॑ण्य॒म् तस्मा॒त् तस्मा॒ दर॑ण्यम् । \newline
34. अर॑ण्यम् प्र॒जाः प्र॒जा अर॑ण्य॒ मर॑ण्यम् प्र॒जाः । \newline
35. प्र॒जा उपोप॑ प्र॒जाः प्र॒जा उप॑ । \newline
36. प्र॒जा इति॑ प्र - जाः । \newline
37. उप॑ जीवन्ति जीव॒न् त्युपोप॑ जीवन्ति । \newline
38. जी॒व॒न्ति॒ परि॒ परि॑ जीवन्ति जीवन्ति॒ परि॑ । \newline
39. परि॑ त्वा त्वा॒ परि॒ परि॑ त्वा । \newline
40. त्वा॒ गि॒र्व॒णो॒ गि॒र्व॒ण॒ स्त्वा॒ त्वा॒ गि॒र्व॒णः॒ । \newline
41. गि॒र्व॒णो॒ गिरो॒ गिरो॑ गिर्वणो गिर्वणो॒ गिरः॑ । \newline
42. गिर॒ इतीति॒ गिरो॒ गिर॒ इति॑ । \newline
43. इत्या॑हा॒हे तीत्या॑ह । \newline
44. आ॒ह॒ य॒था॒य॒जुर् य॑थाय॒जु रा॑हाह यथाय॒जुः । \newline
45. य॒था॒य॒जु रे॒वैव य॑थाय॒जुर् य॑थाय॒जु रे॒व । \newline
46. य॒था॒य॒जुरिति॑ यथा - य॒जुः । \newline
47. ए॒वैत दे॒त दे॒वै वैतत् । \newline
48. ए॒तदिन्द्र॒ स्येन्द्र॑ स्यै॒त दे॒तदिन्द्र॑स्य । \newline
49. इन्द्र॑स्य॒ स्यूः स्यूरिन्द्र॒ स्येन्द्र॑स्य॒ स्यूः । \newline
50. स्यू र॑स्यसि॒ स्यूः स्यू र॑सि । \newline
51. अ॒सीन्द्र॒ स्येन्द्र॑स्या स्य॒सीन्द्र॑स्य । \newline
52. इन्द्र॑स्य ध्रु॒वम् ध्रु॒व मिन्द्र॒ स्येन्द्र॑स्य ध्रु॒वम् । \newline
53. ध्रु॒व म॑स्यसि ध्रु॒वम् ध्रु॒व म॑सि । \newline
54. अ॒सीती त्य॑स्य॒सीति॑ । \newline
55. इत्या॑हा॒हे तीत्या॑ह । \newline
56. आ॒है॒न्द्र मै॒न्द्र मा॑हा है॒न्द्रम् । \newline
57. ऐ॒न्द्रꣳ हि ह्यै᳚न्द्र मै॒न्द्रꣳ हि । \newline
58. हि दे॒वत॑या दे॒वत॑या॒ हि हि दे॒वत॑या । \newline
59. दे॒वत॑या॒ सदः॒ सदो॑ दे॒वत॑या दे॒वत॑या॒ सदः॑ । \newline
60. सदो॒ यं ॅयꣳ सदः॒ सदो॒ यम् । \newline
61. यम् प्र॑थ॒मम् प्र॑थ॒मं ॅयं ॅयम् प्र॑थ॒मम् । \newline
62. प्र॒थ॒मम् ग्र॒न्थिम् ग्र॒न्थिम् प्र॑थ॒मम् प्र॑थ॒मम् ग्र॒न्थिम् । \newline
63. ग्र॒न्थिम् ग्र॑थ्नी॒याद् ग्र॑थ्नी॒याद् ग्र॒न्थिम् ग्र॒न्थिम् ग्र॑थ्नी॒यात् । \newline
64. ग्र॒थ्नी॒याद् यद् यद् ग्र॑थ्नी॒याद् ग्र॑थ्नी॒याद् यत् । \newline
65. यत् तम् तं ॅयद् यत् तम् । \newline
66. तन्न न तम् तन्न । \newline
67. न वि॑स्रꣳ॒॒सये᳚द् विस्रꣳ॒॒सये॒न् न न वि॑स्रꣳ॒॒सये᳚त् । \newline
68. वि॒स्रꣳ॒॒सये॒ दमे॑हे॒ना मे॑हेन विस्रꣳ॒॒सये᳚द् विस्रꣳ॒॒सये॒ दमे॑हेन । \newline
69. वि॒स्रꣳ॒॒सये॒दिति॑ वि - स्रꣳ॒॒सये᳚त् । \newline
70. अमे॑हेना द्ध्व॒र्यु र॑द्ध्व॒र्यु रमे॑हे॒ना मे॑हेना द्ध्व॒र्युः । \newline
71. अ॒द्ध्व॒र्युः प्र प्राद्ध्व॒र्यु र॑द्ध्व॒र्युः प्र । \newline
72. प्र मी॑येत मीयेत॒ प्र प्र मी॑येत । \newline
73. मी॒ये॒त॒ तस्मा॒त् तस्मा᳚न् मीयेत मीयेत॒ तस्मा᳚त् । \newline
74. तस्मा॒थ् स स तस्मा॒त् तस्मा॒थ् सः । \newline
75. स वि॒स्रस्यो॑ वि॒स्रस्यः॒ स स वि॒स्रस्यः॑ । \newline
76. वि॒स्रस्य॒ इति॑ वि - स्रस्यः॑ । \newline

\textbf{Ghana Paata } \newline

1. म॒द्ध्य॒त ऊ॒र्जोर्जा म॑द्ध्य॒तो म॑द्ध्य॒त ऊ॒र्जा भु॑ञ्जते भुञ्जत ऊ॒र्जा म॑द्ध्य॒तो म॑द्ध्य॒त ऊ॒र्जा भु॑ञ्जते । \newline
2. ऊ॒र्जा भु॑ञ्जते भुञ्जत ऊ॒र्जोर्जा भु॑ञ्जते यजमानलो॒के य॑जमानलो॒के भु॑ञ्जत ऊ॒र्जोर्जा भु॑ञ्जते यजमानलो॒के । \newline
3. भु॒ञ्ज॒ते॒ य॒ज॒मा॒न॒लो॒के य॑जमानलो॒के भु॑ञ्जते भुञ्जते यजमानलो॒के वै वै य॑जमानलो॒के भु॑ञ्जते भुञ्जते यजमानलो॒के वै । \newline
4. य॒ज॒मा॒न॒लो॒के वै वै य॑जमानलो॒के य॑जमानलो॒के वै दक्षि॑णानि॒ दक्षि॑णानि॒ वै य॑जमानलो॒के य॑जमानलो॒के वै दक्षि॑णानि । \newline
5. य॒ज॒मा॒न॒लो॒क इति॑ यजमान - लो॒के । \newline
6. वै दक्षि॑णानि॒ दक्षि॑णानि॒ वै वै दक्षि॑णानि छ॒दीꣳषि॑ छ॒दीꣳषि॒ दक्षि॑णानि॒ वै वै दक्षि॑णानि छ॒दीꣳषि॑ । \newline
7. दक्षि॑णानि छ॒दीꣳषि॑ छ॒दीꣳषि॒ दक्षि॑णानि॒ दक्षि॑णानि छ॒दीꣳषि॑ भ्रातृव्यलो॒के भ्रा॑तृव्यलो॒के छ॒दीꣳषि॒ दक्षि॑णानि॒ दक्षि॑णानि छ॒दीꣳषि॑ भ्रातृव्यलो॒के । \newline
8. छ॒दीꣳषि॑ भ्रातृव्यलो॒के भ्रा॑तृव्यलो॒के छ॒दीꣳषि॑ छ॒दीꣳषि॑ भ्रातृव्यलो॒क उत्त॑रा॒ 
ण्युत्त॑राणि भ्रातृव्यलो॒के छ॒दीꣳषि॑ छ॒दीꣳषि॑ भ्रातृव्यलो॒क उत्त॑राणि । \newline
9. भ्रा॒तृ॒व्य॒लो॒क उत्त॑रा॒ ण्युत्त॑राणि भ्रातृव्यलो॒के भ्रा॑तृव्यलो॒क उत्त॑राणि॒ दक्षि॑णानि॒ दक्षि॑णा॒
न्युत्त॑राणि भ्रातृव्यलो॒के भ्रा॑तृव्यलो॒क उत्त॑राणि॒ दक्षि॑णानि । \newline
10. भ्रा॒तृ॒व्य॒लो॒क इति॑ भ्रातृव्य - लो॒के । \newline
11. उत्त॑राणि॒ दक्षि॑णानि॒ दक्षि॑णा॒ न्युत्त॑रा॒ ण्युत्त॑राणि॒ दक्षि॑णा॒ न्युत्त॑रा॒ ण्युत्त॑राणि॒ दक्षि॑णा॒
न्युत्त॑रा॒ ण्युत्त॑राणि॒ दक्षि॑णा॒ न्युत्त॑राणि । \newline
12. उत्त॑रा॒णीत्युत् - त॒रा॒णि॒ । \newline
13. दक्षि॑णा॒ न्युत्त॑रा॒ ण्युत्त॑राणि॒ दक्षि॑णानि॒ दक्षि॑णा॒ न्युत्त॑राणि करोति करो॒ त्युत्त॑राणि॒ 
दक्षि॑णानि॒ दक्षि॑णा॒ न्युत्त॑राणि करोति । \newline
14. उत्त॑राणि करोति करो॒ त्युत्त॑रा॒ ण्युत्त॑राणि करोति॒ यज॑मानं॒ ॅयज॑मानम् करो॒ त्युत्त॑रा॒ ण्युत्त॑राणि करोति॒ यज॑मानम् । \newline
15. उत्त॑रा॒णीत्युत् - त॒रा॒णि॒ । \newline
16. क॒रो॒ति॒ यज॑मानं॒ ॅयज॑मानम् करोति करोति॒ यज॑मान मे॒वैव यज॑मानम् करोति करोति॒ यज॑मान मे॒व । \newline
17. यज॑मान मे॒वैव यज॑मानं॒ ॅयज॑मान मे॒वा य॑जमाना॒ दय॑जमाना दे॒व यज॑मानं॒ ॅयज॑मान मे॒वा य॑जमानात् । \newline
18. ए॒वा य॑जमाना॒ दय॑जमाना दे॒वैवा य॑जमाना॒ दुत्त॑र॒ मुत्त॑र॒ मय॑जमाना दे॒वैवा य॑जमाना॒ दुत्त॑रम् । \newline
19. अय॑जमाना॒ दुत्त॑र॒ मुत्त॑र॒ मय॑जमाना॒ दय॑जमाना॒ दुत्त॑रम् करोति करो॒ त्युत्त॑र॒ मय॑जमाना॒ दय॑जमाना॒ दुत्त॑रम् करोति । \newline
20. उत्त॑रम् करोति करो॒ त्युत्त॑र॒ मुत्त॑रम् करोति॒ तस्मा॒त् तस्मा᳚त् करो॒ त्युत्त॑र॒ मुत्त॑रम् करोति॒ तस्मा᳚त् । \newline
21. उत्त॑र॒मित्युत् - त॒र॒म् । \newline
22. क॒रो॒ति॒ तस्मा॒त् तस्मा᳚त् करोति करोति॒ तस्मा॒द् यज॑मानो॒ यज॑मान॒ स्तस्मा᳚त् करोति करोति॒ तस्मा॒द् यज॑मानः । \newline
23. तस्मा॒द् यज॑मानो॒ यज॑मान॒ स्तस्मा॒त् तस्मा॒द् यज॑मा॒नो ऽय॑जमाना॒ दय॑जमाना॒द् यज॑मान॒ स्तस्मा॒त् तस्मा॒द् यज॑मा॒नो ऽय॑जमानात् । \newline
24. यज॑मा॒नो ऽय॑जमाना॒ दय॑जमाना॒द् यज॑मानो॒ यज॑मा॒नो ऽय॑जमाना॒ दुत्त॑र॒ उत्त॒रो ऽय॑जमाना॒द् यज॑मानो॒ यज॑मा॒नो ऽय॑जमाना॒ दुत्त॑रः । \newline
25. अय॑जमाना॒ दुत्त॑र॒ उत्त॒रो ऽय॑जमाना॒ दय॑जमाना॒ दुत्त॑रो ऽन्तर्व॒र्ता न॑न्तर्व॒र्ता नुत्त॒रो ऽय॑जमाना॒ दय॑जमाना॒ दुत्त॑रो ऽन्तर्व॒र्तान् । \newline
26. उत्त॑रो ऽन्तर्व॒र्ता न॑न्तर्व॒र्ता नुत्त॑र॒ उत्त॑रो ऽन्तर्व॒र्तान् क॑रोति करो त्यन्तर्व॒र्ता नुत्त॑र॒ उत्त॑रो ऽन्तर्व॒र्तान् क॑रोति । \newline
27. उत्त॑र॒ इत्युत् - त॒रः॒ । \newline
28. अ॒न्त॒र्व॒र्तान् क॑रोति करो त्यन्तर्व॒र्ता न॑न्तर्व॒र्तान् क॑रोति॒ व्यावृ॑त्त्यै॒ व्यावृ॑त्त्यै करो त्यन्तर्व॒र्ता न॑न्तर्व॒र्तान् क॑रोति॒ व्यावृ॑त्त्यै । \newline
29. अ॒न्त॒र्व॒र्तानित्य॑न्तः - व॒र्तान् । \newline
30. क॒रो॒ति॒ व्यावृ॑त्त्यै॒ व्यावृ॑त्त्यै करोति करोति॒ व्यावृ॑त्त्यै॒ तस्मा॒त् तस्मा॒द् व्यावृ॑त्त्यै करोति करोति॒ व्यावृ॑त्त्यै॒ तस्मा᳚त् । \newline
31. व्यावृ॑त्त्यै॒ तस्मा॒त् तस्मा॒द् व्यावृ॑त्त्यै॒ व्यावृ॑त्त्यै॒ तस्मा॒ दर॑ण्य॒ मर॑ण्य॒म् तस्मा॒द् व्यावृ॑त्त्यै॒ व्यावृ॑त्त्यै॒ तस्मा॒ दर॑ण्यम् । \newline
32. व्यावृ॑त्त्या॒ इति॑ वि - आवृ॑त्त्यै । \newline
33. तस्मा॒ दर॑ण्य॒ मर॑ण्य॒म् तस्मा॒त् तस्मा॒ दर॑ण्यम् प्र॒जाः प्र॒जा अर॑ण्य॒म् तस्मा॒त् तस्मा॒ दर॑ण्यम् प्र॒जाः । \newline
34. अर॑ण्यम् प्र॒जाः प्र॒जा अर॑ण्य॒ मर॑ण्यम् प्र॒जा उपोप॑ प्र॒जा अर॑ण्य॒ मर॑ण्यम् प्र॒जा उप॑ । \newline
35. प्र॒जा उपोप॑ प्र॒जाः प्र॒जा उप॑ जीवन्ति जीव॒न् त्युप॑ प्र॒जाः प्र॒जा उप॑ जीवन्ति । \newline
36. प्र॒जा इति॑ प्र - जाः । \newline
37. उप॑ जीवन्ति जीव॒न् त्युपोप॑ जीवन्ति॒ परि॒ परि॑ जीव॒न् त्युपोप॑ जीवन्ति॒ परि॑ । \newline
38. जी॒व॒न्ति॒ परि॒ परि॑ जीवन्ति जीवन्ति॒ परि॑ त्वा त्वा॒ परि॑ जीवन्ति जीवन्ति॒ परि॑ त्वा । \newline
39. परि॑ त्वा त्वा॒ परि॒ परि॑ त्वा गिर्वणो गिर्वण स्त्वा॒ परि॒ परि॑ त्वा गिर्वणः । \newline
40. त्वा॒ गि॒र्व॒णो॒ गि॒र्व॒ण॒ स्त्वा॒ त्वा॒ गि॒र्व॒णो॒ गिरो॒ गिरो॑ गिर्वण स्त्वा त्वा गिर्वणो॒ गिरः॑ । \newline
41. गि॒र्व॒णो॒ गिरो॒ गिरो॑ गिर्वणो गिर्वणो॒ गिर॒ इतीति॒ गिरो॑ गिर्वणो गिर्वणो॒ गिर॒ इति॑ । \newline
42. गिर॒ इतीति॒ गिरो॒ गिर॒ इत्या॑हा॒ हेति॒ गिरो॒ गिर॒ इत्या॑ह । \newline
43. इत्या॑हा॒हे तीत्या॑ह यथाय॒जुर् य॑थाय॒जु रा॒हे तीत्या॑ह यथाय॒जुः । \newline
44. आ॒ह॒ य॒था॒य॒जुर् य॑थाय॒जु रा॑हाह यथाय॒जु रे॒वैव य॑थाय॒जु रा॑हाह यथाय॒जु रे॒व । \newline
45. य॒था॒य॒जु रे॒वैव य॑थाय॒जुर् य॑थाय॒जु रे॒वैत दे॒त दे॒व य॑थाय॒जुर् य॑थाय॒जु रे॒वैतत् । \newline
46. य॒था॒य॒जुरिति॑ यथा - य॒जुः । \newline
47. ए॒वैत दे॒त दे॒वै वैत दिन्द्र॒ स्येन्द्र॑ स्यै॒त दे॒वै वैत दिन्द्र॑स्य । \newline
48. ए॒त दिन्द्र॒ स्येन्द्र॑ स्यै॒त दे॒त दिन्द्र॑स्य॒ स्यूः स्यूरिन्द्र॑ स्यै॒त दे॒त दिन्द्र॑स्य॒ स्यूः । \newline
49. इन्द्र॑स्य॒ स्यूः स्यूरिन्द्र॒ स्येन्द्र॑स्य॒ स्यूर॑स्यसि॒ स्यूरिन्द्र॒ स्येन्द्र॑स्य॒ स्यूर॑सि । \newline
50. स्यूर॑स्यसि॒ स्यूः स्यू र॒सीन्द्र॒ स्येन्द्र॑स्यासि॒ स्यूः स्यू र॒सीन्द्र॑स्य । \newline
51. अ॒सीन्द्र॒ स्येन्द्र॑स्यास्य॒ सीन्द्र॑स्य ध्रु॒वम् ध्रु॒व मिन्द्र॑ स्यास्य॒ सीन्द्र॑स्य ध्रु॒वम् । \newline
52. इन्द्र॑स्य ध्रु॒वम् ध्रु॒व मिन्द्र॒ स्येन्द्र॑स्य ध्रु॒व म॑स्यसि ध्रु॒व मिन्द्र॒ स्येन्द्र॑स्य ध्रु॒व म॑सि । \newline
53. ध्रु॒व म॑स्यसि ध्रु॒वम् ध्रु॒व म॒सीती त्य॑सि ध्रु॒वम् ध्रु॒व म॒सीति॑ । \newline
54. अ॒सीती त्य॑स्य॒ सीत्या॑ हा॒हे त्य॑स्य॒ सीत्या॑ह । \newline
55. इत्या॑हा॒हे तीत्या॑ है॒न्द्र मै॒न्द्र मा॒हे तीत्या॑ है॒न्द्रम् । \newline
56. आ॒है॒न्द्र मै॒न्द्र मा॑हा है॒न्द्रꣳ हि ह्यै᳚न्द्र मा॑हा है॒न्द्रꣳ हि । \newline
57. ऐ॒न्द्रꣳ हि ह्यै᳚न्द्र मै॒न्द्रꣳ हि दे॒वत॑या दे॒वत॑या॒ ह्यै᳚न्द्र मै॒न्द्रꣳ हि दे॒वत॑या । \newline
58. हि दे॒वत॑या दे॒वत॑या॒ हि हि दे॒वत॑या॒ सदः॒ सदो॑ दे॒वत॑या॒ हि हि दे॒वत॑या॒ सदः॑ । \newline
59. दे॒वत॑या॒ सदः॒ सदो॑ दे॒वत॑या दे॒वत॑या॒ सदो॒ यं ॅयꣳ सदो॑ दे॒वत॑या दे॒वत॑या॒ सदो॒ यम् । \newline
60. सदो॒ यं ॅयꣳ सदः॒ सदो॒ यम् प्र॑थ॒मम् प्र॑थ॒मं ॅयꣳ सदः॒ सदो॒ यम् प्र॑थ॒मम् । \newline
61. यम् प्र॑थ॒मम् प्र॑थ॒मं ॅयं ॅयम् प्र॑थ॒मम् ग्र॒न्थिम् ग्र॒न्थिम् प्र॑थ॒मं ॅयं ॅयम् प्र॑थ॒मम् ग्र॒न्थिम् । \newline
62. प्र॒थ॒मम् ग्र॒न्थिम् ग्र॒न्थिम् प्र॑थ॒मम् प्र॑थ॒मम् ग्र॒न्थिम् ग्र॑थ्नी॒याद् ग्र॑थ्नी॒याद् ग्र॒न्थिम् प्र॑थ॒मम् प्र॑थ॒मम् ग्र॒न्थिम् ग्र॑थ्नी॒यात् । \newline
63. ग्र॒न्थिम् ग्र॑थ्नी॒याद् ग्र॑थ्नी॒याद् ग्र॒न्थिम् ग्र॒न्थिम् ग्र॑थ्नी॒याद् यद् यद् ग्र॑थ्नी॒याद् ग्र॒न्थिम् ग्र॒न्थिम् ग्र॑थ्नी॒याद् यत् । \newline
64. ग्र॒थ्नी॒याद् यद् यद् ग्र॑थ्नी॒याद् ग्र॑थ्नी॒याद् यत् तम् तं ॅयद् ग्र॑थ्नी॒याद् ग्र॑थ्नी॒याद् यत् तम् । \newline
65. यत् तम् तं ॅयद् यत् तन् न न तं ॅयद् यत् तन् न । \newline
66. तन् न न तम् तन् न वि॑स्रꣳ॒॒सये᳚द् विस्रꣳ॒॒सये॒न् न तम् तन् न वि॑स्रꣳ॒॒सये᳚त् । \newline
67. न वि॑स्रꣳ॒॒सये᳚द् विस्रꣳ॒॒सये॒न् न न वि॑स्रꣳ॒॒सये॒ दमे॑हे॒ना मे॑हेन विस्रꣳ॒॒सये॒न् न न वि॑स्रꣳ॒॒सये॒ दमे॑हेन । \newline
68. वि॒स्रꣳ॒॒सये॒ दमे॑हे॒ना मे॑हेन विस्रꣳ॒॒सये᳚द् विस्रꣳ॒॒सये॒ दमे॑हे नाद्ध्व॒र्यु र॑द्ध्व॒र्यु रमे॑हेन विस्रꣳ॒॒सये᳚द् विस्रꣳ॒॒सये॒ दमे॑हे नाद्ध्व॒र्युः । \newline
69. वि॒स्रꣳ॒॒सये॒दिति॑ वि - स्रꣳ॒॒सये᳚त् । \newline
70. अमे॑हे नाद्ध्व॒र्यु र॑द्ध्व॒र्यु रमे॑हे॒ना मे॑हे नाद्ध्व॒र्युः प्र प्राद्ध्व॒र्यु रमे॑हे॒ना मे॑हे नाद्ध्व॒र्युः प्र । \newline
71. अ॒द्ध्व॒र्युः प्र प्राद्ध्व॒र्यु र॑द्ध्व॒र्युः प्र मी॑येत मीयेत॒ प्राद्ध्व॒र्यु र॑द्ध्व॒र्युः प्र मी॑येत । \newline
72. प्र मी॑येत मीयेत॒ प्र प्र मी॑येत॒ तस्मा॒त् तस्मा᳚न् मीयेत॒ प्र प्र मी॑येत॒ तस्मा᳚त् । \newline
73. मी॒ये॒त॒ तस्मा॒त् तस्मा᳚न् मीयेत मीयेत॒ तस्मा॒थ् स स तस्मा᳚न् मीयेत मीयेत॒ तस्मा॒थ् सः । \newline
74. तस्मा॒थ् स स तस्मा॒त् तस्मा॒थ् स वि॒स्रस्यो॑ वि॒स्रस्यः॒ स तस्मा॒त् तस्मा॒थ् स वि॒स्रस्यः॑ । \newline
75. स वि॒स्रस्यो॑ वि॒स्रस्यः॒ स स वि॒स्रस्यः॑ । \newline
76. वि॒स्रस्य॒ इति॑ वि - स्रस्यः॑ । \newline
\pagebreak
\markright{ TS 6.2.11.1  \hfill https://www.vedavms.in \hfill}

\section{ TS 6.2.11.1 }

\textbf{TS 6.2.11.1 } \newline
\textbf{Samhita Paata} \newline

शिरो॒ वा ए॒तद्-य॒ज्ञ्स्य॒ यद्ध॑वि॒र्द्धानं॑ प्रा॒णा उ॑पर॒वा ह॑वि॒र्द्धाने॑ खायन्ते॒ तस्मा᳚च्छी॒र्॒.षन् प्रा॒णा अ॒धस्ता᳚त् खायन्ते॒ तस्मा॑द॒धस्ता᳚च्छीर्ष्णः प्रा॒णा र॑क्षो॒हणो॑ वलग॒हनो॑ वैष्ण॒वान् ख॑ना॒मीत्या॑ह वैष्ण॒वा हि दे॒वत॑योपर॒वा असु॑रा॒ वै नि॒र्यन्तो॑ दे॒वानां᳚ प्रा॒णेषु॑ वल॒गान् न्य॑खन॒न् तान् बा॑हुमा॒त्रेऽन्व॑विन्द॒न् तस्मा᳚द् बाहुमा॒त्राः खा॑यन्त इ॒दम॒हं तं ॅव॑ल॒गमुद्व॑पामि॒- [  ] \newline

\textbf{Pada Paata} \newline

शिरः॑ । वै । ए॒तत् । य॒ज्ञ्स्य॑ । यत् । ह॒वि॒द्‌र्धान॒मिति॑ हविः-धान᳚म् । प्रा॒णा इति॑ प्र - अ॒नाः । उ॒प॒र॒वा इत्यु॑प - र॒वाः । ह॒वि॒द्‌र्धान॒ इति॑ हविः-धाने᳚ । खा॒य॒न्ते॒ । तस्मा᳚त् । शी॒र्॒.षन्न् । प्रा॒णा इति॑ प्र-अ॒नाः । अ॒धस्ता᳚त् । खा॒य॒न्ते॒ । तस्मा᳚त् । अ॒धस्ता᳚त् । शी॒र्ष्णः । प्रा॒णा इति॑ प्र-अ॒नाः । र॒क्षो॒हण॒ इति॑ रक्षः-हनः॑ । व॒ल॒ग॒हन॒ इति॑ वलग-हनः॑ । वै॒ष्ण॒वान् । ख॒ना॒मि॒ । इति॑ । आ॒ह॒ । वै॒ष्ण॒वाः । हि । दे॒वत॑या । उ॒प॒र॒वा इत्यु॑प - र॒वाः । असु॑राः । वै । नि॒र्यन्त॒ इति॑ निः - यन्तः॑ । दे॒वाना᳚म् । प्रा॒णेष्विति॑ प्र - अ॒नेषु॑ । व॒ल॒गानिति॑ वल-गान् । नीति॑ । अ॒ख॒न॒न्न् । तान् । बा॒हु॒मा॒त्र इति॑ बाहु-मा॒त्रे । अन्विति॑ । अ॒वि॒न्द॒न्न् । तस्मा᳚त् । बा॒हु॒मा॒त्रा इति॑ बाहु - मा॒त्राः । खा॒य॒न्ते॒ । इ॒दम् । अ॒हम् । तम् । व॒ल॒गमिति॑ वल - गम् । उदिति॑ । व॒पा॒मि॒ ।  \newline


\textbf{Krama Paata} \newline

शिरो॒ वै । वा ए॒तत् । ए॒तद् य॒ज्ञ्स्य॑ । य॒ज्ञ्स्य॒ यत् । यद्‍ध॑वि॒र्द्धान᳚म् । ह॒वि॒र्द्धान॑म् प्रा॒णाः । ह॒वि॒र्द्धान॒मिति॑ हविः - धान᳚म् । प्रा॒णा उ॑पर॒वाः । प्रा॒णा इति॑ प्र - अ॒नाः । उ॒प॒र॒वा ह॑वि॒र्द्धाने᳚ । उ॒प॒र॒वा इत्यु॑प - र॒वाः । ह॒वि॒र्द्धाने॑ खायन्ते । ह॒वि॒र्द्धान॒ इति॑ हविः - धाने᳚ । खा॒य॒न्ते॒ तस्मा᳚त् । तस्मा᳚च्छी॒र्.॒षन्न् । शी॒र्.॒षन् प्रा॒णाः । प्रा॒णा अ॒धस्ता᳚त् । प्रा॒णा इति॑ प्र - अ॒नाः । अ॒धस्ता᳚त् खायन्ते । खा॒य॒न्ते॒ तस्मा᳚त् । तस्मा॑द॒धस्ता᳚त् । अ॒धस्ता᳚च्छी॒र्ष्णः । शी॒र्ष्णः प्रा॒णाः । प्रा॒णा र॑क्षो॒हणः॑ । प्रा॒णा इति॑ प्र - अ॒नाः । र॒क्षो॒हणो॑ वलग॒हनः॑ । र॒क्षो॒हण॒ इति॑ रक्षः - हनः॑ । व॒ल॒ग॒हनो॑ वैष्ण॒वान् । व॒ल॒ग॒हन॒ इति॑ वलग - हनः॑ । वै॒ष्ण॒वान् ख॑नामि । ख॒ना॒मीति॑ । इत्या॑ह । आ॒ह॒ वै॒ष्ण॒वाः । वै॒ष्ण॒वा हि । हि दे॒वत॑या । दे॒वत॑योपर॒वाः । उ॒प॒र॒वा असु॑राः । उ॒प॒र॒वा इत्यु॑प - र॒वाः । असु॑रा॒ वै । वै नि॒र्यन्तः॑ । नि॒र्यन्तो॑ दे॒वाना᳚म् । नि॒र्यन्त॒ इति॑ निः - यन्तः॑ । दे॒वाना᳚म् प्रा॒णेषु॑ । प्रा॒णेषु॑ वल॒गान् । प्रा॒णेष्विति॑ प्र - अ॒नेषु॑ । व॒ल॒गान् नि । व॒ल॒गानिति॑ वल - गान् । न्य॑खनन्न् । अ॒ख॒न॒न् तान् । तान् बा॑हुमा॒त्रे । बा॒हु॒मा॒त्रेऽनु॑ । बा॒हु॒मा॒त्र इति॑ बाहु - मा॒त्रे । अन्व॑विन्दन्न् । अ॒वि॒न्द॒न् तस्मा᳚त् । तस्मा᳚द् बाहुमा॒त्राः । बा॒हु॒मा॒त्राः खा॑यन्ते । बा॒हु॒मा॒त्रा इति॑ बाहु - मा॒त्राः । खा॒य॒न्त॒ इ॒दम् । इ॒दम॒हम् । अ॒हम् तम् । तम् ॅव॑ल॒गम् । व॒ल॒गमुत् । व॒ल॒गमिति॑ वल - गम् । 
उद् व॑पामि । व॒पा॒मि॒ यम् \newline

\textbf{Jatai Paata} \newline

1. शिरो॒ वै वै शिरः॒ शिरो॒ वै । \newline
2. वा ए॒त दे॒तद् वै वा ए॒तत् । \newline
3. ए॒तद् य॒ज्ञ्स्य॑ य॒ज्ञ् स्यै॒त दे॒तद् य॒ज्ञ्स्य॑ । \newline
4. य॒ज्ञ्स्य॒ यद् यद् य॒ज्ञ्स्य॑ य॒ज्ञ्स्य॒ यत् । \newline
5. यद्ध॑वि॒र्द्धानꣳ॑ हवि॒र्द्धानं॒ ॅयद् यद्ध॑वि॒र्द्धान᳚म् । \newline
6. ह॒वि॒र्द्धान॑म् प्रा॒णाः प्रा॒णा ह॑वि॒र्द्धानꣳ॑ हवि॒र्द्धान॑म् प्रा॒णाः । \newline
7. ह॒वि॒र्द्धान॒मिति॑ हविः - धान᳚म् । \newline
8. प्रा॒णा उ॑पर॒वा उ॑पर॒वाः प्रा॒णाः प्रा॒णा उ॑पर॒वाः । \newline
9. प्रा॒णा इति॑ प्र - अ॒नाः । \newline
10. उ॒प॒र॒वा ह॑वि॒र्द्धाने॑ हवि॒र्द्धान॑ उपर॒वा उ॑पर॒वा ह॑वि॒र्द्धाने᳚ । \newline
11. उ॒प॒र॒वा इत्यु॑प - र॒वाः । \newline
12. ह॒वि॒र्द्धाने॑ खायन्ते खायन्ते हवि॒र्द्धाने॑ हवि॒र्द्धाने॑ खायन्ते । \newline
13. ह॒वि॒र्द्धान॒ इति॑ हविः - धाने᳚ । \newline
14. खा॒य॒न्ते॒ तस्मा॒त् तस्मा᳚त् खायन्ते खायन्ते॒ तस्मा᳚त् । \newline
15. तस्मा᳚ च्छी॒र्॒.षञ् छी॒र्॒.षन् तस्मा॒त् तस्मा᳚ च्छी॒र्॒.षन्न् । \newline
16. शी॒र्॒.षन् प्रा॒णाः प्रा॒णाः शी॒र्॒.षञ् छी॒र्॒.षन् प्रा॒णाः । \newline
17. प्रा॒णा अ॒धस्ता॑ द॒धस्ता᳚त् प्रा॒णाः प्रा॒णा अ॒धस्ता᳚त् । \newline
18. प्रा॒णा इति॑ प्र - अ॒नाः । \newline
19. अ॒धस्ता᳚त् खायन्ते खायन्ते॒ ऽधस्ता॑ द॒धस्ता᳚त् खायन्ते । \newline
20. खा॒य॒न्ते॒ तस्मा॒त् तस्मा᳚त् खायन्ते खायन्ते॒ तस्मा᳚त् । \newline
21. तस्मा॑ द॒धस्ता॑ द॒धस्ता॒त् तस्मा॒त् तस्मा॑ द॒धस्ता᳚त् । \newline
22. अ॒धस्ता᳚ च्छी॒र्ष्णः शी॒र्ष्णो॑ ऽधस्ता॑ द॒धस्ता᳚ च्छी॒र्ष्णः । \newline
23. शी॒र्ष्णः प्रा॒णाः प्रा॒णाः शी॒र्ष्णः शी॒र्ष्णः प्रा॒णाः । \newline
24. प्रा॒णा र॑क्षो॒हणो॑ रक्षो॒हणः॑ प्रा॒णाः प्रा॒णा र॑क्षो॒हणः॑ । \newline
25. प्रा॒णा इति॑ प्र - अ॒नाः । \newline
26. र॒क्षो॒हणो॑ वलग॒हनो॑ वलग॒हनो॑ रक्षो॒हणो॑ रक्षो॒हणो॑ वलग॒हनः॑ । \newline
27. र॒क्षो॒हण॒ इति॑ रक्षः - हनः॑ । \newline
28. व॒ल॒ग॒हनो॑ वैष्ण॒वान्. वै᳚ष्ण॒वान्. व॑लग॒हनो॑ वलग॒हनो॑ वैष्ण॒वान् । \newline
29. व॒ल॒ग॒हन॒ इति॑ वलग - हनः॑ । \newline
30. वै॒ष्ण॒वान् ख॑नामि खनामि वैष्ण॒वान्. वै᳚ष्ण॒वान् ख॑नामि । \newline
31. ख॒ना॒मीतीति॑ खनामि खना॒मीति॑ । \newline
32. इत्या॑हा॒हे तीत्या॑ह । \newline
33. आ॒ह॒ वै॒ष्ण॒वा वै᳚ष्ण॒वा आ॑हाह वैष्ण॒वाः । \newline
34. वै॒ष्ण॒वा हि हि वै᳚ष्ण॒वा वै᳚ष्ण॒वा हि । \newline
35. हि दे॒वत॑या दे॒वत॑या॒ हि हि दे॒वत॑या । \newline
36. दे॒वत॑ योपर॒वा उ॑पर॒वा दे॒वत॑या दे॒वत॑ योपर॒वाः । \newline
37. उ॒प॒र॒वा असु॑रा॒ असु॑रा उपर॒वा उ॑पर॒वा असु॑राः । \newline
38. उ॒प॒र॒वा इत्यु॑प - र॒वाः । \newline
39. असु॑रा॒ वै वा असु॑रा॒ असु॑रा॒ वै । \newline
40. वै नि॒र्यन्तो॑ नि॒र्यन्तो॒ वै वै नि॒र्यन्तः॑ । \newline
41. नि॒र्यन्तो॑ दे॒वाना᳚म् दे॒वाना᳚म् नि॒र्यन्तो॑ नि॒र्यन्तो॑ दे॒वाना᳚म् । \newline
42. नि॒र्यन्त॒ इति॑ निः - यन्तः॑ । \newline
43. दे॒वाना᳚म् प्रा॒णेषु॑ प्रा॒णेषु॑ दे॒वाना᳚म् दे॒वाना᳚म् प्रा॒णेषु॑ । \newline
44. प्रा॒णेषु॑ वल॒गान्. व॑ल॒गान् प्रा॒णेषु॑ प्रा॒णेषु॑ वल॒गान् । \newline
45. प्रा॒णेष्विति॑ प्र - अ॒नेषु॑ । \newline
46. व॒ल॒गान् नि नि व॑ल॒गान्. व॑ल॒गान् नि । \newline
47. व॒ल॒गानिति॑ वल - गान् । \newline
48. न्य॑खनन् नखन॒न् नि न्य॑खनन्न् । \newline
49. अ॒ख॒न॒न् ताꣳ स्ता न॑खनन् नखन॒न् तान् । \newline
50. तान् बा॑हुमा॒त्रे बा॑हुमा॒त्रे ताꣳ स्तान् बा॑हुमा॒त्रे । \newline
51. बा॒हु॒मा॒त्रे ऽन्वनु॑ बाहुमा॒त्रे बा॑हुमा॒त्रे ऽनु॑ । \newline
52. बा॒हु॒मा॒त्र इति॑ बाहु - मा॒त्रे । \newline
53. अन्व॑विन्दन् नविन्द॒न् नन् वन् व॑विन्दन्न् । \newline
54. अ॒वि॒न्द॒न् तस्मा॒त् तस्मा॑ दविन्दन् नविन्द॒न् तस्मा᳚त् । \newline
55. तस्मा᳚द् बाहुमा॒त्रा बा॑हुमा॒त्रा स्तस्मा॒त् तस्मा᳚द् बाहुमा॒त्राः । \newline
56. बा॒हु॒मा॒त्राः खा॑यन्ते खायन्ते बाहुमा॒त्रा बा॑हुमा॒त्राः खा॑यन्ते । \newline
57. बा॒हु॒मा॒त्रा इति॑ बाहु - मा॒त्राः । \newline
58. खा॒य॒न्त॒ इ॒द मि॒दम् खा॑यन्ते खायन्त इ॒दम् । \newline
59. इ॒द म॒ह म॒ह मि॒द मि॒द म॒हम् । \newline
60. अ॒हम् तम् त म॒ह म॒हम् तम् । \newline
61. तं ॅव॑ल॒गं ॅव॑ल॒गम् तम् तं ॅव॑ल॒गम् । \newline
62. व॒ल॒ग मुदुद् व॑ल॒गं ॅव॑ल॒ग मुत् । \newline
63. व॒ल॒गमिति॑ वल - गम् । \newline
64. उद् व॑पामि वपा॒ म्युदुद् व॑पामि । \newline
65. व॒पा॒मि॒ यं ॅयं ॅव॑पामि वपामि॒ यम् । \newline

\textbf{Ghana Paata } \newline

1. शिरो॒ वै वै शिरः॒ शिरो॒ वा ए॒त दे॒तद् वै शिरः॒ शिरो॒ वा ए॒तत् । \newline
2. वा ए॒त दे॒तद् वै वा ए॒तद् य॒ज्ञ्स्य॑ य॒ज्ञ् स्यै॒तद् वै वा ए॒तद् य॒ज्ञ्स्य॑ । \newline
3. ए॒तद् य॒ज्ञ्स्य॑ य॒ज्ञ्स्यै॒त दे॒तद् य॒ज्ञ्स्य॒ यद् यद् य॒ज्ञ् स्यै॒त दे॒तद् य॒ज्ञ्स्य॒ यत् । \newline
4. य॒ज्ञ्स्य॒ यद् यद् य॒ज्ञ्स्य॑ य॒ज्ञ्स्य॒ यद्ध॑वि॒र्द्धानꣳ॑ हवि॒र्द्धानं॒ ॅयद् य॒ज्ञ्स्य॑ य॒ज्ञ्स्य॒ यद्ध॑वि॒र्द्धान᳚म् । \newline
5. यद्ध॑वि॒र्द्धानꣳ॑ हवि॒र्द्धानं॒ ॅयद् यद्ध॑वि॒र्द्धान॑म् प्रा॒णाः प्रा॒णा ह॑वि॒र्द्धानं॒ ॅयद् यद्ध॑वि॒र्द्धान॑म् प्रा॒णाः । \newline
6. ह॒वि॒र्द्धान॑म् प्रा॒णाः प्रा॒णा ह॑वि॒र्द्धानꣳ॑ हवि॒र्द्धान॑म् प्रा॒णा उ॑पर॒वा उ॑पर॒वाः प्रा॒णा ह॑वि॒र्द्धानꣳ॑ हवि॒र्द्धान॑म् प्रा॒णा उ॑पर॒वाः । \newline
7. ह॒वि॒र्द्धान॒मिति॑ हविः - धान᳚म् । \newline
8. प्रा॒णा उ॑पर॒वा उ॑पर॒वाः प्रा॒णाः प्रा॒णा उ॑पर॒वा ह॑वि॒र्द्धाने॑ हवि॒र्द्धान॑ उपर॒वाः प्रा॒णाः प्रा॒णा उ॑पर॒वा ह॑वि॒र्द्धाने᳚ । \newline
9. प्रा॒णा इति॑ प्र - अ॒नाः । \newline
10. उ॒प॒र॒वा ह॑वि॒र्द्धाने॑ हवि॒र्द्धान॑ उपर॒वा उ॑पर॒वा ह॑वि॒र्द्धाने॑ खायन्ते खायन्ते हवि॒र्द्धान॑ उपर॒वा उ॑पर॒वा ह॑वि॒र्द्धाने॑ खायन्ते । \newline
11. उ॒प॒र॒वा इत्यु॑प - र॒वाः । \newline
12. ह॒वि॒र्द्धाने॑ खायन्ते खायन्ते हवि॒र्द्धाने॑ हवि॒र्द्धाने॑ खायन्ते॒ तस्मा॒त् तस्मा᳚त् खायन्ते हवि॒र्द्धाने॑ हवि॒र्द्धाने॑ खायन्ते॒ तस्मा᳚त् । \newline
13. ह॒वि॒र्द्धान॒ इति॑ हविः - धाने᳚ । \newline
14. खा॒य॒न्ते॒ तस्मा॒त् तस्मा᳚त् खायन्ते खायन्ते॒ तस्मा᳚ च्छी॒र्॒.षञ् छी॒र्॒.षन् तस्मा᳚त् खायन्ते खायन्ते॒ तस्मा᳚
च्छी॒र्॒.षन्न् । \newline
15. तस्मा᳚ च्छी॒र्॒.षञ् छी॒र्॒.षन् तस्मा॒त् तस्मा᳚ च्छी॒र्॒.षन् प्रा॒णाः प्रा॒णाः शी॒र्॒.षन् तस्मा॒त् तस्मा᳚ च्छी॒र्॒.षन् प्रा॒णाः । \newline
16. शी॒र्॒.षन् प्रा॒णाः प्रा॒णाः शी॒र्॒.षञ् छी॒र्॒.षन् प्रा॒णा अ॒धस्ता॑ द॒धस्ता᳚त् प्रा॒णाः शी॒र्॒.षञ् छी॒र्॒.षन् प्रा॒णा अ॒धस्ता᳚त् । \newline
17. प्रा॒णा अ॒धस्ता॑ द॒धस्ता᳚त् प्रा॒णाः प्रा॒णा अ॒धस्ता᳚त् खायन्ते खायन्ते॒ ऽधस्ता᳚त् प्रा॒णाः प्रा॒णा अ॒धस्ता᳚त् खायन्ते । \newline
18. प्रा॒णा इति॑ प्र - अ॒नाः । \newline
19. अ॒धस्ता᳚त् खायन्ते खायन्ते॒ ऽधस्ता॑ द॒धस्ता᳚त् खायन्ते॒ तस्मा॒त् तस्मा᳚त् खायन्ते॒ ऽधस्ता॑ द॒धस्ता᳚त् खायन्ते॒ तस्मा᳚त् । \newline
20. खा॒य॒न्ते॒ तस्मा॒त् तस्मा᳚त् खायन्ते खायन्ते॒ तस्मा॑ द॒धस्ता॑ द॒धस्ता॒त् तस्मा᳚त् खायन्ते खायन्ते॒ तस्मा॑ द॒धस्ता᳚त् । \newline
21. तस्मा॑ द॒धस्ता॑ द॒धस्ता॒त् तस्मा॒त् तस्मा॑ द॒धस्ता᳚ च्छी॒र्ष्णः शी॒र्ष्णो॑ ऽधस्ता॒त् तस्मा॒त् तस्मा॑ द॒धस्ता᳚च्छी॒र्ष्णः । \newline
22. अ॒धस्ता᳚ च्छी॒र्ष्णः शी॒र्ष्णो॑ ऽधस्ता॑ द॒धस्ता᳚ च्छी॒र्ष्णः प्रा॒णाः प्रा॒णाः शी॒र्ष्णो॑ ऽधस्ता॑ 
द॒धस्ता᳚ च्छी॒र्ष्णः प्रा॒णाः । \newline
23. शी॒र्ष्णः प्रा॒णाः प्रा॒णाः शी॒र्ष्णः शी॒र्ष्णः प्रा॒णा र॑क्षो॒हणो॑ रक्षो॒हणः॑ प्रा॒णाः शी॒र्ष्णः शी॒र्ष्णः प्रा॒णा र॑क्षो॒हणः॑ । \newline
24. प्रा॒णा र॑क्षो॒हणो॑ रक्षो॒हणः॑ प्रा॒णाः प्रा॒णा र॑क्षो॒हणो॑ वलग॒हनो॑ वलग॒हनो॑ रक्षो॒हणः॑ प्रा॒णाः प्रा॒णा र॑क्षो॒हणो॑ वलग॒हनः॑ । \newline
25. प्रा॒णा इति॑ प्र - अ॒नाः । \newline
26. र॒क्षो॒हणो॑ वलग॒हनो॑ वलग॒हनो॑ रक्षो॒हणो॑ रक्षो॒हणो॑ वलग॒हनो॑ वैष्ण॒वान्. वै᳚ष्ण॒वान्. व॑लग॒हनो॑ रक्षो॒हणो॑ रक्षो॒हणो॑ वलग॒हनो॑ वैष्ण॒वान् । \newline
27. र॒क्षो॒हण॒ इति॑ रक्षः - हनः॑ । \newline
28. व॒ल॒ग॒हनो॑ वैष्ण॒वान्. वै᳚ष्ण॒वान्. व॑लग॒हनो॑ वलग॒हनो॑ वैष्ण॒वान् ख॑नामि खनामि वैष्ण॒वान्. व॑लग॒हनो॑ वलग॒हनो॑ वैष्ण॒वान् ख॑नामि । \newline
29. व॒ल॒ग॒हन॒ इति॑ वलग - हनः॑ । \newline
30. वै॒ष्ण॒वान् ख॑नामि खनामि वैष्ण॒वान्. वै᳚ष्ण॒वान् ख॑ना॒मी तीति॑ खनामि वैष्ण॒वान्. वै᳚ष्ण॒वान् ख॑ना॒मीति॑ । \newline
31. ख॒ना॒मी तीति॑ खनामि खना॒मीत्या॑ हा॒हेति॑ खनामि खना॒मीत्या॑ह । \newline
32. इत्या॑हा॒हे तीत्या॑ह वैष्ण॒वा वै᳚ष्ण॒वा आ॒हे तीत्या॑ह वैष्ण॒वाः । \newline
33. आ॒ह॒ वै॒ष्ण॒वा वै᳚ष्ण॒वा आ॑हाह वैष्ण॒वा हि हि वै᳚ष्ण॒वा आ॑हाह वैष्ण॒वा हि । \newline
34. वै॒ष्ण॒वा हि हि वै᳚ष्ण॒वा वै᳚ष्ण॒वा हि दे॒वत॑या दे॒वत॑या॒ हि वै᳚ष्ण॒वा वै᳚ष्ण॒वा हि दे॒वत॑या । \newline
35. हि दे॒वत॑या दे॒वत॑या॒ हि हि दे॒वत॑ योपर॒वा उ॑पर॒वा दे॒वत॑या॒ हि हि दे॒वत॑ योपर॒वाः । \newline
36. दे॒वत॑ योपर॒वा उ॑पर॒वा दे॒वत॑या दे॒वत॑ योपर॒वा असु॑रा॒ असु॑रा उपर॒वा दे॒वत॑या दे॒वत॑
योपर॒वा असु॑राः । \newline
37. उ॒प॒र॒वा असु॑रा॒ असु॑रा उपर॒वा उ॑पर॒वा असु॑रा॒ वै वा असु॑रा उपर॒वा उ॑पर॒वा असु॑रा॒ वै । \newline
38. उ॒प॒र॒वा इत्यु॑प - र॒वाः । \newline
39. असु॑रा॒ वै वा असु॑रा॒ असु॑रा॒ वै नि॒र्यन्तो॑ नि॒र्यन्तो॒ वा असु॑रा॒ असु॑रा॒ वै नि॒र्यन्तः॑ । \newline
40. वै नि॒र्यन्तो॑ नि॒र्यन्तो॒ वै वै नि॒र्यन्तो॑ दे॒वाना᳚म् दे॒वाना᳚म् नि॒र्यन्तो॒ वै वै नि॒र्यन्तो॑ दे॒वाना᳚म् । \newline
41. नि॒र्यन्तो॑ दे॒वाना᳚म् दे॒वाना᳚म् नि॒र्यन्तो॑ नि॒र्यन्तो॑ दे॒वाना᳚म् प्रा॒णेषु॑ प्रा॒णेषु॑ दे॒वाना᳚म् नि॒र्यन्तो॑ नि॒र्यन्तो॑ दे॒वाना᳚म् प्रा॒णेषु॑ । \newline
42. नि॒र्यन्त॒ इति॑ निः - यन्तः॑ । \newline
43. दे॒वाना᳚म् प्रा॒णेषु॑ प्रा॒णेषु॑ दे॒वाना᳚म् दे॒वाना᳚म् प्रा॒णेषु॑ वल॒गान्. व॑ल॒गान् प्रा॒णेषु॑ दे॒वाना᳚म् दे॒वाना᳚म् प्रा॒णेषु॑ वल॒गान् । \newline
44. प्रा॒णेषु॑ वल॒गान्. व॑ल॒गान् प्रा॒णेषु॑ प्रा॒णेषु॑ वल॒गान् नि नि व॑ल॒गान् प्रा॒णेषु॑ प्रा॒णेषु॑ वल॒गान् नि । \newline
45. प्रा॒णेष्विति॑ प्र - अ॒नेषु॑ । \newline
46. व॒ल॒गान् नि नि व॑ल॒गान्. व॑ल॒गान् न्य॑खनन् नखन॒न् नि व॑ल॒गान्. व॑ल॒गान् न्य॑खनन्न् । \newline
47. व॒ल॒गानिति॑ वल - गान् । \newline
48. न्य॑खनन् नखन॒न् नि न्य॑खन॒न् ताꣳ स्ता न॑खन॒न् नि न्य॑खन॒न् तान् । \newline
49. अ॒ख॒न॒न् ताꣳ स्ता न॑खनन् नखन॒न् तान् बा॑हुमा॒त्रे बा॑हुमा॒त्रे ता न॑खनन् नखन॒न् तान् बा॑हुमा॒त्रे । \newline
50. तान् बा॑हुमा॒त्रे बा॑हुमा॒त्रे ताꣳ स्तान् बा॑हुमा॒त्रे ऽन्वनु॑ बाहुमा॒त्रे ताꣳ स्तान् बा॑हुमा॒त्रे ऽनु॑ । \newline
51. बा॒हु॒मा॒त्रे ऽन्वनु॑ बाहुमा॒त्रे बा॑हुमा॒त्रे ऽन्व॑विन्दन् नविन्द॒न् ननु॑ बाहुमा॒त्रे बा॑हुमा॒त्रे ऽन्व॑विन्दन्न् । \newline
52. बा॒हु॒मा॒त्र इति॑ बाहु - मा॒त्रे । \newline
53. अन्व॑विन्दन् नविन्द॒न् नन् वन् व॑विन्द॒न् तस्मा॒त् तस्मा॑ दविन्द॒न् नन् वन् व॑विन्द॒न् तस्मा᳚त् । \newline
54. अ॒वि॒न्द॒न् तस्मा॒त् तस्मा॑ दविन्दन् नविन्द॒न् तस्मा᳚द् बाहुमा॒त्रा बा॑हुमा॒त्रा स्तस्मा॑ दविन्दन् नविन्द॒न् तस्मा᳚द् बाहुमा॒त्राः । \newline
55. तस्मा᳚द् बाहुमा॒त्रा बा॑हुमा॒त्रा स्तस्मा॒त् तस्मा᳚द् बाहुमा॒त्राः खा॑यन्ते खायन्ते बाहुमा॒त्रा स्तस्मा॒त् तस्मा᳚द् बाहुमा॒त्राः खा॑यन्ते । \newline
56. बा॒हु॒मा॒त्राः खा॑यन्ते खायन्ते बाहुमा॒त्रा बा॑हुमा॒त्राः खा॑यन्त इ॒द मि॒दम् खा॑यन्ते बाहुमा॒त्रा बा॑हुमा॒त्राः खा॑यन्त इ॒दम् । \newline
57. बा॒हु॒मा॒त्रा इति॑ बाहु - मा॒त्राः । \newline
58. खा॒य॒न्त॒ इ॒द मि॒दम् खा॑यन्ते खायन्त इ॒द म॒ह म॒ह मि॒दम् खा॑यन्ते खायन्त इ॒द म॒हम् । \newline
59. इ॒द म॒ह म॒ह मि॒द मि॒द म॒हम् तम् त म॒ह मि॒द मि॒द म॒हम् तम् । \newline
60. अ॒हम् तम् त म॒ह म॒हम् तं ॅव॑ल॒गं ॅव॑ल॒गम् त म॒ह म॒हम् तं ॅव॑ल॒गम् । \newline
61. तं ॅव॑ल॒गं ॅव॑ल॒गम् तम् तं ॅव॑ल॒ग मुदुद् व॑ल॒गम् तम् तं ॅव॑ल॒ग मुत् । \newline
62. व॒ल॒ग मुदुद् व॑ल॒गं ॅव॑ल॒ग मुद् व॑पामि वपा॒ म्युद् व॑ल॒गं ॅव॑ल॒ग मुद् व॑पामि । \newline
63. व॒ल॒गमिति॑ वल - गम् । \newline
64. उद् व॑पामि वपा॒ म्युदुद् व॑पामि॒ यं ॅयं ॅव॑पा॒ म्युदुद् व॑पामि॒ यम् । \newline
65. व॒पा॒मि॒ यं ॅयं ॅव॑पामि वपामि॒ यन्नो॑ नो॒ यं ॅव॑पामि वपामि॒ यन्नः॑ । \newline
\pagebreak
\markright{ TS 6.2.11.2  \hfill https://www.vedavms.in \hfill}

\section{ TS 6.2.11.2 }

\textbf{TS 6.2.11.2 } \newline
\textbf{Samhita Paata} \newline

यं नः॑ समा॒नो यमस॑मानो निच॒खानेत्या॑ह॒ द्वौ वाव पुरु॑षौ॒ यश्चै॒व स॑मा॒नो यश्चास॑मानो॒ यमे॒वास्मै॒ तौ व॑ल॒गं नि॒खन॑त॒स्तमे॒वोद्व॑पति॒ संतृ॑णत्ति॒ तस्मा॒थ् संतृ॑ण्णा अन्तर॒तः प्रा॒णा न सं भि॑नत्ति॒ तस्मा॒दस॑भिंन्नाः प्रा॒णा अ॒पोऽव॑ नयति॒ तस्मा॑दा॒र्द्रा अ॑न्तर॒तः प्रा॒णा यव॑मती॒रव॑ नय॒- [  ] \newline

\textbf{Pada Paata} \newline

यम् । नः॒ । स॒मा॒नः । यम् । अस॑मानः । नि॒च॒खानिति॑ नि-च॒खान॑ । इति॑ । आ॒ह॒ । द्वौ । वाव । पुरु॑षौ । यः । च॒ । ए॒व । स॒मा॒नः । यः । च॒ । अस॑मानः । यम् । ए॒व । अ॒स्मै॒ । तौ । व॒ल॒गमिति॑ वल - गम् । नि॒खन॑त॒ इति॑ नि - खन॑तः । तम् । ए॒व । उदिति॑ । व॒प॒ति॒ । समिति॑ । तृ॒ण॒त्ति॒ । तस्मा᳚त् । संतृ॑ण्णा॒ इति॒ सं - तृ॒ण्णाः॒ । अ॒न्त॒र॒तः । प्रा॒णा इति॑ प्र - अ॒नाः । न । समिति॑ । भि॒न॒त्ति॒ । तस्मा᳚त् । अस॑भिंन्ना॒ इत्यसं᳚ - भि॒न्नाः॒ । प्रा॒णा इति॑ प्र - अ॒नाः । अ॒पः । अवेति॑ । न॒य॒ति॒ । तस्मा᳚त् । आ॒र्द्राः । अ॒न्त॒र॒तः । प्रा॒णा इति॑ प्र - अ॒नाः । यव॑मती॒रिति॒ यव॑ - म॒तीः॒ । अवेति॑ । न॒य॒ति॒ ।  \newline


\textbf{Krama Paata} \newline

यम् नः॑ । नः॒ स॒मा॒नः । स॒मा॒नो यम् । यमस॑मानः । अस॑मानो निच॒खान॑ । नि॒च॒खानेति॑ । नि॒च॒खानेति॑ नि - च॒खान॑ । इत्या॑ह । आ॒ह॒ द्वौ । द्वौ वाव । वाव पुरु॑षौ । पुरु॑षौ॒ यः । यश्च॑ । चै॒व । ए॒व स॑मा॒नः । स॒मा॒नो यः । यश्च॑ । चास॑मानः । अस॑मानो॒ यम् । यमे॒व । ए॒वास्मै᳚ । अ॒स्मै॒ तौ । तौ व॑ल॒गम् । व॒ल॒गम् नि॒खन॑तः । व॒ल॒गमिति॑ वल - गम् । नि॒खन॑त॒स्तम् । नि॒खन॑त॒ इति॑ नि - खन॑तः । तमे॒व । ए॒वोत् । उद् व॑पति । व॒प॒ति॒ सम् । सम् तृ॑णत्ति । तृ॒ण॒त्ति॒ तस्मा᳚त् । तस्मा॒थ् सन्तृ॑ण्णाः । सन्तृ॑ण्णा अन्तर॒तः । सन्तृ॑ण्णा॒ इति॒ सम् - तृ॒ण्णाः॒ । अ॒न्त॒र॒तः प्रा॒णाः । प्रा॒णा न । प्रा॒णा इति॑ प्र - अ॒नाः । न सम् । सम् भि॑नत्ति । भि॒न॒त्ति॒ तस्मा᳚त् । तस्मा॒दस॑म्भिन्नाः । अस॑म्भिन्नाः प्रा॒णाः । अस॑म्भिन्ना॒ इत्यस᳚म् - भि॒न्नाः॒ । प्रा॒णा अ॒पः । प्रा॒णा इति॑ प्र - अ॒नाः । अ॒पोऽव॑ । अव॑ नयति । न॒य॒ति॒ तस्मा᳚त् । तस्मा᳚दा॒र्द्राः । आ॒र्द्रा अ॑न्तर॒तः । अ॒न्त॒र॒तः प्रा॒णाः । प्रा॒णा यव॑मतीः । प्रा॒णा इति॑ प्र - अ॒नाः । यव॑मती॒रव॑ । यव॑मती॒रिति॒ यव॑ - म॒तीः॒ । अव॑ नयति । न॒य॒त्यूर्क् \newline

\textbf{Jatai Paata} \newline

1. यन्नो॑ नो॒ यं ॅयन्नः॑ । \newline
2. नः॒ स॒मा॒नः स॑मा॒नो नो॑ नः समा॒नः । \newline
3. स॒मा॒नो यं ॅयꣳ स॑मा॒नः स॑मा॒नो यम् । \newline
4. य मस॑मा॒नो ऽस॑मानो॒ यं ॅय मस॑मानः । \newline
5. अस॑मानो निच॒खान॑ निच॒खाना स॑मा॒नो ऽस॑मानो निच॒खान॑ । \newline
6. नि॒च॒खाने तीति॑ निच॒खान॑ निच॒खानेति॑ । \newline
7. नि॒च॒खानेति॑ नि - च॒खान॑ । \newline
8. इत्या॑हा॒हे तीत्या॑ह । \newline
9. आ॒ह॒ द्वौ द्वा वा॑हाह॒ द्वौ । \newline
10. द्वौ वाव वाव द्वौ द्वौ वाव । \newline
11. वाव पुरु॑षौ॒ पुरु॑षौ॒ वाव वाव पुरु॑षौ । \newline
12. पुरु॑षौ॒ यो यः पुरु॑षौ॒ पुरु॑षौ॒ यः । \newline
13. यश्च॑ च॒ यो यश्च॑ । \newline
14. चै॒वैव च॑ चै॒व । \newline
15. ए॒व स॑मा॒नः स॑मा॒न ए॒वैव स॑मा॒नः । \newline
16. स॒मा॒नो यो यः स॑मा॒नः स॑मा॒नो यः । \newline
17. यश्च॑ च॒ यो यश्च॑ । \newline
18. चास॑मा॒नो ऽस॑मानश्च॒ चास॑मानः । \newline
19. अस॑मानो॒ यं ॅय मस॑मा॒नो ऽस॑मानो॒ यम् । \newline
20. य मे॒वैव यं ॅय मे॒व । \newline
21. ए॒वास्मा॑ अस्मा ए॒वैवास्मै᳚ । \newline
22. अ॒स्मै॒ तौ ता व॑स्मा अस्मै॒ तौ । \newline
23. तौ व॑ल॒गं ॅव॑ल॒गम् तौ तौ व॑ल॒गम् । \newline
24. व॒ल॒गम् नि॒खन॑तो नि॒खन॑तो वल॒गं ॅव॑ल॒गम् नि॒खन॑तः । \newline
25. व॒ल॒गमिति॑ वल - गम् । \newline
26. नि॒खन॑त॒ स्तम् तम् नि॒खन॑तो नि॒खन॑त॒ स्तम् । \newline
27. नि॒खन॑त॒ इति॑ नि - खन॑तः । \newline
28. त मे॒वैव तम् त मे॒व । \newline
29. ए॒वोदु दे॒वैवोत् । \newline
30. उद् व॑पति वप॒ त्युदुद् व॑पति । \newline
31. व॒प॒ति॒ सꣳ सं ॅव॑पति वपति॒ सम् । \newline
32. सम् तृ॑णत्ति तृणत्ति॒ सꣳ सम् तृ॑णत्ति । \newline
33. तृ॒ण॒त्ति॒ तस्मा॒त् तस्मा᳚त् तृणत्ति तृणत्ति॒ तस्मा᳚त् । \newline
34. तस्मा॒थ् सन्तृ॑ण्णाः॒ सन्तृ॑ण्णा॒ स्तस्मा॒त् तस्मा॒थ् सन्तृ॑ण्णाः । \newline
35. सन्तृ॑ण्णा अन्तर॒तो᳚ ऽन्तर॒तः सन्तृ॑ण्णाः॒ सन्तृ॑ण्णा अन्तर॒तः । \newline
36. सन्तृ॑ण्णा॒ इति॒ सं - तृ॒ण्णाः॒ । \newline
37. अ॒न्त॒र॒तः प्रा॒णाः प्रा॒णा अ॑न्तर॒तो᳚ ऽन्तर॒तः प्रा॒णाः । \newline
38. प्रा॒णा न न प्रा॒णाः प्रा॒णा न । \newline
39. प्रा॒णा इति॑ प्र - अ॒नाः । \newline
40. न सꣳ सन्न न सम् । \newline
41. सम् भि॑नत्ति भिनत्ति॒ सꣳ सम् भि॑नत्ति । \newline
42. भि॒न॒त्ति॒ तस्मा॒त् तस्मा᳚द् भिनत्ति भिनत्ति॒ तस्मा᳚त् । \newline
43. तस्मा॒ दस॑म्भिन्ना॒ अस॑म्भिन्ना॒ स्तस्मा॒त् तस्मा॒ दस॑म्भिन्नाः । \newline
44. अस॑म्भिन्नाः प्रा॒णाः प्रा॒णा अस॑म्भिन्ना॒ अस॑म्भिन्नाः प्रा॒णाः । \newline
45. अस॑म्भिन्ना॒ इत्यसं᳚ - भि॒न्नाः॒ । \newline
46. प्रा॒णा अ॒पो॑ ऽपः प्रा॒णाः प्रा॒णा अ॒पः । \newline
47. प्रा॒णा इति॑ प्र - अ॒नाः । \newline
48. अ॒पो ऽवावा॒पो॑ ऽपो ऽव॑ । \newline
49. अव॑ नयति नय॒ त्यवाव॑ नयति । \newline
50. न॒य॒ति॒ तस्मा॒त् तस्मा᳚न् नयति नयति॒ तस्मा᳚त् । \newline
51. तस्मा॑ दा॒र्द्रा आ॒र्द्रा स्तस्मा॒त् तस्मा॑ दा॒र्द्राः । \newline
52. आ॒र्द्रा अ॑न्तर॒तो᳚ ऽन्तर॒त आ॒र्द्रा आ॒र्द्रा अ॑न्तर॒तः । \newline
53. अ॒न्त॒र॒तः प्रा॒णाः प्रा॒णा अ॑न्तर॒तो᳚ ऽन्तर॒तः प्रा॒णाः । \newline
54. प्रा॒णा यव॑मती॒र् यव॑मतीः प्रा॒णाः प्रा॒णा यव॑मतीः । \newline
55. प्रा॒णा इति॑ प्र - अ॒नाः । \newline
56. यव॑मती॒ रवाव॒ यव॑मती॒र् यव॑मती॒ रव॑ । \newline
57. यव॑मती॒रिति॒ यव॑ - म॒तीः॒ । \newline
58. अव॑ नयति नय॒ त्यवाव॑ नयति । \newline
59. न॒य॒ त्यूर् गूर्ङ् न॑यति नय॒ त्यूर्क् । \newline

\textbf{Ghana Paata } \newline

1. यन् नो॑ नो॒ यं ॅयन् नः॑ समा॒नः स॑मा॒नो नो॒ यं ॅयन् नः॑ समा॒नः । \newline
2. नः॒ स॒मा॒नः स॑मा॒नो नो॑ नः समा॒नो यं ॅयꣳ स॑मा॒नो नो॑ नः समा॒नो यम् । \newline
3. स॒मा॒नो यं ॅयꣳ स॑मा॒नः स॑मा॒नो यमस॑मा॒नो ऽस॑मानो॒ यꣳ स॑मा॒नः स॑मा॒नो 
यमस॑मानः । \newline
4. यमस॑मा॒नो ऽस॑मानो॒ यं ॅयमस॑मानो निच॒खान॑ निच॒खाना स॑मानो॒ यं ॅयमस॑मानो निच॒खान॑ । \newline
5. अस॑मानो निच॒खान॑ निच॒खाना स॑मा॒नो ऽस॑मानो निच॒खाने तीति॑ निच॒खाना स॑मा॒नो ऽस॑मानो निच॒खानेति॑ । \newline
6. नि॒च॒खाने तीति॑ निच॒खान॑ निच॒खाने त्या॑हा॒हेति॑ निच॒खान॑ निच॒खाने त्या॑ह । \newline
7. नि॒च॒खानेति॑ नि - च॒खान॑ । \newline
8. इत्या॑हा॒हे तीत्या॑ह॒ द्वौ द्वा वा॒हे तीत्या॑ह॒ द्वौ । \newline
9. आ॒ह॒ द्वौ द्वा वा॑हाह॒ द्वौ वाव वाव द्वा वा॑हाह॒ द्वौ वाव । \newline
10. द्वौ वाव वाव द्वौ द्वौ वाव पुरु॑षौ॒ पुरु॑षौ॒ वाव द्वौ द्वौ वाव पुरु॑षौ । \newline
11. वाव पुरु॑षौ॒ पुरु॑षौ॒ वाव वाव पुरु॑षौ॒ यो यः पुरु॑षौ॒ वाव वाव पुरु॑षौ॒ यः । \newline
12. पुरु॑षौ॒ यो यः पुरु॑षौ॒ पुरु॑षौ॒ यश्च॑ च॒ यः पुरु॑षौ॒ पुरु॑षौ॒ यश्च॑ । \newline
13. यश्च॑ च॒ यो यश्चै॒ वैव च॒ यो यश्चै॒व । \newline
14. चै॒वैव च॑ चै॒व स॑मा॒नः स॑मा॒न ए॒व च॑ चै॒व स॑मा॒नः । \newline
15. ए॒व स॑मा॒नः स॑मा॒न ए॒वैव स॑मा॒नो यो यः स॑मा॒न ए॒वैव स॑मा॒नो यः । \newline
16. स॒मा॒नो यो यः स॑मा॒नः स॑मा॒नो यश्च॑ च॒ यः स॑मा॒नः स॑मा॒नो यश्च॑ । \newline
17. यश्च॑ च॒ यो यश्चास॑मा॒नो ऽस॑मानश्च॒ यो यश्चास॑मानः । \newline
18. चास॑मा॒नो ऽस॑मानश्च॒ चास॑मानो॒ यं ॅयमस॑मानश्च॒ चास॑मानो॒ यम् । \newline
19. अस॑मानो॒ यं ॅयमस॑मा॒नो ऽस॑मानो॒ य मे॒वैव यमस॑मा॒नो ऽस॑मानो॒ य मे॒व । \newline
20. य मे॒वैव यं ॅय मे॒वास्मा॑ अस्मा ए॒व यं ॅय मे॒वास्मै᳚ । \newline
21. ए॒वास्मा॑ अस्मा ए॒वै वास्मै॒ तौ ता व॑स्मा ए॒वै वास्मै॒ तौ । \newline
22. अ॒स्मै॒ तौ ता व॑स्मा अस्मै॒ तौ व॑ल॒गं ॅव॑ल॒गम् ता व॑स्मा अस्मै॒ तौ व॑ल॒गम् । \newline
23. तौ व॑ल॒गं ॅव॑ल॒गम् तौ तौ व॑ल॒गम् नि॒खन॑तो नि॒खन॑तो वल॒गम् तौ तौ व॑ल॒गम् नि॒खन॑तः । \newline
24. व॒ल॒गम् नि॒खन॑तो नि॒खन॑तो वल॒गं ॅव॑ल॒गम् नि॒खन॑त॒ स्तम् तन् नि॒खन॑तो वल॒गं ॅव॑ल॒गम् नि॒खन॑त॒ स्तम् । \newline
25. व॒ल॒गमिति॑ वल - गम् । \newline
26. नि॒खन॑त॒ स्तम् तन् नि॒खन॑तो नि॒खन॑त॒ स्त मे॒वैव तन् नि॒खन॑तो नि॒खन॑त॒ स्त मे॒व । \newline
27. नि॒खन॑त॒ इति॑ नि - खन॑तः । \newline
28. त मे॒वैव तम् त मे॒वोदु दे॒व तम् त मे॒वोत् । \newline
29. ए॒वो दुदे॒ वैवोद् व॑पति वप॒ त्युदे॒ वैवोद् व॑पति । \newline
30. उद् व॑पति वप॒ त्युदुद् व॑पति॒ सꣳ सं ॅव॑प॒ त्युदुद् व॑पति॒ सम् । \newline
31. व॒प॒ति॒ सꣳ सं ॅव॑पति वपति॒ सम् तृ॑णत्ति तृणत्ति॒ सं ॅव॑पति वपति॒ सम् तृ॑णत्ति । \newline
32. सम् तृ॑णत्ति तृणत्ति॒ सꣳ सम् तृ॑णत्ति॒ तस्मा॒त् तस्मा᳚त् तृणत्ति॒ सꣳ सम् तृ॑णत्ति॒ तस्मा᳚त् । \newline
33. तृ॒ण॒त्ति॒ तस्मा॒त् तस्मा᳚त् तृणत्ति तृणत्ति॒ तस्मा॒थ् सन्तृ॑ण्णाः॒ सन्तृ॑ण्णा॒ स्तस्मा᳚त् तृणत्ति तृणत्ति॒ तस्मा॒थ् सन्तृ॑ण्णाः । \newline
34. तस्मा॒थ् सन्तृ॑ण्णाः॒ सन्तृ॑ण्णा॒ स्तस्मा॒त् तस्मा॒थ् सन्तृ॑ण्णा अन्तर॒तो᳚ ऽन्तर॒तः सन्तृ॑ण्णा॒ स्तस्मा॒त् तस्मा॒थ् सन्तृ॑ण्णा अन्तर॒तः । \newline
35. सन्तृ॑ण्णा अन्तर॒तो᳚ ऽन्तर॒तः सन्तृ॑ण्णाः॒ सन्तृ॑ण्णा अन्तर॒तः प्रा॒णाः प्रा॒णा अ॑न्तर॒तः सन्तृ॑ण्णाः॒ सन्तृ॑ण्णा अन्तर॒तः प्रा॒णाः । \newline
36. सन्तृ॑ण्णा॒ इति॒ सं - तृ॒ण्णाः॒ । \newline
37. अ॒न्त॒र॒तः प्रा॒णाः प्रा॒णा अ॑न्तर॒तो᳚ ऽन्तर॒तः प्रा॒णा न न प्रा॒णा अ॑न्तर॒तो᳚ ऽन्तर॒तः प्रा॒णा न । \newline
38. प्रा॒णा न न प्रा॒णाः प्रा॒णा न सꣳ सन्न प्रा॒णाः प्रा॒णा न सम् । \newline
39. प्रा॒णा इति॑ प्र - अ॒नाः । \newline
40. न सꣳ सन्न न सम् भि॑नत्ति भिनत्ति॒ सन्न न सम् भि॑नत्ति । \newline
41. सम् भि॑नत्ति भिनत्ति॒ सꣳ सम् भि॑नत्ति॒ तस्मा॒त् तस्मा᳚द् भिनत्ति॒ सꣳ सम् भि॑नत्ति॒ तस्मा᳚त् । \newline
42. भि॒न॒त्ति॒ तस्मा॒त् तस्मा᳚द् भिनत्ति भिनत्ति॒ तस्मा॒ दस॑म्भिन्ना॒ अस॑म्भिन्ना॒ स्तस्मा᳚द् भिनत्ति भिनत्ति॒ तस्मा॒ दस॑म्भिन्नाः । \newline
43. तस्मा॒ दस॑म्भिन्ना॒ अस॑म्भिन्ना॒ स्तस्मा॒त् तस्मा॒ दस॑म्भिन्नाः प्रा॒णाः प्रा॒णा अस॑म्भिन्ना॒ स्तस्मा॒त् तस्मा॒ दस॑म्भिन्नाः प्रा॒णाः । \newline
44. अस॑म्भिन्नाः प्रा॒णाः प्रा॒णा अस॑म्भिन्ना॒ अस॑म्भिन्नाः प्रा॒णा अ॒पो॑ ऽपः प्रा॒णा अस॑म्भिन्ना॒ अस॑म्भिन्नाः प्रा॒णा अ॒पः । \newline
45. अस॑म्भिन्ना॒ इत्यसं᳚ - भि॒न्नाः॒ । \newline
46. प्रा॒णा अ॒पो॑ ऽपः प्रा॒णाः प्रा॒णा अ॒पो ऽवावा॒पः प्रा॒णाः प्रा॒णा अ॒पो ऽव॑ । \newline
47. प्रा॒णा इति॑ प्र - अ॒नाः । \newline
48. अ॒पो ऽवावा॒पो॑ ऽपो ऽव॑ नयति नय॒ त्यवा॒पो॑ ऽपो ऽव॑ नयति । \newline
49. अव॑ नयति नय॒ त्यवाव॑ नयति॒ तस्मा॒त् तस्मा᳚न् नय॒ त्यवाव॑ नयति॒ तस्मा᳚त् । \newline
50. न॒य॒ति॒ तस्मा॒त् तस्मा᳚न् नयति नयति॒ तस्मा॑ दा॒र्द्रा आ॒र्द्रा स्तस्मा᳚न् नयति नयति॒ तस्मा॑ दा॒र्द्राः । \newline
51. तस्मा॑ दा॒र्द्रा आ॒र्द्रा स्तस्मा॒त् तस्मा॑ दा॒र्द्रा अ॑न्तर॒तो᳚ ऽन्तर॒त आ॒र्द्रा स्तस्मा॒त् तस्मा॑ दा॒र्द्रा अ॑न्तर॒तः । \newline
52. आ॒र्द्रा अ॑न्तर॒तो᳚ ऽन्तर॒त आ॒र्द्रा आ॒र्द्रा अ॑न्तर॒तः प्रा॒णाः प्रा॒णा अ॑न्तर॒त आ॒र्द्रा आ॒र्द्रा अ॑न्तर॒तः प्रा॒णाः । \newline
53. अ॒न्त॒र॒तः प्रा॒णाः प्रा॒णा अ॑न्तर॒तो᳚ ऽन्तर॒तः प्रा॒णा यव॑मती॒र् यव॑मतीः प्रा॒णा अ॑न्तर॒तो᳚ ऽन्तर॒तः प्रा॒णा यव॑मतीः । \newline
54. प्रा॒णा यव॑मती॒र् यव॑मतीः प्रा॒णाः प्रा॒णा यव॑मती॒ रवाव॒ यव॑मतीः प्रा॒णाः प्रा॒णा यव॑मती॒ रव॑ । \newline
55. प्रा॒णा इति॑ प्र - अ॒नाः । \newline
56. यव॑मती॒ रवाव॒ यव॑मती॒र् यव॑मती॒ रव॑ नयति नय॒ त्यव॒ यव॑मती॒र् यव॑मती॒ रव॑ नयति । \newline
57. यव॑मती॒रिति॒ यव॑ - म॒तीः॒ । \newline
58. अव॑ नयति नय॒ त्यवाव॑ नय॒ त्यूर् गूर्ङ् न॑य॒ त्यवाव॑ नय॒त्यूर्क् । \newline
59. न॒य॒ त्यूर् गूर्ङ् न॑यति नय॒ त्यूर्ग् वै वा ऊर्ङ् न॑यति नय॒ त्यूर्ग् वै । \newline
\pagebreak
\markright{ TS 6.2.11.3  \hfill https://www.vedavms.in \hfill}

\section{ TS 6.2.11.3 }

\textbf{TS 6.2.11.3 } \newline
\textbf{Samhita Paata} \newline

-त्यूर्ग्वै यवः॑ प्रा॒णा उ॑पर॒वाः प्रा॒णेष्वे॒वोर्जं॑ दधाति ब॒र्॒.हिरव॑ स्तृणाति॒ तस्मा᳚ल्लोम॒शा अ॑न्तर॒तः प्रा॒णा आज्ये॑न॒ व्याघा॑रयति॒ तेजो॒ वा आज्यं॑ प्रा॒णा उ॑पर॒वाः प्रा॒णेष्वे॒व तेजो॑ दधाति॒ हनू॒ वा ए॒ते य॒ज्ञ्स्य॒ यद॑धि॒षव॑णे॒ न सं तृ॑ण॒त्त्य स॑तृंण्णे॒ हि हनू॒ अथो॒ खलु॑ दीर्घसो॒मे स॒तृंद्ये॒ धृत्यै॒ शिरो॒ वा ए॒तद्-य॒ज्ञ्स्य॒ यद्ध॑वि॒र्द्धानं॑- [  ] \newline

\textbf{Pada Paata} \newline

ऊर्क् । वै । यवः॑ । प्रा॒णा इति॑ प्र - अ॒नाः । उ॒प॒र॒वा इत्यु॑प-र॒वाः । प्रा॒णेष्विति॑ प्र - अ॒नेषु॑ । ए॒व । ऊर्ज᳚म् । द॒धा॒ति॒ । ब॒र्॒.हिः । अवेति॑ । स्तृ॒णा॒ति॒ । तस्मा᳚त् । लो॒म॒शाः । अ॒न्त॒र॒तः । प्रा॒णा इति॑ प्र - अ॒नाः । आज्ये॑न । व्याघा॑रय॒तीति॑ वि - आघा॑रयति । तेजः॑ । वै । आज्य᳚म् । प्रा॒णा इति॑ प्र - अ॒नाः । उ॒प॒र॒वा इत्यु॑प - र॒वाः । प्रा॒णेष्विति॑ प्र - अ॒नेषु॑ । ए॒व । तेजः॑ । द॒धा॒ति॒ । हनू॒ इति॑ । वै । ए॒ते इति॑ । य॒ज्ञ्स्य॑ । यत् । अ॒धि॒षव॑णे॒ इत्य॑धि-सव॑ने । न । समिति॑ । तृ॒ण॒त्ति॒ । अस॑तृंण्णे॒ इत्यसं᳚ - तृ॒ण्णे॒ । हि । हनू॒ इति॑ । अथो॒ इति॑ । खलु॑ । दी॒र्घ॒सो॒म इति॑ दीर्घ - सो॒मे । स॒तृंद्ये॒ इति॑ सं-तृद्ये᳚ । धृत्यै᳚ । शिरः॑ । वै । ए॒तत् । य॒ज्ञ्स्य॑ । यत् । ह॒वि॒द्‌र्धान॒मिति॑ हविः - धान᳚म् ।  \newline


\textbf{Krama Paata} \newline

ऊर्ग् वै । वै यवः॑ । यवः॑ प्रा॒णाः । प्रा॒णा उ॑पर॒वाः । प्रा॒णा इति॑ प्र - अ॒नाः । उ॒प॒र॒वाः प्रा॒णेषु॑ । उ॒प॒र॒वा इत्यु॑प - र॒वाः । प्रा॒णेष्वे॒व । प्रा॒णेष्विति॑ प्र - अ॒नेषु॑ । ए॒वोर्ज᳚म् । ऊर्ज॑म् दधाति । द॒धा॒ति॒ ब॒र्.॒हिः । ब॒र्.॒हिरव॑ । अव॑ स्तृणाति । स्तृ॒णा॒ति॒ तस्मा᳚त् । तस्मा᳚ल्लोम॒शाः । लो॒म॒शा अ॑न्तर॒तः । अ॒न्त॒र॒तः प्रा॒णाः । प्रा॒णा आज्ये॑न । प्रा॒णा इति॑ प्र - अ॒नाः । आज्ये॑न॒ व्याघा॑रयति । व्याघा॑रयति॒ तेजः॑ । व्याघा॑रय॒तीति॑ वि - आघा॑रयति । तेजो॒ वै । वा आज्य᳚म् । आज्य॑म् प्रा॒णाः । प्रा॒णा उ॑पर॒वाः । प्रा॒णा इति॑ प्र - अ॒नाः । उ॒प॒र॒वाः प्रा॒णेषु॑ । उ॒प॒र॒वा इत्यु॑प - र॒वाः । प्रा॒णेष्वे॒व । प्रा॒णेष्विति॑ प्र - अ॒नेषु॑ । ए॒व तेजः॑ । तेजो॑ दधाति । द॒धा॒ति॒ हनू᳚ । हनू॒ वै । हनू॒ इति॒ हनू᳚ । वा ए॒ते । ए॒ते य॒ज्ञ्स्य॑ । ए॒ते इत्ये॒ते । य॒ज्ञ्स्य॒ यत् । यद॑धि॒षव॑णे । अ॒धि॒षव॑णे॒ न । अ॒धि॒षव॑णे॒ इत्य॑धि - सव॑ने । न सम् । सम् तृ॑णत्ति । तृ॒ण॒त्यस॑न्तृण्णे । अस॑न्तृण्णे॒ हि । अस॑न्तृण्णे॒ इत्यस᳚म् - तृ॒ण्णे॒ । हि हनू᳚ । हनू॒ अथो᳚ । हनू॒ इति॒ हनू᳚ । अथो॒ खलु॑ । अथो॒ इत्यथो᳚ । खलु॑ दीर्घसो॒मे । दी॒र्घ॒सो॒मे स॒न्तृद्ये᳚ । दी॒र्घ॒सो॒म इति॑ दीर्घ - सो॒मे । स॒न्तृद्ये॒ धृत्यै᳚ । स॒न्तृद्ये॒ इति॑ सम् - तृद्ये᳚ । धृत्यै॒ शिरः॑ । शिरो॒ वै । वा ए॒तत् । ए॒तद् य॒ज्ञ्स्य॑ । य॒ज्ञ्स्य॒ यत् । यद्‍ध॑वि॒र्द्धान᳚म् । ह॒वि॒र्द्धान॑म् प्रा॒णाः । ह॒वि॒र्द्धान॒मिति॑ हविः - धान᳚म् \newline

\textbf{Jatai Paata} \newline

1. ऊर्ग् वै वा ऊर् गूर्ग् वै । \newline
2. वै यवो॒ यवो॒ वै वै यवः॑ । \newline
3. यवः॑ प्रा॒णाः प्रा॒णा यवो॒ यवः॑ प्रा॒णाः । \newline
4. प्रा॒णा उ॑पर॒वा उ॑पर॒वाः प्रा॒णाः प्रा॒णा उ॑पर॒वाः । \newline
5. प्रा॒णा इति॑ प्र - अ॒नाः । \newline
6. उ॒प॒र॒वाः प्रा॒णेषु॑ प्रा॒णेषू॑ पर॒वा उ॑पर॒वाः प्रा॒णेषु॑ । \newline
7. उ॒प॒र॒वा इत्यु॑प - र॒वाः । \newline
8. प्रा॒णे ष्वे॒वैव प्रा॒णेषु॑ प्रा॒णे ष्वे॒व । \newline
9. प्रा॒णेष्विति॑ प्र - अ॒नेषु॑ । \newline
10. ए॒वोर्ज॒ मूर्ज॑ मे॒वैवोर्ज᳚म् । \newline
11. ऊर्ज॑म् दधाति दधा॒ त्यूर्ज॒ मूर्ज॑म् दधाति । \newline
12. द॒धा॒ति॒ ब॒र्॒.हिर् ब॒र्॒.हिर् द॑धाति दधाति ब॒र्॒.हिः । \newline
13. ब॒र्॒.हि रवाव॑ ब॒र्॒.हिर् ब॒र्॒.हि रव॑ । \newline
14. अव॑ स्तृणाति स्तृणा॒ त्यवाव॑ स्तृणाति । \newline
15. स्तृ॒णा॒ति॒ तस्मा॒त् तस्मा᳚थ् स्तृणाति स्तृणाति॒ तस्मा᳚त् । \newline
16. तस्मा᳚ ल्लोम॒शा लो॑म॒शा स्तस्मा॒त् तस्मा᳚ ल्लोम॒शाः । \newline
17. लो॒म॒शा अ॑न्तर॒तो᳚ ऽन्तर॒तो लो॑म॒शा लो॑म॒शा अ॑न्तर॒तः । \newline
18. अ॒न्त॒र॒तः प्रा॒णाः प्रा॒णा अ॑न्तर॒तो᳚ ऽन्तर॒तः प्रा॒णाः । \newline
19. प्रा॒णा आज्ये॒ना ज्ये॑न प्रा॒णाः प्रा॒णा आज्ये॑न । \newline
20. प्रा॒णा इति॑ प्र - अ॒नाः । \newline
21. आज्ये॑न॒ व्याघा॑रयति॒ व्याघा॑रय॒ त्याज्ये॒ना ज्ये॑न॒ व्याघा॑रयति । \newline
22. व्याघा॑रयति॒ तेज॒ स्तेजो॒ व्याघा॑रयति॒ व्याघा॑रयति॒ तेजः॑ । \newline
23. व्याघा॑रय॒तीति॑ वि - आघा॑रयति । \newline
24. तेजो॒ वै वै तेज॒ स्तेजो॒ वै । \newline
25. वा आज्य॒ माज्यं॒ ॅवै वा आज्य᳚म् । \newline
26. आज्य॑म् प्रा॒णाः प्रा॒णा आज्य॒ माज्य॑म् प्रा॒णाः । \newline
27. प्रा॒णा उ॑पर॒वा उ॑पर॒वाः प्रा॒णाः प्रा॒णा उ॑पर॒वाः । \newline
28. प्रा॒णा इति॑ प्र - अ॒नाः । \newline
29. उ॒प॒र॒वाः प्रा॒णेषु॑ प्रा॒णेषू॑ पर॒वा उ॑पर॒वाः प्रा॒णेषु॑ । \newline
30. उ॒प॒र॒वा इत्यु॑प - र॒वाः । \newline
31. प्रा॒णे ष्वे॒वैव प्रा॒णेषु॑ प्रा॒णे ष्वे॒व । \newline
32. प्रा॒णेष्विति॑ प्र - अ॒नेषु॑ । \newline
33. ए॒व तेज॒ स्तेज॑ ए॒वैव तेजः॑ । \newline
34. तेजो॑ दधाति दधाति॒ तेज॒ स्तेजो॑ दधाति । \newline
35. द॒धा॒ति॒ हनू॒ हनू॑ दधाति दधाति॒ हनू᳚ । \newline
36. हनू॒ वै वै हनू॒ हनू॒ वै । \newline
37. हनू॒ इति॒ हनू᳚ । \newline
38. वा ए॒ते ए॒ते वै वा ए॒ते । \newline
39. ए॒ते य॒ज्ञ्स्य॑ य॒ज्ञ् स्यै॒ते ए॒ते य॒ज्ञ्स्य॑ । \newline
40. ए॒ते इत्ये॒ते । \newline
41. य॒ज्ञ्स्य॒ यद् यद् य॒ज्ञ्स्य॑ य॒ज्ञ्स्य॒ यत् । \newline
42. यद॑धि॒षव॑णे अधि॒षव॑णे॒ यद् यद॑धि॒षव॑णे । \newline
43. अ॒धि॒षव॑णे॒ न नाधि॒षव॑णे अधि॒षव॑णे॒ न । \newline
44. अ॒धि॒षव॑णे॒ इत्य॑धि - सव॑ने । \newline
45. न सꣳ सन्न न सम् । \newline
46. सम् तृ॑णत्ति तृणत्ति॒ सꣳ सम् तृ॑णत्ति । \newline
47. तृ॒ण॒ त्त्यस॑न्तृण्णे॒ अस॑न्तृण्णे तृणत्ति तृण॒ त्त्यस॑न्तृण्णे । \newline
48. अस॑न्तृण्णे॒ हि ह्यस॑न्तृण्णे॒ अस॑न्तृण्णे॒ हि । \newline
49. अस॑न्तृण्णे॒ इत्यसं᳚ - तृ॒ण्णे॒ । \newline
50. हि हनू॒ हनू॒ हि हि हनू᳚ । \newline
51. हनू॒ अथो॒ अथो॒ हनू॒ हनू॒ अथो᳚ । \newline
52. हनू॒ इति॒ हनू᳚ । \newline
53. अथो॒ खलु॒ खल्वथो॒ अथो॒ खलु॑ । \newline
54. अथो॒ इत्यथो᳚ । \newline
55. खलु॑ दीर्घसो॒मे दी᳚र्घसो॒मे खलु॒ खलु॑ दीर्घसो॒मे । \newline
56. दी॒र्घ॒सो॒मे स॒न्तृद्ये॑ स॒न्तृद्ये॑ दीर्घसो॒मे दी᳚र्घसो॒मे स॒न्तृद्ये᳚ । \newline
57. दी॒र्घ॒सो॒म इति॑ दीर्घ - सो॒मे । \newline
58. स॒न्तृद्ये॒ धृत्यै॒ धृत्यै॑ स॒न्तृद्ये॑ स॒न्तृद्ये॒ धृत्यै᳚ । \newline
59. स॒न्तृद्ये॒ इति॑ सं - तृद्ये᳚ । \newline
60. धृत्यै॒ शिरः॒ शिरो॒ धृत्यै॒ धृत्यै॒ शिरः॑ । \newline
61. शिरो॒ वै वै शिरः॒ शिरो॒ वै । \newline
62. वा ए॒त दे॒तद् वै वा ए॒तत् । \newline
63. ए॒तद् य॒ज्ञ्स्य॑ य॒ज्ञ् स्यै॒त दे॒तद् य॒ज्ञ्स्य॑ । \newline
64. य॒ज्ञ्स्य॒ यद् यद् य॒ज्ञ्स्य॑ य॒ज्ञ्स्य॒ यत् । \newline
65. यद्ध॑वि॒र्द्धानꣳ॑ हवि॒र्द्धानं॒ ॅयद् यद्ध॑वि॒र्द्धान᳚म् । \newline
66. ह॒वि॒र्द्धान॑म् प्रा॒णाः प्रा॒णा ह॑वि॒र्द्धानꣳ॑ हवि॒र्द्धान॑म् प्रा॒णाः । \newline
67. ह॒वि॒र्द्धान॒मिति॑ हविः - धान᳚म् । \newline

\textbf{Ghana Paata } \newline

1. ऊर्ग् वै वा ऊर् गूर्ग् वै यवो॒ यवो॒ वा ऊर् गूर्ग् वै यवः॑ । \newline
2. वै यवो॒ यवो॒ वै वै यवः॑ प्रा॒णाः प्रा॒णा यवो॒ वै वै यवः॑ प्रा॒णाः । \newline
3. यवः॑ प्रा॒णाः प्रा॒णा यवो॒ यवः॑ प्रा॒णा उ॑पर॒वा उ॑पर॒वाः प्रा॒णा यवो॒ यवः॑ प्रा॒णा उ॑पर॒वाः । \newline
4. प्रा॒णा उ॑पर॒वा उ॑पर॒वाः प्रा॒णाः प्रा॒णा उ॑पर॒वाः प्रा॒णेषु॑ प्रा॒णेषू॑ पर॒वाः प्रा॒णाः प्रा॒णा उ॑पर॒वाः प्रा॒णेषु॑ । \newline
5. प्रा॒णा इति॑ प्र - अ॒नाः । \newline
6. उ॒प॒र॒वाः प्रा॒णेषु॑ प्रा॒णेषू॑ पर॒वा उ॑पर॒वाः प्रा॒णे ष्वे॒वैव प्रा॒णेषू॑ पर॒वा उ॑पर॒वाः प्रा॒णेष्वे॒व । \newline
7. उ॒प॒र॒वा इत्यु॑प - र॒वाः । \newline
8. प्रा॒णे ष्वे॒वैव प्रा॒णेषु॑ प्रा॒णे ष्वे॒वोर्ज॒ मूर्ज॑ मे॒व प्रा॒णेषु॑ प्रा॒णे ष्वे॒वोर्ज᳚म् । \newline
9. प्रा॒णेष्विति॑ प्र - अ॒नेषु॑ । \newline
10. ए॒वोर्ज॒ मूर्ज॑ मे॒वै वोर्ज॑म् दधाति दधा॒ त्यूर्ज॑ मे॒वै वोर्ज॑म् दधाति । \newline
11. ऊर्ज॑म् दधाति दधा॒ त्यूर्ज॒ मूर्ज॑म् दधाति ब॒र्॒.हिर् ब॒र्॒.हिर् द॑धा॒ त्यूर्ज॒ मूर्ज॑म् दधाति ब॒र्॒.हिः । \newline
12. द॒धा॒ति॒ ब॒र्॒.हिर् ब॒र्॒.हिर् द॑धाति दधाति ब॒र्॒.हि रवाव॑ ब॒र्॒.हिर् द॑धाति दधाति ब॒र्॒.हि रव॑ । \newline
13. ब॒र्॒.हि रवाव॑ ब॒र्॒.हिर् ब॒र्॒.हि रव॑ स्तृणाति स्तृणा॒ त्यव॑ ब॒र्॒.हिर् ब॒र्॒.हि रव॑ स्तृणाति । \newline
14. अव॑ स्तृणाति स्तृणा॒ त्यवाव॑ स्तृणाति॒ तस्मा॒त् तस्मा᳚थ् स्तृणा॒ त्यवाव॑ स्तृणाति॒ तस्मा᳚त् । \newline
15. स्तृ॒णा॒ति॒ तस्मा॒त् तस्मा᳚थ् स्तृणाति स्तृणाति॒ तस्मा᳚ ल्लोम॒शा लो॑म॒शा स्तस्मा᳚थ् स्तृणाति स्तृणाति॒ तस्मा᳚
ल्लोम॒शाः । \newline
16. तस्मा᳚ ल्लोम॒शा लो॑म॒शा स्तस्मा॒त् तस्मा᳚ ल्लोम॒शा अ॑न्तर॒तो᳚ ऽन्तर॒तो लो॑म॒शा स्तस्मा॒त् तस्मा᳚ ल्लोम॒शा अ॑न्तर॒तः । \newline
17. लो॒म॒शा अ॑न्तर॒तो᳚ ऽन्तर॒तो लो॑म॒शा लो॑म॒शा अ॑न्तर॒तः प्रा॒णाः प्रा॒णा अ॑न्तर॒तो लो॑म॒शा लो॑म॒शा अ॑न्तर॒तः प्रा॒णाः । \newline
18. अ॒न्त॒र॒तः प्रा॒णाः प्रा॒णा अ॑न्तर॒तो᳚ ऽन्तर॒तः प्रा॒णा आज्ये॒ना ज्ये॑न प्रा॒णा अ॑न्तर॒तो᳚ ऽन्तर॒तः प्रा॒णा आज्ये॑न । \newline
19. प्रा॒णा आज्ये॒ना ज्ये॑न प्रा॒णाः प्रा॒णा आज्ये॑न॒ व्याघा॑रयति॒ व्याघा॑रय॒ त्याज्ये॑न प्रा॒णाः प्रा॒णा आज्ये॑न॒ व्याघा॑रयति । \newline
20. प्रा॒णा इति॑ प्र - अ॒नाः । \newline
21. आज्ये॑न॒ व्याघा॑रयति॒ व्याघा॑रय॒ त्याज्ये॒ना ज्ये॑न॒ व्याघा॑रयति॒ तेज॒ स्तेजो॒ व्याघा॑रय॒ त्याज्ये॒ना ज्ये॑न॒ व्याघा॑रयति॒ तेजः॑ । \newline
22. व्याघा॑रयति॒ तेज॒ स्तेजो॒ व्याघा॑रयति॒ व्याघा॑रयति॒ तेजो॒ वै वै तेजो॒ व्याघा॑रयति॒ व्याघा॑रयति॒ तेजो॒ वै । \newline
23. व्याघा॑रय॒तीति॑ वि - आघा॑रयति । \newline
24. तेजो॒ वै वै तेज॒ स्तेजो॒ वा आज्य॒ माज्यं॒ ॅवै तेज॒ स्तेजो॒ वा आज्य᳚म् । \newline
25. वा आज्य॒ माज्यं॒ ॅवै वा आज्य॑म् प्रा॒णाः प्रा॒णा आज्यं॒ ॅवै वा आज्य॑म् प्रा॒णाः । \newline
26. आज्य॑म् प्रा॒णाः प्रा॒णा आज्य॒ माज्य॑म् प्रा॒णा उ॑पर॒वा उ॑पर॒वाः प्रा॒णा आज्य॒ माज्य॑म् प्रा॒णा उ॑पर॒वाः । \newline
27. प्रा॒णा उ॑पर॒वा उ॑पर॒वाः प्रा॒णाः प्रा॒णा उ॑पर॒वाः प्रा॒णेषु॑ प्रा॒णेषू॑ पर॒वाः प्रा॒णाः प्रा॒णा उ॑पर॒वाः प्रा॒णेषु॑ । \newline
28. प्रा॒णा इति॑ प्र - अ॒नाः । \newline
29. उ॒प॒र॒वाः प्रा॒णेषु॑ प्रा॒णेषू॑ पर॒वा उ॑पर॒वाः प्रा॒णे ष्वे॒वैव प्रा॒णेषू॑ पर॒वा उ॑पर॒वाः प्रा॒णेष्वे॒व । \newline
30. उ॒प॒र॒वा इत्यु॑प - र॒वाः । \newline
31. प्रा॒णेष्वे॒ वैव प्रा॒णेषु॑ प्रा॒णेष्वे॒व तेज॒ स्तेज॑ ए॒व प्रा॒णेषु॑ प्रा॒णेष्वे॒व तेजः॑ । \newline
32. प्रा॒णेष्विति॑ प्र - अ॒नेषु॑ । \newline
33. ए॒व तेज॒ स्तेज॑ ए॒वैव तेजो॑ दधाति दधाति॒ तेज॑ ए॒वैव तेजो॑ दधाति । \newline
34. तेजो॑ दधाति दधाति॒ तेज॒ स्तेजो॑ दधाति॒ हनू॒ हनू॑ दधाति॒ तेज॒ स्तेजो॑ दधाति॒ हनू᳚ । \newline
35. द॒धा॒ति॒ हनू॒ हनू॑ दधाति दधाति॒ हनू॒ वै वै हनू॑ दधाति दधाति॒ हनू॒ वै । \newline
36. हनू॒ वै वै हनू॒ हनू॒ वा ए॒ते ए॒ते वै हनू॒ हनू॒ वा ए॒ते । \newline
37. हनू॒ इति॒ हनू᳚ । \newline
38. वा ए॒ते ए॒ते वै वा ए॒ते य॒ज्ञ्स्य॑ य॒ज्ञ्स्यै॒ते वै वा ए॒ते य॒ज्ञ्स्य॑ । \newline
39. ए॒ते य॒ज्ञ्स्य॑ य॒ज्ञ्स्यै॒ते ए॒ते य॒ज्ञ्स्य॒ यद् यद् य॒ज्ञ्स्यै॒ते ए॒ते य॒ज्ञ्स्य॒ यत् । \newline
40. ए॒ते इत्ये॒ते । \newline
41. य॒ज्ञ्स्य॒ यद् यद् य॒ज्ञ्स्य॑ य॒ज्ञ्स्य॒ यद॑धि॒षव॑णे अधि॒षव॑णे॒ यद् य॒ज्ञ्स्य॑ य॒ज्ञ्स्य॒ यद॑धि॒षव॑णे । \newline
42. यद॑धि॒षव॑णे अधि॒षव॑णे॒ यद् यद॑धि॒षव॑णे॒ न नाधि॒षव॑णे॒ यद् यद॑धि॒षव॑णे॒ न । \newline
43. अ॒धि॒षव॑णे॒ न नाधि॒षव॑णे अधि॒षव॑णे॒ न सꣳ सम् नाधि॒षव॑णे अधि॒षव॑णे॒ न सम् । \newline
44. अ॒धि॒षव॑णे॒ इत्य॑धि - सव॑ने । \newline
45. न सꣳ सन् न न सम् तृ॑णत्ति तृणत्ति॒ सन् न न सम् तृ॑णत्ति । \newline
46. सम् तृ॑णत्ति तृणत्ति॒ सꣳ सम् तृ॑ण॒त्त्य स॑न्तृण्णे॒ अस॑न्तृण्णे तृणत्ति॒ सꣳ सम् तृ॑ण॒त्त्य स॑न्तृण्णे । \newline
47. तृ॒ण॒ त्त्यस॑न्तृण्णे॒ अस॑न्तृण्णे तृणत्ति तृण॒ त्त्यस॑न्तृण्णे॒ हि ह्यस॑न्तृण्णे तृणत्ति तृण॒
त्त्यस॑न्तृण्णे॒ हि । \newline
48. अस॑न्तृण्णे॒ हि ह्यस॑न्तृण्णे॒ अस॑न्तृण्णे॒ हि हनू॒ हनू॒ ह्यस॑न्तृण्णे॒ अस॑न्तृण्णे॒ हि हनू᳚ । \newline
49. अस॑न्तृण्णे॒ इत्यसं᳚ - तृ॒ण्णे॒ । \newline
50. हि हनू॒ हनू॒ हि हि हनू॒ अथो॒ अथो॒ हनू॒ हि हि हनू॒ अथो᳚ । \newline
51. हनू॒ अथो॒ अथो॒ हनू॒ हनू॒ अथो॒ खलु॒ खल्वथो॒ हनू॒ हनू॒ अथो॒ खलु॑ । \newline
52. हनू॒ इति॒ हनू᳚ । \newline
53. अथो॒ खलु॒ खल्वथो॒ अथो॒ खलु॑ दीर्घसो॒मे दी᳚र्घसो॒मे खल्वथो॒ अथो॒ खलु॑ दीर्घसो॒मे । \newline
54. अथो॒ इत्यथो᳚ । \newline
55. खलु॑ दीर्घसो॒मे दी᳚र्घसो॒मे खलु॒ खलु॑ दीर्घसो॒मे स॒न्तृद्ये॑ स॒न्तृद्ये॑ दीर्घसो॒मे खलु॒ खलु॑ दीर्घसो॒मे स॒न्तृद्ये᳚ । \newline
56. दी॒र्घ॒सो॒मे स॒न्तृद्ये॑ स॒न्तृद्ये॑ दीर्घसो॒मे दी᳚र्घसो॒मे स॒न्तृद्ये॒ धृत्यै॒ धृत्यै॑ स॒न्तृद्ये॑ दीर्घसो॒मे दी᳚र्घसो॒मे स॒न्तृद्ये॒ धृत्यै᳚ । \newline
57. दी॒र्घ॒सो॒म इति॑ दीर्घ - सो॒मे । \newline
58. स॒न्तृद्ये॒ धृत्यै॒ धृत्यै॑ स॒न्तृद्ये॑ स॒न्तृद्ये॒ धृत्यै॒ शिरः॒ शिरो॒ धृत्यै॑ स॒न्तृद्ये॑ स॒न्तृद्ये॒ धृत्यै॒ शिरः॑ । \newline
59. स॒न्तृद्ये॒ इति॑ सं - तृद्ये᳚ । \newline
60. धृत्यै॒ शिरः॒ शिरो॒ धृत्यै॒ धृत्यै॒ शिरो॒ वै वै शिरो॒ धृत्यै॒ धृत्यै॒ शिरो॒ वै । \newline
61. शिरो॒ वै वै शिरः॒ शिरो॒ वा ए॒त दे॒तद् वै शिरः॒ शिरो॒ वा ए॒तत् । \newline
62. वा ए॒त दे॒तद् वै वा ए॒तद् य॒ज्ञ्स्य॑ य॒ज्ञ् स्यै॒तद् वै वा ए॒तद् य॒ज्ञ्स्य॑ । \newline
63. ए॒तद् य॒ज्ञ्स्य॑ य॒ज्ञ् स्यै॒त दे॒तद् य॒ज्ञ्स्य॒ यद् यद् य॒ज्ञ् स्यै॒त दे॒तद् य॒ज्ञ्स्य॒ यत् । \newline
64. य॒ज्ञ्स्य॒ यद् यद् य॒ज्ञ्स्य॑ य॒ज्ञ्स्य॒ यद्ध॑वि॒र्द्धानꣳ॑ हवि॒र्द्धानं॒ ॅयद् य॒ज्ञ्स्य॑ य॒ज्ञ्स्य॒ यद्ध॑वि॒र्द्धान᳚म् । \newline
65. यद्ध॑वि॒र्द्धानꣳ॑ हवि॒र्द्धानं॒ ॅयद् यद्ध॑वि॒र्द्धान॑म् प्रा॒णाः प्रा॒णा ह॑वि॒र्द्धानं॒ ॅयद् यद्ध॑वि॒र्द्धान॑म् प्रा॒णाः । \newline
66. ह॒वि॒र्द्धान॑म् प्रा॒णाः प्रा॒णा ह॑वि॒र्द्धानꣳ॑ हवि॒र्द्धान॑म् प्रा॒णा उ॑पर॒वा उ॑पर॒वाः प्रा॒णा ह॑वि॒र्द्धानꣳ॑ हवि॒र्द्धान॑म् प्रा॒णा उ॑पर॒वाः । \newline
67. ह॒वि॒र्द्धान॒मिति॑ हविः - धान᳚म् । \newline
\pagebreak
\markright{ TS 6.2.11.4  \hfill https://www.vedavms.in \hfill}

\section{ TS 6.2.11.4 }

\textbf{TS 6.2.11.4 } \newline
\textbf{Samhita Paata} \newline

प्रा॒णा उ॑पर॒वा हनू॑ अधि॒षव॑णे जि॒ह्वा चर्म॒ ग्रावा॑णो॒ दन्ता॒ मुख॑माहव॒नीयो॒ नासि॑को-त्तरवे॒दि-रु॒दरꣳ॒॒ सदो॑ य॒दा खलु॒ वै जि॒ह्वया॑ द॒थ्स्वधि॒ खाद॒त्यथ॒ मुखं॑ गच्छति य॒दा मुखं॒ गच्छ॒त्यथो॒दरं॑ गच्छति॒ तस्मा᳚द्धवि॒र्द्धाने॒ चर्म॒न्नधि॒ ग्राव॑भिरभि॒षुत्या॑ऽऽ*हव॒नीये॑ हु॒त्वा प्र॒त्यञ्चः॑ प॒रेत्य॒ सद॑सि भक्षयन्ति॒ यो वै वि॒राजो॑ यज्ञ्मु॒खे दोहं॒ ॅवेद॑ दु॒ह ए॒वै ( ) ना॑मि॒यं ॅवै वि॒राट् तस्यै॒ त्वक् चर्मोधो॑ऽधि॒षव॑णे॒ स्तना॑ उपर॒वा ग्रावा॑णो व॒थ्सा ऋ॒त्विजो॑ दुहन्ति॒ सोमः॒ पयो॒ य ए॒वं ॅवेद॑ दु॒ह ए॒वैनां᳚ ॥ \newline

\textbf{Pada Paata} \newline

प्रा॒णा इति॑ प्र - अ॒नाः । उ॒प॒र॒वा इत्यु॑प - र॒वाः । हनू॒ इति॑ । अ॒धि॒षव॑णे॒ इत्य॑धि - सव॑ने । जि॒ह्वा । चर्म॑ । ग्रावा॑णः । दन्ताः᳚ । मुख᳚म् । आ॒ह॒व॒नीय॒ इत्या᳚ - ह॒व॒नीयः॑ । नासि॑का । उ॒त्त॒र॒वे॒दिरित्यु॑त्तर - वे॒दिः । उ॒दर᳚म् । सदः॑ । य॒दा । खलु॑ । वै । जि॒ह्वया᳚ । द॒थ्स्विति॑ दत् -सु । अधीति॑ । खाद॑ति । अथ॑ । मुख᳚म् । ग॒च्छ॒ति॒ । य॒दा । मुख᳚म् । गच्छ॑ति । अथ॑ । उ॒दर᳚म् । ग॒च्छ॒ति॒ । तस्मा᳚त् । ह॒वि॒द्‌र्धान॒ इति॑ हविः - धाने᳚ । चर्मन्न्॑ । अधीति॑ । ग्राव॑भि॒रिति॒ ग्राव॑ - भिः॒ । अ॒भि॒षुत्येत्य॑भि - सुत्य॑ । आ॒ह॒व॒नीय॒ इत्या᳚ - ह॒व॒नीये᳚ । हु॒त्वा । प्र॒त्यञ्चः॑ । प॒रेत्येति॑ परा - इत्य॑ । सद॑सि । भ॒क्ष॒य॒न्ति॒ । यः । वै । वि॒राज॒ इति॑ वि - राजः॑ । य॒ज्ञ्॒मु॒ख इति॑ यज्ञ् - मु॒खे । दोह᳚म् । वेद॑ । दु॒हे । ए॒व ( ) । ए॒ना॒म् । इ॒यम् । वै । वि॒राडिति॑ वि - राट् । तस्यै᳚ । त्वक् । चर्म॑ । ऊधः॑ । अ॒धि॒षव॑णे॒ इत्य॑धि-सव॑ने । स्तनाः᳚ । उ॒प॒र॒वा इत्यु॑प - र॒वाः । ग्रावा॑णः । व॒थ्साः । ऋ॒त्विजः॑ । दु॒ह॒न्ति॒ । सोमः॑ । पयः॑ । यः । ए॒वम् । वेद॑ । दु॒हे । ए॒व । ए॒ना॒म् ॥  \newline


\textbf{Krama Paata} \newline

प्रा॒णा उ॑पर॒वाः । प्रा॒णा इति॑ प्र - अ॒नाः । उ॒प॒र॒वा हनू᳚ । उ॒प॒र॒वा इत्यु॑प - र॒वाः । हनू॑ अधि॒षव॑णे । हनू॒ इति॒ हनू᳚ । अ॒धि॒षव॑णे जि॒ह्वा । अ॒धि॒षव॑णे॒ इत्य॑धि - सव॑ने । जि॒ह्वा चर्म॑ । चर्म॒ ग्रावा॑णः । ग्रावा॑णो॒ दन्ताः᳚ । दन्ता॒ मुख᳚म् । मुख॑माहव॒नीयः॑ । आ॒ह॒व॒नीयो॒ नासि॑का । आ॒ह॒व॒नीय॒ इत्या᳚ - ह॒व॒नीयः॑ । नासि॑कोत्तरवे॒दिः । उ॒त्त॒र॒वे॒दिरु॒दर᳚म् । उ॒त्त॒र॒वे॒दिरित्यु॑त्तर - वे॒दिः । उ॒दरꣳ॒॒ सदः॑ । सदो॑ य॒दा । य॒दा खलु॑ । खलु॒ वै । वै जि॒ह्वया᳚ । जि॒ह्वया॑ द॒थ्सु । द॒थ्स्वधि॑ । द॒थ्स्विति॑ दत् - सु । अधि॒ खाद॑ति । खाद॒त्यथ॑ । अथ॒ मुख᳚म् । मुख॑म् गच्छति । ग॒च्छ॒ति॒ य॒दा । य॒दा मुख᳚म् । मुख॒म् गच्छ॑ति । गच्छ॒त्यथ॑ । अथो॒दर᳚म् । उ॒दर॑म् गच्छति । ग॒च्छ॒ति॒ तस्मा᳚त् । तस्मा᳚द्‍धवि॒र्द्धाने᳚ । ह॒वि॒र्द्धाने॒ चर्मन्न्॑ । ह॒वि॒र्द्धान॒ इति॑ हविः - धाने᳚ । चर्म॒न्नधि॑ । अधि॒ ग्राव॑भिः । ग्राव॑भिरभि॒षुत्य॑ । ग्राव॑भि॒रिति॒ ग्राव॑ - भिः॒ । अ॒भि॒षुत्या॑हव॒नीये᳚ । अ॒भि॒षु॒त्येत्य॑भि - सुत्य॑ । आ॒ह॒व॒नीये॑ हु॒त्वा । आ॒ह॒व॒नीय॒ इत्या᳚ - ह॒व॒नीये᳚ । हु॒त्वा प्र॒त्यञ्चः॑ । प्र॒त्यञ्चः॑ प॒रेत्य॑ । प॒रेत्य॒ सद॑सि । प॒रेत्येति॑ परा - इत्य॑ । सद॑सि भक्षयन्ति । भ॒क्ष॒य॒न्ति॒ यः । यो वै । वै वि॒राजः॑ । वि॒राजो॑ यज्ञ्मु॒खे । वि॒राज॒ इति॑ वि - राजः॑ । य॒ज्ञ्॒मु॒खे दोह᳚म् । य॒ज्ञ्॒मु॒ख इति॑ यज्ञ् - मु॒खे । दोह॒म् ॅवेद॑ । वेद॑ दु॒हे । दु॒ह ए॒व ( ) । ए॒वैना᳚म् । ए॒ना॒मि॒यम् । इ॒यम् ॅवै । वै वि॒राट् । वि॒राट् तस्यै᳚ । वि॒राडिति॑ वि - राट् । तस्यै॒ त्वक् । त्वक् चर्म॑ । चर्मोधः॑ । ऊधो॑ऽधि॒षव॑णे । अ॒धि॒षव॑णे॒ स्तनाः᳚ । अ॒धि॒षव॑णे॒ इत्य॑धि - सव॑ने । स्तना॑ उपर॒वाः । उ॒प॒र॒वा ग्रावा॑णः । उ॒प॒र॒वा इत्यु॑प - र॒वाः । ग्रावा॑णो व॒थ्साः । व॒थ्सा ऋ॒त्विजः॑ । ऋ॒त्विजो॑ दुहन्ति । दु॒ह॒न्ति॒ सोमः॑ । सोमः॒ पयः॑ । पयो॒ यः । य ए॒वम् । ए॒वम् ॅवेद॑ । वेद॑ दु॒हे । दु॒ह ए॒व । ए॒वैना᳚म् । ए॒ना॒मित्ये॑नाम् । \newline

\textbf{Jatai Paata} \newline

1. प्रा॒णा उ॑पर॒वा उ॑पर॒वाः प्रा॒णाः प्रा॒णा उ॑पर॒वाः । \newline
2. प्रा॒णा इति॑ प्र - अ॒नाः । \newline
3. उ॒प॒र॒वा हनू॒ हनू॑ उपर॒वा उ॑पर॒वा हनू᳚ । \newline
4. उ॒प॒र॒वा इत्यु॑प - र॒वाः । \newline
5. हनू॑ अधि॒षव॑णे अधि॒षव॑णे॒ हनू॒ हनू॑ अधि॒षव॑णे । \newline
6. हनू॒ इति॒ हनू᳚ । \newline
7. अ॒धि॒षव॑णे जि॒ह्वा जि॒ह्वा ऽधि॒षव॑णे अधि॒षव॑णे जि॒ह्वा । \newline
8. अ॒धि॒षव॑णे॒ इत्य॑धि - सव॑ने । \newline
9. जि॒ह्वा चर्म॒ चर्म॑ जि॒ह्वा जि॒ह्वा चर्म॑ । \newline
10. चर्म॒ ग्रावा॑णो॒ ग्रावा॑ण॒ श्चर्म॒ चर्म॒ ग्रावा॑णः । \newline
11. ग्रावा॑णो॒ दन्ता॒ दन्ता॒ ग्रावा॑णो॒ ग्रावा॑णो॒ दन्ताः᳚ । \newline
12. दन्ता॒ मुख॒म् मुख॒म् दन्ता॒ दन्ता॒ मुख᳚म् । \newline
13. मुख॑ माहव॒नीय॑ आहव॒नीयो॒ मुख॒म् मुख॑ माहव॒नीयः॑ । \newline
14. आ॒ह॒व॒नीयो॒ नासि॑का॒ नासि॑का ऽऽहव॒नीय॑ आहव॒नीयो॒ नासि॑का । \newline
15. आ॒ह॒व॒नीय॒ इत्या᳚ - ह॒व॒नीयः॑ । \newline
16. नासि॑ कोत्तरवे॒दि रु॑त्तरवे॒दिर् नासि॑का॒ नासि॑ कोत्तरवे॒दिः । \newline
17. उ॒त्त॒र॒वे॒दि रु॒दर॑ मु॒दर॑ मुत्तरवे॒दि रु॑त्तरवे॒दि रु॒दर᳚म् । \newline
18. उ॒त्त॒र॒वे॒दिरित्यु॑त्तर - वे॒दिः । \newline
19. उ॒दरꣳ॒॒ सदः॒ सद॑ उ॒दर॑ मु॒दरꣳ॒॒ सदः॑ । \newline
20. सदो॑ य॒दा य॒दा सदः॒ सदो॑ य॒दा । \newline
21. य॒दा खलु॒ खलु॑ य॒दा य॒दा खलु॑ । \newline
22. खलु॒ वै वै खलु॒ खलु॒ वै । \newline
23. वै जि॒ह्वया॑ जि॒ह्वया॒ वै वै जि॒ह्वया᳚ । \newline
24. जि॒ह्वया॑ द॒थ्सु द॒थ्सु जि॒ह्वया॑ जि॒ह्वया॑ द॒थ्सु । \newline
25. द॒थ्स्वध्यधि॑ द॒थ्सु द॒थ्स्वधि॑ । \newline
26. द॒थ्स्विति॑ दत् - सु । \newline
27. अधि॒ खाद॑ति॒ खाद॒ त्यध्यधि॒ खाद॑ति । \newline
28. खाद॒ त्यथाथ॒ खाद॑ति॒ खाद॒ त्यथ॑ । \newline
29. अथ॒ मुख॒म् मुख॒ मथाथ॒ मुख᳚म् । \newline
30. मुख॑म् गच्छति गच्छति॒ मुख॒म् मुख॑म् गच्छति । \newline
31. ग॒च्छ॒ति॒ य॒दा य॒दा ग॑च्छति गच्छति य॒दा । \newline
32. य॒दा मुख॒म् मुखं॑ ॅय॒दा य॒दा मुख᳚म् । \newline
33. मुख॒म् गच्छ॑ति॒ गच्छ॑ति॒ मुख॒म् मुख॒म् गच्छ॑ति । \newline
34. गच्छ॒ त्यथाथ॒ गच्छ॑ति॒ गच्छ॒ त्यथ॑ । \newline
35. अथो॒दर॑ मु॒दर॒ मथाथो॒दर᳚म् । \newline
36. उ॒दर॑म् गच्छति गच्छ त्यु॒दर॑ मु॒दर॑म् गच्छति । \newline
37. ग॒च्छ॒ति॒ तस्मा॒त् तस्मा᳚द् गच्छति गच्छति॒ तस्मा᳚त् । \newline
38. तस्मा᳚द्धवि॒र्द्धाने॑ हवि॒र्द्धाने॒ तस्मा॒त् तस्मा᳚द्धवि॒र्द्धाने᳚ । \newline
39. ह॒वि॒र्द्धाने॒ चर्मꣳ॒॒श् चर्म॑न्. हवि॒र्द्धाने॑ हवि॒र्द्धाने॒ चर्मन्न्॑ । \newline
40. ह॒वि॒र्द्धान॒ इति॑ हविः - धाने᳚ । \newline
41. चर्म॒न् नध्यधि॒ चर्मꣳ॒॒श् चर्म॒न् नधि॑ । \newline
42. अधि॒ ग्राव॑भि॒र् ग्राव॑भि॒ रध्यधि॒ ग्राव॑भिः । \newline
43. ग्राव॑भि रभि॒षुत्या॑ भि॒षुत्य॒ ग्राव॑भि॒र् ग्राव॑भि रभि॒षुत्य॑ । \newline
44. ग्राव॑भि॒रिति॒ ग्राव॑ - भिः॒ । \newline
45. अ॒भि॒षुत्या॑ हव॒नीय॑ आहव॒नीये॑ ऽभि॒षुत्या॑ भि॒षुत्या॑ हव॒नीये᳚ । \newline
46. अ॒भि॒षुत्येत्य॑भि - सुत्य॑ । \newline
47. आ॒ह॒व॒नीये॑ हु॒त्वा हु॒त्वा ऽऽह॑व॒नीय॑ आहव॒नीये॑ हु॒त्वा । \newline
48. आ॒ह॒व॒नीय॒ इत्या᳚ - ह॒व॒नीये᳚ । \newline
49. हु॒त्वा प्र॒त्यञ्चः॑ प्र॒त्यञ्चो॑ हु॒त्वा हु॒त्वा प्र॒त्यञ्चः॑ । \newline
50. प्र॒त्यञ्चः॑ प॒रेत्य॑ प॒रेत्य॑ प्र॒त्यञ्चः॑ प्र॒त्यञ्चः॑ प॒रेत्य॑ । \newline
51. प॒रेत्य॒ सद॑सि॒ सद॑सि प॒रेत्य॑ प॒रेत्य॒ सद॑सि । \newline
52. प॒रेत्येति॑ परा - इत्य॑ । \newline
53. सद॑सि भक्षयन्ति भक्षयन्ति॒ सद॑सि॒ सद॑सि भक्षयन्ति । \newline
54. भ॒क्ष॒य॒न्ति॒ यो यो भ॑क्षयन्ति भक्षयन्ति॒ यः । \newline
55. यो वै वै यो यो वै । \newline
56. वै वि॒राजो॑ वि॒राजो॒ वै वै वि॒राजः॑ । \newline
57. वि॒राजो॑ यज्ञ्मु॒खे य॑ज्ञ्मु॒खे वि॒राजो॑ वि॒राजो॑ यज्ञ्मु॒खे । \newline
58. वि॒राज॒ इति॑ वि - राजः॑ । \newline
59. य॒ज्ञ्॒मु॒खे दोह॒म् दोहं॑ ॅयज्ञ्मु॒खे य॑ज्ञ्मु॒खे दोह᳚म् । \newline
60. य॒ज्ञ्॒मु॒ख इति॑ यज्ञ् - मु॒खे । \newline
61. दोहं॒ ॅवेद॒ वेद॒ दोह॒म् दोहं॒ ॅवेद॑ । \newline
62. वेद॑ दु॒हे दु॒हे वेद॒ वेद॑ दु॒हे । \newline
63. दु॒ह ए॒वैव दु॒हे दु॒ह ए॒व । \newline
64. ए॒वैना॑ मेना मे॒वै वैना᳚म् । \newline
65. ए॒ना॒ मि॒य मि॒य मे॑ना मेना मि॒यम् । \newline
66. इ॒यं ॅवै वा इ॒य मि॒यं ॅवै । \newline
67. वै वि॒राड् वि॒राड् वै वै वि॒राट् । \newline
68. वि॒राट् तस्यै॒ तस्यै॑ वि॒राड् वि॒राट् तस्यै᳚ । \newline
69. वि॒राडिति॑ वि - राट् । \newline
70. तस्यै॒ त्वक् त्वक् तस्यै॒ तस्यै॒ त्वक् । \newline
71. त्वक् चर्म॒ चर्म॒ त्वक् त्वक् चर्म॑ । \newline
72. चर्मोध॒ ऊध॒ श्चर्म॒ चर्मोधः॑ । \newline
73. ऊधो॑ ऽधि॒षव॑णे अधि॒षव॑णे॒ ऊध॒ ऊधो॑ ऽधि॒षव॑णे । \newline
74. अ॒धि॒षव॑णे॒ स्तनाः॒ स्तना॑ अधि॒षव॑णे अधि॒षव॑णे॒ स्तनाः᳚ । \newline
75. अ॒धि॒षव॑णे॒ इत्य॑धि - सव॑ने । \newline
76. स्तना॑ उपर॒वा उ॑पर॒वाः स्तनाः॒ स्तना॑ उपर॒वाः । \newline
77. उ॒प॒र॒वा ग्रावा॑णो॒ ग्रावा॑ण उपर॒वा उ॑पर॒वा ग्रावा॑णः । \newline
78. उ॒प॒र॒वा इत्यु॑प - र॒वाः । \newline
79. ग्रावा॑णो व॒थ्सा व॒थ्सा ग्रावा॑णो॒ ग्रावा॑णो व॒थ्साः । \newline
80. व॒थ्सा ऋ॒त्विज॑ ऋ॒त्विजो॑ व॒थ्सा व॒थ्सा ऋ॒त्विजः॑ । \newline
81. ऋ॒त्विजो॑ दुहन्ति दुह न्त्यृ॒त्विज॑ ऋ॒त्विजो॑ दुहन्ति । \newline
82. दु॒ह॒न्ति॒ सोमः॒ सोमो॑ दुहन्ति दुहन्ति॒ सोमः॑ । \newline
83. सोमः॒ पयः॒ पयः॒ सोमः॒ सोमः॒ पयः॑ । \newline
84. पयो॒ यो यः पयः॒ पयो॒ यः । \newline
85. य ए॒व मे॒वं ॅयो य ए॒वम् । \newline
86. ए॒वं ॅवेद॒ वेदै॒व मे॒वं ॅवेद॑ । \newline
87. वेद॑ दु॒हे दु॒हे वेद॒ वेद॑ दु॒हे । \newline
88. दु॒ह ए॒वैव दु॒हे दु॒ह ए॒व । \newline
89. ए॒वैना॑ मेना मे॒वैवैना᳚म् । \newline
90. ए॒ना॒मित्ये॑नाम् । \newline

\textbf{Ghana Paata } \newline

1. प्रा॒णा उ॑पर॒वा उ॑पर॒वाः प्रा॒णाः प्रा॒णा उ॑पर॒वा हनू॒ हनू॑ उपर॒वाः प्रा॒णाः प्रा॒णा उ॑पर॒वा हनू᳚ । \newline
2. प्रा॒णा इति॑ प्र - अ॒नाः । \newline
3. उ॒प॒र॒वा हनू॒ हनू॑ उपर॒वा उ॑पर॒वा हनू॑ अधि॒षव॑णे अधि॒षव॑णे॒ हनू॑ उपर॒वा उ॑पर॒वा हनू॑ अधि॒षव॑णे । \newline
4. उ॒प॒र॒वा इत्यु॑प - र॒वाः । \newline
5. हनू॑ अधि॒षव॑णे अधि॒षव॑णे॒ हनू॒ हनू॑ अधि॒षव॑णे जि॒ह्वा जि॒ह्वा ऽधि॒षव॑णे॒ हनू॒ हनू॑ अधि॒षव॑णे जि॒ह्वा । \newline
6. हनू॒ इति॒ हनू᳚ । \newline
7. अ॒धि॒षव॑णे जि॒ह्वा जि॒ह्वा ऽधि॒षव॑णे अधि॒षव॑णे जि॒ह्वा चर्म॒ चर्म॑ जि॒ह्वा ऽधि॒षव॑णे अधि॒षव॑णे जि॒ह्वा चर्म॑ । \newline
8. अ॒धि॒षव॑णे॒ इत्य॑धि - सव॑ने । \newline
9. जि॒ह्वा चर्म॒ चर्म॑ जि॒ह्वा जि॒ह्वा चर्म॒ ग्रावा॑णो॒ ग्रावा॑ण॒ श्चर्म॑ जि॒ह्वा जि॒ह्वा चर्म॒ ग्रावा॑णः । \newline
10. चर्म॒ ग्रावा॑णो॒ ग्रावा॑ण॒ श्चर्म॒ चर्म॒ ग्रावा॑णो॒ दन्ता॒ दन्ता॒ ग्रावा॑ण॒ श्चर्म॒ चर्म॒ ग्रावा॑णो॒ दन्ताः᳚ । \newline
11. ग्रावा॑णो॒ दन्ता॒ दन्ता॒ ग्रावा॑णो॒ ग्रावा॑णो॒ दन्ता॒ मुख॒म् मुख॒म् दन्ता॒ ग्रावा॑णो॒ ग्रावा॑णो॒ दन्ता॒ मुख᳚म् । \newline
12. दन्ता॒ मुख॒म् मुख॒म् दन्ता॒ दन्ता॒ मुख॑ माहव॒नीय॑ आहव॒नीयो॒ मुख॒म् दन्ता॒ दन्ता॒ मुख॑ माहव॒नीयः॑ । \newline
13. मुख॑ माहव॒नीय॑ आहव॒नीयो॒ मुख॒म् मुख॑ माहव॒नीयो॒ नासि॑का॒ नासि॑का ऽऽहव॒नीयो॒ मुख॒म् मुख॑ माहव॒नीयो॒ नासि॑का । \newline
14. आ॒ह॒व॒नीयो॒ नासि॑का॒ नासि॑का ऽऽहव॒नीय॑ आहव॒नीयो॒ नासि॑को त्तरवे॒दि रु॑त्तरवे॒दिर् नासि॑का ऽऽहव॒नीय॑ आहव॒नीयो॒ नासि॑को त्तरवे॒दिः । \newline
15. आ॒ह॒व॒नीय॒ इत्या᳚ - ह॒व॒नीयः॑ । \newline
16. नासि॑को त्तरवे॒दि रु॑त्तरवे॒दिर् नासि॑का॒ नासि॑को त्तरवे॒दि रु॒दर॑ मु॒दर॑ मुत्तरवे॒दिर् नासि॑का॒ नासि॑को त्तरवे॒दि रु॒दर᳚म् । \newline
17. उ॒त्त॒र॒वे॒दि रु॒दर॑ मु॒दर॑ मुत्तरवे॒दि रु॑त्तरवे॒दि रु॒दरꣳ॒॒ सदः॒ सद॑ उ॒दर॑ मुत्तरवे॒दि रु॑त्तरवे॒दि रु॒दरꣳ॒॒ सदः॑ । \newline
18. उ॒त्त॒र॒वे॒दिरित्यु॑त्तर - वे॒दिः । \newline
19. उ॒दरꣳ॒॒ सदः॒ सद॑ उ॒दर॑ मु॒दरꣳ॒॒ सदो॑ य॒दा य॒दा सद॑ उ॒दर॑ मु॒दरꣳ॒॒ सदो॑ य॒दा । \newline
20. सदो॑ य॒दा य॒दा सदः॒ सदो॑ य॒दा खलु॒ खलु॑ य॒दा सदः॒ सदो॑ य॒दा खलु॑ । \newline
21. य॒दा खलु॒ खलु॑ य॒दा य॒दा खलु॒ वै वै खलु॑ य॒दा य॒दा खलु॒ वै । \newline
22. खलु॒ वै वै खलु॒ खलु॒ वै जि॒ह्वया॑ जि॒ह्वया॒ वै खलु॒ खलु॒ वै जि॒ह्वया᳚ । \newline
23. वै जि॒ह्वया॑ जि॒ह्वया॒ वै वै जि॒ह्वया॑ द॒थ्सु द॒थ्सु जि॒ह्वया॒ वै वै जि॒ह्वया॑ द॒थ्सु । \newline
24. जि॒ह्वया॑ द॒थ्सु द॒थ्सु जि॒ह्वया॑ जि॒ह्वया॑ द॒थ्स्व ध्यधि॑ द॒थ्सु जि॒ह्वया॑ जि॒ह्वया॑ द॒थ्स्वधि॑ । \newline
25. द॒थ्स्व ध्यधि॑ द॒थ्सु द॒थ्स्वधि॒ खाद॑ति॒ खाद॒त्यधि॑ द॒थ्सु द॒थ्स्वधि॒ खाद॑ति । \newline
26. द॒थ्स्विति॑ दत् - सु । \newline
27. अधि॒ खाद॑ति॒ खाद॒ त्यध्यधि॒ खाद॒ त्यथाथ॒ खाद॒ त्यध्यधि॒ खाद॒ त्यथ॑ । \newline
28. खाद॒त्य थाथ॒ खाद॑ति॒ खाद॒ त्यथ॒ मुख॒म् मुख॒ मथ॒ खाद॑ति॒ खाद॒ त्यथ॒ मुख᳚म् । \newline
29. अथ॒ मुख॒म् मुख॒ मथाथ॒ मुख॑म् गच्छति गच्छति॒ मुख॒ मथाथ॒ मुख॑म् गच्छति । \newline
30. मुख॑म् गच्छति गच्छति॒ मुख॒म् मुख॑म् गच्छति य॒दा य॒दा ग॑च्छति॒ मुख॒म् मुख॑म् गच्छति य॒दा । \newline
31. ग॒च्छ॒ति॒ य॒दा य॒दा ग॑च्छति गच्छति य॒दा मुख॒म् मुखं॑ ॅय॒दा ग॑च्छति गच्छति य॒दा मुख᳚म् । \newline
32. य॒दा मुख॒म् मुखं॑ ॅय॒दा य॒दा मुख॒म् गच्छ॑ति॒ गच्छ॑ति॒ मुखं॑ ॅय॒दा य॒दा मुख॒म् गच्छ॑ति । \newline
33. मुख॒म् गच्छ॑ति॒ गच्छ॑ति॒ मुख॒म् मुख॒म् गच्छ॒ त्यथाथ॒ गच्छ॑ति॒ मुख॒म् मुख॒म् गच्छ॒ त्यथ॑ । \newline
34. गच्छ॒ त्यथाथ॒ गच्छ॑ति॒ गच्छ॒ त्यथो॒दर॑ मु॒दर॒ मथ॒ गच्छ॑ति॒ गच्छ॒ त्यथो॒दर᳚म् । \newline
35. अथो॒दर॑ मु॒दर॒ मथा थो॒दर॑म् गच्छति गच्छ त्यु॒दर॒ मथा थो॒दर॑म् गच्छति । \newline
36. उ॒दर॑म् गच्छति गच्छ त्यु॒दर॑ मु॒दर॑म् गच्छति॒ तस्मा॒त् तस्मा᳚द् गच्छ त्यु॒दर॑ मु॒दर॑म् गच्छति॒ तस्मा᳚त् । \newline
37. ग॒च्छ॒ति॒ तस्मा॒त् तस्मा᳚द् गच्छति गच्छति॒ तस्मा᳚ द्धवि॒र्द्धाने॑ हवि॒र्द्धाने॒ तस्मा᳚द् गच्छति गच्छति॒ तस्मा᳚
द्धवि॒र्द्धाने᳚ । \newline
38. तस्मा᳚ द्धवि॒र्द्धाने॑ हवि॒र्द्धाने॒ तस्मा॒त् तस्मा᳚ द्धवि॒र्द्धाने॒ चर्मꣳ॒॒ श्चर्म॑न्. हवि॒र्द्धाने॒ तस्मा॒त् तस्मा᳚ द्धवि॒र्द्धाने॒ चर्मन्न्॑ । \newline
39. ह॒वि॒र्द्धाने॒ चर्मꣳ॒॒ श्चर्म॑न्. हवि॒र्द्धाने॑ हवि॒र्द्धाने॒ चर्म॒न् नध्यधि॒ चर्म॑न्. हवि॒र्द्धाने॑ हवि॒र्द्धाने॒ चर्म॒न् नधि॑ । \newline
40. ह॒वि॒र्द्धान॒ इति॑ हविः - धाने᳚ । \newline
41. चर्म॒न् नध्यधि॒ चर्मꣳ॒॒ श्चर्म॒न् नधि॒ ग्राव॑भि॒र् ग्राव॑भि॒ रधि॒ चर्मꣳ॒॒ श्चर्म॒न् नधि॒ ग्राव॑भिः । \newline
42. अधि॒ ग्राव॑भि॒र् ग्राव॑भि॒ रध्यधि॒ ग्राव॑भि रभि॒षुत्या॑ भि॒षुत्य॒ ग्राव॑भि॒ रध्यधि॒ ग्राव॑भि रभि॒षुत्य॑ । \newline
43. ग्राव॑भि रभि॒षुत्या॑ भि॒षुत्य॒ ग्राव॑भि॒र् ग्राव॑भि रभि॒षुत्या॑ हव॒नीय॑ आहव॒नीये॑ ऽभि॒षुत्य॒ ग्राव॑भि॒र् ग्राव॑भि रभि॒षुत्या॑ हव॒नीये᳚ । \newline
44. ग्राव॑भि॒रिति॒ ग्राव॑ - भिः॒ । \newline
45. अ॒भि॒षुत्या॑ हव॒नीय॑ आहव॒नीये॑ ऽभि॒षुत्या॑ भि॒षुत्या॑ हव॒नीये॑ हु॒त्वा हु॒त्वा ऽऽह॑व॒नीये॑ ऽभि॒षुत्या॑ भि॒षुत्या॑ हव॒नीये॑ हु॒त्वा । \newline
46. अ॒भि॒षुत्येत्य॑भि - सुत्य॑ । \newline
47. आ॒ह॒व॒नीये॑ हु॒त्वा हु॒त्वा ऽऽह॑व॒नीय॑ आहव॒नीये॑ हु॒त्वा प्र॒त्यञ्चः॑ प्र॒त्यञ्चो॑ हु॒त्वा ऽऽह॑व॒नीय॑ आहव॒नीये॑ हु॒त्वा प्र॒त्यञ्चः॑ । \newline
48. आ॒ह॒व॒नीय॒ इत्या᳚ - ह॒व॒नीये᳚ । \newline
49. हु॒त्वा प्र॒त्यञ्चः॑ प्र॒त्यञ्चो॑ हु॒त्वा हु॒त्वा प्र॒त्यञ्चः॑ प॒रेत्य॑ प॒रेत्य॑ प्र॒त्यञ्चो॑ हु॒त्वा हु॒त्वा प्र॒त्यञ्चः॑ प॒रेत्य॑ । \newline
50. प्र॒त्यञ्चः॑ प॒रेत्य॑ प॒रेत्य॑ प्र॒त्यञ्चः॑ प्र॒त्यञ्चः॑ प॒रेत्य॒ सद॑सि॒ सद॑सि प॒रेत्य॑ प्र॒त्यञ्चः॑ प्र॒त्यञ्चः॑ प॒रेत्य॒ सद॑सि । \newline
51. प॒रेत्य॒ सद॑सि॒ सद॑सि प॒रेत्य॑ प॒रेत्य॒ सद॑सि भक्षयन्ति भक्षयन्ति॒ सद॑सि प॒रेत्य॑ प॒रेत्य॒ सद॑सि भक्षयन्ति । \newline
52. प॒रेत्येति॑ परा - इत्य॑ । \newline
53. सद॑सि भक्षयन्ति भक्षयन्ति॒ सद॑सि॒ सद॑सि भक्षयन्ति॒ यो यो भ॑क्षयन्ति॒ सद॑सि॒ सद॑सि भक्षयन्ति॒ यः । \newline
54. भ॒क्ष॒य॒न्ति॒ यो यो भ॑क्षयन्ति भक्षयन्ति॒ यो वै वै यो भ॑क्षयन्ति भक्षयन्ति॒ यो वै । \newline
55. यो वै वै यो यो वै वि॒राजो॑ वि॒राजो॒ वै यो यो वै वि॒राजः॑ । \newline
56. वै वि॒राजो॑ वि॒राजो॒ वै वै वि॒राजो॑ यज्ञ्मु॒खे य॑ज्ञ्मु॒खे वि॒राजो॒ वै वै वि॒राजो॑ यज्ञ्मु॒खे । \newline
57. वि॒राजो॑ यज्ञ्मु॒खे य॑ज्ञ्मु॒खे वि॒राजो॑ वि॒राजो॑ यज्ञ्मु॒खे दोह॒म् दोहं॑ ॅयज्ञ्मु॒खे वि॒राजो॑ वि॒राजो॑ यज्ञ्मु॒खे दोह᳚म् । \newline
58. वि॒राज॒ इति॑ वि - राजः॑ । \newline
59. य॒ज्ञ्॒मु॒खे दोह॒म् दोहं॑ ॅयज्ञ्मु॒खे य॑ज्ञ्मु॒खे दोहं॒ ॅवेद॒ वेद॒ दोहं॑ ॅयज्ञ्मु॒खे य॑ज्ञ्मु॒खे दोहं॒ ॅवेद॑ । \newline
60. य॒ज्ञ्॒मु॒ख इति॑ यज्ञ् - मु॒खे । \newline
61. दोहं॒ ॅवेद॒ वेद॒ दोह॒म् दोहं॒ ॅवेद॑ दु॒हे दु॒हे वेद॒ दोह॒म् दोहं॒ ॅवेद॑ दु॒हे । \newline
62. वेद॑ दु॒हे दु॒हे वेद॒ वेद॑ दु॒ह ए॒वैव दु॒हे वेद॒ वेद॑ दु॒ह ए॒व । \newline
63. दु॒ह ए॒वैव दु॒हे दु॒ह ए॒वैना॑ मेना मे॒व दु॒हे दु॒ह ए॒वैना᳚म् । \newline
64. ए॒वैना॑ मेना मे॒वै वैना॑ मि॒य मि॒य मे॑ना मे॒वै वैना॑ मि॒यम् । \newline
65. ए॒ना॒ मि॒य मि॒य मे॑ना मेना मि॒यं ॅवै वा इ॒य मे॑ना मेना मि॒यं ॅवै । \newline
66. इ॒यं ॅवै वा इ॒य मि॒यं ॅवै वि॒राड् वि॒राड् वा इ॒य मि॒यं ॅवै वि॒राट् । \newline
67. वै वि॒राड् वि॒राड् वै वै वि॒राट् तस्यै॒ तस्यै॑ वि॒राड् वै वै वि॒राट् तस्यै᳚ । \newline
68. वि॒राट् तस्यै॒ तस्यै॑ वि॒राड् वि॒राट् तस्यै॒ त्वक् त्वक् तस्यै॑ वि॒राड् वि॒राट् तस्यै॒ त्वक् । \newline
69. वि॒राडिति॑ वि - राट् । \newline
70. तस्यै॒ त्वक् त्वक् तस्यै॒ तस्यै॒ त्वक् चर्म॒ चर्म॒ त्वक् तस्यै॒ तस्यै॒ त्वक् चर्म॑ । \newline
71. त्वक् चर्म॒ चर्म॒ त्वक् त्वक् चर्मोध॒ ऊध॒ श्चर्म॒ त्वक् त्वक् चर्मोधः॑ । \newline
72. चर्मोध॒ ऊध॒ श्चर्म॒ चर्मोधो॑ ऽधि॒षव॑णे अधि॒षव॑णे॒ ऊध॒ श्चर्म॒ चर्मोधो॑ ऽधि॒षव॑णे । \newline
73. ऊधो॑ ऽधि॒षव॑णे अधि॒षव॑णे॒ ऊध॒ ऊधो॑ ऽधि॒षव॑णे॒ स्तनाः॒ स्तना॑ अधि॒षव॑णे॒ ऊध॒ ऊधो॑ ऽधि॒षव॑णे॒ स्तनाः᳚ । \newline
74. अ॒धि॒षव॑णे॒ स्तनाः॒ स्तना॑ अधि॒षव॑णे अधि॒षव॑णे॒ स्तना॑ उपर॒वा उ॑पर॒वाः स्तना॑ अधि॒षव॑णे अधि॒षव॑णे॒ स्तना॑ उपर॒वाः । \newline
75. अ॒धि॒षव॑णे॒ इत्य॑धि - सव॑ने । \newline
76. स्तना॑ उपर॒वा उ॑पर॒वाः स्तनाः॒ स्तना॑ उपर॒वा ग्रावा॑णो॒ ग्रावा॑ण उपर॒वाः स्तनाः॒ स्तना॑ उपर॒वा ग्रावा॑णः । \newline
77. उ॒प॒र॒वा ग्रावा॑णो॒ ग्रावा॑ण उपर॒वा उ॑पर॒वा ग्रावा॑णो व॒थ्सा व॒थ्सा ग्रावा॑ण उपर॒वा उ॑पर॒वा ग्रावा॑णो व॒थ्साः । \newline
78. उ॒प॒र॒वा इत्यु॑प - र॒वाः । \newline
79. ग्रावा॑णो व॒थ्सा व॒थ्सा ग्रावा॑णो॒ ग्रावा॑णो व॒थ्सा ऋ॒त्विज॑ ऋ॒त्विजो॑ व॒थ्सा ग्रावा॑णो॒ ग्रावा॑णो व॒थ्सा ऋ॒त्विजः॑ । \newline
80. व॒थ्सा ऋ॒त्विज॑ ऋ॒त्विजो॑ व॒थ्सा व॒थ्सा ऋ॒त्विजो॑ दुहन्ति दुहन् त्यृ॒त्विजो॑ व॒थ्सा व॒थ्सा ऋ॒त्विजो॑ दुहन्ति । \newline
81. ऋ॒त्विजो॑ दुहन्ति दुहन् त्यृ॒त्विज॑ ऋ॒त्विजो॑ दुहन्ति॒ सोमः॒ सोमो॑ दुहन् त्यृ॒त्विज॑ ऋ॒त्विजो॑ दुहन्ति॒ सोमः॑ । \newline
82. दु॒ह॒न्ति॒ सोमः॒ सोमो॑ दुहन्ति दुहन्ति॒ सोमः॒ पयः॒ पयः॒ सोमो॑ दुहन्ति दुहन्ति॒ सोमः॒ पयः॑ । \newline
83. सोमः॒ पयः॒ पयः॒ सोमः॒ सोमः॒ पयो॒ यो यः पयः॒ सोमः॒ सोमः॒ पयो॒ यः । \newline
84. पयो॒ यो यः पयः॒ पयो॒ य ए॒व मे॒वं ॅयः पयः॒ पयो॒ य ए॒वम् । \newline
85. य ए॒व मे॒वं ॅयो य ए॒वं ॅवेद॒ वेदै॒वं ॅयो य ए॒वं ॅवेद॑ । \newline
86. ए॒वं ॅवेद॒ वेदै॒व मे॒वं ॅवेद॑ दु॒हे दु॒हे वेदै॒व मे॒वं ॅवेद॑ दु॒हे । \newline
87. वेद॑ दु॒हे दु॒हे वेद॒ वेद॑ दु॒ह ए॒वैव दु॒हे वेद॒ वेद॑ दु॒ह ए॒व । \newline
88. दु॒ह ए॒वैव दु॒हे दु॒ह ए॒वैना॑ मेना मे॒व दु॒हे दु॒ह ए॒वैना᳚म् । \newline
89. ए॒वैना॑ मेना मे॒वै वैना᳚म् । \newline
90. ए॒ना॒मित्ये॑नाम् । \newline
\pagebreak


\end{document}