\documentclass[17pt]{extarticle}
\usepackage{babel}
\usepackage{fontspec}
\usepackage{polyglossia}
\usepackage{extsizes}

\usepackage{color}   %May be necessary if you want to color links
\usepackage{hyperref}
\hypersetup{
    colorlinks=true, %set true if you want colored links
    linktoc=all,     %set to all if you want both sections and subsections linked
    linkcolor=black,  %choose some color if you want links to stand out
}

\setmainlanguage{sanskrit}
\setotherlanguages{english} %% or other languages
\setlength{\parindent}{0pt}
\pagestyle{myheadings}
\newfontfamily\devanagarifont[Script=Devanagari]{AdishilaVedic}
\renewcommand{\theHsection}{\thepart.section.\thesection}

\newcommand{\VAR}[1]{}
\newcommand{\BLOCK}[1]{}




\begin{document}
\begin{titlepage}
    \begin{center}
 
\begin{sanskrit}
    { \Large
    कृष्ण यजुर्वेदीय तैत्तिरीय संहिता,पद,जटा,घन पाठः 
    }
    \\
    \vspace{2.5cm}
    \mbox{ \Large
    2.3     द्वितीयकाण्डे तृतीयः प्रश्नः - इष्टिविधानं   }
\end{sanskrit}
\end{center}

\end{titlepage}
\tableofcontents
\phantomsection
\pagebreak

\markright{ TS 2.3.1.1  \hfill https://www.vedavms.in \hfill}

\section{ TS 2.3.1.1 }

\textbf{TS 2.3.1.1 } \newline
\textbf{Samhita Paata} \newline

आ॒दि॒त्येभ्यो॒ भुव॑द्वद्भ्यश्च॒रुं निर्व॑पे॒द्भूति॑काम आदि॒त्या वा ए॒तं भूत्यै॒ प्रति॑ नुदन्ते॒ योऽलं॒ भूत्यै॒ सन् भूतिं॒ न प्रा॒प्नोत्या॑दि॒त्याने॒व भुव॑द्वतः॒ स्वेन॑ भाग॒धेये॒नोप॑ धावति॒ त ए॒वैनं॒ भूतिं॑ गमयन्ति॒ भव॑त्ये॒वा ऽऽदि॒त्येभ्यो॑ धा॒रय॑द्वद्-भ्यश्च॒रुं निर्व॑पे॒दप॑रुद्धो वाऽपरु॒द्ध्यमा॑नो वाऽऽदि॒त्या वा अ॑परो॒द्धार॑ आदि॒त्या अ॑वगमयि॒तार॑ आदि॒त्याने॒व धा॒रय॑द्वतः॒ - [  ] \newline

\textbf{Pada Paata} \newline

आ॒दि॒त्येभ्यः॑ । भुव॑द्वद्भ्य॒ इति॒ भुव॑द्वत् - भ्यः॒ । च॒रुम् । निरिति॑ । व॒पे॒त् । भूति॑काम॒ इति॒ भूति॑ - का॒मः॒ । आ॒दि॒त्याः । वै । ए॒तम् । भूत्यै᳚ । प्रतीति॑ । नु॒द॒न्ते॒ । यः । अल᳚म् । भूत्यै᳚ । सन्न् । भूति᳚म् । न । प्रा॒प्नोतीति॑ प्र - आ॒प्नोति॑ । आ॒दि॒त्यान् । ए॒व । भुव॑द्वत॒ इति॒ भुव॑त् - व॒तः॒ । स्वेन॑ । भा॒ग॒धेये॒नेति॑ भाग - धेये॑न । उपेति॑ । धा॒व॒ति॒ । ते । ए॒व । ए॒न॒म् । भूति᳚म् । ग॒म॒य॒न्ति॒ । भव॑ति । ए॒व । आ॒दि॒त्येभ्यः॑ । धा॒रय॑द्वद्भ्य॒ इति॑ धा॒रय॑द्वत् - भ्यः॒ । च॒रुम् । निरिति॑ । व॒पे॒त् । अप॑रुद्ध॒ इत्यप॑ - रु॒द्धः॒ । वा॒ । अ॒प॒रु॒द्ध्यमा॑न॒ इत्य॑प - रु॒ध्यमा॑नः । वा॒ । आ॒दि॒त्याः । वै । अ॒प॒रो॒द्धार॒ इत्य॑प - रो॒द्धारः॑ । आ॒दि॒त्याः । अ॒व॒ग॒म॒यि॒तार॒ इत्य॑व - ग॒म॒यि॒तारः॑ । आ॒दि॒त्यान् । ए॒व । धा॒रय॑द्वत॒ इति॑ धा॒रय॑त् - व॒तः॒ ।  \newline


\textbf{Krama Paata} \newline

आ॒दि॒त्येभ्यो॒ भुव॑द्वद्भ्यः । भुव॑द्वद्भ्य श्च॒रुम् । भुव॑द्वद्भ्य॒ 
इति॒ भुव॑द्वत् - भ्यः॒ । च॒रुम् निः । निर् व॑पेत् । व॒पे॒द् भूति॑कामः । भूति॑काम आदि॒त्याः । भूति॑काम॒ इति॒ भूति॑ - का॒मः॒ । आ॒दि॒त्या वै । वा ए॒तम् । ए॒तम् भूत्यै᳚ । भूत्यै॒ प्रति॑ । प्रति॑ नुदन्ते । नु॒द॒न्ते॒ यः । योऽल᳚म् । अल॒म् भूत्यै᳚ । भूत्यै॒ सन्न् । सन् भूति᳚म् । भूति॒म् न । न प्रा॒प्नोति॑ । प्रा॒प्नोत्या॑दि॒त्यान् । प्रा॒प्नोतीति॑ प्र - आ॒प्नोति॑ । आ॒दि॒त्याने॒व । ए॒व भुव॑द्वतः । भुव॑द्वतः॒ स्वेन॑ । भुव॑द्वत॒ 
इति॒ भुव॑त् - व॒तः॒ । स्वेन॑ भाग॒धेये॑न । भा॒ग॒धेये॒नोप॑ । भा॒ग॒धेये॒नेति॑ भाग - धेये॑न । उप॑ धावति । धा॒व॒ति॒ ते । 
त ए॒व । ए॒वैन᳚म् । ए॒न॒म् भूति᳚म् । भूति॑म् गमयन्ति । ग॒म॒य॒न्ति॒ भव॑ति । भव॑त्ये॒व । ए॒वादि॒त्येभ्यः॑ । आ॒दि॒त्येभ्यो॑ धा॒रय॑द्वद्भ्यः । धा॒रय॑द्वद्भ्य श्च॒रुम् । धा॒रय॑द्वद्भ्य॒ इति॑ धा॒रय॑द्वत् - भ्यः॒ । च॒रुम् निः । निर् व॑पेत् । व॒पे॒दप॑रुद्धः । अप॑रुद्धो वा । अप॑रुद्ध॒ इत्यप॑ - रु॒द्धः॒ । 
वा॒ ऽप॒रु॒द्ध्यमा॑नः । अ॒प॒रु॒द्ध्यमा॑नो वा । अ॒प॒रु॒द्ध्यमा॑न॒ इत्य॑प - रु॒द्ध्यमा॑नः । वा॒ ऽऽदि॒त्याः । 
आ॒दि॒त्या वै । वा अ॑परो॒द्धारः॑ । अ॒प॒रो॒द्धार॑ आदि॒त्याः । अ॒प॒रो॒द्धार॒ इत्य॑प - रो॒द्धारः॑ । आ॒दि॒त्या अ॑वगमयि॒तारः॑ । अ॒व॒ग॒म॒यि॒तार॑ आदि॒त्यान् । अ॒व॒ग॒म॒यि॒तार॒ इत्य॑व - ग॒म॒यि॒तारः॑ । आ॒दि॒त्याने॒व । ए॒व धा॒रय॑द्वतः । धा॒रय॑द्वतः॒ स्वेन॑ । धा॒रय॑द्वत॒ इति॑ धा॒रय॑त् - व॒तः॒ \newline

\textbf{Jatai Paata} \newline

1. आ॒दि॒त्येभ्यो॒ भुव॑द्वद्भ्यो॒ भुव॑द्वद्भ्य आदि॒त्येभ्य॑ आदि॒त्येभ्यो॒ भुव॑द्वद्भ्यः । \newline
2. भुव॑द्वद्भ्य श्च॒रुम् च॒रुम् भुव॑द्वद्भ्यो॒ भुव॑द्वद्भ्य श्च॒रुम् । \newline
3. भुव॑द्वद्भ्य॒ इति॒ भुव॑द्वत् - भ्यः॒ । \newline
4. च॒रुं निर् णिश्च॒रुम् च॒रुं निः । \newline
5. निर् व॑पेद् वपे॒न् निर् णिर् व॑पेत् । \newline
6. व॒पे॒द् भूति॑कामो॒ भूति॑कामो वपेद् वपे॒द् भूति॑कामः । \newline
7. भूति॑काम आदि॒त्या आ॑दि॒त्या भूति॑कामो॒ भूति॑काम आदि॒त्याः । \newline
8. भूति॑काम॒ इति॒ भूति॑ - का॒मः॒ । \newline
9. आ॒दि॒त्या वै वा आ॑दि॒त्या आ॑दि॒त्या वै । \newline
10. वा ए॒त मे॒तं ॅवै वा ए॒तम् । \newline
11. ए॒तम् भूत्यै॒ भूत्या॑ ए॒त मे॒तम् भूत्यै᳚ । \newline
12. भूत्यै॒ प्रति॒ प्रति॒ भूत्यै॒ भूत्यै॒ प्रति॑ । \newline
13. प्रति॑ नुदन्ते नुदन्ते॒ प्रति॒ प्रति॑ नुदन्ते । \newline
14. नु॒द॒न्ते॒ यो यो नु॑दन्ते नुदन्ते॒ यः । \newline
15. यो ऽल॒ मलं॒ ॅयो यो ऽल᳚म् । \newline
16. अल॒म् भूत्यै॒ भूत्या॒ अल॒ मल॒म् भूत्यै᳚ । \newline
17. भूत्यै॒ सन् थ्सन् भूत्यै॒ भूत्यै॒ सन्न् । \newline
18. सन् भूति॒म् भूतिꣳ॒॒ सन् थ्सन् भूति᳚म् । \newline
19. भूतिं॒ न न भूति॒म् भूतिं॒ न । \newline
20. न प्रा॒प्नोति॑ प्रा॒प्नोति॒ न न प्रा॒प्नोति॑ । \newline
21. प्रा॒प्नो त्या॑दि॒त्या ना॑दि॒त्यान् प्रा॒प्नोति॑ प्रा॒प्नो त्या॑दि॒त्यान् । \newline
22. प्रा॒प्नोतीति॑ प्र - आ॒प्नोति॑ । \newline
23. आ॒दि॒त्या ने॒वैवादि॒त्या ना॑दि॒त्या ने॒व । \newline
24. ए॒व भुव॑द्वतो॒ भुव॑द्वत ए॒वैव भुव॑द्वतः । \newline
25. भुव॑द्वतः॒ स्वेन॒ स्वेन॒ भुव॑द्वतो॒ भुव॑द्वतः॒ स्वेन॑ । \newline
26. भुव॑द्वत॒ इति॒ भुव॑त् - व॒तः॒ । \newline
27. स्वेन॑ भाग॒धेये॑न भाग॒धेये॑न॒ स्वेन॒ स्वेन॑ भाग॒धेये॑न । \newline
28. भा॒ग॒धेये॒नोपोप॑ भाग॒धेये॑न भाग॒धेये॒नोप॑ । \newline
29. भा॒ग॒धेये॒नेति॑ भाग - धेये॑न । \newline
30. उप॑ धावति धाव॒ त्युपोप॑ धावति । \newline
31. धा॒व॒ति॒ ते ते धा॑वति धावति॒ ते । \newline
32. त ए॒वैव ते त ए॒व । \newline
33. ए॒वैन॑ मेन मे॒ वैवैन᳚म् । \newline
34. ए॒न॒म् भूति॒म् भूति॑ मेन मेन॒म् भूति᳚म् । \newline
35. भूति॑म् गमयन्ति गमयन्ति॒ भूति॒म् भूति॑म् गमयन्ति । \newline
36. ग॒म॒य॒न्ति॒ भव॑ति॒ भव॑ति गमयन्ति गमयन्ति॒ भव॑ति । \newline
37. भव॑ त्ये॒वैव भव॑ति॒ भव॑ त्ये॒व । \newline
38. ए॒वादि॒त्येभ्य॑ आदि॒त्येभ्य॑ ए॒वैवादि॒त्येभ्यः॑ । \newline
39. आ॒दि॒त्येभ्यो॑ धा॒रय॑द्वद्भ्यो धा॒रय॑द्वद्भ्य आदि॒त्येभ्य॑ आदि॒त्येभ्यो॑ धा॒रय॑द्वद्भ्यः । \newline
40. धा॒रय॑द्वद्भ्य श्च॒रुम् च॒रुम् धा॒रय॑द्वद्भ्यो धा॒रय॑द्वद्भ्य श्च॒रुम् । \newline
41. धा॒रय॑द्वद्भ्य॒ इति॑ धा॒रय॑द्वत् - भ्यः॒ । \newline
42. च॒रुम् निर् णिश्च॒रुम् च॒रुम् निः । \newline
43. निर् व॑पेद् वपे॒न् निर् णिर् व॑पेत् । \newline
44. व॒पे॒ दप॑रु॒द्धो ऽप॑रुद्धो वपेद् वपे॒ दप॑रुद्धः । \newline
45. अप॑रुद्धो वा॒ वा ऽप॑रु॒द्धो ऽप॑रुद्धो वा । \newline
46. अप॑रुद्ध॒ इत्यप॑ - रु॒द्धः॒ । \newline
47. वा॒ ऽप॒रु॒द्ध्यमा॑नो ऽपरु॒द्ध्यमा॑नो वा वा ऽपरु॒द्ध्यमा॑नः । \newline
48. अ॒प॒रु॒द्ध्यमा॑नो वा वा ऽपरु॒द्ध्यमा॑नो ऽपरु॒द्ध्यमा॑नो वा । \newline
49. अ॒प॒रु॒द्ध्यमा॑न॒ इत्य॑प - रु॒ध्यमा॑नः । \newline
50. वा॒ ऽऽदि॒त्या आ॑दि॒त्या वा॑ वा ऽऽदि॒त्याः । \newline
51. आ॒दि॒त्या वै वा आ॑दि॒त्या आ॑दि॒त्या वै । \newline
52. वा अ॑परो॒द्धारो॑ ऽपरो॒द्धारो॒ वै वा अ॑परो॒द्धारः॑ । \newline
53. अ॒प॒रो॒द्धार॑ आदि॒त्या आ॑दि॒त्या अ॑परो॒द्धारो॑ ऽपरो॒द्धार॑ आदि॒त्याः । \newline
54. अ॒प॒रो॒द्धार॒ इत्य॑प - रो॒द्धारः॑ । \newline
55. आ॒दि॒त्या अ॑वगमयि॒तारो॑ ऽवगमयि॒तार॑ आदि॒त्या आ॑दि॒त्या अ॑वगमयि॒तारः॑ । \newline
56. अ॒व॒ग॒म॒यि॒तार॑ आदि॒त्या ना॑दि॒त्या न॑वगमयि॒तारो॑ ऽवगमयि॒तार॑ आदि॒त्यान् । \newline
57. अ॒व॒ग॒म॒यि॒तार॒ इत्य॑व - ग॒म॒यि॒तारः॑ । \newline
58. आ॒दि॒त्या ने॒वै वादि॒त्या ना॑दि॒त्या ने॒व । \newline
59. ए॒व धा॒रय॑द्वतो धा॒रय॑द्वत ए॒वैव धा॒रय॑द्वतः । \newline
60. धा॒रय॑द्वतः॒ स्वेन॒ स्वेन॑ धा॒रय॑द्वतो धा॒रय॑द्वतः॒ स्वेन॑ । \newline
61. धा॒रय॑द्वत॒ इति॑ धा॒रय॑त् - व॒तः॒ । \newline

\textbf{Ghana Paata } \newline

1. आ॒दि॒त्येभ्यो॒ भुव॑द्वद्भ्यो॒ भुव॑द्वद्भ्य आदि॒त्येभ्य॑ आदि॒त्येभ्यो॒ भुव॑द्वद्भ्य श्च॒रुम् च॒रुम् भुव॑द्वद्भ्य आदि॒त्येभ्य॑ आदि॒त्येभ्यो॒ भुव॑द्वद्भ्य श्च॒रुम् । \newline
2. भुव॑द्वद्भ्य श्च॒रुम् च॒रुम् भुव॑द्वद्भ्यो॒ भुव॑द्वद्भ्यश्च॒रुम् निर् णिश्च॒रुम् भुव॑द्वद्भ्यो॒ भुव॑द्वद्भ्यश्च॒रुम् निः । \newline
3. भुव॑द्वद्भ्य॒ इति॒ भुव॑द्वत् - भ्यः॒ । \newline
4. च॒रुम् निर् णिश् च॒रुम् च॒रुम् निर् व॑पेद् वपे॒न् नि श्च॒रुम् च॒रुम् निर् व॑पेत् । \newline
5. निर् व॑पेद् वपे॒न् निर् णिर् व॑पे॒द् भूति॑कामो॒ भूति॑कामो वपे॒न् निर् णिर् व॑पे॒द् भूति॑कामः । \newline
6. व॒पे॒द् भूति॑कामो॒ भूति॑कामो वपेद् वपे॒द् भूति॑काम आदि॒त्या आ॑दि॒त्या भूति॑कामो वपेद् वपे॒द् भूति॑काम आदि॒त्याः । \newline
7. भूति॑काम आदि॒त्या आ॑दि॒त्या भूति॑कामो॒ भूति॑काम आदि॒त्या वै वा आ॑दि॒त्या भूति॑कामो॒ भूति॑काम आदि॒त्या वै । \newline
8. भूति॑काम॒ इति॒ भूति॑ - का॒मः॒ । \newline
9. आ॒दि॒त्या वै वा आ॑दि॒त्या आ॑दि॒त्या वा ए॒त मे॒तं ॅवा आ॑दि॒त्या आ॑दि॒त्या वा ए॒तम् । \newline
10. वा ए॒त मे॒तं ॅवै वा ए॒तम् भूत्यै॒ भूत्या॑ ए॒तं ॅवै वा ए॒तम् भूत्यै᳚ । \newline
11. ए॒तम् भूत्यै॒ भूत्या॑ ए॒त मे॒तम् भूत्यै॒ प्रति॒ प्रति॒ भूत्या॑ ए॒त मे॒तम् भूत्यै॒ प्रति॑ । \newline
12. भूत्यै॒ प्रति॒ प्रति॒ भूत्यै॒ भूत्यै॒ प्रति॑ नुदन्ते नुदन्ते॒ प्रति॒ भूत्यै॒ भूत्यै॒ प्रति॑ नुदन्ते । \newline
13. प्रति॑ नुदन्ते नुदन्ते॒ प्रति॒ प्रति॑ नुदन्ते॒ यो यो नु॑दन्ते॒ प्रति॒ प्रति॑ नुदन्ते॒ यः । \newline
14. नु॒द॒न्ते॒ यो यो नु॑दन्ते नुदन्ते॒ यो ऽल॒ मलं॒ ॅयो नु॑दन्ते नुदन्ते॒ यो ऽल᳚म् । \newline
15. यो ऽल॒ मलं॒ ॅयो यो ऽल॒म् भूत्यै॒ भूत्या॒ अलं॒ ॅयो यो ऽल॒म् भूत्यै᳚ । \newline
16. अल॒म् भूत्यै॒ भूत्या॒ अल॒ मल॒म् भूत्यै॒ सन् थ्सन् भूत्या॒ अल॒ मल॒म् भूत्यै॒ सन्न् । \newline
17. भूत्यै॒ सन् थ्सन् भूत्यै॒ भूत्यै॒ सन् भूति॒म् भूतिꣳ॒॒ सन् भूत्यै॒ भूत्यै॒ सन् भूति᳚म् । \newline
18. सन् भूति॒म् भूतिꣳ॒॒ सन् थ्सन् भूति॒म् न न भूतिꣳ॒॒ सन् थ्सन् भूति॒म् न । \newline
19. भूति॒म् न न भूति॒म् भूति॒म् न प्रा॒प्नोति॑ प्रा॒प्नोति॒ न भूति॒म् भूति॒म् न प्रा॒प्नोति॑ । \newline
20. न प्रा॒प्नोति॑ प्रा॒प्नोति॒ न न प्रा॒प्नो त्या॑दि॒त्या ना॑दि॒त्यान् प्रा॒प्नोति॒ न न प्रा॒प्नो त्या॑दि॒त्यान् । \newline
21. प्रा॒प्नो त्या॑दि॒त्या ना॑दि॒त्यान् प्रा॒प्नोति॑ प्रा॒प्नो त्या॑दि॒त्या ने॒वैवादि॒त्यान् प्रा॒प्नोति॑ प्रा॒प्नो त्या॑दि॒त्या ने॒व । \newline
22. प्रा॒प्नोतीति॑ प्र - आ॒प्नोति॑ । \newline
23. आ॒दि॒त्या ने॒वैवादि॒त्या ना॑दि॒त्या ने॒व भुव॑द्वतो॒ भुव॑द्वत ए॒वादि॒त्या ना॑दि॒त्या ने॒व भुव॑द्वतः । \newline
24. ए॒व भुव॑द्वतो॒ भुव॑द्वत ए॒वैव भुव॑द्वतः॒ स्वेन॒ स्वेन॒ भुव॑द्वत ए॒वैव भुव॑द्वतः॒ स्वेन॑ । \newline
25. भुव॑द्वतः॒ स्वेन॒ स्वेन॒ भुव॑द्वतो॒ भुव॑द्वतः॒ स्वेन॑ भाग॒धेये॑न भाग॒धेये॑न॒ स्वेन॒ भुव॑द्वतो॒ भुव॑द्वतः॒ स्वेन॑ भाग॒धेये॑न । \newline
26. भुव॑द्वत॒ इति॒ भुव॑त् - व॒तः॒ । \newline
27. स्वेन॑ भाग॒धेये॑न भाग॒धेये॑न॒ स्वेन॒ स्वेन॑ भाग॒धेये॒नोपोप॑ भाग॒धेये॑न॒ स्वेन॒ स्वेन॑ भाग॒धेये॒नोप॑ । \newline
28. भा॒ग॒धेये॒नोपोप॑ भाग॒धेये॑न भाग॒धेये॒नोप॑ धावति धाव॒त्युप॑ भाग॒धेये॑न भाग॒धेये॒नोप॑ धावति । \newline
29. भा॒ग॒धेये॒नेति॑ भाग - धेये॑न । \newline
30. उप॑ धावति धाव॒ त्युपोप॑ धावति॒ ते ते धा॑व॒ त्युपोप॑ धावति॒ ते । \newline
31. धा॒व॒ति॒ ते ते धा॑वति धावति॒ त ए॒वैव ते धा॑वति धावति॒ त ए॒व । \newline
32. त ए॒वैव ते त ए॒वैन॑ मेन मे॒व ते त ए॒वैन᳚म् । \newline
33. ए॒वैन॑ मेन मे॒वैवैन॒म् भूति॒म् भूति॑ मेन मे॒वैवैन॒म् भूति᳚म् । \newline
34. ए॒न॒म् भूति॒म् भूति॑ मेन मेन॒म् भूति॑म् गमयन्ति गमयन्ति॒ भूति॑ मेन मेन॒म् भूति॑म् गमयन्ति । \newline
35. भूति॑म् गमयन्ति गमयन्ति॒ भूति॒म् भूति॑म् गमयन्ति॒ भव॑ति॒ भव॑ति गमयन्ति॒ भूति॒म् भूति॑म् गमयन्ति॒ भव॑ति । \newline
36. ग॒म॒य॒न्ति॒ भव॑ति॒ भव॑ति गमयन्ति गमयन्ति॒ भव॑त्ये॒वैव भव॑ति गमयन्ति गमयन्ति॒ भव॑त्ये॒व । \newline
37. भव॑ त्ये॒वैव भव॑ति॒ भव॑ त्ये॒वादि॒त्येभ्य॑ आदि॒त्येभ्य॑ ए॒व भव॑ति॒ भव॑ त्ये॒वादि॒त्येभ्यः॑ । \newline
38. ए॒वादि॒त्येभ्य॑ आदि॒त्येभ्य॑ ए॒वैवादि॒त्येभ्यो॑ धा॒रय॑द्वद्भ्यो धा॒रय॑द्वद्भ्य आदि॒त्येभ्य॑ ए॒वैवादि॒त्येभ्यो॑ धा॒रय॑द्वद्भ्यः । \newline
39. आ॒दि॒त्येभ्यो॑ धा॒रय॑द्वद्भ्यो धा॒रय॑द्वद्भ्य आदि॒त्येभ्य॑ आदि॒त्येभ्यो॑ धा॒रय॑द्वद्भ्य श्च॒रुम् च॒रुम् धा॒रय॑द्वद्भ्य आदि॒त्येभ्य॑ आदि॒त्येभ्यो॑ धा॒रय॑द्वद्भ्य श्च॒रुम् । \newline
40. धा॒रय॑द्वद्भ्य श्च॒रुम् च॒रुम् धा॒रय॑द्वद्भ्यो धा॒रय॑द्वद्भ्य श्च॒रुम् निर् णिश्च॒रुम् धा॒रय॑द्वद्भ्यो धा॒रय॑द्वद्भ्य श्च॒रुम् निः । \newline
41. धा॒रय॑द्वद्भ्य॒ इति॑ धा॒रय॑द्वत् - भ्यः॒ । \newline
42. च॒रुम् निर् णिश्च॒रुम् च॒रुम् निर् व॑पेद् वपे॒न् निश्च॒रुम् च॒रुम् निर् व॑पेत् । \newline
43. निर् व॑पेद् वपे॒न् निर् णिर् व॑पे॒ दप॑रु॒द्धो ऽप॑रुद्धो वपे॒न् निर् णिर् व॑पे॒ दप॑रुद्धः । \newline
44. व॒पे॒ दप॑रु॒द्धो ऽप॑रुद्धो वपेद् वपे॒ दप॑रुद्धो वा॒ वा ऽप॑रुद्धो वपेद् वपे॒ दप॑रुद्धो वा । \newline
45. अप॑रुद्धो वा॒ वा ऽप॑रु॒द्धो ऽप॑रुद्धो वा ऽपरु॒द्ध्यमा॑नो ऽपरु॒द्ध्यमा॑नो॒ वा ऽप॑रु॒द्धो ऽप॑रुद्धो वा ऽपरु॒द्ध्यमा॑नः । \newline
46. अप॑रुद्ध॒ इत्यप॑ - रु॒द्धः॒ । \newline
47. वा॒ ऽप॒रु॒द्ध्यमा॑नो ऽपरु॒द्ध्यमा॑नो वा वा ऽपरु॒द्ध्यमा॑नो वा वा ऽपरु॒द्ध्यमा॑नो वा वा ऽपरु॒द्ध्यमा॑नो वा । \newline
48. अ॒प॒रु॒द्ध्यमा॑नो वा वा ऽपरु॒द्ध्यमा॑नो ऽपरु॒द्ध्यमा॑नो वा ऽऽदि॒त्या आ॑दि॒त्या वा॑ ऽपरु॒द्ध्यमा॑नो ऽपरु॒द्ध्यमा॑नो वा ऽऽदि॒त्याः । \newline
49. अ॒प॒रु॒द्ध्यमा॑न॒ इत्य॑प - रु॒ध्यमा॑नः । \newline
50. वा॒ ऽऽदि॒त्या आ॑दि॒त्या वा॑ वा ऽऽदि॒त्या वै वा आ॑दि॒त्या वा॑ वा ऽऽदि॒त्या वै । \newline
51. आ॒दि॒त्या वै वा आ॑दि॒त्या आ॑दि॒त्या वा अ॑परो॒द्धारो॑ ऽपरो॒द्धारो॒ वा आ॑दि॒त्या आ॑दि॒त्या वा अ॑परो॒द्धारः॑ । \newline
52. वा अ॑परो॒द्धारो॑ ऽपरो॒द्धारो॒ वै वा अ॑परो॒द्धार॑ आदि॒त्या आ॑दि॒त्या अ॑परो॒द्धारो॒ वै वा अ॑परो॒द्धार॑ आदि॒त्याः । \newline
53. अ॒प॒रो॒द्धार॑ आदि॒त्या आ॑दि॒त्या अ॑परो॒द्धारो॑ ऽपरो॒द्धार॑ आदि॒त्या अ॑वगमयि॒तारो॑ ऽवगमयि॒तार॑ आदि॒त्या अ॑परो॒द्धारो॑ ऽपरो॒द्धार॑ आदि॒त्या अ॑वगमयि॒तारः॑ । \newline
54. अ॒प॒रो॒द्धार॒ इत्य॑प - रो॒द्धारः॑ । \newline
55. आ॒दि॒त्या अ॑वगमयि॒तारो॑ ऽवगमयि॒तार॑ आदि॒त्या आ॑दि॒त्या अ॑वगमयि॒तार॑ आदि॒त्या ना॑दि॒त्या न॑वगमयि॒तार॑ आदि॒त्या आ॑दि॒त्या अ॑वगमयि॒तार॑ आदि॒त्यान् । \newline
56. अ॒व॒ग॒म॒यि॒तार॑ आदि॒त्या ना॑दि॒त्या न॑वगमयि॒तारो॑ ऽवगमयि॒तार॑ आदि॒त्या ने॒वैवादि॒त्या न॑वगमयि॒तारो॑ ऽवगमयि॒तार॑ आदि॒त्या ने॒व । \newline
57. अ॒व॒ग॒म॒यि॒तार॒ इत्य॑व - ग॒म॒यि॒तारः॑ । \newline
58. आ॒दि॒त्या ने॒वैवादि॒त्या ना॑दि॒त्या ने॒व धा॒रय॑द्वतो धा॒रय॑द्वत ए॒वादि॒त्या ना॑दि॒त्या ने॒व धा॒रय॑द्वतः । \newline
59. ए॒व धा॒रय॑द्वतो धा॒रय॑द्वत ए॒वैव धा॒रय॑द्वतः॒ स्वेन॒ स्वेन॑ धा॒रय॑द्वत ए॒वैव धा॒रय॑द्वतः॒ स्वेन॑ । \newline
60. धा॒रय॑द्वतः॒ स्वेन॒ स्वेन॑ धा॒रय॑द्वतो धा॒रय॑द्वतः॒ स्वेन॑ भाग॒धेये॑न भाग॒धेये॑न॒ स्वेन॑ धा॒रय॑द्वतो धा॒रय॑द्वतः॒ स्वेन॑ भाग॒धेये॑न । \newline
61. धा॒रय॑द्वत॒ इति॑ धा॒रय॑त् - व॒तः॒ । \newline
\pagebreak
\markright{ TS 2.3.1.2  \hfill https://www.vedavms.in \hfill}

\section{ TS 2.3.1.2 }

\textbf{TS 2.3.1.2 } \newline
\textbf{Samhita Paata} \newline

स्वेन॑ भाग॒धेये॒नोप॑ धावति॒ त ए॒वैनं॑ ॅवि॒शि दा᳚द्ध्रत्यनपरु॒द्ध्यो भ॑व॒त्यदि॒तेऽनु॑ मन्य॒स्वे-त्य॑परु॒द्ध्यमा॑नोऽस्य प॒दमा द॑दीते॒यं ॅवा अदि॑तिरि॒यमे॒वास्मै॑ रा॒ज्यमनु॑ मन्यते स॒त्याऽऽशीरित्या॑ह स॒त्यामे॒वाऽऽशिषं॑ कुरुत इ॒ह मन॒ इत्या॑ह प्र॒जा ए॒वास्मै॒ सम॑नसः करो॒त्युप॒ प्रेत॑ मरुतः - [  ] \newline

\textbf{Pada Paata} \newline

स्वेन॑ । भा॒ग॒धेये॒नेति॑ भाग - धेये॑न । उपेति॑ । धा॒व॒ति॒ । ते । ए॒व । ए॒न॒म् । वि॒शि । दा॒द्ध्र॒ति॒ । अ॒न॒प॒रु॒द्ध्य इत्य॑नप - रु॒द्ध्यः । भ॒व॒ति॒ । अदि॑ते । अन्विति॑ । म॒न्य॒स्व॒ । इति॑ । अ॒प॒रु॒द्ध्यमा॑न॒ इत्य॑प - रु॒द्ध्यमा॑नः । अ॒स्य॒ । प॒दम् । एति॑ । द॒दी॒त॒ । इ॒यम् । वै । अदि॑तिः । इ॒यम् । ए॒व । अ॒स्मै॒ । रा॒ज्यम् । अन्विति॑ । म॒न्य॒ते॒ । स॒त्या । आ॒शीरित्या᳚ - शीः । इति॑ । आ॒ह॒ । स॒त्याम् । ए॒व । आ॒शिष॒मित्या᳚ - शिष᳚म् । कु॒रु॒ते॒ । इ॒ह । मनः॑ । इति॑ । आ॒ह॒ । प्र॒जा इति॑ प्र - जाः । ए॒व । अ॒स्मै॒ । सम॑नस॒ इति॒ स - म॒न॒सः॒ । क॒रो॒ति॒ । उप॑ । प्रेति॑ । इ॒त॒ । म॒रु॒तः॒ ।  \newline


\textbf{Krama Paata} \newline

स्वेन॑ भाग॒धेये॑न । भा॒ग॒धेये॒नोप॑ । भा॒ग॒धेये॒नेति॑ भाग - धेये॑न । उप॑ धावति । धा॒व॒ति॒ ते । त ए॒व । ए॒वैन᳚म् । ए॒नं॒ ॅवि॒शि । वि॒शि दा᳚ध्रति । दा॒ध्र॒त्य॒न॒प॒रु॒द्ध्यः । अ॒न॒प॒रु॒द्ध्यो भ॑वति । अ॒न॒प॒रु॒द्ध्य इत्य॑नप - रु॒द्ध्यः । भ॒व॒त्यदि॑ते । अदि॒तेऽनु॑ । अनु॑ मन्यस्व । म॒न्य॒स्वेति॑ । इत्य॑परु॒द्ध्यमा॑नः । अ॒प॒रु॒द्ध्यमा॑नोऽस्य । अ॒प॒रु॒द्ध्यमा॑न॒ इत्य॑प - रु॒द्ध्यमा॑नः । अ॒स्य॒ प॒दम् । प॒दमा । आ द॑दीत । द॒दी॒ते॒यम् । इ॒यं ॅवै । वा अदि॑तिः । अदि॑तिरि॒यम् । इ॒यमे॒व । ए॒वास्मै᳚ । अ॒स्मै॒ रा॒ज्यम् । रा॒ज्यमनु॑ । अनु॑मन्यते । म॒न्य॒ते॒ स॒त्या । स॒त्याऽऽशीः । आ॒शीरिति॑ । आ॒शीरित्या᳚ - शीः । इत्या॑ह । आ॒ह॒ स॒त्याम् । स॒त्यामे॒व । ए॒वाशिष᳚म् । आ॒शिष॑म् कुरुते । आ॒शिष॒मित्या᳚ - शिष᳚म् । कु॒रु॒त॒ इ॒ह । इ॒ह मनः॑ । मन॒ इति॑ । इत्या॑ह । आ॒ह॒ प्र॒जाः । प्र॒जा ए॒व । प्र॒जा इति॑ प्र - जाः । ए॒वास्मै᳚ । अ॒स्मै॒ सम॑नसः । सम॑नसः करोति । सम॑नस॒ इति॒ स - म॒न॒सः॒ । क॒रो॒त्युप॑ । उप॒ प्र । प्रेत॑ । इ॒त॒ म॒रु॒तः॒ । म॒रु॒तः॒ सु॒दा॒न॒वः॒ \newline

\textbf{Jatai Paata} \newline

1. स्वेन॑ भाग॒धेये॑न भाग॒धेये॑न॒ स्वेन॒ स्वेन॑ भाग॒धेये॑न । \newline
2. भा॒ग॒धेये॒नोपोप॑ भाग॒धेये॑न भाग॒धेये॒नोप॑ । \newline
3. भा॒ग॒धेये॒नेति॑ भाग - धेये॑न । \newline
4. उप॑ धावति धाव॒ त्युपोप॑ धावति । \newline
5. धा॒व॒ति॒ ते ते धा॑वति धावति॒ ते । \newline
6. त ए॒वैव ते त ए॒व । \newline
7. ए॒वैन॑ मेन मे॒वैवैन᳚म् । \newline
8. ए॒नं॒ ॅवि॒शि वि॒श्ये॑न मेनं ॅवि॒शि । \newline
9. वि॒शि दा᳚द्ध्रति दाद्ध्रति वि॒शि वि॒शि दा᳚द्ध्रति । \newline
10. दा॒द्ध्र॒ त्य॒न॒प॒रु॒द्ध्यो॑ ऽनपरु॒द्ध्यो दा᳚द्ध्रति दाद्ध्र त्यनपरु॒द्ध्यः । \newline
11. अ॒न॒प॒रु॒द्ध्यो भ॑वति भव त्यनपरु॒द्ध्यो॑ ऽनपरु॒द्ध्यो भ॑वति । \newline
12. अ॒न॒प॒रु॒द्ध्य इत्य॑नप - रु॒द्ध्यः । \newline
13. भ॒व॒ त्यदि॒ते ऽदि॑ते भवति भव॒ त्यदि॑ते । \newline
14. अदि॒ते ऽन्वन्वदि॒ते ऽदि॒ते ऽनु॑ । \newline
15. अनु॑ मन्यस्व मन्य॒स्वा न्वनु॑ मन्यस्व । \newline
16. म॒न्य॒स्वे तीति॑ मन्यस्व मन्य॒स्वे ति॑ । \newline
17. इत्य॑परु॒द्ध्यमा॑नो ऽपरु॒द्ध्यमा॑न॒ इती त्य॑परु॒द्ध्यमा॑नः । \newline
18. अ॒प॒रु॒द्ध्यमा॑नो ऽस्यास्या परु॒द्ध्यमा॑नो ऽपरु॒द्ध्यमा॑नो ऽस्य । \newline
19. अ॒प॒रु॒द्ध्यमा॑न॒ इत्य॑प - रु॒द्ध्यमा॑नः । \newline
20. अ॒स्य॒ प॒दम् प॒द म॑स्यास्य प॒दम् । \newline
21. प॒द मा प॒दम् प॒द मा । \newline
22. आ द॑दीत ददी॒ता द॑दीत । \newline
23. द॒दी॒ते॒ य मि॒यम् द॑दीत ददीते॒ यम् । \newline
24. इ॒यं ॅवै वा इ॒य मि॒यं ॅवै । \newline
25. वा अदि॑ति॒ रदि॑ति॒र् वै वा अदि॑तिः । \newline
26. अदि॑ति रि॒य मि॒य मदि॑ति॒ रदि॑ति रि॒यम् । \newline
27. इ॒य मे॒वैवे य मि॒य मे॒व । \newline
28. ए॒वास्मा॑ अस्मा ए॒वैवास्मै᳚ । \newline
29. अ॒स्मै॒ रा॒ज्यꣳ रा॒ज्य म॑स्मा अस्मै रा॒ज्यम् । \newline
30. रा॒ज्य मन्वनु॑ रा॒ज्यꣳ रा॒ज्य मनु॑ । \newline
31. अनु॑ मन्यते मन्य॒ते ऽन्वनु॑ मन्यते । \newline
32. म॒न्य॒ते॒ स॒त्या स॒त्या म॑न्यते मन्यते स॒त्या । \newline
33. स॒त्या ऽऽशीरा॒शीः स॒त्या स॒त्या ऽऽशीः । \newline
34. आ॒शी रिती त्या॒शी रा॒शी रिति॑ । \newline
35. आ॒शीरित्या᳚ - शीः । \newline
36. इत्या॑हा॒हे तीत्या॑ह । \newline
37. आ॒ह॒ स॒त्याꣳ स॒त्या मा॑हाह स॒त्याम् । \newline
38. स॒त्या मे॒वैव स॒त्याꣳ स॒त्या मे॒व । \newline
39. ए॒वाशिष॑ मा॒शिष॑ मे॒वैवाशिष᳚म् । \newline
40. आ॒शिष॑म् कुरुते कुरुत आ॒शिष॑ मा॒शिष॑म् कुरुते । \newline
41. आ॒शिष॒मित्या᳚ - शिष᳚म् । \newline
42. कु॒रु॒त॒ इ॒हे ह कु॑रुते कुरुत इ॒ह । \newline
43. इ॒ह मनो॒ मन॑ इ॒हे ह मनः॑ । \newline
44. मन॒ इतीति॒ मनो॒ मन॒ इति॑ । \newline
45. इत्या॑हा॒हे तीत्या॑ह । \newline
46. आ॒ह॒ प्र॒जाः प्र॒जा आ॑हाह प्र॒जाः । \newline
47. प्र॒जा ए॒वैव प्र॒जाः प्र॒जा ए॒व । \newline
48. प्र॒जा इति॑ प्र - जाः । \newline
49. ए॒वास्मा॑ अस्मा ए॒वैवास्मै᳚ । \newline
50. अ॒स्मै॒ सम॑नसः॒ सम॑नसो ऽस्मा अस्मै॒ सम॑नसः । \newline
51. सम॑नसः करोति करोति॒ सम॑नसः॒ सम॑नसः करोति । \newline
52. सम॑नस॒ इति॒ स - म॒न॒सः॒ । \newline
53. क॒रो॒ त्युपोप॑ करोति करो॒ त्युप॑ । \newline
54. उप॒ प्र प्रोपोप॒ प्र । \newline
55. प्रे ते॑ त॒ प्र प्रे त॑ । \newline
56. इ॒त॒ म॒रु॒तो॒ म॒रु॒त॒ इ॒ते॒ त॒ म॒रु॒तः॒ । \newline
57. म॒रु॒तः॒ सु॒दा॒न॒वः॒ सु॒दा॒न॒वो॒ म॒रु॒तो॒ म॒रु॒तः॒ सु॒दा॒न॒वः॒ । \newline

\textbf{Ghana Paata } \newline

1. स्वेन॑ भाग॒धेये॑न भाग॒धेये॑न॒ स्वेन॒ स्वेन॑ भाग॒धेये॒नोपोप॑ भाग॒धेये॑न॒ स्वेन॒ स्वेन॑ भाग॒धेये॒नोप॑ । \newline
2. भा॒ग॒धेये॒नोपोप॑ भाग॒धेये॑न भाग॒धेये॒नोप॑ धावति धाव॒त्युप॑ भाग॒धेये॑न भाग॒धेये॒नोप॑ धावति । \newline
3. भा॒ग॒धेये॒नेति॑ भाग - धेये॑न । \newline
4. उप॑ धावति धाव॒त्युपोप॑ धावति॒ ते ते धा॑व॒त्युपोप॑ धावति॒ ते । \newline
5. धा॒व॒ति॒ ते ते धा॑वति धावति॒ त ए॒वैव ते धा॑वति धावति॒ त ए॒व । \newline
6. त ए॒वैव ते त ए॒वैन॑ मेन मे॒व ते त ए॒वैन᳚म् । \newline
7. ए॒वैन॑ मेन मे॒वैवैनं॑ ॅवि॒शि वि॒श्ये॑न मे॒वैवैनं॑ ॅवि॒शि । \newline
8. ए॒नं॒ ॅवि॒शि वि॒श्ये॑न मेनं ॅवि॒शि दा᳚द्ध्रति दाद्ध्रति वि॒श्ये॑न मेनं ॅवि॒शि दा᳚द्ध्रति । \newline
9. वि॒शि दा᳚द्ध्रति दाद्ध्रति वि॒शि वि॒शि दा᳚द्ध्र त्यनपरु॒द्ध्यो॑ ऽनपरु॒द्ध्यो दा᳚द्ध्रति वि॒शि वि॒शि दा᳚द्ध्र त्यनपरु॒द्ध्यः । \newline
10. दा॒द्ध्र॒ त्य॒न॒प॒रु॒द्ध्यो॑ ऽनपरु॒द्ध्यो दा᳚द्ध्रति दाद्ध्र त्यनपरु॒द्ध्यो भ॑वति भव त्यनपरु॒द्ध्यो दा᳚द्ध्रति दाद्ध्र त्यनपरु॒द्ध्यो भ॑वति । \newline
11. अ॒न॒प॒रु॒द्ध्यो भ॑वति भव त्यनपरु॒द्ध्यो॑ ऽनपरु॒द्ध्यो भ॑व॒ त्यदि॒ते ऽदि॑ते भव त्यनपरु॒द्ध्यो॑ ऽनपरु॒द्ध्यो भ॑व॒ त्यदि॑ते । \newline
12. अ॒न॒प॒रु॒द्ध्य इत्य॑नप - रु॒द्ध्यः । \newline
13. भ॒व॒ त्यदि॒ते ऽदि॑ते भवति भव॒ त्यदि॒ते ऽन्वन्वदि॑ते भवति भव॒ त्यदि॒ते ऽनु॑ । \newline
14. अदि॒ते ऽन्वन्वदि॒ते ऽदि॒ते ऽनु॑ मन्यस्व मन्य॒स्वान्वदि॒ते ऽदि॒ते ऽनु॑ मन्यस्व । \newline
15. अनु॑ मन्यस्व मन्य॒स्वान्वनु॑ मन्य॒स्वे तीति॑ मन्य॒स्वान्वनु॑ मन्य॒स्वे ति॑ । \newline
16. म॒न्य॒स्वे तीति॑ मन्यस्व मन्य॒स्वे त्य॑परु॒द्ध्यमा॑नो ऽपरु॒द्ध्यमा॑न॒ इति॑ मन्यस्व मन्य॒स्वे त्य॑परु॒द्ध्यमा॑नः । \newline
17. इत्य॑परु॒द्ध्यमा॑नो ऽपरु॒द्ध्यमा॑न॒ इती त्य॑परु॒द्ध्यमा॑नो ऽस्यास्यापरु॒द्ध्यमा॑न॒ इती त्य॑परु॒द्ध्यमा॑नो ऽस्य । \newline
18. अ॒प॒रु॒द्ध्यमा॑नो ऽस्यास्या परु॒द्ध्यमा॑नो ऽपरु॒द्ध्यमा॑नो ऽस्य प॒दम् प॒द म॑स्या परु॒द्ध्यमा॑नो ऽपरु॒द्ध्यमा॑नो ऽस्य प॒दम् । \newline
19. अ॒प॒रु॒द्ध्यमा॑न॒ इत्य॑प - रु॒द्ध्यमा॑नः । \newline
20. अ॒स्य॒ प॒दम् प॒द म॑स्यास्य प॒द मा प॒द म॑स्यास्य प॒द मा । \newline
21. प॒द मा प॒दम् प॒द मा द॑दीत ददी॒ता प॒दम् प॒द मा द॑दीत । \newline
22. आ द॑दीत ददी॒ता द॑दीते॒ य मि॒यम् द॑दी॒ता द॑दीते॒ यम् । \newline
23. द॒दी॒ते॒ य मि॒यम् द॑दीत ददीते॒ यं ॅवै वा इ॒यम् द॑दीत ददीते॒ यं ॅवै । \newline
24. इ॒यं ॅवै वा इ॒य मि॒यं ॅवा अदि॑ति॒ रदि॑ति॒र् वा इ॒य मि॒यं ॅवा अदि॑तिः । \newline
25. वा अदि॑ति॒ रदि॑ति॒र् वै वा अदि॑ति रि॒य मि॒य मदि॑ति॒र् वै वा अदि॑ति रि॒यम् । \newline
26. अदि॑ति रि॒य मि॒य मदि॑ति॒ रदि॑ति रि॒य मे॒वैवे य मदि॑ति॒ रदि॑ति रि॒य मे॒व । \newline
27. इ॒य मे॒वैवे य मि॒य मे॒वास्मा॑ अस्मा ए॒वे य मि॒य मे॒वास्मै᳚ । \newline
28. ए॒वास्मा॑ अस्मा ए॒वैवास्मै॑ रा॒ज्यꣳ रा॒ज्य म॑स्मा ए॒वैवास्मै॑ रा॒ज्यम् । \newline
29. अ॒स्मै॒ रा॒ज्यꣳ रा॒ज्य म॑स्मा अस्मै रा॒ज्य मन्वनु॑ रा॒ज्य म॑स्मा अस्मै रा॒ज्य मनु॑ । \newline
30. रा॒ज्य मन्वनु॑ रा॒ज्यꣳ रा॒ज्य मनु॑ मन्यते मन्य॒ते ऽनु॑ रा॒ज्यꣳ रा॒ज्य मनु॑ मन्यते । \newline
31. अनु॑ मन्यते मन्य॒ते ऽन्वनु॑ मन्यते स॒त्या स॒त्या म॑न्य॒ते ऽन्वनु॑ मन्यते स॒त्या । \newline
32. म॒न्य॒ते॒ स॒त्या स॒त्या म॑न्यते मन्यते स॒त्या ऽऽशी रा॒शीः स॒त्या म॑न्यते मन्यते स॒त्या ऽऽशीः । \newline
33. स॒त्या ऽऽशी रा॒शीः स॒त्या स॒त्या ऽऽशी रितीत्या॒शीः स॒त्या स॒त्या ऽऽशीरिति॑ । \newline
34. आ॒शी रिती त्या॒शी रा॒शी रित्या॑हा॒हे त्या॒शी रा॒शी रित्या॑ह । \newline
35. आ॒शीरित्या᳚ - शीः । \newline
36. इत्या॑हा॒हे तीत्या॑ह स॒त्याꣳ स॒त्या मा॒हे तीत्या॑ह स॒त्याम् । \newline
37. आ॒ह॒ स॒त्याꣳ स॒त्या मा॑हाह स॒त्या मे॒वैव स॒त्या मा॑हाह स॒त्या मे॒व । \newline
38. स॒त्या मे॒वैव स॒त्याꣳ स॒त्या मे॒वाशिष॑ मा॒शिष॑ मे॒व स॒त्याꣳ स॒त्या मे॒वाशिष᳚म् । \newline
39. ए॒वाशिष॑ मा॒शिष॑ मे॒वैवाशिष॑म् कुरुते कुरुत आ॒शिष॑ मे॒वैवाशिष॑म् कुरुते । \newline
40. आ॒शिष॑म् कुरुते कुरुत आ॒शिष॑ मा॒शिष॑म् कुरुत इ॒हे ह कु॑रुत आ॒शिष॑ मा॒शिष॑म् कुरुत इ॒ह । \newline
41. आ॒शिष॒मित्या᳚ - शिष᳚म् । \newline
42. कु॒रु॒त॒ इ॒हे ह कु॑रुते कुरुत इ॒ह मनो॒ मन॑ इ॒ह कु॑रुते कुरुत इ॒ह मनः॑ । \newline
43. इ॒ह मनो॒ मन॑ इ॒हे ह मन॒ इतीति॒ मन॑ इ॒हे ह मन॒ इति॑ । \newline
44. मन॒ इतीति॒ मनो॒ मन॒ इत्या॑हा॒हे ति॒ मनो॒ मन॒ इत्या॑ह । \newline
45. इत्या॑हा॒हे तीत्या॑ह प्र॒जाः प्र॒जा आ॒हे तीत्या॑ह प्र॒जाः । \newline
46. आ॒ह॒ प्र॒जाः प्र॒जा आ॑हाह प्र॒जा ए॒वैव प्र॒जा आ॑हाह प्र॒जा ए॒व । \newline
47. प्र॒जा ए॒वैव प्र॒जाः प्र॒जा ए॒वास्मा॑ अस्मा ए॒व प्र॒जाः प्र॒जा ए॒वास्मै᳚ । \newline
48. प्र॒जा इति॑ प्र - जाः । \newline
49. ए॒वास्मा॑ अस्मा ए॒वैवास्मै॒ सम॑नसः॒ सम॑नसो ऽस्मा ए॒वैवास्मै॒ सम॑नसः । \newline
50. अ॒स्मै॒ सम॑नसः॒ सम॑नसो ऽस्मा अस्मै॒ सम॑नसः करोति करोति॒ सम॑नसो ऽस्मा अस्मै॒ सम॑नसः करोति । \newline
51. सम॑नसः करोति करोति॒ सम॑नसः॒ सम॑नसः करो॒ त्युपोप॑ करोति॒ सम॑नसः॒ सम॑नसः करो॒ त्युप॑ । \newline
52. सम॑नस॒ इति॒ स - म॒न॒सः॒ । \newline
53. क॒रो॒त्युपोप॑ करोति करो॒त्युप॒ प्र प्रोप॑ करोति करो॒त्युप॒ प्र । \newline
54. उप॒ प्र प्रोपोप॒ प्रे ते॑ त॒ प्रोपोप॒ प्रे त॑ । \newline
55. प्रे ते॑ त॒ प्र प्रे त॑ मरुतो मरुत इत॒ प्र प्रे त॑ मरुतः । \newline
56. इ॒त॒ म॒रु॒तो॒ म॒रु॒त॒ इ॒ते॒ त॒ म॒रु॒तः॒ सु॒दा॒न॒वः॒ सु॒दा॒न॒वो॒ म॒रु॒त॒ इ॒ते॒ त॒ म॒रु॒तः॒ सु॒दा॒न॒वः॒ । \newline
57. म॒रु॒तः॒ सु॒दा॒न॒वः॒ सु॒दा॒न॒वो॒ म॒रु॒तो॒ म॒रु॒तः॒ सु॒दा॒न॒व॒ ए॒नैना सु॑दानवो मरुतो मरुतः सुदानव ए॒ना । \newline
\pagebreak
\markright{ TS 2.3.1.3  \hfill https://www.vedavms.in \hfill}

\section{ TS 2.3.1.3 }

\textbf{TS 2.3.1.3 } \newline
\textbf{Samhita Paata} \newline

सुदानव ए॒ना वि॒श्पति॑ना॒ऽभ्य॑मुꣳ राजा॑न॒मित्या॑ह मारु॒ती वै विड् ज्ये॒ष्ठो वि॒श्पति॑र्वि॒शैवैनꣳ॑ रा॒ष्ट्रेण॒ सम॑र्द्धयति॒ यः प॒रस्ता᳚द् ग्राम्यवा॒दी स्यात् तस्य॑ गृ॒हाद्-व्री॒हीना ह॑रेच्छु॒क्लाꣳश्च॑ कृ॒ष्णाꣳश्च॒ वि चि॑नुया॒द्ये शु॒क्लाः स्युस्तमा॑दि॒त्यं च॒रुं निर्व॑पेदादि॒त्या वै दे॒वत॑या॒ विड्विश॑मे॒वाव॑ गच्छ॒ - [  ] \newline

\textbf{Pada Paata} \newline

सु॒दा॒न॒व॒ इति॑ सु - दा॒न॒वः॒ । ए॒ना । वि॒श्पति॑ना । अ॒भीति॑ । अ॒मुम् । राजा॑नम् । इति॑ । आ॒ह॒ । मा॒रु॒ती । वै । विट् । ज्ये॒ष्ठः । वि॒श्पतिः॑ । वि॒शा । ए॒व । ए॒न॒म् । रा॒ष्ट्रेण॑ । समिति॑ । अ॒र्द्ध॒य॒ति॒ । यः । प॒रस्ता᳚त् । ग्रा॒म्य॒वा॒दीति॑ ग्राम्य - वा॒दी । स्यात् । तस्य॑ । गृ॒हात् । व्री॒हीन् । एति॑ । ह॒रे॒त् । शु॒क्लान् । च॒ । कृ॒ष्णान् । च॒ । वीति॑ । चि॒नु॒या॒त् । ये । शु॒क्लाः । स्युः । तम् । आ॒दि॒त्यम् । च॒रुम् । निरीति॑ । व॒पे॒त् । आ॒दि॒त्या । वै । दे॒वत॑या । विट् । विश᳚म् । ए॒व । अवेति॑ । ग॒च्छ॒ति॒ ।  \newline


\textbf{Krama Paata} \newline

सु॒दा॒न॒व॒ ए॒ना । सु॒दा॒न॒व॒ इति॑ सु - दा॒न॒वः॒ । ए॒ना वि॒श्पति॑ना । वि॒श्पति॑ना॒ ऽभि । अ॒भ्य॑मुम् । अ॒मुꣳ राजा॑नम् । राजा॑न॒मिति॑ । इत्या॑ह । आ॒ह॒ मा॒रु॒ती । मा॒रु॒ती वै । वै विट् । विड् ज्ये॒ष्ठः । ज्ये॒ष्ठो वि॒श्पतिः॑ । वि॒श्पति॑र् वि॒शा । वि॒शैव । ए॒वैन᳚म् । ए॒नꣳ॒॒ रा॒ष्ट्रेण॑ । रा॒ष्ट्रेण॒ सम् । सम॑र्द्धयति । अ॒र्द्ध॒य॒ति॒ यः । यः प॒रस्ता᳚त् । प॒रस्ता᳚द् ग्राम्यवा॒दी । ग्रा॒म्य॒वा॒दी स्यात् । ग्रा॒म्य॒वा॒दीति॑ ग्राम्य - वा॒दी । स्यात् तस्य॑ । तस्य॑ गृ॒हात् । गृ॒हाद्,व्री॒हीन् । व्री॒हीना । आ ह॑रेत् । ह॒रे॒च्छु॒क्लान् । शु॒क्लाꣳश्च॑ । च॒ कृ॒ष्णान् । कृ॒ष्णाꣳश्च॑ । च॒ वि । वि चि॑नुयात् । चि॒नु॒या॒द् ये । ये शु॒क्लाः । शु॒क्लाः स्युः । स्यु स्तम् । तमा॑दि॒त्यम् । आ॒दि॒त्यम् च॒रुम् । च॒रुम् निः । निर् व॑पेत् । व॒पे॒दा॒दि॒त्या । आ॒दि॒त्या वै । वै दे॒वत॑या । दे॒वत॑या॒ विट् । विड् विश᳚म् । विश॑मे॒व । ए॒वाव॑ । अव॑ गच्छति । ग॒च्छ॒त्यव॑गता \newline

\textbf{Jatai Paata} \newline

1. सु॒दा॒न॒व॒ ए॒नैना सु॑दानवः सुदानव ए॒ना । \newline
2. सु॒दा॒न॒व॒ इति॑ सु - दा॒न॒वः॒ । \newline
3. ए॒ना वि॒श्पति॑ना वि॒श्पति॑ नै॒नैना वि॒श्पति॑ना । \newline
4. वि॒श्पति॑ना॒ ऽभ्य॑भि वि॒श्पति॑ना वि॒श्पति॑ना॒ ऽभि । \newline
5. अ॒भ्य॑मु म॒मु म॒भ्या᳚(1॒)भ्य॑मुम् । \newline
6. अ॒मुꣳ राजा॑नꣳ॒॒ राजा॑न म॒मु म॒मुꣳ राजा॑नम् । \newline
7. राजा॑न॒ मितीति॒ राजा॑नꣳ॒॒ राजा॑न॒ मिति॑ । \newline
8. इत्या॑हा॒हे तीत्या॑ह । \newline
9. आ॒ह॒ मा॒रु॒ती मा॑रु॒ त्या॑हाह मारु॒ती । \newline
10. मा॒रु॒ती वै वै मा॑रु॒ती मा॑रु॒ती वै । \newline
11. वै विड् विड् वै वै विट् । \newline
12. विड् ज्ये॒ष्ठो ज्ये॒ष्ठो विड् विड् ज्ये॒ष्ठः । \newline
13. ज्ये॒ष्ठो वि॒श्पति॑र् वि॒श्पति॑र् ज्ये॒ष्ठो ज्ये॒ष्ठो वि॒श्पतिः॑ । \newline
14. वि॒श्पति॑र् वि॒शा वि॒शा वि॒श्पति॑र् वि॒श्पति॑र् वि॒शा । \newline
15. वि॒शैवैव वि॒शा वि॒शैव । \newline
16. ए॒वैन॑ मेन मे॒वैवैन᳚म् । \newline
17. ए॒नꣳ॒॒ रा॒ष्ट्रेण॑ रा॒ष्ट्रेणै॑न मेनꣳ रा॒ष्ट्रेण॑ । \newline
18. रा॒ष्ट्रेण॒ सꣳ सꣳ रा॒ष्ट्रेण॑ रा॒ष्ट्रेण॒ सम् । \newline
19. स म॑र्द्धय त्यर्द्धयति॒ सꣳ स म॑र्द्धयति । \newline
20. अ॒र्द्ध॒य॒ति॒ यो यो᳚ ऽर्द्धय त्यर्द्धयति॒ यः । \newline
21. यः प॒रस्ता᳚त् प॒रस्ता॒द् यो यः प॒रस्ता᳚त् । \newline
22. प॒रस्ता᳚द् ग्राम्यवा॒दी ग्रा᳚म्यवा॒दी प॒रस्ता᳚त् प॒रस्ता᳚द् ग्राम्यवा॒दी । \newline
23. ग्रा॒म्य॒वा॒दी स्याथ् स्याद् ग्रा᳚म्यवा॒दी ग्रा᳚म्यवा॒दी स्यात् । \newline
24. ग्रा॒म्य॒वा॒दीति॑ ग्राम्य - वा॒दी । \newline
25. स्यात् तस्य॒ तस्य॒ स्याथ् स्यात् तस्य॑ । \newline
26. तस्य॑ गृ॒हाद् गृ॒हात् तस्य॒ तस्य॑ गृ॒हात् । \newline
27. गृ॒हाद् व्री॒हीन् व्री॒हीन् गृ॒हाद् गृ॒हाद् व्री॒हीन् । \newline
28. व्री॒ही ना व्री॒हीन् व्री॒ही ना । \newline
29. आ ह॑रे द्धरे॒दा ह॑रेत् । \newline
30. ह॒रे॒च् छु॒क्लाञ् छु॒क्लान्. ह॑रे द्धरेच् छु॒क्लान् । \newline
31. शु॒क्लाꣳश्च॑ च शु॒क्लाञ् छु॒क्लाꣳश्च॑ । \newline
32. च॒ कृ॒ष्णान् कृ॒ष्णाꣳश्च॑ च कृ॒ष्णान् । \newline
33. कृ॒ष्णाꣳश्च॑ च कृ॒ष्णान् कृ॒ष्णाꣳश्च॑ । \newline
34. च॒ वि वि च॑ च॒ वि । \newline
35. वि चि॑नुयाच् चिनुया॒द् वि वि चि॑नुयात् । \newline
36. चि॒नु॒या॒द् ये ये चि॑नुयाच् चिनुया॒द् ये । \newline
37. ये शु॒क्लाः शु॒क्ला ये ये शु॒क्लाः । \newline
38. शु॒क्लाः स्युः स्युः शु॒क्लाः शु॒क्लाः स्युः । \newline
39. स्यु स्तम् तꣳ स्युः स्यु स्तम् । \newline
40. त मा॑दि॒त्य मा॑दि॒त्यम् तम् त मा॑दि॒त्यम् । \newline
41. आ॒दि॒त्यम् च॒रुम् च॒रु मा॑दि॒त्य मा॑दि॒त्यम् च॒रुम् । \newline
42. च॒रुम् निर् णिश्च॒रुम् च॒रुम् निः । \newline
43. निर् व॑पेद् वपे॒न् निर् णिर् व॑पेत् । \newline
44. व॒पे॒ दा॒दि॒त्या ऽऽदि॒त्या व॑पेद् वपे दादि॒त्या । \newline
45. आ॒दि॒त्या वै वा आ॑दि॒त्या ऽऽदि॒त्या वै । \newline
46. वै दे॒वत॑या दे॒वत॑या॒ वै वै दे॒वत॑या । \newline
47. दे॒वत॑या॒ विड् विड् दे॒वत॑या दे॒वत॑या॒ विट् । \newline
48. विड् विशं॒ ॅविशं॒ ॅविड् विड् विश᳚म् । \newline
49. विश॑ मे॒वैव विशं॒ ॅविश॑ मे॒व । \newline
50. ए॒वावा वै॒वैवाव॑ । \newline
51. अव॑ गच्छति गच्छ॒ त्यवाव॑ गच्छति । \newline
52. ग॒च्छ॒ त्यव॑ग॒ता ऽव॑गता गच्छति गच्छ॒ त्यव॑गता । \newline

\textbf{Ghana Paata } \newline

1. सु॒दा॒न॒व॒ ए॒नैना सु॑दानवः सुदानव ए॒ना वि॒श्पति॑ना वि॒श्पति॑नै॒ना सु॑दानवः सुदानव ए॒ना वि॒श्पति॑ना । \newline
2. सु॒दा॒न॒व॒ इति॑ सु - दा॒न॒वः॒ । \newline
3. ए॒ना वि॒श्पति॑ना वि॒श्पति॑नै॒नैना वि॒श्पति॑ना॒ ऽभ्य॑भि वि॒श्पति॑नै॒नैना वि॒श्पति॑ना॒ ऽभि । \newline
4. वि॒श्पति॑ना॒ ऽभ्य॑भि वि॒श्पति॑ना वि॒श्पति॑ना॒ ऽभ्य॑मु म॒मु म॒भि वि॒श्पति॑ना वि॒श्पति॑ना॒ ऽभ्य॑मुम् । \newline
5. अ॒भ्य॑मु म॒मु म॒भ्या᳚(1॒)भ्य॑मुꣳ राजा॑नꣳ॒॒ राजा॑न म॒मु म॒भ्या᳚(1॒)भ्य॑मुꣳ राजा॑नम् । \newline
6. अ॒मुꣳ राजा॑नꣳ॒॒ राजा॑न म॒मु म॒मुꣳ राजा॑न॒ मितीति॒ राजा॑न म॒मु म॒मुꣳ राजा॑न॒ मिति॑ । \newline
7. राजा॑न॒ मितीति॒ राजा॑नꣳ॒॒ राजा॑न॒ मित्या॑हा॒हे ति॒ राजा॑नꣳ॒॒ राजा॑न॒ मित्या॑ह । \newline
8. इत्या॑हा॒हे तीत्या॑ह मारु॒ती मा॑रु॒त्या॑हे तीत्या॑ह मारु॒ती । \newline
9. आ॒ह॒ मा॒रु॒ती मा॑रु॒त्या॑हाह मारु॒ती वै वै मा॑रु॒त्या॑हाह मारु॒ती वै । \newline
10. मा॒रु॒ती वै वै मा॑रु॒ती मा॑रु॒ती वै विड् विड् वै मा॑रु॒ती मा॑रु॒ती वै विट् । \newline
11. वै विड् विड् वै वै विड् ज्ये॒ष्ठो ज्ये॒ष्ठो विड् वै वै विड् ज्ये॒ष्ठः । \newline
12. विड् ज्ये॒ष्ठो ज्ये॒ष्ठो विड् विड् ज्ये॒ष्ठो वि॒श्पति॑र् वि॒श्पति॑र् ज्ये॒ष्ठो विड् विड् ज्ये॒ष्ठो वि॒श्पतिः॑ । \newline
13. ज्ये॒ष्ठो वि॒श्पति॑र् वि॒श्पति॑र् ज्ये॒ष्ठो ज्ये॒ष्ठो वि॒श्पति॑र् वि॒शा वि॒शा वि॒श्पति॑र् ज्ये॒ष्ठो ज्ये॒ष्ठो वि॒श्पति॑र् वि॒शा । \newline
14. वि॒श्पति॑र् वि॒शा वि॒शा वि॒श्पति॑र् वि॒श्पति॑र् वि॒शैवैव वि॒शा वि॒श्पति॑र् वि॒श्पति॑र् वि॒शैव । \newline
15. वि॒शैवैव वि॒शा वि॒शैवैन॑ मेन मे॒व वि॒शा वि॒शैवैन᳚म् । \newline
16. ए॒वैन॑ मेन मे॒वैवैनꣳ॑ रा॒ष्ट्रेण॑ रा॒ष्ट्रेणै॑न मे॒वैवैनꣳ॑ रा॒ष्ट्रेण॑ । \newline
17. ए॒नꣳ॒॒ रा॒ष्ट्रेण॑ रा॒ष्ट्रेणै॑न मेनꣳ रा॒ष्ट्रेण॒ सꣳ सꣳ रा॒ष्ट्रेणै॑न मेनꣳ रा॒ष्ट्रेण॒ सम् । \newline
18. रा॒ष्ट्रेण॒ सꣳ सꣳ रा॒ष्ट्रेण॑ रा॒ष्ट्रेण॒ स म॑र्द्धय त्यर्द्धयति॒ सꣳ रा॒ष्ट्रेण॑ रा॒ष्ट्रेण॒ स म॑र्द्धयति । \newline
19. स म॑र्द्धय त्यर्द्धयति॒ सꣳ स म॑र्द्धयति॒ यो यो᳚ ऽर्द्धयति॒ सꣳ स म॑र्द्धयति॒ यः । \newline
20. अ॒र्द्ध॒य॒ति॒ यो यो᳚ ऽर्द्धय त्यर्द्धयति॒ यः प॒रस्ता᳚त् प॒रस्ता॒द् यो᳚ ऽर्द्धय त्यर्द्धयति॒ यः प॒रस्ता᳚त् । \newline
21. यः प॒रस्ता᳚त् प॒रस्ता॒द् यो यः प॒रस्ता᳚द् ग्राम्यवा॒दी ग्रा᳚म्यवा॒दी प॒रस्ता॒द् यो यः प॒रस्ता᳚द् ग्राम्यवा॒दी । \newline
22. प॒रस्ता᳚द् ग्राम्यवा॒दी ग्रा᳚म्यवा॒दी प॒रस्ता᳚त् प॒रस्ता᳚द् ग्राम्यवा॒दी स्याथ् स्याद् ग्रा᳚म्यवा॒दी प॒रस्ता᳚त् प॒रस्ता᳚द् ग्राम्यवा॒दी स्यात् । \newline
23. ग्रा॒म्य॒वा॒दी स्याथ् स्याद् ग्रा᳚म्यवा॒दी ग्रा᳚म्यवा॒दी स्यात् तस्य॒ तस्य॒ स्याद् ग्रा᳚म्यवा॒दी ग्रा᳚म्यवा॒दी स्यात् तस्य॑ । \newline
24. ग्रा॒म्य॒वा॒दीति॑ ग्राम्य - वा॒दी । \newline
25. स्यात् तस्य॒ तस्य॒ स्याथ् स्यात् तस्य॑ गृ॒हाद् गृ॒हात् तस्य॒ स्याथ् स्यात् तस्य॑ गृ॒हात् । \newline
26. तस्य॑ गृ॒हाद् गृ॒हात् तस्य॒ तस्य॑ गृ॒हाद् व्री॒हीन् व्री॒हीन् गृ॒हात् तस्य॒ तस्य॑ गृ॒हाद् व्री॒हीन् । \newline
27. गृ॒हाद् व्री॒हीन् व्री॒हीन् गृ॒हाद् गृ॒हाद् व्री॒ही ना व्री॒हीन् गृ॒हाद् गृ॒हाद् व्री॒ही ना । \newline
28. व्री॒ही ना व्री॒हीन् व्री॒ही ना ह॑रेद्धरे॒दा व्री॒हीन् व्री॒ही ना ह॑रेत् । \newline
29. आ ह॑रेद्धरे॒दा ह॑रेच् छु॒क्लाञ् छु॒क्लान्. ह॑रे॒दा ह॑रेच् छु॒क्लान् । \newline
30. ह॒रे॒च्छु॒क्लाञ् छु॒क्लान्. ह॑रे द्धरेच् छु॒क्लाꣳश्च॑ च शु॒क्लान्. ह॑रे द्धरेच् छु॒क्लाꣳश्च॑ । \newline
31. शु॒क्लाꣳश्च॑ च शु॒क्लाञ् छु॒क्लाꣳश्च॑ कृ॒ष्णान् कृ॒ष्णाꣳश्च॑ शु॒क्लाञ् छु॒क्लाꣳश्च॑ कृ॒ष्णान् । \newline
32. च॒ कृ॒ष्णान् कृ॒ष्णाꣳश्च॑ च कृ॒ष्णाꣳश्च॑ च कृ॒ष्णाꣳश्च॑ च कृ॒ष्णाꣳश्च॑ । \newline
33. कृ॒ष्णाꣳश्च॑ च कृ॒ष्णान् कृ॒ष्णाꣳश्च॒ वि वि च॑ कृ॒ष्णान् कृ॒ष्णाꣳश्च॒ वि । \newline
34. च॒ वि वि च॑ च॒ वि चि॑नुयाच् चिनुया॒द् वि च॑ च॒ वि चि॑नुयात् । \newline
35. वि चि॑नुयाच् चिनुया॒द् वि वि चि॑नुया॒द् ये ये चि॑नुया॒द् वि वि चि॑नुया॒द् ये । \newline
36. चि॒नु॒या॒द् ये ये चि॑नुयाच् चिनुया॒द् ये शु॒क्लाः शु॒क्ला ये चि॑नुयाच् चिनुया॒द् ये शु॒क्लाः । \newline
37. ये शु॒क्लाः शु॒क्ला ये ये शु॒क्लाः स्युः स्युः शु॒क्ला ये ये शु॒क्लाः स्युः । \newline
38. शु॒क्लाः स्युः स्युः शु॒क्लाः शु॒क्लाः स्यु स्तम् तꣳ स्युः शु॒क्लाः शु॒क्लाः स्यु स्तम् । \newline
39. स्युस्तम् तꣳ स्युः स्युस्त मा॑दि॒त्य मा॑दि॒त्यम् तꣳ स्युः स्युस्त मा॑दि॒त्यम् । \newline
40. त मा॑दि॒त्य मा॑दि॒त्यम् तम् त मा॑दि॒त्यम् च॒रुम् च॒रु मा॑दि॒त्यम् तम् त मा॑दि॒त्यम् च॒रुम् । \newline
41. आ॒दि॒त्यम् च॒रुम् च॒रु मा॑दि॒त्य मा॑दि॒त्यम् च॒रुम् निर् णिश्च॒रु मा॑दि॒त्य मा॑दि॒त्यम् च॒रुम् निः । \newline
42. च॒रुम् निर् णिश्च॒रुम् च॒रुम् निर् व॑पेद् वपे॒न् निश्च॒रुम् च॒रुम् निर् व॑पेत् । \newline
43. निर् व॑पेद् वपे॒न् निर् णिर् व॑पे दादि॒त्या ऽऽदि॒त्या व॑पे॒न् निर् णिर् व॑पे दादि॒त्या । \newline
44. व॒पे॒ दा॒दि॒त्या ऽऽदि॒त्या व॑पेद् वपे दादि॒त्या वै वा आ॑दि॒त्या व॑पेद् वपे दादि॒त्या वै । \newline
45. आ॒दि॒त्या वै वा आ॑दि॒त्या ऽऽदि॒त्या वै दे॒वत॑या दे॒वत॑या॒ वा आ॑दि॒त्या ऽऽदि॒त्या वै दे॒वत॑या । \newline
46. वै दे॒वत॑या दे॒वत॑या॒ वै वै दे॒वत॑या॒ विड् विड् दे॒वत॑या॒ वै वै दे॒वत॑या॒ विट् । \newline
47. दे॒वत॑या॒ विड् विड् दे॒वत॑या दे॒वत॑या॒ विड् विशं॒ ॅविशं॒ ॅविड् दे॒वत॑या दे॒वत॑या॒ विड् विश᳚म् । \newline
48. विड् विशं॒ ॅविशं॒ ॅविड् विड् विश॑ मे॒वैव विशं॒ ॅविड् विड् विश॑ मे॒व । \newline
49. विश॑ मे॒वैव विशं॒ ॅविश॑ मे॒वावावै॒व विशं॒ ॅविश॑ मे॒वाव॑ । \newline
50. ए॒वावा वै॒वैवाव॑ गच्छति गच्छ॒ त्यवै॒वैवाव॑ गच्छति । \newline
51. अव॑ गच्छति गच्छ॒ त्यवाव॑ गच्छ॒ त्यव॑ग॒ता ऽव॑गता गच्छ॒ त्यवाव॑ गच्छ॒ त्यव॑गता । \newline
52. ग॒च्छ॒ त्यव॑ग॒ता ऽव॑गता गच्छति गच्छ॒ त्यव॑गता ऽस्या॒स्या व॑गता गच्छति गच्छ॒ त्यव॑गता ऽस्य । \newline
\pagebreak
\markright{ TS 2.3.1.4  \hfill https://www.vedavms.in \hfill}

\section{ TS 2.3.1.4 }

\textbf{TS 2.3.1.4 } \newline
\textbf{Samhita Paata} \newline

-त्यव॑गताऽस्य॒ विडन॑वगतꣳ रा॒ष्ट्रमित्या॑हु॒र्ये कृ॒ष्णाः स्युस्तं ॅवा॑रु॒णं च॒रुं निर्व॑पेद्वारु॒णं ॅवै रा॒ष्ट्रमु॒भे ए॒व विशं॑ च रा॒ष्ट्रं चाव॑ गच्छति॒ यदि॒ नाव॒गच्छे॑दि॒म-म॒हमा॑दि॒त्येभ्यो॑ भा॒गं निर्व॑पा॒म्या ऽमुष्मा॑द॒मुष्यै॑ वि॒शोऽव॑गन्तो॒रिति॒ निर्व॑पेदादि॒त्या ए॒वैनं॑ भाग॒धेयं॑ प्रे॒फ्सन्तो॒ विश॒मव॑ - [  ] \newline

\textbf{Pada Paata} \newline

अव॑ग॒तेत्यव॑ - ग॒ता॒ । अ॒स्य॒ । विट् । अन॑वगत॒मित्यन॑व - ग॒त॒म् । रा॒ष्ट्रम् । इति॑ । आ॒हुः॒ । ये । कृ॒ष्णाः । स्युः । तम् । वा॒रु॒णम् । च॒रुम् । निरिति॑ । व॒पे॒त् । वा॒रु॒णम् । वै । रा॒ष्ट्रम् । उ॒भे इति॑ । ए॒व । विश᳚म् । च॒ । रा॒ष्ट्रम् । च॒ । अवेति॑ । ग॒च्छ॒ति॒ । यदि॑ । न । अ॒व॒गच्छे॒दित्य॑व-गच्छे᳚त् । इ॒मम् । अ॒हम् । आ॒दि॒त्येभ्यः॑ । भा॒गम् । निरिति॑ । व॒पा॒मि॒ । एति॑ । अ॒मुष्मा᳚त् । अ॒मुष्यै᳚ । वि॒शः । अव॑गन्तो॒रित्यव॑-ग॒न्तोः॒ । इति॑ । निरिति॑ । व॒पे॒त् । आ॒दि॒त्याः । ए॒व । ए॒न॒म् । भा॒ग॒धेय॒मिति॑ भाग - धेय᳚म् । प्रे॒फ्सन्त॒ इति॑ प्र-ई॒फ्सन्तः॑ । विश᳚म् । अवेति॑ ।  \newline


\textbf{Krama Paata} \newline

अव॑गता ऽस्य । अव॑ग॒तेत्यव॑ - ग॒ता॒ । अ॒स्य॒ विट् । विडन॑वगतम् । अन॑वगतꣳ रा॒ष्ट्रम् । अन॑वगत॒मित्यन॑व - ग॒त॒म् । रा॒ष्ट्रमिति॑ । इत्या॑हुः । आ॒हु॒र् ये । ये कृ॒ष्णाः । कृ॒ष्णाः स्युः । स्यु स्तम् । तं ॅवा॑रु॒णम् । वा॒रु॒णम् च॒रुम् । च॒रुम् निः । निर् व॑पेत् । व॒पे॒द् वा॒रु॒णम् । वा॒रु॒णं ॅवै । वै रा॒ष्ट्रम् । रा॒ष्ट्रमु॒भे । उ॒भे ए॒व । उ॒भे इत्यु॒भे । ए॒व विश᳚म् । विश॑म् च । च॒ रा॒ष्ट्रम् । रा॒ष्ट्रम् च॑ । चाव॑ । अव॑ गच्छति । ग॒च्छ॒ति॒ यदि॑ । यदि॒ न । नाव॒गच्छे᳚त् । अ॒व॒गच्छे॑दि॒मम् । अ॒व॒गच्छे॒दित्य॑व - गच्छे᳚त् । इ॒मम॒हम् । अ॒हमा॑दि॒त्येभ्यः॑ । आ॒दि॒त्येभ्यो॑ भा॒गम् । भा॒गम् निः । निर् व॑पामि । व॒पा॒म्या । आऽमुष्मा᳚त् । अ॒मुष्मा॑द॒मुष्यै᳚ । अ॒मुष्यै॑ वि॒शः । वि॒शो ऽव॑गन्तोः । अव॑गन्तो॒रिति॑ । अव॑गन्तो॒रित्यव॑ - ग॒न्तोः॒ । इति॒ निः । निर् व॑पेत् । व॒पे॒दा॒दि॒त्याः । आ॒दि॒त्या ए॒व । ए॒वैन᳚म् । ए॒न॒म् भा॒ग॒धेय᳚म् । भा॒ग॒धेय॑म् प्रे॒फ्सन्तः॑ । भा॒ग॒धेय॒मिति॑ भाग - धेय᳚म् । प्रे॒फ्सन्तो॒ विश᳚म् । प्रे॒फ्सन्त॒ इति॑ प्र - ई॒फ्सन्तः॑ । विश॒मव॑ । अव॑ गमयन्ति \newline

\textbf{Jatai Paata} \newline

1. अव॑गता ऽस्या॒स्या व॑ग॒ता ऽव॑गता ऽस्य । \newline
2. अव॑ग॒तेत्यव॑ - ग॒ता॒ । \newline
3. अ॒स्य॒ विड् विड॑स्यास्य॒ विट् । \newline
4. विडन॑वगत॒ मन॑वगतं॒ ॅविड् विडन॑वगतम् । \newline
5. अन॑वगतꣳ रा॒ष्ट्रꣳ रा॒ष्ट्र मन॑वगत॒ मन॑वगतꣳ रा॒ष्ट्रम् । \newline
6. अन॑वगत॒मित्यन॑व - ग॒त॒म् । \newline
7. रा॒ष्ट्र मितीति॑ रा॒ष्ट्रꣳ रा॒ष्ट्र मिति॑ । \newline
8. इत्या॑हु राहु॒ रिती त्या॑हुः । \newline
9. आ॒हु॒र् ये य आ॑हु राहु॒र् ये । \newline
10. ये कृ॒ष्णाः कृ॒ष्णा ये ये कृ॒ष्णाः । \newline
11. कृ॒ष्णाः स्युः स्युः कृ॒ष्णाः कृ॒ष्णाः स्युः । \newline
12. स्यु स्तम् तꣳ स्युः स्यु स्तम् । \newline
13. तं ॅवा॑रु॒णं ॅवा॑रु॒णम् तम् तं ॅवा॑रु॒णम् । \newline
14. वा॒रु॒णम् च॒रुम् च॒रुं ॅवा॑रु॒णं ॅवा॑रु॒णम् च॒रुम् । \newline
15. च॒रुम् निर् णिश्च॒रुम् च॒रुम् निः । \newline
16. निर् व॑पेद् वपे॒न् निर् णिर् व॑पेत् । \newline
17. व॒पे॒द् वा॒रु॒णं ॅवा॑रु॒णं ॅव॑पेद् वपेद् वारु॒णम् । \newline
18. वा॒रु॒णं ॅवै वै वा॑रु॒णं ॅवा॑रु॒णं ॅवै । \newline
19. वै रा॒ष्ट्रꣳ रा॒ष्ट्रं ॅवै वै रा॒ष्ट्रम् । \newline
20. रा॒ष्ट्र मु॒भे उ॒भे रा॒ष्ट्रꣳ रा॒ष्ट्र मु॒भे । \newline
21. उ॒भे ए॒वैवोभे उ॒भे ए॒व । \newline
22. उ॒भे इत्यु॒भे । \newline
23. ए॒व विशं॒ ॅविश॑ मे॒वैव विश᳚म् । \newline
24. विश॑म् च च॒ विशं॒ ॅविश॑म् च । \newline
25. च॒ रा॒ष्ट्रꣳ रा॒ष्ट्रम् च॑ च रा॒ष्ट्रम् । \newline
26. रा॒ष्ट्रम् च॑ च रा॒ष्ट्रꣳ रा॒ष्ट्रम् च॑ । \newline
27. चावाव॑ च॒ चाव॑ । \newline
28. अव॑ गच्छति गच्छ॒ त्यवाव॑ गच्छति । \newline
29. ग॒च्छ॒ति॒ यदि॒ यदि॑ गच्छति गच्छति॒ यदि॑ । \newline
30. यदि॒ न न यदि॒ यदि॒ न । \newline
31. नाव॒गच्छे॑ दव॒गच्छे॒न् न नाव॒गच्छे᳚त् । \newline
32. अ॒व॒गच्छे॑ दि॒म मि॒म म॑व॒गच्छे॑ दव॒गच्छे॑ दि॒मम् । \newline
33. अ॒व॒गच्छे॒दित्य॑व - गच्छे᳚त् । \newline
34. इ॒म म॒ह म॒ह मि॒म मि॒म म॒हम् । \newline
35. अ॒ह मा॑दि॒त्येभ्य॑ आदि॒त्येभ्यो॒ ऽह म॒ह मा॑दि॒त्येभ्यः॑ । \newline
36. आ॒दि॒त्येभ्यो॑ भा॒गम् भा॒ग मा॑दि॒त्येभ्य॑ आदि॒त्येभ्यो॑ भा॒गम् । \newline
37. भा॒गम् निर् णिर् भा॒गम् भा॒गम् निः । \newline
38. निर् व॑पामि वपामि॒ निर् णिर् व॑पामि । \newline
39. व॒पा॒म्या व॑पामि वपा॒म्या । \newline
40. आ ऽमुष्मा॑ द॒मुष्मा॒दा ऽमुष्मा᳚त् । \newline
41. अ॒मुष्मा॑ द॒मुष्या॑ अ॒मुष्या॑ अ॒मुष्मा॑ द॒मुष्मा॑ द॒मुष्यै᳚ । \newline
42. अ॒मुष्यै॑ वि॒शो वि॒शो॑ ऽमुष्या॑ अ॒मुष्यै॑ वि॒शः । \newline
43. वि॒शो ऽव॑गन्तो॒ रव॑गन्तोर् वि॒शो वि॒शो ऽव॑गन्तोः । \newline
44. अव॑गन्तो॒ रिती त्यव॑गन्तो॒ रव॑गन्तो॒ रिति॑ । \newline
45. अव॑गन्तो॒रित्यव॑ - ग॒न्तोः॒ । \newline
46. इति॒ निर् णि रितीति॒ निः । \newline
47. निर् व॑पेद् वपे॒न् निर् णिर् व॑पेत् । \newline
48. व॒पे॒ दा॒दि॒त्या आ॑दि॒त्या व॑पेद् वपे दादि॒त्याः । \newline
49. आ॒दि॒त्या ए॒वै वादि॒त्या आ॑दि॒त्या ए॒व । \newline
50. ए॒वैन॑ मेन मे॒वैवैन᳚म् । \newline
51. ए॒न॒म् भा॒ग॒धेय॑म् भाग॒धेय॑ मेन मेनम् भाग॒धेय᳚म् । \newline
52. भा॒ग॒धेय॑म् प्रे॒फ्सन्तः॑ प्रे॒फ्सन्तो॑ भाग॒धेय॑म् भाग॒धेय॑म् प्रे॒फ्सन्तः॑ । \newline
53. भा॒ग॒धेय॒मिति॑ भाग - धेय᳚म् । \newline
54. प्रे॒फ्सन्तो॒ विशं॒ ॅविश॑म् प्रे॒फ्सन्तः॑ प्रे॒फ्सन्तो॒ विश᳚म् । \newline
55. प्रे॒फ्सन्त॒ इति॑ प्र - ई॒फ्सन्तः॑ । \newline
56. विश॒ मवाव॒ विशं॒ ॅविश॒ मव॑ । \newline
57. अव॑ गमयन्ति गमय॒ न्त्यवाव॑ गमयन्ति । \newline

\textbf{Ghana Paata } \newline

1. अव॑गता ऽस्या॒स्या व॑ग॒ता ऽव॑गता ऽस्य॒ विड् विड॒स्या व॑ग॒ता ऽव॑गता ऽस्य॒ विट् । \newline
2. अव॑ग॒तेत्यव॑ - ग॒ता॒ । \newline
3. अ॒स्य॒ विड् विड॑स्यास्य॒ विडन॑वगत॒ मन॑वगतं॒ ॅविड॑स्यास्य॒ विडन॑वगतम् । \newline
4. विडन॑वगत॒ मन॑वगतं॒ ॅविड् विडन॑वगतꣳ रा॒ष्ट्रꣳ रा॒ष्ट्र मन॑वगतं॒ ॅविड् विडन॑वगतꣳ रा॒ष्ट्रम् । \newline
5. अन॑वगतꣳ रा॒ष्ट्रꣳ रा॒ष्ट्र मन॑वगत॒ मन॑वगतꣳ रा॒ष्ट्र मितीति॑ रा॒ष्ट्र मन॑वगत॒ मन॑वगतꣳ रा॒ष्ट्र मिति॑ । \newline
6. अन॑वगत॒मित्यन॑व - ग॒त॒म् । \newline
7. रा॒ष्ट्र मितीति॑ रा॒ष्ट्रꣳ रा॒ष्ट्र मित्या॑हु राहु॒रिति॑ रा॒ष्ट्रꣳ रा॒ष्ट्र मित्या॑हुः । \newline
8. इत्या॑हु राहु॒रिती त्या॑हु॒र् ये य आ॑हु॒रिती त्या॑हु॒र् ये । \newline
9. आ॒हु॒र् ये य आ॑हु राहु॒र् ये कृ॒ष्णाः कृ॒ष्णा य आ॑हु राहु॒र् ये कृ॒ष्णाः । \newline
10. ये कृ॒ष्णाः कृ॒ष्णा ये ये कृ॒ष्णाः स्युः स्युः कृ॒ष्णा ये ये कृ॒ष्णाः स्युः । \newline
11. कृ॒ष्णाः स्युः स्युः कृ॒ष्णाः कृ॒ष्णाः स्यु स्तम् तꣳ स्युः कृ॒ष्णाः कृ॒ष्णाः स्यु स्तम् । \newline
12. स्यु स्तम् तꣳ स्युः स्यु स्तं ॅवा॑रु॒णं ॅवा॑रु॒णम् तꣳ स्युः स्यु स्तं ॅवा॑रु॒णम् । \newline
13. तं ॅवा॑रु॒णं ॅवा॑रु॒णम् तम् तं ॅवा॑रु॒णम् च॒रुम् च॒रुं ॅवा॑रु॒णम् तम् तं ॅवा॑रु॒णम् च॒रुम् । \newline
14. वा॒रु॒णम् च॒रुम् च॒रुं ॅवा॑रु॒णं ॅवा॑रु॒णम् च॒रुम् निर् णिश्च॒रुं ॅवा॑रु॒णं ॅवा॑रु॒णम् च॒रुम् निः । \newline
15. च॒रुम् निर् णिश्च॒रुम् च॒रुम् निर् व॑पेद् वपे॒न् निश्च॒रुम् च॒रुम् निर् व॑पेत् । \newline
16. निर् व॑पेद् वपे॒न् निर् णिर् व॑पेद् वारु॒णं ॅवा॑रु॒णं ॅव॑पे॒न् निर् णिर् व॑पेद् वारु॒णम् । \newline
17. व॒पे॒द् वा॒रु॒णं ॅवा॑रु॒णं ॅव॑पेद् वपेद् वारु॒णं ॅवै वै वा॑रु॒णं ॅव॑पेद् वपेद् वारु॒णं ॅवै । \newline
18. वा॒रु॒णं ॅवै वै वा॑रु॒णं ॅवा॑रु॒णं ॅवै रा॒ष्ट्रꣳ रा॒ष्ट्रं ॅवै वा॑रु॒णं ॅवा॑रु॒णं ॅवै रा॒ष्ट्रम् । \newline
19. वै रा॒ष्ट्रꣳ रा॒ष्ट्रं ॅवै वै रा॒ष्ट्र मु॒भे उ॒भे रा॒ष्ट्रं ॅवै वै रा॒ष्ट्र मु॒भे । \newline
20. रा॒ष्ट्र मु॒भे उ॒भे रा॒ष्ट्रꣳ रा॒ष्ट्र मु॒भे ए॒वैवोभे रा॒ष्ट्रꣳ रा॒ष्ट्र मु॒भे ए॒व । \newline
21. उ॒भे ए॒वैवोभे उ॒भे ए॒व विशं॒ ॅविश॑ मे॒वोभे उ॒भे ए॒व विश᳚म् । \newline
22. उ॒भे इत्यु॒भे । \newline
23. ए॒व विशं॒ ॅविश॑ मे॒वैव विश॑म् च च॒ विश॑ मे॒वैव विश॑म् च । \newline
24. विश॑म् च च॒ विशं॒ ॅविश॑म् च रा॒ष्ट्रꣳ रा॒ष्ट्रम् च॒ विशं॒ ॅविश॑म् च रा॒ष्ट्रम् । \newline
25. च॒ रा॒ष्ट्रꣳ रा॒ष्ट्रम् च॑ च रा॒ष्ट्रम् च॑ च रा॒ष्ट्रम् च॑ च रा॒ष्ट्रम् च॑ । \newline
26. रा॒ष्ट्रम् च॑ च रा॒ष्ट्रꣳ रा॒ष्ट्रम् चावाव॑ च रा॒ष्ट्रꣳ रा॒ष्ट्रम् चाव॑ । \newline
27. चावाव॑ च॒ चाव॑ गच्छति गच्छ॒त्यव॑ च॒ चाव॑ गच्छति । \newline
28. अव॑ गच्छति गच्छ॒ त्यवाव॑ गच्छति॒ यदि॒ यदि॑ गच्छ॒ त्यवाव॑ गच्छति॒ यदि॑ । \newline
29. ग॒च्छ॒ति॒ यदि॒ यदि॑ गच्छति गच्छति॒ यदि॒ न न यदि॑ गच्छति गच्छति॒ यदि॒ न । \newline
30. यदि॒ न न यदि॒ यदि॒ नाव॒गच्छे॑ दव॒गच्छे॒न् न यदि॒ यदि॒ नाव॒गच्छे᳚त् । \newline
31. नाव॒गच्छे॑ दव॒गच्छे॒न् न नाव॒गच्छे॑ दि॒म मि॒म म॑व॒गच्छे॒न् न नाव॒गच्छे॑ दि॒मम् । \newline
32. अ॒व॒गच्छे॑ दि॒म मि॒म म॑व॒गच्छे॑ दव॒गच्छे॑ दि॒म म॒ह म॒ह मि॒म म॑व॒गच्छे॑ दव॒गच्छे॑ दि॒म म॒हम् । \newline
33. अ॒व॒गच्छे॒दित्य॑व - गच्छे᳚त् । \newline
34. इ॒म म॒ह म॒ह मि॒म मि॒म म॒ह मा॑दि॒त्येभ्य॑ आदि॒त्येभ्यो॒ ऽह मि॒म मि॒म म॒ह मा॑दि॒त्येभ्यः॑ । \newline
35. अ॒ह मा॑दि॒त्येभ्य॑ आदि॒त्येभ्यो॒ ऽह म॒ह मा॑दि॒त्येभ्यो॑ भा॒गम् भा॒ग मा॑दि॒त्येभ्यो॒ ऽह म॒ह मा॑दि॒त्येभ्यो॑ भा॒गम् । \newline
36. आ॒दि॒त्येभ्यो॑ भा॒गम् भा॒ग मा॑दि॒त्येभ्य॑ आदि॒त्येभ्यो॑ भा॒गम् निर् णिर् भा॒ग मा॑दि॒त्येभ्य॑ आदि॒त्येभ्यो॑ भा॒गम् निः । \newline
37. भा॒गम् निर् णिर् भा॒गम् भा॒गम् निर् व॑पामि वपामि॒ निर् भा॒गम् भा॒गम् निर् व॑पामि । \newline
38. निर् व॑पामि वपामि॒ निर् णिर् व॑पा॒म्या व॑पामि॒ निर् णिर् व॑पा॒म्या । \newline
39. व॒पा॒म्या व॑पामि वपा॒म्या ऽमुष्मा॑ द॒मुष्मा॒दा व॑पामि वपा॒म्या ऽमुष्मा᳚त् । \newline
40. आ ऽमुष्मा॑ द॒मुष्मा॒दा ऽमुष्मा॑ द॒मुष्या॑ अ॒मुष्या॑ अ॒मुष्मा॒दा ऽमुष्मा॑ द॒मुष्यै᳚ । \newline
41. अ॒मुष्मा॑ द॒मुष्या॑ अ॒मुष्या॑ अ॒मुष्मा॑ द॒मुष्मा॑ द॒मुष्यै॑ वि॒शो वि॒शो॑ ऽमुष्या॑ अ॒मुष्मा॑ द॒मुष्मा॑ द॒मुष्यै॑ वि॒शः । \newline
42. अ॒मुष्यै॑ वि॒शो वि॒शो॑ ऽमुष्या॑ अ॒मुष्यै॑ वि॒शो ऽव॑गन्तो॒ रव॑गन्तोर् वि॒शो॑ ऽमुष्या॑ अ॒मुष्यै॑ वि॒शो ऽव॑गन्तोः । \newline
43. वि॒शो ऽव॑गन्तो॒ रव॑गन्तोर् वि॒शो वि॒शो ऽव॑गन्तो॒ रिती त्यव॑गन्तोर् वि॒शो वि॒शो ऽव॑गन्तो॒रिति॑ । \newline
44. अव॑गन्तो॒ रिती त्यव॑गन्तो॒ रव॑गन्तो॒ रिति॒ निर् णि रित्यव॑गन्तो॒ रव॑गन्तो॒ रिति॒ निः । \newline
45. अव॑गन्तो॒रित्यव॑ - ग॒न्तोः॒ । \newline
46. इति॒ निर् णिरितीति॒ निर् व॑पेद् वपे॒न् निरितीति॒ निर् व॑पेत् । \newline
47. निर् व॑पेद् वपे॒न् निर् णिर् व॑पे दादि॒त्या आ॑दि॒त्या व॑पे॒न् निर् णिर् व॑पे दादि॒त्याः । \newline
48. व॒पे॒ दा॒दि॒त्या आ॑दि॒त्या व॑पेद् वपे दादि॒त्या ए॒वैवादि॒त्या व॑पेद् वपे दादि॒त्या ए॒व । \newline
49. आ॒दि॒त्या ए॒वैवादि॒त्या आ॑दि॒त्या ए॒वैन॑ मेन मे॒वादि॒त्या आ॑दि॒त्या ए॒वैन᳚म् । \newline
50. ए॒वैन॑ मेन मे॒वैवैन॑म् भाग॒धेय॑म् भाग॒धेय॑ मेन मे॒वैवैन॑म् भाग॒धेय᳚म् । \newline
51. ए॒न॒म् भा॒ग॒धेय॑म् भाग॒धेय॑ मेन मेनम् भाग॒धेय॑म् प्रे॒फ्सन्तः॑ प्रे॒फ्सन्तो॑ भाग॒धेय॑ मेन मेनम् भाग॒धेय॑म् प्रे॒फ्सन्तः॑ । \newline
52. भा॒ग॒धेय॑म् प्रे॒फ्सन्तः॑ प्रे॒फ्सन्तो॑ भाग॒धेय॑म् भाग॒धेय॑म् प्रे॒फ्सन्तो॒ विशं॒ ॅविश॑म् प्रे॒फ्सन्तो॑ भाग॒धेय॑म् भाग॒धेय॑म् प्रे॒फ्सन्तो॒ विश᳚म् । \newline
53. भा॒ग॒धेय॒मिति॑ भाग - धेय᳚म् । \newline
54. प्रे॒फ्सन्तो॒ विशं॒ ॅविश॑म् प्रे॒फ्सन्तः॑ प्रे॒फ्सन्तो॒ विश॒ मवाव॒ विश॑म् प्रे॒फ्सन्तः॑ प्रे॒फ्सन्तो॒ विश॒ मव॑ । \newline
55. प्रे॒फ्सन्त॒ इति॑ प्र - ई॒फ्सन्तः॑ । \newline
56. विश॒ मवाव॒ विशं॒ ॅविश॒ मव॑ गमयन्ति गमय॒न्त्यव॒ विशं॒ ॅविश॒ मव॑ गमयन्ति । \newline
57. अव॑ गमयन्ति गमय॒ न्त्यवाव॑ गमयन्ति॒ यदि॒ यदि॑ गमय॒ न्त्यवाव॑ गमयन्ति॒ यदि॑ । \newline
\pagebreak
\markright{ TS 2.3.1.5  \hfill https://www.vedavms.in \hfill}

\section{ TS 2.3.1.5 }

\textbf{TS 2.3.1.5 } \newline
\textbf{Samhita Paata} \newline

गमयन्ति॒ यदि॒ नाव॒गच्छे॒दाश्व॑त्थान् म॒यूखा᳚न्थ् स॒प्त म॑द्ध्यमे॒षाया॒मुप॑- हन्यादि॒दम॒ह-मा॑दि॒त्यान् ब॑ध्ना॒म्या ऽमुष्मा॑द॒मुष्यै॑ वि॒शोऽव॑गन्तो॒रित्या॑दि॒त्या ए॒वैनं॑ ब॒द्धवी॑रा॒ विश॒मव॑ गमयन्ति॒ यदि॒ नाव॒गच्छे॑दे॒तमे॒वाऽऽदि॒त्यं च॒रुं निर्व॑पेदि॒द्ध्मेऽपि॑ म॒यूखा॒न्थ् सं न॑ह्येदनपरु॒द्ध्यमे॒वाव॑ गच्छ॒त्याश्व॑त्था भवन्तिम॒रुतां॒ ॅवा ए॒त ( ) -दोजो॒ यद॑श्व॒त्थ ओज॑सै॒व विश॒मव॑ गच्छति स॒प्त भ॑वन्ति स॒प्त ग॑णा॒ वै म॒रुतो॑ गण॒श ए॒व विश॒मव॑ गच्छति । \newline

\textbf{Pada Paata} \newline

ग॒म॒य॒न्ति॒ । यदि॑ । न । अ॒व॒गच्छे॒दित्य॑व - गच्छे᳚त् । आश्व॑त्थान् । म॒यूखान्॑ । स॒प्त । म॒द्ध्य॒मे॒षाया॒मिति॑ म॒द्ध्यम - ई॒षाया᳚म् । उपेति॑ । ह॒न्या॒त् । इ॒दम् । अ॒हम् । आ॒दि॒त्यान् । ब॒ध्ना॒मि॒ । एति॑ । अ॒मुष्मा᳚त् । अ॒मुष्यै᳚ । वि॒शः । अव॑गन्तो॒रित्यव॑-ग॒न्तोः॒ । इति॑ । आ॒दि॒त्याः । ए॒व । ए॒न॒म् । ब॒द्धवी॑रा॒ इति॑ ब॒द्ध -वी॒राः॒ । विश᳚म् । अवेति॑ । ग॒म॒य॒न्ति॒ । यदि॑ । न । अ॒व॒गच्छे॒दित्य॑व - गच्छे᳚त् । ए॒तम् । ए॒व । आ॒दि॒त्यम् । च॒रुम् । निरिति॑ । व॒पे॒त् । इ॒द्ध्मे । अपीति॑ । म॒यूखान्॑ । समिति॑ । न॒ह्ये॒त् । अ॒न॒प॒रु॒द्ध्यमित्य॑नप - रु॒द्ध्यम् । ए॒व । अवेति॑ । ग॒च्छ॒ति॒ । आश्व॑त्थाः । भ॒व॒न्ति॒ । म॒रुता᳚म् । वै । ए॒तत् ( ) । ओजः॑ । यत् । अ॒श्व॒त्थः । ओज॑सा । ए॒व । विश᳚म् । अवेति॑ । ग॒च्छ॒ति॒ । स॒प्त । भ॒व॒न्ति॒ । स॒प्तग॑णा॒ इति॑ स॒प्त - ग॒णाः॒ । वै । म॒रुतः॑ । ग॒ण॒श - इति॑ गण - शः । ए॒व । विश᳚म् । अवेति॑ । ग॒च्छ॒ति॒ ॥  \newline


\textbf{Krama Paata} \newline

ग॒म॒य॒न्ति॒ यदि॑ । यदि॒ न । नाव॒गच्छे᳚त् । अ॒व॒गच्छे॒दाश्व॑त्थान् । अ॒व॒गच्छे॒दित्य॑व - गच्छे᳚त् । आश्व॑त्थान् म॒यूखान्॑ । म॒यूखा᳚न्थ् स॒प्त । स॒प्त म॑द्ध्यमे॒षाया᳚म् । म॒द्ध्य॒मे॒षाया॒मुप॑ । म॒द्ध्य॒मे॒षाया॒मिति॑ मद्ध्यम - ई॒षाया᳚म् । उप॑ हन्यात् । ह॒न्या॒दि॒दम् । इ॒दम॒हम् । अ॒हमा॑दि॒त्यान् । आ॒दि॒त्यान् ब॑द्ध्नामि । ब॒द्ध्ना॒म्या । आऽमुष्मा᳚त् । अ॒मुष्मा॑द॒मुष्यै᳚ । अ॒मुष्यै॑ वि॒शः । वि॒शोऽव॑गन्तोः । अव॑गन्तो॒रिति॑ । अव॑गन्तो॒रित्यव॑ - ग॒न्तोः॒ । इत्या॑दि॒त्याः । आ॒दि॒त्या ए॒व । ए॒वैन᳚म् । ए॒न॒म् ब॒द्धवी॑राः । ब॒द्धवी॑रा॒ विश᳚म् । ब॒द्धवी॑रा॒ इति॑ ब॒द्ध - वी॒राः॒ । विश॒मव॑ । अव॑ गमयन्ति । ग॒म॒य॒न्ति॒ यदि॑ । यदि॒ न । नाव॒गच्छे᳚त् । अ॒व॒गच्छे॑दे॒तम् । अ॒व॒गच्छे॒दित्य॑व - गच्छे᳚त् । ए॒तमे॒व । ए॒वादि॒त्यम् । आ॒दि॒त्यं च॒रुम् । च॒रुम् निः । निर् व॑पेत् । व॒पे॒दि॒द्ध्मे । इ॒द्ध्मे ऽपि॑ । अपि॑ म॒यूखान्॑ । म॒यूखा॒न्थ् सम् । सम् न॑ह्येत् । न॒ह्ये॒द॒न॒प॒रु॒द्ध्यम् । अ॒न॒प॒रु॒द्ध्यमे॒व । अ॒न॒प॒रु॒द्ध्यमित्य॑नप - रु॒द्ध्यम् । ए॒वाव॑ । अव॑ गच्छति । ग॒च्छ॒त्याश्व॑त्थाः । आश्व॑त्था भवन्ति । भ॒व॒न्ति॒ म॒रुता᳚म् । म॒रुतां॒ ॅवै । वा ए॒तत् । ए॒तदोजः॑ ( ) । ओजो॒ यत् । यद॑श्व॒त्थः । अ॒श्व॒त्थ ओज॑सा । ओज॑सै॒व । ए॒व विश᳚म् । विश॒मव॑ । अव॑ गच्छति । ग॒च्छ॒ति॒ स॒प्त । स॒प्त भ॑वन्ति । भ॒व॒न्ति॒ स॒प्तग॑णाः । स॒प्तग॑णा॒ वै । स॒प्तग॑णा॒ इति॑ स॒प्त - ग॒णाः॒ । वै म॒रुतः॑ । म॒रुतो॑ गण॒शः । ग॒ण॒श ए॒व । ग॒ण॒श इति॑ गण - शः । ए॒व विश᳚म् । विश॒ मव॑ । अव॑ गच्छति । ग॒च्छ॒तीति॑ गच्छति । \newline

\textbf{Jatai Paata} \newline

1. ग॒म॒य॒न्ति॒ यदि॒ यदि॑ गमयन्ति गमयन्ति॒ यदि॑ । \newline
2. यदि॒ न न यदि॒ यदि॒ न । \newline
3. नाव॒गच्छे॑ दव॒गच्छे॒न् न नाव॒गच्छे᳚त् । \newline
4. अ॒व॒गच्छे॒ दाश्व॑त्था॒ नाश्व॑त्था नव॒गच्छे॑ दव॒गच्छे॒ दाश्व॑त्थान् । \newline
5. अ॒व॒गच्छे॒दित्य॑व - गच्छे᳚त् । \newline
6. आश्व॑त्थान् म॒यूखा᳚न् म॒यूखा॒ नाश्व॑त्था॒ नाश्व॑त्थान् म॒यूखान्॑ । \newline
7. म॒यूखा᳚न् थ्स॒प्त स॒प्त म॒यूखा᳚न् म॒यूखा᳚न् थ्स॒प्त । \newline
8. स॒प्त म॑द्ध्यमे॒षाया᳚म् मद्ध्यमे॒षायाꣳ॑ स॒प्त स॒प्त म॑द्ध्यमे॒षाया᳚म् । \newline
9. म॒द्ध्य॒मे॒षाया॒ मुपोप॑ मद्ध्यमे॒षाया᳚म् मद्ध्यमे॒षाया॒ मुप॑ । \newline
10. म॒द्ध्य॒मे॒षाया॒मिति॑ म॒द्ध्यम - ई॒षाया᳚म् । \newline
11. उप॑ हन्या द्धन्या॒ दुपोप॑ हन्यात् । \newline
12. ह॒न्या॒ दि॒द मि॒दꣳ ह॑न्या द्धन्या दि॒दम् । \newline
13. इ॒द म॒ह म॒ह मि॒द मि॒द म॒हम् । \newline
14. अ॒ह मा॑दि॒त्या ना॑दि॒त्या न॒ह म॒ह मा॑दि॒त्यान् । \newline
15. आ॒दि॒त्यान् ब॑ध्नामि बध्ना म्यादि॒त्या ना॑दि॒त्यान् ब॑ध्नामि । \newline
16. ब॒ध्ना॒म्या ब॑ध्नामि बध्ना॒म्या । \newline
17. आ ऽमुष्मा॑ द॒मुष्मा॒दा ऽमुष्मा᳚त् । \newline
18. अ॒मुष्मा॑ द॒मुष्या॑ अ॒मुष्या॑ अ॒मुष्मा॑ द॒मुष्मा॑ द॒मुष्यै᳚ । \newline
19. अ॒मुष्यै॑ वि॒शो वि॒शो॑ ऽमुष्या॑ अ॒मुष्यै॑ वि॒शः । \newline
20. वि॒शो ऽव॑गन्तो॒ रव॑गन्तोर् वि॒शो वि॒शो ऽव॑गन्तोः । \newline
21. अव॑गन्तो॒ रिती त्यव॑गन्तो॒ रव॑गन्तो॒ रिति॑ । \newline
22. अव॑गन्तो॒रित्यव॑ - ग॒न्तोः॒ । \newline
23. इत्या॑दि॒त्या आ॑दि॒त्या इती त्या॑दि॒त्याः । \newline
24. आ॒दि॒त्या ए॒वै वादि॒त्या आ॑दि॒त्या ए॒व । \newline
25. ए॒वैन॑ मेन मे॒वैवैन᳚म् । \newline
26. ए॒न॒म् ब॒द्धवी॑रा ब॒द्धवी॑रा एन मेनम् ब॒द्धवी॑राः । \newline
27. ब॒द्धवी॑रा॒ विशं॒ ॅविश॑म् ब॒द्धवी॑रा ब॒द्धवी॑रा॒ विश᳚म् । \newline
28. ब॒द्धवी॑रा॒ इति॑ ब॒द्ध - वी॒राः॒ । \newline
29. विश॒ मवाव॒ विशं॒ ॅविश॒ मव॑ । \newline
30. अव॑ गमयन्ति गमय॒ न्त्यवाव॑ गमयन्ति । \newline
31. ग॒म॒य॒न्ति॒ यदि॒ यदि॑ गमयन्ति गमयन्ति॒ यदि॑ । \newline
32. यदि॒ न न यदि॒ यदि॒ न । \newline
33. नाव॒गच्छे॑ दव॒गच्छे॒न् न नाव॒गच्छे᳚त् । \newline
34. अ॒व॒गच्छे॑दे॒त मे॒त म॑व॒गच्छे॑ दव॒गच्छे॑ दे॒तम् । \newline
35. अ॒व॒गच्छे॒दित्य॑व - गच्छे᳚त् । \newline
36. ए॒त मे॒वैवैत मे॒त मे॒व । \newline
37. ए॒वादि॒त्य मा॑दि॒त्य मे॒वै वादि॒त्यम् । \newline
38. आ॒दि॒त्यम् च॒रुम् च॒रु मा॑दि॒त्य मा॑दि॒त्यम् च॒रुम् । \newline
39. च॒रुम् निर् णिश्च॒रुम् च॒रुम् निः । \newline
40. निर् व॑पेद् वपे॒न् निर् णिर् व॑पेत् । \newline
41. व॒पे॒ दि॒द्ध्म इ॒द्ध्मे व॑पेद् वपे दि॒द्ध्मे । \newline
42. इ॒द्ध्मे ऽप्यपी॒द्ध्म इ॒द्ध्मे ऽपि॑ । \newline
43. अपि॑ म॒यूखा᳚न् म॒यूखा॒ नप्यपि॑ म॒यूखान्॑ । \newline
44. म॒यूखा॒न् थ्सꣳ सम् म॒यूखा᳚न् म॒यूखा॒न् थ्सम् । \newline
45. सम् न॑ह्येन् नह्ये॒थ् सꣳ सम् न॑ह्येत् । \newline
46. न॒ह्ये॒ द॒न॒प॒रु॒द्ध्य म॑नपरु॒द्ध्यम् न॑ह्येन् नह्ये दनपरु॒द्ध्यम् । \newline
47. अ॒न॒प॒रु॒द्ध्य मे॒वैवान॑परु॒द्ध्य म॑नपरु॒द्ध्य मे॒व । \newline
48. अ॒न॒प॒रु॒द्ध्यमित्य॑नप - रु॒द्ध्यम् । \newline
49. ए॒वावा वै॒वैवाव॑ । \newline
50. अव॑ गच्छति गच्छ॒ त्यवाव॑ गच्छति । \newline
51. ग॒च्छ॒ त्याश्व॑त्था॒ आश्व॑त्था गच्छति गच्छ॒ त्याश्व॑त्थाः । \newline
52. आश्व॑त्था भवन्ति भव॒ न्त्याश्व॑त्था॒ आश्व॑त्था भवन्ति । \newline
53. भ॒व॒न्ति॒ म॒रुता᳚म् म॒रुता᳚म् भवन्ति भवन्ति म॒रुता᳚म् । \newline
54. म॒रुतां॒ ॅवै वै म॒रुता᳚म् म॒रुतां॒ ॅवै । \newline
55. वा ए॒त दे॒तद् वै वा ए॒तत् । \newline
56. ए॒त दोज॒ ओज॑ ए॒त दे॒त दोजः॑ । \newline
57. ओजो॒ यद् यदोज॒ ओजो॒ यत् । \newline
58. यद॑श्व॒त्थो᳚ ऽश्व॒त्थो यद् यद॑श्व॒त्थः । \newline
59. अ॒श्व॒त्थ ओज॒सौज॑सा ऽश्व॒त्थो᳚ ऽश्व॒त्थ ओज॑सा । \newline
60. ओज॑ सै॒वै वौज॒ सौज॑सै॒व । \newline
61. ए॒व विशं॒ ॅविश॑ मे॒वैव विश᳚म् । \newline
62. विश॒ मवाव॒ विशं॒ ॅविश॒ मव॑ । \newline
63. अव॑ गच्छति गच्छ॒ त्यवाव॑ गच्छति । \newline
64. ग॒च्छ॒ति॒ स॒प्त स॒प्त ग॑च्छति गच्छति स॒प्त । \newline
65. स॒प्त भ॑वन्ति भवन्ति स॒प्त स॒प्त भ॑वन्ति । \newline
66. भ॒व॒न्ति॒ स॒प्तग॑णाः स॒प्तग॑णा भवन्ति भवन्ति स॒प्तग॑णाः । \newline
67. स॒प्तग॑णा॒ वै वै स॒प्तग॑णाः स॒प्तग॑णा॒ वै । \newline
68. स॒प्तग॑णा॒ इति॑ स॒प्त - ग॒णाः॒ । \newline
69. वै म॒रुतो॑ म॒रुतो॒ वै वै म॒रुतः॑ । \newline
70. म॒रुतो॑ गण॒शो ग॑ण॒शो म॒रुतो॑ म॒रुतो॑ गण॒शः । \newline
71. ग॒ण॒श ए॒वैव ग॑ण॒शो ग॑ण॒श ए॒व । \newline
72. ग॒ण॒श इति॑ गण - शः । \newline
73. ए॒व विशं॒ ॅविश॑ मे॒वैव विश᳚म् । \newline
74. विश॒ मवाव॒ विशं॒ ॅविश॒ मव॑ । \newline
75. अव॑ गच्छति गच्छ॒ त्यवाव॑ गच्छति । \newline
76. ग॒च्छ॒तीति॑ गच्छति । \newline

\textbf{Ghana Paata } \newline

1. ग॒म॒य॒न्ति॒ यदि॒ यदि॑ गमयन्ति गमयन्ति॒ यदि॒ न न यदि॑ गमयन्ति गमयन्ति॒ यदि॒ न । \newline
2. यदि॒ न न यदि॒ यदि॒ नाव॒गच्छे॑ दव॒गच्छे॒न् न यदि॒ यदि॒ नाव॒गच्छे᳚त् । \newline
3. नाव॒गच्छे॑ दव॒गच्छे॒न् न नाव॒गच्छे॒ दाश्व॑त्था॒ नाश्व॑त्था नव॒गच्छे॒न् न नाव॒गच्छे॒ दाश्व॑त्थान् । \newline
4. अ॒व॒गच्छे॒ दाश्व॑त्था॒ नाश्व॑त्था नव॒गच्छे॑ दव॒गच्छे॒ दाश्व॑त्थान् म॒यूखा᳚न् म॒यूखा॒ नाश्व॑त्था नव॒गच्छे॑ दव॒गच्छे॒ दाश्व॑त्थान् म॒यूखान्॑ । \newline
5. अ॒व॒गच्छे॒दित्य॑व - गच्छे᳚त् । \newline
6. आश्व॑त्थान् म॒यूखा᳚न् म॒यूखा॒ नाश्व॑त्था॒ नाश्व॑त्थान् म॒यूखा᳚न् थ्स॒प्त स॒प्त म॒यूखा॒ नाश्व॑त्था॒ नाश्व॑त्थान् म॒यूखा᳚न् थ्स॒प्त । \newline
7. म॒यूखा᳚न् थ्स॒प्त स॒प्त म॒यूखा᳚न् म॒यूखा᳚न् थ्स॒प्त म॑द्ध्यमे॒षाया᳚म् मद्ध्यमे॒षायाꣳ॑ स॒प्त म॒यूखा᳚न् म॒यूखा᳚न् थ्स॒प्त म॑द्ध्यमे॒षाया᳚म् । \newline
8. स॒प्त म॑द्ध्यमे॒षाया᳚म् मद्ध्यमे॒षायाꣳ॑ स॒प्त स॒प्त म॑द्ध्यमे॒षाया॒ मुपोप॑ मद्ध्यमे॒षायाꣳ॑ स॒प्त स॒प्त म॑द्ध्यमे॒षाया॒ मुप॑ । \newline
9. म॒द्ध्य॒मे॒षाया॒ मुपोप॑ मद्ध्यमे॒षाया᳚म् मद्ध्यमे॒षाया॒ मुप॑ हन्या द्धन्या॒दुप॑ मद्ध्यमे॒षाया᳚म् मद्ध्यमे॒षाया॒ मुप॑ हन्यात् । \newline
10. म॒द्ध्य॒मे॒षाया॒मिति॑ म॒द्ध्यम - ई॒षाया᳚म् । \newline
11. उप॑ हन्या द्धन्या॒दुपोप॑ हन्यादि॒द मि॒दꣳ ह॑न्या॒दुपोप॑ हन्यादि॒दम् । \newline
12. ह॒न्या॒दि॒द मि॒दꣳ ह॑न्या द्धन्यादि॒द म॒ह म॒ह मि॒दꣳ ह॑न्या द्धन्यादि॒द म॒हम् । \newline
13. इ॒द म॒ह म॒ह मि॒द मि॒द म॒ह मा॑दि॒त्या ना॑दि॒त्या न॒ह मि॒द मि॒द म॒ह मा॑दि॒त्यान् । \newline
14. अ॒ह मा॑दि॒त्या ना॑दि॒त्या न॒ह म॒ह मा॑दि॒त्यान् ब॑ध्नामि बध्नाम्यादि॒त्या न॒ह म॒ह मा॑दि॒त्यान् ब॑ध्नामि । \newline
15. आ॒दि॒त्यान् ब॑ध्नामि बध्नाम्यादि॒त्या ना॑दि॒त्यान् ब॑ध्ना॒म्या ब॑ध्नाम्यादि॒त्या ना॑दि॒त्यान् ब॑ध्ना॒म्या । \newline
16. ब॒ध्ना॒म्या ब॑ध्नामि बध्ना॒म्या ऽमुष्मा॑ द॒मुष्मा॒ दाब॑ध्नामि बध्ना॒म्या ऽमुष्मा᳚त् । \newline
17. आ ऽमुष्मा॑ द॒मुष्मा॒ दाऽमुष्मा॑ द॒मुष्या॑ अ॒मुष्या॑ अ॒मुष्मा॒ दाऽमुष्मा॑ द॒मुष्यै᳚ । \newline
18. अ॒मुष्मा॑ द॒मुष्या॑ अ॒मुष्या॑ अ॒मुष्मा॑ द॒मुष्मा॑ द॒मुष्यै॑ वि॒शो वि॒शो॑ ऽमुष्या॑ अ॒मुष्मा॑ द॒मुष्मा॑ द॒मुष्यै॑ वि॒शः । \newline
19. अ॒मुष्यै॑ वि॒शो वि॒शो॑ ऽमुष्या॑ अ॒मुष्यै॑ वि॒शो ऽव॑गन्तो॒ रव॑गन्तोर् वि॒शो॑ ऽमुष्या॑ अ॒मुष्यै॑ वि॒शो ऽव॑गन्तोः । \newline
20. वि॒शो ऽव॑गन्तो॒ रव॑गन्तोर् वि॒शो वि॒शो ऽव॑गन्तो॒ रिती त्यव॑गन्तोर् वि॒शो वि॒शो ऽव॑गन्तो॒रिति॑ । \newline
21. अव॑गन्तो॒ रिती त्यव॑गन्तो॒ रव॑गन्तो॒ रित्या॑दि॒त्या आ॑दि॒त्या इत्यव॑गन्तो॒ रव॑गन्तो॒ रित्या॑दि॒त्याः । \newline
22. अव॑गन्तो॒रित्यव॑ - ग॒न्तोः॒ । \newline
23. इत्या॑दि॒त्या आ॑दि॒त्या इतीत्या॑दि॒त्या ए॒वैवादि॒त्या इतीत्या॑दि॒त्या ए॒व । \newline
24. आ॒दि॒त्या ए॒वैवादि॒त्या आ॑दि॒त्या ए॒वैन॑ मेन मे॒वादि॒त्या आ॑दि॒त्या ए॒वैन᳚म् । \newline
25. ए॒वैन॑ मेन मे॒वैवैन॑म् ब॒द्धवी॑रा ब॒द्धवी॑रा एन मे॒वैवैन॑म् ब॒द्धवी॑राः । \newline
26. ए॒न॒म् ब॒द्धवी॑रा ब॒द्धवी॑रा एन मेनम् ब॒द्धवी॑रा॒ विशं॒ ॅविश॑म् ब॒द्धवी॑रा एन मेनम् ब॒द्धवी॑रा॒ विश᳚म् । \newline
27. ब॒द्धवी॑रा॒ विशं॒ ॅविश॑म् ब॒द्धवी॑रा ब॒द्धवी॑रा॒ विश॒ मवाव॒ विश॑म् ब॒द्धवी॑रा ब॒द्धवी॑रा॒ विश॒ मव॑ । \newline
28. ब॒द्धवी॑रा॒ इति॑ ब॒द्ध - वी॒राः॒ । \newline
29. विश॒ मवाव॒ विशं॒ ॅविश॒ मव॑ गमयन्ति गमय॒ न्त्यव॒ विशं॒ ॅविश॒ मव॑ गमयन्ति । \newline
30. अव॑ गमयन्ति गमय॒ न्त्यवाव॑ गमयन्ति॒ यदि॒ यदि॑ गमय॒ न्त्यवाव॑ गमयन्ति॒ यदि॑ । \newline
31. ग॒म॒य॒न्ति॒ यदि॒ यदि॑ गमयन्ति गमयन्ति॒ यदि॒ न न यदि॑ गमयन्ति गमयन्ति॒ यदि॒ न । \newline
32. यदि॒ न न यदि॒ यदि॒ नाव॒गच्छे॑ दव॒गच्छे॒न् न यदि॒ यदि॒ नाव॒गच्छे᳚त् । \newline
33. नाव॒गच्छे॑ दव॒गच्छे॒न् न नाव॒गच्छे॑ दे॒त मे॒त म॑व॒गच्छे॒न् न नाव॒गच्छे॑ दे॒तम् । \newline
34. अ॒व॒गच्छे॑ दे॒त मे॒त म॑व॒गच्छे॑ दव॒गच्छे॑ दे॒त मे॒वैवैत म॑व॒गच्छे॑ दव॒गच्छे॑ दे॒त मे॒व । \newline
35. अ॒व॒गच्छे॒दित्य॑व - गच्छे᳚त् । \newline
36. ए॒त मे॒वैवैत मे॒त मे॒वादि॒त्य मा॑दि॒त्य मे॒वैत मे॒त मे॒वादि॒त्यम् । \newline
37. ए॒वादि॒त्य मा॑दि॒त्य मे॒वैवादि॒त्यम् च॒रुम् च॒रु मा॑दि॒त्य मे॒वैवादि॒त्यम् च॒रुम् । \newline
38. आ॒दि॒त्यम् च॒रुम् च॒रु मा॑दि॒त्य मा॑दि॒त्यम् च॒रुम् निर् णिश्च॒रु मा॑दि॒त्य मा॑दि॒त्यम् च॒रुम् निः । \newline
39. च॒रुम् निर् णिश्च॒रुम् च॒रुम् निर् व॑पेद् वपे॒न् निश्च॒रुम् च॒रुम् निर् व॑पेत् । \newline
40. निर् व॑पेद् वपे॒न् निर् णिर् व॑पे दि॒द्ध्म इ॒द्ध्मे व॑पे॒न् निर् णिर् व॑पे दि॒द्ध्मे । \newline
41. व॒पे॒ दि॒द्ध्म इ॒द्ध्मे व॑पेद् वपे दि॒द्ध्मे ऽप्यपी॒द्ध्मे व॑पेद् वपे दि॒द्ध्मे ऽपि॑ । \newline
42. इ॒द्ध्मे ऽप्यपी॒द्ध्म इ॒द्ध्मे ऽपि॑ म॒यूखा᳚न् म॒यूखा॒ नपी॒द्ध्म इ॒द्ध्मे ऽपि॑ म॒यूखान्॑ । \newline
43. अपि॑ म॒यूखा᳚न् म॒यूखा॒ नप्यपि॑ म॒यूखा॒न् थ्सꣳ सम् म॒यूखा॒ नप्यपि॑ म॒यूखा॒न् थ्सम् । \newline
44. म॒यूखा॒न् थ्सꣳ सम् म॒यूखा᳚न् म॒यूखा॒न् थ्सम् न॑ह्येन् नह्ये॒थ् सम् म॒यूखा᳚न् म॒यूखा॒न् थ्सम् न॑ह्येत् । \newline
45. सम् न॑ह्येन् नह्ये॒थ् सꣳ सम् न॑ह्ये दनपरु॒द्ध्य म॑नपरु॒द्ध्यम् न॑ह्ये॒थ् सꣳ सम् न॑ह्ये दनपरु॒द्ध्यम् । \newline
46. न॒ह्ये॒ द॒न॒प॒रु॒द्ध्य म॑नपरु॒द्ध्यम् न॑ह्येन् नह्ये दनपरु॒द्ध्य मे॒वैवान॑परु॒द्ध्यम् न॑ह्येन् नह्ये दनपरु॒द्ध्य मे॒व । \newline
47. अ॒न॒प॒रु॒द्ध्य मे॒वैवान॑परु॒द्ध्य म॑नपरु॒द्ध्य मे॒वावा वै॒वान॑परु॒द्ध्य म॑नपरु॒द्ध्य मे॒वाव॑ । \newline
48. अ॒न॒प॒रु॒द्ध्यमित्य॑नप - रु॒द्ध्यम् । \newline
49. ए॒वावावै॒ वैवाव॑ गच्छति गच्छ॒ त्यवै॒वैवाव॑ गच्छति । \newline
50. अव॑ गच्छति गच्छ॒ त्यवाव॑ गच्छ॒ त्याश्व॑त्था॒ आश्व॑त्था गच्छ॒ त्यवाव॑ गच्छ॒ त्याश्व॑त्थाः । \newline
51. ग॒च्छ॒ त्याश्व॑त्था॒ आश्व॑त्था गच्छति गच्छ॒ त्याश्व॑त्था भवन्ति भव॒ न्त्याश्व॑त्था गच्छति गच्छ॒ त्याश्व॑त्था भवन्ति । \newline
52. आश्व॑त्था भवन्ति भव॒ न्त्याश्व॑त्था॒ आश्व॑त्था भवन्ति म॒रुता᳚म् म॒रुता᳚म् भव॒ न्त्याश्व॑त्था॒ आश्व॑त्था भवन्ति म॒रुता᳚म् । \newline
53. भ॒व॒न्ति॒ म॒रुता᳚म् म॒रुता᳚म् भवन्ति भवन्ति म॒रुतां॒ ॅवै वै म॒रुता᳚म् भवन्ति भवन्ति म॒रुतां॒ ॅवै । \newline
54. म॒रुतां॒ ॅवै वै म॒रुता᳚म् म॒रुतां॒ ॅवा ए॒त दे॒तद् वै म॒रुता᳚म् म॒रुतां॒ ॅवा ए॒तत् । \newline
55. वा ए॒त दे॒तद् वै वा ए॒तदोज॒ ओज॑ ए॒तद् वै वा ए॒तदोजः॑ । \newline
56. ए॒तदोज॒ ओज॑ ए॒त दे॒तदोजो॒ यद् यदोज॑ ए॒त दे॒तदोजो॒ यत् । \newline
57. ओजो॒ यद् यदोज॒ ओजो॒ यद॑श्व॒त्थो᳚ ऽश्व॒त्थो यदोज॒ ओजो॒ यद॑श्व॒त्थः । \newline
58. यद॑श्व॒त्थो᳚ ऽश्व॒त्थो यद् यद॑श्व॒त्थ ओज॒सौज॑सा ऽश्व॒त्थो यद् यद॑श्व॒त्थ ओज॑सा । \newline
59. अ॒श्व॒त्थ ओज॒सौज॑सा ऽश्व॒त्थो᳚ ऽश्व॒त्थ ओज॑ सै॒वैवौज॑सा ऽश्व॒त्थो᳚ ऽश्व॒त्थ ओज॑सै॒व । \newline
60. ओज॑ सै॒वैवौज॒ सौज॑सै॒व विशं॒ ॅविश॑ मे॒वौज॒ सौज॑सै॒व विश᳚म् । \newline
61. ए॒व विशं॒ ॅविश॑ मे॒वैव विश॒ मवाव॒ विश॑ मे॒वैव विश॒ मव॑ । \newline
62. विश॒ मवाव॒ विशं॒ ॅविश॒ मव॑ गच्छति गच्छ॒त्यव॒ विशं॒ ॅविश॒ मव॑ गच्छति । \newline
63. अव॑ गच्छति गच्छ॒ त्यवाव॑ गच्छति स॒प्त स॒प्त ग॑च्छ॒ त्यवाव॑ गच्छति स॒प्त । \newline
64. ग॒च्छ॒ति॒ स॒प्त स॒प्त ग॑च्छति गच्छति स॒प्त भ॑वन्ति भवन्ति स॒प्त ग॑च्छति गच्छति स॒प्त भ॑वन्ति । \newline
65. स॒प्त भ॑वन्ति भवन्ति स॒प्त स॒प्त भ॑वन्ति स॒प्तग॑णाः स॒प्तग॑णा भवन्ति स॒प्त स॒प्त भ॑वन्ति स॒प्तग॑णाः । \newline
66. भ॒व॒न्ति॒ स॒प्तग॑णाः स॒प्तग॑णा भवन्ति भवन्ति स॒प्तग॑णा॒ वै वै स॒प्तग॑णा भवन्ति भवन्ति स॒प्तग॑णा॒ वै । \newline
67. स॒प्तग॑णा॒ वै वै स॒प्तग॑णाः स॒प्तग॑णा॒ वै म॒रुतो॑ म॒रुतो॒ वै स॒प्तग॑णाः स॒प्तग॑णा॒ वै म॒रुतः॑ । \newline
68. स॒प्तग॑णा॒ इति॑ स॒प्त - ग॒णाः॒ । \newline
69. वै म॒रुतो॑ म॒रुतो॒ वै वै म॒रुतो॑ गण॒शो ग॑ण॒शो म॒रुतो॒ वै वै म॒रुतो॑ गण॒शः । \newline
70. म॒रुतो॑ गण॒शो ग॑ण॒शो म॒रुतो॑ म॒रुतो॑ गण॒श ए॒वैव ग॑ण॒शो म॒रुतो॑ म॒रुतो॑ गण॒श ए॒व । \newline
71. ग॒ण॒श ए॒वैव ग॑ण॒शो ग॑ण॒श ए॒व विशं॒ ॅविश॑ मे॒व ग॑ण॒शो ग॑ण॒श ए॒व विश᳚म् । \newline
72. ग॒ण॒श इति॑ गण - शः । \newline
73. ए॒व विशं॒ ॅविश॑ मे॒वैव विश॒ मवाव॒ विश॑ मे॒वैव विश॒ मव॑ । \newline
74. विश॒ मवाव॒ विशं॒ ॅविश॒ मव॑ गच्छति गच्छ॒ त्यव॒ विशं॒ ॅविश॒ मव॑ गच्छति । \newline
75. अव॑ गच्छति गच्छ॒ त्यवाव॑ गच्छति । \newline
76. ग॒च्छ॒तीति॑ गच्छति । \newline
\pagebreak
\markright{ TS 2.3.2.1  \hfill https://www.vedavms.in \hfill}

\section{ TS 2.3.2.1 }

\textbf{TS 2.3.2.1 } \newline
\textbf{Samhita Paata} \newline

दे॒वा वै मृ॒त्योर॑बिभयु॒स्ते प्र॒जाप॑ति॒मुपा॑धाव॒न् तेभ्य॑ ए॒तां प्रा॑जाप॒त्याꣳ श॒तकृ॑ष्णलां॒ निर॑वप॒त् तयै॒वैष्व॒मृत॑मदधा॒द्यो मृ॒त्योर्बि॑भी॒यात् तस्मा॑ ए॒तां प्रा॑जाप॒त्याꣳ श॒तकृ॑ष्णलां॒ निर्व॑पेत् प्र॒जाप॑तिमे॒व स्वेन॑ भाग॒धेये॒नोप॑ धावति॒ स ए॒वास्मि॒न्नायु॑र्दधाति॒ सर्व॒मायु॑रेति श॒तकृ॑ष्णला भवति श॒तायुः॒ पुरु॑षः॒ श॒तेन्द्रि॑य॒ आयु॑ष्ये॒वेन्द्रि॒ये - [  ] \newline

\textbf{Pada Paata} \newline

दे॒वाः । वै । मृ॒त्योः । अ॒बि॒भ॒युः॒ । ते । प्र॒जाप॑ति॒मिति॑ प्र॒जा-प॒ति॒म् । उपेति॑ । अ॒धा॒व॒न्न् । तेभ्यः॑ । ए॒ताम् । प्रा॒जा॒प॒त्यामिति॑ प्राजा - प॒त्याम् । श॒तकृ॑ष्णला॒मिति॑ श॒त-कृ॒ष्ण॒ला॒म् । निरिति॑ । अ॒व॒प॒त् । तया᳚ । ए॒व । ए॒षु॒ । अ॒मृत᳚म् । अ॒द॒धा॒त् । यः । मृ॒त्योः । बि॒भी॒यात् । तस्मै᳚ । ए॒ताम् । प्रा॒जा॒प॒त्यामिति॑ प्राजा - प॒त्याम् । श॒तकृ॑ष्णला॒मिति॑ श॒त - कृ॒ष्ण॒ला॒म् । निरिति॑ । व॒पे॒त् । प्र॒जाप॑ति॒मिति॑ प्र॒जा-प॒ति॒म् । ए॒व । स्वेन॑ । भा॒ग॒धेये॒नेति॑ भाग - धेये॑न । उपेति॑ । धा॒व॒ति॒ । सः । ए॒व । अ॒स्मि॒न्न् । आयुः॑ । द॒धा॒ति॒ । सर्व᳚म् । आयुः॑ । ए॒ति॒ । श॒तकृ॑ष्ण॒लेति॑ श॒त - कृ॒ष्ण॒ला॒ । भ॒व॒ति॒ । श॒तायु॒रिति॑ श॒त-आ॒युः॒ । पुरु॑षः । श॒तेन्द्रि॑य॒ इति॑ श॒त - इ॒न्द्रि॒यः॒ । आयु॑षि । ए॒व । इ॒न्द्रि॒ये ।  \newline


\textbf{Krama Paata} \newline

दे॒वा वै । वै मृ॒त्योः । मृ॒त्योर॑बिभयुः । अ॒बि॒भ॒यु॒स्ते । ते प्र॒जाप॑तिम् । प्र॒जाप॑ति॒मुप॑ । प्र॒जाप॑ति॒मिति॑ प्र॒जा - प॒ति॒म् । उपा॑धावन्न् । अ॒धा॒व॒न् तेभ्यः॑ । तेभ्य॑ ए॒ताम् । ए॒ताम् प्रा॑जाप॒त्याम् । प्रा॒जा॒प॒त्याꣳ श॒तकृ॑ष्णलाम् । प्रा॒जा॒प॒त्यामिति॑ प्राजा - प॒त्याम् । श॒तकृ॑ष्णला॒म् निः । श॒तकृ॑ष्णला॒मिति॑ श॒त - कृ॒ष्ण॒ला॒म् । निर॑वपत् । अ॒व॒प॒त् तया᳚ । तयै॒व । ए॒वैषु॑ । ए॒ष्व॒मृत᳚म् । अ॒मृत॑मदधात् । अ॒द॒धा॒द् यः । यो मृ॒त्योः । मृ॒त्योर् बि॑भी॒यात् । बि॒भी॒यात् तस्मै᳚ । तस्मा॑ ए॒ताम् । ए॒ताम् प्रा॑जाप॒त्याम् । प्रा॒जा॒प॒त्याꣳ श॒तकृ॑ष्णलाम् । प्रा॒जा॒प॒त्यामिति॑ प्राजा - प॒त्याम् । श॒तकृ॑ष्णला॒म् निः । श॒तकृ॑ष्णला॒मिति॑ श॒त - कृ॒ष्ण॒ला॒म् । निर् व॑पेत् । व॒पे॒त् प्र॒जाप॑तिम् । प्र॒जाप॑तिमे॒व । प्र॒जाप॑ति॒मिति॑ प्र॒जा - प॒ति॒म् । ए॒व स्वेन॑ । स्वेन॑ भाग॒धेये॑न । भा॒ग॒धेये॒नोप॑ । भा॒ग॒धेये॒नेति॑ भाग - धेये॑न । उप॑ धावति । धा॒व॒ति॒ सः । स ए॒व । ए॒वास्मिन्न्॑ । अ॒स्मि॒न्नायुः॑ । आयु॑र् दधाति । द॒धा॒ति॒ सर्व᳚म् । सर्व॒मायुः॑ । आयु॑रेति । ए॒ति॒ श॒तकृ॑ष्णला । श॒तकृ॑ष्णला भवति । श॒तकृ॑ष्ण॒लेति॑ श॒त - कृ॒ष्ण॒ला॒ । भ॒व॒ति॒ श॒तायुः॑ । श॒तायुः॒ पुरु॑षः । श॒तायु॒रिति॑ श॒त - आ॒युः॒ । पुरु॑षः श॒तेन्द्रि॑यः । श॒तेन्द्रि॑य॒ आयु॑षि । श॒तेन्द्रि॑य॒ इति॑ श॒त - इ॒न्द्रि॒यः॒ । आयु॑ष्ये॒व । ए॒वेन्द्रि॒ये । इ॒न्द्रि॒ये प्रति॑ \newline

\textbf{Jatai Paata} \newline

1. दे॒वा वै वै दे॒वा दे॒वा वै । \newline
2. वै मृ॒त्योर् मृ॒त्योर् वै वै मृ॒त्योः । \newline
3. मृ॒त्यो र॑बिभयु रबिभयुर् मृ॒त्योर् मृ॒त्यो र॑बिभयुः । \newline
4. अ॒बि॒भ॒यु॒ स्ते ते॑ ऽबिभयु रबिभयु॒ स्ते । \newline
5. ते प्र॒जाप॑तिम् प्र॒जाप॑ति॒म् ते ते प्र॒जाप॑तिम् । \newline
6. प्र॒जाप॑ति॒ मुपोप॑ प्र॒जाप॑तिम् प्र॒जाप॑ति॒ मुप॑ । \newline
7. प्र॒जाप॑ति॒मिति॑ प्र॒जा - प॒ति॒म् । \newline
8. उपा॑धावन् नधाव॒न् नुपोपा॑धावन्न् । \newline
9. अ॒धा॒व॒न् तेभ्य॒ स्तेभ्यो॑ ऽधावन् नधाव॒न् तेभ्यः॑ । \newline
10. तेभ्य॑ ए॒ता मे॒ताम् तेभ्य॒ स्तेभ्य॑ ए॒ताम् । \newline
11. ए॒ताम् प्रा॑जाप॒त्याम् प्रा॑जाप॒त्या मे॒ता मे॒ताम् प्रा॑जाप॒त्याम् । \newline
12. प्रा॒जा॒प॒त्याꣳ श॒तकृ॑ष्णलाꣳ श॒तकृ॑ष्णलाम् प्राजाप॒त्याम् प्रा॑जाप॒त्याꣳ श॒तकृ॑ष्णलाम् । \newline
13. प्रा॒जा॒प॒त्यामिति॑ प्राजा - प॒त्याम् । \newline
14. श॒तकृ॑ष्णला॒म् निर् णिः श॒तकृ॑ष्णलाꣳ श॒तकृ॑ष्णला॒म् निः । \newline
15. श॒तकृ॑ष्णला॒मिति॑ श॒त - कृ॒ष्ण॒ला॒म् । \newline
16. निर॑वप दवप॒न् निर् णि र॑वपत् । \newline
17. अ॒व॒प॒त् तया॒ तया॑ ऽवप दवप॒त् तया᳚ । \newline
18. तयै॒वैव तया॒ तयै॒व । \newline
19. ए॒वैष्वे᳚ ष्वे॒वैवैषु॑ । \newline
20. ए॒ष्व॒मृत॑ म॒मृत॑ मेष्वे ष्व॒मृत᳚म् । \newline
21. अ॒मृत॑ मदधा ददधा द॒मृत॑ म॒मृत॑ मदधात् । \newline
22. अ॒द॒धा॒द् यो यो॑ ऽदधा ददधा॒द् यः । \newline
23. यो मृ॒त्योर् मृ॒त्योर् यो यो मृ॒त्योः । \newline
24. मृ॒त्योर् बि॑भी॒याद् बि॑भी॒यान् मृ॒त्योर् मृ॒त्योर् बि॑भी॒यात् । \newline
25. बि॒भी॒यात् तस्मै॒ तस्मै॑ बिभी॒याद् बि॑भी॒यात् तस्मै᳚ । \newline
26. तस्मा॑ ए॒ता मे॒ताम् तस्मै॒ तस्मा॑ ए॒ताम् । \newline
27. ए॒ताम् प्रा॑जाप॒त्याम् प्रा॑जाप॒त्या मे॒ता मे॒ताम् प्रा॑जाप॒त्याम् । \newline
28. प्रा॒जा॒प॒त्याꣳ श॒तकृ॑ष्णलाꣳ श॒तकृ॑ष्णलाम् प्राजाप॒त्याम् प्रा॑जाप॒त्याꣳ श॒तकृ॑ष्णलाम् । \newline
29. प्रा॒जा॒प॒त्यामिति॑ प्राजा - प॒त्याम् । \newline
30. श॒तकृ॑ष्णला॒म् निर् णिः श॒तकृ॑ष्णलाꣳ श॒तकृ॑ष्णला॒म् निः । \newline
31. श॒तकृ॑ष्णला॒मिति॑ श॒त - कृ॒ष्ण॒ला॒म् । \newline
32. निर् व॑पेद् वपे॒न् निर् णिर् व॑पेत् । \newline
33. व॒पे॒त् प्र॒जाप॑तिम् प्र॒जाप॑तिं ॅवपेद् वपेत् प्र॒जाप॑तिम् । \newline
34. प्र॒जाप॑ति मे॒वैव प्र॒जाप॑तिम् प्र॒जाप॑ति मे॒व । \newline
35. प्र॒जाप॑ति॒मिति॑ प्र॒जा - प॒ति॒म् । \newline
36. ए॒व स्वेन॒ स्वेनै॒वैव स्वेन॑ । \newline
37. स्वेन॑ भाग॒धेये॑न भाग॒धेये॑न॒ स्वेन॒ स्वेन॑ भाग॒धेये॑न । \newline
38. भा॒ग॒धेये॒नोपोप॑ भाग॒धेये॑न भाग॒धेये॒नोप॑ । \newline
39. भा॒ग॒धेये॒नेति॑ भाग - धेये॑न । \newline
40. उप॑ धावति धाव॒ त्युपोप॑ धावति । \newline
41. धा॒व॒ति॒ स स धा॑वति धावति॒ सः । \newline
42. स ए॒वैव स स ए॒व । \newline
43. ए॒वास्मि॑न् नस्मिन् ने॒वैवास्मिन्न्॑ । \newline
44. अ॒स्मि॒न् नायु॒ रायु॑ रस्मिन् नस्मि॒न् नायुः॑ । \newline
45. आयु॑र् दधाति दधा॒ त्यायु॒ रायु॑र् दधाति । \newline
46. द॒धा॒ति॒ सर्वꣳ॒॒ सर्व॑म् दधाति दधाति॒ सर्व᳚म् । \newline
47. सर्व॒ मायु॒ रायुः॒ सर्वꣳ॒॒ सर्व॒ मायुः॑ । \newline
48. आयु॑ रेत्ये॒ त्यायु॒ रायु॑रेति । \newline
49. ए॒ति॒ श॒तकृ॑ष्णला श॒तकृ॑ष्णलैत्येति श॒तकृ॑ष्णला । \newline
50. श॒तकृ॑ष्णला भवति भवति श॒तकृ॑ष्णला श॒तकृ॑ष्णला भवति । \newline
51. श॒तकृ॑ष्ण॒लेति॑ श॒त - कृ॒ष्ण॒ला॒ । \newline
52. भ॒व॒ति॒ श॒तायुः॑ श॒तायु॑र् भवति भवति श॒तायुः॑ । \newline
53. श॒तायुः॒ पुरु॑षः॒ पुरु॑षः श॒तायुः॑ श॒तायुः॒ पुरु॑षः । \newline
54. श॒तायु॒रिति॑ श॒त - आ॒युः॒ । \newline
55. पुरु॑षः श॒तेन्द्रि॑यः श॒तेन्द्रि॑यः॒ पुरु॑षः॒ पुरु॑षः श॒तेन्द्रि॑यः । \newline
56. श॒तेन्द्रि॑य॒ आयु॒ष्यायु॑षि श॒तेन्द्रि॑यः श॒तेन्द्रि॑य॒ आयु॑षि । \newline
57. श॒तेन्द्रि॑य॒ इति॑ श॒त - इ॒न्द्रि॒यः॒ । \newline
58. आयु॑ ष्ये॒वै वायु॒ ष्यायु॑ ष्ये॒व । \newline
59. ए॒वे न्द्रि॒य इ॑न्द्रि॒य ए॒वैवे न्द्रि॒ये । \newline
60. इ॒न्द्रि॒ये प्रति॒ प्रती᳚न्द्रि॒य इ॑न्द्रि॒ये प्रति॑ । \newline

\textbf{Ghana Paata } \newline

1. दे॒वा वै वै दे॒वा दे॒वा वै मृ॒त्योर् मृ॒त्योर् वै दे॒वा दे॒वा वै मृ॒त्योः । \newline
2. वै मृ॒त्योर् मृ॒त्योर् वै वै मृ॒त्यो र॑बिभयु रबिभयुर् मृ॒त्योर् वै वै मृ॒त्यो र॑बिभयुः । \newline
3. मृ॒त्यो र॑बिभयु रबिभयुर् मृ॒त्योर् मृ॒त्यो र॑बिभयु॒ स्ते ते॑ ऽबिभयुर् मृ॒त्योर् मृ॒त्यो र॑बिभयु॒ स्ते । \newline
4. अ॒बि॒भ॒यु॒ स्ते ते॑ ऽबिभयु रबिभयु॒ स्ते प्र॒जाप॑तिम् प्र॒जाप॑ति॒म् ते॑ ऽबिभयु रबिभयु॒ स्ते प्र॒जाप॑तिम् । \newline
5. ते प्र॒जाप॑तिम् प्र॒जाप॑ति॒म् ते ते प्र॒जाप॑ति॒ मुपोप॑ प्र॒जाप॑ति॒म् ते ते प्र॒जाप॑ति॒ मुप॑ । \newline
6. प्र॒जाप॑ति॒ मुपोप॑ प्र॒जाप॑तिम् प्र॒जाप॑ति॒ मुपा॑धावन् नधाव॒न् नुप॑ प्र॒जाप॑तिम् प्र॒जाप॑ति॒ मुपा॑धावन्न् । \newline
7. प्र॒जाप॑ति॒मिति॑ प्र॒जा - प॒ति॒म् । \newline
8. उपा॑धावन् नधाव॒न् नुपोपा॑धाव॒न् तेभ्य॒ स्तेभ्यो॑ ऽधाव॒न् नुपोपा॑धाव॒न् तेभ्यः॑ । \newline
9. अ॒धा॒व॒न् तेभ्य॒ स्तेभ्यो॑ ऽधावन् नधाव॒न् तेभ्य॑ ए॒ता मे॒ताम् तेभ्यो॑ ऽधावन् नधाव॒न् तेभ्य॑ ए॒ताम् । \newline
10. तेभ्य॑ ए॒ता मे॒ताम् तेभ्य॒ स्तेभ्य॑ ए॒ताम् प्रा॑जाप॒त्याम् प्रा॑जाप॒त्या मे॒ताम् तेभ्य॒ स्तेभ्य॑ ए॒ताम् प्रा॑जाप॒त्याम् । \newline
11. ए॒ताम् प्रा॑जाप॒त्याम् प्रा॑जाप॒त्या मे॒ता मे॒ताम् प्रा॑जाप॒त्याꣳ श॒तकृ॑ष्णलाꣳ श॒तकृ॑ष्णलाम् प्राजाप॒त्या मे॒ता मे॒ताम् प्रा॑जाप॒त्याꣳ श॒तकृ॑ष्णलाम् । \newline
12. प्रा॒जा॒प॒त्याꣳ श॒तकृ॑ष्णलाꣳ श॒तकृ॑ष्णलाम् प्राजाप॒त्याम् प्रा॑जाप॒त्याꣳ श॒तकृ॑ष्णला॒म् निर् णिः श॒तकृ॑ष्णलाम् प्राजाप॒त्याम् प्रा॑जाप॒त्याꣳ श॒तकृ॑ष्णला॒म् निः । \newline
13. प्रा॒जा॒प॒त्यामिति॑ प्राजा - प॒त्याम् । \newline
14. श॒तकृ॑ष्णला॒म् निर् णिः श॒तकृ॑ष्णलाꣳ श॒तकृ॑ष्णला॒म् निर॑वप दवप॒न् निः श॒तकृ॑ष्णलाꣳ श॒तकृ॑ष्णला॒म् निर॑वपत् । \newline
15. श॒तकृ॑ष्णला॒मिति॑ श॒त - कृ॒ष्ण॒ला॒म् । \newline
16. निर॑वप दवप॒न् निर् णिर॑वप॒त् तया॒ तया॑ ऽवप॒न् निर् णिर॑वप॒त् तया᳚ । \newline
17. अ॒व॒प॒त् तया॒ तया॑ ऽवप दवप॒त् तयै॒वैव तया॑ ऽवप दवप॒त् तयै॒व । \newline
18. तयै॒वैव तया॒ तयै॒वैष्वे᳚ष्वे॒व तया॒ तयै॒वैषु॑ । \newline
19. ए॒वैष्वे᳚ष्वे॒वैवैष्व॒मृत॑ म॒मृत॑ मेष्वे॒वैवैष्व॒मृत᳚म् । \newline
20. ए॒ष्व॒मृत॑ म॒मृत॑ मेष्वेष्व॒मृत॑ मदधा ददधा द॒मृत॑ मेष्वेष्व॒मृत॑ मदधात् । \newline
21. अ॒मृत॑ मदधा ददधा द॒मृत॑ म॒मृत॑ मदधा॒द् यो यो॑ ऽदधा द॒मृत॑ म॒मृत॑ मदधा॒द् यः । \newline
22. अ॒द॒धा॒द् यो यो॑ ऽदधा ददधा॒द् यो मृ॒त्योर् मृ॒त्योर् यो॑ ऽदधा ददधा॒द् यो मृ॒त्योः । \newline
23. यो मृ॒त्योर् मृ॒त्योर् यो यो मृ॒त्योर् बि॑भी॒याद् बि॑भी॒यान् मृ॒त्योर् यो यो मृ॒त्योर् बि॑भी॒यात् । \newline
24. मृ॒त्योर् बि॑भी॒याद् बि॑भी॒यान् मृ॒त्योर् मृ॒त्योर् बि॑भी॒यात् तस्मै॒ तस्मै॑ बिभी॒यान् मृ॒त्योर् मृ॒त्योर् बि॑भी॒यात् तस्मै᳚ । \newline
25. बि॒भी॒यात् तस्मै॒ तस्मै॑ बिभी॒याद् बि॑भी॒यात् तस्मा॑ ए॒ता मे॒ताम् तस्मै॑ बिभी॒याद् बि॑भी॒यात् तस्मा॑ ए॒ताम् । \newline
26. तस्मा॑ ए॒ता मे॒ताम् तस्मै॒ तस्मा॑ ए॒ताम् प्रा॑जाप॒त्याम् प्रा॑जाप॒त्या मे॒ताम् तस्मै॒ तस्मा॑ ए॒ताम् प्रा॑जाप॒त्याम् । \newline
27. ए॒ताम् प्रा॑जाप॒त्याम् प्रा॑जाप॒त्या मे॒ता मे॒ताम् प्रा॑जाप॒त्याꣳ श॒तकृ॑ष्णलाꣳ श॒तकृ॑ष्णलाम् प्राजाप॒त्या मे॒ता मे॒ताम् प्रा॑जाप॒त्याꣳ श॒तकृ॑ष्णलाम् । \newline
28. प्रा॒जा॒प॒त्याꣳ श॒तकृ॑ष्णलाꣳ श॒तकृ॑ष्णलाम् प्राजाप॒त्याम् प्रा॑जाप॒त्याꣳ श॒तकृ॑ष्णला॒म् निर् णिः श॒तकृ॑ष्णलाम् प्राजाप॒त्याम् प्रा॑जाप॒त्याꣳ श॒तकृ॑ष्णला॒म् निः । \newline
29. प्रा॒जा॒प॒त्यामिति॑ प्राजा - प॒त्याम् । \newline
30. श॒तकृ॑ष्णला॒म् निर् णिः श॒तकृ॑ष्णलाꣳ श॒तकृ॑ष्णला॒म् निर् व॑पेद् वपे॒न् निः श॒तकृ॑ष्णलाꣳ श॒तकृ॑ष्णला॒म् निर् व॑पेत् । \newline
31. श॒तकृ॑ष्णला॒मिति॑ श॒त - कृ॒ष्ण॒ला॒म् । \newline
32. निर् व॑पेद् वपे॒न् निर् णिर् व॑पेत् प्र॒जाप॑तिम् प्र॒जाप॑तिं ॅवपे॒न् निर् णिर् व॑पेत् प्र॒जाप॑तिम् । \newline
33. व॒पे॒त् प्र॒जाप॑तिम् प्र॒जाप॑तिं ॅवपेद् वपेत् प्र॒जाप॑ति मे॒वैव प्र॒जाप॑तिं ॅवपेद् वपेत् प्र॒जाप॑ति मे॒व । \newline
34. प्र॒जाप॑ति मे॒वैव प्र॒जाप॑तिम् प्र॒जाप॑ति मे॒व स्वेन॒ स्वेनै॒व प्र॒जाप॑तिम् प्र॒जाप॑ति मे॒व स्वेन॑ । \newline
35. प्र॒जाप॑ति॒मिति॑ प्र॒जा - प॒ति॒म् । \newline
36. ए॒व स्वेन॒ स्वेनै॒वैव स्वेन॑ भाग॒धेये॑न भाग॒धेये॑न॒ स्वेनै॒वैव स्वेन॑ भाग॒धेये॑न । \newline
37. स्वेन॑ भाग॒धेये॑न भाग॒धेये॑न॒ स्वेन॒ स्वेन॑ भाग॒धेये॒नोपोप॑ भाग॒धेये॑न॒ स्वेन॒ स्वेन॑ भाग॒धेये॒नोप॑ । \newline
38. भा॒ग॒धेये॒नोपोप॑ भाग॒धेये॑न भाग॒धेये॒नोप॑ धावति धाव॒त्युप॑ भाग॒धेये॑न भाग॒धेये॒नोप॑ धावति । \newline
39. भा॒ग॒धेये॒नेति॑ भाग - धेये॑न । \newline
40. उप॑ धावति धाव॒ त्युपोप॑ धावति॒ स स धा॑व॒ त्युपोप॑ धावति॒ सः । \newline
41. धा॒व॒ति॒ स स धा॑वति धावति॒ स ए॒वैव स धा॑वति धावति॒ स ए॒व । \newline
42. स ए॒वैव स स ए॒वास्मि॑न् नस्मिन् ने॒व स स ए॒वास्मिन्न्॑ । \newline
43. ए॒वास्मि॑न् नस्मिन् ने॒वैवास्मि॒न् नायु॒ रायु॑ रस्मिन् ने॒वैवास्मि॒न् नायुः॑ । \newline
44. अ॒स्मि॒न् नायु॒ रायु॑ रस्मिन् नस्मि॒न् नायु॑र् दधाति दधा॒ त्यायु॑ रस्मिन् नस्मि॒न् नायु॑र् दधाति । \newline
45. आयु॑र् दधाति दधा॒ त्यायु॒ रायु॑र् दधाति॒ सर्वꣳ॒॒ सर्व॑म् दधा॒ त्यायु॒ रायु॑र् दधाति॒ सर्व᳚म् । \newline
46. द॒धा॒ति॒ सर्वꣳ॒॒ सर्व॑म् दधाति दधाति॒ सर्व॒ मायु॒रायुः॒ सर्व॑म् दधाति दधाति॒ सर्व॒ मायुः॑ । \newline
47. सर्व॒ मायु॒ रायुः॒ सर्वꣳ॒॒ सर्व॒ मायु॑ रेत्ये॒त्यायुः॒ सर्वꣳ॒॒ सर्व॒ मायु॑रेति । \newline
48. आयु॑ रेत्ये॒त्यायु॒ रायु॑रेति श॒तकृ॑ष्णला श॒तकृ॑ष्णलै॒ त्यायु॒ रायु॑रेति श॒तकृ॑ष्णला । \newline
49. ए॒ति॒ श॒तकृ॑ष्णला श॒तकृ॑ष्णलैत्येति श॒तकृ॑ष्णला भवति भवति श॒तकृ॑ष्णलैत्येति श॒तकृ॑ष्णला भवति । \newline
50. श॒तकृ॑ष्णला भवति भवति श॒तकृ॑ष्णला श॒तकृ॑ष्णला भवति श॒तायुः॑ श॒तायु॑र् भवति श॒तकृ॑ष्णला श॒तकृ॑ष्णला भवति श॒तायुः॑ । \newline
51. श॒तकृ॑ष्ण॒लेति॑ श॒त - कृ॒ष्ण॒ला॒ । \newline
52. भ॒व॒ति॒ श॒तायुः॑ श॒तायु॑र् भवति भवति श॒तायुः॒ पुरु॑षः॒ पुरु॑षः श॒तायु॑र् भवति भवति श॒तायुः॒ पुरु॑षः । \newline
53. श॒तायुः॒ पुरु॑षः॒ पुरु॑षः श॒तायुः॑ श॒तायुः॒ पुरु॑षः श॒तेन्द्रि॑यः श॒तेन्द्रि॑यः॒ पुरु॑षः श॒तायुः॑ श॒तायुः॒ पुरु॑षः श॒तेन्द्रि॑यः । \newline
54. श॒तायु॒रिति॑ श॒त - आ॒युः॒ । \newline
55. पुरु॑षः श॒तेन्द्रि॑यः श॒तेन्द्रि॑यः॒ पुरु॑षः॒ पुरु॑षः श॒तेन्द्रि॑य॒ आयु॒ष्यायु॑षि श॒तेन्द्रि॑यः॒ पुरु॑षः॒ पुरु॑षः श॒तेन्द्रि॑य॒ आयु॑षि । \newline
56. श॒तेन्द्रि॑य॒ आयु॒ ष्यायु॑षि श॒तेन्द्रि॑यः श॒तेन्द्रि॑य॒ आयु॑ ष्ये॒वैवायु॑षि श॒तेन्द्रि॑यः श॒तेन्द्रि॑य॒ आयु॑ष्ये॒व । \newline
57. श॒तेन्द्रि॑य॒ इति॑ श॒त - इ॒न्द्रि॒यः॒ । \newline
58. आयु॑ ष्ये॒वैवायु॒ ष्यायु॑ष्ये॒वे न्द्रि॒य इ॑न्द्रि॒य ए॒वायु॒ ष्यायु॑ष्ये॒वे न्द्रि॒ये । \newline
59. ए॒वे न्द्रि॒य इ॑न्द्रि॒य ए॒वैवे न्द्रि॒ये प्रति॒ प्रती᳚न्द्रि॒य ए॒वैवे न्द्रि॒ये प्रति॑ । \newline
60. इ॒न्द्रि॒ये प्रति॒ प्रती᳚न्द्रि॒य इ॑न्द्रि॒ये प्रति॑ तिष्ठति तिष्ठति॒ प्रती᳚न्द्रि॒य इ॑न्द्रि॒ये प्रति॑ तिष्ठति । \newline
\pagebreak
\markright{ TS 2.3.2.2  \hfill https://www.vedavms.in \hfill}

\section{ TS 2.3.2.2 }

\textbf{TS 2.3.2.2 } \newline
\textbf{Samhita Paata} \newline

प्रति॑ तिष्ठति घृ॒ते भ॑व॒त्यायु॒र्वै घृ॒तम॒मृतꣳ॒॒ हिर॑ण्य॒मायु॑श्चै॒वास्मा॑ अ॒मृतं॑ च स॒मीची॑ दधाति च॒त्वारि॑ चत्वारि कृ॒ष्णला॒न्यव॑ द्यति चतुरव॒त् तस्याऽऽप्त्या॑ एक॒धा ब्र॒ह्मण॒ उप॑ हरत्येक॒धैव यज॑मान॒ आयु॑र्दधात्य॒- सावा॑दि॒त्यो न व्य॑रोचत॒ तस्मै॑ दे॒वाः प्राय॑श्चित्तिमैच्छ॒न् तस्मा॑ ए॒तꣳ सौ॒र्यं च॒रुं निर॑वप॒न् तेनै॒वास्मि॒न् - [  ] \newline

\textbf{Pada Paata} \newline

प्रतीति॑ । ति॒ष्ठ॒ति॒ । घृ॒ते । भ॒व॒ति॒ । आयुः॑ । वै । घृ॒तम् । अ॒मृत᳚म् । हिर॑ण्यम् । आयुः॑ । च॒ । ए॒व । अ॒स्मै॒ । अ॒मृत᳚म् । च॒ । स॒मीची॒ इति॑ । द॒धा॒ति॒ । च॒त्वारि॑चत्वा॒रीति॑ च॒त्वारि॑ - च॒त्वा॒रि॒ । कृ॒ष्णला॑नि । अवेति॑ । द्य॒ति॒ । च॒तु॒र॒व॒त्तस्येति॑ चतुः - अ॒व॒त्तस्य॑ । आप्त्यै᳚ । ए॒क॒धेत्ये॑क - धा । ब्र॒ह्मणे᳚ । उपेति॑ । ह॒र॒ति॒ । ए॒क॒धेत्ये॑क - धा । ए॒व । यज॑माने । आयुः॑ । द॒धा॒ति॒ । अ॒सौ । आ॒दि॒त्यः । न । वीति॑ । अ॒रो॒च॒त॒ । तस्मै᳚ । दे॒वाः । प्राय॑श्चित्तिम् । ऐ॒च्छ॒न्न् । तस्मै᳚ । ए॒तम् । सौ॒र्यम् । च॒रुम् । निरिति॑ । अ॒व॒प॒न्न् । तेन॑ । ए॒व । अ॒स्मि॒न्न् ।  \newline


\textbf{Krama Paata} \newline

प्रति॑ तिष्ठति । ति॒ष्ठ॒ति॒ घृ॒ते । घृ॒ते भ॑वति । भ॒व॒त्यायुः॑ । आयु॒र् वै । वै घृ॒तम् । घृ॒तम॒मृत᳚म् । अ॒मृतꣳ॒॒ हिर॑ण्यम् । हिर॑ण्य॒मायुः॑ । आयु॑श्च । चै॒व । ए॒वास्मै᳚ । अ॒स्मा॒ अ॒मृत᳚म् । अ॒मृत॑म् च । च॒ स॒मीची᳚ । स॒मीची॑ दधाति । स॒मीची॒ इति॑ स॒मीची᳚ । द॒धा॒ति॒ च॒त्वारि॑चत्वारि । च॒त्वारि॑चत्वारि कृ॒ष्णला॑नि । च॒त्वारि॑चत्वा॒रीति॑ च॒त्वारि॑ - च॒त्वा॒रि॒ । कृ॒ष्णला॒न्यव॑ । अव॑ द्यति । द्य॒ति॒ च॒तु॒र॒व॒त्तस्य॑ । च॒तु॒र॒व॒त्तस्याप्त्ये᳚ । च॒तु॒र॒व॒त्तस्येति॑ चतुः - अ॒व॒त्तस्य॑ । आप्त्या॑ एक॒धा । ए॒क॒धा ब्र॒ह्मणे᳚ । ए॒क॒धेत्ये॑क - धा । ब्र॒ह्मण॒ उप॑ । उप॑ हरति । ह॒र॒त्ये॒क॒धा । ए॒क॒धैव । ए॒क॒धेत्ये॑क - धा । ए॒व यज॑माने । यज॑मान॒ आयुः॑ । आयु॑र् दधाति । द॒धा॒त्य॒सौ । अ॒सावा॑दि॒त्यः । आ॒दि॒त्यो न । न वि । व्य॑रोचत । अ॒रो॒च॒त॒ तस्मै᳚ । तस्मै॑ दे॒वाः । दे॒वाः प्राय॑श्चित्तिम् । प्राय॑श्चित्ति मैच्छन्न् । ऐ॒च्छ॒न् तस्मै᳚ । तस्मा॑ ए॒तम् । ए॒तꣳ सौ॒र्यम् । सौ॒र्यम् च॒रुम् । च॒रुम् निः । निर॑वपन्न् । अ॒व॒प॒न् तेन॑ । तेनै॒व । ए॒वास्मिन्न्॑ । अ॒स्मि॒न् रुच᳚म् \newline

\textbf{Jatai Paata} \newline

1. प्रति॑ तिष्ठति तिष्ठति॒ प्रति॒ प्रति॑ तिष्ठति । \newline
2. ति॒ष्ठ॒ति॒ घृ॒ते घृ॒ते ति॑ष्ठति तिष्ठति घृ॒ते । \newline
3. घृ॒ते भ॑वति भवति घृ॒ते घृ॒ते भ॑वति । \newline
4. भ॒व॒ त्यायु॒ रायु॑र् भवति भव॒ त्यायुः॑ । \newline
5. आयु॒र् वै वा आयु॒ रायु॒र् वै । \newline
6. वै घृ॒तम् घृ॒तं ॅवै वै घृ॒तम् । \newline
7. घृ॒त म॒मृत॑ म॒मृत॑म् घृ॒तम् घृ॒त म॒मृत᳚म् । \newline
8. अ॒मृतꣳ॒॒ हिर॑ण्यꣳ॒॒ हिर॑ण्य म॒मृत॑ म॒मृतꣳ॒॒ हिर॑ण्यम् । \newline
9. हिर॑ण्य॒ मायु॒ रायु॒र्॒. हिर॑ण्यꣳ॒॒ हिर॑ण्य॒ मायुः॑ । \newline
10. आयु॑श्च॒ चायु॒ रायु॑श्च । \newline
11. चै॒वैव च॑ चै॒व । \newline
12. ए॒वास्मा॑ अस्मा ए॒वैवास्मै᳚ । \newline
13. अ॒स्मा॒ अ॒मृत॑ म॒मृत॑ मस्मा अस्मा अ॒मृत᳚म् । \newline
14. अ॒मृत॑म् च चा॒मृत॑ म॒मृत॑म् च । \newline
15. च॒ स॒मीची॑ स॒मीची॑ च च स॒मीची᳚ । \newline
16. स॒मीची॑ दधाति दधाति स॒मीची॑ स॒मीची॑ दधाति । \newline
17. स॒मीची॒ इति॑ स॒मीची᳚ । \newline
18. द॒धा॒ति॒ च॒त्वारि॑चत्वारि च॒त्वारि॑चत्वारि दधाति दधाति च॒त्वारि॑चत्वारि । \newline
19. च॒त्वारि॑चत्वारि कृ॒ष्णला॑नि कृ॒ष्णला॑नि च॒त्वारि॑चत्वारि च॒त्वारि॑चत्वारि कृ॒ष्णला॑नि । \newline
20. च॒त्वारि॑चत्वा॒रीति॑ च॒त्वारि॑ - च॒त्वा॒रि॒ । \newline
21. कृ॒ष्णला॒न्यवाव॑ कृ॒ष्णला॑नि कृ॒ष्णला॒न्यव॑ । \newline
22. अव॑ द्यति द्य॒त्यवाव॑ द्यति । \newline
23. द्य॒ति॒ च॒तु॒र॒व॒त्तस्य॑ चतुरव॒त्तस्य॑ द्यति द्यति चतुरव॒त्तस्य॑ । \newline
24. च॒तु॒र॒व॒त्तस्याप्त्या॒ आप्त्यै॑ चतुरव॒त्तस्य॑ चतुरव॒त्तस्याप्त्यै᳚ । \newline
25. च॒तु॒र॒व॒त्तस्येति॑ चतुः - अ॒व॒त्तस्य॑ । \newline
26. आप्त्या॑ एक॒धैक॒धा ऽऽप्त्या॒ आप्त्या॑ एक॒धा । \newline
27. ए॒क॒धा ब्र॒ह्मणे᳚ ब्र॒ह्मण॑ एक॒ धैक॒धा ब्र॒ह्मणे᳚ । \newline
28. ए॒क॒धेत्ये॑क - धा । \newline
29. ब्र॒ह्मण॒ उपोप॑ ब्र॒ह्मणे᳚ ब्र॒ह्मण॒ उप॑ । \newline
30. उप॑ हरति हर॒ त्युपोप॑ हरति । \newline
31. ह॒ र॒त्ये॒क॒ धैक॒धा ह॑रति हर त्येक॒धा । \newline
32. ए॒क॒ धैवै वैक॒ धैक॒धैव । \newline
33. ए॒क॒धेत्ये॑क - धा । \newline
34. ए॒व यज॑माने॒ यज॑मान ए॒वैव यज॑माने । \newline
35. यज॑मान॒ आयु॒ रायु॒र् यज॑माने॒ यज॑मान॒ आयुः॑ । \newline
36. आयु॑र् दधाति दधा॒ त्यायु॒ रायु॑र् दधाति । \newline
37. द॒धा॒त्य॒सा व॒सौ द॑धाति दधात्य॒सौ । \newline
38. अ॒सा वा॑दि॒त्य आ॑दि॒त्यो॑ ऽसा व॒सा वा॑दि॒त्यः । \newline
39. आ॒दि॒त्यो न नादि॒त्य आ॑दि॒त्यो न । \newline
40. न वि वि न न वि । \newline
41. व्य॑रोचता रोचत॒ वि व्य॑रोचत । \newline
42. अ॒रो॒च॒त॒ तस्मै॒ तस्मा॑ अरोचता रोचत॒ तस्मै᳚ । \newline
43. तस्मै॑ दे॒वा दे॒वा स्तस्मै॒ तस्मै॑ दे॒वाः । \newline
44. दे॒वाः प्राय॑श्चित्ति॒म् प्राय॑श्चित्तिम् दे॒वा दे॒वाः प्राय॑श्चित्तिम् । \newline
45. प्राय॑श्चित्ति मैच्छन् नैच्छ॒न् प्राय॑श्चित्ति॒म् प्राय॑श्चित्ति मैच्छन्न् । \newline
46. ऐ॒च्छ॒न् तस्मै॒ तस्मा॑ ऐच्छन् नैच्छ॒न् तस्मै᳚ । \newline
47. तस्मा॑ ए॒त मे॒तम् तस्मै॒ तस्मा॑ ए॒तम् । \newline
48. ए॒तꣳ सौ॒र्यꣳ सौ॒र्य मे॒त मे॒तꣳ सौ॒र्यम् । \newline
49. सौ॒र्यम् च॒रुम् च॒रुꣳ सौ॒र्यꣳ सौ॒र्यम् च॒रुम् । \newline
50. च॒रुम् निर् णिश्च॒रुम् च॒रुम् निः । \newline
51. निर॑वपन् नवप॒न् निर् णि र॑वपन्न् । \newline
52. अ॒व॒प॒न् तेन॒ तेना॑वपन् नवप॒न् तेन॑ । \newline
53. तेनै॒वैव तेन॒ तेनै॒व । \newline
54. ए॒वास्मि॑न् नस्मिन् ने॒वैवास्मिन्न्॑ । \newline
55. अ॒स्मि॒न् रुचꣳ॒॒ रुच॑ मस्मिन् नस्मि॒न् रुच᳚म् । \newline

\textbf{Ghana Paata } \newline

1. प्रति॑ तिष्ठति तिष्ठति॒ प्रति॒ प्रति॑ तिष्ठति घृ॒ते घृ॒ते ति॑ष्ठति॒ प्रति॒ प्रति॑ तिष्ठति घृ॒ते । \newline
2. ति॒ष्ठ॒ति॒ घृ॒ते घृ॒ते ति॑ष्ठति तिष्ठति घृ॒ते भ॑वति भवति घृ॒ते ति॑ष्ठति तिष्ठति घृ॒ते भ॑वति । \newline
3. घृ॒ते भ॑वति भवति घृ॒ते घृ॒ते भ॑व॒ त्यायु॒ रायु॑र् भवति घृ॒ते घृ॒ते भ॑व॒ त्यायुः॑ । \newline
4. भ॒व॒ त्यायु॒ रायु॑र् भवति भव॒ त्यायु॒र् वै वा आयु॑र् भवति भव॒ त्यायु॒र् वै । \newline
5. आयु॒र् वै वा आयु॒ रायु॒र् वै घृ॒तम् घृ॒तं ॅवा आयु॒ रायु॒र् वै घृ॒तम् । \newline
6. वै घृ॒तम् घृ॒तं ॅवै वै घृ॒त म॒मृत॑ म॒मृत॑म् घृ॒तं ॅवै वै घृ॒त म॒मृत᳚म् । \newline
7. घृ॒त म॒मृत॑ म॒मृत॑म् घृ॒तम् घृ॒त म॒मृतꣳ॒॒ हिर॑ण्यꣳ॒॒ हिर॑ण्य म॒मृत॑म् घृ॒तम् घृ॒त म॒मृतꣳ॒॒ हिर॑ण्यम् । \newline
8. अ॒मृतꣳ॒॒ हिर॑ण्यꣳ॒॒ हिर॑ण्य म॒मृत॑ म॒मृतꣳ॒॒ हिर॑ण्य॒ मायु॒रायु॒र्॒. हिर॑ण्य म॒मृत॑ म॒मृतꣳ॒॒ हिर॑ण्य॒ मायुः॑ । \newline
9. हिर॑ण्य॒ मायु॒ रायु॒र्॒. हिर॑ण्यꣳ॒॒ हिर॑ण्य॒ मायु॑श्च॒ चायु॒र्॒. हिर॑ण्यꣳ॒॒ हिर॑ण्य॒ मायु॑श्च । \newline
10. आयु॑श्च॒ चायु॒ रायु॑श्चै॒वैव चायु॒ रायु॑श्चै॒व । \newline
11. चै॒वैव च॑ चै॒वास्मा॑ अस्मा ए॒व च॑ चै॒वास्मै᳚ । \newline
12. ए॒वास्मा॑ अस्मा ए॒वैवास्मा॑ अ॒मृत॑ म॒मृत॑ मस्मा ए॒वैवास्मा॑ अ॒मृत᳚म् । \newline
13. अ॒स्मा॒ अ॒मृत॑ म॒मृत॑ मस्मा अस्मा अ॒मृत॑म् च चा॒मृत॑ मस्मा अस्मा अ॒मृत॑म् च । \newline
14. अ॒मृत॑म् च चा॒मृत॑ म॒मृत॑म् च स॒मीची॑ स॒मीची॑ चा॒मृत॑ म॒मृत॑म् च स॒मीची᳚ । \newline
15. च॒ स॒मीची॑ स॒मीची॑ च च स॒मीची॑ दधाति दधाति स॒मीची॑ च च स॒मीची॑ दधाति । \newline
16. स॒मीची॑ दधाति दधाति स॒मीची॑ स॒मीची॑ दधाति च॒त्वारि॑चत्वारि च॒त्वारि॑चत्वारि दधाति स॒मीची॑ स॒मीची॑ दधाति च॒त्वारि॑चत्वारि । \newline
17. स॒मीची॒ इति॑ स॒मीची᳚ । \newline
18. द॒धा॒ति॒ च॒त्वारि॑चत्वारि च॒त्वारि॑चत्वारि दधाति दधाति च॒त्वारि॑चत्वारि कृ॒ष्णला॑नि कृ॒ष्णला॑नि च॒त्वारि॑चत्वारि दधाति दधाति च॒त्वारि॑चत्वारि कृ॒ष्णला॑नि । \newline
19. च॒त्वारि॑चत्वारि कृ॒ष्णला॑नि कृ॒ष्णला॑नि च॒त्वारि॑चत्वारि च॒त्वारि॑चत्वारि कृ॒ष्णला॒न्यवाव॑ कृ॒ष्णला॑नि च॒त्वारि॑चत्वारि च॒त्वारि॑चत्वारि कृ॒ष्णला॒न्यव॑ । \newline
20. च॒त्वारि॑चत्वा॒रीति॑ च॒त्वारि॑ - च॒त्वा॒रि॒ । \newline
21. कृ॒ष्णला॒न्यवाव॑ कृ॒ष्णला॑नि कृ॒ष्णला॒न्यव॑ द्यति द्य॒त्यव॑ कृ॒ष्णला॑नि कृ॒ष्णला॒न्यव॑ द्यति । \newline
22. अव॑ द्यति द्य॒त्यवाव॑ द्यति चतुरव॒त्तस्य॑ चतुरव॒त्तस्य॑ द्य॒त्यवाव॑ द्यति चतुरव॒त्तस्य॑ । \newline
23. द्य॒ति॒ च॒तु॒र॒व॒त्तस्य॑ चतुरव॒त्तस्य॑ द्यति द्यति चतुरव॒त्त स्याप्त्या॒ आप्त्यै॑ चतुरव॒त्तस्य॑ द्यति द्यति चतुरव॒त्त स्याप्त्यै᳚ । \newline
24. च॒तु॒र॒व॒त्त स्याप्त्या॒ आप्त्यै॑ चतुरव॒त्तस्य॑ चतुरव॒त्त स्याप्त्या॑ एक॒धैक॒धा ऽऽप्त्यै॑ चतुरव॒त्तस्य॑ चतुरव॒त्त स्याप्त्या॑ एक॒धा । \newline
25. च॒तु॒र॒व॒त्तस्येति॑ चतुः - अ॒व॒त्तस्य॑ । \newline
26. आप्त्या॑ एक॒धैक॒धा ऽऽप्त्या॒ आप्त्या॑ एक॒धा ब्र॒ह्मणे᳚ ब्र॒ह्मण॑ एक॒धा ऽऽप्त्या॒ आप्त्या॑ एक॒धा ब्र॒ह्मणे᳚ । \newline
27. ए॒क॒धा ब्र॒ह्मणे᳚ ब्र॒ह्मण॑ एक॒धैक॒धा ब्र॒ह्मण॒ उपोप॑ ब्र॒ह्मण॑ एक॒धैक॒धा ब्र॒ह्मण॒ उप॑ । \newline
28. ए॒क॒धेत्ये॑क - धा । \newline
29. ब्र॒ह्मण॒ उपोप॑ ब्र॒ह्मणे᳚ ब्र॒ह्मण॒ उप॑ हरति हर॒त्युप॑ ब्र॒ह्मणे᳚ ब्र॒ह्मण॒ उप॑ हरति । \newline
30. उप॑ हरति हर॒ त्युपोप॑ हर त्येक॒धैक॒धा ह॑र॒ त्युपोप॑ हर त्येक॒धा । \newline
31. ह॒र॒ त्ये॒क॒धैक॒धा ह॑रति हर त्येक॒धैवैवैक॒धा ह॑रति हर त्येक॒धैव । \newline
32. ए॒क॒धै वैवैक॒धैक॒धैव यज॑माने॒ यज॑मान ए॒वैक॒धैक॒धैव यज॑माने । \newline
33. ए॒क॒धेत्ये॑क - धा । \newline
34. ए॒व यज॑माने॒ यज॑मान ए॒वैव यज॑मान॒ आयु॒ रायु॒र् यज॑मान ए॒वैव यज॑मान॒ आयुः॑ । \newline
35. यज॑मान॒ आयु॒ रायु॒र् यज॑माने॒ यज॑मान॒ आयु॑र् दधाति दधा॒ त्यायु॒र् यज॑माने॒ यज॑मान॒ आयु॑र् दधाति । \newline
36. आयु॑र् दधाति दधा॒ त्यायु॒ रायु॑र् दधा त्य॒सा व॒सौ द॑धा॒ त्यायु॒ रायु॑र् दधात्य॒सौ । \newline
37. द॒धा॒ त्य॒सा व॒सौ द॑धाति दधा त्य॒सा वा॑दि॒त्य आ॑दि॒त्यो॑ ऽसौ द॑धाति दधा त्य॒सा वा॑दि॒त्यः । \newline
38. अ॒सा वा॑दि॒त्य आ॑दि॒त्यो॑ ऽसा व॒सा वा॑दि॒त्यो न नादि॒त्यो॑ ऽसा व॒सा वा॑दि॒त्यो न । \newline
39. आ॒दि॒त्यो न नादि॒त्य आ॑दि॒त्यो न वि वि नादि॒त्य आ॑दि॒त्यो न वि । \newline
40. न वि वि न न व्य॑रोचता रोचत॒ वि न न व्य॑रोचत । \newline
41. व्य॑रोचता रोचत॒ वि व्य॑रोचत॒ तस्मै॒ तस्मा॑ अरोचत॒ वि व्य॑रोचत॒ तस्मै᳚ । \newline
42. अ॒रो॒च॒त॒ तस्मै॒ तस्मा॑ अरोचता रोचत॒ तस्मै॑ दे॒वा दे॒वा स्तस्मा॑ अरोचता रोचत॒ तस्मै॑ दे॒वाः । \newline
43. तस्मै॑ दे॒वा दे॒वा स्तस्मै॒ तस्मै॑ दे॒वाः प्राय॑श्चित्ति॒म् प्राय॑श्चित्तिम् दे॒वा स्तस्मै॒ तस्मै॑ दे॒वाः प्राय॑श्चित्तिम् । \newline
44. दे॒वाः प्राय॑श्चित्ति॒म् प्राय॑श्चित्तिम् दे॒वा दे॒वाः प्राय॑श्चित्ति मैच्छन् नैच्छ॒न् प्राय॑श्चित्तिम् दे॒वा दे॒वाः प्राय॑श्चित्ति मैच्छन्न् । \newline
45. प्राय॑श्चित्ति मैच्छन् नैच्छ॒न् प्राय॑श्चित्ति॒म् प्राय॑श्चित्ति मैच्छ॒न् तस्मै॒ तस्मा॑ ऐच्छ॒न् प्राय॑श्चित्ति॒म् प्राय॑श्चित्ति मैच्छ॒न् तस्मै᳚ । \newline
46. ऐ॒च्छ॒न् तस्मै॒ तस्मा॑ ऐच्छन् नैच्छ॒न् तस्मा॑ ए॒त मे॒तम् तस्मा॑ ऐच्छन् नैच्छ॒न् तस्मा॑ ए॒तम् । \newline
47. तस्मा॑ ए॒त मे॒तम् तस्मै॒ तस्मा॑ ए॒तꣳ सौ॒र्यꣳ सौ॒र्य मे॒तम् तस्मै॒ तस्मा॑ ए॒तꣳ सौ॒र्यम् । \newline
48. ए॒तꣳ सौ॒र्यꣳ सौ॒र्य मे॒त मे॒तꣳ सौ॒र्यम् च॒रुम् च॒रुꣳ सौ॒र्य मे॒त मे॒तꣳ सौ॒र्यम् च॒रुम् । \newline
49. सौ॒र्यम् च॒रुम् च॒रुꣳ सौ॒र्यꣳ सौ॒र्यम् च॒रुम् निर् णिश्च॒रुꣳ सौ॒र्यꣳ सौ॒र्यम् च॒रुम् निः । \newline
50. च॒रुम् निर् णिश्च॒रुम् च॒रुम् निर॑वपन् नवप॒न् निश्च॒रुम् च॒रुम् निर॑वपन्न् । \newline
51. निर॑वपन् नवप॒न् निर् णिर॑वप॒न् तेन॒ तेना॑वप॒न् निर् णिर॑वप॒न् तेन॑ । \newline
52. अ॒व॒प॒न् तेन॒ तेना॑वपन् नवप॒न् तेनै॒वैव तेना॑वपन् नवप॒न् तेनै॒व । \newline
53. तेनै॒वैव तेन॒ तेनै॒वास्मि॑न् नस्मिन् ने॒व तेन॒ तेनै॒वास्मिन्न्॑ । \newline
54. ए॒वास्मि॑न् नस्मिन् ने॒वैवास्मि॒न् रुचꣳ॒॒ रुच॑ मस्मिन् ने॒वैवास्मि॒न् रुच᳚म् । \newline
55. अ॒स्मि॒न् रुचꣳ॒॒ रुच॑ मस्मिन् नस्मि॒न् रुच॑ मदधु रदधू॒ रुच॑ मस्मिन् नस्मि॒न् रुच॑ मदधुः । \newline
\pagebreak
\markright{ TS 2.3.2.3  \hfill https://www.vedavms.in \hfill}

\section{ TS 2.3.2.3 }

\textbf{TS 2.3.2.3 } \newline
\textbf{Samhita Paata} \newline

रुच॑मदधु॒र्यो ब्र॑ह्मवर्च॒सका॑मः॒ स्यात् तस्मा॑ ए॒तꣳ सौ॒र्यं च॒रुं निर्व॑पेद॒मुमे॒वाऽऽदि॒त्यꣳ स्वेन॑ भाग॒धेये॒नोप॑ धावति॒ स ए॒वास्मि॑न् ब्रह्मवर्च॒सं द॑धाति ब्रह्मवर्च॒स्ये॑व भ॑वत्युभ॒यतो॑ रु॒क्मौ भ॑वत उभ॒यत॑ ए॒वास्मि॒न् रुचं॑ दधाति प्रया॒जे प्र॑याजे कृ॒ष्णलं॑ जुहोति दि॒ग्भ्य ए॒वास्मै᳚ ब्रह्मवर्च॒समव॑ रुन्ध आग्ने॒यम॒ष्टाक॑पालं॒ निर्व॑पेथ् सावि॒त्रं द्वाद॑शकपालं॒ भूम्यै॑ - [  ] \newline

\textbf{Pada Paata} \newline

रुच᳚म् । अ॒द॒धुः॒ । यः । ब्र॒ह्म॒व॒र्च॒सका॑म॒ इति॑ ब्रह्मवर्च॒स - का॒मः॒ । स्यात् । तस्मै᳚ । ए॒तम् । सौ॒र्यम् । च॒रुम् । निरिति॑ । व॒पे॒त् । अ॒मुम् । ए॒व । आ॒दि॒त्यम् । स्वेन॑ । भा॒ग॒धेये॒नेति॑ भाग - धेये॑न । उपेति॑ । धा॒व॒ति॒ । सः । ए॒व । अ॒स्मि॒न्न् । ब्र॒ह्म॒व॒र्च॒समिति॑ ब्रह्म - व॒र्च॒सम् । द॒धा॒ति॒ । ब्र॒ह्म॒व॒र्च॒सीति॑ ब्रह्म - व॒र्च॒सी । ए॒व । भ॒व॒ति॒ । उ॒भ॒यतः॑ । रु॒क्मौ । भ॒व॒तः॒ । उ॒भ॒यतः॑ । ए॒व । अ॒स्मि॒न्न् । रुच᳚म् । द॒धा॒ति॒ । प्र॒या॒जेप्र॑याज॒ इति॑ प्रया॒जे - प्र॒या॒जे॒ । कृ॒ष्णल᳚म् । जु॒हो॒ति॒ । दि॒ग्‌भ्य इति॑ दिक् - भ्यः । ए॒व । अ॒स्मै॒ । ब्र॒ह्म॒व॒र्च॒समिति॑ ब्रह्म - व॒र्च॒सम् । अवेति॑ । रु॒न्धे॒ । आ॒ग्ने॒यम् । अ॒ष्टाक॑पाल॒मित्य॒ष्टा - क॒पा॒ल॒म् । निरिति॑ । व॒पे॒त् । सा॒वि॒त्रम् । द्वाद॑शकपाल॒मिति॒ द्वाद॑श - क॒पा॒ल॒म् । भूम्यै᳚ ।  \newline


\textbf{Krama Paata} \newline

रुच॑मदधुः । अ॒द॒धु॒र् यः । यो ब्र॑ह्मवर्च॒सका॑मः । ब्र॒ह्म॒व॒र्च॒सका॑मः॒ स्यात् । ब्र॒ह्म॒व॒र्च॒सका॑म॒ इति॑ ब्रह्मवर्च॒स - का॒मः॒ । स्यात् तस्मै᳚ । तस्मा॑ ए॒तम् । ए॒तꣳ सौ॒र्यम् । सौ॒र्यम् च॒रुम् । च॒रुम् निः । निर् व॑पेत् । व॒पे॒द॒मुम् । अ॒मुमे॒व । ए॒वादि॒त्यम् । आ॒दि॒त्यꣳ स्वेन॑ । स्वेन॑ भाग॒धेये॑न । भा॒ग॒धेये॒नोप॑ । भा॒ग॒धेये॒नेति॑ भाग - धेये॑न । उप॑ धावति । धा॒व॒ति॒ सः । स ए॒व । ए॒वास्मिन्न्॑ । अ॒स्मि॒न् ब्र॒ह्म॒व॒र्च॒सम् । ब्र॒ह्म॒व॒र्च॒सम् द॑धाति । ब्र॒ह्म॒व॒र्च॒समिति॑ ब्रह्म - व॒र्च॒सम् । द॒धा॒ति॒ ब्र॒ह्म॒व॒र्च॒सी । ब्र॒ह्म॒व॒र्च॒स्ये॑व । ब्र॒ह्म॒व॒र्च॒सीति॑ ब्रह्म - व॒र्च॒सी । ए॒व भ॑वति । भ॒व॒त्यु॒भ॒यतः॑ । उ॒भ॒यतो॑ रु॒क्मौ । रु॒क्मौ भ॑वतः । भ॒व॒त॒ उ॒भ॒यतः॑ । उ॒भ॒यत॑ ए॒व । ए॒वास्मिन्न्॑ । अ॒स्मि॒न् रुच᳚म् । रुच॑म् दधाति । द॒धा॒ति॒ प्र॒या॒जेप्र॑याजे । प्र॒या॒जेप्र॑याजे कृ॒ष्णल᳚म् । प्र॒या॒जेप्र॑याज॒ इति॑ प्रया॒जे - प्र॒या॒जे॒ । कृ॒ष्णल॑म् जुहोति । जु॒हो॒ति॒ दि॒ग्भ्यः । दि॒ग्भ्य ए॒व । दि॒ग्भ्य इति॑ दिक् - भ्यः । ए॒वास्मै᳚ । अ॒स्मै॒ ब्र॒ह्म॒व॒र्च॒सम् । ब्र॒ह्म॒व॒र्च॒समव॑ । ब्र॒ह्म॒व॒र्च॒समिति॑ ब्रह्म - व॒र्च॒सम् । अव॑ रुन्धे । रु॒न्ध॒ आ॒ग्ने॒यम् । आ॒ग्ने॒यम॒ष्टाक॑पालम् । अ॒ष्टाक॑पाल॒म् निः । अ॒ष्टाक॑पाल॒मित्य॒ष्टा - क॒पा॒ल॒म् । निर् व॑पेत् । व॒पे॒थ् सा॒वि॒त्रम् । सा॒वि॒त्रम् द्वाद॑शकपालम् । द्वाद॑शकपाल॒म् भूम्यै᳚ । द्वाद॑शकपाल॒मिति॒ द्वाद॑श - क॒पा॒ल॒म् । भूम्यै॑ च॒रुम् \newline

\textbf{Jatai Paata} \newline

1. रुच॑ मदधु रदधू॒ रुचꣳ॒॒ रुच॑ मदधुः । \newline
2. अ॒द॒धु॒र् यो यो॑ ऽदधु रदधु॒र् यः । \newline
3. यो ब्र॑ह्मवर्च॒सका॑मो ब्रह्मवर्च॒सका॑मो॒ यो यो ब्र॑ह्मवर्च॒सका॑मः । \newline
4. ब्र॒ह्म॒व॒र्च॒सका॑मः॒ स्याथ् स्याद् ब्र॑ह्मवर्च॒सका॑मो ब्रह्मवर्च॒सका॑मः॒ स्यात् । \newline
5. ब्र॒ह्म॒व॒र्च॒सका॑म॒ इति॑ ब्रह्मवर्च॒स - का॒मः॒ । \newline
6. स्यात् तस्मै॒ तस्मै॒ स्याथ् स्यात् तस्मै᳚ । \newline
7. तस्मा॑ ए॒त मे॒तम् तस्मै॒ तस्मा॑ ए॒तम् । \newline
8. ए॒तꣳ सौ॒र्यꣳ सौ॒र्य मे॒त मे॒तꣳ सौ॒र्यम् । \newline
9. सौ॒र्यम् च॒रुम् च॒रुꣳ सौ॒र्यꣳ सौ॒र्यम् च॒रुम् । \newline
10. च॒रुम् निर् णिश्च॒रुम् च॒रुम् निः । \newline
11. निर् व॑पेद् वपे॒न् निर् णिर् व॑पेत् । \newline
12. व॒पे॒ द॒मु म॒मुं ॅव॑पेद् वपे द॒मुम् । \newline
13. अ॒मु मे॒वैवामु म॒मु मे॒व । \newline
14. ए॒वादि॒त्य मा॑दि॒त्य मे॒वै वादि॒त्यम् । \newline
15. आ॒दि॒त्यꣳ स्वेन॒ स्वेना॑दि॒त्य मा॑दि॒त्यꣳ स्वेन॑ । \newline
16. स्वेन॑ भाग॒धेये॑न भाग॒धेये॑न॒ स्वेन॒ स्वेन॑ भाग॒धेये॑न । \newline
17. भा॒ग॒धेये॒नोपोप॑ भाग॒धेये॑न भाग॒धेये॒नोप॑ । \newline
18. भा॒ग॒धेये॒नेति॑ भाग - धेये॑न । \newline
19. उप॑ धावति धाव॒ त्युपोप॑ धावति । \newline
20. धा॒व॒ति॒ स स धा॑वति धावति॒ सः । \newline
21. स ए॒वैव स स ए॒व । \newline
22. ए॒वास्मि॑न् नस्मिन् ने॒वैवास्मिन्न्॑ । \newline
23. अ॒स्मि॒न् ब्र॒ह्म॒व॒र्च॒सम् ब्र॑ह्मवर्च॒स म॑स्मिन् नस्मिन् ब्रह्मवर्च॒सम् । \newline
24. ब्र॒ह्म॒व॒र्च॒सम् द॑धाति दधाति ब्रह्मवर्च॒सम् ब्र॑ह्मवर्च॒सम् द॑धाति । \newline
25. ब्र॒ह्म॒व॒र्च॒समिति॑ ब्रह्म - व॒र्च॒सम् । \newline
26. द॒धा॒ति॒ ब्र॒ह्म॒व॒र्च॒सी ब्र॑ह्मवर्च॒सी द॑धाति दधाति ब्रह्मवर्च॒सी । \newline
27. ब्र॒ह्म॒व॒र्च॒ स्ये॑वैव ब्र॑ह्मवर्च॒सी ब्र॑ह्मवर्च॒ स्ये॑व । \newline
28. ब्र॒ह्म॒व॒र्च॒सीति॑ ब्रह्म - व॒र्च॒सी । \newline
29. ए॒व भ॑वति भव त्ये॒वैव भ॑वति । \newline
30. भ॒व॒ त्यु॒भ॒यत॑ उभ॒यतो॑ भवति भव त्युभ॒यतः॑ । \newline
31. उ॒भ॒यतो॑ रु॒क्मौ रु॒क्मा वु॑भ॒यत॑ उभ॒यतो॑ रु॒क्मौ । \newline
32. रु॒क्मौ भ॑वतो भवतो रु॒क्मौ रु॒क्मौ भ॑वतः । \newline
33. भ॒व॒त॒ उ॒भ॒यत॑ उभ॒यतो॑ भवतो भवत उभ॒यतः॑ । \newline
34. उ॒भ॒यत॑ ए॒वैवो भ॒यत॑ उभ॒यत॑ ए॒व । \newline
35. ए॒वास्मि॑न् नस्मिन् ने॒वैवास्मिन्न्॑ । \newline
36. अ॒स्मि॒न् रुचꣳ॒॒ रुच॑ मस्मिन् नस्मि॒न् रुच᳚म् । \newline
37. रुच॑म् दधाति दधाति॒ रुचꣳ॒॒ रुच॑म् दधाति । \newline
38. द॒धा॒ति॒ प्र॒या॒जेप्र॑याजे प्रया॒जेप्र॑याजे दधाति दधाति प्रया॒जेप्र॑याजे । \newline
39. प्र॒या॒जेप्र॑याजे कृ॒ष्णल॑म् कृ॒ष्णल॑म् प्रया॒जेप्र॑याजे प्रया॒जेप्र॑याजे कृ॒ष्णल᳚म् । \newline
40. प्र॒या॒जेप्र॑याज॒ इति॑ प्रया॒जे - प्र॒या॒जे॒ । \newline
41. कृ॒ष्णल॑म् जुहोति जुहोति कृ॒ष्णल॑म् कृ॒ष्णल॑म् जुहोति । \newline
42. जु॒हो॒ति॒ दि॒ग्भ्यो दि॒ग्भ्यो जु॑होति जुहोति दि॒ग्भ्यः । \newline
43. दि॒ग्भ्य ए॒वैव दि॒ग्भ्यो दि॒ग्भ्य ए॒व । \newline
44. दि॒ग्भ्य इति॑ दिक् - भ्यः । \newline
45. ए॒वास्मा॑ अस्मा ए॒वैवास्मै᳚ । \newline
46. अ॒स्मै॒ ब्र॒ह्म॒व॒र्च॒सम् ब्र॑ह्मवर्च॒स म॑स्मा अस्मै ब्रह्मवर्च॒सम् । \newline
47. ब्र॒ह्म॒व॒र्च॒स मवाव॑ ब्रह्मवर्च॒सम् ब्र॑ह्मवर्च॒स मव॑ । \newline
48. ब्र॒ह्म॒व॒र्च॒समिति॑ ब्रह्म - व॒र्च॒सम् । \newline
49. अव॑ रुन्धे रु॒न्धे ऽवाव॑ रुन्धे । \newline
50. रु॒न्ध॒ आ॒ग्ने॒य मा᳚ग्ने॒यꣳ रु॑न्धे रुन्ध आग्ने॒यम् । \newline
51. आ॒ग्ने॒य म॒ष्टाक॑पाल म॒ष्टाक॑पाल माग्ने॒य मा᳚ग्ने॒य म॒ष्टाक॑पालम् । \newline
52. अ॒ष्टाक॑पाल॒म् निर् णिर॒ष्टाक॑पाल म॒ष्टाक॑पाल॒म् निः । \newline
53. अ॒ष्टाक॑पाल॒मित्य॒ष्टा - क॒पा॒ल॒म् । \newline
54. निर् व॑पेद् वपे॒न् निर् णिर् व॑पेत् । \newline
55. व॒पे॒थ् सा॒वि॒त्रꣳ सा॑वि॒त्रं ॅव॑पेद् वपेथ् सावि॒त्रम् । \newline
56. सा॒वि॒त्रम् द्वाद॑शकपाल॒म् द्वाद॑शकपालꣳ सावि॒त्रꣳ सा॑वि॒त्रम् द्वाद॑शकपालम् । \newline
57. द्वाद॑शकपाल॒म् भूम्यै॒ भूम्यै॒ द्वाद॑शकपाल॒म् द्वाद॑शकपाल॒म् भूम्यै᳚ । \newline
58. द्वाद॑शकपाल॒मिति॒ द्वाद॑श - क॒पा॒ल॒म् । \newline
59. भूम्यै॑ च॒रुम् च॒रुम् भूम्यै॒ भूम्यै॑ च॒रुम् । \newline

\textbf{Ghana Paata } \newline

1. रुच॑ मदधु रदधू॒ रुचꣳ॒॒ रुच॑ मदधु॒र् यो यो॑ ऽदधू॒ रुचꣳ॒॒ रुच॑ मदधु॒र् यः । \newline
2. अ॒द॒धु॒र् यो यो॑ ऽदधु रदधु॒र् यो ब्र॑ह्मवर्च॒सका॑मो ब्रह्मवर्च॒सका॑मो॒ यो॑ ऽदधु रदधु॒र् यो ब्र॑ह्मवर्च॒सका॑मः । \newline
3. यो ब्र॑ह्मवर्च॒सका॑मो ब्रह्मवर्च॒सका॑मो॒ यो यो ब्र॑ह्मवर्च॒सका॑मः॒ स्याथ् स्याद् ब्र॑ह्मवर्च॒सका॑मो॒ यो यो ब्र॑ह्मवर्च॒सका॑मः॒ स्यात् । \newline
4. ब्र॒ह्म॒व॒र्च॒सका॑मः॒ स्याथ् स्याद् ब्र॑ह्मवर्च॒सका॑मो ब्रह्मवर्च॒सका॑मः॒ स्यात् तस्मै॒ तस्मै॒ स्याद् ब्र॑ह्मवर्च॒सका॑मो ब्रह्मवर्च॒सका॑मः॒ स्यात् तस्मै᳚ । \newline
5. ब्र॒ह्म॒व॒र्च॒सका॑म॒ इति॑ ब्रह्मवर्च॒स - का॒मः॒ । \newline
6. स्यात् तस्मै॒ तस्मै॒ स्याथ् स्यात् तस्मा॑ ए॒त मे॒तम् तस्मै॒ स्याथ् स्यात् तस्मा॑ ए॒तम् । \newline
7. तस्मा॑ ए॒त मे॒तम् तस्मै॒ तस्मा॑ ए॒तꣳ सौ॒र्यꣳ सौ॒र्य मे॒तम् तस्मै॒ तस्मा॑ ए॒तꣳ सौ॒र्यम् । \newline
8. ए॒तꣳ सौ॒र्यꣳ सौ॒र्य मे॒त मे॒तꣳ सौ॒र्यम् च॒रुम् च॒रुꣳ सौ॒र्य मे॒त मे॒तꣳ सौ॒र्यम् च॒रुम् । \newline
9. सौ॒र्यम् च॒रुम् च॒रुꣳ सौ॒र्यꣳ सौ॒र्यम् च॒रुम् निर् णिश्च॒रुꣳ सौ॒र्यꣳ सौ॒र्यम् च॒रुम् निः । \newline
10. च॒रुम् निर् णिश्च॒रुम् च॒रुम् निर् व॑पेद् वपे॒न् निश्च॒रुम् च॒रुम् निर् व॑पेत् । \newline
11. निर् व॑पेद् वपे॒न् निर् णिर् व॑पे द॒मु म॒मुं ॅव॑पे॒न् निर् णिर् व॑पे द॒मुम् । \newline
12. व॒पे॒ द॒मु म॒मुं ॅव॑पेद् वपे द॒मु मे॒वैवामुं ॅव॑पेद् वपे द॒मु मे॒व । \newline
13. अ॒मु मे॒वैवामु म॒मु मे॒वादि॒त्य मा॑दि॒त्य मे॒वामु म॒मु मे॒वादि॒त्यम् । \newline
14. ए॒वादि॒त्य मा॑दि॒त्य मे॒वैवादि॒त्यꣳ स्वेन॒ स्वेना॑दि॒त्य मे॒वैवादि॒त्यꣳ स्वेन॑ । \newline
15. आ॒दि॒त्यꣳ स्वेन॒ स्वेना॑दि॒त्य मा॑दि॒त्यꣳ स्वेन॑ भाग॒धेये॑न भाग॒धेये॑न॒ स्वेना॑दि॒त्य मा॑दि॒त्यꣳ स्वेन॑ भाग॒धेये॑न । \newline
16. स्वेन॑ भाग॒धेये॑न भाग॒धेये॑न॒ स्वेन॒ स्वेन॑ भाग॒धेये॒नोपोप॑ भाग॒धेये॑न॒ स्वेन॒ स्वेन॑ भाग॒धेये॒नोप॑ । \newline
17. भा॒ग॒धेये॒नोपोप॑ भाग॒धेये॑न भाग॒धेये॒नोप॑ धावति धाव॒त्युप॑ भाग॒धेये॑न भाग॒धेये॒नोप॑ धावति । \newline
18. भा॒ग॒धेये॒नेति॑ भाग - धेये॑न । \newline
19. उप॑ धावति धाव॒ त्युपोप॑ धावति॒ स स धा॑व॒ त्युपोप॑ धावति॒ सः । \newline
20. धा॒व॒ति॒ स स धा॑वति धावति॒ स ए॒वैव स धा॑वति धावति॒ स ए॒व । \newline
21. स ए॒वैव स स ए॒वास्मि॑न् नस्मिन् ने॒व स स ए॒वास्मिन्न्॑ । \newline
22. ए॒वास्मि॑न् नस्मिन् ने॒वैवास्मि॑न् ब्रह्मवर्च॒सम् ब्र॑ह्मवर्च॒स म॑स्मिन् ने॒वैवास्मि॑न् ब्रह्मवर्च॒सम् । \newline
23. अ॒स्मि॒न् ब्र॒ह्म॒व॒र्च॒सम् ब्र॑ह्मवर्च॒स म॑स्मिन् नस्मिन् ब्रह्मवर्च॒सम् द॑धाति दधाति ब्रह्मवर्च॒स म॑स्मिन् नस्मिन् ब्रह्मवर्च॒सम् द॑धाति । \newline
24. ब्र॒ह्म॒व॒र्च॒सम् द॑धाति दधाति ब्रह्मवर्च॒सम् ब्र॑ह्मवर्च॒सम् द॑धाति ब्रह्मवर्च॒सी ब्र॑ह्मवर्च॒सी द॑धाति ब्रह्मवर्च॒सम् ब्र॑ह्मवर्च॒सम् द॑धाति ब्रह्मवर्च॒सी । \newline
25. ब्र॒ह्म॒व॒र्च॒समिति॑ ब्रह्म - व॒र्च॒सम् । \newline
26. द॒धा॒ति॒ ब्र॒ह्म॒व॒र्च॒सी ब्र॑ह्मवर्च॒सी द॑धाति दधाति ब्रह्मवर्च॒स्ये॑वैव ब्र॑ह्मवर्च॒सी द॑धाति दधाति ब्रह्मवर्च॒स्ये॑व । \newline
27. ब्र॒ह्म॒व॒र्च॒स्ये॑वैव ब्र॑ह्मवर्च॒सी ब्र॑ह्मवर्च॒स्ये॑व भ॑वति भवत्ये॒व ब्र॑ह्मवर्च॒सी ब्र॑ह्मवर्च॒स्ये॑व भ॑वति । \newline
28. ब्र॒ह्म॒व॒र्च॒सीति॑ ब्रह्म - व॒र्च॒सी । \newline
29. ए॒व भ॑वति भव त्ये॒वैव भ॑व त्युभ॒यत॑ उभ॒यतो॑ भव त्ये॒वैव भ॑व त्युभ॒यतः॑ । \newline
30. भ॒व॒ त्यु॒भ॒यत॑ उभ॒यतो॑ भवति भव त्युभ॒यतो॑ रु॒क्मौ रु॒क्मा वु॑भ॒यतो॑ भवति भव त्युभ॒यतो॑ रु॒क्मौ । \newline
31. उ॒भ॒यतो॑ रु॒क्मौ रु॒क्मा वु॑भ॒यत॑ उभ॒यतो॑ रु॒क्मौ भ॑वतो भवतो रु॒क्मा वु॑भ॒यत॑ उभ॒यतो॑ रु॒क्मौ भ॑वतः । \newline
32. रु॒क्मौ भ॑वतो भवतो रु॒क्मौ रु॒क्मौ भ॑वत उभ॒यत॑ उभ॒यतो॑ भवतो रु॒क्मौ रु॒क्मौ भ॑वत उभ॒यतः॑ । \newline
33. भ॒व॒त॒ उ॒भ॒यत॑ उभ॒यतो॑ भवतो भवत उभ॒यत॑ ए॒वैवोभ॒यतो॑ भवतो भवत उभ॒यत॑ ए॒व । \newline
34. उ॒भ॒यत॑ ए॒वैवोभ॒यत॑ उभ॒यत॑ ए॒वास्मि॑न् नस्मिन् ने॒वोभ॒यत॑ उभ॒यत॑ ए॒वास्मिन्न्॑ । \newline
35. ए॒वास्मि॑न् नस्मिन् ने॒वैवास्मि॒न् रुचꣳ॒॒ रुच॑ मस्मिन् ने॒वैवास्मि॒न् रुच᳚म् । \newline
36. अ॒स्मि॒न् रुचꣳ॒॒ रुच॑ मस्मिन् नस्मि॒न् रुच॑म् दधाति दधाति॒ रुच॑ मस्मिन् नस्मि॒न् रुच॑म् दधाति । \newline
37. रुच॑म् दधाति दधाति॒ रुचꣳ॒॒ रुच॑म् दधाति प्रया॒जेप्र॑याजे प्रया॒जेप्र॑याजे दधाति॒ रुचꣳ॒॒ रुच॑म् दधाति प्रया॒जेप्र॑याजे । \newline
38. द॒धा॒ति॒ प्र॒या॒जेप्र॑याजे प्रया॒जेप्र॑याजे दधाति दधाति प्रया॒जेप्र॑याजे कृ॒ष्णल॑म् कृ॒ष्णल॑म् प्रया॒जेप्र॑याजे दधाति दधाति प्रया॒जेप्र॑याजे कृ॒ष्णल᳚म् । \newline
39. प्र॒या॒जेप्र॑याजे कृ॒ष्णल॑म् कृ॒ष्णल॑म् प्रया॒जेप्र॑याजे प्रया॒जेप्र॑याजे कृ॒ष्णल॑म् जुहोति जुहोति कृ॒ष्णल॑म् प्रया॒जेप्र॑याजे प्रया॒जेप्र॑याजे कृ॒ष्णल॑म् जुहोति । \newline
40. प्र॒या॒जेप्र॑याज॒ इति॑ प्रया॒जे - प्र॒या॒जे॒ । \newline
41. कृ॒ष्णल॑म् जुहोति जुहोति कृ॒ष्णल॑म् कृ॒ष्णल॑म् जुहोति दि॒ग्भ्यो दि॒ग्भ्यो जु॑होति कृ॒ष्णल॑म् कृ॒ष्णल॑म् जुहोति दि॒ग्भ्यः । \newline
42. जु॒हो॒ति॒ दि॒ग्भ्यो दि॒ग्भ्यो जु॑होति जुहोति दि॒ग्भ्य ए॒वैव दि॒ग्भ्यो जु॑होति जुहोति दि॒ग्भ्य ए॒व । \newline
43. दि॒ग्भ्य ए॒वैव दि॒ग्भ्यो दि॒ग्भ्य ए॒वास्मा॑ अस्मा ए॒व दि॒ग्भ्यो दि॒ग्भ्य ए॒वास्मै᳚ । \newline
44. दि॒ग्भ्य इति॑ दिक् - भ्यः । \newline
45. ए॒वास्मा॑ अस्मा ए॒वैवास्मै᳚ ब्रह्मवर्च॒सम् ब्र॑ह्मवर्च॒स म॑स्मा ए॒वैवास्मै᳚ ब्रह्मवर्च॒सम् । \newline
46. अ॒स्मै॒ ब्र॒ह्म॒व॒र्च॒सम् ब्र॑ह्मवर्च॒स म॑स्मा अस्मै ब्रह्मवर्च॒स मवाव॑ ब्रह्मवर्च॒स म॑स्मा अस्मै ब्रह्मवर्च॒स मव॑ । \newline
47. ब्र॒ह्म॒व॒र्च॒स मवाव॑ ब्रह्मवर्च॒सम् ब्र॑ह्मवर्च॒स मव॑ रुन्धे रु॒न्धे ऽव॑ ब्रह्मवर्च॒सम् ब्र॑ह्मवर्च॒स मव॑ रुन्धे । \newline
48. ब्र॒ह्म॒व॒र्च॒समिति॑ ब्रह्म - व॒र्च॒सम् । \newline
49. अव॑ रुन्धे रु॒न्धे ऽवाव॑ रुन्ध आग्ने॒य मा᳚ग्ने॒यꣳ रु॒न्धे ऽवाव॑ रुन्ध आग्ने॒यम् । \newline
50. रु॒न्ध॒ आ॒ग्ने॒य मा᳚ग्ने॒यꣳ रु॑न्धे रुन्ध आग्ने॒य म॒ष्टाक॑पाल म॒ष्टाक॑पाल माग्ने॒यꣳ रु॑न्धे रुन्ध आग्ने॒य म॒ष्टाक॑पालम् । \newline
51. आ॒ग्ने॒य म॒ष्टाक॑पाल म॒ष्टाक॑पाल माग्ने॒य मा᳚ग्ने॒य म॒ष्टाक॑पाल॒म् निर् णिर॒ष्टाक॑पाल माग्ने॒य मा᳚ग्ने॒य म॒ष्टाक॑पाल॒म् निः । \newline
52. अ॒ष्टाक॑पाल॒म् निर् णिर॒ष्टाक॑पाल म॒ष्टाक॑पाल॒म् निर् व॑पेद् वपे॒न् निर॒ष्टाक॑पाल म॒ष्टाक॑पाल॒म् निर् व॑पेत् । \newline
53. अ॒ष्टाक॑पाल॒मित्य॒ष्टा - क॒पा॒ल॒म् । \newline
54. निर् व॑पेद् वपे॒न् निर् णिर् व॑पेथ् सावि॒त्रꣳ सा॑वि॒त्रं ॅव॑पे॒न् निर् णिर् व॑पेथ् सावि॒त्रम् । \newline
55. व॒पे॒थ् सा॒वि॒त्रꣳ सा॑वि॒त्रं ॅव॑पेद् वपेथ् सावि॒त्रम् द्वाद॑शकपाल॒म् द्वाद॑शकपालꣳ सावि॒त्रं ॅव॑पेद् वपेथ् सावि॒त्रम् द्वाद॑शकपालम् । \newline
56. सा॒वि॒त्रम् द्वाद॑शकपाल॒म् द्वाद॑शकपालꣳ सावि॒त्रꣳ सा॑वि॒त्रम् द्वाद॑शकपाल॒म् भूम्यै॒ भूम्यै॒ द्वाद॑शकपालꣳ सावि॒त्रꣳ सा॑वि॒त्रम् द्वाद॑शकपाल॒म् भूम्यै᳚ । \newline
57. द्वाद॑शकपाल॒म् भूम्यै॒ भूम्यै॒ द्वाद॑शकपाल॒म् द्वाद॑शकपाल॒म् भूम्यै॑ च॒रुम् च॒रुम् भूम्यै॒ द्वाद॑शकपाल॒म् द्वाद॑शकपाल॒म् भूम्यै॑ च॒रुम् । \newline
58. द्वाद॑शकपाल॒मिति॒ द्वाद॑श - क॒पा॒ल॒म् । \newline
59. भूम्यै॑ च॒रुम् च॒रुम् भूम्यै॒ भूम्यै॑ च॒रुं ॅयो यश्च॒रुम् भूम्यै॒ भूम्यै॑ च॒रुं ॅयः । \newline
\pagebreak
\markright{ TS 2.3.2.4  \hfill https://www.vedavms.in \hfill}

\section{ TS 2.3.2.4 }

\textbf{TS 2.3.2.4 } \newline
\textbf{Samhita Paata} \newline

च॒रुं ॅयः का॒मये॑त॒ हिर॑ण्यं ॅविन्देय॒ हिर॑ण्यं॒ मोप॑ नमे॒दिति॒ यदा᳚ग्ने॒यो भव॑त्याग्ने॒यं ॅवै हिर॑ण्यं॒ ॅयस्यै॒व हिर॑ण्यं॒ तेनै॒वैन॑द् विन्दते सावि॒त्रो भ॑वति सवि॒तृप्र॑सूत ए॒वैन॑द्-विन्दते॒ भूम्यै॑ च॒रुर्भ॑वत्य॒स्यामे॒वैन॑द्-विन्दत॒ उपै॑नꣳ॒॒ हिर॑ण्यं नमति॒ वि वा ए॒ष इ॑न्द्रि॒येण॑ वी॒र्ये॑णर्द्ध्यते॒ यो हिर॑ण्यं ॅवि॒न्दत॑ ए॒ता - [  ] \newline

\textbf{Pada Paata} \newline

च॒रुम् । यः । का॒मये॑त । हिर॑ण्यम् । वि॒न्दे॒य॒ । हिर॑ण्यम् । मा॒ । उपेति॑ । न॒मे॒त् । इति॑ । यत् । आ॒ग्ने॒यः । भव॑ति । आ॒ग्ने॒यम् । वै । हिर॑ण्यम् । यस्य॑ । ए॒व । हिर॑ण्यम् । तेन॑ । ए॒व । ए॒न॒त् । वि॒न्द॒ते॒ । सा॒वि॒त्रः । भ॒व॒ति॒ । स॒वि॒तृप्र॑सूत॒ इति॑ सवि॒तृ-प्र॒सू॒तः॒ । ए॒व । ए॒न॒त् । वि॒न्द॒ते॒ । भूम्यै᳚ । च॒रुः । भ॒व॒ति॒ । अ॒स्याम् । ए॒व । ए॒न॒त् । वि॒न्द॒ते॒ । उपेति॑ । ए॒न॒म् । हिर॑ण्यम् । न॒म॒ति॒ । वीति॑ । वै । ए॒षः । इ॒न्द्रि॒येण॑ । वी॒र्ये॑ण । ऋ॒द्ध्य॒ते॒ । यः । हिर॑ण्यम् । वि॒न्दते᳚ । ए॒ताम् ।  \newline


\textbf{Krama Paata} \newline

च॒रुं ॅयः । यः का॒मये॑त । का॒मये॑त॒ हिर॑ण्यम् । हिर॑ण्यं ॅविन्देय । वि॒न्दे॒य॒ हिर॑ण्यम् । हिर॑ण्य॒म् मा । मोप॑ । उप॑ नमेत् । न॒मे॒दिति॑ । इति॒ यत् । यदा᳚ग्ने॒यः । आ॒ग्ने॒यो भव॑ति । भव॑त्याग्ने॒यम् । आ॒ग्ने॒यं ॅवै । वै हिर॑ण्यम् । हिर॑ण्यं॒ ॅयस्य॑ । यस्यै॒व । ए॒व हिर॑ण्यम् । हिर॑ण्य॒म् तेन॑ । तेनै॒व । ए॒वैन॑त् । ए॒न॒द् वि॒न्द॒ते॒ । वि॒न्द॒ते॒ सा॒वि॒त्रः । सा॒वि॒त्रो भ॑वति । भ॒व॒ति॒ स॒वि॒तृप्र॑सूतः । स॒वि॒तृप्र॑सूत ए॒व । स॒वि॒तृप्र॑सूत॒ इति॑ सवि॒तृ - प्र॒सू॒तः॒ । ए॒वैन॑त् । ए॒न॒द् वि॒न्द॒ते॒ । वि॒न्द॒ते॒ भूम्यै᳚ । भूम्यै॑ च॒रुः । च॒रुर् भ॑वति । भ॒व॒त्य॒स्याम् । अ॒स्यामे॒व । ए॒वैन॑त् । ए॒न॒द् वि॒न्द॒ते॒ । वि॒न्द॒त॒ उप॑ । उपै॑नम् । ए॒नꣳ॒॒ हिर॑ण्यम् । हिर॑ण्यम् नमति । न॒म॒ति॒ वि । वि वै । वा ए॒षः । ए॒ष इ॑न्द्रि॒येण॑ । इ॒न्द्रि॒येण॑ वी॒र्ये॑ण । वी॒र्ये॑णर्द्ध्यते । ऋ॒द्ध्य॒ते॒ यः । यो हिर॑ण्यम् । हिर॑ण्यं ॅवि॒न्दते᳚ । वि॒न्दत॑ ए॒ताम् । ए॒तामे॒व \newline

\textbf{Jatai Paata} \newline

1. च॒रुं ॅयो य श्च॒रुम् च॒रुं ॅयः । \newline
2. यः का॒मये॑त का॒मये॑त॒ यो यः का॒मये॑त । \newline
3. का॒मये॑त॒ हिर॑ण्यꣳ॒॒ हिर॑ण्यम् का॒मये॑त का॒मये॑त॒ हिर॑ण्यम् । \newline
4. हिर॑ण्यं ॅविन्देय विन्देय॒ हिर॑ण्यꣳ॒॒ हिर॑ण्यं ॅविन्देय । \newline
5. वि॒न्दे॒य॒ हिर॑ण्यꣳ॒॒ हिर॑ण्यं ॅविन्देय विन्देय॒ हिर॑ण्यम् । \newline
6. हिर॑ण्यम् मा मा॒ हिर॑ण्यꣳ॒॒ हिर॑ण्यम् मा । \newline
7. मोपोप॑ मा॒ मोप॑ । \newline
8. उप॑ नमेन् नमे॒ दुपोप॑ नमेत् । \newline
9. न॒मे॒ दितीति॑ नमेन् नमे॒ दिति॑ । \newline
10. इति॒ यद् यदितीति॒ यत् । \newline
11. यदा᳚ग्ने॒य आ᳚ग्ने॒यो यद् यदा᳚ग्ने॒यः । \newline
12. आ॒ग्ने॒यो भव॑ति॒ भव॑ त्याग्ने॒य आ᳚ग्ने॒यो भव॑ति । \newline
13. भव॑ त्याग्ने॒य मा᳚ग्ने॒यम् भव॑ति॒ भव॑ त्याग्ने॒यम् । \newline
14. आ॒ग्ने॒यं ॅवै वा आ᳚ग्ने॒य मा᳚ग्ने॒यं ॅवै । \newline
15. वै हिर॑ण्यꣳ॒॒ हिर॑ण्यं॒ ॅवै वै हिर॑ण्यम् । \newline
16. हिर॑ण्यं॒ ॅयस्य॒ यस्य॒ हिर॑ण्यꣳ॒॒ हिर॑ण्यं॒ ॅयस्य॑ । \newline
17. यस्यै॒वैव यस्य॒ यस्यै॒व । \newline
18. ए॒व हिर॑ण्यꣳ॒॒ हिर॑ण्य मे॒वैव हिर॑ण्यम् । \newline
19. हिर॑ण्य॒म् तेन॒ तेन॒ हिर॑ण्यꣳ॒॒ हिर॑ण्य॒म् तेन॑ । \newline
20. तेनै॒वैव तेन॒ तेनै॒व । \newline
21. ए॒वैन॑ देन दे॒वैवैन॑त् । \newline
22. ए॒न॒द् वि॒न्द॒ते॒ वि॒न्द॒त॒ ए॒न॒ दे॒न॒द् वि॒न्द॒ते॒ । \newline
23. वि॒न्द॒ते॒ सा॒वि॒त्रः सा॑वि॒त्रो वि॑न्दते विन्दते सावि॒त्रः । \newline
24. सा॒वि॒त्रो भ॑वति भवति सावि॒त्रः सा॑वि॒त्रो भ॑वति । \newline
25. भ॒व॒ति॒ स॒वि॒तृप्र॑सूतः सवि॒तृप्र॑सूतो भवति भवति सवि॒तृप्र॑सूतः । \newline
26. स॒वि॒तृप्र॑सूत ए॒वैव स॑वि॒तृप्र॑सूतः सवि॒तृप्र॑सूत ए॒व । \newline
27. स॒वि॒तृप्र॑सूत॒ इति॑ सवि॒तृ - प्र॒सू॒तः॒ । \newline
28. ए॒वैन॑ देन दे॒वैवैन॑त् । \newline
29. ए॒न॒द् वि॒न्द॒ते॒ वि॒न्द॒त॒ ए॒न॒ दे॒न॒द् वि॒न्द॒ते॒ । \newline
30. वि॒न्द॒ते॒ भूम्यै॒ भूम्यै॑ विन्दते विन्दते॒ भूम्यै᳚ । \newline
31. भूम्यै॑ च॒रुश्च॒रुर् भूम्यै॒ भूम्यै॑ च॒रुः । \newline
32. च॒रुर् भ॑वति भवति च॒रु श्च॒रुर् भ॑वति । \newline
33. भ॒व॒ त्य॒स्या म॒स्याम् भ॑वति भव त्य॒स्याम् । \newline
34. अ॒स्या मे॒वैवास्या म॒स्या मे॒व । \newline
35. ए॒वैन॑ देन दे॒वैवैन॑त् । \newline
36. ए॒न॒द् वि॒न्द॒ते॒ वि॒न्द॒त॒ ए॒न॒ दे॒न॒द् वि॒न्द॒ते॒ । \newline
37. वि॒न्द॒त॒ उपोप॑ विन्दते विन्दत॒ उप॑ । \newline
38. उपै॑न मेन॒ मुपोपै॑नम् । \newline
39. ए॒नꣳ॒॒ हिर॑ण्यꣳ॒॒ हिर॑ण्य मेन मेनꣳ॒॒ हिर॑ण्यम् । \newline
40. हिर॑ण्यम् नमति नमति॒ हिर॑ण्यꣳ॒॒ हिर॑ण्यम् नमति । \newline
41. न॒म॒ति॒ वि वि न॑मति नमति॒ वि । \newline
42. वि वै वै वि वि वै । \newline
43. वा ए॒ष ए॒ष वै वा ए॒षः । \newline
44. ए॒ष इ॑न्द्रि॒येणे᳚ न्द्रि॒येणै॒ष ए॒ष इ॑न्द्रि॒येण॑ । \newline
45. इ॒न्द्रि॒येण॑ वी॒र्ये॑ण वी॒र्ये॑णे न्द्रि॒येणे᳚ न्द्रि॒येण॑ वी॒र्ये॑ण । \newline
46. वी॒र्ये॑ण र्‌द्ध्यत ऋद्ध्यते वी॒र्ये॑ण वी॒र्ये॑ण र्‌द्ध्यते । \newline
47. ऋ॒द्ध्य॒ते॒ यो य ऋ॑द्ध्यत ऋद्ध्यते॒ यः । \newline
48. यो हिर॑ण्यꣳ॒॒ हिर॑ण्यं॒ ॅयो यो हिर॑ण्यम् । \newline
49. हिर॑ण्यं ॅवि॒न्दते॑ वि॒न्दते॒ हिर॑ण्यꣳ॒॒ हिर॑ण्यं ॅवि॒न्दते᳚ । \newline
50. वि॒न्दत॑ ए॒ता मे॒तां ॅवि॒न्दते॑ वि॒न्दत॑ ए॒ताम् । \newline
51. ए॒ता मे॒वैवैता मे॒ता मे॒व । \newline

\textbf{Ghana Paata } \newline

1. च॒रुं ॅयो यश्च॒रुम् च॒रुं ॅयः का॒मये॑त का॒मये॑त॒ यश्च॒रुम् च॒रुं ॅयः का॒मये॑त । \newline
2. यः का॒मये॑त का॒मये॑त॒ यो यः का॒मये॑त॒ हिर॑ण्यꣳ॒॒ हिर॑ण्यम् का॒मये॑त॒ यो यः का॒मये॑त॒ हिर॑ण्यम् । \newline
3. का॒मये॑त॒ हिर॑ण्यꣳ॒॒ हिर॑ण्यम् का॒मये॑त का॒मये॑त॒ हिर॑ण्यं ॅविन्देय विन्देय॒ हिर॑ण्यम् का॒मये॑त का॒मये॑त॒ हिर॑ण्यं ॅविन्देय । \newline
4. हिर॑ण्यं ॅविन्देय विन्देय॒ हिर॑ण्यꣳ॒॒ हिर॑ण्यं ॅविन्देय॒ हिर॑ण्यꣳ॒॒ हिर॑ण्यं ॅविन्देय॒ हिर॑ण्यꣳ॒॒ हिर॑ण्यं ॅविन्देय॒ हिर॑ण्यम् । \newline
5. वि॒न्दे॒य॒ हिर॑ण्यꣳ॒॒ हिर॑ण्यं ॅविन्देय विन्देय॒ हिर॑ण्यम् मा मा॒ हिर॑ण्यं ॅविन्देय विन्देय॒ हिर॑ण्यम् मा । \newline
6. हिर॑ण्यम् मा मा॒ हिर॑ण्यꣳ॒॒ हिर॑ण्य॒म् मोपोप॑ मा॒ हिर॑ण्यꣳ॒॒ हिर॑ण्य॒म् मोप॑ । \newline
7. मोपोप॑ मा॒ मोप॑ नमेन् नमे॒दुप॑ मा॒ मोप॑ नमेत् । \newline
8. उप॑ नमेन् नमे॒ दुपोप॑ नमे॒ दितीति॑ नमे॒ दुपोप॑ नमे॒ दिति॑ । \newline
9. न॒मे॒ दितीति॑ नमेन् नमे॒ दिति॒ यद् यदिति॑ नमेन् नमे॒ दिति॒ यत् । \newline
10. इति॒ यद् यदितीति॒ यदा᳚ग्ने॒य आ᳚ग्ने॒यो यदितीति॒ यदा᳚ग्ने॒यः । \newline
11. यदा᳚ग्ने॒य आ᳚ग्ने॒यो यद् यदा᳚ग्ने॒यो भव॑ति॒ भव॑ त्याग्ने॒यो यद् यदा᳚ग्ने॒यो भव॑ति । \newline
12. आ॒ग्ने॒यो भव॑ति॒ भव॑ त्याग्ने॒य आ᳚ग्ने॒यो भव॑ त्याग्ने॒य मा᳚ग्ने॒यम् भव॑त्याग्ने॒य आ᳚ग्ने॒यो भव॑ त्याग्ने॒यम् । \newline
13. भव॑ त्याग्ने॒य मा᳚ग्ने॒यम् भव॑ति॒ भव॑ त्याग्ने॒यं ॅवै वा आ᳚ग्ने॒यम् भव॑ति॒ भव॑ त्याग्ने॒यं ॅवै । \newline
14. आ॒ग्ने॒यं ॅवै वा आ᳚ग्ने॒य मा᳚ग्ने॒यं ॅवै हिर॑ण्यꣳ॒॒ हिर॑ण्यं॒ ॅवा आ᳚ग्ने॒य मा᳚ग्ने॒यं ॅवै हिर॑ण्यम् । \newline
15. वै हिर॑ण्यꣳ॒॒ हिर॑ण्यं॒ ॅवै वै हिर॑ण्यं॒ ॅयस्य॒ यस्य॒ हिर॑ण्यं॒ ॅवै वै हिर॑ण्यं॒ ॅयस्य॑ । \newline
16. हिर॑ण्यं॒ ॅयस्य॒ यस्य॒ हिर॑ण्यꣳ॒॒ हिर॑ण्यं॒ ॅयस्यै॒वैव यस्य॒ हिर॑ण्यꣳ॒॒ हिर॑ण्यं॒ ॅयस्यै॒व । \newline
17. यस्यै॒वैव यस्य॒ यस्यै॒व हिर॑ण्यꣳ॒॒ हिर॑ण्य मे॒व यस्य॒ यस्यै॒व हिर॑ण्यम् । \newline
18. ए॒व हिर॑ण्यꣳ॒॒ हिर॑ण्य मे॒वैव हिर॑ण्य॒म् तेन॒ तेन॒ हिर॑ण्य मे॒वैव हिर॑ण्य॒म् तेन॑ । \newline
19. हिर॑ण्य॒म् तेन॒ तेन॒ हिर॑ण्यꣳ॒॒ हिर॑ण्य॒म् तेनै॒वैव तेन॒ हिर॑ण्यꣳ॒॒ हिर॑ण्य॒म् तेनै॒व । \newline
20. तेनै॒वैव तेन॒ तेनै॒वैन॑ देन दे॒व तेन॒ तेनै॒वैन॑त् । \newline
21. ए॒वैन॑ देन दे॒वैवैन॑द् विन्दते विन्दत एन दे॒वैवैन॑द् विन्दते । \newline
22. ए॒न॒द् वि॒न्द॒ते॒ वि॒न्द॒त॒ ए॒न॒दे॒न॒द् वि॒न्द॒ते॒ सा॒वि॒त्रः सा॑वि॒त्रो वि॑न्दत एनदेनद् विन्दते सावि॒त्रः । \newline
23. वि॒न्द॒ते॒ सा॒वि॒त्रः सा॑वि॒त्रो वि॑न्दते विन्दते सावि॒त्रो भ॑वति भवति सावि॒त्रो वि॑न्दते विन्दते सावि॒त्रो भ॑वति । \newline
24. सा॒वि॒त्रो भ॑वति भवति सावि॒त्रः सा॑वि॒त्रो भ॑वति सवि॒तृप्र॑सूतः सवि॒तृप्र॑सूतो भवति सावि॒त्रः सा॑वि॒त्रो भ॑वति सवि॒तृप्र॑सूतः । \newline
25. भ॒व॒ति॒ स॒वि॒तृप्र॑सूतः सवि॒तृप्र॑सूतो भवति भवति सवि॒तृप्र॑सूत ए॒वैव स॑वि॒तृप्र॑सूतो भवति भवति सवि॒तृप्र॑सूत ए॒व । \newline
26. स॒वि॒तृप्र॑सूत ए॒वैव स॑वि॒तृप्र॑सूतः सवि॒तृप्र॑सूत ए॒वैन॑देनदे॒व स॑वि॒तृप्र॑सूतः सवि॒तृप्र॑सूत ए॒वैन॑त् । \newline
27. स॒वि॒तृप्र॑सूत॒ इति॑ सवि॒तृ - प्र॒सू॒तः॒ । \newline
28. ए॒वैन॑ देन दे॒वैवैन॑द् विन्दते विन्दत एन दे॒वैवैन॑द् विन्दते । \newline
29. ए॒न॒द् वि॒न्द॒ते॒ वि॒न्द॒त॒ ए॒न॒ दे॒न॒द् वि॒न्द॒ते॒ भूम्यै॒ भूम्यै॑ विन्दत एन देनद् विन्दते॒ भूम्यै᳚ । \newline
30. वि॒न्द॒ते॒ भूम्यै॒ भूम्यै॑ विन्दते विन्दते॒ भूम्यै॑ च॒रुश्च॒रुर् भूम्यै॑ विन्दते विन्दते॒ भूम्यै॑ च॒रुः । \newline
31. भूम्यै॑ च॒रुश्च॒रुर् भूम्यै॒ भूम्यै॑ च॒रुर् भ॑वति भवति च॒रुर् भूम्यै॒ भूम्यै॑ च॒रुर् भ॑वति । \newline
32. च॒रुर् भ॑वति भवति च॒रु श्च॒रुर् भ॑व त्य॒स्या म॒स्याम् भ॑वति च॒रु श्च॒रुर् भ॑व त्य॒स्याम् । \newline
33. भ॒व॒ त्य॒स्या म॒स्याम् भ॑वति भव त्य॒स्या मे॒वैवास्याम् भ॑वति भव त्य॒स्या मे॒व । \newline
34. अ॒स्या मे॒वैवास्या म॒स्या मे॒वैन॑ देन दे॒वास्या म॒स्या मे॒वैन॑त् । \newline
35. ए॒वैन॑ देन दे॒वैवैन॑द् विन्दते विन्दत एन दे॒वैवैन॑द् विन्दते । \newline
36. ए॒न॒द् वि॒न्द॒ते॒ वि॒न्द॒त॒ ए॒न॒दे॒न॒द् वि॒न्द॒त॒ उपोप॑ विन्दत एनदेनद् विन्दत॒ उप॑ । \newline
37. वि॒न्द॒त॒ उपोप॑ विन्दते विन्दत॒ उपै॑न मेन॒ मुप॑ विन्दते विन्दत॒ उपै॑नम् । \newline
38. उपै॑न मेन॒ मुपोपै॑नꣳ॒॒ हिर॑ण्यꣳ॒॒ हिर॑ण्य मेन॒ मुपोपै॑नꣳ॒॒ हिर॑ण्यम् । \newline
39. ए॒नꣳ॒॒ हिर॑ण्यꣳ॒॒ हिर॑ण्य मेन मेनꣳ॒॒ हिर॑ण्यम् नमति नमति॒ हिर॑ण्य मेन मेनꣳ॒॒ हिर॑ण्यम् नमति । \newline
40. हिर॑ण्यम् नमति नमति॒ हिर॑ण्यꣳ॒॒ हिर॑ण्यम् नमति॒ वि वि न॑मति॒ हिर॑ण्यꣳ॒॒ हिर॑ण्यम् नमति॒ वि । \newline
41. न॒म॒ति॒ वि वि न॑मति नमति॒ वि वै वै वि न॑मति नमति॒ वि वै । \newline
42. वि वै वै वि वि वा ए॒ष ए॒ष वै वि वि वा ए॒षः । \newline
43. वा ए॒ष ए॒ष वै वा ए॒ष इ॑न्द्रि॒येणे᳚ न्द्रि॒येणै॒ष वै वा ए॒ष इ॑न्द्रि॒येण॑ । \newline
44. ए॒ष इ॑न्द्रि॒येणे᳚ न्द्रि॒येणै॒ष ए॒ष इ॑न्द्रि॒येण॑ वी॒र्ये॑ण वी॒र्ये॑णे न्द्रि॒येणै॒ष ए॒ष इ॑न्द्रि॒येण॑ वी॒र्ये॑ण । \newline
45. इ॒न्द्रि॒येण॑ वी॒र्ये॑ण वी॒र्ये॑णे न्द्रि॒येणे᳚ न्द्रि॒येण॑ वी॒र्ये॑ण र्‌द्ध्यत ऋद्ध्यते वी॒र्ये॑णे न्द्रि॒येणे᳚ न्द्रि॒येण॑ वी॒र्ये॑ण र्‌द्ध्यते । \newline
46. वी॒र्ये॑ण र्‌द्ध्यत ऋद्ध्यते वी॒र्ये॑ण वी॒र्ये॑ण र्‌द्ध्यते॒ यो य ऋ॑द्ध्यते वी॒र्ये॑ण वी॒र्ये॑ण र्‌द्ध्यते॒ यः । \newline
47. ऋ॒द्ध्य॒ते॒ यो य ऋ॑द्ध्यत ऋद्ध्यते॒ यो हिर॑ण्यꣳ॒॒ हिर॑ण्यं॒ ॅय ऋ॑द्ध्यत ऋद्ध्यते॒ यो हिर॑ण्यम् । \newline
48. यो हिर॑ण्यꣳ॒॒ हिर॑ण्यं॒ ॅयो यो हिर॑ण्यं ॅवि॒न्दते॑ वि॒न्दते॒ हिर॑ण्यं॒ ॅयो यो हिर॑ण्यं ॅवि॒न्दते᳚ । \newline
49. हिर॑ण्यं ॅवि॒न्दते॑ वि॒न्दते॒ हिर॑ण्यꣳ॒॒ हिर॑ण्यं ॅवि॒न्दत॑ ए॒ता मे॒तां ॅवि॒न्दते॒ हिर॑ण्यꣳ॒॒ हिर॑ण्यं ॅवि॒न्दत॑ ए॒ताम् । \newline
50. वि॒न्दत॑ ए॒ता मे॒तां ॅवि॒न्दते॑ वि॒न्दत॑ ए॒ता मे॒वैवैतां ॅवि॒न्दते॑ वि॒न्दत॑ ए॒ता मे॒व । \newline
51. ए॒ता मे॒वैवैता मे॒ता मे॒व निर् णिरे॒वैता मे॒ता मे॒व निः । \newline
\pagebreak
\markright{ TS 2.3.2.5  \hfill https://www.vedavms.in \hfill}

\section{ TS 2.3.2.5 }

\textbf{TS 2.3.2.5 } \newline
\textbf{Samhita Paata} \newline

-मे॒व निर्व॑पे॒द्धिर॑ण्यं ॅवि॒त्त्वा नेन्द्रि॒येण॑ वी॒र्ये॑ण॒ व्यृ॑द्ध्यत ए॒तामे॒व निर्व॑पे॒द्यस्य॒ हिर॑ण्यं॒ नश्ये॒द्यदा᳚ग्ने॒यो भव॑त्याग्ने॒यं ॅवै हिर॑ण्यं॒ ॅयस्यै॒ व हिर॑ण्यं॒ तेनै॒वैन॑द्-विन्दति सावि॒त्रो भ॑वति सवि॒तृ-प्र॑सूत ए॒वैन॑द्-विन्दति॒ भूम्यै॑ च॒रुर्भ॑वत्य॒स्यां ॅवा ए॒तन्न॑श्यति॒ यन्नश्य॑त्य॒स्यामे॒वैन॑द्-विन्द॒तीन्द्र॒ - [  ] \newline

\textbf{Pada Paata} \newline

ए॒व । निरिति॑ । व॒पे॒त् । हिर॑ण्यम् । वि॒त्त्वा । न । इ॒न्द्रि॒येण॑ । वी॒र्ये॑ण । वीति॑ । ऋ॒द्ध्य॒ते॒ । ए॒ताम् । ए॒व । निरिति॑ । व॒पे॒त् । यस्य॑ । हिर॑ण्यम् । नश्ये᳚त् । यत् । आ॒ग्ने॒यः । भव॑ति । आ॒ग्ने॒यम् । वै । हिर॑ण्यम् । यस्य॑ । ए॒व । हिर॑ण्यम् । तेन॑ । ए॒व । ए॒न॒त् । वि॒न्द॒ति॒ । सा॒वि॒त्रः । भ॒व॒ति॒ । स॒वि॒तृप्र॑सूत॒ इति॑ सवि॒तृ - प्र॒सू॒तः॒ । ए॒व । ए॒न॒त् । वि॒न्द॒ति॒ । भूम्यै᳚ । च॒रुः । भ॒व॒ति॒ । अ॒स्याम् । वै । ए॒तत् । न॒श्य॒ति॒ । यत् । नश्य॑ति । अ॒स्याम् । ए॒व । ए॒न॒त् । वि॒न्द॒ति॒ । इन्द्रः॑ ।  \newline


\textbf{Krama Paata} \newline

ए॒व निः । निर् व॑पेत् । व॒पे॒द्धिर॑ण्यम् । हिर॑ण्यं ॅवि॒त्वा । वि॒त्वा न । नेन्द्रि॒येण॑ । इ॒न्द्रि॒येण॑ वी॒र्ये॑ण । वी॒र्ये॑ण॒ वि । व्यृ॑द्ध्यते । ऋ॒द्ध्य॒त॒ ए॒ताम् । ए॒तामे॒व । ए॒व निः । निर् व॑पेत् । व॒पे॒द् यस्य॑ । यस्य॒ हिर॑ण्यम् । हिर॑ण्य॒म् नश्ये᳚त् । नश्ये॒द् यत् । यदा᳚ग्ने॒यः । आ॒ग्ने॒यो भव॑ति । भव॑त्याग्ने॒यम् । आ॒ग्ने॒यं ॅवै । वै हिर॑ण्यम् । हिर॑ण्यं॒ ॅयस्य॑ । यस्यै॒व । ए॒व हिर॑ण्यम् । हिर॑ण्य॒म् तेन॑ । तेनै॒व । ए॒वैन॑त् । ए॒न॒द् वि॒न्द॒ति॒ । वि॒न्द॒ति॒ सा॒वि॒त्रः । सा॒वि॒त्रो भ॑वति । भ॒व॒ति॒ स॒वि॒तृप्र॑सूतः । स॒वि॒तृप्र॑सूत ए॒व । स॒वि॒तृप्र॑सूत॒ इति॑ सवि॒तृ - प्र॒सू॒तः॒ । ए॒वैन॑त् । ए॒न॒द् वि॒न्द॒ति॒ । वि॒न्द॒ति॒ भूम्यै᳚ । भूम्यै॑ च॒रुः । च॒रुर् भ॑वति । भ॒व॒त्य॒स्याम् । अ॒स्यां ॅवै । वा ए॒तत् । ए॒तन्न॑श्यति । न॒श्य॒ति॒ यत् । यन्नश्य॑ति । नश्य॑त्य॒स्याम् । अ॒स्यामे॒व । ए॒वैन॑त् । ए॒न॒द् वि॒न्द॒ति॒ । वि॒न्द॒तीन्द्रः॑ । इन्द्र॒ स्त्वष्टुः॑ \newline

\textbf{Jatai Paata} \newline

1. ए॒व निर् णि रे॒वैव निः । \newline
2. निर् व॑पेद् वपे॒न् निर् णिर् व॑पेत् । \newline
3. व॒पे॒ द्धिर॑ण्यꣳ॒॒ हिर॑ण्यं ॅवपेद् वपे॒ द्धिर॑ण्यम् । \newline
4. हिर॑ण्यं ॅवि॒त्त्वा वि॒त्त्वा हिर॑ण्यꣳ॒॒ हिर॑ण्यं ॅवि॒त्त्वा । \newline
5. वि॒त्त्वा न न वि॒त्त्वा वि॒त्त्वा न । \newline
6. ने न्द्रि॒येणे᳚ न्द्रि॒येण॒ न ने न्द्रि॒येण॑ । \newline
7. इ॒न्द्रि॒येण॑ वी॒र्ये॑ण वी॒र्ये॑णे न्द्रि॒येणे᳚ न्द्रि॒येण॑ वी॒र्ये॑ण । \newline
8. वी॒र्ये॑ण॒ वि वि वी॒र्ये॑ण वी॒र्ये॑ण॒ वि । \newline
9. व्यृ॑द्ध्यत ऋद्ध्यते॒ वि व्यृ॑द्ध्यते । \newline
10. ऋ॒द्ध्य॒त॒ ए॒ता मे॒ता मृ॑द्ध्यत ऋद्ध्यत ए॒ताम् । \newline
11. ए॒ता मे॒वैवैता मे॒ता मे॒व । \newline
12. ए॒व निर् णिरे॒वैव निः । \newline
13. निर् व॑पेद् वपे॒न् निर् णिर् व॑पेत् । \newline
14. व॒पे॒द् यस्य॒ यस्य॑ वपेद् वपे॒द् यस्य॑ । \newline
15. यस्य॒ हिर॑ण्यꣳ॒॒ हिर॑ण्यं॒ ॅयस्य॒ यस्य॒ हिर॑ण्यम् । \newline
16. हिर॑ण्य॒म् नश्ये॒न् नश्ये॒ द्धिर॑ण्यꣳ॒॒ हिर॑ण्य॒म् नश्ये᳚त् । \newline
17. नश्ये॒द् यद् यन् नश्ये॒न् नश्ये॒द् यत् । \newline
18. यदा᳚ग्ने॒य आ᳚ग्ने॒यो यद् यदा᳚ग्ने॒यः । \newline
19. आ॒ग्ने॒यो भव॑ति॒ भव॑ त्याग्ने॒य आ᳚ग्ने॒यो भव॑ति । \newline
20. भव॑ त्याग्ने॒य मा᳚ग्ने॒यम् भव॑ति॒ भव॑ त्याग्ने॒यम् । \newline
21. आ॒ग्ने॒यं ॅवै वा आ᳚ग्ने॒य मा᳚ग्ने॒यं ॅवै । \newline
22. वै हिर॑ण्यꣳ॒॒ हिर॑ण्यं॒ ॅवै वै हिर॑ण्यम् । \newline
23. हिर॑ण्यं॒ ॅयस्य॒ यस्य॒ हिर॑ण्यꣳ॒॒ हिर॑ण्यं॒ ॅयस्य॑ । \newline
24. यस्यै॒वैव यस्य॒ यस्यै॒व । \newline
25. ए॒व हिर॑ण्यꣳ॒॒ हिर॑ण्य मे॒वैव हिर॑ण्यम् । \newline
26. हिर॑ण्य॒म् तेन॒ तेन॒ हिर॑ण्यꣳ॒॒ हिर॑ण्य॒म् तेन॑ । \newline
27. तेनै॒वैव तेन॒ तेनै॒व । \newline
28. ए॒वैन॑ देन दे॒वैवैन॑त् । \newline
29. ए॒न॒द् वि॒न्द॒ति॒ वि॒न्द॒ त्ये॒न॒ दे॒न॒द् वि॒न्द॒ति॒ । \newline
30. वि॒न्द॒ति॒ सा॒वि॒त्रः सा॑वि॒त्रो वि॑न्दति विन्दति सावि॒त्रः । \newline
31. सा॒वि॒त्रो भ॑वति भवति सावि॒त्रः सा॑वि॒त्रो भ॑वति । \newline
32. भ॒व॒ति॒ स॒वि॒तृप्र॑सूतः सवि॒तृप्र॑सूतो भवति भवति सवि॒तृप्र॑सूतः । \newline
33. स॒वि॒तृप्र॑सूत ए॒वैव स॑वि॒तृप्र॑सूतः सवि॒तृप्र॑सूत ए॒व । \newline
34. स॒वि॒तृप्र॑सूत॒ इति॑ सवि॒तृ - प्र॒सू॒तः॒ । \newline
35. ए॒वैन॑ देन दे॒वैवैन॑त् । \newline
36. ए॒न॒द् वि॒न्द॒ति॒ वि॒न्द॒ त्ये॒न॒ दे॒न॒द् वि॒न्द॒ति॒ । \newline
37. वि॒न्द॒ति॒ भूम्यै॒ भूम्यै॑ विन्दति विन्दति॒ भूम्यै᳚ । \newline
38. भूम्यै॑ च॒रु श्च॒रुर् भूम्यै॒ भूम्यै॑ च॒रुः । \newline
39. च॒रुर् भ॑वति भवति च॒रु श्च॒रुर् भ॑वति । \newline
40. भ॒व॒ त्य॒स्या म॒स्याम् भ॑वति भव त्य॒स्याम् । \newline
41. अ॒स्यां ॅवै वा अ॒स्या म॒स्यां ॅवै । \newline
42. वा ए॒त दे॒तद् वै वा ए॒तत् । \newline
43. ए॒तन् न॑श्यति नश्य त्ये॒त दे॒तन् न॑श्यति । \newline
44. न॒श्य॒ति॒ यद् यन् न॑श्यति नश्यति॒ यत् । \newline
45. यन् नश्य॑ति॒ नश्य॑ति॒ यद् यन् नश्य॑ति । \newline
46. नश्य॑ त्य॒स्या म॒स्याम् नश्य॑ति॒ नश्य॑ त्य॒स्याम् । \newline
47. अ॒स्या मे॒वैवास्या म॒स्या मे॒व । \newline
48. ए॒वैन॑ देन दे॒वैवैन॑त् । \newline
49. ए॒न॒द् वि॒न्द॒ति॒ वि॒न्द॒ त्ये॒न॒ दे॒न॒द् वि॒न्द॒ति॒ । \newline
50. वि॒न्द॒तीन्द्र॒ इन्द्रो॑ विन्दति विन्द॒तीन्द्रः॑ । \newline
51. इन्द्र॒ स्त्वष्टु॒ स्त्वष्टु॒ रिन्द्र॒ इन्द्र॒ स्त्वष्टुः॑ । \newline

\textbf{Ghana Paata } \newline

1. ए॒व निर् णिरे॒वैव निर् व॑पेद् वपे॒न् निरे॒वैव निर् व॑पेत् । \newline
2. निर् व॑पेद् वपे॒न् निर् णिर् व॑पे॒ द्धिर॑ण्यꣳ॒॒ हिर॑ण्यं ॅवपे॒न् निर् णिर् व॑पे॒ द्धिर॑ण्यम् । \newline
3. व॒पे॒ द्धिर॑ण्यꣳ॒॒ हिर॑ण्यं ॅवपेद् वपे॒ द्धिर॑ण्यं ॅवि॒त्त्वा वि॒त्त्वा हिर॑ण्यं ॅवपेद् वपे॒ द्धिर॑ण्यं ॅवि॒त्त्वा । \newline
4. हिर॑ण्यं ॅवि॒त्त्वा वि॒त्त्वा हिर॑ण्यꣳ॒॒ हिर॑ण्यं ॅवि॒त्त्वा न न वि॒त्त्वा हिर॑ण्यꣳ॒॒ हिर॑ण्यं ॅवि॒त्त्वा न । \newline
5. वि॒त्त्वा न न वि॒त्त्वा वि॒त्त्वा ने न्द्रि॒येणे᳚ न्द्रि॒येण॒ न वि॒त्त्वा वि॒त्त्वा ने न्द्रि॒येण॑ । \newline
6. ने न्द्रि॒येणे᳚ न्द्रि॒येण॒ न ने न्द्रि॒येण॑ वी॒र्ये॑ण वी॒र्ये॑णे न्द्रि॒येण॒ न ने न्द्रि॒येण॑ वी॒र्ये॑ण । \newline
7. इ॒न्द्रि॒येण॑ वी॒र्ये॑ण वी॒र्ये॑णे न्द्रि॒येणे᳚ न्द्रि॒येण॑ वी॒र्ये॑ण॒ वि वि वी॒र्ये॑णे न्द्रि॒येणे᳚ न्द्रि॒येण॑ वी॒र्ये॑ण॒ वि । \newline
8. वी॒र्ये॑ण॒ वि वि वी॒र्ये॑ण वी॒र्ये॑ण॒ व्यृ॑द्ध्यत ऋद्ध्यते॒ वि वी॒र्ये॑ण वी॒र्ये॑ण॒ व्यृ॑द्ध्यते । \newline
9. व्यृ॑द्ध्यत ऋद्ध्यते॒ वि व्यृ॑द्ध्यत ए॒ता मे॒ता मृ॑द्ध्यते॒ वि व्यृ॑द्ध्यत ए॒ताम् । \newline
10. ऋ॒द्ध्य॒त॒ ए॒ता मे॒ता मृ॑द्ध्यत ऋद्ध्यत ए॒ता मे॒वैवैता मृ॑द्ध्यत ऋद्ध्यत ए॒ता मे॒व । \newline
11. ए॒ता मे॒वैवैता मे॒ता मे॒व निर् णिरे॒वैता मे॒ता मे॒व निः । \newline
12. ए॒व निर् णिरे॒वैव निर् व॑पेद् वपे॒न् निरे॒वैव निर् व॑पेत् । \newline
13. निर् व॑पेद् वपे॒न् निर् णिर् व॑पे॒द् यस्य॒ यस्य॑ वपे॒न् निर् णिर् व॑पे॒द् यस्य॑ । \newline
14. व॒पे॒द् यस्य॒ यस्य॑ वपेद् वपे॒द् यस्य॒ हिर॑ण्यꣳ॒॒ हिर॑ण्यं॒ ॅयस्य॑ वपेद् वपे॒द् यस्य॒ हिर॑ण्यम् । \newline
15. यस्य॒ हिर॑ण्यꣳ॒॒ हिर॑ण्यं॒ ॅयस्य॒ यस्य॒ हिर॑ण्य॒म् नश्ये॒न् नश्ये॒ द्धिर॑ण्यं॒ ॅयस्य॒ यस्य॒ हिर॑ण्य॒म् नश्ये᳚त् । \newline
16. हिर॑ण्य॒म् नश्ये॒न् नश्ये॒ द्धिर॑ण्यꣳ॒॒ हिर॑ण्य॒म् नश्ये॒द् यद् यन् नश्ये॒ द्धिर॑ण्यꣳ॒॒ हिर॑ण्य॒म् नश्ये॒द् यत् । \newline
17. नश्ये॒द् यद् यन् नश्ये॒न् नश्ये॒द् यदा᳚ग्ने॒य आ᳚ग्ने॒यो यन् नश्ये॒न् नश्ये॒द् यदा᳚ग्ने॒यः । \newline
18. यदा᳚ग्ने॒य आ᳚ग्ने॒यो यद् यदा᳚ग्ने॒यो भव॑ति॒ भव॑ त्याग्ने॒यो यद् यदा᳚ग्ने॒यो भव॑ति । \newline
19. आ॒ग्ने॒यो भव॑ति॒ भव॑ त्याग्ने॒य आ᳚ग्ने॒यो भव॑ त्याग्ने॒य मा᳚ग्ने॒यम् भव॑त्याग्ने॒य आ᳚ग्ने॒यो भव॑ त्याग्ने॒यम् । \newline
20. भव॑ त्याग्ने॒य मा᳚ग्ने॒यम् भव॑ति॒ भव॑त्याग्ने॒यं ॅवै वा आ᳚ग्ने॒यम् भव॑ति॒ भव॑ त्याग्ने॒यं ॅवै । \newline
21. आ॒ग्ने॒यं ॅवै वा आ᳚ग्ने॒य मा᳚ग्ने॒यं ॅवै हिर॑ण्यꣳ॒॒ हिर॑ण्यं॒ ॅवा आ᳚ग्ने॒य मा᳚ग्ने॒यं ॅवै हिर॑ण्यम् । \newline
22. वै हिर॑ण्यꣳ॒॒ हिर॑ण्यं॒ ॅवै वै हिर॑ण्यं॒ ॅयस्य॒ यस्य॒ हिर॑ण्यं॒ ॅवै वै हिर॑ण्यं॒ ॅयस्य॑ । \newline
23. हिर॑ण्यं॒ ॅयस्य॒ यस्य॒ हिर॑ण्यꣳ॒॒ हिर॑ण्यं॒ ॅयस्यै॒वैव यस्य॒ हिर॑ण्यꣳ॒॒ हिर॑ण्यं॒ ॅयस्यै॒व । \newline
24. यस्यै॒वैव यस्य॒ यस्यै॒व हिर॑ण्यꣳ॒॒ हिर॑ण्य मे॒व यस्य॒ यस्यै॒व हिर॑ण्यम् । \newline
25. ए॒व हिर॑ण्यꣳ॒॒ हिर॑ण्य मे॒वैव हिर॑ण्य॒म् तेन॒ तेन॒ हिर॑ण्य मे॒वैव हिर॑ण्य॒म् तेन॑ । \newline
26. हिर॑ण्य॒म् तेन॒ तेन॒ हिर॑ण्यꣳ॒॒ हिर॑ण्य॒म् तेनै॒वैव तेन॒ हिर॑ण्यꣳ॒॒ हिर॑ण्य॒म् तेनै॒व । \newline
27. तेनै॒वैव तेन॒ तेनै॒वैन॑ देनदे॒व तेन॒ तेनै॒वैन॑त् । \newline
28. ए॒वैन॑ देन दे॒वैवैन॑द् विन्दति विन्द त्येन दे॒वैवैन॑द् विन्दति । \newline
29. ए॒न॒द् वि॒न्द॒ति॒ वि॒न्द॒ त्ये॒न॒ दे॒न॒द् वि॒न्द॒ति॒ सा॒वि॒त्रः सा॑वि॒त्रो वि॑न्द त्येन देनद् विन्दति सावि॒त्रः । \newline
30. वि॒न्द॒ति॒ सा॒वि॒त्रः सा॑वि॒त्रो वि॑न्दति विन्दति सावि॒त्रो भ॑वति भवति सावि॒त्रो वि॑न्दति विन्दति सावि॒त्रो भ॑वति । \newline
31. सा॒वि॒त्रो भ॑वति भवति सावि॒त्रः सा॑वि॒त्रो भ॑वति सवि॒तृप्र॑सूतः सवि॒तृप्र॑सूतो भवति सावि॒त्रः सा॑वि॒त्रो भ॑वति सवि॒तृप्र॑सूतः । \newline
32. भ॒व॒ति॒ स॒वि॒तृप्र॑सूतः सवि॒तृप्र॑सूतो भवति भवति सवि॒तृप्र॑सूत ए॒वैव स॑वि॒तृप्र॑सूतो भवति भवति सवि॒तृप्र॑सूत ए॒व । \newline
33. स॒वि॒तृप्र॑सूत ए॒वैव स॑वि॒तृप्र॑सूतः सवि॒तृप्र॑सूत ए॒वैन॑ देन दे॒व स॑वि॒तृप्र॑सूतः सवि॒तृप्र॑सूत ए॒वैन॑त् । \newline
34. स॒वि॒तृप्र॑सूत॒ इति॑ सवि॒तृ - प्र॒सू॒तः॒ । \newline
35. ए॒वैन॑ देन दे॒वैवैन॑द् विन्दति विन्द त्येन दे॒वैवैन॑द् विन्दति । \newline
36. ए॒न॒द् वि॒न्द॒ति॒ वि॒न्द॒ त्ये॒न॒ दे॒न॒द् वि॒न्द॒ति॒ भूम्यै॒ भूम्यै॑ विन्द त्येन देनद् विन्दति॒ भूम्यै᳚ । \newline
37. वि॒न्द॒ति॒ भूम्यै॒ भूम्यै॑ विन्दति विन्दति॒ भूम्यै॑ च॒रु श्च॒रुर् भूम्यै॑ विन्दति विन्दति॒ भूम्यै॑ च॒रुः । \newline
38. भूम्यै॑ च॒रु श्च॒रुर् भूम्यै॒ भूम्यै॑ च॒रुर् भ॑वति भवति च॒रुर् भूम्यै॒ भूम्यै॑ च॒रुर् भ॑वति । \newline
39. च॒रुर् भ॑वति भवति च॒रु श्च॒रुर् भ॑व त्य॒स्या म॒स्याम् भ॑वति च॒रु श्च॒रुर् भ॑व त्य॒स्याम् । \newline
40. भ॒व॒ त्य॒स्या म॒स्याम् भ॑वति भव त्य॒स्यां ॅवै वा अ॒स्याम् भ॑वति भव त्य॒स्यां ॅवै । \newline
41. अ॒स्यां ॅवै वा अ॒स्या म॒स्यां ॅवा ए॒त दे॒तद् वा अ॒स्या म॒स्यां ॅवा ए॒तत् । \newline
42. वा ए॒त दे॒तद् वै वा ए॒तन् न॑श्यति नश्य त्ये॒तद् वै वा ए॒तन् न॑श्यति । \newline
43. ए॒तन् न॑श्यति नश्य त्ये॒त दे॒तन् न॑श्यति॒ यद् यन् न॑श्य त्ये॒त दे॒तन् न॑श्यति॒ यत् । \newline
44. न॒श्य॒ति॒ यद् यन् न॑श्यति नश्यति॒ यन् नश्य॑ति॒ नश्य॑ति॒ यन् न॑श्यति नश्यति॒ यन् नश्य॑ति । \newline
45. यन् नश्य॑ति॒ नश्य॑ति॒ यद् यन् नश्य॑त्य॒स्या म॒स्याम् नश्य॑ति॒ यद् यन् नश्य॑ त्य॒स्याम् । \newline
46. नश्य॑त्य॒स्या म॒स्याम् नश्य॑ति॒ नश्य॑ त्य॒स्या मे॒वैवास्याम् नश्य॑ति॒ नश्य॑ त्य॒स्या मे॒व । \newline
47. अ॒स्या मे॒वैवास्या म॒स्या मे॒वैन॑ देन दे॒वास्या म॒स्या मे॒वैन॑त् । \newline
48. ए॒वैन॑ देन दे॒वैवैन॑द् विन्दति विन्द त्येन दे॒वैवैन॑द् विन्दति । \newline
49. ए॒न॒द् वि॒न्द॒ति॒ वि॒न्द॒ त्ये॒न॒ दे॒न॒द् वि॒न्द॒तीन्द्र॒ इन्द्रो॑ विन्द त्येन देनद् विन्द॒तीन्द्रः॑ । \newline
50. वि॒न्द॒तीन्द्र॒ इन्द्रो॑ विन्दति विन्द॒तीन्द्र॒ स्त्वष्टु॒ स्त्वष्टु॒ रिन्द्रो॑ विन्दति विन्द॒तीन्द्र॒ स्त्वष्टुः॑ । \newline
51. इन्द्र॒ स्त्वष्टु॒ स्त्वष्टु॒ रिन्द्र॒ इन्द्र॒ स्त्वष्टुः॒ सोमꣳ॒॒ सोम॒म् त्वष्टु॒ रिन्द्र॒ इन्द्र॒ स्त्वष्टुः॒ सोम᳚म् । \newline
\pagebreak
\markright{ TS 2.3.2.6  \hfill https://www.vedavms.in \hfill}

\section{ TS 2.3.2.6 }

\textbf{TS 2.3.2.6 } \newline
\textbf{Samhita Paata} \newline

-स्त्वष्टुः॒ सोम॑मभी॒षहा॑ऽ पिब॒थ् स विष्व॒ङ् व्या᳚र्च्छ॒थ् स इ॑न्द्रि॒येण॑ सोमपी॒थेन॒ व्या᳚र्द्ध्यत॒ स यदू॒र्द्ध्वमु॒दव॑मी॒त् ते श्या॒माका॑ अभव॒न्थ्स प्र॒जाप॑ति॒मुपा॑धाव॒त् तस्मा॑ ए॒तꣳ सो॑मे॒न्द्रꣳ श्या॑मा॒कं च॒रुं निर॑वप॒त् तेनै॒वास्मि॑न्निन्द्रि॒यꣳ सो॑मपी॒थम॑दधा॒द्वि वा ए॒ष इ॑न्द्रि॒येण॑ सोम॒पीथेन॑र्द्ध्यते॒ यः सोमं॒ ॅवमि॑ति॒ यः सो॑मवा॒मी स्यात् तस्मा॑-  [  ] \newline

\textbf{Pada Paata} \newline

त्वष्टुः॑ । सोम᳚म् । अ॒भी॒षहेत्य॑भि - सहा᳚ । अ॒पि॒ब॒त् । सः । विष्वङ्॑ । वीति॑ । आ॒र्च्छ॒त् । सः । इ॒न्द्रि॒येण॑ । सो॒म॒पी॒थेनेति॑ सोम - पी॒थेन॑ । वीति॑ । आ॒र्द्ध्य॒त॒ । सः । यत् । ऊ॒र्द्ध्वम् । उ॒दव॑मी॒दित्यु॑त् - अव॑मीत् । ते । श्या॒माकाः᳚ । अ॒भ॒व॒न्न् । सः । प्र॒जाप॑ति॒मिति॑ प्र॒जा - प॒ति॒म् । उपेति॑ । अ॒धा॒व॒त् । तस्मै᳚ । ए॒तम् । सो॒मे॒न्द्रम् । श्या॒मा॒कम् । च॒रुम् । निरिति॑ । अ॒व॒प॒त् । तेन॑ । ए॒व । अ॒स्मि॒न्न् । इ॒न्द्रि॒यम् । सो॒म॒पी॒थमिति॑ सोम - पी॒थम् । अ॒द॒धा॒त् । वीति॑ । वै । ए॒षः । इ॒न्द्रि॒येण॑ । सो॒म॒पी॒थेनेति॑ सोम - पी॒थेन॑ । ऋ॒द्ध्य॒ते॒ । यः । सोम᳚म् । वमि॑ति । यः । सो॒म॒वा॒मीति॑ सोम - वा॒मी । स्यात् । तस्मै᳚ ।  \newline


\textbf{Krama Paata} \newline

त्वष्टुः॒ सोम᳚म् । सोम॑मभी॒षहा᳚ । अ॒भी॒षहा॑ ऽपिबत् । अ॒भी॒षहेत्य॑भि - सहा᳚ । अ॒पि॒ब॒थ् सः । स विष्वङ्ङ्॑ । विष्व॒ङ्॒ वि । व्या᳚र्च्छत् । आ॒र्च्छ॒थ् सः । स इ॑न्द्रि॒येण॑ । इ॒न्द्रि॒येण॑ सोमपी॒थेन॑ । सो॒म॒पी॒थेन॒ वि । सो॒म॒पी॒थेनेति॑ सोम - पी॒थेन॑ । व्या᳚र्द्ध्यत । आ॒र्द्ध्य॒त॒ सः । स यत् । यदू॒र्द्ध्वम् । ऊ॒र्द्ध्वमु॒दव॑मीत् । उ॒दव॑मी॒त् ते । उ॒दव॑मी॒दित्यु॑त् - अव॑मीत् । ते श्या॒माकाः᳚ । श्या॒माका॑ अभवन्न् । अ॒भ॒व॒न्थ् सः । स प्र॒जाप॑तिम् । प्र॒जाप॑ति॒मुप॑ । प्र॒जाप॑ति॒मिति॑ प्र॒जा - प॒ति॒म् । उपा॑धावत् । अ॒धा॒व॒त् तस्मै᳚ । तस्मा॑ ए॒तम् । ए॒तꣳ सो॑मे॒न्द्रम् । सो॒मे॒न्द्रꣳ श्या॑मा॒कम् । श्या॒मा॒कम् च॒रुम् । च॒रुम् निः । निर॑वपत् । अ॒व॒प॒त् तेन॑ । तेनै॒व । ए॒वास्मिन्न्॑ । अ॒स्मि॒न्नि॒न्द्रि॒यम् । इ॒न्द्रि॒यꣳ सो॑मपी॒थम् । सो॒म॒पी॒थम॑दधात् । सो॒म॒पी॒थमिति॑ सोम - पी॒थम् । अ॒द॒धा॒द् वि । वि वै । वा ए॒षः । ए॒ष इ॑न्द्रि॒येण॑ । इ॒न्द्रि॒येण॑ सोमपी॒थेन॑ । सो॒म॒पी॒थेन॑र्द्ध्यते । सो॒म॒पी॒थेनेति॑ सोम - पी॒थेन॑ । ऋ॒द्ध्य॒ते॒ यः । यः सोम᳚म् । सोमं॒ ॅवमि॑ति । वमि॑ति॒ यः । यः सो॑मवा॒मी । सो॒म॒वा॒मी स्यात् । सो॒म॒वा॒मीति॑ सोम - वा॒मी । स्यात् तस्मै᳚ । तस्मा॑ ए॒तम् \newline

\textbf{Jatai Paata} \newline

1. त्वष्टुः॒ सोमꣳ॒॒ सोम॒म् त्वष्टु॒ स्त्वष्टुः॒ सोम᳚म् । \newline
2. सोम॑ मभी॒षहा॑ ऽभी॒षहा॒ सोमꣳ॒॒ सोम॑ मभी॒षहा᳚ । \newline
3. अ॒भी॒षहा॑ ऽपिब दपिब दभी॒षहा॑ ऽभी॒षहा॑ ऽपिबत् । \newline
4. अ॒भी॒षहेत्य॑भि - सहा᳚ । \newline
5. अ॒पि॒ब॒थ् स सो॑ ऽपिब दपिब॒थ् सः । \newline
6. स विष्व॒ङ्॒. विष्व॒ङ् ख्स स विष्वङ्॑ । \newline
7. विष्व॒ङ्.॒ वि वि विष्व॒ङ्.॒ विष्व॒ङ्.॒ वि । \newline
8. व्या᳚र्च्छ दार्च्छ॒द् वि व्या᳚र्च्छत् । \newline
9. आ॒र्च्छ॒थ् स स आ᳚र्च्छ दार्च्छ॒थ् सः । \newline
10. स इ॑न्द्रि॒येणे᳚ न्द्रि॒येण॒ स स इ॑न्द्रि॒येण॑ । \newline
11. इ॒न्द्रि॒येण॑ सोमपी॒थेन॑ सोमपी॒थेने᳚ न्द्रि॒येणे᳚ न्द्रि॒येण॑ सोमपी॒थेन॑ । \newline
12. सो॒म॒पी॒थेन॒ वि वि सो॑मपी॒थेन॑ सोमपी॒थेन॒ वि । \newline
13. सो॒म॒पी॒थेनेति॑ सोम - पी॒थेन॑ । \newline
14. व्या᳚र्द्ध्यता र्द्ध्यत॒ वि व्या᳚र्द्ध्यत । \newline
15. आ॒र्द्ध्य॒त॒ स स आ᳚र्द्ध्यता र्द्ध्यत॒ सः । \newline
16. स यद् यथ् स स यत् । \newline
17. यदू॒र्द्ध्व मू॒र्द्ध्वं ॅयद् यदू॒र्द्ध्वम् । \newline
18. ऊ॒र्द्ध्व मु॒दव॑मी दु॒दव॑मी दू॒र्द्ध्व मू॒र्द्ध्व मु॒दव॑मीत् । \newline
19. उ॒दव॑मी॒त् ते त उ॒दव॑मी दु॒दव॑मी॒त् ते । \newline
20. उ॒दव॑मी॒दित्यु॑त् - अव॑मीत् । \newline
21. ते श्या॒माकाः᳚ श्या॒माका॒ स्ते ते श्या॒माकाः᳚ । \newline
22. श्या॒माका॑ अभवन् नभवञ् छ्या॒माकाः᳚ श्या॒माका॑ अभवन्न् । \newline
23. अ॒भ॒व॒न् थ्स सो॑ ऽभवन् नभव॒न् थ्सः । \newline
24. स प्र॒जाप॑तिम् प्र॒जाप॑तिꣳ॒॒ स स प्र॒जाप॑तिम् । \newline
25. प्र॒जाप॑ति॒ मुपोप॑ प्र॒जाप॑तिम् प्र॒जाप॑ति॒ मुप॑ । \newline
26. प्र॒जाप॑ति॒मिति॑ प्र॒जा - प॒ति॒म् । \newline
27. उपा॑धाव दधाव॒ दुपोपा॑ धावत् । \newline
28. अ॒धा॒व॒त् तस्मै॒ तस्मा॑ अधाव दधाव॒त् तस्मै᳚ । \newline
29. तस्मा॑ ए॒त मे॒तम् तस्मै॒ तस्मा॑ ए॒तम् । \newline
30. ए॒तꣳ सो॑मे॒न्द्रꣳ सो॑मे॒न्द्र मे॒त मे॒तꣳ सो॑मे॒न्द्रम् । \newline
31. सो॒मे॒न्द्रꣳ श्या॑मा॒कꣳ श्या॑मा॒कꣳ सो॑मे॒न्द्रꣳ सो॑मे॒न्द्रꣳ श्या॑मा॒कम् । \newline
32. श्या॒मा॒कम् च॒रुम् च॒रुꣳ श्या॑मा॒कꣳ श्या॑मा॒कम् च॒रुम् । \newline
33. च॒रुम् निर् णिश्च॒रुम् च॒रुम् निः । \newline
34. निर॑वप दवप॒न् निर् णि र॑वपत् । \newline
35. अ॒व॒प॒त् तेन॒ तेना॑ वप दवप॒त् तेन॑ । \newline
36. तेनै॒वैव तेन॒ तेनै॒व । \newline
37. ए॒वास्मि॑न् नस्मिन् ने॒वैवास्मिन्न्॑ । \newline
38. अ॒स्मि॒न् नि॒न्द्रि॒य मि॑न्द्रि॒य म॑स्मिन् नस्मिन् निन्द्रि॒यम् । \newline
39. इ॒न्द्रि॒यꣳ सो॑मपी॒थꣳ सो॑मपी॒थ मि॑न्द्रि॒य मि॑न्द्रि॒यꣳ सो॑मपी॒थम् । \newline
40. सो॒म॒पी॒थ म॑दधाददधाथ् सोमपी॒थꣳ सो॑मपी॒थ म॑दधात् । \newline
41. सो॒म॒पी॒थमिति॑ सोम - पी॒थम् । \newline
42. अ॒द॒धा॒द् वि व्य॑दधा ददधा॒द् वि । \newline
43. वि वै वै वि वि वै । \newline
44. वा ए॒ष ए॒ष वै वा ए॒षः । \newline
45. ए॒ष इ॑न्द्रि॒येणे᳚ न्द्रि॒येणै॒ष ए॒ष इ॑न्द्रि॒येण॑ । \newline
46. इ॒न्द्रि॒येण॑ सोमपी॒थेन॑ सोमपी॒थेने᳚ न्द्रि॒येणे᳚ न्द्रि॒येण॑ सोमपी॒थेन॑ । \newline
47. सो॒म॒पी॒थेन॑ र्द्ध्यत ऋद्ध्यते सोमपी॒थेन॑ सोमपी॒थेन॑ र्द्ध्यते । \newline
48. सो॒म॒पी॒थेनेति॑ सोम - पी॒थेन॑ । \newline
49. ऋ॒द्ध्य॒ते॒ यो य ऋ॑द्ध्यत ऋद्ध्यते॒ यः । \newline
50. यः सोमꣳ॒॒ सोमं॒ ॅयो यः सोम᳚म् । \newline
51. सोमं॒ ॅवमि॑ति॒ वमि॑ति॒ सोमꣳ॒॒ सोमं॒ ॅवमि॑ति । \newline
52. वमि॑ति॒ यो यो वमि॑ति॒ वमि॑ति॒ यः । \newline
53. यः सो॑मवा॒मी सो॑मवा॒मी यो यः सो॑मवा॒मी । \newline
54. सो॒म॒वा॒मी स्याथ् स्याथ् सो॑मवा॒मी सो॑मवा॒मी स्यात् । \newline
55. सो॒म॒वा॒मीति॑ सोम - वा॒मी । \newline
56. स्यात् तस्मै॒ तस्मै॒ स्याथ् स्यात् तस्मै᳚ । \newline
57. तस्मा॑ ए॒त मे॒तम् तस्मै॒ तस्मा॑ ए॒तम् । \newline

\textbf{Ghana Paata } \newline

1. त्वष्टुः॒ सोमꣳ॒॒ सोम॒म् त्वष्टु॒ स्त्वष्टुः॒ सोम॑ मभी॒षहा॑ ऽभी॒षहा॒ सोम॒म् त्वष्टु॒ स्त्वष्टुः॒ सोम॑ मभी॒षहा᳚ । \newline
2. सोम॑ मभी॒षहा॑ ऽभी॒षहा॒ सोमꣳ॒॒ सोम॑ मभी॒षहा॑ ऽपिब दपिब दभी॒षहा॒ सोमꣳ॒॒ सोम॑ मभी॒षहा॑ ऽपिबत् । \newline
3. अ॒भी॒षहा॑ ऽपिब दपिब दभी॒षहा॑ ऽभी॒षहा॑ ऽपिब॒थ् स सो॑ ऽपिब दभी॒षहा॑ ऽभी॒षहा॑ ऽपिब॒थ् सः । \newline
4. अ॒भी॒षहेत्य॑भि - सहा᳚ । \newline
5. अ॒पि॒ब॒थ् स सो॑ ऽपिब दपिब॒थ् स विष्व॒ङ्॒. विष्व॒ङ् ख्सो॑ ऽपिब दपिब॒थ् स विष्वङ्॑ । \newline
6. स विष्व॒ङ्॒. विष्व॒ङ् ख्स स विष्व॒ङ्॒. वि वि विष्व॒ङ् ख्स स विष्व॒ङ्॒. वि । \newline
7. विष्व॒ङ्॒. वि वि विष्व॒ङ्॒. विष्व॒ङ् व्या᳚र्च्छ दार्च्छ॒द् वि विष्व॒ङ्॒. विष्व॒ङ् व्या᳚र्च्छत् । \newline
8. व्या᳚र्च्छ दार्च्छ॒द् वि व्या᳚र्च्छ॒थ् स स आ᳚र्च्छ॒द् वि व्या᳚र्च्छ॒थ् सः । \newline
9. आ॒र्च्छ॒थ् स स आ᳚र्च्छ दार्च्छ॒थ् स इ॑न्द्रि॒येणे᳚ न्द्रि॒येण॒ स आ᳚र्च्छ दार्च्छ॒थ् स इ॑न्द्रि॒येण॑ । \newline
10. स इ॑न्द्रि॒येणे᳚ न्द्रि॒येण॒ स स इ॑न्द्रि॒येण॑ सोमपी॒थेन॑ सोमपी॒थेने᳚ न्द्रि॒येण॒ स स इ॑न्द्रि॒येण॑ सोमपी॒थेन॑ । \newline
11. इ॒न्द्रि॒येण॑ सोमपी॒थेन॑ सोमपी॒थेने᳚ न्द्रि॒येणे᳚ न्द्रि॒येण॑ सोमपी॒थेन॒ वि वि सो॑मपी॒थेने᳚ न्द्रि॒येणे᳚ न्द्रि॒येण॑ सोमपी॒थेन॒ वि । \newline
12. सो॒म॒पी॒थेन॒ वि वि सो॑मपी॒थेन॑ सोमपी॒थेन॒ व्या᳚र्द्ध्य तार्द्ध्यत॒ वि सो॑मपी॒थेन॑ सोमपी॒थेन॒ व्या᳚र्द्ध्यत । \newline
13. सो॒म॒पी॒थेनेति॑ सोम - पी॒थेन॑ । \newline
14. व्या᳚र्द्ध्य तार्द्ध्यत॒ वि व्या᳚र्द्ध्यत॒ स स आ᳚र्द्ध्यत॒ वि व्या᳚र्द्ध्यत॒ सः । \newline
15. आ॒र्द्ध्य॒त॒ स स आ᳚र्द्ध्य तार्द्ध्यत॒ स यद् यथ् स आ᳚र्द्ध्य तार्द्ध्यत॒ स यत् । \newline
16. स यद् यथ् स स यदू॒र्द्ध्व मू॒र्द्ध्वं ॅयथ् स स यदू॒र्द्ध्वम् । \newline
17. यदू॒र्द्ध्व मू॒र्द्ध्वं ॅयद् यदू॒र्द्ध्व मु॒दव॑मी दु॒दव॑मी दू॒र्द्ध्वं ॅयद् यदू॒र्द्ध्व मु॒दव॑मीत् । \newline
18. ऊ॒र्द्ध्व मु॒दव॑मी दु॒दव॑मी दू॒र्द्ध्व मू॒र्द्ध्व मु॒दव॑मी॒त् ते त उ॒दव॑मी दू॒र्द्ध्व मू॒र्द्ध्व मु॒दव॑मी॒त् ते । \newline
19. उ॒दव॑मी॒त् ते त उ॒दव॑मी दु॒दव॑मी॒त् ते श्या॒माकाः᳚ श्या॒माका॒ स्त उ॒दव॑मी दु॒दव॑मी॒त् ते श्या॒माकाः᳚ । \newline
20. उ॒दव॑मी॒दित्यु॑त् - अव॑मीत् । \newline
21. ते श्या॒माकाः᳚ श्या॒माका॒ स्ते ते श्या॒माका॑ अभवन् नभवञ् छ्या॒माका॒ स्ते ते श्या॒माका॑ अभवन्न् । \newline
22. श्या॒माका॑ अभवन् नभवञ् छ्या॒माकाः᳚ श्या॒माका॑ अभव॒न् थ्स सो॑ ऽभवञ् छ्या॒माकाः᳚ श्या॒माका॑ अभव॒न् थ्सः । \newline
23. अ॒भ॒व॒न् थ्स सो॑ ऽभवन् नभव॒न् थ्स प्र॒जाप॑तिम् प्र॒जाप॑तिꣳ॒॒ सो॑ ऽभवन् नभव॒न् थ्स प्र॒जाप॑तिम् । \newline
24. स प्र॒जाप॑तिम् प्र॒जाप॑तिꣳ॒॒ स स प्र॒जाप॑ति॒ मुपोप॑ प्र॒जाप॑तिꣳ॒॒ स स प्र॒जाप॑ति॒ मुप॑ । \newline
25. प्र॒जाप॑ति॒ मुपोप॑ प्र॒जाप॑तिम् प्र॒जाप॑ति॒ मुपा॑धाव दधाव॒दुप॑ प्र॒जाप॑तिम् प्र॒जाप॑ति॒ मुपा॑धावत् । \newline
26. प्र॒जाप॑ति॒मिति॑ प्र॒जा - प॒ति॒म् । \newline
27. उपा॑धावद धाव॒ दुपोपा॑धाव॒त् तस्मै॒ तस्मा॑ अधाव॒ दुपोपा॑धाव॒त् तस्मै᳚ । \newline
28. अ॒धा॒व॒त् तस्मै॒ तस्मा॑ अधाव दधाव॒त् तस्मा॑ ए॒त मे॒तम् तस्मा॑ अधाव दधाव॒त् तस्मा॑ ए॒तम् । \newline
29. तस्मा॑ ए॒त मे॒तम् तस्मै॒ तस्मा॑ ए॒तꣳ सो॑मे॒न्द्रꣳ सो॑मे॒न्द्र मे॒तम् तस्मै॒ तस्मा॑ ए॒तꣳ सो॑मे॒न्द्रम् । \newline
30. ए॒तꣳ सो॑मे॒न्द्रꣳ सो॑मे॒न्द्र मे॒त मे॒तꣳ सो॑मे॒न्द्रꣳ श्या॑मा॒कꣳ श्या॑मा॒कꣳ सो॑मे॒न्द्र मे॒त मे॒तꣳ सो॑मे॒न्द्रꣳ श्या॑मा॒कम् । \newline
31. सो॒मे॒न्द्रꣳ श्या॑मा॒कꣳ श्या॑मा॒कꣳ सो॑मे॒न्द्रꣳ सो॑मे॒न्द्रꣳ श्या॑मा॒कम् च॒रुम् च॒रुꣳ श्या॑मा॒कꣳ सो॑मे॒न्द्रꣳ सो॑मे॒न्द्रꣳ श्या॑मा॒कम् च॒रुम् । \newline
32. श्या॒मा॒कम् च॒रुम् च॒रुꣳ श्या॑मा॒कꣳ श्या॑मा॒कम् च॒रुम् निर् णिश्च॒रुꣳ श्या॑मा॒कꣳ श्या॑मा॒कम् च॒रुम् निः । \newline
33. च॒रुम् निर् णिश्च॒रुम् च॒रुम् निर॑वप दवप॒न् निश्च॒रुम् च॒रुम् निर॑वपत् । \newline
34. निर॑वप दवप॒न् निर् णिर॑वप॒त् तेन॒ तेना॑वप॒न् निर् णिर॑वप॒त् तेन॑ । \newline
35. अ॒व॒प॒त् तेन॒ तेना॑वप दवप॒त् तेनै॒वैव तेना॑वप दवप॒त् तेनै॒व । \newline
36. तेनै॒वैव तेन॒ तेनै॒वास्मि॑न् नस्मिन् ने॒व तेन॒ तेनै॒वास्मिन्न्॑ । \newline
37. ए॒वास्मि॑न् नस्मिन् ने॒वैवास्मि॑न् निन्द्रि॒य मि॑न्द्रि॒य म॑स्मिन् ने॒वैवास्मि॑न् निन्द्रि॒यम् । \newline
38. अ॒स्मि॒न् नि॒न्द्रि॒य मि॑न्द्रि॒य म॑स्मिन् नस्मिन् निन्द्रि॒यꣳ सो॑मपी॒थꣳ सो॑मपी॒थ मि॑न्द्रि॒य म॑स्मिन् नस्मिन् निन्द्रि॒यꣳ सो॑मपी॒थम् । \newline
39. इ॒न्द्रि॒यꣳ सो॑मपी॒थꣳ सो॑मपी॒थ मि॑न्द्रि॒य मि॑न्द्रि॒यꣳ सो॑मपी॒थ म॑दधा ददधाथ् सोमपी॒थ मि॑न्द्रि॒य मि॑न्द्रि॒यꣳ सो॑मपी॒थ म॑दधात् । \newline
40. सो॒म॒पी॒थ म॑दधा ददधाथ् सोमपी॒थꣳ सो॑मपी॒थ म॑दधा॒द् वि व्य॑दधाथ् सोमपी॒थꣳ सो॑मपी॒थ म॑दधा॒द् वि । \newline
41. सो॒म॒पी॒थमिति॑ सोम - पी॒थम् । \newline
42. अ॒द॒धा॒द् वि व्य॑दधा ददधा॒द् वि वै वै व्य॑दधा ददधा॒द् वि वै । \newline
43. वि वै वै वि वि वा ए॒ष ए॒ष वै वि वि वा ए॒षः । \newline
44. वा ए॒ष ए॒ष वै वा ए॒ष इ॑न्द्रि॒येणे᳚ न्द्रि॒येणै॒ष वै वा ए॒ष इ॑न्द्रि॒येण॑ । \newline
45. ए॒ष इ॑न्द्रि॒येणे᳚ न्द्रि॒येणै॒ष ए॒ष इ॑न्द्रि॒येण॑ सोमपी॒थेन॑ सोमपी॒थेने᳚ न्द्रि॒येणै॒ष ए॒ष इ॑न्द्रि॒येण॑ सोमपी॒थेन॑ । \newline
46. इ॒न्द्रि॒येण॑ सोमपी॒थेन॑ सोमपी॒थेने᳚ न्द्रि॒येणे᳚ न्द्रि॒येण॑ सोमपी॒थेन॑ र्‌द्ध्यत ऋद्ध्यते सोमपी॒थेने᳚ न्द्रि॒येणे᳚ न्द्रि॒येण॑ सोमपी॒थेन॑ र्‌द्ध्यते । \newline
47. सो॒म॒पी॒थेन॑ र्‌द्ध्यत ऋद्ध्यते सोमपी॒थेन॑ सोमपी॒थेन॑ र्‌द्ध्यते॒ यो य ऋ॑द्ध्यते सोमपी॒थेन॑ सोमपी॒थेन॑ र्‌द्ध्यते॒ यः । \newline
48. सो॒म॒पी॒थेनेति॑ सोम - पी॒थेन॑ । \newline
49. ऋ॒द्ध्य॒ते॒ यो य ऋ॑द्ध्यत ऋद्ध्यते॒ यः सोमꣳ॒॒ सोमं॒ ॅय ऋ॑द्ध्यत ऋद्ध्यते॒ यः सोम᳚म् । \newline
50. यः सोमꣳ॒॒ सोमं॒ ॅयो यः सोमं॒ ॅवमि॑ति॒ वमि॑ति॒ सोमं॒ ॅयो यः सोमं॒ ॅवमि॑ति । \newline
51. सोमं॒ ॅवमि॑ति॒ वमि॑ति॒ सोमꣳ॒॒ सोमं॒ ॅवमि॑ति॒ यो यो वमि॑ति॒ सोमꣳ॒॒ सोमं॒ ॅवमि॑ति॒ यः । \newline
52. वमि॑ति॒ यो यो वमि॑ति॒ वमि॑ति॒ यः सो॑मवा॒मी सो॑मवा॒मी यो वमि॑ति॒ वमि॑ति॒ यः सो॑मवा॒मी । \newline
53. यः सो॑मवा॒मी सो॑मवा॒मी यो यः सो॑मवा॒मी स्याथ् स्याथ् सो॑मवा॒मी यो यः सो॑मवा॒मी स्यात् । \newline
54. सो॒म॒वा॒मी स्याथ् स्याथ् सो॑मवा॒मी सो॑मवा॒मी स्यात् तस्मै॒ तस्मै॒ स्याथ् सो॑मवा॒मी सो॑मवा॒मी स्यात् तस्मै᳚ । \newline
55. सो॒म॒वा॒मीति॑ सोम - वा॒मी । \newline
56. स्यात् तस्मै॒ तस्मै॒ स्याथ् स्यात् तस्मा॑ ए॒त मे॒तम् तस्मै॒ स्याथ् स्यात् तस्मा॑ ए॒तम् । \newline
57. तस्मा॑ ए॒त मे॒तम् तस्मै॒ तस्मा॑ ए॒तꣳ सो॑मे॒न्द्रꣳ सो॑मे॒न्द्र मे॒तम् तस्मै॒ तस्मा॑ ए॒तꣳ सो॑मे॒न्द्रम् । \newline
\pagebreak
\markright{ TS 2.3.2.7  \hfill https://www.vedavms.in \hfill}

\section{ TS 2.3.2.7 }

\textbf{TS 2.3.2.7 } \newline
\textbf{Samhita Paata} \newline

ए॒तꣳ सो॑मे॒न्द्रꣳ श्या॑मा॒कं च॒रुं निर्व॑पे॒थ् सोमं॑ चै॒वेन्द्रं॑ च॒ स्वेन॑ भाग॒धेये॒नोप॑ धावति॒ तावे॒वास्मि॑न्निन्द्रि॒यꣳ सो॑मपी॒थं ध॑त्तो॒ नेन्द्रि॒येण॑ सोमपी॒थेन॒ व्यृ॑द्ध्यते॒ यथ् सौ॒म्यो भव॑ति सोमपी॒थमे॒वाव॑ रुन्धे॒ यदै॒न्द्रो भव॑तीन्द्रि॒यं ॅवै सो॑मपी॒थ इ॑न्द्रि॒यमे॒व सो॑मपी॒थमव॑ रुन्धे श्यामा॒को भ॑वत्ये॒ष वाव स सोमः॑ - [  ] \newline

\textbf{Pada Paata} \newline

ए॒तम् । सो॒मे॒न्द्रम् । श्या॒मा॒कम् । च॒रुम् । निरिति॑ । व॒पे॒त् । सोम᳚म् । च॒ । ए॒व । इन्द्र᳚म् । च॒ । स्वेन॑ । भा॒ग॒धेये॒नेति॑ भाग-धेये॑न । उपेति॑ । धा॒व॒ति॒ । तौ । ए॒व । अ॒स्मि॒न्न् । इ॒न्द्रि॒यम् । सो॒म॒पी॒थमिति॑ सोम - पी॒थम् । ध॒त्तः॒ । न । इ॒न्द्रि॒येण॑ । सो॒म॒पी॒थेनेति॑ सोम - पी॒थेन॑ । वीति॑ । ऋ॒द्ध्य॒ते॒ । यत् । सौ॒म्यः । भव॑ति । सो॒म॒पी॒थमिति॑ सोम - पी॒थम् । ए॒व । अवेति॑ । रु॒न्धे॒ । यत् । ऐ॒न्द्रः । भव॑ति । इ॒न्द्रि॒यम् । वै । सो॒म॒पी॒थ इति॑ सोम - पी॒थः । इ॒न्द्रि॒यम् । ए॒व । सो॒म॒पी॒थमिति॑ सोम - पी॒थम् । अवेति॑ । रु॒न्धे॒ । श्या॒मा॒कः । भ॒व॒ति॒ । ए॒षः । वाव । सः । सोमः॑ ।  \newline


\textbf{Krama Paata} \newline

ए॒तꣳ सो॑मे॒न्द्रम् । सो॒मे॒न्द्रꣳ श्या॑मा॒कम् । श्या॒मा॒कम् च॒रुम् । च॒रुम् निः । निर् व॑पेत् । व॒पे॒थ् सोम᳚म् । सोम॑म् च । चै॒व । ए॒वेन्द्र᳚म् । इन्द्र॑म् च । च॒ स्वेन॑ । स्वेन॑ भाग॒धेये॑न । भा॒ग॒धेये॒नोप॑ । भा॒ग॒धेये॒नेति॑ भाग - धेये॑न । उप॑ धावति । धा॒व॒ति॒ तौ । तावे॒व । ए॒वास्मिन्न्॑ । अ॒स्मि॒न्नि॒न्द्रि॒यम् । इ॒न्द्रि॒यꣳ सो॑मपी॒थम् । सो॒म॒पी॒थम् ध॑त्तः । सो॒म॒पी॒थमिति॑ सोम - पी॒थम् । ध॒त्तो॒ न । नेन्द्रि॒येण॑ । इ॒न्द्रि॒येण॑ सोमपी॒थेन॑ । सो॒म॒पी॒थेन॒ वि । सो॒म॒पी॒थेनेति॑ सोम - पी॒थेन॑ । व्यृ॑ध्यते । ऋ॒द्ध्य॒ते॒ यत् । यथ् सौ॒म्यः । सौ॒म्यो भव॑ति । भव॑ति सोमपी॒थम् । सो॒म॒पी॒थमे॒व । सो॒म॒पी॒थमिति॑ सोम - पी॒थम् । ए॒वाव॑ । अव॑ रुन्धे । रु॒न्धे॒ यत् । यदै॒न्द्रः । ऐ॒न्द्रो भव॑ति । भव॑तीन्द्रि॒यम् । इ॒न्द्रि॒यं ॅवै । वै सो॑मपी॒थः । सो॒म॒पी॒थ इ॑न्द्रि॒यम् । सो॒म॒पी॒थ इति॑ सोम - पी॒थः । इ॒न्द्रि॒यमे॒व । ए॒व सो॑मपी॒थम् । सो॒म॒पी॒थमव॑ । सो॒म॒पी॒थमिति॑ सोम - पी॒थम् । अव॑ रुन्धे । रु॒न्धे॒ श्या॒मा॒कः । श्या॒मा॒को भ॑वति । भ॒व॒त्ये॒षः । ए॒ष वाव । वाव सः । स सोमः॑ । सोमः॑ सा॒क्षात् \newline

\textbf{Jatai Paata} \newline

1. ए॒तꣳ सो॑मे॒न्द्रꣳ सो॑मे॒न्द्र मे॒त मे॒तꣳ सो॑मे॒न्द्रम् । \newline
2. सो॒मे॒न्द्रꣳ श्या॑मा॒कꣳ श्या॑मा॒कꣳ सो॑मे॒न्द्रꣳ सो॑मे॒न्द्रꣳ श्या॑मा॒कम् । \newline
3. श्या॒मा॒कम् च॒रुम् च॒रुꣳ श्या॑मा॒कꣳ श्या॑मा॒कम् च॒रुम् । \newline
4. च॒रुम् निर् णि श्च॒रुम् च॒रुम् निः । \newline
5. निर् व॑पेद् वपे॒न् निर् णिर् व॑पेत् । \newline
6. व॒पे॒थ् सोमꣳ॒॒ सोमं॑ ॅवपेद् वपे॒थ् सोम᳚म् । \newline
7. सोम॑म् च च॒ सोमꣳ॒॒ सोम॑म् च । \newline
8. चै॒वैव च॑ चै॒व । \newline
9. ए॒वे न्द्र॒ मिन्द्र॑ मे॒वैवे न्द्र᳚म् । \newline
10. इन्द्र॑म् च॒ चे न्द्र॒ मिन्द्र॑म् च । \newline
11. च॒ स्वेन॒ स्वेन॑ च च॒ स्वेन॑ । \newline
12. स्वेन॑ भाग॒धेये॑न भाग॒धेये॑न॒ स्वेन॒ स्वेन॑ भाग॒धेये॑न । \newline
13. भा॒ग॒धेये॒नोपोप॑ भाग॒धेये॑न भाग॒धेये॒नोप॑ । \newline
14. भा॒ग॒धेये॒नेति॑ भाग - धेये॑न । \newline
15. उप॑ धावति धाव॒ त्युपोप॑ धावति । \newline
16. धा॒व॒ति॒ तौ तौ धा॑वति धावति॒ तौ । \newline
17. ता वे॒वैव तौ ता वे॒व । \newline
18. ए॒वास्मि॑न् नस्मिन् ने॒वैवास्मिन्न्॑ । \newline
19. अ॒स्मि॒न् नि॒न्द्रि॒य मि॑न्द्रि॒य म॑स्मिन् नस्मिन् निन्द्रि॒यम् । \newline
20. इ॒न्द्रि॒यꣳ सो॑मपी॒थꣳ सो॑मपी॒थ मि॑न्द्रि॒य मि॑न्द्रि॒यꣳ सो॑मपी॒थम् । \newline
21. सो॒म॒पी॒थम् ध॑त्तो धत्तः सोमपी॒थꣳ सो॑मपी॒थम् ध॑त्तः । \newline
22. सो॒म॒पी॒थमिति॑ सोम - पी॒थम् । \newline
23. ध॒त्तो॒ न न ध॑त्तो धत्तो॒ न । \newline
24. ने न्द्रि॒येणे᳚ न्द्रि॒येण॒ न ने न्द्रि॒येण॑ । \newline
25. इ॒न्द्रि॒येण॑ सोमपी॒थेन॑ सोमपी॒थेने᳚ न्द्रि॒येणे᳚ न्द्रि॒येण॑ सोमपी॒थेन॑ । \newline
26. सो॒म॒पी॒थेन॒ वि वि सो॑मपी॒थेन॑ सोमपी॒थेन॒ वि । \newline
27. सो॒म॒पी॒थेनेति॑ सोम - पी॒थेन॑ । \newline
28. व्यृ॑द्ध्यत ऋद्ध्यते॒ वि व्यृ॑द्ध्यते । \newline
29. ऋ॒द्ध्य॒ते॒ यद् यदृ॑द्ध्यत ऋद्ध्यते॒ यत् । \newline
30. यथ् सौ॒म्यः सौ॒म्यो यद् यथ् सौ॒म्यः । \newline
31. सौ॒म्यो भव॑ति॒ भव॑ति सौ॒म्यः सौ॒म्यो भव॑ति । \newline
32. भव॑ति सोमपी॒थꣳ सो॑मपी॒थम् भव॑ति॒ भव॑ति सोमपी॒थम् । \newline
33. सो॒म॒पी॒थ मे॒वैव सो॑मपी॒थꣳ सो॑मपी॒थ मे॒व । \newline
34. सो॒म॒पी॒थमिति॑ सोम - पी॒थम् । \newline
35. ए॒वावा वै॒वैवाव॑ । \newline
36. अव॑ रुन्धे रु॒न्धे ऽवाव॑ रुन्धे । \newline
37. रु॒न्धे॒ यद् यद् रु॑न्धे रुन्धे॒ यत् । \newline
38. यदै॒न्द्र ऐ॒न्द्रो यद् यदै॒न्द्रः । \newline
39. ऐ॒न्द्रो भव॑ति॒ भव॑ त्यै॒न्द्र ऐ॒न्द्रो भव॑ति । \newline
40. भव॑तीन्द्रि॒य मि॑न्द्रि॒यम् भव॑ति॒ भव॑तीन्द्रि॒यम् । \newline
41. इ॒न्द्रि॒यं ॅवै वा इ॑न्द्रि॒य मि॑न्द्रि॒यं ॅवै । \newline
42. वै सो॑मपी॒थः सो॑मपी॒थो वै वै सो॑मपी॒थः । \newline
43. सो॒म॒पी॒थ इ॑न्द्रि॒य मि॑न्द्रि॒यꣳ सो॑मपी॒थः सो॑मपी॒थ इ॑न्द्रि॒यम् । \newline
44. सो॒म॒पी॒थ इति॑ सोम - पी॒थः । \newline
45. इ॒न्द्रि॒य मे॒वैवे न्द्रि॒य मि॑न्द्रि॒य मे॒व । \newline
46. ए॒व सो॑मपी॒थꣳ सो॑मपी॒थ मे॒वैव सो॑मपी॒थम् । \newline
47. सो॒म॒पी॒थ मवाव॑ सोमपी॒थꣳ सो॑मपी॒थ मव॑ । \newline
48. सो॒म॒पी॒थमिति॑ सोम - पी॒थम् । \newline
49. अव॑ रुन्धे रु॒न्धे ऽवाव॑ रुन्धे । \newline
50. रु॒न्धे॒ श्या॒मा॒कः श्या॑मा॒को रु॑न्धे रुन्धे श्यामा॒कः । \newline
51. श्या॒मा॒को भ॑वति भवति श्यामा॒कः श्या॑मा॒को भ॑वति । \newline
52. भ॒व॒ त्ये॒ष ए॒ष भ॑वति भव त्ये॒षः । \newline
53. ए॒ष वाव वावैष ए॒ष वाव । \newline
54. वाव स स वाव वाव सः । \newline
55. स सोमः॒ सोमः॒ स स सोमः॑ । \newline
56. सोमः॑ सा॒क्षाथ् सा॒क्षाथ् सोमः॒ सोमः॑ सा॒क्षात् । \newline

\textbf{Ghana Paata } \newline

1. ए॒तꣳ सो॑मे॒न्द्रꣳ सो॑मे॒न्द्र मे॒त मे॒तꣳ सो॑मे॒न्द्रꣳ श्या॑मा॒कꣳ श्या॑मा॒कꣳ सो॑मे॒न्द्र मे॒त मे॒तꣳ सो॑मे॒न्द्रꣳ श्या॑मा॒कम् । \newline
2. सो॒मे॒न्द्रꣳ श्या॑मा॒कꣳ श्या॑मा॒कꣳ सो॑मे॒न्द्रꣳ सो॑मे॒न्द्रꣳ श्या॑मा॒कम् च॒रुम् च॒रुꣳ श्या॑मा॒कꣳ सो॑मे॒न्द्रꣳ सो॑मे॒न्द्रꣳ श्या॑मा॒कम् च॒रुम् । \newline
3. श्या॒मा॒कम् च॒रुम् च॒रुꣳ श्या॑मा॒कꣳ श्या॑मा॒कम् च॒रुम् निर् णिश्च॒रुꣳ श्या॑मा॒कꣳ श्या॑मा॒कम् च॒रुम् निः । \newline
4. च॒रुम् निर् णिश्च॒रुम् च॒रुम् निर् व॑पेद् वपे॒न् निश्च॒रुम् च॒रुम् निर् व॑पेत् । \newline
5. निर् व॑पेद् वपे॒न् निर् णिर् व॑पे॒थ् सोमꣳ॒॒ सोमं॑ ॅवपे॒न् निर् णिर् व॑पे॒थ् सोम᳚म् । \newline
6. व॒पे॒थ् सोमꣳ॒॒ सोमं॑ ॅवपेद् वपे॒थ् सोम॑म् च च॒ सोमं॑ ॅवपेद् वपे॒थ् सोम॑म् च । \newline
7. सोम॑म् च च॒ सोमꣳ॒॒ सोम॑म् चै॒वैव च॒ सोमꣳ॒॒ सोम॑म् चै॒व । \newline
8. चै॒वैव च॑ चै॒वे न्द्र॒ मिन्द्र॑ मे॒व च॑ चै॒वे न्द्र᳚म् । \newline
9. ए॒वे न्द्र॒ मिन्द्र॑ मे॒वैवे न्द्र॑म् च॒ चे न्द्र॑ मे॒वैवे न्द्र॑म् च । \newline
10. इन्द्र॑म् च॒ चे न्द्र॒ मिन्द्र॑म् च॒ स्वेन॒ स्वेन॒ चे न्द्र॒ मिन्द्र॑म् च॒ स्वेन॑ । \newline
11. च॒ स्वेन॒ स्वेन॑ च च॒ स्वेन॑ भाग॒धेये॑न भाग॒धेये॑न॒ स्वेन॑ च च॒ स्वेन॑ भाग॒धेये॑न । \newline
12. स्वेन॑ भाग॒धेये॑न भाग॒धेये॑न॒ स्वेन॒ स्वेन॑ भाग॒धेये॒नोपोप॑ भाग॒धेये॑न॒ स्वेन॒ स्वेन॑ भाग॒धेये॒नोप॑ । \newline
13. भा॒ग॒धेये॒नोपोप॑ भाग॒धेये॑न भाग॒धेये॒नोप॑ धावति धाव॒त्युप॑ भाग॒धेये॑न भाग॒धेये॒नोप॑ धावति । \newline
14. भा॒ग॒धेये॒नेति॑ भाग - धेये॑न । \newline
15. उप॑ धावति धाव॒ त्युपोप॑ धावति॒ तौ तौ धा॑व॒ त्युपोप॑ धावति॒ तौ । \newline
16. धा॒व॒ति॒ तौ तौ धा॑वति धावति॒ ता वे॒वैव तौ धा॑वति धावति॒ ता वे॒व । \newline
17. ता वे॒वैव तौ ता वे॒वास्मि॑न् नस्मिन् ने॒व तौ ता वे॒वास्मिन्न्॑ । \newline
18. ए॒वास्मि॑न् नस्मिन् ने॒वैवास्मि॑न् निन्द्रि॒य मि॑न्द्रि॒य म॑स्मिन् ने॒वैवास्मि॑न् निन्द्रि॒यम् । \newline
19. अ॒स्मि॒न् नि॒न्द्रि॒य मि॑न्द्रि॒य म॑स्मिन् नस्मिन् निन्द्रि॒यꣳ सो॑मपी॒थꣳ सो॑मपी॒थ मि॑न्द्रि॒य म॑स्मिन् नस्मिन् निन्द्रि॒यꣳ सो॑मपी॒थम् । \newline
20. इ॒न्द्रि॒यꣳ सो॑मपी॒थꣳ सो॑मपी॒थ मि॑न्द्रि॒य मि॑न्द्रि॒यꣳ सो॑मपी॒थम् ध॑त्तो धत्तः सोमपी॒थ मि॑न्द्रि॒य मि॑न्द्रि॒यꣳ सो॑मपी॒थम् ध॑त्तः । \newline
21. सो॒म॒पी॒थम् ध॑त्तो धत्तः सोमपी॒थꣳ सो॑मपी॒थम् ध॑त्तो॒ न न ध॑त्तः सोमपी॒थꣳ सो॑मपी॒थम् ध॑त्तो॒ न । \newline
22. सो॒म॒पी॒थमिति॑ सोम - पी॒थम् । \newline
23. ध॒त्तो॒ न न ध॑त्तो धत्तो॒ ने न्द्रि॒येणे᳚ न्द्रि॒येण॒ न ध॑त्तो धत्तो॒ ने न्द्रि॒येण॑ । \newline
24. ने न्द्रि॒येणे᳚ न्द्रि॒येण॒ न ने न्द्रि॒येण॑ सोमपी॒थेन॑ सोमपी॒थेने᳚ न्द्रि॒येण॒ न ने न्द्रि॒येण॑ सोमपी॒थेन॑ । \newline
25. इ॒न्द्रि॒येण॑ सोमपी॒थेन॑ सोमपी॒थेने᳚ न्द्रि॒येणे᳚ न्द्रि॒येण॑ सोमपी॒थेन॒ वि वि सो॑मपी॒थेने᳚ न्द्रि॒येणे᳚ न्द्रि॒येण॑ सोमपी॒थेन॒ वि । \newline
26. सो॒म॒पी॒थेन॒ वि वि सो॑मपी॒थेन॑ सोमपी॒थेन॒ व्यृ॑द्ध्यत ऋद्ध्यते॒ वि सो॑मपी॒थेन॑ सोमपी॒थेन॒ व्यृ॑द्ध्यते । \newline
27. सो॒म॒पी॒थेनेति॑ सोम - पी॒थेन॑ । \newline
28. व्यृ॑द्ध्यत ऋद्ध्यते॒ वि व्यृ॑द्ध्यते॒ यद् यदृ॑द्ध्यते॒ वि व्यृ॑द्ध्यते॒ यत् । \newline
29. ऋ॒द्ध्य॒ते॒ यद् यदृ॑द्ध्यत ऋद्ध्यते॒ यथ् सौ॒म्यः सौ॒म्यो यदृ॑द्ध्यत ऋद्ध्यते॒ यथ् सौ॒म्यः । \newline
30. यथ् सौ॒म्यः सौ॒म्यो यद् यथ् सौ॒म्यो भव॑ति॒ भव॑ति सौ॒म्यो यद् यथ् सौ॒म्यो भव॑ति । \newline
31. सौ॒म्यो भव॑ति॒ भव॑ति सौ॒म्यः सौ॒म्यो भव॑ति सोमपी॒थꣳ सो॑मपी॒थम् भव॑ति सौ॒म्यः सौ॒म्यो भव॑ति सोमपी॒थम् । \newline
32. भव॑ति सोमपी॒थꣳ सो॑मपी॒थम् भव॑ति॒ भव॑ति सोमपी॒थ मे॒वैव सो॑मपी॒थम् भव॑ति॒ भव॑ति सोमपी॒थ मे॒व । \newline
33. सो॒म॒पी॒थ मे॒वैव सो॑मपी॒थꣳ सो॑मपी॒थ मे॒वावावै॒व सो॑मपी॒थꣳ सो॑मपी॒थ मे॒वाव॑ । \newline
34. सो॒म॒पी॒थमिति॑ सोम - पी॒थम् । \newline
35. ए॒वावा वै॒वै वाव॑ रुन्धे रु॒न्धे ऽवै॒वै वाव॑ रुन्धे । \newline
36. अव॑ रुन्धे रु॒न्धे ऽवाव॑ रुन्धे॒ यद् यद् रु॒न्धे ऽवाव॑ रुन्धे॒ यत् । \newline
37. रु॒न्धे॒ यद् यद् रु॑न्धे रुन्धे॒ यदै॒न्द्र ऐ॒न्द्रो यद् रु॑न्धे रुन्धे॒ यदै॒न्द्रः । \newline
38. यदै॒न्द्र ऐ॒न्द्रो यद् यदै॒न्द्रो भव॑ति॒ भव॑ त्यै॒न्द्रो यद् यदै॒न्द्रो भव॑ति । \newline
39. ऐ॒न्द्रो भव॑ति॒ भव॑ त्यै॒न्द्र ऐ॒न्द्रो भव॑तीन्द्रि॒य मि॑न्द्रि॒यम् भव॑त्यै॒न्द्र ऐ॒न्द्रो भव॑तीन्द्रि॒यम् । \newline
40. भव॑तीन्द्रि॒य मि॑न्द्रि॒यम् भव॑ति॒ भव॑तीन्द्रि॒यं ॅवै वा इ॑न्द्रि॒यम् भव॑ति॒ भव॑तीन्द्रि॒यं ॅवै । \newline
41. इ॒न्द्रि॒यं ॅवै वा इ॑न्द्रि॒य मि॑न्द्रि॒यं ॅवै सो॑मपी॒थः सो॑मपी॒थो वा इ॑न्द्रि॒य मि॑न्द्रि॒यं ॅवै सो॑मपी॒थः । \newline
42. वै सो॑मपी॒थः सो॑मपी॒थो वै वै सो॑मपी॒थ इ॑न्द्रि॒य मि॑न्द्रि॒यꣳ सो॑मपी॒थो वै वै सो॑मपी॒थ इ॑न्द्रि॒यम् । \newline
43. सो॒म॒पी॒थ इ॑न्द्रि॒य मि॑न्द्रि॒यꣳ सो॑मपी॒थः सो॑मपी॒थ इ॑न्द्रि॒य मे॒वैवे न्द्रि॒यꣳ सो॑मपी॒थः सो॑मपी॒थ इ॑न्द्रि॒य मे॒व । \newline
44. सो॒म॒पी॒थ इति॑ सोम - पी॒थः । \newline
45. इ॒न्द्रि॒य मे॒वैवे न्द्रि॒य मि॑न्द्रि॒य मे॒व सो॑मपी॒थꣳ सो॑मपी॒थ मे॒वे न्द्रि॒य मि॑न्द्रि॒य मे॒व सो॑मपी॒थम् । \newline
46. ए॒व सो॑मपी॒थꣳ सो॑मपी॒थ मे॒वैव सो॑मपी॒थ मवाव॑ सोमपी॒थ मे॒वैव सो॑मपी॒थ मव॑ । \newline
47. सो॒म॒पी॒थ मवाव॑ सोमपी॒थꣳ सो॑मपी॒थ मव॑ रुन्धे रु॒न्धे ऽव॑ सोमपी॒थꣳ सो॑मपी॒थ मव॑ रुन्धे । \newline
48. सो॒म॒पी॒थमिति॑ सोम - पी॒थम् । \newline
49. अव॑ रुन्धे रु॒न्धे ऽवाव॑ रुन्धे श्यामा॒कः श्या॑मा॒को रु॒न्धे ऽवाव॑ रुन्धे श्यामा॒कः । \newline
50. रु॒न्धे॒ श्या॒मा॒कः श्या॑मा॒को रु॑न्धे रुन्धे श्यामा॒को भ॑वति भवति श्यामा॒को रु॑न्धे रुन्धे श्यामा॒को भ॑वति । \newline
51. श्या॒मा॒को भ॑वति भवति श्यामा॒कः श्या॑मा॒को भ॑व त्ये॒ष ए॒ष भ॑वति श्यामा॒कः श्या॑मा॒को भ॑व त्ये॒षः । \newline
52. भ॒व॒त्ये॒ष ए॒ष भ॑वति भवत्ये॒ष वाव वावैष भ॑वति भवत्ये॒ष वाव । \newline
53. ए॒ष वाव वावैष ए॒ष वाव स स वावैष ए॒ष वाव सः । \newline
54. वाव स स वाव वाव स सोमः॒ सोमः॒ स वाव वाव स सोमः॑ । \newline
55. स सोमः॒ सोमः॒ स स सोमः॑ सा॒क्षाथ् सा॒क्षाथ् सोमः॒ स स सोमः॑ सा॒क्षात् । \newline
56. सोमः॑ सा॒क्षाथ् सा॒क्षाथ् सोमः॒ सोमः॑ सा॒क्षा दे॒वैव सा॒क्षाथ् सोमः॒ सोमः॑ सा॒क्षा दे॒व । \newline
\pagebreak
\markright{ TS 2.3.2.8  \hfill https://www.vedavms.in \hfill}

\section{ TS 2.3.2.8 }

\textbf{TS 2.3.2.8 } \newline
\textbf{Samhita Paata} \newline

सा॒क्षादे॒व सो॑मपी॒थमव॑ रुन्धे॒ ऽग्नये॑ दा॒त्रे पु॑रो॒डाश॑म॒ष्टाक॑पालं॒ निर्व॑पे॒दिन्द्रा॑य प्रदा॒त्रे पु॑रो॒डाश॒मेका॑दशकपालं प॒शुका॑मो॒ऽग्निरे॒वास्मै॑ प॒शून् प्र॑ज॒नय॑ति वृ॒द्धानिन्द्रः॒ प्र य॑च्छति॒ दधि॒ मधु॑ घृ॒तमापो॑ धा॒ना भ॑वन्त्ये॒तद्वै प॑शू॒नाꣳ रू॒पꣳ रू॒पेणै॒व प॒शूनव॑ रुन्धे पञ्च-गृही॒तं भ॑वति॒ पाङ्क्ता॒ हि प॒शवो॑ बहु रू॒पं भ॑वति बहु रू॒पा हि प॒शवः॒-  [  ] \newline

\textbf{Pada Paata} \newline

सा॒क्षादिति॑ स - अ॒क्षात् । ए॒व । सो॒म॒पी॒थमिति॑ सोम - पी॒थम् । अवेति॑ । रु॒न्धे॒ । अ॒ग्नये᳚ । दा॒त्रे । पु॒रो॒डाश᳚म् । अ॒ष्टाक॑पाल॒मित्य॒ष्टा - क॒पा॒ल॒म् । निरिति॑ । व॒पे॒त् । इन्द्रा॑य । प्र॒दा॒त्र इति॑ प्र - दा॒त्रे । पु॒रो॒डाश᳚म् । एका॑दशकपाल॒मित्येका॑दश - क॒पा॒ल॒म् । प॒शुका॑म॒ इति॑ प॒शु - का॒मः॒ । अ॒ग्निः । ए॒व । अ॒स्मै॒ । प॒शून् । प्र॒ज॒नय॒तीति॑ प्र - ज॒नय॑ति । वृ॒द्धान् । इन्द्रः॑ । प्रेति॑ । य॒च्छ॒ति॒ । दधि॑ । मधु॑ । घृ॒तम् । आपः॑ । धा॒नाः । भ॒व॒न्ति॒ । ए॒तत् । वै । प॒शू॒नाम् । रू॒पम् । रू॒पेण॑ । ए॒व । प॒शून् । अवेति॑ । रु॒न्धे॒ । प॒ञ्च॒गृ॒ही॒तमिति॑ पञ्च-गृ॒ही॒तम् । भ॒व॒ति॒ । पाङ्क्ताः᳚ । हि । प॒शवः॑ । ब॒हु॒रू॒पमिति॑ बहु - रू॒पम् । भ॒व॒ति॒ । ब॒हु॒रू॒पा इति॑ बहु - रू॒पाः । हि । प॒शवः॑ ।  \newline


\textbf{Krama Paata} \newline

सा॒क्षादे॒व । सा॒क्षादिति॑ स - अ॒क्षात् । ए॒व सो॑मपी॒थम् । सो॒म॒पी॒थमव॑ । सो॒म॒पी॒थमिति॑ सोम - पी॒थम् । अव॑ रुन्धे । रु॒न्धे॒ ऽग्नये᳚ । अ॒ग्नये॑ दा॒त्रे । दा॒त्रे पु॑रो॒डाश᳚म् । पु॒रो॒डाश॑म॒ष्टाक॑पालम् । अ॒ष्टाक॑पाल॒म् निः । अ॒ष्टाक॑पाल॒मित्य॒ष्टा - क॒पा॒ल॒म् । निर् व॑पेत् । व॒पे॒दिन्द्रा॑य । इन्द्रा॑य प्रदा॒त्रे । प्र॒दा॒त्रे पु॑रो॒डाश᳚म् । प्र॒दा॒त्र इति॑ प्र - दा॒त्रे । पु॒रो॒डाश॒मेका॑दशकपालम् । एका॑दशकपालम् प॒शुका॑मः । एका॑दशकपाल॒मित्येका॑दश - क॒पा॒ल॒म् । प॒शुका॑मो॒ ऽग्निः । प॒शुका॑म॒ इति॑ प॒शु - का॒मः॒ । अ॒ग्निरे॒व । ए॒वास्मै᳚ । अ॒स्मै॒ प॒शून् । प॒शून् प्र॑ज॒नय॑ति । प्र॒ज॒नय॑ति वृ॒द्धान् । प्र॒ज॒नय॒तीति॑ प्र - ज॒नय॑ति । वृ॒द्धानिन्द्रः॑ । इन्द्रः॒ प्र । प्र य॑च्छति । य॒च्छ॒ति॒ दधि॑ । दधि॒ मधु॑ । मधु॑ घृ॒तम् । घृ॒तमापः॑ । आपो॑ धा॒नाः । धा॒ना भ॑वन्ति । भ॒व॒न्त्ये॒तत् । ए॒तद् वै । वै प॑शू॒नाम् । प॒शू॒नाꣳ रू॒पम् । रू॒पꣳ रू॒पेण॑ । रू॒पेणै॒व । ए॒व प॒शून् । प॒शूनव॑ । अव॑ रुन्धे । रु॒न्धे॒ प॒ञ्च॒गृ॒ही॒तम् । प॒ञ्च॒गृ॒ही॒तम् भ॑वति । प॒ञ्च॒गृ॒ही॒तमिति॑ पञ्च - गृ॒ही॒तम् । भ॒व॒ति॒ पाङ्क्ताः᳚ । पाङ्क्ता॒ हि । हि प॒शवः॑ । प॒शवो॑ बहुरू॒पम् । ब॒हु॒रू॒पम् भ॑वति । ब॒हु॒रू॒पमिति॑ बहु - रू॒पम् । भ॒व॒ति॒ ब॒हु॒रू॒पाः । ब॒हु॒रू॒पा हि । ब॒हु॒रू॒पा इति॑ बहु - रू॒पाः । हि प॒शवः॑ ( ) । प॒शवः॒ समृ॑द्ध्यै \newline

\textbf{Jatai Paata} \newline

1. सा॒क्षा दे॒वैव सा॒क्षाथ् सा॒क्षा दे॒व । \newline
2. सा॒क्षादिति॑ स - अ॒क्षात् । \newline
3. ए॒व सो॑मपी॒थꣳ सो॑मपी॒थ मे॒वैव सो॑मपी॒थम् । \newline
4. सो॒म॒पी॒थ मवाव॑ सोमपी॒थꣳ सो॑मपी॒थ मव॑ । \newline
5. सो॒म॒पी॒थमिति॑ सोम - पी॒थम् । \newline
6. अव॑ रुन्धे रु॒न्धे ऽवाव॑ रुन्धे । \newline
7. रु॒न्धे॒ ऽग्नये॒ ऽग्नये॑ रुन्धे रुन्धे॒ ऽग्नये᳚ । \newline
8. अ॒ग्नये॑ दा॒त्रे दा॒त्रे᳚ ऽग्नये॒ ऽग्नये॑ दा॒त्रे । \newline
9. दा॒त्रे पु॑रो॒डाश॑म् पुरो॒डाश॑म् दा॒त्रे दा॒त्रे पु॑रो॒डाश᳚म् । \newline
10. पु॒रो॒डाश॑ म॒ष्टाक॑पाल म॒ष्टाक॑पालम् पुरो॒डाश॑म् पुरो॒डाश॑ म॒ष्टाक॑पालम् । \newline
11. अ॒ष्टाक॑पाल॒म् निर् णि र॒ष्टाक॑पाल म॒ष्टाक॑पाल॒म् निः । \newline
12. अ॒ष्टाक॑पाल॒मित्य॒ष्टा - क॒पा॒ल॒म् । \newline
13. निर् व॑पेद् वपे॒न् निर् णिर् व॑पेत् । \newline
14. व॒पे॒ दिन्द्रा॒ये न्द्रा॑य वपेद् वपे॒ दिन्द्रा॑य । \newline
15. इन्द्रा॑य प्रदा॒त्रे प्र॑दा॒त्र इन्द्रा॒ये न्द्रा॑य प्रदा॒त्रे । \newline
16. प्र॒दा॒त्रे पु॑रो॒डाश॑म् पुरो॒डाश॑म् प्रदा॒त्रे प्र॑दा॒त्रे पु॑रो॒डाश᳚म् । \newline
17. प्र॒दा॒त्र इति॑ प्र - दा॒त्रे । \newline
18. पु॒रो॒डाश॒ मेका॑दशकपाल॒ मेका॑दशकपालम् पुरो॒डाश॑म् पुरो॒डाश॒ मेका॑दशकपालम् । \newline
19. एका॑दशकपालम् प॒शुका॑मः प॒शुका॑म॒ एका॑दशकपाल॒ मेका॑दशकपालम् प॒शुका॑मः । \newline
20. एका॑दशकपाल॒मित्येका॑दश - क॒पा॒ल॒म् । \newline
21. प॒शुका॑मो॒ ऽग्नि र॒ग्निः प॒शुका॑मः प॒शुका॑मो॒ ऽग्निः । \newline
22. प॒शुका॑म॒ इति॑ प॒शु - का॒मः॒ । \newline
23. अ॒ग्नि रे॒वैवाग्नि र॒ग्नि रे॒व । \newline
24. ए॒वास्मा॑ अस्मा ए॒वैवास्मै᳚ । \newline
25. अ॒स्मै॒ प॒शून् प॒शू न॑स्मा अस्मै प॒शून् । \newline
26. प॒शून् प्र॑ज॒नय॑ति प्रज॒नय॑ति प॒शून् प॒शून् प्र॑ज॒नय॑ति । \newline
27. प्र॒ज॒नय॑ति वृ॒द्धान् वृ॒द्धान् प्र॑ज॒नय॑ति प्रज॒नय॑ति वृ॒द्धान् । \newline
28. प्र॒ज॒नय॒तीति॑ प्र - ज॒नय॑ति । \newline
29. वृ॒द्धा निन्द्र॒ इन्द्रो॑ वृ॒द्धान् वृ॒द्धा निन्द्रः॑ । \newline
30. इन्द्रः॒ प्र प्रे न्द्र॒ इन्द्रः॒ प्र । \newline
31. प्र य॑च्छति यच्छति॒ प्र प्र य॑च्छति । \newline
32. य॒च्छ॒ति॒ दधि॒ दधि॑ यच्छति यच्छति॒ दधि॑ । \newline
33. दधि॒ मधु॒ मधु॒ दधि॒ दधि॒ मधु॑ । \newline
34. मधु॑ घृ॒तम् घृ॒तम् मधु॒ मधु॑ घृ॒तम् । \newline
35. घृ॒त माप॒ आपो॑ घृ॒तम् घृ॒त मापः॑ । \newline
36. आपो॑ धा॒ना धा॒ना आप॒ आपो॑ धा॒नाः । \newline
37. धा॒ना भ॑वन्ति भवन्ति धा॒ना धा॒ना भ॑वन्ति । \newline
38. भ॒व॒न् त्ये॒त दे॒तद् भ॑वन्ति भव न्त्ये॒तत् । \newline
39. ए॒तद् वै वा ए॒त दे॒तद् वै । \newline
40. वै प॑शू॒नाम् प॑शू॒नां ॅवै वै प॑शू॒नाम् । \newline
41. प॒शू॒नाꣳ रू॒पꣳ रू॒पम् प॑शू॒नाम् प॑शू॒नाꣳ रू॒पम् । \newline
42. रू॒पꣳ रू॒पेण॑ रू॒पेण॑ रू॒पꣳ रू॒पꣳ रू॒पेण॑ । \newline
43. रू॒पेणै॒वैव रू॒पेण॑ रू॒पेणै॒व । \newline
44. ए॒व प॒शून् प॒शू ने॒वैव प॒शून् । \newline
45. प॒शू नवाव॑ प॒शून् प॒शू नव॑ । \newline
46. अव॑ रुन्धे रु॒न्धे ऽवाव॑ रुन्धे । \newline
47. रु॒न्धे॒ प॒ञ्च॒गृ॒ही॒तम् प॑ञ्चगृही॒तꣳ रु॑न्धे रुन्धे पञ्चगृही॒तम् । \newline
48. प॒ञ्च॒गृ॒ही॒तम् भ॑वति भवति पञ्चगृही॒तम् प॑ञ्चगृही॒तम् भ॑वति । \newline
49. प॒ञ्च॒गृ॒ही॒तमिति॑ पञ्च - गृ॒ही॒तम् । \newline
50. भ॒व॒ति॒ पाङ्क्ताः॒ पाङ्क्ता॑ भवति भवति॒ पाङ्क्ताः᳚ । \newline
51. पाङ्क्ता॒ हि हि पाङ्क्ताः॒ पाङ्क्ता॒ हि । \newline
52. हि प॒शवः॑ प॒शवो॒ हि हि प॒शवः॑ । \newline
53. प॒शवो॑ बहुरू॒पम् ब॑हुरू॒पम् प॒शवः॑ प॒शवो॑ बहुरू॒पम् । \newline
54. ब॒हु॒रू॒पम् भ॑वति भवति बहुरू॒पम् ब॑हुरू॒पम् भ॑वति । \newline
55. ब॒हु॒रू॒पमिति॑ बहु - रू॒पम् । \newline
56. भ॒व॒ति॒ ब॒हु॒रू॒पा ब॑हुरू॒पा भ॑वति भवति बहुरू॒पाः । \newline
57. ब॒हु॒रू॒पा हि हि ब॑हुरू॒पा ब॑हुरू॒पा हि । \newline
58. ब॒हु॒रू॒पा इति॑ बहु - रू॒पाः । \newline
59. हि प॒शवः॑ प॒शवो॒ हि हि प॒शवः॑ । \newline
60. प॒शवः॒ समृ॑द्ध्यै॒ समृ॑द्ध्यै प॒शवः॑ प॒शवः॒ समृ॑द्ध्यै । \newline

\textbf{Ghana Paata } \newline

1. सा॒क्षा दे॒वैव सा॒क्षाथ् सा॒क्षा दे॒व सो॑मपी॒थꣳ सो॑मपी॒थ मे॒व सा॒क्षाथ् सा॒क्षा दे॒व सो॑मपी॒थम् । \newline
2. सा॒क्षादिति॑ स - अ॒क्षात् । \newline
3. ए॒व सो॑मपी॒थꣳ सो॑मपी॒थ मे॒वैव सो॑मपी॒थ मवाव॑ सोमपी॒थ मे॒वैव सो॑मपी॒थ मव॑ । \newline
4. सो॒म॒पी॒थ मवाव॑ सोमपी॒थꣳ सो॑मपी॒थ मव॑ रुन्धे रु॒न्धे ऽव॑ सोमपी॒थꣳ सो॑मपी॒थ मव॑ रुन्धे । \newline
5. सो॒म॒पी॒थमिति॑ सोम - पी॒थम् । \newline
6. अव॑ रुन्धे रु॒न्धे ऽवाव॑ रुन्धे॒ ऽग्नये॒ ऽग्नये॑ रु॒न्धे ऽवाव॑ रुन्धे॒ ऽग्नये᳚ । \newline
7. रु॒न्धे॒ ऽग्नये॒ ऽग्नये॑ रुन्धे रुन्धे॒ ऽग्नये॑ दा॒त्रे दा॒त्रे᳚ ऽग्नये॑ रुन्धे रुन्धे॒ ऽग्नये॑ दा॒त्रे । \newline
8. अ॒ग्नये॑ दा॒त्रे दा॒त्रे᳚ ऽग्नये॒ ऽग्नये॑ दा॒त्रे पु॑रो॒डाश॑म् पुरो॒डाश॑म् दा॒त्रे᳚ ऽग्नये॒ ऽग्नये॑ दा॒त्रे पु॑रो॒डाश᳚म् । \newline
9. दा॒त्रे पु॑रो॒डाश॑म् पुरो॒डाश॑म् दा॒त्रे दा॒त्रे पु॑रो॒डाश॑ म॒ष्टाक॑पाल म॒ष्टाक॑पालम् पुरो॒डाश॑म् दा॒त्रे दा॒त्रे पु॑रो॒डाश॑ म॒ष्टाक॑पालम् । \newline
10. पु॒रो॒डाश॑ म॒ष्टाक॑पाल म॒ष्टाक॑पालम् पुरो॒डाश॑म् पुरो॒डाश॑ म॒ष्टाक॑पाल॒म् निर् णिर॒ष्टाक॑पालम् पुरो॒डाश॑म् पुरो॒डाश॑ म॒ष्टाक॑पाल॒म् निः । \newline
11. अ॒ष्टाक॑पाल॒म् निर् णिर॒ष्टाक॑पाल म॒ष्टाक॑पाल॒म् निर् व॑पेद् वपे॒न् निर॒ष्टाक॑पाल म॒ष्टाक॑पाल॒म् निर् व॑पेत् । \newline
12. अ॒ष्टाक॑पाल॒मित्य॒ष्टा - क॒पा॒ल॒म् । \newline
13. निर् व॑पेद् वपे॒न् निर् णिर् व॑पे॒दिन्द्रा॒ये न्द्रा॑य वपे॒न् निर् णिर् व॑पे॒दिन्द्रा॑य । \newline
14. व॒पे॒दिन्द्रा॒ये न्द्रा॑य वपेद् वपे॒दिन्द्रा॑य प्रदा॒त्रे प्र॑दा॒त्र इन्द्रा॑य वपेद् वपे॒दिन्द्रा॑य प्रदा॒त्रे । \newline
15. इन्द्रा॑य प्रदा॒त्रे प्र॑दा॒त्र इन्द्रा॒ये न्द्रा॑य प्रदा॒त्रे पु॑रो॒डाश॑म् पुरो॒डाश॑म् प्रदा॒त्र इन्द्रा॒ये न्द्रा॑य प्रदा॒त्रे पु॑रो॒डाश᳚म् । \newline
16. प्र॒दा॒त्रे पु॑रो॒डाश॑म् पुरो॒डाश॑म् प्रदा॒त्रे प्र॑दा॒त्रे पु॑रो॒डाश॒ मेका॑दशकपाल॒ मेका॑दशकपालम् पुरो॒डाश॑म् प्रदा॒त्रे प्र॑दा॒त्रे पु॑रो॒डाश॒ मेका॑दशकपालम् । \newline
17. प्र॒दा॒त्र इति॑ प्र - दा॒त्रे । \newline
18. पु॒रो॒डाश॒ मेका॑दशकपाल॒ मेका॑दशकपालम् पुरो॒डाश॑म् पुरो॒डाश॒ मेका॑दशकपालम् प॒शुका॑मः प॒शुका॑म॒ एका॑दशकपालम् पुरो॒डाश॑म् पुरो॒डाश॒ मेका॑दशकपालम् प॒शुका॑मः । \newline
19. एका॑दशकपालम् प॒शुका॑मः प॒शुका॑म॒ एका॑दशकपाल॒ मेका॑दशकपालम् प॒शुका॑मो॒ ऽग्नि र॒ग्निः प॒शुका॑म॒ एका॑दशकपाल॒ मेका॑दशकपालम् प॒शुका॑मो॒ ऽग्निः । \newline
20. एका॑दशकपाल॒मित्येका॑दश - क॒पा॒ल॒म् । \newline
21. प॒शुका॑मो॒ ऽग्नि र॒ग्निः प॒शुका॑मः प॒शुका॑मो॒ ऽग्नि रे॒वैवाग्निः प॒शुका॑मः प॒शुका॑मो॒ ऽग्निरे॒व । \newline
22. प॒शुका॑म॒ इति॑ प॒शु - का॒मः॒ । \newline
23. अ॒ग्नि रे॒वैवाग्नि र॒ग्नि रे॒वास्मा॑ अस्मा ए॒वाग्नि र॒ग्नि रे॒वास्मै᳚ । \newline
24. ए॒वास्मा॑ अस्मा ए॒वैवास्मै॑ प॒शून् प॒शू न॑स्मा ए॒वैवास्मै॑ प॒शून् । \newline
25. अ॒स्मै॒ प॒शून् प॒शू न॑स्मा अस्मै प॒शून् प्र॑ज॒नय॑ति प्रज॒नय॑ति प॒शू न॑स्मा अस्मै प॒शून् प्र॑ज॒नय॑ति । \newline
26. प॒शून् प्र॑ज॒नय॑ति प्रज॒नय॑ति प॒शून् प॒शून् प्र॑ज॒नय॑ति वृ॒द्धान् वृ॒द्धान् प्र॑ज॒नय॑ति प॒शून् प॒शून् प्र॑ज॒नय॑ति वृ॒द्धान् । \newline
27. प्र॒ज॒नय॑ति वृ॒द्धान् वृ॒द्धान् प्र॑ज॒नय॑ति प्रज॒नय॑ति वृ॒द्धा निन्द्र॒ इन्द्रो॑ वृ॒द्धान् प्र॑ज॒नय॑ति प्रज॒नय॑ति वृ॒द्धा निन्द्रः॑ । \newline
28. प्र॒ज॒नय॒तीति॑ प्र - ज॒नय॑ति । \newline
29. वृ॒द्धा निन्द्र॒ इन्द्रो॑ वृ॒द्धान् वृ॒द्धा निन्द्रः॒ प्र प्रे न्द्रो॑ वृ॒द्धान् वृ॒द्धा निन्द्रः॒ प्र । \newline
30. इन्द्रः॒ प्र प्रे न्द्र॒ इन्द्रः॒ प्र य॑च्छति यच्छति॒ प्रे न्द्र॒ इन्द्रः॒ प्र य॑च्छति । \newline
31. प्र य॑च्छति यच्छति॒ प्र प्र य॑च्छति॒ दधि॒ दधि॑ यच्छति॒ प्र प्र य॑च्छति॒ दधि॑ । \newline
32. य॒च्छ॒ति॒ दधि॒ दधि॑ यच्छति यच्छति॒ दधि॒ मधु॒ मधु॒ दधि॑ यच्छति यच्छति॒ दधि॒ मधु॑ । \newline
33. दधि॒ मधु॒ मधु॒ दधि॒ दधि॒ मधु॑ घृ॒तम् घृ॒तम् मधु॒ दधि॒ दधि॒ मधु॑ घृ॒तम् । \newline
34. मधु॑ घृ॒तम् घृ॒तम् मधु॒ मधु॑ घृ॒त माप॒ आपो॑ घृ॒तम् मधु॒ मधु॑ घृ॒त मापः॑ । \newline
35. घृ॒त माप॒ आपो॑ घृ॒तम् घृ॒त मापो॑ धा॒ना धा॒ना आपो॑ घृ॒तम् घृ॒त मापो॑ धा॒नाः । \newline
36. आपो॑ धा॒ना धा॒ना आप॒ आपो॑ धा॒ना भ॑वन्ति भवन्ति धा॒ना आप॒ आपो॑ धा॒ना भ॑वन्ति । \newline
37. धा॒ना भ॑वन्ति भवन्ति धा॒ना धा॒ना भ॑व न्त्ये॒त दे॒तद् भ॑वन्ति धा॒ना धा॒ना भ॑व न्त्ये॒तत् । \newline
38. भ॒व॒ न्त्ये॒त दे॒तद् भ॑वन्ति भव न्त्ये॒तद् वै वा ए॒तद् भ॑वन्ति भव न्त्ये॒तद् वै । \newline
39. ए॒तद् वै वा ए॒त दे॒तद् वै प॑शू॒नाम् प॑शू॒नां ॅवा ए॒त दे॒तद् वै प॑शू॒नाम् । \newline
40. वै प॑शू॒नाम् प॑शू॒नां ॅवै वै प॑शू॒नाꣳ रू॒पꣳ रू॒पम् प॑शू॒नां ॅवै वै प॑शू॒नाꣳ रू॒पम् । \newline
41. प॒शू॒नाꣳ रू॒पꣳ रू॒पम् प॑शू॒नाम् प॑शू॒नाꣳ रू॒पꣳ रू॒पेण॑ रू॒पेण॑ रू॒पम् प॑शू॒नाम् प॑शू॒नाꣳ रू॒पꣳ रू॒पेण॑ । \newline
42. रू॒पꣳ रू॒पेण॑ रू॒पेण॑ रू॒पꣳ रू॒पꣳ रू॒पेणै॒वैव रू॒पेण॑ रू॒पꣳ रू॒पꣳ रू॒पेणै॒व । \newline
43. रू॒पेणै॒वैव रू॒पेण॑ रू॒पेणै॒व प॒शून् प॒शू ने॒व रू॒पेण॑ रू॒पेणै॒व प॒शून् । \newline
44. ए॒व प॒शून् प॒शू ने॒वैव प॒शू नवाव॑ प॒शू ने॒वैव प॒शू नव॑ । \newline
45. प॒शू नवाव॑ प॒शून् प॒शू नव॑ रुन्धे रु॒न्धे ऽव॑ प॒शून् प॒शू नव॑ रुन्धे । \newline
46. अव॑ रुन्धे रु॒न्धे ऽवाव॑ रुन्धे पञ्चगृही॒तम् प॑ञ्चगृही॒तꣳ रु॒न्धे ऽवाव॑ रुन्धे पञ्चगृही॒तम् । \newline
47. रु॒न्धे॒ प॒ञ्च॒गृ॒ही॒तम् प॑ञ्चगृही॒तꣳ रु॑न्धे रुन्धे पञ्चगृही॒तम् भ॑वति भवति पञ्चगृही॒तꣳ रु॑न्धे रुन्धे पञ्चगृही॒तम् भ॑वति । \newline
48. प॒ञ्च॒गृ॒ही॒तम् भ॑वति भवति पञ्चगृही॒तम् प॑ञ्चगृही॒तम् भ॑वति॒ पाङ्क्ताः॒ पाङ्क्ता॑ भवति पञ्चगृही॒तम् प॑ञ्चगृही॒तम् भ॑वति॒ पाङ्क्ताः᳚ । \newline
49. प॒ञ्च॒गृ॒ही॒तमिति॑ पञ्च - गृ॒ही॒तम् । \newline
50. भ॒व॒ति॒ पाङ्क्ताः॒ पाङ्क्ता॑ भवति भवति॒ पाङ्क्ता॒ हि हि पाङ्क्ता॑ भवति भवति॒ पाङ्क्ता॒ हि । \newline
51. पाङ्क्ता॒ हि हि पाङ्क्ताः॒ पाङ्क्ता॒ हि प॒शवः॑ प॒शवो॒ हि पाङ्क्ताः॒ पाङ्क्ता॒ हि प॒शवः॑ । \newline
52. हि प॒शवः॑ प॒शवो॒ हि हि प॒शवो॑ बहुरू॒पम् ब॑हुरू॒पम् प॒शवो॒ हि हि प॒शवो॑ बहुरू॒पम् । \newline
53. प॒शवो॑ बहुरू॒पम् ब॑हुरू॒पम् प॒शवः॑ प॒शवो॑ बहुरू॒पम् भ॑वति भवति बहुरू॒पम् प॒शवः॑ प॒शवो॑ बहुरू॒पम् भ॑वति । \newline
54. ब॒हु॒रू॒पम् भ॑वति भवति बहुरू॒पम् ब॑हुरू॒पम् भ॑वति बहुरू॒पा ब॑हुरू॒पा भ॑वति बहुरू॒पम् ब॑हुरू॒पम् भ॑वति बहुरू॒पाः । \newline
55. ब॒हु॒रू॒पमिति॑ बहु - रू॒पम् । \newline
56. भ॒व॒ति॒ ब॒हु॒रू॒पा ब॑हुरू॒पा भ॑वति भवति बहुरू॒पा हि हि ब॑हुरू॒पा भ॑वति भवति बहुरू॒पा हि । \newline
57. ब॒हु॒रू॒पा हि हि ब॑हुरू॒पा ब॑हुरू॒पा हि प॒शवः॑ प॒शवो॒ हि ब॑हुरू॒पा ब॑हुरू॒पा हि प॒शवः॑ । \newline
58. ब॒हु॒रू॒पा इति॑ बहु - रू॒पाः । \newline
59. हि प॒शवः॑ प॒शवो॒ हि हि प॒शवः॒ समृ॑द्ध्यै॒ समृ॑द्ध्यै प॒शवो॒ हि हि प॒शवः॒ समृ॑द्ध्यै । \newline
60. प॒शवः॒ समृ॑द्ध्यै॒ समृ॑द्ध्यै प॒शवः॑ प॒शवः॒ समृ॑द्ध्यै प्राजाप॒त्यम् प्रा॑जाप॒त्यꣳ समृ॑द्ध्यै प॒शवः॑ प॒शवः॒ समृ॑द्ध्यै प्राजाप॒त्यम् । \newline
\pagebreak
\markright{ TS 2.3.2.9  \hfill https://www.vedavms.in \hfill}

\section{ TS 2.3.2.9 }

\textbf{TS 2.3.2.9 } \newline
\textbf{Samhita Paata} \newline

समृ॑द्ध्यै प्राजाप॒त्यं भ॑वति प्राजाप॒त्या वै प॒शवः॑ प्र॒जाप॑तिरे॒वास्मै॑ प॒शून् प्रज॑नयत्या॒त्मा वै पुरु॑षस्य॒ मधु॒ यन्मद्ध्व॒ग्नौ जु॒होत्या॒त्मान॑मे॒व तद्-यज॑मानो॒ऽग्नौ प्रद॑धाति प॒ङ्क्त्यौ॑ याज्यानुवा॒क्ये॑ भवतः॒ पाङ्क्तः॒ पुरु॑षः॒ पाङ्क्ताः᳚ प॒शव॑ आ॒त्मान॑मे॒व मृ॒त्योर्नि॒ष्क्रीय॑प॒शूनव॑ रुन्धे ॥ \newline

\textbf{Pada Paata} \newline

समृ॑द्ध्या॒ इति॒ सं - ऋ॒द्ध्यै॒ । प्रा॒जा॒प॒त्यमिति॑ प्राजा - प॒त्यम् । भ॒व॒ति॒ । प्रा॒जा॒प॒त्या इति॑ प्राजा - प॒त्याः । वै । प॒शवः॑ । प्र॒जाप॑ति॒रिति॑ प्र॒जा-प॒तिः॒ । ए॒व । अ॒स्मै॒ । प॒शून् । प्रेति॑ । ज॒न॒य॒ति॒ । आ॒त्मा । वै । पुरु॑षस्य । मधु॑ । यत् । मधु॑ । अ॒ग्नौ । जु॒होति॑ । आ॒त्मान᳚म् । ए॒व । तत् । यज॑मानः । अ॒ग्नौ । प्रेति॑ । द॒धा॒ति॒ । प॒ङ्क्त्यौ᳚ । या॒ज्या॒नु॒वा॒क्ये॑ इति॑ याज्या - अ॒नु॒वा॒क्ये᳚ । भ॒व॒तः॒ । पाङ्क्तः॑ । पुरु॑षः । पाङ्क्ताः᳚ । प॒शवः॑ । आ॒त्मान᳚म् । ए॒व । मृ॒त्योः । नि॒ष्क्रीयेति॑ निः-क्रीय॑ । प॒शून् । अवेति॑ । रु॒न्धे॒ ॥  \newline


\textbf{Krama Paata} \newline

समृ॑द्ध्यै प्राजाप॒त्यम् । समृ॑द्ध्या॒ इति॒ सं - ऋ॒द्ध्यै॒ । प्रा॒जा॒प॒त्यम् भ॑वति । प्रा॒जा॒प॒त्यमिति॑ प्राजा - प॒त्यम् । भ॒व॒ति॒ प्रा॒जा॒प॒त्याः । प्रा॒जा॒प॒त्या वै । प्रा॒जा॒प॒त्या इति॑ प्राजा - प॒त्याः । वै प॒शवः॑ । प॒शवः॑ प्र॒जाप॑तिः । प्र॒जाप॑तिरे॒व । प्र॒जाप॑ति॒रिति॑ प्र॒जा - प॒तिः॒ । ए॒वास्मै᳚ । अ॒स्मै॒ प॒शून् । प॒शून् प्र । प्र ज॑नयति । ज॒न॒य॒त्या॒त्मा । आ॒त्मा वै । वै पुरु॑षस्य । पुरु॑षस्य॒ मधु॑ । मधु॒ यत् । यन्मधु॑ । मद्ध्व॒ग्नौ । अ॒ग्नौ जु॒होति॑ । जु॒होत्या॒त्मान᳚म् । आ॒त्मान॑मे॒व । ए॒व तत् । तद् यज॑मानः । यज॑मानो॒ ऽग्नौ । अ॒ग्नौ प्र । प्र द॑धाति । द॒धा॒ति॒ प॒ङ्क्त्यौ᳚ । प॒ङ्क्त्यौ॑ याज्यानुवा॒क्ये᳚ । या॒ज्या॒नु॒वा॒क्ये॑ भवतः । या॒ज्या॒नु॒वा॒क्ये॑ इति॑ याज्या - अ॒नु॒वा॒क्ये᳚ । भ॒व॒तः॒ पाङ्क्तः॑ । पाङ्क्तः॒ पुरु॑षः । पुरु॑षः॒ पाङ्क्ताः᳚ । पाङ्क्ताः᳚ प॒शवः॑ । प॒शव॑ आ॒त्मान᳚म् । आ॒त्मान॑मे॒व । ए॒व मृ॒त्योः । मृ॒त्योर् नि॒ष्क्रीय॑ । नि॒ष्क्रीय॑ प॒शून् । नि॒ष्क्रीयेति॑ निः - क्रीय॑ । प॒शूनव॑ । अव॑ रुन्धे । रु॒न्ध॒ इति॑ रुन्धे । \newline

\textbf{Jatai Paata} \newline

1. समृ॑द्ध्यै प्राजाप॒त्यम् प्रा॑जाप॒त्यꣳ समृ॑द्ध्यै॒ समृ॑द्ध्यै प्राजाप॒त्यम् । \newline
2. समृ॑द्ध्या॒ इति॒ सं - ऋ॒द्ध्यै॒ । \newline
3. प्रा॒जा॒प॒त्यम् भ॑वति भवति प्राजाप॒त्यम् प्रा॑जाप॒त्यम् भ॑वति । \newline
4. प्रा॒जा॒प॒त्यमिति॑ प्राजा - प॒त्यम् । \newline
5. भ॒व॒ति॒ प्रा॒जा॒प॒त्याः प्रा॑जाप॒त्या भ॑वति भवति प्राजाप॒त्याः । \newline
6. प्रा॒जा॒प॒त्या वै वै प्रा॑जाप॒त्याः प्रा॑जाप॒त्या वै । \newline
7. प्रा॒जा॒प॒त्या इति॑ प्राजा - प॒त्याः । \newline
8. वै प॒शवः॑ प॒शवो॒ वै वै प॒शवः॑ । \newline
9. प॒शवः॑ प्र॒जाप॑तिः प्र॒जाप॑तिः प॒शवः॑ प॒शवः॑ प्र॒जाप॑तिः । \newline
10. प्र॒जाप॑ति रे॒वैव प्र॒जाप॑तिः प्र॒जाप॑ति रे॒व । \newline
11. प्र॒जाप॑ति॒रिति॑ प्र॒जा - प॒तिः॒ । \newline
12. ए॒वास्मा॑ अस्मा ए॒वैवास्मै᳚ । \newline
13. अ॒स्मै॒ प॒शून् प॒शू न॑स्मा अस्मै प॒शून् । \newline
14. प॒शून् प्र प्र प॒शून् प॒शून् प्र । \newline
15. प्र ज॑नयति जनयति॒ प्र प्र ज॑नयति । \newline
16. ज॒न॒य॒ त्या॒त्मा ऽऽत्मा ज॑नयति जनय त्या॒त्मा । \newline
17. आ॒त्मा वै वा आ॒त्मा ऽऽत्मा वै । \newline
18. वै पुरु॑षस्य॒ पुरु॑षस्य॒ वै वै पुरु॑षस्य । \newline
19. पुरु॑षस्य॒ मधु॒ मधु॒ पुरु॑षस्य॒ पुरु॑षस्य॒ मधु॑ । \newline
20. मधु॒ यद् यन् मधु॒ मधु॒ यत् । \newline
21. यन् मधु॒ मधु॒ यद् यन् मधु॑ । \newline
22. मध्व॒ग्ना व॒ग्नौ मधु॒ मध्व॒ग्नौ । \newline
23. अ॒ग्नौ जु॒होति॑ जु॒हो त्य॒ग्ना व॒ग्नौ जु॒होति॑ । \newline
24. जु॒हो त्या॒त्मान॑ मा॒त्मान॑म् जु॒होति॑ जु॒हो त्या॒त्मान᳚म् । \newline
25. आ॒त्मान॑ मे॒वैवात्मान॑ मा॒त्मान॑ मे॒व । \newline
26. ए॒व तत् तदे॒वैव तत् । \newline
27. तद् यज॑मानो॒ यज॑मान॒ स्तत् तद् यज॑मानः । \newline
28. यज॑मानो॒ ऽग्ना व॒ग्नौ यज॑मानो॒ यज॑मानो॒ ऽग्नौ । \newline
29. अ॒ग्नौ प्र प्राग्ना व॒ग्नौ प्र । \newline
30. प्र द॑धाति दधाति॒ प्र प्र द॑धाति । \newline
31. द॒धा॒ति॒ प॒ङ्क्त्यौ॑ प॒ङ्क्त्यौ॑ दधाति दधाति प॒ङ्क्त्यौ᳚ । \newline
32. प॒ङ्क्त्यौ॑ याज्यानुवा॒क्ये॑ याज्यानुवा॒क्ये॑ प॒ङ्क्त्यौ॑ प॒ङ्क्त्यौ॑ याज्यानुवा॒क्ये᳚ । \newline
33. या॒ज्या॒नु॒वा॒क्ये॑ भवतो भवतो याज्यानुवा॒क्ये॑ याज्यानुवा॒क्ये॑ भवतः । \newline
34. या॒ज्या॒नु॒वा॒क्ये॑ इति॑ याज्या - अ॒नु॒वा॒क्ये᳚ । \newline
35. भ॒व॒तः॒ पाङ्क्तः॒ पाङ्क्तो॑ भवतो भवतः॒ पाङ्क्तः॑ । \newline
36. पाङ्क्तः॒ पुरु॑षः॒ पुरु॑षः॒ पाङ्क्तः॒ पाङ्क्तः॒ पुरु॑षः । \newline
37. पुरु॑षः॒ पाङ्क्ताः॒ पाङ्क्ताः॒ पुरु॑षः॒ पुरु॑षः॒ पाङ्क्ताः᳚ । \newline
38. पाङ्क्ताः᳚ प॒शवः॑ प॒शवः॒ पाङ्क्ताः॒ पाङ्क्ताः᳚ प॒शवः॑ । \newline
39. प॒शव॑ आ॒त्मान॑ मा॒त्मान॑म् प॒शवः॑ प॒शव॑ आ॒त्मान᳚म् । \newline
40. आ॒त्मान॑ मे॒वैवात्मान॑ मा॒त्मान॑ मे॒व । \newline
41. ए॒व मृ॒त्योर् मृ॒त्यो रे॒वैव मृ॒त्योः । \newline
42. मृ॒त्योर् नि॒ष्क्रीय॑ नि॒ष्क्रीय॑ मृ॒त्योर् मृ॒त्योर् नि॒ष्क्रीय॑ । \newline
43. नि॒ष्क्रीय॑ प॒शून् प॒शून् नि॒ष्क्रीय॑ नि॒ष्क्रीय॑ प॒शून् । \newline
44. नि॒ष्क्रीयेति॑ निः - क्रीय॑ । \newline
45. प॒शू नवाव॑ प॒शून् प॒शू नव॑ । \newline
46. अव॑ रुन्धे रु॒न्धे ऽवाव॑ रुन्धे । \newline
47. रु॒न्ध॒ इति॑ रुन्धे । \newline

\textbf{Ghana Paata } \newline

1. समृ॑द्ध्यै प्राजाप॒त्यम् प्रा॑जाप॒त्यꣳ समृ॑द्ध्यै॒ समृ॑द्ध्यै प्राजाप॒त्यम् भ॑वति भवति प्राजाप॒त्यꣳ समृ॑द्ध्यै॒ समृ॑द्ध्यै प्राजाप॒त्यम् भ॑वति । \newline
2. समृ॑द्ध्या॒ इति॒ सं - ऋ॒द्ध्यै॒ । \newline
3. प्रा॒जा॒प॒त्यम् भ॑वति भवति प्राजाप॒त्यम् प्रा॑जाप॒त्यम् भ॑वति प्राजाप॒त्याः प्रा॑जाप॒त्या भ॑वति प्राजाप॒त्यम् प्रा॑जाप॒त्यम् भ॑वति प्राजाप॒त्याः । \newline
4. प्रा॒जा॒प॒त्यमिति॑ प्राजा - प॒त्यम् । \newline
5. भ॒व॒ति॒ प्रा॒जा॒प॒त्याः प्रा॑जाप॒त्या भ॑वति भवति प्राजाप॒त्या वै वै प्रा॑जाप॒त्या भ॑वति भवति प्राजाप॒त्या वै । \newline
6. प्रा॒जा॒प॒त्या वै वै प्रा॑जाप॒त्याः प्रा॑जाप॒त्या वै प॒शवः॑ प॒शवो॒ वै प्रा॑जाप॒त्याः प्रा॑जाप॒त्या वै प॒शवः॑ । \newline
7. प्रा॒जा॒प॒त्या इति॑ प्राजा - प॒त्याः । \newline
8. वै प॒शवः॑ प॒शवो॒ वै वै प॒शवः॑ प्र॒जाप॑तिः प्र॒जाप॑तिः प॒शवो॒ वै वै प॒शवः॑ प्र॒जाप॑तिः । \newline
9. प॒शवः॑ प्र॒जाप॑तिः प्र॒जाप॑तिः प॒शवः॑ प॒शवः॑ प्र॒जाप॑ति रे॒वैव प्र॒जाप॑तिः प॒शवः॑ प॒शवः॑ प्र॒जाप॑तिरे॒व । \newline
10. प्र॒जाप॑ति रे॒वैव प्र॒जाप॑तिः प्र॒जाप॑ति रे॒वास्मा॑ अस्मा ए॒व प्र॒जाप॑तिः प्र॒जाप॑ति रे॒वास्मै᳚ । \newline
11. प्र॒जाप॑ति॒रिति॑ प्र॒जा - प॒तिः॒ । \newline
12. ए॒वास्मा॑ अस्मा ए॒वैवास्मै॑ प॒शून् प॒शू न॑स्मा ए॒वैवास्मै॑ प॒शून् । \newline
13. अ॒स्मै॒ प॒शून् प॒शू न॑स्मा अस्मै प॒शून् प्र प्र प॒शू न॑स्मा अस्मै प॒शून् प्र । \newline
14. प॒शून् प्र प्र प॒शून् प॒शून् प्र ज॑नयति जनयति॒ प्र प॒शून् प॒शून् प्र ज॑नयति । \newline
15. प्र ज॑नयति जनयति॒ प्र प्र ज॑नय त्या॒त्मा ऽऽत्मा ज॑नयति॒ प्र प्र ज॑नय त्या॒त्मा । \newline
16. ज॒न॒य॒ त्या॒त्मा ऽऽत्मा ज॑नयति जनय त्या॒त्मा वै वा आ॒त्मा ज॑नयति जनय त्या॒त्मा वै । \newline
17. आ॒त्मा वै वा आ॒त्मा ऽऽत्मा वै पुरु॑षस्य॒ पुरु॑षस्य॒ वा आ॒त्मा ऽऽत्मा वै पुरु॑षस्य । \newline
18. वै पुरु॑षस्य॒ पुरु॑षस्य॒ वै वै पुरु॑षस्य॒ मधु॒ मधु॒ पुरु॑षस्य॒ वै वै पुरु॑षस्य॒ मधु॑ । \newline
19. पुरु॑षस्य॒ मधु॒ मधु॒ पुरु॑षस्य॒ पुरु॑षस्य॒ मधु॒ यद् यन् मधु॒ पुरु॑षस्य॒ पुरु॑षस्य॒ मधु॒ यत् । \newline
20. मधु॒ यद् यन् मधु॒ मधु॒ यन् मधु॒ मधु॒ यन् मधु॒ मधु॒ यन् मधु॑ । \newline
21. यन् मधु॒ मधु॒ यद् यन् मध्व॒ग्ना व॒ग्नौ मधु॒ यद् यन् मध्व॒ग्नौ । \newline
22. मध्व॒ग्ना व॒ग्नौ मधु॒ मध्व॒ग्नौ जु॒होति॑ जु॒हो त्य॒ग्नौ मधु॒ मध्व॒ग्नौ जु॒होति॑ । \newline
23. अ॒ग्नौ जु॒होति॑ जु॒हो त्य॒ग्ना व॒ग्नौ जु॒हो त्या॒त्मान॑ मा॒त्मान॑म् जु॒होत्य॒ग्ना व॒ग्नौ जु॒हो त्या॒त्मान᳚म् । \newline
24. जु॒हो त्या॒त्मान॑ मा॒त्मान॑म् जु॒होति॑ जु॒हो त्या॒त्मान॑ मे॒वैवात्मान॑म् जु॒होति॑ जु॒हो त्या॒त्मान॑ मे॒व । \newline
25. आ॒त्मान॑ मे॒वैवात्मान॑ मा॒त्मान॑ मे॒व तत् तदे॒वात्मान॑ मा॒त्मान॑ मे॒व तत् । \newline
26. ए॒व तत् तदे॒वैव तद् यज॑मानो॒ यज॑मान॒ स्त दे॒वैव तद् यज॑मानः । \newline
27. तद् यज॑मानो॒ यज॑मान॒ स्तत् तद् यज॑मानो॒ ऽग्ना व॒ग्नौ यज॑मान॒ स्तत् तद् यज॑मानो॒ ऽग्नौ । \newline
28. यज॑मानो॒ ऽग्ना व॒ग्नौ यज॑मानो॒ यज॑मानो॒ ऽग्नौ प्र प्राग्नौ यज॑मानो॒ यज॑मानो॒ ऽग्नौ प्र । \newline
29. अ॒ग्नौ प्र प्राग्ना व॒ग्नौ प्र द॑धाति दधाति॒ प्राग्ना व॒ग्नौ प्र द॑धाति । \newline
30. प्र द॑धाति दधाति॒ प्र प्र द॑धाति प॒ङ्क्त्यौ॑ प॒ङ्क्त्यौ॑ दधाति॒ प्र प्र द॑धाति प॒ङ्क्त्यौ᳚ । \newline
31. द॒धा॒ति॒ प॒ङ्क्त्यौ॑ प॒ङ्क्त्यौ॑ दधाति दधाति प॒ङ्क्त्यौ॑ याज्यानुवा॒क्ये॑ याज्यानुवा॒क्ये॑ प॒ङ्क्त्यौ॑ दधाति दधाति प॒ङ्क्त्यौ॑ याज्यानुवा॒क्ये᳚ । \newline
32. प॒ङ्क्त्यौ॑ याज्यानुवा॒क्ये॑ याज्यानुवा॒क्ये॑ प॒ङ्क्त्यौ॑ प॒ङ्क्त्यौ॑ याज्यानुवा॒क्ये॑ भवतो भवतो याज्यानुवा॒क्ये॑ प॒ङ्क्त्यौ॑ प॒ङ्क्त्यौ॑ याज्यानुवा॒क्ये॑ भवतः । \newline
33. या॒ज्या॒नु॒वा॒क्ये॑ भवतो भवतो याज्यानुवा॒क्ये॑ याज्यानुवा॒क्ये॑ भवतः॒ पाङ्क्तः॒ पाङ्क्तो॑ भवतो याज्यानुवा॒क्ये॑ याज्यानुवा॒क्ये॑ भवतः॒ पाङ्क्तः॑ । \newline
34. या॒ज्या॒नु॒वा॒क्ये॑ इति॑ याज्या - अ॒नु॒वा॒क्ये᳚ । \newline
35. भ॒व॒तः॒ पाङ्क्तः॒ पाङ्क्तो॑ भवतो भवतः॒ पाङ्क्तः॒ पुरु॑षः॒ पुरु॑षः॒ पाङ्क्तो॑ भवतो भवतः॒ पाङ्क्तः॒ पुरु॑षः । \newline
36. पाङ्क्तः॒ पुरु॑षः॒ पुरु॑षः॒ पाङ्क्तः॒ पाङ्क्तः॒ पुरु॑षः॒ पाङ्क्ताः॒ पाङ्क्ताः॒ पुरु॑षः॒ पाङ्क्तः॒ पाङ्क्तः॒ पुरु॑षः॒ पाङ्क्ताः᳚ । \newline
37. पुरु॑षः॒ पाङ्क्ताः॒ पाङ्क्ताः॒ पुरु॑षः॒ पुरु॑षः॒ पाङ्क्ताः᳚ प॒शवः॑ प॒शवः॒ पाङ्क्ताः॒ पुरु॑षः॒ पुरु॑षः॒ पाङ्क्ताः᳚ प॒शवः॑ । \newline
38. पाङ्क्ताः᳚ प॒शवः॑ प॒शवः॒ पाङ्क्ताः॒ पाङ्क्ताः᳚ प॒शव॑ आ॒त्मान॑ मा॒त्मान॑म् प॒शवः॒ पाङ्क्ताः॒ पाङ्क्ताः᳚ प॒शव॑ आ॒त्मान᳚म् । \newline
39. प॒शव॑ आ॒त्मान॑ मा॒त्मान॑म् प॒शवः॑ प॒शव॑ आ॒त्मान॑ मे॒वैवात्मान॑म् प॒शवः॑ प॒शव॑ आ॒त्मान॑ मे॒व । \newline
40. आ॒त्मान॑ मे॒वैवात्मान॑ मा॒त्मान॑ मे॒व मृ॒त्योर् मृ॒त्यो रे॒वात्मान॑ मा॒त्मान॑ मे॒व मृ॒त्योः । \newline
41. ए॒व मृ॒त्योर् मृ॒त्यो रे॒वैव मृ॒त्योर् नि॒ष्क्रीय॑ नि॒ष्क्रीय॑ मृ॒त्यो रे॒वैव मृ॒त्योर् नि॒ष्क्रीय॑ । \newline
42. मृ॒त्योर् नि॒ष्क्रीय॑ नि॒ष्क्रीय॑ मृ॒त्योर् मृ॒त्योर् नि॒ष्क्रीय॑ प॒शून् प॒शून् नि॒ष्क्रीय॑ मृ॒त्योर् मृ॒त्योर् नि॒ष्क्रीय॑ प॒शून् । \newline
43. नि॒ष्क्रीय॑ प॒शून् प॒शून् नि॒ष्क्रीय॑ नि॒ष्क्रीय॑ प॒शू नवाव॑ प॒शून् नि॒ष्क्रीय॑ नि॒ष्क्रीय॑ प॒शू नव॑ । \newline
44. नि॒ष्क्रीयेति॑ निः - क्रीय॑ । \newline
45. प॒शू नवाव॑ प॒शून् प॒शू नव॑ रुन्धे रु॒न्धे ऽव॑ प॒शून् प॒शू नव॑ रुन्धे । \newline
46. अव॑ रुन्धे रु॒न्धे ऽवाव॑ रुन्धे । \newline
47. रु॒न्ध॒ इति॑ रुन्धे । \newline
\pagebreak
\markright{ TS 2.3.3.1  \hfill https://www.vedavms.in \hfill}

\section{ TS 2.3.3.1 }

\textbf{TS 2.3.3.1 } \newline
\textbf{Samhita Paata} \newline

दे॒वा वै स॒त्रमा॑स॒त-र्द्धि॑परिमितं॒ ॅयश॑स्कामा॒स्तेषाꣳ॒॒ सोमꣳ॒॒ राजा॑नं॒ ॅयश॑ आर्च्छ॒थ् स गि॒रिमुदै॒त् तम॒ग्निरनूदै॒त् ताव॒ग्नीषोमौ॒ सम॑भवतां॒ ताविन्द्रो॑ य॒ज्ञ्वि॑भ्र॒ष्टोऽनु॒ परै॒त् ताव॑ब्रवीद्या॒जय॑तं॒ मेति॒ तस्मा॑ ए॒तामिष्टिं॒ निर॑वपतामाग्ने॒य-म॒ष्टाक॑पालमै॒न्द्र-मेका॑दशकपालꣳ सौ॒म्यं च॒रुं तयै॒वास्मि॒न् तेज॑ - [  ] \newline

\textbf{Pada Paata} \newline

दे॒वाः । वै । स॒त्रम् । आ॒स॒त॒ । ऋद्धि॑परिमित॒मित्यृद्धि॑ - प॒रि॒मि॒त॒म् । यश॑स्कामा॒ इति॒ यशः॑ - का॒माः॒ । तेषा᳚म् । सोम᳚म् । राजा॑नम् । यशः॑ । आ॒र्च्छ॒त् । सः । गि॒रिम् । उदिति॑ । ऐ॒त् । तम् । अ॒ग्निः । अनु॑ । उदिति॑ । ऐ॒त् । तौ । अ॒ग्नीषोमा॒वित्य॒ग्नी- सोमौ᳚ । समिति॑ । अ॒भ॒व॒ता॒म् । तौ । इन्द्रः॑ । य॒ज्ञ्वि॑भ्रष्ट॒ इति॑ य॒ज्ञ् - वि॒भ्र॒ष्टः॒ । अनु॑ । परेति॑ । ऐ॒त् । तौ । अ॒ब्र॒वी॒त् । या॒जय॑तम् । मा॒ । इति॑ । तस्मै᳚ । ए॒ताम् । इष्टि᳚म् । निरिति॑ । अ॒व॒प॒ता॒म् । आ॒ग्ने॒यम् । अ॒ष्टाक॑पाल॒मित्य॒ष्टा - क॒पा॒ल॒म् । ऐ॒न्द्रम् । एका॑दशकपाल॒मित्येका॑दश - क॒पा॒ल॒म् । सौ॒म्यम् । च॒रुम् । तया᳚ । ए॒व । अ॒स्मि॒न्न् । तेजः॑ ।  \newline


\textbf{Krama Paata} \newline

दे॒वा वै । वै स॒त्रम् । स॒त्रमा॑सत । आ॒स॒तर्द्धि॑परिमितम् । ऋद्धि॑परिमित॒म् ॅयश॑स्कामाः । ऋद्धि॑परिमित॒मित्यृद्धि॑ - प॒रि॒मि॒त॒म् । यश॑स्कामा॒स्तेषा᳚म् । यश॑स्कामा॒ इति॒ यशः॑ - का॒माः॒ । तेषाꣳ॒॒ सोम᳚म् । सोमꣳ॒॒ राजा॑नम् । राजा॑नं॒ ॅयशः॑ । यश॑ आर्च्छत् । आ॒र्च्छ॒थ् सः । स गि॒रिम् । गि॒रिमुत् । उदै᳚त् । ऐ॒त् तम् । तम॒ग्निः । अ॒ग्निरनु॑ । अनूत् । उदै᳚त् । ऐ॒त् तौ । ता व॒ग्नीषोमौ᳚ । अ॒ग्नीषोमौ॒ सम् । अ॒ग्नीषोमा॒वित्य॒ग्नी - सोमौ᳚ । सम॑भवताम् । अ॒भ॒व॒ता॒म् तौ । ताविन्द्रः॑ । इन्द्रो॑ य॒ज्ञ्वि॑भ्रष्टः । य॒ज्ञ्वि॑भ्र॒ष्टोऽनु॑ । य॒ज्ञ्वि॑भ्रष्ट॒ इति॑ य॒ज्ञ् - वि॒भ्र॒ष्टः॒ । अनु॒ परा᳚ । परै᳚त् । ऐ॒त् तौ । ताव॑ब्रवीत् । अ॒ब्र॒वी॒द् या॒जय॑तम् । या॒जय॑त॒म् मा । मेति॑ । इति॒ तस्मै᳚ । तस्मा॑ ए॒ताम् । ए॒तामिष्टि᳚म् । इष्टि॒म् निः । निर॑वपताम् । अ॒व॒प॒ता॒मा॒ग्ने॒यम् । आ॒ग्ने॒यम॒ष्टाक॑पालम् । अ॒ष्टाक॑पालमै॒न्द्रम् । अ॒ष्टाक॑पाल॒मित्य॒ष्टा - क॒पा॒ल॒म् । ऐ॒न्द्रमेका॑दशकपालम् । एका॑दशकपालꣳ सौ॒म्यम् । एका॑दशकपाल॒मित्येका॑दश - क॒पा॒ल॒म् । सौ॒म्यम् च॒रुम् । च॒रुम् तया᳚ । तयै॒व । ए॒वास्मिन्न्॑ । अ॒स्मि॒न् तेजः॑ । तेज॑ इन्द्रि॒यम् \newline

\textbf{Jatai Paata} \newline

1. दे॒वा वै वै दे॒वा दे॒वा वै । \newline
2. वै स॒त्रꣳ स॒त्रं ॅवै वै स॒त्रम् । \newline
3. स॒त्र मा॑सता सत स॒त्रꣳ स॒त्र मा॑सत । \newline
4. आ॒स॒त र्‌द्धि॑परिमित॒ मृद्धि॑परिमित मासतास॒त र्‌द्धि॑परिमितम् । \newline
5. ऋद्धि॑परिमितं॒ ॅयश॑स्कामा॒ यश॑स्कामा॒ ऋद्धि॑परिमित॒ मृद्धि॑परिमितं॒ ॅयश॑स्कामाः । \newline
6. ऋद्धि॑परिमित॒मित्यृद्धि॑ - प॒रि॒मि॒त॒म् । \newline
7. यश॑स्कामा॒ स्तेषा॒म् तेषां॒ ॅयश॑स्कामा॒ यश॑स्कामा॒ स्तेषा᳚म् । \newline
8. यश॑स्कामा॒ इति॒ यशः॑ - का॒माः॒ । \newline
9. तेषाꣳ॒॒ सोमꣳ॒॒ सोम॒म् तेषा॒म् तेषाꣳ॒॒ सोम᳚म् । \newline
10. सोमꣳ॒॒ राजा॑नꣳ॒॒ राजा॑नꣳ॒॒ सोमꣳ॒॒ सोमꣳ॒॒ राजा॑नम् । \newline
11. राजा॑नं॒ ॅयशो॒ यशो॒ राजा॑नꣳ॒॒ राजा॑नं॒ ॅयशः॑ । \newline
12. यश॑ आर्च्छ दार्च्छ॒द् यशो॒ यश॑ आर्च्छत् । \newline
13. आ॒र्च्छ॒थ् स स आ᳚र्च्छ दार्च्छ॒थ् सः । \newline
14. स गि॒रिम् गि॒रिꣳ स स गि॒रिम् । \newline
15. गि॒रि मुदुद् गि॒रिम् गि॒रि मुत् । \newline
16. उदै॑ दै॒ दुदु दै᳚त् । \newline
17. ऐ॒त् तम् त मै॑दै॒त् तम् । \newline
18. त म॒ग्नि र॒ग्नि स्तम् त म॒ग्निः । \newline
19. अ॒ग्नि रन्वन्व॒ग्नि र॒ग्नि रनु॑ । \newline
20. अनू दु दन्वनूत् । \newline
21. उदै॑ दै॒दुदु दै᳚त् । \newline
22. ऐ॒त् तौ ता वै॑दै॒त् तौ । \newline
23. ता व॒ग्नीषोमा॑ व॒ग्नीषोमौ॒ तौ ता व॒ग्नीषोमौ᳚ । \newline
24. अ॒ग्नीषोमौ॒ सꣳ स म॒ग्नीषोमा॑ व॒ग्नीषोमौ॒ सम् । \newline
25. अ॒ग्नीषोमा॒वित्य॒ग्नी - सोमौ᳚ । \newline
26. स म॑भवता मभवताꣳ॒॒ सꣳ स म॑भवताम् । \newline
27. अ॒भ॒व॒ता॒म् तौ ता व॑भवता मभवता॒म् तौ । \newline
28. ता विन्द्र॒ इन्द्र॒ स्तौ ता विन्द्रः॑ । \newline
29. इन्द्रो॑ य॒ज्ञ्वि॑भ्रष्टो य॒ज्ञ्वि॑भ्रष्ट॒ इन्द्र॒ इन्द्रो॑ य॒ज्ञ्वि॑भ्रष्टः । \newline
30. य॒ज्ञ्वि॑भ्र॒ष्टो ऽन्वनु॑ य॒ज्ञ्वि॑भ्रष्टो य॒ज्ञ्वि॑भ्र॒ष्टो ऽनु॑ । \newline
31. य॒ज्ञ्वि॑भ्रष्ट॒ इति॑ य॒ज्ञ् - वि॒भ्र॒ष्टः॒ । \newline
32. अनु॒ परा॒ परा ऽन्वनु॒ परा᳚ । \newline
33. परै॑दै॒त् परा॒ परै᳚त् । \newline
34. ऐ॒त् तौ ता वै॑दै॒त् तौ । \newline
35. ता व॑ब्रवी दब्रवी॒त् तौ ता व॑ब्रवीत् । \newline
36. अ॒ब्र॒वी॒द् या॒जय॑तं ॅया॒जय॑त मब्रवी दब्रवीद् या॒जय॑तम् । \newline
37. या॒जय॑तम् मा मा या॒जय॑तं ॅया॒जय॑तम् मा । \newline
38. मेतीति॑ मा॒ मेति॑ । \newline
39. इति॒ तस्मै॒ तस्मा॒ इतीति॒ तस्मै᳚ । \newline
40. तस्मा॑ ए॒ता मे॒ताम् तस्मै॒ तस्मा॑ ए॒ताम् । \newline
41. ए॒ता मिष्टि॒ मिष्टि॑ मे॒ता मे॒ता मिष्टि᳚म् । \newline
42. इष्टि॒म् निर् णिरिष्टि॒ मिष्टि॒म् निः । \newline
43. निर॑वपता मवपता॒म् निर् णिर॑वपताम् । \newline
44. अ॒व॒प॒ता॒ मा॒ग्ने॒य मा᳚ग्ने॒य म॑वपता मवपता माग्ने॒यम् । \newline
45. आ॒ग्ने॒य म॒ष्टाक॑पाल म॒ष्टाक॑पाल माग्ने॒य मा᳚ग्ने॒य म॒ष्टाक॑पालम् । \newline
46. अ॒ष्टाक॑पाल मै॒न्द्र मै॒न्द्र म॒ष्टाक॑पाल म॒ष्टाक॑पाल मै॒न्द्रम् । \newline
47. अ॒ष्टाक॑पाल॒मित्य॒ष्टा - क॒पा॒ल॒म् । \newline
48. ऐ॒न्द्र मेका॑दशकपाल॒ मेका॑दशकपाल मै॒न्द्र मै॒न्द्र मेका॑दशकपालम् । \newline
49. एका॑दशकपालꣳ सौ॒म्यꣳ सौ॒म्य मेका॑दशकपाल॒ मेका॑दशकपालꣳ सौ॒म्यम् । \newline
50. एका॑दशकपाल॒मित्येका॑दश - क॒पा॒ल॒म् । \newline
51. सौ॒म्यम् च॒रुम् च॒रुꣳ सौ॒म्यꣳ सौ॒म्यम् च॒रुम् । \newline
52. च॒रुम् तया॒ तया॑ च॒रुम् च॒रुम् तया᳚ । \newline
53. तयै॒वैव तया॒ तयै॒व । \newline
54. ए॒वास्मि॑न् नस्मिन् ने॒वैवास्मिन्न्॑ । \newline
55. अ॒स्मि॒न् तेज॒ स्तेजो᳚ ऽस्मिन् नस्मि॒न् तेजः॑ । \newline
56. तेज॑ इन्द्रि॒य मि॑न्द्रि॒यम् तेज॒ स्तेज॑ इन्द्रि॒यम् । \newline

\textbf{Ghana Paata } \newline

1. दे॒वा वै वै दे॒वा दे॒वा वै स॒त्रꣳ स॒त्रं ॅवै दे॒वा दे॒वा वै स॒त्रम् । \newline
2. वै स॒त्रꣳ स॒त्रं ॅवै वै स॒त्र मा॑सता सत स॒त्रं ॅवै वै स॒त्र मा॑सत । \newline
3. स॒त्र मा॑सता सत स॒त्रꣳ स॒त्र मा॑स॒त र्‌द्धि॑परिमित॒ मृद्धि॑परिमित मासत स॒त्रꣳ स॒त्र मा॑स॒त र्‌द्धि॑परिमितम् । \newline
4. आ॒स॒त र्‌द्धि॑परिमित॒ मृद्धि॑परिमित मासतास॒त र्‌द्धि॑परिमितं॒ ॅयश॑स्कामा॒ यश॑स्कामा॒ ऋद्धि॑परिमित मासतास॒त र्‌द्धि॑परिमितं॒ ॅयश॑स्कामाः । \newline
5. ऋद्धि॑परिमितं॒ ॅयश॑स्कामा॒ यश॑स्कामा॒ ऋद्धि॑परिमित॒ मृद्धि॑परिमितं॒ ॅयश॑स्कामा॒ स्तेषा॒म् तेषां॒ ॅयश॑स्कामा॒ ऋद्धि॑परिमित॒ मृद्धि॑परिमितं॒ ॅयश॑स्कामा॒ स्तेषा᳚म् । \newline
6. ऋद्धि॑परिमित॒मित्यृद्धि॑ - प॒रि॒मि॒त॒म् । \newline
7. यश॑स्कामा॒ स्तेषा॒म् तेषां॒ ॅयश॑स्कामा॒ यश॑स्कामा॒ स्तेषाꣳ॒॒ सोमꣳ॒॒ सोम॒म् तेषां॒ ॅयश॑स्कामा॒ यश॑स्कामा॒ स्तेषाꣳ॒॒ सोम᳚म् । \newline
8. यश॑स्कामा॒ इति॒ यशः॑ - का॒माः॒ । \newline
9. तेषाꣳ॒॒ सोमꣳ॒॒ सोम॒म् तेषा॒म् तेषाꣳ॒॒ सोमꣳ॒॒ राजा॑नꣳ॒॒ राजा॑नꣳ॒॒ सोम॒म् तेषा॒म् तेषाꣳ॒॒ सोमꣳ॒॒ राजा॑नम् । \newline
10. सोमꣳ॒॒ राजा॑नꣳ॒॒ राजा॑नꣳ॒॒ सोमꣳ॒॒ सोमꣳ॒॒ राजा॑नं॒ ॅयशो॒ यशो॒ राजा॑नꣳ॒॒ सोमꣳ॒॒ 
सोमꣳ॒॒ राजा॑नं॒ ॅयशः॑ । \newline
11. राजा॑नं॒ ॅयशो॒ यशो॒ राजा॑नꣳ॒॒ राजा॑नं॒ ॅयश॑ आर्च्छ दार्च्छ॒द् यशो॒ राजा॑नꣳ॒॒ राजा॑नं॒ ॅयश॑ आर्च्छत् । \newline
12. यश॑ आर्च्छ दार्च्छ॒द् यशो॒ यश॑ आर्च्छ॒थ् स स आ᳚र्च्छ॒द् यशो॒ यश॑ आर्च्छ॒थ् सः । \newline
13. आ॒र्च्छ॒थ् स स आ᳚र्च्छ दार्च्छ॒थ् स गि॒रिम् गि॒रिꣳ स आ᳚र्च्छ दार्च्छ॒थ् स गि॒रिम् । \newline
14. स गि॒रिम् गि॒रिꣳ स स गि॒रि मुदुद् गि॒रिꣳ स स गि॒रि मुत् । \newline
15. गि॒रि मुदुद् गि॒रिम् गि॒रि मुदै॑दै॒दुद् गि॒रिम् गि॒रि मुदै᳚त् । \newline
16. उदै॑दै॒दुदुदै॒त् तम् त मै॒दुदुदै॒त् तम् । \newline
17. ऐ॒त् तम् त मै॑दै॒त् त म॒ग्नि र॒ग्नि स्त मै॑दै॒त् त म॒ग्निः । \newline
18. त म॒ग्नि र॒ग्नि स्तम् त म॒ग्नि रन्वन्व॒ग्नि स्तम् त म॒ग्नि रनु॑ । \newline
19. अ॒ग्नि रन्वन्व॒ग्नि र॒ग्नि रनूदु दन्व॒ग्नि र॒ग्नि रनूत् । \newline
20. अनू दु दन्वनूदै॑दै॒ दुदन्वनूदै᳚त् । \newline
21. उदै॑दै॒दुदुदै॒त् तौ ता वै॒दुदुदै॒त् तौ । \newline
22. ऐ॒त् तौ ता वै॑दै॒त् ता व॒ग्नीषोमा॑ व॒ग्नीषोमौ॒ ता वै॑दै॒त् ता व॒ग्नीषोमौ᳚ । \newline
23. ता व॒ग्नीषोमा॑ व॒ग्नीषोमौ॒ तौ ता व॒ग्नीषोमौ॒ सꣳ स म॒ग्नीषोमौ॒ तौ ता व॒ग्नीषोमौ॒ सम् । \newline
24. अ॒ग्नीषोमौ॒ सꣳ स म॒ग्नीषोमा॑ व॒ग्नीषोमौ॒ स म॑भवता मभवताꣳ॒॒ स म॒ग्नीषोमा॑ व॒ग्नीषोमौ॒ स म॑भवताम् । \newline
25. अ॒ग्नीषोमा॒वित्य॒ग्नी - सोमौ᳚ । \newline
26. स म॑भवता मभवताꣳ॒॒ सꣳ स म॑भवता॒म् तौ ता व॑भवताꣳ॒॒ सꣳ स म॑भवता॒म् तौ । \newline
27. अ॒भ॒व॒ता॒म् तौ ता व॑भवता मभवता॒म् ता विन्द्र॒ इन्द्र॒स्ता व॑भवता मभवता॒म् ता विन्द्रः॑ । \newline
28. ता विन्द्र॒ इन्द्र॒ स्तौ ता विन्द्रो॑ य॒ज्ञ्वि॑भ्रष्टो य॒ज्ञ्वि॑भ्रष्ट॒ इन्द्र॒ स्तौ ता विन्द्रो॑ य॒ज्ञ्वि॑भ्रष्टः । \newline
29. इन्द्रो॑ य॒ज्ञ्वि॑भ्रष्टो य॒ज्ञ्वि॑भ्रष्ट॒ इन्द्र॒ इन्द्रो॑ य॒ज्ञ्वि॑भ्र॒ष्टो ऽन्वनु॑ य॒ज्ञ्वि॑भ्रष्ट॒ इन्द्र॒ इन्द्रो॑ य॒ज्ञ्वि॑भ्र॒ष्टो ऽनु॑ । \newline
30. य॒ज्ञ्वि॑भ्र॒ष्टो ऽन्वनु॑ य॒ज्ञ्वि॑भ्रष्टो य॒ज्ञ्वि॑भ्र॒ष्टो ऽनु॒ परा॒ परा ऽनु॑ य॒ज्ञ्वि॑भ्रष्टो य॒ज्ञ्वि॑भ्र॒ष्टो ऽनु॒ परा᳚ । \newline
31. य॒ज्ञ्वि॑भ्रष्ट॒ इति॑ य॒ज्ञ् - वि॒भ्र॒ष्टः॒ । \newline
32. अनु॒ परा॒ परा ऽन्वनु॒ परै॑दै॒त् परा ऽन्वनु॒ परै᳚त् । \newline
33. परै॑दै॒त् परा॒ परै॒त् तौ ता वै॒त् परा॒ परै॒त् तौ । \newline
34. ऐ॒त् तौ ता वै॑दै॒त् ता व॑ब्रवी दब्रवी॒त् ता वै॑दै॒त् ता व॑ब्रवीत् । \newline
35. ता व॑ब्रवी दब्रवी॒त् तौ ता व॑ब्रवीद् या॒जय॑तं ॅया॒जय॑त मब्रवी॒त् तौ ता व॑ब्रवीद् या॒जय॑तम् । \newline
36. अ॒ब्र॒वी॒द् या॒जय॑तं ॅया॒जय॑त मब्रवी दब्रवीद् या॒जय॑तम् मा मा या॒जय॑त मब्रवी दब्रवीद् या॒जय॑तम् मा । \newline
37. या॒जय॑तम् मा मा या॒जय॑तं ॅया॒जय॑त॒म् मेतीति॑ मा या॒जय॑तं ॅया॒जय॑त॒म् मेति॑ । \newline
38. मेतीति॑ मा॒ मेति॒ तस्मै॒ तस्मा॒ इति॑ मा॒ मेति॒ तस्मै᳚ । \newline
39. इति॒ तस्मै॒ तस्मा॒ इतीति॒ तस्मा॑ ए॒ता मे॒ताम् तस्मा॒ इतीति॒ तस्मा॑ ए॒ताम् । \newline
40. तस्मा॑ ए॒ता मे॒ताम् तस्मै॒ तस्मा॑ ए॒ता मिष्टि॒ मिष्टि॑ मे॒ताम् तस्मै॒ तस्मा॑ ए॒ता मिष्टि᳚म् । \newline
41. ए॒ता मिष्टि॒ मिष्टि॑ मे॒ता मे॒ता मिष्टि॒म् निर् णिरिष्टि॑ मे॒ता मे॒ता मिष्टि॒म् निः । \newline
42. इष्टि॒म् निर् णिरिष्टि॒ मिष्टि॒म् निर॑वपता मवपता॒म् निरिष्टि॒ मिष्टि॒म् निर॑वपताम् । \newline
43. निर॑वपता मवपता॒म् निर् णिर॑वपता माग्ने॒य मा᳚ग्ने॒य म॑वपता॒म् निर् णिर॑वपता माग्ने॒यम् । \newline
44. अ॒व॒प॒ता॒ मा॒ग्ने॒य मा᳚ग्ने॒य म॑वपता मवपता माग्ने॒य म॒ष्टाक॑पाल म॒ष्टाक॑पाल माग्ने॒य म॑वपता मवपता माग्ने॒य म॒ष्टाक॑पालम् । \newline
45. आ॒ग्ने॒य म॒ष्टाक॑पाल म॒ष्टाक॑पाल माग्ने॒य मा᳚ग्ने॒य म॒ष्टाक॑पाल मै॒न्द्र मै॒न्द्र म॒ष्टाक॑पाल माग्ने॒य मा᳚ग्ने॒य म॒ष्टाक॑पाल मै॒न्द्रम् । \newline
46. अ॒ष्टाक॑पाल मै॒न्द्र मै॒न्द्र म॒ष्टाक॑पाल म॒ष्टाक॑पाल मै॒न्द्र मेका॑दशकपाल॒ मेका॑दशकपाल मै॒न्द्र म॒ष्टाक॑पाल म॒ष्टाक॑पाल मै॒न्द्र मेका॑दशकपालम् । \newline
47. अ॒ष्टाक॑पाल॒मित्य॒ष्टा - क॒पा॒ल॒म् । \newline
48. ऐ॒न्द्र मेका॑दशकपाल॒ मेका॑दशकपाल मै॒न्द्र मै॒न्द्र मेका॑दशकपालꣳ सौ॒म्यꣳ सौ॒म्य मेका॑दशकपाल मै॒न्द्र मै॒न्द्र मेका॑दशकपालꣳ सौ॒म्यम् । \newline
49. एका॑दशकपालꣳ सौ॒म्यꣳ सौ॒म्य मेका॑दशकपाल॒ मेका॑दशकपालꣳ सौ॒म्यम् च॒रुम् च॒रुꣳ सौ॒म्य मेका॑दशकपाल॒ मेका॑दशकपालꣳ सौ॒म्यम् च॒रुम् । \newline
50. एका॑दशकपाल॒मित्येका॑दश - क॒पा॒ल॒म् । \newline
51. सौ॒म्यम् च॒रुम् च॒रुꣳ सौ॒म्यꣳ सौ॒म्यम् च॒रुम् तया॒ तया॑ च॒रुꣳ सौ॒म्यꣳ सौ॒म्यम् च॒रुम् तया᳚ । \newline
52. च॒रुम् तया॒ तया॑ च॒रुम् च॒रुम् तयै॒वैव तया॑ च॒रुम् च॒रुम् तयै॒व । \newline
53. तयै॒वैव तया॒ तयै॒वास्मि॑न् नस्मिन् ने॒व तया॒ तयै॒वास्मिन्न्॑ । \newline
54. ए॒वास्मि॑न् नस्मिन् ने॒वैवास्मि॒न् तेज॒ स्तेजो᳚ ऽस्मिन् ने॒वैवास्मि॒न् तेजः॑ । \newline
55. अ॒स्मि॒न् तेज॒ स्तेजो᳚ ऽस्मिन् नस्मि॒न् तेज॑ इन्द्रि॒य मि॑न्द्रि॒यम् तेजो᳚ ऽस्मिन् नस्मि॒न् तेज॑ इन्द्रि॒यम् । \newline
56. तेज॑ इन्द्रि॒य मि॑न्द्रि॒यम् तेज॒ स्तेज॑ इन्द्रि॒यम् ब्र॑ह्मवर्च॒सम् ब्र॑ह्मवर्च॒स मि॑न्द्रि॒यम् तेज॒ स्तेज॑ इन्द्रि॒यम् ब्र॑ह्मवर्च॒सम् । \newline
\pagebreak
\markright{ TS 2.3.3.2  \hfill https://www.vedavms.in \hfill}

\section{ TS 2.3.3.2 }

\textbf{TS 2.3.3.2 } \newline
\textbf{Samhita Paata} \newline

इन्द्रि॒यं ब्र॑ह्मवर्च॒सम॑धत्तां॒ ॅयो य॒ज्ञ्वि॑भ्रष्टः॒ स्यात् तस्मा॑ ए॒तामिष्टिं॒ निर्व॑पेदाग्ने॒य-म॒ष्टाक॑पालमै॒न्द्र-मेका॑दशकपालꣳ सौ॒म्यं च॒रुं ॅयदा᳚ग्ने॒यो भव॑ति॒ तेज॑ ए॒वास्मि॒न् तेन॑ दधाति॒ यदै॒न्द्रो भव॑तीन्द्रि॒यमे॒वास्मि॒न् तेन॑ दधाति॒ यथ् सौ॒म्यो ब्र॑ह्मवर्च॒सं तेना᳚ ऽऽग्ने॒यस्य॑ च सौ॒म्यस्य॑ चै॒न्द्रे स॒माश्ले॑षये॒त् तेज॑श्चै॒वास्मि॑न् ब्रह्मवर्च॒सं च॑ स॒मीची॑ - [  ] \newline

\textbf{Pada Paata} \newline

इ॒न्द्रि॒यम् । ब्र॒ह्म॒व॒र्च॒समिति॑ ब्रह्म - व॒र्च॒सम् । अ॒ध॒त्ता॒म् । यः । य॒ज्ञ्वि॑भ्रष्ट॒ इति॑ य॒ज्ञ् - वि॒भ्र॒ष्टः॒ । स्यात् । तस्मै᳚ । ए॒ताम् । इष्टि᳚म् । निरिति॑ । व॒पे॒त् । आ॒ग्ने॒यम् । अ॒ष्टाक॑पाल॒मित्य॒ष्टा - क॒पा॒ल॒म् । ऐ॒न्द्रम् । एका॑दशकपाल॒मित्येका॑दश - क॒पा॒ल॒म् । सौ॒म्यम् । च॒रुम् । यत् । आ॒ग्ने॒यः । भव॑ति । तेजः॑ । ए॒व । अ॒स्मि॒न्न् । तेन॑ । द॒धा॒ति॒ । यत् । ऐ॒न्द्रः । भव॑ति । इ॒न्द्रि॒यम् । ए॒व । अ॒स्मि॒न्न् । तेन॑ । द॒धा॒ति॒ । यत् । सौ॒म्यः । ब्र॒ह्म॒व॒र्च॒समिति॑ ब्रह्म-व॒र्च॒सम् । तेन॑ । आ॒ग्ने॒यस्य॑ । च॒ । सौ॒म्यस्य॑ । च॒ । ऐ॒न्द्रे । स॒माश्ले॑षये॒दिति॑ सं - आश्ले॑षयेत् । तेजः॑ । च॒ । ए॒व । अ॒स्मि॒न्न् । ब्र॒ह्म॒व॒र्च॒समिति॑ ब्रह्म - व॒र्च॒सम् । च॒ । स॒मीची॒ इति॑ ।  \newline


\textbf{Krama Paata} \newline

इ॒न्द्रि॒यम् ब्र॑ह्मवर्च॒सम् । ब्र॒ह्म॒व॒र्च॒सम॑धत्ताम् । ब्र॒ह्म॒व॒र्च॒समिति॑ ब्रह्म - व॒र्च॒सम् । अ॒ध॒त्तां॒ ॅयः । यो य॒ज्ञ्वि॑भ्रष्टः । य॒ज्ञ्वि॑भ्रष्टः॒ स्यात् । य॒ज्ञ्वि॑भ्रष्ट॒ इति॑ य॒ज्ञ् - वि॒भ्र॒ष्टः॒ । स्यात् तस्मै᳚ । तस्मा॑ ए॒ताम् । ए॒तामिष्टि᳚म् । इष्टि॒म् निः । निर् व॑पेत् । व॒पे॒दा॒ग्ने॒यम् । आ॒ग्ने॒यम॒ष्टाक॑पालम् । अ॒ष्टाक॑पालमै॒न्द्रम् । अ॒ष्टाक॑पाल॒मित्य॒ष्टा - क॒पा॒ल॒म् । ऐ॒न्द्रमेका॑दशकपालम् । एका॑दशकपालꣳ सौ॒म्यम् । एका॑दशकपाल॒मित्येका॑दश - क॒पा॒ल॒म् । सौ॒म्यम् च॒रुम् । च॒रुं ॅयत् । यदा᳚ग्ने॒यः । आ॒ग्ने॒यो भव॑ति । भव॑ति॒ तेजः॑ । तेज॑ ए॒व । ए॒वास्मिन्न्॑ । अ॒स्मि॒न् तेन॑ । तेन॑ दधाति । द॒धा॒ति॒ यत् । यद् ऐ॒न्द्रः । ऐ॒न्द्रो भव॑ति । भव॑तीन्द्रि॒यम् । इ॒न्द्रि॒यमे॒व । ए॒वास्मिन्न्॑ । अ॒स्मि॒न् तेन॑ । तेन॑ दधाति । द॒धा॒ति॒ यत् । यथ् सौ॒म्यः । सौ॒म्यो ब्र॑ह्मवर्च॒सम् । ब्र॒ह्म॒व॒र्च॒सम् तेन॑ । ब्र॒ह्म॒व॒र्च॒समिति॑ ब्रह्म - व॒र्च॒सम् । तेना᳚ग्ने॒यस्य॑ । आ॒ग्ने॒यस्य॑ च । च॒ सौ॒म्यस्य॑ । सौ॒म्यस्य॑ च । चै॒न्द्रे । ऐ॒न्द्रे स॒माश्ले॑षयेत् । स॒माश्ले॑षये॒त् तेजः॑ । स॒माश्ले॑षये॒दिति॑ सं - आश्ले॑षयेत् । तेज॑श्च । चै॒व । ए॒वास्मिन्न्॑ । अ॒स्मि॒न् ब्र॒ह्म॒व॒र्च॒सम् । ब्र॒ह्म॒व॒र्च॒सम् च॑ । ब्र॒ह्म॒व॒र्च॒मिति॑ ब्रह्म - व॒र्च॒सम् । च॒ स॒मीची᳚ । स॒मिची॑ दधाति । स॒मीची॒ इति॑ स॒मीची᳚ \newline

\textbf{Jatai Paata} \newline

1. इ॒न्द्रि॒यम् ब्र॑ह्मवर्च॒सम् ब्र॑ह्मवर्च॒स मि॑न्द्रि॒य मि॑न्द्रि॒यम् ब्र॑ह्मवर्च॒सम् । \newline
2. ब्र॒ह्म॒व॒र्च॒स म॑धत्ता मधत्ताम् ब्रह्मवर्च॒सम् ब्र॑ह्मवर्च॒स म॑धत्ताम् । \newline
3. ब्र॒ह्म॒व॒र्च॒समिति॑ ब्रह्म - व॒र्च॒सम् । \newline
4. अ॒ध॒त्तां॒ ॅयो यो॑ ऽधत्ता मधत्तां॒ ॅयः । \newline
5. यो य॒ज्ञ्वि॑भ्रष्टो य॒ज्ञ्वि॑भ्रष्टो॒ यो यो य॒ज्ञ्वि॑भ्रष्टः । \newline
6. य॒ज्ञ्वि॑भ्रष्टः॒ स्याथ् स्याद् य॒ज्ञ्वि॑भ्रष्टो य॒ज्ञ्वि॑भ्रष्टः॒ स्यात् । \newline
7. य॒ज्ञ्वि॑भ्रष्ट॒ इति॑ य॒ज्ञ् - वि॒भ्र॒ष्टः॒ । \newline
8. स्यात् तस्मै॒ तस्मै॒ स्याथ् स्यात् तस्मै᳚ । \newline
9. तस्मा॑ ए॒ता मे॒ताम् तस्मै॒ तस्मा॑ ए॒ताम् । \newline
10. ए॒ता मिष्टि॒ मिष्टि॑ मे॒ता मे॒ता मिष्टि᳚म् । \newline
11. इष्टि॒म् निर् णिरिष्टि॒ मिष्टि॒म् निः । \newline
12. निर् व॑पेद् वपे॒न् निर् णिर् व॑पेत् । \newline
13. व॒पे॒दा॒ग्ने॒य मा᳚ग्ने॒यं ॅव॑पेद् वपेदाग्ने॒यम् । \newline
14. आ॒ग्ने॒य म॒ष्टाक॑पाल म॒ष्टाक॑पाल माग्ने॒य मा᳚ग्ने॒य म॒ष्टाक॑पालम् । \newline
15. अ॒ष्टाक॑पाल मै॒न्द्र मै॒न्द्र म॒ष्टाक॑पाल म॒ष्टाक॑पाल मै॒न्द्रम् । \newline
16. अ॒ष्टाक॑पाल॒मित्य॒ष्टा - क॒पा॒ल॒म् । \newline
17. ऐ॒न्द्र मेका॑दशकपाल॒ मेका॑दशकपाल मै॒न्द्र मै॒न्द्र मेका॑दशकपालम् । \newline
18. एका॑दशकपालꣳ सौ॒म्यꣳ सौ॒म्य मेका॑दशकपाल॒ मेका॑दशकपालꣳ सौ॒म्यम् । \newline
19. एका॑दशकपाल॒मित्येका॑दश - क॒पा॒ल॒म् । \newline
20. सौ॒म्यम् च॒रुम् च॒रुꣳ सौ॒म्यꣳ सौ॒म्यम् च॒रुम् । \newline
21. च॒रुं ॅयद् यच् च॒रुम् च॒रुं ॅयत् । \newline
22. यदा᳚ग्ने॒य आ᳚ग्ने॒यो यद् यदा᳚ग्ने॒यः । \newline
23. आ॒ग्ने॒यो भव॑ति॒ भव॑ त्याग्ने॒य आ᳚ग्ने॒यो भव॑ति । \newline
24. भव॑ति॒ तेज॒ स्तेजो॒ भव॑ति॒ भव॑ति॒ तेजः॑ । \newline
25. तेज॑ ए॒वैव तेज॒ स्तेज॑ ए॒व । \newline
26. ए॒वास्मि॑न् नस्मिन् ने॒वैवास्मिन्न्॑ । \newline
27. अ॒स्मि॒न् तेन॒ तेना᳚स्मिन् नस्मि॒न् तेन॑ । \newline
28. तेन॑ दधाति दधाति॒ तेन॒ तेन॑ दधाति । \newline
29. द॒धा॒ति॒ यद् यद् द॑धाति दधाति॒ यत् । \newline
30. यदै॒न्द्र ऐ॒न्द्रो यद् यदै॒न्द्रः । \newline
31. ऐ॒न्द्रो भव॑ति॒ भव॑ त्यै॒न्द्र ऐ॒न्द्रो भव॑ति । \newline
32. भव॑तीन्द्रि॒य मि॑न्द्रि॒यम् भव॑ति॒ भव॑तीन्द्रि॒यम् । \newline
33. इ॒न्द्रि॒य मे॒वैवे न्द्रि॒य मि॑न्द्रि॒य मे॒व । \newline
34. ए॒वास्मि॑न् नस्मिन् ने॒वैवास्मिन्न्॑ । \newline
35. अ॒स्मि॒न् तेन॒ तेना᳚स्मिन् नस्मि॒न् तेन॑ । \newline
36. तेन॑ दधाति दधाति॒ तेन॒ तेन॑ दधाति । \newline
37. द॒धा॒ति॒ यद् यद् द॑धाति दधाति॒ यत् । \newline
38. यथ् सौ॒म्यः सौ॒म्यो यद् यथ् सौ॒म्यः । \newline
39. सौ॒म्यो ब्र॑ह्मवर्च॒सम् ब्र॑ह्मवर्च॒सꣳ सौ॒म्यः सौ॒म्यो ब्र॑ह्मवर्च॒सम् । \newline
40. ब्र॒ह्म॒व॒र्च॒सम् तेन॒ तेन॑ ब्रह्मवर्च॒सम् ब्र॑ह्मवर्च॒सम् तेन॑ । \newline
41. ब्र॒ह्म॒व॒र्च॒समिति॑ ब्रह्म - व॒र्च॒सम् । \newline
42. तेना᳚ग्ने॒यस्या᳚ ग्ने॒यस्य॒ तेन॒ तेना᳚ग्ने॒यस्य॑ । \newline
43. आ॒ग्ने॒यस्य॑ च चाग्ने॒यस्या᳚ ग्ने॒यस्य॑ च । \newline
44. च॒ सौ॒म्यस्य॑ सौ॒म्यस्य॑ च च सौ॒म्यस्य॑ । \newline
45. सौ॒म्यस्य॑ च च सौ॒म्यस्य॑ सौ॒म्यस्य॑ च । \newline
46. चै॒न्द्र ऐ॒न्द्रे च॑ चै॒न्द्रे । \newline
47. ऐ॒न्द्रे स॒माश्ले॑षयेथ् स॒माश्ले॑षयेदै॒न्द्र ऐ॒न्द्रे स॒माश्ले॑षयेत् । \newline
48. स॒माश्ले॑षये॒त् तेज॒ स्तेजः॑ स॒माश्ले॑षयेथ् स॒माश्ले॑षये॒त् तेजः॑ । \newline
49. स॒माश्ले॑षये॒दिति॑ सं - आश्ले॑षयेत् । \newline
50. तेज॑श्च च॒ तेज॒ स्तेज॑श्च । \newline
51. चै॒वैव च॑ चै॒व । \newline
52. ए॒वास्मि॑न् नस्मिन् ने॒वैवास्मिन्न्॑ । \newline
53. अ॒स्मि॒न् ब्र॒ह्म॒व॒र्च॒सम् ब्र॑ह्मवर्च॒स म॑स्मिन् नस्मिन् ब्रह्मवर्च॒सम् । \newline
54. ब्र॒ह्म॒व॒र्च॒सम् च॑ च ब्रह्मवर्च॒सम् ब्र॑ह्मवर्च॒सम् च॑ । \newline
55. ब्र॒ह्म॒व॒र्च॒समिति॑ ब्रह्म - व॒र्च॒सम् । \newline
56. च॒ स॒मीची॑ स॒मीची॑ च च स॒मीची᳚ । \newline
57. स॒मीची॑ दधाति दधाति स॒मीची॑ स॒मीची॑ दधाति । \newline
58. स॒मीची॒ इति॑ स॒मीची᳚ । \newline

\textbf{Ghana Paata } \newline

1. इ॒न्द्रि॒यम् ब्र॑ह्मवर्च॒सम् ब्र॑ह्मवर्च॒स मि॑न्द्रि॒य मि॑न्द्रि॒यम् ब्र॑ह्मवर्च॒स म॑धत्ता मधत्ताम् ब्रह्मवर्च॒स मि॑न्द्रि॒य मि॑न्द्रि॒यम् ब्र॑ह्मवर्च॒स म॑धत्ताम् । \newline
2. ब्र॒ह्म॒व॒र्च॒स म॑धत्ता मधत्ताम् ब्रह्मवर्च॒सम् ब्र॑ह्मवर्च॒स म॑धत्तां॒ ॅयो यो॑ ऽधत्ताम् ब्रह्मवर्च॒सम् ब्र॑ह्मवर्च॒स म॑धत्तां॒ ॅयः । \newline
3. ब्र॒ह्म॒व॒र्च॒समिति॑ ब्रह्म - व॒र्च॒सम् । \newline
4. अ॒ध॒त्तां॒ ॅयो यो॑ ऽधत्ता मधत्तां॒ ॅयो य॒ज्ञ्वि॑भ्रष्टो य॒ज्ञ्वि॑भ्रष्टो॒ यो॑ ऽधत्ता मधत्तां॒ ॅयो य॒ज्ञ्वि॑भ्रष्टः । \newline
5. यो य॒ज्ञ्वि॑भ्रष्टो य॒ज्ञ्वि॑भ्रष्टो॒ यो यो य॒ज्ञ्वि॑भ्रष्टः॒ स्याथ् स्याद् य॒ज्ञ्वि॑भ्रष्टो॒ यो यो य॒ज्ञ्वि॑भ्रष्टः॒ स्यात् । \newline
6. य॒ज्ञ्वि॑भ्रष्टः॒ स्याथ् स्याद् य॒ज्ञ्वि॑भ्रष्टो य॒ज्ञ्वि॑भ्रष्टः॒ स्यात् तस्मै॒ तस्मै॒ स्याद् य॒ज्ञ्वि॑भ्रष्टो य॒ज्ञ्वि॑भ्रष्टः॒ स्यात् तस्मै᳚ । \newline
7. य॒ज्ञ्वि॑भ्रष्ट॒ इति॑ य॒ज्ञ् - वि॒भ्र॒ष्टः॒ । \newline
8. स्यात् तस्मै॒ तस्मै॒ स्याथ् स्यात् तस्मा॑ ए॒ता मे॒ताम् तस्मै॒ स्याथ् स्यात् तस्मा॑ ए॒ताम् । \newline
9. तस्मा॑ ए॒ता मे॒ताम् तस्मै॒ तस्मा॑ ए॒ता मिष्टि॒ मिष्टि॑ मे॒ताम् तस्मै॒ तस्मा॑ ए॒ता मिष्टि᳚म् । \newline
10. ए॒ता मिष्टि॒ मिष्टि॑ मे॒ता मे॒ता मिष्टि॒म् निर् णिरिष्टि॑ मे॒ता मे॒ता मिष्टि॒म् निः । \newline
11. इष्टि॒म् निर् णिरिष्टि॒ मिष्टि॒म् निर् व॑पेद् वपे॒न् निरिष्टि॒ मिष्टि॒म् निर् व॑पेत् । \newline
12. निर् व॑पेद् वपे॒न् निर् णिर् व॑पे दाग्ने॒य मा᳚ग्ने॒यं ॅव॑पे॒न् निर् णिर् व॑पे दाग्ने॒यम् । \newline
13. व॒पे॒ दा॒ग्ने॒य मा᳚ग्ने॒यं ॅव॑पेद् वपेदाग्ने॒य म॒ष्टाक॑पाल म॒ष्टाक॑पाल माग्ने॒यं ॅव॑पेद् वपे दाग्ने॒य म॒ष्टाक॑पालम् । \newline
14. आ॒ग्ने॒य म॒ष्टाक॑पाल म॒ष्टाक॑पाल माग्ने॒य मा᳚ग्ने॒य म॒ष्टाक॑पाल मै॒न्द्र मै॒न्द्र म॒ष्टाक॑पाल माग्ने॒य मा᳚ग्ने॒य म॒ष्टाक॑पाल मै॒न्द्रम् । \newline
15. अ॒ष्टाक॑पाल मै॒न्द्र मै॒न्द्र म॒ष्टाक॑पाल म॒ष्टाक॑पाल मै॒न्द्र मेका॑दशकपाल॒ मेका॑दशकपाल मै॒न्द्र म॒ष्टाक॑पाल म॒ष्टाक॑पाल मै॒न्द्र मेका॑दशकपालम् । \newline
16. अ॒ष्टाक॑पाल॒मित्य॒ष्टा - क॒पा॒ल॒म् । \newline
17. ऐ॒न्द्र मेका॑दशकपाल॒ मेका॑दशकपाल मै॒न्द्र मै॒न्द्र मेका॑दशकपालꣳ सौ॒म्यꣳ सौ॒म्य मेका॑दशकपाल मै॒न्द्र मै॒न्द्र मेका॑दशकपालꣳ सौ॒म्यम् । \newline
18. एका॑दशकपालꣳ सौ॒म्यꣳ सौ॒म्य मेका॑दशकपाल॒ मेका॑दशकपालꣳ सौ॒म्यम् च॒रुम् च॒रुꣳ सौ॒म्य मेका॑दशकपाल॒ मेका॑दशकपालꣳ सौ॒म्यम् च॒रुम् । \newline
19. एका॑दशकपाल॒मित्येका॑दश - क॒पा॒ल॒म् । \newline
20. सौ॒म्यम् च॒रुम् च॒रुꣳ सौ॒म्यꣳ सौ॒म्यम् च॒रुं ॅयद् यच् च॒रुꣳ सौ॒म्यꣳ सौ॒म्यम् च॒रुं ॅयत् । \newline
21. च॒रुं ॅयद् यच् च॒रुम् च॒रुं ॅयदा᳚ग्ने॒य आ᳚ग्ने॒यो यच् च॒रुम् च॒रुं ॅयदा᳚ग्ने॒यः । \newline
22. यदा᳚ग्ने॒य आ᳚ग्ने॒यो यद् यदा᳚ग्ने॒यो भव॑ति॒ भव॑ त्याग्ने॒यो यद् यदा᳚ग्ने॒यो भव॑ति । \newline
23. आ॒ग्ने॒यो भव॑ति॒ भव॑ त्याग्ने॒य आ᳚ग्ने॒यो भव॑ति॒ तेज॒स्तेजो॒ भव॑ त्याग्ने॒य आ᳚ग्ने॒यो भव॑ति॒ तेजः॑ । \newline
24. भव॑ति॒ तेज॒ स्तेजो॒ भव॑ति॒ भव॑ति॒ तेज॑ ए॒वैव तेजो॒ भव॑ति॒ भव॑ति॒ तेज॑ ए॒व । \newline
25. तेज॑ ए॒वैव तेज॒ स्तेज॑ ए॒वास्मि॑न् नस्मिन् ने॒व तेज॒ स्तेज॑ ए॒वास्मिन्न्॑ । \newline
26. ए॒वास्मि॑न् नस्मिन् ने॒वैवास्मि॒न् तेन॒ तेना᳚स्मिन् ने॒वैवास्मि॒न् तेन॑ । \newline
27. अ॒स्मि॒न् तेन॒ तेना᳚स्मिन् नस्मि॒न् तेन॑ दधाति दधाति॒ तेना᳚स्मिन् नस्मि॒न् तेन॑ दधाति । \newline
28. तेन॑ दधाति दधाति॒ तेन॒ तेन॑ दधाति॒ यद् यद् द॑धाति॒ तेन॒ तेन॑ दधाति॒ यत् । \newline
29. द॒धा॒ति॒ यद् यद् द॑धाति दधाति॒ यदै॒न्द्र ऐ॒न्द्रो यद् द॑धाति दधाति॒ यदै॒न्द्रः । \newline
30. यदै॒न्द्र ऐ॒न्द्रो यद् यदै॒न्द्रो भव॑ति॒ भव॑ त्यै॒न्द्रो यद् यदै॒न्द्रो भव॑ति । \newline
31. ऐ॒न्द्रो भव॑ति॒ भव॑ त्यै॒न्द्र ऐ॒न्द्रो भव॑तीन्द्रि॒य मि॑न्द्रि॒यम् भव॑ त्यै॒न्द्र ऐ॒न्द्रो भव॑तीन्द्रि॒यम् । \newline
32. भव॑तीन्द्रि॒य मि॑न्द्रि॒यम् भव॑ति॒ भव॑तीन्द्रि॒य मे॒वैवे न्द्रि॒यम् भव॑ति॒ भव॑तीन्द्रि॒य मे॒व । \newline
33. इ॒न्द्रि॒य मे॒वैवे न्द्रि॒य मि॑न्द्रि॒य मे॒वास्मि॑न् नस्मिन् ने॒वे न्द्रि॒य मि॑न्द्रि॒य मे॒वास्मिन्न्॑ । \newline
34. ए॒वास्मि॑न् नस्मिन् ने॒वैवास्मि॒न् तेन॒ तेना᳚स्मिन् ने॒वैवास्मि॒न् तेन॑ । \newline
35. अ॒स्मि॒न् तेन॒ तेना᳚स्मिन् नस्मि॒न् तेन॑ दधाति दधाति॒ तेना᳚स्मिन् नस्मि॒न् तेन॑ दधाति । \newline
36. तेन॑ दधाति दधाति॒ तेन॒ तेन॑ दधाति॒ यद् यद् द॑धाति॒ तेन॒ तेन॑ दधाति॒ यत् । \newline
37. द॒धा॒ति॒ यद् यद् द॑धाति दधाति॒ यथ् सौ॒म्यः सौ॒म्यो यद् द॑धाति दधाति॒ यथ् सौ॒म्यः । \newline
38. यथ् सौ॒म्यः सौ॒म्यो यद् यथ् सौ॒म्यो ब्र॑ह्मवर्च॒सम् ब्र॑ह्मवर्च॒सꣳ सौ॒म्यो यद् यथ् सौ॒म्यो ब्र॑ह्मवर्च॒सम् । \newline
39. सौ॒म्यो ब्र॑ह्मवर्च॒सम् ब्र॑ह्मवर्च॒सꣳ सौ॒म्यः सौ॒म्यो ब्र॑ह्मवर्च॒सम् तेन॒ तेन॑ ब्रह्मवर्च॒सꣳ सौ॒म्यः सौ॒म्यो ब्र॑ह्मवर्च॒सम् तेन॑ । \newline
40. ब्र॒ह्म॒व॒र्च॒सम् तेन॒ तेन॑ ब्रह्मवर्च॒सम् ब्र॑ह्मवर्च॒सम् तेना᳚ ग्ने॒यस्या᳚ ग्ने॒यस्य॒ तेन॑ ब्रह्मवर्च॒सम् ब्र॑ह्मवर्च॒सम् तेना᳚ग्ने॒यस्य॑ । \newline
41. ब्र॒ह्म॒व॒र्च॒समिति॑ ब्रह्म - व॒र्च॒सम् । \newline
42. तेना᳚ग्ने॒यस्या᳚ ग्ने॒यस्य॒ तेन॒ तेना᳚ग्ने॒यस्य॑ च चाग्ने॒यस्य॒ तेन॒ तेना᳚ग्ने॒यस्य॑ च । \newline
43. आ॒ग्ने॒यस्य॑ च चाग्ने॒यस्या᳚ ग्ने॒यस्य॑ च सौ॒म्यस्य॑ सौ॒म्यस्य॑ चाग्ने॒यस्या᳚ ग्ने॒यस्य॑ च सौ॒म्यस्य॑ । \newline
44. च॒ सौ॒म्यस्य॑ सौ॒म्यस्य॑ च च सौ॒म्यस्य॑ च च सौ॒म्यस्य॑ च च सौ॒म्यस्य॑ च । \newline
45. सौ॒म्यस्य॑ च च सौ॒म्यस्य॑ सौ॒म्यस्य॑ चै॒न्द्र ऐ॒न्द्रे च॑ सौ॒म्यस्य॑ सौ॒म्यस्य॑ चै॒न्द्रे । \newline
46. चै॒न्द्र ऐ॒न्द्रे च॑ चै॒न्द्रे स॒माश्ले॑षयेथ् स॒माश्ले॑षये दै॒न्द्रे च॑ चै॒न्द्रे स॒माश्ले॑षयेत् । \newline
47. ऐ॒न्द्रे स॒माश्ले॑षयेथ् स॒माश्ले॑ष येदै॒न्द्र ऐ॒न्द्रे स॒माश्ले॑षये॒त् तेज॒ स्तेजः॑ स॒माश्ले॑षये दै॒न्द्र ऐ॒न्द्रे स॒माश्ले॑षये॒त् तेजः॑ । \newline
48. स॒माश्ले॑षये॒त् तेज॒ स्तेजः॑ स॒माश्ले॑षयेथ् स॒माश्ले॑षये॒त् तेज॑श्च च॒ तेजः॑ स॒माश्ले॑षयेथ् स॒माश्ले॑षये॒त् तेज॑श्च । \newline
49. स॒माश्ले॑षये॒दिति॑ सं - आश्ले॑षयेत् । \newline
50. तेज॑श्च च॒ तेज॒ स्तेज॑ श्चै॒वैव च॒ तेज॒ स्तेज॑ श्चै॒व । \newline
51. चै॒वैव च॑ चै॒वास्मि॑न् नस्मिन् ने॒व च॑ चै॒वास्मिन्न्॑ । \newline
52. ए॒वास्मि॑न् नस्मिन् ने॒वैवास्मि॑न् ब्रह्मवर्च॒सम् ब्र॑ह्मवर्च॒स म॑स्मिन् ने॒वैवास्मि॑न् ब्रह्मवर्च॒सम् । \newline
53. अ॒स्मि॒न् ब्र॒ह्म॒व॒र्च॒सम् ब्र॑ह्मवर्च॒स म॑स्मिन् नस्मिन् ब्रह्मवर्च॒सम् च॑ च ब्रह्मवर्च॒स म॑स्मिन् नस्मिन् ब्रह्मवर्च॒सम् च॑ । \newline
54. ब्र॒ह्म॒व॒र्च॒सम् च॑ च ब्रह्मवर्च॒सम् ब्र॑ह्मवर्च॒सम् च॑ स॒मीची॑ स॒मीची॑ च ब्रह्मवर्च॒सम् ब्र॑ह्मवर्च॒सम् च॑ स॒मीची᳚ । \newline
55. ब्र॒ह्म॒व॒र्च॒समिति॑ ब्रह्म - व॒र्च॒सम् । \newline
56. च॒ स॒मीची॑ स॒मीची॑ च च स॒मीची॑ दधाति दधाति स॒मीची॑ च च स॒मीची॑ दधाति । \newline
57. स॒मीची॑ दधाति दधाति स॒मीची॑ स॒मीची॑ दधा त्यग्नीषो॒मीय॑ मग्नीषो॒मीय॑म् दधाति स॒मीची॑ स॒मीची॑ दधा त्यग्नीषो॒मीय᳚म् । \newline
58. स॒मीची॒ इति॑ स॒मीची᳚ । \newline
\pagebreak
\markright{ TS 2.3.3.3  \hfill https://www.vedavms.in \hfill}

\section{ TS 2.3.3.3 }

\textbf{TS 2.3.3.3 } \newline
\textbf{Samhita Paata} \newline

दधात्यग्नीषो॒मीय॒-मेका॑दशकपालं॒ निर्व॑पे॒द्यं कामो॒ नोप॒नमे॑दाग्ने॒यो वै ब्रा᳚ह्म॒णः स सोमं॑ पिबति॒ स्वामे॒व दे॒वताꣳ॒॒ स्वेन॑ भाग॒धेये॒नोप॑ धावति॒ सैवैनं॒ कामे॑न॒ सम॑र्द्धय॒त्युपै॑नं॒ कामो॑ नमत्यग्नीषो॒मीय॑-म॒ष्टाक॑पालं॒ निर्व॑पेद्-ब्रह्मवर्च॒सका॑मो॒-ऽग्नीषोमा॑ वे॒व स्वेन॑ भाग॒धेये॒नोप॑ धावति॒ तावे॒वास्मि॑न् ब्रह्मवर्च॒सं ध॑त्तो ब्रह्मवर्च॒स्ये॑व- [  ] \newline

\textbf{Pada Paata} \newline

द॒धा॒ति॒ । अ॒ग्नी॒षो॒मीय॒मित्य॑ग्नी - सो॒मीय᳚म् । एका॑दशकपाल॒मित्येका॑दश - क॒पा॒ल॒म् । निरिति॑ । व॒पे॒त् । यम् । कामः॑ । न । उ॒प॒नमे॒दित्यु॑प - नमे᳚त् । आ॒ग्ने॒यः । वै । ब्रा॒ह्म॒णः । सः । सोम᳚म् । पि॒ब॒ति॒ । स्वाम् । ए॒व । दे॒वता᳚म् । स्वेन॑ । भा॒ग॒धेये॒नेति॑ भाग - धेये॑न । उपेति॑ । धा॒व॒ति॒ । सा । ए॒व । ए॒न॒म् । कामे॑न । समिति॑ । अ॒र्द्ध॒य॒ति॒ । उपेति॑ । ए॒न॒म् । कामः॑ । न॒म॒ति॒ । अ॒ग्नी॒षो॒मीय॒मित्य॑ग्नी - सो॒मीय᳚म् । अ॒ष्टाक॑पाल॒मित्य॒ष्टा - क॒पा॒ल॒म् । निरिति॑ । व॒पे॒त् । ब्र॒ह्म॒व॒र्च॒सका॑म॒ इति॑ ब्रह्मवर्च॒स - का॒मः॒ । अ॒ग्नीषोमा॒वित्य॒ग्नी - सोमौ᳚ । ए॒व । स्वेन॑ । भा॒ग॒धेये॒नेति॑ भाग-धेये॑न । उपेति॑ । धा॒व॒ति॒ । तौ । ए॒व । अ॒स्मि॒न्न् । ब्र॒ह्म॒व॒र्च॒समिति॑ ब्रह्म - व॒र्च॒सम् । ध॒त्तः॒ । ब्र॒ह्म॒व॒र्च॒सीति॑ ब्रह्म - व॒र्च॒सी । ए॒व । 17 (50)  \newline


\textbf{Krama Paata} \newline

द॒धा॒त्य॒ग्नी॒षो॒मीय᳚म् । अ॒ग्नी॒षो॒मीय॒मेका॑दशकपालम् । अ॒ग्नी॒षो॒मीय॒मित्य॑ग्नी - सो॒मीय᳚म् । एका॑दशकपाल॒म् निः । एका॑दशकपाल॒मित्येका॑दश - क॒पा॒ल॒म् । निर् व॑पेत् । व॒पे॒द् यम् । यम् कामः॑ । कामो॒ न । नोप॒नमे᳚त् । उ॒प॒नमे॑दाग्ने॒यः । उ॒प॒नमे॒दित्यु॑प - नमे᳚त् । आ॒ग्ने॒यो वै । वै ब्रा᳚ह्म॒णः । ब्रा॒ह्म॒णः सः । स सोम᳚म् । सोम॑म् पिबति । पि॒ब॒ति॒ स्वाम् । स्वामे॒व । ए॒व दे॒वता᳚म् । दे॒वताꣳ॒॒ स्वेन॑ । स्वेन॑ भाग॒धेये॑न । भा॒ग॒धेये॒नोप॑ । भा॒ग॒धेये॒नेति॑ भाग - धेये॑न । उप॑ धावति । धा॒व॒ति॒ सा । सैव । ए॒वैन᳚म् । ए॒न॒म् कामे॑न । कामे॑न॒ सम् । सम॑र्द्धयति । अ॒र्द्ध॒य॒त्युप॑ । उपै॑नम् । ए॒न॒म् कामः॑ । कामो॑ नमति । न॒म॒त्य॒ग्नी॒षो॒मीय᳚म् । अ॒ग्नी॒षो॒मीय॑म॒ष्टाक॑पालम् । अ॒ग्नी॒षो॒मीय॒मित्य॑ग्नी - सो॒मीय᳚म् । अ॒ष्टाक॑पाल॒म् निः । अ॒ष्टाक॑पाल॒मित्य॒ष्टा - क॒पा॒ल॒म् । निर् व॑पेत् । व॒पे॒द् ब्र॒ह्म॒व॒र्च॒सका॑मः । ब्र॒ह्म॒व॒र्च॒सका॑मो॒ ऽग्नीषोमौ᳚ । ब्र॒ह्म॒व॒र्च॒सका॑म॒ इति॑ ब्रह्मवर्च॒स - का॒मः॒ । अ॒ग्नीषोमा॑वे॒व । अ॒ग्नीषोमा॒वित्य॒ग्नी - सोमौ᳚ । ए॒व स्वेन॑ । स्वेन॑ भाग॒धेये॑न । भा॒ग॒धेये॒नोप॑ । भा॒ग॒धेये॒नेति॑ भाग - धेये॑न । उप॑ धावति । धा॒व॒ति॒ तौ । तावे॒व । ए॒वास्मिन्न्॑ । अ॒स्मि॒न् ब्र॒ह्म॒व॒र्च॒सम् । ब्र॒ह्म॒व॒र्च॒सम् ध॑त्तः । ब्र॒ह्म॒व॒र्च॒समिति॑ ब्रह्म - व॒र्च॒सम् । ध॒त्तो॒ ब्र॒ह्म॒व॒र्च॒सी । ब्र॒ह्म॒व॒र्च॒स्ये॑व । ब्र॒ह्म॒व॒र्च॒सीति॑ ब्रह्म - व॒र्च॒सी । ए॒व भ॑वति \newline

\textbf{Jatai Paata} \newline

1. द॒धा॒ त्य॒ग्नी॒षो॒मीय॑ मग्नीषो॒मीय॑म् दधाति दधा त्यग्नीषो॒मीय᳚म् । \newline
2. अ॒ग्नी॒षो॒मीय॒ मेका॑दशकपाल॒ मेका॑दशकपाल मग्नीषो॒मीय॑ मग्नीषो॒मीय॒ मेका॑दशकपालम् । \newline
3. अ॒ग्नी॒षो॒मीय॒मित्य॑ग्नी - सो॒मीय᳚म् । \newline
4. एका॑दशकपाल॒म् निर् णिरेका॑दशकपाल॒ मेका॑दशकपाल॒म् निः । \newline
5. एका॑दशकपाल॒मित्येका॑दश - क॒पा॒ल॒म् । \newline
6. निर् व॑पेद् वपे॒न् निर् णिर् व॑पेत् । \newline
7. व॒पे॒द् यं ॅयं ॅव॑पेद् वपे॒द् यम् । \newline
8. यम् कामः॒ कामो॒ यं ॅयम् कामः॑ । \newline
9. कामो॒ न न कामः॒ कामो॒ न । \newline
10. नोप॒नमे॑ दुप॒नमे॒न् न नोप॒नमे᳚त् । \newline
11. उ॒प॒नमे॑ दाग्ने॒य आ᳚ग्ने॒य उ॑प॒नमे॑ दुप॒नमे॑ दाग्ने॒यः । \newline
12. उ॒प॒नमे॒दित्यु॑प - नमे᳚त् । \newline
13. आ॒ग्ने॒यो वै वा आ᳚ग्ने॒य आ᳚ग्ने॒यो वै । \newline
14. वै ब्रा᳚ह्म॒णो ब्रा᳚ह्म॒णो वै वै ब्रा᳚ह्म॒णः । \newline
15. ब्रा॒ह्म॒णः स स ब्रा᳚ह्म॒णो ब्रा᳚ह्म॒णः सः । \newline
16. स सोमꣳ॒॒ सोमꣳ॒॒ स स सोम᳚म् । \newline
17. सोम॑म् पिबति पिबति॒ सोमꣳ॒॒ सोम॑म् पिबति । \newline
18. पि॒ब॒ति॒ स्वाꣳ स्वाम् पि॑बति पिबति॒ स्वाम् । \newline
19. स्वा मे॒वैव स्वाꣳ स्वा मे॒व । \newline
20. ए॒व दे॒वता᳚म् दे॒वता॑ मे॒वैव दे॒वता᳚म् । \newline
21. दे॒वताꣳ॒॒ स्वेन॒ स्वेन॑ दे॒वता᳚म् दे॒वताꣳ॒॒ स्वेन॑ । \newline
22. स्वेन॑ भाग॒धेये॑न भाग॒धेये॑न॒ स्वेन॒ स्वेन॑ भाग॒धेये॑न । \newline
23. भा॒ग॒धेये॒नोपोप॑ भाग॒धेये॑न भाग॒धेये॒नोप॑ । \newline
24. भा॒ग॒धेये॒नेति॑ भाग - धेये॑न । \newline
25. उप॑ धावति धाव॒ त्युपोप॑ धावति । \newline
26. धा॒व॒ति॒ सा सा धा॑वति धावति॒ सा । \newline
27. सैवैव सा सैव । \newline
28. ए॒वैन॑ मेन मे॒वैवैन᳚म् । \newline
29. ए॒न॒म् कामे॑न॒ कामे॑नैन मेन॒म् कामे॑न । \newline
30. कामे॑न॒ सꣳ सम् कामे॑न॒ कामे॑न॒ सम् । \newline
31. स म॑र्द्धय त्यर्द्धयति॒ सꣳ स म॑र्द्धयति । \newline
32. अ॒र्द्ध॒य॒ त्युपोपा᳚ र्द्धय त्यर्द्धय॒ त्युप॑ । \newline
33. उपै॑न मेन॒ मुपोपै॑नम् । \newline
34. ए॒न॒म् कामः॒ काम॑ एन मेन॒म् कामः॑ । \newline
35. कामो॑ नमति नमति॒ कामः॒ कामो॑ नमति । \newline
36. न॒म॒ त्य॒ग्नी॒षो॒मीय॑ मग्नीषो॒मीय॑म् नमति नम त्यग्नीषो॒मीय᳚म् । \newline
37. अ॒ग्नी॒षो॒मीय॑ म॒ष्टाक॑पाल म॒ष्टाक॑पाल मग्नीषो॒मीय॑ मग्नीषो॒मीय॑ म॒ष्टाक॑पालम् । \newline
38. अ॒ग्नी॒षो॒मीय॒मित्य॑ग्नी - सो॒मीय᳚म् । \newline
39. अ॒ष्टाक॑पाल॒म् निर् णिर॒ष्टाक॑पाल म॒ष्टाक॑पाल॒म् निः । \newline
40. अ॒ष्टाक॑पाल॒मित्य॒ष्टा - क॒पा॒ल॒म् । \newline
41. निर् व॑पेद् वपे॒न् निर् णिर् व॑पेत् । \newline
42. व॒पे॒द् ब्र॒ह्म॒व॒र्च॒सका॑मो ब्रह्मवर्च॒सका॑मो वपेद् वपेद् ब्रह्मवर्च॒सका॑मः । \newline
43. ब्र॒ह्म॒व॒र्च॒सका॑मो॒ ऽग्नीषोमा॑ व॒ग्नीषोमौ᳚ ब्रह्मवर्च॒सका॑मो ब्रह्मवर्च॒सका॑मो॒ ऽग्नीषोमौ᳚ । \newline
44. ब्र॒ह्म॒व॒र्च॒सका॑म॒ इति॑ ब्रह्मवर्च॒स - का॒मः॒ । \newline
45. अ॒ग्नीषोमा॑ वे॒वै वाग्नीषोमा॑ व॒ग्नीषोमा॑ वे॒व । \newline
46. अ॒ग्नीषोमा॒वित्य॒ग्नी - सोमौ᳚ । \newline
47. ए॒व स्वेन॒ स्वेनै॒वैव स्वेन॑ । \newline
48. स्वेन॑ भाग॒धेये॑न भाग॒धेये॑न॒ स्वेन॒ स्वेन॑ भाग॒धेये॑न । \newline
49. भा॒ग॒धेये॒नोपोप॑ भाग॒धेये॑न भाग॒धेये॒नोप॑ । \newline
50. भा॒ग॒धेये॒नेति॑ भाग - धेये॑न । \newline
51. उप॑ धावति धाव॒ त्युपोप॑ धावति । \newline
52. धा॒व॒ति॒ तौ तौ धा॑वति धावति॒ तौ । \newline
53. ता वे॒वैव तौ ता वे॒व । \newline
54. ए॒वास्मि॑न् नस्मिन् ने॒वैवास्मिन्न्॑ । \newline
55. अ॒स्मि॒न् ब्र॒ह्म॒व॒र्च॒सम् ब्र॑ह्मवर्च॒स म॑स्मिन् नस्मिन् ब्रह्मवर्च॒सम् । \newline
56. ब्र॒ह्म॒व॒र्च॒सम् ध॑त्तो धत्तो ब्रह्मवर्च॒सम् ब्र॑ह्मवर्च॒सम् ध॑त्तः । \newline
57. ब्र॒ह्म॒व॒र्च॒समिति॑ ब्रह्म - व॒र्च॒सम् । \newline
58. ध॒त्तो॒ ब्र॒ह्म॒व॒र्च॒सी ब्र॑ह्मवर्च॒सी ध॑त्तो धत्तो ब्रह्मवर्च॒सी । \newline
59. ब्र॒ह्म॒व॒र्च॒ स्ये॑वैव ब्र॑ह्मवर्च॒सी ब्र॑ह्मवर्च॒ स्ये॑व । \newline
60. ब्र॒ह्म॒व॒र्च॒सीति॑ ब्रह्म - व॒र्च॒सी । \newline
61. ए॒व भ॑वति भव त्ये॒वैव भ॑वति । \newline

\textbf{Ghana Paata } \newline

1. द॒धा॒ त्य॒ग्नी॒षो॒मीय॑ मग्नीषो॒मीय॑म् दधाति दधा त्यग्नीषो॒मीय॒ मेका॑दशकपाल॒ मेका॑दशकपाल मग्नीषो॒मीय॑म् दधाति दधा त्यग्नीषो॒मीय॒ मेका॑दशकपालम् । \newline
2. अ॒ग्नी॒षो॒मीय॒ मेका॑दशकपाल॒ मेका॑दशकपाल मग्नीषो॒मीय॑ मग्नीषो॒मीय॒ मेका॑दशकपाल॒म् निर् णिरेका॑दशकपाल मग्नीषो॒मीय॑ मग्नीषो॒मीय॒ मेका॑दशकपाल॒म् निः । \newline
3. अ॒ग्नी॒षो॒मीय॒मित्य॑ग्नी - सो॒मीय᳚म् । \newline
4. एका॑दशकपाल॒म् निर् णिरेका॑दशकपाल॒ मेका॑दशकपाल॒म् निर् व॑पेद् वपे॒न् निरेका॑दशकपाल॒ मेका॑दशकपाल॒म् निर् व॑पेत् । \newline
5. एका॑दशकपाल॒मित्येका॑दश - क॒पा॒ल॒म् । \newline
6. निर् व॑पेद् वपे॒न् निर् णिर् व॑पे॒द् यं ॅयं ॅव॑पे॒न् निर् णिर् व॑पे॒द् यम् । \newline
7. व॒पे॒द् यं ॅयं ॅव॑पेद् वपे॒द् यम् कामः॒ कामो॒ यं ॅव॑पेद् वपे॒द् यम् कामः॑ । \newline
8. यम् कामः॒ कामो॒ यं ॅयम् कामो॒ न न कामो॒ यं ॅयम् कामो॒ न । \newline
9. कामो॒ न न कामः॒ कामो॒ नोप॒नमे॑ दुप॒नमे॒न् न कामः॒ कामो॒ नोप॒नमे᳚त् । \newline
10. नोप॒नमे॑ दुप॒नमे॒न् न नोप॒नमे॑ दाग्ने॒य आ᳚ग्ने॒य उ॑प॒नमे॒न् न नोप॒नमे॑ दाग्ने॒यः । \newline
11. उ॒प॒नमे॑ दाग्ने॒य आ᳚ग्ने॒य उ॑प॒नमे॑ दुप॒नमे॑ दाग्ने॒यो वै वा आ᳚ग्ने॒य उ॑प॒नमे॑ दुप॒नमे॑ दाग्ने॒यो वै । \newline
12. उ॒प॒नमे॒दित्यु॑प - नमे᳚त् । \newline
13. आ॒ग्ने॒यो वै वा आ᳚ग्ने॒य आ᳚ग्ने॒यो वै ब्रा᳚ह्म॒णो ब्रा᳚ह्म॒णो वा आ᳚ग्ने॒य आ᳚ग्ने॒यो वै ब्रा᳚ह्म॒णः । \newline
14. वै ब्रा᳚ह्म॒णो ब्रा᳚ह्म॒णो वै वै ब्रा᳚ह्म॒णः स स ब्रा᳚ह्म॒णो वै वै ब्रा᳚ह्म॒णः सः । \newline
15. ब्रा॒ह्म॒णः स स ब्रा᳚ह्म॒णो ब्रा᳚ह्म॒णः स सोमꣳ॒॒ सोमꣳ॒॒ स ब्रा᳚ह्म॒णो ब्रा᳚ह्म॒णः स सोम᳚म् । \newline
16. स सोमꣳ॒॒ सोमꣳ॒॒ स स सोम॑म् पिबति पिबति॒ सोमꣳ॒॒ स स सोम॑म् पिबति । \newline
17. सोम॑म् पिबति पिबति॒ सोमꣳ॒॒ सोम॑म् पिबति॒ स्वाꣳ स्वाम् पि॑बति॒ सोमꣳ॒॒ सोम॑म् पिबति॒ स्वाम् । \newline
18. पि॒ब॒ति॒ स्वाꣳ स्वाम् पि॑बति पिबति॒ स्वा मे॒वैव स्वाम् पि॑बति पिबति॒ स्वा मे॒व । \newline
19. स्वा मे॒वैव स्वाꣳ स्वा मे॒व दे॒वता᳚म् दे॒वता॑ मे॒व स्वाꣳ स्वा मे॒व दे॒वता᳚म् । \newline
20. ए॒व दे॒वता᳚म् दे॒वता॑ मे॒वैव दे॒वताꣳ॒॒ स्वेन॒ स्वेन॑ दे॒वता॑ मे॒वैव दे॒वताꣳ॒॒ स्वेन॑ । \newline
21. दे॒वताꣳ॒॒ स्वेन॒ स्वेन॑ दे॒वता᳚म् दे॒वताꣳ॒॒ स्वेन॑ भाग॒धेये॑न भाग॒धेये॑न॒ स्वेन॑ दे॒वता᳚म् दे॒वताꣳ॒॒ स्वेन॑ भाग॒धेये॑न । \newline
22. स्वेन॑ भाग॒धेये॑न भाग॒धेये॑न॒ स्वेन॒ स्वेन॑ भाग॒धेये॒नोपोप॑ भाग॒धेये॑न॒ स्वेन॒ स्वेन॑ भाग॒धेये॒नोप॑ । \newline
23. भा॒ग॒धेये॒नोपोप॑ भाग॒धेये॑न भाग॒धेये॒नोप॑ धावति धाव॒त्युप॑ भाग॒धेये॑न भाग॒धेये॒नोप॑ धावति । \newline
24. भा॒ग॒धेये॒नेति॑ भाग - धेये॑न । \newline
25. उप॑ धावति धाव॒त्युपोप॑ धावति॒ सा सा धा॑व॒त्युपोप॑ धावति॒ सा । \newline
26. धा॒व॒ति॒ सा सा धा॑वति धावति॒ सैवैव सा धा॑वति धावति॒ सैव । \newline
27. सैवैव सा सैवैन॑ मेन मे॒व सा सैवैन᳚म् । \newline
28. ए॒वैन॑ मेन मे॒वैवैन॒म् कामे॑न॒ कामे॑नैन मे॒वैवैन॒म् कामे॑न । \newline
29. ए॒न॒म् कामे॑न॒ कामे॑नैन मेन॒म् कामे॑न॒ सꣳ सम् कामे॑नैन मेन॒म् कामे॑न॒ सम् । \newline
30. कामे॑न॒ सꣳ सम् कामे॑न॒ कामे॑न॒ स म॑र्द्धय त्यर्द्धयति॒ सम् कामे॑न॒ कामे॑न॒ स म॑र्द्धयति । \newline
31. स म॑र्द्धय त्यर्द्धयति॒ सꣳ स म॑र्द्धय॒ त्युपोपा᳚र्द्धयति॒ सꣳ स म॑र्द्धय॒त्युप॑ । \newline
32. अ॒र्द्ध॒य॒ त्युपोपा᳚र्द्धय त्यर्द्धय॒ त्युपै॑न मेन॒ मुपा᳚र्द्धय त्यर्द्धय॒ त्युपै॑नम् । \newline
33. उपै॑न मेन॒ मुपोपै॑न॒म् कामः॒ काम॑ एन॒ मुपोपै॑न॒म् कामः॑ । \newline
34. ए॒न॒म् कामः॒ काम॑ एन मेन॒म् कामो॑ नमति नमति॒ काम॑ एन मेन॒म् कामो॑ नमति । \newline
35. कामो॑ नमति नमति॒ कामः॒ कामो॑ नम त्यग्नीषो॒मीय॑ मग्नीषो॒मीय॑म् नमति॒ कामः॒ कामो॑ नम त्यग्नीषो॒मीय᳚म् । \newline
36. न॒म॒ त्य॒ग्नी॒षो॒मीय॑ मग्नीषो॒मीय॑म् नमति नम त्यग्नीषो॒मीय॑ म॒ष्टाक॑पाल म॒ष्टाक॑पाल मग्नीषो॒मीय॑म् नमति नम त्यग्नीषो॒मीय॑ म॒ष्टाक॑पालम् । \newline
37. अ॒ग्नी॒षो॒मीय॑ म॒ष्टाक॑पाल म॒ष्टाक॑पाल मग्नीषो॒मीय॑ मग्नीषो॒मीय॑ म॒ष्टाक॑पाल॒म् निर् णिर॒ष्टाक॑पाल मग्नीषो॒मीय॑ मग्नीषो॒मीय॑ म॒ष्टाक॑पाल॒म् निः । \newline
38. अ॒ग्नी॒षो॒मीय॒मित्य॑ग्नी - सो॒मीय᳚म् । \newline
39. अ॒ष्टाक॑पाल॒म् निर् णिर॒ष्टाक॑पाल म॒ष्टाक॑पाल॒म् निर् व॑पेद् वपे॒न् निर॒ष्टाक॑पाल म॒ष्टाक॑पाल॒म् निर् व॑पेत् । \newline
40. अ॒ष्टाक॑पाल॒मित्य॒ष्टा - क॒पा॒ल॒म् । \newline
41. निर् व॑पेद् वपे॒न् निर् णिर् व॑पेद् ब्रह्मवर्च॒सका॑मो ब्रह्मवर्च॒सका॑मो वपे॒न् निर् णिर् व॑पेद् ब्रह्मवर्च॒सका॑मः । \newline
42. व॒पे॒द् ब्र॒ह्म॒व॒र्च॒सका॑मो ब्रह्मवर्च॒सका॑मो वपेद् वपेद् ब्रह्मवर्च॒सका॑मो॒ ऽग्नीषोमा॑ व॒ग्नीषोमौ᳚ ब्रह्मवर्च॒सका॑मो वपेद् वपेद् ब्रह्मवर्च॒सका॑मो॒ ऽग्नीषोमौ᳚ । \newline
43. ब्र॒ह्म॒व॒र्च॒सका॑मो॒ ऽग्नीषोमा॑ व॒ग्नीषोमौ᳚ ब्रह्मवर्च॒सका॑मो ब्रह्मवर्च॒सका॑मो॒ ऽग्नीषोमा॑ वे॒वैवाग्नीषोमौ᳚ ब्रह्मवर्च॒सका॑मो ब्रह्मवर्च॒सका॑मो॒ ऽग्नीषोमा॑ वे॒व । \newline
44. ब्र॒ह्म॒व॒र्च॒सका॑म॒ इति॑ ब्रह्मवर्च॒स - का॒मः॒ । \newline
45. अ॒ग्नीषोमा॑ वे॒वै वाग्नीषोमा॑ व॒ग्नीषोमा॑ वे॒व स्वेन॒ स्वेनै॒वाग्नीषोमा॑ व॒ग्नीषोमा॑ वे॒व स्वेन॑ । \newline
46. अ॒ग्नीषोमा॒वित्य॒ग्नी - सोमौ᳚ । \newline
47. ए॒व स्वेन॒ स्वेनै॒वैव स्वेन॑ भाग॒धेये॑न भाग॒धेये॑न॒ स्वेनै॒वैव स्वेन॑ भाग॒धेये॑न । \newline
48. स्वेन॑ भाग॒धेये॑न भाग॒धेये॑न॒ स्वेन॒ स्वेन॑ भाग॒धेये॒नोपोप॑ भाग॒धेये॑न॒ स्वेन॒ स्वेन॑ भाग॒धेये॒नोप॑ । \newline
49. भा॒ग॒धेये॒नोपोप॑ भाग॒धेये॑न भाग॒धेये॒नोप॑ धावति धाव॒त्युप॑ भाग॒धेये॑न भाग॒धेये॒नोप॑ धावति । \newline
50. भा॒ग॒धेये॒नेति॑ भाग - धेये॑न । \newline
51. उप॑ धावति धाव॒त्युपोप॑ धावति॒ तौ तौ धा॑व॒त्युपोप॑ धावति॒ तौ । \newline
52. धा॒व॒ति॒ तौ तौ धा॑वति धावति॒ ता वे॒वैव तौ धा॑वति धावति॒ ता वे॒व । \newline
53. ता वे॒वैव तौ ता वे॒वास्मि॑न् नस्मिन् ने॒व तौ ता वे॒वास्मिन्न्॑ । \newline
54. ए॒वास्मि॑न् नस्मिन् ने॒वैवास्मि॑न् ब्रह्मवर्च॒सम् ब्र॑ह्मवर्च॒स म॑स्मिन् ने॒वैवास्मि॑न् ब्रह्मवर्च॒सम् । \newline
55. अ॒स्मि॒न् ब्र॒ह्म॒व॒र्च॒सम् ब्र॑ह्मवर्च॒स म॑स्मिन् नस्मिन् ब्रह्मवर्च॒सम् ध॑त्तो धत्तो ब्रह्मवर्च॒स म॑स्मिन् नस्मिन् ब्रह्मवर्च॒सम् ध॑त्तः । \newline
56. ब्र॒ह्म॒व॒र्च॒सम् ध॑त्तो धत्तो ब्रह्मवर्च॒सम् ब्र॑ह्मवर्च॒सम् ध॑त्तो ब्रह्मवर्च॒सी ब्र॑ह्मवर्च॒सी ध॑त्तो ब्रह्मवर्च॒सम् ब्र॑ह्मवर्च॒सम् ध॑त्तो ब्रह्मवर्च॒सी । \newline
57. ब्र॒ह्म॒व॒र्च॒समिति॑ ब्रह्म - व॒र्च॒सम् । \newline
58. ध॒त्तो॒ ब्र॒ह्म॒व॒र्च॒सी ब्र॑ह्मवर्च॒सी ध॑त्तो धत्तो ब्रह्मवर्च॒ स्ये॑वैव ब्र॑ह्मवर्च॒सी ध॑त्तो धत्तो ब्रह्मवर्च॒ स्ये॑व । \newline
59. ब्र॒ह्म॒व॒र्च॒ स्ये॑वैव ब्र॑ह्मवर्च॒सी ब्र॑ह्मवर्च॒ स्ये॑व भ॑वति भवत्ये॒व ब्र॑ह्मवर्च॒सी ब्र॑ह्मवर्च॒ स्ये॑व भ॑वति । \newline
60. ब्र॒ह्म॒व॒र्च॒सीति॑ ब्रह्म - व॒र्च॒सी । \newline
61. ए॒व भ॑वति भव त्ये॒वैव भ॑वति॒ यद् यद् भ॑व त्ये॒वैव भ॑वति॒ यत् । \newline
\pagebreak
\markright{ TS 2.3.3.4  \hfill https://www.vedavms.in \hfill}

\section{ TS 2.3.3.4 }

\textbf{TS 2.3.3.4 } \newline
\textbf{Samhita Paata} \newline

भ॑वति॒ यद॒ष्टाक॑पाल॒-स्तेना᳚ऽऽग्ने॒यो यच्छ्या॑मा॒कस्तेन॑ सौ॒म्यः समृ॑द्ध्यै॒ सोमा॑य वा॒जिने᳚ श्यामा॒कं च॒रुं निर्व॑पे॒द्यः क्लैब्या᳚द्बिभी॒याद् रेतो॒ हि वा ए॒तस्मा॒द्-वाजि॑नमप॒क्राम॒त्यथै॒ष क्लैब्या᳚द्बिभाय॒ सोम॑मे॒व वा॒जिनꣳ॒॒ स्वेन॑ भाग॒धेये॒नोप॑ धावति॒ स ए॒वास्मि॒न् रेतो॒ वाजि॑नं दधाति॒ न क्ली॒बो भ॑वतिब्राह्मणस्प॒त्य-मेका॑दशकपालं॒ निर्व॑पे॒द् ग्राम॑कामो॒ - [  ] \newline

\textbf{Pada Paata} \newline

भ॒व॒ति॒ । यत् । अ॒ष्टाक॑पाल॒ इत्य॒ष्टा - क॒पा॒लः॒ । तेन॑ । आ॒ग्ने॒यः । यत् । श्या॒मा॒कः । तेन॑ । सौ॒म्यः । समृ॑द्ध्या॒ इति॒ सं - ऋ॒द्ध्यै॒ । सोमा॑य । वा॒जिने᳚ । श्या॒मा॒कम् । च॒रुम् । निरिति॑ । व॒पे॒त् । यः । क्लैब्या᳚त् । बि॒भी॒यात् । रेतः॑ । हि । वै । ए॒तस्मा᳚त् । वाजि॑नम् । अ॒प॒क्राम॒तीत्य॑प - क्राम॑ति । अथ॑ । ए॒षः । क्लैब्या᳚त् । बि॒भा॒य॒ । सोम᳚म् । ए॒व । वा॒जिन᳚म् । स्वेन॑ । भा॒ग॒धेये॒नेति॑ भाग - धेये॑न । उपेति॑ । धा॒व॒ति॒ । सः । ए॒व । अ॒स्मि॒न्न् । रेतः॑ । वाजि॑नम् । द॒धा॒ति॒ । न । क्ली॒बः । भ॒व॒ति॒ । ब्रा॒ह्म॒ण॒स्प॒त्यमिति॑ ब्राह्मणः - प॒त्यम् । एका॑दशकपाल॒मित्येका॑दश-क॒पा॒ल॒म् । निरिति॑ । व॒पे॒त् । ग्राम॑काम॒ इति॒ ग्राम॑ - का॒मः॒ ।  \newline


\textbf{Krama Paata} \newline

भ॒व॒ति॒ यत् । यद॒ष्टाक॑पालः । अ॒ष्टाक॑पाल॒स्तेन॑ । अ॒ष्टाक॑पाल॒ इत्य॒ष्टा - क॒पा॒लः॒ । तेना᳚ग्ने॒यः । आ॒ग्ने॒यो यत् । यच्छ्या॑मा॒कः । श्या॒मा॒क स्तेन॑ । तेन॑ सौ॒म्यः । सौ॒म्यः समृ॑द्ध्यै । समृ॑द्ध्यै॒ सोमा॑य । समृ॑द्ध्या॒ इति॒ सं - ऋ॒द्ध्यै॒ । सोमा॑य वा॒जिने᳚ । वा॒जिने᳚ श्यामा॒कम् । श्या॒मा॒कम् च॒रुम् । च॒रुम् निः । निर् व॑पेत् । व॒पे॒द् यः । यः क्लैब्या᳚त् । क्लैब्या᳚द् बिभी॒यात् । बि॒भी॒याद् रेतः॑ । रेतो॒ हि । हि वै । वा ए॒तस्मा᳚त् । ए॒तस्मा॒द् वाजि॑नम् । वाजि॑नमप॒क्राम॑ति । अ॒प॒क्राम॒त्यथ॑ । अ॒प॒क्राम॒तीत्य॑प - क्राम॑ति । अथै॒षः । ए॒ष क्लैब्या᳚त् । क्लैब्या᳚द् बिभाय । बि॒भा॒य॒ सोम᳚म् । सोम॑मे॒व । ए॒व वा॒जिन᳚म् । वा॒जिनꣳ॒॒ स्वेन॑ । स्वेन॑ भाग॒धेये॑न । भा॒ग॒धेये॒नोप॑ । भा॒ग॒धेये॒नेति॑ भाग - धेये॑न । उप॑ धावति । धा॒व॒ति॒ सः । स ए॒व । ए॒वास्मिन्न्॑ । अ॒स्मि॒न् रेतः॑ । रेतो॒ वाजि॑नम् । वाजि॑नम् दधाति । द॒धा॒ति॒ न । न क्ली॒बः । क्ली॒बो भ॑वति । भ॒व॒ति॒ ब्रा॒ह्म॒ण॒स्प॒त्यम् । ब्रा॒ह्म॒ण॒स्प॒त्यमेका॑दशकपालम् । ब्रा॒ह्म॒ण॒स्प॒त्यमिति॑ ब्राह्मणः - प॒त्यम् । एका॑दशकपाल॒म् निः । एका॑दशकपाल॒मित्येका॑दश - क॒पा॒ल॒म् । निर् व॑पेत् । व॒पे॒द् ग्राम॑कामः ( ) 
। ग्राम॑कामो॒ ब्रह्म॑णः । ग्राम॑काम॒ इति॒ ग्राम॑ - का॒मः॒ \newline

\textbf{Jatai Paata} \newline

1. भ॒व॒ति॒ यद् यद् भ॑वति भवति॒ यत् । \newline
2. यद॒ष्टाक॑पालो॒ ऽष्टाक॑पालो॒ यद् यद॒ष्टाक॑पालः । \newline
3. अ॒ष्टाक॑पाल॒ स्तेन॒ तेना॒ष्टाक॑पालो॒ ऽष्टाक॑पाल॒ स्तेन॑ । \newline
4. अ॒ष्टाक॑पाल॒ इत्य॒ष्टा - क॒पा॒लः॒ । \newline
5. तेना᳚ग्ने॒य आ᳚ग्ने॒य स्तेन॒ तेना᳚ग्ने॒यः । \newline
6. आ॒ग्ने॒यो यद् यदा᳚ग्ने॒य आ᳚ग्ने॒यो यत् । \newline
7. यच् छ्या॑मा॒कः श्या॑मा॒को यद् यच् छ्या॑मा॒कः । \newline
8. श्या॒मा॒क स्तेन॒ तेन॑ श्यामा॒कः श्या॑मा॒क स्तेन॑ । \newline
9. तेन॑ सौ॒म्यः सौ॒म्य स्तेन॒ तेन॑ सौ॒म्यः । \newline
10. सौ॒म्यः समृ॑द्ध्यै॒ समृ॑द्ध्यै सौ॒म्यः सौ॒म्यः समृ॑द्ध्यै । \newline
11. समृ॑द्ध्यै॒ सोमा॑य॒ सोमा॑य॒ समृ॑द्ध्यै॒ समृ॑द्ध्यै॒ सोमा॑य । \newline
12. समृ॑द्ध्या॒ इति॒ सं - ऋ॒द्ध्यै॒ । \newline
13. सोमा॑य वा॒जिने॑ वा॒जिने॒ सोमा॑य॒ सोमा॑य वा॒जिने᳚ । \newline
14. वा॒जिने᳚ श्यामा॒कꣳ श्या॑मा॒कं ॅवा॒जिने॑ वा॒जिने᳚ श्यामा॒कम् । \newline
15. श्या॒मा॒कम् च॒रुम् च॒रुꣳ श्या॑मा॒कꣳ श्या॑मा॒कम् च॒रुम् । \newline
16. च॒रुम् निर् णिश्च॒रुम् च॒रुम् निः । \newline
17. निर् व॑पेद् वपे॒न् निर् णिर् व॑पेत् । \newline
18. व॒पे॒द् यो यो व॑पेद् वपे॒द् यः । \newline
19. यः क्लैब्या॒त् क्लैब्या॒द् यो यः क्लैब्या᳚त् । \newline
20. क्लैब्या᳚द् बिभी॒याद् बि॑भी॒यात् क्लैब्या॒त् क्लैब्या᳚द् बिभी॒यात् । \newline
21. बि॒भी॒याद् रेतो॒ रेतो॑ बिभी॒याद् बि॑भी॒याद् रेतः॑ । \newline
22. रेतो॒ हि हि रेतो॒ रेतो॒ हि । \newline
23. हि वै वै हि हि वै । \newline
24. वा ए॒तस्मा॑ दे॒तस्मा॒द् वै वा ए॒तस्मा᳚त् । \newline
25. ए॒तस्मा॒द् वाजि॑नं॒ ॅवाजि॑न मे॒तस्मा॑ दे॒तस्मा॒द् वाजि॑नम् । \newline
26. वाजि॑न मप॒क्राम॑ त्यप॒क्राम॑ति॒ वाजि॑नं॒ ॅवाजि॑न मप॒क्राम॑ति । \newline
27. अ॒प॒क्राम॒ त्यथाथा॑ प॒क्राम॑ त्यप॒क्राम॒ त्यथ॑ । \newline
28. अ॒प॒क्राम॒तीत्य॑प - क्राम॑ति । \newline
29. अथै॒ष ए॒षो ऽथाथै॒षः । \newline
30. ए॒ष क्लैब्या॒त् क्लैब्या॑ दे॒ष ए॒ष क्लैब्या᳚त् । \newline
31. क्लैब्या᳚द् बिभाय बिभाय॒ क्लैब्या॒त् क्लैब्या᳚द् बिभाय । \newline
32. बि॒भा॒य॒ सोमꣳ॒॒ सोम॑म् बिभाय बिभाय॒ सोम᳚म् । \newline
33. सोम॑ मे॒वैव सोमꣳ॒॒ सोम॑ मे॒व । \newline
34. ए॒व वा॒जिनं॑ ॅवा॒जिन॑ मे॒वैव वा॒जिन᳚म् । \newline
35. वा॒जिनꣳ॒॒ स्वेन॒ स्वेन॑ वा॒जिनं॑ ॅवा॒जिनꣳ॒॒ स्वेन॑ । \newline
36. स्वेन॑ भाग॒धेये॑न भाग॒धेये॑न॒ स्वेन॒ स्वेन॑ भाग॒धेये॑न । \newline
37. भा॒ग॒धेये॒नोपोप॑ भाग॒धेये॑न भाग॒धेये॒नोप॑ । \newline
38. भा॒ग॒धेये॒नेति॑ भाग - धेये॑न । \newline
39. उप॑ धावति धाव॒ त्युपोप॑ धावति । \newline
40. धा॒व॒ति॒ स स धा॑वति धावति॒ सः । \newline
41. स ए॒वैव स स ए॒व । \newline
42. ए॒वास्मि॑न् नस्मिन् ने॒वैवास्मिन्न्॑ । \newline
43. अ॒स्मि॒न् रेतो॒ रेतो᳚ ऽस्मिन् नस्मि॒न् रेतः॑ । \newline
44. रेतो॒ वाजि॑नं॒ ॅवाजि॑नꣳ॒॒ रेतो॒ रेतो॒ वाजि॑नम् । \newline
45. वाजि॑नम् दधाति दधाति॒ वाजि॑नं॒ ॅवाजि॑नम् दधाति । \newline
46. द॒धा॒ति॒ न न द॑धाति दधाति॒ न । \newline
47. न क्ली॒बः क्ली॒बो न न क्ली॒बः । \newline
48. क्ली॒बो भ॑वति भवति क्ली॒बः क्ली॒बो भ॑वति । \newline
49. भ॒व॒ति॒ ब्रा॒ह्म॒ण॒स्प॒त्यम् ब्रा᳚ह्मणस्प॒त्यम् भ॑वति भवति ब्राह्मणस्प॒त्यम् । \newline
50. ब्रा॒ह्म॒ण॒स्प॒त्य मेका॑दशकपाल॒ मेका॑दशकपालम् ब्राह्मणस्प॒त्यम् ब्रा᳚ह्मणस्प॒त्य मेका॑दशकपालम् । \newline
51. ब्रा॒ह्म॒ण॒स्प॒त्यमिति॑ ब्राह्मणः - प॒त्यम् । \newline
52. एका॑दशकपाल॒म् निर् णिरेका॑दशकपाल॒ मेका॑दशकपाल॒म् निः । \newline
53. एका॑दशकपाल॒मित्येका॑दश - क॒पा॒ल॒म् । \newline
54. निर् व॑पेद् वपे॒न् निर् णिर् व॑पेत् । \newline
55. व॒पे॒द् ग्राम॑कामो॒ ग्राम॑कामो वपेद् वपे॒द् ग्राम॑कामः । \newline
56. ग्राम॑कामो॒ ब्रह्म॑णो॒ ब्रह्म॑णो॒ ग्राम॑कामो॒ ग्राम॑कामो॒ ब्रह्म॑णः । \newline
57. ग्राम॑काम॒ इति॒ ग्राम॑ - का॒मः॒ । \newline

\textbf{Ghana Paata } \newline

1. भ॒व॒ति॒ यद् यद् भ॑वति भवति॒ यद॒ष्टाक॑पालो॒ ऽष्टाक॑पालो॒ यद् भ॑वति भवति॒ यद॒ष्टाक॑पालः । \newline
2. यद॒ष्टाक॑पालो॒ ऽष्टाक॑पालो॒ यद् यद॒ष्टाक॑पाल॒ स्तेन॒ तेना॒ष्टाक॑पालो॒ यद् यद॒ष्टाक॑पाल॒ स्तेन॑ । \newline
3. अ॒ष्टाक॑पाल॒ स्तेन॒ तेना॒ष्टाक॑पालो॒ ऽष्टाक॑पाल॒ स्तेना᳚ग्ने॒य आ᳚ग्ने॒य स्तेना॒ष्टाक॑पालो॒ ऽष्टाक॑पाल॒ स्तेना᳚ग्ने॒यः । \newline
4. अ॒ष्टाक॑पाल॒ इत्य॒ष्टा - क॒पा॒लः॒ । \newline
5. तेना᳚ग्ने॒य आ᳚ग्ने॒य स्तेन॒ तेना᳚ग्ने॒यो यद् यदा᳚ग्ने॒य स्तेन॒ तेना᳚ग्ने॒यो यत् । \newline
6. आ॒ग्ने॒यो यद् यदा᳚ग्ने॒य आ᳚ग्ने॒यो यच् छ्या॑मा॒कः श्या॑मा॒को यदा᳚ग्ने॒य आ᳚ग्ने॒यो यच् छ्या॑मा॒कः । \newline
7. यच् छ्या॑मा॒कः श्या॑मा॒को यद् यच् छ्या॑मा॒क स्तेन॒ तेन॑ श्यामा॒को यद् यच् छ्या॑मा॒क स्तेन॑ । \newline
8. श्या॒मा॒क स्तेन॒ तेन॑ श्यामा॒कः श्या॑मा॒क स्तेन॑ सौ॒म्यः सौ॒म्य स्तेन॑ श्यामा॒कः श्या॑मा॒क स्तेन॑ सौ॒म्यः । \newline
9. तेन॑ सौ॒म्यः सौ॒म्य स्तेन॒ तेन॑ सौ॒म्यः समृ॑द्ध्यै॒ समृ॑द्ध्यै सौ॒म्य स्तेन॒ तेन॑ सौ॒म्यः समृ॑द्ध्यै । \newline
10. सौ॒म्यः समृ॑द्ध्यै॒ समृ॑द्ध्यै सौ॒म्यः सौ॒म्यः समृ॑द्ध्यै॒ सोमा॑य॒ सोमा॑य॒ समृ॑द्ध्यै सौ॒म्यः सौ॒म्यः समृ॑द्ध्यै॒ सोमा॑य । \newline
11. समृ॑द्ध्यै॒ सोमा॑य॒ सोमा॑य॒ समृ॑द्ध्यै॒ समृ॑द्ध्यै॒ सोमा॑य वा॒जिने॑ वा॒जिने॒ सोमा॑य॒ समृ॑द्ध्यै॒ समृ॑द्ध्यै॒ सोमा॑य वा॒जिने᳚ । \newline
12. समृ॑द्ध्या॒ इति॒ सं - ऋ॒द्ध्यै॒ । \newline
13. सोमा॑य वा॒जिने॑ वा॒जिने॒ सोमा॑य॒ सोमा॑य वा॒जिने᳚ श्यामा॒कꣳ श्या॑मा॒कं ॅवा॒जिने॒ सोमा॑य॒ सोमा॑य वा॒जिने᳚ श्यामा॒कम् । \newline
14. वा॒जिने᳚ श्यामा॒कꣳ श्या॑मा॒कं ॅवा॒जिने॑ वा॒जिने᳚ श्यामा॒कम् च॒रुम् च॒रुꣳ श्या॑मा॒कं ॅवा॒जिने॑ वा॒जिने᳚ श्यामा॒कम् च॒रुम् । \newline
15. श्या॒मा॒कम् च॒रुम् च॒रुꣳ श्या॑मा॒कꣳ श्या॑मा॒कम् च॒रुम् निर् णिश्च॒रुꣳ श्या॑मा॒कꣳ श्या॑मा॒कम् च॒रुम् निः । \newline
16. च॒रुम् निर् णिश्च॒रुम् च॒रुम् निर् व॑पेद् वपे॒न् निश्च॒रुम् च॒रुम् निर् व॑पेत् । \newline
17. निर् व॑पेद् वपे॒न् निर् णिर् व॑पे॒द् यो यो व॑पे॒न् निर् णिर् व॑पे॒द् यः । \newline
18. व॒पे॒द् यो यो व॑पेद् वपे॒द् यः क्लैब्या॒त् क्लैब्या॒द् यो व॑पेद् वपे॒द् यः क्लैब्या᳚त् । \newline
19. यः क्लैब्या॒त् क्लैब्या॒द् यो यः क्लैब्या᳚द् बिभी॒याद् बि॑भी॒यात् क्लैब्या॒द् यो यः क्लैब्या᳚द् बिभी॒यात् । \newline
20. क्लैब्या᳚द् बिभी॒याद् बि॑भी॒यात् क्लैब्या॒त् क्लैब्या᳚द् बिभी॒याद् रेतो॒ रेतो॑ बिभी॒यात् क्लैब्या॒त् क्लैब्या᳚द् बिभी॒याद् रेतः॑ । \newline
21. बि॒भी॒याद् रेतो॒ रेतो॑ बिभी॒याद् बि॑भी॒याद् रेतो॒ हि हि रेतो॑ बिभी॒याद् बि॑भी॒याद् रेतो॒ हि । \newline
22. रेतो॒ हि हि रेतो॒ रेतो॒ हि वै वै हि रेतो॒ रेतो॒ हि वै । \newline
23. हि वै वै हि हि वा ए॒तस्मा॑ दे॒तस्मा॒द् वै हि हि वा ए॒तस्मा᳚त् । \newline
24. वा ए॒तस्मा॑ दे॒तस्मा॒द् वै वा ए॒तस्मा॒द् वाजि॑नं॒ ॅवाजि॑न मे॒तस्मा॒द् वै वा ए॒तस्मा॒द् वाजि॑नम् । \newline
25. ए॒तस्मा॒द् वाजि॑नं॒ ॅवाजि॑न मे॒तस्मा॑ दे॒तस्मा॒द् वाजि॑न मप॒क्राम॑ त्यप॒क्राम॑ति॒ वाजि॑न मे॒तस्मा॑ दे॒तस्मा॒द् वाजि॑न मप॒क्राम॑ति । \newline
26. वाजि॑न मप॒क्राम॑ त्यप॒क्राम॑ति॒ वाजि॑नं॒ ॅवाजि॑न मप॒क्राम॒ त्यथाथा॑ प॒क्राम॑ति॒ वाजि॑नं॒ ॅवाजि॑न मप॒क्राम॒त्यथ॑ । \newline
27. अ॒प॒क्राम॒ त्यथाथा॑ प॒क्राम॑ त्यप॒क्राम॒ त्यथै॒ष ए॒षो ऽथा॑प॒क्राम॑ त्यप॒क्राम॒ त्यथै॒षः । \newline
28. अ॒प॒क्राम॒तीत्य॑प - क्राम॑ति । \newline
29. अथै॒ष ए॒षो ऽथाथै॒ष क्लैब्या॒त् क्लैब्या॑ दे॒षो ऽथाथै॒ष क्लैब्या᳚त् । \newline
30. ए॒ष क्लैब्या॒त् क्लैब्या॑ दे॒ष ए॒ष क्लैब्या᳚द् बिभाय बिभाय॒ क्लैब्या॑ दे॒ष ए॒ष क्लैब्या᳚द् बिभाय । \newline
31. क्लैब्या᳚द् बिभाय बिभाय॒ क्लैब्या॒त् क्लैब्या᳚द् बिभाय॒ सोमꣳ॒॒ सोम॑म् बिभाय॒ क्लैब्या॒त् क्लैब्या᳚द् बिभाय॒ सोम᳚म् । \newline
32. बि॒भा॒य॒ सोमꣳ॒॒ सोम॑म् बिभाय बिभाय॒ सोम॑ मे॒वैव सोम॑म् बिभाय बिभाय॒ सोम॑ मे॒व । \newline
33. सोम॑ मे॒वैव सोमꣳ॒॒ सोम॑ मे॒व वा॒जिनं॑ ॅवा॒जिन॑ मे॒व सोमꣳ॒॒ सोम॑ मे॒व वा॒जिन᳚म् । \newline
34. ए॒व वा॒जिनं॑ ॅवा॒जिन॑ मे॒वैव वा॒जिनꣳ॒॒ स्वेन॒ स्वेन॑ वा॒जिन॑ मे॒वैव वा॒जिनꣳ॒॒ स्वेन॑ । \newline
35. वा॒जिनꣳ॒॒ स्वेन॒ स्वेन॑ वा॒जिनं॑ ॅवा॒जिनꣳ॒॒ स्वेन॑ भाग॒धेये॑न भाग॒धेये॑न॒ स्वेन॑ वा॒जिनं॑ ॅवा॒जिनꣳ॒॒ स्वेन॑ भाग॒धेये॑न । \newline
36. स्वेन॑ भाग॒धेये॑न भाग॒धेये॑न॒ स्वेन॒ स्वेन॑ भाग॒धेये॒नोपोप॑ भाग॒धेये॑न॒ स्वेन॒ स्वेन॑ भाग॒धेये॒नोप॑ । \newline
37. भा॒ग॒धेये॒नोपोप॑ भाग॒धेये॑न भाग॒धेये॒नोप॑ धावति धाव॒त्युप॑ भाग॒धेये॑न भाग॒धेये॒नोप॑ धावति । \newline
38. भा॒ग॒धेये॒नेति॑ भाग - धेये॑न । \newline
39. उप॑ धावति धाव॒ त्युपोप॑ धावति॒ स स धा॑व॒ त्युपोप॑ धावति॒ सः । \newline
40. धा॒व॒ति॒ स स धा॑वति धावति॒ स ए॒वैव स धा॑वति धावति॒ स ए॒व । \newline
41. स ए॒वैव स स ए॒वास्मि॑न् नस्मिन् ने॒व स स ए॒वास्मिन्न्॑ । \newline
42. ए॒वास्मि॑न् नस्मिन् ने॒वैवास्मि॒न् रेतो॒ रेतो᳚ ऽस्मिन् ने॒वैवास्मि॒न् रेतः॑ । \newline
43. अ॒स्मि॒न् रेतो॒ रेतो᳚ ऽस्मिन् नस्मि॒न् रेतो॒ वाजि॑नं॒ ॅवाजि॑नꣳ॒॒ रेतो᳚ ऽस्मिन् नस्मि॒न् रेतो॒ वाजि॑नम् । \newline
44. रेतो॒ वाजि॑नं॒ ॅवाजि॑नꣳ॒॒ रेतो॒ रेतो॒ वाजि॑नम् दधाति दधाति॒ वाजि॑नꣳ॒॒ रेतो॒ रेतो॒ वाजि॑नम् दधाति । \newline
45. वाजि॑नम् दधाति दधाति॒ वाजि॑नं॒ ॅवाजि॑नम् दधाति॒ न न द॑धाति॒ वाजि॑नं॒ ॅवाजि॑नम् दधाति॒ न । \newline
46. द॒धा॒ति॒ न न द॑धाति दधाति॒ न क्ली॒बः क्ली॒बो न द॑धाति दधाति॒ न क्ली॒बः । \newline
47. न क्ली॒बः क्ली॒बो न न क्ली॒बो भ॑वति भवति क्ली॒बो न न क्ली॒बो भ॑वति । \newline
48. क्ली॒बो भ॑वति भवति क्ली॒बः क्ली॒बो भ॑वति ब्राह्मणस्प॒त्यम् ब्रा᳚ह्मणस्प॒त्यम् भ॑वति क्ली॒बः क्ली॒बो भ॑वति ब्राह्मणस्प॒त्यम् । \newline
49. भ॒व॒ति॒ ब्रा॒ह्म॒ण॒स्प॒त्यम् ब्रा᳚ह्मणस्प॒त्यम् भ॑वति भवति ब्राह्मणस्प॒त्य मेका॑दशकपाल॒ मेका॑दशकपालम् ब्राह्मणस्प॒त्यम् भ॑वति भवति ब्राह्मणस्प॒त्य मेका॑दशकपालम् । \newline
50. ब्रा॒ह्म॒ण॒स्प॒त्य मेका॑दशकपाल॒ मेका॑दशकपालम् ब्राह्मणस्प॒त्यम् ब्रा᳚ह्मणस्प॒त्य मेका॑दशकपाल॒म् निर् णिरेका॑दशकपालम् ब्राह्मणस्प॒त्यम् ब्रा᳚ह्मणस्प॒त्य मेका॑दशकपाल॒म् निः । \newline
51. ब्रा॒ह्म॒ण॒स्प॒त्यमिति॑ ब्राह्मणः - प॒त्यम् । \newline
52. एका॑दशकपाल॒म् निर् णिरेका॑दशकपाल॒ मेका॑दशकपाल॒म् निर् व॑पेद् वपे॒न् निरेका॑दशकपाल॒ मेका॑दशकपाल॒म् निर् व॑पेत् । \newline
53. एका॑दशकपाल॒मित्येका॑दश - क॒पा॒ल॒म् । \newline
54. निर् व॑पेद् वपे॒न् निर् णिर् व॑पे॒द् ग्राम॑कामो॒ ग्राम॑कामो वपे॒न् निर् णिर् व॑पे॒द् ग्राम॑कामः । \newline
55. व॒पे॒द् ग्राम॑कामो॒ ग्राम॑कामो वपेद् वपे॒द् ग्राम॑कामो॒ ब्रह्म॑णो॒ ब्रह्म॑णो॒ ग्राम॑कामो वपेद् वपे॒द् ग्राम॑कामो॒ ब्रह्म॑णः । \newline
56. ग्राम॑कामो॒ ब्रह्म॑णो॒ ब्रह्म॑णो॒ ग्राम॑कामो॒ ग्राम॑कामो॒ ब्रह्म॑ण॒ स्पति॒म् पति॒म् ब्रह्म॑णो॒ ग्राम॑कामो॒ ग्राम॑कामो॒ ब्रह्म॑ण॒ स्पति᳚म् । \newline
57. ग्राम॑काम॒ इति॒ ग्राम॑ - का॒मः॒ । \newline
\pagebreak
\markright{ TS 2.3.3.5  \hfill https://www.vedavms.in \hfill}

\section{ TS 2.3.3.5 }

\textbf{TS 2.3.3.5 } \newline
\textbf{Samhita Paata} \newline

ब्रह्म॑ण॒स्पति॑मे॒व स्वेन॑ भाग॒धेये॒नोप॑ धावति॒ स ए॒वास्मै॑ सजा॒तान् प्र य॑च्छति ग्रा॒म्ये॑व भ॑वति ग॒णव॑ती याज्यानुवा॒क्ये॑ भवतः सजा॒तैरे॒वैनं॑ ग॒णव॑न्तं करोत्ये॒तामे॒व निर्व॑पे॒द्यः का॒मये॑त॒ ब्रह्म॒न् \newline

\textbf{Pada Paata} \newline

ब्रह्म॑णः । पति᳚म् । ए॒व । स्वेन॑ । भा॒ग॒धेये॒नेति॑ भाग-धेये॑न । उपेति॑ । धा॒व॒ति॒ । सः । ए॒व । अ॒स्मै॒ । स॒जा॒तानिति॑ स - जा॒तान् । प्रेति॑ । य॒च्छ॒ति॒ । ग्रा॒मी । ए॒व । भ॒व॒ति॒ । ग॒णव॑ती॒ इति॑ ग॒ण - व॒ती॒ । या॒ज्या॒नु॒वा॒क्ये॑ इति॑ याज्या - अ॒नु॒वा॒क्ये᳚ । भ॒व॒तः॒ । स॒जा॒तैरिति॑ स - जा॒तैः । ए॒व । ए॒न॒म् । ग॒णव॑न्त॒मिति॑ ग॒ण - व॒न्त॒म् । क॒रो॒ति॒ । ए॒ताम् । ए॒व । निरिति॑ । व॒पे॒त् । यः । का॒मये॑त । ब्रह्मन्न्॑ । विश᳚म् । वीति॑ । ना॒श॒ये॒य॒म् । इति॑ । मा॒रु॒ती इति॑ । या॒ज्या॒नु॒वा॒क्ये॑ इति॑ याज्या - अ॒नु॒वा॒क्ये᳚ । कु॒र्या॒त् । ब्रह्मन्न्॑ । ए॒व । विश᳚म् । वीति॑ । ना॒श॒य॒ति॒ ॥  \newline


\textbf{Krama Paata} \newline

ब्रह्म॑ण॒स्पति᳚म् । पति॑मे॒व । ए॒व स्वेन॑ । स्वेन॑ भाग॒धेये॑न । भा॒ग॒धेये॒नोप॑ । भा॒ग॒धेये॒नेति॑ भाग - धेये॑न । उप॑ धावति । धा॒व॒ति॒ सः । स ए॒व । ए॒वास्मै᳚ । अ॒स्मै॒ स॒जा॒तान् । स॒जा॒तान् प्र । स॒जा॒तानिति॑ स - जा॒तान् । प्र य॑च्छति । य॒च्छ॒ति॒ ग्रा॒मी । ग्रा॒म्ये॑व । ए॒व भ॑वति । भ॒व॒ति॒ ग॒णव॑ती । ग॒णव॑ती याज्यानुवा॒क्ये᳚ । ग॒णव॑ती॒ इति॑ ग॒ण - व॒ती॒ । या॒ज्या॒नु॒वा॒क्ये॑ भवतः । या॒ज्या॒नु॒वा॒क्ये॑ इति॑ याज्या - अ॒नु॒वा॒क्ये᳚ । भ॒व॒तः॒ स॒जा॒तैः । स॒जा॒तैरे॒व । स॒जा॒तैरिति॑ स - जा॒तैः । ए॒वैन᳚म् । ए॒न॒म् ग॒णव॑न्तम् । ग॒णव॑न्तम् करोति । ग॒णव॑न्त॒मिति॑ ग॒ण - व॒न्त॒म् । क॒रो॒त्ये॒ताम् । ए॒तामे॒व । ए॒व निः । निर् व॑पेत् । व॒पे॒द् यः । यः का॒मये॑त । का॒मये॑त॒ ब्रह्मन्न्॑ । ब्रह्म॒न्॒. विश᳚म् । विशं॒ ॅवि । वि ना॑शयेयम् । ना॒श॒ये॒य॒मिति॑ । इति॑ मारु॒ती । मा॒रु॒ती या᳚ज्यानुवा॒क्ये᳚ । मा॒रु॒ती इति॑ मारु॒ती । या॒ज्या॒नु॒वा॒क्ये॑ कुर्यात् । या॒ज्या॒नु॒वा॒क्ये॑ इति॑ याज्या - अ॒नु॒वा॒क्ये᳚ । कु॒र्या॒द् ब्रह्मन्न्॑ । ब्रह्म॑न्ने॒व । ए॒व विश᳚म् । विशं॒ ॅवि । वि ना॑शयति । ना॒श॒य॒तीति॑ नाशयति । \newline

\textbf{Jatai Paata} \newline

1. ब्रह्म॑ण॒ स्पति॒म् पति॒म् ब्रह्म॑णो॒ ब्रह्म॑ण॒ स्पति᳚म् । \newline
2. पति॑ मे॒वैव पति॒म् पति॑ मे॒व । \newline
3. ए॒व स्वेन॒ स्वेनै॒वैव स्वेन॑ । \newline
4. स्वेन॑ भाग॒धेये॑न भाग॒धेये॑न॒ स्वेन॒ स्वेन॑ भाग॒धेये॑न । \newline
5. भा॒ग॒धेये॒नोपोप॑ भाग॒धेये॑न भाग॒धेये॒नोप॑ । \newline
6. भा॒ग॒धेये॒नेति॑ भाग - धेये॑न । \newline
7. उप॑ धावति धाव॒ त्युपोप॑ धावति । \newline
8. धा॒व॒ति॒ स स धा॑वति धावति॒ सः । \newline
9. स ए॒वैव स स ए॒व । \newline
10. ए॒वास्मा॑ अस्मा ए॒वैवास्मै᳚ । \newline
11. अ॒स्मै॒ स॒जा॒तान् थ्स॑जा॒ता न॑स्मा अस्मै सजा॒तान् । \newline
12. स॒जा॒तान् प्र प्र स॑जा॒तान् थ्स॑जा॒तान् प्र । \newline
13. स॒जा॒तानिति॑ स - जा॒तान् । \newline
14. प्र य॑च्छति यच्छति॒ प्र प्र य॑च्छति । \newline
15. य॒च्छ॒ति॒ ग्रा॒मी ग्रा॒मी य॑च्छति यच्छति ग्रा॒मी । \newline
16. ग्रा॒म्ये॑वैव ग्रा॒मी ग्रा॒म्ये॑व । \newline
17. ए॒व भ॑वति भव त्ये॒वैव भ॑वति । \newline
18. भ॒व॒ति॒ ग॒णव॑ती ग॒णव॑ती भवति भवति ग॒णव॑ती । \newline
19. ग॒णव॑ती याज्यानुवा॒क्ये॑ याज्यानुवा॒क्ये॑ ग॒णव॑ती ग॒णव॑ती याज्यानुवा॒क्ये᳚ । \newline
20. ग॒णव॑ती॒ इति॑ ग॒ण - व॒ती॒ । \newline
21. या॒ज्या॒नु॒वा॒क्ये॑ भवतो भवतो याज्यानुवा॒क्ये॑ याज्यानुवा॒क्ये॑ भवतः । \newline
22. या॒ज्या॒नु॒वा॒क्ये॑ इति॑ याज्या - अ॒नु॒वा॒क्ये᳚ । \newline
23. भ॒व॒तः॒ स॒जा॒तैः स॑जा॒तैर् भ॑वतो भवतः सजा॒तैः । \newline
24. स॒जा॒तै रे॒वैव स॑जा॒तैः स॑जा॒तै रे॒व । \newline
25. स॒जा॒तैरिति॑ स - जा॒तैः । \newline
26. ए॒वैन॑ मेन मे॒वैवैन᳚म् । \newline
27. ए॒न॒म् ग॒णव॑न्तम् ग॒णव॑न्त मेन मेनम् ग॒णव॑न्तम् । \newline
28. ग॒णव॑न्तम् करोति करोति ग॒णव॑न्तम् ग॒णव॑न्तम् करोति । \newline
29. ग॒णव॑न्त॒मिति॑ ग॒ण - व॒न्त॒म् । \newline
30. क॒रो॒ त्ये॒ता मे॒ताम् क॑रोति करो त्ये॒ताम् । \newline
31. ए॒ता मे॒वैवैता मे॒ता मे॒व । \newline
32. ए॒व निर् णिरे॒वैव निः । \newline
33. निर् व॑पेद् वपे॒न् निर् णिर् व॑पेत् । \newline
34. व॒पे॒द् यो यो व॑पेद् वपे॒द् यः । \newline
35. यः का॒मये॑त का॒मये॑त॒ यो यः का॒मये॑त । \newline
36. का॒मये॑त॒ ब्रह्म॒न् ब्रह्म॑न् का॒मये॑त का॒मये॑त॒ ब्रह्मन्न्॑ । \newline
37. ब्रह्म॒न्॒. विशं॒ ॅविश॒म् ब्रह्म॒न् ब्रह्म॒न्॒. विश᳚म् । \newline
38. विशं॒ ॅवि वि विशं॒ ॅविशं॒ ॅवि । \newline
39. वि ना॑शयेयम् नाशयेयं॒ ॅवि वि ना॑शयेयम् । \newline
40. ना॒श॒ये॒य॒ मितीति॑ नाशयेयम् नाशयेय॒ मिति॑ । \newline
41. इति॑ मारु॒ती मा॑रु॒ती इतीति॑ मारु॒ती । \newline
42. मा॒रु॒ती या᳚ज्यानुवा॒क्ये॑ याज्यानुवा॒क्ये॑ मारु॒ती मा॑रु॒ती या᳚ज्यानुवा॒क्ये᳚ । \newline
43. मा॒रु॒ती इति॑ मारु॒ती । \newline
44. या॒ज्या॒नु॒वा॒क्ये॑ कुर्यात् कुर्याद् याज्यानुवा॒क्ये॑ याज्यानुवा॒क्ये॑ कुर्यात् । \newline
45. या॒ज्या॒नु॒वा॒क्ये॑ इति॑ याज्या - अ॒नु॒वा॒क्ये᳚ । \newline
46. कु॒र्या॒द् ब्रह्म॒न् ब्रह्म॑न् कुर्यात् कुर्या॒द् ब्रह्मन्न्॑ । \newline
47. ब्रह्म॑न् ने॒वैव ब्रह्म॒न् ब्रह्म॑न् ने॒व । \newline
48. ए॒व विशं॒ ॅविश॑ मे॒वैव विश᳚म् । \newline
49. विशं॒ ॅवि वि विशं॒ ॅविशं॒ ॅवि । \newline
50. वि ना॑शयति नाशयति॒ वि वि ना॑शयति । \newline
51. ना॒श॒य॒तीति॑ नाशयति । \newline

\textbf{Ghana Paata } \newline

1. ब्रह्म॑ण॒ स्पति॒म् पति॒म् ब्रह्म॑णो॒ ब्रह्म॑ण॒ स्पति॑ मे॒वैव पति॒म् ब्रह्म॑णो॒ ब्रह्म॑ण॒ स्पति॑ मे॒व । \newline
2. पति॑ मे॒वैव पति॒म् पति॑ मे॒व स्वेन॒ स्वेनै॒व पति॒म् पति॑ मे॒व स्वेन॑ । \newline
3. ए॒व स्वेन॒ स्वेनै॒वैव स्वेन॑ भाग॒धेये॑न भाग॒धेये॑न॒ स्वेनै॒वैव स्वेन॑ भाग॒धेये॑न । \newline
4. स्वेन॑ भाग॒धेये॑न भाग॒धेये॑न॒ स्वेन॒ स्वेन॑ भाग॒धेये॒नोपोप॑ भाग॒धेये॑न॒ स्वेन॒ स्वेन॑ भाग॒धेये॒नोप॑ । \newline
5. भा॒ग॒धेये॒नोपोप॑ भाग॒धेये॑न भाग॒धेये॒नोप॑ धावति धाव॒त्युप॑ भाग॒धेये॑न भाग॒धेये॒नोप॑ धावति । \newline
6. भा॒ग॒धेये॒नेति॑ भाग - धेये॑न । \newline
7. उप॑ धावति धाव॒ त्युपोप॑ धावति॒ स स धा॑व॒ त्युपोप॑ धावति॒ सः । \newline
8. धा॒व॒ति॒ स स धा॑वति धावति॒ स ए॒वैव स धा॑वति धावति॒ स ए॒व । \newline
9. स ए॒वैव स स ए॒वास्मा॑ अस्मा ए॒व स स ए॒वास्मै᳚ । \newline
10. ए॒वास्मा॑ अस्मा ए॒वैवास्मै॑ सजा॒तान् थ्स॑जा॒ता न॑स्मा ए॒वैवास्मै॑ सजा॒तान् । \newline
11. अ॒स्मै॒ स॒जा॒तान् थ्स॑जा॒ता न॑स्मा अस्मै सजा॒तान् प्र प्र स॑जा॒ता न॑स्मा अस्मै सजा॒तान् प्र । \newline
12. स॒जा॒तान् प्र प्र स॑जा॒तान् थ्स॑जा॒तान् प्र य॑च्छति यच्छति॒ प्र स॑जा॒तान् थ्स॑जा॒तान् प्र य॑च्छति । \newline
13. स॒जा॒तानिति॑ स - जा॒तान् । \newline
14. प्र य॑च्छति यच्छति॒ प्र प्र य॑च्छति ग्रा॒मी ग्रा॒मी य॑च्छति॒ प्र प्र य॑च्छति ग्रा॒मी । \newline
15. य॒च्छ॒ति॒ ग्रा॒मी ग्रा॒मी य॑च्छति यच्छति ग्रा॒म्ये॑वैव ग्रा॒मी य॑च्छति यच्छति ग्रा॒म्ये॑व । \newline
16. ग्रा॒म्ये॑वैव ग्रा॒मी ग्रा॒म्ये॑व भ॑वति भवत्ये॒व ग्रा॒मी ग्रा॒म्ये॑व भ॑वति । \newline
17. ए॒व भ॑वति भवत्ये॒वैव भ॑वति ग॒णव॑ती ग॒णव॑ती भवत्ये॒वैव भ॑वति ग॒णव॑ती । \newline
18. भ॒व॒ति॒ ग॒णव॑ती ग॒णव॑ती भवति भवति ग॒णव॑ती याज्यानुवा॒क्ये॑ याज्यानुवा॒क्ये॑ ग॒णव॑ती भवति भवति ग॒णव॑ती याज्यानुवा॒क्ये᳚ । \newline
19. ग॒णव॑ती याज्यानुवा॒क्ये॑ याज्यानुवा॒क्ये॑ ग॒णव॑ती ग॒णव॑ती याज्यानुवा॒क्ये॑ भवतो भवतो याज्यानुवा॒क्ये॑ ग॒णव॑ती ग॒णव॑ती याज्यानुवा॒क्ये॑ भवतः । \newline
20. ग॒णव॑ती॒ इति॑ ग॒ण - व॒ती॒ । \newline
21. या॒ज्या॒नु॒वा॒क्ये॑ भवतो भवतो याज्यानुवा॒क्ये॑ याज्यानुवा॒क्ये॑ भवतः सजा॒तैः स॑जा॒तैर् भ॑वतो याज्यानुवा॒क्ये॑ याज्यानुवा॒क्ये॑ भवतः सजा॒तैः । \newline
22. या॒ज्या॒नु॒वा॒क्ये॑ इति॑ याज्या - अ॒नु॒वा॒क्ये᳚ । \newline
23. भ॒व॒तः॒ स॒जा॒तैः स॑जा॒तैर् भ॑वतो भवतः सजा॒तै रे॒वैव स॑जा॒तैर् भ॑वतो भवतः सजा॒तैरे॒व । \newline
24. स॒जा॒तै रे॒वैव स॑जा॒तैः स॑जा॒तै रे॒वैन॑ मेन मे॒व स॑जा॒तैः स॑जा॒तै रे॒वैन᳚म् । \newline
25. स॒जा॒तैरिति॑ स - जा॒तैः । \newline
26. ए॒वैन॑ मेन मे॒वैवैन॑म् ग॒णव॑न्तम् ग॒णव॑न्त मेन मे॒वैवैन॑म् ग॒णव॑न्तम् । \newline
27. ए॒न॒म् ग॒णव॑न्तम् ग॒णव॑न्त मेन मेनम् ग॒णव॑न्तम् करोति करोति ग॒णव॑न्त मेन मेनम् ग॒णव॑न्तम् करोति । \newline
28. ग॒णव॑न्तम् करोति करोति ग॒णव॑न्तम् ग॒णव॑न्तम् करोत्ये॒ता मे॒ताम् क॑रोति ग॒णव॑न्तम् ग॒णव॑न्तम् करोत्ये॒ताम् । \newline
29. ग॒णव॑न्त॒मिति॑ ग॒ण - व॒न्त॒म् । \newline
30. क॒रो॒ त्ये॒ता मे॒ताम् क॑रोति करो त्ये॒ता मे॒वैवैताम् क॑रोति करो त्ये॒ता मे॒व । \newline
31. ए॒ता मे॒वैवैता मे॒ता मे॒व निर् णिरे॒वैता मे॒ता मे॒व निः । \newline
32. ए॒व निर् णिरे॒वैव निर् व॑पेद् वपे॒न् निरे॒वैव निर् व॑पेत् । \newline
33. निर् व॑पेद् वपे॒न् निर् णिर् व॑पे॒द् यो यो व॑पे॒न् निर् णिर् व॑पे॒द् यः । \newline
34. व॒पे॒द् यो यो व॑पेद् वपे॒द् यः का॒मये॑त का॒मये॑त॒ यो व॑पेद् वपे॒द् यः का॒मये॑त । \newline
35. यः का॒मये॑त का॒मये॑त॒ यो यः का॒मये॑त॒ ब्रह्म॒न् ब्रह्म॑न् का॒मये॑त॒ यो यः का॒मये॑त॒ ब्रह्मन्न्॑ । \newline
36. का॒मये॑त॒ ब्रह्म॒न् ब्रह्म॑न् का॒मये॑त का॒मये॑त॒ ब्रह्म॒न्॒. विशं॒ ॅविश॒म् ब्रह्म॑न् का॒मये॑त का॒मये॑त॒ ब्रह्म॒न्॒. विश᳚म् । \newline
37. ब्रह्म॒न्॒. विशं॒ ॅविश॒म् ब्रह्म॒न् ब्रह्म॒न्॒. विशं॒ ॅवि वि विश॒म् ब्रह्म॒न् ब्रह्म॒न्॒. विशं॒ ॅवि । \newline
38. विशं॒ ॅवि वि विशं॒ ॅविशं॒ ॅवि ना॑शयेयम् नाशयेयं॒ ॅवि विशं॒ ॅविशं॒ ॅवि ना॑शयेयम् । \newline
39. वि ना॑शयेयम् नाशयेयं॒ ॅवि वि ना॑शयेय॒ मितीति॑ नाशयेयं॒ ॅवि वि ना॑शयेय॒ मिति॑ । \newline
40. ना॒श॒ये॒य॒ मितीति॑ नाशयेयम् नाशयेय॒ मिति॑ मारु॒ती मा॑रु॒ती इति॑ नाशयेयम् नाशयेय॒ मिति॑ मारु॒ती । \newline
41. इति॑ मारु॒ती मा॑रु॒ती इतीति॑ मारु॒ती या᳚ज्यानुवा॒क्ये॑ याज्यानुवा॒क्ये॑ मारु॒ती इतीति॑ मारु॒ती या᳚ज्यानुवा॒क्ये᳚ । \newline
42. मा॒रु॒ती या᳚ज्यानुवा॒क्ये॑ याज्यानुवा॒क्ये॑ मारु॒ती मा॑रु॒ती या᳚ज्यानुवा॒क्ये॑ कुर्यात् कुर्याद् याज्यानुवा॒क्ये॑ मारु॒ती मा॑रु॒ती या᳚ज्यानुवा॒क्ये॑ कुर्यात् । \newline
43. मा॒रु॒ती इति॑ मारु॒ती । \newline
44. या॒ज्या॒नु॒वा॒क्ये॑ कुर्यात् कुर्याद् याज्यानुवा॒क्ये॑ याज्यानुवा॒क्ये॑ कुर्या॒द् ब्रह्म॒न् ब्रह्म॑न् कुर्याद् याज्यानुवा॒क्ये॑ याज्यानुवा॒क्ये॑ कुर्या॒द् ब्रह्मन्न्॑ । \newline
45. या॒ज्या॒नु॒वा॒क्ये॑ इति॑ याज्या - अ॒नु॒वा॒क्ये᳚ । \newline
46. कु॒र्या॒द् ब्रह्म॒न् ब्रह्म॑न् कुर्यात् कुर्या॒द् ब्रह्म॑न् ने॒वैव ब्रह्म॑न् कुर्यात् कुर्या॒द् ब्रह्म॑न् ने॒व । \newline
47. ब्रह्म॑न् ने॒वैव ब्रह्म॒न् ब्रह्म॑न् ने॒व विशं॒ ॅविश॑ मे॒व ब्रह्म॒न् ब्रह्म॑न् ने॒व विश᳚म् । \newline
48. ए॒व विशं॒ ॅविश॑ मे॒वैव विशं॒ ॅवि वि विश॑ मे॒वैव विशं॒ ॅवि । \newline
49. विशं॒ ॅवि वि विशं॒ ॅविशं॒ ॅवि ना॑शयति नाशयति॒ वि विशं॒ ॅविशं॒ ॅवि ना॑शयति । \newline
50. वि ना॑शयति नाशयति॒ वि वि ना॑शयति । \newline
51. ना॒श॒य॒तीति॑ नाशयति । \newline
\pagebreak
\markright{ TS 2.3.4.1  \hfill https://www.vedavms.in \hfill}

\section{ TS 2.3.4.1 }

\textbf{TS 2.3.4.1 } \newline
\textbf{Samhita Paata} \newline

अ॒र्य॒म्णे च॒रुं निर्व॑पथ् सुव॒र्गका॑मो॒ऽसौ वा आ॑दि॒त्यो᳚ऽर्य॒माऽर्य॒मण॑मे॒व स्वेन॑ भाग॒धेये॒नोप॑ धावति॒ स ए॒वैनꣳ॑ सुव॒र्गं ॅलो॒कं ग॑मयत्यर्य॒म्णे च॒रुं निर्व॑पे॒द्यः का॒मये॑त॒ दान॑कामा मे प्र॒जाः स्यु॒रित्य॒सौ वा आ॑दि॒त्यो᳚ऽर्य॒मा यः खलु॒ वै ददा॑ति॒ सो᳚ऽर्य॒माऽर्य॒मण॑मे॒व स्वेन॑ भाग॒धेये॒नोप॑ धावति॒ स ए॒वा - [  ] \newline

\textbf{Pada Paata} \newline

अ॒र्य॒म्णे । च॒रुम् । निरिति॑ । व॒पे॒त् । सु॒व॒र्गका॑म॒ इति॑ सुव॒र्ग-का॒मः॒ । अ॒सौ । वै । आ॒दि॒त्यः । अ॒र्य॒मा । अ॒र्य॒मण᳚म् । ए॒व । स्वेन॑ । भा॒ग॒धेये॒नेति॑ भाग - धेये॑न । उपेति॑ । धा॒व॒ति॒ । सः । ए॒व । ए॒न॒म् । सु॒व॒र्गमिति॑ सुवः - गम् । लो॒कम् । ग॒म॒य॒ति॒ । अ॒र्य॒म्णे । च॒रुम् । निरिति॑ । व॒पे॒त् । यः । का॒मये॑त । दान॑कामा॒ इति॒ दान॑-का॒माः॒ । मे॒ । प्र॒जा इति॑ प्र-जाः । स्युः॒ । इति॑ । अ॒सौ । वै । आ॒दि॒त्यः । अ॒र्य॒मा । यः । खलु॑ । वै । ददा॑ति । सः । अ॒र्य॒मा । अ॒र्य॒मण᳚म् । ए॒व । स्वेन॑ । भा॒ग॒धेये॒नेति॑ भाग - धेये॑न । उपेति॑ । धा॒व॒ति॒ । सः । ए॒व ।  \newline


\textbf{Krama Paata} \newline

अ॒र्य॒म्णे च॒रुम् । च॒रुम् निः । निर् व॑पेत् । व॒पे॒थ् सु॒व॒र्गका॑मः । सु॒व॒र्गका॑मो॒ ऽसौ । सु॒व॒र्गका॑म॒ इति॑ सुव॒र्ग - का॒मः॒ । अ॒सौ वै । वा आ॑दि॒त्यः । आ॒दि॒त्यो᳚ ऽर्य॒मा । अ॒र्य॒मा ऽर्य॒मण᳚म् । अ॒र्य॒मण॑मे॒व । ए॒व स्वेन॑ । स्वेन॑ भाग॒धेये॑न । भा॒ग॒धेये॒नोप॑ । भा॒ग॒धेये॒नेति॑ भाग - धेये॑न । उप॑ धावति । धा॒व॒ति॒ सः । स ए॒व । ए॒वैन᳚म् । ए॒नꣳ॒॒ सु॒व॒र्गम् । सु॒व॒र्गं ॅलो॒कम् । सु॒व॒र्गमिति॑ सुवः - गम् । लो॒कम् ग॑मयति । ग॒म॒य॒त्य॒र्य॒म्णे । अ॒र्य॒म्णे च॒रुम् । च॒रुम् निः । निर् व॑पेत् । व॒पे॒द् यः । यः का॒मये॑त । का॒मये॑त॒ दान॑कामाः । दान॑कामा मे । दान॑कामा॒ इति॒ दान॑ - का॒माः॒ । मे॒ प्र॒जाः । प्र॒जाः स्युः॑ । प्र॒जा इति॑ प्र - जाः । स्यु॒रिति॑ । इत्य॒सौ । अ॒सौ वै । वा आ॑दि॒त्यः । आ॒दि॒त्यो᳚ ऽर्य॒मा । अ॒र्य॒मा यः । यः खलु॑ । खलु॒ वै । वै ददा॑ति । ददा॑ति॒ सः । सो᳚ ऽर्य॒मा । अ॒र्य॒माऽर्य॒मण᳚म् । अ॒र्य॒मण॑मे॒व । ए॒व स्वेन॑ । स्वेन॑ भाग॒धेये॑न । भा॒ग॒धेये॒नोप॑ । भा॒ग॒धेये॒नेति॑ भाग - धेये॑न । उप॑ धावति । धा॒व॒ति॒ सः । स ए॒व । ए॒वास्मै᳚ \newline

\textbf{Jatai Paata} \newline

1. अ॒र्य॒म्णे च॒रुम् च॒रु म॑र्य॒म्णे᳚ ऽर्य॒म्णे च॒रुम् । \newline
2. च॒रुम् निर् णिश्च॒रुम् च॒रुम् निः । \newline
3. निर् व॑पेद् वपे॒न् निर् णिर् व॑पेत् । \newline
4. व॒पे॒थ् सु॒व॒र्गका॑मः सुव॒र्गका॑मो वपेद् वपेथ् सुव॒र्गका॑मः । \newline
5. सु॒व॒र्गका॑मो॒ ऽसा व॒सौ सु॑व॒र्गका॑मः सुव॒र्गका॑मो॒ ऽसौ । \newline
6. सु॒व॒र्गका॑म॒ इति॑ सुव॒र्ग - का॒मः॒ । \newline
7. अ॒सौ वै वा अ॒सा व॒सौ वै । \newline
8. वा आ॑दि॒त्य आ॑दि॒त्यो वै वा आ॑दि॒त्यः । \newline
9. आ॒दि॒त्यो᳚ ऽर्य॒मा ऽर्य॒मा ऽऽदि॒त्य आ॑दि॒त्यो᳚ ऽर्य॒मा । \newline
10. अ॒र्य॒मा ऽर्य॒मण॑ मर्य॒मण॑ मर्य॒मा ऽर्य॒मा ऽर्य॒मण᳚म् । \newline
11. अ॒र्य॒मण॑ मे॒वैवार्य॒मण॑ मर्य॒मण॑ मे॒व । \newline
12. ए॒व स्वेन॒ स्वेनै॒वैव स्वेन॑ । \newline
13. स्वेन॑ भाग॒धेये॑न भाग॒धेये॑न॒ स्वेन॒ स्वेन॑ भाग॒धेये॑न । \newline
14. भा॒ग॒धेये॒नोपोप॑ भाग॒धेये॑न भाग॒धेये॒नोप॑ । \newline
15. भा॒ग॒धेये॒नेति॑ भाग - धेये॑न । \newline
16. उप॑ धावति धाव॒ त्युपोप॑ धावति । \newline
17. धा॒व॒ति॒ स स धा॑वति धावति॒ सः । \newline
18. स ए॒वैव स स ए॒व । \newline
19. ए॒वैन॑ मेन मे॒वैवैन᳚म् । \newline
20. ए॒नꣳ॒॒ सु॒व॒र्गꣳ सु॑व॒र्ग मे॑न मेनꣳ सुव॒र्गम् । \newline
21. सु॒व॒र्गम् ॅलो॒कम् ॅलो॒कꣳ सु॑व॒र्गꣳ सु॑व॒र्गम् ॅलो॒कम् । \newline
22. सु॒व॒र्गमिति॑ सुवः - गम् । \newline
23. लो॒कम् ग॑मयति गमयति लो॒कम् ॅलो॒कम् ग॑मयति । \newline
24. ग॒म॒य॒ त्य॒र्य॒म्णे᳚ ऽर्य॒म्णे ग॑मयति गमय त्यर्य॒म्णे । \newline
25. अ॒र्य॒म्णे च॒रुम् च॒रु म॑र्य॒म्णे᳚ ऽर्य॒म्णे च॒रुम् । \newline
26. च॒रुम् निर् णिश्च॒रुम् च॒रुम् निः । \newline
27. निर् व॑पेद् वपे॒न् निर् णिर् व॑पेत् । \newline
28. व॒पे॒द् यो यो व॑पेद् वपे॒द् यः । \newline
29. यः का॒मये॑त का॒मये॑त॒ यो यः का॒मये॑त । \newline
30. का॒मये॑त॒ दान॑कामा॒ दान॑कामाः का॒मये॑त का॒मये॑त॒ दान॑कामाः । \newline
31. दान॑कामा मे मे॒ दान॑कामा॒ दान॑कामा मे । \newline
32. दान॑कामा॒ इति॒ दान॑ - का॒माः॒ । \newline
33. मे॒ प्र॒जाः प्र॒जा मे॑ मे प्र॒जाः । \newline
34. प्र॒जाः स्युः॑ स्युः प्र॒जाः प्र॒जाः स्युः॑ । \newline
35. प्र॒जा इति॑ प्र - जाः । \newline
36. स्यु॒ रितीति॑ स्युः स्यु॒ रिति॑ । \newline
37. इत्य॒सा व॒सा विती त्य॒सौ । \newline
38. अ॒सौ वै वा अ॒सा व॒सौ वै । \newline
39. वा आ॑दि॒त्य आ॑दि॒त्यो वै वा आ॑दि॒त्यः । \newline
40. आ॒दि॒त्यो᳚ ऽर्य॒मा ऽर्य॒मा ऽऽदि॒त्य आ॑दि॒त्यो᳚ ऽर्य॒मा । \newline
41. अ॒र्य॒मा यो यो᳚ ऽर्य॒मा ऽर्य॒मा यः । \newline
42. यः खलु॒ खलु॒ यो यः खलु॑ । \newline
43. खलु॒ वै वै खलु॒ खलु॒ वै । \newline
44. वै ददा॑ति॒ ददा॑ति॒ वै वै ददा॑ति । \newline
45. ददा॑ति॒ स स ददा॑ति॒ ददा॑ति॒ सः । \newline
46. सो᳚ ऽर्य॒मा ऽर्य॒मा स सो᳚ ऽर्य॒मा । \newline
47. अ॒र्य॒मा ऽर्य॒मण॑ मर्य॒मण॑ मर्य॒मा ऽर्य॒मा ऽर्य॒मण᳚म् । \newline
48. अ॒र्य॒मण॑ मे॒वैवार्य॒मण॑ मर्य॒मण॑ मे॒व । \newline
49. ए॒व स्वेन॒ स्वेनै॒वैव स्वेन॑ । \newline
50. स्वेन॑ भाग॒धेये॑न भाग॒धेये॑न॒ स्वेन॒ स्वेन॑ भाग॒धेये॑न । \newline
51. भा॒ग॒धेये॒नोपोप॑ भाग॒धेये॑न भाग॒धेये॒नोप॑ । \newline
52. भा॒ग॒धेये॒नेति॑ भाग - धेये॑न । \newline
53. उप॑ धावति धाव॒ त्युपोप॑ धावति । \newline
54. धा॒व॒ति॒ स स धा॑वति धावति॒ सः । \newline
55. स ए॒वैव स स ए॒व । \newline
56. ए॒वास्मा॑ अस्मा ए॒वैवास्मै᳚ । \newline

\textbf{Ghana Paata } \newline

1. अ॒र्य॒म्णे च॒रुम् च॒रु म॑र्य॒म्णे᳚ ऽर्य॒म्णे च॒रुम् निर् णिश्च॒रु म॑र्य॒म्णे᳚ ऽर्य॒म्णे च॒रुम् निः । \newline
2. च॒रुम् निर् णिश्च॒रुम् च॒रुम् निर् व॑पेद् वपे॒न् निश्च॒रुम् च॒रुम् निर् व॑पेत् । \newline
3. निर् व॑पेद् वपे॒न् निर् णिर् व॑पेथ् सुव॒र्गका॑मः सुव॒र्गका॑मो वपे॒न् निर् णिर् व॑पेथ् सुव॒र्गका॑मः । \newline
4. व॒पे॒थ् सु॒व॒र्गका॑मः सुव॒र्गका॑मो वपेद् वपेथ् सुव॒र्गका॑मो॒ ऽसा व॒सौ सु॑व॒र्गका॑मो वपेद् वपेथ् सुव॒र्गका॑मो॒ ऽसौ । \newline
5. सु॒व॒र्गका॑मो॒ ऽसा व॒सौ सु॑व॒र्गका॑मः सुव॒र्गका॑मो॒ ऽसौ वै वा अ॒सौ सु॑व॒र्गका॑मः सुव॒र्गका॑मो॒ ऽसौ वै । \newline
6. सु॒व॒र्गका॑म॒ इति॑ सुव॒र्ग - का॒मः॒ । \newline
7. अ॒सौ वै वा अ॒सा व॒सौ वा आ॑दि॒त्य आ॑दि॒त्यो वा अ॒सा व॒सौ वा आ॑दि॒त्यः । \newline
8. वा आ॑दि॒त्य आ॑दि॒त्यो वै वा आ॑दि॒त्यो᳚ ऽर्य॒मा ऽर्य॒मा ऽऽदि॒त्यो वै वा आ॑दि॒त्यो᳚ ऽर्य॒मा । \newline
9. आ॒दि॒त्यो᳚ ऽर्य॒मा ऽर्य॒मा ऽऽदि॒त्य आ॑दि॒त्यो᳚ ऽर्य॒मा ऽर्य॒मण॑ मर्य॒मण॑ मर्य॒मा ऽऽदि॒त्य आ॑दि॒त्यो᳚ ऽर्य॒मा ऽर्य॒मण᳚म् । \newline
10. अ॒र्य॒मा ऽर्य॒मण॑ मर्य॒मण॑ मर्य॒मा ऽर्य॒मा ऽर्य॒मण॑ मे॒वैवार्य॒मण॑ मर्य॒मा ऽर्य॒मा ऽर्य॒मण॑ मे॒व । \newline
11. अ॒र्य॒मण॑ मे॒वैवार्य॒मण॑ मर्य॒मण॑ मे॒व स्वेन॒ स्वेनै॒वार्य॒मण॑ मर्य॒मण॑ मे॒व स्वेन॑ । \newline
12. ए॒व स्वेन॒ स्वेनै॒वैव स्वेन॑ भाग॒धेये॑न भाग॒धेये॑न॒ स्वेनै॒वैव स्वेन॑ भाग॒धेये॑न । \newline
13. स्वेन॑ भाग॒धेये॑न भाग॒धेये॑न॒ स्वेन॒ स्वेन॑ भाग॒धेये॒नोपोप॑ भाग॒धेये॑न॒ स्वेन॒ स्वेन॑ भाग॒धेये॒नोप॑ । \newline
14. भा॒ग॒धेये॒नोपोप॑ भाग॒धेये॑न भाग॒धेये॒नोप॑ धावति धाव॒त्युप॑ भाग॒धेये॑न भाग॒धेये॒नोप॑ धावति । \newline
15. भा॒ग॒धेये॒नेति॑ भाग - धेये॑न । \newline
16. उप॑ धावति धाव॒ त्युपोप॑ धावति॒ स स धा॑व॒ त्युपोप॑ धावति॒ सः । \newline
17. धा॒व॒ति॒ स स धा॑वति धावति॒ स ए॒वैव स धा॑वति धावति॒ स ए॒व । \newline
18. स ए॒वैव स स ए॒वैन॑ मेन मे॒व स स ए॒वैन᳚म् । \newline
19. ए॒वैन॑ मेन मे॒वैवैनꣳ॑ सुव॒र्गꣳ सु॑व॒र्ग मे॑न मे॒वैवैनꣳ॑ सुव॒र्गम् । \newline
20. ए॒नꣳ॒॒ सु॒व॒र्गꣳ सु॑व॒र्ग मे॑न मेनꣳ सुव॒र्गम् ॅलो॒कम् ॅलो॒कꣳ सु॑व॒र्ग मे॑न मेनꣳ सुव॒र्गम् ॅलो॒कम् । \newline
21. सु॒व॒र्गम् ॅलो॒कम् ॅलो॒कꣳ सु॑व॒र्गꣳ सु॑व॒र्गम् ॅलो॒कम् ग॑मयति गमयति लो॒कꣳ सु॑व॒र्गꣳ सु॑व॒र्गम् ॅलो॒कम् ग॑मयति । \newline
22. सु॒व॒र्गमिति॑ सुवः - गम् । \newline
23. लो॒कम् ग॑मयति गमयति लो॒कम् ॅलो॒कम् ग॑मय त्यर्य॒म्णे᳚ ऽर्य॒म्णे ग॑मयति लो॒कम् ॅलो॒कम् ग॑मय त्यर्य॒म्णे । \newline
24. ग॒म॒य॒ त्य॒र्य॒म्णे᳚ ऽर्य॒म्णे ग॑मयति गमय त्यर्य॒म्णे च॒रुम् च॒रु म॑र्य॒म्णे ग॑मयति गमय त्यर्य॒म्णे च॒रुम् । \newline
25. अ॒र्य॒म्णे च॒रुम् च॒रु म॑र्य॒म्णे᳚ ऽर्य॒म्णे च॒रुम् निर् णिश्च॒रु म॑र्य॒म्णे᳚ ऽर्य॒म्णे च॒रुम् निः । \newline
26. च॒रुम् निर् णिश्च॒रुम् च॒रुम् निर् व॑पेद् वपे॒न् निश्च॒रुम् च॒रुम् निर् व॑पेत् । \newline
27. निर् व॑पेद् वपे॒न् निर् णिर् व॑पे॒द् यो यो व॑पे॒न् निर् णिर् व॑पे॒द् यः । \newline
28. व॒पे॒द् यो यो व॑पेद् वपे॒द् यः का॒मये॑त का॒मये॑त॒ यो व॑पेद् वपे॒द् यः का॒मये॑त । \newline
29. यः का॒मये॑त का॒मये॑त॒ यो यः का॒मये॑त॒ दान॑कामा॒ दान॑कामाः का॒मये॑त॒ यो यः का॒मये॑त॒ दान॑कामाः । \newline
30. का॒मये॑त॒ दान॑कामा॒ दान॑कामाः का॒मये॑त का॒मये॑त॒ दान॑कामा मे मे॒ दान॑कामाः का॒मये॑त का॒मये॑त॒ दान॑कामा मे । \newline
31. दान॑कामा मे मे॒ दान॑कामा॒ दान॑कामा मे प्र॒जाः प्र॒जा मे॒ दान॑कामा॒ दान॑कामा मे प्र॒जाः । \newline
32. दान॑कामा॒ इति॒ दान॑ - का॒माः॒ । \newline
33. मे॒ प्र॒जाः प्र॒जा मे॑ मे प्र॒जाः स्युः॑ स्युः प्र॒जा मे॑ मे प्र॒जाः स्युः॑ । \newline
34. प्र॒जाः स्युः॑ स्युः प्र॒जाः प्र॒जाः स्यु॒ रितीति॑ स्युः प्र॒जाः प्र॒जाः स्यु॒रिति॑ । \newline
35. प्र॒जा इति॑ प्र - जाः । \newline
36. स्यु॒रितीति॑ स्युः स्यु॒रित्य॒सा व॒सा विति॑ स्युः स्यु॒रित्य॒सौ । \newline
37. इत्य॒सा व॒सा वितीत्य॒सौ वै वा अ॒सा वितीत्य॒सौ वै । \newline
38. अ॒सौ वै वा अ॒सा व॒सौ वा आ॑दि॒त्य आ॑दि॒त्यो वा अ॒सा व॒सौ वा आ॑दि॒त्यः । \newline
39. वा आ॑दि॒त्य आ॑दि॒त्यो वै वा आ॑दि॒त्यो᳚ ऽर्य॒मा ऽर्य॒मा ऽऽदि॒त्यो वै वा आ॑दि॒त्यो᳚ ऽर्य॒मा । \newline
40. आ॒दि॒त्यो᳚ ऽर्य॒मा ऽर्य॒मा ऽऽदि॒त्य आ॑दि॒त्यो᳚ ऽर्य॒मा यो यो᳚ ऽर्य॒मा ऽऽदि॒त्य आ॑दि॒त्यो᳚ ऽर्य॒मा यः । \newline
41. अ॒र्य॒मा यो यो᳚ ऽर्य॒मा ऽर्य॒मा यः खलु॒ खलु॒ यो᳚ ऽर्य॒मा ऽर्य॒मा यः खलु॑ । \newline
42. यः खलु॒ खलु॒ यो यः खलु॒ वै वै खलु॒ यो यः खलु॒ वै । \newline
43. खलु॒ वै वै खलु॒ खलु॒ वै ददा॑ति॒ ददा॑ति॒ वै खलु॒ खलु॒ वै ददा॑ति । \newline
44. वै ददा॑ति॒ ददा॑ति॒ वै वै ददा॑ति॒ स स ददा॑ति॒ वै वै ददा॑ति॒ सः । \newline
45. ददा॑ति॒ स स ददा॑ति॒ ददा॑ति॒ सो᳚ ऽर्य॒मा ऽर्य॒मा स ददा॑ति॒ ददा॑ति॒ सो᳚ ऽर्य॒मा । \newline
46. सो᳚ ऽर्य॒मा ऽर्य॒मा स सो᳚ ऽर्य॒मा ऽर्य॒मण॑ मर्य॒मण॑ मर्य॒मा स सो᳚ ऽर्य॒मा ऽर्य॒मण᳚म् । \newline
47. अ॒र्य॒मा ऽर्य॒मण॑ मर्य॒मण॑ मर्य॒मा ऽर्य॒मा ऽर्य॒मण॑ मे॒वैवा र्य॒मण॑ मर्य॒मा ऽर्य॒मा ऽर्य॒मण॑ मे॒व । \newline
48. अ॒र्य॒मण॑ मे॒वैवार्य॒मण॑ मर्य॒मण॑ मे॒व स्वेन॒ स्वेनै॒वार्य॒मण॑ मर्य॒मण॑ मे॒व स्वेन॑ । \newline
49. ए॒व स्वेन॒ स्वेनै॒वैव स्वेन॑ भाग॒धेये॑न भाग॒धेये॑न॒ स्वेनै॒वैव स्वेन॑ भाग॒धेये॑न । \newline
50. स्वेन॑ भाग॒धेये॑न भाग॒धेये॑न॒ स्वेन॒ स्वेन॑ भाग॒धेये॒नोपोप॑ भाग॒धेये॑न॒ स्वेन॒ स्वेन॑ भाग॒धेये॒नोप॑ । \newline
51. भा॒ग॒धेये॒नोपोप॑ भाग॒धेये॑न भाग॒धेये॒नोप॑ धावति धाव॒त्युप॑ भाग॒धेये॑न भाग॒धेये॒नोप॑ धावति । \newline
52. भा॒ग॒धेये॒नेति॑ भाग - धेये॑न । \newline
53. उप॑ धावति धाव॒ त्युपोप॑ धावति॒ स स धा॑व॒ त्युपोप॑ धावति॒ सः । \newline
54. धा॒व॒ति॒ स स धा॑वति धावति॒ स ए॒वैव स धा॑वति धावति॒ स ए॒व । \newline
55. स ए॒वैव स स ए॒वास्मा॑ अस्मा ए॒व स स ए॒वास्मै᳚ । \newline
56. ए॒वास्मा॑ अस्मा ए॒वैवास्मै॒ दान॑कामा॒ दान॑कामा अस्मा ए॒वैवास्मै॒ दान॑कामाः । \newline
\pagebreak
\markright{ TS 2.3.4.2  \hfill https://www.vedavms.in \hfill}

\section{ TS 2.3.4.2 }

\textbf{TS 2.3.4.2 } \newline
\textbf{Samhita Paata} \newline

-स्मै॒ दान॑कामाः प्र॒जाः क॑रोति॒ दान॑कामा अस्मै प्र॒जा भ॑वन्त्यर्य॒म्णे च॒रुं निर्व॑पे॒द्यः का॒मये॑त स्व॒स्ति ज॒नता॑मिया॒मित्य॒सौ वा आ॑दि॒त्यो᳚-ऽर्य॒मा- ऽर्य॒मण॑मे॒व स्वेन॑ भाग॒धेये॒नोप॑ धावति॒ स ए॒वैनं॒ तद्-ग॑मयति॒ यत्र॒ जिग॑मिष॒तीन्द्रो॒ वै दे॒वाना॑मानुजाव॒र आ॑सी॒थ् स प्र॒जाप॑ति॒ -मुपा॑धाव॒त् तस्मा॑ ए॒तमै॒न्द्रमा॑नुषू॒कमेका॑दशकपालं॒ नि- [  ] \newline

\textbf{Pada Paata} \newline

अ॒स्मै॒ । दान॑कामा॒ इति॒ दान॑ - का॒माः॒ । प्र॒जा इति॑ प्र - जाः । क॒रो॒ति॒ । दान॑कामा॒ इति॒ दान॑ - का॒माः॒ । अ॒स्मै॒ । प्र॒जा इति॑ प्र - जाः । भ॒व॒न्ति॒ । अ॒र्य॒म्णे । च॒रुम् । निरिति॑ । व॒पे॒त् । यः । का॒मये॑त । स्व॒स्ति । ज॒नता᳚म् । इ॒या॒म् । इति॑ । अ॒सौ । वै । आ॒दि॒त्यः । अ॒र्य॒मा । अ॒र्य॒मण᳚म् । ए॒व । स्वेन॑ । भा॒ग॒धेये॒नेति॑ भाग-धेये॑न । उपेति॑ । धा॒व॒ति॒ । सः । ए॒व । ए॒न॒म् । तत् । ग॒म॒य॒ति॒ । यत्र॑ । जिग॑मिषति । इन्द्रः॑ । वै । दे॒वाना᳚म् । आ॒नु॒जा॒व॒र इत्या॑नु - जा॒व॒रः । आ॒सी॒त् । सः । प्र॒जाप॑ति॒मिति॑ प्र॒जा - प॒ति॒म् । उपेति॑ । अ॒धा॒व॒त् । तस्मै᳚ । ए॒तम् । ऐ॒न्द्रम् । आ॒नु॒षू॒कमित्या॑नु - सू॒कम् । एका॑दशकपाल॒मित्येका॑दश-क॒पा॒ल॒म् । निरिति॑ । 21 (50  \newline


\textbf{Krama Paata} \newline

अ॒स्मै॒ दान॑कामाः । दान॑कामाः प्र॒जाः । दान॑कामा॒ इति॒ दान॑ - का॒माः॒ । प्र॒जाः क॑रोति । प्र॒जा इति॑ प्र - जाः । क॒रो॒ति॒ दान॑कामाः । दान॑कामा अस्मै । दान॑कामा॒ इति॒ दान॑ - का॒माः॒ । अ॒स्मै॒ प्र॒जाः । प्र॒जा भ॑वन्ति । प्र॒जा इति॑ प्र - जाः । भ॒व॒न्त्य॒र्य॒म्णे । अ॒र्य॒म्णे च॒रुम् । च॒रुम् निः । निर् व॑पेत् । व॒पे॒द् यः । यः का॒मये॑त । का॒मये॑त स्व॒स्ति । स्व॒स्ति ज॒नता᳚म् । ज॒नता॑मियाम् । इ॒या॒मिति॑ । इत्य॒सौ । अ॒सौ वै । वा आ॑दि॒त्यः । आ॒दि॒त्यो᳚ऽर्य॒मा । अ॒र्य॒मा ऽर्य॒मण᳚म् । अ॒र्य॒मण॑मे॒व । ए॒व स्वेन॑ । स्वेन॑ भाग॒धेये॑न । भा॒ग॒धेये॒नोप॑ । भा॒ग॒धेये॒नेति॑ भाग - धेये॑न । उप॑ धावति । धा॒व॒ति॒ सः । स ए॒व । ए॒वैन᳚म् । ए॒न॒म् तत् । तद् ग॑मयति । ग॒म॒य॒ति॒ यत्र॑ । यत्र॒ जिग॑मिषति । जिग॑मिष॒तीन्द्रः॑ । इन्द्रो॒ वै । वै दे॒वाना᳚म् । दे॒वाना॑मानुजाव॒रः । आ॒नु॒जा॒व॒र आ॑सीत् । आ॒नु॒जा॒व॒र इत्या॑नु - जा॒व॒रः । आ॒सी॒थ् सः । स प्र॒जाप॑तिम् । प्र॒जाप॑ति॒मुप॑ । प्र॒जाप॑ति॒मिति॑ प्र॒जा - प॒ति॒म् । उपा॑धावत् । अ॒धा॒व॒त् तस्मै᳚ । तस्मा॑ ए॒तम् । ए॒तमै॒न्द्रम् । ऐ॒न्द्रमा॑नुषू॒कम् । आ॒नु॒षू॒कमेका॑दशकपालम् । आ॒नु॒षू॒कमित्या॑नु - सू॒कम् । एका॑दशकपाल॒म् निः । एका॑दशकपाल॒मित्येका॑दश - क॒पा॒ल॒म् । निर॑वपत् \newline

\textbf{Jatai Paata} \newline

1. अ॒स्मै॒ दान॑कामा॒ दान॑कामा अस्मा अस्मै॒ दान॑कामाः । \newline
2. दान॑कामाः प्र॒जाः प्र॒जा दान॑कामा॒ दान॑कामाः प्र॒जाः । \newline
3. दान॑कामा॒ इति॒ दान॑ - का॒माः॒ । \newline
4. प्र॒जाः क॑रोति करोति प्र॒जाः प्र॒जाः क॑रोति । \newline
5. प्र॒जा इति॑ प्र - जाः । \newline
6. क॒रो॒ति॒ दान॑कामा॒ दान॑कामाः करोति करोति॒ दान॑कामाः । \newline
7. दान॑कामा अस्मा अस्मै॒ दान॑कामा॒ दान॑कामा अस्मै । \newline
8. दान॑कामा॒ इति॒ दान॑ - का॒माः॒ । \newline
9. अ॒स्मै॒ प्र॒जाः प्र॒जा अ॑स्मा अस्मै प्र॒जाः । \newline
10. प्र॒जा भ॑वन्ति भवन्ति प्र॒जाः प्र॒जा भ॑वन्ति । \newline
11. प्र॒जा इति॑ प्र - जाः । \newline
12. भ॒व॒ न्त्य॒र्य॒म्णे᳚ ऽर्य॒म्णे भ॑वन्ति भव न्त्यर्य॒म्णे । \newline
13. अ॒र्य॒म्णे च॒रुम् च॒रु म॑र्य॒म्णे᳚ ऽर्य॒म्णे च॒रुम् । \newline
14. च॒रुम् निर् णिश्च॒रुम् च॒रुम् निः । \newline
15. निर् व॑पेद् वपे॒न् निर् णिर् व॑पेत् । \newline
16. व॒पे॒द् यो यो व॑पेद् वपे॒द् यः । \newline
17. यः का॒मये॑त का॒मये॑त॒ यो यः का॒मये॑त । \newline
18. का॒मये॑त स्व॒स्ति स्व॒स्ति का॒मये॑त का॒मये॑त स्व॒स्ति । \newline
19. स्व॒स्ति ज॒नता᳚म् ज॒नताꣳ॑ स्व॒स्ति स्व॒स्ति ज॒नता᳚म् । \newline
20. ज॒नता॑ मिया मियाम् ज॒नता᳚म् ज॒नता॑ मियाम् । \newline
21. इ॒या॒ मितीती॑या मिया॒ मिति॑ । \newline
22. इत्य॒सा व॒सा विती त्य॒सौ । \newline
23. अ॒सौ वै वा अ॒सा व॒सौ वै । \newline
24. वा आ॑दि॒त्य आ॑दि॒त्यो वै वा आ॑दि॒त्यः । \newline
25. आ॒दि॒त्यो᳚ ऽर्य॒मा ऽर्य॒मा ऽऽदि॒त्य आ॑दि॒त्यो᳚ ऽर्य॒मा । \newline
26. अ॒र्य॒मा ऽर्य॒मण॑ मर्य॒मण॑ मर्य॒मा ऽर्य॒मा ऽर्य॒मण᳚म् । \newline
27. अ॒र्य॒मण॑ मे॒वैवार्य॒मण॑ मर्य॒मण॑ मे॒व । \newline
28. ए॒व स्वेन॒ स्वेनै॒वैव स्वेन॑ । \newline
29. स्वेन॑ भाग॒धेये॑न भाग॒धेये॑न॒ स्वेन॒ स्वेन॑ भाग॒धेये॑न । \newline
30. भा॒ग॒धेये॒नोपोप॑ भाग॒धेये॑न भाग॒धेये॒नोप॑ । \newline
31. भा॒ग॒धेये॒नेति॑ भाग - धेये॑न । \newline
32. उप॑ धावति धाव॒ त्युपोप॑ धावति । \newline
33. धा॒व॒ति॒ स स धा॑वति धावति॒ सः । \newline
34. स ए॒वैव स स ए॒व । \newline
35. ए॒वैन॑ मेन मे॒वैवैन᳚म् । \newline
36. ए॒न॒म् तत् तदे॑न मेन॒म् तत् । \newline
37. तद् ग॑मयति गमयति॒ तत् तद् ग॑मयति । \newline
38. ग॒म॒य॒ति॒ यत्र॒ यत्र॑ गमयति गमयति॒ यत्र॑ । \newline
39. यत्र॒ जिग॑मिषति॒ जिग॑मिषति॒ यत्र॒ यत्र॒ जिग॑मिषति । \newline
40. जिग॑मिष॒तीन्द्र॒ इन्द्रो॒ जिग॑मिषति॒ जिग॑मिष॒तीन्द्रः॑ । \newline
41. इन्द्रो॒ वै वा इन्द्र॒ इन्द्रो॒ वै । \newline
42. वै दे॒वाना᳚म् दे॒वानां॒ ॅवै वै दे॒वाना᳚म् । \newline
43. दे॒वाना॑ मानुजाव॒र आ॑नुजाव॒रो दे॒वाना᳚म् दे॒वाना॑ मानुजाव॒रः । \newline
44. आ॒नु॒जा॒व॒र आ॑सी दासी दानुजाव॒र आ॑नुजाव॒र आ॑सीत् । \newline
45. आ॒नु॒जा॒व॒र इत्या॑नु - जा॒व॒रः । \newline
46. आ॒सी॒थ् स स आ॑सी दासी॒थ् सः । \newline
47. स प्र॒जाप॑तिम् प्र॒जाप॑तिꣳ॒॒ स स प्र॒जाप॑तिम् । \newline
48. प्र॒जाप॑ति॒ मुपोप॑ प्र॒जाप॑तिम् प्र॒जाप॑ति॒ मुप॑ । \newline
49. प्र॒जाप॑ति॒मिति॑ प्र॒जा - प॒ति॒म् । \newline
50. उपा॑धाव दधाव॒ दुपोपा॑धावत् । \newline
51. अ॒धा॒व॒त् तस्मै॒ तस्मा॑ अधाव दधाव॒त् तस्मै᳚ । \newline
52. तस्मा॑ ए॒त मे॒तम् तस्मै॒ तस्मा॑ ए॒तम् । \newline
53. ए॒त मै॒न्द्र मै॒न्द्र मे॒त मे॒त मै॒न्द्रम् । \newline
54. ऐ॒न्द्र मा॑नुषू॒क मा॑नुषू॒क मै॒न्द्र मै॒न्द्र मा॑नुषू॒कम् । \newline
55. आ॒नु॒षू॒क मेका॑दशकपाल॒ मेका॑दशकपाल मानुषू॒क मा॑नुषू॒क मेका॑दशकपालम् । \newline
56. आ॒नु॒षू॒कमित्या॑नु - सू॒कम् । \newline
57. एका॑दशकपाल॒म् निर् णिरेका॑दशकपाल॒ मेका॑दशकपाल॒म् निः । \newline
58. एका॑दशकपाल॒मित्येका॑दश - क॒पा॒ल॒म् । \newline
59. निर॑वप दवप॒न् निर् णिर॑वपत् । \newline

\textbf{Ghana Paata } \newline

1. अ॒स्मै॒ दान॑कामा॒ दान॑कामा अस्मा अस्मै॒ दान॑कामाः प्र॒जाः प्र॒जा दान॑कामा अस्मा अस्मै॒ दान॑कामाः प्र॒जाः । \newline
2. दान॑कामाः प्र॒जाः प्र॒जा दान॑कामा॒ दान॑कामाः प्र॒जाः क॑रोति करोति प्र॒जा दान॑कामा॒ दान॑कामाः प्र॒जाः क॑रोति । \newline
3. दान॑कामा॒ इति॒ दान॑ - का॒माः॒ । \newline
4. प्र॒जाः क॑रोति करोति प्र॒जाः प्र॒जाः क॑रोति॒ दान॑कामा॒ दान॑कामाः करोति प्र॒जाः प्र॒जाः क॑रोति॒ दान॑कामाः । \newline
5. प्र॒जा इति॑ प्र - जाः । \newline
6. क॒रो॒ति॒ दान॑कामा॒ दान॑कामाः करोति करोति॒ दान॑कामा अस्मा अस्मै॒ दान॑कामाः करोति करोति॒ दान॑कामा अस्मै । \newline
7. दान॑कामा अस्मा अस्मै॒ दान॑कामा॒ दान॑कामा अस्मै प्र॒जाः प्र॒जा अ॑स्मै॒ दान॑कामा॒ दान॑कामा अस्मै प्र॒जाः । \newline
8. दान॑कामा॒ इति॒ दान॑ - का॒माः॒ । \newline
9. अ॒स्मै॒ प्र॒जाः प्र॒जा अ॑स्मा अस्मै प्र॒जा भ॑वन्ति भवन्ति प्र॒जा अ॑स्मा अस्मै प्र॒जा भ॑वन्ति । \newline
10. प्र॒जा भ॑वन्ति भवन्ति प्र॒जाः प्र॒जा भ॑व न्त्यर्य॒म्णे᳚ ऽर्य॒म्णे भ॑वन्ति प्र॒जाः प्र॒जा भ॑व न्त्यर्य॒म्णे । \newline
11. प्र॒जा इति॑ प्र - जाः । \newline
12. भ॒व॒ न्त्य॒र्य॒म्णे᳚ ऽर्य॒म्णे भ॑वन्ति भव न्त्यर्य॒म्णे च॒रुम् च॒रु म॑र्य॒म्णे भ॑वन्ति भव न्त्यर्य॒म्णे च॒रुम् । \newline
13. अ॒र्य॒म्णे च॒रुम् च॒रु म॑र्य॒म्णे᳚ ऽर्य॒म्णे च॒रुम् निर् णिश्च॒रु म॑र्य॒म्णे᳚ ऽर्य॒म्णे च॒रुम् निः । \newline
14. च॒रुम् निर् णिश्च॒रुम् च॒रुम् निर् व॑पेद् वपे॒न् निश्च॒रुम् च॒रुम् निर् व॑पेत् । \newline
15. निर् व॑पेद् वपे॒न् निर् णिर् व॑पे॒द् यो यो व॑पे॒न् निर् णिर् व॑पे॒द् यः । \newline
16. व॒पे॒द् यो यो व॑पेद् वपे॒द् यः का॒मये॑त का॒मये॑त॒ यो व॑पेद् वपे॒द् यः का॒मये॑त । \newline
17. यः का॒मये॑त का॒मये॑त॒ यो यः का॒मये॑त स्व॒स्ति स्व॒स्ति का॒मये॑त॒ यो यः का॒मये॑त स्व॒स्ति । \newline
18. का॒मये॑त स्व॒स्ति स्व॒स्ति का॒मये॑त का॒मये॑त स्व॒स्ति ज॒नता᳚म् ज॒नताꣳ॑ स्व॒स्ति का॒मये॑त का॒मये॑त स्व॒स्ति ज॒नता᳚म् । \newline
19. स्व॒स्ति ज॒नता᳚म् ज॒नताꣳ॑ स्व॒स्ति स्व॒स्ति ज॒नता॑ मिया मियाम् ज॒नताꣳ॑ स्व॒स्ति स्व॒स्ति ज॒नता॑ मियाम् । \newline
20. ज॒नता॑ मिया मियाम् ज॒नता᳚म् ज॒नता॑ मिया॒ मितीती॑याम् ज॒नता᳚म् ज॒नता॑ मिया॒ मिति॑ । \newline
21. इ॒या॒ मितीती॑या मिया॒ मित्य॒सा व॒सा विती॑या मिया॒ मित्य॒सौ । \newline
22. इत्य॒सा व॒सा वितीत्य॒सौ वै वा अ॒सा वितीत्य॒सौ वै । \newline
23. अ॒सौ वै वा अ॒सा व॒सौ वा आ॑दि॒त्य आ॑दि॒त्यो वा अ॒सा व॒सौ वा आ॑दि॒त्यः । \newline
24. वा आ॑दि॒त्य आ॑दि॒त्यो वै वा आ॑दि॒त्यो᳚ ऽर्य॒मा ऽर्य॒मा ऽऽदि॒त्यो वै वा आ॑दि॒त्यो᳚ ऽर्य॒मा । \newline
25. आ॒दि॒त्यो᳚ ऽर्य॒मा ऽर्य॒मा ऽऽदि॒त्य आ॑दि॒त्यो᳚ ऽर्य॒मा ऽर्य॒मण॑ मर्य॒मण॑ मर्य॒मा ऽऽदि॒त्य आ॑दि॒त्यो᳚ ऽर्य॒मा ऽर्य॒मण᳚म् । \newline
26. अ॒र्य॒मा ऽर्य॒मण॑ मर्य॒मण॑ मर्य॒मा ऽर्य॒मा ऽर्य॒मण॑ मे॒वैवा र्य॒मण॑ मर्य॒मा ऽर्य॒मा ऽर्य॒मण॑ मे॒व । \newline
27. अ॒र्य॒मण॑ मे॒वैवा र्य॒मण॑ मर्य॒मण॑ मे॒व स्वेन॒ स्वेनै॒वा र्य॒मण॑ मर्य॒मण॑ मे॒व स्वेन॑ । \newline
28. ए॒व स्वेन॒ स्वेनै॒वैव स्वेन॑ भाग॒धेये॑न भाग॒धेये॑न॒ स्वेनै॒वैव स्वेन॑ भाग॒धेये॑न । \newline
29. स्वेन॑ भाग॒धेये॑न भाग॒धेये॑न॒ स्वेन॒ स्वेन॑ भाग॒धेये॒नोपोप॑ भाग॒धेये॑न॒ स्वेन॒ स्वेन॑ भाग॒धेये॒नोप॑ । \newline
30. भा॒ग॒धेये॒नोपोप॑ भाग॒धेये॑न भाग॒धेये॒नोप॑ धावति धाव॒त्युप॑ भाग॒धेये॑न भाग॒धेये॒नोप॑ धावति । \newline
31. भा॒ग॒धेये॒नेति॑ भाग - धेये॑न । \newline
32. उप॑ धावति धाव॒ त्युपोप॑ धावति॒ स स धा॑व॒ त्युपोप॑ धावति॒ सः । \newline
33. धा॒व॒ति॒ स स धा॑वति धावति॒ स ए॒वैव स धा॑वति धावति॒ स ए॒व । \newline
34. स ए॒वैव स स ए॒वैन॑ मेन मे॒व स स ए॒वैन᳚म् । \newline
35. ए॒वैन॑ मेन मे॒वैवैन॒म् तत् तदे॑न मे॒वैवैन॒म् तत् । \newline
36. ए॒न॒म् तत् तदे॑न मेन॒म् तद् ग॑मयति गमयति॒ तदे॑न मेन॒म् तद् ग॑मयति । \newline
37. तद् ग॑मयति गमयति॒ तत् तद् ग॑मयति॒ यत्र॒ यत्र॑ गमयति॒ तत् तद् ग॑मयति॒ यत्र॑ । \newline
38. ग॒म॒य॒ति॒ यत्र॒ यत्र॑ गमयति गमयति॒ यत्र॒ जिग॑मिषति॒ जिग॑मिषति॒ यत्र॑ गमयति गमयति॒ यत्र॒ जिग॑मिषति । \newline
39. यत्र॒ जिग॑मिषति॒ जिग॑मिषति॒ यत्र॒ यत्र॒ जिग॑मिष॒तीन्द्र॒ इन्द्रो॒ जिग॑मिषति॒ यत्र॒ यत्र॒ जिग॑मिष॒तीन्द्रः॑ । \newline
40. जिग॑मिष॒तीन्द्र॒ इन्द्रो॒ जिग॑मिषति॒ जिग॑मिष॒तीन्द्रो॒ वै वा इन्द्रो॒ जिग॑मिषति॒ जिग॑मिष॒तीन्द्रो॒ वै । \newline
41. इन्द्रो॒ वै वा इन्द्र॒ इन्द्रो॒ वै दे॒वाना᳚म् दे॒वानां॒ ॅवा इन्द्र॒ इन्द्रो॒ वै दे॒वाना᳚म् । \newline
42. वै दे॒वाना᳚म् दे॒वानां॒ ॅवै वै दे॒वाना॑ मानुजाव॒र आ॑नुजाव॒रो दे॒वानां॒ ॅवै वै दे॒वाना॑ मानुजाव॒रः । \newline
43. दे॒वाना॑ मानुजाव॒र आ॑नुजाव॒रो दे॒वाना᳚म् दे॒वाना॑ मानुजाव॒र आ॑सी दासी दानुजाव॒रो दे॒वाना᳚म् दे॒वाना॑ मानुजाव॒र आ॑सीत् । \newline
44. आ॒नु॒जा॒व॒र आ॑सीदासी दानुजाव॒र आ॑नुजाव॒र आ॑सी॒थ् स स आ॑सी दानुजाव॒र आ॑नुजाव॒र आ॑सी॒थ् सः । \newline
45. आ॒नु॒जा॒व॒र इत्या॑नु - जा॒व॒रः । \newline
46. आ॒सी॒थ् स स आ॑सी दासी॒थ् स प्र॒जाप॑तिम् प्र॒जाप॑तिꣳ॒॒ स आ॑सी दासी॒थ् स प्र॒जाप॑तिम् । \newline
47. स प्र॒जाप॑तिम् प्र॒जाप॑तिꣳ॒॒ स स प्र॒जाप॑ति॒ मुपोप॑ प्र॒जाप॑तिꣳ॒॒ स स प्र॒जाप॑ति॒ मुप॑ । \newline
48. प्र॒जाप॑ति॒ मुपोप॑ प्र॒जाप॑तिम् प्र॒जाप॑ति॒ मुपा॑धाव दधाव॒दुप॑ प्र॒जाप॑तिम् प्र॒जाप॑ति॒ मुपा॑धावत् । \newline
49. प्र॒जाप॑ति॒मिति॑ प्र॒जा - प॒ति॒म् । \newline
50. उपा॑धाव दधाव॒ दुपोपा॑धाव॒त् तस्मै॒ तस्मा॑ अधाव॒ दुपोपा॑धाव॒त् तस्मै᳚ । \newline
51. अ॒धा॒व॒त् तस्मै॒ तस्मा॑ अधाव दधाव॒त् तस्मा॑ ए॒त मे॒तम् तस्मा॑ अधाव दधाव॒त् तस्मा॑ ए॒तम् । \newline
52. तस्मा॑ ए॒त मे॒तम् तस्मै॒ तस्मा॑ ए॒त मै॒न्द्र मै॒न्द्र मे॒तम् तस्मै॒ तस्मा॑ ए॒त मै॒न्द्रम् । \newline
53. ए॒त मै॒न्द्र मै॒न्द्र मे॒त मे॒त मै॒न्द्र मा॑नुषू॒क मा॑नुषू॒क मै॒न्द्र मे॒त मे॒त मै॒न्द्र मा॑नुषू॒कम् । \newline
54. ऐ॒न्द्र मा॑नुषू॒क मा॑नुषू॒क मै॒न्द्र मै॒न्द्र मा॑नुषू॒क मेका॑दशकपाल॒ मेका॑दशकपाल मानुषू॒क मै॒न्द्र मै॒न्द्र मा॑नुषू॒क मेका॑दशकपालम् । \newline
55. आ॒नु॒षू॒क मेका॑दशकपाल॒ मेका॑दशकपाल मानुषू॒क मा॑नुषू॒क मेका॑दशकपाल॒म् निर् णिरेका॑दशकपाल मानुषू॒क मा॑नुषू॒क मेका॑दशकपाल॒म् निः । \newline
56. आ॒नु॒षू॒कमित्या॑नु - सू॒कम् । \newline
57. एका॑दशकपाल॒म् निर् णिरेका॑दशकपाल॒ मेका॑दशकपाल॒म् निर॑वप दवप॒न् निरेका॑दशकपाल॒ मेका॑दशकपाल॒म् निर॑वपत् । \newline
58. एका॑दशकपाल॒मित्येका॑दश - क॒पा॒ल॒म् । \newline
59. निर॑वप दवप॒न् निर् णिर॑वप॒त् तेन॒ तेना॑वप॒न् निर् णिर॑वप॒त् तेन॑ । \newline
\pagebreak
\markright{ TS 2.3.4.3  \hfill https://www.vedavms.in \hfill}

\section{ TS 2.3.4.3 }

\textbf{TS 2.3.4.3 } \newline
\textbf{Samhita Paata} \newline

-र॑वप॒त् तेनै॒वैन॒मग्रं॑ दे॒वता॑नां॒ पर्य॑णयद् बु॒द्ध्नव॑ती॒ अग्र॑वती याज्यानुवा॒क्ये॑ अकरोद् बु॒द्ध्ना-दे॒वैन॒मग्रं॒ पर्य॑णय॒द्यो रा॑ज॒न्य॑ आनुजाव॒रः स्यात् तस्मा॑ ए॒तमै॒न्द्रमा॑नुषू॒क-मेका॑दशकपालं॒ निर्व॑पे॒दिन्द्र॑मे॒व स्वेन॑ भाग॒धेये॒नोप॑ धावति॒ स ए॒वैन॒मग्रꣳ॑ समा॒नानां॒ परि॑णयति बु॒द्ध्नव॑ती॒ अग्र॑वती याज्यानुवा॒क्ये॑ भवतो बु॒द्ध्नादे॒वैन॒मग्रं॒ [  ] \newline

\textbf{Pada Paata} \newline

अ॒व॒प॒त् । तेन॑ । ए॒व । ए॒न॒म् । अग्र᳚म् । दे॒वता॑नाम् । परीति॑ । अ॒न॒य॒त् । बु॒द्ध्नव॑ती॒ इति॑ बु॒द्ध्न - व॒ती॒ । अग्र॑वती॒ इत्यग्र॑ - व॒ती॒ । या॒ज्या॒नु॒वा॒क्ये॑ इति॑ याज्या-अ॒नु॒वा॒क्ये᳚ । अ॒क॒रो॒त् । बु॒द्ध्नात् । ए॒व । ए॒न॒म् । अग्र᳚म् । परीति॑ । अ॒न॒य॒त् । यः । रा॒ज॒न्यः॑ । आ॒नु॒जा॒व॒र इत्या॑नु - जा॒व॒रः । स्यात् । तस्मै᳚ । ए॒तम् । ऐ॒न्द्रम् । आ॒नु॒षू॒कमित्या॑नु - सू॒कम् । एका॑दशकपाल॒मित्येका॑दश-क॒पा॒ल॒म् । निरिति॑ । व॒पे॒त् । इन्द्र᳚म् । ए॒व । स्वेन॑ । भा॒ग॒धेये॒नेति॑ भाग-धेये॑न । उपेति॑ । धा॒व॒ति॒ । सः । ए॒व । ए॒न॒म् । अग्र᳚म् । स॒मा॒नाना᳚म् । परीति॑ । न॒य॒ति॒ । बु॒द्ध्नव॑ती॒ इति॑ बु॒द्ध्न - व॒ती॒ । अग्र॑वती॒ इत्यग्र॑ - व॒ती॒ । या॒ज्या॒नु॒वा॒क्ये॑ इति॑ याज्या - अ॒नु॒वा॒क्ये᳚ । भ॒व॒तः॒ । बु॒द्ध्नात् । ए॒व । ए॒न॒म् । अग्र᳚म् ।  \newline


\textbf{Krama Paata} \newline

अ॒व॒प॒त् तेन॑ । तेनै॒व । ए॒वैन᳚म् । ए॒न॒मग्र᳚म् । अग्र॑म् दे॒वता॑नाम् । दे॒वता॑ना॒म् परि॑ । पर्य॑णयत् । अ॒न॒य॒द् बु॒द्ध्नव॑ती । बु॒द्ध्नव॑ती॒ अग्र॑वती । बु॒द्ध्नव॑ती॒ इति॑ बु॒द्ध्न - व॒ती॒ । अग्र॑वती याज्यानुवा॒क्ये᳚ । अग्र॑वती॒ इत्यग्र॑ - व॒ती॒ । या॒ज्या॒नु॒वा॒क्ये॑ अकरोत् । या॒ज्या॒नु॒वा॒क्ये॑ इति॑ याज्या - अ॒नु॒वा॒क्ये᳚ । अ॒क॒रो॒द् बु॒द्ध्नात् । बु॒द्ध्नादे॒व । ए॒वैन᳚म् । ए॒न॒मग्र᳚म् । अग्र॒म् परि॑ । पर्य॑णयत् । अ॒न॒य॒द् यः । यो रा॑ज॒न्यः॑ । रा॒ज॒न्य॑ आनुजाव॒रः । आ॒नु॒जा॒व॒रः स्यात् । आ॒नु॒जा॒व॒र इत्या॑नु - जा॒व॒रः । स्यात् तस्मै᳚ । तस्मा॑ ए॒तम् । ए॒तमै॒न्द्रम् । ऐ॒न्द्रमा॑नुषू॒कम् । आ॒नु॒षू॒क,मेका॑दशकपालम् । आ॒नु॒षू॒कमित्या॑नु - सू॒कम् । एका॑दशकापाल॒म् निः । एका॑दशकपाल॒मित्येका॑दश - क॒पा॒ल॒म् । निर् व॑पेत् । व॒पे॒दिन्द्र᳚म् । इन्द्र॑मे॒व । ए॒व स्वेन॑ । स्वेन॑ भाग॒धेये॑न । भा॒ग॒धेये॒नोप॑ । भा॒ग॒धेये॒नेति॑ भाग - धेये॑न । उप॑ धावति । धा॒व॒ति॒ सः । स ए॒व । ए॒वैन᳚म् । ए॒न॒मग्र᳚म् । अग्रꣳ॑ समा॒नाना᳚म् । स॒मा॒नाना॒म् परि॑ । परि॑णयति । न॒य॒ति॒ बु॒द्ध्नव॑ती । बु॒द्ध्नव॑ती॒ अग्र॑वती । बु॒द्ध्नव॑ती॒ इति॑ बु॒द्ध्न - व॒ती॒ । अग्र॑वती याज्यानुवा॒क्ये᳚ । अग्र॑वती॒ इत्यग्र॑ - व॒ती॒ । या॒ज्या॒नु॒वा॒क्ये॑ भवतः । या॒ज्या॒नु॒वा॒क्ये॑ इति॑ याज्या - अ॒नु॒वा॒क्ये᳚ । भ॒व॒तो॒ बु॒द्ध्नात् । बु॒द्ध्नादे॒व । ए॒वैन᳚म् । ए॒न॒मग्र᳚म् । अग्र॒म् परि॑ \newline

\textbf{Jatai Paata} \newline

1. अ॒व॒प॒त् तेन॒ तेना॑वप दवप॒त् तेन॑ । \newline
2. तेनै॒वैव तेन॒ तेनै॒व । \newline
3. ए॒वैन॑ मेन मे॒वैवैन᳚म् । \newline
4. ए॒न॒ मग्र॒ मग्र॑ मेन मेन॒ मग्र᳚म् । \newline
5. अग्र॑म् दे॒वता॑नाम् दे॒वता॑ना॒ मग्र॒ मग्र॑म् दे॒वता॑नाम् । \newline
6. दे॒वता॑ना॒म् परि॒ परि॑ दे॒वता॑नाम् दे॒वता॑ना॒म् परि॑ । \newline
7. पर्य॑णय दनय॒त् परि॒ पर्य॑णयत् । \newline
8. अ॒न॒य॒द् बु॒द्ध्नव॑ती बु॒द्ध्नव॑ती अनय दनयद् बु॒द्ध्नव॑ती । \newline
9. बु॒द्ध्नव॑ती॒ अग्र॑वती॒ अग्र॑वती बु॒द्ध्नव॑ती बु॒द्ध्नव॑ती॒ अग्र॑वती । \newline
10. बु॒द्ध्नव॑ती॒ इति॑ बु॒द्ध्न - व॒ती॒ । \newline
11. अग्र॑वती याज्यानुवा॒क्ये॑ याज्यानुवा॒क्ये॑ अग्र॑वती॒ अग्र॑वती याज्यानुवा॒क्ये᳚ । \newline
12. अग्र॑वती॒ इत्यग्र॑ - व॒ती॒ । \newline
13. या॒ज्या॒नु॒वा॒क्ये॑ अकरोदकरोद् याज्यानुवा॒क्ये॑ याज्यानुवा॒क्ये॑ अकरोत् । \newline
14. या॒ज्या॒नु॒वा॒क्ये॑ इति॑ याज्या - अ॒नु॒वा॒क्ये᳚ । \newline
15. अ॒क॒रो॒द् बु॒द्ध्नाद् बु॒द्ध्ना द॑करो दकरोद् बु॒द्ध्नात् । \newline
16. बु॒द्ध्ना दे॒वैव बु॒द्ध्नाद् बु॒द्ध्ना दे॒व । \newline
17. ए॒वैन॑ मेन मे॒वैवैन᳚म् । \newline
18. ए॒न॒ मग्र॒ मग्र॑ मेन मेन॒ मग्र᳚म् । \newline
19. अग्र॒म् परि॒ पर्यग्र॒ मग्र॒म् परि॑ । \newline
20. पर्य॑णय दनय॒त् परि॒ पर्य॑णयत् । \newline
21. अ॒न॒य॒द् यो यो॑ ऽनय दनय॒द् यः । \newline
22. यो रा॑ज॒न्यो॑ राज॒न्यो॑ यो यो रा॑ज॒न्यः॑ । \newline
23. रा॒ज॒न्य॑ आनुजाव॒र आ॑नुजाव॒रो रा॑ज॒न्यो॑ राज॒न्य॑ आनुजाव॒रः । \newline
24. आ॒नु॒जा॒व॒रः स्याथ् स्यादा॑नुजाव॒र आ॑नुजाव॒रः स्यात् । \newline
25. आ॒नु॒जा॒व॒र इत्या॑नु - जा॒व॒रः । \newline
26. स्यात् तस्मै॒ तस्मै॒ स्याथ् स्यात् तस्मै᳚ । \newline
27. तस्मा॑ ए॒त मे॒तम् तस्मै॒ तस्मा॑ ए॒तम् । \newline
28. ए॒त मै॒न्द्र मै॒न्द्र मे॒त मे॒त मै॒न्द्रम् । \newline
29. ऐ॒न्द्र मा॑नुषू॒क मा॑नुषू॒क मै॒न्द्र मै॒न्द्र मा॑नुषू॒कम् । \newline
30. आ॒नु॒षू॒क मेका॑दशकपाल॒ मेका॑दशकपाल मानुषू॒क मा॑नुषू॒क मेका॑दशकपालम् । \newline
31. आ॒नु॒षू॒कमित्या॑नु - सू॒कम् । \newline
32. एका॑दशकपाल॒म् निर् णिरेका॑दशकपाल॒ मेका॑दशकपाल॒म् निः । \newline
33. एका॑दशकपाल॒मित्येका॑दश - क॒पा॒ल॒म् । \newline
34. निर् व॑पेद् वपे॒न् निर् णिर् व॑पेत् । \newline
35. व॒पे॒ दिन्द्र॒ मिन्द्रं॑ ॅवपेद् वपे॒ दिन्द्र᳚म् । \newline
36. इन्द्र॑ मे॒वैवे न्द्र॒ मिन्द्र॑ मे॒व । \newline
37. ए॒व स्वेन॒ स्वेनै॒वैव स्वेन॑ । \newline
38. स्वेन॑ भाग॒धेये॑न भाग॒धेये॑न॒ स्वेन॒ स्वेन॑ भाग॒धेये॑न । \newline
39. भा॒ग॒धेये॒नोपोप॑ भाग॒धेये॑न भाग॒धेये॒नोप॑ । \newline
40. भा॒ग॒धेये॒नेति॑ भाग - धेये॑न । \newline
41. उप॑ धावति धाव॒ त्युपोप॑ धावति । \newline
42. धा॒व॒ति॒ स स धा॑वति धावति॒ सः । \newline
43. स ए॒वैव स स ए॒व । \newline
44. ए॒वैन॑ मेन मे॒वैवैन᳚म् । \newline
45. ए॒न॒ मग्र॒ मग्र॑ मेन मेन॒ मग्र᳚म् । \newline
46. अग्रꣳ॑ समा॒नानाꣳ॑ समा॒नाना॒ मग्र॒ मग्रꣳ॑ समा॒नाना᳚म् । \newline
47. स॒मा॒नाना॒म् परि॒ परि॑ समा॒नानाꣳ॑ समा॒नाना॒म् परि॑ । \newline
48. परि॑ णयति नयति॒ परि॒ परि॑ णयति । \newline
49. न॒य॒ति॒ बु॒द्ध्नव॑ती बु॒द्ध्नव॑ती नयति नयति बु॒द्ध्नव॑ती । \newline
50. बु॒द्ध्नव॑ती॒ अग्र॑वती॒ अग्र॑वती बु॒द्ध्नव॑ती बु॒द्ध्नव॑ती॒ अग्र॑वती । \newline
51. बु॒द्ध्नव॑ती॒ इति॑ बु॒द्ध्न - व॒ती॒ । \newline
52. अग्र॑वती याज्यानुवा॒क्ये॑ याज्यानुवा॒क्ये॑ अग्र॑वती॒ अग्र॑वती याज्यानुवा॒क्ये᳚ । \newline
53. अग्र॑वती॒ इत्यग्र॑ - व॒ती॒ । \newline
54. या॒ज्या॒नु॒वा॒क्ये॑ भवतो भवतो याज्यानुवा॒क्ये॑ याज्यानुवा॒क्ये॑ भवतः । \newline
55. या॒ज्या॒नु॒वा॒क्ये॑ इति॑ याज्या - अ॒नु॒वा॒क्ये᳚ । \newline
56. भ॒व॒तो॒ बु॒द्ध्नाद् बु॒द्ध्नाद् भ॑वतो भवतो बु॒द्ध्नात् । \newline
57. बु॒द्ध्ना दे॒वैव बु॒द्ध्नाद् बु॒द्ध्ना दे॒व । \newline
58. ए॒वैन॑ मेन मे॒वैवैन᳚म् । \newline
59. ए॒न॒ मग्र॒ मग्र॑ मेन मेन॒ मग्र᳚म् । \newline
60. अग्र॒म् परि॒ पर्यग्र॒ मग्र॒म् परि॑ । \newline

\textbf{Ghana Paata } \newline

1. अ॒व॒प॒त् तेन॒ तेना॑वप दवप॒त् तेनै॒वैव तेना॑वप दवप॒त् तेनै॒व । \newline
2. तेनै॒वैव तेन॒ तेनै॒वैन॑ मेन मे॒व तेन॒ तेनै॒वैन᳚म् । \newline
3. ए॒वैन॑ मेन मे॒वैवैन॒ मग्र॒ मग्र॑ मेन मे॒वैवैन॒ मग्र᳚म् । \newline
4. ए॒न॒ मग्र॒ मग्र॑ मेन मेन॒ मग्र॑म् दे॒वता॑नाम् दे॒वता॑ना॒ मग्र॑ मेन मेन॒ मग्र॑म् दे॒वता॑नाम् । \newline
5. अग्र॑म् दे॒वता॑नाम् दे॒वता॑ना॒ मग्र॒ मग्र॑म् दे॒वता॑ना॒म् परि॒ परि॑ दे॒वता॑ना॒ मग्र॒ मग्र॑म् दे॒वता॑ना॒म् परि॑ । \newline
6. दे॒वता॑ना॒म् परि॒ परि॑ दे॒वता॑नाम् दे॒वता॑ना॒म् पर्य॑णय दनय॒त् परि॑ दे॒वता॑नाम् दे॒वता॑ना॒म् पर्य॑णयत् । \newline
7. पर्य॑णय दनय॒त् परि॒ पर्य॑णयद् बु॒द्ध्नव॑ती बु॒द्ध्नव॑ती अनय॒त् परि॒ पर्य॑णयद् बु॒द्ध्नव॑ती । \newline
8. अ॒न॒य॒द् बु॒द्ध्नव॑ती बु॒द्ध्नव॑ती अनय दनयद् बु॒द्ध्नव॑ती॒ अग्र॑वती॒ अग्र॑वती बु॒द्ध्नव॑ती अनय दनयद् बु॒द्ध्नव॑ती॒ अग्र॑वती । \newline
9. बु॒द्ध्नव॑ती॒ अग्र॑वती॒ अग्र॑वती बु॒द्ध्नव॑ती बु॒द्ध्नव॑ती॒ अग्र॑वती याज्यानुवा॒क्ये॑ याज्यानुवा॒क्ये॑ अग्र॑वती बु॒द्ध्नव॑ती बु॒द्ध्नव॑ती॒ अग्र॑वती याज्यानुवा॒क्ये᳚ । \newline
10. बु॒द्ध्नव॑ती॒ इति॑ बु॒द्ध्न - व॒ती॒ । \newline
11. अग्र॑वती याज्यानुवा॒क्ये॑ याज्यानुवा॒क्ये॑ अग्र॑वती॒ अग्र॑वती याज्यानुवा॒क्ये॑ अकरो दकरोद् याज्यानुवा॒क्ये॑ अग्र॑वती॒ अग्र॑वती याज्यानुवा॒क्ये॑ अकरोत् । \newline
12. अग्र॑वती॒ इत्यग्र॑ - व॒ती॒ । \newline
13. या॒ज्या॒नु॒वा॒क्ये॑ अकरो दकरोद् याज्यानुवा॒क्ये॑ याज्यानुवा॒क्ये॑ अकरोद् बु॒द्ध्नाद् बु॒द्ध्ना द॑करोद् याज्यानुवा॒क्ये॑ याज्यानुवा॒क्ये॑ अकरोद् बु॒द्ध्नात् । \newline
14. या॒ज्या॒नु॒वा॒क्ये॑ इति॑ याज्या - अ॒नु॒वा॒क्ये᳚ । \newline
15. अ॒क॒रो॒द् बु॒द्ध्नाद् बु॒द्ध्ना द॑करो दकरोद् बु॒द्ध्ना दे॒वैव बु॒द्ध्ना द॑करो दकरोद् बु॒द्ध्नादे॒व । \newline
16. बु॒द्ध्ना दे॒वैव बु॒द्ध्नाद् बु॒द्ध्ना दे॒वैन॑ मेन मे॒व बु॒द्ध्नाद् बु॒द्ध्ना दे॒वैन᳚म् । \newline
17. ए॒वैन॑ मेन मे॒वैवैन॒ मग्र॒ मग्र॑ मेन मे॒वैवैन॒ मग्र᳚म् । \newline
18. ए॒न॒ मग्र॒ मग्र॑ मेन मेन॒ मग्र॒म् परि॒ पर्यग्र॑ मेन मेन॒ मग्र॒म् परि॑ । \newline
19. अग्र॒म् परि॒ पर्यग्र॒ मग्र॒म् पर्य॑णय दनय॒त् पर्यग्र॒ मग्र॒म् पर्य॑णयत् । \newline
20. पर्य॑णय दनय॒त् परि॒ पर्य॑णय॒द् यो यो॑ ऽनय॒त् परि॒ पर्य॑णय॒द् यः । \newline
21. अ॒न॒य॒द् यो यो॑ ऽनय दनय॒द् यो रा॑ज॒न्यो॑ राज॒न्यो᳚(1॒) यो॑ ऽनय दनय॒द् यो रा॑ज॒न्यः॑ । \newline
22. यो रा॑ज॒न्यो॑ राज॒न्यो॑ यो यो रा॑ज॒न्य॑ आनुजाव॒र आ॑नुजाव॒रो रा॑ज॒न्यो॑ यो यो रा॑ज॒न्य॑ आनुजाव॒रः । \newline
23. रा॒ज॒न्य॑ आनुजाव॒र आ॑नुजाव॒रो रा॑ज॒न्यो॑ राज॒न्य॑ आनुजाव॒रः स्याथ् स्यादा॑नुजाव॒रो रा॑ज॒न्यो॑ राज॒न्य॑ आनुजाव॒रः स्यात् । \newline
24. आ॒नु॒जा॒व॒रः स्याथ् स्यादा॑नुजाव॒र आ॑नुजाव॒रः स्यात् तस्मै॒ तस्मै॒ स्यादा॑नुजाव॒र आ॑नुजाव॒रः स्यात् तस्मै᳚ । \newline
25. आ॒नु॒जा॒व॒र इत्या॑नु - जा॒व॒रः । \newline
26. स्यात् तस्मै॒ तस्मै॒ स्याथ् स्यात् तस्मा॑ ए॒त मे॒तम् तस्मै॒ स्याथ् स्यात् तस्मा॑ ए॒तम् । \newline
27. तस्मा॑ ए॒त मे॒तम् तस्मै॒ तस्मा॑ ए॒त मै॒न्द्र मै॒न्द्र मे॒तम् तस्मै॒ तस्मा॑ ए॒त मै॒न्द्रम् । \newline
28. ए॒त मै॒न्द्र मै॒न्द्र मे॒त मे॒त मै॒न्द्र मा॑नुषू॒क मा॑नुषू॒क मै॒न्द्र मे॒त मे॒त मै॒न्द्र मा॑नुषू॒कम् । \newline
29. ऐ॒न्द्र मा॑नुषू॒क मा॑नुषू॒क मै॒न्द्र मै॒न्द्र मा॑नुषू॒क मेका॑दशकपाल॒ मेका॑दशकपाल मानुषू॒क मै॒न्द्र मै॒न्द्र मा॑नुषू॒क मेका॑दशकपालम् । \newline
30. आ॒नु॒षू॒क मेका॑दशकपाल॒ मेका॑दशकपाल मानुषू॒क मा॑नुषू॒क मेका॑दशकपाल॒म् निर् णिरेका॑दशकपाल मानुषू॒क मा॑नुषू॒क मेका॑दशकपाल॒म् निः । \newline
31. आ॒नु॒षू॒कमित्या॑नु - सू॒कम् । \newline
32. एका॑दशकपाल॒म् निर् णिरेका॑दशकपाल॒ मेका॑दशकपाल॒म् निर् व॑पेद् वपे॒न् निरेका॑दशकपाल॒ मेका॑दशकपाल॒म् निर् व॑पेत् । \newline
33. एका॑दशकपाल॒मित्येका॑दश - क॒पा॒ल॒म् । \newline
34. निर् व॑पेद् वपे॒न् निर् णिर् व॑पे॒दिन्द्र॒ मिन्द्रं॑ ॅवपे॒न् निर् णिर् व॑पे॒दिन्द्र᳚म् । \newline
35. व॒पे॒दिन्द्र॒ मिन्द्रं॑ ॅवपेद् वपे॒दिन्द्र॑ मे॒वैवे न्द्रं॑ ॅवपेद् वपे॒दिन्द्र॑ मे॒व । \newline
36. इन्द्र॑ मे॒वैवे न्द्र॒ मिन्द्र॑ मे॒व स्वेन॒ स्वेनै॒वे न्द्र॒ मिन्द्र॑ मे॒व स्वेन॑ । \newline
37. ए॒व स्वेन॒ स्वेनै॒वैव स्वेन॑ भाग॒धेये॑न भाग॒धेये॑न॒ स्वेनै॒वैव स्वेन॑ भाग॒धेये॑न । \newline
38. स्वेन॑ भाग॒धेये॑न भाग॒धेये॑न॒ स्वेन॒ स्वेन॑ भाग॒धेये॒नोपोप॑ भाग॒धेये॑न॒ स्वेन॒ स्वेन॑ भाग॒धेये॒नोप॑ । \newline
39. भा॒ग॒धेये॒नोपोप॑ भाग॒धेये॑न भाग॒धेये॒नोप॑ धावति धाव॒त्युप॑ भाग॒धेये॑न भाग॒धेये॒नोप॑ धावति । \newline
40. भा॒ग॒धेये॒नेति॑ भाग - धेये॑न । \newline
41. उप॑ धावति धाव॒ त्युपोप॑ धावति॒ स स धा॑व॒ त्युपोप॑ धावति॒ सः । \newline
42. धा॒व॒ति॒ स स धा॑वति धावति॒ स ए॒वैव स धा॑वति धावति॒ स ए॒व । \newline
43. स ए॒वैव स स ए॒वैन॑ मेन मे॒व स स ए॒वैन᳚म् । \newline
44. ए॒वैन॑ मेन मे॒वैवैन॒ मग्र॒ मग्र॑ मेन मे॒वैवैन॒ मग्र᳚म् । \newline
45. ए॒न॒ मग्र॒ मग्र॑ मेन मेन॒ मग्रꣳ॑ समा॒नानाꣳ॑ समा॒नाना॒ मग्र॑ मेन मेन॒ मग्रꣳ॑ समा॒नाना᳚म् । \newline
46. अग्रꣳ॑ समा॒नानाꣳ॑ समा॒नाना॒ मग्र॒ मग्रꣳ॑ समा॒नाना॒म् परि॒ परि॑ समा॒नाना॒ मग्र॒ मग्रꣳ॑ समा॒नाना॒म् परि॑ । \newline
47. स॒मा॒नाना॒म् परि॒ परि॑ समा॒नानाꣳ॑ समा॒नाना॒म् परि॑ णयति नयति॒ परि॑ समा॒नानाꣳ॑ समा॒नाना॒म् परि॑ णयति । \newline
48. परि॑ णयति नयति॒ परि॒ परि॑ णयति बु॒द्ध्नव॑ती बु॒द्ध्नव॑ती नयति॒ परि॒ परि॑ णयति बु॒द्ध्नव॑ती । \newline
49. न॒य॒ति॒ बु॒द्ध्नव॑ती बु॒द्ध्नव॑ती नयति नयति बु॒द्ध्नव॑ती॒ अग्र॑वती॒ अग्र॑वती बु॒द्ध्नव॑ती नयति नयति बु॒द्ध्नव॑ती॒ अग्र॑वती । \newline
50. बु॒द्ध्नव॑ती॒ अग्र॑वती॒ अग्र॑वती बु॒द्ध्नव॑ती बु॒द्ध्नव॑ती॒ अग्र॑वती याज्यानुवा॒क्ये॑ याज्यानुवा॒क्ये॑ अग्र॑वती बु॒द्ध्नव॑ती बु॒द्ध्नव॑ती॒ अग्र॑वती याज्यानुवा॒क्ये᳚ । \newline
51. बु॒द्ध्नव॑ती॒ इति॑ बु॒द्ध्न - व॒ती॒ । \newline
52. अग्र॑वती याज्यानुवा॒क्ये॑ याज्यानुवा॒क्ये॑ अग्र॑वती॒ अग्र॑वती याज्यानुवा॒क्ये॑ भवतो भवतो याज्यानुवा॒क्ये॑ अग्र॑वती॒ अग्र॑वती याज्यानुवा॒क्ये॑ भवतः । \newline
53. अग्र॑वती॒ इत्यग्र॑ - व॒ती॒ । \newline
54. या॒ज्या॒नु॒वा॒क्ये॑ भवतो भवतो याज्यानुवा॒क्ये॑ याज्यानुवा॒क्ये॑ भवतो बु॒द्ध्नाद् बु॒द्ध्नाद् भ॑वतो याज्यानुवा॒क्ये॑ याज्यानुवा॒क्ये॑ भवतो बु॒द्ध्नात् । \newline
55. या॒ज्या॒नु॒वा॒क्ये॑ इति॑ याज्या - अ॒नु॒वा॒क्ये᳚ । \newline
56. भ॒व॒तो॒ बु॒द्ध्नाद् बु॒द्ध्नाद् भ॑वतो भवतो बु॒द्ध्ना दे॒वैव बु॒द्ध्नाद् भ॑वतो भवतो बु॒द्ध्नादे॒व । \newline
57. बु॒द्ध्ना दे॒वैव बु॒द्ध्नाद् बु॒द्ध्ना दे॒वैन॑ मेन मे॒व बु॒द्ध्नाद् बु॒द्ध्ना दे॒वैन᳚म् । \newline
58. ए॒वैन॑ मेन मे॒वैवैन॒ मग्र॒ मग्र॑ मेन मे॒वैवैन॒ मग्र᳚म् । \newline
59. ए॒न॒ मग्र॒ मग्र॑ मेन मेन॒ मग्र॒म् परि॒ पर्यग्र॑ मेन मेन॒ मग्र॒म् परि॑ । \newline
60. अग्र॒म् परि॒ पर्यग्र॒ मग्र॒म् परि॑ णयति नयति॒ पर्यग्र॒ मग्र॒म् परि॑ णयति । \newline
\pagebreak
\markright{ TS 2.3.4.4  \hfill https://www.vedavms.in \hfill}

\section{ TS 2.3.4.4 }

\textbf{TS 2.3.4.4 } \newline
\textbf{Samhita Paata} \newline

परि॑ णयत्यानुषू॒को भ॑वत्ये॒षा ह्ये॑तस्य॑ दे॒वता॒ य आ॑नुजाव॒रः समृ॑द्ध्यै॒ यो ब्रा᳚ह्म॒ण आ॑नुजाव॒रः स्यात् तस्मा॑ ए॒तं बा॑र्.हस्प॒त्यमा॑नुषू॒कं च॒रुं निर्व॑पे॒द् बृह॒स्पति॑मे॒व स्वेन॑ भाग॒धेये॒नोप॑ धावति॒ स ए॒वैन॒मग्रꣳ॑ समा॒नानां॒ परि॑णयति बु॒द्ध्नव॑ती॒ अग्र॑वती याज्यानुवा॒क्ये॑ भवतो बु॒द्ध्नादे॒वैन॒मग्रं॒ परि॑ णयत्यानुषू॒को भ॑वत्ये॒षा ह्ये॑तस्य॑ ( ) दे॒वता॒ य आ॑नुजाव॒रः समृ॑द्ध्यै ॥ \newline

\textbf{Pada Paata} \newline

परीति॑ । न॒य॒ति॒ । आ॒नु॒षू॒क इत्या॑नु - सू॒कः । भ॒व॒ति॒ । ए॒षा । हि । ए॒तस्य॑ । दे॒वता᳚ । यः । आ॒नु॒जा॒व॒र इत्या॑नु - जा॒व॒रः । समृ॑द्ध्या॒ इति॒ सं- ऋ॒द्ध्यै॒ । यः । ब्रा॒ह्म॒णः । आ॒नु॒जा॒व॒र इत्या॑नु - जा॒व॒रः । स्यात् । तस्मै᳚ । ए॒तम् । बा॒र्.॒ह॒स्प॒त्यम् । आ॒नु॒षू॒कमित्या॑नु-सू॒कम् । च॒रुम् । निरिति॑ । व॒पे॒त् । बृह॒स्पति᳚म् । ए॒व । स्वेन॑ । भा॒ग॒धेये॒नेति॑ भाग - धेये॑न । उपेति॑ । धा॒व॒ति॒ । सः । ए॒व । ए॒न॒म् । अग्र᳚म् । स॒मा॒नाना᳚म् । परीति॑ । न॒य॒ति॒ । बु॒द्ध्नव॑ती॒ इति॑ बु॒द्ध्न-व॒ती॒ । अग्र॑वती॒ इत्यग्र॑ - व॒ती॒ । या॒ज्या॒नु॒वा॒क्ये॑ इति॑ याज्या - अ॒नु॒वा॒क्ये᳚ । भ॒व॒तः॒ । बु॒द्ध्नात् । ए॒व । ए॒न॒म् । अग्र᳚म् । परीति॑ । न॒य॒ति॒ । आ॒नु॒षू॒क इत्या॑नु - सू॒कः । भ॒व॒ति॒ । ए॒षा । हि । ए॒तस्य॑ ( ) । दे॒वता᳚ । यः । आ॒नु॒जा॒व॒र इत्या॑नु - जा॒व॒रः । समृ॑द्ध्या॒ इति॒ सं - ऋ॒द्ध्यै॒ ॥  \newline


\textbf{Krama Paata} \newline

परि॑णयति । न॒य॒त्या॒नु॒षू॒कः । आ॒नु॒षू॒को भ॑वति । आ॒नु॒षू॒क इत्या॑नु - सू॒कः । भ॒व॒त्ये॒षा । ए॒षाहि । ह्ये॑तस्य॑ । ए॒तस्य॑ दे॒वता᳚ । दे॒वता॒ यः । य आ॑नुजाव॒रः । आ॒नु॒जा॒व॒रः समृ॑द्ध्यै । आ॒नु॒जा॒व॒र इत्या॑नु - जा॒व॒रः । समृ॑द्ध्यै॒ यः । समृ॑द्ध्या॒ इति॒ सं - ऋ॒द्ध्यै॒ । यो ब्रा᳚ह्म॒णः । ब्रा॒ह्म॒ण आ॑नुजाव॒रः । आ॒नु॒जा॒व॒रः स्यात् । आ॒नु॒जा॒व॒र इत्या॑नु - जा॒व॒रः । स्यात् तस्मै᳚ । तस्मा॑ ए॒तम् । ए॒तम् बा॑र्.हस्प॒त्यम् । बा॒र्॒.ह॒स्प॒त्य,मा॑नुषू॒कम् । आ॒नु॒षू॒कम् च॒रुम् । आ॒नु॒षू॒कमित्या॑नु - सू॒कम् । च॒रुम् निः । निर् व॑पेत् । व॒पे॒द् बृह॒स्पति᳚म् । बृह॒स्पति॑मे॒व । ए॒व स्वेन॑ । स्वेन॑ भाग॒धेये॑न । भा॒ग॒धेये॒नोप॑ । भा॒ग॒धेये॒नेति॑ भाग - धेये॑न । उप॑ धावति । धा॒व॒ति॒ सः । स ए॒व । ए॒वैन᳚म् । ए॒न॒मग्र᳚म् । अग्रꣳ॑ समा॒नाना᳚म् । स॒मा॒नाना॒म् परि॑ । परि॑णयति । न॒य॒ति॒ बु॒द्ध्नव॑ती । बु॒द्ध्नव॑ती॒ अग्र॑वती । बु॒द्ध्नव॑ती॒ इति॑ बु॒द्ध्न - व॒ती॒ । अग्र॑वती याज्यानुवा॒क्ये᳚ । अग्र॑वती॒ इत्यग्र॑ - व॒ती॒ । या॒ज्या॒नु॒वा॒क्ये॑ भवतः । या॒ज्या॒नु॒वा॒क्ये॑ इति॑ याज्या - अ॒नु॒वा॒क्ये᳚ । भ॒व॒तो॒ बु॒द्ध्नात् । बु॒द्ध्नादे॒व । ए॒वैन᳚म् । ए॒न॒मग्र᳚म् । अग्र॒म् परि॑ । परि॑णयति । न॒य॒त्या॒नु॒षू॒कः । आ॒नु॒षू॒को भ॑वति । आ॒नु॒षू॒क इत्या॑नु - सू॒कः । भ॒व॒त्ये॒षा । ए॒षा हि । ह्ये॑तस्य॑ ( ) । ए॒तस्य॑ दे॒वता᳚ । दे॒वता॒ यः । य आ॑नुजाव॒रः । आ॒नु॒जा॒व॒रः समृ॑द्ध्यै । आ॒नु॒जा॒व॒र इत्या॑नु - जा॒व॒रः । समृ॑द्ध्या॒ इति॒ सं - ऋ॒द्ध्यै॒ । \newline

\textbf{Jatai Paata} \newline

1. परि॑ णयति नयति॒ परि॒ परि॑ णयति । \newline
2. न॒य॒ त्या॒नु॒षू॒क आ॑नुषू॒को न॑यति नय त्यानुषू॒कः । \newline
3. आ॒नु॒षू॒को भ॑वति भव त्यानुषू॒क आ॑नुषू॒को भ॑वति । \newline
4. आ॒नु॒षू॒क इत्या॑नु - सू॒कः । \newline
5. भ॒व॒ त्ये॒षैषा भ॑वति भव त्ये॒षा । \newline
6. ए॒षा हि ह्ये॑षैषा हि । \newline
7. ह्ये॑त स्यै॒तस्य॒ हि ह्ये॑तस्य॑ । \newline
8. ए॒तस्य॑ दे॒वता॑ दे॒वतै॒त स्यै॒तस्य॑ दे॒वता᳚ । \newline
9. दे॒वता॒ यो यो दे॒वता॑ दे॒वता॒ यः । \newline
10. य आ॑नुजाव॒र आ॑नुजाव॒रो यो य आ॑नुजाव॒रः । \newline
11. आ॒नु॒जा॒व॒रः समृ॑द्ध्यै॒ समृ॑द्ध्या आनुजाव॒र आ॑नुजाव॒रः समृ॑द्ध्यै । \newline
12. आ॒नु॒जा॒व॒र इत्या॑नु - जा॒व॒रः । \newline
13. समृ॑द्ध्यै॒ यो यः समृ॑द्ध्यै॒ समृ॑द्ध्यै॒ यः । \newline
14. समृ॑द्ध्या॒ इति॒ सं - ऋ॒द्ध्यै॒ । \newline
15. यो ब्रा᳚ह्म॒णो ब्रा᳚ह्म॒णो यो यो ब्रा᳚ह्म॒णः । \newline
16. ब्रा॒ह्म॒ण आ॑नुजाव॒र आ॑नुजाव॒रो ब्रा᳚ह्म॒णो ब्रा᳚ह्म॒ण आ॑नुजाव॒रः । \newline
17. आ॒नु॒जा॒व॒रः स्याथ् स्यादा॑नुजाव॒र आ॑नुजाव॒रः स्यात् । \newline
18. आ॒नु॒जा॒व॒र इत्या॑नु - जा॒व॒रः । \newline
19. स्यात् तस्मै॒ तस्मै॒ स्याथ् स्यात् तस्मै᳚ । \newline
20. तस्मा॑ ए॒त मे॒तम् तस्मै॒ तस्मा॑ ए॒तम् । \newline
21. ए॒तम् बा॑र्.हस्प॒त्यम् बा॑र्.हस्प॒त्य मे॒त मे॒तम् बा॑र्.हस्प॒त्यम् । \newline
22. बा॒र्॒.ह॒स्प॒त्य मा॑नुषू॒क मा॑नुषू॒कम् बा॑र्.हस्प॒त्यम् बा॑र्.हस्प॒त्य मा॑नुषू॒कम् । \newline
23. आ॒नु॒षू॒कम् च॒रुम् च॒रु मा॑नुषू॒क मा॑नुषू॒कम् च॒रुम् । \newline
24. आ॒नु॒षू॒कमित्या॑नु - सू॒कम् । \newline
25. च॒रुम् निर् णिश्च॒रुम् च॒रुम् निः । \newline
26. निर् व॑पेद् वपे॒न् निर् णिर् व॑पेत् । \newline
27. व॒पे॒द् बृह॒स्पति॒म् बृह॒स्पतिं॑ ॅवपेद् वपे॒द् बृह॒स्पति᳚म् । \newline
28. बृह॒स्पति॑ मे॒वैव बृह॒स्पति॒म् बृह॒स्पति॑ मे॒व । \newline
29. ए॒व स्वेन॒ स्वेनै॒वैव स्वेन॑ । \newline
30. स्वेन॑ भाग॒धेये॑न भाग॒धेये॑न॒ स्वेन॒ स्वेन॑ भाग॒धेये॑न । \newline
31. भा॒ग॒धेये॒नोपोप॑ भाग॒धेये॑न भाग॒धेये॒नोप॑ । \newline
32. भा॒ग॒धेये॒नेति॑ भाग - धेये॑न । \newline
33. उप॑ धावति धाव॒ त्युपोप॑ धावति । \newline
34. धा॒व॒ति॒ स स धा॑वति धावति॒ सः । \newline
35. स ए॒वैव स स ए॒व । \newline
36. ए॒वैन॑ मेन मे॒वैवैन᳚म् । \newline
37. ए॒न॒ मग्र॒ मग्र॑ मेन मेन॒ मग्र᳚म् । \newline
38. अग्रꣳ॑ समा॒नानाꣳ॑ समा॒नाना॒ मग्र॒ मग्रꣳ॑ समा॒नाना᳚म् । \newline
39. स॒मा॒नाना॒म् परि॒ परि॑ समा॒नानाꣳ॑ समा॒नाना॒म् परि॑ । \newline
40. परि॑ णयति नयति॒ परि॒ परि॑ णयति । \newline
41. न॒य॒ति॒ बु॒द्ध्नव॑ती बु॒द्ध्नव॑ती नयति नयति बु॒द्ध्नव॑ती । \newline
42. बु॒द्ध्नव॑ती॒ अग्र॑वती॒ अग्र॑वती बु॒द्ध्नव॑ती बु॒द्ध्नव॑ती॒ अग्र॑वती । \newline
43. बु॒द्ध्नव॑ती॒ इति॑ बु॒द्ध्न - व॒ती॒ । \newline
44. अग्र॑वती याज्यानुवा॒क्ये॑ याज्यानुवा॒क्ये॑ अग्र॑वती॒ अग्र॑वती याज्यानुवा॒क्ये᳚ । \newline
45. अग्र॑वती॒ इत्यग्र॑ - व॒ती॒ । \newline
46. या॒ज्या॒नु॒वा॒क्ये॑ भवतो भवतो याज्यानुवा॒क्ये॑ याज्यानुवा॒क्ये॑ भवतः । \newline
47. या॒ज्या॒नु॒वा॒क्ये॑ इति॑ याज्या - अ॒नु॒वा॒क्ये᳚ । \newline
48. भ॒व॒तो॒ बु॒द्ध्नाद् बु॒द्ध्नाद् भ॑वतो भवतो बु॒द्ध्नात् । \newline
49. बु॒द्ध्ना दे॒वैव बु॒द्ध्नाद् बु॒द्ध्ना दे॒व । \newline
50. ए॒वैन॑ मेन मे॒वैवैन᳚म् । \newline
51. ए॒न॒ मग्र॒ मग्र॑ मेन मेन॒ मग्र᳚म् । \newline
52. अग्र॒म् परि॒ पर्यग्र॒ मग्र॒म् परि॑ । \newline
53. परि॑ णयति नयति॒ परि॒ परि॑ णयति । \newline
54. न॒य॒ त्या॒नु॒षू॒क आ॑नुषू॒को न॑यति नय त्यानुषू॒कः । \newline
55. आ॒नु॒षू॒को भ॑वति भव त्यानुषू॒क आ॑नुषू॒को भ॑वति । \newline
56. आ॒नु॒षू॒क इत्या॑नु - सू॒कः । \newline
57. भ॒व॒ त्ये॒षैषा भ॑वति भव त्ये॒षा । \newline
58. ए॒षा हि ह्ये॑षैषा हि । \newline
59. ह्ये॑त स्यै॒तस्य॒ हि ह्ये॑तस्य॑ । \newline
60. ए॒तस्य॑ दे॒वता॑ दे॒वतै॒त स्यै॒तस्य॑ दे॒वता᳚ । \newline
61. दे॒वता॒ यो यो दे॒वता॑ दे॒वता॒ यः । \newline
62. य आ॑नुजाव॒र आ॑नुजाव॒रो यो य आ॑नुजाव॒रः । \newline
63. आ॒नु॒जा॒व॒रः समृ॑द्ध्यै॒ समृ॑द्ध्या आनुजाव॒र आ॑नुजाव॒रः समृ॑द्ध्यै । \newline
64. आ॒नु॒जा॒व॒र इत्या॑नु - जा॒व॒रः । \newline
65. समृ॑द्ध्या॒ इति॒ सं - ऋ॒द्ध्यै॒ । \newline

\textbf{Ghana Paata } \newline

1. परि॑ णयति नयति॒ परि॒ परि॑ णय त्यानुषू॒क आ॑नुषू॒को न॑यति॒ परि॒ परि॑ णय त्यानुषू॒कः । \newline
2. न॒य॒ त्या॒नु॒षू॒क आ॑नुषू॒को न॑यति नय त्यानुषू॒को भ॑वति भव त्यानुषू॒को न॑यति नय त्यानुषू॒को भ॑वति । \newline
3. आ॒नु॒षू॒को भ॑वति भव त्यानुषू॒क आ॑नुषू॒को भ॑व त्ये॒षैषा भ॑व त्यानुषू॒क आ॑नुषू॒को भ॑व त्ये॒षा । \newline
4. आ॒नु॒षू॒क इत्या॑नु - सू॒कः । \newline
5. भ॒व॒ त्ये॒षैषा भ॑वति भव त्ये॒षा हि ह्ये॑षा भ॑वति भव त्ये॒षा हि । \newline
6. ए॒षा हि ह्ये॑षैषा ह्ये॑त स्यै॒तस्य॒ ह्ये॑षैषा ह्ये॑तस्य॑ । \newline
7. ह्ये॑त स्यै॒तस्य॒ हि ह्ये॑तस्य॑ दे॒वता॑ दे॒वतै॒तस्य॒ हि ह्ये॑तस्य॑ दे॒वता᳚ । \newline
8. ए॒तस्य॑ दे॒वता॑ दे॒व तै॒त स्यै॒तस्य॑ दे॒वता॒ यो यो दे॒व तै॒त स्यै॒तस्य॑ दे॒वता॒ यः । \newline
9. दे॒वता॒ यो यो दे॒वता॑ दे॒वता॒ य आ॑नुजाव॒र आ॑नुजाव॒रो यो दे॒वता॑ दे॒वता॒ य आ॑नुजाव॒रः । \newline
10. य आ॑नुजाव॒र आ॑नुजाव॒रो यो य आ॑नुजाव॒रः समृ॑द्ध्यै॒ समृ॑द्ध्या आनुजाव॒रो यो य आ॑नुजाव॒रः समृ॑द्ध्यै । \newline
11. आ॒नु॒जा॒व॒रः समृ॑द्ध्यै॒ समृ॑द्ध्या आनुजाव॒र आ॑नुजाव॒रः समृ॑द्ध्यै॒ यो यः समृ॑द्ध्या आनुजाव॒र आ॑नुजाव॒रः समृ॑द्ध्यै॒ यः । \newline
12. आ॒नु॒जा॒व॒र इत्या॑नु - जा॒व॒रः । \newline
13. समृ॑द्ध्यै॒ यो यः समृ॑द्ध्यै॒ समृ॑द्ध्यै॒ यो ब्रा᳚ह्म॒णो ब्रा᳚ह्म॒णो यः समृ॑द्ध्यै॒ समृ॑द्ध्यै॒ यो ब्रा᳚ह्म॒णः । \newline
14. समृ॑द्ध्या॒ इति॒ सं - ऋ॒द्ध्यै॒ । \newline
15. यो ब्रा᳚ह्म॒णो ब्रा᳚ह्म॒णो यो यो ब्रा᳚ह्म॒ण आ॑नुजाव॒र आ॑नुजाव॒रो ब्रा᳚ह्म॒णो यो यो ब्रा᳚ह्म॒ण आ॑नुजाव॒रः । \newline
16. ब्रा॒ह्म॒ण आ॑नुजाव॒र आ॑नुजाव॒रो ब्रा᳚ह्म॒णो ब्रा᳚ह्म॒ण आ॑नुजाव॒रः स्याथ् स्या दा॑नुजाव॒रो ब्रा᳚ह्म॒णो ब्रा᳚ह्म॒ण आ॑नुजाव॒रः स्यात् । \newline
17. आ॒नु॒जा॒व॒रः स्याथ् स्या दा॑नुजाव॒र आ॑नुजाव॒रः स्यात् तस्मै॒ तस्मै॒ स्या दा॑नुजाव॒र आ॑नुजाव॒रः स्यात् तस्मै᳚ । \newline
18. आ॒नु॒जा॒व॒र इत्या॑नु - जा॒व॒रः । \newline
19. स्यात् तस्मै॒ तस्मै॒ स्याथ् स्यात् तस्मा॑ ए॒त मे॒तम् तस्मै॒ स्याथ् स्यात् तस्मा॑ ए॒तम् । \newline
20. तस्मा॑ ए॒त मे॒तम् तस्मै॒ तस्मा॑ ए॒तम् बा॑र्.हस्प॒त्यम् बा॑र्.हस्प॒त्य मे॒तम् तस्मै॒ तस्मा॑ ए॒तम् बा॑र्.हस्प॒त्यम् । \newline
21. ए॒तम् बा॑र्.हस्प॒त्यम् बा॑र्.हस्प॒त्य मे॒त मे॒तम् बा॑र्.हस्प॒त्य मा॑नुषू॒क मा॑नुषू॒कम् बा॑र्.हस्प॒त्य मे॒त मे॒तम् बा॑र्.हस्प॒त्य मा॑नुषू॒कम् । \newline
22. बा॒र्॒.ह॒स्प॒त्य मा॑नुषू॒क मा॑नुषू॒कम् बा॑र्.हस्प॒त्यम् बा॑र्.हस्प॒त्य मा॑नुषू॒कम् च॒रुम् च॒रु मा॑नुषू॒कम् बा॑र्.हस्प॒त्यम् बा॑र्.हस्प॒त्य मा॑नुषू॒कम् च॒रुम् । \newline
23. आ॒नु॒षू॒कम् च॒रुम् च॒रु मा॑नुषू॒क मा॑नुषू॒कम् च॒रुम् निर् णिश्च॒रु मा॑नुषू॒क मा॑नुषू॒कम् च॒रुम् निः । \newline
24. आ॒नु॒षू॒कमित्या॑नु - सू॒कम् । \newline
25. च॒रुम् निर् णिश्च॒रुम् च॒रुम् निर् व॑पेद् वपे॒न् निश्च॒रुम् च॒रुम् निर् व॑पेत् । \newline
26. निर् व॑पेद् वपे॒न् निर् णिर् व॑पे॒द् बृह॒स्पति॒म् बृह॒स्पतिं॑ ॅवपे॒न् निर् णिर् व॑पे॒द् बृह॒स्पति᳚म् । \newline
27. व॒पे॒द् बृह॒स्पति॒म् बृह॒स्पतिं॑ ॅवपेद् वपे॒द् बृह॒स्पति॑ मे॒वैव बृह॒स्पतिं॑ ॅवपेद् वपे॒द् बृह॒स्पति॑ मे॒व । \newline
28. बृह॒स्पति॑ मे॒वैव बृह॒स्पति॒म् बृह॒स्पति॑ मे॒व स्वेन॒ स्वेनै॒व बृह॒स्पति॒म् बृह॒स्पति॑ मे॒व स्वेन॑ । \newline
29. ए॒व स्वेन॒ स्वेनै॒वैव स्वेन॑ भाग॒धेये॑न भाग॒धेये॑न॒ स्वेनै॒वैव स्वेन॑ भाग॒धेये॑न । \newline
30. स्वेन॑ भाग॒धेये॑न भाग॒धेये॑न॒ स्वेन॒ स्वेन॑ भाग॒धेये॒नोपोप॑ भाग॒धेये॑न॒ स्वेन॒ स्वेन॑ भाग॒धेये॒नोप॑ । \newline
31. भा॒ग॒धेये॒नोपोप॑ भाग॒धेये॑न भाग॒धेये॒नोप॑ धावति धाव॒त्युप॑ भाग॒धेये॑न भाग॒धेये॒नोप॑ धावति । \newline
32. भा॒ग॒धेये॒नेति॑ भाग - धेये॑न । \newline
33. उप॑ धावति धाव॒ त्युपोप॑ धावति॒ स स धा॑व॒ त्युपोप॑ धावति॒ सः । \newline
34. धा॒व॒ति॒ स स धा॑वति धावति॒ स ए॒वैव स धा॑वति धावति॒ स ए॒व । \newline
35. स ए॒वैव स स ए॒वैन॑ मेन मे॒व स स ए॒वैन᳚म् । \newline
36. ए॒वैन॑ मेन मे॒वैवैन॒ मग्र॒ मग्र॑ मेन मे॒वैवैन॒ मग्र᳚म् । \newline
37. ए॒न॒ मग्र॒ मग्र॑ मेन मेन॒ मग्रꣳ॑ समा॒नानाꣳ॑ समा॒नाना॒ मग्र॑ मेन मेन॒ मग्रꣳ॑ समा॒नाना᳚म् । \newline
38. अग्रꣳ॑ समा॒नानाꣳ॑ समा॒नाना॒ मग्र॒ मग्रꣳ॑ समा॒नाना॒म् परि॒ परि॑ समा॒नाना॒ मग्र॒ मग्रꣳ॑ समा॒नाना॒म् परि॑ । \newline
39. स॒मा॒नाना॒म् परि॒ परि॑ समा॒नानाꣳ॑ समा॒नाना॒म् परि॑ णयति नयति॒ परि॑ समा॒नानाꣳ॑ समा॒नाना॒म् परि॑ णयति । \newline
40. परि॑ णयति नयति॒ परि॒ परि॑ णयति बु॒द्ध्नव॑ती बु॒द्ध्नव॑ती नयति॒ परि॒ परि॑ णयति बु॒द्ध्नव॑ती । \newline
41. न॒य॒ति॒ बु॒द्ध्नव॑ती बु॒द्ध्नव॑ती नयति नयति बु॒द्ध्नव॑ती॒ अग्र॑वती॒ अग्र॑वती बु॒द्ध्नव॑ती नयति नयति बु॒द्ध्नव॑ती॒ अग्र॑वती । \newline
42. बु॒द्ध्नव॑ती॒ अग्र॑वती॒ अग्र॑वती बु॒द्ध्नव॑ती बु॒द्ध्नव॑ती॒ अग्र॑वती याज्यानुवा॒क्ये॑ याज्यानुवा॒क्ये॑ अग्र॑वती बु॒द्ध्नव॑ती बु॒द्ध्नव॑ती॒ अग्र॑वती याज्यानुवा॒क्ये᳚ । \newline
43. बु॒द्ध्नव॑ती॒ इति॑ बु॒द्ध्न - व॒ती॒ । \newline
44. अग्र॑वती याज्यानुवा॒क्ये॑ याज्यानुवा॒क्ये॑ अग्र॑वती॒ अग्र॑वती याज्यानुवा॒क्ये॑ भवतो भवतो याज्यानुवा॒क्ये॑ अग्र॑वती॒ अग्र॑वती याज्यानुवा॒क्ये॑ भवतः । \newline
45. अग्र॑वती॒ इत्यग्र॑ - व॒ती॒ । \newline
46. या॒ज्या॒नु॒वा॒क्ये॑ भवतो भवतो याज्यानुवा॒क्ये॑ याज्यानुवा॒क्ये॑ भवतो बु॒द्ध्नाद् बु॒द्ध्नाद् भ॑वतो याज्यानुवा॒क्ये॑ याज्यानुवा॒क्ये॑ भवतो बु॒द्ध्नात् । \newline
47. या॒ज्या॒नु॒वा॒क्ये॑ इति॑ याज्या - अ॒नु॒वा॒क्ये᳚ । \newline
48. भ॒व॒तो॒ बु॒द्ध्नाद् बु॒द्ध्नाद् भ॑वतो भवतो बु॒द्ध्ना दे॒वैव बु॒द्ध्नाद् भ॑वतो भवतो बु॒द्ध्ना दे॒व । \newline
49. बु॒द्ध्ना दे॒वैव बु॒द्ध्नाद् बु॒द्ध्ना दे॒वैन॑ मेन मे॒व बु॒द्ध्नाद् बु॒द्ध्ना दे॒वैन᳚म् । \newline
50. ए॒वैन॑ मेन मे॒वैवैन॒ मग्र॒ मग्र॑ मेन मे॒वैवैन॒ मग्र᳚म् । \newline
51. ए॒न॒ मग्र॒ मग्र॑ मेन मेन॒ मग्र॒म् परि॒ पर्यग्र॑ मेन मेन॒ मग्र॒म् परि॑ । \newline
52. अग्र॒म् परि॒ पर्यग्र॒ मग्र॒म् परि॑ णयति नयति॒ पर्यग्र॒ मग्र॒म् परि॑ णयति । \newline
53. परि॑ णयति नयति॒ परि॒ परि॑ णय त्यानुषू॒क आ॑नुषू॒को न॑यति॒ परि॒ परि॑ णय त्यानुषू॒कः । \newline
54. न॒य॒ त्या॒नु॒षू॒क आ॑नुषू॒को न॑यति नय त्यानुषू॒को भ॑वति भव त्यानुषू॒को न॑यति नय त्यानुषू॒को भ॑वति । \newline
55. आ॒नु॒षू॒को भ॑वति भव त्यानुषू॒क आ॑नुषू॒को भ॑व त्ये॒षैषा भ॑व त्यानुषू॒क आ॑नुषू॒को भ॑व त्ये॒षा । \newline
56. आ॒नु॒षू॒क इत्या॑नु - सू॒कः । \newline
57. भ॒व॒ त्ये॒षैषा भ॑वति भव त्ये॒षा हि ह्ये॑षा भ॑वति भव त्ये॒षा हि । \newline
58. ए॒षा हि ह्ये॑षैषा ह्ये॑तस्यै॒तस्य॒ ह्ये॑षैषा ह्ये॑तस्य॑ । \newline
59. ह्ये॑त स्यै॒तस्य॒ हि ह्ये॑तस्य॑ दे॒वता॑ दे॒व तै॒तस्य॒ हि ह्ये॑तस्य॑ दे॒वता᳚ । \newline
60. ए॒तस्य॑ दे॒वता॑ दे॒व तै॒त स्यै॒तस्य॑ दे॒वता॒ यो यो दे॒व तै॒त स्यै॒तस्य॑ दे॒वता॒ यः । \newline
61. दे॒वता॒ यो यो दे॒वता॑ दे॒वता॒ य आ॑नुजाव॒र आ॑नुजाव॒रो यो दे॒वता॑ दे॒वता॒ य आ॑नुजाव॒रः । \newline
62. य आ॑नुजाव॒र आ॑नुजाव॒रो यो य आ॑नुजाव॒रः समृ॑द्ध्यै॒ समृ॑द्ध्या आनुजाव॒रो यो य आ॑नुजाव॒रः समृ॑द्ध्यै । \newline
63. आ॒नु॒जा॒व॒रः समृ॑द्ध्यै॒ समृ॑द्ध्या आनुजाव॒र आ॑नुजाव॒रः समृ॑द्ध्यै । \newline
64. आ॒नु॒जा॒व॒र इत्या॑नु - जा॒व॒रः । \newline
65. समृ॑द्ध्या॒ इति॒ सं - ऋ॒द्ध्यै॒ । \newline
\pagebreak
\markright{ TS 2.3.5.1  \hfill https://www.vedavms.in \hfill}

\section{ TS 2.3.5.1 }

\textbf{TS 2.3.5.1 } \newline
\textbf{Samhita Paata} \newline

प्र॒जाप॑ते॒स्त्रय॑स्त्रिꣳशद् दुहि॒तर॑ आस॒न् ताः सोमा॑य॒ राज्ञे॑ऽददा॒त् तासाꣳ॑ रोहि॒णीमुपै॒त् ता ईर्ष्य॑न्तीः॒ पुन॑रगच्छ॒न् ता अन्वै॒त् ताः पुन॑रयाचत॒ ता अ॑स्मै॒ न पुन॑रददा॒थ् सो᳚ऽब्रवीदृ॒तम॑मीष्व॒ यथा॑ समाव॒च्छ उ॑पै॒ष्याम्यथ॑ ते॒ पुन॑र्दास्या॒मीति॒ स ऋ॒तमा॑मी॒त् ता अ॑स्मै॒ पुन॑रददा॒त् तासाꣳ॑ रोहि॒णीमे॒वोप॒त् - [  ] \newline

\textbf{Pada Paata} \newline

प्र॒जाप॑ते॒रिति॑ प्र॒जा - प॒तेः॒ । त्रय॑स्त्रिꣳश॒दिति॒ त्रयः॑ - त्रिꣳ॒॒श॒त् । दु॒हि॒तरः॑ । आ॒स॒न्न् । ताः । सोमा॑य । राज्ञे᳚ । अ॒द॒दा॒त् । तासा᳚म् । रो॒हि॒णीम् । उपेति॑ । ऐ॒त् । ताः । ईर्ष्य॑न्तीः । पुनः॑ । अ॒ग॒च्छ॒न्न् । ताः । अन्विति॑ । ऐ॒त् । ताः । पुनः॑ । अ॒या॒च॒त॒ । ताः । अ॒स्मै॒ । न । पुनः॑ । अ॒द॒दा॒त् । सः । अ॒ब्र॒वी॒त् । ऋ॒तम् । अ॒मी॒ष्व॒ । यथा᳚ । स॒मा॒व॒च्छ इति॑ समावत् - शः । उ॒पै॒ष्यामीत्यु॑प - ए॒ष्यामि॑ । अथ॑ । ते॒ । पुनः॑ । दा॒स्या॒मि॒ । इति॑ । सः । ऋ॒तम् । आ॒मी॒त् । ताः । अ॒स्मै॒ । पुनः॑ । अ॒द॒दा॒त् । तासा᳚म् । रो॒हि॒णीम् । ए॒व । उपेति॑ ।  \newline


\textbf{Krama Paata} \newline

प्र॒जाप॑ते॒स्त्रय॑स्त्रिꣳशत् । प्र॒जाप॑ते॒रिति॑ प्र॒जा - प॒तेः॒ । त्रय॑स्त्रिꣳशद् दुहि॒तरः॑ । त्रय॑स्त्रिꣳश॒दिति॒ त्रयः॑ - त्रिꣳ॒॒श॒त् । दु॒हि॒तर॑ आसन्न् । आ॒स॒न् ताः । ताः सोमा॑य । सोमा॑य॒ राज्ञे᳚ । राज्ञे॑ ऽददात् । अ॒द॒दा॒त् तासा᳚म् । तासाꣳ॑ रोहि॒णीम् । रो॒हि॒णीमुप॑ । ऊपै᳚त् । ऐ॒त् ताः । ता ईर्ष्य॑न्तीः । ईर्ष्य॑न्तीः॒ पुनः॑ । पुन॑रगच्छन्न् । अ॒ग॒च्छ॒न् ताः । ता अनु॑ । अन्वै᳚त् । ऐ॒त् ताः । ताः पुनः॑ । पुन॑रयाचत । अ॒या॒च॒त॒ ताः । ता अ॑स्मै । अ॒स्मै॒ न । न पुनः॑ । पुन॑रददात् । अ॒द॒दा॒थ् सः । सो᳚ऽब्रवीत् । अ॒ब्र॒वी॒दृ॒तम् । ऋ॒तम॑मीष्व । अ॒मी॒ष्व॒ यथा᳚ । यथा॑ समाव॒च्छः । स॒मा॒व॒च्छ उ॑पै॒ष्यामि॑ । स॒मा॒व॒च्छ इति॑ समावत् - शः । उ॒पै॒ष्याम्यथ॑ । उ॒पै॒ष्यामीत्यु॑प - ए॒ष्यामि॑ । अथ॑ ते । ते॒ पुनः॑ । पुन॑र् दास्यामि । दा॒स्या॒मीति॑ । इति॒ सः । स ऋ॒तम् । ऋ॒तमा॑मीत् । आ॒मी॒त् ताः । ता अ॑स्मै । अ॒स्मै॒ पुनः॑ । पुन॑रददात् । अ॒द॒दा॒त् तासा᳚म् । तासाꣳ॑ रोहि॒णीम् । रो॒हि॒णीमे॒व । 
ए॒वोप॑ । उपै᳚त् \newline

\textbf{Jatai Paata} \newline

1. प्र॒जाप॑ते॒ स्त्रय॑स्त्रिꣳश॒त् त्रय॑स्त्रिꣳशत् प्र॒जाप॑तेः प्र॒जाप॑ते॒ स्त्रय॑स्त्रिꣳशत् । \newline
2. प्र॒जाप॑ते॒रिति॑ प्र॒जा - प॒तेः॒ । \newline
3. त्रय॑स्त्रिꣳशद् दुहि॒तरो॑ दुहि॒तर॒स्त्रय॑ स्त्रिꣳश॒त् त्रय॑स्त्रिꣳशद् दुहि॒तरः॑ । \newline
4. त्रय॑स्त्रिꣳश॒दिति॒ त्रयः॑ - त्रिꣳ॒॒श॒त् । \newline
5. दु॒हि॒तर॑ आसन् नासन् दुहि॒तरो॑ दुहि॒तर॑ आसन्न् । \newline
6. आ॒स॒न् ता स्ता आ॑सन् नास॒न् ताः । \newline
7. ताः सोमा॑य॒ सोमा॑य॒ तास्ताः सोमा॑य । \newline
8. सोमा॑य॒ राज्ञे॒ राज्ञे॒ सोमा॑य॒ सोमा॑य॒ राज्ञे᳚ । \newline
9. राज्ञे॑ ऽददा दददा॒द् राज्ञे॒ राज्ञे॑ ऽददात् । \newline
10. अ॒द॒दा॒त् तासा॒म् तासा॑ मददा दददा॒त् तासा᳚म् । \newline
11. तासाꣳ॑ रोहि॒णीꣳ रो॑हि॒णीम् तासा॒म् तासाꣳ॑ रोहि॒णीम् । \newline
12. रो॒हि॒णी मुपोप॑ रोहि॒णीꣳ रो॑हि॒णी मुप॑ । \newline
13. उपै॑ दै॒दुपोपै᳚त् । \newline
14. ऐ॒त् ता स्ता ऐ॑दै॒त् ताः । \newline
15. ता ईर्ष्य॑न्ती॒ रीर्ष्य॑न्ती॒ स्ता स्ता ईर्ष्य॑न्तीः । \newline
16. ईर्ष्य॑न्तीः॒ पुनः॒ पुन॒ रीर्ष्य॑न्ती॒री र्ष्य॑न्तीः॒ पुनः॑ । \newline
17. पुन॑ रगच्छन् नगच्छ॒न् पुनः॒ पुन॑ रगच्छन्न् । \newline
18. अ॒ग॒च्छ॒न् ता स्ता अ॑गच्छन् नगच्छ॒न् ताः । \newline
19. ता अन्वनु॒ ता स्ता अनु॑ । \newline
20. अन्वै॑ दै॒दन्वन्वै᳚त् । \newline
21. ऐ॒त् ता स्ता ऐ॑दै॒त् ताः । \newline
22. ताः पुनः॒ पुन॒ स्ता स्ताः पुनः॑ । \newline
23. पुन॑ रयाचता याचत॒ पुनः॒ पुन॑ रयाचत । \newline
24. अ॒या॒च॒त॒ तास्ता अ॑याचता याचत॒ ताः । \newline
25. ता अ॑स्मा अस्मै॒ ता स्ता अ॑स्मै । \newline
26. अ॒स्मै॒ न नास्मा॑ अस्मै॒ न । \newline
27. न पुनः॒ पुन॒र् न न पुनः॑ । \newline
28. पुन॑ रददा दददा॒त् पुनः॒ पुन॑ रददात् । \newline
29. अ॒द॒दा॒थ् स सो॑ ऽददा दददा॒थ् सः । \newline
30. सो᳚ ऽब्रवी दब्रवी॒थ् स सो᳚ ऽब्रवीत् । \newline
31. अ॒ब्र॒वी॒दृ॒त मृ॒त म॑ब्रवी दब्रवीदृ॒तम् । \newline
32. ऋ॒त म॑मीष्वा मीष्व॒ र्त मृ॒त म॑मीष्व । \newline
33. अ॒मी॒ष्व॒ यथा॒ यथा॑ ऽमीष्वा मीष्व॒ यथा᳚ । \newline
34. यथा॑ समाव॒च्छः स॑माव॒च्छो यथा॒ यथा॑ समाव॒च्छः । \newline
35. स॒मा॒व॒च्छ उ॑पै॒ष्या म्यु॑पै॒ष्यामि॑ समाव॒च्छः स॑माव॒च्छ उ॑पै॒ष्यामि॑ । \newline
36. स॒मा॒व॒च्छ इति॑ समावत् - शः । \newline
37. उ॒पै॒ष्या म्यथाथो॑ पै॒ष्या म्यु॑पै॒ष्या म्यथ॑ । \newline
38. उ॒पै॒ष्यामीत्यु॑प - ए॒ष्यामि॑ । \newline
39. अथ॑ ते॒ ते ऽथाथ॑ ते । \newline
40. ते॒ पुनः॒ पुन॑ स्ते ते॒ पुनः॑ । \newline
41. पुन॑र् दास्यामि दास्यामि॒ पुनः॒ पुन॑र् दास्यामि । \newline
42. दा॒स्या॒मीतीति॑ दास्यामि दास्या॒मीति॑ । \newline
43. इति॒ स स इतीति॒ सः । \newline
44. स ऋ॒त मृ॒तꣳ स स ऋ॒तम् । \newline
45. ऋ॒त मा॑मी दामीदृ॒त मृ॒त मा॑मीत् । \newline
46. आ॒मी॒त् तास्ता आ॑मी दामी॒त् ताः । \newline
47. ता अ॑स्मा अस्मै॒ ता स्ता अ॑स्मै । \newline
48. अ॒स्मै॒ पुनः॒ पुन॑ रस्मा अस्मै॒ पुनः॑ । \newline
49. पुन॑ रददा दददा॒त् पुनः॒ पुन॑ रददात् । \newline
50. अ॒द॒दा॒त् तासा॒म् तासा॑ मददा दददा॒त् तासा᳚म् । \newline
51. तासाꣳ॑ रोहि॒णीꣳ रो॑हि॒णीम् तासा॒म् तासाꣳ॑ रोहि॒णीम् । \newline
52. रो॒हि॒णी मे॒वैव रो॑हि॒णीꣳ रो॑हि॒णी मे॒व । \newline
53. ए॒वोपो पै॒वैवोप॑ । \newline
54. उपै॑ दै॒दु पोपै᳚त् । \newline

\textbf{Ghana Paata } \newline

1. प्र॒जाप॑ते॒ स्त्रय॑स्त्रिꣳश॒त् त्रय॑स्त्रिꣳशत् प्र॒जाप॑तेः प्र॒जाप॑ते॒ स्त्रय॑स्त्रिꣳशद् दुहि॒तरो॑ दुहि॒तर॒ स्त्रय॑स्त्रिꣳशत् प्र॒जाप॑तेः प्र॒जाप॑ते॒ स्त्रय॑स्त्रिꣳशद् दुहि॒तरः॑ । \newline
2. प्र॒जाप॑ते॒रिति॑ प्र॒जा - प॒तेः॒ । \newline
3. त्रय॑स्त्रिꣳशद् दुहि॒तरो॑ दुहि॒तर॒ स्त्रय॑स्त्रिꣳश॒त् त्रय॑स्त्रिꣳशद् दुहि॒तर॑ आसन् नासन् दुहि॒तर॒ स्त्रय॑स्त्रिꣳश॒त् त्रय॑स्त्रिꣳशद् दुहि॒तर॑ आसन्न् । \newline
4. त्रय॑स्त्रिꣳश॒दिति॒ त्रयः॑ - त्रिꣳ॒॒श॒त् । \newline
5. दु॒हि॒तर॑ आसन् नासन् दुहि॒तरो॑ दुहि॒तर॑ आस॒न् ता स्ता आ॑सन् दुहि॒तरो॑ दुहि॒तर॑ आस॒न् ताः । \newline
6. आ॒स॒न् ता स्ता आ॑सन् नास॒न् ताः सोमा॑य॒ सोमा॑य॒ ता आ॑सन् नास॒न् ताः सोमा॑य । \newline
7. ताः सोमा॑य॒ सोमा॑य॒ ता स्ताः सोमा॑य॒ राज्ञे॒ राज्ञे॒ सोमा॑य॒ ता स्ताः सोमा॑य॒ राज्ञे᳚ । \newline
8. सोमा॑य॒ राज्ञे॒ राज्ञे॒ सोमा॑य॒ सोमा॑य॒ राज्ञे॑ ऽददा दददा॒द् राज्ञे॒ सोमा॑य॒ सोमा॑य॒ राज्ञे॑ ऽददात् । \newline
9. राज्ञे॑ ऽददा दददा॒द् राज्ञे॒ राज्ञे॑ ऽददा॒त् तासा॒म् तासा॑ मददा॒द् राज्ञे॒ राज्ञे॑ ऽददा॒त् तासा᳚म् । \newline
10. अ॒द॒दा॒त् तासा॒म् तासा॑ मददा दददा॒त् तासाꣳ॑ रोहि॒णीꣳ रो॑हि॒णीम् तासा॑ मददा दददा॒त् तासाꣳ॑ रोहि॒णीम् । \newline
11. तासाꣳ॑ रोहि॒णीꣳ रो॑हि॒णीम् तासा॒म् तासाꣳ॑ रोहि॒णी मुपोप॑ रोहि॒णीम् तासा॒म् तासाꣳ॑ रोहि॒णी मुप॑ । \newline
12. रो॒हि॒णी मुपोप॑ रोहि॒णीꣳ रो॑हि॒णी मुपै॑दै॒दुप॑ रोहि॒णीꣳ रो॑हि॒णी मुपै᳚त् । \newline
13. उपै॑दै॒ दुपोपै॒त् ता स्ता ऐ॒दुपोपै॒त् ताः । \newline
14. ऐ॒त् ता स्ता ऐ॑दै॒त् ता ईर्ष्य॑न्ती॒ रीर्ष्य॑न्ती॒ स्ता ऐ॑दै॒त् ता ईर्ष्य॑न्तीः । \newline
15. ता ईर्ष्य॑न्ती॒ रीर्ष्य॑न्ती॒ स्ता स्ता ईर्ष्य॑न्तीः॒ पुनः॒ पुन॒ रीर्ष्य॑न्ती॒ स्ता स्ता ईर्ष्य॑न्तीः॒ पुनः॑ । \newline
16. ईर्ष्य॑न्तीः॒ पुनः॒ पुन॒ रीर्ष्य॑न्ती॒ रीर्ष्य॑न्तीः॒ पुन॑ रगच्छन् नगच्छ॒न् पुन॒ रीर्ष्य॑न्ती॒ रीर्ष्य॑न्तीः॒ पुन॑ रगच्छन्न् । \newline
17. पुन॑ रगच्छन् नगच्छ॒न् पुनः॒ पुन॑ रगच्छ॒न् ता स्ता अ॑गच्छ॒न् पुनः॒ पुन॑ रगच्छ॒न् ताः । \newline
18. अ॒ग॒च्छ॒न् ता स्ता अ॑गच्छन् नगच्छ॒न् ता अन्वनु॒ ता अ॑गच्छन् नगच्छ॒न् ता अनु॑ । \newline
19. ता अन्वनु॒ ता स्ता अन्वै॑ दै॒दनु॒ ता स्ता अन्वै᳚त् । \newline
20. अन्वै॑ दै॒दन्वन्वै॒त् ता स्ता ऐ॒दन्वन्वै॒त् ताः । \newline
21. ऐ॒त् ता स्ता ऐ॑दै॒त् ताः पुनः॒ पुन॒ स्ता ऐ॑दै॒त् ताः पुनः॑ । \newline
22. ताः पुनः॒ पुन॒ स्ता स्ताः पुन॑ रयाचता याचत॒ पुन॒ स्ता स्ताः पुन॑ रयाचत । \newline
23. पुन॑ रयाचता याचत॒ पुनः॒ पुन॑ रयाचत॒ ता स्ता अ॑याचत॒ पुनः॒ पुन॑ रयाचत॒ ताः । \newline
24. अ॒या॒च॒त॒ ता स्ता अ॑याचता याचत॒ ता अ॑स्मा अस्मै॒ ता अ॑याचता याचत॒ ता अ॑स्मै । \newline
25. ता अ॑स्मा अस्मै॒ ता स्ता अ॑स्मै॒ न नास्मै॒ ता स्ता अ॑स्मै॒ न । \newline
26. अ॒स्मै॒ न नास्मा॑ अस्मै॒ न पुनः॒ पुन॒र् नास्मा॑ अस्मै॒ न पुनः॑ । \newline
27. न पुनः॒ पुन॒र् न न पुन॑ रददा दददा॒त् पुन॒र् न न पुन॑ रददात् । \newline
28. पुन॑ रददा दददा॒त् पुनः॒ पुन॑ रददा॒थ् स सो॑ ऽददा॒त् पुनः॒ पुन॑ रददा॒थ् सः । \newline
29. अ॒द॒दा॒थ् स सो॑ ऽददा दददा॒थ् सो᳚ ऽब्रवीदब्रवी॒थ् सो॑ ऽददा दददा॒थ् सो᳚ ऽब्रवीत् । \newline
30. सो᳚ ऽब्रवी दब्रवी॒थ् स सो᳚ ऽब्रवीदृ॒त मृ॒त म॑ब्रवी॒थ् स सो᳚ ऽब्रवीदृ॒तम् । \newline
31. अ॒ब्र॒वी॒दृ॒त मृ॒त म॑ब्रवी दब्रवीदृ॒त म॑मीष्वा मीष्व॒ र्त म॑ब्रवी दब्रवीदृ॒त म॑मीष्व । \newline
32. ऋ॒त म॑मीष्वामीष्व॒ र्त मृ॒त म॑मीष्व॒ यथा॒ यथा॑ ऽमीष्व॒ र्त मृ॒त म॑मीष्व॒ यथा᳚ । \newline
33. अ॒मी॒ष्व॒ यथा॒ यथा॑ ऽमीष्वामीष्व॒ यथा॑ समाव॒च्छः स॑माव॒च्छो यथा॑ ऽमीष्वा मीष्व॒ यथा॑ समाव॒च्छः । \newline
34. यथा॑ समाव॒च्छः स॑माव॒च्छो यथा॒ यथा॑ समाव॒च्छ उ॑पै॒ष्या म्यु॑पै॒ष्यामि॑ समाव॒च्छो यथा॒ यथा॑ समाव॒च्छ उ॑पै॒ष्यामि॑ । \newline
35. स॒मा॒व॒च्छ उ॑पै॒ष्या म्यु॑पै॒ष्यामि॑ समाव॒च्छः स॑माव॒च्छ उ॑पै॒ष्या म्यथाथो॑ पै॒ष्यामि॑ समाव॒च्छः स॑माव॒च्छ उ॑पै॒ष्याम्यथ॑ । \newline
36. स॒मा॒व॒च्छ इति॑ समावत् - शः । \newline
37. उ॒पै॒ष्या म्यथाथो॑ पै॒ष्या म्यु॑पै॒ष्याम्यथ॑ ते॒ ते ऽथो॑पै॒ष्या म्यु॑पै॒ष्याम्यथ॑ ते । \newline
38. उ॒पै॒ष्यामीत्यु॑प - ए॒ष्यामि॑ । \newline
39. अथ॑ ते॒ ते ऽथाथ॑ ते॒ पुनः॒ पुन॒ स्ते ऽथाथ॑ ते॒ पुनः॑ । \newline
40. ते॒ पुनः॒ पुन॑ स्ते ते॒ पुन॑र् दास्यामि दास्यामि॒ पुन॑ स्ते ते॒ पुन॑र् दास्यामि । \newline
41. पुन॑र् दास्यामि दास्यामि॒ पुनः॒ पुन॑र् दास्या॒मीतीति॑ दास्यामि॒ पुनः॒ पुन॑र् दास्या॒मीति॑ । \newline
42. दा॒स्या॒मीतीति॑ दास्यामि दास्या॒मीति॒ स स इति॑ दास्यामि दास्या॒मीति॒ सः । \newline
43. इति॒ स स इतीति॒ स ऋ॒त मृ॒तꣳ स इतीति॒ स ऋ॒तम् । \newline
44. स ऋ॒त मृ॒तꣳ स स ऋ॒त मा॑मी दामीदृ॒तꣳ स स ऋ॒त मा॑मीत् । \newline
45. ऋ॒त मा॑मी दामीदृ॒त मृ॒त मा॑मी॒त् ता स्ता आ॑मीदृ॒त मृ॒त मा॑मी॒त् ताः । \newline
46. आ॒मी॒त् ता स्ता आ॑मी दामी॒त् ता अ॑स्मा अस्मै॒ ता आ॑मी दामी॒त् ता अ॑स्मै । \newline
47. ता अ॑स्मा अस्मै॒ ता स्ता अ॑स्मै॒ पुनः॒ पुन॑ रस्मै॒ ता स्ता अ॑स्मै॒ पुनः॑ । \newline
48. अ॒स्मै॒ पुनः॒ पुन॑ रस्मा अस्मै॒ पुन॑ रददा दददा॒त् पुन॑ रस्मा अस्मै॒ पुन॑ रददात् । \newline
49. पुन॑ रददा दददा॒त् पुनः॒ पुन॑ रददा॒त् तासा॒म् तासा॑ मददा॒त् पुनः॒ पुन॑ रददा॒त् तासा᳚म् । \newline
50. अ॒द॒दा॒त् तासा॒म् तासा॑ मददा दददा॒त् तासाꣳ॑ रोहि॒णीꣳ रो॑हि॒णीम् तासा॑ मददा दददा॒त् तासाꣳ॑ रोहि॒णीम् । \newline
51. तासाꣳ॑ रोहि॒णीꣳ रो॑हि॒णीम् तासा॒म् तासाꣳ॑ रोहि॒णी मे॒वैव रो॑हि॒णीम् तासा॒म् तासाꣳ॑ रोहि॒णी मे॒व । \newline
52. रो॒हि॒णी मे॒वैव रो॑हि॒णीꣳ रो॑हि॒णी मे॒वोपोपै॒व रो॑हि॒णीꣳ रो॑हि॒णी मे॒वोप॑ । \newline
53. ए॒वोपो पै॒वैवो पै॑दै॒ दुपै॒ वैवोपै᳚त् । \newline
54. उपै॑ दै॒दुपोपै॒त् तम् त मै॒दुपोपै॒त् तम् । \newline
\pagebreak
\markright{ TS 2.3.5.2  \hfill https://www.vedavms.in \hfill}

\section{ TS 2.3.5.2 }

\textbf{TS 2.3.5.2 } \newline
\textbf{Samhita Paata} \newline

तं ॅयक्ष्म॑ आर्च्छ॒द्-राजा॑नं॒ ॅयक्ष्म॑ आर॒दिति॒ तद्-रा॑जय॒क्ष्मस्य॒ जन्म॒ यत् पापी॑या॒नभ॑व॒त् तत् पा॑पय॒क्ष्मस्य॒ यज्जा॒याभ्योऽवि॑न्द॒त् तज्जा॒येन्य॑स्य॒य ए॒वमे॒तेषां॒ ॅयक्ष्मा॑णां॒ जन्म॒ वेद॒ नैन॑मे॒ते यक्ष्मा॑ विन्दन्ति॒स ए॒ता ए॒व न॑म॒स्यन्नुपा॑धाव॒त् ता अ॑ब्रुव॒न्. वरं॑ ॅवृणामहै समाव॒च्छ ए॒व न॒ उपा॑य॒ इति॒ तस्मा॑ ए॒त - [  ] \newline

\textbf{Pada Paata} \newline

ऐ॒त् । तम् । यक्ष्मः॑ । आ॒र्च्छ॒त् । राजा॑नम् । यक्ष्मः॑ । आ॒र॒त् । इति॑ । तत् । रा॒ज॒य॒क्ष्मस्येति॑ राज -य॒क्ष्मस्य॑ । जन्म॑ । यत् । पापी॑यान् । अभ॑वत् । तत् । पा॒प॒य॒क्ष्मस्येति॑ पाप - य॒क्ष्मस्य॑ । यत् । जा॒याभ्यः॑ । अवि॑न्दत् । तत् । जा॒येन्य॑स्य । यः । ए॒वम् । ए॒तेषा᳚म् । यक्ष्मा॑णाम् । जन्म॑ । वेद॑ । न ।  ए॒न॒म् । ए॒ते । यक्ष्माः᳚ । वि॒न्द॒न्ति॒ । सः । ए॒ताः । ए॒व । न॒म॒स्यन्न् । उपेति॑ । अ॒धा॒व॒त् । ताः । अ॒ब्रु॒व॒न्न् । वर᳚म् । वृ॒णा॒म॒है॒ । स॒मा॒व॒च्छ इति॑ समावत् - शः । ए॒व । नः॒ । उपेति॑ । अ॒यः॒ । इति॑ । तस्मै᳚ । ए॒तम् ।  \newline


\textbf{Krama Paata} \newline

ऐ॒त् तम् । तं ॅयक्ष्मः॑ । यक्ष्म॑ आर्छत् । आ॒र्छ॒द् राजा॑नम् । राजा॑नं॒ ॅयक्ष्मः॑ । यक्ष्म॑ आरत् । आ॒र॒दिति॑ । इति॒ तत् । तद् रा॑जय॒क्ष्मस्य॑ । रा॒ज॒य॒क्ष्मस्य॒ जन्म॑ । रा॒ज॒य॒क्ष्मस्येति॑ राज - य॒क्ष्मस्य॑ । जन्म॒ यत् । यत् पापी॑यान् । पापी॑या॒नभ॑वत् । अभ॑व॒त् तत् । तत् पा॑पय॒क्ष्मस्य॑ । पा॒प॒य॒क्ष्मस्य॒ यत् । पा॒प॒य॒क्ष्मस्येति॑ पाप - य॒क्ष्मस्य॑ । यज्जा॒याभ्यः॑ । जा॒याभ्यो ऽवि॑न्दत् । अवि॑न्द॒त् तत् । तज्जा॒येन्य॑स्य । जा॒येन्य॑स्य॒ यः । य ए॒वम् । ए॒वमे॒तेषा᳚म् । ए॒तेषां॒ ॅयक्ष्मा॑णाम् । यक्ष्मा॑णा॒म् जन्म॑ । जन्म॒ वेद॑ । वेद॒ न । नैन᳚म् । ए॒न॒मे॒ते । ए॒ते यक्ष्माः᳚ । यक्ष्मा॑ विन्दन्ति । वि॒न्द॒न्ति॒ सः । स ए॒ताः । ए॒ता ए॒व । ए॒व न॑म॒स्यन्न् । न॒म॒स्यन्नुप॑ । उपा॑धावत् । अ॒धा॒व॒त् ताः । ता अ॑ब्रुवन्न् । अ॒ब्रु॒व॒न् वर᳚म् । वरं॑ ॅवृणामहै । वृ॒णा॒म॒है॒ स॒मा॒व॒च्छः । स॒मा॒व॒च्छ ए॒व । स॒मा॒व॒च्छ इति॑ समावत् - शः । ए॒व नः॑ । न॒ उप॑ । उपा॑यः । अ॒य॒ इति॑ । इति॒ तस्मै᳚ । तस्मा॑ ए॒तम् । ए॒तमा॑दि॒त्यम् \newline

\textbf{Jatai Paata} \newline

1. ऐ॒त् तम् त मै॑दै॒त् तम् । \newline
2. तं ॅयक्ष्मो॒ यक्ष्म॒ स्तम् तं ॅयक्ष्मः॑ । \newline
3. यक्ष्म॑ आर्च्छ दार्च्छ॒द् यक्ष्मो॒ यक्ष्म॑ आर्च्छत् । \newline
4. आ॒र्च्छ॒द् राजा॑नꣳ॒॒ राजा॑न मार्च्छ दार्च्छ॒द् राजा॑नम् । \newline
5. राजा॑नं॒ ॅयक्ष्मो॒ यक्ष्मो॒ राजा॑नꣳ॒॒ राजा॑नं॒ ॅयक्ष्मः॑ । \newline
6. यक्ष्म॑ आर दार॒द् यक्ष्मो॒ यक्ष्म॑ आरत् । \newline
7. आ॒र॒ दिती त्या॑र दार॒ दिति॑ । \newline
8. इति॒ तत् तदितीति॒ तत् । \newline
9. तद् रा॑जय॒क्ष्मस्य॑ राजय॒क्ष्मस्य॒ तत् तद् रा॑जय॒क्ष्मस्य॑ । \newline
10. रा॒ज॒य॒क्ष्मस्य॒ जन्म॒ जन्म॑ राजय॒क्ष्मस्य॑ राजय॒क्ष्मस्य॒ जन्म॑ । \newline
11. रा॒ज॒य॒क्ष्मस्येति॑ राज - य॒क्ष्मस्य॑ । \newline
12. जन्म॒ यद् यज् जन्म॒ जन्म॒ यत् । \newline
13. यत् पापी॑या॒न् पापी॑या॒न्.॒ यद् यत् पापी॑यान् । \newline
14. पापी॑या॒ नभ॑व॒ दभ॑व॒त् पापी॑या॒न् पापी॑या॒ नभ॑वत् । \newline
15. अभ॑व॒त् तत् तदभ॑व॒ दभ॑व॒त् तत् । \newline
16. तत् पा॑पय॒क्ष्मस्य॑ पापय॒क्ष्मस्य॒ तत् तत् पा॑पय॒क्ष्मस्य॑ । \newline
17. पा॒प॒य॒क्ष्मस्य॒ यद् यत् पा॑पय॒क्ष्मस्य॑ पापय॒क्ष्मस्य॒ यत् । \newline
18. पा॒प॒य॒क्ष्मस्येति॑ पाप - य॒क्ष्मस्य॑ । \newline
19. यज् जा॒याभ्यो॑ जा॒याभ्यो॒ यद् यज् जा॒याभ्यः॑ । \newline
20. जा॒याभ्यो ऽवि॑न्द॒ दवि॑न्दज् जा॒याभ्यो॑ जा॒याभ्यो ऽवि॑न्दत् । \newline
21. अवि॑न्द॒त् तत् तदवि॑न्द॒ दवि॑न्द॒त् तत् । \newline
22. तज् जा॒येन्य॑स्य जा॒येन्य॑स्य॒ तत् तज् जा॒येन्य॑स्य । \newline
23. जा॒येन्य॑स्य॒ यो यो जा॒येन्य॑स्य जा॒येन्य॑स्य॒ यः । \newline
24. य ए॒व मे॒वं ॅयो य ए॒वम् । \newline
25. ए॒व मे॒तेषा॑ मे॒तेषा॑ मे॒व मे॒व मे॒तेषा᳚म् । \newline
26. ए॒तेषां॒ ॅयक्ष्मा॑णां॒ ॅयक्ष्मा॑णा मे॒तेषा॑ मे॒तेषां॒ ॅयक्ष्मा॑णाम् । \newline
27. यक्ष्मा॑णा॒म् जन्म॒ जन्म॒ यक्ष्मा॑णां॒ ॅयक्ष्मा॑णा॒म् जन्म॑ । \newline
28. जन्म॒ वेद॒ वेद॒ जन्म॒ जन्म॒ वेद॑ । \newline
29. वेद॒ न न वेद॒ वेद॒ न । \newline
30. नैन॑ मेन॒म् न नैन᳚म् । \newline
31. ए॒न॒ मे॒त ए॒त ए॑न मेन मे॒ते । \newline
32. ए॒ते यक्ष्मा॒ यक्ष्मा॑ ए॒त ए॒ते यक्ष्माः᳚ । \newline
33. यक्ष्मा॑ विन्दन्ति विन्दन्ति॒ यक्ष्मा॒ यक्ष्मा॑ विन्दन्ति । \newline
34. वि॒न्द॒न्ति॒ स स वि॑न्दन्ति विन्दन्ति॒ सः । \newline
35. स ए॒ता ए॒ताः स स ए॒ताः । \newline
36. ए॒ता ए॒वैवैता ए॒ता ए॒व । \newline
37. ए॒व न॑म॒स्यन् न॑म॒स्यन् ने॒वैव न॑म॒स्यन्न् । \newline
38. न॒म॒स्यन् नुपोप॑ नम॒स्यन् न॑म॒स्यन् नुप॑ । \newline
39. उपा॑धाव दधाव॒ दुपोपा॑ धावत् । \newline
40. अ॒धा॒व॒त् तास्ता अ॑धाव दधाव॒त् ताः । \newline
41. ता अ॑ब्रुवन् नब्रुव॒न् ता स्ता अ॑ब्रुवन्न् । \newline
42. अ॒ब्रु॒व॒न्॒. वरं॒ ॅवर॑ मब्रुवन् नब्रुव॒न्॒. वर᳚म् । \newline
43. वरं॑ ॅवृणामहै वृणामहै॒ वरं॒ ॅवरं॑ ॅवृणामहै । \newline
44. वृ॒णा॒म॒है॒ स॒मा॒व॒च्छः स॑माव॒च्छो वृ॑णामहै वृणामहै समाव॒च्छः । \newline
45. स॒मा॒व॒च्छ ए॒वैव स॑माव॒च्छः स॑माव॒च्छ ए॒व । \newline
46. स॒मा॒व॒च्छ इति॑ समावत् - शः । \newline
47. ए॒व नो॑ न ए॒वैव नः॑ । \newline
48. न॒ उपोप॑ नो न॒ उप॑ । \newline
49. उपा॑यो ऽय॒ उपोपा॑यः । \newline
50. अ॒य॒ इती त्य॑यो ऽय॒ इति॑ । \newline
51. इति॒ तस्मै॒ तस्मा॒ इतीति॒ तस्मै᳚ । \newline
52. तस्मा॑ ए॒त मे॒तम् तस्मै॒ तस्मा॑ ए॒तम् । \newline
53. ए॒त मा॑दि॒त्य मा॑दि॒त्य मे॒त मे॒त मा॑दि॒त्यम् । \newline

\textbf{Ghana Paata } \newline

1. ऐ॒त् तम् त मै॑दै॒त् तं ॅयक्ष्मो॒ यक्ष्म॒ स्त मै॑दै॒त् तं ॅयक्ष्मः॑ । \newline
2. तं ॅयक्ष्मो॒ यक्ष्म॒ स्तम् तं ॅयक्ष्म॑ आर्च्छ दार्च्छ॒द् यक्ष्म॒ स्तम् तं ॅयक्ष्म॑ आर्च्छत् । \newline
3. यक्ष्म॑ आर्च्छ दार्च्छ॒द् यक्ष्मो॒ यक्ष्म॑ आर्च्छ॒द् राजा॑नꣳ॒॒ राजा॑न मार्च्छ॒द् यक्ष्मो॒ यक्ष्म॑ आर्च्छ॒द् राजा॑नम् । \newline
4. आ॒र्च्छ॒द् राजा॑नꣳ॒॒ राजा॑न मार्च्छ दार्च्छ॒द् राजा॑नं॒ ॅयक्ष्मो॒ यक्ष्मो॒ राजा॑न मार्च्छ दार्च्छ॒द् राजा॑नं॒ ॅयक्ष्मः॑ । \newline
5. राजा॑नं॒ ॅयक्ष्मो॒ यक्ष्मो॒ राजा॑नꣳ॒॒ राजा॑नं॒ ॅयक्ष्म॑ आर दार॒द् यक्ष्मो॒ राजा॑नꣳ॒॒ राजा॑नं॒ ॅयक्ष्म॑ आरत् । \newline
6. यक्ष्म॑ आर दार॒द् यक्ष्मो॒ यक्ष्म॑ आर॒दिती त्या॑र॒द् यक्ष्मो॒ यक्ष्म॑ आर॒दिति॑ । \newline
7. आ॒र॒ दिती त्या॑र दार॒दिति॒ तत् तदि त्या॑र दार॒दिति॒ तत् । \newline
8. इति॒ तत् तदितीति॒ तद् रा॑जय॒क्ष्मस्य॑ राजय॒क्ष्मस्य॒ तदितीति॒ तद् रा॑जय॒क्ष्मस्य॑ । \newline
9. तद् रा॑जय॒क्ष्मस्य॑ राजय॒क्ष्मस्य॒ तत् तद् रा॑जय॒क्ष्मस्य॒ जन्म॒ जन्म॑ राजय॒क्ष्मस्य॒ तत् तद् रा॑जय॒क्ष्मस्य॒ जन्म॑ । \newline
10. रा॒ज॒य॒क्ष्मस्य॒ जन्म॒ जन्म॑ राजय॒क्ष्मस्य॑ राजय॒क्ष्मस्य॒ जन्म॒ यद् यज् जन्म॑ राजय॒क्ष्मस्य॑ राजय॒क्ष्मस्य॒ जन्म॒ यत् । \newline
11. रा॒ज॒य॒क्ष्मस्येति॑ राज - य॒क्ष्मस्य॑ । \newline
12. जन्म॒ यद् यज् जन्म॒ जन्म॒ यत् पापी॑या॒न् पापी॑या॒न्॒. यज् जन्म॒ जन्म॒ यत् पापी॑यान् । \newline
13. यत् पापी॑या॒न् पापी॑या॒न्॒. यद् यत् पापी॑या॒ नभ॑व॒ दभ॑व॒त् पापी॑या॒न्॒. यद् यत् पापी॑या॒ नभ॑वत् । \newline
14. पापी॑या॒ नभ॑व॒ दभ॑व॒त् पापी॑या॒न् पापी॑या॒ नभ॑व॒त् तत् तदभ॑व॒त् पापी॑या॒न् पापी॑या॒ नभ॑व॒त् तत् । \newline
15. अभ॑व॒त् तत् तदभ॑व॒ दभ॑व॒त् तत् पा॑पय॒क्ष्मस्य॑ पापय॒क्ष्मस्य॒ तदभ॑व॒ दभ॑व॒त् तत् पा॑पय॒क्ष्मस्य॑ । \newline
16. तत् पा॑पय॒क्ष्मस्य॑ पापय॒क्ष्मस्य॒ तत् तत् पा॑पय॒क्ष्मस्य॒ यद् यत् पा॑पय॒क्ष्मस्य॒ तत् तत् पा॑पय॒क्ष्मस्य॒ यत् । \newline
17. पा॒प॒य॒क्ष्मस्य॒ यद् यत् पा॑पय॒क्ष्मस्य॑ पापय॒क्ष्मस्य॒ यज् जा॒याभ्यो॑ जा॒याभ्यो॒ यत् पा॑पय॒क्ष्मस्य॑ पापय॒क्ष्मस्य॒ यज् जा॒याभ्यः॑ । \newline
18. पा॒प॒य॒क्ष्मस्येति॑ पाप - य॒क्ष्मस्य॑ । \newline
19. यज् जा॒याभ्यो॑ जा॒याभ्यो॒ यद् यज् जा॒याभ्यो ऽवि॑न्द॒ दवि॑न्दज् जा॒याभ्यो॒ यद् यज् जा॒याभ्यो ऽवि॑न्दत् । \newline
20. जा॒याभ्यो ऽवि॑न्द॒ दवि॑न्दज् जा॒याभ्यो॑ जा॒याभ्यो ऽवि॑न्द॒त् तत् तदवि॑न्दज् जा॒याभ्यो॑ जा॒याभ्यो ऽवि॑न्द॒त् तत् । \newline
21. अवि॑न्द॒त् तत् तदवि॑न्द॒ दवि॑न्द॒त् तज् जा॒येन्य॑स्य जा॒येन्य॑स्य॒ तदवि॑न्द॒ दवि॑न्द॒त् तज् जा॒येन्य॑स्य । \newline
22. तज् जा॒येन्य॑स्य जा॒येन्य॑स्य॒ तत् तज् जा॒येन्य॑स्य॒ यो यो जा॒येन्य॑स्य॒ तत् तज् जा॒येन्य॑स्य॒ यः । \newline
23. जा॒येन्य॑स्य॒ यो यो जा॒येन्य॑स्य जा॒येन्य॑स्य॒ य ए॒व मे॒वं ॅयो जा॒येन्य॑स्य जा॒येन्य॑स्य॒ य ए॒वम् । \newline
24. य ए॒व मे॒वं ॅयो य ए॒व मे॒तेषा॑ मे॒तेषा॑ मे॒वं ॅयो य ए॒व मे॒तेषा᳚म् । \newline
25. ए॒व मे॒तेषा॑ मे॒तेषा॑ मे॒व मे॒व मे॒तेषां॒ ॅयक्ष्मा॑णां॒ ॅयक्ष्मा॑णा मे॒तेषा॑ मे॒व मे॒व मे॒तेषां॒ ॅयक्ष्मा॑णाम् । \newline
26. ए॒तेषां॒ ॅयक्ष्मा॑णां॒ ॅयक्ष्मा॑णा मे॒तेषा॑ मे॒तेषां॒ ॅयक्ष्मा॑णा॒म् जन्म॒ जन्म॒ यक्ष्मा॑णा मे॒तेषा॑ मे॒तेषां॒ ॅयक्ष्मा॑णा॒म् जन्म॑ । \newline
27. यक्ष्मा॑णा॒म् जन्म॒ जन्म॒ यक्ष्मा॑णां॒ ॅयक्ष्मा॑णा॒म् जन्म॒ वेद॒ वेद॒ जन्म॒ यक्ष्मा॑णां॒ ॅयक्ष्मा॑णा॒म् जन्म॒ वेद॑ । \newline
28. जन्म॒ वेद॒ वेद॒ जन्म॒ जन्म॒ वेद॒ न न वेद॒ जन्म॒ जन्म॒ वेद॒ न । \newline
29. वेद॒ न न वेद॒ वेद॒ नैन॑ मेन॒म् न वेद॒ वेद॒ नैन᳚म् । \newline
30. नैन॑ मेन॒म् न नैन॑ मे॒त ए॒त ए॑न॒म् न नैन॑ मे॒ते । \newline
31. ए॒न॒ मे॒त ए॒त ए॑न मेन मे॒ते यक्ष्मा॒ यक्ष्मा॑ ए॒त ए॑न मेन मे॒ते यक्ष्माः᳚ । \newline
32. ए॒ते यक्ष्मा॒ यक्ष्मा॑ ए॒त ए॒ते यक्ष्मा॑ विन्दन्ति विन्दन्ति॒ यक्ष्मा॑ ए॒त ए॒ते यक्ष्मा॑ विन्दन्ति । \newline
33. यक्ष्मा॑ विन्दन्ति विन्दन्ति॒ यक्ष्मा॒ यक्ष्मा॑ विन्दन्ति॒ स स वि॑न्दन्ति॒ यक्ष्मा॒ यक्ष्मा॑ विन्दन्ति॒ सः । \newline
34. वि॒न्द॒न्ति॒ स स वि॑न्दन्ति विन्दन्ति॒ स ए॒ता ए॒ताः स वि॑न्दन्ति विन्दन्ति॒ स ए॒ताः । \newline
35. स ए॒ता ए॒ताः स स ए॒ता ए॒वैवैताः स स ए॒ता ए॒व । \newline
36. ए॒ता ए॒वैवैता ए॒ता ए॒व न॑म॒स्यन् न॑म॒स्यन् ने॒वैता ए॒ता ए॒व न॑म॒स्यन्न् । \newline
37. ए॒व न॑म॒स्यन् न॑म॒स्यन् ने॒वैव न॑म॒स्यन् नुपोप॑ नम॒स्यन् ने॒वैव न॑म॒स्यन् नुप॑ । \newline
38. न॒म॒स्यन् नुपोप॑ नम॒स्यन् न॑म॒स्यन् नुपा॑धाव दधाव॒ दुप॑ नम॒स्यन् न॑म॒स्यन् नुपा॑धावत् । \newline
39. उपा॑धाव दधाव॒ दुपोपा॑धाव॒त् ता स्ता अ॑धाव॒ दुपोपा॑धाव॒त् ताः । \newline
40. अ॒धा॒व॒त् ता स्ता अ॑धाव दधाव॒त् ता अ॑ब्रुवन् नब्रुव॒न् ता अ॑धाव दधाव॒त् ता अ॑ब्रुवन्न् । \newline
41. ता अ॑ब्रुवन् नब्रुव॒न् ता स्ता अ॑ब्रुव॒न्॒. वरं॒ ॅवर॑ मब्रुव॒न् ता स्ता अ॑ब्रुव॒न्॒. वर᳚म् । \newline
42. अ॒ब्रु॒व॒न्॒. वरं॒ ॅवर॑ मब्रुवन् नब्रुव॒न्॒. वरं॑ ॅवृणामहै वृणामहै॒ वर॑ मब्रुवन् नब्रुव॒न्॒. वरं॑ ॅवृणामहै । \newline
43. वरं॑ ॅवृणामहै वृणामहै॒ वरं॒ ॅवरं॑ ॅवृणामहै समाव॒च्छः स॑माव॒च्छो वृ॑णामहै॒ वरं॒ ॅवरं॑ ॅवृणामहै समाव॒च्छः । \newline
44. वृ॒णा॒म॒है॒ स॒मा॒व॒च्छः स॑माव॒च्छो वृ॑णामहै वृणामहै समाव॒च्छ ए॒वैव स॑माव॒च्छो वृ॑णामहै वृणामहै समाव॒च्छ ए॒व । \newline
45. स॒मा॒व॒च्छ ए॒वैव स॑माव॒च्छः स॑माव॒च्छ ए॒व नो॑ न ए॒व स॑माव॒च्छः स॑माव॒च्छ ए॒व नः॑ । \newline
46. स॒मा॒व॒च्छ इति॑ समावत् - शः । \newline
47. ए॒व नो॑ न ए॒वैव न॒ उपोप॑ न ए॒वैव न॒ उप॑ । \newline
48. न॒ उपोप॑ नो न॒ उपा॑यो ऽय॒ उप॑ नो न॒ उपा॑यः । \newline
49. उपा॑यो ऽय॒ उपोपा॑य॒ इतीत्य॑य॒ उपोपा॑य॒ इति॑ । \newline
50. अ॒य॒ इतीत्य॑यो ऽय॒ इति॒ तस्मै॒ तस्मा॒ इत्य॑यो ऽय॒ इति॒ तस्मै᳚ । \newline
51. इति॒ तस्मै॒ तस्मा॒ इतीति॒ तस्मा॑ ए॒त मे॒तम् तस्मा॒ इतीति॒ तस्मा॑ ए॒तम् । \newline
52. तस्मा॑ ए॒त मे॒तम् तस्मै॒ तस्मा॑ ए॒त मा॑दि॒त्य मा॑दि॒त्य मे॒तम् तस्मै॒ तस्मा॑ ए॒त मा॑दि॒त्यम् । \newline
53. ए॒त मा॑दि॒त्य मा॑दि॒त्य मे॒त मे॒त मा॑दि॒त्यम् च॒रुम् च॒रु मा॑दि॒त्य मे॒त मे॒त मा॑दि॒त्यम् च॒रुम् । \newline
\pagebreak
\markright{ TS 2.3.5.3  \hfill https://www.vedavms.in \hfill}

\section{ TS 2.3.5.3 }

\textbf{TS 2.3.5.3 } \newline
\textbf{Samhita Paata} \newline

-मा॑दि॒त्यं च॒रुं निर॑वप॒न् तेनै॒वैनं॑ पा॒पाथ् स्रामा॑दमुञ्च॒न्. यः पा॑पय॒क्ष्मगृ॑हीतः॒ स्यात् तस्मा॑ ए॒तमा॑दि॒त्यं च॒रुं निर्व॑पेदादि॒त्याने॒व स्वेन॑ भाग॒धेये॒नोप॑ धावति॒ त ए॒वैनं॑ पा॒पाथ् स्रामा᳚न्मुञ्चन्त्य-मावा॒स्या॑यां॒ निर्व॑पेद॒मुमे॒वैन-॑मा॒प्याय॑मान॒-मन्वा प्या॑ययति॒ नवो॑नवो भवति॒ जाय॑मान॒ इति॑ पुरोऽनुवा॒क्या॑ भव॒त्यायु॑रे॒वास्मि॒न् ( ) तया॑ दधाति॒ यमा॑दि॒त्या अꣳ॒॒शुमा᳚प्या॒यय॒न्तीति॑ या॒ज्यैवैन॑मे॒तया᳚ प्याययति ॥ \newline

\textbf{Pada Paata} \newline

आ॒दि॒त्यम् । च॒रुम् । निरिति॑ । अ॒व॒प॒न्न् । तेन॑ । ए॒व । ए॒न॒म् । पा॒पात् । स्रामा᳚त् । अ॒मु॒ञ्च॒न्न् । यः । पा॒प॒य॒क्ष्मगृ॑हीत॒ इति॑ पापय॒क्ष्म - गृ॒ही॒तः॒ । स्यात् । तस्मै᳚ । ए॒तम् । आ॒दि॒त्यम् । च॒रुम् । निरिति॑ । व॒पे॒त् । आ॒दि॒त्यान् । ए॒व । स्वेन॑ । भा॒ग॒धेये॒नेति॑ भाग - धेये॑न । उपेति॑ । ध॒व॒ति॒ । ते । ए॒व । ए॒न॒म् । पा॒पात् । स्रामा᳚त् । मु॒ञ्च॒न्ति॒ । अ॒मा॒वा॒स्या॑या॒मित्य॑मा - वा॒स्या॑याम् । निरिति॑ । व॒पे॒त् । अ॒मुम् । ए॒व । ए॒न॒म् । आ॒प्याय॑मान॒मित्या᳚ - प्याय॑मानम् । अनु॑ । एति॑ । प्या॒य॒य॒ति॒ । नवो॑नव॒ इति॒ नवः॑ - न॒वः॒ । भ॒व॒ति॒ । जाय॑मानः । इति॑ । पु॒रो॒नु॒वा॒क्येति॑ पुरः - अ॒नु॒वा॒क्या᳚ । भ॒व॒ति॒ । आयुः॑ । ए॒व । अ॒स्मि॒न्न् ( ) । तया᳚ । द॒धा॒ति॒ । यम् । आ॒दि॒त्याः । अꣳ॒॒शुम् । आ॒प्या॒यय॒न्तीत्या᳚ - प्या॒यय॑न्ति । इति॑ । या॒ज्या᳚ । एति॑ । ए॒व । ए॒न॒म् ।  ए॒तया᳚ । प्या॒य॒य॒ति॒ ॥  \newline


\textbf{Krama Paata} \newline

आ॒दि॒त्यम् च॒रुम् । च॒रुम् निः । निर॑वपन्न् । अ॒व॒प॒न् तेन॑ । तेनै॒व । ए॒वैन᳚म् । ए॒न॒म् पा॒पात् । पा॒पाथ् स्रामा᳚त् । स्रामा॑दमुञ्चन्न् । अ॒मु॒ञ्च॒न् यः । यः पा॑पय॒क्ष्मगृ॑हीतः । पा॒प॒य॒क्ष्मगृ॑हीतः॒ स्यात् । पा॒प॒य॒क्ष्मगृ॑हीत॒ इति॑ पापय॒क्ष्म - गृ॒ही॒तः॒ । स्यात् तस्मै᳚ । तस्मा॑ ए॒तम् । ए॒तमा॑दि॒त्यम् । आ॒दि॒त्यम् च॒रुम् । च॒रुम् निः । निर् व॑पेत् । व॒पे॒दा॒दि॒त्यान् । आ॒दि॒त्याने॒व । ए॒व स्वेन॑ । स्वेन॑ भाग॒धेये॑न । भा॒ग॒धेये॒नोप॑ । भा॒ग॒धेये॒नेति॑ भाग - धेये॑न । उप॑ धावति । धा॒व॒ति॒ ते । त ए॒व । ए॒वैन᳚म् । ए॒न॒म् पा॒पात् । पा॒पाथ् स्रामा᳚त् । स्रामा᳚न् मुञ्चन्ति । मु॒ञ्च॒न्त्य॒मा॒वा॒स्या॑याम् । अ॒मा॒वा॒स्या॑या॒म् निः । अ॒मा॒वा॒स्या॑या॒ मित्य॑मा - वा॒स्या॑याम् । निर् व॑पेत् । व॒पे॒द॒मुम् । अ॒मुमे॒व । ए॒वैन᳚म् । ए॒न॒मा॒प्याय॑मानम् । आ॒प्याय॑मान॒मनु॑ । आ॒प्याय॑मान॒मित्या᳚ - प्याय॑मानम् । अन्वा । आ प्या॑ययति । प्या॒य॒य॒ति॒ नवो॑नवः । नवो॑नवो भवति । नवो॑नव॒ इति॒ नवः॑ - न॒वः॒ । भ॒व॒ति॒ जाय॑मानः । जाय॑मान॒ इति॑ । इति॑ पुरोनुवा॒क्या᳚ । पु॒रो॒नु॒वा॒क्या॑ भवति । पु॒रो॒नु॒वा॒क्येति॑ पुरः - अ॒नु॒वा॒क्या᳚ । भ॒व॒त्यायुः॑ । आयु॑रे॒व । ए॒वास्मिन्न्॑ । अ॒स्मि॒न् तया᳚ । तया॑ दधाति ( ) । द॒धा॒ति॒ यम् । यमा॑दि॒त्याः । आ॒दि॒त्या अꣳ॒॒शुम् । अꣳ॒॒शुमा᳚प्या॒यय॑न्ति । आ॒प्या॒यय॒न्तीति॑ । आ॒प्या॒यय॒न्तीत्या᳚ - प्या॒यय॑न्ति । इति॑ या॒ज्या᳚ । या॒ज्या । ऐव । ए॒वैन᳚म् । ए॒न॒मे॒तया᳚ । ए॒तया᳚ प्याययति । प्या॒य॒य॒तीति॑ प्याययति । \newline

\textbf{Jatai Paata} \newline

1. आ॒दि॒त्यम् च॒रुम् च॒रु मा॑दि॒त्य मा॑दि॒त्यम् च॒रुम् । \newline
2. च॒रुम् निर् णिश्च॒रुम् च॒रुम् निः । \newline
3. निर॑वपन् नवप॒न् निर् णि र॑वपन्न् । \newline
4. अ॒व॒प॒न् तेन॒ तेना॑वपन् नवप॒न् तेन॑ । \newline
5. तेनै॒वैव तेन॒ तेनै॒व । \newline
6. ए॒वैन॑ मेन मे॒वैवैन᳚म् । \newline
7. ए॒न॒म् पा॒पात् पा॒पा दे॑न मेनम् पा॒पात् । \newline
8. पा॒पाथ् स्रामा॒थ् स्रामा᳚त् पा॒पात् पा॒पाथ् स्रामा᳚त् । \newline
9. स्रामा॑ दमुञ्चन् नमुञ्च॒न् थ्स्रामा॒थ् स्रामा॑ दमुञ्चन्न् । \newline
10. अ॒मु॒ञ्च॒न्॒. यो यो॑ ऽमुञ्चन् नमुञ्च॒न्॒. यः । \newline
11. यः पा॑पय॒क्ष्मगृ॑हीतः पापय॒क्ष्मगृ॑हीतो॒ यो यः पा॑पय॒क्ष्मगृ॑हीतः । \newline
12. पा॒प॒य॒क्ष्मगृ॑हीतः॒ स्याथ् स्यात् पा॑पय॒क्ष्मगृ॑हीतः पापय॒क्ष्मगृ॑हीतः॒ स्यात् । \newline
13. पा॒प॒य॒क्ष्मगृ॑हीत॒ इति॑ पापय॒क्ष्म - गृ॒ही॒तः॒ । \newline
14. स्यात् तस्मै॒ तस्मै॒ स्याथ् स्यात् तस्मै᳚ । \newline
15. तस्मा॑ ए॒त मे॒तम् तस्मै॒ तस्मा॑ ए॒तम् । \newline
16. ए॒त मा॑दि॒त्य मा॑दि॒त्य मे॒त मे॒त मा॑दि॒त्यम् । \newline
17. आ॒दि॒त्यम् च॒रुम् च॒रु मा॑दि॒त्य मा॑दि॒त्यम् च॒रुम् । \newline
18. च॒रुम् निर् णिश्च॒रुम् च॒रुम् निः । \newline
19. निर् व॑पेद् वपे॒न् निर् णिर् व॑पेत् । \newline
20. व॒पे॒ दा॒दि॒त्या ना॑दि॒त्यान्. व॑पेद् वपे दादि॒त्यान् । \newline
21. आ॒दि॒त्या ने॒वैवादि॒त्या ना॑दि॒त्या ने॒व । \newline
22. ए॒व स्वेन॒ स्वेनै॒वैव स्वेन॑ । \newline
23. स्वेन॑ भाग॒धेये॑न भाग॒धेये॑न॒ स्वेन॒ स्वेन॑ भाग॒धेये॑न । \newline
24. भा॒ग॒धेये॒नोपोप॑ भाग॒धेये॑न भाग॒धेये॒नोप॑ । \newline
25. भा॒ग॒धेये॒नेति॑ भाग - धेये॑न । \newline
26. उप॑ धावति धाव॒ त्युपोप॑ धावति । \newline
27. धा॒व॒ति॒ ते ते धा॑वति धावति॒ ते । \newline
28. त ए॒वैव ते त ए॒व । \newline
29. ए॒वैन॑ मेन मे॒वैवैन᳚म् । \newline
30. ए॒न॒म् पा॒पात् पा॒पा दे॑न मेनम् पा॒पात् । \newline
31. पा॒पाथ् स्रामा॒थ् स्रामा᳚त् पा॒पात् पा॒पाथ् स्रामा᳚त् । \newline
32. स्रामा᳚न् मुञ्चन्ति मुञ्चन्ति॒ स्रामा॒थ् स्रामा᳚न् मुञ्चन्ति । \newline
33. मु॒ञ्च॒ न्त्य॒मा॒वा॒स्या॑या ममावा॒स्या॑याम् मुञ्चन्ति मुञ्च न्त्यमावा॒स्या॑याम् । \newline
34. अ॒मा॒वा॒स्या॑या॒म् निर् णिर॑मावा॒स्या॑या ममावा॒स्या॑या॒म् निः । \newline
35. अ॒मा॒वा॒स्या॑या॒मित्य॑मा - वा॒स्या॑याम् । \newline
36. निर् व॑पेद् वपे॒न् निर् णिर् व॑पेत् । \newline
37. व॒पे॒ द॒मु म॒मुं ॅव॑पेद् वपे द॒मुम् । \newline
38. अ॒मु मे॒वैवामु म॒मु मे॒व । \newline
39. ए॒वैन॑ मेन मे॒वैवैन᳚म् । \newline
40. ए॒न॒ मा॒प्याय॑मान मा॒प्याय॑मान मेन मेन मा॒प्याय॑मानम् । \newline
41. आ॒प्याय॑मान॒ मन्वन्वा॒ प्याय॑मान मा॒प्याय॑मान॒ मनु॑ । \newline
42. आ॒प्याय॑मान॒मित्या᳚ - प्याय॑मानम् । \newline
43. अन्वा ऽन्वन्वा । \newline
44. आ प्या॑ययति प्यायय॒त्या प्या॑ययति । \newline
45. प्या॒य॒य॒ति॒ नवो॑नवो॒ नवो॑नवः प्याययति प्याययति॒ नवो॑नवः । \newline
46. नवो॑नवो भवति भवति॒ नवो॑नवो॒ नवो॑नवो भवति । \newline
47. नवो॑नव॒ इति॒ नवः॑ - न॒वः॒ । \newline
48. भ॒व॒ति॒ जाय॑मानो॒ जाय॑मानो भवति भवति॒ जाय॑मानः । \newline
49. जाय॑मान॒ इतीति॒ जाय॑मानो॒ जाय॑मान॒ इति॑ । \newline
50. इति॑ पुरोनुवा॒क्या॑ पुरोनुवा॒क्ये॑तीति॑ पुरोनुवा॒क्या᳚ । \newline
51. पु॒रो॒नु॒वा॒क्या॑ भवति भवति पुरोनुवा॒क्या॑ पुरोनुवा॒क्या॑ भवति । \newline
52. पु॒रो॒नु॒वा॒क्येति॑ पुरः - अ॒नु॒वा॒क्या᳚ । \newline
53. भ॒व॒ त्यायु॒ रायु॑र् भवति भव॒ त्यायुः॑ । \newline
54. आयु॑ रे॒वैवायु॒ रायु॑ रे॒व । \newline
55. ए॒वास्मि॑न् नस्मिन् ने॒वैवास्मिन्न्॑ । \newline
56. अ॒स्मि॒न् तया॒ तया᳚ ऽस्मिन् नस्मि॒न् तया᳚ । \newline
57. तया॑ दधाति दधाति॒ तया॒ तया॑ दधाति । \newline
58. द॒धा॒ति॒ यं ॅयम् द॑धाति दधाति॒ यम् । \newline
59. य मा॑दि॒त्या आ॑दि॒त्या यं ॅय मा॑दि॒त्याः । \newline
60. आ॒दि॒त्या अꣳ॒॒शु मꣳ॒॒शु मा॑दि॒त्या आ॑दि॒त्या अꣳ॒॒शुम् । \newline
61. अꣳ॒॒शु मा᳚प्या॒यय॑ न्त्याप्या॒यय॑ न्त्यꣳ॒॒शु मꣳ॒॒शु मा᳚प्या॒यय॑न्ति । \newline
62. आ॒प्या॒य य॒न्तीती त्या᳚प्या॒यय॑ न्त्याप्या॒य य॒न्तीति॑ । \newline
63. आ॒प्या॒यय॒न्तीत्या᳚ - प्या॒यय॑न्ति । \newline
64. इति॑ या॒ज्या॑ या॒ज्ये॑तीति॑ या॒ज्या᳚ । \newline
65. या॒ज्या॑ ऽऽया॒ज्या॑ या॒ज्या᳚ । \newline
66. ऐवैवैव । \newline
67. ए॒वैन॑ मेन मे॒वैवैन᳚म् । \newline
68. ए॒न॒ मे॒तयै॒तयै॑न मेन मे॒तया᳚ । \newline
69. ए॒तया᳚ प्याययति प्यायय त्ये॒तयै॒तया᳚ प्याययति । \newline
70. प्या॒य॒य॒तीति॑ प्याययति । \newline

\textbf{Ghana Paata } \newline

1. आ॒दि॒त्यम् च॒रुम् च॒रु मा॑दि॒त्य मा॑दि॒त्यम् च॒रुम् निर् णिश्च॒रु मा॑दि॒त्य मा॑दि॒त्यम् च॒रुम् निः । \newline
2. च॒रुम् निर् णिश्च॒रुम् च॒रुम् निर॑वपन् नवप॒न् निश्च॒रुम् च॒रुम् निर॑वपन्न् । \newline
3. निर॑वपन् नवप॒न् निर् णिर॑वप॒न् तेन॒ तेना॑वप॒न् निर् णिर॑वप॒न् तेन॑ । \newline
4. अ॒व॒प॒न् तेन॒ तेना॑वपन् नवप॒न् तेनै॒वैव तेना॑वपन् नवप॒न् तेनै॒व । \newline
5. तेनै॒वैव तेन॒ तेनै॒वैन॑ मेन मे॒व तेन॒ तेनै॒वैन᳚म् । \newline
6. ए॒वैन॑ मेन मे॒वैवैन॑म् पा॒पात् पा॒पादे॑न मे॒वैवैन॑म् पा॒पात् । \newline
7. ए॒न॒म् पा॒पात् पा॒पादे॑न मेनम् पा॒पाथ् स्रामा॒थ् स्रामा᳚त् पा॒पादे॑न मेनम् पा॒पाथ् स्रामा᳚त् । \newline
8. पा॒पाथ् स्रामा॒थ् स्रामा᳚त् पा॒पात् पा॒पाथ् स्रामा॑ दमुञ्चन् नमुञ्च॒न् थ्स्रामा᳚त् पा॒पात् पा॒पाथ् स्रामा॑ दमुञ्चन्न् । \newline
9. स्रामा॑ दमुञ्चन् नमुञ्च॒न् थ्स्रामा॒थ् स्रामा॑ दमुञ्च॒न्॒. यो यो॑ ऽमुञ्च॒न् थ्स्रामा॒थ् स्रामा॑ दमुञ्च॒न्॒. यः । \newline
10. अ॒मु॒ञ्च॒न्॒. यो यो॑ ऽमुञ्चन् नमुञ्च॒न्॒. यः पा॑पय॒क्ष्मगृ॑हीतः पापय॒क्ष्मगृ॑हीतो॒ यो॑ ऽमुञ्चन् नमुञ्च॒न्॒. यः पा॑पय॒क्ष्मगृ॑हीतः । \newline
11. यः पा॑पय॒क्ष्मगृ॑हीतः पापय॒क्ष्मगृ॑हीतो॒ यो यः पा॑पय॒क्ष्मगृ॑हीतः॒ स्याथ् स्यात् पा॑पय॒क्ष्मगृ॑हीतो॒ यो यः पा॑पय॒क्ष्मगृ॑हीतः॒ स्यात् । \newline
12. पा॒प॒य॒क्ष्मगृ॑हीतः॒ स्याथ् स्यात् पा॑पय॒क्ष्मगृ॑हीतः पापय॒क्ष्मगृ॑हीतः॒ स्यात् तस्मै॒ तस्मै॒ स्यात् पा॑पय॒क्ष्मगृ॑हीतः पापय॒क्ष्मगृ॑हीतः॒ स्यात् तस्मै᳚ । \newline
13. पा॒प॒य॒क्ष्मगृ॑हीत॒ इति॑ पापय॒क्ष्म - गृ॒ही॒तः॒ । \newline
14. स्यात् तस्मै॒ तस्मै॒ स्याथ् स्यात् तस्मा॑ ए॒त मे॒तम् तस्मै॒ स्याथ् स्यात् तस्मा॑ ए॒तम् । \newline
15. तस्मा॑ ए॒त मे॒तम् तस्मै॒ तस्मा॑ ए॒त मा॑दि॒त्य मा॑दि॒त्य मे॒तम् तस्मै॒ तस्मा॑ ए॒त मा॑दि॒त्यम् । \newline
16. ए॒त मा॑दि॒त्य मा॑दि॒त्य मे॒त मे॒त मा॑दि॒त्यम् च॒रुम् च॒रु मा॑दि॒त्य मे॒त मे॒त मा॑दि॒त्यम् च॒रुम् । \newline
17. आ॒दि॒त्यम् च॒रुम् च॒रु मा॑दि॒त्य मा॑दि॒त्यम् च॒रुम् निर् णिश्च॒रु मा॑दि॒त्य मा॑दि॒त्यम् च॒रुम् निः । \newline
18. च॒रुम् निर् णिश्च॒रुम् च॒रुम् निर् व॑पेद् वपे॒न् निश्च॒रुम् च॒रुम् निर् व॑पेत् । \newline
19. निर् व॑पेद् वपे॒न् निर् णिर् व॑पे दादि॒त्या ना॑दि॒त्यान्. व॑पे॒न् निर् णिर् व॑पे दादि॒त्यान् । \newline
20. व॒पे॒ दा॒दि॒त्या ना॑दि॒त्यान्. व॑पेद् वपे दादि॒त्या ने॒वैवादि॒त्यान्. व॑पेद् वपे दादि॒त्या ने॒व । \newline
21. आ॒दि॒त्या ने॒वैवादि॒त्या ना॑दि॒त्या ने॒व स्वेन॒ स्वेनै॒वादि॒त्या ना॑दि॒त्या ने॒व स्वेन॑ । \newline
22. ए॒व स्वेन॒ स्वेनै॒वैव स्वेन॑ भाग॒धेये॑न भाग॒धेये॑न॒ स्वेनै॒वैव स्वेन॑ भाग॒धेये॑न । \newline
23. स्वेन॑ भाग॒धेये॑न भाग॒धेये॑न॒ स्वेन॒ स्वेन॑ भाग॒धेये॒नोपोप॑ भाग॒धेये॑न॒ स्वेन॒ स्वेन॑ भाग॒धेये॒नोप॑ । \newline
24. भा॒ग॒धेये॒नोपोप॑ भाग॒धेये॑न भाग॒धेये॒नोप॑ धावति धाव॒त्युप॑ भाग॒धेये॑न भाग॒धेये॒नोप॑ धावति । \newline
25. भा॒ग॒धेये॒नेति॑ भाग - धेये॑न । \newline
26. उप॑ धावति धाव॒ त्युपोप॑ धावति॒ ते ते धा॑व॒ त्युपोप॑ धावति॒ ते । \newline
27. धा॒व॒ति॒ ते ते धा॑वति धावति॒ त ए॒वैव ते धा॑वति धावति॒ त ए॒व । \newline
28. त ए॒वैव ते त ए॒वैन॑ मेन मे॒व ते त ए॒वैन᳚म् । \newline
29. ए॒वैन॑ मेन मे॒वैवैन॑म् पा॒पात् पा॒पादे॑न मे॒वैवैन॑म् पा॒पात् । \newline
30. ए॒न॒म् पा॒पात् पा॒पादे॑न मेनम् पा॒पाथ् स्रामा॒थ् स्रामा᳚त् पा॒पादे॑न मेनम् पा॒पाथ् स्रामा᳚त् । \newline
31. पा॒पाथ् स्रामा॒थ् स्रामा᳚त् पा॒पात् पा॒पाथ् स्रामा᳚न् मुञ्चन्ति मुञ्चन्ति॒ स्रामा᳚त् पा॒पात् पा॒पाथ् स्रामा᳚न् मुञ्चन्ति । \newline
32. स्रामा᳚न् मुञ्चन्ति मुञ्चन्ति॒ स्रामा॒थ् स्रामा᳚न् मुञ्च न्त्यमावा॒स्या॑या ममावा॒स्या॑याम् मुञ्चन्ति॒ स्रामा॒थ् 
स्रामा᳚न् मुञ्च न्त्यमावा॒स्या॑याम् । \newline
33. मु॒ञ्च॒ न्त्य॒मा॒वा॒स्या॑या ममावा॒स्या॑याम् मुञ्चन्ति मुञ्च न्त्यमावा॒स्या॑या॒म् निर् णिर॑मावा॒स्या॑याम् मुञ्चन्ति मुञ्च न्त्यमावा॒स्या॑या॒म् निः । \newline
34. अ॒मा॒वा॒स्या॑या॒म् निर् णिर॑मावा॒स्या॑या ममावा॒स्या॑या॒म् निर् व॑पेद् वपे॒न् निर॑मावा॒स्या॑या ममावा॒स्या॑या॒म् निर् व॑पेत् । \newline
35. अ॒मा॒वा॒स्या॑या॒मित्य॑मा - वा॒स्या॑याम् । \newline
36. निर् व॑पेद् वपे॒न् निर् णिर् व॑पेद॒मु म॒मुं ॅव॑पे॒न् निर् णिर् व॑पेद॒मुम् । \newline
37. व॒पे॒ द॒मु म॒मुं ॅव॑पेद् वपे द॒मु मे॒वैवामुं ॅव॑पेद् वपे द॒मु मे॒व । \newline
38. अ॒मु मे॒वैवामु म॒मु मे॒वैन॑ मेन मे॒वामु म॒मु मे॒वैन᳚म् । \newline
39. ए॒वैन॑ मेन मे॒वैवैन॑ मा॒प्याय॑मान मा॒प्याय॑मान मेन मे॒वैवैन॑ मा॒प्याय॑मानम् । \newline
40. ए॒न॒ मा॒प्याय॑मान मा॒प्याय॑मान मेन मेन मा॒प्याय॑मान॒ मन्वन्वा॒ प्याय॑मान मेन मेन मा॒प्याय॑मान॒ मनु॑ । \newline
41. आ॒प्याय॑मान॒ मन्वन्वा॒ प्याय॑मान मा॒प्याय॑मान॒ मन्वा ऽन्वा॒प्याय॑मान मा॒प्याय॑मान॒ मन्वा । \newline
42. आ॒प्याय॑मान॒मित्या᳚ - प्याय॑मानम् । \newline
43. अन्वा ऽन्वन्वा प्या॑ययति प्यायय॒त्या ऽन्वन्वा प्या॑ययति । \newline
44. आ प्या॑ययति प्यायय॒त्या प्या॑ययति॒ नवो॑नवो॒ नवो॑नवः प्यायय॒त्या प्या॑ययति॒ नवो॑नवः । \newline
45. प्या॒य॒य॒ति॒ नवो॑नवो॒ नवो॑नवः प्याययति प्याययति॒ नवो॑नवो भवति भवति॒ नवो॑नवः प्याययति प्याययति॒ नवो॑नवो भवति । \newline
46. नवो॑नवो भवति भवति॒ नवो॑नवो॒ नवो॑नवो भवति॒ जाय॑मानो॒ जाय॑मानो भवति॒ नवो॑नवो॒ नवो॑नवो भवति॒ जाय॑मानः । \newline
47. नवो॑नव॒ इति॒ नवः॑ - न॒वः॒ । \newline
48. भ॒व॒ति॒ जाय॑मानो॒ जाय॑मानो भवति भवति॒ जाय॑मान॒ इतीति॒ जाय॑मानो भवति भवति॒ जाय॑मान॒ इति॑ । \newline
49. जाय॑मान॒ इतीति॒ जाय॑मानो॒ जाय॑मान॒ इति॑ पुरोनुवा॒क्या॑ पुरोनुवा॒क्ये॑ति॒ जाय॑मानो॒ जाय॑मान॒ इति॑ पुरोनुवा॒क्या᳚ । \newline
50. इति॑ पुरोनुवा॒क्या॑ पुरोनुवा॒क्ये॑तीति॑ पुरोनुवा॒क्या॑ भवति भवति पुरोनुवा॒क्ये॑तीति॑ पुरोनुवा॒क्या॑ भवति । \newline
51. पु॒रो॒नु॒वा॒क्या॑ भवति भवति पुरोनुवा॒क्या॑ पुरोनुवा॒क्या॑ भव॒ त्यायु॒ रायु॑र् भवति पुरोनुवा॒क्या॑ पुरोनुवा॒क्या॑ भव॒ त्यायुः॑ । \newline
52. पु॒रो॒नु॒वा॒क्येति॑ पुरः - अ॒नु॒वा॒क्या᳚ । \newline
53. भ॒व॒ त्यायु॒ रायु॑र् भवति भव॒ त्यायु॑ रे॒वैवायु॑र् भवति भव॒ त्यायु॑ रे॒व । \newline
54. आयु॑ रे॒वैवायु॒ रायु॑ रे॒वास्मि॑न् नस्मिन् ने॒वायु॒ रायु॑ रे॒वास्मिन्न्॑ । \newline
55. ए॒वास्मि॑न् नस्मिन् ने॒वैवास्मि॒न् तया॒ तया᳚ ऽस्मिन् ने॒वैवास्मि॒न् तया᳚ । \newline
56. अ॒स्मि॒न् तया॒ तया᳚ ऽस्मिन् नस्मि॒न् तया॑ दधाति दधाति॒ तया᳚ ऽस्मिन् नस्मि॒न् तया॑ दधाति । \newline
57. तया॑ दधाति दधाति॒ तया॒ तया॑ दधाति॒ यं ॅयम् द॑धाति॒ तया॒ तया॑ दधाति॒ यम् । \newline
58. द॒धा॒ति॒ यं ॅयम् द॑धाति दधाति॒ य मा॑दि॒त्या आ॑दि॒त्या यम् द॑धाति दधाति॒ य मा॑दि॒त्याः । \newline
59. य मा॑दि॒त्या आ॑दि॒त्या यं ॅय मा॑दि॒त्या अꣳ॒॒शु मꣳ॒॒शु मा॑दि॒त्या यं ॅय मा॑दि॒त्या अꣳ॒॒शुम् । \newline
60. आ॒दि॒त्या अꣳ॒॒शु मꣳ॒॒शु मा॑दि॒त्या आ॑दि॒त्या अꣳ॒॒शु मा᳚प्या॒यय॑ न्त्याप्या॒यय॑ न्त्यꣳ॒॒शु मा॑दि॒त्या आ॑दि॒त्या अꣳ॒॒शु मा᳚प्या॒यय॑न्ति । \newline
61. अꣳ॒॒शु मा᳚प्या॒यय॑ न्त्याप्या॒यय॑ न्त्यꣳ॒॒शु मꣳ॒॒शु मा᳚प्या॒यय॒न्तीती त्या᳚प्या॒य य॑न्त्यꣳ॒॒शु मꣳ॒॒शु मा᳚प्या॒यय॒न्तीति॑ । \newline
62. आ॒प्या॒यय॒न्ती तीत्या᳚प्या॒यय॑ न्त्याप्या॒यय॒ न्तीति॑ या॒ज्या॑ या॒ज्ये त्या᳚प्या॒यय॑ न्त्याप्या॒यय॒ न्तीति॑ या॒ज्या᳚ । \newline
63. आ॒प्या॒यय॒न्तीत्या᳚ - प्या॒यय॑न्ति । \newline
64. इति॑ या॒ज्या॑ या॒ज्ये॑तीति॑ या॒ज्या॑ ऽऽया॒ज्ये॑तीति॑ या॒ज्या᳚ । \newline
65. या॒ज्या॑ ऽऽया॒ज्या॑ या॒ज्यै॑वैवा या॒ज्या॑ या॒ज्यै॑व । \newline
66. ऐवैवैवैन॑ मेन मे॒वैवैन᳚म् । \newline
67. ए॒वैन॑ मेन मे॒वैवैन॑ मे॒तयै॒त यै॑न मे॒वैवैन॑ मे॒तया᳚ । \newline
68. ए॒न॒ मे॒तयै॒तयै॑न मेन मे॒तया᳚ प्याययति प्यायय त्ये॒तयै॑न मेन मे॒तया᳚ प्याययति । \newline
69. ए॒तया᳚ प्याययति प्यायय त्ये॒तयै॒तया᳚ प्याययति । \newline
70. प्या॒य॒य॒तीति॑ प्याययति । \newline
\pagebreak
\markright{ TS 2.3.6.1  \hfill https://www.vedavms.in \hfill}

\section{ TS 2.3.6.1 }

\textbf{TS 2.3.6.1 } \newline
\textbf{Samhita Paata} \newline

प्र॒जाप॑ति र्दे॒वेभ्यो॒ऽन्नाद्यं॒ ॅव्यादि॑श॒थ् सो᳚ऽब्रवी॒द्यदि॒मान् ॅलो॒कान॒भ्य॑ति॒रिच्या॑तै॒ तन्ममा॑स॒दिति॒ तदि॒मान् ॅलो॒कान॒भ्यत्य॑रिच्य॒तेन्द्रꣳ॒॒ राजा॑न॒मिन्द्र॑मधिरा॒जमिन्द्रꣳ॑ स्व॒राजा॑नं॒ ततो॒ वै स इ॒मान् ॅलो॒काꣳ स्त्रे॒धाऽदु॑ह॒त् तत् त्रि॒धातो᳚स्त्रिधातु॒त्वं ॅयं का॒मये॑तान्ना॒दः स्या॒दिति॒ तस्मा॑ ए॒तं त्रि॒धातुं॒ निर्व॑पे॒दिन्द्रा॑य॒ राज्ञे॑ पुरो॒डाश॒ - [  ] \newline

\textbf{Pada Paata} \newline

प्र॒जाप॑ति॒रिति॑ प्र॒जा - प॒तिः॒ । दे॒वेभ्यः॑ । अ॒न्नाद्य॒मित्य॑न्न - अद्य᳚म् । व्यादि॑श॒दिति॑ वि - आदि॑शत् । सः । अ॒ब्र॒वी॒त् । यत् । इ॒मान् । लो॒कान् । अ॒भीति॑ । अ॒ति॒रिच्या॑ता॒ इत्य॑ति - रिच्या॑तै । तत् । मम॑ । अ॒स॒त् । इति॑ । तत् । इ॒मान् । लो॒कान् । अ॒भि । अतीति॑ । अ॒रि॒च्य॒त । इन्द्र᳚म् । राजा॑नम् । इन्द्र᳚म् । अ॒धि॒रा॒जमित्य॑धि-रा॒जम् ।   इन्द्र᳚म् । स्व॒राजा॑न॒मिति॑ स्व - राजा॑नम् । ततः॑ । वै । सः । इ॒मान् । लो॒कान् । त्रे॒धा । अ॒दु॒ह॒त् । तत् । त्रि॒धातो॒रिति॑ त्रि - धातोः᳚ । त्रि॒धा॒तु॒त्वमिति॑ त्रिधातु - त्वम् । यम् । का॒मये॑त । अ॒न्ना॒द इत्य॑न्न - अ॒दः । स्या॒त् । इति॑ । तस्मै᳚ । ए॒तम् । त्रि॒धातु॒मिति॑ त्रि - धातु᳚म् । निरिति॑ । व॒पे॒त् । इन्द्रा॑य । राज्ञे᳚ । पु॒रो॒डाश᳚म् ।  \newline


\textbf{Krama Paata} \newline

प्र॒जाप॑तिर्,दे॒वेभ्यः॑ । प्र॒जाप॑ति॒रिति॑ प्र॒जा - प॒तिः॒ । दे॒वेभ्यो॒ऽन्नाद्य᳚म् । अ॒न्नाद्यं॒ ॅव्यादि॑शत् । अ॒न्नाद्य॒मित्य॑न्न - अद्य᳚म् । व्यादि॑श॒थ् सः । व्यादि॑श॒दिति॑ वि - आदि॑शत् । सो᳚ ऽब्रवीत् । अ॒ब्र॒वी॒द् यत् । यदि॒मान् । इ॒मान् ॅलो॒कान् । लो॒कान॒भि । अ॒भ्य॑ति॒रिच्या॑तै । अ॒ति॒रिच्या॑तै॒ तत् । अ॒ति॒रिच्या॑ता॒ इत्य॑ति - रिच्या॑तै । तन्मम॑ । ममा॑सत् । अ॒स॒दिति॑ । इति॒ तत् । तदि॒मान् । इ॒मान् ॅलो॒कान् । लो॒कान॒भि । अ॒भ्यति॑ । अत्य॑रिच्यत । अ॒रि॒च्य॒तेन्द्र᳚म् । इन्द्रꣳ॒॒ राजा॑नम् । राजा॑न॒मिन्द्र᳚म् । इन्द्र॑मधिरा॒जम् । अ॒धि॒रा॒जमिन्द्र᳚म् । अ॒धि॒रा॒जामित्य॑धि - रा॒जम् । इन्द्रꣳ॑ स्व॒राजा॑नम् । स्व॒राजा॑न॒म् ततः॑ । स्व॒राजा॑न॒मिति॑ स्व - राजा॑नम् । ततो॒ वै । वै सः । स इ॒मान् । इ॒मान् ॅलो॒कान् । लो॒काꣳ स्त्रे॒धा । त्रे॒धा ऽदु॑हत् । अ॒दु॒ह॒त् तत् । तत् त्रि॒धातोः᳚ । त्रि॒धातो᳚ स्त्रिधातु॒त्वम् । त्रि॒धातो॒रिति॑ त्रि - धातोः᳚ । त्रि॒धा॒तु॒त्वं ॅयम् । त्रि॒धा॒तु॒त्वमिति॑ त्रिधातु - त्वम् । यम् का॒मये॑त । का॒मये॑तान्ना॒दः । अ॒न्ना॒दः स्या᳚त् । अ॒न्ना॒द इत्य॑न्न - अ॒दः । स्या॒दिति॑ । इति॒ तस्मै᳚ । तस्मा॑ ए॒तम् । ए॒तम् त्रि॒धातु᳚म् । त्रि॒धातु॒म् निः । त्रि॒धातु॒मिति॑ त्रि - धातु᳚म् । निर् व॑पेत् । व॒पे॒दिन्द्रा॑य । इन्द्रा॑य॒ राज्ञे᳚ । राज्ञे॑ पुरो॒डाश᳚म् । पु॒रो॒डाश॒मेका॑दशकपालम् \newline

\textbf{Jatai Paata} \newline

1. प्र॒जाप॑तिर् दे॒वेभ्यो॑ दे॒वेभ्यः॑ प्र॒जाप॑तिः प्र॒जाप॑तिर् दे॒वेभ्यः॑ । \newline
2. प्र॒जाप॑ति॒रिति॑ प्र॒जा - प॒तिः॒ । \newline
3. दे॒वेभ्यो॒ ऽन्नाद्य॑ म॒न्नाद्य॑म् दे॒वेभ्यो॑ दे॒वेभ्यो॒ ऽन्नाद्य᳚म् । \newline
4. अ॒न्नाद्यं॒ ॅव्यादि॑श॒द् व्यादि॑श द॒न्नाद्य॑ म॒न्नाद्यं॒ ॅव्यादि॑शत् । \newline
5. अ॒न्नाद्य॒मित्य॑न्न - अद्य᳚म् । \newline
6. व्यादि॑श॒थ् स स व्यादि॑श॒द् व्यादि॑श॒थ् सः । \newline
7. व्यादि॑श॒दिति॑ वि - आदि॑शत् । \newline
8. सो᳚ ऽब्रवी दब्रवी॒थ् स सो᳚ ऽब्रवीत् । \newline
9. अ॒ब्र॒वी॒द् यद् यद॑ब्रवी दब्रवी॒द् यत् । \newline
10. यदि॒मा नि॒मान्. यद् यदि॒मान् । \newline
11. इ॒मान् ॅलो॒कान् ॅलो॒का नि॒मा नि॒मान् ॅलो॒कान् । \newline
12. लो॒का न॒भ्य॑भि लो॒कान् ॅलो॒का न॒भि । \newline
13. अ॒भ्य॑ति॒रिच्या॑ता अति॒रिच्या॑ता अ॒भ्या᳚(1॒)भ्य॑ति॒रिच्या॑तै । \newline
14. अ॒ति॒रिच्या॑तै॒ तत् त द॑ति॒रिच्या॑ता अति॒रिच्या॑तै॒ तत् । \newline
15. अ॒ति॒रिच्या॑ता॒ इत्य॑ति - रिच्या॑तै । \newline
16. तन् मम॒ मम॒ तत् तन् मम॑ । \newline
17. ममा॑स दस॒न् मम॒ ममा॑सत् । \newline
18. अ॒स॒ दिती त्य॑स दस॒ दिति॑ । \newline
19. इति॒ तत् तदितीति॒ तत् । \newline
20. तदि॒मा नि॒मान् तत् तदि॒मान् । \newline
21. इ॒मान् ॅलो॒कान् ॅलो॒का नि॒मा नि॒मान् ॅलो॒कान् । \newline
22. लो॒का न॒भ्य॑भि लो॒कान् ॅलो॒का न॒भि । \newline
23. अ॒भ्य त्यत्य॒ भ्य॑भ्यति॑ । \newline
24. अत्य॑रिच्यता रिच्य॒ता त्यत्य॑रिच्यत । \newline
25. अ॒रि॒च्य॒तेन्द्र॒ मिन्द्र॑ मरिच्यता रिच्य॒ते न्द्र᳚म् । \newline
26. इन्द्रꣳ॒॒ राजा॑नꣳ॒॒ राजा॑न॒ मिन्द्र॒ मिन्द्रꣳ॒॒ राजा॑नम् । \newline
27. राजा॑न॒ मिन्द्र॒ मिन्द्रꣳ॒॒ राजा॑नꣳ॒॒ राजा॑न॒ मिन्द्र᳚म् । \newline
28. इन्द्र॑ मधिरा॒ज म॑धिरा॒ज मिन्द्र॒ मिन्द्र॑ मधिरा॒जम् । \newline
29. अ॒धि॒रा॒ज मिन्द्र॒ मिन्द्र॑ मधिरा॒ज म॑धिरा॒ज मिन्द्र᳚म् । \newline
30. अ॒धि॒रा॒जमित्य॑धि - रा॒जम् । \newline
31. इन्द्रꣳ॑ स्व॒राजा॑नꣳ स्व॒राजा॑न॒ मिन्द्र॒ मिन्द्रꣳ॑ स्व॒राजा॑नम् । \newline
32. स्व॒राजा॑न॒म् तत॒ स्ततः॑ स्व॒राजा॑नꣳ स्व॒राजा॑न॒म् ततः॑ । \newline
33. स्व॒राजा॑न॒मिति॑ स्व - राजा॑नम् । \newline
34. ततो॒ वै वै तत॒ स्ततो॒ वै । \newline
35. वै स स वै वै सः । \newline
36. स इ॒मा नि॒मान् थ्स स इ॒मान् । \newline
37. इ॒मान् ॅलो॒कान् ॅलो॒का नि॒मा नि॒मान् ॅलो॒कान् । \newline
38. लो॒काꣳ स्त्रे॒धा त्रे॒धा लो॒कान् ॅलो॒काꣳ स्त्रे॒धा । \newline
39. त्रे॒धा ऽदु॑ह ददुहत् त्रे॒धा त्रे॒धा ऽदु॑हत् । \newline
40. अ॒दु॒ह॒त् तत् तद॑दुह ददुह॒त् तत् । \newline
41. तत् त्रि॒धातो᳚ स्त्रि॒धातो॒ स्तत् तत् त्रि॒धातोः᳚ । \newline
42. त्रि॒धातो᳚ स्त्रिधातु॒त्वम् त्रि॑धातु॒त्वम् त्रि॒धातो᳚ स्त्रि॒धातो᳚ स्त्रिधातु॒त्वम् । \newline
43. त्रि॒धातो॒रिति॑ त्रि - धातोः᳚ । \newline
44. त्रि॒धा॒तु॒त्वं ॅयं ॅयम् त्रि॑धातु॒त्वम् त्रि॑धातु॒त्वं ॅयम् । \newline
45. त्रि॒धा॒तु॒त्वमिति॑ त्रिधातु - त्वम् । \newline
46. यम् का॒मये॑त का॒मये॑त॒ यं ॅयम् का॒मये॑त । \newline
47. का॒मये॑ता न्ना॒दो᳚ ऽन्ना॒दः का॒मये॑त का॒मये॑ता न्ना॒दः । \newline
48. अ॒न्ना॒दः स्या᳚थ् स्या दन्ना॒दो᳚ ऽन्ना॒दः स्या᳚त् । \newline
49. अ॒न्ना॒द इत्य॑न्न - अ॒दः । \newline
50. स्या॒ दितीति॑ स्याथ् स्या॒ दिति॑ । \newline
51. इति॒ तस्मै॒ तस्मा॒ इतीति॒ तस्मै᳚ । \newline
52. तस्मा॑ ए॒त मे॒तम् तस्मै॒ तस्मा॑ ए॒तम् । \newline
53. ए॒तम् त्रि॒धातु॑म् त्रि॒धातु॑ मे॒त मे॒तम् त्रि॒धातु᳚म् । \newline
54. त्रि॒धातु॒म् निर् णिष् ट्रि॒धातु॑म् त्रि॒धातु॒म् निः । \newline
55. त्रि॒धातु॒मिति॑ त्रि - धातु᳚म् । \newline
56. निर् व॑पेद् वपे॒न् निर् णिर् व॑पेत् । \newline
57. व॒पे॒ दिन्द्रा॒ये न्द्रा॑य वपेद् वपे॒ दिन्द्रा॑य । \newline
58. इन्द्रा॑य॒ राज्ञे॒ राज्ञ्॒ इन्द्रा॒ये न्द्रा॑य॒ राज्ञे᳚ । \newline
59. राज्ञे॑ पुरो॒डाश॑म् पुरो॒डाशꣳ॒॒ राज्ञे॒ राज्ञे॑ पुरो॒डाश᳚म् । \newline
60. पु॒रो॒डाश॒ मेका॑दशकपाल॒ मेका॑दशकपालम् पुरो॒डाश॑म् पुरो॒डाश॒ मेका॑दशकपालम् । \newline

\textbf{Ghana Paata } \newline

1. प्र॒जाप॑तिर् दे॒वेभ्यो॑ दे॒वेभ्यः॑ प्र॒जाप॑तिः प्र॒जाप॑तिर् दे॒वेभ्यो॒ ऽन्नाद्य॑ म॒न्नाद्य॑म् दे॒वेभ्यः॑ प्र॒जाप॑तिः प्र॒जाप॑तिर् दे॒वेभ्यो॒ ऽन्नाद्य᳚म् । \newline
2. प्र॒जाप॑ति॒रिति॑ प्र॒जा - प॒तिः॒ । \newline
3. दे॒वेभ्यो॒ ऽन्नाद्य॑ म॒न्नाद्य॑म् दे॒वेभ्यो॑ दे॒वेभ्यो॒ ऽन्नाद्यं॒ ॅव्यादि॑श॒द् व्यादि॑श द॒न्नाद्य॑म् दे॒वेभ्यो॑ दे॒वेभ्यो॒ ऽन्नाद्यं॒ ॅव्यादि॑शत् । \newline
4. अ॒न्नाद्यं॒ ॅव्यादि॑श॒द् व्यादि॑श द॒न्नाद्य॑ म॒न्नाद्यं॒ ॅव्यादि॑श॒थ् स स व्यादि॑श द॒न्नाद्य॑ म॒न्नाद्यं॒ ॅव्यादि॑श॒थ् सः । \newline
5. अ॒न्नाद्य॒मित्य॑न्न - अद्य᳚म् । \newline
6. व्यादि॑श॒थ् स स व्यादि॑श॒द् व्यादि॑श॒थ् सो᳚ ऽब्रवी दब्रवी॒थ् स व्यादि॑श॒द् व्यादि॑श॒थ् सो᳚ ऽब्रवीत् । \newline
7. व्यादि॑श॒दिति॑ वि - आदि॑शत् । \newline
8. सो᳚ ऽब्रवी दब्रवी॒थ् स सो᳚ ऽब्रवी॒द् यद् यद॑ब्रवी॒थ् स सो᳚ ऽब्रवी॒द् यत् । \newline
9. अ॒ब्र॒वी॒द् यद् यद॑ब्रवी दब्रवी॒द् यदि॒मा नि॒मान्. यद॑ब्रवी दब्रवी॒द् यदि॒मान् । \newline
10. यदि॒मा नि॒मान्. यद् यदि॒मान् ॅलो॒कान् ॅलो॒का नि॒मान्. यद् यदि॒मान् ॅलो॒कान् । \newline
11. इ॒मान् ॅलो॒कान् ॅलो॒का नि॒मा नि॒मान् ॅलो॒का न॒भ्य॑भि लो॒का नि॒मा नि॒मान् ॅलो॒का न॒भि । \newline
12. लो॒का न॒भ्य॑भि लो॒कान् ॅलो॒का न॒भ्य॑ ति॒रिच्या॑ता अति॒रिच्या॑ता अ॒भि लो॒कान् ॅलो॒का न॒भ्य॑ति॒रिच्या॑तै । \newline
13. अ॒भ्य॑ति॒रिच्या॑ता अति॒रिच्या॑ता अ॒भ्या᳚(1॒)भ्य॑ति॒रिच्या॑तै॒ तत् तद॑ति॒रिच्या॑ता अ॒भ्या᳚(1॒)भ्य॑ति॒रिच्या॑तै॒ तत् । \newline
14. अ॒ति॒रिच्या॑तै॒ तत् तद॑ति॒रिच्या॑ता अति॒रिच्या॑तै॒ तन् मम॒ मम॒ तद॑ति॒रिच्या॑ता अति॒रिच्या॑तै॒ तन् मम॑ । \newline
15. अ॒ति॒रिच्या॑ता॒ इत्य॑ति - रिच्या॑तै । \newline
16. तन् मम॒ मम॒ तत् तन् ममा॑स दस॒न् मम॒ तत् तन् ममा॑सत् । \newline
17. ममा॑स दस॒न् मम॒ ममा॑स॒ दितीत्य॑स॒न् मम॒ ममा॑स॒दिति॑ । \newline
18. अ॒स॒ दितीत्य॑स दस॒दिति॒ तत् तदि त्य॑स दस॒दिति॒ तत् । \newline
19. इति॒ तत् तदितीति॒ तदि॒मा नि॒मान् तदितीति॒ तदि॒मान् । \newline
20. तदि॒मा नि॒मान् तत् तदि॒मान् ॅलो॒कान् ॅलो॒का नि॒मान् तत् तदि॒मान् ॅलो॒कान् । \newline
21. इ॒मान् ॅलो॒कान् ॅलो॒का नि॒मा नि॒मान् ॅलो॒का न॒भ्य॑भि लो॒का नि॒मा नि॒मान् ॅलो॒का न॒भि । \newline
22. लो॒का न॒भ्य॑भि लो॒कान् ॅलो॒का न॒भ्य त्यत्य॒भि लो॒कान् ॅलो॒का न॒भ्यति॑ । \newline
23. अ॒भ्यत्य त्य॒भ्य॑भ्य त्य॑रिच्यता रिच्य॒ तात्य॒भ्य॑भ्य त्य॑रिच्यत । \newline
24. अत्य॑रिच्यता रिच्य॒तात्य त्य॑रिच्य॒ते न्द्र॒ मिन्द्र॑ मरिच्य॒ता त्यत्य॑रिच्य॒ते न्द्र᳚म् । \newline
25. अ॒रि॒च्य॒ते न्द्र॒ मिन्द्र॑ मरिच्यता रिच्य॒ते न्द्रꣳ॒॒ राजा॑नꣳ॒॒ राजा॑न॒ मिन्द्र॑ मरिच्यता रिच्य॒ते न्द्रꣳ॒॒ राजा॑नम् । \newline
26. इन्द्रꣳ॒॒ राजा॑नꣳ॒॒ राजा॑न॒ मिन्द्र॒ मिन्द्रꣳ॒॒ राजा॑न॒ मिन्द्र॒ मिन्द्रꣳ॒॒ राजा॑न॒ मिन्द्र॒ मिन्द्रꣳ॒॒ राजा॑न॒ मिन्द्र᳚म् । \newline
27. राजा॑न॒ मिन्द्र॒ मिन्द्रꣳ॒॒ राजा॑नꣳ॒॒ राजा॑न॒ मिन्द्र॑ मधिरा॒ज म॑धिरा॒ज मिन्द्रꣳ॒॒ राजा॑नꣳ॒॒ राजा॑न॒ मिन्द्र॑ मधिरा॒जम् । \newline
28. इन्द्र॑ मधिरा॒ज म॑धिरा॒ज मिन्द्र॒ मिन्द्र॑ मधिरा॒ज मिन्द्र॒ मिन्द्र॑ मधिरा॒ज मिन्द्र॒ मिन्द्र॑ मधिरा॒ज मिन्द्र᳚म् । \newline
29. अ॒धि॒रा॒ज मिन्द्र॒ मिन्द्र॑ मधिरा॒ज म॑धिरा॒ज मिन्द्रꣳ॑ स्व॒राजा॑नꣳ स्व॒राजा॑न॒ मिन्द्र॑ मधिरा॒ज म॑धिरा॒ज मिन्द्रꣳ॑ स्व॒राजा॑नम् । \newline
30. अ॒धि॒रा॒जमित्य॑धि - रा॒जम् । \newline
31. इन्द्रꣳ॑ स्व॒राजा॑नꣳ स्व॒राजा॑न॒ मिन्द्र॒ मिन्द्रꣳ॑ स्व॒राजा॑न॒म् तत॒ स्ततः॑ स्व॒राजा॑न॒ मिन्द्र॒ मिन्द्रꣳ॑ स्व॒राजा॑न॒म् ततः॑ । \newline
32. स्व॒राजा॑न॒म् तत॒ स्ततः॑ स्व॒राजा॑नꣳ स्व॒राजा॑न॒म् ततो॒ वै वै ततः॑ स्व॒राजा॑नꣳ स्व॒राजा॑न॒म् ततो॒ वै । \newline
33. स्व॒राजा॑न॒मिति॑ स्व - राजा॑नम् । \newline
34. ततो॒ वै वै तत॒ स्ततो॒ वै स स वै तत॒ स्ततो॒ वै सः । \newline
35. वै स स वै वै स इ॒मा नि॒मान् थ्स वै वै स इ॒मान् । \newline
36. स इ॒मा नि॒मान् थ्स स इ॒मान् ॅलो॒कान् ॅलो॒का नि॒मान् थ्स स इ॒मान् ॅलो॒कान् । \newline
37. इ॒मान् ॅलो॒कान् ॅलो॒का नि॒मा नि॒मान् ॅलो॒काꣳ स्त्रे॒धा त्रे॒धा लो॒का नि॒मा नि॒मान् ॅलो॒काꣳ स्त्रे॒धा । \newline
38. लो॒काꣳ स्त्रे॒धा त्रे॒धा लो॒कान् ॅलो॒काꣳ स्त्रे॒धा ऽदु॑ह ददुहत् त्रे॒धा लो॒कान् ॅलो॒काꣳ स्त्रे॒धा ऽदु॑हत् । \newline
39. त्रे॒धा ऽदु॑ह ददुहत् त्रे॒धा त्रे॒धा ऽदु॑ह॒त् तत् तद॑दुहत् त्रे॒धा त्रे॒धा ऽदु॑ह॒त् तत् । \newline
40. अ॒दु॒ह॒त् तत् तद॑दुह ददुह॒त् तत् त्रि॒धातो᳚ स्त्रि॒धातो॒ स्तद॑दुह ददुह॒त् तत् त्रि॒धातोः᳚ । \newline
41. तत् त्रि॒धातो᳚ स्त्रि॒धातो॒ स्तत् तत् त्रि॒धातो᳚ स्त्रिधातु॒त्वम् त्रि॑धातु॒त्वम् त्रि॒धातो॒ स्तत् तत् त्रि॒धातो᳚ स्त्रिधातु॒त्वम् । \newline
42. त्रि॒धातो᳚ स्त्रिधातु॒त्वम् त्रि॑धातु॒त्वम् त्रि॒धातो᳚ स्त्रि॒धातो᳚ स्त्रिधातु॒त्वं ॅयं ॅयम् त्रि॑धातु॒त्वम् त्रि॒धातो᳚ स्त्रि॒धातो᳚स्त्रिधातु॒त्वं ॅयम् । \newline
43. त्रि॒धातो॒रिति॑ त्रि - धातोः᳚ । \newline
44. त्रि॒धा॒तु॒त्वं ॅयं ॅयम् त्रि॑धातु॒त्वम् त्रि॑धातु॒त्वं ॅयम् का॒मये॑त का॒मये॑त॒ यम् त्रि॑धातु॒त्वम् त्रि॑धातु॒त्वं ॅयम् का॒मये॑त । \newline
45. त्रि॒धा॒तु॒त्वमिति॑ त्रिधातु - त्वम् । \newline
46. यम् का॒मये॑त का॒मये॑त॒ यं ॅयम् का॒मये॑ता न्ना॒दो᳚ ऽन्ना॒दः का॒मये॑त॒ यं ॅयम् का॒मये॑तान्ना॒दः । \newline
47. का॒मये॑ता न्ना॒दो᳚ ऽन्ना॒दः का॒मये॑त का॒मये॑तान्ना॒दः स्या᳚थ् स्यादन्ना॒दः का॒मये॑त का॒मये॑तान्ना॒दः स्या᳚त् । \newline
48. अ॒न्ना॒दः स्या᳚थ् स्यादन्ना॒दो᳚ ऽन्ना॒दः स्या॒दितीति॑ स्यादन्ना॒दो᳚ ऽन्ना॒दः स्या॒दिति॑ । \newline
49. अ॒न्ना॒द इत्य॑न्न - अ॒दः । \newline
50. स्या॒दितीति॑ स्याथ् स्या॒दिति॒ तस्मै॒ तस्मा॒ इति॑ स्याथ् स्या॒दिति॒ तस्मै᳚ । \newline
51. इति॒ तस्मै॒ तस्मा॒ इतीति॒ तस्मा॑ ए॒त मे॒तम् तस्मा॒ इतीति॒ तस्मा॑ ए॒तम् । \newline
52. तस्मा॑ ए॒त मे॒तम् तस्मै॒ तस्मा॑ ए॒तम् त्रि॒धातु॑म् त्रि॒धातु॑ मे॒तम् तस्मै॒ तस्मा॑ ए॒तम् त्रि॒धातु᳚म् । \newline
53. ए॒तम् त्रि॒धातु॑म् त्रि॒धातु॑ मे॒त मे॒तम् त्रि॒धातु॒म् निर् णिष् ट्रि॒धातु॑ मे॒त मे॒तम् त्रि॒धातु॒म् निः । \newline
54. त्रि॒धातु॒म् निर् णिष् ट्रि॒धातु॑म् त्रि॒धातु॒म् निर् व॑पेद् वपे॒न् निष् ट्रि॒धातु॑म् त्रि॒धातु॒म् निर् व॑पेत् । \newline
55. त्रि॒धातु॒मिति॑ त्रि - धातु᳚म् । \newline
56. निर् व॑पेद् वपे॒न् निर् णिर् व॑पे॒ दिन्द्रा॒ये न्द्रा॑य वपे॒न् निर् णिर् व॑पे॒ दिन्द्रा॑य । \newline
57. व॒पे॒ दिन्द्रा॒ये न्द्रा॑य वपेद् वपे॒ दिन्द्रा॑य॒ राज्ञे॒ राज्ञ्॒ इन्द्रा॑य वपेद् वपे॒ दिन्द्रा॑य॒ राज्ञे᳚ । \newline
58. इन्द्रा॑य॒ राज्ञे॒ राज्ञ्॒ इन्द्रा॒ये न्द्रा॑य॒ राज्ञे॑ पुरो॒डाश॑म् पुरो॒डाशꣳ॒॒ राज्ञ्॒ इन्द्रा॒ये न्द्रा॑य॒ राज्ञे॑ पुरो॒डाश᳚म् । \newline
59. राज्ञे॑ पुरो॒डाश॑म् पुरो॒डाशꣳ॒॒ राज्ञे॒ राज्ञे॑ पुरो॒डाश॒ मेका॑दशकपाल॒ मेका॑दशकपालम् पुरो॒डाशꣳ॒॒ राज्ञे॒ राज्ञे॑ पुरो॒डाश॒ मेका॑दशकपालम् । \newline
60. पु॒रो॒डाश॒ मेका॑दशकपाल॒ मेका॑दशकपालम् पुरो॒डाश॑म् पुरो॒डाश॒ मेका॑दशकपाल॒ मिन्द्रा॒ये न्द्रा॒ यैका॑दशकपालम् पुरो॒डाश॑म् पुरो॒डाश॒ मेका॑दशकपाल॒ मिन्द्रा॑य । \newline
\pagebreak
\markright{ TS 2.3.6.2  \hfill https://www.vedavms.in \hfill}

\section{ TS 2.3.6.2 }

\textbf{TS 2.3.6.2 } \newline
\textbf{Samhita Paata} \newline

-मेका॑दशकपाल॒-मिन्द्रा॑या-धिरा॒जायेन्द्रा॑य स्व॒राज्ञे॒ऽयं ॅवा इन्द्रो॒ राजा॒ऽयमिन्द्रो॑-ऽधिरा॒जो॑-ऽसाविन्द्रः॑ स्व॒राडि॒माने॒व लो॒कान्थ्-स्वेन॑ भाग॒धेये॒नोप॑ धावति॒ त ए॒वास्मा॒ अन्नं॒ प्रय॑च्छन्त्यन्ना॒द ए॒व भ॑वति॒ यथा॑ व॒थ्सेन॒ प्रत्तां॒ गां दु॒ह ए॒वमे॒वेमान् ॅलो॒कान् प्रत्ता॒न् काम॑म॒न्नाद्यं॑ दुह उत्ता॒नेषु॑ क॒पाले॒ष्वधि॑ श्रय॒त्यया॑तयामत्वाय॒ त्रयः॑ ( ) पुरो॒डाशा॑ भवन्ति॒ त्रय॑ इ॒मे लो॒का ए॒षां ॅलो॒काना॒माप्त्या॒ उत्त॑र‌उत्तरो॒ ज्याया᳚न् भवत्ये॒वमि॑व॒ हीमे लो॒काः समृ॑द्ध्यै॒ सर्वे॑षामभि-ग॒मय॒न्नव॑ द्य॒त्यछ॑म्बट्कारं ॅव्य॒त्यास॒मन्वा॒हानि॑र्दाहाय ॥ \newline

\textbf{Pada Paata} \newline

एका॑दशकपाल॒मित्येका॑दश - क॒पा॒ल॒म् । इन्द्रा॑य । अ॒धि॒रा॒जायेत्य॑धि - रा॒जाय॑ । इन्द्रा॑य । स्व॒राज्ञ्॒ इति॑ स्व - राज्ञे᳚ । अ॒यम् । वै । इन्द्रः॑ । राजा᳚ । अ॒यम् । इन्द्रः॑ । अ॒धि॒रा॒ज इत्य॑धि - रा॒जः । अ॒सौ । इन्द्रः॑ । स्व॒राडिति॑ स्व - राट् । इ॒मान् । ए॒व । लो॒कान् । स्वेन॑ । भा॒ग॒धेये॒नेति॑ भाग - धेये॑न । उपेति॑ । धा॒व॒ति॒ । ते । ए॒व । अ॒स्मै॒ । अन्न᳚म् । प्रेति॑ । य॒च्छ॒न्ति॒ । अ॒न्ना॒द इत्य॑न्न - अ॒दः । ए॒व । भ॒व॒ति॒ । यथा᳚ । व॒थ्सेन॑ । प्रत्ता᳚म् । गाम् । दु॒हे । ए॒वम् । ए॒व ।  इ॒मान् । लो॒कान् । प्रत्तान्॑ । काम᳚म् । अ॒न्नाद्य॒मित्य॑न्न - अद्य᳚म् । दु॒हे॒ । उ॒त्ता॒नेष्वित्यु॑त्-ता॒नेषु॑ । क॒पाले॑षु । अधीति॑ । श्र॒य॒ति॒ । अया॑तयामत्वा॒येत्यया॑तयाम - त्वा॒य॒ । त्रयः॑ ( ) । पु॒रो॒डाशाः᳚ । भ॒व॒न्ति॒ । त्रयः॑ । इ॒मे । लो॒काः । ए॒षाम् । लो॒काना᳚म् । आप्त्यै᳚ । उत्त॑र उत्तर॒ इत्युत्त॑रः - उ॒त्त॒रः॒ । ज्यायान्॑ । भ॒व॒ति॒ । ए॒वम् । इ॒व॒ । हि । इ॒मे । लो॒काः । समृ॑द्ध्या॒ इति॒ सं-ऋ॒द्ध्यै॒ ।  सर्वे॑षाम् । अ॒भि॒ग॒मय॒न्नित्य॑भि - ग॒मयन्न्॑ । अवेति॑ । द्य॒ति॒ । अछ॑बंट्कार॒मित्यछ॑बंट् - का॒र॒म् । व्य॒त्यास॒मिति॑ वि - अ॒त्यास᳚म् । अन्विति॑ । आ॒ह॒ । अनि॑र्दाहा॒येत्यनिः॑ - दा॒हा॒य॒ ॥  \newline


\textbf{Krama Paata} \newline

एका॑दशकपाल॒मिन्द्रा॑य । एका॑दशकपाल॒मित्येका॑दश - क॒पा॒ल॒म् । इन्द्रा॑याधिरा॒जाय॑ । अ॒धि॒रा॒जायेन्द्रा॑य । अ॒धि॒रा॒जायेत्य॑धि - रा॒जाय॑ । इन्द्रा॑य स्व॒राज्ञे᳚ । स्व॒राज्ञे॒ ऽयम् । स्व॒राज्ञ्॒ इति॑ स्व - राज्ञे᳚ । अ॒यं ॅवै । वा इन्द्रः॑ । इन्द्रो॒ राजा᳚ । राजा॒ऽयम् । अ॒यमिन्द्रः॑ । इन्द्रो॑ ऽधिरा॒जः । अ॒धि॒रा॒जो॑ ऽसौ । अ॒धि॒रा॒ज इत्य॑धि - रा॒जः । अ॒साविन्द्रः॑ । इन्द्रः॑ स्व॒राट् । स्व॒राडि॒मान् । स्व॒राडिति॑ स्व - राट् । इ॒माने॒व । ए॒व लो॒कान् । लो॒कान्थ् स्वेन॑ । स्वेन॑ भाग॒धेये॑न । भा॒ग॒धेये॒नोप॑ । भा॒ग॒धेये॒नेति॑ भाग - धेये॑न । उप॑ धावति । धा॒व॒ति॒ ते । त ए॒व । ए॒वास्मै᳚ । अ॒स्मा॒ अन्न᳚म् । अन्न॒म् प्र । प्र य॑च्छन्ति । य॒च्छ॒न्त्य॒न्ना॒दः । अ॒न्ना॒द ए॒व । अ॒न्ना॒द इत्य॑न्न - अ॒दः । ए॒व भ॑वति । भ॒व॒ति॒ यथा᳚ । यथा॑ व॒थ्सेन॑ । व॒थ्सेन॒ प्रत्ता᳚म् । प्रत्ता॒म् गाम् । गाम् दु॒हे । दु॒ह ए॒वम् । ए॒वमे॒व । ए॒वेमान् । इ॒मान् ॅलो॒कान् । लो॒कान् प्रत्तान्॑ । प्रत्ता॒न् काम᳚म् । काम॑म॒न्नाद्य᳚म् । अ॒न्नाद्य॑म् दुहे । अ॒न्नाद्य॒मित्य॑न्न - अद्य᳚म् । दु॒ह॒ उ॒त्ता॒नेषु॑ । उ॒त्ता॒नेषु॑ क॒पाले॑षु । उ॒त्ता॒नेष्वित्यु॑त् - ता॒नेषु॑ । क॒पाले॒ष्वधि॑ । अधि॑ श्रयति । श्र॒य॒त्यया॑तयामत्वाय । अया॑तयामत्वाय॒ त्रयः॑ । अया॑तयामत्वा॒येत्यया॑तयाम - त्वा॒य॒ । त्रयः॑ पुरो॒डाशाः᳚ । पु॒रो॒डाशा॑ भवन्ति । भ॒व॒न्ति॒ त्रयः॑ ( ) । त्रय॑ इ॒मे । इ॒मे लो॒काः । लो॒का ए॒षाम् । ए॒षां ॅलो॒काना᳚म् । लो॒काना॒माप्त्यै᳚ । आप्त्या॒ उत्त॑र उत्तरः । उत्त॑र उत्तरो॒ ज्यायान्॑ । उत्त॑र उत्तर॒ इत्युत्त॑रः - उ॒त्त॒रः॒ । ज्याया᳚न् भवति । भ॒व॒त्ये॒वम् । ए॒वमि॑व । इ॒व॒ हि । हीमे । इ॒मे लो॒काः । लो॒काः समृ॑द्ध्यै । समृ॑द्ध्यै॒ सर्वे॑षाम् । समृ॑द्ध्या॒ इति॒ सं - ऋ॒द्ध्यै॒ । सर्वे॑षामभिग॒मयन्न्॑ । अ॒भि॒ग॒मय॒न्नव॑ । अ॒भि॒ग॒मय॒न्नित्य॑भि - ग॒मयन्न्॑ । अव॑ द्यति । द्य॒त्यछ॑म्बट्कारम् । अछ॑म्बट्कारं ॅव्य॒त्यास᳚म् । अछ॑म्बट्कार॒मित्यछ॑म्बट् - का॒र॒म् । व्य॒त्यास॒मनु॑ । व्य॒त्यास॒मिति॑ वि - अ॒त्यास᳚म् । अन्वा॑ह । आ॒हानि॑र्दाहाय । अनि॑र्दाहा॒येत्यनिः॑ - दा॒हा॒य॒ । \newline

\textbf{Jatai Paata} \newline

1. एका॑दशकपाल॒ मिन्द्रा॒ये न्द्रा॒यैका॑दशकपाल॒ मेका॑दशकपाल॒ मिन्द्रा॑य । \newline
2. एका॑दशकपाल॒मित्येका॑दश - क॒पा॒ल॒म् । \newline
3. इन्द्रा॑या धिरा॒जाया॑ धिरा॒जाये न्द्रा॒ये न्द्रा॑या धिरा॒जाय॑ । \newline
4. अ॒धि॒रा॒जाये न्द्रा॒ये न्द्रा॑या धिरा॒जाया॑ धिरा॒जाये न्द्रा॑य । \newline
5. अ॒धि॒रा॒जायेत्य॑धि - रा॒जाय॑ । \newline
6. इन्द्रा॑य स्व॒राज्ञे᳚ स्व॒राज्ञ्॒ इन्द्रा॒ये न्द्रा॑य स्व॒राज्ञे᳚ । \newline
7. स्व॒राज्ञे॒ ऽय म॒यꣳ स्व॒राज्ञे᳚ स्व॒राज्ञे॒ ऽयम् । \newline
8. स्व॒राज्ञ्॒ इति॑ स्व - राज्ञे᳚ । \newline
9. अ॒यं ॅवै वा अ॒य म॒यं ॅवै । \newline
10. वा इन्द्र॒ इन्द्रो॒ वै वा इन्द्रः॑ । \newline
11. इन्द्रो॒ राजा॒ राजेन्द्र॒ इन्द्रो॒ राजा᳚ । \newline
12. राजा॒ ऽय म॒यꣳ राजा॒ राजा॒ ऽयम् । \newline
13. अ॒य मिन्द्र॒ इन्द्रो॒ ऽय म॒य मिन्द्रः॑ । \newline
14. इन्द्रो॑ ऽधिरा॒जो॑ ऽधिरा॒ज इन्द्र॒ इन्द्रो॑ ऽधिरा॒जः । \newline
15. अ॒धि॒रा॒जो॑ ऽसा व॒सा व॑धिरा॒जो॑ ऽधिरा॒जो॑ ऽसौ । \newline
16. अ॒धि॒रा॒ज इत्य॑धि - रा॒जः । \newline
17. अ॒सा विन्द्र॒ इन्द्रो॒ ऽसा व॒सा विन्द्रः॑ । \newline
18. इन्द्रः॑ स्व॒राट् थ्स्व॒राडिन्द्र॒ इन्द्रः॑ स्व॒राट् । \newline
19. स्व॒रा डि॒मा नि॒मान् थ्स्व॒राट् थ्स्व॒रा डि॒मान् । \newline
20. स्व॒राडिति॑ स्व - राट् । \newline
21. इ॒मा ने॒वैवे मा नि॒मा ने॒व । \newline
22. ए॒व लो॒कान् ॅलो॒का ने॒वैव लो॒कान् । \newline
23. लो॒कान् थ्स्वेन॒ स्वेन॑ लो॒कान् ॅलो॒कान् थ्स्वेन॑ । \newline
24. स्वेन॑ भाग॒धेये॑न भाग॒धेये॑न॒ स्वेन॒ स्वेन॑ भाग॒धेये॑न । \newline
25. भा॒ग॒धेये॒नोपोप॑ भाग॒धेये॑न भाग॒धेये॒नोप॑ । \newline
26. भा॒ग॒धेये॒नेति॑ भाग - धेये॑न । \newline
27. उप॑ धावति धाव॒ त्युपोप॑ धावति । \newline
28. धा॒व॒ति॒ ते ते धा॑वति धावति॒ ते । \newline
29. त ए॒वैव ते त ए॒व । \newline
30. ए॒वास्मा॑ अस्मा ए॒वैवास्मै᳚ । \newline
31. अ॒स्मा॒ अन्न॒ मन्न॑ मस्मा अस्मा॒ अन्न᳚म् । \newline
32. अन्न॒म् प्र प्रान्न॒ मन्न॒म् प्र । \newline
33. प्र य॑च्छन्ति यच्छन्ति॒ प्र प्र य॑च्छन्ति । \newline
34. य॒च्छ॒ न्त्य॒न्ना॒दो᳚ ऽन्ना॒दो य॑च्छन्ति यच्छ न्त्यन्ना॒दः । \newline
35. अ॒न्ना॒द ए॒वैवा न्ना॒दो᳚ ऽन्ना॒द ए॒व । \newline
36. अ॒न्ना॒द इत्य॑न्न - अ॒दः । \newline
37. ए॒व भ॑वति भव त्ये॒वैव भ॑वति । \newline
38. भ॒व॒ति॒ यथा॒ यथा॑ भवति भवति॒ यथा᳚ । \newline
39. यथा॑ व॒थ्सेन॑ व॒थ्सेन॒ यथा॒ यथा॑ व॒थ्सेन॑ । \newline
40. व॒थ्सेन॒ प्रत्ता॒म् प्रत्तां᳚ ॅव॒थ्सेन॑ व॒थ्सेन॒ प्रत्ता᳚म् । \newline
41. प्रत्ता॒म् गाम् गाम् प्रत्ता॒म् प्रत्ता॒म् गाम् । \newline
42. गाम् दु॒हे दु॒हे गाम् गाम् दु॒हे । \newline
43. दु॒ह ए॒व मे॒वम् दु॒हे दु॒ह ए॒वम् । \newline
44. ए॒व मे॒वैवैव मे॒व मे॒व । \newline
45. ए॒वे मा नि॒मा ने॒वैवे मान् । \newline
46. इ॒मान् ॅलो॒कान् ॅलो॒का नि॒मा नि॒मान् ॅलो॒कान् । \newline
47. लो॒कान् प्रत्ता॒न् प्रत्ता᳚न् ॅलो॒कान् ॅलो॒कान् प्रत्तान्॑ । \newline
48. प्रत्ता॒न् काम॒म् काम॒म् प्रत्ता॒न् प्रत्ता॒न् काम᳚म् । \newline
49. काम॑ म॒न्नाद्य॑ म॒न्नाद्य॒म् काम॒म् काम॑ म॒न्नाद्य᳚म् । \newline
50. अ॒न्नाद्य॑म् दुहे दुहे॒ ऽन्नाद्य॑ म॒न्नाद्य॑म् दुहे । \newline
51. अ॒न्नाद्य॒मित्य॑न्न - अद्य᳚म् । \newline
52. दु॒ह॒ उ॒त्ता॒नेषू᳚ त्ता॒नेषु॑ दुहे दुह उत्ता॒नेषु॑ । \newline
53. उ॒त्ता॒नेषु॑ क॒पाले॑षु क॒पाले॑षू त्ता॒नेषू᳚ त्ता॒नेषु॑ क॒पाले॑षु । \newline
54. उ॒त्ता॒नेष्वित्यु॑त् - ता॒नेषु॑ । \newline
55. क॒पाले॒ ष्वध्यधि॑ क॒पाले॑षु क॒पाले॒ ष्वधि॑ । \newline
56. अधि॑ श्रयति श्रय॒ त्यध्यधि॑ श्रयति । \newline
57. श्र॒य॒ त्यया॑तयामत्वा॒या या॑तयामत्वाय श्रयति श्रय॒ त्यया॑तयामत्वाय । \newline
58. अया॑तयामत्वाय॒ त्रय॒स्त्रयो ऽया॑तयामत्वा॒या या॑तयामत्वाय॒ त्रयः॑ । \newline
59. अया॑तयामत्वा॒येत्यया॑तयाम - त्वा॒य॒ । \newline
60. त्रयः॑ पुरो॒डाशाः᳚ पुरो॒डाशा॒ स्त्रय॒ स्त्रयः॑ पुरो॒डाशाः᳚ । \newline
61. पु॒रो॒डाशा॑ भवन्ति भवन्ति पुरो॒डाशाः᳚ पुरो॒डाशा॑ भवन्ति । \newline
62. भ॒व॒न्ति॒ त्रय॒ स्त्रयो॑ भवन्ति भवन्ति॒ त्रयः॑ । \newline
63. त्रय॑ इ॒म इ॒मे त्रय॒ स्त्रय॑ इ॒मे । \newline
64. इ॒मे लो॒का लो॒का इ॒म इ॒मे लो॒काः । \newline
65. लो॒का ए॒षा मे॒षाम् ॅलो॒का लो॒का ए॒षाम् । \newline
66. ए॒षाम् ॅलो॒काना᳚म् ॅलो॒काना॑ मे॒षा मे॒षाम् ॅलो॒काना᳚म् । \newline
67. लो॒काना॒ माप्त्या॒ आप्त्यै॑ लो॒काना᳚म् ॅलो॒काना॒ माप्त्यै᳚ । \newline
68. आप्त्या॒ उत्त॑र‌उत्तर॒ उत्त॑र‌उत्तर॒ आप्त्या॒ आप्त्या॒ उत्त॑र‌उत्तरः । \newline
69. उत्त॑र‌उत्तरो॒ ज्याया॒न् ज्याया॒ नुत्त॑र‌उत्तर॒ उत्त॑र‌उत्तरो॒ ज्यायान्॑ । \newline
70. उत्त॑र‌उत्तर॒ इत्युत्त॑रः - उ॒त्त॒रः॒ । \newline
71. ज्याया᳚न् भवति भवति॒ ज्याया॒न् ज्याया᳚न् भवति । \newline
72. भ॒व॒ त्ये॒व मे॒वम् भ॑वति भव त्ये॒वम् । \newline
73. ए॒व मि॑वे वै॒व मे॒व मि॑व । \newline
74. इ॒व॒ हि हीवे॑ व॒ हि । \newline
75. हीम इ॒मे हि हीमे । \newline
76. इ॒मे लो॒का लो॒का इ॒म इ॒मे लो॒काः । \newline
77. लो॒काः समृ॑द्ध्यै॒ समृ॑द्ध्यै लो॒का लो॒काः समृ॑द्ध्यै । \newline
78. समृ॑द्ध्यै॒ सर्वे॑षाꣳ॒॒ सर्वे॑षाꣳ॒॒ समृ॑द्ध्यै॒ समृ॑द्ध्यै॒ सर्वे॑षाम् । \newline
79. समृ॑द्ध्या॒ इति॒ सं - ऋ॒द्ध्यै॒ । \newline
80. सर्वे॑षा मभिग॒मय॑न् नभिग॒मय॒न् थ्सर्वे॑षाꣳ॒॒ सर्वे॑षा मभिग॒मयन्न्॑ । \newline
81. अ॒भि॒ग॒मय॒न् नवावा॑ भिग॒मय॑न् नभिग॒मय॒न् नव॑ । \newline
82. अ॒भि॒ग॒मय॒न्नित्य॑भि - ग॒मयन्न्॑ । \newline
83. अव॑ द्यति द्य॒त्यवाव॑ द्यति । \newline
84. द्य॒त्यछं॑बट्कार॒ मछं॑बट्कारम् द्यति द्य॒त्यछं॑बट्कारम् । \newline
85. अछं॑बट्कारं ॅव्य॒त्यासं॑ ॅव्य॒त्यास॒ मछं॑बट्कार॒ मछं॑बट्कारं ॅव्य॒त्यास᳚म् । \newline
86. अछं॑बट्कार॒मित्यछं॑बट् - का॒र॒म् । \newline
87. व्य॒त्यास॒ मन्वनु॑ व्य॒त्यासं॑ ॅव्य॒त्यास॒ मनु॑ । \newline
88. व्य॒त्यास॒मिति॑ वि - अ॒त्यास᳚म् । \newline
89. अन्वा॑ हा॒हा न्वन्वा॑ह । \newline
90. आ॒हा नि॑र्दाहा॒या नि॑र्दाहाया हा॒हा नि॑र्दाहाय । \newline
91. अनि॑र्दाहा॒येत्यनिः॑ - दा॒हा॒य॒ । \newline

\textbf{Ghana Paata } \newline

1. एका॑दशकपाल॒ मिन्द्रा॒ये न्द्रा॒यैका॑दशकपाल॒ मेका॑दशकपाल॒ मिन्द्रा॑या धिरा॒जाया॑ धिरा॒जाये न्द्रा॒यैका॑दशकपाल॒ मेका॑दशकपाल॒ मिन्द्रा॑या धिरा॒जाय॑ । \newline
2. एका॑दशकपाल॒मित्येका॑दश - क॒पा॒ल॒म् । \newline
3. इन्द्रा॑या धिरा॒जाया॑ धिरा॒जाये न्द्रा॒ये न्द्रा॑या धिरा॒जाये न्द्रा॒ये न्द्रा॑या धिरा॒जाये न्द्रा॒ये न्द्रा॑या धिरा॒जाये न्द्रा॑य । \newline
4. अ॒धि॒रा॒जाये न्द्रा॒ये न्द्रा॑या धिरा॒जाया॑ धिरा॒जाये न्द्रा॑य स्व॒राज्ञे᳚ स्व॒राज्ञ्॒ इन्द्रा॑या धिरा॒जाया॑ धिरा॒जाये न्द्रा॑य स्व॒राज्ञे᳚ । \newline
5. अ॒धि॒रा॒जायेत्य॑धि - रा॒जाय॑ । \newline
6. इन्द्रा॑य स्व॒राज्ञे᳚ स्व॒राज्ञ्॒ इन्द्रा॒ये न्द्रा॑य स्व॒राज्ञे॒ ऽय म॒यꣳ स्व॒राज्ञ्॒ इन्द्रा॒ये न्द्रा॑य स्व॒राज्ञे॒ ऽयम् । \newline
7. स्व॒राज्ञे॒ ऽय म॒यꣳ स्व॒राज्ञे᳚ स्व॒राज्ञे॒ ऽयं ॅवै वा अ॒यꣳ स्व॒राज्ञे᳚ स्व॒राज्ञे॒ ऽयं ॅवै । \newline
8. स्व॒राज्ञ्॒ इति॑ स्व - राज्ञे᳚ । \newline
9. अ॒यं ॅवै वा अ॒य म॒यं ॅवा इन्द्र॒ इन्द्रो॒ वा अ॒य म॒यं ॅवा इन्द्रः॑ । \newline
10. वा इन्द्र॒ इन्द्रो॒ वै वा इन्द्रो॒ राजा॒ राजेन्द्रो॒ वै वा इन्द्रो॒ राजा᳚ । \newline
11. इन्द्रो॒ राजा॒ राजेन्द्र॒ इन्द्रो॒ राजा॒ ऽय म॒यꣳ राजेन्द्र॒ इन्द्रो॒ राजा॒ ऽयम् । \newline
12. राजा॒ ऽय म॒यꣳ राजा॒ राजा॒ ऽय मिन्द्र॒ इन्द्रो॒ ऽयꣳ राजा॒ राजा॒ ऽय मिन्द्रः॑ । \newline
13. अ॒य मिन्द्र॒ इन्द्रो॒ ऽय म॒य मिन्द्रो॑ ऽधिरा॒जो॑ ऽधिरा॒ज इन्द्रो॒ ऽय म॒य मिन्द्रो॑ ऽधिरा॒जः । \newline
14. इन्द्रो॑ ऽधिरा॒जो॑ ऽधिरा॒ज इन्द्र॒ इन्द्रो॑ ऽधिरा॒जो॑ ऽसा व॒सा व॑धिरा॒ज इन्द्र॒ इन्द्रो॑ ऽधिरा॒जो॑ ऽसौ । \newline
15. अ॒धि॒रा॒जो॑ ऽसा व॒सा व॑धिरा॒जो॑ ऽधिरा॒जो॑ ऽसा विन्द्र॒ इन्द्रो॒ ऽसा व॑धिरा॒जो॑ ऽधिरा॒जो॑ ऽसा विन्द्रः॑ । \newline
16. अ॒धि॒रा॒ज इत्य॑धि - रा॒जः । \newline
17. अ॒सा विन्द्र॒ इन्द्रो॒ ऽसा व॒सा विन्द्रः॑ स्व॒राट् थ्स्व॒राडिन्द्रो॒ ऽसा व॒सा विन्द्रः॑ स्व॒राट् । \newline
18. इन्द्रः॑ स्व॒राट् थ्स्व॒राडिन्द्र॒ इन्द्रः॑ स्व॒राडि॒मा नि॒मान् थ्स्व॒राडिन्द्र॒ इन्द्रः॑ स्व॒राडि॒मान् । \newline
19. स्व॒राडि॒मा नि॒मान् थ्स्व॒राट् थ्स्व॒राडि॒मा ने॒वैवे मान् थ्स्व॒राट् थ्स्व॒राडि॒मा ने॒व । \newline
20. स्व॒राडिति॑ स्व - राट् । \newline
21. इ॒मा ने॒वैवे मा नि॒मा ने॒व लो॒कान् ॅलो॒का ने॒वे मा नि॒मा ने॒व लो॒कान् । \newline
22. ए॒व लो॒कान् ॅलो॒का ने॒वैव लो॒कान् थ्स्वेन॒ स्वेन॑ लो॒का ने॒वैव लो॒कान् थ्स्वेन॑ । \newline
23. लो॒कान् थ्स्वेन॒ स्वेन॑ लो॒कान् ॅलो॒कान् थ्स्वेन॑ भाग॒धेये॑न भाग॒धेये॑न॒ स्वेन॑ लो॒कान् ॅलो॒कान् थ्स्वेन॑ भाग॒धेये॑न । \newline
24. स्वेन॑ भाग॒धेये॑न भाग॒धेये॑न॒ स्वेन॒ स्वेन॑ भाग॒धेये॒नोपोप॑ भाग॒धेये॑न॒ स्वेन॒ स्वेन॑ भाग॒धेये॒नोप॑ । \newline
25. भा॒ग॒धेये॒नोपोप॑ भाग॒धेये॑न भाग॒धेये॒नोप॑ धावति धाव॒त्युप॑ भाग॒धेये॑न भाग॒धेये॒नोप॑ धावति । \newline
26. भा॒ग॒धेये॒नेति॑ भाग - धेये॑न । \newline
27. उप॑ धावति धाव॒ त्युपोप॑ धावति॒ ते ते धा॑व॒ त्युपोप॑ धावति॒ ते । \newline
28. धा॒व॒ति॒ ते ते धा॑वति धावति॒ त ए॒वैव ते धा॑वति धावति॒ त ए॒व । \newline
29. त ए॒वैव ते त ए॒वास्मा॑ अस्मा ए॒व ते त ए॒वास्मै᳚ । \newline
30. ए॒वास्मा॑ अस्मा ए॒वैवास्मा॒ अन्न॒ मन्न॑ मस्मा ए॒वैवास्मा॒ अन्न᳚म् । \newline
31. अ॒स्मा॒ अन्न॒ मन्न॑ मस्मा अस्मा॒ अन्न॒म् प्र प्रान्न॑ मस्मा अस्मा॒ अन्न॒म् प्र । \newline
32. अन्न॒म् प्र प्रान्न॒ मन्न॒म् प्र य॑च्छन्ति यच्छन्ति॒ प्रान्न॒ मन्न॒म् प्र य॑च्छन्ति । \newline
33. प्र य॑च्छन्ति यच्छन्ति॒ प्र प्र य॑च्छ न्त्यन्ना॒दो᳚ ऽन्ना॒दो य॑च्छन्ति॒ प्र प्र य॑च्छ न्त्यन्ना॒दः । \newline
34. य॒च्छ॒ न्त्य॒न्ना॒दो᳚ ऽन्ना॒दो य॑च्छन्ति यच्छ न्त्यन्ना॒द ए॒वैवा न्ना॒दो य॑च्छन्ति यच्छ न्त्यन्ना॒द ए॒व । \newline
35. अ॒न्ना॒द ए॒वैवा न्ना॒दो᳚ ऽन्ना॒द ए॒व भ॑वति भव त्ये॒वा न्ना॒दो᳚ ऽन्ना॒द ए॒व भ॑वति । \newline
36. अ॒न्ना॒द इत्य॑न्न - अ॒दः । \newline
37. ए॒व भ॑वति भव त्ये॒वैव भ॑वति॒ यथा॒ यथा॑ भव त्ये॒वैव भ॑वति॒ यथा᳚ । \newline
38. भ॒व॒ति॒ यथा॒ यथा॑ भवति भवति॒ यथा॑ व॒थ्सेन॑ व॒थ्सेन॒ यथा॑ भवति भवति॒ यथा॑ व॒थ्सेन॑ । \newline
39. यथा॑ व॒थ्सेन॑ व॒थ्सेन॒ यथा॒ यथा॑ व॒थ्सेन॒ प्रत्ता॒म् प्रत्तां᳚ ॅव॒थ्सेन॒ यथा॒ यथा॑ व॒थ्सेन॒ प्रत्ता᳚म् । \newline
40. व॒थ्सेन॒ प्रत्ता॒म् प्रत्तां᳚ ॅव॒थ्सेन॑ व॒थ्सेन॒ प्रत्ता॒म् गाम् गाम् प्रत्तां᳚ ॅव॒थ्सेन॑ व॒थ्सेन॒ प्रत्ता॒म् गाम् । \newline
41. प्रत्ता॒म् गाम् गाम् प्रत्ता॒म् प्रत्ता॒म् गाम् दु॒हे दु॒हे गाम् प्रत्ता॒म् प्रत्ता॒म् गाम् दु॒हे । \newline
42. गाम् दु॒हे दु॒हे गाम् गाम् दु॒ह ए॒व मे॒वम् दु॒हे गाम् गाम् दु॒ह ए॒वम् । \newline
43. दु॒ह ए॒व मे॒वम् दु॒हे दु॒ह ए॒व मे॒वैवैवम् दु॒हे दु॒ह ए॒व मे॒व । \newline
44. ए॒व मे॒वैवैव मे॒व मे॒वे मा नि॒मा ने॒वैव मे॒व मे॒वे मान् । \newline
45. ए॒वे मा नि॒मा ने॒वैवे मान् ॅलो॒कान् ॅलो॒का नि॒मा ने॒वैवे मान् ॅलो॒कान् । \newline
46. इ॒मान् ॅलो॒कान् ॅलो॒का नि॒मा नि॒मान् ॅलो॒कान् प्रत्ता॒न् प्रत्ता᳚न् ॅलो॒का नि॒मा नि॒मान् ॅलो॒कान् प्रत्तान्॑ । \newline
47. लो॒कान् प्रत्ता॒न् प्रत्ता᳚न् ॅलो॒कान् ॅलो॒कान् प्रत्ता॒न् काम॒म् काम॒म् प्रत्ता᳚न् ॅलो॒कान् ॅलो॒कान् प्रत्ता॒न् काम᳚म् । \newline
48. प्रत्ता॒न् काम॒म् काम॒म् प्रत्ता॒न् प्रत्ता॒न् काम॑ म॒न्नाद्य॑ म॒न्नाद्य॒म् काम॒म् प्रत्ता॒न् प्रत्ता॒न् काम॑ म॒न्नाद्य᳚म् । \newline
49. काम॑ म॒न्नाद्य॑ म॒न्नाद्य॒म् काम॒म् काम॑ म॒न्नाद्य॑म् दुहे दुहे॒ ऽन्नाद्य॒म् काम॒म् काम॑ म॒न्नाद्य॑म् दुहे । \newline
50. अ॒न्नाद्य॑म् दुहे दुहे॒ ऽन्नाद्य॑ म॒न्नाद्य॑म् दुह उत्ता॒नेषू᳚ त्ता॒नेषु॑ दुहे॒ ऽन्नाद्य॑ म॒न्नाद्य॑म् दुह उत्ता॒नेषु॑ । \newline
51. अ॒न्नाद्य॒मित्य॑न्न - अद्य᳚म् । \newline
52. दु॒ह॒ उ॒त्ता॒नेषू᳚ त्ता॒नेषु॑ दुहे दुह उत्ता॒नेषु॑ क॒पाले॑षु क॒पाले॑षू त्ता॒नेषु॑ दुहे दुह उत्ता॒नेषु॑ क॒पाले॑षु । \newline
53. उ॒त्ता॒नेषु॑ क॒पाले॑षु क॒पाले॑षू त्ता॒नेषू᳚ त्ता॒नेषु॑ क॒पाले॒ ष्वध्यधि॑ क॒पाले॑षू त्ता॒नेषू᳚ त्ता॒नेषु॑ क॒पाले॒ष्वधि॑ । \newline
54. उ॒त्ता॒नेष्वित्यु॑त् - ता॒नेषु॑ । \newline
55. क॒पाले॒ ष्वध्यधि॑ क॒पाले॑षु क॒पाले॒ ष्वधि॑ श्रयति श्रय॒त्यधि॑ क॒पाले॑षु क॒पाले॒ ष्वधि॑ श्रयति । \newline
56. अधि॑ श्रयति श्रय॒ त्यध्यधि॑ श्रय॒ त्यया॑तयामत्वा॒या या॑तयामत्वाय श्रय॒ त्यध्यधि॑ श्रय॒ त्यया॑तयामत्वाय । \newline
57. श्र॒य॒ त्यया॑तयामत्वा॒या या॑तयामत्वाय श्रयति श्रय॒ त्यया॑तयामत्वाय॒ त्रय॒ स्त्रयो ऽया॑तयामत्वाय श्रयति श्रय॒ त्यया॑तयामत्वाय॒ त्रयः॑ । \newline
58. अया॑तयामत्वाय॒ त्रय॒ स्त्रयो ऽया॑तयामत्वा॒या या॑तयामत्वाय॒ त्रयः॑ पुरो॒डाशाः᳚ पुरो॒डाशा॒ स्त्रयो ऽया॑तयामत्वा॒या या॑तयामत्वाय॒ त्रयः॑ पुरो॒डाशाः᳚ । \newline
59. अया॑तयामत्वा॒येत्यया॑तयाम - त्वा॒य॒ । \newline
60. त्रयः॑ पुरो॒डाशाः᳚ पुरो॒डाशा॒ स्त्रय॒ स्त्रयः॑ पुरो॒डाशा॑ भवन्ति भवन्ति पुरो॒डाशा॒ स्त्रय॒ स्त्रयः॑ पुरो॒डाशा॑ भवन्ति । \newline
61. पु॒रो॒डाशा॑ भवन्ति भवन्ति पुरो॒डाशाः᳚ पुरो॒डाशा॑ भवन्ति॒ त्रय॒ स्त्रयो॑ भवन्ति पुरो॒डाशाः᳚ पुरो॒डाशा॑ भवन्ति॒ त्रयः॑ । \newline
62. भ॒व॒न्ति॒ त्रय॒ स्त्रयो॑ भवन्ति भवन्ति॒ त्रय॑ इ॒म इ॒मे त्रयो॑ भवन्ति भवन्ति॒ त्रय॑ इ॒मे । \newline
63. त्रय॑ इ॒म इ॒मे त्रय॒ स्त्रय॑ इ॒मे लो॒का लो॒का इ॒मे त्रय॒ स्त्रय॑ इ॒मे लो॒काः । \newline
64. इ॒मे लो॒का लो॒का इ॒म इ॒मे लो॒का ए॒षा मे॒षाम् ॅलो॒का इ॒म इ॒मे लो॒का ए॒षाम् । \newline
65. लो॒का ए॒षा मे॒षाम् ॅलो॒का लो॒का ए॒षाम् ॅलो॒काना᳚म् ॅलो॒काना॑ मे॒षाम् ॅलो॒का लो॒का ए॒षाम् ॅलो॒काना᳚म् । \newline
66. ए॒षाम् ॅलो॒काना᳚म् ॅलो॒काना॑ मे॒षा मे॒षाम् ॅलो॒काना॒ माप्त्या॒ आप्त्यै॑ लो॒काना॑ मे॒षा मे॒षाम् ॅलो॒काना॒ माप्त्यै᳚ । \newline
67. लो॒काना॒ माप्त्या॒ आप्त्यै॑ लो॒काना᳚म् ॅलो॒काना॒ माप्त्या॒ उत्त॑र‌उत्तर॒ उत्त॑र‌उत्तर॒ आप्त्यै॑ लो॒काना᳚म् ॅलो॒काना॒ माप्त्या॒ उत्त॑र‌उत्तरः । \newline
68. आप्त्या॒ उत्त॑र‌उत्तर॒ उत्त॑र‌उत्तर॒ आप्त्या॒ आप्त्या॒ उत्त॑र‌उत्तरो॒ ज्याया॒न् ज्याया॒ नुत्त॑र‌उत्तर॒ आप्त्या॒ आप्त्या॒ उत्त॑र‌उत्तरो॒ ज्यायान्॑ । \newline
69. उत्त॑र‌उत्तरो॒ ज्याया॒न् ज्याया॒ नुत्त॑र‌उत्तर॒ उत्त॑र‌उत्तरो॒ ज्याया᳚न् भवति भवति॒ ज्याया॒ नुत्त॑र‌उत्तर॒ उत्त॑र‌उत्तरो॒ ज्याया᳚न् भवति । \newline
70. उत्त॑र‌उत्तर॒ इत्युत्त॑रः - उ॒त्त॒रः॒ । \newline
71. ज्याया᳚न् भवति भवति॒ ज्याया॒न् ज्याया᳚न् भवत्ये॒व मे॒वम् भ॑वति॒ ज्याया॒न् ज्याया᳚न् भवत्ये॒वम् । \newline
72. भ॒व॒त्ये॒व मे॒वम् भ॑वति भवत्ये॒व मि॑वे वै॒वम् भ॑वति भवत्ये॒व मि॑व । \newline
73. ए॒व मि॑वे वै॒व मे॒व मि॑व॒ हि हीवै॒व मे॒व मि॑व॒ हि । \newline
74. इ॒व॒ हि हीवे॑ व॒ हीम इ॒मे हीवे॑ व॒ हीमे । \newline
75. हीम इ॒मे हि हीमे लो॒का लो॒का इ॒मे हि हीमे लो॒काः । \newline
76. इ॒मे लो॒का लो॒का इ॒म इ॒मे लो॒काः समृ॑द्ध्यै॒ समृ॑द्ध्यै लो॒का इ॒म इ॒मे लो॒काः समृ॑द्ध्यै । \newline
77. लो॒काः समृ॑द्ध्यै॒ समृ॑द्ध्यै लो॒का लो॒काः समृ॑द्ध्यै॒ सर्वे॑षाꣳ॒॒ सर्वे॑षाꣳ॒॒ समृ॑द्ध्यै लो॒का लो॒काः समृ॑द्ध्यै॒ सर्वे॑षाम् । \newline
78. समृ॑द्ध्यै॒ सर्वे॑षाꣳ॒॒ सर्वे॑षाꣳ॒॒ समृ॑द्ध्यै॒ समृ॑द्ध्यै॒ सर्वे॑षा मभिग॒मय॑न् नभिग॒मय॒न् थ्सर्वे॑षाꣳ॒॒ समृ॑द्ध्यै॒ समृ॑द्ध्यै॒ सर्वे॑षा मभिग॒मयन्न्॑ । \newline
79. समृ॑द्ध्या॒ इति॒ सं - ऋ॒द्ध्यै॒ । \newline
80. सर्वे॑षा मभिग॒मय॑न् नभिग॒मय॒न् थ्सर्वे॑षाꣳ॒॒ सर्वे॑षा मभिग॒मय॒न् नवावा॑ भिग॒मय॒न् थ्सर्वे॑षाꣳ॒॒ सर्वे॑षा मभिग॒मय॒न् नव॑ । \newline
81. अ॒भि॒ग॒मय॒न् नवावा॑भिग॒मय॑न् नभिग॒मय॒न् नव॑ द्यति द्य॒त्यवा॑भिग॒मय॑न् नभिग॒मय॒न् नव॑ द्यति । \newline
82. अ॒भि॒ग॒मय॒न्नित्य॑भि - ग॒मयन्न्॑ । \newline
83. अव॑ द्यति द्य॒त्यवाव॑ द्य॒त्यछं॑बट्कार॒ मछं॑बट्कारम् द्य॒त्यवाव॑ द्य॒त्यछं॑बट्कारम् । \newline
84. द्य॒त्यछं॑बट्कार॒ मछं॑बट्कारम् द्यति द्य॒त्यछं॑बट्कारं ॅव्य॒त्यासं॑ ॅव्य॒त्यास॒ मछं॑बट्कारम् द्यति द्य॒त्यछं॑बट्कारं ॅव्य॒त्यास᳚म् । \newline
85. अछं॑बट्कारं ॅव्य॒त्यासं॑ ॅव्य॒त्यास॒ मछं॑बट्कार॒ मछं॑बट्कारं ॅव्य॒त्यास॒ मन्वनु॑ व्य॒त्यास॒ मछं॑बट्कार॒ मछं॑बट्कारं ॅव्य॒त्यास॒ मनु॑ । \newline
86. अछं॑बट्कार॒मित्यछं॑बट् - का॒र॒म् । \newline
87. व्य॒त्यास॒ मन्वनु॑ व्य॒त्यासं॑ ॅव्य॒त्यास॒ मन्वा॑हा॒हानु॑ व्य॒त्यासं॑ ॅव्य॒त्यास॒ मन्वा॑ह । \newline
88. व्य॒त्यास॒मिति॑ वि - अ॒त्यास᳚म् । \newline
89. अन्वा॑हा॒हा न्वन्वा॒हा नि॑र्दाहा॒या नि॑र्दाहाया॒हा न्वन्वा॒हा नि॑र्दाहाय । \newline
90. आ॒हा नि॑र्दाहा॒या नि॑र्दाहायाहा॒हा नि॑र्दाहाय । \newline
91. अनि॑र्दाहा॒येत्यनिः॑ - दा॒हा॒य॒ । \newline
\pagebreak
\markright{ TS 2.3.7.1  \hfill https://www.vedavms.in \hfill}

\section{ TS 2.3.7.1 }

\textbf{TS 2.3.7.1 } \newline
\textbf{Samhita Paata} \newline

दे॒वा॒सु॒राः संॅय॑त्ता आस॒न् तान् दे॒वानसु॑रा अजय॒न् ते दे॒वाः प॑राजिग्या॒ना असु॑राणां॒ ॅवैश्य॒मुपा॑ऽऽय॒न् तेभ्य॑ इन्द्रि॒यं ॅवी॒र्य॑मपा᳚क्राम॒त् तदिन्द्रो॑ऽचाय॒त् तदन्वपा᳚क्राम॒त् तद॑व॒रुधं॒ नाश॑क्नो॒त् तद॑स्मादभ्य॒र्द्धो॑ ऽचर॒थ् स प्र॒जाप॑ति॒मुपा॑धाव॒त् तमे॒तया॒ सर्व॑पृष्ठयाऽयाजय॒त् तयै॒वास्मि॑न्निन्द्रि॒यं ॅवी॒र्य॑मदधा॒द्य इ॑न्द्रि॒यका॑मो - [  ] \newline

\textbf{Pada Paata} \newline

दे॒वा॒सु॒रा इति॑ देव - अ॒सु॒राः । संॅय॑त्ता॒ इति॒ सं-य॒त्ताः॒ । आ॒स॒न्न् । तान् । दे॒वान् । असु॑राः । अ॒ज॒य॒न्न् । ते । दे॒वाः । प॒रा॒जि॒ग्या॒ना इति॑ परा - जि॒ग्या॒नाः । असु॑राणाम् । वैश्य᳚म् । उपेति॑ । आ॒य॒न्न् । तेभ्यः॑ । इ॒न्द्रि॒यम् । वी॒र्य᳚म् । अपेति॑ । अ॒क्रा॒म॒त् । तत् । इन्द्रः॑ । अ॒चा॒य॒त् । तत् । अनु॑ । अपेति॑ । अ॒क्रा॒म॒त् । तत् । अ॒व॒रुध॒मित्य॑व - रुध᳚म् । न । अ॒श॒क्नो॒त् । तत् । अ॒स्मा॒त् । अ॒भ्य॒र्द्ध इत्य॑भि - अ॒र्द्धः । अ॒च॒र॒त् । सः । प्र॒जाप॑ति॒मिति॑ प्र॒जा - प॒ति॒म् । उपेति॑ । अ॒धा॒व॒त् । तम् । ए॒तया᳚ । सर्व॑पृष्ठ॒येति॒ सर्व॑ - पृ॒ष्ठ॒या॒ । अ॒या॒ज॒य॒त् । तया᳚ । ए॒व । अ॒स्मि॒न्न् । इ॒न्द्रि॒यम् । वी॒र्य᳚म् । अ॒द॒धा॒त् । यः । इ॒न्द्रि॒यका॑म॒ इती᳚न्द्रि॒य - का॒मः॒ ।  \newline


\textbf{Krama Paata} \newline

दे॒वा॒सु॒राः सम्ॅय॑त्ताः । दे॒वा॒सु॒रा इति॑ देव - अ॒सु॒राः । सम्ॅय॑त्ता आसन्न् । सम्ॅय॑त्ता॒ इति॒ सं - य॒त्ताः॒ । आ॒स॒न् तान् । तान् दे॒वान् । दे॒वानसु॑राः । असु॑रा अजयन्न् । अ॒ज॒य॒न् ते । ते दे॒वाः । दे॒वाः प॑राजिग्या॒नाः । प॒रा॒जि॒ग्या॒ना असु॑राणाम् । प॒रा॒जि॒ग्या॒ना इति॑ परा - जि॒ग्या॒नाः । असु॑राणां॒ ॅवैश्य᳚म् । वैश्य॒मुप॑ । उपा॑यन्न् । आ॒य॒न् तेभ्यः॑ । तेभ्य॑ इन्द्रि॒यम् । इ॒न्द्रि॒यं ॅवी॒र्य᳚म् । वी॒र्य॑मप॑ । अपा᳚क्रामत् । अ॒क्रा॒म॒त् तत् । तदिन्द्रः॑ । इन्द्रो॑ऽचायत् । अ॒चा॒य॒त् तत् । तदनु॑ । अन्वप॑ । अपा᳚क्रामत् । अ॒क्रा॒म॒त् तत् । तद॑व॒रुध᳚म् । अ॒व॒रुध॒म् न । अ॒व॒रुध॒मित्य॑व - रुध᳚म् । नाश॑क्नोत् । अ॒श॒क्नो॒त् तत् । तद॑स्मात् । अ॒स्मा॒द॒भ्य॒र्द्धः । अ॒भ्य॒र्द्धो॑ ऽचरत् । अ॒भ्य॒र्द्ध इत्य॑भि - अ॒र्द्धः । अ॒च॒र॒थ् सः । स प्र॒जाप॑तिम् । प्र॒जाप॑ति॒मुप॑ । प्र॒जाप॑ति॒मिति॑ प्र॒जा - प॒ति॒म् । उपा॑धावत् । अ॒धा॒व॒त् तम् । तमे॒तया᳚ । ए॒तया॒ सर्व॑पृष्ठया । सर्व॑पृष्ठया ऽयाजयत् । सर्व॑पृष्ठ॒येति॒ सर्व॑ - पृ॒ष्ठ॒या॒ । अ॒या॒ज॒य॒त् तया᳚ । तयै॒व । ए॒वास्मिन्न्॑ । अ॒स्मि॒न्नि॒न्द्रि॒यम् । इ॒न्द्रि॒यं ॅवी॒र्य᳚म् । वी॒र्य॑मदधात् । अ॒द॒धा॒द् यः । य इ॑न्द्रि॒यका॑मः । इ॒न्द्रि॒यका॑मो वी॒र्य॑कामः । इ॒न्द्रि॒यका॑म॒ इती᳚न्द्रि॒य - का॒मः॒ \newline

\textbf{Jatai Paata} \newline

1. दे॒वा॒सु॒राः संॅय॑त्ताः॒ संॅय॑त्ता देवासु॒रा दे॑वासु॒राः संॅय॑त्ताः । \newline
2. दे॒वा॒सु॒रा इति॑ देव - अ॒सु॒राः । \newline
3. संॅय॑त्ता आसन् नास॒न् थ्संॅय॑त्ताः॒ संॅय॑त्ता आसन्न् । \newline
4. संॅय॑त्ता॒ इति॒ सं - य॒त्ताः॒ । \newline
5. आ॒स॒न् ताꣳ स्ता ना॑सन् नास॒न् तान् । \newline
6. तान् दे॒वान् दे॒वान् ताꣳ स्तान् दे॒वान् । \newline
7. दे॒वा नसु॑रा॒ असु॑रा दे॒वान् दे॒वा नसु॑राः । \newline
8. असु॑रा अजयन् नजय॒न् नसु॑रा॒ असु॑रा अजयन्न् । \newline
9. अ॒ज॒य॒न् ते ते॑ ऽजयन् नजय॒न् ते । \newline
10. ते दे॒वा दे॒वा स्ते ते दे॒वाः । \newline
11. दे॒वाः प॑राजिग्या॒नाः प॑राजिग्या॒ना दे॒वा दे॒वाः प॑राजिग्या॒नाः । \newline
12. प॒रा॒जि॒ग्या॒ना असु॑राणा॒ मसु॑राणाम् पराजिग्या॒नाः प॑राजिग्या॒ना असु॑राणाम् । \newline
13. प॒रा॒जि॒ग्या॒ना इति॑ परा - जि॒ग्या॒नाः । \newline
14. असु॑राणां॒ ॅवैश्यं॒ ॅवैश्य॒ मसु॑राणा॒ मसु॑राणां॒ ॅवैश्य᳚म् । \newline
15. वैश्य॒ मुपोप॒ वैश्यं॒ ॅवैश्य॒ मुप॑ । \newline
16. उपा॑यन् नाय॒न् नुपोपा॑यन्न् । \newline
17. आ॒य॒न् तेभ्य॒ स्तेभ्य॑ आयन् नाय॒न् तेभ्यः॑ । \newline
18. तेभ्य॑ इन्द्रि॒य मि॑न्द्रि॒यम् तेभ्य॒ स्तेभ्य॑ इन्द्रि॒यम् । \newline
19. इ॒न्द्रि॒यं ॅवी॒र्यं॑ ॅवी॒र्य॑ मिन्द्रि॒य मि॑न्द्रि॒यं ॅवी॒र्य᳚म् । \newline
20. वी॒र्य॑ मपाप॑ वी॒र्यं॑ ॅवी॒र्य॑ मप॑ । \newline
21. अपा᳚ क्राम दक्राम॒ दपापा᳚ क्रामत् । \newline
22. अ॒क्रा॒म॒त् तत् तद॑क्राम दक्राम॒त् तत् । \newline
23. तदिन्द्र॒ इन्द्र॒ स्तत् तदिन्द्रः॑ । \newline
24. इन्द्रो॑ ऽचाय दचाय॒ दिन्द्र॒ इन्द्रो॑ ऽचायत् । \newline
25. अ॒चा॒य॒त् तत् तद॑चाय दचाय॒त् तत् । \newline
26. तदन्वनु॒ तत् तदनु॑ । \newline
27. अन्व पापा न्वन्वप॑ । \newline
28. अपा᳚क्राम दक्राम॒ दपापा᳚ क्रामत् । \newline
29. अ॒क्रा॒म॒त् तत् तद॑क्राम दक्राम॒त् तत् । \newline
30. तद॑व॒रुध॑ मव॒रुध॒म् तत् तद॑व॒रुध᳚म् । \newline
31. अ॒व॒रुध॒म् न नाव॒रुध॑ मव॒रुध॒म् न । \newline
32. अ॒व॒रुध॒मित्य॑व - रुध᳚म् । \newline
33. नाश॑क्नो दशक्नो॒न् न नाश॑क्नोत् । \newline
34. अ॒श॒क्नो॒त् तत् तद॑शक्नो दशक्नो॒त् तत् । \newline
35. तद॑स्मा दस्मा॒त् तत् तद॑स्मात् । \newline
36. अ॒स्मा॒ द॒भ्य॒र्द्धो᳚ ऽभ्य॒र्द्धो᳚ ऽस्मा दस्मा दभ्य॒र्द्धः । \newline
37. अ॒भ्य॒र्द्धो॑ ऽचर दचर दभ्य॒र्द्धो᳚ ऽभ्य॒र्द्धो॑ ऽचरत् । \newline
38. अ॒भ्य॒र्द्ध इत्य॑भि - अ॒र्द्धः । \newline
39. अ॒च॒र॒थ् स सो॑ ऽचर दचर॒थ् सः । \newline
40. स प्र॒जाप॑तिम् प्र॒जाप॑तिꣳ॒॒ स स प्र॒जाप॑तिम् । \newline
41. प्र॒जाप॑ति॒ मुपोप॑ प्र॒जाप॑तिम् प्र॒जाप॑ति॒ मुप॑ । \newline
42. प्र॒जाप॑ति॒मिति॑ प्र॒जा - प॒ति॒म् । \newline
43. उपा॑धाव दधाव॒ दुपोपा॑ धावत् । \newline
44. अ॒धा॒व॒त् तम् त म॑धाव दधाव॒त् तम् । \newline
45. त मे॒त यै॒तया॒ तम् त मे॒तया᳚ । \newline
46. ए॒तया॒ सर्व॑पृष्ठया॒ सर्व॑पृष्ठयै॒त यै॒तया॒ सर्व॑पृष्ठया । \newline
47. सर्व॑पृष्ठया ऽयाजयदयाजय॒थ् सर्व॑पृष्ठया॒ सर्व॑पृष्ठया ऽयाजयत् । \newline
48. सर्व॑पृष्ठ॒येति॒ सर्व॑ - पृ॒ष्ठ॒या॒ । \newline
49. अ॒या॒ज॒य॒त् तया॒ तया॑ ऽयाजय दयाजय॒त् तया᳚ । \newline
50. तयै॒वैव तया॒ तयै॒व । \newline
51. ए॒वास्मि॑न् नस्मिन् ने॒वैवास्मिन्न्॑ । \newline
52. अ॒स्मि॒न् नि॒न्द्रि॒य मि॑न्द्रि॒य म॑स्मिन् नस्मिन् निन्द्रि॒यम् । \newline
53. इ॒न्द्रि॒यं ॅवी॒र्यं॑ ॅवी॒र्य॑ मिन्द्रि॒य मि॑न्द्रि॒यं ॅवी॒र्य᳚म् । \newline
54. वी॒र्य॑ मदधा ददधाद् वी॒र्यं॑ ॅवी॒र्य॑ मदधात् । \newline
55. अ॒द॒धा॒द् यो यो॑ ऽदधा ददधा॒द् यः । \newline
56. य इ॑न्द्रि॒यका॑म इन्द्रि॒यका॑मो॒ यो य इ॑न्द्रि॒यका॑मः । \newline
57. इ॒न्द्रि॒यका॑मो वी॒र्य॑कामो वी॒र्य॑काम इन्द्रि॒यका॑म इन्द्रि॒यका॑मो वी॒र्य॑कामः । \newline
58. इ॒न्द्रि॒यका॑म॒ इती᳚न्द्रि॒य - का॒मः॒ । \newline

\textbf{Ghana Paata } \newline

1. दे॒वा॒सु॒राः संॅय॑त्ताः॒ संॅय॑त्ता देवासु॒रा दे॑वासु॒राः संॅय॑त्ता आसन् नास॒न् थ्संॅय॑त्ता देवासु॒रा दे॑वासु॒राः संॅय॑त्ता आसन्न् । \newline
2. दे॒वा॒सु॒रा इति॑ देव - अ॒सु॒राः । \newline
3. संॅय॑त्ता आसन् नास॒न् थ्संॅय॑त्ताः॒ संॅय॑त्ता आस॒न् ताꣳ स्ता ना॑स॒न् थ्संॅय॑त्ताः॒ संॅय॑त्ता आस॒न् तान् । \newline
4. संॅय॑त्ता॒ इति॒ सं - य॒त्ताः॒ । \newline
5. आ॒स॒न् ताꣳ स्ता ना॑सन् नास॒न् तान् दे॒वान् दे॒वान् ता ना॑सन् नास॒न् तान् दे॒वान् । \newline
6. तान् दे॒वान् दे॒वान् ताꣳ स्तान् दे॒वा नसु॑रा॒ असु॑रा दे॒वान् ताꣳ स्तान् दे॒वा नसु॑राः । \newline
7. दे॒वा नसु॑रा॒ असु॑रा दे॒वान् दे॒वा नसु॑रा अजयन् नजय॒न् नसु॑रा दे॒वान् दे॒वा नसु॑रा अजयन्न् । \newline
8. असु॑रा अजयन् नजय॒न् नसु॑रा॒ असु॑रा अजय॒न् ते ते॑ ऽजय॒न् नसु॑रा॒ असु॑रा अजय॒न् ते । \newline
9. अ॒ज॒य॒न् ते ते॑ ऽजयन् नजय॒न् ते दे॒वा दे॒वा स्ते॑ ऽजयन् नजय॒न् ते दे॒वाः । \newline
10. ते दे॒वा दे॒वा स्ते ते दे॒वाः प॑राजिग्या॒नाः प॑राजिग्या॒ना दे॒वा स्ते ते दे॒वाः प॑राजिग्या॒नाः । \newline
11. दे॒वाः प॑राजिग्या॒नाः प॑राजिग्या॒ना दे॒वा दे॒वाः प॑राजिग्या॒ना असु॑राणा॒ मसु॑राणाम् पराजिग्या॒ना दे॒वा दे॒वाः प॑राजिग्या॒ना असु॑राणाम् । \newline
12. प॒रा॒जि॒ग्या॒ना असु॑राणा॒ मसु॑राणाम् पराजिग्या॒नाः प॑राजिग्या॒ना असु॑राणां॒ ॅवैश्यं॒ ॅवैश्य॒ मसु॑राणाम् पराजिग्या॒नाः प॑राजिग्या॒ना असु॑राणां॒ ॅवैश्य᳚म् । \newline
13. प॒रा॒जि॒ग्या॒ना इति॑ परा - जि॒ग्या॒नाः । \newline
14. असु॑राणां॒ ॅवैश्यं॒ ॅवैश्य॒ मसु॑राणा॒ मसु॑राणां॒ ॅवैश्य॒ मुपोप॒ वैश्य॒ मसु॑राणा॒ मसु॑राणां॒ ॅवैश्य॒ मुप॑ । \newline
15. वैश्य॒ मुपोप॒ वैश्यं॒ ॅवैश्य॒ मुपा॑यन् नाय॒न् नुप॒ वैश्यं॒ ॅवैश्य॒ मुपा॑यन्न् । \newline
16. उपा॑यन् नाय॒न् नुपोपा॑य॒न् तेभ्य॒ स्तेभ्य॑ आय॒न् नुपोपा॑य॒न् तेभ्यः॑ । \newline
17. आ॒य॒न् तेभ्य॒ स्तेभ्य॑ आयन् नाय॒न् तेभ्य॑ इन्द्रि॒य मि॑न्द्रि॒यम् तेभ्य॑ आयन् नाय॒न् तेभ्य॑ इन्द्रि॒यम् । \newline
18. तेभ्य॑ इन्द्रि॒य मि॑न्द्रि॒यम् तेभ्य॒ स्तेभ्य॑ इन्द्रि॒यं ॅवी॒र्यं॑ ॅवी॒र्य॑ मिन्द्रि॒यम् तेभ्य॒ स्तेभ्य॑ इन्द्रि॒यं ॅवी॒र्य᳚म् । \newline
19. इ॒न्द्रि॒यं ॅवी॒र्यं॑ ॅवी॒र्य॑ मिन्द्रि॒य मि॑न्द्रि॒यं ॅवी॒र्य॑ मपाप॑ वी॒र्य॑ मिन्द्रि॒य मि॑न्द्रि॒यं ॅवी॒र्य॑ मप॑ । \newline
20. वी॒र्य॑ मपाप॑ वी॒र्यं॑ ॅवी॒र्य॑ मपा᳚क्राम दक्राम॒दप॑ वी॒र्यं॑ ॅवी॒र्य॑ मपा᳚क्रामत् । \newline
21. अपा᳚क्राम दक्राम॒ दपापा᳚क्राम॒त् तत् तद॑क्राम॒ दपापा᳚क्राम॒त् तत् । \newline
22. अ॒क्रा॒म॒त् तत् तद॑क्राम दक्राम॒त् तदिन्द्र॒ इन्द्र॒ स्तद॑क्राम दक्राम॒त् तदिन्द्रः॑ । \newline
23. तदिन्द्र॒ इन्द्र॒ स्तत् तदिन्द्रो॑ ऽचाय दचाय॒ दिन्द्र॒ स्तत् तदिन्द्रो॑ ऽचायत् । \newline
24. इन्द्रो॑ ऽचाय दचाय॒ दिन्द्र॒ इन्द्रो॑ ऽचाय॒त् तत् तद॑चाय॒ दिन्द्र॒ इन्द्रो॑ ऽचाय॒त् तत् । \newline
25. अ॒चा॒य॒त् तत् तद॑चाय दचाय॒त् तदन्वनु॒ तद॑चाय दचाय॒त् तदनु॑ । \newline
26. तदन्वनु॒ तत् तदन्व पापानु॒ तत् तदन्वप॑ । \newline
27. अन्वपापा न्वन्वपा᳚क्राम दक्राम॒ दपान्वन्व पा᳚क्रामत् । \newline
28. अपा᳚क्राम दक्राम॒ दपापा᳚क्राम॒त् तत् तद॑क्राम॒ दपापा᳚क्राम॒त् तत् । \newline
29. अ॒क्रा॒म॒त् तत् तद॑क्राम दक्राम॒त् तद॑व॒रुध॑ मव॒रुध॒म् तद॑क्राम दक्राम॒त् तद॑व॒रुध᳚म् । \newline
30. तद॑व॒रुध॑ मव॒रुध॒म् तत् तद॑व॒रुध॒म् न नाव॒रुध॒म् तत् तद॑व॒रुध॒म् न । \newline
31. अ॒व॒रुध॒म् न नाव॒रुध॑ मव॒रुध॒म् नाश॑क्नो दशक्नो॒न् नाव॒रुध॑ मव॒रुध॒म् नाश॑क्नोत् । \newline
32. अ॒व॒रुध॒मित्य॑व - रुध᳚म् । \newline
33. नाश॑क्नो दशक्नो॒न् न नाश॑क्नो॒त् तत् तद॑शक्नो॒न् न नाश॑क्नो॒त् तत् । \newline
34. अ॒श॒क्नो॒त् तत् तद॑शक्नो दशक्नो॒त् तद॑स्मा दस्मा॒त् तद॑शक्नो दशक्नो॒त् तद॑स्मात् । \newline
35. तद॑स्मा दस्मा॒त् तत् तद॑स्मा दभ्य॒र्द्धो᳚ ऽभ्य॒र्द्धो᳚ ऽस्मा॒त् तत् तद॑स्मा दभ्य॒र्द्धः । \newline
36. अ॒स्मा॒ द॒भ्य॒र्द्धो᳚ ऽभ्य॒र्द्धो᳚ ऽस्मा दस्मा दभ्य॒र्द्धो॑ ऽचर दचर दभ्य॒र्द्धो᳚ ऽस्मा दस्मा दभ्य॒र्द्धो॑ ऽचरत् । \newline
37. अ॒भ्य॒र्द्धो॑ ऽचर दचर दभ्य॒र्द्धो᳚ ऽभ्य॒र्द्धो॑ ऽचर॒थ् स सो॑ ऽचर दभ्य॒र्द्धो᳚ ऽभ्य॒र्द्धो॑ ऽचर॒थ् सः । \newline
38. अ॒भ्य॒र्द्ध इत्य॑भि - अ॒र्द्धः । \newline
39. अ॒च॒र॒थ् स सो॑ ऽचर दचर॒थ् स प्र॒जाप॑तिम् प्र॒जाप॑तिꣳ॒॒ सो॑ ऽचर दचर॒थ् स प्र॒जाप॑तिम् । \newline
40. स प्र॒जाप॑तिम् प्र॒जाप॑तिꣳ॒॒ स स प्र॒जाप॑ति॒ मुपोप॑ प्र॒जाप॑तिꣳ॒॒ स स प्र॒जाप॑ति॒ मुप॑ । \newline
41. प्र॒जाप॑ति॒ मुपोप॑ प्र॒जाप॑तिम् प्र॒जाप॑ति॒ मुपा॑धाव दधाव॒दुप॑ प्र॒जाप॑तिम् प्र॒जाप॑ति॒ मुपा॑धावत् । \newline
42. प्र॒जाप॑ति॒मिति॑ प्र॒जा - प॒ति॒म् । \newline
43. उपा॑धाव दधाव॒ दुपोपा॑धाव॒त् तम् त म॑धाव॒ दुपोपा॑धाव॒त् तम् । \newline
44. अ॒धा॒व॒त् तम् त म॑धाव दधाव॒त् त मे॒तयै॒तया॒ त म॑धाव दधाव॒त् त मे॒तया᳚ । \newline
45. त मे॒तयै॒तया॒ तम् त मे॒तया॒ सर्व॑पृष्ठया॒ सर्व॑पृष्ठयै॒तया॒ तम् त मे॒तया॒ सर्व॑पृष्ठया । \newline
46. ए॒तया॒ सर्व॑पृष्ठया॒ सर्व॑पृष्ठ यै॒तयै॒तया॒ सर्व॑पृष्ठया ऽयाजय दयाजय॒थ् सर्व॑पृष्ठ यै॒तयै॒तया॒ सर्व॑पृष्ठया ऽयाजयत् । \newline
47. सर्व॑पृष्ठया ऽयाजय दयाजय॒थ् सर्व॑पृष्ठया॒ सर्व॑पृष्ठया ऽयाजय॒त् तया॒ तया॑ ऽयाजय॒थ् सर्व॑पृष्ठया॒ सर्व॑पृष्ठया ऽयाजय॒त् तया᳚ । \newline
48. सर्व॑पृष्ठ॒येति॒ सर्व॑ - पृ॒ष्ठ॒या॒ । \newline
49. अ॒या॒ज॒य॒त् तया॒ तया॑ ऽयाजय दयाजय॒त् तयै॒वैव तया॑ ऽयाजय दयाजय॒त् तयै॒व । \newline
50. तयै॒वैव तया॒ तयै॒वास्मि॑न् नस्मिन् ने॒व तया॒ तयै॒वास्मिन्न्॑ । \newline
51. ए॒वास्मि॑न् नस्मिन् ने॒वैवास्मि॑न् निन्द्रि॒य मि॑न्द्रि॒य म॑स्मिन् ने॒वैवास्मि॑न् निन्द्रि॒यम् । \newline
52. अ॒स्मि॒न् नि॒न्द्रि॒य मि॑न्द्रि॒य म॑स्मिन् नस्मिन् निन्द्रि॒यं ॅवी॒र्यं॑ ॅवी॒र्य॑ मिन्द्रि॒य म॑स्मिन् नस्मिन् निन्द्रि॒यं ॅवी॒र्य᳚म् । \newline
53. इ॒न्द्रि॒यं ॅवी॒र्यं॑ ॅवी॒र्य॑ मिन्द्रि॒य मि॑न्द्रि॒यं ॅवी॒र्य॑ मदधा ददधाद् वी॒र्य॑ मिन्द्रि॒य मि॑न्द्रि॒यं ॅवी॒र्य॑ मदधात् । \newline
54. वी॒र्य॑ मदधा ददधाद् वी॒र्यं॑ ॅवी॒र्य॑ मदधा॒द् यो यो॑ ऽदधाद् वी॒र्यं॑ ॅवी॒र्य॑ मदधा॒द् यः । \newline
55. अ॒द॒धा॒द् यो यो॑ ऽदधा ददधा॒द् य इ॑न्द्रि॒यका॑म इन्द्रि॒यका॑मो॒ यो॑ ऽदधा ददधा॒द् य इ॑न्द्रि॒यका॑मः । \newline
56. य इ॑न्द्रि॒यका॑म इन्द्रि॒यका॑मो॒ यो य इ॑न्द्रि॒यका॑मो वी॒र्य॑कामो वी॒र्य॑काम इन्द्रि॒यका॑मो॒ यो य इ॑न्द्रि॒यका॑मो वी॒र्य॑कामः । \newline
57. इ॒न्द्रि॒यका॑मो वी॒र्य॑कामो वी॒र्य॑काम इन्द्रि॒यका॑म इन्द्रि॒यका॑मो वी॒र्य॑कामः॒ स्याथ् स्याद् वी॒र्य॑काम इन्द्रि॒यका॑म इन्द्रि॒यका॑मो वी॒र्य॑कामः॒ स्यात् । \newline
58. इ॒न्द्रि॒यका॑म॒ इती᳚न्द्रि॒य - का॒मः॒ । \newline
\pagebreak
\markright{ TS 2.3.7.2  \hfill https://www.vedavms.in \hfill}

\section{ TS 2.3.7.2 }

\textbf{TS 2.3.7.2 } \newline
\textbf{Samhita Paata} \newline

वी॒र्य॑कामः॒ स्यात् तमे॒तया॒ सर्व॑पृष्ठया याजयेदे॒ता ए॒व दे॒वताः॒ स्वेन॑ भाग॒धेये॒नोप॑ धावति॒ ता ए॒वास्मि॑न्निन्द्रि॒यं ॅवी॒र्यं॑ दधति॒यदिन्द्रा॑य॒ राथ॑न्तराय नि॒र्वप॑ति॒ यदे॒वाग्ने-स्तेज॒स्तदे॒वाव॑ रुन्धे॒यदिन्द्रा॑य॒ बार्.ह॑ताय॒ यदे॒वेन्द्र॑स्य॒ तेज॒स्तदे॒वाव॑ रुन्धे॒यदिन्द्रा॑य वैरू॒पाय॒ यदे॒व स॑वि॒तुस्तेज॒स्त - [  ] \newline

\textbf{Pada Paata} \newline

वी॒र्य॑काम॒ इति॑ वी॒र्य॑ - का॒मः॒ । स्यात् । तम् । ए॒तया᳚ । सर्व॑पृष्ठ॒येति॒ सर्व॑-पृ॒ष्ठ॒या॒ । या॒ज॒ये॒त् । ए॒ताः । ए॒व । दे॒वताः᳚ । स्वेन॑ । भा॒ग॒धेये॒नेति॑ भाग-धेये॑न । उपेति॑ । धा॒व॒ति॒ । ताः । ए॒व । अ॒स्मि॒न्न् । इ॒न्द्रि॒यम् । वी॒र्य᳚म् । द॒ध॒ति॒ । यत् । इन्द्रा॑य । राथ॑न्तरा॒येति॒ राथ᳚म् - त॒रा॒य॒ । नि॒र्वप॒तीति॑ निः - वप॑ति । यत् । ए॒व । अ॒ग्नेः । तेजः॑ ।   तत् । ए॒व । अवेति॑ । रु॒न्धे॒ । यत् । इन्द्रा॑य । बार्.ह॑ताय । यत् । ए॒व । इन्द्र॑स्य । तेजः॑ । तत् । ए॒व । अवेति॑ ।   रु॒न्धे॒ । यत् । इन्द्रा॑य । वै॒रू॒पाय॑ । यत् । ए॒व । स॒वि॒तुः । तेजः॑ । तत् ।  \newline


\textbf{Krama Paata} \newline

वी॒र्य॑कामः॒ स्यात् । वी॒र्य॑काम॒ इति॑ वी॒र्य॑ - का॒मः॒ । स्यात् तम् । तमे॒तया᳚ । ए॒तया॒ सर्व॑पृष्ठया । सर्व॑पृष्ठया याजयेत् । सर्व॑पृष्ठ॒येति॒ सर्व॑ - पृ॒ष्ठ॒या॒ । या॒ज॒ये॒दे॒ताः । ए॒ता ए॒व । ए॒व दे॒वताः᳚ । दे॒वताः॒ स्वेन॑ । स्वेन॑ भाग॒धेये॑न । भा॒ग॒धेये॒नोप॑ । भा॒ग॒धेये॒नेति॑ भाग - धेये॑न । उप॑ धावति । धा॒व॒ति॒ ताः । ता ए॒व । ए॒वास्मिन्न्॑ । अ॒स्मि॒न्नि॒न्द्रि॒यम् । इ॒न्द्रि॒यं ॅवी॒र्य᳚म् । वी॒र्य॑म् दधति । द॒ध॒ति॒ यत् । यदिन्द्रा॑य । इन्द्रा॑य॒ राथ॑न्तराय । राथ॑न्तराय नि॒र्वप॑ति । राथ॑न्तरा॒येति॒ राथं᳚ - त॒रा॒य॒ । नि॒र्वप॑ति॒ यत् । नि॒र्वप॒तीति॑ निः - वप॑ति । यदे॒व । ए॒वाग्नेः । अ॒ग्नेस्तेजः॑ । तेज॒स्तत् । तदे॒व । ए॒वाव॑ । अव॑ रुन्धे । रु॒न्धे॒ यत् । यदिन्द्रा॑य । इन्द्रा॑य॒ बार्.ह॑ताय । बार्.ह॑ताय॒ यत् । यदे॒व । ए॒वेन्द्र॑स्य । इन्द्र॑स्य॒ तेजः॑ । तेज॒स्तत् । तदे॒व । ए॒वाव॑ । अव॑ रुन्धे । रु॒न्धे॒ यत् । यदिन्द्रा॑य । इन्द्रा॑य वैरू॒पाय॑ । वै॒रू॒पाय॒ यत् । यदे॒व । ए॒व स॑वि॒तुः । स॒वि॒तुस्तेजः॑ । तेज॒स्तत् । तदे॒व \newline

\textbf{Jatai Paata} \newline

1. वी॒र्य॑कामः॒ स्याथ् स्याद् वी॒र्य॑कामो वी॒र्य॑कामः॒ स्यात् । \newline
2. वी॒र्य॑काम॒ इति॑ वी॒र्य॑ - का॒मः॒ । \newline
3. स्यात् तम् तꣳ स्याथ् स्यात् तम् । \newline
4. त मे॒तयै॒तया॒ तम् त मे॒तया᳚ । \newline
5. ए॒तया॒ सर्व॑पृष्ठया॒ सर्व॑पृष्ठ यै॒तयै॒तया॒ सर्व॑पृष्ठया । \newline
6. सर्व॑पृष्ठया याजयेद् याजये॒थ् सर्व॑पृष्ठया॒ सर्व॑पृष्ठया याजयेत् । \newline
7. सर्व॑पृष्ठ॒येति॒ सर्व॑ - पृ॒ष्ठ॒या॒ । \newline
8. या॒ज॒ये॒ दे॒ता ए॒ता या॑जयेद् याजये दे॒ताः । \newline
9. ए॒ता ए॒वैवैता ए॒ता ए॒व । \newline
10. ए॒व दे॒वता॑ दे॒वता॑ ए॒वैव दे॒वताः᳚ । \newline
11. दे॒वताः॒ स्वेन॒ स्वेन॑ दे॒वता॑ दे॒वताः॒ स्वेन॑ । \newline
12. स्वेन॑ भाग॒धेये॑न भाग॒धेये॑न॒ स्वेन॒ स्वेन॑ भाग॒धेये॑न । \newline
13. भा॒ग॒धेये॒नोपोप॑ भाग॒धेये॑न भाग॒धेये॒नोप॑ । \newline
14. भा॒ग॒धेये॒नेति॑ भाग - धेये॑न । \newline
15. उप॑ धावति धाव॒ त्युपोप॑ धावति । \newline
16. धा॒व॒ति॒ तास्ता धा॑वति धावति॒ ताः । \newline
17. ता ए॒वैव ता स्ता ए॒व । \newline
18. ए॒वास्मि॑न् नस्मिन् ने॒वैवास्मिन्न्॑ । \newline
19. अ॒स्मि॒न् नि॒न्द्रि॒य मि॑न्द्रि॒य म॑स्मिन् नस्मिन् निन्द्रि॒यम् । \newline
20. इ॒न्द्रि॒यं ॅवी॒र्यं॑ ॅवी॒र्य॑ मिन्द्रि॒य मि॑न्द्रि॒यं ॅवी॒र्य᳚म् । \newline
21. वी॒र्य॑म् दधति दधति वी॒र्यं॑ ॅवी॒र्य॑म् दधति । \newline
22. द॒ध॒ति॒ यद् यद् द॑धति दधति॒ यत् । \newline
23. यदिन्द्रा॒ये न्द्रा॑य॒ यद् यदिन्द्रा॑य । \newline
24. इन्द्रा॑य॒ राथ॑न्तराय॒ राथ॑न्तरा॒ये न्द्रा॒ये न्द्रा॑य॒ राथ॑न्तराय । \newline
25. राथ॑न्तराय नि॒र्वप॑ति नि॒र्वप॑ति॒ राथ॑न्तराय॒ राथ॑न्तराय नि॒र्वप॑ति । \newline
26. राथ॑न्तरा॒येति॒ राथ᳚म् - त॒रा॒य॒ । \newline
27. नि॒र्वप॑ति॒ यद् यन् नि॒र्वप॑ति नि॒र्वप॑ति॒ यत् । \newline
28. नि॒र्वप॒तीति॑ निः - वप॑ति । \newline
29. यदे॒वैव यद् यदे॒व । \newline
30. ए॒वाग्ने र॒ग्ने रे॒वैवाग्नेः । \newline
31. अ॒ग्ने स्तेज॒ स्तेजो॒ ऽग्ने र॒ग्ने स्तेजः॑ । \newline
32. तेज॒ स्तत् तत् तेज॒ स्तेज॒ स्तत् । \newline
33. तदे॒वैव तत् तदे॒व । \newline
34. ए॒वावा वै॒वैवाव॑ । \newline
35. अव॑ रुन्धे रु॒न्धे ऽवाव॑ रुन्धे । \newline
36. रु॒न्धे॒ यद् यद् रु॑न्धे रुन्धे॒ यत् । \newline
37. यदिन्द्रा॒ये न्द्रा॑य॒ यद् यदिन्द्रा॑य । \newline
38. इन्द्रा॑य॒ बार्.ह॑ताय॒ बार्.ह॑ता॒ये न्द्रा॒ये न्द्रा॑य॒ बार्.ह॑ताय । \newline
39. बार्.ह॑ताय॒ यद् यद् बार्.ह॑ताय॒ बार्.ह॑ताय॒ यत् । \newline
40. यदे॒वैव यद् यदे॒व । \newline
41. ए॒वे न्द्र॒स्ये न्द्र॑स्यै॒वैवे न्द्र॑स्य । \newline
42. इन्द्र॑स्य॒ तेज॒ स्तेज॒ इन्द्र॒स्ये न्द्र॑स्य॒ तेजः॑ । \newline
43. तेज॒ स्तत् तत् तेज॒ स्तेज॒ स्तत् । \newline
44. तदे॒वैव तत् तदे॒व । \newline
45. ए॒वावा वै॒वैवाव॑ । \newline
46. अव॑ रुन्धे रु॒न्धे ऽवाव॑ रुन्धे । \newline
47. रु॒न्धे॒ यद् यद् रु॑न्धे रुन्धे॒ यत् । \newline
48. यदिन्द्रा॒ये न्द्रा॑य॒ यद् यदिन्द्रा॑य । \newline
49. इन्द्रा॑य वैरू॒पाय॑ वैरू॒पाये न्द्रा॒ये न्द्रा॑य वैरू॒पाय॑ । \newline
50. वै॒रू॒पाय॒ यद् यद् वै॑रू॒पाय॑ वैरू॒पाय॒ यत् । \newline
51. यदे॒वैव यद् यदे॒व । \newline
52. ए॒व स॑वि॒तुः स॑वि॒तु रे॒वैव स॑वि॒तुः । \newline
53. स॒वि॒तु स्तेज॒ स्तेजः॑ सवि॒तुः स॑वि॒तु स्तेजः॑ । \newline
54. तेज॒ स्तत् तत् तेज॒ स्तेज॒ स्तत् । \newline
55. तदे॒वैव तत् तदे॒व । \newline

\textbf{Ghana Paata } \newline

1. वी॒र्य॑कामः॒ स्याथ् स्याद् वी॒र्य॑कामो वी॒र्य॑कामः॒ स्यात् तम् तꣳ स्याद् वी॒र्य॑कामो वी॒र्य॑कामः॒ स्यात् तम् । \newline
2. वी॒र्य॑काम॒ इति॑ वी॒र्य॑ - का॒मः॒ । \newline
3. स्यात् तम् तꣳ स्याथ् स्यात् त मे॒तयै॒तया॒ तꣳ स्याथ् स्यात् त मे॒तया᳚ । \newline
4. त मे॒तयै॒तया॒ तम् त मे॒तया॒ सर्व॑पृष्ठया॒ सर्व॑पृष्ठ यै॒तया॒ तम् त मे॒तया॒ सर्व॑पृष्ठया । \newline
5. ए॒तया॒ सर्व॑पृष्ठया॒ सर्व॑पृष्ठ यै॒तयै॒तया॒ सर्व॑पृष्ठया याजयेद् याजये॒थ् सर्व॑पृष्ठ यै॒तयै॒तया॒ सर्व॑पृष्ठया याजयेत् । \newline
6. सर्व॑पृष्ठया याजयेद् याजये॒थ् सर्व॑पृष्ठया॒ सर्व॑पृष्ठया याजये दे॒ता ए॒ता या॑जये॒थ् सर्व॑पृष्ठया॒ सर्व॑पृष्ठया याजये दे॒ताः । \newline
7. सर्व॑पृष्ठ॒येति॒ सर्व॑ - पृ॒ष्ठ॒या॒ । \newline
8. या॒ज॒ये॒ दे॒ता ए॒ता या॑जयेद् याजये दे॒ता ए॒वैवैता या॑जयेद् याजये दे॒ता ए॒व । \newline
9. ए॒ता ए॒वैवैता ए॒ता ए॒व दे॒वता॑ दे॒वता॑ ए॒वैता ए॒ता ए॒व दे॒वताः᳚ । \newline
10. ए॒व दे॒वता॑ दे॒वता॑ ए॒वैव दे॒वताः॒ स्वेन॒ स्वेन॑ दे॒वता॑ ए॒वैव दे॒वताः॒ स्वेन॑ । \newline
11. दे॒वताः॒ स्वेन॒ स्वेन॑ दे॒वता॑ दे॒वताः॒ स्वेन॑ भाग॒धेये॑न भाग॒धेये॑न॒ स्वेन॑ दे॒वता॑ दे॒वताः॒ स्वेन॑ भाग॒धेये॑न । \newline
12. स्वेन॑ भाग॒धेये॑न भाग॒धेये॑न॒ स्वेन॒ स्वेन॑ भाग॒धेये॒नोपोप॑ भाग॒धेये॑न॒ स्वेन॒ स्वेन॑ भाग॒धेये॒नोप॑ । \newline
13. भा॒ग॒धेये॒नोपोप॑ भाग॒धेये॑न भाग॒धेये॒नोप॑ धावति धाव॒त्युप॑ भाग॒धेये॑न भाग॒धेये॒नोप॑ धावति । \newline
14. भा॒ग॒धेये॒नेति॑ भाग - धेये॑न । \newline
15. उप॑ धावति धाव॒ त्युपोप॑ धावति॒ ता स्ता धा॑व॒ त्युपोप॑ धावति॒ ताः । \newline
16. धा॒व॒ति॒ ता स्ता धा॑वति धावति॒ ता ए॒वैव ता धा॑वति धावति॒ ता ए॒व । \newline
17. ता ए॒वैव ता स्ता ए॒वास्मि॑न् नस्मिन् ने॒व ता स्ता ए॒वास्मिन्न्॑ । \newline
18. ए॒वास्मि॑न् नस्मिन् ने॒वैवास्मि॑न् निन्द्रि॒य मि॑न्द्रि॒य म॑स्मिन् ने॒वैवास्मि॑न् निन्द्रि॒यम् । \newline
19. अ॒स्मि॒न् नि॒न्द्रि॒य मि॑न्द्रि॒य म॑स्मिन् नस्मिन् निन्द्रि॒यं ॅवी॒र्यं॑ ॅवी॒र्य॑ मिन्द्रि॒य म॑स्मिन् नस्मिन् निन्द्रि॒यं ॅवी॒र्य᳚म् । \newline
20. इ॒न्द्रि॒यं ॅवी॒र्यं॑ ॅवी॒र्य॑ मिन्द्रि॒य मि॑न्द्रि॒यं ॅवी॒र्य॑म् दधति दधति वी॒र्य॑ मिन्द्रि॒य मि॑न्द्रि॒यं ॅवी॒र्य॑म् दधति । \newline
21. वी॒र्य॑म् दधति दधति वी॒र्यं॑ ॅवी॒र्य॑म् दधति॒ यद् यद् द॑धति वी॒र्यं॑ ॅवी॒र्य॑म् दधति॒ यत् । \newline
22. द॒ध॒ति॒ यद् यद् द॑धति दधति॒ यदिन्द्रा॒ये न्द्रा॑य॒ यद् द॑धति दधति॒ यदिन्द्रा॑य । \newline
23. यदिन्द्रा॒ये न्द्रा॑य॒ यद् यदिन्द्रा॑य॒ राथ॑न्तराय॒ राथ॑न्तरा॒ये न्द्रा॑य॒ यद् यदिन्द्रा॑य॒ राथ॑न्तराय । \newline
24. इन्द्रा॑य॒ राथ॑न्तराय॒ राथ॑न्तरा॒ये न्द्रा॒ये न्द्रा॑य॒ राथ॑न्तराय नि॒र्वप॑ति नि॒र्वप॑ति॒ राथ॑न्तरा॒ये न्द्रा॒ये न्द्रा॑य॒ राथ॑न्तराय नि॒र्वप॑ति । \newline
25. राथ॑न्तराय नि॒र्वप॑ति नि॒र्वप॑ति॒ राथ॑न्तराय॒ राथ॑न्तराय नि॒र्वप॑ति॒ यद् यन् नि॒र्वप॑ति॒ राथ॑न्तराय॒ राथ॑न्तराय नि॒र्वप॑ति॒ यत् । \newline
26. राथ॑न्तरा॒येति॒ राथ᳚म् - त॒रा॒य॒ । \newline
27. नि॒र्वप॑ति॒ यद् यन् नि॒र्वप॑ति नि॒र्वप॑ति॒ यदे॒वैव यन् नि॒र्वप॑ति नि॒र्वप॑ति॒ यदे॒व । \newline
28. नि॒र्वप॒तीति॑ निः - वप॑ति । \newline
29. यदे॒वैव यद् यदे॒वाग्ने र॒ग्ने रे॒व यद् यदे॒वाग्नेः । \newline
30. ए॒वाग्ने र॒ग्ने रे॒वैवाग्ने स्तेज॒ स्तेजो॒ ऽग्ने रे॒वैवाग्ने स्तेजः॑ । \newline
31. अ॒ग्ने स्तेज॒ स्तेजो॒ ऽग्ने र॒ग्ने स्तेज॒ स्तत् तत् तेजो॒ ऽग्ने र॒ग्ने स्तेज॒ स्तत् । \newline
32. तेज॒ स्तत् तत् तेज॒ स्तेज॒ स्त दे॒वैव तत् तेज॒ स्तेज॒ स्तदे॒व । \newline
33. तदे॒वैव तत् तदे॒ वावावै॒व तत् तदे॒वाव॑ । \newline
34. ए॒वावा वै॒वैवाव॑ रुन्धे रु॒न्धे ऽवै॒वैवाव॑ रुन्धे । \newline
35. अव॑ रुन्धे रु॒न्धे ऽवाव॑ रुन्धे॒ यद् यद् रु॒न्धे ऽवाव॑ रुन्धे॒ यत् । \newline
36. रु॒न्धे॒ यद् यद् रु॑न्धे रुन्धे॒ यदिन्द्रा॒ये न्द्रा॑य॒ यद् रु॑न्धे रुन्धे॒ यदिन्द्रा॑य । \newline
37. यदिन्द्रा॒ये न्द्रा॑य॒ यद् यदिन्द्रा॑य॒ बार्.ह॑ताय॒ बार्.ह॑ता॒ये न्द्रा॑य॒ यद् यदिन्द्रा॑य॒ बार्.ह॑ताय । \newline
38. इन्द्रा॑य॒ बार्.ह॑ताय॒ बार्.ह॑ता॒ये न्द्रा॒ये न्द्रा॑य॒ बार्.ह॑ताय॒ यद् यद् बार्.ह॑ता॒ये न्द्रा॒ये न्द्रा॑य॒ बार्.ह॑ताय॒ यत् । \newline
39. बार्.ह॑ताय॒ यद् यद् बार्.ह॑ताय॒ बार्.ह॑ताय॒ यदे॒वैव यद् बार्.ह॑ताय॒ बार्.ह॑ताय॒ यदे॒व । \newline
40. यदे॒वैव यद् यदे॒वे न्द्र॒स्ये न्द्र॑स्यै॒व यद् यदे॒वे न्द्र॑स्य । \newline
41. ए॒वे न्द्र॒स्ये न्द्र॑स्यै॒वैवे न्द्र॑स्य॒ तेज॒ स्तेज॒ इन्द्र॑स्यै॒वैवे न्द्र॑स्य॒ तेजः॑ । \newline
42. इन्द्र॑स्य॒ तेज॒ स्तेज॒ इन्द्र॒स्ये न्द्र॑स्य॒ तेज॒ स्तत् तत् तेज॒ इन्द्र॒स्ये न्द्र॑स्य॒ तेज॒ स्तत् । \newline
43. तेज॒ स्तत् तत् तेज॒ स्तेज॒ स्त दे॒वैव तत् तेज॒ स्तेज॒ स्तदे॒व । \newline
44. तदे॒वैव तत् तदे॒वा वावै॒व तत् तदे॒वाव॑ । \newline
45. ए॒वावा वै॒वैवाव॑ रुन्धे रु॒न्धे ऽवै॒वैवाव॑ रुन्धे । \newline
46. अव॑ रुन्धे रु॒न्धे ऽवाव॑ रुन्धे॒ यद् यद् रु॒न्धे ऽवाव॑ रुन्धे॒ यत् । \newline
47. रु॒न्धे॒ यद् यद् रु॑न्धे रुन्धे॒ यदिन्द्रा॒ये न्द्रा॑य॒ यद् रु॑न्धे रुन्धे॒ यदिन्द्रा॑य । \newline
48. यदिन्द्रा॒ये न्द्रा॑य॒ यद् यदिन्द्रा॑य वैरू॒पाय॑ वैरू॒पाये न्द्रा॑य॒ यद् यदिन्द्रा॑य वैरू॒पाय॑ । \newline
49. इन्द्रा॑य वैरू॒पाय॑ वैरू॒पाये न्द्रा॒ये न्द्रा॑य वैरू॒पाय॒ यद् यद् वै॑रू॒पाये न्द्रा॒ये न्द्रा॑य वैरू॒पाय॒ यत् । \newline
50. वै॒रू॒पाय॒ यद् यद् वै॑रू॒पाय॑ वैरू॒पाय॒ यदे॒वैव यद् वै॑रू॒पाय॑ वैरू॒पाय॒ यदे॒व । \newline
51. यदे॒वैव यद् यदे॒व स॑वि॒तुः स॑वि॒तुरे॒व यद् यदे॒व स॑वि॒तुः । \newline
52. ए॒व स॑वि॒तुः स॑वि॒तु रे॒वैव स॑वि॒तु स्तेज॒ स्तेजः॑ सवि॒तु रे॒वैव स॑वि॒तु स्तेजः॑ । \newline
53. स॒वि॒तु स्तेज॒ स्तेजः॑ सवि॒तुः स॑वि॒तु स्तेज॒ स्तत् तत् तेजः॑ सवि॒तुः स॑वि॒तु स्तेज॒ स्तत् । \newline
54. तेज॒ स्तत् तत् तेज॒ स्तेज॒ स्तदे॒वैव तत् तेज॒ स्तेज॒ स्तदे॒व । \newline
55. तदे॒वैव तत् तदे॒वा वावै॒व तत् तदे॒वाव॑ । \newline
\pagebreak
\markright{ TS 2.3.7.3  \hfill https://www.vedavms.in \hfill}

\section{ TS 2.3.7.3 }

\textbf{TS 2.3.7.3 } \newline
\textbf{Samhita Paata} \newline

दे॒वाव॑ रुन्धे॒ यदिन्द्रा॑य वैरा॒जाय॒ यदे॒व धा॒तुस्तेज॒स्तदे॒वाव॑ रुन्धे॒ यदिन्द्रा॑य शाक्व॒राय॒ यदे॒व म॒रुतां॒ तेज॒स्तदे॒वाव॑ रुन्धे॒यदिन्द्रा॑य रैव॒ताय॒ यदे॒व बृह॒स्पते॒स्तेज॒स्तदे॒वाव॑ रुन्ध ए॒ताव॑न्ति॒ वै तेजाꣳ॑सि॒ तान्ये॒वाव॑ रुन्ध उत्ता॒नेषु॑ क॒पाले॒ष्वधि॑ श्रय॒त्यया॑तयामत्वाय॒ द्वाद॑शकपालः पुरो॒डाशो॑ - [  ] \newline

\textbf{Pada Paata} \newline

ए॒व । अवेति॑ । रु॒न्धे॒ । यत् । इन्द्रा॑य । वै॒रा॒जाय॑ । यत् । ए॒व । धा॒तुः । तेजः॑ । तत् । ए॒व । अवेति॑ । रु॒न्धे॒ । यत् । इन्द्रा॑य । शा॒क्व॒राय॑ । यत् । ए॒व । म॒रुता᳚म् । तेजः॑ । तत् । ए॒व । अवेति॑ । रु॒न्धे॒ । यत् । इन्द्रा॑य । रै॒व॒ताय॑ । यत् । ए॒व । बृह॒स्पतेः᳚ । तेजः॑ । तत् । ए॒व । अवेति॑ । रु॒न्धे॒ । ए॒ताव॑न्ति । वै । तेजाꣳ॑सि । तानि॑ । ए॒व । अवेति॑ । रु॒न्धे॒ । उ॒त्ता॒नेष्वित्यु॑त् - ता॒नेषु॑ । क॒पाले॑षु । अधीति॑ । श्र॒य॒ति॒ । अया॑तयामत्वा॒येत्यया॑तयाम - त्वा॒य॒ । द्वाद॑शकपाल॒ इति॒ द्वाद॑श - क॒पा॒लः॒ । पु॒रो॒डाशः॑ ।  \newline


\textbf{Krama Paata} \newline

ए॒वाव॑ । अव॑ रुन्धे । रु॒न्धे॒ यत् । यदिन्द्रा॑य । इन्द्रा॑य वैरा॒जाय॑ । वै॒रा॒जाय॒ यत् । यदे॒व । ए॒व धा॒तुः । धा॒तु स्तेजः॑ । तेज॒स्तत् । तदे॒व । ए॒वाव॑ । अव॑ रुन्धे । रु॒न्धे॒ यत् । यदिन्द्रा॑य । इन्द्रा॑य शाक्व॒राय॑ । शा॒क्व॒राय॒ यत् । यदे॒व । ए॒व म॒रुता᳚म् । म॒रुता॒म् तेजः॑ । तेज॒ स्तत् । तदे॒व । ए॒वाव॑ । अव॑ रुन्धे । रु॒न्धे॒ यत् । यदिन्द्रा॑य । इन्द्रा॑य रैव॒ताय॑ । रै॒व॒ताय॒ यत् । यदे॒व । ए॒व बृह॒स्पतेः᳚ । बृह॒स्पते॒स्तेजः॑ । तेज॒स्तत् । तदे॒व । ए॒वाव॑ । अव॑ रुन्धे । रु॒न्ध॒ ए॒ताव॑न्ति । ए॒ताव॑न्ति॒ वै । वै तेजाꣳ॑सि । तेजाꣳ॑सि॒ तानि॑ । तान्ये॒व । ए॒वाव॑ । अव॑ रुन्धे । रु॒न्ध॒ उ॒त्ता॒नेषु॑ । उ॒त्ता॒नेषु॑ क॒पाले॑षु । उ॒त्ता॒नेष्वित्यु॑त् - ता॒नेषु॑ । क॒पाले॒ष्वधि॑ । अधि॑ श्रयति । श्र॒य॒त्यया॑तयामत्वाय । अया॑तयामत्वाय॒ द्वाद॑शकपालः । अया॑तयामत्वा॒येत्यया॑तयाम - त्वा॒य॒ । द्वाद॑शकपालः पु॒रोडाशः॑ ( ) । द्वाद॑शकपाल॒ इति॒ द्वाद॑श - क॒पा॒लः॒ । पु॒रो॒डाशो॑ भवति \newline
भ॒व॒ति॒ वै॒श्व॒दे॒व॒त्वाय॑ । वै॒श्व॒दे॒व॒त्वाय॑ सम॒न्तम् । वै॒श्व॒दे॒व॒त्वायेति॑ वैश्वदेव - त्वाय॑ । स॒म॒न्तम् प॒र्यव॑द्यति । स॒म॒न्तमिति॑ सं - अ॒न्तम् । प॒र्यव॑द्यति सम॒न्तम् । प॒र्यव॑द्य॒तीति॑ परि - अव॑द्यति । स॒म॒न्तमे॒व । स॒म॒न्तमिति॑ सं - अ॒न्तम् । ए॒वेन्द्रि॒यम् । इ॒न्द्रि॒यं ॅवी॒र्य᳚म् । वी॒र्यं॑ ॅयज॑माने । यज॑माने दधाति । द॒धा॒ति॒ व्य॒त्यास᳚म् । व्य॒त्यास॒मनु॑ । व्य॒त्यास॒मिति॑ वि - अ॒त्यास᳚म् । अन्वा॑ह । आ॒हानि॑र्दाहाय । अनि॑र्दाहा॒याश्वः॑ । अनि॑र्दाहा॒येत्यनिः॑ - दा॒हा॒य॒ । अश्व॑ ऋष॒भः । ऋ॒ष॒भो वृ॒ष्णिः । वृ॒ष्णिर् ब॒स्तः । ब॒स्तः सा । सा दक्षि॑णा । दक्षि॑णा वृष॒त्वाय॑ । वृ॒ष॒त्वायै॒तया᳚ । वृ॒ष॒त्वायेति॑ वृष - त्वाय॑ । ए॒तयै॒व । ए॒व य॑जेत । य॒जे॒ता॒भि॒श॒स्यमा॑नः । अ॒भि॒श॒स्यमा॑न ए॒ताः । अ॒भि॒श॒स्यमा॑न॒ इत्य॑भि - श॒स्यमा॑नः । ए॒ताश्च॑ । चेत् । इद् वै । वा अ॑स्य । अ॒स्य॒ दे॒वताः᳚ । दे॒वता॒ अन्न᳚म् । अन्न॑म॒दन्ति॑ । अ॒दन्त्य॒दन्ति॑ । अ॒दन्त्यु॑ । उ॒वे॒व । ए॒वास्य॑ । अ॒स्य॒ म॒नु॒ष्याः᳚ । म॒नु॒ष्या॑ इति॑ मनु॒ष्याः᳚ । \newline

\textbf{Jatai Paata} \newline

1. ए॒वावा वै॒वैवाव॑ । \newline
2. अव॑ रुन्धे रु॒न्धे ऽवाव॑ रुन्धे । \newline
3. रु॒न्धे॒ यद् यद् रु॑न्धे रुन्धे॒ यत् । \newline
4. यदिन्द्रा॒ये न्द्रा॑य॒ यद् यदिन्द्रा॑य । \newline
5. इन्द्रा॑य वैरा॒जाय॑ वैरा॒जाये न्द्रा॒ये न्द्रा॑य वैरा॒जाय॑ । \newline
6. वै॒रा॒जाय॒ यद् यद् वै॑रा॒जाय॑ वैरा॒जाय॒ यत् । \newline
7. यदे॒वैव यद् यदे॒व । \newline
8. ए॒व धा॒तुर् धा॒तु रे॒वैव धा॒तुः । \newline
9. धा॒तु स्तेज॒ स्तेजो॑ धा॒तुर् धा॒तु स्तेजः॑ । \newline
10. तेज॒ स्तत् तत् तेज॒ स्तेज॒ स्तत् । \newline
11. तदे॒वैव तत् तदे॒व । \newline
12. ए॒वावा वै॒वैवाव॑ । \newline
13. अव॑ रुन्धे रु॒न्धे ऽवाव॑ रुन्धे । \newline
14. रु॒न्धे॒ यद् यद् रु॑न्धे रुन्धे॒ यत् । \newline
15. यदिन्द्रा॒ये न्द्रा॑य॒ यद् यदिन्द्रा॑य । \newline
16. इन्द्रा॑य शाक्व॒राय॑ शाक्व॒राये न्द्रा॒ये न्द्रा॑य शाक्व॒राय॑ । \newline
17. शा॒क्व॒राय॒ यद् यच् छा᳚क्व॒राय॑ शाक्व॒राय॒ यत् । \newline
18. यदे॒वैव यद् यदे॒व । \newline
19. ए॒व म॒रुता᳚म् म॒रुता॑ मे॒वैव म॒रुता᳚म् । \newline
20. म॒रुता॒म् तेज॒ स्तेजो॑ म॒रुता᳚म् म॒रुता॒म् तेजः॑ । \newline
21. तेज॒ स्तत् तत् तेज॒ स्तेज॒ स्तत् । \newline
22. तदे॒वैव तत् तदे॒व । \newline
23. ए॒वावा वै॒वैवाव॑ । \newline
24. अव॑ रुन्धे रु॒न्धे ऽवाव॑ रुन्धे । \newline
25. रु॒न्धे॒ यद् यद् रु॑न्धे रुन्धे॒ यत् । \newline
26. यदिन्द्रा॒ये न्द्रा॑य॒ यद् यदिन्द्रा॑य । \newline
27. इन्द्रा॑य रैव॒ताय॑ रैव॒ताये न्द्रा॒ये न्द्रा॑य रैव॒ताय॑ । \newline
28. रै॒व॒ताय॒ यद् यद् रै॑व॒ताय॑ रैव॒ताय॒ यत् । \newline
29. यदे॒वैव यद् यदे॒व । \newline
30. ए॒व बृह॒स्पते॒र् बृह॒स्पते॑ रे॒वैव बृह॒स्पतेः᳚ । \newline
31. बृह॒स्पते॒ स्तेज॒ स्तेजो॒ बृह॒स्पते॒र् बृह॒स्पते॒ स्तेजः॑ । \newline
32. तेज॒ स्तत् तत् तेज॒ स्तेज॒ स्तत् । \newline
33. तदे॒वैव तत् तदे॒व । \newline
34. ए॒वावा वै॒वैवाव॑ । \newline
35. अव॑ रुन्धे रु॒न्धे ऽवाव॑ रुन्धे । \newline
36. रु॒न्ध॒ ए॒ताव॑ न्त्ये॒ताव॑न्ति रुन्धे रुन्ध ए॒ताव॑न्ति । \newline
37. ए॒ताव॑न्ति॒ वै वा ए॒ताव॑ न्त्ये॒ताव॑न्ति॒ वै । \newline
38. वै तेजाꣳ॑सि॒ तेजाꣳ॑सि॒ वै वै तेजाꣳ॑सि । \newline
39. तेजाꣳ॑सि॒ तानि॒ तानि॒ तेजाꣳ॑सि॒ तेजाꣳ॑सि॒ तानि॑ । \newline
40. तान्ये॒वैव तानि॒ तान्ये॒व । \newline
41. ए॒वावा वै॒वैवाव॑ । \newline
42. अव॑ रुन्धे रु॒न्धे ऽवाव॑ रुन्धे । \newline
43. रु॒न्ध॒ उ॒त्ता॒नेषू᳚ त्ता॒नेषु॑ रुन्धे रुन्ध उत्ता॒नेषु॑ । \newline
44. उ॒त्ता॒नेषु॑ क॒पाले॑षु क॒पाले॑षू त्ता॒नेषू᳚ त्ता॒नेषु॑ क॒पाले॑षु । \newline
45. उ॒त्ता॒नेष्वित्यु॑त् - ता॒नेषु॑ । \newline
46. क॒पाले॒ ष्वध्यधि॑ क॒पाले॑षु क॒पाले॒ ष्वधि॑ । \newline
47. अधि॑ श्रयति श्रय॒ त्यध्यधि॑ श्रयति । \newline
48. श्र॒य॒ त्यया॑तयामत्वा॒या या॑तयामत्वाय श्रयति श्रय॒ त्यया॑तयामत्वाय । \newline
49. अया॑तयामत्वाय॒ द्वाद॑शकपालो॒ द्वाद॑शकपा॒लो ऽया॑तयामत्वा॒या या॑तयामत्वाय॒ द्वाद॑शकपालः । \newline
50. अया॑तयामत्वा॒येत्यया॑तयाम - त्वा॒य॒ । \newline
51. द्वाद॑शकपालः पुरो॒डाशः॑ पुरो॒डाशो॒ द्वाद॑शकपालो॒ द्वाद॑शकपालः पुरो॒डाशः॑ । \newline
52. द्वाद॑शकपाल॒ इति॒ द्वाद॑श - क॒पा॒लः॒ । \newline
53. पु॒रो॒डाशो॑ भवति भवति पुरो॒डाशः॑ पुरो॒डाशो॑ भवति । \newline

\textbf{Ghana Paata } \newline

1. ए॒वावा वै॒वैवाव॑ रुन्धे रु॒न्धे ऽवै॒वैवाव॑ रुन्धे । \newline
2. अव॑ रुन्धे रु॒न्धे ऽवाव॑ रुन्धे॒ यद् यद् रु॒न्धे ऽवाव॑ रुन्धे॒ यत् । \newline
3. रु॒न्धे॒ यद् यद् रु॑न्धे रुन्धे॒ यदिन्द्रा॒ये न्द्रा॑य॒ यद् रु॑न्धे रुन्धे॒ यदिन्द्रा॑य । \newline
4. यदिन्द्रा॒ये न्द्रा॑य॒ यद् यदिन्द्रा॑य वैरा॒जाय॑ वैरा॒जाये न्द्रा॑य॒ यद् यदिन्द्रा॑य वैरा॒जाय॑ । \newline
5. इन्द्रा॑य वैरा॒जाय॑ वैरा॒जाये न्द्रा॒ये न्द्रा॑य वैरा॒जाय॒ यद् यद् वै॑रा॒जाये न्द्रा॒ये न्द्रा॑य वैरा॒जाय॒ यत् । \newline
6. वै॒रा॒जाय॒ यद् यद् वै॑रा॒जाय॑ वैरा॒जाय॒ यदे॒वैव यद् वै॑रा॒जाय॑ वैरा॒जाय॒ यदे॒व । \newline
7. यदे॒वैव यद् यदे॒व धा॒तुर् धा॒तु रे॒व यद् यदे॒व धा॒तुः । \newline
8. ए॒व धा॒तुर् धा॒तु रे॒वैव धा॒तु स्तेज॒ स्तेजो॑ धा॒तु रे॒वैव धा॒तु स्तेजः॑ । \newline
9. धा॒तु स्तेज॒ स्तेजो॑ धा॒तुर् धा॒तु स्तेज॒ स्तत् तत् तेजो॑ धा॒तुर् धा॒तु स्तेज॒ स्तत् । \newline
10. तेज॒ स्तत् तत् तेज॒ स्तेज॒ स्तदे॒वैव तत् तेज॒ स्तेज॒ स्तदे॒व । \newline
11. तदे॒वैव तत् तदे॒वा वावै॒व तत् तदे॒वाव॑ । \newline
12. ए॒वावा वै॒वैवाव॑ रुन्धे रु॒न्धे ऽवै॒वैवाव॑ रुन्धे । \newline
13. अव॑ रुन्धे रु॒न्धे ऽवाव॑ रुन्धे॒ यद् यद् रु॒न्धे ऽवाव॑ रुन्धे॒ यत् । \newline
14. रु॒न्धे॒ यद् यद् रु॑न्धे रुन्धे॒ यदिन्द्रा॒ये न्द्रा॑य॒ यद् रु॑न्धे रुन्धे॒ यदिन्द्रा॑य । \newline
15. यदिन्द्रा॒ये न्द्रा॑य॒ यद् यदिन्द्रा॑य शाक्व॒राय॑ शाक्व॒राये न्द्रा॑य॒ यद् यदिन्द्रा॑य शाक्व॒राय॑ । \newline
16. इन्द्रा॑य शाक्व॒राय॑ शाक्व॒राये न्द्रा॒ये न्द्रा॑य शाक्व॒राय॒ यद् यच्छा᳚क्व॒राये न्द्रा॒ये न्द्रा॑य शाक्व॒राय॒ यत् । \newline
17. शा॒क्व॒राय॒ यद् यच्छा᳚क्व॒राय॑ शाक्व॒राय॒ यदे॒वैव यच्छा᳚क्व॒राय॑ शाक्व॒राय॒ यदे॒व । \newline
18. यदे॒वैव यद् यदे॒व म॒रुता᳚म् म॒रुता॑ मे॒व यद् यदे॒व म॒रुता᳚म् । \newline
19. ए॒व म॒रुता᳚म् म॒रुता॑ मे॒वैव म॒रुता॒म् तेज॒ स्तेजो॑ म॒रुता॑ मे॒वैव म॒रुता॒म् तेजः॑ । \newline
20. म॒रुता॒म् तेज॒ स्तेजो॑ म॒रुता᳚म् म॒रुता॒म् तेज॒ स्तत् तत् तेजो॑ म॒रुता᳚म् म॒रुता॒म् तेज॒ स्तत् । \newline
21. तेज॒ स्तत् तत् तेज॒ स्तेज॒ स्तदे॒वैव तत् तेज॒ स्तेज॒ स्तदे॒व । \newline
22. तदे॒वैव तत् तदे॒वा वावै॒व तत् तदे॒वाव॑ । \newline
23. ए॒वावा वै॒वैवाव॑ रुन्धे रु॒न्धे ऽवै॒वैवाव॑ रुन्धे । \newline
24. अव॑ रुन्धे रु॒न्धे ऽवाव॑ रुन्धे॒ यद् यद् रु॒न्धे ऽवाव॑ रुन्धे॒ यत् । \newline
25. रु॒न्धे॒ यद् यद् रु॑न्धे रुन्धे॒ यदिन्द्रा॒ये न्द्रा॑य॒ यद् रु॑न्धे रुन्धे॒ यदिन्द्रा॑य । \newline
26. यदिन्द्रा॒ये न्द्रा॑य॒ यद् यदिन्द्रा॑य रैव॒ताय॑ रैव॒ताये न्द्रा॑य॒ यद् यदिन्द्रा॑य रैव॒ताय॑ । \newline
27. इन्द्रा॑य रैव॒ताय॑ रैव॒ताये न्द्रा॒ये न्द्रा॑य रैव॒ताय॒ यद् यद् रै॑व॒ताये न्द्रा॒ये न्द्रा॑य रैव॒ताय॒ यत् । \newline
28. रै॒व॒ताय॒ यद् यद् रै॑व॒ताय॑ रैव॒ताय॒ यदे॒वैव यद् रै॑व॒ताय॑ रैव॒ताय॒ यदे॒व । \newline
29. यदे॒वैव यद् यदे॒व बृह॒स्पते॒र् बृह॒स्पते॑ रे॒व यद् यदे॒व बृह॒स्पतेः᳚ । \newline
30. ए॒व बृह॒स्पते॒र् बृह॒स्पते॑ रे॒वैव बृह॒स्पते॒ स्तेज॒ स्तेजो॒ बृह॒स्पते॑ रे॒वैव बृह॒स्पते॒ स्तेजः॑ । \newline
31. बृह॒स्पते॒ स्तेज॒ स्तेजो॒ बृह॒स्पते॒र् बृह॒स्पते॒ स्तेज॒ स्तत् तत् तेजो॒ बृह॒स्पते॒र् बृह॒स्पते॒ स्तेज॒ स्तत् । \newline
32. तेज॒ स्तत् तत् तेज॒ स्तेज॒ स्तदे॒वैव तत् तेज॒ स्तेज॒ स्तदे॒व । \newline
33. तदे॒वैव तत् तदे॒वा वावै॒व तत् तदे॒वाव॑ । \newline
34. ए॒वावा वै॒वैवाव॑ रुन्धे रु॒न्धे ऽवै॒वैवाव॑ रुन्धे । \newline
35. अव॑ रुन्धे रु॒न्धे ऽवाव॑ रुन्ध ए॒ताव॑ न्त्ये॒ताव॑न्ति रु॒न्धे ऽवाव॑ रुन्ध ए॒ताव॑न्ति । \newline
36. रु॒न्ध॒ ए॒ताव॑ न्त्ये॒ताव॑न्ति रुन्धे रुन्ध ए॒ताव॑न्ति॒ वै वा ए॒ताव॑न्ति रुन्धे रुन्ध ए॒ताव॑न्ति॒ वै । \newline
37. ए॒ताव॑न्ति॒ वै वा ए॒ताव॑ न्त्ये॒ताव॑न्ति॒ वै तेजाꣳ॑सि॒ तेजाꣳ॑सि॒ वा ए॒ताव॑ न्त्ये॒ताव॑न्ति॒ वै तेजाꣳ॑सि । \newline
38. वै तेजाꣳ॑सि॒ तेजाꣳ॑सि॒ वै वै तेजाꣳ॑सि॒ तानि॒ तानि॒ तेजाꣳ॑सि॒ वै वै तेजाꣳ॑सि॒ तानि॑ । \newline
39. तेजाꣳ॑सि॒ तानि॒ तानि॒ तेजाꣳ॑सि॒ तेजाꣳ॑सि॒ तान्ये॒वैव तानि॒ तेजाꣳ॑सि॒ तेजाꣳ॑सि॒ तान्ये॒व । \newline
40. तान्ये॒वैव तानि॒ ता न्ये॒वावावै॒व तानि॒ तान्ये॒वाव॑ । \newline
41. ए॒वावा वै॒वैवाव॑ रुन्धे रु॒न्धे ऽवै॒वैवाव॑ रुन्धे । \newline
42. अव॑ रुन्धे रु॒न्धे ऽवाव॑ रुन्ध उत्ता॒नेषू᳚ त्ता॒नेषु॑ रु॒न्धे ऽवाव॑ रुन्ध उत्ता॒नेषु॑ । \newline
43. रु॒न्ध॒ उ॒त्ता॒नेषू᳚त्ता॒नेषु॑ रुन्धे रुन्ध उत्ता॒नेषु॑ क॒पाले॑षु क॒पाले॑षू त्ता॒नेषु॑ रुन्धे रुन्ध उत्ता॒नेषु॑ क॒पाले॑षु । \newline
44. उ॒त्ता॒नेषु॑ क॒पाले॑षु क॒पाले॑षू त्ता॒नेषू᳚ त्ता॒नेषु॑ क॒पाले॒ ष्वध्यधि॑ क॒पाले॑षू त्ता॒नेषू᳚ त्ता॒नेषु॑ क॒पाले॒ ष्वधि॑ । \newline
45. उ॒त्ता॒नेष्वित्यु॑त् - ता॒नेषु॑ । \newline
46. क॒पाले॒ ष्वध्यधि॑ क॒पाले॑षु क॒पाले॒ ष्वधि॑ श्रयति श्रय॒त्यधि॑ क॒पाले॑षु क॒पाले॒ ष्वधि॑ श्रयति । \newline
47. अधि॑ श्रयति श्रय॒ त्यध्यधि॑ श्रय॒ त्यया॑तयामत्वा॒या या॑तयामत्वाय श्रय॒ त्यध्यधि॑ श्रय॒ त्यया॑तयामत्वाय । \newline
48. श्र॒य॒ त्यया॑तयामत्वा॒या या॑तयामत्वाय श्रयति श्रय॒ त्यया॑तयामत्वाय॒ द्वाद॑शकपालो॒ द्वाद॑शकपा॒लो ऽया॑तयामत्वाय श्रयति श्रय॒ त्यया॑तयामत्वाय॒ द्वाद॑शकपालः । \newline
49. अया॑तयामत्वाय॒ द्वाद॑शकपालो॒ द्वाद॑शकपा॒लो ऽया॑तयामत्वा॒या या॑तयामत्वाय॒ द्वाद॑शकपालः पुरो॒डाशः॑ पुरो॒डाशो॒ द्वाद॑शकपा॒लो ऽया॑तयामत्वा॒या या॑तयामत्वाय॒ द्वाद॑शकपालः पुरो॒डाशः॑ । \newline
50. अया॑तयामत्वा॒येत्यया॑तयाम - त्वा॒य॒ । \newline
51. द्वाद॑शकपालः पुरो॒डाशः॑ पुरो॒डाशो॒ द्वाद॑शकपालो॒ द्वाद॑शकपालः पुरो॒डाशो॑ भवति भवति पुरो॒डाशो॒ द्वाद॑शकपालो॒ द्वाद॑शकपालः पुरो॒डाशो॑ भवति । \newline
52. द्वाद॑शकपाल॒ इति॒ द्वाद॑श - क॒पा॒लः॒ । \newline
53. पु॒रो॒डाशो॑ भवति भवति पुरो॒डाशः॑ पुरो॒डाशो॑ भवति वैश्वदेव॒त्वाय॑ वैश्वदेव॒त्वाय॑ भवति पुरो॒डाशः॑ पुरो॒डाशो॑ भवति वैश्वदेव॒त्वाय॑ । \newline
\pagebreak
\markright{ TS 2.3.7.4  \hfill https://www.vedavms.in \hfill}

\section{ TS 2.3.7.4 }

\textbf{TS 2.3.7.4 } \newline
\textbf{Samhita Paata} \newline

भवति वैश्वदेव॒त्वाय॑ सम॒न्तं प॒र्यव॑द्यति सम॒न्तमे॒वेन्द्रि॒यं ॅवी॒र्यं॑ ॅयज॑माने दधाति व्य॒त्यास॒-मन्वा॒हानि॑र्दाहा॒याश्व॑ ऋष॒भो वृ॒ष्णिर्ब॒स्तः सादक्षि॑णावृष॒त्वायै॒तयै॒वय॑जेताभिश॒स्यमा॑न ए॒ताश्चेद्वा अ॑स्यदे॒वता॒ अन्न॑म॒दन्त्य॒दन्त्यु॑वे॒वास्य॑ मनु॒ष्याः᳚ ॥ \newline

\textbf{Pada Paata} \newline

भ॒व॒ति॒ । वै॒श्व॒दे॒व॒त्वायेति॑ वैश्वदेव - त्वाय॑ । स॒म॒न्तमिति॑ सं - अ॒न्तम् । प॒र्यव॑द्य॒तीति॑ परि - अव॑द्यति । स॒म॒न्तमिति॑ सं - अ॒न्तम् । ए॒व ।  इ॒न्द्रि॒यम् । वी॒र्य᳚म् । यज॑माने । द॒धा॒ति॒ । व्य॒त्यास॒मिति॑ वि - अ॒त्यास᳚म् । अन्विति॑ । आ॒ह॒ । अनि॑र्दाहा॒येत्यनिः॑ - दा॒हा॒य॒ । अश्वः॑ ।   ऋ॒ष॒भः । वृ॒ष्णिः । ब॒स्तः । सा । दक्षि॑णा । वृ॒ष॒त्वायेति॑ वृष - त्वाय॑ । ए॒तया᳚ । ए॒व । य॒जे॒त॒ । अ॒भि॒श॒स्यमा॑न॒ इत्य॑भि - श॒स्यमा॑नः । ए॒ताः । च॒ । इत् । वै । अ॒स्य॒ । दे॒वताः᳚ । अन्न᳚म् । अ॒दन्ति॑ । अ॒दन्ति॑ । उ॒ । ए॒व । अ॒स्य॒ । म॒नु॒ष्याः᳚ ॥  \newline



\textbf{Jatai Paata} \newline

1. भ॒व॒ति॒ वै॒श्व॒दे॒व॒त्वाय॑ वैश्वदेव॒त्वाय॑ भवति भवति वैश्वदेव॒त्वाय॑ । \newline
2. वै॒श्व॒दे॒व॒त्वाय॑ सम॒न्तꣳ स॑म॒न्तं ॅवै᳚श्वदेव॒त्वाय॑ वैश्वदेव॒त्वाय॑ सम॒न्तम् । \newline
3. वै॒श्व॒दे॒व॒त्वायेति॑ वैश्वदेव - त्वाय॑ । \newline
4. स॒म॒न्तम् प॒र्यव॑द्यति प॒र्यव॑द्यति सम॒न्तꣳ स॑म॒न्तम् प॒र्यव॑द्यति । \newline
5. स॒म॒न्तमिति॑ सं - अ॒न्तम् । \newline
6. प॒र्यव॑द्यति सम॒न्तꣳ स॑म॒न्तम् प॒र्यव॑द्यति प॒र्यव॑द्यति सम॒न्तम् । \newline
7. प॒र्यव॑द्य॒तीति॑ परि - अव॑द्यति । \newline
8. स॒म॒न्त मे॒वैव स॑म॒न्तꣳ स॑म॒न्त मे॒व । \newline
9. स॒म॒न्तमिति॑ सं - अ॒न्तम् । \newline
10. ए॒वे न्द्रि॒य मि॑न्द्रि॒य मे॒वैवे न्द्रि॒यम् । \newline
11. इ॒न्द्रि॒यं ॅवी॒र्यं॑ ॅवी॒र्य॑ मिन्द्रि॒य मि॑न्द्रि॒यं ॅवी॒र्य᳚म् । \newline
12. वी॒र्यं॑ ॅयज॑माने॒ यज॑माने वी॒र्यं॑ ॅवी॒र्यं॑ ॅयज॑माने । \newline
13. यज॑माने दधाति दधाति॒ यज॑माने॒ यज॑माने दधाति । \newline
14. द॒धा॒ति॒ व्य॒त्यासं॑ ॅव्य॒त्यास॑म् दधाति दधाति व्य॒त्यास᳚म् । \newline
15. व्य॒त्यास॒ मन्वनु॑ व्य॒त्यासं॑ ॅव्य॒त्यास॒ मनु॑ । \newline
16. व्य॒त्यास॒मिति॑ वि - अ॒त्यास᳚म् । \newline
17. अन्वा॑ हा॒हा न्वन्वा॑ह । \newline
18. आ॒हा नि॑र्दाहा॒या नि॑र्दाहाया हा॒हा नि॑र्दाहाय । \newline
19. अनि॑र्दाहा॒याश्वो ऽश्वो ऽनि॑र्दाहा॒या नि॑र्दाहा॒याश्वः॑ । \newline
20. अनि॑र्दाहा॒येत्यनिः॑ - दा॒हा॒य॒ । \newline
21. अश्व॑ ऋष॒भ ऋ॑ष॒भो अश्वो ऽश्व॑ ऋष॒भः । \newline
22. ऋ॒ष॒भो वृ॒ष्णिर् वृ॒ष्णिर्. ऋ॑ष॒भ ऋ॑ष॒भो वृ॒ष्णिः । \newline
23. वृ॒ष्णिर् ब॒स्तो ब॒स्तो वृ॒ष्णिर् वृ॒ष्णिर् ब॒स्तः । \newline
24. ब॒स्तः सा सा ब॒स्तो ब॒स्तः सा । \newline
25. सा दक्षि॑णा॒ दक्षि॑णा॒ सा सा दक्षि॑णा । \newline
26. दक्षि॑णा वृष॒त्वाय॑ वृष॒त्वाय॒ दक्षि॑णा॒ दक्षि॑णा वृष॒त्वाय॑ । \newline
27. वृ॒ष॒त्वा यै॒तयै॒तया॑ वृष॒त्वाय॑ वृष॒त्वा यै॒तया᳚ । \newline
28. वृ॒ष॒त्वायेति॑ वृष - त्वाय॑ । \newline
29. ए॒त यै॒वैवैत यै॒तयै॒व । \newline
30. ए॒व य॑जेत यजे तै॒वैव य॑जेत । \newline
31. य॒जे॒ता॒ भि॒श॒स्यमा॑नो ऽभिश॒स्यमा॑नो यजेत यजेता भिश॒स्यमा॑नः । \newline
32. अ॒भि॒श॒स्यमा॑न ए॒ता ए॒ता अ॑भिश॒स्यमा॑नो ऽभिश॒स्यमा॑न ए॒ताः । \newline
33. अ॒भि॒श॒स्यमा॑न॒ इत्य॑भि - श॒स्यमा॑नः । \newline
34. ए॒ताश्च॑ चै॒ता ए॒ताश्च॑ । \newline
35. चे दिच् च॒ चे त् । \newline
36. इद् वै वा इदिद् वै । \newline
37. वा अ॑स्यास्य॒ वै वा अ॑स्य । \newline
38. अ॒स्य॒ दे॒वता॑ दे॒वता॑ अस्यास्य दे॒वताः᳚ । \newline
39. दे॒वता॒ अन्न॒ मन्न॑म् दे॒वता॑ दे॒वता॒ अन्न᳚म् । \newline
40. अन्न॑ म॒द न्त्य॒द न्त्यन्न॒ मन्न॑ म॒दन्ति॑ । \newline
41. अ॒दन्त्य॒दन्ति॑ । \newline
42. अ॒दन्त्यु॑ वु व॒द न्त्य॒दन्त्यु॑ । \newline
43. उ॒ वे॒वैव वु॑ वे॒व । \newline
44. ए॒वा स्या᳚ स्यै॒वैवास्य॑ । \newline
45. अ॒स्य॒ म॒नु॒ष्या॑ मनु॒ष्या॑ अस्यास्य मनु॒ष्याः᳚ । \newline
46. म॒नु॒ष्या॑ इति॑ मनु॒ष्याः᳚ । \newline

\textbf{Ghana Paata } \newline

1. भ॒व॒ति॒ वै॒श्व॒दे॒व॒त्वाय॑ वैश्वदेव॒त्वाय॑ भवति भवति वैश्वदेव॒त्वाय॑ सम॒न्तꣳ स॑म॒न्तं ॅवै᳚श्वदेव॒त्वाय॑ भवति भवति वैश्वदेव॒त्वाय॑ सम॒न्तम् । \newline
2. वै॒श्व॒दे॒व॒त्वाय॑ सम॒न्तꣳ स॑म॒न्तं ॅवै᳚श्वदेव॒त्वाय॑ वैश्वदेव॒त्वाय॑ सम॒न्तम् प॒र्यव॑द्यति प॒र्यव॑द्यति सम॒न्तं ॅवै᳚श्वदेव॒त्वाय॑ वैश्वदेव॒त्वाय॑ सम॒न्तम् प॒र्यव॑द्यति । \newline
3. वै॒श्व॒दे॒व॒त्वायेति॑ वैश्वदेव - त्वाय॑ । \newline
4. स॒म॒न्तम् प॒र्यव॑द्यति प॒र्यव॑द्यति सम॒न्तꣳ स॑म॒न्तम् प॒र्यव॑द्यति सम॒न्तꣳ स॑म॒न्तम् प॒र्यव॑द्यति सम॒न्तꣳ स॑म॒न्तम् प॒र्यव॑द्यति सम॒न्तम् । \newline
5. स॒म॒न्तमिति॑ सं - अ॒न्तम् । \newline
6. प॒र्यव॑द्यति सम॒न्तꣳ स॑म॒न्तम् प॒र्यव॑द्यति प॒र्यव॑द्यति सम॒न्त मे॒वैव स॑म॒न्तम् प॒र्यव॑द्यति प॒र्यव॑द्यति सम॒न्त मे॒व । \newline
7. प॒र्यव॑द्य॒तीति॑ परि - अव॑द्यति । \newline
8. स॒म॒न्त मे॒वैव स॑म॒न्तꣳ स॑म॒न्त मे॒वे न्द्रि॒य मि॑न्द्रि॒य मे॒व स॑म॒न्तꣳ स॑म॒न्त मे॒वे न्द्रि॒यम् । \newline
9. स॒म॒न्तमिति॑ सं - अ॒न्तम् । \newline
10. ए॒वे न्द्रि॒य मि॑न्द्रि॒य मे॒वैवे न्द्रि॒यं ॅवी॒र्यं॑ ॅवी॒र्य॑ मिन्द्रि॒य मे॒वैवे न्द्रि॒यं ॅवी॒र्य᳚म् । \newline
11. इ॒न्द्रि॒यं ॅवी॒र्यं॑ ॅवी॒र्य॑ मिन्द्रि॒य मि॑न्द्रि॒यं ॅवी॒र्यं॑ ॅयज॑माने॒ यज॑माने वी॒र्य॑ मिन्द्रि॒य मि॑न्द्रि॒यं ॅवी॒र्यं॑ ॅयज॑माने । \newline
12. वी॒र्यं॑ ॅयज॑माने॒ यज॑माने वी॒र्यं॑ ॅवी॒र्यं॑ ॅयज॑माने दधाति दधाति॒ यज॑माने वी॒र्यं॑ ॅवी॒र्यं॑ ॅयज॑माने दधाति । \newline
13. यज॑माने दधाति दधाति॒ यज॑माने॒ यज॑माने दधाति व्य॒त्यासं॑ ॅव्य॒त्यास॑म् दधाति॒ यज॑माने॒ यज॑माने दधाति व्य॒त्यास᳚म् । \newline
14. द॒धा॒ति॒ व्य॒त्यासं॑ ॅव्य॒त्यास॑म् दधाति दधाति व्य॒त्यास॒ मन्वनु॑ व्य॒त्यास॑म् दधाति दधाति व्य॒त्यास॒ मनु॑ । \newline
15. व्य॒त्यास॒ मन्वनु॑ व्य॒त्यासं॑ ॅव्य॒त्यास॒ मन्वा॑हा॒हानु॑ व्य॒त्यासं॑ ॅव्य॒त्यास॒ मन्वा॑ह । \newline
16. व्य॒त्यास॒मिति॑ वि - अ॒त्यास᳚म् । \newline
17. अन्वा॑हा॒हा न्वन्वा॒हा नि॑र्दाहा॒या नि॑र्दाहाया॒हा न्वन्वा॒हा नि॑र्दाहाय । \newline
18. आ॒हा नि॑र्दाहा॒या नि॑र्दाहायाहा॒हा नि॑र्दाहा॒याश्वो ऽश्वो ऽनि॑र्दाहायाहा॒हा नि॑र्दाहा॒याश्वः॑ । \newline
19. अनि॑र्दाहा॒याश्वो ऽश्वो ऽनि॑र्दाहा॒या नि॑र्दाहा॒याश्व॑ ऋष॒भ ऋ॑ष॒भो अश्वो ऽनि॑र्दाहा॒या नि॑र्दाहा॒याश्व॑ ऋष॒भः । \newline
20. अनि॑र्दाहा॒येत्यनिः॑ - दा॒हा॒य॒ । \newline
21. अश्व॑ ऋष॒भ ऋ॑ष॒भो अश्वो ऽश्व॑ ऋष॒भो वृ॒ष्णिर् वृ॒ष्णिर्. ऋ॑ष॒भो अश्वो ऽश्व॑ ऋष॒भो वृ॒ष्णिः । \newline
22. ऋ॒ष॒भो वृ॒ष्णिर् वृ॒ष्णिर्. ऋ॑ष॒भ ऋ॑ष॒भो वृ॒ष्णिर् ब॒स्तो ब॒स्तो वृ॒ष्णिर्. ऋ॑ष॒भ ऋ॑ष॒भो वृ॒ष्णिर् ब॒स्तः । \newline
23. वृ॒ष्णिर् ब॒स्तो ब॒स्तो वृ॒ष्णिर् वृ॒ष्णिर् ब॒स्तः सा सा ब॒स्तो वृ॒ष्णिर् वृ॒ष्णिर् ब॒स्तः सा । \newline
24. ब॒स्तः सा सा ब॒स्तो ब॒स्तः सा दक्षि॑णा॒ दक्षि॑णा॒ सा ब॒स्तो ब॒स्तः सा दक्षि॑णा । \newline
25. सा दक्षि॑णा॒ दक्षि॑णा॒ सा सा दक्षि॑णा वृष॒त्वाय॑ वृष॒त्वाय॒ दक्षि॑णा॒ सा सा दक्षि॑णा वृष॒त्वाय॑ । \newline
26. दक्षि॑णा वृष॒त्वाय॑ वृष॒त्वाय॒ दक्षि॑णा॒ दक्षि॑णा वृष॒त्वा यै॒त यै॒तया॑ वृष॒त्वाय॒ दक्षि॑णा॒ दक्षि॑णा वृष॒त्वायै॒तया᳚ । \newline
27. वृ॒ष॒त्वा यै॒तयै॒तया॑ वृष॒त्वाय॑ वृष॒त्वा यै॒तयै॒ वैवैतया॑ वृष॒त्वाय॑ वृष॒त्वा यै॒तयै॒व । \newline
28. वृ॒ष॒त्वायेति॑ वृष - त्वाय॑ । \newline
29. ए॒त यै॒वैवैत यै॒तयै॒व य॑जेत यजेतै॒वैत यै॒तयै॒व य॑जेत । \newline
30. ए॒व य॑जेत यजेतै॒वैव य॑जेता भिश॒स्यमा॑नो ऽभिश॒स्यमा॑नो यजेतै॒वैव य॑जेता भिश॒स्यमा॑नः । \newline
31. य॒जे॒ता॒ भि॒श॒स्यमा॑नो ऽभिश॒स्यमा॑नो यजेत यजेता भिश॒स्यमा॑न ए॒ता ए॒ता अ॑भिश॒स्यमा॑नो यजेत यजेता भिश॒स्यमा॑न ए॒ताः । \newline
32. अ॒भि॒श॒स्यमा॑न ए॒ता ए॒ता अ॑भिश॒स्यमा॑नो ऽभिश॒स्यमा॑न ए॒ताश्च॑ चै॒ता अ॑भिश॒स्यमा॑नो ऽभिश॒स्यमा॑न ए॒ताश्च॑ । \newline
33. अ॒भि॒श॒स्यमा॑न॒ इत्य॑भि - श॒स्यमा॑नः । \newline
34. ए॒ताश्च॑ चै॒ता ए॒ताश्चे दिच् चै॒ता ए॒ताश्चेत् । \newline
35. चे दिच् च॒ चेद् वै वा इच् च॒ चेद् वै । \newline
36. इद् वै वा इदिद् वा अ॑स्यास्य॒ वा इदिद् वा अ॑स्य । \newline
37. वा अ॑स्यास्य॒ वै वा अ॑स्य दे॒वता॑ दे॒वता॑ अस्य॒ वै वा अ॑स्य दे॒वताः᳚ । \newline
38. अ॒स्य॒ दे॒वता॑ दे॒वता॑ अस्यास्य दे॒वता॒ अन्न॒ मन्न॑म् दे॒वता॑ अस्यास्य दे॒वता॒ अन्न᳚म् । \newline
39. दे॒वता॒ अन्न॒ मन्न॑म् दे॒वता॑ दे॒वता॒ अन्न॑ म॒द न्त्य॒द न्त्यन्न॑म् दे॒वता॑ दे॒वता॒ अन्न॑ म॒दन्ति॑ । \newline
40. अन्न॑ म॒द न्त्य॒द न्त्यन्न॒ मन्न॑ म॒दन्ति॑ । \newline
41. अ॒दन्त्य॒दन्ति॑ । \newline
42. अ॒द न्त्यु॑ वु व॒द न्त्य॒द न्त्यु॑ वे॒वैवो॑ ऽद न्त्य॒द न्त्यु॑ वे॒व । \newline
43. उ॒ वे॒वैव वु॑ वे॒वास्या᳚स्यै॒व वु॑ वे॒वास्य॑ । \newline
44. ए॒वास्या᳚ स्यै॒वैवास्य॑ मनु॒ष्या॑ मनु॒ष्या॑ अस्यै॒वैवास्य॑ मनु॒ष्याः᳚ । \newline
45. अ॒स्य॒ म॒नु॒ष्या॑ मनु॒ष्या॑ अस्यास्य मनु॒ष्याः᳚ । \newline
46. म॒नु॒ष्या॑ इति॑ मनु॒ष्याः᳚ । \newline
\pagebreak
\markright{ TS 2.3.8.1  \hfill https://www.vedavms.in \hfill}

\section{ TS 2.3.8.1 }

\textbf{TS 2.3.8.1 } \newline
\textbf{Samhita Paata} \newline

रज॑नो॒ वै कौ॑णे॒यः क्र॑तु॒जितं॒ जान॑किं चक्षु॒र्वन्य॑मया॒त् तस्मा॑ ए॒तामिष्टिं॒ निर॑वपद॒ग्नये॒ भ्राज॑स्वते पुरो॒डाश॑म॒ष्टाक॑पालꣳ सौ॒र्यं च॒रुम॒ग्नये॒ भ्राज॑स्वते पुरो॒डाश॑म॒ष्टाक॑पालं॒ तयै॒वास्मि॒न् चक्षु॑रदधा॒द्-य-श्चक्षु॑कामः॒ स्यात् तस्मा॑ ए॒तामिष्टिं॒ निर्व॑पेद॒ग्नये॒ भ्राज॑स्वते पुरो॒डाश॑म॒ष्टाक॑पालꣳ सौ॒र्यं च॒रुम॒ग्नये॒ भ्राज॑स्वते पुरो॒डाश॑म॒ष्टाक॑पालम॒ग्ने र्वै चक्षु॑षा मनु॒ष्या॑ वि - [  ] \newline

\textbf{Pada Paata} \newline

रज॑नः । वै । कौ॒णे॒यः । क्र॒तु॒जित॒मिति॑ क्रतु - जित᳚म् । जान॑किम् । च॒क्षु॒र्वन्य॒मिति॑ चक्षुः - वन्य᳚म् । अ॒या॒त् । तस्मै᳚ । ए॒ताम् । इष्टि᳚म् । निरिति॑ । अ॒व॒प॒त् । अ॒ग्नये᳚ । भ्राज॑स्वते । पु॒रो॒डाश᳚म् । अ॒ष्टाक॑पाल॒मित्य॒ष्टा - क॒पा॒ल॒म् । सौ॒र्यम् । च॒रुम् । अ॒ग्नये᳚ । भ्राज॑स्वते । पु॒रो॒डाश᳚म् । अ॒ष्टाक॑पाल॒मित्य॒ष्टा - क॒पा॒ल॒म् । तया᳚ । ए॒व । अ॒स्मि॒न्न् । चक्षुः॑ । अ॒द॒धा॒त् । यः । चक्षु॑ष्काम॒ इति॒ चक्षुः॑ - का॒मः॒ । स्यात् । तस्मै᳚ । ए॒ताम् । इष्टि᳚म् । निरिति॑ । व॒पे॒त् । अ॒ग्नये᳚ । भ्राज॑स्वते । पु॒रो॒डाश᳚म् । अ॒ष्टाक॑पाल॒मित्य॒ष्टा - क॒पा॒ल॒म् । सौ॒र्यम् । च॒रुम् । अ॒ग्नये᳚ । भ्राज॑स्वते । पु॒रो॒डाश᳚म् । अ॒ष्टाक॑पाल॒मित्य॒ष्टा - क॒पा॒ल॒म् । अ॒ग्नेः । वै । चक्षु॑षा । म॒नु॒ष्याः᳚ । वीति॑ ।  \newline


\textbf{Krama Paata} \newline

रज॑नो॒ वै । वै कौ॑णे॒यः । कौ॒णे॒यः क्र॑तु॒जित᳚म् । क्र॒तु॒जित॒म् जान॑किम् । क्र॒तु॒जित॒मिति॑ क्रतु - जित᳚म् । जान॑किम् चक्षु॒र्वन्य᳚म् । च॒क्षु॒र्वन्य॑मयात् । च॒क्षु॒र्वन्य॒ मिति॑ चक्षुः - वन्य᳚म् । अ॒या॒त् तस्मै᳚ । तस्मा॑ ए॒ताम् । ए॒तामिष्टि᳚म् । इष्टि॒म् निः । निर॑वपत् । अ॒व॒प॒द॒ग्नये᳚ । अ॒ग्नये॒ भ्राज॑स्वते । भ्राज॑स्वते पुरो॒डाश᳚म् । पु॒रो॒डाश॑म॒ष्टाक॑पालम् । अ॒ष्टाक॑पालꣳ सौ॒र्यम् । अ॒ष्टाक॑पाल॒मित्य॒ष्टा - क॒पा॒ल॒म् । सौ॒र्यम् च॒रुम् । च॒रुम॒ग्नये᳚ । अ॒ग्नये॒ भ्राज॑स्वते । भ्राज॑स्वते पुरो॒डाश᳚म् । पु॒रो॒डाश॑म॒ष्टाक॑पालम् । अ॒ष्टाक॑पाल॒म् तया᳚ । अ॒ष्टाक॑पाल॒मित्य॒ष्टा - क॒पा॒ल॒म् । तयै॒व । ए॒वास्मिन्न्॑ । अ॒स्मि॒न् चक्षुः॑ । चक्षु॑रदधात् । अ॒द॒धा॒द् यः । यश्चक्षु॑ष्कामः । चक्षु॑ष्कामः॒ स्यात् । चक्षु॑ष्काम॒ इति॒ चक्षुः॑ - का॒मः॒ । स्यात् तस्मै᳚ । तस्मा॑ ए॒ताम् । ए॒तामिष्टि᳚म् । इष्टि॒म् निः । निर् व॑पेत् । व॒पे॒द॒ग्नये᳚ । अ॒ग्नये॒ भ्राज॑स्वते । भ्राज॑स्वते पुरो॒डाश᳚म् । पु॒रो॒डाश॑म॒ष्टाक॑पालम् । अ॒ष्टाक॑पालꣳ सौ॒र्यम् । अ॒ष्टाक॑पाल॒मित्य॒ष्टा - क॒पा॒ल॒म् । सौ॒र्यम् च॒रुम् । च॒रुम॒ग्नये᳚ । अ॒ग्नये॒ भ्राज॑स्वते । भ्राज॑स्वते पुरो॒डाश᳚म् । पु॒रो॒डाश॑म॒ष्टाक॑पालम् । अ॒ष्टाक॑पालम॒ग्नेः । अ॒ष्टाक॑पाल॒मित्य॒ष्टा - क॒पा॒ल॒म् । अ॒ग्नेर् वै । वै चक्षु॑षा । चक्षु॑षा मनु॒ष्याः᳚ । म॒नु॒ष्या॑ वि । वि प॑श्यन्ति \newline

\textbf{Jatai Paata} \newline

1. रज॑नो॒ वै वै रज॑नो॒ रज॑नो॒ वै । \newline
2. वै कौ॑णे॒यः कौ॑णे॒यो वै वै कौ॑णे॒यः । \newline
3. कौ॒णे॒यः क्र॑तु॒जित॑म् क्रतु॒जित॑म् कौणे॒यः कौ॑णे॒यः क्र॑तु॒जित᳚म् । \newline
4. क्र॒तु॒जित॒म् जान॑कि॒म् जान॑किम् क्रतु॒जित॑म् क्रतु॒जित॒म् जान॑किम् । \newline
5. क्र॒तु॒जित॒मिति॑ क्रतु - जित᳚म् । \newline
6. जान॑किम् चक्षु॒र्वन्य॑म् चक्षु॒र्वन्य॒म् जान॑कि॒म् जान॑किम् चक्षु॒र्वन्य᳚म् । \newline
7. च॒क्षु॒र्वन्य॑ मयादयाच् चक्षु॒र्वन्य॑म् चक्षु॒र्वन्य॑ मयात् । \newline
8. च॒क्षु॒र्वन्य॒मिति॑ चक्षुः - वन्य᳚म् । \newline
9. अ॒या॒त् तस्मै॒ तस्मा॑ अयादया॒त् तस्मै᳚ । \newline
10. तस्मा॑ ए॒ता मे॒ताम् तस्मै॒ तस्मा॑ ए॒ताम् । \newline
11. ए॒ता मिष्टि॒ मिष्टि॑ मे॒ता मे॒ता मिष्टि᳚म् । \newline
12. इष्टि॒म् निर् णिरिष्टि॒ मिष्टि॒म् निः । \newline
13. निर॑वप दवप॒न् निर् णिर॑वपत् । \newline
14. अ॒व॒प॒ द॒ग्नये॒ ऽग्नये॑ ऽवप दवप द॒ग्नये᳚ । \newline
15. अ॒ग्नये॒ भ्राज॑स्वते॒ भ्राज॑स्वते॒ ऽग्नये॒ ऽग्नये॒ भ्राज॑स्वते । \newline
16. भ्राज॑स्वते पुरो॒डाश॑म् पुरो॒डाश॒म् भ्राज॑स्वते॒ भ्राज॑स्वते पुरो॒डाश᳚म् । \newline
17. पु॒रो॒डाश॑ म॒ष्टाक॑पाल म॒ष्टाक॑पालम् पुरो॒डाश॑म् पुरो॒डाश॑ म॒ष्टाक॑पालम् । \newline
18. अ॒ष्टाक॑पालꣳ सौ॒र्यꣳ सौ॒र्य म॒ष्टाक॑पाल म॒ष्टाक॑पालꣳ सौ॒र्यम् । \newline
19. अ॒ष्टाक॑पाल॒मित्य॒ष्टा - क॒पा॒ल॒म् । \newline
20. सौ॒र्यम् च॒रुम् च॒रुꣳ सौ॒र्यꣳ सौ॒र्यम् च॒रुम् । \newline
21. च॒रु म॒ग्नये॒ ऽग्नये॑ च॒रुम् च॒रु म॒ग्नये᳚ । \newline
22. अ॒ग्नये॒ भ्राज॑स्वते॒ भ्राज॑स्वते॒ ऽग्नये॒ ऽग्नये॒ भ्राज॑स्वते । \newline
23. भ्राज॑स्वते पुरो॒डाश॑म् पुरो॒डाश॒म् भ्राज॑स्वते॒ भ्राज॑स्वते पुरो॒डाश᳚म् । \newline
24. पु॒रो॒डाश॑ म॒ष्टाक॑पाल म॒ष्टाक॑पालम् पुरो॒डाश॑म् पुरो॒डाश॑ म॒ष्टाक॑पालम् । \newline
25. अ॒ष्टाक॑पाल॒म् तया॒ तया॒ ऽष्टाक॑पाल म॒ष्टाक॑पाल॒म् तया᳚ । \newline
26. अ॒ष्टाक॑पाल॒मित्य॒ष्टा - क॒पा॒ल॒म् । \newline
27. तयै॒वैव तया॒ तयै॒व । \newline
28. ए॒वास्मि॑न् नस्मिन् ने॒वैवास्मिन्न्॑ । \newline
29. अ॒स्मि॒न् चक्षु॒ श्चक्षु॑रस्मिन् नस्मि॒न् चक्षुः॑ । \newline
30. चक्षु॑ रदधा ददधा॒च् चक्षु॒ श्चक्षु॑ रदधात् । \newline
31. अ॒द॒धा॒द् यो यो॑ ऽदधा ददधा॒द् यः । \newline
32. यश्चक्षु॑ष्काम॒ श्चक्षु॑ष्कामो॒ यो यश्चक्षु॑ष्कामः । \newline
33. चक्षु॑ष्कामः॒ स्याथ् स्याच् चक्षु॑ष्काम॒ श्चक्षु॑ष्कामः॒ स्यात् । \newline
34. चक्षु॑ष्काम॒ इति॒ चक्षुः॑ - का॒मः॒ । \newline
35. स्यात् तस्मै॒ तस्मै॒ स्याथ् स्यात् तस्मै᳚ । \newline
36. तस्मा॑ ए॒ता मे॒ताम् तस्मै॒ तस्मा॑ ए॒ताम् । \newline
37. ए॒ता मिष्टि॒ मिष्टि॑ मे॒ता मे॒ता मिष्टि᳚म् । \newline
38. इष्टि॒म् निर् णिरिष्टि॒ मिष्टि॒म् निः । \newline
39. निर् व॑पेद् वपे॒न् निर् णिर् व॑पेत् । \newline
40. व॒पे॒ द॒ग्नये॒ ऽग्नये॑ वपेद् वपे द॒ग्नये᳚ । \newline
41. अ॒ग्नये॒ भ्राज॑स्वते॒ भ्राज॑स्वते॒ ऽग्नये॒ ऽग्नये॒ भ्राज॑स्वते । \newline
42. भ्राज॑स्वते पुरो॒डाश॑म् पुरो॒डाश॒म् भ्राज॑स्वते॒ भ्राज॑स्वते पुरो॒डाश᳚म् । \newline
43. पु॒रो॒डाश॑ म॒ष्टाक॑पाल म॒ष्टाक॑पालम् पुरो॒डाश॑म् पुरो॒डाश॑ म॒ष्टाक॑पालम् । \newline
44. अ॒ष्टाक॑पालꣳ सौ॒र्यꣳ सौ॒र्य म॒ष्टाक॑पाल म॒ष्टाक॑पालꣳ सौ॒र्यम् । \newline
45. अ॒ष्टाक॑पाल॒मित्य॒ष्टा - क॒पा॒ल॒म् । \newline
46. सौ॒र्यम् च॒रुम् च॒रुꣳ सौ॒र्यꣳ सौ॒र्यम् च॒रुम् । \newline
47. च॒रु म॒ग्नये॒ ऽग्नये॑ च॒रुम् च॒रु म॒ग्नये᳚ । \newline
48. अ॒ग्नये॒ भ्राज॑स्वते॒ भ्राज॑स्वते॒ ऽग्नये॒ ऽग्नये॒ भ्राज॑स्वते । \newline
49. भ्राज॑स्वते पुरो॒डाश॑म् पुरो॒डाश॒म् भ्राज॑स्वते॒ भ्राज॑स्वते पुरो॒डाश᳚म् । \newline
50. पु॒रो॒डाश॑ म॒ष्टाक॑पाल म॒ष्टाक॑पालम् पुरो॒डाश॑म् पुरो॒डाश॑ म॒ष्टाक॑पालम् । \newline
51. अ॒ष्टाक॑पाल म॒ग्ने र॒ग्ने र॒ष्टाक॑पाल म॒ष्टाक॑पाल म॒ग्नेः । \newline
52. अ॒ष्टाक॑पाल॒मित्य॒ष्टा - क॒पा॒ल॒म् । \newline
53. अ॒ग्नेर् वै वा अ॒ग्ने र॒ग्नेर् वै । \newline
54. वै चक्षु॑षा॒ चक्षु॑षा॒ वै वै चक्षु॑षा । \newline
55. चक्षु॑षा मनु॒ष्या॑ मनु॒ष्या᳚ श्चक्षु॑षा॒ चक्षु॑षा मनु॒ष्याः᳚ । \newline
56. म॒नु॒ष्या॑ वि वि म॑नु॒ष्या॑ मनु॒ष्या॑ वि । \newline
57. वि प॑श्यन्ति पश्यन्ति॒ वि वि प॑श्यन्ति । \newline

\textbf{Ghana Paata } \newline

1. रज॑नो॒ वै वै रज॑नो॒ रज॑नो॒ वै कौ॑णे॒यः कौ॑णे॒यो वै रज॑नो॒ रज॑नो॒ वै कौ॑णे॒यः । \newline
2. वै कौ॑णे॒यः कौ॑णे॒यो वै वै कौ॑णे॒यः क्र॑तु॒जित॑म् क्रतु॒जित॑म् कौणे॒यो वै वै कौ॑णे॒यः क्र॑तु॒जित᳚म् । \newline
3. कौ॒णे॒यः क्र॑तु॒जित॑म् क्रतु॒जित॑म् कौणे॒यः कौ॑णे॒यः क्र॑तु॒जित॒म् जान॑कि॒म् जान॑किम् क्रतु॒जित॑म् कौणे॒यः कौ॑णे॒यः क्र॑तु॒जित॒म् जान॑किम् । \newline
4. क्र॒तु॒जित॒म् जान॑कि॒म् जान॑किम् क्रतु॒जित॑म् क्रतु॒जित॒म् जान॑किम् चक्षु॒र्वन्य॑म् चक्षु॒र्वन्य॒म् जान॑किम् क्रतु॒जित॑म् क्रतु॒जित॒म् जान॑किम् चक्षु॒र्वन्य᳚म् । \newline
5. क्र॒तु॒जित॒मिति॑ क्रतु - जित᳚म् । \newline
6. जान॑किम् चक्षु॒र्वन्य॑म् चक्षु॒र्वन्य॒म् जान॑कि॒म् जान॑किम् चक्षु॒र्वन्य॑ मयादयाच् चक्षु॒र्वन्य॒म् जान॑कि॒म् जान॑किम् चक्षु॒र्वन्य॑ मयात् । \newline
7. च॒क्षु॒र्वन्य॑ मयादयाच् चक्षु॒र्वन्य॑म् चक्षु॒र्वन्य॑ मया॒त् तस्मै॒ तस्मा॑ अयाच् चक्षु॒र्वन्य॑म् चक्षु॒र्वन्य॑ मया॒त् तस्मै᳚ । \newline
8. च॒क्षु॒र्वन्य॒मिति॑ चक्षुः - वन्य᳚म् । \newline
9. अ॒या॒त् तस्मै॒ तस्मा॑ अयादया॒त् तस्मा॑ ए॒ता मे॒ताम् तस्मा॑ अयादया॒त् तस्मा॑ ए॒ताम् । \newline
10. तस्मा॑ ए॒ता मे॒ताम् तस्मै॒ तस्मा॑ ए॒ता मिष्टि॒ मिष्टि॑ मे॒ताम् तस्मै॒ तस्मा॑ ए॒ता मिष्टि᳚म् । \newline
11. ए॒ता मिष्टि॒ मिष्टि॑ मे॒ता मे॒ता मिष्टि॒म् निर् णिरिष्टि॑ मे॒ता मे॒ता मिष्टि॒म् निः । \newline
12. इष्टि॒म् निर् णिरिष्टि॒ मिष्टि॒म् निर॑वप दवप॒न् निरिष्टि॒ मिष्टि॒म् निर॑वपत् । \newline
13. निर॑वप दवप॒न् निर् णिर॑वप द॒ग्नये॒ ऽग्नये॑ ऽवप॒न् निर् णिर॑वप द॒ग्नये᳚ । \newline
14. अ॒व॒प॒ द॒ग्नये॒ ऽग्नये॑ ऽवप दवप द॒ग्नये॒ भ्राज॑स्वते॒ भ्राज॑स्वते॒ ऽग्नये॑ ऽवप दवप द॒ग्नये॒ भ्राज॑स्वते । \newline
15. अ॒ग्नये॒ भ्राज॑स्वते॒ भ्राज॑स्वते॒ ऽग्नये॒ ऽग्नये॒ भ्राज॑स्वते पुरो॒डाश॑म् पुरो॒डाश॒म् भ्राज॑स्वते॒ ऽग्नये॒ ऽग्नये॒ भ्राज॑स्वते पुरो॒डाश᳚म् । \newline
16. भ्राज॑स्वते पुरो॒डाश॑म् पुरो॒डाश॒म् भ्राज॑स्वते॒ भ्राज॑स्वते पुरो॒डाश॑ म॒ष्टाक॑पाल म॒ष्टाक॑पालम् पुरो॒डाश॒म् भ्राज॑स्वते॒ भ्राज॑स्वते पुरो॒डाश॑ म॒ष्टाक॑पालम् । \newline
17. पु॒रो॒डाश॑ म॒ष्टाक॑पाल म॒ष्टाक॑पालम् पुरो॒डाश॑म् पुरो॒डाश॑ म॒ष्टाक॑पालꣳ सौ॒र्यꣳ सौ॒र्य म॒ष्टाक॑पालम् पुरो॒डाश॑म् पुरो॒डाश॑ म॒ष्टाक॑पालꣳ सौ॒र्यम् । \newline
18. अ॒ष्टाक॑पालꣳ सौ॒र्यꣳ सौ॒र्य म॒ष्टाक॑पाल म॒ष्टाक॑पालꣳ सौ॒र्यम् च॒रुम् च॒रुꣳ सौ॒र्य म॒ष्टाक॑पाल म॒ष्टाक॑पालꣳ सौ॒र्यम् च॒रुम् । \newline
19. अ॒ष्टाक॑पाल॒मित्य॒ष्टा - क॒पा॒ल॒म् । \newline
20. सौ॒र्यम् च॒रुम् च॒रुꣳ सौ॒र्यꣳ सौ॒र्यम् च॒रु म॒ग्नये॒ ऽग्नये॑ च॒रुꣳ सौ॒र्यꣳ सौ॒र्यम् च॒रु म॒ग्नये᳚ । \newline
21. च॒रु म॒ग्नये॒ ऽग्नये॑ च॒रुम् च॒रु म॒ग्नये॒ भ्राज॑स्वते॒ भ्राज॑स्वते॒ ऽग्नये॑ च॒रुम् च॒रु म॒ग्नये॒ भ्राज॑स्वते । \newline
22. अ॒ग्नये॒ भ्राज॑स्वते॒ भ्राज॑स्वते॒ ऽग्नये॒ ऽग्नये॒ भ्राज॑स्वते पुरो॒डाश॑म् पुरो॒डाश॒म् भ्राज॑स्वते॒ ऽग्नये॒ ऽग्नये॒ भ्राज॑स्वते पुरो॒डाश᳚म् । \newline
23. भ्राज॑स्वते पुरो॒डाश॑म् पुरो॒डाश॒म् भ्राज॑स्वते॒ भ्राज॑स्वते पुरो॒डाश॑ म॒ष्टाक॑पाल म॒ष्टाक॑पालम् पुरो॒डाश॒म् भ्राज॑स्वते॒ भ्राज॑स्वते पुरो॒डाश॑ म॒ष्टाक॑पालम् । \newline
24. पु॒रो॒डाश॑ म॒ष्टाक॑पाल म॒ष्टाक॑पालम् पुरो॒डाश॑म् पुरो॒डाश॑ म॒ष्टाक॑पाल॒म् तया॒ तया॒ ऽष्टाक॑पालम् पुरो॒डाश॑म् पुरो॒डाश॑ म॒ष्टाक॑पाल॒म् तया᳚ । \newline
25. अ॒ष्टाक॑पाल॒म् तया॒ तया॒ ऽष्टाक॑पाल म॒ष्टाक॑पाल॒म् तयै॒वैव तया॒ ऽष्टाक॑पाल म॒ष्टाक॑पाल॒म् तयै॒व । \newline
26. अ॒ष्टाक॑पाल॒मित्य॒ष्टा - क॒पा॒ल॒म् । \newline
27. तयै॒वैव तया॒ तयै॒वास्मि॑न् नस्मिन् ने॒व तया॒ तयै॒वास्मिन्न्॑ । \newline
28. ए॒वास्मि॑न् नस्मिन् ने॒वैवास्मि॒न् चक्षु॒ श्चक्षु॑ रस्मिन् ने॒वैवास्मि॒न् चक्षुः॑ । \newline
29. अ॒स्मि॒न् चक्षु॒ श्चक्षु॑ रस्मिन् नस्मि॒न् चक्षु॑ रदधा ददधा॒च् चक्षु॑ रस्मिन् नस्मि॒न् चक्षु॑ रदधात् । \newline
30. चक्षु॑ रदधा ददधा॒च् चक्षु॒ श्चक्षु॑ रदधा॒द् यो यो॑ ऽदधा॒च् चक्षु॒ श्चक्षु॑ रदधा॒द् यः । \newline
31. अ॒द॒धा॒द् यो यो॑ ऽदधा ददधा॒द् यश्चक्षु॑ष्काम॒ श्चक्षु॑ष्कामो॒ यो॑ ऽदधा ददधा॒द् यश्चक्षु॑ष्कामः । \newline
32. यश्चक्षु॑ष्काम॒ श्चक्षु॑ष्कामो॒ यो यश्चक्षु॑ष्कामः॒ स्याथ् स्याच् चक्षु॑ष्कामो॒ यो यश्चक्षु॑ष्कामः॒ स्यात् । \newline
33. चक्षु॑ष्कामः॒ स्याथ् स्याच् चक्षु॑ष्काम॒ श्चक्षु॑ष्कामः॒ स्यात् तस्मै॒ तस्मै॒ स्याच् चक्षु॑ष्काम॒ श्चक्षु॑ष्कामः॒ स्यात् तस्मै᳚ । \newline
34. चक्षु॑ष्काम॒ इति॒ चक्षुः॑ - का॒मः॒ । \newline
35. स्यात् तस्मै॒ तस्मै॒ स्याथ् स्यात् तस्मा॑ ए॒ता मे॒ताम् तस्मै॒ स्याथ् स्यात् तस्मा॑ ए॒ताम् । \newline
36. तस्मा॑ ए॒ता मे॒ताम् तस्मै॒ तस्मा॑ ए॒ता मिष्टि॒ मिष्टि॑ मे॒ताम् तस्मै॒ तस्मा॑ ए॒ता मिष्टि᳚म् । \newline
37. ए॒ता मिष्टि॒ मिष्टि॑ मे॒ता मे॒ता मिष्टि॒म् निर् णिरिष्टि॑ मे॒ता मे॒ता मिष्टि॒म् निः । \newline
38. इष्टि॒म् निर् णिरिष्टि॒ मिष्टि॒म् निर् व॑पेद् वपे॒न् निरिष्टि॒ मिष्टि॒म् निर् व॑पेत् । \newline
39. निर् व॑पेद् वपे॒न् निर् णिर् व॑पे द॒ग्नये॒ ऽग्नये॑ वपे॒न् निर् णिर् व॑पे द॒ग्नये᳚ । \newline
40. व॒पे॒ द॒ग्नये॒ ऽग्नये॑ वपेद् वपे द॒ग्नये॒ भ्राज॑स्वते॒ भ्राज॑स्वते॒ ऽग्नये॑ वपेद् वपे द॒ग्नये॒ भ्राज॑स्वते । \newline
41. अ॒ग्नये॒ भ्राज॑स्वते॒ भ्राज॑स्वते॒ ऽग्नये॒ ऽग्नये॒ भ्राज॑स्वते पुरो॒डाश॑म् पुरो॒डाश॒म् भ्राज॑स्वते॒ ऽग्नये॒ ऽग्नये॒ भ्राज॑स्वते पुरो॒डाश᳚म् । \newline
42. भ्राज॑स्वते पुरो॒डाश॑म् पुरो॒डाश॒म् भ्राज॑स्वते॒ भ्राज॑स्वते पुरो॒डाश॑ म॒ष्टाक॑पाल म॒ष्टाक॑पालम् पुरो॒डाश॒म् भ्राज॑स्वते॒ भ्राज॑स्वते पुरो॒डाश॑ म॒ष्टाक॑पालम् । \newline
43. पु॒रो॒डाश॑ म॒ष्टाक॑पाल म॒ष्टाक॑पालम् पुरो॒डाश॑म् पुरो॒डाश॑ म॒ष्टाक॑पालꣳ सौ॒र्यꣳ सौ॒र्य म॒ष्टाक॑पालम् पुरो॒डाश॑म् पुरो॒डाश॑ म॒ष्टाक॑पालꣳ सौ॒र्यम् । \newline
44. अ॒ष्टाक॑पालꣳ सौ॒र्यꣳ सौ॒र्य म॒ष्टाक॑पाल म॒ष्टाक॑पालꣳ सौ॒र्यम् च॒रुम् च॒रुꣳ सौ॒र्य म॒ष्टाक॑पाल म॒ष्टाक॑पालꣳ सौ॒र्यम् च॒रुम् । \newline
45. अ॒ष्टाक॑पाल॒मित्य॒ष्टा - क॒पा॒ल॒म् । \newline
46. सौ॒र्यम् च॒रुम् च॒रुꣳ सौ॒र्यꣳ सौ॒र्यम् च॒रु म॒ग्नये॒ ऽग्नये॑ च॒रुꣳ सौ॒र्यꣳ सौ॒र्यम् च॒रु म॒ग्नये᳚ । \newline
47. च॒रु म॒ग्नये॒ ऽग्नये॑ च॒रुम् च॒रु म॒ग्नये॒ भ्राज॑स्वते॒ भ्राज॑स्वते॒ ऽग्नये॑ च॒रुम् च॒रु म॒ग्नये॒ भ्राज॑स्वते । \newline
48. अ॒ग्नये॒ भ्राज॑स्वते॒ भ्राज॑स्वते॒ ऽग्नये॒ ऽग्नये॒ भ्राज॑स्वते पुरो॒डाश॑म् पुरो॒डाश॒म् भ्राज॑स्वते॒ ऽग्नये॒ ऽग्नये॒ भ्राज॑स्वते पुरो॒डाश᳚म् । \newline
49. भ्राज॑स्वते पुरो॒डाश॑म् पुरो॒डाश॒म् भ्राज॑स्वते॒ भ्राज॑स्वते पुरो॒डाश॑ म॒ष्टाक॑पाल म॒ष्टाक॑पालम् पुरो॒डाश॒म् भ्राज॑स्वते॒ भ्राज॑स्वते पुरो॒डाश॑ म॒ष्टाक॑पालम् । \newline
50. पु॒रो॒डाश॑ म॒ष्टाक॑पाल म॒ष्टाक॑पालम् पुरो॒डाश॑म् पुरो॒डाश॑ म॒ष्टाक॑पाल म॒ग्ने र॒ग्ने र॒ष्टाक॑पालम् पुरो॒डाश॑म् पुरो॒डाश॑ म॒ष्टाक॑पाल म॒ग्नेः । \newline
51. अ॒ष्टाक॑पाल म॒ग्ने र॒ग्ने र॒ष्टाक॑पाल म॒ष्टाक॑पाल म॒ग्नेर् वै वा अ॒ग्ने र॒ष्टाक॑पाल म॒ष्टाक॑पाल म॒ग्नेर् वै । \newline
52. अ॒ष्टाक॑पाल॒मित्य॒ष्टा - क॒पा॒ल॒म् । \newline
53. अ॒ग्नेर् वै वा अ॒ग्ने र॒ग्नेर् वै चक्षु॑षा॒ चक्षु॑षा॒ वा अ॒ग्ने र॒ग्नेर् वै चक्षु॑षा । \newline
54. वै चक्षु॑षा॒ चक्षु॑षा॒ वै वै चक्षु॑षा मनु॒ष्या॑ मनु॒ष्या᳚ श्चक्षु॑षा॒ वै वै चक्षु॑षा मनु॒ष्याः᳚ । \newline
55. चक्षु॑षा मनु॒ष्या॑ मनु॒ष्या᳚ श्चक्षु॑षा॒ चक्षु॑षा मनु॒ष्या॑ वि वि म॑नु॒ष्या᳚ श्चक्षु॑षा॒ चक्षु॑षा मनु॒ष्या॑ वि । \newline
56. म॒नु॒ष्या॑ वि वि म॑नु॒ष्या॑ मनु॒ष्या॑ वि प॑श्यन्ति पश्यन्ति॒ वि म॑नु॒ष्या॑ मनु॒ष्या॑ वि प॑श्यन्ति । \newline
57. वि प॑श्यन्ति पश्यन्ति॒ वि वि प॑श्यन्ति॒ सूर्य॑स्य॒ सूर्य॑स्य पश्यन्ति॒ वि वि प॑श्यन्ति॒ सूर्य॑स्य । \newline
\pagebreak
\markright{ TS 2.3.8.2  \hfill https://www.vedavms.in \hfill}

\section{ TS 2.3.8.2 }

\textbf{TS 2.3.8.2 } \newline
\textbf{Samhita Paata} \newline

प॑श्यन्ति॒ सूर्य॑स्य दे॒वा अ॒ग्निं चै॒व सूर्यं॑ च॒ स्वेन॑ भाग॒धेये॒नोप॑ धावति॒ तावे॒वास्मि॒न् चक्षु॑र्द्धत्त॒श्चक्षु॑ष्माने॒व भ॑वति॒ यदा᳚ग्ने॒यौ भव॑त॒श्चक्षु॑षी ए॒वास्मि॒न् तत् प्रति॑ दधाति॒ यथ् सौ॒र्यो नासि॑कां॒ तेना॒भितः॑ सौ॒र्यमा᳚ग्ने॒यौ भ॑वत॒स्तस्मा॑द॒भितो॒ नासि॑कां॒ चक्षु॑षी॒ तस्मा॒न्नासि॑कया॒ चक्षु॑षी॒ विधृ॑ते समा॒नी या᳚ज्यानुवा॒क्ये॑ भवतः समा॒नꣳ हि ( ) चक्षुः॒ समृ॑द्ध्या॒ उदु॒त्यं जा॒तवे॑दसꣳ स॒प्त त्वा॑ ह॒रितो॒ रथे॑ चि॒त्रं दे॒वाना॒मुद॑गा॒दनी॑क॒मिति॒ पिण्डा॒न् प्रय॑च्छति॒ चक्षु॑रे॒वास्मै॒ प्रय॑च्छति॒ यदे॒व तस्य॒ तत् ॥ \newline

\textbf{Pada Paata} \newline

प॒श्य॒न्ति॒ । सूर्य॑स्य । दे॒वाः । अ॒ग्निम् । च॒ । ए॒व । सूर्य᳚म् । च॒ । स्वेन॑ । भा॒ग॒धेये॒नेति॑ भाग - धेये॑न । उपेति॑ । धा॒व॒ति॒ । तौ । ए॒व । अ॒स्मि॒न्न् । चक्षुः॑ । ध॒त्तः॒ । चक्षु॑ष्मान् । ए॒व । भ॒व॒ति॒ । यत् । आ॒ग्ने॒यौ । भव॑तः । चक्षु॑षी॒ इति॑ । ए॒व । अ॒स्मि॒न्न् । तत् । प्रतीति॑ । द॒धा॒ति॒ । यत् । सौ॒र्यः । नासि॑काम् । तेन॑ । अ॒भितः॑ । सौ॒र्यम् । आ॒ग्ने॒यौ । भ॒व॒तः॒ । तस्मा᳚त् । अ॒भितः॑ । नासि॑काम् । चक्षु॑षी॒ इति॑ । तस्मा᳚त् । नासि॑कया । चक्षु॑षी॒ इति॑ । विधृ॑ते॒ इति॒ वि - धृ॒ते॒ । स॒मा॒नी इति॑ । या॒ज्या॒नु॒वा॒क्ये॑ इति॑ याज्या - अ॒नु॒वा॒क्ये᳚ । भ॒व॒तः॒ । स॒मा॒नम् । हि ( ) । चक्षुः॑ । समृ॑द्ध्या॒ इति॒ सं-ऋ॒द्ध्यै॒ । उदिति॑ ।   उ॒ । त्यम् । जा॒तवे॑दस॒मिति॑ जा॒त - वे॒द॒स॒म् । स॒प्त । त्वा॒ । ह॒रितः॑ । रथे᳚ । चि॒त्रम् । दे॒वाना᳚म् । उदिति॑ । अ॒गा॒त् । अनी॑कम् । इति॑ । पिण्डान्॑ । प्रेति॑ । य॒च्छ॒ति॒ । चक्षुः॑ । ए॒व । अ॒स्मै॒ । प्रेति॑ । य॒च्छ॒ति॒ । यत् । ए॒व । तस्य॑ । तत् ॥  \newline


\textbf{Krama Paata} \newline

प॒श्य॒न्ति॒ सूर्य॑स्य । सूर्य॑स्य दे॒वाः । दे॒वा अ॒ग्निम् । अ॒ग्निम् च॑ । चै॒व । ए॒व सूर्य᳚म् । सूर्य॑म् च । च॒ स्वेन॑ । स्वेन॑ भाग॒धेये॑न । भा॒ग॒धेये॒नोप॑ । भा॒ग॒धेये॒नेति॑ भाग - धेये॑न । उप॑ धावति । धा॒व॒ति॒ तौ । तावे॒व । ए॒वास्मिन्न्॑ । अ॒स्मि॒न् चक्षुः॑ । चक्षु॑र् धत्तः । ध॒त्त॒ श्चक्षु॑ष्मान् । चक्षु॑ष्माने॒व । ए॒व भ॑वति । भ॒व॒ति॒ यत् । यदा᳚ग्ने॒यौ । आ॒ग्ने॒यौ भव॑तः । भव॑त॒श्चक्षु॑षी । चक्षु॑षी ए॒व । चक्षु॑षी॒ इति॒ चक्षु॑षी । ए॒वास्मिन्न्॑ । अ॒स्मि॒न् तत् । तत् प्रति॑ । प्रति॑ दधाति । द॒धा॒ति॒ यत् । यथ् सौ॒र्यः । सौ॒र्यो नासि॑काम् । नासि॑का॒म् तेन॑ । तेना॒भितः॑ । अ॒भितः॑ सौ॒र्यम् । सौ॒र्यमा᳚ग्ने॒यौ । आ॒ग्ने॒यौ भ॑वतः । भ॒व॒त॒स्तस्मा᳚त् । तस्मा॑द॒भितः॑ । अ॒भितो॒ नासि॑काम् । नासि॑का॒म् चक्षु॑षी । चक्षु॑षी॒ तस्मा᳚त् । चक्षु॑षी॒ इति॒ चक्षु॑षी । तस्मा॒न्नासि॑कया । नासि॑कया॒ चक्षु॑षी । चक्षु॑षी॒ विधृ॑ते । चक्षु॑षी॒ इति॒ चक्षु॑षी । विधृ॑ते समा॒नी । विधृ॑ते॒ इति॒ वि - धृ॒ते॒ । स॒मा॒नी या᳚ज्यानुवा॒क्ये᳚ । स॒मा॒नी इति॑ समा॒नी । या॒ज्या॒नु॒वा॒क्ये॑ भवतः । या॒ज्या॒नु॒वा॒क्ये॑ इति॑ याज्या - अ॒नु॒वा॒क्ये᳚ । भ॒व॒तः॒ स॒मा॒नम् । स॒मा॒नꣳ हि ( ) । हि चक्षुः॑ । चक्षुः॒ समृ॑द्ध्यै । समृ॑द्ध्या॒ उत् । समृ॑द्ध्या॒ इति॒ सं - ऋ॒द्ध्यै॒ । उदु॑ । उ॒ त्यम् । त्यम् जा॒तवे॑दसम् । जा॒तवे॑दसꣳ स॒प्त । जा॒तवे॑दस॒मिति॑ जा॒त - वे॒द॒स॒म् । स॒प्त त्वा᳚ । त्वा॒ ह॒रितः॑ । ह॒रितो॒ रथे᳚ । रथे॑ चि॒त्रम् । चि॒त्रम् दे॒वाना᳚म् । दे॒वाना॒मुत् । उद॑गात् । अ॒गा॒दनी॑कम् । अनी॑क॒मिति॑ । इति॒ पिण्डान्॑ । पिण्डा॒न् प्र । प्र य॑च्छति । य॒च्छ॒ति॒ चक्षुः॑ । चक्षु॑रे॒व । ए॒वास्मै᳚ । अ॒स्मै॒ प्र । प्र य॑च्छति । य॒च्छ॒ति॒ यत् । यदे॒व । ए॒व तस्य॑ । तस्य॒ तत् । तदिति॒ तत् । \newline

\textbf{Jatai Paata} \newline

1. प॒श्य॒न्ति॒ सूर्य॑स्य॒ सूर्य॑स्य पश्यन्ति पश्यन्ति॒ सूर्य॑स्य । \newline
2. सूर्य॑स्य दे॒वा दे॒वाः सूर्य॑स्य॒ सूर्य॑स्य दे॒वाः । \newline
3. दे॒वा अ॒ग्नि म॒ग्निम् दे॒वा दे॒वा अ॒ग्निम् । \newline
4. अ॒ग्निम् च॑ चा॒ग्नि म॒ग्निम् च॑ । \newline
5. चै॒वैव च॑ चै॒व । \newline
6. ए॒व सूर्यꣳ॒॒ सूर्य॑ मे॒वैव सूर्य᳚म् । \newline
7. सूर्य॑म् च च॒ सूर्यꣳ॒॒ सूर्य॑म् च । \newline
8. च॒ स्वेन॒ स्वेन॑ च च॒ स्वेन॑ । \newline
9. स्वेन॑ भाग॒धेये॑न भाग॒धेये॑न॒ स्वेन॒ स्वेन॑ भाग॒धेये॑न । \newline
10. भा॒ग॒धेये॒नोपोप॑ भाग॒धेये॑न भाग॒धेये॒नोप॑ । \newline
11. भा॒ग॒धेये॒नेति॑ भाग - धेये॑न । \newline
12. उप॑ धावति धाव॒ त्युपोप॑ धावति । \newline
13. धा॒व॒ति॒ तौ तौ धा॑वति धावति॒ तौ । \newline
14. ता वे॒वैव तौ ता वे॒व । \newline
15. ए॒वास्मि॑न् नस्मिन् ने॒वैवास्मिन्न्॑ । \newline
16. अ॒स्मि॒न् चक्षु॒ श्चक्षु॑रस्मिन् नस्मि॒न् चक्षुः॑ । \newline
17. चक्षु॑र् धत्तो धत्त॒ श्चक्षु॒ श्चक्षु॑र् धत्तः । \newline
18. ध॒त्त॒ श्चक्षु॑ष्माꣳ॒॒ श्चक्षु॑ष्मान् धत्तो धत्त॒श्चक्षु॑ष्मान् । \newline
19. चक्षु॑ष्मा ने॒वैव चक्षु॑ष्माꣳ॒॒ श्चक्षु॑ष्मा ने॒व । \newline
20. ए॒व भ॑वति भव त्ये॒वैव भ॑वति । \newline
21. भ॒व॒ति॒ यद् यद् भ॑वति भवति॒ यत् । \newline
22. यदा᳚ग्ने॒या वा᳚ग्ने॒यौ यद् यदा᳚ग्ने॒यौ । \newline
23. आ॒ग्ने॒यौ भव॑तो॒ भव॑त आग्ने॒या वा᳚ग्ने॒यौ भव॑तः । \newline
24. भव॑त॒ श्चक्षु॑षी॒ चक्षु॑षी॒ भव॑तो॒ भव॑त॒ श्चक्षु॑षी । \newline
25. चक्षु॑षी ए॒वैव चक्षु॑षी॒ चक्षु॑षी ए॒व । \newline
26. चक्षु॑षी॒ इति॒ चक्षु॑षी । \newline
27. ए॒वास्मि॑न् नस्मिन् ने॒वैवास्मिन्न्॑ । \newline
28. अ॒स्मि॒न् तत् तद॑स्मिन् नस्मि॒न् तत् । \newline
29. तत् प्रति॒ प्रति॒ तत् तत् प्रति॑ । \newline
30. प्रति॑ दधाति दधाति॒ प्रति॒ प्रति॑ दधाति । \newline
31. द॒धा॒ति॒ यद् यद् द॑धाति दधाति॒ यत् । \newline
32. यथ् सौ॒र्यः सौ॒र्यो यद् यथ् सौ॒र्यः । \newline
33. सौ॒र्यो नासि॑का॒म् नासि॑काꣳ सौ॒र्यः सौ॒र्यो नासि॑काम् । \newline
34. नासि॑का॒म् तेन॒ तेन॒ नासि॑का॒म् नासि॑का॒म् तेन॑ । \newline
35. तेना॒भितो॒ ऽभित॒ स्तेन॒ तेना॒भितः॑ । \newline
36. अ॒भितः॑ सौ॒र्यꣳ सौ॒र्य म॒भितो॒ ऽभितः॑ सौ॒र्यम् । \newline
37. सौ॒र्य मा᳚ग्ने॒या वा᳚ग्ने॒यौ सौ॒र्यꣳ सौ॒र्य मा᳚ग्ने॒यौ । \newline
38. आ॒ग्ने॒यौ भ॑वतो भवत आग्ने॒या वा᳚ग्ने॒यौ भ॑वतः । \newline
39. भ॒व॒त॒ स्तस्मा॒त् तस्मा᳚द् भवतो भवत॒ स्तस्मा᳚त् । \newline
40. तस्मा॑ द॒भितो॒ ऽभित॒ स्तस्मा॒त् तस्मा॑ द॒भितः॑ । \newline
41. अ॒भितो॒ नासि॑का॒म् नासि॑का म॒भितो॒ ऽभितो॒ नासि॑काम् । \newline
42. नासि॑का॒म् चक्षु॑षी॒ चक्षु॑षी॒ नासि॑का॒म् नासि॑का॒म् चक्षु॑षी । \newline
43. चक्षु॑षी॒ तस्मा॒त् तस्मा॒च् चक्षु॑षी॒ चक्षु॑षी॒ तस्मा᳚त् । \newline
44. चक्षु॑षी॒ इति॒ चक्षु॑षी । \newline
45. तस्मा॒न् नासि॑कया॒ नासि॑कया॒ तस्मा॒त् तस्मा॒न् नासि॑कया । \newline
46. नासि॑कया॒ चक्षु॑षी॒ चक्षु॑षी॒ नासि॑कया॒ नासि॑कया॒ चक्षु॑षी । \newline
47. चक्षु॑षी॒ विधृ॑ते॒ विधृ॑ते॒ चक्षु॑षी॒ चक्षु॑षी॒ विधृ॑ते । \newline
48. चक्षु॑षी॒ इति॒ चक्षु॑षी । \newline
49. विधृ॑ते समा॒नी स॑मा॒नी विधृ॑ते॒ विधृ॑ते समा॒नी । \newline
50. विधृ॑ते॒ इति॒ वि - धृ॒ते॒ । \newline
51. स॒मा॒नी या᳚ज्यानुवा॒क्ये॑ याज्यानुवा॒क्ये॑ समा॒नी स॑मा॒नी या᳚ज्यानुवा॒क्ये᳚ । \newline
52. स॒मा॒नी इति॑ समा॒नी । \newline
53. या॒ज्या॒नु॒वा॒क्ये॑ भवतो भवतो याज्यानुवा॒क्ये॑ याज्यानुवा॒क्ये॑ भवतः । \newline
54. या॒ज्या॒नु॒वा॒क्ये॑ इति॑ याज्या - अ॒नु॒वा॒क्ये᳚ । \newline
55. भ॒व॒तः॒ स॒मा॒नꣳ स॑मा॒नम् भ॑वतो भवतः समा॒नम् । \newline
56. स॒मा॒नꣳ हि हि स॑मा॒नꣳ स॑मा॒नꣳ हि । \newline
57. हि चक्षु॒ श्चक्षु॒र्॒. हि हि चक्षुः॑ । \newline
58. चक्षुः॒ समृ॑द्ध्यै॒ समृ॑द्ध्यै॒ चक्षु॒ श्चक्षुः॒ समृ॑द्ध्यै । \newline
59. समृ॑द्ध्या॒ उदुथ् समृ॑द्ध्यै॒ समृ॑द्ध्या॒ उत् । \newline
60. समृ॑द्ध्या॒ इति॒ सं - ऋ॒द्ध्यै॒ । \newline
61. उदु॑ वु॒ वु दुदु॑ । \newline
62. उ॒ त्यम् त्य मु॑ वु॒ त्यम् । \newline
63. त्यम् जा॒तवे॑दसम् जा॒तवे॑दस॒म् त्यम् त्यम् जा॒तवे॑दसम् । \newline
64. जा॒तवे॑दसꣳ स॒प्त स॒प्त जा॒तवे॑दसम् जा॒तवे॑दसꣳ स॒प्त । \newline
65. जा॒तवे॑दस॒मिति॑ जा॒त - वे॒द॒स॒म् । \newline
66. स॒प्त त्वा᳚ त्वा स॒प्त स॒प्त त्वा᳚ । \newline
67. त्वा॒ ह॒रितो॑ ह॒रित॑ स्त्वा त्वा ह॒रितः॑ । \newline
68. ह॒रितो॒ रथे॒ रथे॑ ह॒रितो॑ ह॒रितो॒ रथे᳚ । \newline
69. रथे॑ चि॒त्रम् चि॒त्रꣳ रथे॒ रथे॑ चि॒त्रम् । \newline
70. चि॒त्रम् दे॒वाना᳚म् दे॒वाना᳚म् चि॒त्रम् चि॒त्रम् दे॒वाना᳚म् । \newline
71. दे॒वाना॒ मुदुद् दे॒वाना᳚म् दे॒वाना॒ मुत् । \newline
72. उद॑गा दगा॒ दुदु द॑गात् । \newline
73. अ॒गा॒ दनी॑क॒ मनी॑क मगा दगा॒ दनी॑कम् । \newline
74. अनी॑क॒ मिती त्यनी॑क॒ मनी॑क॒ मिति॑ । \newline
75. इति॒ पिण्डा॒न् पिण्डा॒ नितीति॒ पिण्डान्॑ । \newline
76. पिण्डा॒न् प्र प्र पिण्डा॒न् पिण्डा॒न् प्र । \newline
77. प्र य॑च्छति यच्छति॒ प्र प्र य॑च्छति । \newline
78. य॒च्छ॒ति॒ चक्षु॒ श्चक्षु॑र् यच्छति यच्छति॒ चक्षुः॑ । \newline
79. चक्षु॑ रे॒वैव चक्षु॒ श्चक्षु॑ रे॒व । \newline
80. ए॒वास्मा॑ अस्मा ए॒वैवास्मै᳚ । \newline
81. अ॒स्मै॒ प्र प्रास्मा॑ अस्मै॒ प्र । \newline
82. प्र य॑च्छति यच्छति॒ प्र प्र य॑च्छति । \newline
83. य॒च्छ॒ति॒ यद् यद् य॑च्छति यच्छति॒ यत् । \newline
84. यदे॒वैव यद् यदे॒व । \newline
85. ए॒व तस्य॒ तस्यै॒वैव तस्य॑ । \newline
86. तस्य॒ तत् तत् तस्य॒ तस्य॒ तत् । \newline
87. तदिति॒ तत् । \newline

\textbf{Ghana Paata } \newline

1. प॒श्य॒न्ति॒ सूर्य॑स्य॒ सूर्य॑स्य पश्यन्ति पश्यन्ति॒ सूर्य॑स्य दे॒वा दे॒वाः सूर्य॑स्य पश्यन्ति पश्यन्ति॒ सूर्य॑स्य दे॒वाः । \newline
2. सूर्य॑स्य दे॒वा दे॒वाः सूर्य॑स्य॒ सूर्य॑स्य दे॒वा अ॒ग्नि म॒ग्निम् दे॒वाः सूर्य॑स्य॒ सूर्य॑स्य दे॒वा अ॒ग्निम् । \newline
3. दे॒वा अ॒ग्नि म॒ग्निम् दे॒वा दे॒वा अ॒ग्निम् च॑ चा॒ग्निम् दे॒वा दे॒वा अ॒ग्निम् च॑ । \newline
4. अ॒ग्निम् च॑ चा॒ग्नि म॒ग्निम् चै॒वैव चा॒ग्नि म॒ग्निम् चै॒व । \newline
5. चै॒वैव च॑ चै॒व सूर्यꣳ॒॒ सूर्य॑ मे॒व च॑ चै॒व सूर्य᳚म् । \newline
6. ए॒व सूर्यꣳ॒॒ सूर्य॑ मे॒वैव सूर्य॑म् च च॒ सूर्य॑ मे॒वैव सूर्य॑म् च । \newline
7. सूर्य॑म् च च॒ सूर्यꣳ॒॒ सूर्य॑म् च॒ स्वेन॒ स्वेन॑ च॒ सूर्यꣳ॒॒ सूर्य॑म् च॒ स्वेन॑ । \newline
8. च॒ स्वेन॒ स्वेन॑ च च॒ स्वेन॑ भाग॒धेये॑न भाग॒धेये॑न॒ स्वेन॑ च च॒ स्वेन॑ भाग॒धेये॑न । \newline
9. स्वेन॑ भाग॒धेये॑न भाग॒धेये॑न॒ स्वेन॒ स्वेन॑ भाग॒धेये॒नोपोप॑ भाग॒धेये॑न॒ स्वेन॒ स्वेन॑ भाग॒धेये॒नोप॑ । \newline
10. भा॒ग॒धेये॒नोपोप॑ भाग॒धेये॑न भाग॒धेये॒नोप॑ धावति धाव॒त्युप॑ भाग॒धेये॑न भाग॒धेये॒नोप॑ धावति । \newline
11. भा॒ग॒धेये॒नेति॑ भाग - धेये॑न । \newline
12. उप॑ धावति धाव॒ त्युपोप॑ धावति॒ तौ तौ धा॑व॒ त्युपोप॑ धावति॒ तौ । \newline
13. धा॒व॒ति॒ तौ तौ धा॑वति धावति॒ ता वे॒वैव तौ धा॑वति धावति॒ ता वे॒व । \newline
14. ता वे॒वैव तौ ता वे॒वास्मि॑न् नस्मिन् ने॒व तौ ता वे॒वास्मिन्न्॑ । \newline
15. ए॒वास्मि॑न् नस्मिन् ने॒वैवास्मि॒न् चक्षु॒ श्चक्षु॑रस्मिन् ने॒वैवास्मि॒न् चक्षुः॑ । \newline
16. अ॒स्मि॒न् चक्षु॒ श्चक्षु॑रस्मिन् नस्मि॒न् चक्षु॑र् धत्तो धत्त॒ श्चक्षु॑रस्मिन् नस्मि॒न् चक्षु॑र् धत्तः । \newline
17. चक्षु॑र् धत्तो धत्त॒ श्चक्षु॒ श्चक्षु॑र् धत्त॒ श्चक्षु॑ष्माꣳ॒॒ श्चक्षु॑ष्मान् धत्त॒ श्चक्षु॒ श्चक्षु॑र् धत्त॒ श्चक्षु॑ष्मान् । \newline
18. ध॒त्त॒ श्चक्षु॑ष्माꣳ॒॒ श्चक्षु॑ष्मान् धत्तो धत्त॒श्चक्षु॑ष्मा ने॒वैव चक्षु॑ष्मान् धत्तो धत्त॒ श्चक्षु॑ष्मा ने॒व । \newline
19. चक्षु॑ष्मा ने॒वैव चक्षु॑ष्माꣳ॒॒ श्चक्षु॑ष्मा ने॒व भ॑वति भवत्ये॒व चक्षु॑ष्माꣳ॒॒ श्चक्षु॑ष्मा ने॒व भ॑वति । \newline
20. ए॒व भ॑वति भव त्ये॒वैव भ॑वति॒ यद् यद् भ॑व त्ये॒वैव भ॑वति॒ यत् । \newline
21. भ॒व॒ति॒ यद् यद् भ॑वति भवति॒ यदा᳚ग्ने॒या वा᳚ग्ने॒यौ यद् भ॑वति भवति॒ यदा᳚ग्ने॒यौ । \newline
22. यदा᳚ग्ने॒या वा᳚ग्ने॒यौ यद् यदा᳚ग्ने॒यौ भव॑तो॒ भव॑त आग्ने॒यौ यद् यदा᳚ग्ने॒यौ भव॑तः । \newline
23. आ॒ग्ने॒यौ भव॑तो॒ भव॑त आग्ने॒या वा᳚ग्ने॒यौ भव॑त॒ श्चक्षु॑षी॒ चक्षु॑षी॒ भव॑त आग्ने॒या वा᳚ग्ने॒यौ भव॑त॒ श्चक्षु॑षी । \newline
24. भव॑त॒ श्चक्षु॑षी॒ चक्षु॑षी॒ भव॑तो॒ भव॑त॒ श्चक्षु॑षी ए॒वैव चक्षु॑षी॒ भव॑तो॒ भव॑त॒ श्चक्षु॑षी ए॒व । \newline
25. चक्षु॑षी ए॒वैव चक्षु॑षी॒ चक्षु॑षी ए॒वास्मि॑न् नस्मिन् ने॒व चक्षु॑षी॒ चक्षु॑षी ए॒वास्मिन्न्॑ । \newline
26. चक्षु॑षी॒ इति॒ चक्षु॑षी । \newline
27. ए॒वास्मि॑न् नस्मिन् ने॒वैवास्मि॒न् तत् तद॑स्मिन् ने॒वैवास्मि॒न् तत् । \newline
28. अ॒स्मि॒न् तत् तद॑स्मिन् नस्मि॒न् तत् प्रति॒ प्रति॒ तद॑स्मिन् नस्मि॒न् तत् प्रति॑ । \newline
29. तत् प्रति॒ प्रति॒ तत् तत् प्रति॑ दधाति दधाति॒ प्रति॒ तत् तत् प्रति॑ दधाति । \newline
30. प्रति॑ दधाति दधाति॒ प्रति॒ प्रति॑ दधाति॒ यद् यद् द॑धाति॒ प्रति॒ प्रति॑ दधाति॒ यत् । \newline
31. द॒धा॒ति॒ यद् यद् द॑धाति दधाति॒ यथ् सौ॒र्यः सौ॒र्यो यद् द॑धाति दधाति॒ यथ् सौ॒र्यः । \newline
32. यथ् सौ॒र्यः सौ॒र्यो यद् यथ् सौ॒र्यो नासि॑का॒म् नासि॑काꣳ सौ॒र्यो यद् यथ् सौ॒र्यो नासि॑काम् । \newline
33. सौ॒र्यो नासि॑का॒म् नासि॑काꣳ सौ॒र्यः सौ॒र्यो नासि॑का॒म् तेन॒ तेन॒ नासि॑काꣳ सौ॒र्यः सौ॒र्यो नासि॑का॒म् तेन॑ । \newline
34. नासि॑का॒म् तेन॒ तेन॒ नासि॑का॒म् नासि॑का॒म् तेना॒भितो॒ ऽभित॒ स्तेन॒ नासि॑का॒म् नासि॑का॒म् तेना॒भितः॑ । \newline
35. तेना॒भितो॒ ऽभित॒ स्तेन॒ तेना॒भितः॑ सौ॒र्यꣳ सौ॒र्य म॒भित॒ स्तेन॒ तेना॒भितः॑ सौ॒र्यम् । \newline
36. अ॒भितः॑ सौ॒र्यꣳ सौ॒र्य म॒भितो॒ ऽभितः॑ सौ॒र्य मा᳚ग्ने॒या वा᳚ग्ने॒यौ सौ॒र्य म॒भितो॒ ऽभितः॑ सौ॒र्य मा᳚ग्ने॒यौ । \newline
37. सौ॒र्य मा᳚ग्ने॒या वा᳚ग्ने॒यौ सौ॒र्यꣳ सौ॒र्य मा᳚ग्ने॒यौ भ॑वतो भवत आग्ने॒यौ सौ॒र्यꣳ सौ॒र्य मा᳚ग्ने॒यौ भ॑वतः । \newline
38. आ॒ग्ने॒यौ भ॑वतो भवत आग्ने॒या वा᳚ग्ने॒यौ भ॑वत॒ स्तस्मा॒त् तस्मा᳚द् भवत आग्ने॒या वा᳚ग्ने॒यौ भ॑वत॒ स्तस्मा᳚त् । \newline
39. भ॒व॒त॒ स्तस्मा॒त् तस्मा᳚द् भवतो भवत॒ स्तस्मा॑ द॒भितो॒ ऽभित॒ स्तस्मा᳚द् भवतो भवत॒ स्तस्मा॑ द॒भितः॑ । \newline
40. तस्मा॑ द॒भितो॒ ऽभित॒ स्तस्मा॒त् तस्मा॑ द॒भितो॒ नासि॑का॒म् नासि॑का म॒भित॒ स्तस्मा॒त् तस्मा॑ द॒भितो॒ नासि॑काम् । \newline
41. अ॒भितो॒ नासि॑का॒म् नासि॑का म॒भितो॒ ऽभितो॒ नासि॑का॒म् चक्षु॑षी॒ चक्षु॑षी॒ नासि॑का म॒भितो॒ ऽभितो॒ नासि॑का॒म् चक्षु॑षी । \newline
42. नासि॑का॒म् चक्षु॑षी॒ चक्षु॑षी॒ नासि॑का॒म् नासि॑का॒म् चक्षु॑षी॒ तस्मा॒त् तस्मा॒च् चक्षु॑षी॒ नासि॑का॒म् नासि॑का॒म् चक्षु॑षी॒ तस्मा᳚त् । \newline
43. चक्षु॑षी॒ तस्मा॒त् तस्मा॒च् चक्षु॑षी॒ चक्षु॑षी॒ तस्मा॒न् नासि॑कया॒ नासि॑कया॒ तस्मा॒च् चक्षु॑षी॒ चक्षु॑षी॒ तस्मा॒न् नासि॑कया । \newline
44. चक्षु॑षी॒ इति॒ चक्षु॑षी । \newline
45. तस्मा॒न् नासि॑कया॒ नासि॑कया॒ तस्मा॒त् तस्मा॒न् नासि॑कया॒ चक्षु॑षी॒ चक्षु॑षी॒ नासि॑कया॒ तस्मा॒त् तस्मा॒न् नासि॑कया॒ चक्षु॑षी । \newline
46. नासि॑कया॒ चक्षु॑षी॒ चक्षु॑षी॒ नासि॑कया॒ नासि॑कया॒ चक्षु॑षी॒ विधृ॑ते॒ विधृ॑ते॒ चक्षु॑षी॒ नासि॑कया॒ नासि॑कया॒ चक्षु॑षी॒ विधृ॑ते । \newline
47. चक्षु॑षी॒ विधृ॑ते॒ विधृ॑ते॒ चक्षु॑षी॒ चक्षु॑षी॒ विधृ॑ते समा॒नी स॑मा॒नी विधृ॑ते॒ चक्षु॑षी॒ चक्षु॑षी॒ विधृ॑ते समा॒नी । \newline
48. चक्षु॑षी॒ इति॒ चक्षु॑षी । \newline
49. विधृ॑ते समा॒नी स॑मा॒नी विधृ॑ते॒ विधृ॑ते समा॒नी या᳚ज्यानुवा॒क्ये॑ याज्यानुवा॒क्ये॑ समा॒नी विधृ॑ते॒ विधृ॑ते समा॒नी या᳚ज्यानुवा॒क्ये᳚ । \newline
50. विधृ॑ते॒ इति॒ वि - धृ॒ते॒ । \newline
51. स॒मा॒नी या᳚ज्यानुवा॒क्ये॑ याज्यानुवा॒क्ये॑ समा॒नी स॑मा॒नी या᳚ज्यानुवा॒क्ये॑ भवतो भवतो याज्यानुवा॒क्ये॑ समा॒नी स॑मा॒नी या᳚ज्यानुवा॒क्ये॑ भवतः । \newline
52. स॒मा॒नी इति॑ समा॒नी । \newline
53. या॒ज्या॒नु॒वा॒क्ये॑ भवतो भवतो याज्यानुवा॒क्ये॑ याज्यानुवा॒क्ये॑ भवतः समा॒नꣳ स॑मा॒नम् भ॑वतो याज्यानुवा॒क्ये॑ याज्यानुवा॒क्ये॑ भवतः समा॒नम् । \newline
54. या॒ज्या॒नु॒वा॒क्ये॑ इति॑ याज्या - अ॒नु॒वा॒क्ये᳚ । \newline
55. भ॒व॒तः॒ स॒मा॒नꣳ स॑मा॒नम् भ॑वतो भवतः समा॒नꣳ हि हि स॑मा॒नम् भ॑वतो भवतः समा॒नꣳ हि । \newline
56. स॒मा॒नꣳ हि हि स॑मा॒नꣳ स॑मा॒नꣳ हि चक्षु॒ श्चक्षु॒र्॒. हि स॑मा॒नꣳ स॑मा॒नꣳ हि चक्षुः॑ । \newline
57. हि चक्षु॒ श्चक्षु॒र्॒. हि हि चक्षुः॒ समृ॑द्ध्यै॒ समृ॑द्ध्यै॒ चक्षु॒र्॒. हि हि चक्षुः॒ समृ॑द्ध्यै । \newline
58. चक्षुः॒ समृ॑द्ध्यै॒ समृ॑द्ध्यै॒ चक्षु॒ श्चक्षुः॒ समृ॑द्ध्या॒ उदुथ् समृ॑द्ध्यै॒ चक्षु॒ श्चक्षुः॒ समृ॑द्ध्या॒ उत् । \newline
59. समृ॑द्ध्या॒ उदुथ् समृ॑द्ध्यै॒ समृ॑द्ध्या॒ उदु॑ वु॒ वुथ् समृ॑द्ध्यै॒ समृ॑द्ध्या॒ उदु॑ । \newline
60. समृ॑द्ध्या॒ इति॒ सं - ऋ॒द्ध्यै॒ । \newline
61. उदु॑ वु॒ वुदुदु॒ त्यम् त्य मु॒ वुदुदु॒ त्यम् । \newline
62. उ॒ त्यम् त्य मु॑ वु॒ त्यम् जा॒तवे॑दसम् जा॒तवे॑दस॒म् त्य मु॑ वु॒ त्यम् जा॒तवे॑दसम् । \newline
63. त्यम् जा॒तवे॑दसम् जा॒तवे॑दस॒म् त्यम् त्यम् जा॒तवे॑दसꣳ स॒प्त स॒प्त जा॒तवे॑दस॒म् त्यम् त्यम् जा॒तवे॑दसꣳ स॒प्त । \newline
64. जा॒तवे॑दसꣳ स॒प्त स॒प्त जा॒तवे॑दसम् जा॒तवे॑दसꣳ स॒प्त त्वा᳚ त्वा स॒प्त जा॒तवे॑दसम् जा॒तवे॑दसꣳ स॒प्त त्वा᳚ । \newline
65. जा॒तवे॑दस॒मिति॑ जा॒त - वे॒द॒स॒म् । \newline
66. स॒प्त त्वा᳚ त्वा स॒प्त स॒प्त त्वा॑ ह॒रितो॑ ह॒रित॑ स्त्वा स॒प्त स॒प्त त्वा॑ ह॒रितः॑ । \newline
67. त्वा॒ ह॒रितो॑ ह॒रित॑ स्त्वा त्वा ह॒रितो॒ रथे॒ रथे॑ ह॒रित॑ स्त्वा त्वा ह॒रितो॒ रथे᳚ । \newline
68. ह॒रितो॒ रथे॒ रथे॑ ह॒रितो॑ ह॒रितो॒ रथे॑ चि॒त्रम् चि॒त्रꣳ रथे॑ ह॒रितो॑ ह॒रितो॒ रथे॑ चि॒त्रम् । \newline
69. रथे॑ चि॒त्रम् चि॒त्रꣳ रथे॒ रथे॑ चि॒त्रम् दे॒वाना᳚म् दे॒वाना᳚म् चि॒त्रꣳ रथे॒ रथे॑ चि॒त्रम् दे॒वाना᳚म् । \newline
70. चि॒त्रम् दे॒वाना᳚म् दे॒वाना᳚म् चि॒त्रम् चि॒त्रम् दे॒वाना॒ मुदुद् दे॒वाना᳚म् चि॒त्रम् चि॒त्रम् दे॒वाना॒ मुत् । \newline
71. दे॒वाना॒ मुदुद् दे॒वाना᳚म् दे॒वाना॒ मुद॑गा दगा॒दुद् दे॒वाना᳚म् दे॒वाना॒ मुद॑गात् । \newline
72. उद॑गा दगा॒ दुदु द॑गा॒ दनी॑क॒ मनी॑क मगा॒दुदु द॑गा॒ दनी॑कम् । \newline
73. अ॒गा॒ दनी॑क॒ मनी॑क मगा दगा॒ दनी॑क॒ मिती त्यनी॑क मगा दगा॒ दनी॑क॒ मिति॑ । \newline
74. अनी॑क॒ मिती त्यनी॑क॒ मनी॑क॒ मिति॒ पिण्डा॒न् पिण्डा॒ नित्यनी॑क॒ मनी॑क॒ मिति॒ पिण्डान्॑ । \newline
75. इति॒ पिण्डा॒न् पिण्डा॒ नितीति॒ पिण्डा॒न् प्र प्र पिण्डा॒ नितीति॒ पिण्डा॒न् प्र । \newline
76. पिण्डा॒न् प्र प्र पिण्डा॒न् पिण्डा॒न् प्र य॑च्छति यच्छति॒ प्र पिण्डा॒न् पिण्डा॒न् प्र य॑च्छति । \newline
77. प्र य॑च्छति यच्छति॒ प्र प्र य॑च्छति॒ चक्षु॒ श्चक्षु॑र् यच्छति॒ प्र प्र य॑च्छति॒ चक्षुः॑ । \newline
78. य॒च्छ॒ति॒ चक्षु॒ श्चक्षु॑र् यच्छति यच्छति॒ चक्षु॑रे॒वैव चक्षु॑र् यच्छति यच्छति॒ चक्षु॑रे॒व । \newline
79. चक्षु॑ रे॒वैव चक्षु॒ श्चक्षु॑ रे॒वास्मा॑ अस्मा ए॒व चक्षु॒ श्चक्षु॑ रे॒वास्मै᳚ । \newline
80. ए॒वास्मा॑ अस्मा ए॒वैवास्मै॒ प्र प्रास्मा॑ ए॒वैवास्मै॒ प्र । \newline
81. अ॒स्मै॒ प्र प्रास्मा॑ अस्मै॒ प्र य॑च्छति यच्छति॒ प्रास्मा॑ अस्मै॒ प्र य॑च्छति । \newline
82. प्र य॑च्छति यच्छति॒ प्र प्र य॑च्छति॒ यद् यद् य॑च्छति॒ प्र प्र य॑च्छति॒ यत् । \newline
83. य॒च्छ॒ति॒ यद् यद् य॑च्छति यच्छति॒ यदे॒वैव यद् य॑च्छति यच्छति॒ यदे॒व । \newline
84. यदे॒वैव यद् यदे॒व तस्य॒ तस्यै॒व यद् यदे॒व तस्य॑ । \newline
85. ए॒व तस्य॒ तस्यै॒वैव तस्य॒ तत् तत् तस्यै॒वैव तस्य॒ तत् । \newline
86. तस्य॒ तत् तत् तस्य॒ तस्य॒ तत् । \newline
87. तदिति॒ तत् । \newline
\pagebreak
\markright{ TS 2.3.9.1  \hfill https://www.vedavms.in \hfill}

\section{ TS 2.3.9.1 }

\textbf{TS 2.3.9.1 } \newline
\textbf{Samhita Paata} \newline

ध्रु॒वो॑ऽसि ध्रु॒वो॑ऽहꣳ स॑जा॒तेषु॑ भूयासं॒ धीर॒श्चेत्ता॑ वसु॒विद् ध्रु॒वो॑ऽसि ध्रु॒वो॑ऽहꣳ स॑जा॒तेषु॑ भूयासमु॒ग्रश्चेत्ता॑ वसु॒विद् ध्रु॒वो॑ऽसि ध्रु॒वो॑ऽहꣳ स॑जा॒तेषु॑ भूयासमभि॒भूश्चेत्ता॑ वसु॒विदा म॑नम॒स्याम॑नस्य देवा॒ ये स॑जा॒ताः कु॑मा॒राः सम॑नस॒स्तान॒हं का॑मये हृ॒दा ते मां का॑मयन्ताꣳ हृ॒दा तान् म॒ आम॑नसः कृधि॒ स्वाहा ऽऽम॑नम॒स्या - [  ] \newline

\textbf{Pada Paata} \newline

ध्रु॒वः । अ॒सि॒ । ध्रु॒वः । अ॒हम् । स॒जा॒तेष्विति॑ स - जा॒तेषु॑ । भू॒या॒स॒म् । धीरः॑ । चेत्ता᳚ । व॒सु॒विदिति॑ वसु - वित् । ध्रु॒वः । अ॒सि॒ । ध्रु॒वः । अ॒हम् । स॒जा॒तेष्विति॑ स - जा॒तेषु॑ । भू॒या॒स॒म् । उ॒ग्रः । चेत्ता᳚ । व॒सु॒विदिति॑ वसु - वित् । ध्रु॒वः । अ॒सि॒ ।  ध्रु॒वः । अ॒हम् । स॒जा॒तेष्विति॑ स - जा॒तेषु॑ । भू॒या॒स॒म् । अ॒भि॒भूरित्य॑भि - भूः । चेत्ता᳚ । व॒सु॒विदिति॑ वसु - वित् । आम॑न॒मित्या - म॒न॒म् । अ॒सि॒ । आम॑न॒स्येत्या - म॒न॒स्य॒ । दे॒वाः॒ । ये ।  स॒जा॒ता इति॑ स - जा॒ताः । कु॒मा॒राः । सम॑नस॒ इति॒ स - म॒न॒सः॒ । तान् । अ॒हम् । का॒म॒ये॒ । हृ॒दा । ते । माम् । का॒म॒य॒न्ता॒म् । हृ॒दा । तान् । मे॒ । आम॑नस॒ इत्या - म॒न॒सः॒ । कृ॒धि॒ । स्वाहा᳚ । आम॑न॒मित्या - म॒न॒म् । अ॒सि॒ ।  \newline


\textbf{Krama Paata} \newline

ध्रु॒वो॑ऽसि । अ॒सि॒ ध्रु॒वः । ध्रु॒वो॑ऽहम् । अ॒हꣳ स॑जा॒तेषु॑ । स॒जा॒तेषु॑ भूयासम् । स॒जा॒तेष्विति॑ स - जा॒तेषु॑ । भू॒या॒स॒म् धीरः॑ । धीर॒ श्चेत्ता᳚ । चेत्ता॑ वसु॒वित् । व॒सु॒विद् ध्रु॒वः । व॒सु॒विदिति॑ वसु - वित् । ध्रु॒वो॑ऽसि । अ॒सि॒ ध्रु॒वः । ध्रु॒वो॑ऽहम् । अ॒हꣳ स॑जा॒तेषु॑ । स॒जा॒तेषु॑ भूयासम् । स॒जा॒तेष्विति॑ स - जा॒तेषु॑ । भू॒या॒स॒मु॒ग्रः । उ॒ग्रश्चेत्ता᳚ । चेत्ता॑ वसु॒वित् । व॒सु॒विद् ध्रु॒वः । व॒सु॒विदिति॑ वसु - वित् । ध्रु॒वो॑ऽसि । अ॒सि॒ ध्रु॒वः । ध्रु॒वो॑ऽहम् । अ॒हꣳ स॑जा॒तेषु॑ । स॒जा॒तेषु॑ भूयासम् । स॒जा॒तेष्विति॑ स - जा॒तेषु॑ । भू॒या॒स॒म॒भि॒भूः । अ॒भि॒भूश्चेत्ता᳚ । अ॒भि॒भूरित्य॑भि - भूः । चेत्ता॑ वसु॒वित् । व॒सु॒विदाम॑नम् । व॒सु॒विदिति॑ वसु - वित् । आम॑नमसि । आम॑न॒मित्या - म॒न॒म् । अ॒स्याम॑नस्य । आम॑नस्य देवाः । आम॑न॒स्येत्या - म॒न॒स्य॒ । दे॒वा॒ ये । ये स॑जा॒ताः । स॒जा॒ताः कु॑मा॒राः । स॒जा॒ता इति॑ स - जा॒ताः । कु॒मा॒राः सम॑नसः । सम॑नस॒स्तान् । सम॑नस॒ इति॒ स - म॒न॒सः॒ । तान॒हम् । अ॒हम् का॑मये । का॒म॒ये॒ हृ॒दा । हृ॒दा ते । ते माम् । माम् का॑मयन्ताम् । का॒म॒य॒न्ताꣳ॒॒ हृ॒दा । हृ॒दा तान् । तान्मे᳚ । म॒ आम॑नसः । आम॑नसः कृधि । आम॑नस॒ इत्या - म॒न॒सः॒ । कृ॒धि॒ स्वाहा᳚ । स्वाहा ऽऽम॑नम् । आम॑नमसि । आम॑न॒मित्या - म॒न॒म् । अ॒स्याम॑नस्य \newline

\textbf{Jatai Paata} \newline

1. ध्रु॒वो᳚ ऽस्यसि ध्रु॒वो ध्रु॒वो॑ ऽसि । \newline
2. अ॒सि॒ ध्रु॒वो ध्रु॒वो᳚ ऽस्यसि ध्रु॒वः । \newline
3. ध्रु॒वो॑ ऽह म॒हम् ध्रु॒वो ध्रु॒वो॑ ऽहम् । \newline
4. अ॒हꣳ स॑जा॒तेषु॑ सजा॒तेष्व॒ह म॒हꣳ स॑जा॒तेषु॑ । \newline
5. स॒जा॒तेषु॑ भूयासम् भूयासꣳ सजा॒तेषु॑ सजा॒तेषु॑ भूयासम् । \newline
6. स॒जा॒तेष्विति॑ स - जा॒तेषु॑ । \newline
7. भू॒या॒स॒म् धीरो॒ धीरो॑ भूयासम् भूयास॒म् धीरः॑ । \newline
8. धीर॒ श्चेत्ता॒ चेत्ता॒ धीरो॒ धीर॒ श्चेत्ता᳚ । \newline
9. चेत्ता॑ वसु॒विद् व॑सु॒विच् चेत्ता॒ चेत्ता॑ वसु॒वित् । \newline
10. व॒सु॒विद् ध्रु॒वो ध्रु॒वो व॑सु॒विद् व॑सु॒विद् ध्रु॒वः । \newline
11. व॒सु॒विदिति॑ वसु - वित् । \newline
12. ध्रु॒वो᳚ ऽस्यसि ध्रु॒वो ध्रु॒वो॑ ऽसि । \newline
13. अ॒सि॒ ध्रु॒वो ध्रु॒वो᳚ ऽस्यसि ध्रु॒वः । \newline
14. ध्रु॒वो॑ ऽह म॒हम् ध्रु॒वो ध्रु॒वो॑ ऽहम् । \newline
15. अ॒हꣳ स॑जा॒तेषु॑ सजा॒ते ष्व॒ह म॒हꣳ स॑जा॒तेषु॑ । \newline
16. स॒जा॒तेषु॑ भूयासम् भूयासꣳ सजा॒तेषु॑ सजा॒तेषु॑ भूयासम् । \newline
17. स॒जा॒तेष्विति॑ स - जा॒तेषु॑ । \newline
18. भू॒या॒स॒ मु॒ग्र उ॒ग्रो भू॑यासम् भूयास मु॒ग्रः । \newline
19. उ॒ग्र श्चेत्ता॒ चेत्तो॒ग्र उ॒ग्र श्चेत्ता᳚ । \newline
20. चेत्ता॑ वसु॒विद् व॑सु॒विच् चेत्ता॒ चेत्ता॑ वसु॒वित् । \newline
21. व॒सु॒विद् ध्रु॒वो ध्रु॒वो व॑सु॒विद् व॑सु॒विद् ध्रु॒वः । \newline
22. व॒सु॒विदिति॑ वसु - वित् । \newline
23. ध्रु॒वो᳚ ऽस्यसि ध्रु॒वो ध्रु॒वो॑ ऽसि । \newline
24. अ॒सि॒ ध्रु॒वो ध्रु॒वो᳚ ऽस्यसि ध्रु॒वः । \newline
25. ध्रु॒वो॑ ऽह म॒हम् ध्रु॒वो ध्रु॒वो॑ ऽहम् । \newline
26. अ॒हꣳ स॑जा॒तेषु॑ सजा॒ते ष्व॒ह म॒हꣳ स॑जा॒तेषु॑ । \newline
27. स॒जा॒तेषु॑ भूयासम् भूयासꣳ सजा॒तेषु॑ सजा॒तेषु॑ भूयासम् । \newline
28. स॒जा॒तेष्विति॑ स - जा॒तेषु॑ । \newline
29. भू॒या॒स॒ म॒भि॒भू र॑भि॒भूर् भू॑यासम् भूयास मभि॒भूः । \newline
30. अ॒भि॒भू श्चेत्ता॒ चेत्ता॑ ऽभि॒भू र॑भि॒भू श्चेत्ता᳚ । \newline
31. अ॒भि॒भूरित्य॑भि - भूः । \newline
32. चेत्ता॑ वसु॒विद् व॑सु॒विच् चेत्ता॒ चेत्ता॑ वसु॒वित् । \newline
33. व॒सु॒वि दाम॑न॒ माम॑नं ॅवसु॒विद् व॑सु॒वि दाम॑नम् । \newline
34. व॒सु॒विदिति॑ वसु - वित् । \newline
35. आम॑न मस्य॒स्याम॑न॒ माम॑न मसि । \newline
36. आम॑न॒मित्या - म॒न॒म् । \newline
37. अ॒स्याम॑न॒स्या म॑नस्यास्य॒स्या म॑नस्य । \newline
38. आम॑नस्य देवा देवा॒ आम॑न॒स्या म॑नस्य देवाः । \newline
39. आम॑न॒स्येत्या - म॒न॒स्य॒ । \newline
40. दे॒वा॒ ये ये दे॑वा देवा॒ ये । \newline
41. ये स॑जा॒ताः स॑जा॒ता ये ये स॑जा॒ताः । \newline
42. स॒जा॒ताः कु॑मा॒राः कु॑मा॒राः स॑जा॒ताः स॑जा॒ताः कु॑मा॒राः । \newline
43. स॒जा॒ता इति॑ स - जा॒ताः । \newline
44. कु॒मा॒राः सम॑नसः॒ सम॑नसः कुमा॒राः कु॑मा॒राः सम॑नसः । \newline
45. सम॑नस॒ स्ताꣳ स्तान् थ्सम॑नसः॒ सम॑नस॒ स्तान् । \newline
46. सम॑नस॒ इति॒ स - म॒न॒सः॒ । \newline
47. ता न॒ह म॒हम् ताꣳ स्ता न॒हम् । \newline
48. अ॒हम् का॑मये कामये॒ ऽह म॒हम् का॑मये । \newline
49. का॒म॒ये॒ हृ॒दा हृ॒दा का॑मये कामये हृ॒दा । \newline
50. हृ॒दा ते ते हृ॒दा हृ॒दा ते । \newline
51. ते माम् माम् ते ते माम् । \newline
52. माम् का॑मयन्ताम् कामयन्ता॒म् माम् माम् का॑मयन्ताम् । \newline
53. का॒म॒य॒न्ताꣳ॒॒ हृ॒दा हृ॒दा का॑मयन्ताम् कामयन्ताꣳ हृ॒दा । \newline
54. हृ॒दा ताꣳस् तान्. हृ॒दा हृ॒दा तान् । \newline
55. तान् मे॑ मे॒ ताꣳ स्तान् मे᳚ । \newline
56. म॒ आम॑नस॒ आम॑नसो मे म॒ आम॑नसः । \newline
57. आम॑नसः कृधि कृ॒ध्या म॑नस॒ आम॑नसः कृधि । \newline
58. आम॑नस॒ इत्या - म॒न॒सः॒ । \newline
59. कृ॒धि॒ स्वाहा॒ स्वाहा॑ कृधि कृधि॒ स्वाहा᳚ । \newline
60. स्वाहा ऽऽम॑न॒ माम॑नꣳ॒॒ स्वाहा॒ स्वाहा ऽऽम॑नम् । \newline
61. आम॑न मस्य॒स्याम॑न॒ माम॑न मसि । \newline
62. आम॑न॒मित्या - म॒न॒म् । \newline
63. अ॒स्या म॑न॒स्या म॑नस्यास्य॒स्या म॑नस्य । \newline

\textbf{Ghana Paata } \newline

1. ध्रु॒वो᳚ ऽस्यसि ध्रु॒वो ध्रु॒वो॑ ऽसि ध्रु॒वो ध्रु॒वो॑ ऽसि ध्रु॒वो ध्रु॒वो॑ ऽसि ध्रु॒वः । \newline
2. अ॒सि॒ ध्रु॒वो ध्रु॒वो᳚ ऽस्यसि ध्रु॒वो॑ ऽह म॒हम् ध्रु॒वो᳚ ऽस्यसि ध्रु॒वो॑ ऽहम् । \newline
3. ध्रु॒वो॑ ऽह म॒हम् ध्रु॒वो ध्रु॒वो॑ ऽहꣳ स॑जा॒तेषु॑ सजा॒ते ष्व॒हम् ध्रु॒वो ध्रु॒वो॑ ऽहꣳ स॑जा॒तेषु॑ । \newline
4. अ॒हꣳ स॑जा॒तेषु॑ सजा॒ते ष्व॒ह म॒हꣳ स॑जा॒तेषु॑ भूयासम् भूयासꣳ सजा॒ते ष्व॒ह म॒हꣳ स॑जा॒तेषु॑ भूयासम् । \newline
5. स॒जा॒तेषु॑ भूयासम् भूयासꣳ सजा॒तेषु॑ सजा॒तेषु॑ भूयास॒म् धीरो॒ धीरो॑ भूयासꣳ सजा॒तेषु॑ सजा॒तेषु॑ भूयास॒म् धीरः॑ । \newline
6. स॒जा॒तेष्विति॑ स - जा॒तेषु॑ । \newline
7. भू॒या॒स॒म् धीरो॒ धीरो॑ भूयासम् भूयास॒म् धीर॒ श्चेत्ता॒ चेत्ता॒ धीरो॑ भूयासम् भूयास॒म् धीर॒ श्चेत्ता᳚ । \newline
8. धीर॒श्चेत्ता॒ चेत्ता॒ धीरो॒ धीर॒ श्चेत्ता॑ वसु॒विद् व॑सु॒विच् चेत्ता॒ धीरो॒ धीर॒ श्चेत्ता॑ वसु॒वित् । \newline
9. चेत्ता॑ वसु॒विद् व॑सु॒विच् चेत्ता॒ चेत्ता॑ वसु॒विद् ध्रु॒वो ध्रु॒वो व॑सु॒विच् चेत्ता॒ चेत्ता॑ वसु॒विद् ध्रु॒वः । \newline
10. व॒सु॒विद् ध्रु॒वो ध्रु॒वो व॑सु॒विद् व॑सु॒विद् ध्रु॒वो᳚ ऽस्यसि ध्रु॒वो व॑सु॒विद् व॑सु॒विद् ध्रु॒वो॑ ऽसि । \newline
11. व॒सु॒विदिति॑ वसु - वित् । \newline
12. ध्रु॒वो᳚ ऽस्यसि ध्रु॒वो ध्रु॒वो॑ ऽसि ध्रु॒वो ध्रु॒वो॑ ऽसि ध्रु॒वो ध्रु॒वो॑ ऽसि ध्रु॒वः । \newline
13. अ॒सि॒ ध्रु॒वो ध्रु॒वो᳚ ऽस्यसि ध्रु॒वो॑ ऽह म॒हम् ध्रु॒वो᳚ ऽस्यसि ध्रु॒वो॑ ऽहम् । \newline
14. ध्रु॒वो॑ ऽह म॒हम् ध्रु॒वो ध्रु॒वो॑ ऽहꣳ स॑जा॒तेषु॑ सजा॒ते ष्व॒हम् ध्रु॒वो ध्रु॒वो॑ ऽहꣳ स॑जा॒तेषु॑ । \newline
15. अ॒हꣳ स॑जा॒तेषु॑ सजा॒ते ष्व॒ह म॒हꣳ स॑जा॒तेषु॑ भूयासम् भूयासꣳ सजा॒ते ष्व॒ह म॒हꣳ स॑जा॒तेषु॑ भूयासम् । \newline
16. स॒जा॒तेषु॑ भूयासम् भूयासꣳ सजा॒तेषु॑ सजा॒तेषु॑ भूयास मु॒ग्र उ॒ग्रो भू॑यासꣳ सजा॒तेषु॑ सजा॒तेषु॑ भूयास मु॒ग्रः । \newline
17. स॒जा॒तेष्विति॑ स - जा॒तेषु॑ । \newline
18. भू॒या॒स॒ मु॒ग्र उ॒ग्रो भू॑यासम् भूयास मु॒ग्रश्चेत्ता॒ चेत्तो॒ग्रो भू॑यासम् भूयास मु॒ग्रश्चेत्ता᳚ । \newline
19. उ॒ग्रश्चेत्ता॒ चेत्तो॒ग्र उ॒ग्रश्चेत्ता॑ वसु॒विद् व॑सु॒विच् चेत्तो॒ग्र उ॒ग्रश्चेत्ता॑ वसु॒वित् । \newline
20. चेत्ता॑ वसु॒विद् व॑सु॒विच् चेत्ता॒ चेत्ता॑ वसु॒विद् ध्रु॒वो ध्रु॒वो व॑सु॒विच् चेत्ता॒ चेत्ता॑ वसु॒विद् ध्रु॒वः । \newline
21. व॒सु॒विद् ध्रु॒वो ध्रु॒वो व॑सु॒विद् व॑सु॒विद् ध्रु॒वो᳚ ऽस्यसि ध्रु॒वो व॑सु॒विद् व॑सु॒विद् ध्रु॒वो॑ ऽसि । \newline
22. व॒सु॒विदिति॑ वसु - वित् । \newline
23. ध्रु॒वो᳚ ऽस्यसि ध्रु॒वो ध्रु॒वो॑ ऽसि ध्रु॒वो ध्रु॒वो॑ ऽसि ध्रु॒वो ध्रु॒वो॑ ऽसि ध्रु॒वः । \newline
24. अ॒सि॒ ध्रु॒वो ध्रु॒वो᳚ ऽस्यसि ध्रु॒वो॑ ऽह म॒हम् ध्रु॒वो᳚ ऽस्यसि ध्रु॒वो॑ ऽहम् । \newline
25. ध्रु॒वो॑ ऽह म॒हम् ध्रु॒वो ध्रु॒वो॑ ऽहꣳ स॑जा॒तेषु॑ सजा॒ते ष्व॒हम् ध्रु॒वो ध्रु॒वो॑ ऽहꣳ स॑जा॒तेषु॑ । \newline
26. अ॒हꣳ स॑जा॒तेषु॑ सजा॒तेष्व॒ह म॒हꣳ स॑जा॒तेषु॑ भूयासम् भूयासꣳ सजा॒ते ष्व॒ह म॒हꣳ स॑जा॒तेषु॑ भूयासम् । \newline
27. स॒जा॒तेषु॑ भूयासम् भूयासꣳ सजा॒तेषु॑ सजा॒तेषु॑ भूयास मभि॒भू र॑भि॒भूर् भू॑यासꣳ सजा॒तेषु॑ सजा॒तेषु॑ भूयास मभि॒भूः । \newline
28. स॒जा॒तेष्विति॑ स - जा॒तेषु॑ । \newline
29. भू॒या॒स॒ म॒भि॒भू र॑भि॒भूर् भू॑यासम् भूयास मभि॒भू श्चेत्ता॒ चेत्ता॑ ऽभि॒भूर् भू॑यासम् भूयास मभि॒भू श्चेत्ता᳚ । \newline
30. अ॒भि॒भू श्चेत्ता॒ चेत्ता॑ ऽभि॒भू र॑भि॒भू श्चेत्ता॑ वसु॒विद् व॑सु॒विच् चेत्ता॑ ऽभि॒भू र॑भि॒भू श्चेत्ता॑ वसु॒वित् । \newline
31. अ॒भि॒भूरित्य॑भि - भूः । \newline
32. चेत्ता॑ वसु॒विद् व॑सु॒विच् चेत्ता॒ चेत्ता॑ वसु॒वि दाम॑न॒ माम॑नं ॅवसु॒विच् चेत्ता॒ चेत्ता॑ वसु॒वि दाम॑नम् । \newline
33. व॒सु॒वि दाम॑न॒ माम॑नं ॅवसु॒विद् व॑सु॒वि दाम॑न मस्य॒स्याम॑नं ॅवसु॒विद् व॑सु॒वि दाम॑न मसि । \newline
34. व॒सु॒विदिति॑ वसु - वित् । \newline
35. आम॑न मस्य॒स्याम॑न॒ माम॑न म॒स्या म॑न॒स्या म॑नस्या॒ स्याम॑न॒ माम॑न म॒स्या म॑नस्य । \newline
36. आम॑न॒मित्या - म॒न॒म् । \newline
37. अ॒स्या म॑न॒स्या म॑नस्या स्य॒स्या म॑नस्य देवा देवा॒ आम॑नस्या स्य॒स्या म॑नस्य देवाः । \newline
38. आम॑नस्य देवा देवा॒ आम॑न॒स्या म॑नस्य देवा॒ ये ये दे॑वा॒ आम॑न॒स्या म॑नस्य देवा॒ ये । \newline
39. आम॑न॒स्येत्या - म॒न॒स्य॒ । \newline
40. दे॒वा॒ ये ये दे॑वा देवा॒ ये स॑जा॒ताः स॑जा॒ता ये दे॑वा देवा॒ ये स॑जा॒ताः । \newline
41. ये स॑जा॒ताः स॑जा॒ता ये ये स॑जा॒ताः कु॑मा॒राः कु॑मा॒राः स॑जा॒ता ये ये स॑जा॒ताः कु॑मा॒राः । \newline
42. स॒जा॒ताः कु॑मा॒राः कु॑मा॒राः स॑जा॒ताः स॑जा॒ताः कु॑मा॒राः सम॑नसः॒ सम॑नसः कुमा॒राः स॑जा॒ताः स॑जा॒ताः कु॑मा॒राः सम॑नसः । \newline
43. स॒जा॒ता इति॑ स - जा॒ताः । \newline
44. कु॒मा॒राः सम॑नसः॒ सम॑नसः कुमा॒राः कु॑मा॒राः सम॑नस॒ स्ताꣳ स्तान् थ्सम॑नसः कुमा॒राः कु॑मा॒राः सम॑नस॒ स्तान् । \newline
45. सम॑नस॒ स्ताꣳ स्तान् थ्सम॑नसः॒ सम॑नस॒ स्ता न॒ह म॒हम् तान् थ्सम॑नसः॒ सम॑नस॒ स्ता न॒हम् । \newline
46. सम॑नस॒ इति॒ स - म॒न॒सः॒ । \newline
47. ता न॒ह म॒हम् ताꣳ स्ता न॒हम् का॑मये कामये॒ ऽहम् ताꣳ स्ता न॒हम् का॑मये । \newline
48. अ॒हम् का॑मये कामये॒ ऽह म॒हम् का॑मये हृ॒दा हृ॒दा का॑मये॒ ऽह म॒हम् का॑मये हृ॒दा । \newline
49. का॒म॒ये॒ हृ॒दा हृ॒दा का॑मये कामये हृ॒दा ते ते हृ॒दा का॑मये कामये हृ॒दा ते । \newline
50. हृ॒दा ते ते हृ॒दा हृ॒दा ते माम् माम् ते हृ॒दा हृ॒दा ते माम् । \newline
51. ते माम् माम् ते ते माम् का॑मयन्ताम् कामयन्ता॒म् माम् ते ते माम् का॑मयन्ताम् । \newline
52. माम् का॑मयन्ताम् कामयन्ता॒म् माम् माम् का॑मयन्ताꣳ हृ॒दा हृ॒दा का॑मयन्ता॒म् माम् माम् का॑मयन्ताꣳ हृ॒दा । \newline
53. का॒म॒य॒न्ताꣳ॒॒ हृ॒दा हृ॒दा का॑मयन्ताम् कामयन्ताꣳ हृ॒दा ताꣳ स्तान्. हृ॒दा का॑मयन्ताम् कामयन्ताꣳ हृ॒दा तान् । \newline
54. हृ॒दा ताꣳ स्तान्. हृ॒दा हृ॒दा तान् मे॑ मे॒ तान्. हृ॒दा हृ॒दा तान् मे᳚ । \newline
55. तान् मे॑ मे॒ ताꣳ स्तान् म॒ आम॑नस॒ आम॑नसो मे॒ ताꣳ स्तान् म॒ आम॑नसः । \newline
56. म॒ आम॑नस॒ आम॑नसो मे म॒ आम॑नसः कृधि कृ॒ध्याम॑नसो मे म॒ आम॑नसः कृधि । \newline
57. आम॑नसः कृधि कृ॒ध्याम॑नस॒ आम॑नसः कृधि॒ स्वाहा॒ स्वाहा॑ कृ॒ध्याम॑नस॒ आम॑नसः कृधि॒ स्वाहा᳚ । \newline
58. आम॑नस॒ इत्या - म॒न॒सः॒ । \newline
59. कृ॒धि॒ स्वाहा॒ स्वाहा॑ कृधि कृधि॒ स्वाहा ऽऽम॑न॒ माम॑नꣳ॒॒ स्वाहा॑ कृधि कृधि॒ स्वाहा ऽऽम॑नम् । \newline
60. स्वाहा ऽऽम॑न॒ माम॑नꣳ॒॒ स्वाहा॒ स्वाहा ऽऽम॑न मस्य॒ स्याम॑नꣳ॒॒ स्वाहा॒ स्वाहा ऽऽम॑न मसि । \newline
61. आम॑न मस्य॒स्याम॑न॒ माम॑न म॒स्या म॑न॒स्या म॑नस्या॒ स्याम॑न॒ माम॑न म॒स्याम॑नस्य । \newline
62. आम॑न॒मित्या - म॒न॒म् । \newline
63. अ॒स्याम॑न॒स्या म॑नस्या स्य॒स्या म॑नस्य देवा देवा॒ आम॑नस्या स्य॒स्या म॑नस्य देवाः । \newline
\pagebreak
\markright{ TS 2.3.9.2  \hfill https://www.vedavms.in \hfill}

\section{ TS 2.3.9.2 }

\textbf{TS 2.3.9.2 } \newline
\textbf{Samhita Paata} \newline

-म॑नस्य देवा॒ याः स्त्रियः॒ सम॑नस॒स्ता अ॒हं का॑मये हृ॒दा ता मां का॑मयन्ताꣳ हृ॒दा ता म॒ आम॑नसः कृधि॒ स्वाहा॑ वैश्वदे॒वीꣳसा᳚ङ्ग्रह॒णीं निर्व॑पे॒द्ग्राम॑कामो वैश्वदे॒वा वै स॑जा॒ता विश्वा॑ने॒व दे॒वान्थ्स्वेन॑ भाग॒धेये॒नोप॑ धावति॒ त ए॒वास्मै॑ सजा॒तान् प्र य॑च्छन्ति ग्रा॒म्ये॑व भ॑वति सांग्रह॒णी भ॑वति मनो॒ग्रह॑णं॒ ॅवैस॒ग्रंह॑णं॒ मन॑ ए॒व स॑जा॒तानां᳚ - [  ] \newline

\textbf{Pada Paata} \newline

आम॑न॒स्येत्या - म॒न॒स्य॒ । दे॒वाः॒ । याः । स्त्रियः॑ । सम॑नस॒ इति॒ स - म॒न॒सः॒ । ताः । अ॒हम् । का॒म॒ये॒ । हृ॒दा । ताः । माम् । का॒म॒य॒न्ता॒म् । हृ॒दा । ताः । मे॒ । आम॑नस॒ इत्या - म॒न॒सः॒ । कृ॒धि॒ । स्वाहा᳚ । वै॒श्व॒दे॒वीमिति॑ वैश्व - दे॒वीम् । सा॒ग्रं॒ह॒णीमिति॑ सां - ग्र॒ह॒णीम् । निरिति॑ । व॒पे॒त् । ग्राम॑काम॒ इति॒ ग्राम॑ - का॒मः॒ । वै॒श्व॒दे॒वा इति॑ वैश्व - दे॒वाः । वै । स॒जा॒ता इति॑ स - जा॒ताः । विश्वान्॑ । ए॒व । दे॒वान् । स्वेन॑ । भा॒ग॒धेये॒नेति॑ भाग - धेये॑न । उपेति॑ । धा॒व॒ति॒ । ते । ए॒व । अ॒स्मै॒ । स॒जा॒तानिति॑ स-जा॒तान् । प्रेति॑ । य॒च्छ॒न्ति॒ । ग्रा॒मी ।  ए॒व । भ॒व॒ति॒ । सा॒ग्रं॒ह॒णीति॑ सां - ग्र॒ह॒णी । भ॒व॒ति॒ । म॒नो॒ग्रह॑ण॒मिति॑ मनः - ग्रह॑णम् । वै । स॒ग्रंह॑ण॒मिति॑ सं - ग्रह॑णम् । मनः॑ ।  ए॒व । स॒जा॒ताना॒मिति॑ स - जा॒ताना᳚म् ।  \newline


\textbf{Krama Paata} \newline

आम॑नस्य देवाः । आम॑न॒स्येत्या - म॒न॒स्य॒ । दे॒वा॒ याः । याः स्त्रियः॑ । स्त्रियः॒ सम॑नसः । सम॑नस॒स्ताः । सम॑नस॒ इति॒ स - म॒न॒सः॒ । ता अ॒हम् । अ॒हम् का॑मये । का॒म॒ये॒ हृ॒दा । हृ॒दा ताः । ता माम् । माम् का॑मयन्ताम् । का॒म॒य॒न्ताꣳ॒॒ हृ॒दा । हृ॒दा ताः । ता मे᳚ । म॒ आम॑नसः । आम॑नसः कृधि । आम॑नस॒ इत्या - म॒न॒सः॒ । कृ॒धि॒ स्वाहा᳚ । स्वाहा॑ वैश्वदे॒वीम् । वै॒श्व॒दे॒वीꣳ सा᳚ङ्ग्रह॒णीम् । वै॒श्व॒दे॒वीमिति॑ वैश्व - दे॒वीम् । सा॒ङ्ग्र॒ह॒णीम् निः । सा॒ङ्ग्र॒ह॒णीमिति॑ सां - ग्र॒ह॒णीम् । निर् व॑पेत् । व॒पे॒द् ग्राम॑कामः । ग्राम॑कामो वैश्वदे॒वाः । ग्राम॑काम॒ इति॒ ग्राम॑ - का॒मः॒ । वै॒श्व॒दे॒वा वै । वै॒श्व॒दे॒वा इति॑ वैश्व - दे॒वाः । वै स॑जा॒ताः । स॒जा॒ता विश्वान्॑ । स॒जा॒ता इति॑ स - जा॒ताः । विश्वा॑ने॒व । ए॒व दे॒वान् । दे॒वान्थ् स्वेन॑ । स्वेन॑ भाग॒धेये॑न । भा॒ग॒धेये॒नोप॑ । भा॒ग॒धेये॒नेति॑ भाग - धेये॑न । उप॑ धावति । धा॒व॒ति॒ ते । त ए॒व । ए॒वास्मै᳚ । अ॒स्मै॒ स॒जा॒तान् । स॒जा॒तान् प्र । स॒जा॒तानिति॑ स - जा॒तान् । प्र य॑च्छन्ति । य॒च्छ॒न्ति॒ ग्रा॒मी । ग्रा॒म्ये॑व । ए॒व भ॑वति । भ॒व॒ति॒ सा॒ङ्ग्र॒ह॒णी । सा॒ङ्ग्र॒ह॒णी भ॑वति । सा॒ङ्ग्र॒ह॒णीति॑ साम् - ग्र॒ह॒णी । भ॒व॒ति॒ म॒नो॒ग्रह॑णम् । म॒नो॒ग्रह॑णं॒ ॅवै । म॒नो॒ग्रह॑ण॒मिति॑ मनः - ग्रह॑णम् । वै स॒ङ्ग्रह॑णम् । स॒ङ्ग्रह॑ण॒म् मनः॑ । स॒ङ्ग्रह॑ण॒मिति॑ सं - ग्रह॑णम् । मन॑ ए॒व । ए॒व स॑जा॒ताना᳚म् । स॒जा॒ताना᳚म् गृह्णाति । स॒जा॒ताना॒मिति॑ स - जा॒ताना᳚म् \newline

\textbf{Jatai Paata} \newline

1. आम॑नस्य देवा देवा॒ आम॑न॒स्या म॑नस्य देवाः । \newline
2. आम॑न॒स्येत्या - म॒न॒स्य॒ । \newline
3. दे॒वा॒ या या दे॑वा देवा॒ याः । \newline
4. याः स्त्रियः॒ स्त्रियो॒ या याः स्त्रियः॑ । \newline
5. स्त्रियः॒ सम॑नसः॒ सम॑नसः॒ स्त्रियः॒ स्त्रियः॒ सम॑नसः । \newline
6. सम॑नस॒ स्ता स्ताः सम॑नसः॒ सम॑नस॒ स्ताः । \newline
7. सम॑नस॒ इति॒ स - म॒न॒सः॒ । \newline
8. ता अ॒ह म॒हम् ता स्ता अ॒हम् । \newline
9. अ॒हम् का॑मये कामये॒ ऽह म॒हम् का॑मये । \newline
10. का॒म॒ये॒ हृ॒दा हृ॒दा का॑मये कामये हृ॒दा । \newline
11. हृ॒दा ता स्ता हृ॒दा हृ॒दा ताः । \newline
12. ता माम् माम् तास्ता माम् । \newline
13. माम् का॑मयन्ताम् कामयन्ता॒म् माम् माम् का॑मयन्ताम् । \newline
14. का॒म॒य॒न्ताꣳ॒॒ हृ॒दा हृ॒दा का॑मयन्ताम् कामयन्ताꣳ हृ॒दा । \newline
15. हृ॒दा ता स्ता हृ॒दा हृ॒दा ताः । \newline
16. ता मे॑ मे॒ ता स्ता मे᳚ । \newline
17. म॒ आम॑नस॒ आम॑नसो मे म॒ आम॑नसः । \newline
18. आम॑नसः कृधि कृ॒ध्या म॑नस॒ आम॑नसः कृधि । \newline
19. आम॑नस॒ इत्या - म॒न॒सः॒ । \newline
20. कृ॒धि॒ स्वाहा॒ स्वाहा॑ कृधि कृधि॒ स्वाहा᳚ । \newline
21. स्वाहा॑ वैश्वदे॒वीं ॅवै᳚श्वदे॒वीꣳ स्वाहा॒ स्वाहा॑ वैश्वदे॒वीम् । \newline
22. वै॒श्व॒दे॒वीꣳ सा᳚ङ्ग्रह॒णीꣳ सा᳚ङ्ग्रह॒णीं ॅवै᳚श्वदे॒वीं ॅवै᳚श्वदे॒वीꣳ सा᳚ङ्ग्रह॒णीम् । \newline
23. वै॒श्व॒दे॒वीमिति॑ वैश्व - दे॒वीम् । \newline
24. सा॒ङ्ग्र॒ह॒णीम् निर् णिः सा᳚ङ्ग्रह॒णीꣳ सा᳚ङ्ग्रह॒णीम् निः । \newline
25. सा॒ङ्ग्र॒ह॒णीमिति॑ सां - ग्र॒ह॒णीम् । \newline
26. निर् व॑पेद् वपे॒न् निर् णिर् व॑पेत् । \newline
27. व॒पे॒द् ग्राम॑कामो॒ ग्राम॑कामो वपेद् वपे॒द् ग्राम॑कामः । \newline
28. ग्राम॑कामो वैश्वदे॒वा वै᳚श्वदे॒वा ग्राम॑कामो॒ ग्राम॑कामो वैश्वदे॒वाः । \newline
29. ग्राम॑काम॒ इति॒ ग्राम॑ - का॒मः॒ । \newline
30. वै॒श्व॒दे॒वा वै वै वै᳚श्वदे॒वा वै᳚श्वदे॒वा वै । \newline
31. वै॒श्व॒दे॒वा इति॑ वैश्व - दे॒वाः । \newline
32. वै स॑जा॒ताः स॑जा॒ता वै वै स॑जा॒ताः । \newline
33. स॒जा॒ता विश्वा॒न्॒. विश्वा᳚न् थ्सजा॒ताः स॑जा॒ता विश्वान्॑ । \newline
34. स॒जा॒ता इति॑ स - जा॒ताः । \newline
35. विश्वा॑ ने॒वैव विश्वा॒न्॒. विश्वा॑ ने॒व । \newline
36. ए॒व दे॒वान् दे॒वा ने॒वैव दे॒वान् । \newline
37. दे॒वान् थ्स्वेन॒ स्वेन॑ दे॒वान् दे॒वान् थ्स्वेन॑ । \newline
38. स्वेन॑ भाग॒धेये॑न भाग॒धेये॑न॒ स्वेन॒ स्वेन॑ भाग॒धेये॑न । \newline
39. भा॒ग॒धेये॒नोपोप॑ भाग॒धेये॑न भाग॒धेये॒नोप॑ । \newline
40. भा॒ग॒धेये॒नेति॑ भाग - धेये॑न । \newline
41. उप॑ धावति धाव॒ त्युपोप॑ धावति । \newline
42. धा॒व॒ति॒ ते ते धा॑वति धावति॒ ते । \newline
43. त ए॒वैव ते त ए॒व । \newline
44. ए॒वास्मा॑ अस्मा ए॒वैवास्मै᳚ । \newline
45. अ॒स्मै॒ स॒जा॒तान् थ्स॑जा॒ता न॑स्मा अस्मै सजा॒तान् । \newline
46. स॒जा॒तान् प्र प्र स॑जा॒तान् थ्स॑जा॒तान् प्र । \newline
47. स॒जा॒तानिति॑ स - जा॒तान् । \newline
48. प्र य॑च्छन्ति यच्छन्ति॒ प्र प्र य॑च्छन्ति । \newline
49. य॒च्छ॒न्ति॒ ग्रा॒मी ग्रा॒मी य॑च्छन्ति यच्छन्ति ग्रा॒मी । \newline
50. ग्रा॒म्ये॑वैव ग्रा॒मी ग्रा॒म्ये॑व । \newline
51. ए॒व भ॑वति भव त्ये॒वैव भ॑वति । \newline
52. भ॒व॒ति॒ सा॒ङ्ग्र॒ह॒णी सा᳚ङ्ग्रह॒णी भ॑वति भवति साङ्ग्रह॒णी । \newline
53. सा॒ङ्ग्र॒ह॒णी भ॑वति भवति साङ्ग्रह॒णी सा᳚ङ्ग्रह॒णी भ॑वति । \newline
54. सा॒ङ्ग्र॒ह॒णीति॑ सां - ग्र॒ह॒णी । \newline
55. भ॒व॒ति॒ म॒नो॒ग्रह॑णम् मनो॒ग्रह॑णम् भवति भवति मनो॒ग्रह॑णम् । \newline
56. म॒नो॒ग्रह॑णं॒ ॅवै वै म॑नो॒ग्रह॑णम् मनो॒ग्रह॑णं॒ ॅवै । \newline
57. म॒नो॒ग्रह॑ण॒मिति॑ मनः - ग्रह॑णम् । \newline
58. वै स॒ङ्ग्रह॑णꣳ स॒ङ्ग्रह॑णं॒ ॅवै वै स॒ङ्ग्रह॑णम् । \newline
59. स॒ङ्ग्रह॑ण॒म् मनो॒ मनः॑ स॒ङ्ग्रह॑णꣳ स॒ङ्ग्रह॑ण॒म् मनः॑ । \newline
60. स॒ङ्ग्रह॑ण॒मिति॑ सं - ग्रह॑णम् । \newline
61. मन॑ ए॒वैव मनो॒ मन॑ ए॒व । \newline
62. ए॒व स॑जा॒तानाꣳ॑ सजा॒ताना॑ मे॒वैव स॑जा॒ताना᳚म् । \newline
63. स॒जा॒ताना᳚म् गृह्णाति गृह्णाति सजा॒तानाꣳ॑ सजा॒ताना᳚म् गृह्णाति । \newline
64. स॒जा॒ताना॒मिति॑ स - जा॒ताना᳚म् । \newline

\textbf{Ghana Paata } \newline

1. आम॑नस्य देवा देवा॒ आम॑न॒स्या म॑नस्य देवा॒ या या दे॑वा॒ आम॑न॒स्या म॑नस्य देवा॒ याः । \newline
2. आम॑न॒स्येत्या - म॒न॒स्य॒ । \newline
3. दे॒वा॒ या या दे॑वा देवा॒ याः स्त्रियः॒ स्त्रियो॒ या दे॑वा देवा॒ याः स्त्रियः॑ । \newline
4. याः स्त्रियः॒ स्त्रियो॒ या याः स्त्रियः॒ सम॑नसः॒ सम॑नसः॒ स्त्रियो॒ या याः स्त्रियः॒ सम॑नसः । \newline
5. स्त्रियः॒ सम॑नसः॒ सम॑नसः॒ स्त्रियः॒ स्त्रियः॒ सम॑नस॒ स्ता स्ताः सम॑नसः॒ स्त्रियः॒ स्त्रियः॒ सम॑नस॒ स्ताः । \newline
6. सम॑नस॒ स्ता स्ताः सम॑नसः॒ सम॑नस॒ स्ता अ॒ह म॒हम् ताः सम॑नसः॒ सम॑नस॒ स्ता अ॒हम् । \newline
7. सम॑नस॒ इति॒ स - म॒न॒सः॒ । \newline
8. ता अ॒ह म॒हम् ता स्ता अ॒हम् का॑मये कामये॒ ऽहम् ता स्ता अ॒हम् का॑मये । \newline
9. अ॒हम् का॑मये कामये॒ ऽह म॒हम् का॑मये हृ॒दा हृ॒दा का॑मये॒ ऽह म॒हम् का॑मये हृ॒दा । \newline
10. का॒म॒ये॒ हृ॒दा हृ॒दा का॑मये कामये हृ॒दा ता स्ता हृ॒दा का॑मये कामये हृ॒दा ताः । \newline
11. हृ॒दा ता स्ता हृ॒दा हृ॒दा ता माम् माम् ता हृ॒दा हृ॒दा ता माम् । \newline
12. ता माम् माम् ता स्ता माम् का॑मयन्ताम् कामयन्ता॒म् माम् ता स्ता माम् का॑मयन्ताम् । \newline
13. माम् का॑मयन्ताम् कामयन्ता॒म् माम् माम् का॑मयन्ताꣳ हृ॒दा हृ॒दा का॑मयन्ता॒म् माम् माम् का॑मयन्ताꣳ हृ॒दा । \newline
14. का॒म॒य॒न्ताꣳ॒॒ हृ॒दा हृ॒दा का॑मयन्ताम् कामयन्ताꣳ हृ॒दा ता स्ता हृ॒दा का॑मयन्ताम् कामयन्ताꣳ हृ॒दा ताः । \newline
15. हृ॒दा ता स्ता हृ॒दा हृ॒दा ता मे॑ मे॒ ता हृ॒दा हृ॒दा ता मे᳚ । \newline
16. ता मे॑ मे॒ ता स्ता म॒ आम॑नस॒ आम॑नसो मे॒ ता स्ता म॒ आम॑नसः । \newline
17. म॒ आम॑नस॒ आम॑नसो मे म॒ आम॑नसः कृधि कृ॒ध्याम॑नसो मे म॒ आम॑नसः कृधि । \newline
18. आम॑नसः कृधि कृ॒ध्याम॑नस॒ आम॑नसः कृधि॒ स्वाहा॒ स्वाहा॑ कृ॒ध्याम॑नस॒ आम॑नसः कृधि॒ स्वाहा᳚ । \newline
19. आम॑नस॒ इत्या - म॒न॒सः॒ । \newline
20. कृ॒धि॒ स्वाहा॒ स्वाहा॑ कृधि कृधि॒ स्वाहा॑ वैश्वदे॒वीं ॅवै᳚श्वदे॒वीꣳ स्वाहा॑ कृधि कृधि॒ स्वाहा॑ वैश्वदे॒वीम् । \newline
21. स्वाहा॑ वैश्वदे॒वीं ॅवै᳚श्वदे॒वीꣳ स्वाहा॒ स्वाहा॑ वैश्वदे॒वीꣳ सा᳚ङ्ग्रह॒णीꣳ सा᳚ङ्ग्रह॒णीं ॅवै᳚श्वदे॒वीꣳ स्वाहा॒ स्वाहा॑ वैश्वदे॒वीꣳ सा᳚ङ्ग्रह॒णीम् । \newline
22. वै॒श्व॒दे॒वीꣳ सा᳚ङ्ग्रह॒णीꣳ सा᳚ङ्ग्रह॒णीं ॅवै᳚श्वदे॒वीं ॅवै᳚श्वदे॒वीꣳ सा᳚ङ्ग्रह॒णीम् निर् णिः सा᳚ङ्ग्रह॒णीं ॅवै᳚श्वदे॒वीं ॅवै᳚श्वदे॒वीꣳ सा᳚ङ्ग्रह॒णीम् निः । \newline
23. वै॒श्व॒दे॒वीमिति॑ वैश्व - दे॒वीम् । \newline
24. सा॒ङ्ग्र॒ह॒णीम् निर् णिः सा᳚ङ्ग्रह॒णीꣳ सा᳚ङ्ग्रह॒णीम् निर् व॑पेद् वपे॒न् निः सा᳚ङ्ग्रह॒णीꣳ सा᳚ङ्ग्रह॒णीम् निर् व॑पेत् । \newline
25. सा॒ङ्ग्र॒ह॒णीमिति॑ सां - ग्र॒ह॒णीम् । \newline
26. निर् व॑पेद् वपे॒न् निर् णिर् व॑पे॒द् ग्राम॑कामो॒ ग्राम॑कामो वपे॒न् निर् णिर् व॑पे॒द् ग्राम॑कामः । \newline
27. व॒पे॒द् ग्राम॑कामो॒ ग्राम॑कामो वपेद् वपे॒द् ग्राम॑कामो वैश्वदे॒वा वै᳚श्वदे॒वा ग्राम॑कामो वपेद् वपे॒द् ग्राम॑कामो वैश्वदे॒वाः । \newline
28. ग्राम॑कामो वैश्वदे॒वा वै᳚श्वदे॒वा ग्राम॑कामो॒ ग्राम॑कामो वैश्वदे॒वा वै वै वै᳚श्वदे॒वा ग्राम॑कामो॒ ग्राम॑कामो वैश्वदे॒वा वै । \newline
29. ग्राम॑काम॒ इति॒ ग्राम॑ - का॒मः॒ । \newline
30. वै॒श्व॒दे॒वा वै वै वै᳚श्वदे॒वा वै᳚श्वदे॒वा वै स॑जा॒ताः स॑जा॒ता वै वै᳚श्वदे॒वा वै᳚श्वदे॒वा वै स॑जा॒ताः । \newline
31. वै॒श्व॒दे॒वा इति॑ वैश्व - दे॒वाः । \newline
32. वै स॑जा॒ताः स॑जा॒ता वै वै स॑जा॒ता विश्वा॒न्॒. विश्वा᳚न् थ्सजा॒ता वै वै स॑जा॒ता विश्वान्॑ । \newline
33. स॒जा॒ता विश्वा॒न्॒. विश्वा᳚न् थ्सजा॒ताः स॑जा॒ता विश्वा॑ ने॒वैव विश्वा᳚न् थ्सजा॒ताः स॑जा॒ता विश्वा॑ ने॒व । \newline
34. स॒जा॒ता इति॑ स - जा॒ताः । \newline
35. विश्वा॑ ने॒वैव विश्वा॒न्॒. विश्वा॑ ने॒व दे॒वान् दे॒वा ने॒व विश्वा॒न्॒. विश्वा॑ ने॒व दे॒वान् । \newline
36. ए॒व दे॒वान् दे॒वा ने॒वैव दे॒वान् थ्स्वेन॒ स्वेन॑ दे॒वा ने॒वैव दे॒वान् थ्स्वेन॑ । \newline
37. दे॒वान् थ्स्वेन॒ स्वेन॑ दे॒वान् दे॒वान् थ्स्वेन॑ भाग॒धेये॑न भाग॒धेये॑न॒ स्वेन॑ दे॒वान् दे॒वान् थ्स्वेन॑ भाग॒धेये॑न । \newline
38. स्वेन॑ भाग॒धेये॑न भाग॒धेये॑न॒ स्वेन॒ स्वेन॑ भाग॒धेये॒नोपोप॑ भाग॒धेये॑न॒ स्वेन॒ स्वेन॑ भाग॒धेये॒नोप॑ । \newline
39. भा॒ग॒धेये॒नोपोप॑ भाग॒धेये॑न भाग॒धेये॒नोप॑ धावति धाव॒त्युप॑ भाग॒धेये॑न भाग॒धेये॒नोप॑ धावति । \newline
40. भा॒ग॒धेये॒नेति॑ भाग - धेये॑न । \newline
41. उप॑ धावति धाव॒ त्युपोप॑ धावति॒ ते ते धा॑व॒ त्युपोप॑ धावति॒ ते । \newline
42. धा॒व॒ति॒ ते ते धा॑वति धावति॒ त ए॒वैव ते धा॑वति धावति॒ त ए॒व । \newline
43. त ए॒वैव ते त ए॒वास्मा॑ अस्मा ए॒व ते त ए॒वास्मै᳚ । \newline
44. ए॒वास्मा॑ अस्मा ए॒वैवास्मै॑ सजा॒तान् थ्स॑जा॒ता न॑स्मा ए॒वैवास्मै॑ सजा॒तान् । \newline
45. अ॒स्मै॒ स॒जा॒तान् थ्स॑जा॒ता न॑स्मा अस्मै सजा॒तान् प्र प्र स॑जा॒ता न॑स्मा अस्मै सजा॒तान् प्र । \newline
46. स॒जा॒तान् प्र प्र स॑जा॒तान् थ्स॑जा॒तान् प्र य॑च्छन्ति यच्छन्ति॒ प्र स॑जा॒तान् थ्स॑जा॒तान् प्र य॑च्छन्ति । \newline
47. स॒जा॒तानिति॑ स - जा॒तान् । \newline
48. प्र य॑च्छन्ति यच्छन्ति॒ प्र प्र य॑च्छन्ति ग्रा॒मी ग्रा॒मी य॑च्छन्ति॒ प्र प्र य॑च्छन्ति ग्रा॒मी । \newline
49. य॒च्छ॒न्ति॒ ग्रा॒मी ग्रा॒मी य॑च्छन्ति यच्छन्ति ग्रा॒म्ये॑वैव ग्रा॒मी य॑च्छन्ति यच्छन्ति ग्रा॒म्ये॑व । \newline
50. ग्रा॒म्ये॑वैव ग्रा॒मी ग्रा॒म्ये॑व भ॑वति भवत्ये॒व ग्रा॒मी ग्रा॒म्ये॑व भ॑वति । \newline
51. ए॒व भ॑वति भवत्ये॒वैव भ॑वति साङ्ग्रह॒णी सा᳚ङ्ग्रह॒णी भ॑वत्ये॒वैव भ॑वति साङ्ग्रह॒णी । \newline
52. भ॒व॒ति॒ सा॒ङ्ग्र॒ह॒णी सा᳚ङ्ग्रह॒णी भ॑वति भवति साङ्ग्रह॒णी भ॑वति भवति साङ्ग्रह॒णी भ॑वति भवति साङ्ग्रह॒णी भ॑वति । \newline
53. सा॒ङ्ग्र॒ह॒णी भ॑वति भवति साङ्ग्रह॒णी सा᳚ङ्ग्रह॒णी भ॑वति मनो॒ग्रह॑णम् मनो॒ग्रह॑णम् भवति साङ्ग्रह॒णी सा᳚ङ्ग्रह॒णी भ॑वति मनो॒ग्रह॑णम् । \newline
54. सा॒ङ्ग्र॒ह॒णीति॑ सां - ग्र॒ह॒णी । \newline
55. भ॒व॒ति॒ म॒नो॒ग्रह॑णम् मनो॒ग्रह॑णम् भवति भवति मनो॒ग्रह॑णं॒ ॅवै वै म॑नो॒ग्रह॑णम् भवति भवति मनो॒ग्रह॑णं॒ ॅवै । \newline
56. म॒नो॒ग्रह॑णं॒ ॅवै वै म॑नो॒ग्रह॑णम् मनो॒ग्रह॑णं॒ ॅवै स॒ङ्ग्रह॑णꣳ स॒ङ्ग्रह॑णं॒ ॅवै म॑नो॒ग्रह॑णम् मनो॒ग्रह॑णं॒ ॅवै स॒ङ्ग्रह॑णम् । \newline
57. म॒नो॒ग्रह॑ण॒मिति॑ मनः - ग्रह॑णम् । \newline
58. वै स॒ङ्ग्रह॑णꣳ स॒ङ्ग्रह॑णं॒ ॅवै वै स॒ङ्ग्रह॑ण॒म् मनो॒ मनः॑ स॒ङ्ग्रह॑णं॒ ॅवै वै स॒ङ्ग्रह॑ण॒म् मनः॑ । \newline
59. स॒ङ्ग्रह॑ण॒म् मनो॒ मनः॑ स॒ङ्ग्रह॑णꣳ स॒ङ्ग्रह॑ण॒म् मन॑ ए॒वैव मनः॑ स॒ङ्ग्रह॑णꣳ स॒ङ्ग्रह॑ण॒म् मन॑ ए॒व । \newline
60. स॒ङ्ग्रह॑ण॒मिति॑ सं - ग्रह॑णम् । \newline
61. मन॑ ए॒वैव मनो॒ मन॑ ए॒व स॑जा॒तानाꣳ॑ सजा॒ताना॑ मे॒व मनो॒ मन॑ ए॒व स॑जा॒ताना᳚म् । \newline
62. ए॒व स॑जा॒तानाꣳ॑ सजा॒ताना॑ मे॒वैव स॑जा॒ताना᳚म् गृह्णाति गृह्णाति सजा॒ताना॑ मे॒वैव स॑जा॒ताना᳚म् गृह्णाति । \newline
63. स॒जा॒ताना᳚म् गृह्णाति गृह्णाति सजा॒तानाꣳ॑ सजा॒ताना᳚म् गृह्णाति ध्रु॒वो ध्रु॒वो गृ॑ह्णाति सजा॒तानाꣳ॑ सजा॒ताना᳚म् गृह्णाति ध्रु॒वः । \newline
64. स॒जा॒ताना॒मिति॑ स - जा॒ताना᳚म् । \newline
\pagebreak
\markright{ TS 2.3.9.3  \hfill https://www.vedavms.in \hfill}

\section{ TS 2.3.9.3 }

\textbf{TS 2.3.9.3 } \newline
\textbf{Samhita Paata} \newline

गृह्णाति ध्रु॒वो॑ऽसि ध्रु॒वो॑ऽहꣳ स॑जा॒तेषु॑ भूयास॒मिति॑ परि॒धीन् परि॑ दधात्या॒शिष॑मे॒वैतामा शा॒स्तेऽथो॑ ए॒तदे॒व सर्वꣳ॑ सजा॒तेष्वधि॑ भवति॒ यस्यै॒वं ॅवि॒दुष॑ ए॒ते प॑रि॒धयः॑ परिधी॒यन्त॒ आम॑ नम॒स्याम॑नस्य देवा॒ इति॑ ति॒स्र आहु॑ती र्जुहोत्ये॒ताव॑न्तो॒ वै स॑जा॒ता ये म॒हान्तो॒ ये क्षु॑ल्ल॒का याः स्त्रिय॒स्ताने॒वाव॑ रुन्धे॒ ( ) त ए॑न॒मव॑रुद्धा॒ उप॑ तिष्ठन्ते ॥ \newline

\textbf{Pada Paata} \newline

गृ॒ह्णा॒ति॒ । ध्रु॒वः । अ॒सि॒ । ध्रु॒वः । अ॒हम् । स॒जा॒तेष्विति॑ स-जा॒तेषु॑ । भू॒या॒स॒म् । इति॑ । प॒रि॒धीनिति॑ परि - धीन् । परीति॑ । द॒धा॒ति॒ । आ॒शिष॒मित्या᳚ - शिष᳚म् । ए॒व । ए॒ताम् । एति॑ । शा॒स्ते॒ । अथो॒ इति॑ । ए॒तत् । ए॒व । सर्व᳚म् । स॒जा॒तेष्विति॑ स-जा॒तेषु॑ । अधीति॑ । भ॒व॒ति॒ । यस्य॑ । ए॒वम् । वि॒दुषः॑ । ए॒ते । प॒रि॒धय॒ इति॑ परि - धयः॑ । प॒रि॒धी॒यन्त॒ इति॑ परि - धी॒यन्ते᳚ । आम॑न॒मित्या - म॒न॒म् । अ॒सि॒ । आम॑न॒स्येत्या - म॒न॒स्य॒ । दे॒वाः॒ । इति॑ । ति॒स्रः । आहु॑ती॒रित्या - हु॒तीः॒ । जु॒हो॒ति॒ । ए॒ताव॑न्तः । वै । स॒जा॒ता इति॑ स-जा॒ताः । ये । म॒हान्तः॑ । ये । क्षु॒ल्ल॒काः । याः । स्त्रियः॑ । तान् ।  ए॒व । अवेति॑ । रु॒न्धे॒ ( ) । ते । ए॒न॒म् । अव॑रुद्धा॒ इत्यव॑ - रु॒द्धाः॒ । उपेति॑ । ति॒ष्ठ॒न्ते॒ ॥  \newline


\textbf{Krama Paata} \newline

गृ॒ह्णा॒ति॒ ध्रु॒वः । ध्रु॒वो॑ऽसि । अ॒सि॒ ध्रु॒वः । ध्रु॒वो॑ऽहम् । अ॒हꣳ स॑जा॒तेषु॑ । स॒जा॒तेषु॑ भूयासम् । स॒जा॒तेष्विति॑ स - जा॒तेषु॑ । भू॒या॒स॒मिति॑ । इति॑ परि॒धीन् । प॒रि॒धीन् परि॑ । प॒रि॒धीनिति॑ परि - धीन् । परि॑ दधाति । द॒धा॒त्या॒शिष᳚म् । आ॒शिष॑मे॒व । आ॒शिष॒मित्या᳚ - शिष᳚म् । ए॒वैताम् । ए॒तामा । आ शा᳚स्ते । शा॒स्ते ऽथो᳚ । अथो॑ ए॒तत् । अथो॒ इत्यथो᳚ । ए॒तदे॒व । ए॒व सर्व᳚म् । सर्वꣳ॑ सजा॒तेषु॑ । स॒जा॒तेष्वधि॑ । स॒जा॒तेष्विति॑ स - जा॒तेषु॑ । अधि॑ भवति । भ॒व॒ति॒ यस्य॑ । यस्यै॒वम् । ए॒वं ॅवि॒दुषः॑ । वि॒दुष॑ ए॒ते । ए॒ते प॑रि॒धयः॑ । प॒रि॒धयः॑ परिधी॒यन्ते᳚ । प॒रि॒धय॒ इति॑ परि - धयः॑ । प॒रि॒धी॒यन्त॒ आम॑नम् । प॒रि॒धी॒यन्त॒ इति॑ परि - धी॒यन्ते᳚ । आम॑नमसि । आम॑न॒मित्या - म॒न॒म् । अ॒स्याम॑नस्य । आम॑नस्य देवाः । आम॑न॒स्येत्या - म॒न॒स्य॒ । दे॒वा॒ इति॑ । इति॑ ति॒स्रः । ति॒स्र आहु॑तीः । आहु॑तीर् जुहोति । आहु॑ती॒रित्या - हु॒तीः॒ । जु॒हो॒त्ये॒ताव॑न्तः । ए॒ताव॑न्तो॒ वै । वै स॑जा॒ताः । स॒जा॒ता ये । स॒जा॒ता इति॑ स - जा॒ताः । ये म॒हान्तः॑ । म॒हान्तो॒ ये । ये क्षु॑ल्ल॒काः । क्षु॒ल्ल॒का याः । याः स्त्रियः॑ । स्त्रिय॒स्तान् । ताने॒व । ए॒वाव॑ । अव॑ रुन्धे ( ) । रु॒न्धे॒ ते । त ए॑नम् । ए॒न॒मव॑रुद्धाः । अव॑रुद्धा॒ उप॑ । अव॑रुद्धा॒ इत्यव॑ - रु॒द्धाः॒ । उप॑ तिष्ठन्ते । ति॒ष्ठ॒न्त॒ इति॑ तिष्ठन्ते । \newline

\textbf{Jatai Paata} \newline

1. गृ॒ह्णा॒ति॒ ध्रु॒वो ध्रु॒वो गृ॑ह्णाति गृह्णाति ध्रु॒वः । \newline
2. ध्रु॒वो᳚ ऽस्यसि ध्रु॒वो ध्रु॒वो॑ ऽसि । \newline
3. अ॒सि॒ ध्रु॒वो ध्रु॒वो᳚ ऽस्यसि ध्रु॒वः । \newline
4. ध्रु॒वो॑ ऽह म॒हम् ध्रु॒वो ध्रु॒वो॑ ऽहम् । \newline
5. अ॒हꣳ स॑जा॒तेषु॑ सजा॒ते ष्व॒ह म॒हꣳ स॑जा॒तेषु॑ । \newline
6. स॒जा॒तेषु॑ भूयासम् भूयासꣳ सजा॒तेषु॑ सजा॒तेषु॑ भूयासम् । \newline
7. स॒जा॒तेष्विति॑ स - जा॒तेषु॑ । \newline
8. भू॒या॒स॒ मितीति॑ भूयासम् भूयास॒ मिति॑ । \newline
9. इति॑ परि॒धीन् प॑रि॒धी नितीति॑ परि॒धीन् । \newline
10. प॒रि॒धीन् परि॒ परि॑ परि॒धीन् प॑रि॒धीन् परि॑ । \newline
11. प॒रि॒धीनिति॑ परि - धीन् । \newline
12. परि॑ दधाति दधाति॒ परि॒ परि॑ दधाति । \newline
13. द॒धा॒ त्या॒शिष॑ मा॒शिष॑म् दधाति दधा त्या॒शिष᳚म् । \newline
14. आ॒शिष॑ मे॒वैवाशिष॑ मा॒शिष॑ मे॒व । \newline
15. आ॒शिष॒मित्या᳚ - शिष᳚म् । \newline
16. ए॒वैता मे॒ता मे॒वैवैताम् । \newline
17. ए॒ता मैता मे॒ता मा । \newline
18. आ शा᳚स्ते शास्त॒ आ शा᳚स्ते । \newline
19. शा॒स्ते ऽथो॒ अथो॑ शास्ते शा॒स्ते ऽथो᳚ । \newline
20. अथो॑ ए॒त दे॒त दथो॒ अथो॑ ए॒तत् । \newline
21. अथो॒ इत्यथो᳚ । \newline
22. ए॒त दे॒वैवैत दे॒त दे॒व । \newline
23. ए॒व सर्वꣳ॒॒ सर्व॑ मे॒वैव सर्व᳚म् । \newline
24. सर्वꣳ॑ सजा॒तेषु॑ सजा॒तेषु॒ सर्वꣳ॒॒ सर्वꣳ॑ सजा॒तेषु॑ । \newline
25. स॒जा॒ते ष्वध्यधि॑ सजा॒तेषु॑ सजा॒ते ष्वधि॑ । \newline
26. स॒जा॒तेष्विति॑ स - जा॒तेषु॑ । \newline
27. अधि॑ भवति भव॒ त्यध्यधि॑ भवति । \newline
28. भ॒व॒ति॒ यस्य॒ यस्य॑ भवति भवति॒ यस्य॑ । \newline
29. यस्यै॒व मे॒वं ॅयस्य॒ यस्यै॒वम् । \newline
30. ए॒वं ॅवि॒दुषो॑ वि॒दुष॑ ए॒व मे॒वं ॅवि॒दुषः॑ । \newline
31. वि॒दुष॑ ए॒त ए॒ते वि॒दुषो॑ वि॒दुष॑ ए॒ते । \newline
32. ए॒ते प॑रि॒धयः॑ परि॒धय॑ ए॒त ए॒ते प॑रि॒धयः॑ । \newline
33. प॒रि॒धयः॑ परिधी॒यन्ते॑ परिधी॒यन्ते॑ परि॒धयः॑ परि॒धयः॑ परिधी॒यन्ते᳚ । \newline
34. प॒रि॒धय॒ इति॑ परि - धयः॑ । \newline
35. प॒रि॒धी॒यन्त॒ आम॑न॒ माम॑नम् परिधी॒यन्ते॑ परिधी॒यन्त॒ आम॑नम् । \newline
36. प॒रि॒धी॒यन्त॒ इति॑ परि - धी॒यन्ते᳚ । \newline
37. आम॑न मस्य॒स्याम॑न॒ माम॑न मसि । \newline
38. आम॑न॒मित्या - म॒न॒म् । \newline
39. अ॒स्या म॑न॒स्या म॑नस्या स्य॒स्या म॑नस्य । \newline
40. आम॑नस्य देवा देवा॒ आम॑न॒स्या म॑नस्य देवाः । \newline
41. आम॑न॒स्येत्या - म॒न॒स्य॒ । \newline
42. दे॒वा॒ इतीति॑ देवा देवा॒ इति॑ । \newline
43. इति॑ ति॒स्र स्ति॒स्र इतीति॑ ति॒स्रः । \newline
44. ति॒स्र आहु॑ती॒ राहु॑ती स्ति॒स्र स्ति॒स्र आहु॑तीः । \newline
45. आहु॑तीर् जुहोति जुहो॒ त्याहु॑ती॒ राहु॑तीर् जुहोति । \newline
46. आहु॑ती॒रित्या - हु॒तीः॒ । \newline
47. जु॒हो॒ त्ये॒ताव॑न्त ए॒ताव॑न्तो जुहोति जुहो त्ये॒ताव॑न्तः । \newline
48. ए॒ताव॑न्तो॒ वै वा ए॒ताव॑न्त ए॒ताव॑न्तो॒ वै । \newline
49. वै स॑जा॒ताः स॑जा॒ता वै वै स॑जा॒ताः । \newline
50. स॒जा॒ता ये ये स॑जा॒ताः स॑जा॒ता ये । \newline
51. स॒जा॒ता इति॑ स - जा॒ताः । \newline
52. ये म॒हान्तो॑ म॒हान्तो॒ ये ये म॒हान्तः॑ । \newline
53. म॒हान्तो॒ ये ये म॒हान्तो॑ म॒हान्तो॒ ये । \newline
54. ये क्षु॑ल्ल॒काः क्षु॑ल्ल॒का ये ये क्षु॑ल्ल॒काः । \newline
55. क्षु॒ल्ल॒का या याः क्षु॑ल्ल॒काः क्षु॑ल्ल॒का याः । \newline
56. याः स्त्रियः॒ स्त्रियो॒ या याः स्त्रियः॑ । \newline
57. स्त्रिय॒ स्ताꣳ स्तान् थ्‌स्त्रियः॒ स्त्रिय॒ स्तान् । \newline
58. ता ने॒वैव ताꣳ स्ता ने॒व । \newline
59. ए॒वावा वै॒वै वाव॑ । \newline
60. अव॑ रुन्धे रु॒न्धे ऽवाव॑ रुन्धे । \newline
61. रु॒न्धे॒ ते ते रु॑न्धे रुन्धे॒ ते । \newline
62. त ए॑न मेन॒म् ते त ए॑नम् । \newline
63. ए॒न॒ मव॑रुद्धा॒ अव॑रुद्धा एन मेन॒ मव॑रुद्धाः । \newline
64. अव॑रुद्धा॒ उपोपाव॑रुद्धा॒ अव॑रुद्धा॒ उप॑ । \newline
65. अव॑रुद्धा॒ इत्यव॑ - रु॒द्धाः॒ । \newline
66. उप॑ तिष्ठन्ते तिष्ठन्त॒ उपोप॑ तिष्ठन्ते । \newline
67. ति॒ष्ठ॒न्त॒ इति॑ तिष्ठन्ते । \newline

\textbf{Ghana Paata } \newline

1. गृ॒ह्णा॒ति॒ ध्रु॒वो ध्रु॒वो गृ॑ह्णाति गृह्णाति ध्रु॒वो᳚ ऽस्यसि ध्रु॒वो गृ॑ह्णाति गृह्णाति ध्रु॒वो॑ ऽसि । \newline
2. ध्रु॒वो᳚ ऽस्यसि ध्रु॒वो ध्रु॒वो॑ ऽसि ध्रु॒वो ध्रु॒वो॑ ऽसि ध्रु॒वो ध्रु॒वो॑ ऽसि ध्रु॒वः । \newline
3. अ॒सि॒ ध्रु॒वो ध्रु॒वो᳚ ऽस्यसि ध्रु॒वो॑ ऽह म॒हम् ध्रु॒वो᳚ ऽस्यसि ध्रु॒वो॑ ऽहम् । \newline
4. ध्रु॒वो॑ ऽह म॒हम् ध्रु॒वो ध्रु॒वो॑ ऽहꣳ स॑जा॒तेषु॑ सजा॒ते ष्व॒हम् ध्रु॒वो ध्रु॒वो॑ ऽहꣳ स॑जा॒तेषु॑ । \newline
5. अ॒हꣳ स॑जा॒तेषु॑ सजा॒ते ष्व॒ह म॒हꣳ स॑जा॒तेषु॑ भूयासम् भूयासꣳ सजा॒ते ष्व॒ह म॒हꣳ स॑जा॒तेषु॑ भूयासम् । \newline
6. स॒जा॒तेषु॑ भूयासम् भूयासꣳ सजा॒तेषु॑ सजा॒तेषु॑ भूयास॒ मितीति॑ भूयासꣳ सजा॒तेषु॑ सजा॒तेषु॑ भूयास॒ मिति॑ । \newline
7. स॒जा॒तेष्विति॑ स - जा॒तेषु॑ । \newline
8. भू॒या॒स॒ मितीति॑ भूयासम् भूयास॒ मिति॑ परि॒धीन् प॑रि॒धी निति॑ भूयासम् भूयास॒ मिति॑ परि॒धीन् । \newline
9. इति॑ परि॒धीन् प॑रि॒धी नितीति॑ परि॒धीन् परि॒ परि॑ परि॒धी नितीति॑ परि॒धीन् परि॑ । \newline
10. प॒रि॒धीन् परि॒ परि॑ परि॒धीन् प॑रि॒धीन् परि॑ दधाति दधाति॒ परि॑ परि॒धीन् प॑रि॒धीन् परि॑ दधाति । \newline
11. प॒रि॒धीनिति॑ परि - धीन् । \newline
12. परि॑ दधाति दधाति॒ परि॒ परि॑ दधा त्या॒शिष॑ मा॒शिष॑म् दधाति॒ परि॒ परि॑ दधा त्या॒शिष᳚म् । \newline
13. द॒धा॒ त्या॒शिष॑ मा॒शिष॑म् दधाति दधा त्या॒शिष॑ मे॒वैवाशिष॑म् दधाति दधा त्या॒शिष॑ मे॒व । \newline
14. आ॒शिष॑ मे॒वैवाशिष॑ मा॒शिष॑ मे॒वैता मे॒ता मे॒वाशिष॑ मा॒शिष॑ मे॒वैताम् । \newline
15. आ॒शिष॒मित्या᳚ - शिष᳚म् । \newline
16. ए॒वैता मे॒ता मे॒वैवैता मैता मे॒वैवैता मा । \newline
17. ए॒ता मैता मे॒ता मा शा᳚स्ते शास्त॒ ऐता मे॒ता मा शा᳚स्ते । \newline
18. आ शा᳚स्ते शास्त॒ आ शा॒स्ते ऽथो॒ अथो॑ शास्त॒ आ शा॒स्ते ऽथो᳚ । \newline
19. शा॒स्ते ऽथो॒ अथो॑ शास्ते शा॒स्ते ऽथो॑ ए॒त दे॒त दथो॑ शास्ते शा॒स्ते ऽथो॑ ए॒तत् । \newline
20. अथो॑ ए॒त दे॒त दथो॒ अथो॑ ए॒त दे॒वैवैत दथो॒ अथो॑ ए॒तदे॒व । \newline
21. अथो॒ इत्यथो᳚ । \newline
22. ए॒त दे॒वैवैत दे॒त दे॒व सर्वꣳ॒॒ सर्व॑ मे॒वैत दे॒त दे॒व सर्व᳚म् । \newline
23. ए॒व सर्वꣳ॒॒ सर्व॑ मे॒वैव सर्वꣳ॑ सजा॒तेषु॑ सजा॒तेषु॒ सर्व॑ मे॒वैव सर्वꣳ॑ सजा॒तेषु॑ । \newline
24. सर्वꣳ॑ सजा॒तेषु॑ सजा॒तेषु॒ सर्वꣳ॒॒ सर्वꣳ॑ सजा॒ते ष्वध्यधि॑ सजा॒तेषु॒ सर्वꣳ॒॒ सर्वꣳ॑ सजा॒ते ष्वधि॑ । \newline
25. स॒जा॒ते ष्वध्यधि॑ सजा॒तेषु॑ सजा॒ते ष्वधि॑ भवति भव॒ त्यधि॑ सजा॒तेषु॑ सजा॒ते ष्वधि॑ भवति । \newline
26. स॒जा॒तेष्विति॑ स - जा॒तेषु॑ । \newline
27. अधि॑ भवति भव॒ त्यध्यधि॑ भवति॒ यस्य॒ यस्य॑ भव॒ त्यध्यधि॑ भवति॒ यस्य॑ । \newline
28. भ॒व॒ति॒ यस्य॒ यस्य॑ भवति भवति॒ यस्यै॒व मे॒वं ॅयस्य॑ भवति भवति॒ यस्यै॒वम् । \newline
29. यस्यै॒व मे॒वं ॅयस्य॒ यस्यै॒वं ॅवि॒दुषो॑ वि॒दुष॑ ए॒वं ॅयस्य॒ यस्यै॒वं ॅवि॒दुषः॑ । \newline
30. ए॒वं ॅवि॒दुषो॑ वि॒दुष॑ ए॒व मे॒वं ॅवि॒दुष॑ ए॒त ए॒ते वि॒दुष॑ ए॒व मे॒वं ॅवि॒दुष॑ ए॒ते । \newline
31. वि॒दुष॑ ए॒त ए॒ते वि॒दुषो॑ वि॒दुष॑ ए॒ते प॑रि॒धयः॑ परि॒धय॑ ए॒ते वि॒दुषो॑ वि॒दुष॑ ए॒ते प॑रि॒धयः॑ । \newline
32. ए॒ते प॑रि॒धयः॑ परि॒धय॑ ए॒त ए॒ते प॑रि॒धयः॑ परिधी॒यन्ते॑ परिधी॒यन्ते॑ परि॒धय॑ ए॒त ए॒ते प॑रि॒धयः॑ परिधी॒यन्ते᳚ । \newline
33. प॒रि॒धयः॑ परिधी॒यन्ते॑ परिधी॒यन्ते॑ परि॒धयः॑ परि॒धयः॑ परिधी॒यन्त॒ आम॑न॒ माम॑नम् परिधी॒यन्ते॑ परि॒धयः॑ परि॒धयः॑ परिधी॒यन्त॒ आम॑नम् । \newline
34. प॒रि॒धय॒ इति॑ परि - धयः॑ । \newline
35. प॒रि॒धी॒यन्त॒ आम॑न॒ माम॑नम् परिधी॒यन्ते॑ परिधी॒यन्त॒ आम॑न मस्य॒स्याम॑नम् परिधी॒यन्ते॑ परिधी॒यन्त॒ आम॑न मसि । \newline
36. प॒रि॒धी॒यन्त॒ इति॑ परि - धी॒यन्ते᳚ । \newline
37. आम॑न मस्य॒स्याम॑न॒ माम॑न म॒स्या म॑न॒स्या म॑नस्या॒ स्याम॑न॒ माम॑न म॒स्याम॑नस्य । \newline
38. आम॑न॒मित्या - म॒न॒म् । \newline
39. अ॒स्या म॑न॒स्या म॑न स्यास्य॒ स्याम॑नस्य देवा देवा॒ आम॑नस्या स्य॒स्या म॑नस्य देवाः । \newline
40. आम॑नस्य देवा देवा॒ आम॑न॒स्या म॑नस्य देवा॒ इतीति॑ देवा॒ आम॑न॒स्या म॑नस्य देवा॒ इति॑ । \newline
41. आम॑न॒स्येत्या - म॒न॒स्य॒ । \newline
42. दे॒वा॒ इतीति॑ देवा देवा॒ इति॑ ति॒स्र स्ति॒स्र इति॑ देवा देवा॒ इति॑ ति॒स्रः । \newline
43. इति॑ ति॒स्र स्ति॒स्र इतीति॑ ति॒स्र आहु॑ती॒ राहु॑ती स्ति॒स्र इतीति॑ ति॒स्र आहु॑तीः । \newline
44. ति॒स्र आहु॑ती॒ राहु॑ती स्ति॒स्र स्ति॒स्र आहु॑तीर् जुहोति जुहो॒ त्याहु॑ती स्ति॒स्र स्ति॒स्र आहु॑तीर् जुहोति । \newline
45. आहु॑तीर् जुहोति जुहो॒ त्याहु॑ती॒ राहु॑तीर् जुहो त्ये॒ताव॑न्त ए॒ताव॑न्तो जुहो॒ त्याहु॑ती॒ राहु॑तीर् जुहो त्ये॒ताव॑न्तः । \newline
46. आहु॑ती॒रित्या - हु॒तीः॒ । \newline
47. जु॒हो॒ त्ये॒ताव॑न्त ए॒ताव॑न्तो जुहोति जुहो त्ये॒ताव॑न्तो॒ वै वा ए॒ताव॑न्तो जुहोति जुहो त्ये॒ताव॑न्तो॒ वै । \newline
48. ए॒ताव॑न्तो॒ वै वा ए॒ताव॑न्त ए॒ताव॑न्तो॒ वै स॑जा॒ताः स॑जा॒ता वा ए॒ताव॑न्त ए॒ताव॑न्तो॒ वै स॑जा॒ताः । \newline
49. वै स॑जा॒ताः स॑जा॒ता वै वै स॑जा॒ता ये ये स॑जा॒ता वै वै स॑जा॒ता ये । \newline
50. स॒जा॒ता ये ये स॑जा॒ताः स॑जा॒ता ये म॒हान्तो॑ म॒हान्तो॒ ये स॑जा॒ताः स॑जा॒ता ये म॒हान्तः॑ । \newline
51. स॒जा॒ता इति॑ स - जा॒ताः । \newline
52. ये म॒हान्तो॑ म॒हान्तो॒ ये ये म॒हान्तो॒ ये ये म॒हान्तो॒ ये ये म॒हान्तो॒ ये । \newline
53. म॒हान्तो॒ ये ये म॒हान्तो॑ म॒हान्तो॒ ये क्षु॑ल्ल॒काः क्षु॑ल्ल॒का ये म॒हान्तो॑ म॒हान्तो॒ ये क्षु॑ल्ल॒काः । \newline
54. ये क्षु॑ल्ल॒काः क्षु॑ल्ल॒का ये ये क्षु॑ल्ल॒का या याः क्षु॑ल्ल॒का ये ये क्षु॑ल्ल॒का याः । \newline
55. क्षु॒ल्ल॒का या याः क्षु॑ल्ल॒काः क्षु॑ल्ल॒का याः स्त्रियः॒ स्त्रियो॒ याः क्षु॑ल्ल॒काः क्षु॑ल्ल॒का याः स्त्रियः॑ । \newline
56. याः स्त्रियः॒ स्त्रियो॒ या याः स्त्रिय॒ स्ताꣳ स्तान् थ्‌स्त्रियो॒ या याः स्त्रिय॒ स्तान् । \newline
57. स्त्रिय॒ स्ताꣳ स्तान् थ्‌स्त्रियः॒ स्त्रिय॒ स्ता ने॒वैव तान् थ्‌स्त्रियः॒ स्त्रिय॒ स्ता ने॒व । \newline
58. ता ने॒वैव ताꣳ स्ता ने॒वावावै॒व ताꣳ स्ता ने॒वाव॑ । \newline
59. ए॒वावा वै॒वैवाव॑ रुन्धे रु॒न्धे ऽवै॒वैवाव॑ रुन्धे । \newline
60. अव॑ रुन्धे रु॒न्धे ऽवाव॑ रुन्धे॒ ते ते रु॒न्धे ऽवाव॑ रुन्धे॒ ते । \newline
61. रु॒न्धे॒ ते ते रु॑न्धे रुन्धे॒ त ए॑न मेन॒म् ते रु॑न्धे रुन्धे॒ त ए॑नम् । \newline
62. त ए॑न मेन॒म् ते त ए॑न॒ मव॑रुद्धा॒ अव॑रुद्धा एन॒म् ते त ए॑न॒ मव॑रुद्धाः । \newline
63. ए॒न॒ मव॑रुद्धा॒ अव॑रुद्धा एन मेन॒ मव॑रुद्धा॒ उपोपाव॑रुद्धा एन मेन॒ मव॑रुद्धा॒ उप॑ । \newline
64. अव॑रुद्धा॒ उपोपाव॑रुद्धा॒ अव॑रुद्धा॒ उप॑ तिष्ठन्ते तिष्ठन्त॒ उपाव॑रुद्धा॒ अव॑रुद्धा॒ उप॑ तिष्ठन्ते । \newline
65. अव॑रुद्धा॒ इत्यव॑ - रु॒द्धाः॒ । \newline
66. उप॑ तिष्ठन्ते तिष्ठन्त॒ उपोप॑ तिष्ठन्ते । \newline
67. ति॒ष्ठ॒न्त॒ इति॑ तिष्ठन्ते । \newline
\pagebreak
\markright{ TS 2.3.10.1  \hfill https://www.vedavms.in \hfill}

\section{ TS 2.3.10.1 }

\textbf{TS 2.3.10.1 } \newline
\textbf{Samhita Paata} \newline

यन्नव॒मैत् तन्नव॑नीतमभव॒द्-यदस॑र्प॒त् तथ् स॒र्पिर॑भव॒द्-यदद्धि॑यत॒ तद् घृ॒तम॑भवद॒श्विनोः᳚ प्रा॒णो॑ऽसि॒ तस्य॑ ते दत्तां॒ ॅययोः᳚ प्रा॒णोऽसि॒ स्वाहेन्द्र॑स्य प्रा॒णो॑ऽसि॒ तस्य॑ ते ददातु॒ यस्य॑ प्रा॒णोऽसि॒ स्वाहा॑ मि॒त्रावरु॑णयोः प्रा॒णो॑ऽसि॒ तस्य॑ ते दत्तां॒ ॅययोः᳚ प्रा॒णोऽसि॒ स्वाहा॒ विश्वे॑षां दे॒वानां᳚ प्रा॒णो॑ऽसि॒ - [  ] \newline

\textbf{Pada Paata} \newline

यत् । नव᳚म् । ऐत् । तत् । नव॑नीत॒मिति॒ नव॑ - नी॒त॒म् । अ॒भ॒व॒त् । यत् । अस॑र्पत् । तत् । स॒र्पिः । अ॒भ॒व॒त् । यत् । अद्ध्रि॑यत । तत् । घृ॒तम् । अ॒भ॒व॒त् । अ॒श्विनोः᳚ । प्रा॒ण इति॑ प्र-अ॒नः । अ॒सि॒ । तस्य॑ । ते॒ । द॒त्ता॒म् । ययोः᳚ । प्रा॒ण इति॑ प्र - अ॒नः । असि॑ । स्वाहा᳚ । इन्द्र॑स्य । प्रा॒ण इति॑ प्र-अ॒नः । अ॒सि॒ । तस्य॑ । ते॒ । द॒दा॒तु॒ । यस्य॑ । प्रा॒ण इति॑ प्र - अ॒नः । असि॑ । स्वाहा᳚ । मि॒त्रावरु॑णयो॒रिति॑ मि॒त्रा - वरु॑णयोः । प्रा॒ण इति॑ प्र - अ॒नः । अ॒सि॒ । तस्य॑ । ते॒ । द॒त्ता॒म् । ययोः᳚ । प्रा॒ण इति॑ प्र - अ॒नः । असि॑ । स्वाहा᳚ । विश्वे॑षाम् । दे॒वाना᳚म् । प्रा॒ण इति॑ प्र - अ॒नः । अ॒सि॒ ।  \newline


\textbf{Krama Paata} \newline

यन्नव᳚म् । नव॒मैत् । ऐत्तत् । तन्नव॑नीतम् । नव॑नीतमभवत् । नव॑नीत॒मिति॒ नव॑ - नी॒त॒म् । अ॒भ॒व॒द् यत् । यदस॑र्पत् । अस॑र्प॒त् तत् । तथ् स॒र्पिः । स॒र्पिर॑भवत् । अ॒भ॒व॒द् यत् । यदध्रि॑यत । अध्रि॑यत॒ तत् । तद् घृ॒तम् । घृ॒तम॑भवत् । अ॒भ॒व॒द॒श्विनोः᳚ । अ॒श्विनोः᳚ प्रा॒णः । प्रा॒णो॑ऽसि । प्रा॒ण इति॑ प्र - अ॒नः । अ॒सि॒ तस्य॑ । तस्य॑ ते । ते॒ द॒त्ता॒म् । द॒त्तां॒ ॅययोः᳚ । ययोः᳚ प्रा॒णः । प्रा॒णोऽसि॑ । प्रा॒ण इति॑ प्र - अ॒नः । असि॒ स्वाहा᳚ । स्वाहेन्द्र॑स्य । इन्द्र॑स्य प्रा॒णः । प्रा॒णो॑ऽसि । प्रा॒ण इति॑ प्र - अ॒नः । अ॒सि॒ तस्य॑ । तस्य॑ ते । ते॒ द॒दा॒तु॒ । द॒दा॒तु॒ यस्य॑ । यस्य॑ प्रा॒णः । प्रा॒णोऽसि॑ । प्रा॒ण इति॑ प्र - अ॒नः । असि॒ स्वाहा᳚ । स्वाहा॑ मि॒त्रावरु॑णयोः । मि॒त्रावरु॑णयोः प्रा॒णः । मि॒त्रावरु॑णयो॒रिति॑ मि॒त्रा - वरु॑णयोः । प्रा॒णो॑ऽसि । प्रा॒ण इति॑ प्र - अ॒नः । अ॒सि॒ तस्य॑ । तस्य॑ ते । ते॒ द॒त्ता॒म् । द॒त्तां॒ ॅययोः᳚ । ययोः᳚ प्रा॒णः । प्रा॒णोऽसि॑ । प्रा॒ण इति॑ प्र - अ॒नः । असि॒ स्वाहा᳚ । स्वाहा॒ विश्वे॑षाम् । विश्वे॑षाम् दे॒वाना᳚म् । दे॒वाना᳚म् प्रा॒णः । प्रा॒णो॑ऽसि । प्रा॒ण इति॑ प्र - अ॒नः । अ॒सि॒ तस्य॑ \newline

\textbf{Jatai Paata} \newline

1. यन् नव॒म् नवं॒ ॅयद् यन् नव᳚म् । \newline
2. नव॒ मैदैन् नव॒म् नव॒ मैत् । \newline
3. ऐत् तत् तदैदैत् तत् । \newline
4. तन् नव॑नीत॒म् नव॑नीत॒म् तत् तन् नव॑नीतम् । \newline
5. नव॑नीत मभव दभव॒न् नव॑नीत॒म् नव॑नीत मभवत् । \newline
6. नव॑नीत॒मिति॒ नव॑ - नी॒त॒म् । \newline
7. अ॒भ॒व॒द् यद् यद॑भव दभव॒द् यत् । \newline
8. यदस॑र्प॒ दस॑र्प॒द् यद् यदस॑र्पत् । \newline
9. अस॑र्प॒त् तत् तदस॑र्प॒ दस॑र्प॒त् तत् । \newline
10. तथ् स॒र्पिः स॒र्पि स्तत् तथ् स॒र्पिः । \newline
11. स॒र्पि र॑भव दभवथ् स॒र्पिः स॒र्पि र॑भवत् । \newline
12. अ॒भ॒व॒द् यद् यद॑भव दभव॒द् यत् । \newline
13. यदद्ध्रि॑य॒ता द्ध्रि॑यत॒ यद् यदद्ध्रि॑यत । \newline
14. अद्ध्रि॑यत॒ तत् तदद्ध्रि॑य॒ता द्ध्रि॑यत॒ तत् । \newline
15. तद् घृ॒तम् घृ॒तम् तत् तद् घृ॒तम् । \newline
16. घृ॒त म॑भव दभवद् घृ॒तम् घृ॒त म॑भवत् । \newline
17. अ॒भ॒व॒ द॒श्विनो॑ र॒श्विनो॑ रभव दभव द॒श्विनोः᳚ । \newline
18. अ॒श्विनोः᳚ प्रा॒णः प्रा॒णो᳚ ऽश्विनो॑ र॒श्विनोः᳚ प्रा॒णः । \newline
19. प्रा॒णो᳚ ऽस्यसि प्रा॒णः प्रा॒णो॑ ऽसि । \newline
20. प्रा॒ण इति॑ प्र - अ॒नः । \newline
21. अ॒सि॒ तस्य॒ तस्या᳚स्यसि॒ तस्य॑ । \newline
22. तस्य॑ ते ते॒ तस्य॒ तस्य॑ ते । \newline
23. ते॒ द॒त्ता॒म् द॒त्ता॒म् ते॒ ते॒ द॒त्ता॒म् । \newline
24. द॒त्तां॒ ॅययो॒र् ययो᳚र् दत्ताम् दत्तां॒ ॅययोः᳚ । \newline
25. ययोः᳚ प्रा॒णः प्रा॒णो ययो॒र् ययोः᳚ प्रा॒णः । \newline
26. प्रा॒णो ऽस्यसि॑ प्रा॒णः प्रा॒णो ऽसि॑ । \newline
27. प्रा॒ण इति॑ प्र - अ॒नः । \newline
28. असि॒ स्वाहा॒ स्वाहा ऽस्यसि॒ स्वाहा᳚ । \newline
29. स्वाहेन्द्र॒स्ये न्द्र॑स्य॒ स्वाहा॒ स्वाहेन्द्र॑स्य । \newline
30. इन्द्र॑स्य प्रा॒णः प्रा॒ण इन्द्र॒स्ये न्द्र॑स्य प्रा॒णः । \newline
31. प्रा॒णो᳚ ऽस्यसि प्रा॒णः प्रा॒णो॑ ऽसि । \newline
32. प्रा॒ण इति॑ प्र - अ॒नः । \newline
33. अ॒सि॒ तस्य॒ तस्या᳚स्यसि॒ तस्य॑ । \newline
34. तस्य॑ ते ते॒ तस्य॒ तस्य॑ ते । \newline
35. ते॒ द॒दा॒तु॒ द॒दा॒तु॒ ते॒ ते॒ द॒दा॒तु॒ । \newline
36. द॒दा॒तु॒ यस्य॒ यस्य॑ ददातु ददातु॒ यस्य॑ । \newline
37. यस्य॑ प्रा॒णः प्रा॒णो यस्य॒ यस्य॑ प्रा॒णः । \newline
38. प्रा॒णो ऽस्यसि॑ प्रा॒णः प्रा॒णो ऽसि॑ । \newline
39. प्रा॒ण इति॑ प्र - अ॒नः । \newline
40. असि॒ स्वाहा॒ स्वाहा ऽस्यसि॒ स्वाहा᳚ । \newline
41. स्वाहा॑ मि॒त्रावरु॑णयोर् मि॒त्रावरु॑णयोः॒ स्वाहा॒ स्वाहा॑ मि॒त्रावरु॑णयोः । \newline
42. मि॒त्रावरु॑णयोः प्रा॒णः प्रा॒णो मि॒त्रावरु॑णयोर् मि॒त्रावरु॑णयोः प्रा॒णः । \newline
43. मि॒त्रावरु॑णयो॒रिति॑ मि॒त्रा - वरु॑णयोः । \newline
44. प्रा॒णो᳚ ऽस्यसि प्रा॒णः प्रा॒णो॑ ऽसि । \newline
45. प्रा॒ण इति॑ प्र - अ॒नः । \newline
46. अ॒सि॒ तस्य॒ तस्या᳚स्यसि॒ तस्य॑ । \newline
47. तस्य॑ ते ते॒ तस्य॒ तस्य॑ ते । \newline
48. ते॒ द॒त्ता॒म् द॒त्ता॒म् ते॒ ते॒ द॒त्ता॒म् । \newline
49. द॒त्तां॒ ॅययो॒र् ययो᳚र् दत्ताम् दत्तां॒ ॅययोः᳚ । \newline
50. ययोः᳚ प्रा॒णः प्रा॒णो ययो॒र् ययोः᳚ प्रा॒णः । \newline
51. प्रा॒णो ऽस्यसि॑ प्रा॒णः प्रा॒णो ऽसि॑ । \newline
52. प्रा॒ण इति॑ प्र - अ॒नः । \newline
53. असि॒ स्वाहा॒ स्वाहा ऽस्यसि॒ स्वाहा᳚ । \newline
54. स्वाहा॒ विश्वे॑षां॒ ॅविश्वे॑षाꣳ॒॒ स्वाहा॒ स्वाहा॒ विश्वे॑षाम् । \newline
55. विश्वे॑षाम् दे॒वाना᳚म् दे॒वानां॒ ॅविश्वे॑षां॒ ॅविश्वे॑षाम् दे॒वाना᳚म् । \newline
56. दे॒वाना᳚म् प्रा॒णः प्रा॒णो दे॒वाना᳚म् दे॒वाना᳚म् प्रा॒णः । \newline
57. प्रा॒णो᳚ ऽस्यसि प्रा॒णः प्रा॒णो॑ ऽसि । \newline
58. प्रा॒ण इति॑ प्र - अ॒नः । \newline
59. अ॒सि॒ तस्य॒ तस्या᳚स्यसि॒ तस्य॑ । \newline

\textbf{Ghana Paata } \newline

1. यन् नव॒म् नवं॒ ॅयद् यन् नव॒ मैदैन् नवं॒ ॅयद् यन् नव॒ मैत् । \newline
2. नव॒ मैदैन् नव॒म् नव॒ मैत् तत् तदैन् नव॒म् नव॒ मैत् तत् । \newline
3. ऐत् तत् तदैदैत् तन् नव॑नीत॒म् नव॑नीत॒म् तदैदैत् तन् नव॑नीतम् । \newline
4. तन् नव॑नीत॒म् नव॑नीत॒म् तत् तन् नव॑नीत मभव दभव॒न् नव॑नीत॒म् तत् तन् नव॑नीत मभवत् । \newline
5. नव॑नीत मभव दभव॒न् नव॑नीत॒म् नव॑नीत मभव॒द् यद् यद॑भव॒न् नव॑नीत॒म् नव॑नीत मभव॒द् यत् । \newline
6. नव॑नीत॒मिति॒ नव॑ - नी॒त॒म् । \newline
7. अ॒भ॒व॒द् यद् यद॑भव दभव॒द् यदस॑र्प॒ दस॑र्प॒द् यद॑भव दभव॒द् यदस॑र्पत् । \newline
8. यदस॑र्प॒ दस॑र्प॒द् यद् यदस॑र्प॒त् तत् तदस॑र्प॒द् यद् यदस॑र्प॒त् तत् । \newline
9. अस॑र्प॒त् तत् तदस॑र्प॒ दस॑र्प॒त् तथ् स॒र्पिः स॒र्पि स्तदस॑र्प॒ दस॑र्प॒त् तथ् स॒र्पिः । \newline
10. तथ् स॒र्पिः स॒र्पि स्तत् तथ् स॒र्पि र॑भव दभवथ् स॒र्पि स्तत् तथ् स॒र्पि र॑भवत् । \newline
11. स॒र्पि र॑भव दभवथ् स॒र्पिः स॒र्पि र॑भव॒द् यद् यद॑भवथ् स॒र्पिः स॒र्पि र॑भव॒द् यत् । \newline
12. अ॒भ॒व॒द् यद् यद॑भव दभव॒द् यदद्ध्रि॑य॒ता द्ध्रि॑यत॒ यद॑भव दभव॒द् यदद्ध्रि॑यत । \newline
13. यदद्ध्रि॑य॒ता द्ध्रि॑यत॒ यद् यदद्ध्रि॑यत॒ तत् तदद्ध्रि॑यत॒ यद् यदद्ध्रि॑यत॒ तत् । \newline
14. अद्ध्रि॑यत॒ तत् तदद्ध्रि॑य॒ता द्ध्रि॑यत॒ तद् घृ॒तम् घृ॒तम् तदद्ध्रि॑य॒ता द्ध्रि॑यत॒ तद् घृ॒तम् । \newline
15. तद् घृ॒तम् घृ॒तम् तत् तद् घृ॒त म॑भव दभवद् घृ॒तम् तत् तद् घृ॒त म॑भवत् । \newline
16. घृ॒त म॑भव दभवद् घृ॒तम् घृ॒त म॑भव द॒श्विनो॑ र॒श्विनो॑ रभवद् घृ॒तम् घृ॒त म॑भव द॒श्विनोः᳚ । \newline
17. अ॒भ॒व॒ द॒श्विनो॑ र॒श्विनो॑ रभव दभव द॒श्विनोः᳚ प्रा॒णः प्रा॒णो᳚ ऽश्विनो॑ रभव दभव द॒श्विनोः᳚ प्रा॒णः । \newline
18. अ॒श्विनोः᳚ प्रा॒णः प्रा॒णो᳚ ऽश्विनो॑ र॒श्विनोः᳚ प्रा॒णो᳚ ऽस्यसि प्रा॒णो᳚ ऽश्विनो॑ र॒श्विनोः᳚ प्रा॒णो॑ ऽसि । \newline
19. प्रा॒णो᳚ ऽस्यसि प्रा॒णः प्रा॒णो॑ ऽसि॒ तस्य॒ तस्या॑सि प्रा॒णः प्रा॒णो॑ ऽसि॒ तस्य॑ । \newline
20. प्रा॒ण इति॑ प्र - अ॒नः । \newline
21. अ॒सि॒ तस्य॒ तस्या᳚स्यसि॒ तस्य॑ ते ते॒ तस्या᳚स्यसि॒ तस्य॑ ते । \newline
22. तस्य॑ ते ते॒ तस्य॒ तस्य॑ ते दत्ताम् दत्ताम् ते॒ तस्य॒ तस्य॑ ते दत्ताम् । \newline
23. ते॒ द॒त्ता॒म् द॒त्ता॒म् ते॒ ते॒ द॒त्तां॒ ॅययो॒र् ययो᳚र् दत्ताम् ते ते दत्तां॒ ॅययोः᳚ । \newline
24. द॒त्तां॒ ॅययो॒र् ययो᳚र् दत्ताम् दत्तां॒ ॅययोः᳚ प्रा॒णः प्रा॒णो ययो᳚र् दत्ताम् दत्तां॒ ॅययोः᳚ प्रा॒णः । \newline
25. ययोः᳚ प्रा॒णः प्रा॒णो ययो॒र् ययोः᳚ प्रा॒णो ऽस्यसि॑ प्रा॒णो ययो॒र् ययोः᳚ प्रा॒णो ऽसि॑ । \newline
26. प्रा॒णो ऽस्यसि॑ प्रा॒णः प्रा॒णो ऽसि॒ स्वाहा॒ स्वाहा ऽसि॑ प्रा॒णः प्रा॒णो ऽसि॒ स्वाहा᳚ । \newline
27. प्रा॒ण इति॑ प्र - अ॒नः । \newline
28. असि॒ स्वाहा॒ स्वाहा ऽस्यसि॒ स्वाहेन्द्र॒स्ये न्द्र॑स्य॒ स्वाहा ऽस्यसि॒ स्वाहेन्द्र॑स्य । \newline
29. स्वाहेन्द्र॒स्ये न्द्र॑स्य॒ स्वाहा॒ स्वाहेन्द्र॑स्य प्रा॒णः प्रा॒ण इन्द्र॑स्य॒ स्वाहा॒ स्वाहेन्द्र॑स्य प्रा॒णः । \newline
30. इन्द्र॑स्य प्रा॒णः प्रा॒ण इन्द्र॒स्ये न्द्र॑स्य प्रा॒णो᳚ ऽस्यसि प्रा॒ण इन्द्र॒स्ये न्द्र॑स्य प्रा॒णो॑ ऽसि । \newline
31. प्रा॒णो᳚ ऽस्यसि प्रा॒णः प्रा॒णो॑ ऽसि॒ तस्य॒ तस्या॑सि प्रा॒णः प्रा॒णो॑ ऽसि॒ तस्य॑ । \newline
32. प्रा॒ण इति॑ प्र - अ॒नः । \newline
33. अ॒सि॒ तस्य॒ तस्या᳚स्यसि॒ तस्य॑ ते ते॒ तस्या᳚स्यसि॒ तस्य॑ ते । \newline
34. तस्य॑ ते ते॒ तस्य॒ तस्य॑ ते ददातु ददातु ते॒ तस्य॒ तस्य॑ ते ददातु । \newline
35. ते॒ द॒दा॒तु॒ द॒दा॒तु॒ ते॒ ते॒ द॒दा॒तु॒ यस्य॒ यस्य॑ ददातु ते ते ददातु॒ यस्य॑ । \newline
36. द॒दा॒तु॒ यस्य॒ यस्य॑ ददातु ददातु॒ यस्य॑ प्रा॒णः प्रा॒णो यस्य॑ ददातु ददातु॒ यस्य॑ प्रा॒णः । \newline
37. यस्य॑ प्रा॒णः प्रा॒णो यस्य॒ यस्य॑ प्रा॒णो ऽस्यसि॑ प्रा॒णो यस्य॒ यस्य॑ प्रा॒णो ऽसि॑ । \newline
38. प्रा॒णो ऽस्यसि॑ प्रा॒णः प्रा॒णो ऽसि॒ स्वाहा॒ स्वाहा ऽसि॑ प्रा॒णः प्रा॒णो ऽसि॒ स्वाहा᳚ । \newline
39. प्रा॒ण इति॑ प्र - अ॒नः । \newline
40. असि॒ स्वाहा॒ स्वाहा ऽस्यसि॒ स्वाहा॑ मि॒त्रावरु॑णयोर् मि॒त्रावरु॑णयोः॒ स्वाहा ऽस्यसि॒ स्वाहा॑ मि॒त्रावरु॑णयोः । \newline
41. स्वाहा॑ मि॒त्रावरु॑णयोर् मि॒त्रावरु॑णयोः॒ स्वाहा॒ स्वाहा॑ मि॒त्रावरु॑णयोः प्रा॒णः प्रा॒णो मि॒त्रावरु॑णयोः॒ स्वाहा॒ स्वाहा॑ मि॒त्रावरु॑णयोः प्रा॒णः । \newline
42. मि॒त्रावरु॑णयोः प्रा॒णः प्रा॒णो मि॒त्रावरु॑णयोर् मि॒त्रावरु॑णयोः प्रा॒णो᳚ ऽस्यसि प्रा॒णो मि॒त्रावरु॑णयोर् मि॒त्रावरु॑णयोः प्रा॒णो॑ ऽसि । \newline
43. मि॒त्रावरु॑णयो॒रिति॑ मि॒त्रा - वरु॑णयोः । \newline
44. प्रा॒णो᳚ ऽस्यसि प्रा॒णः प्रा॒णो॑ ऽसि॒ तस्य॒ तस्या॑सि प्रा॒णः प्रा॒णो॑ ऽसि॒ तस्य॑ । \newline
45. प्रा॒ण इति॑ प्र - अ॒नः । \newline
46. अ॒सि॒ तस्य॒ तस्या᳚स्यसि॒ तस्य॑ ते ते॒ तस्या᳚स्यसि॒ तस्य॑ ते । \newline
47. तस्य॑ ते ते॒ तस्य॒ तस्य॑ ते दत्ताम् दत्ताम् ते॒ तस्य॒ तस्य॑ ते दत्ताम् । \newline
48. ते॒ द॒त्ता॒म् द॒त्ता॒म् ते॒ ते॒ द॒त्तां॒ ॅययो॒र् ययो᳚र् दत्ताम् ते ते दत्तां॒ ॅययोः᳚ । \newline
49. द॒त्तां॒ ॅययो॒र् ययो᳚र् दत्ताम् दत्तां॒ ॅययोः᳚ प्रा॒णः प्रा॒णो ययो᳚र् दत्ताम् दत्तां॒ ॅययोः᳚ प्रा॒णः । \newline
50. ययोः᳚ प्रा॒णः प्रा॒णो ययो॒र् ययोः᳚ प्रा॒णो ऽस्यसि॑ प्रा॒णो ययो॒र् ययोः᳚ प्रा॒णो ऽसि॑ । \newline
51. प्रा॒णो ऽस्यसि॑ प्रा॒णः प्रा॒णो ऽसि॒ स्वाहा॒ स्वाहा ऽसि॑ प्रा॒णः प्रा॒णो ऽसि॒ स्वाहा᳚ । \newline
52. प्रा॒ण इति॑ प्र - अ॒नः । \newline
53. असि॒ स्वाहा॒ स्वाहा ऽस्यसि॒ स्वाहा॒ विश्वे॑षां॒ ॅविश्वे॑षाꣳ॒॒ स्वाहा ऽस्यसि॒ स्वाहा॒ विश्वे॑षाम् । \newline
54. स्वाहा॒ विश्वे॑षां॒ ॅविश्वे॑षाꣳ॒॒ स्वाहा॒ स्वाहा॒ विश्वे॑षाम् दे॒वाना᳚म् दे॒वानां॒ ॅविश्वे॑षाꣳ॒॒ स्वाहा॒ स्वाहा॒ विश्वे॑षाम् दे॒वाना᳚म् । \newline
55. विश्वे॑षाम् दे॒वाना᳚म् दे॒वानां॒ ॅविश्वे॑षां॒ ॅविश्वे॑षाम् दे॒वाना᳚म् प्रा॒णः प्रा॒णो दे॒वानां॒ ॅविश्वे॑षां॒ ॅविश्वे॑षाम् दे॒वाना᳚म् प्रा॒णः । \newline
56. दे॒वाना᳚म् प्रा॒णः प्रा॒णो दे॒वाना᳚म् दे॒वाना᳚म् प्रा॒णो᳚ ऽस्यसि प्रा॒णो दे॒वाना᳚म् दे॒वाना᳚म् प्रा॒णो॑ ऽसि । \newline
57. प्रा॒णो᳚ ऽस्यसि प्रा॒णः प्रा॒णो॑ ऽसि॒ तस्य॒ तस्या॑सि प्रा॒णः प्रा॒णो॑ ऽसि॒ तस्य॑ । \newline
58. प्रा॒ण इति॑ प्र - अ॒नः । \newline
59. अ॒सि॒ तस्य॒ तस्या᳚स्यसि॒ तस्य॑ ते ते॒ तस्या᳚स्यसि॒ तस्य॑ ते । \newline
\pagebreak
\markright{ TS 2.3.10.2  \hfill https://www.vedavms.in \hfill}

\section{ TS 2.3.10.2 }

\textbf{TS 2.3.10.2 } \newline
\textbf{Samhita Paata} \newline

तस्य॑ ते ददतु॒ येषां᳚ प्रा॒णोऽसि॒ स्वाहा॑ घृ॒तस्य॒ धारा॑म॒मृत॑स्य॒ पन्था॒मिन्द्रे॑ण द॒त्तां प्रय॑तां म॒रुद्भिः॑ । तत् त्वा॒ विष्णुः॒ पर्य॑पश्य॒त् तत् त्वेडा॒ गव्यैर॑यत् ॥ पा॒व॒मा॒नेन॑ त्वा॒ स्तोमे॑न गाय॒त्रस्य॑ वर्त॒न्योपाꣳ॒॒शो र्वी॒र्ये॑ण दे॒वस्त्वा॑ सवि॒तोथ् सृ॑जतु जी॒वात॑वे जीवन॒स्यायै॑ बृहद्-रथन्त॒रयो᳚स्त्वा॒ स्तोमे॑न त्रि॒ष्टुभो॑ वर्त॒न्या शु॒क्रस्य॑ वी॒र्ये॑ण दे॒वस्त्वा॑ सवि॒तोथ् - [  ] \newline

\textbf{Pada Paata} \newline

तस्य॑ । ते॒ । द॒द॒तु॒ । येषा᳚म् । प्रा॒ण इति॑ प्र-अ॒नः । असि॑ । स्वाहा᳚ । घृ॒तस्य॑ । धारा᳚म् । अ॒मृत॑स्य । पन्था᳚म् । इन्द्रे॑ण । द॒त्ताम् । प्रय॑ता॒मिति॒ प्र - य॒ता॒म् । म॒रुद्भि॒रिति॑ म॒रुत् - भिः॒ ॥ तत् । त्वा॒ । विष्णुः॑ । परीति॑ । अ॒प॒श्य॒त् । तत् । त्वा॒ । इडा᳚ । गवि॑ । ऐर॑यत् ॥ पा॒व॒मा॒नेन॑ । त्वा॒ । स्तोमे॑न । गा॒य॒त्रस्य॑ । व॒र्त॒न्या । उ॒पाꣳ॒॒शोरित्यु॑प-अꣳ॒॒शोः । वी॒र्ये॑ण । दे॒वः । त्वा॒ । स॒वि॒ता । उदिति॑ । सृ॒ज॒तु॒ । जी॒वात॑वे । जी॒व॒न॒स्यायै᳚ । बृ॒ह॒द्र॒थ॒न्त॒रयो॒रिति॑ बृहत् - र॒थ॒न्त॒रयोः᳚ । त्वा॒ । स्तोमे॑न । त्रि॒ष्टुभः॑ । व॒र्त॒न्या । शु॒क्रस्य॑ । वी॒र्ये॑ण । दे॒वः । त्वा॒ । स॒वि॒ता । उदिति॑ ।  \newline


\textbf{Krama Paata} \newline

तस्य॑ ते । ते॒ द॒द॒तु॒ । द॒द॒तु॒ येषा᳚म् । येषा᳚म् प्रा॒णः । प्रा॒णोऽसि॑ । प्रा॒ण इति॑ प्र - अ॒नः । असि॒ स्वाहा᳚ । स्वाहा॑ घृ॒तस्य॑ । घृ॒तस्य॒ धारा᳚म् । धारा॑म॒मृत॑स्य । अ॒मृत॑स्य॒ पन्था᳚म् । पन्था॒मिन्द्रे॑ण । इन्द्रे॑ण द॒त्ताम् । द॒त्ताम् प्रय॑ताम् । प्रय॑ताम् म॒रुद्भिः॑ । प्रय॑ता॒मिति॒ प्र - य॒ता॒म् । म॒रुद्भि॒रिति॑ म॒रुत् - भिः॒ ॥ तत् त्वा᳚ । त्वा॒ विष्णुः॑ । विष्णुः॒ परि॑ । पर्य॑पश्यत् । अ॒प॒श्य॒त् तत् । तत् त्वा᳚ । त्वेडा᳚ । इडा॒ गवि॑ । गव्यैर॑यत् । ऐर॑य॒दित्यैर॑यत् ॥ पा॒व॒मा॒नेन॑ त्वा । त्वा॒ स्तोमे॑न । स्तोमे॑न गाय॒त्रस्य॑ । गा॒य॒त्रस्य॑ वर्त॒न्या । व॒र्त॒न्यो॒पाꣳ॒॒शोः । उ॒पाꣳ॒॒शोर् वी॒र्ये॑ण । उ॒पाꣳ॒॒शोरित्यु॑प - अꣳ॒॒शोः । वी॒र्ये॑ण दे॒वः । दे॒वस्त्वा᳚ । त्वा॒ स॒वि॒ता । स॒वि॒तोत् । उथ् सृ॑जतु । सृ॒ज॒तु॒ जी॒वात॑वे । जी॒वात॑वे जीवन॒स्यायै᳚ । जी॒व॒न॒स्यायै॑ बृहद्रथन्त॒रयोः᳚ । बृ॒ह॒द्र॒थ॒न्त॒रयो᳚स्त्वा । बृ॒ह॒द्र॒थ॒न्त॒रयो॒रिति॑ बृहत् - र॒थ॒न्त॒रयोः᳚ । त्वा॒ स्तोमे॑न । स्तोमे॑न त्रि॒ष्टुभः॑ । त्रि॒ष्टुभो॑ वर्त॒न्या । व॒र्त॒न्या शु॒क्रस्य॑ । शु॒क्रस्य॑ वी॒र्ये॑ण । वी॒र्ये॑ण दे॒वः । दे॒वस्त्वा᳚ । त्वा॒ स॒वि॒ता । स॒वि॒तोत् । उथ् सृ॑जतु \newline

\textbf{Jatai Paata} \newline

1. तस्य॑ ते ते॒ तस्य॒ तस्य॑ ते । \newline
2. ते॒ द॒द॒तु॒ द॒द॒तु॒ ते॒ ते॒ द॒द॒तु॒ । \newline
3. द॒द॒तु॒ येषां॒ ॅयेषा᳚म् ददतु ददतु॒ येषा᳚म् । \newline
4. येषा᳚म् प्रा॒णः प्रा॒णो येषां॒ ॅयेषा᳚म् प्रा॒णः । \newline
5. प्रा॒णो ऽस्यसि॑ प्रा॒णः प्रा॒णो ऽसि॑ । \newline
6. प्रा॒ण इति॑ प्र - अ॒नः । \newline
7. असि॒ स्वाहा॒ स्वाहा ऽस्यसि॒ स्वाहा᳚ । \newline
8. स्वाहा॑ घृ॒तस्य॑ घृ॒तस्य॒ स्वाहा॒ स्वाहा॑ घृ॒तस्य॑ । \newline
9. घृ॒तस्य॒ धारा॒म् धारा᳚म् घृ॒तस्य॑ घृ॒तस्य॒ धारा᳚म् । \newline
10. धारा॑ म॒मृत॑स्या॒ मृत॑स्य॒ धारा॒म् धारा॑ म॒मृत॑स्य । \newline
11. अ॒मृत॑स्य॒ पन्था॒म् पन्था॑ म॒मृत॑स्या॒ मृत॑स्य॒ पन्था᳚म् । \newline
12. पन्था॒ मिन्द्रे॒णे न्द्रे॑ण॒ पन्था॒म् पन्था॒ मिन्द्रे॑ण । \newline
13. इन्द्रे॑ण द॒त्ताम् द॒त्ता मिन्द्रे॒णे न्द्रे॑ण द॒त्ताम् । \newline
14. द॒त्ताम् प्रय॑ता॒म् प्रय॑ताम् द॒त्ताम् द॒त्ताम् प्रय॑ताम् । \newline
15. प्रय॑ताम् म॒रुद्भि॑र् म॒रुद्भिः॒ प्रय॑ता॒म् प्रय॑ताम् म॒रुद्भिः॑ । \newline
16. प्रय॑ता॒मिति॒ प्र - य॒ता॒म् । \newline
17. म॒रुद्भि॒रिति॑ म॒रुत् - भिः॒ । \newline
18. तत् त्वा᳚ त्वा॒ तत् तत् त्वा᳚ । \newline
19. त्वा॒ विष्णु॒र् विष्णु॑ स्त्वा त्वा॒ विष्णुः॑ । \newline
20. विष्णुः॒ परि॒ परि॒ विष्णु॒र् विष्णुः॒ परि॑ । \newline
21. पर्य॑पश्य दपश्य॒त् परि॒ पर्य॑पश्यत् । \newline
22. अ॒प॒श्य॒त् तत् तद॑पश्य दपश्य॒त् तत् । \newline
23. तत् त्वा᳚ त्वा॒ तत् तत् त्वा᳚ । \newline
24. त्वेडेडा᳚ त्वा॒ त्वेडा᳚ । \newline
25. इडा॒ गवि॒ गवीडेडा॒ गवि॑ । \newline
26. गव्यै र॑य॒ दैर॑य॒द् गवि॒ गव्यैर॑यत् । \newline
27. ऐर॑य॒दित्यैर॑यत् । \newline
28. पा॒व॒मा॒नेन॑ त्वा त्वा पावमा॒नेन॑ पावमा॒नेन॑ त्वा । \newline
29. त्वा॒ स्तोमे॑न॒ स्तोमे॑न त्वा त्वा॒ स्तोमे॑न । \newline
30. स्तोमे॑न गाय॒त्रस्य॑ गाय॒त्रस्य॒ स्तोमे॑न॒ स्तोमे॑न गाय॒त्रस्य॑ । \newline
31. गा॒य॒त्रस्य॑ वर्त॒न्या व॑र्त॒न्या गा॑य॒त्रस्य॑ गाय॒त्रस्य॑ वर्त॒न्या । \newline
32. व॒र्त॒न्यो पाꣳ॒॒शो रु॑पाꣳ॒॒शोर् व॑र्त॒न्या व॑र्त॒न्यो पाꣳ॒॒शोः । \newline
33. उ॒पाꣳ॒॒शोर् वी॒र्ये॑ण वी॒र्ये॑णोपाꣳ॒॒शो रु॑पाꣳ॒॒शोर् वी॒र्ये॑ण । \newline
34. उ॒पाꣳ॒॒शोरित्यु॑प - अꣳ॒॒शोः । \newline
35. वी॒र्ये॑ण दे॒वो दे॒वो वी॒र्ये॑ण वी॒र्ये॑ण दे॒वः । \newline
36. दे॒वस्त्वा᳚ त्वा दे॒वो दे॒व स्त्वा᳚ । \newline
37. त्वा॒ स॒वि॒ता स॑वि॒ता त्वा᳚ त्वा सवि॒ता । \newline
38. स॒वि॒तो दुथ् स॑वि॒ता स॑वि॒तोत् । \newline
39. उथ् सृ॑जतु सृज॒तू दुथ् सृ॑जतु । \newline
40. सृ॒ज॒तु॒ जी॒वात॑वे जी॒वात॑वे सृजतु सृजतु जी॒वात॑वे । \newline
41. जी॒वात॑वे जीवन॒स्यायै॑ जीवन॒स्यायै॑ जी॒वात॑वे जी॒वात॑वे जीवन॒स्यायै᳚ । \newline
42. जी॒व॒न॒स्यायै॑ बृहद्रथन्त॒रयो᳚र् बृहद्रथन्त॒रयो᳚र् जीवन॒स्यायै॑ जीवन॒स्यायै॑ बृहद्रथन्त॒रयोः᳚ । \newline
43. बृ॒ह॒द्र॒थ॒न्त॒रयो᳚ स्त्वा त्वा बृहद्रथन्त॒रयो᳚र् बृहद्रथन्त॒रयो᳚ स्त्वा । \newline
44. बृ॒ह॒द्र॒थ॒न्त॒रयो॒रिति॑ बृहत् - र॒थ॒न्त॒रयोः᳚ । \newline
45. त्वा॒ स्तोमे॑न॒ स्तोमे॑न त्वा त्वा॒ स्तोमे॑न । \newline
46. स्तोमे॑न त्रि॒ष्टुभ॑ स्त्रि॒ष्टुभः॒ स्तोमे॑न॒ स्तोमे॑न त्रि॒ष्टुभः॑ । \newline
47. त्रि॒ष्टुभो॑ वर्त॒न्या व॑र्त॒न्या त्रि॒ष्टुभ॑ स्त्रि॒ष्टुभो॑ वर्त॒न्या । \newline
48. व॒र्त॒न्या शु॒क्रस्य॑ शु॒क्रस्य॑ वर्त॒न्या व॑र्त॒न्या शु॒क्रस्य॑ । \newline
49. शु॒क्रस्य॑ वी॒र्ये॑ण वी॒र्ये॑ण शु॒क्रस्य॑ शु॒क्रस्य॑ वी॒र्ये॑ण । \newline
50. वी॒र्ये॑ण दे॒वो दे॒वो वी॒र्ये॑ण वी॒र्ये॑ण दे॒वः । \newline
51. दे॒व स्त्वा᳚ त्वा दे॒वो दे॒व स्त्वा᳚ । \newline
52. त्वा॒ स॒वि॒ता स॑वि॒ता त्वा᳚ त्वा सवि॒ता । \newline
53. स॒वि॒तोदुथ् स॑वि॒ता स॑वि॒तोत् । \newline
54. उथ् सृ॑जतु सृज॒तू दुथ् सृ॑जतु । \newline

\textbf{Ghana Paata } \newline

1. तस्य॑ ते ते॒ तस्य॒ तस्य॑ ते ददतु ददतु ते॒ तस्य॒ तस्य॑ ते ददतु । \newline
2. ते॒ द॒द॒तु॒ द॒द॒तु॒ ते॒ ते॒ द॒द॒तु॒ येषां॒ ॅयेषा᳚म् ददतु ते ते ददतु॒ येषा᳚म् । \newline
3. द॒द॒तु॒ येषां॒ ॅयेषा᳚म् ददतु ददतु॒ येषा᳚म् प्रा॒णः प्रा॒णो येषा᳚म् ददतु ददतु॒ येषा᳚म् प्रा॒णः । \newline
4. येषा᳚म् प्रा॒णः प्रा॒णो येषां॒ ॅयेषा᳚म् प्रा॒णो ऽस्यसि॑ प्रा॒णो येषां॒ ॅयेषा᳚म् प्रा॒णो ऽसि॑ । \newline
5. प्रा॒णो ऽस्यसि॑ प्रा॒णः प्रा॒णो ऽसि॒ स्वाहा॒ स्वाहा ऽसि॑ प्रा॒णः प्रा॒णो ऽसि॒ स्वाहा᳚ । \newline
6. प्रा॒ण इति॑ प्र - अ॒नः । \newline
7. असि॒ स्वाहा॒ स्वाहा ऽस्यसि॒ स्वाहा॑ घृ॒तस्य॑ घृ॒तस्य॒ स्वाहा ऽस्यसि॒ स्वाहा॑ घृ॒तस्य॑ । \newline
8. स्वाहा॑ घृ॒तस्य॑ घृ॒तस्य॒ स्वाहा॒ स्वाहा॑ घृ॒तस्य॒ धारा॒म् धारा᳚म् घृ॒तस्य॒ स्वाहा॒ स्वाहा॑ घृ॒तस्य॒ धारा᳚म् । \newline
9. घृ॒तस्य॒ धारा॒म् धारा᳚म् घृ॒तस्य॑ घृ॒तस्य॒ धारा॑ म॒मृत॑स्या॒ मृत॑स्य॒ धारा᳚म् घृ॒तस्य॑ घृ॒तस्य॒ धारा॑ म॒मृत॑स्य । \newline
10. धारा॑ म॒मृत॑स्या॒ मृत॑स्य॒ धारा॒म् धारा॑ म॒मृत॑स्य॒ पन्था॒म् पन्था॑ म॒मृत॑स्य॒ धारा॒म् धारा॑ म॒मृत॑स्य॒ पन्था᳚म् । \newline
11. अ॒मृत॑स्य॒ पन्था॒म् पन्था॑ म॒मृत॑स्या॒ मृत॑स्य॒ पन्था॒ मिन्द्रे॒णे न्द्रे॑ण॒ पन्था॑ म॒मृत॑स्या॒ मृत॑स्य॒ पन्था॒ मिन्द्रे॑ण । \newline
12. पन्था॒ मिन्द्रे॒णे न्द्रे॑ण॒ पन्था॒म् पन्था॒ मिन्द्रे॑ण द॒त्ताम् द॒त्ता मिन्द्रे॑ण॒ पन्था॒म् पन्था॒ मिन्द्रे॑ण द॒त्ताम् । \newline
13. इन्द्रे॑ण द॒त्ताम् द॒त्ता मिन्द्रे॒णे न्द्रे॑ण द॒त्ताम् प्रय॑ता॒म् प्रय॑ताम् द॒त्ता मिन्द्रे॒णे न्द्रे॑ण द॒त्ताम् प्रय॑ताम् । \newline
14. द॒त्ताम् प्रय॑ता॒म् प्रय॑ताम् द॒त्ताम् द॒त्ताम् प्रय॑ताम् म॒रुद्भि॑र् म॒रुद्भिः॒ प्रय॑ताम् द॒त्ताम् द॒त्ताम् प्रय॑ताम् म॒रुद्भिः॑ । \newline
15. प्रय॑ताम् म॒रुद्भि॑र् म॒रुद्भिः॒ प्रय॑ता॒म् प्रय॑ताम् म॒रुद्भिः॑ । \newline
16. प्रय॑ता॒मिति॒ प्र - य॒ता॒म् । \newline
17. म॒रुद्भि॒रिति॑ म॒रुत् - भिः॒ । \newline
18. तत् त्वा᳚ त्वा॒ तत् तत् त्वा॒ विष्णु॒र् विष्णु॑ स्त्वा॒ तत् तत् त्वा॒ विष्णुः॑ । \newline
19. त्वा॒ विष्णु॒र् विष्णु॑ स्त्वा त्वा॒ विष्णुः॒ परि॒ परि॒ विष्णु॑ स्त्वा त्वा॒ विष्णुः॒ परि॑ । \newline
20. विष्णुः॒ परि॒ परि॒ विष्णु॒र् विष्णुः॒ पर्य॑पश्य दपश्य॒त् परि॒ विष्णु॒र् विष्णुः॒ पर्य॑पश्यत् । \newline
21. पर्य॑पश्य दपश्य॒त् परि॒ पर्य॑पश्य॒त् तत् तद॑पश्य॒त् परि॒ पर्य॑पश्य॒त् तत् । \newline
22. अ॒प॒श्य॒त् तत् तद॑पश्य दपश्य॒त् तत् त्वा᳚ त्वा॒ तद॑पश्य दपश्य॒त् तत् त्वा᳚ । \newline
23. तत् त्वा᳚ त्वा॒ तत् तत् त्वेडेडा᳚ त्वा॒ तत् तत् त्वेडा᳚ । \newline
24. त्वेडेडा᳚ त्वा॒ त्वेडा॒ गवि॒ गवीडा᳚ त्वा॒ त्वेडा॒ गवि॑ । \newline
25. इडा॒ गवि॒ गवीडेडा॒ गव्यै र॑य॒दैर॑य॒द् गवीडेडा॒ गव्यै र॑यत् । \newline
26. गव्यै र॑य॒दैर॑य॒द् गवि॒ गव्यै र॑यत् । \newline
27. ऐर॑य॒दित्यैर॑यत् । \newline
28. पा॒व॒मा॒नेन॑ त्वा त्वा पावमा॒नेन॑ पावमा॒नेन॑ त्वा॒ स्तोमे॑न॒ स्तोमे॑न त्वा पावमा॒नेन॑ पावमा॒नेन॑ त्वा॒ स्तोमे॑न । \newline
29. त्वा॒ स्तोमे॑न॒ स्तोमे॑न त्वा त्वा॒ स्तोमे॑न गाय॒त्रस्य॑ गाय॒त्रस्य॒ स्तोमे॑न त्वा त्वा॒ स्तोमे॑न गाय॒त्रस्य॑ । \newline
30. स्तोमे॑न गाय॒त्रस्य॑ गाय॒त्रस्य॒ स्तोमे॑न॒ स्तोमे॑न गाय॒त्रस्य॑ वर्त॒न्या व॑र्त॒न्या गा॑य॒त्रस्य॒ स्तोमे॑न॒ स्तोमे॑न गाय॒त्रस्य॑ वर्त॒न्या । \newline
31. गा॒य॒त्रस्य॑ वर्त॒न्या व॑र्त॒न्या गा॑य॒त्रस्य॑ गाय॒त्रस्य॑ वर्त॒न्योपाꣳ॒॒शो रु॑पाꣳ॒॒शोर् व॑र्त॒न्या गा॑य॒त्रस्य॑ गाय॒त्रस्य॑ वर्त॒न्योपाꣳ॒॒शोः । \newline
32. व॒र्त॒न्योपाꣳ॒॒शो रु॑पाꣳ॒॒शोर् व॑र्त॒न्या व॑र्त॒न्योपाꣳ॒॒शोर् वी॒र्ये॑ण वी॒र्ये॑णोपाꣳ॒॒शोर् व॑र्त॒न्या व॑र्त॒न्योपाꣳ॒॒शोर् वी॒र्ये॑ण । \newline
33. उ॒पाꣳ॒॒शोर् वी॒र्ये॑ण वी॒र्ये॑णोपाꣳ॒॒शो रु॑पाꣳ॒॒शोर् वी॒र्ये॑ण दे॒वो दे॒वो वी॒र्ये॑णोपाꣳ॒॒शो रु॑पाꣳ॒॒शोर् वी॒र्ये॑ण दे॒वः । \newline
34. उ॒पाꣳ॒॒शोरित्यु॑प - अꣳ॒॒शोः । \newline
35. वी॒र्ये॑ण दे॒वो दे॒वो वी॒र्ये॑ण वी॒र्ये॑ण दे॒व स्त्वा᳚ त्वा दे॒वो वी॒र्ये॑ण वी॒र्ये॑ण दे॒व स्त्वा᳚ । \newline
36. दे॒व स्त्वा᳚ त्वा दे॒वो दे॒व स्त्वा॑ सवि॒ता स॑वि॒ता त्वा॑ दे॒वो दे॒व स्त्वा॑ सवि॒ता । \newline
37. त्वा॒ स॒वि॒ता स॑वि॒ता त्वा᳚ त्वा सवि॒तोदुथ् स॑वि॒ता त्वा᳚ त्वा सवि॒तोत् । \newline
38. स॒वि॒तोदुथ् स॑वि॒ता स॑वि॒तोथ् सृ॑जतु सृज॒तूथ् स॑वि॒ता स॑वि॒तोथ् सृ॑जतु । \newline
39. उथ् सृ॑जतु सृज॒तूदुथ् सृ॑जतु जी॒वात॑वे जी॒वात॑वे सृज॒तूदुथ् सृ॑जतु जी॒वात॑वे । \newline
40. सृ॒ज॒तु॒ जी॒वात॑वे जी॒वात॑वे सृजतु सृजतु जी॒वात॑वे जीवन॒स्यायै॑ जीवन॒स्यायै॑ जी॒वात॑वे सृजतु सृजतु जी॒वात॑वे जीवन॒स्यायै᳚ । \newline
41. जी॒वात॑वे जीवन॒स्यायै॑ जीवन॒स्यायै॑ जी॒वात॑वे जी॒वात॑वे जीवन॒स्यायै॑ बृहद्रथन्त॒रयो᳚र् बृहद्रथन्त॒रयो᳚र् जीवन॒स्यायै॑ जी॒वात॑वे जी॒वात॑वे जीवन॒स्यायै॑ बृहद्रथन्त॒रयोः᳚ । \newline
42. जी॒व॒न॒स्यायै॑ बृहद्रथन्त॒रयो᳚र् बृहद्रथन्त॒रयो᳚र् जीवन॒स्यायै॑ जीवन॒स्यायै॑ बृहद्रथन्त॒रयो᳚ स्त्वा त्वा बृहद्रथन्त॒रयो᳚र् जीवन॒स्यायै॑ जीवन॒स्यायै॑ बृहद्रथन्त॒रयो᳚ स्त्वा । \newline
43. बृ॒ह॒द्र॒थ॒न्त॒रयो᳚ स्त्वा त्वा बृहद्रथन्त॒रयो᳚र् बृहद्रथन्त॒रयो᳚ स्त्वा॒ स्तोमे॑न॒ स्तोमे॑न त्वा बृहद्रथन्त॒रयो᳚र् बृहद्रथन्त॒रयो᳚ स्त्वा॒ स्तोमे॑न । \newline
44. बृ॒ह॒द्र॒थ॒न्त॒रयो॒रिति॑ बृहत् - र॒थ॒न्त॒रयोः᳚ । \newline
45. त्वा॒ स्तोमे॑न॒ स्तोमे॑न त्वा त्वा॒ स्तोमे॑न त्रि॒ष्टुभ॑ स्त्रि॒ष्टुभः॒ स्तोमे॑न त्वा त्वा॒ स्तोमे॑न त्रि॒ष्टुभः॑ । \newline
46. स्तोमे॑न त्रि॒ष्टुभ॑ स्त्रि॒ष्टुभः॒ स्तोमे॑न॒ स्तोमे॑न त्रि॒ष्टुभो॑ वर्त॒न्या व॑र्त॒न्या त्रि॒ष्टुभः॒ स्तोमे॑न॒ स्तोमे॑न त्रि॒ष्टुभो॑ वर्त॒न्या । \newline
47. त्रि॒ष्टुभो॑ वर्त॒न्या व॑र्त॒न्या त्रि॒ष्टुभ॑ स्त्रि॒ष्टुभो॑ वर्त॒न्या शु॒क्रस्य॑ शु॒क्रस्य॑ वर्त॒न्या त्रि॒ष्टुभ॑ स्त्रि॒ष्टुभो॑ वर्त॒न्या शु॒क्रस्य॑ । \newline
48. व॒र्त॒न्या शु॒क्रस्य॑ शु॒क्रस्य॑ वर्त॒न्या व॑र्त॒न्या शु॒क्रस्य॑ वी॒र्ये॑ण वी॒र्ये॑ण शु॒क्रस्य॑ वर्त॒न्या व॑र्त॒न्या शु॒क्रस्य॑ वी॒र्ये॑ण । \newline
49. शु॒क्रस्य॑ वी॒र्ये॑ण वी॒र्ये॑ण शु॒क्रस्य॑ शु॒क्रस्य॑ वी॒र्ये॑ण दे॒वो दे॒वो वी॒र्ये॑ण शु॒क्रस्य॑ शु॒क्रस्य॑ वी॒र्ये॑ण दे॒वः । \newline
50. वी॒र्ये॑ण दे॒वो दे॒वो वी॒र्ये॑ण वी॒र्ये॑ण दे॒व स्त्वा᳚ त्वा दे॒वो वी॒र्ये॑ण वी॒र्ये॑ण दे॒व स्त्वा᳚ । \newline
51. दे॒व स्त्वा᳚ त्वा दे॒वो दे॒व स्त्वा॑ सवि॒ता स॑वि॒ता त्वा॑ दे॒वो दे॒व स्त्वा॑ सवि॒ता । \newline
52. त्वा॒ स॒वि॒ता स॑वि॒ता त्वा᳚ त्वा सवि॒तोदुथ् स॑वि॒ता त्वा᳚ त्वा सवि॒तोत् । \newline
53. स॒वि॒तोदुथ् स॑वि॒ता स॑वि॒तोथ् सृ॑जतु सृज॒तूथ् स॑वि॒ता स॑वि॒तोथ् सृ॑जतु । \newline
54. उथ् सृ॑जतु सृज॒तूदुथ् सृ॑जतु जी॒वात॑वे जी॒वात॑वे सृज॒तूदुथ् सृ॑जतु जी॒वात॑वे । \newline
\pagebreak
\markright{ TS 2.3.10.3  \hfill https://www.vedavms.in \hfill}

\section{ TS 2.3.10.3 }

\textbf{TS 2.3.10.3 } \newline
\textbf{Samhita Paata} \newline

सृ॑जतु जी॒वात॑वे जीवन॒स्याया॑ अ॒ग्नेस्त्वा॒ मात्र॑या॒ जग॑त्यै वर्त॒न्याऽऽग्र॑य॒णस्य॑ वी॒र्ये॑ण दे॒वस्त्वा॑ सवि॒तोथ् सृ॑जतु जी॒वात॑वे जीवन॒स्याया॑ इ॒मम॑ग्न॒ आयु॑षे॒ वर्च॑से कृधि प्रि॒यꣳ रेतो॑ वरुण सोम राजन्न् । मा॒ तेवा᳚स्मा अदिते॒ शर्म॑ यच्छ॒ विश्वे॑ देवा॒ जर॑दष्टि॒र्यथाऽस॑त् ॥ अ॒ग्निरायु॑ष्मा॒न्थ् स वन॒स्पति॑भि॒रायु॑ष्मा॒न् तेन॒ त्वाऽऽयु॒षाऽऽयु॑ष्मन्तं करोमि॒ सोम॒ आयु॑ष्मा॒न्थ् ( ) स ओष॑धीभि र्य॒ज्ञ् आयु॑ष्मा॒न्थ् स दक्षि॑णाभि॒ र्ब्रह्माऽऽयु॑ष्म॒त् तद्-ब्रा᳚ह्म॒णैरायु॑ष्मद् दे॒वा आयु॑ष्मन्त॒स्ते॑ऽमृते॑न पि॒तर॒ आयु॑ष्मन्त॒स्ते स्व॒धयाऽऽयु॑ष्मन्त॒स्तेन॒ त्वा ऽऽयु॒षाऽऽ यु॑ष्मन्तं करोमि ॥ \newline

\textbf{Pada Paata} \newline

सृ॒ज॒तु॒ । जी॒वात॑वे । जी॒व॒न॒स्यायै᳚ । अ॒ग्नेः । त्वा॒ । मात्र॑या । जग॑त्यै । व॒र्त॒न्या । आ॒ग्र॒य॒णस्य॑ । वी॒र्ये॑ण । दे॒वः । त्वा॒ । स॒वि॒ता । उदिति॑ । सृ॒ज॒तु॒ । जी॒वात॑वे । जी॒व॒न॒स्यायै᳚ । इ॒मम् । अ॒ग्ने॒ । आयु॑षे । वर्च॑से । कृ॒धि॒ । प्रि॒यम् । रेतः॑ । व॒रु॒ण॒ । सो॒म॒ । रा॒ज॒न्न् ॥ मा॒ता । इ॒व॒ । अ॒स्मै॒ । अ॒दि॒ते॒ । शर्म॑ । य॒च्छ॒ । विश्वे᳚ । दे॒वाः॒ । जर॑दष्टि॒रिति॒ जर॑त् - अ॒ष्टिः॒ । यथा᳚ । अस॑त् ॥ अ॒ग्निः । आयु॑ष्मान् । सः । वन॒स्पति॑भि॒रिति॒ वन॒स्पति॑-भिः॒ । आयु॑ष्मान् । तेन॑ । त्वा॒ । आयु॑षा । आयु॑ष्मन्तम् । क॒रो॒मि॒ । सोमः॑ । आयु॑ष्मान् ( ) । सः । ओष॑धीभि॒रित्योष॑धि - भिः॒ । य॒ज्ञ्ः । आयु॑ष्मान् । सः । दक्षि॑णाभिः । ब्रह्म॑ । आयु॑ष्मत् । तत् । ब्रा॒ह्म॒णैः । आयु॑ष्मत् । दे॒वाः । आयु॑ष्मन्तः । ते । अ॒मृते॑न । पि॒तरः॑ । आयु॑ष्मन्तः । ते । स्व॒धयेति॑ स्व - धया᳚ । आयु॑ष्मन्तः । तेन॑ । त्वा॒ । आयु॑षा । आयु॑ष्मन्तम् । क॒रो॒मि॒ ॥  \newline


\textbf{Krama Paata} \newline

सृ॒ज॒तु॒ जी॒वात॑वे । जी॒वात॑वे जीवन॒स्यायै᳚ । जी॒व॒न॒स्याया॑ अ॒ग्नेः । अ॒ग्नेस्त्वा᳚ । त्वा॒ मात्र॑या । मात्र॑या॒ जग॑त्यै । जग॑त्यै वर्त॒न्या । व॒र्त॒न्या ऽऽग्र॑य॒णस्य॑ । आ॒ग्र॒य॒णस्य॑ वी॒र्ये॑ण । वी॒र्ये॑ण दे॒वः । दे॒वस्त्वा᳚ । त्वा॒ स॒वि॒ता । स॒वि॒तोत् । उथ् सृ॑जतु । सृ॒ज॒तु॒ जी॒वात॑वे । जी॒वात॑वे जीवन॒स्यायै᳚ । जी॒वा॒न॒स्याया॑ इ॒मम् । इ॒मम॑ग्ने । अ॒ग्न॒ आयु॑षे । आयु॑षे॒ वर्च॑से । वर्च॑से कृधि । कृ॒धि॒ प्रि॒यम् । प्रि॒यꣳ रेतः॑ । रेतो॑ वरुण । व॒रु॒ण॒ सो॒म॒ । सो॒म॒ रा॒ज॒न्न् । रा॒ज॒न्निति॑ राजन्न् ॥ मा॒तेव॑ । इ॒वा॒स्मै॒ । अ॒स्मा॒ अ॒दि॒ते॒ । अ॒दि॒ते॒ शर्म॑ । शर्म॑ यच्छ । य॒च्छ॒ विश्वे᳚ । विश्वे॑ देवाः । दे॒वा॒ जर॑दष्टिः । जर॑दष्टि॒र् यथा᳚ । जर॑दष्टि॒रिति॒ जर॑त् - अ॒ष्टिः॒ । यथा ऽस॑त् । अस॒दित्यस॑त् ॥ अ॒ग्निर् आयु॑ष्मान् । आयु॑ष्मा॒न्थ् सः । स वन॒स्पति॑भिः । वन॒स्पति॑भि॒र् आयु॑ष्मान् । वन॒स्पति॑भि॒रिति॒ वन॒स्पति॑ - भिः॒ । आयु॑ष्मा॒न् तेन॑ । तेन॑ त्वा । त्वा ऽऽयु॑षा । आयु॒षा ऽऽयु॑ष्मन्तम् । आयु॑ष्मन्तम् करोमि । क॒रो॒मि॒ सोमः॑ । सोम॒ आयु॑ष्मान् ( ) । आयु॑ष्मा॒न्थ् सः । स ओष॑धीभिः । ओष॑धीभिर् य॒ज्ञ्ः । ओष॑धीभि॒रित्योष॑धि - भिः॒ । य॒ज्ञ् आयु॑ष्मान् । आयु॑ष्मा॒न्थ् सः । स दक्षि॑णाभिः । दक्षि॑णाभि॒र् ब्रह्म॑ । ब्रह्मायु॑ष्मत् । आयु॑ष्म॒त् तत् । तद् ब्रा᳚ह्म॒णैः । ब्रा॒ह्म॒णैरायु॑ष्मत् । आयु॑ष्मद् दे॒वाः । दे॒वा आयु॑ष्मन्तः । आयु॑ष्मन्त॒स्ते । ते॑ऽमृते॑न । अ॒मृते॑न पि॒तरः॑ । पि॒तर॒ आयु॑ष्मन्तः । आयु॑ष्मन्त॒स्ते । ते स्व॒धया᳚ । स्व॒धया ऽऽयु॑ष्मन्तः । स्व॒धयेति॑ स्व - धया᳚ । आयु॑ष्मन्त॒स्तेन॑ । तेन॑ त्वा । त्वा ऽऽयु॑षा । आयु॒षा ऽऽयु॑ष्मन्तम् । आयु॑ष्मन्तम् करोमि । क॒रो॒मीति॑ करोमि । \newline

\textbf{Jatai Paata} \newline

1. सृ॒ज॒तु॒ जी॒वात॑वे जी॒वात॑वे सृजतु सृजतु जी॒वात॑वे । \newline
2. जी॒वात॑वे जीवन॒स्यायै॑ जीवन॒स्यायै॑ जी॒वात॑वे जी॒वात॑वे जीवन॒स्यायै᳚ । \newline
3. जी॒व॒न॒स्याया॑ अ॒ग्नेर॒ग्नेर् जी॑वन॒स्यायै॑ जीवन॒स्याया॑ अ॒ग्नेः । \newline
4. अ॒ग्ने स्त्वा᳚ त्वा॒ ऽग्ने र॒ग्ने स्त्वा᳚ । \newline
5. त्वा॒ मात्र॑या॒ मात्र॑या त्वा त्वा॒ मात्र॑या । \newline
6. मात्र॑या॒ जग॑त्यै॒ जग॑त्यै॒ मात्र॑या॒ मात्र॑या॒ जग॑त्यै । \newline
7. जग॑त्यै वर्त॒न्या व॑र्त॒न्या जग॑त्यै॒ जग॑त्यै वर्त॒न्या । \newline
8. व॒र्त॒न्या ऽऽग्र॑य॒णस्या᳚ ग्रय॒णस्य॑ वर्त॒न्या व॑र्त॒न्या ऽऽग्र॑य॒णस्य॑ । \newline
9. आ॒ग्र॒य॒णस्य॑ वी॒र्ये॑ण वी॒र्ये॑णा ग्रय॒णस्या᳚ ग्रय॒णस्य॑ वी॒र्ये॑ण । \newline
10. वी॒र्ये॑ण दे॒वो दे॒वो वी॒र्ये॑ण वी॒र्ये॑ण दे॒वः । \newline
11. दे॒व स्त्वा᳚ त्वा दे॒वो दे॒व स्त्वा᳚ । \newline
12. त्वा॒ स॒वि॒ता स॑वि॒ता त्वा᳚ त्वा सवि॒ता । \newline
13. स॒वि॒तो दुथ् स॑वि॒ता स॑वि॒तोत् । \newline
14. उथ् सृ॑जतु सृज॒तू दुथ् सृ॑जतु । \newline
15. सृ॒ज॒तु॒ जी॒वात॑वे जी॒वात॑वे सृजतु सृजतु जी॒वात॑वे । \newline
16. जी॒वात॑वे जीवन॒स्यायै॑ जीवन॒स्यायै॑ जी॒वात॑वे जी॒वात॑वे जीवन॒स्यायै᳚ । \newline
17. जी॒व॒न॒स्याया॑ इ॒म मि॒मम् जी॑वन॒स्यायै॑ जीवन॒स्याया॑ इ॒मम् । \newline
18. इ॒म म॑ग्ने ऽग्न इ॒म मि॒म म॑ग्ने । \newline
19. अ॒ग्न॒ आयु॑ष॒ आयु॑षे ऽग्ने ऽग्न॒ आयु॑षे । \newline
20. आयु॑षे॒ वर्च॑से॒ वर्च॑स॒ आयु॑ष॒ आयु॑षे॒ वर्च॑से । \newline
21. वर्च॑से कृधि कृधि॒ वर्च॑से॒ वर्च॑से कृधि । \newline
22. कृ॒धि॒ प्रि॒यम् प्रि॒यम् कृ॑धि कृधि प्रि॒यम् । \newline
23. प्रि॒यꣳ रेतो॒ रेतः॑ प्रि॒यम् प्रि॒यꣳ रेतः॑ । \newline
24. रेतो॑ वरुण वरुण॒ रेतो॒ रेतो॑ वरुण । \newline
25. व॒रु॒ण॒ सो॒म॒ सो॒म॒ व॒रु॒ण॒ व॒रु॒ण॒ सो॒म॒ । \newline
26. सो॒म॒ रा॒ज॒न् रा॒ज॒न् थ्सो॒म॒ सो॒म॒ रा॒ज॒न्न् । \newline
27. रा॒ज॒न्निति॑ राजन्न् । \newline
28. मा॒तेवे॑ व मा॒ता मा॒तेव॑ । \newline
29. इ॒वा॒स्मा॒ अ॒स्मा॒ इ॒वे॒ वा॒स्मै॒ । \newline
30. अ॒स्मा॒ अ॒दि॒ते॒ ऽदि॒ते॒ ऽस्मा॒ अ॒स्मा॒ अ॒दि॒ते॒ । \newline
31. अ॒दि॒ते॒ शर्म॒ शर्मा॑दिते ऽदिते॒ शर्म॑ । \newline
32. शर्म॑ यच्छ यच्छ॒ शर्म॒ शर्म॑ यच्छ । \newline
33. य॒च्छ॒ विश्वे॒ विश्वे॑ यच्छ यच्छ॒ विश्वे᳚ । \newline
34. विश्वे॑ देवा देवा॒ विश्वे॒ विश्वे॑ देवाः । \newline
35. दे॒वा॒ जर॑दष्टि॒र् जर॑दष्टिर् देवा देवा॒ जर॑दष्टिः । \newline
36. जर॑दष्टि॒र् यथा॒ यथा॒ जर॑दष्टि॒र् जर॑दष्टि॒र् यथा᳚ । \newline
37. जर॑दष्टि॒रिति॒ जर॑त् - अ॒ष्टिः॒ । \newline
38. यथा ऽस॒दस॒द् यथा॒ यथा ऽस॑त् । \newline
39. अस॒दित्यस॑त् । \newline
40. अ॒ग्नि रायु॑ष्मा॒ नायु॑ष्मा न॒ग्नि र॒ग्नि रायु॑ष्मान् । \newline
41. आयु॑ष्मा॒न् थ्स स आयु॑ष्मा॒ नायु॑ष्मा॒न् थ्सः । \newline
42. स वन॒स्पति॑भि॒र् वन॒स्पति॑भिः॒ स स वन॒स्पति॑भिः । \newline
43. वन॒स्पति॑भि॒ रायु॑ष्मा॒ नायु॑ष्मा॒न्॒. वन॒स्पति॑भि॒र् वन॒स्पति॑भि॒ रायु॑ष्मान् । \newline
44. वन॒स्पति॑भि॒रिति॒ वन॒स्पति॑ - भिः॒ । \newline
45. आयु॑ष्मा॒न् तेन॒ तेनायु॑ष्मा॒ नायु॑ष्मा॒न् तेन॑ । \newline
46. तेन॑ त्वा त्वा॒ तेन॒ तेन॑ त्वा । \newline
47. त्वा ऽऽयु॒षा ऽऽयु॑षा त्वा॒ त्वा ऽऽयु॑षा । \newline
48. आयु॒षा ऽऽयु॑ष्मन्त॒ मायु॑ष्मन्त॒ मायु॒षा ऽऽयु॒षा ऽऽयु॑ष्मन्तम् । \newline
49. आयु॑ष्मन्तम् करोमि करो॒ म्यायु॑ष्मन्त॒ मायु॑ष्मन्तम् करोमि । \newline
50. क॒रो॒मि॒ सोमः॒ सोमः॑ करोमि करोमि॒ सोमः॑ । \newline
51. सोम॒ आयु॑ष्मा॒ नायु॑ष्मा॒न् थ्सोमः॒ सोम॒ आयु॑ष्मान् । \newline
52. आयु॑ष्मा॒न् थ्स स आयु॑ष्मा॒ नायु॑ष्मा॒न् थ्सः । \newline
53. स ओष॑धीभि॒ रोष॑धीभिः॒ स स ओष॑धीभिः । \newline
54. ओष॑धीभिर् य॒ज्ञो य॒ज्ञ् ओष॑धीभि॒ रोष॑धीभिर् य॒ज्ञ्ः । \newline
55. ओष॑धीभि॒रित्योष॑धि - भिः॒ । \newline
56. य॒ज्ञ् आयु॑ष्मा॒ नायु॑ष्मान्. य॒ज्ञो य॒ज्ञ् आयु॑ष्मान् । \newline
57. आयु॑ष्मा॒न् थ्स स आयु॑ष्मा॒ नायु॑ष्मा॒न् थ्सः । \newline
58. स दक्षि॑णाभि॒र् दक्षि॑णाभिः॒ स स दक्षि॑णाभिः । \newline
59. दक्षि॑णाभि॒र् ब्रह्म॒ ब्रह्म॒ दक्षि॑णाभि॒र् दक्षि॑णाभि॒र् ब्रह्म॑ । \newline
60. ब्रह्मायु॑ष्म॒ दायु॑ष्म॒द् ब्रह्म॒ ब्रह्मायु॑ष्मत् । \newline
61. आयु॑ष्म॒त् तत् तदायु॑ष्म॒ दायु॑ष्म॒त् तत् । \newline
62. तद् ब्रा᳚ह्म॒णैर् ब्रा᳚ह्म॒णै स्तत् तद् ब्रा᳚ह्म॒णैः । \newline
63. ब्रा॒ह्म॒णै रायु॑ष्म॒दायु॑ष्मद् ब्राह्म॒णैर् ब्रा᳚ह्म॒णै रायु॑ष्मत् । \newline
64. आयु॑ष्मद् दे॒वा दे॒वा आयु॑ष्म॒ दायु॑ष्मद् दे॒वाः । \newline
65. दे॒वा आयु॑ष्मन्त॒ आयु॑ष्मन्तो दे॒वा दे॒वा आयु॑ष्मन्तः । \newline
66. आयु॑ष्मन्त॒ स्ते त आयु॑ष्मन्त॒ आयु॑ष्मन्त॒ स्ते । \newline
67. ते॑ ऽमृते॑ना॒ मृते॑न॒ ते ते॑ ऽमृते॑न । \newline
68. अ॒मृते॑न पि॒तरः॑ पि॒तरो॒ ऽमृते॑ना॒ मृते॑न पि॒तरः॑ । \newline
69. पि॒तर॒ आयु॑ष्मन्त॒ आयु॑ष्मन्तः पि॒तरः॑ पि॒तर॒ आयु॑ष्मन्तः । \newline
70. आयु॑ष्मन्त॒ स्ते त आयु॑ष्मन्त॒ आयु॑ष्मन्त॒ स्ते । \newline
71. ते स्व॒धया᳚ स्व॒धया॒ ते ते स्व॒धया᳚ । \newline
72. स्व॒धया ऽऽयु॑ष्मन्त॒ आयु॑ष्मन्तः स्व॒धया᳚ स्व॒धया ऽऽयु॑ष्मन्तः । \newline
73. स्व॒धयेति॑ स्व - धया᳚ । \newline
74. आयु॑ष्मन्त॒ स्तेन॒ तेनायु॑ष्मन्त॒ आयु॑ष्मन्त॒ स्तेन॑ । \newline
75. तेन॑ त्वा त्वा॒ तेन॒ तेन॑ त्वा । \newline
76. त्वा ऽऽयु॒षा ऽऽयु॑षा त्वा॒ त्वा ऽऽयु॑षा । \newline
77. आयु॒षा ऽऽयु॑ष्मन्त॒ मायु॑ष्मन्त॒ मायु॒षा ऽऽयु॒षा ऽऽयु॑ष्मन्तम् । \newline
78. आयु॑ष्मन्तम् करोमि करो॒ म्यायु॑ष्मन्त॒ मायु॑ष्मन्तम् करोमि । \newline
79. क॒रो॒मीति॑ करोमि । \newline

\textbf{Ghana Paata } \newline

1. सृ॒ज॒तु॒ जी॒वात॑वे जी॒वात॑वे सृजतु सृजतु जी॒वात॑वे जीवन॒स्यायै॑ जीवन॒स्यायै॑ जी॒वात॑वे सृजतु सृजतु जी॒वात॑वे जीवन॒स्यायै᳚ । \newline
2. जी॒वात॑वे जीवन॒स्यायै॑ जीवन॒स्यायै॑ जी॒वात॑वे जी॒वात॑वे जीवन॒स्याया॑ अ॒ग्ने र॒ग्नेर् जी॑वन॒स्यायै॑ जी॒वात॑वे जी॒वात॑वे जीवन॒स्याया॑ अ॒ग्नेः । \newline
3. जी॒व॒न॒स्याया॑ अ॒ग्ने र॒ग्नेर् जी॑वन॒स्यायै॑ जीवन॒स्याया॑ अ॒ग्ने स्त्वा᳚ त्वा॒ ऽग्नेर् जी॑वन॒स्यायै॑ जीवन॒स्याया॑ अ॒ग्ने स्त्वा᳚ । \newline
4. अ॒ग्ने स्त्वा᳚ त्वा॒ ऽग्ने र॒ग्ने स्त्वा॒ मात्र॑या॒ मात्र॑या त्वा॒ ऽग्ने र॒ग्ने स्त्वा॒ मात्र॑या । \newline
5. त्वा॒ मात्र॑या॒ मात्र॑या त्वा त्वा॒ मात्र॑या॒ जग॑त्यै॒ जग॑त्यै॒ मात्र॑या त्वा त्वा॒ मात्र॑या॒ जग॑त्यै । \newline
6. मात्र॑या॒ जग॑त्यै॒ जग॑त्यै॒ मात्र॑या॒ मात्र॑या॒ जग॑त्यै वर्त॒न्या व॑र्त॒न्या जग॑त्यै॒ मात्र॑या॒ मात्र॑या॒ जग॑त्यै वर्त॒न्या । \newline
7. जग॑त्यै वर्त॒न्या व॑र्त॒न्या जग॑त्यै॒ जग॑त्यै वर्त॒न्या ऽऽग्र॑य॒णस्या᳚ ग्रय॒णस्य॑ वर्त॒न्या जग॑त्यै॒ जग॑त्यै वर्त॒न्या ऽऽग्र॑य॒णस्य॑ । \newline
8. व॒र्त॒न्या ऽऽग्र॑य॒णस्या᳚ ग्रय॒णस्य॑ वर्त॒न्या व॑र्त॒न्या ऽऽग्र॑य॒णस्य॑ वी॒र्ये॑ण वी॒र्ये॑णा ग्रय॒णस्य॑ वर्त॒न्या व॑र्त॒न्या ऽऽग्र॑य॒णस्य॑ वी॒र्ये॑ण । \newline
9. आ॒ग्र॒य॒णस्य॑ वी॒र्ये॑ण वी॒र्ये॑णा ग्रय॒णस्या᳚ ग्रय॒णस्य॑ वी॒र्ये॑ण दे॒वो दे॒वो वी॒र्ये॑णा ग्रय॒णस्या᳚ ग्रय॒णस्य॑ वी॒र्ये॑ण दे॒वः । \newline
10. वी॒र्ये॑ण दे॒वो दे॒वो वी॒र्ये॑ण वी॒र्ये॑ण दे॒व स्त्वा᳚ त्वा दे॒वो वी॒र्ये॑ण वी॒र्ये॑ण दे॒व स्त्वा᳚ । \newline
11. दे॒व स्त्वा᳚ त्वा दे॒वो दे॒व स्त्वा॑ सवि॒ता स॑वि॒ता त्वा॑ दे॒वो दे॒व स्त्वा॑ सवि॒ता । \newline
12. त्वा॒ स॒वि॒ता स॑वि॒ता त्वा᳚ त्वा सवि॒तोदुथ् स॑वि॒ता त्वा᳚ त्वा सवि॒तोत् । \newline
13. स॒वि॒तोदुथ् स॑वि॒ता स॑वि॒तोथ् सृ॑जतु सृज॒तूथ् स॑वि॒ता स॑वि॒तोथ् सृ॑जतु । \newline
14. उथ् सृ॑जतु सृज॒तूदुथ् सृ॑जतु जी॒वात॑वे जी॒वात॑वे सृज॒तूदुथ् सृ॑जतु जी॒वात॑वे । \newline
15. सृ॒ज॒तु॒ जी॒वात॑वे जी॒वात॑वे सृजतु सृजतु जी॒वात॑वे जीवन॒स्यायै॑ जीवन॒स्यायै॑ जी॒वात॑वे सृजतु सृजतु जी॒वात॑वे जीवन॒स्यायै᳚ । \newline
16. जी॒वात॑वे जीवन॒स्यायै॑ जीवन॒स्यायै॑ जी॒वात॑वे जी॒वात॑वे जीवन॒स्याया॑ इ॒म मि॒मम् जी॑वन॒स्यायै॑ जी॒वात॑वे जी॒वात॑वे जीवन॒स्याया॑ इ॒मम् । \newline
17. जी॒व॒न॒स्याया॑ इ॒म मि॒मम् जी॑वन॒स्यायै॑ जीवन॒स्याया॑ इ॒म म॑ग्ने ऽग्न इ॒मम् जी॑वन॒स्यायै॑ जीवन॒स्याया॑ इ॒म म॑ग्ने । \newline
18. इ॒म म॑ग्ने ऽग्न इ॒म मि॒म म॑ग्न॒ आयु॑ष॒ आयु॑षे ऽग्न इ॒म मि॒म म॑ग्न॒ आयु॑षे । \newline
19. अ॒ग्न॒ आयु॑ष॒ आयु॑षे ऽग्ने ऽग्न॒ आयु॑षे॒ वर्च॑से॒ वर्च॑स॒ आयु॑षे ऽग्ने ऽग्न॒ आयु॑षे॒ वर्च॑से । \newline
20. आयु॑षे॒ वर्च॑से॒ वर्च॑स॒ आयु॑ष॒ आयु॑षे॒ वर्च॑से कृधि कृधि॒ वर्च॑स॒ आयु॑ष॒ आयु॑षे॒ वर्च॑से कृधि । \newline
21. वर्च॑से कृधि कृधि॒ वर्च॑से॒ वर्च॑से कृधि प्रि॒यम् प्रि॒यम् कृ॑धि॒ वर्च॑से॒ वर्च॑से कृधि प्रि॒यम् । \newline
22. कृ॒धि॒ प्रि॒यम् प्रि॒यम् कृ॑धि कृधि प्रि॒यꣳ रेतो॒ रेतः॑ प्रि॒यम् कृ॑धि कृधि प्रि॒यꣳ रेतः॑ । \newline
23. प्रि॒यꣳ रेतो॒ रेतः॑ प्रि॒यम् प्रि॒यꣳ रेतो॑ वरुण वरुण॒ रेतः॑ प्रि॒यम् प्रि॒यꣳ रेतो॑ वरुण । \newline
24. रेतो॑ वरुण वरुण॒ रेतो॒ रेतो॑ वरुण सोम सोम वरुण॒ रेतो॒ रेतो॑ वरुण सोम । \newline
25. व॒रु॒ण॒ सो॒म॒ सो॒म॒ व॒रु॒ण॒ व॒रु॒ण॒ सो॒म॒ रा॒ज॒न् रा॒ज॒न् थ्सो॒म॒ व॒रु॒ण॒ व॒रु॒ण॒ सो॒म॒ रा॒ज॒न्न् । \newline
26. सो॒म॒ रा॒ज॒न् रा॒ज॒न् थ्सो॒म॒ सो॒म॒ रा॒ज॒न्न् । \newline
27. रा॒ज॒न्निति॑ राजन्न् । \newline
28. मा॒तेवे॑ व मा॒ता मा॒तेवा᳚स्मा अस्मा इव मा॒ता मा॒तेवा᳚स्मै । \newline
29. इ॒वा॒स्मा॒ अ॒स्मा॒ इ॒वे॒ वा॒स्मा॒ अ॒दि॒ते॒ ऽदि॒ते॒ ऽस्मा॒ इ॒वे॒ वा॒स्मा॒ अ॒दि॒ते॒ । \newline
30. अ॒स्मा॒ अ॒दि॒ते॒ ऽदि॒ते॒ ऽस्मा॒ अ॒स्मा॒ अ॒दि॒ते॒ शर्म॒ शर्मा॑दिते ऽस्मा अस्मा अदिते॒ शर्म॑ । \newline
31. अ॒दि॒ते॒ शर्म॒ शर्मा॑दिते ऽदिते॒ शर्म॑ यच्छ यच्छ॒ शर्मा॑दिते ऽदिते॒ शर्म॑ यच्छ । \newline
32. शर्म॑ यच्छ यच्छ॒ शर्म॒ शर्म॑ यच्छ॒ विश्वे॒ विश्वे॑ यच्छ॒ शर्म॒ शर्म॑ यच्छ॒ विश्वे᳚ । \newline
33. य॒च्छ॒ विश्वे॒ विश्वे॑ यच्छ यच्छ॒ विश्वे॑ देवा देवा॒ विश्वे॑ यच्छ यच्छ॒ विश्वे॑ देवाः । \newline
34. विश्वे॑ देवा देवा॒ विश्वे॒ विश्वे॑ देवा॒ जर॑दष्टि॒र् जर॑दष्टिर् देवा॒ विश्वे॒ विश्वे॑ देवा॒ जर॑दष्टिः । \newline
35. दे॒वा॒ जर॑दष्टि॒र् जर॑दष्टिर् देवा देवा॒ जर॑दष्टि॒र् यथा॒ यथा॒ जर॑दष्टिर् देवा देवा॒ जर॑दष्टि॒र् यथा᳚ । \newline
36. जर॑दष्टि॒र् यथा॒ यथा॒ जर॑दष्टि॒र् जर॑दष्टि॒र् यथा ऽस॒ दस॒द् यथा॒ जर॑दष्टि॒र् जर॑दष्टि॒र् यथा ऽस॑त् । \newline
37. जर॑दष्टि॒रिति॒ जर॑त् - अ॒ष्टिः॒ । \newline
38. यथा ऽस॒ दस॒द् यथा॒ यथा ऽस॑त् । \newline
39. अस॒दित्यस॑त् । \newline
40. अ॒ग्नि रायु॑ष्मा॒ नायु॑ष्मा न॒ग्नि र॒ग्नि रायु॑ष्मा॒न् थ्स स आयु॑ष्मा न॒ग्नि र॒ग्नि रायु॑ष्मा॒न् थ्सः । \newline
41. आयु॑ष्मा॒न् थ्स स आयु॑ष्मा॒ नायु॑ष्मा॒न् थ्स वन॒स्पति॑भि॒र् वन॒स्पति॑भिः॒ स आयु॑ष्मा॒ नायु॑ष्मा॒न् थ्स वन॒स्पति॑भिः । \newline
42. स वन॒स्पति॑भि॒र् वन॒स्पति॑भिः॒ स स वन॒स्पति॑भि॒ रायु॑ष्मा॒ नायु॑ष्मा॒न्॒. वन॒स्पति॑भिः॒ स स वन॒स्पति॑भि॒ रायु॑ष्मान् । \newline
43. वन॒स्पति॑भि॒ रायु॑ष्मा॒ नायु॑ष्मा॒न्॒. वन॒स्पति॑भि॒र् वन॒स्पति॑भि॒ रायु॑ष्मा॒न् तेन॒ तेनायु॑ष्मा॒न्॒. वन॒स्पति॑भि॒र् वन॒स्पति॑भि॒ रायु॑ष्मा॒न् तेन॑ । \newline
44. वन॒स्पति॑भि॒रिति॒ वन॒स्पति॑ - भिः॒ । \newline
45. आयु॑ष्मा॒न् तेन॒ तेनायु॑ष्मा॒ नायु॑ष्मा॒न् तेन॑ त्वा त्वा॒ तेनायु॑ष्मा॒ नायु॑ष्मा॒न् तेन॑ त्वा । \newline
46. तेन॑ त्वा त्वा॒ तेन॒ तेन॒ त्वा ऽऽयु॒षा ऽऽयु॑षा त्वा॒ तेन॒ तेन॒ त्वा ऽऽयु॑षा । \newline
47. त्वा ऽऽयु॒षा ऽऽयु॑षा त्वा॒ त्वा ऽऽयु॒षा ऽऽयु॑ष्मन्त॒ मायु॑ष्मन्त॒ मायु॑षा त्वा॒ त्वा ऽऽयु॒षा ऽऽयु॑ष्मन्तम् । \newline
48. आयु॒षा ऽऽयु॑ष्मन्त॒ मायु॑ष्मन्त॒ मायु॒षा ऽऽयु॒षा ऽऽयु॑ष्मन्तम् करोमि करो॒ म्यायु॑ष्मन्त॒ मायु॒षा ऽऽयु॒षा ऽऽयु॑ष्मन्तम् करोमि । \newline
49. आयु॑ष्मन्तम् करोमि करो॒ म्यायु॑ष्मन्त॒ मायु॑ष्मन्तम् करोमि॒ सोमः॒ सोमः॑ करो॒ म्यायु॑ष्मन्त॒ मायु॑ष्मन्तम् करोमि॒ सोमः॑ । \newline
50. क॒रो॒मि॒ सोमः॒ सोमः॑ करोमि करोमि॒ सोम॒ आयु॑ष्मा॒ नायु॑ष्मा॒न् थ्सोमः॑ करोमि करोमि॒ सोम॒ आयु॑ष्मान् । \newline
51. सोम॒ आयु॑ष्मा॒ नायु॑ष्मा॒न् थ्सोमः॒ सोम॒ आयु॑ष्मा॒न् थ्स स आयु॑ष्मा॒न् थ्सोमः॒ सोम॒ आयु॑ष्मा॒न् थ्सः । \newline
52. आयु॑ष्मा॒न् थ्स स आयु॑ष्मा॒ नायु॑ष्मा॒न् थ्स ओष॑धीभि॒ रोष॑धीभिः॒ स आयु॑ष्मा॒ नायु॑ष्मा॒न् थ्स ओष॑धीभिः । \newline
53. स ओष॑धीभि॒ रोष॑धीभिः॒ स स ओष॑धीभिर् य॒ज्ञो य॒ज्ञ् ओष॑धीभिः॒ स स ओष॑धीभिर् य॒ज्ञ्ः । \newline
54. ओष॑धीभिर् य॒ज्ञो य॒ज्ञ् ओष॑धीभि॒ रोष॑धीभिर् य॒ज्ञ् आयु॑ष्मा॒ नायु॑ष्मान्. य॒ज्ञ् ओष॑धीभि॒ रोष॑धीभिर् य॒ज्ञ् आयु॑ष्मान् । \newline
55. ओष॑धीभि॒रित्योष॑धि - भिः॒ । \newline
56. य॒ज्ञ् आयु॑ष्मा॒ नायु॑ष्मान्. य॒ज्ञो य॒ज्ञ् आयु॑ष्मा॒न् थ्स स आयु॑ष्मान्. य॒ज्ञो य॒ज्ञ् आयु॑ष्मा॒न् थ्सः । \newline
57. आयु॑ष्मा॒न् थ्स स आयु॑ष्मा॒ नायु॑ष्मा॒न् थ्स दक्षि॑णाभि॒र् दक्षि॑णाभिः॒ स आयु॑ष्मा॒ नायु॑ष्मा॒न् थ्स दक्षि॑णाभिः । \newline
58. स दक्षि॑णाभि॒र् दक्षि॑णाभिः॒ स स दक्षि॑णाभि॒र् ब्रह्म॒ ब्रह्म॒ दक्षि॑णाभिः॒ स स दक्षि॑णाभि॒र् ब्रह्म॑ । \newline
59. दक्षि॑णाभि॒र् ब्रह्म॒ ब्रह्म॒ दक्षि॑णाभि॒र् दक्षि॑णाभि॒र् ब्रह्मायु॑ष्म॒ दायु॑ष्म॒द् ब्रह्म॒ दक्षि॑णाभि॒र् दक्षि॑णाभि॒र् ब्रह्मायु॑ष्मत् । \newline
60. ब्रह्मायु॑ष्म॒ दायु॑ष्म॒द् ब्रह्म॒ ब्रह्मायु॑ष्म॒त् तत् तदायु॑ष्म॒द् ब्रह्म॒ ब्रह्मायु॑ष्म॒त् तत् । \newline
61. आयु॑ष्म॒त् तत् तदायु॑ष्म॒ दायु॑ष्म॒त् तद् ब्रा᳚ह्म॒णैर् ब्रा᳚ह्म॒णै स्तदायु॑ष्म॒ दायु॑ष्म॒त् तद् ब्रा᳚ह्म॒णैः । \newline
62. तद् ब्रा᳚ह्म॒णैर् ब्रा᳚ह्म॒णै स्तत् तद् ब्रा᳚ह्म॒णै रायु॑ष्म॒ दायु॑ष्मद् ब्राह्म॒णै स्तत् तद् ब्रा᳚ह्म॒णै रायु॑ष्मत् । \newline
63. ब्रा॒ह्म॒णै रायु॑ष्म॒ दायु॑ष्मद् ब्राह्म॒णैर् ब्रा᳚ह्म॒णै रायु॑ष्मद् दे॒वा दे॒वा आयु॑ष्मद् ब्राह्म॒णैर् ब्रा᳚ह्म॒णै रायु॑ष्मद् दे॒वाः । \newline
64. आयु॑ष्मद् दे॒वा दे॒वा आयु॑ष्म॒ दायु॑ष्मद् दे॒वा आयु॑ष्मन्त॒ आयु॑ष्मन्तो दे॒वा आयु॑ष्म॒ दायु॑ष्मद् दे॒वा आयु॑ष्मन्तः । \newline
65. दे॒वा आयु॑ष्मन्त॒ आयु॑ष्मन्तो दे॒वा दे॒वा आयु॑ष्मन्त॒ स्ते त आयु॑ष्मन्तो दे॒वा दे॒वा आयु॑ष्मन्त॒ स्ते । \newline
66. आयु॑ष्मन्त॒ स्ते त आयु॑ष्मन्त॒ आयु॑ष्मन्त॒ स्ते॑ ऽमृते॑ना॒मृते॑न॒ त आयु॑ष्मन्त॒ आयु॑ष्मन्त॒ स्ते॑ ऽमृते॑न । \newline
67. ते॑ ऽमृते॑ना॒मृते॑न॒ ते ते॑ ऽमृते॑न पि॒तरः॑ पि॒तरो॒ ऽमृते॑न॒ ते ते॑ ऽमृते॑न पि॒तरः॑ । \newline
68. अ॒मृते॑न पि॒तरः॑ पि॒तरो॒ ऽमृते॑ना॒मृते॑न पि॒तर॒ आयु॑ष्मन्त॒ आयु॑ष्मन्तः पि॒तरो॒ ऽमृते॑ना॒मृते॑न पि॒तर॒ आयु॑ष्मन्तः । \newline
69. पि॒तर॒ आयु॑ष्मन्त॒ आयु॑ष्मन्तः पि॒तरः॑ पि॒तर॒ आयु॑ष्मन्त॒ स्ते त आयु॑ष्मन्तः पि॒तरः॑ पि॒तर॒ आयु॑ष्मन्त॒ स्ते । \newline
70. आयु॑ष्मन्त॒ स्ते त आयु॑ष्मन्त॒ आयु॑ष्मन्त॒ स्ते स्व॒धया᳚ स्व॒धया॒ त आयु॑ष्मन्त॒ आयु॑ष्मन्त॒ स्ते स्व॒धया᳚ । \newline
71. ते स्व॒धया᳚ स्व॒धया॒ ते ते स्व॒धया ऽऽयु॑ष्मन्त॒ आयु॑ष्मन्तः स्व॒धया॒ ते ते स्व॒धया ऽऽयु॑ष्मन्तः । \newline
72. स्व॒धया ऽऽयु॑ष्मन्त॒ आयु॑ष्मन्तः स्व॒धया᳚ स्व॒धया ऽऽयु॑ष्मन्त॒ स्तेन॒ तेनायु॑ष्मन्तः स्व॒धया᳚ स्व॒धया ऽऽयु॑ष्मन्त॒ स्तेन॑ । \newline
73. स्व॒धयेति॑ स्व - धया᳚ । \newline
74. आयु॑ष्मन्त॒ स्तेन॒ तेनायु॑ष्मन्त॒ आयु॑ष्मन्त॒ स्तेन॑ त्वा त्वा॒ तेनायु॑ष्मन्त॒ आयु॑ष्मन्त॒ स्तेन॑ त्वा । \newline
75. तेन॑ त्वा त्वा॒ तेन॒ तेन॒ त्वा ऽऽयु॒षा ऽऽयु॑षा त्वा॒ तेन॒ तेन॒ त्वा ऽऽयु॑षा । \newline
76. त्वा ऽऽयु॒षा ऽऽयु॑षा त्वा॒ त्वा ऽऽयु॒षा ऽऽयु॑ष्मन्त॒ मायु॑ष्मन्त॒ मायु॑षा त्वा॒ त्वा ऽऽयु॒षा ऽऽयु॑ष्मन्तम् । \newline
77. आयु॒षा ऽऽयु॑ष्मन्त॒ मायु॑ष्मन्त॒ मायु॒षा ऽऽयु॒षा ऽऽयु॑ष्मन्तम् करोमि करो॒ म्यायु॑ष्मन्त॒ मायु॒षा ऽऽयु॒षा ऽऽयु॑ष्मन्तम् करोमि । \newline
78. आयु॑ष्मन्तम् करोमि करो॒ म्यायु॑ष्मन्त॒ मायु॑ष्मन्तम् करोमि । \newline
79. क॒रो॒मीति॑ करोमि । \newline
\pagebreak
\markright{ TS 2.3.11.1  \hfill https://www.vedavms.in \hfill}

\section{ TS 2.3.11.1 }

\textbf{TS 2.3.11.1 } \newline
\textbf{Samhita Paata} \newline

अ॒ग्निं ॅवा ए॒तस्य॒ शरी॑रं गच्छति॒ सोमꣳ॒॒ रसो॒ वरु॑ण एनं ॅवरुणपा॒शेन॑ गृह्णाति॒ सर॑स्वतीं॒ ॅवाग॒ग्नाविष्णू॑ आ॒त्मा यस्य॒ ज्योगा॒मय॑ति॒ यो ज्योगा॑मयावी॒ स्याद्यो वा॑ का॒मये॑त॒ सर्व॒मायु॑रिया॒मिति॒ तस्मा॑ ए॒तामिष्टिं॒ निर्व॑पेदाग्ने॒य -म॒ष्टाक॑पालꣳ सौ॒म्यं च॒रुं ॅवा॑रु॒णं दश॑कपालꣳ सारस्व॒तं च॒रुमा᳚ग्नावैष्ण॒व-मेका॑दशकपाल-म॒ग्नेरे॒वास्य॒ शरी॑रं निष्क्री॒णाति॒ सोमा॒द्रसं॑ - [  ] \newline

\textbf{Pada Paata} \newline

अ॒ग्निम् । वै । ए॒तस्य॑ । शरी॑रम् । ग॒च्छ॒ति॒ । सोम᳚म् । रसः॑ । वरु॑णः । ए॒न॒म् । व॒रु॒ण॒पा॒शेनेति॑ वरुण - पा॒शेन॑ । गृ॒ह्णा॒ति॒ । सर॑स्वतीम् । वाक् । अ॒ग्नाविष्णू॒ इत्य॒ग्ना - विष्णू᳚ । आ॒त्मा । यस्य॑ । ज्योक् । आ॒मय॑ति । यः । ज्योगा॑मया॒वीति॒ ज्योक् - आ॒म॒या॒वी॒ । स्यात् । यः । वा॒ । का॒मये॑त । सर्व᳚म् । आयुः॑ । इ॒या॒म् । इति॑ । तस्मै᳚ । ए॒ताम् । इष्टि᳚म् । निरिति॑ । व॒पे॒त् । आ॒ग्ने॒यम् । अ॒ष्टाक॑पाल॒मित्य॒ष्टा - क॒पा॒ल॒म् । सौ॒म्यम् । च॒रुम् । वा॒रु॒णम् । दश॑कपाल॒मिति॒ दश॑ - क॒पा॒ल॒म् । सा॒र॒स्व॒तम् । च॒रुम् । आ॒ग्ना॒वै॒ष्ण॒वमित्या᳚ग्ना - वै॒ष्ण॒वम् । एका॑दशकपाल॒मित्येका॑दश - क॒पा॒ल॒म् । अ॒ग्नेः । ए॒व । अ॒स्य॒ । शरी॑रम् । नि॒ष्क्री॒णातीति॑ निः - क्री॒णाति॑ । सोमा᳚त् । रस᳚म् ।  \newline


\textbf{Krama Paata} \newline

अ॒ग्निं ॅवै । वा ए॒तस्य॑ । ए॒तस्य॒ शरी॑रम् । शरी॑रम् गच्छति । ग॒च्छ॒ति॒ सोम᳚म् । सोमꣳ॒॒ रसः॑ । रसो॒ वरु॑णः । वरु॑ण एनम् । ए॒नं॒ ॅव॒रु॒ण॒पा॒शेन॑ । व॒रु॒ण॒पा॒शेन॑ गृह्णाति । व॒रु॒ण॒पा॒शेनेति॑ वरुण - पा॒शेन॑ । गृ॒ह्णा॒ति॒ सर॑स्वतीम् । सर॑स्वतीं॒ ॅवाक् । वाग॒ग्नाविष्णू᳚ । अ॒ग्नाविष्णू॑ आ॒त्मा । अ॒ग्नाविष्णू॒ इत्य॒ग्ना - विष्णू᳚ । आ॒त्मा यस्य॑ । यस्य॒ ज्योक् । ज्योगा॒मय॑ति । आ॒मय॑ति॒ यः । यो ज्योगा॑मयावी । ज्योगा॑मयावी॒ स्यात् । ज्योगा॑मया॒वीति॒ ज्योक् - आ॒म॒या॒वी॒ । स्याद् यः । यो वा᳚ । वा॒ का॒मये॑त । का॒मये॑त॒ सर्व᳚म् । सर्व॒मायुः॑ । आयु॑रियाम् । इ॒या॒मिति॑ । इति॒ तस्मै᳚ । तस्मा॑ ए॒ताम् । ए॒तामिष्टि᳚म् । इष्टि॒म् निः । निर् व॑पेत् । व॒पे॒दा॒ग्ने॒यम् । आ॒ग्ने॒यम॒ष्टाक॑पालम् । अ॒ष्टाक॑पालꣳ सौ॒म्यम् । अ॒ष्टाक॑पाल॒मित्य॒ष्टा - क॒पा॒ल॒म् । सौ॒म्यम् च॒रुम् । च॒रुं ॅवा॑रु॒णम् । वा॒रु॒णम् दश॑कपालम् । दश॑कपालꣳ सारस्व॒तम् । दश॑कपाल॒मिति॒ दश॑ - क॒पा॒ल॒म् । सा॒र॒स्व॒तम् च॒रुम् । च॒रुमा᳚ग्नावैष्ण॒वम् । आ॒ग्ना॒वै॒ष्ण॒वमेका॑दशकपालम् । आ॒ग्ना॒वै॒ष्ण॒वमित्या᳚ग्ना - वै॒ष्ण॒वम् । एका॑दशकपालम॒ग्नेः । एका॑दशकपाल॒मित्येका॑दश - क॒पा॒ल॒म् । अ॒ग्नेरे॒व । ए॒वास्य॑ । अ॒स्य॒ शरी॑रम् । शरी॑रन्निष्क्री॒णाति॑ । नि॒ष्क्री॒णाति॒ सोमा᳚त् । नि॒ष्क्री॒णातीति॑ निः - क्री॒णाति॑ । सोमा॒द् रस᳚म् । रसं॑ ॅवारु॒णेन॑ \newline

\textbf{Jatai Paata} \newline

1. अ॒ग्निं ॅवै वा अ॒ग्नि म॒ग्निं ॅवै । \newline
2. वा ए॒त स्यै॒तस्य॒ वै वा ए॒तस्य॑ । \newline
3. ए॒तस्य॒ शरी॑रꣳ॒॒ शरी॑र मे॒तस्यै॒तस्य॒ शरी॑रम् । \newline
4. शरी॑रम् गच्छति गच्छति॒ शरी॑रꣳ॒॒ शरी॑रम् गच्छति । \newline
5. ग॒च्छ॒ति॒ सोमꣳ॒॒ सोम॑म् गच्छति गच्छति॒ सोम᳚म् । \newline
6. सोमꣳ॒॒ रसो॒ रसः॒ सोमꣳ॒॒ सोमꣳ॒॒ रसः॑ । \newline
7. रसो॒ वरु॑णो॒ वरु॑णो॒ रसो॒ रसो॒ वरु॑णः । \newline
8. वरु॑ण एन मेनं॒ ॅवरु॑णो॒ वरु॑ण एनम् । \newline
9. ए॒नं॒ ॅव॒रु॒ण॒पा॒शेन॑ वरुणपा॒शेनै॑न मेनं ॅवरुणपा॒शेन॑ । \newline
10. व॒रु॒ण॒पा॒शेन॑ गृह्णाति गृह्णाति वरुणपा॒शेन॑ वरुणपा॒शेन॑ गृह्णाति । \newline
11. व॒रु॒ण॒पा॒शेनेति॑ वरुण - पा॒शेन॑ । \newline
12. गृ॒ह्णा॒ति॒ सर॑स्वतीꣳ॒॒ सर॑स्वतीम् गृह्णाति गृह्णाति॒ सर॑स्वतीम् । \newline
13. सर॑स्वतीं॒ ॅवाग् वाख् सर॑स्वतीꣳ॒॒ सर॑स्वतीं॒ ॅवाक् । \newline
14. वाग॒ग्नाविष्णू॑ अ॒ग्नाविष्णू॒ वाग् वाग॒ग्नाविष्णू᳚ । \newline
15. अ॒ग्नाविष्णू॑ आ॒त्मा ऽऽत्मा ऽग्नाविष्णू॑ अ॒ग्नाविष्णू॑ आ॒त्मा । \newline
16. अ॒ग्नाविष्णू॒ इत्य॒ग्ना - विष्णू᳚ । \newline
17. आ॒त्मा यस्य॒ यस्या॒त्मा ऽऽत्मा यस्य॑ । \newline
18. यस्य॒ ज्योग् ज्योग् यस्य॒ यस्य॒ ज्योक् । \newline
19. ज्योगा॒मय॑ त्या॒मय॑ति॒ ज्योग् ज्योगा॒मय॑ति । \newline
20. आ॒मय॑ति॒ यो य आ॒मय॑ त्या॒मय॑ति॒ यः । \newline
21. यो ज्योगा॑मयावी॒ ज्योगा॑मयावी॒ यो यो ज्योगा॑मयावी । \newline
22. ज्योगा॑मयावी॒ स्याथ् स्याज् ज्योगा॑मयावी॒ ज्योगा॑मयावी॒ स्यात् । \newline
23. ज्योगा॑मया॒वीति॒ ज्योक् - आ॒म॒या॒वी॒ । \newline
24. स्याद् यो यः स्याथ् स्याद् यः । \newline
25. यो वा॑ वा॒ यो यो वा᳚ । \newline
26. वा॒ का॒मये॑त का॒मये॑त वा वा का॒मये॑त । \newline
27. का॒मये॑त॒ सर्वꣳ॒॒ सर्व॑म् का॒मये॑त का॒मये॑त॒ सर्व᳚म् । \newline
28. सर्व॒ मायु॒ रायुः॒ सर्वꣳ॒॒ सर्व॒ मायुः॑ । \newline
29. आयु॑ रिया मिया॒ मायु॒ रायु॑ रियाम् । \newline
30. इ॒या॒ मितीती॑या मिया॒ मिति॑ । \newline
31. इति॒ तस्मै॒ तस्मा॒ इतीति॒ तस्मै᳚ । \newline
32. तस्मा॑ ए॒ता मे॒ताम् तस्मै॒ तस्मा॑ ए॒ताम् । \newline
33. ए॒ता मिष्टि॒ मिष्टि॑ मे॒ता मे॒ता मिष्टि᳚म् । \newline
34. इष्टि॒म् निर् णिरिष्टि॒ मिष्टि॒म् निः । \newline
35. निर् व॑पेद् वपे॒न् निर् णिर् व॑पेत् । \newline
36. व॒पे॒ दा॒ग्ने॒य मा᳚ग्ने॒यं ॅव॑पेद् वपे दाग्ने॒यम् । \newline
37. आ॒ग्ने॒य म॒ष्टाक॑पाल म॒ष्टाक॑पाल माग्ने॒य मा᳚ग्ने॒य म॒ष्टाक॑पालम् । \newline
38. अ॒ष्टाक॑पालꣳ सौ॒म्यꣳ सौ॒म्य म॒ष्टाक॑पाल म॒ष्टाक॑पालꣳ सौ॒म्यम् । \newline
39. अ॒ष्टाक॑पाल॒मित्य॒ष्टा - क॒पा॒ल॒म् । \newline
40. सौ॒म्यम् च॒रुम् च॒रुꣳ सौ॒म्यꣳ सौ॒म्यम् च॒रुम् । \newline
41. च॒रुं ॅवा॑रु॒णं ॅवा॑रु॒णम् च॒रुम् च॒रुं ॅवा॑रु॒णम् । \newline
42. वा॒रु॒णम् दश॑कपाल॒म् दश॑कपालं ॅवारु॒णं ॅवा॑रु॒णम् दश॑कपालम् । \newline
43. दश॑कपालꣳ सारस्व॒तꣳ सा॑रस्व॒तम् दश॑कपाल॒म् दश॑कपालꣳ सारस्व॒तम् । \newline
44. दश॑कपाल॒मिति॒ दश॑ - क॒पा॒ल॒म् । \newline
45. सा॒र॒स्व॒तम् च॒रुम् च॒रुꣳ सा॑रस्व॒तꣳ सा॑रस्व॒तम् च॒रुम् । \newline
46. च॒रु मा᳚ग्नावैष्ण॒व मा᳚ग्नावैष्ण॒वम् च॒रुम् च॒रु मा᳚ग्नावैष्ण॒वम् । \newline
47. आ॒ग्ना॒वै॒ष्ण॒व मेका॑दशकपाल॒ मेका॑दशकपाल माग्नावैष्ण॒व मा᳚ग्नावैष्ण॒व मेका॑दशकपालम् । \newline
48. आ॒ग्ना॒वै॒ष्ण॒वमित्या᳚ग्ना - वै॒ष्ण॒वम् । \newline
49. एका॑दशकपाल म॒ग्ने र॒ग्ने रेका॑दशकपाल॒ मेका॑दशकपाल म॒ग्नेः । \newline
50. एका॑दशकपाल॒मित्येका॑दश - क॒पा॒ल॒म् । \newline
51. अ॒ग्ने रे॒वै वाग्ने र॒ग्ने रे॒व । \newline
52. ए॒वा स्या᳚स्यै॒ वैवास्य॑ । \newline
53. अ॒स्य॒ शरी॑रꣳ॒॒ शरी॑र मस्यास्य॒ शरी॑रम् । \newline
54. शरी॑रम् निष्क्री॒णाति॑ निष्क्री॒णाति॒ शरी॑रꣳ॒॒ शरी॑रम् निष्क्री॒णाति॑ । \newline
55. नि॒ष्क्री॒णाति॒ सोमा॒थ् सोमा᳚न् निष्क्री॒णाति॑ निष्क्री॒णाति॒ सोमा᳚त् । \newline
56. नि॒ष्क्री॒णातीति॑ निः - क्री॒णाति॑ । \newline
57. सोमा॒द् रसꣳ॒॒ रसꣳ॒॒ सोमा॒थ् सोमा॒द् रस᳚म् । \newline
58. रसं॑ ॅवारु॒णेन॑ वारु॒णेन॒ रसꣳ॒॒ रसं॑ ॅवारु॒णेन॑ । \newline

\textbf{Ghana Paata } \newline

1. अ॒ग्निं ॅवै वा अ॒ग्नि म॒ग्निं ॅवा ए॒तस्यै॒तस्य॒ वा अ॒ग्नि म॒ग्निं ॅवा ए॒तस्य॑ । \newline
2. वा ए॒तस्यै॒तस्य॒ वै वा ए॒तस्य॒ शरी॑रꣳ॒॒ शरी॑र मे॒तस्य॒ वै वा ए॒तस्य॒ शरी॑रम् । \newline
3. ए॒तस्य॒ शरी॑रꣳ॒॒ शरी॑र मे॒तस्यै॒तस्य॒ शरी॑रम् गच्छति गच्छति॒ शरी॑र मे॒तस्यै॒तस्य॒ शरी॑रम् गच्छति । \newline
4. शरी॑रम् गच्छति गच्छति॒ शरी॑रꣳ॒॒ शरी॑रम् गच्छति॒ सोमꣳ॒॒ सोम॑म् गच्छति॒ शरी॑रꣳ॒॒ शरी॑रम् गच्छति॒ सोम᳚म् । \newline
5. ग॒च्छ॒ति॒ सोमꣳ॒॒ सोम॑म् गच्छति गच्छति॒ सोमꣳ॒॒ रसो॒ रसः॒ सोम॑म् गच्छति गच्छति॒ सोमꣳ॒॒ रसः॑ । \newline
6. सोमꣳ॒॒ रसो॒ रसः॒ सोमꣳ॒॒ सोमꣳ॒॒ रसो॒ वरु॑णो॒ वरु॑णो॒ रसः॒ सोमꣳ॒॒ सोमꣳ॒॒ रसो॒ वरु॑णः । \newline
7. रसो॒ वरु॑णो॒ वरु॑णो॒ रसो॒ रसो॒ वरु॑ण एन मेनं॒ ॅवरु॑णो॒ रसो॒ रसो॒ वरु॑ण एनम् । \newline
8. वरु॑ण एन मेनं॒ ॅवरु॑णो॒ वरु॑ण एनं ॅवरुणपा॒शेन॑ वरुणपा॒शेनै॑नं॒ ॅवरु॑णो॒ वरु॑ण एनं ॅवरुणपा॒शेन॑ । \newline
9. ए॒नं॒ ॅव॒रु॒ण॒पा॒शेन॑ वरुणपा॒शेनै॑न मेनं ॅवरुणपा॒शेन॑ गृह्णाति गृह्णाति वरुणपा॒शेनै॑न मेनं ॅवरुणपा॒शेन॑ गृह्णाति । \newline
10. व॒रु॒ण॒पा॒शेन॑ गृह्णाति गृह्णाति वरुणपा॒शेन॑ वरुणपा॒शेन॑ गृह्णाति॒ सर॑स्वतीꣳ॒॒ सर॑स्वतीम् गृह्णाति वरुणपा॒शेन॑ वरुणपा॒शेन॑ गृह्णाति॒ सर॑स्वतीम् । \newline
11. व॒रु॒ण॒पा॒शेनेति॑ वरुण - पा॒शेन॑ । \newline
12. गृ॒ह्णा॒ति॒ सर॑स्वतीꣳ॒॒ सर॑स्वतीम् गृह्णाति गृह्णाति॒ सर॑स्वतीं॒ ॅवाग् वाख् सर॑स्वतीम् गृह्णाति गृह्णाति॒ सर॑स्वतीं॒ ॅवाक् । \newline
13. सर॑स्वतीं॒ ॅवाग् वाख् सर॑स्वतीꣳ॒॒ सर॑स्वतीं॒ ॅवाग॒ग्नाविष्णू॑ अ॒ग्नाविष्णू॒ वाख् सर॑स्वतीꣳ॒॒ सर॑स्वतीं॒ ॅवाग॒ग्नाविष्णू᳚ । \newline
14. वाग॒ग्नाविष्णू॑ अ॒ग्नाविष्णू॒ वाग् वाग॒ग्नाविष्णू॑ आ॒त्मा ऽऽत्मा ऽग्नाविष्णू॒ वाग् वाग॒ग्नाविष्णू॑ आ॒त्मा । \newline
15. अ॒ग्नाविष्णू॑ आ॒त्मा ऽऽत्मा ऽग्नाविष्णू॑ अ॒ग्नाविष्णू॑ आ॒त्मा यस्य॒ यस्या॒त्मा ऽग्नाविष्णू॑ अ॒ग्नाविष्णू॑ आ॒त्मा यस्य॑ । \newline
16. अ॒ग्नाविष्णू॒ इत्य॒ग्ना - विष्णू᳚ । \newline
17. आ॒त्मा यस्य॒ यस्या॒त्मा ऽऽत्मा यस्य॒ ज्योग् ज्योग् यस्या॒त्मा ऽऽत्मा यस्य॒ ज्योक् । \newline
18. यस्य॒ ज्योग् ज्योग् यस्य॒ यस्य॒ ज्योगा॒मय॑ त्या॒मय॑ति॒ ज्योग् यस्य॒ यस्य॒ ज्योगा॒मय॑ति । \newline
19. ज्योगा॒मय॑ त्या॒मय॑ति॒ ज्योग् ज्योगा॒मय॑ति॒ यो य आ॒मय॑ति॒ ज्योग् ज्योगा॒मय॑ति॒ यः । \newline
20. आ॒मय॑ति॒ यो य आ॒मय॑ त्या॒मय॑ति॒ यो ज्योगा॑मयावी॒ ज्योगा॑मयावी॒ य आ॒मय॑ त्या॒मय॑ति॒ यो ज्योगा॑मयावी । \newline
21. यो ज्योगा॑मयावी॒ ज्योगा॑मयावी॒ यो यो ज्योगा॑मयावी॒ स्याथ् स्याज् ज्योगा॑मयावी॒ यो यो ज्योगा॑मयावी॒ स्यात् । \newline
22. ज्योगा॑मयावी॒ स्याथ् स्याज् ज्योगा॑मयावी॒ ज्योगा॑मयावी॒ स्याद् यो यः स्याज् ज्योगा॑मयावी॒ ज्योगा॑मयावी॒ स्याद् यः । \newline
23. ज्योगा॑मया॒वीति॒ ज्योक् - आ॒म॒या॒वी॒ । \newline
24. स्याद् यो यः स्याथ् स्याद् यो वा॑ वा॒ यः स्याथ् स्याद् यो वा᳚ । \newline
25. यो वा॑ वा॒ यो यो वा॑ का॒मये॑त का॒मये॑त वा॒ यो यो वा॑ का॒मये॑त । \newline
26. वा॒ का॒मये॑त का॒मये॑त वा वा का॒मये॑त॒ सर्वꣳ॒॒ सर्व॑म् का॒मये॑त वा वा का॒मये॑त॒ सर्व᳚म् । \newline
27. का॒मये॑त॒ सर्वꣳ॒॒ सर्व॑म् का॒मये॑त का॒मये॑त॒ सर्व॒ मायु॒रायुः॒ सर्व॑म् का॒मये॑त का॒मये॑त॒ सर्व॒ मायुः॑ । \newline
28. सर्व॒ मायु॒रायुः॒ सर्वꣳ॒॒ सर्व॒ मायु॑रिया मिया॒ मायुः॒ सर्वꣳ॒॒ सर्व॒ मायु॑रियाम् । \newline
29. आयु॑रिया मिया॒ मायु॒ रायु॑रिया॒ मितीती॑या॒ मायु॒ रायु॑रिया॒ मिति॑ । \newline
30. इ॒या॒ मितीती॑या मिया॒ मिति॒ तस्मै॒ तस्मा॒ इती॑या मिया॒ मिति॒ तस्मै᳚ । \newline
31. इति॒ तस्मै॒ तस्मा॒ इतीति॒ तस्मा॑ ए॒ता मे॒ताम् तस्मा॒ इतीति॒ तस्मा॑ ए॒ताम् । \newline
32. तस्मा॑ ए॒ता मे॒ताम् तस्मै॒ तस्मा॑ ए॒ता मिष्टि॒ मिष्टि॑ मे॒ताम् तस्मै॒ तस्मा॑ ए॒ता मिष्टि᳚म् । \newline
33. ए॒ता मिष्टि॒ मिष्टि॑ मे॒ता मे॒ता मिष्टि॒म् निर् णिरिष्टि॑ मे॒ता मे॒ता मिष्टि॒म् निः । \newline
34. इष्टि॒म् निर् णिरिष्टि॒ मिष्टि॒म् निर् व॑पेद् वपे॒न् निरिष्टि॒ मिष्टि॒म् निर् व॑पेत् । \newline
35. निर् व॑पेद् वपे॒न् निर् णिर् व॑पे दाग्ने॒य मा᳚ग्ने॒यं ॅव॑पे॒न् निर् णिर् व॑पे दाग्ने॒यम् । \newline
36. व॒पे॒ दा॒ग्ने॒य मा᳚ग्ने॒यं ॅव॑पेद् वपे दाग्ने॒य म॒ष्टाक॑पाल म॒ष्टाक॑पाल माग्ने॒यं ॅव॑पेद् वपे दाग्ने॒य म॒ष्टाक॑पालम् । \newline
37. आ॒ग्ने॒य म॒ष्टाक॑पाल म॒ष्टाक॑पाल माग्ने॒य मा᳚ग्ने॒य म॒ष्टाक॑पालꣳ सौ॒म्यꣳ सौ॒म्य म॒ष्टाक॑पाल माग्ने॒य मा᳚ग्ने॒य म॒ष्टाक॑पालꣳ सौ॒म्यम् । \newline
38. अ॒ष्टाक॑पालꣳ सौ॒म्यꣳ सौ॒म्य म॒ष्टाक॑पाल म॒ष्टाक॑पालꣳ सौ॒म्यम् च॒रुम् च॒रुꣳ सौ॒म्य म॒ष्टाक॑पाल म॒ष्टाक॑पालꣳ सौ॒म्यम् च॒रुम् । \newline
39. अ॒ष्टाक॑पाल॒मित्य॒ष्टा - क॒पा॒ल॒म् । \newline
40. सौ॒म्यम् च॒रुम् च॒रुꣳ सौ॒म्यꣳ सौ॒म्यम् च॒रुं ॅवा॑रु॒णं ॅवा॑रु॒णम् च॒रुꣳ सौ॒म्यꣳ सौ॒म्यम् च॒रुं ॅवा॑रु॒णम् । \newline
41. च॒रुं ॅवा॑रु॒णं ॅवा॑रु॒णम् च॒रुम् च॒रुं ॅवा॑रु॒णम् दश॑कपाल॒म् दश॑कपालं ॅवारु॒णम् च॒रुम् च॒रुं ॅवा॑रु॒णम् दश॑कपालम् । \newline
42. वा॒रु॒णम् दश॑कपाल॒म् दश॑कपालं ॅवारु॒णं ॅवा॑रु॒णम् दश॑कपालꣳ सारस्व॒तꣳ सा॑रस्व॒तम् दश॑कपालं ॅवारु॒णं ॅवा॑रु॒णम् दश॑कपालꣳ सारस्व॒तम् । \newline
43. दश॑कपालꣳ सारस्व॒तꣳ सा॑रस्व॒तम् दश॑कपाल॒म् दश॑कपालꣳ सारस्व॒तम् च॒रुम् च॒रुꣳ सा॑रस्व॒तम् दश॑कपाल॒म् दश॑कपालꣳ सारस्व॒तम् च॒रुम् । \newline
44. दश॑कपाल॒मिति॒ दश॑ - क॒पा॒ल॒म् । \newline
45. सा॒र॒स्व॒तम् च॒रुम् च॒रुꣳ सा॑रस्व॒तꣳ सा॑रस्व॒तम् च॒रु मा᳚ग्नावैष्ण॒व मा᳚ग्नावैष्ण॒वम् च॒रुꣳ सा॑रस्व॒तꣳ सा॑रस्व॒तम् च॒रु मा᳚ग्नावैष्ण॒वम् । \newline
46. च॒रु मा᳚ग्नावैष्ण॒व मा᳚ग्नावैष्ण॒वम् च॒रुम् च॒रु मा᳚ग्नावैष्ण॒व मेका॑दशकपाल॒ मेका॑दशकपाल माग्नावैष्ण॒वम् च॒रुम् च॒रु मा᳚ग्नावैष्ण॒व मेका॑दशकपालम् । \newline
47. आ॒ग्ना॒वै॒ष्ण॒व मेका॑दशकपाल॒ मेका॑दशकपाल माग्नावैष्ण॒व मा᳚ग्नावैष्ण॒व मेका॑दशकपाल म॒ग्ने र॒ग्ने रेका॑दशकपाल माग्नावैष्ण॒व मा᳚ग्नावैष्ण॒व मेका॑दशकपाल म॒ग्नेः । \newline
48. आ॒ग्ना॒वै॒ष्ण॒वमित्या᳚ग्ना - वै॒ष्ण॒वम् । \newline
49. एका॑दशकपाल म॒ग्ने र॒ग्ने रेका॑दशकपाल॒ मेका॑दशकपाल म॒ग्ने रे॒वैवाग्ने रेका॑दशकपाल॒ मेका॑दशकपाल म॒ग्नेरे॒व । \newline
50. एका॑दशकपाल॒मित्येका॑दश - क॒पा॒ल॒म् । \newline
51. अ॒ग्ने रे॒वैवाग्ने र॒ग्ने रे॒वास्या᳚ स्यै॒वाग्ने र॒ग्ने रे॒वास्य॑ । \newline
52. ए॒वास्या᳚ स्यै॒वैवास्य॒ शरी॑रꣳ॒॒ शरी॑र मस्यै॒वैवास्य॒ शरी॑रम् । \newline
53. अ॒स्य॒ शरी॑रꣳ॒॒ शरी॑र मस्यास्य॒ शरी॑रम् निष्क्री॒णाति॑ निष्क्री॒णाति॒ शरी॑र मस्यास्य॒ शरी॑रम् निष्क्री॒णाति॑ । \newline
54. शरी॑रम् निष्क्री॒णाति॑ निष्क्री॒णाति॒ शरी॑रꣳ॒॒ शरी॑रम् निष्क्री॒णाति॒ सोमा॒थ् सोमा᳚न् निष्क्री॒णाति॒ शरी॑रꣳ॒॒ शरी॑रम् निष्क्री॒णाति॒ सोमा᳚त् । \newline
55. नि॒ष्क्री॒णाति॒ सोमा॒थ् सोमा᳚न् निष्क्री॒णाति॑ निष्क्री॒णाति॒ सोमा॒द् रसꣳ॒॒ रसꣳ॒॒ सोमा᳚न् निष्क्री॒णाति॑ निष्क्री॒णाति॒ सोमा॒द् रस᳚म् । \newline
56. नि॒ष्क्री॒णातीति॑ निः - क्री॒णाति॑ । \newline
57. सोमा॒द् रसꣳ॒॒ रसꣳ॒॒ सोमा॒थ् सोमा॒द् रसं॑ ॅवारु॒णेन॑ वारु॒णेन॒ रसꣳ॒॒ सोमा॒थ् सोमा॒द् रसं॑ ॅवारु॒णेन॑ । \newline
58. रसं॑ ॅवारु॒णेन॑ वारु॒णेन॒ रसꣳ॒॒ रसं॑ ॅवारु॒णेनै॒वैव वा॑रु॒णेन॒ रसꣳ॒॒ रसं॑ ॅवारु॒णेनै॒व । \newline
\pagebreak
\markright{ TS 2.3.11.2  \hfill https://www.vedavms.in \hfill}

\section{ TS 2.3.11.2 }

\textbf{TS 2.3.11.2 } \newline
\textbf{Samhita Paata} \newline

ॅवारु॒णेनै॒वैनं॑ ॅवरुणपा॒शान् मु॑ञ्चति सारस्व॒तेन॒ वाचं॑ दधात्य॒ग्निः सर्वा॑ दे॒वता॒ विष्णु॑र्य॒ज्ञो दे॒वता॑भिश्चै॒वैनं॑ ॅय॒ज्ञेन॑ च भिषज्यत्यु॒त यदी॒तासु॒ र्भव॑ति॒ जीव॑त्ये॒व यन्नव॒मैत् तन्नव॑नीतम- भव॒दित्याज्य॒- मवे᳚क्षतेरू॒पमे॒वास्यै॒-तन्म॑हि॒मानं॒ ॅव्याच॑ष्टे॒ऽश्विनोः᳚ प्रा॒णो॑ऽसीत्या॑हा॒श्विनौ॒ वै दे॒वानां᳚ - [  ] \newline

\textbf{Pada Paata} \newline

वा॒रु॒णेन॑ । ए॒व । ए॒न॒म् । व॒रु॒ण॒पा॒शादिति॑ वरुण - पा॒शात् । मु॒ञ्च॒ति॒ । सा॒र॒स्व॒तेन॑ । वाच᳚म् । द॒धा॒ति॒ । अ॒ग्निः । सर्वाः᳚ । दे॒वताः᳚ । विष्णुः॑ । य॒ज्ञ्ः । दे॒वता॑भिः । च॒ । ए॒व । ए॒न॒म् । य॒ज्ञेन॑ । च॒ । भि॒ष॒ज्य॒ति॒ । उ॒त । यदि॑ । इ॒तासु॒रिती॒त - अ॒सुः॒ । भव॑ति । जीव॑ति । ए॒व । यत् । नव᳚म् । ऐत् । तत् । नव॑नीत॒मिति॒ नव॑-नी॒त॒म् । अ॒भ॒व॒त् । इति॑ । आज्य᳚म् । अवेति॑ । ई॒क्ष॒ते॒ । रू॒पम् । ए॒व । अ॒स्य॒ । ए॒तत् । म॒हि॒मान᳚म् । व्याच॑ष्ट॒ इति॑ वि-आच॑ष्टे । अ॒श्विनोः᳚ । प्रा॒ण इति॑ प्र - अ॒नः । अ॒सि॒ । इति॑ । आ॒ह॒ । अ॒श्विनौ᳚ । वै । दे॒वाना᳚म् ।  \newline


\textbf{Krama Paata} \newline

वा॒रु॒णेनै॒व । ए॒वैन᳚म् । ए॒नं॒ ॅव॒रु॒ण॒पा॒शात् । व॒रु॒ण॒पा॒शान् मु॑ञ्चति । व॒रु॒ण॒पा॒शादिति॑ वरुण - पा॒शात् । मु॒ञ्च॒ति॒ सा॒र॒स्व॒तेन॑ । सा॒र॒स्व॒तेन॒ वाच᳚म् । वाच॑म् दधाति । द॒धा॒त्य॒ग्निः । अ॒ग्निः सर्वाः᳚ । सर्वा॑ दे॒वताः᳚ । दे॒वता॒ विष्णुः॑ । विष्णु॑र् य॒ज्ञ्ः । य॒ज्ञो दे॒वता॑भिः । दे॒वता॑भिश्च । चै॒व । ए॒वैन᳚म् । ए॒नं॒ ॅय॒ज्ञेन॑ । य॒ज्ञेन॑ च । च॒ भि॒॒ष॒ज्य॒ति॒ । भि॒ष॒ज्य॒त्यु॒त । उ॒त यदि॑ । यदी॒तासुः॑ । इ॒तासु॒र् भव॑ति । इ॒तासु॒रिती॒त - अ॒सुः॒ । भव॑ति॒ जीव॑ति । जीव॑त्ये॒व । ए॒व यत् । यन्नव᳚म् । नव॒मैत् । ऐत् तत् । तन्नव॑नीतम् । नव॑नीत मभवत् । नव॑नीत॒मिति॒ नव॑ - नी॒त॒॒म् । अ॒भ॒व॒दिति॑ । इत्याज्य᳚म् । आज्य॒मव॑ । अवे᳚क्षते । ई॒क्ष॒ते॒ रू॒पम् । रू॒पमे॒व । ए॒वास्य॑ । अ॒स्यै॒तत् । ए॒तन्म॑हि॒मान᳚म् । म॒हि॒मानं॒ ॅव्याच॑ष्टे । व्याच॑ष्टे॒ऽश्विनोः᳚ । व्याच॑ष्ट॒ इति॑ वि - आच॑ष्टे । अ॒श्विनोः᳚ प्रा॒णः । प्रा॒णो॑ऽसि । प्रा॒ण इति॑ प्र - अ॒नः । अ॒सीति॑ । इत्या॑ह । आ॒हा॒श्विनौ᳚ । अ॒श्विनौ॒ वै । वै दे॒वाना᳚म् । दे॒वानां᳚ भि॒षजौ᳚ \newline

\textbf{Jatai Paata} \newline

1. वा॒रु॒णे नै॒वैव वा॑रु॒णेन॑ वारु॒णे नै॒व । \newline
2. ए॒वैन॑ मेन मे॒वैवैन᳚म् । \newline
3. ए॒नं॒ ॅव॒रु॒ण॒पा॒शाद् व॑रुणपा॒शादे॑न मेनं ॅवरुणपा॒शात् । \newline
4. व॒रु॒ण॒पा॒शान् मु॑ञ्चति मुञ्चति वरुणपा॒शाद् व॑रुणपा॒शान् मु॑ञ्चति । \newline
5. व॒रु॒ण॒पा॒शादिति॑ वरुण - पा॒शात् । \newline
6. मु॒ञ्च॒ति॒ सा॒र॒स्व॒तेन॑ सारस्व॒तेन॑ मुञ्चति मुञ्चति सारस्व॒तेन॑ । \newline
7. सा॒र॒स्व॒तेन॒ वाचं॒ ॅवाचꣳ॑ सारस्व॒तेन॑ सारस्व॒तेन॒ वाच᳚म् । \newline
8. वाच॑म् दधाति दधाति॒ वाचं॒ ॅवाच॑म् दधाति । \newline
9. द॒धा॒ त्य॒ग्नि र॒ग्निर् द॑धाति दधा त्य॒ग्निः । \newline
10. अ॒ग्निः सर्वाः॒ सर्वा॑ अ॒ग्नि र॒ग्निः सर्वाः᳚ । \newline
11. सर्वा॑ दे॒वता॑ दे॒वताः॒ सर्वाः॒ सर्वा॑ दे॒वताः᳚ । \newline
12. दे॒वता॒ विष्णु॒र् विष्णु॑र् दे॒वता॑ दे॒वता॒ विष्णुः॑ । \newline
13. विष्णु॑र् य॒ज्ञो य॒ज्ञो विष्णु॒र् विष्णु॑र् य॒ज्ञ्ः । \newline
14. य॒ज्ञो दे॒वता॑भिर् दे॒वता॑भिर् य॒ज्ञो य॒ज्ञो दे॒वता॑भिः । \newline
15. दे॒वता॑भिश्च च दे॒वता॑भिर् दे॒वता॑भिश्च । \newline
16. चै॒वैव च॑ चै॒व । \newline
17. ए॒वैन॑ मेन मे॒वैवैन᳚म् । \newline
18. ए॒नं॒ ॅय॒ज्ञेन॑ य॒ज्ञेनै॑न मेनं ॅय॒ज्ञेन॑ । \newline
19. य॒ज्ञेन॑ च च य॒ज्ञेन॑ य॒ज्ञेन॑ च । \newline
20. च॒ भि॒ष॒ज्य॒ति॒ भि॒ष॒ज्य॒ति॒ च॒ च॒ भि॒ष॒ज्य॒ति॒ । \newline
21. भि॒ष॒ज्य॒ त्यु॒तोत भि॑षज्यति भिषज्य त्यु॒त । \newline
22. उ॒त यदि॒ यद्यु॒तोत यदि॑ । \newline
23. यदी॒तासु॑ रि॒तासु॒र् यदि॒ यदी॒तासुः॑ । \newline
24. इ॒तासु॒र् भव॑ति॒ भव॑ती॒ तासु॑ रि॒तासु॒र् भव॑ति । \newline
25. इ॒तासु॒रिती॒त - अ॒सुः॒ । \newline
26. भव॑ति॒ जीव॑ति॒ जीव॑ति॒ भव॑ति॒ भव॑ति॒ जीव॑ति । \newline
27. जीव॑ त्ये॒वैव जीव॑ति॒ जीव॑ त्ये॒व । \newline
28. ए॒व यद् यदे॒वैव यत् । \newline
29. यन् नव॒म् नवं॒ ॅयद् यन् नव᳚म् । \newline
30. नव॒ मैदैन् नव॒म् नव॒ मैत् । \newline
31. ऐत् तत् तदैदैत् तत् । \newline
32. तन् नव॑नीत॒म् नव॑नीत॒म् तत् तन् नव॑नीतम् । \newline
33. नव॑नीत मभव दभव॒न् नव॑नीत॒म् नव॑नीत मभवत् । \newline
34. नव॑नीत॒मिति॒ नव॑ - नी॒त॒म् । \newline
35. अ॒भ॒व॒ दिती त्य॑भव दभव॒ दिति॑ । \newline
36. इत्याज्य॒ माज्य॒ मिती त्याज्य᳚म् । \newline
37. आज्य॒ मवावाज्य॒ माज्य॒ मव॑ । \newline
38. अवे᳚क्षत ईक्ष॒ते ऽवावे᳚क्षते । \newline
39. ई॒क्ष॒ते॒ रू॒पꣳ रू॒प मी᳚क्षत ईक्षते रू॒पम् । \newline
40. रू॒प मे॒वैव रू॒पꣳ रू॒प मे॒व । \newline
41. ए॒वा स्या᳚स्यै॒ वैवास्य॑ । \newline
42. अ॒स्यै॒त दे॒त द॑स्या स्यै॒तत् । \newline
43. ए॒तन् म॑हि॒मान॑म् महि॒मान॑ मे॒तदे॒तन् म॑हि॒मान᳚म् । \newline
44. म॒हि॒मानं॒ ॅव्याच॑ष्टे॒ व्याच॑ष्टे महि॒मान॑म् महि॒मानं॒ ॅव्याच॑ष्टे । \newline
45. व्याच॑ष्टे॒ ऽश्विनो॑ र॒श्विनो॒र् व्याच॑ष्टे॒ व्याच॑ष्टे॒ ऽश्विनोः᳚ । \newline
46. व्याच॑ष्ट॒ इति॑ वि - आच॑ष्टे । \newline
47. अ॒श्विनोः᳚ प्रा॒णः प्रा॒णो᳚ ऽश्विनो॑ र॒श्विनोः᳚ प्रा॒णः । \newline
48. प्रा॒णो᳚ ऽस्यसि प्रा॒णः प्रा॒णो॑ ऽसि । \newline
49. प्रा॒ण इति॑ प्र - अ॒नः । \newline
50. अ॒सीती त्य॑स्य॒सीति॑ । \newline
51. इत्या॑हा॒हे तीत्या॑ह । \newline
52. आ॒हा॒श्विना॑ व॒श्विना॑ वाहाहा॒ श्विनौ᳚ । \newline
53. अ॒श्विनौ॒ वै वा अ॒श्विना॑ व॒श्विनौ॒ वै । \newline
54. वै दे॒वाना᳚म् दे॒वानां॒ ॅवै वै दे॒वाना᳚म् । \newline
55. दे॒वाना᳚म् भि॒षजौ॑ भि॒षजौ॑ दे॒वाना᳚म् दे॒वाना᳚म् भि॒षजौ᳚ । \newline

\textbf{Ghana Paata } \newline

1. वा॒रु॒णे नै॒वैव वा॑रु॒णेन॑ वारु॒णे नै॒वैन॑ मेन मे॒व वा॑रु॒णेन॑ वारु॒णे नै॒वैन᳚म् । \newline
2. ए॒वैन॑ मेन मे॒वैवैनं॑ ॅवरुणपा॒शाद् व॑रुणपा॒शा दे॑न मे॒वैवैनं॑ ॅवरुणपा॒शात् । \newline
3. ए॒नं॒ ॅव॒रु॒ण॒पा॒शाद् व॑रुणपा॒शा दे॑न मेनं ॅवरुणपा॒शान् मु॑ञ्चति मुञ्चति वरुणपा॒शा दे॑न मेनं ॅवरुणपा॒शान् मु॑ञ्चति । \newline
4. व॒रु॒ण॒पा॒शान् मु॑ञ्चति मुञ्चति वरुणपा॒शाद् व॑रुणपा॒शान् मु॑ञ्चति सारस्व॒तेन॑ सारस्व॒तेन॑ मुञ्चति वरुणपा॒शाद् व॑रुणपा॒शान् मु॑ञ्चति सारस्व॒तेन॑ । \newline
5. व॒रु॒ण॒पा॒शादिति॑ वरुण - पा॒शात् । \newline
6. मु॒ञ्च॒ति॒ सा॒र॒स्व॒तेन॑ सारस्व॒तेन॑ मुञ्चति मुञ्चति सारस्व॒तेन॒ वाचं॒ ॅवाचꣳ॑ सारस्व॒तेन॑ मुञ्चति मुञ्चति सारस्व॒तेन॒ वाच᳚म् । \newline
7. सा॒र॒स्व॒तेन॒ वाचं॒ ॅवाचꣳ॑ सारस्व॒तेन॑ सारस्व॒तेन॒ वाच॑म् दधाति दधाति॒ वाचꣳ॑ सारस्व॒तेन॑ सारस्व॒तेन॒ वाच॑म् दधाति । \newline
8. वाच॑म् दधाति दधाति॒ वाचं॒ ॅवाच॑म् दधा त्य॒ग्नि र॒ग्निर् द॑धाति॒ वाचं॒ ॅवाच॑म् दधा त्य॒ग्निः । \newline
9. द॒धा॒ त्य॒ग्नि र॒ग्निर् द॑धाति दधा त्य॒ग्निः सर्वाः॒ सर्वा॑ अ॒ग्निर् द॑धाति दधा त्य॒ग्निः सर्वाः᳚ । \newline
10. अ॒ग्निः सर्वाः॒ सर्वा॑ अ॒ग्नि र॒ग्निः सर्वा॑ दे॒वता॑ दे॒वताः॒ सर्वा॑ अ॒ग्नि र॒ग्निः सर्वा॑ दे॒वताः᳚ । \newline
11. सर्वा॑ दे॒वता॑ दे॒वताः॒ सर्वाः॒ सर्वा॑ दे॒वता॒ विष्णु॒र् विष्णु॑र् दे॒वताः॒ सर्वाः॒ सर्वा॑ दे॒वता॒ विष्णुः॑ । \newline
12. दे॒वता॒ विष्णु॒र् विष्णु॑र् दे॒वता॑ दे॒वता॒ विष्णु॑र् य॒ज्ञो य॒ज्ञो विष्णु॑र् दे॒वता॑ दे॒वता॒ विष्णु॑र् य॒ज्ञ्ः । \newline
13. विष्णु॑र् य॒ज्ञो य॒ज्ञो विष्णु॒र् विष्णु॑र् य॒ज्ञो दे॒वता॑भिर् दे॒वता॑भिर् य॒ज्ञो विष्णु॒र् विष्णु॑र् य॒ज्ञो दे॒वता॑भिः । \newline
14. य॒ज्ञो दे॒वता॑भिर् दे॒वता॑भिर् य॒ज्ञो य॒ज्ञो दे॒वता॑भिश्च च दे॒वता॑भिर् य॒ज्ञो य॒ज्ञो दे॒वता॑भिश्च । \newline
15. दे॒वता॑भिश्च च दे॒वता॑भिर् दे॒वता॑भि श्चै॒वैव च॑ दे॒वता॑भिर् दे॒वता॑भिश्चै॒व । \newline
16. चै॒वैव च॑ चै॒वैन॑ मेन मे॒व च॑ चै॒वैन᳚म् । \newline
17. ए॒वैन॑ मेन मे॒वैवैनं॑ ॅय॒ज्ञेन॑ य॒ज्ञेनै॑न मे॒वैवैनं॑ ॅय॒ज्ञेन॑ । \newline
18. ए॒नं॒ ॅय॒ज्ञेन॑ य॒ज्ञेनै॑न मेनं ॅय॒ज्ञेन॑ च च य॒ज्ञेनै॑न मेनं ॅय॒ज्ञेन॑ च । \newline
19. य॒ज्ञेन॑ च च य॒ज्ञेन॑ य॒ज्ञेन॑ च भिषज्यति भिषज्यति च य॒ज्ञेन॑ य॒ज्ञेन॑ च भिषज्यति । \newline
20. च॒ भि॒ष॒ज्य॒ति॒ भि॒ष॒ज्य॒ति॒ च॒ च॒ भि॒ष॒ज्य॒ त्यु॒तोत भि॑षज्यति च च भिषज्य त्यु॒त । \newline
21. भि॒ष॒ज्य॒ त्यु॒तोत भि॑षज्यति भिषज्य त्यु॒त यदि॒ यद्यु॒त भि॑षज्यति भिषज्य त्यु॒त यदि॑ । \newline
22. उ॒त यदि॒ यद्यु॒तोत यदी॒तासु॑ रि॒तासु॒र् यद्यु॒तोत यदी॒तासुः॑ । \newline
23. यदी॒तासु॑ रि॒तासु॒र् यदि॒ यदी॒तासु॒र् भव॑ति॒ भव॑ती॒तासु॒र् यदि॒ यदी॒तासु॒र् भव॑ति । \newline
24. इ॒तासु॒र् भव॑ति॒ भव॑ती॒ तासु॑ रि॒तासु॒र् भव॑ति॒ जीव॑ति॒ जीव॑ति॒ भव॑ती॒ तासु॑ रि॒तासु॒र् भव॑ति॒ जीव॑ति । \newline
25. इ॒तासु॒रिती॒त - अ॒सुः॒ । \newline
26. भव॑ति॒ जीव॑ति॒ जीव॑ति॒ भव॑ति॒ भव॑ति॒ जीव॑ त्ये॒वैव जीव॑ति॒ भव॑ति॒ भव॑ति॒ जीव॑ त्ये॒व । \newline
27. जीव॑ त्ये॒वैव जीव॑ति॒ जीव॑ त्ये॒व यद् यदे॒व जीव॑ति॒ जीव॑ त्ये॒व यत् । \newline
28. ए॒व यद् यदे॒वैव यन् नव॒म् नवं॒ ॅयदे॒वैव यन् नव᳚म् । \newline
29. यन् नव॒म् नवं॒ ॅयद् यन् नव॒ मैदैन् नवं॒ ॅयद् यन् नव॒ मैत् । \newline
30. नव॒ मैदैन् नव॒म् नव॒ मैत् तत् तदैन् नव॒म् नव॒ मैत् तत् । \newline
31. ऐत् तत् तदैदैत् तन् नव॑नीत॒म् नव॑नीत॒म् तदैदैत् तन् नव॑नीतम् । \newline
32. तन् नव॑नीत॒म् नव॑नीत॒म् तत् तन् नव॑नीत मभव दभव॒न् नव॑नीत॒म् तत् तन् नव॑नीत मभवत् । \newline
33. नव॑नीत मभव दभव॒न् नव॑नीत॒म् नव॑नीत मभव॒ दिती त्य॑भव॒न् नव॑नीत॒म् नव॑नीत मभव॒दिति॑ । \newline
34. नव॑नीत॒मिति॒ नव॑ - नी॒त॒म् । \newline
35. अ॒भ॒व॒ दिती त्य॑भव दभव॒ दित्याज्य॒ माज्य॒ मित्य॑भव दभव॒ दित्याज्य᳚म् । \newline
36. इत्याज्य॒ माज्य॒ मितीत्याज्य॒ मवावाज्य॒ मितीत्याज्य॒ मव॑ । \newline
37. आज्य॒ मवावाज्य॒ माज्य॒ मवे᳚क्षत ईक्ष॒ते ऽवाज्य॒ माज्य॒ मवे᳚क्षते । \newline
38. अवे᳚क्षत ईक्ष॒ते ऽवावे᳚क्षते रू॒पꣳ रू॒प मी᳚क्ष॒ते ऽवावे᳚क्षते रू॒पम् । \newline
39. ई॒क्ष॒ते॒ रू॒पꣳ रू॒प मी᳚क्षत ईक्षते रू॒प मे॒वैव रू॒प मी᳚क्षत ईक्षते रू॒प मे॒व । \newline
40. रू॒प मे॒वैव रू॒पꣳ रू॒प मे॒वास्या᳚स्यै॒व रू॒पꣳ रू॒प मे॒वास्य॑ । \newline
41. ए॒वास्या᳚ स्यै॒वैवा स्यै॒त दे॒त द॑स्यै॒ वैवास्यै॒तत् । \newline
42. अ॒स्यै॒ तदे॒त द॑स्या स्यै॒तन् म॑हि॒मान॑म् महि॒मान॑ मे॒त द॑स्या स्यै॒तन् म॑हि॒मान᳚म् । \newline
43. ए॒तन् म॑हि॒मान॑म् महि॒मान॑ मे॒त दे॒तन् म॑हि॒मानं॒ ॅव्याच॑ष्टे॒ व्याच॑ष्टे महि॒मान॑ मे॒त दे॒तन् म॑हि॒मानं॒ ॅव्याच॑ष्टे । \newline
44. म॒हि॒मानं॒ ॅव्याच॑ष्टे॒ व्याच॑ष्टे महि॒मान॑म् महि॒मानं॒ ॅव्याच॑ष्टे॒ ऽश्विनो॑ र॒श्विनो॒र् व्याच॑ष्टे महि॒मान॑म् महि॒मानं॒ ॅव्याच॑ष्टे॒ ऽश्विनोः᳚ । \newline
45. व्याच॑ष्टे॒ ऽश्विनो॑ र॒श्विनो॒र् व्याच॑ष्टे॒ व्याच॑ष्टे॒ ऽश्विनोः᳚ प्रा॒णः प्रा॒णो᳚ ऽश्विनो॒र् व्याच॑ष्टे॒ व्याच॑ष्टे॒ ऽश्विनोः᳚ प्रा॒णः । \newline
46. व्याच॑ष्ट॒ इति॑ वि - आच॑ष्टे । \newline
47. अ॒श्विनोः᳚ प्रा॒णः प्रा॒णो᳚ ऽश्विनो॑ र॒श्विनोः᳚ प्रा॒णो᳚ ऽस्यसि प्रा॒णो᳚ ऽश्विनो॑ र॒श्विनोः᳚ प्रा॒णो॑ ऽसि । \newline
48. प्रा॒णो᳚ ऽस्यसि प्रा॒णः प्रा॒णो॑ ऽसीती त्य॑सि प्रा॒णः प्रा॒णो॑ ऽसीति॑ । \newline
49. प्रा॒ण इति॑ प्र - अ॒नः । \newline
50. अ॒सीती त्य॑स्य॒सी त्या॑हा॒हे त्य॑स्य॒सी त्या॑ह । \newline
51. इत्या॑हा॒हे तीत्या॑हा॒ श्विना॑ व॒श्विना॑ वा॒हे तीत्या॑हा॒श्विनौ᳚ । \newline
52. आ॒हा॒श्विना॑ व॒श्विना॑ वाहाहा॒ श्विनौ॒ वै वा अ॒श्विना॑ वाहाहा॒ श्विनौ॒ वै । \newline
53. अ॒श्विनौ॒ वै वा अ॒श्विना॑ व॒श्विनौ॒ वै दे॒वाना᳚म् दे॒वानां॒ ॅवा अ॒श्विना॑ व॒श्विनौ॒ वै दे॒वाना᳚म् । \newline
54. वै दे॒वाना᳚म् दे॒वानां॒ ॅवै वै दे॒वाना᳚म् भि॒षजौ॑ भि॒षजौ॑ दे॒वानां॒ ॅवै वै दे॒वाना᳚म् भि॒षजौ᳚ । \newline
55. दे॒वाना᳚म् भि॒षजौ॑ भि॒षजौ॑ दे॒वाना᳚म् दे॒वाना᳚म् भि॒षजौ॒ ताभ्या॒म् ताभ्या᳚म् भि॒षजौ॑ दे॒वाना᳚म् दे॒वाना᳚म् भि॒षजौ॒ ताभ्या᳚म् । \newline
\pagebreak
\markright{ TS 2.3.11.3  \hfill https://www.vedavms.in \hfill}

\section{ TS 2.3.11.3 }

\textbf{TS 2.3.11.3 } \newline
\textbf{Samhita Paata} \newline

भि॒षजौ॒ ताभ्या॑मे॒वास्मै॑ भेष॒जं क॑रो॒तीन्द्र॑स्य प्रा॒णो॑ ऽसीत्या॑हेन्द्रि॒य- मे॒वास्मि॑न्ने॒तेन॑ दधाति मि॒त्रावरु॑णयोः प्रा॒णो॑ऽसीत्या॑ह प्राणापा॒नावे॒- वास्मि॑न्ने॒तेन॑ दधाति॒ विश्वे॑षां दे॒वानां᳚ प्रा॒णो॑ऽसीत्या॑ह वी॒र्य॑मे॒वास्मि॑न्ने॒तेन॑ दधाति घृ॒तस्य॒ धारा॑म॒मृत॑स्य॒ पन्था॒मित्या॑ह यथाय॒जुरे॒वैतत् पा॑वमा॒नेन॑ त्वा॒ स्तोमे॒नेत्या॑ - [  ] \newline

\textbf{Pada Paata} \newline

भि॒षजौ᳚ । ताभ्या᳚म् । ए॒व । अ॒स्मै॒ । भे॒ष॒जम् । क॒रो॒ति॒ । इन्द्र॑स्य । प्रा॒ण इति॑ प्र - अ॒नः । अ॒सि॒ । इति॑ । आ॒ह॒ । इ॒न्द्रि॒यम् । ए॒व । अ॒स्मि॒न्न् । ए॒तेन॑ । द॒धा॒ति॒ । मि॒त्रावरु॑णया॒रिति॑ मि॒त्रा - वरु॑णयोः । प्रा॒ण इति॑ प्र - अ॒नः । अ॒सि॒ । इति॑ । आ॒ह॒ । प्रा॒णा॒पा॒नाविति॑ प्राण - अ॒पा॒नौ । ए॒व । अ॒स्मि॒न्न् । ए॒तेन॑ । द॒धा॒ति॒ । विश्वे॑षाम् । दे॒वाना᳚म् । प्रा॒ण इति॑ प्र - अ॒नः । अ॒सि॒ । इति॑ । आ॒ह॒ । वी॒र्य᳚म् । ए॒व । अ॒स्मि॒न्न् । ए॒तेन॑ । द॒धा॒ति॒ । घृ॒तस्य॑ । धारा᳚म् । अ॒मृत॑स्य । पन्था᳚म् । इति॑ । आ॒ह॒ । य॒था॒य॒जुरिति॑ यथा - य॒जुः । ए॒व । ए॒तत् । पा॒व॒मा॒नेन॑ । त्वा॒ । स्तोमे॑न । इति॑ ।  \newline


\textbf{Krama Paata} \newline

भि॒षजौ॒ ताभ्या᳚म् । ताभ्या॑मे॒व । ए॒वास्मै᳚ । अ॒स्मै॒ भे॒ष॒जम् । भे॒ष॒जम् क॑रोति । क॒रो॒तीन्द्र॑स्य । इन्द्र॑स्य प्रा॒णः । प्रा॒णो॑ऽसि । प्रा॒ण इति॑ प्र - अ॒नः । अ॒सीति॑ । इत्या॑ह । आ॒हे॒न्द्रि॒यम् । इ॒न्द्रि॒य मे॒व । ए॒वास्मिन्न्॑ । अ॒स्मि॒न्ने॒तेन॑ । ए॒तेन॑ दधाति । द॒धा॒ति॒ मि॒त्रावरु॑णयोः । मि॒त्रावरु॑णयोः प्रा॒णः । मि॒त्रावरु॑णयो॒रिति॑ मि॒त्रा - वरु॑णयोः । प्रा॒णो॑ऽसि । प्रा॒ण इति॑ प्र - अ॒नः । अ॒सीति॑ । इत्या॑ह । आ॒ह॒ प्रा॒णा॒पा॒नौ । प्रा॒णा॒पा॒नावे॒व । प्रा॒णा॒पा॒नाविति॑ प्राण - अ॒पा॒नौ । ए॒वास्मिन्न्॑ । अ॒स्मि॒न्ने॒तेन॑ । ए॒तेन॑ दधाति । द॒धा॒ति॒ विश्वे॑षाम् । विश्वे॑षाम् दे॒वाना᳚म् । दे॒वाना᳚म् प्रा॒णः । प्रा॒णो॑ऽसि । प्रा॒ण इति॑ प्र - अ॒नः । अ॒सीति॑ । इत्या॑ह । आ॒ह॒ वी॒र्य᳚म् । वी॒र्य॑मे॒व । ए॒वास्मिन्न्॑ । अ॒स्मि॒न्ने॒तेन॑ । ए॒तेन॑ दधाति । द॒धा॒ति॒ घृ॒तस्य॑ । घृ॒तस्य॒ धारा᳚म् । धारा॑म॒मृत॑स्य । अ॒मृत॑स्य॒ पन्था᳚म् । पन्था॒मिति॑ । इत्या॑ह । आ॒ह॒ य॒था॒य॒जुः । य॒था॒य॒जुरे॒व । य॒था॒य॒जुरिति॑ यथा - य॒जुः । ए॒वैतत् । ए॒तत् पा॑वमा॒नेन॑ । पा॒व॒मा॒नेन॑ त्वा । त्वा॒ स्तोमे॑न । स्तोमे॒नेति॑ । इत्या॑ह \newline

\textbf{Jatai Paata} \newline

1. भि॒षजौ॒ ताभ्या॒म् ताभ्या᳚म् भि॒षजौ॑ भि॒षजौ॒ ताभ्या᳚म् । \newline
2. ताभ्या॑ मे॒वैव ताभ्या॒म् ताभ्या॑ मे॒व । \newline
3. ए॒वास्मा॑ अस्मा ए॒वैवास्मै᳚ । \newline
4. अ॒स्मै॒ भे॒ष॒जम् भे॑ष॒ज म॑स्मा अस्मै भेष॒जम् । \newline
5. भे॒ष॒जम् क॑रोति करोति भेष॒जम् भे॑ष॒जम् क॑रोति । \newline
6. क॒रो॒ती न्द्र॒स्ये न्द्र॑स्य करोति करो॒ती न्द्र॑स्य । \newline
7. इन्द्र॑स्य प्रा॒णः प्रा॒ण इन्द्र॒स्ये न्द्र॑स्य प्रा॒णः । \newline
8. प्रा॒णो᳚ ऽस्यसि प्रा॒णः प्रा॒णो॑ ऽसि । \newline
9. प्रा॒ण इति॑ प्र - अ॒नः । \newline
10. अ॒सीती त्य॑स्य॒सीति॑ । \newline
11. इत्या॑हा॒हे तीत्या॑ह । \newline
12. आ॒हे॒ न्द्रि॒य मि॑न्द्रि॒य मा॑हाहे न्द्रि॒यम् । \newline
13. इ॒न्द्रि॒य मे॒वैवे न्द्रि॒य मि॑न्द्रि॒य मे॒व । \newline
14. ए॒वास्मि॑न् नस्मिन् ने॒वैवास्मिन्न्॑ । \newline
15. अ॒स्मि॒न् ने॒ते नै॒तेना᳚स्मिन् नस्मिन् ने॒तेन॑ । \newline
16. ए॒तेन॑ दधाति दधा त्ये॒ते नै॒तेन॑ दधाति । \newline
17. द॒धा॒ति॒ मि॒त्रावरु॑णयोर् मि॒त्रावरु॑णयोर् दधाति दधाति मि॒त्रावरु॑णयोः । \newline
18. मि॒त्रावरु॑णयोः प्रा॒णः प्रा॒णो मि॒त्रावरु॑णयोर् मि॒त्रावरु॑णयोः प्रा॒णः । \newline
19. मि॒त्रावरु॑णयो॒रिति॑ मि॒त्रा - वरु॑णयोः । \newline
20. प्रा॒णो᳚ ऽस्यसि प्रा॒णः प्रा॒णो॑ ऽसि । \newline
21. प्रा॒ण इति॑ प्र - अ॒नः । \newline
22. अ॒सीती त्य॑स्य॒सीति॑ । \newline
23. इत्या॑हा॒हे तीत्या॑ह । \newline
24. आ॒ह॒ प्रा॒णा॒पा॒नौ प्रा॑णापा॒ना वा॑हाह प्राणापा॒नौ । \newline
25. प्रा॒णा॒पा॒ना वे॒वैव प्रा॑णापा॒नौ प्रा॑णापा॒ना वे॒व । \newline
26. प्रा॒णा॒पा॒नाविति॑ प्राण - अ॒पा॒नौ । \newline
27. ए॒वास्मि॑न् नस्मिन् ने॒वैवास्मिन्न्॑ । \newline
28. अ॒स्मि॒न् ने॒ते नै॒तेना᳚स्मिन् नस्मिन् ने॒तेन॑ । \newline
29. ए॒तेन॑ दधाति दधा त्ये॒ते नै॒तेन॑ दधाति । \newline
30. द॒धा॒ति॒ विश्वे॑षां॒ ॅविश्वे॑षाम् दधाति दधाति॒ विश्वे॑षाम् । \newline
31. विश्वे॑षाम् दे॒वाना᳚म् दे॒वानां॒ ॅविश्वे॑षां॒ ॅविश्वे॑षाम् दे॒वाना᳚म् । \newline
32. दे॒वाना᳚म् प्रा॒णः प्रा॒णो दे॒वाना᳚म् दे॒वाना᳚म् प्रा॒णः । \newline
33. प्रा॒णो᳚ ऽस्यसि प्रा॒णः प्रा॒णो॑ ऽसि । \newline
34. प्रा॒ण इति॑ प्र - अ॒नः । \newline
35. अ॒सीती त्य॑स्य॒सीति॑ । \newline
36. इत्या॑हा॒हे तीत्या॑ह । \newline
37. आ॒ह॒ वी॒र्यं॑ ॅवी॒र्य॑ माहाह वी॒र्य᳚म् । \newline
38. वी॒र्य॑ मे॒वैव वी॒र्यं॑ ॅवी॒र्य॑ मे॒व । \newline
39. ए॒वास्मि॑न् नस्मिन् ने॒वैवास्मिन्न्॑ । \newline
40. अ॒स्मि॒न् ने॒ते नै॒तेना᳚स्मिन् नस्मिन् ने॒तेन॑ । \newline
41. ए॒तेन॑ दधाति दधा त्ये॒तेनै॒तेन॑ दधाति । \newline
42. द॒धा॒ति॒ घृ॒तस्य॑ घृ॒तस्य॑ दधाति दधाति घृ॒तस्य॑ । \newline
43. घृ॒तस्य॒ धारा॒म् धारा᳚म् घृ॒तस्य॑ घृ॒तस्य॒ धारा᳚म् । \newline
44. धारा॑ म॒मृत॑स्या॒ मृत॑स्य॒ धारा॒म् धारा॑ म॒मृत॑स्य । \newline
45. अ॒मृत॑स्य॒ पन्था॒म् पन्था॑ म॒मृत॑स्या॒ मृत॑स्य॒ पन्था᳚म् । \newline
46. पन्था॒ मितीति॒ पन्था॒म् पन्था॒ मिति॑ । \newline
47. इत्या॑हा॒हे तीत्या॑ह । \newline
48. आ॒ह॒ य॒था॒य॒जुर् य॑थाय॒जु रा॑हाह यथाय॒जुः । \newline
49. य॒था॒य॒जु रे॒वैव य॑थाय॒जुर् य॑थाय॒जु रे॒व । \newline
50. य॒था॒य॒जुरिति॑ यथा - य॒जुः । \newline
51. ए॒वैत दे॒त दे॒वैवैतत् । \newline
52. ए॒तत् पा॑वमा॒नेन॑ पावमा॒ने नै॒तदे॒तत् पा॑वमा॒नेन॑ । \newline
53. पा॒व॒मा॒नेन॑ त्वा त्वा पावमा॒नेन॑ पावमा॒नेन॑ त्वा । \newline
54. त्वा॒ स्तोमे॑न॒ स्तोमे॑न त्वा त्वा॒ स्तोमे॑न । \newline
55. स्तोमे॒ने तीति॒ स्तोमे॑न॒ स्तोमे॒ने ति॑ । \newline
56. इत्या॑हा॒हे तीत्या॑ह । \newline

\textbf{Ghana Paata } \newline

1. भि॒षजौ॒ ताभ्या॒म् ताभ्या᳚म् भि॒षजौ॑ भि॒षजौ॒ ताभ्या॑ मे॒वैव ताभ्या᳚म् भि॒षजौ॑ भि॒षजौ॒ ताभ्या॑ मे॒व । \newline
2. ताभ्या॑ मे॒वैव ताभ्या॒म् ताभ्या॑ मे॒वास्मा॑ अस्मा ए॒व ताभ्या॒म् ताभ्या॑ मे॒वास्मै᳚ । \newline
3. ए॒वास्मा॑ अस्मा ए॒वैवास्मै॑ भेष॒जम् भे॑ष॒ज म॑स्मा ए॒वैवास्मै॑ भेष॒जम् । \newline
4. अ॒स्मै॒ भे॒ष॒जम् भे॑ष॒ज म॑स्मा अस्मै भेष॒जम् क॑रोति करोति भेष॒ज म॑स्मा अस्मै भेष॒जम् क॑रोति । \newline
5. भे॒ष॒जम् क॑रोति करोति भेष॒जम् भे॑ष॒जम् क॑रो॒तीन्द्र॒स्ये न्द्र॑स्य करोति भेष॒जम् भे॑ष॒जम् क॑रो॒तीन्द्र॑स्य । \newline
6. क॒रो॒तीन्द्र॒स्ये न्द्र॑स्य करोति करो॒तीन्द्र॑स्य प्रा॒णः प्रा॒ण इन्द्र॑स्य करोति करो॒तीन्द्र॑स्य प्रा॒णः । \newline
7. इन्द्र॑स्य प्रा॒णः प्रा॒ण इन्द्र॒स्ये न्द्र॑स्य प्रा॒णो᳚ ऽस्यसि प्रा॒ण इन्द्र॒स्ये न्द्र॑स्य प्रा॒णो॑ ऽसि । \newline
8. प्रा॒णो᳚ ऽस्यसि प्रा॒णः प्रा॒णो॑ ऽसीती त्य॑सि प्रा॒णः प्रा॒णो॑ ऽसीति॑ । \newline
9. प्रा॒ण इति॑ प्र - अ॒नः । \newline
10. अ॒सीती त्य॑स्य॒सी त्या॑हा॒हे त्य॑स्य॒सीत्या॑ह । \newline
11. इत्या॑हा॒हे तीत्या॑हे न्द्रि॒य मि॑न्द्रि॒य मा॒हे तीत्या॑हे न्द्रि॒यम् । \newline
12. आ॒हे॒ न्द्रि॒य मि॑न्द्रि॒य मा॑हाहे न्द्रि॒य मे॒वैवे न्द्रि॒य मा॑हाहे न्द्रि॒य मे॒व । \newline
13. इ॒न्द्रि॒य मे॒वैवे न्द्रि॒य मि॑न्द्रि॒य मे॒वास्मि॑न् नस्मिन् ने॒वे न्द्रि॒य मि॑न्द्रि॒य मे॒वास्मिन्न्॑ । \newline
14. ए॒वास्मि॑न् नस्मिन् ने॒वैवास्मि॑न् ने॒तेनै॒तेना᳚स्मिन् ने॒वैवास्मि॑न् ने॒तेन॑ । \newline
15. अ॒स्मि॒न् ने॒तेनै॒तेना᳚स्मिन् नस्मिन् ने॒तेन॑ दधाति दधा त्ये॒तेना᳚स्मिन् नस्मिन् ने॒तेन॑ दधाति । \newline
16. ए॒तेन॑ दधाति दधा त्ये॒तेनै॒तेन॑ दधाति मि॒त्रावरु॑णयोर् मि॒त्रावरु॑णयोर् दधा त्ये॒तेनै॒तेन॑ दधाति मि॒त्रावरु॑णयोः । \newline
17. द॒धा॒ति॒ मि॒त्रावरु॑णयोर् मि॒त्रावरु॑णयोर् दधाति दधाति मि॒त्रावरु॑णयोः प्रा॒णः प्रा॒णो मि॒त्रावरु॑णयोर् दधाति दधाति मि॒त्रावरु॑णयोः प्रा॒णः । \newline
18. मि॒त्रावरु॑णयोः प्रा॒णः प्रा॒णो मि॒त्रावरु॑णयोर् मि॒त्रावरु॑णयोः प्रा॒णो᳚ ऽस्यसि प्रा॒णो मि॒त्रावरु॑णयोर् मि॒त्रावरु॑णयोः प्रा॒णो॑ ऽसि । \newline
19. मि॒त्रावरु॑णयो॒रिति॑ मि॒त्रा - वरु॑णयोः । \newline
20. प्रा॒णो᳚ ऽस्यसि प्रा॒णः प्रा॒णो॑ ऽसीती त्य॑सि प्रा॒णः प्रा॒णो॑ ऽसीति॑ । \newline
21. प्रा॒ण इति॑ प्र - अ॒नः । \newline
22. अ॒सीती त्य॑स्य॒सी त्या॑हा॒हे त्य॑स्य॒सीत्या॑ह । \newline
23. इत्या॑हा॒हे तीत्या॑ह प्राणापा॒नौ प्रा॑णापा॒ना वा॒हे तीत्या॑ह प्राणापा॒नौ । \newline
24. आ॒ह॒ प्रा॒णा॒पा॒नौ प्रा॑णापा॒ना वा॑हाह प्राणापा॒ना वे॒वैव प्रा॑णापा॒ना वा॑हाह प्राणापा॒ना वे॒व । \newline
25. प्रा॒णा॒पा॒ना वे॒वैव प्रा॑णापा॒नौ प्रा॑णापा॒ना वे॒वास्मि॑न् नस्मिन् ने॒व प्रा॑णापा॒नौ प्रा॑णापा॒ना वे॒वास्मिन्न्॑ । \newline
26. प्रा॒णा॒पा॒नाविति॑ प्राण - अ॒पा॒नौ । \newline
27. ए॒वास्मि॑न् नस्मिन् ने॒वैवास्मि॑न् ने॒तेनै॒तेना᳚स्मिन् ने॒वैवास्मि॑न् ने॒तेन॑ । \newline
28. अ॒स्मि॒न् ने॒तेनै॒तेना᳚स्मिन् नस्मिन् ने॒तेन॑ दधाति दधा त्ये॒तेना᳚स्मिन् नस्मिन् ने॒तेन॑ दधाति । \newline
29. ए॒तेन॑ दधाति दधा त्ये॒तेनै॒तेन॑ दधाति॒ विश्वे॑षां॒ ॅविश्वे॑षाम् दधा त्ये॒तेनै॒तेन॑ दधाति॒ विश्वे॑षाम् । \newline
30. द॒धा॒ति॒ विश्वे॑षां॒ ॅविश्वे॑षाम् दधाति दधाति॒ विश्वे॑षाम् दे॒वाना᳚म् दे॒वानां॒ ॅविश्वे॑षाम् दधाति दधाति॒ विश्वे॑षाम् दे॒वाना᳚म् । \newline
31. विश्वे॑षाम् दे॒वाना᳚म् दे॒वानां॒ ॅविश्वे॑षां॒ ॅविश्वे॑षाम् दे॒वाना᳚म् प्रा॒णः प्रा॒णो दे॒वानां॒ ॅविश्वे॑षां॒ ॅविश्वे॑षाम् दे॒वाना᳚म् प्रा॒णः । \newline
32. दे॒वाना᳚म् प्रा॒णः प्रा॒णो दे॒वाना᳚म् दे॒वाना᳚म् प्रा॒णो᳚ ऽस्यसि प्रा॒णो दे॒वाना᳚म् दे॒वाना᳚म् प्रा॒णो॑ ऽसि । \newline
33. प्रा॒णो᳚ ऽस्यसि प्रा॒णः प्रा॒णो॑ ऽसीतीत्य॑सि प्रा॒णः प्रा॒णो॑ ऽसीति॑ । \newline
34. प्रा॒ण इति॑ प्र - अ॒नः । \newline
35. अ॒सीती त्य॑स्य॒सी त्या॑हा॒हे त्य॑स्य॒सीत्या॑ह । \newline
36. इत्या॑हा॒हे तीत्या॑ह वी॒र्यं॑ ॅवी॒र्य॑ मा॒हे तीत्या॑ह वी॒र्य᳚म् । \newline
37. आ॒ह॒ वी॒र्यं॑ ॅवी॒र्य॑ माहाह वी॒र्य॑ मे॒वैव वी॒र्य॑ माहाह वी॒र्य॑ मे॒व । \newline
38. वी॒र्य॑ मे॒वैव वी॒र्यं॑ ॅवी॒र्य॑ मे॒वास्मि॑न् नस्मिन् ने॒व वी॒र्यं॑ ॅवी॒र्य॑ मे॒वास्मिन्न्॑ । \newline
39. ए॒वास्मि॑न् नस्मिन् ने॒वैवास्मि॑न् ने॒तेनै॒तेना᳚स्मिन् ने॒वैवास्मि॑न् ने॒तेन॑ । \newline
40. अ॒स्मि॒न् ने॒तेनै॒तेना᳚स्मिन् नस्मिन् ने॒तेन॑ दधाति दधा त्ये॒तेना᳚स्मिन् नस्मिन् ने॒तेन॑ दधाति । \newline
41. ए॒तेन॑ दधाति दधा त्ये॒तेनै॒तेन॑ दधाति घृ॒तस्य॑ घृ॒तस्य॑ दधा त्ये॒तेनै॒तेन॑ दधाति घृ॒तस्य॑ । \newline
42. द॒धा॒ति॒ घृ॒तस्य॑ घृ॒तस्य॑ दधाति दधाति घृ॒तस्य॒ धारा॒म् धारा᳚म् घृ॒तस्य॑ दधाति दधाति घृ॒तस्य॒ धारा᳚म् । \newline
43. घृ॒तस्य॒ धारा॒म् धारा᳚म् घृ॒तस्य॑ घृ॒तस्य॒ धारा॑ म॒मृत॑स्या॒ मृत॑स्य॒ धारा᳚म् घृ॒तस्य॑ घृ॒तस्य॒ धारा॑ म॒मृत॑स्य । \newline
44. धारा॑ म॒मृत॑स्या॒ मृत॑स्य॒ धारा॒म् धारा॑ म॒मृत॑स्य॒ पन्था॒म् पन्था॑ म॒मृत॑स्य॒ धारा॒म् धारा॑ म॒मृत॑स्य॒ पन्था᳚म् । \newline
45. अ॒मृत॑स्य॒ पन्था॒म् पन्था॑ म॒मृत॑स्या॒ मृत॑स्य॒ पन्था॒ मितीति॒ पन्था॑ म॒मृत॑स्या॒ मृत॑स्य॒ पन्था॒ मिति॑ । \newline
46. पन्था॒ मितीति॒ पन्था॒म् पन्था॒ मित्या॑हा॒हे ति॒ पन्था॒म् पन्था॒ मित्या॑ह । \newline
47. इत्या॑हा॒हे तीत्या॑ह यथाय॒जुर् य॑थाय॒जुरा॒हे तीत्या॑ह यथाय॒जुः । \newline
48. आ॒ह॒ य॒था॒य॒जुर् य॑थाय॒जु रा॑हाह यथाय॒जु रे॒वैव य॑थाय॒जु रा॑हाह यथाय॒जु रे॒व । \newline
49. य॒था॒य॒जु रे॒वैव य॑थाय॒जुर् य॑थाय॒जु रे॒वैत दे॒तदे॒व य॑थाय॒जुर् य॑थाय॒जु रे॒वैतत् । \newline
50. य॒था॒य॒जुरिति॑ यथा - य॒जुः । \newline
51. ए॒वैत दे॒त दे॒वैवैतत् पा॑वमा॒नेन॑ पावमा॒नेनै॒त दे॒वैवैतत् पा॑वमा॒नेन॑ । \newline
52. ए॒तत् पा॑वमा॒नेन॑ पावमा॒नेनै॒त दे॒तत् पा॑वमा॒नेन॑ त्वा त्वा पावमा॒नेनै॒त दे॒तत् पा॑वमा॒नेन॑ त्वा । \newline
53. पा॒व॒मा॒नेन॑ त्वा त्वा पावमा॒नेन॑ पावमा॒नेन॑ त्वा॒ स्तोमे॑न॒ स्तोमे॑न त्वा पावमा॒नेन॑ पावमा॒नेन॑ त्वा॒ स्तोमे॑न । \newline
54. त्वा॒ स्तोमे॑न॒ स्तोमे॑न त्वा त्वा॒ स्तोमे॒ने तीति॒ स्तोमे॑न त्वा त्वा॒ स्तोमे॒ने ति॑ । \newline
55. स्तोमे॒ने तीति॒ स्तोमे॑न॒ स्तोमे॒ने त्या॑हा॒हे ति॒ स्तोमे॑न॒ स्तोमे॒ने त्या॑ह । \newline
56. इत्या॑हा॒हे तीत्या॑ह प्रा॒णम् प्रा॒ण मा॒हे तीत्या॑ह प्रा॒णम् । \newline
\pagebreak
\markright{ TS 2.3.11.4  \hfill https://www.vedavms.in \hfill}

\section{ TS 2.3.11.4 }

\textbf{TS 2.3.11.4 } \newline
\textbf{Samhita Paata} \newline

-ह प्रा॒णमे॒वास्मि॑न्ने॒तेन॑ दधाति बृहद्-रथन्त॒रयो᳚स्त्वा॒ स्तोमे॒नेत्या॒हौज॑ ए॒वास्मि॑न्ने॒तेन॑ दधात्य॒ग्नेस्त्वा॒ मात्र॒येत्या॑हा॒-ऽऽत्मान॑-मे॒वास्मि॑न्ने॒तेन॑ दधात्यृ॒त्विजः॒ पर्या॑हु॒र्याव॑न्त ए॒वर्त्विज॒स्त ए॑नं भिषज्यन्ति ब्र॒ह्मणो॒ हस्त॑मन्वा॒रभ्य॒ पर्या॑हुरेक॒धैव यज॑मान॒ आयु॑र्दधति॒ यदे॒व तस्य॒ तद्धिर॑ण्याद् - [  ] \newline

\textbf{Pada Paata} \newline

आ॒ह॒ । प्रा॒णमिति॑ प्र - अ॒नम् । ए॒व । अ॒स्मि॒न्न् । ए॒तेन॑ । द॒धा॒ति॒ । बृ॒ह॒द्र॒थ॒न्त॒रयो॒रिति॑ बृहत्-र॒थ॒न्त॒रयोः᳚ । त्वा॒ । स्तोमे॑न । इति॑ । आ॒ह॒ । ओजः॑ । ए॒व । अ॒स्मि॒न्न् । ए॒तेन॑ । द॒धा॒ति॒ । अ॒ग्नेः । त्वा॒ । मात्र॑या । इति॑ । आ॒ह॒ । आ॒त्मान᳚म् । ए॒व । अ॒स्मि॒न्न् । ए॒तेन॑ । द॒धा॒ति॒ । ऋ॒त्विजः॑ । परीति॑ । आ॒हुः॒ । याव॑न्तः । ए॒व । ऋ॒त्विजः॑ । ते । ए॒न॒म् । भि॒ष॒ज्य॒न्ति॒ । ब्र॒ह्मणः॑ । हस्त᳚म् । अ॒न्वा॒रभ्येत्य॑नु - आ॒रभ्य॑ । परीति॑ । आ॒हुः॒ । ए॒क॒धेत्ये॑क - धा । ए॒व । यज॑माने । आयुः॑ । द॒ध॒ति॒ । यत् । ए॒व । तस्य॑ । तत् । हिर॑ण्यात् ।  \newline


\textbf{Krama Paata} \newline

आ॒ह॒ प्रा॒णम् । प्रा॒णमे॒व । प्रा॒ण मिति॑ प्र - अ॒नम् । ए॒वास्मिन्न्॑ । अ॒स्मि॒न्ने॒तेन॑ । ए॒तेन॑ दधाति । द॒धा॒ति॒ बृ॒ह॒द्र॒थ॒न्त॒रयोः᳚ । बृ॒ह॒द्र॒थ॒न्त॒रयो᳚स्त्वा । बृ॒ह॒द्र॒थ॒न्त॒रयो॒रिति॑ बृहत् - र॒थ॒न्त॒रयोः᳚ । त्वा॒ स्तोमे॑न । स्तोमे॒नेति॑ । इत्या॑ह । आ॒हौजः॑ । ओज॑ ए॒व । ए॒वास्मिन्न्॑ । अ॒स्मि॒न्ने॒तेन॑ । ए॒तेन॑ दधाति । द॒धा॒त्य॒ग्नेः । अ॒ग्नेस्त्वा᳚ । त्वा॒ मात्र॑या । मात्र॒येति॑ । इत्या॑ह । आ॒हा॒त्मान᳚म् । आ॒त्मान॑मे॒व । ए॒वास्मिन्न्॑ । अ॒स्मि॒न्ने॒तेन॑ । ए॒तेन॑ दधाति । द॒धा॒त्यृ॒त्विजः॑ । ऋ॒त्विजः॒ परि॑ । पर्या॑हुः । आ॒हु॒र् याव॑न्तः । याव॑न्त ए॒व । ए॒वर्त्विजः॑ । ऋ॒त्विज॒स्ते । त ए॑नम् । ए॒न॒म् भि॒ष॒ज्य॒न्ति॒ । भि॒ष॒ज्य॒न्ति॒ ब्र॒ह्मणः॑ । ब्र॒ह्मणो॒ हस्त᳚म् । हस्त॑मन्वा॒रभ्य॑ । अ॒न्वा॒रभ्य॒ परि॑ । अ॒न्वा॒रभ्येत्य॑नु - आ॒रभ्य॑ । पर्या॑हुः । आ॒हु॒रे॒क॒धा । ए॒क॒धैव । ए॒क॒धेत्ये॑क - धा । ए॒व यज॑माने । यज॑मान॒ आयुः॑ । आयु॑र् दधति । द॒ध॒ति॒ यत् । यदे॒व । ए॒व तस्य॑ । तस्य॒ तत् । तद्धिर॑ण्यात् । हिर॑ण्याद् घृ॒तम् \newline

\textbf{Jatai Paata} \newline

1. आ॒ह॒ प्रा॒णम् प्रा॒ण मा॑हाह प्रा॒णम् । \newline
2. प्रा॒ण मे॒वैव प्रा॒णम् प्रा॒ण मे॒व । \newline
3. प्रा॒णमिति॑ प्र - अ॒नम् । \newline
4. ए॒वास्मि॑न् नस्मिन् ने॒वैवास्मिन्न्॑ । \newline
5. अ॒स्मि॒न् ने॒ते नै॒तेना᳚स्मिन् नस्मिन् ने॒तेन॑ । \newline
6. ए॒तेन॑ दधाति दधा त्ये॒तेनै॒तेन॑ दधाति । \newline
7. द॒धा॒ति॒ बृ॒ह॒द्र॒थ॒न्त॒रयो᳚र् बृहद्रथन्त॒रयो᳚र् दधाति दधाति बृहद्रथन्त॒रयोः᳚ । \newline
8. बृ॒ह॒द्र॒थ॒न्त॒रयो᳚ स्त्वा त्वा बृहद्रथन्त॒रयो᳚र् बृहद्रथन्त॒रयो᳚ स्त्वा । \newline
9. बृ॒ह॒द्र॒थ॒न्त॒रयो॒रिति॑ बृहत् - र॒थ॒न्त॒रयोः᳚ । \newline
10. त्वा॒ स्तोमे॑न॒ स्तोमे॑न त्वा त्वा॒ स्तोमे॑न । \newline
11. स्तोमे॒ने तीति॒ स्तोमे॑न॒ स्तोमे॒ने ति॑ । \newline
12. इत्या॑हा॒हे तीत्या॑ह । \newline
13. आ॒हौज॒ ओज॑ आहा॒हौजः॑ । \newline
14. ओज॑ ए॒वैवौज॒ ओज॑ ए॒व । \newline
15. ए॒वास्मि॑न् नस्मिन् ने॒वैवास्मिन्न्॑ । \newline
16. अ॒स्मि॒न् ने॒ते नै॒तेना᳚स्मिन् नस्मिन् ने॒तेन॑ । \newline
17. ए॒तेन॑ दधाति दधा त्ये॒तेनै॒तेन॑ दधाति । \newline
18. द॒धा॒ त्य॒ग्ने र॒ग्नेर् द॑धाति दधा त्य॒ग्नेः । \newline
19. अ॒ग्ने स्त्वा᳚ त्वा॒ ऽग्ने र॒ग्ने स्त्वा᳚ । \newline
20. त्वा॒ मात्र॑या॒ मात्र॑या त्वा त्वा॒ मात्र॑या । \newline
21. मात्र॒येतीति॒ मात्र॑या॒ मात्र॒येति॑ । \newline
22. इत्या॑हा॒हे तीत्या॑ह । \newline
23. आ॒हा॒त्मान॑ मा॒त्मान॑ माहाहा॒त्मान᳚म् । \newline
24. आ॒त्मान॑ मे॒वैवात्मान॑ मा॒त्मान॑ मे॒व । \newline
25. ए॒वास्मि॑न् नस्मिन् ने॒वैवास्मिन्न्॑ । \newline
26. अ॒स्मि॒न् ने॒ते नै॒तेना᳚स्मिन् नस्मिन् ने॒तेन॑ । \newline
27. ए॒तेन॑ दधाति दधा त्ये॒तेनै॒तेन॑ दधाति । \newline
28. द॒धा॒ त्यृ॒त्विज॑ ऋ॒त्विजो॑ दधाति दधा त्यृ॒त्विजः॑ । \newline
29. ऋ॒त्विजः॒ परि॒ पर्यृ॒त्विज॑ ऋ॒त्विजः॒ परि॑ । \newline
30. पर्या॑हु राहुः॒ परि॒ पर्या॑हुः । \newline
31. आ॒हु॒र् याव॑न्तो॒ याव॑न्त आहु राहु॒र् याव॑न्तः । \newline
32. याव॑न्त ए॒वैव याव॑न्तो॒ याव॑न्त ए॒व । \newline
33. ए॒व र्त्विज॑ ऋ॒त्विज॑ ए॒वैव र्त्विजः॑ । \newline
34. ऋ॒त्विज॒ स्ते त ऋ॒त्विज॑ ऋ॒त्विज॒ स्ते । \newline
35. त ए॑न मेन॒म् ते त ए॑नम् । \newline
36. ए॒न॒म् भि॒ष॒ज्य॒न्ति॒ भि॒ष॒ज्य॒ न्त्ये॒न॒ मे॒न॒म् भि॒ष॒ज्य॒न्ति॒ । \newline
37. भि॒ष॒ज्य॒न्ति॒ ब्र॒ह्मणो᳚ ब्र॒ह्मणो॑ भिषज्यन्ति भिषज्यन्ति ब्र॒ह्मणः॑ । \newline
38. ब्र॒ह्मणो॒ हस्तꣳ॒॒ हस्त॑म् ब्र॒ह्मणो᳚ ब्र॒ह्मणो॒ हस्त᳚म् । \newline
39. हस्त॑ मन्वा॒रभ्या᳚ न्वा॒रभ्य॒ हस्तꣳ॒॒ हस्त॑ मन्वा॒रभ्य॑ । \newline
40. अ॒न्वा॒रभ्य॒ परि॒ पर्य॑न्वा॒रभ्या᳚ न्वा॒रभ्य॒ परि॑ । \newline
41. अ॒न्वा॒रभ्येत्य॑नु - आ॒रभ्य॑ । \newline
42. पर्या॑हु राहुः॒ परि॒ पर्या॑हुः । \newline
43. आ॒हु॒ रे॒क॒ धैक॒धा ऽऽहु॑ राहु रेक॒धा । \newline
44. ए॒क॒ धैवैवैक॒ धैक॒धैव । \newline
45. ए॒क॒धेत्ये॑क - धा । \newline
46. ए॒व यज॑माने॒ यज॑मान ए॒वैव यज॑माने । \newline
47. यज॑मान॒ आयु॒ रायु॒र् यज॑माने॒ यज॑मान॒ आयुः॑ । \newline
48. आयु॑र् दधति दध॒ त्यायु॒ रायु॑र् दधति । \newline
49. द॒ध॒ति॒ यद् यद् द॑धति दधति॒ यत् । \newline
50. यदे॒वैव यद् यदे॒व । \newline
51. ए॒व तस्य॒ तस्यै॒वैव तस्य॑ । \newline
52. तस्य॒ तत् तत् तस्य॒ तस्य॒ तत् । \newline
53. तद्धिर॑ण्या॒ द्धिर॑ण्या॒त् तत् त द्धिर॑ण्यात् । \newline
54. हिर॑ण्याद् घृ॒तम् घृ॒तꣳ हिर॑ण्या॒ द्धिर॑ण्याद् घृ॒तम् । \newline

\textbf{Ghana Paata } \newline

1. आ॒ह॒ प्रा॒णम् प्रा॒ण मा॑हाह प्रा॒ण मे॒वैव प्रा॒ण मा॑हाह प्रा॒ण मे॒व । \newline
2. प्रा॒ण मे॒वैव प्रा॒णम् प्रा॒ण मे॒वास्मि॑न् नस्मिन् ने॒व प्रा॒णम् प्रा॒ण मे॒वास्मिन्न्॑ । \newline
3. प्रा॒णमिति॑ प्र - अ॒नम् । \newline
4. ए॒वास्मि॑न् नस्मिन् ने॒वैवास्मि॑न् ने॒तेनै॒तेना᳚स्मिन् ने॒वैवास्मि॑न् ने॒तेन॑ । \newline
5. अ॒स्मि॒न् ने॒तेनै॒तेना᳚स्मिन् नस्मिन् ने॒तेन॑ दधाति दधा त्ये॒तेना᳚स्मिन् नस्मिन् ने॒तेन॑ दधाति । \newline
6. ए॒तेन॑ दधाति दधा त्ये॒तेनै॒तेन॑ दधाति बृहद्रथन्त॒रयो᳚र् बृहद्रथन्त॒रयो᳚र् दधा त्ये॒तेनै॒तेन॑ दधाति बृहद्रथन्त॒रयोः᳚ । \newline
7. द॒धा॒ति॒ बृ॒ह॒द्र॒थ॒न्त॒रयो᳚र् बृहद्रथन्त॒रयो᳚र् दधाति दधाति बृहद्रथन्त॒रयो᳚ स्त्वा त्वा बृहद्रथन्त॒रयो᳚र् दधाति दधाति बृहद्रथन्त॒रयो᳚ स्त्वा । \newline
8. बृ॒ह॒द्र॒थ॒न्त॒रयो᳚ स्त्वा त्वा बृहद्रथन्त॒रयो᳚र् बृहद्रथन्त॒रयो᳚ स्त्वा॒ स्तोमे॑न॒ स्तोमे॑न त्वा बृहद्रथन्त॒रयो᳚र् बृहद्रथन्त॒रयो᳚ स्त्वा॒ स्तोमे॑न । \newline
9. बृ॒ह॒द्र॒थ॒न्त॒रयो॒रिति॑ बृहत् - र॒थ॒न्त॒रयोः᳚ । \newline
10. त्वा॒ स्तोमे॑न॒ स्तोमे॑न त्वा त्वा॒ स्तोमे॒ने तीति॒ स्तोमे॑न त्वा त्वा॒ स्तोमे॒ने ति॑ । \newline
11. स्तोमे॒ने तीति॒ स्तोमे॑न॒ स्तोमे॒ने त्या॑हा॒हे ति॒ स्तोमे॑न॒ स्तोमे॒ने त्या॑ह । \newline
12. इत्या॑हा॒हे ती त्या॒हौज॒ ओज॑ आ॒हे ती त्या॒हौजः॑ । \newline
13. आ॒हौज॒ ओज॑ आहा॒हौज॑ ए॒वैवौज॑ आहा॒हौज॑ ए॒व । \newline
14. ओज॑ ए॒वैवौज॒ ओज॑ ए॒वास्मि॑न् नस्मिन् ने॒वौज॒ ओज॑ ए॒वास्मिन्न्॑ । \newline
15. ए॒वास्मि॑न् नस्मिन् ने॒वैवास्मि॑न् ने॒तेनै॒तेना᳚स्मिन् ने॒वैवास्मि॑न् ने॒तेन॑ । \newline
16. अ॒स्मि॒न् ने॒तेनै॒तेना᳚स्मिन् नस्मिन् ने॒तेन॑ दधाति दधा त्ये॒तेना᳚स्मिन् नस्मिन् ने॒तेन॑ दधाति । \newline
17. ए॒तेन॑ दधाति दधा त्ये॒तेनै॒तेन॑ दधा त्य॒ग्ने र॒ग्नेर् द॑धा त्ये॒तेनै॒तेन॑ दधा त्य॒ग्नेः । \newline
18. द॒धा॒ त्य॒ग्ने र॒ग्नेर् द॑धाति दधा त्य॒ग्ने स्त्वा᳚ त्वा॒ ऽग्नेर् द॑धाति दधा त्य॒ग्ने स्त्वा᳚ । \newline
19. अ॒ग्ने स्त्वा᳚ त्वा॒ ऽग्ने र॒ग्ने स्त्वा॒ मात्र॑या॒ मात्र॑या त्वा॒ ऽग्ने र॒ग्ने स्त्वा॒ मात्र॑या । \newline
20. त्वा॒ मात्र॑या॒ मात्र॑या त्वा त्वा॒ मात्र॒येतीति॒ मात्र॑या त्वा त्वा॒ मात्र॒येति॑ । \newline
21. मात्र॒येतीति॒ मात्र॑या॒ मात्र॒ये त्या॑हा॒हे ति॒ मात्र॑या॒ मात्र॒ये त्या॑ह । \newline
22. इत्या॑हा॒हे तीत्या॑हा॒त्मान॑ मा॒त्मान॑ मा॒हे तीत्या॑हा॒त्मान᳚म् । \newline
23. आ॒हा॒त्मान॑ मा॒त्मान॑ माहाहा॒त्मान॑ मे॒वैवात्मान॑ माहाहा॒त्मान॑ मे॒व । \newline
24. आ॒त्मान॑ मे॒वैवात्मान॑ मा॒त्मान॑ मे॒वास्मि॑न् नस्मिन् ने॒वात्मान॑ मा॒त्मान॑ मे॒वास्मिन्न्॑ । \newline
25. ए॒वास्मि॑न् नस्मिन् ने॒वैवास्मि॑न् ने॒तेनै॒तेना᳚स्मिन् ने॒वैवास्मि॑न् ने॒तेन॑ । \newline
26. अ॒स्मि॒न् ने॒तेनै॒तेना᳚स्मिन् नस्मिन् ने॒तेन॑ दधाति दधा त्ये॒तेना᳚स्मिन् नस्मिन् ने॒तेन॑ दधाति । \newline
27. ए॒तेन॑ दधाति दधा त्ये॒तेनै॒तेन॑ दधा त्यृ॒त्विज॑ ऋ॒त्विजो॑ दधा त्ये॒तेनै॒तेन॑ दधा त्यृ॒त्विजः॑ । \newline
28. द॒धा॒ त्यृ॒त्विज॑ ऋ॒त्विजो॑ दधाति दधा त्यृ॒त्विजः॒ परि॒ पर्यृ॒त्विजो॑ दधाति दधा त्यृ॒त्विजः॒ परि॑ । \newline
29. ऋ॒त्विजः॒ परि॒ पर्यृ॒त्विज॑ ऋ॒त्विजः॒ पर्या॑हु राहुः॒ पर्यृ॒त्विज॑ ऋ॒त्विजः॒ पर्या॑हुः । \newline
30. पर्या॑हु राहुः॒ परि॒ पर्या॑हु॒र् याव॑न्तो॒ याव॑न्त आहुः॒ परि॒ पर्या॑हु॒र् याव॑न्तः । \newline
31. आ॒हु॒र् याव॑न्तो॒ याव॑न्त आहु राहु॒र् याव॑न्त ए॒वैव याव॑न्त आहु राहु॒र् याव॑न्त ए॒व । \newline
32. याव॑न्त ए॒वैव याव॑न्तो॒ याव॑न्त ए॒व र्‌त्विज॑ ऋ॒त्विज॑ ए॒व याव॑न्तो॒ याव॑न्त ए॒व र्‌त्विजः॑ । \newline
33. ए॒व र्‌त्विज॑ ऋ॒त्विज॑ ए॒वैव र्‌त्विज॒ स्ते त ऋ॒त्विज॑ ए॒वैव र्‌त्विज॒ स्ते । \newline
34. ऋ॒त्विज॒ स्ते त ऋ॒त्विज॑ ऋ॒त्विज॒ स्त ए॑न मेन॒म् त ऋ॒त्विज॑ ऋ॒त्विज॒ स्त ए॑नम् । \newline
35. त ए॑न मेन॒म् ते त ए॑नम् भिषज्यन्ति भिषज्य न्त्येन॒म् ते त ए॑नम् भिषज्यन्ति । \newline
36. ए॒न॒म् भि॒ष॒ज्य॒न्ति॒ भि॒ष॒ज्य॒ न्त्ये॒न॒ मे॒न॒म् भि॒ष॒ज्य॒न्ति॒ ब्र॒ह्मणो᳚ ब्र॒ह्मणो॑ भिषज्य न्त्येन मेनम् भिषज्यन्ति ब्र॒ह्मणः॑ । \newline
37. भि॒ष॒ज्य॒न्ति॒ ब्र॒ह्मणो᳚ ब्र॒ह्मणो॑ भिषज्यन्ति भिषज्यन्ति ब्र॒ह्मणो॒ हस्तꣳ॒॒ हस्त॑म् ब्र॒ह्मणो॑ भिषज्यन्ति भिषज्यन्ति ब्र॒ह्मणो॒ हस्त᳚म् । \newline
38. ब्र॒ह्मणो॒ हस्तꣳ॒॒ हस्त॑म् ब्र॒ह्मणो᳚ ब्र॒ह्मणो॒ हस्त॑ मन्वा॒रभ्या᳚ न्वा॒रभ्य॒ हस्त॑म् ब्र॒ह्मणो᳚ ब्र॒ह्मणो॒ हस्त॑ मन्वा॒रभ्य॑ । \newline
39. हस्त॑ मन्वा॒रभ्या᳚ न्वा॒रभ्य॒ हस्तꣳ॒॒ हस्त॑ मन्वा॒रभ्य॒ परि॒ पर्य॑न्वा॒रभ्य॒ हस्तꣳ॒॒ हस्त॑ मन्वा॒रभ्य॒ परि॑ । \newline
40. अ॒न्वा॒रभ्य॒ परि॒ पर्य॑ न्वा॒रभ्या᳚ न्वा॒रभ्य॒ पर्या॑हुराहुः॒ पर्य॑न्वा॒रभ्या᳚ न्वा॒रभ्य॒ पर्या॑हुः । \newline
41. अ॒न्वा॒रभ्येत्य॑नु - आ॒रभ्य॑ । \newline
42. पर्या॑हु राहुः॒ परि॒ पर्या॑हु रेक॒धैक॒धा ऽऽहुः॒ परि॒ पर्या॑हु रेक॒धा । \newline
43. आ॒हु॒ रे॒क॒धैक॒धा ऽऽहु॑ राहु रेक॒धैवैवैक॒धा ऽऽहु॑ राहु रेक॒धैव । \newline
44. ए॒क॒धै वैवै क॒धैक॒धैव यज॑माने॒ यज॑मान ए॒वैक॒धैक॒धैव यज॑माने । \newline
45. ए॒क॒धेत्ये॑क - धा । \newline
46. ए॒व यज॑माने॒ यज॑मान ए॒वैव यज॑मान॒ आयु॒ रायु॒र् यज॑मान ए॒वैव यज॑मान॒ आयुः॑ । \newline
47. यज॑मान॒ आयु॒ रायु॒र् यज॑माने॒ यज॑मान॒ आयु॑र् दधति दध॒ त्यायु॒र् यज॑माने॒ यज॑मान॒ आयु॑र् दधति । \newline
48. आयु॑र् दधति दध॒ त्यायु॒ रायु॑र् दधति॒ यद् यद् द॑ध॒ त्यायु॒ रायु॑र् दधति॒ यत् । \newline
49. द॒ध॒ति॒ यद् यद् द॑धति दधति॒ यदे॒वैव यद् द॑धति दधति॒ यदे॒व । \newline
50. यदे॒वैव यद् यदे॒व तस्य॒ तस्यै॒व यद् यदे॒व तस्य॑ । \newline
51. ए॒व तस्य॒ तस्यै॒वैव तस्य॒ तत् तत् तस्यै॒वैव तस्य॒ तत् । \newline
52. तस्य॒ तत् तत् तस्य॒ तस्य॒ त द्धिर॑ण्या॒ द्धिर॑ण्या॒त् तत् तस्य॒ तस्य॒ तद्धिर॑ण्यात् । \newline
53. तद्धिर॑ण्या॒ द्धिर॑ण्या॒त् तत् तद्धिर॑ण्याद् घृ॒तम् घृ॒तꣳ हिर॑ण्या॒त् तत् तद्धिर॑ण्याद् घृ॒तम् । \newline
54. हिर॑ण्याद् घृ॒तम् घृ॒तꣳ हिर॑ण्या॒ द्धिर॑ण्याद् घृ॒तम् निर् णिर् घृ॒तꣳ हिर॑ण्या॒ द्धिर॑ण्याद् घृ॒तम् निः । \newline
\pagebreak
\markright{ TS 2.3.11.5  \hfill https://www.vedavms.in \hfill}

\section{ TS 2.3.11.5 }

\textbf{TS 2.3.11.5 } \newline
\textbf{Samhita Paata} \newline

घृ॒तं निष्पि॑ब॒त्यायु॒र्वै घृ॒तम॒मृतꣳ॒॒ हिर॑ण्यम॒मृता॑दे॒वा ऽऽयु॒र्निष्पि॑बति श॒तमा॑नं भवति श॒तायुः॒ पुरु॑षः श॒तेन्द्रि॑य॒ आयु॑ष्ये॒वेन्द्रि॒ये प्रति॑तिष्ठ॒त्यथो॒ खलु॒ याव॑तीः॒ समा॑ ए॒ष्यन् मन्ये॑त॒ ताव॑न्मानꣳ स्या॒थ् समृ॑द्ध्या इ॒मम॑ग्न॒ आयु॑षे॒ वर्च॑से कृ॒धीत्या॒हा ऽऽयु॑रे॒वास्मि॒न्. वर्चो॑ दधाति॒ विश्वे॑ देवा॒ जर॑दष्टि॒र्यथा ऽस॒दित्या॑ ( ) -ह॒ जर॑दष्टिमे॒वैनं॑ करोत्य॒ग्नि-रायु॑ष्मा॒निति॒ हस्तं॑ गृह्णात्ये॒ते वै दे॒वा आयु॑ष्मन्त॒स्त ए॒वास्मि॒न्नायु॑र्दधति॒ सर्व॒मायु॑रेति ॥ \newline

\textbf{Pada Paata} \newline

घृ॒तम् । निरिति॑ । पि॒ब॒ति॒ । आयुः॑ । वै । घृ॒तम् । अ॒मृत᳚म् । हिर॑ण्यम् । अ॒मृता᳚त् । ए॒व । आयुः॑ । निरिति॑ । पि॒ब॒ति॒ । श॒तमा॑न॒मिति॑ श॒त - मा॒न॒म् । भ॒व॒ति॒ । श॒तायु॒रिति॑ श॒त - आ॒युः॒ । पुरु॑षः । श॒तेन्द्रि॑य॒ इति॑ श॒त - इ॒न्द्रि॒यः॒ । आयु॑षि । ए॒व । इ॒न्द्रि॒ये । प्रतीति॑ । ति॒ष्ठ॒ति॒ । अथो॒ इति॑ । खलु॑ । याव॑तीः । समाः᳚ । ए॒ष्यन्न् । मन्ये॑त । ताव॑न्मान॒मिति॒ ताव॑त् - मा॒न॒म् । स्या॒त् । समृ॑द्ध्या॒ इति॒ सम् - ऋ॒द्ध्यै॒ । इ॒मम् । अ॒ग्ने॒ । आयु॑षे । वर्च॑से । कृ॒धि॒ । इति॑ । आ॒ह॒ । आयुः॑ । ए॒व । अ॒स्मि॒न्न् । वर्चः॑ । द॒धा॒ति॒ । विश्वे᳚ । दे॒वाः॒ । जर॑दष्टि॒रिति॒ जर॑त् - अ॒ष्टिः॒ । यथा᳚ । अस॑त् । इति॑ ( ) । आ॒ह॒ । जर॑दष्टि॒मिति॒ जर॑त् - अ॒ष्टि॒म् । ए॒व । ए॒न॒म् । क॒रो॒ति॒ । अ॒ग्निः । आयु॑ष्मान् । इति॑ । हस्त᳚म् । गृ॒ह्णा॒ति॒ । ए॒ते । वै । दे॒वाः । आयु॑ष्मन्तः । ते । ए॒व । अ॒स्मि॒न्न् । आयुः॑ । द॒ध॒ति॒ । सर्व᳚म् । आयुः॑ । ए॒ति॒ ॥  \newline


\textbf{Krama Paata} \newline

घृ॒तम् निः । निष्पि॑बति । पि॒ब॒त्यायुः॑ । आयु॒र् वै । वै घृ॒तम् । घृ॒तम॒मृत᳚म् । अ॒मृतꣳ॒॒ हिर॑ण्यम् । हिर॑ण्यम॒मृता᳚त् । अ॒मृता॑दे॒व । ए॒वायुः॑ । आयु॒र् निः । निष्पि॑बति । पि॒ब॒ति॒ श॒तमा॑नम् । श॒तमा॑नम् भवति । श॒तमा॑न॒मिति॑ श॒त - मा॒न॒म् । भ॒व॒ति॒ श॒तायुः॑ । श॒तायुः॒ पुरु॑षः । श॒तायु॒रिति॑ श॒त - आ॒युः॒ । पुरु॑षः श॒तेन्द्रि॑यः । श॒तेन्द्रि॑य॒ आयु॑षि । श॒तेन्द्रि॑य॒ इति॑ श॒त - इ॒न्द्रि॒यः॒ । आयु॑ष्ये॒व । ए॒वेन्द्रि॒ये । इ॒न्द्रि॒ये प्रति॑ । प्रति॑ तिष्ठति । ति॒ष्ठ॒त्यथो᳚ । अथो॒ खलु॑ । अथो॒ इत्यथो᳚ । खलु॒ याव॑तीः । याव॑तीः॒ समाः᳚ । समा॑ ए॒ष्यन्न् । ए॒ष्यन् मन्ये॑त । मन्ये॑त॒ ताव॑न्मानम् । ताव॑न्मानꣳ स्यात् । ताव॑न्मान॒मिति॒ ताव॑त् - मा॒न॒म् । स्या॒थ् समृ॑द्ध्यै । समृ॑द्ध्या इ॒मम् । समृ॑द्ध्या॒ इति॒ सं - ऋ॒द्ध्यै॒ । इ॒मम॑ग्ने । अ॒ग्न॒ आयु॑षे । आयु॑षे॒ वर्च॑से । वर्च॑से कृधि । कृ॒धीति॑ । इत्या॑ह । आ॒हायुः॑ । आयु॑रे॒व । ए॒वास्मिन्न्॑ । अ॒स्मि॒न् वर्चः॑ । वर्चो॑ दधाति । द॒धा॒ति॒ विश्वे᳚ । विश्वे॑ देवाः । दे॒वा॒ जर॑दष्टिः । जर॑दष्टि॒र् यथा᳚ । जर॑दष्टि॒रिति॒ जर॑त् - अ॒ष्टिः॒ । यथा ऽस॑त् । अस॒दिति॑ ( ) । इत्या॑ह । आ॒ह॒ जर॑दष्टिम् । जर॑दष्टिमे॒व । जर॑दष्टि॒मिति॒ जर॑त् - अ॒ष्टि॒म् । ए॒वैन᳚म् । ए॒न॒म् क॒रो॒ति॒ । क॒रो॒त्य॒ग्निः । अ॒ग्निरायु॑ष्मान् । आयु॑ष्मा॒निति॑ । इति॒ हस्त᳚म् । हस्त॑म् गृह्णाति । गृ॒ह्णा॒त्ये॒ते । ए॒ते वै । वै दे॒वाः । दे॒वा आयु॑ष्मन्तः । आयु॑ष्मन्त॒स्ते । त ए॒व । ए॒वास्मिन्न्॑ । अ॒स्मि॒न्नायुः॑ । आयु॑र् दधति । द॒ध॒ति॒ सर्व᳚म् । सर्व॒मायुः॑ । आयु॑रेति । ए॒तीत्ये॑ति । \newline

\textbf{Jatai Paata} \newline

1. घृ॒तम् निर् णिर् घृ॒तम् घृ॒तम् निः । \newline
2. निष् पि॑बति पिबति॒ निर् णिष् पि॑बति । \newline
3. पि॒ब॒ त्यायु॒ रायुः॑ पिबति पिब॒ त्यायुः॑ । \newline
4. आयु॒र् वै वा आयु॒ रायु॒र् वै । \newline
5. वै घृ॒तम् घृ॒तं ॅवै वै घृ॒तम् । \newline
6. घृ॒त म॒मृत॑ म॒मृत॑म् घृ॒तम् घृ॒त म॒मृत᳚म् । \newline
7. अ॒मृतꣳ॒॒ हिर॑ण्यꣳ॒॒ हिर॑ण्य म॒मृत॑ म॒मृतꣳ॒॒ हिर॑ण्यम् । \newline
8. हिर॑ण्य म॒मृता॑ द॒मृता॒ द्धिर॑ण्यꣳ॒॒ हिर॑ण्य म॒मृता᳚त् । \newline
9. अ॒मृता॑ दे॒वै वामृता॑ द॒मृता॑ दे॒व । \newline
10. ए॒वायु॒ रायु॑ रे॒वैवायुः॑ । \newline
11. आयु॒र् निर् णि रायु॒ रायु॒र् निः । \newline
12. निष् पि॑बति पिबति॒ निर् णिष् पि॑बति । \newline
13. पि॒ब॒ति॒ श॒तमा॑नꣳ श॒तमा॑नम् पिबति पिबति श॒तमा॑नम् । \newline
14. श॒तमा॑नम् भवति भवति श॒तमा॑नꣳ श॒तमा॑नम् भवति । \newline
15. श॒तमा॑न॒मिति॑ श॒त - मा॒न॒म् । \newline
16. भ॒व॒ति॒ श॒तायुः॑ श॒तायु॑र् भवति भवति श॒तायुः॑ । \newline
17. श॒तायुः॒ पुरु॑षः॒ पुरु॑षः श॒तायुः॑ श॒तायुः॒ पुरु॑षः । \newline
18. श॒तायु॒रिति॑ श॒त - आ॒युः॒ । \newline
19. पुरु॑षः श॒तेन्द्रि॑यः श॒तेन्द्रि॑यः॒ पुरु॑षः॒ पुरु॑षः श॒तेन्द्रि॑यः । \newline
20. श॒तेन्द्रि॑य॒ आयु॒ष्यायु॑षि श॒तेन्द्रि॑यः श॒तेन्द्रि॑य॒ आयु॑षि । \newline
21. श॒तेन्द्रि॑य॒ इति॑ श॒त - इ॒न्द्रि॒यः॒ । \newline
22. आयु॑ ष्ये॒वैवायु॒ ष्यायु॑ ष्ये॒व । \newline
23. ए॒वे न्द्रि॒य इ॑न्द्रि॒य ए॒वैवे न्द्रि॒ये । \newline
24. इ॒न्द्रि॒ये प्रति॒ प्रती᳚न्द्रि॒य इ॑न्द्रि॒ये प्रति॑ । \newline
25. प्रति॑ तिष्ठति तिष्ठति॒ प्रति॒ प्रति॑ तिष्ठति । \newline
26. ति॒ष्ठ॒ त्यथो॒ अथो॑ तिष्ठति तिष्ठ॒ त्यथो᳚ । \newline
27. अथो॒ खलु॒ खल्वथो॒ अथो॒ खलु॑ । \newline
28. अथो॒ इत्यथो᳚ । \newline
29. खलु॒ याव॑ती॒र् याव॑तीः॒ खलु॒ खलु॒ याव॑तीः । \newline
30. याव॑तीः॒ समाः॒ समा॒ याव॑ती॒र् याव॑तीः॒ समाः᳚ । \newline
31. समा॑ ए॒ष्यन् ने॒ष्यन् थ्समाः॒ समा॑ ए॒ष्यन्न् । \newline
32. ए॒ष्यन् मन्ये॑त॒ मन्ये॑तै॒ष्यन् ने॒ष्यन् मन्ये॑त । \newline
33. मन्ये॑त॒ ताव॑न्मान॒म् ताव॑न्मान॒म् मन्ये॑त॒ मन्ये॑त॒ ताव॑न्मानम् । \newline
34. ताव॑न्मानꣳ स्याथ् स्या॒त् ताव॑न्मान॒म् ताव॑न्मानꣳ स्यात् । \newline
35. ताव॑न्मान॒मिति॒ ताव॑त् - मा॒न॒म् । \newline
36. स्या॒थ् समृ॑द्ध्यै॒ समृ॑द्ध्यै स्याथ् स्या॒थ् समृ॑द्ध्यै । \newline
37. समृ॑द्ध्या इ॒म मि॒मꣳ समृ॑द्ध्यै॒ समृ॑द्ध्या इ॒मम् । \newline
38. समृ॑द्ध्या॒ इति॒ सम् - ऋ॒द्ध्यै॒ । \newline
39. इ॒म म॑ग्ने ऽग्न इ॒म मि॒म म॑ग्ने । \newline
40. अ॒ग्न॒ आयु॑ष॒ आयु॑षे ऽग्ने ऽग्न॒ आयु॑षे । \newline
41. आयु॑षे॒ वर्च॑से॒ वर्च॑स॒ आयु॑ष॒ आयु॑षे॒ वर्च॑से । \newline
42. वर्च॑से कृधि कृधि॒ वर्च॑से॒ वर्च॑से कृधि । \newline
43. कृ॒धीतीति॑ कृधि कृ॒धीति॑ । \newline
44. इत्या॑हा॒हे तीत्या॑ह । \newline
45. आ॒हायु॒ रायु॑ राहा॒ हायुः॑ । \newline
46. आयु॑ रे॒वै वायु॒ रायु॑रे॒व । \newline
47. ए॒वास्मि॑न् नस्मिन् ने॒वैवास्मिन्न्॑ । \newline
48. अ॒स्मि॒न्॒. वर्चो॒ वर्चो᳚ ऽस्मिन् नस्मि॒न्॒. वर्चः॑ । \newline
49. वर्चो॑ दधाति दधाति॒ वर्चो॒ वर्चो॑ दधाति । \newline
50. द॒धा॒ति॒ विश्वे॒ विश्वे॑ दधाति दधाति॒ विश्वे᳚ । \newline
51. विश्वे॑ देवा देवा॒ विश्वे॒ विश्वे॑ देवाः । \newline
52. दे॒वा॒ जर॑दष्टि॒र् जर॑दष्टिर् देवा देवा॒ जर॑दष्टिः । \newline
53. जर॑दष्टि॒र् यथा॒ यथा॒ जर॑दष्टि॒र् जर॑दष्टि॒र् यथा᳚ । \newline
54. जर॑दष्टि॒रिति॒ जर॑त् - अ॒ष्टिः॒ । \newline
55. यथा ऽस॒ दस॒द् यथा॒ यथा ऽस॑त् । \newline
56. अस॒दिती त्यस॒ दस॒दिति॑ । \newline
57. इत्या॑हा॒हे तीत्या॑ह । \newline
58. आ॒ह॒ जर॑दष्टि॒म् जर॑दष्टि माहाह॒ जर॑दष्टिम् । \newline
59. जर॑दष्टि मे॒वैव जर॑दष्टि॒म् जर॑दष्टि मे॒व । \newline
60. जर॑दष्टि॒मिति॒ जर॑त् - अ॒ष्टि॒म् । \newline
61. ए॒वैन॑ मेन मे॒वैवैन᳚म् । \newline
62. ए॒न॒म् क॒रो॒ति॒ क॒रो॒ त्ये॒न॒ मे॒न॒म् क॒रो॒ति॒ । \newline
63. क॒रो॒ त्य॒ग्नि र॒ग्निः क॑रोति करो त्य॒ग्निः । \newline
64. अ॒ग्नि रायु॑ष्मा॒ नायु॑ष्मा न॒ग्नि र॒ग्नि रायु॑ष्मान् । \newline
65. आयु॑ष्मा॒ निती त्यायु॑ष्मा॒ नायु॑ष्मा॒ निति॑ । \newline
66. इति॒ हस्तꣳ॒॒ हस्त॒ मितीति॒ हस्त᳚म् । \newline
67. हस्त॑म् गृह्णाति गृह्णाति॒ हस्तꣳ॒॒ हस्त॑म् गृह्णाति । \newline
68. गृ॒ह्णा॒ त्ये॒त ए॒ते गृ॑ह्णाति गृह्णा त्ये॒ते । \newline
69. ए॒ते वै वा ए॒त ए॒ते वै । \newline
70. वै दे॒वा दे॒वा वै वै दे॒वाः । \newline
71. दे॒वा आयु॑ष्मन्त॒ आयु॑ष्मन्तो दे॒वा दे॒वा आयु॑ष्मन्तः । \newline
72. आयु॑ष्मन्त॒ स्ते त आयु॑ष्मन्त॒ आयु॑ष्मन्त॒ स्ते । \newline
73. त ए॒वैव ते त ए॒व । \newline
74. ए॒वास्मि॑न् नस्मिन् ने॒वैवास्मिन्न्॑ । \newline
75. अ॒स्मि॒न् नायु॒ रायु॑ रस्मिन् नस्मि॒न् नायुः॑ । \newline
76. आयु॑र् दधति दध॒ त्यायु॒ रायु॑र् दधति । \newline
77. द॒ध॒ति॒ सर्वꣳ॒॒ सर्व॑म् दधति दधति॒ सर्व᳚म् । \newline
78. सर्व॒ मायु॒ रायुः॒ सर्वꣳ॒॒ सर्व॒ मायुः॑ । \newline
79. आयु॑ रेत्ये॒त्यायु॒ रायु॑रेति । \newline
80. ए॒तीत्ये॑ति । \newline

\textbf{Ghana Paata } \newline

1. घृ॒तम् निर् णिर् घृ॒तम् घृ॒तम् निष् पि॑बति पिबति॒ निर् घृ॒तम् घृ॒तम् निष् पि॑बति । \newline
2. निष् पि॑बति पिबति॒ निर् णिष् पि॑ब॒ त्यायु॒ रायुः॑ पिबति॒ निर् णिष् पि॑ब॒ त्यायुः॑ । \newline
3. पि॒ब॒ त्यायु॒ रायुः॑ पिबति पिब॒ त्यायु॒र् वै वा आयुः॑ पिबति पिब॒ त्यायु॒र् वै । \newline
4. आयु॒र् वै वा आयु॒ रायु॒र् वै घृ॒तम् घृ॒तं ॅवा आयु॒ रायु॒र् वै घृ॒तम् । \newline
5. वै घृ॒तम् घृ॒तं ॅवै वै घृ॒त म॒मृत॑ म॒मृत॑म् घृ॒तं ॅवै वै घृ॒त म॒मृत᳚म् । \newline
6. घृ॒त म॒मृत॑ म॒मृत॑म् घृ॒तम् घृ॒त म॒मृतꣳ॒॒ हिर॑ण्यꣳ॒॒ हिर॑ण्य म॒मृत॑म् घृ॒तम् घृ॒त म॒मृतꣳ॒॒ हिर॑ण्यम् । \newline
7. अ॒मृतꣳ॒॒ हिर॑ण्यꣳ॒॒ हिर॑ण्य म॒मृत॑ म॒मृतꣳ॒॒ हिर॑ण्य म॒मृता॑ द॒मृता॒ द्धिर॑ण्य म॒मृत॑ म॒मृतꣳ॒॒ हिर॑ण्य म॒मृता᳚त् । \newline
8. हिर॑ण्य म॒मृता॑ द॒मृता॒ द्धिर॑ण्यꣳ॒॒ हिर॑ण्य म॒मृता॑ दे॒वैवामृता॒ द्धिर॑ण्यꣳ॒॒ हिर॑ण्य म॒मृता॑दे॒व । \newline
9. अ॒मृता॑ दे॒वैवा मृता॑ द॒मृता॑ दे॒वायु॒ रायु॑ रे॒वा मृता॑ द॒मृता॑ दे॒वायुः॑ । \newline
10. ए॒वायु॒ रायु॑ रे॒वैवायु॒र् निर् णिरायु॑ रे॒वैवायु॒र् निः । \newline
11. आयु॒र् निर् णिरायु॒ रायु॒र् निष् पि॑बति पिबति॒ निरायु॒ रायु॒र् निष् पि॑बति । \newline
12. निष् पि॑बति पिबति॒ निर् णिष् पि॑बति श॒तमा॑नꣳ श॒तमा॑नम् पिबति॒ निर् णिष् पि॑बति श॒तमा॑नम् । \newline
13. पि॒ब॒ति॒ श॒तमा॑नꣳ श॒तमा॑नम् पिबति पिबति श॒तमा॑नम् भवति भवति श॒तमा॑नम् पिबति पिबति श॒तमा॑नम् भवति । \newline
14. श॒तमा॑नम् भवति भवति श॒तमा॑नꣳ श॒तमा॑नम् भवति श॒तायुः॑ श॒तायु॑र् भवति श॒तमा॑नꣳ श॒तमा॑नम् भवति श॒तायुः॑ । \newline
15. श॒तमा॑न॒मिति॑ श॒त - मा॒न॒म् । \newline
16. भ॒व॒ति॒ श॒तायुः॑ श॒तायु॑र् भवति भवति श॒तायुः॒ पुरु॑षः॒ पुरु॑षः श॒तायु॑र् भवति भवति श॒तायुः॒ पुरु॑षः । \newline
17. श॒तायुः॒ पुरु॑षः॒ पुरु॑षः श॒तायुः॑ श॒तायुः॒ पुरु॑षः श॒तेन्द्रि॑यः श॒तेन्द्रि॑यः॒ पुरु॑षः श॒तायुः॑ श॒तायुः॒ पुरु॑षः श॒तेन्द्रि॑यः । \newline
18. श॒तायु॒रिति॑ श॒त - आ॒युः॒ । \newline
19. पुरु॑षः श॒तेन्द्रि॑यः श॒तेन्द्रि॑यः॒ पुरु॑षः॒ पुरु॑षः श॒तेन्द्रि॑य॒ आयु॒ ष्यायु॑षि श॒तेन्द्रि॑यः॒ पुरु॑षः॒ पुरु॑षः श॒तेन्द्रि॑य॒ आयु॑षि । \newline
20. श॒तेन्द्रि॑य॒ आयु॒ ष्यायु॑षि श॒तेन्द्रि॑यः श॒तेन्द्रि॑य॒ आयु॑ ष्ये॒वैवायु॑षि श॒तेन्द्रि॑यः श॒तेन्द्रि॑य॒ आयु॑ष्ये॒व । \newline
21. श॒तेन्द्रि॑य॒ इति॑ श॒त - इ॒न्द्रि॒यः॒ । \newline
22. आयु॑ ष्ये॒वै वायु॒ ष्यायु॑ ष्ये॒वे न्द्रि॒य इ॑न्द्रि॒य ए॒वायु॒ ष्यायु॑ ष्ये॒वे न्द्रि॒ये । \newline
23. ए॒वे न्द्रि॒य इ॑न्द्रि॒य ए॒वैवे न्द्रि॒ये प्रति॒ प्रती᳚न्द्रि॒य ए॒वैवे न्द्रि॒ये प्रति॑ । \newline
24. इ॒न्द्रि॒ये प्रति॒ प्रती᳚न्द्रि॒य इ॑न्द्रि॒ये प्रति॑ तिष्ठति तिष्ठति॒ प्रती᳚न्द्रि॒य इ॑न्द्रि॒ये प्रति॑ तिष्ठति । \newline
25. प्रति॑ तिष्ठति तिष्ठति॒ प्रति॒ प्रति॑ तिष्ठ॒ त्यथो॒ अथो॑ तिष्ठति॒ प्रति॒ प्रति॑ तिष्ठ॒ त्यथो᳚ । \newline
26. ति॒ष्ठ॒ त्यथो॒ अथो॑ तिष्ठति तिष्ठ॒ त्यथो॒ खलु॒ खल्वथो॑ तिष्ठति तिष्ठ॒ त्यथो॒ खलु॑ । \newline
27. अथो॒ खलु॒ खल्वथो॒ अथो॒ खलु॒ याव॑ती॒र् याव॑तीः॒ खल्वथो॒ अथो॒ खलु॒ याव॑तीः । \newline
28. अथो॒ इत्यथो᳚ । \newline
29. खलु॒ याव॑ती॒र् याव॑तीः॒ खलु॒ खलु॒ याव॑तीः॒ समाः॒ समा॒ याव॑तीः॒ खलु॒ खलु॒ याव॑तीः॒ समाः᳚ । \newline
30. याव॑तीः॒ समाः॒ समा॒ याव॑ती॒र् याव॑तीः॒ समा॑ ए॒ष्यन् ने॒ष्यन् थ्समा॒ याव॑ती॒र् याव॑तीः॒ समा॑ ए॒ष्यन्न् । \newline
31. समा॑ ए॒ष्यन् ने॒ष्यन् थ्समाः॒ समा॑ ए॒ष्यन् मन्ये॑त॒ मन्ये॑तै॒ष्यन् थ्समाः॒ समा॑ ए॒ष्यन् मन्ये॑त । \newline
32. ए॒ष्यन् मन्ये॑त॒ मन्ये॑तै॒ष्यन् ने॒ष्यन् मन्ये॑त॒ ताव॑न्मान॒म् ताव॑न्मान॒म् मन्ये॑तै॒ष्यन् ने॒ष्यन् मन्ये॑त॒ ताव॑न्मानम् । \newline
33. मन्ये॑त॒ ताव॑न्मान॒म् ताव॑न्मान॒म् मन्ये॑त॒ मन्ये॑त॒ ताव॑न्मानꣳ स्याथ् स्या॒त् ताव॑न्मान॒म् मन्ये॑त॒ मन्ये॑त॒ ताव॑न्मानꣳ स्यात् । \newline
34. ताव॑न्मानꣳ स्याथ् स्या॒त् ताव॑न्मान॒म् ताव॑न्मानꣳ स्या॒थ् समृ॑द्ध्यै॒ समृ॑द्ध्यै स्या॒त् ताव॑न्मान॒म् ताव॑न्मानꣳ स्या॒थ् समृ॑द्ध्यै । \newline
35. ताव॑न्मान॒मिति॒ ताव॑त् - मा॒न॒म् । \newline
36. स्या॒थ् समृ॑द्ध्यै॒ समृ॑द्ध्यै स्याथ् स्या॒थ् समृ॑द्ध्या इ॒म मि॒मꣳ समृ॑द्ध्यै स्याथ् स्या॒थ् समृ॑द्ध्या इ॒मम् । \newline
37. समृ॑द्ध्या इ॒म मि॒मꣳ समृ॑द्ध्यै॒ समृ॑द्ध्या इ॒म म॑ग्ने ऽग्न इ॒मꣳ समृ॑द्ध्यै॒ समृ॑द्ध्या इ॒म म॑ग्ने । \newline
38. समृ॑द्ध्या॒ इति॒ सम् - ऋ॒द्ध्यै॒ । \newline
39. इ॒म म॑ग्ने ऽग्न इ॒म मि॒म म॑ग्न॒ आयु॑ष॒ आयु॑षे ऽग्न इ॒म मि॒म म॑ग्न॒ आयु॑षे । \newline
40. अ॒ग्न॒ आयु॑ष॒ आयु॑षे ऽग्ने ऽग्न॒ आयु॑षे॒ वर्च॑से॒ वर्च॑स॒ आयु॑षे ऽग्ने ऽग्न॒ आयु॑षे॒ वर्च॑से । \newline
41. आयु॑षे॒ वर्च॑से॒ वर्च॑स॒ आयु॑ष॒ आयु॑षे॒ वर्च॑से कृधि कृधि॒ वर्च॑स॒ आयु॑ष॒ आयु॑षे॒ वर्च॑से कृधि । \newline
42. वर्च॑से कृधि कृधि॒ वर्च॑से॒ वर्च॑से कृ॒धीतीति॑ कृधि॒ वर्च॑से॒ वर्च॑से कृ॒धीति॑ । \newline
43. कृ॒धीतीति॑ कृधि कृ॒धी त्या॑हा॒हे ति॑ कृधि कृ॒धी त्या॑ह । \newline
44. इत्या॑हा॒हे तीत्या॒हायु॒ रायु॑रा॒हे तीत्या॒हायुः॑ । \newline
45. आ॒हायु॒ रायु॑ राहा॒हायु॑ रे॒वैवायु॑ राहा॒हायु॑ रे॒व । \newline
46. आयु॑ रे॒वैवायु॒ रायु॑ रे॒वास्मि॑न् नस्मिन् ने॒वायु॒ रायु॑ रे॒वास्मिन्न्॑ । \newline
47. ए॒वास्मि॑न् नस्मिन् ने॒वैवास्मि॒न्॒. वर्चो॒ वर्चो᳚ ऽस्मिन् ने॒वैवास्मि॒न्॒. वर्चः॑ । \newline
48. अ॒स्मि॒न्॒. वर्चो॒ वर्चो᳚ ऽस्मिन् नस्मि॒न्॒. वर्चो॑ दधाति दधाति॒ वर्चो᳚ ऽस्मिन् नस्मि॒न्॒. वर्चो॑ दधाति । \newline
49. वर्चो॑ दधाति दधाति॒ वर्चो॒ वर्चो॑ दधाति॒ विश्वे॒ विश्वे॑ दधाति॒ वर्चो॒ वर्चो॑ दधाति॒ विश्वे᳚ । \newline
50. द॒धा॒ति॒ विश्वे॒ विश्वे॑ दधाति दधाति॒ विश्वे॑ देवा देवा॒ विश्वे॑ दधाति दधाति॒ विश्वे॑ देवाः । \newline
51. विश्वे॑ देवा देवा॒ विश्वे॒ विश्वे॑ देवा॒ जर॑दष्टि॒र् जर॑दष्टिर् देवा॒ विश्वे॒ विश्वे॑ देवा॒ जर॑दष्टिः । \newline
52. दे॒वा॒ जर॑दष्टि॒र् जर॑दष्टिर् देवा देवा॒ जर॑दष्टि॒र् यथा॒ यथा॒ जर॑दष्टिर् देवा देवा॒ जर॑दष्टि॒र् यथा᳚ । \newline
53. जर॑दष्टि॒र् यथा॒ यथा॒ जर॑दष्टि॒र् जर॑दष्टि॒र् यथा ऽस॒दस॒द् यथा॒ जर॑दष्टि॒र् जर॑दष्टि॒र् यथा ऽस॑त् । \newline
54. जर॑दष्टि॒रिति॒ जर॑त् - अ॒ष्टिः॒ । \newline
55. यथा ऽस॒दस॒द् यथा॒ यथा ऽस॒दिती त्यस॒द् यथा॒ यथा ऽस॒दिति॑ । \newline
56. अस॒दिती त्यस॒ दस॒ दित्या॑हा॒हे त्यस॒दस॒ दित्या॑ह । \newline
57. इत्या॑हा॒हे तीत्या॑ह॒ जर॑दष्टि॒म् जर॑दष्टि मा॒हे तीत्या॑ह॒ जर॑दष्टिम् । \newline
58. आ॒ह॒ जर॑दष्टि॒म् जर॑दष्टि माहाह॒ जर॑दष्टि मे॒वैव जर॑दष्टि माहाह॒ जर॑दष्टि मे॒व । \newline
59. जर॑दष्टि मे॒वैव जर॑दष्टि॒म् जर॑दष्टि मे॒वैन॑ मेन मे॒व जर॑दष्टि॒म् जर॑दष्टि मे॒वैन᳚म् । \newline
60. जर॑दष्टि॒मिति॒ जर॑त् - अ॒ष्टि॒म् । \newline
61. ए॒वैन॑ मेन मे॒वैवैन॑म् करोति करोत्येन मे॒वैवैन॑म् करोति । \newline
62. ए॒न॒म् क॒रो॒ति॒ क॒रो॒ त्ये॒न॒ मे॒न॒म् क॒रो॒ त्य॒ग्नि र॒ग्निः क॑रोत्येन मेनम् करो त्य॒ग्निः । \newline
63. क॒रो॒ त्य॒ग्नि र॒ग्निः क॑रोति करो त्य॒ग्नि रायु॑ष्मा॒ नायु॑ष्मा न॒ग्निः क॑रोति करो त्य॒ग्नि रायु॑ष्मान् । \newline
64. अ॒ग्नि रायु॑ष्मा॒ नायु॑ष्मा न॒ग्नि र॒ग्नि रायु॑ष्मा॒ निती त्यायु॑ष्मा न॒ग्नि र॒ग्नि रायु॑ष्मा॒ निति॑ । \newline
65. आयु॑ष्मा॒ निती त्यायु॑ष्मा॒ नायु॑ष्मा॒ निति॒ हस्तꣳ॒॒ हस्त॒ मित्यायु॑ष्मा॒ नायु॑ष्मा॒ निति॒ हस्त᳚म् । \newline
66. इति॒ हस्तꣳ॒॒ हस्त॒ मितीति॒ हस्त॑म् गृह्णाति गृह्णाति॒ हस्त॒ मितीति॒ हस्त॑म् गृह्णाति । \newline
67. हस्त॑म् गृह्णाति गृह्णाति॒ हस्तꣳ॒॒ हस्त॑म् गृह्णात्ये॒त ए॒ते गृ॑ह्णाति॒ हस्तꣳ॒॒ हस्त॑म् गृह्णात्ये॒ते । \newline
68. गृ॒ह्णा॒त्ये॒त ए॒ते गृ॑ह्णाति गृह्णात्ये॒ते वै वा ए॒ते गृ॑ह्णाति गृह्णात्ये॒ते वै । \newline
69. ए॒ते वै वा ए॒त ए॒ते वै दे॒वा दे॒वा वा ए॒त ए॒ते वै दे॒वाः । \newline
70. वै दे॒वा दे॒वा वै वै दे॒वा आयु॑ष्मन्त॒ आयु॑ष्मन्तो दे॒वा वै वै दे॒वा आयु॑ष्मन्तः । \newline
71. दे॒वा आयु॑ष्मन्त॒ आयु॑ष्मन्तो दे॒वा दे॒वा आयु॑ष्मन्त॒ स्ते त आयु॑ष्मन्तो दे॒वा दे॒वा आयु॑ष्मन्त॒ स्ते । \newline
72. आयु॑ष्मन्त॒ स्ते त आयु॑ष्मन्त॒ आयु॑ष्मन्त॒ स्त ए॒वैव त आयु॑ष्मन्त॒ आयु॑ष्मन्त॒ स्त ए॒व । \newline
73. त ए॒वैव ते त ए॒वास्मि॑न् नस्मिन् ने॒व ते त ए॒वास्मिन्न्॑ । \newline
74. ए॒वास्मि॑न् नस्मिन् ने॒वैवास्मि॒न् नायु॒ रायु॑ रस्मिन् ने॒वैवास्मि॒न् नायुः॑ । \newline
75. अ॒स्मि॒न् नायु॒ रायु॑ रस्मिन् नस्मि॒न् नायु॑र् दधति दध॒ त्यायु॑ रस्मिन् नस्मि॒न् नायु॑र् दधति । \newline
76. आयु॑र् दधति दध॒ त्यायु॒ रायु॑र् दधति॒ सर्वꣳ॒॒ सर्व॑म् दध॒ त्यायु॒ रायु॑र् दधति॒ सर्व᳚म् । \newline
77. द॒ध॒ति॒ सर्वꣳ॒॒ सर्व॑म् दधति दधति॒ सर्व॒ मायु॒ रायुः॒ सर्व॑म् दधति दधति॒ सर्व॒ मायुः॑ । \newline
78. सर्व॒ मायु॒रायुः॒ सर्वꣳ॒॒ सर्व॒ मायु॑ रेत्ये॒त्यायुः॒ सर्वꣳ॒॒ सर्व॒ मायु॑रेति । \newline
79. आयु॑ रेत्ये॒त्यायु॒ रायु॑रेति । \newline
80. ए॒तीत्ये॑ति । \newline
\pagebreak
\markright{ TS 2.3.12.1  \hfill https://www.vedavms.in \hfill}

\section{ TS 2.3.12.1 }

\textbf{TS 2.3.12.1 } \newline
\textbf{Samhita Paata} \newline

प्र॒जाप॑ति॒ र्वरु॑णा॒याश्व॑मनय॒थ् स स्वां दे॒वता॑मार्च्छ॒थ् स पर्य॑दीर्यत॒ स ए॒तं ॅवा॑रु॒णं चतु॑ष्-कपालमपश्य॒त् तं निर॑वप॒त् ततो॒ वै स व॑रुण- पा॒शाद॑मुच्यत॒ वरु॑णो॒ वा ए॒तं गृ॑ह्णाति॒ योऽश्वं॑ प्रतिगृ॒ह्णाति॒ याव॒तोऽश्वा᳚न् प्रतिगृह्णी॒यात् ताव॑तो वारु॒णान् चतु॑ष्कपाला॒न् निर्व॑पे॒द्-वरु॑णमे॒व स्वेन॑ भाग॒धेये॒नोप॑ धावति॒ स ए॒वैनं॑ ॅवरुणपा॒शान् -मु॑ञ्चति॒ - [  ] \newline

\textbf{Pada Paata} \newline

प्र॒जाप॑ति॒रिति॑ प्र॒जा - प॒तिः॒ । वरु॑णाय । अश्व᳚म् । अ॒न॒य॒त् । सः । स्वाम् । दे॒वता᳚म् । आ॒र्च्छ॒त् । सः । परीति॑ । अ॒दी॒र्य॒त॒ । सः । ए॒तम् । वा॒रु॒णम् । चतु॑ष्कपाल॒मिति॒ चतुः॑ - क॒पा॒ल॒म् । अ॒प॒श्य॒त् । तम् । निरिति॑ । अ॒व॒प॒त् । ततः॑ । वै । सः । व॒रु॒ण॒पा॒शादिति॑ वरुण - पा॒शात् । अ॒मु॒च्य॒त॒ । वरु॑णः । वै । ए॒तम् । गृ॒ह्णा॒ति॒ । यः । अश्व᳚म् । प्र॒ति॒गृ॒ह्णातीति॑ प्रति - गृ॒ह्णाति॑ । याव॑तः । अश्वान्॑ । प्र॒ति॒गृ॒ह्णी॒यादिति॑ प्रति - गृ॒ह्णी॒यात् । ताव॑तः । वा॒रु॒णान् । चतु॑ष्कपाला॒निति॒ चतुः॑ - क॒पा॒ला॒न् । निरिति॑ । व॒पे॒त् । वरु॑णम् । ए॒व । स्वेन॑ । भा॒ग॒धेये॒नेति॑ भाग - धेये॑न । उपेति॑ । धा॒व॒ति॒ । सः । ए॒व । ए॒न॒म् । व॒रु॒ण॒पा॒शादिति॑ वरुण - पा॒शात् । मु॒ञ्च॒ति॒ ।  \newline


\textbf{Krama Paata} \newline

प्र॒जाप॑ति॒र् वरु॑णाय । प्र॒जाप॑ति॒रिति॑ प्र॒जा - प॒तिः॒ । वरु॑णा॒याश्व᳚म् । अश्व॑मनयत् । अ॒न॒य॒थ् सः । स स्वाम् । स्वाम् दे॒वता᳚म् । दे॒वता॑मार्च्छत् । आ॒र्च्छ॒थ् सः । स परि॑ । पर्य॑दीर्यत । अ॒दी॒र्य॒त॒ सः । स ए॒तम् । ए॒तं ॅवा॑रु॒णम् । वा॒रु॒णम् चतु॑ष्कपालम् । चतु॑ष्कपालमपश्यत् । चतु॑ष्कपाल॒मिति॒ चतुः॑ - क॒पा॒ल॒म् । अ॒प॒श्य॒त् तम् । तम् निः । निर॑वपत् । अ॒व॒प॒त् ततः॑ । ततो॒ वै । वै सः । स व॑रुणपा॒शात् । व॒रु॒ण॒पा॒शाद॑मुच्यत । व॒रु॒ण॒पा॒शादिति॑ वरुण - पा॒शात् । अ॒मु॒च्य॒त॒ वरु॑णः । वरु॑णो॒ वै । वा ए॒तम् । ए॒तम् गृ॑ह्णाति । गृ॒ह्णा॒ति॒ यः । यो ऽश्व᳚म् । अश्व॑म् प्रतिगृ॒ह्णाति॑ । प्र॒ति॒गृ॒ह्णाति॒ याव॑तः । प्र॒ति॒गृ॒ह्णातीति॑ प्रति - गृ॒ह्णाति॑ । याव॒तो ऽश्वान्॑ । अश्वा᳚न् प्रतिगृह्णी॒यात् । प्र॒ति॒गृ॒ह्णी॒यात् ताव॑तः । प्र॒ति॒गृ॒ह्णी॒यादिति॑ प्रति - गृ॒ह्णी॒यात् । ताव॑तो वारु॒णान् । वा॒रु॒णान् चतु॑ष्कपालान् । चतु॑ष्कपाला॒न् निः । चतु॑ष्कपाला॒निति॒ चतुः॑ - क॒पा॒ला॒न् । निर् व॑पेत् । व॒पे॒द् वरु॑णम् । वरु॑णमे॒व । ए॒व स्वेन॑ । स्वेन॑ भाग॒धेये॑न । भा॒ग॒धेये॒नोप॑ । भा॒ग॒धेये॒नेति॑ भाग - धेये॑न । उप॑ धावति । धा॒व॒ति॒ सः । स ए॒व । ए॒वैन᳚म् । ए॒नं॒ ॅव॒रु॒ण॒पा॒शात् । व॒रु॒ण॒पा॒शान् मु॑ञ्चति । व॒रु॒ण॒पा॒शादिति॑ वरुण - पा॒शात् । मु॒ञ्च॒ति॒ चतु॑ष्कपालाः \newline

\textbf{Jatai Paata} \newline

1. प्र॒जाप॑ति॒र् वरु॑णाय॒ वरु॑णाय प्र॒जाप॑तिः प्र॒जाप॑ति॒र् वरु॑णाय । \newline
2. प्र॒जाप॑ति॒रिति॑ प्र॒जा - प॒तिः॒ । \newline
3. वरु॑णा॒याश्व॒ मश्वं॒ ॅवरु॑णाय॒ वरु॑णा॒याश्व᳚म् । \newline
4. अश्व॑ मनय दनय॒ दश्व॒ मश्व॑ मनयत् । \newline
5. अ॒न॒य॒थ् स सो॑ ऽनय दनय॒थ् सः । \newline
6. स स्वाꣳ स्वाꣳ स स स्वाम् । \newline
7. स्वाम् दे॒वता᳚म् दे॒वताꣳ॒॒ स्वाꣳ स्वाम् दे॒वता᳚म् । \newline
8. दे॒वता॑ मार्च्छ दार्च्छद् दे॒वता᳚म् दे॒वता॑ मार्च्छत् । \newline
9. आ॒र्च्छ॒थ् स स आ᳚र्च्छ दार्च्छ॒थ् सः । \newline
10. स परि॒ परि॒ स स परि॑ । \newline
11. पर्य॑दीर्यता दीर्यत॒ परि॒ पर्य॑दीर्यत । \newline
12. अ॒दी॒र्य॒त॒ स सो॑ ऽदीर्यता दीर्यत॒ सः । \newline
13. स ए॒त मे॒तꣳ स स ए॒तम् । \newline
14. ए॒तं ॅवा॑रु॒णं ॅवा॑रु॒ण मे॒त मे॒तं ॅवा॑रु॒णम् । \newline
15. वा॒रु॒णम् चतु॑ष्कपाल॒म् चतु॑ष्कपालं ॅवारु॒णं ॅवा॑रु॒णम् चतु॑ष्कपालम् । \newline
16. चतु॑ष्कपाल मपश्यदपश्य॒च् चतु॑ष्कपाल॒म् चतु॑ष्कपाल मपश्यत् । \newline
17. चतु॑ष्कपाल॒मिति॒ चतुः॑ - क॒पा॒ल॒म् । \newline
18. अ॒प॒श्य॒त् तम् त म॑पश्य दपश्य॒त् तम् । \newline
19. तम् निर् णिष् टम् तम् निः । \newline
20. निर॑वप दवप॒न् निर् णि र॑वपत् । \newline
21. अ॒व॒प॒त् तत॒ स्ततो॑ ऽवप दवप॒त् ततः॑ । \newline
22. ततो॒ वै वै तत॒ स्ततो॒ वै । \newline
23. वै स स वै वै सः । \newline
24. स व॑रुणपा॒शाद् व॑रुणपा॒शाथ् स स व॑रुणपा॒शात् । \newline
25. व॒रु॒ण॒पा॒शा द॑मुच्यता मुच्यत वरुणपा॒शाद् व॑रुणपा॒शा द॑मुच्यत । \newline
26. व॒रु॒ण॒पा॒शादिति॑ वरुण - पा॒शात् । \newline
27. अ॒मु॒च्य॒त॒ वरु॑णो॒ वरु॑णो ऽमुच्यता मुच्यत॒ वरु॑णः । \newline
28. वरु॑णो॒ वै वै वरु॑णो॒ वरु॑णो॒ वै । \newline
29. वा ए॒त मे॒तं ॅवै वा ए॒तम् । \newline
30. ए॒तम् गृ॑ह्णाति गृह्णात्ये॒त मे॒तम् गृ॑ह्णाति । \newline
31. गृ॒ह्णा॒ति॒ यो यो गृ॑ह्णाति गृह्णाति॒ यः । \newline
32. यो ऽश्व॒ मश्वं॒ ॅयो यो ऽश्व᳚म् । \newline
33. अश्व॑म् प्रतिगृ॒ह्णाति॑ प्रतिगृ॒ह्णा त्यश्व॒ मश्व॑म् प्रतिगृ॒ह्णाति॑ । \newline
34. प्र॒ति॒गृ॒ह्णाति॒ याव॑तो॒ याव॑तः प्रतिगृ॒ह्णाति॑ प्रतिगृ॒ह्णाति॒ याव॑तः । \newline
35. प्र॒ति॒गृ॒ह्णातीति॑ प्रति - गृ॒ह्णाति॑ । \newline
36. याव॒तो ऽश्वा॒ नश्वा॒न्॒. याव॑तो॒ याव॒तो ऽश्वान्॑ । \newline
37. अश्वा᳚न् प्रतिगृह्णी॒यात् प्र॑तिगृह्णी॒या दश्वा॒ नश्वा᳚न् प्रतिगृह्णी॒यात् । \newline
38. प्र॒ति॒गृ॒ह्णी॒यात् ताव॑त॒ स्ताव॑तः प्रतिगृह्णी॒यात् प्र॑तिगृह्णी॒यात् ताव॑तः । \newline
39. प्र॒ति॒गृ॒ह्णी॒यादिति॑ प्रति - गृ॒ह्णी॒यात् । \newline
40. ताव॑तो वारु॒णान्. वा॑रु॒णान् ताव॑त॒ स्ताव॑तो वारु॒णान् । \newline
41. वा॒रु॒णान् चतु॑ष्कपालाꣳ॒॒श् चतु॑ष्कपालान्. वारु॒णान्. वा॑रु॒णान् चतु॑ष्कपालान् । \newline
42. चतु॑ष्कपाला॒न् निर् णिश्चतु॑ष्कपालाꣳ॒॒ श्चतु॑ष्कपाला॒न् निः । \newline
43. चतु॑ष्कपाला॒निति॒ चतुः॑ - क॒पा॒ला॒न् । \newline
44. निर् व॑पेद् वपे॒न् निर् णिर् व॑पेत् । \newline
45. व॒पे॒द् वरु॑णं॒ ॅवरु॑णं ॅवपेद् वपे॒द् वरु॑णम् । \newline
46. वरु॑ण मे॒वैव वरु॑णं॒ ॅवरु॑ण मे॒व । \newline
47. ए॒व स्वेन॒ स्वेनै॒वैव स्वेन॑ । \newline
48. स्वेन॑ भाग॒धेये॑न भाग॒धेये॑न॒ स्वेन॒ स्वेन॑ भाग॒धेये॑न । \newline
49. भा॒ग॒धेये॒नोपोप॑ भाग॒धेये॑न भाग॒धेये॒नोप॑ । \newline
50. भा॒ग॒धेये॒नेति॑ भाग - धेये॑न । \newline
51. उप॑ धावति धाव॒ त्युपोप॑ धावति । \newline
52. धा॒व॒ति॒ स स धा॑वति धावति॒ सः । \newline
53. स ए॒वैव स स ए॒व । \newline
54. ए॒वैन॑ मेन मे॒वैवैन᳚म् । \newline
55. ए॒नं॒ ॅव॒रु॒ण॒पा॒शाद् व॑रुणपा॒शादे॑न मेनं ॅवरुणपा॒शात् । \newline
56. व॒रु॒ण॒पा॒शान् मु॑ञ्चति मुञ्चति वरुणपा॒शाद् व॑रुणपा॒शान् मु॑ञ्चति । \newline
57. व॒रु॒ण॒पा॒शादिति॑ वरुण - पा॒शात् । \newline
58. मु॒ञ्च॒ति॒ चतु॑ष्कपाला॒ श्चतु॑ष्कपाला मुञ्चति मुञ्चति॒ चतु॑ष्कपालाः । \newline

\textbf{Ghana Paata } \newline

1. प्र॒जाप॑ति॒र् वरु॑णाय॒ वरु॑णाय प्र॒जाप॑तिः प्र॒जाप॑ति॒र् वरु॑णा॒याश्व॒ मश्वं॒ ॅवरु॑णाय प्र॒जाप॑तिः प्र॒जाप॑ति॒र् वरु॑णा॒याश्व᳚म् । \newline
2. प्र॒जाप॑ति॒रिति॑ प्र॒जा - प॒तिः॒ । \newline
3. वरु॑णा॒याश्व॒ मश्वं॒ ॅवरु॑णाय॒ वरु॑णा॒याश्व॑ मनय दनय॒ दश्वं॒ ॅवरु॑णाय॒ वरु॑णा॒याश्व॑ मनयत् । \newline
4. अश्व॑ मनय दनय॒ दश्व॒ मश्व॑ मनय॒थ् स सो॑ ऽनय॒दश्व॒ मश्व॑ मनय॒थ् सः । \newline
5. अ॒न॒य॒थ् स सो॑ ऽनय दनय॒थ् स स्वाꣳ स्वाꣳ सो॑ ऽनय दनय॒थ् स स्वाम् । \newline
6. स स्वाꣳ स्वाꣳ स स स्वाम् दे॒वता᳚म् दे॒वताꣳ॒॒ स्वाꣳ स स स्वाम् दे॒वता᳚म् । \newline
7. स्वाम् दे॒वता᳚म् दे॒वताꣳ॒॒ स्वाꣳ स्वाम् दे॒वता॑ मार्च्छ दार्च्छद् दे॒वताꣳ॒॒ स्वाꣳ स्वाम् दे॒वता॑ मार्च्छत् । \newline
8. दे॒वता॑ मार्च्छ दार्च्छद् दे॒वता᳚म् दे॒वता॑ मार्च्छ॒थ् स स आ᳚र्च्छद् दे॒वता᳚म् दे॒वता॑ मार्च्छ॒थ् सः । \newline
9. आ॒र्च्छ॒थ् स स आ᳚र्च्छ दार्च्छ॒थ् स परि॒ परि॒ स आ᳚र्च्छ दार्च्छ॒थ् स परि॑ । \newline
10. स परि॒ परि॒ स स पर्य॑दीर्यतादीर्यत॒ परि॒ स स पर्य॑दीर्यत । \newline
11. पर्य॑दीर्यता दीर्यत॒ परि॒ पर्य॑दीर्यत॒ स सो॑ ऽदीर्यत॒ परि॒ पर्य॑दीर्यत॒ सः । \newline
12. अ॒दी॒र्य॒त॒ स सो॑ ऽदीर्यता दीर्यत॒ स ए॒त मे॒तꣳ सो॑ ऽदीर्यता दीर्यत॒ स ए॒तम् । \newline
13. स ए॒त मे॒तꣳ स स ए॒तं ॅवा॑रु॒णं ॅवा॑रु॒ण मे॒तꣳ स स ए॒तं ॅवा॑रु॒णम् । \newline
14. ए॒तं ॅवा॑रु॒णं ॅवा॑रु॒ण मे॒त मे॒तं ॅवा॑रु॒णम् चतु॑ष्कपाल॒म् चतु॑ष्कपालं ॅवारु॒ण मे॒त मे॒तं ॅवा॑रु॒णम् चतु॑ष्कपालम् । \newline
15. वा॒रु॒णम् चतु॑ष्कपाल॒म् चतु॑ष्कपालं ॅवारु॒णं ॅवा॑रु॒णम् चतु॑ष्कपाल मपश्य दपश्य॒च् चतु॑ष्कपालं ॅवारु॒णं ॅवा॑रु॒णम् चतु॑ष्कपाल मपश्यत् । \newline
16. चतु॑ष्कपाल मपश्य दपश्य॒च् चतु॑ष्कपाल॒म् चतु॑ष्कपाल मपश्य॒त् तम् त म॑पश्य॒च् चतु॑ष्कपाल॒म् चतु॑ष्कपाल मपश्य॒त् तम् । \newline
17. चतु॑ष्कपाल॒मिति॒ चतुः॑ - क॒पा॒ल॒म् । \newline
18. अ॒प॒श्य॒त् तम् त म॑पश्य दपश्य॒त् तम् निर् णिष्ट म॑पश्य दपश्य॒त् तम् निः । \newline
19. तम् निर् णिष् टम् तम् निर॑वप दवप॒न् निष् टम् तम् निर॑वपत् । \newline
20. निर॑वप दवप॒न् निर् णिर॑वप॒त् तत॒ स्ततो॑ ऽवप॒न् निर् णिर॑वप॒त् ततः॑ । \newline
21. अ॒व॒प॒त् तत॒ स्ततो॑ ऽवप दवप॒त् ततो॒ वै वै ततो॑ ऽवप दवप॒त् ततो॒ वै । \newline
22. ततो॒ वै वै तत॒ स्ततो॒ वै स स वै तत॒ स्ततो॒ वै सः । \newline
23. वै स स वै वै स व॑रुणपा॒शाद् व॑रुणपा॒शाथ् स वै वै स व॑रुणपा॒शात् । \newline
24. स व॑रुणपा॒शाद् व॑रुणपा॒शाथ् स स व॑रुणपा॒शा द॑मुच्यता मुच्यत वरुणपा॒शाथ् स स व॑रुणपा॒शा द॑मुच्यत । \newline
25. व॒रु॒ण॒पा॒शा द॑मुच्यता मुच्यत वरुणपा॒शाद् व॑रुणपा॒शा द॑मुच्यत॒ वरु॑णो॒ वरु॑णो ऽमुच्यत वरुणपा॒शाद् व॑रुणपा॒शा द॑मुच्यत॒ वरु॑णः । \newline
26. व॒रु॒ण॒पा॒शादिति॑ वरुण - पा॒शात् । \newline
27. अ॒मु॒च्य॒त॒ वरु॑णो॒ वरु॑णो ऽमुच्यता मुच्यत॒ वरु॑णो॒ वै वै वरु॑णो ऽमुच्यता मुच्यत॒ वरु॑णो॒ वै । \newline
28. वरु॑णो॒ वै वै वरु॑णो॒ वरु॑णो॒ वा ए॒त मे॒तं ॅवै वरु॑णो॒ वरु॑णो॒ वा ए॒तम् । \newline
29. वा ए॒त मे॒तं ॅवै वा ए॒तम् गृ॑ह्णाति गृह्णा त्ये॒तं ॅवै वा ए॒तम् गृ॑ह्णाति । \newline
30. ए॒तम् गृ॑ह्णाति गृह्णा त्ये॒त मे॒तम् गृ॑ह्णाति॒ यो यो गृ॑ह्णा त्ये॒त मे॒तम् गृ॑ह्णाति॒ यः । \newline
31. गृ॒ह्णा॒ति॒ यो यो गृ॑ह्णाति गृह्णाति॒ यो ऽश्व॒ मश्वं॒ ॅयो गृ॑ह्णाति गृह्णाति॒ यो ऽश्व᳚म् । \newline
32. यो ऽश्व॒ मश्वं॒ ॅयो यो ऽश्व॑म् प्रतिगृ॒ह्णाति॑ प्रतिगृ॒ह्णा त्यश्वं॒ ॅयो यो ऽश्व॑म् प्रतिगृ॒ह्णाति॑ । \newline
33. अश्व॑म् प्रतिगृ॒ह्णाति॑ प्रतिगृ॒ह्णा त्यश्व॒ मश्व॑म् प्रतिगृ॒ह्णाति॒ याव॑तो॒ याव॑तः प्रतिगृ॒ह्णा त्यश्व॒ मश्व॑म् प्रतिगृ॒ह्णाति॒ याव॑तः । \newline
34. प्र॒ति॒गृ॒ह्णाति॒ याव॑तो॒ याव॑तः प्रतिगृ॒ह्णाति॑ प्रतिगृ॒ह्णाति॒ याव॒तो ऽश्वा॒ नश्वा॒न्॒. याव॑तः प्रतिगृ॒ह्णाति॑ प्रतिगृ॒ह्णाति॒ याव॒तो ऽश्वान्॑ । \newline
35. प्र॒ति॒गृ॒ह्णातीति॑ प्रति - गृ॒ह्णाति॑ । \newline
36. याव॒तो ऽश्वा॒ नश्वा॒न्॒. याव॑तो॒ याव॒तो ऽश्वा᳚न् प्रतिगृह्णी॒यात् प्र॑तिगृह्णी॒या दश्वा॒न्॒. याव॑तो॒ याव॒तो ऽश्वा᳚न् प्रतिगृह्णी॒यात् । \newline
37. अश्वा᳚न् प्रतिगृह्णी॒यात् प्र॑तिगृह्णी॒या दश्वा॒ नश्वा᳚न् प्रतिगृह्णी॒यात् ताव॑त॒ स्ताव॑तः प्रतिगृह्णी॒या दश्वा॒ 
नश्वा᳚न् प्रतिगृह्णी॒यात् ताव॑तः । \newline
38. प्र॒ति॒गृ॒ह्णी॒यात् ताव॑त॒ स्ताव॑तः प्रतिगृह्णी॒यात् प्र॑तिगृह्णी॒यात् ताव॑तो वारु॒णान्. वा॑रु॒णान् ताव॑तः प्रतिगृह्णी॒यात् प्र॑तिगृह्णी॒यात् ताव॑तो वारु॒णान् । \newline
39. प्र॒ति॒गृ॒ह्णी॒यादिति॑ प्रति - गृ॒ह्णी॒यात् । \newline
40. ताव॑तो वारु॒णान्. वा॑रु॒णान् ताव॑त॒ स्ताव॑तो वारु॒णान् चतु॑ष्कपालाꣳ॒॒ श्चतु॑ष्कपालान्. वारु॒णान् ताव॑त॒ स्ताव॑तो वारु॒णान् चतु॑ष्कपालान् । \newline
41. वा॒रु॒णान् चतु॑ष्कपालाꣳ॒॒ श्चतु॑ष्कपालान्. वारु॒णान्. वा॑रु॒णान् चतु॑ष्कपाला॒न् निर् णिश्चतु॑ष्कपालान्. वारु॒णान्. वा॑रु॒णान् चतु॑ष्कपाला॒न् निः । \newline
42. चतु॑ष्कपाला॒न् निर् णिश्चतु॑ष्कपालाꣳ॒॒ श्चतु॑ष्कपाला॒न् निर् व॑पेद् वपे॒न् निश्चतु॑ष्कपालाꣳ॒॒ श्चतु॑ष्कपाला॒न् निर् व॑पेत् । \newline
43. चतु॑ष्कपाला॒निति॒ चतुः॑ - क॒पा॒ला॒न् । \newline
44. निर् व॑पेद् वपे॒न् निर् णिर् व॑पे॒द् वरु॑णं॒ ॅवरु॑णं ॅवपे॒न् निर् णिर् व॑पे॒द् वरु॑णम् । \newline
45. व॒पे॒द् वरु॑णं॒ ॅवरु॑णं ॅवपेद् वपे॒द् वरु॑ण मे॒वैव वरु॑णं ॅवपेद् वपे॒द् वरु॑ण मे॒व । \newline
46. वरु॑ण मे॒वैव वरु॑णं॒ ॅवरु॑ण मे॒व स्वेन॒ स्वेनै॒व वरु॑णं॒ ॅवरु॑ण मे॒व स्वेन॑ । \newline
47. ए॒व स्वेन॒ स्वेनै॒वैव स्वेन॑ भाग॒धेये॑न भाग॒धेये॑न॒ स्वेनै॒वैव स्वेन॑ भाग॒धेये॑न । \newline
48. स्वेन॑ भाग॒धेये॑न भाग॒धेये॑न॒ स्वेन॒ स्वेन॑ भाग॒धेये॒नोपोप॑ भाग॒धेये॑न॒ स्वेन॒ स्वेन॑ भाग॒धेये॒नोप॑ । \newline
49. भा॒ग॒धेये॒नोपोप॑ भाग॒धेये॑न भाग॒धेये॒नोप॑ धावति धाव॒ त्युप॑ भाग॒धेये॑न भाग॒धेये॒नोप॑ धावति । \newline
50. भा॒ग॒धेये॒नेति॑ भाग - धेये॑न । \newline
51. उप॑ धावति धाव॒ त्युपोप॑ धावति॒ स स धा॑व॒ त्युपोप॑ धावति॒ सः । \newline
52. धा॒व॒ति॒ स स धा॑वति धावति॒ स ए॒वैव स धा॑वति धावति॒ स ए॒व । \newline
53. स ए॒वैव स स ए॒वैन॑ मेन मे॒व स स ए॒वैन᳚म् । \newline
54. ए॒वैन॑ मेन मे॒वैवैनं॑ ॅवरुणपा॒शाद् व॑रुणपा॒शा दे॑न मे॒वैवैनं॑ ॅवरुणपा॒शात् । \newline
55. ए॒नं॒ ॅव॒रु॒ण॒पा॒शाद् व॑रुणपा॒शादे॑न मेनं ॅवरुणपा॒शान् मु॑ञ्चति मुञ्चति वरुणपा॒शा दे॑न मेनं ॅवरुणपा॒शान् मु॑ञ्चति । \newline
56. व॒रु॒ण॒पा॒शान् मु॑ञ्चति मुञ्चति वरुणपा॒शाद् व॑रुणपा॒शान् मु॑ञ्चति॒ चतु॑ष्कपाला॒ श्चतु॑ष्कपाला मुञ्चति वरुणपा॒शाद् व॑रुणपा॒शान् मु॑ञ्चति॒ चतु॑ष्कपालाः । \newline
57. व॒रु॒ण॒पा॒शादिति॑ वरुण - पा॒शात् । \newline
58. मु॒ञ्च॒ति॒ चतु॑ष्कपाला॒ श्चतु॑ष्कपाला मुञ्चति मुञ्चति॒ चतु॑ष्कपाला भवन्ति भवन्ति॒ चतु॑ष्कपाला मुञ्चति मुञ्चति॒ चतु॑ष्कपाला भवन्ति । \newline
\pagebreak
\markright{ TS 2.3.12.2  \hfill https://www.vedavms.in \hfill}

\section{ TS 2.3.12.2 }

\textbf{TS 2.3.12.2 } \newline
\textbf{Samhita Paata} \newline

चतु॑ष्कपाला भवन्ति॒ चतु॑ष्पा॒द्ध्यश्वः॒ समृ॑द्ध्या॒ एक॒मति॑रिक्तं॒ निर्व॑पे॒द्-यमे॒व प्र॑तिग्रा॒ही भव॑ति॒ यं ॅवा॒ नाद्ध्येति॒ तस्मा॑दे॒व व॑रुणपा॒शान् मु॑च्यते॒ यद्यप॑रं प्रतिग्रा॒ही स्याथ् सौ॒र्यमेक॑कपाल॒मनु॒ निर्व॑पेद॒मुमे॒वा ऽऽदि॒त्यमु॑च्चा॒रं कु॑रुते॒ ऽपो॑ऽवभृ॒थमवै᳚त्य॒फ्सु वै वरु॑णः सा॒क्षादे॒व वरु॑ण॒मव॑ यजते ऽपोन॒प्त्रीयं॑ च॒रुं ( ) पुन॒रेत्य॒ निर्व॑पेद॒फ्सु यो॑नि॒र्वा अश्वः॒ स्वामे॒वैनं॒ॅयोनिं॑ गमयति॒ स ए॑नꣳ शा॒न्त उप॑ तिष्ठते ॥ \newline

\textbf{Pada Paata} \newline

चतु॑ष्कपाला॒ इति॒ चतुः॑ - क॒पा॒लाः॒ । भ॒व॒न्ति॒ । चतु॑ष्पा॒दिति॒ चतुः॑ - पा॒त् । हि । अश्वः॑ । समृ॑द्ध्या॒ इति॒ सं - ऋ॒द्ध्यै॒ । एक᳚म् । अति॑रिक्त॒मित्यति॑ - रि॒क्त॒म् । निरिति॑ । व॒पे॒त् । यम् । ए॒व । प्र॒ति॒ग्रा॒हीति॑ प्रति - ग्रा॒ही । भव॑ति । यम् । वा॒ । न । अ॒द्ध्येतीत्य॑धि - एति॑ । तस्मा᳚त् । ए॒व । व॒रु॒ण॒पा॒शादिति॑ वरुण - पा॒शात् ।   मु॒च्य॒ते॒ । यदि॑ । अप॑रम् । प्र॒ति॒ग्रा॒हीति॑ प्रति -  ग्रा॒ही । स्यात् । सौ॒र्यम् । एक॑कपाल॒मित्येक॑ - क॒पा॒ल॒म् । अनु॑ । निरिति॑ । व॒पे॒त् । अ॒मुम् । ए॒व । आ॒दि॒त्यम् । उ॒च्चा॒रमित्यु॑त् - चा॒रम् । कु॒रु॒ते॒ । अ॒पः । अ॒व॒भृ॒थमित्य॑व -भृ॒थम् । अवेति॑ । ए॒ति॒ । अ॒फ्स्वित्य॑प् - सु । वै । वरु॑णः । सा॒क्षादिति॑ स-अ॒क्षात् । ए॒व । वरु॑णम् । अवेति॑ । य॒ज॒ते॒ । अ॒पो॒न॒प्त्रीय॒मित्य॑पः - न॒प्त्रीय᳚म् । च॒रुम् ( ) । पुनः॑ । एत्येत्या᳚ - इत्य॑ । निरिति॑ । व॒पे॒त् । अ॒फ्सुयो॑नि॒रित्य॒फ्सु - यो॒निः॒ । वै । अश्वः॑ । स्वाम् । ए॒व । ए॒न॒म् । योनि᳚म् । ग॒म॒य॒ति॒ । सः । ए॒न॒म् । शा॒न्तः । उपेति॑ । ति॒ष्ठ॒ते॒ ॥  \newline


\textbf{Krama Paata} \newline

चतु॑ष्कपाला भवन्ति । चतु॑ष्कपाला॒ इति॒ चतुः॑ - क॒पा॒लाः॒ । भ॒व॒न्ति॒ चतु॑ष्पात् । चतु॑ष्पा॒द्धि । चतु॑ष्पा॒दिति॒ चतुः॑ - पा॒त्॒ । ह्यश्वः॑ । अश्वः॒ समृ॑द्ध्यै । सम॑द्ध्या॒ एक᳚म् । समृ॑द्ध्या॒ इति॒ सं - ऋ॒द्ध्यै॒ । एक॒मति॑रिक्तम् । अति॑रिक्त॒म् निः । अति॑रिक्त॒मित्यति॑ - रि॒क्त॒म् । निर् व॑पेत् । व॒पे॒द् यम् । यमे॒व । ए॒व प्र॑तिग्रा॒ही । प्र॒ति॒ग्रा॒ही भव॑ति । प्र॒ति॒ग्रा॒हीति॑ प्रति - ग्रा॒ही । भव॑ति॒ यम् । यं ॅवा᳚ । वा॒ न । नाद्ध्येति॑ । अ॒द्ध्येति॒ तस्मा᳚त् । अ॒द्ध्येतीत्य॑धि - एति॑ । तस्मा॑दे॒व । ए॒व व॑रुणपा॒शात् । व॒रु॒ण॒पा॒शान् मु॑च्यते । व॒रु॒ण॒पा॒शादिति॑ वरुण - पा॒शात् । मु॒च्य॒ते॒ यदि॑ । यद्यप॑रम् । अप॑रम् प्रतिग्रा॒ही । प्र॒ति॒ग्रा॒ही स्यात् । प्र॒ति॒ग्रा॒हीति॑ प्रति - ग्रा॒ही । स्याथ् सौ॒र्यम् । सौ॒र्यमेक॑कपालम् । एक॑कपाल॒मनु॑ । एक॑कपाल॒मित्येक॑ - क॒पा॒ल॒म् । अनु॒ निः । निर् व॑पेत् । व॒पे॒द॒मुम् । अ॒मुमे॒व । ए॒वादि॒त्यम् । आ॒दि॒त्यमु॑च्चा॒रम् । उ॒च्चा॒रम् कु॑रुते । उ॒च्चा॒रमित्यु॑त् - चा॒रम् । कु॒रु॒ते॒ ऽपः । अ॒पो॑ ऽवभृ॒थम् । अ॒व॒भृ॒थमव॑ । अ॒व॒भृ॒थमित्य॑व - भृ॒थम् । अवै॑ति । ए॒त्य॒फ्सु । अ॒फ्सु वै । अ॒फ्स्वित्य॑प् - सु । वै वरु॑णः । वरु॑णः सा॒क्षात् । सा॒क्षादे॒व । सा॒क्षादिति॑ स - अ॒क्षात् । ए॒व वरु॑णम् । वरु॑ण॒मव॑ । अव॑ यजते । य॒ज॒ते॒ ऽपो॒न॒प्त्रीय᳚म् । अ॒पो॒न॒प्त्रीय॑म् च॒रुम् ( ) । अ॒पो॒न॒प्त्रीय॒मित्य॑पः - न॒प्त्रीय᳚म् । च॒रुम् पुनः॑ । पुन॒रेत्य॑ । एत्य॒ निः । एत्येत्या᳚ - इत्य॑ । निर् व॑पेत् । व॒पे॒द॒फ्सुयो॑निः । अ॒फ्सुयो॑नि॒र् वै । अ॒फ्सुयो॑नि॒रित्य॒फ्सु - यो॒निः॒ । वा अश्वः॑ । अश्वः॒ स्वाम् । स्वामे॒व । ए॒वैन᳚म् । ए॒नं॒ ॅयोनि᳚म् । योनिं॑ गमयति । ग॒म॒य॒ति॒ सः । स ए॑नम् । ए॒नꣳ॒॒ शा॒न्तः । शा॒न्त उप॑ । उप॑ तिष्ठते । ति॒ष्ठ॒त॒ इति॑ तिष्ठते । \newline

\textbf{Jatai Paata} \newline

1. चतु॑ष्कपाला भवन्ति भवन्ति॒ चतु॑ष्कपाला॒ श्चतु॑ष्कपाला भवन्ति । \newline
2. चतु॑ष्कपाला॒ इति॒ चतुः॑ - क॒पा॒लाः॒ । \newline
3. भ॒व॒न्ति॒ चतु॑ष्पा॒च् चतु॑ष्पाद् भवन्ति भवन्ति॒ चतु॑ष्पात् । \newline
4. चतु॑ष्पा॒ द्धि हि चतु॑ष्पा॒च् चतु॑ष्पा॒ द्धि । \newline
5. चतु॑ष्पा॒दिति॒ चतुः॑ - पा॒त् । \newline
6. ह्यश्वो ऽश्वो॒ हि ह्यश्वः॑ । \newline
7. अश्वः॒ समृ॑द्ध्यै॒ समृ॑द्ध्या॒ अश्वो ऽश्वः॒ समृ॑द्ध्यै । \newline
8. समृ॑द्ध्या॒ एक॒ मेकꣳ॒॒ समृ॑द्ध्यै॒ समृ॑द्ध्या॒ एक᳚म् । \newline
9. समृ॑द्ध्या॒ इति॒ सं - ऋ॒द्ध्यै॒ । \newline
10. एक॒ मति॑रिक्त॒ मति॑रिक्त॒ मेक॒ मेक॒ मति॑रिक्तम् । \newline
11. अति॑रिक्त॒म् निर् णिरति॑रिक्त॒ मति॑रिक्त॒म् निः । \newline
12. अति॑रिक्त॒मित्यति॑ - रि॒क्त॒म् । \newline
13. निर् व॑पेद् वपे॒न् निर् णिर् व॑पेत् । \newline
14. व॒पे॒द् यं ॅयं ॅव॑पेद् वपे॒द् यम् । \newline
15. य मे॒वैव यं ॅय मे॒व । \newline
16. ए॒व प्र॑तिग्रा॒ही प्र॑तिग्रा॒ह्ये॑वैव प्र॑तिग्रा॒ही । \newline
17. प्र॒ति॒ग्रा॒ही भव॑ति॒ भव॑ति प्रतिग्रा॒ही प्र॑तिग्रा॒ही भव॑ति । \newline
18. प्र॒ति॒ग्रा॒हीति॑ प्रति - ग्रा॒ही । \newline
19. भव॑ति॒ यं ॅयम् भव॑ति॒ भव॑ति॒ यम् । \newline
20. यं ॅवा॑ वा॒ यं ॅयं ॅवा᳚ । \newline
21. वा॒ न न वा॑ वा॒ न । \newline
22. नाद्ध्ये त्य॒द्ध्येति॒ न नाद्ध्येति॑ । \newline
23. अ॒द्ध्येति॒ तस्मा॒त् तस्मा॑ द॒द्ध्ये त्य॒द्ध्येति॒ तस्मा᳚त् । \newline
24. अ॒द्ध्येतीत्य॑धि - एति॑ । \newline
25. तस्मा॑ दे॒वैव तस्मा॒त् तस्मा॑ दे॒व । \newline
26. ए॒व व॑रुणपा॒शाद् व॑रुणपा॒शा दे॒वैव व॑रुणपा॒शात् । \newline
27. व॒रु॒ण॒पा॒शान् मु॑च्यते मुच्यते वरुणपा॒शाद् व॑रुणपा॒शान् मु॑च्यते । \newline
28. व॒रु॒ण॒पा॒शादिति॑ वरुण - पा॒शात् । \newline
29. मु॒च्य॒ते॒ यदि॒ यदि॑ मुच्यते मुच्यते॒ यदि॑ । \newline
30. यद्यप॑र॒ मप॑रं॒ ॅयदि॒ यद्यप॑रम् । \newline
31. अप॑रम् प्रतिग्रा॒ही प्र॑तिग्रा॒ह्यप॑र॒ मप॑रम् प्रतिग्रा॒ही । \newline
32. प्र॒ति॒ग्रा॒ही स्याथ् स्यात् प्र॑तिग्रा॒ही प्र॑तिग्रा॒ही स्यात् । \newline
33. प्र॒ति॒ग्रा॒हीति॑ प्रति - ग्रा॒ही । \newline
34. स्याथ् सौ॒र्यꣳ सौ॒र्यꣳ स्याथ् स्याथ् सौ॒र्यम् । \newline
35. सौ॒र्य मेक॑कपाल॒ मेक॑कपालꣳ सौ॒र्यꣳ सौ॒र्य मेक॑कपालम् । \newline
36. एक॑कपाल॒ मन्वन्वेक॑कपाल॒ मेक॑कपाल॒ मनु॑ । \newline
37. एक॑कपाल॒मित्येक॑ - क॒पा॒ल॒म् । \newline
38. अनु॒ निर् णि रन्वनु॒ निः । \newline
39. निर् व॑पेद् वपे॒न् निर् णिर् व॑पेत् । \newline
40. व॒पे॒ द॒मु म॒मुं ॅव॑पेद् वपे द॒मुम् । \newline
41. अ॒मु मे॒वैवामु म॒मु मे॒व । \newline
42. ए॒वादि॒त्य मा॑दि॒त्य मे॒वैवादि॒त्यम् । \newline
43. आ॒दि॒त्य मु॑च्चा॒र मु॑च्चा॒र मा॑दि॒त्य मा॑दि॒त्य मु॑च्चा॒रम् । \newline
44. उ॒च्चा॒रम् कु॑रुते कुरुत उच्चा॒र मु॑च्चा॒रम् कु॑रुते । \newline
45. उ॒च्चा॒रमित्यु॑त् - चा॒रम् । \newline
46. कु॒रु॒ते॒ ऽपो॑ ऽपः कु॑रुते कुरुते॒ ऽपः । \newline
47. अ॒पो॑ ऽवभृ॒थ म॑वभृ॒थ म॒पो᳚(1॒) ऽपो॑ ऽवभृ॒थम् । \newline
48. अ॒व॒भृ॒थ मवावा॑ वभृ॒थ म॑वभृ॒थ मव॑ । \newline
49. अ॒व॒भृ॒थमित्य॑व - भृ॒थम् । \newline
50. अवै᳚ त्ये॒त्यवा वै॑ति । \newline
51. ए॒त्य॒ फ्स्वा᳚(1॒)फ्स्वे᳚ त्येत्य॒फ्सु । \newline
52. अ॒फ्सु वै वा अ॒फ्स्व॑फ्सु वै । \newline
53. अ॒फ्स्वित्य॑प् - सु । \newline
54. वै वरु॑णो॒ वरु॑णो॒ वै वै वरु॑णः । \newline
55. वरु॑णः सा॒क्षाथ् सा॒क्षाद् वरु॑णो॒ वरु॑णः सा॒क्षात् । \newline
56. सा॒क्षा दे॒वैव सा॒क्षाथ् सा॒क्षा दे॒व । \newline
57. सा॒क्षादिति॑ स - अ॒क्षात् । \newline
58. ए॒व वरु॑णं॒ ॅवरु॑ण मे॒वैव वरु॑णम् । \newline
59. वरु॑ण॒ मवाव॒ वरु॑णं॒ ॅवरु॑ण॒ मव॑ । \newline
60. अव॑ यजते यज॒ते ऽवाव॑ यजते । \newline
61. य॒ज॒ते॒ ऽपो॒न॒प्त्रीय॑ मपोन॒प्त्रीयं॑ ॅयजते यजते ऽपोन॒प्त्रीय᳚म् । \newline
62. अ॒पो॒न॒प्त्रीय॑म् च॒रुम् च॒रु म॑पोन॒प्त्रीय॑ मपोन॒प्त्रीय॑म् च॒रुम् । \newline
63. अ॒पो॒न॒प्त्रीय॒मित्य॑पः - न॒प्त्रीय᳚म् । \newline
64. च॒रुम् पुनः॒ पुन॑ श्च॒रुम् च॒रुम् पुनः॑ । \newline
65. पुन॒ रेत्ये त्यैत्येत्य॒ पुनः॒ पुन॒ रेत्येत्य॑ । \newline
66. एत्येत्य॒ निर् णिरेत्ये त्यैत्येत्य॒ निः । \newline
67. एत्येत्या᳚ - इत्य॑ । \newline
68. निर् व॑पेद् वपे॒न् निर् णिर् व॑पेत् । \newline
69. व॒पे ॒द॒फ्सुयो॑नि र॒फ्सुयो॑निर् वपेद् वपे द॒फ्सुयो॑निः । \newline
70. अ॒फ्सुयो॑नि॒र् वै वा अ॒फ्सुयो॑नि र॒फ्सुयो॑नि॒र् वै । \newline
71. अ॒फ्सुयो॑नि॒रित्य॒फ्सु - यो॒निः॒ । \newline
72. वा अश्वो ऽश्वो॒ वै वा अश्वः॑ । \newline
73. अश्वः॒ स्वाꣳ स्वा मश्वो ऽश्वः॒ स्वाम् । \newline
74. स्वा मे॒वैव स्वाꣳ स्वा मे॒व । \newline
75. ए॒वैन॑ मेन मे॒वैवैन᳚म् । \newline
76. ए॒नं॒ ॅयोनिं॒ ॅयोनि॑ मेन मेनं॒ ॅयोनि᳚म् । \newline
77. योनि॑म् गमयति गमयति॒ योनिं॒ ॅयोनि॑म् गमयति । \newline
78. ग॒म॒य॒ति॒ स स ग॑मयति गमयति॒ सः । \newline
79. स ए॑न मेनꣳ॒॒ स स ए॑नम् । \newline
80. ए॒नꣳ॒॒ शा॒न्तः शा॒न्त ए॑न मेनꣳ शा॒न्तः । \newline
81. शा॒न्त उपोप॑ शा॒न्तः शा॒न्त उप॑ । \newline
82. उप॑ तिष्ठते तिष्ठत॒ उपोप॑ तिष्ठते । \newline
83. ति॒ष्ठ॒त॒ इति॑ तिष्ठते । \newline

\textbf{Ghana Paata } \newline

1. चतु॑ष्कपाला भवन्ति भवन्ति॒ चतु॑ष्कपाला॒ श्चतु॑ष्कपाला भवन्ति॒ चतु॑ष्पा॒च् चतु॑ष्पाद् भवन्ति॒ चतु॑ष्कपाला॒ श्चतु॑ष्कपाला भवन्ति॒ चतु॑ष्पात् । \newline
2. चतु॑ष्कपाला॒ इति॒ चतुः॑ - क॒पा॒लाः॒ । \newline
3. भ॒व॒न्ति॒ चतु॑ष्पा॒च् चतु॑ष्पाद् भवन्ति भवन्ति॒ चतु॑ष्पा॒ द्धि हि चतु॑ष्पाद् भवन्ति भवन्ति॒ चतु॑ष्पा॒ द्धि । \newline
4. चतु॑ष्पा॒ द्धि हि चतु॑ष्पा॒च् चतु॑ष्पा॒ द्ध्यश्वो ऽश्वो॒ हि चतु॑ष्पा॒च् चतु॑ष्पा॒ द्ध्यश्वः॑ । \newline
5. चतु॑ष्पा॒दिति॒ चतुः॑ - पा॒त् । \newline
6. ह्यश्वो ऽश्वो॒ हि ह्यश्वः॒ समृ॑द्ध्यै॒ समृ॑द्ध्या॒ अश्वो॒ हि ह्यश्वः॒ समृ॑द्ध्यै । \newline
7. अश्वः॒ समृ॑द्ध्यै॒ समृ॑द्ध्या॒ अश्वो ऽश्वः॒ समृ॑द्ध्या॒ एक॒ मेकꣳ॒॒ समृ॑द्ध्या॒ अश्वो ऽश्वः॒ समृ॑द्ध्या॒ एक᳚म् । \newline
8. समृ॑द्ध्या॒ एक॒ मेकꣳ॒॒ समृ॑द्ध्यै॒ समृ॑द्ध्या॒ एक॒ मति॑रिक्त॒ मति॑रिक्त॒ मेकꣳ॒॒ समृ॑द्ध्यै॒ समृ॑द्ध्या॒ एक॒ मति॑रिक्तम् । \newline
9. समृ॑द्ध्या॒ इति॒ सं - ऋ॒द्ध्यै॒ । \newline
10. एक॒ मति॑रिक्त॒ मति॑रिक्त॒ मेक॒ मेक॒ मति॑रिक्त॒म् निर् णिरति॑रिक्त॒ मेक॒ मेक॒ मति॑रिक्त॒म् निः । \newline
11. अति॑रिक्त॒म् निर् णिरति॑रिक्त॒ मति॑रिक्त॒म् निर् व॑पेद् वपे॒न् निरति॑रिक्त॒ मति॑रिक्त॒म् निर् व॑पेत् । \newline
12. अति॑रिक्त॒मित्यति॑ - रि॒क्त॒म् । \newline
13. निर् व॑पेद् वपे॒न् निर् णिर् व॑पे॒द् यं ॅयं ॅव॑पे॒न् निर् णिर् व॑पे॒द् यम् । \newline
14. व॒पे॒द् यं ॅयं ॅव॑पेद् वपे॒द् य मे॒वैव यं ॅव॑पेद् वपे॒द् य मे॒व । \newline
15. य मे॒वैव यं ॅय मे॒व प्र॑तिग्रा॒ही प्र॑तिग्रा॒ह्ये॑व यं ॅय मे॒व प्र॑तिग्रा॒ही । \newline
16. ए॒व प्र॑तिग्रा॒ही प्र॑तिग्रा॒ह्ये॑वैव प्र॑तिग्रा॒ही भव॑ति॒ भव॑ति प्रतिग्रा॒ह्ये॑वैव प्र॑तिग्रा॒ही भव॑ति । \newline
17. प्र॒ति॒ग्रा॒ही भव॑ति॒ भव॑ति प्रतिग्रा॒ही प्र॑तिग्रा॒ही भव॑ति॒ यं ॅयम् भव॑ति प्रतिग्रा॒ही प्र॑तिग्रा॒ही भव॑ति॒ यम् । \newline
18. प्र॒ति॒ग्रा॒हीति॑ प्रति - ग्रा॒ही । \newline
19. भव॑ति॒ यं ॅयम् भव॑ति॒ भव॑ति॒ यं ॅवा॑ वा॒ यम् भव॑ति॒ भव॑ति॒ यं ॅवा᳚ । \newline
20. यं ॅवा॑ वा॒ यं ॅयं ॅवा॒ न न वा॒ यं ॅयं ॅवा॒ न । \newline
21. वा॒ न न वा॑ वा॒ नाद्ध्ये त्य॒द्ध्येति॒ न वा॑ वा॒ नाद्ध्येति॑ । \newline
22. नाद्ध्ये त्य॒द्ध्येति॒ न नाद्ध्येति॒ तस्मा॒त् तस्मा॑ द॒द्ध्येति॒ न नाद्ध्येति॒ तस्मा᳚त् । \newline
23. अ॒द्ध्येति॒ तस्मा॒त् तस्मा॑ द॒द्ध्ये त्य॒द्ध्येति॒ तस्मा॑ दे॒वैव तस्मा॑द॒द्ध्ये त्य॒द्ध्येति॒ तस्मा॑दे॒व । \newline
24. अ॒द्ध्येतीत्य॑धि - एति॑ । \newline
25. तस्मा॑ दे॒वैव तस्मा॒त् तस्मा॑दे॒व व॑रुणपा॒शाद् व॑रुणपा॒शा दे॒व तस्मा॒त् तस्मा॑ दे॒व व॑रुणपा॒शात् । \newline
26. ए॒व व॑रुणपा॒शाद् व॑रुणपा॒शा दे॒वैव व॑रुणपा॒शान् मु॑च्यते मुच्यते वरुणपा॒शा दे॒वैव व॑रुणपा॒शान् मु॑च्यते । \newline
27. व॒रु॒ण॒पा॒शान् मु॑च्यते मुच्यते वरुणपा॒शाद् व॑रुणपा॒शान् मु॑च्यते॒ यदि॒ यदि॑ मुच्यते वरुणपा॒शाद् व॑रुणपा॒शान् मु॑च्यते॒ यदि॑ । \newline
28. व॒रु॒ण॒पा॒शादिति॑ वरुण - पा॒शात् । \newline
29. मु॒च्य॒ते॒ यदि॒ यदि॑ मुच्यते मुच्यते॒ यद्यप॑र॒ मप॑रं॒ ॅयदि॑ मुच्यते मुच्यते॒ यद्यप॑रम् । \newline
30. यद्यप॑र॒ मप॑रं॒ ॅयदि॒ यद्यप॑रम् प्रतिग्रा॒ही प्र॑तिग्रा॒ह्यप॑रं॒ ॅयदि॒ यद्यप॑रम् प्रतिग्रा॒ही । \newline
31. अप॑रम् प्रतिग्रा॒ही प्र॑तिग्रा॒ह्यप॑र॒ मप॑रम् प्रतिग्रा॒ही स्याथ् स्यात् प्र॑तिग्रा॒ह्यप॑र॒ मप॑रम् प्रतिग्रा॒ही स्यात् । \newline
32. प्र॒ति॒ग्रा॒ही स्याथ् स्यात् प्र॑तिग्रा॒ही प्र॑तिग्रा॒ही स्याथ् सौ॒र्यꣳ सौ॒र्यꣳ स्यात् प्र॑तिग्रा॒ही प्र॑तिग्रा॒ही स्याथ् सौ॒र्यम् । \newline
33. प्र॒ति॒ग्रा॒हीति॑ प्रति - ग्रा॒ही । \newline
34. स्याथ् सौ॒र्यꣳ सौ॒र्यꣳ स्याथ् स्याथ् सौ॒र्य मेक॑कपाल॒ मेक॑कपालꣳ सौ॒र्यꣳ स्याथ् स्याथ् सौ॒र्य मेक॑कपालम् । \newline
35. सौ॒र्य मेक॑कपाल॒ मेक॑कपालꣳ सौ॒र्यꣳ सौ॒र्य मेक॑कपाल॒ मन्व न्वेक॑कपालꣳ सौ॒र्यꣳ सौ॒र्य मेक॑कपाल॒ मनु॑ । \newline
36. एक॑कपाल॒ मन्वन्वेक॑कपाल॒ मेक॑कपाल॒ मनु॒ निर् णिरन्वेक॑कपाल॒ मेक॑कपाल॒ मनु॒ निः । \newline
37. एक॑कपाल॒मित्येक॑ - क॒पा॒ल॒म् । \newline
38. अनु॒ निर् णि रन्वनु॒ निर् व॑पेद् वपे॒न् नि रन्वनु॒ निर् व॑पेत् । \newline
39. निर् व॑पेद् वपे॒न् निर् णिर् व॑पेद॒मु म॒मुं ॅव॑पे॒न् निर् णिर् व॑पेद॒मुम् । \newline
40. व॒पे॒द॒मु म॒मुं ॅव॑पेद् वपेद॒मु मे॒वैवामुं ॅव॑पेद् वपेद॒मु मे॒व । \newline
41. अ॒मु मे॒वैवामु म॒मु मे॒वादि॒त्य मा॑दि॒त्य मे॒वामु म॒मु मे॒वादि॒त्यम् । \newline
42. ए॒वादि॒त्य मा॑दि॒त्य मे॒वैवादि॒त्य मु॑च्चा॒र मु॑च्चा॒र मा॑दि॒त्य मे॒वैवादि॒त्य मु॑च्चा॒रम् । \newline
43. आ॒दि॒त्य मु॑च्चा॒र मु॑च्चा॒र मा॑दि॒त्य मा॑दि॒त्य मु॑च्चा॒रम् कु॑रुते कुरुत उच्चा॒र मा॑दि॒त्य मा॑दि॒त्य मु॑च्चा॒रम् कु॑रुते । \newline
44. उ॒च्चा॒रम् कु॑रुते कुरुत उच्चा॒र मु॑च्चा॒रम् कु॑रुते॒ ऽपो॑ ऽपः कु॑रुत उच्चा॒र मु॑च्चा॒रम् कु॑रुते॒ ऽपः । \newline
45. उ॒च्चा॒रमित्यु॑त् - चा॒रम् । \newline
46. कु॒रु॒ते॒ ऽपो॑ ऽपः कु॑रुते कुरुते॒ ऽपो॑ ऽवभृ॒थ म॑वभृ॒थ म॒पः कु॑रुते कुरुते॒ ऽपो॑ ऽवभृ॒थम् । \newline
47. अ॒पो॑ ऽवभृ॒थ म॑वभृ॒थ म॒पो᳚(1॒) ऽपो॑ ऽवभृ॒थ मवावा॑वभृ॒थ म॒पो᳚(1॒) ऽपो॑ ऽवभृ॒थ मव॑ । \newline
48. अ॒व॒भृ॒थ मवावा॑वभृ॒थ म॑वभृ॒थ मवै᳚ त्ये॒त्यवा॑ वभृ॒थ म॑वभृ॒थ मवै॑ति । \newline
49. अ॒व॒भृ॒थमित्य॑व - भृ॒थम् । \newline
50. अवै᳚त्ये॒ त्यवावै᳚ त्य॒फ्स्वा᳚(1॒) फ्स्वे᳚ त्यवावै᳚ त्य॒फ्सु । \newline
51. ए॒त्य॒फ्स्वा᳚(1॒) फ्स्वे᳚ त्येत्य॒फ्सु वै वा अ॒फ्स्वे᳚त्ये त्य॒फ्सु वै । \newline
52. अ॒फ्सु वै वा अ॒फ्स्व॑फ्सु वै वरु॑णो॒ वरु॑णो॒ वा अ॒फ्स्व॑फ्सु वै वरु॑णः । \newline
53. अ॒फ्स्वित्य॑प् - सु । \newline
54. वै वरु॑णो॒ वरु॑णो॒ वै वै वरु॑णः सा॒क्षाथ् सा॒क्षाद् वरु॑णो॒ वै वै वरु॑णः सा॒क्षात् । \newline
55. वरु॑णः सा॒क्षाथ् सा॒क्षाद् वरु॑णो॒ वरु॑णः सा॒क्षा दे॒वैव सा॒क्षाद् वरु॑णो॒ वरु॑णः सा॒क्षा दे॒व । \newline
56. सा॒क्षा दे॒वैव सा॒क्षाथ् सा॒क्षा दे॒व वरु॑णं॒ ॅवरु॑ण मे॒व सा॒क्षाथ् सा॒क्षादे॒व वरु॑णम् । \newline
57. सा॒क्षादिति॑ स - अ॒क्षात् । \newline
58. ए॒व वरु॑णं॒ ॅवरु॑ण मे॒वैव वरु॑ण॒ मवाव॒ वरु॑ण मे॒वैव वरु॑ण॒ मव॑ । \newline
59. वरु॑ण॒ मवाव॒ वरु॑णं॒ ॅवरु॑ण॒ मव॑ यजते यज॒ते ऽव॒ वरु॑णं॒ ॅवरु॑ण॒ मव॑ यजते । \newline
60. अव॑ यजते यज॒ते ऽवाव॑ यजते ऽपोन॒प्त्रीय॑ मपोन॒प्त्रीयं॑ ॅयज॒ते ऽवाव॑ यजते ऽपोन॒प्त्रीय᳚म् । \newline
61. य॒ज॒ते॒ ऽपो॒न॒प्त्रीय॑ मपोन॒प्त्रीयं॑ ॅयजते यजते ऽपोन॒प्त्रीय॑म् च॒रुम् च॒रु म॑पोन॒प्त्रीयं॑ ॅयजते यजते ऽपोन॒प्त्रीय॑म् च॒रुम् । \newline
62. अ॒पो॒न॒प्त्रीय॑म् च॒रुम् च॒रु म॑पोन॒प्त्रीय॑ मपोन॒प्त्रीय॑म् च॒रुम् पुनः॒ पुन॑श्च॒रु म॑पोन॒प्त्रीय॑ मपोन॒प्त्रीय॑म् च॒रुम् पुनः॑ । \newline
63. अ॒पो॒न॒प्त्रीय॒मित्य॑पः - न॒प्त्रीय᳚म् । \newline
64. च॒रुम् पुनः॒ पुन॑श्च॒रुम् च॒रुम् पुन॒ रेत्ये त्यैत्येत्य॒ पुन॑श्च॒रुम् च॒रुम् पुन॒ रेत्येत्य॑ । \newline
65. पुन॒ रेत्येत्यैत्येत्य॒ पुनः॒ पुन॒ रेत्येत्य॒ निर् णिरेत्येत्य॒ पुनः॒ पुन॒ रेत्येत्य॒ निः । \newline
66. एत्येत्य॒ निर् णिरेत्ये त्यैत्येत्य॒ निर् व॑पेद् वपे॒न् निरेत्ये त्यैत्येत्य॒ निर् व॑पेत् । \newline
67. एत्येत्या᳚ - इत्य॑ । \newline
68. निर् व॑पेद् वपे॒न् निर् णिर् व॑पे द॒फ्सुयो॑नि र॒फ्सुयो॑निर् वपे॒न् निर् णिर् व॑पे द॒फ्सुयो॑निः । \newline
69. व॒पे॒द॒फ्सुयो॑नि र॒फ्सुयो॑निर् वपेद् वपे द॒फ्सुयो॑नि॒र् वै वा अ॒फ्सुयो॑निर् वपेद् वपे द॒फ्सुयो॑नि॒र् वै । \newline
70. अ॒फ्सुयो॑नि॒र् वै वा अ॒फ्सुयो॑नि र॒फ्सुयो॑नि॒र् वा अश्वो ऽश्वो॒ वा अ॒फ्सुयो॑नि र॒फ्सुयो॑नि॒र् वा अश्वः॑ । \newline
71. अ॒फ्सुयो॑नि॒रित्य॒फ्सु - यो॒निः॒ । \newline
72. वा अश्वो ऽश्वो॒ वै वा अश्वः॒ स्वाꣳ स्वा मश्वो॒ वै वा अश्वः॒ स्वाम् । \newline
73. अश्वः॒ स्वाꣳ स्वा मश्वो ऽश्वः॒ स्वा मे॒वैव स्वा मश्वो ऽश्वः॒ स्वा मे॒व । \newline
74. स्वा मे॒वैव स्वाꣳ स्वा मे॒वैन॑ मेन मे॒व स्वाꣳ स्वा मे॒वैन᳚म् । \newline
75. ए॒वैन॑ मेन मे॒वैवैनं॒ ॅयोनिं॒ ॅयोनि॑ मेन मे॒वैवैनं॒ ॅयोनि᳚म् । \newline
76. ए॒नं॒ ॅयोनिं॒ ॅयोनि॑ मेन मेनं॒ ॅयोनि॑म् गमयति गमयति॒ योनि॑ मेन मेनं॒ ॅयोनि॑म् गमयति । \newline
77. योनि॑म् गमयति गमयति॒ योनिं॒ ॅयोनि॑म् गमयति॒ स स ग॑मयति॒ योनिं॒ ॅयोनि॑म् गमयति॒ सः । \newline
78. ग॒म॒य॒ति॒ स स ग॑मयति गमयति॒ स ए॑न मेनꣳ॒॒ स ग॑मयति गमयति॒ स ए॑नम् । \newline
79. स ए॑न मेनꣳ॒॒ स स ए॑नꣳ शा॒न्तः शा॒न्त ए॑नꣳ॒॒ स स ए॑नꣳ शा॒न्तः । \newline
80. ए॒नꣳ॒॒ शा॒न्तः शा॒न्त ए॑न मेनꣳ शा॒न्त उपोप॑ शा॒न्त ए॑न मेनꣳ शा॒न्त उप॑ । \newline
81. शा॒न्त उपोप॑ शा॒न्तः शा॒न्त उप॑ तिष्ठते तिष्ठत॒ उप॑ शा॒न्तः शा॒न्त उप॑ तिष्ठते । \newline
82. उप॑ तिष्ठते तिष्ठत॒ उपोप॑ तिष्ठते । \newline
83. ति॒ष्ठ॒त॒ इति॑ तिष्ठते । \newline
\pagebreak
\markright{ TS 2.3.13.1  \hfill https://www.vedavms.in \hfill}

\section{ TS 2.3.13.1 }

\textbf{TS 2.3.13.1 } \newline
\textbf{Samhita Paata} \newline

या वा॑मिन्द्रावरुणा यत॒व्या॑ त॒नूस्तये॒ममꣳ ह॑सो मुञ्चतं॒ ॅया वा॑मिन्द्रा वरुणा सह॒स्या॑ रक्ष॒स्या॑ तेज॒स्या॑ त॒नूस्तये॒ ममꣳ ह॑सो मुञ्चतं॒ ॅयो वा॑मिन्द्रा वरुणा व॒ग्नौ स्राम॒स्तं ॅवा॑ मे॒ तेना व॑यजे॒यो वा॑मिन्द्रा वरुणा द्वि॒पाथ्सु॑ प॒शुषु॒ चतु॑ष्पाथ्सु गो॒ष्ठे गृ॒हेष्व॒फ्‌स्वोष॑धीषु॒ वन॒स्पति॑षु॒ स्राम॒स्तं ॅवा॑ मे॒ तेनाव॑ यज॒ इन्द्रो॒ वा ए॒तस्ये᳚ - [  ] \newline

\textbf{Pada Paata} \newline

या । वा॒म् । इ॒न्द्रा॒व॒रु॒णेती᳚न्द्रा - व॒रु॒णा॒ । य॒त॒व्या᳚ । त॒नूः । तया᳚ । इ॒मम् । अꣳह॑सः । मु॒ञ्च॒त॒म् । या । वा॒म् । इ॒न्द्रा॒व॒रु॒णेती᳚न्द्रा - व॒रु॒णा॒ । स॒ह॒स्या᳚ । र॒क्ष॒स्या᳚ । ते॒ज॒स्या᳚ । त॒नूः । तया᳚ । इ॒मम् । अꣳह॑सः । मु॒ञ्च॒त॒म् ।  यः । वा॒म् । इ॒न्द्रा॒व॒रु॒णा॒विती᳚न्द्रा - व॒रु॒णौ॒ । अ॒ग्नौ । स्रामः॑ । तम् । वा॒म् । ए॒तेन॑ । अवेति॑ । य॒जे॒ । यः । वा॒म् । इ॒न्द्रा॒व॒रु॒णेती᳚न्द्रा - व॒रु॒णा॒ । द्वि॒पाथ्स्विति॑ द्वि॒पात् - सु॒ । प॒शुषु॑ । चतु॑ष्पा॒थ्स्विति॒ चतु॑ष्पात् - सु॒ । गो॒ष्ठ इति॑ गो- स्थे । गृ॒हेषु॑ । अ॒फ्स्वित्य॑प् - सु । ओष॑धीषु । वन॒स्पति॑षु । स्रामः॑ । तम् । वा॒म् । ए॒तेन॑ । अवेति॑ । य॒जे॒ । इन्द्रः॑ । वै । ए॒तस्य॑ ।  \newline


\textbf{Krama Paata} \newline

या वा᳚म् । वा॒मि॒न्द्रा॒व॒रु॒णा॒ । इ॒न्द्रा॒व॒रु॒णा॒ य॒त॒व्या᳚ । इ॒न्द्रा॒व॒रु॒णेती᳚न्द्रा - व॒रु॒णा॒ । य॒त॒व्या॑ त॒नूः । त॒नूस्तया᳚ । तये॒मम् । इ॒ममꣳह॑सः । अꣳह॑सो मुञ्चतम् । मु॒ञ्च॒त॒म् ॅया । या वा᳚म् । वा॒मि॒न्द्रा॒व॒रु॒णा॒ । इ॒न्द्रा॒व॒रु॒णा॒ स॒ह॒स्या᳚ । इ॒न्द्रा॒व॒रु॒णेती᳚न्द्रा - व॒रु॒णा॒ । स॒ह॒स्या॑ रक्ष॒स्या᳚ । र॒क्ष॒स्या॑ तेज॒स्या᳚ । ते॒ज॒स्या॑ त॒नूः । त॒नूस्तया᳚ । तये॒मम् । इ॒ममꣳह॑सः । अꣳह॑सो मुञ्चतम् । मु॒ञ्च॒त॒म् ॅयः । यो वा᳚म् । वा॒ मि॒न्द्रा॒व॒रु॒णौ॒ । इ॒न्द्रा॒व॒रु॒णा॒व॒ग्नौ । इ॒न्द्रा॒व॒रु॒णा॒विती᳚न्द्रा - व॒रु॒णौ॒ । अ॒ग्नौ स्रामः॑ । स्राम॒स्तम् । तम् ॅवा᳚म् । वा॒मे॒तेन॑ । ए॒तेनाव॑ । अव॑ यजे । य॒जे॒ यः । यो वा᳚म् । वा॒मि॒न्द्रा॒व॒रु॒णा॒ । इ॒न्द्रा॒व॒रु॒णा॒ द्वि॒पाथ्सु॑ । इ॒न्द्रा॒व॒रु॒णेती᳚न्द्रा - व॒रु॒णा॒ । द्वि॒पाथ्सु॑ प॒शुषु॑ । द्वि॒पाथ्स्विति॑ द्वि॒पात् - सु॒ । प॒शुषु॒ चतु॑ष्पाथ्सु । चतु॑ष्पाथ्सु गो॒ष्ठे । चतु॑ष्पा॒थ्स्विति॒ चतु॑ष्पात् - सु॒ । गो॒ष्ठे गृ॒हेषु॑ । गो॒ष्ठ इति॑ गो - स्थे । गृ॒हेष्व॒फ्सु । अ॒फ्स्वोष॑धीषु । अ॒फ्स्वित्य॑प् - सु । ओष॑धीषु॒ वन॒स्पति॑षु । वन॒स्पति॑षु॒ स्रामः॑ । स्राम॒स्तम् । तम् ॅवा᳚म् । वा॒ मे॒तेन॑ । ए॒तेनाव॑ । अव॑ यजे । य॒ज॒ इन्द्रः॑ । इन्द्रो॒ वै । वा ए॒तस्य॑ । ए॒तस्ये᳚न्द्रि॒येण॑ \newline

\textbf{Jatai Paata} \newline

1. या वां᳚ ॅवां॒ ॅया या वा᳚म् । \newline
2. वा॒ मि॒न्द्रा॒व॒रु॒ णे॒न्द्रा॒व॒रु॒णा॒ वां॒ ॅवा॒ मि॒न्द्रा॒व॒रु॒णा॒ । \newline
3. इ॒न्द्रा॒व॒रु॒णा॒ य॒त॒व्या॑ यत॒व्ये᳚न्द्रावरु णेन्द्रावरुणा यत॒व्या᳚ । \newline
4. इ॒न्द्रा॒व॒रु॒णेती᳚न्द्रा - व॒रु॒णा॒ । \newline
5. य॒त॒व्या॑ त॒नू स्त॒नूर् य॑त॒व्या॑ यत॒व्या॑ त॒नूः । \newline
6. त॒नू स्तया॒ तया॑ त॒नू स्त॒नू स्तया᳚ । \newline
7. तये॒म मि॒मम् तया॒ तये॒मम् । \newline
8. इ॒म मꣳह॒सो ऽꣳह॑स इ॒म मि॒म मꣳह॑सः । \newline
9. अꣳह॑सो मुञ्चतम् मुञ्चत॒ मꣳह॒सोऽ ꣳह॑सो मुञ्चतम् । \newline
10. मु॒ञ्च॒तं॒ ॅया या मु॑ञ्चतम् मुञ्चतं॒ ॅया । \newline
11. या वां᳚ ॅवां॒ ॅया या वा᳚म् । \newline
12. वा॒ मि॒न्द्रा॒व॒रु॒ णे॒न्द्रा॒व॒रु॒णा॒ वां॒ ॅवा॒ मि॒न्द्रा॒व॒रु॒णा॒ । \newline
13. इ॒न्द्रा॒व॒रु॒णा॒ स॒ह॒स्या॑ सह॒स्ये᳚न्द्रावरु णेन्द्रावरुणा सह॒स्या᳚ । \newline
14. इ॒न्द्रा॒व॒रु॒णेती᳚न्द्रा - व॒रु॒णा॒ । \newline
15. स॒ह॒स्या॑ रक्ष॒स्या॑ रक्ष॒स्या॑ सह॒स्या॑ सह॒स्या॑ रक्ष॒स्या᳚ । \newline
16. र॒क्ष॒स्या॑ तेज॒स्या॑ तेज॒स्या॑ रक्ष॒स्या॑ रक्ष॒स्या॑ तेज॒स्या᳚ । \newline
17. ते॒ज॒स्या॑ त॒नू स्त॒नू स्ते॑ज॒स्या॑ तेज॒स्या॑ त॒नूः । \newline
18. त॒नू स्तया॒ तया॑ त॒नू स्त॒नू स्तया᳚ । \newline
19. तये॒म मि॒मम् तया॒ तये॒मम् । \newline
20. इ॒म मꣳह॒सो ऽꣳह॑स इ॒म मि॒म मꣳह॑सः । \newline
21. अꣳह॑सो मुञ्चतम् मुञ्चत॒ मꣳह॒सो ऽꣳह॑सो मुञ्चतम् । \newline
22. मु॒ञ्च॒तं॒ ॅयो यो मु॑ञ्चतम् मुञ्चतं॒ ॅयः । \newline
23. यो वां᳚ ॅवां॒ ॅयो यो वा᳚म् । \newline
24. वा॒ मि॒न्द्रा॒व॒रु॒णा॒ वि॒न्द्रा॒व॒रु॒णौ॒ वां॒ ॅवा॒ मि॒न्द्रा॒व॒रु॒णौ॒ । \newline
25. इ॒न्द्रा॒व॒रु॒णा॒ व॒ग्ना व॒ग्ना वि॑न्द्रावरुणा विन्द्रावरुणा व॒ग्नौ । \newline
26. इ॒न्द्रा॒व॒रु॒णा॒विती᳚न्द्रा - व॒रु॒णौ॒ । \newline
27. अ॒ग्नौ स्रामः॒ स्रामो॒ ऽग्ना व॒ग्नौ स्रामः॑ । \newline
28. स्राम॒ स्तम् तꣳ स्रामः॒ स्राम॒ स्तम् । \newline
29. तं ॅवां᳚ ॅवा॒म् तम् तं ॅवा᳚म् । \newline
30. वा॒ मे॒तेनै॒तेन॑ वां ॅवा मे॒तेन॑ । \newline
31. ए॒तेनावा वै॒ते नै॒तेनाव॑ । \newline
32. अव॑ यजे य॒जे ऽवाव॑ यजे । \newline
33. य॒जे॒ यो यो य॑जे यजे॒ यः । \newline
34. यो वां᳚ ॅवां॒ ॅयो यो वा᳚म् । \newline
35. वा॒ मि॒न्द्रा॒व॒रु॒ णे॒न्द्रा॒व॒रु॒णा॒ वां॒ ॅवा॒ मि॒न्द्रा॒व॒रु॒णा॒ । \newline
36. इ॒न्द्रा॒व॒रु॒णा॒ द्वि॒पाथ्सु॑ द्वि॒पा थ्स्वि॑न्द्रावरु णेन्द्रावरुणा द्वि॒पाथ्सु॑ । \newline
37. इ॒न्द्रा॒व॒रु॒णेती᳚न्द्रा - व॒रु॒णा॒ । \newline
38. द्वि॒पाथ्सु॑ प॒शुषु॑ प॒शुषु॑ द्वि॒पाथ्सु॑ द्वि॒पाथ्सु॑ प॒शुषु॑ । \newline
39. द्वि॒पाथ्स्विति॑ द्वि॒पात् - सु॒ । \newline
40. प॒शुषु॒ चतु॑ष्पाथ्सु॒ चतु॑ष्पाथ्सु प॒शुषु॑ प॒शुषु॒ चतु॑ष्पाथ्सु । \newline
41. चतु॑ष्पाथ्सु गो॒ष्ठे गो॒ष्ठे चतु॑ष्पाथ्सु॒ चतु॑ष्पाथ्सु गो॒ष्ठे । \newline
42. चतु॑ष्पा॒थ्स्विति॒ चतु॑ष्पात् - सु॒ । \newline
43. गो॒ष्ठे गृ॒हेषु॑ गृ॒हेषु॑ गो॒ष्ठे गो॒ष्ठे गृ॒हेषु॑ । \newline
44. गो॒ष्ठ इति॑ गो - स्थे । \newline
45. गृ॒हे ष्व॒फ्स्व॑फ्सु गृ॒हेषु॑ गृ॒हे ष्व॒फ्सु । \newline
46. अ॒फ् स्वोष॑धी॒ ष्वोष॑धी ष्व॒ फ्स्व॑ फ्स्वोष॑धीषु । \newline
47. अ॒फ्स्वित्य॑प् - सु । \newline
48. ओष॑धीषु॒ वन॒स्पति॑षु॒ वन॒स्पति॒ ष्वोष॑धी॒ ष्वोष॑धीषु॒ वन॒स्पति॑षु । \newline
49. वन॒स्पति॑षु॒ स्रामः॒ स्रामो॒ वन॒स्पति॑षु॒ वन॒स्पति॑षु॒ स्रामः॑ । \newline
50. स्राम॒ स्तम् तꣳ स्रामः॒ स्राम॒ स्तम् । \newline
51. तं ॅवां᳚ ॅवा॒म् तम् तं ॅवा᳚म् । \newline
52. वा॒ मे॒तेनै॒तेन॑ वां ॅवा मे॒तेन॑ । \newline
53. ए॒तेनावा वै॒ते नै॒तेनाव॑ । \newline
54. अव॑ यजे य॒जे ऽवाव॑ यजे । \newline
55. य॒ज॒ इन्द्र॒ इन्द्रो॑ यजे यज॒ इन्द्रः॑ । \newline
56. इन्द्रो॒ वै वा इन्द्र॒ इन्द्रो॒ वै । \newline
57. वा ए॒तस्यै॒तस्य॒ वै वा ए॒तस्य॑ । \newline
58. ए॒तस्ये᳚ न्द्रि॒येणे᳚ न्द्रि॒ये णै॒त स्यै॒तस्ये᳚ न्द्रि॒येण॑ । \newline

\textbf{Ghana Paata } \newline

1. या वां᳚ ॅवां॒ ॅया या वा॑ मिन्द्रावरु णेन्द्रावरुणा वां॒ ॅया या वा॑ मिन्द्रावरुणा । \newline
2. वा॒ मि॒न्द्रा॒व॒रु॒ णे॒न्द्रा॒व॒रु॒णा॒ वां॒ ॅवा॒ मि॒न्द्रा॒व॒रु॒णा॒ य॒त॒व्या॑ यत॒व्ये᳚न्द्रावरुणा वां ॅवा मिन्द्रावरुणा यत॒व्या᳚ । \newline
3. इ॒न्द्रा॒व॒रु॒णा॒ य॒त॒व्या॑ यत॒व्ये᳚न्द्रावरु णेन्द्रावरुणा यत॒व्या॑ त॒नू स्त॒नूर् य॑त॒व्ये᳚न्द्रावरु णेन्द्रावरुणा यत॒व्या॑ त॒नूः । \newline
4. इ॒न्द्रा॒व॒रु॒णेती᳚न्द्रा - व॒रु॒णा॒ । \newline
5. य॒त॒व्या॑ त॒नू स्त॒नूर् य॑त॒व्या॑ यत॒व्या॑ त॒नू स्तया॒ तया॑ त॒नूर् य॑त॒व्या॑ यत॒व्या॑ त॒नू स्तया᳚ । \newline
6. त॒नू स्तया॒ तया॑ त॒नू स्त॒नू स्तये॒म मि॒मम् तया॑ त॒नू स्त॒नू स्तये॒मम् । \newline
7. तये॒म मि॒मम् तया॒ तये॒म मꣳह॒सो ऽꣳह॑स इ॒मम् तया॒ तये॒म मꣳह॑सः । \newline
8. इ॒म मꣳह॒सो ऽꣳह॑स इ॒म मि॒म मꣳह॑सो मुञ्चतम् मुञ्चत॒ मꣳह॑स इ॒म मि॒म मꣳह॑सो मुञ्चतम् । \newline
9. अꣳह॑सो मुञ्चतम् मुञ्चत॒ मꣳह॒सो ऽꣳह॑सो मुञ्चतं॒ ॅया या मु॑ञ्चत॒ मꣳह॒सो ऽꣳह॑सो मुञ्चतं॒ ॅया । \newline
10. मु॒ञ्च॒तं॒ ॅया या मु॑ञ्चतम् मुञ्चतं॒ ॅया वां᳚ ॅवां॒ ॅया मु॑ञ्चतम् मुञ्चतं॒ ॅया वा᳚म् । \newline
11. या वां᳚ ॅवां॒ ॅया या वा॑ मिन्द्रावरु णेन्द्रावरुणा वां॒ ॅया या वा॑ मिन्द्रावरुणा । \newline
12. वा॒ मि॒न्द्रा॒व॒रु॒ णे॒न्द्रा॒व॒रु॒णा॒ वां॒ ॅवा॒ मि॒न्द्रा॒व॒रु॒णा॒ स॒ह॒स्या॑ सह॒स्ये᳚न्द्रावरुणा वां ॅवा मिन्द्रावरुणा सह॒स्या᳚ । \newline
13. इ॒न्द्रा॒व॒रु॒णा॒ स॒ह॒स्या॑ सह॒स्ये᳚न्द्रावरु णेन्द्रावरुणा सह॒स्या॑ रक्ष॒स्या॑ रक्ष॒स्या॑ सह॒स्ये᳚न्द्रावरु णेन्द्रावरुणा सह॒स्या॑ रक्ष॒स्या᳚ । \newline
14. इ॒न्द्रा॒व॒रु॒णेती᳚न्द्रा - व॒रु॒णा॒ । \newline
15. स॒ह॒स्या॑ रक्ष॒स्या॑ रक्ष॒स्या॑ सह॒स्या॑ सह॒स्या॑ रक्ष॒स्या॑ तेज॒स्या॑ तेज॒स्या॑ रक्ष॒स्या॑ सह॒स्या॑ सह॒स्या॑ रक्ष॒स्या॑ तेज॒स्या᳚ । \newline
16. र॒क्ष॒स्या॑ तेज॒स्या॑ तेज॒स्या॑ रक्ष॒स्या॑ रक्ष॒स्या॑ तेज॒स्या॑ त॒नू स्त॒नू स्ते॑ज॒स्या॑ रक्ष॒स्या॑ रक्ष॒स्या॑ तेज॒स्या॑ त॒नूः । \newline
17. ते॒ज॒स्या॑ त॒नू स्त॒नू स्ते॑ज॒स्या॑ तेज॒स्या॑ त॒नू स्तया॒ तया॑ त॒नू स्ते॑ज॒स्या॑ तेज॒स्या॑ त॒नू स्तया᳚ । \newline
18. त॒नू स्तया॒ तया॑ त॒नू स्त॒नू स्तये॒म मि॒मम् तया॑ त॒नू स्त॒नू स्तये॒मम् । \newline
19. तये॒म मि॒मम् तया॒ तये॒म मꣳह॒सो ऽꣳह॑स इ॒मम् तया॒ तये॒म मꣳह॑सः । \newline
20. इ॒म मꣳह॒सो ऽꣳह॑स इ॒म मि॒म मꣳह॑सो मुञ्चतम् मुञ्चत॒ मꣳह॑स इ॒म मि॒म मꣳह॑सो मुञ्चतम् । \newline
21. अꣳह॑सो मुञ्चतम् मुञ्चत॒ मꣳह॒सो ऽꣳह॑सो मुञ्चतं॒ ॅयो यो मु॑ञ्चत॒ मꣳह॒सो ऽꣳह॑सो मुञ्चतं॒ ॅयः । \newline
22. मु॒ञ्च॒तं॒ ॅयो यो मु॑ञ्चतम् मुञ्चतं॒ ॅयो वां᳚ ॅवां॒ ॅयो मु॑ञ्चतम् मुञ्चतं॒ ॅयो वा᳚म् । \newline
23. यो वां᳚ ॅवां॒ ॅयो यो वा॑ मिन्द्रावरुणा विन्द्रावरुणौ वां॒ ॅयो यो वा॑ मिन्द्रावरुणौ । \newline
24. वा॒ मि॒न्द्रा॒व॒रु॒णा॒ वि॒न्द्रा॒व॒रु॒णौ॒ वां॒ ॅवा॒ मि॒न्द्रा॒व॒रु॒णा॒ व॒ग्ना व॒ग्ना वि॑न्द्रावरुणौ वां ॅवा मिन्द्रावरुणा व॒ग्नौ । \newline
25. इ॒न्द्रा॒व॒रु॒णा॒ व॒ग्ना व॒ग्ना वि॑न्द्रावरुणा विन्द्रावरुणा व॒ग्नौ स्रामः॒ स्रामो॒ ऽग्ना वि॑न्द्रावरुणा विन्द्रावरुणा व॒ग्नौ स्रामः॑ । \newline
26. इ॒न्द्रा॒व॒रु॒णा॒विती᳚न्द्रा - व॒रु॒णौ॒ । \newline
27. अ॒ग्नौ स्रामः॒ स्रामो॒ ऽग्ना व॒ग्नौ स्राम॒ स्तम् तꣳ स्रामो॒ ऽग्ना व॒ग्नौ स्राम॒ स्तम् । \newline
28. स्राम॒ स्तम् तꣳ स्रामः॒ स्राम॒ स्तं ॅवां᳚ ॅवा॒म् तꣳ स्रामः॒ स्राम॒ स्तं ॅवा᳚म् । \newline
29. तं ॅवां᳚ ॅवा॒म् तम् तं ॅवा॑ मे॒तेनै॒तेन॑ वा॒म् तम् तं ॅवा॑ मे॒तेन॑ । \newline
30. वा॒ मे॒तेनै॒तेन॑ वां ॅवा मे॒तेनावा वै॒तेन॑ वां ॅवा मे॒तेनाव॑ । \newline
31. ए॒तेनावा वै॒ते नै॒तेनाव॑ यजे य॒जे ऽवै॒ते नै॒तेनाव॑ यजे । \newline
32. अव॑ यजे य॒जे ऽवाव॑ यजे॒ यो यो य॒जे ऽवाव॑ यजे॒ यः । \newline
33. य॒जे॒ यो यो य॑जे यजे॒ यो वां᳚ ॅवां॒ ॅयो य॑जे यजे॒ यो वा᳚म् । \newline
34. यो वां᳚ ॅवां॒ ॅयो यो वा॑ मिन्द्रावरुणे न्द्रावरुणा वां॒ ॅयो यो वा॑ मिन्द्रावरुणा । \newline
35. वा॒ मि॒न्द्रा॒व॒रु॒णे॒ न्द्रा॒व॒रु॒णा॒ वां॒ ॅवा॒ मि॒न्द्रा॒व॒रु॒णा॒ द्वि॒पाथ्सु॑ द्वि॒पाथ् स्वि॑न्द्रावरुणा वां ॅवा मिन्द्रावरुणा द्वि॒पाथ्सु॑ । \newline
36. इ॒न्द्रा॒व॒रु॒णा॒ द्वि॒पाथ्सु॑ द्वि॒पाथ् स्वि॑न्द्रावरुणे न्द्रावरुणा द्वि॒पाथ्सु॑ प॒शुषु॑ प॒शुषु॑ द्वि॒पाथ् स्वि॑न्द्रावरुणे न्द्रावरुणा द्वि॒पाथ्सु॑ प॒शुषु॑ । \newline
37. इ॒न्द्रा॒व॒रु॒णेती᳚न्द्रा - व॒रु॒णा॒ । \newline
38. द्वि॒पाथ्सु॑ प॒शुषु॑ प॒शुषु॑ द्वि॒पाथ्सु॑ द्वि॒पाथ्सु॑ प॒शुषु॒ चतु॑ष्पाथ्सु॒ चतु॑ष्पाथ्सु प॒शुषु॑ द्वि॒पाथ्सु॑ द्वि॒पाथ्सु॑ प॒शुषु॒ चतु॑ष्पाथ्सु । \newline
39. द्वि॒पाथ्स्विति॑ द्वि॒पात् - सु॒ । \newline
40. प॒शुषु॒ चतु॑ष्पाथ्सु॒ चतु॑ष्पाथ्सु प॒शुषु॑ प॒शुषु॒ चतु॑ष्पाथ्सु गो॒ष्ठे गो॒ष्ठे चतु॑ष्पाथ्सु प॒शुषु॑ प॒शुषु॒ चतु॑ष्पाथ्सु गो॒ष्ठे । \newline
41. चतु॑ष्पाथ्सु गो॒ष्ठे गो॒ष्ठे चतु॑ष्पाथ्सु॒ चतु॑ष्पाथ्सु गो॒ष्ठे गृ॒हेषु॑ गृ॒हेषु॑ गो॒ष्ठे चतु॑ष्पाथ्सु॒ चतु॑ष्पाथ्सु गो॒ष्ठे गृ॒हेषु॑ । \newline
42. चतु॑ष्पा॒थ्स्विति॒ चतु॑ष्पात् - सु॒ । \newline
43. गो॒ष्ठे गृ॒हेषु॑ गृ॒हेषु॑ गो॒ष्ठे गो॒ष्ठे गृ॒हे ष्व॒फ्स्व॑फ्सु गृ॒हेषु॑ गो॒ष्ठे गो॒ष्ठे गृ॒हेष्व॒फ्सु । \newline
44. गो॒ष्ठ इति॑ गो - स्थे । \newline
45. गृ॒हे ष्व॒फ्स्व॑फ्सु गृ॒हेषु॑ गृ॒हे ष्व॒फ्स्वोष॑धी॒ ष्वोष॑धी ष्व॒फ्सु गृ॒हेषु॑ गृ॒हे ष्व॒फ्स्वोष॑धीषु । \newline
46. अ॒फ्स्वोष॑धी॒ ष्वोष॑धी ष्व॒फ्स्व॑ फ्स्वोष॑धीषु॒ वन॒स्पति॑षु॒ वन॒स्पति॒ ष्वोष॑धी ष्व॒फ्स्व॑ फ्स्वोष॑धीषु॒ वन॒स्पति॑षु । \newline
47. अ॒फ्स्वित्य॑प् - सु । \newline
48. ओष॑धीषु॒ वन॒स्पति॑षु॒ वन॒स्पति॒ ष्वोष॑धी॒ ष्वोष॑धीषु॒ वन॒स्पति॑षु॒ स्रामः॒ स्रामो॒ वन॒स्पति॒ ष्वोष॑धी॒ ष्वोष॑धीषु॒ वन॒स्पति॑षु॒ स्रामः॑ । \newline
49. वन॒स्पति॑षु॒ स्रामः॒ स्रामो॒ वन॒स्पति॑षु॒ वन॒स्पति॑षु॒ स्राम॒स्तम् तꣳ स्रामो॒ वन॒स्पति॑षु॒ वन॒स्पति॑षु॒ स्राम॒स्तम् । \newline
50. स्राम॒ स्तम् तꣳ स्रामः॒ स्राम॒ स्तं ॅवां᳚ ॅवा॒म् तꣳ स्रामः॒ स्राम॒ स्तं ॅवा᳚म् । \newline
51. तं ॅवां᳚ ॅवा॒म् तम् तं ॅवा॑ मे॒तेनै॒तेन॑ वा॒म् तम् तं ॅवा॑ मे॒तेन॑ । \newline
52. वा॒ मे॒तेनै॒तेन॑ वां ॅवा मे॒तेनावा वै॒तेन॑ वां ॅवा मे॒तेनाव॑ । \newline
53. ए॒तेनावा वै॒ते नै॒तेनाव॑ यजे य॒जे ऽवै॒ते नै॒तेनाव॑ यजे । \newline
54. अव॑ यजे य॒जे ऽवाव॑ यज॒ इन्द्र॒ इन्द्रो॑ य॒जे ऽवाव॑ यज॒ इन्द्रः॑ । \newline
55. य॒ज॒ इन्द्र॒ इन्द्रो॑ यजे यज॒ इन्द्रो॒ वै वा इन्द्रो॑ यजे यज॒ इन्द्रो॒ वै । \newline
56. इन्द्रो॒ वै वा इन्द्र॒ इन्द्रो॒ वा ए॒तस्यै॒तस्य॒ वा इन्द्र॒ इन्द्रो॒ वा ए॒तस्य॑ । \newline
57. वा ए॒तस्यै॒तस्य॒ वै वा ए॒तस्ये᳚ न्द्रि॒येणे᳚ न्द्रि॒ये णै॒तस्य॒ वै वा ए॒तस्ये᳚ न्द्रि॒येण॑ । \newline
58. ए॒तस्ये᳚ न्द्रि॒येणे᳚ न्द्रि॒ये णै॒त स्यै॒तस्ये᳚ न्द्रि॒येणापापे᳚ न्द्रि॒येणै॒त स्यै॒तस्ये᳚ न्द्रि॒येणाप॑ । \newline
\pagebreak
\markright{ TS 2.3.13.2  \hfill https://www.vedavms.in \hfill}

\section{ TS 2.3.13.2 }

\textbf{TS 2.3.13.2 } \newline
\textbf{Samhita Paata} \newline

-न्द्रि॒येणाप॑ क्रामति॒ वरु॑ण एनं ॅवरुणपा॒शेन॑ गृह्णाति॒ यः पा॒प्मना॑ गृही॒तो भव॑ति॒ यः पा॒प्मना॑ गृही॒तः स्यात् तस्मा॑ ए॒तामै᳚न्द्रावरु॒णीं प॑य॒स्यां᳚ निर्व॑पे॒दिन्द्र॑ ए॒वास्मि॑न्निन्द्रि॒यं द॑धाति॒ वरु॑ण एनं ॅवरुणपा॒शान्-मु॑ञ्चति पय॒स्या॑ भवति॒ पयो॒ हि वा ए॒तस्मा॑दप॒क्राम॒त्यथै॒ष पा॒प्मना॑ गृही॒तो यत् प॑य॒स्या॑ भव॑ति॒ पय॑ ए॒वास्मि॒न् तया॑ दधाति पय॒स्या॑यां - [  ] \newline

\textbf{Pada Paata} \newline

इ॒न्द्रि॒येण॑ । अपेति॑ । क्रा॒म॒ति॒ । वरु॑णः । ए॒न॒म् । व॒रु॒ण॒पा॒शेनेति॑ वरुण - पा॒शेन॑ । गृ॒ह्णा॒ति॒ । यः । पा॒प्मना᳚ । गृ॒ही॒तः । भव॑ति । यः ।   पा॒प्मना᳚ । गृ॒ही॒तः । स्यात् । तस्मै᳚ । ए॒ताम् । ऐ॒न्द्रा॒व॒रु॒णीमित्यै᳚न्द्रा - व॒रु॒णीम् । प॒य॒स्या᳚म् । निरिति॑ । व॒पे॒त् । इन्द्रः॑ । ए॒व । अ॒स्मि॒न्न् । इ॒न्द्रि॒यम् । द॒धा॒ति॒ । वरु॑णः । ए॒न॒म् । व॒रु॒ण॒पा॒शादिति॑ वरुण - पा॒शात् । मु॒ञ्च॒ति॒ । प॒य॒स्या᳚ । भ॒व॒ति॒ । पयः॑ । हि । वै । ए॒तस्मा᳚त् । अ॒प॒क्राम॒तीत्य॑प - क्राम॑ति । अथ॑ । ए॒षः । पा॒प्मना᳚ । गृ॒ही॒तः । यत् । प॒य॒स्या᳚ । भव॑ति । पयः॑ । ए॒व । अ॒स्मि॒न्न् । तया᳚ । द॒धा॒ति॒ । प॒य॒स्या॑याम् ।  \newline


\textbf{Krama Paata} \newline

इ॒न्द्रि॒येणाप॑ । अप॑ क्रामति । क्रा॒म॒ति॒ वरु॑णः । वरु॑ण एनम् । ए॒नं॒ ॅव॒रु॒ण॒पा॒शेन॑ । व॒रु॒ण॒पा॒शेन॑ गृह्णाति । व॒रु॒ण॒पा॒शेनेति॑ वरुण - पा॒शेन॑ । गृ॒ह्णा॒ति॒ यः । यः पा॒प्मना᳚ । पा॒प्मना॑ गृही॒तः । गृ॒ही॒तो भव॑ति । भव॑ति॒ यः । यः पा॒प्मना᳚ । पा॒प्मना॑ गृही॒तः । गृ॒ही॒तः स्यात् । स्यात् तस्मै᳚ । तस्मा॑ ए॒ताम् । ए॒तामै᳚न्द्रावरु॒णीम् । ऐ॒न्द्रा॒वरु॒णीम् प॑य॒स्या᳚म् । ऐ॒न्द्रा॒व॒रु॒णीमित्यै᳚न्द्रा - व॒रु॒णीम् । प॒य॒स्या᳚म् निः । निर्व॑पेत् । व॒पे॒दिन्द्रः॑ । इन्द्र॑ ए॒व । ए॒वास्मिन्न्॑ । अ॒स्मि॒न्नि॒न्द्रि॒यम् । इ॒न्द्रि॒यम् द॑धाति । द॒धा॒ति॒ वरु॑णः । वरु॑ण एनम् । ए॒न॒म् ॅव॒रु॒ण॒पा॒शात् । व॒रु॒ण॒पा॒शान् मु॑ञ्चति । व॒रु॒ण॒पा॒शादिति॑ वरुण - पा॒शात् । मु॒ञ्च॒ति॒ प॒य॒स्या᳚ । प॒य॒स्या॑ भवति । भ॒व॒ति॒ पयः॑ । पयो॒ हि । हि वै । वा ए॒तस्मा᳚त् । ए॒तस्मा॑दप॒क्राम॑ति । अ॒प॒क्राम॒त्यथ॑ । अ॒प॒क्राम॒तीत्य॑प - क्राम॑ति । अथै॒षः । ए॒ष पा॒प्मना᳚ । पा॒प्मना॑ गृही॒तः । गृ॒ही॒तो यत् । यत् प॑य॒स्या᳚ । प॒य॒स्या॑ भव॑ति । भव॑ति॒ पयः॑ । पय॑ ए॒व । ए॒वास्मिन्न्॑ । अ॒स्मि॒न् तया᳚ । तया॑ दधाति । द॒धा॒ति॒ प॒य॒स्या॑याम् । प॒य॒स्या॑याम् पुरो॒डाश᳚म् \newline

\textbf{Jatai Paata} \newline

1. इ॒न्द्रि॒येणापापे᳚ न्द्रि॒येणे᳚ न्द्रि॒येणाप॑ । \newline
2. अप॑ क्रामति क्राम॒ त्यपाप॑ क्रामति । \newline
3. क्रा॒म॒ति॒ वरु॑णो॒ वरु॑णः क्रामति क्रामति॒ वरु॑णः । \newline
4. वरु॑ण एन मेनं॒ ॅवरु॑णो॒ वरु॑ण एनम् । \newline
5. ए॒नं॒ ॅव॒रु॒ण॒पा॒शेन॑ वरुणपा॒शेनै॑न मेनं ॅवरुणपा॒शेन॑ । \newline
6. व॒रु॒ण॒पा॒शेन॑ गृह्णाति गृह्णाति वरुणपा॒शेन॑ वरुणपा॒शेन॑ गृह्णाति । \newline
7. व॒रु॒ण॒पा॒शेनेति॑ वरुण - पा॒शेन॑ । \newline
8. गृ॒ह्णा॒ति॒ यो यो गृ॑ह्णाति गृह्णाति॒ यः । \newline
9. यः पा॒प्मना॑ पा॒प्मना॒ यो यः पा॒प्मना᳚ । \newline
10. पा॒प्मना॑ गृही॒तो गृ॑ही॒तः पा॒प्मना॑ पा॒प्मना॑ गृही॒तः । \newline
11. गृ॒ही॒तो भव॑ति॒ भव॑ति गृही॒तो गृ॑ही॒तो भव॑ति । \newline
12. भव॑ति॒ यो यो भव॑ति॒ भव॑ति॒ यः । \newline
13. यः पा॒प्मना॑ पा॒प्मना॒ यो यः पा॒प्मना᳚ । \newline
14. पा॒प्मना॑ गृही॒तो गृ॑ही॒तः पा॒प्मना॑ पा॒प्मना॑ गृही॒तः । \newline
15. गृ॒ही॒तः स्याथ् स्याद् गृ॑ही॒तो गृ॑ही॒तः स्यात् । \newline
16. स्यात् तस्मै॒ तस्मै॒ स्याथ् स्यात् तस्मै᳚ । \newline
17. तस्मा॑ ए॒ता मे॒ताम् तस्मै॒ तस्मा॑ ए॒ताम् । \newline
18. ए॒ता मै᳚न्द्रावरु॒णी मै᳚न्द्रावरु॒णी मे॒ता मे॒ता मै᳚न्द्रावरु॒णीम् । \newline
19. ऐ॒न्द्रा॒व॒रु॒णीम् प॑य॒स्या᳚म् पय॒स्या॑ मैन्द्रावरु॒णी मै᳚न्द्रावरु॒णीम् प॑य॒स्या᳚म् । \newline
20. ऐ॒न्द्रा॒व॒रु॒णीमित्यै᳚न्द्रा - व॒रु॒णीम् । \newline
21. प॒य॒स्या᳚म् निर् णिष् प॑य॒स्या᳚म् पय॒स्या᳚म् निः । \newline
22. निर् व॑पेद् वपे॒न् निर् णिर् व॑पेत् । \newline
23. व॒पे॒ दिन्द्र॒ इन्द्रो॑ वपेद् वपे॒ दिन्द्रः॑ । \newline
24. इन्द्र॑ ए॒वैवे न्द्र॒ इन्द्र॑ ए॒व । \newline
25. ए॒वास्मि॑न् नस्मिन् ने॒वैवास्मिन्न्॑ । \newline
26. अ॒स्मि॒न् नि॒न्द्रि॒य मि॑न्द्रि॒य म॑स्मिन् नस्मिन् निन्द्रि॒यम् । \newline
27. इ॒न्द्रि॒यम् द॑धाति दधातीन्द्रि॒य मि॑न्द्रि॒यम् द॑धाति । \newline
28. द॒धा॒ति॒ वरु॑णो॒ वरु॑णो दधाति दधाति॒ वरु॑णः । \newline
29. वरु॑ण एन मेनं॒ ॅवरु॑णो॒ वरु॑ण एनम् । \newline
30. ए॒नं॒ ॅव॒रु॒ण॒पा॒शाद् व॑रुणपा॒शादे॑न मेनं ॅवरुणपा॒शात् । \newline
31. व॒रु॒ण॒पा॒शान् मु॑ञ्चति मुञ्चति वरुणपा॒शाद् व॑रुणपा॒शान् मु॑ञ्चति । \newline
32. व॒रु॒ण॒पा॒शादिति॑ वरुण - पा॒शात् । \newline
33. मु॒ञ्च॒ति॒ प॒य॒स्या॑ पय॒स्या॑ मुञ्चति मुञ्चति पय॒स्या᳚ । \newline
34. प॒य॒स्या॑ भवति भवति पय॒स्या॑ पय॒स्या॑ भवति । \newline
35. भ॒व॒ति॒ पयः॒ पयो॑ भवति भवति॒ पयः॑ । \newline
36. पयो॒ हि हि पयः॒ पयो॒ हि । \newline
37. हि वै वै हि हि वै । \newline
38. वा ए॒तस्मा॑ दे॒तस्मा॒द् वै वा ए॒तस्मा᳚त् । \newline
39. ए॒तस्मा॑ दप॒क्राम॑ त्यप॒क्राम॑ त्ये॒तस्मा॑ दे॒तस्मा॑ दप॒क्राम॑ति । \newline
40. अ॒प॒क्राम॒ त्यथाथा॑ प॒क्राम॑ त्यप॒क्राम॒ त्यथ॑ । \newline
41. अ॒प॒क्राम॒तीत्य॑प - क्राम॑ति । \newline
42. अथै॒ष ए॒षो ऽथाथै॒षः । \newline
43. ए॒ष पा॒प्मना॑ पा॒प्मनै॒ष ए॒ष पा॒प्मना᳚ । \newline
44. पा॒प्मना॑ गृही॒तो गृ॑ही॒तः पा॒प्मना॑ पा॒प्मना॑ गृही॒तः । \newline
45. गृ॒ही॒तो यद् यद् गृ॑ही॒तो गृ॑ही॒तो यत् । \newline
46. यत् प॑य॒स्या॑ पय॒स्या॑ यद् यत् प॑य॒स्या᳚ । \newline
47. प॒य॒स्या॑ भव॑ति॒ भव॑ति पय॒स्या॑ पय॒स्या॑ भव॑ति । \newline
48. भव॑ति॒ पयः॒ पयो॒ भव॑ति॒ भव॑ति॒ पयः॑ । \newline
49. पय॑ ए॒वैव पयः॒ पय॑ ए॒व । \newline
50. ए॒वास्मि॑न् नस्मिन् ने॒वैवास्मिन्न्॑ । \newline
51. अ॒स्मि॒न् तया॒ तया᳚ ऽस्मिन् नस्मि॒न् तया᳚ । \newline
52. तया॑ दधाति दधाति॒ तया॒ तया॑ दधाति । \newline
53. द॒धा॒ति॒ प॒य॒स्या॑याम् पय॒स्या॑याम् दधाति दधाति पय॒स्या॑याम् । \newline
54. प॒य॒स्या॑याम् पुरो॒डाश॑म् पुरो॒डाश॑म् पय॒स्या॑याम् पय॒स्या॑याम् पुरो॒डाश᳚म् । \newline

\textbf{Ghana Paata } \newline

1. इ॒न्द्रि॒येणापापे᳚ न्द्रि॒येणे᳚ न्द्रि॒येणाप॑ क्रामति क्राम॒ त्यपे᳚ न्द्रि॒येणे᳚ न्द्रि॒येणाप॑ क्रामति । \newline
2. अप॑ क्रामति क्राम॒ त्यपाप॑ क्रामति॒ वरु॑णो॒ वरु॑णः क्राम॒ त्यपाप॑ क्रामति॒ वरु॑णः । \newline
3. क्रा॒म॒ति॒ वरु॑णो॒ वरु॑णः क्रामति क्रामति॒ वरु॑ण एन मेनं॒ ॅवरु॑णः क्रामति क्रामति॒ वरु॑ण एनम् । \newline
4. वरु॑ण एन मेनं॒ ॅवरु॑णो॒ वरु॑ण एनं ॅवरुणपा॒शेन॑ वरुणपा॒शेनै॑नं॒ ॅवरु॑णो॒ वरु॑ण एनं ॅवरुणपा॒शेन॑ । \newline
5. ए॒नं॒ ॅव॒रु॒ण॒पा॒शेन॑ वरुणपा॒शेनै॑न मेनं ॅवरुणपा॒शेन॑ गृह्णाति गृह्णाति वरुणपा॒शेनै॑न मेनं ॅवरुणपा॒शेन॑ गृह्णाति । \newline
6. व॒रु॒ण॒पा॒शेन॑ गृह्णाति गृह्णाति वरुणपा॒शेन॑ वरुणपा॒शेन॑ गृह्णाति॒ यो यो गृ॑ह्णाति वरुणपा॒शेन॑ वरुणपा॒शेन॑ गृह्णाति॒ यः । \newline
7. व॒रु॒ण॒पा॒शेनेति॑ वरुण - पा॒शेन॑ । \newline
8. गृ॒ह्णा॒ति॒ यो यो गृ॑ह्णाति गृह्णाति॒ यः पा॒प्मना॑ पा॒प्मना॒ यो गृ॑ह्णाति गृह्णाति॒ यः पा॒प्मना᳚ । \newline
9. यः पा॒प्मना॑ पा॒प्मना॒ यो यः पा॒प्मना॑ गृही॒तो गृ॑ही॒तः पा॒प्मना॒ यो यः पा॒प्मना॑ गृही॒तः । \newline
10. पा॒प्मना॑ गृही॒तो गृ॑ही॒तः पा॒प्मना॑ पा॒प्मना॑ गृही॒तो भव॑ति॒ भव॑ति गृही॒तः पा॒प्मना॑ पा॒प्मना॑ गृही॒तो भव॑ति । \newline
11. गृ॒ही॒तो भव॑ति॒ भव॑ति गृही॒तो गृ॑ही॒तो भव॑ति॒ यो यो भव॑ति गृही॒तो गृ॑ही॒तो भव॑ति॒ यः । \newline
12. भव॑ति॒ यो यो भव॑ति॒ भव॑ति॒ यः पा॒प्मना॑ पा॒प्मना॒ यो भव॑ति॒ भव॑ति॒ यः पा॒प्मना᳚ । \newline
13. यः पा॒प्मना॑ पा॒प्मना॒ यो यः पा॒प्मना॑ गृही॒तो गृ॑ही॒तः पा॒प्मना॒ यो यः पा॒प्मना॑ गृही॒तः । \newline
14. पा॒प्मना॑ गृही॒तो गृ॑ही॒तः पा॒प्मना॑ पा॒प्मना॑ गृही॒तः स्याथ् स्याद् गृ॑ही॒तः पा॒प्मना॑ पा॒प्मना॑ गृही॒तः स्यात् । \newline
15. गृ॒ही॒तः स्याथ् स्याद् गृ॑ही॒तो गृ॑ही॒तः स्यात् तस्मै॒ तस्मै॒ स्याद् गृ॑ही॒तो गृ॑ही॒तः स्यात् तस्मै᳚ । \newline
16. स्यात् तस्मै॒ तस्मै॒ स्याथ् स्यात् तस्मा॑ ए॒ता मे॒ताम् तस्मै॒ स्याथ् स्यात् तस्मा॑ ए॒ताम् । \newline
17. तस्मा॑ ए॒ता मे॒ताम् तस्मै॒ तस्मा॑ ए॒ता मै᳚न्द्रावरु॒णी मै᳚न्द्रावरु॒णी मे॒ताम् तस्मै॒ तस्मा॑ ए॒ता मै᳚न्द्रावरु॒णीम् । \newline
18. ए॒ता मै᳚न्द्रावरु॒णी मै᳚न्द्रावरु॒णी मे॒ता मे॒ता मै᳚न्द्रावरु॒णीम् प॑य॒स्या᳚म् पय॒स्या॑ मैन्द्रावरु॒णी मे॒ता मे॒ता मै᳚न्द्रावरु॒णीम् प॑य॒स्या᳚म् । \newline
19. ऐ॒न्द्रा॒व॒रु॒णीम् प॑य॒स्या᳚म् पय॒स्या॑ मैन्द्रावरु॒णी मै᳚न्द्रावरु॒णीम् प॑य॒स्या᳚म् निर् णिष् प॑य॒स्या॑ मैन्द्रावरु॒णी मै᳚न्द्रावरु॒णीम् प॑य॒स्या᳚म् निः । \newline
20. ऐ॒न्द्रा॒व॒रु॒णीमित्यै᳚न्द्रा - व॒रु॒णीम् । \newline
21. प॒य॒स्या᳚म् निर् णिष् प॑य॒स्या᳚म् पय॒स्या᳚म् निर् व॑पेद् वपे॒न् निष् प॑य॒स्या᳚म् पय॒स्या᳚म् निर् व॑पेत् । \newline
22. निर् व॑पेद् वपे॒न् निर् णिर् व॑पे॒ दिन्द्र॒ इन्द्रो॑ वपे॒न् निर् णिर् व॑पे॒ दिन्द्रः॑ । \newline
23. व॒पे॒ दिन्द्र॒ इन्द्रो॑ वपेद् वपे॒ दिन्द्र॑ ए॒वैवे न्द्रो॑ वपेद् वपे॒ दिन्द्र॑ ए॒व । \newline
24. इन्द्र॑ ए॒वैवे न्द्र॒ इन्द्र॑ ए॒वास्मि॑न् नस्मिन् ने॒वे न्द्र॒ इन्द्र॑ ए॒वास्मिन्न्॑ । \newline
25. ए॒वास्मि॑न् नस्मिन् ने॒वैवास्मि॑न् निन्द्रि॒य मि॑न्द्रि॒य म॑स्मिन् ने॒वैवास्मि॑न् निन्द्रि॒यम् । \newline
26. अ॒स्मि॒न् नि॒न्द्रि॒य मि॑न्द्रि॒य म॑स्मिन् नस्मिन् निन्द्रि॒यम् द॑धाति दधातीन्द्रि॒य म॑स्मिन् नस्मिन् निन्द्रि॒यम् द॑धाति । \newline
27. इ॒न्द्रि॒यम् द॑धाति दधातीन्द्रि॒य मि॑न्द्रि॒यम् द॑धाति॒ वरु॑णो॒ वरु॑णो दधातीन्द्रि॒य मि॑न्द्रि॒यम् द॑धाति॒ वरु॑णः । \newline
28. द॒धा॒ति॒ वरु॑णो॒ वरु॑णो दधाति दधाति॒ वरु॑ण एन मेनं॒ ॅवरु॑णो दधाति दधाति॒ वरु॑ण एनम् । \newline
29. वरु॑ण एन मेनं॒ ॅवरु॑णो॒ वरु॑ण एनं ॅवरुणपा॒शाद् व॑रुणपा॒शा दे॑नं॒ ॅवरु॑णो॒ वरु॑ण एनं ॅवरुणपा॒शात् । \newline
30. ए॒नं॒ ॅव॒रु॒ण॒पा॒शाद् व॑रुणपा॒शा दे॑न मेनं ॅवरुणपा॒शान् मु॑ञ्चति मुञ्चति वरुणपा॒शा दे॑न मेनं ॅवरुणपा॒शान् मु॑ञ्चति । \newline
31. व॒रु॒ण॒पा॒शान् मु॑ञ्चति मुञ्चति वरुणपा॒शाद् व॑रुणपा॒शान् मु॑ञ्चति पय॒स्या॑ पय॒स्या॑ मुञ्चति वरुणपा॒शाद् व॑रुणपा॒शान् मु॑ञ्चति पय॒स्या᳚ । \newline
32. व॒रु॒ण॒पा॒शादिति॑ वरुण - पा॒शात् । \newline
33. मु॒ञ्च॒ति॒ प॒य॒स्या॑ पय॒स्या॑ मुञ्चति मुञ्चति पय॒स्या॑ भवति भवति पय॒स्या॑ मुञ्चति मुञ्चति पय॒स्या॑ भवति । \newline
34. प॒य॒स्या॑ भवति भवति पय॒स्या॑ पय॒स्या॑ भवति॒ पयः॒ पयो॑ भवति पय॒स्या॑ पय॒स्या॑ भवति॒ पयः॑ । \newline
35. भ॒व॒ति॒ पयः॒ पयो॑ भवति भवति॒ पयो॒ हि हि पयो॑ भवति भवति॒ पयो॒ हि । \newline
36. पयो॒ हि हि पयः॒ पयो॒ हि वै वै हि पयः॒ पयो॒ हि वै । \newline
37. हि वै वै हि हि वा ए॒तस्मा॑ दे॒तस्मा॒द् वै हि हि वा ए॒तस्मा᳚त् । \newline
38. वा ए॒तस्मा॑ दे॒तस्मा॒द् वै वा ए॒तस्मा॑ दप॒क्राम॑ त्यप॒क्राम॑ त्ये॒तस्मा॒द् वै वा ए॒तस्मा॑ दप॒क्राम॑ति । \newline
39. ए॒तस्मा॑ दप॒क्राम॑ त्यप॒क्राम॑ त्ये॒तस्मा॑ दे॒तस्मा॑ दप॒क्राम॒ त्यथाथा॑ प॒क्राम॑ त्ये॒तस्मा॑ दे॒तस्मा॑ दप॒क्राम॒ त्यथ॑ । \newline
40. अ॒प॒क्राम॒ त्यथाथा॑ प॒क्राम॑ त्यप॒क्राम॒त्य थै॒ष ए॒षो ऽथा॑प॒क्राम॑ त्यप॒क्राम॒ त्यथै॒षः । \newline
41. अ॒प॒क्राम॒तीत्य॑प - क्राम॑ति । \newline
42. अथै॒ष ए॒षो ऽथाथै॒ष पा॒प्मना॑ पा॒प्मनै॒षो ऽथाथै॒ष पा॒प्मना᳚ । \newline
43. ए॒ष पा॒प्मना॑ पा॒प्मनै॒ष ए॒ष पा॒प्मना॑ गृही॒तो गृ॑ही॒तः पा॒प्मनै॒ष ए॒ष पा॒प्मना॑ गृही॒तः । \newline
44. पा॒प्मना॑ गृही॒तो गृ॑ही॒तः पा॒प्मना॑ पा॒प्मना॑ गृही॒तो यद् यद् गृ॑ही॒तः पा॒प्मना॑ पा॒प्मना॑ गृही॒तो यत् । \newline
45. गृ॒ही॒तो यद् यद् गृ॑ही॒तो गृ॑ही॒तो यत् प॑य॒स्या॑ पय॒स्या॑ यद् गृ॑ही॒तो गृ॑ही॒तो यत् प॑य॒स्या᳚ । \newline
46. यत् प॑य॒स्या॑ पय॒स्या॑ यद् यत् प॑य॒स्या॑ भव॑ति॒ भव॑ति पय॒स्या॑ यद् यत् प॑य॒स्या॑ भव॑ति । \newline
47. प॒य॒स्या॑ भव॑ति॒ भव॑ति पय॒स्या॑ पय॒स्या॑ भव॑ति॒ पयः॒ पयो॒ भव॑ति पय॒स्या॑ पय॒स्या॑ भव॑ति॒ पयः॑ । \newline
48. भव॑ति॒ पयः॒ पयो॒ भव॑ति॒ भव॑ति॒ पय॑ ए॒वैव पयो॒ भव॑ति॒ भव॑ति॒ पय॑ ए॒व । \newline
49. पय॑ ए॒वैव पयः॒ पय॑ ए॒वास्मि॑न् नस्मिन् ने॒व पयः॒ पय॑ ए॒वास्मिन्न्॑ । \newline
50. ए॒वास्मि॑न् नस्मिन् ने॒वैवास्मि॒न् तया॒ तया᳚ ऽस्मिन् ने॒वैवास्मि॒न् तया᳚ । \newline
51. अ॒स्मि॒न् तया॒ तया᳚ ऽस्मिन् नस्मि॒न् तया॑ दधाति दधाति॒ तया᳚ ऽस्मिन् नस्मि॒न् तया॑ दधाति । \newline
52. तया॑ दधाति दधाति॒ तया॒ तया॑ दधाति पय॒स्या॑याम् पय॒स्या॑याम् दधाति॒ तया॒ तया॑ दधाति पय॒स्या॑याम् । \newline
53. द॒धा॒ति॒ प॒य॒स्या॑याम् पय॒स्या॑याम् दधाति दधाति पय॒स्या॑याम् पुरो॒डाश॑म् पुरो॒डाश॑म् पय॒स्या॑याम् दधाति दधाति पय॒स्या॑याम् पुरो॒डाश᳚म् । \newline
54. प॒य॒स्या॑याम् पुरो॒डाश॑म् पुरो॒डाश॑म् पय॒स्या॑याम् पय॒स्या॑याम् पुरो॒डाश॒ मवाव॑ पुरो॒डाश॑म् पय॒स्या॑याम् पय॒स्या॑याम् पुरो॒डाश॒ मव॑ । \newline
\pagebreak
\markright{ TS 2.3.13.3  \hfill https://www.vedavms.in \hfill}

\section{ TS 2.3.13.3 }

\textbf{TS 2.3.13.3 } \newline
\textbf{Samhita Paata} \newline

पुरो॒डाश॒मव॑ दधात्यात्म॒न्वन्त॑मे॒वैनं॑ करो॒त्यथो॑ आ॒यत॑नवन्तमे॒व च॑तु॒र्द्धा व्यू॑हति दि॒क्ष्वे॑व प्रति॑तिष्ठति॒ पुनः॒ समू॑हति दि॒ग्भ्य ए॒वास्मै॑ भेष॒जं क॑रोति स॒मूह्याव॑ द्यति॒ यथाऽऽवि॑द्धं निष्कृ॒न्तति॑ ता॒दृगे॒व तद्यो वा॑मिन्द्रावरुणाव॒ग्नौ स्राम॒स्तं ॅवा॑मे॒तेनाव॑ यज॒ इत्या॑ह॒ दुरि॑ष्‌ट्या ए॒वैनं॑ पाति॒ ( ) यो वा॑ मिन्द्रा वरुणा द्वि॒पाथ्सु॑ प॒शुषु॒ स्राम॒स्तं ॅवा॑ मे॒ तेनाव॑ यज॒ इत्या॑है॒ताव॑ती॒र्वा आप॒ ओष॑धयो॒ वन॒स्पत॑यः प्र॒जाः प॒शव॑ उपजीव॒नीया॒स्ता ए॒वास्मै॑ वरुणपा॒शान्-मु॑ञ्चति ॥ \newline

\textbf{Pada Paata} \newline

पु॒रो॒डाश᳚म् । अवेति॑ । द॒धा॒ति॒ । आ॒त्म॒न्वन्त॒मित्या᳚त्मन्न् - वन्त᳚म् । ए॒व । ए॒न॒म् । क॒रो॒ति॒ । अथो॒ इति॑ । आ॒यत॑नवन्त॒मित्या॒यत॑न - व॒न्त॒म् । ए॒व । च॒तु॒र्द्धेति॑ चतुः - धा । वीति॑ । ऊ॒ह॒ति॒ । दि॒क्षु । ए॒व । प्रतीति॑ । ति॒ष्ठ॒ति॒ । पुनः॑ । समिति॑ । ऊ॒ह॒ति॒ । दि॒ग्भ्य इति॑ दिक् - भ्यः । ए॒व ।   अ॒स्मै॒ । भे॒ष॒जम् । क॒रो॒ति॒ । स॒मूह्येति॑ सं - ऊह्य॑ । अवेति॑ । द्य॒ति॒ । यथा᳚ । आवि॑द्ध॒मित्या - वि॒द्ध॒म् । नि॒ष्कृ॒न्ततीति॑ निः - कृ॒न्तति॑ । ता॒दृक् । ए॒व । तत् । यः । वा॒म् । इ॒न्द्रा॒व॒रु॒णा॒विती᳚न्द्रा - व॒रु॒णौ॒ । अ॒ग्नौ । स्रामः॑ । तम् । वा॒म् । ए॒तेन॑ । अवेति॑ । य॒जे॒ । इति॑ । आ॒ह॒ । दुरि॑ष्ट्या॒ इति॒ दुः - इ॒ष्ट्याः॒ । ए॒व । ए॒न॒म् । पा॒ति॒ ( ) । यः । वा॒म् । इ॒न्द्रा॒व॒रु॒णेती᳚न्द्रा - व॒रु॒णा॒ । द्वि॒पाथ्स्विति॑ द्वि॒पात् - सु॒ । प॒शुषु॑ । स्रामः॑ । तम् । वा॒म् ।  ए॒तेन॑ । अवेति॑ । य॒जे॒ । इति॑ । आ॒ह॒ । ए॒ताव॑तीः । वै । आपः॑ । ओष॑धयः । वन॒स्पत॑यः । प्र॒जा इति॑ प्र - जाः । प॒शवः॑ । उ॒प॒जी॒व॒नीया॒ इत्यु॑प - जी॒व॒नीयाः᳚ । ताः । ए॒व । अ॒स्मै॒ । व॒रु॒ण॒पा॒शादिति॑ वरुण - पा॒शात् । मु॒ञ्च॒ति॒ ॥  \newline


\textbf{Krama Paata} \newline

पु॒रो॒डाश॒मव॑ । अव॑ दधाति । द॒धा॒त्या॒त्म॒न्वन्त᳚म् । आ॒त्म॒न्वन्त॑मे॒व । आ॒त्म॒न्वन्त॒मित्या᳚त्मन्न् - वन्त᳚म् । ए॒वैन᳚म् । ए॒न॒म् क॒रो॒ति॒ । क॒रो॒त्यथो᳚ । अथो॑ आ॒यत॑नवन्तम् । अथो॒ इत्यथो᳚ । आ॒यत॑नवन्तमे॒व । आ॒यत॑नवन्त॒मित्या॒यत॑न - व॒न्त॒म् । ए॒व च॑तु॒र्द्धा । च॒तु॒र्द्धा वि । च॒तु॒र्द्धेति॑ चतुः - धा । व्यू॑हति । ऊ॒ह॒ति॒ दि॒क्षु । दि॒क्ष्वे॑व । ए॒व प्रति॑ । प्रति॑ तिष्ठति । ति॒ष्ठ॒ति॒ पुनः॑ । पुनः॒ सम् । समू॑हति । ऊ॒ह॒ति॒ दि॒ग्भ्यः । दि॒ग्भ्य ए॒व । दि॒ग्भ्य इति॑ दिक् - भ्यः । ए॒वास्मै᳚ । अ॒स्मै॒ भे॒ष॒जम् । भे॒ष॒जम् क॑रोति । क॒रो॒ति॒ स॒मूह्य॑ । स॒मूह्याव॑ । स॒मूह्येति॑ सं - ऊह्य॑ । अव॑ द्यति । द्य॒ति॒ यथा᳚ । यथा ऽऽवि॑द्धम् । आवि॑द्धं निष्कृ॒न्तति॑ । आवि॑द्ध॒मित्या - वि॒द्ध॒म् । नि॒ष्कृ॒न्तति॑ ता॒दृक् । नि॒ष्कृ॒न्ततीति॑ निः - कृ॒न्तति॑ । ता॒दृगे॒व । ए॒व तत् । तद् यः । यो वा᳚म् । वा॒मि॒न्द्रा॒व॒रु॒णौ॒ । इ॒न्द्रा॒व॒रु॒णा॒व॒ग्नौ । इ॒न्द्रा॒व॒रु॒णा॒विती᳚न्द्रा - व॒रु॒णौ॒ । अ॒ग्नौ स्रामः॑ । स्राम॒स्तम् । तं ॅवा᳚म् । वा॒मे॒तेन॑ । ए॒तेनाव॑ । अव॑ यजे । य॒ज॒ इति॑ । इत्या॑ह । आ॒ह॒ दुरि॑ष्ट्याः । दुरि॑ष्ट्या ए॒व । दुरि॑ष्ट्या॒ इति॒ दुः - इ॒ष्ट्याः॒ । ए॒वैन᳚म् । ए॒न॒म् पा॒ति॒ ( ) । पा॒ति॒ यः । यो वा᳚म् । वा॒मि॒न्द्रा॒व॒रु॒णा॒ । इ॒न्द्रा॒व॒रु॒णा॒ द्वि॒पाथ्सु॑ । इ॒न्द्रा॒व॒रु॒णेती᳚न्द्रा - व॒रु॒णा॒ । द्वि॒पाथ्सु॑ प॒शुषु॑ । द्वि॒पाथ्स्विति॑ द्वि॒पात् - सु॒ । प॒शुषु॒ स्रामः॑ । स्राम॒स्तम् । तं ॅवा᳚म् । वा॒मे॒तेन॑ । ए॒तेनाव॑ । अव॑ यजे । य॒ज॒ इति॑ । इत्या॑ह । आ॒है॒ताव॑तीः । ए॒ताव॑ती॒र् वै । वा आपः॑ । आप॒ ओष॑धयः । ओष॑धयो॒ वन॒स्पत॑यः । वन॒स्पत॑यः प्र॒जाः । प्र॒जाः प॒शवः॑ । प्र॒जा इति॑ प्र - जाः । प॒शव॑ उपजीव॒नीयाः᳚ । उ॒प॒जी॒व॒नीया॒स्ताः । उ॒प॒जी॒व॒नीया॒ इत्यु॑प - जी॒व॒नीयाः᳚ । ता ए॒व । ए॒वास्मै᳚ । अ॒स्मै॒ व॒रु॒ण॒पा॒शात् । व॒रु॒ण॒पा॒शान् मु॑ञ्चति । व॒रु॒ण॒पा॒शादिति॑ वरुण - पा॒शात् । मु॒ञ्च॒तीति॑ मुञ्चति । \newline

\textbf{Jatai Paata} \newline

1. पु॒रो॒डाश॒ मवाव॑ पुरो॒डाश॑म् पुरो॒डाश॒ मव॑ । \newline
2. अव॑ दधाति दधा॒ त्यवाव॑ दधाति । \newline
3. द॒धा॒ त्या॒त्म॒न्वन्त॑ मात्म॒न्वन्त॑म् दधाति दधा त्यात्म॒न्वन्त᳚म् । \newline
4. आ॒त्म॒न्वन्त॑ मे॒वै वात्म॒न्वन्त॑ मात्म॒न्वन्त॑ मे॒व । \newline
5. आ॒त्म॒न्वन्त॒मित्या᳚त्मन्न् - वन्त᳚म् । \newline
6. ए॒वैन॑ मेन मे॒वैवैन᳚म् । \newline
7. ए॒न॒म् क॒रो॒ति॒ क॒रो॒ त्ये॒न॒ मे॒न॒म् क॒रो॒ति॒ । \newline
8. क॒रो॒ त्यथो॒ अथो॑ करोति करो॒ त्यथो᳚ । \newline
9. अथो॑ आ॒यत॑नवन्त मा॒यत॑नवन्त॒ मथो॒ अथो॑ आ॒यत॑नवन्तम् । \newline
10. अथो॒ इत्यथो᳚ । \newline
11. आ॒यत॑नवन्त मे॒वै वायत॑नवन्त मा॒यत॑नवन्त मे॒व । \newline
12. आ॒यत॑नवन्त॒मित्या॒यत॑न - व॒न्त॒म् । \newline
13. ए॒व च॑तु॒र्द्धा च॑तु॒र्द्धैवैव च॑तु॒र्द्धा । \newline
14. च॒तु॒र्द्धा वि वि च॑तु॒र्द्धा च॑तु॒र्द्धा वि । \newline
15. च॒तु॒र्द्धेति॑ चतुः - धा । \newline
16. व्यू॑ह त्यूहति॒ वि व्यू॑हति । \newline
17. ऊ॒ह॒ति॒ दि॒क्षु दि॒क्षू॑ह त्यूहति दि॒क्षु । \newline
18. दि॒क्ष्वे॑वैव दि॒क्षु दि॒क्ष्वे॑व । \newline
19. ए॒व प्रति॒ प्रत्ये॒वैव प्रति॑ । \newline
20. प्रति॑ तिष्ठति तिष्ठति॒ प्रति॒ प्रति॑ तिष्ठति । \newline
21. ति॒ष्ठ॒ति॒ पुनः॒ पुन॑ स्तिष्ठति तिष्ठति॒ पुनः॑ । \newline
22. पुनः॒ सꣳ सम् पुनः॒ पुनः॒ सम् । \newline
23. स मू॑ह त्यूहति॒ सꣳ स मू॑हति । \newline
24. ऊ॒ह॒ति॒ दि॒ग्भ्यो दि॒ग्भ्य ऊ॑ह त्यूहति दि॒ग्भ्यः । \newline
25. दि॒ग्भ्य ए॒वैव दि॒ग्भ्यो दि॒ग्भ्य ए॒व । \newline
26. दि॒ग्भ्य इति॑ दिक् - भ्यः । \newline
27. ए॒वास्मा॑ अस्मा ए॒वैवास्मै᳚ । \newline
28. अ॒स्मै॒ भे॒ष॒जम् भे॑ष॒ज म॑स्मा अस्मै भेष॒जम् । \newline
29. भे॒ष॒जम् क॑रोति करोति भेष॒जम् भे॑ष॒जम् क॑रोति । \newline
30. क॒रो॒ति॒ स॒मूह्य॑ स॒मूह्य॑ करोति करोति स॒मूह्य॑ । \newline
31. स॒मूह्यावाव॑ स॒मूह्य॑ स॒मूह्याव॑ । \newline
32. स॒मूह्येति॑ सं - ऊह्य॑ । \newline
33. अव॑ द्यति द्य॒ त्यवाव॑ द्यति । \newline
34. द्य॒ति॒ यथा॒ यथा᳚ द्यति द्यति॒ यथा᳚ । \newline
35. यथा ऽऽवि॑द्ध॒ मावि॑द्धं॒ ॅयथा॒ यथा ऽऽवि॑द्धम् । \newline
36. आवि॑द्धम् निष्कृ॒न्तति॑ निष्कृ॒ न्तत्यावि॑द्ध॒ मावि॑द्धम् निष्कृ॒न्तति॑ । \newline
37. आवि॑द्ध॒मित्या - वि॒द्ध॒म् । \newline
38. नि॒ष्कृ॒न्तति॑ ता॒दृक् ता॒दृङ् नि॑ष्कृ॒न्तति॑ निष्कृ॒न्तति॑ ता॒दृक् । \newline
39. नि॒ष्कृ॒न्ततीति॑ निः - कृ॒न्तति॑ । \newline
40. ता॒दृगे॒वैव ता॒दृक् ता॒दृगे॒व । \newline
41. ए॒व तत् तदे॒वैव तत् । \newline
42. तद् यो य स्तत् तद् यः । \newline
43. यो वां᳚ ॅवां॒ ॅयो यो वा᳚म् । \newline
44. वा॒ मि॒न्द्रा॒व॒रु॒णा॒ वि॒न्द्रा॒व॒रु॒णौ॒ वां॒ ॅवा॒ मि॒न्द्रा॒व॒रु॒णौ॒ । \newline
45. इ॒न्द्रा॒व॒रु॒णा॒ व॒ग्ना व॒ग्ना वि॑न्द्रावरुणा विन्द्रावरुणा व॒ग्नौ । \newline
46. इ॒न्द्रा॒व॒रु॒णा॒विती᳚न्द्रा - व॒रु॒णौ॒ । \newline
47. अ॒ग्नौ स्रामः॒ स्रामो॒ ऽग्ना व॒ग्नौ स्रामः॑ । \newline
48. स्राम॒ स्तम् तꣳ स्रामः॒ स्राम॒ स्तम् । \newline
49. तं ॅवां᳚ ॅवा॒म् तम् तं ॅवा᳚म् । \newline
50. वा॒ मे॒तेनै॒तेन॑ वां ॅवा मे॒तेन॑ । \newline
51. ए॒तेनावा वै॒ते नै॒तेनाव॑ । \newline
52. अव॑ यजे य॒जे ऽवाव॑ यजे । \newline
53. य॒ज॒ इतीति॑ यजे यज॒ इति॑ । \newline
54. इत्या॑हा॒हे तीत्या॑ह । \newline
55. आ॒ह॒ दुरि॑ष्ट्या॒ दुरि॑ष्ट्या आहाह॒ दुरि॑ष्ट्याः । \newline
56. दुरि॑ष्ट्या ए॒वैव दुरि॑ष्ट्या॒ दुरि॑ष्ट्या ए॒व । \newline
57. दुरि॑ष्ट्या॒ इति॒ दुः - इ॒ष्ट्याः॒ । \newline
58. ए॒वैन॑ मेन मे॒वैवैन᳚म् । \newline
59. ए॒न॒म् पा॒ति॒ पा॒त्ये॒न॒ मे॒न॒म् पा॒ति॒ । \newline
60. पा॒ति॒ यो यः पा॑ति पाति॒ यः । \newline
61. यो वां᳚ ॅवां॒ ॅयो यो वा᳚म् । \newline
62. वा॒ मि॒न्द्रा॒व॒रु॒ णे॒न्द्रा॒व॒रु॒णा॒ वां॒ ॅवा॒ मि॒न्द्रा॒व॒रु॒णा॒ । \newline
63. इ॒न्द्रा॒व॒रु॒णा॒ द्वि॒पाथ्सु॑ द्वि॒पा थ्स्वि॑न्द्रावरु णेन्द्रावरुणा द्वि॒पाथ्सु॑ । \newline
64. इ॒न्द्रा॒व॒रु॒णेती᳚न्द्रा - व॒रु॒णा॒ । \newline
65. द्वि॒पाथ्सु॑ प॒शुषु॑ प॒शुषु॑ द्वि॒पाथ्सु॑ द्वि॒पाथ्सु॑ प॒शुषु॑ । \newline
66. द्वि॒पाथ्स्विति॑ द्वि॒पात् - सु॒ । \newline
67. प॒शुषु॒ स्रामः॒ स्रामः॑ प॒शुषु॑ प॒शुषु॒ स्रामः॑ । \newline
68. स्राम॒ स्तम् तꣳ स्रामः॒ स्राम॒ स्तम् । \newline
69. तं ॅवां᳚ ॅवा॒म् तम् तं ॅवा᳚म् । \newline
70. वा॒ मे॒तेनै॒तेन॑ वां ॅवा मे॒तेन॑ । \newline
71. ए॒तेनावा वै॒ते नै॒तेनाव॑ । \newline
72. अव॑ यजे य॒जे ऽवाव॑ यजे । \newline
73. य॒ज॒ इतीति॑ यजे यज॒ इति॑ । \newline
74. इत्या॑हा॒हे तीत्या॑ह । \newline
75. आ॒है॒ताव॑ती रे॒ताव॑ती राहाहै॒ताव॑तीः । \newline
76. ए॒ताव॑ती॒र् वै वा ए॒ताव॑ती रे॒ताव॑ती॒र् वै । \newline
77. वा आप॒ आपो॒ वै वा आपः॑ । \newline
78. आप॒ ओष॑धय॒ ओष॑धय॒ आप॒ आप॒ ओष॑धयः । \newline
79. ओष॑धयो॒ वन॒स्पत॑यो॒ वन॒स्पत॑य॒ ओष॑धय॒ ओष॑धयो॒ वन॒स्पत॑यः । \newline
80. वन॒स्पत॑यः प्र॒जाः प्र॒जा वन॒स्पत॑यो॒ वन॒स्पत॑यः प्र॒जाः । \newline
81. प्र॒जाः प॒शवः॑ प॒शवः॑ प्र॒जाः प्र॒जाः प॒शवः॑ । \newline
82. प्र॒जा इति॑ प्र - जाः । \newline
83. प॒शव॑ उपजीव॒नीया॑ उपजीव॒नीयाः᳚ प॒शवः॑ प॒शव॑ उपजीव॒नीयाः᳚ । \newline
84. उ॒प॒जी॒व॒नीया॒ स्ता स्ता उ॑पजीव॒नीया॑ उपजीव॒नीया॒ स्ताः । \newline
85. उ॒प॒जी॒व॒नीया॒ इत्यु॑प - जी॒व॒नीयाः᳚ । \newline
86. ता ए॒वैव ता स्ता ए॒व । \newline
87. ए॒वास्मा॑ अस्मा ए॒वैवास्मै᳚ । \newline
88. अ॒स्मै॒ व॒रु॒ण॒पा॒शाद् व॑रुणपा॒शा द॑स्मा अस्मै वरुणपा॒शात् । \newline
89. व॒रु॒ण॒पा॒शान् मु॑ञ्चति मुञ्चति वरुणपा॒शाद् व॑रुणपा॒शान् मु॑ञ्चति । \newline
90. व॒रु॒ण॒पा॒शादिति॑ वरुण - पा॒शात् । \newline
91. मु॒ञ्च॒तीति॑ मुञ्चति । \newline

\textbf{Ghana Paata } \newline

1. पु॒रो॒डाश॒ मवाव॑ पुरो॒डाश॑म् पुरो॒डाश॒ मव॑ दधाति दधा॒त्यव॑ पुरो॒डाश॑म् पुरो॒डाश॒ मव॑ दधाति । \newline
2. अव॑ दधाति दधा॒ त्यवाव॑ दधा त्यात्म॒न्वन्त॑ मात्म॒न्वन्त॑म् दधा॒त्यवाव॑ दधा त्यात्म॒न्वन्त᳚म् । \newline
3. द॒धा॒ त्या॒त्म॒न्वन्त॑ मात्म॒न्वन्त॑म् दधाति दधा त्यात्म॒न्वन्त॑ मे॒वैवात्म॒न्वन्त॑म् दधाति दधा त्यात्म॒न्वन्त॑ मे॒व । \newline
4. आ॒त्म॒न्वन्त॑ मे॒वैवात्म॒न्वन्त॑ मात्म॒न्वन्त॑ मे॒वैन॑ मेन मे॒वात्म॒न्वन्त॑ मात्म॒न्वन्त॑ मे॒वैन᳚म् । \newline
5. आ॒त्म॒न्वन्त॒मित्या᳚त्मन्न् - वन्त᳚म् । \newline
6. ए॒वैन॑ मेन मे॒वैवैन॑म् करोति करो त्येन मे॒वैवैन॑म् करोति । \newline
7. ए॒न॒म् क॒रो॒ति॒ क॒रो॒ त्ये॒न॒ मे॒न॒म् क॒रो॒ त्यथो॒ अथो॑ करो त्येन मेनम् करो॒ त्यथो᳚ । \newline
8. क॒रो॒ त्यथो॒ अथो॑ करोति करो॒ त्यथो॑ आ॒यत॑नवन्त मा॒यत॑नवन्त॒ मथो॑ करोति करो॒ त्यथो॑ आ॒यत॑नवन्तम् । \newline
9. अथो॑ आ॒यत॑नवन्त मा॒यत॑नवन्त॒ मथो॒ अथो॑ आ॒यत॑नवन्त मे॒वैवायत॑नवन्त॒ मथो॒ अथो॑ आ॒यत॑नवन्त मे॒व । \newline
10. अथो॒ इत्यथो᳚ । \newline
11. आ॒यत॑नवन्त मे॒वैवायत॑नवन्त मा॒यत॑नवन्त मे॒व च॑तु॒र्द्धा च॑तु॒र्द्धै वायत॑नवन्त मा॒यत॑नवन्त मे॒व च॑तु॒र्द्धा । \newline
12. आ॒यत॑नवन्त॒मित्या॒यत॑न - व॒न्त॒म् । \newline
13. ए॒व च॑तु॒र्द्धा च॑तु॒र्द्धैवैव च॑तु॒र्द्धा वि वि च॑तु॒र्द्धैवैव च॑तु॒र्द्धा वि । \newline
14. च॒तु॒र्द्धा वि वि च॑तु॒र्द्धा च॑तु॒र्द्धा व्यू॑ह त्यूहति॒ वि च॑तु॒र्द्धा च॑तु॒र्द्धा व्यू॑हति । \newline
15. च॒तु॒र्द्धेति॑ चतुः - धा । \newline
16. व्यू॑ह त्यूहति॒ वि व्यू॑हति दि॒क्षु दि॒क्षूह॑ति॒ वि व्यू॑हति दि॒क्षु । \newline
17. ऊ॒ह॒ति॒ दि॒क्षु दि॒क्षू॑ह त्यूहति दि॒क्ष्वे॑वैव दि॒क्षू॑ह त्यूहति दि॒क्ष्वे॑व । \newline
18. दि॒क्ष्वे॑वैव दि॒क्षु दि॒क्ष्वे॑व प्रति॒ प्रत्ये॒व दि॒क्षु दि॒क्ष्वे॑व प्रति॑ । \newline
19. ए॒व प्रति॒ प्रत्ये॒वैव प्रति॑ तिष्ठति तिष्ठति॒ प्रत्ये॒वैव प्रति॑ तिष्ठति । \newline
20. प्रति॑ तिष्ठति तिष्ठति॒ प्रति॒ प्रति॑ तिष्ठति॒ पुनः॒ पुन॑ स्तिष्ठति॒ प्रति॒ प्रति॑ तिष्ठति॒ पुनः॑ । \newline
21. ति॒ष्ठ॒ति॒ पुनः॒ पुन॑ स्तिष्ठति तिष्ठति॒ पुनः॒ सꣳ सम् पुन॑ स्तिष्ठति तिष्ठति॒ पुनः॒ सम् । \newline
22. पुनः॒ सꣳ सम् पुनः॒ पुनः॒ स मू॑ह त्यूहति॒ सम् पुनः॒ पुनः॒ स मू॑हति । \newline
23. स मू॑हत्यूहति॒ सꣳ स मू॑हति दि॒ग्भ्यो दि॒ग्भ्य ऊ॑हति॒ सꣳ स मू॑हति दि॒ग्भ्यः । \newline
24. ऊ॒ह॒ति॒ दि॒ग्भ्यो दि॒ग्भ्य ऊ॑हत्यूहति दि॒ग्भ्य ए॒वैव दि॒ग्भ्य ऊ॑ह त्यूहति दि॒ग्भ्य ए॒व । \newline
25. दि॒ग्भ्य ए॒वैव दि॒ग्भ्यो दि॒ग्भ्य ए॒वास्मा॑ अस्मा ए॒व दि॒ग्भ्यो दि॒ग्भ्य ए॒वास्मै᳚ । \newline
26. दि॒ग्भ्य इति॑ दिक् - भ्यः । \newline
27. ए॒वास्मा॑ अस्मा ए॒वैवास्मै॑ भेष॒जम् भे॑ष॒ज म॑स्मा ए॒वैवास्मै॑ भेष॒जम् । \newline
28. अ॒स्मै॒ भे॒ष॒जम् भे॑ष॒ज म॑स्मा अस्मै भेष॒जम् क॑रोति करोति भेष॒ज म॑स्मा अस्मै भेष॒जम् क॑रोति । \newline
29. भे॒ष॒जम् क॑रोति करोति भेष॒जम् भे॑ष॒जम् क॑रोति स॒मूह्य॑ स॒मूह्य॑ करोति भेष॒जम् भे॑ष॒जम् क॑रोति स॒मूह्य॑ । \newline
30. क॒रो॒ति॒ स॒मूह्य॑ स॒मूह्य॑ करोति करोति स॒मूह्यावाव॑ स॒मूह्य॑ करोति करोति स॒मूह्याव॑ । \newline
31. स॒मूह्यावाव॑ स॒मूह्य॑ स॒मूह्याव॑ द्यति द्य॒त्यव॑ स॒मूह्य॑ स॒मूह्याव॑ द्यति । \newline
32. स॒मूह्येति॑ सं - ऊह्य॑ । \newline
33. अव॑ द्यति द्य॒त्यवाव॑ द्यति॒ यथा॒ यथा᳚ द्य॒त्यवाव॑ द्यति॒ यथा᳚ । \newline
34. द्य॒ति॒ यथा॒ यथा᳚ द्यति द्यति॒ यथा ऽऽवि॑द्ध॒ मावि॑द्धं॒ ॅयथा᳚ द्यति द्यति॒ यथा ऽऽवि॑द्धम् । \newline
35. यथा ऽऽवि॑द्ध॒ मावि॑द्धं॒ ॅयथा॒ यथा ऽऽवि॑द्धम् निष्कृ॒न्तति॑ निष्कृ॒ न्तत्यावि॑द्धं॒ ॅयथा॒ यथा ऽऽवि॑द्धम् निष्कृ॒न्तति॑ । \newline
36. आवि॑द्धम् निष्कृ॒न्तति॑ निष्कृ॒न्त त्यावि॑द्ध॒ मावि॑द्धम् निष्कृ॒न्तति॑ ता॒दृक् ता॒दृङ् नि॑ष्कृ॒न्त त्यावि॑द्ध॒ मावि॑द्धम् निष्कृ॒न्तति॑ ता॒दृक् । \newline
37. आवि॑द्ध॒मित्या - वि॒द्ध॒म् । \newline
38. नि॒ष्कृ॒न्तति॑ ता॒दृक् ता॒दृङ् नि॑ष्कृ॒न्तति॑ निष्कृ॒न्तति॑ ता॒दृगे॒वैव ता॒दृङ् नि॑ष्कृ॒न्तति॑ निष्कृ॒न्तति॑ ता॒दृगे॒व । \newline
39. नि॒ष्कृ॒न्ततीति॑ निः - कृ॒न्तति॑ । \newline
40. ता॒दृगे॒वैव ता॒दृक् ता॒दृगे॒व तत् तदे॒व ता॒दृक् ता॒दृगे॒व तत् । \newline
41. ए॒व तत् तदे॒वैव तद् यो य स्तदे॒वैव तद् यः । \newline
42. तद् यो य स्तत् तद् यो वां᳚ ॅवां॒ ॅय स्तत् तद् यो वा᳚म् । \newline
43. यो वां᳚ ॅवां॒ ॅयो यो वा॑ मिन्द्रावरुणा विन्द्रावरुणौ वां॒ ॅयो यो वा॑ मिन्द्रावरुणौ । \newline
44. वा॒ मि॒न्द्रा॒व॒रु॒णा॒ वि॒न्द्रा॒व॒रु॒णौ॒ वां॒ ॅवा॒ मि॒न्द्रा॒व॒रु॒णा॒ व॒ग्ना व॒ग्ना वि॑न्द्रावरुणौ वां ॅवा मिन्द्रावरुणा व॒ग्नौ । \newline
45. इ॒न्द्रा॒व॒रु॒णा॒ व॒ग्ना व॒ग्ना वि॑न्द्रावरुणा विन्द्रावरुणा व॒ग्नौ स्रामः॒ स्रामो॒ ऽग्ना वि॑न्द्रावरुणा विन्द्रावरुणा व॒ग्नौ स्रामः॑ । \newline
46. इ॒न्द्रा॒व॒रु॒णा॒विती᳚न्द्रा - व॒रु॒णौ॒ । \newline
47. अ॒ग्नौ स्रामः॒ स्रामो॒ ऽग्ना व॒ग्नौ स्राम॒ स्तम् तꣳ स्रामो॒ ऽग्ना व॒ग्नौ स्राम॒ स्तम् । \newline
48. स्राम॒ स्तम् तꣳ स्रामः॒ स्राम॒स्तं ॅवां᳚ ॅवा॒म् तꣳ स्रामः॒ स्राम॒ स्तं ॅवा᳚म् । \newline
49. तं ॅवां᳚ ॅवा॒म् तम् तं ॅवा॑ मे॒तेनै॒तेन॑ वा॒म् तम् तं ॅवा॑ मे॒तेन॑ । \newline
50. वा॒ मे॒तेनै॒तेन॑ वां ॅवा मे॒तेनावा वै॒तेन॑ वां ॅवा मे॒तेनाव॑ । \newline
51. ए॒तेनावा वै॒तेनै॒तेनाव॑ यजे य॒जे ऽवै॒तेनै॒तेनाव॑ यजे । \newline
52. अव॑ यजे य॒जे ऽवाव॑ यज॒ इतीति॑ य॒जे ऽवाव॑ यज॒ इति॑ । \newline
53. य॒ज॒ इतीति॑ यजे यज॒ इत्या॑हा॒हे ति॑ यजे यज॒ इत्या॑ह । \newline
54. इत्या॑हा॒हे तीत्या॑ह॒ दुरि॑ष्ट्या॒ दुरि॑ष्ट्या आ॒हे तीत्या॑ह॒ दुरि॑ष्ट्याः । \newline
55. आ॒ह॒ दुरि॑ष्ट्या॒ दुरि॑ष्ट्या आहाह॒ दुरि॑ष्ट्या ए॒वैव दुरि॑ष्ट्या आहाह॒ दुरि॑ष्ट्या ए॒व । \newline
56. दुरि॑ष्ट्या ए॒वैव दुरि॑ष्ट्या॒ दुरि॑ष्ट्या ए॒वैन॑ मेन मे॒व दुरि॑ष्ट्या॒ दुरि॑ष्ट्या ए॒वैन᳚म् । \newline
57. दुरि॑ष्ट्या॒ इति॒ दुः - इ॒ष्ट्याः॒ । \newline
58. ए॒वैन॑ मेन मे॒वैवैन॑म् पाति पात्येन मे॒वैवैन॑म् पाति । \newline
59. ए॒न॒म् पा॒ति॒ पा॒त्ये॒न॒ मे॒न॒म् पा॒ति॒ यो यः पा᳚त्येन मेनम् पाति॒ यः । \newline
60. पा॒ति॒ यो यः पा॑ति पाति॒ यो वां᳚ ॅवां॒ ॅयः पा॑ति पाति॒ यो वा᳚म् । \newline
61. यो वां᳚ ॅवां॒ ॅयो यो वा॑ मिन्द्रावरु णेन्द्रावरुणा वां॒ ॅयो यो वा॑ मिन्द्रावरुणा । \newline
62. वा॒ मि॒न्द्रा॒व॒रु॒ णे॒न्द्रा॒व॒रु॒णा॒ वां॒ ॅवा॒ मि॒न्द्रा॒व॒रु॒णा॒ द्वि॒पाथ्सु॑ द्वि॒पाथ् स्वि॑न्द्रावरुणा वां ॅवा मिन्द्रावरुणा द्वि॒पाथ्सु॑ । \newline
63. इ॒न्द्रा॒व॒रु॒णा॒ द्वि॒पाथ्सु॑ द्वि॒पाथ् स्वि॑न्द्रावरु णेन्द्रावरुणा द्वि॒पाथ्सु॑ प॒शुषु॑ प॒शुषु॑ द्वि॒पाथ्स्वि॑न्द्रावरु णेन्द्रावरुणा द्वि॒पाथ्सु॑ प॒शुषु॑ । \newline
64. इ॒न्द्रा॒व॒रु॒णेती᳚न्द्रा - व॒रु॒णा॒ । \newline
65. द्वि॒पाथ्सु॑ प॒शुषु॑ प॒शुषु॑ द्वि॒पाथ्सु॑ द्वि॒पाथ्सु॑ प॒शुषु॒ स्रामः॒ स्रामः॑ प॒शुषु॑ द्वि॒पाथ्सु॑ द्वि॒पाथ्सु॑ प॒शुषु॒ स्रामः॑ । \newline
66. द्वि॒पाथ्स्विति॑ द्वि॒पात् - सु॒ । \newline
67. प॒शुषु॒ स्रामः॒ स्रामः॑ प॒शुषु॑ प॒शुषु॒ स्राम॒ स्तम् तꣳ स्रामः॑ प॒शुषु॑ प॒शुषु॒ स्राम॒ स्तम् । \newline
68. स्राम॒ स्तम् तꣳ स्रामः॒ स्राम॒ स्तं ॅवां᳚ ॅवा॒म् तꣳ स्रामः॒ स्राम॒ स्तं ॅवा᳚म् । \newline
69. तं ॅवां᳚ ॅवा॒म् तम् तं ॅवा॑ मे॒तेनै॒तेन॑ वा॒म् तम् तं ॅवा॑ मे॒तेन॑ । \newline
70. वा॒ मे॒तेनै॒तेन॑ वां ॅवा मे॒तेनावा वै॒तेन॑ वां ॅवा मे॒तेनाव॑ । \newline
71. ए॒तेनावा वै॒तेनै॒तेनाव॑ यजे य॒जे ऽवै॒तेनै॒तेनाव॑ यजे । \newline
72. अव॑ यजे य॒जे ऽवाव॑ यज॒ इतीति॑ य॒जे ऽवाव॑ यज॒ इति॑ । \newline
73. य॒ज॒ इतीति॑ यजे यज॒ इत्या॑हा॒हे ति॑ यजे यज॒ इत्या॑ह । \newline
74. इत्या॑हा॒हे ती त्या॑है॒ताव॑ती रे॒ताव॑ती रा॒हे तीत्या॑है॒ताव॑तीः । \newline
75. आ॒है॒ताव॑ती रे॒ताव॑ती राहा है॒ताव॑ती॒र् वै वा ए॒ताव॑ती राहा है॒ताव॑ती॒र् वै । \newline
76. ए॒ताव॑ती॒र् वै वा ए॒ताव॑ती रे॒ताव॑ती॒र् वा आप॒ आपो॒ वा ए॒ताव॑ती रे॒ताव॑ती॒र् वा आपः॑ । \newline
77. वा आप॒ आपो॒ वै वा आप॒ ओष॑धय॒ ओष॑धय॒ आपो॒ वै वा आप॒ ओष॑धयः । \newline
78. आप॒ ओष॑धय॒ ओष॑धय॒ आप॒ आप॒ ओष॑धयो॒ वन॒स्पत॑यो॒ वन॒स्पत॑य॒ ओष॑धय॒ आप॒ आप॒ ओष॑धयो॒ वन॒स्पत॑यः । \newline
79. ओष॑धयो॒ वन॒स्पत॑यो॒ वन॒स्पत॑य॒ ओष॑धय॒ ओष॑धयो॒ वन॒स्पत॑यः प्र॒जाः प्र॒जा वन॒स्पत॑य॒ ओष॑धय॒ ओष॑धयो॒ वन॒स्पत॑यः प्र॒जाः । \newline
80. वन॒स्पत॑यः प्र॒जाः प्र॒जा वन॒स्पत॑यो॒ वन॒स्पत॑यः प्र॒जाः प॒शवः॑ प॒शवः॑ प्र॒जा वन॒स्पत॑यो॒ वन॒स्पत॑यः प्र॒जाः प॒शवः॑ । \newline
81. प्र॒जाः प॒शवः॑ प॒शवः॑ प्र॒जाः प्र॒जाः प॒शव॑ उपजीव॒नीया॑ उपजीव॒नीयाः᳚ प॒शवः॑ प्र॒जाः प्र॒जाः प॒शव॑ उपजीव॒नीयाः᳚ । \newline
82. प्र॒जा इति॑ प्र - जाः । \newline
83. प॒शव॑ उपजीव॒नीया॑ उपजीव॒नीयाः᳚ प॒शवः॑ प॒शव॑ उपजीव॒नीया॒ स्ता स्ता उ॑पजीव॒नीयाः᳚ प॒शवः॑ प॒शव॑ उपजीव॒नीया॒ स्ताः । \newline
84. उ॒प॒जी॒व॒नीया॒ स्ता स्ता उ॑पजीव॒नीया॑ उपजीव॒नीया॒ स्ता ए॒वैव ता उ॑पजीव॒नीया॑ उपजीव॒नीया॒ स्ता ए॒व । \newline
85. उ॒प॒जी॒व॒नीया॒ इत्यु॑प - जी॒व॒नीयाः᳚ । \newline
86. ता ए॒वैव ता स्ता ए॒वास्मा॑ अस्मा ए॒व ता स्ता ए॒वास्मै᳚ । \newline
87. ए॒वास्मा॑ अस्मा ए॒वैवास्मै॑ वरुणपा॒शाद् व॑रुणपा॒शा द॑स्मा ए॒वैवास्मै॑ वरुणपा॒शात् । \newline
88. अ॒स्मै॒ व॒रु॒ण॒पा॒शाद् व॑रुणपा॒शा द॑स्मा अस्मै वरुणपा॒शान् मु॑ञ्चति मुञ्चति वरुणपा॒शा द॑स्मा अस्मै वरुणपा॒शान् मु॑ञ्चति । \newline
89. व॒रु॒ण॒पा॒शान् मु॑ञ्चति मुञ्चति वरुणपा॒शाद् व॑रुणपा॒शान् मु॑ञ्चति । \newline
90. व॒रु॒ण॒पा॒शादिति॑ वरुण - पा॒शात् । \newline
91. मु॒ञ्च॒तीति॑ मुञ्चति । \newline
\pagebreak
\markright{ TS 2.3.14.1  \hfill https://www.vedavms.in \hfill}

\section{ TS 2.3.14.1 }

\textbf{TS 2.3.14.1 } \newline
\textbf{Samhita Paata} \newline

स प्र॑त्न॒व >1, न्नि काव्ये >2, न्द्रं॑ ॅवो वि॒श्वत॒स्परी>3, न्द्रं॒ नरः॑>4 ॥त्वं नः॑ सोम वि॒श्वतो॒ रक्षा॑ राजन्नघाय॒तः । न रि॑ष्ये॒त् त्वाव॑तः॒ सखाः᳚ ॥या ते॒ धामा॑नि दि॒वि या पृ॑थि॒व्यां ॅया पर्व॑ते॒ष्वोष॑धीष्व॒फ्सु ॥ तेभि॑र्नो॒ विश्वः᳚ सु॒मना॒ अहे॑ड॒न् राजन्᳚थ्-सोम॒ प्रति॑ ह॒व्या गृ॑भाय ॥ अग्नी॑षोमा॒ सवे॑दसा॒ सहू॑ती वनतं॒ गिरः॑ । सं दे॑व॒त्रा ब॑भूवथुः ॥ यु॒व - [  ] \newline

\textbf{Pada Paata} \newline

सः । प्र॒त्न॒वदिति॑ प्रत्न - वत् । नीति॑ । काव्या᳚ । इन्द्र᳚म् । वः॒ । वि॒श्वतः॑ । परीति॑ । इन्द्र᳚म् । नरः॑ ॥ त्वम् । नः॒ । सो॒म॒ । वि॒श्वतः॑ । रक्ष॑ । रा॒ज॒न्न् । अ॒घा॒य॒त इत्य॑घ - य॒तः ॥ न । रि॒ष्ये॒त् । त्वाव॑त॒ इति॒ त्व - व॒तः॒ । सखा᳚ ॥ या । ते॒ । धामा॑नि । दि॒वि । या । पृ॒थि॒व्याम् । या । पर्व॑तेषु । ओष॑धीषु । अ॒फ्सित्य॑प् - सु ॥ तेभिः॑ । नः॒ । विश्वैः᳚ । सु॒मना॒ इति॑ सु-मनाः᳚ । अहे॑डन्न् । राजन्न्॑ । सो॒म॒ । प्रतीति॑ । ह॒व्या । गृ॒भा॒य॒ ॥ अग्नी॑षो॒मेत्यग्नी᳚ - सो॒मा॒ । सवे॑द॒सेति॒ स - वे॒द॒सा॒ । सहू॑ती॒ इति॒ स - हू॒ती॒ । व॒न॒त॒म् । गिरः॑ ॥ समिति॑ । दे॒व॒त्रेति॑ देव - त्रा । ब॒भू॒व॒थुः॒ ॥ यु॒वम् ।  \newline


\textbf{Krama Paata} \newline

स प्र॑त्न॒वत् । प्र॒त्न॒वन्नि । प्र॒त्न॒वदिति॑ प्रत्न - वत् । नि काव्या᳚ । काव्येन्द्र᳚म् । इन्द्रं॑ ॅवः । वो॒ वि॒श्वतः॑ । वि॒श्वत॒स्परि॑ । परीन्द्र᳚म् । इन्द्र॒म् नरः॑ । नर॒ इति॒ नरः॑ ॥ त्वम् नः॑ । नः॒ सो॒म॒ । सो॒म॒ वि॒श्वतः॑ । वि॒श्वतो॒ रक्ष॑ । रक्षा॑ राजन्न् । रा॒ज॒न्न॒घा॒य॒तः । अ॒घा॒य॒त इत्य॑घ - य॒तः ॥ न रि॑ष्येत् । रि॒ष्ये॒त् त्वाव॑तः । त्वाव॑तः॒ सखा᳚ । त्वाव॑त॒ इति॒ त्व - व॒तः॒ । सखेति॒ सखा᳚ ॥ या ते᳚ । ते॒ धामा॑नि । धामा॑नि दि॒वि । दि॒वि या । या पृ॑थि॒व्याम् । पृ॒थि॒व्यां ॅया । या पर्व॑तेषु । पर्व॑ते॒ष्वोष॑धीषु । ओष॑धीष्व॒प्सु । अ॒फ्स्वित्य॑प् - सु ॥ तेभि॑र् नः । नो॒ विश्वैः᳚ । विश्वैः᳚ सु॒मनाः᳚ । सु॒मना॒ अहे॑डन्न् । सु॒मना॒ इति॑ सु - मनाः᳚ । अहे॑ड॒न् राजन्न्॑ । राज᳚न्थ् सोम । सो॒म॒ प्रति॑ । प्रति॑ ह॒व्या । ह॒व्या गृ॑भाय । गृ॒भा॒येति॑ गृभाय ॥ अग्नी॑षोमा॒ सवे॑दसा । अग्नी॑षो॒मेत्यग्नी᳚ - सो॒मा॒ । सवे॑दसा॒ सहू॑ती । सवे॑द॒सेति॒ स - वे॒द॒सा॒ । सहू॑ती वनतम् । सहू॑ती॒ इति॒ स - हू॒ती॒ । व॒न॒त॒म् गिरः॑ । गिर॒ इति॒ गिरः॑ ॥ सम् दे॑व॒त्रा । दे॒व॒त्रा ब॑भूवथुः । दे॒व॒त्रेति॑ देव - त्रा । ब॒भू॒व॒थु॒रिति॑ बभूवथुः ॥ यु॒वमे॒तानि॑ \newline

\textbf{Jatai Paata} \newline

1. स प्र॑त्न॒वत् प्र॑त्न॒वथ् स स प्र॑त्न॒वत् । \newline
2. प्र॒त्न॒वन् नि नि प्र॑त्न॒वत् प्र॑त्न॒वन् नि । \newline
3. प्र॒त्न॒वदिति॑ प्रत्न - वत् । \newline
4. नि काव्या॒ काव्या॒ नि नि काव्या᳚ । \newline
5. काव्येन्द्र॒ मिन्द्र॒म् काव्या॒ काव्येन्द्र᳚म् । \newline
6. इन्द्रं॑ ॅवो व॒ इन्द्र॒ मिन्द्रं॑ ॅवः । \newline
7. वो॒ वि॒श्वतो॑ वि॒श्वतो॑ वो वो वि॒श्वतः॑ । \newline
8. वि॒श्वत॒ स्परि॒ परि॑ वि॒श्वतो॑ वि॒श्वत॒ स्परि॑ । \newline
9. परीन्द्र॒ मिन्द्र॒म् परि॒ परीन्द्र᳚म् । \newline
10. इन्द्र॒म् नरो॒ नर॒ इन्द्र॒ मिन्द्र॒म् नरः॑ । \newline
11. नर॒ इति॒ नरः॑ । \newline
12. त्वम् नो॑ न॒ स्त्वम् त्वम् नः॑ । \newline
13. नः॒ सो॒म॒ सो॒म॒ नो॒ नः॒ सो॒म॒ । \newline
14. सो॒म॒ वि॒श्वतो॑ वि॒श्वतः॑ सोम सोम वि॒श्वतः॑ । \newline
15. वि॒श्वतो॒ रक्ष॒ रक्ष॑ वि॒श्वतो॑ वि॒श्वतो॒ रक्ष॑ । \newline
16. रक्षा॑ राजन् राज॒न् रक्ष॒ रक्षा॑ राजन्न् । \newline
17. रा॒ज॒न् न॒घा॒य॒तो अ॑घाय॒तो रा॑जन् राजन् नघाय॒तः । \newline
18. अ॒घा॒य॒त इत्य॑घ - य॒तः । \newline
19. न रि॑ष्येद् रिष्ये॒न् न न रि॑ष्येत् । \newline
20. रि॒ष्ये॒त् त्वाव॑त॒ स्त्वाव॑तो रिष्येद् रिष्ये॒त् त्वाव॑तः । \newline
21. त्वाव॑तः॒ सखा॒ सखा॒ त्वाव॑त॒ स्त्वाव॑तः॒ सखा᳚ । \newline
22. त्वाव॑त॒ इति॒ त्व - व॒तः॒ । \newline
23. सखेति॒ सखा᳚ । \newline
24. या ते॑ ते॒ या या ते᳚ । \newline
25. ते॒ धामा॑नि॒ धामा॑नि ते ते॒ धामा॑नि । \newline
26. धामा॑नि दि॒वि दि॒वि धामा॑नि॒ धामा॑नि दि॒वि । \newline
27. दि॒वि या या दि॒वि दि॒वि या । \newline
28. या पृ॑थि॒व्याम् पृ॑थि॒व्यां ॅया या पृ॑थि॒व्याम् । \newline
29. पृ॒थि॒व्यां ॅया या पृ॑थि॒व्याम् पृ॑थि॒व्यां ॅया । \newline
30. या पर्व॑तेषु॒ पर्व॑तेषु॒ या या पर्व॑तेषु । \newline
31. पर्व॑ते॒ ष्वोष॑धी॒ ष्वोष॑धीषु॒ पर्व॑तेषु॒ पर्व॑ते॒ ष्वोष॑धीषु । \newline
32. ओष॑धी ष्व॒फ्स्व॑फ् स्वोष॑धी॒ ष्वोष॑धी ष्व॒फ्सु । \newline
33. अ॒फ्सित्य॑प् - सु । \newline
34. तेभि॑र् नो न॒ स्तेभि॒ स्तेभि॑र् नः । \newline
35. नो॒ विश्वै॒र् विश्वै᳚र् नो नो॒ विश्वैः᳚ । \newline
36. विश्वैः᳚ सु॒मनाः᳚ सु॒मना॒ विश्वै॒र् विश्वैः᳚ सु॒मनाः᳚ । \newline
37. सु॒मना॒ अहे॑ड॒न् नहे॑डन् थ्सु॒मनाः᳚ सु॒मना॒ अहे॑डन्न् । \newline
38. सु॒मना॒ इति॑ सु - मनाः᳚ । \newline
39. अहे॑ड॒न् राज॒न् राज॒न् नहे॑ड॒न् नहे॑ड॒न् राजन्न्॑ । \newline
40. राजन्᳚ थ्सोम सोम॒ राज॒न् राजन्᳚ थ्सोम । \newline
41. सो॒म॒ प्रति॒ प्रति॑ षोम सोम॒ प्रति॑ । \newline
42. प्रति॑ ह॒व्या ह॒व्या प्रति॒ प्रति॑ ह॒व्या । \newline
43. ह॒व्या गृ॑भाय गृभाय ह॒व्या ह॒व्या गृ॑भाय । \newline
44. गृ॒भा॒येति॑ गृभाय । \newline
45. अग्नी॑षोमा॒ सवे॑दसा॒ सवे॑द॒सा ऽग्नी॑षो॒मा ऽग्नी॑षोमा॒ सवे॑दसा । \newline
46. अग्नी॑षो॒मेत्यग्नी᳚ - सो॒मा॒ । \newline
47. सवे॑दसा॒ सहू॑ती॒ सहू॑ती॒ सवे॑दसा॒ सवे॑दसा॒ सहू॑ती । \newline
48. सवे॑द॒सेति॒ स - वे॒द॒सा॒ । \newline
49. सहू॑ती वनतं ॅवनतꣳ॒॒ सहू॑ती॒ सहू॑ती वनतम् । \newline
50. सहू॑ती॒ इति॒ स - हू॒ती॒ । \newline
51. व॒न॒त॒म् गिरो॒ गिरो॑ वनतं ॅवनत॒म् गिरः॑ । \newline
52. गिर॒ इति॒ गिरः॑ । \newline
53. सम् दे॑व॒त्रा दे॑व॒त्रा सꣳ सम् दे॑व॒त्रा । \newline
54. दे॒व॒त्रा ब॑भूवथुर् बभूवथुर् देव॒त्रा दे॑व॒त्रा ब॑भूवथुः । \newline
55. दे॒व॒त्रेति॑ देव - त्रा । \newline
56. ब॒भू॒व॒थु॒रिति॑ बभूवथुः । \newline
57. यु॒व मे॒ता न्ये॒तानि॑ यु॒वं ॅयु॒व मे॒तानि॑ । \newline

\textbf{Ghana Paata } \newline

1. स प्र॑त्न॒वत् प्र॑त्न॒वथ् स स प्र॑त्न॒वन् नि नि प्र॑त्न॒वथ् स स प्र॑त्न॒वन् नि । \newline
2. प्र॒त्न॒वन् नि नि प्र॑त्न॒वत् प्र॑त्न॒वन् नि काव्या॒ काव्या॒ नि प्र॑त्न॒वत् प्र॑त्न॒वन् नि काव्या᳚ । \newline
3. प्र॒त्न॒वदिति॑ प्रत्न - वत् । \newline
4. नि काव्या॒ काव्या॒ नि नि काव्येन्द्र॒ मिन्द्र॒म् काव्या॒ नि नि काव्येन्द्र᳚म् । \newline
5. काव्येन्द्र॒ मिन्द्र॒म् काव्या॒ काव्येन्द्रं॑ ॅवो व॒ इन्द्र॒म् काव्या॒ काव्येन्द्रं॑ ॅवः । \newline
6. इन्द्रं॑ ॅवो व॒ इन्द्र॒ मिन्द्रं॑ ॅवो वि॒श्वतो॑ वि॒श्वतो॑ व॒ इन्द्र॒ मिन्द्रं॑ ॅवो वि॒श्वतः॑ । \newline
7. वो॒ वि॒श्वतो॑ वि॒श्वतो॑ वो वो वि॒श्वत॒ स्परि॒ परि॑ वि॒श्वतो॑ वो वो वि॒श्वत॒ स्परि॑ । \newline
8. वि॒श्वत॒ स्परि॒ परि॑ वि॒श्वतो॑ वि॒श्वत॒ स्परीन्द्र॒ मिन्द्र॒म् परि॑ वि॒श्वतो॑ वि॒श्वत॒ स्परीन्द्र᳚म् । \newline
9. परीन्द्र॒ मिन्द्र॒म् परि॒ परीन्द्र॒म् नरो॒ नर॒ इन्द्र॒म् परि॒ परीन्द्र॒म् नरः॑ । \newline
10. इन्द्र॒म् नरो॒ नर॒ इन्द्र॒ मिन्द्र॒म् नरः॑ । \newline
11. नर॒ इति॒ नरः॑ । \newline
12. त्वम् नो॑ न॒स्त्वम् त्वम् नः॑ सोम सोम न॒स्त्वम् त्वम् नः॑ सोम । \newline
13. नः॒ सो॒म॒ सो॒म॒ नो॒ नः॒ सो॒म॒ वि॒श्वतो॑ वि॒श्वतः॑ सोम नो नः सोम वि॒श्वतः॑ । \newline
14. सो॒म॒ वि॒श्वतो॑ वि॒श्वतः॑ सोम सोम वि॒श्वतो॒ रक्ष॒ रक्ष॑ वि॒श्वतः॑ सोम सोम वि॒श्वतो॒ रक्ष॑ । \newline
15. वि॒श्वतो॒ रक्ष॒ रक्ष॑ वि॒श्वतो॑ वि॒श्वतो॒ रक्षा॑ राजन् राज॒न् रक्ष॑ वि॒श्वतो॑ वि॒श्वतो॒ रक्षा॑ राजन्न् । \newline
16. रक्षा॑ राजन् राज॒न् रक्ष॒ रक्षा॑ राजन् नघाय॒तो अ॑घाय॒तो रा॑ज॒न् रक्ष॒ रक्षा॑ राजन् नघाय॒तः । \newline
17. रा॒ज॒न् न॒घा॒य॒तो अ॑घाय॒तो रा॑जन् राजन् नघाय॒तः । \newline
18. अ॒घा॒य॒त इत्य॑घ - य॒तः । \newline
19. न रि॑ष्येद् रिष्ये॒न् न न रि॑ष्ये॒त् त्वाव॑त॒ स्त्वाव॑तो रिष्ये॒न् न न रि॑ष्ये॒त् त्वाव॑तः । \newline
20. रि॒ष्ये॒त् त्वाव॑त॒ स्त्वाव॑तो रिष्येद् रिष्ये॒त् त्वाव॑तः॒ सखा॒ सखा॒ त्वाव॑तो रिष्येद् रिष्ये॒त् त्वाव॑तः॒ सखा᳚ । \newline
21. त्वाव॑तः॒ सखा॒ सखा॒ त्वाव॑त॒ स्त्वाव॑तः॒ सखा᳚ । \newline
22. त्वाव॑त॒ इति॒ त्व - व॒तः॒ । \newline
23. सखेति॒ सखा᳚ । \newline
24. या ते॑ ते॒ या या ते॒ धामा॑नि॒ धामा॑नि ते॒ या या ते॒ धामा॑नि । \newline
25. ते॒ धामा॑नि॒ धामा॑नि ते ते॒ धामा॑नि दि॒वि दि॒वि धामा॑नि ते ते॒ धामा॑नि दि॒वि । \newline
26. धामा॑नि दि॒वि दि॒वि धामा॑नि॒ धामा॑नि दि॒वि या या दि॒वि धामा॑नि॒ धामा॑नि दि॒वि या । \newline
27. दि॒वि या या दि॒वि दि॒वि या पृ॑थि॒व्याम् पृ॑थि॒व्यां ॅया दि॒वि दि॒वि या पृ॑थि॒व्याम् । \newline
28. या पृ॑थि॒व्याम् पृ॑थि॒व्यां ॅया या पृ॑थि॒व्यां ॅया या पृ॑थि॒व्यां ॅया या पृ॑थि॒व्यां ॅया । \newline
29. पृ॒थि॒व्यां ॅया या पृ॑थि॒व्याम् पृ॑थि॒व्यां ॅया पर्व॑तेषु॒ पर्व॑तेषु॒ या पृ॑थि॒व्याम् पृ॑थि॒व्यां ॅया पर्व॑तेषु । \newline
30. या पर्व॑तेषु॒ पर्व॑तेषु॒ या या पर्व॑ते॒ ष्वोष॑धी॒ ष्वोष॑धीषु॒ पर्व॑तेषु॒ या या पर्व॑ते॒ ष्वोष॑धीषु । \newline
31. पर्व॑ते॒ ष्वोष॑धी॒ ष्वोष॑धीषु॒ पर्व॑तेषु॒ पर्व॑ते॒ ष्वोष॑धी ष्व॒फ्स्व॑ फ्स्वोष॑धीषु॒ पर्व॑तेषु॒ पर्व॑ते॒ ष्वोष॑धीष्व॒फ्सु । \newline
32. ओष॑धी ष्व॒फ्स्व॑ फ्स्वोष॑धी॒ ष्वोष॑धी ष्व॒फ्सु । \newline
33. अ॒फ्सित्य॑प् - सु । \newline
34. तेभि॑र् नो न॒ स्तेभि॒ स्तेभि॑र् नो॒ विश्वै॒र् विश्वै᳚र् न॒स्तेभि॒ स्तेभि॑र् नो॒ विश्वैः᳚ । \newline
35. नो॒ विश्वै॒र् विश्वै᳚र् नो नो॒ विश्वैः᳚ सु॒मनाः᳚ सु॒मना॒ विश्वै᳚र् नो नो॒ विश्वैः᳚ सु॒मनाः᳚ । \newline
36. विश्वैः᳚ सु॒मनाः᳚ सु॒मना॒ विश्वै॒र् विश्वैः᳚ सु॒मना॒ अहे॑ड॒न् नहे॑डन् थ्सु॒मना॒ विश्वै॒र् विश्वैः᳚ सु॒मना॒ अहे॑डन्न् । \newline
37. सु॒मना॒ अहे॑ड॒न् नहे॑डन् थ्सु॒मनाः᳚ सु॒मना॒ अहे॑ड॒न् राज॒न् राज॒न् नहे॑डन् थ्सु॒मनाः᳚ सु॒मना॒ अहे॑ड॒न् राजन्न्॑ । \newline
38. सु॒मना॒ इति॑ सु - मनाः᳚ । \newline
39. अहे॑ड॒न् राज॒न् राज॒न् नहे॑ड॒न् नहे॑ड॒न् राजन्᳚ थ्सोम सोम॒ राज॒न् नहे॑ड॒न् नहे॑ड॒न् राजन्᳚ थ्सोम । \newline
40. राजन्᳚ थ्सोम सोम॒ राज॒न् राजन्᳚ थ्सोम॒ प्रति॒ प्रति॑ षोम॒ राज॒न् राजन्᳚ थ्सोम॒ प्रति॑ । \newline
41. सो॒म॒ प्रति॒ प्रति॑ षोम सोम॒ प्रति॑ ह॒व्या ह॒व्या प्रति॑ षोम सोम॒ प्रति॑ ह॒व्या । \newline
42. प्रति॑ ह॒व्या ह॒व्या प्रति॒ प्रति॑ ह॒व्या गृ॑भाय गृभाय ह॒व्या प्रति॒ प्रति॑ ह॒व्या गृ॑भाय । \newline
43. ह॒व्या गृ॑भाय गृभाय ह॒व्या ह॒व्या गृ॑भाय । \newline
44. गृ॒भा॒येति॑ गृभाय । \newline
45. अग्नी॑षोमा॒ सवे॑दसा॒ सवे॑द॒सा ऽग्नी॑षो॒मा ऽग्नी॑षोमा॒ सवे॑दसा॒ सहू॑ती॒ सहू॑ती॒ सवे॑द॒सा ऽग्नी॑षो॒मा ऽग्नी॑षोमा॒ सवे॑दसा॒ सहू॑ती । \newline
46. अग्नी॑षो॒मेत्यग्नी᳚ - सो॒मा॒ । \newline
47. सवे॑दसा॒ सहू॑ती॒ सहू॑ती॒ सवे॑दसा॒ सवे॑दसा॒ सहू॑ती वनतं ॅवनतꣳ॒॒ सहू॑ती॒ सवे॑दसा॒ सवे॑दसा॒ सहू॑ती वनतम् । \newline
48. सवे॑द॒सेति॒ स - वे॒द॒सा॒ । \newline
49. सहू॑ती वनतं ॅवनतꣳ॒॒ सहू॑ती॒ सहू॑ती वनत॒म् गिरो॒ गिरो॑ वनतꣳ॒॒ सहू॑ती॒ सहू॑ती वनत॒म् गिरः॑ । \newline
50. सहू॑ती॒ इति॒ स - हू॒ती॒ । \newline
51. व॒न॒त॒म् गिरो॒ गिरो॑ वनतं ॅवनत॒म् गिरः॑ । \newline
52. गिर॒ इति॒ गिरः॑ । \newline
53. सम् दे॑व॒त्रा दे॑व॒त्रा सꣳ सम् दे॑व॒त्रा ब॑भूवथुर् बभूवथुर् देव॒त्रा सꣳ सम् दे॑व॒त्रा ब॑भूवथुः । \newline
54. दे॒व॒त्रा ब॑भूवथुर् बभूवथुर् देव॒त्रा दे॑व॒त्रा ब॑भूवथुः । \newline
55. दे॒व॒त्रेति॑ देव - त्रा । \newline
56. ब॒भू॒व॒थु॒रिति॑ बभूवथुः । \newline
57. यु॒व मे॒ता न्ये॒तानि॑ यु॒वं ॅयु॒व मे॒तानि॑ दि॒वि दि॒व्ये॑तानि॑ यु॒वं ॅयु॒व मे॒तानि॑ दि॒वि । \newline
\pagebreak
\markright{ TS 2.3.14.2  \hfill https://www.vedavms.in \hfill}

\section{ TS 2.3.14.2 }

\textbf{TS 2.3.14.2 } \newline
\textbf{Samhita Paata} \newline

-मे॒तानि॑ दि॒वि रो॑च॒नान्य॒ग्निश्च॑ सोम॒ सक्र॑तू अधत्तं ॥ यु॒वꣳ सिन्धूꣳ॑ र॒भिश॑स्तेरव॒द्या-दग्नी॑षोमा॒-वमु॑ञ्चतं गृभी॒तान् ॥ अग्नी॑षोमावि॒मꣳ सु मे॑ शृणु॒तं ॅवृ॑षणा॒ हवं᳚ । प्रति॑ सू॒क्तानि॑ हर्यतं॒ भव॑तं दा॒शुषे॒ मयः॑ ॥ आऽन्यं दि॒वो मा॑त॒रिश्वा॑ जभा॒राऽम॑थ्नाद॒न्यं परि॑ श्ये॒नो अद्रेः᳚ । अग्नी॑षोमा॒ ब्रह्म॑णा वावृधा॒नोरुं ॅय॒ज्ञाय॑ चक्रथुरु लो॒कं ॥अग्नी॑षोमा ह॒विषः॒ प्रस्थि॑तस्य वी॒तꣳ - [  ] \newline

\textbf{Pada Paata} \newline

ए॒तानि॑ । दि॒वि । रो॒च॒नानि॑ । अ॒ग्निः । च॒ । सो॒म॒ । सक्र॑तू॒ इति॒ स - क्र॒तू॒ । अ॒ध॒त्त॒म् ॥ यु॒वम् । सिन्धून्॑ । अ॒भिश॑स्ते॒रित्य॒भि - श॒स्तेः॒ । अ॒व॒द्यात् । अग्नी॑षोमा॒वित्यग्नी᳚ - सो॒मौ॒ । अमु॑ञ्चतम् । गृ॒भी॒तान् ॥ अग्नी॑षोमा॒वित्यग्नी᳚ - सो॒मौ॒ । इ॒मम् । स्विति॑ । मे॒ । शृ॒णु॒तम् । वृ॒ष॒णा॒ । हव᳚म् । प्रतीति॑ ॥ सू॒क्ता॒नीति॑ सु - उ॒क्तानि॑ । ह॒र्य॒त॒म् । भव॑तम् । दा॒शुषे᳚ । मयः॑ ॥ एति॑ । अ॒न्यम् । दि॒वः । मा॒त॒रिश्वा᳚ । ज॒भा॒र॒ । अम॑थ्नात् । अ॒न्यम् । परीति॑ । श्ये॒नः । अद्रेः᳚ ॥ अग्नी॑षो॒मेत्यग्नी᳚ - सो॒मा॒ । ब्रह्म॑णा । वा॒वृ॒धा॒ना । उ॒रुम् । य॒ज्ञाय॑ । च॒क्र॒थुः॒ । उ॒ । लो॒कम् ॥ अग्नी॑षो॒मेत्यग्नी᳚ - सो॒मा॒ । ह॒विषः॑ । प्रस्थि॑त॒स्येति॒ प्र - स्थि॒त॒स्य॒ । वी॒तम् ।  \newline


\textbf{Krama Paata} \newline

ए॒तानि॑ दि॒वि । दि॒वि रो॑च॒नानि॑ । रो॒च॒नान्य॒ग्निः । अ॒ग्निश्च॑ । च॒ सो॒म॒ । सो॒म॒ सक्र॑तू । सक्र॑तू अधत्तम् । सक्र॑तू॒ इति॒ स - क्र॒तू॒ । अ॒ध॒त्त॒,मित्य॑धत्तम् ॥ यु॒वꣳ सिन्धून्॑ । सिन्धूꣳ॑ र॒भिश॑स्तेः । अ॒भिश॑स्तेरव॒द्यात् । अ॒भिश॑स्ते॒रित्य॒भि - श॒स्तेः॒ । अ॒व॒द्यादग्नी॑षोमौ । अग्नी॑षोमा॒वमु॑ञ्चतम् । अग्नी॑षोमा॒वित्यग्नी᳚ - सो॒मौ॒ । अमु॑ञ्चतम् गृभी॒तान् । गृ॒भी॒तानिति॑ गृभी॒तान् ॥ अग्नी॑षोमा वि॒मम् । अग्नी॑षोमा॒,वित्यग्नी᳚ - सो॒मौ॒ । इ॒मꣳ सु । सु मे᳚ । मे॒ शृ॒णु॒तम् । शृ॒णु॒तं ॅवृ॑षणा । वृ॒ष॒णा॒ हव᳚म् । हव॒मिति॒ हव᳚म् ॥ प्रति॑ सू॒क्तानि॑ । सू॒क्तानि॑ हर्यतम् । सू॒क्तानीति॑ सु - उ॒क्तानि॑ । ह॒र्य॒त॒म् भव॑तम् । भव॑तम् दा॒शुषे᳚ । दा॒शुषे॒ मयः॑ । मय॒ इति॒ मयः॑ ॥ आ ऽन्यम् । अ॒न्यम् दि॒वः । दि॒वो मा॑त॒रिश्वा᳚ । मा॒त॒रिश्वा॑ जभार । ज॒भा॒राम॑थ्नात् । अम॑थ्नाद॒न्यम् । अ॒न्यम् परि॑ । परि॑ श्ये॒नः । श्ये॒नो अद्रेः᳚ । अद्रे॒रित्यद्रेः᳚ ॥ अग्नी॑षोमा॒ ब्रह्म॑णा । अग्नी॑षो॒मेत्यग्नी᳚ - सो॒मा॒ । ब्रह्म॑णा वावृधा॒ना । वा॒वृ॒धा॒नोरुम् । उ॒रुं ॅय॒ज्ञाय॑ । य॒ज्ञाय॑ चक्रथुः । च॒क्र॒थु॒रु॒ । उ॒ लो॒कम् । लो॒कमिति॑ लो॒कम् ॥ अग्नी॑षोमा ह॒विषः॑ । अग्नी॑षो॒मेत्यग्नी᳚ - सो॒मा॒ । ह॒विषः॒ प्रस्थि॑तस्य । प्रस्थि॑तस्य वी॒तम् । प्रस्थि॑त॒स्येति॒ प्र - स्थि॒त॒स्य॒ । वी॒तꣳ हर्य॑तम् \newline

\textbf{Jatai Paata} \newline

1. ए॒तानि॑ दि॒वि दि॒व्ये॑ता न्ये॒तानि॑ दि॒वि । \newline
2. दि॒वि रो॑च॒नानि॑ रोच॒नानि॑ दि॒वि दि॒वि रो॑च॒नानि॑ । \newline
3. रो॒च॒ना न्य॒ग्नि र॒ग्नी रो॑च॒नानि॑ रोच॒ना न्य॒ग्निः । \newline
4. अ॒ग्निश्च॑ चा॒ग्नि र॒ग्निश्च॑ । \newline
5. च॒ सो॒म॒ सो॒म॒ च॒ च॒ सो॒म॒ । \newline
6. सो॒म॒ सक्र॑तू॒ सक्र॑तू सोम सोम॒ सक्र॑तू । \newline
7. सक्र॑तू अधत्त मधत्तꣳ॒॒ सक्र॑तू॒ सक्र॑तू अधत्तम् । \newline
8. सक्र॑तू॒ इति॒ स - क्र॒तू॒ । \newline
9. अ॒ध॒त्त॒मित्य॑धत्तम् । \newline
10. यु॒वꣳ सिन्धू॒न् थ्सिन्धून्॑. यु॒वं ॅयु॒वꣳ सिन्धून्॑ । \newline
11. सिन्धूꣳ॑ र॒भिश॑स्ते र॒भिश॑स्तेः॒ सिन्धू॒न् थ्सिन्धूꣳ॑ र॒भिश॑स्तेः । \newline
12. अ॒भिश॑स्ते रव॒द्या द॑व॒द्या द॒भिश॑स्ते र॒भिश॑स्ते रव॒द्यात् । \newline
13. अ॒भिश॑स्ते॒रित्य॒भि - श॒स्तेः॒ । \newline
14. अ॒व॒द्या दग्नी॑षोमा॒ वग्नी॑षोमा वव॒द्या द॑व॒द्या दग्नी॑षोमौ । \newline
15. अग्नी॑षोमा॒ वमु॑ञ्चत॒ ममु॑ञ्चत॒ मग्नी॑षोमा॒ वग्नी॑षोमा॒ वमु॑ञ्चतम् । \newline
16. अग्नी॑षोमा॒वित्यग्नी᳚ - सो॒मौ॒ । \newline
17. अमु॑ञ्चतम् गृभी॒तान् गृ॑भी॒ता नमु॑ञ्चत॒ ममु॑ञ्चतम् गृभी॒तान् । \newline
18. गृ॒भी॒तानिति॑ गृभी॒तान् । \newline
19. अग्नी॑षोमा वि॒म मि॒म मग्नी॑षोमा॒ वग्नी॑षोमा वि॒मम् । \newline
20. अग्नी॑षोमा॒वित्यग्नी᳚ - सो॒मौ॒ । \newline
21. इ॒मꣳ सु स्वि॑म मि॒मꣳ सु । \newline
22. सु मे॑ मे॒ सु सु मे᳚ । \newline
23. मे॒ शृ॒णु॒तꣳ शृ॑णु॒तम् मे॑ मे शृणु॒तम् । \newline
24. शृ॒णु॒तं ॅवृ॑षणा वृषणा शृणु॒तꣳ शृ॑णु॒तं ॅवृ॑षणा । \newline
25. वृ॒ष॒णा॒ हवꣳ॒॒ हवं॑ ॅवृषणा वृषणा॒ हव᳚म् । \newline
26. हव॒मिति॒ हव᳚म् । \newline
27. प्रति॑ सू॒क्तानि॑ सू॒क्तानि॑ प्रति॒ प्रति॑ सू॒क्तानि॑ । \newline
28. सू॒क्तानि॑ हर्यतꣳ हर्यतꣳ सू॒क्तानि॑ सू॒क्तानि॑ हर्यतम् । \newline
29. सू॒क्तानीति॑ सु - उ॒क्तानि॑ । \newline
30. ह॒र्य॒त॒म् भव॑त॒म् भव॑तꣳ हर्यतꣳ हर्यत॒म् भव॑तम् । \newline
31. भव॑तम् दा॒शुषे॑ दा॒शुषे॒ भव॑त॒म् भव॑तम् दा॒शुषे᳚ । \newline
32. दा॒शुषे॒ मयो॒ मयो॑ दा॒शुषे॑ दा॒शुषे॒ मयः॑ । \newline
33. मय॒ इति॒ मयः॑ । \newline
34. आ ऽन्य म॒न्य मा ऽन्यम् । \newline
35. अ॒न्यम् दि॒वो दि॒वो अ॒न्य म॒न्यम् दि॒वः । \newline
36. दि॒वो मा॑त॒रिश्वा॑ मात॒रिश्वा॑ दि॒वो दि॒वो मा॑त॒रिश्वा᳚ । \newline
37. मा॒त॒रिश्वा॑ जभार जभार मात॒रिश्वा॑ मात॒रिश्वा॑ जभार । \newline
38. ज॒भा॒रा म॑थ्ना॒द म॑थ्नाज् जभार जभा॒रा म॑थ्नात् । \newline
39. अम॑थ्ना द॒न्य म॒न्य मम॑थ्ना॒ दम॑थ्ना द॒न्यम् । \newline
40. अ॒न्यम् परि॒ पर्य॒न्य म॒न्यम् परि॑ । \newline
41. परि॑ श्ये॒नः श्ये॒नः परि॒ परि॑ श्ये॒नः । \newline
42. श्ये॒नो अद्रे॒ रद्रेः᳚ श्ये॒नः श्ये॒नो अद्रेः᳚ । \newline
43. अद्रे॒रित्यद्रेः᳚ । \newline
44. अग्नी॑षोमा॒ ब्रह्म॑णा॒ ब्रह्म॒णा ऽग्नी॑षो॒मा ऽग्नी॑षोमा॒ ब्रह्म॑णा । \newline
45. अग्नी॑षो॒मेत्यग्नी᳚ - सो॒मा॒ । \newline
46. ब्रह्म॑णा वावृधा॒ना वा॑वृधा॒ना ब्रह्म॑णा॒ ब्रह्म॑णा वावृधा॒ना । \newline
47. वा॒वृ॒धा॒नोरु मु॒रुं ॅवा॑वृधा॒ना वा॑वृधा॒नोरुम् । \newline
48. उ॒रुं ॅय॒ज्ञाय॑ य॒ज्ञायो॒रु मु॒रुं ॅय॒ज्ञाय॑ । \newline
49. य॒ज्ञाय॑ चक्रथु श्चक्रथुर् य॒ज्ञाय॑ य॒ज्ञाय॑ चक्रथुः । \newline
50. च॒क्र॒थु॒रु॒ वु॒ च॒क्र॒थु॒ श्च॒क्र॒थु॒रु॒ । \newline
51. उ॒ लो॒कम् ॅलो॒क मु॑ वु लो॒कम् । \newline
52. लो॒कमिति॑ लो॒कम् । \newline
53. अग्नी॑षोमा ह॒विषो॑ ह॒विषो ऽग्नी॑षो॒मा ऽग्नी॑षोमा ह॒विषः॑ । \newline
54. अग्नी॑षो॒मेत्यग्नी᳚ - सो॒मा॒ । \newline
55. ह॒विषः॒ प्रस्थि॑तस्य॒ प्रस्थि॑तस्य ह॒विषो॑ ह॒विषः॒ प्रस्थि॑तस्य । \newline
56. प्रस्थि॑तस्य वी॒तं ॅवी॒तम् प्रस्थि॑तस्य॒ प्रस्थि॑तस्य वी॒तम् । \newline
57. प्रस्थि॑त॒स्येति॒ प्र - स्थि॒त॒स्य॒ । \newline
58. वी॒तꣳ हर्य॑तꣳ॒॒ हर्य॑तं ॅवी॒तं ॅवी॒तꣳ हर्य॑तम् । \newline

\textbf{Ghana Paata } \newline

1. ए॒तानि॑ दि॒वि दि॒व्ये॑ता न्ये॒तानि॑ दि॒वि रो॑च॒नानि॑ रोच॒नानि॑ दि॒व्ये॑ता न्ये॒तानि॑ दि॒वि रो॑च॒नानि॑ । \newline
2. दि॒वि रो॑च॒नानि॑ रोच॒नानि॑ दि॒वि दि॒वि रो॑च॒ना न्य॒ग्नि र॒ग्नी रो॑च॒नानि॑ दि॒वि दि॒वि रो॑च॒ना न्य॒ग्निः । \newline
3. रो॒च॒ना न्य॒ग्नि र॒ग्नी रो॑च॒नानि॑ रोच॒ना न्य॒ग्निश्च॑ चा॒ग्नी रो॑च॒नानि॑ रोच॒ना न्य॒ग्निश्च॑ । \newline
4. अ॒ग्निश्च॑ चा॒ग्नि र॒ग्निश्च॑ सोम सोम चा॒ग्नि र॒ग्निश्च॑ सोम । \newline
5. च॒ सो॒म॒ सो॒म॒ च॒ च॒ सो॒म॒ सक्र॑तू॒ सक्र॑तू सोम च च सोम॒ सक्र॑तू । \newline
6. सो॒म॒ सक्र॑तू॒ सक्र॑तू सोम सोम॒ सक्र॑तू अधत्त मधत्तꣳ॒॒ सक्र॑तू सोम सोम॒ सक्र॑तू अधत्तम् । \newline
7. सक्र॑तू अधत्त मधत्तꣳ॒॒ सक्र॑तू॒ सक्र॑तू अधत्तम् । \newline
8. सक्र॑तू॒ इति॒ स - क्र॒तू॒ । \newline
9. अ॒ध॒त्त॒मित्य॑धत्तम् । \newline
10. यु॒वꣳ सिन्धू॒न् थ्सिन्धून्॑. यु॒वं ॅयु॒वꣳ सिन्धूꣳ॑ र॒भिश॑स्ते र॒भिश॑स्तेः॒ सिन्धून्॑. यु॒वं ॅयु॒वꣳ सिन्धूꣳ॑ र॒भिश॑स्तेः । \newline
11. सिन्धूꣳ॑ र॒भिश॑स्ते र॒भिश॑स्तेः॒ सिन्धू॒न् थ्सिन्धूꣳ॑ र॒भिश॑स्ते रव॒द्या द॑व॒द्या द॒भिश॑स्तेः॒ सिन्धू॒न् थ्सिन्धूꣳ॑ र॒भिश॑स्ते रव॒द्यात् । \newline
12. अ॒भिश॑स्ते रव॒द्या द॑व॒द्या द॒भिश॑स्ते र॒भिश॑स्ते रव॒द्या दग्नी॑षोमा॒ वग्नी॑षोमा वव॒द्या द॒भिश॑स्ते र॒भिश॑स्ते रव॒द्या दग्नी॑षोमौ । \newline
13. अ॒भिश॑स्ते॒रित्य॒भि - श॒स्तेः॒ । \newline
14. अ॒व॒द्या दग्नी॑षोमा॒ वग्नी॑षोमा वव॒द्या द॑व॒द्या दग्नी॑षोमा॒ वमु॑ञ्चत॒ ममु॑ञ्चत॒ मग्नी॑षोमा वव॒द्या द॑व॒द्या दग्नी॑षोमा॒ वमु॑ञ्चतम् । \newline
15. अग्नी॑षोमा॒ वमु॑ञ्चत॒ ममु॑ञ्चत॒ मग्नी॑षोमा॒ वग्नी॑षोमा॒ वमु॑ञ्चतम् गृभी॒तान् गृ॑भी॒ता नमु॑ञ्चत॒ मग्नी॑षोमा॒ वग्नी॑षोमा॒ वमु॑ञ्चतम् गृभी॒तान् । \newline
16. अग्नी॑षोमा॒वित्यग्नी᳚ - सो॒मौ॒ । \newline
17. अमु॑ञ्चतम् गृभी॒तान् गृ॑भी॒ता नमु॑ञ्चत॒ ममु॑ञ्चतम् गृभी॒तान् । \newline
18. गृ॒भी॒तानिति॑ गृभी॒तान् । \newline
19. अग्नी॑षोमा वि॒म मि॒म मग्नी॑षोमा॒ वग्नी॑षोमा वि॒मꣳ सु स्वि॑म मग्नी॑षोमा॒ वग्नी॑षोमा वि॒मꣳ सु । \newline
20. अग्नी॑षोमा॒वित्यग्नी᳚ - सो॒मौ॒ । \newline
21. इ॒मꣳ सु स्वि॑म मि॒मꣳ सु मे॑ मे॒ स्वि॑म मि॒मꣳ सु मे᳚ । \newline
22. सु मे॑ मे॒ सु सु मे॑ शृणु॒तꣳ शृ॑णु॒तम् मे॒ सु सु मे॑ शृणु॒तम् । \newline
23. मे॒ शृ॒णु॒तꣳ शृ॑णु॒तम् मे॑ मे शृणु॒तं ॅवृ॑षणा वृषणा शृणु॒तम् मे॑ मे शृणु॒तं ॅवृ॑षणा । \newline
24. शृ॒णु॒तं ॅवृ॑षणा वृषणा शृणु॒तꣳ शृ॑णु॒तं ॅवृ॑षणा॒ हवꣳ॒॒ हवं॑ ॅवृषणा शृणु॒तꣳ शृ॑णु॒तं ॅवृ॑षणा॒ हव᳚म् । \newline
25. वृ॒ष॒णा॒ हवꣳ॒॒ हवं॑ ॅवृषणा वृषणा॒ हव᳚म् । \newline
26. हव॒मिति॒ हव᳚म् । \newline
27. प्रति॑ सू॒क्तानि॑ सू॒क्तानि॑ प्रति॒ प्रति॑ सू॒क्तानि॑ हर्यतꣳ हर्यतꣳ सू॒क्तानि॑ प्रति॒ प्रति॑ सू॒क्तानि॑ हर्यतम् । \newline
28. सू॒क्तानि॑ हर्यतꣳ हर्यतꣳ सू॒क्तानि॑ सू॒क्तानि॑ हर्यत॒म् भव॑त॒म् भव॑तꣳ हर्यतꣳ सू॒क्तानि॑ सू॒क्तानि॑ हर्यत॒म् भव॑तम् । \newline
29. सू॒क्तानीति॑ सु - उ॒क्तानि॑ । \newline
30. ह॒र्य॒त॒म् भव॑त॒म् भव॑तꣳ हर्यतꣳ हर्यत॒म् भव॑तम् दा॒शुषे॑ दा॒शुषे॒ भव॑तꣳ हर्यतꣳ हर्यत॒म् भव॑तम् दा॒शुषे᳚ । \newline
31. भव॑तम् दा॒शुषे॑ दा॒शुषे॒ भव॑त॒म् भव॑तम् दा॒शुषे॒ मयो॒ मयो॑ दा॒शुषे॒ भव॑त॒म् भव॑तम् दा॒शुषे॒ मयः॑ । \newline
32. दा॒शुषे॒ मयो॒ मयो॑ दा॒शुषे॑ दा॒शुषे॒ मयः॑ । \newline
33. मय॒ इति॒ मयः॑ । \newline
34. आ ऽन्य म॒न्य मा ऽन्यम् दि॒वो दि॒वो अ॒न्य मा ऽन्यम् दि॒वः । \newline
35. अ॒न्यम् दि॒वो दि॒वो अ॒न्य म॒न्यम् दि॒वो मा॑त॒रिश्वा॑ मात॒रिश्वा॑ दि॒वो अ॒न्य म॒न्यम् दि॒वो मा॑त॒रिश्वा᳚ । \newline
36. दि॒वो मा॑त॒रिश्वा॑ मात॒रिश्वा॑ दि॒वो दि॒वो मा॑त॒रिश्वा॑ जभार जभार मात॒रिश्वा॑ दि॒वो दि॒वो मा॑त॒रिश्वा॑ जभार । \newline
37. मा॒त॒रिश्वा॑ जभार जभार मात॒रिश्वा॑ मात॒रिश्वा॑ जभा॒रा म॑थ्ना॒ दम॑थ्नाज् जभार मात॒रिश्वा॑ मात॒रिश्वा॑ जभा॒रा म॑थ्नात् । \newline
38. ज॒भा॒रा म॑थ्ना॒ दम॑थ्नाज् जभार जभा॒रा म॑थ्ना द॒न्य म॒न्य मम॑थ्नाज् जभार जभा॒रा म॑थ्ना द॒न्यम् । \newline
39. अम॑थ्ना द॒न्य म॒न्य मम॑थ्ना॒ दम॑थ्ना द॒न्यम् परि॒ पर्य॒न्य मम॑थ्ना॒ दम॑थ्ना द॒न्यम् परि॑ । \newline
40. अ॒न्यम् परि॒ पर्य॒न्य म॒न्यम् परि॑ श्ये॒नः श्ये॒नः पर्य॒न्य म॒न्यम् परि॑ श्ये॒नः । \newline
41. परि॑ श्ये॒नः श्ये॒नः परि॒ परि॑ श्ये॒नो अद्रे॒ रद्रेः᳚ श्ये॒नः परि॒ परि॑ श्ये॒नो अद्रेः᳚ । \newline
42. श्ये॒नो अद्रे॒ रद्रेः᳚ श्ये॒नः श्ये॒नो अद्रेः᳚ । \newline
43. अद्रे॒रित्यद्रेः᳚ । \newline
44. अग्नी॑षोमा॒ ब्रह्म॑णा॒ ब्रह्म॒णा ऽग्नी॑षो॒मा ऽग्नी॑षोमा॒ ब्रह्म॑णा वावृधा॒ना वा॑वृधा॒ना ब्रह्म॒णा ऽग्नी॑षो॒मा ऽग्नी॑षोमा॒ ब्रह्म॑णा वावृधा॒ना । \newline
45. अग्नी॑षो॒मेत्यग्नी᳚ - सो॒मा॒ । \newline
46. ब्रह्म॑णा वावृधा॒ना वा॑वृधा॒ना ब्रह्म॑णा॒ ब्रह्म॑णा वावृधा॒नोरु मु॒रुं ॅवा॑वृधा॒ना ब्रह्म॑णा॒ ब्रह्म॑णा वावृधा॒नोरुम् । \newline
47. वा॒वृ॒धा॒नोरु मु॒रुं ॅवा॑वृधा॒ना वा॑वृधा॒नोरुं ॅय॒ज्ञाय॑ य॒ज्ञायो॒रुं ॅवा॑वृधा॒ना वा॑वृधा॒नोरुं ॅय॒ज्ञाय॑ । \newline
48. उ॒रुं ॅय॒ज्ञाय॑ य॒ज्ञायो॒रु मु॒रुं ॅय॒ज्ञाय॑ चक्रथु श्चक्रथुर् य॒ज्ञायो॒रु मु॒रुं ॅय॒ज्ञाय॑ चक्रथुः । \newline
49. य॒ज्ञाय॑ चक्रथु श्चक्रथुर् य॒ज्ञाय॑ य॒ज्ञाय॑ चक्रथुरु वु चक्रथुर् य॒ज्ञाय॑ य॒ज्ञाय॑ चक्रथुरु । \newline
50. च॒क्र॒थु॒रु॒ वु॒ च॒क्र॒थु॒ श्च॒क्र॒थु॒रु॒ लो॒कम् ॅलो॒क मु॑ चक्रथु श्चक्रथुरु लो॒कम् । \newline
51. उ॒ लो॒कम् ॅलो॒क मु॑ वु लो॒कम् । \newline
52. लो॒कमिति॑ लो॒कम् । \newline
53. अग्नी॑षोमा ह॒विषो॑ ह॒विषो ऽग्नी॑षो॒मा ऽग्नी॑षोमा ह॒विषः॒ प्रस्थि॑तस्य॒ प्रस्थि॑तस्य ह॒विषो ऽग्नी॑षो॒मा ऽग्नी॑षोमा ह॒विषः॒ प्रस्थि॑तस्य । \newline
54. अग्नी॑षो॒मेत्यग्नी᳚ - सो॒मा॒ । \newline
55. ह॒विषः॒ प्रस्थि॑तस्य॒ प्रस्थि॑तस्य ह॒विषो॑ ह॒विषः॒ प्रस्थि॑तस्य वी॒तं ॅवी॒तम् प्रस्थि॑तस्य ह॒विषो॑ ह॒विषः॒ प्रस्थि॑तस्य वी॒तम् । \newline
56. प्रस्थि॑तस्य वी॒तं ॅवी॒तम् प्रस्थि॑तस्य॒ प्रस्थि॑तस्य वी॒तꣳ हर्य॑तꣳ॒॒ हर्य॑तं ॅवी॒तम् प्रस्थि॑तस्य॒ प्रस्थि॑तस्य वी॒तꣳ हर्य॑तम् । \newline
57. प्रस्थि॑त॒स्येति॒ प्र - स्थि॒त॒स्य॒ । \newline
58. वी॒तꣳ हर्य॑तꣳ॒॒ हर्य॑तं ॅवी॒तं ॅवी॒तꣳ हर्य॑तं ॅवृषणा वृषणा॒ हर्य॑तं ॅवी॒तं ॅवी॒तꣳ हर्य॑तं ॅवृषणा । \newline
\pagebreak
\markright{ TS 2.3.14.3  \hfill https://www.vedavms.in \hfill}

\section{ TS 2.3.14.3 }

\textbf{TS 2.3.14.3 } \newline
\textbf{Samhita Paata} \newline

हर्य॑तं ॅवृषणा जु॒षेथां᳚ । सु॒शर्मा॑णा॒ स्वव॑सा॒ हि भू॒तमथा॑ धत्तं॒ ॅयज॑मानाय॒ शं ॅयोः ॥ आप्या॑यस्व॒ >5, सं ते᳚ >6 ॥ ग॒णानां᳚ त्वा ग॒णप॑तिꣳ हवामहे क॒विं क॑वी॒ना-मु॑प॒मश्र॑वस्तमं । ज्ये॒ष्ठ॒राजं॒ ब्रह्म॑णां ब्रह्मणस्पत॒ आ नः॑ शृ॒ण्वन्नू॒तिभिः॑ सीद॒ साद॑नं । स इज्जने॑न॒ स वि॒शा स जन्म॑ना॒ स पु॒त्रैर्वाजं॑ भरते॒ धना॒ नृभिः॑ । दे॒वानां॒ ॅयः पि॒तर॑मा॒विवा॑सति - [  ] \newline

\textbf{Pada Paata} \newline

हर्य॑तम् । वृ॒ष॒णा॒ । जु॒षेथा᳚म् ॥ सु॒शर्मा॒णेति॑ सु - शर्मा॑णा । स्वव॒सेति॑ सु - अव॑सा । हि । भू॒तम् । अथ॑ । ध॒त्त॒म् । यज॑मानाय । शम् । योः ॥ एति॑ । प्या॒य॒स्व॒ । समिति॑ । ते॒ ॥ ग॒णाना᳚म् । त्वा॒ । ग॒णप॑ति॒मिति॑ ग॒ण - प॒ति॒म् । ह॒वा॒म॒हे॒ । क॒विम् । क॒वी॒नाम् । उ॒प॒मश्र॑वस्तम॒मित्यु॑प॒मश्र॑वः-त॒म॒म् ॥ ज्ये॒ष्ठ॒राज॒मिति॑ ज्येष्ठ-राज᳚म् । ब्रह्म॑णाम् । ब्र॒ह्म॒णः॒ । प॒ते॒ । एति॑ । नः॒ । शृ॒ण्वन्न् । ऊ॒तिभि॒रित्यू॒ति - भिः॒ । सी॒द॒ । साद॑नम् ॥ सः । इत् । जने॑न । सः । वि॒शा । सः । जन्म॑ना । सः । पु॒त्रैः । वाज᳚म् । भ॒र॒ते॒ । धना᳚ । नृभि॒रिति॒ नृ - भिः॒ ॥ दे॒वाना᳚म् । यः । पि॒तर᳚म् । आ॒विवा॑स॒तीत्या᳚ - विवा॑सति ।  \newline


\textbf{Krama Paata} \newline

हर्य॑तं ॅवृषणा । वृ॒ष॒णा॒ जु॒षेथा᳚म् । जु॒षेथा॒मिति॑ जु॒षेथा᳚म् ॥ सु॒शर्मा॑णा॒ स्वव॑सा । सु॒शर्मा॒णेति॑ सु - शर्मा॑णा । स्वव॑सा॒ हि । स्वव॒सेति॑ सु - अव॑सा । हि भू॒तम् । भू॒तमथ॑ । अथा॑ धत्तम् । ध॒त्तं॒ ॅयज॑मानाय । यज॑मानाय॒ शम् । शम् ॅयोः । योरिति॒ योः ॥ आ प्या॑यस्व । प्या॒य॒स्व॒ सम् । सम् ते᳚ । त॒ इति॑ ते ॥ ग॒णाना᳚म् त्वा । त्वा॒ ग॒णप॑तिम् । ग॒णप॑तिꣳ हवामहे । ग॒णप॑ति॒मिति॑ ग॒ण - प॒ति॒म् । ह॒वा॒म॒हे॒ क॒विम् । क॒विम् क॑वी॒नाम् । क॒वी॒नामु॑प॒मश्र॑वस्तमम् । उ॒प॒मश्र॑वस्तम॒मित्यु॑प॒मश्र॑वः - त॒म॒म् ॥ ज्ये॒ष्ठ॒राज॒म् ब्रह्म॑णाम् । ज्ये॒ष्ठ॒राज॒मिति॑ ज्येष्ठ - राज᳚म् । ब्रह्म॑णाम् ब्रह्मणः । ब्र॒ह्म॒ण॒स्प॒ते॒ । प॒त॒ आ । आ नः॑ । नः॒ शृ॒ण्वन्न् । शृ॒ण्वन्नू॒तिभिः॑ । ऊ॒तिभिः॑ सीद । ऊ॒तिभि॒रित्यू॒ति - भिः॒ । सी॒द॒ साद॑नम् । साद॑न॒मिति॒ साद॑नम् ॥ स इत् । इज्जने॑न । जने॑न॒ सः । स वि॒शा । वि॒शा सः । स जन्म॑ना । जन्म॑ना॒ सः । स पु॒त्रैः । पु॒त्रैर् वाज᳚म् । वाज॑म् भरते । भ॒र॒ते॒ धना᳚ । धना॒ नृभिः॑ । नृभि॒रिति॒ नृ - भिः॒ ॥ दे॒वानां॒ ॅयः । यः पि॒तर᳚म् । पि॒तर॑मा॒विवा॑सति । आ॒विवा॑सति श्र॒द्धाम॑नाः । आ॒विवा॑स॒तीत्या᳚ - विवा॑सति \newline

\textbf{Jatai Paata} \newline

1. हर्य॑तं ॅवृषणा वृषणा॒ हर्य॑तꣳ॒॒ हर्य॑तं ॅवृषणा । \newline
2. वृ॒ष॒णा॒ जु॒षेथा᳚म् जु॒षेथां᳚ ॅवृषणा वृषणा जु॒षेथा᳚म् । \newline
3. जु॒षेथा॒मिति॑ जु॒षेथा᳚म् । \newline
4. सु॒शर्मा॑णा॒ स्वव॑सा॒ स्वव॑सा सु॒शर्मा॑णा सु॒शर्मा॑णा॒ स्वव॑सा । \newline
5. सु॒शर्मा॒णेति॑ सु - शर्मा॑णा । \newline
6. स्वव॑सा॒ हि हि स्वव॑सा॒ स्वव॑सा॒ हि । \newline
7. स्वव॒सेति॑ सु - अव॑सा । \newline
8. हि भू॒तम् भू॒तꣳ हि हि भू॒तम् । \newline
9. भू॒त मथाथ॑ भू॒तम् भू॒त मथ॑ । \newline
10. अथा॑ धत्तम् धत्त॒ मथाथा॑ धत्तम् । \newline
11. ध॒त्तं॒ ॅयज॑मानाय॒ यज॑मानाय धत्तम् धत्तं॒ ॅयज॑मानाय । \newline
12. यज॑मानाय॒ शꣳ शं ॅयज॑मानाय॒ यज॑मानाय॒ शम् । \newline
13. शं ॅयोर् योः शꣳ शं ॅयोः । \newline
14. योरिति॒ योः । \newline
15. आ प्या॑यस्व प्याय॒स्वा प्या॑यस्व । \newline
16. प्या॒य॒स्व॒ सꣳ सम् प्या॑यस्व प्यायस्व॒ सम् । \newline
17. सम् ते॑ ते॒ सꣳ सम् ते᳚ । \newline
18. त॒ इति॑ ते । \newline
19. ग॒णाना᳚म् त्वा त्वा ग॒णाना᳚म् ग॒णाना᳚म् त्वा । \newline
20. त्वा॒ ग॒णप॑तिम् ग॒णप॑तिम् त्वा त्वा ग॒णप॑तिम् । \newline
21. ग॒णप॑तिꣳ हवामहे हवामहे ग॒णप॑तिम् ग॒णप॑तिꣳ हवामहे । \newline
22. ग॒णप॑ति॒मिति॑ ग॒ण - प॒ति॒म् । \newline
23. ह॒वा॒म॒हे॒ क॒विम् क॒विꣳ ह॑वामहे हवामहे क॒विम् । \newline
24. क॒विम् क॑वी॒नाम् क॑वी॒नाम् क॒विम् क॒विम् क॑वी॒नाम् । \newline
25. क॒वी॒ना मु॑प॒मश्र॑वस्तम मुप॒मश्र॑वस्तमम् कवी॒नाम् क॑वी॒ना मु॑प॒मश्र॑वस्तमम् । \newline
26. उ॒प॒मश्र॑वस्तम॒मित्यु॑प॒मश्र॑वः - त॒म॒म् । \newline
27. ज्ये॒ष्ठ॒राज॒म् ब्रह्म॑णा॒म् ब्रह्म॑णाम् ज्येष्ठ॒राज॑म् ज्येष्ठ॒राज॒म् ब्रह्म॑णाम् । \newline
28. ज्ये॒ष्ठ॒राज॒मिति॑ ज्येष्ठ - राज᳚म् । \newline
29. ब्रह्म॑णाम् ब्रह्मणो ब्रह्मणो॒ ब्रह्म॑णा॒म् ब्रह्म॑णाम् ब्रह्मणः । \newline
30. ब्र॒ह्म॒ण॒ स्प॒ते॒ प॒ते॒ ब्र॒ह्म॒णो॒ ब्र॒ह्म॒ण॒ स्प॒ते॒ । \newline
31. प॒त॒ आ प॑ते पत॒ आ । \newline
32. आ नो॑ न॒ आ नः॑ । \newline
33. नः॒ शृ॒ण्वञ् छृ॒ण्वन् नो॑ नः शृ॒ण्वन्न् । \newline
34. शृ॒ण्वन् नू॒तिभि॑ रू॒तिभिः॑ शृ॒ण्वञ् छृ॒ण्वन् नू॒तिभिः॑ । \newline
35. ऊ॒तिभिः॑ सीद सीदो॒तिभि॑ रू॒तिभिः॑ सीद । \newline
36. ऊ॒तिभि॒रित्यू॒ति - भिः॒ । \newline
37. सी॒द॒ साद॑नꣳ॒॒ साद॑नꣳ सीद सीद॒ साद॑नम् । \newline
38. साद॑न॒मिति॒ साद॑नम् । \newline
39. स इदिथ् स स इत् । \newline
40. इज् जने॑न॒ जने॒ने दिज् जने॑न । \newline
41. जने॑न॒ स स जने॑न॒ जने॑न॒ सः । \newline
42. स वि॒शा वि॒शा स स वि॒शा । \newline
43. वि॒शा स स वि॒शा वि॒शा सः । \newline
44. स जन्म॑ना॒ जन्म॑ना॒ स स जन्म॑ना । \newline
45. जन्म॑ना॒ स स जन्म॑ना॒ जन्म॑ना॒ सः । \newline
46. स पु॒त्रैः पु॒त्रैः स स पु॒त्रैः । \newline
47. पु॒त्रैर् वाजं॒ ॅवाज॑म् पु॒त्रैः पु॒त्रैर् वाज᳚म् । \newline
48. वाज॑म् भरते भरते॒ वाजं॒ ॅवाज॑म् भरते । \newline
49. भ॒र॒ते॒ धना॒ धना॑ भरते भरते॒ धना᳚ । \newline
50. धना॒ नृभि॒र् नृभि॒र् धना॒ धना॒ नृभिः॑ । \newline
51. नृभि॒रिति॒ नृ - भिः॒ । \newline
52. दे॒वानां॒ ॅयो यो दे॒वाना᳚म् दे॒वानां॒ ॅयः । \newline
53. यः पि॒तर॑म् पि॒तरं॒ ॅयो यः पि॒तर᳚म् । \newline
54. पि॒तर॑ मा॒विवा॑स त्या॒विवा॑सति पि॒तर॑म् पि॒तर॑ मा॒विवा॑सति । \newline
55. आ॒विवा॑सति श्र॒द्धाम॑नाः श्र॒द्धाम॑ना आ॒विवा॑स त्या॒विवा॑सति श्र॒द्धाम॑नाः । \newline
56. आ॒विवा॑स॒तीत्या᳚ - विवा॑सति । \newline

\textbf{Ghana Paata } \newline

1. हर्य॑तं ॅवृषणा वृषणा॒ हर्य॑तꣳ॒॒ हर्य॑तं ॅवृषणा जु॒षेथा᳚म् जु॒षेथां᳚ ॅवृषणा॒ हर्य॑तꣳ॒॒ हर्य॑तं ॅवृषणा जु॒षेथा᳚म् । \newline
2. वृ॒ष॒णा॒ जु॒षेथा᳚म् जु॒षेथां᳚ ॅवृषणा वृषणा जु॒षेथा᳚म् । \newline
3. जु॒षेथा॒मिति॑ जु॒षेथा᳚म् । \newline
4. सु॒शर्मा॑णा॒ स्वव॑सा॒ स्वव॑सा सु॒शर्मा॑णा सु॒शर्मा॑णा॒ स्वव॑सा॒ हि हि स्वव॑सा सु॒शर्मा॑णा सु॒शर्मा॑णा॒ स्वव॑सा॒ हि । \newline
5. सु॒शर्मा॒णेति॑ सु - शर्मा॑णा । \newline
6. स्वव॑सा॒ हि हि स्वव॑सा॒ स्वव॑सा॒ हि भू॒तम् भू॒तꣳ हि स्वव॑सा॒ स्वव॑सा॒ हि भू॒तम् । \newline
7. स्वव॒सेति॑ सु - अव॑सा । \newline
8. हि भू॒तम् भू॒तꣳ हि हि भू॒त मथाथ॑ भू॒तꣳ हि हि भू॒त मथ॑ । \newline
9. भू॒त मथाथ॑ भू॒तम् भू॒त मथा॑ धत्तम् धत्त॒ मथ॑ भू॒तम् भू॒त मथा॑ धत्तम् । \newline
10. अथा॑ धत्तम् धत्त॒ मथाथा॑ धत्तं॒ ॅयज॑मानाय॒ यज॑मानाय धत्त॒ मथाथा॑ धत्तं॒ ॅयज॑मानाय । \newline
11. ध॒त्तं॒ ॅयज॑मानाय॒ यज॑मानाय धत्तम् धत्तं॒ ॅयज॑मानाय॒ शꣳ शं ॅयज॑मानाय धत्तम् धत्तं॒ ॅयज॑मानाय॒ शम् । \newline
12. यज॑मानाय॒ शꣳ शं ॅयज॑मानाय॒ यज॑मानाय॒ शं ॅयोर् योः शं ॅयज॑मानाय॒ यज॑मानाय॒ शं ॅयोः । \newline
13. शं ॅयोर् योः शꣳ शं ॅयोः । \newline
14. योरिति॒ योः । \newline
15. आ प्या॑यस्व प्याय॒स्वा प्या॑यस्व॒ सꣳ सम् प्या॑य॒स्वा प्या॑यस्व॒ सम् । \newline
16. प्या॒य॒स्व॒ सꣳ सम् प्या॑यस्व प्यायस्व॒ सम् ते॑ ते॒ सम् प्या॑यस्व प्यायस्व॒ सम् ते᳚ । \newline
17. सम् ते॑ ते॒ सꣳ सम् ते᳚ । \newline
18. त॒ इति॑ ते । \newline
19. ग॒णाना᳚म् त्वा त्वा ग॒णाना᳚म् ग॒णाना᳚म् त्वा ग॒णप॑तिम् ग॒णप॑तिम् त्वा ग॒णाना᳚म् ग॒णाना᳚म् त्वा ग॒णप॑तिम् । \newline
20. त्वा॒ ग॒णप॑तिम् ग॒णप॑तिम् त्वा त्वा ग॒णप॑तिꣳ हवामहे हवामहे ग॒णप॑तिम् त्वा त्वा ग॒णप॑तिꣳ हवामहे । \newline
21. ग॒णप॑तिꣳ हवामहे हवामहे ग॒णप॑तिम् ग॒णप॑तिꣳ हवामहे क॒विम् क॒विꣳ ह॑वामहे ग॒णप॑तिम् ग॒णप॑तिꣳ हवामहे क॒विम् । \newline
22. ग॒णप॑ति॒मिति॑ ग॒ण - प॒ति॒म् । \newline
23. ह॒वा॒म॒हे॒ क॒विम् क॒विꣳ ह॑वामहे हवामहे क॒विम् क॑वी॒नाम् क॑वी॒नाम् क॒विꣳ ह॑वामहे हवामहे क॒विम् क॑वी॒नाम् । \newline
24. क॒विम् क॑वी॒नाम् क॑वी॒नाम् क॒विम् क॒विम् क॑वी॒ना मु॑प॒मश्र॑वस्तम मुप॒मश्र॑वस्तमम् कवी॒नाम् क॒विम् क॒विम् क॑वी॒ना मु॑प॒मश्र॑वस्तमम् । \newline
25. क॒वी॒ना मु॑प॒मश्र॑वस्तम मुप॒मश्र॑वस्तमम् कवी॒नाम् क॑वी॒ना मु॑प॒मश्र॑वस्तमम् । \newline
26. उ॒प॒मश्र॑वस्तम॒मित्यु॑प॒मश्र॑वः - त॒म॒म् । \newline
27. ज्ये॒ष्ठ॒राज॒म् ब्रह्म॑णा॒म् ब्रह्म॑णाम् ज्येष्ठ॒राज॑म् ज्येष्ठ॒राज॒म् ब्रह्म॑णाम् ब्रह्मणो ब्रह्मणो॒ ब्रह्म॑णाम् ज्येष्ठ॒राज॑म् ज्येष्ठ॒राज॒म् ब्रह्म॑णाम् ब्रह्मणः । \newline
28. ज्ये॒ष्ठ॒राज॒मिति॑ ज्येष्ठ - राज᳚म् । \newline
29. ब्रह्म॑णाम् ब्रह्मणो ब्रह्मणो॒ ब्रह्म॑णा॒म् ब्रह्म॑णाम् ब्रह्मण स्पते पते ब्रह्मणो॒ ब्रह्म॑णा॒म् ब्रह्म॑णाम् ब्रह्मण स्पते । \newline
30. ब्र॒ह्म॒ण॒ स्प॒ते॒ प॒ते॒ ब्र॒ह्म॒णो॒ ब्र॒ह्म॒ण॒ स्प॒त॒ आ प॑ते ब्रह्मणो ब्रह्मण स्पत॒ आ । \newline
31. प॒त॒ आ प॑ते पत॒ आ नो॑ न॒ आ प॑ते पत॒ आ नः॑ । \newline
32. आ नो॑ न॒ आ नः॑ शृ॒ण्वञ् छृ॒ण्वन् न॒ आ नः॑ शृ॒ण्वन्न् । \newline
33. नः॒ शृ॒ण्वञ् छृ॒ण्वन् नो॑ नः शृ॒ण्वन् नू॒तिभि॑ रू॒तिभिः॑ शृ॒ण्वन् नो॑ नः शृ॒ण्वन् नू॒तिभिः॑ । \newline
34. शृ॒ण्वन् नू॒तिभि॑ रू॒तिभिः॑ शृ॒ण्वञ् छृ॒ण्वन् नू॒तिभिः॑ सीद सीदो॒तिभिः॑ शृ॒ण्वञ् छृ॒ण्वन् नू॒तिभिः॑ सीद । \newline
35. ऊ॒तिभिः॑ सीद सीदो॒तिभि॑ रू॒तिभिः॑ सीद॒ साद॑नꣳ॒॒ साद॑नꣳ सीदो॒तिभि॑ रू॒तिभिः॑ सीद॒ साद॑नम् । \newline
36. ऊ॒तिभि॒रित्यू॒ति - भिः॒ । \newline
37. सी॒द॒ साद॑नꣳ॒॒ साद॑नꣳ सीद सीद॒ साद॑नम् । \newline
38. साद॑न॒मिति॒ साद॑नम् । \newline
39. स इदिथ् स स इज् जने॑न॒ जने॒नेथ् स स इज् जने॑न । \newline
40. इज् जने॑न॒ जने॒ने दिज् जने॑न॒ स स जने॒ने दिज् जने॑न॒ सः । \newline
41. जने॑न॒ स स जने॑न॒ जने॑न॒ स वि॒शा वि॒शा स जने॑न॒ जने॑न॒ स वि॒शा । \newline
42. स वि॒शा वि॒शा स स वि॒शा स स वि॒शा स स वि॒शा सः । \newline
43. वि॒शा स स वि॒शा वि॒शा स जन्म॑ना॒ जन्म॑ना॒ स वि॒शा वि॒शा स जन्म॑ना । \newline
44. स जन्म॑ना॒ जन्म॑ना॒ स स जन्म॑ना॒ स स जन्म॑ना॒ स स जन्म॑ना॒ सः । \newline
45. जन्म॑ना॒ स स जन्म॑ना॒ जन्म॑ना॒ स पु॒त्रैः पु॒त्रैः स जन्म॑ना॒ जन्म॑ना॒ स पु॒त्रैः । \newline
46. स पु॒त्रैः पु॒त्रैः स स पु॒त्रैर् वाजं॒ ॅवाज॑म् पु॒त्रैः स स पु॒त्रैर् वाज᳚म् । \newline
47. पु॒त्रैर् वाजं॒ ॅवाज॑म् पु॒त्रैः पु॒त्रैर् वाज॑म् भरते भरते॒ वाज॑म् पु॒त्रैः पु॒त्रैर् वाज॑म् भरते । \newline
48. वाज॑म् भरते भरते॒ वाजं॒ ॅवाज॑म् भरते॒ धना॒ धना॑ भरते॒ वाजं॒ ॅवाज॑म् भरते॒ धना᳚ । \newline
49. भ॒र॒ते॒ धना॒ धना॑ भरते भरते॒ धना॒ नृभि॒र् नृभि॒र् धना॑ भरते भरते॒ धना॒ नृभिः॑ । \newline
50. धना॒ नृभि॒र् नृभि॒र् धना॒ धना॒ नृभिः॑ । \newline
51. नृभि॒रिति॒ नृ - भिः॒ । \newline
52. दे॒वानां॒ ॅयो यो दे॒वाना᳚म् दे॒वानां॒ ॅयः पि॒तर॑म् पि॒तरं॒ ॅयो दे॒वाना᳚म् दे॒वानां॒ ॅयः पि॒तर᳚म् । \newline
53. यः पि॒तर॑म् पि॒तरं॒ ॅयो यः पि॒तर॑ मा॒विवा॑स त्या॒विवा॑सति पि॒तरं॒ ॅयो यः पि॒तर॑ मा॒विवा॑सति । \newline
54. पि॒तर॑ मा॒विवा॑स त्या॒विवा॑सति पि॒तर॑म् पि॒तर॑ मा॒विवा॑सति श्र॒द्धाम॑नाः श्र॒द्धाम॑ना आ॒विवा॑सति पि॒तर॑म् पि॒तर॑ मा॒विवा॑सति श्र॒द्धाम॑नाः । \newline
55. आ॒विवा॑सति श्र॒द्धाम॑नाः श्र॒द्धाम॑ना आ॒विवा॑स त्या॒विवा॑सति श्र॒द्धाम॑ना ह॒विषा॑ ह॒विषा᳚ श्र॒द्धाम॑ना आ॒विवा॑स त्या॒विवा॑सति श्र॒द्धाम॑ना ह॒विषा᳚ । \newline
56. आ॒विवा॑स॒तीत्या᳚ - विवा॑सति । \newline
\pagebreak
\markright{ TS 2.3.14.4  \hfill https://www.vedavms.in \hfill}

\section{ TS 2.3.14.4 }

\textbf{TS 2.3.14.4 } \newline
\textbf{Samhita Paata} \newline

श्र॒द्धाम॑ना ह॒विषा॒ ब्रह्म॑ण॒स्पतिं᳚ ॥ स सु॒ष्टुभा॒ स ऋक्व॑ता ग॒णेन॑ व॒लꣳ रु॑रोज फलि॒गꣳ रवे॑ण । बृह॒स्पति॑रु॒स्त्रिया॑ हव्य॒सूदः॒ कनि॑क्रद॒द्- वाव॑शती॒रुदा॑जत् ॥ मरु॑तो॒ यद्ध॑ वो दि॒वो>7, या वः॒ शर्म॑ >8 ॥अ॒र्य॒मा ऽऽया॑ति वृष॒भस्तुवि॑ष्मान् दा॒ता वसू॑नां पुरुहू॒तो अर्.हन्न्॑ । स॒ह॒स्रा॒क्षो गो᳚त्र॒भिद्-वज्र॑बाहुर॒स्मासु॑ दे॒वो द्रवि॑णं दधातु ॥ये ते᳚ऽर्यमन् ब॒हवो॑ देव॒यानाः॒ पन्था॑नो - [  ] \newline

\textbf{Pada Paata} \newline

श्र॒द्धाम॑ना॒ इति॑ श्र॒द्धा - म॒नाः॒ । ह॒विषा᳚ । ब्रह्म॑णः । पति᳚म् ॥ सः । सु॒ष्टुभेति॑ सु - स्तुभा᳚ । सः । ऋक्व॑ता । ग॒णेन॑ । व॒लम् । रु॒रो॒ज॒ । फ॒लि॒गम् । रवे॑ण ॥ बृह॒स्पतिः॑ । उ॒स्त्रियाः᳚ । ह॒व्य॒सूद॒ इति॑ हव्य - सूदः॑ । कनि॑क्रदत् । वाव॑शतीः । उदिति॑ । आ॒ज॒त् ॥ मरु॑तः । यत् । ह॒ । वः॒ । दि॒वः । या । वः॒ । शर्म॑ ॥ अ॒र्य॒मा । एति॑ । या॒ति॒ । वृ॒ष॒भः । तुवि॑ष्मान् । दा॒ता । वसू॑नाम् । पु॒रु॒हू॒त इति॑ पुरु - हू॒तः । अर्.हन्न्॑ ॥ स॒ह॒स्रा॒क्ष इति॑ सहस्र - अ॒क्षः । गो॒त्र॒भिदिति॑ गोत्र-भित् । वज्र॑बाहु॒रिति॒ वज्र॑ - बा॒हुः॒ । अ॒स्मासु॑ । दे॒वः । द्रवि॑णम् । द॒धा॒तु॒ ॥ ये । ते॒ । अ॒र्य॒म॒न्न् । ब॒हवः॑ । दे॒व॒याना॒ इति॑ देव - यानाः᳚ । पन्था॑नः ।  \newline


\textbf{Krama Paata} \newline

श्र॒द्धाम॑ना ह॒विषा᳚ । श्र॒द्धाम॑ना॒ इति॑ श्र॒द्धा - म॒नाः॒ । ह॒विषा॒ ब्रह्म॑णः । ब्रह्म॑ण॒स्पति᳚म् । पति॒मिति॒ पति᳚म् ॥ स सु॒ष्टुभा᳚ । सु॒ष्टुभा॒ सः । सु॒ष्टुभेति॑ सु - स्तुभा᳚ । स ऋक्व॑ता । ऋक्व॑ता ग॒णेन॑ । ग॒णेन॑ व॒लम् । व॒लꣳ रु॑रोज । रु॒रो॒ज॒ फ॒लि॒गम् । फ॒लि॒गꣳ रवे॑ण । रवे॒णेति॒ रवे॑ण ॥ बृह॒स्पति॑रु॒स्रियाः᳚ । उ॒स्रिया॑ हव्य॒सूदः॑ । ह॒व्य॒सूदः॒ कनि॑क्रदत् । ह॒व्य॒सूद॒ इति॑ हव्य - सूदः॑ । कनि॑क्रद॒द् वाव॑शतीः । वाव॑शती॒रुत् । उदा॑जत् । आ॒ज॒दित्या॑जत् ॥ मरु॑तो॒ यत् । यद्ध॑ । ह॒ वः॒ । वो॒ दि॒वः । दि॒वो या । या वः॑ । वः॒ शर्म॑ । शर्मेति॒ शर्म॑ ॥ अ॒र्य॒मा ऽऽया॑ति । आ या॑ति । या॒ति॒ वृ॒ष॒भः । वृ॒ष॒भस्तुवि॑ष्मान् । तुवि॑ष्मान् दा॒ता । दा॒ता वसू॑नाम् । वसू॑नाम् पुरुहू॒तः । पु॒रु॒हू॒तो अर्.हन्न्॑ । पु॒रु॒हू॒त इति॑ पुरु - हू॒तः । अर्.ह॒न्नित्यर्.हन्न्॑ ॥ स॒ह॒स्रा॒क्षो गो᳚त्र॒भित् । स॒ह॒स्रा॒क्ष इति॑ सहस्र - अ॒क्षः । गो॒त्र॒भिद् वज्र॑बाहुः । गो॒त्र॒भिदिति॑ गोत्र - भित् । वज्र॑बाहुर॒स्मासु॑ । वज्र॑बाहु॒रिति॒ वज्र॑ - बा॒हुः॒ । अ॒स्मासु॑ दे॒वः । दे॒वो द्रवि॑णम् । द्रवि॑णम् दधातु । द॒धा॒त्विति॑ दधातु ॥ ये ते᳚ । ते॒ऽर्य॒म॒न्न्॒ । अ॒र्य॒म॒न् ब॒हवः॑ । ब॒हवो॑ देव॒यानाः᳚ । दे॒व॒यानाः॒ पन्था॑नः । दे॒व॒याना॒ इति॑ देव - यानाः᳚ । पन्था॑नो राजन्न् \newline

\textbf{Jatai Paata} \newline

1. श्र॒द्धाम॑ना ह॒विषा॑ ह॒विषा᳚ श्र॒द्धाम॑नाः श्र॒द्धाम॑ना ह॒विषा᳚ । \newline
2. श्र॒द्धाम॑ना॒ इति॑ श्र॒द्धा - म॒नाः॒ । \newline
3. ह॒विषा॒ ब्रह्म॑णो॒ ब्रह्म॑णो ह॒विषा॑ ह॒विषा॒ ब्रह्म॑णः । \newline
4. ब्रह्म॑ण॒ स्पति॒म् पति॒म् ब्रह्म॑णो॒ ब्रह्म॑ण॒ स्पति᳚म् । \newline
5. पति॒मिति॒ पति᳚म् । \newline
6. स सु॒ष्टुभा॑ सु॒ष्टुभा॒ स स सु॒ष्टुभा᳚ । \newline
7. सु॒ष्टुभा॒ स स सु॒ष्टुभा॑ सु॒ष्टुभा॒ सः । \newline
8. सु॒ष्टुभेति॑ सु - स्तुभा᳚ । \newline
9. स ऋक्व॒त र्‌क्व॑ता॒ स स ऋक्व॑ता । \newline
10. ऋक्व॑ता ग॒णेन॑ ग॒णेन र्‌क्व॒त र्‌क्व॑ता ग॒णेन॑ । \newline
11. ग॒णेन॑ व॒लं ॅव॒लम् ग॒णेन॑ ग॒णेन॑ व॒लम् । \newline
12. व॒लꣳ रु॑रोज रुरोज व॒लं ॅव॒लꣳ रु॑रोज । \newline
13. रु॒रो॒ज॒ फ॒लि॒गम् फ॑लि॒गꣳ रु॑रोज रुरोज फलि॒गम् । \newline
14. फ॒लि॒गꣳ रवे॑ण॒ रवे॑ण फलि॒गम् फ॑लि॒गꣳ रवे॑ण । \newline
15. रवे॒णेति॒ रवे॑ण । \newline
16. बृह॒स्पति॑ रु॒स्रिया॑ उ॒स्रिया॒ बृह॒स्पति॒र् बृह॒स्पति॑ रु॒स्रियाः᳚ । \newline
17. उ॒स्रिया॑ हव्य॒सूदो॑ हव्य॒सूद॑ उ॒स्रिया॑ उ॒स्रिया॑ हव्य॒सूदः॑ । \newline
18. ह॒व्य॒सूदः॒ कनि॑क्रद॒त् कनि॑क्रद द्धव्य॒सूदो॑ हव्य॒सूदः॒ कनि॑क्रदत् । \newline
19. ह॒व्य॒सूद॒ इति॑ हव्य - सूदः॑ । \newline
20. कनि॑क्रद॒द् वाव॑शती॒र् वाव॑शतीः॒ कनि॑क्रद॒त् कनि॑क्रद॒द् वाव॑शतीः । \newline
21. वाव॑शती॒ रुदुद् वाव॑शती॒र् वाव॑शती॒ रुत् । \newline
22. उदा॑जदाज॒ दुदुदा॑जत् । \newline
23. आ॒ज॒दित्या॑जत् । \newline
24. मरु॑तो॒ यद् यन् मरु॑तो॒ मरु॑तो॒ यत् । \newline
25. य द्ध॑ ह॒ यद् य द्ध॑ । \newline
26. ह॒ वो॒ वो॒ ह॒ ह॒ वः॒ । \newline
27. वो॒ दि॒वो दि॒वो वो॑ वो दि॒वः । \newline
28. दि॒वो या या दि॒वो दि॒वो या । \newline
29. या वो॑ वो॒ या या वः॑ । \newline
30. वः॒ शर्म॒ शर्म॑ वो वः॒ शर्म॑ । \newline
31. शर्मेति॒ शर्म॑ । \newline
32. अ॒र्य॒मा ऽऽ या॑ति या॒त्या ऽर्य॒मा । \newline
33. आ या॑ति या॒त्या या॑ति । \newline
34. या॒ति॒ वृ॒ष॒भो वृ॑ष॒भो या॑ति याति वृष॒भः । \newline
35. वृ॒ष॒भ स्तुवि॑ष्मा॒न् तुवि॑ष्मान् वृष॒भो वृ॑ष॒भ स्तुवि॑ष्मान् । \newline
36. तुवि॑ष्मान् दा॒ता दा॒ता तुवि॑ष्मा॒न् तुवि॑ष्मान् दा॒ता । \newline
37. दा॒ता वसू॑नां॒ ॅवसू॑नाम् दा॒ता दा॒ता वसू॑नाम् । \newline
38. वसू॑नाम् पुरुहू॒तः पु॑रुहू॒तो वसू॑नां॒ ॅवसू॑नाम् पुरुहू॒तः । \newline
39. पु॒रु॒हू॒तो अर्.ह॒न् नर्.ह॑न् पुरुहू॒तः पु॑रुहू॒तो अर्.हन्न्॑ । \newline
40. पु॒रु॒हू॒त इति॑ पुरु - हू॒तः । \newline
41. अर्.ह॒न्नित्यर्.हन्न्॑ । \newline
42. स॒ह॒स्रा॒क्षो गो᳚त्र॒भिद् गो᳚त्र॒भिथ् स॑हस्रा॒क्षः स॑हस्रा॒क्षो गो᳚त्र॒भित् । \newline
43. स॒ह॒स्रा॒क्ष इति॑ सहस्र - अ॒क्षः । \newline
44. गो॒त्र॒भिद् वज्र॑बाहु॒र् वज्र॑बाहुर् गोत्र॒भिद् गो᳚त्र॒भिद् वज्र॑बाहुः । \newline
45. गो॒त्र॒भिदिति॑ गोत्र - भित् । \newline
46. वज्र॑बाहु र॒स्मा स्व॒स्मासु॒ वज्र॑बाहु॒र् वज्र॑बाहु र॒स्मासु॑ । \newline
47. वज्र॑बाहु॒रिति॒ वज्र॑ - बा॒हुः॒ । \newline
48. अ॒स्मासु॑ दे॒वो दे॒वो अ॒स्मा स्व॒स्मासु॑ दे॒वः । \newline
49. दे॒वो द्रवि॑ण॒म् द्रवि॑णम् दे॒वो दे॒वो द्रवि॑णम् । \newline
50. द्रवि॑णम् दधातु दधातु॒ द्रवि॑ण॒म् द्रवि॑णम् दधातु । \newline
51. द॒धा॒त्विति॑ दधातु । \newline
52. ये ते॑ ते॒ ये ये ते᳚ । \newline
53. ते॒ ऽर्य॒म॒न् न॒र्य॒म॒न् ते॒ ते॒ ऽर्य॒म॒न्न् । \newline
54. अ॒र्य॒म॒न् ब॒हवो॑ ब॒हवो᳚ ऽर्यमन् नर्यमन् ब॒हवः॑ । \newline
55. ब॒हवो॑ देव॒याना॑ देव॒याना॑ ब॒हवो॑ ब॒हवो॑ देव॒यानाः᳚ । \newline
56. दे॒व॒यानाः॒ पन्था॑नः॒ पन्था॑नो देव॒याना॑ देव॒यानाः॒ पन्था॑नः । \newline
57. दे॒व॒याना॒ इति॑ देव - यानाः᳚ । \newline
58. पन्था॑नो राजन् राज॒न् पन्था॑नः॒ पन्था॑नो राजन्न् । \newline

\textbf{Ghana Paata } \newline

1. श्र॒द्धाम॑ना ह॒विषा॑ ह॒विषा᳚ श्र॒द्धाम॑नाः श्र॒द्धाम॑ना ह॒विषा॒ ब्रह्म॑णो॒ ब्रह्म॑णो ह॒विषा᳚ श्र॒द्धाम॑नाः श्र॒द्धाम॑ना ह॒विषा॒ ब्रह्म॑णः । \newline
2. श्र॒द्धाम॑ना॒ इति॑ श्र॒द्धा - म॒नाः॒ । \newline
3. ह॒विषा॒ ब्रह्म॑णो॒ ब्रह्म॑णो ह॒विषा॑ ह॒विषा॒ ब्रह्म॑ण॒ स्पति॒म् पति॒म् ब्रह्म॑णो ह॒विषा॑ ह॒विषा॒ ब्रह्म॑ण॒ स्पति᳚म् । \newline
4. ब्रह्म॑ण॒ स्पति॒म् पति॒म् ब्रह्म॑णो॒ ब्रह्म॑ण॒ स्पति᳚म् । \newline
5. पति॒मिति॒ पति᳚म् । \newline
6. स सु॒ष्टुभा॑ सु॒ष्टुभा॒ स स सु॒ष्टुभा॒ स स सु॒ष्टुभा॒ स स सु॒ष्टुभा॒ सः । \newline
7. सु॒ष्टुभा॒ स स सु॒ष्टुभा॑ सु॒ष्टुभा॒ स ऋक्व॒त र्‌क्व॑ता॒ स सु॒ष्टुभा॑ सु॒ष्टुभा॒ स ऋक्व॑ता । \newline
8. सु॒ष्टुभेति॑ सु - स्तुभा᳚ । \newline
9. स ऋक्व॒त र्‌क्व॑ता॒ स स ऋक्व॑ता ग॒णेन॑ ग॒णेन र्‌क्व॑ता॒ स स ऋक्व॑ता ग॒णेन॑ । \newline
10. ऋक्व॑ता ग॒णेन॑ ग॒णेन र्‌क्व॒त र्‌क्व॑ता ग॒णेन॑ व॒लं ॅव॒लम् ग॒णेन र्‌क्व॒त र्‌क्व॑ता ग॒णेन॑ व॒लम् । \newline
11. ग॒णेन॑ व॒लं ॅव॒लम् ग॒णेन॑ ग॒णेन॑ व॒लꣳ रु॑रोज रुरोज व॒लम् ग॒णेन॑ ग॒णेन॑ व॒लꣳ रु॑रोज । \newline
12. व॒लꣳ रु॑रोज रुरोज व॒लं ॅव॒लꣳ रु॑रोज फलि॒गम् फ॑लि॒गꣳ रु॑रोज व॒लं ॅव॒लꣳ रु॑रोज फलि॒गम् । \newline
13. रु॒रो॒ज॒ फ॒लि॒गम् फ॑लि॒गꣳ रु॑रोज रुरोज फलि॒गꣳ रवे॑ण॒ रवे॑ण फलि॒गꣳ रु॑रोज रुरोज फलि॒गꣳ रवे॑ण । \newline
14. फ॒लि॒गꣳ रवे॑ण॒ रवे॑ण फलि॒गम् फ॑लि॒गꣳ रवे॑ण । \newline
15. रवे॒णेति॒ रवे॑ण । \newline
16. बृह॒स्पति॑ रु॒स्रिया॑ उ॒स्रिया॒ बृह॒स्पति॒र् बृह॒स्पति॑ रु॒स्रिया॑ हव्य॒सूदो॑ हव्य॒सूद॑ उ॒स्रिया॒ बृह॒स्पति॒र् बृह॒स्पति॑ रु॒स्रिया॑ हव्य॒सूदः॑ । \newline
17. उ॒स्रिया॑ हव्य॒सूदो॑ हव्य॒सूद॑ उ॒स्रिया॑ उ॒स्रिया॑ हव्य॒सूदः॒ कनि॑क्रद॒त् कनि॑क्रद द्धव्य॒सूद॑ उ॒स्रिया॑ उ॒स्रिया॑ हव्य॒सूदः॒ कनि॑क्रदत् । \newline
18. ह॒व्य॒सूदः॒ कनि॑क्रद॒त् कनि॑क्रद द्धव्य॒सूदो॑ हव्य॒सूदः॒ कनि॑क्रद॒द् वाव॑शती॒र् वाव॑शतीः॒ कनि॑क्रद द्धव्य॒सूदो॑ हव्य॒सूदः॒ कनि॑क्रद॒द् वाव॑शतीः । \newline
19. ह॒व्य॒सूद॒ इति॑ हव्य - सूदः॑ । \newline
20. कनि॑क्रद॒द् वाव॑शती॒र् वाव॑शतीः॒ कनि॑क्रद॒त् कनि॑क्रद॒द् वाव॑शती॒ रुदुद् वाव॑शतीः॒ कनि॑क्रद॒त् कनि॑क्रद॒द् वाव॑शती॒रुत् । \newline
21. वाव॑शती॒ रुदुद् वाव॑शती॒र् वाव॑शती॒ रुदा॑ज दाज॒दुद् वाव॑शती॒र् वाव॑शती॒ रुदा॑जत् । \newline
22. उदा॑जदाज॒ दुदुदा॑जत् । \newline
23. आ॒ज॒दित्या॑जत् । \newline
24. मरु॑तो॒ यद् यन् मरु॑तो॒ मरु॑तो॒ यद्ध॑ ह॒ यन् मरु॑तो॒ मरु॑तो॒ यद्ध॑ । \newline
25. यद्ध॑ ह॒ यद् यद्ध॑ वो वो ह॒ यद् यद्ध॑ वः । \newline
26. ह॒ वो॒ वो॒ ह॒ ह॒ वो॒ दि॒वो दि॒वो वो॑ ह ह वो दि॒वः । \newline
27. वो॒ दि॒वो दि॒वो वो॑ वो दि॒वो या या दि॒वो वो॑ वो दि॒वो या । \newline
28. दि॒वो या या दि॒वो दि॒वो या वो॑ वो॒ या दि॒वो दि॒वो या वः॑ । \newline
29. या वो॑ वो॒ या या वः॒ शर्म॒ शर्म॑ वो॒ या या वः॒ शर्म॑ । \newline
30. वः॒ शर्म॒ शर्म॑ वो वः॒ शर्म॑ । \newline
31. शर्मेति॒ शर्म॑ । \newline
32. अ॒र्य॒मा ऽऽ या॑ति या॒त्या ऽर्य॒मा ऽर्य॒मा ऽऽ या॑ति वृष॒भो वृ॑ष॒भो या॒त्या ऽर्य॒मा ऽर्य॒मा ऽऽ या॑ति वृष॒भः । \newline
33. आ या॑ति या॒त्या या॑ति वृष॒भो वृ॑ष॒भो या॒त्या या॑ति वृष॒भः । \newline
34. या॒ति॒ वृ॒ष॒भो वृ॑ष॒भो या॑ति याति वृष॒भ स्तुवि॑ष्मा॒न् तुवि॑ष्मान् वृष॒भो या॑ति याति वृष॒भ स्तुवि॑ष्मान् । \newline
35. वृ॒ष॒भ स्तुवि॑ष्मा॒न् तुवि॑ष्मान् वृष॒भो वृ॑ष॒भ स्तुवि॑ष्मान् दा॒ता दा॒ता तुवि॑ष्मान् वृष॒भो वृ॑ष॒भ स्तुवि॑ष्मान् दा॒ता । \newline
36. तुवि॑ष्मान् दा॒ता दा॒ता तुवि॑ष्मा॒न् तुवि॑ष्मान् दा॒ता वसू॑नां॒ ॅवसू॑नाम् दा॒ता तुवि॑ष्मा॒न् तुवि॑ष्मान् दा॒ता वसू॑नाम् । \newline
37. दा॒ता वसू॑नां॒ ॅवसू॑नाम् दा॒ता दा॒ता वसू॑नाम् पुरुहू॒तः पु॑रुहू॒तो वसू॑नाम् दा॒ता दा॒ता वसू॑नाम् पुरुहू॒तः । \newline
38. वसू॑नाम् पुरुहू॒तः पु॑रुहू॒तो वसू॑नां॒ ॅवसू॑नाम् पुरुहू॒तो अर्.ह॒न् नर्.ह॑न् पुरुहू॒तो वसू॑नां॒ ॅवसू॑नाम् 
पुरुहू॒तो अर्.हन्न्॑ । \newline
39. पु॒रु॒हू॒तो अर्.ह॒न् नर्.ह॑न् पुरुहू॒तः पु॑रुहू॒तो अर्.हन्न्॑ । \newline
40. पु॒रु॒हू॒त इति॑ पुरु - हू॒तः । \newline
41. अर्.ह॒न्नित्यर्.हन्न्॑ । \newline
42. स॒ह॒स्रा॒क्षो गो᳚त्र॒भिद् गो᳚त्र॒भिथ् स॑हस्रा॒क्षः स॑हस्रा॒क्षो गो᳚त्र॒भिद् वज्र॑बाहु॒र् वज्र॑बाहुर् गोत्र॒भिथ् स॑हस्रा॒क्षः स॑हस्रा॒क्षो गो᳚त्र॒भिद् वज्र॑बाहुः । \newline
43. स॒ह॒स्रा॒क्ष इति॑ सहस्र - अ॒क्षः । \newline
44. गो॒त्र॒भिद् वज्र॑बाहु॒र् वज्र॑बाहुर् गोत्र॒भिद् गो᳚त्र॒भिद् वज्र॑बाहु र॒स्मा स्व॒स्मासु॒ वज्र॑बाहुर् गोत्र॒भिद् गो᳚त्र॒भिद् वज्र॑बाहु र॒स्मासु॑ । \newline
45. गो॒त्र॒भिदिति॑ गोत्र - भित् । \newline
46. वज्र॑बाहु र॒स्मा स्व॒स्मासु॒ वज्र॑बाहु॒र् वज्र॑बाहु र॒स्मासु॑ दे॒वो दे॒वो अ॒स्मासु॒ वज्र॑बाहु॒र् वज्र॑बाहु र॒स्मासु॑ दे॒वः । \newline
47. वज्र॑बाहु॒रिति॒ वज्र॑ - बा॒हुः॒ । \newline
48. अ॒स्मासु॑ दे॒वो दे॒वो अ॒स्मा स्व॒स्मासु॑ दे॒वो द्रवि॑ण॒म् द्रवि॑णम् दे॒वो अ॒स्मा स्व॒स्मासु॑ दे॒वो द्रवि॑णम् । \newline
49. दे॒वो द्रवि॑ण॒म् द्रवि॑णम् दे॒वो दे॒वो द्रवि॑णम् दधातु दधातु॒ द्रवि॑णम् दे॒वो दे॒वो द्रवि॑णम् दधातु । \newline
50. द्रवि॑णम् दधातु दधातु॒ द्रवि॑ण॒म् द्रवि॑णम् दधातु । \newline
51. द॒धा॒त्विति॑ दधातु । \newline
52. ये ते॑ ते॒ ये ये ते᳚ ऽर्यमन् नर्यमन् ते॒ ये ये ते᳚ ऽर्यमन्न् । \newline
53. ते॒ ऽर्य॒म॒न् न॒र्य॒म॒न् ते॒ ते॒ ऽर्य॒म॒न् ब॒हवो॑ ब॒हवो᳚ ऽर्यमन् ते ते ऽर्यमन् ब॒हवः॑ । \newline
54. अ॒र्य॒म॒न् ब॒हवो॑ ब॒हवो᳚ ऽर्यमन् नर्यमन् ब॒हवो॑ देव॒याना॑ देव॒याना॑ ब॒हवो᳚ ऽर्यमन् नर्यमन् ब॒हवो॑ देव॒यानाः᳚ । \newline
55. ब॒हवो॑ देव॒याना॑ देव॒याना॑ ब॒हवो॑ ब॒हवो॑ देव॒यानाः॒ पन्था॑नः॒ पन्था॑नो देव॒याना॑ ब॒हवो॑ ब॒हवो॑ देव॒यानाः॒ पन्था॑नः । \newline
56. दे॒व॒यानाः॒ पन्था॑नः॒ पन्था॑नो देव॒याना॑ देव॒यानाः॒ पन्था॑नो राजन् राज॒न् पन्था॑नो देव॒याना॑ देव॒यानाः॒ पन्था॑नो राजन्न् । \newline
57. दे॒व॒याना॒ इति॑ देव - यानाः᳚ । \newline
58. पन्था॑नो राजन् राज॒न् पन्था॑नः॒ पन्था॑नो राजन् दि॒वो दि॒वो रा॑ज॒न् पन्था॑नः॒ पन्था॑नो राजन् दि॒वः । \newline
\pagebreak
\markright{ TS 2.3.14.5  \hfill https://www.vedavms.in \hfill}

\section{ TS 2.3.14.5 }

\textbf{TS 2.3.14.5 } \newline
\textbf{Samhita Paata} \newline

राजन् दि॒व आ॒चर॑न्ति । तेभि॑र्नो देव॒ महि॒ शर्म॑ यच्छ॒ शं न॑ एधि द्वि॒पदे॒ शं चतु॑ष्पदे ॥ बु॒द्ध्नादग्र॒-मङ्गि॑रोभि-र्गृणा॒नो वि पर्व॑तस्य दृꣳहि॒तान्यै॑रत् । रु॒जद्रोधाꣳ॑सिकृ॒त्रिमा᳚ण्येषाꣳ॒॒सोम॑स्य॒तामद॒इन्द्र॑श्चकार ॥ बु॒द्ध्नादग्रे॑ण॒ वि मि॑माय॒ मानै॒र्वज्रे॑ण॒ खान्य॑तृणन्न॒दीनां᳚ । वृथा॑ ऽसृजत् प॒थिभि॑ र्दीर्घया॒थैः सोम॑स्य॒ ता मद॒ इन्द्र॑श्चकार ॥ \newline

\textbf{Pada Paata} \newline

रा॒ज॒न्न् । दि॒वः । आ॒चर॒न्तीत्या᳚ - चर॑न्ति ॥ तेभिः॑ । नः॒ । दे॒व॒ । महि॑ । शर्म॑ । य॒च्छ॒ । शम् । नः॒ । ए॒धि॒ । द्वि॒पद॒ इति॑ द्वि - पदे᳚ । शम् । चतु॑ष्पद॒ इति॒ चतुः॑ - प॒दे॒ ॥ बु॒द्ध्नात् । अग्र᳚म् । अङ्गि॑रोभि॒रित्यङ्गि॑रः-भिः॒ । गृ॒णा॒नः । वीति॑ । पर्व॑तस्य । दृꣳ॒॒हि॒तानि॑ । ऐ॒र॒त् ॥ रु॒जत् । रोधाꣳ॑सि । कृ॒त्रिमा॑णि । ए॒षा॒म् । सोम॑स्य । ता । मदे᳚ । इन्द्रः॑ । च॒का॒र॒ ॥ बु॒द्ध्नात् । अग्रे॑ण । वीति॑ । मि॒मा॒य॒ । मानैः᳚ । वज्रे॑ण । खानि॑ । अ॒तृ॒ण॒त् । न॒दीना᳚म् ॥ वृथा᳚ । अ॒सृ॒ज॒त् । प॒थिभि॒रिति॑ प॒थि - भिः॒ । दी॒र्घ॒या॒थैरिति॑ दीर्घ - या॒थैः । सोम॑स्य । ता । मदे᳚ । इन्द्रः॑ । च॒का॒र॒ ॥  \newline


\textbf{Krama Paata} \newline

रा॒ज॒न् दि॒वः । दि॒व आ॒चर॑न्ति । आ॒चर॒न्तीत्या᳚ - चर॑न्ति ॥ तेभि॑र् नः । नो॒ दे॒व॒ । दे॒व॒ महि॑ । महि॒ शर्म॑ । शर्म॑ यच्छ । य॒च्छ॒ शम् । शम् नः॑ । न॒ ए॒धि॒ । ए॒धि॒ द्वि॒पदे᳚ । द्वि॒पदे॒ शम् । द्वि॒पद॒ इति॑ द्वि - पदे᳚ । शम् चतु॑ष्पदे । चतु॑ष्पद॒ इति॒ चतुः॑ - प॒दे॒ ॥ बु॒द्ध्नादग्र᳚म् । अग्र॒मङ्गि॑रोभिः । अङ्गि॑रोभिर् गृणा॒नः । अङ्गि॑रोभि॒रित्यङ्गि॑रः - भिः॒ । गृ॒णा॒नो वि । वि पर्व॑तस्य । पर्व॑तस्य दृꣳहि॒तानि॑ । दृꣳ॒॒हि॒तान्यै॑रत् । ऐ॒र॒दित्यै॑रत् ॥ रु॒जद् रोधाꣳ॑सि । रोधाꣳ॑सि कृ॒त्रिमा॑णि । कृ॒त्रिमा᳚ण्येषाम् । ए॒षाꣳ॒॒ सोम॑स्य । सोम॑स्य॒ ता । ता मदे᳚ । मद॒ इन्द्रः॑ । इन्द्र॑श्चकार । च॒का॒रेति॑ चकार ॥ बु॒द्ध्नादग्रे॑ण । अग्रे॑ण॒ वि । वि मि॑माय । मि॒मा॒य॒ मानैः᳚ । मानै॒र् वज्रे॑ण । वज्रे॑ण॒ खानि॑ । खान्य॑तृणत् । अ॒तृ॒ण॒न् न॒दीना᳚म् । न॒दीना॒मिति॑ न॒दीना᳚म् ॥ वृथा॑ ऽसृजत् । अ॒सृ॒ज॒त् प॒थिभिः॑ । प॒थिभि॑र् दीर्घया॒थैः । प॒थिभि॒रिति॑ प॒थि - भिः॒ । दी॒र्घ॒या॒थैः सोम॑स्य । दी॒र्घ॒या॒थैरिति॑ दीर्घ - या॒थैः । सोम॑स्य॒ ता । ता मदे᳚ । मद॒ इन्द्रः॑ । इन्द्र॑श्चकार । च॒का॒रेति॑ चकार । \newline

\textbf{Jatai Paata} \newline

1. रा॒ज॒न् दि॒वो दि॒वो रा॑जन् राजन् दि॒वः । \newline
2. दि॒व आ॒चर॑ न्त्या॒चर॑न्ति दि॒वो दि॒व आ॒चर॑न्ति । \newline
3. आ॒चर॒न्तीत्या᳚ - चर॑न्ति । \newline
4. तेभि॑र् नो न॒ स्तेभि॒ स्तेभि॑र् नः । \newline
5. नो॒ दे॒व॒ दे॒व॒ नो॒ नो॒ दे॒व॒ । \newline
6. दे॒व॒ महि॒ महि॑ देव देव॒ महि॑ । \newline
7. महि॒ शर्म॒ शर्म॒ महि॒ महि॒ शर्म॑ । \newline
8. शर्म॑ यच्छ यच्छ॒ शर्म॒ शर्म॑ यच्छ । \newline
9. य॒च्छ॒ शꣳ शं ॅय॑च्छ यच्छ॒ शम् । \newline
10. शम् नो॑ नः॒ शꣳ शम् नः॑ । \newline
11. न॒ ए॒ध्ये॒धि॒ नो॒ न॒ ए॒धि॒ । \newline
12. ए॒धि॒ द्वि॒पदे᳚ द्वि॒पद॑ एध्येधि द्वि॒पदे᳚ । \newline
13. द्वि॒पदे॒ शꣳ शम् द्वि॒पदे᳚ द्वि॒पदे॒ शम् । \newline
14. द्वि॒पद॒ इति॑ द्वि - पदे᳚ । \newline
15. शम् चतु॑ष्पदे॒ चतु॑ष्पदे॒ शꣳ शम् चतु॑ष्पदे । \newline
16. चतु॑ष्पद॒ इति॒ चतुः॑ - प॒दे॒ । \newline
17. बु॒द्ध्नादग्र॒ मग्र॑म् बु॒द्ध्नाद् बु॒द्ध्नादग्र᳚म् । \newline
18. अग्र॒ मङ्गि॑रोभि॒ रङ्गि॑रोभि॒ रग्र॒ मग्र॒ मङ्गि॑रोभिः । \newline
19. अङ्गि॑रोभिर् गृणा॒नो गृ॑णा॒नो अङ्गि॑रोभि॒ रङ्गि॑रोभिर् गृणा॒नः । \newline
20. अङ्गि॑रोभि॒रित्यङ्गि॑रः - भिः॒ । \newline
21. गृ॒णा॒नो वि वि गृ॑णा॒नो गृ॑णा॒नो वि । \newline
22. वि पर्व॑तस्य॒ पर्व॑तस्य॒ वि वि पर्व॑तस्य । \newline
23. पर्व॑तस्य दृꣳहि॒तानि॑ दृꣳहि॒तानि॒ पर्व॑तस्य॒ पर्व॑तस्य दृꣳहि॒तानि॑ । \newline
24. दृꣳ॒॒हि॒ता न्यै॑रदैरद् दृꣳहि॒तानि॑ दृꣳहि॒ता न्यै॑रत् । \newline
25. ऐ॒र॒दित्यै॑रत् । \newline
26. रु॒जद् रोधाꣳ॑सि॒ रोधाꣳ॑सि रु॒जद् रु॒जद् रोधाꣳ॑सि । \newline
27. रोधाꣳ॑सि कृ॒त्रिमा॑णि कृ॒त्रिमा॑णि॒ रोधाꣳ॑सि॒ रोधाꣳ॑सि कृ॒त्रिमा॑णि । \newline
28. कृ॒त्रिमा᳚ ण्येषा मेषाम् कृ॒त्रिमा॑णि कृ॒त्रिमा᳚ ण्येषाम् । \newline
29. ए॒षाꣳ॒॒ सोम॑स्य॒ सोम॑स्यैषा मेषाꣳ॒॒ सोम॑स्य । \newline
30. सोम॑स्य॒ ता ता सोम॑स्य॒ सोम॑स्य॒ ता । \newline
31. ता मदे॒ मदे॒ ता ता मदे᳚ । \newline
32. मद॒ इन्द्र॒ इन्द्रो॒ मदे॒ मद॒ इन्द्रः॑ । \newline
33. इन्द्र॑ श्चकार चका॒रे न्द्र॒ इन्द्र॑ श्चकार । \newline
34. च॒का॒रेति॑ चकार । \newline
35. बु॒द्ध्ना दग्रे॒णाग्रे॑ण बु॒द्ध्नाद् बु॒द्ध्ना दग्रे॑ण । \newline
36. अग्रे॑ण॒ वि व्यग्रे॒णा ग्रे॑ण॒ वि । \newline
37. वि मि॑माय मिमाय॒ वि वि मि॑माय । \newline
38. मि॒मा॒य॒ मानै॒र् मानै᳚र् मिमाय मिमाय॒ मानैः᳚ । \newline
39. मानै॒र् वज्रे॑ण॒ वज्रे॑ण॒ मानै॒र् मानै॒र् वज्रे॑ण । \newline
40. वज्रे॑ण॒ खानि॒ खानि॒ वज्रे॑ण॒ वज्रे॑ण॒ खानि॑ । \newline
41. खान्य॑तृण दतृण॒त् खानि॒ खान्य॑तृणत् । \newline
42. अ॒तृ॒ण॒न् न॒दीना᳚म् न॒दीना॑ मतृण दतृणन् न॒दीना᳚म् । \newline
43. न॒दीना॒मिति॑ न॒दीना᳚म् । \newline
44. वृथा॑ ऽसृज दसृज॒द् वृथा॒ वृथा॑ ऽसृजत् । \newline
45. अ॒सृ॒ज॒त् प॒थिभिः॑ प॒थिभि॑ रसृज दसृजत् प॒थिभिः॑ । \newline
46. प॒थिभि॑र् दीर्घया॒थैर् दी᳚र्घया॒थैः प॒थिभिः॑ प॒थिभि॑र् दीर्घया॒थैः । \newline
47. प॒थिभि॒रिति॑ प॒थि - भिः॒ । \newline
48. दी॒र्घ॒या॒थैः सोम॑स्य॒ सोम॑स्य दीर्घया॒थैर् दी᳚र्घया॒थैः सोम॑स्य । \newline
49. दी॒र्घ॒या॒थैरिति॑ दीर्घ - या॒थैः । \newline
50. सोम॑स्य॒ ता ता सोम॑स्य॒ सोम॑स्य॒ ता । \newline
51. ता मदे॒ मदे॒ ता ता मदे᳚ । \newline
52. मद॒ इन्द्र॒ इन्द्रो॒ मदे॒ मद॒ इन्द्रः॑ । \newline
53. इन्द्र॑ श्चकार चका॒रे न्द्र॒ इन्द्र॑ श्चकार । \newline
54. च॒का॒रेति॑ चकार । \newline

\textbf{Ghana Paata } \newline

1. रा॒ज॒न् दि॒वो दि॒वो रा॑जन् राजन् दि॒व आ॒चर॑ न्त्या॒चर॑न्ति दि॒वो रा॑जन् राजन् दि॒व आ॒चर॑न्ति । \newline
2. दि॒व आ॒चर॑ न्त्या॒चर॑न्ति दि॒वो दि॒व आ॒चर॑न्ति । \newline
3. आ॒चर॒न्तीत्या᳚ - चर॑न्ति । \newline
4. तेभि॑र् नो न॒ स्तेभि॒ स्तेभि॑र् नो देव देव न॒ स्तेभि॒ स्तेभि॑र् नो देव । \newline
5. नो॒ दे॒व॒ दे॒व॒ नो॒ नो॒ दे॒व॒ महि॒ महि॑ देव नो नो देव॒ महि॑ । \newline
6. दे॒व॒ महि॒ महि॑ देव देव॒ महि॒ शर्म॒ शर्म॒ महि॑ देव देव॒ महि॒ शर्म॑ । \newline
7. महि॒ शर्म॒ शर्म॒ महि॒ महि॒ शर्म॑ यच्छ यच्छ॒ शर्म॒ महि॒ महि॒ शर्म॑ यच्छ । \newline
8. शर्म॑ यच्छ यच्छ॒ शर्म॒ शर्म॑ यच्छ॒ शꣳ शं ॅय॑च्छ॒ शर्म॒ शर्म॑ यच्छ॒ शम् । \newline
9. य॒च्छ॒ शꣳ शं ॅय॑च्छ यच्छ॒ शम् नो॑ नः॒ शं ॅय॑च्छ यच्छ॒ शम् नः॑ । \newline
10. शम् नो॑ नः॒ शꣳ शम् न॑ एध्येधि नः॒ शꣳ शम् न॑ एधि । \newline
11. न॒ ए॒ध्ये॒धि॒ नो॒ न॒ ए॒धि॒ द्वि॒पदे᳚ द्वि॒पद॑ एधि नो न एधि द्वि॒पदे᳚ । \newline
12. ए॒धि॒ द्वि॒पदे᳚ द्वि॒पद॑ एध्येधि द्वि॒पदे॒ शꣳ शम् द्वि॒पद॑ एध्येधि द्वि॒पदे॒ शम् । \newline
13. द्वि॒पदे॒ शꣳ शम् द्वि॒पदे᳚ द्वि॒पदे॒ शम् चतु॑ष्पदे॒ चतु॑ष्पदे॒ शम् द्वि॒पदे᳚ द्वि॒पदे॒ शम् चतु॑ष्पदे । \newline
14. द्वि॒पद॒ इति॑ द्वि - पदे᳚ । \newline
15. शम् चतु॑ष्पदे॒ चतु॑ष्पदे॒ शꣳ शम् चतु॑ष्पदे । \newline
16. चतु॑ष्पद॒ इति॒ चतुः॑ - प॒दे॒ । \newline
17. बु॒द्ध्नादग्र॒ मग्र॑म् बु॒द्ध्नाद् बु॒द्ध्नादग्र॒ मङ्गि॑रोभि॒ रङ्गि॑रोभि॒ रग्र॑म् बु॒द्ध्नाद् बु॒द्ध्नादग्र॒ मङ्गि॑रोभिः । \newline
18. अग्र॒ मङ्गि॑रोभि॒ रङ्गि॑रोभि॒ रग्र॒ मग्र॒ मङ्गि॑रोभिर् गृणा॒नो गृ॑णा॒नो अङ्गि॑रोभि॒ रग्र॒ मग्र॒ मङ्गि॑रोभिर् गृणा॒नः । \newline
19. अङ्गि॑रोभिर् गृणा॒नो गृ॑णा॒नो अङ्गि॑रोभि॒ रङ्गि॑रोभिर् गृणा॒नो वि वि गृ॑णा॒नो अङ्गि॑रोभि॒ रङ्गि॑रोभिर् गृणा॒नो वि । \newline
20. अङ्गि॑रोभि॒रित्यङ्गि॑रः - भिः॒ । \newline
21. गृ॒णा॒नो वि वि गृ॑णा॒नो गृ॑णा॒नो वि पर्व॑तस्य॒ पर्व॑तस्य॒ वि गृ॑णा॒नो गृ॑णा॒नो वि पर्व॑तस्य । \newline
22. वि पर्व॑तस्य॒ पर्व॑तस्य॒ वि वि पर्व॑तस्य दृꣳहि॒तानि॑ दृꣳहि॒तानि॒ पर्व॑तस्य॒ वि वि पर्व॑तस्य दृꣳहि॒तानि॑ । \newline
23. पर्व॑तस्य दृꣳहि॒तानि॑ दृꣳहि॒तानि॒ पर्व॑तस्य॒ पर्व॑तस्य दृꣳहि॒ता न्यै॑ रदैरद् दृꣳहि॒तानि॒ पर्व॑तस्य॒ पर्व॑तस्य दृꣳहि॒ता न्यै॑रत् । \newline
24. दृꣳ॒॒हि॒ता न्यै॑ रदैरद् दृꣳहि॒तानि॑ दृꣳहि॒ता न्यै॑रत् । \newline
25. ऐ॒र॒दित्यै॑रत् । \newline
26. रु॒जद् रोधाꣳ॑सि॒ रोधाꣳ॑सि रु॒जद् रु॒जद् रोधाꣳ॑सि कृ॒त्रिमा॑णि कृ॒त्रिमा॑णि॒ रोधाꣳ॑सि रु॒जद् रु॒जद् रोधाꣳ॑सि कृ॒त्रिमा॑णि । \newline
27. रोधाꣳ॑सि कृ॒त्रिमा॑णि कृ॒त्रिमा॑णि॒ रोधाꣳ॑सि॒ रोधाꣳ॑सि कृ॒त्रिमा᳚ण्येषा मेषाम् कृ॒त्रिमा॑णि॒ रोधाꣳ॑सि॒ रोधाꣳ॑सि कृ॒त्रिमा᳚ण्येषाम् । \newline
28. कृ॒त्रिमा᳚ ण्येषा मेषाम् कृ॒त्रिमा॑णि कृ॒त्रिमा᳚ ण्येषाꣳ॒॒ सोम॑स्य॒ सोम॑स्यैषाम् कृ॒त्रिमा॑णि कृ॒त्रिमा᳚ण्येषाꣳ॒॒ सोम॑स्य । \newline
29. ए॒षाꣳ॒॒ सोम॑स्य॒ सोम॑स्यैषा मेषाꣳ॒॒ सोम॑स्य॒ ता ता सोम॑स्यैषा मेषाꣳ॒॒ सोम॑स्य॒ ता । \newline
30. सोम॑स्य॒ ता ता सोम॑स्य॒ सोम॑स्य॒ ता मदे॒ मदे॒ ता सोम॑स्य॒ सोम॑स्य॒ ता मदे᳚ । \newline
31. ता मदे॒ मदे॒ ता ता मद॒ इन्द्र॒ इन्द्रो॒ मदे॒ ता ता मद॒ इन्द्रः॑ । \newline
32. मद॒ इन्द्र॒ इन्द्रो॒ मदे॒ मद॒ इन्द्र॑ श्चकार चका॒रे न्द्रो॒ मदे॒ मद॒ इन्द्र॑ श्चकार । \newline
33. इन्द्र॑ श्चकार चका॒रे न्द्र॒ इन्द्र॑ श्चकार । \newline
34. च॒का॒रेति॑ चकार । \newline
35. बु॒द्ध्ना दग्रे॒णाग्रे॑ण बु॒द्ध्नाद् बु॒द्ध्ना दग्रे॑ण॒ वि व्यग्रे॑ण बु॒द्ध्नाद् बु॒द्ध्ना दग्रे॑ण॒ वि । \newline
36. अग्रे॑ण॒ वि व्यग्रे॒णाग्रे॑ण॒ वि मि॑माय मिमाय॒ व्यग्रे॒णाग्रे॑ण॒ वि मि॑माय । \newline
37. वि मि॑माय मिमाय॒ वि वि मि॑माय॒ मानै॒र् मानै᳚र् मिमाय॒ वि वि मि॑माय॒ मानैः᳚ । \newline
38. मि॒मा॒य॒ मानै॒र् मानै᳚र् मिमाय मिमाय॒ मानै॒र् वज्रे॑ण॒ वज्रे॑ण॒ मानै᳚र् मिमाय मिमाय॒ मानै॒र् वज्रे॑ण । \newline
39. मानै॒र् वज्रे॑ण॒ वज्रे॑ण॒ मानै॒र् मानै॒र् वज्रे॑ण॒ खानि॒ खानि॒ वज्रे॑ण॒ मानै॒र् मानै॒र् वज्रे॑ण॒ खानि॑ । \newline
40. वज्रे॑ण॒ खानि॒ खानि॒ वज्रे॑ण॒ वज्रे॑ण॒ खान्य॑तृण दतृण॒त् खानि॒ वज्रे॑ण॒ वज्रे॑ण॒ खान्य॑तृणत् । \newline
41. खान्य॑तृण दतृण॒त् खानि॒ खान्य॑तृणन् न॒दीना᳚म् न॒दीना॑ मतृण॒त् खानि॒ खान्य॑तृणन् न॒दीना᳚म् । \newline
42. अ॒तृ॒ण॒न् न॒दीना᳚म् न॒दीना॑ मतृण दतृणन् न॒दीना᳚म् । \newline
43. न॒दीना॒मिति॑ न॒दीना᳚म् । \newline
44. वृथा॑ ऽसृज दसृज॒द् वृथा॒ वृथा॑ ऽसृजत् प॒थिभिः॑ प॒थिभि॑ रसृज॒द् वृथा॒ वृथा॑ ऽसृजत् प॒थिभिः॑ । \newline
45. अ॒सृ॒ज॒त् प॒थिभिः॑ प॒थिभि॑ रसृज दसृजत् प॒थिभि॑र् दीर्घया॒थैर् दी᳚र्घया॒थैः प॒थिभि॑ रसृज दसृजत् प॒थिभि॑र् दीर्घया॒थैः । \newline
46. प॒थिभि॑र् दीर्घया॒थैर् दी᳚र्घया॒थैः प॒थिभिः॑ प॒थिभि॑र् दीर्घया॒थैः सोम॑स्य॒ सोम॑स्य दीर्घया॒थैः प॒थिभिः॑ प॒थिभि॑र् दीर्घया॒थैः सोम॑स्य । \newline
47. प॒थिभि॒रिति॑ प॒थि - भिः॒ । \newline
48. दी॒र्घ॒या॒थैः सोम॑स्य॒ सोम॑स्य दीर्घया॒थैर् दी᳚र्घया॒थैः सोम॑स्य॒ ता ता सोम॑स्य दीर्घया॒थैर् दी᳚र्घया॒थैः सोम॑स्य॒ ता । \newline
49. दी॒र्घ॒या॒थैरिति॑ दीर्घ - या॒थैः । \newline
50. सोम॑स्य॒ ता ता सोम॑स्य॒ सोम॑स्य॒ ता मदे॒ मदे॒ ता सोम॑स्य॒ सोम॑स्य॒ ता मदे᳚ । \newline
51. ता मदे॒ मदे॒ ता ता मद॒ इन्द्र॒ इन्द्रो॒ मदे॒ ता ता मद॒ इन्द्रः॑ । \newline
52. मद॒ इन्द्र॒ इन्द्रो॒ मदे॒ मद॒ इन्द्र॑ श्चकार चका॒रे न्द्रो॒ मदे॒ मद॒ इन्द्र॑ श्चकार । \newline
53. इन्द्र॑ श्चकार चका॒रे न्द्र॒ इन्द्र॑ श्चकार । \newline
54. च॒का॒रेति॑ चकार । \newline
\pagebreak
\markright{ TS 2.3.14.6  \hfill https://www.vedavms.in \hfill}

\section{ TS 2.3.14.6 }

\textbf{TS 2.3.14.6 } \newline
\textbf{Samhita Paata} \newline

प्र यो ज॒ज्ञे वि॒द्वाꣳ अ॒स्य बन्धुं॒ ॅविश्वा॑नि दे॒वो जनि॑मा विवक्ति । ब्रह्म॒ ब्रह्म॑ण॒ उज्ज॑भार॒ मद्ध्या᳚न्नी॒चादु॒च्चा स्व॒धया॒ऽभि प्रत॑स्थौ ॥म॒हान् म॒ही अ॑स्तभाय॒द्वि जा॒तो द्याꣳ सद्म॒ पार्थि॑वं च॒ रजः॑ । स बु॒द्ध्नादा᳚ष्ट ज॒नुषा॒ऽभ्यग्रं॒ बृह॒स्पति॑ र्दे॒वता॒यस्य॑ स॒म्राट् ॥ बु॒द्ध्नाद्यो अग्र॑म॒भ्यर्त्योज॑सा॒ बृह॒स्पति॒मा वि॑वासन्ति दे॒वाः ( ) । भि॒नद्व॒लं ॅवि पुरो॑ दर्दरीति॒ कनि॑क्रद॒थ् सुव॑र॒पो जि॑गाय ॥ \newline

\textbf{Pada Paata} \newline

प्रेति॑ । यः । ज॒ज्ञे । वि॒द्वान् । अ॒स्य । बन्धु᳚म् । विश्वा॑नि । दे॒वः । जनि॑मा । वि॒व॒क्ति॒ ॥ ब्रह्म॑ । ब्रह्म॑णः । उदिति॑ । ज॒भा॒र॒ । मद्ध्या᳚त् । नी॒चा । उ॒च्चा । स्व॒धयेति॑ स्व - धया᳚ । अ॒भि । प्रेति॑ । त॒स्थौ॒ ॥ म॒हान् । म॒ही इति॑ । अ॒स्त॒भा॒य॒त् । वीति॑ । जा॒तः । द्याम् । सद्म॑ । पार्त्थि॑वम् । च॒ । रजः॑ ॥ सः । बु॒द्ध्नात् । आ॒ष्ट॒ । ज॒नुषा᳚ । अ॒भीति॑ । अग्र᳚म् । बृह॒स्पतिः॑ । दे॒वता᳚ । यस्य॑ । स॒म्राडिति॑ सं-राट् ॥ बु॒द्ध्नात् । यः । अग्र᳚म् । अ॒भ्यर्तीत्य॑भि-अर्ति॑ । ओज॑सा । बृह॒स्पति᳚म् । एति॑ । वि॒वा॒स॒न्ति॒ । दे॒वाः ( ) ॥ भि॒नत् । व॒लम् । वीति॑ । पुरः॑ । द॒र्द॒री॒ति॒ । कनि॑क्रदत् । सुवः॑ । अ॒पः । जि॒गा॒य॒ ॥  \newline


\textbf{Krama Paata} \newline

प्र यः । यो ज॒ज्ञे । ज॒ज्ञे वि॒द्वान् । वि॒द्वाꣳ अ॒स्य । अ॒स्य बन्धु᳚म् । बन्धु॒म् ॅविश्वा॑नि । विश्वा॑नि दे॒वः । दे॒वो जनि॑मा । जनि॑मा विवक्ति । वि॒व॒क्तीति॑ विवक्ति ॥ बह्म॒ ब्रह्म॑णः । ब्रह्म॑ण॒ उत् । उज्ज॑भार । ज॒भा॒र॒ मद्ध्या᳚त् । मद्ध्या᳚न्नी॒चा । नी॒चादु॒च्चा । उ॒च्चा स्व॒धया᳚ । स्व॒धया॒ ऽभि । स्व॒धयेति॑ स्व - धया᳚ । अ॒भि प्र । प्र त॑स्थौ । त॒स्था॒विति॑ तस्थौ ॥ म॒हान् म॒ही । म॒ही अ॑स्तभायत् । म॒ही इति॑ म॒ही । अ॒स्त॒भा॒य॒द् वि । वि जा॒तः । जा॒तो द्याम् । द्याꣳ सद्म॑ । सद्म॒ पार्त्थि॑वम् । पार्त्थि॑वम् च । च॒ रजः॑ । रज॒ इति॒ रजः॑ ॥ स बु॒द्ध्नात् । बु॒द्ध्नादा᳚ष्ट । आ॒ष्ट॒ ज॒नुषा᳚ । ज॒नुषा॒ ऽभि । अ॒भ्यग्र᳚म् । अग्र॒म् बृह॒स्पतिः॑ । बृह॒स्पति॑र् दे॒वता᳚ । दे॒वता॒ यस्य॑ । यस्य॑ स॒म्राट् । स॒म्राडिति॑ सं - राट् ॥ बु॒द्ध्नाद् यः । यो अग्र᳚म् । अग्र॑म॒भ्यर्ति॑ । अ॒भ्यर्त्योज॑सा । अ॒भ्यर्तीत्य॑भि - अर्ति॑ । ओज॑सा॒ बृह॒स्पति᳚म् । बृह॒स्पति॒ मा । आ वि॑वासन्ति । वि॒वा॒स॒न्ति॒ दे॒वाः ( ) । दे॒वा इति॑ दे॒वाः ॥ भि॒नद् व॒लम् । व॒लं ॅवि । वि पुरः॑ । पुरो॑ दर्दरीति । द॒र्द॒री॒ति॒ कनि॑क्रदत् । कनि॑क्रद॒थ् सुवः॑ । सुव॑र॒पः । अ॒पो जि॑गाय । जि॒गा॒येति॑ जिगाय । \newline

\textbf{Jatai Paata} \newline

1. प्र यो यः प्र प्र यः । \newline
2. यो ज॒ज्ञे ज॒ज्ञे यो यो ज॒ज्ञे । \newline
3. ज॒ज्ञे वि॒द्वान्. वि॒द्वान् ज॒ज्ञे ज॒ज्ञे वि॒द्वान् । \newline
4. वि॒द्वाꣳ अ॒स्यास्य वि॒द्वान्. वि॒द्वाꣳ अ॒स्य । \newline
5. अ॒स्य बन्धु॒म् बन्धु॑ म॒स्यास्य बन्धु᳚म् । \newline
6. बन्धुं॒ ॅविश्वा॑नि॒ विश्वा॑नि॒ बन्धु॒म् बन्धुं॒ ॅविश्वा॑नि । \newline
7. विश्वा॑नि दे॒वो दे॒वो विश्वा॑नि॒ विश्वा॑नि दे॒वः । \newline
8. दे॒वो जनि॑मा॒ जनि॑मा दे॒वो दे॒वो जनि॑मा । \newline
9. जनि॑मा विवक्ति विवक्ति॒ जनि॑मा॒ जनि॑मा विवक्ति । \newline
10. वि॒व॒क्तीति॑ विवक्ति । \newline
11. ब्रह्म॒ ब्रह्म॑णो॒ ब्रह्म॑णो॒ ब्रह्म॒ ब्रह्म॒ ब्रह्म॑णः । \newline
12. ब्रह्म॑ण॒ उदुद् ब्रह्म॑णो॒ ब्रह्म॑ण॒ उत् । \newline
13. उज् ज॑भार जभा॒रोदुज् ज॑भार । \newline
14. ज॒भा॒र॒ मद्ध्या॒न् मद्ध्या᳚ज् जभार जभार॒ मद्ध्या᳚त् । \newline
15. मद्ध्या᳚न् नी॒चा नी॒चा मद्ध्या॒न् मद्ध्या᳚न् नी॒चा । \newline
16. नी॒चा दु॒च्चोच्चा नी॒चा नी॒चा दु॒च्चा । \newline
17. उ॒च्चा स्व॒धया᳚ स्व॒धयो॒च्चोच्चा स्व॒धया᳚ । \newline
18. स्व॒धया॒ ऽभ्य॑भि स्व॒धया᳚ स्व॒धया॒ ऽभि । \newline
19. स्व॒धयेति॑ स्व - धया᳚ । \newline
20. अ॒भि प्र प्राभ्य॑भि प्र । \newline
21. प्र त॑स्थौ तस्थौ॒ प्र प्र त॑स्थौ । \newline
22. त॒स्था॒विति॑ तस्थौ । \newline
23. म॒हान् म॒ही म॒ही म॒हान् म॒हान् म॒ही । \newline
24. म॒ही अ॑स्तभाय दस्तभायन् म॒ही म॒ही अ॑स्तभायत् । \newline
25. म॒ही इति॑ म॒ही । \newline
26. अ॒स्त॒भा॒य॒द् वि व्य॑स्तभाय दस्तभाय॒द् वि । \newline
27. वि जा॒तो जा॒तो वि वि जा॒तः । \newline
28. जा॒तो द्याम् द्याम् जा॒तो जा॒तो द्याम् । \newline
29. द्याꣳ सद्म॒ सद्म॒ द्याम् द्याꣳ सद्म॑ । \newline
30. सद्म॒ पार्त्थि॑व॒म् पार्त्थि॑वꣳ॒॒ सद्म॒ सद्म॒ पार्त्थि॑वम् । \newline
31. पार्त्थि॑वम् च च॒ पार्त्थि॑व॒म् पार्त्थि॑वम् च । \newline
32. च॒ रजो॒ रज॑श्च च॒ रजः॑ । \newline
33. रज॒ इति॒ रजः॑ । \newline
34. स बु॒द्ध्नाद् बु॒द्ध्नाथ् स स बु॒द्ध्नात् । \newline
35. बु॒द्ध्नादा᳚ष्टाष्ट बु॒द्ध्नाद् बु॒द्ध्नादा᳚ष्ट । \newline
36. आ॒ष्ट॒ ज॒नुषा॑ ज॒नुषा᳚ ऽऽष्टाष्ट ज॒नुषा᳚ । \newline
37. ज॒नुषा॒ ऽभ्य॑भि ज॒नुषा॑ ज॒नुषा॒ ऽभि । \newline
38. अ॒भ्यग्र॒ मग्र॑ म॒भ्य॑भ्यग्र᳚म् । \newline
39. अग्र॒म् बृह॒स्पति॒र् बृह॒स्पति॒ रग्र॒ मग्र॒म् बृह॒स्पतिः॑ । \newline
40. बृह॒स्पति॑र् दे॒वता॑ दे॒वता॒ बृह॒स्पति॒र् बृह॒स्पति॑र् दे॒वता᳚ । \newline
41. दे॒वता॒ यस्य॒ यस्य॑ दे॒वता॑ दे॒वता॒ यस्य॑ । \newline
42. यस्य॑ स॒म्राट् थ्स॒म्राड् यस्य॒ यस्य॑ स॒म्राट् । \newline
43. स॒म्राडिति॑ सं - राट् । \newline
44. बु॒द्ध्नाद् यो यो बु॒द्ध्नाद् बु॒द्ध्नाद् यः । \newline
45. यो अग्र॒ मग्रं॒ ॅयो यो अग्र᳚म् । \newline
46. अग्र॑ म॒भ्यर्त्य॒भ्यर्त्यग्र॒ मग्र॑ म॒भ्यर्ति॑ । \newline
47. अ॒भ्यर् त्योज॒ सौज॑सा॒ ऽभ्यर्त्य॒भ्यर्त्योज॑सा । \newline
48. अ॒भ्यर्तीत्य॑भि - अर्ति॑ । \newline
49. ओज॑सा॒ बृह॒स्पति॒म् बृह॒स्पति॒ मोज॒सौज॑सा॒ बृह॒स्पति᳚म् । \newline
50. बृह॒स्पति॒ मा बृह॒स्पति॒म् बृह॒स्पति॒ मा । \newline
51. आ वि॑वासन्ति विवास॒न्त्या वि॑वासन्ति । \newline
52. वि॒वा॒स॒न्ति॒ दे॒वा दे॒वा वि॑वासन्ति विवासन्ति दे॒वाः । \newline
53. दे॒वा इति॑ दे॒वाः । \newline
54. भि॒नद् व॒लं ॅव॒लम् भि॒नद् भि॒नद् व॒लम् । \newline
55. व॒लं ॅवि वि व॒लं ॅव॒लं ॅवि । \newline
56. वि पुरः॒ पुरो॒ वि वि पुरः॑ । \newline
57. पुरो॑ दर्दरीति दर्दरीति॒ पुरः॒ पुरो॑ दर्दरीति । \newline
58. द॒र्द॒री॒ति॒ कनि॑क्रद॒त् कनि॑क्रदद् दर्दरीति दर्दरीति॒ कनि॑क्रदत् । \newline
59. कनि॑क्रद॒थ् सुवः॒ सुवः॒ कनि॑क्रद॒त् कनि॑क्रद॒थ् सुवः॑ । \newline
60. सुव॑ र॒पो अ॒पः सुवः॒ सुव॑ र॒पः । \newline
61. अ॒पो जि॑गाय जिगाया॒पो अ॒पो जि॑गाय । \newline
62. जि॒गा॒येति॑ जिगाय । \newline

\textbf{Ghana Paata } \newline

1. प्र यो यः प्र प्र यो ज॒ज्ञे ज॒ज्ञे यः प्र प्र यो ज॒ज्ञे । \newline
2. यो ज॒ज्ञे ज॒ज्ञे यो यो ज॒ज्ञे वि॒द्वान्. वि॒द्वान् ज॒ज्ञे यो यो ज॒ज्ञे वि॒द्वान् । \newline
3. ज॒ज्ञे वि॒द्वान्. वि॒द्वान् ज॒ज्ञे ज॒ज्ञे वि॒द्वाꣳ अ॒स्यास्य वि॒द्वान् ज॒ज्ञे ज॒ज्ञे वि॒द्वाꣳ अ॒स्य । \newline
4. वि॒द्वाꣳ अ॒स्यास्य वि॒द्वान्. वि॒द्वाꣳ अ॒स्य बन्धु॒म् बन्धु॑ म॒स्य वि॒द्वान्. वि॒द्वाꣳ अ॒स्य बन्धु᳚म् । \newline
5. अ॒स्य बन्धु॒म् बन्धु॑ म॒स्यास्य बन्धुं॒ ॅविश्वा॑नि॒ विश्वा॑नि॒ बन्धु॑ म॒स्यास्य बन्धुं॒ ॅविश्वा॑नि । \newline
6. बन्धुं॒ ॅविश्वा॑नि॒ विश्वा॑नि॒ बन्धु॒म् बन्धुं॒ ॅविश्वा॑नि दे॒वो दे॒वो विश्वा॑नि॒ बन्धु॒म् बन्धुं॒ ॅविश्वा॑नि दे॒वः । \newline
7. विश्वा॑नि दे॒वो दे॒वो विश्वा॑नि॒ विश्वा॑नि दे॒वो जनि॑मा॒ जनि॑मा दे॒वो विश्वा॑नि॒ विश्वा॑नि दे॒वो जनि॑मा । \newline
8. दे॒वो जनि॑मा॒ जनि॑मा दे॒वो दे॒वो जनि॑मा विवक्ति विवक्ति॒ जनि॑मा दे॒वो दे॒वो जनि॑मा विवक्ति । \newline
9. जनि॑मा विवक्ति विवक्ति॒ जनि॑मा॒ जनि॑मा विवक्ति । \newline
10. वि॒व॒क्तीति॑ विवक्ति । \newline
11. ब्रह्म॒ ब्रह्म॑णो॒ ब्रह्म॑णो॒ ब्रह्म॒ ब्रह्म॒ ब्रह्म॑ण॒ उदुद् ब्रह्म॑णो॒ ब्रह्म॒ ब्रह्म॒ ब्रह्म॑ण॒ उत् । \newline
12. ब्रह्म॑ण॒ उदुद् ब्रह्म॑णो॒ ब्रह्म॑ण॒ उज् ज॑भार जभा॒रोद् ब्रह्म॑णो॒ ब्रह्म॑ण॒ उज् ज॑भार । \newline
13. उज् ज॑भार जभा॒रोदुज् ज॑भार॒ मद्ध्या॒न् मद्ध्या᳚ज् जभा॒रोदुज् ज॑भार॒ मद्ध्या᳚त् । \newline
14. ज॒भा॒र॒ मद्ध्या॒न् मद्ध्या᳚ज् जभार जभार॒ मद्ध्या᳚न् नी॒चा नी॒चा मद्ध्या᳚ज् जभार जभार॒ मद्ध्या᳚न् नी॒चा । \newline
15. मद्ध्या᳚न् नी॒चा नी॒चा मद्ध्या॒न् मद्ध्या᳚न् नी॒चा दु॒च्चोच्चा नी॒चा मद्ध्या॒न् मद्ध्या᳚न् नी॒चा दु॒च्चा । \newline
16. नी॒चा दु॒च्चोच्चा नी॒चा नी॒चा दु॒च्चा स्व॒धया᳚ स्व॒धयो॒च्चा नी॒चा नी॒चा दु॒च्चा स्व॒धया᳚ । \newline
17. उ॒च्चा स्व॒धया᳚ स्व॒धयो॒च्चोच्चा स्व॒धया॒ ऽभ्य॑भि स्व॒धयो॒च्चोच्चा स्व॒धया॒ ऽभि । \newline
18. स्व॒धया॒ ऽभ्य॑भि स्व॒धया᳚ स्व॒धया॒ ऽभि प्र प्राभि स्व॒धया᳚ स्व॒धया॒ ऽभि प्र । \newline
19. स्व॒धयेति॑ स्व - धया᳚ । \newline
20. अ॒भि प्र प्राभ्य॑भि प्र त॑स्थौ तस्थौ॒ प्राभ्य॑भि प्र त॑स्थौ । \newline
21. प्र त॑स्थौ तस्थौ॒ प्र प्र त॑स्थौ । \newline
22. त॒स्था॒विति॑ तस्थौ । \newline
23. म॒हान् म॒ही म॒ही म॒हान् म॒हान् म॒ही अ॑स्तभाय दस्तभायन् म॒ही म॒हान् म॒हान् म॒ही अ॑स्तभायत् । \newline
24. म॒ही अ॑स्तभाय दस्तभायन् म॒ही म॒ही अ॑स्तभाय॒द् वि व्य॑स्तभायन् म॒ही म॒ही अ॑स्तभाय॒द् वि । \newline
25. म॒ही इति॑ म॒ही । \newline
26. अ॒स्त॒भा॒य॒द् वि व्य॑स्तभाय दस्तभाय॒द् वि जा॒तो जा॒तो व्य॑स्तभाय दस्तभाय॒द् वि जा॒तः । \newline
27. वि जा॒तो जा॒तो वि वि जा॒तो द्याम् द्याम् जा॒तो वि वि जा॒तो द्याम् । \newline
28. जा॒तो द्याम् द्याम् जा॒तो जा॒तो द्याꣳ सद्म॒ सद्म॒ द्याम् जा॒तो जा॒तो द्याꣳ सद्म॑ । \newline
29. द्याꣳ सद्म॒ सद्म॒ द्याम् द्याꣳ सद्म॒ पार्त्थि॑व॒म् पार्त्थि॑वꣳ॒॒ सद्म॒ द्याम् द्याꣳ सद्म॒ पार्त्थि॑वम् । \newline
30. सद्म॒ पार्त्थि॑व॒म् पार्त्थि॑वꣳ॒॒ सद्म॒ सद्म॒ पार्त्थि॑वम् च च॒ पार्त्थि॑वꣳ॒॒ सद्म॒ सद्म॒ पार्त्थि॑वम् च । \newline
31. पार्त्थि॑वम् च च॒ पार्त्थि॑व॒म् पार्त्थि॑वम् च॒ रजो॒ रज॑श्च॒ पार्त्थि॑व॒म् पार्त्थि॑वम् च॒ रजः॑ । \newline
32. च॒ रजो॒ रज॑श्च च॒ रजः॑ । \newline
33. रज॒ इति॒ रजः॑ । \newline
34. स बु॒द्ध्नाद् बु॒द्ध्नाथ् स स बु॒द्ध्ना दा᳚ष्टाष्ट बु॒द्ध्नाथ् स स बु॒द्ध्नादा᳚ष्ट । \newline
35. बु॒द्ध्ना दा᳚ष्टाष्ट बु॒द्ध्नाद् बु॒द्ध्नादा᳚ष्ट ज॒नुषा॑ ज॒नुषा᳚ ऽऽष्ट बु॒द्ध्नाद् बु॒द्ध्नादा᳚ष्ट ज॒नुषा᳚ । \newline
36. आ॒ष्ट॒ ज॒नुषा॑ ज॒नुषा᳚ ऽऽष्टाष्ट ज॒नुषा॒ ऽभ्य॑भि ज॒नुषा᳚ ऽऽष्टाष्ट ज॒नुषा॒ ऽभि । \newline
37. ज॒नुषा॒ ऽभ्य॑भि ज॒नुषा॑ ज॒नुषा॒ ऽभ्यग्र॒ मग्र॑ म॒भि ज॒नुषा॑ ज॒नुषा॒ ऽभ्यग्र᳚म् । \newline
38. अ॒भ्यग्र॒ मग्र॑ म॒भ्य॑भ्यग्र॒म् बृह॒स्पति॒र् बृह॒स्पति॒ रग्र॑ म॒भ्य॑भ्यग्र॒म् बृह॒स्पतिः॑ । \newline
39. अग्र॒म् बृह॒स्पति॒र् बृह॒स्पति॒ रग्र॒ मग्र॒म् बृह॒स्पति॑र् दे॒वता॑ दे॒वता॒ बृह॒स्पति॒ रग्र॒ मग्र॒म् बृह॒स्पति॑र् दे॒वता᳚ । \newline
40. बृह॒स्पति॑र् दे॒वता॑ दे॒वता॒ बृह॒स्पति॒र् बृह॒स्पति॑र् दे॒वता॒ यस्य॒ यस्य॑ दे॒वता॒ बृह॒स्पति॒र् बृह॒स्पति॑र् दे॒वता॒ यस्य॑ । \newline
41. दे॒वता॒ यस्य॒ यस्य॑ दे॒वता॑ दे॒वता॒ यस्य॑ स॒म्राट् थ्स॒म्राड् यस्य॑ दे॒वता॑ दे॒वता॒ यस्य॑ स॒म्राट् । \newline
42. यस्य॑ स॒म्राट् थ्स॒म्राड् यस्य॒ यस्य॑ स॒म्राट् । \newline
43. स॒म्राडिति॑ सं - राट् । \newline
44. बु॒द्ध्नाद् यो यो बु॒द्ध्नाद् बु॒द्ध्नाद् यो अग्र॒ मग्रं॒ ॅयो बु॒द्ध्नाद् बु॒द्ध्नाद् यो अग्र᳚म् । \newline
45. यो अग्र॒ मग्रं॒ ॅयो यो अग्र॑ म॒भ्यर् त्य॒भ्यर् त्यग्रं॒ ॅयो यो अग्र॑ म॒भ्यर्ति॑ । \newline
46. अग्र॑ म॒भ्यर् त्य॒भ्यर् त्यग्र॒ मग्र॑ म॒भ्यर् त्योज॒सौज॑सा॒ ऽभ्यर्त्यग्र॒ मग्र॑ म॒भ्यर्त्योज॑सा । \newline
47. अ॒भ्यर् त्योज॒सौज॑सा॒ ऽभ्यर्त्य॒भ्यर् त्योज॑सा॒ बृह॒स्पति॒म् बृह॒स्पति॒ मोज॑सा॒ ऽभ्यर्त्य॒भ्यर्त्योज॑सा॒ बृह॒स्पति᳚म् । \newline
48. अ॒भ्यर्तीत्य॑भि - अर्ति॑ । \newline
49. ओज॑सा॒ बृह॒स्पति॒म् बृह॒स्पति॒ मोज॒सौज॑सा॒ बृह॒स्पति॒ मा बृह॒स्पति॒ मोज॒सौज॑सा॒ बृह॒स्पति॒ मा । \newline
50. बृह॒स्पति॒ मा बृह॒स्पति॒म् बृह॒स्पति॒ मा वि॑वासन्ति विवास॒न्त्या बृह॒स्पति॒म् बृह॒स्पति॒ मा वि॑वासन्ति । \newline
51. आ वि॑वासन्ति विवास॒न्त्या वि॑वासन्ति दे॒वा दे॒वा वि॑वास॒न्त्या वि॑वासन्ति दे॒वाः । \newline
52. वि॒वा॒स॒न्ति॒ दे॒वा दे॒वा वि॑वासन्ति विवासन्ति दे॒वाः । \newline
53. दे॒वा इति॑ दे॒वाः । \newline
54. भि॒नद् व॒लं ॅव॒लम् भि॒नद् भि॒नद् व॒लं ॅवि वि व॒लम् भि॒नद् भि॒नद् व॒लं ॅवि । \newline
55. व॒लं ॅवि वि व॒लं ॅव॒लं ॅवि पुरः॒ पुरो॒ वि व॒लं ॅव॒लं ॅवि पुरः॑ । \newline
56. वि पुरः॒ पुरो॒ वि वि पुरो॑ दर्दरीति दर्दरीति॒ पुरो॒ वि वि पुरो॑ दर्दरीति । \newline
57. पुरो॑ दर्दरीति दर्दरीति॒ पुरः॒ पुरो॑ दर्दरीति॒ कनि॑क्रद॒त् कनि॑क्रदद् दर्दरीति॒ पुरः॒ पुरो॑ दर्दरीति॒ कनि॑क्रदत् । \newline
58. द॒र्द॒री॒ति॒ कनि॑क्रद॒त् कनि॑क्रदद् दर्दरीति दर्दरीति॒ कनि॑क्रद॒थ् सुवः॒ सुवः॒ कनि॑क्रदद् दर्दरीति दर्दरीति॒ कनि॑क्रद॒थ् सुवः॑ । \newline
59. कनि॑क्रद॒थ् सुवः॒ सुवः॒ कनि॑क्रद॒त् कनि॑क्रद॒थ् सुव॑ र॒पो अ॒पः सुवः॒ कनि॑क्रद॒त् कनि॑क्रद॒थ् सुव॑ र॒पः । \newline
60. सुव॑ र॒पो अ॒पः सुवः॒ सुव॑ र॒पो जि॑गाय जिगाया॒पः सुवः॒ सुव॑ र॒पो जि॑गाय । \newline
61. अ॒पो जि॑गाय जिगाया॒पो अ॒पो जि॑गाय । \newline
62. जि॒गा॒येति॑ जिगाय । \newline
\pagebreak


\end{document}