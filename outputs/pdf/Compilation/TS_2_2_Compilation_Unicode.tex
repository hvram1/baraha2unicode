\documentclass[17pt]{extarticle}
\usepackage{babel}
\usepackage{fontspec}
\usepackage{polyglossia}
\usepackage{extsizes}



\setmainlanguage{sanskrit}
\setotherlanguages{english} %% or other languages
\setlength{\parindent}{0pt}
\pagestyle{myheadings}
\newfontfamily\devanagarifont[Script=Devanagari]{AdishilaVedic}


\newcommand{\VAR}[1]{}
\newcommand{\BLOCK}[1]{}




\begin{document}
\begin{titlepage}
    \begin{center}
 
\begin{sanskrit}
    { \Huge
    कृष्ण यजुर्वेदीय तैत्तिरीय संहिता,पद,जटा,घन पाठः 
    }
    \\
    \vspace{2.5cm}
    \mbox{ \Huge
    2.2     द्वितीयकाण्डे द्वितीयः प्रश्नः - इष्टिविधानं   }
\end{sanskrit}
\end{center}

\end{titlepage}
\tableofcontents
\pagebreak

\markright{ TS 2.2.1.1  \hfill https://www.vedavms.in \hfill}
\addcontentsline{toc}{section}{ TS 2.2.1.1 }
\section*{ TS 2.2.1.1 }

\textbf{TS 2.2.1.1 } \newline
\textbf{Samhita Paata} \newline

प्र॒जापतिः॑ प्र॒जा अ॑सृजत॒ ताः सृ॒ष्टा॒ इन्द्रा॒ग्नी अपा॑गूहताꣳ॒॒ सो॑ऽचायत् प्र॒जाप॑तिरिन्द्रा॒ग्नी वै मे᳚ प्र॒जा अपा॑घुक्षता॒मिति॒ स ए॒तमै᳚न्द्रा॒ग्न- मेका॑दशकपाल-मपश्य॒त् तं निर॑वप॒त् ताव॑स्मै प्र॒जाः प्रासा॑धयता- मिन्द्रा॒ग्नी वा ए॒तस्य॑ प्र॒जामप॑ गूहतो॒ योऽलं॑ प्र॒जायै॒ सन् प्र॒जां न वि॒न्दत॑ ऐन्द्रा॒ग्न-मेका॑दशकपालं॒ निर्व॑पेत्-प्र॒जाका॑म इन्द्रा॒ग्नी-  [  ] \newline

\textbf{Pada Paata} \newline

प्र॒जाप॑ति॒रिति॑ प्र॒जा-प॒तिः॒ । प्र॒जा इति॑ प्र - जाः । अ॒सृ॒ज॒त॒ । ताः । सृ॒ष्टाः । इ॒न्द्रा॒ग्नी इती᳚न्द्र - अ॒ग्नी । अपेति॑ । अ॒गू॒ह॒ता॒म् । सः । अ॒चा॒य॒त् । प्र॒जाप॑ति॒रिति॑ प्र॒जा - प॒तिः॒ । इ॒न्द्रा॒ग्नी इती᳚न्द्र-अ॒ग्नी । वै । मे॒ । प्र॒जा इति॑ प्र - जाः । अपेति॑ । अ॒घु॒क्ष॒ता॒म् । इति॑ । सः । ए॒तम् । ऐ॒न्द्रा॒ग्नमित्यै᳚न्द्र - अ॒ग्नम् । एका॑दशकपाल॒मित्येका॑दश - क॒पा॒ल॒म् । अ॒प॒श्य॒त् । तम् । निरिति॑ । अ॒व॒प॒त् । तौ । अ॒स्मै॒ । प्र॒जा इति॑ प्र-जाः । प्रेति॑ । अ॒सा॒ध॒य॒ता॒म् । इ॒न्द्रा॒ग्नी इती᳚न्द्र - अ॒ग्नी । वै । ए॒तस्य॑ । प्र॒जामिति॑ प्र - जाम् । अपेति॑ । गू॒ह॒तः॒ । यः । अल᳚म् । प्र॒जाया॒ इति॑ प्र - जायै᳚ । सन्न् । प्र॒जामिति॑ प्र - जाम् । न । वि॒न्दते᳚ । ऐ॒न्द्रा॒ग्नमित्यै᳚न्द्र - अ॒ग्नम् । एका॑दशकपाल॒मित्येका॑दश -क॒पा॒ल॒म् । निरिति॑ । व॒पे॒त् । प्र॒जाका॑म॒ इति॑ प्र॒जा - का॒मः॒ । इ॒न्द्रा॒ग्नी इती᳚न्द्र - अ॒ग्नी ।  \newline


\textbf{Krama Paata} \newline

प्र॒जाप॑तिः प्र॒जाः । प्र॒जाप॑ति॒रिति॑ प्र॒जा - प॒तिः॒ । प्र॒जा अ॑सृजत । प्र॒जा इति॑ प्र - जाः । अ॒सृ॒ज॒त॒ ताः । ताः सृ॒ष्टाः । सृ॒ष्टा इ॑न्द्रा॒ग्नी । इ॒न्द्रा॒ग्नी अप॑ । इ॒न्द्रा॒ग्नी इती᳚न्द्र - अ॒ग्नी । अपा॑गूहताम् । अ॒गू॒ह॒ताꣳ॒॒ सः । सो॑ ऽचायत् । अ॒चा॒य॒त्,प्र॒जाप॑तिः । प्र॒जाप॑तिरिन्द्रा॒ग्नी । प्र॒जाप॑ति॒रिति॑ प्र॒जा - प॒तिः॒ । इ॒न्द्रा॒ग्नी वै । इ॒न्द्रा॒ग्नी इती᳚न्द्र - अ॒ग्नी । वै मे᳚ । मे॒ प्र॒जाः । प्र॒जा अप॑ । प्र॒जा इति॑ प्र - जाः । अपा॑घुक्षताम् । अ॒घु॒क्ष॒ता॒मिति॑ । इति॒ सः । स ए॒तम् । ए॒तमै᳚न्द्रा॒ग्नम् । ऐ॒न्द्रा॒ग्न,मेका॑दशकपालम् । ऐ॒न्द्रा॒ग्नमित्यै᳚न्द्र - अ॒ग्नम् । एका॑दशकपालमपश्यत् । एका॑दशकपाल॒मित्येका॑दश - क॒पा॒ल॒म् । अ॒प॒श्य॒त् तम् । तम् निः । निर॑वपत् । अ॒व॒प॒त् तौ । ताव॑स्मै । अ॒स्मै॒ प्र॒जाः । प्र॒जाः प्र । प्र॒जा इति॑ प्र - जाः । प्रासा॑धयताम् । अ॒सा॒ध॒य॒ता॒,मि॒न्द्रा॒ग्नी । इ॒न्द्रा॒ग्नी वै । इ॒न्द्रा॒ग्नी इती᳚न्द्र - अ॒ग्नी । वा ए॒तस्य॑ । ए॒तस्य॑ प्र॒जाम् । प्र॒जामप॑ । प्र॒जामिति॑ प्र - जाम् । अप॑ गूहतः । गू॒ह॒तो॒ यः । यो ऽल᳚म् । अल॑म् प्र॒जायै᳚ । प्र॒जायै॒ सन्न् । प्र॒जाया॒ इति॑ प्र - जायै᳚ । सन्,प्र॒जाम् । प्र॒जाम् न । प्र॒जामिति॑ प्र - जाम् । न वि॒न्दते᳚ । वि॒न्दत॑ ऐन्द्रा॒ग्नम् । ऐ॒न्द्रा॒ग्न,मेका॑दशकपालम् । ऐ॒न्द्रा॒ग्नमित्यै᳚न्द्र - अ॒ग्नम् । एका॑दशकपाल॒म् निः । एका॑दशकपाल॒मित्येका॑दश - क॒पा॒ल॒म् । निर् व॑पेत् । व॒पे॒त्,प्र॒जाका॑मः । प्र॒जाका॑म इन्द्रा॒ग्नी । प्र॒जाका॑म॒ इति॑ प्र॒जा - का॒मः॒ । इ॒न्द्रा॒ग्नी ए॒व । इ॒न्द्रा॒ग्नी इती᳚न्द्र - अ॒ग्नी \newline

\textbf{Jatai Paata} \newline

1. प्र॒जाप॑तिः प्र॒जाः प्र॒जाः प्र॒जाप॑तिः प्र॒जाप॑तिः प्र॒जाः । \newline
2. प्र॒जाप॑ति॒रिति॑ प्र॒जा - प॒तिः॒ । \newline
3. प्र॒जा अ॑सृजता सृजत प्र॒जाः प्र॒जा अ॑सृजत । \newline
4. प्र॒जा इति॑ प्र - जाः । \newline
5. अ॒सृ॒ज॒त॒ तास्ता अ॑सृजता सृजत॒ ताः । \newline
6. ताः सृ॒ष्टाः सृ॒ष्टा स्ता स्ताः सृ॒ष्टाः । \newline
7. सृ॒ष्टा इ॑न्द्रा॒ग्नी इ॑न्द्रा॒ग्नी सृ॒ष्टाः सृ॒ष्टा इ॑न्द्रा॒ग्नी । \newline
8. इ॒न्द्रा॒ग्नी अपापे᳚ न्द्रा॒ग्नी इ॑न्द्रा॒ग्नी अप॑ । \newline
9. इ॒न्द्रा॒ग्नी इती᳚न्द्र - अ॒ग्नी । \newline
10. अपा॑गूहता मगूहता॒ मपापा॑गूहताम् । \newline
11. अ॒गू॒ह॒ताꣳ॒॒ स सो॑ ऽगूहता मगूहताꣳ॒॒ सः । \newline
12. सो॑ ऽचाय दचाय॒थ् स सो॑ ऽचायत् । \newline
13. अ॒चा॒य॒त् प्र॒जाप॑तिः प्र॒जाप॑ति रचाय दचायत् प्र॒जाप॑तिः । \newline
14. प्र॒जाप॑ति रिन्द्रा॒ग्नी इ॑न्द्रा॒ग्नी प्र॒जाप॑तिः प्र॒जाप॑ति रिन्द्रा॒ग्नी । \newline
15. प्र॒जाप॑ति॒रिति॑ प्र॒जा - प॒तिः॒ । \newline
16. इ॒न्द्रा॒ग्नी वै वा इ॑न्द्रा॒ग्नी इ॑न्द्रा॒ग्नी वै । \newline
17. इ॒न्द्रा॒ग्नी इती᳚न्द्र - अ॒ग्नी । \newline
18. वै मे॑ मे॒ वै वै मे᳚ । \newline
19. मे॒ प्र॒जाः प्र॒जा मे॑ मे प्र॒जाः । \newline
20. प्र॒जा अपाप॑ प्र॒जाः प्र॒जा अप॑ । \newline
21. प्र॒जा इति॑ प्र - जाः । \newline
22. अपा॑घुक्षता मघुक्षता॒ मपापा॑ घुक्षताम् । \newline
23. अ॒घु॒क्ष॒ता॒ मिती त्य॑घुक्षता मघुक्षता॒ मिति॑ । \newline
24. इति॒ स स इतीति॒ सः । \newline
25. स ए॒त मे॒तꣳ स स ए॒तम् । \newline
26. ए॒त मै᳚न्द्रा॒ग्न मै᳚न्द्रा॒ग्न मे॒त मे॒त मै᳚न्द्रा॒ग्नम् । \newline
27. ऐ॒न्द्रा॒ग्न मेका॑दशकपाल॒ मेका॑दशकपाल मैन्द्रा॒ग्न मै᳚न्द्रा॒ग्न मेका॑दशकपालम् । \newline
28. ऐ॒न्द्रा॒ग्नमित्यै᳚न्द्र - अ॒ग्नम् । \newline
29. एका॑दशकपाल मपश्य दपश्य॒ देका॑दशकपाल॒ मेका॑दशकपाल मपश्यत् । \newline
30. एका॑दशकपाल॒मित्येका॑दश - क॒पा॒ल॒म् । \newline
31. अ॒प॒श्य॒त् तम् त म॑पश्य दपश्य॒त् तम् । \newline
32. तम् निर् णिष् टम् तम् निः । \newline
33. निर॑वप दवप॒न् निर् णिर॑वपत् । \newline
34. अ॒व॒प॒त् तौ ता व॑वप दवप॒त् तौ । \newline
35. ता व॑स्मा अस्मै॒ तौ ता व॑स्मै । \newline
36. अ॒स्मै॒ प्र॒जाः प्र॒जा अ॑स्मा अस्मै प्र॒जाः । \newline
37. प्र॒जाः प्र प्र प्र॒जाः प्र॒जाः प्र । \newline
38. प्र॒जा इति॑ प्र - जाः । \newline
39. प्रासा॑धयता मसाधयता॒म् प्र प्रासा॑धयताम् । \newline
40. अ॒सा॒ध॒य॒ता॒ मि॒न्द्रा॒ग्नी इ॑न्द्रा॒ग्नी अ॑साधयता मसाधयता मिन्द्रा॒ग्नी । \newline
41. इ॒न्द्रा॒ग्नी वै वा इ॑न्द्रा॒ग्नी इ॑न्द्रा॒ग्नी वै । \newline
42. इ॒न्द्रा॒ग्नी इती᳚न्द्र - अ॒ग्नी । \newline
43. वा ए॒तस्यै॒तस्य॒ वै वा ए॒तस्य॑ । \newline
44. ए॒तस्य॑ प्र॒जाम् प्र॒जा मे॒तस्यै॒तस्य॑ प्र॒जाम् । \newline
45. प्र॒जा मपाप॑ प्र॒जाम् प्र॒जा मप॑ । \newline
46. प्र॒जामिति॑ प्र - जाम् । \newline
47. अप॑ गूहतो गूह॒तो ऽपाप॑ गूहतः । \newline
48. गू॒ह॒तो॒ यो यो गू॑हतो गूहतो॒ यः । \newline
49. यो ऽल॒ मलं॒ ॅयो यो ऽल᳚म् । \newline
50. अल॑म् प्र॒जायै᳚ प्र॒जाया॒ अल॒ मल॑म् प्र॒जायै᳚ । \newline
51. प्र॒जायै॒ सन् थ्सन् प्र॒जायै᳚ प्र॒जायै॒ सन्न् । \newline
52. प्र॒जाया॒ इति॑ प्र - जायै᳚ । \newline
53. सन् प्र॒जाम् प्र॒जाꣳ सन् थ्सन् प्र॒जाम् । \newline
54. प्र॒जाम् न न प्र॒जाम् प्र॒जाम् न । \newline
55. प्र॒जामिति॑ प्र - जाम् । \newline
56. न वि॒न्दते॑ वि॒न्दते॒ न न वि॒न्दते᳚ । \newline
57. वि॒न्दत॑ ऐन्द्रा॒ग्न मै᳚न्द्रा॒ग्नं ॅवि॒न्दते॑ वि॒न्दत॑ ऐन्द्रा॒ग्नम् । \newline
58. ऐ॒न्द्रा॒ग्न मेका॑दशकपाल॒ मेका॑दशकपाल मैन्द्रा॒ग्न मै᳚न्द्रा॒ग्न मेका॑दशकपालम् । \newline
59. ऐ॒न्द्रा॒ग्नमित्यै᳚न्द्र - अ॒ग्नम् । \newline
60. एका॑दशकपाल॒म् निर् णिरेका॑दशकपाल॒ मेका॑दशकपाल॒म् निः । \newline
61. एका॑दशकपाल॒मित्येका॑दश - क॒पा॒ल॒म् । \newline
62. निर् व॑पेद् वपे॒न् निर् णिर् व॑पेत् । \newline
63. व॒पे॒त् प्र॒जाका॑मः प्र॒जाका॑मो वपेद् वपेत् प्र॒जाका॑मः । \newline
64. प्र॒जाका॑म इन्द्रा॒ग्नी इ॑न्द्रा॒ग्नी प्र॒जाका॑मः प्र॒जाका॑म इन्द्रा॒ग्नी । \newline
65. प्र॒जाका॑म॒ इति॑ प्र॒जा - का॒मः॒ । \newline
66. इ॒न्द्रा॒ग्नी ए॒वैवे न्द्रा॒ग्नी इ॑न्द्रा॒ग्नी ए॒व । \newline
67. इ॒न्द्रा॒ग्नी इती᳚न्द्र - अ॒ग्नी । \newline

\textbf{Ghana Paata } \newline

1. प्र॒जाप॑तिः प्र॒जाः प्र॒जाः प्र॒जाप॑तिः प्र॒जाप॑तिः प्र॒जा अ॑सृजता सृजत प्र॒जाः प्र॒जाप॑तिः प्र॒जाप॑तिः प्र॒जा अ॑सृजत । \newline
2. प्र॒जाप॑ति॒रिति॑ प्र॒जा - प॒तिः॒ । \newline
3. प्र॒जा अ॑सृजता सृजत प्र॒जाः प्र॒जा अ॑सृजत॒ ता स्ता अ॑सृजत प्र॒जाः प्र॒जा अ॑सृजत॒ ताः । \newline
4. प्र॒जा इति॑ प्र - जाः । \newline
5. अ॒सृ॒ज॒त॒ ता स्ता अ॑सृजता सृजत॒ ताः सृ॒ष्टाः सृ॒ष्टा स्ता अ॑सृजता सृजत॒ ताः सृ॒ष्टाः । \newline
6. ताः सृ॒ष्टाः सृ॒ ष्टा स्तास्ताः सृ॒ष्टा इ॑न्द्रा॒ग्नी इ॑न्द्रा॒ग्नी सृ॒ष्टा स्ता स्ताः सृ॒ष्टा इ॑न्द्रा॒ग्नी । \newline
7. सृ॒ष्टा इ॑न्द्रा॒ग्नी इ॑न्द्रा॒ग्नी सृ॒ष्टाः सृ॒ष्टा इ॑न्द्रा॒ग्नी अपापे᳚ न्द्रा॒ग्नी सृ॒ष्टाः सृ॒ष्टा इ॑न्द्रा॒ग्नी अप॑ । \newline
8. इ॒न्द्रा॒ग्नी अपापे᳚ न्द्रा॒ग्नी इ॑न्द्रा॒ग्नी अपा॑गूहता मगूहता॒ मपे᳚ न्द्रा॒ग्नी इ॑न्द्रा॒ग्नी अपा॑गूहताम् । \newline
9. इ॒न्द्रा॒ग्नी इती᳚न्द्र - अ॒ग्नी । \newline
10. अपा॑गूहता मगूहता॒ मपापा॑ गूहताꣳ॒॒ स सो॑ ऽगूहता॒ मपापा॑ गूहताꣳ॒॒ सः । \newline
11. अ॒गू॒ह॒ताꣳ॒॒ स सो॑ ऽगूहता मगूहताꣳ॒॒ सो॑ ऽचाय दचाय॒थ् सो॑ ऽगूहता मगूहताꣳ॒॒ सो॑ ऽचायत् । \newline
12. सो॑ ऽचाय दचाय॒थ् स सो॑ ऽचायत् प्र॒जाप॑तिः प्र॒जाप॑ति रचाय॒थ् स सो॑ ऽचायत् प्र॒जाप॑तिः । \newline
13. अ॒चा॒य॒त् प्र॒जाप॑तिः प्र॒जाप॑ति रचाय दचायत् प्र॒जाप॑ति रिन्द्रा॒ग्नी इ॑न्द्रा॒ग्नी प्र॒जाप॑ति रचाय दचायत् प्र॒जाप॑ति रिन्द्रा॒ग्नी । \newline
14. प्र॒जाप॑ति रिन्द्रा॒ग्नी इ॑न्द्रा॒ग्नी प्र॒जाप॑तिः प्र॒जाप॑ति रिन्द्रा॒ग्नी वै वा इ॑न्द्रा॒ग्नी प्र॒जाप॑तिः प्र॒जाप॑ति रिन्द्रा॒ग्नी वै । \newline
15. प्र॒जाप॑ति॒रिति॑ प्र॒जा - प॒तिः॒ । \newline
16. इ॒न्द्रा॒ग्नी वै वा इ॑न्द्रा॒ग्नी इ॑न्द्रा॒ग्नी वै मे॑ मे॒ वा इ॑न्द्रा॒ग्नी इ॑न्द्रा॒ग्नी वै मे᳚ । \newline
17. इ॒न्द्रा॒ग्नी इती᳚न्द्र - अ॒ग्नी । \newline
18. वै मे॑ मे॒ वै वै मे᳚ प्र॒जाः प्र॒जा मे॒ वै वै मे᳚ प्र॒जाः । \newline
19. मे॒ प्र॒जाः प्र॒जा मे॑ मे प्र॒जा अपाप॑ प्र॒जा मे॑ मे प्र॒जा अप॑ । \newline
20. प्र॒जा अपाप॑ प्र॒जाः प्र॒जा अपा॑घुक्षता मघुक्षता॒ मप॑ प्र॒जाः प्र॒जा अपा॑घुक्षताम् । \newline
21. प्र॒जा इति॑ प्र - जाः । \newline
22. अपा॑घुक्षता मघुक्षता॒ मपापा॑ घुक्षता॒ मिती त्य॑घुक्षता॒ मपापा॑ घुक्षता॒ मिति॑ । \newline
23. अ॒घु॒क्ष॒ता॒ मितीत्य॑ घुक्षता मघुक्षता॒ मिति॒ स स इत्य॑घुक्षता मघुक्षता॒ मिति॒ सः । \newline
24. इति॒ स स इतीति॒ स ए॒त मे॒तꣳ स इतीति॒ स ए॒तम् । \newline
25. स ए॒त मे॒तꣳ स स ए॒त मै᳚न्द्रा॒ग्न मै᳚न्द्रा॒ग्न मे॒तꣳ स स ए॒त मै᳚न्द्रा॒ग्नम् । \newline
26. ए॒त मै᳚न्द्रा॒ग्न  मै᳚न्द्रा॒ग्न मे॒त मे॒त मै᳚न्द्रा॒ग्न मेका॑दशकपाल॒ मेका॑दशकपाल मैन्द्रा॒ग्न मे॒त मे॒त मै᳚न्द्रा॒ग्न मेका॑दशकपालम् । \newline
27. ऐ॒न्द्रा॒ग्न मेका॑दशकपाल॒ मेका॑दशकपाल मैन्द्रा॒ग्न मै᳚न्द्रा॒ग्न मेका॑दशकपाल मपश्य दपश्य॒ देका॑दशकपाल मैन्द्रा॒ग्न मै᳚न्द्रा॒ग्न मेका॑दशकपाल मपश्यत् । \newline
28. ऐ॒न्द्रा॒ग्नमित्यै᳚न्द्र - अ॒ग्नम् । \newline
29. एका॑दशकपाल मपश्य दपश्य॒ देका॑दशकपाल॒ मेका॑दशकपाल मपश्य॒त् तम् त म॑पश्य॒ देका॑दशकपाल॒ मेका॑दशकपाल मपश्य॒त् तम् । \newline
30. एका॑दशकपाल॒मित्येका॑दश - क॒पा॒ल॒म् । \newline
31. अ॒प॒श्य॒त् तम् त म॑पश्य दपश्य॒त् तम् निर् णिष्ट म॑पश्य दपश्य॒त् तम् निः । \newline
32. तन्निर् णिष् टम् तन् निर॑वप दवप॒न् निष् टम् तन् निर॑वपत् । \newline
33. निर॑वप दवप॒न् निर् णिर॑वप॒त् तौ ता व॑वप॒न् निर् णिर॑वप॒त् तौ । \newline
34. अ॒व॒प॒त् तौ ता व॑वप दवप॒त् ता व॑स्मा अस्मै॒ ता व॑वप दवप॒त् ता व॑स्मै । \newline
35. ता व॑स्मा अस्मै॒ तौ ता व॑स्मै प्र॒जाः प्र॒जा अ॑स्मै॒ तौ ता व॑स्मै प्र॒जाः । \newline
36. अ॒स्मै॒ प्र॒जाः प्र॒जा अ॑स्मा अस्मै प्र॒जाः प्र प्र प्र॒जा अ॑स्मा अस्मै प्र॒जाः प्र । \newline
37. प्र॒जाः प्र प्र प्र॒जाः प्र॒जाः प्रासा॑धयता मसाधयता॒म् प्र प्र॒जाः प्र॒जाः प्रासा॑धयताम् । \newline
38. प्र॒जा इति॑ प्र - जाः । \newline
39. प्रासा॑धयता मसाधयता॒म् प्र प्रासा॑धयता मिन्द्रा॒ग्नी इ॑न्द्रा॒ग्नी अ॑साधयता॒म् प्र प्रासा॑धयता मिन्द्रा॒ग्नी । \newline
40. अ॒सा॒ध॒य॒ता॒ मि॒न्द्रा॒ग्नी इ॑न्द्रा॒ग्नी अ॑साधयता मसाधयता मिन्द्रा॒ग्नी वै वा इ॑न्द्रा॒ग्नी अ॑साधयता मसाधयता मिन्द्रा॒ग्नी वै । \newline
41. इ॒न्द्रा॒ग्नी वै वा इ॑न्द्रा॒ग्नी इ॑न्द्रा॒ग्नी वा ए॒तस्यै॒तस्य॒ वा इ॑न्द्रा॒ग्नी इ॑न्द्रा॒ग्नी वा ए॒तस्य॑ । \newline
42. इ॒न्द्रा॒ग्नी इती᳚न्द्र - अ॒ग्नी । \newline
43. वा ए॒तस्यै॒तस्य॒ वै वा ए॒तस्य॑ प्र॒जाम् प्र॒जा मे॒तस्य॒ वै वा ए॒तस्य॑ प्र॒जाम् । \newline
44. ए॒तस्य॑ प्र॒जाम् प्र॒जा मे॒त स्यै॒तस्य॑ प्र॒जा मपाप॑ प्र॒जा मे॒त स्यै॒तस्य॑ प्र॒जा मप॑ । \newline
45. प्र॒जा मपाप॑ प्र॒जाम् प्र॒जा मप॑ गूहतो गूह॒तो ऽप॑ प्र॒जाम् प्र॒जा मप॑ गूहतः । \newline
46. प्र॒जामिति॑ प्र - जाम् । \newline
47. अप॑ गूहतो गूह॒तो ऽपाप॑ गूहतो॒ यो यो गू॑ह॒तो ऽपाप॑ गूहतो॒ यः । \newline
48. गू॒ह॒तो॒ यो यो गू॑हतो गूहतो॒ यो ऽल॒ मलं॒ ॅयो गू॑हतो गूहतो॒ यो ऽल᳚म् । \newline
49. यो ऽल॒ मलं॒ ॅयो यो ऽल॑म् प्र॒जायै᳚ प्र॒जाया॒ अलं॒ ॅयो यो ऽल॑म् प्र॒जायै᳚ । \newline
50. अल॑म् प्र॒जायै᳚ प्र॒जाया॒ अल॒ मल॑म् प्र॒जायै॒ सन् थ्सन् प्र॒जाया॒ अल॒ मल॑म् प्र॒जायै॒ सन्न् । \newline
51. प्र॒जायै॒ सन् थ्सन् प्र॒जायै᳚ प्र॒जायै॒ सन् प्र॒जाम् प्र॒जाꣳ सन् प्र॒जायै᳚ प्र॒जायै॒ सन् प्र॒जाम् । \newline
52. प्र॒जाया॒ इति॑ प्र - जायै᳚ । \newline
53. सन् प्र॒जाम् प्र॒जाꣳ सन् थ्सन् प्र॒जाम् न न प्र॒जाꣳ सन् थ्सन् प्र॒जाम् न । \newline
54. प्र॒जाम् न न प्र॒जाम् प्र॒जाम् न वि॒न्दते॑ वि॒न्दते॒ न प्र॒जाम् प्र॒जाम् न वि॒न्दते᳚ । \newline
55. प्र॒जामिति॑ प्र - जाम् । \newline
56. न वि॒न्दते॑ वि॒न्दते॒ न न वि॒न्दत॑ ऐन्द्रा॒ग्न मै᳚न्द्रा॒ग्नं ॅवि॒न्दते॒ न न वि॒न्दत॑ ऐन्द्रा॒ग्नम् । \newline
57. वि॒न्दत॑ ऐन्द्रा॒ग्न मै᳚न्द्रा॒ग्नं ॅवि॒न्दते॑ वि॒न्दत॑ ऐन्द्रा॒ग्न मेका॑दशकपाल॒ मेका॑दशकपाल मैन्द्रा॒ग्नं ॅवि॒न्दते॑ वि॒न्दत॑ ऐन्द्रा॒ग्न मेका॑दशकपालम् । \newline
58. ऐ॒न्द्रा॒ग्न मेका॑दशकपाल॒ मेका॑दशकपाल मैन्द्रा॒ग्न मै᳚न्द्रा॒ग्न मेका॑दशकपाल॒म् निर् णिरेका॑दशकपाल मैन्द्रा॒ग्न मै᳚न्द्रा॒ग्न मेका॑दशकपाल॒म् निः । \newline
59. ऐ॒न्द्रा॒ग्नमित्यै᳚न्द्र - अ॒ग्नम् । \newline
60. एका॑दशकपाल॒म् निर् णिरेका॑दशकपाल॒ मेका॑दशकपाल॒म् निर् व॑पेद् वपे॒न् नि रेका॑दशकपाल॒ मेका॑दशकपाल॒म् निर् व॑पेत् । \newline
61. एका॑दशकपाल॒मित्येका॑दश - क॒पा॒ल॒म् । \newline
62. निर् व॑पेद् वपे॒न् निर् णिर् व॑पेत् प्र॒जाका॑मः प्र॒जाका॑मो वपे॒न् निर् णिर् व॑पेत् प्र॒जाका॑मः । \newline
63. व॒पे॒त् प्र॒जाका॑मः प्र॒जाका॑मो वपेद् वपेत् प्र॒जाका॑म इन्द्रा॒ग्नी इ॑न्द्रा॒ग्नी प्र॒जाका॑मो वपेद् वपेत् प्र॒जाका॑म इन्द्रा॒ग्नी । \newline
64. प्र॒जाका॑म इन्द्रा॒ग्नी इ॑न्द्रा॒ग्नी प्र॒जाका॑मः प्र॒जाका॑म इन्द्रा॒ग्नी ए॒वैवे न्द्रा॒ग्नी प्र॒जाका॑मः प्र॒जाका॑म इन्द्रा॒ग्नी ए॒व । \newline
65. प्र॒जाका॑म॒ इति॑ प्र॒जा - का॒मः॒ । \newline
66. इ॒न्द्रा॒ग्नी ए॒वैवे न्द्रा॒ग्नी इ॑न्द्रा॒ग्नी ए॒व स्वेन॒ स्वेनै॒वे न्द्रा॒ग्नी इ॑न्द्रा॒ग्नी ए॒व स्वेन॑ । \newline
67. इ॒न्द्रा॒ग्नी इती᳚न्द्र - अ॒ग्नी । \newline
\pagebreak
\markright{ TS 2.2.1.2  \hfill https://www.vedavms.in \hfill}
\addcontentsline{toc}{section}{ TS 2.2.1.2 }
\section*{ TS 2.2.1.2 }

\textbf{TS 2.2.1.2 } \newline
\textbf{Samhita Paata} \newline

ए॒व स्वेन॑ भाग॒धेये॒नोप॑ धावति॒ तावे॒वास्मै᳚ प्र॒जां प्र सा॑धयतो वि॒न्दते᳚ प्र॒जा-मै᳚न्द्रा॒ग्न-मेका॑दशकपालं॒ निर्व॑पे॒थ् स्पर्द्ध॑मानः॒ क्षेत्रे॑ वा सजा॒तेषु॑ वेन्द्रा॒ग्नी ए॒व स्वेन॑ भाग॒धेये॒नोप॑ धावति॒ ताभ्या॑मे॒वेन्द्रि॒यं ॅवी॒र्यं॑ भ्रातृ॑व्यस्य वृङ्क्ते॒ वि पा॒प्मना॒ भ्रातृ॑व्येण जय॒तेऽप॒ वा ए॒तस्मा॑दिन्द्रि॒यं ॅवी॒र्यं॑ क्रामति॒ यः स॑ग्रां॒म-मु॑पप्र॒यात्यै᳚न्द्रा॒ग्न-मेका॑दशकपालं॒ नि - [  ] \newline

\textbf{Pada Paata} \newline

ए॒व । स्वेन॑ । भा॒ग॒धेये॒नेति॑ भाग - धेये॑न । उपेति॑ । धा॒व॒ति॒ । तौ । ए॒व । अ॒स्मै॒ । प्र॒जामिति॑ प्र - जाम् । प्रेति॑ । सा॒ध॒य॒तः॒ । वि॒न्दते᳚ । प्र॒जामिति॑ प्र - जाम् । ऐ॒न्द्रा॒ग्नमित्यै᳚न्द्र - अ॒ग्नम् । एका॑दशकपाल॒मित्येका॑दश - क॒पा॒ल॒म् । निरिति॑ । व॒पे॒त् । स्पर्द्ध॑मानः । क्षेत्रे᳚ । वा॒ । स॒जा॒तेष्विति॑ स - जा॒तेषु॑ । वा॒ । इ॒न्द्रा॒ग्नी इती᳚न्द्र - अ॒ग्नी । ए॒व । स्वेन॑ । भा॒ग॒धेये॒नेति॑ भाग-धेये॑न । उपेति॑ । धा॒व॒ति॒ । ताभ्या᳚म् । ए॒व । इ॒न्द्रि॒यम् । वी॒र्य᳚म् । भ्रातृ॑व्यस्य । वृ॒ङ्क्ते॒ । वीति॑ । पा॒प्मना᳚ । भ्रातृ॑व्येण । ज॒य॒ते॒ । अपेति॑ । वै । ए॒तस्मा᳚त् । इ॒न्द्रि॒यम् । वी॒र्य᳚म् । क्रा॒म॒ति॒ । यः । स॒ग्रां॒ममिति॑ सं - ग्रा॒मम् । उ॒प॒प्र॒यातीत्यु॑प - प्र॒याति॑ । ऐ॒न्द्रा॒ग्नमित्यै᳚न्द्र-अ॒ग्नम् । एका॑दशकपाल॒मित्येका॑दश - क॒पा॒ल॒म् । निरिति॑ ।  \newline


\textbf{Krama Paata} \newline

ए॒व स्वेन॑ । स्वेन॑ भाग॒धेये॑न । भा॒ग॒धेये॒नोप॑ । भा॒ग॒धेये॒नेति॑ भाग - धेये॑न । उप॑ धावति । धा॒व॒ति॒ तौ । ता वे॒व । ए॒वास्मै᳚ । अ॒स्मै॒ प्र॒जाम् । प्र॒जाम् प्र । प्र॒जामिति॑ प्र - जाम् । प्र सा॑धयतः । सा॒ध॒य॒तो॒ वि॒न्दते᳚ । वि॒न्दते᳚ प्र॒जाम् । प्र॒जामै᳚न्द्रा॒ग्नम् । प्र॒जामिति॑ प्र - जाम् । ऐ॒न्द्रा॒ग्नमेका॑दशकपालम् । ऐ॒न्द्रा॒ग्नमित्यै᳚न्द्र - अ॒ग्नम् । एका॑दशकपाल॒म् निः । एका॑दशकपाल॒मित्येका॑दश - क॒पा॒ल॒म् । निर् व॑पेत् । व॒पे॒थ् स्पर्द्ध॑मानः । स्पर्द्ध॑मानः॒ क्षेत्रे᳚ । क्षेत्रे॑ वा । वा॒ स॒जा॒तेषु॑ । स॒जा॒तेषु॑ वा । स॒जा॒तेष्विति॑ स - जा॒तेषु॑ । वे॒न्द्रा॒ग्नी । इ॒न्द्रा॒ग्नी ए॒व । इ॒न्द्रा॒ग्नी इती᳚न्द्र - अ॒ग्नी । ए॒व स्वेन॑ । स्वेन॑ भाग॒धेये॑न । भा॒ग॒धेये॒नोप॑ । भा॒ग॒धेये॒नेति॑ भाग - धेये॑न । उप॑ धावति । धा॒व॒ति॒ ताभ्या᳚म् । ताभ्या॑मे॒व । ए॒वेन्द्रि॒यम् । इ॒न्द्रि॒यं ॅवी॒र्य᳚म् । वी॒र्य॑म् भ्रातृ॑व्यस्य । भ्रातृ॑व्यस्य वृङ्क्ते । वृ॒ङ्क्ते॒ वि । वि पा॒प्मना᳚ । पा॒प्मना॒ भ्रातृ॑व्येण । भ्रातृ॑व्येण जयते । ज॒य॒ते ऽप॑ । अप॒ वै । वा ए॒तस्मा᳚त् । ए॒तस्मा॑दिन्द्रि॒यम् । इ॒न्द्रि॒यं ॅवी॒र्य᳚म् । वी॒र्य॑म् क्रामति । क्रा॒म॒ति॒ यः । यः स॑ङ्ग्रा॒मम् । स॒ङ्ग्रा॒ममु॑पप्र॒याति॑ । स॒ङ्ग्रा॒ममिति॑ सं - ग्रा॒मम् । उ॒प॒प्र॒यात्यै᳚न्द्रा॒ग्नम् । उ॒प॒प्र॒यातीत्यु॑प - प्र॒याति॑ । ऐ॒न्द्रा॒ग्न,मेका॑दशकपालम् । ऐ॒न्द्रा॒ग्नमित्यै᳚न्द्र - अ॒ग्नम् । एका॑दशकपाल॒म् निः । एका॑दशकपाल॒मित्येका॑दश - क॒पा॒ल॒म् । निर् व॑पेत् \newline

\textbf{Jatai Paata} \newline

1. ए॒व स्वेन॒ स्वेनै॒वैव स्वेन॑ । \newline
2. स्वेन॑ भाग॒धेये॑न भाग॒धेये॑न॒ स्वेन॒ स्वेन॑ भाग॒धेये॑न । \newline
3. भा॒ग॒धेये॒नोपोप॑ भाग॒धेये॑न भाग॒धेये॒नोप॑ । \newline
4. भा॒ग॒धेये॒नेति॑ भाग - धेये॑न । \newline
5. उप॑ धावति धाव॒ त्युपोप॑ धावति । \newline
6. धा॒व॒ति॒ तौ तौ धा॑वति धावति॒ तौ । \newline
7. ता वे॒वैव तौ ता वे॒व । \newline
8. ए॒वास्मा॑ अस्मा ए॒वैवास्मै᳚ । \newline
9. अ॒स्मै॒ प्र॒जाम् प्र॒जा म॑स्मा अस्मै प्र॒जाम् । \newline
10. प्र॒जाम् प्र प्र प्र॒जाम् प्र॒जाम् प्र । \newline
11. प्र॒जामिति॑ प्र - जाम् । \newline
12. प्र सा॑धयतः साधयतः॒ प्र प्र सा॑धयतः । \newline
13. सा॒ध॒य॒तो॒ वि॒न्दते॑ वि॒न्दते॑ साधयतः साधयतो वि॒न्दते᳚ । \newline
14. वि॒न्दते᳚ प्र॒जाम् प्र॒जां ॅवि॒न्दते॑ वि॒न्दते᳚ प्र॒जाम् । \newline
15. प्र॒जा मै᳚न्द्रा॒ग्न मै᳚न्द्रा॒ग्नम् प्र॒जाम् प्र॒जा मै᳚न्द्रा॒ग्नम् । \newline
16. प्र॒जामिति॑ प्र - जाम् । \newline
17. ऐ॒न्द्रा॒ग्न मेका॑दशकपाल॒ मेका॑दशकपाल मैन्द्रा॒ग्न मै᳚न्द्रा॒ग्न मेका॑दशकपालम् । \newline
18. ऐ॒न्द्रा॒ग्नमित्यै᳚न्द्र - अ॒ग्नम् । \newline
19. एका॑दशकपाल॒म् निर् णिरेका॑दशकपाल॒ मेका॑दशकपाल॒म् निः । \newline
20. एका॑दशकपाल॒मित्येका॑दश - क॒पा॒ल॒म् । \newline
21. निर् व॑पेद् वपे॒न् निर् णिर् व॑पेत् । \newline
22. व॒पे॒थ् स्पर्द्ध॑मानः॒ स्पर्द्ध॑मानो वपेद् वपे॒थ् स्पर्द्ध॑मानः । \newline
23. स्पर्द्ध॑मानः॒ क्षेत्रे॒ क्षेत्रे॒ स्पर्द्ध॑मानः॒ स्पर्द्ध॑मानः॒ क्षेत्रे᳚ । \newline
24. क्षेत्रे॑ वा वा॒ क्षेत्रे॒ क्षेत्रे॑ वा । \newline
25. वा॒ स॒जा॒तेषु॑ सजा॒तेषु॑ वा वा सजा॒तेषु॑ । \newline
26. स॒जा॒तेषु॑ वा वा सजा॒तेषु॑ सजा॒तेषु॑ वा । \newline
27. स॒जा॒तेष्विति॑ स - जा॒तेषु॑ । \newline
28. वे॒न्द्रा॒ग्नी इ॑न्द्रा॒ग्नी वा॑ वेन्द्रा॒ग्नी । \newline
29. इ॒न्द्रा॒ग्नी ए॒वैवे न्द्रा॒ग्नी इ॑न्द्रा॒ग्नी ए॒व । \newline
30. इ॒न्द्रा॒ग्नी इती᳚न्द्र - अ॒ग्नी । \newline
31. ए॒व स्वेन॒ स्वेनै॒वैव स्वेन॑ । \newline
32. स्वेन॑ भाग॒धेये॑न भाग॒धेये॑न॒ स्वेन॒ स्वेन॑ भाग॒धेये॑न । \newline
33. भा॒ग॒धेये॒नोपोप॑ भाग॒धेये॑न भाग॒धेये॒नोप॑ । \newline
34. भा॒ग॒धेये॒नेति॑ भाग - धेये॑न । \newline
35. उप॑ धावति धाव॒ त्युपोप॑ धावति । \newline
36. धा॒व॒ति॒ ताभ्या॒म् ताभ्या᳚म् धावति धावति॒ ताभ्या᳚म् । \newline
37. ताभ्या॑ मे॒वैव ताभ्या॒म् ताभ्या॑ मे॒व । \newline
38. ए॒वे न्द्रि॒य मि॑न्द्रि॒य मे॒वैवे न्द्रि॒यम् । \newline
39. इ॒न्द्रि॒यं ॅवी॒र्यं॑ ॅवी॒र्य॑ मिन्द्रि॒य मि॑न्द्रि॒यं ॅवी॒र्य᳚म् । \newline
40. वी॒र्य॑म् भ्रातृ॑व्यस्य॒ भ्रातृ॑व्यस्य वी॒र्यं॑ ॅवी॒र्य॑म् भ्रातृ॑व्यस्य । \newline
41. भ्रातृ॑व्यस्य वृङ्क्ते वृङ्क्ते॒ भ्रातृ॑व्यस्य॒ भ्रातृ॑व्यस्य वृङ्क्ते । \newline
42. वृ॒ङ्क्ते॒ वि वि वृ॑ङ्क्ते वृङ्क्ते॒ वि । \newline
43. वि पा॒प्मना॑ पा॒प्मना॒ वि वि पा॒प्मना᳚ । \newline
44. पा॒प्मना॒ भ्रातृ॑व्येण॒ भ्रातृ॑व्येण पा॒प्मना॑ पा॒प्मना॒ भ्रातृ॑व्येण । \newline
45. भ्रातृ॑व्येण जयते जयते॒ भ्रातृ॑व्येण॒ भ्रातृ॑व्येण जयते । \newline
46. ज॒य॒ते ऽपाप॑ जयते जय॒ते ऽप॑ । \newline
47. अप॒ वै वा अपाप॒ वै । \newline
48. वा ए॒तस्मा॑ दे॒तस्मा॒द् वै वा ए॒तस्मा᳚त् । \newline
49. ए॒तस्मा॑ दिन्द्रि॒य मि॑न्द्रि॒य मे॒तस्मा॑ दे॒तस्मा॑ दिन्द्रि॒यम् । \newline
50. इ॒न्द्रि॒यं ॅवी॒र्यं॑ ॅवी॒र्य॑ मिन्द्रि॒य मि॑न्द्रि॒यं ॅवी॒र्य᳚म् । \newline
51. वी॒र्य॑म् क्रामति क्रामति वी॒र्यं॑ ॅवी॒र्य॑म् क्रामति । \newline
52. क्रा॒म॒ति॒ यो यः क्रा॑मति क्रामति॒ यः । \newline
53. यः स॑ङ्ग्रा॒मꣳ स॑ङ्ग्रा॒मं ॅयो यः स॑ङ्ग्रा॒मम् । \newline
54. स॒ङ्ग्रा॒म मु॑पप्र॒या त्यु॑पप्र॒याति॑ सङ्ग्रा॒मꣳ स॑ङ्ग्रा॒म मु॑पप्र॒याति॑ । \newline
55. स॒ङ्ग्रा॒ममिति॑ सं - ग्रा॒मम् । \newline
56. उ॒प॒प्र॒या त्यै᳚न्द्रा॒ग्न मै᳚न्द्रा॒ग्न मु॑पप्र॒या त्यु॑पप्र॒या त्यै᳚न्द्रा॒ग्नम् । \newline
57. उ॒प॒प्र॒यातीत्यु॑प - प्र॒याति॑ । \newline
58. ऐ॒न्द्रा॒ग्न मेका॑दशकपाल॒ मेका॑दशकपाल मैन्द्रा॒ग्न मै᳚न्द्रा॒ग्न मेका॑दशकपालम् । \newline
59. ऐ॒न्द्रा॒ग्नमित्यै᳚न्द्र - अ॒ग्नम् । \newline
60. एका॑दशकपाल॒म् निर् णिरेका॑दशकपाल॒ मेका॑दशकपाल॒म् निः । \newline
61. एका॑दशकपाल॒मित्येका॑दश - क॒पा॒ल॒म् । \newline
62. निर् व॑पेद् वपे॒न् निर् णिर् व॑पेत् । \newline

\textbf{Ghana Paata } \newline

1. ए॒व स्वेन॒ स्वेनै॒वैव स्वेन॑ भाग॒धेये॑न भाग॒धेये॑न॒ स्वेनै॒वैव स्वेन॑ भाग॒धेये॑न । \newline
2. स्वेन॑ भाग॒धेये॑न भाग॒धेये॑न॒ स्वेन॒ स्वेन॑ भाग॒धेये॒नोपोप॑ भाग॒धेये॑न॒ स्वेन॒ स्वेन॑ भाग॒धेये॒नोप॑ । \newline
3. भा॒ग॒धेये॒नोपोप॑ भाग॒धेये॑न भाग॒धेये॒नोप॑ धावति धाव॒त्युप॑ भाग॒धेये॑न भाग॒धेये॒नोप॑ धावति । \newline
4. भा॒ग॒धेये॒नेति॑ भाग - धेये॑न । \newline
5. उप॑ धावति धाव॒ त्युपोप॑ धावति॒ तौ तौ धा॑व॒ त्युपोप॑ धावति॒ तौ । \newline
6. धा॒व॒ति॒ तौ तौ धा॑वति धावति॒ ता वे॒वैव तौ धा॑वति धावति॒ ता वे॒व । \newline
7. ता वे॒वैव तौ ता वे॒वास्मा॑ अस्मा ए॒व तौ ता वे॒वास्मै᳚ । \newline
8. ए॒वास्मा॑ अस्मा ए॒वैवास्मै᳚ प्र॒जाम् प्र॒जा म॑स्मा ए॒वैवास्मै᳚ प्र॒जाम् । \newline
9. अ॒स्मै॒ प्र॒जाम् प्र॒जा म॑स्मा अस्मै प्र॒जाम् प्र प्र प्र॒जा म॑स्मा अस्मै प्र॒जाम् प्र । \newline
10. प्र॒जाम् प्र प्र प्र॒जाम् प्र॒जाम् प्र सा॑धयतः साधयतः॒ प्र प्र॒जाम् प्र॒जाम् प्र सा॑धयतः । \newline
11. प्र॒जामिति॑ प्र - जाम् । \newline
12. प्र सा॑धयतः साधयतः॒ प्र प्र सा॑धयतो वि॒न्दते॑ वि॒न्दते॑ साधयतः॒ प्र प्र सा॑धयतो वि॒न्दते᳚ । \newline
13. सा॒ध॒य॒तो॒ वि॒न्दते॑ वि॒न्दते॑ साधयतः साधयतो वि॒न्दते᳚ प्र॒जाम् प्र॒जां ॅवि॒न्दते॑ साधयतः साधयतो वि॒न्दते᳚ प्र॒जाम् । \newline
14. वि॒न्दते᳚ प्र॒जाम् प्र॒जां ॅवि॒न्दते॑ वि॒न्दते᳚ प्र॒जा मै᳚न्द्रा॒ग्न मै᳚न्द्रा॒ग्नम् प्र॒जां ॅवि॒न्दते॑ वि॒न्दते᳚ प्र॒जा मै᳚न्द्रा॒ग्नम् । \newline
15. प्र॒जा मै᳚न्द्रा॒ग्न मै᳚न्द्रा॒ग्नम् प्र॒जाम् प्र॒जा मै᳚न्द्रा॒ग्न मेका॑दशकपाल॒ मेका॑दशकपाल मैन्द्रा॒ग्नम् प्र॒जाम् प्र॒जा मै᳚न्द्रा॒ग्न मेका॑दशकपालम् । \newline
16. प्र॒जामिति॑ प्र - जाम् । \newline
17. ऐ॒न्द्रा॒ग्न मेका॑दशकपाल॒ मेका॑दशकपाल मैन्द्रा॒ग्न मै᳚न्द्रा॒ग्न मेका॑दशकपाल॒म् निर् णिरेका॑दशकपाल मैन्द्रा॒ग्न मै᳚न्द्रा॒ग्न मेका॑दशकपाल॒म् निः । \newline
18. ऐ॒न्द्रा॒ग्नमित्यै᳚न्द्र - अ॒ग्नम् । \newline
19. एका॑दशकपाल॒म् निर् णिरेका॑दशकपाल॒ मेका॑दशकपाल॒म् निर् व॑पेद् वपे॒न् निरेका॑दशकपाल॒ मेका॑दशकपाल॒म् निर् व॑पेत् । \newline
20. एका॑दशकपाल॒मित्येका॑दश - क॒पा॒ल॒म् । \newline
21. निर् व॑पेद् वपे॒न् निर् णिर् व॑पे॒थ् स्पर्द्ध॑मानः॒ स्पर्द्ध॑मानो वपे॒न् निर् णिर् व॑पे॒थ् स्पर्द्ध॑मानः । \newline
22. व॒पे॒थ् स्पर्द्ध॑मानः॒ स्पर्द्ध॑मानो वपेद् वपे॒थ् स्पर्द्ध॑मानः॒ क्षेत्रे॒ क्षेत्रे॒ स्पर्द्ध॑मानो वपेद् वपे॒थ् स्पर्द्ध॑मानः॒ क्षेत्रे᳚ । \newline
23. स्पर्द्ध॑मानः॒ क्षेत्रे॒ क्षेत्रे॒ स्पर्द्ध॑मानः॒ स्पर्द्ध॑मानः॒ क्षेत्रे॑ वा वा॒ क्षेत्रे॒ स्पर्द्ध॑मानः॒ स्पर्द्ध॑मानः॒ क्षेत्रे॑ वा । \newline
24. क्षेत्रे॑ वा वा॒ क्षेत्रे॒ क्षेत्रे॑ वा सजा॒तेषु॑ सजा॒तेषु॑ वा॒ क्षेत्रे॒ क्षेत्रे॑ वा सजा॒तेषु॑ । \newline
25. वा॒ स॒जा॒तेषु॑ सजा॒तेषु॑ वा वा सजा॒तेषु॑ वा वा सजा॒तेषु॑ वा वा सजा॒तेषु॑ वा । \newline
26. स॒जा॒तेषु॑ वा वा सजा॒तेषु॑ सजा॒तेषु॑ वेन्द्रा॒ग्नी इ॑न्द्रा॒ग्नी वा॑ सजा॒तेषु॑ सजा॒तेषु॑ वेन्द्रा॒ग्नी । \newline
27. स॒जा॒तेष्विति॑ स - जा॒तेषु॑ । \newline
28. वे॒न्द्रा॒ग्नी इ॑न्द्रा॒ग्नी वा॑ वेन्द्रा॒ग्नी ए॒वैवे न्द्रा॒ग्नी वा॑ वेन्द्रा॒ग्नी ए॒व । \newline
29. इ॒न्द्रा॒ग्नी ए॒वैवे न्द्रा॒ग्नी इ॑न्द्रा॒ग्नी ए॒व स्वेन॒ स्वेनै॒वे न्द्रा॒ग्नी इ॑न्द्रा॒ग्नी ए॒व स्वेन॑ । \newline
30. इ॒न्द्रा॒ग्नी इती᳚न्द्र - अ॒ग्नी । \newline
31. ए॒व स्वेन॒ स्वेनै॒वैव स्वेन॑ भाग॒धेये॑न भाग॒धेये॑न॒ स्वेनै॒वैव स्वेन॑ भाग॒धेये॑न । \newline
32. स्वेन॑ भाग॒धेये॑न भाग॒धेये॑न॒ स्वेन॒ स्वेन॑ भाग॒धेये॒नोपोप॑ भाग॒धेये॑न॒ स्वेन॒ स्वेन॑ भाग॒धेये॒नोप॑ । \newline
33. भा॒ग॒धेये॒नोपोप॑ भाग॒धेये॑न भाग॒धेये॒नोप॑ धावति धाव॒त्युप॑ भाग॒धेये॑न भाग॒धेये॒नोप॑ धावति । \newline
34. भा॒ग॒धेये॒नेति॑ भाग - धेये॑न । \newline
35. उप॑ धावति धाव॒ त्युपोप॑ धावति॒ ताभ्या॒म् ताभ्या᳚म् धाव॒ त्युपोप॑ धावति॒ ताभ्या᳚म् । \newline
36. धा॒व॒ति॒ ताभ्या॒म् ताभ्या᳚म् धावति धावति॒ ताभ्या॑ मे॒वैव ताभ्या᳚म् धावति धावति॒ ताभ्या॑ मे॒व । \newline
37. ताभ्या॑ मे॒वैव ताभ्या॒म् ताभ्या॑ मे॒वे न्द्रि॒य मि॑न्द्रि॒य मे॒व ताभ्या॒म् ताभ्या॑ मे॒वे न्द्रि॒यम् । \newline
38. ए॒वे न्द्रि॒य मि॑न्द्रि॒य मे॒वैवे न्द्रि॒यं ॅवी॒र्यं॑ ॅवी॒र्य॑ मिन्द्रि॒य मे॒वैवे न्द्रि॒यं ॅवी॒र्य᳚म् । \newline
39. इ॒न्द्रि॒यं ॅवी॒र्यं॑ ॅवी॒र्य॑ मिन्द्रि॒य मि॑न्द्रि॒यं ॅवी॒र्य॑म् भ्रातृ॑व्यस्य॒ भ्रातृ॑व्यस्य वी॒र्य॑ मिन्द्रि॒य मि॑न्द्रि॒यं ॅवी॒र्य॑म् भ्रातृ॑व्यस्य । \newline
40. वी॒र्य॑म् भ्रातृ॑व्यस्य॒ भ्रातृ॑व्यस्य वी॒र्यं॑ ॅवी॒र्य॑म् भ्रातृ॑व्यस्य वृङ्क्ते वृङ्क्ते॒ भ्रातृ॑व्यस्य वी॒र्यं॑ ॅवी॒र्य॑म् भ्रातृ॑व्यस्य वृङ्क्ते । \newline
41. भ्रातृ॑व्यस्य वृङ्क्ते वृङ्क्ते॒ भ्रातृ॑व्यस्य॒ भ्रातृ॑व्यस्य वृङ्क्ते॒ वि वि वृ॑ङ्क्ते॒ भ्रातृ॑व्यस्य॒ भ्रातृ॑व्यस्य वृङ्क्ते॒ वि । \newline
42. वृ॒ङ्क्ते॒ वि वि वृ॑ङ्क्ते वृङ्क्ते॒ वि पा॒प्मना॑ पा॒प्मना॒ वि वृ॑ङ्क्ते वृङ्क्ते॒ वि पा॒प्मना᳚ । \newline
43. वि पा॒प्मना॑ पा॒प्मना॒ वि वि पा॒प्मना॒ भ्रातृ॑व्येण॒ भ्रातृ॑व्येण पा॒प्मना॒ वि वि पा॒प्मना॒ भ्रातृ॑व्येण । \newline
44. पा॒प्मना॒ भ्रातृ॑व्येण॒ भ्रातृ॑व्येण पा॒प्मना॑ पा॒प्मना॒ भ्रातृ॑व्येण जयते जयते॒ भ्रातृ॑व्येण पा॒प्मना॑ पा॒प्मना॒ भ्रातृ॑व्येण जयते । \newline
45. भ्रातृ॑व्येण जयते जयते॒ भ्रातृ॑व्येण॒ भ्रातृ॑व्येण जय॒ते ऽपाप॑ जयते॒ भ्रातृ॑व्येण॒ भ्रातृ॑व्येण जय॒ते ऽप॑ । \newline
46. ज॒य॒ते ऽपाप॑ जयते जय॒ते ऽप॒ वै वा अप॑ जयते जय॒ते ऽप॒ वै । \newline
47. अप॒ वै वा अपाप॒ वा ए॒तस्मा॑ दे॒तस्मा॒द् वा अपाप॒ वा ए॒तस्मा᳚त् । \newline
48. वा ए॒तस्मा॑ दे॒तस्मा॒द् वै वा ए॒तस्मा॑ दिन्द्रि॒य मि॑न्द्रि॒य मे॒तस्मा॒द् वै वा ए॒तस्मा॑ दिन्द्रि॒यम् । \newline
49. ए॒तस्मा॑ दिन्द्रि॒य मि॑न्द्रि॒य मे॒तस्मा॑ दे॒तस्मा॑ दिन्द्रि॒यं ॅवी॒र्यं॑ ॅवी॒र्य॑ मिन्द्रि॒य मे॒तस्मा॑ दे॒तस्मा॑ दिन्द्रि॒यं ॅवी॒र्य᳚म् । \newline
50. इ॒न्द्रि॒यं ॅवी॒र्यं॑ ॅवी॒र्य॑ मिन्द्रि॒य मि॑न्द्रि॒यं ॅवी॒र्य॑म् क्रामति क्रामति वी॒र्य॑ मिन्द्रि॒य मि॑न्द्रि॒यं ॅवी॒र्य॑म् क्रामति । \newline
51. वी॒र्य॑म् क्रामति क्रामति वी॒र्यं॑ ॅवी॒र्य॑म् क्रामति॒ यो यः क्रा॑मति वी॒र्यं॑ ॅवी॒र्य॑म् क्रामति॒ यः । \newline
52. क्रा॒म॒ति॒ यो यः क्रा॑मति क्रामति॒ यः स॑ङ्ग्रा॒मꣳ स॑ङ्ग्रा॒मं ॅयः क्रा॑मति क्रामति॒ यः स॑ङ्ग्रा॒मम् । \newline
53. यः स॑ङ्ग्रा॒मꣳ स॑ङ्ग्रा॒मं ॅयो यः स॑ङ्ग्रा॒म मु॑पप्र॒या त्यु॑पप्र॒याति॑ सङ्ग्रा॒मं ॅयो यः स॑ङ्ग्रा॒म मु॑पप्र॒याति॑ । \newline
54. स॒ङ्ग्रा॒म मु॑पप्र॒या त्यु॑पप्र॒याति॑ सङ्ग्रा॒मꣳ स॑ङ्ग्रा॒म मु॑पप्र॒या त्यै᳚न्द्रा॒ग्न मै᳚न्द्रा॒ग्न मु॑पप्र॒याति॑ सङ्ग्रा॒मꣳ स॑ङ्ग्रा॒म मु॑पप्र॒या त्यै᳚न्द्रा॒ग्नम् । \newline
55. स॒ङ्ग्रा॒ममिति॑ सं - ग्रा॒मम् । \newline
56. उ॒प॒प्र॒या त्यै᳚न्द्रा॒ग्न मै᳚न्द्रा॒ग्न मु॑पप्र॒या त्यु॑पप्र॒या त्यै᳚न्द्रा॒ग्न मेका॑दशकपाल॒ मेका॑दशकपाल मैन्द्रा॒ग्न मु॑पप्र॒या त्यु॑पप्र॒या त्यै᳚न्द्रा॒ग्न मेका॑दशकपालम् । \newline
57. उ॒प॒प्र॒यातीत्यु॑प - प्र॒याति॑ । \newline
58. ऐ॒न्द्रा॒ग्न मेका॑दशकपाल॒ मेका॑दशकपाल मैन्द्रा॒ग्न मै᳚न्द्रा॒ग्न मेका॑दशकपाल॒म् निर् णिरेका॑दशकपाल मैन्द्रा॒ग्न मै᳚न्द्रा॒ग्न मेका॑दशकपाल॒म् निः । \newline
59. ऐ॒न्द्रा॒ग्नमित्यै᳚न्द्र - अ॒ग्नम् । \newline
60. एका॑दशकपाल॒म् निर् णिरेका॑दशकपाल॒ मेका॑दशकपाल॒म् निर् व॑पेद् वपे॒न् निरेका॑दशकपाल॒ मेका॑दशकपाल॒म् निर् व॑पेत् । \newline
61. एका॑दशकपाल॒मित्येका॑दश - क॒पा॒ल॒म् । \newline
62. निर् व॑पेद् वपे॒न् निर् णिर् व॑पेथ् सङ्ग्रा॒मꣳ स॑ङ्ग्रा॒मं ॅव॑पे॒न् निर् णिर् व॑पेथ् सङ्ग्रा॒मम् । \newline
\pagebreak
\markright{ TS 2.2.1.3  \hfill https://www.vedavms.in \hfill}
\addcontentsline{toc}{section}{ TS 2.2.1.3 }
\section*{ TS 2.2.1.3 }

\textbf{TS 2.2.1.3 } \newline
\textbf{Samhita Paata} \newline

-र्व॑पेथ् संग्रा॒म-मु॑पप्रया॒स्यन्नि॑न्द्रा॒ग्नी ए॒व स्वेन॑ भाग॒धेये॒नोप॑ धावति॒ तावे॒वास्मि॑न्निन्द्रि॒यं ॅवी॒र्यं॑ धत्तः स॒हेन्द्रि॒येण॑ वी॒र्ये॑णोप॒ प्र या॑ति॒ जय॑ति॒ तꣳ स॑ग्रां॒मं ॅवि वा ए॒ष इ॑न्द्रि॒येण॑ वी॒र्ये॑णर्द्ध्यते॒ यः स॑ग्रां॒मं जय॑त्यैन्द्रा॒ग्न-मेका॑दशकपालं॒ निर्व॑पेथ् संग्रा॒मं जि॒त्वेन्द्रा॒ग्नी ए॒व स्वेन॑ भाग॒धेये॒नोप॑ धावति॒ तावे॒वास्मि॑न्निद्रि॒यं ॅवी॒र्यं॑ - [  ] \newline

\textbf{Pada Paata} \newline

व॒पे॒त् । स॒ग्रां॒ममिति॑ सं - ग्रा॒मम् । उ॒प॒प्र॒या॒स्यन्नित्यु॑प - प्र॒या॒स्यन्न् । इ॒न्द्रा॒ग्नी इती᳚न्द्र - अ॒ग्नी । ए॒व । स्वेन॑ । भा॒ग॒धेये॒नेति॑ भाग - धेये॑न । उपेति॑ । धा॒व॒ति॒ । तौ । ए॒व । अ॒स्मि॒न्न् । इ॒न्द्रि॒यम् । वी॒र्य᳚म् । ध॒त्तः॒ । स॒ह । इ॒न्द्रि॒येण॑ । वी॒र्ये॑ण । उप॑ । प्रेति॑ । या॒ति॒ । जय॑ति । तम् । स॒ग्रां॒ममिति॑ सं - ग्रा॒मम् । वीति॑ । वै । ए॒षः । इ॒न्द्रि॒येण॑ । वी॒र्ये॑ण । ऋ॒द्ध्य॒ते॒ । यः । सं॒ग्रा॒ममिति॑ सं - ग्रा॒मम् । जय॑ति । ऐ॒न्द्रा॒ग्नमित्यै᳚न्द्र - अ॒ग्नम् । एका॑दशकपाल॒मित्येका॑दश - क॒पा॒ल॒म् । निरिति॑ । व॒पे॒त् । स॒ग्रां॒ममिति॑ सं - ग्रा॒मम् । जि॒त्वा । इ॒न्द्रा॒ग्नी इती᳚न्द्र - अ॒ग्नी । ए॒व । स्वेन॑ । भा॒ग॒धेये॒नेति॑ भाग - धेये॑न । उपेति॑ । धा॒व॒ति॒ । तौ । ए॒व । अ॒स्मि॒न्न् । इ॒न्द्रि॒यम् । वी॒र्य᳚म् ।  \newline


\textbf{Krama Paata} \newline

व॒पे॒थ् स॒ङ्ग्रा॒मम् । स॒ङ्ग्रा॒म,मु॑पप्रया॒स्यन्न् । स॒ङ्ग्रा॒ममिति॑ सं - ग्रा॒मम् । उ॒प॒प्र॒या॒स्यन्नि॑न्द्रा॒ग्नी । उ॒प॒प्र॒या॒स्यन्नित्यु॑प - प्र॒या॒स्यन्न् । इ॒न्द्रा॒ग्नी ए॒व । इ॒न्द्रा॒ग्नी इती᳚न्द्र - अ॒ग्नी । ए॒व स्वेन॑ । स्वेन॑ भाग॒धेये॑न । भा॒ग॒धेये॒नोप॑ । भा॒ग॒धेये॒नेति॑ भाग - धेये॑न । उप॑ धावति । धा॒व॒ति॒ तौ । ता वे॒व । ए॒वास्मिन्न्॑ । अ॒स्मि॒न्नि॒न्द्रि॒यम् । इ॒न्द्रि॒यं ॅवी॒र्य᳚म् । वी॒र्य॑म् धत्तः । ध॒त्तः॒ स॒ह । स॒हेन्द्रि॒येण॑ । इ॒न्द्रि॒येण॑ वी॒र्ये॑ण । वी॒र्ये॑णोप॑ । उप॒ प्र । प्र या॑ति । या॒ति॒ जय॑ति । जय॑ति॒ तम् । तꣳ स॑ङ्ग्रा॒मम् । स॒ङ्ग्रा॒मं ॅवि । स॒ङ्ग्रा॒ममिति॑ सं - ग्रा॒मम् । वि वै । वा ए॒षः । ए॒ष इ॑न्द्रि॒येण॑ । इ॒न्द्रि॒येण॑ वी॒र्ये॑ण । वी॒र्ये॑णर्द्ध्यते । ऋ॒द्ध्य॒ते॒ यः । यः स॑ङ्ग्रा॒मम् । स॒ङ्ग्रा॒मम् जय॑ति । स॒ङ्ग्रा॒ममिति॑ सं - ग्रा॒मम् । जय॑त्यैन्द्रा॒ग्नम् । ऐ॒न्द्रा॒ग्नमेका॑दशकपालम् । ऐ॒न्द्रा॒ग्नमित्यै᳚न्द्र - अ॒ग्नम् । एका॑दशकपाल॒म् निः । एका॑दशकपाल॒मित्येका॑दश - क॒पा॒ल॒म् । निर् व॑पेत् । व॒पे॒थ् स॒ङ्ग्रा॒मम् । स॒ङ्ग्रा॒मम् जि॒त्वा । स॒ङ्ग्रा॒ममिति॑ सं - ग्रा॒मम् । जि॒त्वेन्द्रा॒ग्नी । इ॒न्द्रा॒ग्नी ए॒व । इ॒न्द्रा॒ग्नी इती᳚न्द्र - अ॒ग्नी । ए॒व स्वेन॑ । स्वेन॑ भाग॒धेये॑न । भा॒ग॒धेये॒नोप॑ । भा॒ग॒धेये॒नेति॑ भाग - धेये॑न । उप॑ धावति । धा॒व॒ति॒ तौ । तावे॒व । ए॒वास्मिन्न्॑ । अ॒स्मि॒न्नि॒न्द्रि॒यम् । इ॒न्द्रि॒यं ॅवी॒र्यं᳚ । वी॒र्य॑म् धत्तः \newline

\textbf{Jatai Paata} \newline

1. व॒पे॒थ् स॒ङ्ग्रा॒मꣳ स॑ङ्ग्रा॒मं ॅव॑पेद् वपेथ् सङ्ग्रा॒मम् । \newline
2. स॒ङ्ग्रा॒म मु॑पप्रया॒स्यन् नु॑पप्रया॒स्यन् थ्स॑ङ्ग्रा॒मꣳ स॑ङ्ग्रा॒म मु॑पप्रया॒स्यन्न् । \newline
3. स॒ङ्ग्रा॒ममिति॑ सं - ग्रा॒मम् । \newline
4. उ॒प॒प्र॒या॒स्यन् नि॑न्द्रा॒ग्नी इ॑न्द्रा॒ग्नी उ॑पप्रया॒स्यन् नु॑पप्रया॒स्यन् नि॑न्द्रा॒ग्नी । \newline
5. उ॒प॒प्र॒या॒स्यन्नित्यु॑प - प्र॒या॒स्यन्न् । \newline
6. इ॒न्द्रा॒ग्नी ए॒वैवे न्द्रा॒ग्नी इ॑न्द्रा॒ग्नी ए॒व । \newline
7. इ॒न्द्रा॒ग्नी इती᳚न्द्र - अ॒ग्नी । \newline
8. ए॒व स्वेन॒ स्वे नै॒वैव स्वेन॑ । \newline
9. स्वेन॑ भाग॒धेये॑न भाग॒धेये॑न॒ स्वेन॒ स्वेन॑ भाग॒धेये॑न । \newline
10. भा॒ग॒धेये॒नोपोप॑ भाग॒धेये॑न भाग॒धेये॒नोप॑ । \newline
11. भा॒ग॒धेये॒नेति॑ भाग - धेये॑न । \newline
12. उप॑ धावति धाव॒ त्युपोप॑ धावति । \newline
13. धा॒व॒ति॒ तौ तौ धा॑वति धावति॒ तौ । \newline
14. ता वे॒ वैव तौ ता वे॒व । \newline
15. ए॒वास्मि॑न् नस्मिन् ने॒वैवास्मिन्न्॑ । \newline
16. अ॒स्मि॒न् नि॒न्द्रि॒य मि॑न्द्रि॒य म॑स्मिन् नस्मिन् निन्द्रि॒यम् । \newline
17. इ॒न्द्रि॒यं ॅवी॒र्यं॑ ॅवी॒र्य॑ मिन्द्रि॒य मि॑न्द्रि॒यं ॅवी॒र्य᳚म् । \newline
18. वी॒र्य॑म् धत्तो धत्तो वी॒र्यं॑ ॅवी॒र्य॑म् धत्तः । \newline
19. ध॒त्तः॒ स॒ह स॒ह ध॑त्तो धत्तः स॒ह । \newline
20. स॒हे न्द्रि॒येणे᳚ न्द्रि॒येण॑ स॒ह स॒हे न्द्रि॒येण॑ । \newline
21. इ॒न्द्रि॒येण॑ वी॒र्ये॑ण वी॒र्ये॑णे न्द्रि॒येणे᳚ न्द्रि॒येण॑ वी॒र्ये॑ण । \newline
22. वी॒र्ये॑ णोपोप॑ वी॒र्ये॑ण वी॒र्ये॑ णोप॑ । \newline
23. उप॒ प्र प्रोपोप॒ प्र । \newline
24. प्र या॑ति याति॒ प्र प्र या॑ति । \newline
25. या॒ति॒ जय॑ति॒ जय॑ति याति याति॒ जय॑ति । \newline
26. जय॑ति॒ तम् तम् जय॑ति॒ जय॑ति॒ तम् । \newline
27. तꣳ स॑ङ्ग्रा॒मꣳ स॑ङ्ग्रा॒मम् तम् तꣳ स॑ङ्ग्रा॒मम् । \newline
28. स॒ङ्ग्रा॒मं ॅवि वि स॑ङ्ग्रा॒मꣳ स॑ङ्ग्रा॒मं ॅवि । \newline
29. स॒ङ्ग्रा॒ममिति॑ सं - ग्रा॒मम् । \newline
30. वि वै वै वि वि वै । \newline
31. वा ए॒ष ए॒ष वै वा ए॒षः । \newline
32. ए॒ष इ॑न्द्रि॒येणे᳚ न्द्रि॒येणै॒ष ए॒ष इ॑न्द्रि॒येण॑ । \newline
33. इ॒न्द्रि॒येण॑ वी॒र्ये॑ण वी॒र्ये॑णे न्द्रि॒येणे᳚ न्द्रि॒येण॑ वी॒र्ये॑ण । \newline
34. वी॒र्ये॑ण र्द्ध्यत ऋद्ध्यते वी॒र्ये॑ण वी॒र्ये॑ण र्द्ध्यते । \newline
35. ऋ॒द्ध्य॒ते॒ यो य ऋ॑द्ध्यत ऋद्ध्यते॒ यः । \newline
36. यः स॑ङ्ग्रा॒मꣳ स॑ङ्ग्रा॒मं ॅयो यः स॑ङ्ग्रा॒मम् । \newline
37. स॒ङ्ग्रा॒मम् जय॑ति॒ जय॑ति सङ्ग्रा॒मꣳ स॑ङ्ग्रा॒मम् जय॑ति । \newline
38. स॒ङ्ग्रा॒ममिति॑ सं - ग्रा॒मम् । \newline
39. जय॑ त्यैन्द्रा॒ग्न मै᳚न्द्रा॒ग्नम् जय॑ति॒ जय॑ त्यैन्द्रा॒ग्नम् । \newline
40. ऐ॒न्द्रा॒ग्न मेका॑दशकपाल॒ मेका॑दशकपाल मैन्द्रा॒ग्न मै᳚न्द्रा॒ग्न मेका॑दशकपालम् । \newline
41. ऐ॒न्द्रा॒ग्नमित्यै᳚न्द्र - अ॒ग्नम् । \newline
42. एका॑दशकपाल॒म् निर् णिरेका॑दशकपाल॒ मेका॑दशकपाल॒म् निः । \newline
43. एका॑दशकपाल॒मित्येका॑दश - क॒पा॒ल॒म् । \newline
44. निर् व॑पेद् वपे॒न् निर् णिर् व॑पेत् । \newline
45. व॒पे॒थ् स॒ङ्ग्रा॒मꣳ स॑ङ्ग्रा॒मं ॅव॑पेद् वपेथ् सङ्ग्रा॒मम् । \newline
46. स॒ङ्ग्रा॒मम् जि॒त्वा जि॒त्वा स॑ङ्ग्रा॒मꣳ स॑ङ्ग्रा॒मम् जि॒त्वा । \newline
47. स॒ङ्ग्रा॒ममिति॑ सं - ग्रा॒मम् । \newline
48. जि॒त्वेन्द्रा॒ग्नी इ॑न्द्रा॒ग्नी जि॒त्वा जि॒त्वेन्द्रा॒ग्नी । \newline
49. इ॒न्द्रा॒ग्नी ए॒वैवे न्द्रा॒ग्नी इ॑न्द्रा॒ग्नी ए॒व । \newline
50. इ॒न्द्रा॒ग्नी इती᳚न्द्र - अ॒ग्नी । \newline
51. ए॒व स्वेन॒ स्वे नै॒वैव स्वेन॑ । \newline
52. स्वेन॑ भाग॒धेये॑न भाग॒धेये॑न॒ स्वेन॒ स्वेन॑ भाग॒धेये॑न । \newline
53. भा॒ग॒धेये॒नोपोप॑ भाग॒धेये॑न भाग॒धेये॒नोप॑ । \newline
54. भा॒ग॒धेये॒नेति॑ भाग - धेये॑न । \newline
55. उप॑ धावति धाव॒ त्युपोप॑ धावति । \newline
56. धा॒व॒ति॒ तौ तौ धा॑वति धावति॒ तौ । \newline
57. ता वे॒वैव तौ ता वे॒व । \newline
58. ए॒वास्मि॑न् नस्मिन् ने॒वैवास्मिन्न्॑ । \newline
59. अ॒स्मि॒न् नि॒न्द्रि॒य मि॑न्द्रि॒य म॑स्मिन् नस्मिन् निन्द्रि॒यम् । \newline
60. इ॒न्द्रि॒यं ॅवी॒र्यं॑ ॅवी॒र्य॑ मिन्द्रि॒य मि॑न्द्रि॒यं ॅवी॒र्य᳚म् । \newline
61. वी॒र्य॑म् धत्तो धत्तो वी॒र्यं॑ ॅवी॒र्य॑म् धत्तः । \newline

\textbf{Ghana Paata } \newline

1. व॒पे॒थ् स॒ङ्ग्रा॒मꣳ स॑ङ्ग्रा॒मं ॅव॑पेद् वपेथ् सङ्ग्रा॒म 
मु॑पप्रया॒स्यन् नु॑पप्रया॒स्यन् थ्स॑ङ्ग्रा॒मं ॅव॑पेद् वपेथ् सङ्ग्रा॒म मु॑पप्रया॒स्यन्न् । \newline
2. स॒ङ्ग्रा॒म मु॑पप्रया॒स्यन् नु॑पप्रया॒स्यन् थ्स॑ङ्ग्रा॒मꣳ स॑ङ्ग्रा॒म मु॑पप्रया॒स्यन् नि॑न्द्रा॒ग्नी इ॑न्द्रा॒ग्नी उ॑पप्रया॒स्यन् थ्स॑ङ्ग्रा॒मꣳ स॑ङ्ग्रा॒म मु॑पप्रया॒स्यन् नि॑न्द्रा॒ग्नी । \newline
3. स॒ङ्ग्रा॒ममिति॑ सं - ग्रा॒मम् । \newline
4. उ॒प॒प्र॒या॒स्यन् नि॑न्द्रा॒ग्नी इ॑न्द्रा॒ग्नी उ॑पप्रया॒स्यन् नु॑पप्रया॒स्यन् नि॑न्द्रा॒ग्नी ए॒वैवे न्द्रा॒ग्नी उ॑पप्रया॒स्यन् नु॑पप्रया॒स्यन् नि॑न्द्रा॒ग्नी ए॒व । \newline
5. उ॒प॒प्र॒या॒स्यन्नित्यु॑प - प्र॒या॒स्यन्न् । \newline
6. इ॒न्द्रा॒ग्नी ए॒वैवे न्द्रा॒ग्नी इ॑न्द्रा॒ग्नी ए॒व स्वेन॒ स्वेनै॒वे न्द्रा॒ग्नी इ॑न्द्रा॒ग्नी ए॒व स्वेन॑ । \newline
7. इ॒न्द्रा॒ग्नी इती᳚न्द्र - अ॒ग्नी । \newline
8. ए॒व स्वेन॒ स्वेनै॒वैव स्वेन॑ भाग॒धेये॑न भाग॒धेये॑न॒ स्वेनै॒वैव स्वेन॑ भाग॒धेये॑न । \newline
9. स्वेन॑ भाग॒धेये॑न भाग॒धेये॑न॒ स्वेन॒ स्वेन॑ भाग॒धेये॒नोपोप॑ भाग॒धेये॑न॒ स्वेन॒ स्वेन॑ भाग॒धेये॒नोप॑ । \newline
10. भा॒ग॒धेये॒नोपोप॑ भाग॒धेये॑न भाग॒धेये॒नोप॑ धावति धाव॒त्युप॑ भाग॒धेये॑न भाग॒धेये॒नोप॑ धावति । \newline
11. भा॒ग॒धेये॒नेति॑ भाग - धेये॑न । \newline
12. उप॑ धावति धाव॒ त्युपोप॑ धावति॒ तौ तौ धा॑व॒ त्युपोप॑ धावति॒ तौ । \newline
13. धा॒व॒ति॒ तौ तौ धा॑वति धावति॒ ता वे॒वैव तौ धा॑वति धावति॒ ता वे॒व । \newline
14. ता वे॒वैव तौ ता वे॒वास्मि॑न् नस्मिन् ने॒व तौ ता वे॒वास्मिन्न्॑ । \newline
15. ए॒वास्मि॑न् नस्मिन् ने॒वैवास्मि॑न् निन्द्रि॒य मि॑न्द्रि॒य म॑स्मिन् ने॒वैवास्मि॑न् निन्द्रि॒यम् । \newline
16. अ॒स्मि॒न् नि॒न्द्रि॒य मि॑न्द्रि॒य म॑स्मिन् नस्मिन् निन्द्रि॒यं ॅवी॒र्यं॑ ॅवी॒र्य॑ मिन्द्रि॒य म॑स्मिन् नस्मिन् निन्द्रि॒यं ॅवी॒र्य᳚म् । \newline
17. इ॒न्द्रि॒यं ॅवी॒र्यं॑ ॅवी॒र्य॑ मिन्द्रि॒य मि॑न्द्रि॒यं ॅवी॒र्य॑म् धत्तो धत्तो वी॒र्य॑ मिन्द्रि॒य मि॑न्द्रि॒यं ॅवी॒र्य॑म् धत्तः । \newline
18. वी॒र्य॑म् धत्तो धत्तो वी॒र्यं॑ ॅवी॒र्य॑म् धत्तः स॒ह स॒ह ध॑त्तो वी॒र्यं॑ ॅवी॒र्य॑म् धत्तः स॒ह । \newline
19. ध॒त्तः॒ स॒ह स॒ह ध॑त्तो धत्तः स॒हे न्द्रि॒येणे᳚ न्द्रि॒येण॑ स॒ह ध॑त्तो धत्तः स॒हे न्द्रि॒येण॑ । \newline
20. स॒हे न्द्रि॒येणे᳚ न्द्रि॒येण॑ स॒ह स॒हे न्द्रि॒येण॑ वी॒र्ये॑ण वी॒र्ये॑णे न्द्रि॒येण॑ स॒ह स॒हे न्द्रि॒येण॑ वी॒र्ये॑ण । \newline
21. इ॒न्द्रि॒येण॑ वी॒र्ये॑ण वी॒र्ये॑णे न्द्रि॒येणे᳚ न्द्रि॒येण॑ वी॒र्ये॑णोपोप॑ वी॒र्ये॑णे न्द्रि॒येणे᳚ न्द्रि॒येण॑ वी॒र्ये॑णोप॑ । \newline
22. वी॒र्ये॑णोपोप॑ वी॒र्ये॑ण वी॒र्ये॑णोप॒ प्र प्रोप॑ वी॒र्ये॑ण वी॒र्ये॑णोप॒ प्र । \newline
23. उप॒ प्र प्रोपोप॒ प्र या॑ति याति॒ प्रोपोप॒ प्र या॑ति । \newline
24. प्र या॑ति याति॒ प्र प्र या॑ति॒ जय॑ति॒ जय॑ति याति॒ प्र प्र या॑ति॒ जय॑ति । \newline
25. या॒ति॒ जय॑ति॒ जय॑ति याति याति॒ जय॑ति॒ तम् तम् जय॑ति याति याति॒ जय॑ति॒ तम् । \newline
26. जय॑ति॒ तम् तम् जय॑ति॒ जय॑ति॒ तꣳ स॑ङ्ग्रा॒मꣳ स॑ङ्ग्रा॒मम् तम् जय॑ति॒ जय॑ति॒ तꣳ स॑ङ्ग्रा॒मम् । \newline
27. तꣳ स॑ङ्ग्रा॒मꣳ स॑ङ्ग्रा॒मम् तम् तꣳ स॑ङ्ग्रा॒मं ॅवि वि स॑ङ्ग्रा॒मम् तम् तꣳ स॑ङ्ग्रा॒मं ॅवि । \newline
28. स॒ङ्ग्रा॒मं ॅवि वि स॑ङ्ग्रा॒मꣳ स॑ङ्ग्रा॒मं ॅवि वै वै वि स॑ङ्ग्रा॒मꣳ स॑ङ्ग्रा॒मं ॅवि वै । \newline
29. स॒ङ्ग्रा॒ममिति॑ सं - ग्रा॒मम् । \newline
30. वि वै वै वि वि वा ए॒ष ए॒ष वै वि वि वा ए॒षः । \newline
31. वा ए॒ष ए॒ष वै वा ए॒ष इ॑न्द्रि॒येणे᳚ न्द्रि॒येणै॒ष वै वा ए॒ष इ॑न्द्रि॒येण॑ । \newline
32. ए॒ष इ॑न्द्रि॒येणे᳚ न्द्रि॒येणै॒ष ए॒ष इ॑न्द्रि॒येण॑ वी॒र्ये॑ण वी॒र्ये॑णे न्द्रि॒येणै॒ष ए॒ष इ॑न्द्रि॒येण॑ वी॒र्ये॑ण । \newline
33. इ॒न्द्रि॒येण॑ वी॒र्ये॑ण वी॒र्ये॑णे न्द्रि॒येणे᳚ न्द्रि॒येण॑ वी॒र्ये॑ण र्द्ध्यत ऋद्ध्यते वी॒र्ये॑णे न्द्रि॒येणे᳚ न्द्रि॒येण॑ वी॒र्ये॑ण र्द्ध्यते । \newline
34. वी॒र्ये॑ण र्द्ध्यत ऋद्ध्यते वी॒र्ये॑ण वी॒र्ये॑ण र्द्ध्यते॒ यो य ऋ॑द्ध्यते वी॒र्ये॑ण वी॒र्ये॑ण र्द्ध्यते॒ यः । \newline
35. ऋ॒द्ध्य॒ते॒ यो य ऋ॑द्ध्यत ऋद्ध्यते॒ यः स॑ङ्ग्रा॒मꣳ स॑ङ्ग्रा॒मं ॅय ऋ॑द्ध्यत ऋद्ध्यते॒ यः स॑ङ्ग्रा॒मम् । \newline
36. यः स॑ङ्ग्रा॒मꣳ स॑ङ्ग्रा॒मं ॅयो यः स॑ङ्ग्रा॒मम् जय॑ति॒ जय॑ति सङ्ग्रा॒मं ॅयो यः स॑ङ्ग्रा॒मम् जय॑ति । \newline
37. स॒ङ्ग्रा॒मम् जय॑ति॒ जय॑ति सङ्ग्रा॒मꣳ स॑ङ्ग्रा॒मम् जय॑ त्यैन्द्रा॒ग्न मै᳚न्द्रा॒ग्नम् जय॑ति सङ्ग्रा॒मꣳ स॑ङ्ग्रा॒मम् जय॑ त्यैन्द्रा॒ग्नम् । \newline
38. स॒ङ्ग्रा॒ममिति॑ सं - ग्रा॒मम् । \newline
39. जय॑ त्यैन्द्रा॒ग्न मै᳚न्द्रा॒ग्नम् जय॑ति॒ जय॑ त्यैन्द्रा॒ग्न मेका॑दशकपाल॒ मेका॑दशकपाल मैन्द्रा॒ग्नम् जय॑ति॒ जय॑ त्यैन्द्रा॒ग्न मेका॑दशकपालम् । \newline
40. ऐ॒न्द्रा॒ग्न मेका॑दशकपाल॒ मेका॑दशकपाल मैन्द्रा॒ग्न मै᳚न्द्रा॒ग्न मेका॑दशकपाल॒म् निर् णिरेका॑दशकपाल मैन्द्रा॒ग्न मै᳚न्द्रा॒ग्न मेका॑दशकपाल॒म् निः । \newline
41. ऐ॒न्द्रा॒ग्नमित्यै᳚न्द्र - अ॒ग्नम् । \newline
42. एका॑दशकपाल॒म् निर् णिरेका॑दशकपाल॒ मेका॑दशकपाल॒म् निर् व॑पेद् वपे॒न् निरेका॑दशकपाल॒ मेका॑दशकपाल॒म् निर् व॑पेत् । \newline
43. एका॑दशकपाल॒मित्येका॑दश - क॒पा॒ल॒म् । \newline
44. निर् व॑पेद् वपे॒न् निर् णिर् व॑पेथ् सङ्ग्रा॒मꣳ स॑ङ्ग्रा॒मं ॅव॑पे॒न् निर् णिर् व॑पेथ् सङ्ग्रा॒मम् । \newline
45. व॒पे॒थ् स॒ङ्ग्रा॒मꣳ स॑ङ्ग्रा॒मं ॅव॑पेद् वपेथ् सङ्ग्रा॒मम् जि॒त्वा जि॒त्वा स॑ङ्ग्रा॒मं ॅव॑पेद् वपेथ् सङ्ग्रा॒मम् जि॒त्वा । \newline
46. स॒ङ्ग्रा॒मम् जि॒त्वा जि॒त्वा स॑ङ्ग्रा॒मꣳ स॑ङ्ग्रा॒मम् जि॒त्वेन्द्रा॒ग्नी इ॑न्द्रा॒ग्नी जि॒त्वा स॑ङ्ग्रा॒मꣳ स॑ङ्ग्रा॒मम् जि॒त्वेन्द्रा॒ग्नी । \newline
47. स॒ङ्ग्रा॒ममिति॑ सं - ग्रा॒मम् । \newline
48. जि॒त्वेन्द्रा॒ग्नी इ॑न्द्रा॒ग्नी जि॒त्वा जि॒त्वेन्द्रा॒ग्नी ए॒वैवे न्द्रा॒ग्नी जि॒त्वा जि॒त्वेन्द्रा॒ग्नी ए॒व । \newline
49. इ॒न्द्रा॒ग्नी ए॒वैवे न्द्रा॒ग्नी इ॑न्द्रा॒ग्नी ए॒व स्वेन॒ स्वेनै॒वे न्द्रा॒ग्नी इ॑न्द्रा॒ग्नी ए॒व स्वेन॑ । \newline
50. इ॒न्द्रा॒ग्नी इती᳚न्द्र - अ॒ग्नी । \newline
51. ए॒व स्वेन॒ स्वेनै॒वैव स्वेन॑ भाग॒धेये॑न भाग॒धेये॑न॒ स्वेनै॒वैव स्वेन॑ भाग॒धेये॑न । \newline
52. स्वेन॑ भाग॒धेये॑न भाग॒धेये॑न॒ स्वेन॒ स्वेन॑ भाग॒धेये॒नोपोप॑ भाग॒धेये॑न॒ स्वेन॒ स्वेन॑ भाग॒धेये॒नोप॑ । \newline
53. भा॒ग॒धेये॒नोपोप॑ भाग॒धेये॑न भाग॒धेये॒नोप॑ धावति धाव॒त्युप॑ भाग॒धेये॑न भाग॒धेये॒नोप॑ धावति । \newline
54. भा॒ग॒धेये॒नेति॑ भाग - धेये॑न । \newline
55. उप॑ धावति धाव॒ त्युपोप॑ धावति॒ तौ तौ धा॑व॒ त्युपोप॑ धावति॒ तौ । \newline
56. धा॒व॒ति॒ तौ तौ धा॑वति धावति॒ ता वे॒वैव तौ धा॑वति धावति॒ ता वे॒व । \newline
57. ता वे॒वैव तौ ता वे॒वास्मि॑न् नस्मिन् ने॒व तौ ता वे॒वास्मिन्न्॑ । \newline
58. ए॒वास्मि॑न् नस्मिन् ने॒वैवास्मि॑न् निन्द्रि॒य मि॑न्द्रि॒य म॑स्मिन् ने॒वैवास्मि॑न् निन्द्रि॒यम् । \newline
59. अ॒स्मि॒न् नि॒न्द्रि॒य मि॑न्द्रि॒य म॑स्मिन् नस्मिन् निन्द्रि॒यं ॅवी॒र्यं॑ ॅवी॒र्य॑ मिन्द्रि॒य म॑स्मिन् नस्मिन् निन्द्रि॒यं ॅवी॒र्य᳚म् । \newline
60. इ॒न्द्रि॒यं ॅवी॒र्यं॑ ॅवी॒र्य॑ मिन्द्रि॒य मि॑न्द्रि॒यं ॅवी॒र्य॑म् धत्तो धत्तो वी॒र्य॑ मिन्द्रि॒य मि॑न्द्रि॒यं ॅवी॒र्य॑म् धत्तः । \newline
61. वी॒र्य॑म् धत्तो धत्तो वी॒र्यं॑ ॅवी॒र्य॑म् धत्तो॒ न न ध॑त्तो वी॒र्यं॑ ॅवी॒र्य॑म् धत्तो॒ न । \newline
\pagebreak
\markright{ TS 2.2.1.4  \hfill https://www.vedavms.in \hfill}
\addcontentsline{toc}{section}{ TS 2.2.1.4 }
\section*{ TS 2.2.1.4 }

\textbf{TS 2.2.1.4 } \newline
\textbf{Samhita Paata} \newline

धत्तो॒ नेन्द्रि॒येण॑ वी॒ये॑र्ण॒ व्यृ॑द्ध्य॒तेऽप॒ वा ए॒तस्मा॑दिन्द्रि॒यं ॅवी॒र्यं॑ क्रामति॒ य एति॑ ज॒नता॑मैन्द्रा॒ग्न-मेका॑दशकपालं॒ निर्व॑पे-ज्ज॒नता॑मे॒ष्यन्नि॑न्द्रा॒ग्नी ए॒व स्वेन॑ भाग॒धेये॒नोप॑ धावति॒ तावे॒वास्मि॑न्निन्द्रि॒यं ॅवी॒र्यं॑ धत्तः स॒हेन्द्रि॒येण॑ वी॒र्ये॑ण ज॒नता॑मेति पौ॒ष्णं च॒रुमनु॒ निर्व॑पेत् पू॒षा वा इ॑न्द्रि॒यस्य॑ वी॒र्य॑स्यानुप्रदा॒ता पू॒षण॑मे॒व - [  ] \newline

\textbf{Pada Paata} \newline

ध॒त्तः॒ । न । इ॒न्द्रि॒येण॑ । वी॒र्ये॑ण । वीति॑ । ऋ॒द्ध्य॒ते॒ । अपेति॑ । वै । ए॒तस्मा᳚त् । इ॒न्द्रि॒यम् । वी॒र्य᳚म् । क्रा॒म॒ति॒ । यः । एति॑ । ज॒नता᳚म् । ऐ॒न्द्रा॒ग्नमित्यै᳚न्द्र - अ॒ग्नम् । एका॑दशकपाल॒मित्येका॑दश-क॒पा॒ल॒म् । निरिति॑ । व॒पे॒त् । ज॒नता᳚म् । ए॒ष्यन्न् । इ॒न्द्रा॒ग्नी इती᳚न्द्र-अ॒ग्नी । ए॒व । स्वेन॑ । भा॒ग॒धेये॒नेति॑ भाग - धेये॑न । उपेति॑ । धा॒व॒ति॒ । तौ । ए॒व । अ॒स्मि॒न्न् । इ॒न्द्रि॒यम् । वी॒र्य᳚म् । ध॒त्तः॒ । स॒ह । इ॒न्द्रि॒येण॑ । वी॒र्ये॑ण । ज॒नता᳚म् । ए॒ति॒ । पौ॒ष्णम् । च॒रुम् । अनु॑ । निरिति॑ । व॒पे॒त् । पू॒षा । वै । इ॒न्द्रि॒यस्य॑ । वी॒र्य॑स्य । अ॒नु॒प्र॒दा॒तेत्य॑नु - प्र॒दा॒ता । पू॒षण᳚म् । ए॒व ।  \newline


\textbf{Krama Paata} \newline

ध॒त्तो॒ न । नेन्द्रि॒येण॑ । इ॒न्द्रि॒येण॑ वी॒र्ये॑ण । वी॒र्ये॑ण॒ वि । व्यृ॑ध्यते । ऋ॒ध्य॒तेऽप॑ । अप॒ वै । वा ए॒तस्मा᳚त् । ए॒तस्मा॑दिन्द्रि॒यम् । इ॒न्द्रि॒यं ॅवी॒र्य᳚म् । वी॒र्य॑म् क्रामति । क्रा॒म॒ति॒ यः । य एति॑ । एति॑ ज॒नता᳚म् । ज॒नता॑मैन्द्रा॒ग्नम् । ऐ॒न्द्रा॒ग्नमेका॑दशकपालम् । ऐ॒न्द्रा॒ग्नमित्यै᳚न्द्र - अ॒ग्नम् । एका॑दशकपाल॒म् निः । एका॑दशकपाल॒मित्येका॑दश - क॒पा॒ल॒म् । निर् व॑पेत् । व॒पे॒ज्ज॒नता᳚म् । ज॒नता॑मे॒ष्यन्न् । ए॒ष्यन्नि॑न्द्रा॒ग्नी । इ॒न्द्रा॒ग्नी ए॒व । इ॒न्द्रा॒ग्नी इती᳚न्द्र - अ॒ग्नी । ए॒व स्वेन॑ । स्वेन॑ भाग॒धेये॑न । भा॒ग॒धेये॒नोप॑ । भा॒ग॒धेये॒नेति॑ भाग - धेये॑न । उप॑ धावति । धा॒व॒ति॒ तौ । तावे॒व । ए॒वास्मिन्न्॑ । अ॒स्मि॒न्नि॒न्द्रि॒यम् । इ॒न्द्रि॒यं ॅवी॒र्य᳚म् । वी॒र्यं॑ धत्तः । ध॒त्तः॒ स॒ह । स॒हेन्द्रि॒येण॑ । इ॒न्द्रि॒येण॑ वी॒र्ये॑ण । वी॒र्ये॑ण ज॒नता᳚म् । ज॒नता॑मेति । ए॒ति॒ पौ॒ष्णम् । पौ॒ष्णम् च॒रुम् । च॒रुमनु॑ । अनु॒ निः । निर् व॑पेत् । व॒पे॒त् पू॒षा । पू॒षा वै । वा इ॑न्द्रि॒यस्य॑ । इ॒न्द्रि॒यस्य॑ वी॒र्य॑स्य । वी॒र्य॑स्यानुप्रदा॒ता । अ॒नु॒प्र॒दा॒ता पू॒षण᳚म् । अ॒नु॒प्र॒दा॒तेत्य॑नु - प्र॒दा॒ता । पू॒षण॑मे॒व ( ) । ए॒व स्वेन॑ \newline

\textbf{Jatai Paata} \newline

1. ध॒त्तो॒ न न ध॑त्तो धत्तो॒ न । \newline
2. ने न्द्रि॒येणे᳚ न्द्रि॒येण॒ न ने न्द्रि॒येण॑ । \newline
3. इ॒न्द्रि॒येण॑ वी॒र्ये॑ण वी॒र्ये॑णे न्द्रि॒येणे᳚ न्द्रि॒येण॑ वी॒र्ये॑ण । \newline
4. वी॒र्ये॑ण॒ वि वि वी॒र्ये॑ण वी॒र्ये॑ण॒ वि । \newline
5. व्यृ॑द्ध्यत ऋद्ध्यते॒ वि व्यृ॑द्ध्यते । \newline
6. ऋ॒द्ध्य॒ते ऽपाप॑ र्द्ध्यत ऋद्ध्य॒ते ऽप॑ । \newline
7. अप॒ वै वा अपाप॒ वै । \newline
8. वा ए॒तस्मा॑ दे॒तस्मा॒द् वै वा ए॒तस्मा᳚त् । \newline
9. ए॒तस्मा॑ दिन्द्रि॒य मि॑न्द्रि॒य मे॒तस्मा॑ दे॒तस्मा॑ दिन्द्रि॒यम् । \newline
10. इ॒न्द्रि॒यं ॅवी॒र्यं॑ ॅवी॒र्य॑ मिन्द्रि॒य मि॑न्द्रि॒यं ॅवी॒र्य᳚म् । \newline
11. वी॒र्य॑म् क्रामति क्रामति वी॒र्यं॑ ॅवी॒र्य॑म् क्रामति । \newline
12. क्रा॒म॒ति॒ यो यः क्रा॑मति क्रामति॒ यः । \newline
13. य एत्येति॒ यो य एति॑ । \newline
14. एति॑ ज॒नता᳚म् ज॒नता॒ मेत्येति॑ ज॒नता᳚म् । \newline
15. ज॒नता॑ मैन्द्रा॒ग्न मै᳚न्द्रा॒ग्नम् ज॒नता᳚म् ज॒नता॑ मैन्द्रा॒ग्नम् । \newline
16. ऐ॒न्द्रा॒ग्न मेका॑दशकपाल॒ मेका॑दशकपाल मैन्द्रा॒ग्न मै᳚न्द्रा॒ग्न मेका॑दशकपालम् । \newline
17. ऐ॒न्द्रा॒ग्नमित्यै᳚न्द्र - अ॒ग्नम् । \newline
18. एका॑दशकपाल॒म् निर् णिरेका॑दशकपाल॒ मेका॑दशकपाल॒म् निः । \newline
19. एका॑दशकपाल॒मित्येका॑दश - क॒पा॒ल॒म् । \newline
20. निर् व॑पेद् वपे॒न् निर् णिर् व॑पेत् । \newline
21. व॒पे॒ज् ज॒नता᳚म् ज॒नतां᳚ ॅवपेद् वपेज् ज॒नता᳚म् । \newline
22. ज॒नता॑ मे॒ष्यन् ने॒ष्यन् ज॒नता᳚म् ज॒नता॑ मे॒ष्यन्न् । \newline
23. ए॒ष्यन् नि॑न्द्रा॒ग्नी इ॑न्द्रा॒ग्नी ए॒ष्यन् ने॒ष्यन् नि॑न्द्रा॒ग्नी । \newline
24. इ॒न्द्रा॒ग्नी ए॒वैवे न्द्रा॒ग्नी इ॑न्द्रा॒ग्नी ए॒व । \newline
25. इ॒न्द्रा॒ग्नी इती᳚न्द्र - अ॒ग्नी । \newline
26. ए॒व स्वेन॒ स्वे नै॒वैव स्वेन॑ । \newline
27. स्वेन॑ भाग॒धेये॑न भाग॒धेये॑न॒ स्वेन॒ स्वेन॑ भाग॒धेये॑न । \newline
28. भा॒ग॒धेये॒नोपोप॑ भाग॒धेये॑न भाग॒धेये॒नोप॑ । \newline
29. भा॒ग॒धेये॒नेति॑ भाग - धेये॑न । \newline
30. उप॑ धावति धाव॒ त्युपोप॑ धावति । \newline
31. धा॒व॒ति॒ तौ तौ धा॑वति धावति॒ तौ । \newline
32. ता वे॒ वैव तौ ता वे॒व । \newline
33. ए॒वास्मि॑न् नस्मिन् ने॒वैवास्मिन्न्॑ । \newline
34. अ॒स्मि॒न् नि॒न्द्रि॒य मि॑न्द्रि॒य म॑स्मिन् नस्मिन् निन्द्रि॒यम् । \newline
35. इ॒न्द्रि॒यं ॅवी॒र्यं॑ ॅवी॒र्य॑ मिन्द्रि॒य मि॑न्द्रि॒यं ॅवी॒र्य᳚म् । \newline
36. वी॒र्य॑म् धत्तो धत्तो वी॒र्यं॑ ॅवी॒र्य॑म् धत्तः । \newline
37. ध॒त्तः॒ स॒ह स॒ह ध॑त्तो धत्तः स॒ह । \newline
38. स॒हे न्द्रि॒येणे᳚ न्द्रि॒येण॑ स॒ह स॒हे न्द्रि॒येण॑ । \newline
39. इ॒न्द्रि॒येण॑ वी॒र्ये॑ण वी॒र्ये॑णे न्द्रि॒येणे᳚ न्द्रि॒येण॑ वी॒र्ये॑ण । \newline
40. वी॒र्ये॑ण ज॒नता᳚म् ज॒नतां᳚ ॅवी॒र्ये॑ण वी॒र्ये॑ण ज॒नता᳚म् । \newline
41. ज॒नता॑ मेत्येति ज॒नता᳚म् ज॒नता॑ मेति । \newline
42. ए॒ति॒ पौ॒ष्णम् पौ॒ष्ण मे᳚त्येति पौ॒ष्णम् । \newline
43. पौ॒ष्णम् च॒रुम् च॒रुम् पौ॒ष्णम् पौ॒ष्णम् च॒रुम् । \newline
44. च॒रु मन्वनु॑ च॒रुम् च॒रु मनु॑ । \newline
45. अनु॒ निर् णि रन्वनु॒ निः । \newline
46. निर् व॑पेद् वपे॒न् निर् णिर् व॑पेत् । \newline
47. व॒पे॒त् पू॒षा पू॒षा व॑पेद् वपेत् पू॒षा । \newline
48. पू॒षा वै वै पू॒षा पू॒षा वै । \newline
49. वा इ॑न्द्रि॒यस्ये᳚ न्द्रि॒यस्य॒ वै वा इ॑न्द्रि॒यस्य॑ । \newline
50. इ॒न्द्रि॒यस्य॑ वी॒र्य॑स्य वी॒र्य॑स्ये न्द्रि॒यस्ये᳚ न्द्रि॒यस्य॑ वी॒र्य॑स्य । \newline
51. वी॒र्य॑स्या नुप्रदा॒ता ऽनु॑प्रदा॒ता वी॒र्य॑स्य वी॒र्य॑स्या नुप्रदा॒ता । \newline
52. अ॒नु॒प्र॒दा॒ता पू॒षण॑म् पू॒षण॑ मनुप्रदा॒ता ऽनु॑प्रदा॒ता पू॒षण᳚म् । \newline
53. अ॒नु॒प्र॒दा॒तेत्य॑नु - प्र॒दा॒ता । \newline
54. पू॒षण॑ मे॒वैव पू॒षण॑म् पू॒षण॑ मे॒व । \newline
55. ए॒व स्वेन॒ स्वे नै॒वैव स्वेन॑ । \newline

\textbf{Ghana Paata } \newline

1. ध॒त्तो॒ न न ध॑त्तो धत्तो॒ ने न्द्रि॒येणे᳚ न्द्रि॒येण॒ न ध॑त्तो धत्तो॒ ने न्द्रि॒येण॑ । \newline
2. ने न्द्रि॒येणे᳚ न्द्रि॒येण॒ न ने न्द्रि॒येण॑ वी॒र्ये॑ण वी॒र्ये॑णे न्द्रि॒येण॒ न ने न्द्रि॒येण॑ वी॒र्ये॑ण । \newline
3. इ॒न्द्रि॒येण॑ वी॒र्ये॑ण वी॒र्ये॑णे न्द्रि॒येणे᳚ न्द्रि॒येण॑ वी॒र्ये॑ण॒ वि वि वी॒र्ये॑णे न्द्रि॒येणे᳚ न्द्रि॒येण॑ वी॒र्ये॑ण॒ वि । \newline
4. वी॒र्ये॑ण॒ वि वि वी॒र्ये॑ण वी॒र्ये॑ण॒ व्यृ॑द्ध्यत ऋद्ध्यते॒ वि वी॒र्ये॑ण वी॒र्ये॑ण॒ व्यृ॑द्ध्यते । \newline
5. व्यृ॑द्ध्यत ऋद्ध्यते॒ वि व्यृ॑द्ध्य॒ते ऽपाप॑ र्द्ध्यते॒ वि व्यृ॑द्ध्य॒ते ऽप॑ । \newline
6. ऋ॒द्ध्य॒ते ऽपाप॑ र्द्ध्यत ऋद्ध्य॒ते ऽप॒ वै वा अप॑ र्द्ध्यत ऋद्ध्य॒ते ऽप॒ वै । \newline
7. अप॒ वै वा अपाप॒ वा ए॒तस्मा॑ दे॒तस्मा॒द् वा अपाप॒ वा ए॒तस्मा᳚त् । \newline
8. वा ए॒तस्मा॑ दे॒तस्मा॒द् वै वा ए॒तस्मा॑ दिन्द्रि॒य मि॑न्द्रि॒य मे॒तस्मा॒द् वै वा ए॒तस्मा॑ दिन्द्रि॒यम् । \newline
9. ए॒तस्मा॑ दिन्द्रि॒य मि॑न्द्रि॒य मे॒तस्मा॑ दे॒तस्मा॑ दिन्द्रि॒यं ॅवी॒र्यं॑ ॅवी॒र्य॑ मिन्द्रि॒य मे॒तस्मा॑ दे॒तस्मा॑ दिन्द्रि॒यं ॅवी॒र्य᳚म् । \newline
10. इ॒न्द्रि॒यं ॅवी॒र्यं॑ ॅवी॒र्य॑ मिन्द्रि॒य मि॑न्द्रि॒यं ॅवी॒र्य॑म् क्रामति क्रामति वी॒र्य॑ मिन्द्रि॒य मि॑न्द्रि॒यं ॅवी॒र्य॑म् क्रामति । \newline
11. वी॒र्य॑म् क्रामति क्रामति वी॒र्यं॑ ॅवी॒र्य॑म् क्रामति॒ यो यः क्रा॑मति वी॒र्यं॑ ॅवी॒र्य॑म् क्रामति॒ यः । \newline
12. क्रा॒म॒ति॒ यो यः क्रा॑मति क्रामति॒ य एत्येति॒ यः क्रा॑मति क्रामति॒ य एति॑ । \newline
13. य एत्येति॒ यो य एति॑ ज॒नता᳚म् ज॒नता॒ मेति॒ यो य एति॑ ज॒नता᳚म् । \newline
14. एति॑ ज॒नता᳚म् ज॒नता॒ मेत्येति॑ ज॒नता॑ मैन्द्रा॒ग्न मै᳚न्द्रा॒ग्नम् ज॒नता॒ मेत्येति॑ ज॒नता॑ मैन्द्रा॒ग्नम् । \newline
15. ज॒नता॑ मैन्द्रा॒ग्न मै᳚न्द्रा॒ग्नम् ज॒नता᳚म् ज॒नता॑ मैन्द्रा॒ग्न मेका॑दशकपाल॒ मेका॑दशकपाल मैन्द्रा॒ग्नम् ज॒नता᳚म् ज॒नता॑ मैन्द्रा॒ग्न मेका॑दशकपालम् । \newline
16. ऐ॒न्द्रा॒ग्न मेका॑दशकपाल॒ मेका॑दशकपाल मैन्द्रा॒ग्न मै᳚न्द्रा॒ग्न मेका॑दशकपाल॒म् निर् णिरेका॑दशकपाल मैन्द्रा॒ग्न मै᳚न्द्रा॒ग्न मेका॑दशकपाल॒म् निः । \newline
17. ऐ॒न्द्रा॒ग्नमित्यै᳚न्द्र - अ॒ग्नम् । \newline
18. एका॑दशकपाल॒म् निर् णिरेका॑दशकपाल॒ मेका॑दशकपाल॒म् निर् व॑पेद् वपे॒न् निरेका॑दशकपाल॒ मेका॑दशकपाल॒म् निर् व॑पेत् । \newline
19. एका॑दशकपाल॒मित्येका॑दश - क॒पा॒ल॒म् । \newline
20. निर् व॑पेद् वपे॒न् निर् णिर् व॑पेज् ज॒नता᳚म् ज॒नतां᳚ ॅवपे॒न् निर् णिर् व॑पेज् ज॒नता᳚म् । \newline
21. व॒पे॒ज् ज॒नता᳚म् ज॒नतां᳚ ॅवपेद् वपेज् ज॒नता॑ मे॒ष्यन् ने॒ष्यन् ज॒नतां᳚ ॅवपेद् वपेज् ज॒नता॑ मे॒ष्यन्न् । \newline
22. ज॒नता॑ मे॒ष्यन् ने॒ष्यन् ज॒नता᳚म् ज॒नता॑ मे॒ष्यन् नि॑न्द्रा॒ग्नी इ॑न्द्रा॒ग्नी ए॒ष्यन् ज॒नता᳚म् ज॒नता॑ मे॒ष्यन् नि॑न्द्रा॒ग्नी । \newline
23. ए॒ष्यन् नि॑न्द्रा॒ग्नी इ॑न्द्रा॒ग्नी ए॒ष्यन् ने॒ष्यन् नि॑न्द्रा॒ग्नी ए॒वैवे न्द्रा॒ग्नी ए॒ष्यन् ने॒ष्यन् नि॑न्द्रा॒ग्नी ए॒व । \newline
24. इ॒न्द्रा॒ग्नी ए॒वैवे न्द्रा॒ग्नी इ॑न्द्रा॒ग्नी ए॒व स्वेन॒ स्वेनै॒वे न्द्रा॒ग्नी इ॑न्द्रा॒ग्नी ए॒व स्वेन॑ । \newline
25. इ॒न्द्रा॒ग्नी इती᳚न्द्र - अ॒ग्नी । \newline
26. ए॒व स्वेन॒ स्वेनै॒वैव स्वेन॑ भाग॒धेये॑न भाग॒धेये॑न॒ स्वेनै॒वैव स्वेन॑ भाग॒धेये॑न । \newline
27. स्वेन॑ भाग॒धेये॑न भाग॒धेये॑न॒ स्वेन॒ स्वेन॑ भाग॒धेये॒नोपोप॑ भाग॒धेये॑न॒ स्वेन॒ स्वेन॑ भाग॒धेये॒नोप॑ । \newline
28. भा॒ग॒धेये॒नोपोप॑ भाग॒धेये॑न भाग॒धेये॒नोप॑ धावति धाव॒त्युप॑ भाग॒धेये॑न भाग॒धेये॒नोप॑ धावति । \newline
29. भा॒ग॒धेये॒नेति॑ भाग - धेये॑न । \newline
30. उप॑ धावति धाव॒ त्युपोप॑ धावति॒ तौ तौ धा॑व॒ त्युपोप॑ धावति॒ तौ । \newline
31. धा॒व॒ति॒ तौ तौ धा॑वति धावति॒ ता वे॒वैव तौ धा॑वति धावति॒ ता वे॒व । \newline
32. ता वे॒वैव तौ ता वे॒वास्मि॑न् नस्मिन् ने॒व तौ ता वे॒वास्मिन्न्॑ । \newline
33. ए॒वास्मि॑न् नस्मिन् ने॒वैवास्मि॑न् निन्द्रि॒य मि॑न्द्रि॒य म॑स्मिन् ने॒वैवास्मि॑न् निन्द्रि॒यम् । \newline
34. अ॒स्मि॒न् नि॒न्द्रि॒य मि॑न्द्रि॒य म॑स्मिन् नस्मिन् निन्द्रि॒यं ॅवी॒र्यं॑ ॅवी॒र्य॑ मिन्द्रि॒य म॑स्मिन् नस्मिन् निन्द्रि॒यं ॅवी॒र्य᳚म् । \newline
35. इ॒न्द्रि॒यं ॅवी॒र्यं॑ ॅवी॒र्य॑ मिन्द्रि॒य मि॑न्द्रि॒यं ॅवी॒र्य॑म् धत्तो धत्तो वी॒र्य॑ मिन्द्रि॒य मि॑न्द्रि॒यं ॅवी॒र्य॑म् धत्तः । \newline
36. वी॒र्य॑म् धत्तो धत्तो वी॒र्यं॑ ॅवी॒र्य॑म् धत्तः स॒ह स॒ह ध॑त्तो वी॒र्यं॑ ॅवी॒र्य॑म् धत्तः स॒ह । \newline
37. ध॒त्तः॒ स॒ह स॒ह ध॑त्तो धत्तः स॒हे न्द्रि॒येणे᳚ न्द्रि॒येण॑ स॒ह ध॑त्तो धत्तः स॒हे न्द्रि॒येण॑ । \newline
38. स॒हे न्द्रि॒येणे᳚ न्द्रि॒येण॑ स॒ह स॒हे न्द्रि॒येण॑ वी॒र्ये॑ण वी॒र्ये॑णे न्द्रि॒येण॑ स॒ह स॒हे न्द्रि॒येण॑ वी॒र्ये॑ण । \newline
39. इ॒न्द्रि॒येण॑ वी॒र्ये॑ण वी॒र्ये॑णे न्द्रि॒येणे᳚ न्द्रि॒येण॑ वी॒र्ये॑ण ज॒नता᳚म् ज॒नतां᳚ ॅवी॒र्ये॑णे न्द्रि॒येणे᳚ न्द्रि॒येण॑ वी॒र्ये॑ण ज॒नता᳚म् । \newline
40. वी॒र्ये॑ण ज॒नता᳚म् ज॒नतां᳚ ॅवी॒र्ये॑ण वी॒र्ये॑ण ज॒नता॑ मेत्येति ज॒नतां᳚ ॅवी॒र्ये॑ण वी॒र्ये॑ण ज॒नता॑ मेति । \newline
41. ज॒नता॑ मेत्येति ज॒नता᳚म् ज॒नता॑ मेति पौ॒ष्णम् पौ॒ष्ण मे॑ति ज॒नता᳚म् ज॒नता॑ मेति पौ॒ष्णम् । \newline
42. ए॒ति॒ पौ॒ष्णम् पौ॒ष्ण मे᳚त्येति पौ॒ष्णम् च॒रुम् च॒रुम् पौ॒ष्ण मे᳚त्येति पौ॒ष्णम् च॒रुम् । \newline
43. पौ॒ष्णम् च॒रुम् च॒रुम् पौ॒ष्णम् पौ॒ष्णम् च॒रु मन्वनु॑ च॒रुम् पौ॒ष्णम् पौ॒ष्णम् च॒रु मनु॑ । \newline
44. च॒रु मन्वनु॑ च॒रुम् च॒रु मनु॒ निर् णिरनु॑ च॒रुम् च॒रु मनु॒ निः । \newline
45. अनु॒ निर् णिरन्वनु॒ निर् व॑पेद् वपे॒न् निरन्वनु॒ निर् व॑पेत् । \newline
46. निर् व॑पेद् वपे॒न् निर् णिर् व॑पेत् पू॒षा पू॒षा व॑पे॒न् निर् णिर् व॑पेत् पू॒षा । \newline
47. व॒पे॒त् पू॒षा पू॒षा व॑पेद् वपेत् पू॒षा वै वै पू॒षा व॑पेद् वपेत् पू॒षा वै । \newline
48. पू॒षा वै वै पू॒षा पू॒षा वा इ॑न्द्रि॒यस्ये᳚ न्द्रि॒यस्य॒ वै पू॒षा पू॒षा वा इ॑न्द्रि॒यस्य॑ । \newline
49. वा इ॑न्द्रि॒यस्ये᳚ न्द्रि॒यस्य॒ वै वा इ॑न्द्रि॒यस्य॑ वी॒र्य॑स्य वी॒र्य॑स्ये न्द्रि॒यस्य॒ वै वा इ॑न्द्रि॒यस्य॑ वी॒र्य॑स्य । \newline
50. इ॒न्द्रि॒यस्य॑ वी॒र्य॑स्य वी॒र्य॑स्ये न्द्रि॒यस्ये᳚ न्द्रि॒यस्य॑ वी॒र्य॑स्या नुप्रदा॒ता ऽनु॑प्रदा॒ता वी॒र्य॑स्ये न्द्रि॒यस्ये᳚ न्द्रि॒यस्य॑ वी॒र्य॑स्या नुप्रदा॒ता । \newline
51. वी॒र्य॑स्या नुप्रदा॒ता ऽनु॑प्रदा॒ता वी॒र्य॑स्य वी॒र्य॑स्या नुप्रदा॒ता पू॒षण॑म् पू॒षण॑ मनुप्रदा॒ता वी॒र्य॑स्य वी॒र्य॑स्या नुप्रदा॒ता पू॒षण᳚म् । \newline
52. अ॒नु॒प्र॒दा॒ता पू॒षण॑म् पू॒षण॑ मनुप्रदा॒ता ऽनु॑प्रदा॒ता पू॒षण॑ मे॒वैव पू॒षण॑ मनुप्रदा॒ता ऽनु॑प्रदा॒ता पू॒षण॑ मे॒व । \newline
53. अ॒नु॒प्र॒दा॒तेत्य॑नु - प्र॒दा॒ता । \newline
54. पू॒षण॑ मे॒वैव पू॒षण॑म् पू॒षण॑ मे॒व स्वेन॒ स्वेनै॒व पू॒षण॑म् पू॒षण॑ मे॒व स्वेन॑ । \newline
55. ए॒व स्वेन॒ स्वेनै॒वैव स्वेन॑ भाग॒धेये॑न भाग॒धेये॑न॒ स्वेनै॒वैव स्वेन॑ भाग॒धेये॑न । \newline
\pagebreak
\markright{ TS 2.2.1.5  \hfill https://www.vedavms.in \hfill}
\addcontentsline{toc}{section}{ TS 2.2.1.5 }
\section*{ TS 2.2.1.5 }

\textbf{TS 2.2.1.5 } \newline
\textbf{Samhita Paata} \newline

स्वेन॑ भाग॒धेये॒नोप॑ धावति॒ स ए॒वास्मा॑ इन्द्रि॒यं ॅवी॒र्य॑मनु॒ प्रय॑च्छति क्षैत्रप॒त्यं च॒रुं निर्व॑पे-ज्ज॒नता॑मा॒गत्ये॒यं ॅवै क्षेत्र॑स्य॒ पति॑र॒स्यामे॒व प्रति॑ तिष्ठत्यैन्द्रा॒ग्न-मेका॑दशकपाल-मु॒परि॑ष्टा॒-न्निर्व॑पेद॒स्यामे॒व प्र॑ति॒ष्ठाये᳚न्द्रि॒यं ॅवी॒र्य॑मु॒परि॑ष्टा-दा॒त्मन् ध॑त्ते ॥ \newline

\textbf{Pada Paata} \newline

स्वेन॑ । भा॒ग॒धेये॒नेति॑ भाग - धेये॑न । उपेति॑ । धा॒व॒ति॒ । सः । ए॒व । अ॒स्मै॒ । इ॒न्द्रि॒यम् । वी॒र्य᳚म् । अनु॑ । प्रेति॑ । य॒च्छ॒ति॒ । क्षै॒त्र॒प॒त्यमिति॑ क्षैत्र - प॒त्यम् । च॒रुम् । निरिति॑ । व॒पे॒त् । ज॒नता᳚म् । आ॒गत्येत्या᳚ - गत्य॑ । इ॒यम् । वै । क्षेत्र॑स्य । पतिः॑ । अ॒स्याम् । ए॒व । प्रतीति॑ । ति॒ष्ठ॒ति॒ । ऐ॒न्द्रा॒ग्नमित्यै᳚न्द्र - अ॒ग्नम् । एका॑दशकपाल॒मित्येका॑दश-क॒पा॒ल॒म् । उ॒परि॑ष्टात् । निरिति॑ । व॒पे॒त् । अ॒स्याम् । ए॒व । प्र॒ति॒ष्ठायेति॑ प्रति - स्थाय॑ । इ॒न्द्रि॒यम् । वी॒र्य᳚म् । उ॒परि॑ष्टात् । आ॒त्मन्न् । ध॒त्ते॒ ॥  \newline


\textbf{Krama Paata} \newline

स्वेन॑ भाग॒धेये॑न । भा॒ग॒धेये॒नोप॑ । भा॒ग॒धेये॒नेति॑ भाग - धेये॑न । उप॑ धावति । धा॒व॒ति॒ सः । स ए॒व । ए॒वास्मै᳚ । अ॒स्मा॒ इ॒न्द्रि॒यम् । इ॒न्द्रि॒यं ॅवी॒र्य᳚म् । वी॒र्य॑मनु॑ । अनु॒ प्र । प्र य॑च्छति । य॒च्छ॒ति॒ क्षै॒त्र॒प॒त्यम् । क्षै॒त्र॒प॒त्यम् च॒रुम् । क्षै॒त्र॒प॒त्यमिति॑ क्षैत्र - प॒त्यम् । च॒रुम् निः । निर् व॑पेत् । व॒पे॒ज्ज॒नता᳚म् । ज॒नता॑मा॒गत्य॑ । आ॒गत्ये॒यम् । आ॒गत्येत्या᳚ - गत्य॑ । इ॒यं ॅवै । वै क्षेत्र॑स्य । क्षेत्र॑स्य॒ पतिः॑ । पति॑र॒स्याम् । अ॒स्यामे॒व । ए॒व प्रति॑ । प्रति॑ तिष्ठति । ति॒ष्ठ॒त्यै॒न्द्रा॒ग्नम् । ऐ॒न्द्रा॒ग्नमेका॑दशकपालम् । ऐ॒न्द्रा॒ग्नमित्यै᳚न्द्र - अ॒ग्नम् । एका॑दशकपालमु॒परि॑ष्टात् । एका॑दशकपाल॒मित्येका॑दश - क॒पा॒ल॒म् । उ॒परि॑ष्टा॒न्निः । निर् व॑पेत् । व॒पे॒द॒स्याम् । अ॒स्यामे॒व । ए॒व प्र॑ति॒ष्ठाय॑ । प्र॒ति॒ष्ठाये᳚न्द्रि॒यम् । प्र॒ति॒ष्ठायेति॑ प्रति - स्थाय॑ । इ॒न्द्रि॒यं ॅवी॒र्य᳚म् । वी॒र्य॑मु॒परि॑ष्टात् । उ॒परि॑ष्टादा॒त्मन्न् । आ॒त्मन् ध॑त्ते । ध॒त्त॒ इति॑ धत्ते । \newline

\textbf{Jatai Paata} \newline

1. स्वेन॑ भाग॒धेये॑न भाग॒धेये॑न॒ स्वेन॒ स्वेन॑ भाग॒धेये॑न । \newline
2. भा॒ग॒धेये॒नोपोप॑ भाग॒धेये॑न भाग॒धेये॒नोप॑ । \newline
3. भा॒ग॒धेये॒नेति॑ भाग - धेये॑न । \newline
4. उप॑ धावति धाव॒ त्युपोप॑ धावति । \newline
5. धा॒व॒ति॒ स स धा॑वति धावति॒ सः । \newline
6. स ए॒वैव स स ए॒व । \newline
7. ए॒वास्मा॑ अस्मा ए॒वैवास्मै᳚ । \newline
8. अ॒स्मा॒ इ॒न्द्रि॒य मि॑न्द्रि॒य म॑स्मा अस्मा इन्द्रि॒यम् । \newline
9. इ॒न्द्रि॒यं ॅवी॒र्यं॑ ॅवी॒र्य॑ मिन्द्रि॒य मि॑न्द्रि॒यं ॅवी॒र्य᳚म् । \newline
10. वी॒र्य॑ मन्वनु॑ वी॒र्यं॑ ॅवी॒र्य॑ मनु॑ । \newline
11. अनु॒ प्र प्राण्वनु॒ प्र । \newline
12. प्र य॑च्छति यच्छति॒ प्र प्र य॑च्छति । \newline
13. य॒च्छ॒ति॒ क्षै॒त्र॒प॒त्यम् क्षै᳚त्रप॒त्यं ॅय॑च्छति यच्छति क्षैत्रप॒त्यम् । \newline
14. क्षै॒त्र॒प॒त्यम् च॒रुम् च॒रुम् क्षै᳚त्रप॒त्यम् क्षै᳚त्रप॒त्यम् च॒रुम् । \newline
15. क्षै॒त्र॒प॒त्यमिति॑ क्षैत्र - प॒त्यम् । \newline
16. च॒रुम् निर् णिश्च॒रुम् च॒रुम् निः । \newline
17. निर् व॑पेद् वपे॒न् निर् णिर् व॑पेत् । \newline
18. व॒पे॒ज् ज॒नता᳚म् ज॒नतां᳚ ॅवपेद् वपेज् ज॒नता᳚म् । \newline
19. ज॒नता॑ मा॒ग त्या॒गत्य॑ ज॒नता᳚म् ज॒नता॑ मा॒गत्य॑ । \newline
20. आ॒गत्ये॒ य मि॒य मा॒ग त्या॒गत्ये॒ यम् । \newline
21. आ॒गत्येत्या᳚ - गत्य॑ । \newline
22. इ॒यं ॅवै वा इ॒य मि॒यं ॅवै । \newline
23. वै क्षेत्र॑स्य॒ क्षेत्र॑स्य॒ वै वै क्षेत्र॑स्य । \newline
24. क्षेत्र॑स्य॒ पति॒ष् पतिः॒ क्षेत्र॑स्य॒ क्षेत्र॑स्य॒ पतिः॑ । \newline
25. पति॑ र॒स्या म॒स्याम् पति॒ष् पति॑ र॒स्याम् । \newline
26. अ॒स्या मे॒वैवास्या म॒स्या मे॒व । \newline
27. ए॒व प्रति॒ प्र त्ये॒वैव प्रति॑ । \newline
28. प्रति॑ तिष्ठति तिष्ठति॒ प्रति॒ प्रति॑ तिष्ठति । \newline
29. ति॒ष्ठ॒ त्यै॒न्द्रा॒ग्न मै᳚न्द्रा॒ग्नम् ति॑ष्ठति तिष्ठ त्यैन्द्रा॒ग्नम् । \newline
30. ऐ॒न्द्रा॒ग्न मेका॑दशकपाल॒ मेका॑दशकपाल मैन्द्रा॒ग्न मै᳚न्द्रा॒ग्न मेका॑दशकपालम् । \newline
31. ऐ॒न्द्रा॒ग्नमित्यै᳚न्द्र - अ॒ग्नम् । \newline
32. एका॑दशकपाल मु॒परि॑ष्टा दु॒परि॑ष्टा॒ देका॑दशकपाल॒ मेका॑दशकपाल मु॒परि॑ष्टात् । \newline
33. एका॑दशकपाल॒मित्येका॑दश - क॒पा॒ल॒म् । \newline
34. उ॒परि॑ष्टा॒न् निर् णि रु॒परि॑ष्टा दु॒परि॑ष्टा॒न् निः । \newline
35. निर् व॑पेद् वपे॒न् निर् णिर् व॑पेत् । \newline
36. व॒पे॒ द॒स्या म॒स्यां ॅव॑पेद् वपे द॒स्याम् । \newline
37. अ॒स्या मे॒वैवास्या म॒स्या मे॒व । \newline
38. ए॒व प्र॑ति॒ष्ठाय॑ प्रति॒ष्ठा यै॒वैव प्र॑ति॒ष्ठाय॑ । \newline
39. प्र॒ति॒ष्ठाये᳚ न्द्रि॒य मि॑न्द्रि॒यम् प्र॑ति॒ष्ठाय॑ प्रति॒ष्ठाये᳚ न्द्रि॒यम् । \newline
40. प्र॒ति॒ष्ठायेति॑ प्रति - स्थाय॑ । \newline
41. इ॒न्द्रि॒यं ॅवी॒र्यं॑ ॅवी॒र्य॑ मिन्द्रि॒य मि॑न्द्रि॒यं ॅवी॒र्य᳚म् । \newline
42. वी॒र्य॑ मु॒परि॑ष्टा दु॒परि॑ष्टाद् वी॒र्यं॑ ॅवी॒र्य॑ मु॒परि॑ष्टात् । \newline
43. उ॒परि॑ष्टा दा॒त्मन् ना॒त्मन् नु॒परि॑ष्टा दु॒परि॑ष्टा दा॒त्मन्न् । \newline
44. आ॒त्मन् ध॑त्ते धत्त आ॒त्मन् ना॒त्मन् ध॑त्ते । \newline
45. ध॒त्त॒ इति॑ धत्ते । \newline

\textbf{Ghana Paata } \newline

1. स्वेन॑ भाग॒धेये॑न भाग॒धेये॑न॒ स्वेन॒ स्वेन॑ भाग॒धेये॒नोपोप॑ भाग॒धेये॑न॒ स्वेन॒ स्वेन॑ भाग॒धेये॒नोप॑ । \newline
2. भा॒ग॒धेये॒नोपोप॑ भाग॒धेये॑न भाग॒धेये॒नोप॑ धावति धाव॒त्युप॑ भाग॒धेये॑न भाग॒धेये॒नोप॑ धावति । \newline
3. भा॒ग॒धेये॒नेति॑ भाग - धेये॑न । \newline
4. उप॑ धावति धाव॒ त्युपोप॑ धावति॒ स स धा॑व॒ त्युपोप॑ धावति॒ सः । \newline
5. धा॒व॒ति॒ स स धा॑वति धावति॒ स ए॒वैव स धा॑वति धावति॒ स ए॒व । \newline
6. स ए॒वैव स स ए॒वास्मा॑ अस्मा ए॒व स स ए॒वास्मै᳚ । \newline
7. ए॒वास्मा॑ अस्मा ए॒वैवास्मा॑ इन्द्रि॒य मि॑न्द्रि॒य म॑स्मा ए॒वैवास्मा॑ इन्द्रि॒यम् । \newline
8. अ॒स्मा॒ इ॒न्द्रि॒य मि॑न्द्रि॒य म॑स्मा अस्मा इन्द्रि॒यं ॅवी॒र्यं॑ ॅवी॒र्य॑ मिन्द्रि॒य म॑स्मा अस्मा इन्द्रि॒यं ॅवी॒र्य᳚म् । \newline
9. इ॒न्द्रि॒यं ॅवी॒र्यं॑ ॅवी॒र्य॑ मिन्द्रि॒य मि॑न्द्रि॒यं ॅवी॒र्य॑ मन्वनु॑ वी॒र्य॑ मिन्द्रि॒य मि॑न्द्रि॒यं ॅवी॒र्य॑ मनु॑ । \newline
10. वी॒र्य॑ मन्वनु॑ वी॒र्यं॑ ॅवी॒र्य॑ मनु॒ प्र प्राणु॑ वी॒र्यं॑ ॅवी॒र्य॑ मनु॒ प्र । \newline
11. अनु॒ प्र प्राण्वनु॒ प्र य॑च्छति यच्छति॒ प्राण्वनु॒ प्र य॑च्छति । \newline
12. प्र य॑च्छति यच्छति॒ प्र प्र य॑च्छति क्षैत्रप॒त्यम् क्षै᳚त्रप॒त्यं ॅय॑च्छति॒ प्र प्र य॑च्छति क्षैत्रप॒त्यम् । \newline
13. य॒च्छ॒ति॒ क्षै॒त्र॒प॒त्यम् क्षै᳚त्रप॒त्यं ॅय॑च्छति यच्छति क्षैत्रप॒त्यम् च॒रुम् च॒रुम् क्षै᳚त्रप॒त्यं ॅय॑च्छति यच्छति क्षैत्रप॒त्यम् च॒रुम् । \newline
14. क्षै॒त्र॒प॒त्यम् च॒रुम् च॒रुम् क्षै᳚त्रप॒त्यम् क्षै᳚त्रप॒त्यम् च॒रुम् निर् णिश्च॒रुम् क्षै᳚त्रप॒त्यम् क्षै᳚त्रप॒त्यम् च॒रुम् निः । \newline
15. क्षै॒त्र॒प॒त्यमिति॑ क्षैत्र - प॒त्यम् । \newline
16. च॒रुम् निर् णिश्च॒रुम् च॒रुम् निर् व॑पेद् वपे॒न् निश्च॒रुम् च॒रुम् निर् व॑पेत् । \newline
17. निर् व॑पेद् वपे॒न् निर् णिर् व॑पेज् ज॒नता᳚म् ज॒नतां᳚ ॅवपे॒न् निर् णिर् व॑पेज् ज॒नता᳚म् । \newline
18. व॒पे॒ज् ज॒नता᳚म् ज॒नतां᳚ ॅवपेद् वपेज् ज॒नता॑ मा॒गत्या॒गत्य॑ ज॒नतां᳚ ॅवपेद् वपेज् ज॒नता॑ मा॒गत्य॑ । \newline
19. ज॒नता॑ मा॒गत्या॒गत्य॑ ज॒नता᳚म् ज॒नता॑ मा॒गत्ये॒ य मि॒य मा॒गत्य॑ ज॒नता᳚म् ज॒नता॑ मा॒गत्ये॒ यम् । \newline
20. आ॒गत्ये॒ य मि॒य मा॒ग त्या॒गत्ये॒ यं ॅवै वा इ॒य मा॒ग त्या॒गत्ये॒ यं ॅवै । \newline
21. आ॒गत्येत्या᳚ - गत्य॑ । \newline
22. इ॒यं ॅवै वा इ॒य मि॒यं ॅवै क्षेत्र॑स्य॒ क्षेत्र॑स्य॒ वा इ॒य मि॒यं ॅवै क्षेत्र॑स्य । \newline
23. वै क्षेत्र॑स्य॒ क्षेत्र॑स्य॒ वै वै क्षेत्र॑स्य॒ पति॒ष् पतिः॒ क्षेत्र॑स्य॒ वै वै क्षेत्र॑स्य॒ पतिः॑ । \newline
24. क्षेत्र॑स्य॒ पति॒ष् पतिः॒ क्षेत्र॑स्य॒ क्षेत्र॑स्य॒ पति॑ र॒स्या म॒स्याम् पतिः॒ क्षेत्र॑स्य॒ क्षेत्र॑स्य॒ पति॑ र॒स्याम् । \newline
25. पति॑ र॒स्या म॒स्याम् पति॒ष् पति॑ र॒स्या मे॒वैवास्याम् पति॒ष् पति॑ र॒स्या मे॒व । \newline
26. अ॒स्या मे॒वैवास्या म॒स्या मे॒व प्रति॒ प्रत्ये॒वास्या म॒स्या मे॒व प्रति॑ । \newline
27. ए॒व प्रति॒ प्रत्ये॒वैव प्रति॑ तिष्ठति तिष्ठति॒ प्रत्ये॒वैव प्रति॑ तिष्ठति । \newline
28. प्रति॑ तिष्ठति तिष्ठति॒ प्रति॒ प्रति॑ तिष्ठ त्यैन्द्रा॒ग्न मै᳚न्द्रा॒ग्नम् ति॑ष्ठति॒ प्रति॒ प्रति॑ तिष्ठ त्यैन्द्रा॒ग्नम् । \newline
29. ति॒ष्ठ॒ त्यै॒न्द्रा॒ग्न मै᳚न्द्रा॒ग्नम् ति॑ष्ठति तिष्ठ त्यैन्द्रा॒ग्न मेका॑दशकपाल॒ मेका॑दशकपाल मैन्द्रा॒ग्नम् ति॑ष्ठति तिष्ठ त्यैन्द्रा॒ग्न मेका॑दशकपालम् । \newline
30. ऐ॒न्द्रा॒ग्न मेका॑दशकपाल॒ मेका॑दशकपाल मैन्द्रा॒ग्न मै᳚न्द्रा॒ग्न मेका॑दशकपाल मु॒परि॑ष्टा दु॒परि॑ष्टा॒ देका॑दशकपाल मैन्द्रा॒ग्न मै᳚न्द्रा॒ग्न मेका॑दशकपाल मु॒परि॑ष्टात् । \newline
31. ऐ॒न्द्रा॒ग्नमित्यै᳚न्द्र - अ॒ग्नम् । \newline
32. एका॑दशकपाल मु॒परि॑ष्टा दु॒परि॑ष्टा॒ देका॑दशकपाल॒ मेका॑दशकपाल 
मु॒परि॑ष्टा॒म् निर् णिरु॒परि॑ष्टा॒ देका॑दशकपाल॒ मेका॑दशकपाल मु॒परि॑ष्टा॒म् निः । \newline
33. एका॑दशकपाल॒मित्येका॑दश - क॒पा॒ल॒म् । \newline
34. उ॒परि॑ष्टा॒न् निर् णिरु॒परि॑ष्टा दु॒परि॑ष्टा॒न् निर् व॑पेद् वपे॒न् निरु॒परि॑ष्टा दु॒परि॑ष्टा॒न् निर् व॑पेत् । \newline
35. निर् व॑पेद् वपे॒न् निर् णिर् व॑पेद॒स्या म॒स्यां ॅव॑पे॒न् निर् णिर् व॑पेद॒स्याम् । \newline
36. व॒पे॒द॒स्या म॒स्यां ॅव॑पेद् वपेद॒स्या मे॒वैवास्यां ॅव॑पेद् वपेद॒स्या मे॒व । \newline
37. अ॒स्या मे॒वैवास्या म॒स्या मे॒व प्र॑ति॒ष्ठाय॑ प्रति॒ष्ठायै॒वास्या म॒स्या मे॒व प्र॑ति॒ष्ठाय॑ । \newline
38. ए॒व प्र॑ति॒ष्ठाय॑ प्रति॒ष्ठायै॒वैव प्र॑ति॒ष्ठाये᳚ न्द्रि॒य मि॑न्द्रि॒यम् प्र॑ति॒ष्ठायै॒वैव प्र॑ति॒ष्ठाये᳚ न्द्रि॒यम् । \newline
39. प्र॒ति॒ष्ठाये᳚ न्द्रि॒य मि॑न्द्रि॒यम् प्र॑ति॒ष्ठाय॑ प्रति॒ष्ठाये᳚ न्द्रि॒यं ॅवी॒र्यं॑ ॅवी॒र्य॑ मिन्द्रि॒यम् प्र॑ति॒ष्ठाय॑ प्रति॒ष्ठाये᳚ न्द्रि॒यं ॅवी॒र्य᳚म् । \newline
40. प्र॒ति॒ष्ठायेति॑ प्रति - स्थाय॑ । \newline
41. इ॒न्द्रि॒यं ॅवी॒र्यं॑ ॅवी॒र्य॑ मिन्द्रि॒य मि॑न्द्रि॒यं ॅवी॒र्य॑ मु॒परि॑ष्टा दु॒परि॑ष्टाद् वी॒र्य॑ मिन्द्रि॒य मि॑न्द्रि॒यं ॅवी॒र्य॑ मु॒परि॑ष्टात् । \newline
42. वी॒र्य॑ मु॒परि॑ष्टा दु॒परि॑ष्टाद् वी॒र्यं॑ ॅवी॒र्य॑ मु॒परि॑ष्टा दा॒त्मन् ना॒त्मन् नु॒परि॑ष्टाद् वी॒र्यं॑ ॅवी॒र्य॑ मु॒परि॑ष्टा दा॒त्मन्न् । \newline
43. उ॒परि॑ष्टा दा॒त्मन् ना॒त्मन् नु॒परि॑ष्टा दु॒परि॑ष्टा दा॒त्मन् ध॑त्ते धत्त आ॒त्मन् नु॒परि॑ष्टा दु॒परि॑ष्टा दा॒त्मन् ध॑त्ते । \newline
44. आ॒त्मन् ध॑त्ते धत्त आ॒त्मन् ना॒त्मन् ध॑त्ते । \newline
45. ध॒त्त॒ इति॑ धत्ते । \newline
\pagebreak
\markright{ TS 2.2.2.1  \hfill https://www.vedavms.in \hfill}
\addcontentsline{toc}{section}{ TS 2.2.2.1 }
\section*{ TS 2.2.2.1 }

\textbf{TS 2.2.2.1 } \newline
\textbf{Samhita Paata} \newline

अ॒ग्नये॑ पथि॒कृते॑ पुरो॒डाश॑-म॒ष्टाक॑पालं॒ निर्व॑पे॒द्यो द॑र्.शपूर्णमासया॒जी सन्न॑मावा॒स्यां᳚ ॅवा पौर्णमा॒सीं ॅवा॑ऽतिपा॒दये᳚त् प॒थो वा ए॒षोद्ध्यप॑थेनैति॒ यो द॑र्.शपूर्णमासया॒जी सन्न॑मावा॒स्यां᳚ ॅवा पौर्णमा॒सीं ॅवा॑तिपा॒दय॑त्य॒ग्निमे॒व प॑थि॒कृतꣳ॒॒ स्वेन॑ भाग॒धेये॒नोप॑ धावति॒ स ए॒वैन॒मप॑था॒त् पन्था॒मपि॑ नयत्यन॒ड्वान् दक्षि॑णा व॒ही ह्ये॑ष समृ॑द्ध्या अ॒ग्नये᳚ व्र॒तप॑तये - [  ] \newline

\textbf{Pada Paata} \newline

अ॒ग्नये᳚ । प॒थि॒कृत॒ इति॑ पथि - कृते᳚ । पु॒रो॒डाश᳚म् । अ॒ष्टाक॑पाल॒मित्य॒ष्टा - क॒पा॒ल॒म् । निरिति॑ । व॒पे॒त् । यः । द॒र्॒.श॒पू॒र्ण॒मा॒स॒या॒जीति॑ दर्.शपूर्णमास - या॒जी । सन्न् । अ॒मा॒वा॒स्या॑मित्य॑मा - वा॒स्या᳚म् । वा॒ । पौ॒र्ण॒मा॒सीमिति॑ पौर्ण - मा॒सीम् । वा॒ । अ॒ति॒पा॒दये॒दित्य॑ति - पा॒दये᳚त् । प॒थः । वै । ए॒षः । अधीति॑ । अप॑थेन । ए॒ति॒ । यः । द॒र्.॒श॒पू॒र्ण॒मा॒स॒या॒जीति॑ दर्.शपूर्णमास - या॒जी । सन्न् । अ॒मा॒वा॒स्या॑मित्य॑मा - वा॒स्या᳚म् । वा॒ । पौ॒र्ण॒मा॒सीमिति॑ पौर्ण - मा॒सीम् । वा॒ । अ॒ति॒पा॒दय॒तीत्य॑ति - पा॒दय॑ति । अ॒ग्निम् । ए॒व । प॒थि॒कृत॒मिति॑ पथि - कृत᳚म् । स्वेन॑ । भा॒ग॒धेये॒नेति॑ भाग - धेये॑न । उपेति॑ । धा॒व॒ति॒ । सः । ए॒व । ए॒न॒म् । अप॑थात् । पन्था᳚म् । अपीति॑ । न॒य॒ति॒ । अ॒न॒ड्वान् । दक्षि॑णा । व॒ही । हि । ए॒षः । समृ॑द्ध्या॒ इति॒ सं - ऋ॒द्ध्यै॒ । अ॒ग्नये᳚ । व्र॒तप॑तय॒ इति॑ व्र॒त-प॒त॒ये॒ ।  \newline


\textbf{Krama Paata} \newline

अ॒ग्नये॑ पथि॒कृते᳚ । प॒थि॒कृते॑ पुरो॒डाश᳚म् । प॒थि॒कृत॒ इति॑ पथि - कृते᳚ । पु॒रो॒डाश॑म॒ष्टाक॑पालम् । अ॒ष्टाक॑पाल॒म् निः । अ॒ष्टाक॑पाल॒मित्य॒ष्टा - क॒पा॒ल॒म् । निर् व॑पेत् । व॒पे॒द् यः । यो द॑र्.शपूर्णमासया॒जी । द॒र्.॒श॒पू॒र्ण॒मा॒स॒या॒जी सन्न् । द॒र्.॒श॒पू॒र्ण॒मा॒स॒या॒जीति॑ दर्.शपूर्णमास - या॒जी । सन्न॑मावा॒स्या᳚म् । अ॒मा॒वा॒स्यां᳚ ॅवा । अ॒मा॒वा॒स्या॑मित्य॑मा - वा॒स्या᳚म् । वा॒ पौ॒र्ण॒मा॒सीम् । पा॒र्ण॒मा॒सीं ॅवा᳚ । पा॒र्ण॒मा॒सीमिति॑ पौर्ण - मा॒सीम् । वा॒ऽति॒पा॒दये᳚त् । अ॒ति॒पा॒दये᳚त् प॒थः । अ॒ति॒पा॒दये॒दित्य॑ति - पा॒दये᳚त् । प॒थो वै । वा ए॒षः । ए॒षोऽधि॑ । अध्यप॑थेन । अप॑थेनैति । ए॒ति॒ यः । यो द॑र्.शपूर्णमासया॒जी । द॒र्॒श॒पू॒र्ण॒मा॒स॒या॒जी सन्न् । 
द॒र्.॒श॒पू॒र्ण॒मा॒स॒या॒जीति॑ दर्.शपूर्णमास - या॒जी । सन्न॑मावा॒स्या᳚म् । अ॒मा॒वा॒स्यां᳚ ॅवा । अ॒मा॒वा॒स्या॑मित्य॑मा - वा॒स्या᳚म् । वा॒ पौ॒र्ण॒मा॒सीम् । पौ॒र्ण॒मा॒सीं ॅवा᳚ । पौ॒र्ण॒मा॒सीमिति॑ पौर्ण - मा॒सीम् । वा॒ऽति॒पा॒दय॑ति । अ॒ति॒पा॒दय॑त्य॒ग्निम् । अ॒ति॒पा॒दय॒तीत्य॑ति - पा॒दय॑ति । अ॒ग्निमे॒व । ए॒व प॑थि॒कृत᳚म् । प॒थि॒कृतꣳ॒॒ स्वेन॑ । प॒थि॒कृत॒मिति॑ पथि - कृत᳚म् । स्वेन॑ भाग॒धेये॑न । भा॒ग॒धेये॒नोप॑ । भा॒ग॒धेये॒नेति॑ भाग - धेये॑न । उप॑ धावति । धा॒व॒ति॒ सः । स ए॒व । ए॒वैन᳚म् । ए॒न॒मप॑थात् । अप॑था॒त्,पन्था᳚म् । पन्था॒मपि॑ । अपि॑ नयति । न॒य॒त्य॒न॒ड्वान् । अ॒न॒ड्वान् दक्षि॑णा । दक्षि॑णा व॒ही । व॒ही हि । ह्ये॑षः । ए॒ष समृ॑द्ध्यै । समृ॑द्धा अ॒ग्नये᳚ । समृ॑द्धा॒ इति॒ सं - ऋ॒द्ध्यै॒ । अ॒ग्नये᳚ व्र॒तप॑तये । व्र॒तप॑तये पुरो॒डाश᳚म् । व्र॒तप॑तय॒ इति॑ व्र॒त - प॒त॒ये॒ \newline

\textbf{Jatai Paata} \newline

1. अ॒ग्नये॑ पथि॒कृते॑ पथि॒कृते॒ ऽग्नये॒ ऽग्नये॑ पथि॒कृते᳚ । \newline
2. प॒थि॒कृते॑ पुरो॒डाश॑म् पुरो॒डाश॑म् पथि॒कृते॑ पथि॒कृते॑ पुरो॒डाश᳚म् । \newline
3. प॒थि॒कृत॒ इति॑ पथि - कृते᳚ । \newline
4. पु॒रो॒डाश॑ म॒ष्टाक॑पाल म॒ष्टाक॑पालम् पुरो॒डाश॑म् पुरो॒डाश॑ म॒ष्टाक॑पालम् । \newline
5. अ॒ष्टाक॑पाल॒म् निर् णिर॒ष्टाक॑पाल म॒ष्टाक॑पाल॒म् निः । \newline
6. अ॒ष्टाक॑पाल॒मित्य॒ष्टा - क॒पा॒ल॒म् । \newline
7. निर् व॑पेद् वपे॒न् निर् णिर् व॑पेत् । \newline
8. व॒पे॒द् यो यो व॑पेद् वपे॒द् यः । \newline
9. यो द॑र्.शपूर्णमासया॒जी द॑र्.शपूर्णमासया॒जी यो यो द॑र्.शपूर्णमासया॒जी । \newline
10. द॒र्॒.श॒पू॒र्ण॒मा॒स॒या॒जी सन् थ्सन् द॑र्.शपूर्णमासया॒जी द॑र्.शपूर्णमासया॒जी सन्न् । \newline
11. द॒र्॒.श॒पू॒र्ण॒मा॒स॒या॒जीति॑ दर्.शपूर्णमास - या॒जी । \newline
12. सन् न॑मावा॒स्या॑ ममावा॒स्याꣳ॑ सन् थ्सन् न॑मावा॒स्या᳚म् । \newline
13. अ॒मा॒वा॒स्यां᳚ ॅवा वा ऽमावा॒स्या॑ ममावा॒स्यां᳚ ॅवा । \newline
14. अ॒मा॒वा॒स्या॑मित्य॑मा - वा॒स्या᳚म् । \newline
15. वा॒ पौ॒र्ण॒मा॒सीम् पौ᳚र्णमा॒सीं ॅवा॑ वा पौर्णमा॒सीम् । \newline
16. पौ॒र्ण॒मा॒सीं ॅवा॑ वा पौर्णमा॒सीम् पौ᳚र्णमा॒सीं ॅवा᳚ । \newline
17. पौ॒र्ण॒मा॒सीमिति॑ पौर्ण - मा॒सीम् । \newline
18. वा॒ ऽति॒पा॒दये॑ दतिपा॒दये᳚द् वा वा ऽतिपा॒दये᳚त् । \newline
19. अ॒ति॒पा॒दये᳚त् प॒थः प॒थो॑ ऽतिपा॒दये॑ दतिपा॒दये᳚त् प॒थः । \newline
20. अ॒ति॒पा॒दये॒दित्य॑ति - पा॒दये᳚त् । \newline
21. प॒थो वै वै प॒थः प॒थो वै । \newline
22. वा ए॒ष ए॒ष वै वा ए॒षः । \newline
23. ए॒षोऽ ध्य ध्ये॒ष ए॒षो ऽधि॑ । \newline
24. अध्यप॑थे॒ना प॑थे॒ना ध्यध्यप॑थेन । \newline
25. अप॑थेनै त्ये॒ त्यप॑थे॒ना प॑थेनैति । \newline
26. ए॒ति॒ यो य ए᳚त्येति॒ यः । \newline
27. यो द॑र्.शपूर्णमासया॒जी द॑र्.शपूर्णमासया॒जी यो यो द॑र्.शपूर्णमासया॒जी । \newline
28. द॒र्॒.श॒पू॒र्ण॒मा॒स॒या॒जी सन् थ्सन् द॑र्.शपूर्णमासया॒जी द॑र्.शपूर्णमासया॒जी सन्न् । \newline
29. द॒र्.॒श॒पू॒र्ण॒मा॒स॒या॒जीति॑ दर्.शपूर्णमास - या॒जी । \newline
30. सन् न॑मावा॒स्या॑ ममावा॒स्याꣳ॑ सन् थ्सन् न॑मावा॒स्या᳚म् । \newline
31. अ॒मा॒वा॒स्यां᳚ ॅवा वा ऽमावा॒स्या॑ ममावा॒स्यां᳚ ॅवा । \newline
32. अ॒मा॒वा॒स्या॑मित्य॑मा - वा॒स्या᳚म् । \newline
33. वा॒ पौ॒र्ण॒मा॒सीम् पौ᳚र्णमा॒सीं ॅवा॑ वा पौर्णमा॒सीम् । \newline
34. पौ॒र्ण॒मा॒सीं ॅवा॑ वा पौर्णमा॒सीम् पौ᳚र्णमा॒सीं ॅवा᳚ । \newline
35. पौ॒र्ण॒मा॒सीमिति॑ पौर्ण - मा॒सीम् । \newline
36. वा॒ ऽति॒पा॒दय॑ त्यतिपा॒दय॑ति वा वा ऽतिपा॒दय॑ति । \newline
37. अ॒ति॒पा॒दय॑ त्य॒ग्नि म॒ग्नि म॑तिपा॒दय॑ त्यतिपा॒दय॑ त्य॒ग्निम् । \newline
38. अ॒ति॒पा॒दय॒तीत्य॑ति - पा॒दय॑ति । \newline
39. अ॒ग्नि मे॒वैवाग्नि म॒ग्नि मे॒व । \newline
40. ए॒व प॑थि॒कृत॑म् पथि॒कृत॑ मे॒वैव प॑थि॒कृत᳚म् । \newline
41. प॒थि॒कृतꣳ॒॒ स्वेन॒ स्वेन॑ पथि॒कृत॑म् पथि॒कृतꣳ॒॒ स्वेन॑ । \newline
42. प॒थि॒कृत॒मिति॑ पथि - कृत᳚म् । \newline
43. स्वेन॑ भाग॒धेये॑न भाग॒धेये॑न॒ स्वेन॒ स्वेन॑ भाग॒धेये॑न । \newline
44. भा॒ग॒धेये॒नोपोप॑ भाग॒धेये॑न भाग॒धेये॒नोप॑ । \newline
45. भा॒ग॒धेये॒नेति॑ भाग - धेये॑न । \newline
46. उप॑ धावति धाव॒ त्युपोप॑ धावति । \newline
47. धा॒व॒ति॒ स स धा॑वति धावति॒ सः । \newline
48. स ए॒वैव स स ए॒व । \newline
49. ए॒वैन॑ मेन मे॒वैवैन᳚म् । \newline
50. ए॒न॒ मप॑था॒ दप॑थादेन मेन॒ मप॑थात् । \newline
51. अप॑था॒त् पन्था॒म् पन्था॒ मप॑था॒ दप॑था॒त् पन्था᳚म् । \newline
52. पन्था॒ मप्यपि॒ पन्था॒म् पन्था॒ मपि॑ । \newline
53. अपि॑ नयति नय॒ त्यप्यपि॑ नयति । \newline
54. न॒य॒ त्य॒न॒ड्वा न॑न॒ड्वान् न॑यति नय त्यन॒ड्वान् । \newline
55. अ॒न॒ड्वान् दक्षि॑णा॒ दक्षि॑णा ऽन॒ड्वा न॑न॒ड्वान् दक्षि॑णा । \newline
56. दक्षि॑णा व॒ही व॒ही दक्षि॑णा॒ दक्षि॑णा व॒ही । \newline
57. व॒ही हि हि व॒ही व॒ही हि । \newline
58. ह्ये॑ष ए॒ष हि ह्ये॑षः । \newline
59. ए॒ष समृ॑द्ध्यै॒ समृ॑द्ध्या ए॒ष ए॒ष समृ॑द्ध्यै । \newline
60. समृ॑द्ध्या अ॒ग्नये॒ ऽग्नये॒ समृ॑द्ध्यै॒ समृ॑द्ध्या अ॒ग्नये᳚ । \newline
61. समृ॑द्ध्या॒ इति॒ सं - ऋ॒द्ध्यै॒ । \newline
62. अ॒ग्नये᳚ व्र॒तप॑तये व्र॒तप॑तये॒ ऽग्नये॒ ऽग्नये᳚ व्र॒तप॑तये । \newline
63. व्र॒तप॑तये पुरो॒डाश॑म् पुरो॒डाशं॑ ॅव्र॒तप॑तये व्र॒तप॑तये पुरो॒डाश᳚म् । \newline
64. व्र॒तप॑तय॒ इति॑ व्र॒त - प॒त॒ये॒ । \newline

\textbf{Ghana Paata } \newline

1. अ॒ग्नये॑ पथि॒कृते॑ पथि॒कृते॒ ऽग्नये॒ ऽग्नये॑ पथि॒कृते॑ पुरो॒डाश॑म् पुरो॒डाश॑म् पथि॒कृते॒ ऽग्नये॒ ऽग्नये॑ पथि॒कृते॑ पुरो॒डाश᳚म् । \newline
2. प॒थि॒कृते॑ पुरो॒डाश॑म् पुरो॒डाश॑म् पथि॒कृते॑ पथि॒कृते॑ पुरो॒डाश॑ म॒ष्टाक॑पाल म॒ष्टाक॑पालम् पुरो॒डाश॑म् पथि॒कृते॑ पथि॒कृते॑ पुरो॒डाश॑ म॒ष्टाक॑पालम् । \newline
3. प॒थि॒कृत॒ इति॑ पथि - कृते᳚ । \newline
4. पु॒रो॒डाश॑ म॒ष्टाक॑पाल म॒ष्टाक॑पालम् पुरो॒डाश॑म् पुरो॒डाश॑ म॒ष्टाक॑पाल॒म् निर् णिर॒ष्टाक॑पालम् पुरो॒डाश॑म् पुरो॒डाश॑ म॒ष्टाक॑पाल॒म् निः । \newline
5. अ॒ष्टाक॑पाल॒म् निर् णिर॒ष्टाक॑पाल म॒ष्टाक॑पाल॒म् निर् व॑पेद् वपे॒न् निर॒ष्टाक॑पाल म॒ष्टाक॑पाल॒म् निर् व॑पेत् । \newline
6. अ॒ष्टाक॑पाल॒मित्य॒ष्टा - क॒पा॒ल॒म् । \newline
7. निर् व॑पेद् वपे॒न् निर् णिर् व॑पे॒द् यो यो व॑पे॒न् निर् णिर् व॑पे॒द् यः । \newline
8. व॒पे॒द् यो यो व॑पेद् वपे॒द् यो द॑र्.शपूर्णमासया॒जी द॑र्.शपूर्णमासया॒जी यो व॑पेद् वपे॒द् यो द॑र्.शपूर्णमासया॒जी । \newline
9. यो द॑र्.शपूर्णमासया॒जी द॑र्.शपूर्णमासया॒जी यो यो द॑र्.शपूर्णमासया॒जी सन् थ्सन् द॑र्.शपूर्णमासया॒जी यो यो द॑र्.शपूर्णमासया॒जी सन्न् । \newline
10. द॒र्॒.श॒पू॒र्ण॒मा॒स॒या॒जी सन् थ्सन् द॑र्.शपूर्णमासया॒जी द॑र्.शपूर्णमासया॒जी सन् न॑मावा॒स्या॑ ममावा॒स्याꣳ॑ सन् द॑र्.शपूर्णमासया॒जी द॑र्.शपूर्णमासया॒जी सन् न॑मावा॒स्या᳚म् । \newline
11. द॒र्॒.श॒पू॒र्ण॒मा॒स॒या॒जीति॑ दर्.शपूर्णमास - या॒जी । \newline
12. सन् न॑मावा॒स्या॑ ममावा॒स्याꣳ॑ सन् थ्सन् न॑मावा॒स्यां᳚ ॅवा वा ऽमावा॒स्याꣳ॑ सन् थ्सन् न॑मावा॒स्यां᳚ ॅवा । \newline
13. अ॒मा॒वा॒स्यां᳚ ॅवा वा ऽमावा॒स्या॑ ममावा॒स्यां᳚ ॅवा पौर्णमा॒सीम् पौ᳚र्णमा॒सीं ॅवा॑ ऽमावा॒स्या॑ ममावा॒स्यां᳚ ॅवा पौर्णमा॒सीम् । \newline
14. अ॒मा॒वा॒स्या॑मित्य॑मा - वा॒स्या᳚म् । \newline
15. वा॒ पौ॒र्ण॒मा॒सीम् पौ᳚र्णमा॒सीं ॅवा॑ वा पौर्णमा॒सीं ॅवा॑ वा पौर्णमा॒सीं ॅवा॑ वा पौर्णमा॒सीं ॅवा᳚ । \newline
16. पौ॒र्ण॒मा॒सीं ॅवा॑ वा पौर्णमा॒सीम् पौ᳚र्णमा॒सीं ॅवा॑ ऽतिपा॒दये॑ दतिपा॒दये᳚द् वा पौर्णमा॒सीम् पौ᳚र्णमा॒सीं ॅवा॑ ऽतिपा॒दये᳚त् । \newline
17. पौ॒र्ण॒मा॒सीमिति॑ पौर्ण - मा॒सीम् । \newline
18. वा॒ ऽति॒पा॒दये॑ दतिपा॒दये᳚द् वा वा ऽतिपा॒दये᳚त् प॒थः प॒थो॑ ऽतिपा॒दये᳚द् वा वा ऽतिपा॒दये᳚त् प॒थः । \newline
19. अ॒ति॒पा॒दये᳚त् प॒थः प॒थो॑ ऽतिपा॒दये॑ दतिपा॒दये᳚त् प॒थो वै वै प॒थो॑ ऽतिपा॒दये॑ दतिपा॒दये᳚त् प॒थो वै । \newline
20. अ॒ति॒पा॒दये॒दित्य॑ति - पा॒दये᳚त् । \newline
21. प॒थो वै वै प॒थः प॒थो वा ए॒ष ए॒ष वै प॒थः प॒थो वा ए॒षः । \newline
22. वा ए॒ष ए॒ष वै वा ए॒षो ऽध्यध्ये॒ष वै वा ए॒षो ऽधि॑ । \newline
23. ए॒षो ऽध्यध्ये॒ष ए॒षो ऽध्यप॑थे॒ना प॑थे॒ना ध्ये॒ष ए॒षो ऽध्यप॑थेन । \newline
24. अध्यप॑थे॒ना प॑थे॒ना ध्यध्यप॑थे नैत्ये॒त्य प॑थे॒ना ध्यध्य प॑थेनैति । \newline
25. अप॑थे नैत्ये॒त्य प॑थे॒ना प॑थेनैति॒ यो य ए॒त्य प॑थे॒ना प॑थेनैति॒ यः । \newline
26. ए॒ति॒ यो य ए᳚त्येति॒ यो द॑र्.शपूर्णमासया॒जी द॑र्.शपूर्णमासया॒जी य ए᳚त्येति॒ यो द॑र्.शपूर्णमासया॒जी । \newline
27. यो द॑र्.शपूर्णमासया॒जी द॑र्.शपूर्णमासया॒जी यो यो द॑र्.शपूर्णमासया॒जी सन् थ्सन् द॑र्.शपूर्णमासया॒जी यो यो द॑र्.शपूर्णमासया॒जी सन्न् । \newline
28. द॒र्॒.श॒पू॒र्ण॒मा॒स॒या॒जी सन् थ्सन् द॑र्.शपूर्णमासया॒जी द॑र्.शपूर्णमासया॒जी सन् न॑मावा॒स्या॑ ममावा॒स्याꣳ॑ सन् द॑र्.शपूर्णमासया॒जी द॑र्.शपूर्णमासया॒जी सन् न॑मावा॒स्या᳚म् । \newline
29. द॒र्.॒श॒पू॒र्ण॒मा॒स॒या॒जीति॑ दर्.शपूर्णमास - या॒जी । \newline
30. सन् न॑मावा॒स्या॑ ममावा॒स्याꣳ॑ सन् थ्सन् न॑मावा॒स्यां᳚ ॅवा वा ऽमावा॒स्याꣳ॑ सन् थ्सन् न॑मावा॒स्यां᳚ ॅवा । \newline
31. अ॒मा॒वा॒स्यां᳚ ॅवा वा ऽमावा॒स्या॑ ममावा॒स्यां᳚ ॅवा पौर्णमा॒सीम् पौ᳚र्णमा॒सीं ॅवा॑ ऽमावा॒स्या॑ ममावा॒स्यां᳚ ॅवा पौर्णमा॒सीम् । \newline
32. अ॒मा॒वा॒स्या॑मित्य॑मा - वा॒स्या᳚म् । \newline
33. वा॒ पौ॒र्ण॒मा॒सीम् पौ᳚र्णमा॒सीं ॅवा॑ वा पौर्णमा॒सीं ॅवा॑ वा पौर्णमा॒सीं ॅवा॑ वा पौर्णमा॒सीं ॅवा᳚ । \newline
34. पौ॒र्ण॒मा॒सीं ॅवा॑ वा पौर्णमा॒सीम् पौ᳚र्णमा॒सीं ॅवा॑ ऽतिपा॒दय॑ त्यतिपा॒दय॑ति वा पौर्णमा॒सीम् पौ᳚र्णमा॒सीं ॅवा॑ ऽतिपा॒दय॑ति । \newline
35. पौ॒र्ण॒मा॒सीमिति॑ पौर्ण - मा॒सीम् । \newline
36. वा॒ ऽति॒पा॒दय॑ त्यतिपा॒दय॑ति वा वा ऽतिपा॒दय॑ त्य॒ग्नि म॒ग्नि म॑तिपा॒दय॑ति वा वा ऽतिपा॒दय॑ त्य॒ग्निम् । \newline
37. अ॒ति॒पा॒दय॑ त्य॒ग्नि म॒ग्नि म॑तिपा॒दय॑ त्यतिपा॒दय॑ त्य॒ग्नि मे॒वैवाग्नि म॑तिपा॒दय॑ त्यतिपा॒दय॑ त्य॒ग्नि मे॒व । \newline
38. अ॒ति॒पा॒दय॒तीत्य॑ति - पा॒दय॑ति । \newline
39. अ॒ग्नि मे॒वैवाग्नि म॒ग्नि मे॒व प॑थि॒कृत॑म् पथि॒कृत॑ मे॒वाग्नि म॒ग्नि मे॒व प॑थि॒कृत᳚म् । \newline
40. ए॒व प॑थि॒कृत॑म् पथि॒कृत॑ मे॒वैव प॑थि॒कृतꣳ॒॒ स्वेन॒ स्वेन॑ पथि॒कृत॑ मे॒वैव प॑थि॒कृतꣳ॒॒ स्वेन॑ । \newline
41. प॒थि॒कृतꣳ॒॒ स्वेन॒ स्वेन॑ पथि॒कृत॑म् पथि॒कृतꣳ॒॒ स्वेन॑ भाग॒धेये॑न भाग॒धेये॑न॒ स्वेन॑ पथि॒कृत॑म् पथि॒कृतꣳ॒॒ स्वेन॑ भाग॒धेये॑न । \newline
42. प॒थि॒कृत॒मिति॑ पथि - कृत᳚म् । \newline
43. स्वेन॑ भाग॒धेये॑न भाग॒धेये॑न॒ स्वेन॒ स्वेन॑ भाग॒धेये॒नोपोप॑ भाग॒धेये॑न॒ स्वेन॒ स्वेन॑ भाग॒धेये॒नोप॑ । \newline
44. भा॒ग॒धेये॒नोपोप॑ भाग॒धेये॑न भाग॒धेये॒नोप॑ धावति धाव॒त्युप॑ भाग॒धेये॑न भाग॒धेये॒नोप॑ धावति । \newline
45. भा॒ग॒धेये॒नेति॑ भाग - धेये॑न । \newline
46. उप॑ धावति धाव॒ त्युपोप॑ धावति॒ स स धा॑व॒ त्युपोप॑ धावति॒ सः । \newline
47. धा॒व॒ति॒ स स धा॑वति धावति॒ स ए॒वैव स धा॑वति धावति॒ स ए॒व । \newline
48. स ए॒वैव स स ए॒वैन॑ मेन मे॒व स स ए॒वैन᳚म् । \newline
49. ए॒वैन॑ मेन मे॒वैवैन॒ मप॑था॒ दप॑था देन मे॒वैवैन॒ मप॑थात् । \newline
50. ए॒न॒ मप॑था॒ दप॑था देन मेन॒ मप॑था॒त् पन्था॒म् पन्था॒ मप॑थादेन मेन॒ मप॑था॒त् पन्था᳚म् । \newline
51. अप॑था॒त् पन्था॒म् पन्था॒ मप॑था॒ दप॑था॒त् पन्था॒ मप्यपि॒ पन्था॒ मप॑था॒ दप॑था॒त् पन्था॒ मपि॑ । \newline
52. पन्था॒ मप्यपि॒ पन्था॒म् पन्था॒ मपि॑ नयति नय॒त्यपि॒ पन्था॒म् पन्था॒ मपि॑ नयति । \newline
53. अपि॑ नयति नय॒ त्यप्यपि॑ नय त्यन॒ड्वा न॑न॒ड्वान् न॑य॒ त्यप्यपि॑ नय त्यन॒ड्वान् । \newline
54. न॒य॒ त्य॒न॒ड्वा न॑न॒ड्वान् न॑यति नय त्यन॒ड्वान् दक्षि॑णा॒ दक्षि॑णा ऽन॒ड्वान् न॑यति नय त्यन॒ड्वान् दक्षि॑णा । \newline
55. अ॒न॒ड्वान् दक्षि॑णा॒ दक्षि॑णा ऽन॒ड्वा न॑न॒ड्वान् दक्षि॑णा व॒ही व॒ही दक्षि॑णा ऽन॒ड्वा न॑न॒ड्वान् दक्षि॑णा व॒ही । \newline
56. दक्षि॑णा व॒ही व॒ही दक्षि॑णा॒ दक्षि॑णा व॒ही हि हि व॒ही दक्षि॑णा॒ दक्षि॑णा व॒ही हि । \newline
57. व॒ही हि हि व॒ही व॒ही ह्ये॑ष ए॒ष हि व॒ही व॒ही ह्ये॑षः । \newline
58. ह्ये॑ष ए॒ष हि ह्ये॑ष समृ॑द्ध्यै॒ समृ॑द्ध्या ए॒ष हि ह्ये॑ष समृ॑द्ध्यै । \newline
59. ए॒ष समृ॑द्ध्यै॒ समृ॑द्ध्या ए॒ष ए॒ष समृ॑द्ध्या अ॒ग्नये॒ ऽग्नये॒ समृ॑द्ध्या ए॒ष ए॒ष समृ॑द्ध्या अ॒ग्नये᳚ । \newline
60. समृ॑द्ध्या अ॒ग्नये॒ ऽग्नये॒ समृ॑द्ध्यै॒ समृ॑द्ध्या अ॒ग्नये᳚ व्र॒तप॑तये व्र॒तप॑तये॒ ऽग्नये॒ समृ॑द्ध्यै॒ समृ॑द्ध्या अ॒ग्नये᳚ व्र॒तप॑तये । \newline
61. समृ॑द्ध्या॒ इति॒ सं - ऋ॒द्ध्यै॒ । \newline
62. अ॒ग्नये᳚ व्र॒तप॑तये व्र॒तप॑तये॒ ऽग्नये॒ ऽग्नये᳚ व्र॒तप॑तये पुरो॒डाश॑म् पुरो॒डाशं॑ ॅव्र॒तप॑तये॒ ऽग्नये॒ ऽग्नये᳚ व्र॒तप॑तये पुरो॒डाश᳚म् । \newline
63. व्र॒तप॑तये पुरो॒डाश॑म् पुरो॒डाशं॑ ॅव्र॒तप॑तये व्र॒तप॑तये पुरो॒डाश॑ म॒ष्टाक॑पाल म॒ष्टाक॑पालम् पुरो॒डाशं॑ ॅव्र॒तप॑तये व्र॒तप॑तये पुरो॒डाश॑ म॒ष्टाक॑पालम् । \newline
64. व्र॒तप॑तय॒ इति॑ व्र॒त - प॒त॒ये॒ । \newline
\pagebreak
\markright{ TS 2.2.2.2  \hfill https://www.vedavms.in \hfill}
\addcontentsline{toc}{section}{ TS 2.2.2.2 }
\section*{ TS 2.2.2.2 }

\textbf{TS 2.2.2.2 } \newline
\textbf{Samhita Paata} \newline

पुरो॒डाश॑म॒ष्टाक॑पालं॒ निर्व॑पे॒द्य आहि॑ताग्निः॒ सन्न॑व्र॒त्यमि॑व॒ चरे॑द॒ग्निमे॒व व्र॒तप॑तिꣳ॒॒ स्वेन॑ भाग॒धेये॒नोप॑ धावति॒ स ए॒वैनं॑ ॅव्र॒तमा ल॑भंयति॒ व्रत्यो॑ भवत्य॒ग्नये॑ रक्षो॒घ्ने पु॑रो॒डाश॑म॒ष्टाक॑पालं॒ निर्व॑पे॒द्यꣳ रक्षाꣳ॑सि॒ सचे॑रन्न॒ग्निमे॒व र॑क्षो॒हणꣳ॒॒ स्वेन॑ भाग॒धेये॒नोप॑ धावति॒ स ए॒वास्मा॒द्-रक्षाꣳ॒॒स्यप॑ हन्ति॒ निशि॑तायां॒ निर्व॑पे॒ - [  ] \newline

\textbf{Pada Paata} \newline

पु॒रो॒डाश᳚म् । अ॒ष्टाक॑पाल॒मित्य॒ष्टा - क॒पा॒ल॒म् । निरिति॑ । व॒पे॒त् । यः । आहि॑ताग्नि॒रित्याहि॑त-अ॒ग्निः॒ । सन्न् । अ॒व्र॒त्यम् । इ॒व॒ । चरे᳚त् । अ॒ग्निम् । ए॒व । व्र॒तप॑ति॒मिति॑ व्र॒त - प॒ति॒म् । स्वेन॑ । भा॒ग॒धेये॒नेति॑ भाग - धेये॑न । उपेति॑ । धा॒व॒ति॒ । सः । ए॒व । ए॒न॒म् । व्र॒तम् । एति॑ । ल॒भं॒य॒ति॒ । व्रत्यः॑ । भ॒व॒ति॒ । अ॒ग्नये᳚ । र॒क्षो॒घ्न इति॑ रक्षः - घ्ने । पु॒रो॒डाश᳚म् । अ॒ष्टाक॑पाल॒मित्य॒ष्टा - क॒पा॒ल॒म् । निरिति॑ । व॒पे॒त् । यम् । रक्षाꣳ॑सि । सचे॑रन्न् । अ॒ग्निम् । ए॒व । र॒क्षो॒हण॒मिति॑ रक्षः - हन᳚म् । स्वेन॑ । भा॒ग॒धेये॒नेति॑ भाग-धेये॑न । उपेति॑ । धा॒व॒ति॒ । सः । ए॒व । अ॒स्मा॒त् । रक्षाꣳ॑सि । अपेति॑ । ह॒न्ति॒ । निशि॑ताया॒मिति॒ नि - शि॒ता॒या॒म् । निरिति॑ । व॒पे॒त् ।  \newline


\textbf{Krama Paata} \newline

पु॒रो॒डाश॑म॒ष्टाक॑पालम् । अ॒ष्टाक॑पाल॒म् निः । अ॒ष्टाक॑पा॒लमित्य॒ष्टा - क॒पा॒ल॒म् । निर् व॑पेत् । व॒पे॒द् यः । य आहि॑ताग्निः । आहि॑ताग्निः॒ सन्न् । आहि॑ताग्नि॒रित्याहि॑त - अ॒ग्निः॒ । सन्न॑व्र॒त्यम् । अ॒व्र॒त्यमि॑व । इ॒व॒ चरे᳚त् । चरे॑द॒ग्निम् । अ॒ग्निमे॒व । ए॒व व्र॒तप॑तिम् । व्र॒तप॑तिꣳ॒॒ स्वेन॑ । व्र॒तप॑ति॒मिति॑ व्र॒त - प॒ति॒म् । स्वेन॑ भाग॒धेये॑न । भा॒ग॒धेये॒नोप॑ । भा॒ग॒धेये॒नेति॑ भाग - धेये॑न । उप॑ धावति । धा॒व॒ति॒ सः । स ए॒व । ए॒वैन᳚म् । ए॒नं॒ ॅव्र॒तम् । व्र॒तमा । आ ल॑म्भयति । ल॒म्भ॒य॒ति॒ व्रत्यः॑ । व्रत्यो॑ भवति । भ॒व॒त्य॒ग्नये᳚ । अ॒ग्नये॑ रक्षो॒घ्ने । र॒क्षो॒घ्ने पु॑रो॒डाश᳚म् । र॒क्षो॒घ्न इति॑ रक्षः - घ्ने । पु॒रो॒डाश॑म॒ष्टाक॑पालम् । अ॒ष्टाक॑पाल॒म् निः । अ॒ष्टाक॑पाल॒मित्य॒ष्टा - क॒पा॒ल॒म् । निर् व॑पेत् । व॒पे॒द् यम् । यꣳ रक्षाꣳ॑सि । रक्षाꣳ॑सि॒ सचे॑रन्न् । सचे॑रन्न॒ग्निम् । अ॒ग्निमे॒व । ए॒व र॑क्षो॒हण᳚म् । र॒क्षो॒हणꣳ॒॒ स्वेन॑ । र॒क्षो॒हण॒मिति॑ रक्षः - हन᳚म् । स्वेन॑ भाग॒धेये॑न । भा॒ग॒धेये॒नोप॑ । भा॒ग॒धेये॒नेति॑ भाग - धेये॑न । उप॑ धावति । धा॒व॒ति॒ सः । स ए॒व । ए॒वास्मा᳚त् । अ॒स्मा॒द् रक्षाꣳ॑सि । रक्षाꣳ॒॒स्यप॑ । अप॑ हन्ति । ह॒न्ति॒ निशि॑तायाम् । निशि॑ताया॒म् निः । निशि॑ताया॒मिति॒ नि - शि॒ता॒या॒म् । निर् व॑पेत् । व॒पे॒न्निशि॑तायाम् \newline

\textbf{Jatai Paata} \newline

1. पु॒रो॒डाश॑ म॒ष्टाक॑पाल म॒ष्टाक॑पालम् पुरो॒डाश॑म् पुरो॒डाश॑ म॒ष्टाक॑पालम् । \newline
2. अ॒ष्टाक॑पाल॒म् निर् णिर॒ष्टाक॑पाल म॒ष्टाक॑पाल॒म् निः । \newline
3. अ॒ष्टाक॑पाल॒मित्य॒ष्टा - क॒पा॒ल॒म् । \newline
4. निर् व॑पेद् वपे॒न् निर् णिर् व॑पेत् । \newline
5. व॒पे॒द् यो यो व॑पेद् वपे॒द् यः । \newline
6. य आहि॑ताग्नि॒ राहि॑ताग्नि॒र् यो य आहि॑ताग्निः । \newline
7. आहि॑ताग्निः॒ सन् थ्सन् नाहि॑ताग्नि॒ राहि॑ताग्निः॒ सन्न् । \newline
8. आहि॑ताग्नि॒रित्याहि॑त - अ॒ग्निः॒ । \newline
9. सन् न॑व्र॒त्य म॑व्र॒त्यꣳ सन् थ्सन् न॑व्र॒त्यम् । \newline
10. अ॒व्र॒त्य मि॑वे वाव्र॒त्य म॑व्र॒त्य मि॑व । \newline
11. इ॒व॒ चरे॒च् चरे॑ दिवे व॒ चरे᳚त् । \newline
12. चरे॑ द॒ग्नि म॒ग्निम् चरे॒च् चरे॑ द॒ग्निम् । \newline
13. अ॒ग्नि मे॒वैवाग्नि म॒ग्नि मे॒व । \newline
14. ए॒व व्र॒तप॑तिं ॅव्र॒तप॑ति मे॒वैव व्र॒तप॑तिम् । \newline
15. व्र॒तप॑तिꣳ॒॒ स्वेन॒ स्वेन॑ व्र॒तप॑तिं ॅव्र॒तप॑तिꣳ॒॒ स्वेन॑ । \newline
16. व्र॒तप॑ति॒मिति॑ व्र॒त - प॒ति॒म् । \newline
17. स्वेन॑ भाग॒धेये॑न भाग॒धेये॑न॒ स्वेन॒ स्वेन॑ भाग॒धेये॑न । \newline
18. भा॒ग॒धेये॒नोपोप॑ भाग॒धेये॑न भाग॒धेये॒नोप॑ । \newline
19. भा॒ग॒धेये॒नेति॑ भाग - धेये॑न । \newline
20. उप॑ धावति धाव॒त्युपोप॑ धावति । \newline
21. धा॒व॒ति॒ स स धा॑वति धावति॒ सः । \newline
22. स ए॒वैव स स ए॒व । \newline
23. ए॒वैन॑ मेन मे॒वैवैन᳚म् । \newline
24. ए॒नं॒ ॅव्र॒तं ॅव्र॒त मे॑न मेनं ॅव्र॒तम् । \newline
25. व्र॒त मा व्र॒तं ॅव्र॒त मा । \newline
26. आ लं॑भयति लंभय॒त्या लं॑भयति । \newline
27. लं॒भ॒य॒ति॒ व्रत्यो॒ व्रत्यो॑ लंभयति लंभयति॒ व्रत्यः॑ । \newline
28. व्रत्यो॑ भवति भवति॒ व्रत्यो॒ व्रत्यो॑ भवति । \newline
29. भ॒व॒ त्य॒ग्नये॒ ऽग्नये॑ भवति भव त्य॒ग्नये᳚ । \newline
30. अ॒ग्नये॑ रक्षो॒घ्ने र॑क्षो॒घ्ने᳚ ऽग्नये॒ ऽग्नये॑ रक्षो॒घ्ने । \newline
31. र॒क्षो॒घ्ने पु॑रो॒डाश॑म् पुरो॒डाशꣳ॑ रक्षो॒घ्ने र॑क्षो॒घ्ने पु॑रो॒डाश᳚म् । \newline
32. र॒क्षो॒घ्न इति॑ रक्षः - घ्ने । \newline
33. पु॒रो॒डाश॑ म॒ष्टाक॑पाल म॒ष्टाक॑पालम् पुरो॒डाश॑म् पुरो॒डाश॑ म॒ष्टाक॑पालम् । \newline
34. अ॒ष्टाक॑पाल॒म् निर् णिर॒ष्टाक॑पाल म॒ष्टाक॑पाल॒म् निः । \newline
35. अ॒ष्टाक॑पाल॒मित्य॒ष्टा - क॒पा॒ल॒म् । \newline
36. निर् व॑पेद् वपे॒न् निर् णिर् व॑पेत् । \newline
37. व॒पे॒द् यं ॅयं ॅव॑पेद् वपे॒द् यम् । \newline
38. यꣳ रक्षाꣳ॑सि॒ रक्षाꣳ॑सि॒ यं ॅयꣳ रक्षाꣳ॑सि । \newline
39. रक्षाꣳ॑सि॒ सचे॑र॒न् थ्सचे॑र॒न् रक्षाꣳ॑सि॒ रक्षाꣳ॑सि॒ सचे॑रन्न् । \newline
40. सचे॑रन् न॒ग्नि म॒ग्निꣳ सचे॑र॒न् थ्सचे॑रन् न॒ग्निम् । \newline
41. अ॒ग्नि मे॒वैवाग्नि म॒ग्नि मे॒व । \newline
42. ए॒व र॑क्षो॒हणꣳ॑ रक्षो॒हण॑ मे॒वैव र॑क्षो॒हण᳚म् । \newline
43. र॒क्षो॒हणꣳ॒॒ स्वेन॒ स्वेन॑ रक्षो॒हणꣳ॑ रक्षो॒हणꣳ॒॒ स्वेन॑ । \newline
44. र॒क्षो॒हण॒मिति॑ रक्षः - हन᳚म् । \newline
45. स्वेन॑ भाग॒धेये॑न भाग॒धेये॑न॒ स्वेन॒ स्वेन॑ भाग॒धेये॑न । \newline
46. भा॒ग॒धेये॒नोपोप॑ भाग॒धेये॑न भाग॒धेये॒नोप॑ । \newline
47. भा॒ग॒धेये॒नेति॑ भाग - धेये॑न । \newline
48. उप॑ धावति धाव॒ त्युपोप॑ धावति । \newline
49. धा॒व॒ति॒ स स धा॑वति धावति॒ सः । \newline
50. स ए॒वैव स स ए॒व । \newline
51. ए॒वास्मा॑ दस्मा दे॒वैवास्मा᳚त् । \newline
52. अ॒स्मा॒द् रक्षाꣳ॑सि॒ रक्षाꣳ॑ स्यस्मा दस्मा॒द् रक्षाꣳ॑सि । \newline
53. रक्षाꣳ॒॒ स्यपाप॒ रक्षाꣳ॑सि॒ रक्षाꣳ॒॒ स्यप॑ । \newline
54. अप॑ हन्ति ह॒ न्त्यपाप॑ हन्ति । \newline
55. ह॒न्ति॒ निशि॑ताया॒म् निशि॑तायाꣳ हन्ति हन्ति॒ निशि॑तायाम् । \newline
56. निशि॑ताया॒म् निर् णिर् णिशि॑ताया॒म् निशि॑ताया॒म् निः । \newline
57. निशि॑ताया॒मिति॒ नि - शि॒ता॒या॒म् । \newline
58. निर् व॑पेद् वपे॒न् निर् णिर् व॑पेत् । \newline
59. व॒पे॒न् निशि॑ताया॒म् निशि॑तायां ॅवपेद् वपे॒न् निशि॑तायाम् । \newline

\textbf{Ghana Paata } \newline

1. पु॒रो॒डाश॑ म॒ष्टाक॑पाल म॒ष्टाक॑पालम् पुरो॒डाश॑म् पुरो॒डाश॑ म॒ष्टाक॑पाल॒म् निर् णिर॒ष्टाक॑पालम् पुरो॒डाश॑म् पुरो॒डाश॑ म॒ष्टाक॑पाल॒म् निः । \newline
2. अ॒ष्टाक॑पाल॒म् निर् णिर॒ष्टाक॑पाल म॒ष्टाक॑पाल॒म् निर् व॑पेद् वपे॒न् निर॒ष्टाक॑पाल म॒ष्टाक॑पाल॒म् निर् व॑पेत् । \newline
3. अ॒ष्टाक॑पाल॒मित्य॒ष्टा - क॒पा॒ल॒म् । \newline
4. निर् व॑पेद् वपे॒न् निर् णिर् व॑पे॒द् यो यो व॑पे॒न् निर् णिर् व॑पे॒द् यः । \newline
5. व॒पे॒द् यो यो व॑पेद् वपे॒द् य आहि॑ताग्नि॒ राहि॑ताग्नि॒र् यो व॑पेद् वपे॒द् य आहि॑ताग्निः । \newline
6. य आहि॑ताग्नि॒ राहि॑ताग्नि॒र् यो य आहि॑ताग्निः॒ सन् थ्सन् नाहि॑ताग्नि॒र् यो य आहि॑ताग्निः॒ सन्न् । \newline
7. आहि॑ताग्निः॒ सन् थ्सन् नाहि॑ताग्नि॒ राहि॑ताग्निः॒ सन् न॑व्र॒त्य म॑व्र॒त्यꣳ सन् नाहि॑ताग्नि॒ राहि॑ताग्निः॒ सन् न॑व्र॒त्यम् । \newline
8. आहि॑ताग्नि॒रित्याहि॑त - अ॒ग्निः॒ । \newline
9. सन् न॑व्र॒त्य म॑व्र॒त्यꣳ सन् थ्सन् न॑व्र॒त्य मि॑वे वाव्र॒त्यꣳ सन् थ्सन् न॑व्र॒त्य मि॑व । \newline
10. अ॒व्र॒त्य मि॑वे वाव्र॒त्य म॑व्र॒त्य मि॑व॒ चरे॒च् चरे॑ दिवाव्र॒त्य म॑व्र॒त्य मि॑व॒ चरे᳚त् । \newline
11. इ॒व॒ चरे॒च् चरे॑ दिवे व॒ चरे॑ द॒ग्नि म॒ग्निम् चरे॑ दिवे व॒ चरे॑ द॒ग्निम् । \newline
12. चरे॑द॒ग्नि म॒ग्निम् चरे॒च् चरे॑ द॒ग्नि मे॒वैवाग्निम् चरे॒च् चरे॑ द॒ग्नि मे॒व । \newline
13. अ॒ग्नि मे॒वैवाग्नि म॒ग्नि मे॒व व्र॒तप॑तिं ॅव्र॒तप॑ति मे॒वाग्नि म॒ग्नि मे॒व व्र॒तप॑तिम् । \newline
14. ए॒व व्र॒तप॑तिं ॅव्र॒तप॑ति मे॒वैव व्र॒तप॑तिꣳ॒॒ स्वेन॒ स्वेन॑ व्र॒तप॑ति मे॒वैव व्र॒तप॑तिꣳ॒॒ स्वेन॑ । \newline
15. व्र॒तप॑तिꣳ॒॒ स्वेन॒ स्वेन॑ व्र॒तप॑तिं ॅव्र॒तप॑तिꣳ॒॒ स्वेन॑ भाग॒धेये॑न भाग॒धेये॑न॒ स्वेन॑ व्र॒तप॑तिं ॅव्र॒तप॑तिꣳ॒॒ स्वेन॑ भाग॒धेये॑न । \newline
16. व्र॒तप॑ति॒मिति॑ व्र॒त - प॒ति॒म् । \newline
17. स्वेन॑ भाग॒धेये॑न भाग॒धेये॑न॒ स्वेन॒ स्वेन॑ भाग॒धेये॒नोपोप॑ भाग॒धेये॑न॒ स्वेन॒ स्वेन॑ भाग॒धेये॒नोप॑ । \newline
18. भा॒ग॒धेये॒नोपोप॑ भाग॒धेये॑न भाग॒धेये॒नोप॑ धावति धाव॒त्युप॑ भाग॒धेये॑न भाग॒धेये॒नोप॑ धावति । \newline
19. भा॒ग॒धेये॒नेति॑ भाग - धेये॑न । \newline
20. उप॑ धावति धाव॒ त्युपोप॑ धावति॒ स स धा॑व॒ त्युपोप॑ धावति॒ सः । \newline
21. धा॒व॒ति॒ स स धा॑वति धावति॒ स ए॒वैव स धा॑वति धावति॒ स ए॒व । \newline
22. स ए॒वैव स स ए॒वैन॑ मेन मे॒व स स ए॒वैन᳚म् । \newline
23. ए॒वैन॑ मेन मे॒वैवैनं॑ ॅव्र॒तं ॅव्र॒त मे॑न मे॒वैवैनं॑ ॅव्र॒तम् । \newline
24. ए॒नं॒ ॅव्र॒तं ॅव्र॒त मे॑न मेनं ॅव्र॒त मा व्र॒त मे॑न मेनं ॅव्र॒त मा । \newline
25. व्र॒त मा व्र॒तं ॅव्र॒त मा लं॑भयति लंभय॒त्या व्र॒तं ॅव्र॒त मा लं॑भयति । \newline
26. आ लं॑भयति लंभय॒त्या लं॑भयति॒ व्रत्यो॒ व्रत्यो॑ लंभय॒त्या लं॑भयति॒ व्रत्यः॑ । \newline
27. लं॒भ॒य॒ति॒ व्रत्यो॒ व्रत्यो॑ लंभयति लंभयति॒ व्रत्यो॑ भवति भवति॒ व्रत्यो॑ लंभयति लंभयति॒ व्रत्यो॑ भवति । \newline
28. व्रत्यो॑ भवति भवति॒ व्रत्यो॒ व्रत्यो॑ भवत्य॒ग्नये॒ ऽग्नये॑ भवति॒ व्रत्यो॒ व्रत्यो॑ भवत्य॒ग्नये᳚ । \newline
29. भ॒व॒त्य॒ग्नये॒ ऽग्नये॑ भवति भवत्य॒ग्नये॑ रक्षो॒घ्ने र॑क्षो॒घ्ने᳚ ऽग्नये॑ भवति भवत्य॒ग्नये॑ रक्षो॒घ्ने । \newline
30. अ॒ग्नये॑ रक्षो॒घ्ने र॑क्षो॒घ्ने᳚ ऽग्नये॒ ऽग्नये॑ रक्षो॒घ्ने पु॑रो॒डाश॑म् पुरो॒डाशꣳ॑ रक्षो॒घ्ने᳚ ऽग्नये॒ ऽग्नये॑ रक्षो॒घ्ने पु॑रो॒डाश᳚म् । \newline
31. र॒क्षो॒घ्ने पु॑रो॒डाश॑म् पुरो॒डाशꣳ॑ रक्षो॒घ्ने र॑क्षो॒घ्ने पु॑रो॒डाश॑ म॒ष्टाक॑पाल म॒ष्टाक॑पालम् पुरो॒डाशꣳ॑ रक्षो॒घ्ने र॑क्षो॒घ्ने पु॑रो॒डाश॑ म॒ष्टाक॑पालम् । \newline
32. र॒क्षो॒घ्न इति॑ रक्षः - घ्ने । \newline
33. पु॒रो॒डाश॑ म॒ष्टाक॑पाल म॒ष्टाक॑पालम् पुरो॒डाश॑म् पुरो॒डाश॑ म॒ष्टाक॑पाल॒म् निर् णिर॒ष्टाक॑पालम् पुरो॒डाश॑म् पुरो॒डाश॑ म॒ष्टाक॑पाल॒म् निः । \newline
34. अ॒ष्टाक॑पाल॒न्निर् णिर॒ष्टाक॑पाल म॒ष्टाक॑पाल॒म् निर् व॑पेद् वपे॒न् निर॒ष्टाक॑पाल म॒ष्टाक॑पाल॒म् निर् व॑पेत् । \newline
35. अ॒ष्टाक॑पाल॒मित्य॒ष्टा - क॒पा॒ल॒म् । \newline
36. निर् व॑पेद् वपे॒न् निर् णिर् व॑पे॒द् यं ॅयं ॅव॑पे॒न् निर् णिर् व॑पे॒द् यम् । \newline
37. व॒पे॒द् यं ॅयं ॅव॑पेद् वपे॒द् यꣳ रक्षाꣳ॑सि॒ रक्षाꣳ॑सि॒ यं ॅव॑पेद् वपे॒द् यꣳ रक्षाꣳ॑सि । \newline
38. यꣳ रक्षाꣳ॑सि॒ रक्षाꣳ॑सि॒ यं ॅयꣳ रक्षाꣳ॑सि॒ सचे॑र॒न् थ्सचे॑र॒न् रक्षाꣳ॑सि॒ यं ॅयꣳ रक्षाꣳ॑सि॒ सचे॑रन्न् । \newline
39. रक्षाꣳ॑सि॒ सचे॑र॒न् थ्सचे॑र॒न् रक्षाꣳ॑सि॒ रक्षाꣳ॑सि॒ सचे॑रन् न॒ग्नि म॒ग्निꣳ सचे॑र॒न् रक्षाꣳ॑सि॒ रक्षाꣳ॑सि॒ सचे॑रन् न॒ग्निम् । \newline
40. सचे॑रन् न॒ग्नि म॒ग्निꣳ सचे॑र॒न् थ्सचे॑रन् न॒ग्नि मे॒वैवाग्निꣳ सचे॑र॒न् थ्सचे॑रन् न॒ग्नि मे॒व । \newline
41. अ॒ग्नि मे॒वैवाग्नि म॒ग्नि मे॒व र॑क्षो॒हणꣳ॑ रक्षो॒हण॑ मे॒वाग्नि म॒ग्नि मे॒व र॑क्षो॒हण᳚म् । \newline
42. ए॒व र॑क्षो॒हणꣳ॑ रक्षो॒हण॑ मे॒वैव र॑क्षो॒हणꣳ॒॒ स्वेन॒ स्वेन॑ रक्षो॒हण॑ मे॒वैव र॑क्षो॒हणꣳ॒॒ स्वेन॑ । \newline
43. र॒क्षो॒हणꣳ॒॒ स्वेन॒ स्वेन॑ रक्षो॒हणꣳ॑ रक्षो॒हणꣳ॒॒ स्वेन॑ भाग॒धेये॑न भाग॒धेये॑न॒ स्वेन॑ रक्षो॒हणꣳ॑ रक्षो॒हणꣳ॒॒ स्वेन॑ भाग॒धेये॑न । \newline
44. र॒क्षो॒हण॒मिति॑ रक्षः - हन᳚म् । \newline
45. स्वेन॑ भाग॒धेये॑न भाग॒धेये॑न॒ स्वेन॒ स्वेन॑ भाग॒धेये॒नोपोप॑ भाग॒धेये॑न॒ स्वेन॒ स्वेन॑ भाग॒धेये॒नोप॑ । \newline
46. भा॒ग॒धेये॒नोपोप॑ भाग॒धेये॑न भाग॒धेये॒नोप॑ धावति धाव॒त्युप॑ भाग॒धेये॑न भाग॒धेये॒नोप॑ धावति । \newline
47. भा॒ग॒धेये॒नेति॑ भाग - धेये॑न । \newline
48. उप॑ धावति धाव॒ त्युपोप॑ धावति॒ स स धा॑व॒ त्युपोप॑ धावति॒ सः । \newline
49. धा॒व॒ति॒ स स धा॑वति धावति॒ स ए॒वैव स धा॑वति धावति॒ स ए॒व । \newline
50. स ए॒वैव स स ए॒वास्मा॑ दस्मादे॒व स स ए॒वास्मा᳚त् । \newline
51. ए॒वास्मा॑ दस्मा दे॒वैवास्मा॒द् रक्षाꣳ॑सि॒ रक्षाꣳ॑ स्यस्मा दे॒वैवास्मा॒द् रक्षाꣳ॑सि । \newline
52. अ॒स्मा॒द् रक्षाꣳ॑सि॒ रक्षाꣳ॑ स्यस्मा दस्मा॒द् रक्षाꣳ॒॒स्यपाप॒ रक्षाꣳ॑ स्यस्मा दस्मा॒द् रक्षाꣳ॒॒स्यप॑ । \newline
53. रक्षाꣳ॒॒ स्यपाप॒ रक्षाꣳ॑सि॒ रक्षाꣳ॒॒स्यप॑ हन्ति ह॒न्त्यप॒ रक्षाꣳ॑सि॒ रक्षाꣳ॒॒ स्यप॑ हन्ति । \newline
54. अप॑ हन्ति ह॒न्त्यपाप॑ हन्ति॒ निशि॑ताया॒म् निशि॑तायाꣳ ह॒न्त्यपाप॑ हन्ति॒ निशि॑तायाम् । \newline
55. ह॒न्ति॒ निशि॑ताया॒म् निशि॑तायाꣳ हन्ति हन्ति॒ निशि॑ताया॒म् निर् णिर् णिशि॑तायाꣳ हन्ति हन्ति॒ निशि॑ताया॒म् निः । \newline
56. निशि॑ताया॒म् निर् णिर् णिशि॑ताया॒म् निशि॑ताया॒म् निर् व॑पेद् वपे॒न् निर् णिशि॑ताया॒म् निशि॑ताया॒म् निर् व॑पेत् । \newline
57. निशि॑ताया॒मिति॒ नि - शि॒ता॒या॒म् । \newline
58. निर् व॑पेद् वपे॒न् निर् णिर् व॑पे॒न् निशि॑ताया॒म् निशि॑तायां ॅवपे॒न् निर् णिर् व॑पे॒न् निशि॑तायाम् । \newline
59. व॒पे॒न् निशि॑ताया॒म् निशि॑तायां ॅवपेद् वपे॒न् निशि॑तायाꣳ॒॒ हि हि निशि॑तायां ॅवपेद् वपे॒न् निशि॑तायाꣳ॒॒ हि । \newline
\pagebreak
\markright{ TS 2.2.2.3  \hfill https://www.vedavms.in \hfill}
\addcontentsline{toc}{section}{ TS 2.2.2.3 }
\section*{ TS 2.2.2.3 }

\textbf{TS 2.2.2.3 } \newline
\textbf{Samhita Paata} \newline

-न्निशि॑तायाꣳ॒॒ हि रक्षाꣳ॑सि प्रे॒रते॑ स॒प्रेंर्णा᳚न्ये॒वैना॑नि हन्ति॒ परि॑श्रिते याजये॒द्-रक्ष॑सा॒-मन॑न्ववचाराय रक्षो॒घ्नी या᳚ज्यानुवा॒क्ये॑ भवतो॒ रक्ष॑साꣳ॒॒ स्तृत्या॑ अ॒ग्नये॑ रु॒द्रव॑ते पुरो॒डाश॑म॒ष्टाक॑पालं॒ निर्व॑पेदभि॒चर॑न्ने॒षा वा अ॑स्य घो॒रा त॒नूर्यद्-रु॒द्रस्तस्मा॑ ए॒वैन॒मावृ॑श्चति ता॒जगार्ति॒-मार्च्छ॑त्य॒ग्नये॑ सुरभि॒मते॑ पुरो॒डाश॑म॒ष्टाक॑पालं॒ निर्व॑पे॒द्यस्य॒ गावो॑ वा॒ पुरु॑षा - [  ] \newline

\textbf{Pada Paata} \newline

निशि॑ताया॒मिति॒ नि - शि॒ता॒या॒म् । हि । रक्षाꣳ॑सि । प्रे॒रत॒ इति॑ प्र - ई॒रते᳚ । स॒प्रेंर्णा॒नीति॑ सं - प्रेर्णा॑नि । ए॒व । ए॒ना॒नि॒ । ह॒न्ति॒ । परि॑श्रित॒ इति॒ परि॑ - श्रि॒ते॒ । या॒ज॒ये॒त् । रक्ष॑साम् । अन॑न्ववचारा॒येत्यन॑नु - अ॒व॒चा॒रा॒य॒ । र॒क्षो॒घ्नी इति॑ रक्षः - घ्नी । या॒ज्या॒नु॒वा॒क्ये॑ इति॑ याज्या - अ॒नु॒वा॒क्ये᳚ । भ॒व॒तः॒ । रक्ष॑साम् । स्तृत्यै᳚ । अ॒ग्नये᳚ । रु॒द्रव॑त॒ इति॑ रु॒द्र - व॒ते॒ । पु॒रो॒डाश᳚म् । अ॒ष्टाक॑पाल॒मित्य॒ष्टा - क॒पा॒ल॒म् । निरिति॑ । व॒पे॒त् । अ॒भि॒चर॒न्नित्य॑भि - चरन्न्॑ । ए॒षा । वै । अ॒स्य॒ । घो॒रा । त॒नूः । यत् । रु॒द्रः । तस्मै᳚ । ए॒व । ए॒न॒म् । एति॑ । वृ॒श्च॒ति॒ । ता॒जक् । आर्ति᳚म् । एति॑ । ऋ॒च्छ॒ति॒ । अ॒ग्नये᳚ । सु॒र॒भि॒मत॒ इति॑ सुरभि - मते᳚ । पु॒रो॒डाश᳚म् । अ॒ष्टाक॑पाल॒मित्य॒ष्टा - क॒पा॒ल॒म् । निरिति॑ । व॒पे॒त् । यस्य॑ । गावः॑ । वा॒ । पुरु॑षाः ।  \newline


\textbf{Krama Paata} \newline

निशि॑तायाꣳ॒॒ हि । निशि॑ताया॒मिति॒ नि - शि॒ता॒या॒म् । हि रक्षाꣳ॑सि । रक्षाꣳ॑सि प्रे॒रते᳚ । प्रे॒रते॑ स॒म्प्रेर्णा॑नि । प्रे॒रत॒ इति॑ प्र - ई॒रते᳚ । स॒म्प्रेर्णा᳚न्ये॒व । स॒म्प्रेर्णा॒नीति॑ सं - प्रेर्णा॑नि । ए॒वैना॑नि । ए॒ना॒नि॒ ह॒न्ति॒ । ह॒न्ति॒ परि॑श्रिते । परि॑श्रिते याजयेत् । परि॑श्रित॒ इति॒ परि॑ - श्रि॒ते॒ । या॒ज॒ये॒द् रक्ष॑साम् । रक्ष॑सा॒मन॑न्ववचाराय । अन॑न्ववचाराय रक्षो॒घ्नी । अन॑न्ववचारा॒येत्यन॑नु - अ॒व॒चा॒रा॒य॒ । र॒क्षो॒घ्नी या᳚ज्यानुवा॒क्ये᳚ । र॒क्षो॒घ्नी इति॑ रक्षः - घ्नी । या॒ज्या॒नु॒वा॒क्ये॑ भवतः । या॒ज्य॒नु॒वा॒क्ये॑ इति॑ याज्या - अ॒नु॒वा॒क्ये᳚ । भ॒व॒तो॒ रक्ष॑साम् । रक्ष॑साꣳ॒॒ स्तृत्यै᳚ । स्तृत्या॑ अ॒ग्नये᳚ । अ॒ग्नये॑ रु॒द्रव॑ते । रु॒द्रव॑ते पुरो॒डाश᳚म् । रु॒द्रव॑त॒ इति॑ रु॒द्र - व॒ते॒ । पु॒रो॒डाश॑म॒ष्टाक॑पालम् । अ॒ष्टाक॑पाल॒म् निः । अ॒ष्टाक॑पाल॒मित्य॒ष्टा - क॒पा॒ल॒म् । निर् व॑पेत् । व॒पे॒द॒भि॒चरन्न्॑ । अ॒भि॒चर॑न्ने॒षा । अ॒भि॒चर॒न्नित्य॑भि - चरन्न्॑ । ए॒षा वै । वा अ॑स्य । अ॒स्य॒ घो॒रा । घो॒रा त॒नूः । त॒नूर् यत् । यद् रु॒द्रः । रु॒द्रस्तस्मै᳚ । तस्मा॑ ए॒व । ए॒वैन᳚म् । ए॒न॒ मा । आ वृ॑श्चति । वृ॒श्च॒ति॒ ता॒जक् । ता॒जगार्ति᳚म् । आर्ति॒मा । आर्च्छ॑ति । ऋ॒च्छ॒त्य॒ग्नये᳚ । अ॒ग्नये॑ सुरभि॒मते᳚ । सु॒र॒भि॒मते॑ पुरो॒डाश᳚म् । सु॒र॒भि॒मत॒ इति॑ सुरभि - मते᳚ । पु॒रो॒डाश॑म॒ष्टाक॑पालम् । अ॒ष्टाक॑पाल॒म् निः । अ॒ष्टाक॑पाल॒मित्य॒ष्टा - क॒पा॒ल॒म् । निर् व॑पेत् । व॒पे॒द् यस्य॑ । यस्य॒ गावः॑ । गावो॑ वा । वा॒ पुरु॑षाः । पुरु॑षा वा \newline

\textbf{Jatai Paata} \newline

1. निशि॑तायाꣳ॒॒ हि हि निशि॑ताया॒म् निशि॑तायाꣳ॒॒ हि । \newline
2. निशि॑ताया॒मिति॒ नि - शि॒ता॒या॒म् । \newline
3. हि रक्षाꣳ॑सि॒ रक्षाꣳ॑सि॒ हि हि रक्षाꣳ॑सि । \newline
4. रक्षाꣳ॑सि प्रे॒रते᳚ प्रे॒रते॒ रक्षाꣳ॑सि॒ रक्षाꣳ॑सि प्रे॒रते᳚ । \newline
5. प्रे॒रते॑ सं॒प्रेर्णा॑नि सं॒प्रेर्णा॑नि प्रे॒रते᳚ प्रे॒रते॑ सं॒प्रेर्णा॑नि । \newline
6. प्रे॒रत॒ इति॑ प्र - ई॒रते᳚ । \newline
7. सं॒प्रेर्णा᳚ न्ये॒वैव सं॒प्रेर्णा॑नि सं॒प्रेर्णा᳚ न्ये॒व । \newline
8. सं॒प्रेर्णा॒नीति॑ सं - प्रेर्णा॑नि । \newline
9. ए॒वैना᳚ न्येना न्ये॒ वैवैना॑नि । \newline
10. ए॒ना॒नि॒ ह॒न्ति॒ ह॒ न्त्ये॒ना॒ न्ये॒ना॒नि॒ ह॒न्ति॒ । \newline
11. ह॒न्ति॒ परि॑श्रिते॒ परि॑श्रिते हन्ति हन्ति॒ परि॑श्रिते । \newline
12. परि॑श्रिते याजयेद् याजये॒त् परि॑श्रिते॒ परि॑श्रिते याजयेत् । \newline
13. परि॑श्रित॒ इति॒ परि॑ - श्रि॒ते॒ । \newline
14. या॒ज॒ये॒द् रक्ष॑साꣳ॒॒ रक्ष॑सां ॅयाजयेद् याजये॒द् रक्ष॑साम् । \newline
15. रक्ष॑सा॒ मन॑न्ववचारा॒या न॑न्ववचाराय॒ रक्ष॑साꣳ॒॒ रक्ष॑सा॒ मन॑न्ववचाराय । \newline
16. अन॑न्ववचाराय रक्षो॒घ्नी र॑क्षो॒घ्नी अन॑न्ववचारा॒या न॑न्ववचाराय रक्षो॒घ्नी । \newline
17. अन॑न्ववचारा॒येत्यन॑नु - अ॒व॒चा॒रा॒य॒ । \newline
18. र॒क्षो॒घ्नी या᳚ज्यानुवा॒क्ये॑ याज्यानुवा॒क्ये॑ रक्षो॒घ्नी र॑क्षो॒घ्नी या᳚ज्यानुवा॒क्ये᳚ । \newline
19. र॒क्षो॒घ्नी इति॑ रक्षः - घ्नी । \newline
20. या॒ज्या॒नु॒वा॒क्ये॑ भवतो भवतो याज्यानुवा॒क्ये॑ याज्यानुवा॒क्ये॑ भवतः । \newline
21. या॒ज्या॒नु॒वा॒क्ये॑ इति॑ याज्या - अ॒नु॒वा॒क्ये᳚ । \newline
22. भ॒व॒तो॒ रक्ष॑साꣳ॒॒ रक्ष॑साम् भवतो भवतो॒ रक्ष॑साम् । \newline
23. रक्ष॑साꣳ॒॒ स्तृत्यै॒ स्तृत्यै॒ रक्ष॑साꣳ॒॒ रक्ष॑साꣳ॒॒ स्तृत्यै᳚ । \newline
24. स्तृत्या॑ अ॒ग्नये॒ ऽग्नये॒ स्तृत्यै॒ स्तृत्या॑ अ॒ग्नये᳚ । \newline
25. अ॒ग्नये॑ रु॒द्रव॑ते रु॒द्रव॑ते॒ ऽग्नये॒ ऽग्नये॑ रु॒द्रव॑ते । \newline
26. रु॒द्रव॑ते पुरो॒डाश॑म् पुरो॒डाशꣳ॑ रु॒द्रव॑ते रु॒द्रव॑ते पुरो॒डाश᳚म् । \newline
27. रु॒द्रव॑त॒ इति॑ रु॒द्र - व॒ते॒ । \newline
28. पु॒रो॒डाश॑ म॒ष्टाक॑पाल म॒ष्टाक॑पालम् पुरो॒डाश॑म् पुरो॒डाश॑ म॒ष्टाक॑पालम् । \newline
29. अ॒ष्टाक॑पाल॒म् निर् णिर॒ष्टाक॑पाल म॒ष्टाक॑पाल॒म् निः । \newline
30. अ॒ष्टाक॑पाल॒मित्य॒ष्टा - क॒पा॒ल॒म् । \newline
31. निर् व॑पेद् वपे॒न् निर् णिर् व॑पेत् । \newline
32. व॒पे॒ द॒भि॒चर॑न् नभि॒चरन्॑. वपेद् वपे दभि॒चरन्न्॑ । \newline
33. अ॒भि॒चर॑न् ने॒षैषा ऽभि॒चर॑न् नभि॒चर॑न् ने॒षा । \newline
34. अ॒भि॒चर॒न्नित्य॑भि - चरन्न्॑ । \newline
35. ए॒षा वै वा ए॒षैषा वै । \newline
36. वा अ॑स्यास्य॒ वै वा अ॑स्य । \newline
37. अ॒स्य॒ घो॒रा घो॒रा ऽस्या᳚स्य घो॒रा । \newline
38. घो॒रा त॒नू स्त॒नूर् घो॒रा घो॒रा त॒नूः । \newline
39. त॒नूर् यद् यत् त॒नू स्त॒नूर् यत् । \newline
40. यद् रु॒द्रो रु॒द्रो यद् यद् रु॒द्रः । \newline
41. रु॒द्र स्तस्मै॒ तस्मै॑ रु॒द्रो रु॒द्र स्तस्मै᳚ । \newline
42. तस्मा॑ ए॒वैव तस्मै॒ तस्मा॑ ए॒व । \newline
43. ए॒वैन॑ मेन मे॒वैवैन᳚म् । \newline
44. ए॒न॒ मैन॑ मेन॒ मा । \newline
45. आ वृ॑श्चति वृश्च॒त्या वृ॑श्चति । \newline
46. वृ॒श्च॒ति॒ ता॒जक् ता॒जग् वृ॑श्चति वृश्चति ता॒जक् । \newline
47. ता॒जगार्ति॒ मार्ति॑म् ता॒जक् ता॒जगार्ति᳚म् । \newline
48. आर्ति॒ मा ऽऽर्ति॒ मार्ति॒ मा । \newline
49. आर्च्छ॑ त्यृच्छ॒ त्यार्च्छ॑ति । \newline
50. ऋ॒च्छ॒ त्य॒ग्नये॒ ऽग्नय॑ ऋच्छ त्यृच्छ त्य॒ग्नये᳚ । \newline
51. अ॒ग्नये॑ सुरभि॒मते॑ सुरभि॒मते॒ ऽग्नये॒ ऽग्नये॑ सुरभि॒मते᳚ । \newline
52. सु॒र॒भि॒मते॑ पुरो॒डाश॑म् पुरो॒डाशꣳ॑ सुरभि॒मते॑ सुरभि॒मते॑ पुरो॒डाश᳚म् । \newline
53. सु॒र॒भि॒मत॒ इति॑ सुरभि - मते᳚ । \newline
54. पु॒रो॒डाश॑ म॒ष्टाक॑पाल म॒ष्टाक॑पालम् पुरो॒डाश॑म् पुरो॒डाश॑ म॒ष्टाक॑पालम् । \newline
55. अ॒ष्टाक॑पाल॒म् निर् णिर॒ष्टाक॑पाल म॒ष्टाक॑पाल॒म् निः । \newline
56. अ॒ष्टाक॑पाल॒मित्य॒ष्टा - क॒पा॒ल॒म् । \newline
57. निर् व॑पेद् वपे॒न् निर् णिर् व॑पेत् । \newline
58. व॒पे॒द् यस्य॒ यस्य॑ वपेद् वपे॒द् यस्य॑ । \newline
59. यस्य॒ गावो॒ गावो॒ यस्य॒ यस्य॒ गावः॑ । \newline
60. गावो॑ वा वा॒ गावो॒ गावो॑ वा । \newline
61. वा॒ पुरु॑षाः॒ पुरु॑षा वा वा॒ पुरु॑षाः । \newline
62. पुरु॑षा वा वा॒ पुरु॑षाः॒ पुरु॑षा वा । \newline

\textbf{Ghana Paata } \newline

1. निशि॑तायाꣳ॒॒ हि हि निशि॑ताया॒म् निशि॑तायाꣳ॒॒ हि रक्षाꣳ॑सि॒ रक्षाꣳ॑सि॒ हि निशि॑ताया॒म् निशि॑तायाꣳ॒॒ हि रक्षाꣳ॑सि । \newline
2. निशि॑ताया॒मिति॒ नि - शि॒ता॒या॒म् । \newline
3. हि रक्षाꣳ॑सि॒ रक्षाꣳ॑सि॒ हि हि रक्षाꣳ॑सि प्रे॒रते᳚ प्रे॒रते॒ रक्षाꣳ॑सि॒ हि हि रक्षाꣳ॑सि प्रे॒रते᳚ । \newline
4. रक्षाꣳ॑सि प्रे॒रते᳚ प्रे॒रते॒ रक्षाꣳ॑सि॒ रक्षाꣳ॑सि प्रे॒रते॑ सं॒प्रेर्णा॑नि सं॒प्रेर्णा॑नि प्रे॒रते॒ रक्षाꣳ॑सि॒ रक्षाꣳ॑सि प्रे॒रते॑ सं॒प्रेर्णा॑नि । \newline
5. प्रे॒रते॑ सं॒प्रेर्णा॑नि सं॒प्रेर्णा॑नि प्रे॒रते᳚ प्रे॒रते॑ सं॒प्रेर्णा᳚ न्ये॒वैव सं॒प्रेर्णा॑नि प्रे॒रते᳚ प्रे॒रते॒ सं॒प्रेर्णा᳚ न्ये॒व । \newline
6. प्रे॒रत॒ इति॑ प्र - ई॒रते᳚ । \newline
7. सं॒प्रेर्णा᳚ न्ये॒वैव सं॒प्रेर्णा॑नि सं॒प्रेर्णा᳚ न्ये॒वैना᳚ न्येनान्ये॒व सं॒प्रेर्णा॑नि सं॒प्रेर्णा᳚ न्ये॒वैना॑नि । \newline
8. सं॒प्रेर्णा॒नीति॑ सं - प्रेर्णा॑नि । \newline
9. ए॒वैना᳚ न्येना न्ये॒वैवैना॑नि हन्ति हन्त्येना न्ये॒वैवैना॑नि हन्ति । \newline
10. ए॒ना॒नि॒ ह॒न्ति॒ ह॒न्त्ये॒ना॒ न्ये॒ना॒नि॒ ह॒न्ति॒ परि॑श्रिते॒ परि॑श्रिते हन्त्येना न्येनानि हन्ति॒ परि॑श्रिते । \newline
11. ह॒न्ति॒ परि॑श्रिते॒ परि॑श्रिते हन्ति हन्ति॒ परि॑श्रिते याजयेद् याजये॒त् परि॑श्रिते हन्ति हन्ति॒ परि॑श्रिते याजयेत् । \newline
12. परि॑श्रिते याजयेद् याजये॒त् परि॑श्रिते॒ परि॑श्रिते याजये॒द् रक्ष॑साꣳ॒॒ रक्ष॑सां ॅयाजये॒त् परि॑श्रिते॒ परि॑श्रिते याजये॒द् रक्ष॑साम् । \newline
13. परि॑श्रित॒ इति॒ परि॑ - श्रि॒ते॒ । \newline
14. या॒ज॒ये॒द् रक्ष॑साꣳ॒॒ रक्ष॑सां ॅयाजयेद् याजये॒द् रक्ष॑सा॒ मन॑न्ववचारा॒या न॑न्ववचाराय॒ रक्ष॑सां ॅयाजयेद् याजये॒द् रक्ष॑सा॒ मन॑न्ववचाराय । \newline
15. रक्ष॑सा॒ मन॑न्ववचारा॒या न॑न्ववचाराय॒ रक्ष॑साꣳ॒॒ रक्ष॑सा॒ मन॑न्ववचाराय रक्षो॒घ्नी र॑क्षो॒घ्नी अन॑न्ववचाराय॒ रक्ष॑साꣳ॒॒ रक्ष॑सा॒ मन॑न्ववचाराय रक्षो॒घ्नी । \newline
16. अन॑न्ववचाराय रक्षो॒घ्नी र॑क्षो॒घ्नी अन॑न्ववचारा॒या न॑न्ववचाराय रक्षो॒घ्नी या᳚ज्यानुवा॒क्ये॑ याज्यानुवा॒क्ये॑ रक्षो॒घ्नी अन॑न्ववचारा॒या न॑न्ववचाराय रक्षो॒घ्नी या᳚ज्यानुवा॒क्ये᳚ । \newline
17. अन॑न्ववचारा॒येत्यन॑नु - अ॒व॒चा॒रा॒य॒ । \newline
18. र॒क्षो॒घ्नी या᳚ज्यानुवा॒क्ये॑ याज्यानुवा॒क्ये॑ रक्षो॒घ्नी र॑क्षो॒घ्नी या᳚ज्यानुवा॒क्ये॑ भवतो भवतो याज्यानुवा॒क्ये॑ रक्षो॒घ्नी र॑क्षो॒घ्नी या᳚ज्यानुवा॒क्ये॑ भवतः । \newline
19. र॒क्षो॒घ्नी इति॑ रक्षः - घ्नी । \newline
20. या॒ज्या॒नु॒वा॒क्ये॑ भवतो भवतो याज्यानुवा॒क्ये॑ याज्यानुवा॒क्ये॑ भवतो॒ रक्ष॑साꣳ॒॒ रक्ष॑साम् भवतो याज्यानुवा॒क्ये॑ याज्यानुवा॒क्ये॑ भवतो॒ रक्ष॑साम् । \newline
21. या॒ज्या॒नु॒वा॒क्ये॑ इति॑ याज्या - अ॒नु॒वा॒क्ये᳚ । \newline
22. भ॒व॒तो॒ रक्ष॑साꣳ॒॒ रक्ष॑साम् भवतो भवतो॒ रक्ष॑साꣳ॒॒ स्तृत्यै॒ स्तृत्यै॒ रक्ष॑साम् भवतो भवतो॒ रक्ष॑साꣳ॒॒ स्तृत्यै᳚ । \newline
23. रक्ष॑साꣳ॒॒ स्तृत्यै॒ स्तृत्यै॒ रक्ष॑साꣳ॒॒ रक्ष॑साꣳ॒॒ स्तृत्या॑ अ॒ग्नये॒ ऽग्नये॒ स्तृत्यै॒ रक्ष॑साꣳ॒॒ रक्ष॑साꣳ॒॒ स्तृत्या॑ अ॒ग्नये᳚ । \newline
24. स्तृत्या॑ अ॒ग्नये॒ ऽग्नये॒ स्तृत्यै॒ स्तृत्या॑ अ॒ग्नये॑ रु॒द्रव॑ते रु॒द्रव॑ते॒ ऽग्नये॒ स्तृत्यै॒ स्तृत्या॑ अ॒ग्नये॑ रु॒द्रव॑ते । \newline
25. अ॒ग्नये॑ रु॒द्रव॑ते रु॒द्रव॑ते॒ ऽग्नये॒ ऽग्नये॑ रु॒द्रव॑ते पुरो॒डाश॑म् पुरो॒डाशꣳ॑ रु॒द्रव॑ते॒ ऽग्नये॒ ऽग्नये॑ रु॒द्रव॑ते पुरो॒डाश᳚म् । \newline
26. रु॒द्रव॑ते पुरो॒डाश॑म् पुरो॒डाशꣳ॑ रु॒द्रव॑ते रु॒द्रव॑ते पुरो॒डाश॑ म॒ष्टाक॑पाल म॒ष्टाक॑पालम् पुरो॒डाशꣳ॑ रु॒द्रव॑ते रु॒द्रव॑ते पुरो॒डाश॑ म॒ष्टाक॑पालम् । \newline
27. रु॒द्रव॑त॒ इति॑ रु॒द्र - व॒ते॒ । \newline
28. पु॒रो॒डाश॑ म॒ष्टाक॑पाल म॒ष्टाक॑पालम् पुरो॒डाश॑म् पुरो॒डाश॑ म॒ष्टाक॑पाल॒म् निर् णिर॒ष्टाक॑पालम् पुरो॒डाश॑म् पुरो॒डाश॑ म॒ष्टाक॑पाल॒म् निः । \newline
29. अ॒ष्टाक॑पाल॒म् निर् णिर॒ष्टाक॑पाल म॒ष्टाक॑पाल॒म् निर् व॑पेद् वपे॒न् निर॒ष्टाक॑पाल म॒ष्टाक॑पाल॒म् निर् व॑पेत् । \newline
30. अ॒ष्टाक॑पाल॒मित्य॒ष्टा - क॒पा॒ल॒म् । \newline
31. निर् व॑पेद् वपे॒न् निर् णिर् व॑पे दभि॒चर॑न् नभि॒चरन्॑. वपे॒न् निर् णिर् व॑पे दभि॒चरन्न्॑ । \newline
32. व॒पे॒ द॒भि॒चर॑न् नभि॒चरन्॑. वपेद् वपे दभि॒चर॑न् ने॒षैषा ऽभि॒चरन्॑. वपेद् वपे दभि॒चर॑न् ने॒षा । \newline
33. अ॒भि॒चर॑न् ने॒षैषा ऽभि॒चर॑न् नभि॒चर॑न् ने॒षा वै वा ए॒षा ऽभि॒चर॑न् नभि॒चर॑न् ने॒षा वै । \newline
34. अ॒भि॒चर॒न्नित्य॑भि - चरन्न्॑ । \newline
35. ए॒षा वै वा ए॒षैषा वा अ॑स्यास्य॒ वा ए॒षैषा वा अ॑स्य । \newline
36. वा अ॑स्यास्य॒ वै वा अ॑स्य घो॒रा घो॒रा ऽस्य॒ वै वा अ॑स्य घो॒रा । \newline
37. अ॒स्य॒ घो॒रा घो॒रा ऽस्या᳚स्य घो॒रा त॒नू स्त॒नूर् घो॒रा ऽस्या᳚स्य घो॒रा त॒नूः । \newline
38. घो॒रा त॒नू स्त॒नूर् घो॒रा घो॒रा त॒नूर् यद् यत् त॒नूर् घो॒रा घो॒रा त॒नूर् यत् । \newline
39. त॒नूर् यद् यत् त॒नू स्त॒नूर् यद् रु॒द्रो रु॒द्रो यत् त॒नू स्त॒नूर् यद् रु॒द्रः । \newline
40. यद् रु॒द्रो रु॒द्रो यद् यद् रु॒द्र स्तस्मै॒ तस्मै॑ रु॒द्रो यद् यद् रु॒द्र स्तस्मै᳚ । \newline
41. रु॒द्र स्तस्मै॒ तस्मै॑ रु॒द्रो रु॒द्र स्तस्मा॑ ए॒वैव तस्मै॑ रु॒द्रो रु॒द्र स्तस्मा॑ ए॒व । \newline
42. तस्मा॑ ए॒वैव तस्मै॒ तस्मा॑ ए॒वैन॑ मेन मे॒व तस्मै॒ तस्मा॑ ए॒वैन᳚म् । \newline
43. ए॒वैन॑ मेन मे॒वैवैन॒ मैन॑ मे॒वैवैन॒ मा । \newline
44. ए॒न॒ मैन॑ मेन॒ मा वृ॑श्चति वृश्च॒त्यैन॑ मेन॒ मा वृ॑श्चति । \newline
45. आ वृ॑श्चति वृश्च॒त्या वृ॑श्चति ता॒जक् ता॒जग् वृ॑श्च॒त्या वृ॑श्चति ता॒जक् । \newline
46. वृ॒श्च॒ति॒ ता॒जक् ता॒जग् वृ॑श्चति वृश्चति ता॒जगार्ति॒ मार्ति॑म् ता॒जग् वृ॑श्चति वृश्चति ता॒जगार्ति᳚म् । \newline
47. ता॒जगार्ति॒ मार्ति॑म् ता॒जक् ता॒जगार्ति॒ मा ऽऽर्ति॑म् ता॒जक् ता॒जगार्ति॒ मा । \newline
48. आर्ति॒ मा ऽऽर्ति॒ मार्ति॒ मार्च्छ॑ त्यृच्छ॒त्या ऽऽर्ति॒ मार्ति॒ मार्च्छ॑ति । \newline
49. आर्च्छ॑ त्यृच्छ॒ त्यार्च्छ॑ त्य॒ग्नये॒ ऽग्नय॑ ऋच्छ॒ त्यार्च्छ॑ त्य॒ग्नये᳚ । \newline
50. ऋ॒च्छ॒ त्य॒ग्नये॒ ऽग्नय॑ ऋच्छ त्यृच्छ त्य॒ग्नये॑ सुरभि॒मते॑ सुरभि॒मते॒ ऽग्नय॑ ऋच्छ त्यृच्छ त्य॒ग्नये॑ सुरभि॒मते᳚ । \newline
51. अ॒ग्नये॑ सुरभि॒मते॑ सुरभि॒मते॒ ऽग्नये॒ ऽग्नये॑ सुरभि॒मते॑ पुरो॒डाश॑म् पुरो॒डाशꣳ॑ सुरभि॒मते॒ ऽग्नये॒ ऽग्नये॑ सुरभि॒मते॑ पुरो॒डाश᳚म् । \newline
52. सु॒र॒भि॒मते॑ पुरो॒डाश॑म् पुरो॒डाशꣳ॑ सुरभि॒मते॑ सुरभि॒मते॑ पुरो॒डाश॑ म॒ष्टाक॑पाल म॒ष्टाक॑पालम् पुरो॒डाशꣳ॑ सुरभि॒मते॑ सुरभि॒मते॑ पुरो॒डाश॑ म॒ष्टाक॑पालम् । \newline
53. सु॒र॒भि॒मत॒ इति॑ सुरभि - मते᳚ । \newline
54. पु॒रो॒डाश॑ म॒ष्टाक॑पाल म॒ष्टाक॑पालम् पुरो॒डाश॑म् पुरो॒डाश॑ म॒ष्टाक॑पाल॒म् निर् णिर॒ष्टाक॑पालम् पुरो॒डाश॑म् पुरो॒डाश॑ म॒ष्टाक॑पाल॒म् निः । \newline
55. अ॒ष्टाक॑पाल॒म् निर् णिर॒ष्टाक॑पाल म॒ष्टाक॑पाल॒म् निर् व॑पेद् वपे॒न् निर॒ष्टाक॑पाल म॒ष्टाक॑पाल॒म् निर् व॑पेत् । \newline
56. अ॒ष्टाक॑पाल॒मित्य॒ष्टा - क॒पा॒ल॒म् । \newline
57. निर् व॑पेद् वपे॒न् निर् णिर् व॑पे॒द् यस्य॒ यस्य॑ वपे॒न् निर् णिर् व॑पे॒द् यस्य॑ । \newline
58. व॒पे॒द् यस्य॒ यस्य॑ वपेद् वपे॒द् यस्य॒ गावो॒ गावो॒ यस्य॑ वपेद् वपे॒द् यस्य॒ गावः॑ । \newline
59. यस्य॒ गावो॒ गावो॒ यस्य॒ यस्य॒ गावो॑ वा वा॒ गावो॒ यस्य॒ यस्य॒ गावो॑ वा । \newline
60. गावो॑ वा वा॒ गावो॒ गावो॑ वा॒ पुरु॑षाः॒ पुरु॑षा वा॒ गावो॒ गावो॑ वा॒ पुरु॑षाः । \newline
61. वा॒ पुरु॑षाः॒ पुरु॑षा वा वा॒ पुरु॑षा वा वा॒ पुरु॑षा वा वा॒ पुरु॑षा वा । \newline
62. पुरु॑षा वा वा॒ पुरु॑षाः॒ पुरु॑षा वा प्र॒मीये॑रन् प्र॒मीये॑रन्. वा॒ पुरु॑षाः॒ पुरु॑षा वा प्र॒मीये॑रन्न् । \newline
\pagebreak
\markright{ TS 2.2.2.4  \hfill https://www.vedavms.in \hfill}
\addcontentsline{toc}{section}{ TS 2.2.2.4 }
\section*{ TS 2.2.2.4 }

\textbf{TS 2.2.2.4 } \newline
\textbf{Samhita Paata} \newline

वा प्र॒मीये॑र॒न्. यो वा॑ बिभी॒यादे॒षा वा अ॑स्य भेष॒ज्या॑ त॒नूर्यथ् सु॑रभि॒मती॒तयै॒वास्मै॑ भेष॒जं क॑रोति सुरभि॒मते॑ भवति पूतीग॒न्धस्या- प॑हत्या अ॒ग्नये॒ क्षाम॑वते पुरो॒डाश॑म॒ष्टाक॑पालं॒ निर्व॑पेथ् संग्रा॒मे संॅय॑त्ते भाग॒धेये॑नै॒वैनꣳ॑ शमयि॒त्वा परा॑न॒भि निर्दि॑शति॒ यमव॑रेषां॒ ॅविद्ध्य॑न्ति॒ जीव॑ति॒ स यं परे॑षां॒ प्र स मी॑यते॒ जय॑ति॒ तꣳ स॑ग्रां॒म - [  ] \newline

\textbf{Pada Paata} \newline

वा॒ । प्र॒मीये॑र॒न्निति॑ प्र - मीये॑रन्न् । यः । वा॒ । बि॒भी॒यात् । ए॒षा । वै । अ॒स्य॒ । भे॒ष॒ज्या᳚ । त॒नूः । यत् । सु॒र॒भि॒मतीति॑ सुरभि-मती᳚ । तया᳚ । ए॒व । अ॒स्मै॒ । भे॒ष॒जम् । क॒रो॒ति॒ । सु॒र॒भि॒मत॒ इति॑ सुरभि - मते᳚ । भ॒व॒ति॒ । पू॒ती॒ग॒न्धस्येति॑ पूती - ग॒न्धस्य॑ । अप॑हत्या॒ इत्यप॑ - ह॒त्यै॒ । अ॒ग्नये᳚ । क्षाम॑वत॒ इति॒ क्षाम॑ - व॒ते॒ । पु॒रो॒डाश᳚म् । अ॒ष्टाक॑पाल॒मित्य॒ष्टा - क॒पा॒ल॒म् । निरिति॑ । व॒पे॒त् । स॒ग्रां॒म इति॑ सं - ग्रा॒मे । संॅय॑त्त॒ इति॒ सं - य॒त्ते॒ । भा॒ग॒धेये॒नेति॑ भाग - धेये॑न । ए॒व । ए॒न॒म् । श॒म॒यि॒त्वा । परान्॑ । अ॒भि । निरिति॑ । दि॒श॒ति॒ । यम् । अव॑रेषाम् । विद्ध्य॑न्ति । जीव॑ति । सः । यम् । परे॑षाम् । प्रेति॑ । सः । मी॒य॒ते॒ । जय॑ति । तम् । स॒ग्रां॒ममिति॑ सं - ग्रा॒मम् ।  \newline


\textbf{Krama Paata} \newline

वा॒ प्र॒मीये॑रन्न् । प्र॒मीये॑र॒न्॒. यः । प्र॒मीये॑र॒न्निति॑ प्र - मीये॑रन्न् । यो वा᳚ । वा॒ बि॒भी॒यात् । बि॒भी॒यादे॒षा । ए॒षा वै । वा अ॑स्य । अ॒स्य॒ भे॒ष॒ज्या᳚ । भे॒ष॒ज्या॑ त॒नूः । त॒नूर् यत् । यथ् सु॑रभि॒मती᳚ । सु॒र॒भि॒मती॒ तया᳚ । सु॒र॒भि॒मतीति॑ सुरभि - मती᳚ । तयै॒व । ए॒वास्मै᳚ । अ॒स्मै॒ भे॒ष॒जम् । भे॒ष॒जम् क॑रोति । क॒रो॒ति॒ सु॒र॒भि॒मते᳚ । सु॒र॒भि॒मते॑ भवति । सु॒र॒भि॒मत॒ इति॑ सुरभि - मते᳚ । भ॒व॒ति॒ पू॒ती॒ग॒न्धस्य॑ । पू॒ती॒ग॒न्धस्याप॑हत्यै । पू॒ती॒ग॒न्धस्येति॑ पूति - ग॒न्धस्य॑ । अप॑हत्या अ॒ग्नये᳚ । अप॑हत्या॒ इत्यप॑ - ह॒त्यै॒ । अ॒ग्नये॒ क्षाम॑वते । क्षाम॑वते पुरो॒डाश᳚म् । क्षाम॑वत॒ इति॒ क्षाम॑ - व॒ते॒ । पु॒रो॒डाश॑म॒ष्टाक॑पालम् । अ॒ष्टाक॑पाल॒म् निः । अ॒ष्टाक॑पाल॒मित्य॒ष्टा - क॒पा॒ल॒म् । निर् व॑पेत् । व॒पे॒थ् स॒ङ्ग्रा॒मे । स॒ङ्ग्रा॒मे संॅय॑त्ते । स॒ङ्ग्रा॒म इति॑ सं - ग्रा॒मे । संॅय॑त्ते भाग॒धेये॑न । संॅय॑त्त॒ इति॒ सं - य॒त्ते॒ । भा॒ग॒धेये॑नै॒व । भा॒ग॒धेये॒नेति॑ भाग - धेये॑न । ए॒वैन᳚म् । ए॒नꣳ॒॒ श॒म॒यि॒त्वा । श॒म॒यि॒त्वा परान्॑ । परा॑न॒भि । अ॒भि निः । निर् दि॑शति । दि॒श॒ति॒ यम् । यमव॑रेषाम् । अव॑रेषां॒ ॅविध्य॑न्ति । विध्य॑न्ति॒ जीव॑ति । जीव॑ति॒ सः । स यम् । यम् परे॑षाम् । परे॑षा॒म् प्र । प्र सः । स मी॑यते । मी॒य॒ते॒ जय॑ति । जय॑ति॒ तम् । तꣳ स॑ङ्ग्रा॒मम् । स॒ङ्ग्रा॒मम॒भि । स॒ङ्ग्रा॒ममिति॑ सं - ग्रा॒मम् \newline

\textbf{Jatai Paata} \newline

1. वा॒ प्र॒मीये॑रन् प्र॒मीये॑रन्. वा वा प्र॒मीये॑रन्न् । \newline
2. प्र॒मीये॑र॒न्॒. यो यः प्र॒मीये॑रन् प्र॒मीये॑र॒न्॒. यः । \newline
3. प्र॒मीये॑र॒न्निति॑ प्र - मीये॑रन्न् । \newline
4. यो वा॑ वा॒ यो यो वा᳚ । \newline
5. वा॒ बि॒भी॒याद् बि॑भी॒याद् वा॑ वा बिभी॒यात् । \newline
6. बि॒भी॒या दे॒षैषा बि॑भी॒याद् बि॑भी॒या दे॒षा । \newline
7. ए॒षा वै वा ए॒षैषा वै । \newline
8. वा अ॑स्यास्य॒ वै वा अ॑स्य । \newline
9. अ॒स्य॒ भे॒ष॒ज्या॑ भेष॒ज्या᳚ ऽस्यास्य भेष॒ज्या᳚ । \newline
10. भे॒ष॒ज्या॑ त॒नू स्त॒नूर् भे॑ष॒ज्या॑ भेष॒ज्या॑ त॒नूः । \newline
11. त॒नूर् यद् यत् त॒नू स्त॒नूर् यत् । \newline
12. यथ् सु॑रभि॒मती॑ सुरभि॒मती॒ यद् यथ् सु॑रभि॒मती᳚ । \newline
13. सु॒र॒भि॒मती॒ तया॒ तया॑ सुरभि॒मती॑ सुरभि॒मती॒ तया᳚ । \newline
14. सु॒र॒भि॒मतीति॑ सुरभि - मती᳚ । \newline
15. तयै॒वैव तया॒ तयै॒व । \newline
16. ए॒वास्मा॑ अस्मा ए॒वैवास्मै᳚ । \newline
17. अ॒स्मै॒ भे॒ष॒जम् भे॑ष॒ज म॑स्मा अस्मै भेष॒जम् । \newline
18. भे॒ष॒जम् क॑रोति करोति भेष॒जम् भे॑ष॒जम् क॑रोति । \newline
19. क॒रो॒ति॒ सु॒र॒भि॒मते॑ सुरभि॒मते॑ करोति करोति सुरभि॒मते᳚ । \newline
20. सु॒र॒भि॒मते॑ भवति भवति सुरभि॒मते॑ सुरभि॒मते॑ भवति । \newline
21. सु॒र॒भि॒मत॒ इति॑ सुरभि - मते᳚ । \newline
22. भ॒व॒ति॒ पू॒ती॒ग॒न्धस्य॑ पूतीग॒न्धस्य॑ भवति भवति पूतीग॒न्धस्य॑ । \newline
23. पू॒ती॒ग॒न्धस्या प॑हत्या॒ अप॑हत्यै पूतीग॒न्धस्य॑ पूतीग॒न्धस्या प॑हत्यै । \newline
24. पू॒ती॒ग॒न्धस्येति॑ पूति - ग॒न्धस्य॑ । \newline
25. अप॑हत्या अ॒ग्नये॒ ऽग्नये ऽप॑हत्या॒ अप॑हत्या अ॒ग्नये᳚ । \newline
26. अप॑हत्या॒ इत्यप॑ - ह॒त्यै॒ । \newline
27. अ॒ग्नये॒ क्षाम॑वते॒ क्षाम॑वते॒ ऽग्नये॒ ऽग्नये॒ क्षाम॑वते । \newline
28. क्षाम॑वते पुरो॒डाश॑म् पुरो॒डाश॒म् क्षाम॑वते॒ क्षाम॑वते पुरो॒डाश᳚म् । \newline
29. क्षाम॑वत॒ इति॒ क्षाम॑ - व॒ते॒ । \newline
30. पु॒रो॒डाश॑ म॒ष्टाक॑पाल म॒ष्टाक॑पालम् पुरो॒डाश॑म् पुरो॒डाश॑ म॒ष्टाक॑पालम् । \newline
31. अ॒ष्टाक॑पाल॒म् निर् णिर॒ष्टाक॑पाल म॒ष्टाक॑पाल॒म् निः । \newline
32. अ॒ष्टाक॑पाल॒मित्य॒ष्टा - क॒पा॒ल॒म् । \newline
33. निर् व॑पेद् वपे॒न् निर् णिर् व॑पेत् । \newline
34. व॒पे॒थ् स॒ङ्ग्रा॒मे स॑ङ्ग्रा॒मे व॑पेद् वपेथ् सङ्ग्रा॒मे । \newline
35. स॒ङ्ग्रा॒मे संॅय॑त्ते॒ संॅय॑त्ते सङ्ग्रा॒मे स॑ङ्ग्रा॒मे संॅय॑त्ते । \newline
36. स॒ङ्ग्रा॒म इति॑ सं - ग्रा॒मे । \newline
37. संॅय॑त्ते भाग॒धेये॑न भाग॒धेये॑न॒ संॅय॑त्ते॒ संॅय॑त्ते भाग॒धेये॑न । \newline
38. संॅय॑त्त॒ इति॒ सं - य॒त्ते॒ । \newline
39. भा॒ग॒धेये॑नै॒वैव भा॑ग॒धेये॑न भाग॒धेये॑नै॒व । \newline
40. भा॒ग॒धेये॒नेति॑ भाग - धेये॑न । \newline
41. ए॒वैन॑ मेन मे॒वैवैन᳚म् । \newline
42. ए॒नꣳ॒॒ श॒म॒यि॒त्वा श॑मयि॒त्वैन॑ मेनꣳ शमयि॒त्वा । \newline
43. श॒म॒यि॒त्वा परा॒न् परा᳚ञ् छमयि॒त्वा श॑मयि॒त्वा परान्॑ । \newline
44. परा॑ न॒भ्य॑भि परा॒न् परा॑ न॒भि । \newline
45. अ॒भि निर् णि र॒भ्य॑भि निः । \newline
46. निर् दि॑शति दिशति॒ निर् णिर् दि॑शति । \newline
47. दि॒श॒ति॒ यं ॅयम् दि॑शति दिशति॒ यम् । \newline
48. य मव॑रेषा॒ मव॑रेषां॒ ॅयं ॅय मव॑रेषाम् । \newline
49. अव॑रेषां॒ ॅविद्ध्य॑न्ति॒ विद्ध्य॒ न्त्यव॑रेषा॒ मव॑रेषां॒ ॅविद्ध्य॑न्ति । \newline
50. विद्ध्य॑न्ति॒ जीव॑ति॒ जीव॑ति॒ विद्ध्य॑न्ति॒ विद्ध्य॑न्ति॒ जीव॑ति । \newline
51. जीव॑ति॒ स स जीव॑ति॒ जीव॑ति॒ सः । \newline
52. स यं ॅयꣳ स स यम् । \newline
53. यम् परे॑षा॒म् परे॑षां॒ ॅयं ॅयम् परे॑षाम् । \newline
54. परे॑षा॒म् प्र प्र परे॑षा॒म् परे॑षा॒म् प्र । \newline
55. प्र स स प्र प्र सः । \newline
56. स मी॑यते मीयते॒ स स मी॑यते । \newline
57. मी॒य॒ते॒ जय॑ति॒ जय॑ति मीयते मीयते॒ जय॑ति । \newline
58. जय॑ति॒ तम् तम् जय॑ति॒ जय॑ति॒ तम् । \newline
59. तꣳ स॑ङ्ग्रा॒मꣳ स॑ङ्ग्रा॒मम् तम् तꣳ स॑ङ्ग्रा॒मम् । \newline
60. स॒ङ्ग्रा॒म म॒भ्य॑भि स॑ङ्ग्रा॒मꣳ स॑ङ्ग्रा॒म म॒भि । \newline
61. स॒ङ्ग्रा॒ममिति॑ सं - ग्रा॒मम् । \newline

\textbf{Ghana Paata } \newline

1. वा॒ प्र॒मीये॑रन् प्र॒मीये॑रन्. वा वा प्र॒मीये॑र॒न्॒. यो यः प्र॒मीये॑रन्. वा वा प्र॒मीये॑र॒न्॒. यः । \newline
2. प्र॒मीये॑र॒न्॒. यो यः प्र॒मीये॑रन् प्र॒मीये॑र॒न्॒. यो वा॑ वा॒ यः प्र॒मीये॑रन् प्र॒मीये॑र॒न्॒. यो वा᳚ । \newline
3. प्र॒मीये॑र॒न्निति॑ प्र - मीये॑रन्न् । \newline
4. यो वा॑ वा॒ यो यो वा॑ बिभी॒याद् बि॑भी॒याद् वा॒ यो यो वा॑ बिभी॒यात् । \newline
5. वा॒ बि॒भी॒याद् बि॑भी॒याद् वा॑ वा बिभी॒या दे॒षैषा बि॑भी॒याद् वा॑ वा बिभी॒या दे॒षा । \newline
6. बि॒भी॒या दे॒षैषा बि॑भी॒याद् बि॑भी॒या दे॒षा वै वा ए॒षा बि॑भी॒याद् बि॑भी॒या दे॒षा वै । \newline
7. ए॒षा वै वा ए॒षैषा वा अ॑स्यास्य॒ वा ए॒षैषा वा अ॑स्य । \newline
8. वा अ॑स्यास्य॒ वै वा अ॑स्य भेष॒ज्या॑ भेष॒ज्या᳚ ऽस्य॒ वै वा अ॑स्य भेष॒ज्या᳚ । \newline
9. अ॒स्य॒ भे॒ष॒ज्या॑ भेष॒ज्या᳚ ऽस्यास्य भेष॒ज्या॑ त॒नू स्त॒नूर् भे॑ष॒ज्या᳚ ऽस्यास्य भेष॒ज्या॑ त॒नूः । \newline
10. भे॒ष॒ज्या॑ त॒नू स्त॒नूर् भे॑ष॒ज्या॑ भेष॒ज्या॑ त॒नूर् यद् यत् त॒नूर् भे॑ष॒ज्या॑ भेष॒ज्या॑ त॒नूर् यत् । \newline
11. त॒नूर् यद् यत् त॒नू स्त॒नूर् यथ् सु॑रभि॒मती॑ सुरभि॒मती॒ यत् त॒नू स्त॒नूर् यथ् सु॑रभि॒मती᳚ । \newline
12. यथ् सु॑रभि॒मती॑ सुरभि॒मती॒ यद् यथ् सु॑रभि॒मती॒ तया॒ तया॑ सुरभि॒मती॒ यद् यथ् सु॑रभि॒मती॒ तया᳚ । \newline
13. सु॒र॒भि॒मती॒ तया॒ तया॑ सुरभि॒मती॑ सुरभि॒मती॒ तयै॒वैव तया॑ सुरभि॒मती॑ सुरभि॒मती॒ तयै॒व । \newline
14. सु॒र॒भि॒मतीति॑ सुरभि - मती᳚ । \newline
15. तयै॒वैव तया॒ तयै॒वास्मा॑ अस्मा ए॒व तया॒ तयै॒वास्मै᳚ । \newline
16. ए॒वास्मा॑ अस्मा ए॒वैवास्मै॑ भेष॒जम् भे॑ष॒ज म॑स्मा ए॒वैवास्मै॑ भेष॒जम् । \newline
17. अ॒स्मै॒ भे॒ष॒जम् भे॑ष॒ज म॑स्मा अस्मै भेष॒जम् क॑रोति करोति भेष॒ज म॑स्मा अस्मै भेष॒जम् क॑रोति । \newline
18. भे॒ष॒जम् क॑रोति करोति भेष॒जम् भे॑ष॒जम् क॑रोति सुरभि॒मते॑ सुरभि॒मते॑ करोति भेष॒जम् भे॑ष॒जम् क॑रोति सुरभि॒मते᳚ । \newline
19. क॒रो॒ति॒ सु॒र॒भि॒मते॑ सुरभि॒मते॑ करोति करोति सुरभि॒मते॑ भवति भवति सुरभि॒मते॑ करोति करोति सुरभि॒मते॑ भवति । \newline
20. सु॒र॒भि॒मते॑ भवति भवति सुरभि॒मते॑ सुरभि॒मते॑ भवति पूतीग॒न्धस्य॑ पूतीग॒न्धस्य॑ भवति सुरभि॒मते॑ सुरभि॒मते॑ भवति पूतीग॒न्धस्य॑ । \newline
21. सु॒र॒भि॒मत॒ इति॑ सुरभि - मते᳚ । \newline
22. भ॒व॒ति॒ पू॒ती॒ग॒न्धस्य॑ पूतीग॒न्धस्य॑ भवति भवति पूतीग॒न्धस्या प॑हत्या॒ अप॑हत्यै पूतीग॒न्धस्य॑ भवति भवति पूतीग॒न्धस्या प॑हत्यै । \newline
23. पू॒ती॒ग॒न्धस्या प॑हत्या॒ अप॑हत्यै पूतीग॒न्धस्य॑ पूतीग॒न्धस्या प॑हत्या अ॒ग्नये॒ ऽग्नये ऽप॑हत्यै पूतीग॒न्धस्य॑ पूतीग॒न्धस्या प॑हत्या अ॒ग्नये᳚ । \newline
24. पू॒ती॒ग॒न्धस्येति॑ पूति - ग॒न्धस्य॑ । \newline
25. अप॑हत्या अ॒ग्नये॒ ऽग्नये ऽप॑हत्या॒ अप॑हत्या अ॒ग्नये॒ क्षाम॑वते॒ क्षाम॑वते॒ ऽग्नये ऽप॑हत्या॒ अप॑हत्या अ॒ग्नये॒ क्षाम॑वते । \newline
26. अप॑हत्या॒ इत्यप॑ - ह॒त्यै॒ । \newline
27. अ॒ग्नये॒ क्षाम॑वते॒ क्षाम॑वते॒ ऽग्नये॒ ऽग्नये॒ क्षाम॑वते पुरो॒डाश॑म् पुरो॒डाश॒म् क्षाम॑वते॒ ऽग्नये॒ ऽग्नये॒ क्षाम॑वते पुरो॒डाश᳚म् । \newline
28. क्षाम॑वते पुरो॒डाश॑म् पुरो॒डाश॒म् क्षाम॑वते॒ क्षाम॑वते पुरो॒डाश॑ म॒ष्टाक॑पाल म॒ष्टाक॑पालम् पुरो॒डाश॒म् क्षाम॑वते॒ क्षाम॑वते पुरो॒डाश॑ म॒ष्टाक॑पालम् । \newline
29. क्षाम॑वत॒ इति॒ क्षाम॑ - व॒ते॒ । \newline
30. पु॒रो॒डाश॑ म॒ष्टाक॑पाल म॒ष्टाक॑पालम् पुरो॒डाश॑म् पुरो॒डाश॑ म॒ष्टाक॑पाल॒म् निर् णिर॒ष्टाक॑पालम् पुरो॒डाश॑म् पुरो॒डाश॑ म॒ष्टाक॑पाल॒म् निः । \newline
31. अ॒ष्टाक॑पाल॒म् निर् णिर॒ष्टाक॑पाल म॒ष्टाक॑पाल॒म् निर् व॑पेद् वपे॒न् निर॒ष्टाक॑पाल म॒ष्टाक॑पाल॒म् निर् व॑पेत् । \newline
32. अ॒ष्टाक॑पाल॒मित्य॒ष्टा - क॒पा॒ल॒म् । \newline
33. निर् व॑पेद् वपे॒न् निर् णिर् व॑पेथ् सङ्ग्रा॒मे स॑ङ्ग्रा॒मे व॑पे॒न् निर् णिर् व॑पेथ् सङ्ग्रा॒मे । \newline
34. व॒पे॒थ् स॒ङ्ग्रा॒मे स॑ङ्ग्रा॒मे व॑पेद् वपेथ् सङ्ग्रा॒मे संॅय॑त्ते॒ संॅय॑त्ते सङ्ग्रा॒मे व॑पेद् वपेथ् सङ्ग्रा॒मे संॅय॑त्ते । \newline
35. स॒ङ्ग्रा॒मे संॅय॑त्ते॒ संॅय॑त्ते सङ्ग्रा॒मे स॑ङ्ग्रा॒मे संॅय॑त्ते भाग॒धेये॑न भाग॒धेये॑न॒ संॅय॑त्ते सङ्ग्रा॒मे स॑ङ्ग्रा॒मे संॅय॑त्ते भाग॒धेये॑न । \newline
36. स॒ङ्ग्रा॒म इति॑ सं - ग्रा॒मे । \newline
37. संॅय॑त्ते भाग॒धेये॑न भाग॒धेये॑न॒ संॅय॑त्ते॒ संॅय॑त्ते भाग॒धेये॑ नै॒वैव भा॑ग॒धेये॑न॒ संॅय॑त्ते॒ संॅय॑त्ते भाग॒धेये॑नै॒व । \newline
38. संॅय॑त्त॒ इति॒ सं - य॒त्ते॒ । \newline
39. भा॒ग॒धेये॑नै॒वैव भा॑ग॒धेये॑न भाग॒धेये॑नै॒वैन॑ मेन मे॒व भा॑ग॒धेये॑न भाग॒धेये॑नै॒वैन᳚म् । \newline
40. भा॒ग॒धेये॒नेति॑ भाग - धेये॑न । \newline
41. ए॒वैन॑ मेन मे॒वैवैनꣳ॑ शमयि॒त्वा श॑मयि॒त्वैन॑ मे॒वैवैनꣳ॑ शमयि॒त्वा । \newline
42. ए॒नꣳ॒॒ श॒म॒यि॒त्वा श॑मयि॒त्वैन॑ मेनꣳ शमयि॒त्वा परा॒न् परा᳚ञ् छमयि॒त्वैन॑ मेनꣳ शमयि॒त्वा परान्॑ । \newline
43. श॒म॒यि॒त्वा परा॒न् परा᳚ञ् छमयि॒त्वा श॑मयि॒त्वा परा॑ न॒भ्य॑भि परा᳚ञ् छमयि॒त्वा श॑मयि॒त्वा परा॑ न॒भि । \newline
44. परा॑ न॒भ्य॑भि परा॒न् परा॑ न॒भि निर् णिर॒भि परा॒न् परा॑ न॒भि निः । \newline
45. अ॒भि निर् णिर॒भ्य॑भि निर् दि॑शति दिशति॒ निर॒भ्य॑भि निर् दि॑शति । \newline
46. निर् दि॑शति दिशति॒ निर् णिर् दि॑शति॒ यं ॅयम् दि॑शति॒ निर् णिर् दि॑शति॒ यम् । \newline
47. दि॒श॒ति॒ यं ॅयम् दि॑शति दिशति॒ य मव॑रेषा॒ मव॑रेषां॒ ॅयम् दि॑शति दिशति॒ य मव॑रेषाम् । \newline
48. य मव॑रेषा॒ मव॑रेषां॒ ॅयं ॅय मव॑रेषां॒ ॅविद्ध्य॑न्ति॒ विद्ध्य॒ न्त्यव॑रेषां॒ ॅयं ॅय मव॑रेषां॒ ॅविद्ध्य॑न्ति । \newline
49. अव॑रेषां॒ ॅविद्ध्य॑न्ति॒ विद्ध्य॒ न्त्यव॑रेषा॒ मव॑रेषां॒ ॅविद्ध्य॑न्ति॒ जीव॑ति॒ जीव॑ति॒ विद्ध्य॒ न्त्यव॑रेषा॒ मव॑रेषां॒ ॅविद्ध्य॑न्ति॒ जीव॑ति । \newline
50. विद्ध्य॑न्ति॒ जीव॑ति॒ जीव॑ति॒ विद्ध्य॑न्ति॒ विद्ध्य॑न्ति॒ जीव॑ति॒ स स जीव॑ति॒ विद्ध्य॑न्ति॒ विद्ध्य॑न्ति॒ जीव॑ति॒ सः । \newline
51. जीव॑ति॒ स स जीव॑ति॒ जीव॑ति॒ स यं ॅयꣳ स जीव॑ति॒ जीव॑ति॒ स यम् । \newline
52. स यं ॅयꣳ स स यम् परे॑षा॒म् परे॑षां॒ ॅयꣳ स स यम् परे॑षाम् । \newline
53. यम् परे॑षा॒म् परे॑षां॒ ॅयं ॅयम् परे॑षा॒म् प्र प्र परे॑षां॒ ॅयं ॅयम् परे॑षा॒म् प्र । \newline
54. परे॑षा॒म् प्र प्र परे॑षा॒म् परे॑षा॒म् प्र स स प्र परे॑षा॒म् परे॑षा॒म् प्र सः । \newline
55. प्र स स प्र प्र स मी॑यते मीयते॒ स प्र प्र स मी॑यते । \newline
56. स मी॑यते मीयते॒ स स मी॑यते॒ जय॑ति॒ जय॑ति मीयते॒ स स मी॑यते॒ जय॑ति । \newline
57. मी॒य॒ते॒ जय॑ति॒ जय॑ति मीयते मीयते॒ जय॑ति॒ तम् तम् जय॑ति मीयते मीयते॒ जय॑ति॒ तम् । \newline
58. जय॑ति॒ तम् तम् जय॑ति॒ जय॑ति॒ तꣳ स॑ङ्ग्रा॒मꣳ स॑ङ्ग्रा॒मम् तम् जय॑ति॒ जय॑ति॒ तꣳ स॑ङ्ग्रा॒मम् । \newline
59. तꣳ स॑ङ्ग्रा॒मꣳ स॑ङ्ग्रा॒मम् तम् तꣳ स॑ङ्ग्रा॒म म॒भ्य॑भि स॑ङ्ग्रा॒मम् तम् तꣳ स॑ङ्ग्रा॒म म॒भि । \newline
60. स॒ङ्ग्रा॒म म॒भ्य॑भि स॑ङ्ग्रा॒मꣳ स॑ङ्ग्रा॒म म॒भि वै वा अ॒भि स॑ङ्ग्रा॒मꣳ स॑ङ्ग्रा॒म म॒भि वै । \newline
61. स॒ङ्ग्रा॒ममिति॑ सं - ग्रा॒मम् । \newline
\pagebreak
\markright{ TS 2.2.2.5  \hfill https://www.vedavms.in \hfill}
\addcontentsline{toc}{section}{ TS 2.2.2.5 }
\section*{ TS 2.2.2.5 }

\textbf{TS 2.2.2.5 } \newline
\textbf{Samhita Paata} \newline

-म॒भि वा ए॒ष ए॒तानु॑च्यति॒ येषां᳚ पूर्वाप॒रा अ॒न्वञ्चः॑ प्र॒मीय॑न्ते पुरुषाहु॒तिर्ह्य॑स्य प्रि॒यत॑मा॒ग्नये॒ क्षाम॑वते पुरो॒डाश॑म॒ष्टाक॑पालं॒ निर्व॑पेद्-भाग॒धेये॑नै॒वैनꣳ॑ शमयति॒ नैषां᳚ पु॒राऽऽ*यु॒षोप॑रः॒ प्रमी॑यते॒ऽभि वा ए॒ष ए॒तस्य॑ गृ॒हानु॑च्यति॒ यस्य॑ गृ॒हान् दह॑त्य॒ग्नये॒ क्षाम॑वते पुरो॒डाश॑म॒ष्टाक॑पालं॒ निर्व॑पेद्-भाग॒धेये॑नै॒वैनꣳ॑ शमयति॒ ना ( )-स्याप॑रं गृ॒हान् द॑हति ॥ \newline

\textbf{Pada Paata} \newline

अ॒भीति॑ । वै । ए॒षः । ए॒तान् । उ॒च्य॒ति॒ । येषा᳚म् । पू॒र्वा॒प॒रा इति॑ पूर्व - अ॒प॒राः । अ॒न्वञ्चः॑ । प्र॒मीय॑न्त॒ इति॑ प्र - मीय॑न्ते । पु॒रु॒षा॒हु॒तिरिति॑ पुरुष - आ॒हु॒तिः । हि । अ॒स्य॒ । प्रि॒यत॒मेति॑ प्रि॒य - त॒मा॒ । अ॒ग्नये᳚ । क्षाम॑वत॒ इति॒ क्षाम॑ - व॒ते॒ । पु॒रो॒डाश᳚म् । अ॒ष्टाक॑पाल॒मित्य॒ष्टा - क॒पा॒ल॒म् । निरिति॑ । व॒पे॒त् । भा॒ग॒धेये॒नेति॑ भाग - धेये॑न । ए॒व । ए॒न॒म् । श॒म॒य॒ति॒ । न । ए॒षा॒म् । पु॒रा । आयु॑षः । अप॑रः । प्रेति॑ । मी॒य॒ते॒ । अ॒भीति॑ । वै । ए॒षः । ए॒तस्य॑ । गृ॒हान् । उ॒च्य॒ति॒ । यस्य॑ । गृ॒हान् । दह॑ति । अ॒ग्नये᳚ । क्षाम॑वत॒ इति॒ क्षाम॑ - व॒ते॒ । पु॒रो॒डाश᳚म् । अ॒ष्टाक॑पाल॒मित्य॒ष्टा - क॒पा॒ल॒म् । निरिति॑ । व॒पे॒त् । भा॒ग॒धेये॒नेति॑ भाग - धेये॑न । ए॒व । ए॒न॒म् । श॒य॒म॒ति॒ । न ( ) । अ॒स्य॒ । अप॑रम् । गृ॒हान् । द॒ह॒ति॒ ॥  \newline


\textbf{Krama Paata} \newline

अ॒भि वै । वा ए॒षः । ए॒ष ए॒तान् । ए॒तानु॑च्यति । उ॒च्य॒ति॒ येषा᳚म् । येषा᳚म् पूर्वाप॒राः । पू॒र्वा॒प॒रा अ॒न्वञ्चः॑ । पू॒र्वा॒प॒रा इति॑ पूर्व - अ॒प॒राः । अ॒न्वञ्चः॑ प्र॒मीय॑न्ते । प्र॒मीय॑न्ते पुरुषाहु॒तिः । प्र॒मीय॑न्त॒ इति॑ प्र - मीय॑न्ते । पु॒रु॒षा॒हु॒तिर्. हि । पु॒रु॒षा॒हु॒तिरिति॑ पुरुष - आ॒हु॒तिः । ह्य॑स्य । अ॒स्य॒ प्रि॒यत॑मा । प्रि॒यत॑मा॒ऽग्नये᳚ । प्रि॒यत॒मेति॑ प्रि॒य - त॒मा॒ । अ॒ग्नये॒ क्षाम॑वते । क्षाम॑वते पुरो॒डाश᳚म् । क्षाम॑वत॒ इति॒ क्षाम॑ - व॒ते॒ । पु॒रो॒डाश॑म॒ष्टाक॑पालम् । अ॒ष्टाक॑पाल॒म् निः । अ॒ष्टाक॑पाल॒मित्य॒ष्टा - क॒पा॒ल॒म् । निर् व॑पेत् । व॒पे॒द्,भा॒ग॒धेये॑न । भा॒ग॒धेये॑नै॒व । भा॒ग॒धेये॒नेति॑ भाग - धेये॑न । ए॒वैन᳚म् । ए॒नꣳ॒॒ श॒म॒य॒ति॒ । श॒म॒य॒ति॒ न । नैषा᳚म् । ए॒षा॒म् पु॒रा । पु॒रा ऽऽयु॑षः । आयु॒षो ऽप॑रः । अप॑रः॒ प्र । प्र मी॑यते । मी॒य॒ते॒ऽभि । अ॒भि वै । वा ए॒षः । ए॒ष ए॒तस्य॑ । ए॒तस्य॑ गृ॒हान् । गृ॒हानु॑च्यति । उ॒च्य॒ति॒ यस्य॑ । यस्य॑ गृ॒हान् । गृ॒हान् दह॑ति । दह॑त्य॒ग्नये᳚ । अ॒ग्नये॒ क्षाम॑वते । क्षाम॑वते पुरो॒डाश᳚म् । क्षाम॑वत॒ इति॒ क्षाम॑ - व॒ते॒ । पु॒रो॒डाश॑म॒ष्टाक॑पालम् । अ॒ष्टाक॑पाल॒म् निः । अ॒ष्टाक॑पाल॒मित्य॒ष्टा - क॒पा॒ल॒म् । निर् व॑पेत् । व॒पे॒द् भा॒ग॒धेये॑न । भा॒ग॒धेये॑नै॒व । भा॒ग॒धेये॒नेति॑ भाग - धेये॑न । ए॒वैन᳚म् । ए॒नꣳ॒॒ श॒म॒य॒ति॒ । श॒म॒य॒ति॒ न ( ) । नास्य॑ । अ॒स्याप॑रम् । अप॑रम् गृ॒हान् । गृ॒हान् द॑हति । द॒ह॒तीति॑ दहति । \newline

\textbf{Jatai Paata} \newline

1. अ॒भि वै वा अ॒भ्य॑भि वै । \newline
2. वा ए॒ष ए॒ष वै वा ए॒षः । \newline
3. ए॒ष ए॒ता ने॒ता ने॒ष ए॒ष ए॒तान् । \newline
4. ए॒ता नु॑च्य त्युच्य त्ये॒ता ने॒ता नु॑च्यति । \newline
5. उ॒च्य॒ति॒ येषां॒ ॅयेषा॑ मुच्य त्युच्यति॒ येषा᳚म् । \newline
6. येषा᳚म् पूर्वाप॒राः पू᳚र्वाप॒रा येषां॒ ॅयेषा᳚म् पूर्वाप॒राः । \newline
7. पू॒र्वा॒प॒रा अ॒न्वञ्चो॒ ऽन्वञ्चः॑ पूर्वाप॒राः पू᳚र्वाप॒रा अ॒न्वञ्चः॑ । \newline
8. पू॒र्वा॒प॒रा इति॑ पूर्व - अ॒प॒राः । \newline
9. अ॒न्वञ्चः॑ प्र॒मीय॑न्ते प्र॒मीय॑न्ते॒ ऽन्वञ्चो॒ ऽन्वञ्चः॑ प्र॒मीय॑न्ते । \newline
10. प्र॒मीय॑न्ते पुरुषाहु॒तिः पु॑रुषाहु॒तिः प्र॒मीय॑न्ते प्र॒मीय॑न्ते पुरुषाहु॒तिः । \newline
11. प्र॒मीय॑न्त॒ इति॑ प्र - मीय॑न्ते । \newline
12. पु॒रु॒षा॒हु॒तिर्. हि हि पु॑रुषाहु॒तिः पु॑रुषाहु॒तिर्. हि । \newline
13. पु॒रु॒षा॒हु॒तिरिति॑ पुरुष - आ॒हु॒तिः । \newline
14. ह्य॑स्यास्य॒ हि ह्य॑स्य । \newline
15. अ॒स्य॒ प्रि॒यत॑मा प्रि॒यत॑मा ऽस्यास्य प्रि॒यत॑मा । \newline
16. प्रि॒यत॑मा॒ ऽग्नये॒ ऽग्नये᳚ प्रि॒यत॑मा प्रि॒यत॑मा॒ ऽग्नये᳚ । \newline
17. प्रि॒यत॒मेति॑ प्रि॒य - त॒मा॒ । \newline
18. अ॒ग्नये॒ क्षाम॑वते॒ क्षाम॑वते॒ ऽग्नये॒ ऽग्नये॒ क्षाम॑वते । \newline
19. क्षाम॑वते पुरो॒डाश॑म् पुरो॒डाश॒म् क्षाम॑वते॒ क्षाम॑वते पुरो॒डाश᳚म् । \newline
20. क्षाम॑वत॒ इति॒ क्षाम॑ - व॒ते॒ । \newline
21. पु॒रो॒डाश॑ म॒ष्टाक॑पाल म॒ष्टाक॑पालम् पुरो॒डाश॑म् पुरो॒डाश॑ म॒ष्टाक॑पालम् । \newline
22. अ॒ष्टाक॑पाल॒म् निर् णिर॒ष्टाक॑पाल म॒ष्टाक॑पाल॒म् निः । \newline
23. अ॒ष्टाक॑पाल॒मित्य॒ष्टा - क॒पा॒ल॒म् । \newline
24. निर् व॑पेद् वपे॒न् निर् णिर् व॑पेत् । \newline
25. व॒पे॒द् भा॒ग॒धेये॑न भाग॒धेये॑न वपेद् वपेद् भाग॒धेये॑न । \newline
26. भा॒ग॒धेये॑नै॒वैव भा॑ग॒धेये॑न भाग॒धेये॑नै॒व । \newline
27. भा॒ग॒धेये॒नेति॑ भाग - धेये॑न । \newline
28. ए॒वैन॑ मेन मे॒वैवैन᳚म् । \newline
29. ए॒नꣳ॒॒ श॒म॒य॒ति॒ श॒म॒य॒त्ये॒न॒ मे॒नꣳ॒॒ श॒म॒य॒ति॒ । \newline
30. श॒म॒य॒ति॒ न न श॑मयति शमयति॒ न । \newline
31. नैषा॑ मेषा॒म् न नैषा᳚म् । \newline
32. ए॒षा॒म् पु॒रा पु॒रैषा॑ मेषाम् पु॒रा । \newline
33. पु॒रा ऽऽयु॑ष॒ आयु॑षः पु॒रा पु॒रा ऽऽयु॑षः । \newline
34. आयु॒षो ऽप॒रो ऽप॑र॒ आयु॑ष॒ आयु॒षो ऽप॑रः । \newline
35. अप॑रः॒ प्र प्रा प॒रो ऽप॑रः॒ प्र । \newline
36. प्र मी॑यते मीयते॒ प्र प्र मी॑यते । \newline
37. मी॒य॒ते॒ ऽभ्य॑भि मी॑यते मीयते॒ ऽभि । \newline
38. अ॒भि वै वा अ॒भ्य॑भि वै । \newline
39. वा ए॒ष ए॒ष वै वा ए॒षः । \newline
40. ए॒ष ए॒त स्यै॒त स्यै॒ष ए॒ष ए॒तस्य॑ । \newline
41. ए॒तस्य॑ गृ॒हान् गृ॒हा ने॒त स्यै॒तस्य॑ गृ॒हान् । \newline
42. गृ॒हा नु॑च्य त्युच्यति गृ॒हान् गृ॒हा नु॑च्यति । \newline
43. उ॒च्य॒ति॒ यस्य॒ यस्यो᳚ च्यत्युच्यति॒ यस्य॑ । \newline
44. यस्य॑ गृ॒हान् गृ॒हान्. यस्य॒ यस्य॑ गृ॒हान् । \newline
45. गृ॒हान् दह॑ति॒ दह॑ति गृ॒हान् गृ॒हान् दह॑ति । \newline
46. दह॑ त्य॒ग्नये॒ ऽग्नये॒ दह॑ति॒ दह॑ त्य॒ग्नये᳚ । \newline
47. अ॒ग्नये॒ क्षाम॑वते॒ क्षाम॑वते॒ ऽग्नये॒ ऽग्नये॒ क्षाम॑वते । \newline
48. क्षाम॑वते पुरो॒डाश॑म् पुरो॒डाश॒म् क्षाम॑वते॒ क्षाम॑वते पुरो॒डाश᳚म् । \newline
49. क्षाम॑वत॒ इति॒ क्षाम॑ - व॒ते॒ । \newline
50. पु॒रो॒डाश॑ म॒ष्टाक॑पाल म॒ष्टाक॑पालम् पुरो॒डाश॑म् पुरो॒डाश॑ म॒ष्टाक॑पालम् । \newline
51. अ॒ष्टाक॑पाल॒म् निर् णिर॒ष्टाक॑पाल म॒ष्टाक॑पाल॒म् निः । \newline
52. अ॒ष्टाक॑पाल॒मित्य॒ष्टा - क॒पा॒ल॒म् । \newline
53. निर् व॑पेद् वपे॒न् निर् णिर् व॑पेत् । \newline
54. व॒पे॒द् भा॒ग॒धेये॑न भाग॒धेये॑न वपेद् वपेद् भाग॒धेये॑न । \newline
55. भा॒ग॒धेये॑नै॒वैव भा॑ग॒धेये॑न भाग॒धेये॑नै॒व । \newline
56. भा॒ग॒धेये॒नेति॑ भाग - धेये॑न । \newline
57. ए॒वैन॑ मेन मे॒वैवैन᳚म् । \newline
58. ए॒नꣳ॒॒ श॒म॒य॒ति॒ श॒म॒य॒त्ये॒न॒ मे॒नꣳ॒॒ श॒म॒य॒ति॒ । \newline
59. श॒म॒य॒ति॒ न न श॑मयति शमयति॒ न । \newline
60. नास्या᳚स्य॒ न नास्य॑ । \newline
61. अ॒स्याप॑र॒ मप॑र मस्या॒स्याप॑रम् । \newline
62. अप॑रम् गृ॒हान् गृ॒हा नप॑र॒ मप॑रम् गृ॒हान् । \newline
63. गृ॒हान् द॑हति दहति गृ॒हान् गृ॒हान् द॑हति । \newline
64. द॒ह॒तीति॑ दहति । \newline

\textbf{Ghana Paata } \newline

1. अ॒भि वै वा अ॒भ्य॑भि वा ए॒ष ए॒ष वा अ॒भ्य॑भि वा ए॒षः । \newline
2. वा ए॒ष ए॒ष वै वा ए॒ष ए॒ता ने॒ता ने॒ष वै वा ए॒ष ए॒तान् । \newline
3. ए॒ष ए॒ता ने॒ता ने॒ष ए॒ष ए॒ता नु॑च्य त्युच्य त्ये॒ता ने॒ष ए॒ष ए॒ता नु॑च्यति । \newline
4. ए॒ता नु॑च्य त्युच्य त्ये॒ता ने॒ता नु॑च्यति॒ येषां॒ ॅयेषा॑ मुच्य त्ये॒ता ने॒ता नु॑च्यति॒ येषा᳚म् । \newline
5. उ॒च्य॒ति॒ येषां॒ ॅयेषा॑ मुच्य त्युच्यति॒ येषा᳚म् पूर्वाप॒राः पू᳚र्वाप॒रा येषा॑ मुच्य त्युच्यति॒ येषा᳚म् पूर्वाप॒राः । \newline
6. येषा᳚म् पूर्वाप॒राः पू᳚र्वाप॒रा येषां॒ ॅयेषा᳚म् पूर्वाप॒रा अ॒न्वञ्चो॒ ऽन्वञ्चः॑ पूर्वाप॒रा येषां॒ ॅयेषा᳚म् पूर्वाप॒रा अ॒न्वञ्चः॑ । \newline
7. पू॒र्वा॒प॒रा अ॒न्वञ्चो॒ ऽन्वञ्चः॑ पूर्वाप॒राः पू᳚र्वाप॒रा अ॒न्वञ्चः॑ प्र॒मीय॑न्ते प्र॒मीय॑न्ते॒ ऽन्वञ्चः॑ पूर्वाप॒राः पू᳚र्वाप॒रा अ॒न्वञ्चः॑ प्र॒मीय॑न्ते । \newline
8. पू॒र्वा॒प॒रा इति॑ पूर्व - अ॒प॒राः । \newline
9. अ॒न्वञ्चः॑ प्र॒मीय॑न्ते प्र॒मीय॑न्ते॒ ऽन्वञ्चो॒ ऽन्वञ्चः॑ प्र॒मीय॑न्ते पुरुषाहु॒तिः पु॑रुषाहु॒तिः प्र॒मीय॑न्ते॒ ऽन्वञ्चो॒ ऽन्वञ्चः॑ प्र॒मीय॑न्ते पुरुषाहु॒तिः । \newline
10. प्र॒मीय॑न्ते पुरुषाहु॒तिः पु॑रुषाहु॒तिः प्र॒मीय॑न्ते प्र॒मीय॑न्ते पुरुषाहु॒तिर्. हि हि पु॑रुषाहु॒तिः प्र॒मीय॑न्ते प्र॒मीय॑न्ते पुरुषाहु॒तिर्. हि । \newline
11. प्र॒मीय॑न्त॒ इति॑ प्र - मीय॑न्ते । \newline
12. पु॒रु॒षा॒हु॒तिर्. हि हि पु॑रुषाहु॒तिः पु॑रुषाहु॒तिर् ह्य॑स्यास्य॒ हि पु॑रुषाहु॒तिः पु॑रुषाहु॒तिर् ह्य॑स्य । \newline
13. पु॒रु॒षा॒हु॒तिरिति॑ पुरुष - आ॒हु॒तिः । \newline
14. ह्य॑स्यास्य॒ हि ह्य॑स्य प्रि॒यत॑मा प्रि॒यत॑मा ऽस्य॒ हि ह्य॑स्य प्रि॒यत॑मा । \newline
15. अ॒स्य॒ प्रि॒यत॑मा प्रि॒यत॑मा ऽस्यास्य प्रि॒यत॑मा॒ ऽग्नये॒ ऽग्नये᳚ प्रि॒यत॑मा ऽस्यास्य प्रि॒यत॑मा॒ ऽग्नये᳚ । \newline
16. प्रि॒यत॑मा॒ ऽग्नये॒ ऽग्नये᳚ प्रि॒यत॑मा प्रि॒यत॑मा॒ ऽग्नये॒ क्षाम॑वते॒ क्षाम॑वते॒ ऽग्नये᳚ प्रि॒यत॑मा प्रि॒यत॑मा॒ ऽग्नये॒ क्षाम॑वते । \newline
17. प्रि॒यत॒मेति॑ प्रि॒य - त॒मा॒ । \newline
18. अ॒ग्नये॒ क्षाम॑वते॒ क्षाम॑वते॒ ऽग्नये॒ ऽग्नये॒ क्षाम॑वते पुरो॒डाश॑म् पुरो॒डाश॒म् क्षाम॑वते॒ ऽग्नये॒ ऽग्नये॒ क्षाम॑वते पुरो॒डाश᳚म् । \newline
19. क्षाम॑वते पुरो॒डाश॑म् पुरो॒डाश॒म् क्षाम॑वते॒ क्षाम॑वते पुरो॒डाश॑ म॒ष्टाक॑पाल म॒ष्टाक॑पालम् पुरो॒डाश॒म् क्षाम॑वते॒ क्षाम॑वते पुरो॒डाश॑ म॒ष्टाक॑पालम् । \newline
20. क्षाम॑वत॒ इति॒ क्षाम॑ - व॒ते॒ । \newline
21. पु॒रो॒डाश॑ म॒ष्टाक॑पाल म॒ष्टाक॑पालम् पुरो॒डाश॑म् पुरो॒डाश॑ म॒ष्टाक॑पाल॒म् निर् णिर॒ष्टाक॑पालम् पुरो॒डाश॑म् पुरो॒डाश॑ म॒ष्टाक॑पाल॒म् निः । \newline
22. अ॒ष्टाक॑पाल॒म् निर् णिर॒ष्टाक॑पाल म॒ष्टाक॑पाल॒म् निर् व॑पेद् वपे॒न् निर॒ष्टाक॑पाल म॒ष्टाक॑पाल॒म् निर् व॑पेत् । \newline
23. अ॒ष्टाक॑पाल॒मित्य॒ष्टा - क॒पा॒ल॒म् । \newline
24. निर् व॑पेद् वपे॒न् निर् णिर् व॑पेद् भाग॒धेये॑न भाग॒धेये॑न वपे॒न् निर् णिर् व॑पेद् भाग॒धेये॑न । \newline
25. व॒पे॒द् भा॒ग॒धेये॑न भाग॒धेये॑न वपेद् वपेद् भाग॒धेये॑ नै॒वैव भा॑ग॒धेये॑न वपेद् वपेद् भाग॒धेये॑नै॒व । \newline
26. भा॒ग॒धेये॑ नै॒वैव भा॑ग॒धेये॑न भाग॒धेये॑ नै॒वैन॑ मेन मे॒व भा॑ग॒धेये॑न भाग॒धेये॑ नै॒वैन᳚म् । \newline
27. भा॒ग॒धेये॒नेति॑ भाग - धेये॑न । \newline
28. ए॒वैन॑ मेन मे॒वैवैनꣳ॑ शमयति शमयत्येन मे॒वैवैनꣳ॑ शमयति । \newline
29. ए॒नꣳ॒॒ श॒म॒य॒ति॒ श॒म॒य॒त्ये॒न॒ मे॒नꣳ॒॒ श॒म॒य॒ति॒ न न श॑मयत्येन मेनꣳ शमयति॒ न । \newline
30. श॒म॒य॒ति॒ न न श॑मयति शमयति॒ नैषा॑ मेषा॒म् न श॑मयति शमयति॒ नैषा᳚म् । \newline
31. नैषा॑ मेषा॒म् न नैषा᳚म् पु॒रा पु॒रैषा॒म् न नैषा᳚म् पु॒रा । \newline
32. ए॒षा॒म् पु॒रा पु॒रैषा॑ मेषाम् पु॒रा ऽऽयु॑ष॒ आयु॑षः पु॒रैषा॑ मेषाम् पु॒रा ऽऽयु॑षः । \newline
33. पु॒रा ऽऽयु॑ष॒ आयु॑षः पु॒रा पु॒रा ऽऽयु॒षो ऽप॒रो ऽप॑र॒ आयु॑षः पु॒रा पु॒रा ऽऽयु॒षो ऽप॑रः । \newline
34. आयु॒षो ऽप॒रो ऽप॑र॒ आयु॑ष॒ आयु॒षो ऽप॑रः॒ प्र प्राप॑र॒ आयु॑ष॒ आयु॒षो ऽप॑रः॒ प्र । \newline
35. अप॑रः॒ प्र प्राप॒रो ऽप॑रः॒ प्र मी॑यते मीयते॒ प्राप॒रो ऽप॑रः॒ प्र मी॑यते । \newline
36. प्र मी॑यते मीयते॒ प्र प्र मी॑यते॒ ऽभ्य॑भि मी॑यते॒ प्र प्र मी॑यते॒ ऽभि । \newline
37. मी॒य॒ते॒ ऽभ्य॑भि मी॑यते मीयते॒ ऽभि वै वा अ॒भि मी॑यते मीयते॒ ऽभि वै । \newline
38. अ॒भि वै वा अ॒भ्य॑भि वा ए॒ष ए॒ष वा अ॒भ्य॑भि वा ए॒षः । \newline
39. वा ए॒ष ए॒ष वै वा ए॒ष ए॒त स्यै॒त स्यै॒ष वै वा ए॒ष ए॒तस्य॑ । \newline
40. ए॒ष ए॒त स्यै॒त स्यै॒ष ए॒ष ए॒तस्य॑ गृ॒हान् गृ॒हा ने॒त स्यै॒ष ए॒ष ए॒तस्य॑ गृ॒हान् । \newline
41. ए॒तस्य॑ गृ॒हान् गृ॒हा ने॒तस्यै॒तस्य॑ गृ॒हा नु॑च्य त्युच्यति गृ॒हा ने॒त स्यै॒तस्य॑ गृ॒हा नु॑च्यति । \newline
42. गृ॒हा नु॑च्य त्युच्यति गृ॒हान् गृ॒हा नु॑च्यति॒ यस्य॒ यस्यो᳚च्यति गृ॒हान् गृ॒हा नु॑च्यति॒ यस्य॑ । \newline
43. उ॒च्य॒ति॒ यस्य॒ यस्यो᳚च्य त्युच्यति॒ यस्य॑ गृ॒हान् गृ॒हान्. यस्यो᳚च्य त्युच्यति॒ यस्य॑ गृ॒हान् । \newline
44. यस्य॑ गृ॒हान् गृ॒हान्. यस्य॒ यस्य॑ गृ॒हान् दह॑ति॒ दह॑ति गृ॒हान्. यस्य॒ यस्य॑ गृ॒हान् दह॑ति । \newline
45. गृ॒हान् दह॑ति॒ दह॑ति गृ॒हान् गृ॒हान् दह॑त्य॒ग्नये॒ ऽग्नये॒ दह॑ति गृ॒हान् गृ॒हान् दह॑त्य॒ग्नये᳚ । \newline
46. दह॑त्य॒ग्नये॒ ऽग्नये॒ दह॑ति॒ दह॑त्य॒ग्नये॒ क्षाम॑वते॒ क्षाम॑वते॒ ऽग्नये॒ दह॑ति॒ दह॑त्य॒ग्नये॒ क्षाम॑वते । \newline
47. अ॒ग्नये॒ क्षाम॑वते॒ क्षाम॑वते॒ ऽग्नये॒ ऽग्नये॒ क्षाम॑वते पुरो॒डाश॑म् पुरो॒डाश॒म् क्षाम॑वते॒ ऽग्नये॒ ऽग्नये॒ क्षाम॑वते पुरो॒डाश᳚म् । \newline
48. क्षाम॑वते पुरो॒डाश॑म् पुरो॒डाश॒म् क्षाम॑वते॒ क्षाम॑वते पुरो॒डाश॑ म॒ष्टाक॑पाल म॒ष्टाक॑पालम् पुरो॒डाश॒म् क्षाम॑वते॒ क्षाम॑वते पुरो॒डाश॑ म॒ष्टाक॑पालम् । \newline
49. क्षाम॑वत॒ इति॒ क्षाम॑ - व॒ते॒ । \newline
50. पु॒रो॒डाश॑ म॒ष्टाक॑पाल म॒ष्टाक॑पालम् पुरो॒डाश॑म् पुरो॒डाश॑ म॒ष्टाक॑पाल॒म् निर् णिर॒ष्टाक॑पालम् पुरो॒डाश॑म् पुरो॒डाश॑ म॒ष्टाक॑पाल॒म् निः । \newline
51. अ॒ष्टाक॑पाल॒म् निर् णिर॒ष्टाक॑पाल म॒ष्टाक॑पाल॒म् निर् व॑पेद् वपे॒न् निर॒ष्टाक॑पाल म॒ष्टाक॑पाल॒म् निर् व॑पेत् । \newline
52. अ॒ष्टाक॑पाल॒मित्य॒ष्टा - क॒पा॒ल॒म् । \newline
53. निर् व॑पेद् वपे॒न् निर् णिर् व॑पेद् भाग॒धेये॑न भाग॒धेये॑न वपे॒न् निर् णिर् व॑पेद् भाग॒धेये॑न । \newline
54. व॒पे॒द् भा॒ग॒धेये॑न भाग॒धेये॑न वपेद् वपेद् भाग॒धेये॑ नै॒वैव भा॑ग॒धेये॑न वपेद् वपेद् भाग॒धेये॑ नै॒व । \newline
55. भा॒ग॒धेये॑ नै॒वैव भा॑ग॒धेये॑न भाग॒धेये॑ नै॒वैन॑ मेन मे॒व भा॑ग॒धेये॑न भाग॒धेये॑ नै॒वैन᳚म् । \newline
56. भा॒ग॒धेये॒नेति॑ भाग - धेये॑न । \newline
57. ए॒वैन॑ मेन मे॒वैवैनꣳ॑ शमयति शमयत्येन मे॒वैवैनꣳ॑ शमयति । \newline
58. ए॒नꣳ॒॒ श॒म॒य॒ति॒ श॒म॒य॒त्ये॒न॒ मे॒नꣳ॒॒ श॒म॒य॒ति॒ न न श॑मयत्येन मेनꣳ शमयति॒ न । \newline
59. श॒म॒य॒ति॒ न न श॑मयति शमयति॒ नास्या᳚स्य॒ न श॑मयति शमयति॒ नास्य॑ । \newline
60. नास्या᳚स्य॒ न नास्या प॑र॒ मप॑र मस्य॒ न नास्याप॑रम् । \newline
61. अ॒स्याप॑र॒ मप॑र मस्या॒स्याप॑रम् गृ॒हान् गृ॒हा नप॑र मस्या॒स्याप॑रम् गृ॒हान् । \newline
62. अप॑रम् गृ॒हान् गृ॒हा नप॑र॒ मप॑रम् गृ॒हान् द॑हति दहति गृ॒हा नप॑र॒ मप॑रम् गृ॒हान् द॑हति । \newline
63. गृ॒हान् द॑हति दहति गृ॒हान् गृ॒हान् द॑हति । \newline
64. द॒ह॒तीति॑ दहति । \newline
\pagebreak
\markright{ TS 2.2.3.1  \hfill https://www.vedavms.in \hfill}
\addcontentsline{toc}{section}{ TS 2.2.3.1 }
\section*{ TS 2.2.3.1 }

\textbf{TS 2.2.3.1 } \newline
\textbf{Samhita Paata} \newline

अ॒ग्नये॒ कामा॑य पुरो॒डाश॑म॒ष्टाक॑पालं॒ निर्व॑पे॒द्यं कामो॒ नोप॒नमे॑द॒ग्निमे॒व कामꣳ॒॒ स्वेन॑ भाग॒धेये॒नोप॑ धावति॒ स ए॒वैनं॒ कामे॑न॒ सम॑र्द्धय॒त्युपै॑नं॒ कामो॑ नमत्य॒ग्नये॒ य वि॑ष्ठाय पुरो॒डाश॑म॒ष्टाक॑पालं॒ निर्व॑पे॒थ् स्पर्द्ध॑मानः॒ क्षेत्रे॑ वा सजा॒तेषु॑ वा॒ऽग्निमे॒व यवि॑ष्ठꣳ॒॒ स्वेन॑ भाग॒धेये॒नोप॑ धावति॒ तेनै॒वेन्द्रि॒यं ॅवी॒र्यं॑ भ्रातृ॑व्यस्य - [  ] \newline

\textbf{Pada Paata} \newline

अ॒ग्नये᳚ । कामा॑य । पु॒रो॒डाश᳚म् । अ॒ष्टाक॑पाल॒मित्य॒ष्टा - क॒पा॒ल॒म् । निरिति॑ । व॒पे॒त् । यम् । कामः॑ । न । उ॒प॒नमे॒दित्यु॑प - नमे᳚त् । अ॒ग्निम् । ए॒व । काम᳚म् । स्वेन॑ । भा॒ग॒धेये॒नेति॑ भाग - धेये॑न । उपेति॑ । धा॒व॒ति॒ । सः । ए॒व । ए॒न॒म् । कामे॑न । समिति॑ । अ॒र्द्ध॒य॒ति॒ । उपेति॑ । ए॒न॒म् । कामः॑ । न॒म॒ति॒ । अ॒ग्नये᳚ । यवि॑ष्ठाय । पु॒रो॒डाश᳚म् । अ॒ष्टाक॑पाल॒मित्य॒ष्टा-क॒पा॒ल॒म् । निरिति॑ । व॒पे॒त् । स्पर्द्ध॑मानः । क्षेत्रे᳚ । वा॒ । स॒जा॒तेष्विति॑ स - जा॒तेषु॑ । वा॒ । अ॒ग्निम् । ए॒व । यवि॑ष्ठम् । स्वेन॑ । भा॒ग॒धेये॒नेति॑ भाग - धेये॑न । उपेति॑ । धा॒व॒ति॒ । तेन॑ । ए॒व । इ॒न्द्रि॒यम् । वी॒र्य᳚म् । भ्रातृ॑व्यस्य ।  \newline


\textbf{Krama Paata} \newline

अ॒ग्नये॒ कामा॑य । कामा॑य पुरो॒डाश᳚म् । पु॒रो॒डाश॑म॒ष्टाक॑पालम् । अ॒ष्टाक॑पाल॒म् निः । अ॒ष्टाक॑पाल॒मित्य॒ष्टा - क॒पा॒ल॒म् । निर् व॑पेत् । व॒पे॒द् यम् । यम् कामः॑ । कामो॒ न । नोप॒नमे᳚त् । उ॒प॒नमे॑द॒ग्निम् । उ॒प॒नमे॒दित्यु॑प - नमे᳚त् । अ॒ग्निमे॒व । ए॒व काम᳚म् । कामꣳ॒॒ स्वेन॑ । स्वेन॑ भाग॒धेये॑न । भा॒ग॒धेये॒नोप॑ । भा॒ग॒धेये॒नेति॑ भाग - धेये॑न । उप॑ धावति । धा॒व॒ति॒ सः । स ए॒व । ए॒वैन᳚म् । ए॒न॒म् कामे॑न । कामे॑न॒ सम् । सम॑र्द्धयति । अ॒र्द्ध॒य॒त्युप॑ । उपै॑नम् । ए॒न॒म् कामः॑ । कामो॑ नमति । न॒म॒त्य॒ग्नये᳚ । अ॒ग्नये॒ यवि॑ष्ठाय । यवि॑ष्ठाय पुरो॒डाश᳚म् । पु॒रो॒डाश॑म॒ष्टाक॑पालम् । अ॒ष्टाक॑पाल॒म् निः । अ॒ष्टाक॑पाल॒मित्य॒ष्टा - क॒पा॒ल॒म् । निर् व॑पेत् । व॒पे॒थ् स्पर्द्ध॑मानः । स्पर्द्ध॑मानः॒ क्षेत्रे᳚ । क्षेत्रे॑ वा । वा॒ स॒जा॒तेषु॑ । स॒जा॒तेषु॑ वा । स॒जा॒तेष्विति॑ स - जा॒तेषु॑ । वा॒ ऽग्निम् । अ॒ग्निमे॒व । ए॒व यवि॑ष्ठम् । यवि॑ष्ठꣳ॒॒ स्वेन॑ । स्वेन॑ भाग॒धेये॑न । भा॒ग॒धेये॒नोप॑ । भा॒ग॒धेये॒नेति॑ भाग - धेये॑न । उप॑ धावति । धा॒व॒ति॒ तेन॑ । तेनै॒व । ए॒वेन्द्रि॒यम् । इ॒न्द्रि॒यं ॅवी॒र्य᳚म् । वी॒र्य॑म् भ्रातृ॑व्यस्य । भ्रातृ॑व्यस्य युवते \newline

\textbf{Jatai Paata} \newline

1. अ॒ग्नये॒ कामा॑य॒ कामा॑या॒ग्नये॒ ऽग्नये॒ कामा॑य । \newline
2. कामा॑य पुरो॒डाश॑म् पुरो॒डाश॒म् कामा॑य॒ कामा॑य पुरो॒डाश᳚म् । \newline
3. पु॒रो॒डाश॑ म॒ष्टाक॑पाल म॒ष्टाक॑पालम् पुरो॒डाश॑म् पुरो॒डाश॑ म॒ष्टाक॑पालम् । \newline
4. अ॒ष्टाक॑पाल॒म् निर् णिर॒ष्टाक॑पाल म॒ष्टाक॑पाल॒म् निः । \newline
5. अ॒ष्टाक॑पाल॒मित्य॒ष्टा - क॒पा॒ल॒म् । \newline
6. निर् व॑पेद् वपे॒न् निर् णिर् व॑पेत् । \newline
7. व॒पे॒द् यं ॅयं ॅव॑पेद् वपे॒द् यम् । \newline
8. यम् कामः॒ कामो॒ यं ॅयम् कामः॑ । \newline
9. कामो॒ न न कामः॒ कामो॒ न । \newline
10. नोप॒नमे॑ दुप॒नमे॒न् न नोप॒नमे᳚त् । \newline
11. उ॒प॒नमे॑द॒ग्नि म॒ग्नि मु॑प॒नमे॑ दुप॒नमे॑ द॒ग्निम् । \newline
12. उ॒प॒नमे॒दित्यु॑प - नमे᳚त् । \newline
13. अ॒ग्नि मे॒वैवाग्नि म॒ग्नि मे॒व । \newline
14. ए॒व काम॒म् काम॑ मे॒वैव काम᳚म् । \newline
15. कामꣳ॒॒ स्वेन॒ स्वेन॒ काम॒म् कामꣳ॒॒ स्वेन॑ । \newline
16. स्वेन॑ भाग॒धेये॑न भाग॒धेये॑न॒ स्वेन॒ स्वेन॑ भाग॒धेये॑न । \newline
17. भा॒ग॒धेये॒नोपोप॑ भाग॒धेये॑न भाग॒धेये॒नोप॑ । \newline
18. भा॒ग॒धेये॒नेति॑ भाग - धेये॑न । \newline
19. उप॑ धावति धाव॒ त्युपोप॑ धावति । \newline
20. धा॒व॒ति॒ स स धा॑वति धावति॒ सः । \newline
21. स ए॒वैव स स ए॒व । \newline
22. ए॒वैन॑ मेन मे॒वैवैन᳚म् । \newline
23. ए॒न॒म् कामे॑न॒ कामे॑नैन मेन॒म् कामे॑न । \newline
24. कामे॑न॒ सꣳ सम् कामे॑न॒ कामे॑न॒ सम् । \newline
25. स म॑र्द्धय त्यर्द्धयति॒ सꣳ स म॑र्द्धयति । \newline
26. अ॒र्द्ध॒य॒ त्युपोपा᳚र्द्धय त्यर्द्धय॒ त्युप॑ । \newline
27. उपै॑न मेन॒ मुपोपै॑नम् । \newline
28. ए॒न॒म् कामः॒ काम॑ एन मेन॒म् कामः॑ । \newline
29. कामो॑ नमति नमति॒ कामः॒ कामो॑ नमति । \newline
30. न॒म॒ त्य॒ग्नये॒ ऽग्नये॑ नमति नम त्य॒ग्नये᳚ । \newline
31. अ॒ग्नये॒ यवि॑ष्ठाय॒ यवि॑ष्ठाया॒ ग्नये॒ ऽग्नये॒ यवि॑ष्ठाय । \newline
32. यवि॑ष्ठाय पुरो॒डाश॑म् पुरो॒डाशं॒ ॅयवि॑ष्ठाय॒ यवि॑ष्ठाय पुरो॒डाश᳚म् । \newline
33. पु॒रो॒डाश॑ म॒ष्टाक॑पाल म॒ष्टाक॑पालम् पुरो॒डाश॑म् पुरो॒डाश॑ म॒ष्टाक॑पालम् । \newline
34. अ॒ष्टाक॑पाल॒म् निर् णिर॒ष्टाक॑पाल म॒ष्टाक॑पाल॒म् निः । \newline
35. अ॒ष्टाक॑पाल॒मित्य॒ष्टा - क॒पा॒ल॒म् । \newline
36. निर् व॑पेद् वपे॒न् निर् णिर् व॑पेत् । \newline
37. व॒पे॒थ् स्पर्द्ध॑मानः॒ स्पर्द्ध॑मानो वपेद् वपे॒थ् स्पर्द्ध॑मानः । \newline
38. स्पर्द्ध॑मानः॒ क्षेत्रे॒ क्षेत्रे॒ स्पर्द्ध॑मानः॒ स्पर्द्ध॑मानः॒ क्षेत्रे᳚ । \newline
39. क्षेत्रे॑ वा वा॒ क्षेत्रे॒ क्षेत्रे॑ वा । \newline
40. वा॒ स॒जा॒तेषु॑ सजा॒तेषु॑ वा वा सजा॒तेषु॑ । \newline
41. स॒जा॒तेषु॑ वा वा सजा॒तेषु॑ सजा॒तेषु॑ वा । \newline
42. स॒जा॒तेष्विति॑ स - जा॒तेषु॑ । \newline
43. वा॒ ऽग्नि म॒ग्निं ॅवा॑ वा॒ ऽग्निम् । \newline
44. अ॒ग्नि मे॒वैवाग्नि म॒ग्नि मे॒व । \newline
45. ए॒व यवि॑ष्ठं॒ ॅयवि॑ष्ठ मे॒वैव यवि॑ष्ठम् । \newline
46. यवि॑ष्ठꣳ॒॒ स्वेन॒ स्वेन॒ यवि॑ष्ठं॒ ॅयवि॑ष्ठꣳ॒॒ स्वेन॑ । \newline
47. स्वेन॑ भाग॒धेये॑न भाग॒धेये॑न॒ स्वेन॒ स्वेन॑ भाग॒धेये॑न । \newline
48. भा॒ग॒धेये॒नोपोप॑ भाग॒धेये॑न भाग॒धेये॒नोप॑ । \newline
49. भा॒ग॒धेये॒नेति॑ भाग - धेये॑न । \newline
50. उप॑ धावति धाव॒ त्युपोप॑ धावति । \newline
51. धा॒व॒ति॒ तेन॒ तेन॑ धावति धावति॒ तेन॑ । \newline
52. तेनै॒वैव तेन॒ तेनै॒व । \newline
53. ए॒वे न्द्रि॒य मि॑न्द्रि॒य मे॒वैवे न्द्रि॒यम् । \newline
54. इ॒न्द्रि॒यं ॅवी॒र्यं॑ ॅवी॒र्य॑ मिन्द्रि॒य मि॑न्द्रि॒यं ॅवी॒र्य᳚म् । \newline
55. वी॒र्य॑म् भ्रातृ॑व्यस्य॒ भ्रातृ॑व्यस्य वी॒र्यं॑ ॅवी॒र्य॑म् भ्रातृ॑व्यस्य । \newline
56. भ्रातृ॑व्यस्य युवते युवते॒ भ्रातृ॑व्यस्य॒ भ्रातृ॑व्यस्य युवते । \newline

\textbf{Ghana Paata } \newline

1. अ॒ग्नये॒ कामा॑य॒ कामा॑या॒ग्नये॒ ऽग्नये॒ कामा॑य पुरो॒डाश॑म् पुरो॒डाश॒म् कामा॑या॒ग्नये॒ ऽग्नये॒ कामा॑य पुरो॒डाश᳚म् । \newline
2. कामा॑य पुरो॒डाश॑म् पुरो॒डाश॒म् कामा॑य॒ कामा॑य पुरो॒डाश॑ म॒ष्टाक॑पाल म॒ष्टाक॑पालम् पुरो॒डाश॒म् कामा॑य॒ कामा॑य पुरो॒डाश॑ म॒ष्टाक॑पालम् । \newline
3. पु॒रो॒डाश॑ म॒ष्टाक॑पाल म॒ष्टाक॑पालम् पुरो॒डाश॑म् पुरो॒डाश॑ म॒ष्टाक॑पाल॒म् निर् णिर॒ष्टाक॑पालम् पुरो॒डाश॑म् पुरो॒डाश॑ म॒ष्टाक॑पाल॒म् निः । \newline
4. अ॒ष्टाक॑पाल॒म् निर् णिर॒ष्टाक॑पाल म॒ष्टाक॑पाल॒म् निर् व॑पेद् वपे॒न् निर॒ष्टाक॑पाल म॒ष्टाक॑पाल॒म् निर् व॑पेत् । \newline
5. अ॒ष्टाक॑पाल॒मित्य॒ष्टा - क॒पा॒ल॒म् । \newline
6. निर् व॑पेद् वपे॒न् निर् णिर् व॑पे॒द् यं ॅयं ॅव॑पे॒न् निर् णिर् व॑पे॒द् यम् । \newline
7. व॒पे॒द् यं ॅयं ॅव॑पेद् वपे॒द् यम् कामः॒ कामो॒ यं ॅव॑पेद् वपे॒द् यम् कामः॑ । \newline
8. यम् कामः॒ कामो॒ यं ॅयम् कामो॒ न न कामो॒ यं ॅयम् कामो॒ न । \newline
9. कामो॒ न न कामः॒ कामो॒ नोप॒नमे॑ दुप॒नमे॒न् न कामः॒ कामो॒ नोप॒नमे᳚त् । \newline
10. नोप॒नमे॑ दुप॒नमे॒न् न नोप॒नमे॑ द॒ग्नि म॒ग्नि मु॑प॒नमे॒न् न नोप॒नमे॑ द॒ग्निम् । \newline
11. उ॒प॒नमे॑ द॒ग्नि म॒ग्नि मु॑प॒नमे॑ दुप॒नमे॑ द॒ग्नि मे॒वैवाग्नि मु॑प॒नमे॑ दुप॒नमे॑ द॒ग्नि मे॒व । \newline
12. उ॒प॒नमे॒दित्यु॑प - नमे᳚त् । \newline
13. अ॒ग्नि मे॒वैवाग्नि म॒ग्नि मे॒व काम॒म् काम॑ मे॒वाग्नि म॒ग्नि मे॒व काम᳚म् । \newline
14. ए॒व काम॒म् काम॑ मे॒वैव कामꣳ॒॒ स्वेन॒ स्वेन॒ काम॑ मे॒वैव कामꣳ॒॒ स्वेन॑ । \newline
15. कामꣳ॒॒ स्वेन॒ स्वेन॒ काम॒म् कामꣳ॒॒ स्वेन॑ भाग॒धेये॑न भाग॒धेये॑न॒ स्वेन॒ काम॒म् कामꣳ॒॒ स्वेन॑ भाग॒धेये॑न । \newline
16. स्वेन॑ भाग॒धेये॑न भाग॒धेये॑न॒ स्वेन॒ स्वेन॑ भाग॒धेये॒नोपोप॑ भाग॒धेये॑न॒ स्वेन॒ स्वेन॑ भाग॒धेये॒नोप॑ । \newline
17. भा॒ग॒धेये॒नोपोप॑ भाग॒धेये॑न भाग॒धेये॒नोप॑ धावति धाव॒त्युप॑ भाग॒धेये॑न भाग॒धेये॒नोप॑ धावति । \newline
18. भा॒ग॒धेये॒नेति॑ भाग - धेये॑न । \newline
19. उप॑ धावति धाव॒ त्युपोप॑ धावति॒ स स धा॑व॒ त्युपोप॑ धावति॒ सः । \newline
20. धा॒व॒ति॒ स स धा॑वति धावति॒ स ए॒वैव स धा॑वति धावति॒ स ए॒व । \newline
21. स ए॒वैव स स ए॒वैन॑ मेन मे॒व स स ए॒वैन᳚म् । \newline
22. ए॒वैन॑ मेन मे॒वै वैन॒म् कामे॑न॒ कामे॑नैन मे॒वै वैन॒म् कामे॑न । \newline
23. ए॒न॒म् कामे॑न॒ कामे॑नैन मेन॒म् कामे॑न॒ सꣳ सम् कामे॑नैन मेन॒म् कामे॑न॒ सम् । \newline
24. कामे॑न॒ सꣳ सम् कामे॑न॒ कामे॑न॒ स म॑र्द्धय त्यर्द्धयति॒ सम् कामे॑न॒ कामे॑न॒ स म॑र्द्धयति । \newline
25. स म॑र्द्धय त्यर्द्धयति॒ सꣳ स म॑र्द्धय॒ त्युपोपा᳚र्द्धयति॒ सꣳ स म॑र्द्धय॒ त्युप॑ । \newline
26. अ॒र्द्ध॒य॒ त्युपो पा᳚र्द्धय त्यर्द्धय॒ त्युपै॑न मेन॒ मुपा᳚र्द्धय त्यर्द्धय॒ त्युपै॑नम् । \newline
27. उपै॑न मेन॒ मुपोपै॑न॒म् कामः॒ काम॑ एन॒ मुपोपै॑न॒म् कामः॑ । \newline
28. ए॒न॒म् कामः॒ काम॑ एन मेन॒म् कामो॑ नमति नमति॒ काम॑ एन मेन॒म् कामो॑ नमति । \newline
29. कामो॑ नमति नमति॒ कामः॒ कामो॑ नम त्य॒ग्नये॒ ऽग्नये॑ नमति॒ कामः॒ कामो॑ नमत्य॒ग्नये᳚ । \newline
30. न॒म॒ त्य॒ग्नये॒ ऽग्नये॑ नमति नम त्य॒ग्नये॒ यवि॑ष्ठाय॒ यवि॑ष्ठाया॒ ग्नये॑ नमति नम त्य॒ग्नये॒ यवि॑ष्ठाय । \newline
31. अ॒ग्नये॒ यवि॑ष्ठाय॒ यवि॑ष्ठाया॒ ग्नये॒ ऽग्नये॒ यवि॑ष्ठाय पुरो॒डाश॑म् पुरो॒डाशं॒ ॅयवि॑ष्ठाया॒ ग्नये॒ ऽग्नये॒ यवि॑ष्ठाय पुरो॒डाश᳚म् । \newline
32. यवि॑ष्ठाय पुरो॒डाश॑म् पुरो॒डाशं॒ ॅयवि॑ष्ठाय॒ यवि॑ष्ठाय पुरो॒डाश॑ म॒ष्टाक॑पाल म॒ष्टाक॑पालम् पुरो॒डाशं॒ ॅयवि॑ष्ठाय॒ यवि॑ष्ठाय पुरो॒डाश॑ म॒ष्टाक॑पालम् । \newline
33. पु॒रो॒डाश॑ म॒ष्टाक॑पाल म॒ष्टाक॑पालम् पुरो॒डाश॑म् पुरो॒डाश॑ म॒ष्टाक॑पाल॒म् निर् णिर॒ष्टाक॑पालम् पुरो॒डाश॑म् पुरो॒डाश॑ म॒ष्टाक॑पाल॒म् निः । \newline
34. अ॒ष्टाक॑पाल॒म् निर् णिर॒ष्टाक॑पाल म॒ष्टाक॑पाल॒म् निर् व॑पेद् वपे॒न् निर॒ष्टाक॑पाल म॒ष्टाक॑पाल॒म् निर् व॑पेत् । \newline
35. अ॒ष्टाक॑पाल॒मित्य॒ष्टा - क॒पा॒ल॒म् । \newline
36. निर् व॑पेद् वपे॒न् निर् णिर् व॑पे॒थ् स्पर्द्ध॑मानः॒ स्पर्द्ध॑मानो वपे॒न् निर् णिर् व॑पे॒थ् स्पर्द्ध॑मानः । \newline
37. व॒पे॒थ् स्पर्द्ध॑मानः॒ स्पर्द्ध॑मानो वपेद् वपे॒थ् स्पर्द्ध॑मानः॒ क्षेत्रे॒ क्षेत्रे॒ स्पर्द्ध॑मानो वपेद् वपे॒थ् स्पर्द्ध॑मानः॒ क्षेत्रे᳚ । \newline
38. स्पर्द्ध॑मानः॒ क्षेत्रे॒ क्षेत्रे॒ स्पर्द्ध॑मानः॒ स्पर्द्ध॑मानः॒ क्षेत्रे॑ वा वा॒ क्षेत्रे॒ स्पर्द्ध॑मानः॒ स्पर्द्ध॑मानः॒ क्षेत्रे॑ वा । \newline
39. क्षेत्रे॑ वा वा॒ क्षेत्रे॒ क्षेत्रे॑ वा सजा॒तेषु॑ सजा॒तेषु॑ वा॒ क्षेत्रे॒ क्षेत्रे॑ वा सजा॒तेषु॑ । \newline
40. वा॒ स॒जा॒तेषु॑ सजा॒तेषु॑ वा वा सजा॒तेषु॑ वा वा सजा॒तेषु॑ वा वा सजा॒तेषु॑ वा । \newline
41. स॒जा॒तेषु॑ वा वा सजा॒तेषु॑ सजा॒तेषु॑ वा॒ ऽग्नि म॒ग्निं ॅवा॑ सजा॒तेषु॑ सजा॒तेषु॑ वा॒ ऽग्निम् । \newline
42. स॒जा॒तेष्विति॑ स - जा॒तेषु॑ । \newline
43. वा॒ ऽग्नि म॒ग्निं ॅवा॑ वा॒ ऽग्नि मे॒वै वाग्निं ॅवा॑ वा॒ ऽग्नि मे॒व । \newline
44. अ॒ग्नि मे॒वै वाग्नि म॒ग्नि मे॒व यवि॑ष्ठं॒ ॅयवि॑ष्ठ मे॒वाग्नि म॒ग्नि मे॒व यवि॑ष्ठम् । \newline
45. ए॒व यवि॑ष्ठं॒ ॅयवि॑ष्ठ मे॒वैव यवि॑ष्ठꣳ॒॒ स्वेन॒ स्वेन॒ यवि॑ष्ठ मे॒वैव यवि॑ष्ठꣳ॒॒ स्वेन॑ । \newline
46. यवि॑ष्ठꣳ॒॒ स्वेन॒ स्वेन॒ यवि॑ष्ठं॒ ॅयवि॑ष्ठꣳ॒॒ स्वेन॑ भाग॒धेये॑न भाग॒धेये॑न॒ स्वेन॒ यवि॑ष्ठं॒ ॅयवि॑ष्ठꣳ॒॒ स्वेन॑ भाग॒धेये॑न । \newline
47. स्वेन॑ भाग॒धेये॑न भाग॒धेये॑न॒ स्वेन॒ स्वेन॑ भाग॒धेये॒नोपोप॑ भाग॒धेये॑न॒ स्वेन॒ स्वेन॑ भाग॒धेये॒नोप॑ । \newline
48. भा॒ग॒धेये॒नोपोप॑ भाग॒धेये॑न भाग॒धेये॒नोप॑ धावति धाव॒ त्युप॑ भाग॒धेये॑न भाग॒धेये॒नोप॑ धावति । \newline
49. भा॒ग॒धेये॒नेति॑ भाग - धेये॑न । \newline
50. उप॑ धावति धाव॒ त्युपोप॑ धावति॒ तेन॒ तेन॑ धाव॒ त्युपोप॑ धावति॒ तेन॑ । \newline
51. धा॒व॒ति॒ तेन॒ तेन॑ धावति धावति॒ तेनै॒वैव तेन॑ धावति धावति॒ तेनै॒व । \newline
52. तेनै॒वैव तेन॒ तेनै॒वे न्द्रि॒य मि॑न्द्रि॒य मे॒व तेन॒ तेनै॒वे न्द्रि॒यम् । \newline
53. ए॒वे न्द्रि॒य मि॑न्द्रि॒य मे॒वैवे न्द्रि॒यं ॅवी॒र्यं॑ ॅवी॒र्य॑ मिन्द्रि॒य मे॒वैवे न्द्रि॒यं ॅवी॒र्य᳚म् । \newline
54. इ॒न्द्रि॒यं ॅवी॒र्यं॑ ॅवी॒र्य॑ मिन्द्रि॒य मि॑न्द्रि॒यं ॅवी॒र्य॑म् भ्रातृ॑व्यस्य॒ भ्रातृ॑व्यस्य वी॒र्य॑ मिन्द्रि॒य मि॑न्द्रि॒यं ॅवी॒र्य॑म् भ्रातृ॑व्यस्य । \newline
55. वी॒र्य॑म् भ्रातृ॑व्यस्य॒ भ्रातृ॑व्यस्य वी॒र्यं॑ ॅवी॒र्य॑म् भ्रातृ॑व्यस्य युवते युवते॒ भ्रातृ॑व्यस्य वी॒र्यं॑ ॅवी॒र्य॑म् भ्रातृ॑व्यस्य युवते । \newline
56. भ्रातृ॑व्यस्य युवते युवते॒ भ्रातृ॑व्यस्य॒ भ्रातृ॑व्यस्य युवते॒ वि वि यु॑वते॒ भ्रातृ॑व्यस्य॒ भ्रातृ॑व्यस्य युवते॒ वि । \newline
\pagebreak
\markright{ TS 2.2.3.2  \hfill https://www.vedavms.in \hfill}
\addcontentsline{toc}{section}{ TS 2.2.3.2 }
\section*{ TS 2.2.3.2 }

\textbf{TS 2.2.3.2 } \newline
\textbf{Samhita Paata} \newline

युवते॒ विपा॒प्मना॒ भ्रातृ॑व्येण जयते॒ऽग्नये॒ यवि॑ष्ठाय पुरो॒डाश॑म॒ष्टाक॑पालं॒ निर्व॑पेदभिच॒र्यमा॑णो॒ ऽग्निमे॒व यवि॑ष्ठꣳ॒॒ स्वेन॑ भाग॒धेये॒नोप॑ धावति॒ स ए॒वास्मा॒द्-रक्षाꣳ॑सि यवयति॒ नैन॑-मभि॒चरन्᳚थ् स्तृणुते॒ऽग्नय॒ आयु॑ष्मते पुरो॒डाश॑म॒ष्टाक॑पालं॒ निर्व॑पे॒द्यः का॒मये॑त॒ सर्व॒मायु॑रिया॒-मित्य॒ग्नि- मे॒वाऽऽ*यु॑ष्मन्तꣳ॒॒ स्वेन॑ भाग॒धेये॒नोप॑ धावति॒ स ए॒वास्मि॒ - [  ] \newline

\textbf{Pada Paata} \newline

यु॒व॒ते॒ । वीति॑ । पा॒प्मना᳚ । भ्रातृ॑व्येण । ज॒य॒ते॒ । अ॒ग्नये᳚ । यवि॑ष्ठाय । पु॒रो॒डाश᳚म् । अ॒ष्टाक॑पाल॒मित्य॒ष्टा - क॒पा॒ल॒म् । निरिति॑ । व॒पे॒त् । अ॒भि॒च॒र्यमा॑ण॒ इत्य॑भि - च॒र्यमा॑णः । अ॒ग्निम् । ए॒व । यवि॑ष्ठम् । स्वेन॑ । भा॒ग॒धेये॒नेति॑ भाग - धेये॑न । उपेति॑ । धा॒व॒ति॒ । सः । ए॒व । अ॒स्मा॒त् । रक्षाꣳ॑सि । य॒व॒य॒ति॒ । न । ए॒न॒म् । अ॒भि॒चर॒न्नित्य॑भि - चरन्न्॑ । स्तृ॒णु॒ते॒ । अ॒ग्नये᳚ । आयु॑ष्मते । पु॒रो॒डाश᳚म् । अ॒ष्टाक॑पाल॒मित्य॒ष्टा-क॒पा॒ल॒म् । निरिति॑ । व॒पे॒त् । यः । का॒मये॑त । सर्व᳚म् । आयुः॑ । इ॒या॒म् । इति॑ । अ॒ग्निम् । ए॒व । आयु॑ष्मन्तम् । स्वेन॑ । भा॒ग॒धेये॒नेति॑ भाग - धेये॑न । उपेति॑ । धा॒व॒ति॒ । सः । ए॒व । अ॒स्मि॒न्न् ।  \newline


\textbf{Krama Paata} \newline

यु॒व॒ते॒ वि । वि पा॒प्मना᳚ । पा॒प्मना॒ भ्रातृ॑व्येण । भ्रातृ॑व्येण जयते । ज॒य॒ते॒ ऽग्नये᳚ । अ॒ग्नये॒ यवि॑ष्ठाय । यवि॑ष्ठाय पुरो॒डाश᳚म् । पु॒रो॒डाश॑म॒ष्टाक॑पालम् । अ॒ष्टाक॑पाल॒म् निः । अ॒ष्टाक॑पाल॒मित्य॒ष्टा - क॒पा॒ल॒म् । निर् व॑पेत् । व॒पे॒द॒भि॒च॒र्यमा॑णः । अ॒भि॒च॒र्यमा॑णो॒ ऽग्निम् । अ॒भि॒च॒र्यमा॑ण॒ इत्य॑भि - च॒र्यमा॑णः । अ॒ग्निमे॒व । ए॒व यवि॑ष्ठम् । यवि॑ष्ठꣳ॒॒ स्वेन॑ । स्वेन॑ भाग॒धेये॑न । भा॒ग॒धेये॒नोप॑ । भा॒ग॒धेये॒नेति॑ भाग - धेये॑न । उप॑ धावति । धा॒व॒ति॒ सः । स ए॒व । ए॒वास्मा᳚त् । अ॒स्मा॒द् रक्षाꣳ॑सि । रक्षाꣳ॑सि यवयति । य॒व॒य॒ति॒ न । नैन᳚म् । ए॒न॒म॒भि॒चरन्न्॑ । अ॒भि॒चर᳚न्थ् स्तृणुते । अ॒भि॒चर॒न्नित्य॑भि - चरन्न्॑ । स्तृ॒णु॒ते॒ ऽग्नये᳚ । अ॒ग्नय॒ आयु॑ष्मते । आयु॑ष्मते पुरो॒डाश᳚म् । पु॒रो॒डाश॑म॒ष्टाक॑पालम् । अ॒ष्टाक॑पाल॒म् निः । अ॒ष्टाक॑पाल॒मित्य॒ष्टा - क॒पा॒ल॒म् । निर् व॑पेत् । व॒पे॒द् यः । यः का॒मये॑त । का॒मये॑त॒ सर्व᳚म् । सर्व॒मायुः॑ । आयु॑रियाम् । इ॒या॒मिति॑ । इत्य॒ग्निम् । अ॒ग्निमे॒व । ए॒वायु॑ष्मन्तम् । आयु॑ष्मन्तꣳ॒॒ स्वेन॑ । स्वेन॑ भाग॒धेये॑न । भा॒ग॒धेये॒नोप॑ । भा॒ग॒धेये॒नेति॑ भाग - धेये॑न । उप॑ धावति । धा॒व॒ति॒ सः । स ए॒व । ए॒वास्मिन्न्॑ । अ॒स्मि॒न्नायुः॑ \newline

\textbf{Jatai Paata} \newline

1. यु॒व॒ते॒ वि वि यु॑वते युवते॒ वि । \newline
2. वि पा॒प्मना॑ पा॒प्मना॒ वि वि पा॒प्मना᳚ । \newline
3. पा॒प्मना॒ भ्रातृ॑व्येण॒ भ्रातृ॑व्येण पा॒प्मना॑ पा॒प्मना॒ भ्रातृ॑व्येण । \newline
4. भ्रातृ॑व्येण जयते जयते॒ भ्रातृ॑व्येण॒ भ्रातृ॑व्येण जयते । \newline
5. ज॒य॒ते॒ ऽग्नये॒ ऽग्नये॑ जयते जयते॒ ऽग्नये᳚ । \newline
6. अ॒ग्नये॒ यवि॑ष्ठाय॒ यवि॑ष्ठाया॒ ग्नये॒ ऽग्नये॒ यवि॑ष्ठाय । \newline
7. यवि॑ष्ठाय पुरो॒डाश॑म् पुरो॒डाशं॒ ॅयवि॑ष्ठाय॒ यवि॑ष्ठाय पुरो॒डाश᳚म् । \newline
8. पु॒रो॒डाश॑ म॒ष्टाक॑पाल म॒ष्टाक॑पालम् पुरो॒डाश॑म् पुरो॒डाश॑ म॒ष्टाक॑पालम् । \newline
9. अ॒ष्टाक॑पाल॒म् निर् णिर॒ष्टाक॑पाल म॒ष्टाक॑पाल॒म् निः । \newline
10. अ॒ष्टाक॑पाल॒मित्य॒ष्टा - क॒पा॒ल॒म् । \newline
11. निर् व॑पेद् वपे॒न् निर् णिर् व॑पेत् । \newline
12. व॒पे॒ द॒भि॒च॒र्यमा॑णो ऽभिच॒र्यमा॑णो वपेद् वपे दभिच॒र्यमा॑णः । \newline
13. अ॒भि॒च॒र्यमा॑णो॒ ऽग्नि म॒ग्नि म॑भिच॒र्यमा॑णो ऽभिच॒र्यमा॑णो॒ ऽग्निम् । \newline
14. अ॒भि॒च॒र्यमा॑ण॒ इत्य॑भि - च॒र्यमा॑णः । \newline
15. अ॒ग्नि मे॒वैवाग्नि म॒ग्नि मे॒व । \newline
16. ए॒व यवि॑ष्ठं॒ ॅयवि॑ष्ठ मे॒वैव यवि॑ष्ठम् । \newline
17. यवि॑ष्ठꣳ॒॒ स्वेन॒ स्वेन॒ यवि॑ष्ठं॒ ॅयवि॑ष्ठꣳ॒॒ स्वेन॑ । \newline
18. स्वेन॑ भाग॒धेये॑न भाग॒धेये॑न॒ स्वेन॒ स्वेन॑ भाग॒धेये॑न । \newline
19. भा॒ग॒धेये॒नोपोप॑ भाग॒धेये॑न भाग॒धेये॒नोप॑ । \newline
20. भा॒ग॒धेये॒नेति॑ भाग - धेये॑न । \newline
21. उप॑ धावति धाव॒ त्युपोप॑ धावति । \newline
22. धा॒व॒ति॒ स स धा॑वति धावति॒ सः । \newline
23. स ए॒वैव स स ए॒व । \newline
24. ए॒वास्मा॑ दस्मा दे॒वैवास्मा᳚त् । \newline
25. अ॒स्मा॒द् रक्षाꣳ॑सि॒ रक्षाꣳ॑ स्यस्मा दस्मा॒द् रक्षाꣳ॑सि । \newline
26. रक्षाꣳ॑सि यवयति यवयति॒ रक्षाꣳ॑सि॒ रक्षाꣳ॑सि यवयति । \newline
27. य॒व॒य॒ति॒ न न य॑वयति यवयति॒ न । \newline
28. नैन॑ मेन॒म् न नैन᳚म् । \newline
29. ए॒न॒ म॒भि॒चर॑न् नभि॒चर॑न् नेन मेन मभि॒चरन्न्॑ । \newline
30. अ॒भि॒चरन्᳚ थ्स्तृणुते स्तृणुते ऽभि॒चर॑न् नभि॒चरन्᳚ थ्स्तृणुते । \newline
31. अ॒भि॒चर॒न्नित्य॑भि - चरन्न्॑ । \newline
32. स्तृ॒णु॒ते॒ ऽग्नये॒ ऽग्नये᳚ स्तृणुते स्तृणुते॒ ऽग्नये᳚ । \newline
33. अ॒ग्नय॒ आयु॑ष्मत॒ आयु॑ष्मते॒ ऽग्नये॒ ऽग्नय॒ आयु॑ष्मते । \newline
34. आयु॑ष्मते पुरो॒डाश॑म् पुरो॒डाश॒ मायु॑ष्मत॒ आयु॑ष्मते पुरो॒डाश᳚म् । \newline
35. पु॒रो॒डाश॑ म॒ष्टाक॑पाल म॒ष्टाक॑पालम् पुरो॒डाश॑म् पुरो॒डाश॑ म॒ष्टाक॑पालम् । \newline
36. अ॒ष्टाक॑पाल॒म् निर् णिर॒ष्टाक॑पाल म॒ष्टाक॑पाल॒म् निः । \newline
37. अ॒ष्टाक॑पाल॒मित्य॒ष्टा - क॒पा॒ल॒म् । \newline
38. निर् व॑पेद् वपे॒न् निर् णिर् व॑पेत् । \newline
39. व॒पे॒द् यो यो व॑पेद् वपे॒द् यः । \newline
40. यः का॒मये॑त का॒मये॑त॒ यो यः का॒मये॑त । \newline
41. का॒मये॑त॒ सर्वꣳ॒॒ सर्व॑म् का॒मये॑त का॒मये॑त॒ सर्व᳚म् । \newline
42. सर्व॒ मायु॒ रायुः॒ सर्वꣳ॒॒ सर्व॒ मायुः॑ । \newline
43. आयु॑ रिया मिया॒ मायु॒ रायु॑ रियाम् । \newline
44. इ॒या॒ मितीती॑या मिया॒ मिति॑ । \newline
45. इत्य॒ग्नि म॒ग्नि मिती त्य॒ग्निम् । \newline
46. अ॒ग्नि मे॒वैवाग्नि म॒ग्नि मे॒व । \newline
47. ए॒वायु॑ष्मन्त॒ मायु॑ष्मन्त मे॒वैवायु॑ष्मन्तम् । \newline
48. आयु॑ष्मन्तꣳ॒॒ स्वेन॒ स्वेनायु॑ष्मन्त॒ मायु॑ष्मन्तꣳ॒॒ स्वेन॑ । \newline
49. स्वेन॑ भाग॒धेये॑न भाग॒धेये॑न॒ स्वेन॒ स्वेन॑ भाग॒धेये॑न । \newline
50. भा॒ग॒धेये॒नोपोप॑ भाग॒धेये॑न भाग॒धेये॒नोप॑ । \newline
51. भा॒ग॒धेये॒नेति॑ भाग - धेये॑न । \newline
52. उप॑ धावति धाव॒ त्युपोप॑ धावति । \newline
53. धा॒व॒ति॒ स स धा॑वति धावति॒ सः । \newline
54. स ए॒वैव स स ए॒व । \newline
55. ए॒वास्मि॑न् नस्मिन् ने॒वैवास्मिन्न्॑ । \newline
56. अ॒स्मि॒न् नायु॒ रायु॑ रस्मिन् नस्मि॒न् नायुः॑ । \newline

\textbf{Ghana Paata } \newline

1. यु॒व॒ते॒ वि वि यु॑वते युवते॒ वि पा॒प्मना॑ पा॒प्मना॒ वि यु॑वते युवते॒ वि पा॒प्मना᳚ । \newline
2. वि पा॒प्मना॑ पा॒प्मना॒ वि वि पा॒प्मना॒ भ्रातृ॑व्येण॒ भ्रातृ॑व्येण पा॒प्मना॒ वि वि पा॒प्मना॒ भ्रातृ॑व्येण । \newline
3. पा॒प्मना॒ भ्रातृ॑व्येण॒ भ्रातृ॑व्येण पा॒प्मना॑ पा॒प्मना॒ भ्रातृ॑व्येण जयते जयते॒ भ्रातृ॑व्येण पा॒प्मना॑ पा॒प्मना॒ भ्रातृ॑व्येण जयते । \newline
4. भ्रातृ॑व्येण जयते जयते॒ भ्रातृ॑व्येण॒ भ्रातृ॑व्येण जयते॒ ऽग्नये॒ ऽग्नये॑ जयते॒ भ्रातृ॑व्येण॒ भ्रातृ॑व्येण जयते॒ ऽग्नये᳚ । \newline
5. ज॒य॒ते॒ ऽग्नये॒ ऽग्नये॑ जयते जयते॒ ऽग्नये॒ यवि॑ष्ठाय॒ यवि॑ष्ठाया॒ ग्नये॑ जयते जयते॒ ऽग्नये॒ यवि॑ष्ठाय । \newline
6. अ॒ग्नये॒ यवि॑ष्ठाय॒ यवि॑ष्ठाया॒ ग्नये॒ ऽग्नये॒ यवि॑ष्ठाय पुरो॒डाश॑म् पुरो॒डाशं॒ ॅयवि॑ष्ठाया॒ ग्नये॒ ऽग्नये॒ यवि॑ष्ठाय पुरो॒डाश᳚म् । \newline
7. यवि॑ष्ठाय पुरो॒डाश॑म् पुरो॒डाशं॒ ॅयवि॑ष्ठाय॒ यवि॑ष्ठाय पुरो॒डाश॑ म॒ष्टाक॑पाल म॒ष्टाक॑पालम् पुरो॒डाशं॒ ॅयवि॑ष्ठाय॒ यवि॑ष्ठाय पुरो॒डाश॑ म॒ष्टाक॑पालम् । \newline
8. पु॒रो॒डाश॑ म॒ष्टाक॑पाल म॒ष्टाक॑पालम् पुरो॒डाश॑म् पुरो॒डाश॑ म॒ष्टाक॑पाल॒म् निर् णिर॒ष्टाक॑पालम् पुरो॒डाश॑म् पुरो॒डाश॑ म॒ष्टाक॑पाल॒म् निः । \newline
9. अ॒ष्टाक॑पाल॒म् निर् णिर॒ष्टाक॑पाल म॒ष्टाक॑पाल॒म् निर् व॑पेद् वपे॒न् निर॒ष्टाक॑पाल म॒ष्टाक॑पाल॒म् निर् व॑पेत् । \newline
10. अ॒ष्टाक॑पाल॒मित्य॒ष्टा - क॒पा॒ल॒म् । \newline
11. निर् व॑पेद् वपे॒न् निर् णिर् व॑पे दभिच॒र्यमा॑णो ऽभिच॒र्यमा॑णो वपे॒न् निर् णिर् व॑पे दभिच॒र्यमा॑णः । \newline
12. व॒पे॒ द॒भि॒च॒र्यमा॑णो ऽभिच॒र्यमा॑णो वपेद् वपे दभिच॒र्यमा॑णो॒ ऽग्नि म॒ग्नि म॑भिच॒र्यमा॑णो वपेद् वपे दभिच॒र्यमा॑णो॒ ऽग्निम् । \newline
13. अ॒भि॒च॒र्यमा॑णो॒ ऽग्नि म॒ग्नि म॑भिच॒र्यमा॑णो ऽभिच॒र्यमा॑णो॒ ऽग्नि मे॒वैवाग्नि म॑भिच॒र्यमा॑णो ऽभिच॒र्यमा॑णो॒ ऽग्नि मे॒व । \newline
14. अ॒भि॒च॒र्यमा॑ण॒ इत्य॑भि - च॒र्यमा॑णः । \newline
15. अ॒ग्नि मे॒वैवाग्नि म॒ग्नि मे॒व यवि॑ष्ठं॒ ॅयवि॑ष्ठ मे॒वाग्नि म॒ग्नि मे॒व यवि॑ष्ठम् । \newline
16. ए॒व यवि॑ष्ठं॒ ॅयवि॑ष्ठ मे॒वैव यवि॑ष्ठꣳ॒॒ स्वेन॒ स्वेन॒ यवि॑ष्ठ मे॒वैव यवि॑ष्ठꣳ॒॒ स्वेन॑ । \newline
17. यवि॑ष्ठꣳ॒॒ स्वेन॒ स्वेन॒ यवि॑ष्ठं॒ ॅयवि॑ष्ठꣳ॒॒ स्वेन॑ भाग॒धेये॑न भाग॒धेये॑न॒ स्वेन॒ यवि॑ष्ठं॒ ॅयवि॑ष्ठꣳ॒॒ स्वेन॑ भाग॒धेये॑न । \newline
18. स्वेन॑ भाग॒धेये॑न भाग॒धेये॑न॒ स्वेन॒ स्वेन॑ भाग॒धेये॒नोपोप॑ भाग॒धेये॑न॒ स्वेन॒ स्वेन॑ भाग॒धेये॒नोप॑ । \newline
19. भा॒ग॒धेये॒नोपोप॑ भाग॒धेये॑न भाग॒धेये॒नोप॑ धावति धाव॒ त्युप॑ भाग॒धेये॑न भाग॒धेये॒नोप॑ धावति । \newline
20. भा॒ग॒धेये॒नेति॑ भाग - धेये॑न । \newline
21. उप॑ धावति धाव॒ त्युपोप॑ धावति॒ स स धा॑व॒ त्युपोप॑ धावति॒ सः । \newline
22. धा॒व॒ति॒ स स धा॑वति धावति॒ स ए॒वैव स धा॑वति धावति॒ स ए॒व । \newline
23. स ए॒वैव स स ए॒वास्मा॑ दस्मा दे॒व स स ए॒वास्मा᳚त् । \newline
24. ए॒वास्मा॑ दस्मा दे॒वैवास्मा॒द् रक्षाꣳ॑सि॒ रक्षाꣳ॑ स्यस्मा दे॒वैवास्मा॒द् रक्षाꣳ॑सि । \newline
25. अ॒स्मा॒द् रक्षाꣳ॑सि॒ रक्षाꣳ॑ स्यस्मा दस्मा॒द् रक्षाꣳ॑सि यवयति यवयति॒ रक्षाꣳ॑ स्यस्मा दस्मा॒द् रक्षाꣳ॑सि यवयति । \newline
26. रक्षाꣳ॑सि यवयति यवयति॒ रक्षाꣳ॑सि॒ रक्षाꣳ॑सि यवयति॒ न न य॑वयति॒ रक्षाꣳ॑सि॒ रक्षाꣳ॑सि यवयति॒ न । \newline
27. य॒व॒य॒ति॒ न न य॑वयति यवयति॒ नैन॑ मेन॒म् न य॑वयति यवयति॒ नैन᳚म् । \newline
28. नैन॑ मेन॒म् न नैन॑ मभि॒चर॑न् नभि॒चर॑न् नेन॒म् न नैन॑ मभि॒चरन्न्॑ । \newline
29. ए॒न॒ म॒भि॒चर॑न् नभि॒चर॑न् नेन मेन मभि॒चरन्᳚ थ्स्तृणुते स्तृणुते ऽभि॒चर॑न् नेन मेन मभि॒चरन्᳚ थ्स्तृणुते । \newline
30. अ॒भि॒चरन्᳚ थ्स्तृणुते स्तृणुते ऽभि॒चर॑न् नभि॒चरन्᳚ थ्स्तृणुते॒ ऽग्नये॒ ऽग्नये᳚ स्तृणुते ऽभि॒चर॑न् नभि॒चरन्᳚ थ्स्तृणुते॒ ऽग्नये᳚ । \newline
31. अ॒भि॒चर॒न्नित्य॑भि - चरन्न्॑ । \newline
32. स्तृ॒णु॒ते॒ ऽग्नये॒ ऽग्नये᳚ स्तृणुते स्तृणुते॒ ऽग्नय॒ आयु॑ष्मत॒ आयु॑ष्मते॒ ऽग्नये᳚ स्तृणुते स्तृणुते॒ ऽग्नय॒ आयु॑ष्मते । \newline
33. अ॒ग्नय॒ आयु॑ष्मत॒ आयु॑ष्मते॒ ऽग्नये॒ ऽग्नय॒ आयु॑ष्मते पुरो॒डाश॑म् पुरो॒डाश॒ मायु॑ष्मते॒ ऽग्नये॒ ऽग्नय॒ आयु॑ष्मते पुरो॒डाश᳚म् । \newline
34. आयु॑ष्मते पुरो॒डाश॑म् पुरो॒डाश॒ मायु॑ष्मत॒ आयु॑ष्मते पुरो॒डाश॑ म॒ष्टाक॑पाल म॒ष्टाक॑पालम् पुरो॒डाश॒ मायु॑ष्मत॒ आयु॑ष्मते पुरो॒डाश॑ म॒ष्टाक॑पालम् । \newline
35. पु॒रो॒डाश॑ म॒ष्टाक॑पाल म॒ष्टाक॑पालम् पुरो॒डाश॑म् पुरो॒डाश॑ म॒ष्टाक॑पाल॒म् निर् णिर॒ष्टाक॑पालम् पुरो॒डाश॑म् पुरो॒डाश॑ म॒ष्टाक॑पाल॒म् निः । \newline
36. अ॒ष्टाक॑पाल॒म् निर् णिर॒ष्टाक॑पाल म॒ष्टाक॑पाल॒म् निर् व॑पेद् वपे॒न् निर॒ष्टाक॑पाल म॒ष्टाक॑पाल॒म् निर् व॑पेत् । \newline
37. अ॒ष्टाक॑पाल॒मित्य॒ष्टा - क॒पा॒ल॒म् । \newline
38. निर् व॑पेद् वपे॒न् निर् णिर् व॑पे॒द् यो यो व॑पे॒न् निर् णिर् व॑पे॒द् यः । \newline
39. व॒पे॒द् यो यो व॑पेद् वपे॒द् यः का॒मये॑त का॒मये॑त॒ यो व॑पेद् वपे॒द् यः का॒मये॑त । \newline
40. यः का॒मये॑त का॒मये॑त॒ यो यः का॒मये॑त॒ सर्वꣳ॒॒ सर्व॑म् का॒मये॑त॒ यो यः का॒मये॑त॒ सर्व᳚म् । \newline
41. का॒मये॑त॒ सर्वꣳ॒॒ सर्व॑म् का॒मये॑त का॒मये॑त॒ सर्व॒ मायु॒ रायुः॒ सर्व॑म् का॒मये॑त का॒मये॑त॒ सर्व॒ मायुः॑ । \newline
42. सर्व॒ मायु॒ रायुः॒ सर्वꣳ॒॒ सर्व॒ मायु॑ रिया मिया॒ मायुः॒ सर्वꣳ॒॒ सर्व॒ मायु॑ रियाम् । \newline
43. आयु॑ रिया मिया॒ मायु॒ रायु॑ रिया॒ मितीती॑या॒ मायु॒ रायु॑ रिया॒ मिति॑ । \newline
44. इ॒या॒ मितीती॑या मिया॒ मित्य॒ग्नि म॒ग्नि मिती॑या मिया॒ मित्य॒ग्निम् । \newline
45. इत्य॒ग्नि म॒ग्नि मिती त्य॒ग्नि मे॒वैवाग्नि मितीत्य॒ग्नि मे॒व । \newline
46. अ॒ग्नि मे॒वैवाग्नि म॒ग्नि मे॒वायु॑ष्मन्त॒ मायु॑ष्मन्त मे॒वाग्नि म॒ग्नि मे॒वायु॑ष्मन्तम् । \newline
47. ए॒वायु॑ष्मन्त॒ मायु॑ष्मन्त मे॒वै वायु॑ष्मन्तꣳ॒॒ स्वेन॒ स्वेनायु॑ष्मन्त मे॒वै वायु॑ष्मन्तꣳ॒॒ स्वेन॑ । \newline
48. आयु॑ष्मन्तꣳ॒॒ स्वेन॒ स्वेनायु॑ष्मन्त॒ मायु॑ष्मन्तꣳ॒॒ स्वेन॑ भाग॒धेये॑न भाग॒धेये॑न॒ स्वेनायु॑ष्मन्त॒ मायु॑ष्मन्तꣳ॒॒ स्वेन॑ भाग॒धेये॑न । \newline
49. स्वेन॑ भाग॒धेये॑न भाग॒धेये॑न॒ स्वेन॒ स्वेन॑ भाग॒धेये॒नोपोप॑ भाग॒धेये॑न॒ स्वेन॒ स्वेन॑ भाग॒धेये॒नोप॑ । \newline
50. भा॒ग॒धेये॒नोपोप॑ भाग॒धेये॑न भाग॒धेये॒नोप॑ धावति धाव॒ त्युप॑ भाग॒धेये॑न भाग॒धेये॒नोप॑ धावति । \newline
51. भा॒ग॒धेये॒नेति॑ भाग - धेये॑न । \newline
52. उप॑ धावति धाव॒ त्युपोप॑ धावति॒ स स धा॑व॒ त्युपोप॑ धावति॒ सः । \newline
53. धा॒व॒ति॒ स स धा॑वति धावति॒ स ए॒वैव स धा॑वति धावति॒ स ए॒व । \newline
54. स ए॒वैव स स ए॒वास्मि॑न् नस्मिन् ने॒व स स ए॒वास्मिन्न्॑ । \newline
55. ए॒वास्मि॑न् नस्मिन् ने॒वैवास्मि॒न् नायु॒ रायु॑ रस्मिन् ने॒वैवास्मि॒न् नायुः॑ । \newline
56. अ॒स्मि॒न् नायु॒ रायु॑ रस्मिन् नस्मि॒न् नायु॑र् दधाति दधा॒ त्यायु॑ रस्मिन् नस्मि॒न् नायु॑र् दधाति । \newline
\pagebreak
\markright{ TS 2.2.3.3  \hfill https://www.vedavms.in \hfill}
\addcontentsline{toc}{section}{ TS 2.2.3.3 }
\section*{ TS 2.2.3.3 }

\textbf{TS 2.2.3.3 } \newline
\textbf{Samhita Paata} \newline

-न्नायु॑र्दधाति॒ सर्व॒माय॑रेत्य॒ग्नये॑ जा॒तवे॑दसे पुरो॒डाश॑-म॒ष्टाक॑पालं॒ निर्व॑पे॒द्-भूति॑कामो॒ऽग्निमे॒व जा॒तवे॑दसꣳ॒॒ स्वेन॑ भाग॒धेये॒नोप॑ धावति॒ स ए॒वैनं॒ भूतिं॑ गमयति॒ भव॑त्ये॒वाग्नये॒ रुक्म॑ते पुरो॒डाश॑-म॒ष्टाक॑पालं॒ निर्व॑पे॒द्-रुक्का॑मो॒ऽग्निमे॒व रुक्म॑न्तꣳ॒॒ स्वेन॑ भाग॒धेये॒नोप॑ धावति॒ स ए॒वास्मि॒न् रुचं॑ दधाति॒रोच॑त ए॒वाग्नये॒ तेज॑स्वते पुरो॒डाश॑ - [  ] \newline

\textbf{Pada Paata} \newline

आयुः॑ । द॒धा॒ति॒ । सर्व᳚म् । आयुः॑ । ए॒ति॒ । अ॒ग्नये᳚ । जा॒तवे॑दस॒ इति॑ जा॒त - वे॒द॒से॒ । पु॒रो॒डाश᳚म् । अ॒ष्टाक॑पाल॒मित्य॒ष्टा - क॒पा॒ल॒म् । निरिति॑ । व॒पे॒त् । भूति॑काम॒ इति॒ भूति॑ - का॒मः॒ । अ॒ग्निम् । ए॒व । जा॒तवे॑दस॒मिति॑ जा॒त - वे॒द॒स॒म् । स्वेन॑ । भा॒ग॒धेये॒नेति॑ भाग - धेये॑न । उपेति॑ । धा॒व॒ति॒ । सः । ए॒व । ए॒न॒म् । भूति᳚म् । ग॒म॒य॒ति॒ । भव॑ति । ए॒व । अ॒ग्नये᳚ । रुक्म॑ते । पु॒रो॒डाश᳚म् । अ॒ष्टाक॑पाल॒मित्य॒ष्टा-क॒पा॒ल॒म् । निरिति॑ । व॒पे॒त् । रुक्का॑म॒ इति॒ रुक् - का॒मः॒ । अ॒ग्निम् । ए॒व । रुक्म॑न्तम् । स्वेन॑ । भा॒ग॒धेये॒नेति॑ भाग - धेये॑न । उपेति॑ । धा॒व॒ति॒ । सः । ए॒व । अ॒स्मि॒न्न् । रुच᳚म् । द॒धा॒ति॒ । रोच॑ते । ए॒व । अ॒ग्नये᳚ । तेज॑स्वते । पु॒रो॒डाश᳚म् ।  \newline


\textbf{Krama Paata} \newline

आयु॑र् दधाति । द॒धा॒ति॒ सर्व᳚म् । सर्व॒मायुः॑ । आयु॑रेति । ए॒त्य॒ग्नये᳚ । अ॒ग्नये॑ जा॒तवे॑दसे । जा॒तवे॑दसे पुरो॒डाश᳚म् । जा॒तवे॑दस॒ इति॑ जा॒त - वे॒द॒से॒ । पु॒रो॒डाश॑म॒ष्टाक॑पालम् । अ॒ष्टाक॑पाल॒म् निः । अ॒ष्टाक॑पाल॒मित्य॒ष्टा - क॒पा॒ल॒म् । निर् व॑पेत् । व॒पे॒द्,भूति॑कामः । भूति॑कामो॒ ऽग्निम् । भूति॑काम॒ इति॒ भूति॑ - का॒मः॒ । अ॒ग्निमे॒व । ए॒व जा॒तवे॑दसम् । जा॒तवे॑दसꣳ॒॒ स्वेन॑ । जा॒तवे॑दस॒मिति॑ जा॒त - वे॒द॒स॒म् । स्वेन॑ भाग॒धेये॑न । भा॒ग॒धेये॒नोप॑ । भा॒ग॒धेये॒नेति॑ भाग - धेये॑न । उप॑ धावति । धा॒व॒ति॒ सः । स ए॒व । ए॒वैन᳚म् । ए॒न॒म् भूति᳚म् । भूति॑म् गमयति । ग॒म॒य॒ति॒ भव॑ति । भव॑त्ये॒व । ए॒वाग्नये᳚ । अ॒ग्नये॒ रुक्म॑ते । रुक्म॑ते पुरो॒डाश᳚म् । पु॒रो॒डाश॑म॒ष्टाक॑पालम् । अ॒ष्टाक॑पाल॒म् निः । अ॒ष्टाक॑पाल॒मित्य॒ष्टा - क॒पा॒ल॒म् । निर् व॑पेत् । व॒पे॒द् रुक्का॑मः । रुक्का॑मो॒ऽग्निम् । रुक्का॑म॒ इति॒ रुक् - का॒मः॒ । अ॒ग्निमे॒व । ए॒व रुक्म॑न्तम् । रुक्म॑न्तꣳ॒॒ स्वेन॑ । स्वेन॑ भाग॒धेये॑न । भा॒ग॒धेये॒नोप॑ । भा॒ग॒धेये॒नेति॑ भाग - धेये॑न । उप॑ धावति । धा॒व॒ति॒ सः । स ए॒व । ए॒वास्मिन्न्॑ । अ॒स्मि॒न् रुच᳚म् । रुच॑म् दधाति । द॒धा॒ति॒ रोच॑ते । रोच॑त ए॒व । ए॒वाग्नये᳚ । अ॒ग्नये॒ तेज॑स्वते । तेज॑स्वते पुरो॒डाश᳚म् ( ) । पु॒रो॒डाश॑म॒ष्टाक॑पालम् \newline

\textbf{Jatai Paata} \newline

1. आयु॑र् दधाति दधा॒ त्यायु॒ रायु॑र् दधाति । \newline
2. द॒धा॒ति॒ सर्वꣳ॒॒ सर्व॑म् दधाति दधाति॒ सर्व᳚म् । \newline
3. सर्व॒ मायु॒रायुः॒ सर्वꣳ॒॒ सर्व॒ मायुः॑ । \newline
4. आयु॑ रेत्ये॒त्यायु॒ रायु॑रेति । \newline
5. ए॒त्य॒ग्नये॒ ऽग्नय॑ एत्ये त्य॒ग्नये᳚ । \newline
6. अ॒ग्नये॑ जा॒तवे॑दसे जा॒तवे॑दसे॒ ऽग्नये॒ ऽग्नये॑ जा॒तवे॑दसे । \newline
7. जा॒तवे॑दसे पुरो॒डाश॑म् पुरो॒डाश॑म् जा॒तवे॑दसे जा॒तवे॑दसे पुरो॒डाश᳚म् । \newline
8. जा॒तवे॑दस॒ इति॑ जा॒त - वे॒द॒से॒ । \newline
9. पु॒रो॒डाश॑ म॒ष्टाक॑पाल म॒ष्टाक॑पालम् पुरो॒डाश॑म् पुरो॒डाश॑ म॒ष्टाक॑पालम् । \newline
10. अ॒ष्टाक॑पाल॒म् निर् णिर॒ष्टाक॑पाल म॒ष्टाक॑पाल॒म् निः । \newline
11. अ॒ष्टाक॑पाल॒मित्य॒ष्टा - क॒पा॒ल॒म् । \newline
12. निर् व॑पेद् वपे॒न् निर् णिर् व॑पेत् । \newline
13. व॒पे॒द् भूति॑कामो॒ भूति॑कामो वपेद् वपे॒द् भूति॑कामः । \newline
14. भूति॑कामो॒ ऽग्नि म॒ग्निम् भूति॑कामो॒ भूति॑कामो॒ ऽग्निम् । \newline
15. भूति॑काम॒ इति॒ भूति॑ - का॒मः॒ । \newline
16. अ॒ग्नि मे॒वैवाग्नि म॒ग्नि मे॒व । \newline
17. ए॒व जा॒तवे॑दसम् जा॒तवे॑दस मे॒वैव जा॒तवे॑दसम् । \newline
18. जा॒तवे॑दसꣳ॒॒ स्वेन॒ स्वेन॑ जा॒तवे॑दसम् जा॒तवे॑दसꣳ॒॒ स्वेन॑ । \newline
19. जा॒तवे॑दस॒मिति॑ जा॒त - वे॒द॒स॒म् । \newline
20. स्वेन॑ भाग॒धेये॑न भाग॒धेये॑न॒ स्वेन॒ स्वेन॑ भाग॒धेये॑न । \newline
21. भा॒ग॒धेये॒नोपोप॑ भाग॒धेये॑न भाग॒धेये॒नोप॑ । \newline
22. भा॒ग॒धेये॒नेति॑ भाग - धेये॑न । \newline
23. उप॑ धावति धाव॒ त्युपोप॑ धावति । \newline
24. धा॒व॒ति॒ स स धा॑वति धावति॒ सः । \newline
25. स ए॒वैव स स ए॒व । \newline
26. ए॒वैन॑ मेन मे॒वैवैन᳚म् । \newline
27. ए॒न॒म् भूति॒म् भूति॑ मेन मेन॒म् भूति᳚म् । \newline
28. भूति॑म् गमयति गमयति॒ भूति॒म् भूति॑म् गमयति । \newline
29. ग॒म॒य॒ति॒ भव॑ति॒ भव॑ति गमयति गमयति॒ भव॑ति । \newline
30. भव॑त्ये॒वैव भव॑ति॒ भव॑त्ये॒व । \newline
31. ए॒वाग्नये॒ ऽग्नय॑ ए॒वैवाग्नये᳚ । \newline
32. अ॒ग्नये॒ रुक्म॑ते॒ रुक्म॑ते॒ ऽग्नये॒ ऽग्नये॒ रुक्म॑ते । \newline
33. रुक्म॑ते पुरो॒डाश॑म् पुरो॒डाशꣳ॒॒ रुक्म॑ते॒ रुक्म॑ते पुरो॒डाश᳚म् । \newline
34. पु॒रो॒डाश॑ म॒ष्टाक॑पाल म॒ष्टाक॑पालम् पुरो॒डाश॑म् पुरो॒डाश॑ म॒ष्टाक॑पालम् । \newline
35. अ॒ष्टाक॑पाल॒म् निर् णिर॒ष्टाक॑पाल म॒ष्टाक॑पाल॒म् निः । \newline
36. अ॒ष्टाक॑पाल॒मित्य॒ष्टा - क॒पा॒ल॒म् । \newline
37. निर् व॑पेद् वपे॒न् निर् णिर् व॑पेत् । \newline
38. व॒पे॒द् रुक्का॑मो॒ रुक्का॑मो वपेद् वपे॒द् रुक्का॑मः । \newline
39. रुक्का॑मो॒ ऽग्नि म॒ग्निꣳ रुक्का॑मो॒ रुक्का॑मो॒ ऽग्निम् । \newline
40. रुक्का॑म॒ इति॒ रुक् - का॒मः॒ । \newline
41. अ॒ग्नि मे॒वैवाग्नि म॒ग्नि मे॒व । \newline
42. ए॒व रुक्म॑न्तꣳ॒॒ रुक्म॑न्त मे॒वैव रुक्म॑न्तम् । \newline
43. रुक्म॑न्तꣳ॒॒ स्वेन॒ स्वेन॒ रुक्म॑न्तꣳ॒॒ रुक्म॑न्तꣳ॒॒ स्वेन॑ । \newline
44. स्वेन॑ भाग॒धेये॑न भाग॒धेये॑न॒ स्वेन॒ स्वेन॑ भाग॒धेये॑न । \newline
45. भा॒ग॒धेये॒नोपोप॑ भाग॒धेये॑न भाग॒धेये॒नोप॑ । \newline
46. भा॒ग॒धेये॒नेति॑ भाग - धेये॑न । \newline
47. उप॑ धावति धाव॒ त्युपोप॑ धावति । \newline
48. धा॒व॒ति॒ स स धा॑वति धावति॒ सः । \newline
49. स ए॒वैव स स ए॒व । \newline
50. ए॒वास्मि॑न् नस्मिन् ने॒वैवास्मिन्न्॑ । \newline
51. अ॒स्मि॒न् रुचꣳ॒॒ रुच॑ मस्मिन् नस्मि॒न् रुच᳚म् । \newline
52. रुच॑म् दधाति दधाति॒ रुचꣳ॒॒ रुच॑म् दधाति । \newline
53. द॒धा॒ति॒ रोच॑ते॒ रोच॑ते दधाति दधाति॒ रोच॑ते । \newline
54. रोच॑त ए॒वैव रोच॑ते॒ रोच॑त ए॒व । \newline
55. ए॒वाग्नये॒ ऽग्नय॑ ए॒वैवाग्नये᳚ । \newline
56. अ॒ग्नये॒ तेज॑स्वते॒ तेज॑स्वते॒ ऽग्नये॒ ऽग्नये॒ तेज॑स्वते । \newline
57. तेज॑स्वते पुरो॒डाश॑म् पुरो॒डाश॒म् तेज॑स्वते॒ तेज॑स्वते पुरो॒डाश᳚म् । \newline
58. पु॒रो॒डाश॑ म॒ष्टाक॑पाल म॒ष्टाक॑पालम् पुरो॒डाश॑म् पुरो॒डाश॑ म॒ष्टाक॑पालम् । \newline

\textbf{Ghana Paata } \newline

1. आयु॑र् दधाति दधा॒ त्यायु॒ रायु॑र् दधाति॒ सर्वꣳ॒॒ सर्व॑म् दधा॒ त्यायु॒ रायु॑र् दधाति॒ सर्व᳚म् । \newline
2. द॒धा॒ति॒ सर्वꣳ॒॒ सर्व॑म् दधाति दधाति॒ सर्व॒ मायु॒रायुः॒ सर्व॑म् दधाति दधाति॒ सर्व॒ मायुः॑ । \newline
3. सर्व॒ मायु॒ रायुः॒ सर्वꣳ॒॒ सर्व॒ मायु॑ रेत्ये॒ त्यायुः॒ सर्वꣳ॒॒ सर्व॒ मायु॑ रेति । \newline
4. आयु॑ रेत्ये॒ त्यायु॒ रायु॑ रेत्य॒ग्नये॒ ऽग्नय॑ ए॒त्यायु॒ रायु॑ रेत्य॒ग्नये᳚ । \newline
5. ए॒त्य॒ग्नये॒ ऽग्नय॑ एत्ये त्य॒ग्नये॑ जा॒तवे॑दसे जा॒तवे॑दसे॒ ऽग्नय॑ एत्ये त्य॒ग्नये॑ जा॒तवे॑दसे । \newline
6. अ॒ग्नये॑ जा॒तवे॑दसे जा॒तवे॑दसे॒ ऽग्नये॒ ऽग्नये॑ जा॒तवे॑दसे पुरो॒डाश॑म् पुरो॒डाश॑म् जा॒तवे॑दसे॒ ऽग्नये॒ ऽग्नये॑ जा॒तवे॑दसे पुरो॒डाश᳚म् । \newline
7. जा॒तवे॑दसे पुरो॒डाश॑म् पुरो॒डाश॑म् जा॒तवे॑दसे जा॒तवे॑दसे पुरो॒डाश॑ म॒ष्टाक॑पाल म॒ष्टाक॑पालम् पुरो॒डाश॑म् जा॒तवे॑दसे जा॒तवे॑दसे पुरो॒डाश॑ म॒ष्टाक॑पालम् । \newline
8. जा॒तवे॑दस॒ इति॑ जा॒त - वे॒द॒से॒ । \newline
9. पु॒रो॒डाश॑ म॒ष्टाक॑पाल म॒ष्टाक॑पालम् पुरो॒डाश॑म् पुरो॒डाश॑ म॒ष्टाक॑पाल॒म् निर् णिर॒ष्टाक॑पालम् पुरो॒डाश॑म् पुरो॒डाश॑ म॒ष्टाक॑पाल॒म् निः । \newline
10. अ॒ष्टाक॑पाल॒म् निर् णिर॒ष्टाक॑पाल म॒ष्टाक॑पाल॒म् निर् व॑पेद् वपे॒न् निर॒ष्टाक॑पाल म॒ष्टाक॑पाल॒म् निर् व॑पेत् । \newline
11. अ॒ष्टाक॑पाल॒मित्य॒ष्टा - क॒पा॒ल॒म् । \newline
12. निर् व॑पेद् वपे॒न् निर् णिर् व॑पे॒द् भूति॑कामो॒ भूति॑कामो वपे॒न् निर् णिर् व॑पे॒द् भूति॑कामः । \newline
13. व॒पे॒द् भूति॑कामो॒ भूति॑कामो वपेद् वपे॒द् भूति॑कामो॒ ऽग्नि म॒ग्निम् भूति॑कामो वपेद् वपे॒द् भूति॑कामो॒ ऽग्निम् । \newline
14. भूति॑कामो॒ ऽग्नि म॒ग्निम् भूति॑कामो॒ भूति॑कामो॒ ऽग्नि मे॒वैवाग्निम् भूति॑कामो॒ भूति॑कामो॒ ऽग्नि मे॒व । \newline
15. भूति॑काम॒ इति॒ भूति॑ - का॒मः॒ । \newline
16. अ॒ग्नि मे॒वैवाग्नि म॒ग्नि मे॒व जा॒तवे॑दसम् जा॒तवे॑दस मे॒वाग्नि म॒ग्नि मे॒व जा॒तवे॑दसम् । \newline
17. ए॒व जा॒तवे॑दसम् जा॒तवे॑दस मे॒वैव जा॒तवे॑दसꣳ॒॒ स्वेन॒ स्वेन॑ जा॒तवे॑दस मे॒वैव जा॒तवे॑दसꣳ॒॒ स्वेन॑ । \newline
18. जा॒तवे॑दसꣳ॒॒ स्वेन॒ स्वेन॑ जा॒तवे॑दसम् जा॒तवे॑दसꣳ॒॒ स्वेन॑ भाग॒धेये॑न भाग॒धेये॑न॒ स्वेन॑ जा॒तवे॑दसम् जा॒तवे॑दसꣳ॒॒ स्वेन॑ भाग॒धेये॑न । \newline
19. जा॒तवे॑दस॒मिति॑ जा॒त - वे॒द॒स॒म् । \newline
20. स्वेन॑ भाग॒धेये॑न भाग॒धेये॑न॒ स्वेन॒ स्वेन॑ भाग॒धेये॒नोपोप॑ भाग॒धेये॑न॒ स्वेन॒ स्वेन॑ भाग॒धेये॒नोप॑ । \newline
21. भा॒ग॒धेये॒नोपोप॑ भाग॒धेये॑न भाग॒धेये॒नोप॑ धावति धाव॒ त्युप॑ भाग॒धेये॑न भाग॒धेये॒नोप॑ धावति । \newline
22. भा॒ग॒धेये॒नेति॑ भाग - धेये॑न । \newline
23. उप॑ धावति धाव॒ त्युपोप॑ धावति॒ स स धा॑व॒ त्युपोप॑ धावति॒ सः । \newline
24. धा॒व॒ति॒ स स धा॑वति धावति॒ स ए॒वैव स धा॑वति धावति॒ स ए॒व । \newline
25. स ए॒वैव स स ए॒वैन॑ मेन मे॒व स स ए॒वैन᳚म् । \newline
26. ए॒वैन॑ मेन मे॒वैवैन॒म् भूति॒म् भूति॑ मेन मे॒वैवैन॒म् भूति᳚म् । \newline
27. ए॒न॒म् भूति॒म् भूति॑ मेन मेन॒म् भूति॑म् गमयति गमयति॒ भूति॑ मेन मेन॒म् भूति॑म् गमयति । \newline
28. भूति॑म् गमयति गमयति॒ भूति॒म् भूति॑म् गमयति॒ भव॑ति॒ भव॑ति गमयति॒ भूति॒म् भूति॑म् गमयति॒ भव॑ति । \newline
29. ग॒म॒य॒ति॒ भव॑ति॒ भव॑ति गमयति गमयति॒ भव॑ त्ये॒वैव भव॑ति गमयति गमयति॒ भव॑त्ये॒व । \newline
30. भव॑ त्ये॒वैव भव॑ति॒ भव॑ त्ये॒वाग्नये॒ ऽग्नय॑ ए॒व भव॑ति॒ भव॑ त्ये॒वाग्नये᳚ । \newline
31. ए॒वाग्नये॒ ऽग्नय॑ ए॒वैवाग्नये॒ रुक्म॑ते॒ रुक्म॑ते॒ ऽग्नय॑ ए॒वैवाग्नये॒ रुक्म॑ते । \newline
32. अ॒ग्नये॒ रुक्म॑ते॒ रुक्म॑ते॒ ऽग्नये॒ ऽग्नये॒ रुक्म॑ते पुरो॒डाश॑म् पुरो॒डाशꣳ॒॒ रुक्म॑ते॒ ऽग्नये॒ ऽग्नये॒ रुक्म॑ते पुरो॒डाश᳚म् । \newline
33. रुक्म॑ते पुरो॒डाश॑म् पुरो॒डाशꣳ॒॒ रुक्म॑ते॒ रुक्म॑ते पुरो॒डाश॑ म॒ष्टाक॑पाल म॒ष्टाक॑पालम् पुरो॒डाशꣳ॒॒ रुक्म॑ते॒ रुक्म॑ते पुरो॒डाश॑ म॒ष्टाक॑पालम् । \newline
34. पु॒रो॒डाश॑ म॒ष्टाक॑पाल म॒ष्टाक॑पालम् पुरो॒डाश॑म् पुरो॒डाश॑ म॒ष्टाक॑पाल॒म् निर् णिर॒ष्टाक॑पालम् पुरो॒डाश॑म् पुरो॒डाश॑ म॒ष्टाक॑पाल॒म् निः । \newline
35. अ॒ष्टाक॑पाल॒म् निर् णिर॒ष्टाक॑पाल म॒ष्टाक॑पाल॒म् निर् व॑पेद् वपे॒न् निर॒ष्टाक॑पाल म॒ष्टाक॑पाल॒म् निर् व॑पेत् । \newline
36. अ॒ष्टाक॑पाल॒मित्य॒ष्टा - क॒पा॒ल॒म् । \newline
37. निर् व॑पेद् वपे॒न् निर् णिर् व॑पे॒द् रुक्का॑मो॒ रुक्का॑मो वपे॒न् निर् णिर् व॑पे॒द् रुक्का॑मः । \newline
38. व॒पे॒द् रुक्का॑मो॒ रुक्का॑मो वपेद् वपे॒द् रुक्का॑मो॒ ऽग्नि म॒ग्निꣳ रुक्का॑मो वपेद् वपे॒द् रुक्का॑मो॒ ऽग्निम् । \newline
39. रुक्का॑मो॒ ऽग्नि म॒ग्निꣳ रुक्का॑मो॒ रुक्का॑मो॒ ऽग्नि मे॒वैवाग्निꣳ रुक्का॑मो॒ रुक्का॑मो॒ ऽग्नि मे॒व । \newline
40. रुक्का॑म॒ इति॒ रुक् - का॒मः॒ । \newline
41. अ॒ग्नि मे॒वैवाग्नि म॒ग्नि मे॒व रुक्म॑न्तꣳ॒॒ रुक्म॑न्त मे॒वाग्नि म॒ग्नि मे॒व रुक्म॑न्तम् । \newline
42. ए॒व रुक्म॑न्तꣳ॒॒ रुक्म॑न्त मे॒वैव रुक्म॑न्तꣳ॒॒ स्वेन॒ स्वेन॒ रुक्म॑न्त मे॒वैव रुक्म॑न्तꣳ॒॒ स्वेन॑ । \newline
43. रुक्म॑न्तꣳ॒॒ स्वेन॒ स्वेन॒ रुक्म॑न्तꣳ॒॒ रुक्म॑न्तꣳ॒॒ स्वेन॑ भाग॒धेये॑न भाग॒धेये॑न॒ स्वेन॒ रुक्म॑न्तꣳ॒॒ रुक्म॑न्तꣳ॒॒ स्वेन॑ भाग॒धेये॑न । \newline
44. स्वेन॑ भाग॒धेये॑न भाग॒धेये॑न॒ स्वेन॒ स्वेन॑ भाग॒धेये॒नोपोप॑ भाग॒धेये॑न॒ स्वेन॒ स्वेन॑ भाग॒धेये॒नोप॑ । \newline
45. भा॒ग॒धेये॒नोपोप॑ भाग॒धेये॑न भाग॒धेये॒नोप॑ धावति धाव॒त्युप॑ भाग॒धेये॑न भाग॒धेये॒नोप॑ धावति । \newline
46. भा॒ग॒धेये॒नेति॑ भाग - धेये॑न । \newline
47. उप॑ धावति धाव॒ त्युपोप॑ धावति॒ स स धा॑व॒ त्युपोप॑ धावति॒ सः । \newline
48. धा॒व॒ति॒ स स धा॑वति धावति॒ स ए॒वैव स धा॑वति धावति॒ स ए॒व । \newline
49. स ए॒वैव स स ए॒वास्मि॑न् नस्मिन् ने॒व स स ए॒वास्मिन्न्॑ । \newline
50. ए॒वास्मि॑न् नस्मिन् ने॒वैवास्मि॒न् रुचꣳ॒॒ रुच॑ मस्मिन् ने॒वैवास्मि॒न् रुच᳚म् । \newline
51. अ॒स्मि॒न् रुचꣳ॒॒ रुच॑ मस्मिन् नस्मि॒न् रुच॑म् दधाति दधाति॒ रुच॑ मस्मिन् नस्मि॒न् रुच॑म् दधाति । \newline
52. रुच॑म् दधाति दधाति॒ रुचꣳ॒॒ रुच॑म् दधाति॒ रोच॑ते॒ रोच॑ते दधाति॒ रुचꣳ॒॒ रुच॑म् दधाति॒ रोच॑ते । \newline
53. द॒धा॒ति॒ रोच॑ते॒ रोच॑ते दधाति दधाति॒ रोच॑त ए॒वैव रोच॑ते दधाति दधाति॒ रोच॑त ए॒व । \newline
54. रोच॑त ए॒वैव रोच॑ते॒ रोच॑त ए॒वाग्नये॒ ऽग्नय॑ ए॒व रोच॑ते॒ रोच॑त ए॒वाग्नये᳚ । \newline
55. ए॒वाग्नये॒ ऽग्नय॑ ए॒वैवाग्नये॒ तेज॑स्वते॒ तेज॑स्वते॒ ऽग्नय॑ ए॒वैवाग्नये॒ तेज॑स्वते । \newline
56. अ॒ग्नये॒ तेज॑स्वते॒ तेज॑स्वते॒ ऽग्नये॒ ऽग्नये॒ तेज॑स्वते पुरो॒डाश॑म् पुरो॒डाश॒म् तेज॑स्वते॒ ऽग्नये॒ ऽग्नये॒ तेज॑स्वते पुरो॒डाश᳚म् । \newline
57. तेज॑स्वते पुरो॒डाश॑म् पुरो॒डाश॒म् तेज॑स्वते॒ तेज॑स्वते पुरो॒डाश॑ म॒ष्टाक॑पाल म॒ष्टाक॑पालम् पुरो॒डाश॒म् तेज॑स्वते॒ तेज॑स्वते पुरो॒डाश॑ म॒ष्टाक॑पालम् । \newline
58. पु॒रो॒डाश॑ म॒ष्टाक॑पाल म॒ष्टाक॑पालम् पुरो॒डाश॑म् पुरो॒डाश॑ म॒ष्टाक॑पाल॒म् निर् णिर॒ष्टाक॑पालम् पुरो॒डाश॑म् पुरो॒डाश॑ म॒ष्टाक॑पाल॒म् निः । \newline
\pagebreak
\markright{ TS 2.2.3.4  \hfill https://www.vedavms.in \hfill}
\addcontentsline{toc}{section}{ TS 2.2.3.4 }
\section*{ TS 2.2.3.4 }

\textbf{TS 2.2.3.4 } \newline
\textbf{Samhita Paata} \newline

-म॒ष्टाक॑पालं॒ निर्व॑पे॒त् तेज॑स्कामो॒ऽग्निमे॒व तेज॑स्वन्तꣳ॒॒ स्वेन॑ भाग॒धेये॒नोप॑ धावति॒ स ए॒वास्मि॒न् तेजो॑ दधाति तेज॒स्व्ये॑व भ॑वत्य॒ग्नये॑ साह॒न्त्याय॑ पुरो॒डाश॑म॒ष्टाक॑पालं॒ निर्व॑पे॒थ् सीक्ष॑माणो॒ ऽग्निमे॒व सा॑ह॒न्त्यꣳ स्वेन॑ भाग॒धेये॒नोप॑ धावति॒ तेनै॒व स॑हते॒ यꣳ सीक्ष॑ते ॥ \newline

\textbf{Pada Paata} \newline

अ॒ष्टाक॑पाल॒मित्य॒ष्टा - क॒पा॒ल॒म् । निरिति॑ । व॒पे॒त् । तेज॑स्काम॒ इति॒ तेजः॑ - का॒मः॒ । अ॒ग्निम् । ए॒व । तेज॑स्वन्तम् । स्वेन॑ । भा॒ग॒धेये॒नेति॑ भाग-धेये॑न । उपेति॑ । धा॒व॒ति॒ । सः । ए॒व । अ॒स्मि॒न्न् । तेजः॑ । द॒धा॒ति॒ । ते॒ज॒स्वी । ए॒व । भ॒व॒ति॒ । अ॒ग्नये᳚ । सा॒ह॒न्त्याय॑ । पु॒रो॒डाश᳚म् । अ॒ष्टाक॑पाल॒मित्य॒ष्टा - क॒पा॒ल॒म् । निरिति॑ । व॒पे॒त् । सीक्ष॑माणः । अ॒ग्निम् । ए॒व । सा॒ह॒न्त्यम् । स्वेन॑ । भा॒ग॒धेये॒नेति॑ भाग - धेये॑न । उपेति॑ । धा॒व॒ति॒ । तेन॑ । ए॒व । स॒ह॒ते॒ । यम् । सीक्ष॑ते ॥  \newline


\textbf{Krama Paata} \newline

अ॒ष्टाक॑पाल॒म् निः । अ॒ष्टाक॑पाल॒मित्य॒ष्टा - क॒पा॒ल॒म् । निर् व॑पेत् । व॒पे॒त् तेज॑स्कामः । तेज॑स्कामो॒ऽग्निम् । तेज॑स्काम॒ इति॒ तेजः॑ - का॒मः॒ । अ॒ग्निमे॒व । ए॒व तेज॑स्वन्तम् । तेज॑स्वन्तꣳ॒॒ स्वेन॑ । स्वेन॑ भाग॒धेये॑न । भा॒ग॒धेये॒नोप॑ । भा॒ग॒धेये॒नेति॑ भाग - धेये॑न । उप॑ धावति । धा॒व॒ति॒ सः । स ए॒व । ए॒वास्मिन्न्॑ । अ॒स्मि॒न् तेजः॑ । तेजो॑ दधाति । द॒धा॒ति॒ ते॒ज॒स्वी । ते॒ज॒स्व्ये॑व । ए॒व भ॑वति । भ॒व॒त्य॒ग्नये᳚ । अ॒ग्नये॑ साह॒न्त्याय॑ । सा॒ह॒न्त्याय॑ पुरो॒डाश᳚म् । पु॒रो॒डाश॑म॒ष्टाक॑पालम् । अ॒ष्टाक॑पाल॒म् निः । अ॒ष्टाक॑पाल॒मित्य॒ष्टा - क॒पा॒ल॒म् । निर् व॑पेत् । व॒पे॒थ् सीक्ष॑माणः । सीक्ष॑माणो॒ऽग्निम् । अ॒ग्निमे॒व । ए॒व सा॑ह॒न्त्यम् । सा॒ह॒न्त्यꣳ स्वेन॑ । स्वेन॑ भाग॒धेये॑न । भा॒ग॒धेये॒नोप॑ । भा॒ग॒धेये॒नेति॑ भाग - धेये॑न । उप॑ धावति । धा॒व॒ति॒ तेन॑ । तेनै॒व । ए॒व स॑हते । स॒ह॒ते॒ यम् । यꣳ सीक्ष॑ते । सीक्ष॑त॒ इति॒ सीक्ष॑ते । \newline

\textbf{Jatai Paata} \newline

1. अ॒ष्टाक॑पाल॒म् निर् णिर॒ष्टाक॑पाल म॒ष्टाक॑पाल॒म् निः । \newline
2. अ॒ष्टाक॑पाल॒मित्य॒ष्टा - क॒पा॒ल॒म् । \newline
3. निर् व॑पेद् वपे॒न् निर् णिर् व॑पेत् । \newline
4. व॒पे॒त् तेज॑स्काम॒ स्तेज॑स्कामो वपेद् वपे॒त् तेज॑स्कामः । \newline
5. तेज॑स्कामो॒ ऽग्नि म॒ग्निम् तेज॑स्काम॒ स्तेज॑स्कामो॒ ऽग्निम् । \newline
6. तेज॑स्काम॒ इति॒ तेजः॑ - का॒मः॒ । \newline
7. अ॒ग्नि मे॒वैवाग्नि म॒ग्नि मे॒व । \newline
8. ए॒व तेज॑स्वन्त॒म् तेज॑स्वन्त मे॒वैव तेज॑स्वन्तम् । \newline
9. तेज॑स्वन्तꣳ॒॒ स्वेन॒ स्वेन॒ तेज॑स्वन्त॒म् तेज॑स्वन्तꣳ॒॒ स्वेन॑ । \newline
10. स्वेन॑ भाग॒धेये॑न भाग॒धेये॑न॒ स्वेन॒ स्वेन॑ भाग॒धेये॑न । \newline
11. भा॒ग॒धेये॒नोपोप॑ भाग॒धेये॑न भाग॒धेये॒नोप॑ । \newline
12. भा॒ग॒धेये॒नेति॑ भाग - धेये॑न । \newline
13. उप॑ धावति धाव॒ त्युपोप॑ धावति । \newline
14. धा॒व॒ति॒ स स धा॑वति धावति॒ सः । \newline
15. स ए॒वैव स स ए॒व । \newline
16. ए॒वास्मि॑न् नस्मिन् ने॒वैवास्मिन्न्॑ । \newline
17. अ॒स्मि॒न् तेज॒ स्तेजो᳚ ऽस्मिन् नस्मि॒न् तेजः॑ । \newline
18. तेजो॑ दधाति दधाति॒ तेज॒ स्तेजो॑ दधाति । \newline
19. द॒धा॒ति॒ ते॒ज॒स्वी ते॑ज॒स्वी द॑धाति दधाति तेज॒स्वी । \newline
20. ते॒ज॒ स्व्ये॑वैव ते॑ज॒स्वी ते॑ज॒ स्व्ये॑व । \newline
21. ए॒व भ॑वति भव त्ये॒वैव भ॑वति । \newline
22. भ॒व॒ त्य॒ग्नये॒ ऽग्नये॑ भवति भव त्य॒ग्नये᳚ । \newline
23. अ॒ग्नये॑ साह॒न्त्याय॑ साह॒न्त्याया॒ग्नये॒ ऽग्नये॑ साह॒न्त्याय॑ । \newline
24. सा॒ह॒न्त्याय॑ पुरो॒डाश॑म् पुरो॒डाशꣳ॑ साह॒न्त्याय॑ साह॒न्त्याय॑ पुरो॒डाश᳚म् । \newline
25. पु॒रो॒डाश॑ म॒ष्टाक॑पाल म॒ष्टाक॑पालम् पुरो॒डाश॑म् पुरो॒डाश॑ म॒ष्टाक॑पालम् । \newline
26. अ॒ष्टाक॑पाल॒म् निर् णिर॒ष्टाक॑पाल म॒ष्टाक॑पाल॒म् निः । \newline
27. अ॒ष्टाक॑पाल॒मित्य॒ष्टा - क॒पा॒ल॒म् । \newline
28. निर् व॑पेद् वपे॒न् निर् णिर् व॑पेत् । \newline
29. व॒पे॒थ् सीक्ष॑माणः॒ सीक्ष॑माणो वपेद् वपे॒थ् सीक्ष॑माणः । \newline
30. सीक्ष॑माणो॒ ऽग्नि म॒ग्निꣳ सीक्ष॑माणः॒ सीक्ष॑माणो॒ ऽग्निम् । \newline
31. अ॒ग्नि मे॒वैवाग्नि म॒ग्नि मे॒व । \newline
32. ए॒व सा॑ह॒न्त्यꣳ सा॑ह॒न्त्य मे॒वैव सा॑ह॒न्त्यम् । \newline
33. सा॒ह॒न्त्यꣳ स्वेन॒ स्वेन॑ साह॒न्त्यꣳ सा॑ह॒न्त्यꣳ स्वेन॑ । \newline
34. स्वेन॑ भाग॒धेये॑न भाग॒धेये॑न॒ स्वेन॒ स्वेन॑ भाग॒धेये॑न । \newline
35. भा॒ग॒धेये॒नोपोप॑ भाग॒धेये॑न भाग॒धेये॒नोप॑ । \newline
36. भा॒ग॒धेये॒नेति॑ भाग - धेये॑न । \newline
37. उप॑ धावति धाव॒ त्युपोप॑ धावति । \newline
38. धा॒व॒ति॒ तेन॒ तेन॑ धावति धावति॒ तेन॑ । \newline
39. तेनै॒वैव तेन॒ तेनै॒व । \newline
40. ए॒व स॑हते सहत ए॒वैव स॑हते । \newline
41. स॒ह॒ते॒ यं ॅयꣳ स॑हते सहते॒ यम् । \newline
42. यꣳ सीक्ष॑ते॒ सीक्ष॑ते॒ यं ॅयꣳ सीक्ष॑ते । \newline
43. सीक्ष॑त॒ इति॒ सीक्ष॑ते । \newline

\textbf{Ghana Paata } \newline

1. अ॒ष्टाक॑पाल॒म् निर् णिर॒ष्टाक॑पाल म॒ष्टाक॑पाल॒म् निर् व॑पेद् वपे॒न् निर॒ष्टाक॑पाल म॒ष्टाक॑पाल॒म् निर् व॑पेत् । \newline
2. अ॒ष्टाक॑पाल॒मित्य॒ष्टा - क॒पा॒ल॒म् । \newline
3. निर् व॑पेद् वपे॒न् निर् णिर् व॑पे॒त् तेज॑स्काम॒ स्तेज॑स्कामो वपे॒न् निर् णिर् व॑पे॒त् तेज॑स्कामः । \newline
4. व॒पे॒त् तेज॑स्काम॒ स्तेज॑स्कामो वपेद् वपे॒त् तेज॑स्कामो॒ ऽग्नि म॒ग्निम् तेज॑स्कामो वपेद् वपे॒त् तेज॑स्कामो॒ ऽग्निम् । \newline
5. तेज॑स्कामो॒ ऽग्नि म॒ग्निम् तेज॑स्काम॒ स्तेज॑स्कामो॒ ऽग्नि मे॒वैवाग्निम् तेज॑स्काम॒ स्तेज॑स्कामो॒ ऽग्नि मे॒व । \newline
6. तेज॑स्काम॒ इति॒ तेजः॑ - का॒मः॒ । \newline
7. अ॒ग्नि मे॒वैवाग्नि म॒ग्नि मे॒व तेज॑स्वन्त॒म् तेज॑स्वन्त मे॒वाग्नि म॒ग्नि मे॒व तेज॑स्वन्तम् । \newline
8. ए॒व तेज॑स्वन्त॒म् तेज॑स्वन्त मे॒वैव तेज॑स्वन्तꣳ॒॒ स्वेन॒ स्वेन॒ तेज॑स्वन्त मे॒वैव तेज॑स्वन्तꣳ॒॒ स्वेन॑ । \newline
9. तेज॑स्वन्तꣳ॒॒ स्वेन॒ स्वेन॒ तेज॑स्वन्त॒म् तेज॑स्वन्तꣳ॒॒ स्वेन॑ भाग॒धेये॑न भाग॒धेये॑न॒ स्वेन॒ तेज॑स्वन्त॒म् तेज॑स्वन्तꣳ॒॒ स्वेन॑ भाग॒धेये॑न । \newline
10. स्वेन॑ भाग॒धेये॑न भाग॒धेये॑न॒ स्वेन॒ स्वेन॑ भाग॒धेये॒नोपोप॑ भाग॒धेये॑न॒ स्वेन॒ स्वेन॑ भाग॒धेये॒नोप॑ । \newline
11. भा॒ग॒धेये॒नोपोप॑ भाग॒धेये॑न भाग॒धेये॒नोप॑ धावति धाव॒त्युप॑ भाग॒धेये॑न भाग॒धेये॒नोप॑ धावति । \newline
12. भा॒ग॒धेये॒नेति॑ भाग - धेये॑न । \newline
13. उप॑ धावति धाव॒ त्युपोप॑ धावति॒ स स धा॑व॒ त्युपोप॑ धावति॒ सः । \newline
14. धा॒व॒ति॒ स स धा॑वति धावति॒ स ए॒वैव स धा॑वति धावति॒ स ए॒व । \newline
15. स ए॒वैव स स ए॒वास्मि॑न् नस्मिन् ने॒व स स ए॒वास्मिन्न्॑ । \newline
16. ए॒वास्मि॑न् नस्मिन् ने॒वैवास्मि॒न् तेज॒ स्तेजो᳚ ऽस्मिन् ने॒वैवास्मि॒न् तेजः॑ । \newline
17. अ॒स्मि॒न् तेज॒ स्तेजो᳚ ऽस्मिन् नस्मि॒न् तेजो॑ दधाति दधाति॒ तेजो᳚ ऽस्मिन् नस्मि॒न् तेजो॑ दधाति । \newline
18. तेजो॑ दधाति दधाति॒ तेज॒ स्तेजो॑ दधाति तेज॒स्वी ते॑ज॒स्वी द॑धाति॒ तेज॒ स्तेजो॑ दधाति तेज॒स्वी । \newline
19. द॒धा॒ति॒ ते॒ज॒स्वी ते॑ज॒स्वी द॑धाति दधाति तेज॒ स्व्ये॑वैव ते॑ज॒स्वी द॑धाति दधाति तेज॒ स्व्ये॑व । \newline
20. ते॒ज॒ स्व्ये॑वैव ते॑ज॒स्वी ते॑ज॒ स्व्ये॑व भ॑वति भवत्ये॒व ते॑ज॒स्वी ते॑ज॒ स्व्ये॑व भ॑वति । \newline
21. ए॒व भ॑वति भव त्ये॒वैव भ॑व त्य॒ग्नये॒ ऽग्नये॑ भव त्ये॒वैव भ॑व त्य॒ग्नये᳚ । \newline
22. भ॒व॒ त्य॒ग्नये॒ ऽग्नये॑ भवति भव त्य॒ग्नये॑ साह॒न्त्याय॑ साह॒न्त्याया॒ग्नये॑ भवति भवत्य॒ग्नये॑ साह॒न्त्याय॑ । \newline
23. अ॒ग्नये॑ साह॒न्त्याय॑ साह॒न्त्याया॒ ग्नये॒ ऽग्नये॑ साह॒न्त्याय॑ पुरो॒डाश॑म् पुरो॒डाशꣳ॑ साह॒न्त्याया॒ग्नये॒ ऽग्नये॑ साह॒न्त्याय॑ पुरो॒डाश᳚म् । \newline
24. सा॒ह॒न्त्याय॑ पुरो॒डाश॑म् पुरो॒डाशꣳ॑ साह॒न्त्याय॑ साह॒न्त्याय॑ पुरो॒डाश॑ म॒ष्टाक॑पाल म॒ष्टाक॑पालम् पुरो॒डाशꣳ॑ साह॒न्त्याय॑ साह॒न्त्याय॑ पुरो॒डाश॑ म॒ष्टाक॑पालम् । \newline
25. पु॒रो॒डाश॑ म॒ष्टाक॑पाल म॒ष्टाक॑पालम् पुरो॒डाश॑म् पुरो॒डाश॑ म॒ष्टाक॑पाल॒म् निर् णिर॒ष्टाक॑पालम् पुरो॒डाश॑म् पुरो॒डाश॑ म॒ष्टाक॑पाल॒म् निः । \newline
26. अ॒ष्टाक॑पाल॒म् निर् णिर॒ष्टाक॑पाल म॒ष्टाक॑पाल॒म् निर् व॑पेद् वपे॒न् निर॒ष्टाक॑पाल म॒ष्टाक॑पाल॒म् निर् व॑पेत् । \newline
27. अ॒ष्टाक॑पाल॒मित्य॒ष्टा - क॒पा॒ल॒म् । \newline
28. निर् व॑पेद् वपे॒न् निर् णिर् व॑पे॒थ् सीक्ष॑माणः॒ सीक्ष॑माणो वपे॒न् निर् णिर् व॑पे॒थ् सीक्ष॑माणः । \newline
29. व॒पे॒थ् सीक्ष॑माणः॒ सीक्ष॑माणो वपेद् वपे॒थ् सीक्ष॑माणो॒ ऽग्नि म॒ग्निꣳ सीक्ष॑माणो वपेद् वपे॒थ् सीक्ष॑माणो॒ ऽग्निम् । \newline
30. सीक्ष॑माणो॒ ऽग्नि म॒ग्निꣳ सीक्ष॑माणः॒ सीक्ष॑माणो॒ ऽग्नि मे॒वैवाग्निꣳ सीक्ष॑माणः॒ सीक्ष॑माणो॒ ऽग्नि मे॒व । \newline
31. अ॒ग्नि मे॒वैवाग्नि म॒ग्नि मे॒व सा॑ह॒न्त्यꣳ सा॑ह॒न्त्य मे॒वाग्नि म॒ग्नि मे॒व सा॑ह॒न्त्यम् । \newline
32. ए॒व सा॑ह॒न्त्यꣳ सा॑ह॒न्त्य मे॒वैव सा॑ह॒न्त्यꣳ स्वेन॒ स्वेन॑ साह॒न्त्य मे॒वैव सा॑ह॒न्त्यꣳ स्वेन॑ । \newline
33. सा॒ह॒न्त्यꣳ स्वेन॒ स्वेन॑ साह॒न्त्यꣳ सा॑ह॒न्त्यꣳ स्वेन॑ भाग॒धेये॑न भाग॒धेये॑न॒ स्वेन॑ साह॒न्त्यꣳ सा॑ह॒न्त्यꣳ स्वेन॑ भाग॒धेये॑न । \newline
34. स्वेन॑ भाग॒धेये॑न भाग॒धेये॑न॒ स्वेन॒ स्वेन॑ भाग॒धेये॒नोपोप॑ भाग॒धेये॑न॒ स्वेन॒ स्वेन॑ भाग॒धेये॒नोप॑ । \newline
35. भा॒ग॒धेये॒नोपोप॑ भाग॒धेये॑न भाग॒धेये॒नोप॑ धावति धाव॒त्युप॑ भाग॒धेये॑न भाग॒धेये॒नोप॑ धावति । \newline
36. भा॒ग॒धेये॒नेति॑ भाग - धेये॑न । \newline
37. उप॑ धावति धाव॒ त्युपोप॑ धावति॒ तेन॒ तेन॑ धाव॒ त्युपोप॑ धावति॒ तेन॑ । \newline
38. धा॒व॒ति॒ तेन॒ तेन॑ धावति धावति॒ तेनै॒वैव तेन॑ धावति धावति॒ तेनै॒व । \newline
39. तेनै॒वैव तेन॒ तेनै॒व स॑हते सहत ए॒व तेन॒ तेनै॒व स॑हते । \newline
40. ए॒व स॑हते सहत ए॒वैव स॑हते॒ यं ॅयꣳ स॑हत ए॒वैव स॑हते॒ यम् । \newline
41. स॒ह॒ते॒ यं ॅयꣳ स॑हते सहते॒ यꣳ सीक्ष॑ते॒ सीक्ष॑ते॒ यꣳ स॑हते सहते॒ यꣳ सीक्ष॑ते । \newline
42. यꣳ सीक्ष॑ते॒ सीक्ष॑ते॒ यं ॅयꣳ सीक्ष॑ते । \newline
43. सीक्ष॑त॒ इति॒ सीक्ष॑ते । \newline
\pagebreak
\markright{ TS 2.2.4.1  \hfill https://www.vedavms.in \hfill}
\addcontentsline{toc}{section}{ TS 2.2.4.1 }
\section*{ TS 2.2.4.1 }

\textbf{TS 2.2.4.1 } \newline
\textbf{Samhita Paata} \newline

अ॒ग्नयेऽन्न॑वते पुरो॒डाश॑म॒ष्टाक॑पालं॒ निर्व॑पे॒द्यः का॒मये॒तान्न॑वान्थ्-स्या॒मित्य॒ग्नि-मे॒वा-न्न॑वन्तꣳ॒॒ स्वेन॑ भाग॒धेये॒नोप॑ धावति॒ स ए॒वैन॒मन्न॑वन्तं करो॒त्यन्न॑वाने॒व भ॑वत्य॒ग्नये᳚ऽन्ना॒दाय॑ पुरो॒डाश॑म॒ष्टाक॑पालं॒ निर्व॑पे॒द्यः का॒मये॑तान्ना॒दः स्या॒मित्य॒ग्नि-मे॒वान्ना॒दꣳ स्वेन॑ भाग॒धेये॒नोप॑ धावति॒ स ए॒वैन॑मन्ना॒दं क॑रोत्यन्ना॒द - [  ] \newline

\textbf{Pada Paata} \newline

अ॒ग्नये᳚ । अन्न॑वत॒ इत्यन्न॑ - व॒ते॒ । पु॒रो॒डाश᳚म् । अ॒ष्टाक॑पाल॒मित्य॒ष्टा - क॒पा॒ल॒म् । निरिति॑ । व॒पे॒त् । यः । का॒मये॑त । अन्न॑वा॒नित्यन्न॑ - वा॒न् । स्या॒म् । इति॑ । अ॒ग्निम् । ए॒व । अन्न॑वन्त॒मित्यन्न॑ - व॒न्त॒म् । स्वेन॑ । भा॒ग॒धेये॒नेति॑ भाग - धेये॑न । उपेति॑ । धा॒व॒ति॒ । सः । ए॒व । ए॒न॒म् । अन्न॑वन्त॒मित्यन्न॑ - व॒न्त॒म् । क॒रो॒ति॒ । अन्न॑वा॒नित्यन्न॑ - वा॒न् । ए॒व । भ॒व॒ति॒ । अ॒ग्नये᳚ । अ॒न्ना॒दायेत्य॑न्न - अ॒दाय॑ । पु॒रो॒डाश᳚म् । अ॒ष्टाक॑पाल॒मित्य॒ष्टा - क॒पा॒ल॒म् । निरिति॑ । व॒पे॒त् । यः । का॒मये॑त । अ॒न्ना॒द इत्य॑न्न - अ॒दः । स्या॒म् । इति॑ । अ॒ग्निम् । ए॒व । अ॒न्ना॒दमित्य॑न्न - अ॒दम् । स्वेन॑ । भा॒ग॒धेये॒नेति॑ भाग - धेये॑न । उपेति॑ । धा॒व॒ति॒ । सः । ए॒व । ए॒न॒म् । अ॒न्ना॒दमित्य॑न्न - अ॒दम् । क॒रो॒ति॒ । अ॒न्ना॒द इत्य॑न्न - अ॒दः ।  \newline


\textbf{Krama Paata} \newline

अ॒ग्नये ऽन्न॑वते । अन्न॑वते पुरो॒डाश᳚म् । अन्न॑वत॒ इत्यन्न॑ - व॒ते॒ । पु॒रो॒डाश॑म॒ष्टाक॑पालम् । अ॒ष्टाक॑पाल॒म् निः । अ॒ष्टाक॑पाल॒मित्य॒ष्टा - क॒पा॒ल॒म् । निर् व॑पेत् । व॒पे॒द् यः । यः का॒मये॑त । का॒मये॒तान्न॑वान् । अन्न॑वान्थ् स्याम् । अन्न॑वा॒नित्यन्न॑ - वा॒न्॒ । स्या॒मिति॑ । इत्य॒ग्निम् । अ॒ग्निमे॒व । ए॒वान्न॑वन्तम् । अन्न॑वन्तꣳ॒॒ स्वेन॑ । अन्न॑वन्त॒मित्यन्न॑ - व॒न्त॒म् । स्वेन॑ भाग॒धेये॑न । भा॒ग॒धेये॒नोप॑ । भा॒ग॒धेये॒नेति॑ भाग - धेये॑न । उप॑ धावति । धा॒व॒ति॒ सः । स ए॒व । ए॒वैन᳚म् । ए॒न॒मन्न॑वन्तम् । अन्न॑वन्तम् करोति । अन्न॑वन्त॒मित्यन्न॑ - व॒न्त॒म् । क॒रो॒त्यन्न॑वान् । अन्न॑वाने॒व । अन्न॑वा॒नित्यन्न॑ - वा॒न्॒ । ए॒व भ॑वति । भ॒व॒त्य॒ग्नये᳚ । अ॒ग्नये᳚ ऽन्ना॒दाय॑ । अ॒न्ना॒दाय॑ पुरो॒डाश᳚म् । अ॒न्ना॒दायेत्य॑न्न - अ॒दाय॑ । पु॒रो॒डाश॑म॒ष्टाक॑पालम् । अ॒ष्टाक॑पाल॒म् निः । अ॒ष्टाक॑पाल॒मित्य॒ष्टा - क॒पा॒ल॒म् । निर् व॑पेत् । व॒पे॒द् यः । यः का॒मये॑त । का॒मये॑तान्ना॒दः । अ॒न्ना॒दः स्या᳚म् । अ॒न्ना॒द इत्य॑न्न - अ॒दः । स्या॒मिति॑ । इत्य॒ग्निम् । अ॒ग्निमे॒व । ए॒वान्ना॒दम् । अ॒न्ना॒दꣳ स्वेन॑ । अ॒न्ना॒दमित्य॑न्न - अ॒दम् । स्वेन॑ भाग॒धेये॑न । भा॒ग॒धेये॒नोप॑ । भा॒ग॒धेये॒नेति॑ भाग - धेये॑न । उप॑ धावति । धा॒व॒ति॒ सः । स ए॒व । ए॒वैन᳚म् । ए॒न॒म॒न्ना॒दम् । अ॒न्ना॒दम् क॑रोति । अ॒न्ना॒दमित्य॑न्न - अ॒दम् । क॒रो॒त्य॒न्ना॒दः । अ॒न्ना॒द ए॒व । अ॒न्ना॒द इत्य॑न्न - अ॒दः \newline

\textbf{Jatai Paata} \newline

1. अ॒ग्नये ऽन्न॑व॒ते ऽन्न॑वते॒ ऽग्नये॒ ऽग्नये ऽन्न॑वते । \newline
2. अन्न॑वते पुरो॒डाश॑म् पुरो॒डाश॒ मन्न॑व॒ते ऽन्न॑वते पुरो॒डाश᳚म् । \newline
3. अन्न॑वत॒ इत्यन्न॑ - व॒ते॒ । \newline
4. पु॒रो॒डाश॑ म॒ष्टाक॑पाल म॒ष्टाक॑पालम् पुरो॒डाश॑म् पुरो॒डाश॑ म॒ष्टाक॑पालम् । \newline
5. अ॒ष्टाक॑पाल॒म् निर् णिर॒ष्टाक॑पाल म॒ष्टाक॑पाल॒म् निः । \newline
6. अ॒ष्टाक॑पाल॒मित्य॒ष्टा - क॒पा॒ल॒म् । \newline
7. निर् व॑पेद् वपे॒न् निर् णिर् व॑पेत् । \newline
8. व॒पे॒द् यो यो व॑पेद् वपे॒द् यः । \newline
9. यः का॒मये॑त का॒मये॑त॒ यो यः का॒मये॑त । \newline
10. का॒मये॒ता न्न॑वा॒ नन्न॑वान् का॒मये॑त का॒मये॒ता न्न॑वान् । \newline
11. अन्न॑वान् थ्स्याꣳ स्या॒ मन्न॑वा॒ नन्न॑वान् थ्स्याम् । \newline
12. अन्न॑वा॒नित्यन्न॑ - वा॒न् । \newline
13. स्या॒ मितीति॑ स्याꣳ स्या॒ मिति॑ । \newline
14. इत्य॒ग्नि म॒ग्नि मिती त्य॒ग्निम् । \newline
15. अ॒ग्नि मे॒वैवाग्नि म॒ग्नि मे॒व । \newline
16. ए॒वान्न॑वन्त॒ मन्न॑वन्त मे॒वैवान्न॑वन्तम् । \newline
17. अन्न॑वन्तꣳ॒॒ स्वेन॒ स्वेनान्न॑वन्त॒ मन्न॑वन्तꣳ॒॒ स्वेन॑ । \newline
18. अन्न॑वन्त॒मित्यन्न॑ - व॒न्त॒म् । \newline
19. स्वेन॑ भाग॒धेये॑न भाग॒धेये॑न॒ स्वेन॒ स्वेन॑ भाग॒धेये॑न । \newline
20. भा॒ग॒धेये॒नोपोप॑ भाग॒धेये॑न भाग॒धेये॒नोप॑ । \newline
21. भा॒ग॒धेये॒नेति॑ भाग - धेये॑न । \newline
22. उप॑ धावति धाव॒ त्युपोप॑ धावति । \newline
23. धा॒व॒ति॒ स स धा॑वति धावति॒ सः । \newline
24. स ए॒वैव स स ए॒व । \newline
25. ए॒वैन॑ मेन मे॒वैवैन᳚म् । \newline
26. ए॒न॒ मन्न॑वन्त॒ मन्न॑वन्त मेन मेन॒ मन्न॑वन्तम् । \newline
27. अन्न॑वन्तम् करोति करो॒ त्यन्न॑वन्त॒ मन्न॑वन्तम् करोति । \newline
28. अन्न॑वन्त॒मित्यन्न॑ - व॒न्त॒म् । \newline
29. क॒रो॒ त्यन्न॑वा॒ नन्न॑वान् करोति करो॒ त्यन्न॑वान् । \newline
30. अन्न॑वा ने॒वैवान्न॑वा॒ नन्न॑वा ने॒व । \newline
31. अन्न॑वा॒नित्यन्न॑ - वा॒न् । \newline
32. ए॒व भ॑वति भव त्ये॒वैव भ॑वति । \newline
33. भ॒व॒ त्य॒ग्नये॒ ऽग्नये॑ भवति भव त्य॒ग्नये᳚ । \newline
34. अ॒ग्नये᳚ ऽन्ना॒दाया᳚ न्ना॒दाया॒ग्नये॒ ऽग्नये᳚ ऽन्ना॒दाय॑ । \newline
35. अ॒न्ना॒दाय॑ पुरो॒डाश॑म् पुरो॒डाश॑ मन्ना॒दाया᳚ न्ना॒दाय॑ पुरो॒डाश᳚म् । \newline
36. अ॒न्ना॒दायेत्य॑न्न - अ॒दाय॑ । \newline
37. पु॒रो॒डाश॑ म॒ष्टाक॑पाल म॒ष्टाक॑पालम् पुरो॒डाश॑म् पुरो॒डाश॑ म॒ष्टाक॑पालम् । \newline
38. अ॒ष्टाक॑पाल॒म् निर् णिर॒ष्टाक॑पाल म॒ष्टाक॑पाल॒म् निः । \newline
39. अ॒ष्टाक॑पाल॒मित्य॒ष्टा - क॒पा॒ल॒म् । \newline
40. निर् व॑पेद् वपे॒न् निर् णिर् व॑पेत् । \newline
41. व॒पे॒द् यो यो व॑पेद् वपे॒द् यः । \newline
42. यः का॒मये॑त का॒मये॑त॒ यो यः का॒मये॑त । \newline
43. का॒मये॑ता न्ना॒दो᳚ ऽन्ना॒दः का॒मये॑त का॒मये॑ता न्ना॒दः । \newline
44. अ॒न्ना॒दः स्याꣳ॑ स्या मन्ना॒दो᳚ ऽन्ना॒दः स्या᳚म् । \newline
45. अ॒न्ना॒द इत्य॑न्न - अ॒दः । \newline
46. स्या॒ मितीति॑ स्याꣳ स्या॒ मिति॑ । \newline
47. इत्य॒ग्नि म॒ग्नि मिती त्य॒ग्निम् । \newline
48. अ॒ग्नि मे॒वैवाग्नि म॒ग्नि मे॒व । \newline
49. ए॒वान्ना॒द म॑न्ना॒द मे॒वैवा न्ना॒दम् । \newline
50. अ॒न्ना॒दꣳ स्वेन॒ स्वेना᳚न्ना॒द म॑न्ना॒दꣳ स्वेन॑ । \newline
51. अ॒न्ना॒दमित्य॑न्न - अ॒दम् । \newline
52. स्वेन॑ भाग॒धेये॑न भाग॒धेये॑न॒ स्वेन॒ स्वेन॑ भाग॒धेये॑न । \newline
53. भा॒ग॒धेये॒नोपोप॑ भाग॒धेये॑न भाग॒धेये॒नोप॑ । \newline
54. भा॒ग॒धेये॒नेति॑ भाग - धेये॑न । \newline
55. उप॑ धावति धाव॒ त्युपोप॑ धावति । \newline
56. धा॒व॒ति॒ स स धा॑वति धावति॒ सः । \newline
57. स ए॒वैव स स ए॒व । \newline
58. ए॒वैन॑ मेन मे॒वैवैन᳚म् । \newline
59. ए॒न॒ म॒न्ना॒द म॑न्ना॒द मे॑न मेन मन्ना॒दम् । \newline
60. अ॒न्ना॒दम् क॑रोति करो त्यन्ना॒द म॑न्ना॒दम् क॑रोति । \newline
61. अ॒न्ना॒दमित्य॑न्न - अ॒दम् । \newline
62. क॒रो॒ त्य॒न्ना॒दो᳚ ऽन्ना॒दः क॑रोति करो त्यन्ना॒दः । \newline
63. अ॒न्ना॒द ए॒वैवा न्ना॒दो᳚ ऽन्ना॒द ए॒व । \newline
64. अ॒न्ना॒द इत्य॑न्न - अ॒दः । \newline

\textbf{Ghana Paata } \newline

1. अ॒ग्नये ऽन्न॑व॒ते ऽन्न॑वते॒ ऽग्नये॒ ऽग्नये ऽन्न॑वते पुरो॒डाश॑म् पुरो॒डाश॒ मन्न॑वते॒ ऽग्नये॒ ऽग्नये ऽन्न॑वते पुरो॒डाश᳚म् । \newline
2. अन्न॑वते पुरो॒डाश॑म् पुरो॒डाश॒ मन्न॑व॒ते ऽन्न॑वते पुरो॒डाश॑ म॒ष्टाक॑पाल म॒ष्टाक॑पालम् पुरो॒डाश॒ मन्न॑व॒ते ऽन्न॑वते पुरो॒डाश॑ म॒ष्टाक॑पालम् । \newline
3. अन्न॑वत॒ इत्यन्न॑ - व॒ते॒ । \newline
4. पु॒रो॒डाश॑ म॒ष्टाक॑पाल म॒ष्टाक॑पालम् पुरो॒डाश॑म् पुरो॒डाश॑ म॒ष्टाक॑पाल॒म् निर् णिर॒ष्टाक॑पालम् पुरो॒डाश॑म् पुरो॒डाश॑ म॒ष्टाक॑पाल॒म् निः । \newline
5. अ॒ष्टाक॑पाल॒म् निर् णिर॒ष्टाक॑पाल म॒ष्टाक॑पाल॒म् निर् व॑पेद् वपे॒न् निर॒ष्टाक॑पाल म॒ष्टाक॑पाल॒म् निर् व॑पेत् । \newline
6. अ॒ष्टाक॑पाल॒मित्य॒ष्टा - क॒पा॒ल॒म् । \newline
7. निर् व॑पेद् वपे॒न् निर् णिर् व॑पे॒द् यो यो व॑पे॒न् निर् णिर् व॑पे॒द् यः । \newline
8. व॒पे॒द् यो यो व॑पेद् वपे॒द् यः का॒मये॑त का॒मये॑त॒ यो व॑पेद् वपे॒द् यः का॒मये॑त । \newline
9. यः का॒मये॑त का॒मये॑त॒ यो यः का॒मये॒ता न्न॑वा॒ नन्न॑वान् का॒मये॑त॒ यो यः का॒मये॒ता न्न॑वान् । \newline
10. का॒मये॒तान्न॑वा॒ नन्न॑वान् का॒मये॑त का॒मये॒ता न्न॑वान् थ्स्याꣳ स्या॒ मन्न॑वान् का॒मये॑त का॒मये॒तान्न॑वान् थ्स्याम् । \newline
11. अन्न॑वान् थ्स्याꣳ स्या॒ मन्न॑वा॒ नन्न॑वान् थ्स्या॒ मितीति॑ स्या॒ मन्न॑वा॒ नन्न॑वान् थ्स्या॒ मिति॑ । \newline
12. अन्न॑वा॒नित्यन्न॑ - वा॒न् । \newline
13. स्या॒ मितीति॑ स्याꣳ स्या॒ मित्य॒ग्नि म॒ग्नि मिति॑ स्याꣳ स्या॒ मित्य॒ग्निम् । \newline
14. इत्य॒ग्नि म॒ग्नि मिती त्य॒ग्नि मे॒वै वाग्नि मिती त्य॒ग्नि मे॒व । \newline
15. अ॒ग्नि मे॒वैवाग्नि म॒ग्नि मे॒वा न्न॑वन्त॒ मन्न॑वन्त मे॒वाग्नि म॒ग्नि मे॒वा न्न॑वन्तम् । \newline
16. ए॒वान्न॑वन्त॒ मन्न॑वन्त मे॒वै वान्न॑वन्तꣳ॒॒ स्वेन॒ स्वेना न्न॑वन्त मे॒वैवा न्न॑वन्तꣳ॒॒ स्वेन॑ । \newline
17. अन्न॑वन्तꣳ॒॒ स्वेन॒ स्वेना न्न॑वन्त॒ मन्न॑वन्तꣳ॒॒ स्वेन॑ भाग॒धेये॑न भाग॒धेये॑न॒ स्वेनान्न॑वन्त॒ मन्न॑वन्तꣳ॒॒ स्वेन॑ भाग॒धेये॑न । \newline
18. अन्न॑वन्त॒मित्यन्न॑ - व॒न्त॒म् । \newline
19. स्वेन॑ भाग॒धेये॑न भाग॒धेये॑न॒ स्वेन॒ स्वेन॑ भाग॒धेये॒नोपोप॑ भाग॒धेये॑न॒ स्वेन॒ स्वेन॑ भाग॒धेये॒नोप॑ । \newline
20. भा॒ग॒धेये॒नोपोप॑ भाग॒धेये॑न भाग॒धेये॒नोप॑ धावति धाव॒त्युप॑ भाग॒धेये॑न भाग॒धेये॒नोप॑ धावति । \newline
21. भा॒ग॒धेये॒नेति॑ भाग - धेये॑न । \newline
22. उप॑ धावति धाव॒ त्युपोप॑ धावति॒ स स धा॑व॒ त्युपोप॑ धावति॒ सः । \newline
23. धा॒व॒ति॒ स स धा॑वति धावति॒ स ए॒वैव स धा॑वति धावति॒ स ए॒व । \newline
24. स ए॒वैव स स ए॒वैन॑ मेन मे॒व स स ए॒वैन᳚म् । \newline
25. ए॒वैन॑ मेन मे॒वैवैन॒ मन्न॑वन्त॒ मन्न॑वन्त मेन मे॒वैवैन॒ मन्न॑वन्तम् । \newline
26. ए॒न॒ मन्न॑वन्त॒ मन्न॑वन्त मेन मेन॒ मन्न॑वन्तम् करोति करो॒ त्यन्न॑वन्त मेन मेन॒ मन्न॑वन्तम् करोति । \newline
27. अन्न॑वन्तम् करोति करो॒ त्यन्न॑वन्त॒ मन्न॑वन्तम् करो॒ त्यन्न॑वा॒ नन्न॑वान् करो॒ त्यन्न॑वन्त॒ मन्न॑वन्तम् करो॒ त्यन्न॑वान् । \newline
28. अन्न॑वन्त॒मित्यन्न॑ - व॒न्त॒म् । \newline
29. क॒रो॒ त्यन्न॑वा॒ नन्न॑वान् करोति करो॒ त्यन्न॑वा ने॒वैवान्न॑वान् करोति करो॒ त्यन्न॑वा ने॒व । \newline
30. अन्न॑वा ने॒वैवा न्न॑वा॒ नन्न॑वा ने॒व भ॑वति भव त्ये॒वान्न॑वा॒ नन्न॑वा ने॒व भ॑वति । \newline
31. अन्न॑वा॒नित्यन्न॑ - वा॒न् । \newline
32. ए॒व भ॑वति भव त्ये॒वैव भ॑व त्य॒ग्नये॒ ऽग्नये॑ भव त्ये॒वैव भ॑व त्य॒ग्नये᳚ । \newline
33. भ॒व॒त्य॒ग्नये॒ ऽग्नये॑ भवति भवत्य॒ग्नये᳚ ऽन्ना॒दाया᳚ न्ना॒दाया॒ग्नये॑ भवति भवत्य॒ग्नये᳚ ऽन्ना॒दाय॑ । \newline
34. अ॒ग्नये᳚ ऽन्ना॒दाया᳚ न्ना॒दाया॒ ग्नये॒ ऽग्नये᳚ ऽन्ना॒दाय॑ पुरो॒डाश॑म् पुरो॒डाश॑ मन्ना॒दाया॒ ग्नये॒ ऽग्नये᳚ ऽन्ना॒दाय॑ पुरो॒डाश᳚म् । \newline
35. अ॒न्ना॒दाय॑ पुरो॒डाश॑म् पुरो॒डाश॑ मन्ना॒दाया᳚ न्ना॒दाय॑ पुरो॒डाश॑ म॒ष्टाक॑पाल म॒ष्टाक॑पालम् पुरो॒डाश॑ मन्ना॒दाया᳚ न्ना॒दाय॑ पुरो॒डाश॑ म॒ष्टाक॑पालम् । \newline
36. अ॒न्ना॒दायेत्य॑न्न - अ॒दाय॑ । \newline
37. पु॒रो॒डाश॑ म॒ष्टाक॑पाल म॒ष्टाक॑पालम् पुरो॒डाश॑म् पुरो॒डाश॑ म॒ष्टाक॑पाल॒म् निर् णिर॒ष्टाक॑पालम् पुरो॒डाश॑म् पुरो॒डाश॑ म॒ष्टाक॑पाल॒म् निः । \newline
38. अ॒ष्टाक॑पाल॒म् निर् णिर॒ष्टाक॑पाल म॒ष्टाक॑पाल॒म् निर् व॑पेद् वपे॒न् निर॒ष्टाक॑पाल म॒ष्टाक॑पाल॒म् निर् व॑पेत् । \newline
39. अ॒ष्टाक॑पाल॒मित्य॒ष्टा - क॒पा॒ल॒म् । \newline
40. निर् व॑पेद् वपे॒न् निर् णिर् व॑पे॒द् यो यो व॑पे॒न् निर् णिर् व॑पे॒द् यः । \newline
41. व॒पे॒द् यो यो व॑पेद् वपे॒द् यः का॒मये॑त का॒मये॑त॒ यो व॑पेद् वपे॒द् यः का॒मये॑त । \newline
42. यः का॒मये॑त का॒मये॑त॒ यो यः का॒मये॑ता न्ना॒दो᳚ ऽन्ना॒दः का॒मये॑त॒ यो यः का॒मये॑ता न्ना॒दः । \newline
43. का॒मये॑ता न्ना॒दो᳚ ऽन्ना॒दः का॒मये॑त का॒मये॑ता न्ना॒दः स्याꣳ॑ स्या मन्ना॒दः का॒मये॑त का॒मये॑ता न्ना॒दः स्या᳚म् । \newline
44. अ॒न्ना॒दः स्याꣳ॑ स्या मन्ना॒दो᳚ ऽन्ना॒दः स्या॒ मितीति॑ स्या मन्ना॒दो᳚ ऽन्ना॒दः स्या॒ मिति॑ । \newline
45. अ॒न्ना॒द इत्य॑न्न - अ॒दः । \newline
46. स्या॒ मितीति॑ स्याꣳ स्या॒ मित्य॒ग्नि म॒ग्नि मिति॑ स्याꣳ स्या॒ मित्य॒ग्निम् । \newline
47. इत्य॒ग्नि म॒ग्नि मितीत्य॒ग्नि मे॒वैवाग्नि मितीत्य॒ग्नि मे॒व । \newline
48. अ॒ग्नि मे॒वैवाग्नि म॒ग्नि मे॒वान्ना॒द म॑न्ना॒द मे॒वाग्नि म॒ग्नि मे॒वान्ना॒दम् । \newline
49. ए॒वान्ना॒द म॑न्ना॒द मे॒वैवा न्ना॒दꣳ स्वेन॒ स्वेना᳚न्ना॒द मे॒वैवा न्ना॒दꣳ स्वेन॑ । \newline
50. अ॒न्ना॒दꣳ स्वेन॒ स्वेना᳚न्ना॒द म॑न्ना॒दꣳ स्वेन॑ भाग॒धेये॑न भाग॒धेये॑न॒ स्वेना᳚न्ना॒द म॑न्ना॒दꣳ स्वेन॑ भाग॒धेये॑न । \newline
51. अ॒न्ना॒दमित्य॑न्न - अ॒दम् । \newline
52. स्वेन॑ भाग॒धेये॑न भाग॒धेये॑न॒ स्वेन॒ स्वेन॑ भाग॒धेये॒नोपोप॑ भाग॒धेये॑न॒ स्वेन॒ स्वेन॑ भाग॒धेये॒नोप॑ । \newline
53. भा॒ग॒धेये॒नोपोप॑ भाग॒धेये॑न भाग॒धेये॒नोप॑ धावति धाव॒त्युप॑ भाग॒धेये॑न भाग॒धेये॒नोप॑ धावति । \newline
54. भा॒ग॒धेये॒नेति॑ भाग - धेये॑न । \newline
55. उप॑ धावति धाव॒ त्युपोप॑ धावति॒ स स धा॑व॒ त्युपोप॑ धावति॒ सः । \newline
56. धा॒व॒ति॒ स स धा॑वति धावति॒ स ए॒वैव स धा॑वति धावति॒ स ए॒व । \newline
57. स ए॒वैव स स ए॒वैन॑ मेन मे॒व स स ए॒वैन᳚म् । \newline
58. ए॒वैन॑ मेन मे॒वैवैन॑ मन्ना॒द म॑न्ना॒द मे॑न मे॒वैवैन॑ मन्ना॒दम् । \newline
59. ए॒न॒ म॒न्ना॒द म॑न्ना॒द मे॑न मेन मन्ना॒दम् क॑रोति करो त्यन्ना॒द मे॑न मेन मन्ना॒दम् क॑रोति । \newline
60. अ॒न्ना॒दम् क॑रोति करो त्यन्ना॒द म॑न्ना॒दम् क॑रो त्यन्ना॒दो᳚ ऽन्ना॒दः क॑रो त्यन्ना॒द म॑न्ना॒दम् क॑रो त्यन्ना॒दः । \newline
61. अ॒न्ना॒दमित्य॑न्न - अ॒दम् । \newline
62. क॒रो॒ त्य॒न्ना॒दो᳚ ऽन्ना॒दः क॑रोति करो त्यन्ना॒द ए॒वैवान्ना॒दः क॑रोति करो त्यन्ना॒द ए॒व । \newline
63. अ॒न्ना॒द ए॒वैवान्ना॒दो᳚ ऽन्ना॒द ए॒व भ॑वति भव त्ये॒वान्ना॒दो᳚ ऽन्ना॒द ए॒व भ॑वति । \newline
64. अ॒न्ना॒द इत्य॑न्न - अ॒दः । \newline
\pagebreak
\markright{ TS 2.2.4.2  \hfill https://www.vedavms.in \hfill}
\addcontentsline{toc}{section}{ TS 2.2.4.2 }
\section*{ TS 2.2.4.2 }

\textbf{TS 2.2.4.2 } \newline
\textbf{Samhita Paata} \newline

ए॒व भ॑वत्य॒ग्नयेऽन्न॑पतये पुरो॒डाश॑म॒ष्टाक॑पालं॒ निर्व॑पे॒द्यःका॒मये॒तान्न॑पतिः स्या॒मित्य॒ग्नि-मे॒वान्न॑पतिꣳ॒॒ स्वेन॑ भाग॒धेये॒नोप॑ धावति॒ स ए॒वैन॒मन्न॑पतिं करो॒त्यन्न॑पतिरे॒व भ॑वत्य॒ग्नये॒ पव॑मानाय पुरो॒डाश॑म॒ष्टाक॑पालं॒ निर्व॑पेद॒ग्नये॑ पाव॒काया॒ग्नये॒ शुच॑ये॒ ज्योगा॑मयावी॒ यद॒ग्नये॒ पव॑मानाय नि॒र्वप॑ति प्रा॒णमे॒वास्मि॒न् तेन॑ दधाति॒ यद॒ग्नये॑ - [  ] \newline

\textbf{Pada Paata} \newline

ए॒व । भ॒व॒ति॒ । अ॒ग्नये᳚ । अन्न॑पतय॒ इत्यन्न॑ - प॒त॒ये॒ । पु॒रो॒डाश᳚म् । अ॒ष्टाक॑पाल॒मित्य॒ष्टा - क॒पा॒ल॒म् । निरिति॑ । व॒पे॒त् । यः । का॒मये॑त । अन्न॑पति॒रित्यन्न॑ - प॒तिः॒ । स्या॒म् । इति॑ । अ॒ग्निम् । ए॒व । अन्न॑पति॒मित्यन्न॑ - प॒ति॒म् । स्वेन॑ । भा॒ग॒धेये॒नेति॑ भाग - धेये॑न । उपेति॑ । धा॒व॒ति॒ । सः । ए॒व । ए॒न॒म् । अन्न॑पति॒मित्यन्न॑ - प॒ति॒म् । क॒रो॒ति॒ । अन्न॑पति॒रित्यन्न॑ - प॒तिः॒ । ए॒व । भ॒व॒ति॒ । अ॒ग्नये᳚ । पव॑मानाय । पु॒रो॒डाश᳚म् । अ॒ष्टाक॑पाल॒मित्य॒ष्टा - क॒पा॒ल॒म् । निरिति॑ । व॒पे॒त् । अ॒ग्नये᳚ । पा॒व॒काय॑ । अ॒ग्नये᳚ । शुच॑ये । ज्योगा॑मया॒वीति॒ ज्योक् - आ॒म॒या॒वी॒ । यत् । अ॒ग्नये᳚ । पव॑मानाय । नि॒र्वप॒तीति॑ निः - वप॑ति । प्रा॒णमिति॑ प्र - अ॒नम् । ए॒व । अ॒स्मि॒न्न् । तेन॑ । द॒धा॒ति॒ । यत् । अ॒ग्नये᳚ ।  \newline


\textbf{Krama Paata} \newline

ए॒व भ॑वति । भ॒व॒त्य॒ग्नये᳚ । अ॒ग्नये ऽन्न॑पतये । अन्न॑पतये पुरो॒डाश᳚म् । अन्न॑पतय॒ इत्यन्न॑ - प॒त॒ये॒ । पु॒रो॒डाश॑म॒ष्टाक॑पालम् । अ॒ष्टाक॑पाल॒म् निः । अ॒ष्टाक॑पाल॒मित्य॒ष्टा - क॒पा॒ल॒म् । निर् व॑पेत् । व॒पे॒द् यः । यः का॒मये॑त । का॒मये॒तान्न॑पतिः । अन्न॑पतिः स्याम् । अन्न॑पति॒रित्यन्न॑ - प॒तिः॒ । स्या॒मिति॑ । इत्य॒ग्निम् । अ॒ग्निमे॒व । ए॒वान्न॑पतिम् । अन्न॑पतिꣳ॒॒ स्वेन॑ । अन्न॑पति॒मित्यन्न॑ - प॒ति॒म् । स्वेन॑ भाग॒धेये॑न । भा॒ग॒धेये॒नोप॑ । भा॒ग॒धेये॒नेति॑ भाग - धेये॑न । उप॑ धावति । धा॒व॒ति॒ सः । स ए॒व । ए॒वैन᳚म् । ए॒न॒मन्न॑पतिम् । अन्न॑पतिम् करोति । अन्न॑पति॒मित्यन्न॑ - प॒ति॒॒म् । क॒रो॒त्यन्न॑पतिः । अन्न॑पतिरे॒व । अन्न॑पति॒रित्यन्न॑ - प॒तिः॒ । ए॒व भ॑वति । भ॒व॒त्य॒ग्नये᳚ । अ॒ग्नये॒ पव॑मानाय । पव॑मानाय पुरो॒डाश᳚म् । पु॒रो॒डाश॑म॒ष्टाक॑पालम् । अ॒ष्टाक॑पाल॒म् निः । अ॒ष्टाक॑पाल॒मित्य॒ष्टा - क॒पा॒ल॒म् । निर् व॑पेत् । व॒पे॒द॒ग्नये᳚ । अ॒ग्नये॑ पाव॒काय॑ । पा॒व॒काया॒ग्नये᳚ । अ॒ग्नये॒ शुच॑ये । शुच॑ये॒ ज्योगा॑मयावी । ज्योगा॑मयावी॒ यत् । ज्योगा॑मया॒वीति॒ ज्योक् - आ॒म॒या॒वी॒ । यद॒ग्नये᳚ । अ॒ग्नये॒ पव॑मानय । पव॑मानाय नि॒र्वप॑ति । नि॒र्वप॑ति प्रा॒णम् । नि॒र्वप॒तीति॑ निः - वप॑ति । प्रा॒णमे॒व । प्रा॒णमिति॑ प्र - अ॒नम् । ए॒वास्मिन्न्॑ । अ॒स्मि॒न् तेन॑ । तेन॑ दधाति । द॒धा॒ति॒ यत् । यद॒ग्नये᳚ । अ॒ग्नये॑ पाव॒काय॑ \newline

\textbf{Jatai Paata} \newline

1. ए॒व भ॑वति भव त्ये॒वैव भ॑वति । \newline
2. भ॒व॒ त्य॒ग्नये॒ ऽग्नये॑ भवति भव त्य॒ग्नये᳚ । \newline
3. अ॒ग्नये ऽन्न॑पत॒ये ऽन्न॑पतये॒ ऽग्नये॒ ऽग्नये ऽन्न॑पतये । \newline
4. अन्न॑पतये पुरो॒डाश॑म् पुरो॒डाश॒ मन्न॑पत॒ये ऽन्न॑पतये पुरो॒डाश᳚म् । \newline
5. अन्न॑पतय॒ इत्यन्न॑ - प॒त॒ये॒ । \newline
6. पु॒रो॒डाश॑ म॒ष्टाक॑पाल म॒ष्टाक॑पालम् पुरो॒डाश॑म् पुरो॒डाश॑ म॒ष्टाक॑पालम् । \newline
7. अ॒ष्टाक॑पाल॒म् निर् णिर॒ष्टाक॑पाल म॒ष्टाक॑पाल॒म् निः । \newline
8. अ॒ष्टाक॑पाल॒मित्य॒ष्टा - क॒पा॒ल॒म् । \newline
9. निर् व॑पेद् वपे॒न् निर् णिर् व॑पेत् । \newline
10. व॒पे॒द् यो यो व॑पेद् वपे॒द् यः । \newline
11. यः का॒मये॑त का॒मये॑त॒ यो यः का॒मये॑त । \newline
12. का॒मये॒ता न्न॑पति॒ रन्न॑पतिः का॒मये॑त का॒मये॒ता न्न॑पतिः । \newline
13. अन्न॑पतिः स्याꣳ स्या॒ मन्न॑पति॒ रन्न॑पतिः स्याम् । \newline
14. अन्न॑पति॒रित्यन्न॑ - प॒तिः॒ । \newline
15. स्या॒ मितीति॑ स्याꣳ स्या॒ मिति॑ । \newline
16. इत्य॒ग्नि म॒ग्नि मिती त्य॒ग्निम् । \newline
17. अ॒ग्नि मे॒वैवाग्नि म॒ग्नि मे॒व । \newline
18. ए॒वा न्न॑पति॒ मन्न॑पति मे॒वैवा न्न॑पतिम् । \newline
19. अन्न॑पतिꣳ॒॒ स्वेन॒ स्वेना न्न॑पति॒ मन्न॑पतिꣳ॒॒ स्वेन॑ । \newline
20. अन्न॑पति॒मित्यन्न॑ - प॒ति॒म् । \newline
21. स्वेन॑ भाग॒धेये॑न भाग॒धेये॑न॒ स्वेन॒ स्वेन॑ भाग॒धेये॑न । \newline
22. भा॒ग॒धेये॒नोपोप॑ भाग॒धेये॑न भाग॒धेये॒नोप॑ । \newline
23. भा॒ग॒धेये॒नेति॑ भाग - धेये॑न । \newline
24. उप॑ धावति धाव॒ त्युपोप॑ धावति । \newline
25. धा॒व॒ति॒ स स धा॑वति धावति॒ सः । \newline
26. स ए॒वैव स स ए॒व । \newline
27. ए॒वैन॑ मेन मे॒वैवैन᳚म् । \newline
28. ए॒न॒ मन्न॑पति॒ मन्न॑पति मेन मेन॒ मन्न॑पतिम् । \newline
29. अन्न॑पतिम् करोति करो॒ त्यन्न॑पति॒ मन्न॑पतिम् करोति । \newline
30. अन्न॑पति॒मित्यन्न॑ - प॒ति॒म् । \newline
31. क॒रो॒ त्यन्न॑पति॒ रन्न॑पतिः करोति करो॒ त्यन्न॑पतिः । \newline
32. अन्न॑पति रे॒वैवा न्न॑पति॒ रन्न॑पति रे॒व । \newline
33. अन्न॑पति॒रित्यन्न॑ - प॒तिः॒ । \newline
34. ए॒व भ॑वति भव त्ये॒वैव भ॑वति । \newline
35. भ॒व॒ त्य॒ग्नये॒ ऽग्नये॑ भवति भव त्य॒ग्नये᳚ । \newline
36. अ॒ग्नये॒ पव॑मानाय॒ पव॑मानाया॒ग्नये॒ ऽग्नये॒ पव॑मानाय । \newline
37. पव॑मानाय पुरो॒डाश॑म् पुरो॒डाश॒म् पव॑मानाय॒ पव॑मानाय पुरो॒डाश᳚म् । \newline
38. पु॒रो॒डाश॑ म॒ष्टाक॑पाल म॒ष्टाक॑पालम् पुरो॒डाश॑म् पुरो॒डाश॑ म॒ष्टाक॑पालम् । \newline
39. अ॒ष्टाक॑पाल॒म् निर् णिर॒ष्टाक॑पाल म॒ष्टाक॑पाल॒म् निः । \newline
40. अ॒ष्टाक॑पाल॒मित्य॒ष्टा - क॒पा॒ल॒म् । \newline
41. निर् व॑पेद् वपे॒न् निर् णिर् व॑पेत् । \newline
42. व॒पे॒ द॒ग्नये॒ ऽग्नये॑ वपेद् वपे द॒ग्नये᳚ । \newline
43. अ॒ग्नये॑ पाव॒काय॑ पाव॒काया॒ग्नये॒ ऽग्नये॑ पाव॒काय॑ । \newline
44. पा॒व॒काया॒ग्नये॒ ऽग्नये॑ पाव॒काय॑ पाव॒काया॒ग्नये᳚ । \newline
45. अ॒ग्नये॒ शुच॑ये॒ शुच॑ये॒ ऽग्नये॒ ऽग्नये॒ शुच॑ये । \newline
46. शुच॑ये॒ ज्योगा॑मयावी॒ ज्योगा॑मयावी॒ शुच॑ये॒ शुच॑ये॒ ज्योगा॑मयावी । \newline
47. ज्योगा॑मयावी॒ यद् यज् ज्योगा॑मयावी॒ ज्योगा॑मयावी॒ यत् । \newline
48. ज्योगा॑मया॒वीति॒ ज्योक् - आ॒म॒या॒वी॒ । \newline
49. यद॒ग्नये॒ ऽग्नये॒ यद् यद॒ग्नये᳚ । \newline
50. अ॒ग्नये॒ पव॑मानाय॒ पव॑मानाया॒ग्नये॒ ऽग्नये॒ पव॑मानाय । \newline
51. पव॑मानाय नि॒र्वप॑ति नि॒र्वप॑ति॒ पव॑मानाय॒ पव॑मानाय नि॒र्वप॑ति । \newline
52. नि॒र्वप॑ति प्रा॒णम् प्रा॒णम् नि॒र्वप॑ति नि॒र्वप॑ति प्रा॒णम् । \newline
53. नि॒र्वप॒तीति॑ निः - वप॑ति । \newline
54. प्रा॒ण मे॒वैव प्रा॒णम् प्रा॒ण मे॒व । \newline
55. प्रा॒णमिति॑ प्र - अ॒नम् । \newline
56. ए॒वास्मि॑न् नस्मिन् ने॒वैवास्मिन्न्॑ । \newline
57. अ॒स्मि॒न् तेन॒ तेना᳚स्मिन् नस्मि॒न् तेन॑ । \newline
58. तेन॑ दधाति दधाति॒ तेन॒ तेन॑ दधाति । \newline
59. द॒धा॒ति॒ यद् यद् द॑धाति दधाति॒ यत् । \newline
60. यद॒ग्नये॒ ऽग्नये॒ यद् यद॒ग्नये᳚ । \newline
61. अ॒ग्नये॑ पाव॒काय॑ पाव॒काया॒ग्नये॒ ऽग्नये॑ पाव॒काय॑ । \newline

\textbf{Ghana Paata } \newline

1. ए॒व भ॑वति भव त्ये॒वैव भ॑व त्य॒ग्नये॒ ऽग्नये॑ भव त्ये॒वैव भ॑व त्य॒ग्नये᳚ । \newline
2. भ॒व॒ त्य॒ग्नये॒ ऽग्नये॑ भवति भव त्य॒ग्नये ऽन्न॑पत॒ये ऽन्न॑पतये॒ ऽग्नये॑ भवति भव त्य॒ग्नये ऽन्न॑पतये । \newline
3. अ॒ग्नये ऽन्न॑पत॒ये ऽन्न॑पतये॒ ऽग्नये॒ ऽग्नये ऽन्न॑पतये पुरो॒डाश॑म् पुरो॒डाश॒ मन्न॑पतये॒ ऽग्नये॒ ऽग्नये ऽन्न॑पतये पुरो॒डाश᳚म् । \newline
4. अन्न॑पतये पुरो॒डाश॑म् पुरो॒डाश॒ मन्न॑पत॒ये ऽन्न॑पतये पुरो॒डाश॑ म॒ष्टाक॑पाल म॒ष्टाक॑पालम् पुरो॒डाश॒ मन्न॑पत॒ये ऽन्न॑पतये पुरो॒डाश॑ म॒ष्टाक॑पालम् । \newline
5. अन्न॑पतय॒ इत्यन्न॑ - प॒त॒ये॒ । \newline
6. पु॒रो॒डाश॑ म॒ष्टाक॑पाल म॒ष्टाक॑पालम् पुरो॒डाश॑म् पुरो॒डाश॑ म॒ष्टाक॑पाल॒म् निर् णिर॒ष्टाक॑पालम् पुरो॒डाश॑म् पुरो॒डाश॑ म॒ष्टाक॑पाल॒म् निः । \newline
7. अ॒ष्टाक॑पाल॒म् निर् णिर॒ष्टाक॑पाल म॒ष्टाक॑पाल॒म् निर् व॑पेद् वपे॒न् निर॒ष्टाक॑पाल म॒ष्टाक॑पाल॒म् निर् व॑पेत् । \newline
8. अ॒ष्टाक॑पाल॒मित्य॒ष्टा - क॒पा॒ल॒म् । \newline
9. निर् व॑पेद् वपे॒न् निर् णिर् व॑पे॒द् यो यो व॑पे॒न् निर् णिर् व॑पे॒द् यः । \newline
10. व॒पे॒द् यो यो व॑पेद् वपे॒द् यः का॒मये॑त का॒मये॑त॒ यो व॑पेद् वपे॒द् यः का॒मये॑त । \newline
11. यः का॒मये॑त का॒मये॑त॒ यो यः का॒मये॒ता न्न॑पति॒ रन्न॑पतिः का॒मये॑त॒ यो यः का॒मये॒ता न्न॑पतिः । \newline
12. का॒मये॒ता न्न॑पति॒ रन्न॑पतिः का॒मये॑त का॒मये॒ता न्न॑पतिः स्याꣳ स्या॒ मन्न॑पतिः का॒मये॑त का॒मये॒ता न्न॑पतिः स्याम् । \newline
13. अन्न॑पतिः स्याꣳ स्या॒ मन्न॑पति॒ रन्न॑पतिः स्या॒ मितीति॑ स्या॒ मन्न॑पति॒ रन्न॑पतिः स्या॒ मिति॑ । \newline
14. अन्न॑पति॒रित्यन्न॑ - प॒तिः॒ । \newline
15. स्या॒ मितीति॑ स्याꣳ स्या॒ मित्य॒ग्नि म॒ग्नि मिति॑ स्याꣳ स्या॒ मित्य॒ग्निम् । \newline
16. इत्य॒ग्नि म॒ग्नि मिती त्य॒ग्नि मे॒वैवाग्नि मिती त्य॒ग्नि मे॒व । \newline
17. अ॒ग्नि मे॒वैवाग्नि म॒ग्नि मे॒वान्न॑पति॒ मन्न॑पति मे॒वाग्नि म॒ग्नि मे॒वान्न॑पतिम् । \newline
18. ए॒वान्न॑पति॒ मन्न॑पति मे॒वैवान्न॑पतिꣳ॒॒ स्वेन॒ स्वेना न्न॑पति मे॒वैवा न्न॑पतिꣳ॒॒ स्वेन॑ । \newline
19. अन्न॑पतिꣳ॒॒ स्वेन॒ स्वेनान्न॑पति॒ मन्न॑पतिꣳ॒॒ स्वेन॑ भाग॒धेये॑न भाग॒धेये॑न॒ स्वेनान्न॑पति॒ मन्न॑पतिꣳ॒॒ स्वेन॑ भाग॒धेये॑न । \newline
20. अन्न॑पति॒मित्यन्न॑ - प॒ति॒म् । \newline
21. स्वेन॑ भाग॒धेये॑न भाग॒धेये॑न॒ स्वेन॒ स्वेन॑ भाग॒धेये॒नोपोप॑ भाग॒धेये॑न॒ स्वेन॒ स्वेन॑ भाग॒धेये॒नोप॑ । \newline
22. भा॒ग॒धेये॒नोपोप॑ भाग॒धेये॑न भाग॒धेये॒नोप॑ धावति धाव॒ त्युप॑ भाग॒धेये॑न भाग॒धेये॒नोप॑ धावति । \newline
23. भा॒ग॒धेये॒नेति॑ भाग - धेये॑न । \newline
24. उप॑ धावति धाव॒ त्युपोप॑ धावति॒ स स धा॑व॒ त्युपोप॑ धावति॒ सः । \newline
25. धा॒व॒ति॒ स स धा॑वति धावति॒ स ए॒वैव स धा॑वति धावति॒ स ए॒व । \newline
26. स ए॒वैव स स ए॒वैन॑ मेन मे॒व स स ए॒वैन᳚म् । \newline
27. ए॒वैन॑ मेन मे॒वैवैन॒ मन्न॑पति॒ मन्न॑पति मेन मे॒वैवैन॒ मन्न॑पतिम् । \newline
28. ए॒न॒ मन्न॑पति॒ मन्न॑पति मेन मेन॒ मन्न॑पतिम् करोति करो॒ त्यन्न॑पति मेन मेन॒ मन्न॑पतिम् करोति । \newline
29. अन्न॑पतिम् करोति करो॒ त्यन्न॑पति॒ मन्न॑पतिम् करो॒ त्यन्न॑पति॒ रन्न॑पतिः करो॒ त्यन्न॑पति॒ मन्न॑पतिम् करो॒ त्यन्न॑पतिः । \newline
30. अन्न॑पति॒मित्यन्न॑ - प॒ति॒म् । \newline
31. क॒रो॒ त्यन्न॑पति॒ रन्न॑पतिः करोति करो॒ त्यन्न॑पति रे॒वैवा न्न॑पतिः करोति करो॒ त्यन्न॑पति रे॒व । \newline
32. अन्न॑पति रे॒वैवा न्न॑पति॒ रन्न॑पति रे॒व भ॑वति भव त्ये॒वान्न॑पति॒ रन्न॑पति रे॒व भ॑वति । \newline
33. अन्न॑पति॒रित्यन्न॑ - प॒तिः॒ । \newline
34. ए॒व भ॑वति भव त्ये॒वैव भ॑व त्य॒ग्नये॒ ऽग्नये॑ भव त्ये॒वैव भ॑व त्य॒ग्नये᳚ । \newline
35. भ॒व॒त्य॒ग्नये॒ ऽग्नये॑ भवति भवत्य॒ग्नये॒ पव॑मानाय॒ पव॑मानाया॒ग्नये॑ भवति भवत्य॒ग्नये॒ पव॑मानाय । \newline
36. अ॒ग्नये॒ पव॑मानाय॒ पव॑मानाया॒ग्नये॒ ऽग्नये॒ पव॑मानाय पुरो॒डाश॑म् पुरो॒डाश॒म् पव॑मानाया॒ग्नये॒ ऽग्नये॒ पव॑मानाय पुरो॒डाश᳚म् । \newline
37. पव॑मानाय पुरो॒डाश॑म् पुरो॒डाश॒म् पव॑मानाय॒ पव॑मानाय पुरो॒डाश॑ म॒ष्टाक॑पाल म॒ष्टाक॑पालम् पुरो॒डाश॒म् पव॑मानाय॒ पव॑मानाय पुरो॒डाश॑ म॒ष्टाक॑पालम् । \newline
38. पु॒रो॒डाश॑ म॒ष्टाक॑पाल म॒ष्टाक॑पालम् पुरो॒डाश॑म् पुरो॒डाश॑ म॒ष्टाक॑पाल॒म् निर् णिर॒ष्टाक॑पालम् पुरो॒डाश॑म् पुरो॒डाश॑ म॒ष्टाक॑पाल॒म् निः । \newline
39. अ॒ष्टाक॑पाल॒म् निर् णिर॒ष्टाक॑पाल म॒ष्टाक॑पाल॒म् निर् व॑पेद् वपे॒न् निर॒ष्टाक॑पाल म॒ष्टाक॑पाल॒म् निर् व॑पेत् । \newline
40. अ॒ष्टाक॑पाल॒मित्य॒ष्टा - क॒पा॒ल॒म् । \newline
41. निर् व॑पेद् वपे॒न् निर् णिर् व॑पे द॒ग्नये॒ ऽग्नये॑ वपे॒न् निर् णिर् व॑पे द॒ग्नये᳚ । \newline
42. व॒पे॒ द॒ग्नये॒ ऽग्नये॑ वपेद् वपे द॒ग्नये॑ पाव॒काय॑ पाव॒काया॒ग्नये॑ वपेद् वपे द॒ग्नये॑ पाव॒काय॑ । \newline
43. अ॒ग्नये॑ पाव॒काय॑ पाव॒काया॒ग्नये॒ ऽग्नये॑ पाव॒काया॒ग्नये॒ ऽग्नये॑ पाव॒काया॒ग्नये॒ ऽग्नये॑ पाव॒काया॒ग्नये᳚ । \newline
44. पा॒व॒काया॒ग्नये॒ ऽग्नये॑ पाव॒काय॑ पाव॒काया॒ग्नये॒ शुच॑ये॒ शुच॑ये॒ ऽग्नये॑ पाव॒काय॑ पाव॒काया॒ग्नये॒ शुच॑ये । \newline
45. अ॒ग्नये॒ शुच॑ये॒ शुच॑ये॒ ऽग्नये॒ ऽग्नये॒ शुच॑ये॒ ज्योगा॑मयावी॒ ज्योगा॑मयावी॒ शुच॑ये॒ ऽग्नये॒ ऽग्नये॒ शुच॑ये॒ ज्योगा॑मयावी । \newline
46. शुच॑ये॒ ज्योगा॑मयावी॒ ज्योगा॑मयावी॒ शुच॑ये॒ शुच॑ये॒ ज्योगा॑मयावी॒ यद् यज् ज्योगा॑मयावी॒ शुच॑ये॒ शुच॑ये॒ ज्योगा॑मयावी॒ यत् । \newline
47. ज्योगा॑मयावी॒ यद् यज् ज्योगा॑मयावी॒ ज्योगा॑मयावी॒ यद॒ग्नये॒ ऽग्नये॒ यज् ज्योगा॑मयावी॒ ज्योगा॑मयावी॒ यद॒ग्नये᳚ । \newline
48. ज्योगा॑मया॒वीति॒ ज्योक् - आ॒म॒या॒वी॒ । \newline
49. यद॒ग्नये॒ ऽग्नये॒ यद् यद॒ग्नये॒ पव॑मानाय॒ पव॑मानाया॒ग्नये॒ यद् यद॒ग्नये॒ पव॑मानाय । \newline
50. अ॒ग्नये॒ पव॑मानाय॒ पव॑मानाया॒ग्नये॒ ऽग्नये॒ पव॑मानाय नि॒र्वप॑ति नि॒र्वप॑ति॒ पव॑मानाया॒ग्नये॒ ऽग्नये॒ पव॑मानाय नि॒र्वप॑ति । \newline
51. पव॑मानाय नि॒र्वप॑ति नि॒र्वप॑ति॒ पव॑मानाय॒ पव॑मानाय नि॒र्वप॑ति प्रा॒णम् प्रा॒णम् नि॒र्वप॑ति॒ पव॑मानाय॒ पव॑मानाय नि॒र्वप॑ति प्रा॒णम् । \newline
52. नि॒र्वप॑ति प्रा॒णम् प्रा॒णम् नि॒र्वप॑ति नि॒र्वप॑ति प्रा॒ण मे॒वैव प्रा॒णम् नि॒र्वप॑ति नि॒र्वप॑ति प्रा॒ण मे॒व । \newline
53. नि॒र्वप॒तीति॑ निः - वप॑ति । \newline
54. प्रा॒ण मे॒वैव प्रा॒णम् प्रा॒ण मे॒वास्मि॑न् नस्मिन् ने॒व प्रा॒णम् प्रा॒ण मे॒वास्मिन्न्॑ । \newline
55. प्रा॒णमिति॑ प्र - अ॒नम् । \newline
56. ए॒वास्मि॑न् नस्मिन् ने॒वैवास्मि॒न् तेन॒ तेना᳚स्मिन् ने॒वैवास्मि॒न् तेन॑ । \newline
57. अ॒स्मि॒न् तेन॒ तेना᳚स्मिन् नस्मि॒न् तेन॑ दधाति दधाति॒ तेना᳚स्मिन् नस्मि॒न् तेन॑ दधाति । \newline
58. तेन॑ दधाति दधाति॒ तेन॒ तेन॑ दधाति॒ यद् यद् द॑धाति॒ तेन॒ तेन॑ दधाति॒ यत् । \newline
59. द॒धा॒ति॒ यद् यद् द॑धाति दधाति॒ यद॒ग्नये॒ ऽग्नये॒ यद् द॑धाति दधाति॒ यद॒ग्नये᳚ । \newline
60. यद॒ग्नये॒ ऽग्नये॒ यद् यद॒ग्नये॑ पाव॒काय॑ पाव॒काया॒ग्नये॒ यद् यद॒ग्नये॑ पाव॒काय॑ । \newline
61. अ॒ग्नये॑ पाव॒काय॑ पाव॒काया॒ग्नये॒ ऽग्नये॑ पाव॒काय॒ वाचं॒ ॅवाच॑म् पाव॒काया॒ग्नये॒ ऽग्नये॑ पाव॒काय॒ वाच᳚म् । \newline
\pagebreak
\markright{ TS 2.2.4.3  \hfill https://www.vedavms.in \hfill}
\addcontentsline{toc}{section}{ TS 2.2.4.3 }
\section*{ TS 2.2.4.3 }

\textbf{TS 2.2.4.3 } \newline
\textbf{Samhita Paata} \newline

पाव॒काय॒ वाच॑मे॒वास्मि॒न् तेन॑ दधाति॒ यद॒ग्नये॒ शुच॑य॒ आयु॑रे॒वास्मि॒न् तेन॑ दधात्यु॒त यदी॒तासु॒र्भव॑ति॒ जीव॑त्ये॒वैतामे॒व निर्व॑पे॒च्चक्षु॑ष्कामो॒ यद॒ग्नये॒ पव॑मानाय नि॒र्वप॑ति प्रा॒णमे॒वास्मि॒न् तेन॑ दधाति॒ यद॒ग्नये॑ पाव॒काय॒ वाच॑मे॒वास्मि॒न् तेन॑ दधाति॒ यद॒ग्नये॒ शुच॑ये॒ चक्षु॑रे॒वास्मि॒न् तेन॑ दधा - [  ] \newline

\textbf{Pada Paata} \newline

पा॒व॒काय॑ । वाच᳚म् । ए॒व । अ॒स्मि॒न्न् । तेन॑ । द॒धा॒ति॒ । यत् । अ॒ग्नये᳚ । शुच॑ये । आयुः॑ । ए॒व । अ॒स्मि॒न्न् । तेन॑ । द॒धा॒ति॒ । उ॒त । यदि॑ । इ॒तासु॒रिती॒त-अ॒सुः॒ । भव॑ति । जीव॑ति । ए॒व । ए॒ताम् । ए॒व । निरिति॑ । व॒पे॒त् । चक्षु॑ष्काम॒ इति॒ चक्षुः॑ - का॒मः॒ । यत् । अ॒ग्नये᳚ । पव॑मानाय । नि॒र्वप॒तीति॑ निः - वप॑ति । प्रा॒णमिति॑ प्र-अ॒नम् । ए॒व । अ॒स्मि॒न्न् । तेन॑ । द॒धा॒ति॒ । यत् । अ॒ग्नये᳚ । पा॒व॒काय॑ । वाच᳚म् । ए॒व । अ॒स्मि॒न्न् । तेन॑ । द॒धा॒ति॒ । यत् । अ॒ग्नये᳚ । शुच॑ये । चक्षुः॑ । ए॒व । अ॒स्मि॒न्न् । तेन॑ । द॒धा॒ति॒ ।  \newline


\textbf{Krama Paata} \newline

पा॒व॒काय॒ वाच᳚म् । वाच॑मे॒व । ए॒वास्मिन्न्॑ । अ॒स्मि॒न् तेन॑ । तेन॑ दधाति । द॒धा॒ति॒ यत् । यद॒ग्नये᳚ । अ॒ग्नये॒ शुच॑ये । शुच॑य॒ आयुः॑ । आयु॑रे॒व । ए॒वास्मिन्न्॑ । अ॒स्मि॒न् तेन॑ । तेन॑ दधाति । द॒धा॒त्यु॒त । उ॒त यदि॑ । यदी॒तासुः॑ । इ॒तासु॒र् भ॑वति । इ॒तासु॒रिती॒त - अ॒सुः॒ । भव॑ति॒ जीव॑ति । जीव॑त्ये॒व । ए॒वैताम् । ए॒तामे॒व । ए॒व निः । निर् व॑पेत् । व॒पे॒च्चक्षु॑ष्कामः । चक्षु॑ष्कामो॒ यत् । चक्षु॑ष्काम॒ इति॒ चक्षुः॑ - का॒मः॒ । यद॒ग्नये᳚ । अ॒ग्नये॒ पव॑मानाय । पव॑मानाय नि॒र्वप॑ति । नि॒र्वप॑ति प्रा॒णम् । नि॒र्वप॒तीति॑ निः - वप॑ति । प्रा॒णमे॒व । प्रा॒णमिति॑ प्र - अ॒नम् । ए॒वास्मिन्न्॑ । अ॒स्मि॒न् तेन॑ । तेन॑ दधाति । द॒धा॒ति॒ यत् । यद॒ग्नये᳚ । अ॒ग्नये॑ पाव॒काय॑ । पा॒व॒काय॒ वाच᳚म् । वाच॑मे॒व । ए॒वास्मिन्न्॑ । अ॒स्मि॒न् तेन॑ । तेन॑ दधाति । द॒धा॒ति॒ यत् । यद॒ग्नये᳚ । अ॒ग्नये॒ शुच॑ये । शुच॑ये॒ चक्षुः॑ । चक्षु॑रे॒व । ए॒वास्मिन्न्॑ । अ॒स्मि॒न् तेन॑ । तेन॑ दधाति । द॒धा॒त्यु॒त \newline

\textbf{Jatai Paata} \newline

1. पा॒व॒काय॒ वाचं॒ ॅवाच॑म् पाव॒काय॑ पाव॒काय॒ वाच᳚म् । \newline
2. वाच॑ मे॒वैव वाचं॒ ॅवाच॑ मे॒व । \newline
3. ए॒वास्मि॑न् नस्मिन् ने॒वैवास्मिन्न्॑ । \newline
4. अ॒स्मि॒न् तेन॒ तेना᳚स्मिन् नस्मि॒न् तेन॑ । \newline
5. तेन॑ दधाति दधाति॒ तेन॒ तेन॑ दधाति । \newline
6. द॒धा॒ति॒ यद् यद् द॑धाति दधाति॒ यत् । \newline
7. यद॒ग्नये॒ ऽग्नये॒ यद् यद॒ग्नये᳚ । \newline
8. अ॒ग्नये॒ शुच॑ये॒ शुच॑ये॒ ऽग्नये॒ ऽग्नये॒ शुच॑ये । \newline
9. शुच॑य॒ आयु॒ रायुः॒ शुच॑ये॒ शुच॑य॒ आयुः॑ । \newline
10. आयु॑ रे॒वैवायु॒ रायु॑ रे॒व । \newline
11. ए॒वास्मि॑न् नस्मिन् ने॒वैवास्मिन्न्॑ । \newline
12. अ॒स्मि॒न् तेन॒ तेना᳚स्मिन् नस्मि॒न् तेन॑ । \newline
13. तेन॑ दधाति दधाति॒ तेन॒ तेन॑ दधाति । \newline
14. द॒धा॒ त्यु॒तोत द॑धाति दधा त्यु॒त । \newline
15. उ॒त यदि॒ यद्यु॒तोत यदि॑ । \newline
16. यदी॒तासु॑ रि॒तासु॒र् यदि॒ यदी॒तासुः॑ । \newline
17. इ॒तासु॒र् भव॑ति॒ भव॑ती॒तासु॑ रि॒तासु॒र् भव॑ति । \newline
18. इ॒तासु॒रिती॒त - अ॒सुः॒ । \newline
19. भव॑ति॒ जीव॑ति॒ जीव॑ति॒ भव॑ति॒ भव॑ति॒ जीव॑ति । \newline
20. जीव॑ त्ये॒वैव जीव॑ति॒ जीव॑ त्ये॒व । \newline
21. ए॒वैता मे॒ता मे॒वैवैताम् । \newline
22. ए॒ता मे॒वैवैता मे॒ता मे॒व । \newline
23. ए॒व निर् णिरे॒वैव निः । \newline
24. निर् व॑पेद् वपे॒न् निर् णिर् व॑पेत् । \newline
25. व॒पे॒च् चक्षु॑ष्काम॒ श्चक्षु॑ष्कामो वपेद् वपे॒च् चक्षु॑ष्कामः । \newline
26. चक्षु॑ष्कामो॒ यद् यच् चक्षु॑ष्काम॒ श्चक्षु॑ष्कामो॒ यत् । \newline
27. चक्षु॑ष्काम॒ इति॒ चक्षुः॑ - का॒मः॒ । \newline
28. यद॒ग्नये॒ ऽग्नये॒ यद् यद॒ग्नये᳚ । \newline
29. अ॒ग्नये॒ पव॑मानाय॒ पव॑मानाया॒ग्नये॒ ऽग्नये॒ पव॑मानाय । \newline
30. पव॑मानाय नि॒र्वप॑ति नि॒र्वप॑ति॒ पव॑मानाय॒ पव॑मानाय नि॒र्वप॑ति । \newline
31. नि॒र्वप॑ति प्रा॒णम् प्रा॒णन्नि॒र्वप॑ति नि॒र्वप॑ति प्रा॒णम् । \newline
32. नि॒र्वप॒तीति॑ निः - वप॑ति । \newline
33. प्रा॒ण मे॒वैव प्रा॒णम् प्रा॒ण मे॒व । \newline
34. प्रा॒णमिति॑ प्र - अ॒नम् । \newline
35. ए॒वास्मि॑न् नस्मिन् ने॒वैवास्मिन्न्॑ । \newline
36. अ॒स्मि॒न् तेन॒ तेना᳚स्मिन् नस्मि॒न् तेन॑ । \newline
37. तेन॑ दधाति दधाति॒ तेन॒ तेन॑ दधाति । \newline
38. द॒धा॒ति॒ यद् यद् द॑धाति दधाति॒ यत् । \newline
39. यद॒ग्नये॒ ऽग्नये॒ यद् यद॒ग्नये᳚ । \newline
40. अ॒ग्नये॑ पाव॒काय॑ पाव॒काया॒ग्नये॒ ऽग्नये॑ पाव॒काय॑ । \newline
41. पा॒व॒काय॒ वाचं॒ ॅवाच॑म् पाव॒काय॑ पाव॒काय॒ वाच᳚म् । \newline
42. वाच॑ मे॒वैव वाचं॒ ॅवाच॑ मे॒व । \newline
43. ए॒वास्मि॑न् नस्मिन् ने॒वैवास्मिन्न्॑ । \newline
44. अ॒स्मि॒न् तेन॒ तेना᳚स्मिन् नस्मि॒न् तेन॑ । \newline
45. तेन॑ दधाति दधाति॒ तेन॒ तेन॑ दधाति । \newline
46. द॒धा॒ति॒ यद् यद् द॑धाति दधाति॒ यत् । \newline
47. यद॒ग्नये॒ ऽग्नये॒ यद् यद॒ग्नये᳚ । \newline
48. अ॒ग्नये॒ शुच॑ये॒ शुच॑ये॒ ऽग्नये॒ ऽग्नये॒ शुच॑ये । \newline
49. शुच॑ये॒ चक्षु॒ श्चक्षुः॒ शुच॑ये॒ शुच॑ये॒ चक्षुः॑ । \newline
50. चक्षु॑ रे॒वैव चक्षु॒ श्चक्षु॑ रे॒व । \newline
51. ए॒वास्मि॑न् नस्मिन् ने॒वैवास्मिन्न्॑ । \newline
52. अ॒स्मि॒न् तेन॒ तेना᳚स्मिन् नस्मि॒न् तेन॑ । \newline
53. तेन॑ दधाति दधाति॒ तेन॒ तेन॑ दधाति । \newline
54. द॒धा॒ त्यु॒तोत द॑धाति दधा त्यु॒त । \newline

\textbf{Ghana Paata } \newline

1. पा॒व॒काय॒ वाचं॒ ॅवाच॑म् पाव॒काय॑ पाव॒काय॒ वाच॑ मे॒वैव वाच॑म् पाव॒काय॑ पाव॒काय॒ वाच॑ मे॒व । \newline
2. वाच॑ मे॒वैव वाचं॒ ॅवाच॑ मे॒वास्मि॑न् नस्मिन् ने॒व वाचं॒ ॅवाच॑ मे॒वास्मिन्न्॑ । \newline
3. ए॒वास्मि॑न् नस्मिन् ने॒वैवास्मि॒न् तेन॒ तेना᳚स्मिन् ने॒वैवास्मि॒न् तेन॑ । \newline
4. अ॒स्मि॒न् तेन॒ तेना᳚स्मिन् नस्मि॒न् तेन॑ दधाति दधाति॒ तेना᳚स्मिन् नस्मि॒न् तेन॑ दधाति । \newline
5. तेन॑ दधाति दधाति॒ तेन॒ तेन॑ दधाति॒ यद् यद् द॑धाति॒ तेन॒ तेन॑ दधाति॒ यत् । \newline
6. द॒धा॒ति॒ यद् यद् द॑धाति दधाति॒ यद॒ग्नये॒ ऽग्नये॒ यद् द॑धाति दधाति॒ यद॒ग्नये᳚ । \newline
7. यद॒ग्नये॒ ऽग्नये॒ यद् यद॒ग्नये॒ शुच॑ये॒ शुच॑ये॒ ऽग्नये॒ यद् यद॒ग्नये॒ शुच॑ये । \newline
8. अ॒ग्नये॒ शुच॑ये॒ शुच॑ये॒ ऽग्नये॒ ऽग्नये॒ शुच॑य॒ आयु॒ रायुः॒ शुच॑ये॒ ऽग्नये॒ ऽग्नये॒ शुच॑य॒ आयुः॑ । \newline
9. शुच॑य॒ आयु॒ रायुः॒ शुच॑ये॒ शुच॑य॒ आयु॑ रे॒वैवायुः॒ शुच॑ये॒ शुच॑य॒ आयु॑रे॒व । \newline
10. आयु॑ रे॒वैवायु॒ रायु॑ रे॒वास्मि॑न् नस्मिन् ने॒वायु॒ रायु॑ रे॒वास्मिन्न्॑ । \newline
11. ए॒वास्मि॑न् नस्मिन् ने॒वैवास्मि॒न् तेन॒ तेना᳚स्मिन् ने॒वैवास्मि॒न् तेन॑ । \newline
12. अ॒स्मि॒न् तेन॒ तेना᳚स्मिन् नस्मि॒न् तेन॑ दधाति दधाति॒ तेना᳚स्मिन् नस्मि॒न् तेन॑ दधाति । \newline
13. तेन॑ दधाति दधाति॒ तेन॒ तेन॑ दधा त्यु॒तोत द॑धाति॒ तेन॒ तेन॑ दधा त्यु॒त । \newline
14. द॒धा॒ त्यु॒तोत द॑धाति दधात्यु॒त यदि॒ यद्यु॒त द॑धाति दधा त्यु॒त यदि॑ । \newline
15. उ॒त यदि॒ यद्यु॒तोत यदी॒तासु॑ रि॒तासु॒र् यद्यु॒तोत यदी॒तासुः॑ । \newline
16. यदी॒तासु॑ रि॒तासु॒र् यदि॒ यदी॒तासु॒र् भव॑ति॒ भव॑ती॒तासु॒र् यदि॒ यदी॒तासु॒र् भव॑ति । \newline
17. इ॒तासु॒र् भव॑ति॒ भव॑ती॒तासु॑ रि॒तासु॒र् भव॑ति॒ जीव॑ति॒ जीव॑ति॒ भव॑ती॒तासु॑ रि॒तासु॒र् भव॑ति॒ जीव॑ति । \newline
18. इ॒तासु॒रिती॒त - अ॒सुः॒ । \newline
19. भव॑ति॒ जीव॑ति॒ जीव॑ति॒ भव॑ति॒ भव॑ति॒ जीव॑त्ये॒वैव जीव॑ति॒ भव॑ति॒ भव॑ति॒ जीव॑त्ये॒व । \newline
20. जीव॑ त्ये॒वैव जीव॑ति॒ जीव॑ त्ये॒वैता मे॒ता मे॒व जीव॑ति॒ जीव॑ त्ये॒वैताम् । \newline
21. ए॒वैता मे॒ता मे॒वैवैता मे॒वैवैता मे॒वैवैता मे॒व । \newline
22. ए॒ता मे॒वैवैता मे॒ता मे॒व निर् णिरे॒वैता मे॒ता मे॒व निः । \newline
23. ए॒व निर् णिरे॒वैव निर् व॑पेद् वपे॒न् निरे॒वैव निर् व॑पेत् । \newline
24. निर् व॑पेद् वपे॒न् निर् णिर् व॑पे॒च् चक्षु॑ष्काम॒ श्चक्षु॑ष्कामो वपे॒न् निर् णिर् व॑पे॒च् चक्षु॑ष्कामः । \newline
25. व॒पे॒च् चक्षु॑ष्काम॒ श्चक्षु॑ष्कामो वपेद् वपे॒च् चक्षु॑ष्कामो॒ यद् यच् चक्षु॑ष्कामो वपेद् वपे॒च् चक्षु॑ष्कामो॒ यत् । \newline
26. चक्षु॑ष्कामो॒ यद् यच् चक्षु॑ष्काम॒ श्चक्षु॑ष्कामो॒ यद॒ग्नये॒ ऽग्नये॒ यच् चक्षु॑ष्काम॒ श्चक्षु॑ष्कामो॒ यद॒ग्नये᳚ । \newline
27. चक्षु॑ष्काम॒ इति॒ चक्षुः॑ - का॒मः॒ । \newline
28. यद॒ग्नये॒ ऽग्नये॒ यद् यद॒ग्नये॒ पव॑मानाय॒ पव॑मानाया॒ग्नये॒ यद् यद॒ग्नये॒ पव॑मानाय । \newline
29. अ॒ग्नये॒ पव॑मानाय॒ पव॑मानाया॒ग्नये॒ ऽग्नये॒ पव॑मानाय नि॒र्वप॑ति नि॒र्वप॑ति॒ पव॑मानाया॒ग्नये॒ ऽग्नये॒ पव॑मानाय नि॒र्वप॑ति । \newline
30. पव॑मानाय नि॒र्वप॑ति नि॒र्वप॑ति॒ पव॑मानाय॒ पव॑मानाय नि॒र्वप॑ति प्रा॒णम् प्रा॒णम् नि॒र्वप॑ति॒ पव॑मानाय॒ पव॑मानाय नि॒र्वप॑ति प्रा॒णम् । \newline
31. नि॒र्वप॑ति प्रा॒णम् प्रा॒णम् नि॒र्वप॑ति नि॒र्वप॑ति प्रा॒ण मे॒वैव प्रा॒णम् नि॒र्वप॑ति नि॒र्वप॑ति प्रा॒ण मे॒व । \newline
32. नि॒र्वप॒तीति॑ निः - वप॑ति । \newline
33. प्रा॒ण मे॒वैव प्रा॒णम् प्रा॒ण मे॒वास्मि॑न् नस्मिन् ने॒व प्रा॒णम् प्रा॒ण मे॒वास्मिन्न्॑ । \newline
34. प्रा॒णमिति॑ प्र - अ॒नम् । \newline
35. ए॒वास्मि॑न् नस्मिन् ने॒वैवास्मि॒न् तेन॒ तेना᳚स्मिन् ने॒वैवास्मि॒न् तेन॑ । \newline
36. अ॒स्मि॒न् तेन॒ तेना᳚स्मिन् नस्मि॒न् तेन॑ दधाति दधाति॒ तेना᳚स्मिन् नस्मि॒न् तेन॑ दधाति । \newline
37. तेन॑ दधाति दधाति॒ तेन॒ तेन॑ दधाति॒ यद् यद् द॑धाति॒ तेन॒ तेन॑ दधाति॒ यत् । \newline
38. द॒धा॒ति॒ यद् यद् द॑धाति दधाति॒ यद॒ग्नये॒ ऽग्नये॒ यद् द॑धाति दधाति॒ यद॒ग्नये᳚ । \newline
39. यद॒ग्नये॒ ऽग्नये॒ यद् यद॒ग्नये॑ पाव॒काय॑ पाव॒काया॒ग्नये॒ यद् यद॒ग्नये॑ पाव॒काय॑ । \newline
40. अ॒ग्नये॑ पाव॒काय॑ पाव॒काया॒ग्नये॒ ऽग्नये॑ पाव॒काय॒ वाचं॒ ॅवाच॑म् पाव॒काया॒ग्नये॒ ऽग्नये॑ पाव॒काय॒ वाच᳚म् । \newline
41. पा॒व॒काय॒ वाचं॒ ॅवाच॑म् पाव॒काय॑ पाव॒काय॒ वाच॑ मे॒वैव वाच॑म् पाव॒काय॑ पाव॒काय॒ वाच॑ मे॒व । \newline
42. वाच॑ मे॒वैव वाचं॒ ॅवाच॑ मे॒वास्मि॑न् नस्मिन् ने॒व वाचं॒ ॅवाच॑ मे॒वास्मिन्न्॑ । \newline
43. ए॒वास्मि॑न् नस्मिन् ने॒वैवास्मि॒न् तेन॒ तेना᳚स्मिन् ने॒वैवास्मि॒न् तेन॑ । \newline
44. अ॒स्मि॒न् तेन॒ तेना᳚स्मिन् नस्मि॒न् तेन॑ दधाति दधाति॒ तेना᳚स्मिन् नस्मि॒न् तेन॑ दधाति । \newline
45. तेन॑ दधाति दधाति॒ तेन॒ तेन॑ दधाति॒ यद् यद् द॑धाति॒ तेन॒ तेन॑ दधाति॒ यत् । \newline
46. द॒धा॒ति॒ यद् यद् द॑धाति दधाति॒ यद॒ग्नये॒ ऽग्नये॒ यद् द॑धाति दधाति॒ यद॒ग्नये᳚ । \newline
47. यद॒ग्नये॒ ऽग्नये॒ यद् यद॒ग्नये॒ शुच॑ये॒ शुच॑ये॒ ऽग्नये॒ यद् यद॒ग्नये॒ शुच॑ये । \newline
48. अ॒ग्नये॒ शुच॑ये॒ शुच॑ये॒ ऽग्नये॒ ऽग्नये॒ शुच॑ये॒ चक्षु॒ श्चक्षुः॒ शुच॑ये॒ ऽग्नये॒ ऽग्नये॒ शुच॑ये॒ चक्षुः॑ । \newline
49. शुच॑ये॒ चक्षु॒ श्चक्षुः॒ शुच॑ये॒ शुच॑ये॒ चक्षु॑ रे॒वैव चक्षुः॒ शुच॑ये॒ शुच॑ये॒ चक्षु॑ रे॒व । \newline
50. चक्षु॑ रे॒वैव चक्षु॒ श्चक्षु॑ रे॒वास्मि॑न् नस्मिन् ने॒व चक्षु॒ श्चक्षु॑ रे॒वास्मिन्न्॑ । \newline
51. ए॒वास्मि॑न् नस्मिन् ने॒वैवास्मि॒न् तेन॒ तेना᳚स्मिन् ने॒वैवास्मि॒न् तेन॑ । \newline
52. अ॒स्मि॒न् तेन॒ तेना᳚स्मिन् नस्मि॒न् तेन॑ दधाति दधाति॒ तेना᳚स्मिन् नस्मि॒न् तेन॑ दधाति । \newline
53. तेन॑ दधाति दधाति॒ तेन॒ तेन॑ दधा त्यु॒तोत द॑धाति॒ तेन॒ तेन॑ दधा त्यु॒त । \newline
54. द॒धा॒ त्यु॒तोत द॑धाति दधा त्यु॒त यदि॒ यद्यु॒त द॑धाति दधा त्यु॒त यदि॑ । \newline
\pagebreak
\markright{ TS 2.2.4.4  \hfill https://www.vedavms.in \hfill}
\addcontentsline{toc}{section}{ TS 2.2.4.4 }
\section*{ TS 2.2.4.4 }

\textbf{TS 2.2.4.4 } \newline
\textbf{Samhita Paata} \newline

-त्यु॒त यद्य॒न्धो भव॑ति॒ प्रैव प॑श्यत्य॒ग्नये॑ पु॒त्रव॑ते पुरो॒डाश॑म॒ष्टाक॑पालं॒ निर्व॑पे॒दिन्द्रा॑य पु॒त्रिणे॑ पुरो॒डाश॒मेका॑दशकपालं प्र॒जाका॑मो॒ऽग्निरे॒वास्मै᳚ प्र॒जांप्र॑ज॒नय॑ति वृ॒द्धामिन्द्रः॒ प्र य॑च्छत्य॒ग्नये॒ रस॑वतेऽजक्षी॒रे च॒रुं निर्व॑पे॒द्यः का॒मये॑त॒ रस॑वान्थ्-स्या॒मित्य॒ग्निमे॒व रस॑वन्तꣳ॒॒ स्वेन॑ भाग॒धेये॒नोप॑ धावति॒ स ए॒वैनꣳ॒॒ रस॑वन्तं करोति॒ - [  ] \newline

\textbf{Pada Paata} \newline

उ॒त । यदि॑ । अ॒न्धः । भव॑ति । प्रेति॑ । ए॒व । प॒श्य॒ति॒ । अ॒ग्नये᳚ । पु॒त्रव॑त॒ इति॑ पु॒त्र - व॒ते॒ । पु॒रो॒डाश᳚म् । अ॒ष्टाक॑पाल॒मित्य॒ष्टा - क॒पा॒ल॒म् । निरिति॑ । व॒पे॒त् । इन्द्रा॑य । पु॒त्रिणे᳚ । पु॒रो॒डाश᳚म् । एका॑दशकपाल॒मित्येका॑दश - क॒पा॒ल॒म् । प्र॒जाका॑म॒ इति॑ प्र॒जा - का॒मः॒ । अ॒ग्निः । ए॒व । अ॒स्मै॒ । प्र॒जामिति॑ प्र-जाम् । प्र॒ज॒नय॒तीति॑ प्र - ज॒नय॑ति । वृ॒द्धाम् । इन्द्रः॑ । प्रेति॑ । य॒च्छ॒ति॒ । अ॒ग्नये᳚ । रस॑वत॒ इति॒ रस॑-व॒ते॒ । अ॒ज॒क्षी॒र इत्य॑ज - क्षी॒रे । च॒रुम् । निरिति॑ । व॒पे॒त् । यः । का॒मये॑त । रस॑वा॒निति॒ रस॑ - वा॒न् । स्या॒म् । इति॑ । अ॒ग्निम् । ए॒व । रस॑वन्त॒मिति॒ रस॑ - व॒न्त॒म् । स्वेन॑ । भा॒ग॒धेये॒नेति॑ भाग - धेये॑न । उपेति॑ । धा॒व॒ति॒ । सः । ए॒व । ए॒न॒म् । रस॑वन्त॒मिति॒ रस॑ - व॒न्त॒म् । क॒रो॒ति॒ ।  \newline


\textbf{Krama Paata} \newline

उ॒त यदि॑ । यद्य॒न्धः । अ॒न्धो भव॑ति । भव॑ति॒ प्र । प्रैव । ए॒व प॑श्यति । प॒श्य॒त्य॒ग्नये᳚ । अ॒ग्नये॑ पु॒त्रव॑ते । पु॒त्रव॑ते पुरो॒डाश᳚म् । पु॒त्रव॑त॒ इति॑ पु॒त्र - व॒ते॒ । पु॒रो॒डाश॑म॒ष्टाक॑पालम् । अ॒ष्टाक॑पाल॒म् निः । अ॒ष्टाक॑पाल॒मित्य॒ष्टा - क॒पा॒ल॒म् । निर् व॑पेत् । व॒पे॒दिन्द्रा॑य । इन्द्रा॑य पु॒त्रिणे᳚ । पु॒त्रिणे॑ पुरो॒डाश᳚म् । पु॒रो॒डाश॒मेका॑दशकपालम् । एका॑दशकपालम् प्र॒जाका॑मः । एका॑दशकपाल॒मित्येका॑दश - क॒पा॒ल॒म् । प्र॒जाका॑मो॒ ऽग्निः । प्र॒जाका॑म॒ इति॑ प्र॒जा - का॒मः॒ । अ॒ग्निरे॒व । ए॒वास्मै᳚ । अ॒स्मै॒ प्र॒जाम् । प्र॒जाम् प्र॑ज॒नय॑ति । प्र॒जामिति॑ प्र - जाम् । प्र॒ज॒नय॑ति वृ॒द्धाम् । प्र॒ज॒नय॒तीति॑ प्र - ज॒नय॑ति । वृ॒द्धामिन्द्रः॑ । इन्द्रः॒ प्र । प्र य॑च्छति । य॒च्छ॒त्य॒ग्नये᳚ । अ॒ग्नये॒ रस॑वते । रस॑वते ऽजक्षी॒रे । रस॑वत॒ इति॒ रस॑ - व॒ते॒ । अ॒ज॒क्षी॒रे च॒रुम् । अ॒ज॒क्षी॒र इत्य॑ज - क्षी॒रे । च॒रुम् निः । निर् व॑पेत् । व॒पे॒द् यः । यः का॒मये॑त । का॒मये॑त॒ रस॑वान् । रस॑वान्थ् स्याम् । रस॑वा॒निति॒ रस॑ - वा॒न्॒ । स्या॒मिति॑ । इत्य॒ग्निम् । अ॒ग्निमे॒व । ए॒व रस॑वन्तम् । रस॑वन्तꣳ॒॒ स्वेन॑ । रस॑वन्त॒मिति॒ रस॑ - व॒न्त॒॒म् । स्वेन॑ भाग॒धेये॑न । भा॒ग॒धेये॒नोप॑ । भा॒ग॒धेये॒नेति॑ भाग - धेये॑न । उप॑ धावति । धा॒व॒ति॒ सः । स ए॒व । ए॒वैन᳚म् । ए॒नꣳ॒॒ रस॑वन्तम् । रस॑वन्तम् करोति । रस॑वन्त॒मिति॒ रस॑ - व॒न्त॒म् । क॒रो॒ति॒ रस॑वान् \newline

\textbf{Jatai Paata} \newline

1. उ॒त यदि॒ यद्यु॒तोत यदि॑ । \newline
2. यद्य॒न्धो᳚ ऽन्धो यदि॒ यद्य॒न्धः । \newline
3. अ॒न्धो भव॑ति॒ भव॑ त्य॒न्धो᳚ ऽन्धो भव॑ति । \newline
4. भव॑ति॒ प्र प्र भव॑ति॒ भव॑ति॒ प्र । \newline
5. प्रैवैव प्र प्रैव । \newline
6. ए॒व प॑श्यति पश्य त्ये॒वैव प॑श्यति । \newline
7. प॒श्य॒ त्य॒ग्नये॒ ऽग्नये॑ पश्यति पश्य त्य॒ग्नये᳚ । \newline
8. अ॒ग्नये॑ पु॒त्रव॑ते पु॒त्रव॑ते॒ ऽग्नये॒ ऽग्नये॑ पु॒त्रव॑ते । \newline
9. पु॒त्रव॑ते पुरो॒डाश॑म् पुरो॒डाश॑म् पु॒त्रव॑ते पु॒त्रव॑ते पुरो॒डाश᳚म् । \newline
10. पु॒त्रव॑त॒ इति॑ पु॒त्र - व॒ते॒ । \newline
11. पु॒रो॒डाश॑ म॒ष्टाक॑पाल म॒ष्टाक॑पालम् पुरो॒डाश॑म् पुरो॒डाश॑ म॒ष्टाक॑पालम् । \newline
12. अ॒ष्टाक॑पाल॒म् निर् णिर॒ष्टाक॑पाल म॒ष्टाक॑पाल॒म् निः । \newline
13. अ॒ष्टाक॑पाल॒मित्य॒ष्टा - क॒पा॒ल॒म् । \newline
14. निर् व॑पेद् वपे॒न् निर् णिर् व॑पेत् । \newline
15. व॒पे॒ दिन्द्रा॒ये न्द्रा॑य वपेद् वपे॒ दिन्द्रा॑य । \newline
16. इन्द्रा॑य पु॒त्रिणे॑ पु॒त्रिण॒ इन्द्रा॒ये न्द्रा॑य पु॒त्रिणे᳚ । \newline
17. पु॒त्रिणे॑ पुरो॒डाश॑म् पुरो॒डाश॑म् पु॒त्रिणे॑ पु॒त्रिणे॑ पुरो॒डाश᳚म् । \newline
18. पु॒रो॒डाश॒ मेका॑दशकपाल॒ मेका॑दशकपालम् पुरो॒डाश॑म् पुरो॒डाश॒ मेका॑दशकपालम् । \newline
19. एका॑दशकपालम् प्र॒जाका॑मः प्र॒जाका॑म॒ एका॑दशकपाल॒ मेका॑दशकपालम् प्र॒जाका॑मः । \newline
20. एका॑दशकपाल॒मित्येका॑दश - क॒पा॒ल॒म् । \newline
21. प्र॒जाका॑मो॒ ऽग्नि र॒ग्निः प्र॒जाका॑मः प्र॒जाका॑मो॒ ऽग्निः । \newline
22. प्र॒जाका॑म॒ इति॑ प्र॒जा - का॒मः॒ । \newline
23. अ॒ग्नि रे॒वैवाग्नि र॒ग्नि रे॒व । \newline
24. ए॒वास्मा॑ अस्मा ए॒वैवास्मै᳚ । \newline
25. अ॒स्मै॒ प्र॒जाम् प्र॒जा म॑स्मा अस्मै प्र॒जाम् । \newline
26. प्र॒जाम् प्र॑ज॒नय॑ति प्रज॒नय॑ति प्र॒जाम् प्र॒जाम् प्र॑ज॒नय॑ति । \newline
27. प्र॒जामिति॑ प्र - जाम् । \newline
28. प्र॒ज॒नय॑ति वृ॒द्धां ॅवृ॒द्धाम् प्र॑ज॒नय॑ति प्रज॒नय॑ति वृ॒द्धाम् । \newline
29. प्र॒ज॒नय॒तीति॑ प्र - ज॒नय॑ति । \newline
30. वृ॒द्धा मिन्द्र॒ इन्द्रो॑ वृ॒द्धां ॅवृ॒द्धा मिन्द्रः॑ । \newline
31. इन्द्रः॒ प्र प्रे न्द्र॒ इन्द्रः॒ प्र । \newline
32. प्र य॑च्छति यच्छति॒ प्र प्र य॑च्छति । \newline
33. य॒च्छ॒ त्य॒ग्नये॒ ऽग्नये॑ यच्छति यच्छ त्य॒ग्नये᳚ । \newline
34. अ॒ग्नये॒ रस॑वते॒ रस॑वते॒ ऽग्नये॒ ऽग्नये॒ रस॑वते । \newline
35. रस॑वते ऽजक्षी॒रे॑ ऽजक्षी॒रे रस॑वते॒ रस॑वते ऽजक्षी॒रे । \newline
36. रस॑वत॒ इति॒ रस॑ - व॒ते॒ । \newline
37. अ॒ज॒क्षी॒रे च॒रुम् च॒रु म॑जक्षी॒रे॑ ऽजक्षी॒रे च॒रुम् । \newline
38. अ॒ज॒क्षी॒र इत्य॑ज - क्षी॒रे । \newline
39. च॒रुम् निर् णिश्च॒रुम् च॒रुम् निः । \newline
40. निर् व॑पेद् वपे॒न् निर् णिर् व॑पेत् । \newline
41. व॒पे॒द् यो यो व॑पेद् वपे॒द् यः । \newline
42. यः का॒मये॑त का॒मये॑त॒ यो यः का॒मये॑त । \newline
43. का॒मये॑त॒ रस॑वा॒न् रस॑वान् का॒मये॑त का॒मये॑त॒ रस॑वान् । \newline
44. रस॑वान् थ्स्याꣳ स्याꣳ॒॒ रस॑वा॒न् रस॑वान् थ्स्याम् । \newline
45. रस॑वा॒निति॒ रस॑ - वा॒न् । \newline
46. स्या॒ मितीति॑ स्याꣳ स्या॒ मिति॑ । \newline
47. इत्य॒ग्नि म॒ग्नि मिती त्य॒ग्निम् । \newline
48. अ॒ग्नि मे॒वैवाग्नि म॒ग्नि मे॒व । \newline
49. ए॒व रस॑वन्तꣳ॒॒ रस॑वन्त मे॒वैव रस॑वन्तम् । \newline
50. रस॑वन्तꣳ॒॒ स्वेन॒ स्वेन॒ रस॑वन्तꣳ॒॒ रस॑वन्तꣳ॒॒ स्वेन॑ । \newline
51. रस॑वन्त॒मिति॒ रस॑ - व॒न्त॒म् । \newline
52. स्वेन॑ भाग॒धेये॑न भाग॒धेये॑न॒ स्वेन॒ स्वेन॑ भाग॒धेये॑न । \newline
53. भा॒ग॒धेये॒नोपोप॑ भाग॒धेये॑न भाग॒धेये॒नोप॑ । \newline
54. भा॒ग॒धेये॒नेति॑ भाग - धेये॑न । \newline
55. उप॑ धावति धाव॒ त्युपोप॑ धावति । \newline
56. धा॒व॒ति॒ स स धा॑वति धावति॒ सः । \newline
57. स ए॒वैव स स ए॒व । \newline
58. ए॒वैन॑ मेन मे॒वैवैन᳚म् । \newline
59. ए॒नꣳ॒॒ रस॑वन्तꣳ॒॒ रस॑वन्त मेन मेनꣳ॒॒ रस॑वन्तम् । \newline
60. रस॑वन्तम् करोति करोति॒ रस॑वन्तꣳ॒॒ रस॑वन्तम् करोति । \newline
61. रस॑वन्त॒मिति॒ रस॑ - व॒न्त॒म् । \newline
62. क॒रो॒ति॒ रस॑वा॒न् रस॑वान् करोति करोति॒ रस॑वान् । \newline

\textbf{Ghana Paata } \newline

1. उ॒त यदि॒ यद्यु॒तोत यद्य॒न्धो᳚ ऽन्धो यद्यु॒तोत यद्य॒न्धः । \newline
2. यद्य॒न्धो᳚ ऽन्धो यदि॒ यद्य॒न्धो भव॑ति॒ भव॑ त्य॒न्धो यदि॒ यद्य॒न्धो भव॑ति । \newline
3. अ॒न्धो भव॑ति॒ भव॑ त्य॒न्धो᳚ ऽन्धो भव॑ति॒ प्र प्र भव॑ त्य॒न्धो᳚ ऽन्धो भव॑ति॒ प्र । \newline
4. भव॑ति॒ प्र प्र भव॑ति॒ भव॑ति॒ प्रैवैव प्र भव॑ति॒ भव॑ति॒ प्रैव । \newline
5. प्रैवैव प्र प्रैव प॑श्यति पश्य त्ये॒व प्र प्रैव प॑श्यति । \newline
6. ए॒व प॑श्यति पश्य त्ये॒वैव प॑श्य त्य॒ग्नये॒ ऽग्नये॑ पश्य त्ये॒वैव प॑श्य त्य॒ग्नये᳚ । \newline
7. प॒श्य॒ त्य॒ग्नये॒ ऽग्नये॑ पश्यति पश्य त्य॒ग्नये॑ पु॒त्रव॑ते पु॒त्रव॑ते॒ ऽग्नये॑ पश्यति पश्य त्य॒ग्नये॑ पु॒त्रव॑ते । \newline
8. अ॒ग्नये॑ पु॒त्रव॑ते पु॒त्रव॑ते॒ ऽग्नये॒ ऽग्नये॑ पु॒त्रव॑ते पुरो॒डाश॑म् पुरो॒डाश॑म् पु॒त्रव॑ते॒ ऽग्नये॒ ऽग्नये॑ पु॒त्रव॑ते पुरो॒डाश᳚म् । \newline
9. पु॒त्रव॑ते पुरो॒डाश॑म् पुरो॒डाश॑म् पु॒त्रव॑ते पु॒त्रव॑ते पुरो॒डाश॑ म॒ष्टाक॑पाल म॒ष्टाक॑पालम् पुरो॒डाश॑म् पु॒त्रव॑ते पु॒त्रव॑ते पुरो॒डाश॑ म॒ष्टाक॑पालम् । \newline
10. पु॒त्रव॑त॒ इति॑ पु॒त्र - व॒ते॒ । \newline
11. पु॒रो॒डाश॑ म॒ष्टाक॑पाल म॒ष्टाक॑पालम् पुरो॒डाश॑म् पुरो॒डाश॑ म॒ष्टाक॑पाल॒म् निर् णिर॒ष्टाक॑पालम् पुरो॒डाश॑म् पुरो॒डाश॑ म॒ष्टाक॑पाल॒म् निः । \newline
12. अ॒ष्टाक॑पाल॒म् निर् णिर॒ष्टाक॑पाल म॒ष्टाक॑पाल॒म् निर् व॑पेद् वपे॒न् निर॒ष्टाक॑पाल म॒ष्टाक॑पाल॒म् निर् व॑पेत् । \newline
13. अ॒ष्टाक॑पाल॒मित्य॒ष्टा - क॒पा॒ल॒म् । \newline
14. निर् व॑पेद् वपे॒न् निर् णिर् व॑पे॒दिन्द्रा॒ये न्द्रा॑य वपे॒न् निर् णिर् व॑पे॒दिन्द्रा॑य । \newline
15. व॒पे॒दिन्द्रा॒ये न्द्रा॑य वपेद् वपे॒दिन्द्रा॑य पु॒त्रिणे॑ पु॒त्रिण॒ इन्द्रा॑य वपेद् वपे॒दिन्द्रा॑य पु॒त्रिणे᳚ । \newline
16. इन्द्रा॑य पु॒त्रिणे॑ पु॒त्रिण॒ इन्द्रा॒ये न्द्रा॑य पु॒त्रिणे॑ पुरो॒डाश॑म् पुरो॒डाश॑म् पु॒त्रिण॒ इन्द्रा॒ये न्द्रा॑य पु॒त्रिणे॑ पुरो॒डाश᳚म् । \newline
17. पु॒त्रिणे॑ पुरो॒डाश॑म् पुरो॒डाश॑म् पु॒त्रिणे॑ पु॒त्रिणे॑ पुरो॒डाश॒ मेका॑दशकपाल॒ मेका॑दशकपालम् पुरो॒डाश॑म् पु॒त्रिणे॑ पु॒त्रिणे॑ पुरो॒डाश॒ मेका॑दशकपालम् । \newline
18. पु॒रो॒डाश॒ मेका॑दशकपाल॒ मेका॑दशकपालम् पुरो॒डाश॑म् पुरो॒डाश॒ मेका॑दशकपालम् प्र॒जाका॑मः प्र॒जाका॑म॒ एका॑दशकपालम् पुरो॒डाश॑म् पुरो॒डाश॒ मेका॑दशकपालम् प्र॒जाका॑मः । \newline
19. एका॑दशकपालम् प्र॒जाका॑मः प्र॒जाका॑म॒ एका॑दशकपाल॒ मेका॑दशकपालम् प्र॒जाका॑मो॒ ऽग्निर॒ग्निः प्र॒जाका॑म॒ एका॑दशकपाल॒ मेका॑दशकपालम् प्र॒जाका॑मो॒ ऽग्निः । \newline
20. एका॑दशकपाल॒मित्येका॑दश - क॒पा॒ल॒म् । \newline
21. प्र॒जाका॑मो॒ ऽग्नि र॒ग्निः प्र॒जाका॑मः प्र॒जाका॑मो॒ ऽग्नि रे॒वैवाग्निः प्र॒जाका॑मः प्र॒जाका॑मो॒ ऽग्नि रे॒व । \newline
22. प्र॒जाका॑म॒ इति॑ प्र॒जा - का॒मः॒ । \newline
23. अ॒ग्नि रे॒वैवाग्नि र॒ग्नि रे॒वास्मा॑ अस्मा ए॒वाग्नि र॒ग्नि रे॒वास्मै᳚ । \newline
24. ए॒वास्मा॑ अस्मा ए॒वैवास्मै᳚ प्र॒जाम् प्र॒जा म॑स्मा ए॒वैवास्मै᳚ प्र॒जाम् । \newline
25. अ॒स्मै॒ प्र॒जाम् प्र॒जा म॑स्मा अस्मै प्र॒जाम् प्र॑ज॒नय॑ति प्रज॒नय॑ति प्र॒जा म॑स्मा अस्मै प्र॒जाम् प्र॑ज॒नय॑ति । \newline
26. प्र॒जाम् प्र॑ज॒नय॑ति प्रज॒नय॑ति प्र॒जाम् प्र॒जाम् प्र॑ज॒नय॑ति वृ॒द्धां ॅवृ॒द्धाम् प्र॑ज॒नय॑ति प्र॒जाम् प्र॒जाम् प्र॑ज॒नय॑ति वृ॒द्धाम् । \newline
27. प्र॒जामिति॑ प्र - जाम् । \newline
28. प्र॒ज॒नय॑ति वृ॒द्धां ॅवृ॒द्धाम् प्र॑ज॒नय॑ति प्रज॒नय॑ति वृ॒द्धा मिन्द्र॒ इन्द्रो॑ वृ॒द्धाम् प्र॑ज॒नय॑ति प्रज॒नय॑ति वृ॒द्धा मिन्द्रः॑ । \newline
29. प्र॒ज॒नय॒तीति॑ प्र - ज॒नय॑ति । \newline
30. वृ॒द्धा मिन्द्र॒ इन्द्रो॑ वृ॒द्धां ॅवृ॒द्धा मिन्द्रः॒ प्र प्रे न्द्रो॑ वृ॒द्धां ॅवृ॒द्धा मिन्द्रः॒ प्र । \newline
31. इन्द्रः॒ प्र प्रे न्द्र॒ इन्द्रः॒ प्र य॑च्छति यच्छति॒ प्रे न्द्र॒ इन्द्रः॒ प्र य॑च्छति । \newline
32. प्र य॑च्छति यच्छति॒ प्र प्र य॑च्छ त्य॒ग्नये॒ ऽग्नये॑ यच्छति॒ प्र प्र य॑च्छ त्य॒ग्नये᳚ । \newline
33. य॒च्छ॒ त्य॒ग्नये॒ ऽग्नये॑ यच्छति यच्छ त्य॒ग्नये॒ रस॑वते॒ रस॑वते॒ ऽग्नये॑ यच्छति यच्छ त्य॒ग्नये॒ रस॑वते । \newline
34. अ॒ग्नये॒ रस॑वते॒ रस॑वते॒ ऽग्नये॒ ऽग्नये॒ रस॑वते ऽजक्षी॒रे॑ ऽजक्षी॒रे रस॑वते॒ ऽग्नये॒ ऽग्नये॒ रस॑वते ऽजक्षी॒रे । \newline
35. रस॑वते ऽजक्षी॒रे॑ ऽजक्षी॒रे रस॑वते॒ रस॑वते ऽजक्षी॒रे च॒रुम् च॒रु म॑जक्षी॒रे रस॑वते॒ रस॑वते ऽजक्षी॒रे च॒रुम् । \newline
36. रस॑वत॒ इति॒ रस॑ - व॒ते॒ । \newline
37. अ॒ज॒क्षी॒रे च॒रुम् च॒रु म॑जक्षी॒रे॑ ऽजक्षी॒रे च॒रुम् निर् णिश्च॒रु म॑जक्षी॒रे॑ ऽजक्षी॒रे च॒रुम् निः । \newline
38. अ॒ज॒क्षी॒र इत्य॑ज - क्षी॒रे । \newline
39. च॒रुम् निर् णिश्च॒रुम् च॒रुम् निर् व॑पेद् वपे॒न् निश्च॒रुम् च॒रुम् निर् व॑पेत् । \newline
40. निर् व॑पेद् वपे॒न् निर् णिर् व॑पे॒द् यो यो व॑पे॒न् निर् णिर् व॑पे॒द् यः । \newline
41. व॒पे॒द् यो यो व॑पेद् वपे॒द् यः का॒मये॑त का॒मये॑त॒ यो व॑पेद् वपे॒द् यः का॒मये॑त । \newline
42. यः का॒मये॑त का॒मये॑त॒ यो यः का॒मये॑त॒ रस॑वा॒न् रस॑वान् का॒मये॑त॒ यो यः का॒मये॑त॒ रस॑वान् । \newline
43. का॒मये॑त॒ रस॑वा॒न् रस॑वान् का॒मये॑त का॒मये॑त॒ रस॑वान् थ्स्याꣳ स्याꣳ॒॒ रस॑वान् का॒मये॑त का॒मये॑त॒ रस॑वान् थ्स्याम् । \newline
44. रस॑वान् थ्स्याꣳ स्याꣳ॒॒ रस॑वा॒न् रस॑वान् थ्स्या॒ मितीति॑ स्याꣳ॒॒ रस॑वा॒न् रस॑वान् थ्स्या॒ मिति॑ । \newline
45. रस॑वा॒निति॒ रस॑ - वा॒न् । \newline
46. स्या॒ मितीति॑ स्याꣳ स्या॒ मित्य॒ग्नि म॒ग्नि मिति॑ स्याꣳ स्या॒ मित्य॒ग्निम् । \newline
47. इत्य॒ग्नि म॒ग्नि मिती त्य॒ग्नि मे॒वैवाग्नि मिती त्य॒ग्नि मे॒व । \newline
48. अ॒ग्नि मे॒वैवाग्नि म॒ग्नि मे॒व रस॑वन्तꣳ॒॒ रस॑वन्त मे॒वाग्नि म॒ग्नि मे॒व रस॑वन्तम् । \newline
49. ए॒व रस॑वन्तꣳ॒॒ रस॑वन्त मे॒वैव रस॑वन्तꣳ॒॒ स्वेन॒ स्वेन॒ रस॑वन्त मे॒वैव रस॑वन्तꣳ॒॒ स्वेन॑ । \newline
50. रस॑वन्तꣳ॒॒ स्वेन॒ स्वेन॒ रस॑वन्तꣳ॒॒ रस॑वन्तꣳ॒॒ स्वेन॑ भाग॒धेये॑न भाग॒धेये॑न॒ स्वेन॒ रस॑वन्तꣳ॒॒ रस॑वन्तꣳ॒॒ स्वेन॑ भाग॒धेये॑न । \newline
51. रस॑वन्त॒मिति॒ रस॑ - व॒न्त॒म् । \newline
52. स्वेन॑ भाग॒धेये॑न भाग॒धेये॑न॒ स्वेन॒ स्वेन॑ भाग॒धेये॒नोपोप॑ भाग॒धेये॑न॒ स्वेन॒ स्वेन॑ भाग॒धेये॒नोप॑ । \newline
53. भा॒ग॒धेये॒नोपोप॑ भाग॒धेये॑न भाग॒धेये॒नोप॑ धावति धाव॒त्युप॑ भाग॒धेये॑न भाग॒धेये॒नोप॑ धावति । \newline
54. भा॒ग॒धेये॒नेति॑ भाग - धेये॑न । \newline
55. उप॑ धावति धाव॒ त्युपोप॑ धावति॒ स स धा॑व॒ त्युपोप॑ धावति॒ सः । \newline
56. धा॒व॒ति॒ स स धा॑वति धावति॒ स ए॒वैव स धा॑वति धावति॒ स ए॒व । \newline
57. स ए॒वैव स स ए॒वैन॑ मेन मे॒व स स ए॒वैन᳚म् । \newline
58. ए॒वैन॑ मेन मे॒वैवैनꣳ॒॒ रस॑वन्तꣳ॒॒ रस॑वन्त मेन मे॒वैवैनꣳ॒॒ रस॑वन्तम् । \newline
59. ए॒नꣳ॒॒ रस॑वन्तꣳ॒॒ रस॑वन्त मेन मेनꣳ॒॒ रस॑वन्तम् करोति करोति॒ रस॑वन्त मेन मेनꣳ॒॒ रस॑वन्तम् करोति । \newline
60. रस॑वन्तम् करोति करोति॒ रस॑वन्तꣳ॒॒ रस॑वन्तम् करोति॒ रस॑वा॒न् रस॑वान् करोति॒ रस॑वन्तꣳ॒॒ रस॑वन्तम् करोति॒ रस॑वान् । \newline
61. रस॑वन्त॒मिति॒ रस॑ - व॒न्त॒म् । \newline
62. क॒रो॒ति॒ रस॑वा॒न् रस॑वान् करोति करोति॒ रस॑वा ने॒वैव रस॑वान् करोति करोति॒ रस॑वा ने॒व । \newline
\pagebreak
\markright{ TS 2.2.4.5  \hfill https://www.vedavms.in \hfill}
\addcontentsline{toc}{section}{ TS 2.2.4.5 }
\section*{ TS 2.2.4.5 }

\textbf{TS 2.2.4.5 } \newline
\textbf{Samhita Paata} \newline

रस॑वाने॒व भ॑वत्यजक्षी॒रे भ॑वत्याग्ने॒यी वा ए॒षा यद॒जा सा॒क्षादे॒व रस॒मव॑ रुन्धे॒ऽग्नये॒ वसु॑मते पुरो॒डाश॑म॒ष्टाक॑पालं॒ निर्व॑पे॒द्यः का॒मये॑त॒ वसु॑मान्थ्-स्या॒मित्य॒ग्नि-मे॒व वसु॑मन्तꣳ॒॒ स्वेन॑ भाग॒धेये॒नोप॑ धावति॒ स ए॒वैनं॒ ॅवसु॑मन्तं करोति॒ वसु॑माने॒व भ॑वत्य॒ग्नये॑ वाज॒सृते॑ पुरो॒डाश॑म॒ष्टाक॑पालं॒ निव॑र्पेथ् संग्रा॒मे सं ॅय॑त्ते॒ वाजं॒ - [  ] \newline

\textbf{Pada Paata} \newline

रस॑वा॒निति॒ रस॑ - वा॒न् । ए॒व । भ॒व॒ति॒ । अ॒ज॒क्षी॒र इत्य॑ज - क्षी॒रे । भ॒व॒ति॒ । आ॒ग्ने॒यी । वै । ए॒षा । यत् । अ॒जा । सा॒क्षादिति॑ स - अ॒क्षात् । ए॒व । रस᳚म् । अवेति॑ । रु॒न्धे॒ । अ॒ग्नये᳚ । वसु॑मत॒ इति॒ वसु॑ - म॒ते॒ । पु॒रो॒डाश᳚म् । अ॒ष्टाक॑पाल॒मित्य॒ष्टा - क॒पा॒ल॒म् । निरिति॑ । व॒पे॒त् । यः । का॒मये॑त । वसु॑मा॒निति॒ वसु॑ - मा॒न् । स्या॒म् । इति॑ । अ॒ग्निम् । ए॒व । वसु॑मन्त॒मिति॒ वसु॑ - म॒न्त॒म् । स्वेन॑ । भा॒ग॒धेये॒नेति॑ भाग - धेये॑न । उपेति॑ । धा॒व॒ति॒ । सः । ए॒व । ए॒न॒म् । वसु॑मन्त॒मिति॒ वसु॑ - म॒न्त॒म् । क॒रो॒ति॒ । वसु॑मा॒निति॒॒ वसु॑ - मा॒न् । ए॒व । भ॒व॒ति॒ । अ॒ग्नये᳚ । वा॒ज॒सृत॒ इति॑ वाज - सृते᳚ । पु॒रो॒डाश᳚म् । अ॒ष्टाक॑पाल॒मित्य॒ष्टा - क॒पा॒ल॒म् । निरिति॑ । व॒पे॒त् । स॒ग्रां॒म इति॑ सं - ग्रा॒मे । संॅय॑त्त॒ इति॒ सं - य॒त्ते॒ । वाज᳚म् ।  \newline


\textbf{Krama Paata} \newline

रस॑वाने॒व । रस॑वा॒निति॒ रस॑ - वा॒न्॒ । ए॒व भ॑वति । भ॒व॒त्य॒ज॒क्षी॒रे । अ॒ज॒क्षी॒रे भ॑वति । अ॒ज॒क्षी॒र इत्य॑ज - क्षी॒रे । भ॒व॒त्या॒ग्ने॒यी । आ॒ग्ने॒यी वै । वा ए॒षा । ए॒षा यत् । यद॒जा । अ॒जा सा॒क्षात् । सा॒क्षादे॒व । सा॒क्षादिति॑ स - अ॒क्षात् । ए॒व रस᳚म् । रस॒मव॑ । अव॑ रुन्धे । रु॒न्धे॒ ऽग्नये᳚ । अ॒ग्नये॒ वसु॑मते । वसु॑मते पुरो॒डाश᳚म् । वसु॑मत॒ इति॒ वसु॑ - म॒ते॒ । पु॒रो॒डाश॑म॒ष्टाक॑पालम् । अ॒ष्टाक॑पाल॒म् निः । अ॒ष्टाक॑पाल॒मित्य॒ष्टा - क॒पा॒ल॒म् । निर् व॑पेत् । व॒पे॒द् यः । यः का॒मये॑त । का॒मये॑त॒ वसु॑मान् । वसु॑मान्थ् स्याम् । वसु॑मा॒निति॒ वसु॑ - मा॒न्॒ । स्या॒मिति॑ । इत्य॒ग्निम् । अ॒ग्निमे॒व । ए॒व वसु॑मन्तम् । वसु॑मन्तꣳ॒॒ स्वेन॑ । वसु॑वन्त॒मिति॒ वसु॑ - म॒न्त॒॒म् । स्वेन॑ भाग॒धेये॑न । भा॒ग॒धेये॒नोप॑ । भा॒ग॒धेये॒नेति॑ भाग - धेये॑न । उप॑ धावति । धा॒व॒ति॒ सः । स ए॒व । ए॒वैन᳚म् । ए॒नं॒ ॅवसु॑मन्तम् । वसु॑मन्तम् करोति । वसु॑मन्त॒मिति॒ वसु॑ - म॒न्त॒॒म् । क॒रो॒ति॒ वसु॑मान् । वसु॑माने॒व । वसु॑मा॒निति॒ वसु॑ - मा॒न्॒ । ए॒व भ॑वति । भ॒व॒त्य॒ग्नये᳚ । अ॒ग्नये॑ वाज॒सृते᳚ । वा॒ज॒सृते॑ पुरो॒डाश᳚म् । वा॒ज॒सृत॒ इति॑ वाज - सृते᳚ । पु॒रो॒डाश॑म॒ष्टाक॑पालम् । अ॒ष्टाक॑पाल॒म् निः । अ॒ष्टाक॑पाल॒मित्य॒ष्टा - क॒पा॒ल॒म् । निर् व॑पेत् । व॒पे॒थ् स॒ङ्ग्रा॒मे । स॒ङ्ग्रा॒मे सम्ॅय॑त्ते । स॒ङ्ग्रा॒म इति॑ सं - ग्रा॒मे । सम्ॅय॑त्ते॒ वाज᳚म् । सम्ॅय॑त्त॒ इति॒ सं - य॒त्ते॒ । वाजं॒ ॅवै \newline

\textbf{Jatai Paata} \newline

1. रस॑वा ने॒वैव रस॑वा॒न् रस॑वा ने॒व । \newline
2. रस॑वा॒निति॒ रस॑ - वा॒न् । \newline
3. ए॒व भ॑वति भव त्ये॒वैव भ॑वति । \newline
4. भ॒व॒ त्य॒ज॒क्षी॒रे॑ ऽजक्षी॒रे भ॑वति भव त्यजक्षी॒रे । \newline
5. अ॒ज॒क्षी॒रे भ॑वति भव त्यजक्षी॒रे॑ ऽजक्षी॒रे भ॑वति । \newline
6. अ॒ज॒क्षी॒र इत्य॑ज - क्षी॒रे । \newline
7. भ॒व॒ त्या॒ग्ने॒य्या᳚ग्ने॒यी भ॑वति भव त्याग्ने॒यी । \newline
8. आ॒ग्ने॒यी वै वा आ᳚ग्ने॒य्या᳚ग्ने॒यी वै । \newline
9. वा ए॒षैषा वै वा ए॒षा । \newline
10. ए॒षा यद् यदे॒षैषा यत् । \newline
11. यद॒जा ऽजा यद् यद॒जा । \newline
12. अ॒जा सा॒क्षाथ् सा॒क्षा द॒जा ऽजा सा॒क्षात् । \newline
13. सा॒क्षा दे॒वैव सा॒क्षाथ् सा॒क्षा दे॒व । \newline
14. सा॒क्षादिति॑ स - अ॒क्षात् । \newline
15. ए॒व रसꣳ॒॒ रस॑ मे॒वैव रस᳚म् । \newline
16. रस॒ मवाव॒ रसꣳ॒॒ रस॒ मव॑ । \newline
17. अव॑ रुन्धे रु॒न्धे ऽवाव॑ रुन्धे । \newline
18. रु॒न्धे॒ ऽग्नये॒ ऽग्नये॑ रुन्धे रुन्धे॒ ऽग्नये᳚ । \newline
19. अ॒ग्नये॒ वसु॑मते॒ वसु॑मते॒ ऽग्नये॒ ऽग्नये॒ वसु॑मते । \newline
20. वसु॑मते पुरो॒डाश॑म् पुरो॒डाशं॒ ॅवसु॑मते॒ वसु॑मते पुरो॒डाश᳚म् । \newline
21. वसु॑मत॒ इति॒ वसु॑ - म॒ते॒ । \newline
22. पु॒रो॒डाश॑ म॒ष्टाक॑पाल म॒ष्टाक॑पालम् पुरो॒डाश॑म् पुरो॒डाश॑ म॒ष्टाक॑पालम् । \newline
23. अ॒ष्टाक॑पाल॒म् निर् णिर॒ष्टाक॑पाल म॒ष्टाक॑पाल॒म् निः । \newline
24. अ॒ष्टाक॑पाल॒मित्य॒ष्टा - क॒पा॒ल॒म् । \newline
25. निर् व॑पेद् वपे॒न् निर् णिर् व॑पेत् । \newline
26. व॒पे॒द् यो यो व॑पेद् वपे॒द् यः । \newline
27. यः का॒मये॑त का॒मये॑त॒ यो यः का॒मये॑त । \newline
28. का॒मये॑त॒ वसु॑मा॒न्॒. वसु॑मान् का॒मये॑त का॒मये॑त॒ वसु॑मान् । \newline
29. वसु॑मान् थ्स्याꣳ स्यां॒ ॅवसु॑मा॒न्॒. वसु॑मान् थ्स्याम् । \newline
30. वसु॑मा॒निति॒ वसु॑ - मा॒न् । \newline
31. स्या॒ मितीति॑ स्याꣳ स्या॒ मिति॑ । \newline
32. इत्य॒ग्नि म॒ग्नि मिती त्य॒ग्निम् । \newline
33. अ॒ग्नि मे॒वैवाग्नि म॒ग्नि मे॒व । \newline
34. ए॒व वसु॑मन्तं॒ ॅवसु॑मन्त मे॒वैव वसु॑मन्तम् । \newline
35. वसु॑मन्तꣳ॒॒ स्वेन॒ स्वेन॒ वसु॑मन्तं॒ ॅवसु॑मन्तꣳ॒॒ स्वेन॑ । \newline
36. वसु॑मन्त॒मिति॒ वसु॑ - म॒न्त॒म् । \newline
37. स्वेन॑ भाग॒धेये॑न भाग॒धेये॑न॒ स्वेन॒ स्वेन॑ भाग॒धेये॑न । \newline
38. भा॒ग॒धेये॒नोपोप॑ भाग॒धेये॑न भाग॒धेये॒नोप॑ । \newline
39. भा॒ग॒धेये॒नेति॑ भाग - धेये॑न । \newline
40. उप॑ धावति धाव॒ त्युपोप॑ धावति । \newline
41. धा॒व॒ति॒ स स धा॑वति धावति॒ सः । \newline
42. स ए॒वैव स स ए॒व । \newline
43. ए॒वैन॑ मेन मे॒वैवैन᳚म् । \newline
44. ए॒नं॒ ॅवसु॑मन्तं॒ ॅवसु॑मन्त मेन मेनं॒ ॅवसु॑मन्तम् । \newline
45. वसु॑मन्तम् करोति करोति॒ वसु॑मन्तं॒ ॅवसु॑मन्तम् करोति । \newline
46. वसु॑मन्त॒मिति॒ वसु॑ - म॒न्त॒म् । \newline
47. क॒रो॒ति॒ वसु॑मा॒न्॒. वसु॑मान् करोति करोति॒ वसु॑मान् । \newline
48. वसु॑मा ने॒वैव वसु॑मा॒न्॒. वसु॑मा ने॒व । \newline
49. वसु॑मा॒निति॒॒ वसु॑ - मा॒न् । \newline
50. ए॒व भ॑वति भव त्ये॒वैव भ॑वति । \newline
51. भ॒व॒ त्य॒ग्नये॒ ऽग्नये॑ भवति भव त्य॒ग्नये᳚ । \newline
52. अ॒ग्नये॑ वाज॒सृते॑ वाज॒सृते॒ ऽग्नये॒ ऽग्नये॑ वाज॒सृते᳚ । \newline
53. वा॒ज॒सृते॑ पुरो॒डाश॑म् पुरो॒डाशं॑ ॅवाज॒सृते॑ वाज॒सृते॑ पुरो॒डाश᳚म् । \newline
54. वा॒ज॒सृत॒ इति॑ वाज - सृते᳚ । \newline
55. पु॒रो॒डाश॑ म॒ष्टाक॑पाल म॒ष्टाक॑पालम् पुरो॒डाश॑म् पुरो॒डाश॑ म॒ष्टाक॑पालम् । \newline
56. अ॒ष्टाक॑पाल॒म् निर् णिर॒ष्टाक॑पाल म॒ष्टाक॑पाल॒म् निः । \newline
57. अ॒ष्टाक॑पाल॒मित्य॒ष्टा - क॒पा॒ल॒म् । \newline
58. निर् व॑पेद् वपे॒न् निर् णिर् व॑पेत् । \newline
59. व॒पे॒थ् स॒ङ्ग्रा॒मे स॑ङ्ग्रा॒मे व॑पेद् वपेथ् सङ्ग्रा॒मे । \newline
60. स॒ङ्ग्रा॒मे संॅय॑त्ते॒ संॅय॑त्ते सङ्ग्रा॒मे स॑ङ्ग्रा॒मे संॅय॑त्ते । \newline
61. स॒ङ्ग्रा॒म इति॑ सं - ग्रा॒मे । \newline
62. संॅय॑त्ते॒ वाजं॒ ॅवाजꣳ॒॒ संॅय॑त्ते॒ संॅय॑त्ते॒ वाज᳚म् । \newline
63. संॅय॑त्त॒ इति॒ सं - य॒त्ते॒ । \newline
64. वाजं॒ ॅवै वै वाजं॒ ॅवाजं॒ ॅवै । \newline

\textbf{Ghana Paata } \newline

1. रस॑वा ने॒वैव रस॑वा॒न् रस॑वा ने॒व भ॑वति भवत्ये॒व रस॑वा॒न् रस॑वा ने॒व भ॑वति । \newline
2. रस॑वा॒निति॒ रस॑ - वा॒न् । \newline
3. ए॒व भ॑वति भव त्ये॒वैव भ॑व त्यजक्षी॒रे॑ ऽजक्षी॒रे भ॑व त्ये॒वैव भ॑व त्यजक्षी॒रे । \newline
4. भ॒व॒ त्य॒ज॒क्षी॒रे॑ ऽजक्षी॒रे भ॑वति भव त्यजक्षी॒रे भ॑वति भव त्यजक्षी॒रे भ॑वति भव त्यजक्षी॒रे भ॑वति । \newline
5. अ॒ज॒क्षी॒रे भ॑वति भव त्यजक्षी॒रे॑ ऽजक्षी॒रे भ॑व त्याग्ने॒य्या᳚ग्ने॒यी भ॑व त्यजक्षी॒रे॑ ऽजक्षी॒रे भ॑व त्याग्ने॒यी । \newline
6. अ॒ज॒क्षी॒र इत्य॑ज - क्षी॒रे । \newline
7. भ॒व॒ त्या॒ग्ने॒य्या᳚ग्ने॒यी भ॑वति भव त्याग्ने॒यी वै वा आ᳚ग्ने॒यी भ॑वति भव त्याग्ने॒यी वै । \newline
8. आ॒ग्ने॒यी वै वा आ᳚ग्ने॒य्या᳚ग्ने॒यी वा ए॒षैषा वा आ᳚ग्ने॒य्या᳚ग्ने॒यी वा ए॒षा । \newline
9. वा ए॒षैषा वै वा ए॒षा यद् यदे॒षा वै वा ए॒षा यत् । \newline
10. ए॒षा यद् यदे॒षैषा यद॒जा ऽजा यदे॒षैषा यद॒जा । \newline
11. यद॒जा ऽजा यद् यद॒जा सा॒क्षाथ् सा॒क्षा द॒जा यद् यद॒जा सा॒क्षात् । \newline
12. अ॒जा सा॒क्षाथ् सा॒क्षा द॒जा ऽजा सा॒क्षा दे॒वैव सा॒क्षा द॒जा ऽजा सा॒क्षा दे॒व । \newline
13. सा॒क्षा दे॒वैव सा॒क्षाथ् सा॒क्षादे॒व रसꣳ॒॒ रस॑ मे॒व सा॒क्षाथ् सा॒क्षा दे॒व रस᳚म् । \newline
14. सा॒क्षादिति॑ स - अ॒क्षात् । \newline
15. ए॒व रसꣳ॒॒ रस॑ मे॒वैव रस॒ मवाव॒ रस॑ मे॒वैव रस॒ मव॑ । \newline
16. रस॒ मवाव॒ रसꣳ॒॒ रस॒ मव॑ रुन्धे रु॒न्धे ऽव॒ रसꣳ॒॒ रस॒ मव॑ रुन्धे । \newline
17. अव॑ रुन्धे रु॒न्धे ऽवाव॑ रुन्धे॒ ऽग्नये॒ ऽग्नये॑ रु॒न्धे ऽवाव॑ रुन्धे॒ ऽग्नये᳚ । \newline
18. रु॒न्धे॒ ऽग्नये॒ ऽग्नये॑ रुन्धे रुन्धे॒ ऽग्नये॒ वसु॑मते॒ वसु॑मते॒ ऽग्नये॑ रुन्धे रुन्धे॒ ऽग्नये॒ वसु॑मते । \newline
19. अ॒ग्नये॒ वसु॑मते॒ वसु॑मते॒ ऽग्नये॒ ऽग्नये॒ वसु॑मते पुरो॒डाश॑म् पुरो॒डाशं॒ ॅवसु॑मते॒ ऽग्नये॒ ऽग्नये॒ वसु॑मते पुरो॒डाश᳚म् । \newline
20. वसु॑मते पुरो॒डाश॑म् पुरो॒डाशं॒ ॅवसु॑मते॒ वसु॑मते पुरो॒डाश॑ म॒ष्टाक॑पाल म॒ष्टाक॑पालम् पुरो॒डाशं॒ ॅवसु॑मते॒ वसु॑मते पुरो॒डाश॑ म॒ष्टाक॑पालम् । \newline
21. वसु॑मत॒ इति॒ वसु॑ - म॒ते॒ । \newline
22. पु॒रो॒डाश॑ म॒ष्टाक॑पाल म॒ष्टाक॑पालम् पुरो॒डाश॑म् पुरो॒डाश॑ म॒ष्टाक॑पाल॒म् निर् णिर॒ष्टाक॑पालम् पुरो॒डाश॑म् पुरो॒डाश॑ म॒ष्टाक॑पाल॒म् निः । \newline
23. अ॒ष्टाक॑पाल॒म् निर् णिर॒ष्टाक॑पाल म॒ष्टाक॑पाल॒म् निर् व॑पेद् वपे॒न् निर॒ष्टाक॑पाल म॒ष्टाक॑पाल॒म् निर् व॑पेत् । \newline
24. अ॒ष्टाक॑पाल॒मित्य॒ष्टा - क॒पा॒ल॒म् । \newline
25. निर् व॑पेद् वपे॒न् निर् णिर् व॑पे॒द् यो यो व॑पे॒न् निर् णिर् व॑पे॒द् यः । \newline
26. व॒पे॒द् यो यो व॑पेद् वपे॒द् यः का॒मये॑त का॒मये॑त॒ यो व॑पेद् वपे॒द् यः का॒मये॑त । \newline
27. यः का॒मये॑त का॒मये॑त॒ यो यः का॒मये॑त॒ वसु॑मा॒न्॒. वसु॑मान् का॒मये॑त॒ यो यः का॒मये॑त॒ वसु॑मान् । \newline
28. का॒मये॑त॒ वसु॑मा॒न्॒. वसु॑मान् का॒मये॑त का॒मये॑त॒ वसु॑मान् थ्स्याꣳ स्यां॒ ॅवसु॑मान् का॒मये॑त का॒मये॑त॒ वसु॑मान् थ्स्याम् । \newline
29. वसु॑मान् थ्स्याꣳ स्यां॒ ॅवसु॑मा॒न्॒. वसु॑मान् थ्स्या॒ मितीति॑ स्यां॒ ॅवसु॑मा॒न्॒. वसु॑मान् थ्स्या॒ मिति॑ । \newline
30. वसु॑मा॒निति॒ वसु॑ - मा॒न् । \newline
31. स्या॒ मितीति॑ स्याꣳ स्या॒ मित्य॒ग्नि म॒ग्नि मिति॑ स्याꣳ स्या॒ मित्य॒ग्निम् । \newline
32. इत्य॒ग्नि म॒ग्नि मिती त्य॒ग्नि मे॒वैवाग्नि मिती त्य॒ग्नि मे॒व । \newline
33. अ॒ग्नि मे॒वैवाग्नि म॒ग्नि मे॒व वसु॑मन्तं॒ ॅवसु॑मन्त मे॒वाग्नि म॒ग्नि मे॒व वसु॑मन्तम् । \newline
34. ए॒व वसु॑मन्तं॒ ॅवसु॑मन्त मे॒वैव वसु॑मन्तꣳ॒॒ स्वेन॒ स्वेन॒ वसु॑मन्त मे॒वैव वसु॑मन्तꣳ॒॒ स्वेन॑ । \newline
35. वसु॑मन्तꣳ॒॒ स्वेन॒ स्वेन॒ वसु॑मन्तं॒ ॅवसु॑मन्तꣳ॒॒ स्वेन॑ भाग॒धेये॑न भाग॒धेये॑न॒ स्वेन॒ वसु॑मन्तं॒ ॅवसु॑मन्तꣳ॒॒ स्वेन॑ भाग॒धेये॑न । \newline
36. वसु॑मन्त॒मिति॒ वसु॑ - म॒न्त॒म् । \newline
37. स्वेन॑ भाग॒धेये॑न भाग॒धेये॑न॒ स्वेन॒ स्वेन॑ भाग॒धेये॒नोपोप॑ भाग॒धेये॑न॒ स्वेन॒ स्वेन॑ भाग॒धेये॒नोप॑ । \newline
38. भा॒ग॒धेये॒नोपोप॑ भाग॒धेये॑न भाग॒धेये॒नोप॑ धावति धाव॒त्युप॑ भाग॒धेये॑न भाग॒धेये॒नोप॑ धावति । \newline
39. भा॒ग॒धेये॒नेति॑ भाग - धेये॑न । \newline
40. उप॑ धावति धाव॒ त्युपोप॑ धावति॒ स स धा॑व॒ त्युपोप॑ धावति॒ सः । \newline
41. धा॒व॒ति॒ स स धा॑वति धावति॒ स ए॒वैव स धा॑वति धावति॒ स ए॒व । \newline
42. स ए॒वैव स स ए॒वैन॑ मेन मे॒व स स ए॒वैन᳚म् । \newline
43. ए॒वैन॑ मेन मे॒वैवैनं॒ ॅवसु॑मन्तं॒ ॅवसु॑मन्त मेन मे॒वैवैनं॒ ॅवसु॑मन्तम् । \newline
44. ए॒नं॒ ॅवसु॑मन्तं॒ ॅवसु॑मन्त मेन मेनं॒ ॅवसु॑मन्तम् करोति करोति॒ वसु॑मन्त मेन मेनं॒ ॅवसु॑मन्तम् करोति । \newline
45. वसु॑मन्तम् करोति करोति॒ वसु॑मन्तं॒ ॅवसु॑मन्तम् करोति॒ वसु॑मा॒न्॒. वसु॑मान् करोति॒ वसु॑मन्तं॒ ॅवसु॑मन्तम् करोति॒ वसु॑मान् । \newline
46. वसु॑मन्त॒मिति॒ वसु॑ - म॒न्त॒म् । \newline
47. क॒रो॒ति॒ वसु॑मा॒न्॒. वसु॑मान् करोति करोति॒ वसु॑मा ने॒वैव वसु॑मान् करोति करोति॒ वसु॑मा ने॒व । \newline
48. वसु॑मा ने॒वैव वसु॑मा॒न्॒. वसु॑मा ने॒व भ॑वति भव त्ये॒व वसु॑मा॒न्॒. वसु॑मा ने॒व भ॑वति । \newline
49. वसु॑मा॒निति॒॒ वसु॑ - मा॒न् । \newline
50. ए॒व भ॑वति भव त्ये॒वैव भ॑व त्य॒ग्नये॒ ऽग्नये॑ भव त्ये॒वैव भ॑व त्य॒ग्नये᳚ । \newline
51. भ॒व॒ त्य॒ग्नये॒ ऽग्नये॑ भवति भव त्य॒ग्नये॑ वाज॒सृते॑ वाज॒सृते॒ ऽग्नये॑ भवति भव त्य॒ग्नये॑ वाज॒सृते᳚ । \newline
52. अ॒ग्नये॑ वाज॒सृते॑ वाज॒सृते॒ ऽग्नये॒ ऽग्नये॑ वाज॒सृते॑ पुरो॒डाश॑म् पुरो॒डाशं॑ ॅवाज॒सृते॒ ऽग्नये॒ ऽग्नये॑ वाज॒सृते॑ पुरो॒डाश᳚म् । \newline
53. वा॒ज॒सृते॑ पुरो॒डाश॑म् पुरो॒डाशं॑ ॅवाज॒सृते॑ वाज॒सृते॑ पुरो॒डाश॑ म॒ष्टाक॑पाल म॒ष्टाक॑पालम् पुरो॒डाशं॑ ॅवाज॒सृते॑ वाज॒सृते॑ पुरो॒डाश॑ म॒ष्टाक॑पालम् । \newline
54. वा॒ज॒सृत॒ इति॑ वाज - सृते᳚ । \newline
55. पु॒रो॒डाश॑ म॒ष्टाक॑पाल म॒ष्टाक॑पालम् पुरो॒डाश॑म् पुरो॒डाश॑ म॒ष्टाक॑पाल॒म् निर् णिर॒ष्टाक॑पालम् पुरो॒डाश॑म् पुरो॒डाश॑ म॒ष्टाक॑पाल॒म् निः । \newline
56. अ॒ष्टाक॑पाल॒म् निर् णिर॒ष्टाक॑पाल म॒ष्टाक॑पाल॒म् निर् व॑पेद् वपे॒न् निर॒ष्टाक॑पाल म॒ष्टाक॑पाल॒म् निर् व॑पेत् । \newline
57. अ॒ष्टाक॑पाल॒मित्य॒ष्टा - क॒पा॒ल॒म् । \newline
58. निर् व॑पेद् वपे॒न् निर् णिर् व॑पेथ् सङ्ग्रा॒मे स॑ङ्ग्रा॒मे व॑पे॒न् निर् णिर् व॑पेथ् सङ्ग्रा॒मे । \newline
59. व॒पे॒थ् स॒ङ्ग्रा॒मे स॑ङ्ग्रा॒मे व॑पेद् वपेथ् सङ्ग्रा॒मे संॅय॑त्ते॒ संॅय॑त्ते सङ्ग्रा॒मे व॑पेद् वपेथ् सङ्ग्रा॒मे संॅय॑त्ते । \newline
60. स॒ङ्ग्रा॒मे संॅय॑त्ते॒ संॅय॑त्ते सङ्ग्रा॒मे स॑ङ्ग्रा॒मे संॅय॑त्ते॒ वाजं॒ ॅवाजꣳ॒॒ संॅय॑त्ते सङ्ग्रा॒मे स॑ङ्ग्रा॒मे संॅय॑त्ते॒ वाज᳚म् । \newline
61. स॒ङ्ग्रा॒म इति॑ सं - ग्रा॒मे । \newline
62. संॅय॑त्ते॒ वाजं॒ ॅवाजꣳ॒॒ संॅय॑त्ते॒ संॅय॑त्ते॒ वाजं॒ ॅवै वै वाजꣳ॒॒ संॅय॑त्ते॒ संॅय॑त्ते॒ वाजं॒ ॅवै । \newline
63. संॅय॑त्त॒ इति॒ सं - य॒त्ते॒ । \newline
64. वाजं॒ ॅवै वै वाजं॒ ॅवाजं॒ ॅवा ए॒ष ए॒ष वै वाजं॒ ॅवाजं॒ ॅवा ए॒षः । \newline
\pagebreak
\markright{ TS 2.2.4.6  \hfill https://www.vedavms.in \hfill}
\addcontentsline{toc}{section}{ TS 2.2.4.6 }
\section*{ TS 2.2.4.6 }

\textbf{TS 2.2.4.6 } \newline
\textbf{Samhita Paata} \newline

ॅवा ए॒ष सि॑सीर्.षति॒ यः स॑ग्रां॒मं जिगी॑षत्य॒ग्निः खलु॒ वै दे॒वानां᳚ ॅवाज॒सृद॒ग्निमे॒व वा॑ज॒सृतꣳ॒॒ स्वेन॑ भाग॒धेये॒नोप॑ धावति॒ धाव॑ति॒ वाजꣳ॒॒ हन्ति॑ वृ॒त्रं जय॑ति॒ तꣳ स॑ग्रां॒ममथो॑ अ॒ग्निरि॑व॒ न प्र॑ति॒धृषे॑ भवत्य॒ग्नये᳚ऽग्नि॒वते॑ पुरो॒डाश॑म॒ष्टाक॑पालं॒ निर्व॑पे॒द्-यस्या॒ग्ना- व॒ग्नि- म॑भ्यु॒द्धरे॑यु॒-र्निर्दि॑ष्टभागो॒ वा ए॒तयो॑र॒न्योऽनि॑र्दिष्टभागो॒ऽन्यस्तौ स॒भंव॑न्तौ॒ यज॑मान - [  ] \newline

\textbf{Pada Paata} \newline

वै । ए॒षः । सि॒सी॒र्.॒ष॒ति॒ । यः । स॒ग्रां॒ममिति॑ सं - ग्रा॒मम् । जिगी॑षति । अ॒ग्निः । खलु॑ । वै । दे॒वाना᳚म् । वा॒ज॒सृदिति॑ वाज - सृत् । अ॒ग्निम् । ए॒व । वा॒ज॒सृत॒मिति॑ वाज-सृत᳚म् । स्वेन॑ । भा॒ग॒धेये॒नेति॑ भाग - धेये॑न । उपेति॑ । धा॒व॒ति॒ । धाव॑ति । वाज᳚म् । हन्ति॑ । वृ॒त्रम् । जय॑ति । तम् । स॒ग्रां॒ममिति॑ सं - ग्रा॒मम् । अथो॒ इति॑ । अ॒ग्निः । इ॒व॒ । न । प्र॒ति॒धृष॒ इति॑ प्रति - धृषे᳚ । भ॒व॒ति॒ । अ॒ग्नये᳚ । अ॒ग्नि॒वत॒ इत्य॑ग्नि - वते᳚ । पु॒रो॒डाश᳚म् । अ॒ष्टाक॑पाल॒मित्य॒ष्टा - क॒पा॒ल॒म् । निरिति॑ । व॒पे॒त् । यस्य॑ । अ॒ग्नौ । अ॒ग्निम् । अ॒भ्यु॒द्धरे॑यु॒रित्य॑भि - उ॒द्धरे॑युः । निर्दि॑ष्टभाग॒ इति॒ निर्दि॑ष्ट - भा॒गः॒ । वै । ए॒तयोः᳚ । अ॒न्यः । अनि॑र्दिष्टभाग॒ इत्यनि॑र्दिष्ट - भा॒गः॒ । अ॒न्यः । तौ । स॒भंव॑न्ता॒विति॑ सं - भव॑न्तौ । यज॑मानम् ।  \newline


\textbf{Krama Paata} \newline

वा ए॒षः । ए॒ष सि॑सीर्.षति । सि॒सी॒र्॒.ष॒ति॒ यः । यः स॑ङ्ग्रा॒मम् । स॒ङ्ग्रा॒मम् जिगी॑षति । स॒ङ्ग्रा॒ममिति॑ सं - ग्रा॒मम् । जिगी॑षत्य॒ग्निः । अ॒ग्निः खलु॑ । खलु॒ वै । वै दे॒वाना᳚म् । दे॒वानां᳚ ॅवाज॒सृत् । वा॒ज॒सृद॒ग्निम् । वा॒ज॒सृदिति॑ वाज - सृत् । अ॒ग्निमे॒व । ए॒व वा॑ज॒सृत᳚म् । वा॒ज॒सृतꣳ॒॒ स्वेन॑ । वा॒ज॒सृत॒मिति॑ वाज - सृत᳚म् । स्वेन॑ भाग॒धेये॑न । भा॒ग॒धेये॒नोप॑ । भा॒ग॒धेये॒नेति॑ भाग - धेये॑न । उप॑ धावति । धा॒व॒ति॒ धाव॑ति । धाव॑ति॒ वाज᳚म् । वाजꣳ॒॒ हन्ति॑ । हन्ति॑ वृ॒त्रम् । वृ॒त्रम् जय॑ति । जय॑ति॒ तम् । तꣳ स॑ङ्ग्रा॒मम् । स॒ङ्ग्रा॒ममथो᳚ । स॒ङ्ग्रा॒ममिति॑ सं - ग्रा॒मम् । अथो॑ अ॒ग्निः । अथो॒ इत्यथो᳚ । अ॒ग्निरि॑व । इ॒व॒ न । न प्र॑ति॒धृषे᳚ । प्र॒ति॒धृषे॑ भवति । प्र॒ति॒धृष॒ इति॑ प्रति - धृषे᳚ । भ॒व॒त्य॒ग्नये᳚ । अ॒ग्नये᳚ ऽग्नि॒वते᳚ । अ॒ग्नि॒वते॑ पुरो॒डाश᳚म् । अ॒ग्नि॒वत॒ इत्य॑ग्नि - वते᳚ । पु॒रो॒डाश॑म॒ष्टाक॑पालम् । अ॒ष्टाक॑पाल॒म् निः । अ॒ष्टाक॑पाल॒मित्य॒ष्टा - क॒पा॒ल॒म् । निर् व॑पेत् । व॒पे॒द्यस्य॑ । यस्या॒ग्नौ । अ॒ग्नाव॒ग्निम् । अ॒ग्निम॑भ्यु॒द्धरे॑युः । अ॒भ्यु॒द्धरे॑यु॒र् निर्दि॑ष्टभागः । अ॒भ्यु॒द्धरे॑यु॒रित्य॑भि - उ॒द्धरे॑युः । निर्दि॑ष्टभागो॒ वै । निर्दि॑ष्टभाग॒ इति॒ निर्दि॑ष्ट - भा॒गः॒ । वा ए॒तयोः᳚ । ए॒तयो॑र॒न्यः । अ॒न्योऽनि॑र्दिष्टभागः । अनि॑र्दिष्टभागो॒ ऽन्यः । अनि॑र्दिष्टभाग॒ इत्यनि॑र्दिष्ट - भा॒गः॒ । अ॒न्यस्तौ । तौ सं॒भव॑न्तौ । सं॒भव॑न्तौ॒ यज॑मानम् । सं॒भव॑न्ता॒विति॑ सं - भव॑न्तौ । यज॑मानम॒भि \newline

\textbf{Jatai Paata} \newline

1. वा ए॒ष ए॒ष वै वा ए॒षः । \newline
2. ए॒ष सि॑सीर्.षति सिसीर्.षत्ये॒ष ए॒ष सि॑सीर्.षति । \newline
3. सि॒सी॒र्॒.ष॒ति॒ यो यः सि॑सीर्.षति सिसीर्.षति॒ यः । \newline
4. यः स॑ङ्ग्रा॒मꣳ स॑ङ्ग्रा॒मं ॅयो यः स॑ङ्ग्रा॒मम् । \newline
5. स॒ङ्ग्रा॒मम् जिगी॑षति॒ जिगी॑षति सङ्ग्रा॒मꣳ स॑ङ्ग्रा॒मम् जिगी॑षति । \newline
6. स॒ङ्ग्रा॒ममिति॑ सं - ग्रा॒मम् । \newline
7. जिगी॑ष त्य॒ग्नि र॒ग्निर् जिगी॑षति॒ जिगी॑ष त्य॒ग्निः । \newline
8. अ॒ग्निः खलु॒ खल्व॒ग्नि र॒ग्निः खलु॑ । \newline
9. खलु॒ वै वै खलु॒ खलु॒ वै । \newline
10. वै दे॒वाना᳚म् दे॒वानां॒ ॅवै वै दे॒वाना᳚म् । \newline
11. दे॒वानां᳚ ॅवाज॒सृद् वा॑ज॒सृद् दे॒वाना᳚म् दे॒वानां᳚ ॅवाज॒सृत् । \newline
12. वा॒ज॒सृद॒ग्नि म॒ग्निं ॅवा॑ज॒सृद् वा॑ज॒सृद॒ग्निम् । \newline
13. वा॒ज॒सृदिति॑ वाज - सृत् । \newline
14. अ॒ग्नि मे॒वैवाग्नि म॒ग्नि मे॒व । \newline
15. ए॒व वा॑ज॒सृतं॑ ॅवाज॒सृत॑ मे॒वैव वा॑ज॒सृत᳚म् । \newline
16. वा॒ज॒सृतꣳ॒॒ स्वेन॒ स्वेन॑ वाज॒सृतं॑ ॅवाज॒सृतꣳ॒॒ स्वेन॑ । \newline
17. वा॒ज॒सृत॒मिति॑ वाज - सृत᳚म् । \newline
18. स्वेन॑ भाग॒धेये॑न भाग॒धेये॑न॒ स्वेन॒ स्वेन॑ भाग॒धेये॑न । \newline
19. भा॒ग॒धेये॒नोपोप॑ भाग॒धेये॑न भाग॒धेये॒नोप॑ । \newline
20. भा॒ग॒धेये॒नेति॑ भाग - धेये॑न । \newline
21. उप॑ धावति धाव॒ त्युपोप॑ धावति । \newline
22. धा॒व॒ति॒ धाव॑ति॒ धाव॑ति धावति धावति॒ धाव॑ति । \newline
23. धाव॑ति॒ वाजं॒ ॅवाज॒म् धाव॑ति॒ धाव॑ति॒ वाज᳚म् । \newline
24. वाजꣳ॒॒ हन्ति॒ हन्ति॒ वाजं॒ ॅवाजꣳ॒॒ हन्ति॑ । \newline
25. हन्ति॑ वृ॒त्रं ॅवृ॒त्रꣳ हन्ति॒ हन्ति॑ वृ॒त्रम् । \newline
26. वृ॒त्रम् जय॑ति॒ जय॑ति वृ॒त्रं ॅवृ॒त्रम् जय॑ति । \newline
27. जय॑ति॒ तम् तम् जय॑ति॒ जय॑ति॒ तम् । \newline
28. तꣳ स॑ङ्ग्रा॒मꣳ स॑ङ्ग्रा॒मम् तम् तꣳ स॑ङ्ग्रा॒मम् । \newline
29. स॒ङ्ग्रा॒म मथो॒ अथो॑ सङ्ग्रा॒मꣳ स॑ङ्ग्रा॒म मथो᳚ । \newline
30. स॒ङ्ग्रा॒ममिति॑ सं - ग्रा॒मम् । \newline
31. अथो॑ अ॒ग्नि र॒ग्नि रथो॒ अथो॑ अ॒ग्निः । \newline
32. अथो॒ इत्यथो᳚ । \newline
33. अ॒ग्नि रि॑वे वा॒ग्नि र॒ग्नि रि॑व । \newline
34. इ॒व॒ न ने वे॑ व॒ न । \newline
35. न प्र॑ति॒धृषे᳚ प्रति॒धृषे॒ न न प्र॑ति॒धृषे᳚ । \newline
36. प्र॒ति॒धृषे॑ भवति भवति प्रति॒धृषे᳚ प्रति॒धृषे॑ भवति । \newline
37. प्र॒ति॒धृष॒ इति॑ प्रति - धृषे᳚ । \newline
38. भ॒व॒ त्य॒ग्नये॒ ऽग्नये॑ भवति भव त्य॒ग्नये᳚ । \newline
39. अ॒ग्नये᳚ ऽग्नि॒वते᳚ ऽग्नि॒वते॒ ऽग्नये॒ ऽग्नये᳚ ऽग्नि॒वते᳚ । \newline
40. अ॒ग्नि॒वते॑ पुरो॒डाश॑म् पुरो॒डाश॑ मग्नि॒वते᳚ ऽग्नि॒वते॑ पुरो॒डाश᳚म् । \newline
41. अ॒ग्नि॒वत॒ इत्य॑ग्नि - वते᳚ । \newline
42. पु॒रो॒डाश॑ म॒ष्टाक॑पाल म॒ष्टाक॑पालम् पुरो॒डाश॑म् पुरो॒डाश॑ म॒ष्टाक॑पालम् । \newline
43. अ॒ष्टाक॑पाल॒म् निर् णिर॒ष्टाक॑पाल म॒ष्टाक॑पाल॒म् निः । \newline
44. अ॒ष्टाक॑पाल॒मित्य॒ष्टा - क॒पा॒ल॒म् । \newline
45. निर् व॑पेद् वपे॒न् निर् णिर् व॑पेत् । \newline
46. व॒पे॒द् यस्य॒ यस्य॑ वपेद् वपे॒द् यस्य॑ । \newline
47. यस्या॒ग्ना व॒ग्नौ यस्य॒ यस्या॒ग्नौ । \newline
48. अ॒ग्ना व॒ग्नि म॒ग्नि म॒ग्ना व॒ग्ना व॒ग्निम् । \newline
49. अ॒ग्नि म॑भ्यु॒द्धरे॑यु रभ्यु॒द्धरे॑यु र॒ग्नि म॒ग्नि म॑भ्यु॒द्धरे॑युः । \newline
50. अ॒भ्यु॒द्धरे॑यु॒र् निर्दि॑ष्टभागो॒ निर्दि॑ष्टभागो ऽभ्यु॒द्धरे॑यु रभ्यु॒द्धरे॑यु॒र् निर्दि॑ष्टभागः । \newline
51. अ॒भ्यु॒द्धरे॑यु॒रित्य॑भि - उ॒द्धरे॑युः । \newline
52. निर्दि॑ष्टभागो॒ वै वै निर्दि॑ष्टभागो॒ निर्दि॑ष्टभागो॒ वै । \newline
53. निर्दि॑ष्टभाग॒ इति॒ निर्दि॑ष्ट - भा॒गः॒ । \newline
54. वा ए॒तयो॑ रे॒तयो॒र् वै वा ए॒तयोः᳚ । \newline
55. ए॒तयो॑ र॒न्यो᳚ ऽन्य ए॒तयो॑ रे॒तयो॑ र॒न्यः । \newline
56. अ॒न्यो ऽनि॑र्दिष्टभा॒गो ऽनि॑र्दिष्टभागो॒ ऽन्यो᳚ ऽन्यो ऽनि॑र्दिष्टभागः । \newline
57. अनि॑र्दिष्टभागो॒ ऽन्यो᳚ ऽन्यो ऽनि॑र्दिष्टभा॒गो ऽनि॑र्दिष्टभागो॒ ऽन्यः । \newline
58. अनि॑र्दिष्टभाग॒ इत्यनि॑र्दिष्ट - भा॒गः॒ । \newline
59. अ॒न्य स्तौ ता व॒न्यो᳚ ऽन्य स्तौ । \newline
60. तौ सं॒भव॑न्तौ सं॒भव॑न्तौ॒ तौ तौ सं॒भव॑न्तौ । \newline
61. सं॒भव॑न्तौ॒ यज॑मानं॒ ॅयज॑मानꣳ सं॒भव॑न्तौ सं॒भव॑न्तौ॒ यज॑मानम् । \newline
62. सं॒भव॑न्ता॒विति॑ सं - भव॑न्तौ । \newline
63. यज॑मान म॒भ्य॑भि यज॑मानं॒ ॅयज॑मान म॒भि । \newline

\textbf{Ghana Paata } \newline

1. वा ए॒ष ए॒ष वै वा ए॒ष सि॑सीर्.षति सिसीर्.ष त्ये॒ष वै वा ए॒ष सि॑सीर्.षति । \newline
2. ए॒ष सि॑सीर्.षति सिसीर्.ष त्ये॒ष ए॒ष सि॑सीर्.षति॒ यो यः सि॑सीर्.ष त्ये॒ष ए॒ष सि॑सीर्.षति॒ यः । \newline
3. सि॒सी॒र्॒.ष॒ति॒ यो यः सि॑सीर्.षति सिसीर्.षति॒ यः स॑ङ्ग्रा॒मꣳ स॑ङ्ग्रा॒मं ॅयः सि॑सीर्.षति सिसीर्.षति॒ यः स॑ङ्ग्रा॒मम् । \newline
4. यः स॑ङ्ग्रा॒मꣳ स॑ङ्ग्रा॒मं ॅयो यः स॑ङ्ग्रा॒मम् जिगी॑षति॒ जिगी॑षति सङ्ग्रा॒मं ॅयो यः स॑ङ्ग्रा॒मम् जिगी॑षति । \newline
5. स॒ङ्ग्रा॒मम् जिगी॑षति॒ जिगी॑षति सङ्ग्रा॒मꣳ स॑ङ्ग्रा॒मम् जिगी॑ष त्य॒ग्नि र॒ग्निर् जिगी॑षति सङ्ग्रा॒मꣳ स॑ङ्ग्रा॒मम् जिगी॑ष त्य॒ग्निः । \newline
6. स॒ङ्ग्रा॒ममिति॑ सं - ग्रा॒मम् । \newline
7. जिगी॑ष त्य॒ग्नि र॒ग्निर् जिगी॑षति॒ जिगी॑ष त्य॒ग्निः खलु॒ खल्व॒ग्निर् जिगी॑षति॒ जिगी॑ष त्य॒ग्निः खलु॑ । \newline
8. अ॒ग्निः खलु॒ खल्व॒ग्नि र॒ग्निः खलु॒ वै वै खल्व॒ग्नि र॒ग्निः खलु॒ वै । \newline
9. खलु॒ वै वै खलु॒ खलु॒ वै दे॒वाना᳚म् दे॒वानां॒ ॅवै खलु॒ खलु॒ वै दे॒वाना᳚म् । \newline
10. वै दे॒वाना᳚म् दे॒वानां॒ ॅवै वै दे॒वानां᳚ ॅवाज॒सृद् वा॑ज॒सृद् दे॒वानां॒ ॅवै वै दे॒वानां᳚ ॅवाज॒सृत् । \newline
11. दे॒वानां᳚ ॅवाज॒सृद् वा॑ज॒सृद् दे॒वाना᳚म् दे॒वानां᳚ ॅवाज॒सृ द॒ग्नि म॒ग्निं ॅवा॑ज॒सृद् दे॒वाना᳚म् दे॒वानां᳚ ॅवाज॒सृ द॒ग्निम् । \newline
12. वा॒ज॒सृ द॒ग्नि म॒ग्निं ॅवा॑ज॒सृद् वा॑ज॒सृ द॒ग्नि मे॒वैवाग्निं ॅवा॑ज॒सृद् वा॑ज॒सृ द॒ग्नि मे॒व । \newline
13. वा॒ज॒सृदिति॑ वाज - सृत् । \newline
14. अ॒ग्नि मे॒वैवाग्नि म॒ग्नि मे॒व वा॑ज॒सृतं॑ ॅवाज॒सृत॑ मे॒वाग्नि म॒ग्नि मे॒व वा॑ज॒सृत᳚म् । \newline
15. ए॒व वा॑ज॒सृतं॑ ॅवाज॒सृत॑ मे॒वैव वा॑ज॒सृतꣳ॒॒ स्वेन॒ स्वेन॑ वाज॒सृत॑ मे॒वैव वा॑ज॒सृतꣳ॒॒ स्वेन॑ । \newline
16. वा॒ज॒सृतꣳ॒॒ स्वेन॒ स्वेन॑ वाज॒सृतं॑ ॅवाज॒सृतꣳ॒॒ स्वेन॑ भाग॒धेये॑न भाग॒धेये॑न॒ स्वेन॑ वाज॒सृतं॑ ॅवाज॒सृतꣳ॒॒ स्वेन॑ भाग॒धेये॑न । \newline
17. वा॒ज॒सृत॒मिति॑ वाज - सृत᳚म् । \newline
18. स्वेन॑ भाग॒धेये॑न भाग॒धेये॑न॒ स्वेन॒ स्वेन॑ भाग॒धेये॒नोपोप॑ भाग॒धेये॑न॒ स्वेन॒ स्वेन॑ भाग॒धेये॒नोप॑ । \newline
19. भा॒ग॒धेये॒नोपोप॑ भाग॒धेये॑न भाग॒धेये॒नोप॑ धावति धाव॒त्युप॑ भाग॒धेये॑न भाग॒धेये॒नोप॑ धावति । \newline
20. भा॒ग॒धेये॒नेति॑ भाग - धेये॑न । \newline
21. उप॑ धावति धाव॒ त्युपोप॑ धावति॒ धाव॑ति॒ धाव॑ति धाव॒ त्युपोप॑ धावति॒ धाव॑ति । \newline
22. धा॒व॒ति॒ धाव॑ति॒ धाव॑ति धावति धावति॒ धाव॑ति॒ वाजं॒ ॅवाज॒म् धाव॑ति धावति धावति॒ धाव॑ति॒ वाज᳚म् । \newline
23. धाव॑ति॒ वाजं॒ ॅवाज॒म् धाव॑ति॒ धाव॑ति॒ वाजꣳ॒॒ हन्ति॒ हन्ति॒ वाज॒म् धाव॑ति॒ धाव॑ति॒ वाजꣳ॒॒ हन्ति॑ । \newline
24. वाजꣳ॒॒ हन्ति॒ हन्ति॒ वाजं॒ ॅवाजꣳ॒॒ हन्ति॑ वृ॒त्रं ॅवृ॒त्रꣳ हन्ति॒ वाजं॒ ॅवाजꣳ॒॒ हन्ति॑ वृ॒त्रम् । \newline
25. हन्ति॑ वृ॒त्रं ॅवृ॒त्रꣳ हन्ति॒ हन्ति॑ वृ॒त्रम् जय॑ति॒ जय॑ति वृ॒त्रꣳ हन्ति॒ हन्ति॑ वृ॒त्रम् जय॑ति । \newline
26. वृ॒त्रम् जय॑ति॒ जय॑ति वृ॒त्रं ॅवृ॒त्रम् जय॑ति॒ तम् तम् जय॑ति वृ॒त्रं ॅवृ॒त्रम् जय॑ति॒ तम् । \newline
27. जय॑ति॒ तम् तम् जय॑ति॒ जय॑ति॒ तꣳ स॑ङ्ग्रा॒मꣳ स॑ङ्ग्रा॒मम् तम् जय॑ति॒ जय॑ति॒ तꣳ स॑ङ्ग्रा॒मम् । \newline
28. तꣳ स॑ङ्ग्रा॒मꣳ स॑ङ्ग्रा॒मम् तम् तꣳ स॑ङ्ग्रा॒म मथो॒ अथो॑ सङ्ग्रा॒मम् तम् तꣳ स॑ङ्ग्रा॒म मथो᳚ । \newline
29. स॒ङ्ग्रा॒म मथो॒ अथो॑ सङ्ग्रा॒मꣳ स॑ङ्ग्रा॒म मथो॑ अ॒ग्नि र॒ग्नि रथो॑ सङ्ग्रा॒मꣳ स॑ङ्ग्रा॒म मथो॑ अ॒ग्निः । \newline
30. स॒ङ्ग्रा॒ममिति॑ सं - ग्रा॒मम् । \newline
31. अथो॑ अ॒ग्नि र॒ग्नि रथो॒ अथो॑ अ॒ग्नि रि॑वे वा॒ग्नि रथो॒ अथो॑ अ॒ग्नि रि॑व । \newline
32. अथो॒ इत्यथो᳚ । \newline
33. अ॒ग्नि रि॑वे वा॒ग्नि र॒ग्नि रि॑व॒ न ने वा॒ग्नि र॒ग्नि रि॑व॒ न । \newline
34. इ॒व॒ न ने वे॑ व॒ न प्र॑ति॒धृषे᳚ प्रति॒धृषे॒ ने वे॑ व॒ न प्र॑ति॒धृषे᳚ । \newline
35. न प्र॑ति॒धृषे᳚ प्रति॒धृषे॒ न न प्र॑ति॒धृषे॑ भवति भवति प्रति॒धृषे॒ न न प्र॑ति॒धृषे॑ भवति । \newline
36. प्र॒ति॒धृषे॑ भवति भवति प्रति॒धृषे᳚ प्रति॒धृषे॑ भव त्य॒ग्नये॒ ऽग्नये॑ भवति प्रति॒धृषे᳚ प्रति॒धृषे॑ भव त्य॒ग्नये᳚ । \newline
37. प्र॒ति॒धृष॒ इति॑ प्रति - धृषे᳚ । \newline
38. भ॒व॒ त्य॒ग्नये॒ ऽग्नये॑ भवति भव त्य॒ग्नये᳚ ऽग्नि॒वते᳚ ऽग्नि॒वते॒ ऽग्नये॑ भवति भव त्य॒ग्नये᳚ ऽग्नि॒वते᳚ । \newline
39. अ॒ग्नये᳚ ऽग्नि॒वते᳚ ऽग्नि॒वते॒ ऽग्नये॒ ऽग्नये᳚ ऽग्नि॒वते॑ पुरो॒डाश॑म् पुरो॒डाश॑ मग्नि॒वते॒ ऽग्नये॒ ऽग्नये᳚ ऽग्नि॒वते॑ पुरो॒डाश᳚म् । \newline
40. अ॒ग्नि॒वते॑ पुरो॒डाश॑म् पुरो॒डाश॑ मग्नि॒वते᳚ ऽग्नि॒वते॑ पुरो॒डाश॑ म॒ष्टाक॑पाल म॒ष्टाक॑पालम् पुरो॒डाश॑ मग्नि॒वते᳚ ऽग्नि॒वते॑ पुरो॒डाश॑ म॒ष्टाक॑पालम् । \newline
41. अ॒ग्नि॒वत॒ इत्य॑ग्नि - वते᳚ । \newline
42. पु॒रो॒डाश॑ म॒ष्टाक॑पाल म॒ष्टाक॑पालम् पुरो॒डाश॑म् पुरो॒डाश॑ म॒ष्टाक॑पाल॒म् निर् णिर॒ष्टाक॑पालम् पुरो॒डाश॑म् पुरो॒डाश॑ म॒ष्टाक॑पाल॒म् निः । \newline
43. अ॒ष्टाक॑पाल॒म् निर् णिर॒ष्टाक॑पाल म॒ष्टाक॑पाल॒म् निर् व॑पेद् वपे॒न् निर॒ष्टाक॑पाल म॒ष्टाक॑पाल॒म् निर् व॑पेत् । \newline
44. अ॒ष्टाक॑पाल॒मित्य॒ष्टा - क॒पा॒ल॒म् । \newline
45. निर् व॑पेद् वपे॒न् निर् णिर् व॑पे॒द् यस्य॒ यस्य॑ वपे॒न् निर् णिर् व॑पे॒द् यस्य॑ । \newline
46. व॒पे॒द् यस्य॒ यस्य॑ वपेद् वपे॒द् यस्या॒ग्ना व॒ग्नौ यस्य॑ वपेद् वपे॒द् यस्या॒ग्नौ । \newline
47. यस्या॒ग्ना व॒ग्नौ यस्य॒ यस्या॒ग्ना व॒ग्नि म॒ग्नि म॒ग्नौ यस्य॒ यस्या॒ग्ना व॒ग्निम् । \newline
48. अ॒ग्ना व॒ग्नि म॒ग्नि म॒ग्ना व॒ग्ना व॒ग्नि म॑भ्यु॒द्धरे॑यु रभ्यु॒द्धरे॑यु र॒ग्नि म॒ग्ना व॒ग्ना व॒ग्नि म॑भ्यु॒द्धरे॑युः । \newline
49. अ॒ग्नि म॑भ्यु॒द्धरे॑यु रभ्यु॒द्धरे॑यु र॒ग्नि म॒ग्नि म॑भ्यु॒द्धरे॑यु॒र् निर्दि॑ष्टभागो॒ निर्दि॑ष्टभागो ऽभ्यु॒द्धरे॑यु र॒ग्नि म॒ग्नि म॑भ्यु॒द्धरे॑यु॒र् निर्दि॑ष्टभागः । \newline
50. अ॒भ्यु॒द्धरे॑यु॒र् निर्दि॑ष्टभागो॒ निर्दि॑ष्टभागो ऽभ्यु॒द्धरे॑यु रभ्यु॒द्धरे॑यु॒र् निर्दि॑ष्टभागो॒ वै वै निर्दि॑ष्टभागो ऽभ्यु॒द्धरे॑यु रभ्यु॒द्धरे॑यु॒र् निर्दि॑ष्टभागो॒ वै । \newline
51. अ॒भ्यु॒द्धरे॑यु॒रित्य॑भि - उ॒द्धरे॑युः । \newline
52. निर्दि॑ष्टभागो॒ वै वै निर्दि॑ष्टभागो॒ निर्दि॑ष्टभागो॒ वा ए॒तयो॑ रे॒तयो॒र् वै निर्दि॑ष्टभागो॒ निर्दि॑ष्टभागो॒ वा ए॒तयोः᳚ । \newline
53. निर्दि॑ष्टभाग॒ इति॒ निर्दि॑ष्ट - भा॒गः॒ । \newline
54. वा ए॒तयो॑ रे॒तयो॒र् वै वा ए॒तयो॑ र॒न्यो᳚ ऽन्य ए॒तयो॒र् वै वा ए॒तयो॑ र॒न्यः । \newline
55. ए॒तयो॑ र॒न्यो᳚ ऽन्य ए॒तयो॑ रे॒तयो॑ र॒न्यो ऽनि॑र्दिष्टभा॒गो ऽनि॑र्दिष्टभागो॒ ऽन्य ए॒तयो॑ रे॒तयो॑ र॒न्यो ऽनि॑र्दिष्टभागः । \newline
56. अ॒न्यो ऽनि॑र्दिष्टभा॒गो ऽनि॑र्दिष्टभागो॒ ऽन्यो᳚ ऽन्यो ऽनि॑र्दिष्टभागो॒ ऽन्यो᳚ ऽन्यो ऽनि॑र्दिष्टभागो॒ ऽन्यो᳚ ऽन्यो ऽनि॑र्दिष्टभागो॒ ऽन्यः । \newline
57. अनि॑र्दिष्टभागो॒ ऽन्यो᳚ ऽन्यो ऽनि॑र्दिष्टभा॒गो ऽनि॑र्दिष्टभागो॒ ऽन्यस्तौ ता व॒न्यो ऽनि॑र्दिष्टभा॒गो ऽनि॑र्दिष्टभागो॒ ऽन्यस्तौ । \newline
58. अनि॑र्दिष्टभाग॒ इत्यनि॑र्दिष्ट - भा॒गः॒ । \newline
59. अ॒न्यस्तौ ता व॒न्यो᳚ ऽन्यस्तौ सं॒भव॑न्तौ सं॒भव॑न्तौ॒ ता व॒न्यो᳚ ऽन्यस्तौ सं॒भव॑न्तौ । \newline
60. तौ सं॒भव॑न्तौ सं॒भव॑न्तौ॒ तौ तौ सं॒भव॑न्तौ॒ यज॑मानं॒ ॅयज॑मानꣳ सं॒भव॑न्तौ॒ तौ तौ सं॒भव॑न्तौ॒ यज॑मानम् । \newline
61. सं॒भव॑न्तौ॒ यज॑मानं॒ ॅयज॑मानꣳ सं॒भव॑न्तौ सं॒भव॑न्तौ॒ यज॑मान म॒भ्य॑भि यज॑मानꣳ सं॒भव॑न्तौ सं॒भव॑न्तौ॒ यज॑मान म॒भि । \newline
62. सं॒भव॑न्ता॒विति॑ सं - भव॑न्तौ । \newline
63. यज॑मान म॒भ्य॑भि यज॑मानं॒ ॅयज॑मान म॒भि सꣳ स म॒भि यज॑मानं॒ ॅयज॑मान म॒भि सम् । \newline
\pagebreak
\markright{ TS 2.2.4.7  \hfill https://www.vedavms.in \hfill}
\addcontentsline{toc}{section}{ TS 2.2.4.7 }
\section*{ TS 2.2.4.7 }

\textbf{TS 2.2.4.7 } \newline
\textbf{Samhita Paata} \newline

-म॒भि सं भ॑वतः॒ स ई᳚श्व॒र आर्ति॒मार्तो॒-र्यद॒ग्नये᳚ऽग्नि॒वते॑ नि॒र्वप॑ति भाग॒धेये॑नै॒वैनौ॑ शमयति॒ नाऽऽ*र्ति॒मार्च्छ॑ति॒ यज॑मानो॒ऽग्नये॒ ज्योति॑ष्मते पुरो॒डाश॑म॒ष्टाक॑पालं॒ निर्व॑पे॒द्-यस्या॒-ग्निरुद्धृ॒तोऽहु॑ते-ऽग्निहो॒त्र उ॒द्वाये॒दप॑र आ॒दीप्या॑नू॒द्धृत्य॒ इत्या॑हु॒स्तत् तथा॒ न का॒र्यं॑ ॅयद्-भा॑ग॒धेय॑म॒भि पूर्व॑ उद्ध्रि॒यते॒ किमप॑रो॒ऽभ्यु - [  ] \newline

\textbf{Pada Paata} \newline

अ॒भि । समिति॑ । भ॒व॒तः॒ । सः । ई॒श्व॒रः । आर्ति᳚म् । आर्तो॒रित्या - अ॒र्तोः॒ । यत् । अ॒ग्नये᳚ । अ॒ग्नि॒वत॒ इत्य॑ग्नि - वते᳚ । नि॒र्वप॒तीति॑ निः-वप॑ति । भा॒ग॒धेये॒नेति॑ भाग - धेये॑न । ए॒व । ए॒नौ॒ । श॒म॒य॒ति॒ । न । आर्ति᳚म् । एति॑ । ऋ॒च्छ॒ति॒ । यज॑मानः । अ॒ग्नये᳚ । ज्योति॑ष्मते । पु॒रो॒डाश᳚म् । अ॒ष्टाक॑पाल॒मित्य॒ष्टा - क॒पा॒ल॒म् । निरिति॑ । व॒पे॒त् । यस्य॑ । अ॒ग्निः । उद्धृ॑त॒ इत्युत् - हृ॒तः॒ । अहु॑ते । अ॒ग्नि॒हो॒त्र इत्य॑ग्नि - हो॒त्रे । उ॒द्वाये॒दित्यु॑त् - वाये᳚त् । अप॑रः । आ॒दीप्येत्या᳚ - दीप्य॑ । अ॒नू॒द्धृत्य॒ इत्य॑नु - उ॒द्धृत्यः॑ । इति॑ । आ॒हुः॒ । तत् । तथा᳚ । न । का॒र्य᳚म् । यत् । भा॒ग॒धेय॒मिति॑ भाग - धेय᳚म् । अ॒भीति॑ । पूर्वः॑ । उ॒द्ध्रि॒यत॒ इत्यु॑त् - ह्रि॒यते᳚ । किम् । अप॑रः । अ॒भि । उदिति॑ ।  \newline


\textbf{Krama Paata} \newline

अ॒भि सम् । सं भ॑वतः । भ॒व॒तः॒ सः । स ई᳚श्व॒रः । ई॒श्व॒र आर्ति᳚म् । आर्ति॒मार्तोः᳚ । आर्तो॒र् यत् । आर्तो॒रित्या - अ॒र्तोः॒ । यद॒ग्नये᳚ । अ॒ग्नये᳚ ऽग्नि॒वते᳚ । अ॒ग्नि॒वते॑ नि॒र्वप॑ति । अ॒ग्नि॒वत॒ इत्य॑ग्नि - वते᳚ । नि॒र्वप॑ति भाग॒धेये॑न । नि॒र्वप॒तीति॑ निः - वप॑ति । भा॒ग॒धेये॑नै॒व । भा॒ग॒धेये॒नेति॑ भाग - धेये॑न । ए॒वैनौ᳚ । ए॒नौ॒ श॒म॒य॒ति॒ । श॒म॒य॒ति॒ न । नार्ति᳚म् । आर्ति॒मा । आर्च्छ॑ति । ऋ॒च्छ॒ति॒ यज॑मानः । यज॑मानो॒ ऽग्नये᳚ । अ॒ग्नये॒ ज्योति॑ष्मते । ज्योति॑ष्मते पुरो॒डाश᳚म् । पु॒रो॒डाश॑म॒ष्टाक॑पालम् । अ॒ष्टाक॑पाल॒म् निः । अ॒ष्टाक॑पाल॒मित्य॒ष्टा - क॒पा॒ल॒म् । निर् व॑पेत् । व॒पे॒द्यस्य॑ । यस्या॒ग्निः । अ॒ग्निरुद्धृ॑तः । उद्धृ॒तो ऽहु॑ते । उद्धृ॑त॒ इत्युत् - हृ॒तः॒ । अहु॑ते ऽग्निहो॒त्रे । अ॒ग्नि॒हो॒त्र उ॒द्वाये᳚त् । अ॒ग्नि॒हो॒त्र इत्य॑ग्नि - हो॒त्रे । उ॒द्वाये॒दप॑रः । उ॒द्वाये॒दित्यु॑त् - वाये᳚त् । अप॑र आ॒दीप्य॑ । आ॒दीप्या॑नू॒द्धृत्यः॑ । आ॒दीप्येत्या᳚ - दीप्य॑ । अ॒नू॒द्धृत्य॒ इति॑ । अ॒नू॒द्धृत्य॒ इत्य॑नु - उ॒द्धृत्यः॑ । इत्या॑हुः । आ॒हु॒स्तत् । तत्,तथा᳚ । तथा॒ न । न का॒र्य᳚म् । का॒र्यं॑ ॅयत् । यद् भा॑ग॒धेय᳚म् । भा॒ग॒धेय॑म॒भि । भा॒ग॒धेय॒मिति॑ भाग - धेय᳚म् । अ॒भि पूर्वः॑ । पूर्व॑ उद्ध्रि॒यते᳚ । उ॒द्ध्रि॒यते॒ किम् । उ॒द्ध्रि॒यत॒ इत्यु॑त् - ह्रि॒यते᳚ । किमप॑रः । अप॑रो॒ऽभि । अ॒भ्युत् । उद्ध्रि॑येत \newline

\textbf{Jatai Paata} \newline

1. अ॒भि सꣳ स म॒भ्य॑भि सम् । \newline
2. सम् भ॑वतो भवतः॒ सꣳ सम् भ॑वतः । \newline
3. भ॒व॒तः॒ स स भ॑वतो भवतः॒ सः । \newline
4. स ई᳚श्व॒र ई᳚श्व॒रः स स ई᳚श्व॒रः । \newline
5. ई॒श्व॒र आर्ति॒ मार्ति॑ मीश्व॒र ई᳚श्व॒र आर्ति᳚म् । \newline
6. आर्ति॒ मार्तो॒ रार्तो॒ रार्ति॒ मार्ति॒ मार्तोः᳚ । \newline
7. आर्तो॒र् यद् यदार्तो॒ रार्तो॒र् यत् । \newline
8. आर्तो॒रित्या - अ॒र्तोः॒ । \newline
9. यद॒ग्नये॒ ऽग्नये॒ यद् यद॒ग्नये᳚ । \newline
10. अ॒ग्नये᳚ ऽग्नि॒वते᳚ ऽग्नि॒वते॒ ऽग्नये॒ ऽग्नये᳚ ऽग्नि॒वते᳚ । \newline
11. अ॒ग्नि॒वते॑ नि॒र्वप॑ति नि॒र्वप॑ त्यग्नि॒वते᳚ ऽग्नि॒वते॑ नि॒र्वप॑ति । \newline
12. अ॒ग्नि॒वत॒ इत्य॑ग्नि - वते᳚ । \newline
13. नि॒र्वप॑ति भाग॒धेये॑न भाग॒धेये॑न नि॒र्वप॑ति नि॒र्वप॑ति भाग॒धेये॑न । \newline
14. नि॒र्वप॒तीति॑ निः - वप॑ति । \newline
15. भा॒ग॒धेये॑नै॒वैव भा॑ग॒धेये॑न भाग॒धेये॑नै॒व । \newline
16. भा॒ग॒धेये॒नेति॑ भाग - धेये॑न । \newline
17. ए॒वैना॑ वेना वे॒वैवैनौ᳚ । \newline
18. ए॒नौ॒ श॒म॒य॒ति॒ श॒म॒य॒ त्ये॒ना॒ वे॒नौ॒ श॒म॒य॒ति॒ । \newline
19. श॒म॒य॒ति॒ न न श॑मयति शमयति॒ न । \newline
20. नार्ति॒ मार्ति॒म् न नार्ति᳚म् । \newline
21. आर्ति॒ मा ऽऽर्ति॒ मार्ति॒ मा । \newline
22. आर्च्छ॑ त्यृच्छ॒ त्यार्च्छ॑ति । \newline
23. ऋ॒च्छ॒ति॒ यज॑मानो॒ यज॑मान ऋच्छ त्यृच्छति॒ यज॑मानः । \newline
24. यज॑मानो॒ ऽग्नये॒ ऽग्नये॒ यज॑मानो॒ यज॑मानो॒ ऽग्नये᳚ । \newline
25. अ॒ग्नये॒ ज्योति॑ष्मते॒ ज्योति॑ष्मते॒ ऽग्नये॒ ऽग्नये॒ ज्योति॑ष्मते । \newline
26. ज्योति॑ष्मते पुरो॒डाश॑म् पुरो॒डाश॒म् ज्योति॑ष्मते॒ ज्योति॑ष्मते पुरो॒डाश᳚म् । \newline
27. पु॒रो॒डाश॑ म॒ष्टाक॑पाल म॒ष्टाक॑पालम् पुरो॒डाश॑म् पुरो॒डाश॑ म॒ष्टाक॑पालम् । \newline
28. अ॒ष्टाक॑पाल॒म् निर् णिर॒ष्टाक॑पाल म॒ष्टाक॑पाल॒म् निः । \newline
29. अ॒ष्टाक॑पाल॒मित्य॒ष्टा - क॒पा॒ल॒म् । \newline
30. निर् व॑पेद् वपे॒न् निर् णिर् व॑पेत् । \newline
31. व॒पे॒द् यस्य॒ यस्य॑ वपेद् वपे॒द् यस्य॑ । \newline
32. यस्या॒ग्नि र॒ग्निर् यस्य॒ यस्या॒ग्निः । \newline
33. अ॒ग्नि रुद्धृ॑त॒ उद्धृ॑तो॒ ऽग्नि र॒ग्नि रुद्धृ॑तः । \newline
34. उद्धृ॒तो ऽहु॒ते ऽहु॑त॒ उद्धृ॑त॒ उद्धृ॒तो ऽहु॑ते । \newline
35. उद्धृ॑त॒ इत्युत् - हृ॒तः॒ । \newline
36. अहु॑ते ऽग्निहो॒त्रे᳚ ऽग्निहो॒त्रे ऽहु॒ते ऽहु॑ते ऽग्निहो॒त्रे । \newline
37. अ॒ग्नि॒हो॒त्र उ॒द्वाये॑ दु॒द्वाये॑ दग्निहो॒त्रे᳚ ऽग्निहो॒त्र उ॒द्वाये᳚त् । \newline
38. अ॒ग्नि॒हो॒त्र इत्य॑ग्नि - हो॒त्रे । \newline
39. उ॒द्वाये॒ दप॒रो ऽप॑र उ॒द्वाये॑ दु॒द्वाये॒ दप॑रः । \newline
40. उ॒द्वाये॒दित्यु॑त् - वाये᳚त् । \newline
41. अप॑र आ॒दीप्या॒ दीप्याप॒रो ऽप॑र आ॒दीप्य॑ । \newline
42. आ॒दीप्या॑ नू॒द्धृत्यो॑ ऽनू॒द्धृत्य॑ आ॒दीप्या॒ दीप्या॑ नू॒द्धृत्यः॑ । \newline
43. आ॒दीप्येत्या᳚ - दीप्य॑ । \newline
44. अ॒नू॒द्धृत्य॒ इती त्य॑नू॒द्धृत्यो॑ ऽनू॒द्धृत्य॒ इति॑ । \newline
45. अ॒नू॒द्धृत्य॒ इत्य॑नु - उ॒द्धृत्यः॑ । \newline
46. इत्या॑हु राहु॒ रिती त्या॑हुः । \newline
47. आ॒हु॒ स्तत् तदा॑हु राहु॒ स्तत् । \newline
48. तत् तथा॒ तथा॒ तत् तत् तथा᳚ । \newline
49. तथा॒ न न तथा॒ तथा॒ न । \newline
50. न का॒र्य॑म् का॒र्य॑म् न न का॒र्य᳚म् । \newline
51. का॒र्यं॑ ॅयद् यत् का॒र्य॑म् का॒र्यं॑ ॅयत् । \newline
52. यद् भा॑ग॒धेय॑म् भाग॒धेयं॒ ॅयद् यद् भा॑ग॒धेय᳚म् । \newline
53. भा॒ग॒धेय॑ म॒भ्य॑भि भा॑ग॒धेय॑म् भाग॒धेय॑ म॒भि । \newline
54. भा॒ग॒धेय॒मिति॑ भाग - धेय᳚म् । \newline
55. अ॒भि पूर्वः॒ पूर्वो॒ ऽभ्य॑भि पूर्वः॑ । \newline
56. पूर्व॑ उद्ध्रि॒यत॑ उद्ध्रि॒यते॒ पूर्वः॒ पूर्व॑ उद्ध्रि॒यते᳚ । \newline
57. उ॒द्ध्रि॒यते॒ किम् कि मु॑द्ध्रि॒यत॑ उद्ध्रि॒यते॒ किम् । \newline
58. उ॒द्ध्रि॒यत॒ इत्यु॑त् - ह्रि॒यते᳚ । \newline
59. कि मप॒रो ऽप॑रः॒ किम् कि मप॑रः । \newline
60. अप॑रो॒ ऽभ्य॑भ्यप॒रो ऽप॑रो॒ ऽभि । \newline
61. अ॒भ्यु दुद॒भ्य॑भ्युत् । \newline
62. उद्ध्रि॑येत ह्रिये॒ तोदु द्ध्रि॑येत । \newline

\textbf{Ghana Paata } \newline

1. अ॒भि सꣳ स म॒भ्य॑भि सम् भ॑वतो भवतः॒ स म॒भ्य॑भि सम् भ॑वतः । \newline
2. सम् भ॑वतो भवतः॒ सꣳ सम् भ॑वतः॒ स स भ॑वतः॒ सꣳ सम् भ॑वतः॒ सः । \newline
3. भ॒व॒तः॒ स स भ॑वतो भवतः॒ स ई᳚श्व॒र ई᳚श्व॒रः स भ॑वतो भवतः॒ स ई᳚श्व॒रः । \newline
4. स ई᳚श्व॒र ई᳚श्व॒रः स स ई᳚श्व॒र आर्ति॒ मार्ति॑ मीश्व॒रः स स ई᳚श्व॒र आर्ति᳚म् । \newline
5. ई॒श्व॒र आर्ति॒ मार्ति॑ मीश्व॒र ई᳚श्व॒र आर्ति॒ मार्तो॒ रार्तो॒ रार्ति॑ मीश्व॒र ई᳚श्व॒र आर्ति॒ मार्तोः᳚ । \newline
6. आर्ति॒ मार्तो॒ रार्तो॒ रार्ति॒ मार्ति॒ मार्तो॒र् यद् यदार्तो॒ रार्ति॒ मार्ति॒ मार्तो॒र् यत् । \newline
7. आर्तो॒र् यद् यदार्तो॒ रार्तो॒र् यद॒ग्नये॒ ऽग्नये॒ यदार्तो॒ रार्तो॒र् यद॒ग्नये᳚ । \newline
8. आर्तो॒रित्या - अ॒र्तोः॒ । \newline
9. यद॒ग्नये॒ ऽग्नये॒ यद् यद॒ग्नये᳚ ऽग्नि॒वते᳚ ऽग्नि॒वते॒ ऽग्नये॒ यद् यद॒ग्नये᳚ ऽग्नि॒वते᳚ । \newline
10. अ॒ग्नये᳚ ऽग्नि॒वते᳚ ऽग्नि॒वते॒ ऽग्नये॒ ऽग्नये᳚ ऽग्नि॒वते॑ नि॒र्वप॑ति नि॒र्वप॑ त्यग्नि॒वते॒ ऽग्नये॒ ऽग्नये᳚ ऽग्नि॒वते॑ नि॒र्वप॑ति । \newline
11. अ॒ग्नि॒वते॑ नि॒र्वप॑ति नि॒र्वप॑ त्यग्नि॒वते᳚ ऽग्नि॒वते॑ नि॒र्वप॑ति भाग॒धेये॑न भाग॒धेये॑न नि॒र्वप॑ त्यग्नि॒वते᳚ ऽग्नि॒वते॑ नि॒र्वप॑ति भाग॒धेये॑न । \newline
12. अ॒ग्नि॒वत॒ इत्य॑ग्नि - वते᳚ । \newline
13. नि॒र्वप॑ति भाग॒धेये॑न भाग॒धेये॑न नि॒र्वप॑ति नि॒र्वप॑ति भाग॒धेये॑नै॒वैव भा॑ग॒धेये॑न नि॒र्वप॑ति नि॒र्वप॑ति भाग॒धेये॑नै॒व । \newline
14. नि॒र्वप॒तीति॑ निः - वप॑ति । \newline
15. भा॒ग॒धेये॑नै॒वैव भा॑ग॒धेये॑न भाग॒धेये॑नै॒वैना॑ वेना वे॒व भा॑ग॒धेये॑न भाग॒धेये॑नै॒वैनौ᳚ । \newline
16. भा॒ग॒धेये॒नेति॑ भाग - धेये॑न । \newline
17. ए॒वैना॑ वेना वे॒वैवैनौ॑ शमयति शमय त्येना वे॒वैवैनौ॑ शमयति । \newline
18. ए॒नौ॒ श॒म॒य॒ति॒ श॒म॒य॒ त्ये॒ना॒ वे॒नौ॒ श॒म॒य॒ति॒ न न श॑मय त्येना वेनौ शमयति॒ न । \newline
19. श॒म॒य॒ति॒ न न श॑मयति शमयति॒ नार्ति॒ मार्ति॒म् न श॑मयति शमयति॒ नार्ति᳚म् । \newline
20. नार्ति॒ मार्ति॒म् न नार्ति॒ मा ऽऽर्ति॒म् न नार्ति॒ मा । \newline
21. आर्ति॒ मा ऽऽर्ति॒ मार्ति॒ मार्च्छ॑ त्यृच्छ॒त्या ऽऽर्ति॒ मार्ति॒ मार्च्छ॑ति । \newline
22. आर्च्छ॑ त्यृच्छ॒ त्यार्च्छ॑ति॒ यज॑मानो॒ यज॑मान ऋच्छ॒ त्यार्च्छ॑ति॒ यज॑मानः । \newline
23. ऋ॒च्छ॒ति॒ यज॑मानो॒ यज॑मान ऋच्छ त्यृच्छति॒ यज॑मानो॒ ऽग्नये॒ ऽग्नये॒ यज॑मान ऋच्छ त्यृच्छति॒ यज॑मानो॒ ऽग्नये᳚ । \newline
24. यज॑मानो॒ ऽग्नये॒ ऽग्नये॒ यज॑मानो॒ यज॑मानो॒ ऽग्नये॒ ज्योति॑ष्मते॒ ज्योति॑ष्मते॒ ऽग्नये॒ यज॑मानो॒ यज॑मानो॒ ऽग्नये॒ ज्योति॑ष्मते । \newline
25. अ॒ग्नये॒ ज्योति॑ष्मते॒ ज्योति॑ष्मते॒ ऽग्नये॒ ऽग्नये॒ ज्योति॑ष्मते पुरो॒डाश॑म् पुरो॒डाश॒म् ज्योति॑ष्मते॒ ऽग्नये॒ ऽग्नये॒ ज्योति॑ष्मते पुरो॒डाश᳚म् । \newline
26. ज्योति॑ष्मते पुरो॒डाश॑म् पुरो॒डाश॒म् ज्योति॑ष्मते॒ ज्योति॑ष्मते पुरो॒डाश॑ म॒ष्टाक॑पाल म॒ष्टाक॑पालम् पुरो॒डाश॒म् ज्योति॑ष्मते॒ ज्योति॑ष्मते पुरो॒डाश॑ म॒ष्टाक॑पालम् । \newline
27. पु॒रो॒डाश॑ म॒ष्टाक॑पाल म॒ष्टाक॑पालम् पुरो॒डाश॑म् पुरो॒डाश॑ म॒ष्टाक॑पाल॒म् निर् णिर॒ष्टाक॑पालम् पुरो॒डाश॑म् पुरो॒डाश॑ म॒ष्टाक॑पाल॒म् निः । \newline
28. अ॒ष्टाक॑पाल॒म् निर् णिर॒ष्टाक॑पाल म॒ष्टाक॑पाल॒म् निर् व॑पेद् वपे॒न् निर॒ष्टाक॑पाल म॒ष्टाक॑पाल॒म् निर् व॑पेत् । \newline
29. अ॒ष्टाक॑पाल॒मित्य॒ष्टा - क॒पा॒ल॒म् । \newline
30. निर् व॑पेद् वपे॒न् निर् णिर् व॑पे॒द् यस्य॒ यस्य॑ वपे॒न् निर् णिर् व॑पे॒द् यस्य॑ । \newline
31. व॒पे॒द् यस्य॒ यस्य॑ वपेद् वपे॒द् यस्या॒ग्नि र॒ग्निर् यस्य॑ वपेद् वपे॒द् यस्या॒ग्निः । \newline
32. यस्या॒ग्नि र॒ग्निर् यस्य॒ यस्या॒ग्नि रुद्धृ॑त॒ उद्धृ॑तो॒ ऽग्निर् यस्य॒ यस्या॒ग्नि रुद्धृ॑तः । \newline
33. अ॒ग्नि रुद्धृ॑त॒ उद्धृ॑तो॒ ऽग्नि र॒ग्नि रुद्धृ॒तो ऽहु॒ते ऽहु॑त॒ उद्धृ॑तो॒ ऽग्नि र॒ग्नि रुद्धृ॒तो ऽहु॑ते । \newline
34. उद्धृ॒तो ऽहु॒ते ऽहु॑त॒ उद्धृ॑त॒ उद्धृ॒तो ऽहु॑ते ऽग्निहो॒त्रे᳚ ऽग्निहो॒त्रे ऽहु॑त॒ उद्धृ॑त॒ उद्धृ॒तो ऽहु॑ते ऽग्निहो॒त्रे । \newline
35. उद्धृ॑त॒ इत्युत् - हृ॒तः॒ । \newline
36. अहु॑ते ऽग्निहो॒त्रे᳚ ऽग्निहो॒त्रे ऽहु॒ते ऽहु॑ते ऽग्निहो॒त्र उ॒द्वाये॑ दु॒द्वाये॑ दग्निहो॒त्रे ऽहु॒ते ऽहु॑ते ऽग्निहो॒त्र उ॒द्वाये᳚त् । \newline
37. अ॒ग्नि॒हो॒त्र उ॒द्वाये॑ दु॒द्वाये॑ दग्निहो॒त्रे᳚ ऽग्निहो॒त्र उ॒द्वाये॒ दप॒रो ऽप॑र उ॒द्वाये॑ दग्निहो॒त्रे᳚ ऽग्निहो॒त्र उ॒द्वाये॒ दप॑रः । \newline
38. अ॒ग्नि॒हो॒त्र इत्य॑ग्नि - हो॒त्रे । \newline
39. उ॒द्वाये॒ दप॒रो ऽप॑र उ॒द्वाये॑ दु॒द्वाये॒ दप॑र आ॒दीप्या॒ दीप्याप॑र उ॒द्वाये॑ दु॒द्वाये॒ दप॑र आ॒दीप्य॑ । \newline
40. उ॒द्वाये॒दित्यु॑त् - वाये᳚त् । \newline
41. अप॑र आ॒दीप्या॒ दीप्याप॒रो ऽप॑र आ॒दीप्या॑ नू॒द्धृत्यो॑ ऽनू॒द्धृत्य॑ आ॒दीप्याप॒रो ऽप॑र आ॒दीप्या॑ नू॒द्धृत्यः॑ । \newline
42. आ॒दीप्या॑ नू॒द्धृत्यो॑ ऽनू॒द्धृत्य॑ आ॒दी प्या॒दीप्या॑ नू॒द्धृत्य॒ इती त्य॑नू॒द्धृत्य॑ आ॒दीप्या॒दीप्या॑ नू॒द्धृत्य॒ इति॑ । \newline
43. आ॒दीप्येत्या᳚ - दीप्य॑ । \newline
44. अ॒नू॒द्धृत्य॒ इती त्य॑नू॒द्धृत्यो॑ ऽनू॒द्धृत्य॒ इत्या॑हु राहु॒ रित्य॑नू॒द्धृत्यो॑ ऽनू॒द्धृत्य॒ इत्या॑हुः । \newline
45. अ॒नू॒द्धृत्य॒ इत्य॑नु - उ॒द्धृत्यः॑ । \newline
46. इत्या॑हु राहु॒ रितीत्या॑हु॒ स्तत् तदा॑हु॒ रिती त्या॑हु॒ स्तत् । \newline
47. आ॒हु॒ स्तत् तदा॑हु राहु॒ स्तत् तथा॒ तथा॒ तदा॑हु राहु॒ स्तत् तथा᳚ । \newline
48. तत् तथा॒ तथा॒ तत् तत् तथा॒ न न तथा॒ तत् तत् तथा॒ न । \newline
49. तथा॒ न न तथा॒ तथा॒ न का॒र्य॑म् का॒र्य॑म् न तथा॒ तथा॒ न का॒र्य᳚म् । \newline
50. न का॒र्य॑म् का॒र्य॑म् न न का॒र्यं॑ ॅयद् यत् का॒र्य॑म् न न का॒र्यं॑ ॅयत् । \newline
51. का॒र्यं॑ ॅयद् यत् का॒र्य॑म् का॒र्यं॑ ॅयद् भा॑ग॒धेय॑म् भाग॒धेयं॒ ॅयत् का॒र्य॑म् का॒र्यं॑ ॅयद् भा॑ग॒धेय᳚म् । \newline
52. यद् भा॑ग॒धेय॑म् भाग॒धेयं॒ ॅयद् यद् भा॑ग॒धेय॑ म॒भ्य॑भि भा॑ग॒धेयं॒ ॅयद् यद् भा॑ग॒धेय॑ म॒भि । \newline
53. भा॒ग॒धेय॑ म॒भ्य॑भि भा॑ग॒धेय॑म् भाग॒धेय॑ म॒भि पूर्वः॒ पूर्वो॒ ऽभि भा॑ग॒धेय॑म् भाग॒धेय॑ म॒भि पूर्वः॑ । \newline
54. भा॒ग॒धेय॒मिति॑ भाग - धेय᳚म् । \newline
55. अ॒भि पूर्वः॒ पूर्वो॒ ऽभ्य॑भि पूर्व॑ उद्ध्रि॒यत॑ उद्ध्रि॒यते॒ पूर्वो॒ ऽभ्य॑भि पूर्व॑ उद्ध्रि॒यते᳚ । \newline
56. पूर्व॑ उद्ध्रि॒यत॑ उद्ध्रि॒यते॒ पूर्वः॒ पूर्व॑ उद्ध्रि॒यते॒ किम् कि मु॑द्ध्रि॒यते॒ पूर्वः॒ पूर्व॑ उद्ध्रि॒यते॒ किम् । \newline
57. उ॒द्ध्रि॒यते॒ किम् कि मु॑द्ध्रि॒यत॑ उद्ध्रि॒यते॒ कि मप॒रो ऽप॑रः॒ कि मु॑द्ध्रि॒यत॑ उद्ध्रि॒यते॒ कि मप॑रः । \newline
58. उ॒द्ध्रि॒यत॒ इत्यु॑त् - ह्रि॒यते᳚ । \newline
59. कि मप॒रो ऽप॑रः॒ किम् कि मप॑रो॒ ऽभ्य॑भ्यप॑रः॒ किम् कि मप॑रो॒ ऽभि । \newline
60. अप॑रो॒ ऽभ्य॑भ्यप॒रो ऽप॑रो॒ ऽभ्यु दु द॒भ्यप॒रो ऽप॑रो॒ ऽभ्युत् । \newline
61. अ॒भ्यु दु द॒भ्य॑ भ्युद्ध्रि॑येत ह्रिये॒तो द॒भ्य॑ भ्युद्ध्रि॑येत । \newline
62. उद्ध्रि॑येत ह्रिये॒तो दुद्ध्रि॑ये॒ते तीति॑ ह्रिये॒तो दुद्ध्रि॑ये॒ते ति॑ । \newline
\pagebreak
\markright{ TS 2.2.4.8  \hfill https://www.vedavms.in \hfill}
\addcontentsline{toc}{section}{ TS 2.2.4.8 }
\section*{ TS 2.2.4.8 }

\textbf{TS 2.2.4.8 } \newline
\textbf{Samhita Paata} \newline

-द्ध्रि॑ये॒तेति॒ तान्ये॒वा व॒क्षाणा॑नि सन्नि॒धाय॑ मन्थेदि॒तः प्र॑थ॒मं ज॑ज्ञे अ॒ग्निः स्वाद्योने॒रधि॑ जा॒तवे॑दाः । स गा॑यत्रि॒या त्रि॒ष्टुभा॒ जग॑त्या दे॒वेभ्यो॑ ह॒व्यं ॅव॑हतु प्रजा॒नन्निति॒ छन्दो॑भिरे॒वैनꣳ॒॒ स्वाद्योनेः॒ प्रज॑नयत्ये॒ष वा व सो᳚ऽग्निरित्या॑हु॒ र्ज्योति॒स्त्वा अ॑स्य॒ परा॑पतित॒मिति॒ यद॒ग्नये॒ ज्योति॑ष्मते नि॒र्वप॑ति॒ यदे॒वास्य॒ ( ) ज्योतिः॒ परा॑पतितं॒ तदे॒वाव॑ रुन्धे ॥ \newline

\textbf{Pada Paata} \newline

ह्रि॒ये॒त॒ । इति॑ । तानि॑ । ए॒व । अ॒व॒क्षाणा॒नीत्य॑व - क्षाणा॑नि । स॒न्नि॒धायेति॑ सं - नि॒धाय॑ । म॒न्थे॒त् । इ॒तः । प्र॒थ॒मम् । ज॒ज्ञे॒ । अ॒ग्निः । स्वात् । योनेः᳚ । अधीति॑ । जा॒तवे॑दा॒ इति॑ जा॒त - वे॒दाः॒ ॥ सः । गा॒य॒त्रि॒या । त्रि॒ष्टुभा᳚ । जग॑त्या । दे॒वेभ्यः॑ । ह॒व्यम् । व॒ह॒तु॒ । प्र॒जा॒नन्निति॑ प्र - जा॒नन्न् । इति॑ । छन्दो॑भि॒रिति॒ छन्दः॑ - भिः॒ । ए॒व । ए॒न॒म् । स्वात् । योनेः᳚ । प्रेति॑ । ज॒न॒य॒ति॒ । ए॒षः । वाव । सः । अ॒ग्निः । इति॑ । आ॒हुः॒ । ज्योतिः॑ । तु । वै । अ॒स्य॒ । परा॑पतित॒मिति॒ परा᳚ - प॒ति॒त॒म् । इति॑ । यत् । अ॒ग्नये᳚ । ज्योति॑ष्मते । नि॒र्वप॒तीति॑ निः - वप॑ति । यत् । ए॒व । अ॒स्य॒ ( ) । ज्योतिः॑ । परा॑पतित॒मिति॒ परा᳚ - प॒ति॒त॒म् । तत् । ए॒व । अवेति॑ । रु॒न्धे॒ ॥  \newline


\textbf{Krama Paata} \newline

ह्रि॒ये॒तेति॑ । इति॒ तानि॑ । तान्ये॒व । ए॒वाव॒क्षाणा॑नि । अ॒व॒क्षाणा॑नि सन्नि॒धाय॑ । अ॒व॒क्षाणा॒नीत्य॑व - क्षाणा॑नि । स॒न्नि॒धाय॑ मन्थेत् । स॒न्नि॒धायेति॑ सं - नि॒धाय॑ । म॒न्थे॒दि॒तः । इ॒तः प्र॑थ॒मम् । प्र॒थ॒मम् ज॑ज्ञे । ज॒ज्ञे॒ अ॒ग्निः । अ॒ग्निः स्वात् । स्वाद्योनेः᳚ । योने॒रधि॑ । अधि॑ जा॒तवे॑दाः । जा॒तवे॑दा॒ इति॑ जा॒त - वे॒दाः॒ । स गा॑यत्रि॒या । गा॒य॒त्रि॒या त्रि॒ष्टुभा᳚ । त्रि॒ष्टुभा॒ जग॑त्या । जग॑त्या दे॒वेभ्यः॑ । दे॒वेभ्यो॑ ह॒व्यम् । ह॒व्यं ॅव॑हतु । व॒ह॒तु॒ प्र॒जा॒नन्न् । प्र॒जा॒नन्निति॑ । प्र॒जा॒नन्निति॑ प्र - जा॒नन्न् । इति॒ छन्दो॑भिः । छन्दो॑भिरे॒व । छन्दो॑भि॒रिति॒ छन्दः॑ - भिः॒ । ए॒वैन᳚म् । ए॒नꣳ॒॒ स्वात् । स्वाद्योनेः᳚ । योनेः॒ प्र । प्र ज॑नयति । ज॒न॒य॒त्ये॒षः । ए॒ष वाव । वाव सः । सो᳚ऽग्निः । अ॒ग्निरिति॑ । इत्या॑हुः । अ॒हु॒र्,ज्योतिः॑ । ज्योति॒स्तु । त्वै । वा अ॑स्य । अ॒स्य॒ परा॑पतितम् । परा॑पतित॒मिति॑ । परा॑पतित॒मिति॒ परा᳚ - प॒ति॒त॒म् । इति॒ यत् । यद॒ग्नये᳚ । अ॒ग्नये॒ ज्योति॑ष्मते । ज्योति॑ष्मते नि॒र्वप॑ति । नि॒र्वप॑ति॒ यत् । नि॒र्वप॒तीति॑ निः - वप॑ति । यदे॒व । ए॒वास्य॑ ( ) । अ॒स्य॒ ज्योतिः॑ । ज्योतिः॒ परा॑पतितम् । परा॑पतित॒म् तत् । परा॑पतित॒मिति॒ परा᳚ - प॒ति॒त॒म् । तदे॒व । ए॒वाव॑ । अव॑ रुन्धे । रु॒न्ध॒ इति॑ रुन्धे । \newline

\textbf{Jatai Paata} \newline

1. ह्रि॒ये॒ते तीति॑ ह्रियेत ह्रिये॒ते ति॑ । \newline
2. इति॒ तानि॒ तानीतीति॒ तानि॑ । \newline
3. ता न्ये॒वैव तानि॒ ता न्ये॒व । \newline
4. ए॒वाव॒क्षाणा᳚ न्यव॒क्षाणा᳚ न्ये॒वैवाव॒ क्षाणा॑नि । \newline
5. अ॒व॒क्षाणा॑नि सन्नि॒धाय॑ सन्नि॒धाया॑ व॒क्षाणा᳚ न्यव॒क्षाणा॑नि सन्नि॒धाय॑ । \newline
6. अ॒व॒क्षाणा॒नीत्य॑व - क्षाणा॑नि । \newline
7. स॒न्नि॒धाय॑ मन्थेन् मन्थेथ् सन्नि॒धाय॑ सन्नि॒धाय॑ मन्थेत् । \newline
8. स॒न्नि॒धायेति॑ सं - नि॒धाय॑ । \newline
9. म॒न्थे॒दि॒त इ॒तो म॑न्थेन् मन्थेदि॒तः । \newline
10. इ॒तः प्र॑थ॒मम् प्र॑थ॒म मि॒त इ॒तः प्र॑थ॒मम् । \newline
11. प्र॒थ॒मम् ज॑ज्ञे जज्ञे प्रथ॒मम् प्र॑थ॒मम् ज॑ज्ञे । \newline
12. ज॒ज्ञे॒ अ॒ग्नि र॒ग्निर् ज॑ज्ञे जज्ञे अ॒ग्निः । \newline
13. अ॒ग्निः स्वाथ् स्वा द॒ग्नि र॒ग्निः स्वात् । \newline
14. स्वाद् योने॒र् योनेः॒ स्वाथ् स्वाद् योनेः᳚ । \newline
15. योने॒ रध्यधि॒ योने॒र् योने॒ रधि॑ । \newline
16. अधि॑ जा॒तवे॑दा जा॒तवे॑दा॒ अध्यधि॑ जा॒तवे॑दाः । \newline
17. जा॒तवे॑दा॒ इति॑ जा॒त - वे॒दाः॒ । \newline
18. स गा॑यत्रि॒या गा॑यत्रि॒या स स गा॑यत्रि॒या । \newline
19. गा॒य॒त्रि॒या त्रि॒ष्टुभा᳚ त्रि॒ष्टुभा॑ गायत्रि॒या गा॑यत्रि॒या त्रि॒ष्टुभा᳚ । \newline
20. त्रि॒ष्टुभा॒ जग॑त्या॒ जग॑त्या त्रि॒ष्टुभा᳚ त्रि॒ष्टुभा॒ जग॑त्या । \newline
21. जग॑त्या दे॒वेभ्यो॑ दे॒वेभ्यो॒ जग॑त्या॒ जग॑त्या दे॒वेभ्यः॑ । \newline
22. दे॒वेभ्यो॑ ह॒व्यꣳ ह॒व्यम् दे॒वेभ्यो॑ दे॒वेभ्यो॑ ह॒व्यम् । \newline
23. ह॒व्यं ॅव॑हतु वहतु ह॒व्यꣳ ह॒व्यं ॅव॑हतु । \newline
24. व॒ह॒तु॒ प्र॒जा॒नन् प्र॑जा॒नन्. व॑हतु वहतु प्रजा॒नन्न् । \newline
25. प्र॒जा॒नन् नितीति॑ प्रजा॒नन् प्र॑जा॒नन् निति॑ । \newline
26. प्र॒जा॒नन्निति॑ प्र - जा॒नन्न् । \newline
27. इति॒ छन्दो॑भि॒ श्छन्दो॑भि॒ रितीति॒ छन्दो॑भिः । \newline
28. छन्दो॑भि रे॒वैव छन्दो॑भि॒ श्छन्दो॑भि रे॒व । \newline
29. छन्दो॑भि॒रिति॒ छन्दः॑ - भिः॒ । \newline
30. ए॒वैन॑ मेन मे॒वैवैन᳚म् । \newline
31. ए॒नꣳ॒॒ स्वाथ् स्वादे॑न मेनꣳ॒॒ स्वात् । \newline
32. स्वाद् योने॒र् योनेः॒ स्वाथ् स्वाद् योनेः᳚ । \newline
33. योनेः॒ प्र प्र योने॒र् योनेः॒ प्र । \newline
34. प्र ज॑नयति जनयति॒ प्र प्र ज॑नयति । \newline
35. ज॒न॒य॒ त्ये॒ष ए॒ष ज॑नयति जनय त्ये॒षः । \newline
36. ए॒ष वाव वावैष ए॒ष वाव । \newline
37. वाव स स वाव वाव सः । \newline
38. सो᳚ ऽग्नि र॒ग्निः स सो᳚ ऽग्निः । \newline
39. अ॒ग्नि रिती त्य॒ग्नि र॒ग्नि रिति॑ । \newline
40. इत्या॑हु राहु॒ रिती त्या॑हुः । \newline
41. आ॒हु॒र् ज्योति॒र् ज्योति॑ राहु राहु॒र् ज्योतिः॑ । \newline
42. ज्योति॒ स्तु तु ज्योति॒र् ज्योति॒ स्तु । \newline
43. त्वै वै तुत् वै । \newline
44. वा अ॑स्यास्य॒ वै वा अ॑स्य । \newline
45. अ॒स्य॒ परा॑पतित॒म् परा॑पतित मस्यास्य॒ परा॑पतितम् । \newline
46. परा॑पतित॒ मितीति॒ परा॑पतित॒म् परा॑पतित॒ मिति॑ । \newline
47. परा॑पतित॒मिति॒ परा᳚ - प॒ति॒त॒म् । \newline
48. इति॒ यद् यदितीति॒ यत् । \newline
49. यद॒ग्नये॒ ऽग्नये॒ यद् यद॒ग्नये᳚ । \newline
50. अ॒ग्नये॒ ज्योति॑ष्मते॒ ज्योति॑ष्मते॒ ऽग्नये॒ ऽग्नये॒ ज्योति॑ष्मते । \newline
51. ज्योति॑ष्मते नि॒र्वप॑ति नि॒र्वप॑ति॒ ज्योति॑ष्मते॒ ज्योति॑ष्मते नि॒र्वप॑ति । \newline
52. नि॒र्वप॑ति॒ यद् यन् नि॒र्वप॑ति नि॒र्वप॑ति॒ यत् । \newline
53. नि॒र्वप॒तीति॑ निः - वप॑ति । \newline
54. यदे॒वैव यद् यदे॒व । \newline
55. ए॒वास्या᳚स्यै॒वैवास्य॑ । \newline
56. अ॒स्य॒ ज्योति॒र् ज्योति॑ रस्यास्य॒ ज्योतिः॑ । \newline
57. ज्योतिः॒ परा॑पतित॒म् परा॑पतित॒म् ज्योति॒र् ज्योतिः॒ परा॑पतितम् । \newline
58. परा॑पतित॒म् तत् तत् परा॑पतित॒म् परा॑पतित॒म् तत् । \newline
59. परा॑पतित॒मिति॒ परा᳚ - प॒ति॒त॒म् । \newline
60. तदे॒वैव तत् तदे॒व । \newline
61. ए॒वावा वै॒वैवाव॑ । \newline
62. अव॑ रुन्धे रु॒न्धे ऽवाव॑ रुन्धे । \newline
63. रु॒न्ध॒ इति॑ रुन्धे । \newline

\textbf{Ghana Paata } \newline

1. ह्रि॒ये॒ते तीति॑ ह्रियेत ह्रिये॒ते ति॒ तानि॒ तानीति॑ ह्रियेत ह्रिये॒ते ति॒ तानि॑ । \newline
2. इति॒ तानि॒ तानीतीति॒ तान्ये॒वैव तानीतीति॒ तान्ये॒व । \newline
3. तान्ये॒वैव तानि॒ तान्ये॒वाव॒क्षा णा᳚न्यव॒क्षा णा᳚न्ये॒व तानि॒ तान्ये॒वाव॒क्षाणा॑नि । \newline
4. ए॒वाव॒क्षा णा᳚न्यव॒क्षा णा᳚न्ये॒ वैवाव॒क्षाणा॑नि सन्नि॒धाय॑ सन्नि॒धाया॑ व॒क्षाणा᳚न्ये॒ वैवाव॒क्षाणा॑नि सन्नि॒धाय॑ । \newline
5. अ॒व॒क्षाणा॑नि सन्नि॒धाय॑ सन्नि॒धाया॑ व॒क्षाणा᳚ न्यव॒क्षाणा॑नि सन्नि॒धाय॑ मन्थेन् मन्थेथ् सन्नि॒धाया॑ व॒क्षाणा᳚ न्यव॒क्षाणा॑नि सन्नि॒धाय॑ मन्थेत् । \newline
6. अ॒व॒क्षाणा॒नीत्य॑व - क्षाणा॑नि । \newline
7. स॒न्नि॒धाय॑ मन्थेन् मन्थेथ् सन्नि॒धाय॑ सन्नि॒धाय॑ मन्थेदि॒त इ॒तो म॑न्थेथ् सन्नि॒धाय॑ सन्नि॒धाय॑ मन्थेदि॒तः । \newline
8. स॒न्नि॒धायेति॑ सं - नि॒धाय॑ । \newline
9. म॒न्थे॒दि॒त इ॒तो म॑न्थेन् मन्थेदि॒तः प्र॑थ॒मम् प्र॑थ॒म मि॒तो म॑न्थेन् मन्थेदि॒तः प्र॑थ॒मम् । \newline
10. इ॒तः प्र॑थ॒मम् प्र॑थ॒म मि॒त इ॒तः प्र॑थ॒मम् ज॑ज्ञे जज्ञे प्रथ॒म मि॒त इ॒तः प्र॑थ॒मम् ज॑ज्ञे । \newline
11. प्र॒थ॒मम् ज॑ज्ञे जज्ञे प्रथ॒मम् प्र॑थ॒मम् ज॑ज्ञे अ॒ग्नि र॒ग्निर् ज॑ज्ञे प्रथ॒मम् प्र॑थ॒मम् ज॑ज्ञे अ॒ग्निः । \newline
12. ज॒ज्ञे॒ अ॒ग्नि र॒ग्निर् ज॑ज्ञे जज्ञे अ॒ग्निः स्वाथ् स्वाद॒ग्निर् ज॑ज्ञे जज्ञे अ॒ग्निः स्वात् । \newline
13. अ॒ग्निः स्वाथ् स्वाद॒ग्नि र॒ग्निः स्वाद् योने॒र् योनेः॒ स्वाद॒ग्नि र॒ग्निः स्वाद् योनेः᳚ । \newline
14. स्वाद् योने॒र् योनेः॒ स्वाथ् स्वाद् योने॒ रध्यधि॒ योनेः॒ स्वाथ् स्वाद् योने॒रधि॑ । \newline
15. योने॒ रध्यधि॒ योने॒र् योने॒ रधि॑ जा॒तवे॑दा जा॒तवे॑दा॒ अधि॒ योने॒र् योने॒ रधि॑ जा॒तवे॑दाः । \newline
16. अधि॑ जा॒तवे॑दा जा॒तवे॑दा॒ अध्यधि॑ जा॒तवे॑दाः । \newline
17. जा॒तवे॑दा॒ इति॑ जा॒त - वे॒दाः॒ । \newline
18. स गा॑यत्रि॒या गा॑यत्रि॒या स स गा॑यत्रि॒या त्रि॒ष्टुभा᳚ त्रि॒ष्टुभा॑ गायत्रि॒या स स गा॑यत्रि॒या त्रि॒ष्टुभा᳚ । \newline
19. गा॒य॒त्रि॒या त्रि॒ष्टुभा᳚ त्रि॒ष्टुभा॑ गायत्रि॒या गा॑यत्रि॒या त्रि॒ष्टुभा॒ जग॑त्या॒ जग॑त्या त्रि॒ष्टुभा॑ गायत्रि॒या गा॑यत्रि॒या त्रि॒ष्टुभा॒ जग॑त्या । \newline
20. त्रि॒ष्टुभा॒ जग॑त्या॒ जग॑त्या त्रि॒ष्टुभा᳚ त्रि॒ष्टुभा॒ जग॑त्या दे॒वेभ्यो॑ दे॒वेभ्यो॒ जग॑त्या त्रि॒ष्टुभा᳚ त्रि॒ष्टुभा॒ जग॑त्या दे॒वेभ्यः॑ । \newline
21. जग॑त्या दे॒वेभ्यो॑ दे॒वेभ्यो॒ जग॑त्या॒ जग॑त्या दे॒वेभ्यो॑ ह॒व्यꣳ ह॒व्यम् दे॒वेभ्यो॒ जग॑त्या॒ जग॑त्या दे॒वेभ्यो॑ ह॒व्यम् । \newline
22. दे॒वेभ्यो॑ ह॒व्यꣳ ह॒व्यम् दे॒वेभ्यो॑ दे॒वेभ्यो॑ ह॒व्यं ॅव॑हतु वहतु ह॒व्यम् दे॒वेभ्यो॑ दे॒वेभ्यो॑ ह॒व्यं ॅव॑हतु । \newline
23. ह॒व्यं ॅव॑हतु वहतु ह॒व्यꣳ ह॒व्यं ॅव॑हतु प्रजा॒नन् प्र॑जा॒नन्. व॑हतु ह॒व्यꣳ ह॒व्यं ॅव॑हतु प्रजा॒नन्न् । \newline
24. व॒ह॒तु॒ प्र॒जा॒नन् प्र॑जा॒नन्. व॑हतु वहतु प्रजा॒नन् नितीति॑ प्रजा॒नन्. व॑हतु वहतु प्रजा॒नन् निति॑ । \newline
25. प्र॒जा॒नन् नितीति॑ प्रजा॒नन् प्र॑जा॒नन् निति॒ छन्दो॑भि॒ श्छन्दो॑भि॒ रिति॑ प्रजा॒नन् प्र॑जा॒नन् निति॒ छन्दो॑भिः । \newline
26. प्र॒जा॒नन्निति॑ प्र - जा॒नन्न् । \newline
27. इति॒ छन्दो॑भि॒ श्छन्दो॑भि॒ रितीति॒ छन्दो॑भि रे॒वैव छन्दो॑भि॒ रितीति॒ छन्दो॑भि रे॒व । \newline
28. छन्दो॑भि रे॒वैव छन्दो॑भि॒ श्छन्दो॑भि रे॒वैन॑ मेन मे॒व छन्दो॑भि॒ श्छन्दो॑भि रे॒वैन᳚म् । \newline
29. छन्दो॑भि॒रिति॒ छन्दः॑ - भिः॒ । \newline
30. ए॒वैन॑ मेन मे॒वैवैनꣳ॒॒ स्वाथ् स्वादे॑न मे॒वैवैनꣳ॒॒ स्वात् । \newline
31. ए॒नꣳ॒॒ स्वाथ् स्वादे॑न मेनꣳ॒॒ स्वाद् योने॒र् योनेः॒ स्वादे॑न मेनꣳ॒॒ स्वाद् योनेः᳚ । \newline
32. स्वाद् योने॒र् योनेः॒ स्वाथ् स्वाद् योनेः॒ प्र प्र योनेः॒ स्वाथ् स्वाद् योनेः॒ प्र । \newline
33. योनेः॒ प्र प्र योने॒र् योनेः॒ प्र ज॑नयति जनयति॒ प्र योने॒र् योनेः॒ प्र ज॑नयति । \newline
34. प्र ज॑नयति जनयति॒ प्र प्र ज॑नय त्ये॒ष ए॒ष ज॑नयति॒ प्र प्र ज॑नय त्ये॒षः । \newline
35. ज॒न॒य॒ त्ये॒ष ए॒ष ज॑नयति जनय त्ये॒ष वाव वावैष ज॑नयति जनय त्ये॒ष वाव । \newline
36. ए॒ष वाव वावैष ए॒ष वाव स स वावैष ए॒ष वाव सः । \newline
37. वाव स स वाव वाव सो᳚ ऽग्नि र॒ग्निः स वाव वाव सो᳚ ऽग्निः । \newline
38. सो᳚ ऽग्नि र॒ग्निः स सो᳚ ऽग्नि रिती त्य॒ग्निः स सो᳚ ऽग्नि रिति॑ । \newline
39. अ॒ग्नि रिती त्य॒ग्नि र॒ग्नि रित्या॑हु राहु॒ रित्य॒ग्नि र॒ग्नि रित्या॑हुः । \newline
40. इत्या॑हु राहु॒ रितीत्या॑हु॒र् ज्योति॒र् ज्योति॑ राहु॒ रितीत्या॑हु॒र् ज्योतिः॑ । \newline
41. आ॒हु॒र् ज्योति॒र् ज्योति॑ राहु राहु॒र् ज्योति॒ स्तु तु ज्योति॑ राहु राहु॒र् ज्योति॒ स्तु । \newline
42. ज्योति॒ स्तु तु ज्योति॒र् ज्योति॒ स्त्वै वै तु ज्योति॒र् ज्योति॒ स्त्वै । \newline
43. त्वै वै तु त्वा अ॑स्यास्य॒ वै तु त्वा अ॑स्य । \newline
44. वा अ॑स्यास्य॒ वै वा अ॑स्य॒ परा॑पतित॒म् परा॑पतित मस्य॒ वै वा अ॑स्य॒ परा॑पतितम् । \newline
45. अ॒स्य॒ परा॑पतित॒म् परा॑पतित मस्यास्य॒ परा॑पतित॒ मितीति॒ परा॑पतित मस्यास्य॒ परा॑पतित॒ मिति॑ । \newline
46. परा॑पतित॒ मितीति॒ परा॑पतित॒म् परा॑पतित॒ मिति॒ यद् यदिति॒ परा॑पतित॒म् परा॑पतित॒ मिति॒ यत् । \newline
47. परा॑पतित॒मिति॒ परा᳚ - प॒ति॒त॒म् । \newline
48. इति॒ यद् यदितीति॒ यद॒ग्नये॒ ऽग्नये॒ यदितीति॒ यद॒ग्नये᳚ । \newline
49. यद॒ग्नये॒ ऽग्नये॒ यद् यद॒ग्नये॒ ज्योति॑ष्मते॒ ज्योति॑ष्मते॒ ऽग्नये॒ यद् यद॒ग्नये॒ ज्योति॑ष्मते । \newline
50. अ॒ग्नये॒ ज्योति॑ष्मते॒ ज्योति॑ष्मते॒ ऽग्नये॒ ऽग्नये॒ ज्योति॑ष्मते नि॒र्वप॑ति नि॒र्वप॑ति॒ ज्योति॑ष्मते॒ ऽग्नये॒ ऽग्नये॒ ज्योति॑ष्मते नि॒र्वप॑ति । \newline
51. ज्योति॑ष्मते नि॒र्वप॑ति नि॒र्वप॑ति॒ ज्योति॑ष्मते॒ ज्योति॑ष्मते नि॒र्वप॑ति॒ यद् यन् नि॒र्वप॑ति॒ ज्योति॑ष्मते॒ ज्योति॑ष्मते नि॒र्वप॑ति॒ यत् । \newline
52. नि॒र्वप॑ति॒ यद् यन् नि॒र्वप॑ति नि॒र्वप॑ति॒ यदे॒वैव यन् नि॒र्वप॑ति नि॒र्वप॑ति॒ यदे॒व । \newline
53. नि॒र्वप॒तीति॑ निः - वप॑ति । \newline
54. यदे॒वैव यद् यदे॒वास्या᳚ स्यै॒व यद् यदे॒वास्य॑ । \newline
55. ए॒वास्या᳚ स्यै॒वैवास्य॒ ज्योति॒र् ज्योति॑ रस्यै॒वैवास्य॒ ज्योतिः॑ । \newline
56. अ॒स्य॒ ज्योति॒र् ज्योति॑ रस्यास्य॒ ज्योतिः॒ परा॑पतित॒म् परा॑पतित॒म् ज्योति॑ रस्यास्य॒ ज्योतिः॒ परा॑पतितम् । \newline
57. ज्योतिः॒ परा॑पतित॒म् परा॑पतित॒म् ज्योति॒र् ज्योतिः॒ परा॑पतित॒म् तत् तत् परा॑पतित॒म् ज्योति॒र् ज्योतिः॒ परा॑पतित॒म् तत् । \newline
58. परा॑पतित॒म् तत् तत् परा॑पतित॒म् परा॑पतित॒म् तदे॒वैव तत् परा॑पतित॒म् परा॑पतित॒म् तदे॒व । \newline
59. परा॑पतित॒मिति॒ परा᳚ - प॒ति॒त॒म् । \newline
60. तदे॒वैव तत् तदे॒वावा वै॒व तत् तदे॒वाव॑ । \newline
61. ए॒वावावै॒ वैवाव॑ रुन्धे रु॒न्धे ऽवै॒वैवाव॑ रुन्धे । \newline
62. अव॑ रुन्धे रु॒न्धे ऽवाव॑ रुन्धे । \newline
63. रु॒न्ध॒ इति॑ रुन्धे । \newline
\pagebreak
\markright{ TS 2.2.5.1  \hfill https://www.vedavms.in \hfill}
\addcontentsline{toc}{section}{ TS 2.2.5.1 }
\section*{ TS 2.2.5.1 }

\textbf{TS 2.2.5.1 } \newline
\textbf{Samhita Paata} \newline

वै॒श्वा॒न॒रं द्वाद॑शकपालं॒ निर्व॑पेद्-वारु॒णं च॒रुं द॑धि॒क्राव्‌ण्णे॑ च॒रुम॑भिश॒स्यमा॑नो॒ यद्-वै᳚श्वान॒रो द्वाद॑शकपालो॒ भव॑ति संॅवथ्स॒रो वा अ॒ग्नि र्वै᳚श्वान॒रः सं॑ॅवथ्स॒रेणै॒वैनꣳ॑ स्वदय॒त्यप॑ पा॒पं ॅवर्णꣳ॑ हते वारु॒णेनै॒वैनं॑ ॅवरुणपा॒शान् मु॑ञ्चति दधि॒क्राव्‌ण्णा॑ पुनाति॒ हिर॑ण्यं॒ दक्षि॑णा प॒वित्रं॒ ॅवै हिर॑ण्यं पु॒नात्ये॒वैन॑मा॒द्य॑-म॒स्यान्नं॑ भवत्ये॒तामे॒व निर्व॑पेत् प्र॒जाका॑मः संॅवथ्स॒रो - [  ] \newline

\textbf{Pada Paata} \newline

वै॒श्वा॒न॒रम् । द्वाद॑शकपाल॒मिति॒ द्वाद॑श - क॒पा॒ल॒म् । निरिति॑ । व॒पे॒त् । वा॒रु॒णम् । च॒रुम् । द॒धि॒क्राव्‌ण्ण॒ इति॑ दधि - क्राव्‌ण्णे᳚ । च॒रुम् । अ॒भि॒श॒स्यमा॑न॒ इत्य॑भि - श॒स्यमा॑नः । यत् । वै॒श्वा॒न॒रः । द्वाद॑शकपाल॒ इति॒ द्वाद॑श - क॒पा॒लः॒ । भव॑ति । सं॒ॅव॒थ्स॒र इति॑ सं - व॒थ्स॒रः । वै । अ॒ग्निः । वै॒श्वा॒न॒रः । सं॒ॅव॒थ्स॒रेणेति॑ सं - व॒थ्स॒रेण॑ । ए॒व । ए॒न॒म् । स्व॒द॒य॒ति॒ । अपेति॑ । पा॒पम् । वर्ण᳚म् । ह॒ते॒ । वा॒रु॒णेन॑ । ए॒व । ए॒न॒म् । व॒रु॒ण॒पा॒शादिति॑ वरुण - पा॒शात् । मु॒ञ्च॒ति॒ । द॒धि॒क्राव्‌ण्णेति॑ दधि -क्राव्‌ण्णा᳚ । पु॒ना॒ति॒ । हिर॑ण्यम् । दक्षि॑णा । प॒वित्र᳚म् । वै । हिर॑ण्यम् । पु॒नाति॑ । ए॒व । ए॒न॒म् । आ॒द्य᳚म् । अ॒स्य॒ । अन्न᳚म् । भ॒व॒ति॒ । ए॒ताम् । ए॒व । निरिति॑ । व॒पे॒त् । प्र॒जाका॑म॒ इति॑ प्र॒जा - का॒मः॒ । सं॒ॅव॒थ्स॒र इति॑ सं - व॒थ्स॒रः ।  \newline


\textbf{Krama Paata} \newline

वै॒श्वा॒न॒रम् द्वाद॑शकपालम् । द्वाद॑शकपाल॒म् निः । द्वाद॑शकपाल॒मिति॒ द्वाद॑श - क॒पा॒ल॒म् । निर् व॑पेत् । व॒पे॒द् वा॒रु॒णम् । वा॒रु॒णम् च॒रुम् । च॒रुम् द॑धि॒क्राव्.ण्णे᳚ । द॒धि॒क्राव्.ण्णे॑ च॒रुम् । द॒धि॒क्राव्.ण्ण॒ इति॑ दधि - क्राव्.ण्णे᳚ । च॒रुम॑भिश॒स्यमा॑नः । अ॒भि॒श॒स्यमा॑नो॒ यत् । अ॒भि॒॒शस्यमा॑न॒ इत्य॑भि - श॒स्यमा॑नः । यद् वै᳚श्वान॒रः । वै॒श्वा॒न॒रो द्वाद॑शकपालः । द्वाद॑शकपालो॒ भव॑ति । द्वाद॑शकपाल॒ इति॒ द्वाद॑श - क॒पा॒लः॒ । भव॑ति सम्ॅवथ्स॒रः । स॒म्ॅव॒थ्स॒रो वै । स॒म्ॅव॒थ्स॒र इति॑ सम् - व॒थ्स॒रः । वा अ॒ग्निः । अ॒ग्निर्,वै᳚श्वान॒रः । वै॒श्वा॒न॒रः स॑म्ॅवथ्स॒रेण॑ । स॒म्ॅव॒थ्स॒रेणै॒व । स॒म्ॅव॒थ्स॒रेणेति॑ सं - व॒थ्स॒रेण॑ । ए॒वैन᳚म् । ए॒नꣳ॒॒ स्व॒द॒य॒ति॒ । स्व॒द॒य॒त्यप॑ । अप॑ पा॒पम् । पा॒पं ॅवर्ण᳚म् । वर्णꣳ॑ हते । ह॒ते॒ वा॒रु॒णेन॑ । वा॒रु॒णेनै॒व । ए॒वैन᳚म् । ए॒नं॒ ॅव॒रु॒ण॒पा॒शात् । व॒रु॒ण॒पा॒शान् मु॑ञ्चति । व॒रु॒ण॒पा॒शादिति॑ वरुण - पा॒शात् । मु॒ञ्च॒ति॒ द॒धि॒क्राव्.ण्णा᳚ । द॒धि॒क्राव्.ण्णा॑ पुनाति । द॒धि॒क्राव्.ण्णेति॑ दधि - क्राव्.ण्णा᳚ । पु॒ना॒ति॒ हिर॑ण्यम् । हिर॑ण्य॒म् दक्षि॑णा । दक्षि॑णा प॒वित्र᳚म् । प॒वित्रं॒ ॅवै । वै हिर॑ण्यम् । हिर॑ण्यम् पु॒नाति॑ । पु॒नात्ये॒व । ए॒वैन᳚म् । ए॒न॒मा॒द्य᳚म् । आ॒द्य॑मस्य । अ॒स्यान्न᳚म् । अन्न॑म् भवति । भ॒व॒त्ये॒ताम् । ए॒तामे॒व । ए॒व निः । निर् व॑पेत् । व॒पे॒त्,प्र॒जाका॑मः । प्र॒जाका॑मः सम्ॅवथ्स॒रः । प्र॒जाका॑म॒ इति॑ प्र॒जा - का॒मः॒ । स॒म्ॅव॒थ्स॒रो वै । स॒म्ॅव॒थ्स॒र इति॑ सं - व॒थ्स॒रः \newline

\textbf{Jatai Paata} \newline

1. वै॒श्वा॒न॒रम् द्वाद॑शकपाल॒म् द्वाद॑शकपालं ॅवैश्वान॒रं ॅवै᳚श्वान॒रम् द्वाद॑शकपालम् । \newline
2. द्वाद॑शकपाल॒म् निर् णिर् द्वाद॑शकपाल॒म् द्वाद॑शकपाल॒म् निः । \newline
3. द्वाद॑शकपाल॒मिति॒ द्वाद॑श - क॒पा॒ल॒म् । \newline
4. निर् व॑पेद् वपे॒न् निर् णिर् व॑पेत् । \newline
5. व॒पे॒द् वा॒रु॒णं ॅवा॑रु॒णं ॅव॑पेद् वपेद् वारु॒णम् । \newline
6. वा॒रु॒णम् च॒रुम् च॒रुं ॅवा॑रु॒णं ॅवा॑रु॒णम् च॒रुम् । \newline
7. च॒रुम् द॑धि॒क्राव्.ण्णे॑ दधि॒क्राव्.ण्णे॑ च॒रुम् च॒रुम् द॑धि॒क्राव्.ण्णे᳚ । \newline
8. द॒धि॒क्राव्.ण्णे॑ च॒रुम् च॒रुम् द॑धि॒क्राव्.ण्णे॑ दधि॒क्राव्.ण्णे॑ च॒रुम् । \newline
9. द॒धि॒क्राव्.ण्ण॒ इति॑ दधि - क्राव्.ण्णे᳚ । \newline
10. च॒रु म॑भिश॒स्यमा॑नो ऽभिश॒स्यमा॑न श्च॒रुम् च॒रु म॑भिश॒स्यमा॑नः । \newline
11. अ॒भि॒श॒स्यमा॑नो॒ यद् यद॑भिश॒स्यमा॑नो ऽभिश॒स्यमा॑नो॒ यत् । \newline
12. अ॒भि॒श॒स्यमा॑न॒ इत्य॑भि - श॒स्यमा॑नः । \newline
13. यद् वै᳚श्वान॒रो वै᳚श्वान॒रो यद् यद् वै᳚श्वान॒रः । \newline
14. वै॒श्वा॒न॒रो द्वाद॑शकपालो॒ द्वाद॑शकपालो वैश्वान॒रो वै᳚श्वान॒रो द्वाद॑शकपालः । \newline
15. द्वाद॑शकपालो॒ भव॑ति॒ भव॑ति॒ द्वाद॑शकपालो॒ द्वाद॑शकपालो॒ भव॑ति । \newline
16. द्वाद॑शकपाल॒ इति॒ द्वाद॑श - क॒पा॒लः॒ । \newline
17. भव॑ति संॅवथ्स॒रः सं॑ॅवथ्स॒रो भव॑ति॒ भव॑ति संॅवथ्स॒रः । \newline
18. सं॒ॅव॒थ्स॒रो वै वै सं॑ॅवथ्स॒रः सं॑ॅवथ्स॒रो वै । \newline
19. सं॒ॅव॒थ्स॒र इति॑ सं - व॒थ्स॒रः । \newline
20. वा अ॒ग्नि र॒ग्निर् वै वा अ॒ग्निः । \newline
21. अ॒ग्निर् वै᳚श्वान॒रो वै᳚श्वान॒रो᳚ ऽग्नि र॒ग्निर् वै᳚श्वान॒रः । \newline
22. वै॒श्वा॒न॒रः सं॑ॅवथ्स॒रेण॑ संॅवथ्स॒रेण॑ वैश्वान॒रो वै᳚श्वान॒रः सं॑ॅवथ्स॒रेण॑ । \newline
23. सं॒ॅव॒थ्स॒ रेणै॒वैव सं॑ॅवथ्स॒रेण॑ संॅवथ्स॒ रेणै॒व । \newline
24. सं॒ॅव॒थ्स॒रेणेति॑ सं - व॒थ्स॒रेण॑ । \newline
25. ए॒वैन॑ मेन मे॒वैवैन᳚म् । \newline
26. ए॒नꣳ॒॒ स्व॒द॒य॒ति॒ स्व॒द॒य॒त्ये॒न॒ मे॒नꣳ॒॒ स्व॒द॒य॒ति॒ । \newline
27. स्व॒द॒य॒ त्यपाप॑ स्वदयति स्वदय॒ त्यप॑ । \newline
28. अप॑ पा॒पम् पा॒प मपाप॑ पा॒पम् । \newline
29. पा॒पं ॅवर्णं॒ ॅवर्ण॑म् पा॒पम् पा॒पं ॅवर्ण᳚म् । \newline
30. वर्णꣳ॑ हते हते॒ वर्णं॒ ॅवर्णꣳ॑ हते । \newline
31. ह॒ते॒ वा॒रु॒णेन॑ वारु॒णेन॑ हते हते वारु॒णेन॑ । \newline
32. वा॒रु॒णे नै॒वैव वा॑रु॒णेन॑ वारु॒णे नै॒व । \newline
33. ए॒वैन॑ मेन मे॒वैवैन᳚म् । \newline
34. ए॒नं॒ ॅव॒रु॒ण॒पा॒शाद् व॑रुणपा॒शा दे॑न मेनं ॅवरुणपा॒शात् । \newline
35. व॒रु॒ण॒पा॒शान् मु॑ञ्चति मुञ्चति वरुणपा॒शाद् व॑रुणपा॒शान् मु॑ञ्चति । \newline
36. व॒रु॒ण॒पा॒शादिति॑ वरुण - पा॒शात् । \newline
37. मु॒ञ्च॒ति॒ द॒धि॒क्राव्.ण्णा॑ दधि॒क्राव्.ण्णा॑ मुञ्चति मुञ्चति दधि॒क्राव्.ण्णा᳚ । \newline
38. द॒धि॒क्राव्.ण्णा॑ पुनाति पुनाति दधि॒क्राव्.ण्णा॑ दधि॒क्राव्.ण्णा॑ पुनाति । \newline
39. द॒धि॒क्राव्.ण्णेति॑ दधि - क्राव्.ण्णा᳚ । \newline
40. पु॒ना॒ति॒ हिर॑ण्यꣳ॒॒ हिर॑ण्यम् पुनाति पुनाति॒ हिर॑ण्यम् । \newline
41. हिर॑ण्य॒म् दक्षि॑णा॒ दक्षि॑णा॒ हिर॑ण्यꣳ॒॒ हिर॑ण्य॒म् दक्षि॑णा । \newline
42. दक्षि॑णा प॒वित्र॑म् प॒वित्र॒म् दक्षि॑णा॒ दक्षि॑णा प॒वित्र᳚म् । \newline
43. प॒वित्रं॒ ॅवै वै प॒वित्र॑म् प॒वित्रं॒ ॅवै । \newline
44. वै हिर॑ण्यꣳ॒॒ हिर॑ण्यं॒ ॅवै वै हिर॑ण्यम् । \newline
45. हिर॑ण्यम् पु॒नाति॑ पु॒नाति॒ हिर॑ण्यꣳ॒॒ हिर॑ण्यम् पु॒नाति॑ । \newline
46. पु॒ना त्ये॒वैव पु॒नाति॑ पु॒ना त्ये॒व । \newline
47. ए॒वैन॑ मेन मे॒वैवैन᳚म् । \newline
48. ए॒न॒ मा॒द्य॑ मा॒द्य॑ मेन मेन मा॒द्य᳚म् । \newline
49. आ॒द्य॑ मस्यास्या॒द्य॑ मा॒द्य॑ मस्य । \newline
50. अ॒स्यान्न॒ मन्न॑ मस्या॒ स्यान्न᳚म् । \newline
51. अन्न॑म् भवति भव॒त्यन्न॒ मन्न॑म् भवति । \newline
52. भ॒व॒ त्ये॒ता मे॒ताम् भ॑वति भव त्ये॒ताम् । \newline
53. ए॒ता मे॒वैवैता मे॒ता मे॒व । \newline
54. ए॒व निर् णि रे॒वैव निः । \newline
55. निर् व॑पेद् वपे॒न् निर् णिर् व॑पेत् । \newline
56. व॒पे॒त् प्र॒जाका॑मः प्र॒जाका॑मो वपेद् वपेत् प्र॒जाका॑मः । \newline
57. प्र॒जाका॑मः संॅवथ्स॒रः सं॑ॅवथ्स॒रः प्र॒जाका॑मः प्र॒जाका॑मः संॅवथ्स॒रः । \newline
58. प्र॒जाका॑म॒ इति॑ प्र॒जा - का॒मः॒ । \newline
59. सं॒ॅव॒थ्स॒रो वै वै सं॑ॅवथ्स॒रः सं॑ॅवथ्स॒रो वै । \newline
60. सं॒ॅव॒थ्स॒र इति॑ सं - व॒थ्स॒रः । \newline

\textbf{Ghana Paata } \newline

1. वै॒श्वा॒न॒रम् द्वाद॑शकपाल॒म् द्वाद॑शकपालं ॅवैश्वान॒रं ॅवै᳚श्वान॒रम् द्वाद॑शकपाल॒म् निर् णिर् द्वाद॑शकपालं ॅवैश्वान॒रं ॅवै᳚श्वान॒रम् द्वाद॑शकपाल॒म् निः । \newline
2. द्वाद॑शकपाल॒म् निर् णिर् द्वाद॑शकपाल॒म् द्वाद॑शकपाल॒म् निर् व॑पेद् वपे॒न् निर् द्वाद॑शकपाल॒म् द्वाद॑शकपाल॒म् निर् व॑पेत् । \newline
3. द्वाद॑शकपाल॒मिति॒ द्वाद॑श - क॒पा॒ल॒म् । \newline
4. निर् व॑पेद् वपे॒न् निर् णिर् व॑पेद् वारु॒णं ॅवा॑रु॒णं ॅव॑पे॒न् निर् णिर् व॑पेद् वारु॒णम् । \newline
5. व॒पे॒द् वा॒रु॒णं ॅवा॑रु॒णं ॅव॑पेद् वपेद् वारु॒णम् च॒रुम् च॒रुं ॅवा॑रु॒णं ॅव॑पेद् वपेद् वारु॒णम् च॒रुम् । \newline
6. वा॒रु॒णम् च॒रुम् च॒रुं ॅवा॑रु॒णं ॅवा॑रु॒णम् च॒रुम् द॑धि॒क्राव्.ण्णे॑ दधि॒क्राव्.ण्णे॑ च॒रुं ॅवा॑रु॒णं ॅवा॑रु॒णम् च॒रुम् द॑धि॒क्राव्.ण्णे᳚ । \newline
7. च॒रुम् द॑धि॒क्राव्.ण्णे॑ दधि॒क्राव्.ण्णे॑ च॒रुम् च॒रुम् द॑धि॒क्राव्.ण्णे॑ च॒रुम् च॒रुम् द॑धि॒क्राव्.ण्णे॑ च॒रुम् च॒रुम् द॑धि॒क्राव्.ण्णे॑ च॒रुम् । \newline
8. द॒धि॒क्राव्.ण्णे॑ च॒रुम् च॒रुम् द॑धि॒क्राव्.ण्णे॑ दधि॒क्राव्.ण्णे॑ च॒रु म॑भिश॒स्यमा॑नो ऽभिश॒स्यमा॑न श्च॒रुम् द॑धि॒क्राव्.ण्णे॑ दधि॒क्राव्.ण्णे॑ च॒रु म॑भिश॒स्यमा॑नः । \newline
9. द॒धि॒क्राव्.ण्ण॒ इति॑ दधि - क्राव्.ण्णे᳚ । \newline
10. च॒रु म॑भिश॒स्यमा॑नो ऽभिश॒स्यमा॑न श्च॒रुम् च॒रु म॑भिश॒स्यमा॑नो॒ यद् यद॑भिश॒स्यमा॑न श्च॒रुम् च॒रु म॑भिश॒स्यमा॑नो॒ यत् । \newline
11. अ॒भि॒श॒स्यमा॑नो॒ यद् यद॑भिश॒स्यमा॑नो ऽभिश॒स्यमा॑नो॒ यद् वै᳚श्वान॒रो वै᳚श्वान॒रो यद॑भिश॒स्यमा॑नो ऽभिश॒स्यमा॑नो॒ यद् वै᳚श्वान॒रः । \newline
12. अ॒भि॒श॒स्यमा॑न॒ इत्य॑भि - श॒स्यमा॑नः । \newline
13. यद् वै᳚श्वान॒रो वै᳚श्वान॒रो यद् यद् वै᳚श्वान॒रो द्वाद॑शकपालो॒ द्वाद॑शकपालो वैश्वान॒रो यद् यद् वै᳚श्वान॒रो द्वाद॑शकपालः । \newline
14. वै॒श्वा॒न॒रो द्वाद॑शकपालो॒ द्वाद॑शकपालो वैश्वान॒रो वै᳚श्वान॒रो द्वाद॑शकपालो॒ भव॑ति॒ भव॑ति॒ द्वाद॑शकपालो वैश्वान॒रो वै᳚श्वान॒रो द्वाद॑शकपालो॒ भव॑ति । \newline
15. द्वाद॑शकपालो॒ भव॑ति॒ भव॑ति॒ द्वाद॑शकपालो॒ द्वाद॑शकपालो॒ भव॑ति संॅवथ्स॒रः सं॑ॅवथ्स॒रो भव॑ति॒ द्वाद॑शकपालो॒ द्वाद॑शकपालो॒ भव॑ति संॅवथ्स॒रः । \newline
16. द्वाद॑शकपाल॒ इति॒ द्वाद॑श - क॒पा॒लः॒ । \newline
17. भव॑ति संॅवथ्स॒रः सं॑ॅवथ्स॒रो भव॑ति॒ भव॑ति संॅवथ्स॒रो वै वै सं॑ॅवथ्स॒रो भव॑ति॒ भव॑ति संॅवथ्स॒रो वै । \newline
18. सं॒ॅव॒थ्स॒रो वै वै सं॑ॅवथ्स॒रः सं॑ॅवथ्स॒रो वा अ॒ग्नि र॒ग्निर् वै सं॑ॅवथ्स॒रः सं॑ॅवथ्स॒रो वा अ॒ग्निः । \newline
19. सं॒ॅव॒थ्स॒र इति॑ सं - व॒थ्स॒रः । \newline
20. वा अ॒ग्नि र॒ग्निर् वै वा अ॒ग्निर् वै᳚श्वान॒रो वै᳚श्वान॒रो᳚ ऽग्निर् वै वा अ॒ग्निर् वै᳚श्वान॒रः । \newline
21. अ॒ग्निर् वै᳚श्वान॒रो वै᳚श्वान॒रो᳚ ऽग्नि र॒ग्निर् वै᳚श्वान॒रः सं॑ॅवथ्स॒रेण॑ संॅवथ्स॒रेण॑ वैश्वान॒रो᳚ ऽग्नि र॒ग्निर् वै᳚श्वान॒रः सं॑ॅवथ्स॒रेण॑ । \newline
22. वै॒श्वा॒न॒रः सं॑ॅवथ्स॒रेण॑ संॅवथ्स॒रेण॑ वैश्वान॒रो वै᳚श्वान॒रः सं॑ॅवथ्स॒रे णै॒वैव सं॑ॅवथ्स॒रेण॑ वैश्वान॒रो वै᳚श्वान॒रः सं॑ॅवथ्स॒रे णै॒व । \newline
23. सं॒ॅव॒थ्स॒रे णै॒वैव सं॑ॅवथ्स॒रेण॑ संॅवथ्स॒रे णै॒वैन॑ मेन मे॒व सं॑ॅवथ्स॒रेण॑ संॅवथ्स॒रे णै॒वैन᳚म् । \newline
24. सं॒ॅव॒थ्स॒रेणेति॑ सं - व॒थ्स॒रेण॑ । \newline
25. ए॒वैन॑ मेन मे॒वैवैनꣳ॑ स्वदयति स्वदय त्येन मे॒वैवैनꣳ॑ स्वदयति । \newline
26. ए॒नꣳ॒॒ स्व॒द॒य॒ति॒ स्व॒द॒य॒ त्ये॒न॒ मे॒नꣳ॒॒ स्व॒द॒य॒ त्यपाप॑ स्वदय त्येन मेनꣳ स्वदय॒ त्यप॑ । \newline
27. स्व॒द॒य॒ त्यपाप॑ स्वदयति स्वदय॒ त्यप॑ पा॒पम् पा॒प मप॑ स्वदयति स्वदय॒ त्यप॑ पा॒पम् । \newline
28. अप॑ पा॒पम् पा॒प मपाप॑ पा॒पं ॅवर्णं॒ ॅवर्ण॑म् पा॒प मपाप॑ पा॒पं ॅवर्ण᳚म् । \newline
29. पा॒पं ॅवर्णं॒ ॅवर्ण॑म् पा॒पम् पा॒पं ॅवर्णꣳ॑ हते हते॒ वर्ण॑म् पा॒पम् पा॒पं ॅवर्णꣳ॑ हते । \newline
30. वर्णꣳ॑ हते हते॒ वर्णं॒ ॅवर्णꣳ॑ हते वारु॒णेन॑ वारु॒णेन॑ हते॒ वर्णं॒ ॅवर्णꣳ॑ हते वारु॒णेन॑ । \newline
31. ह॒ते॒ वा॒रु॒णेन॑ वारु॒णेन॑ हते हते वारु॒णेनै॒वैव वा॑रु॒णेन॑ हते हते वारु॒णेनै॒व । \newline
32. वा॒रु॒णेनै॒ वैव वा॑रु॒णेन॑ वारु॒णेनै॒वैन॑ मेन मे॒व वा॑रु॒णेन॑ वारु॒णेनै॒वैन᳚म् । \newline
33. ए॒वैन॑ मेन मे॒वैवैनं॑ ॅवरुणपा॒शाद् व॑रुणपा॒शा दे॑न मे॒वैवैनं॑ ॅवरुणपा॒शात् । \newline
34. ए॒नं॒ ॅव॒रु॒ण॒पा॒शाद् व॑रुणपा॒शा दे॑न मेनं ॅवरुणपा॒शान् मु॑ञ्चति मुञ्चति वरुणपा॒शा दे॑न मेनं ॅवरुणपा॒शान् मु॑ञ्चति । \newline
35. व॒रु॒ण॒पा॒शान् मु॑ञ्चति मुञ्चति वरुणपा॒शाद् व॑रुणपा॒शान् मु॑ञ्चति दधि॒क्राव्.ण्णा॑ दधि॒क्राव्.ण्णा॑ मुञ्चति वरुणपा॒शाद् व॑रुणपा॒शान् मु॑ञ्चति दधि॒क्राव्.ण्णा᳚ । \newline
36. व॒रु॒ण॒पा॒शादिति॑ वरुण - पा॒शात् । \newline
37. मु॒ञ्च॒ति॒ द॒धि॒क्राव्.ण्णा॑ दधि॒क्राव्.ण्णा॑ मुञ्चति मुञ्चति दधि॒क्राव्.ण्णा॑ पुनाति पुनाति दधि॒क्राव्.ण्णा॑ मुञ्चति मुञ्चति दधि॒क्राव्.ण्णा॑ पुनाति । \newline
38. द॒धि॒क्राव्.ण्णा॑ पुनाति पुनाति दधि॒क्राव्.ण्णा॑ दधि॒क्राव्.ण्णा॑ पुनाति॒ हिर॑ण्यꣳ॒॒ हिर॑ण्यम् पुनाति दधि॒क्राव्.ण्णा॑ दधि॒क्राव्.ण्णा॑ पुनाति॒ हिर॑ण्यम् । \newline
39. द॒धि॒क्राव्.ण्णेति॑ दधि - क्राव्.ण्णा᳚ । \newline
40. पु॒ना॒ति॒ हिर॑ण्यꣳ॒॒ हिर॑ण्यम् पुनाति पुनाति॒ हिर॑ण्य॒म् दक्षि॑णा॒ दक्षि॑णा॒ हिर॑ण्यम् पुनाति पुनाति॒ हिर॑ण्य॒म् दक्षि॑णा । \newline
41. हिर॑ण्य॒म् दक्षि॑णा॒ दक्षि॑णा॒ हिर॑ण्यꣳ॒॒ हिर॑ण्य॒म् दक्षि॑णा प॒वित्र॑म् प॒वित्र॒म् दक्षि॑णा॒ हिर॑ण्यꣳ॒॒ हिर॑ण्य॒म् दक्षि॑णा प॒वित्र᳚म् । \newline
42. दक्षि॑णा प॒वित्र॑म् प॒वित्र॒म् दक्षि॑णा॒ दक्षि॑णा प॒वित्रं॒ ॅवै वै प॒वित्र॒म् दक्षि॑णा॒ दक्षि॑णा प॒वित्रं॒ ॅवै । \newline
43. प॒वित्रं॒ ॅवै वै प॒वित्र॑म् प॒वित्रं॒ ॅवै हिर॑ण्यꣳ॒॒ हिर॑ण्यं॒ ॅवै प॒वित्र॑म् प॒वित्रं॒ ॅवै हिर॑ण्यम् । \newline
44. वै हिर॑ण्यꣳ॒॒ हिर॑ण्यं॒ ॅवै वै हिर॑ण्यम् पु॒नाति॑ पु॒नाति॒ हिर॑ण्यं॒ ॅवै वै हिर॑ण्यम् पु॒नाति॑ । \newline
45. हिर॑ण्यम् पु॒नाति॑ पु॒नाति॒ हिर॑ण्यꣳ॒॒ हिर॑ण्यम् पु॒ना त्ये॒वैव पु॒नाति॒ हिर॑ण्यꣳ॒॒ हिर॑ण्यम् पु॒ना त्ये॒व । \newline
46. पु॒ना त्ये॒वैव पु॒नाति॑ पु॒ना त्ये॒वैन॑ मेन मे॒व पु॒नाति॑ पु॒ना त्ये॒वैन᳚म् । \newline
47. ए॒वैन॑ मेन मे॒वैवैन॑ मा॒द्य॑ मा॒द्य॑ मेन मे॒वैवैन॑ मा॒द्य᳚म् । \newline
48. ए॒न॒ मा॒द्य॑ मा॒द्य॑ मेन मेन मा॒द्य॑ मस्यास्या॒द्य॑ मेन मेन मा॒द्य॑ मस्य । \newline
49. आ॒द्य॑ मस्यास्या॒द्य॑ मा॒द्य॑ म॒स्यान्न॒ मन्न॑ मस्या॒द्य॑ मा॒द्य॑ म॒स्यान्न᳚म् । \newline
50. अ॒स्यान्न॒ मन्न॑ मस्या॒स्यान्न॑म् भवति भव॒त्यन्न॑ मस्या॒स्यान्न॑म् भवति । \newline
51. अन्न॑म् भवति भव॒त्यन्न॒ मन्न॑म् भवत्ये॒ता मे॒ताम् भ॑व॒त्यन्न॒ मन्न॑म् भवत्ये॒ताम् । \newline
52. भ॒व॒त्ये॒ता मे॒ताम् भ॑वति भवत्ये॒ता मे॒वैवैताम् भ॑वति भवत्ये॒ता मे॒व । \newline
53. ए॒ता मे॒वैवैता मे॒ता मे॒व निर् णिरे॒वैता मे॒ता मे॒व निः । \newline
54. ए॒व निर् णिरे॒वैव निर् व॑पेद् वपे॒न् निरे॒वैव निर् व॑पेत् । \newline
55. निर् व॑पेद् वपे॒न् निर् णिर् व॑पेत् प्र॒जाका॑मः प्र॒जाका॑मो वपे॒न् निर् णिर् व॑पेत् प्र॒जाका॑मः । \newline
56. व॒पे॒त् प्र॒जाका॑मः प्र॒जाका॑मो वपेद् वपेत् प्र॒जाका॑मः संॅवथ्स॒रः सं॑ॅवथ्स॒रः प्र॒जाका॑मो वपेद् वपेत् प्र॒जाका॑मः संॅवथ्स॒रः । \newline
57. प्र॒जाका॑मः संॅवथ्स॒रः सं॑ॅवथ्स॒रः प्र॒जाका॑मः प्र॒जाका॑मः संॅवथ्स॒रो वै वै सं॑ॅवथ्स॒रः प्र॒जाका॑मः प्र॒जाका॑मः संॅवथ्स॒रो वै । \newline
58. प्र॒जाका॑म॒ इति॑ प्र॒जा - का॒मः॒ । \newline
59. सं॒ॅव॒थ्स॒रो वै वै सं॑ॅवथ्स॒रः सं॑ॅवथ्स॒रो वा ए॒त स्यै॒तस्य॒ वै सं॑ॅवथ्स॒रः सं॑ॅवथ्स॒रो वा ए॒तस्य॑ । \newline
60. सं॒ॅव॒थ्स॒र इति॑ सं - व॒थ्स॒रः । \newline
\pagebreak
\markright{ TS 2.2.5.2  \hfill https://www.vedavms.in \hfill}
\addcontentsline{toc}{section}{ TS 2.2.5.2 }
\section*{ TS 2.2.5.2 }

\textbf{TS 2.2.5.2 } \newline
\textbf{Samhita Paata} \newline

वा ए॒तस्या शा᳚न्तो॒ योनिं॑ प्र॒जायै॑ पशू॒नां निर्द॑हति॒ योऽलं॑ प्र॒जायै॒ सन् प्र॒जां न वि॒न्दते॒ यद्-वै᳚श्वान॒रो द्वाद॑शकपालो॒ भव॑ति संॅवथ्स॒रो वा अ॒ग्नि र्वै᳚श्वान॒रः सं॑ॅवथ्स॒रमे॒व भा॑ग॒धेये॑न शमयति॒ सो᳚ऽस्मै शा॒न्तः स्वाद्योनेः᳚ प्र॒जां प्रज॑नयति वारु॒णेनै॒वैनं॑ ॅवरुणपा॒शान् मु॑ञ्चति दधि॒क्राव्‌ण्णा॑ पुनाति॒ हिर॑ण्यं॒ दक्षि॑णा प॒वित्रं॒ ॅवै हिर॑ण्यं पु॒नात्ये॒वैनं॑- [  ] \newline

\textbf{Pada Paata} \newline

वै । ए॒तस्य॑ । अशा᳚न्तः । योनि᳚म् । प्र॒जाया॒ इति॑ प्र - जायै᳚ । प॒शू॒नाम् । निरिति॑ । द॒ह॒ति॒ । यः । अल᳚म् । प्र॒जाया॒ इति॑ प्र-जायै᳚ । सन्न् । प्र॒जामिति॑ प्र - जाम् । न । वि॒न्दते᳚ । यत् । वै॒श्वा॒न॒रः । द्वाद॑शकपाल॒ इति॒ द्वाद॑श - क॒पा॒लः॒ । भव॑ति । सं॒ॅव॒थ्स॒र इति॑ सं - व॒थ्स॒रः । वै । अ॒ग्निः । वै॒श्वा॒न॒रः । सं॒ॅव॒थ्स॒रमिति॑ सं - व॒थ्स॒रम् । ए॒व । भा॒ग॒धेये॒नेति॑ भाग - धेये॑न । श॒म॒य॒ति॒ । सः । अ॒स्मै॒ । शा॒न्तः । स्वात् । योनेः᳚ । प्र॒जामिति॑ प्र - जाम् । प्रेति॑ । ज॒न॒य॒ति॒ । वा॒रु॒णेन॑ । ए॒व । ए॒न॒म् । व॒रु॒ण॒पा॒शादिति॑ वरुण - पा॒शात् । मु॒ञ्च॒ति॒ । द॒धि॒क्राव्‌ण्णेति॑ दधि - क्राव्‌ण्णा᳚ । पु॒ना॒ति॒ । हिर॑ण्यम् । दक्षि॑णा । प॒वित्र᳚म् । वै । हिर॑ण्यम् । पु॒नाति॑ । ए॒व । ए॒न॒म् ।  \newline


\textbf{Krama Paata} \newline

वा ए॒तस्य॑ । ए॒तस्याशा᳚न्तः । अशा᳚न्तो॒ योनि᳚म् । योनि॑म् प्र॒जायै᳚ । प्र॒जायै॑ पशू॒नाम् । प्र॒जाया॒ इति॑ प्र - जायै᳚ । प॒शू॒नाम् निः । निर् द॑हति । द॒ह॒ति॒ यः । योऽल᳚म् । अल॑म् प्र॒जायै᳚ । प्र॒जायै॒ सन्न् । प्र॒जाया॒ इति॑ प्र - जायै᳚ । सन्,प्र॒जाम् । प्र॒जाम् न । प्र॒जामिति॑ प्र - जाम् । न वि॒न्दते᳚ । वि॒न्दते॒ यत् । यद् वै᳚श्वान॒रः । वै॒श्वा॒न॒रो द्वाद॑शकपालः । द्वाद॑शकपालो॒ भव॑ति । द्वाद॑शकपाल॒ इति॒ द्वाद॑श - क॒पा॒लः॒ । भव॑ति सम्ॅवथ्स॒रः । स॒म्ॅव॒थ्स॒रो वै । स॒म्ॅव॒थ्स॒र इति॑ सं - व॒थ्स॒रः । वा अ॒ग्निः । अ॒ग्निर् वै᳚श्वान॒रः । वै॒श्वा॒न॒रः स॑म्ॅवथ्स॒रम् । स॒म्ॅव॒थ्स॒रमे॒व । स॒म्ॅव॒थ्स॒रमिति॑ सं - व॒थ्स॒रम् । ए॒व भा॑ग॒धेये॑न । भा॒ग॒धेये॑न शमयति । भा॒ग॒धेये॒नेति॑ भाग - धेये॑न । श॒म॒य॒ति॒ सः । सो᳚ऽस्मै । अ॒स्मै॒ शा॒न्तः । शा॒न्तः स्वात् । स्वाद् योनेः᳚ । योनेः᳚ प्र॒जाम् । प्र॒जाम् प्र । प्र॒जामिति॑ प्र - जाम् । प्र ज॑नयति । ज॒न॒य॒ति॒ वा॒रु॒णेन॑ । वा॒रु॒णेनै॒व । ए॒वैन᳚म् । ए॒नं॒ ॅव॒रु॒ण॒पा॒शात् । व॒रु॒ण॒पा॒शान् मु॑ञ्चति । व॒रु॒ण॒पा॒शादिति॑ वरुण - पा॒शात् । मु॒ञ्च॒ति॒ द॒धि॒क्राव्.ण्णा᳚ । द॒धि॒क्राव्.ण्णा॑ पुनाति । द॒धि॒क्राव्.ण्णेति॑ दधि - क्राव्ण्णा᳚ । पु॒ना॒ति॒ हिर॑ण्यम् । हिर॑ण्य॒म् दक्षि॑णा । दक्षि॑णा प॒वित्र᳚म् । प॒वित्रं॒ ॅवै । वै हिर॑ण्यम् । हिर॑ण्यम् पु॒नाति॑ । पु॒नात्ये॒व । ए॒वैन᳚म् । ए॒नं॒ ॅवि॒न्दते᳚ \newline

\textbf{Jatai Paata} \newline

1. वा ए॒त स्यै॒तस्य॒ वै वा ए॒तस्य॑ । \newline
2. ए॒तस्याशा॒न्तो ऽशा᳚न्त ए॒त स्यै॒तस्याशा᳚न्तः । \newline
3. अशा᳚न्तो॒ योनिं॒ ॅयोनि॒ मशा॒न्तो ऽशा᳚न्तो॒ योनि᳚म् । \newline
4. योनि॑म् प्र॒जायै᳚ प्र॒जायै॒ योनिं॒ ॅयोनि॑म् प्र॒जायै᳚ । \newline
5. प्र॒जायै॑ पशू॒नाम् प॑शू॒नाम् प्र॒जायै᳚ प्र॒जायै॑ पशू॒नाम् । \newline
6. प्र॒जाया॒ इति॑ प्र - जायै᳚ । \newline
7. प॒शू॒नाम् निर् णिष् प॑शू॒नाम् प॑शू॒नाम् निः । \newline
8. निर् द॑हति दहति॒ निर् णिर् द॑हति । \newline
9. द॒ह॒ति॒ यो यो द॑हति दहति॒ यः । \newline
10. यो ऽल॒ मलं॒ ॅयो यो ऽल᳚म् । \newline
11. अल॑म् प्र॒जायै᳚ प्र॒जाया॒ अल॒ मल॑म् प्र॒जायै᳚ । \newline
12. प्र॒जायै॒ सन् थ्सन् प्र॒जायै᳚ प्र॒जायै॒ सन्न् । \newline
13. प्र॒जाया॒ इति॑ प्र - जायै᳚ । \newline
14. सन् प्र॒जाम् प्र॒जाꣳ सन् थ्सन् प्र॒जाम् । \newline
15. प्र॒जाम् न न प्र॒जाम् प्र॒जाम् न । \newline
16. प्र॒जामिति॑ प्र - जाम् । \newline
17. न वि॒न्दते॑ वि॒न्दते॒ न न वि॒न्दते᳚ । \newline
18. वि॒न्दते॒ यद् यद् वि॒न्दते॑ वि॒न्दते॒ यत् । \newline
19. यद् वै᳚श्वान॒रो वै᳚श्वान॒रो यद् यद् वै᳚श्वान॒रः । \newline
20. वै॒श्वा॒न॒रो द्वाद॑शकपालो॒ द्वाद॑शकपालो वैश्वान॒रो वै᳚श्वान॒रो द्वाद॑शकपालः । \newline
21. द्वाद॑शकपालो॒ भव॑ति॒ भव॑ति॒ द्वाद॑शकपालो॒ द्वाद॑शकपालो॒ भव॑ति । \newline
22. द्वाद॑शकपाल॒ इति॒ द्वाद॑श - क॒पा॒लः॒ । \newline
23. भव॑ति संॅवथ्स॒रः सं॑ॅवथ्स॒रो भव॑ति॒ भव॑ति संॅवथ्स॒रः । \newline
24. सं॒ॅव॒थ्स॒रो वै वै सं॑ॅवथ्स॒रः सं॑ॅवथ्स॒रो वै । \newline
25. सं॒ॅव॒थ्स॒र इति॑ सं - व॒थ्स॒रः । \newline
26. वा अ॒ग्नि र॒ग्निर् वै वा अ॒ग्निः । \newline
27. अ॒ग्निर् वै᳚श्वान॒रो वै᳚श्वान॒रो᳚ ऽग्निर॒ग्निर् वै᳚श्वान॒रः । \newline
28. वै॒श्वा॒न॒रः सं॑ॅवथ्स॒रꣳ सं॑ॅवथ्स॒रं ॅवै᳚श्वान॒रो वै᳚श्वान॒रः सं॑ॅवथ्स॒रम् । \newline
29. सं॒ॅव॒थ्स॒र मे॒वैव सं॑ॅवथ्स॒रꣳ सं॑ॅवथ्स॒र मे॒व । \newline
30. सं॒ॅव॒थ्स॒रमिति॑ सं - व॒थ्स॒रम् । \newline
31. ए॒व भा॑ग॒धेये॑न भाग॒धेये॑नै॒वैव भा॑ग॒धेये॑न । \newline
32. भा॒ग॒धेये॑न शमयति शमयति भाग॒धेये॑न भाग॒धेये॑न शमयति । \newline
33. भा॒ग॒धेये॒नेति॑ भाग - धेये॑न । \newline
34. श॒म॒य॒ति॒ स स श॑मयति शमयति॒ सः । \newline
35. सो᳚ ऽस्मा अस्मै॒ स सो᳚ ऽस्मै । \newline
36. अ॒स्मै॒ शा॒न्तः शा॒न्तो᳚ ऽस्मा अस्मै शा॒न्तः । \newline
37. शा॒न्तः स्वाथ् स्वाच्छा॒न्तः शा॒न्तः स्वात् । \newline
38. स्वाद् योने॒र् योनेः॒ स्वाथ् स्वाद् योनेः᳚ । \newline
39. योनेः᳚ प्र॒जाम् प्र॒जां ॅयोने॒र् योनेः᳚ प्र॒जाम् । \newline
40. प्र॒जाम् प्र प्र प्र॒जाम् प्र॒जाम् प्र । \newline
41. प्र॒जामिति॑ प्र - जाम् । \newline
42. प्र ज॑नयति जनयति॒ प्र प्र ज॑नयति । \newline
43. ज॒न॒य॒ति॒ वा॒रु॒णेन॑ वारु॒णेन॑ जनयति जनयति वारु॒णेन॑ । \newline
44. वा॒रु॒णे नै॒वैव वा॑रु॒णेन॑ वारु॒णे नै॒व । \newline
45. ए॒वैन॑ मेन मे॒वैवैन᳚म् । \newline
46. ए॒नं॒ ॅव॒रु॒ण॒पा॒शाद् व॑रुणपा॒शादे॑न मेनं ॅवरुणपा॒शात् । \newline
47. व॒रु॒ण॒पा॒शान् मु॑ञ्चति मुञ्चति वरुणपा॒शाद् व॑रुणपा॒शान् मु॑ञ्चति । \newline
48. व॒रु॒ण॒पा॒शादिति॑ वरुण - पा॒शात् । \newline
49. मु॒ञ्च॒ति॒ द॒धि॒क्राव्.ण्णा॑ दधि॒क्राव्.ण्णा॑ मुञ्चति मुञ्चति दधि॒क्राव्.ण्णा᳚ । \newline
50. द॒धि॒क्राव्.ण्णा॑ पुनाति पुनाति दधि॒क्राव्.ण्णा॑ दधि॒क्राव्.ण्णा॑ पुनाति । \newline
51. द॒धि॒क्राव्.ण्णेति॑ दधि - क्राव्.ण्णा᳚ । \newline
52. पु॒ना॒ति॒ हिर॑ण्यꣳ॒॒ हिर॑ण्यम् पुनाति पुनाति॒ हिर॑ण्यम् । \newline
53. हिर॑ण्य॒म् दक्षि॑णा॒ दक्षि॑णा॒ हिर॑ण्यꣳ॒॒ हिर॑ण्य॒म् दक्षि॑णा । \newline
54. दक्षि॑णा प॒वित्र॑म् प॒वित्र॒म् दक्षि॑णा॒ दक्षि॑णा प॒वित्र᳚म् । \newline
55. प॒वित्रं॒ ॅवै वै प॒वित्र॑म् प॒वित्रं॒ ॅवै । \newline
56. वै हिर॑ण्यꣳ॒॒ हिर॑ण्यं॒ ॅवै वै हिर॑ण्यम् । \newline
57. हिर॑ण्यम् पु॒नाति॑ पु॒नाति॒ हिर॑ण्यꣳ॒॒ हिर॑ण्यम् पु॒नाति॑ । \newline
58. पु॒ना त्ये॒वैव पु॒नाति॑ पु॒ना त्ये॒व । \newline
59. ए॒वैन॑ मेन मे॒वैवैन᳚म् । \newline
60. ए॒नं॒ ॅवि॒न्दते॑ वि॒न्दत॑ एन मेनं ॅवि॒न्दते᳚ । \newline

\textbf{Ghana Paata } \newline

1. वा ए॒त स्यै॒तस्य॒ वै वा ए॒तस्या शा॒न्तो ऽशा᳚न्त ए॒तस्य॒ वै वा ए॒तस्या शा᳚न्तः । \newline
2. ए॒तस्या शा॒न्तो ऽशा᳚न्त ए॒त स्यै॒तस्या शा᳚न्तो॒ योनिं॒ ॅयोनि॒ मशा᳚न्त ए॒त स्यै॒तस्या शा᳚न्तो॒ योनि᳚म् । \newline
3. अशा᳚न्तो॒ योनिं॒ ॅयोनि॒ मशा॒न्तो ऽशा᳚न्तो॒ योनि॑म् प्र॒जायै᳚ प्र॒जायै॒ योनि॒ मशा॒न्तो ऽशा᳚न्तो॒ योनि॑म् प्र॒जायै᳚ । \newline
4. योनि॑म् प्र॒जायै᳚ प्र॒जायै॒ योनिं॒ ॅयोनि॑म् प्र॒जायै॑ पशू॒नाम् प॑शू॒नाम् प्र॒जायै॒ योनिं॒ ॅयोनि॑म् प्र॒जायै॑ पशू॒नाम् । \newline
5. प्र॒जायै॑ पशू॒नाम् प॑शू॒नाम् प्र॒जायै᳚ प्र॒जायै॑ पशू॒नाम् निर् णिष् प॑शू॒नाम् प्र॒जायै᳚ प्र॒जायै॑ पशू॒नाम् निः । \newline
6. प्र॒जाया॒ इति॑ प्र - जायै᳚ । \newline
7. प॒शू॒नाम् निर् णिष् प॑शू॒नाम् प॑शू॒नाम् निर् द॑हति दहति॒ निष् प॑शू॒नाम् प॑शू॒नाम् निर् द॑हति । \newline
8. निर् द॑हति दहति॒ निर् णिर् द॑हति॒ यो यो द॑हति॒ निर् णिर् द॑हति॒ यः । \newline
9. द॒ह॒ति॒ यो यो द॑हति दहति॒ यो ऽल॒ मलं॒ ॅयो द॑हति दहति॒ यो ऽल᳚म् । \newline
10. यो ऽल॒ मलं॒ ॅयो यो ऽल॑म् प्र॒जायै᳚ प्र॒जाया॒ अलं॒ ॅयो यो ऽल॑म् प्र॒जायै᳚ । \newline
11. अल॑म् प्र॒जायै᳚ प्र॒जाया॒ अल॒ मल॑म् प्र॒जायै॒ सन् थ्सन् प्र॒जाया॒ अल॒ मल॑म् प्र॒जायै॒ सन्न् । \newline
12. प्र॒जायै॒ सन् थ्सन् प्र॒जायै᳚ प्र॒जायै॒ सन् प्र॒जाम् प्र॒जाꣳ सन् प्र॒जायै᳚ प्र॒जायै॒ सन् प्र॒जाम् । \newline
13. प्र॒जाया॒ इति॑ प्र - जायै᳚ । \newline
14. सन् प्र॒जाम् प्र॒जाꣳ सन् थ्सन् प्र॒जाम् न न प्र॒जाꣳ सन् थ्सन् प्र॒जाम् न । \newline
15. प्र॒जाम् न न प्र॒जाम् प्र॒जाम् न वि॒न्दते॑ वि॒न्दते॒ न प्र॒जाम् प्र॒जाम् न वि॒न्दते᳚ । \newline
16. प्र॒जामिति॑ प्र - जाम् । \newline
17. न वि॒न्दते॑ वि॒न्दते॒ न न वि॒न्दते॒ यद् यद् वि॒न्दते॒ न न वि॒न्दते॒ यत् । \newline
18. वि॒न्दते॒ यद् यद् वि॒न्दते॑ वि॒न्दते॒ यद् वै᳚श्वान॒रो वै᳚श्वान॒रो यद् वि॒न्दते॑ वि॒न्दते॒ यद् वै᳚श्वान॒रः । \newline
19. यद् वै᳚श्वान॒रो वै᳚श्वान॒रो यद् यद् वै᳚श्वान॒रो द्वाद॑शकपालो॒ द्वाद॑शकपालो वैश्वान॒रो यद् यद् वै᳚श्वान॒रो द्वाद॑शकपालः । \newline
20. वै॒श्वा॒न॒रो द्वाद॑शकपालो॒ द्वाद॑शकपालो वैश्वान॒रो वै᳚श्वान॒रो द्वाद॑शकपालो॒ भव॑ति॒ भव॑ति॒ द्वाद॑शकपालो वैश्वान॒रो वै᳚श्वान॒रो द्वाद॑शकपालो॒ भव॑ति । \newline
21. द्वाद॑शकपालो॒ भव॑ति॒ भव॑ति॒ द्वाद॑शकपालो॒ द्वाद॑शकपालो॒ भव॑ति संॅवथ्स॒रः सं॑ॅवथ्स॒रो भव॑ति॒ द्वाद॑शकपालो॒ द्वाद॑शकपालो॒ भव॑ति संॅवथ्स॒रः । \newline
22. द्वाद॑शकपाल॒ इति॒ द्वाद॑श - क॒पा॒लः॒ । \newline
23. भव॑ति संॅवथ्स॒रः सं॑ॅवथ्स॒रो भव॑ति॒ भव॑ति संॅवथ्स॒रो वै वै सं॑ॅवथ्स॒रो भव॑ति॒ भव॑ति संॅवथ्स॒रो वै । \newline
24. सं॒ॅव॒थ्स॒रो वै वै सं॑ॅवथ्स॒रः सं॑ॅवथ्स॒रो वा अ॒ग्नि र॒ग्निर् वै सं॑ॅवथ्स॒रः सं॑ॅवथ्स॒रो वा अ॒ग्निः । \newline
25. सं॒ॅव॒थ्स॒र इति॑ सं - व॒थ्स॒रः । \newline
26. वा अ॒ग्नि र॒ग्निर् वै वा अ॒ग्निर् वै᳚श्वान॒रो वै᳚श्वान॒रो᳚ ऽग्निर् वै वा अ॒ग्निर् वै᳚श्वान॒रः । \newline
27. अ॒ग्निर् वै᳚श्वान॒रो वै᳚श्वान॒रो᳚ ऽग्नि र॒ग्निर् वै᳚श्वान॒रः सं॑ॅवथ्स॒रꣳ सं॑ॅवथ्स॒रं ॅवै᳚श्वान॒रो᳚ ऽग्नि र॒ग्निर् वै᳚श्वान॒रः सं॑ॅवथ्स॒रम् । \newline
28. वै॒श्वा॒न॒रः सं॑ॅवथ्स॒रꣳ सं॑ॅवथ्स॒रं ॅवै᳚श्वान॒रो वै᳚श्वान॒रः सं॑ॅवथ्स॒र मे॒वैव सं॑ॅवथ्स॒रं ॅवै᳚श्वान॒रो वै᳚श्वान॒रः सं॑ॅवथ्स॒र मे॒व । \newline
29. सं॒ॅव॒थ्स॒र मे॒वैव सं॑ॅवथ्स॒रꣳ सं॑ॅवथ्स॒र मे॒व भा॑ग॒धेये॑न भाग॒धेये॑नै॒व सं॑ॅवथ्स॒रꣳ सं॑ॅवथ्स॒र मे॒व भा॑ग॒धेये॑न । \newline
30. सं॒ॅव॒थ्स॒रमिति॑ सं - व॒थ्स॒रम् । \newline
31. ए॒व भा॑ग॒धेये॑न भाग॒धेये॑नै॒वैव भा॑ग॒धेये॑न शमयति शमयति भाग॒धेये॑नै॒वैव भा॑ग॒धेये॑न शमयति । \newline
32. भा॒ग॒धेये॑न शमयति शमयति भाग॒धेये॑न भाग॒धेये॑न शमयति॒ स स श॑मयति भाग॒धेये॑न भाग॒धेये॑न शमयति॒ सः । \newline
33. भा॒ग॒धेये॒नेति॑ भाग - धेये॑न । \newline
34. श॒म॒य॒ति॒ स स श॑मयति शमयति॒ सो᳚ ऽस्मा अस्मै॒ स श॑मयति शमयति॒ सो᳚ ऽस्मै । \newline
35. सो᳚ ऽस्मा अस्मै॒ स सो᳚ ऽस्मै शा॒न्तः शा॒न्तो᳚ ऽस्मै॒ स सो᳚ ऽस्मै शा॒न्तः । \newline
36. अ॒स्मै॒ शा॒न्तः शा॒न्तो᳚ ऽस्मा अस्मै शा॒न्तः स्वाथ् स्वाच् छा॒न्तो᳚ ऽस्मा अस्मै शा॒न्तः स्वात् । \newline
37. शा॒न्तः स्वाथ् स्वाच् छा॒न्तः शा॒न्तः स्वाद् योने॒र् योनेः॒ स्वाच् छा॒न्तः शा॒न्तः स्वाद् योनेः᳚ । \newline
38. स्वाद् योने॒र् योनेः॒ स्वाथ् स्वाद् योनेः᳚ प्र॒जाम् प्र॒जां ॅयोनेः॒ स्वाथ् स्वाद् योनेः᳚ प्र॒जाम् । \newline
39. योनेः᳚ प्र॒जाम् प्र॒जां ॅयोने॒र् योनेः᳚ प्र॒जाम् प्र प्र प्र॒जां ॅयोने॒र् योनेः᳚ प्र॒जाम् प्र । \newline
40. प्र॒जाम् प्र प्र प्र॒जाम् प्र॒जाम् प्र ज॑नयति जनयति॒ प्र प्र॒जाम् प्र॒जाम् प्र ज॑नयति । \newline
41. प्र॒जामिति॑ प्र - जाम् । \newline
42. प्र ज॑नयति जनयति॒ प्र प्र ज॑नयति वारु॒णेन॑ वारु॒णेन॑ जनयति॒ प्र प्र ज॑नयति वारु॒णेन॑ । \newline
43. ज॒न॒य॒ति॒ वा॒रु॒णेन॑ वारु॒णेन॑ जनयति जनयति वारु॒णे नै॒वैव वा॑रु॒णेन॑ जनयति जनयति वारु॒णे नै॒व । \newline
44. वा॒रु॒णे नै॒वैव वा॑रु॒णेन॑ वारु॒णे नै॒वैन॑ मेन मे॒व वा॑रु॒णेन॑ वारु॒णे नै॒वैन᳚म् । \newline
45. ए॒वैन॑ मेन मे॒वैवैनं॑ ॅवरुणपा॒शाद् व॑रुणपा॒शा दे॑न मे॒वैवैनं॑ ॅवरुणपा॒शात् । \newline
46. ए॒नं॒ ॅव॒रु॒ण॒पा॒शाद् व॑रुणपा॒शा दे॑न मेनं ॅवरुणपा॒शान् मु॑ञ्चति मुञ्चति वरुणपा॒शा दे॑न मेनं ॅवरुणपा॒शान् मु॑ञ्चति । \newline
47. व॒रु॒ण॒पा॒शान् मु॑ञ्चति मुञ्चति वरुणपा॒शाद् व॑रुणपा॒शान् मु॑ञ्चति दधि॒क्राव्.ण्णा॑ दधि॒क्राव्.ण्णा॑ मुञ्चति वरुणपा॒शाद् व॑रुणपा॒शान् मु॑ञ्चति दधि॒क्राव्.ण्णा᳚ । \newline
48. व॒रु॒ण॒पा॒शादिति॑ वरुण - पा॒शात् । \newline
49. मु॒ञ्च॒ति॒ द॒धि॒क्राव्.ण्णा॑ दधि॒क्राव्.ण्णा॑ मुञ्चति मुञ्चति दधि॒क्राव्.ण्णा॑ पुनाति पुनाति दधि॒क्राव्.ण्णा॑ मुञ्चति मुञ्चति दधि॒क्राव्.ण्णा॑ पुनाति । \newline
50. द॒धि॒क्राव्.ण्णा॑ पुनाति पुनाति दधि॒क्राव्.ण्णा॑ दधि॒क्राव्.ण्णा॑ पुनाति॒ हिर॑ण्यꣳ॒॒ हिर॑ण्यम् पुनाति दधि॒क्राव्.ण्णा॑ दधि॒क्राव्.ण्णा॑ पुनाति॒ हिर॑ण्यम् । \newline
51. द॒धि॒क्राव्.ण्णेति॑ दधि - क्राव्.ण्णा᳚ । \newline
52. पु॒ना॒ति॒ हिर॑ण्यꣳ॒॒ हिर॑ण्यम् पुनाति पुनाति॒ हिर॑ण्य॒म् दक्षि॑णा॒ दक्षि॑णा॒ हिर॑ण्यम् पुनाति पुनाति॒ हिर॑ण्य॒म् दक्षि॑णा । \newline
53. हिर॑ण्य॒म् दक्षि॑णा॒ दक्षि॑णा॒ हिर॑ण्यꣳ॒॒ हिर॑ण्य॒म् दक्षि॑णा प॒वित्र॑म् प॒वित्र॒म् दक्षि॑णा॒ हिर॑ण्यꣳ॒॒ हिर॑ण्य॒म् दक्षि॑णा प॒वित्र᳚म् । \newline
54. दक्षि॑णा प॒वित्र॑म् प॒वित्र॒म् दक्षि॑णा॒ दक्षि॑णा प॒वित्रं॒ ॅवै वै प॒वित्र॒म् दक्षि॑णा॒ दक्षि॑णा प॒वित्रं॒ ॅवै । \newline
55. प॒वित्रं॒ ॅवै वै प॒वित्र॑म् प॒वित्रं॒ ॅवै हिर॑ण्यꣳ॒॒ हिर॑ण्यं॒ ॅवै प॒वित्र॑म् प॒वित्रं॒ ॅवै हिर॑ण्यम् । \newline
56. वै हिर॑ण्यꣳ॒॒ हिर॑ण्यं॒ ॅवै वै हिर॑ण्यम् पु॒नाति॑ पु॒नाति॒ हिर॑ण्यं॒ ॅवै वै हिर॑ण्यम् पु॒नाति॑ । \newline
57. हिर॑ण्यम् पु॒नाति॑ पु॒नाति॒ हिर॑ण्यꣳ॒॒ हिर॑ण्यम् पु॒ना त्ये॒वैव पु॒नाति॒ हिर॑ण्यꣳ॒॒ हिर॑ण्यम् पु॒नात्ये॒व । \newline
58. पु॒ना त्ये॒वैव पु॒नाति॑ पु॒ना त्ये॒वैन॑ मेन मे॒व पु॒नाति॑ पु॒ना त्ये॒वैन᳚म् । \newline
59. ए॒वैन॑ मेन मे॒वैवैनं॑ ॅवि॒न्दते॑ वि॒न्दत॑ एन मे॒वैवैनं॑ ॅवि॒न्दते᳚ । \newline
60. ए॒नं॒ ॅवि॒न्दते॑ वि॒न्दत॑ एन मेनं ॅवि॒न्दते᳚ प्र॒जाम् प्र॒जां ॅवि॒न्दत॑ एन मेनं ॅवि॒न्दते᳚ प्र॒जाम् । \newline
\pagebreak
\markright{ TS 2.2.5.3  \hfill https://www.vedavms.in \hfill}
\addcontentsline{toc}{section}{ TS 2.2.5.3 }
\section*{ TS 2.2.5.3 }

\textbf{TS 2.2.5.3 } \newline
\textbf{Samhita Paata} \newline

ॅवि॒न्दते᳚ प्र॒जां ॅवै᳚श्वान॒रं द्वाद॑शकपालं॒ निर्व॑पेत् पु॒त्रे जा॒तेयद॒ष्टाक॑पालो॒ भव॑ति गायत्रि॒यैवैनं॑ ब्रह्मवर्च॒सेन॑ पुनाति॒ यन्नव॑कपाल-स्त्रि॒वृतै॒वास्मि॒न् तेजो॑ दधाति॒ यद्-दश॑कपालो वि॒राजै॒वास्मि॑न्न॒न्नाद्यं॑ दधाति॒ यदेका॑दशकपाल- स्त्रि॒ष्टुभै॒वास्मि॑न्निन्द्रि॒यं द॑धाति॒ यद्-द्वाद॑शकपालो॒ जग॑त्यै॒वास्मि॑न् प॒शून् द॑धाति॒ यस्मि॑न् जा॒त ए॒तामिष्टिं॑ नि॒र्वप॑ति पू॒त- [  ] \newline

\textbf{Pada Paata} \newline

वि॒न्दते᳚ । प्र॒जामिति॑ प्र - जाम् । वै॒श्वा॒न॒रम् । द्वाद॑शकपाल॒मिति॒ द्वाद॑श - क॒पा॒ल॒म् । निरिति॑ । व॒पे॒त् । पु॒त्रे । जा॒ते । यत् । अ॒ष्टाक॑पाल॒ इत्य॒ष्टा - क॒पा॒लः॒ । भव॑ति । गा॒य॒त्रि॒या । ए॒व । ए॒न॒म् । ब्र॒ह्म॒व॒र्च॒सेनेति॑ ब्रह्म - व॒र्च॒सेन॑ । पु॒ना॒ति॒ । यत् । नव॑कपाल॒ इति॒ नव॑ - क॒पा॒लः॒ । त्रि॒वृतेति॑ त्रि - वृता᳚ । ए॒व । अ॒स्मि॒न्न् । तेजः॑ । द॒धा॒ति॒ । यत् । दश॑कपाल॒ इति॒ दश॑ - क॒पा॒लः॒ । वि॒राजेति॑ वि - राजा᳚ । ए॒व । अ॒स्मि॒न्न् । अ॒न्नाद्य॒मित्य॑न्न - अद्य᳚म् । द॒धा॒ति॒ । यत् । एका॑दशकपाल॒ इत्येका॑दश - क॒पा॒लः॒ । त्रि॒ष्टुभा᳚ । ए॒व । अ॒स्मि॒न्न् । इ॒न्द्रि॒यम् । द॒धा॒ति॒ । यत् । द्वाद॑शकपाल॒ इति॒ द्वाद॑श - क॒पा॒लः॒ । जग॑त्या । ए॒व । अ॒स्मि॒न्न् । प॒शून् । द॒धा॒ति॒ । यस्मिन्न्॑ । जा॒ते । ए॒ताम् । इष्टि᳚म् । नि॒र्वप॒तीति॑ निः - वप॑ति । पू॒तः ।  \newline


\textbf{Krama Paata} \newline

वि॒न्दते᳚ प्र॒जाम् । प्र॒जां ॅवै᳚श्वान॒रम् । प्र॒जामिति॑ प्र - जाम् । वै॒श्वा॒न॒रम् द्वाद॑शकपालम् । द्वाद॑शकपाल॒म् निः । द्वाद॑शकपाल॒मिति॒ द्वाद॑श - क॒पा॒ल॒म् । निर् व॑पेत् । व॒पे॒त्,पु॒त्रे । पु॒त्रे जा॒ते । जा॒ते यत् । यद॒ष्टाक॑पालः । अ॒ष्टाक॑पालो॒ भव॑ति । अ॒ष्टाक॑पाल॒ इत्य॒ष्टा - क॒पा॒लः॒ । भव॑ति गायत्रि॒या । गा॒य॒त्रि॒यैव । ए॒वैन᳚म् । ए॒न॒म् ब्र॒ह्म॒व॒र्च॒सेन॑ । ब्र॒ह्म॒व॒र्च॒सेन॑ पुनाति । ब्र॒ह्म॒व॒र्च॒सेनेति॑ ब्रह्म - व॒र्च॒सेन॑ । पु॒ना॒ति॒ यत् । यन्नव॑कपालः । नव॑कपालस्त्रि॒वृता᳚ । नव॑कपाल॒ इति॒ नव॑ - क॒पा॒लः॒ । त्रि॒वृतै॒व । त्रि॒वृतेति॑ त्रि - वृता᳚ । ए॒वास्मिन्न्॑ । अ॒स्मि॒न्,तेजः॑ । तेजो॑दधाति । द॒धा॒ति॒ यत् । यद् दश॑कपालः । दश॑कपालो वि॒राजा᳚ । दश॑कपाल॒ इति॒ दश॑ - क॒पा॒लः॒ । वि॒राजै॒व । वि॒राजेति॑ वि - राजा᳚ । ए॒वास्मिन्न्॑ । अ॒स्मि॒न्न॒न्नाद्य᳚म् । अ॒न्नाद्य॑म् दधाति । अ॒न्नाद्य॒मित्य॑न्न - अद्य᳚म् । द॒धा॒ति॒ यत् । यदेका॑दशकपालः । एका॑दशकपालस्त्रि॒ष्टुभा᳚ । एका॑दशकपाल॒ इत्येका॑दश - क॒पा॒लः॒ । त्रि॒ष्टुभै॒व । ए॒वास्मिन्न्॑ । अ॒स्मि॒न्नि॒न्द्रि॒यम् । इ॒न्द्रि॒यम् द॑धाति । द॒धा॒ति॒ यत् । यद् द्वाद॑शकपालः । द्वाद॑शकपालो॒ जग॑त्या । द्वाद॑शकपाल॒ इति॒ द्वाद॑श - क॒पा॒लः॒ । जग॑त्यै॒व । ए॒वास्मिन्न्॑ । अ॒स्मि॒न् प॒शून् । प॒शून् द॑धाति । द॒धा॒ति॒ यस्मिन्न्॑ । यस्मि॑न् जा॒ते । जा॒त ए॒ताम् । ए॒तामिष्टि᳚म् । इष्टि॑म् नि॒र्वप॑ति । नि॒र्वप॑ति पू॒तः । नि॒र्वप॒तीति॑ निः - वप॑ति । पू॒त ए॒व \newline

\textbf{Jatai Paata} \newline

1. वि॒न्दते᳚ प्र॒जाम् प्र॒जां ॅवि॒न्दते॑ वि॒न्दते᳚ प्र॒जाम् । \newline
2. प्र॒जां ॅवै᳚श्वान॒रं ॅवै᳚श्वान॒रम् प्र॒जाम् प्र॒जां ॅवै᳚श्वान॒रम् । \newline
3. प्र॒जामिति॑ प्र - जाम् । \newline
4. वै॒श्वा॒न॒रम् द्वाद॑शकपाल॒म् द्वाद॑शकपालं ॅवैश्वान॒रं ॅवै᳚श्वान॒रम् द्वाद॑शकपालम् । \newline
5. द्वाद॑शकपाल॒म् निर् णिर् द्वाद॑शकपाल॒म् द्वाद॑शकपाल॒म् निः । \newline
6. द्वाद॑शकपाल॒मिति॒ द्वाद॑श - क॒पा॒ल॒म् । \newline
7. निर् व॑पेद् वपे॒न् निर् णिर् व॑पेत् । \newline
8. व॒पे॒त् पु॒त्रे पु॒त्रे व॑पेद् वपेत् पु॒त्रे । \newline
9. पु॒त्रे जा॒ते जा॒ते पु॒त्रे पु॒त्रे जा॒ते । \newline
10. जा॒ते यद् यज् जा॒ते जा॒ते यत् । \newline
11. यद॒ष्टाक॑पालो॒ ऽष्टाक॑पालो॒ यद् यद॒ष्टाक॑पालः । \newline
12. अ॒ष्टाक॑पालो॒ भव॑ति॒ भव॑ त्य॒ष्टाक॑पालो॒ ऽष्टाक॑पालो॒ भव॑ति । \newline
13. अ॒ष्टाक॑पाल॒ इत्य॒ष्टा - क॒पा॒लः॒ । \newline
14. भव॑ति गायत्रि॒या गा॑यत्रि॒या भव॑ति॒ भव॑ति गायत्रि॒या । \newline
15. गा॒य॒त्रि॒यैवैव गा॑यत्रि॒या गा॑यत्रि॒यैव । \newline
16. ए॒वैन॑ मेन मे॒वैवैन᳚म् । \newline
17. ए॒न॒म् ब्र॒ह्म॒व॒र्च॒सेन॑ ब्रह्मवर्च॒सेनै॑न मेनम् ब्रह्मवर्च॒सेन॑ । \newline
18. ब्र॒ह्म॒व॒र्च॒सेन॑ पुनाति पुनाति ब्रह्मवर्च॒सेन॑ ब्रह्मवर्च॒सेन॑ पुनाति । \newline
19. ब्र॒ह्म॒व॒र्च॒सेनेति॑ ब्रह्म - व॒र्च॒सेन॑ । \newline
20. पु॒ना॒ति॒ यद् यत् पु॑नाति पुनाति॒ यत् । \newline
21. यन् नव॑कपालो॒ नव॑कपालो॒ यद् यन् नव॑कपालः । \newline
22. नव॑कपाल स्त्रि॒वृता᳚ त्रि॒वृता॒ नव॑कपालो॒ नव॑कपाल स्त्रि॒वृता᳚ । \newline
23. नव॑कपाल॒ इति॒ नव॑ - क॒पा॒लः॒ । \newline
24. त्रि॒वृतै॒वैव त्रि॒वृता᳚ त्रि॒वृतै॒व । \newline
25. त्रि॒वृतेति॑ त्रि - वृता᳚ । \newline
26. ए॒वास्मि॑न् नस्मिन् ने॒वैवास्मिन्न्॑ । \newline
27. अ॒स्मि॒न् तेज॒ स्तेजो᳚ ऽस्मिन् नस्मि॒न् तेजः॑ । \newline
28. तेजो॑ दधाति दधाति॒ तेज॒ स्तेजो॑ दधाति । \newline
29. द॒धा॒ति॒ यद् यद् द॑धाति दधाति॒ यत् । \newline
30. यद् दश॑कपालो॒ दश॑कपालो॒ यद् यद् दश॑कपालः । \newline
31. दश॑कपालो वि॒राजा॑ वि॒राजा॒ दश॑कपालो॒ दश॑कपालो वि॒राजा᳚ । \newline
32. दश॑कपाल॒ इति॒ दश॑ - क॒पा॒लः॒ । \newline
33. वि॒राजै॒वैव वि॒राजा॑ वि॒राजै॒व । \newline
34. वि॒राजेति॑ वि - राजा᳚ । \newline
35. ए॒वास्मि॑न् नस्मिन् ने॒वैवास्मिन्न्॑ । \newline
36. अ॒स्मि॒न् न॒न्नाद्य॑ म॒न्नाद्य॑ मस्मिन् नस्मिन् न॒न्नाद्य᳚म् । \newline
37. अ॒न्नाद्य॑म् दधाति दधा त्य॒न्नाद्य॑ म॒न्नाद्य॑म् दधाति । \newline
38. अ॒न्नाद्य॒मित्य॑न्न - अद्य᳚म् । \newline
39. द॒धा॒ति॒ यद् यद् द॑धाति दधाति॒ यत् । \newline
40. यदेका॑दशकपाल॒ एका॑दशकपालो॒ यद् यदेका॑दशकपालः । \newline
41. एका॑दशकपाल स्त्रि॒ष्टुभा᳚ त्रि॒ष्टु भैका॑दशकपाल॒ एका॑दशकपाल स्त्रि॒ष्टुभा᳚ । \newline
42. एका॑दशकपाल॒ इत्येका॑दश - क॒पा॒लः॒ । \newline
43. त्रि॒ष्टुभै॒वैव त्रि॒ष्टुभा᳚ त्रि॒ष्टुभै॒व । \newline
44. ए॒वास्मि॑न् नस्मिन् ने॒वैवास्मिन्न्॑ । \newline
45. अ॒स्मि॒न् नि॒न्द्रि॒य मि॑न्द्रि॒य म॑स्मिन् नस्मिन् निन्द्रि॒यम् । \newline
46. इ॒न्द्रि॒यम् द॑धाति दधातीन्द्रि॒य मि॑न्द्रि॒यम् द॑धाति । \newline
47. द॒धा॒ति॒ यद् यद् द॑धाति दधाति॒ यत् । \newline
48. यद् द्वाद॑शकपालो॒ द्वाद॑शकपालो॒ यद् यद् द्वाद॑शकपालः । \newline
49. द्वाद॑शकपालो॒ जग॑त्या॒ जग॑त्या॒ द्वाद॑शकपालो॒ द्वाद॑शकपालो॒ जग॑त्या । \newline
50. द्वाद॑शकपाल॒ इति॒ द्वाद॑श - क॒पा॒लः॒ । \newline
51. जग॑ त्यै॒वैव जग॑त्या॒ जग॑ त्यै॒व । \newline
52. ए॒वास्मि॑न् नस्मिन् ने॒वैवास्मिन्न्॑ । \newline
53. अ॒स्मि॒न् प॒शून् प॒शू न॑स्मिन् नस्मिन् प॒शून् । \newline
54. प॒शून् द॑धाति दधाति प॒शून् प॒शून् द॑धाति । \newline
55. द॒धा॒ति॒ यस्मि॒न्॒. यस्मि॑न् दधाति दधाति॒ यस्मिन्न्॑ । \newline
56. यस्मि॑न् जा॒ते जा॒ते यस्मि॒न्॒. यस्मि॑न् जा॒ते । \newline
57. जा॒त ए॒ता मे॒ताम् जा॒ते जा॒त ए॒ताम् । \newline
58. ए॒ता मिष्टि॒ मिष्टि॑ मे॒ता मे॒ता मिष्टि᳚म् । \newline
59. इष्टि॑म् नि॒र्वप॑ति नि॒र्वप॒तीष्टि॒ मिष्टि॑म् नि॒र्वप॑ति । \newline
60. नि॒र्वप॑ति पू॒तः पू॒तो नि॒र्वप॑ति नि॒र्वप॑ति पू॒तः । \newline
61. नि॒र्वप॒तीति॑ निः - वप॑ति । \newline
62. पू॒त ए॒वैव पू॒तः पू॒त ए॒व । \newline

\textbf{Ghana Paata } \newline

1. वि॒न्दते᳚ प्र॒जाम् प्र॒जां ॅवि॒न्दते॑ वि॒न्दते᳚ प्र॒जां ॅवै᳚श्वान॒रं ॅवै᳚श्वान॒रम् प्र॒जां ॅवि॒न्दते॑ वि॒न्दते᳚ प्र॒जां ॅवै᳚श्वान॒रम् । \newline
2. प्र॒जां ॅवै᳚श्वान॒रं ॅवै᳚श्वान॒रम् प्र॒जाम् प्र॒जां ॅवै᳚श्वान॒रम् द्वाद॑शकपाल॒म् द्वाद॑शकपालं ॅवैश्वान॒रम् प्र॒जाम् प्र॒जां ॅवै᳚श्वान॒रम् द्वाद॑शकपालम् । \newline
3. प्र॒जामिति॑ प्र - जाम् । \newline
4. वै॒श्वा॒न॒रम् द्वाद॑शकपाल॒म् द्वाद॑शकपालं ॅवैश्वान॒रं ॅवै᳚श्वान॒रम् द्वाद॑शकपाल॒म् निर् णिर् द्वाद॑शकपालं ॅवैश्वान॒रं ॅवै᳚श्वान॒रम् द्वाद॑शकपाल॒म् निः । \newline
5. द्वाद॑शकपाल॒म् निर् णिर् द्वाद॑शकपाल॒म् द्वाद॑शकपाल॒म् निर् व॑पेद् वपे॒न् निर् द्वाद॑शकपाल॒म् द्वाद॑शकपाल॒म् निर् व॑पेत् । \newline
6. द्वाद॑शकपाल॒मिति॒ द्वाद॑श - क॒पा॒ल॒म् । \newline
7. निर् व॑पेद् वपे॒न् निर् णिर् व॑पेत् पु॒त्रे पु॒त्रे व॑पे॒न् निर् णिर् व॑पेत् पु॒त्रे । \newline
8. व॒पे॒त् पु॒त्रे पु॒त्रे व॑पेद् वपेत् पु॒त्रे जा॒ते जा॒ते पु॒त्रे व॑पेद् वपेत् पु॒त्रे जा॒ते । \newline
9. पु॒त्रे जा॒ते जा॒ते पु॒त्रे पु॒त्रे जा॒ते यद् यज् जा॒ते पु॒त्रे पु॒त्रे जा॒ते यत् । \newline
10. जा॒ते यद् यज् जा॒ते जा॒ते यद॒ष्टाक॑पालो॒ ऽष्टाक॑पालो॒ यज् जा॒ते जा॒ते यद॒ष्टाक॑पालः । \newline
11. यद॒ष्टाक॑पालो॒ ऽष्टाक॑पालो॒ यद् यद॒ष्टाक॑पालो॒ भव॑ति॒ भव॑ त्य॒ष्टाक॑पालो॒ यद् यद॒ष्टाक॑पालो॒ भव॑ति । \newline
12. अ॒ष्टाक॑पालो॒ भव॑ति॒ भव॑ त्य॒ष्टाक॑पालो॒ ऽष्टाक॑पालो॒ भव॑ति गायत्रि॒या गा॑यत्रि॒या भव॑ त्य॒ष्टाक॑पालो॒ ऽष्टाक॑पालो॒ भव॑ति गायत्रि॒या । \newline
13. अ॒ष्टाक॑पाल॒ इत्य॒ष्टा - क॒पा॒लः॒ । \newline
14. भव॑ति गायत्रि॒या गा॑यत्रि॒या भव॑ति॒ भव॑ति गायत्रि॒यै वैव गा॑यत्रि॒या भव॑ति॒ भव॑ति गायत्रि॒यैव । \newline
15. गा॒य॒त्रि॒यै वैव गा॑यत्रि॒या गा॑यत्रि॒यै वैन॑ मेन मे॒व गा॑यत्रि॒या गा॑यत्रि॒यै वैन᳚म् । \newline
16. ए॒वैन॑ मेन मे॒वैवैन॑म् ब्रह्मवर्च॒सेन॑ ब्रह्मवर्च॒सेनै॑न मे॒वैवैन॑म् ब्रह्मवर्च॒सेन॑ । \newline
17. ए॒न॒म् ब्र॒ह्म॒व॒र्च॒सेन॑ ब्रह्मवर्च॒सेनै॑न मेनम् ब्रह्मवर्च॒सेन॑ पुनाति पुनाति ब्रह्मवर्च॒सेनै॑न मेनम् ब्रह्मवर्च॒सेन॑ पुनाति । \newline
18. ब्र॒ह्म॒व॒र्च॒सेन॑ पुनाति पुनाति ब्रह्मवर्च॒सेन॑ ब्रह्मवर्च॒सेन॑ पुनाति॒ यद् यत् पु॑नाति ब्रह्मवर्च॒सेन॑ ब्रह्मवर्च॒सेन॑ पुनाति॒ यत् । \newline
19. ब्र॒ह्म॒व॒र्च॒सेनेति॑ ब्रह्म - व॒र्च॒सेन॑ । \newline
20. पु॒ना॒ति॒ यद् यत् पु॑नाति पुनाति॒ यन् नव॑कपालो॒ नव॑कपालो॒ यत् पु॑नाति पुनाति॒ यन् नव॑कपालः । \newline
21. यन् नव॑कपालो॒ नव॑कपालो॒ यद् यन् नव॑कपाल स्त्रि॒वृता᳚ त्रि॒वृता॒ नव॑कपालो॒ यद् यन् नव॑कपाल स्त्रि॒वृता᳚ । \newline
22. नव॑कपाल स्त्रि॒वृता᳚ त्रि॒वृता॒ नव॑कपालो॒ नव॑कपाल स्त्रि॒वृतै॒वैव त्रि॒वृता॒ नव॑कपालो॒ नव॑कपाल स्त्रि॒वृतै॒व । \newline
23. नव॑कपाल॒ इति॒ नव॑ - क॒पा॒लः॒ । \newline
24. त्रि॒वृतै॒वैव त्रि॒वृता᳚ त्रि॒वृतै॒वास्मि॑न् नस्मिन् ने॒व त्रि॒वृता᳚ त्रि॒वृतै॒वास्मिन्न्॑ । \newline
25. त्रि॒वृतेति॑ त्रि - वृता᳚ । \newline
26. ए॒वास्मि॑न् नस्मिन् ने॒वैवास्मि॒न् तेज॒ स्तेजो᳚ ऽस्मिन् ने॒वैवास्मि॒न् तेजः॑ । \newline
27. अ॒स्मि॒न् तेज॒ स्तेजो᳚ ऽस्मिन् नस्मि॒न् तेजो॑ दधाति दधाति॒ तेजो᳚ ऽस्मिन् नस्मि॒न् तेजो॑ दधाति । \newline
28. तेजो॑ दधाति दधाति॒ तेज॒ स्तेजो॑ दधाति॒ यद् यद् द॑धाति॒ तेज॒ स्तेजो॑ दधाति॒ यत् । \newline
29. द॒धा॒ति॒ यद् यद् द॑धाति दधाति॒ यद् दश॑कपालो॒ दश॑कपालो॒ यद् द॑धाति दधाति॒ यद् दश॑कपालः । \newline
30. यद् दश॑कपालो॒ दश॑कपालो॒ यद् यद् दश॑कपालो वि॒राजा॑ वि॒राजा॒ दश॑कपालो॒ यद् यद् दश॑कपालो वि॒राजा᳚ । \newline
31. दश॑कपालो वि॒राजा॑ वि॒राजा॒ दश॑कपालो॒ दश॑कपालो वि॒राजै॒वैव वि॒राजा॒ दश॑कपालो॒ दश॑कपालो वि॒राजै॒व । \newline
32. दश॑कपाल॒ इति॒ दश॑ - क॒पा॒लः॒ । \newline
33. वि॒रा जै॒वैव वि॒राजा॑ वि॒रा जै॒वास्मि॑न् नस्मिन् ने॒व वि॒राजा॑ वि॒रा जै॒वास्मिन्न्॑ । \newline
34. वि॒राजेति॑ वि - राजा᳚ । \newline
35. ए॒वास्मि॑न् नस्मिन् ने॒वैवास्मि॑न् न॒न्नाद्य॑ म॒न्नाद्य॑ मस्मिन् ने॒वैवास्मि॑न् न॒न्नाद्य᳚म् । \newline
36. अ॒स्मि॒न् न॒न्नाद्य॑ म॒न्नाद्य॑ मस्मिन् नस्मिन् न॒न्नाद्य॑म् दधाति दधा त्य॒न्नाद्य॑ मस्मिन् नस्मिन् न॒न्नाद्य॑म् दधाति । \newline
37. अ॒न्नाद्य॑म् दधाति दधा त्य॒न्नाद्य॑ म॒न्नाद्य॑म् दधाति॒ यद् यद् द॑धा त्य॒न्नाद्य॑ म॒न्नाद्य॑म् दधाति॒ यत् । \newline
38. अ॒न्नाद्य॒मित्य॑न्न - अद्य᳚म् । \newline
39. द॒धा॒ति॒ यद् यद् द॑धाति दधाति॒ यदेका॑दशकपाल॒ एका॑दशकपालो॒ यद् द॑धाति दधाति॒ यदेका॑दशकपालः । \newline
40. यदेका॑दशकपाल॒ एका॑दशकपालो॒ यद् यदेका॑दशकपाल स्त्रि॒ष्टुभा᳚ त्रि॒ष्टुभैका॑दशकपालो॒ यद् यदेका॑दशकपाल स्त्रि॒ष्टुभा᳚ । \newline
41. एका॑दशकपाल स्त्रि॒ष्टुभा᳚ त्रि॒ष्टुभैका॑दशकपाल॒ एका॑दशकपाल स्त्रि॒ष्टुभै॒ वैव त्रि॒ष्टुभैका॑दशकपाल॒ एका॑दशकपाल स्त्रि॒ष्टुभै॒व । \newline
42. एका॑दशकपाल॒ इत्येका॑दश - क॒पा॒लः॒ । \newline
43. त्रि॒ष्टुभै॒वैव त्रि॒ष्टुभा᳚ त्रि॒ष्टुभै॒वास्मि॑न् नस्मिन् ने॒व त्रि॒ष्टुभा᳚ त्रि॒ष्टुभै॒वास्मिन्न्॑ । \newline
44. ए॒वास्मि॑न् नस्मिन् ने॒वैवास्मि॑न् निन्द्रि॒य मि॑न्द्रि॒य म॑स्मिन् ने॒वैवास्मि॑न् निन्द्रि॒यम् । \newline
45. अ॒स्मि॒न् नि॒न्द्रि॒य मि॑न्द्रि॒य म॑स्मिन् नस्मिन् निन्द्रि॒यम् द॑धाति दधातीन्द्रि॒य म॑स्मिन् नस्मिन् निन्द्रि॒यम् द॑धाति । \newline
46. इ॒न्द्रि॒यम् द॑धाति दधातीन्द्रि॒य मि॑न्द्रि॒यम् द॑धाति॒ यद् यद् द॑धातीन्द्रि॒य मि॑न्द्रि॒यम् द॑धाति॒ यत् । \newline
47. द॒धा॒ति॒ यद् यद् द॑धाति दधाति॒ यद् द्वाद॑शकपालो॒ द्वाद॑शकपालो॒ यद् द॑धाति दधाति॒ यद् द्वाद॑शकपालः । \newline
48. यद् द्वाद॑शकपालो॒ द्वाद॑शकपालो॒ यद् यद् द्वाद॑शकपालो॒ जग॑त्या॒ जग॑त्या॒ द्वाद॑शकपालो॒ यद् यद् द्वाद॑शकपालो॒ जग॑त्या । \newline
49. द्वाद॑शकपालो॒ जग॑त्या॒ जग॑त्या॒ द्वाद॑शकपालो॒ द्वाद॑शकपालो॒ जग॑ त्यै॒वैव जग॑त्या॒ द्वाद॑शकपालो॒ द्वाद॑शकपालो॒ जग॑ त्यै॒व । \newline
50. द्वाद॑शकपाल॒ इति॒ द्वाद॑श - क॒पा॒लः॒ । \newline
51. जग॑त्यै॒वैव जग॑त्या॒ जग॑ त्यै॒वास्मि॑न् नस्मिन् ने॒व जग॑त्या॒ जग॑ त्यै॒वास्मिन्न्॑ । \newline
52. ए॒वास्मि॑न् नस्मिन् ने॒वैवास्मि॑न् प॒शून् प॒शू न॑स्मिन् ने॒वैवास्मि॑न् प॒शून् । \newline
53. अ॒स्मि॒न् प॒शून् प॒शू न॑स्मिन् नस्मिन् प॒शून् द॑धाति दधाति प॒शू न॑स्मिन् नस्मिन् प॒शून् द॑धाति । \newline
54. प॒शून् द॑धाति दधाति प॒शून् प॒शून् द॑धाति॒ यस्मि॒न्॒. यस्मि॑न् दधाति प॒शून् प॒शून् द॑धाति॒ यस्मिन्न्॑ । \newline
55. द॒धा॒ति॒ यस्मि॒न्॒. यस्मि॑न् दधाति दधाति॒ यस्मि॑न् जा॒ते जा॒ते यस्मि॑न् दधाति दधाति॒ यस्मि॑न् जा॒ते । \newline
56. यस्मि॑न् जा॒ते जा॒ते यस्मि॒न्॒. यस्मि॑न् जा॒त ए॒ता मे॒ताम् जा॒ते यस्मि॒न्॒. यस्मि॑न् जा॒त ए॒ताम् । \newline
57. जा॒त ए॒ता मे॒ताम् जा॒ते जा॒त ए॒ता मिष्टि॒ मिष्टि॑ मे॒ताम् जा॒ते जा॒त ए॒ता मिष्टि᳚म् । \newline
58. ए॒ता मिष्टि॒ मिष्टि॑ मे॒ता मे॒ता मिष्टि॑म् नि॒र्वप॑ति नि॒र्वप॒तीष्टि॑ मे॒ता मे॒ता मिष्टि॑म् नि॒र्वप॑ति । \newline
59. इष्टि॑म् नि॒र्वप॑ति नि॒र्वप॒तीष्टि॒ मिष्टि॑म् नि॒र्वप॑ति पू॒तः पू॒तो नि॒र्वप॒तीष्टि॒ मिष्टि॑म् नि॒र्वप॑ति पू॒तः । \newline
60. नि॒र्वप॑ति पू॒तः पू॒तो नि॒र्वप॑ति नि॒र्वप॑ति पू॒त ए॒वैव पू॒तो नि॒र्वप॑ति नि॒र्वप॑ति पू॒त ए॒व । \newline
61. नि॒र्वप॒तीति॑ निः - वप॑ति । \newline
62. पू॒त ए॒वैव पू॒तः पू॒त ए॒व ते॑ज॒स्वी ते॑ज॒ स्व्ये॑व पू॒तः पू॒त ए॒व ते॑ज॒स्वी । \newline
\pagebreak
\markright{ TS 2.2.5.4  \hfill https://www.vedavms.in \hfill}
\addcontentsline{toc}{section}{ TS 2.2.5.4 }
\section*{ TS 2.2.5.4 }

\textbf{TS 2.2.5.4 } \newline
\textbf{Samhita Paata} \newline

ए॒व ते॑ज॒स्व्य॑न्ना॒द इ॑न्द्रिया॒वी प॑शु॒मान् भ॑व॒त्यव॒ वा ए॒ष सु॑व॒र्गाल्लो॒काच्छि॑द्यते॒ यो द॑र्.शपूर्णमासया॒जी सन्न॑मावा॒स्यां᳚ ॅवा पौर्णमा॒सीं ॅवा॑तिपा॒दय॑ति सुव॒र्गाय॒ हि लो॒काय॑ दर्.शपूर्णमा॒सा वि॒ज्येते॑ वैश्वान॒रं द्वाद॑शकपालं॒ निर्व॑पेदमावा॒स्यां᳚ ॅवा पौर्णमा॒सीं ॅवा॑ऽति॒पाद्य॑ संॅवथ्स॒रो वा अ॒ग्नि र्वै᳚श्वान॒रः सं॑ॅवथ्स॒रमे॒व प्री॑णा॒त्यथो॑ संॅवथ्स॒रमे॒वास्मा॒ उप॑ दधाति सुव॒र्गस्य॑ लो॒कस्य॒ सम॑ष्ट्या॒-  [  ] \newline

\textbf{Pada Paata} \newline

ए॒व । ते॒ज॒स्वी । अ॒न्ना॒द इत्य॑न्न - अ॒दः । इ॒न्द्रि॒या॒वी । प॒शु॒मानिति॑ पशु - मान् । भ॒व॒ति॒ । अवेति॑ । वै । ए॒षः । सु॒व॒र्गादिति॑ सुवः - गात् । लो॒कात् । छि॒द्य॒ते॒ । यः । द॒र्॒.श॒पू॒र्ण॒मा॒स॒या॒जीति॑ दर्.शपूर्णमास - या॒जी । सन्न् । अ॒मा॒वा॒स्या॑मित्य॑मा - वा॒स्या᳚म् । वा॒ । पौ॒र्ण॒मा॒सीमिति॑ पौर्ण - मा॒सीम् । वा॒ । अ॒ति॒पा॒दय॒तीत्य॑ति - पा॒दय॑ति । सु॒व॒र्गायेति॑ सुवः - गाय॑ । हि । लो॒काय॑ । द॒र्॒.श॒पू॒र्ण॒मा॒साविति॑ दर्.श - पू॒र्ण॒मा॒सौ । इ॒ज्येते॒ इति॑ । वै॒श्वा॒न॒रम् । द्वाद॑शकपाल॒मिति॒ द्वाद॑श - क॒पा॒ल॒म् । निरिति॑ । व॒पे॒त् । अ॒मा॒वा॒स्या॑मित्य॑मा - वा॒स्या᳚म् । वा॒ । पौ॒र्ण॒मा॒सीमिति॑ पौर्ण - मा॒सीम् । वा॒ । अ॒ति॒पाद्येत्य॑ति - पाद्य॑ । सं॒ॅव॒थ्स॒र इति॑ सं - व॒थ्स॒रः । वै । अ॒ग्निः । वै॒श्वा॒न॒रः । सं॒ॅव॒थ्स॒रमिति॑ सं - व॒थ्स॒रम् । ए॒व । प्री॒णा॒ति॒ । अथो॒ इति॑ । सं॒ॅव॒थ्स॒रमिति॑ सं - व॒थ्स॒रम् । ए॒व । अ॒स्मै॒ । उपेति॑ । द॒धा॒ति॒ । सु॒व॒र्गस्येति॑ सुवः - गस्य॑ । लो॒कस्य॑ । सम॑ष्ट्या॒ इति॒ सं - अ॒ष्ट्यै॒ ।  \newline


\textbf{Krama Paata} \newline

ए॒व ते॑ज॒स्वी । ते॒ज॒स्व्य॑न्ना॒दः । अ॒न्ना॒द इ॑न्द्रिया॒वी । अ॒न्ना॒द इत्य॑न्न - अ॒दः । इ॒न्द्रि॒या॒वी प॑शु॒मान् । प॒शु॒मान् भ॑वति । प॒शु॒मानिति॑ पशु - मान् । भ॒व॒त्यव॑ । अव॒ वै । वा ए॒षः । ए॒ष सु॑व॒र्गात् । सु॒व॒र्गाल्लो॒कात् । सु॒व॒र्गादिति॑ सुवः - गात् । लो॒काच्छि॑द्यते । छि॒द्य॒ते॒ यः । यो द॑र्.शपूर्णमासया॒जी । द॒र्॒.श॒पू॒र्ण॒मा॒स॒या॒जी सन्न् । द॒र्॒.श॒पू॒र्ण॒मा॒स॒या॒जीति॑ दर्.शपूर्णमास - या॒जी । सन्न॑मावा॒स्या᳚म् । अ॒मा॒वा॒स्यां᳚ ॅवा । अ॒मा॒वा॒स्या॑मित्य॑मा - वा॒स्या᳚म् । वा॒ पौ॒र्ण॒मा॒सीम् । पौ॒र्ण॒मा॒सीं ॅवा᳚ । पौ॒र्ण॒मा॒सीमिति॑ पौर्ण - मा॒सीम् । वा॒ ऽति॒पा॒दय॑ति । अ॒ति॒पा॒दय॑ति सुव॒र्गाय॑ । अ॒ति॒पा॒दय॒तीत्य॑ति - पा॒दय॑ति । सु॒व॒र्गाय॒ हि । सु॒व॒र्गायेति॑ सुवः - गाय॑ । हि लो॒काय॑ । लो॒काय॑ दर्.शपूर्णमा॒सौ । द॒र्॒.श॒पू॒र्ण॒मा॒सा वि॒ज्येते᳚ । द॒र्॒श॒पू॒र्ण॒मा॒साविति॑ दर्.श - पू॒र्ण॒मा॒सौ । इ॒ज्येते॑ वैश्वान॒रम् । इ॒ज्येते॒ इती॒ज्येते᳚ । वै॒श्वा॒न॒रं द्वाद॑शकपालम् । द्वाद॑शकपाल॒म् निः । द्वाद॑शकपाल॒मिति॒ द्वाद॑श - क॒पा॒ल॒म् । निर् व॑पेत् । व॒पे॒द॒मा॒वा॒स्या᳚म् । अ॒मा॒वा॒स्यां᳚ ॅवा । अ॒मा॒वा॒स्या॑मित्य॑मा - वा॒स्या᳚म् । वा॒ पौ॒र्ण॒मा॒सीम् । पौ॒र्ण॒मा॒सीं ॅवा᳚ । पौ॒र्ण॒मा॒सीमिति॑ पौर्ण - मा॒सीम् । वा॒ ऽति॒पाद्य॑ । अ॒ति॒पाद्य॑ सम्ॅवथ्स॒रः । अ॒ति॒पाद्येत्य॑ति - पाद्य॑ । स॒म्ॅव॒थ्स॒रो वै । स॒म्ॅव॒थ्स॒र इति॑ सं - व॒थ्स॒रः । वा अ॒ग्निः । अ॒ग्निर् वै᳚श्वान॒रः । वै॒श्वा॒न॒रः स॑म्ॅवथ्स॒रम् । स॒म्ॅव॒थ्स॒रमे॒व । स॒म्ॅव॒थ्स॒रमिति॑ सं - व॒थ्स॒रम् । ए॒व प्री॑णाति । प्री॒णा॒त्यथो᳚ । अथो॑ सम्ॅवथ्स॒रम् । अथो॒ इत्यथो᳚ । स॒म्ॅव॒थ्स॒रमे॒व । स॒म्ॅव॒थ्स॒रमिति॑ सं - व॒थ्स॒रम् । ए॒वास्मै᳚ । अ॒स्मा॒ उप॑ । उप॑ दधाति । द॒धा॒ति॒ सु॒व॒र्गस्य॑ । सु॒व॒र्गस्य॑ लो॒कस्य॑ । सु॒व॒र्गस्येति॑ सुवः - गस्य॑ । लो॒कस्य॒ सम॑ष्ट्यै । सम॑ष्ट्या॒ अथो᳚ । सम॑ष्ट्या॒ इति॒ सं - अ॒ष्ट्यै॒ \newline

\textbf{Jatai Paata} \newline

1. ए॒व ते॑ज॒स्वी ते॑ज॒स्व्ये॑वैव ते॑ज॒स्वी । \newline
2. ते॒ज॒स्व्य॑न्ना॒दो᳚ ऽन्ना॒द स्ते॑ज॒स्वी ते॑ज॒ स्व्य॑न्ना॒दः । \newline
3. अ॒न्ना॒द इ॑न्द्रिया॒वी न्द्रि॑या॒ व्य॑न्ना॒दो᳚ ऽन्ना॒द इ॑न्द्रिया॒वी । \newline
4. अ॒न्ना॒द इत्य॑न्न - अ॒दः । \newline
5. इ॒न्द्रि॒या॒वी प॑शु॒मान् प॑शु॒मा नि॑न्द्रिया॒वी न्द्रि॑या॒वी प॑शु॒मान् । \newline
6. प॒शु॒मान् भ॑वति भवति पशु॒मान् प॑शु॒मान् भ॑वति । \newline
7. प॒शु॒मानिति॑ पशु - मान् । \newline
8. भ॒व॒ त्यवाव॑ भवति भव॒ त्यव॑ । \newline
9. अव॒ वै वा अवाव॒ वै । \newline
10. वा ए॒ष ए॒ष वै वा ए॒षः । \newline
11. ए॒ष सु॑व॒र्गाथ् सु॑व॒र्गा दे॒ष ए॒ष सु॑व॒र्गात् । \newline
12. सु॒व॒र्गा ल्लो॒का ल्लो॒काथ् सु॑व॒र्गाथ् सु॑व॒र्गा ल्लो॒कात् । \newline
13. सु॒व॒र्गादिति॑ सुवः - गात् । \newline
14. लो॒काच् छि॑द्यते छिद्यते लो॒का ल्लो॒काच् छि॑द्यते । \newline
15. छि॒द्य॒ते॒ यो य श्छि॑द्यते छिद्यते॒ यः । \newline
16. यो द॑र्.शपूर्णमासया॒जी द॑र्.शपूर्णमासया॒जी यो यो द॑र्.शपूर्णमासया॒जी । \newline
17. द॒र्॒.श॒पू॒र्ण॒मा॒स॒या॒जी सन् थ्सन् द॑र्.शपूर्णमासया॒जी द॑र्.शपूर्णमासया॒जी सन्न् । \newline
18. द॒र्॒.श॒पू॒र्ण॒मा॒स॒या॒जीति॑ दर्.शपूर्णमास - या॒जी । \newline
19. सन् न॑मावा॒स्या॑ ममावा॒स्याꣳ॑ सन् थ्सन् न॑मावा॒स्या᳚म् । \newline
20. अ॒मा॒वा॒स्यां᳚ ॅवा वा ऽमावा॒स्या॑ ममावा॒स्यां᳚ ॅवा । \newline
21. अ॒मा॒वा॒स्या॑मित्य॑मा - वा॒स्या᳚म् । \newline
22. वा॒ पौ॒र्ण॒मा॒सीम् पौ᳚र्णमा॒सीं ॅवा॑ वा पौर्णमा॒सीम् । \newline
23. पौ॒र्ण॒मा॒सीं ॅवा॑ वा पौर्णमा॒सीम् पौ᳚र्णमा॒सीं ॅवा᳚ । \newline
24. पौ॒र्ण॒मा॒सीमिति॑ पौर्ण - मा॒सीम् । \newline
25. वा॒ ऽति॒पा॒दय॑ त्यतिपा॒दय॑ति वा वा ऽतिपा॒दय॑ति । \newline
26. अ॒ति॒पा॒दय॑ति सुव॒र्गाय॑ सुव॒र्गाया॑ तिपा॒दय॑ त्यतिपा॒दय॑ति सुव॒र्गाय॑ । \newline
27. अ॒ति॒पा॒दय॒तीत्य॑ति - पा॒दय॑ति । \newline
28. सु॒व॒र्गाय॒ हि हि सु॑व॒र्गाय॑ सुव॒र्गाय॒ हि । \newline
29. सु॒व॒र्गायेति॑ सुवः - गाय॑ । \newline
30. हि लो॒काय॑ लो॒काय॒ हि हि लो॒काय॑ । \newline
31. लो॒काय॑ दर्.शपूर्णमा॒सौ द॑र्.शपूर्णमा॒सौ लो॒काय॑ लो॒काय॑ दर्.शपूर्णमा॒सौ । \newline
32. द॒र्॒.श॒पू॒र्ण॒मा॒सा वि॒ज्येते॑ इ॒ज्येते॑ दर्.शपूर्णमा॒सौ द॑र्.शपूर्णमा॒सा वि॒ज्येते᳚ । \newline
33. द॒र्॒.श॒पू॒र्ण॒मा॒साविति॑ दर्.श - पू॒र्ण॒मा॒सौ । \newline
34. इ॒ज्येते॑ वैश्वान॒रं ॅवै᳚श्वान॒र मि॒ज्येते॑ इ॒ज्येते॑ वैश्वान॒रम् । \newline
35. इ॒ज्येते॒ इती॒ज्येते᳚ । \newline
36. वै॒श्वा॒न॒रम् द्वाद॑शकपाल॒म् द्वाद॑शकपालं ॅवैश्वान॒रं ॅवै᳚श्वान॒रम् द्वाद॑शकपालम् । \newline
37. द्वाद॑शकपाल॒म् निर् णिर् द्वाद॑शकपाल॒म् द्वाद॑शकपाल॒म् निः । \newline
38. द्वाद॑शकपाल॒मिति॒ द्वाद॑श - क॒पा॒ल॒म् । \newline
39. निर् व॑पेद् वपे॒न् निर् णिर् व॑पेत् । \newline
40. व॒पे॒द॒मा॒वा॒स्या॑ ममावा॒स्यां᳚ ॅवपेद् वपेदमावा॒स्या᳚म् । \newline
41. अ॒मा॒वा॒स्यां᳚ ॅवा वा ऽमावा॒स्या॑ ममावा॒स्यां᳚ ॅवा । \newline
42. अ॒मा॒वा॒स्या॑मित्य॑मा - वा॒स्या᳚म् । \newline
43. वा॒ पौ॒र्ण॒मा॒सीम् पौ᳚र्णमा॒सीं ॅवा॑ वा पौर्णमा॒सीम् । \newline
44. पौ॒र्ण॒मा॒सीं ॅवा॑ वा पौर्णमा॒सीम् पौ᳚र्णमा॒सीं ॅवा᳚ । \newline
45. पौ॒र्ण॒मा॒सीमिति॑ पौर्ण - मा॒सीम् । \newline
46. वा॒ ऽति॒पाद्या॑ ति॒पाद्य॑ वा वा ऽति॒पाद्य॑ । \newline
47. अ॒ति॒पाद्य॑ संॅवथ्स॒रः सं॑ॅवथ्स॒रो॑ ऽति॒पाद्या॑ति॒पाद्य॑ संॅवथ्स॒रः । \newline
48. अ॒ति॒पाद्येत्य॑ति - पाद्य॑ । \newline
49. सं॒ॅव॒थ्स॒रो वै वै सं॑ॅवथ्स॒रः सं॑ॅवथ्स॒रो वै । \newline
50. सं॒ॅव॒थ्स॒र इति॑ सं - व॒थ्स॒रः । \newline
51. वा अ॒ग्नि र॒ग्निर् वै वा अ॒ग्निः । \newline
52. अ॒ग्निर् वै᳚श्वान॒रो वै᳚श्वान॒रो᳚ ऽग्नि र॒ग्निर् वै᳚श्वान॒रः । \newline
53. वै॒श्वा॒न॒रः सं॑ॅवथ्स॒रꣳ सं॑ॅवथ्स॒रं ॅवै᳚श्वान॒रो वै᳚श्वान॒रः सं॑ॅवथ्स॒रम् । \newline
54. सं॒ॅव॒थ्स॒र मे॒वैव सं॑ॅवथ्स॒रꣳ सं॑ॅवथ्स॒र मे॒व । \newline
55. सं॒ॅव॒थ्स॒रमिति॑ सं - व॒थ्स॒रम् । \newline
56. ए॒व प्री॑णाति प्रीणा त्ये॒वैव प्री॑णाति । \newline
57. प्री॒णा॒ त्यथो॒ अथो᳚ प्रीणाति प्रीणा॒ त्यथो᳚ । \newline
58. अथो॑ संॅवथ्स॒रꣳ सं॑ॅवथ्स॒र मथो॒ अथो॑ संॅवथ्स॒रम् । \newline
59. अथो॒ इत्यथो᳚ । \newline
60. सं॒ॅव॒थ्स॒र मे॒वैव सं॑ॅवथ्स॒रꣳ सं॑ॅवथ्स॒र मे॒व । \newline
61. सं॒ॅव॒थ्स॒रमिति॑ सं - व॒थ्स॒रम् । \newline
62. ए॒वास्मा॑ अस्मा ए॒वैवास्मै᳚ । \newline
63. अ॒स्मा॒ उपोपा᳚स्मा अस्मा॒ उप॑ । \newline
64. उप॑ दधाति दधा॒ त्युपोप॑ दधाति । \newline
65. द॒धा॒ति॒ सु॒व॒र्गस्य॑ सुव॒र्गस्य॑ दधाति दधाति सुव॒र्गस्य॑ । \newline
66. सु॒व॒र्गस्य॑ लो॒कस्य॑ लो॒कस्य॑ सुव॒र्गस्य॑ सुव॒र्गस्य॑ लो॒कस्य॑ । \newline
67. सु॒व॒र्गस्येति॑ सुवः - गस्य॑ । \newline
68. लो॒कस्य॒ सम॑ष्ट्यै॒ सम॑ष्ट्यै लो॒कस्य॑ लो॒कस्य॒ सम॑ष्ट्यै । \newline
69. सम॑ष्ट्या॒ अथो॒ अथो॒ सम॑ष्ट्यै॒ सम॑ष्ट्या॒ अथो᳚ । \newline
70. सम॑ष्ट्या॒ इति॒ सं - अ॒ष्ट्यै॒ । \newline

\textbf{Ghana Paata } \newline

1. ए॒व ते॑ज॒स्वी ते॑ज॒ स्व्ये॑वैव ते॑ज॒ स्व्य॑न्ना॒दो᳚ ऽन्ना॒द स्ते॑ज॒ स्व्ये॑वैव ते॑ज॒ स्व्य॑न्ना॒दः । \newline
2. ते॒ज॒ स्व्य॑न्ना॒दो᳚ ऽन्ना॒द स्ते॑ज॒स्वी ते॑ज॒ स्व्य॑न्ना॒द इ॑न्द्रिया॒वी न्द्रि॑या॒ व्य॑न्ना॒द स्ते॑ज॒स्वी ते॑ज॒ स्व्य॑न्ना॒द इ॑न्द्रिया॒वी । \newline
3. अ॒न्ना॒द इ॑न्द्रिया॒वी न्द्रि॑या॒ व्य॑न्ना॒दो᳚ ऽन्ना॒द इ॑न्द्रिया॒वी प॑शु॒मान् प॑शु॒मा नि॑न्द्रिया॒ व्य॑न्ना॒दो᳚ ऽन्ना॒द इ॑न्द्रिया॒वी प॑शु॒मान् । \newline
4. अ॒न्ना॒द इत्य॑न्न - अ॒दः । \newline
5. इ॒न्द्रि॒या॒वी प॑शु॒मान् प॑शु॒मा नि॑न्द्रिया॒वी न्द्रि॑या॒वी प॑शु॒मान् भ॑वति भवति पशु॒मा नि॑न्द्रिया॒वी न्द्रि॑या॒वी प॑शु॒मान् भ॑वति । \newline
6. प॒शु॒मान् भ॑वति भवति पशु॒मान् प॑शु॒मान् भ॑व॒ त्यवाव॑ भवति पशु॒मान् प॑शु॒मान् भ॑व॒ त्यव॑ । \newline
7. प॒शु॒मानिति॑ पशु - मान् । \newline
8. भ॒व॒ त्यवाव॑ भवति भव॒त्यव॒ वै वा अव॑ भवति भव॒ त्यव॒ वै । \newline
9. अव॒ वै वा अवाव॒ वा ए॒ष ए॒ष वा अवाव॒ वा ए॒षः । \newline
10. वा ए॒ष ए॒ष वै वा ए॒ष सु॑व॒र्गाथ् सु॑व॒र्गादे॒ष वै वा ए॒ष सु॑व॒र्गात् । \newline
11. ए॒ष सु॑व॒र्गाथ् सु॑व॒र्गादे॒ष ए॒ष सु॑व॒र्गा ल्लो॒का ल्लो॒काथ् सु॑व॒र्गादे॒ष ए॒ष सु॑व॒र्गा ल्लो॒कात् । \newline
12. सु॒व॒र्गा ल्लो॒का ल्लो॒काथ् सु॑व॒र्गाथ् सु॑व॒र्गा ल्लो॒काच् छि॑द्यते छिद्यते लो॒काथ् सु॑व॒र्गाथ् सु॑व॒र्गा ल्लो॒काच् छि॑द्यते । \newline
13. सु॒व॒र्गादिति॑ सुवः - गात् । \newline
14. लो॒काच् छि॑द्यते छिद्यते लो॒का ल्लो॒काच् छि॑द्यते॒ यो य श्छि॑द्यते लो॒का ल्लो॒काच् छि॑द्यते॒ यः । \newline
15. छि॒द्य॒ते॒ यो य श्छि॑द्यते छिद्यते॒ यो द॑र्.शपूर्णमासया॒जी द॑र्.शपूर्णमासया॒जी य श्छि॑द्यते छिद्यते॒ यो द॑र्.शपूर्णमासया॒जी । \newline
16. यो द॑र्.शपूर्णमासया॒जी द॑र्.शपूर्णमासया॒जी यो यो द॑र्.शपूर्णमासया॒जी सन् थ्सन् द॑र्.शपूर्णमासया॒जी यो यो द॑र्.शपूर्णमासया॒जी सन्न् । \newline
17. द॒र्॒.श॒पू॒र्ण॒मा॒स॒या॒जी सन् थ्सन् द॑र्.शपूर्णमासया॒जी द॑र्.शपूर्णमासया॒जी सन् न॑मावा॒स्या॑ ममावा॒स्याꣳ॑ सन् द॑र्.शपूर्णमासया॒जी द॑र्.शपूर्णमासया॒जी सन् न॑मावा॒स्या᳚म् । \newline
18. द॒र्॒.श॒पू॒र्ण॒मा॒स॒या॒जीति॑ दर्.शपूर्णमास - या॒जी । \newline
19. सन् न॑मावा॒स्या॑ ममावा॒स्याꣳ॑ सन् थ्सन् न॑मावा॒स्यां᳚ ॅवा वा ऽमावा॒स्याꣳ॑ सन् थ्सन् न॑मावा॒स्यां᳚ ॅवा । \newline
20. अ॒मा॒वा॒स्यां᳚ ॅवा वा ऽमावा॒स्या॑ ममावा॒स्यां᳚ ॅवा पौर्णमा॒सीम् पौ᳚र्णमा॒सीं ॅवा॑ ऽमावा॒स्या॑ ममावा॒स्यां᳚ ॅवा पौर्णमा॒सीम् । \newline
21. अ॒मा॒वा॒स्या॑मित्य॑मा - वा॒स्या᳚म् । \newline
22. वा॒ पौ॒र्ण॒मा॒सीम् पौ᳚र्णमा॒सीं ॅवा॑ वा पौर्णमा॒सीं ॅवा॑ वा पौर्णमा॒सीं ॅवा॑ वा पौर्णमा॒सीं ॅवा᳚ । \newline
23. पौ॒र्ण॒मा॒सीं ॅवा॑ वा पौर्णमा॒सीम् पौ᳚र्णमा॒सीं ॅवा॑ ऽतिपा॒दय॑ त्यतिपा॒दय॑ति वा पौर्णमा॒सीम् पौ᳚र्णमा॒सीं ॅवा॑ ऽतिपा॒दय॑ति । \newline
24. पौ॒र्ण॒मा॒सीमिति॑ पौर्ण - मा॒सीम् । \newline
25. वा॒ ऽति॒पा॒दय॑ त्यतिपा॒दय॑ति वा वा ऽतिपा॒दय॑ति सुव॒र्गाय॑ सुव॒र्गाया॑ तिपा॒दय॑ति वा वा ऽतिपा॒दय॑ति सुव॒र्गाय॑ । \newline
26. अ॒ति॒पा॒दय॑ति सुव॒र्गाय॑ सुव॒र्गाया॑ तिपा॒दय॑ त्यतिपा॒दय॑ति सुव॒र्गाय॒ हि हि सु॑व॒र्गाया॑ तिपा॒दय॑ त्यतिपा॒दय॑ति सुव॒र्गाय॒ हि । \newline
27. अ॒ति॒पा॒दय॒तीत्य॑ति - पा॒दय॑ति । \newline
28. सु॒व॒र्गाय॒ हि हि सु॑व॒र्गाय॑ सुव॒र्गाय॒ हि लो॒काय॑ लो॒काय॒ हि सु॑व॒र्गाय॑ सुव॒र्गाय॒ हि लो॒काय॑ । \newline
29. सु॒व॒र्गायेति॑ सुवः - गाय॑ । \newline
30. हि लो॒काय॑ लो॒काय॒ हि हि लो॒काय॑ दर्.शपूर्णमा॒सौ द॑र्.शपूर्णमा॒सौ लो॒काय॒ हि हि लो॒काय॑ दर्.शपूर्णमा॒सौ । \newline
31. लो॒काय॑ दर्.शपूर्णमा॒सौ द॑र्.शपूर्णमा॒सौ लो॒काय॑ लो॒काय॑ दर्.शपूर्णमा॒सा वि॒ज्येते॑ इ॒ज्येते॑ दर्.शपूर्णमा॒सौ लो॒काय॑ लो॒काय॑ दर्.शपूर्णमा॒सा वि॒ज्येते᳚ । \newline
32. द॒र्॒.श॒पू॒र्ण॒मा॒सा वि॒ज्येते॑ इ॒ज्येते॑ दर्.शपूर्णमा॒सौ द॑र्.शपूर्णमा॒सा वि॒ज्येते॑ वैश्वान॒रं ॅवै᳚श्वान॒र मि॒ज्येते॑ दर्.शपूर्णमा॒सौ द॑र्.शपूर्णमा॒सा वि॒ज्येते॑ वैश्वान॒रम् । \newline
33. द॒र्॒.श॒पू॒र्ण॒मा॒साविति॑ दर्.श - पू॒र्ण॒मा॒सौ । \newline
34. इ॒ज्येते॑ वैश्वान॒रं ॅवै᳚श्वान॒र मि॒ज्येते॑ इ॒ज्येते॑ वैश्वान॒रम् द्वाद॑शकपाल॒म् द्वाद॑शकपालं ॅवैश्वान॒र मि॒ज्येते॑ इ॒ज्येते॑ वैश्वान॒रम् द्वाद॑शकपालम् । \newline
35. इ॒ज्येते॒ इती॒ज्येते᳚ । \newline
36. वै॒श्वा॒न॒रम् द्वाद॑शकपाल॒म् द्वाद॑शकपालं ॅवैश्वान॒रं ॅवै᳚श्वान॒रम् द्वाद॑शकपाल॒म् निर् णिर् द्वाद॑शकपालं ॅवैश्वान॒रं ॅवै᳚श्वान॒रम् द्वाद॑शकपाल॒म् निः । \newline
37. द्वाद॑शकपाल॒म् निर् णिर् द्वाद॑शकपाल॒म् द्वाद॑शकपाल॒म् निर् व॑पेद् वपे॒न् निर् द्वाद॑शकपाल॒म् द्वाद॑शकपाल॒म् निर् व॑पेत् । \newline
38. द्वाद॑शकपाल॒मिति॒ द्वाद॑श - क॒पा॒ल॒म् । \newline
39. निर् व॑पेद् वपे॒न् निर् णिर् व॑पेदमावा॒स्या॑ ममावा॒स्यां᳚ ॅवपे॒न् निर् णिर् व॑पेदमावा॒स्या᳚म् । \newline
40. व॒पे॒द॒मा॒वा॒स्या॑ ममावा॒स्यां᳚ ॅवपेद् वपेदमावा॒स्यां᳚ ॅवा वा ऽमावा॒स्यां᳚ ॅवपेद् वपेदमावा॒स्यां᳚ ॅवा । \newline
41. अ॒मा॒वा॒स्यां᳚ ॅवा वा ऽमावा॒स्या॑ ममावा॒स्यां᳚ ॅवा पौर्णमा॒सीम् पौ᳚र्णमा॒सीं ॅवा॑ ऽमावा॒स्या॑ ममावा॒स्यां᳚ ॅवा पौर्णमा॒सीम् । \newline
42. अ॒मा॒वा॒स्या॑मित्य॑मा - वा॒स्या᳚म् । \newline
43. वा॒ पौ॒र्ण॒मा॒सीम् पौ᳚र्णमा॒सीं ॅवा॑ वा पौर्णमा॒सीं ॅवा॑ वा पौर्णमा॒सीं ॅवा॑ वा पौर्णमा॒सीं ॅवा᳚ । \newline
44. पौ॒र्ण॒मा॒सीं ॅवा॑ वा पौर्णमा॒सीम् पौ᳚र्णमा॒सीं ॅवा॑ ऽति॒पाद्या॑ति॒पाद्य॑ वा पौर्णमा॒सीम् पौ᳚र्णमा॒सीं ॅवा॑ ऽति॒पाद्य॑ । \newline
45. पौ॒र्ण॒मा॒सीमिति॑ पौर्ण - मा॒सीम् । \newline
46. वा॒ ऽति॒पाद्या॑ ति॒पाद्य॑ वा वा ऽति॒पाद्य॑ संॅवथ्स॒रः सं॑ॅवथ्स॒रो॑ ऽति॒पाद्य॑ वा वा ऽति॒पाद्य॑ संॅवथ्स॒रः । \newline
47. अ॒ति॒पाद्य॑ संॅवथ्स॒रः सं॑ॅवथ्स॒रो॑ ऽति॒पाद्या॑ ति॒पाद्य॑ संॅवथ्स॒रो वै वै सं॑ॅवथ्स॒रो॑ ऽति॒पाद्या॑ ति॒पाद्य॑ संॅवथ्स॒रो वै । \newline
48. अ॒ति॒पाद्येत्य॑ति - पाद्य॑ । \newline
49. सं॒ॅव॒थ्स॒रो वै वै सं॑ॅवथ्स॒रः सं॑ॅवथ्स॒रो वा अ॒ग्नि र॒ग्निर् वै सं॑ॅवथ्स॒रः सं॑ॅवथ्स॒रो वा अ॒ग्निः । \newline
50. सं॒ॅव॒थ्स॒र इति॑ सं - व॒थ्स॒रः । \newline
51. वा अ॒ग्नि र॒ग्निर् वै वा अ॒ग्निर् वै᳚श्वान॒रो वै᳚श्वान॒रो᳚ ऽग्निर् वै वा अ॒ग्निर् वै᳚श्वान॒रः । \newline
52. अ॒ग्निर् वै᳚श्वान॒रो वै᳚श्वान॒रो᳚ ऽग्निर॒ग्निर् वै᳚श्वान॒रः सं॑ॅवथ्स॒रꣳ सं॑ॅवथ्स॒रं ॅवै᳚श्वान॒रो᳚ ऽग्नि र॒ग्निर् वै᳚श्वान॒रः सं॑ॅवथ्स॒रम् । \newline
53. वै॒श्वा॒न॒रः सं॑ॅवथ्स॒रꣳ सं॑ॅवथ्स॒रं ॅवै᳚श्वान॒रो वै᳚श्वान॒रः सं॑ॅवथ्स॒र मे॒वैव सं॑ॅवथ्स॒रं ॅवै᳚श्वान॒रो वै᳚श्वान॒रः सं॑ॅवथ्स॒र मे॒व । \newline
54. सं॒ॅव॒थ्स॒र मे॒वैव सं॑ॅवथ्स॒रꣳ सं॑ॅवथ्स॒र मे॒व प्री॑णाति प्रीणात्ये॒व सं॑ॅवथ्स॒रꣳ सं॑ॅवथ्स॒र मे॒व प्री॑णाति । \newline
55. सं॒ॅव॒थ्स॒रमिति॑ सं - व॒थ्स॒रम् । \newline
56. ए॒व प्री॑णाति प्रीणा त्ये॒वैव प्री॑णा॒त्यथो॒ अथो᳚ प्रीणा त्ये॒वैव प्री॑णा॒ त्यथो᳚ । \newline
57. प्री॒णा॒ त्यथो॒ अथो᳚ प्रीणाति प्रीणा॒ त्यथो॑ संॅवथ्स॒रꣳ सं॑ॅवथ्स॒र मथो᳚ प्रीणाति प्रीणा॒ त्यथो॑ संॅवथ्स॒रम् । \newline
58. अथो॑ संॅवथ्स॒रꣳ सं॑ॅवथ्स॒र मथो॒ अथो॑ संॅवथ्स॒र मे॒वैव सं॑ॅवथ्स॒र मथो॒ अथो॑ संॅवथ्स॒र मे॒व । \newline
59. अथो॒ इत्यथो᳚ । \newline
60. सं॒ॅव॒थ्स॒र मे॒वैव सं॑ॅवथ्स॒रꣳ सं॑ॅवथ्स॒र मे॒वास्मा॑ अस्मा ए॒व सं॑ॅवथ्स॒रꣳ सं॑ॅवथ्स॒र मे॒वास्मै᳚ । \newline
61. सं॒ॅव॒थ्स॒रमिति॑ सं - व॒थ्स॒रम् । \newline
62. ए॒वास्मा॑ अस्मा ए॒वैवास्मा॒ उपोपा᳚स्मा ए॒वैवास्मा॒ उप॑ । \newline
63. अ॒स्मा॒ उपोपा᳚स्मा अस्मा॒ उप॑ दधाति दधा॒ त्युपा᳚स्मा अस्मा॒ उप॑ दधाति । \newline
64. उप॑ दधाति दधा॒ त्युपोप॑ दधाति सुव॒र्गस्य॑ सुव॒र्गस्य॑ दधा॒ त्युपोप॑ दधाति सुव॒र्गस्य॑ । \newline
65. द॒धा॒ति॒ सु॒व॒र्गस्य॑ सुव॒र्गस्य॑ दधाति दधाति सुव॒र्गस्य॑ लो॒कस्य॑ लो॒कस्य॑ सुव॒र्गस्य॑ दधाति दधाति सुव॒र्गस्य॑ लो॒कस्य॑ । \newline
66. सु॒व॒र्गस्य॑ लो॒कस्य॑ लो॒कस्य॑ सुव॒र्गस्य॑ सुव॒र्गस्य॑ लो॒कस्य॒ सम॑ष्ट्यै॒ सम॑ष्ट्यै लो॒कस्य॑ सुव॒र्गस्य॑ सुव॒र्गस्य॑ लो॒कस्य॒ सम॑ष्ट्यै । \newline
67. सु॒व॒र्गस्येति॑ सुवः - गस्य॑ । \newline
68. लो॒कस्य॒ सम॑ष्ट्यै॒ सम॑ष्ट्यै लो॒कस्य॑ लो॒कस्य॒ सम॑ष्ट्या॒ अथो॒ अथो॒ सम॑ष्ट्यै लो॒कस्य॑ लो॒कस्य॒ सम॑ष्ट्या॒ अथो᳚ । \newline
69. सम॑ष्ट्या॒ अथो॒ अथो॒ सम॑ष्ट्यै॒ सम॑ष्ट्या॒ अथो॑ दे॒वता॑ दे॒वता॒ अथो॒ सम॑ष्ट्यै॒ सम॑ष्ट्या॒ अथो॑ दे॒वताः᳚ । \newline
70. सम॑ष्ट्या॒ इति॒ सं - अ॒ष्ट्यै॒ । \newline
\pagebreak
\markright{ TS 2.2.5.5  \hfill https://www.vedavms.in \hfill}
\addcontentsline{toc}{section}{ TS 2.2.5.5 }
\section*{ TS 2.2.5.5 }

\textbf{TS 2.2.5.5 } \newline
\textbf{Samhita Paata} \newline

अथो॑ दे॒वता॑ ए॒वान्वा॒रभ्य॑ सुव॒र्गं ॅलो॒कमे॑ति वीर॒हा वा ए॒ष दे॒वानां॒ ॅयो᳚ऽग्नि-मु॑द्वा॒सय॑ते॒ न वा ए॒तस्य॑ ब्राह्म॒णा ऋ॑ता॒यवः॑ पु॒राऽन्न॑मक्षन्ना-ग्ने॒यम॒ष्टाक॑पालं॒ निर्व॑पेद्-वैश्वान॒रं द्वाद॑शकपाल-म॒ग्निमु॑द्वासयि॒ष्यन्. यद॒ष्टाक॑पालो॒ भव॑त्य॒ष्टाक्ष॑रा गाय॒त्रीगा॑य॒त्रो᳚ ऽग्निर्यावा॑ने॒वाग्निस्तस्मा॑ आति॒थ्यं क॑रो॒त्यथो॒ यथा॒ जनं॑ ॅय॒ते॑ऽव॒सं क॒रोति॑ ता॒दृ - [  ] \newline

\textbf{Pada Paata} \newline

अथो॒ इति॑ । दे॒वताः᳚ । ए॒व । अ॒न्वा॒रभ्येत्य॑नु - आ॒रभ्य॑ । सु॒व॒र्गमिति॑ सुवः - गम् । लो॒कम् । ए॒ति॒ । वी॒र॒हेति॑ वीर-हा । वै । ए॒षः । दे॒वाना᳚म् । यः । अ॒ग्निम् । उ॒द्वा॒सय॑त॒ इत्यु॑त् - वा॒सय॑ते । न । वै । ए॒तस्य॑ । ब्रा॒ह्म॒णाः । ऋ॒ता॒यव॒ इत्यृ॑त - यवः॑ । पु॒रा । अन्न᳚म् । अ॒क्ष॒न्न् । आ॒ग्ने॒यम् । अ॒ष्टाक॑पाल॒मित्य॒ष्टा - क॒पा॒ल॒म् । निरिति॑ । व॒पे॒त् । वै॒श्वा॒न॒रम् । द्वाद॑शकपाल॒मिति॒ द्वाद॑श-क॒पा॒ल॒म् । अ॒ग्निम् । उ॒द्वा॒स॒यि॒ष्यन्नित्यु॑त् - वा॒स॒यि॒ष्यन्न् । यत् । अ॒ष्टाक॑पाल॒ इत्य॒ष्टा - क॒पा॒लः॒ । भव॑ति । अ॒ष्टाक्ष॒रेत्य॒ष्टा - अ॒क्ष॒रा॒ । गा॒य॒त्री । गा॒य॒त्रः । अ॒ग्निः । यावान्॑ । ए॒व । अ॒ग्निः । तस्मै᳚ । आ॒ति॒थ्यम् । क॒रो॒ति॒ । अथो॒ इति॑ । यथा᳚ । जन᳚म् । य॒ते । अ॒व॒सम् । क॒रोति॑ । ता॒दृक् ।  \newline


\textbf{Krama Paata} \newline

अथो॑ दे॒वताः᳚ । अथो॒ इत्यथो᳚ । दे॒वता॑ ए॒व । ए॒वान्वा॒रभ्य॑ । अ॒न्वा॒रभ्य॑ सुव॒र्गम् । अ॒न्वा॒रभ्येत्य॑नु - आ॒रभ्य॑ । सु॒व॒र्गं ॅलो॒कम् । सु॒व॒र्गमिति॑ सुवः - गम् । लो॒कमे॑ति । ए॒ति॒ वी॒र॒हा । वी॒र॒हा वै । वी॒र॒हेति॑ वीर - हा । वा ए॒षः । ए॒ष दे॒वाना᳚म् । दे॒वानां॒ ॅयः । यो᳚ऽग्निम् । अ॒ग्नमु॑द्वा॒सय॑ते । उ॒द्वा॒सय॑ते॒ न । उ॒द्वा॒सय॑त॒ इत्यु॑त् - वा॒सय॑ते । न वै । वा ए॒तस्य॑ । ए॒तस्य॑ ब्राह्म॒णाः । ब्रा॒ह्म॒णा ऋ॑ता॒यवः॑ । ऋ॒ता॒यवः॑ पु॒रा । ऋ॒ता॒यव॒ इत्यृ॑त - यवः॑ । पु॒राऽन्न᳚म् । अन्न॑मक्षन्न् । अ॒क्ष॒न्ना॒ग्ने॒यम् । आ॒ग्ने॒यम॒ष्टाक॑पालम् । अ॒ष्टाक॑पाल॒म् निः । अ॒ष्टाक॑पाल॒मित्य॒ष्टा - क॒पा॒ल॒म् । निर् व॑पेत् । व॒पे॒द् वै॒श्वा॒न॒रम् । वै॒श्वा॒न॒रम् द्वाद॑शकपालम् । द्वाद॑शकपालम॒ग्निम् । द्वाद॑शकपाल॒मिति॒ द्वाद॑श - क॒पा॒ल॒म् । अ॒ग्निमु॑द्वासयि॒ष्यन्न् । उ॒द्वा॒स॒यि॒ष्यन्न् यत् । उ॒द्वा॒स॒यि॒ष्यन्नित्यु॑त् - वा॒स॒यि॒ष्यन्न् । यद॒ष्टाक॑पालः । अ॒ष्टाक॑पालो॒ भव॑ति । अ॒ष्टाक॑पाल॒ इत्य॒ष्टा - क॒पा॒लः॒ । भव॑त्य॒ष्टाक्ष॑रा । अ॒ष्टाक्ष॑रा गाय॒त्री । अ॒ष्टाक्ष॒रेत्य॒ष्टा - अ॒क्ष॒रा॒ । गा॒य॒त्री गा॑य॒त्रः । गा॒य॒त्रो᳚ऽग्निः । अ॒ग्निर् यावान्॑ । यावा॑ने॒व । ए॒वाग्निः । अ॒ग्निस्तस्मै᳚ । तस्मा॑ आति॒थ्यम् । आ॒ति॒थ्यम् क॑रोति । क॒रो॒त्यथो᳚ । अथो॒ यथा᳚ । अथो॒ इत्यथो᳚ । यथा॒ जन᳚म् । जनं॑ ॅय॒ते । य॒ते॑ऽव॒सम् । अ॒व॒सम् क॒रोति॑ । क॒रोति॑ ता॒दृक् । ता॒दृगे॒व \newline

\textbf{Jatai Paata} \newline

1. अथो॑ दे॒वता॑ दे॒वता॒ अथो॒ अथो॑ दे॒वताः᳚ । \newline
2. अथो॒ इत्यथो᳚ । \newline
3. दे॒वता॑ ए॒वैव दे॒वता॑ दे॒वता॑ ए॒व । \newline
4. ए॒वा न्वा॒रभ्या᳚ न्वा॒र भ्यै॒वैवा न्वा॒रभ्य॑ । \newline
5. अ॒न्वा॒रभ्य॑ सुव॒र्गꣳ सु॑व॒र्ग म॑न्वा॒रभ्या᳚ न्वा॒रभ्य॑ सुव॒र्गम् । \newline
6. अ॒न्वा॒रभ्येत्य॑नु - आ॒रभ्य॑ । \newline
7. सु॒व॒र्गम् ॅलो॒कम् ॅलो॒कꣳ सु॑व॒र्गꣳ सु॑व॒र्गम् ॅलो॒कम् । \newline
8. सु॒व॒र्गमिति॑ सुवः - गम् । \newline
9. लो॒क मे᳚त्येति लो॒कम् ॅलो॒क मे॑ति । \newline
10. ए॒ति॒ वी॒र॒हा वी॑र॒है त्ये॑ति वीर॒हा । \newline
11. वी॒र॒हा वै वै वी॑र॒हा वी॑र॒हा वै । \newline
12. वी॒र॒हेति॑ वीर - हा । \newline
13. वा ए॒ष ए॒ष वै वा ए॒षः । \newline
14. ए॒ष दे॒वाना᳚म् दे॒वाना॑ मे॒ष ए॒ष दे॒वाना᳚म् । \newline
15. दे॒वानां॒ ॅयो यो दे॒वाना᳚म् दे॒वानां॒ ॅयः । \newline
16. यो᳚ ऽग्नि म॒ग्निं ॅयो यो᳚ ऽग्निम् । \newline
17. अ॒ग्नि मु॑द्वा॒सय॑त उद्वा॒सय॑ते॒ ऽग्नि म॒ग्नि मु॑द्वा॒सय॑ते । \newline
18. उ॒द्वा॒सय॑ते॒ न नोद्वा॒सय॑त उद्वा॒सय॑ते॒ न । \newline
19. उ॒द्वा॒सय॑त॒ इत्यु॑त् - वा॒सय॑ते । \newline
20. न वै वै न न वै । \newline
21. वा ए॒त स्यै॒तस्य॒ वै वा ए॒तस्य॑ । \newline
22. ए॒तस्य॑ ब्राह्म॒णा ब्रा᳚ह्म॒णा ए॒त स्यै॒तस्य॑ ब्राह्म॒णाः । \newline
23. ब्रा॒ह्म॒णा ऋ॑ता॒यव॑ ऋता॒यवो᳚ ब्राह्म॒णा ब्रा᳚ह्म॒णा ऋ॑ता॒यवः॑ । \newline
24. ऋ॒ता॒यवः॑ पु॒रा पु॒रर्ता॒यव॑ ऋता॒यवः॑ पु॒रा । \newline
25. ऋ॒ता॒यव॒ इत्यृ॑त - यवः॑ । \newline
26. पु॒रा ऽन्न॒ मन्न॑म् पु॒रा पु॒रा ऽन्न᳚म् । \newline
27. अन्न॑ मक्षन् नक्ष॒न् नन्न॒ मन्न॑ मक्षन्न् । \newline
28. अ॒क्ष॒न् ना॒ग्ने॒य मा᳚ग्ने॒य म॑क्षन् नक्षन् नाग्ने॒यम् । \newline
29. आ॒ग्ने॒य म॒ष्टाक॑पाल म॒ष्टाक॑पाल माग्ने॒य मा᳚ग्ने॒य म॒ष्टाक॑पालम् । \newline
30. अ॒ष्टाक॑पाल॒म् निर् णिर॒ष्टाक॑पाल म॒ष्टाक॑पाल॒म् निः । \newline
31. अ॒ष्टाक॑पाल॒मित्य॒ष्टा - क॒पा॒ल॒म् । \newline
32. निर् व॑पेद् वपे॒न् निर् णिर् व॑पेत् । \newline
33. व॒पे॒द् वै॒श्वा॒न॒रं ॅवै᳚श्वान॒रं ॅव॑पेद् वपेद् वैश्वान॒रम् । \newline
34. वै॒श्वा॒न॒रम् द्वाद॑शकपाल॒म् द्वाद॑शकपालं ॅवैश्वान॒रं ॅवै᳚श्वान॒रम् द्वाद॑शकपालम् । \newline
35. द्वाद॑शकपाल म॒ग्नि म॒ग्निम् द्वाद॑शकपाल॒म् द्वाद॑शकपाल म॒ग्निम् । \newline
36. द्वाद॑शकपाल॒मिति॒ द्वाद॑श - क॒पा॒ल॒म् । \newline
37. अ॒ग्नि मु॑द्वासयि॒ष्यन् नु॑द्वासयि॒ष्यन् न॒ग्नि म॒ग्नि मु॑द्वासयि॒ष्यन्न् । \newline
38. उ॒द्वा॒स॒यि॒ष्यन्. यद् यदु॑द्वासयि॒ष्यन् नु॑द्वासयि॒ष्यन्. यत् । \newline
39. उ॒द्वा॒स॒यि॒ष्यन्नित्यु॑त् - वा॒स॒यि॒ष्यन्न् । \newline
40. यद॒ष्टाक॑पालो॒ ऽष्टाक॑पालो॒ यद् यद॒ष्टाक॑पालः । \newline
41. अ॒ष्टाक॑पालो॒ भव॑ति॒ भव॑ त्य॒ष्टाक॑पालो॒ ऽष्टाक॑पालो॒ भव॑ति । \newline
42. अ॒ष्टाक॑पाल॒ इत्य॒ष्टा - क॒पा॒लः॒ । \newline
43. भव॑ त्य॒ष्टाक्ष॑रा॒ ऽष्टाक्ष॑रा॒ भव॑ति॒ भव॑ त्य॒ष्टाक्ष॑रा । \newline
44. अ॒ष्टाक्ष॑रा गाय॒त्री गा॑य॒ त्र्य॑ष्टाक्ष॑रा॒ ऽष्टाक्ष॑रा गाय॒त्री । \newline
45. अ॒ष्टाक्ष॒रेत्य॒ष्टा - अ॒क्ष॒रा॒ । \newline
46. गा॒य॒त्री गा॑य॒त्रो गा॑य॒त्रो गा॑य॒त्री गा॑य॒त्री गा॑य॒त्रः । \newline
47. गा॒य॒त्रो᳚ ऽग्नि र॒ग्निर् गा॑य॒त्रो गा॑य॒त्रो᳚ ऽग्निः । \newline
48. अ॒ग्निर् यावा॒न्॒. यावा॑ न॒ग्नि र॒ग्निर् यावान्॑ । \newline
49. यावा॑ ने॒वैव यावा॒न्॒. यावा॑ ने॒व । \newline
50. ए॒वाग्नि र॒ग्नि रे॒वैवाग्निः । \newline
51. अ॒ग्निस् तस्मै॒ तस्मा॑ अ॒ग्नि र॒ग्नि स्तस्मै᳚ । \newline
52. तस्मा॑ आति॒थ्य मा॑ति॒थ्यम् तस्मै॒ तस्मा॑ आति॒थ्यम् । \newline
53. आ॒ति॒थ्यम् क॑रोति करो त्याति॒थ्य मा॑ति॒थ्यम् क॑रोति । \newline
54. क॒रो॒ त्यथो॒ अथो॑ करोति करो॒ त्यथो᳚ । \newline
55. अथो॒ यथा॒ यथा ऽथो॒ अथो॒ यथा᳚ । \newline
56. अथो॒ इत्यथो᳚ । \newline
57. यथा॒ जन॒म् जनं॒ ॅयथा॒ यथा॒ जन᳚म् । \newline
58. जनं॑ ॅय॒ते य॒ते जन॒म् जनं॑ ॅय॒ते । \newline
59. य॒ते॑ ऽव॒स म॑व॒सं ॅय॒ते य॒ते॑ ऽव॒सम् । \newline
60. अ॒व॒सम् क॒रोति॑ क॒रो त्य॑व॒स म॑व॒सम् क॒रोति॑ । \newline
61. क॒रोति॑ ता॒दृक् ता॒दृक् क॒रोति॑ क॒रोति॑ ता॒दृक् । \newline
62. ता॒दृ गे॒वैव ता॒दृक् ता॒दृ गे॒व । \newline

\textbf{Ghana Paata } \newline

1. अथो॑ दे॒वता॑ दे॒वता॒ अथो॒ अथो॑ दे॒वता॑ ए॒वैव दे॒वता॒ अथो॒ अथो॑ दे॒वता॑ ए॒व । \newline
2. अथो॒ इत्यथो᳚ । \newline
3. दे॒वता॑ ए॒वैव दे॒वता॑ दे॒वता॑ ए॒वा न्वा॒रभ्या᳚ न्वा॒रभ्यै॒व दे॒वता॑ दे॒वता॑ ए॒वा न्वा॒रभ्य॑ । \newline
4. ए॒वा न्वा॒रभ्या᳚ न्वा॒रभ्यै॒ वैवान्वा॒रभ्य॑ सुव॒र्गꣳ सु॑व॒र्ग म॑न्वा॒रभ्यै॒ वैवान्वा॒रभ्य॑ सुव॒र्गम् । \newline
5. अ॒न्वा॒रभ्य॑ सुव॒र्गꣳ सु॑व॒र्ग म॑न्वा॒रभ्या᳚ न्वा॒रभ्य॑ सुव॒र्गम् ॅलो॒कम् ॅलो॒कꣳ सु॑व॒र्ग म॑न्वा॒रभ्या᳚ न्वा॒रभ्य॑ सुव॒र्गम् ॅलो॒कम् । \newline
6. अ॒न्वा॒रभ्येत्य॑नु - आ॒रभ्य॑ । \newline
7. सु॒व॒र्गम् ॅलो॒कम् ॅलो॒कꣳ सु॑व॒र्गꣳ सु॑व॒र्गम् ॅलो॒क मे᳚त्येति लो॒कꣳ सु॑व॒र्गꣳ सु॑व॒र्गम् ॅलो॒क मे॑ति । \newline
8. सु॒व॒र्गमिति॑ सुवः - गम् । \newline
9. लो॒क मे᳚त्येति लो॒कम् ॅलो॒क मे॑ति वीर॒हा वी॑र॒हैति॑ लो॒कम् ॅलो॒क मे॑ति वीर॒हा । \newline
10. ए॒ति॒ वी॒र॒हा वी॑र॒है त्ये॑ति वीर॒हा वै वै वी॑र॒है त्ये॑ति वीर॒हा वै । \newline
11. वी॒र॒हा वै वै वी॑र॒हा वी॑र॒हा वा ए॒ष ए॒ष वै वी॑र॒हा वी॑र॒हा वा ए॒षः । \newline
12. वी॒र॒हेति॑ वीर - हा । \newline
13. वा ए॒ष ए॒ष वै वा ए॒ष दे॒वाना᳚म् दे॒वाना॑ मे॒ष वै वा ए॒ष दे॒वाना᳚म् । \newline
14. ए॒ष दे॒वाना᳚म् दे॒वाना॑ मे॒ष ए॒ष दे॒वानां॒ ॅयो यो दे॒वाना॑ मे॒ष ए॒ष दे॒वानां॒ ॅयः । \newline
15. दे॒वानां॒ ॅयो यो दे॒वाना᳚म् दे॒वानां॒ ॅयो᳚ ऽग्नि म॒ग्निं ॅयो दे॒वाना᳚म् दे॒वानां॒ ॅयो᳚ ऽग्निम् । \newline
16. यो᳚ ऽग्नि म॒ग्निं ॅयो यो᳚ ऽग्नि मु॑द्वा॒सय॑त उद्वा॒सय॑ते॒ ऽग्निं ॅयो यो᳚ ऽग्नि मु॑द्वा॒सय॑ते । \newline
17. अ॒ग्नि मु॑द्वा॒सय॑त उद्वा॒सय॑ते॒ ऽग्नि म॒ग्नि मु॑द्वा॒सय॑ते॒ न नोद्वा॒सय॑ते॒ ऽग्नि म॒ग्नि मु॑द्वा॒सय॑ते॒ न । \newline
18. उ॒द्वा॒सय॑ते॒ न नोद्वा॒सय॑त उद्वा॒सय॑ते॒ न वै वै नोद्वा॒सय॑त उद्वा॒सय॑ते॒ न वै । \newline
19. उ॒द्वा॒सय॑त॒ इत्यु॑त् - वा॒सय॑ते । \newline
20. न वै वै न न वा ए॒त स्यै॒तस्य॒ वै न न वा ए॒तस्य॑ । \newline
21. वा ए॒त स्यै॒तस्य॒ वै वा ए॒तस्य॑ ब्राह्म॒णा ब्रा᳚ह्म॒णा ए॒तस्य॒ वै वा ए॒तस्य॑ ब्राह्म॒णाः । \newline
22. ए॒तस्य॑ ब्राह्म॒णा ब्रा᳚ह्म॒णा ए॒त स्यै॒तस्य॑ ब्राह्म॒णा ऋ॑ता॒यव॑ ऋता॒यवो᳚ ब्राह्म॒णा ए॒त स्यै॒तस्य॑ ब्राह्म॒णा ऋ॑ता॒यवः॑ । \newline
23. ब्रा॒ह्म॒णा ऋ॑ता॒यव॑ ऋता॒यवो᳚ ब्राह्म॒णा ब्रा᳚ह्म॒णा ऋ॑ता॒यवः॑ पु॒रा पु॒रर्ता॒यवो᳚ ब्राह्म॒णा ब्रा᳚ह्म॒णा ऋ॑ता॒यवः॑ पु॒रा । \newline
24. ऋ॒ता॒यवः॑ पु॒रा पु॒रर्ता॒यव॑ ऋता॒यवः॑ पु॒रा ऽन्न॒ मन्न॑म् पु॒रर्ता॒यव॑ ऋता॒यवः॑ पु॒रा ऽन्न᳚म् । \newline
25. ऋ॒ता॒यव॒ इत्यृ॑त - यवः॑ । \newline
26. पु॒रा ऽन्न॒ मन्न॑म् पु॒रा पु॒रा ऽन्न॑ मक्षन् नक्ष॒न् नन्न॑म् पु॒रा पु॒रा ऽन्न॑ मक्षन्न् । \newline
27. अन्न॑ मक्षन् नक्ष॒न् नन्न॒ मन्न॑ मक्षन् नाग्ने॒य मा᳚ग्ने॒य म॑क्ष॒न् नन्न॒ मन्न॑ मक्षन् नाग्ने॒यम् । \newline
28. अ॒क्ष॒न् ना॒ग्ने॒य मा᳚ग्ने॒य म॑क्षन् नक्षन् नाग्ने॒य म॒ष्टाक॑पाल म॒ष्टाक॑पाल माग्ने॒य म॑क्षन् नक्षन् नाग्ने॒य म॒ष्टाक॑पालम् । \newline
29. आ॒ग्ने॒य म॒ष्टाक॑पाल म॒ष्टाक॑पाल माग्ने॒य मा᳚ग्ने॒य म॒ष्टाक॑पाल॒म् निर् णिर॒ष्टाक॑पाल माग्ने॒य मा᳚ग्ने॒य म॒ष्टाक॑पाल॒म् निः । \newline
30. अ॒ष्टाक॑पाल॒म् निर् णिर॒ष्टाक॑पाल म॒ष्टाक॑पाल॒म् निर् व॑पेद् वपे॒न् निर॒ष्टाक॑पाल म॒ष्टाक॑पाल॒म् निर् व॑पेत् । \newline
31. अ॒ष्टाक॑पाल॒मित्य॒ष्टा - क॒पा॒ल॒म् । \newline
32. निर् व॑पेद् वपे॒न् निर् णिर् व॑पेद् वैश्वान॒रं ॅवै᳚श्वान॒रं ॅव॑पे॒न् निर् णिर् व॑पेद् वैश्वान॒रम् । \newline
33. व॒पे॒द् वै॒श्वा॒न॒रं ॅवै᳚श्वान॒रं ॅव॑पेद् वपेद् वैश्वान॒रम् द्वाद॑शकपाल॒म् द्वाद॑शकपालं ॅवैश्वान॒रं ॅव॑पेद् वपेद् वैश्वान॒रम् द्वाद॑शकपालम् । \newline
34. वै॒श्वा॒न॒रम् द्वाद॑शकपाल॒म् द्वाद॑शकपालं ॅवैश्वान॒रं ॅवै᳚श्वान॒रम् द्वाद॑शकपाल म॒ग्नि म॒ग्निम् द्वाद॑शकपालं ॅवैश्वान॒रं ॅवै᳚श्वान॒रम् द्वाद॑शकपाल म॒ग्निम् । \newline
35. द्वाद॑शकपाल म॒ग्नि म॒ग्निम् द्वाद॑शकपाल॒म् द्वाद॑शकपाल म॒ग्नि मु॑द्वासयि॒ष्यन् नु॑द्वासयि॒ष्यन् न॒ग्निम् द्वाद॑शकपाल॒म् द्वाद॑शकपाल म॒ग्नि मु॑द्वासयि॒ष्यन्न् । \newline
36. द्वाद॑शकपाल॒मिति॒ द्वाद॑श - क॒पा॒ल॒म् । \newline
37. अ॒ग्नि मु॑द्वासयि॒ष्यन् नु॑द्वासयि॒ष्यन् न॒ग्नि म॒ग्नि मु॑द्वासयि॒ष्यन्. यद् यदु॑द्वासयि॒ष्यन् न॒ग्नि म॒ग्नि मु॑द्वासयि॒ष्यन्. यत् । \newline
38. उ॒द्वा॒स॒यि॒ष्यन्. यद् यदु॑द्वासयि॒ष्यन् नु॑द्वासयि॒ष्यन्. यद॒ष्टाक॑पालो॒ ऽष्टाक॑पालो॒ यदु॑द्वासयि॒ष्यन् नु॑द्वासयि॒ष्यन्. यद॒ष्टाक॑पालः । \newline
39. उ॒द्वा॒स॒यि॒ष्यन्नित्यु॑त् - वा॒स॒यि॒ष्यन्न् । \newline
40. यद॒ष्टाक॑पालो॒ ऽष्टाक॑पालो॒ यद् यद॒ष्टाक॑पालो॒ भव॑ति॒ भव॑ त्य॒ष्टाक॑पालो॒ यद् यद॒ष्टाक॑पालो॒ भव॑ति । \newline
41. अ॒ष्टाक॑पालो॒ भव॑ति॒ भव॑ त्य॒ष्टाक॑पालो॒ ऽष्टाक॑पालो॒ भव॑ त्य॒ष्टाक्ष॑रा॒ ऽष्टाक्ष॑रा॒ भव॑ त्य॒ष्टाक॑पालो॒ ऽष्टाक॑पालो॒ भव॑ त्य॒ष्टाक्ष॑रा । \newline
42. अ॒ष्टाक॑पाल॒ इत्य॒ष्टा - क॒पा॒लः॒ । \newline
43. भव॑ त्य॒ष्टाक्ष॑रा॒ ऽष्टाक्ष॑रा॒ भव॑ति॒ भव॑ त्य॒ष्टाक्ष॑रा गाय॒त्री गा॑य॒ त्र्य॑ष्टाक्ष॑रा॒ भव॑ति॒ भव॑ त्य॒ष्टाक्ष॑रा गाय॒त्री । \newline
44. अ॒ष्टाक्ष॑रा गाय॒त्री गा॑य॒ त्र्य॑ष्टाक्ष॑रा॒ ऽष्टाक्ष॑रा गाय॒त्री गा॑य॒त्रो गा॑य॒त्रो गा॑य॒ त्र्य॑ष्टाक्ष॑रा॒ ऽष्टाक्ष॑रा गाय॒त्री गा॑य॒त्रः । \newline
45. अ॒ष्टाक्ष॒रेत्य॒ष्टा - अ॒क्ष॒रा॒ । \newline
46. गा॒य॒त्री गा॑य॒त्रो गा॑य॒त्रो गा॑य॒त्री गा॑य॒त्री गा॑य॒त्रो᳚ ऽग्नि र॒ग्निर् गा॑य॒त्रो गा॑य॒त्री गा॑य॒त्री गा॑य॒त्रो᳚ ऽग्निः । \newline
47. गा॒य॒त्रो᳚ ऽग्नि र॒ग्निर् गा॑य॒त्रो गा॑य॒त्रो᳚ ऽग्निर् यावा॒न्॒. यावा॑ न॒ग्निर् गा॑य॒त्रो गा॑य॒त्रो᳚ ऽग्निर् यावान्॑ । \newline
48. अ॒ग्निर् यावा॒न्॒. यावा॑ न॒ग्नि र॒ग्निर् यावा॑ ने॒वैव यावा॑ न॒ग्नि र॒ग्निर् यावा॑ ने॒व । \newline
49. यावा॑ ने॒वैव यावा॒न्॒. यावा॑ ने॒वाग्नि र॒ग्नि रे॒व यावा॒न्॒. यावा॑ ने॒वाग्निः । \newline
50. ए॒वाग्नि र॒ग्नि रे॒वैवाग्नि स्तस्मै॒ तस्मा॑ अ॒ग्नि रे॒वैवाग्नि स्तस्मै᳚ । \newline
51. अ॒ग्नि स्तस्मै॒ तस्मा॑ अ॒ग्नि र॒ग्नि स्तस्मा॑ आति॒थ्य मा॑ति॒थ्यम् तस्मा॑ अ॒ग्नि र॒ग्नि स्तस्मा॑ आति॒थ्यम् । \newline
52. तस्मा॑ आति॒थ्य मा॑ति॒थ्यम् तस्मै॒ तस्मा॑ आति॒थ्यम् क॑रोति करो त्याति॒थ्यम् तस्मै॒ तस्मा॑ आति॒थ्यम् क॑रोति । \newline
53. आ॒ति॒थ्यम् क॑रोति करो त्याति॒थ्य मा॑ति॒थ्यम् क॑रो॒ त्यथो॒ अथो॑ करो त्याति॒थ्य मा॑ति॒थ्यम् क॑रो॒ त्यथो᳚ । \newline
54. क॒रो॒ त्यथो॒ अथो॑ करोति करो॒ त्यथो॒ यथा॒ यथा ऽथो॑ करोति करो॒ त्यथो॒ यथा᳚ । \newline
55. अथो॒ यथा॒ यथा ऽथो॒ अथो॒ यथा॒ जन॒म् जनं॒ ॅयथा ऽथो॒ अथो॒ यथा॒ जन᳚म् । \newline
56. अथो॒ इत्यथो᳚ । \newline
57. यथा॒ जन॒म् जनं॒ ॅयथा॒ यथा॒ जनं॑ ॅय॒ते य॒ते जनं॒ ॅयथा॒ यथा॒ जनं॑ ॅय॒ते । \newline
58. जनं॑ ॅय॒ते य॒ते जन॒म् जनं॑ ॅय॒ते॑ ऽव॒स म॑व॒सं ॅय॒ते जन॒म् जनं॑ ॅय॒ते॑ ऽव॒सम् । \newline
59. य॒ते॑ ऽव॒स म॑व॒सं ॅय॒ते य॒ते॑ ऽव॒सम् क॒रोति॑ क॒रो त्य॑व॒सं ॅय॒ते य॒ते॑ ऽव॒सम् क॒रोति॑ । \newline
60. अ॒व॒सम् क॒रोति॑ क॒रो त्य॑व॒स म॑व॒सम् क॒रोति॑ ता॒दृक् ता॒दृक् क॒रो त्य॑व॒स म॑व॒सम् क॒रोति॑ ता॒दृक् । \newline
61. क॒रोति॑ ता॒दृक् ता॒दृक् क॒रोति॑ क॒रोति॑ ता॒दृगे॒वैव ता॒दृक् क॒रोति॑ क॒रोति॑ ता॒दृगे॒व । \newline
62. ता॒दृगे॒वैव ता॒दृक् ता॒दृगे॒व तत् तदे॒व ता॒दृक् ता॒दृगे॒व तत् । \newline
\pagebreak
\markright{ TS 2.2.5.6  \hfill https://www.vedavms.in \hfill}
\addcontentsline{toc}{section}{ TS 2.2.5.6 }
\section*{ TS 2.2.5.6 }

\textbf{TS 2.2.5.6 } \newline
\textbf{Samhita Paata} \newline

-गे॒व तद्-द्वाद॑शकपालो वैश्वान॒रो भ॑वति॒ द्वाद॑श॒ मासाः᳚ संॅवथ्स॒रः सं॑ॅवथ्स॒रः खलु॒ वा अ॒ग्नेर्योनिः॒ स्वामे॒वैनं॒ ॅयोनिं॑ गमय-त्या॒द्य॑म॒स्यान्नं॑ भवति वैश्वान॒रं द्वाद॑शकपालं॒ निर्व॑पेन्मारु॒तꣳ स॒प्तक॑पालं॒ ग्राम॑काम आहव॒नीये॑ वैश्वान॒रमधि॑ श्रयति॒ गार्.ह॑पत्ये मारु॒तं पा॑पवस्य॒सस्य॒ विधृ॑त्यै॒ द्वाद॑शकपालो वैश्वान॒रो भ॑वति॒ द्वाद॑श॒ मासाः᳚ संॅवथ्स॒रः सं॑ॅवथ्स॒रेणै॒वास्मै॑ सजा॒ताꣳश्च्या॑वयति मारु॒तो भ॑वति - [  ] \newline

\textbf{Pada Paata} \newline

ए॒व । तत् । द्वाद॑शकपाल॒ इति॒ द्वाद॑श - क॒पा॒लः॒ । वै॒श्वा॒न॒रः । भ॒व॒ति॒ । द्वाद॑श । मासाः᳚ । सं॒ॅव॒थ्स॒र इति॑ सं - व॒थ्स॒रः । सं॒ॅव॒थ्स॒र इति॑ सं - व॒थ्स॒रः । खलु॑ । वै । अ॒ग्नेः । योनिः॑ । स्वाम् । ए॒व । ए॒न॒म् । योनि᳚म् । ग॒म॒य॒ति॒ । आ॒द्य᳚म् । अ॒स्य॒ । अन्न᳚म् । भ॒व॒ति॒ । वै॒श्वा॒न॒रम् । द्वाद॑शकपाल॒मिति॒ द्वाद॑श - क॒पा॒ल॒म् । निरिति॑ । व॒पे॒त् । मा॒रु॒तम् । स॒प्तक॑पाल॒मिति॑ स॒प्त - क॒पा॒ल॒म् । ग्राम॑काम॒ इति॒ ग्राम॑ - का॒मः॒ । आ॒ह॒व॒नीय॒ इत्या᳚ -ह॒व॒नीये᳚ । वै॒श्वा॒न॒रम् । अधीति॑ । श्र॒य॒ति॒ । गार्.ह॑पत्य॒ इति॒ गार्.ह॑ - प॒त्ये॒ । मा॒रु॒तम् । पा॒प॒व॒स्य॒सस्येति॑ पाप - व॒स्य॒सस्य॑ । विधृ॑त्या॒ इति॒ वि - धृ॒त्यै॒ । द्वाद॑शकपाल॒ इति॒ द्वाद॑श - क॒पा॒लः॒ । वै॒श्वा॒न॒रः । भ॒व॒ति॒ । द्वाद॑श । मासाः᳚ । सं॒ॅव॒थ्स॒र इति॑ सं-व॒थ्स॒रः । सं॒ॅव॒थ्स॒रेणेति॑ सं - व॒थ्स॒रेण॑ । ए॒व । अ॒स्मै॒ । स॒जा॒तानिति॑ स - जा॒तान् । च्या॒व॒य॒ति॒ । मा॒रु॒तः । भ॒व॒ति॒ ।  \newline


\textbf{Krama Paata} \newline

ए॒व तत् । तद् द्वाद॑शकपालः । द्वाद॑शकपालो वैश्वान॒रः । द्वाद॑शकपाल॒ इति॒ द्वाद॑श - क॒पा॒लः॒ । वै॒श्वा॒न॒रो भ॑वति । भ॒व॒ति॒ द्वाद॑श । द्वाद॑श॒ मासाः᳚ । मासाः᳚ सम्ॅवथ्स॒रः । स॒म्ॅव॒थ्स॒रः स॑म्ॅवथ्स॒रः । स॒म्ॅव॒थ्स॒र इति॑ सं - व॒थ्स॒रः । स॒म्ॅव॒थ्स॒रः खलु॑ । स॒म्ॅव॒थ्स॒र इति॑ सं - व॒थ्स॒रः । खलु॒ वै । वा अ॒ग्नेः । अ॒ग्नेर् योनिः॑ । योनिः॒ स्वाम् । स्वामे॒व । ए॒वैन᳚म् । ए॒नं॒ ॅयोनि᳚म् । योनि॑म् गमयति । ग॒म॒य॒त्या॒द्य᳚म् । आ॒द्य॑मस्य । अ॒स्यान्न᳚म् । अन्नं॑ भवति । भ॒व॒ति॒ वै॒श्वा॒न॒रम् । वै॒श्वा॒न॒रम् द्वाद॑शकपालम् । द्वाद॑शकपाल॒म् निः । द्वाद॑शकपाल॒मिति॒ द्वाद॑श - क॒पा॒ल॒म् । निर् व॑पेत् । व॒पे॒न् मा॒रु॒तम् । मा॒रु॒तꣳ स॒प्तक॑पालम् । स॒प्तक॑पाल॒म् ग्राम॑कामः । स॒प्तक॑पाल॒मिति॑ स॒प्त - क॒पा॒ल॒म् । ग्राम॑काम आहव॒नीये᳚ । ग्राम॑काम॒ इति॒ ग्राम॑ - का॒मः॒ । आ॒ह॒व॒नीये॑ वैश्वान॒रम् । आ॒ह॒व॒नीय॒ इत्या᳚ - ह॒व॒नीये᳚ । वै॒श्वा॒न॒रमधि॑ । अधि॑ श्रयति । श्र॒य॒ति॒ गार्.ह॑पत्ये । गार्.ह॑पत्ये मारु॒तम् । गार्.ह॑पत्य॒ इति॒ गार्.ह॑ - प॒त्ये॒ । मा॒रु॒तम् पा॑पवस्य॒सस्य॑ । पा॒प॒व॒स्य॒सस्य॒ विधृ॑त्यै । पा॒प॒व॒स्य॒सस्येति॑ पाप - व॒स्य॒सस्य॑ । विधृ॑त्यै॒ द्वाद॑शकपालः । विधृ॑त्या॒ इति॒ वि - धृ॒त्यै॒ । द्वाद॑शकपालो वैश्वान॒रः । द्वाद॑शकपाल॒ इति॒ द्वाद॑श - क॒पा॒लः॒ । वै॒श्वा॒न॒रो भ॑वति । भ॒व॒ति॒ द्वाद॑श । द्वाद॑श॒मासाः᳚ । मासाः᳚ सम्ॅवथ्स॒रः । स॒म्ॅव॒थ्स॒रः स॑म्ॅवथ्स॒रेण॑ । स॒म्ॅव॒थ्स॒र इति॑ सं - व॒थ्स॒रः । स॒म्ॅव॒थ्स॒रेणै॒व । स॒म्ॅव॒थ्स॒रेणेति॑ सं - व॒थ्स॒रेण॑ । ए॒वास्मै᳚ । अ॒स्मै॒ स॒जा॒तान् । स॒जा॒ताꣳश्च्या॑वयति । स॒जा॒तानिति॑ स - जा॒तान् । च्या॒व॒य॒ति॒ मा॒रु॒तः । मा॒रु॒तो भ॑वति ( ) । भ॒व॒ति॒ म॒रुतः॑ \newline

\textbf{Jatai Paata} \newline

1. ए॒व तत् तदे॒वैव तत् । \newline
2. तद् द्वाद॑शकपालो॒ द्वाद॑शकपाल॒स्तत् तद् द्वाद॑शकपालः । \newline
3. द्वाद॑शकपालो वैश्वान॒रो वै᳚श्वान॒रो द्वाद॑शकपालो॒ द्वाद॑शकपालो वैश्वान॒रः । \newline
4. द्वाद॑शकपाल॒ इति॒ द्वाद॑श - क॒पा॒लः॒ । \newline
5. वै॒श्वा॒न॒रो भ॑वति भवति वैश्वान॒रो वै᳚श्वान॒रो भ॑वति । \newline
6. भ॒व॒ति॒ द्वाद॑श॒ द्वाद॑श भवति भवति॒ द्वाद॑श । \newline
7. द्वाद॑श॒ मासा॒ मासा॒ द्वाद॑श॒ द्वाद॑श॒ मासाः᳚ । \newline
8. मासाः᳚ संॅवथ्स॒रः सं॑ॅवथ्स॒रो मासा॒ मासाः᳚ संॅवथ्स॒रः । \newline
9. सं॒ॅव॒थ्स॒रः सं॑ॅवथ्स॒रः । \newline
10. सं॒ॅव॒थ्स॒र इति॑ सं - व॒थ्स॒रः । \newline
11. सं॒ॅव॒थ्स॒रः खलु॒ खलु॑ संॅवथ्स॒रः सं॑ॅवथ्स॒रः खलु॑ । \newline
12. सं॒ॅव॒थ्स॒र इति॑ सं - व॒थ्स॒रः । \newline
13. खलु॒ वै वै खलु॒ खलु॒ वै । \newline
14. वा अ॒ग्ने र॒ग्नेर् वै वा अ॒ग्नेः । \newline
15. अ॒ग्नेर् योनि॒र् योनि॑ र॒ग्ने र॒ग्नेर् योनिः॑ । \newline
16. योनिः॒ स्वाꣳ स्वां ॅयोनि॒र् योनिः॒ स्वाम् । \newline
17. स्वा मे॒वैव स्वाꣳ स्वा मे॒व । \newline
18. ए॒वैन॑ मेन मे॒वैवैन᳚म् । \newline
19. ए॒नं॒ ॅयोनिं॒ ॅयोनि॑ मेन मेनं॒ ॅयोनि᳚म् । \newline
20. योनि॑म् गमयति गमयति॒ योनिं॒ ॅयोनि॑म् गमयति । \newline
21. ग॒म॒य॒ त्या॒द्य॑ मा॒द्य॑म् गमयति गमय त्या॒द्य᳚म् । \newline
22. आ॒द्य॑ मस्यास्या॒द्य॑ मा॒द्य॑ मस्य । \newline
23. अ॒स्यान्न॒ मन्न॑ मस्या॒ स्यान्न᳚म् । \newline
24. अन्न॑म् भवति भव॒ त्यन्न॒ मन्न॑म् भवति । \newline
25. भ॒व॒ति॒ वै॒श्वा॒न॒रं ॅवै᳚श्वान॒रम् भ॑वति भवति वैश्वान॒रम् । \newline
26. वै॒श्वा॒न॒रम् द्वाद॑शकपाल॒म् द्वाद॑शकपालं ॅवैश्वान॒रं ॅवै᳚श्वान॒रम् द्वाद॑शकपालम् । \newline
27. द्वाद॑शकपाल॒म् निर् णिर् द्वाद॑शकपाल॒म् द्वाद॑शकपाल॒म् निः । \newline
28. द्वाद॑शकपाल॒मिति॒ द्वाद॑श - क॒पा॒ल॒म् । \newline
29. निर् व॑पेद् वपे॒न् निर् णिर् व॑पेत् । \newline
30. व॒पे॒न् मा॒रु॒तम् मा॑रु॒तं ॅव॑पेद् वपेन् मारु॒तम् । \newline
31. मा॒रु॒तꣳ स॒प्तक॑पालꣳ स॒प्तक॑पालम् मारु॒तम् मा॑रु॒तꣳ स॒प्तक॑पालम् । \newline
32. स॒प्तक॑पाल॒म् ग्राम॑कामो॒ ग्राम॑कामः स॒प्तक॑पालꣳ स॒प्तक॑पाल॒म् ग्राम॑कामः । \newline
33. स॒प्तक॑पाल॒मिति॑ स॒प्त - क॒पा॒ल॒म् । \newline
34. ग्राम॑काम आहव॒नीय॑ आहव॒नीये॒ ग्राम॑कामो॒ ग्राम॑काम आहव॒नीये᳚ । \newline
35. ग्राम॑काम॒ इति॒ ग्राम॑ - का॒मः॒ । \newline
36. आ॒ह॒व॒नीये॑ वैश्वान॒रं ॅवै᳚श्वान॒र मा॑हव॒नीय॑ आहव॒नीये॑ वैश्वान॒रम् । \newline
37. आ॒ह॒व॒नीय॒ इत्या᳚ - ह॒व॒नीये᳚ । \newline
38. वै॒श्वा॒न॒र मध्यधि॑ वैश्वान॒रं ॅवै᳚श्वान॒र मधि॑ । \newline
39. अधि॑ श्रयति श्रय॒ त्यध्यधि॑ श्रयति । \newline
40. श्र॒य॒ति॒ गार्.ह॑पत्ये॒ गार्.ह॑पत्ये श्रयति श्रयति॒ गार्.ह॑पत्ये । \newline
41. गार्.ह॑पत्ये मारु॒तम् मा॑रु॒तम् गार्.ह॑पत्ये॒ गार्.ह॑पत्ये मारु॒तम् । \newline
42. गार्.ह॑पत्य॒ इति॒ गार्.ह॑ - प॒त्ये॒ । \newline
43. मा॒रु॒तम् पा॑पवस्य॒सस्य॑ पापवस्य॒सस्य॑ मारु॒तम् मा॑रु॒तम् पा॑पवस्य॒सस्य॑ । \newline
44. पा॒प॒व॒स्य॒सस्य॒ विधृ॑त्यै॒ विधृ॑त्यै पापवस्य॒सस्य॑ पापवस्य॒सस्य॒ विधृ॑त्यै । \newline
45. पा॒प॒व॒स्य॒सस्येति॑ पाप - व॒स्य॒सस्य॑ । \newline
46. विधृ॑त्यै॒ द्वाद॑शकपालो॒ द्वाद॑शकपालो॒ विधृ॑त्यै॒ विधृ॑त्यै॒ द्वाद॑शकपालः । \newline
47. विधृ॑त्या॒ इति॒ वि - धृ॒त्यै॒ । \newline
48. द्वाद॑शकपालो वैश्वान॒रो वै᳚श्वान॒रो द्वाद॑शकपालो॒ द्वाद॑शकपालो वैश्वान॒रः । \newline
49. द्वाद॑शकपाल॒ इति॒ द्वाद॑श - क॒पा॒लः॒ । \newline
50. वै॒श्वा॒न॒रो भ॑वति भवति वैश्वान॒रो वै᳚श्वान॒रो भ॑वति । \newline
51. भ॒व॒ति॒ द्वाद॑श॒ द्वाद॑श भवति भवति॒ द्वाद॑श । \newline
52. द्वाद॑श॒ मासा॒ मासा॒ द्वाद॑श॒ द्वाद॑श॒ मासाः᳚ । \newline
53. मासाः᳚ संॅवथ्स॒रः सं॑ॅवथ्स॒रो मासा॒ मासाः᳚ संॅवथ्स॒रः । \newline
54. सं॒ॅव॒थ्स॒रः सं॑ॅवथ्स॒रेण॑ संॅवथ्स॒रेण॑ संॅवथ्स॒रः सं॑ॅवथ्स॒रः सं॑ॅवथ्स॒रेण॑ । \newline
55. सं॒ॅव॒थ्स॒र इति॑ सं - व॒थ्स॒रः । \newline
56. सं॒ॅव॒थ्स॒रे णै॒वैव सं॑ॅवथ्स॒रेण॑ संॅवथ्स॒रे णै॒व । \newline
57. सं॒ॅव॒थ्स॒रेणेति॑ सं - व॒थ्स॒रेण॑ । \newline
58. ए॒वास्मा॑ अस्मा ए॒वैवास्मै᳚ । \newline
59. अ॒स्मै॒ स॒जा॒तान् थ्स॑जा॒ता न॑स्मा अस्मै सजा॒तान् । \newline
60. स॒जा॒ताꣳ श्च्या॑वयति च्यावयति सजा॒तान् थ्स॑जा॒ताꣳ श्च्या॑वयति । \newline
61. स॒जा॒तानिति॑ स - जा॒तान् । \newline
62. च्या॒व॒य॒ति॒ मा॒रु॒तो मा॑रु॒त श्च्या॑वयति च्यावयति मारु॒तः । \newline
63. मा॒रु॒तो भ॑वति भवति मारु॒तो मा॑रु॒तो भ॑वति । \newline
64. भ॒व॒ति॒ म॒रुतो॑ म॒रुतो॑ भवति भवति म॒रुतः॑ । \newline

\textbf{Ghana Paata } \newline

1. ए॒व तत् तदे॒वैव तद् द्वाद॑शकपालो॒ द्वाद॑शकपाल॒ स्तदे॒वैव तद् द्वाद॑शकपालः । \newline
2. तद् द्वाद॑शकपालो॒ द्वाद॑शकपाल॒ स्तत् तद् द्वाद॑शकपालो वैश्वान॒रो वै᳚श्वान॒रो द्वाद॑शकपाल॒ स्तत् तद् द्वाद॑शकपालो वैश्वान॒रः । \newline
3. द्वाद॑शकपालो वैश्वान॒रो वै᳚श्वान॒रो द्वाद॑शकपालो॒ द्वाद॑शकपालो वैश्वान॒रो भ॑वति भवति वैश्वान॒रो द्वाद॑शकपालो॒ द्वाद॑शकपालो वैश्वान॒रो भ॑वति । \newline
4. द्वाद॑शकपाल॒ इति॒ द्वाद॑श - क॒पा॒लः॒ । \newline
5. वै॒श्वा॒न॒रो भ॑वति भवति वैश्वान॒रो वै᳚श्वान॒रो भ॑वति॒ द्वाद॑श॒ द्वाद॑श भवति वैश्वान॒रो वै᳚श्वान॒रो भ॑वति॒ द्वाद॑श । \newline
6. भ॒व॒ति॒ द्वाद॑श॒ द्वाद॑श भवति भवति॒ द्वाद॑श॒ मासा॒ मासा॒ द्वाद॑श भवति भवति॒ द्वाद॑श॒ मासाः᳚ । \newline
7. द्वाद॑श॒ मासा॒ मासा॒ द्वाद॑श॒ द्वाद॑श॒ मासाः᳚ संॅवथ्स॒रः सं॑ॅवथ्स॒रो मासा॒ द्वाद॑श॒ द्वाद॑श॒ मासाः᳚ संॅवथ्स॒रः । \newline
8. मासाः᳚ संॅवथ्स॒रः सं॑ॅवथ्स॒रो मासा॒ मासाः᳚ संॅवथ्स॒रः । \newline
9. सं॒ॅव॒थ्स॒रः सं॑ॅवथ्स॒रः । \newline
10. सं॒ॅव॒थ्स॒र इति॑ सं - व॒थ्स॒रः । \newline
11. सं॒ॅव॒थ्स॒रः खलु॒ खलु॑ संॅवथ्स॒रः सं॑ॅवथ्स॒रः खलु॒ वै वै खलु॑ संॅवथ्स॒रः सं॑ॅवथ्स॒रः खलु॒ वै । \newline
12. सं॒ॅव॒थ्स॒र इति॑ सं - व॒थ्स॒रः । \newline
13. खलु॒ वै वै खलु॒ खलु॒ वा अ॒ग्ने र॒ग्नेर् वै खलु॒ खलु॒ वा अ॒ग्नेः । \newline
14. वा अ॒ग्ने र॒ग्नेर् वै वा अ॒ग्नेर् योनि॒र् योनि॑ र॒ग्नेर् वै वा अ॒ग्नेर् योनिः॑ । \newline
15. अ॒ग्नेर् योनि॒र् योनि॑ र॒ग्ने र॒ग्नेर् योनिः॒ स्वाꣳ स्वां ॅयोनि॑ र॒ग्ने र॒ग्नेर् योनिः॒ स्वाम् । \newline
16. योनिः॒ स्वाꣳ स्वां ॅयोनि॒र् योनिः॒ स्वा मे॒वैव स्वां ॅयोनि॒र् योनिः॒ स्वा मे॒व । \newline
17. स्वा मे॒वैव स्वाꣳ स्वा मे॒वैन॑ मेन मे॒व स्वाꣳ स्वा मे॒वैन᳚म् । \newline
18. ए॒वैन॑ मेन मे॒वैवैनं॒ ॅयोनिं॒ ॅयोनि॑ मेन मे॒वैवैनं॒ ॅयोनि᳚म् । \newline
19. ए॒नं॒ ॅयोनिं॒ ॅयोनि॑ मेन मेनं॒ ॅयोनि॑म् गमयति गमयति॒ योनि॑ मेन मेनं॒ ॅयोनि॑म् गमयति । \newline
20. योनि॑म् गमयति गमयति॒ योनिं॒ ॅयोनि॑म् गमय त्या॒द्य॑ मा॒द्य॑म् गमयति॒ योनिं॒ ॅयोनि॑म् गमय त्या॒द्य᳚म् । \newline
21. ग॒म॒य॒ त्या॒द्य॑ मा॒द्य॑म् गमयति गमय त्या॒द्य॑ मस्या स्या॒द्य॑म् गमयति गमय त्या॒द्य॑ मस्य । \newline
22. आ॒द्य॑ मस्या स्या॒द्य॑ मा॒द्य॑ म॒स्यान्न॒ मन्न॑ मस्या॒द्य॑ मा॒द्य॑ म॒स्यान्न᳚म् । \newline
23. अ॒स्यान्न॒ मन्न॑ मस्या॒ स्यान्न॑म् भवति भव॒ त्यन्न॑ मस्या॒ स्यान्न॑म् भवति । \newline
24. अन्न॑म् भवति भव॒ त्यन्न॒ मन्न॑म् भवति वैश्वान॒रं ॅवै᳚श्वान॒रम् भ॑व॒ त्यन्न॒ मन्न॑म् भवति वैश्वान॒रम् । \newline
25. भ॒व॒ति॒ वै॒श्वा॒न॒रं ॅवै᳚श्वान॒रम् भ॑वति भवति वैश्वान॒रम् द्वाद॑शकपाल॒म् द्वाद॑शकपालं ॅवैश्वान॒रम् भ॑वति भवति वैश्वान॒रम् द्वाद॑शकपालम् । \newline
26. वै॒श्वा॒न॒रम् द्वाद॑शकपाल॒म् द्वाद॑शकपालं ॅवैश्वान॒रं ॅवै᳚श्वान॒रम् द्वाद॑शकपाल॒म् निर् णिर् द्वाद॑शकपालं ॅवैश्वान॒रं ॅवै᳚श्वान॒रम् द्वाद॑शकपाल॒म् निः । \newline
27. द्वाद॑शकपाल॒म् निर् णिर् द्वाद॑शकपाल॒म् द्वाद॑शकपाल॒म् निर् व॑पेद् वपे॒न् निर् द्वाद॑शकपाल॒म् द्वाद॑शकपाल॒म् निर् व॑पेत् । \newline
28. द्वाद॑शकपाल॒मिति॒ द्वाद॑श - क॒पा॒ल॒म् । \newline
29. निर् व॑पेद् वपे॒न् निर् णिर् व॑पेन् मारु॒तम् मा॑रु॒तं ॅव॑पे॒न् निर् णिर् व॑पेन् मारु॒तम् । \newline
30. व॒पे॒न् मा॒रु॒तम् मा॑रु॒तं ॅव॑पेद् वपेन् मारु॒तꣳ स॒प्तक॑पालꣳ स॒प्तक॑पालम् मारु॒तं ॅव॑पेद् वपेन् मारु॒तꣳ स॒प्तक॑पालम् । \newline
31. मा॒रु॒तꣳ स॒प्तक॑पालꣳ स॒प्तक॑पालम् मारु॒तम् मा॑रु॒तꣳ स॒प्तक॑पाल॒म् ग्राम॑कामो॒ ग्राम॑कामः स॒प्तक॑पालम् मारु॒तम् मा॑रु॒तꣳ स॒प्तक॑पाल॒म् ग्राम॑कामः । \newline
32. स॒प्तक॑पाल॒म् ग्राम॑कामो॒ ग्राम॑कामः स॒प्तक॑पालꣳ स॒प्तक॑पाल॒म् ग्राम॑काम आहव॒नीय॑ आहव॒नीये॒ ग्राम॑कामः स॒प्तक॑पालꣳ स॒प्तक॑पाल॒म् ग्राम॑काम आहव॒नीये᳚ । \newline
33. स॒प्तक॑पाल॒मिति॑ स॒प्त - क॒पा॒ल॒म् । \newline
34. ग्राम॑काम आहव॒नीय॑ आहव॒नीये॒ ग्राम॑कामो॒ ग्राम॑काम आहव॒नीये॑ वैश्वान॒रं ॅवै᳚श्वान॒र मा॑हव॒नीये॒ ग्राम॑कामो॒ ग्राम॑काम आहव॒नीये॑ वैश्वान॒रम् । \newline
35. ग्राम॑काम॒ इति॒ ग्राम॑ - का॒मः॒ । \newline
36. आ॒ह॒व॒नीये॑ वैश्वान॒रं ॅवै᳚श्वान॒र मा॑हव॒नीय॑ आहव॒नीये॑ वैश्वान॒र मध्यधि॑ वैश्वान॒र मा॑हव॒नीय॑ आहव॒नीये॑ वैश्वान॒र मधि॑ । \newline
37. आ॒ह॒व॒नीय॒ इत्या᳚ - ह॒व॒नीये᳚ । \newline
38. वै॒श्वा॒न॒र मध्यधि॑ वैश्वान॒रं ॅवै᳚श्वान॒र मधि॑ श्रयति श्रय॒त्यधि॑ वैश्वान॒रं ॅवै᳚श्वान॒र मधि॑ श्रयति । \newline
39. अधि॑ श्रयति श्रय॒ त्यध्यधि॑ श्रयति॒ गार्.ह॑पत्ये॒ गार्.ह॑पत्ये श्रय॒ त्यध्यधि॑ श्रयति॒ गार्.ह॑पत्ये । \newline
40. श्र॒य॒ति॒ गार्.ह॑पत्ये॒ गार्.ह॑पत्ये श्रयति श्रयति॒ गार्.ह॑पत्ये मारु॒तम् मा॑रु॒तम् गार्.ह॑पत्ये श्रयति श्रयति॒ गार्.ह॑पत्ये मारु॒तम् । \newline
41. गार्.ह॑पत्ये मारु॒तम् मा॑रु॒तम् गार्.ह॑पत्ये॒ गार्.ह॑पत्ये मारु॒तम् पा॑पवस्य॒सस्य॑ पापवस्य॒सस्य॑ मारु॒तम् गार्.ह॑पत्ये॒ गार्.ह॑पत्ये मारु॒तम् पा॑पवस्य॒सस्य॑ । \newline
42. गार्.ह॑पत्य॒ इति॒ गार्.ह॑ - प॒त्ये॒ । \newline
43. मा॒रु॒तम् पा॑पवस्य॒सस्य॑ पापवस्य॒सस्य॑ मारु॒तम् मा॑रु॒तम् पा॑पवस्य॒सस्य॒ विधृ॑त्यै॒ विधृ॑त्यै पापवस्य॒सस्य॑ मारु॒तम् मा॑रु॒तम् पा॑पवस्य॒सस्य॒ विधृ॑त्यै । \newline
44. पा॒प॒व॒स्य॒सस्य॒ विधृ॑त्यै॒ विधृ॑त्यै पापवस्य॒सस्य॑ पापवस्य॒सस्य॒ विधृ॑त्यै॒ द्वाद॑शकपालो॒ द्वाद॑शकपालो॒ विधृ॑त्यै पापवस्य॒सस्य॑ पापवस्य॒सस्य॒ विधृ॑त्यै॒ द्वाद॑शकपालः । \newline
45. पा॒प॒व॒स्य॒सस्येति॑ पाप - व॒स्य॒सस्य॑ । \newline
46. विधृ॑त्यै॒ द्वाद॑शकपालो॒ द्वाद॑शकपालो॒ विधृ॑त्यै॒ विधृ॑त्यै॒ द्वाद॑शकपालो वैश्वान॒रो वै᳚श्वान॒रो द्वाद॑शकपालो॒ विधृ॑त्यै॒ विधृ॑त्यै॒ द्वाद॑शकपालो वैश्वान॒रः । \newline
47. विधृ॑त्या॒ इति॒ वि - धृ॒त्यै॒ । \newline
48. द्वाद॑शकपालो वैश्वान॒रो वै᳚श्वान॒रो द्वाद॑शकपालो॒ द्वाद॑शकपालो वैश्वान॒रो भ॑वति भवति वैश्वान॒रो द्वाद॑शकपालो॒ द्वाद॑शकपालो वैश्वान॒रो भ॑वति । \newline
49. द्वाद॑शकपाल॒ इति॒ द्वाद॑श - क॒पा॒लः॒ । \newline
50. वै॒श्वा॒न॒रो भ॑वति भवति वैश्वान॒रो वै᳚श्वान॒रो भ॑वति॒ द्वाद॑श॒ द्वाद॑श भवति वैश्वान॒रो वै᳚श्वान॒रो भ॑वति॒ द्वाद॑श । \newline
51. भ॒व॒ति॒ द्वाद॑श॒ द्वाद॑श भवति भवति॒ द्वाद॑श॒ मासा॒ मासा॒ द्वाद॑श भवति भवति॒ द्वाद॑श॒ मासाः᳚ । \newline
52. द्वाद॑श॒ मासा॒ मासा॒ द्वाद॑श॒ द्वाद॑श॒ मासाः᳚ संॅवथ्स॒रः सं॑ॅवथ्स॒रो मासा॒ द्वाद॑श॒ द्वाद॑श॒ मासाः᳚ संॅवथ्स॒रः । \newline
53. मासाः᳚ संॅवथ्स॒रः सं॑ॅवथ्स॒रो मासा॒ मासाः᳚ संॅवथ्स॒रः सं॑ॅवथ्स॒रेण॑ संॅवथ्स॒रेण॑ संॅवथ्स॒रो मासा॒ मासाः᳚ संॅवथ्स॒रः सं॑ॅवथ्स॒रेण॑ । \newline
54. सं॒ॅव॒थ्स॒रः सं॑ॅवथ्स॒रेण॑ संॅवथ्स॒रेण॑ संॅवथ्स॒रः सं॑ॅवथ्स॒रः 
सं॑ॅवथ्स॒रे णै॒वैव सं॑ॅवथ्स॒रेण॑ संॅवथ्स॒रः सं॑ॅवथ्स॒रः सं॑ॅवथ्स॒रे णै॒व । \newline
55. सं॒ॅव॒थ्स॒र इति॑ सं - व॒थ्स॒रः । \newline
56. सं॒ॅव॒थ्स॒रे णै॒वैव सं॑ॅवथ्स॒रेण॑ संॅवथ्स॒रे णै॒वास्मा॑ अस्मा ए॒व सं॑ॅवथ्स॒रेण॑ संॅवथ्स॒रे णै॒वास्मै᳚ । \newline
57. सं॒ॅव॒थ्स॒रेणेति॑ सं - व॒थ्स॒रेण॑ । \newline
58. ए॒वास्मा॑ अस्मा ए॒वैवास्मै॑ सजा॒तान् थ्स॑जा॒ता न॑स्मा ए॒वैवास्मै॑ सजा॒तान् । \newline
59. अ॒स्मै॒ स॒जा॒तान् थ्स॑जा॒ता न॑स्मा अस्मै सजा॒ताꣳ श्च्या॑वयति च्यावयति सजा॒ता न॑स्मा अस्मै सजा॒ताꣳ
श्च्या॑वयति । \newline
60. स॒जा॒ताꣳ श्च्या॑वयति च्यावयति सजा॒तान् थ्स॑जा॒ताꣳ श्च्या॑वयति मारु॒तो मा॑रु॒त श्च्या॑वयति सजा॒तान् थ्स॑जा॒ताꣳ श्च्या॑वयति मारु॒तः । \newline
61. स॒जा॒तानिति॑ स - जा॒तान् । \newline
62. च्या॒व॒य॒ति॒ मा॒रु॒तो मा॑रु॒त श्च्या॑वयति च्यावयति मारु॒तो भ॑वति भवति मारु॒त श्च्या॑वयति च्यावयति मारु॒तो भ॑वति । \newline
63. मा॒रु॒तो भ॑वति भवति मारु॒तो मा॑रु॒तो भ॑वति म॒रुतो॑ म॒रुतो॑ भवति मारु॒तो मा॑रु॒तो भ॑वति म॒रुतः॑ । \newline
64. भ॒व॒ति॒ म॒रुतो॑ म॒रुतो॑ भवति भवति म॒रुतो॒ वै वै म॒रुतो॑ भवति भवति म॒रुतो॒ वै । \newline
\pagebreak
\markright{ TS 2.2.5.7  \hfill https://www.vedavms.in \hfill}
\addcontentsline{toc}{section}{ TS 2.2.5.7 }
\section*{ TS 2.2.5.7 }

\textbf{TS 2.2.5.7 } \newline
\textbf{Samhita Paata} \newline

म॒रुतो॒ वै दे॒वानां॒ ॅविशो॑ देववि॒शेनै॒वास्मै॑ मनुष्य वि॒शमव॑ रुन्धे स॒प्तक॑पालो भवति स॒प्त ग॑णा॒ वै म॒रुतो॑ गण॒श ए॒वास्मै॑ सजा॒तानव॑ रुन्धे ऽनू॒च्यमा॑न॒ आ सा॑दयति॒ विश॑मे॒वास्मा॒ अनु॑वर्त्मानं करोति ॥ \newline

\textbf{Pada Paata} \newline

म॒रुतः॑ । वै । दे॒वाना᳚म् । विशः॑ । दे॒व॒वि॒शेनेति॑ देव - वि॒शेन॑ । ए॒व । अ॒स्मै॒ । म॒नु॒ष्य॒वि॒शमिति॑ मनुष्य - वि॒शम् । अवेति॑ । रु॒न्धे॒ । स॒प्तक॑पाल॒ इति॑ स॒प्त - क॒पा॒लः॒ । भ॒व॒ति॒ । स॒प्तग॑णा॒ इति॑ स॒प्त - ग॒णाः॒ । वै । म॒रुतः॑ । ग॒ण॒श इति॑ गण-शः । ए॒व । अ॒स्मै॒ । स॒जा॒तानिति॑ स - जा॒तान् । अवेति॑ । रु॒न्धे॒ । अ॒नू॒च्यमा॑न॒ इत्य॑नु - उ॒च्यमा॑ने । एति॑ । सा॒द॒य॒ति॒ । विश᳚म् । ए॒व । अ॒स्मै॒ । अनु॑वर्त्मान॒मित्यनु॑ - व॒र्त्मा॒न॒म् । क॒रो॒ति॒ ॥  \newline


\textbf{Krama Paata} \newline

म॒रुतो॒ वै । वै दे॒वाना᳚म् । दे॒वानां॒ ॅविशः॑ । विशो॑ देववि॒शेन॑ । दे॒व॒वि॒शेनै॒व । दे॒व॒वि॒शेनेति॑ देव - वि॒शेन॑ । ए॒वास्मै᳚ । अ॒स्मै॒ म॒नु॒ष्य॒वि॒शम् । म॒नु॒ष्य॒वि॒शमव॑ । म॒नु॒ष्य॒वि॒शमिति॑ मनुष्य - वि॒शम् । अव॑ रुन्धे । रु॒न्धे॒ स॒प्तक॑पालः । स॒प्तक॑पालो भवति । स॒प्तक॑पाल॒ इति॑ स॒प्त - क॒पा॒लः॒ । भ॒व॒ति॒ स॒प्तग॑णाः । स॒प्तग॑णा॒ वै । स॒प्तग॑णा॒ इति॑ स॒प्त - ग॒णाः॒ । वै म॒रुतः॑ । म॒रुतो॑ गण॒शः । ग॒ण॒श ए॒व । ग॒ण॒श इति॑ गण - शः । ए॒वास्मै᳚ । अ॒स्मै॒ स॒जा॒तान् । स॒जा॒तानव॑ । स॒जा॒तानिति॑ स - जा॒तान् । अव॑ रुन्धे । रु॒न्धे॒ ऽनू॒च्यमा॑ने । अ॒नू॒च्यमा॑न॒ आ । अ॒नू॒च्यमा॑न॒ इत्य॑नु - उ॒च्यमा॑ने । आ सा॑दयति । सा॒द॒य॒ति॒ विश᳚म् । विश॑मे॒व । ए॒वास्मै᳚ । अ॒स्मा॒ अनु॑वर्त्मानम् । अनु॑वर्त्मानम् करोति । अनु॑वर्त्मान॒मित्यनु॑ - व॒र्त्मा॒न॒॒म् । क॒रो॒तीति॑ करोति । \newline

\textbf{Jatai Paata} \newline

1. म॒रुतो॒ वै वै म॒रुतो॑ म॒रुतो॒ वै । \newline
2. वै दे॒वाना᳚म् दे॒वानां॒ ॅवै वै दे॒वाना᳚म् । \newline
3. दे॒वानां॒ ॅविशो॒ विशो॑ दे॒वाना᳚म् दे॒वानां॒ ॅविशः॑ । \newline
4. विशो॑ देववि॒शेन॑ देववि॒शेन॒ विशो॒ विशो॑ देववि॒शेन॑ । \newline
5. दे॒व॒वि॒शे नै॒वैव दे॑ववि॒शेन॑ देववि॒शे नै॒व । \newline
6. दे॒व॒वि॒शेनेति॑ देव - वि॒शेन॑ । \newline
7. ए॒वास्मा॑ अस्मा ए॒वैवास्मै᳚ । \newline
8. अ॒स्मै॒ म॒नु॒ष्य॒वि॒शम् म॑नुष्यवि॒श म॑स्मा अस्मै मनुष्यवि॒शम् । \newline
9. म॒नु॒ष्य॒वि॒श मवाव॑ मनुष्यवि॒शम् म॑नुष्यवि॒श मव॑ । \newline
10. म॒नु॒ष्य॒वि॒शमिति॑ मनुष्य - वि॒शम् । \newline
11. अव॑ रुन्धे रु॒न्धे ऽवाव॑ रुन्धे । \newline
12. रु॒न्धे॒ स॒प्तक॑पालः स॒प्तक॑पालो रुन्धे रुन्धे स॒प्तक॑पालः । \newline
13. स॒प्तक॑पालो भवति भवति स॒प्तक॑पालः स॒प्तक॑पालो भवति । \newline
14. स॒प्तक॑पाल॒ इति॑ स॒प्त - क॒पा॒लः॒ । \newline
15. भ॒व॒ति॒ स॒प्तग॑णाः स॒प्तग॑णा भवति भवति स॒प्तग॑णाः । \newline
16. स॒प्तग॑णा॒ वै वै स॒प्तग॑णाः स॒प्तग॑णा॒ वै । \newline
17. स॒प्तग॑णा॒ इति॑ स॒प्त - ग॒णाः॒ । \newline
18. वै म॒रुतो॑ म॒रुतो॒ वै वै म॒रुतः॑ । \newline
19. म॒रुतो॑ गण॒शो ग॑ण॒शो म॒रुतो॑ म॒रुतो॑ गण॒शः । \newline
20. ग॒ण॒श ए॒वैव ग॑ण॒शो ग॑ण॒श ए॒व । \newline
21. ग॒ण॒श इति॑ गण - शः । \newline
22. ए॒वास्मा॑ अस्मा ए॒वैवास्मै᳚ । \newline
23. अ॒स्मै॒ स॒जा॒तान् थ्स॑जा॒ता न॑स्मा अस्मै सजा॒तान् । \newline
24. स॒जा॒ता नवाव॑ सजा॒तान् थ्स॑जा॒ता नव॑ । \newline
25. स॒जा॒तानिति॑ स - जा॒तान् । \newline
26. अव॑ रुन्धे रु॒न्धे ऽवाव॑ रुन्धे । \newline
27. रु॒न्धे॒ ऽनू॒च्यमा॑ने ऽनू॒च्यमा॑ने रुन्धे रुन्धे ऽनू॒च्यमा॑ने । \newline
28. अ॒नू॒च्यमा॑न॒ आ ऽनू॒च्यमा॑ने ऽनू॒च्यमा॑न॒ आ । \newline
29. अ॒नू॒च्यमा॑न॒ इत्य॑नु - उ॒च्यमा॑ने । \newline
30. आ सा॑दयति सादय॒त्या सा॑दयति । \newline
31. सा॒द॒य॒ति॒ विशं॒ ॅविशꣳ॑ सादयति सादयति॒ विश᳚म् । \newline
32. विश॑ मे॒वैव विशं॒ ॅविश॑ मे॒व । \newline
33. ए॒वास्मा॑ अस्मा ए॒वैवास्मै᳚ । \newline
34. अ॒स्मा॒ अनु॑वर्त्मान॒ मनु॑वर्त्मान मस्मा अस्मा॒ अनु॑वर्त्मानम् । \newline
35. अनु॑वर्त्मानम् करोति करो॒त्यनु॑वर्त्मान॒ मनु॑वर्त्मानम् करोति । \newline
36. अनु॑वर्त्मान॒मित्यनु॑ - व॒र्त्मा॒न॒म् । \newline
37. क॒रो॒तीति॑ करोति । \newline

\textbf{Ghana Paata } \newline

1. म॒रुतो॒ वै वै म॒रुतो॑ म॒रुतो॒ वै दे॒वाना᳚म् दे॒वानां॒ ॅवै म॒रुतो॑ म॒रुतो॒ वै दे॒वाना᳚म् । \newline
2. वै दे॒वाना᳚म् दे॒वानां॒ ॅवै वै दे॒वानां॒ ॅविशो॒ विशो॑ दे॒वानां॒ ॅवै वै दे॒वानां॒ ॅविशः॑ । \newline
3. दे॒वानां॒ ॅविशो॒ विशो॑ दे॒वाना᳚म् दे॒वानां॒ ॅविशो॑ देववि॒शेन॑ देववि॒शेन॒ विशो॑ दे॒वाना᳚म् दे॒वानां॒ ॅविशो॑ देववि॒शेन॑ । \newline
4. विशो॑ देववि॒शेन॑ देववि॒शेन॒ विशो॒ विशो॑ देववि॒शे नै॒वैव दे॑ववि॒शेन॒ विशो॒ विशो॑ देववि॒शे नै॒व । \newline
5. दे॒व॒वि॒शे नै॒वैव दे॑ववि॒शेन॑ देववि॒शे नै॒वास्मा॑ अस्मा ए॒व दे॑ववि॒शेन॑ देववि॒शे नै॒वास्मै᳚ । \newline
6. दे॒व॒वि॒शेनेति॑ देव - वि॒शेन॑ । \newline
7. ए॒वास्मा॑ अस्मा ए॒वैवास्मै॑ मनुष्यवि॒शम् म॑नुष्यवि॒श म॑स्मा ए॒वैवास्मै॑ मनुष्यवि॒शम् । \newline
8. अ॒स्मै॒ म॒नु॒ष्य॒वि॒शम् म॑नुष्यवि॒श म॑स्मा अस्मै मनुष्यवि॒श मवाव॑ मनुष्यवि॒श म॑स्मा अस्मै मनुष्यवि॒श मव॑ । \newline
9. म॒नु॒ष्य॒वि॒श मवाव॑ मनुष्यवि॒शम् म॑नुष्यवि॒श मव॑ रुन्धे रु॒न्धे ऽव॑ मनुष्यवि॒शम् म॑नुष्यवि॒श मव॑ रुन्धे । \newline
10. म॒नु॒ष्य॒वि॒शमिति॑ मनुष्य - वि॒शम् । \newline
11. अव॑ रुन्धे रु॒न्धे ऽवाव॑ रुन्धे स॒प्तक॑पालः स॒प्तक॑पालो रु॒न्धे ऽवाव॑ रुन्धे स॒प्तक॑पालः । \newline
12. रु॒न्धे॒ स॒प्तक॑पालः स॒प्तक॑पालो रुन्धे रुन्धे स॒प्तक॑पालो भवति भवति स॒प्तक॑पालो रुन्धे रुन्धे स॒प्तक॑पालो भवति । \newline
13. स॒प्तक॑पालो भवति भवति स॒प्तक॑पालः स॒प्तक॑पालो भवति स॒प्तग॑णाः स॒प्तग॑णा भवति स॒प्तक॑पालः स॒प्तक॑पालो भवति स॒प्तग॑णाः । \newline
14. स॒प्तक॑पाल॒ इति॑ स॒प्त - क॒पा॒लः॒ । \newline
15. भ॒व॒ति॒ स॒प्तग॑णाः स॒प्तग॑णा भवति भवति स॒प्तग॑णा॒ वै वै स॒प्तग॑णा भवति भवति स॒प्तग॑णा॒ वै । \newline
16. स॒प्तग॑णा॒ वै वै स॒प्तग॑णाः स॒प्तग॑णा॒ वै म॒रुतो॑ म॒रुतो॒ वै स॒प्तग॑णाः स॒प्तग॑णा॒ वै म॒रुतः । \newline
17. स॒प्तग॑णा॒ इति॑ स॒प्त - ग॒णाः॒ । \newline
18. वै म॒रुतो॑ म॒रुतो॒ वै वै म॒रुतो॑ गण॒शो ग॑ण॒शो म॒रुतो॒ वै वै म॒रुतो॑ गण॒शः । \newline
19. म॒रुतो॑ गण॒शो ग॑ण॒शो म॒रुतो॑ म॒रुतो॑ गण॒श ए॒वैव ग॑ण॒शो म॒रुतो॑ म॒रुतो॑ गण॒श ए॒व । \newline
20. ग॒ण॒श ए॒वैव ग॑ण॒शो ग॑ण॒श ए॒वास्मा॑ अस्मा ए॒व ग॑ण॒शो ग॑ण॒श ए॒वास्मै᳚ । \newline
21. ग॒ण॒श इति॑ गण - शः । \newline
22. ए॒वास्मा॑ अस्मा ए॒वैवास्मै॑ सजा॒तान् थ्स॑जा॒ता न॑स्मा ए॒वैवास्मै॑ सजा॒तान् । \newline
23. अ॒स्मै॒ स॒जा॒तान् थ्स॑जा॒ता न॑स्मा अस्मै सजा॒ता नवाव॑ सजा॒ता न॑स्मा अस्मै सजा॒ता नव॑ । \newline
24. स॒जा॒ता नवाव॑ सजा॒तान् थ्स॑जा॒ता नव॑ रुन्धे रु॒न्धे ऽव॑ सजा॒तान् थ्स॑जा॒ता नव॑ रुन्धे । \newline
25. स॒जा॒तानिति॑ स - जा॒तान् । \newline
26. अव॑ रुन्धे रु॒न्धे ऽवाव॑ रुन्धे ऽनू॒च्यमा॑ने ऽनू॒च्यमा॑ने रु॒न्धे ऽवाव॑ रुन्धे ऽनू॒च्यमा॑ने । \newline
27. रु॒न्धे॒ ऽनू॒च्यमा॑ने ऽनू॒च्यमा॑ने रुन्धे रुन्धे ऽनू॒च्यमा॑न॒ आ ऽनू॒च्यमा॑ने रुन्धे रुन्धे ऽनू॒च्यमा॑न॒ आ । \newline
28. अ॒नू॒च्यमा॑न॒ आ ऽनू॒च्यमा॑ने ऽनू॒च्यमा॑न॒ आ सा॑दयति सादय॒त्या ऽनू॒च्यमा॑ने ऽनू॒च्यमा॑न॒ आ सा॑दयति । \newline
29. अ॒नू॒च्यमा॑न॒ इत्य॑नु - उ॒च्यमा॑ने । \newline
30. आ सा॑दयति सादय॒त्या सा॑दयति॒ विशं॒ ॅविशꣳ॑ सादय॒त्या सा॑दयति॒ विश᳚म् । \newline
31. सा॒द॒य॒ति॒ विशं॒ ॅविशꣳ॑ सादयति सादयति॒ विश॑ मे॒वैव विशꣳ॑ सादयति सादयति॒ विश॑ मे॒व । \newline
32. विश॑ मे॒वैव विशं॒ ॅविश॑ मे॒वास्मा॑ अस्मा ए॒व विशं॒ ॅविश॑ मे॒वास्मै᳚ । \newline
33. ए॒वास्मा॑ अस्मा ए॒वैवास्मा॒ अनु॑वर्त्मान॒ मनु॑वर्त्मान मस्मा ए॒वैवास्मा॒ अनु॑वर्त्मानम् । \newline
34. अ॒स्मा॒ अनु॑वर्त्मान॒ मनु॑वर्त्मान मस्मा अस्मा॒ अनु॑वर्त्मानम् करोति करो॒ त्यनु॑वर्त्मान मस्मा अस्मा॒ अनु॑वर्त्मानम् करोति । \newline
35. अनु॑वर्त्मानम् करोति करो॒ त्यनु॑वर्त्मान॒ मनु॑वर्त्मानम् करोति । \newline
36. अनु॑वर्त्मान॒मित्यनु॑ - व॒र्त्मा॒न॒म् । \newline
37. क॒रो॒तीति॑ करोति । \newline
\pagebreak
\markright{ TS 2.2.6.1  \hfill https://www.vedavms.in \hfill}
\addcontentsline{toc}{section}{ TS 2.2.6.1 }
\section*{ TS 2.2.6.1 }

\textbf{TS 2.2.6.1 } \newline
\textbf{Samhita Paata} \newline

आ॒दि॒त्यं च॒रुं निर्व॑पेथ् संग्रा॒म-मु॑पप्रया॒स्यन्नि॒यं ॅवा अदि॑तिर॒स्यामे॒व पूर्वे॒ प्रति॑तिष्ठन्ति वैश्वान॒रं द्वाद॑शकपालं॒ निर्व॑पेदा॒यत॑नं ग॒त्वासं॑ॅवथ्स॒रो वा अ॒ग्नि र्वै᳚श्वान॒रः सं॑ॅवथ्स॒रः खलु॒ वै दे॒वाना॑-मा॒यत॑नमे॒तस्मा॒द्वा आ॒यत॑नाद्-दे॒वा असु॑रानजय॒न॒. यद्-वै᳚श्वान॒रं द्वाद॑शकपालं नि॒र्वप॑ति दे॒वाना॑मे॒वायत॑ने यतते॒ जय॑ति॒ तꣳ स॑ग्रां॒ममे॒तस्मि॒न् वा ए॒तौ मृ॑जाते॒ - [  ] \newline

\textbf{Pada Paata} \newline

आ॒दि॒त्यम् । च॒रुम् । निरिति॑ । व॒पे॒त् । स॒ग्रां॒ममिति॑ सं - ग्रा॒मम् । उ॒प॒प्र॒या॒स्यन्नित्यु॑प - प्र॒या॒स्यन्न् । इ॒यम् । वै । अदि॑तिः । अ॒स्याम् । ए॒व । पूर्वे᳚ । प्रतीति॑ । ति॒ष्ठ॒न्ति॒ । वै॒श्वा॒न॒रम् । द्वाद॑शकपाल॒मिति॒ द्वाद॑श - क॒पा॒ल॒म् । निरिति॑ । व॒पे॒त् । आ॒यत॑न॒मित्या᳚ - यत॑नम् । ग॒त्वा । सं॒ॅव॒थ्स॒र इति॑ सं - व॒थ्स॒रः । वै । अ॒ग्निः । वै॒श्वा॒न॒रः । सं॒ॅव॒थ्स॒र इति॑ सं-व॒थ्स॒रः । खलु॑ । वै । दे॒वाना᳚म् । आ॒यत॑न॒मित्या᳚ - यत॑नम् । ए॒तस्मा᳚त् । वै । आ॒यत॑ना॒दित्या᳚-यत॑नात् । दे॒वाः । असु॑रान् । अ॒ज॒य॒न्न् । यत् । वै॒श्वा॒न॒रम् । द्वाद॑शकपाल॒मिति॒ द्वाद॑श - क॒पा॒ल॒म् । नि॒र्वप॒तीति॑ निः - वप॑ति । दे॒वाना᳚म् । ए॒व । आ॒यत॑न॒ इत्या᳚ - यत॑ने । य॒त॒ते॒ । जय॑ति । तम् । स॒ग्रां॒ममिति॑ सं - ग्रा॒मम् । ए॒तस्मिन्न्॑ । वै । ए॒तौ । मृ॒जा॒ते॒ इति॑ ।  \newline


\textbf{Krama Paata} \newline

आ॒दि॒त्यम् च॒रुम् । च॒रुम् निः । निर् व॑पेत् । व॒पे॒थ् स॒ङ्ग्रा॒मम् । स॒ङ्ग्रा॒ममु॑पप्रया॒स्यन्न् । स॒ङ्ग्रा॒ममिति॑ सं - ग्रा॒मम् । उ॒प॒प्र॒या॒स्यन्नि॒यम् । उ॒प॒प्र॒या॒स्यन्नित्यु॑प - प्र॒या॒स्यन्न् । इ॒यं ॅवै । वा अदि॑तिः । अदि॑तिर॒स्याम् । अ॒स्यामे॒व । ए॒व पूर्वे᳚ । पूर्वे॒ प्रति॑ । प्रति॑ तिष्ठन्ति । ति॒ष्ठ॒न्ति॒ वै॒श्वा॒न॒रम् । वै॒श्वा॒न॒रम् द्वाद॑शकपालम् । द्वाद॑शकपाल॒म् निः । द्वाद॑शकपाल॒मिति॒ द्वाद॑श - क॒पा॒ल॒म् । निर् व॑पेत् । व॒पे॒दा॒यत॑नम् । आ॒यत॑नम् ग॒त्वा । आ॒यत॑न॒मित्या᳚ - यत॑नम् । ग॒त्वा स॑म्ॅवथ्स॒रः । स॒म्ॅव॒थ्स॒रो वै । स॒म्ॅव॒थ्स॒र इति॑ सं - व॒थ्स॒रः । वा अ॒ग्निः । अ॒ग्निर् वै᳚श्वान॒रः । वै॒श्वा॒न॒रः स॑म्ॅवथ्स॒रः । स॒म्ॅव॒थ्स॒रः खलु॑ । स॒म्ॅव॒थ्स॒र इति॑ सं - व॒थ्स॒रः । खलु॒ वै । वै दे॒वाना᳚म् । दे॒वाना॑मा॒यत॑नम् । आ॒यत॑नमे॒तस्मा᳚त् । आ॒यत॑न॒मित्या᳚ - यत॑नम् । ए॒तस्मा॒द् वै । वा आ॒यत॑नात् । आ॒यत॑नाद् दे॒वाः । आ॒यत॑ना॒दित्या᳚ - यत॑नात् । दे॒वा असु॑रान् । असु॑रानजयन्न् । अ॒ज॒य॒न्॒. यत् । यद् वै᳚शान॒रम् । वै॒श्वा॒न॒रम् द्वाद॑शकपालम् । द्वाद॑शकपालम् नि॒र्वप॑ति । द्वाद॑शकपाल॒मिति॒ द्वाद॑श - क॒पा॒ल॒म् । नि॒र्वप॑ति दे॒वाना᳚म् । नि॒र्वप॒तीति॑ निः - वप॑ति । दे॒वाना॑मे॒व । ए॒वायत॑ने । आ॒यत॑ने यतते । आ॒यत॑न॒ इत्या᳚ - यत॑ने । य॒त॒ते॒ जय॑ति । जय॑ति॒ तम् । तꣳ स॑ङ्ग्रा॒मम् । स॒ङ्ग्रा॒ममे॒तस्मिन्न्॑ । स॒ङ्ग्रा॒ममिति॑ सं - ग्रा॒मम् । ए॒तस्मि॒न् वै । वा ए॒तौ । ए॒तौ मृ॑जाते । मृ॒जा॒ते॒ यः । मृ॒जा॒ते॒ इति॑ मृजाते \newline

\textbf{Jatai Paata} \newline

1. आ॒दि॒त्यम् च॒रुम् च॒रु मा॑दि॒त्य मा॑दि॒त्यम् च॒रुम् । \newline
2. च॒रुम् निर् णिश्च॒रुम् च॒रुम् निः । \newline
3. निर् व॑पेद् वपे॒न् निर् णिर् व॑पेत् । \newline
4. व॒पे॒थ् स॒ङ्ग्रा॒मꣳ स॑ङ्ग्रा॒मं ॅव॑पेद् वपेथ् सङ्ग्रा॒मम् । \newline
5. स॒ङ्ग्रा॒म मु॑पप्रया॒स्यन् नु॑पप्रया॒स्यन् थ्स॑ङ्ग्रा॒मꣳ स॑ङ्ग्रा॒म मु॑पप्रया॒स्यन्न् । \newline
6. स॒ङ्ग्रा॒ममिति॑ सं - ग्रा॒मम् । \newline
7. उ॒प॒प्र॒या॒स्यन् नि॒य मि॒य मु॑पप्रया॒स्यन् नु॑पप्रया॒स्यन् नि॒यम् । \newline
8. उ॒प॒प्र॒या॒स्यन्नित्यु॑प - प्र॒या॒स्यन्न् । \newline
9. इ॒यं ॅवै वा इ॒य मि॒यं ॅवै । \newline
10. वा अदि॑ति॒ रदि॑ति॒र् वै वा अदि॑तिः । \newline
11. अदि॑ति र॒स्या म॒स्या मदि॑ति॒ रदि॑ति र॒स्याम् । \newline
12. अ॒स्या मे॒वैवास्या म॒स्या मे॒व । \newline
13. ए॒व पूर्वे॒ पूर्व॑ ए॒वैव पूर्वे᳚ । \newline
14. पूर्वे॒ प्रति॒ प्रति॒ पूर्वे॒ पूर्वे॒ प्रति॑ । \newline
15. प्रति॑ तिष्ठन्ति तिष्ठन्ति॒ प्रति॒ प्रति॑ तिष्ठन्ति । \newline
16. ति॒ष्ठ॒न्ति॒ वै॒श्वा॒न॒रं ॅवै᳚श्वान॒रम् ति॑ष्ठन्ति तिष्ठन्ति वैश्वान॒रम् । \newline
17. वै॒श्वा॒न॒रम् द्वाद॑शकपाल॒म् द्वाद॑शकपालं ॅवैश्वान॒रं ॅवै᳚श्वान॒रम् द्वाद॑शकपालम् । \newline
18. द्वाद॑शकपाल॒म् निर् णिर् द्वाद॑शकपाल॒म् द्वाद॑शकपाल॒म् निः । \newline
19. द्वाद॑शकपाल॒मिति॒ द्वाद॑श - क॒पा॒ल॒म् । \newline
20. निर् व॑पेद् वपे॒न् निर् णिर् व॑पेत् । \newline
21. व॒पे॒ दा॒यत॑न मा॒यत॑नं ॅवपेद् वपे दा॒यत॑नम् । \newline
22. आ॒यत॑नम् ग॒त्वा ग॒त्वा ऽऽयत॑न मा॒यत॑नम् ग॒त्वा । \newline
23. आ॒यत॑न॒मित्या᳚ - यत॑नम् । \newline
24. ग॒त्वा सं॑ॅवथ्स॒रः सं॑ॅवथ्स॒रो ग॒त्वा ग॒त्वा सं॑ॅवथ्स॒रः । \newline
25. सं॒ॅव॒थ्स॒रो वै वै सं॑ॅवथ्स॒रः सं॑ॅवथ्स॒रो वै । \newline
26. सं॒ॅव॒थ्स॒र इति॑ सं - व॒थ्स॒रः । \newline
27. वा अ॒ग्नि र॒ग्निर् वै वा अ॒ग्निः । \newline
28. अ॒ग्निर् वै᳚श्वान॒रो वै᳚श्वान॒रो᳚ ऽग्नि र॒ग्निर् वै᳚श्वान॒रः । \newline
29. वै॒श्वा॒न॒रः सं॑ॅवथ्स॒रः सं॑ॅवथ्स॒रो वै᳚श्वान॒रो वै᳚श्वान॒रः सं॑ॅवथ्स॒रः । \newline
30. सं॒ॅव॒थ्स॒रः खलु॒ खलु॑ संॅवथ्स॒रः सं॑ॅवथ्स॒रः खलु॑ । \newline
31. सं॒ॅव॒थ्स॒र इति॑ सं - व॒थ्स॒रः । \newline
32. खलु॒ वै वै खलु॒ खलु॒ वै । \newline
33. वै दे॒वाना᳚म् दे॒वानां॒ ॅवै वै दे॒वाना᳚म् । \newline
34. दे॒वाना॑ मा॒यत॑न मा॒यत॑नम् दे॒वाना᳚म् दे॒वाना॑ मा॒यत॑नम् । \newline
35. आ॒यत॑न मे॒तस्मा॑ दे॒तस्मा॑ दा॒यत॑न मा॒यत॑न मे॒तस्मा᳚त् । \newline
36. आ॒यत॑न॒मित्या᳚ - यत॑नम् । \newline
37. ए॒तस्मा॒द् वै वा ए॒तस्मा॑ दे॒तस्मा॒द् वै । \newline
38. वा आ॒यत॑ना दा॒यत॑ना॒द् वै वा आ॒यत॑नात् । \newline
39. आ॒यत॑नाद् दे॒वा दे॒वा आ॒यत॑ना दा॒यत॑नाद् दे॒वाः । \newline
40. आ॒यत॑ना॒दित्या᳚ - यत॑नात् । \newline
41. दे॒वा असु॑रा॒ नसु॑रान् दे॒वा दे॒वा असु॑रान् । \newline
42. असु॑रा नजयन् नजय॒न् नसु॑रा॒ नसु॑रा नजयन्न् । \newline
43. अ॒ज॒य॒न्॒. यद् यद॑जयन् नजय॒न्॒. यत् । \newline
44. यद् वै᳚श्वान॒रं ॅवै᳚श्वान॒रं ॅयद् यद् वै᳚श्वान॒रम् । \newline
45. वै॒श्वा॒न॒रम् द्वाद॑शकपाल॒म् द्वाद॑शकपालं ॅवैश्वान॒रं ॅवै᳚श्वान॒रम् द्वाद॑शकपालम् । \newline
46. द्वाद॑शकपालम् नि॒र्वप॑ति नि॒र्वप॑ति॒ द्वाद॑शकपाल॒म् द्वाद॑शकपालम् नि॒र्वप॑ति । \newline
47. द्वाद॑शकपाल॒मिति॒ द्वाद॑श - क॒पा॒ल॒म् । \newline
48. नि॒र्वप॑ति दे॒वाना᳚म् दे॒वाना᳚म् नि॒र्वप॑ति नि॒र्वप॑ति दे॒वाना᳚म् । \newline
49. नि॒र्वप॒तीति॑ निः - वप॑ति । \newline
50. दे॒वाना॑ मे॒वैव दे॒वाना᳚म् दे॒वाना॑ मे॒व । \newline
51. ए॒वायत॑न आ॒यत॑न ए॒वैवायत॑ने । \newline
52. आ॒यत॑ने यतते यतत आ॒यत॑न आ॒यत॑ने यतते । \newline
53. आ॒यत॑न॒ इत्या᳚ - यत॑ने । \newline
54. य॒त॒ते॒ जय॑ति॒ जय॑ति यतते यतते॒ जय॑ति । \newline
55. जय॑ति॒ तम् तम् जय॑ति॒ जय॑ति॒ तम् । \newline
56. तꣳ स॑ङ्ग्रा॒मꣳ स॑ङ्ग्रा॒मम् तम् तꣳ स॑ङ्ग्रा॒मम् । \newline
57. स॒ङ्ग्रा॒म मे॒तस्मि॑न् ने॒तस्मि᳚न् थ्सङ्ग्रा॒मꣳ स॑ङ्ग्रा॒म मे॒तस्मिन्न्॑ । \newline
58. स॒ङ्ग्रा॒ममिति॑ सं - ग्रा॒मम् । \newline
59. ए॒तस्मि॒न्॒. वै वा ए॒तस्मि॑न् ने॒तस्मि॒न्॒. वै । \newline
60. वा ए॒ता वे॒तौ वै वा ए॒तौ । \newline
61. ए॒तौ मृ॑जाते मृजाते ए॒ता वे॒तौ मृ॑जाते । \newline
62. मृ॒जा॒ते॒ यो यो मृ॑जाते मृजाते॒ यः । \newline
63. मृ॒जा॒ते॒ इति॑ मृजाते । \newline

\textbf{Ghana Paata } \newline

1. आ॒दि॒त्यम् च॒रुम् च॒रु मा॑दि॒त्य मा॑दि॒त्यम् च॒रुम् निर् णिश्च॒रु मा॑दि॒त्य मा॑दि॒त्यम् च॒रुम् निः । \newline
2. च॒रुम् निर् णिश्च॒रुम् च॒रुम् निर् व॑पेद् वपे॒न् निश्च॒रुम् च॒रुम् निर् व॑पेत् । \newline
3. निर् व॑पेद् वपे॒न् निर् णिर् व॑पेथ् सङ्ग्रा॒मꣳ स॑ङ्ग्रा॒मं ॅव॑पे॒न् निर् णिर् व॑पेथ् सङ्ग्रा॒मम् । \newline
4. व॒पे॒थ् स॒ङ्ग्रा॒मꣳ स॑ङ्ग्रा॒मं ॅव॑पेद् वपेथ् सङ्ग्रा॒म मु॑पप्रया॒स्यन् नु॑पप्रया॒स्यन् थ्स॑ङ्ग्रा॒मं ॅव॑पेद् वपेथ् सङ्ग्रा॒म मु॑पप्रया॒स्यन्न् । \newline
5. स॒ङ्ग्रा॒म मु॑पप्रया॒स्यन् नु॑पप्रया॒स्यन् थ्स॑ङ्ग्रा॒मꣳ स॑ङ्ग्रा॒म मु॑पप्रया॒स्यन् नि॒य मि॒य मु॑पप्रया॒स्यन् थ्स॑ङ्ग्रा॒मꣳ स॑ङ्ग्रा॒म मु॑पप्रया॒स्यन् नि॒यम् । \newline
6. स॒ङ्ग्रा॒ममिति॑ सं - ग्रा॒मम् । \newline
7. उ॒प॒प्र॒या॒स्यन् नि॒य मि॒य मु॑पप्रया॒स्यन् नु॑पप्रया॒स्यन् नि॒यं ॅवै वा इ॒य मु॑पप्रया॒स्यन् नु॑पप्रया॒स्यन् नि॒यं ॅवै । \newline
8. उ॒प॒प्र॒या॒स्यन्नित्यु॑प - प्र॒या॒स्यन्न् । \newline
9. इ॒यं ॅवै वा इ॒य मि॒यं ॅवा अदि॑ति॒ रदि॑ति॒र् वा इ॒य मि॒यं ॅवा अदि॑तिः । \newline
10. वा अदि॑ति॒ रदि॑ति॒र् वै वा अदि॑ति र॒स्या म॒स्या मदि॑ति॒र् वै वा अदि॑ति र॒स्याम् । \newline
11. अदि॑ति र॒स्या म॒स्या मदि॑ति॒ रदि॑ति र॒स्या मे॒वैवास्या मदि॑ति॒ रदि॑ति र॒स्या मे॒व । \newline
12. अ॒स्या मे॒वैवास्या म॒स्या मे॒व पूर्वे॒ पूर्व॑ ए॒वास्या म॒स्या मे॒व पूर्वे᳚ । \newline
13. ए॒व पूर्वे॒ पूर्व॑ ए॒वैव पूर्वे॒ प्रति॒ प्रति॒ पूर्व॑ ए॒वैव पूर्वे॒ प्रति॑ । \newline
14. पूर्वे॒ प्रति॒ प्रति॒ पूर्वे॒ पूर्वे॒ प्रति॑ तिष्ठन्ति तिष्ठन्ति॒ प्रति॒ पूर्वे॒ पूर्वे॒ प्रति॑ तिष्ठन्ति । \newline
15. प्रति॑ तिष्ठन्ति तिष्ठन्ति॒ प्रति॒ प्रति॑ तिष्ठन्ति वैश्वान॒रं ॅवै᳚श्वान॒रम् ति॑ष्ठन्ति॒ प्रति॒ प्रति॑ तिष्ठन्ति वैश्वान॒रम् । \newline
16. ति॒ष्ठ॒न्ति॒ वै॒श्वा॒न॒रं ॅवै᳚श्वान॒रम् ति॑ष्ठन्ति तिष्ठन्ति वैश्वान॒रम् द्वाद॑शकपाल॒म् द्वाद॑शकपालं ॅवैश्वान॒रम् ति॑ष्ठन्ति तिष्ठन्ति वैश्वान॒रम् द्वाद॑शकपालम् । \newline
17. वै॒श्वा॒न॒रम् द्वाद॑शकपाल॒म् द्वाद॑शकपालं ॅवैश्वान॒रं ॅवै᳚श्वान॒रम् द्वाद॑शकपाल॒म् निर् णिर् द्वाद॑शकपालं ॅवैश्वान॒रं ॅवै᳚श्वान॒रम् द्वाद॑शकपाल॒म् निः । \newline
18. द्वाद॑शकपाल॒म् निर् णिर् द्वाद॑शकपाल॒म् द्वाद॑शकपाल॒म् निर् व॑पेद् वपे॒न् निर् द्वाद॑शकपाल॒म् द्वाद॑शकपाल॒म् निर् व॑पेत् । \newline
19. द्वाद॑शकपाल॒मिति॒ द्वाद॑श - क॒पा॒ल॒म् । \newline
20. निर् व॑पेद् वपे॒न् निर् णिर् व॑पे दा॒यत॑न मा॒यत॑नं ॅवपे॒न् निर् णिर् व॑पे दा॒यत॑नम् । \newline
21. व॒पे॒ दा॒यत॑न मा॒यत॑नं ॅवपेद् वपे दा॒यत॑नम् ग॒त्वा ग॒त्वा ऽऽयत॑नं ॅवपेद् वपे दा॒यत॑नम् ग॒त्वा । \newline
22. आ॒यत॑नम् ग॒त्वा ग॒त्वा ऽऽयत॑न मा॒यत॑नम् ग॒त्वा सं॑ॅवथ्स॒रः सं॑ॅवथ्स॒रो ग॒त्वा ऽऽयत॑न मा॒यत॑नम् ग॒त्वा सं॑ॅवथ्स॒रः । \newline
23. आ॒यत॑न॒मित्या᳚ - यत॑नम् । \newline
24. ग॒त्वा सं॑ॅवथ्स॒रः सं॑ॅवथ्स॒रो ग॒त्वा ग॒त्वा सं॑ॅवथ्स॒रो वै वै सं॑ॅवथ्स॒रो ग॒त्वा ग॒त्वा सं॑ॅवथ्स॒रो वै । \newline
25. सं॒ॅव॒थ्स॒रो वै वै सं॑ॅवथ्स॒रः सं॑ॅवथ्स॒रो वा अ॒ग्नि र॒ग्निर् वै सं॑ॅवथ्स॒रः सं॑ॅवथ्स॒रो वा अ॒ग्निः । \newline
26. सं॒ॅव॒थ्स॒र इति॑ सं - व॒थ्स॒रः । \newline
27. वा अ॒ग्नि र॒ग्निर् वै वा अ॒ग्निर् वै᳚श्वान॒रो वै᳚श्वान॒रो᳚ ऽग्निर् वै वा अ॒ग्निर् वै᳚श्वान॒रः । \newline
28. अ॒ग्निर् वै᳚श्वान॒रो वै᳚श्वान॒रो᳚ ऽग्नि र॒ग्निर् वै᳚श्वान॒रः सं॑ॅवथ्स॒रः सं॑ॅवथ्स॒रो वै᳚श्वान॒रो᳚ ऽग्नि र॒ग्निर् वै᳚श्वान॒रः सं॑ॅवथ्स॒रः । \newline
29. वै॒श्वा॒न॒रः सं॑ॅवथ्स॒रः सं॑ॅवथ्स॒रो वै᳚श्वान॒रो वै᳚श्वान॒रः सं॑ॅवथ्स॒रः खलु॒ खलु॑ संॅवथ्स॒रो वै᳚श्वान॒रो वै᳚श्वान॒रः सं॑ॅवथ्स॒रः खलु॑ । \newline
30. सं॒ॅव॒थ्स॒रः खलु॒ खलु॑ संॅवथ्स॒रः सं॑ॅवथ्स॒रः खलु॒ वै वै खलु॑ संॅवथ्स॒रः सं॑ॅवथ्स॒रः खलु॒ वै । \newline
31. सं॒ॅव॒थ्स॒र इति॑ सं - व॒थ्स॒रः । \newline
32. खलु॒ वै वै खलु॒ खलु॒ वै दे॒वाना᳚म् दे॒वानां॒ ॅवै खलु॒ खलु॒ वै दे॒वाना᳚म् । \newline
33. वै दे॒वाना᳚म् दे॒वानां॒ ॅवै वै दे॒वाना॑ मा॒यत॑न मा॒यत॑नम् दे॒वानां॒ ॅवै वै दे॒वाना॑ मा॒यत॑नम् । \newline
34. दे॒वाना॑ मा॒यत॑न मा॒यत॑नम् दे॒वाना᳚म् दे॒वाना॑ मा॒यत॑न मे॒तस्मा॑ दे॒तस्मा॑ दा॒यत॑नम् दे॒वाना᳚म् दे॒वाना॑ मा॒यत॑न मे॒तस्मा᳚त् । \newline
35. आ॒यत॑न मे॒तस्मा॑ दे॒तस्मा॑ दा॒यत॑न मा॒यत॑न मे॒तस्मा॒द् वै वा ए॒तस्मा॑ दा॒यत॑न मा॒यत॑न मे॒तस्मा॒द् वै । \newline
36. आ॒यत॑न॒मित्या᳚ - यत॑नम् । \newline
37. ए॒तस्मा॒द् वै वा ए॒तस्मा॑ दे॒तस्मा॒द् वा आ॒यत॑ना दा॒यत॑ना॒द् वा ए॒तस्मा॑ दे॒तस्मा॒द् वा आ॒यत॑नात् । \newline
38. वा आ॒यत॑ना दा॒यत॑ना॒द् वै वा आ॒यत॑नाद् दे॒वा दे॒वा आ॒यत॑ना॒द् वै वा आ॒यत॑नाद् दे॒वाः । \newline
39. आ॒यत॑नाद् दे॒वा दे॒वा आ॒यत॑ना दा॒यत॑नाद् दे॒वा असु॑रा॒ नसु॑रान् दे॒वा आ॒यत॑ना दा॒यत॑नाद् दे॒वा असु॑रान् । \newline
40. आ॒यत॑ना॒दित्या᳚ - यत॑नात् । \newline
41. दे॒वा असु॑रा॒ नसु॑रान् दे॒वा दे॒वा असु॑रा नजयन् नजय॒न् नसु॑रान् दे॒वा दे॒वा असु॑रा नजयन्न् । \newline
42. असु॑रा नजयन् नजय॒न् नसु॑रा॒ नसु॑रा नजय॒न्॒. यद् यद॑जय॒न् नसु॑रा॒ नसु॑रा नजय॒न्॒. यत् । \newline
43. अ॒ज॒य॒न्॒. यद् यद॑जयन् नजय॒न्॒. यद् वै᳚श्वान॒रं ॅवै᳚श्वान॒रं ॅयद॑जयन् नजय॒न्॒. यद् वै᳚श्वान॒रम् । \newline
44. यद् वै᳚श्वान॒रं ॅवै᳚श्वान॒रं ॅयद् यद् वै᳚श्वान॒रम् द्वाद॑शकपाल॒म् द्वाद॑शकपालं ॅवैश्वान॒रं ॅयद् यद् वै᳚श्वान॒रम् द्वाद॑शकपालम् । \newline
45. वै॒श्वा॒न॒रम् द्वाद॑शकपाल॒म् द्वाद॑शकपालं ॅवैश्वान॒रं ॅवै᳚श्वान॒रम् द्वाद॑शकपालम् नि॒र्वप॑ति नि॒र्वप॑ति॒ द्वाद॑शकपालं ॅवैश्वान॒रं ॅवै᳚श्वान॒रम् द्वाद॑शकपालम् नि॒र्वप॑ति । \newline
46. द्वाद॑शकपालम् नि॒र्वप॑ति नि॒र्वप॑ति॒ द्वाद॑शकपाल॒म् द्वाद॑शकपालम् नि॒र्वप॑ति दे॒वाना᳚म् दे॒वाना᳚म् नि॒र्वप॑ति॒ द्वाद॑शकपाल॒म् द्वाद॑शकपालम् नि॒र्वप॑ति दे॒वाना᳚म् । \newline
47. द्वाद॑शकपाल॒मिति॒ द्वाद॑श - क॒पा॒ल॒म् । \newline
48. नि॒र्वप॑ति दे॒वाना᳚म् दे॒वाना᳚म् नि॒र्वप॑ति नि॒र्वप॑ति दे॒वाना॑ मे॒वैव दे॒वाना᳚म् नि॒र्वप॑ति नि॒र्वप॑ति दे॒वाना॑ मे॒व । \newline
49. नि॒र्वप॒तीति॑ निः - वप॑ति । \newline
50. दे॒वाना॑ मे॒वैव दे॒वाना᳚म् दे॒वाना॑ मे॒वायत॑न आ॒यत॑न ए॒व दे॒वाना᳚म् दे॒वाना॑ मे॒वायत॑ने । \newline
51. ए॒वायत॑न आ॒यत॑न ए॒वैवायत॑ने यतते यतत आ॒यत॑न ए॒वैवायत॑ने यतते । \newline
52. आ॒यत॑ने यतते यतत आ॒यत॑न आ॒यत॑ने यतते॒ जय॑ति॒ जय॑ति यतत आ॒यत॑न आ॒यत॑ने यतते॒ जय॑ति । \newline
53. आ॒यत॑न॒ इत्या᳚ - यत॑ने । \newline
54. य॒त॒ते॒ जय॑ति॒ जय॑ति यतते यतते॒ जय॑ति॒ तम् तम् जय॑ति यतते यतते॒ जय॑ति॒ तम् । \newline
55. जय॑ति॒ तम् तम् जय॑ति॒ जय॑ति॒ तꣳ स॑ङ्ग्रा॒मꣳ स॑ङ्ग्रा॒मम् तम् जय॑ति॒ जय॑ति॒ तꣳ स॑ङ्ग्रा॒मम् । \newline
56. तꣳ स॑ङ्ग्रा॒मꣳ स॑ङ्ग्रा॒मम् तम् तꣳ स॑ङ्ग्रा॒म मे॒तस्मि॑न् ने॒तस्मि᳚न् थ्सङ्ग्रा॒मम् तम् तꣳ स॑ङ्ग्रा॒म मे॒तस्मिन्न्॑ । \newline
57. स॒ङ्ग्रा॒म मे॒तस्मि॑न् ने॒तस्मि᳚न् थ्सङ्ग्रा॒मꣳ स॑ङ्ग्रा॒म मे॒तस्मि॒न्॒. वै वा ए॒तस्मि᳚न् थ्सङ्ग्रा॒मꣳ स॑ङ्ग्रा॒म मे॒तस्मि॒न्॒. वै । \newline
58. स॒ङ्ग्रा॒ममिति॑ सं - ग्रा॒मम् । \newline
59. ए॒तस्मि॒न्॒. वै वा ए॒तस्मि॑न् ने॒तस्मि॒न्॒. वा ए॒ता वे॒तौ वा ए॒तस्मि॑न् ने॒तस्मि॒न्॒. वा ए॒तौ । \newline
60. वा ए॒ता वे॒तौ वै वा ए॒तौ मृ॑जाते मृजाते ए॒तौ वै वा ए॒तौ मृ॑जाते । \newline
61. ए॒तौ मृ॑जाते मृजाते ए॒ता वे॒तौ मृ॑जाते॒ यो यो मृ॑जाते ए॒ता वे॒तौ मृ॑जाते॒ यः । \newline
62. मृ॒जा॒ते॒ यो यो मृ॑जाते मृजाते॒ यो वि॑द्विषा॒णयो᳚र् विद्विषा॒णयो॒र् यो मृ॑जाते मृजाते॒ यो वि॑द्विषा॒णयोः᳚ । \newline
63. मृ॒जा॒ते॒ इति॑ मृजाते । \newline
\pagebreak
\markright{ TS 2.2.6.2  \hfill https://www.vedavms.in \hfill}
\addcontentsline{toc}{section}{ TS 2.2.6.2 }
\section*{ TS 2.2.6.2 }

\textbf{TS 2.2.6.2 } \newline
\textbf{Samhita Paata} \newline

यो वि॑द्विषा॒णयो॒रन्न॒मत्ति॑ वैश्वान॒रं द्वाद॑शकपालं॒ निर्व॑पेद्-विद्विषा॒णयो॒रन्नं॑ ज॒ग्ध्वा सं॑ॅवथ्स॒रो वा अ॒ग्नि-र्वै᳚श्वान॒रः सं॑ॅवथ्स॒र स्व॑दितमे॒वात्ति॒ नास्मि॑न् मृजाते संॅवथ्स॒राय॒ वा ए॒तौ सम॑माते॒ यौ स॑म॒माते॒ तयो॒र्यः पूर्वो॑ऽभि॒द्रुह्य॑ति॒ तं ॅवरु॑णो गृह्णाति वैश्वान॒रं द्वाद॑शकपालं॒ निर्व॑पेथ् सममा॒नयोः॒ पूर्वो॑ऽभि॒द्रुह्य॑ संॅवथ्स॒रो वा अ॒ग्निर्वै᳚श्वान॒रः सं॑ॅवथ्स॒रमे॒वाऽऽ*प्त्वा नि॑र्वरु॒णं - [  ] \newline

\textbf{Pada Paata} \newline

यः । वि॒द्वि॒षा॒णयो॒रिति॑ वि-द्वि॒षा॒णयोः᳚ । अन्न᳚म् । अत्ति॑ । वै॒श्वा॒न॒रम् । द्वाद॑शकपाल॒मिति॒ द्वाद॑श - क॒पा॒ल॒म् । निरिति॑ । व॒पे॒त् । वि॒द्वि॒षा॒णयो॒रिति॑ वि - द्वि॒षा॒णयोः᳚ । अन्न᳚म् । ज॒ग्ध्वा । सं॒ॅव॒थ्स॒र इति॑ सं - व॒थ्स॒रः । वै । अ॒ग्निः । वै॒श्वा॒न॒रः । सं॒ॅव॒थ्स॒रस्व॑दित॒मिति॑ संॅवथ्स॒र - स्व॒दि॒त॒म् । ए॒व । अ॒त्ति॒ । न । अ॒स्मि॒न्न् । मृ॒जा॒ते॒ इति॑ । सं॒ॅव॒थ्स॒रायेति॑ सं - व॒थ्स॒राय॑ । वै । ए॒तौ । समिति॑ । अ॒मा॒ते॒ इति॑ । यौ । स॒म॒माते॒ इति॑ सं-अ॒माते᳚ । तयोः᳚ । यः । पूर्वः॑ । अ॒भि॒द्रुह्य॒तीत्य॑भि - द्रुह्य॑ति । तम् । वरु॑णः । गृ॒ह्णा॒ति॒ । वै॒श्वा॒न॒रम् । द्वाद॑शकपाल॒मिति॒ द्वाद॑श - क॒पा॒ल॒म् । निरिति॑ । व॒पे॒त् । स॒म॒मा॒नयो॒रिति॑ सं - अ॒मा॒नयोः᳚ । पूर्वः॑ । अ॒भि॒द्रुह्येत्य॑भि - द्रुह्य॑ । सं॒ॅव॒थ्स॒र इति॑ सं - व॒थ्स॒रः । वै । अ॒ग्निः । वै॒श्वा॒न॒रः । सं॒ॅव॒थ्स॒रमिति॑ सं - व॒थ्स॒रम् । ए॒व । आ॒प्त्वा । नि॒र्व॒रु॒णमिति॑ निः - व॒रु॒णम् ।  \newline


\textbf{Krama Paata} \newline

यो वि॑द्विषा॒णयोः᳚ । वि॒द्वि॒षा॒णयो॒रन्न᳚म् । वि॒द्वि॒षा॒णयो॒रिति॑ वि - द्वि॒षा॒णयोः᳚ । अन्न॒मत्ति॑ । अत्ति॑ वैश्वान॒रम् । वै॒श्वा॒न॒रम् द्वाद॑शकपालम् । द्वाद॑शकपाल॒म् निः । द्वाद॑शकपाल॒मिति॒ द्वाद॑श - क॒पा॒ल॒म् । निर् व॑पेत् । व॒पे॒द् वि॒द्वि॒षा॒णयोः᳚ । वि॒द्वि॒षा॒णयो॒रन्न᳚म् । वि॒द्वि॒षा॒णयो॒रिति॑ वि - द्वि॒षा॒णयोः᳚ । अन्न॑म् ज॒ग्द्ध्वा । ज॒ग्द्ध्वा स॑म्ॅवथ्स॒रः । स॒म्ॅव॒थ्स॒रो वै । स॒म्ॅव॒थ्स॒र इति॑ सं - व॒थ्स॒रः । वा अ॒ग्निः । अ॒ग्निर् वै᳚श्वान॒रः । वै॒श्वा॒न॒रः स॑म्ॅवथ्स॒रस्व॑दितम् । स॒म्ॅव॒थ्स॒रस्व॑दितमे॒व । स॒म्ॅव॒थ्स॒रस्व॑दित॒मिति॑ सम्ॅवथ्स॒र - स्व॒दि॒त॒म् । ए॒वात्ति॑ । अ॒त्ति॒ न । नास्मिन्न्॑ । अ॒स्मि॒न् मृ॒जा॒ते॒ । मृ॒जा॒ते॒ स॒म्ॅव॒थ्स॒राय॑ । मृ॒जा॒ते॒ इति॑ मृजाते । स॒म्ॅव॒थ्स॒राय॒ वै । स॒म्ॅव॒थ्स॒रायेति॑ सं - व॒थ्स॒राय॑ । वा ए॒तौ । ए॒तौ सम् । सम॑माते । अ॒मा॒ते॒ यौ । अ॒मा॒ते॒ इत्य॑माते । यौ स॑म॒माते᳚ । स॒म॒माते॒ तयोः᳚ । स॒म॒माते॒ इति॑ सं - अ॒माते᳚ । तयो॒र् यः । यः पूर्वः॑ । पूर्वो॑ ऽभि॒द्रुह्य॑ति । अ॒भि॒द्रुह्य॑ति॒ तम् । अ॒भि॒द्रुह्य॒तीत्य॑भि - द्रुह्य॑ति । तं ॅवरु॑णः । वरु॑णो गृह्णाति । गृ॒ह्णा॒ति॒ वै॒श्वा॒न॒रम् । वै॒श्वा॒न॒रम् द्वाद॑शकपालम् । द्वाद॑शकपाल॒म् निः । द्वाद॑शकपाल॒मिति॒ द्वाद॑श - क॒पा॒ल॒म् । निर् व॑पेत् । व॒पे॒थ् स॒म॒मा॒नयोः᳚ । स॒म॒म॒नयोः॒ पूर्वः॑ । स॒म॒मा॒नयो॒रिति॑ सं - अ॒मा॒नयोः᳚ । पूर्वो॑ऽभि॒द्रुह्य॑ । अ॒भि॒द्रुह्य॑ सम्ॅवथ्स॒रः । अ॒भि॒द्रुह्येत्य॑भि - द्रुह्य॑ । स॒म्ॅव॒थ्स॒रो वै । स॒म्ॅव॒थ्स॒र इति॑ सं - व॒थ्स॒रः । वा अ॒ग्निः । अ॒ग्निर् वै᳚श्वान॒रः । वै॒श्वा॒न॒रः स॑म्ॅवथ्स॒रम् । स॒म्ॅव॒थ्स॒रमे॒व । स॒म्ॅव॒थ्स॒रमिति॑ सं - व॒थ्स॒रम् । ए॒वाप्त्वा । आ॒प्त्वा नि॑र्वरु॒णम् । नि॒र्व॒रु॒णम् प॒रस्ता᳚त् । नि॒र्व॒रु॒णमिति॑ निः - व॒रु॒णम् \newline

\textbf{Jatai Paata} \newline

1. यो वि॑द्विषा॒णयो᳚र् विद्विषा॒णयो॒र् यो यो वि॑द्विषा॒णयोः᳚ । \newline
2. वि॒द्वि॒षा॒णयो॒रन्न॒ मन्नं॑ ॅविद्विषा॒णयो᳚र् विद्विषा॒णयो॒रन्न᳚म् । \newline
3. वि॒द्वि॒षा॒णयो॒रिति॑ वि - द्वि॒षा॒णयोः᳚ । \newline
4. अन्न॒ म त्त्य त्त्यन्न॒ मन्न॒ मत्ति॑ । \newline
5. अत्ति॑ वैश्वान॒रं ॅवै᳚श्वान॒र मत्त्यत्ति॑ वैश्वान॒रम् । \newline
6. वै॒श्वा॒न॒रम् द्वाद॑शकपाल॒म् द्वाद॑शकपालं ॅवैश्वान॒रं ॅवै᳚श्वान॒रम् द्वाद॑शकपालम् । \newline
7. द्वाद॑शकपाल॒म् निर् णिर् द्वाद॑शकपाल॒म् द्वाद॑शकपाल॒म् निः । \newline
8. द्वाद॑शकपाल॒मिति॒ द्वाद॑श - क॒पा॒ल॒म् । \newline
9. निर् व॑पेद् वपे॒न् निर् णिर् व॑पेत् । \newline
10. व॒पे॒द् वि॒द्वि॒षा॒णयो᳚र् विद्विषा॒णयो᳚र् वपेद् वपेद् विद्विषा॒णयोः᳚ । \newline
11. वि॒द्वि॒षा॒णयो॒ रन्न॒ मन्नं॑ ॅविद्विषा॒णयो᳚र् विद्विषा॒णयो॒ रन्न᳚म् । \newline
12. वि॒द्वि॒षा॒णयो॒रिति॑ वि - द्वि॒षा॒णयोः᳚ । \newline
13. अन्न॑म् ज॒ग्ध्वा ज॒ग्ध्वा ऽन्न॒ मन्न॑म् ज॒ग्ध्वा । \newline
14. ज॒ग्ध्वा सं॑ॅवथ्स॒रः सं॑ॅवथ्स॒रो ज॒ग्ध्वा ज॒ग्ध्वा सं॑ॅवथ्स॒रः । \newline
15. सं॒ॅव॒थ्स॒रो वै वै सं॑ॅवथ्स॒रः सं॑ॅवथ्स॒रो वै । \newline
16. सं॒ॅव॒थ्स॒र इति॑ सं - व॒थ्स॒रः । \newline
17. वा अ॒ग्नि र॒ग्निर् वै वा अ॒ग्निः । \newline
18. अ॒ग्निर् वै᳚श्वान॒रो वै᳚श्वान॒रो᳚ ऽग्नि र॒ग्निर् वै᳚श्वान॒रः । \newline
19. वै॒श्वा॒न॒रः सं॑ॅवथ्स॒रस्व॑दितꣳ संॅवथ्स॒रस्व॑दितं ॅवैश्वान॒रो वै᳚श्वान॒रः सं॑ॅवथ्स॒रस्व॑दितम् । \newline
20. सं॒ॅव॒थ्स॒रस्व॑दित मे॒वैव सं॑ॅवथ्स॒रस्व॑दितꣳ संॅवथ्स॒रस्व॑दित मे॒व । \newline
21. सं॒ॅव॒थ्स॒रस्व॑दित॒मिति॑ संॅवथ्स॒र - स्व॒दि॒त॒म् । \newline
22. ए॒वा त्त्य॑ त्त्ये॒ वैवात्ति॑ । \newline
23. अ॒त्ति॒ न ना त्त्य॑त्ति॒ न । \newline
24. नास्मि॑न् नस्मि॒न् न नास्मिन्न्॑ । \newline
25. अ॒स्मि॒न् मृ॒जा॒ते॒ मृ॒जा॒ते॒ अ॒स्मि॒न् न॒स्मि॒न् मृ॒जा॒ते॒ । \newline
26. मृ॒जा॒ते॒ सं॒ॅव॒थ्स॒राय॑ संॅवथ्स॒राय॑ मृजाते मृजाते संॅवथ्स॒राय॑ । \newline
27. मृ॒जा॒ते॒ इति॑ मृजाते । \newline
28. सं॒ॅव॒थ्स॒राय॒ वै वै सं॑ॅवथ्स॒राय॑ संॅवथ्स॒राय॒ वै । \newline
29. सं॒ॅव॒थ्स॒रायेति॑ सं - व॒थ्स॒राय॑ । \newline
30. वा ए॒ता वे॒तौ वै वा ए॒तौ । \newline
31. ए॒तौ सꣳ स मे॒ता वे॒तौ सम् । \newline
32. स म॑माते अमाते॒ सꣳ स म॑माते । \newline
33. अ॒मा॒ते॒ यौ या व॑माते अमाते॒ यौ । \newline
34. अ॒मा॒ते॒ इत्य॑माते । \newline
35. यौ स॑म॒माते॑ सम॒माते॒ यौ यौ स॑म॒माते᳚ । \newline
36. स॒म॒माते॒ तयो॒ स्तयोः᳚ सम॒माते॑ सम॒माते॒ तयोः᳚ । \newline
37. स॒म॒माते॒ इति॑ सं - अ॒माते᳚ । \newline
38. तयो॒र् यो य स्तयो॒ स्तयो॒र् यः । \newline
39. यः पूर्वः॒ पूर्वो॒ यो यः पूर्वः॑ । \newline
40. पूर्वो॑ ऽभि॒द्रुह्य॑ त्यभि॒द्रुह्य॑ति॒ पूर्वः॒ पूर्वो॑ ऽभि॒द्रुह्य॑ति । \newline
41. अ॒भि॒द्रुह्य॑ति॒ तम् त म॑भि॒द्रुह्य॑ त्यभि॒द्रुह्य॑ति॒ तम् । \newline
42. अ॒भि॒द्रुह्य॒तीत्य॑भि - द्रुह्य॑ति । \newline
43. तं ॅवरु॑णो॒ वरु॑ण॒ स्तम् तं ॅवरु॑णः । \newline
44. वरु॑णो गृह्णाति गृह्णाति॒ वरु॑णो॒ वरु॑णो गृह्णाति । \newline
45. गृ॒ह्णा॒ति॒ वै॒श्वा॒न॒रं ॅवै᳚श्वान॒रम् गृ॑ह्णाति गृह्णाति वैश्वान॒रम् । \newline
46. वै॒श्वा॒न॒रम् द्वाद॑शकपाल॒म् द्वाद॑शकपालं ॅवैश्वान॒रं ॅवै᳚श्वान॒रम् द्वाद॑शकपालम् । \newline
47. द्वाद॑शकपाल॒म् निर् णिर् द्वाद॑शकपाल॒म् द्वाद॑शकपाल॒म् निः । \newline
48. द्वाद॑शकपाल॒मिति॒ द्वाद॑श - क॒पा॒ल॒म् । \newline
49. निर् व॑पेद् वपे॒न् निर् णिर् व॑पेत् । \newline
50. व॒पे॒थ् स॒म॒मा॒नयोः᳚ सममा॒नयो᳚र् वपेद् वपेथ् सममा॒नयोः᳚ । \newline
51. स॒म॒मा॒नयोः॒ पूर्वः॒ पूर्वः॑ सममा॒नयोः᳚ सममा॒नयोः॒ पूर्वः॑ । \newline
52. स॒म॒मा॒नयो॒रिति॑ सं - अ॒मा॒नयोः᳚ । \newline
53. पूर्वो॑ ऽभि॒द्रुह्या॑ भि॒द्रुह्य॒ पूर्वः॒ पूर्वो॑ ऽभि॒द्रुह्य॑ । \newline
54. अ॒भि॒द्रुह्य॑ संॅवथ्स॒रः सं॑ॅवथ्स॒रो॑ ऽभि॒द्रुह्या॑ भि॒द्रुह्य॑ संॅवथ्स॒रः । \newline
55. अ॒भि॒द्रुह्येत्य॑भि - द्रुह्य॑ । \newline
56. सं॒ॅव॒थ्स॒रो वै वै सं॑ॅवथ्स॒रः सं॑ॅवथ्स॒रो वै । \newline
57. सं॒ॅव॒थ्स॒र इति॑ सं - व॒थ्स॒रः । \newline
58. वा अ॒ग्नि र॒ग्निर् वै वा अ॒ग्निः । \newline
59. अ॒ग्निर् वै᳚श्वान॒रो वै᳚श्वान॒रो᳚ ऽग्निर॒ग्निर् वै᳚श्वान॒रः । \newline
60. वै॒श्वा॒न॒रः सं॑ॅवथ्स॒रꣳ सं॑ॅवथ्स॒रं ॅवै᳚श्वान॒रो वै᳚श्वान॒रः सं॑ॅवथ्स॒रम् । \newline
61. सं॒ॅव॒थ्स॒र मे॒वैव सं॑ॅवथ्स॒रꣳ सं॑ॅवथ्स॒र मे॒व । \newline
62. सं॒ॅव॒थ्स॒रमिति॑ सं - व॒थ्स॒रम् । \newline
63. ए॒वाप्त्वा ऽऽप्त्वैवै वाप्त्वा । \newline
64. आ॒प्त्वा नि॑र्वरु॒णम् नि॑र्वरु॒ण मा॒प्त्वा ऽऽप्त्वा नि॑र्वरु॒णम् । \newline
65. नि॒र्व॒रु॒णम् प॒रस्ता᳚त् प॒रस्ता᳚न् निर्वरु॒णम् नि॑र्वरु॒णम् प॒रस्ता᳚त् । \newline
66. नि॒र्व॒रु॒णमिति॑ निः - व॒रु॒णम् । \newline

\textbf{Ghana Paata } \newline

1. यो वि॑द्विषा॒णयो᳚र् विद्विषा॒णयो॒र् यो यो वि॑द्विषा॒णयो॒ रन्न॒ मन्नं॑ ॅविद्विषा॒णयो॒र् यो यो वि॑द्विषा॒णयो॒ रन्न᳚म् । \newline
2. वि॒द्वि॒षा॒णयो॒ रन्न॒ मन्नं॑ ॅविद्विषा॒णयो᳚र् विद्विषा॒णयो॒ रन्न॒ मत्त्यत्त्यन्नं॑ ॅविद्विषा॒णयो᳚र् विद्विषा॒णयो॒ रन्न॒ मत्ति॑ । \newline
3. वि॒द्वि॒षा॒णयो॒रिति॑ वि - द्वि॒षा॒णयोः᳚ । \newline
4. अन्न॒ मत्त्यत्त्यन्न॒ मन्न॒ मत्ति॑ वैश्वान॒रं ॅवै᳚श्वान॒र मत्त्यन्न॒ मन्न॒ मत्ति॑ वैश्वान॒रम् । \newline
5. अत्ति॑ वैश्वान॒रं ॅवै᳚श्वान॒र मत्त्यत्ति॑ वैश्वान॒रम् द्वाद॑शकपाल॒म् द्वाद॑शकपालं ॅवैश्वान॒र मत्त्यत्ति॑ वैश्वान॒रम् द्वाद॑शकपालम् । \newline
6. वै॒श्वा॒न॒रम् द्वाद॑शकपाल॒म् द्वाद॑शकपालं ॅवैश्वान॒रं ॅवै᳚श्वान॒रम् द्वाद॑शकपाल॒म् निर् णिर् द्वाद॑शकपालं ॅवैश्वान॒रं ॅवै᳚श्वान॒रम् द्वाद॑शकपाल॒म् निः । \newline
7. द्वाद॑शकपाल॒म् निर् णिर् द्वाद॑शकपाल॒म् द्वाद॑शकपाल॒म् निर् व॑पेद् वपे॒न् निर् द्वाद॑शकपाल॒म् द्वाद॑शकपाल॒म् निर् व॑पेत् । \newline
8. द्वाद॑शकपाल॒मिति॒ द्वाद॑श - क॒पा॒ल॒म् । \newline
9. निर् व॑पेद् वपे॒न् निर् णिर् व॑पेद् विद्विषा॒णयो᳚र् विद्विषा॒णयो᳚र् वपे॒न् निर् णिर् व॑पेद् विद्विषा॒णयोः᳚ । \newline
10. व॒पे॒द् वि॒द्वि॒षा॒णयो᳚र् विद्विषा॒णयो᳚र् वपेद् वपेद् विद्विषा॒णयो॒ रन्न॒ मन्नं॑ ॅविद्विषा॒णयो᳚र् वपेद् वपेद् विद्विषा॒णयो॒ रन्न᳚म् । \newline
11. वि॒द्वि॒षा॒णयो॒ रन्न॒ मन्नं॑ ॅविद्विषा॒णयो᳚र् विद्विषा॒णयो॒ रन्न॑म् ज॒ग्ध्वा ज॒ग्ध्वा ऽन्नं॑ ॅविद्विषा॒णयो᳚र् विद्विषा॒णयो॒ रन्न॑म् ज॒ग्ध्वा । \newline
12. वि॒द्वि॒षा॒णयो॒रिति॑ वि - द्वि॒षा॒णयोः᳚ । \newline
13. अन्न॑म् ज॒ग्ध्वा ज॒ग्ध्वा ऽन्न॒ मन्न॑म् ज॒ग्ध्वा सं॑ॅवथ्स॒रः सं॑ॅवथ्स॒रो ज॒ग्ध्वा ऽन्न॒ मन्न॑म् ज॒ग्ध्वा सं॑ॅवथ्स॒रः । \newline
14. ज॒ग्ध्वा सं॑ॅवथ्स॒रः सं॑ॅवथ्स॒रो ज॒ग्ध्वा ज॒ग्ध्वा सं॑ॅवथ्स॒रो वै वै सं॑ॅवथ्स॒रो ज॒ग्ध्वा ज॒ग्ध्वा सं॑ॅवथ्स॒रो वै । \newline
15. सं॒ॅव॒थ्स॒रो वै वै सं॑ॅवथ्स॒रः सं॑ॅवथ्स॒रो वा अ॒ग्नि र॒ग्निर् वै सं॑ॅवथ्स॒रः सं॑ॅवथ्स॒रो वा अ॒ग्निः । \newline
16. सं॒ॅव॒थ्स॒र इति॑ सं - व॒थ्स॒रः । \newline
17. वा अ॒ग्नि र॒ग्निर् वै वा अ॒ग्निर् वै᳚श्वान॒रो वै᳚श्वान॒रो᳚ ऽग्निर् वै वा अ॒ग्निर् वै᳚श्वान॒रः । \newline
18. अ॒ग्निर् वै᳚श्वान॒रो वै᳚श्वान॒रो᳚ ऽग्नि र॒ग्निर् वै᳚श्वान॒रः सं॑ॅवथ्स॒रस्व॑दितꣳ संॅवथ्स॒रस्व॑दितं ॅवैश्वान॒रो᳚ ऽग्नि र॒ग्निर् वै᳚श्वान॒रः सं॑ॅवथ्स॒रस्व॑दितम् । \newline
19. वै॒श्वा॒न॒रः सं॑ॅवथ्स॒रस्व॑दितꣳ संॅवथ्स॒रस्व॑दितं ॅवैश्वान॒रो वै᳚श्वान॒रः सं॑ॅवथ्स॒रस्व॑दित मे॒वैव सं॑ॅवथ्स॒रस्व॑दितं ॅवैश्वान॒रो वै᳚श्वान॒रः सं॑ॅवथ्स॒रस्व॑दित मे॒व । \newline
20. सं॒ॅव॒थ्स॒रस्व॑दित मे॒वैव सं॑ॅवथ्स॒रस्व॑दितꣳ संॅवथ्स॒रस्व॑दित मे॒वा त्त्य॑त्त्ये॒व सं॑ॅवथ्स॒रस्व॑दितꣳ संॅवथ्स॒रस्व॑दित मे॒वात्ति॑ । \newline
21. सं॒ॅव॒थ्स॒रस्व॑दित॒मिति॑ संॅवथ्स॒र - स्व॒दि॒त॒म् । \newline
22. ए॒वा त्त्य॑त्त्ये॒वैवात्ति॒ न ना त्त्ये॒वैवात्ति॒ न । \newline
23. अ॒त्ति॒ न नात्त्य॑त्ति॒ नास्मि॑न् नस्मि॒न् नात्त्य॑त्ति॒ नास्मिन्न्॑ । \newline
24. नास्मि॑न् नस्मि॒न् न नास्मि॑न् मृजाते मृजाते अस्मि॒न् न नास्मि॑न् मृजाते । \newline
25. अ॒स्मि॒न् मृ॒जा॒ते॒ मृ॒जा॒ते॒ अ॒स्मि॒न् न॒स्मि॒न् मृ॒जा॒ते॒ सं॒ॅव॒थ्स॒राय॑ संॅवथ्स॒राय॑ मृजाते अस्मिन् नस्मिन् मृजाते संॅवथ्स॒राय॑ । \newline
26. मृ॒जा॒ते॒ सं॒ॅव॒थ्स॒राय॑ संॅवथ्स॒राय॑ मृजाते मृजाते संॅवथ्स॒राय॒ वै वै सं॑ॅवथ्स॒राय॑ मृजाते मृजाते संॅवथ्स॒राय॒ वै । \newline
27. मृ॒जा॒ते॒ इति॑ मृजाते । \newline
28. सं॒ॅव॒थ्स॒राय॒ वै वै सं॑ॅवथ्स॒राय॑ संॅवथ्स॒राय॒ वा ए॒ता वे॒तौ वै सं॑ॅवथ्स॒राय॑ संॅवथ्स॒राय॒ वा ए॒तौ । \newline
29. सं॒ॅव॒थ्स॒रायेति॑ सं - व॒थ्स॒राय॑ । \newline
30. वा ए॒ता वे॒तौ वै वा ए॒तौ सꣳ स मे॒तौ वै वा ए॒तौ सम् । \newline
31. ए॒तौ सꣳ स मे॒ता वे॒तौ स म॑माते अमाते॒ स मे॒ता वे॒तौ स म॑माते । \newline
32. स म॑माते अमाते॒ सꣳ स म॑माते॒ यौ या व॑माते॒ सꣳ स म॑माते॒ यौ । \newline
33. अ॒मा॒ते॒ यौ या व॑माते अमाते॒ यौ स॑म॒माते॑ सम॒माते॒ या व॑माते अमाते॒ यौ स॑म॒माते᳚ । \newline
34. अ॒मा॒ते॒ इत्य॑माते । \newline
35. यौ स॑म॒माते॑ सम॒माते॒ यौ यौ स॑म॒माते॒ तयो॒ स्तयोः᳚ सम॒माते॒ यौ यौ स॑म॒माते॒ तयोः᳚ । \newline
36. स॒म॒माते॒ तयो॒ स्तयोः᳚ सम॒माते॑ सम॒माते॒ तयो॒र् यो य स्तयोः᳚ सम॒माते॑ सम॒माते॒ तयो॒र् यः । \newline
37. स॒म॒माते॒ इति॑ सं - अ॒माते᳚ । \newline
38. तयो॒र् यो य स्तयो॒ स्तयो॒र् यः पूर्वः॒ पूर्वो॒ य स्तयो॒ स्तयो॒र् यः पूर्वः॑ । \newline
39. यः पूर्वः॒ पूर्वो॒ यो यः पूर्वो॑ ऽभि॒द्रुह्य॑ त्यभि॒द्रुह्य॑ति॒ पूर्वो॒ यो यः पूर्वो॑ ऽभि॒द्रुह्य॑ति । \newline
40. पूर्वो॑ ऽभि॒द्रुह्य॑ त्यभि॒द्रुह्य॑ति॒ पूर्वः॒ पूर्वो॑ ऽभि॒द्रुह्य॑ति॒ तम् त म॑भि॒द्रुह्य॑ति॒ पूर्वः॒ पूर्वो॑ ऽभि॒द्रुह्य॑ति॒ तम् । \newline
41. अ॒भि॒द्रुह्य॑ति॒ तम् त म॑भि॒द्रुह्य॑ त्यभि॒द्रुह्य॑ति॒ तं ॅवरु॑णो॒ वरु॑ण॒ स्त म॑भि॒द्रुह्य॑ त्यभि॒द्रुह्य॑ति॒ तं ॅवरु॑णः । \newline
42. अ॒भि॒द्रुह्य॒तीत्य॑भि - द्रुह्य॑ति । \newline
43. तं ॅवरु॑णो॒ वरु॑ण॒ स्तम् तं ॅवरु॑णो गृह्णाति गृह्णाति॒ वरु॑ण॒स्तम् तं ॅवरु॑णो गृह्णाति । \newline
44. वरु॑णो गृह्णाति गृह्णाति॒ वरु॑णो॒ वरु॑णो गृह्णाति वैश्वान॒रं ॅवै᳚श्वान॒रम् गृ॑ह्णाति॒ वरु॑णो॒ वरु॑णो गृह्णाति वैश्वान॒रम् । \newline
45. गृ॒ह्णा॒ति॒ वै॒श्वा॒न॒रं ॅवै᳚श्वान॒रम् गृ॑ह्णाति गृह्णाति वैश्वान॒रम् द्वाद॑शकपाल॒म् द्वाद॑शकपालं ॅवैश्वान॒रम् गृ॑ह्णाति गृह्णाति वैश्वान॒रम् द्वाद॑शकपालम् । \newline
46. वै॒श्वा॒न॒रम् द्वाद॑शकपाल॒म् द्वाद॑शकपालं ॅवैश्वान॒रं ॅवै᳚श्वान॒रम् द्वाद॑शकपाल॒म् निर् णिर् द्वाद॑शकपालं ॅवैश्वान॒रं ॅवै᳚श्वान॒रम् द्वाद॑शकपाल॒म् निः । \newline
47. द्वाद॑शकपाल॒म् निर् णिर् द्वाद॑शकपाल॒म् द्वाद॑शकपाल॒म् निर् व॑पेद् वपे॒न् निर् द्वाद॑शकपाल॒म् द्वाद॑शकपाल॒म् निर् व॑पेत् । \newline
48. द्वाद॑शकपाल॒मिति॒ द्वाद॑श - क॒पा॒ल॒म् । \newline
49. निर् व॑पेद् वपे॒न् निर् णिर् व॑पेथ् सममा॒नयोः᳚ सममा॒नयो᳚र् वपे॒न् निर् णिर् व॑पेथ् सममा॒नयोः᳚ । \newline
50. व॒पे॒थ् स॒म॒मा॒नयोः᳚ सममा॒नयो᳚र् वपेद् वपेथ् सममा॒नयोः॒ पूर्वः॒ पूर्वः॑ सममा॒नयो᳚र् वपेद् वपेथ् सममा॒नयोः॒ पूर्वः॑ । \newline
51. स॒म॒मा॒नयोः॒ पूर्वः॒ पूर्वः॑ सममा॒नयोः᳚ सममा॒नयोः॒ पूर्वो॑ ऽभि॒द्रुह्या॑ भि॒द्रुह्य॒ पूर्वः॑ सममा॒नयोः᳚ सममा॒नयोः॒ पूर्वो॑ ऽभि॒द्रुह्य॑ । \newline
52. स॒म॒मा॒नयो॒रिति॑ सं - अ॒मा॒नयोः᳚ । \newline
53. पूर्वो॑ ऽभि॒द्रुह्या॑ भि॒द्रुह्य॒ पूर्वः॒ पूर्वो॑ ऽभि॒द्रुह्य॑ संॅवथ्स॒रः सं॑ॅवथ्स॒रो॑ ऽभि॒द्रुह्य॒ पूर्वः॒ पूर्वो॑ ऽभि॒द्रुह्य॑ संॅवथ्स॒रः । \newline
54. अ॒भि॒द्रुह्य॑ संॅवथ्स॒रः सं॑ॅवथ्स॒रो॑ ऽभि॒द्रुह्या॑ भि॒द्रुह्य॑ संॅवथ्स॒रो वै वै सं॑ॅवथ्स॒रो॑ ऽभि॒द्रुह्या॑ भि॒द्रुह्य॑ संॅवथ्स॒रो वै । \newline
55. अ॒भि॒द्रुह्येत्य॑भि - द्रुह्य॑ । \newline
56. सं॒ॅव॒थ्स॒रो वै वै सं॑ॅवथ्स॒रः सं॑ॅवथ्स॒रो वा अ॒ग्नि र॒ग्निर् वै सं॑ॅवथ्स॒रः सं॑ॅवथ्स॒रो वा अ॒ग्निः । \newline
57. सं॒ॅव॒थ्स॒र इति॑ सं - व॒थ्स॒रः । \newline
58. वा अ॒ग्नि र॒ग्निर् वै वा अ॒ग्निर् वै᳚श्वान॒रो वै᳚श्वान॒रो᳚ ऽग्निर् वै वा अ॒ग्निर् वै᳚श्वान॒रः । \newline
59. अ॒ग्निर् वै᳚श्वान॒रो वै᳚श्वान॒रो᳚ ऽग्नि र॒ग्निर् वै᳚श्वान॒रः सं॑ॅवथ्स॒रꣳ सं॑ॅवथ्स॒रं ॅवै᳚श्वान॒रो᳚ ऽग्नि र॒ग्निर् वै᳚श्वान॒रः सं॑ॅवथ्स॒रम् । \newline
60. वै॒श्वा॒न॒रः सं॑ॅवथ्स॒रꣳ सं॑ॅवथ्स॒रं ॅवै᳚श्वान॒रो वै᳚श्वान॒रः सं॑ॅवथ्स॒र मे॒वैव सं॑ॅवथ्स॒रं ॅवै᳚श्वान॒रो वै᳚श्वान॒रः सं॑ॅवथ्स॒र मे॒व । \newline
61. सं॒ॅव॒थ्स॒र मे॒वैव सं॑ॅवथ्स॒रꣳ सं॑ॅवथ्स॒र मे॒वाप्त्वा ऽऽप्त्वैव सं॑ॅवथ्स॒रꣳ सं॑ॅवथ्स॒र मे॒वाप्त्वा । \newline
62. सं॒ॅव॒थ्स॒रमिति॑ सं - व॒थ्स॒रम् । \newline
63. ए॒वाप्त्वा ऽऽप्त्वैवै वाप्त्वा नि॑र्वरु॒णम् नि॑र्वरु॒ण मा॒प्त्वैवै वाप्त्वा नि॑र्वरु॒णम् । \newline
64. आ॒प्त्वा नि॑र्वरु॒णम् नि॑र्वरु॒ण मा॒प्त्वा ऽऽप्त्वा नि॑र्वरु॒णम् प॒रस्ता᳚त् प॒रस्ता᳚न् निर्वरु॒ण मा॒प्त्वा ऽऽप्त्वा नि॑र्वरु॒णम् प॒रस्ता᳚त् । \newline
65. नि॒र्व॒रु॒णम् प॒रस्ता᳚त् प॒रस्ता᳚न् निर्वरु॒णम् नि॑र्वरु॒णम् प॒रस्ता॑ द॒भ्य॑भि प॒रस्ता᳚न् निर्वरु॒णम् नि॑र्वरु॒णम् प॒रस्ता॑ द॒भि । \newline
66. नि॒र्व॒रु॒णमिति॑ निः - व॒रु॒णम् । \newline
\pagebreak
\markright{ TS 2.2.6.3  \hfill https://www.vedavms.in \hfill}
\addcontentsline{toc}{section}{ TS 2.2.6.3 }
\section*{ TS 2.2.6.3 }

\textbf{TS 2.2.6.3 } \newline
\textbf{Samhita Paata} \newline

प॒रस्ता॑द॒भि द्रु॑ह्यति॒ नैनं॒ ॅवरु॑णो गृह्णात्या॒व्यं॑ ॅवा ए॒ष प्रति॑ गृह्णाति॒ योऽविं॑ प्रतिगृ॒ह्णाति॑ वैश्वान॒रं द्वाद॑शकपालं॒ निर्व॑पे॒दविं॑ प्रति॒गृह्य॑ संॅवथ्स॒रो वा अ॒ग्निर्वै᳚श्वान॒रः सं॑ॅवथ्स॒र-स्व॑दितामे॒व प्रति॑गृह्णाति॒ नाऽऽ*व्यं॑ प्रति॑गृह्णात्या॒त्मनो॒ वा ए॒ष मात्रा॑माप्नोति॒ य उ॑भ॒याद॑त् प्रतिगृ॒ह्णात्यश्वं॑ ॅवा॒ पुरु॑षं ॅवा वैश्वान॒रं द्वाद॑शकपालं॒ निर्व॑पेदुभ॒याद॑त् - [  ] \newline

\textbf{Pada Paata} \newline

प॒रस्ता᳚त् । अ॒भीति॑ । द्रु॒ह्य॒ति॒ । न । ए॒न॒म् । वरु॑णः । गृ॒ह्णा॒ति॒ । आ॒व्य᳚म् । वै । ए॒षः । प्रतीति॑ । गृ॒ह्णा॒ति॒ । यः । अवि᳚म् । प्र॒ति॒गृ॒ह्णातीति॑ प्रति - गृ॒ह्णाति॑ । वै॒श्वा॒न॒रम् । द्वाद॑शकपाल॒मिति॒ द्वाद॑श - क॒पा॒ल॒म् । निरिति॑ । व॒पे॒त् । अवि᳚म् । प्र॒ति॒गृह्येति॑ प्रति - गृह्य॑ । सं॒ॅव॒थ्स॒र इति॑ सं - व॒थ्स॒रः । वै । अ॒ग्निः । वै॒श्वा॒न॒रः । सं॒ॅव॒थ्स॒रस्व॑दिता॒मिति॑ संॅवथ्स॒र - स्व॒दि॒ता॒म् । ए॒व । प्रतीति॑ । गृ॒ह्णा॒ति॒ । न । आ॒व्य᳚म् । प्रतीति॑ । गृ॒ह्णा॒ति॒ । आ॒त्मनः॑ । वै । ए॒षः । मात्रा᳚म् । आ॒प्नो॒ति॒ । यः । उ॒भ॒याद॑त् । प्र॒ति॒गृ॒ह्णातीति॑ प्रति - गृ॒ह्णाति॑ । अश्व᳚म् । वा॒ । पुरु॑षम् । वा॒ । वै॒श्वा॒न॒रम् । द्वाद॑शकपाल॒मिति॒ द्वाद॑श - क॒पा॒ल॒म् । निरिति॑ । व॒पे॒त् । उ॒भ॒याद॑त् ।  \newline


\textbf{Krama Paata} \newline

प॒रस्ता॑द॒भि । अ॒भि द्रु॑ह्यति । द्रु॒ह्य॒ति॒ न । नैन᳚म् । ए॒नं॒ ॅवरु॑णः । वरु॑णो गृह्णाति । गृ॒ह्णा॒त्या॒व्य᳚म् । आ॒व्यं॑ ॅवै । वा ए॒षः । ए॒ष प्रति॑ । प्रति॑ गृह्णाति । गृ॒ह्णा॒ति॒ यः । योऽवि᳚म् । अवि॑म् प्रतिगृ॒ह्णाति॑ । प्र॒ति॒गृ॒ह्णाति॑ वैश्वान॒रम् । प्र॒ति॒गृ॒ह्णातीति॑ प्रति - गृ॒ह्णाति॑ । वै॒श्वा॒न॒रम् द्वाद॑शकपालम् । द्वाद॑शकपाल॒म् निः । द्वाद॑शकपाल॒मिति॒ द्वाद॑श - क॒पा॒ल॒म् । निर् व॑पेत् । व॒पे॒दवि᳚म् । अवि॑म् प्रति॒गृह्य॑ । प्र॒ति॒गृह्य॑ सम्ॅवथ्स॒रः । प्र॒ति॒गृह्येति॑ प्रति - गृह्य॑ । स॒म्ॅव॒थ्स॒रो वै । स॒म्ॅव॒थ्स॒र इति॑ सं - व॒थ्स॒रः । वा अ॒ग्निः । अ॒ग्निर् वै᳚श्वान॒रः । वै॒श्वा॒न॒रः स॑म्ॅवथ्स॒रस्व॑दिताम् । स॒म्ॅव॒थ्स॒रस्व॑दितामे॒व । स॒म्ॅव॒थ्स॒रस्व॑दिता॒मिति॑ सम्ॅवथ्स॒र - स्व॒दि॒ता॒॒म् । ए॒व प्रति॑ । प्रति॑ गृह्णाति । गृ॒ह्णा॒ति॒ न । ना॒व्य᳚म् । आ॒व्य॑म् प्रति॑ । प्रति॑ गृह्णाति । गृ॒ह्णा॒त्या॒त्मनः॑ । आ॒त्मनो॒ वै । वा ए॒षः । ए॒ष मात्रा᳚म् । मात्रा॑माप्नोति । आ॒प्नो॒ति॒ यः । य उ॑भ॒याद॑त् । उ॒भ॒याद॑त्,प्रतिगृ॒ह्णाति॑ । प्र॒ति॒गृ॒ह्णात्यश्व᳚म् । प्र॒ति॒गृ॒ह्णातीति॑ प्रति - गृ॒ह्णाति॑ । अश्वं॑ ॅवा । वा॒ पुरु॑षम् । पुरु॑षं ॅवा । वा॒ वै॒श्वा॒न॒रम् । वै॒श्वा॒न॒रम् द्वाद॑शकपालम् । दाद॑शकपाल॒म् निः । द्वाद॑शकपाल॒मिति॒ द्वाद॑श - क॒पा॒ल॒म् । निर् व॑पेत् । व॒पे॒दु॒भ॒याद॑त् । उ॒भ॒याद॑त् प्रति॒गृह्य॑ \newline

\textbf{Jatai Paata} \newline

1. प॒रस्ता॑ द॒भ्य॑भि प॒रस्ता᳚त् प॒रस्ता॑ द॒भि । \newline
2. अ॒भि द्रु॑ह्यति द्रुह्य त्य॒भ्य॑भि द्रु॑ह्यति । \newline
3. द्रु॒ह्य॒ति॒ न न द्रु॑ह्यति द्रुह्यति॒ न । \newline
4. नैन॑ मेन॒म् न नैन᳚म् । \newline
5. ए॒नं॒ ॅवरु॑णो॒ वरु॑ण एन मेनं॒ ॅवरु॑णः । \newline
6. वरु॑णो गृह्णाति गृह्णाति॒ वरु॑णो॒ वरु॑णो गृह्णाति । \newline
7. गृ॒ह्णा॒ त्या॒व्य॑ मा॒व्य॑म् गृह्णाति गृह्णा त्या॒व्य᳚म् । \newline
8. आ॒व्यं॑ ॅवै वा आ॒व्य॑ मा॒व्यं॑ ॅवै । \newline
9. वा ए॒ष ए॒ष वै वा ए॒षः । \newline
10. ए॒ष प्रति॒ प्रत्ये॒ष ए॒ष प्रति॑ । \newline
11. प्रति॑ गृह्णाति गृह्णाति॒ प्रति॒ प्रति॑ गृह्णाति । \newline
12. गृ॒ह्णा॒ति॒ यो यो गृ॑ह्णाति गृह्णाति॒ यः । \newline
13. यो ऽवि॒ मविं॒ ॅयो यो ऽवि᳚म् । \newline
14. अवि॑म् प्रतिगृ॒ह्णाति॑ प्रतिगृ॒ह्णा त्यवि॒ मवि॑म् प्रतिगृ॒ह्णाति॑ । \newline
15. प्र॒ति॒गृ॒ह्णाति॑ वैश्वान॒रं ॅवै᳚श्वान॒रम् प्र॑तिगृ॒ह्णाति॑ प्रतिगृ॒ह्णाति॑ वैश्वान॒रम् । \newline
16. प्र॒ति॒गृ॒ह्णातीति॑ प्रति - गृ॒ह्णाति॑ । \newline
17. वै॒श्वा॒न॒रम् द्वाद॑शकपाल॒म् द्वाद॑शकपालं ॅवैश्वान॒रं ॅवै᳚श्वान॒रम् द्वाद॑शकपालम् । \newline
18. द्वाद॑शकपाल॒म् निर् णिर् द्वाद॑शकपाल॒म् द्वाद॑शकपाल॒म् निः । \newline
19. द्वाद॑शकपाल॒मिति॒ द्वाद॑श - क॒पा॒ल॒म् । \newline
20. निर् व॑पेद् वपे॒न् निर् णिर् व॑पेत् । \newline
21. व॒पे॒ दवि॒ मविं॑ ॅवपेद् वपे॒ दवि᳚म् । \newline
22. अवि॑म् प्रति॒गृह्य॑ प्रति॒गृह्यावि॒ मवि॑म् प्रति॒गृह्य॑ । \newline
23. प्र॒ति॒गृह्य॑ संॅवथ्स॒रः सं॑ॅवथ्स॒रः प्र॑ति॒गृह्य॑ प्रति॒गृह्य॑ संॅवथ्स॒रः । \newline
24. प्र॒ति॒गृह्येति॑ प्रति - गृह्य॑ । \newline
25. सं॒ॅव॒थ्स॒रो वै वै सं॑ॅवथ्स॒रः सं॑ॅवथ्स॒रो वै । \newline
26. सं॒ॅव॒थ्स॒र इति॑ सं - व॒थ्स॒रः । \newline
27. वा अ॒ग्नि र॒ग्निर् वै वा अ॒ग्निः । \newline
28. अ॒ग्निर् वै᳚श्वान॒रो वै᳚श्वान॒रो᳚ ऽग्नि र॒ग्निर् वै᳚श्वान॒रः । \newline
29. वै॒श्वा॒न॒रः सं॑ॅवथ्स॒रस्व॑दिताꣳ संॅवथ्स॒रस्व॑दितां ॅवैश्वान॒रो वै᳚श्वान॒रः सं॑ॅवथ्स॒रस्व॑दिताम् । \newline
30. सं॒ॅव॒थ्स॒रस्व॑दिता मे॒वैव सं॑ॅवथ्स॒रस्व॑दिताꣳ संॅवथ्स॒रस्व॑दिता मे॒व । \newline
31. सं॒ॅव॒थ्स॒रस्व॑दिता॒मिति॑ संॅवथ्स॒र - स्व॒दि॒ता॒म् । \newline
32. ए॒व प्रति॒ प्रत्ये॒वैव प्रति॑ । \newline
33. प्रति॑ गृह्णाति गृह्णाति॒ प्रति॒ प्रति॑ गृह्णाति । \newline
34. गृ॒ह्णा॒ति॒ न न गृ॑ह्णाति गृह्णाति॒ न । \newline
35. नाव्य॑ मा॒व्य॑म् न नाव्य᳚म् । \newline
36. आ॒व्य॑म् प्रति॒ प्रत्या॒व्य॑ मा॒व्य॑म् प्रति॑ । \newline
37. प्रति॑ गृह्णाति गृह्णाति॒ प्रति॒ प्रति॑ गृह्णाति । \newline
38. गृ॒ह्णा॒ त्या॒त्मन॑ आ॒त्मनो॑ गृह्णाति गृह्णा त्या॒त्मनः॑ । \newline
39. आ॒त्मनो॒ वै वा आ॒त्मन॑ आ॒त्मनो॒ वै । \newline
40. वा ए॒ष ए॒ष वै वा ए॒षः । \newline
41. ए॒ष मात्रा॒म् मात्रा॑ मे॒ष ए॒ष मात्रा᳚म् । \newline
42. मात्रा॑ माप्नो त्याप्नोति॒ मात्रा॒म् मात्रा॑ माप्नोति । \newline
43. आ॒प्नो॒ति॒ यो य आ᳚प्नो त्याप्नोति॒ यः । \newline
44. य उ॑भ॒याद॑ दुभ॒याद॒द् यो य उ॑भ॒याद॑त् । \newline
45. उ॒भ॒याद॑त् प्रतिगृ॒ह्णाति॑ प्रतिगृ॒ह्णा त्यु॑भ॒याद॑ दुभ॒याद॑त् प्रतिगृ॒ह्णाति॑ । \newline
46. प्र॒ति॒गृ॒ह्णा त्यश्व॒ मश्व॑म् प्रतिगृ॒ह्णाति॑ प्रतिगृ॒ह्णा त्यश्व᳚म् । \newline
47. प्र॒ति॒गृ॒ह्णातीति॑ प्रति - गृ॒ह्णाति॑ । \newline
48. अश्वं॑ ॅवा॒ वा ऽश्व॒ मश्वं॑ ॅवा । \newline
49. वा॒ पुरु॑ष॒म् पुरु॑षं ॅवा वा॒ पुरु॑षम् । \newline
50. पुरु॑षं ॅवा वा॒ पुरु॑ष॒म् पुरु॑षं ॅवा । \newline
51. वा॒ वै॒श्वा॒न॒रं ॅवै᳚श्वान॒रं ॅवा॑ वा वैश्वान॒रम् । \newline
52. वै॒श्वा॒न॒रम् द्वाद॑शकपाल॒म् द्वाद॑शकपालं ॅवैश्वान॒रं ॅवै᳚श्वान॒रम् द्वाद॑शकपालम् । \newline
53. द्वाद॑शकपाल॒म् निर् णिर् द्वाद॑शकपाल॒म् द्वाद॑शकपाल॒म् निः । \newline
54. द्वाद॑शकपाल॒मिति॒ द्वाद॑श - क॒पा॒ल॒म् । \newline
55. निर् व॑पेद् वपे॒न् निर् णिर् व॑पेत् । \newline
56. व॒पे॒ दु॒भ॒याद॑ दुभ॒याद॑द् वपेद् वपे दुभ॒याद॑त् । \newline
57. उ॒भ॒याद॑त् प्रति॒गृह्य॑ प्रति॒गृह्यो॑ भ॒या द॑दुभ॒याद॑त् प्रति॒गृह्य॑ । \newline

\textbf{Ghana Paata } \newline

1. प॒रस्ता॑ द॒भ्य॑भि प॒रस्ता᳚त् प॒रस्ता॑ द॒भि द्रु॑ह्यति द्रुह्यत्य॒भि प॒रस्ता᳚त् प॒रस्ता॑ द॒भि द्रु॑ह्यति । \newline
2. अ॒भि द्रु॑ह्यति द्रुह्य त्य॒भ्य॑भि द्रु॑ह्यति॒ न न द्रु॑ह्य त्य॒भ्य॑भि द्रु॑ह्यति॒ न । \newline
3. द्रु॒ह्य॒ति॒ न न द्रु॑ह्यति द्रुह्यति॒ नैन॑ मेन॒म् न द्रु॑ह्यति द्रुह्यति॒ नैन᳚म् । \newline
4. नैन॑ मेन॒म् न नैनं॒ ॅवरु॑णो॒ वरु॑ण एन॒म् न नैनं॒ ॅवरु॑णः । \newline
5. ए॒नं॒ ॅवरु॑णो॒ वरु॑ण एन मेनं॒ ॅवरु॑णो गृह्णाति गृह्णाति॒ वरु॑ण एन मेनं॒ ॅवरु॑णो गृह्णाति । \newline
6. वरु॑णो गृह्णाति गृह्णाति॒ वरु॑णो॒ वरु॑णो गृह्णा त्या॒व्य॑ मा॒व्य॑म् गृह्णाति॒ वरु॑णो॒ वरु॑णो गृह्णा त्या॒व्य᳚म् । \newline
7. गृ॒ह्णा॒ त्या॒व्य॑ मा॒व्य॑म् गृह्णाति गृह्णा त्या॒व्यं॑ ॅवै वा आ॒व्य॑म् गृह्णाति गृह्णा त्या॒व्यं॑ ॅवै । \newline
8. आ॒व्यं॑ ॅवै वा आ॒व्य॑ मा॒व्यं॑ ॅवा ए॒ष ए॒ष वा आ॒व्य॑ मा॒व्यं॑ ॅवा ए॒षः । \newline
9. वा ए॒ष ए॒ष वै वा ए॒ष प्रति॒ प्रत्ये॒ष वै वा ए॒ष प्रति॑ । \newline
10. ए॒ष प्रति॒ प्रत्ये॒ष ए॒ष प्रति॑ गृह्णाति गृह्णाति॒ प्रत्ये॒ष ए॒ष प्रति॑ गृह्णाति । \newline
11. प्रति॑ गृह्णाति गृह्णाति॒ प्रति॒ प्रति॑ गृह्णाति॒ यो यो गृ॑ह्णाति॒ प्रति॒ प्रति॑ गृह्णाति॒ यः । \newline
12. गृ॒ह्णा॒ति॒ यो यो गृ॑ह्णाति गृह्णाति॒ यो ऽवि॒ मविं॒ ॅयो गृ॑ह्णाति गृह्णाति॒ यो ऽवि᳚म् । \newline
13. यो ऽवि॒ मविं॒ ॅयो यो ऽवि॑म् प्रतिगृ॒ह्णाति॑ प्रतिगृ॒ह्णा त्यविं॒ ॅयो यो ऽवि॑म् प्रतिगृ॒ह्णाति॑ । \newline
14. अवि॑म् प्रतिगृ॒ह्णाति॑ प्रतिगृ॒ह्णा त्यवि॒ मवि॑म् प्रतिगृ॒ह्णाति॑ वैश्वान॒रं ॅवै᳚श्वान॒रम् प्र॑तिगृ॒ह्णा त्यवि॒ मवि॑म् प्रतिगृ॒ह्णाति॑ वैश्वान॒रम् । \newline
15. प्र॒ति॒गृ॒ह्णाति॑ वैश्वान॒रं ॅवै᳚श्वान॒रम् प्र॑तिगृ॒ह्णाति॑ प्रतिगृ॒ह्णाति॑ वैश्वान॒रम् द्वाद॑शकपाल॒म् द्वाद॑शकपालं ॅवैश्वान॒रम् प्र॑तिगृ॒ह्णाति॑ प्रतिगृ॒ह्णाति॑ वैश्वान॒रम् द्वाद॑शकपालम् । \newline
16. प्र॒ति॒गृ॒ह्णातीति॑ प्रति - गृ॒ह्णाति॑ । \newline
17. वै॒श्वा॒न॒रम् द्वाद॑शकपाल॒म् द्वाद॑शकपालं ॅवैश्वान॒रं ॅवै᳚श्वान॒रम् द्वाद॑शकपाल॒म् निर् णिर् द्वाद॑शकपालं ॅवैश्वान॒रं ॅवै᳚श्वान॒रम् द्वाद॑शकपाल॒म् निः । \newline
18. द्वाद॑शकपाल॒म् निर् णिर् द्वाद॑शकपाल॒म् द्वाद॑शकपाल॒म् निर् व॑पेद् वपे॒न् निर् द्वाद॑शकपाल॒म् द्वाद॑शकपाल॒म् निर् व॑पेत् । \newline
19. द्वाद॑शकपाल॒मिति॒ द्वाद॑श - क॒पा॒ल॒म् । \newline
20. निर् व॑पेद् वपे॒न् निर् णिर् व॑पे॒ दवि॒ मविं॑ ॅवपे॒न् निर् णिर् व॑पे॒ दवि᳚म् । \newline
21. व॒पे॒ दवि॒ मविं॑ ॅवपेद् वपे॒ दवि॑म् प्रति॒गृह्य॑ प्रति॒गृह्याविं॑ ॅवपेद् वपे॒ दवि॑म् प्रति॒गृह्य॑ । \newline
22. अवि॑म् प्रति॒गृह्य॑ प्रति॒गृह्यावि॒ मवि॑म् प्रति॒गृह्य॑ संॅवथ्स॒रः सं॑ॅवथ्स॒रः प्र॑ति॒गृह्यावि॒ मवि॑म् प्रति॒गृह्य॑ संॅवथ्स॒रः । \newline
23. प्र॒ति॒गृह्य॑ संॅवथ्स॒रः सं॑ॅवथ्स॒रः प्र॑ति॒गृह्य॑ प्रति॒गृह्य॑ संॅवथ्स॒रो वै वै सं॑ॅवथ्स॒रः प्र॑ति॒गृह्य॑ प्रति॒गृह्य॑ संॅवथ्स॒रो वै । \newline
24. प्र॒ति॒गृह्येति॑ प्रति - गृह्य॑ । \newline
25. सं॒ॅव॒थ्स॒रो वै वै सं॑ॅवथ्स॒रः सं॑ॅवथ्स॒रो वा अ॒ग्नि र॒ग्निर् वै सं॑ॅवथ्स॒रः सं॑ॅवथ्स॒रो वा अ॒ग्निः । \newline
26. सं॒ॅव॒थ्स॒र इति॑ सं - व॒थ्स॒रः । \newline
27. वा अ॒ग्नि र॒ग्निर् वै वा अ॒ग्निर् वै᳚श्वान॒रो वै᳚श्वान॒रो᳚ ऽग्निर् वै वा अ॒ग्निर् वै᳚श्वान॒रः । \newline
28. अ॒ग्निर् वै᳚श्वान॒रो वै᳚श्वान॒रो᳚ ऽग्नि र॒ग्निर् वै᳚श्वान॒रः सं॑ॅवथ्स॒रस्व॑दिताꣳ संॅवथ्स॒रस्व॑दितां ॅवैश्वान॒रो᳚ ऽग्निर॒ग्निर् वै᳚श्वान॒रः सं॑ॅवथ्स॒रस्व॑दिताम् । \newline
29. वै॒श्वा॒न॒रः सं॑ॅवथ्स॒रस्व॑दिताꣳ संॅवथ्स॒रस्व॑दितां ॅवैश्वान॒रो वै᳚श्वान॒रः सं॑ॅवथ्स॒रस्व॑दिता मे॒वैव सं॑ॅवथ्स॒रस्व॑दितां ॅवैश्वान॒रो वै᳚श्वान॒रः सं॑ॅवथ्स॒रस्व॑दिता मे॒व । \newline
30. सं॒ॅव॒थ्स॒रस्व॑दिता मे॒वैव सं॑ॅवथ्स॒रस्व॑दिताꣳ संॅवथ्स॒रस्व॑दिता मे॒व प्रति॒ प्रत्ये॒व सं॑ॅवथ्स॒रस्व॑दिताꣳ संॅवथ्स॒रस्व॑दिता मे॒व प्रति॑ । \newline
31. सं॒ॅव॒थ्स॒रस्व॑दिता॒मिति॑ संॅवथ्स॒र - स्व॒दि॒ता॒म् । \newline
32. ए॒व प्रति॒ प्रत्ये॒वैव प्रति॑ गृह्णाति गृह्णाति॒ प्र त्ये॒वैव प्रति॑ गृह्णाति । \newline
33. प्रति॑ गृह्णाति गृह्णाति॒ प्रति॒ प्रति॑ गृह्णाति॒ न न गृ॑ह्णाति॒ प्रति॒ प्रति॑ गृह्णाति॒ न । \newline
34. गृ॒ह्णा॒ति॒ न न गृ॑ह्णाति गृह्णाति॒ नाव्य॑ मा॒व्य॑म् न गृ॑ह्णाति गृह्णाति॒ नाव्य᳚म् । \newline
35. नाव्य॑ मा॒व्य॑म् न नाव्य॑म् प्रति॒ प्रत्या॒व्य॑म् न नाव्य॑म् प्रति॑ । \newline
36. आ॒व्य॑म् प्रति॒ प्रत्या॒व्य॑ मा॒व्य॑म् प्रति॑ गृह्णाति गृह्णाति॒ प्रत्या॒व्य॑ मा॒व्य॑म् प्रति॑ गृह्णाति । \newline
37. प्रति॑ गृह्णाति गृह्णाति॒ प्रति॒ प्रति॑ गृह्णात्या॒त्मन॑ आ॒त्मनो॑ गृह्णाति॒ प्रति॒ प्रति॑ गृह्णात्या॒त्मनः॑ । \newline
38. गृ॒ह्णा॒ त्या॒त्मन॑ आ॒त्मनो॑ गृह्णाति गृह्णा त्या॒त्मनो॒ वै वा आ॒त्मनो॑ गृह्णाति गृह्णा त्या॒त्मनो॒ वै । \newline
39. आ॒त्मनो॒ वै वा आ॒त्मन॑ आ॒त्मनो॒ वा ए॒ष ए॒ष वा आ॒त्मन॑ आ॒त्मनो॒ वा ए॒षः । \newline
40. वा ए॒ष ए॒ष वै वा ए॒ष मात्रा॒म् मात्रा॑ मे॒ष वै वा ए॒ष मात्रा᳚म् । \newline
41. ए॒ष मात्रा॒म् मात्रा॑ मे॒ष ए॒ष मात्रा॑ माप्नो त्याप्नोति॒ मात्रा॑ मे॒ष ए॒ष मात्रा॑ माप्नोति । \newline
42. मात्रा॑ माप्नो त्याप्नोति॒ मात्रा॒म् मात्रा॑ माप्नोति॒ यो य आ᳚प्नोति॒ मात्रा॒म् मात्रा॑ माप्नोति॒ यः । \newline
43. आ॒प्नो॒ति॒ यो य आ᳚प्नो त्याप्नोति॒ य उ॑भ॒या द॑दु भ॒याद॒द् य आ᳚प्नो त्याप्नोति॒ य उ॑भ॒याद॑त् । \newline
44. य उ॑भ॒याद॑ दुभ॒या द॒द् यो य उ॑भ॒याद॑त् प्रतिगृ॒ह्णाति॑ प्रतिगृ॒ह्णा त्यु॑भ॒याद॒द् यो य उ॑भ॒याद॑त् प्रतिगृ॒ह्णाति॑ । \newline
45. उ॒भ॒याद॑त् प्रतिगृ॒ह्णाति॑ प्रतिगृ॒ह्णा त्यु॑भ॒या द॑दुभ॒याद॑त् प्रतिगृ॒ह्णा त्यश्व॒ मश्व॑म् प्रतिगृ॒ह्णा त्यु॑भ॒या द॑दुभ॒याद॑त् प्रतिगृ॒ह्णा त्यश्व᳚म् । \newline
46. प्र॒ति॒गृ॒ह्णात्यश्व॒ मश्व॑म् प्रतिगृ॒ह्णाति॑ प्रतिगृ॒ह्णा त्यश्वं॑ ॅवा॒ वा ऽश्व॑म् प्रतिगृ॒ह्णाति॑ प्रतिगृ॒ह्णा त्यश्वं॑ ॅवा । \newline
47. प्र॒ति॒गृ॒ह्णातीति॑ प्रति - गृ॒ह्णाति॑ । \newline
48. अश्वं॑ ॅवा॒ वा ऽश्व॒ मश्वं॑ ॅवा॒ पुरु॑ष॒म् पुरु॑षं॒ ॅवा ऽश्व॒ मश्वं॑ ॅवा॒ पुरु॑षम् । \newline
49. वा॒ पुरु॑ष॒म् पुरु॑षं ॅवा वा॒ पुरु॑षं ॅवा वा॒ पुरु॑षं ॅवा वा॒ पुरु॑षं ॅवा । \newline
50. पुरु॑षं ॅवा वा॒ पुरु॑ष॒म् पुरु॑षं ॅवा वैश्वान॒रं ॅवै᳚श्वान॒रं ॅवा॒ पुरु॑ष॒म् 
पुरु॑षं ॅवा वैश्वान॒रम् । \newline
51. वा॒ वै॒श्वा॒न॒रं ॅवै᳚श्वान॒रं ॅवा॑ वा वैश्वान॒रम् द्वाद॑शकपाल॒म् द्वाद॑शकपालं ॅवैश्वान॒रं ॅवा॑ वा वैश्वान॒रम् द्वाद॑शकपालम् । \newline
52. वै॒श्वा॒न॒रम् द्वाद॑शकपाल॒म् द्वाद॑शकपालं ॅवैश्वान॒रं ॅवै᳚श्वान॒रम् द्वाद॑शकपाल॒म् निर् णिर् द्वाद॑शकपालं ॅवैश्वान॒रं ॅवै᳚श्वान॒रम् द्वाद॑शकपाल॒म् निः । \newline
53. द्वाद॑शकपाल॒म् निर् णिर् द्वाद॑शकपाल॒म् द्वाद॑शकपाल॒म् निर् व॑पेद् वपे॒न् निर् द्वाद॑शकपाल॒म् द्वाद॑शकपाल॒म् निर् व॑पेत् । \newline
54. द्वाद॑शकपाल॒मिति॒ द्वाद॑श - क॒पा॒ल॒म् । \newline
55. निर् व॑पेद् वपे॒न् निर् णिर् व॑पे दुभ॒या द॑दुभ॒याद॑द् वपे॒न् निर् णिर् व॑पे दुभ॒याद॑त् । \newline
56. व॒पे॒दु॒भ॒या द॑दुभ॒याद॑द् वपेद् वपे दुभ॒याद॑त् प्रति॒गृह्य॑ प्रति॒गृह्यो॑ भ॒याद॑द् वपेद् वपे दुभ॒याद॑त् प्रति॒गृह्य॑ । \newline
57. उ॒भ॒याद॑त् प्रति॒गृह्य॑ प्रति॒गृह्यो॑ भ॒याद॑ दुभ॒याद॑त् प्रति॒गृह्य॑ संॅवथ्स॒रः सं॑ॅवथ्स॒रः प्र॑ति॒गृह्यो॑ भ॒या द॑दुभ॒याद॑त् प्रति॒गृह्य॑ संॅवथ्स॒रः । \newline
\pagebreak
\markright{ TS 2.2.6.4  \hfill https://www.vedavms.in \hfill}
\addcontentsline{toc}{section}{ TS 2.2.6.4 }
\section*{ TS 2.2.6.4 }

\textbf{TS 2.2.6.4 } \newline
\textbf{Samhita Paata} \newline

प्रति॒गृह्य॑ संवथ्स॒रो वा अ॒ग्निर्वै᳚श्वान॒रः सं॑ॅवथ्स॒र-स्व॑दितमे॒व प्रति॑ गृह्णाति॒ नात्मनो॒ मात्रा॑माप्नोति वैश्वान॒रं द्वाद॑शकपालं॒ निर्व॑पेथ्-स॒निमे॒ष्यन्थ्-सं॑ॅवथ्स॒रो वा अ॒ग्निर्वै᳚श्वान॒रो य॒दा खलु॒ वै सं॑ॅवथ्स॒रं ज॒नता॑यां॒ चर॒त्यथ॒ स ध॑ना॒र्घो भ॑वति॒यद्-वै᳚श्वान॒रं द्वाद॑शकपालं नि॒र्वप॑ति संॅवथ्स॒र-सा॑तामे॒व स॒निम॒भि प्रच्य॑वते॒ दान॑कामा अस्मै प्र॒जा भ॑वन्ति॒ यो वै सं॑ॅवथ्स॒रं - [  ] \newline

\textbf{Pada Paata} \newline

प्र॒ति॒गृह्येति॑ प्रति - गृह्य॑ । सं॒ॅव॒थ्स॒र इति॑ सं - व॒थ्स॒रः । वै । अ॒ग्निः । वै॒श्वा॒न॒रः । सं॒ॅव॒थ्स॒रस्व॑दित॒मिति॑ संॅवथ्स॒र - स्व॒दि॒त॒म् । ए॒व । प्रतीति॑ । गृ॒ह्णा॒ति॒ । न । आ॒त्मनः॑ । मात्रा᳚म् । आ॒प्नो॒ति॒ । वै॒श्वा॒न॒रम् । द्वाद॑शकपाल॒मिति॒ द्वाद॑श - क॒पा॒ल॒म् । निरिति॑ । व॒पे॒त् । स॒निम् । ए॒ष्यन्न् । सं॒ॅव॒थ्स॒र इति॑ सं - व॒थ्स॒रः । वै । अ॒ग्निः । वै॒श्वा॒न॒रः । य॒दा । खलु॑ । वै । सं॒ॅव॒थ्स॒रमिति॑ सं - व॒थ्स॒रम् । ज॒नता॑याम् । चर॑ति । अथ॑ । सः । ध॒ना॒र्घ इति॑ धन - अ॒र्घः । भ॒व॒ति॒॒ । यत् । वै॒श्वा॒न॒रम् । द्वाद॑शकपाल॒मिति॒ द्वाद॑श - क॒पा॒ल॒म् । नि॒र्वप॒तीति॑ निः - वप॑ति । सं॒ॅव॒थ्स॒रसा॑ता॒मिति॑ संॅवथ्स॒र-सा॒ता॒म् । ए॒व । स॒निम् । अ॒भि । प्रेति॑ । च्य॒व॒ते॒ । दान॑कामा॒ इति॒ दान॑ - का॒माः॒ । अ॒स्मै॒ । प्र॒जा इति॑ प्र - जाः । भ॒व॒न्ति॒ । यः । वै । सं॒ॅव॒थ्स॒रमिति॑ सं - व॒थ्स॒रम् ।  \newline


\textbf{Krama Paata} \newline

प्र॒ति॒गृह्य॑ सम्ॅवथ्स॒रः । प्र॒ति॒गृह्येति॑ प्रति - गृह्य॑ । स॒म्ॅव॒थ्स॒रो वै । स॒म्ॅव॒थ्स॒र इति॑ सं - व॒थ्स॒रः । वा अ॒ग्निः । अ॒ग्निर् वै᳚श्वान॒रः । वै॒श्वा॒न॒रः स॑म्ॅवथ्स॒रस्व॑दितम् । स॒म्ॅव॒थ्स॒रस्व॑दितमे॒व । स॒म्ॅव॒थ्स॒रस्व॑दित॒मिति॑ सम्ॅवथ्स॒र - स्व॒दि॒त॒म् । ए॒व प्रति॑ । प्रति॑ गृह्णाति । गृ॒ह्णा॒ति॒ न । नात्मनः॑ । आ॒त्मनो॒ मात्रा᳚म् । मात्रा॑माप्नोति । आ॒प्नो॒ति॒ वै॒श्वा॒न॒रम् । वै॒श्वा॒न॒रम् द्वाद॑शकपालम् । द्वाद॑शकपाल॒म् निः । द्वाद॑शकपाल॒मिति॒ द्वाद॑श - क॒पा॒ल॒म् । निर् व॑पेत् । व॒पे॒थ् स॒निम् । स॒निमे॒ष्यन्न् । ए॒ष्यन्थ् स॑म्ॅवथ्स॒रः । स॒म्ॅव॒थ्स॒रो वै । स॒म्ॅव॒थ्स॒र इति॑ सं - व॒थ्स॒रः । वा अ॒ग्निः । अ॒ग्निर् वै᳚श्वान॒रः । वै॒श्वा॒न॒रो य॒दा । य॒दा खलु॑ । खलु॒ वै । वै स॑म्ॅवथ्स॒रम् । स॒म्ॅव॒थ्स॒रम् ज॒नता॑याम् । स॒म्ॅव॒थ्स॒रमिति॑ सं - व॒थ्स॒रम् । ज॒नता॑या॒म् चर॑ति । चर॒त्यथ॑ । अथ॒ सः । स ध॑ना॒र्घः । ध॒ना॒र्घो भ॑वति । ध॒ना॒र्घ इति॑ धन - अ॒र्घः । भ॒व॒ति॒ यत् । यद् वै᳚शान॒रम् । वै॒श्वा॒न॒रम् द्वाद॑शकपालम् । द्वाद॑शकपालम् नि॒र्वप॑ति । द्वाद॑शकपाल॒मिति॒ द्वाद॑श - क॒पा॒ल॒म् । नि॒र्वप॑ति सम्ॅवथ्स॒रसा॑ताम् । नि॒र्वप॒तीति॑ निः - वप॑ति । स॒म्ॅव॒थ्स॒रसा॑तामे॒व । स॒म्ॅव॒थ्स॒रसा॑ता॒मिति॑ सम्ॅवथ्स॒र - सा॒ता॒म् । ए॒व स॒निम् । स॒निम॒भि । अ॒भि प्र । प्र च्य॑वते । च्य॒व॒ते॒ दान॑कामाः । दान॑कामा अस्मै । दान॑कामा॒ इति॒ दान॑ - का॒माः॒ । अ॒स्मै॒ प्र॒जाः । प्र॒जा भ॑वन्ति । प्र॒जा इति॑ प्र - जाः । भ॒व॒न्ति॒ यः । यो वै । वै स॑म्ॅवथ्स॒रम् ( ) । स॒म्ॅव॒थ्स॒रम् प्र॒युज्य॑ । स॒म्ॅव॒थ्स॒रमिति॑ सं - व॒थ्स॒रम् \newline

\textbf{Jatai Paata} \newline

1. प्र॒ति॒गृह्य॑ संॅवथ्स॒रः सं॑ॅवथ्स॒रः प्र॑ति॒गृह्य॑ प्रति॒गृह्य॑ संॅवथ्स॒रः । \newline
2. प्र॒ति॒गृह्येति॑ प्रति - गृह्य॑ । \newline
3. सं॒ॅव॒थ्स॒रो वै वै सं॑ॅवथ्स॒रः सं॑ॅवथ्स॒रो वै । \newline
4. सं॒ॅव॒थ्स॒र इति॑ सं - व॒थ्स॒रः । \newline
5. वा अ॒ग्नि र॒ग्निर् वै वा अ॒ग्निः । \newline
6. अ॒ग्निर् वै᳚श्वान॒रो वै᳚श्वान॒रो᳚ ऽग्नि र॒ग्निर् वै᳚श्वान॒रः । \newline
7. वै॒श्वा॒न॒रः सं॑ॅवथ्स॒रस्व॑दितꣳ संॅवथ्स॒रस्व॑दितं ॅवैश्वान॒रो वै᳚श्वान॒रः सं॑ॅवथ्स॒रस्व॑दितम् । \newline
8. सं॒ॅव॒थ्स॒रस्व॑दित मे॒वैव सं॑ॅवथ्स॒रस्व॑दितꣳ संॅवथ्स॒रस्व॑दित मे॒व । \newline
9. सं॒ॅव॒थ्स॒रस्व॑दित॒मिति॑ संॅवथ्स॒र - स्व॒दि॒त॒म् । \newline
10. ए॒व प्रति॒ प्रत्ये॒वैव प्रति॑ । \newline
11. प्रति॑ गृह्णाति गृह्णाति॒ प्रति॒ प्रति॑ गृह्णाति । \newline
12. गृ॒ह्णा॒ति॒ न न गृ॑ह्णाति गृह्णाति॒ न । \newline
13. नात्मन॑ आ॒त्मनो॒ न नात्मनः॑ । \newline
14. आ॒त्मनो॒ मात्रा॒म् मात्रा॑ मा॒त्मन॑ आ॒त्मनो॒ मात्रा᳚म् । \newline
15. मात्रा॑ माप्नो त्याप्नोति॒ मात्रा॒म् मात्रा॑ माप्नोति । \newline
16. आ॒प्नो॒ति॒ वै॒श्वा॒न॒रं ॅवै᳚श्वान॒र मा᳚प्नो त्याप्नोति वैश्वान॒रम् । \newline
17. वै॒श्वा॒न॒रम् द्वाद॑शकपाल॒म् द्वाद॑शकपालं ॅवैश्वान॒रं ॅवै᳚श्वान॒रम् द्वाद॑शकपालम् । \newline
18. द्वाद॑शकपाल॒म् निर् णिर् द्वाद॑शकपाल॒म् द्वाद॑शकपाल॒म् निः । \newline
19. द्वाद॑शकपाल॒मिति॒ द्वाद॑श - क॒पा॒ल॒म् । \newline
20. निर् व॑पेद् वपे॒न् निर् णिर् व॑पेत् । \newline
21. व॒पे॒थ् स॒निꣳ स॒निं ॅव॑पेद् वपेथ् स॒निम् । \newline
22. स॒नि मे॒ष्यन् ने॒ष्यन् थ्स॒निꣳ स॒नि मे॒ष्यन्न् । \newline
23. ए॒ष्यन् थ्सं॑ॅवथ्स॒रः सं॑ॅवथ्स॒र ए॒ष्यन् ने॒ष्यन् थ्सं॑ॅवथ्स॒रः । \newline
24. सं॒ॅव॒थ्स॒रो वै वै सं॑ॅवथ्स॒रः सं॑ॅवथ्स॒रो वै । \newline
25. सं॒ॅव॒थ्स॒र इति॑ सं - व॒थ्स॒रः । \newline
26. वा अ॒ग्नि र॒ग्निर् वै वा अ॒ग्निः । \newline
27. अ॒ग्निर् वै᳚श्वान॒रो वै᳚श्वान॒रो᳚ ऽग्निर॒ग्निर् वै᳚श्वान॒रः । \newline
28. वै॒श्वा॒न॒रो य॒दा य॒दा वै᳚श्वान॒रो वै᳚श्वान॒रो य॒दा । \newline
29. य॒दा खलु॒ खलु॑ य॒दा य॒दा खलु॑ । \newline
30. खलु॒ वै वै खलु॒ खलु॒ वै । \newline
31. वै सं॑ॅवथ्स॒रꣳ सं॑ॅवथ्स॒रं ॅवै वै सं॑ॅवथ्स॒रम् । \newline
32. सं॒ॅव॒थ्स॒रम् ज॒नता॑याम् ज॒नता॑याꣳ संॅवथ्स॒रꣳ सं॑ॅवथ्स॒रम् ज॒नता॑याम् । \newline
33. सं॒ॅव॒थ्स॒रमिति॑ सं - व॒थ्स॒रम् । \newline
34. ज॒नता॑या॒म् चर॑ति॒ चर॑ति ज॒नता॑याम् ज॒नता॑या॒म् चर॑ति । \newline
35. चर॒ त्यथाथ॒ चर॑ति॒ चर॒ त्यथ॑ । \newline
36. अथ॒ स सो ऽथाथ॒ सः । \newline
37. स ध॑ना॒र्घो ध॑ना॒र्घः स स ध॑ना॒र्घः । \newline
38. ध॒ना॒र्घो भ॑वति भवति धना॒र्घो ध॑ना॒र्घो भ॑वति । \newline
39. ध॒ना॒र्घ इति॑ धन - अ॒र्घः । \newline
40. भ॒व॒ति॒ यद् यद् भ॑वति भवति॒ यत् । \newline
41. यद् वै᳚श्वान॒रं ॅवै᳚श्वान॒रं ॅयद् यद् वै᳚श्वान॒रम् । \newline
42. वै॒श्वा॒न॒रम् द्वाद॑शकपाल॒म् द्वाद॑शकपालं ॅवैश्वान॒रं ॅवै᳚श्वान॒रम् द्वाद॑शकपालम् । \newline
43. द्वाद॑शकपालम् नि॒र्वप॑ति नि॒र्वप॑ति॒ द्वाद॑शकपाल॒म् द्वाद॑शकपालम् नि॒र्वप॑ति । \newline
44. द्वाद॑शकपाल॒मिति॒ द्वाद॑श - क॒पा॒ल॒म् । \newline
45. नि॒र्वप॑ति संॅवथ्स॒रसा॑ताꣳ संॅवथ्स॒रसा॑ताम् नि॒र्वप॑ति नि॒र्वप॑ति संॅवथ्स॒रसा॑ताम् । \newline
46. नि॒र्वप॒तीति॑ निः - वप॑ति । \newline
47. सं॒ॅव॒थ्स॒रसा॑ता मे॒वैव सं॑ॅवथ्स॒रसा॑ताꣳ संॅवथ्स॒रसा॑ता मे॒व । \newline
48. सं॒ॅव॒थ्स॒रसा॑ता॒मिति॑ संॅवथ्स॒र - सा॒ता॒म् । \newline
49. ए॒व स॒निꣳ स॒नि मे॒वैव स॒निम् । \newline
50. स॒नि म॒भ्य॑भि स॒निꣳ स॒नि म॒भि । \newline
51. अ॒भि प्र प्राभ्य॑भि प्र । \newline
52. प्र च्य॑वते च्यवते॒ प्र प्र च्य॑वते । \newline
53. च्य॒व॒ते॒ दान॑कामा॒ दान॑कामा श्च्यवते च्यवते॒ दान॑कामाः । \newline
54. दान॑कामा अस्मा अस्मै॒ दान॑कामा॒ दान॑कामा अस्मै । \newline
55. दान॑कामा॒ इति॒ दान॑ - का॒माः॒ । \newline
56. अ॒स्मै॒ प्र॒जाः प्र॒जा अ॑स्मा अस्मै प्र॒जाः । \newline
57. प्र॒जा भ॑वन्ति भवन्ति प्र॒जाः प्र॒जा भ॑वन्ति । \newline
58. प्र॒जा इति॑ प्र - जाः । \newline
59. भ॒व॒न्ति॒ यो यो भ॑वन्ति भवन्ति॒ यः । \newline
60. यो वै वै यो यो वै । \newline
61. वै सं॑ॅवथ्स॒रꣳ सं॑ॅवथ्स॒रं ॅवै वै सं॑ॅवथ्स॒रम् । \newline
62. सं॒ॅव॒थ्स॒रम् प्र॒युज्य॑ प्र॒युज्य॑ संॅवथ्स॒रꣳ सं॑ॅवथ्स॒रम् प्र॒युज्य॑ । \newline
63. सं॒ॅव॒थ्स॒रमिति॑ सं - व॒थ्स॒रम् । \newline

\textbf{Ghana Paata } \newline

1. प्र॒ति॒गृह्य॑ संॅवथ्स॒रः सं॑ॅवथ्स॒रः प्र॑ति॒गृह्य॑ प्रति॒गृह्य॑ संॅवथ्स॒रो वै वै सं॑ॅवथ्स॒रः प्र॑ति॒गृह्य॑ प्रति॒गृह्य॑ संॅवथ्स॒रो वै । \newline
2. प्र॒ति॒गृह्येति॑ प्रति - गृह्य॑ । \newline
3. सं॒ॅव॒थ्स॒रो वै वै सं॑ॅवथ्स॒रः सं॑ॅवथ्स॒रो वा अ॒ग्नि र॒ग्निर् वै सं॑ॅवथ्स॒रः सं॑ॅवथ्स॒रो वा अ॒ग्निः । \newline
4. सं॒ॅव॒थ्स॒र इति॑ सं - व॒थ्स॒रः । \newline
5. वा अ॒ग्नि र॒ग्निर् वै वा अ॒ग्निर् वै᳚श्वान॒रो वै᳚श्वान॒रो᳚ ऽग्निर् वै वा अ॒ग्निर् वै᳚श्वान॒रः । \newline
6. अ॒ग्निर् वै᳚श्वान॒रो वै᳚श्वान॒रो᳚ ऽग्नि र॒ग्निर् वै᳚श्वान॒रः सं॑ॅवथ्स॒रस्व॑दितꣳ संॅवथ्स॒रस्व॑दितं ॅवैश्वान॒रो᳚ ऽग्नि र॒ग्निर् वै᳚श्वान॒रः सं॑ॅवथ्स॒रस्व॑दितम् । \newline
7. वै॒श्वा॒न॒रः सं॑ॅवथ्स॒रस्व॑दितꣳ संॅवथ्स॒रस्व॑दितं ॅवैश्वान॒रो वै᳚श्वान॒रः सं॑ॅवथ्स॒रस्व॑दित मे॒वैव सं॑ॅवथ्स॒रस्व॑दितं ॅवैश्वान॒रो वै᳚श्वान॒रः सं॑ॅवथ्स॒रस्व॑दित मे॒व । \newline
8. सं॒ॅव॒थ्स॒रस्व॑दित मे॒वैव सं॑ॅवथ्स॒रस्व॑दितꣳ संॅवथ्स॒रस्व॑दित मे॒व प्रति॒ प्रत्ये॒व सं॑ॅवथ्स॒रस्व॑दितꣳ संॅवथ्स॒रस्व॑दित मे॒व प्रति॑ । \newline
9. सं॒ॅव॒थ्स॒रस्व॑दित॒मिति॑ संॅवथ्स॒र - स्व॒दि॒त॒म् । \newline
10. ए॒व प्रति॒ प्रत्ये॒वैव प्रति॑ गृह्णाति गृह्णाति॒ प्रत्ये॒वैव प्रति॑ गृह्णाति । \newline
11. प्रति॑ गृह्णाति गृह्णाति॒ प्रति॒ प्रति॑ गृह्णाति॒ न न गृ॑ह्णाति॒ प्रति॒ प्रति॑ गृह्णाति॒ न । \newline
12. गृ॒ह्णा॒ति॒ न न गृ॑ह्णाति गृह्णाति॒ नात्मन॑ आ॒त्मनो॒ न गृ॑ह्णाति गृह्णाति॒ नात्मनः॑ । \newline
13. नात्मन॑ आ॒त्मनो॒ न नात्मनो॒ मात्रा॒म् मात्रा॑ मा॒त्मनो॒ न नात्मनो॒ मात्रा᳚म् । \newline
14. आ॒त्मनो॒ मात्रा॒म् मात्रा॑ मा॒त्मन॑ आ॒त्मनो॒ मात्रा॑ माप्नो त्याप्नोति॒ मात्रा॑ मा॒त्मन॑ आ॒त्मनो॒ मात्रा॑ माप्नोति । \newline
15. मात्रा॑ माप्नो त्याप्नोति॒ मात्रा॒म् मात्रा॑ माप्नोति वैश्वान॒रं ॅवै᳚श्वान॒र मा᳚प्नोति॒ मात्रा॒म् मात्रा॑ माप्नोति वैश्वान॒रम् । \newline
16. आ॒प्नो॒ति॒ वै॒श्वा॒न॒रं ॅवै᳚श्वान॒र मा᳚प्नो त्याप्नोति वैश्वान॒रम् द्वाद॑शकपाल॒म् द्वाद॑शकपालं ॅवैश्वान॒र मा᳚प्नो त्याप्नोति वैश्वान॒रम् द्वाद॑शकपालम् । \newline
17. वै॒श्वा॒न॒रम् द्वाद॑शकपाल॒म् द्वाद॑शकपालं ॅवैश्वान॒रं ॅवै᳚श्वान॒रम् द्वाद॑शकपाल॒म् निर् णिर् द्वाद॑शकपालं ॅवैश्वान॒रं ॅवै᳚श्वान॒रम् द्वाद॑शकपाल॒म् निः । \newline
18. द्वाद॑शकपाल॒म् निर् णिर् द्वाद॑शकपाल॒म् द्वाद॑शकपाल॒म् निर् व॑पेद् वपे॒न् निर् द्वाद॑शकपाल॒म् द्वाद॑शकपाल॒म् निर् व॑पेत् । \newline
19. द्वाद॑शकपाल॒मिति॒ द्वाद॑श - क॒पा॒ल॒म् । \newline
20. निर् व॑पेद् वपे॒न् निर् णिर् व॑पेथ् स॒निꣳ स॒निं ॅव॑पे॒न् निर् णिर् व॑पेथ् स॒निम् । \newline
21. व॒पे॒थ् स॒निꣳ स॒निं ॅव॑पेद् वपेथ् स॒नि मे॒ष्यन् ने॒ष्यन् थ्स॒निं ॅव॑पेद् वपेथ् स॒नि मे॒ष्यन्न् । \newline
22. स॒नि मे॒ष्यन् ने॒ष्यन् थ्स॒निꣳ स॒नि मे॒ष्यन् थ्सं॑ॅवथ्स॒रः सं॑ॅवथ्स॒र ए॒ष्यन् थ्स॒निꣳ स॒नि मे॒ष्यन् थ्सं॑ॅवथ्स॒रः । \newline
23. ए॒ष्यन् थ्सं॑ॅवथ्स॒रः सं॑ॅवथ्स॒र ए॒ष्यन् ने॒ष्यन् थ्सं॑ॅवथ्स॒रो वै वै सं॑ॅवथ्स॒र ए॒ष्यन् ने॒ष्यन् थ्सं॑ॅवथ्स॒रो वै । \newline
24. सं॒ॅव॒थ्स॒रो वै वै सं॑ॅवथ्स॒रः सं॑ॅवथ्स॒रो वा अ॒ग्नि र॒ग्निर् वै सं॑ॅवथ्स॒रः सं॑ॅवथ्स॒रो वा अ॒ग्निः । \newline
25. सं॒ॅव॒थ्स॒र इति॑ सं - व॒थ्स॒रः । \newline
26. वा अ॒ग्नि र॒ग्निर् वै वा अ॒ग्निर् वै᳚श्वान॒रो वै᳚श्वान॒रो᳚ ऽग्निर् वै वा अ॒ग्निर् वै᳚श्वान॒रः । \newline
27. अ॒ग्निर् वै᳚श्वान॒रो वै᳚श्वान॒रो᳚ ऽग्नि र॒ग्निर् वै᳚श्वान॒रो य॒दा य॒दा वै᳚श्वान॒रो᳚ ऽग्नि र॒ग्निर् वै᳚श्वान॒रो य॒दा । \newline
28. वै॒श्वा॒न॒रो य॒दा य॒दा वै᳚श्वान॒रो वै᳚श्वान॒रो य॒दा खलु॒ खलु॑ य॒दा वै᳚श्वान॒रो वै᳚श्वान॒रो य॒दा खलु॑ । \newline
29. य॒दा खलु॒ खलु॑ य॒दा य॒दा खलु॒ वै वै खलु॑ य॒दा य॒दा खलु॒ वै । \newline
30. खलु॒ वै वै खलु॒ खलु॒ वै सं॑ॅवथ्स॒रꣳ सं॑ॅवथ्स॒रं ॅवै खलु॒ खलु॒ वै सं॑ॅवथ्स॒रम् । \newline
31. वै सं॑ॅवथ्स॒रꣳ सं॑ॅवथ्स॒रं ॅवै वै सं॑ॅवथ्स॒रम् ज॒नता॑याम् ज॒नता॑याꣳ संॅवथ्स॒रं ॅवै वै सं॑ॅवथ्स॒रम् ज॒नता॑याम् । \newline
32. सं॒ॅव॒थ्स॒रम् ज॒नता॑याम् ज॒नता॑याꣳ संॅवथ्स॒रꣳ सं॑ॅवथ्स॒रम् ज॒नता॑या॒म् चर॑ति॒ चर॑ति ज॒नता॑याꣳ संॅवथ्स॒रꣳ सं॑ॅवथ्स॒रम् ज॒नता॑या॒म् चर॑ति । \newline
33. सं॒ॅव॒थ्स॒रमिति॑ सं - व॒थ्स॒रम् । \newline
34. ज॒नता॑या॒म् चर॑ति॒ चर॑ति ज॒नता॑याम् ज॒नता॑या॒म् चर॒ त्यथाथ॒ चर॑ति ज॒नता॑याम् ज॒नता॑या॒म् चर॒ त्यथ॑ । \newline
35. चर॒ त्यथाथ॒ चर॑ति॒ चर॒ त्यथ॒ स सो ऽथ॒ चर॑ति॒ चर॒ त्यथ॒ सः । \newline
36. अथ॒ स सो ऽथाथ॒ स ध॑ना॒र्घो ध॑ना॒र्घः सो ऽथाथ॒ स ध॑ना॒र्घः । \newline
37. स ध॑ना॒र्घो ध॑ना॒र्घः स स ध॑ना॒र्घो भ॑वति भवति धना॒र्घः स स ध॑ना॒र्घो भ॑वति । \newline
38. ध॒ना॒र्घो भ॑वति भवति धना॒र्घो ध॑ना॒र्घो भ॑वति॒ यद् यद् भ॑वति धना॒र्घो ध॑ना॒र्घो भ॑वति॒ यत् । \newline
39. ध॒ना॒र्घ इति॑ धन - अ॒र्घः । \newline
40. भ॒व॒ति॒ यद् यद् भ॑वति भवति॒ यद् वै᳚श्वान॒रं ॅवै᳚श्वान॒रं ॅयद् भ॑वति भवति॒ यद् वै᳚श्वान॒रम् । \newline
41. यद् वै᳚श्वान॒रं ॅवै᳚श्वान॒रं ॅयद् यद् वै᳚श्वान॒रम् द्वाद॑शकपाल॒म् द्वाद॑शकपालं ॅवैश्वान॒रं ॅयद् यद् वै᳚श्वान॒रम् द्वाद॑शकपालम् । \newline
42. वै॒श्वा॒न॒रम् द्वाद॑शकपाल॒म् द्वाद॑शकपालं ॅवैश्वान॒रं ॅवै᳚श्वान॒रम् द्वाद॑शकपालम् नि॒र्वप॑ति नि॒र्वप॑ति॒ द्वाद॑शकपालं ॅवैश्वान॒रं ॅवै᳚श्वान॒रम् द्वाद॑शकपालम् नि॒र्वप॑ति । \newline
43. द्वाद॑शकपालम् नि॒र्वप॑ति नि॒र्वप॑ति॒ द्वाद॑शकपाल॒म् द्वाद॑शकपालम् नि॒र्वप॑ति संॅवथ्स॒रसा॑ताꣳ संॅवथ्स॒रसा॑ताम् नि॒र्वप॑ति॒ द्वाद॑शकपाल॒म् द्वाद॑शकपालम् नि॒र्वप॑ति संॅवथ्स॒रसा॑ताम् । \newline
44. द्वाद॑शकपाल॒मिति॒ द्वाद॑श - क॒पा॒ल॒म् । \newline
45. नि॒र्वप॑ति संॅवथ्स॒रसा॑ताꣳ संॅवथ्स॒रसा॑ताम् नि॒र्वप॑ति नि॒र्वप॑ति संॅवथ्स॒रसा॑ता मे॒वैव सं॑ॅवथ्स॒रसा॑ताम् नि॒र्वप॑ति नि॒र्वप॑ति संॅवथ्स॒रसा॑ता मे॒व । \newline
46. नि॒र्वप॒तीति॑ निः - वप॑ति । \newline
47. सं॒ॅव॒थ्स॒रसा॑ता मे॒वैव सं॑ॅवथ्स॒रसा॑ताꣳ संॅवथ्स॒रसा॑ता मे॒व स॒निꣳ स॒नि मे॒व सं॑ॅवथ्स॒रसा॑ताꣳ संॅवथ्स॒रसा॑ता मे॒व स॒निम् । \newline
48. सं॒ॅव॒थ्स॒रसा॑ता॒मिति॑ संॅवथ्स॒र - सा॒ता॒म् । \newline
49. ए॒व स॒निꣳ स॒नि मे॒वैव स॒नि म॒भ्य॑भि स॒नि मे॒वैव स॒नि म॒भि । \newline
50. स॒नि म॒भ्य॑भि स॒निꣳ स॒नि म॒भि प्र प्राभि स॒निꣳ स॒नि म॒भि प्र । \newline
51. अ॒भि प्र प्राभ्य॑भि प्र च्य॑वते च्यवते॒ प्राभ्य॑भि प्र च्य॑वते । \newline
52. प्र च्य॑वते च्यवते॒ प्र प्र च्य॑वते॒ दान॑कामा॒ दान॑कामा श्च्यवते॒ प्र प्र च्य॑वते॒ दान॑कामाः । \newline
53. च्य॒व॒ते॒ दान॑कामा॒ दान॑कामा श्च्यवते च्यवते॒ दान॑कामा अस्मा अस्मै॒ दान॑कामा श्च्यवते च्यवते॒ दान॑कामा अस्मै । \newline
54. दान॑कामा अस्मा अस्मै॒ दान॑कामा॒ दान॑कामा अस्मै प्र॒जाः प्र॒जा अ॑स्मै॒ दान॑कामा॒ दान॑कामा अस्मै प्र॒जाः । \newline
55. दान॑कामा॒ इति॒ दान॑ - का॒माः॒ । \newline
56. अ॒स्मै॒ प्र॒जाः प्र॒जा अ॑स्मा अस्मै प्र॒जा भ॑वन्ति भवन्ति प्र॒जा अ॑स्मा अस्मै प्र॒जा भ॑वन्ति । \newline
57. प्र॒जा भ॑वन्ति भवन्ति प्र॒जाः प्र॒जा भ॑वन्ति॒ यो यो भ॑वन्ति प्र॒जाः प्र॒जा भ॑वन्ति॒ यः । \newline
58. प्र॒जा इति॑ प्र - जाः । \newline
59. भ॒व॒न्ति॒ यो यो भ॑वन्ति भवन्ति॒ यो वै वै यो भ॑वन्ति भवन्ति॒ यो वै । \newline
60. यो वै वै यो यो वै सं॑ॅवथ्स॒रꣳ सं॑ॅवथ्स॒रं ॅवै यो यो वै सं॑ॅवथ्स॒रम् । \newline
61. वै सं॑ॅवथ्स॒रꣳ सं॑ॅवथ्स॒रं ॅवै वै सं॑ॅवथ्स॒रम् प्र॒युज्य॑ प्र॒युज्य॑ संॅवथ्स॒रं ॅवै वै सं॑ॅवथ्स॒रम् प्र॒युज्य॑ । \newline
62. सं॒ॅव॒थ्स॒रम् प्र॒युज्य॑ प्र॒युज्य॑ संॅवथ्स॒रꣳ सं॑ॅवथ्स॒रम् प्र॒युज्य॒ न न प्र॒युज्य॑ संॅवथ्स॒रꣳ सं॑ॅवथ्स॒रम् प्र॒युज्य॒ न । \newline
63. सं॒ॅव॒थ्स॒रमिति॑ सं - व॒थ्स॒रम् । \newline
\pagebreak
\markright{ TS 2.2.6.5  \hfill https://www.vedavms.in \hfill}
\addcontentsline{toc}{section}{ TS 2.2.6.5 }
\section*{ TS 2.2.6.5 }

\textbf{TS 2.2.6.5 } \newline
\textbf{Samhita Paata} \newline

प्र॒युज्य॒ न वि॑मु॒ञ्चत्य॑प्रतिष्ठा॒नो वै स भ॑वत्ये॒तमे॒व वै᳚श्वान॒रं पुन॑रा॒गत्य॒ निर्व॑पे॒द्यमे॒व प्र॑यु॒ङ्क्ते तं भा॑ग॒धेये॑न॒ वि मु॑ञ्चति॒ प्रति॑ष्ठित्यै॒ यया॒ रज्वो᳚त्त॒मां गामा॒जेत् तां भ्रातृ॑व्याय॒ प्र हि॑णुया॒न्निर्.ऋ॑तिमे॒वास्मै॒ प्र हि॑णोति ॥ \newline

\textbf{Pada Paata} \newline

प्र॒युज्येति॑ प्र - युज्य॑ । न । वि॒मु॒ञ्चतीति॑ वि - मु॒ञ्चति॑ । अ॒प्र॒ति॒ष्ठा॒न इत्य॑प्रति - स्था॒नः । वै । सः । भ॒व॒ति॒ । ए॒तम् । ए॒व । वै॒श्वा॒न॒रम् । पुनः॑ । आ॒गत्येत्या᳚ - गत्य॑ । निरिति॑ । व॒पे॒त् । यम् । ए॒व । प्र॒यु॒ङ्क्त इति॑ प्र - यु॒ङ्क्ते । तम् । भा॒ग॒धेये॒नेति॑ भाग - धेये॑न । वीति॑ । मु॒ञ्च॒ति॒ । प्रति॑ष्ठित्या॒ इति॒ प्रति॑ - स्थि॒त्यै॒ । यया᳚ । रज्वा᳚ । उ॒त्त॒मामित्यु॑त् - त॒माम् । गाम् । आ॒जेदित्या᳚ - अ॒जेत् । ताम् । भ्रातृ॑व्याय । प्रेति॑ । हि॒णु॒या॒त् । निर्.ऋ॑ति॒मिति॒ निः - ऋ॒ति॒म् । ए॒व । अ॒स्मै॒ । प्रेति॑ । हि॒णो॒ति॒ ॥  \newline


\textbf{Krama Paata} \newline

प्र॒युज्य॒ न । प्र॒युज्येति॑ प्र - युज्य॑ । न वि॑मु॒ञ्चति॑ । वि॒मु॒ञ्चत्य॑प्रतिष्ठा॒नः । वि॒मु॒ञ्चतीति॑ वि - मु॒ञ्चति॑ । अ॒प्र॒ति॒ष्ठा॒नो वै । अ॒प्र॒ति॒ष्ठा॒न इत्य॑प्रति - स्था॒नः । वै सः । स भ॑वति । भ॒व॒त्ये॒तम् । ए॒तमे॒व । ए॒व वै᳚श्वान॒रम् । वै॒श्वा॒न॒रम् पुनः॑ । पुन॑रा॒गत्य॑ । आ॒गत्य॒ निः । आ॒गत्येत्या᳚ - गत्य॑ । निर् व॑पेत् । व॒पे॒द् यम् । यमे॒व । ए॒व प्र॑यु॒ङ्क्ते । प्र॒यु॒ङ्क्ते तम् । प्र॒यु॒ङ्क्त इति॑ प्र - यु॒ङ्क्ते । तम् भा॑ग॒धेये॑न । भा॒ग॒धेये॑न॒ वि । भा॒ग॒धेये॒नेति॑ भाग - धेये॑न । वि मु॑ञ्चति । मु॒ञ्च॒ति॒ प्रति॑ष्ठित्यै । प्रति॑ष्ठित्यै॒ यया᳚ । प्रति॑ष्ठत्या॒ इति॒ प्रति॑ - स्थि॒त्यै॒ । यया॒ रज्वा᳚ । रज्वो᳚त्त॒माम् । उ॒त्त॒माम् गाम् । उ॒त्त॒मामित्यु॑त् - त॒माम् । गामा॒जेत् । आ॒जेत् ताम् । आ॒जेदित्या᳚ - अ॒जेत् । ताम् भ्रातृ॑व्याय । भातृ॑व्याय॒ प्र । प्र हि॑णुयात् । हि॒णु॒या॒न्निर्.ऋ॑तिम् । निर्.ऋ॑तिमे॒व । निर्.ऋ॑ति॒मिति॒ निः - ऋ॒ति॒म् । ए॒वास्मै᳚ । अ॒स्मै॒ प्र । प्र हि॑णोति । हि॒णो॒तीति॑ हिणोति । \newline

\textbf{Jatai Paata} \newline

1. प्र॒युज्य॒ न न प्र॒युज्य॑ प्र॒युज्य॒ न । \newline
2. प्र॒युज्येति॑ प्र - युज्य॑ । \newline
3. न वि॑मु॒ञ्चति॑ विमु॒ञ्चति॒ न न वि॑मु॒ञ्चति॑ । \newline
4. वि॒मु॒ञ्च त्य॑प्रतिष्ठा॒नो᳚ ऽप्रतिष्ठा॒नो वि॑मु॒ञ्चति॑ विमु॒ञ्च त्य॑प्रतिष्ठा॒नः । \newline
5. वि॒मु॒ञ्चतीति॑ वि - मु॒ञ्चति॑ । \newline
6. अ॒प्र॒ति॒ष्ठा॒नो वै वा अ॑प्रतिष्ठा॒नो᳚ ऽप्रतिष्ठा॒नो वै । \newline
7. अ॒प्र॒ति॒ष्ठा॒न इत्य॑प्रति - स्था॒नः । \newline
8. वै स स वै वै सः । \newline
9. स भ॑वति भवति॒ स स भ॑वति । \newline
10. भ॒व॒ त्ये॒त मे॒तम् भ॑वति भव त्ये॒तम् । \newline
11. ए॒त मे॒वैवैत मे॒त मे॒व । \newline
12. ए॒व वै᳚श्वान॒रं ॅवै᳚श्वान॒र मे॒वैव वै᳚श्वान॒रम् । \newline
13. वै॒श्वा॒न॒रम् पुनः॒ पुन॑र् वैश्वान॒रं ॅवै᳚श्वान॒रम् पुनः॑ । \newline
14. पुन॑ रा॒गत्या॒ गत्य॒ पुनः॒ पुन॑ रा॒गत्य॑ । \newline
15. आ॒गत्य॒ निर् णिरा॒गत्या॒ गत्य॒ निः । \newline
16. आ॒गत्येत्या᳚ - गत्य॑ । \newline
17. निर् व॑पेद् वपे॒न् निर् णिर् व॑पेत् । \newline
18. व॒पे॒द् यं ॅयं ॅव॑पेद् वपे॒द् यम् । \newline
19. य मे॒वैव यं ॅय मे॒व । \newline
20. ए॒व प्र॑यु॒ङ्क्ते प्र॑यु॒ङ्क्त ए॒वैव प्र॑यु॒ङ्क्ते । \newline
21. प्र॒यु॒ङ्क्ते तम् तम् प्र॑यु॒ङ्क्ते प्र॑यु॒ङ्क्ते तम् । \newline
22. प्र॒यु॒ङ्क्त इति॑ प्र - यु॒ङ्क्ते । \newline
23. तम् भा॑ग॒धेये॑न भाग॒धेये॑न॒ तम् तम् भा॑ग॒धेये॑न । \newline
24. भा॒ग॒धेये॑न॒ वि वि भा॑ग॒धेये॑न भाग॒धेये॑न॒ वि । \newline
25. भा॒ग॒धेये॒नेति॑ भाग - धेये॑न । \newline
26. वि मु॑ञ्चति मुञ्चति॒ वि वि मु॑ञ्चति । \newline
27. मु॒ञ्च॒ति॒ प्रति॑ष्ठित्यै॒ प्रति॑ष्ठित्यै मुञ्चति मुञ्चति॒ प्रति॑ष्ठित्यै । \newline
28. प्रति॑ष्ठित्यै॒ यया॒ यया॒ प्रति॑ष्ठित्यै॒ प्रति॑ष्ठित्यै॒ यया᳚ । \newline
29. प्रति॑ष्ठित्या॒ इति॒ प्रति॑ - स्थि॒त्यै॒ । \newline
30. यया॒ रज्वा॒ रज्वा॒ यया॒ यया॒ रज्वा᳚ । \newline
31. रज्वो᳚त्त॒मा मु॑त्त॒माꣳ रज्वा॒ रज्वो᳚त्त॒माम् । \newline
32. उ॒त्त॒माम् गाम् गा मु॑त्त॒मा मु॑त्त॒माम् गाम् । \newline
33. उ॒त्त॒मामित्यु॑त् - त॒माम् । \newline
34. गा मा॒जे दा॒जेद् गाम् गा मा॒जेत् । \newline
35. आ॒जेत् ताम् ता मा॒जे दा॒जेत् ताम् । \newline
36. आ॒जेदित्या᳚ - अ॒जेत् । \newline
37. ताम् भ्रातृ॑व्याय॒ भ्रातृ॑व्याय॒ ताम् ताम् भ्रातृ॑व्याय । \newline
38. भ्रातृ॑व्याय॒ प्र प्र भ्रातृ॑व्याय॒ भ्रातृ॑व्याय॒ प्र । \newline
39. प्र हि॑णुया द्धिणुया॒त् प्र प्र हि॑णुयात् । \newline
40. हि॒णु॒या॒न् निर्.ऋ॑ति॒म् निर्.ऋ॑तिꣳ हिणुया द्धिणुया॒न् निर्.ऋ॑तिम् । \newline
41. निर्.ऋ॑ति मे॒वैव निर्.ऋ॑ति॒म् निर्.ऋ॑ति मे॒व । \newline
42. निर्.ऋ॑ति॒मिति॒ निः - ऋ॒ति॒म् । \newline
43. ए॒वास्मा॑ अस्मा ए॒वैवास्मै᳚ । \newline
44. अ॒स्मै॒ प्र प्रास्मा॑ अस्मै॒ प्र । \newline
45. प्र हि॑णोति हिणोति॒ प्र प्र हि॑णोति । \newline
46. हि॒णो॒तीति॑ हिणोति । \newline

\textbf{Ghana Paata } \newline

1. प्र॒युज्य॒ न न प्र॒युज्य॑ प्र॒युज्य॒ न वि॑मु॒ञ्चति॑ विमु॒ञ्चति॒ न प्र॒युज्य॑ प्र॒युज्य॒ न वि॑मु॒ञ्चति॑ । \newline
2. प्र॒युज्येति॑ प्र - युज्य॑ । \newline
3. न वि॑मु॒ञ्चति॑ विमु॒ञ्चति॒ न न वि॑मु॒ञ्च त्य॑प्रतिष्ठा॒नो᳚ ऽप्रतिष्ठा॒नो वि॑मु॒ञ्चति॒ न न वि॑मु॒ञ्च त्य॑प्रतिष्ठा॒नः । \newline
4. वि॒मु॒ञ्च त्य॑प्रतिष्ठा॒नो᳚ ऽप्रतिष्ठा॒नो वि॑मु॒ञ्चति॑ विमु॒ञ्च त्य॑प्रतिष्ठा॒नो वै वा अ॑प्रतिष्ठा॒नो वि॑मु॒ञ्चति॑ विमु॒ञ्च त्य॑प्रतिष्ठा॒नो वै । \newline
5. वि॒मु॒ञ्चतीति॑ वि - मु॒ञ्चति॑ । \newline
6. अ॒प्र॒ति॒ष्ठा॒नो वै वा अ॑प्रतिष्ठा॒नो᳚ ऽप्रतिष्ठा॒नो वै स स वा अ॑प्रतिष्ठा॒नो᳚ ऽप्रतिष्ठा॒नो वै सः । \newline
7. अ॒प्र॒ति॒ष्ठा॒न इत्य॑प्रति - स्था॒नः । \newline
8. वै स स वै वै स भ॑वति भवति॒ स वै वै स भ॑वति । \newline
9. स भ॑वति भवति॒ स स भ॑व त्ये॒त मे॒तम् भ॑वति॒ स स भ॑व त्ये॒तम् । \newline
10. भ॒व॒ त्ये॒त मे॒तम् भ॑वति भव त्ये॒त मे॒वैवैतम् भ॑वति भव त्ये॒त मे॒व । \newline
11. ए॒त मे॒वैवैत मे॒त मे॒व वै᳚श्वान॒रं ॅवै᳚श्वान॒र मे॒वैत मे॒त मे॒व वै᳚श्वान॒रम् । \newline
12. ए॒व वै᳚श्वान॒रं ॅवै᳚श्वान॒र मे॒वैव वै᳚श्वान॒रम् पुनः॒ पुन॑र् वैश्वान॒र मे॒वैव वै᳚श्वान॒रम् पुनः॑ । \newline
13. वै॒श्वा॒न॒रम् पुनः॒ पुन॑र् वैश्वान॒रं ॅवै᳚श्वान॒रम् पुन॑ रा॒गत्या॒ गत्य॒ पुन॑र् वैश्वान॒रं ॅवै᳚श्वान॒रम् पुन॑ रा॒गत्य॑ । \newline
14. पुन॑ रा॒गत्या॒ गत्य॒ पुनः॒ पुन॑ रा॒गत्य॒ निर् णिरा॒गत्य॒ पुनः॒ पुन॑ रा॒गत्य॒ निः । \newline
15. आ॒गत्य॒ निर् णिरा॒गत्या॒ गत्य॒ निर् व॑पेद् वपे॒न् निरा॒गत्या॒ गत्य॒ निर् व॑पेत् । \newline
16. आ॒गत्येत्या᳚ - गत्य॑ । \newline
17. निर् व॑पेद् वपे॒न् निर् णिर् व॑पे॒द् यं ॅयं ॅव॑पे॒न् निर् णिर् व॑पे॒द् यम् । \newline
18. व॒पे॒द् यं ॅयं ॅव॑पेद् वपे॒द् य मे॒वैव यं ॅव॑पेद् वपे॒द् य मे॒व । \newline
19. य मे॒वैव यं ॅय मे॒व प्र॑यु॒ङ्क्ते प्र॑यु॒ङ्क्त ए॒व यं ॅय मे॒व प्र॑यु॒ङ्क्ते । \newline
20. ए॒व प्र॑यु॒ङ्क्ते प्र॑यु॒ङ्क्त ए॒वैव प्र॑यु॒ङ्क्ते तम् तम् प्र॑यु॒ङ्क्त ए॒वैव प्र॑यु॒ङ्क्ते तम् । \newline
21. प्र॒यु॒ङ्क्ते तम् तम् प्र॑यु॒ङ्क्ते प्र॑यु॒ङ्क्ते तम् भा॑ग॒धेये॑न भाग॒धेये॑न॒ तम् प्र॑यु॒ङ्क्ते प्र॑यु॒ङ्क्ते तम् भा॑ग॒धेये॑न । \newline
22. प्र॒यु॒ङ्क्त इति॑ प्र - यु॒ङ्क्ते । \newline
23. तम् भा॑ग॒धेये॑न भाग॒धेये॑न॒ तम् तम् भा॑ग॒धेये॑न॒ वि वि भा॑ग॒धेये॑न॒ तम् तम् भा॑ग॒धेये॑न॒ वि । \newline
24. भा॒ग॒धेये॑न॒ वि वि भा॑ग॒धेये॑न भाग॒धेये॑न॒ वि मु॑ञ्चति मुञ्चति॒ वि भा॑ग॒धेये॑न भाग॒धेये॑न॒ वि मु॑ञ्चति । \newline
25. भा॒ग॒धेये॒नेति॑ भाग - धेये॑न । \newline
26. वि मु॑ञ्चति मुञ्चति॒ वि वि मु॑ञ्चति॒ प्रति॑ष्ठित्यै॒ प्रति॑ष्ठित्यै मुञ्चति॒ वि वि मु॑ञ्चति॒ प्रति॑ष्ठित्यै । \newline
27. मु॒ञ्च॒ति॒ प्रति॑ष्ठित्यै॒ प्रति॑ष्ठित्यै मुञ्चति मुञ्चति॒ प्रति॑ष्ठित्यै॒ यया॒ यया॒ प्रति॑ष्ठित्यै मुञ्चति मुञ्चति॒ प्रति॑ष्ठित्यै॒ यया᳚ । \newline
28. प्रति॑ष्ठित्यै॒ यया॒ यया॒ प्रति॑ष्ठित्यै॒ प्रति॑ष्ठित्यै॒ यया॒ रज्वा॒ रज्वा॒ यया॒ प्रति॑ष्ठित्यै॒ प्रति॑ष्ठित्यै॒ यया॒ रज्वा᳚ । \newline
29. प्रति॑ष्ठित्या॒ इति॒ प्रति॑ - स्थि॒त्यै॒ । \newline
30. यया॒ रज्वा॒ रज्वा॒ यया॒ यया॒ रज्वो᳚त्त॒मा मु॑त्त॒माꣳ रज्वा॒ यया॒ यया॒ रज्वो᳚त्त॒माम् । \newline
31. रज्वो᳚त्त॒मा मु॑त्त॒माꣳ रज्वा॒ रज्वो᳚त्त॒माम् गाम् गा मु॑त्त॒माꣳ रज्वा॒ रज्वो᳚त्त॒माम् गाम् । \newline
32. उ॒त्त॒माम् गाम् गा मु॑त्त॒मा मु॑त्त॒माम् गा मा॒जे दा॒जेद् गा मु॑त्त॒मा मु॑त्त॒माम् गा मा॒जेत् । \newline
33. उ॒त्त॒मामित्यु॑त् - त॒माम् । \newline
34. गा मा॒जे दा॒जेद् गाम् गा मा॒जेत् ताम् ता मा॒जेद् गाम् गा मा॒जेत् ताम् । \newline
35. आ॒जेत् ताम् ता मा॒जे दा॒जेत् ताम् भ्रातृ॑व्याय॒ भ्रातृ॑व्याय॒ ता मा॒जे दा॒जेत् ताम् भ्रातृ॑व्याय । \newline
36. आ॒जेदित्या᳚ - अ॒जेत् । \newline
37. ताम् भ्रातृ॑व्याय॒ भ्रातृ॑व्याय॒ ताम् ताम् भ्रातृ॑व्याय॒ प्र प्र भ्रातृ॑व्याय॒ ताम् ताम् भ्रातृ॑व्याय॒ प्र । \newline
38. भ्रातृ॑व्याय॒ प्र प्र भ्रातृ॑व्याय॒ भ्रातृ॑व्याय॒ प्र हि॑णुया द्धिणुया॒त् प्र भ्रातृ॑व्याय॒ भ्रातृ॑व्याय॒ प्र हि॑णुयात् । \newline
39. प्र हि॑णुया द्धिणुया॒त् प्र प्र हि॑णुया॒न् निर्.ऋ॑ति॒म् निर्.ऋ॑तिꣳ हिणुया॒त् प्र प्र हि॑णुया॒न् निर्.ऋ॑तिम् । \newline
40. हि॒णु॒या॒न् निर्.ऋ॑ति॒म् निर्.ऋ॑तिꣳ हिणुया द्धिणुया॒न् निर्.ऋ॑ति मे॒वैव निर्.ऋ॑तिꣳ हिणुया द्धिणुया॒न् निर्.ऋ॑ति मे॒व । \newline
41. निर्.ऋ॑ति मे॒वैव निर्.ऋ॑ति॒म् निर्.ऋ॑ति मे॒वास्मा॑ अस्मा ए॒व निर्.ऋ॑ति॒म् निर्.ऋ॑ति मे॒वास्मै᳚ । \newline
42. निर्.ऋ॑ति॒मिति॒ निः - ऋ॒ति॒म् । \newline
43. ए॒वास्मा॑ अस्मा ए॒वैवास्मै॒ प्र प्रास्मा॑ ए॒वैवास्मै॒ प्र । \newline
44. अ॒स्मै॒ प्र प्रास्मा॑ अस्मै॒ प्र हि॑णोति हिणोति॒ प्रास्मा॑ अस्मै॒ प्र हि॑णोति । \newline
45. प्र हि॑णोति हिणोति॒ प्र प्र हि॑णोति । \newline
46. हि॒णो॒तीति॑ हिणोति । \newline
\pagebreak
\markright{ TS 2.2.7.1  \hfill https://www.vedavms.in \hfill}
\addcontentsline{toc}{section}{ TS 2.2.7.1 }
\section*{ TS 2.2.7.1 }

\textbf{TS 2.2.7.1 } \newline
\textbf{Samhita Paata} \newline

ऐ॒न्द्रं च॒रु निंर्व॑पेत् प॒शुका॑म ऐ॒न्द्रा वै प॒शव॒ इन्द्र॑मे॒व स्वेन॑ भाग॒धेये॒नोप॑ धावति॒ स ए॒वास्मै॑ प॒शून् प्रय॑च्छति पशु॒माने॒व भ॑वति च॒रुर्भ॑वति॒ स्वादे॒वास्मै॒ योनेः᳚ प॒शून् प्रज॑नय॒तीन्द्रा॑येन्द्रि॒याव॑ते पुरो॒डाश॒मेका॑दशकपालं॒ निर्व॑पेत् प॒शुका॑म इन्द्रि॒यं ॅवै प॒शव॒ इन्द्र॑मे॒वेन्द्रि॒याव॑न्तꣳ॒॒ स्वेन ॑भाग॒धेये॒नोप॑ धावति॒ स - [  ] \newline

\textbf{Pada Paata} \newline

ऐ॒न्द्रम् । च॒रुम् । निरिति॑ । व॒पे॒त् । प॒शुका॑म॒ इति॑ प॒शु - का॒मः॒ । ऐ॒न्द्राः । वै । प॒शवः॑ । इन्द्र᳚म् । ए॒व । स्वेन॑ । भा॒ग॒धेये॒नेति॑ भाग - धेये॑न । उपेति॑ । धा॒व॒ति॒ । सः । ए॒व । अ॒स्मै॒ । प॒शून् । प्रेति॑ । य॒च्छ॒ति॒ । प॒शु॒मानिति॑ पशु - मान् । ए॒व । भ॒व॒ति॒ । च॒रुः । भ॒व॒ति॒ । स्वात् । ए॒व । अ॒स्मै॒ । योनेः᳚ । प॒शून् । प्रेति॑ । ज॒न॒य॒ति॒ । इन्द्रा॑य । इ॒न्द्रि॒याव॑त॒ इती᳚न्द्रि॒य - व॒ते॒ । पु॒रो॒डाश᳚म् । एका॑दशकपाल॒मित्येका॑दश - क॒पा॒ल॒म् । निरिति॑ । व॒पे॒त् । प॒शुका॑म॒ इति॑ प॒शु - का॒मः॒ । इ॒न्द्रि॒यम् । वै । प॒शवः॑ । इन्द्र᳚म् । ए॒व । इ॒न्द्रि॒याव॑न्त॒मिती᳚न्द्रि॒य - व॒न्त॒म् । स्वेन॑ । भा॒ग॒धेये॒नेति॑ भाग - धेये॑न । उपेति॑ । धा॒व॒ति॒ । सः ।  \newline


\textbf{Krama Paata} \newline

ऐ॒न्द्रम् च॒रुम् । च॒रुम् निः । निर् व॑पेत् । व॒पे॒त् प॒शुका॑मः । प॒शुका॑म ऐ॒न्द्राः । प॒शुका॑म॒ इति॑ प॒शु - का॒मः॒ । ऐ॒न्द्रा वै । वै प॒शवः॑ । प॒शव॒ इन्द्र᳚म् । इन्द्र॑मे॒व । ए॒व स्वेन॑ । स्वेन॑ भाग॒धेये॑न । भा॒ग॒धेये॒नोप॑ । भा॒ग॒धेये॒नेति॑ भाग - धेये॑न । उप॑ धावति । धा॒व॒ति॒ सः । स ए॒व । ए॒वास्मै᳚ । अ॒स्मै॒ प॒शून् । प॒शून् प्र । प्र य॑च्छति । य॒च्छ॒ति॒ प॒शु॒मान् । प॒शु॒माने॒व । प॒शु॒मानिति॑ पशु - मान् । ए॒व भ॑वति । भ॒व॒ति॒ च॒रुः । च॒रुर् भ॑वति । भ॒व॒ति॒ स्वात् । स्वादे॒व । ए॒वास्मै᳚ । अ॒स्मै॒ योनेः᳚ । योनेः᳚ प॒शून् । प॒शून् प्र । प्र ज॑नयति । ज॒न॒य॒तीन्द्रा॑य । इन्द्रा॑येन्द्रि॒याव॑ते । इ॒न्द्रि॒याव॑ते पुरो॒डाश᳚म् । इ॒न्द्रि॒याव॑त॒ इती᳚न्द्रि॒य - व॒ते॒ । पु॒रो॒डाश॒मेका॑दशकपालम् । एका॑दशकपाल॒म् निः । एका॑दशकपाल॒मित्येका॑दश - क॒पा॒ल॒म् । निर् व॑पेत् । व॒पे॒त् प॒शुका॑मः । प॒शुका॑म इन्द्रि॒यम् । प॒शुका॑म॒ इति॑ प॒शु - का॒मः॒ । इ॒न्द्रि॒यं ॅवै । वै प॒शवः॑ । प॒शव॒ इन्द्र᳚म् । इन्द्र॑मे॒व । ए॒वेन्द्रि॒याव॑न्तम् । इ॒न्द्रि॒याव॑न्तꣳ॒॒ स्वेन॑ । इ॒न्द्रि॒याव॑न्त॒मिती᳚न्द्रि॒य - व॒न्त॒म् । स्वेन॑ भाग॒धेये॑न । भा॒ग॒धेये॒नोप॑ । भा॒ग॒धेये॒नेति॑ भाग - धेये॑न । उप॑ धावति । धा॒व॒ति॒ सः । स ए॒व \newline

\textbf{Jatai Paata} \newline

1. ऐ॒न्द्रम् च॒रुम् च॒रु मै॒न्द्र मै॒न्द्रम् च॒रुम् । \newline
2. च॒रुम् निर् णिश्च॒रुम् च॒रुम् निः । \newline
3. निर् व॑पेद् वपे॒न् निर् णिर् व॑पेत् । \newline
4. व॒पे॒त् प॒शुका॑मः प॒शुका॑मो वपेद् वपेत् प॒शुका॑मः । \newline
5. प॒शुका॑म ऐ॒न्द्रा ऐ॒न्द्राः प॒शुका॑मः प॒शुका॑म ऐ॒न्द्राः । \newline
6. प॒शुका॑म॒ इति॑ प॒शु - का॒मः॒ । \newline
7. ऐ॒न्द्रा वै वा ऐ॒न्द्रा ऐ॒न्द्रा वै । \newline
8. वै प॒शवः॑ प॒शवो॒ वै वै प॒शवः॑ । \newline
9. प॒शव॒ इन्द्र॒ मिन्द्र॑म् प॒शवः॑ प॒शव॒ इन्द्र᳚म् । \newline
10. इन्द्र॑ मे॒वैवे न्द्र॒ मिन्द्र॑ मे॒व । \newline
11. ए॒व स्वेन॒ स्वेनै॒वैव स्वेन॑ । \newline
12. स्वेन॑ भाग॒धेये॑न भाग॒धेये॑न॒ स्वेन॒ स्वेन॑ भाग॒धेये॑न । \newline
13. भा॒ग॒धेये॒नोपोप॑ भाग॒धेये॑न भाग॒धेये॒नोप॑ । \newline
14. भा॒ग॒धेये॒नेति॑ भाग - धेये॑न । \newline
15. उप॑ धावति धाव॒ त्युपोप॑ धावति । \newline
16. धा॒व॒ति॒ स स धा॑वति धावति॒ सः । \newline
17. स ए॒वैव स स ए॒व । \newline
18. ए॒वास्मा॑ अस्मा ए॒वैवास्मै᳚ । \newline
19. अ॒स्मै॒ प॒शून् प॒शू न॑स्मा अस्मै प॒शून् । \newline
20. प॒शून् प्र प्र प॒शून् प॒शून् प्र । \newline
21. प्र य॑च्छति यच्छति॒ प्र प्र य॑च्छति । \newline
22. य॒च्छ॒ति॒ प॒शु॒मान् प॑शु॒मान्. य॑च्छति यच्छति पशु॒मान् । \newline
23. प॒शु॒मा ने॒वैव प॑शु॒मान् प॑शु॒मा ने॒व । \newline
24. प॒शु॒मानिति॑ पशु - मान् । \newline
25. ए॒व भ॑वति भव त्ये॒वैव भ॑वति । \newline
26. भ॒व॒ति॒ च॒रु श्च॒रुर् भ॑वति भवति च॒रुः । \newline
27. च॒रुर् भ॑वति भवति च॒रु श्च॒रुर् भ॑वति । \newline
28. भ॒व॒ति॒ स्वाथ् स्वाद् भ॑वति भवति॒ स्वात् । \newline
29. स्वा दे॒वैव स्वाथ् स्वा दे॒व । \newline
30. ए॒वास्मा॑ अस्मा ए॒वैवास्मै᳚ । \newline
31. अ॒स्मै॒ योने॒र् योने॑ रस्मा अस्मै॒ योनेः᳚ । \newline
32. योनेः᳚ प॒शून् प॒शून्. योने॒र् योनेः᳚ प॒शून् । \newline
33. प॒शून् प्र प्र प॒शून् प॒शून् प्र । \newline
34. प्र ज॑नयति जनयति॒ प्र प्र ज॑नयति । \newline
35. ज॒न॒य॒तीन्द्रा॒ये न्द्रा॑य जनयति जनय॒तीन्द्रा॑य । \newline
36. इन्द्रा॑ये न्द्रि॒याव॑त इन्द्रि॒याव॑त॒ इन्द्रा॒ये न्द्रा॑ये न्द्रि॒याव॑ते । \newline
37. इ॒न्द्रि॒याव॑ते पुरो॒डाश॑म् पुरो॒डाश॑ मिन्द्रि॒याव॑त इन्द्रि॒याव॑ते पुरो॒डाश᳚म् । \newline
38. इ॒न्द्रि॒याव॑त॒ इती᳚न्द्रि॒य - व॒ते॒ । \newline
39. पु॒रो॒डाश॒ मेका॑दशकपाल॒ मेका॑दशकपालम् पुरो॒डाश॑म् पुरो॒डाश॒ मेका॑दशकपालम् । \newline
40. एका॑दशकपाल॒म् निर् णिरेका॑दशकपाल॒ मेका॑दशकपाल॒म् निः । \newline
41. एका॑दशकपाल॒मित्येका॑दश - क॒पा॒ल॒म् । \newline
42. निर् व॑पेद् वपे॒न् निर् णिर् व॑पेत् । \newline
43. व॒पे॒त् प॒शुका॑मः प॒शुका॑मो वपेद् वपेत् प॒शुका॑मः । \newline
44. प॒शुका॑म इन्द्रि॒य मि॑न्द्रि॒यम् प॒शुका॑मः प॒शुका॑म इन्द्रि॒यम् । \newline
45. प॒शुका॑म॒ इति॑ प॒शु - का॒मः॒ । \newline
46. इ॒न्द्रि॒यं ॅवै वा इ॑न्द्रि॒य मि॑न्द्रि॒यं ॅवै । \newline
47. वै प॒शवः॑ प॒शवो॒ वै वै प॒शवः॑ । \newline
48. प॒शव॒ इन्द्र॒ मिन्द्र॑म् प॒शवः॑ प॒शव॒ इन्द्र᳚म् । \newline
49. इन्द्र॑ मे॒वैवे न्द्र॒ मिन्द्र॑ मे॒व । \newline
50. ए॒वे न्द्रि॒याव॑न्त मिन्द्रि॒याव॑न्त मे॒वैवे न्द्रि॒याव॑न्तम् । \newline
51. इ॒न्द्रि॒याव॑न्तꣳ॒॒ स्वेन॒ स्वेने᳚ न्द्रि॒याव॑न्त मिन्द्रि॒याव॑न्तꣳ॒॒ स्वेन॑ । \newline
52. इ॒न्द्रि॒याव॑न्त॒मिती᳚न्द्रि॒य - व॒न्त॒म् । \newline
53. स्वेन॑ भाग॒धेये॑न भाग॒धेये॑न॒ स्वेन॒ स्वेन॑ भाग॒धेये॑न । \newline
54. भा॒ग॒धेये॒नोपोप॑ भाग॒धेये॑न भाग॒धेये॒नोप॑ । \newline
55. भा॒ग॒धेये॒नेति॑ भाग - धेये॑न । \newline
56. उप॑ धावति धाव॒ त्युपोप॑ धावति । \newline
57. धा॒व॒ति॒ स स धा॑वति धावति॒ सः । \newline
58. स ए॒वैव स स ए॒व । \newline

\textbf{Ghana Paata } \newline

1. ऐ॒न्द्रम् च॒रुम् च॒रु मै॒न्द्र मै॒न्द्रम् च॒रुम् निर् णिश्च॒रु मै॒न्द्र मै॒न्द्रम् च॒रुम् निः । \newline
2. च॒रुम् निर् णिश्च॒रुम् च॒रुम् निर् व॑पेद् वपे॒न् निश्च॒रुम् च॒रुम् निर् व॑पेत् । \newline
3. निर् व॑पेद् वपे॒न् निर् णिर् व॑पेत् प॒शुका॑मः प॒शुका॑मो वपे॒न् निर् णिर् व॑पेत् प॒शुका॑मः । \newline
4. व॒पे॒त् प॒शुका॑मः प॒शुका॑मो वपेद् वपेत् प॒शुका॑म ऐ॒न्द्रा ऐ॒न्द्राः प॒शुका॑मो वपेद् वपेत् प॒शुका॑म ऐ॒न्द्राः । \newline
5. प॒शुका॑म ऐ॒न्द्रा ऐ॒न्द्राः प॒शुका॑मः प॒शुका॑म ऐ॒न्द्रा वै वा ऐ॒न्द्राः प॒शुका॑मः प॒शुका॑म ऐ॒न्द्रा वै । \newline
6. प॒शुका॑म॒ इति॑ प॒शु - का॒मः॒ । \newline
7. ऐ॒न्द्रा वै वा ऐ॒न्द्रा ऐ॒न्द्रा वै प॒शवः॑ प॒शवो॒ वा ऐ॒न्द्रा ऐ॒न्द्रा वै प॒शवः॑ । \newline
8. वै प॒शवः॑ प॒शवो॒ वै वै प॒शव॒ इन्द्र॒ मिन्द्र॑म् प॒शवो॒ वै वै प॒शव॒ इन्द्र᳚म् । \newline
9. प॒शव॒ इन्द्र॒ मिन्द्र॑म् प॒शवः॑ प॒शव॒ इन्द्र॑ मे॒वैवे न्द्र॑म् प॒शवः॑ प॒शव॒ इन्द्र॑ मे॒व । \newline
10. इन्द्र॑ मे॒वैवे न्द्र॒ मिन्द्र॑ मे॒व स्वेन॒ स्वेनै॒वे न्द्र॒ मिन्द्र॑ मे॒व स्वेन॑ । \newline
11. ए॒व स्वेन॒ स्वेनै॒वैव स्वेन॑ भाग॒धेये॑न भाग॒धेये॑न॒ स्वेनै॒वैव स्वेन॑ भाग॒धेये॑न । \newline
12. स्वेन॑ भाग॒धेये॑न भाग॒धेये॑न॒ स्वेन॒ स्वेन॑ भाग॒धेये॒नोपोप॑ भाग॒धेये॑न॒ स्वेन॒ स्वेन॑ भाग॒धेये॒नोप॑ । \newline
13. भा॒ग॒धेये॒नोपोप॑ भाग॒धेये॑न भाग॒धेये॒नोप॑ धावति धाव॒ त्युप॑ भाग॒धेये॑न भाग॒धेये॒नोप॑ धावति । \newline
14. भा॒ग॒धेये॒नेति॑ भाग - धेये॑न । \newline
15. उप॑ धावति धाव॒ त्युपोप॑ धावति॒ स स धा॑व॒ त्युपोप॑ धावति॒ सः । \newline
16. धा॒व॒ति॒ स स धा॑वति धावति॒ स ए॒वैव स धा॑वति धावति॒ स ए॒व । \newline
17. स ए॒वैव स स ए॒वास्मा॑ अस्मा ए॒व स स ए॒वास्मै᳚ । \newline
18. ए॒वास्मा॑ अस्मा ए॒वैवास्मै॑ प॒शून् प॒शू न॑स्मा ए॒वैवास्मै॑ प॒शून् । \newline
19. अ॒स्मै॒ प॒शून् प॒शू न॑स्मा अस्मै प॒शून् प्र प्र प॒शू न॑स्मा अस्मै प॒शून् प्र । \newline
20. प॒शून् प्र प्र प॒शून् प॒शून् प्र य॑च्छति यच्छति॒ प्र प॒शून् प॒शून् प्र य॑च्छति । \newline
21. प्र य॑च्छति यच्छति॒ प्र प्र य॑च्छति पशु॒मान् प॑शु॒मान्. य॑च्छति॒ प्र प्र य॑च्छति पशु॒मान् । \newline
22. य॒च्छ॒ति॒ प॒शु॒मान् प॑शु॒मान्. य॑च्छति यच्छति पशु॒मा ने॒वैव प॑शु॒मान्. य॑च्छति यच्छति पशु॒मा ने॒व । \newline
23. प॒शु॒मा ने॒वैव प॑शु॒मान् प॑शु॒मा ने॒व भ॑वति भव त्ये॒व प॑शु॒मान् प॑शु॒मा ने॒व भ॑वति । \newline
24. प॒शु॒मानिति॑ पशु - मान् । \newline
25. ए॒व भ॑वति भव त्ये॒वैव भ॑वति च॒रु श्च॒रुर् भ॑व त्ये॒वैव भ॑वति च॒रुः । \newline
26. भ॒व॒ति॒ च॒रु श्च॒रुर् भ॑वति भवति च॒रुर् भ॑वति भवति च॒रुर् भ॑वति भवति च॒रुर् भ॑वति । \newline
27. च॒रुर् भ॑वति भवति च॒रु श्च॒रुर् भ॑वति॒ स्वाथ् स्वाद् भ॑वति च॒रु श्च॒रुर् भ॑वति॒ स्वात् । \newline
28. भ॒व॒ति॒ स्वाथ् स्वाद् भ॑वति भवति॒ स्वादे॒वैव स्वाद् भ॑वति भवति॒ स्वादे॒व । \newline
29. स्वादे॒वैव स्वाथ् स्वादे॒वास्मा॑ अस्मा ए॒व स्वाथ् स्वादे॒वास्मै᳚ । \newline
30. ए॒वास्मा॑ अस्मा ए॒वैवास्मै॒ योने॒र् योने॑रस्मा ए॒वैवास्मै॒ योनेः᳚ । \newline
31. अ॒स्मै॒ योने॒र् योने॑ रस्मा अस्मै॒ योनेः᳚ प॒शून् प॒शून्. योने॑ रस्मा अस्मै॒ योनेः᳚ प॒शून् । \newline
32. योनेः᳚ प॒शून् प॒शून्. योने॒र् योनेः᳚ प॒शून् प्र प्र प॒शून्. योने॒र् योनेः᳚ प॒शून् प्र । \newline
33. प॒शून् प्र प्र प॒शून् प॒शून् प्र ज॑नयति जनयति॒ प्र प॒शून् प॒शून् प्र ज॑नयति । \newline
34. प्र ज॑नयति जनयति॒ प्र प्र ज॑नय॒तीन्द्रा॒ये न्द्रा॑य जनयति॒ प्र प्र ज॑नय॒तीन्द्रा॑य । \newline
35. ज॒न॒य॒तीन्द्रा॒ये न्द्रा॑य जनयति जनय॒तीन्द्रा॑ये न्द्रि॒याव॑त इन्द्रि॒याव॑त॒ इन्द्रा॑य जनयति जनय॒तीन्द्रा॑ये न्द्रि॒याव॑ते । \newline
36. इन्द्रा॑ये न्द्रि॒याव॑त इन्द्रि॒याव॑त॒ इन्द्रा॒ये न्द्रा॑ये न्द्रि॒याव॑ते पुरो॒डाश॑म् पुरो॒डाश॑ मिन्द्रि॒याव॑त॒ इन्द्रा॒ये न्द्रा॑ये न्द्रि॒याव॑ते पुरो॒डाश᳚म् । \newline
37. इ॒न्द्रि॒याव॑ते पुरो॒डाश॑म् पुरो॒डाश॑ मिन्द्रि॒याव॑त इन्द्रि॒याव॑ते पुरो॒डाश॒ मेका॑दशकपाल॒ मेका॑दशकपालम् पुरो॒डाश॑ मिन्द्रि॒याव॑त इन्द्रि॒याव॑ते पुरो॒डाश॒ मेका॑दशकपालम् । \newline
38. इ॒न्द्रि॒याव॑त॒ इती᳚न्द्रि॒य - व॒ते॒ । \newline
39. पु॒रो॒डाश॒ मेका॑दशकपाल॒ मेका॑दशकपालम् पुरो॒डाश॑म् पुरो॒डाश॒ मेका॑दशकपाल॒म् निर् णिरेका॑दशकपालम् पुरो॒डाश॑म् पुरो॒डाश॒ मेका॑दशकपाल॒म् निः । \newline
40. एका॑दशकपाल॒म् निर् णिरेका॑दशकपाल॒ मेका॑दशकपाल॒म् निर् व॑पेद् वपे॒न् निरेका॑दशकपाल॒ मेका॑दशकपाल॒म् निर् व॑पेत् । \newline
41. एका॑दशकपाल॒मित्येका॑दश - क॒पा॒ल॒म् । \newline
42. निर् व॑पेद् वपे॒न् निर् णिर् व॑पेत् प॒शुका॑मः प॒शुका॑मो वपे॒न् निर् णिर् व॑पेत् प॒शुका॑मः । \newline
43. व॒पे॒त् प॒शुका॑मः प॒शुका॑मो वपेद् वपेत् प॒शुका॑म इन्द्रि॒य मि॑न्द्रि॒यम् प॒शुका॑मो वपेद् वपेत् प॒शुका॑म इन्द्रि॒यम् । \newline
44. प॒शुका॑म इन्द्रि॒य मि॑न्द्रि॒यम् प॒शुका॑मः प॒शुका॑म इन्द्रि॒यं ॅवै वा इ॑न्द्रि॒यम् प॒शुका॑मः प॒शुका॑म इन्द्रि॒यं ॅवै । \newline
45. प॒शुका॑म॒ इति॑ प॒शु - का॒मः॒ । \newline
46. इ॒न्द्रि॒यं ॅवै वा इ॑न्द्रि॒य मि॑न्द्रि॒यं ॅवै प॒शवः॑ प॒शवो॒ वा इ॑न्द्रि॒य मि॑न्द्रि॒यं ॅवै प॒शवः॑ । \newline
47. वै प॒शवः॑ प॒शवो॒ वै वै प॒शव॒ इन्द्र॒ मिन्द्र॑म् प॒शवो॒ वै वै प॒शव॒ इन्द्र᳚म् । \newline
48. प॒शव॒ इन्द्र॒ मिन्द्र॑म् प॒शवः॑ प॒शव॒ इन्द्र॑ मे॒वैवे न्द्र॑म् प॒शवः॑ प॒शव॒ इन्द्र॑ मे॒व । \newline
49. इन्द्र॑ मे॒वैवे न्द्र॒ मिन्द्र॑ मे॒वे न्द्रि॒याव॑न्त मिन्द्रि॒याव॑न्त मे॒वे न्द्र॒ मिन्द्र॑ मे॒वे न्द्रि॒याव॑न्तम् । \newline
50. ए॒वे न्द्रि॒याव॑न्त मिन्द्रि॒याव॑न्त मे॒वैवे न्द्रि॒याव॑न्तꣳ॒॒ स्वेन॒ स्वेने᳚ न्द्रि॒याव॑न्त मे॒वैवे न्द्रि॒याव॑न्तꣳ॒॒ स्वेन॑ । \newline
51. इ॒न्द्रि॒याव॑न्तꣳ॒॒ स्वेन॒ स्वेने᳚ न्द्रि॒याव॑न्त मिन्द्रि॒याव॑न्तꣳ॒॒ स्वेन॑ भाग॒धेये॑न भाग॒धेये॑न॒ स्वेने᳚ न्द्रि॒याव॑न्त मिन्द्रि॒याव॑न्तꣳ॒॒ स्वेन॑ भाग॒धेये॑न । \newline
52. इ॒न्द्रि॒याव॑न्त॒मिती᳚न्द्रि॒य - व॒न्त॒म् । \newline
53. स्वेन॑ भाग॒धेये॑न भाग॒धेये॑न॒ स्वेन॒ स्वेन॑ भाग॒धेये॒नोपोप॑ भाग॒धेये॑न॒ स्वेन॒ स्वेन॑ भाग॒धेये॒नोप॑ । \newline
54. भा॒ग॒धेये॒नोपोप॑ भाग॒धेये॑न भाग॒धेये॒नोप॑ धावति धाव॒ त्युप॑ भाग॒धेये॑न भाग॒धेये॒नोप॑ धावति । \newline
55. भा॒ग॒धेये॒नेति॑ भाग - धेये॑न । \newline
56. उप॑ धावति धाव॒ त्युपोप॑ धावति॒ स स धा॑व॒ त्युपोप॑ धावति॒ सः । \newline
57. धा॒व॒ति॒ स स धा॑वति धावति॒ स ए॒वैव स धा॑वति धावति॒ स ए॒व । \newline
58. स ए॒वैव स स ए॒वास्मा॑ अस्मा ए॒व स स ए॒वास्मै᳚ । \newline
\pagebreak
\markright{ TS 2.2.7.2  \hfill https://www.vedavms.in \hfill}
\addcontentsline{toc}{section}{ TS 2.2.7.2 }
\section*{ TS 2.2.7.2 }

\textbf{TS 2.2.7.2 } \newline
\textbf{Samhita Paata} \newline

ए॒वास्मा॑ इन्द्रि॒यं प॒शून् प्रय॑च्छति पशु॒माने॒व भ॑व॒तीन्द्रा॑य-घ॒र्मव॑ते पुरो॒डाश॒मेका॑दशकपालं॒ निर्व॑पेद्-ब्रह्मवर्च॒सका॑मो ब्रह्मवर्च॒सं ॅवै घ॒र्म इन्द्र॑मे॒व घ॒र्मव॑न्तꣳ॒॒ स्वेन॑ भाग॒धेये॒नोप॑ धावति॒ स ए॒वास्मि॑न् ब्रह्मवर्च॒सं द॑धाति ब्रह्मवर्च॒स्ये॑व भ॑व॒तीन्द्रा॑या॒र्कव॑ते पुरो॒डाश॒मेका॑दशकपालं॒ निर्व॑पे॒दन्न॑कामो॒ऽर्को वै दे॒वाना॒मन्न॒मिन्द्र॑मे॒वार्कव॑न्तꣳ॒॒ स्वेन॑ भाग॒धेये॒नो - [  ] \newline

\textbf{Pada Paata} \newline

ए॒व । अ॒स्मै॒ । इ॒न्द्रि॒यम् । प॒शून् । प्रेति॑ । य॒च्छ॒ति॒ । प॒शु॒मानिति॑ पशु - मान् । ए॒व । भ॒व॒ति॒ । इन्द्रा॑य । घ॒र्मव॑त॒ इति॑ घ॒र्म - व॒ते॒ । पु॒रो॒डाश᳚म् । एका॑दशकपाल॒मित्येका॑दश - क॒पा॒ल॒म् । निरिति॑ । व॒पे॒त् । ब्र॒ह्म॒व॒र्च॒सका॑म॒ इति॑ ब्रह्मवर्च॒स - का॒मः॒ । ब्र॒ह्म॒व॒र्च॒समिति॑ ब्रह्म - व॒र्च॒सम् । वै । घ॒र्मः । इन्द्र᳚म् । ए॒व । घ॒र्मव॑न्त॒मिति॑ घ॒र्म - व॒न्त॒म् । स्वेन॑ । भा॒ग॒धेये॒नेति॑ भाग - धेये॑न । उपेति॑ । धा॒व॒ति॒ । सः । ए॒व । अ॒स्मि॒न्न् । ब्र॒ह्म॒व॒र्च॒समिति॑ ब्रह्म - व॒र्च॒सम् । द॒धा॒ति॒ । ब्र॒ह्म॒व॒र्च॒सीति॑ ब्रह्म - व॒र्च॒सी । ए॒व । भ॒व॒ति॒ । इन्द्रा॑य । अ॒र्कव॑त॒ इत्य॒र्क - व॒ते॒ । पु॒रो॒डाश᳚म् । एका॑दशकपाल॒मित्येका॑दश - क॒पा॒ल॒म् । निरिति॑ । व॒पे॒त् । अन्न॑काम॒ इत्यन्न॑ - का॒मः॒ । अ॒र्कः । वै । दे॒वाना᳚म् । अन्न᳚म् । इन्द्र᳚म् । ए॒व । अ॒र्कव॑न्त॒मित्य॒र्क - व॒न्त॒म् । स्वेन॑ । भा॒ग॒धेये॒नेति॑ भाग - धेये॑न ।  \newline


\textbf{Krama Paata} \newline

ए॒वास्मै᳚ । अ॒स्मा॒ इ॒न्द्रि॒यम् । इ॒न्द्रि॒यम् प॒शून् । प॒शून् प्र । प्र य॑च्छति । य॒च्छ॒ति॒ प॒शु॒मान् । प॒शु॒माने॒व । प॒शु॒मानिति॑ पशु - मान् । ए॒व भ॑वति । भ॒व॒तीन्द्रा॑य । इन्द्रा॑य घ॒र्मव॑ते । घ॒र्मव॑ते पुरो॒डाश᳚म् । घ॒र्मव॑त॒ इति॑ घ॒र्म - व॒ते॒ । पु॒रो॒डाश॒मेका॑दशकपालम् । एका॑दशकपाल॒म् निः । एका॑दशकपाल॒मित्येका॑दश - क॒पा॒ल॒म् । निर् व॑पेत् । व॒पे॒द्,ब्र॒ह्म॒व॒र्च॒सका॑मः । ब्र॒ह्म॒व॒र्च॒सका॑मो ब्रह्मवर्च॒सम् । ब्र॒ह्म॒व॒र्च॒सका॑म॒ इति॑ ब्रह्मवर्च॒स - का॒मः॒ । ब्र॒ह्म॒व॒र्च॒सं ॅवै । ब्र॒ह्म॒व॒र्च॒समिति॑ ब्रह्म - व॒र्च॒सम् । वै घ॒र्मः । घ॒र्म इन्द्र᳚म् । इन्द्र॑मे॒व । ए॒व घ॒र्मव॑न्तम् । घ॒र्मव॑न्तꣳ॒॒ स्वेन॑ । घ॒र्मव॑न्त॒मिति॑ घ॒र्म - व॒न्त॒म् । स्वेन॑ भाग॒धेये॑न । भा॒ग॒धेये॒नोप॑ । भा॒ग॒धेये॒नेति॑ भाग - धेये॑न । उप॑ धावति । धा॒व॒ति॒ सः । स ए॒व । ए॒वास्मिन्न्॑ । अ॒स्मि॒न् ब्र॒ह्म॒व॒र्च॒सम् । ब्र॒ह्म॒व॒र्च॒सम् द॑धाति । ब्र॒ह्म॒व॒र्च॒समिति॑ ब्रह्म - व॒र्च॒सम् । द॒धा॒ति॒ ब्र॒ह्म॒व॒र्च॒सी । ब्र॒ह्म॒व॒र्च॒स्ये॑व । ब्र॒ह्म॒व॒र्च॒सीति॑ ब्रह्म - व॒र्च॒सी । ए॒व भ॑वति । भ॒व॒तीन्द्रा॑य । इन्द्रा॑या॒र्कव॑ते । अ॒र्कव॑ते पुरो॒डाश᳚म् । अ॒र्कव॑त॒ इत्य॒र्क - व॒ते॒ । पु॒रो॒डाश॒मेका॑दशकपालम् । एका॑दशपाल॒म् निः । एका॑दशकपाल॒मित्येका॑दश - क॒पा॒ल॒म् । निर् व॑पेत् । व॒पे॒दन्न॑कामः । अन्न॑कामो॒ ऽर्कः । अन्न॑काम॒ इत्यन्न॑ - का॒मः॒ । अ॒र्को वै । वै दे॒वाना᳚म् । दे॒वाना॒मन्न᳚म् । अन्न॒मिन्द्र᳚म् । इन्द्र॑मे॒व । ए॒वार्कव॑न्तम् । अ॒र्कव॑न्तꣳ॒॒ स्वेन॑ । अ॒र्कव॑न्त॒मित्य॒र्क - व॒न्त॒म् । स्वेन॑ भाग॒धेये॑न । भा॒ग॒धेये॒नोप॑ । भा॒ग॒धेये॒नेति॑ भाग - धेये॑न \newline

\textbf{Jatai Paata} \newline

1. ए॒वास्मा॑ अस्मा ए॒वैवास्मै᳚ । \newline
2. अ॒स्मा॒ इ॒न्द्रि॒य मि॑न्द्रि॒य म॑स्मा अस्मा इन्द्रि॒यम् । \newline
3. इ॒न्द्रि॒यम् प॒शून् प॒शू नि॑न्द्रि॒य मि॑न्द्रि॒यम् प॒शून् । \newline
4. प॒शून् प्र प्र प॒शून् प॒शून् प्र । \newline
5. प्र य॑च्छति यच्छति॒ प्र प्र य॑च्छति । \newline
6. य॒च्छ॒ति॒ प॒शु॒मान् प॑शु॒मान्. य॑च्छति यच्छति पशु॒मान् । \newline
7. प॒शु॒मा ने॒वैव प॑शु॒मान् प॑शु॒मा ने॒व । \newline
8. प॒शु॒मानिति॑ पशु - मान् । \newline
9. ए॒व भ॑वति भव त्ये॒वैव भ॑वति । \newline
10. भ॒व॒तीन्द्रा॒ये न्द्रा॑य भवति भव॒तीन्द्रा॑य । \newline
11. इन्द्रा॑य घ॒र्मव॑ते घ॒र्मव॑त॒ इन्द्रा॒ये न्द्रा॑य घ॒र्मव॑ते । \newline
12. घ॒र्मव॑ते पुरो॒डाश॑म् पुरो॒डाश॑म् घ॒र्मव॑ते घ॒र्मव॑ते पुरो॒डाश᳚म् । \newline
13. घ॒र्मव॑त॒ इति॑ घ॒र्म - व॒ते॒ । \newline
14. पु॒रो॒डाश॒ मेका॑दशकपाल॒ मेका॑दशकपालम् पुरो॒डाश॑म् पुरो॒डाश॒ मेका॑दशकपालम् । \newline
15. एका॑दशकपाल॒म् निर् णिरेका॑दशकपाल॒ मेका॑दशकपाल॒म् निः । \newline
16. एका॑दशकपाल॒मित्येका॑दश - क॒पा॒ल॒म् । \newline
17. निर् व॑पेद् वपे॒न् निर् णिर् व॑पेत् । \newline
18. व॒पे॒द् ब्र॒ह्म॒व॒र्च॒सका॑मो ब्रह्मवर्च॒सका॑मो वपेद् वपेद् ब्रह्मवर्च॒सका॑मः । \newline
19. ब्र॒ह्म॒व॒र्च॒सका॑मो ब्रह्मवर्च॒सम् ब्र॑ह्मवर्च॒सम् ब्र॑ह्मवर्च॒सका॑मो ब्रह्मवर्च॒सका॑मो ब्रह्मवर्च॒सम् । \newline
20. ब्र॒ह्म॒व॒र्च॒सका॑म॒ इति॑ ब्रह्मवर्च॒स - का॒मः॒ । \newline
21. ब्र॒ह्म॒व॒र्च॒सं ॅवै वै ब्र॑ह्मवर्च॒सम् ब्र॑ह्मवर्च॒सं ॅवै । \newline
22. ब्र॒ह्म॒व॒र्च॒समिति॑ ब्रह्म - व॒र्च॒सम् । \newline
23. वै घ॒र्मो घ॒र्मो वै वै घ॒र्मः । \newline
24. घ॒र्म इन्द्र॒ मिन्द्र॑म् घ॒र्मो घ॒र्म इन्द्र᳚म् । \newline
25. इन्द्र॑ मे॒वैवे न्द्र॒ मिन्द्र॑ मे॒व । \newline
26. ए॒व घ॒र्मव॑न्तम् घ॒र्मव॑न्त मे॒वैव घ॒र्मव॑न्तम् । \newline
27. घ॒र्मव॑न्तꣳ॒॒ स्वेन॒ स्वेन॑ घ॒र्मव॑न्तम् घ॒र्मव॑न्तꣳ॒॒ स्वेन॑ । \newline
28. घ॒र्मव॑न्त॒मिति॑ घ॒र्म - व॒न्त॒म् । \newline
29. स्वेन॑ भाग॒धेये॑न भाग॒धेये॑न॒ स्वेन॒ स्वेन॑ भाग॒धेये॑न । \newline
30. भा॒ग॒धेये॒नोपोप॑ भाग॒धेये॑न भाग॒धेये॒नोप॑ । \newline
31. भा॒ग॒धेये॒नेति॑ भाग - धेये॑न । \newline
32. उप॑ धावति धाव॒ त्युपोप॑ धावति । \newline
33. धा॒व॒ति॒ स स धा॑वति धावति॒ सः । \newline
34. स ए॒वैव स स ए॒व । \newline
35. ए॒वास्मि॑न् नस्मिन् ने॒वैवास्मिन्न्॑ । \newline
36. अ॒स्मि॒न् ब्र॒ह्म॒व॒र्च॒सम् ब्र॑ह्मवर्च॒स म॑स्मिन् नस्मिन् ब्रह्मवर्च॒सम् । \newline
37. ब्र॒ह्म॒व॒र्च॒सम् द॑धाति दधाति ब्रह्मवर्च॒सम् ब्र॑ह्मवर्च॒सम् द॑धाति । \newline
38. ब्र॒ह्म॒व॒र्च॒समिति॑ ब्रह्म - व॒र्च॒सम् । \newline
39. द॒धा॒ति॒ ब्र॒ह्म॒व॒र्च॒सी ब्र॑ह्मवर्च॒सी द॑धाति दधाति ब्रह्मवर्च॒सी । \newline
40. ब्र॒ह्म॒व॒र्च॒ स्ये॑वैव ब्र॑ह्मवर्च॒सी ब्र॑ह्मवर्च॒ स्ये॑व । \newline
41. ब्र॒ह्म॒व॒र्च॒सीति॑ ब्रह्म - व॒र्च॒सी । \newline
42. ए॒व भ॑वति भव त्ये॒वैव भ॑वति । \newline
43. भ॒व॒तीन्द्रा॒ये न्द्रा॑य भवति भव॒तीन्द्रा॑य । \newline
44. इन्द्रा॑या॒ र्कव॑ते॒ ऽर्कव॑त॒ इन्द्रा॒ये न्द्रा॑या॒ र्कव॑ते । \newline
45. अ॒र्कव॑ते पुरो॒डाश॑म् पुरो॒डाश॑ म॒र्कव॑ते॒ ऽर्कव॑ते पुरो॒डाश᳚म् । \newline
46. अ॒र्कव॑त॒ इत्य॒र्क - व॒ते॒ । \newline
47. पु॒रो॒डाश॒ मेका॑दशकपाल॒ मेका॑दशकपालम् पुरो॒डाश॑म् पुरो॒डाश॒ मेका॑दशकपालम् । \newline
48. एका॑दशकपाल॒म् निर् णिरेका॑दशकपाल॒ मेका॑दशकपाल॒म् निः । \newline
49. एका॑दशकपाल॒मित्येका॑दश - क॒पा॒ल॒म् । \newline
50. निर् व॑पेद् वपे॒न् निर् णिर् व॑पेत् । \newline
51. व॒पे॒ दन्न॑का॒मो ऽन्न॑कामो वपेद् वपे॒ दन्न॑कामः । \newline
52. अन्न॑कामो॒ ऽर्को᳚ ऽर्को ऽन्न॑का॒मो ऽन्न॑कामो॒ ऽर्कः । \newline
53. अन्न॑काम॒ इत्यन्न॑ - का॒मः॒ । \newline
54. अ॒र्को वै वा अ॒र्को᳚ ऽर्को वै । \newline
55. वै दे॒वाना᳚म् दे॒वानां॒ ॅवै वै दे॒वाना᳚म् । \newline
56. दे॒वाना॒ मन्न॒ मन्न॑म् दे॒वाना᳚म् दे॒वाना॒ मन्न᳚म् । \newline
57. अन्न॒ मिन्द्र॒ मिन्द्र॒ मन्न॒ मन्न॒ मिन्द्र᳚म् । \newline
58. इन्द्र॑ मे॒वैवे न्द्र॒ मिन्द्र॑ मे॒व । \newline
59. ए॒वार्कव॑न्त म॒र्कव॑न्त मे॒वै वार्कव॑न्तम् । \newline
60. अ॒र्कव॑न्तꣳ॒॒ स्वेन॒ स्वेना॒ र्कव॑न्त म॒र्कव॑न्तꣳ॒॒ स्वेन॑ । \newline
61. अ॒र्कव॑न्त॒मित्य॒र्क - व॒न्त॒म् । \newline
62. स्वेन॑ भाग॒धेये॑न भाग॒धेये॑न॒ स्वेन॒ स्वेन॑ भाग॒धेये॑न । \newline
63. भा॒ग॒धेये॒नोपोप॑ भाग॒धेये॑न भाग॒धेये॒नोप॑ । \newline
64. भा॒ग॒धेये॒नेति॑ भाग - धेये॑न । \newline

\textbf{Ghana Paata } \newline

1. ए॒वास्मा॑ अस्मा ए॒वैवास्मा॑ इन्द्रि॒य मि॑न्द्रि॒य म॑स्मा ए॒वैवास्मा॑ इन्द्रि॒यम् । \newline
2. अ॒स्मा॒ इ॒न्द्रि॒य मि॑न्द्रि॒य म॑स्मा अस्मा इन्द्रि॒यम् प॒शून् प॒शू नि॑न्द्रि॒य म॑स्मा अस्मा इन्द्रि॒यम् प॒शून् । \newline
3. इ॒न्द्रि॒यम् प॒शून् प॒शू नि॑न्द्रि॒य मि॑न्द्रि॒यम् प॒शून् प्र प्र प॒शू नि॑न्द्रि॒य मि॑न्द्रि॒यम् प॒शून् प्र । \newline
4. प॒शून् प्र प्र प॒शून् प॒शून् प्र य॑च्छति यच्छति॒ प्र प॒शून् प॒शून् प्र य॑च्छति । \newline
5. प्र य॑च्छति यच्छति॒ प्र प्र य॑च्छति पशु॒मान् प॑शु॒मान्. य॑च्छति॒ प्र प्र य॑च्छति पशु॒मान् । \newline
6. य॒च्छ॒ति॒ प॒शु॒मान् प॑शु॒मान्. य॑च्छति यच्छति पशु॒मा ने॒वैव प॑शु॒मान्. य॑च्छति यच्छति पशु॒मा ने॒व । \newline
7. प॒शु॒मा ने॒वैव प॑शु॒मान् प॑शु॒मा ने॒व भ॑वति भवत्ये॒व प॑शु॒मान् प॑शु॒मा ने॒व भ॑वति । \newline
8. प॒शु॒मानिति॑ पशु - मान् । \newline
9. ए॒व भ॑वति भव त्ये॒वैव भ॑व॒तीन्द्रा॒ये न्द्रा॑य भव त्ये॒वैव भ॑व॒तीन्द्रा॑य । \newline
10. भ॒व॒तीन्द्रा॒ये न्द्रा॑य भवति भव॒तीन्द्रा॑य घ॒र्मव॑ते घ॒र्मव॑त॒ इन्द्रा॑य भवति भव॒तीन्द्रा॑य घ॒र्मव॑ते । \newline
11. इन्द्रा॑य घ॒र्मव॑ते घ॒र्मव॑त॒ इन्द्रा॒ये न्द्रा॑य घ॒र्मव॑ते पुरो॒डाश॑म् पुरो॒डाश॑म् घ॒र्मव॑त॒ इन्द्रा॒ये न्द्रा॑य घ॒र्मव॑ते पुरो॒डाश᳚म् । \newline
12. घ॒र्मव॑ते पुरो॒डाश॑म् पुरो॒डाश॑म् घ॒र्मव॑ते घ॒र्मव॑ते पुरो॒डाश॒ मेका॑दशकपाल॒ मेका॑दशकपालम् पुरो॒डाश॑म् घ॒र्मव॑ते घ॒र्मव॑ते पुरो॒डाश॒ मेका॑दशकपालम् । \newline
13. घ॒र्मव॑त॒ इति॑ घ॒र्म - व॒ते॒ । \newline
14. पु॒रो॒डाश॒ मेका॑दशकपाल॒ मेका॑दशकपालम् पुरो॒डाश॑म् पुरो॒डाश॒ मेका॑दशकपाल॒म् निर् णिरेका॑दशकपालम् पुरो॒डाश॑म् पुरो॒डाश॒ मेका॑दशकपाल॒म् निः । \newline
15. एका॑दशकपाल॒म् निर् णिरेका॑दशकपाल॒ मेका॑दशकपाल॒म् निर् व॑पेद् वपे॒न् निरेका॑दशकपाल॒ मेका॑दशकपाल॒म् निर् व॑पेत् । \newline
16. एका॑दशकपाल॒मित्येका॑दश - क॒पा॒ल॒म् । \newline
17. निर् व॑पेद् वपे॒न् निर् णिर् व॑पेद् ब्रह्मवर्च॒सका॑मो ब्रह्मवर्च॒सका॑मो वपे॒न् निर् णिर् व॑पेद् ब्रह्मवर्च॒सका॑मः । \newline
18. व॒पे॒द् ब्र॒ह्म॒व॒र्च॒सका॑मो ब्रह्मवर्च॒सका॑मो वपेद् वपेद् ब्रह्मवर्च॒सका॑मो ब्रह्मवर्च॒सम् ब्र॑ह्मवर्च॒सम् ब्र॑ह्मवर्च॒सका॑मो वपेद् वपेद् ब्रह्मवर्च॒सका॑मो ब्रह्मवर्च॒सम् । \newline
19. ब्र॒ह्म॒व॒र्च॒सका॑मो ब्रह्मवर्च॒सम् ब्र॑ह्मवर्च॒सम् ब्र॑ह्मवर्च॒सका॑मो ब्रह्मवर्च॒सका॑मो ब्रह्मवर्च॒सं ॅवै वै ब्र॑ह्मवर्च॒सम् ब्र॑ह्मवर्च॒सका॑मो ब्रह्मवर्च॒सका॑मो ब्रह्मवर्च॒सं ॅवै । \newline
20. ब्र॒ह्म॒व॒र्च॒सका॑म॒ इति॑ ब्रह्मवर्च॒स - का॒मः॒ । \newline
21. ब्र॒ह्म॒व॒र्च॒सं ॅवै वै ब्र॑ह्मवर्च॒सम् ब्र॑ह्मवर्च॒सं ॅवै घ॒र्मो घ॒र्मो वै ब्र॑ह्मवर्च॒सम् ब्र॑ह्मवर्च॒सं ॅवै घ॒र्मः । \newline
22. ब्र॒ह्म॒व॒र्च॒समिति॑ ब्रह्म - व॒र्च॒सम् । \newline
23. वै घ॒र्मो घ॒र्मो वै वै घ॒र्म इन्द्र॒ मिन्द्र॑म् घ॒र्मो वै वै घ॒र्म इन्द्र᳚म् । \newline
24. घ॒र्म इन्द्र॒ मिन्द्र॑म् घ॒र्मो घ॒र्म इन्द्र॑ मे॒वैवे न्द्र॑म् घ॒र्मो घ॒र्म इन्द्र॑ मे॒व । \newline
25. इन्द्र॑ मे॒वैवे न्द्र॒ मिन्द्र॑ मे॒व घ॒र्मव॑न्तम् घ॒र्मव॑न्त मे॒वे न्द्र॒ मिन्द्र॑ मे॒व घ॒र्मव॑न्तम् । \newline
26. ए॒व घ॒र्मव॑न्तम् घ॒र्मव॑न्त मे॒वैव घ॒र्मव॑न्तꣳ॒॒ स्वेन॒ स्वेन॑ घ॒र्मव॑न्त मे॒वैव घ॒र्मव॑न्तꣳ॒॒ स्वेन॑ । \newline
27. घ॒र्मव॑न्तꣳ॒॒ स्वेन॒ स्वेन॑ घ॒र्मव॑न्तम् घ॒र्मव॑न्तꣳ॒॒ स्वेन॑ भाग॒धेये॑न भाग॒धेये॑न॒ स्वेन॑ घ॒र्मव॑न्तम् घ॒र्मव॑न्तꣳ॒॒ स्वेन॑ भाग॒धेये॑न । \newline
28. घ॒र्मव॑न्त॒मिति॑ घ॒र्म - व॒न्त॒म् । \newline
29. स्वेन॑ भाग॒धेये॑न भाग॒धेये॑न॒ स्वेन॒ स्वेन॑ भाग॒धेये॒नोपोप॑ भाग॒धेये॑न॒ स्वेन॒ स्वेन॑ भाग॒धेये॒नोप॑ । \newline
30. भा॒ग॒धेये॒नोपोप॑ भाग॒धेये॑न भाग॒धेये॒नोप॑ धावति धाव॒त्युप॑ भाग॒धेये॑न भाग॒धेये॒नोप॑ धावति । \newline
31. भा॒ग॒धेये॒नेति॑ भाग - धेये॑न । \newline
32. उप॑ धावति धाव॒ त्युपोप॑ धावति॒ स स धा॑व॒ त्युपोप॑ धावति॒ सः । \newline
33. धा॒व॒ति॒ स स धा॑वति धावति॒ स ए॒वैव स धा॑वति धावति॒ स ए॒व । \newline
34. स ए॒वैव स स ए॒वास्मि॑न् नस्मिन् ने॒व स स ए॒वास्मिन्न्॑ । \newline
35. ए॒वास्मि॑न् नस्मिन् ने॒वैवास्मि॑न् ब्रह्मवर्च॒सम् ब्र॑ह्मवर्च॒स म॑स्मिन् ने॒वैवास्मि॑न् ब्रह्मवर्च॒सम् । \newline
36. अ॒स्मि॒न् ब्र॒ह्म॒व॒र्च॒सम् ब्र॑ह्मवर्च॒स म॑स्मिन् नस्मिन् ब्रह्मवर्च॒सम् द॑धाति दधाति ब्रह्मवर्च॒स म॑स्मिन् नस्मिन् ब्रह्मवर्च॒सम् द॑धाति । \newline
37. ब्र॒ह्म॒व॒र्च॒सम् द॑धाति दधाति ब्रह्मवर्च॒सम् ब्र॑ह्मवर्च॒सम् द॑धाति ब्रह्मवर्च॒सी ब्र॑ह्मवर्च॒सी द॑धाति ब्रह्मवर्च॒सम् ब्र॑ह्मवर्च॒सम् द॑धाति ब्रह्मवर्च॒सी । \newline
38. ब्र॒ह्म॒व॒र्च॒समिति॑ ब्रह्म - व॒र्च॒सम् । \newline
39. द॒धा॒ति॒ ब्र॒ह्म॒व॒र्च॒सी ब्र॑ह्मवर्च॒सी द॑धाति दधाति ब्रह्मवर्च॒ स्ये॑वैव ब्र॑ह्मवर्च॒सी द॑धाति दधाति ब्रह्मवर्च॒ स्ये॑व । \newline
40. ब्र॒ह्म॒व॒र्च॒ स्ये॑वैव ब्र॑ह्मवर्च॒सी ब्र॑ह्मवर्च॒ स्ये॑व भ॑वति भवत्ये॒व ब्र॑ह्मवर्च॒सी ब्र॑ह्मवर्च॒ स्ये॑व भ॑वति । \newline
41. ब्र॒ह्म॒व॒र्च॒सीति॑ ब्रह्म - व॒र्च॒सी । \newline
42. ए॒व भ॑वति भव त्ये॒वैव भ॑व॒तीन्द्रा॒ये न्द्रा॑य भव त्ये॒वैव भ॑व॒तीन्द्रा॑य । \newline
43. भ॒व॒ तीन्द्रा॒ये न्द्रा॑य भवति भव॒ तीन्द्रा॑या॒र्कव॑ते॒ ऽर्कव॑त॒ इन्द्रा॑य भवति भव॒ तीन्द्रा॑या॒र्कव॑ते । \newline
44. इन्द्रा॑या॒र्कव॑ते॒ ऽर्कव॑त॒ इन्द्रा॒ये न्द्रा॑या॒र्कव॑ते पुरो॒डाश॑म् पुरो॒डाश॑ म॒र्कव॑त॒ इन्द्रा॒ये न्द्रा॑या॒र्कव॑ते पुरो॒डाश᳚म् । \newline
45. अ॒र्कव॑ते पुरो॒डाश॑म् पुरो॒डाश॑ म॒र्कव॑ते॒ ऽर्कव॑ते पुरो॒डाश॒ मेका॑दशकपाल॒ मेका॑दशकपालम् पुरो॒डाश॑ म॒र्कव॑ते॒ ऽर्कव॑ते पुरो॒डाश॒ मेका॑दशकपालम् । \newline
46. अ॒र्कव॑त॒ इत्य॒र्क - व॒ते॒ । \newline
47. पु॒रो॒डाश॒ मेका॑दशकपाल॒ मेका॑दशकपालम् पुरो॒डाश॑म् पुरो॒डाश॒ मेका॑दशकपाल॒म् निर् णिरेका॑दशकपालम् पुरो॒डाश॑म् पुरो॒डाश॒ मेका॑दशकपाल॒म् निः । \newline
48. एका॑दशकपाल॒म् निर् णिरेका॑दशकपाल॒ मेका॑दशकपाल॒म् निर् व॑पेद् वपे॒न् निरेका॑दशकपाल॒ मेका॑दशकपाल॒म् निर् व॑पेत् । \newline
49. एका॑दशकपाल॒मित्येका॑दश - क॒पा॒ल॒म् । \newline
50. निर् व॑पेद् वपे॒न् निर् णिर् व॑पे॒ दन्न॑का॒मो ऽन्न॑कामो वपे॒न् निर् णिर् व॑पे॒ दन्न॑कामः । \newline
51. व॒पे॒ दन्न॑का॒मो ऽन्न॑कामो वपेद् वपे॒ दन्न॑कामो॒ ऽर्को᳚ ऽर्को ऽन्न॑कामो वपेद् वपे॒ दन्न॑कामो॒ ऽर्कः । \newline
52. अन्न॑कामो॒ ऽर्को᳚ ऽर्को ऽन्न॑का॒मो ऽन्न॑कामो॒ ऽर्को वै वा अ॒र्को ऽन्न॑का॒मो ऽन्न॑कामो॒ ऽर्को वै । \newline
53. अन्न॑काम॒ इत्यन्न॑ - का॒मः॒ । \newline
54. अ॒र्को वै वा अ॒र्को᳚ ऽर्को वै दे॒वाना᳚म् दे॒वानां॒ ॅवा अ॒र्को᳚ ऽर्को वै दे॒वाना᳚म् । \newline
55. वै दे॒वाना᳚म् दे॒वानां॒ ॅवै वै दे॒वाना॒ मन्न॒ मन्न॑म् दे॒वानां॒ ॅवै वै दे॒वाना॒ मन्न᳚म् । \newline
56. दे॒वाना॒ मन्न॒ मन्न॑म् दे॒वाना᳚म् दे॒वाना॒ मन्न॒ मिन्द्र॒ मिन्द्र॒ मन्न॑म् दे॒वाना᳚म् दे॒वाना॒ मन्न॒ मिन्द्र᳚म् । \newline
57. अन्न॒ मिन्द्र॒ मिन्द्र॒ मन्न॒ मन्न॒ मिन्द्र॑ मे॒वैवे न्द्र॒ मन्न॒ मन्न॒ मिन्द्र॑ मे॒व । \newline
58. इन्द्र॑ मे॒वैवे न्द्र॒ मिन्द्र॑ मे॒वार्कव॑न्त म॒र्कव॑न्त मे॒वे न्द्र॒ मिन्द्र॑ मे॒वार्कव॑न्तम् । \newline
59. ए॒वार्कव॑न्त म॒र्कव॑न्त मे॒वैवार्कव॑न्तꣳ॒॒ स्वेन॒ स्वेना॒र्कव॑न्त मे॒वैवार्कव॑न्तꣳ॒॒ स्वेन॑ । \newline
60. अ॒र्कव॑न्तꣳ॒॒ स्वेन॒ स्वेना॒र्कव॑न्त म॒र्कव॑न्तꣳ॒॒ स्वेन॑ भाग॒धेये॑न भाग॒धेये॑न॒ स्वेना॒र्कव॑न्त म॒र्कव॑न्तꣳ॒॒ स्वेन॑ भाग॒धेये॑न । \newline
61. अ॒र्कव॑न्त॒मित्य॒र्क - व॒न्त॒म् । \newline
62. स्वेन॑ भाग॒धेये॑न भाग॒धेये॑न॒ स्वेन॒ स्वेन॑ भाग॒धेये॒नोपोप॑ भाग॒धेये॑न॒ स्वेन॒ स्वेन॑ भाग॒धेये॒नोप॑ । \newline
63. भा॒ग॒धेये॒नोपोप॑ भाग॒धेये॑न भाग॒धेये॒नोप॑ धावति धाव॒ त्युप॑ भाग॒धेये॑न भाग॒धेये॒नोप॑ धावति । \newline
64. भा॒ग॒धेये॒नेति॑ भाग - धेये॑न । \newline
\pagebreak
\markright{ TS 2.2.7.3  \hfill https://www.vedavms.in \hfill}
\addcontentsline{toc}{section}{ TS 2.2.7.3 }
\section*{ TS 2.2.7.3 }

\textbf{TS 2.2.7.3 } \newline
\textbf{Samhita Paata} \newline

-प॑ धावति॒ स ए॒वास्मा॒ अन्नं॒ प्रय॑च्छत्यन्ना॒द ए॒व भ॑व॒तीन्द्रा॑य घ॒र्मव॑ते पुरो॒डाश॒मेका॑दशकपालं॒ निर्व॑पे॒दिन्द्रा॑ये-न्द्रि॒याव॑त॒ इन्द्रा॑या॒ऽर्कव॑ते॒ भूति॑कामो॒ यदिन्द्रा॑य घ॒र्मव॑ते नि॒र्वप॑ति॒ शिर॑ ए॒वास्य॒ तेन॑ करोति॒ यदिन्द्रा॑येन्द्रि॒याव॑त आ॒त्मान॑मे॒वास्य॒ तेन॑ करोति॒यदिन्द्रा॑या॒ऽर्कव॑ते भू॒त ए॒वान्नाद्ये॒ प्रति॑तिष्ठति॒ भव॑त्ये॒वेन्द्रा॑या - [  ] \newline

\textbf{Pada Paata} \newline

उपेति॑ । धा॒व॒ति॒ । सः । ए॒व । अ॒स्मै॒ । अन्न᳚म् । प्रेति॑ । य॒च्छ॒ति॒ । अ॒न्ना॒द इत्य॑न्न - अ॒दः । ए॒व । भ॒व॒ति॒ । इन्द्रा॑य । घ॒र्मव॑त॒ इति॑ घ॒र्म - व॒ते॒ । पु॒रो॒डाश᳚म् । एका॑दशकपाल॒मित्येका॑दश - क॒पा॒ल॒म् । निरिति॑ । व॒पे॒त् । इन्द्रा॑य । इ॒न्द्रि॒याव॑त॒ इती᳚न्द्रि॒य - व॒ते॒ । इन्द्रा॑य । अ॒र्कव॑त॒ इत्य॒र्क - व॒ते॒ । भूति॑काम॒ इति॒ भूति॑ - का॒मः॒ । यत् । इन्द्रा॑य । घ॒र्मव॑त॒ इति॑ घ॒र्म - व॒ते॒ । नि॒र्वप॒तीति॑ निः-वप॑ति । शिरः॑ । ए॒व । अ॒स्य॒ । तेन॑ । क॒रो॒ति॒ । यत् । इन्द्रा॑य । इ॒न्द्रि॒याव॑त॒ इती᳚न्द्रि॒य - व॒ते॒ । आ॒त्मान᳚म् । ए॒व । अ॒स्य॒ । तेन॑ । क॒रो॒ति॒ । यत् । इन्द्रा॑य । अ॒र्कव॑त॒ इत्य॒र्क - व॒ते॒ । भू॒तः । ए॒व । अ॒न्नाद्य॒ इत्य॑न्न - अद्ये᳚ । प्रतीति॑ । ति॒ष्ठ॒ति॒ । भव॑ति । ए॒व । इन्द्रा॑य ।  \newline


\textbf{Krama Paata} \newline

उप॑ धावति । धा॒व॒ति॒ सः । स ए॒व । ए॒वास्मै᳚ । अ॒स्मा॒ अन्न᳚म् । अन्न॒म् प्र । प्र य॑च्छति । य॒च्छ॒त्य॒न्ना॒दः । अ॒न्ना॒द ए॒व । अ॒न्ना॒द इत्य॑न्न - अ॒दः । ए॒व भ॑वति । भ॒व॒तीन्द्रा॑य । इन्द्रा॑य घ॒र्मव॑ते । घ॒र्मव॑ते पुरो॒डाश᳚म् । घ॒र्मव॑त॒ इति॑ घ॒र्म - व॒ते॒ । पु॒रो॒डाश॒मेका॑दशकपालम् । एका॑दशकपाल॒म् निः । एका॑दशकपाल॒मित्येका॑दश - क॒पा॒ल॒म् । निर् व॑पेत् । व॒पे॒दिन्द्रा॑य । इन्द्रा॑येन्द्रि॒याव॑ते । इ॒न्द्रि॒याव॑त॒ इन्द्रा॑य । इ॒न्द्रि॒याव॑त॒ इती᳚न्द्रि॒य - व॒ते॒ । इन्द्रा॑या॒र्कव॑ते । अ॒र्कव॑ते॒ भूति॑कामः । अ॒र्कव॑त॒ इत्य॒र्क - व॒ते॒ । भूति॑कामो॒ यत् । भूति॑काम॒ इति॒ भूति॑ - का॒मः॒ । यदिन्द्रा॑य । इन्द्रा॑य घ॒र्मव॑ते । घ॒र्मव॑ते नि॒र्वप॑ति । घ॒र्मव॑त॒ इति॑ घ॒र्म - व॒ते॒ । नि॒र्वप॑ति॒ शिरः॑ । नि॒र्वप॒तीति॑ निः - वप॑ति । शिर॑ ए॒व । ए॒वास्य॑ । अ॒स्य॒ तेन॑ । तेन॑ करोति । क॒रो॒ति॒ यत् । यदिन्द्रा॑य । इन्द्रा॑येन्द्रि॒याव॑ते । इ॒न्द्रि॒याव॑त आ॒त्मान᳚म् । इ॒न्द्रि॒याव॑त॒ इती᳚न्द्रि॒य - व॒ते॒ । आ॒त्मान॑मे॒व । ए॒वास्य॑ । अ॒स्य॒ तेन॑ । तेन॑ करोति । क॒रो॒ति॒ यत् । यदिन्द्रा॑य । इन्द्रा॑या॒र्कव॑ते । अ॒र्कव॑ते भू॒तः । अ॒र्कव॑त॒ इत्य॒र्क - व॒ते॒ । भू॒त ए॒व । ए॒वान्नाद्ये᳚ । अ॒न्नाद्ये॒ प्रति॑ । अ॒न्नाद्य॒ इत्य॑न्न - अद्ये᳚ । प्रति॑ तिष्ठति । ति॒ष्ठ॒ति॒ भव॑ति । भव॑त्ये॒व । ए॒वेन्द्रा॑य । इन्द्रा॑याꣳहो॒मुचे᳚ \newline

\textbf{Jatai Paata} \newline

1. उप॑ धावति धाव॒ त्युपोप॑ धावति । \newline
2. धा॒व॒ति॒ स स धा॑वति धावति॒ सः । \newline
3. स ए॒वैव स स ए॒व । \newline
4. ए॒वास्मा॑ अस्मा ए॒वैवास्मै᳚ । \newline
5. अ॒स्मा॒ अन्न॒ मन्न॑ मस्मा अस्मा॒ अन्न᳚म् । \newline
6. अन्न॒म् प्र प्रान्न॒ मन्न॒म् प्र । \newline
7. प्र य॑च्छति यच्छति॒ प्र प्र य॑च्छति । \newline
8. य॒च्छ॒ त्य॒न्ना॒दो᳚ ऽन्ना॒दो य॑च्छति यच्छ त्यन्ना॒दः । \newline
9. अ॒न्ना॒द ए॒वैवान्ना॒दो᳚ ऽन्ना॒द ए॒व । \newline
10. अ॒न्ना॒द इत्य॑न्न - अ॒दः । \newline
11. ए॒व भ॑वति भव त्ये॒वैव भ॑वति । \newline
12. भ॒व॒तीन्द्रा॒ये न्द्रा॑य भवति भव॒तीन्द्रा॑य । \newline
13. इन्द्रा॑य घ॒र्मव॑ते घ॒र्मव॑त॒ इन्द्रा॒ये न्द्रा॑य घ॒र्मव॑ते । \newline
14. घ॒र्मव॑ते पुरो॒डाश॑म् पुरो॒डाश॑म् घ॒र्मव॑ते घ॒र्मव॑ते पुरो॒डाश᳚म् । \newline
15. घ॒र्मव॑त॒ इति॑ घ॒र्म - व॒ते॒ । \newline
16. पु॒रो॒डाश॒ मेका॑दशकपाल॒ मेका॑दशकपालम् पुरो॒डाश॑म् पुरो॒डाश॒ मेका॑दशकपालम् । \newline
17. एका॑दशकपाल॒म् निर् णिरेका॑दशकपाल॒ मेका॑दशकपाल॒म् निः । \newline
18. एका॑दशकपाल॒मित्येका॑दश - क॒पा॒ल॒म् । \newline
19. निर् व॑पेद् वपे॒न् निर् णिर् व॑पेत् । \newline
20. व॒पे॒ दिन्द्रा॒ये न्द्रा॑य वपेद् वपे॒ दिन्द्रा॑य । \newline
21. इन्द्रा॑ये न्द्रि॒याव॑त इन्द्रि॒याव॑त॒ इन्द्रा॒ये न्द्रा॑ये न्द्रि॒याव॑ते । \newline
22. इ॒न्द्रि॒याव॑त॒ इन्द्रा॒ये न्द्रा॑ये न्द्रि॒याव॑त इन्द्रि॒याव॑त॒ इन्द्रा॑य । \newline
23. इ॒न्द्रि॒याव॑त॒ इती᳚न्द्रि॒य - व॒ते॒ । \newline
24. इन्द्रा॑या॒ र्कव॑ते॒ ऽर्कव॑त॒ इन्द्रा॒ये न्द्रा॑या॒ र्कव॑ते । \newline
25. अ॒र्कव॑ते॒ भूति॑कामो॒ भूति॑कामो॒ ऽर्कव॑ते॒ ऽर्कव॑ते॒ भूति॑कामः । \newline
26. अ॒र्कव॑त॒ इत्य॒र्क - व॒ते॒ । \newline
27. भूति॑कामो॒ यद् यद् भूति॑कामो॒ भूति॑कामो॒ यत् । \newline
28. भूति॑काम॒ इति॒ भूति॑ - का॒मः॒ । \newline
29. यदिन्द्रा॒ये न्द्रा॑य॒ यद् यदिन्द्रा॑य । \newline
30. इन्द्रा॑य घ॒र्मव॑ते घ॒र्मव॑त॒ इन्द्रा॒ये न्द्रा॑य घ॒र्मव॑ते । \newline
31. घ॒र्मव॑ते नि॒र्वप॑ति नि॒र्वप॑ति घ॒र्मव॑ते घ॒र्मव॑ते नि॒र्वप॑ति । \newline
32. घ॒र्मव॑त॒ इति॑ घ॒र्म - व॒ते॒ । \newline
33. नि॒र्वप॑ति॒ शिरः॒ शिरो॑ नि॒र्वप॑ति नि॒र्वप॑ति॒ शिरः॑ । \newline
34. नि॒र्वप॒तीति॑ निः - वप॑ति । \newline
35. शिर॑ ए॒वैव शिरः॒ शिर॑ ए॒व । \newline
36. ए॒वास्या᳚ स्यै॒वै वास्य॑ । \newline
37. अ॒स्य॒ तेन॒ तेना᳚ स्यास्य॒ तेन॑ । \newline
38. तेन॑ करोति करोति॒ तेन॒ तेन॑ करोति । \newline
39. क॒रो॒ति॒ यद् यत् क॑रोति करोति॒ यत् । \newline
40. यदिन्द्रा॒ये न्द्रा॑य॒ यद् यदिन्द्रा॑य । \newline
41. इन्द्रा॑ये न्द्रि॒याव॑त इन्द्रि॒याव॑त॒ इन्द्रा॒ये न्द्रा॑ये न्द्रि॒याव॑ते । \newline
42. इ॒न्द्रि॒याव॑त आ॒त्मान॑ मा॒त्मान॑ मिन्द्रि॒याव॑त इन्द्रि॒याव॑त आ॒त्मान᳚म् । \newline
43. इ॒न्द्रि॒याव॑त॒ इती᳚न्द्रि॒य - व॒ते॒ । \newline
44. आ॒त्मान॑ मे॒वैवात्मान॑ मा॒त्मान॑ मे॒व । \newline
45. ए॒वास्या᳚ स्यै॒वै वास्य॑ । \newline
46. अ॒स्य॒ तेन॒ तेना᳚स्यास्य॒ तेन॑ । \newline
47. तेन॑ करोति करोति॒ तेन॒ तेन॑ करोति । \newline
48. क॒रो॒ति॒ यद् यत् क॑रोति करोति॒ यत् । \newline
49. यदिन्द्रा॒ये न्द्रा॑य॒ यद् यदिन्द्रा॑य । \newline
50. इन्द्रा॑या॒ र्कव॑ते॒ ऽर्कव॑त॒ इन्द्रा॒ये न्द्रा॑या॒ र्कव॑ते । \newline
51. अ॒र्कव॑ते भू॒तो भू॒तो᳚ ऽर्कव॑ते॒ ऽर्कव॑ते भू॒तः । \newline
52. अ॒र्कव॑त॒ इत्य॒र्क - व॒ते॒ । \newline
53. भू॒त ए॒वैव भू॒तो भू॒त ए॒व । \newline
54. ए॒वा न्नाद्ये॒ ऽन्नाद्य॑ ए॒वैवा न्नाद्ये᳚ । \newline
55. अ॒न्नाद्ये॒ प्रति॒ प्र त्य॒न्नाद्ये॒ ऽन्नाद्ये॒ प्रति॑ । \newline
56. अ॒न्नाद्य॒ इत्य॑न्न - अद्ये᳚ । \newline
57. प्रति॑ तिष्ठति तिष्ठति॒ प्रति॒ प्रति॑ तिष्ठति । \newline
58. ति॒ष्ठ॒ति॒ भव॑ति॒ भव॑ति तिष्ठति तिष्ठति॒ भव॑ति । \newline
59. भव॑ त्ये॒वैव भव॑ति॒ भव॑ त्ये॒व । \newline
60. ए॒वे न्द्रा॒ये न्द्रा॑यै॒वैवे न्द्रा॑य । \newline
61. इन्द्रा॑या ꣳहो॒मुचे ऽꣳ॑हो॒मुच॒ इन्द्रा॒ये न्द्रा॑या ꣳहो॒मुचे᳚ । \newline

\textbf{Ghana Paata } \newline

1. उप॑ धावति धाव॒ त्युपोप॑ धावति॒ स स धा॑व॒ त्युपोप॑ धावति॒ सः । \newline
2. धा॒व॒ति॒ स स धा॑वति धावति॒ स ए॒वैव स धा॑वति धावति॒ स ए॒व । \newline
3. स ए॒वैव स स ए॒वास्मा॑ अस्मा ए॒व स स ए॒वास्मै᳚ । \newline
4. ए॒वास्मा॑ अस्मा ए॒वैवास्मा॒ अन्न॒ मन्न॑ मस्मा ए॒वैवास्मा॒ अन्न᳚म् । \newline
5. अ॒स्मा॒ अन्न॒ मन्न॑ मस्मा अस्मा॒ अन्न॒म् प्र प्रान्न॑ मस्मा अस्मा॒ अन्न॒म् प्र । \newline
6. अन्न॒म् प्र प्रान्न॒ मन्न॒म् प्र य॑च्छति यच्छति॒ प्रान्न॒ मन्न॒म् प्र य॑च्छति । \newline
7. प्र य॑च्छति यच्छति॒ प्र प्र य॑च्छ त्यन्ना॒दो᳚ ऽन्ना॒दो य॑च्छति॒ प्र प्र य॑च्छ त्यन्ना॒दः । \newline
8. य॒च्छ॒ त्य॒न्ना॒दो᳚ ऽन्ना॒दो य॑च्छति यच्छ त्यन्ना॒द ए॒वैवान्ना॒दो य॑च्छति यच्छ त्यन्ना॒द ए॒व । \newline
9. अ॒न्ना॒द ए॒वैवान्ना॒दो᳚ ऽन्ना॒द ए॒व भ॑वति भव त्ये॒वान्ना॒दो᳚ ऽन्ना॒द ए॒व भ॑वति । \newline
10. अ॒न्ना॒द इत्य॑न्न - अ॒दः । \newline
11. ए॒व भ॑वति भव त्ये॒वैव भ॑व॒तीन्द्रा॒ये न्द्रा॑य भव त्ये॒वैव भ॑व॒तीन्द्रा॑य । \newline
12. भ॒व॒तीन्द्रा॒ये न्द्रा॑य भवति भव॒तीन्द्रा॑य घ॒र्मव॑ते घ॒र्मव॑त॒ इन्द्रा॑य भवति भव॒तीन्द्रा॑य घ॒र्मव॑ते । \newline
13. इन्द्रा॑य घ॒र्मव॑ते घ॒र्मव॑त॒ इन्द्रा॒ये न्द्रा॑य घ॒र्मव॑ते पुरो॒डाश॑म् पुरो॒डाश॑म् घ॒र्मव॑त॒ इन्द्रा॒ये न्द्रा॑य घ॒र्मव॑ते पुरो॒डाश᳚म् । \newline
14. घ॒र्मव॑ते पुरो॒डाश॑म् पुरो॒डाश॑म् घ॒र्मव॑ते घ॒र्मव॑ते पुरो॒डाश॒ मेका॑दशकपाल॒ मेका॑दशकपालम् पुरो॒डाश॑म् घ॒र्मव॑ते घ॒र्मव॑ते पुरो॒डाश॒ मेका॑दशकपालम् । \newline
15. घ॒र्मव॑त॒ इति॑ घ॒र्म - व॒ते॒ । \newline
16. पु॒रो॒डाश॒ मेका॑दशकपाल॒ मेका॑दशकपालम् पुरो॒डाश॑म् पुरो॒डाश॒ मेका॑दशकपाल॒म् निर् णिरेका॑दशकपालम् पुरो॒डाश॑म् पुरो॒डाश॒ मेका॑दशकपाल॒म् निः । \newline
17. एका॑दशकपाल॒म् निर् णिरेका॑दशकपाल॒ मेका॑दशकपाल॒म् निर् व॑पेद् वपे॒न् निरेका॑दशकपाल॒ मेका॑दशकपाल॒म् निर् व॑पेत् । \newline
18. एका॑दशकपाल॒मित्येका॑दश - क॒पा॒ल॒म् । \newline
19. निर् व॑पेद् वपे॒न् निर् णिर् व॑पे॒ दिन्द्रा॒ये न्द्रा॑य वपे॒न् निर् णिर् व॑पे॒ दिन्द्रा॑य । \newline
20. व॒पे॒ दिन्द्रा॒ये न्द्रा॑य वपेद् वपे॒ दिन्द्रा॑ये न्द्रि॒याव॑त इन्द्रि॒याव॑त॒ इन्द्रा॑य वपेद् वपे॒ दिन्द्रा॑ये न्द्रि॒याव॑ते । \newline
21. इन्द्रा॑ये न्द्रि॒याव॑त इन्द्रि॒याव॑त॒ इन्द्रा॒ये न्द्रा॑ये न्द्रि॒याव॑त॒ इन्द्रा॒ये न्द्रा॑ये न्द्रि॒याव॑त॒ इन्द्रा॒ये न्द्रा॑ये न्द्रि॒याव॑त॒ इन्द्रा॑य । \newline
22. इ॒न्द्रि॒याव॑त॒ इन्द्रा॒ये न्द्रा॑ये न्द्रि॒याव॑त इन्द्रि॒याव॑त॒ इन्द्रा॑या॒र्कव॑ते॒ ऽर्कव॑त॒ इन्द्रा॑ये न्द्रि॒याव॑त इन्द्रि॒याव॑त॒ इन्द्रा॑या॒र्कव॑ते । \newline
23. इ॒न्द्रि॒याव॑त॒ इती᳚न्द्रि॒य - व॒ते॒ । \newline
24. इन्द्रा॑या॒र्कव॑ते॒ ऽर्कव॑त॒ इन्द्रा॒ये न्द्रा॑या॒र्कव॑ते॒ भूति॑कामो॒ भूति॑कामो॒ ऽर्कव॑त॒ इन्द्रा॒ये न्द्रा॑या॒र्कव॑ते॒ भूति॑कामः । \newline
25. अ॒र्कव॑ते॒ भूति॑कामो॒ भूति॑कामो॒ ऽर्कव॑ते॒ ऽर्कव॑ते॒ भूति॑कामो॒ यद् यद् भूति॑कामो॒ ऽर्कव॑ते॒ ऽर्कव॑ते॒ भूति॑कामो॒ यत् । \newline
26. अ॒र्कव॑त॒ इत्य॒र्क - व॒ते॒ । \newline
27. भूति॑कामो॒ यद् यद् भूति॑कामो॒ भूति॑कामो॒ यदिन्द्रा॒ये न्द्रा॑य॒ यद् भूति॑कामो॒ भूति॑कामो॒ यदिन्द्रा॑य । \newline
28. भूति॑काम॒ इति॒ भूति॑ - का॒मः॒ । \newline
29. यदिन्द्रा॒ये न्द्रा॑य॒ यद् यदिन्द्रा॑य घ॒र्मव॑ते घ॒र्मव॑त॒ इन्द्रा॑य॒ यद् यदिन्द्रा॑य घ॒र्मव॑ते । \newline
30. इन्द्रा॑य घ॒र्मव॑ते घ॒र्मव॑त॒ इन्द्रा॒ये न्द्रा॑य घ॒र्मव॑ते नि॒र्वप॑ति नि॒र्वप॑ति घ॒र्मव॑त॒ इन्द्रा॒ये न्द्रा॑य घ॒र्मव॑ते नि॒र्वप॑ति । \newline
31. घ॒र्मव॑ते नि॒र्वप॑ति नि॒र्वप॑ति घ॒र्मव॑ते घ॒र्मव॑ते नि॒र्वप॑ति॒ शिरः॒ शिरो॑ नि॒र्वप॑ति घ॒र्मव॑ते घ॒र्मव॑ते नि॒र्वप॑ति॒ शिरः॑ । \newline
32. घ॒र्मव॑त॒ इति॑ घ॒र्म - व॒ते॒ । \newline
33. नि॒र्वप॑ति॒ शिरः॒ शिरो॑ नि॒र्वप॑ति नि॒र्वप॑ति॒ शिर॑ ए॒वैव शिरो॑ नि॒र्वप॑ति नि॒र्वप॑ति॒ शिर॑ ए॒व । \newline
34. नि॒र्वप॒तीति॑ निः - वप॑ति । \newline
35. शिर॑ ए॒वैव शिरः॒ शिर॑ ए॒वास्या᳚ स्यै॒व शिरः॒ शिर॑ ए॒वास्य॑ । \newline
36. ए॒वास्या᳚ स्यै॒वै वास्य॒ तेन॒ तेना᳚ स्यै॒वै वास्य॒ तेन॑ । \newline
37. अ॒स्य॒ तेन॒ तेना᳚ स्यास्य॒ तेन॑ करोति करोति॒ तेना᳚ स्यास्य॒ तेन॑ करोति । \newline
38. तेन॑ करोति करोति॒ तेन॒ तेन॑ करोति॒ यद् यत् क॑रोति॒ तेन॒ तेन॑ करोति॒ यत् । \newline
39. क॒रो॒ति॒ यद् यत् क॑रोति करोति॒ यदिन्द्रा॒ये न्द्रा॑य॒ यत् क॑रोति करोति॒ यदिन्द्रा॑य । \newline
40. यदिन्द्रा॒ये न्द्रा॑य॒ यद् यदिन्द्रा॑ये न्द्रि॒याव॑त इन्द्रि॒याव॑त॒ इन्द्रा॑य॒ यद् यदिन्द्रा॑ये न्द्रि॒याव॑ते । \newline
41. इन्द्रा॑ये न्द्रि॒याव॑त इन्द्रि॒याव॑त॒ इन्द्रा॒ये न्द्रा॑ये न्द्रि॒याव॑त आ॒त्मान॑ मा॒त्मान॑ मिन्द्रि॒याव॑त॒ इन्द्रा॒ये न्द्रा॑ये न्द्रि॒याव॑त आ॒त्मान᳚म् । \newline
42. इ॒न्द्रि॒याव॑त आ॒त्मान॑ मा॒त्मान॑ मिन्द्रि॒याव॑त इन्द्रि॒याव॑त आ॒त्मान॑ मे॒वैवात्मान॑ मिन्द्रि॒याव॑त इन्द्रि॒याव॑त आ॒त्मान॑ मे॒व । \newline
43. इ॒न्द्रि॒याव॑त॒ इती᳚न्द्रि॒य - व॒ते॒ । \newline
44. आ॒त्मान॑ मे॒वैवात्मान॑ मा॒त्मान॑ मे॒वास्या᳚ स्यै॒वात्मान॑ मा॒त्मान॑ मे॒वास्य॑ । \newline
45. ए॒वास्या᳚ स्यै॒वै वास्य॒ तेन॒ तेना᳚ स्यै॒वै वास्य॒ तेन॑ । \newline
46. अ॒स्य॒ तेन॒ तेना᳚ स्यास्य॒ तेन॑ करोति करोति॒ तेना᳚ स्यास्य॒ तेन॑ करोति । \newline
47. तेन॑ करोति करोति॒ तेन॒ तेन॑ करोति॒ यद् यत् क॑रोति॒ तेन॒ तेन॑ करोति॒ यत् । \newline
48. क॒रो॒ति॒ यद् यत् क॑रोति करोति॒ यदिन्द्रा॒ये न्द्रा॑य॒ यत् क॑रोति करोति॒ यदिन्द्रा॑य । \newline
49. यदिन्द्रा॒ये न्द्रा॑य॒ यद् यदिन्द्रा॑या॒र्कव॑ते॒ ऽर्कव॑त॒ इन्द्रा॑य॒ यद् यदिन्द्रा॑या॒र्कव॑ते । \newline
50. इन्द्रा॑या॒र्कव॑ते॒ ऽर्कव॑त॒ इन्द्रा॒ये न्द्रा॑या॒र्कव॑ते भू॒तो भू॒तो᳚ ऽर्कव॑त॒ इन्द्रा॒ये न्द्रा॑या॒र्कव॑ते भू॒तः । \newline
51. अ॒र्कव॑ते भू॒तो भू॒तो᳚ ऽर्कव॑ते॒ ऽर्कव॑ते भू॒त ए॒वैव भू॒तो᳚ ऽर्कव॑ते॒ ऽर्कव॑ते भू॒त ए॒व । \newline
52. अ॒र्कव॑त॒ इत्य॒र्क - व॒ते॒ । \newline
53. भू॒त ए॒वैव भू॒तो भू॒त ए॒वान्नाद्ये॒ ऽन्नाद्य॑ ए॒व भू॒तो भू॒त ए॒वान्नाद्ये᳚ । \newline
54. ए॒वान्नाद्ये॒ ऽन्नाद्य॑ ए॒वैवान्नाद्ये॒ प्रति॒ प्रत्य॒न्नाद्य॑ ए॒वैवान्नाद्ये॒ प्रति॑ । \newline
55. अ॒न्नाद्ये॒ प्रति॒ प्रत्य॒न्नाद्ये॒ ऽन्नाद्ये॒ प्रति॑ तिष्ठति तिष्ठति॒ प्रत्य॒न्नाद्ये॒ ऽन्नाद्ये॒ प्रति॑ तिष्ठति । \newline
56. अ॒न्नाद्य॒ इत्य॑न्न - अद्ये᳚ । \newline
57. प्रति॑ तिष्ठति तिष्ठति॒ प्रति॒ प्रति॑ तिष्ठति॒ भव॑ति॒ भव॑ति तिष्ठति॒ प्रति॒ प्रति॑ तिष्ठति॒ भव॑ति । \newline
58. ति॒ष्ठ॒ति॒ भव॑ति॒ भव॑ति तिष्ठति तिष्ठति॒ भव॑ त्ये॒वैव भव॑ति तिष्ठति तिष्ठति॒ भव॑ त्ये॒व । \newline
59. भव॑ त्ये॒वैव भव॑ति॒ भव॑ त्ये॒वे न्द्रा॒ये न्द्रा॑यै॒व भव॑ति॒ भव॑ त्ये॒वे न्द्रा॑य । \newline
60. ए॒वे न्द्रा॒ये न्द्रा॑यै॒ वैवे न्द्रा॑या ꣳहो॒मुचे ऽꣳ॑हो॒मुच॒ इन्द्रा॑यै॒वैवे न्द्रा॑या ꣳहो॒मुचे᳚ । \newline
61. इन्द्रा॑या ꣳहो॒मुचे ऽꣳ॑हो॒मुच॒ इन्द्रा॒ये न्द्रा॑या ꣳहो॒मुचे॑ पुरो॒डाश॑म् पुरो॒डाश॑ मꣳहो॒मुच॒ इन्द्रा॒ये न्द्रा॑या ꣳहो॒मुचे॑ पुरो॒डाश᳚म् । \newline
\pagebreak
\markright{ TS 2.2.7.4  \hfill https://www.vedavms.in \hfill}
\addcontentsline{toc}{section}{ TS 2.2.7.4 }
\section*{ TS 2.2.7.4 }

\textbf{TS 2.2.7.4 } \newline
\textbf{Samhita Paata} \newline

-ऽꣳ हो॒मुचे॑ पुरो॒डाश॒मेका॑दशकपालं॒ निर्व॑पे॒द्यः पा॒प्मना॑ गृही॒तः स्यात् पा॒प्मा वा अꣳह॒ इन्द्र॑मे॒वाऽꣳ हो॒मुचꣳ॒॒ स्वेन॑ भाग॒धेये॒नोप॑ धावति॒ स ए॒वैनं॑ पा॒प्मनोऽꣳह॑सो मुञ्च॒तीन्द्रा॑य वैमृ॒धाय॑ पुरो॒डाश॒मेका॑दशकपालं॒ निर्व॑पे॒द्यं मृधो॒ऽभि प्र॒वेपे॑रन्-रा॒ष्ट्राणि॑ वा॒ऽभि स॑मि॒युरिन्द्र॑मे॒व वै॑मृ॒धꣳ स्वेन॑ भाग॒धेये॒नोप॑ धावति॒ स ए॒वास्मा॒न्मृधो - [  ] \newline

\textbf{Pada Paata} \newline

अꣳ॒॒हो॒मुच॒ इत्यꣳ॑हः - मुचे᳚ । पु॒रो॒डाश᳚म् । एका॑दशकपाल॒मित्येका॑दश - क॒पा॒ल॒म् । निरिति॑ । व॒पे॒त् । यः । पा॒प्मना᳚ । गृ॒ही॒तः । स्यात् । पा॒प्मा । वै । अꣳहः॑ । इन्द्र᳚म् । ए॒व । अꣳ॒॒हो॒मुच॒मित्यꣳ॑हः - मुच᳚म् । स्वेन॑ । भा॒ग॒धेये॒नेति॑ भाग-धेये॑न । उपेति॑ । धा॒व॒ति॒ । सः । ए॒व । ए॒न॒म् । पा॒प्मनः॑ । अꣳह॑सः । मु॒ञ्च॒ति॒ । इन्द्रा॑य । वै॒मृ॒धाय॑ । पु॒रो॒डाश᳚म् । एका॑दशकपाल॒मित्येका॑दश - क॒पा॒ल॒म् । निरिति॑ । व॒पे॒त् । यम् । मृधः॑ । अ॒भीति॑ । प्र॒वेपे॑र॒न्निति॑ प्र - वेपे॑रन्न् । रा॒ष्ट्राणि॑ । वा॒ । अ॒भीति॑ । स॒मि॒युरिति॑ सं - इ॒युः । इन्द्र᳚म् । ए॒व । वै॒मृ॒धम् । स्वेन॑ । भा॒ग॒धेये॒नेति॑ भाग - धेये॑न । उपेति॑ । धा॒व॒ति॒ । सः । ए॒व । अ॒स्मा॒त् । मृधः॑ ।  \newline


\textbf{Krama Paata} \newline

अꣳ॒॒हो॒मुचे॑ पुरो॒डाश᳚म् । अꣳ॒॒हो॒मुच॒ इत्यꣳ॑हः - मुचे᳚ । पु॒रो॒डाश॒मेका॑दशपालम् । एका॑दशकपाल॒म् निः । एका॑दशकपाल॒मित्येका॑दश - क॒पा॒ल॒म् । निर् व॑पेत् । व॒पे॒द् यः । यः पा॒प्मना᳚ । पा॒प्मना॑ गृही॒तः । गृ॒ही॒तः स्यात् । स्यात् पा॒प्मा । पा॒प्मा वै । वा अꣳहः॑ । अꣳह॒ इन्द्र᳚म् । इन्द्र॑मे॒व । ए॒वाꣳहो॒मुच᳚म् । अꣳ॒॒हो॒मुचꣳ॒॒ स्वेन॑ । अꣳ॒॒हो॒मुच॒मित्यꣳ॑हः - मुच᳚म् । स्वेन॑ भाग॒धेये॑न । भा॒ग॒धेये॒नोप॑ । भा॒ग॒धेये॒नेति॑ भाग - धेये॑न । उप॑ धावति । धा॒व॒ति॒ सः । स ए॒व । ए॒वैन᳚म् । ए॒न॒म् पा॒प्मनः॑ । पा॒प्मनो ऽꣳह॑सः । अꣳह॑सो मुञ्चति । मु॒ञ्च॒तीन्द्रा॑य । इन्द्रा॑य वैमृ॒धाय॑ । वै॒मृ॒धाय॑ पुरो॒डाश᳚म् । पु॒रो॒डाश॒मेका॑दशकपालम् । एका॑दशकपाल॒म् निः । एका॑दशकपाल॒मित्येका॑दश - क॒पा॒ल॒म् । निर् व॑पेत् । व॒पे॒द् यम् । यम् मृधः॑ । मृधो॒ऽभि । अ॒भि प्र॒वेपे॑रन्न् । प्र॒वेपे॑रन् रा॒ष्ट्राणि॑ । प्र॒वेपे॑र॒न्निति॑ प्र - वेपे॑रन्न् । रा॒ष्ट्राणि॑ वा । वा॒ऽभि । अ॒भि स॑मि॒युः । स॒मि॒युरिन्द्र᳚म् । स॒मि॒युरिति॑ सं - इ॒युः । इन्द्र॑मे॒व । ए॒व वै॑मृ॒धम् । वै॒मृ॒धꣳ स्वेन॑ । स्वेन॑ भाग॒धेये॑न । भा॒ग॒धेये॒नोप॑ । भा॒ग॒धेये॒नेति॑ भाग - धेये॑न । उप॑ धावति । धा॒व॒ति॒ सः । स ए॒व । ए॒वास्मा᳚त् । अ॒स्मा॒न् मृधः॑ । मृधोऽप॑ \newline

\textbf{Jatai Paata} \newline

1. अꣳ॒॒हो॒मुचे॑ पुरो॒डाश॑म् पुरो॒डाश॑ मꣳहो॒मुचे ऽꣳ॑हो॒मुचे॑ पुरो॒डाश᳚म् । \newline
2. अꣳ॒॒हो॒मुच॒ इत्यꣳ॑हः - मुचे᳚ । \newline
3. पु॒रो॒डाश॒ मेका॑दशकपाल॒ मेका॑दशकपालम् पुरो॒डाश॑म् पुरो॒डाश॒ मेका॑दशकपालम् । \newline
4. एका॑दशकपाल॒म् निर् णिरेका॑दशकपाल॒ मेका॑दशकपाल॒म् निः । \newline
5. एका॑दशकपाल॒मित्येका॑दश - क॒पा॒ल॒म् । \newline
6. निर् व॑पेद् वपे॒न् निर् णिर् व॑पेत् । \newline
7. व॒पे॒द् यो यो व॑पेद् वपे॒द् यः । \newline
8. यः पा॒प्मना॑ पा॒प्मना॒ यो यः पा॒प्मना᳚ । \newline
9. पा॒प्मना॑ गृही॒तो गृ॑ही॒तः पा॒प्मना॑ पा॒प्मना॑ गृही॒तः । \newline
10. गृ॒ही॒तः स्याथ् स्याद् गृ॑ही॒तो गृ॑ही॒तः स्यात् । \newline
11. स्यात् पा॒प्मा पा॒प्मा स्याथ् स्यात् पा॒प्मा । \newline
12. पा॒प्मा वै वै पा॒प्मा पा॒प्मा वै । \newline
13. वा अꣳहो ऽꣳहो॒ वै वा अꣳहः॑ । \newline
14. अꣳह॒ इन्द्र॒ मिन्द्र॒ मꣳहो ऽꣳह॒ इन्द्र᳚म् । \newline
15. इन्द्र॑ मे॒वैवे न्द्र॒ मिन्द्र॑ मे॒व । \newline
16. ए॒वा ꣳहो॒मुच॑ मꣳहो॒मुच॑ मे॒वैवा ꣳहो॒मुच᳚म् । \newline
17. अꣳ॒॒हो॒मुच॒ ꣳ॒स्वेन॒ स्वेना ꣳ॑हो॒मुच॑ मꣳहो॒मुचꣳ॒॒ स्वेन॑ । \newline
18. अꣳ॒॒हो॒मुच॒मित्यꣳ॑हः - मुच᳚म् । \newline
19. स्वेन॑ भाग॒धेये॑न भाग॒धेये॑न॒ स्वेन॒ स्वेन॑ भाग॒धेये॑न । \newline
20. भा॒ग॒धेये॒नोपोप॑ भाग॒धेये॑न भाग॒धेये॒नोप॑ । \newline
21. भा॒ग॒धेये॒नेति॑ भाग - धेये॑न । \newline
22. उप॑ धावति धाव॒ त्युपोप॑ धावति । \newline
23. धा॒व॒ति॒ स स धा॑वति धावति॒ सः । \newline
24. स ए॒वैव स स ए॒व । \newline
25. ए॒वैन॑ मेन मे॒वैवैन᳚म् । \newline
26. ए॒न॒म् पा॒प्मनः॑ पा॒प्मन॑ एन मेनम् पा॒प्मनः॑ । \newline
27. पा॒प्मनो ऽꣳह॒सो ऽꣳह॑स स्पा॒प्मनः॑ पा॒प्मनो ऽꣳह॑सः । \newline
28. अꣳह॑सो मुञ्चति मुञ्च॒ त्यꣳह॒सो ऽꣳह॑सो मुञ्चति । \newline
29. मु॒ञ्च॒तीन्द्रा॒ये न्द्रा॑य मुञ्चति मुञ्च॒तीन्द्रा॑य । \newline
30. इन्द्रा॑य वैमृ॒धाय॑ वैमृ॒धाये न्द्रा॒ये न्द्रा॑य वैमृ॒धाय॑ । \newline
31. वै॒मृ॒धाय॑ पुरो॒डाश॑म् पुरो॒डाशं॑ ॅवैमृ॒धाय॑ वैमृ॒धाय॑ पुरो॒डाश᳚म् । \newline
32. पु॒रो॒डाश॒ मेका॑दशकपाल॒ मेका॑दशकपालम् पुरो॒डाश॑म् पुरो॒डाश॒ मेका॑दशकपालम् । \newline
33. एका॑दशकपाल॒म् निर् णिरेका॑दशकपाल॒ मेका॑दशकपाल॒म् निः । \newline
34. एका॑दशकपाल॒मित्येका॑दश - क॒पा॒ल॒म् । \newline
35. निर् व॑पेद् वपे॒न् निर् णिर् व॑पेत् । \newline
36. व॒पे॒द् यं ॅयं ॅव॑पेद् वपे॒द् यम् । \newline
37. यम् मृधो॒ मृधो॒ यं ॅयम् मृधः॑ । \newline
38. मृधो॒ ऽभ्य॑भि मृधो॒ मृधो॒ ऽभि । \newline
39. अ॒भि प्र॒वेपे॑रन् प्र॒वेपे॑रन् न॒भ्य॑भि प्र॒वेपे॑रन्न् । \newline
40. प्र॒वेपे॑रन् रा॒ष्ट्राणि॑ रा॒ष्ट्राणि॑ प्र॒वेपे॑रन् प्र॒वेपे॑रन् रा॒ष्ट्राणि॑ । \newline
41. प्र॒वेपे॑र॒न्निति॑ प्र - वेपे॑रन्न् । \newline
42. रा॒ष्ट्राणि॑ वा वा रा॒ष्ट्राणि॑ रा॒ष्ट्राणि॑ वा । \newline
43. वा॒ ऽभ्य॑भि वा॑ वा॒ ऽभि । \newline
44. अ॒भि स॑मि॒युः स॑मि॒यु र॒भ्य॑भि स॑मि॒युः । \newline
45. स॒मि॒यु रिन्द्र॒ मिन्द्रꣳ॑ समि॒युः स॑मि॒यु रिन्द्र᳚म् । \newline
46. स॒मि॒युरिति॑ सं - इ॒युः । \newline
47. इन्द्र॑ मे॒वैवे न्द्र॒ मिन्द्र॑ मे॒व । \newline
48. ए॒व वै॑मृ॒धं ॅवै॑मृ॒ध मे॒वैव वै॑मृ॒धम् । \newline
49. वै॒मृ॒धꣳ स्वेन॒ स्वेन॑ वैमृ॒धं ॅवै॑मृ॒धꣳ स्वेन॑ । \newline
50. स्वेन॑ भाग॒धेये॑न भाग॒धेये॑न॒ स्वेन॒ स्वेन॑ भाग॒धेये॑न । \newline
51. भा॒ग॒धेये॒नोपोप॑ भाग॒धेये॑न भाग॒धेये॒नोप॑ । \newline
52. भा॒ग॒धेये॒नेति॑ भाग - धेये॑न । \newline
53. उप॑ धावति धाव॒ त्युपोप॑ धावति । \newline
54. धा॒व॒ति॒ स स धा॑वति धावति॒ सः । \newline
55. स ए॒वैव स स ए॒व । \newline
56. ए॒वास्मा॑ दस्मा दे॒वै वास्मा᳚त् । \newline
57. अ॒स्मा॒न् मृधो॒ मृधो᳚ ऽस्मा दस्मा॒न् मृधः॑ । \newline
58. मृधो ऽपाप॒ मृधो॒ मृधो ऽप॑ । \newline

\textbf{Ghana Paata } \newline

1. अꣳ॒॒हो॒मुचे॑ पुरो॒डाश॑म् पुरो॒डाश॑ मꣳहो॒मुचे ऽꣳ॑हो॒मुचे॑ पुरो॒डाश॒ मेका॑दशकपाल॒ मेका॑दशकपालम् पुरो॒डाश॑ मꣳहो॒मुचे ऽꣳ॑हो॒मुचे॑ पुरो॒डाश॒ मेका॑दशकपालम् । \newline
2. अꣳ॒॒हो॒मुच॒ इत्यꣳ॑हः - मुचे᳚ । \newline
3. पु॒रो॒डाश॒ मेका॑दशकपाल॒ मेका॑दशकपालम् पुरो॒डाश॑म् पुरो॒डाश॒ मेका॑दशकपाल॒म् निर् णिरेका॑दशकपालम् पुरो॒डाश॑म् पुरो॒डाश॒ मेका॑दशकपाल॒म् निः । \newline
4. एका॑दशकपाल॒म् निर् णिरेका॑दशकपाल॒ मेका॑दशकपाल॒म् निर् व॑पेद् वपे॒न् निरेका॑दशकपाल॒ मेका॑दशकपाल॒म् निर् व॑पेत् । \newline
5. एका॑दशकपाल॒मित्येका॑दश - क॒पा॒ल॒म् । \newline
6. निर् व॑पेद् वपे॒न् निर् णिर् व॑पे॒द् यो यो व॑पे॒न् निर् णिर् व॑पे॒द् यः । \newline
7. व॒पे॒द् यो यो व॑पेद् वपे॒द् यः पा॒प्मना॑ पा॒प्मना॒ यो व॑पेद् वपे॒द् यः पा॒प्मना᳚ । \newline
8. यः पा॒प्मना॑ पा॒प्मना॒ यो यः पा॒प्मना॑ गृही॒तो गृ॑ही॒तः पा॒प्मना॒ यो यः पा॒प्मना॑ गृही॒तः । \newline
9. पा॒प्मना॑ गृही॒तो गृ॑ही॒तः पा॒प्मना॑ पा॒प्मना॑ गृही॒तः स्याथ् स्याद् गृ॑ही॒तः पा॒प्मना॑ पा॒प्मना॑ गृही॒तः स्यात् । \newline
10. गृ॒ही॒तः स्याथ् स्याद् गृ॑ही॒तो गृ॑ही॒तः स्यात् पा॒प्मा पा॒प्मा स्याद् गृ॑ही॒तो गृ॑ही॒तः स्यात् पा॒प्मा । \newline
11. स्यात् पा॒प्मा पा॒प्मा स्याथ् स्यात् पा॒प्मा वै वै पा॒प्मा स्याथ् स्यात् पा॒प्मा वै । \newline
12. पा॒प्मा वै वै पा॒प्मा पा॒प्मा वा अꣳहो ऽꣳहो॒ वै पा॒प्मा पा॒प्मा वा अꣳहः॑ । \newline
13. वा अꣳहो ऽꣳहो॒ वै वा अꣳह॒ इन्द्र॒ मिन्द्र॒ मꣳहो॒ वै वा अꣳह॒ इन्द्र᳚म् । \newline
14. अꣳह॒ इन्द्र॒ मिन्द्र॒ मꣳहो ऽꣳह॒ इन्द्र॑ मे॒वैवे न्द्र॒ मꣳहो ऽꣳह॒ इन्द्र॑ मे॒व । \newline
15. इन्द्र॑ मे॒वैवे न्द्र॒ मिन्द्र॑ मे॒वाꣳहो॒मुच॑ मꣳहो॒मुच॑ मे॒वे न्द्र॒ मिन्द्र॑ मे॒वाꣳहो॒मुच᳚म् । \newline
16. ए॒वाꣳहो॒मुच॑ मꣳहो॒मुच॑ मे॒वैवाꣳहो॒मुचꣳ॒॒ स्वेन॒ स्वेनाꣳ॑हो॒मुच॑ मे॒वैवाꣳ॑हो॒मुचꣳ॒॒ स्वेन॑ । \newline
17. अꣳ॒॒हो॒मुचꣳ॒॒ स्वेन॒ स्वेनाꣳ॑हो॒मुच॑ मꣳहो॒मुचꣳ॒॒ स्वेन॑ भाग॒धेये॑न भाग॒धेये॑न॒ स्वेनाꣳ॑हो॒मुच॑ मꣳहो॒मुचꣳ॒॒ स्वेन॑ भाग॒धेये॑न । \newline
18. अꣳ॒॒हो॒मुच॒मित्यꣳ॑हः - मुच᳚म् । \newline
19. स्वेन॑ भाग॒धेये॑न भाग॒धेये॑न॒ स्वेन॒ स्वेन॑ भाग॒धेये॒नोपोप॑ भाग॒धेये॑न॒ स्वेन॒ स्वेन॑ भाग॒धेये॒नोप॑ । \newline
20. भा॒ग॒धेये॒नोपोप॑ भाग॒धेये॑न भाग॒धेये॒नोप॑ धावति धाव॒त्युप॑ भाग॒धेये॑न भाग॒धेये॒नोप॑ धावति । \newline
21. भा॒ग॒धेये॒नेति॑ भाग - धेये॑न । \newline
22. उप॑ धावति धाव॒ त्युपोप॑ धावति॒ स स धा॑व॒ त्युपोप॑ धावति॒ सः । \newline
23. धा॒व॒ति॒ स स धा॑वति धावति॒ स ए॒वैव स धा॑वति धावति॒ स ए॒व । \newline
24. स ए॒वैव स स ए॒वैन॑ मेन मे॒व स स ए॒वैन᳚म् । \newline
25. ए॒वैन॑ मेन मे॒वैवैन॑म् पा॒प्मनः॑ पा॒प्मन॑ एन मे॒वैवैन॑म् पा॒प्मनः॑ । \newline
26. ए॒न॒म् पा॒प्मनः॑ पा॒प्मन॑ एन मेनम् पा॒प्मनो ऽꣳह॒सो ऽꣳह॑स स्पा॒प्मन॑ एन मेनम् पा॒प्मनो ऽꣳह॑सः । \newline
27. पा॒प्मनो ऽꣳह॒सो ऽꣳह॑स स्पा॒प्मनः॑ पा॒प्मनो ऽꣳह॑सो मुञ्चति मुञ्च॒ त्यꣳह॑स स्पा॒प्मनः॑ पा॒प्मनो ऽꣳह॑सो मुञ्चति । \newline
28. अꣳह॑सो मुञ्चति मुञ्च॒ त्यꣳह॒सो ऽꣳह॑सो मुञ्च॒तीन्द्रा॒ये न्द्रा॑य मुञ्च॒ त्यꣳह॒सो ऽꣳह॑सो मुञ्च॒तीन्द्रा॑य । \newline
29. मु॒ञ्च॒तीन्द्रा॒ये न्द्रा॑य मुञ्चति मुञ्च॒तीन्द्रा॑य वैमृ॒धाय॑ वैमृ॒धाये न्द्रा॑य मुञ्चति मुञ्च॒तीन्द्रा॑य वैमृ॒धाय॑ । \newline
30. इन्द्रा॑य वैमृ॒धाय॑ वैमृ॒धाये न्द्रा॒ये न्द्रा॑य वैमृ॒धाय॑ पुरो॒डाश॑म् पुरो॒डाशं॑ ॅवैमृ॒धाये न्द्रा॒ये न्द्रा॑य वैमृ॒धाय॑ पुरो॒डाश᳚म् । \newline
31. वै॒मृ॒धाय॑ पुरो॒डाश॑म् पुरो॒डाशं॑ ॅवैमृ॒धाय॑ वैमृ॒धाय॑ पुरो॒डाश॒ मेका॑दशकपाल॒ मेका॑दशकपालम् पुरो॒डाशं॑ ॅवैमृ॒धाय॑ वैमृ॒धाय॑ पुरो॒डाश॒ मेका॑दशकपालम् । \newline
32. पु॒रो॒डाश॒ मेका॑दशकपाल॒ मेका॑दशकपालम् पुरो॒डाश॑म् पुरो॒डाश॒ मेका॑दशकपाल॒म् निर् णिरेका॑दशकपालम् पुरो॒डाश॑म् पुरो॒डाश॒ मेका॑दशकपाल॒म् निः । \newline
33. एका॑दशकपाल॒म् निर् णिरेका॑दशकपाल॒ मेका॑दशकपाल॒म् निर् व॑पेद् वपे॒न् निरेका॑दशकपाल॒ मेका॑दशकपाल॒म् निर् व॑पेत् । \newline
34. एका॑दशकपाल॒मित्येका॑दश - क॒पा॒ल॒म् । \newline
35. निर् व॑पेद् वपे॒न् निर् णिर् व॑पे॒द् यं ॅयं ॅव॑पे॒न् निर् णिर् व॑पे॒द् यम् । \newline
36. व॒पे॒द् यं ॅयं ॅव॑पेद् वपे॒द् यम् मृधो॒ मृधो॒ यं ॅव॑पेद् वपे॒द् यम् मृधः॑ । \newline
37. यम् मृधो॒ मृधो॒ यं ॅयम् मृधो॒ ऽभ्य॑भि मृधो॒ यं ॅयम् मृधो॒ ऽभि । \newline
38. मृधो॒ ऽभ्य॑भि मृधो॒ मृधो॒ ऽभि प्र॒वेपे॑रन् प्र॒वेपे॑रन् न॒भि मृधो॒ मृधो॒ ऽभि प्र॒वेपे॑रन्न् । \newline
39. अ॒भि प्र॒वेपे॑रन् प्र॒वेपे॑रन् न॒भ्य॑भि प्र॒वेपे॑रन् रा॒ष्ट्राणि॑ रा॒ष्ट्राणि॑ प्र॒वेपे॑रन् न॒भ्य॑भि प्र॒वेपे॑रन् रा॒ष्ट्राणि॑ । \newline
40. प्र॒वेपे॑रन् रा॒ष्ट्राणि॑ रा॒ष्ट्राणि॑ प्र॒वेपे॑रन् प्र॒वेपे॑रन् रा॒ष्ट्राणि॑ वा वा रा॒ष्ट्राणि॑ प्र॒वेपे॑रन् प्र॒वेपे॑रन् रा॒ष्ट्राणि॑ वा । \newline
41. प्र॒वेपे॑र॒न्निति॑ प्र - वेपे॑रन्न् । \newline
42. रा॒ष्ट्राणि॑ वा वा रा॒ष्ट्राणि॑ रा॒ष्ट्राणि॑ वा॒ ऽभ्य॑भि वा॑ रा॒ष्ट्राणि॑ रा॒ष्ट्राणि॑ वा॒ ऽभि । \newline
43. वा॒ ऽभ्य॑भि वा॑ वा॒ ऽभि स॑मि॒युः स॑मि॒यु र॒भि वा॑ वा॒ ऽभि स॑मि॒युः । \newline
44. अ॒भि स॑मि॒युः स॑मि॒यु र॒भ्य॑भि स॑मि॒यु रिन्द्र॒ मिन्द्रꣳ॑ समि॒यु र॒भ्य॑भि स॑मि॒यु रिन्द्र᳚म् । \newline
45. स॒मि॒यु रिन्द्र॒ मिन्द्रꣳ॑ समि॒युः स॑मि॒यु रिन्द्र॑ मे॒वैवे न्द्रꣳ॑ समि॒युः स॑मि॒यु रिन्द्र॑ मे॒व । \newline
46. स॒मि॒युरिति॑ सं - इ॒युः । \newline
47. इन्द्र॑ मे॒वैवे न्द्र॒ मिन्द्र॑ मे॒व वै॑मृ॒धं ॅवै॑मृ॒ध मे॒वे न्द्र॒ मिन्द्र॑ मे॒व वै॑मृ॒धम् । \newline
48. ए॒व वै॑मृ॒धं ॅवै॑मृ॒ध मे॒वैव वै॑मृ॒धꣳ स्वेन॒ स्वेन॑ वैमृ॒ध मे॒वैव वै॑मृ॒धꣳ स्वेन॑ । \newline
49. वै॒मृ॒धꣳ स्वेन॒ स्वेन॑ वैमृ॒धं ॅवै॑मृ॒धꣳ स्वेन॑ भाग॒धेये॑न भाग॒धेये॑न॒ स्वेन॑ वैमृ॒धं ॅवै॑मृ॒धꣳ स्वेन॑ भाग॒धेये॑न । \newline
50. स्वेन॑ भाग॒धेये॑न भाग॒धेये॑न॒ स्वेन॒ स्वेन॑ भाग॒धेये॒नोपोप॑ भाग॒धेये॑न॒ स्वेन॒ स्वेन॑ भाग॒धेये॒नोप॑ । \newline
51. भा॒ग॒धेये॒नोपोप॑ भाग॒धेये॑न भाग॒धेये॒नोप॑ धावति धाव॒त्युप॑ भाग॒धेये॑न भाग॒धेये॒नोप॑ धावति । \newline
52. भा॒ग॒धेये॒नेति॑ भाग - धेये॑न । \newline
53. उप॑ धावति धाव॒ त्युपोप॑ धावति॒ स स धा॑व॒ त्युपोप॑ धावति॒ सः । \newline
54. धा॒व॒ति॒ स स धा॑वति धावति॒ स ए॒वैव स धा॑वति धावति॒ स ए॒व । \newline
55. स ए॒वैव स स ए॒वास्मा॑ दस्मा दे॒व स स ए॒वास्मा᳚त् । \newline
56. ए॒वास्मा॑ दस्मा दे॒वैवास्मा॒न् मृधो॒ मृधो᳚ ऽस्मा दे॒वैवास्मा॒न् मृधः॑ । \newline
57. अ॒स्मा॒न् मृधो॒ मृधो᳚ ऽस्मा दस्मा॒न् मृधो ऽपाप॒ मृधो᳚ ऽस्मा दस्मा॒न् मृधो ऽप॑ । \newline
58. मृधो ऽपाप॒ मृधो॒ मृधो ऽप॑ हन्ति ह॒न्त्यप॒ मृधो॒ मृधो ऽप॑ हन्ति । \newline
\pagebreak
\markright{ TS 2.2.7.5  \hfill https://www.vedavms.in \hfill}
\addcontentsline{toc}{section}{ TS 2.2.7.5 }
\section*{ TS 2.2.7.5 }

\textbf{TS 2.2.7.5 } \newline
\textbf{Samhita Paata} \newline

ऽप॑ ह॒न्तीन्द्रा॑य त्रा॒त्रे पु॑रो॒डाश॒मेका॑दशकपालं॒ निर्व॑पेद्-ब॒द्धो वा॒ परि॑यत्तो॒ वेन्द्र॑मे॒व त्रा॒तारꣳ॒॒ स्वेन॑ भाग॒धेये॒नोप॑ धावति॒ स ए॒वैनं॑ त्रायत॒ इन्द्रा॑यार्काश्वमे॒धव॑ते पुरो॒डाश॒मेका॑दशकपालं॒ निर्व॑पे॒द्यं म॑हाय॒ज्ञो नोप॒नमे॑दे॒ते वै म॑हाय॒ज्ञ्स्यान्त्ये॑ त॒नू यद॑र्काश्वमे॒धाविन्द्र॑मे॒वार्का᳚श्वमे॒ध- व॑न्तꣳ॒॒ स्वेन॑ भाग॒धेये॒नोप॑ धावति॒ स ए॒वास्मा॑ ( ) अन्त॒तो म॑हाय॒ज्ञ्ं च्या॑वय॒त्युपै॑नं महाय॒ज्ञो न॑मति ॥ \newline

\textbf{Pada Paata} \newline

अपेति॑ । ह॒न्ति॒ । इन्द्रा॑य । त्रा॒त्रे । पु॒रो॒डाश᳚म् । एका॑दशकपाल॒मित्येका॑दश - क॒पा॒ल॒म् । निरिति॑ । व॒पे॒त् । ब॒द्धः । वा॒ । परि॑यत्त॒ इति॒ परि॑ - य॒त्तः॒ । वा॒ । इन्द्र᳚म् । ए॒व । त्रा॒तार᳚म् । स्वेन॑ । भा॒ग॒धेये॒नेति॑ भाग - धेये॑न । उपेति॑ । धा॒व॒ति॒ । सः । ए॒व । ए॒न॒म् । त्रा॒य॒ते॒ । इन्द्रा॑य । अ॒र्का॒श्व॒मे॒धव॑त॒ इत्य॑र्काश्वमे॒ध-व॒ते॒ । पु॒रो॒डाश᳚म् । एका॑दशकपाल॒मित्येका॑दश - क॒पा॒ल॒म् । निरिति॑ । व॒पे॒त् । यम् । म॒हा॒य॒ज्ञ् इति॑ महा - य॒ज्ञ्ः । न । उ॒प॒नमे॒दित्यु॑प - नमे᳚त् । ए॒ते इति॑ । वै । म॒हा॒य॒ज्ञ्स्येति॑ महा - य॒ज्ञ्स्य॑ । अन्त्ये॒ इति॑ । त॒नू इति॑ । यत् । अ॒र्का॒श्व॒मे॒धावित्य॑र्क - अ॒श्व॒मे॒धौ । इन्द्र᳚म् । ए॒व । अ॒र्का॒श्व॒मे॒धव॑न्त॒मित्य॑र्काश्वमे॒ध - व॒न्त॒म् । स्वेन॑ । भा॒ग॒धेये॒नेति॑ भाग - धेये॑न । उपेति॑ । धा॒व॒ति॒ । सः । ए॒व । अ॒स्मै॒ ( ) । अ॒न्त॒तः । म॒हा॒य॒ज्ञ्मिति॑ महा - य॒ज्ञ्म् । च्या॒व॒य॒ति॒ । उपेति॑ । ए॒न॒म् । म॒हा॒य॒ज्ञ् इति॑ महा - य॒ज्ञ्ः । न॒म॒ति॒ ॥  \newline


\textbf{Krama Paata} \newline

अप॑ हन्ति । ह॒न्तीन्द्रा॑य । इन्द्रा॑य त्रा॒त्रे । त्रा॒त्रे पु॑रो॒डाश᳚म् । पु॒रो॒डाश॒मेका॑दशकपालम् । एका॑दशकपाल॒म् निः । एका॑दशकपाल॒मित्येका॑दश - क॒पा॒ल॒म् । निर् व॑पेत् । व॒पे॒द्,ब॒द्धः । ब॒द्धो वा᳚ । वा॒ परि॑यत्तः । परि॑यत्तो वा । परि॑यत्त॒ इति॒ परि॑ - य॒त्तः॒ । वेन्द्र᳚म् । इन्द्र॑मे॒व । ए॒व त्रा॒तार᳚म् । त्रा॒तारꣳ॒॒ स्वेन॑ । स्वेन॑ भाग॒धेये॑न । भा॒ग॒धेये॒नोप॑ । भा॒ग॒धेये॒नेति॑ भाग - धेये॑न । उप॑ धावति । धा॒व॒ति॒ सः । स ए॒व । ए॒वैन᳚म् । ए॒न॒म् त्रा॒य॒ते॒ । त्रा॒य॒त॒ इन्द्रा॑य । इन्द्रा॑यार्काश्वमे॒धव॑ते । अ॒र्का॒श्व॒मे॒धव॑ते पुरो॒डाश᳚म् । अ॒र्का॒श्व॒मेधव॑त॒ इत्य॑र्काश्वमे॒ध - व॒ते॒ । पु॒रो॒डाश॒मेका॑दशकपालम् । एका॑दशकपाल॒म् निः । एका॑दशकपाल॒मित्येका॑दश - क॒पा॒ल॒म् । निर् व॑पेत् । व॒पे॒द् यम् । यम् म॑हाय॒ज्ञ्ः । म॒हा॒य॒ज्ञो न । म॒हा॒य॒ज्ञ् इति॑ महा - य॒ज्ञ्ः । नोप॒नमे᳚त् । उ॒प॒नमे॑दे॒ते । उ॒प॒नमे॒दित्यु॑प - नमे᳚त् । ए॒ते वै । ए॒ते इत्ये॒ते । वै म॑हाय॒ज्ञ्स्य॑ । म॒हा॒य॒ज्ञ्स्यान्त्ये᳚ । म॒हा॒य॒ज्ञ्स्येति॑ महा - य॒ज्ञ्स्य॑ । अन्त्ये॑ त॒नू । अन्त्ये॒ इत्यन्त्ये᳚ । त॒नू यत् । त॒नू इति॑ त॒नू । यद॑र्काश्वमे॒धौ । अ॒र्का॒श्व॒मे॒धाविन्द्र᳚म् । अ॒र्का॒श्व॒मे॒धावित्य॑र्क - अ॒श्व॒मे॒धौ । इन्द्र॒॑मेव । ए॒वार्का᳚श्वमे॒धव॑न्तम् । अ॒र्का॒श्व॒मे॒धव॑न्तꣳ॒॒ स्वेन॑ । अ॒र्का॒श्व॒मे॒धव॑न्त॒मित्य॑र्काश्वमे॒ध - व॒न्त॒म् । स्वेन॑ भाग॒धेये॑न । भा॒ग॒धेये॒नोप॑ । भा॒ग॒धेये॒नेति॑ भाग - धेये॑न । उप॑ धावति । धा॒व॒ति॒ सः । स ए॒व । ए॒वास्मै᳚ ( ) । अ॒स्मा॒ अ॒न्त॒तः । अ॒न्त॒तो म॑हाय॒ज्ञ्म् । म॒हा॒य॒ज्ञ्म् च्या॑वयति । म॒हा॒य॒ज्ञ्मिति॑ महा - य॒ज्ञ्म् । च्या॒व॒य॒त्युप॑ । उपै॑नम् । ए॒न॒म् म॒हा॒य॒ज्ञ्ः । म॒हा॒य॒ज्ञो न॑मति । म॒हा॒य॒ज्ञ् इति॑ महा - य॒ज्ञ्ः । न॒म॒तीति॑ नमति । \newline

\textbf{Jatai Paata} \newline

1. अप॑ हन्ति ह॒ न्त्यपाप॑ हन्ति । \newline
2. ह॒न्तीन्द्रा॒ये न्द्रा॑य हन्ति ह॒न्तीन्द्रा॑य । \newline
3. इन्द्रा॑य त्रा॒त्रे त्रा॒त्र इन्द्रा॒ये न्द्रा॑य त्रा॒त्रे । \newline
4. त्रा॒त्रे पु॑रो॒डाश॑म् पुरो॒डाश॑म् त्रा॒त्रे त्रा॒त्रे पु॑रो॒डाश᳚म् । \newline
5. पु॒रो॒डाश॒ मेका॑दशकपाल॒ मेका॑दशकपालम् पुरो॒डाश॑म् पुरो॒डाश॒ मेका॑दशकपालम् । \newline
6. एका॑दशकपाल॒म् निर् णिरेका॑दशकपाल॒ मेका॑दशकपाल॒म् निः । \newline
7. एका॑दशकपाल॒मित्येका॑दश - क॒पा॒ल॒म् । \newline
8. निर् व॑पेद् वपे॒न् निर् णिर् व॑पेत् । \newline
9. व॒पे॒द् ब॒द्धो ब॒द्धो व॑पेद् वपेद् ब॒द्धः । \newline
10. ब॒द्धो वा॑ वा ब॒द्धो ब॒द्धो वा᳚ । \newline
11. वा॒ परि॑यत्तः॒ परि॑यत्तो वा वा॒ परि॑यत्तः । \newline
12. परि॑यत्तो वा वा॒ परि॑यत्तः॒ परि॑यत्तो वा । \newline
13. परि॑यत्त॒ इति॒ परि॑ - य॒त्तः॒ । \newline
14. वेन्द्र॒ मिन्द्रं॑ ॅवा॒ वेन्द्र᳚म् । \newline
15. इन्द्र॑ मे॒वैवे न्द्र॒ मिन्द्र॑ मे॒व । \newline
16. ए॒व त्रा॒तार॑म् त्रा॒तार॑ मे॒वैव त्रा॒तार᳚म् । \newline
17. त्रा॒तारꣳ॒॒ स्वेन॒ स्वेन॑ त्रा॒तार॑म् त्रा॒तारꣳ॒॒ स्वेन॑ । \newline
18. स्वेन॑ भाग॒धेये॑न भाग॒धेये॑न॒ स्वेन॒ स्वेन॑ भाग॒धेये॑न । \newline
19. भा॒ग॒धेये॒नोपोप॑ भाग॒धेये॑न भाग॒धेये॒नोप॑ । \newline
20. भा॒ग॒धेये॒नेति॑ भाग - धेये॑न । \newline
21. उप॑ धावति धाव॒ त्युपोप॑ धावति । \newline
22. धा॒व॒ति॒ स स धा॑वति धावति॒ सः । \newline
23. स ए॒वैव स स ए॒व । \newline
24. ए॒वैन॑ मेन मे॒वैवैन᳚म् । \newline
25. ए॒न॒म् त्रा॒य॒ते॒ त्रा॒य॒त॒ ए॒न॒ मे॒न॒म् त्रा॒य॒ते॒ । \newline
26. त्रा॒य॒त॒ इन्द्रा॒ये न्द्रा॑य त्रायते त्रायत॒ इन्द्रा॑य । \newline
27. इन्द्रा॑या र्काश्वमे॒धव॑ते ऽर्काश्वमे॒धव॑त॒ इन्द्रा॒ये न्द्रा॑या र्काश्वमे॒धव॑ते । \newline
28. अ॒र्का॒श्व॒मे॒धव॑ते पुरो॒डाश॑म् पुरो॒डाश॑ मर्काश्वमे॒धव॑ते ऽर्काश्वमे॒धव॑ते पुरो॒डाश᳚म् । \newline
29. अ॒र्का॒श्व॒मे॒धव॑त॒ इत्य॑र्काश्वमे॒ध - व॒ते॒ । \newline
30. पु॒रो॒डाश॒ मेका॑दशकपाल॒ मेका॑दशकपालम् पुरो॒डाश॑म् पुरो॒डाश॒ मेका॑दशकपालम् । \newline
31. एका॑दशकपाल॒म् निर् णिरेका॑दशकपाल॒ मेका॑दशकपाल॒म् निः । \newline
32. एका॑दशकपाल॒मित्येका॑दश - क॒पा॒ल॒म् । \newline
33. निर् व॑पेद् वपे॒न् निर् णिर् व॑पेत् । \newline
34. व॒पे॒द् यं ॅयं ॅव॑पेद् वपे॒द् यम् । \newline
35. यम् म॑हाय॒ज्ञो म॑हाय॒ज्ञो यं ॅयम् म॑हाय॒ज्ञ्ः । \newline
36. म॒हा॒य॒ज्ञो न न म॑हाय॒ज्ञो म॑हाय॒ज्ञो न । \newline
37. म॒हा॒य॒ज्ञ् इति॑ महा - य॒ज्ञ्ः । \newline
38. नोप॒नमे॑ दुप॒नमे॒न् न नोप॒नमे᳚त् । \newline
39. उ॒प॒नमे॑ दे॒ते ए॒ते उ॑प॒नमे॑ दुप॒नमे॑ दे॒ते । \newline
40. उ॒प॒नमे॒दित्यु॑प - नमे᳚त् । \newline
41. ए॒ते वै वा ए॒ते ए॒ते वै । \newline
42. ए॒ते इत्ये॒ते । \newline
43. वै म॑हाय॒ज्ञ्स्य॑ महाय॒ज्ञ्स्य॒ वै वै म॑हाय॒ज्ञ्स्य॑ । \newline
44. म॒हा॒य॒ज्ञ्स्या न्त्ये॒ अन्त्ये॑ महाय॒ज्ञ्स्य॑ महाय॒ज्ञ्स्या न्त्ये᳚ । \newline
45. म॒हा॒य॒ज्ञ्स्येति॑ महा - य॒ज्ञ्स्य॑ । \newline
46. अन्त्ये॑ त॒नू त॒नू अन्त्ये॒ अन्त्ये॑ त॒नू । \newline
47. अन्त्ये॒ इत्यन्त्ये᳚ । \newline
48. त॒नू यद् यत् त॒नू त॒नू यत् । \newline
49. त॒नू इति॑ त॒नू । \newline
50. यद॑र्काश्वमे॒धा व॑र्काश्वमे॒धौ यद् यद॑र्काश्वमे॒धौ । \newline
51. अ॒र्का॒श्व॒मे॒धा विन्द्र॒ मिन्द्र॑ मर्काश्वमे॒धा व॑र्काश्वमे॒धा विन्द्र᳚म् । \newline
52. अ॒र्का॒श्व॒मे॒धावित्य॑र्क - अ॒श्व॒मे॒धौ । \newline
53. इन्द्र॑ मे॒वैवे न्द्र॒ मिन्द्र॑ मे॒व । \newline
54. ए॒वार्का᳚श्वमे॒धव॑न्त मर्काश्वमे॒धव॑न्त मे॒वै वार्का᳚श्वमे॒धव॑न्तम् । \newline
55. अ॒र्का॒श्व॒मे॒धव॑न्तꣳ॒॒ स्वेन॒ स्वेना᳚र्काश्वमे॒धव॑न्त मर्काश्वमे॒धव॑न्तꣳ॒॒ स्वेन॑ । \newline
56. अ॒र्का॒श्व॒मे॒धव॑न्त॒मित्य॑र्काश्वमे॒ध - व॒न्त॒म् । \newline
57. स्वेन॑ भाग॒धेये॑न भाग॒धेये॑न॒ स्वेन॒ स्वेन॑ भाग॒धेये॑न । \newline
58. भा॒ग॒धेये॒नोपोप॑ भाग॒धेये॑न भाग॒धेये॒नोप॑ । \newline
59. भा॒ग॒धेये॒नेति॑ भाग - धेये॑न । \newline
60. उप॑ धावति धाव॒ त्युपोप॑ धावति । \newline
61. धा॒व॒ति॒ स स धा॑वति धावति॒ सः । \newline
62. स ए॒वैव स स ए॒व । \newline
63. ए॒वास्मा॑ अस्मा ए॒वैवास्मै᳚ । \newline
64. अ॒स्मा॒ अ॒न्त॒तो᳚ ऽन्त॒तो᳚ ऽस्मा अस्मा अन्त॒तः । \newline
65. अ॒न्त॒तो म॑हाय॒ज्ञ्म् म॑हाय॒ज्ञ् म॑न्त॒तो᳚ ऽन्त॒तो म॑हाय॒ज्ञ्म् । \newline
66. म॒हा॒य॒ज्ञ्म् च्या॑वयति च्यावयति महाय॒ज्ञ्म् म॑हाय॒ज्ञ्म् च्या॑वयति । \newline
67. म॒हा॒य॒ज्ञ्मिति॑ महा - य॒ज्ञ्म् । \newline
68. च्या॒व॒य॒ त्युपोप॑ च्यावयति च्यावय॒ त्युप॑ । \newline
69. उपै॑न मेन॒ मुपोपै॑नम् । \newline
70. ए॒न॒म् म॒हा॒य॒ज्ञो म॑हाय॒ज्ञ् ए॑न मेनम् महाय॒ज्ञ्ः । \newline
71. म॒हा॒य॒ज्ञो न॑मति नमति महाय॒ज्ञो म॑हाय॒ज्ञो न॑मति । \newline
72. म॒हा॒य॒ज्ञ् इति॑ महा - य॒ज्ञ्ः । \newline
73. न॒म॒तीति॑ नमति । \newline

\textbf{Ghana Paata } \newline

1. अप॑ हन्ति ह॒न्त्यपाप॑ ह॒न्तीन्द्रा॒ये न्द्रा॑य ह॒न्त्यपाप॑ ह॒न्तीन्द्रा॑य । \newline
2. ह॒न्तीन्द्रा॒ये न्द्रा॑य हन्ति ह॒न्तीन्द्रा॑य त्रा॒त्रे त्रा॒त्र इन्द्रा॑य हन्ति ह॒न्तीन्द्रा॑य त्रा॒त्रे । \newline
3. इन्द्रा॑य त्रा॒त्रे त्रा॒त्र इन्द्रा॒ये न्द्रा॑य त्रा॒त्रे पु॑रो॒डाश॑म् पुरो॒डाश॑म् त्रा॒त्र इन्द्रा॒ये न्द्रा॑य त्रा॒त्रे पु॑रो॒डाश᳚म् । \newline
4. त्रा॒त्रे पु॑रो॒डाश॑म् पुरो॒डाश॑म् त्रा॒त्रे त्रा॒त्रे पु॑रो॒डाश॒ मेका॑दशकपाल॒ मेका॑दशकपालम् पुरो॒डाश॑म् त्रा॒त्रे त्रा॒त्रे पु॑रो॒डाश॒ मेका॑दशकपालम् । \newline
5. पु॒रो॒डाश॒ मेका॑दशकपाल॒ मेका॑दशकपालम् पुरो॒डाश॑म् पुरो॒डाश॒ मेका॑दशकपाल॒म् निर् णिरेका॑दशकपालम् पुरो॒डाश॑म् पुरो॒डाश॒ मेका॑दशकपाल॒म् निः । \newline
6. एका॑दशकपाल॒म् निर् णिरेका॑दशकपाल॒ मेका॑दशकपाल॒म् निर् व॑पेद् वपे॒न् निरेका॑दशकपाल॒ मेका॑दशकपाल॒म् निर् व॑पेत् । \newline
7. एका॑दशकपाल॒मित्येका॑दश - क॒पा॒ल॒म् । \newline
8. निर् व॑पेद् वपे॒न् निर् णिर् व॑पेद् ब॒द्धो ब॒द्धो व॑पे॒न् निर् णिर् व॑पेद् ब॒द्धः । \newline
9. व॒पे॒द् ब॒द्धो ब॒द्धो व॑पेद् वपेद् ब॒द्धो वा॑ वा ब॒द्धो व॑पेद् वपेद् ब॒द्धो वा᳚ । \newline
10. ब॒द्धो वा॑ वा ब॒द्धो ब॒द्धो वा॒ परि॑यत्तः॒ परि॑यत्तो वा ब॒द्धो ब॒द्धो वा॒ परि॑यत्तः । \newline
11. वा॒ परि॑यत्तः॒ परि॑यत्तो वा वा॒ परि॑यत्तो वा वा॒ परि॑यत्तो वा वा॒ परि॑यत्तो वा । \newline
12. परि॑यत्तो वा वा॒ परि॑यत्तः॒ परि॑यत्तो॒ वेन्द्र॒ मिन्द्रं॑ ॅवा॒ परि॑यत्तः॒ परि॑यत्तो॒ वेन्द्र᳚म् । \newline
13. परि॑यत्त॒ इति॒ परि॑ - य॒त्तः॒ । \newline
14. वेन्द्र॒ मिन्द्रं॑ ॅवा॒ वेन्द्र॑ मे॒वैवे न्द्रं॑ ॅवा॒ वेन्द्र॑ मे॒व । \newline
15. इन्द्र॑ मे॒वैवे न्द्र॒ मिन्द्र॑ मे॒व त्रा॒तार॑म् त्रा॒तार॑ मे॒वे न्द्र॒ मिन्द्र॑ मे॒व त्रा॒तार᳚म् । \newline
16. ए॒व त्रा॒तार॑म् त्रा॒तार॑ मे॒वैव त्रा॒तारꣳ॒॒ स्वेन॒ स्वेन॑ त्रा॒तार॑ मे॒वैव त्रा॒तारꣳ॒॒ स्वेन॑ । \newline
17. त्रा॒तारꣳ॒॒ स्वेन॒ स्वेन॑ त्रा॒तार॑म् त्रा॒तारꣳ॒॒ स्वेन॑ भाग॒धेये॑न भाग॒धेये॑न॒ स्वेन॑ त्रा॒तार॑म् त्रा॒तारꣳ॒॒ स्वेन॑ भाग॒धेये॑न । \newline
18. स्वेन॑ भाग॒धेये॑न भाग॒धेये॑न॒ स्वेन॒ स्वेन॑ भाग॒धेये॒नोपोप॑ भाग॒धेये॑न॒ स्वेन॒ स्वेन॑ भाग॒धेये॒नोप॑ । \newline
19. भा॒ग॒धेये॒नोपोप॑ भाग॒धेये॑न भाग॒धेये॒नोप॑ धावति धाव॒ त्युप॑ भाग॒धेये॑न भाग॒धेये॒नोप॑ धावति । \newline
20. भा॒ग॒धेये॒नेति॑ भाग - धेये॑न । \newline
21. उप॑ धावति धाव॒ त्युपोप॑ धावति॒ स स धा॑व॒ त्युपोप॑ धावति॒ सः । \newline
22. धा॒व॒ति॒ स स धा॑वति धावति॒ स ए॒वैव स धा॑वति धावति॒ स ए॒व । \newline
23. स ए॒वैव स स ए॒वैन॑ मेन मे॒व स स ए॒वैन᳚म् । \newline
24. ए॒वैन॑ मेन मे॒वैवैन॑म् त्रायते त्रायत एन मे॒वैवैन॑म् त्रायते । \newline
25. ए॒न॒म् त्रा॒य॒ते॒ त्रा॒य॒त॒ ए॒न॒ मे॒न॒म् त्रा॒य॒त॒ इन्द्रा॒ये न्द्रा॑य त्रायत एन मेनम् त्रायत॒ इन्द्रा॑य । \newline
26. त्रा॒य॒त॒ इन्द्रा॒ये न्द्रा॑य त्रायते त्रायत॒ इन्द्रा॑यार्काश्वमे॒धव॑ते ऽर्काश्वमे॒धव॑त॒ इन्द्रा॑य त्रायते त्रायत॒ इन्द्रा॑यार्काश्वमे॒धव॑ते । \newline
27. इन्द्रा॑यार्काश्वमे॒धव॑ते ऽर्काश्वमे॒धव॑त॒ इन्द्रा॒ये न्द्रा॑यार्काश्वमे॒धव॑ते पुरो॒डाश॑म् पुरो॒डाश॑ मर्काश्वमे॒धव॑त॒ इन्द्रा॒ये न्द्रा॑यार्काश्वमे॒धव॑ते पुरो॒डाश᳚म् । \newline
28. अ॒र्का॒श्व॒मे॒धव॑ते पुरो॒डाश॑म् पुरो॒डाश॑ मर्काश्वमे॒धव॑ते ऽर्काश्वमे॒धव॑ते पुरो॒डाश॒ मेका॑दशकपाल॒ मेका॑दशकपालम् पुरो॒डाश॑ मर्काश्वमे॒धव॑ते ऽर्काश्वमे॒धव॑ते पुरो॒डाश॒ मेका॑दशकपालम् । \newline
29. अ॒र्का॒श्व॒मे॒धव॑त॒ इत्य॑र्काश्वमे॒ध - व॒ते॒ । \newline
30. पु॒रो॒डाश॒ मेका॑दशकपाल॒ मेका॑दशकपालम् पुरो॒डाश॑म् पुरो॒डाश॒ मेका॑दशकपाल॒म् निर् णिरेका॑दशकपालम् पुरो॒डाश॑म् पुरो॒डाश॒ मेका॑दशकपाल॒म् निः । \newline
31. एका॑दशकपाल॒म् निर् णिरेका॑दशकपाल॒ मेका॑दशकपाल॒म् निर् व॑पेद् वपे॒न् निरेका॑दशकपाल॒ मेका॑दशकपाल॒म् निर् व॑पेत् । \newline
32. एका॑दशकपाल॒मित्येका॑दश - क॒पा॒ल॒म् । \newline
33. निर् व॑पेद् वपे॒न् निर् णिर् व॑पे॒द् यं ॅयं ॅव॑पे॒न् निर् णिर् व॑पे॒द् यम् । \newline
34. व॒पे॒द् यं ॅयं ॅव॑पेद् वपे॒द् यम् म॑हाय॒ज्ञो म॑हाय॒ज्ञो यं ॅव॑पेद् वपे॒द् यम् म॑हाय॒ज्ञ्ः । \newline
35. यम् म॑हाय॒ज्ञो म॑हाय॒ज्ञो यं ॅयम् म॑हाय॒ज्ञो न न म॑हाय॒ज्ञो यं ॅयम् म॑हाय॒ज्ञो न । \newline
36. म॒हा॒य॒ज्ञो न न म॑हाय॒ज्ञो म॑हाय॒ज्ञो नोप॒नमे॑ दुप॒नमे॒न् न म॑हाय॒ज्ञो म॑हाय॒ज्ञो नोप॒नमे᳚त् । \newline
37. म॒हा॒य॒ज्ञ् इति॑ महा - य॒ज्ञ्ः । \newline
38. नोप॒नमे॑ दुप॒नमे॒न् न नोप॒नमे॑ दे॒ते ए॒ते उ॑प॒नमे॒न् न नोप॒नमे॑ दे॒ते । \newline
39. उ॒प॒नमे॑ दे॒ते ए॒ते उ॑प॒नमे॑ दुप॒नमे॑ दे॒ते वै वा ए॒ते उ॑प॒नमे॑ दुप॒नमे॑ दे॒ते वै । \newline
40. उ॒प॒नमे॒दित्यु॑प - नमे᳚त् । \newline
41. ए॒ते वै वा ए॒ते ए॒ते वै म॑हाय॒ज्ञ्स्य॑ महाय॒ज्ञ्स्य॒ वा ए॒ते ए॒ते वै म॑हाय॒ज्ञ्स्य॑ । \newline
42. ए॒ते इत्ये॒ते । \newline
43. वै म॑हाय॒ज्ञ्स्य॑ महाय॒ज्ञ्स्य॒ वै वै म॑हाय॒ज्ञ्स्यान्त्ये॒ अन्त्ये॑ महाय॒ज्ञ्स्य॒ वै वै म॑हाय॒ज्ञ्स्यान्त्ये᳚ । \newline
44. म॒हा॒य॒ज्ञ्स्यान्त्ये॒ अन्त्ये॑ महाय॒ज्ञ्स्य॑ महाय॒ज्ञ्स्यान्त्ये॑ त॒नू त॒नू अन्त्ये॑ महाय॒ज्ञ्स्य॑ महाय॒ज्ञ्स्यान्त्ये॑ त॒नू । \newline
45. म॒हा॒य॒ज्ञ्स्येति॑ महा - य॒ज्ञ्स्य॑ । \newline
46. अन्त्ये॑ त॒नू त॒नू अन्त्ये॒ अन्त्ये॑ त॒नू यद् यत् त॒नू अन्त्ये॒ अन्त्ये॑ त॒नू यत् । \newline
47. अन्त्ये॒ इत्यन्त्ये᳚ । \newline
48. त॒नू यद् यत् त॒नू त॒नू यद॑र्काश्वमे॒धा व॑र्काश्वमे॒धौ यत् त॒नू त॒नू यद॑र्काश्वमे॒धौ । \newline
49. त॒नू इति॑ त॒नू । \newline
50. यद॑र्काश्वमे॒धा व॑र्काश्वमे॒धौ यद् यद॑र्काश्वमे॒धा विन्द्र॒ मिन्द्र॑ मर्काश्वमे॒धौ यद् यद॑र्काश्वमे॒धा विन्द्र᳚म् । \newline
51. अ॒र्का॒श्व॒मे॒धा विन्द्र॒ मिन्द्र॑ मर्काश्वमे॒धा व॑र्काश्वमे॒धा विन्द्र॑ मे॒वैवे न्द्र॑ मर्काश्वमे॒धा व॑र्काश्वमे॒धा विन्द्र॑ मे॒व । \newline
52. अ॒र्का॒श्व॒मे॒धावित्य॑र्क - अ॒श्व॒मे॒धौ । \newline
53. इन्द्र॑ मे॒वैवे न्द्र॒ मिन्द्र॑ मे॒वार्का᳚श्वमे॒धव॑न्त मर्काश्वमे॒धव॑न्त मे॒वे न्द्र॒ मिन्द्र॑ मे॒वार्का᳚श्वमे॒धव॑न्तम् । \newline
54. ए॒वार्का᳚श्वमे॒धव॑न्त मर्काश्वमे॒धव॑न्त मे॒वैवार्का᳚श्वमे॒धव॑न्तꣳ॒॒ स्वेन॒ स्वेना᳚र्काश्वमे॒धव॑न्त मे॒वैवार्का᳚श्वमे॒धव॑न्तꣳ॒॒ स्वेन॑ । \newline
55. अ॒र्का॒श्व॒मे॒धव॑न्तꣳ॒॒ स्वेन॒ स्वेना᳚र्काश्वमे॒धव॑न्त मर्काश्वमे॒धव॑न्तꣳ॒॒ स्वेन॑ भाग॒धेये॑न भाग॒धेये॑न॒ स्वेना᳚र्काश्वमे॒धव॑न्त मर्काश्वमे॒धव॑न्तꣳ॒॒ स्वेन॑ भाग॒धेये॑न । \newline
56. अ॒र्का॒श्व॒मे॒धव॑न्त॒मित्य॑र्काश्वमे॒ध - व॒न्त॒म् । \newline
57. स्वेन॑ भाग॒धेये॑न भाग॒धेये॑न॒ स्वेन॒ स्वेन॑ भाग॒धेये॒नोपोप॑ भाग॒धेये॑न॒ स्वेन॒ स्वेन॑ भाग॒धेये॒नोप॑ । \newline
58. भा॒ग॒धेये॒नोपोप॑ भाग॒धेये॑न भाग॒धेये॒नोप॑ धावति धाव॒ त्युप॑ भाग॒धेये॑न भाग॒धेये॒नोप॑ धावति । \newline
59. भा॒ग॒धेये॒नेति॑ भाग - धेये॑न । \newline
60. उप॑ धावति धाव॒ त्युपोप॑ धावति॒ स स धा॑व॒ त्युपोप॑ धावति॒ सः । \newline
61. धा॒व॒ति॒ स स धा॑वति धावति॒ स ए॒वैव स धा॑वति धावति॒ स ए॒व । \newline
62. स ए॒वैव स स ए॒वास्मा॑ अस्मा ए॒व स स ए॒वास्मै᳚ । \newline
63. ए॒वास्मा॑ अस्मा ए॒वैवास्मा॑ अन्त॒तो᳚ ऽन्त॒तो᳚ ऽस्मा ए॒वैवास्मा॑ अन्त॒तः । \newline
64. अ॒स्मा॒ अ॒न्त॒तो᳚ ऽन्त॒तो᳚ ऽस्मा अस्मा अन्त॒तो म॑हाय॒ज्ञ्म् म॑हाय॒ज्ञ् म॑न्त॒तो᳚ ऽस्मा अस्मा अन्त॒तो म॑हाय॒ज्ञ्म् । \newline
65. अ॒न्त॒तो म॑हाय॒ज्ञ्म् म॑हाय॒ज्ञ् म॑न्त॒तो᳚ ऽन्त॒तो म॑हाय॒ज्ञ्म् च्या॑वयति च्यावयति महाय॒ज्ञ् म॑न्त॒तो᳚ ऽन्त॒तो म॑हाय॒ज्ञ्म् च्या॑वयति । \newline
66. म॒हा॒य॒ज्ञ्म् च्या॑वयति च्यावयति महाय॒ज्ञ्म् म॑हाय॒ज्ञ्म् च्या॑वय॒ त्युपोप॑ च्यावयति महाय॒ज्ञ्म् म॑हाय॒ज्ञ्म् च्या॑वय॒ त्युप॑ । \newline
67. म॒हा॒य॒ज्ञ्मिति॑ महा - य॒ज्ञ्म् । \newline
68. च्या॒व॒य॒ त्युपोप॑ च्यावयति च्यावय॒ त्युपै॑न मेन॒ मुप॑ च्यावयति च्यावय॒ त्युपै॑नम् । \newline
69. उपै॑न मेन॒ मुपोपै॑नम् महाय॒ज्ञो म॑हाय॒ज्ञ् ए॑न॒ मुपोपै॑नम् महाय॒ज्ञ्ः । \newline
70. ए॒न॒म् म॒हा॒य॒ज्ञो म॑हाय॒ज्ञ् ए॑न मेनम् महाय॒ज्ञो न॑मति नमति महाय॒ज्ञ् ए॑न मेनम् महाय॒ज्ञो न॑मति । \newline
71. म॒हा॒य॒ज्ञो न॑मति नमति महाय॒ज्ञो म॑हाय॒ज्ञो न॑मति । \newline
72. म॒हा॒य॒ज्ञ् इति॑ महा - य॒ज्ञ्ः । \newline
73. न॒म॒तीति॑ नमति । \newline
\pagebreak
\markright{ TS 2.2.8.1  \hfill https://www.vedavms.in \hfill}
\addcontentsline{toc}{section}{ TS 2.2.8.1 }
\section*{ TS 2.2.8.1 }

\textbf{TS 2.2.8.1 } \newline
\textbf{Samhita Paata} \newline

इन्द्रा॒यान्वृ॑जवे पुरो॒डाश॒मेका॑दशकपालं॒ निर्व॑पे॒द् ग्राम॑काम॒ इन्द्र॑मे॒वान्वृ॑जुꣳ॒॒ स्वेन॑ भाग॒धेये॒नोप॑ धावति॒ स ए॒वास्मै॑ सजा॒ताननु॑कान् करोति ग्रा॒म्ये॑व भ॑वतीन्द्रा॒ण्यै च॒रुं निर्व॑पे॒द्यस्य॒ सेनाऽसꣳ॑शितेव॒ स्यादि॑न्द्रा॒णी वै सेना॑यै दे॒वते᳚न्द्रा॒णीमे॒व स्वेन॑ भाग॒धेये॒नोप॑ धावति॒ सैवास्य॒ सेनाꣳ॒॒ सꣳ श्य॑ति॒ बल्ब॑जा॒नपी॒-  [  ] \newline

\textbf{Pada Paata} \newline

इ॒न्द्रा॑य । अन्वृ॑जव॒ इत्यनु॑ - ऋ॒ज॒वे॒ । पु॒रो॒डाश᳚म् । एका॑दशकपाल॒मित्येका॑दश - क॒पा॒ल॒म् । निरिति॑ । व॒पे॒त् । ग्राम॑काम॒ इति॒ ग्राम॑ - का॒मः॒ । इन्द्र᳚म् । ए॒व । अन्वृ॑जु॒मित्यनु॑-ऋ॒जु॒म् । स्वेन॑ । भा॒ग॒धेये॒नेति॑ भाग-धेये॑न । उपेति॑ । धा॒व॒ति॒ । सः । ए॒व । अ॒स्मै॒ । स॒जा॒तानिति॑ स - जा॒तान् । अनु॑का॒नित्यनु॑ - का॒न् । क॒रो॒ति॒ । ग्रा॒मी । ए॒व । भ॒व॒ति॒ । इ॒न्द्रा॒ण्यै । च॒रुम् । निरिति॑ । व॒पे॒त् । यस्य॑ । सेना᳚ । असꣳ॑शि॒तेत्यसं᳚ - शि॒ता॒ । इ॒व॒ । स्यात् । इ॒न्द्रा॒णी । वै । सेना॑यै । दे॒वता᳚ । इ॒न्द्रा॒णीम् । ए॒व । स्वेन॑ । भा॒ग॒धेये॒नेति॑ भाग - धेये॑न । उपेति॑ । धा॒व॒ति॒ । सा । ए॒व । अ॒स्य॒ । सेना᳚म् । समिति॑ । श्य॒ति॒ । बल्ब॑जान् । अपीति॑ ।  \newline


\textbf{Krama Paata} \newline

इन्द्रा॒यान्वृ॑जवे । अन्वृ॑जवे पुरो॒डाश᳚म् । अन्वृ॑जव॒ इत्यनु॑ - ऋ॒ज॒वे॒ । पु॒रो॒डाश॒मेका॑दशकालम् । एका॑दशकपाल॒म् निः । एका॑दशकपाल॒मित्येका॑दश - क॒पा॒ल॒म् । निर् व॑पेत् । व॒पे॒द् ग्राम॑कामः । ग्राम॑काम॒ इन्द्र᳚म् । ग्राम॑काम॒ इति॒ ग्राम॑ - का॒मः॒ । इन्द्र॑मे॒व । ए॒वान्वृ॑जुम् । अन्वृ॑जुꣳ॒॒ स्वेन॑ । अन्वृ॑जु॒मित्यनु॑ - ऋ॒जु॒म् । स्वेन॑ भाग॒धेये॑न । भा॒ग॒धेये॒नोप॑ । भा॒ग॒धेये॒नेति॑ भाग - धेये॑न । उप॑ धावति । धा॒व॒ति॒ सः । स ए॒व । ए॒वास्मै᳚ । अ॒स्मै॒ स॒जा॒तान् । स॒जा॒ताननु॑कान् । स॒जा॒तानिति॑ स - जा॒तान् । अनु॑कान् करोति । अनु॑का॒नित्यनु॑ - का॒न्॒ । क॒रो॒ति॒ ग्रा॒मी । ग्रा॒म्ये॑व । ए॒व भ॑वति । भ॒व॒ती॒न्द्रा॒ण्यै । इ॒न्द्रा॒ण्यै च॒रुम् । च॒रुम् निः । निर् व॑पेत् । व॒पे॒द् यस्य॑ । यस्य॒ सेना᳚ । सेना ऽसꣳ॑शिता । असꣳ॑शितेव । असꣳ॑शि॒तेत्यसं᳚ - शि॒ता॒ । इ॒व॒ स्यात् । स्यादि॑न्द्रा॒णी । इ॒न्द्रा॒णी वै । वै सेना॑यै । सेना॑यै दे॒वता᳚ । दे॒वते᳚न्द्रा॒णीम् । इ॒न्द्रा॒णीमे॒व । ए॒व स्वेन॑ । स्वेन॑ भाग॒धेये॑न । भा॒ग॒धेये॒नोप॑ । भा॒ग॒धेये॒नेति॑ भाग - धेये॑न । उप॑ धावति । धा॒व॒ति॒ सा । सैव । ए॒वास्य॑ । अ॒स्य॒ सेना᳚म् । सेनाꣳ॒॒ सम् । सꣳ श्य॑ति । श्य॒ति॒ बल्ब॑जान् । बल्ब॑जा॒नपि॑ । अपी॒द्ध्मे \newline

\textbf{Jatai Paata} \newline

1. इन्द्रा॒या न्वृ॑ज॒वे ऽन्वृ॑जव॒ इन्द्रा॒ येन्द्रा॒या न्वृ॑जवे । \newline
2. अन्वृ॑जवे पुरो॒डाश॑म् पुरो॒डाश॒ मन्वृ॑ज॒वे ऽन्वृ॑जवे पुरो॒डाश᳚म् । \newline
3. अन्वृ॑जव॒ इत्यनु॑ - ऋ॒ज॒वे॒ । \newline
4. पु॒रो॒डाश॒ मेका॑दशकपाल॒ मेका॑दशकपालम् पुरो॒डाश॑म् पुरो॒डाश॒ मेका॑दशकपालम् । \newline
5. एका॑दशकपाल॒म् निर् णिरेका॑दशकपाल॒ मेका॑दशकपाल॒म् निः । \newline
6. एका॑दशकपाल॒मित्येका॑दश - क॒पा॒ल॒म् । \newline
7. निर् व॑पेद् वपे॒न् निर् णिर् व॑पेत् । \newline
8. व॒पे॒द् ग्राम॑कामो॒ ग्राम॑कामो वपेद् वपे॒द् ग्राम॑कामः । \newline
9. ग्राम॑काम॒ इन्द्र॒ मिन्द्र॒म् ग्राम॑कामो॒ ग्राम॑काम॒ इन्द्र᳚म् । \newline
10. ग्राम॑काम॒ इति॒ ग्राम॑ - का॒मः॒ । \newline
11. इन्द्र॑ मे॒वैवे न्द्र॒ मिन्द्र॑ मे॒व । \newline
12. ए॒वा न्वृ॑जु॒ मन्वृ॑जु मे॒वैवा न्वृ॑जुम् । \newline
13. अन्वृ॑जुꣳ॒॒ स्वेन॒ स्वेना न्वृ॑जु॒ मन्वृ॑जुꣳ॒॒ स्वेन॑ । \newline
14. अन्वृ॑जु॒मित्यनु॑ - ऋ॒जु॒म् । \newline
15. स्वेन॑ भाग॒धेये॑न भाग॒धेये॑न॒ स्वेन॒ स्वेन॑ भाग॒धेये॑न । \newline
16. भा॒ग॒धेये॒नोपोप॑ भाग॒धेये॑न भाग॒धेये॒नोप॑ । \newline
17. भा॒ग॒धेये॒नेति॑ भाग - धेये॑न । \newline
18. उप॑ धावति धाव॒ त्युपोप॑ धावति । \newline
19. धा॒व॒ति॒ स स धा॑वति धावति॒ सः । \newline
20. स ए॒वैव स स ए॒व । \newline
21. ए॒वास्मा॑ अस्मा ए॒वैवास्मै᳚ । \newline
22. अ॒स्मै॒ स॒जा॒तान् थ्स॑जा॒ता न॑स्मा अस्मै सजा॒तान् । \newline
23. स॒जा॒ता ननु॑का॒ ननु॑कान् थ्सजा॒तान् थ्स॑जा॒ता ननु॑कान् । \newline
24. स॒जा॒तानिति॑ स - जा॒तान् । \newline
25. अनु॑कान् करोति करो॒ त्यनु॑का॒ ननु॑कान् करोति । \newline
26. अनु॑का॒नित्यनु॑ - का॒न् । \newline
27. क॒रो॒ति॒ ग्रा॒मी ग्रा॒मी क॑रोति करोति ग्रा॒मी । \newline
28. ग्रा॒म्ये॑वैव ग्रा॒मी ग्रा॒म्ये॑व । \newline
29. ए॒व भ॑वति भव त्ये॒वैव भ॑वति । \newline
30. भ॒व॒ती॒न्द्रा॒ण्या इ॑न्द्रा॒ण्यै भ॑वति भवतीन्द्रा॒ण्यै । \newline
31. इ॒न्द्रा॒ण्यै च॒रुम् च॒रु मि॑न्द्रा॒ण्या इ॑न्द्रा॒ण्यै च॒रुम् । \newline
32. च॒रुम् निर् णिश् च॒रुम् च॒रुम् निः । \newline
33. निर् व॑पेद् वपे॒न् निर् णिर् व॑पेत् । \newline
34. व॒पे॒द् यस्य॒ यस्य॑ वपेद् वपे॒द् यस्य॑ । \newline
35. यस्य॒ सेना॒ सेना॒ यस्य॒ यस्य॒ सेना᳚ । \newline
36. सेना ऽसꣳ॑शि॒ता ऽसꣳ॑शिता॒ सेना॒ सेना ऽसꣳ॑शिता । \newline
37. असꣳ॑शितेवे॒ वासꣳ॑शि॒ता ऽसꣳ॑शितेव । \newline
38. असꣳ॑शि॒तेत्यसं᳚ - शि॒ता॒ । \newline
39. इ॒व॒ स्याथ् स्या दि॑वे व॒ स्यात् । \newline
40. स्या दि॑न्द्रा॒णी न्द्रा॒णी स्याथ् स्या दि॑न्द्रा॒णी । \newline
41. इ॒न्द्रा॒णी वै वा इ॑न्द्रा॒णी न्द्रा॒णी वै । \newline
42. वै सेना॑यै॒ सेना॑यै॒ वै वै सेना॑यै । \newline
43. सेना॑यै दे॒वता॑ दे॒वता॒ सेना॑यै॒ सेना॑यै दे॒वता᳚ । \newline
44. दे॒वते᳚ न्द्रा॒णी मि॑न्द्रा॒णीम् दे॒वता॑ दे॒वते᳚ न्द्रा॒णीम् । \newline
45. इ॒न्द्रा॒णी मे॒वैवे न्द्रा॒णी मि॑न्द्रा॒णी मे॒व । \newline
46. ए॒व स्वेन॒ स्वे नै॒वैव स्वेन॑ । \newline
47. स्वेन॑ भाग॒धेये॑न भाग॒धेये॑न॒ स्वेन॒ स्वेन॑ भाग॒धेये॑न । \newline
48. भा॒ग॒धेये॒नोपोप॑ भाग॒धेये॑न भाग॒धेये॒नोप॑ । \newline
49. भा॒ग॒धेये॒नेति॑ भाग - धेये॑न । \newline
50. उप॑ धावति धाव॒ त्युपोप॑ धावति । \newline
51. धा॒व॒ति॒ सा सा धा॑वति धावति॒ सा । \newline
52. सैवैव सा सैव । \newline
53. ए॒वास्या᳚ स्यै॒ वैवास्य॑ । \newline
54. अ॒स्य॒ सेनाꣳ॒॒ सेना॑ मस्यास्य॒ सेना᳚म् । \newline
55. सेनाꣳ॒॒ सꣳ सꣳ सेनाꣳ॒॒ सेनाꣳ॒॒ सम् । \newline
56. सꣳ श्य॑ति श्यति॒ सꣳ सꣳ श्य॑ति । \newline
57. श्य॒ति॒ बल्ब॑जा॒न् बल्ब॑जाञ् छ्‌यति श्यति॒ बल्ब॑जान् । \newline
58. बल्ब॑जा॒ नप्यपि॒ बल्ब॑जा॒न् बल्ब॑जा॒ नपि॑ । \newline
59. अपी॒द्ध्म इ॒द्ध्मे ऽप्यपी॒द्ध्मे । \newline

\textbf{Ghana Paata } \newline

1. इन्द्रा॒यान्वृ॑ज॒वे ऽन्वृ॑जव॒ इन्द्रा॒ येन्द्रा॑यान्वृ॑जवे पुरो॒डाश॑म् पुरो॒डाश॒ मन्वृ॑जव॒ इन्द्रा॒
येन्द्रा॑यान्वृ॑जवे पुरो॒डाश᳚म् । \newline
2. अन्वृ॑जवे पुरो॒डाश॑म् पुरो॒डाश॒ मन्वृ॑ज॒वे ऽन्वृ॑जवे पुरो॒डाश॒ मेका॑दशकपाल॒ मेका॑दशकपालम् पुरो॒डाश॒ मन्वृ॑ज॒वे ऽन्वृ॑जवे पुरो॒डाश॒ मेका॑दशकपालम् । \newline
3. अन्वृ॑जव॒ इत्यनु॑ - ऋ॒ज॒वे॒ । \newline
4. पु॒रो॒डाश॒ मेका॑दशकपाल॒ मेका॑दशकपालम् पुरो॒डाश॑म् पुरो॒डाश॒ मेका॑दशकपाल॒म् निर् णिरेका॑दशकपालम् पुरो॒डाश॑म् पुरो॒डाश॒ मेका॑दशकपाल॒म् निः । \newline
5. एका॑दशकपाल॒म् निर् णिरेका॑दशकपाल॒ मेका॑दशकपाल॒म् निर् व॑पेद् वपे॒न् निरेका॑दशकपाल॒ मेका॑दशकपाल॒म् निर् व॑पेत् । \newline
6. एका॑दशकपाल॒मित्येका॑दश - क॒पा॒ल॒म् । \newline
7. निर् व॑पेद् वपे॒न् निर् णिर् व॑पे॒द् ग्राम॑कामो॒ ग्राम॑कामो वपे॒न् निर् णिर् व॑पे॒द् ग्राम॑कामः । \newline
8. व॒पे॒द् ग्राम॑कामो॒ ग्राम॑कामो वपेद् वपे॒द् ग्राम॑काम॒ इन्द्र॒ मिन्द्र॒म् ग्राम॑कामो वपेद् वपे॒द् ग्राम॑काम॒ इन्द्र᳚म् । \newline
9. ग्राम॑काम॒ इन्द्र॒ मिन्द्र॒म् ग्राम॑कामो॒ ग्राम॑काम॒ इन्द्र॑ मे॒वैवे न्द्र॒म् ग्राम॑कामो॒ ग्राम॑काम॒ इन्द्र॑ मे॒व । \newline
10. ग्राम॑काम॒ इति॒ ग्राम॑ - का॒मः॒ । \newline
11. इन्द्र॑ मे॒वैवे न्द्र॒ मिन्द्र॑ मे॒वान्वृ॑जु॒ मन्वृ॑जु मे॒वे न्द्र॒ मिन्द्र॑ मे॒वान्वृ॑जुम् । \newline
12. ए॒वान्वृ॑जु॒ मन्वृ॑जु मे॒वैवान्वृ॑जुꣳ॒॒ स्वेन॒ स्वेनान्वृ॑जु मे॒वैवान्वृ॑जुꣳ॒॒ स्वेन॑ । \newline
13. अन्वृ॑जुꣳ॒॒ स्वेन॒ स्वेनान्वृ॑जु॒ मन्वृ॑जुꣳ॒॒ स्वेन॑ भाग॒धेये॑न भाग॒धेये॑न॒ स्वेनान्वृ॑जु॒ मन्वृ॑जुꣳ॒॒ स्वेन॑ भाग॒धेये॑न । \newline
14. अन्वृ॑जु॒मित्यनु॑ - ऋ॒जु॒म् । \newline
15. स्वेन॑ भाग॒धेये॑न भाग॒धेये॑न॒ स्वेन॒ स्वेन॑ भाग॒धेये॒नोपोप॑ भाग॒धेये॑न॒ स्वेन॒ स्वेन॑ भाग॒धेये॒नोप॑ । \newline
16. भा॒ग॒धेये॒नोपोप॑ भाग॒धेये॑न भाग॒धेये॒नोप॑ धावति धाव॒ त्युप॑ भाग॒धेये॑न भाग॒धेये॒नोप॑ धावति । \newline
17. भा॒ग॒धेये॒नेति॑ भाग - धेये॑न । \newline
18. उप॑ धावति धाव॒ त्युपोप॑ धावति॒ स स धा॑व॒ त्युपोप॑ धावति॒ सः । \newline
19. धा॒व॒ति॒ स स धा॑वति धावति॒ स ए॒वैव स धा॑वति धावति॒ स ए॒व । \newline
20. स ए॒वैव स स ए॒वास्मा॑ अस्मा ए॒व स स ए॒वास्मै᳚ । \newline
21. ए॒वास्मा॑ अस्मा ए॒वैवास्मै॑ सजा॒तान् थ्स॑जा॒ता न॑स्मा ए॒वैवास्मै॑ सजा॒तान् । \newline
22. अ॒स्मै॒ स॒जा॒तान् थ्स॑जा॒ता न॑स्मा अस्मै सजा॒ता ननु॑का॒ ननु॑कान् थ्सजा॒ता न॑स्मा अस्मै सजा॒ता ननु॑कान् । \newline
23. स॒जा॒ता ननु॑का॒ ननु॑कान् थ्सजा॒तान् थ्स॑जा॒ता ननु॑कान् करोति करो॒ त्यनु॑कान् थ्सजा॒तान् थ्स॑जा॒ता ननु॑कान् करोति । \newline
24. स॒जा॒तानिति॑ स - जा॒तान् । \newline
25. अनु॑कान् करोति करो॒ त्यनु॑का॒ ननु॑कान् करोति ग्रा॒मी ग्रा॒मी क॑रो॒ त्यनु॑का॒ ननु॑कान् करोति ग्रा॒मी । \newline
26. अनु॑का॒नित्यनु॑ - का॒न् । \newline
27. क॒रो॒ति॒ ग्रा॒मी ग्रा॒मी क॑रोति करोति ग्रा॒म्ये॑वैव ग्रा॒मी क॑रोति करोति ग्रा॒म्ये॑व । \newline
28. ग्रा॒म्ये॑वैव ग्रा॒मी ग्रा॒म्ये॑व भ॑वति भव त्ये॒व ग्रा॒मी ग्रा॒म्ये॑व भ॑वति । \newline
29. ए॒व भ॑वति भव त्ये॒वैव भ॑वतीन्द्रा॒ण्या इ॑न्द्रा॒ण्यै भ॑व त्ये॒वैव भ॑वतीन्द्रा॒ण्यै । \newline
30. भ॒व॒ती॒न्द्रा॒ण्या इ॑न्द्रा॒ण्यै भ॑वति भवतीन्द्रा॒ण्यै च॒रुम् च॒रु मि॑न्द्रा॒ण्यै भ॑वति भवतीन्द्रा॒ण्यै च॒रुम् । \newline
31. इ॒न्द्रा॒ण्यै च॒रुम् च॒रु मि॑न्द्रा॒ण्या इ॑न्द्रा॒ण्यै च॒रुम् निर् णिश् च॒रु मि॑न्द्रा॒ण्या इ॑न्द्रा॒ण्यै च॒रुम् निः । \newline
32. च॒रुम् निर् णिश् च॒रुम् च॒रुम् निर् व॑पेद् वपे॒न् निश् च॒रुम् च॒रुम् निर् व॑पेत् । \newline
33. निर् व॑पेद् वपे॒न् निर् णिर् व॑पे॒द् यस्य॒ यस्य॑ वपे॒न् निर् णिर् व॑पे॒द् यस्य॑ । \newline
34. व॒पे॒द् यस्य॒ यस्य॑ वपेद् वपे॒द् यस्य॒ सेना॒ सेना॒ यस्य॑ वपेद् वपे॒द् यस्य॒ सेना᳚ । \newline
35. यस्य॒ सेना॒ सेना॒ यस्य॒ यस्य॒ सेना ऽसꣳ॑शि॒ता ऽसꣳ॑शिता॒ सेना॒ यस्य॒ यस्य॒ सेना ऽसꣳ॑शिता । \newline
36. सेना ऽसꣳ॑शि॒ता ऽसꣳ॑शिता॒ सेना॒ सेना ऽसꣳ॑शितेवे॒ वासꣳ॑शिता॒ सेना॒ सेना ऽसꣳ॑शितेव । \newline
37. असꣳ॑शितेवे॒ वासꣳ॑शि॒ता ऽसꣳ॑शितेव॒ स्याथ् स्या दि॒वासꣳ॑शि॒ता ऽसꣳ॑शितेव॒ स्यात् । \newline
38. असꣳ॑शि॒तेत्यसं᳚ - शि॒ता॒ । \newline
39. इ॒व॒ स्याथ् स्या दि॑वे व॒ स्या दि॑न्द्रा॒णीन्द्रा॒णी स्या दि॑वे व॒ स्या दि॑न्द्रा॒णी । \newline
40. स्या दि॑न्द्रा॒णीन्द्रा॒णी स्याथ् स्या दि॑न्द्रा॒णी वै वा इ॑न्द्रा॒णी स्याथ् स्या दि॑न्द्रा॒णी वै । \newline
41. इ॒न्द्रा॒णी वै वा इ॑न्द्रा॒णीन्द्रा॒णी वै सेना॑यै॒ सेना॑यै॒ वा इ॑न्द्रा॒णीन्द्रा॒णी वै सेना॑यै । \newline
42. वै सेना॑यै॒ सेना॑यै॒ वै वै सेना॑यै दे॒वता॑ दे॒वता॒ सेना॑यै॒ वै वै सेना॑यै दे॒वता᳚ । \newline
43. सेना॑यै दे॒वता॑ दे॒वता॒ सेना॑यै॒ सेना॑यै दे॒वते᳚न्द्रा॒णी मि॑न्द्रा॒णीम् दे॒वता॒ सेना॑यै॒ सेना॑यै दे॒वते᳚न्द्रा॒णीम् । \newline
44. दे॒वते᳚न्द्रा॒णी मि॑न्द्रा॒णीम् दे॒वता॑ दे॒वते᳚न्द्रा॒णी मे॒वैवे न्द्रा॒णीम् दे॒वता॑ दे॒वते᳚न्द्रा॒णी मे॒व । \newline
45. इ॒न्द्रा॒णी मे॒वैवे न्द्रा॒णी मि॑न्द्रा॒णी मे॒व स्वेन॒ स्वेनै॒वे न्द्रा॒णी मि॑न्द्रा॒णी मे॒व स्वेन॑ । \newline
46. ए॒व स्वेन॒ स्वेनै॒वैव स्वेन॑ भाग॒धेये॑न भाग॒धेये॑न॒ स्वेनै॒वैव स्वेन॑ भाग॒धेये॑न । \newline
47. स्वेन॑ भाग॒धेये॑न भाग॒धेये॑न॒ स्वेन॒ स्वेन॑ भाग॒धेये॒नोपोप॑ भाग॒धेये॑न॒ स्वेन॒ स्वेन॑ भाग॒धेये॒नोप॑ । \newline
48. भा॒ग॒धेये॒नोपोप॑ भाग॒धेये॑न भाग॒धेये॒नोप॑ धावति धाव॒ त्युप॑ भाग॒धेये॑न भाग॒धेये॒नोप॑ धावति । \newline
49. भा॒ग॒धेये॒नेति॑ भाग - धेये॑न । \newline
50. उप॑ धावति धाव॒ त्युपोप॑ धावति॒ सा सा धा॑व॒ त्युपोप॑ धावति॒ सा । \newline
51. धा॒व॒ति॒ सा सा धा॑वति धावति॒ सैवैव सा धा॑वति धावति॒ सैव । \newline
52. सैवैव सा सैवास्या᳚ स्यै॒व सा सैवास्य॑ । \newline
53. ए॒वास्या᳚ स्यै॒वैवास्य॒ सेनाꣳ॒॒ सेना॑ मस्यै॒वै वास्य॒ सेना᳚म् । \newline
54. अ॒स्य॒ सेनाꣳ॒॒ सेना॑ मस्यास्य॒ सेनाꣳ॒॒ सꣳ सꣳ सेना॑ मस्यास्य॒ सेनाꣳ॒॒ सम् । \newline
55. सेनाꣳ॒॒ सꣳ सꣳ सेनाꣳ॒॒ सेनाꣳ॒॒ सꣳ श्य॑ति श्यति॒ सꣳ सेनाꣳ॒॒ सेनाꣳ॒॒ सꣳ श्य॑ति । \newline
56. सꣳ श्य॑ति श्यति॒ सꣳ सꣳ श्य॑ति॒ बल्ब॑जा॒न् बल्ब॑जाञ् छ्‌यति॒ सꣳ सꣳ श्य॑ति॒ बल्ब॑जान् । \newline
57. श्य॒ति॒ बल्ब॑जा॒न् बल्ब॑जाञ् छ्‌यति श्यति॒ बल्ब॑जा॒ नप्यपि॒ बल्ब॑जाञ् छ्‌यति श्यति॒ बल्ब॑जा॒ नपि॑ । \newline
58. बल्ब॑जा॒ नप्यपि॒ बल्ब॑जा॒न् बल्ब॑जा॒ नपी॒द्ध्म इ॒द्ध्मे ऽपि॒ बल्ब॑जा॒न् बल्ब॑जा॒ नपी॒द्ध्मे । \newline
59. अपी॒द्ध्म इ॒द्ध्मे ऽप्यपी॒द्ध्मे सꣳ स मि॒द्ध्मे ऽप्यपी॒द्ध्मे सम् । \newline
\pagebreak
\markright{ TS 2.2.8.2  \hfill https://www.vedavms.in \hfill}
\addcontentsline{toc}{section}{ TS 2.2.8.2 }
\section*{ TS 2.2.8.2 }

\textbf{TS 2.2.8.2 } \newline
\textbf{Samhita Paata} \newline

-द्ध्मे सं न॑ह्ये॒द्-गौ-र्यत्राधि॑ष्कन्ना॒न्यमे॑ह॒त् ततो॒ बल्ब॑जा॒ उद॑तिष्ठ॒न् गवा॑मे॒वैनं॑ न्या॒यम॑पि॒नीय॒ गा वे॑दय॒तीन्द्रा॑य मन्यु॒मते॒ मन॑स्वते पुरो॒डाश॒मेका॑दशकपालं॒ निर्व॑पेथ् संग्रा॒मे संॅय॑त्त इन्द्रि॒येण॒ वै म॒न्युना॒ मन॑सा संग्रा॒मं ज॑य॒तीन्द्र॑मे॒व म॑न्यु॒मन्तं॒ मन॑स्वन्तꣳ॒॒ स्वेन॑ भाग॒धेये॒नोप॑ धावति॒ स ए॒वास्मि॑न्निन्द्रि॒यं म॒न्युं मनो॑ दधाति॒ जय॑ति॒ तꣳ - [  ] \newline

\textbf{Pada Paata} \newline

इ॒द्ध्मे । समिति॑ । न॒ह्ये॒त् । गौः । यत्र॑ । अधि॑ष्क॒न्नेत्यधि॑ - स्क॒न्ना॒ । न्यमे॑ह॒दिति॑ नि - अमे॑हत् । ततः॑ । बल्ब॑जाः । उदिति॑ । अ॒ति॒ष्ठ॒न्न् । गवा᳚म् । ए॒व । ए॒न॒म् । न्या॒यमिति॑ नि - आ॒यम् । अ॒पि॒नीयेत्य॑पि -नीय॑ । गाः । वे॒द॒य॒ति॒ । इन्द्रा॑य । म॒न्यु॒मत॒ इति॑ मन्यु - मते᳚ । मन॑स्वते । पु॒रो॒डाश᳚म् । एका॑दशकपाल॒मित्येका॑दश - क॒पा॒ल॒म् । निरिति॑ । व॒पे॒त् । स॒ग्रां॒म इति॑ सं - ग्रा॒मे । संॅय॑त्त॒ इति॒ सं - य॒त्ते॒ । इ॒न्द्रि॒येण॑ । वै । म॒न्युना᳚ । मन॑सा । स॒ग्रां॒ममिति॑ सं - ग्रा॒मम् । ज॒य॒ति॒ । इन्द्र᳚म् । ए॒व । म॒न्यु॒मन्त॒मिति॑ मन्यु - मन्त᳚म् । मन॑स्वन्तम् । स्वेन॑ । भा॒ग॒धेये॒नेति॑ भाग-धेये॑न । उपेति॑ । धा॒व॒ति॒ । सः । ए॒व । अ॒स्मि॒न्न् । इ॒न्द्रि॒यम् । म॒न्युम् । मनः॑ । द॒धा॒ति॒ । जय॑ति । तम् ।  \newline


\textbf{Krama Paata} \newline

इ॒द्ध्मे सम् । सन्न॑ह्येत् । न॒ह्ये॒द्,गौः । गौर्,यत्र॑ । यत्राधि॑ष्कन्ना । अधि॑ष्कन्ना॒ न्यमे॑हत् । अधि॑ष्क॒न्नेत्यधि॑ - स्क॒न्ना॒ । न्यमे॑ह॒त्,ततः॑ । न्यमे॑ह॒दिति॑ नि - अमे॑हत् । ततो॒ बल्ब॑जाः । बल्ब॑जा॒ उत् । उद॑तिष्ठन्न् । अ॒ति॒ष्ठ॒न् गवा᳚म् । गवा॑मे॒व । ए॒वैन᳚म् । ए॒न॒म् न्या॒यम् । न्या॒यम॑पि॒नीय॑ । न्या॒यमिति॑ नि - आ॒यम् । अ॒पि॒नीय॒ गाः । अ॒पि॒नीयेयत्य॑पि - नीय॑ । गा वे॑दयति । वे॒द॒य॒तीन्द्रा॑य । इन्द्रा॑य मन्यु॒मते᳚ । म॒न्यु॒मते॒ मन॑स्वते । म॒न्यु॒मत॒ इति॑ मन्यु - मते᳚ । मन॑स्वते पुरो॒डाश᳚म् । पु॒रो॒डाश॒मेका॑दशकपालम् । एका॑दशकपाल॒म् निः । एका॑दशकपाल॒मित्येका॑दश - क॒पा॒ल॒म् । निर् व॑पेत् । व॒पे॒थ् स॒ङ्ग्रा॒मे । स॒ङ्ग्रा॒मे सम्ॅय॑त्ते । स॒ङ्ग्रा॒म इति॑ सं - ग्रा॒मे । सम्ॅय॑त्त इन्द्रि॒येण॑ । सम्ॅय॑त्त॒ इति॒ सं - य॒त्ते॒ । इ॒न्द्रि॒येण॒ वै । वै म॒न्युना᳚ । म॒न्युना॒ मन॑सा । मन॑सा सङ्ग्रा॒मम् । स॒ङ्ग्रा॒मम् ज॑यति । स॒ङ्ग्रा॒ममिति॑ सं - ग्रा॒मम् । ज॒य॒तीन्द्र᳚म् । इन्द्र॑मे॒व । ए॒व म॑न्यु॒मन्त᳚म् । म॒न्यु॒मन्त॒म् मन॑स्वन्तम् । म॒न्यु॒मन्त॒मिति॑ मन्यु - मन्त᳚म् । मन॑स्वन्तꣳ॒॒ स्वेन॑ । स्वेन॑ भाग॒धेये॑न । भा॒ग॒धेये॒नोप॑ । भा॒ग॒धेये॒नेति॑ भाग - धेये॑न । उप॑ धावति । धा॒व॒ति॒ सः । स ए॒व । ए॒वास्मिन्न्॑ । अ॒स्मि॒न्नि॒न्द्रि॒यम् । इ॒न्द्रि॒यम् म॒न्युम् । म॒न्युम् मनः॑ । मनो॑ दधाति । द॒धा॒ति॒ जय॑ति । जय॑ति॒ तम् । तꣳ स॑ङ्ग्रा॒मम् \newline

\textbf{Jatai Paata} \newline

1. इ॒द्ध्मे सꣳ स मि॒द्ध्म इ॒द्ध्मे सम् । \newline
2. सन्न॑ह्येन् नह्ये॒थ् सꣳ सन्न॑ह्येत् । \newline
3. न॒ह्ये॒द् गौर् गौर् न॑ह्येन् नह्ये॒द् गौः । \newline
4. गौर् यत्र॒ यत्र॒ गौर् गौर् यत्र॑ । \newline
5. यत्राधि॑ष्क॒न्ना ऽधि॑ष्कन्ना॒ यत्र॒ यत्राधि॑ष्कन्ना । \newline
6. अधि॑ष्कन्ना॒ न्यमे॑ह॒न् न्यमे॑ह॒ दधि॑ष्क॒न्ना ऽधि॑ष्कन्ना॒ न्यमे॑हत् । \newline
7. अधि॑ष्क॒न्नेत्यधि॑ - स्क॒न्ना॒ । \newline
8. न्यमे॑ह॒त् तत॒ स्ततो॒ न्यमे॑ह॒न् न्यमे॑ह॒त् ततः॑ । \newline
9. न्यमे॑ह॒दिति॑ नि - अमे॑हत् । \newline
10. ततो॒ बल्ब॑जा॒ बल्ब॑जा॒ स्तत॒ स्ततो॒ बल्ब॑जाः । \newline
11. बल्ब॑जा॒ उदुद् बल्ब॑जा॒ बल्ब॑जा॒ उत् । \newline
12. उद॑तिष्ठन् नतिष्ठ॒न् नुदु द॑तिष्ठन्न् । \newline
13. अ॒ति॒ष्ठ॒न् गवा॒म् गवा॑ मतिष्ठन् नतिष्ठ॒न् गवा᳚म् । \newline
14. गवा॑ मे॒वैव गवा॒म् गवा॑ मे॒व । \newline
15. ए॒वैन॑ मेन मे॒वैवैन᳚म् । \newline
16. ए॒न॒म् न्या॒यम् न्या॒य मे॑न मेनम् न्या॒यम् । \newline
17. न्या॒य म॑पि॒नीया॑ पि॒नीय॑ न्या॒यम् न्या॒य म॑पि॒नीय॑ । \newline
18. न्या॒यमिति॑ नि - आ॒यम् । \newline
19. अ॒पि॒नीय॒ गा गा अ॑पि॒नीया॑ पि॒नीय॒ गाः । \newline
20. अ॒पि॒नीयेत्य॑पि - नीय॑ । \newline
21. गा वे॑दयति वेदयति॒ गा गा वे॑दयति । \newline
22. वे॒द॒य॒तीन्द्रा॒ये न्द्रा॑य वेदयति वेदय॒तीन्द्रा॑य । \newline
23. इन्द्रा॑य मन्यु॒मते॑ मन्यु॒मत॒ इन्द्रा॒ये न्द्रा॑य मन्यु॒मते᳚ । \newline
24. म॒न्यु॒मते॒ मन॑स्वते॒ मन॑स्वते मन्यु॒मते॑ मन्यु॒मते॒ मन॑स्वते । \newline
25. म॒न्यु॒मत॒ इति॑ मन्यु - मते᳚ । \newline
26. मन॑स्वते पुरो॒डाश॑म् पुरो॒डाश॒म् मन॑स्वते॒ मन॑स्वते पुरो॒डाश᳚म् । \newline
27. पु॒रो॒डाश॒ मेका॑दशकपाल॒ मेका॑दशकपालम् पुरो॒डाश॑म् पुरो॒डाश॒ मेका॑दशकपालम् । \newline
28. एका॑दशकपाल॒म् निर् णिरेका॑दशकपाल॒ मेका॑दशकपाल॒म् निः । \newline
29. एका॑दशकपाल॒मित्येका॑दश - क॒पा॒ल॒म् । \newline
30. निर् व॑पेद् वपे॒न् निर् णिर् व॑पेत् । \newline
31. व॒पे॒थ् स॒ङ्ग्रा॒मे स॑ङ्ग्रा॒मे व॑पेद् वपेथ् सङ्ग्रा॒मे । \newline
32. स॒ङ्ग्रा॒मे संॅय॑त्ते॒ संॅय॑त्ते सङ्ग्रा॒मे स॑ङ्ग्रा॒मे संॅय॑त्ते । \newline
33. स॒ङ्ग्रा॒म इति॑ सं - ग्रा॒मे । \newline
34. संॅय॑त्त इन्द्रि॒येणे᳚ न्द्रि॒येण॒ संॅय॑त्ते॒ संॅय॑त्त इन्द्रि॒येण॑ । \newline
35. संॅय॑त्त॒ इति॒ सं - य॒त्ते॒ । \newline
36. इ॒न्द्रि॒येण॒ वै वा इ॑न्द्रि॒येणे᳚ न्द्रि॒येण॒ वै । \newline
37. वै म॒न्युना॑ म॒न्युना॒ वै वै म॒न्युना᳚ । \newline
38. म॒न्युना॒ मन॑सा॒ मन॑सा म॒न्युना॑ म॒न्युना॒ मन॑सा । \newline
39. मन॑सा सङ्ग्रा॒मꣳ स॑ङ्ग्रा॒मम् मन॑सा॒ मन॑सा सङ्ग्रा॒मम् । \newline
40. स॒ङ्ग्रा॒मम् ज॑यति जयति सङ्ग्रा॒मꣳ स॑ङ्ग्रा॒मम् ज॑यति । \newline
41. स॒ङ्ग्रा॒ममिति॑ सं - ग्रा॒मम् । \newline
42. ज॒य॒तीन्द्र॒ मिन्द्र॑म् जयति जय॒तीन्द्र᳚म् । \newline
43. इन्द्र॑ मे॒वैवे न्द्र॒ मिन्द्र॑ मे॒व । \newline
44. ए॒व म॑न्यु॒मन्त॑म् मन्यु॒मन्त॑ मे॒वैव म॑न्यु॒मन्त᳚म् । \newline
45. म॒न्यु॒मन्त॒म् मन॑स्वन्त॒म् मन॑स्वन्तम् मन्यु॒मन्त॑म् मन्यु॒मन्त॒म् मन॑स्वन्तम् । \newline
46. म॒न्यु॒मन्त॒मिति॑ मन्यु - मन्त᳚म् । \newline
47. मन॑स्वन्तꣳ॒॒ स्वेन॒ स्वेन॒ मन॑स्वन्त॒म् मन॑स्वन्तꣳ॒॒ स्वेन॑ । \newline
48. स्वेन॑ भाग॒धेये॑न भाग॒धेये॑न॒ स्वेन॒ स्वेन॑ भाग॒धेये॑न । \newline
49. भा॒ग॒धेये॒नोपोप॑ भाग॒धेये॑न भाग॒धेये॒नोप॑ । \newline
50. भा॒ग॒धेये॒नेति॑ भाग - धेये॑न । \newline
51. उप॑ धावति धाव॒ त्युपोप॑ धावति । \newline
52. धा॒व॒ति॒ स स धा॑वति धावति॒ सः । \newline
53. स ए॒वैव स स ए॒व । \newline
54. ए॒वास्मि॑न् नस्मिन् ने॒वैवास्मिन्न्॑ । \newline
55. अ॒स्मि॒न् नि॒न्द्रि॒य मि॑न्द्रि॒य म॑स्मिन् नस्मिन् निन्द्रि॒यम् । \newline
56. इ॒न्द्रि॒यम् म॒न्युम् म॒न्यु मि॑न्द्रि॒य मि॑न्द्रि॒यम् म॒न्युम् । \newline
57. म॒न्युम् मनो॒ मनो॑ म॒न्युम् म॒न्युम् मनः॑ । \newline
58. मनो॑ दधाति दधाति॒ मनो॒ मनो॑ दधाति । \newline
59. द॒धा॒ति॒ जय॑ति॒ जय॑ति दधाति दधाति॒ जय॑ति । \newline
60. जय॑ति॒ तम् तम् जय॑ति॒ जय॑ति॒ तम् । \newline
61. तꣳ स॑ङ्ग्रा॒मꣳ स॑ङ्ग्रा॒मम् तम् तꣳ स॑ङ्ग्रा॒मम् । \newline

\textbf{Ghana Paata } \newline

1. इ॒द्ध्मे सꣳ स मि॒द्ध्म इ॒द्ध्मे सम् न॑ह्येन् नह्ये॒थ् स मि॒द्ध्म इ॒द्ध्मे सम् न॑ह्येत् । \newline
2. सम् न॑ह्येन् नह्ये॒थ् सꣳ सम् न॑ह्ये॒द् गौर् गौर् न॑ह्ये॒थ् सꣳ सम् न॑ह्ये॒द् गौः । \newline
3. न॒ह्ये॒द् गौर् गौर् न॑ह्येन् नह्ये॒द् गौर् यत्र॒ यत्र॒ गौर् न॑ह्येन् नह्ये॒द् गौर् यत्र॑ । \newline
4. गौर् यत्र॒ यत्र॒ गौर् गौर् यत्राधि॑ष्क॒न्ना ऽधि॑ष्कन्ना॒ यत्र॒ गौर् गौर् यत्राधि॑ष्कन्ना । \newline
5. यत्राधि॑ष्क॒न्ना ऽधि॑ष्कन्ना॒ यत्र॒ यत्राधि॑ष्कन्ना॒ न्यमे॑ह॒न् न्यमे॑ह॒ दधि॑ष्कन्ना॒ यत्र॒ यत्राधि॑ष्कन्ना॒ न्यमे॑हत् । \newline
6. अधि॑ष्कन्ना॒ न्यमे॑ह॒न् न्यमे॑ह॒ दधि॑ष्क॒न्ना ऽधि॑ष्कन्ना॒ न्यमे॑ह॒त् तत॒ स्ततो॒ न्यमे॑ह॒ दधि॑ष्क॒न्ना ऽधि॑ष्कन्ना॒ न्यमे॑ह॒त् ततः॑ । \newline
7. अधि॑ष्क॒न्नेत्यधि॑ - स्क॒न्ना॒ । \newline
8. न्यमे॑ह॒त् तत॒ स्ततो॒ न्यमे॑ह॒न् न्यमे॑ह॒त् ततो॒ बल्ब॑जा॒ बल्ब॑जा॒ स्ततो॒ न्यमे॑ह॒न् न्यमे॑ह॒त् ततो॒ बल्ब॑जाः । \newline
9. न्यमे॑ह॒दिति॑ नि - अमे॑हत् । \newline
10. ततो॒ बल्ब॑जा॒ बल्ब॑जा॒ स्तत॒ स्ततो॒ बल्ब॑जा॒ उदुद् बल्ब॑जा॒ स्तत॒ स्ततो॒ बल्ब॑जा॒ उत् । \newline
11. बल्ब॑जा॒ उदुद् बल्ब॑जा॒ बल्ब॑जा॒ उद॑तिष्ठन् नतिष्ठ॒न् नुद् बल्ब॑जा॒ बल्ब॑जा॒ उद॑तिष्ठन्न् । \newline
12. उद॑तिष्ठन् नतिष्ठ॒न् नुदुद॑तिष्ठ॒न् गवा॒म् गवा॑ मतिष्ठ॒न् नुदुद॑तिष्ठ॒न् गवा᳚म् । \newline
13. अ॒ति॒ष्ठ॒न् गवा॒म् गवा॑ मतिष्ठन् नतिष्ठ॒न् गवा॑ मे॒वैव गवा॑ मतिष्ठन् नतिष्ठ॒न् गवा॑ मे॒व । \newline
14. गवा॑ मे॒वैव गवा॒म् गवा॑ मे॒वैन॑ मेन मे॒व गवा॒म् गवा॑ मे॒वैन᳚म् । \newline
15. ए॒वैन॑ मेन मे॒वैवैन॑म् न्या॒यम् न्या॒य मे॑न मे॒वैवैन॑म् न्या॒यम् । \newline
16. ए॒न॒म् न्या॒यम् न्या॒य मे॑न मेनम् न्या॒य म॑पि॒नीया॑ पि॒नीय॑ न्या॒य मे॑न मेनम् न्या॒य म॑पि॒नीय॑ । \newline
17. न्या॒य म॑पि॒नीया॑ पि॒नीय॑ न्या॒यम् न्या॒य म॑पि॒नीय॒ गा गा अ॑पि॒नीय॑ न्या॒यम् न्या॒य म॑पि॒नीय॒ गाः । \newline
18. न्या॒यमिति॑ नि - आ॒यम् । \newline
19. अ॒पि॒नीय॒ गा गा अ॑पि॒नीया॑ पि॒नीय॒ गा वे॑दयति वेदयति॒ गा अ॑पि॒नीया॑ पि॒नीय॒ गा वे॑दयति । \newline
20. अ॒पि॒नीयेत्य॑पि - नीय॑ । \newline
21. गा वे॑दयति वेदयति॒ गा गा वे॑दय॒तीन्द्रा॒ये न्द्रा॑य वेदयति॒ गा गा वे॑दय॒तीन्द्रा॑य । \newline
22. वे॒द॒य॒तीन्द्रा॒ये न्द्रा॑य वेदयति वेदय॒तीन्द्रा॑य मन्यु॒मते॑ मन्यु॒मत॒ इन्द्रा॑य वेदयति वेदय॒तीन्द्रा॑य मन्यु॒मते᳚ । \newline
23. इन्द्रा॑य मन्यु॒मते॑ मन्यु॒मत॒ इन्द्रा॒ये न्द्रा॑य मन्यु॒मते॒ मन॑स्वते॒ मन॑स्वते मन्यु॒मत॒ इन्द्रा॒ये न्द्रा॑य मन्यु॒मते॒ मन॑स्वते । \newline
24. म॒न्यु॒मते॒ मन॑स्वते॒ मन॑स्वते मन्यु॒मते॑ मन्यु॒मते॒ मन॑स्वते पुरो॒डाश॑म् पुरो॒डाश॒म् मन॑स्वते मन्यु॒मते॑ मन्यु॒मते॒ मन॑स्वते पुरो॒डाश᳚म् । \newline
25. म॒न्यु॒मत॒ इति॑ मन्यु - मते᳚ । \newline
26. मन॑स्वते पुरो॒डाश॑म् पुरो॒डाश॒म् मन॑स्वते॒ मन॑स्वते पुरो॒डाश॒ मेका॑दशकपाल॒ मेका॑दशकपालम् पुरो॒डाश॒म् मन॑स्वते॒ मन॑स्वते पुरो॒डाश॒ मेका॑दशकपालम् । \newline
27. पु॒रो॒डाश॒ मेका॑दशकपाल॒ मेका॑दशकपालम् पुरो॒डाश॑म् पुरो॒डाश॒ मेका॑दशकपाल॒म् निर् णिरेका॑दशकपालम् पुरो॒डाश॑म् पुरो॒डाश॒ मेका॑दशकपाल॒म् निः । \newline
28. एका॑दशकपाल॒म् निर् णिरेका॑दशकपाल॒ मेका॑दशकपाल॒म् निर् व॑पेद् वपे॒न् निरेका॑दशकपाल॒ मेका॑दशकपाल॒म् निर् व॑पेत् । \newline
29. एका॑दशकपाल॒मित्येका॑दश - क॒पा॒ल॒म् । \newline
30. निर् व॑पेद् वपे॒न् निर् णिर् व॑पेथ् सङ्ग्रा॒मे स॑ङ्ग्रा॒मे व॑पे॒न् निर् णिर् व॑पेथ् सङ्ग्रा॒मे । \newline
31. व॒पे॒थ् स॒ङ्ग्रा॒मे स॑ङ्ग्रा॒मे व॑पेद् वपेथ् सङ्ग्रा॒मे संॅय॑त्ते॒ संॅय॑त्ते सङ्ग्रा॒मे व॑पेद् वपेथ् सङ्ग्रा॒मे संॅय॑त्ते । \newline
32. स॒ङ्ग्रा॒मे संॅय॑त्ते॒ संॅय॑त्ते सङ्ग्रा॒मे स॑ङ्ग्रा॒मे संॅय॑त्त इन्द्रि॒येणे᳚ न्द्रि॒येण॒ संॅय॑त्ते सङ्ग्रा॒मे स॑ङ्ग्रा॒मे संॅय॑त्त इन्द्रि॒येण॑ । \newline
33. स॒ङ्ग्रा॒म इति॑ सं - ग्रा॒मे । \newline
34. संॅय॑त्त इन्द्रि॒येणे᳚ न्द्रि॒येण॒ संॅय॑त्ते॒ संॅय॑त्त इन्द्रि॒येण॒ वै वा इ॑न्द्रि॒येण॒ संॅय॑त्ते॒ संॅय॑त्त इन्द्रि॒येण॒ वै । \newline
35. संॅय॑त्त॒ इति॒ सं - य॒त्ते॒ । \newline
36. इ॒न्द्रि॒येण॒ वै वा इ॑न्द्रि॒येणे᳚ न्द्रि॒येण॒ वै म॒न्युना॑ म॒न्युना॒ वा इ॑न्द्रि॒येणे᳚ न्द्रि॒येण॒ वै म॒न्युना᳚ । \newline
37. वै म॒न्युना॑ म॒न्युना॒ वै वै म॒न्युना॒ मन॑सा॒ मन॑सा म॒न्युना॒ वै वै म॒न्युना॒ मन॑सा । \newline
38. म॒न्युना॒ मन॑सा॒ मन॑सा म॒न्युना॑ म॒न्युना॒ मन॑सा सङ्ग्रा॒मꣳ स॑ङ्ग्रा॒मम् मन॑सा म॒न्युना॑ म॒न्युना॒ मन॑सा सङ्ग्रा॒मम् । \newline
39. मन॑सा सङ्ग्रा॒मꣳ स॑ङ्ग्रा॒मम् मन॑सा॒ मन॑सा सङ्ग्रा॒मम् ज॑यति जयति सङ्ग्रा॒मम् मन॑सा॒ मन॑सा सङ्ग्रा॒मम् ज॑यति । \newline
40. स॒ङ्ग्रा॒मम् ज॑यति जयति सङ्ग्रा॒मꣳ स॑ङ्ग्रा॒मम् ज॑य॒तीन्द्र॒ मिन्द्र॑म् जयति सङ्ग्रा॒मꣳ स॑ङ्ग्रा॒मम् ज॑य॒तीन्द्र᳚म् । \newline
41. स॒ङ्ग्रा॒ममिति॑ सं - ग्रा॒मम् । \newline
42. ज॒य॒तीन्द्र॒ मिन्द्र॑म् जयति जय॒तीन्द्र॑ मे॒वैवे न्द्र॑म् जयति जय॒तीन्द्र॑ मे॒व । \newline
43. इन्द्र॑ मे॒वैवे न्द्र॒ मिन्द्र॑ मे॒व म॑न्यु॒मन्त॑म् मन्यु॒मन्त॑ मे॒वे न्द्र॒ मिन्द्र॑ मे॒व म॑न्यु॒मन्त᳚म् । \newline
44. ए॒व म॑न्यु॒मन्त॑म् मन्यु॒मन्त॑ मे॒वैव म॑न्यु॒मन्त॒म् मन॑स्वन्त॒म् मन॑स्वन्तम् मन्यु॒मन्त॑ मे॒वैव म॑न्यु॒मन्त॒म् मन॑स्वन्तम् । \newline
45. म॒न्यु॒मन्त॒म् मन॑स्वन्त॒म् मन॑स्वन्तम् मन्यु॒मन्त॑म् मन्यु॒मन्त॒म् मन॑स्वन्तꣳ॒॒ स्वेन॒ स्वेन॒ मन॑स्वन्तम् मन्यु॒मन्त॑म् मन्यु॒मन्त॒म् मन॑स्वन्तꣳ॒॒ स्वेन॑ । \newline
46. म॒न्यु॒मन्त॒मिति॑ मन्यु - मन्त᳚म् । \newline
47. मन॑स्वन्तꣳ॒॒ स्वेन॒ स्वेन॒ मन॑स्वन्त॒म् मन॑स्वन्तꣳ॒॒ स्वेन॑ भाग॒धेये॑न भाग॒धेये॑न॒ स्वेन॒ मन॑स्वन्त॒म् मन॑स्वन्तꣳ॒॒ स्वेन॑ भाग॒धेये॑न । \newline
48. स्वेन॑ भाग॒धेये॑न भाग॒धेये॑न॒ स्वेन॒ स्वेन॑ भाग॒धेये॒नोपोप॑ भाग॒धेये॑न॒ स्वेन॒ स्वेन॑ भाग॒धेये॒नोप॑ । \newline
49. भा॒ग॒धेये॒नोपोप॑ भाग॒धेये॑न भाग॒धेये॒नोप॑ धावति धाव॒ त्युप॑ भाग॒धेये॑न भाग॒धेये॒नोप॑ धावति । \newline
50. भा॒ग॒धेये॒नेति॑ भाग - धेये॑न । \newline
51. उप॑ धावति धाव॒ त्युपोप॑ धावति॒ स स धा॑व॒ त्युपोप॑ धावति॒ सः । \newline
52. धा॒व॒ति॒ स स धा॑वति धावति॒ स ए॒वैव स धा॑वति धावति॒ स ए॒व । \newline
53. स ए॒वैव स स ए॒वास्मि॑न् नस्मिन् ने॒व स स ए॒वास्मिन्न्॑ । \newline
54. ए॒वास्मि॑न् नस्मिन् ने॒वैवास्मि॑न् निन्द्रि॒य मि॑न्द्रि॒य म॑स्मिन् ने॒वैवास्मि॑न् निन्द्रि॒यम् । \newline
55. अ॒स्मि॒न् नि॒न्द्रि॒य मि॑न्द्रि॒य म॑स्मिन् नस्मिन् निन्द्रि॒यम् म॒न्युम् म॒न्यु मि॑न्द्रि॒य म॑स्मिन् नस्मिन् निन्द्रि॒यम् म॒न्युम् । \newline
56. इ॒न्द्रि॒यम् म॒न्युम् म॒न्यु मि॑न्द्रि॒य मि॑न्द्रि॒यम् म॒न्युम् मनो॒ मनो॑ म॒न्यु मि॑न्द्रि॒य मि॑न्द्रि॒यम् म॒न्युम् मनः॑ । \newline
57. म॒न्युम् मनो॒ मनो॑ म॒न्युम् म॒न्युम् मनो॑ दधाति दधाति॒ मनो॑ म॒न्युम् म॒न्युम् मनो॑ दधाति । \newline
58. मनो॑ दधाति दधाति॒ मनो॒ मनो॑ दधाति॒ जय॑ति॒ जय॑ति दधाति॒ मनो॒ मनो॑ दधाति॒ जय॑ति । \newline
59. द॒धा॒ति॒ जय॑ति॒ जय॑ति दधाति दधाति॒ जय॑ति॒ तम् तम् जय॑ति दधाति दधाति॒ जय॑ति॒ तम् । \newline
60. जय॑ति॒ तम् तम् जय॑ति॒ जय॑ति॒ तꣳ स॑ङ्ग्रा॒मꣳ स॑ङ्ग्रा॒मम् तम् जय॑ति॒ जय॑ति॒ तꣳ स॑ङ्ग्रा॒मम् । \newline
61. तꣳ स॑ङ्ग्रा॒मꣳ स॑ङ्ग्रा॒मम् तम् तꣳ स॑ङ्ग्रा॒म मे॒ता मे॒ताꣳ स॑ङ्ग्रा॒मम् तम् तꣳ स॑ङ्ग्रा॒म मे॒ताम् । \newline
\pagebreak
\markright{ TS 2.2.8.3  \hfill https://www.vedavms.in \hfill}
\addcontentsline{toc}{section}{ TS 2.2.8.3 }
\section*{ TS 2.2.8.3 }

\textbf{TS 2.2.8.3 } \newline
\textbf{Samhita Paata} \newline

स॑ग्रां॒ममे॒तामे॒व निर्व॑पे॒द्यो ह॒तम॑नाः स्व॒यं पा॑प इव॒ स्यादे॒तानि॒ हि वा ए॒तस्मा॒ दप॑क्रान्ता॒न्यथै॒ष ह॒तम॑नाः स्व॒यं पा॑प॒ इन्द्र॑मे॒व म॑न्यु॒मन्तं॒ मन॑स्वन्तꣳ॒॒ स्वेन॑ भाग॒धेये॒नोप॑ धावति॒ स ए॒वास्मि॑न्निन्द्रि॒यं म॒न्युं मनो॑ दधाति॒ न ह॒तम॑नाः स्व॒यं पा॑पो भव॒तीन्द्रा॑य दा॒त्रे पु॑रो॒डाश॒मेका॑दशकपालं॒ निर्व॑पे॒द्यः का॒मये॑त॒ दान॑कामा मे प्र॒जाः स्यु॒ - [  ] \newline

\textbf{Pada Paata} \newline

स॒ग्रां॒ममिति॑ सं -ग्रा॒मम् । ए॒ताम् । ए॒व । निरिति॑ । व॒पे॒त् । यः । ह॒तम॑ना॒ इति॑ ह॒त - म॒नाः॒ । स्व॒यंपा॑प॒ इति॑ स्व॒यं - पा॒पः॒ । इ॒व॒ । स्यात् । ए॒तानि॑ । हि । वै । ए॒तस्मा᳚त् । अप॑क्रान्ता॒नीत्यप॑ - क्रा॒न्ता॒नि॒ । अथ॑ । ए॒षः । ह॒तम॑ना॒ इति॑ ह॒त - म॒नाः॒ । स्व॒यंपा॑प॒ इति॑ स्व॒यं - पा॒पः॒ । इन्द्र᳚म् । ए॒व । म॒न्यु॒मन्त॒मिति॑ मन्यु - मन्त᳚म् । मन॑स्वन्तम् । स्वेन॑ । भा॒ग॒धेये॒नेति॑ भाग-धेये॑न । उपेति॑ । धा॒व॒ति॒ । सः । ए॒व । अ॒स्मि॒न्न् । इ॒न्द्रि॒यम् । म॒न्युम् । मनः॑ । द॒धा॒ति॒ । न । ह॒तम॑ना॒ इति॑ ह॒त-म॒नाः॒ । स्व॒यंपा॑प॒ इति॑ स्व॒यं-पा॒पः॒ । भ॒व॒ति॒ । इन्द्रा॑य । दा॒त्रे । पु॒रो॒डाश᳚म् । एका॑दशकपाल॒मित्येका॑दश - क॒पा॒ल॒म् । निरिति॑ । व॒पे॒त् । यः । का॒मये॑त । दान॑कामा॒ इति॒ दान॑ - का॒माः॒ । मे॒ । प्र॒जा इति॑ प्र - जाः । स्युः॒ ।  \newline


\textbf{Krama Paata} \newline

स॒ङ्ग्रा॒ममे॒ताम् । स॒ङ्ग्रा॒ममिति॑ सं - ग्रा॒मम् । ए॒तामे॒व । ए॒व निः । निर् व॑पेत् । व॒पे॒द् यः । यो ह॒तम॑नाः । ह॒तम॑नाः स्व॒यम्पा॑पः । ह॒तम॑ना॒ इति॑ ह॒त - म॒नाः॒ । स्व॒यम्पा॑प इव । स्व॒यम्पा॑प॒ इति॑ स्व॒यम् - पा॒पः॒ । इ॒व॒ स्यात् । स्या॒दे॒तानि॑ । ए॒तानि॒ हि । हि वै । वा ए॒तस्मा᳚त् । ए॒तस्मा॒दप॑क्रान्तानि । अप॑क्रान्ता॒न्यथ॑ । अप॑क्रान्ता॒नीत्यप॑ - क्रा॒न्ता॒नि॒ । अथै॒षः । ए॒ष ह॒तम॑नाः । ह॒तम॑नाः स्व॒यम्पा॑पः । ह॒तम॑ना॒ इति॑ ह॒त - म॒नाः॒ । स्व॒यम्पा॑प॒ इन्द्र᳚म् । स्व॒यम्पा॑प॒ इति॑ स्व॒यम् - पा॒पः॒ । इन्द्र॑मे॒व । ए॒व म॑न्यु॒मन्त᳚म् । म॒न्यु॒मन्त॒म्,मन॑स्वन्तम् । म॒न्यु॒मन्त॒मिति॑मन्यु - मन्त᳚म् । मन॑स्वन्तꣳ॒॒ स्वेन॑ । स्वेन॑ भाग॒धेये॑न । भा॒ग॒धेये॒नोप॑ । भा॒ग॒धेये॒नेति॑ भाग - धेये॑न । उप॑ धावति । धा॒व॒ति॒ सः । स ए॒व । ए॒वास्मिन्न्॑ । अ॒स्मि॒न्नि॒न्द्रि॒यम् । इ॒न्द्रि॒यम् म॒न्युम् । म॒न्युम् मनः॑ । मनो॑ दधाति । द॒धा॒ति॒ न । न ह॒तम॑नाः । ह॒तम॑नाः स्व॒यम्पा॑पः । ह॒तम॑ना॒ इति॑ ह॒त - म॒नाः॒ । स्व॒यम्पा॑पो भवति । स्व॒यम्पा॑प॒ इति॑ स्व॒यम् - पा॒पः॒ । भ॒व॒तीन्द्रा॑य । इन्द्रा॑य दा॒त्रे । दा॒त्रे पु॑रो॒डाश᳚म् । पु॒रो॒डाश॒मेका॑दशकपालम् । एका॑दशकपाल॒म् निः । एका॑दशकपाल॒मित्येका॑दश - क॒पा॒ल॒म् । निर् व॑पेत् । व॒पे॒द् यः । यः का॒मये॑त । का॒मये॑त॒ दान॑कामाः । दान॑कामा मे । दान॑कामा॒ इति॒ दान॑ - का॒माः॒ । मे॒ प्र॒जाः । प्र॒जाः स्युः॑ । प्र॒जा इति॑ प्र - जाः । स्यु॒रिति॑ \newline

\textbf{Jatai Paata} \newline

1. स॒ङ्ग्रा॒म मे॒ता मे॒ताꣳ स॑ङ्ग्रा॒मꣳ स॑ङ्ग्रा॒म मे॒ताम् । \newline
2. स॒ङ्ग्रा॒ममिति॑ सं - ग्रा॒मम् । \newline
3. ए॒ता मे॒वैवैता मे॒ता मे॒व । \newline
4. ए॒व निर् णि रे॒वैव निः । \newline
5. निर् व॑पेद् वपे॒न् निर् णिर् व॑पेत् । \newline
6. व॒पे॒द् यो यो व॑पेद् वपे॒द् यः । \newline
7. यो ह॒तम॑ना ह॒तम॑ना॒ यो यो ह॒तम॑नाः । \newline
8. ह॒तम॑नाः स्व॒यंपा॑पः स्व॒यंपा॑पो ह॒तम॑ना ह॒तम॑नाः स्व॒यंपा॑पः । \newline
9. ह॒तम॑ना॒ इति॑ ह॒त - म॒नाः॒ । \newline
10. स्व॒यंपा॑प इवे व स्व॒यंपा॑पः स्व॒यंपा॑प इव । \newline
11. स्व॒यंपा॑प॒ इति॑ स्व॒यं - पा॒पः॒ । \newline
12. इ॒व॒ स्याथ् स्या दि॑वे व॒ स्यात् । \newline
13. स्या दे॒ता न्ये॒तानि॒ स्याथ् स्या दे॒तानि॑ । \newline
14. ए॒तानि॒ हि ह्ये॑ता न्ये॒तानि॒ हि । \newline
15. हि वै वै हि हि वै । \newline
16. वा ए॒तस्मा॑ दे॒तस्मा॒द् वै वा ए॒तस्मा᳚त् । \newline
17. ए॒तस्मा॒ दप॑क्रान्ता॒ न्यप॑क्रान्ता न्ये॒तस्मा॑ दे॒तस्मा॒ दप॑क्रान्तानि । \newline
18. अप॑क्रान्ता॒ न्यथाथा प॑क्रान्ता॒ न्यप॑क्रान्ता॒ न्यथ॑ । \newline
19. अप॑क्रान्ता॒नीत्यप॑ - क्रा॒न्ता॒नि॒ । \newline
20. अथै॒ष ए॒षो ऽथा थै॒षः । \newline
21. ए॒ष ह॒तम॑ना ह॒तम॑ना ए॒ष ए॒ष ह॒तम॑नाः । \newline
22. ह॒तम॑नाः स्व॒यंपा॑पः स्व॒यंपा॑पो ह॒तम॑ना ह॒तम॑नाः स्व॒यंपा॑पः । \newline
23. ह॒तम॑ना॒ इति॑ ह॒त - म॒नाः॒ । \newline
24. स्व॒यंपा॑प॒ इन्द्र॒ मिन्द्रꣳ॑ स्व॒यंपा॑पः स्व॒यंपा॑प॒ इन्द्र᳚म् । \newline
25. स्व॒यंपा॑प॒ इति॑ स्व॒यं - पा॒पः॒ । \newline
26. इन्द्र॑ मे॒वैवे न्द्र॒ मिन्द्र॑ मे॒व । \newline
27. ए॒व म॑न्यु॒मन्त॑म् मन्यु॒मन्त॑ मे॒वैव म॑न्यु॒मन्त᳚म् । \newline
28. म॒न्यु॒मन्त॒म् मन॑स्वन्त॒म् मन॑स्वन्तम् मन्यु॒मन्त॑म् मन्यु॒मन्त॒म् मन॑स्वन्तम् । \newline
29. म॒न्यु॒मन्त॒मिति॑ मन्यु - मन्त᳚म् । \newline
30. मन॑स्वन्तꣳ॒॒ स्वेन॒ स्वेन॒ मन॑स्वन्त॒म् मन॑स्वन्तꣳ॒॒ स्वेन॑ । \newline
31. स्वेन॑ भाग॒धेये॑न भाग॒धेये॑न॒ स्वेन॒ स्वेन॑ भाग॒धेये॑न । \newline
32. भा॒ग॒धेये॒नोपोप॑ भाग॒धेये॑न भाग॒धेये॒नोप॑ । \newline
33. भा॒ग॒धेये॒नेति॑ भाग - धेये॑न । \newline
34. उप॑ धावति धाव॒ त्युपोप॑ धावति । \newline
35. धा॒व॒ति॒ स स धा॑वति धावति॒ सः । \newline
36. स ए॒वैव स स ए॒व । \newline
37. ए॒वास्मि॑न् नस्मिन् ने॒वैवास्मिन्न्॑ । \newline
38. अ॒स्मि॒न् नि॒न्द्रि॒य मि॑न्द्रि॒य म॑स्मिन् नस्मिन् निन्द्रि॒यम् । \newline
39. इ॒न्द्रि॒यम् म॒न्युम् म॒न्यु मि॑न्द्रि॒य मि॑न्द्रि॒यम् म॒न्युम् । \newline
40. म॒न्युम् मनो॒ मनो॑ म॒न्युम् म॒न्युम् मनः॑ । \newline
41. मनो॑ दधाति दधाति॒ मनो॒ मनो॑ दधाति । \newline
42. द॒धा॒ति॒ न न द॑धाति दधाति॒ न । \newline
43. न ह॒तम॑ना ह॒तम॑ना॒ न न ह॒तम॑नाः । \newline
44. ह॒तम॑नाः स्व॒यंपा॑पः स्व॒यंपा॑पो ह॒तम॑ना ह॒तम॑नाः स्व॒यंपा॑पः । \newline
45. ह॒तम॑ना॒ इति॑ ह॒त - म॒नाः॒ । \newline
46. स्व॒यंपा॑पो भवति भवति स्व॒यंपा॑पः स्व॒यंपा॑पो भवति । \newline
47. स्व॒यंपा॑प॒ इति॑ स्व॒यं - पा॒पः॒ । \newline
48. भ॒व॒तीन्द्रा॒ये न्द्रा॑य भवति भव॒तीन्द्रा॑य । \newline
49. इन्द्रा॑य दा॒त्रे दा॒त्र इन्द्रा॒ये न्द्रा॑य दा॒त्रे । \newline
50. दा॒त्रे पु॑रो॒डाश॑म् पुरो॒डाश॑म् दा॒त्रे दा॒त्रे पु॑रो॒डाश᳚म् । \newline
51. पु॒रो॒डाश॒ मेका॑दशकपाल॒ मेका॑दशकपालम् पुरो॒डाश॑म् पुरो॒डाश॒ मेका॑दशकपालम् । \newline
52. एका॑दशकपाल॒म् निर् णिरेका॑दशकपाल॒ मेका॑दशकपाल॒म् निः । \newline
53. एका॑दशकपाल॒मित्येका॑दश - क॒पा॒ल॒म् । \newline
54. निर् व॑पेद् वपे॒न् निर् णिर् व॑पेत् । \newline
55. व॒पे॒द् यो यो व॑पेद् वपे॒द् यः । \newline
56. यः का॒मये॑त का॒मये॑त॒ यो यः का॒मये॑त । \newline
57. का॒मये॑त॒ दान॑कामा॒ दान॑कामाः का॒मये॑त का॒मये॑त॒ दान॑कामाः । \newline
58. दान॑कामा मे मे॒ दान॑कामा॒ दान॑कामा मे । \newline
59. दान॑कामा॒ इति॒ दान॑ - का॒माः॒ । \newline
60. मे॒ प्र॒जाः प्र॒जा मे॑ मे प्र॒जाः । \newline
61. प्र॒जाः स्युः॑ स्युः प्र॒जाः प्र॒जाः स्युः॑ । \newline
62. प्र॒जा इति॑ प्र - जाः । \newline
63. स्यु॒ रितीति॑ स्युः स्यु॒ रिति॑ । \newline

\textbf{Ghana Paata } \newline

1. स॒ङ्ग्रा॒म मे॒ता मे॒ताꣳ स॑ङ्ग्रा॒मꣳ स॑ङ्ग्रा॒म मे॒ता मे॒वैवैताꣳ स॑ङ्ग्रा॒मꣳ स॑ङ्ग्रा॒म मे॒ता मे॒व । \newline
2. स॒ङ्ग्रा॒ममिति॑ सं - ग्रा॒मम् । \newline
3. ए॒ता मे॒वैवैता मे॒ता मे॒व निर् णि रे॒वैता मे॒ता मे॒व निः । \newline
4. ए॒व निर् णि रे॒वैव निर् व॑पेद् वपे॒न् नि रे॒वैव निर् व॑पेत् । \newline
5. निर् व॑पेद् वपे॒न् निर् णिर् व॑पे॒द् यो यो व॑पे॒न् निर् णिर् व॑पे॒द् यः । \newline
6. व॒पे॒द् यो यो व॑पेद् वपे॒द् यो ह॒तम॑ना ह॒तम॑ना॒ यो व॑पेद् वपे॒द् यो ह॒तम॑नाः । \newline
7. यो ह॒तम॑ना ह॒तम॑ना॒ यो यो ह॒तम॑नाः स्व॒यंपा॑पः स्व॒यंपा॑पो ह॒तम॑ना॒ यो यो ह॒तम॑नाः स्व॒यंपा॑पः । \newline
8. ह॒तम॑नाः स्व॒यंपा॑पः स्व॒यंपा॑पो ह॒तम॑ना ह॒तम॑नाः स्व॒यंपा॑प इवे व स्व॒यंपा॑पो ह॒तम॑ना ह॒तम॑नाः स्व॒यंपा॑प इव । \newline
9. ह॒तम॑ना॒ इति॑ ह॒त - म॒नाः॒ । \newline
10. स्व॒यंपा॑प इवे व स्व॒यंपा॑पः स्व॒यंपा॑प इव॒ स्याथ् स्यादि॑व स्व॒यंपा॑पः स्व॒यंपा॑प इव॒ स्यात् । \newline
11. स्व॒यंपा॑प॒ इति॑ स्व॒यं - पा॒पः॒ । \newline
12. इ॒व॒ स्याथ् स्यादि॑वे व॒ स्यादे॒ता न्ये॒तानि॒ स्यादि॑वे व॒ स्यादे॒तानि॑ । \newline
13. स्यादे॒ता न्ये॒तानि॒ स्याथ् स्यादे॒तानि॒ हि ह्ये॑तानि॒ स्याथ् स्यादे॒तानि॒ हि । \newline
14. ए॒तानि॒ हि ह्ये॑ता न्ये॒तानि॒ हि वै वै ह्ये॑ता न्ये॒तानि॒ हि वै । \newline
15. हि वै वै हि हि वा ए॒तस्मा॑ दे॒तस्मा॒द् वै हि हि वा ए॒तस्मा᳚त् । \newline
16. वा ए॒तस्मा॑ दे॒तस्मा॒द् वै वा ए॒तस्मा॒ दप॑क्रान्ता॒ न्यप॑क्रान्ता न्ये॒तस्मा॒द् वै वा ए॒तस्मा॒ दप॑क्रान्तानि । \newline
17. ए॒तस्मा॒ दप॑क्रान्ता॒ न्यप॑क्रान्ता न्ये॒तस्मा॑ दे॒तस्मा॒ दप॑क्रान्ता॒ न्यथाथा प॑क्रान्ता न्ये॒तस्मा॑ दे॒तस्मा॒ दप॑क्रान्ता॒ न्यथ॑ । \newline
18. अप॑क्रान्ता॒ न्यथाथा प॑क्रान्ता॒ न्यप॑क्रान्ता॒ न्यथै॒ष ए॒षो ऽथाप॑क्रान्ता॒ न्यप॑क्रान्ता॒ न्यथै॒षः । \newline
19. अप॑क्रान्ता॒नीत्यप॑ - क्रा॒न्ता॒नि॒ । \newline
20. अथै॒ष ए॒षो ऽथाथै॒ष ह॒तम॑ना ह॒तम॑ना ए॒षो ऽथाथै॒ष ह॒तम॑नाः । \newline
21. ए॒ष ह॒तम॑ना ह॒तम॑ना ए॒ष ए॒ष ह॒तम॑नाः स्व॒यंपा॑पः स्व॒यंपा॑पो ह॒तम॑ना ए॒ष ए॒ष ह॒तम॑नाः स्व॒यंपा॑पः । \newline
22. ह॒तम॑नाः स्व॒यंपा॑पः स्व॒यंपा॑पो ह॒तम॑ना ह॒तम॑नाः स्व॒यंपा॑प॒ इन्द्र॒ मिन्द्रꣳ॑ स्व॒यंपा॑पो ह॒तम॑ना ह॒तम॑नाः स्व॒यंपा॑प॒ इन्द्र᳚म् । \newline
23. ह॒तम॑ना॒ इति॑ ह॒त - म॒नाः॒ । \newline
24. स्व॒यंपा॑प॒ इन्द्र॒ मिन्द्रꣳ॑ स्व॒यंपा॑पः स्व॒यंपा॑प॒ इन्द्र॑ मे॒वैवे न्द्रꣳ॑ स्व॒यंपा॑पः स्व॒यंपा॑प॒ इन्द्र॑ मे॒व । \newline
25. स्व॒यंपा॑प॒ इति॑ स्व॒यं - पा॒पः॒ । \newline
26. इन्द्र॑ मे॒वैवे न्द्र॒ मिन्द्र॑ मे॒व म॑न्यु॒मन्त॑म् मन्यु॒मन्त॑ मे॒वे न्द्र॒ मिन्द्र॑ मे॒व म॑न्यु॒मन्त᳚म् । \newline
27. ए॒व म॑न्यु॒मन्त॑म् मन्यु॒मन्त॑ मे॒वैव म॑न्यु॒मन्त॒म् मन॑स्वन्त॒म् मन॑स्वन्तम् मन्यु॒मन्त॑ मे॒वैव म॑न्यु॒मन्त॒म् मन॑स्वन्तम् । \newline
28. म॒न्यु॒मन्त॒म् मन॑स्वन्त॒म् मन॑स्वन्तम् मन्यु॒मन्त॑म् मन्यु॒मन्त॒म् मन॑स्वन्तꣳ॒॒ स्वेन॒ स्वेन॒ मन॑स्वन्तम् मन्यु॒मन्त॑म् मन्यु॒मन्त॒म् मन॑स्वन्तꣳ॒॒ स्वेन॑ । \newline
29. म॒न्यु॒मन्त॒मिति॑ मन्यु - मन्त᳚म् । \newline
30. मन॑स्वन्तꣳ॒॒ स्वेन॒ स्वेन॒ मन॑स्वन्त॒म् मन॑स्वन्तꣳ॒॒ स्वेन॑ भाग॒धेये॑न भाग॒धेये॑न॒ स्वेन॒ मन॑स्वन्त॒म् मन॑स्वन्तꣳ॒॒ स्वेन॑ भाग॒धेये॑न । \newline
31. स्वेन॑ भाग॒धेये॑न भाग॒धेये॑न॒ स्वेन॒ स्वेन॑ भाग॒धेये॒नोपोप॑ भाग॒धेये॑न॒ स्वेन॒ स्वेन॑ भाग॒धेये॒नोप॑ । \newline
32. भा॒ग॒धेये॒नोपोप॑ भाग॒धेये॑न भाग॒धेये॒नोप॑ धावति धाव॒ त्युप॑ भाग॒धेये॑न भाग॒धेये॒नोप॑ धावति । \newline
33. भा॒ग॒धेये॒नेति॑ भाग - धेये॑न । \newline
34. उप॑ धावति धाव॒ त्युपोप॑ धावति॒ स स धा॑व॒ त्युपोप॑ धावति॒ सः । \newline
35. धा॒व॒ति॒ स स धा॑वति धावति॒ स ए॒वैव स धा॑वति धावति॒ स ए॒व । \newline
36. स ए॒वैव स स ए॒वास्मि॑न् नस्मिन् ने॒व स स ए॒वास्मिन्न्॑ । \newline
37. ए॒वास्मि॑न् नस्मिन् ने॒वैवास्मि॑न् निन्द्रि॒य मि॑न्द्रि॒य म॑स्मिन् ने॒वैवास्मि॑न् निन्द्रि॒यम् । \newline
38. अ॒स्मि॒न् नि॒न्द्रि॒य मि॑न्द्रि॒य म॑स्मिन् नस्मिन् निन्द्रि॒यम् म॒न्युम् म॒न्यु मि॑न्द्रि॒य म॑स्मिन् नस्मिन् निन्द्रि॒यम् म॒न्युम् । \newline
39. इ॒न्द्रि॒यम् म॒न्युम् म॒न्यु मि॑न्द्रि॒य मि॑न्द्रि॒यम् म॒न्युम् मनो॒ मनो॑ म॒न्यु मि॑न्द्रि॒य मि॑न्द्रि॒यम् म॒न्युम् मनः॑ । \newline
40. म॒न्युम् मनो॒ मनो॑ म॒न्युम् म॒न्युम् मनो॑ दधाति दधाति॒ मनो॑ म॒न्युम् म॒न्युम् मनो॑ दधाति । \newline
41. मनो॑ दधाति दधाति॒ मनो॒ मनो॑ दधाति॒ न न द॑धाति॒ मनो॒ मनो॑ दधाति॒ न । \newline
42. द॒धा॒ति॒ न न द॑धाति दधाति॒ न ह॒तम॑ना ह॒तम॑ना॒ न द॑धाति दधाति॒ न ह॒तम॑नाः । \newline
43. न ह॒तम॑ना ह॒तम॑ना॒ न न ह॒तम॑नाः स्व॒यंपा॑पः स्व॒यंपा॑पो ह॒तम॑ना॒ न न ह॒तम॑नाः स्व॒यंपा॑पः । \newline
44. ह॒तम॑नाः स्व॒यंपा॑पः स्व॒यंपा॑पो ह॒तम॑ना ह॒तम॑नाः स्व॒यंपा॑पो भवति भवति स्व॒यंपा॑पो ह॒तम॑ना ह॒तम॑नाः स्व॒यंपा॑पो भवति । \newline
45. ह॒तम॑ना॒ इति॑ ह॒त - म॒नाः॒ । \newline
46. स्व॒यंपा॑पो भवति भवति स्व॒यंपा॑पः स्व॒यंपा॑पो भव॒तीन्द्रा॒ये न्द्रा॑य भवति स्व॒यंपा॑पः स्व॒यंपा॑पो भव॒तीन्द्रा॑य । \newline
47. स्व॒यंपा॑प॒ इति॑ स्व॒यं - पा॒पः॒ । \newline
48. भ॒व॒तीन्द्रा॒ये न्द्रा॑य भवति भव॒तीन्द्रा॑य दा॒त्रे दा॒त्र इन्द्रा॑य भवति भव॒तीन्द्रा॑य दा॒त्रे । \newline
49. इन्द्रा॑य दा॒त्रे दा॒त्र इन्द्रा॒ये न्द्रा॑य दा॒त्रे पु॑रो॒डाश॑म् पुरो॒डाश॑म् दा॒त्र इन्द्रा॒ये न्द्रा॑य दा॒त्रे पु॑रो॒डाश᳚म् । \newline
50. दा॒त्रे पु॑रो॒डाश॑म् पुरो॒डाश॑म् दा॒त्रे दा॒त्रे पु॑रो॒डाश॒ मेका॑दशकपाल॒ मेका॑दशकपालम् पुरो॒डाश॑म् दा॒त्रे दा॒त्रे पु॑रो॒डाश॒ मेका॑दशकपालम् । \newline
51. पु॒रो॒डाश॒ मेका॑दशकपाल॒ मेका॑दशकपालम् पुरो॒डाश॑म् पुरो॒डाश॒ मेका॑दशकपाल॒म् निर् णिरेका॑दशकपालम् पुरो॒डाश॑म् पुरो॒डाश॒ मेका॑दशकपाल॒म् निः । \newline
52. एका॑दशकपाल॒म् निर् णिरेका॑दशकपाल॒ मेका॑दशकपाल॒म् निर् व॑पेद् वपे॒न् निरेका॑दशकपाल॒ मेका॑दशकपाल॒म् निर् व॑पेत् । \newline
53. एका॑दशकपाल॒मित्येका॑दश - क॒पा॒ल॒म् । \newline
54. निर् व॑पेद् वपे॒न् निर् णिर् व॑पे॒द् यो यो व॑पे॒न् निर् णिर् व॑पे॒द् यः । \newline
55. व॒पे॒द् यो यो व॑पेद् वपे॒द् यः का॒मये॑त का॒मये॑त॒ यो व॑पेद् वपे॒द् यः का॒मये॑त । \newline
56. यः का॒मये॑त का॒मये॑त॒ यो यः का॒मये॑त॒ दान॑कामा॒ दान॑कामाः का॒मये॑त॒ यो यः का॒मये॑त॒ दान॑कामाः । \newline
57. का॒मये॑त॒ दान॑कामा॒ दान॑कामाः का॒मये॑त का॒मये॑त॒ दान॑कामा मे मे॒ दान॑कामाः का॒मये॑त का॒मये॑त॒ दान॑कामा मे । \newline
58. दान॑कामा मे मे॒ दान॑कामा॒ दान॑कामा मे प्र॒जाः प्र॒जा मे॒ दान॑कामा॒ दान॑कामा मे प्र॒जाः । \newline
59. दान॑कामा॒ इति॒ दान॑ - का॒माः॒ । \newline
60. मे॒ प्र॒जाः प्र॒जा मे॑ मे प्र॒जाः स्युः॑ स्युः प्र॒जा मे॑ मे प्र॒जाः स्युः॑ । \newline
61. प्र॒जाः स्युः॑ स्युः प्र॒जाः प्र॒जाः स्यु॒ रितीति॑ स्युः प्र॒जाः प्र॒जाः स्यु॒रिति॑ । \newline
62. प्र॒जा इति॑ प्र - जाः । \newline
63. स्यु॒ रितीति॑ स्युः स्यु॒ रितीन्द्र॒ मिन्द्र॒ मिति॑ स्युः स्यु॒ रितीन्द्र᳚म् । \newline
\pagebreak
\markright{ TS 2.2.8.4  \hfill https://www.vedavms.in \hfill}
\addcontentsline{toc}{section}{ TS 2.2.8.4 }
\section*{ TS 2.2.8.4 }

\textbf{TS 2.2.8.4 } \newline
\textbf{Samhita Paata} \newline

-रितीन्द्र॑मे॒व दा॒तारꣳ॒॒ स्वेन॑ भाग॒धेये॒नोप॑ धावति॒ स ए॒वास्मै॒ दान॑कामाः प्र॒जाः क॑रोति॒ दान॑कामा अस्मै प्र॒जा भ॑व॒न्तीन्द्रा॑य प्रदा॒त्रे पु॑रो॒डाश॒मेका॑दशकपालं॒ निर्व॑पे॒द्यस्मै॒ प्रत्त॑मिव॒ सन्न प्र॑दी॒येतेन्द्र॑मे॒व प्र॑दा॒तारꣳ॒॒ स्वेन॑ भाग॒धेये॒नोप॑धावति॒ स ए॒वास्मै॒ प्रदा॑पय॒तीन्द्रा॑य सु॒त्राम्णे॑ पुरो॒डाश॒मेका॑दशकपालं॒ निर्व॑पे॒दप॑रुद्धो वा - [  ] \newline

\textbf{Pada Paata} \newline

इति॑ । इन्द्र᳚म् । ए॒व । दा॒तार᳚म् । स्वेन॑ । भा॒ग॒धेये॒नेति॑ भाग - धेये॑न । उपेति॑ । धा॒व॒ति॒ । सः । ए॒व । अ॒स्मै॒ । दान॑कामा॒ इति॒ दान॑-का॒माः॒ । प्र॒जा इति॑ प्र - जाः । क॒रो॒ति॒ । दान॑कामा॒ इति॒ दान॑ - का॒माः॒ । अ॒स्मै॒ । प्र॒जा इति॑ प्र - जाः । भ॒व॒न्ति॒ । इन्द्रा॑य । प्र॒दा॒त्र इति॑ प्र - दा॒त्रे । पु॒रो॒डाश᳚म् । एका॑दशकपाल॒मित्येका॑दश - क॒पा॒ल॒म् । निरिति॑ । व॒पे॒त् । यस्मै᳚ । प्रत्त᳚म् । इ॒व॒ । सत् । न । प्र॒दी॒येतेति॑ प्र - दी॒येत॑ । इन्द्र᳚म् । ए॒व । प्र॒दा॒तार॒मिति॑ प्र - दा॒तार᳚म् । स्वेन॑ । भा॒ग॒धेये॒नेति॑ भाग - धेये॑न । उपेति॑ । धा॒व॒ति॒ । सः । ए॒व । अ॒स्मै॒ । प्रेति॑ । दा॒प॒य॒ति॒ । इन्द्रा॑य । सु॒त्रांण॒ इति॑ सु - त्रांणे᳚ । पु॒रो॒डाश᳚म् । एका॑दशकपाल॒मित्येका॑दश - क॒पा॒ल॒म् । निरिति॑ । व॒पे॒त् । अप॑रुद्ध॒ इत्यप॑ - रु॒द्धः॒ । वा॒ ।  \newline


\textbf{Krama Paata} \newline

इतीन्द्र᳚म् । इन्द्र॑मे॒व । ए॒व दा॒तार᳚म् । दा॒तारꣳ॒॒ स्वेन॑ । स्वेन॑ भाग॒धेये॑न । भा॒ग॒धेये॒नोप॑ । भा॒ग॒धेये॒नेति॑ भाग - धेये॑न । उप॑ धावति । धा॒व॒ति॒ सः । स ए॒व । ए॒वास्मै᳚ । अ॒स्मै॒ दान॑कामाः । दान॑कामाः प्र॒जाः । दान॑कामा॒ इति॒ दान॑ - का॒माः॒ । प्र॒जाः क॑रोति । प्र॒जा इति॑ प्र - जाः । क॒रो॒ति॒ दान॑कामाः । दान॑कामा अस्मै । दान॑कामा॒ इति॒ दान॑ - का॒माः॒ । अ॒स्मै॒ प्र॒जाः । प्र॒जा भ॑वन्ति । प्र॒जा इति॑ प्र - जाः । भ॒व॒न्तीन्द्रा॑य । इन्द्रा॑य प्रदा॒त्रे । प्र॒दा॒त्रे पु॑रो॒डाश᳚म् । प्र॒दा॒त्र इति॑ प्र - दा॒त्रे । पु॒रो॒डाश॒मेका॑दशकपालम् । एका॑दशकपाल॒म् निः । एका॑दशकपाल॒मित्येका॑दश - क॒पा॒ल॒म् । निर् व॑पेत् । व॒पे॒द् यस्मै᳚ । यस्मै॒ प्रत्त᳚म् । प्रत्त॑मिव । इ॒व॒ सत् । सन्न । न प्र॑दी॒येत॑ । प्र॒दी॒येतेन्द्र᳚म् । प्र॒दी॒येतेति॑ प्र - दी॒येत॑ । इन्द्र॑मे॒व । ए॒व प्र॑दा॒तार᳚म् । प्र॒दा॒तारꣳ॒॒ स्वेन॑ । प्र॒दा॒तार॒मिति॑ प्र - दा॒तार᳚म् । स्वेन॑ भाग॒धेये॑न । भा॒ग॒धेये॒नोप॑ । भा॒ग॒धेये॒नेति॑ भाग - धेये॑न । उप॑ धावति । धा॒व॒ति॒ सः । स ए॒व । ए॒वास्मै᳚ । अ॒स्मै॒ प्र । प्र दा॑पयति । दा॒प॒य॒तीन्द्रा॑य । इन्द्रा॑य सु॒त्राम्णे᳚ । सु॒त्राम्णे॑ पुरो॒डाश᳚म् । सु॒त्राम्ण॒ इति॑ सु - त्राम्णे᳚ । पु॒रो॒डाश॒मेका॑दशकपालम् । एका॑दशकपाल॒म् निः । एका॑दशकपाल॒मित्येका॑दश - क॒पा॒ल॒म् । निर् व॑पेत् । व॒पे॒दप॑रुद्धः । अप॑रुद्धो वा । अप॑रुद्ध॒ इत्यप॑ - रु॒द्धः॒ । 
वा॒ ऽप॒रु॒द्ध्यमा॑नः \newline

\textbf{Jatai Paata} \newline

1. इतीन्द्र॒ मिन्द्र॒ मितीतीन्द्र᳚म् । \newline
2. इन्द्र॑ मे॒वैवे न्द्र॒ मिन्द्र॑ मे॒व । \newline
3. ए॒व दा॒तार॑म् दा॒तार॑ मे॒वैव दा॒तार᳚म् । \newline
4. दा॒तारꣳ॒॒ स्वेन॒ स्वेन॑ दा॒तार॑म् दा॒तारꣳ॒॒ स्वेन॑ । \newline
5. स्वेन॑ भाग॒धेये॑न भाग॒धेये॑न॒ स्वेन॒ स्वेन॑ भाग॒धेये॑न । \newline
6. भा॒ग॒धेये॒नोपोप॑ भाग॒धेये॑न भाग॒धेये॒नोप॑ । \newline
7. भा॒ग॒धेये॒नेति॑ भाग - धेये॑न । \newline
8. उप॑ धावति धाव॒ त्युपोप॑ धावति । \newline
9. धा॒व॒ति॒ स स धा॑वति धावति॒ सः । \newline
10. स ए॒वैव स स ए॒व । \newline
11. ए॒वास्मा॑ अस्मा ए॒वैवास्मै᳚ । \newline
12. अ॒स्मै॒ दान॑कामा॒ दान॑कामा अस्मा अस्मै॒ दान॑कामाः । \newline
13. दान॑कामाः प्र॒जाः प्र॒जा दान॑कामा॒ दान॑कामाः प्र॒जाः । \newline
14. दान॑कामा॒ इति॒ दान॑ - का॒माः॒ । \newline
15. प्र॒जाः क॑रोति करोति प्र॒जाः प्र॒जाः क॑रोति । \newline
16. प्र॒जा इति॑ प्र - जाः । \newline
17. क॒रो॒ति॒ दान॑कामा॒ दान॑कामाः करोति करोति॒ दान॑कामाः । \newline
18. दान॑कामा अस्मा अस्मै॒ दान॑कामा॒ दान॑कामा अस्मै । \newline
19. दान॑कामा॒ इति॒ दान॑ - का॒माः॒ । \newline
20. अ॒स्मै॒ प्र॒जाः प्र॒जा अ॑स्मा अस्मै प्र॒जाः । \newline
21. प्र॒जा भ॑वन्ति भवन्ति प्र॒जाः प्र॒जा भ॑वन्ति । \newline
22. प्र॒जा इति॑ प्र - जाः । \newline
23. भ॒व॒न्तीन्द्रा॒ये न्द्रा॑य भवन्ति भव॒न्तीन्द्रा॑य । \newline
24. इन्द्रा॑य प्रदा॒त्रे प्र॑दा॒त्र इन्द्रा॒ये न्द्रा॑य प्रदा॒त्रे । \newline
25. प्र॒दा॒त्रे पु॑रो॒डाश॑म् पुरो॒डाश॑म् प्रदा॒त्रे प्र॑दा॒त्रे पु॑रो॒डाश᳚म् । \newline
26. प्र॒दा॒त्र इति॑ प्र - दा॒त्रे । \newline
27. पु॒रो॒डाश॒ मेका॑दशकपाल॒ मेका॑दशकपालम् पुरो॒डाश॑म् पुरो॒डाश॒ मेका॑दशकपालम् । \newline
28. एका॑दशकपाल॒म् निर् णिरेका॑दशकपाल॒ मेका॑दशकपाल॒म् निः । \newline
29. एका॑दशकपाल॒मित्येका॑दश - क॒पा॒ल॒म् । \newline
30. निर् व॑पेद् वपे॒न् निर् णिर् व॑पेत् । \newline
31. व॒पे॒द् यस्मै॒ यस्मै॑ वपेद् वपे॒द् यस्मै᳚ । \newline
32. यस्मै॒ प्रत्त॒म् प्रत्तं॒ ॅयस्मै॒ यस्मै॒ प्रत्त᳚म् । \newline
33. प्रत्त॑ मिवे व॒ प्रत्त॒म् प्रत्त॑ मिव । \newline
34. इ॒व॒ सथ् सदि॑वे व॒ सत् । \newline
35. सन् न न सथ् सन् न । \newline
36. न प्र॑दी॒येत॑ प्रदी॒येत॒ न न प्र॑दी॒येत॑ । \newline
37. प्र॒दी॒येते न्द्र॒ मिन्द्र॑म् प्रदी॒येत॑ प्रदी॒येते न्द्र᳚म् । \newline
38. प्र॒दी॒येतेति॑ प्र - दी॒येत॑ । \newline
39. इन्द्र॑ मे॒वैवे न्द्र॒ मिन्द्र॑ मे॒व । \newline
40. ए॒व प्र॑दा॒तार॑म् प्रदा॒तार॑ मे॒वैव प्र॑दा॒तार᳚म् । \newline
41. प्र॒दा॒तारꣳ॒॒ स्वेन॒ स्वेन॑ प्रदा॒तार॑म् प्रदा॒तारꣳ॒॒ स्वेन॑ । \newline
42. प्र॒दा॒तार॒मिति॑ प्र - दा॒तार᳚म् । \newline
43. स्वेन॑ भाग॒धेये॑न भाग॒धेये॑न॒ स्वेन॒ स्वेन॑ भाग॒धेये॑न । \newline
44. भा॒ग॒धेये॒नोपोप॑ भाग॒धेये॑न भाग॒धेये॒नोप॑ । \newline
45. भा॒ग॒धेये॒नेति॑ भाग - धेये॑न । \newline
46. उप॑ धावति धाव॒ त्युपोप॑ धावति । \newline
47. धा॒व॒ति॒ स स धा॑वति धावति॒ सः । \newline
48. स ए॒वैव स स ए॒व । \newline
49. ए॒वास्मा॑ अस्मा ए॒वैवास्मै᳚ । \newline
50. अ॒स्मै॒ प्र प्रास्मा॑ अस्मै॒ प्र । \newline
51. प्र दा॑पयति दापयति॒ प्र प्र दा॑पयति । \newline
52. दा॒प॒य॒तीन्द्रा॒ये न्द्रा॑य दापयति दापय॒तीन्द्रा॑य । \newline
53. इन्द्रा॑य सु॒त्रांणे॑ सु॒त्रांण॒ इन्द्रा॒ये न्द्रा॑य सु॒त्रांणे᳚ । \newline
54. सु॒त्रांणे॑ पुरो॒डाश॑म् पुरो॒डाशꣳ॑ सु॒त्रांणे॑ सु॒त्रांणे॑ पुरो॒डाश᳚म् । \newline
55. सु॒त्रांण॒ इति॑ सु - त्रांणे᳚ । \newline
56. पु॒रो॒डाश॒ मेका॑दशकपाल॒ मेका॑दशकपालम् पुरो॒डाश॑म् पुरो॒डाश॒ मेका॑दशकपालम् । \newline
57. एका॑दशकपाल॒म् निर् णिरेका॑दशकपाल॒ मेका॑दशकपाल॒म् निः । \newline
58. एका॑दशकपाल॒मित्येका॑दश - क॒पा॒ल॒म् । \newline
59. निर् व॑पेद् वपे॒न् निर् णिर् व॑पेत् । \newline
60. व॒पे॒ दप॑रु॒द्धो ऽप॑रुद्धो वपेद् वपे॒ दप॑रुद्धः । \newline
61. अप॑रुद्धो वा॒ वा ऽप॑रु॒द्धो ऽप॑रुद्धो वा । \newline
62. अप॑रुद्ध॒ इत्यप॑ - रु॒द्धः॒ । \newline
63. वा॒ ऽप॒रु॒द्ध्यमा॑नो ऽपरु॒द्ध्यमा॑नो वा वा ऽपरु॒द्ध्यमा॑नः । \newline

\textbf{Ghana Paata } \newline

1. इतीन्द्र॒ मिन्द्र॒ मितीतीन्द्र॑ मे॒वैवे न्द्र॒ मितीतीन्द्र॑ मे॒व । \newline
2. इन्द्र॑ मे॒वैवे न्द्र॒ मिन्द्र॑ मे॒व दा॒तार॑म् दा॒तार॑ मे॒वे न्द्र॒ मिन्द्र॑ मे॒व दा॒तार᳚म् । \newline
3. ए॒व दा॒तार॑म् दा॒तार॑ मे॒वैव दा॒तारꣳ॒॒ स्वेन॒ स्वेन॑ दा॒तार॑ मे॒वैव दा॒तारꣳ॒॒ स्वेन॑ । \newline
4. दा॒तारꣳ॒॒ स्वेन॒ स्वेन॑ दा॒तार॑म् दा॒तारꣳ॒॒ स्वेन॑ भाग॒धेये॑न भाग॒धेये॑न॒ स्वेन॑ दा॒तार॑म् दा॒तारꣳ॒॒ स्वेन॑ भाग॒धेये॑न । \newline
5. स्वेन॑ भाग॒धेये॑न भाग॒धेये॑न॒ स्वेन॒ स्वेन॑ भाग॒धेये॒नोपोप॑ भाग॒धेये॑न॒ स्वेन॒ स्वेन॑ भाग॒धेये॒नोप॑ । \newline
6. भा॒ग॒धेये॒नोपोप॑ भाग॒धेये॑न भाग॒धेये॒नोप॑ धावति धाव॒ त्युप॑ भाग॒धेये॑न भाग॒धेये॒नोप॑ धावति । \newline
7. भा॒ग॒धेये॒नेति॑ भाग - धेये॑न । \newline
8. उप॑ धावति धाव॒ त्युपोप॑ धावति॒ स स धा॑व॒ त्युपोप॑ धावति॒ सः । \newline
9. धा॒व॒ति॒ स स धा॑वति धावति॒ स ए॒वैव स धा॑वति धावति॒ स ए॒व । \newline
10. स ए॒वैव स स ए॒वास्मा॑ अस्मा ए॒व स स ए॒वास्मै᳚ । \newline
11. ए॒वास्मा॑ अस्मा ए॒वैवास्मै॒ दान॑कामा॒ दान॑कामा अस्मा ए॒वैवास्मै॒ दान॑कामाः । \newline
12. अ॒स्मै॒ दान॑कामा॒ दान॑कामा अस्मा अस्मै॒ दान॑कामाः प्र॒जाः प्र॒जा दान॑कामा अस्मा अस्मै॒ दान॑कामाः प्र॒जाः । \newline
13. दान॑कामाः प्र॒जाः प्र॒जा दान॑कामा॒ दान॑कामाः प्र॒जाः क॑रोति करोति प्र॒जा दान॑कामा॒ दान॑कामाः प्र॒जाः क॑रोति । \newline
14. दान॑कामा॒ इति॒ दान॑ - का॒माः॒ । \newline
15. प्र॒जाः क॑रोति करोति प्र॒जाः प्र॒जाः क॑रोति॒ दान॑कामा॒ दान॑कामाः करोति प्र॒जाः प्र॒जाः क॑रोति॒ दान॑कामाः । \newline
16. प्र॒जा इति॑ प्र - जाः । \newline
17. क॒रो॒ति॒ दान॑कामा॒ दान॑कामाः करोति करोति॒ दान॑कामा अस्मा अस्मै॒ दान॑कामाः करोति करोति॒ दान॑कामा अस्मै । \newline
18. दान॑कामा अस्मा अस्मै॒ दान॑कामा॒ दान॑कामा अस्मै प्र॒जाः प्र॒जा अ॑स्मै॒ दान॑कामा॒ दान॑कामा अस्मै प्र॒जाः । \newline
19. दान॑कामा॒ इति॒ दान॑ - का॒माः॒ । \newline
20. अ॒स्मै॒ प्र॒जाः प्र॒जा अ॑स्मा अस्मै प्र॒जा भ॑वन्ति भवन्ति प्र॒जा अ॑स्मा अस्मै प्र॒जा भ॑वन्ति । \newline
21. प्र॒जा भ॑वन्ति भवन्ति प्र॒जाः प्र॒जा भ॑व॒न्तीन्द्रा॒ये न्द्रा॑य भवन्ति प्र॒जाः प्र॒जा भ॑व॒न्तीन्द्रा॑य । \newline
22. प्र॒जा इति॑ प्र - जाः । \newline
23. भ॒व॒न्तीन्द्रा॒ये न्द्रा॑य भवन्ति भव॒न्तीन्द्रा॑य प्रदा॒त्रे प्र॑दा॒त्र इन्द्रा॑य भवन्ति भव॒न्तीन्द्रा॑य प्रदा॒त्रे । \newline
24. इन्द्रा॑य प्रदा॒त्रे प्र॑दा॒त्र इन्द्रा॒ये न्द्रा॑य प्रदा॒त्रे पु॑रो॒डाश॑म् पुरो॒डाश॑म् प्रदा॒त्र इन्द्रा॒ये न्द्रा॑य प्रदा॒त्रे पु॑रो॒डाश᳚म् । \newline
25. प्र॒दा॒त्रे पु॑रो॒डाश॑म् पुरो॒डाश॑म् प्रदा॒त्रे प्र॑दा॒त्रे पु॑रो॒डाश॒ मेका॑दशकपाल॒ मेका॑दशकपालम् पुरो॒डाश॑म् प्रदा॒त्रे प्र॑दा॒त्रे पु॑रो॒डाश॒ मेका॑दशकपालम् । \newline
26. प्र॒दा॒त्र इति॑ प्र - दा॒त्रे । \newline
27. पु॒रो॒डाश॒ मेका॑दशकपाल॒ मेका॑दशकपालम् पुरो॒डाश॑म् पुरो॒डाश॒ मेका॑दशकपाल॒म् निर् णिरेका॑दशकपालम् पुरो॒डाश॑म् पुरो॒डाश॒ मेका॑दशकपाल॒म् निः । \newline
28. एका॑दशकपाल॒म् निर् णिरेका॑दशकपाल॒ मेका॑दशकपाल॒म् निर् व॑पेद् वपे॒न् निरेका॑दशकपाल॒ मेका॑दशकपाल॒म् निर् व॑पेत् । \newline
29. एका॑दशकपाल॒मित्येका॑दश - क॒पा॒ल॒म् । \newline
30. निर् व॑पेद् वपे॒न् निर् णिर् व॑पे॒द् यस्मै॒ यस्मै॑ वपे॒न् निर् णिर् व॑पे॒द् यस्मै᳚ । \newline
31. व॒पे॒द् यस्मै॒ यस्मै॑ वपेद् वपे॒द् यस्मै॒ प्रत्त॒म् प्रत्तं॒ ॅयस्मै॑ वपेद् वपे॒द् यस्मै॒ प्रत्त᳚म् । \newline
32. यस्मै॒ प्रत्त॒म् प्रत्तं॒ ॅयस्मै॒ यस्मै॒ प्रत्त॑ मिवे व॒ प्रत्तं॒ ॅयस्मै॒ यस्मै॒ प्रत्त॑ मिव । \newline
33. प्रत्त॑ मिवे व॒ प्रत्त॒म् प्रत्त॑ मिव॒ सथ् सदि॑व॒ प्रत्त॒म् प्रत्त॑ मिव॒ सत् । \newline
34. इ॒व॒ सथ् सदि॑वे व॒ सन् न न सदि॑वे व॒ सन् न । \newline
35. सन् न न सथ् सन् न प्र॑दी॒येत॑ प्रदी॒येत॒ न सथ् सन् न प्र॑दी॒येत॑ । \newline
36. न प्र॑दी॒येत॑ प्रदी॒येत॒ न न प्र॑दी॒येते न्द्र॒ मिन्द्र॑म् प्रदी॒येत॒ न न प्र॑दी॒येते न्द्र᳚म् । \newline
37. प्र॒दी॒येते न्द्र॒ मिन्द्र॑म् प्रदी॒येत॑ प्रदी॒येते न्द्र॑ मे॒वैवे न्द्र॑म् प्रदी॒येत॑ प्रदी॒येते न्द्र॑ मे॒व । \newline
38. प्र॒दी॒येतेति॑ प्र - दी॒येत॑ । \newline
39. इन्द्र॑ मे॒वैवे न्द्र॒ मिन्द्र॑ मे॒व प्र॑दा॒तार॑म् प्रदा॒तार॑ मे॒वे न्द्र॒ मिन्द्र॑ मे॒व प्र॑दा॒तार᳚म् । \newline
40. ए॒व प्र॑दा॒तार॑म् प्रदा॒तार॑ मे॒वैव प्र॑दा॒तारꣳ॒॒ स्वेन॒ स्वेन॑ प्रदा॒तार॑ मे॒वैव प्र॑दा॒तारꣳ॒॒ स्वेन॑ । \newline
41. प्र॒दा॒तारꣳ॒॒ स्वेन॒ स्वेन॑ प्रदा॒तार॑म् प्रदा॒तारꣳ॒॒ स्वेन॑ भाग॒धेये॑न भाग॒धेये॑न॒ स्वेन॑ प्रदा॒तार॑म् प्रदा॒तारꣳ॒॒ स्वेन॑ भाग॒धेये॑न । \newline
42. प्र॒दा॒तार॒मिति॑ प्र - दा॒तार᳚म् । \newline
43. स्वेन॑ भाग॒धेये॑न भाग॒धेये॑न॒ स्वेन॒ स्वेन॑ भाग॒धेये॒नोपोप॑ भाग॒धेये॑न॒ स्वेन॒ स्वेन॑ भाग॒धेये॒नोप॑ । \newline
44. भा॒ग॒धेये॒नोपोप॑ भाग॒धेये॑न भाग॒धेये॒नोप॑ धावति धाव॒ त्युप॑ भाग॒धेये॑न भाग॒धेये॒नोप॑ धावति । \newline
45. भा॒ग॒धेये॒नेति॑ भाग - धेये॑न । \newline
46. उप॑ धावति धाव॒ त्युपोप॑ धावति॒ स स धा॑व॒ त्युपोप॑ धावति॒ सः । \newline
47. धा॒व॒ति॒ स स धा॑वति धावति॒ स ए॒वैव स धा॑वति धावति॒ स ए॒व । \newline
48. स ए॒वैव स स ए॒वास्मा॑ अस्मा ए॒व स स ए॒वास्मै᳚ । \newline
49. ए॒वास्मा॑ अस्मा ए॒वैवास्मै॒ प्र प्रास्मा॑ ए॒वैवास्मै॒ प्र । \newline
50. अ॒स्मै॒ प्र प्रास्मा॑ अस्मै॒ प्र दा॑पयति दापयति॒ प्रास्मा॑ अस्मै॒ प्र दा॑पयति । \newline
51. प्र दा॑पयति दापयति॒ प्र प्र दा॑पय॒तीन्द्रा॒ये न्द्रा॑य दापयति॒ प्र प्र दा॑पय॒तीन्द्रा॑य । \newline
52. दा॒प॒य॒तीन्द्रा॒ये न्द्रा॑य दापयति दापय॒तीन्द्रा॑य सु॒त्रांणे॑ सु॒त्रांण॒ इन्द्रा॑य दापयति दापय॒तीन्द्रा॑य सु॒त्रांणे᳚ । \newline
53. इन्द्रा॑य सु॒त्रांणे॑ सु॒त्रांण॒ इन्द्रा॒ये न्द्रा॑य सु॒त्रांणे॑ पुरो॒डाश॑म् पुरो॒डाशꣳ॑ सु॒त्रांण॒ इन्द्रा॒ये न्द्रा॑य सु॒त्रांणे॑ पुरो॒डाश᳚म् । \newline
54. सु॒त्रांणे॑ पुरो॒डाश॑म् पुरो॒डाशꣳ॑ सु॒त्रांणे॑ सु॒त्रांणे॑ पुरो॒डाश॒ मेका॑दशकपाल॒ मेका॑दशकपालम् पुरो॒डाशꣳ॑ सु॒त्रांणे॑ सु॒त्रांणे॑ पुरो॒डाश॒ मेका॑दशकपालम् । \newline
55. सु॒त्रांण॒ इति॑ सु - त्रांणे᳚ । \newline
56. पु॒रो॒डाश॒ मेका॑दशकपाल॒ मेका॑दशकपालम् पुरो॒डाश॑म् पुरो॒डाश॒ मेका॑दशकपाल॒म् निर् णिरेका॑दशकपालम् पुरो॒डाश॑म् पुरो॒डाश॒ मेका॑दशकपाल॒म् निः । \newline
57. एका॑दशकपाल॒म् निर् णिरेका॑दशकपाल॒ मेका॑दशकपाल॒म् निर् व॑पेद् वपे॒न् निरेका॑दशकपाल॒ मेका॑दशकपाल॒म् निर् व॑पेत् । \newline
58. एका॑दशकपाल॒मित्येका॑दश - क॒पा॒ल॒म् । \newline
59. निर् व॑पेद् वपे॒न् निर् णिर् व॑पे॒ दप॑रु॒द्धो ऽप॑रुद्धो वपे॒न् निर् णिर् व॑पे॒ दप॑रुद्धः । \newline
60. व॒पे॒ दप॑रु॒द्धो ऽप॑रुद्धो वपेद् वपे॒ दप॑रुद्धो वा॒ वा ऽप॑रुद्धो वपेद् वपे॒ दप॑रुद्धो वा । \newline
61. अप॑रुद्धो वा॒ वा ऽप॑रु॒द्धो ऽप॑रुद्धो वा ऽपरु॒द्ध्यमा॑नो ऽपरु॒द्ध्यमा॑नो॒ वा ऽप॑रु॒द्धो ऽप॑रुद्धो वा ऽपरु॒द्ध्यमा॑नः । \newline
62. अप॑रुद्ध॒ इत्यप॑ - रु॒द्धः॒ । \newline
63. वा॒ ऽप॒रु॒द्ध्यमा॑नो ऽपरु॒द्ध्यमा॑नो वा वा ऽपरु॒द्ध्यमा॑नो वा वा ऽपरु॒द्ध्यमा॑नो वा वा ऽपरु॒द्ध्यमा॑नो वा । \newline
\pagebreak
\markright{ TS 2.2.8.5  \hfill https://www.vedavms.in \hfill}
\addcontentsline{toc}{section}{ TS 2.2.8.5 }
\section*{ TS 2.2.8.5 }

\textbf{TS 2.2.8.5 } \newline
\textbf{Samhita Paata} \newline

ऽपरु॒द्धयमा॑नो॒ वेन्द्र॑मे॒व सु॒त्रामा॑णꣳ॒॒ स्वेन॑ भाग॒धेये॒नोप॑ धावति॒ स ए॒वैनं॑ त्रायते ऽनपरु॒द्ध्यो भ॑व॒तीन्द्रो॒ वै स॒दृङ् दे॒वता॑भिरासी॒थ् स न व्या॒वृत॑मगच्छ॒थ् स प्र॒जाप॑ति॒मुपा॑धाव॒त् तस्मा॑ ए॒तमै॒न्द्रमेका॑दशकपालं॒ निर॑वप॒त् तेनै॒-वास्मि॑न्निन्द्रि॒यम॑दधा॒-च्छक्व॑री याज्यानुवा॒क्ये॑ अकरो॒द्-वज्रो॒ वै शक्व॑री॒ स ए॑नं॒ ॅवज्रो॒ भूत्या॑ ऐन्ध॒ - [  ] \newline

\textbf{Pada Paata} \newline

अ॒प॒रु॒द्ध्यमा॑न॒ इत्य॑प - रु॒द्ध्यमा॑नः । वा॒ । इन्द्र᳚म् । ए॒व । सु॒त्रामा॑ण॒मिति॑ सु - त्रामा॑णम् । स्वेन॑ । भा॒ग॒धेये॒नेति॑ भाग-धेये॑न । उपेति॑ । धा॒व॒ति॒ । सः । ए॒व । ए॒न॒म् । त्रा॒य॒ते॒ । अ॒न॒प॒रु॒द्ध्य इत्य॑नप - रु॒द्ध्यः । भ॒व॒ति॒ । इन्द्रः॑ । वै । स॒दृङ्ङिति॑ स - दृङ् । दे॒वता॑भिः । आ॒सी॒त् । सः । न । व्या॒वृत॒मिति॑ वि - आ॒वृत᳚म् । अ॒ग॒च्छ॒त् । सः । प्र॒जाप॑ति॒मिति॑ प्र॒जा - प॒ति॒म् । उपेति॑ । अ॒धा॒व॒त् । तस्मै᳚ । ए॒तम् । ऐ॒न्द्रम् । एका॑दशकपाल॒मित्येका॑दश - क॒पा॒ल॒म् । निरिति॑ । अ॒व॒प॒त् । तेन॑ । ए॒व । अ॒स्मि॒न्न् । इ॒न्द्रि॒यम् । अ॒द॒धा॒त् । शक्व॑री॒ इति॑ । या॒ज्या॒नु॒वा॒क्ये॑ इति॑ याज्या - अ॒नु॒वा॒क्ये᳚ । अ॒क॒रो॒त् । वज्रः॑ । वै । शक्व॑री । सः । ए॒न॒म् । वज्रः॑ । भूत्यै᳚ । ऐ॒न्ध॒ ।  \newline


\textbf{Krama Paata} \newline

अ॒प॒रु॒द्ध्यमा॑नो वा । अ॒प॒रु॒द्ध्यमा॑न॒ इत्य॑प - रु॒द्ध्यमा॑नः । वेन्द्र᳚म् । इन्द्र॑मे॒व । ए॒व सु॒त्रामा॑णम् । सु॒त्रामा॑णꣳ॒॒ स्वेन॑ । सु॒त्रामा॑ण॒मिति॑ सु - त्रामा॑णम् । स्वेन॑ भाग॒धेये॑न । भा॒ग॒धेये॒नोप॑ । भा॒ग॒धेये॒नेति॑ भाग - धेये॑न । उप॑ धावति । धा॒व॒ति॒ सः । स ए॒व । ए॒वैन᳚म् । ए॒न॒म् त्रा॒य॒ते॒ । त्रा॒य॒ते॒ ऽन॒प॒रु॒द्ध्यः । अ॒न॒प॒रु॒द्ध्यो भ॑वति । अ॒न॒प॒रु॒द्ध्य इत्य॑नप - रु॒द्ध्यः । भ॒व॒तीन्द्रः॑ । इन्द्रो॒ वै । वै स॒दृङ् । स॒दृङ् दे॒वता॑भिः । स॒दृङ्ङिति॑ स - दृङ् । दे॒वता॑भिरासीत् । आ॒सी॒थ् सः । स न । न व्या॒वृत᳚म् । व्या॒वृत॑मगच्छत् । व्या॒वृत॒मिति॑ वि - आ॒वृत᳚म् । अ॒ग॒च्छ॒थ् सः । स प्र॒जाप॑तिम् । प्र॒जाप॑ति॒मुप॑ । प्र॒जाप॑ति॒मिति॑ प्र॒जा - प॒ति॒म् । उपा॑धावत् । अ॒धा॒व॒त्,तस्मै᳚ । तस्मा॑ ए॒तम् । ए॒तमै॒न्द्रम् । ऐ॒न्द्रमेका॑दशकपालम् । एका॑दशकपाल॒म् निः । एका॑दशकपाल॒मित्येका॑दश - क॒पा॒ल॒म् । निर॑वपत् । अ॒व॒प॒त् तेन॑ । तेनै॒व । ए॒वास्मिन्न्॑ । अ॒स्मि॒न्नि॒न्द्रि॒यम् । इ॒न्द्रि॒यम॑दधात् । अ॒द॒धा॒च्छक्व॑री । शक्व॑री याज्यानुवा॒क्ये᳚ । शक्व॑री॒ इति॒ शक्व॑री । या॒ज्या॒नु॒वा॒क्ये॑ अकरोत् । या॒ज्या॒नु॒वा॒क्ये॑ इति॑ याज्या - अ॒नु॒वा॒क्ये᳚ । अ॒क॒रो॒द्,वज्रः॑ । वज्रो॒ वै । वै शक्व॑री । शक्व॑री॒ सः । स ए॑नम् । ए॒नं॒ ॅवज्रः॑ । वज्रो॒ भूत्यै᳚ । भूत्या॑ ऐन्ध । ऐ॒न्ध॒ सः \newline

\textbf{Jatai Paata} \newline

1. अ॒प॒रु॒द्ध्यमा॑नो वा वा ऽपरु॒द्ध्यमा॑नो ऽपरु॒द्ध्यमा॑नो वा । \newline
2. अ॒प॒रु॒द्ध्यमा॑न॒ इत्य॑प - रु॒द्ध्यमा॑नः । \newline
3. वेन्द्र॒ मिन्द्रं॑ ॅवा॒ वेन्द्र᳚म् । \newline
4. इन्द्र॑ मे॒वैवे न्द्र॒ मिन्द्र॑ मे॒व । \newline
5. ए॒व सु॒त्रामा॑णꣳ सु॒त्रामा॑ण मे॒वैव सु॒त्रामा॑णम् । \newline
6. सु॒त्रामा॑णꣳ॒॒ स्वेन॒ स्वेन॑ सु॒त्रामा॑णꣳ सु॒त्रामा॑णꣳ॒॒ स्वेन॑ । \newline
7. सु॒त्रामा॑ण॒मिति॑ सु - त्रामा॑णम् । \newline
8. स्वेन॑ भाग॒धेये॑न भाग॒धेये॑न॒ स्वेन॒ स्वेन॑ भाग॒धेये॑न । \newline
9. भा॒ग॒धेये॒नोपोप॑ भाग॒धेये॑न भाग॒धेये॒नोप॑ । \newline
10. भा॒ग॒धेये॒नेति॑ भाग - धेये॑न । \newline
11. उप॑ धावति धाव॒ त्युपोप॑ धावति । \newline
12. धा॒व॒ति॒ स स धा॑वति धावति॒ सः । \newline
13. स ए॒वैव स स ए॒व । \newline
14. ए॒वैन॑ मेन मे॒वैवैन᳚म् । \newline
15. ए॒न॒म् त्रा॒य॒ते॒ त्रा॒य॒त॒ ए॒न॒ मे॒न॒म् त्रा॒य॒ते॒ । \newline
16. त्रा॒य॒ते॒ ऽन॒प॒रु॒द्ध्यो॑ ऽनपरु॒द्ध्य स्त्रा॑यते त्रायते ऽनपरु॒द्ध्यः । \newline
17. अ॒न॒प॒रु॒द्ध्यो भ॑वति भव त्यनपरु॒द्ध्यो॑ ऽनपरु॒द्ध्यो भ॑वति । \newline
18. अ॒न॒प॒रु॒द्ध्य इत्य॑नप - रु॒द्ध्यः । \newline
19. भ॒व॒तीन्द्र॒ इन्द्रो॑ भवति भव॒तीन्द्रः॑ । \newline
20. इन्द्रो॒ वै वा इन्द्र॒ इन्द्रो॒ वै । \newline
21. वै स॒दृङ् ख्‌स॒दृङ् वै वै स॒दृङ् । \newline
22. स॒दृङ् दे॒वता॑भिर् दे॒वता॑भिः स॒दृङ् ख्‌स॒दृङ् दे॒वता॑भिः । \newline
23. स॒दृङ्ङिति॑ स - दृङ् । \newline
24. दे॒वता॑भि रासी दासीद् दे॒वता॑भिर् दे॒वता॑भि रासीत् । \newline
25. आ॒सी॒थ् स स आ॑सी दासी॒थ् सः । \newline
26. स न न स स न । \newline
27. न व्या॒वृतं॑ ॅव्या॒वृत॒म् न न व्या॒वृत᳚म् । \newline
28. व्या॒वृत॑ मगच्छ दगच्छद् व्या॒वृतं॑ ॅव्या॒वृत॑ मगच्छत् । \newline
29. व्या॒वृत॒मिति॑ वि - आ॒वृत᳚म् । \newline
30. अ॒ग॒च्छ॒थ् स सो॑ ऽगच्छ दगच्छ॒थ् सः । \newline
31. स प्र॒जाप॑तिम् प्र॒जाप॑तिꣳ॒॒ स स प्र॒जाप॑तिम् । \newline
32. प्र॒जाप॑ति॒ मुपोप॑ प्र॒जाप॑तिम् प्र॒जाप॑ति॒ मुप॑ । \newline
33. प्र॒जाप॑ति॒मिति॑ प्र॒जा - प॒ति॒म् । \newline
34. उपा॑धाव दधाव॒ दुपोपा॑ धावत् । \newline
35. अ॒धा॒व॒त् तस्मै॒ तस्मा॑ अधाव दधाव॒त् तस्मै᳚ । \newline
36. तस्मा॑ ए॒त मे॒तम् तस्मै॒ तस्मा॑ ए॒तम् । \newline
37. ए॒त मै॒न्द्र मै॒न्द्र मे॒त मे॒त मै॒न्द्रम् । \newline
38. ऐ॒न्द्र मेका॑दशकपाल॒ मेका॑दशकपाल मै॒न्द्र मै॒न्द्र मेका॑दशकपालम् । \newline
39. एका॑दशकपाल॒म् निर् णिरेका॑दशकपाल॒ मेका॑दशकपाल॒म् निः । \newline
40. एका॑दशकपाल॒मित्येका॑दश - क॒पा॒ल॒म् । \newline
41. निर॑वप दवप॒न् निर् णि र॑वपत् । \newline
42. अ॒व॒प॒त् तेन॒ तेना॑ वप दवप॒त् तेन॑ । \newline
43. तेनै॒वैव तेन॒ तेनै॒व । \newline
44. ए॒वास्मि॑न् नस्मिन् ने॒वैवास्मिन्न्॑ । \newline
45. अ॒स्मि॒न् नि॒न्द्रि॒य मि॑न्द्रि॒य म॑स्मिन् नस्मिन् निन्द्रि॒यम् । \newline
46. इ॒न्द्रि॒य म॑दधा ददधा दिन्द्रि॒य मि॑न्द्रि॒य म॑दधात् । \newline
47. अ॒द॒धा॒च् छक्व॑री॒ शक्व॑री अदधा ददधा॒च् छक्व॑री । \newline
48. शक्व॑री याज्यानुवा॒क्ये॑ याज्यानुवा॒क्ये॑ शक्व॑री॒ शक्व॑री याज्यानुवा॒क्ये᳚ । \newline
49. शक्व॑री॒ इति॒ शक्व॑री । \newline
50. या॒ज्या॒नु॒वा॒क्ये॑ अकरोदकरोद् याज्यानुवा॒क्ये॑ याज्यानुवा॒क्ये॑ अकरोत् । \newline
51. या॒ज्या॒नु॒वा॒क्ये॑ इति॑ याज्या - अ॒नु॒वा॒क्ये᳚ । \newline
52. अ॒क॒रो॒द् वज्रो॒ वज्रो॑ ऽकरो दकरो॒द् वज्रः॑ । \newline
53. वज्रो॒ वै वै वज्रो॒ वज्रो॒ वै । \newline
54. वै शक्व॑री॒ शक्व॑री॒ वै वै शक्व॑री । \newline
55. शक्व॑री॒ स स शक्व॑री॒ शक्व॑री॒ सः । \newline
56. स ए॑न मेनꣳ॒॒ स स ए॑नम् । \newline
57. ए॒नं॒ ॅवज्रो॒ वज्र॑ एन मेनं॒ ॅवज्रः॑ । \newline
58. वज्रो॒ भूत्यै॒ भूत्यै॒ वज्रो॒ वज्रो॒ भूत्यै᳚ । \newline
59. भूत्या॑ ऐन्धैन्ध॒ भूत्यै॒ भूत्या॑ ऐन्ध । \newline
60. ऐ॒न्ध॒ स स ऐ᳚न्धैन्ध॒ सः । \newline

\textbf{Ghana Paata } \newline

1. अ॒प॒रु॒द्ध्यमा॑नो वा वा ऽपरु॒द्ध्यमा॑नो ऽपरु॒द्ध्यमा॑नो॒ वेन्द्र॒ मिन्द्रं॑ ॅवा ऽपरु॒द्ध्यमा॑नो ऽपरु॒द्ध्यमा॑नो॒ वेन्द्र᳚म् । \newline
2. अ॒प॒रु॒द्ध्यमा॑न॒ इत्य॑प - रु॒द्ध्यमा॑नः । \newline
3. वेन्द्र॒ मिन्द्रं॑ ॅवा॒ वेन्द्र॑ मे॒वैवे न्द्रं॑ ॅवा॒ वेन्द्र॑ मे॒व । \newline
4. इन्द्र॑ मे॒वैवे न्द्र॒ मिन्द्र॑ मे॒व सु॒त्रामा॑णꣳ सु॒त्रामा॑ण मे॒वे न्द्र॒ मिन्द्र॑ मे॒व सु॒त्रामा॑णम् । \newline
5. ए॒व सु॒त्रामा॑णꣳ सु॒त्रामा॑ण मे॒वैव सु॒त्रामा॑णꣳ॒॒ स्वेन॒ स्वेन॑ सु॒त्रामा॑ण मे॒वैव सु॒त्रामा॑णꣳ॒॒ स्वेन॑ । \newline
6. सु॒त्रामा॑णꣳ॒॒ स्वेन॒ स्वेन॑ सु॒त्रामा॑णꣳ सु॒त्रामा॑णꣳ॒॒ स्वेन॑ भाग॒धेये॑न भाग॒धेये॑न॒ स्वेन॑ सु॒त्रामा॑णꣳ सु॒त्रामा॑णꣳ॒॒ स्वेन॑ भाग॒धेये॑न । \newline
7. सु॒त्रामा॑ण॒मिति॑ सु - त्रामा॑णम् । \newline
8. स्वेन॑ भाग॒धेये॑न भाग॒धेये॑न॒ स्वेन॒ स्वेन॑ भाग॒धेये॒नोपोप॑ भाग॒धेये॑न॒ स्वेन॒ स्वेन॑ भाग॒धेये॒नोप॑ । \newline
9. भा॒ग॒धेये॒नोपोप॑ भाग॒धेये॑न भाग॒धेये॒नोप॑ धावति धाव॒ त्युप॑ भाग॒धेये॑न भाग॒धेये॒नोप॑ धावति । \newline
10. भा॒ग॒धेये॒नेति॑ भाग - धेये॑न । \newline
11. उप॑ धावति धाव॒ त्युपोप॑ धावति॒ स स धा॑व॒ त्युपोप॑ धावति॒ सः । \newline
12. धा॒व॒ति॒ स स धा॑वति धावति॒ स ए॒वैव स धा॑वति धावति॒ स ए॒व । \newline
13. स ए॒वैव स स ए॒वैन॑ मेन मे॒व स स ए॒वैन᳚म् । \newline
14. ए॒वैन॑ मेन मे॒वैवैन॑म् त्रायते त्रायत एन मे॒वैवैन॑म् त्रायते । \newline
15. ए॒न॒म् त्रा॒य॒ते॒ त्रा॒य॒त॒ ए॒न॒ मे॒न॒म् त्रा॒य॒ते॒ ऽन॒प॒रु॒द्ध्यो॑ ऽनपरु॒द्ध्य स्त्रा॑यत एन मेनम् त्रायते ऽनपरु॒द्ध्यः । \newline
16. त्रा॒य॒ते॒ ऽन॒प॒रु॒द्ध्यो॑ ऽनपरु॒द्ध्य स्त्रा॑यते त्रायते ऽनपरु॒द्ध्यो भ॑वति भव त्यनपरु॒द्ध्य स्त्रा॑यते त्रायते ऽनपरु॒द्ध्यो भ॑वति । \newline
17. अ॒न॒प॒रु॒द्ध्यो भ॑वति भव त्यनपरु॒द्ध्यो॑ ऽनपरु॒द्ध्यो भ॑व॒तीन्द्र॒ इन्द्रो॑ भव त्यनपरु॒द्ध्यो॑ ऽनपरु॒द्ध्यो भ॑व॒तीन्द्रः॑ । \newline
18. अ॒न॒प॒रु॒द्ध्य इत्य॑नप - रु॒द्ध्यः । \newline
19. भ॒व॒तीन्द्र॒ इन्द्रो॑ भवति भव॒तीन्द्रो॒ वै वा इन्द्रो॑ भवति भव॒तीन्द्रो॒ वै । \newline
20. इन्द्रो॒ वै वा इन्द्र॒ इन्द्रो॒ वै स॒दृङ् ख्‌स॒दृङ् वा इन्द्र॒ इन्द्रो॒ वै स॒दृङ् । \newline
21. वै स॒दृङ् ख्‌स॒दृङ् वै वै स॒दृङ् दे॒वता॑भिर् दे॒वता॑भिः स॒दृङ् वै वै स॒दृङ् दे॒वता॑भिः । \newline
22. स॒दृङ् दे॒वता॑भिर् दे॒वता॑भिः स॒दृङ् ख्‌स॒दृङ् दे॒वता॑भि रासी दासीद् दे॒वता॑भिः स॒दृङ् ख्‌स॒दृङ् दे॒वता॑भि रासीत् । \newline
23. स॒दृङ्ङिति॑ स - दृङ् । \newline
24. दे॒वता॑भि रासी दासीद् दे॒वता॑भिर् दे॒वता॑भि रासी॒थ् स स आ॑सीद् दे॒वता॑भिर् दे॒वता॑भि रासी॒थ् सः । \newline
25. आ॒सी॒थ् स स आ॑सी दासी॒थ् स न न स आ॑सी दासी॒थ् स न । \newline
26. स न न स स न व्या॒वृतं॑ ॅव्या॒वृत॒म् न स स न व्या॒वृत᳚म् । \newline
27. न व्या॒वृतं॑ ॅव्या॒वृत॒म् न न व्या॒वृत॑ मगच्छ दगच्छद् व्या॒वृत॒म् न न व्या॒वृत॑ मगच्छत् । \newline
28. व्या॒वृत॑ मगच्छ दगच्छद् व्या॒वृतं॑ ॅव्या॒वृत॑ मगच्छ॒थ् स सो॑ ऽगच्छद् व्या॒वृतं॑ ॅव्या॒वृत॑ मगच्छ॒थ् सः । \newline
29. व्या॒वृत॒मिति॑ वि - आ॒वृत᳚म् । \newline
30. अ॒ग॒च्छ॒थ् स सो॑ ऽगच्छ दगच्छ॒थ् स प्र॒जाप॑तिम् प्र॒जाप॑तिꣳ॒॒ सो॑ ऽगच्छ दगच्छ॒थ् स प्र॒जाप॑तिम् । \newline
31. स प्र॒जाप॑तिम् प्र॒जाप॑तिꣳ॒॒ स स प्र॒जाप॑ति॒ मुपोप॑ प्र॒जाप॑तिꣳ॒॒ स स प्र॒जाप॑ति॒ मुप॑ । \newline
32. प्र॒जाप॑ति॒ मुपोप॑ प्र॒जाप॑तिम् प्र॒जाप॑ति॒ मुपा॑धाव दधाव॒ दुप॑ प्र॒जाप॑तिम् प्र॒जाप॑ति॒ मुपा॑धावत् । \newline
33. प्र॒जाप॑ति॒मिति॑ प्र॒जा - प॒ति॒म् । \newline
34. उपा॑धाव दधाव॒ दुपोपा॑धाव॒त् तस्मै॒ तस्मा॑ अधाव॒ दुपोपा॑धाव॒त् तस्मै᳚ । \newline
35. अ॒धा॒व॒त् तस्मै॒ तस्मा॑ अधाव दधाव॒त् तस्मा॑ ए॒त मे॒तम् तस्मा॑ अधाव दधाव॒त् तस्मा॑ ए॒तम् । \newline
36. तस्मा॑ ए॒त मे॒तम् तस्मै॒ तस्मा॑ ए॒त मै॒न्द्र मै॒न्द्र मे॒तम् तस्मै॒ तस्मा॑ ए॒त मै॒न्द्रम् । \newline
37. ए॒त मै॒न्द्र मै॒न्द्र मे॒त मे॒त मै॒न्द्र मेका॑दशकपाल॒ मेका॑दशकपाल मै॒न्द्र मे॒त मे॒त मै॒न्द्र मेका॑दशकपालम् । \newline
38. ऐ॒न्द्र मेका॑दशकपाल॒ मेका॑दशकपाल मै॒न्द्र मै॒न्द्र मेका॑दशकपाल॒म् निर् णिरेका॑दशकपाल मै॒न्द्र मै॒न्द्र मेका॑दशकपाल॒म् निः । \newline
39. एका॑दशकपाल॒म् निर् णिरेका॑दशकपाल॒ मेका॑दशकपाल॒म् नि र॑वप दवप॒न् निरेका॑दशकपाल॒ मेका॑दशकपाल॒म् निर॑वपत् । \newline
40. एका॑दशकपाल॒मित्येका॑दश - क॒पा॒ल॒म् । \newline
41. निर॑वप दवप॒न् निर् णिर॑वप॒त् तेन॒ तेना॑वप॒न् निर् णिर॑वप॒त् तेन॑ । \newline
42. अ॒व॒प॒त् तेन॒ तेना॑वप दवप॒त् तेनै॒वैव तेना॑वप दवप॒त् तेनै॒व । \newline
43. तेनै॒वैव तेन॒ तेनै॒वास्मि॑न् नस्मिन् ने॒व तेन॒ तेनै॒वास्मिन्न्॑ । \newline
44. ए॒वास्मि॑न् नस्मिन् ने॒वैवास्मि॑न् निन्द्रि॒य मि॑न्द्रि॒य म॑स्मिन् ने॒वैवास्मि॑न् निन्द्रि॒यम् । \newline
45. अ॒स्मि॒न् नि॒न्द्रि॒य मि॑न्द्रि॒य म॑स्मिन् नस्मिन् निन्द्रि॒य म॑दधा ददधा दिन्द्रि॒य म॑स्मिन् नस्मिन् निन्द्रि॒य म॑दधात् । \newline
46. इ॒न्द्रि॒य म॑दधा ददधा दिन्द्रि॒य मि॑न्द्रि॒य म॑दधा॒च् छक्व॑री॒ शक्व॑री अदधा दिन्द्रि॒य मि॑न्द्रि॒य म॑दधा॒च् छक्व॑री । \newline
47. अ॒द॒धा॒च् छक्व॑री॒ शक्व॑री अदधाद दधा॒च् छक्व॑री याज्यानुवा॒क्ये॑ याज्यानुवा॒क्ये॑ शक्व॑री अदधा ददधा॒च् छक्व॑री याज्यानुवा॒क्ये᳚ । \newline
48. शक्व॑री याज्यानुवा॒क्ये॑ याज्यानुवा॒क्ये॑ शक्व॑री॒ शक्व॑री याज्यानुवा॒क्ये॑ अकरो दकरोद् याज्यानुवा॒क्ये॑ शक्व॑री॒ शक्व॑री याज्यानुवा॒क्ये॑ अकरोत् । \newline
49. शक्व॑री॒ इति॒ शक्व॑री । \newline
50. या॒ज्या॒नु॒वा॒क्ये॑ अकरो दकरोद् याज्यानुवा॒क्ये॑ याज्यानुवा॒क्ये॑ अकरो॒द् वज्रो॒ वज्रो॑ ऽकरोद् याज्यानुवा॒क्ये॑ याज्यानुवा॒क्ये॑ अकरो॒द् वज्रः॑ । \newline
51. या॒ज्या॒नु॒वा॒क्ये॑ इति॑ याज्या - अ॒नु॒वा॒क्ये᳚ । \newline
52. अ॒क॒रो॒द् वज्रो॒ वज्रो॑ ऽकरो दकरो॒द् वज्रो॒ वै वै वज्रो॑ ऽकरो दकरो॒द् वज्रो॒ वै । \newline
53. वज्रो॒ वै वै वज्रो॒ वज्रो॒ वै शक्व॑री॒ शक्व॑री॒ वै वज्रो॒ वज्रो॒ वै शक्व॑री । \newline
54. वै शक्व॑री॒ शक्व॑री॒ वै वै शक्व॑री॒ स स शक्व॑री॒ वै वै शक्व॑री॒ सः । \newline
55. शक्व॑री॒ स स शक्व॑री॒ शक्व॑री॒ स ए॑न मेनꣳ॒॒ स शक्व॑री॒ शक्व॑री॒ स ए॑नम् । \newline
56. स ए॑न मेनꣳ॒॒ स स ए॑नं॒ ॅवज्रो॒ वज्र॑ एनꣳ॒॒ स स ए॑नं॒ ॅवज्रः॑ । \newline
57. ए॒नं॒ ॅवज्रो॒ वज्र॑ एन मेनं॒ ॅवज्रो॒ भूत्यै॒ भूत्यै॒ वज्र॑ एन मेनं॒ ॅवज्रो॒ भूत्यै᳚ । \newline
58. वज्रो॒ भूत्यै॒ भूत्यै॒ वज्रो॒ वज्रो॒ भूत्या॑ ऐन्धैन्ध॒ भूत्यै॒ वज्रो॒ वज्रो॒ भूत्या॑ ऐन्ध । \newline
59. भूत्या॑ ऐन्धैन्ध॒ भूत्यै॒ भूत्या॑ ऐन्ध॒ स स ऐ᳚न्ध॒ भूत्यै॒ भूत्या॑ ऐन्ध॒ सः । \newline
60. ऐ॒न्ध॒ स स ऐ᳚न्धैन्ध॒ सो॑ ऽभव दभव॒थ् स ऐ᳚न्धैन्ध॒ सो॑ ऽभवत् । \newline
\pagebreak
\markright{ TS 2.2.8.6  \hfill https://www.vedavms.in \hfill}
\addcontentsline{toc}{section}{ TS 2.2.8.6 }
\section*{ TS 2.2.8.6 }

\textbf{TS 2.2.8.6 } \newline
\textbf{Samhita Paata} \newline

सो॑ऽभव॒थ् सो॑ऽबिभेद्-भू॒तः प्र मा॑ धक्ष्य॒तीति॒ स प्र॒जाप॑तिं॒ पुन॒रुपा॑धाव॒थ् स प्र॒जाप॑तिः॒ शक्व॑र्या॒ अधि॑ रे॒वतीं॒ निर॑मिमीत॒ शान्त्या॒ अप्र॑दाहाय॒ योऽलꣳ॑ श्रि॒यै सन्थ्-स॒दृङ्ख्स॑मा॒नैः स्यात् तस्मा॑ ए॒तमै॒न्द्रमेका॑दशकपालं॒ निर्व॑पे॒दिन्द्र॑मे॒व स्वेन॑ भाग॒धेये॒नोप॑ धावति॒ स ए॒वास्मि॑न्निन्द्रि॒यं द॑धाति रे॒वती॑ पुरोऽनुवा॒क्या॑ भवति॒ ( ) शान्त्या॒ अप्र॑दाहाय॒ शक्व॑री या॒ज्या॑ वज्रो॒ वै शक्व॑री॒स ए॑नं॒ ॅवज्रो॒ भूत्या॑ इन्धे॒ भव॑त्ये॒व ॥ \newline

\textbf{Pada Paata} \newline

सः । अ॒भ॒व॒त् । सः । अ॒बि॒भे॒त् । भू॒तः । प्रेति॑ । मा॒ । ध॒क्ष्य॒ति॒ । इति॑ । सः । प्र॒जाप॑ति॒मिति॑ प्र॒जा-प॒ति॒म् । पुनः॑ । उपेति॑ । अ॒धा॒व॒त् । सः । प्र॒जाप॑ति॒रिति॑ प्र॒जा - प॒तिः॒ । शक्व॑र्याः । अधीति॑ । रे॒वती᳚म् । निरिति॑ । अ॒मि॒मी॒त॒ । शान्त्यै᳚ । अप्र॑दाहा॒येत्यप्र॑ - दा॒हा॒य॒ । यः । अल᳚म् । श्रि॒यै । सन्न् । स॒दृङ्ङिति॑ स - दृङ्ङ् । स॒मा॒नैः । स्यात् । तस्मै᳚ । ए॒तम् । ऐ॒न्द्रम् । एका॑दशकपाल॒मित्येका॑दश - क॒पा॒ल॒म् । निरिति॑ । व॒पे॒त् । इन्द्र᳚म् । ए॒व । स्वेन॑ । भा॒ग॒धेये॒नेति॑ भाग-धेये॑न । उपेति॑ । धा॒व॒ति॒ । सः । ए॒व । अ॒स्मि॒न्न् । इ॒न्द्रि॒यम् । द॒धा॒ति॒ । रे॒वती᳚ । पु॒रो॒नु॒वा॒क्येति॑ पुरः - अ॒नु॒वा॒क्या᳚ । भ॒व॒ति॒ ( ) । शान्त्यै᳚ । अप्र॑दाहा॒येत्यप्र॑ - दा॒हा॒य॒ । शक्व॑री । या॒ज्या᳚ । वज्रः॑ । वै । शक्व॑री । सः । ए॒न॒म् । वज्रः॑ । भूत्यै᳚ । इ॒न्धे॒ । भव॑ति । ए॒व ॥  \newline


\textbf{Krama Paata} \newline

सो॑ ऽभवत् । अ॒भ॒व॒थ् सः । सो॑ ऽबिभेत् । अ॒बि॒भे॒द् भू॒तः । भू॒तः प्र । प्र मा᳚ । मा॒ ध॒क्ष्य॒ति॒ । ध॒क्ष्य॒तीति॑ । इति॒ सः । स प्र॒जाप॑तिम् । प्र॒जाप॑ति॒म् पुनः॑ । प्र॒जाप॑ति॒मिति॑ प्र॒जा - प॒ति॒म् । पुन॒रुप॑ । उपा॑धावत् । अ॒धा॒व॒थ् सः । सः प्र॒जाप॑तिः । प्र॒जाप॑तिः॒ शक्व॑र्याः । प्र॒जाप॑ति॒रिति॑ प्र॒जा - प॒तिः॒ । शक्व॑र्या॒ अधि॑ । अधि॑ रे॒वती᳚म् । रे॒वती॒म् निः । निर॑मिमीत । अ॒मि॒मी॒त॒ शान्त्यै᳚ । शान्त्या॒ अप्र॑दाहाय । अप्र॑दाहाय॒ यः । अप्र॑दाहा॒येत्यप्र॑ - दा॒हा॒य॒ । योऽल᳚म् । अलꣳ॑ श्रि॒यै । श्रि॒यै सन्न् । सन्थ् स॒दृङ् । स॒दृङ्ख् स॑मा॒नैः । स॒दृङ्ङिति॑ स - दृङ् । स॒मा॒नैः स्यात् । स्यात् तस्मै᳚ । तस्मा॑ ए॒तम् । ए॒तमै॒न्द्रम् । ऐ॒न्द्रमेका॑दशकपालम् । एका॑दशकपाल॒म् निः । एका॑दशकपाल॒मित्येका॑दश - क॒पा॒ल॒म् । निर् व॑पेत् । व॒पे॒दिन्द्र᳚म् । इन्द्र॑मे॒व । ए॒व स्वेन॑ । स्वेन॑ भाग॒धेये॑न । भा॒ग॒धेये॒नोप॑ । भा॒ग॒धेये॒नेति॑ भाग - धेये॑न । उप॑ धावति । धा॒व॒ति॒ सः । स ए॒व । ए॒वास्मिन्न्॑ । अ॒स्मि॒न्नि॒न्द्रि॒यम् । इ॒न्द्रि॒यम् द॑धाति । द॒धा॒ति॒ रे॒वती᳚ । रे॒वती॑ पुरोनुवा॒क्या᳚ । पु॒रो॒नु॒वा॒क्या॑ भवति ( ) । पु॒रो॒नु॒वाक्येति॑ पुरः - अ॒नु॒वा॒क्या᳚ । भ॒व॒ति॒ शान्त्यै᳚ । शान्त्या॒ अप्र॑दाहाय । अप्र॑दाहाय॒ शक्व॑री । अप्र॑दाहा॒येत्यप्र॑ - दा॒हा॒य॒ । शक्व॑री या॒ज्या᳚ । या॒ज्या॑ वज्रः॑ । वज्रो॒ वै । वै शक्व॑री । शक्व॑री॒ सः । स ए॑नम् । ए॒नं॒ ॅवज्रः॑ । वज्रो॒ भूत्यै᳚ । भूत्या॑ इन्धे । इ॒न्धे॒ भव॑ति । भव॑त्ये॒व । ए॒वेत्ये॒व । \newline

\textbf{Jatai Paata} \newline

1. सो॑ ऽभव दभव॒थ् स सो॑ ऽभवत् । \newline
2. अ॒भ॒व॒थ् स सो॑ ऽभव दभव॒थ् सः । \newline
3. सो॑ ऽबिभे दबिभे॒थ् स सो॑ ऽबिभेत् । \newline
4. अ॒बि॒भे॒द् भू॒तो भू॒तो॑ ऽबिभे दबिभेद् भू॒तः । \newline
5. भू॒तः प्र प्र भू॒तो भू॒तः प्र । \newline
6. प्र मा॑ मा॒ प्र प्र मा᳚ । \newline
7. मा॒ ध॒क्ष्य॒ति॒ ध॒क्ष्य॒ति॒ मा॒ मा॒ ध॒क्ष्य॒ति॒ । \newline
8. ध॒क्ष्य॒तीतीति॑ धक्ष्यति धक्ष्य॒तीति॑ । \newline
9. इति॒ स स इतीति॒ सः । \newline
10. स प्र॒जाप॑तिम् प्र॒जाप॑तिꣳ॒॒ स स प्र॒जाप॑तिम् । \newline
11. प्र॒जाप॑ति॒म् पुनः॒ पुनः॑ प्र॒जाप॑तिम् प्र॒जाप॑ति॒म् पुनः॑ । \newline
12. प्र॒जाप॑ति॒मिति॑ प्र॒जा - प॒ति॒म् । \newline
13. पुन॒ रुपोप॒ पुनः॒ पुन॒ रुप॑ । \newline
14. उपा॑धाव दधाव॒ दुपोपा॑ धावत् । \newline
15. अ॒धा॒व॒थ् स सो॑ ऽधाव दधाव॒थ् सः । \newline
16. स प्र॒जाप॑तिः प्र॒जाप॑तिः॒ स स प्र॒जाप॑तिः । \newline
17. प्र॒जाप॑तिः॒ शक्व॑र्याः॒ शक्व॑र्याः प्र॒जाप॑तिः प्र॒जाप॑तिः॒ शक्व॑र्याः । \newline
18. प्र॒जाप॑ति॒रिति॑ प्र॒जा - प॒तिः॒ । \newline
19. शक्व॑र्या॒ अध्यधि॒ शक्व॑र्याः॒ शक्व॑र्या॒ अधि॑ । \newline
20. अधि॑ रे॒वतीꣳ॑ रे॒वती॒ मध्यधि॑ रे॒वती᳚म् । \newline
21. रे॒वती॒म् निर् णी रे॒वतीꣳ॑ रे॒वती॒म् निः । \newline
22. निर॑मिमीता मिमीत॒ निर् णि र॑मिमीत । \newline
23. अ॒मि॒मी॒त॒ शान्त्यै॒ शान्त्या॑ अमिमीता मिमीत॒ शान्त्यै᳚ । \newline
24. शान्त्या॒ अप्र॑दाहा॒या प्र॑दाहाय॒ शान्त्यै॒ शान्त्या॒ अप्र॑दाहाय । \newline
25. अप्र॑दाहाय॒ यो यो ऽप्र॑दाहा॒या प्र॑दाहाय॒ यः । \newline
26. अप्र॑दाहा॒येत्यप्र॑ - दा॒हा॒य॒ । \newline
27. यो ऽल॒ मलं॒ ॅयो यो ऽल᳚म् । \newline
28. अलꣳ॑ श्रि॒यै श्रि॒या अल॒ मलꣳ॑ श्रि॒यै । \newline
29. श्रि॒यै सन् थ्सञ् छ्रि॒यै श्रि॒यै सन्न् । \newline
30. सन् थ्स॒दृङ् ख्‌स॒दृङ् ख्‌सन् थ्सन् थ्स॒दृङ् । \newline
31. स॒दृङ् ख्‌स॑मा॒नैः स॑मा॒नैः स॒दृङ् ख्‌स॒दृङ् ख्‌स॑मा॒नैः । \newline
32. स॒दृङ्ङिति॑ स - दृङ् । \newline
33. स॒मा॒नैः स्याथ् स्याथ् स॑मा॒नैः स॑मा॒नैः स्यात् । \newline
34. स्यात् तस्मै॒ तस्मै॒ स्याथ् स्यात् तस्मै᳚ । \newline
35. तस्मा॑ ए॒त मे॒तम् तस्मै॒ तस्मा॑ ए॒तम् । \newline
36. ए॒त मै॒न्द्र मै॒न्द्र मे॒त मे॒त मै॒न्द्रम् । \newline
37. ऐ॒न्द्र मेका॑दशकपाल॒ मेका॑दशकपाल मै॒न्द्र मै॒न्द्र मेका॑दशकपालम् । \newline
38. एका॑दशकपाल॒म् निर् णिरेका॑दशकपाल॒ मेका॑दशकपाल॒म् निः । \newline
39. एका॑दशकपाल॒मित्येका॑दश - क॒पा॒ल॒म् । \newline
40. निर् व॑पेद् वपे॒न् निर् णिर् व॑पेत् । \newline
41. व॒पे॒दिन्द्र॒ मिन्द्रं॑ ॅवपेद् वपे॒दिन्द्र᳚म् । \newline
42. इन्द्र॑ मे॒वैवे न्द्र॒ मिन्द्र॑ मे॒व । \newline
43. ए॒व स्वेन॒ स्वेनै॒वैव स्वेन॑ । \newline
44. स्वेन॑ भाग॒धेये॑न भाग॒धेये॑न॒ स्वेन॒ स्वेन॑ भाग॒धेये॑न । \newline
45. भा॒ग॒धेये॒नोपोप॑ भाग॒धेये॑न भाग॒धेये॒नोप॑ । \newline
46. भा॒ग॒धेये॒नेति॑ भाग - धेये॑न । \newline
47. उप॑ धावति धाव॒ त्युपोप॑ धावति । \newline
48. धा॒व॒ति॒ स स धा॑वति धावति॒ सः । \newline
49. स ए॒वैव स स ए॒व । \newline
50. ए॒वास्मि॑न् नस्मिन् ने॒वैवास्मिन्न्॑ । \newline
51. अ॒स्मि॒न् नि॒न्द्रि॒य मि॑न्द्रि॒य म॑स्मिन् नस्मिन् निन्द्रि॒यम् । \newline
52. इ॒न्द्रि॒यम् द॑धाति दधातीन्द्रि॒य मि॑न्द्रि॒यम् द॑धाति । \newline
53. द॒धा॒ति॒ रे॒वती॑ रे॒वती॑ दधाति दधाति रे॒वती᳚ । \newline
54. रे॒वती॑ पुरोनुवा॒क्या॑ पुरोनुवा॒क्या॑ रे॒वती॑ रे॒वती॑ पुरोनुवा॒क्या᳚ । \newline
55. पु॒रो॒नु॒वा॒क्या॑ भवति भवति पुरोनुवा॒क्या॑ पुरोनुवा॒क्या॑ भवति । \newline
56. पु॒रो॒नु॒वा॒क्येति॑ पुरः - अ॒नु॒वा॒क्या᳚ । \newline
57. भ॒व॒ति॒ शान्त्यै॒ शान्त्यै॑ भवति भवति॒ शान्त्यै᳚ । \newline
58. शान्त्या॒ अप्र॑दाहा॒या प्र॑दाहाय॒ शान्त्यै॒ शान्त्या॒ अप्र॑दाहाय । \newline
59. अप्र॑दाहाय॒ शक्व॑री॒ शक्व॒र्य प्र॑दाहा॒या प्र॑दाहाय॒ शक्व॑री । \newline
60. अप्र॑दाहा॒येत्यप्र॑ - दा॒हा॒य॒ । \newline
61. शक्व॑री या॒ज्या॑ या॒ज्या॑ शक्व॑री॒ शक्व॑री या॒ज्या᳚ । \newline
62. या॒ज्या॑ वज्रो॒ वज्रो॑ या॒ज्या॑ या॒ज्या॑ वज्रः॑ । \newline
63. वज्रो॒ वै वै वज्रो॒ वज्रो॒ वै । \newline
64. वै शक्व॑री॒ शक्व॑री॒ वै वै शक्व॑री । \newline
65. शक्व॑री॒ स स शक्व॑री॒ शक्व॑री॒ सः । \newline
66. स ए॑न मेनꣳ॒॒ स स ए॑नम् । \newline
67. ए॒नं॒ ॅवज्रो॒ वज्र॑ एन मेनं॒ ॅवज्रः॑ । \newline
68. वज्रो॒ भूत्यै॒ भूत्यै॒ वज्रो॒ वज्रो॒ भूत्यै᳚ । \newline
69. भूत्या॑ इन्ध इन्धे॒ भूत्यै॒ भूत्या॑ इन्धे । \newline
70. इ॒न्धे॒ भव॑ति॒ भव॑तीन्ध इन्धे॒ भव॑ति । \newline
71. भव॑ त्ये॒वैव भव॑ति॒ भव॑ त्ये॒व । \newline
72. ए॒वे त्ये॒व । \newline

\textbf{Ghana Paata } \newline

1. सो॑ ऽभवद भव॒थ् स सो॑ ऽभव॒थ् स सो॑ ऽभव॒थ् स सो॑ ऽभव॒थ् सः । \newline
2. अ॒भ॒व॒थ् स सो॑ ऽभव दभव॒थ् सो॑ ऽबिभे दबिभे॒थ् सो॑ ऽभव दभव॒थ् सो॑ ऽबिभेत् । \newline
3. सो॑ ऽबिभे दबिभे॒थ् स सो॑ ऽबिभेद् भू॒तो भू॒तो॑ ऽबिभे॒थ् स सो॑ ऽबिभेद् भू॒तः । \newline
4. अ॒बि॒भे॒द् भू॒तो भू॒तो॑ ऽबिभे दबिभेद् भू॒तः प्र प्र भू॒तो॑ ऽबिभे दबिभेद् भू॒तः प्र । \newline
5. भू॒तः प्र प्र भू॒तो भू॒तः प्र मा॑ मा॒ प्र भू॒तो भू॒तः प्र मा᳚ । \newline
6. प्र मा॑ मा॒ प्र प्र मा॑ धक्ष्यति धक्ष्यति मा॒ प्र प्र मा॑ धक्ष्यति । \newline
7. मा॒ ध॒क्ष्य॒ति॒ ध॒क्ष्य॒ति॒ मा॒ मा॒ ध॒क्ष्य॒तीतीति॑ धक्ष्यति मा मा धक्ष्य॒तीति॑ । \newline
8. ध॒क्ष्य॒तीतीति॑ धक्ष्यति धक्ष्य॒तीति॒ स स इति॑ धक्ष्यति धक्ष्य॒तीति॒ सः । \newline
9. इति॒ स स इतीति॒ स प्र॒जाप॑तिम् प्र॒जाप॑तिꣳ॒॒ स इतीति॒ स प्र॒जाप॑तिम् । \newline
10. स प्र॒जाप॑तिम् प्र॒जाप॑तिꣳ॒॒ स स प्र॒जाप॑ति॒म् पुनः॒ पुनः॑ प्र॒जाप॑तिꣳ॒॒ स स प्र॒जाप॑ति॒म् पुनः॑ । \newline
11. प्र॒जाप॑ति॒म् पुनः॒ पुनः॑ प्र॒जाप॑तिम् प्र॒जाप॑ति॒म् पुन॒ रुपोप॒ पुनः॑ प्र॒जाप॑तिम् प्र॒जाप॑ति॒म् पुन॒ रुप॑ । \newline
12. प्र॒जाप॑ति॒मिति॑ प्र॒जा - प॒ति॒म् । \newline
13. पुन॒ रुपोप॒ पुनः॒ पुन॒ रुपा॑धाव दधाव॒ दुप॒ पुनः॒ पुन॒ रुपा॑धावत् । \newline
14. उपा॑धाव दधाव॒ दुपोपा॑धाव॒थ् स सो॑ ऽधाव॒ दुपोपा॑धाव॒थ् सः । \newline
15. अ॒धा॒व॒थ् स सो॑ ऽधाव दधाव॒थ् स प्र॒जाप॑तिः प्र॒जाप॑तिः॒ सो॑ ऽधाव दधाव॒थ् स प्र॒जाप॑तिः । \newline
16. स प्र॒जाप॑तिः प्र॒जाप॑तिः॒ स स प्र॒जाप॑तिः॒ शक्व॑र्याः॒ शक्व॑र्याः प्र॒जाप॑तिः॒ स स प्र॒जाप॑तिः॒ शक्व॑र्याः । \newline
17. प्र॒जाप॑तिः॒ शक्व॑र्याः॒ शक्व॑र्याः प्र॒जाप॑तिः प्र॒जाप॑तिः॒ शक्व॑र्या॒ अध्यधि॒ शक्व॑र्याः प्र॒जाप॑तिः प्र॒जाप॑तिः॒ शक्व॑र्या॒ अधि॑ । \newline
18. प्र॒जाप॑ति॒रिति॑ प्र॒जा - प॒तिः॒ । \newline
19. शक्व॑र्या॒ अध्यधि॒ शक्व॑र्याः॒ शक्व॑र्या॒ अधि॑ रे॒वतीꣳ॑ रे॒वती॒ मधि॒ शक्व॑र्याः॒ शक्व॑र्या॒ अधि॑ रे॒वती᳚म् । \newline
20. अधि॑ रे॒वतीꣳ॑ रे॒वती॒ मध्यधि॑ रे॒वती॒म् निर् णी रे॒वती॒ मध्यधि॑ रे॒वती॒म् निः । \newline
21. रे॒वती॒म् निर् णी रे॒वतीꣳ॑ रे॒वती॒म् निर॑मिमीता मिमीत॒ नी रे॒वतीꣳ॑ रे॒वती॒म् निर॑मिमीत । \newline
22. निर॑मिमीता मिमीत॒ निर् णिर॑मिमीत॒ शान्त्यै॒ शान्त्या॑ अमिमीत॒ निर् णिर॑मिमीत॒ शान्त्यै᳚ । \newline
23. अ॒मि॒मी॒त॒ शान्त्यै॒ शान्त्या॑ अमिमीता मिमीत॒ शान्त्या॒ अप्र॑दाहा॒या प्र॑दाहाय॒ शान्त्या॑ अमिमीता मिमीत॒ शान्त्या॒ अप्र॑दाहाय । \newline
24. शान्त्या॒ अप्र॑दाहा॒या प्र॑दाहाय॒ शान्त्यै॒ शान्त्या॒ अप्र॑दाहाय॒ यो यो ऽप्र॑दाहाय॒ शान्त्यै॒ शान्त्या॒ अप्र॑दाहाय॒ यः । \newline
25. अप्र॑दाहाय॒ यो यो ऽप्र॑दाहा॒या प्र॑दाहाय॒ यो ऽल॒ मलं॒ ॅयो ऽप्र॑दाहा॒या प्र॑दाहाय॒ यो ऽल᳚म् । \newline
26. अप्र॑दाहा॒येत्यप्र॑ - दा॒हा॒य॒ । \newline
27. यो ऽल॒ मलं॒ ॅयो यो ऽलꣳ॑ श्रि॒यै श्रि॒या अलं॒ ॅयो यो ऽलꣳ॑ श्रि॒यै । \newline
28. अलꣳ॑ श्रि॒यै श्रि॒या अल॒ मलꣳ॑ श्रि॒यै सन् थ्सञ् छ्रि॒या अल॒ मलꣳ॑ श्रि॒यै सन्न् । \newline
29. श्रि॒यै सन् थ्सञ् छ्रि॒यै श्रि॒यै सन् थ्स॒दृङ् ख्‍स॒दृङ् ख्‍सञ् छ्रि॒यै श्रि॒यै सन् थ्स॒दृङ् । \newline
30. सन् थ्स॒दृङ् ख्‌स॒दृङ् ख्‍सन् थ्सन् थ्स॒दृङ् ख्‌स॑मा॒नैः स॑मा॒नैः स॒दृङ् ख्‌सन् थ्सन् थ्स॒दृङ् ख्‌स॑मा॒नैः । \newline
31. स॒दृङ् ख्‌स॑मा॒नैः स॑मा॒नैः स॒दृङ् ख्‌स॒दृङ् ख्‌स॑मा॒नैः स्याथ् स्याथ् स॑मा॒नैः स॒दृङ् ख्‌स॒दृङ् ख्‌स॑मा॒नैः स्यात् । \newline
32. स॒दृङ्ङिति॑ स - दृङ् । \newline
33. स॒मा॒नैः स्याथ् स्याथ् स॑मा॒नैः स॑मा॒नैः स्यात् तस्मै॒ तस्मै॒ स्याथ् स॑मा॒नैः स॑मा॒नैः स्यात् तस्मै᳚ । \newline
34. स्यात् तस्मै॒ तस्मै॒ स्याथ् स्यात् तस्मा॑ ए॒त मे॒तम् तस्मै॒ स्याथ् स्यात् तस्मा॑ ए॒तम् । \newline
35. तस्मा॑ ए॒त मे॒तम् तस्मै॒ तस्मा॑ ए॒त मै॒न्द्र मै॒न्द्र मे॒तम् तस्मै॒ तस्मा॑ ए॒त मै॒न्द्रम् । \newline
36. ए॒त मै॒न्द्र मै॒न्द्र मे॒त मे॒त मै॒न्द्र मेका॑दशकपाल॒ मेका॑दशकपाल मै॒न्द्र मे॒त मे॒त मै॒न्द्र मेका॑दशकपालम् । \newline
37. ऐ॒न्द्र मेका॑दशकपाल॒ मेका॑दशकपाल मै॒न्द्र मै॒न्द्र मेका॑दशकपाल॒म् निर् णिरेका॑दशकपाल मै॒न्द्र मै॒न्द्र मेका॑दशकपाल॒म् निः । \newline
38. एका॑दशकपाल॒म् निर् णिरेका॑दशकपाल॒ मेका॑दशकपाल॒म् निर् व॑पेद् वपे॒न् निरेका॑दशकपाल॒ मेका॑दशकपाल॒म् निर् व॑पेत् । \newline
39. एका॑दशकपाल॒मित्येका॑दश - क॒पा॒ल॒म् । \newline
40. निर् व॑पेद् वपे॒न् निर् णिर् व॑पे॒ दिन्द्र॒ मिन्द्रं॑ ॅवपे॒न् निर् णिर् व॑पे॒ दिन्द्र᳚म् । \newline
41. व॒पे॒दिन्द्र॒ मिन्द्रं॑ ॅवपेद् वपे॒ दिन्द्र॑ मे॒वैवे न्द्रं॑ ॅवपेद् वपे॒ दिन्द्र॑ मे॒व । \newline
42. इन्द्र॑ मे॒वैवे न्द्र॒ मिन्द्र॑ मे॒व स्वेन॒ स्वेनै॒वे न्द्र॒ मिन्द्र॑ मे॒व स्वेन॑ । \newline
43. ए॒व स्वेन॒ स्वेनै॒वैव स्वेन॑ भाग॒धेये॑न भाग॒धेये॑न॒ स्वेनै॒वैव स्वेन॑ भाग॒धेये॑न । \newline
44. स्वेन॑ भाग॒धेये॑न भाग॒धेये॑न॒ स्वेन॒ स्वेन॑ भाग॒धेये॒नोपोप॑ भाग॒धेये॑न॒ स्वेन॒ स्वेन॑ भाग॒धेये॒नोप॑ । \newline
45. भा॒ग॒धेये॒नोपोप॑ भाग॒धेये॑न भाग॒धेये॒नोप॑ धावति धाव॒ त्युप॑ भाग॒धेये॑न भाग॒धेये॒नोप॑ धावति । \newline
46. भा॒ग॒धेये॒नेति॑ भाग - धेये॑न । \newline
47. उप॑ धावति धाव॒ त्युपोप॑ धावति॒ स स धा॑व॒ त्युपोप॑ धावति॒ सः । \newline
48. धा॒व॒ति॒ स स धा॑वति धावति॒ स ए॒वैव स धा॑वति धावति॒ स ए॒व । \newline
49. स ए॒वैव स स ए॒वास्मि॑न् नस्मिन् ने॒व स स ए॒वास्मिन्न्॑ । \newline
50. ए॒वास्मि॑न् नस्मिन् ने॒वैवास्मि॑न् निन्द्रि॒य मि॑न्द्रि॒य म॑स्मिन् ने॒वैवास्मि॑न् निन्द्रि॒यम् । \newline
51. अ॒स्मि॒न् नि॒न्द्रि॒य मि॑न्द्रि॒य म॑स्मिन् नस्मिन् निन्द्रि॒यम् द॑धाति दधातीन्द्रि॒य म॑स्मिन् नस्मिन् निन्द्रि॒यम् द॑धाति । \newline
52. इ॒न्द्रि॒यम् द॑धाति दधातीन्द्रि॒य मि॑न्द्रि॒यम् द॑धाति रे॒वती॑ रे॒वती॑ दधातीन्द्रि॒य मि॑न्द्रि॒यम् द॑धाति रे॒वती᳚ । \newline
53. द॒धा॒ति॒ रे॒वती॑ रे॒वती॑ दधाति दधाति रे॒वती॑ पुरोनुवा॒क्या॑ पुरोनुवा॒क्या॑ रे॒वती॑ दधाति दधाति रे॒वती॑ पुरोनुवा॒क्या᳚ । \newline
54. रे॒वती॑ पुरोनुवा॒क्या॑ पुरोनुवा॒क्या॑ रे॒वती॑ रे॒वती॑ पुरोनुवा॒क्या॑ भवति भवति पुरोनुवा॒क्या॑ रे॒वती॑ रे॒वती॑ पुरोनुवा॒क्या॑ भवति । \newline
55. पु॒रो॒नु॒वा॒क्या॑ भवति भवति पुरोनुवा॒क्या॑ पुरोनुवा॒क्या॑ भवति॒ शान्त्यै॒ शान्त्यै॑ भवति पुरोनुवा॒क्या॑ पुरोनुवा॒क्या॑ भवति॒ शान्त्यै᳚ । \newline
56. पु॒रो॒नु॒वा॒क्येति॑ पुरः - अ॒नु॒वा॒क्या᳚ । \newline
57. भ॒व॒ति॒ शान्त्यै॒ शान्त्यै॑ भवति भवति॒ शान्त्या॒ अप्र॑दाहा॒या प्र॑दाहाय॒ शान्त्यै॑ भवति भवति॒ शान्त्या॒ अप्र॑दाहाय । \newline
58. शान्त्या॒ अप्र॑दाहा॒या प्र॑दाहाय॒ शान्त्यै॒ शान्त्या॒ अप्र॑दाहाय॒ शक्व॑री॒ शक्व॒र्य प्र॑दाहाय॒ शान्त्यै॒ शान्त्या॒ अप्र॑दाहाय॒ शक्व॑री । \newline
59. अप्र॑दाहाय॒ शक्व॑री॒ शक्व॒र्य प्र॑दाहा॒या प्र॑दाहाय॒ शक्व॑री या॒ज्या॑ या॒ज्या॑ शक्व॒र्य प्र॑दाहा॒या प्र॑दाहाय॒ शक्व॑री या॒ज्या᳚ । \newline
60. अप्र॑दाहा॒येत्यप्र॑ - दा॒हा॒य॒ । \newline
61. शक्व॑री या॒ज्या॑ या॒ज्या॑ शक्व॑री॒ शक्व॑री या॒ज्या॑ वज्रो॒ वज्रो॑ या॒ज्या॑ शक्व॑री॒ शक्व॑री या॒ज्या॑ वज्रः॑ । \newline
62. या॒ज्या॑ वज्रो॒ वज्रो॑ या॒ज्या॑ या॒ज्या॑ वज्रो॒ वै वै वज्रो॑ या॒ज्या॑ या॒ज्या॑ वज्रो॒ वै । \newline
63. वज्रो॒ वै वै वज्रो॒ वज्रो॒ वै शक्व॑री॒ शक्व॑री॒ वै वज्रो॒ वज्रो॒ वै शक्व॑री । \newline
64. वै शक्व॑री॒ शक्व॑री॒ वै वै शक्व॑री॒ स स शक्व॑री॒ वै वै शक्व॑री॒ सः । \newline
65. शक्व॑री॒ स स शक्व॑री॒ शक्व॑री॒ स ए॑न मेनꣳ॒॒ स शक्व॑री॒ शक्व॑री॒ स ए॑नम् । \newline
66. स ए॑न मेनꣳ॒॒ स स ए॑नं॒ ॅवज्रो॒ वज्र॑ एनꣳ॒॒ स स ए॑नं॒ ॅवज्रः॑ । \newline
67. ए॒नं॒ ॅवज्रो॒ वज्र॑ एन मेनं॒ ॅवज्रो॒ भूत्यै॒ भूत्यै॒ वज्र॑ एन मेनं॒ ॅवज्रो॒ भूत्यै᳚ । \newline
68. वज्रो॒ भूत्यै॒ भूत्यै॒ वज्रो॒ वज्रो॒ भूत्या॑ इन्ध इन्धे॒ भूत्यै॒ वज्रो॒ वज्रो॒ भूत्या॑ इन्धे । \newline
69. भूत्या॑ इन्ध इन्धे॒ भूत्यै॒ भूत्या॑ इन्धे॒ भव॑ति॒ भव॑तीन्धे॒ भूत्यै॒ भूत्या॑ इन्धे॒ भव॑ति । \newline
70. इ॒न्धे॒ भव॑ति॒ भव॑तीन्ध इन्धे॒ भव॑ त्ये॒वैव भव॑तीन्ध इन्धे॒ भव॑ त्ये॒व । \newline
71. भव॑ त्ये॒वैव भव॑ति॒ भव॑ त्ये॒व । \newline
72. ए॒वेत्ये॒व । \newline
\pagebreak
\markright{ TS 2.2.9.1  \hfill https://www.vedavms.in \hfill}
\addcontentsline{toc}{section}{ TS 2.2.9.1 }
\section*{ TS 2.2.9.1 }

\textbf{TS 2.2.9.1 } \newline
\textbf{Samhita Paata} \newline

आ॒ग्ना॒-वै॒ष्ण॒व-मेका॑दशकपालं॒ निर्व॑पेदभि॒चर॒न्थ्-सर॑स्व॒त्याज्य॑ भागा॒ स्याद्-बा॑र्.हस्प॒त् यश्च॒रुर्यदा᳚ग्ना-वैष्ण॒व एका॑दशकपालो॒ भव॑त्य॒ग्निः सर्वा॑ दे॒वता॒ विष्णु॑र्य॒ज्ञो दे॒वता॑भिश्चै॒वैनं॑ ॅय॒ज्ञेन॑ चा॒भि च॑रति॒सर॑स्व॒त्याज्य॑भागा भवति॒ वाग्वै सर॑स्वती वा॒चैवैन॑म॒भि च॑रति बार्.हस्प॒त्यश्च॒रु र्भ॑वति॒ ब्रह्म॒ वै दे॒वानां॒ बृह॒स्पति॒ र्ब्रह्म॑णै॒वैन॑म॒भि च॑रति॒ - [  ] \newline

\textbf{Pada Paata} \newline

आ॒ग्ना॒वै॒ष्ण॒वमित्या᳚ग्ना - वै॒ष्ण॒वम् । एका॑दशकपाल॒मित्येका॑दश - क॒पा॒ल॒म् । निरिति॑ । व॒पे॒त् । अ॒भि॒चर॒न्नित्य॑भि - चर॒न्न्॑ । सर॑स्वती । आज्य॑भा॒गेत्याज्य॑ - भा॒गा॒ । स्यात् । बा॒र्.॒ह॒स्प॒त्यः । च॒रुः । यत् । आ॒ग्ना॒वै॒ष्ण॒व इत्या᳚ग्ना - वै॒ष्ण॒वः । एका॑दशकपाल॒ इत्येका॑दश-क॒पा॒लः॒ । भव॑ति । अ॒ग्निः । सर्वाः᳚ । दे॒वताः᳚ । विष्णुः॑ । य॒ज्ञ्ः । दे॒वता॑भिः । च॒ । ए॒व । ए॒न॒म् । य॒ज्ञेन॑ । च॒ । अ॒भीति॑ । च॒र॒ति॒ । सर॑स्वती । आज्य॑भा॒गेत्याज्य॑-भा॒गा॒ । भ॒व॒ति॒ । वाक् । वै । सर॑स्वती । वा॒चा । ए॒व । ए॒न॒म् । अ॒भीति॑ । च॒र॒ति॒ । बा॒र्.॒ह॒स्प॒त्यः । च॒रुः । भ॒व॒ति॒ । ब्रह्म॑ । वै । दे॒वाना᳚म् । बृह॒स्पतिः॑ । ब्रह्म॑णा । ए॒व । ए॒न॒म् । अ॒भीति॑ । च॒र॒ति॒ ।  \newline


\textbf{Krama Paata} \newline

आ॒ग्ना॒वै॒ष्ण॒वमेका॑दशकपालम् । आ॒ग्ना॒वै॒ष्ण॒वमित्या᳚ग्ना - वै॒ष्ण॒वम् । एका॑दशकपाल॒म् निः । एका॑दशकपाल॒मित्येका॑दश - क॒पा॒ल॒म् । निर् व॑पेत् । व॒पे॒द॒भि॒चरन्न्॑ । अ॒भि॒चर॒न्थ् सर॑स्वती । अ॒भि॒चर॒न्नित्य॑भि - चरन्न्॑ । सर॑स्व॒त्याज्य॑भागा । आज्य॑भागा॒ स्यात् । आज्य॑भा॒गेत्याज्य॑ - भा॒गा॒ । स्याद्,बा॑र्.हस्प॒त्यः । बा॒र्॒.ह॒स्प॒त्य श्च॒रुः । च॒रुर् यत् । यदा᳚ग्नावैष्ण॒वः । आ॒ग्ना॒वै॒ष्ण॒व एका॑दशकपालः । आ॒ग्ना॒वै॒ष्ण॒व इत्या᳚ग्ना - वै॒ष्ण॒वः । एका॑दशकपालो॒ भव॑ति । एका॑दशकपाल॒ इत्येका॑दश - क॒पा॒लः॒ । भव॑त्य॒ग्निः । अ॒ग्निः सर्वाः᳚ । सर्वा॑ दे॒वताः᳚ । दे॒वता॒ विष्णुः॑ । विष्णु॑र्,य॒ज्ञ्ः । य॒ज्ञो दे॒वता॑भिः । दे॒वता॑भिश्च । चै॒व । ए॒वैन᳚म् । ए॒नं॒ ॅय॒ज्ञेन॑ । य॒ज्ञेन॑ च । चा॒भि । अ॒भि च॑रति । च॒र॒ति॒ सर॑स्वती । सर॑स्व॒त्याज्य॑भागा । आज्य॑भागा भवति । आज्य॑भा॒गेत्याज्य॑ - भा॒गा॒ । भ॒व॒ति॒ वाक् । वाग् वै । वै सर॑स्वती । सर॑स्वती वा॒चा । वा॒चैव । ए॒वैन᳚म् । ए॒न॒म॒भि । अ॒भि च॑रति । च॒र॒ति॒ बा॒र्॒.ह॒स्प॒त्यः । बा॒र्॒.ह॒स्प॒त्य श्च॒रुः । च॒रुर्,भ॑वति । भ॒व॒ति॒ ब्रह्म॑ । ब्रह्म॒ वै । वै दे॒वाना᳚म् । दे॒वाना॒म् बृह॒स्पतिः॑ । बृह॒स्पति॒र् ब्रह्म॑णा । ब्रह्म॑णै॒व । ए॒वैन᳚म् । ए॒न॒म॒भि । अ॒भि च॑रति । च॒र॒ति॒ प्रति॑ \newline

\textbf{Jatai Paata} \newline

1. आ॒ग्ना॒वै॒ष्ण॒व मेका॑दशकपाल॒ मेका॑दशकपाल माग्नावैष्ण॒व मा᳚ग्नावैष्ण॒व मेका॑दशकपालम् । \newline
2. आ॒ग्ना॒वै॒ष्ण॒वमित्या᳚ग्ना - वै॒ष्ण॒वम् । \newline
3. एका॑दशकपाल॒म् निर् णिरेका॑दशकपाल॒ मेका॑दशकपाल॒म् निः । \newline
4. एका॑दशकपाल॒मित्येका॑दश - क॒पा॒ल॒म् । \newline
5. निर् व॑पेद् वपे॒न् निर् णिर् व॑पेत् । \newline
6. व॒पे॒ द॒भि॒चर॑न् नभि॒चरन्॑. वपेद् वपे दभि॒चरन्न्॑ । \newline
7. अ॒भि॒चर॒न् थ्सर॑स्वती॒ सर॑स्व त्यभि॒चर॑न् नभि॒चर॒न् थ्सर॑स्वती । \newline
8. अ॒भि॒चर॒न्नित्य॑भि - चरन्न्॑ । \newline
9. सर॑स्व॒ त्याज्य॑भा॒गा ऽऽज्य॑भागा॒ सर॑स्वती॒ सर॑स्व॒ त्याज्य॑भागा । \newline
10. आज्य॑भागा॒ स्याथ् स्या दाज्य॑भा॒गा ऽऽज्य॑भागा॒ स्यात् । \newline
11. आज्य॑भा॒गेत्याज्य॑ - भा॒गा॒ । \newline
12. स्याद् बा॑र्.हस्प॒त्यो बा॑र्.हस्प॒त्यः स्याथ् स्याद् बा॑र्.हस्प॒त्यः । \newline
13. बा॒र्॒.ह॒स्प॒त्य श्च॒रु श्च॒रुर् बा॑र्.हस्प॒त्यो बा॑र्.हस्प॒त्य श्च॒रुः । \newline
14. च॒रुर् यद् यच् च॒रु श्च॒रुर् यत् । \newline
15. यदा᳚ग्नावैष्ण॒व आ᳚ग्नावैष्ण॒वो यद् यदा᳚ग्नावैष्ण॒वः । \newline
16. आ॒ग्ना॒वै॒ष्ण॒व एका॑दशकपाल॒ एका॑दशकपाल आग्नावैष्ण॒व आ᳚ग्नावैष्ण॒व एका॑दशकपालः । \newline
17. आ॒ग्ना॒वै॒ष्ण॒व इत्या᳚ग्ना - वै॒ष्ण॒वः । \newline
18. एका॑दशकपालो॒ भव॑ति॒ भव॒त्येका॑दशकपाल॒ एका॑दशकपालो॒ भव॑ति । \newline
19. एका॑दशकपाल॒ इत्येका॑दश - क॒पा॒लः॒ । \newline
20. भव॑ त्य॒ग्नि र॒ग्निर् भव॑ति॒ भव॑ त्य॒ग्निः । \newline
21. अ॒ग्निः सर्वाः॒ सर्वा॑ अ॒ग्नि र॒ग्निः सर्वाः᳚ । \newline
22. सर्वा॑ दे॒वता॑ दे॒वताः॒ सर्वाः॒ सर्वा॑ दे॒वताः᳚ । \newline
23. दे॒वता॒ विष्णु॒र् विष्णु॑र् दे॒वता॑ दे॒वता॒ विष्णुः॑ । \newline
24. विष्णु॑र् य॒ज्ञो य॒ज्ञो विष्णु॒र् विष्णु॑र् य॒ज्ञ्ः । \newline
25. य॒ज्ञो दे॒वता॑भिर् दे॒वता॑भिर् य॒ज्ञो य॒ज्ञो दे॒वता॑भिः । \newline
26. दे॒वता॑भिश्च च दे॒वता॑भिर् दे॒वता॑भिश्च । \newline
27. चै॒वैव च॑ चै॒व । \newline
28. ए॒वैन॑ मेन मे॒वैवैन᳚म् । \newline
29. ए॒नं॒ ॅय॒ज्ञेन॑ य॒ज्ञेनै॑न मेनं ॅय॒ज्ञेन॑ । \newline
30. य॒ज्ञेन॑ च च य॒ज्ञेन॑ य॒ज्ञेन॑ च । \newline
31. चा॒भ्य॑भि च॑ चा॒भि । \newline
32. अ॒भि च॑रति चर त्य॒भ्य॑भि च॑रति । \newline
33. च॒र॒ति॒ सर॑स्वती॒ सर॑स्वती चरति चरति॒ सर॑स्वती । \newline
34. सर॑स्व॒ त्याज्य॑भा॒गा ऽऽज्य॑भागा॒ सर॑स्वती॒ सर॑स्व॒ त्याज्य॑भागा । \newline
35. आज्य॑भागा भवति भव॒ त्याज्य॑भा॒गा ऽऽज्य॑भागा भवति । \newline
36. आज्य॑भा॒गेत्याज्य॑ - भा॒गा॒ । \newline
37. भ॒व॒ति॒ वाग् वाग् भ॑वति भवति॒ वाक् । \newline
38. वाग् वै वै वाग् वाग् वै । \newline
39. वै सर॑स्वती॒ सर॑स्वती॒ वै वै सर॑स्वती । \newline
40. सर॑स्वती वा॒चा वा॒चा सर॑स्वती॒ सर॑स्वती वा॒चा । \newline
41. वा॒चैवैव वा॒चा वा॒चैव । \newline
42. ए॒वैन॑ मेन मे॒वैवैन᳚म् । \newline
43. ए॒न॒ म॒भ्या᳚(1॒)भ्ये॑न मेन म॒भि । \newline
44. अ॒भि च॑रति चर त्य॒भ्य॑भि च॑रति । \newline
45. च॒र॒ति॒ बा॒र्॒.ह॒स्प॒त्यो बा॑र्.हस्प॒त्य श्च॑रति चरति बार्.हस्प॒त्यः । \newline
46. बा॒र्॒.ह॒स्प॒त्य श्च॒रु श्च॒रुर् बा॑र्.हस्प॒त्यो बा॑र्.हस्प॒त्य श्च॒रुः । \newline
47. च॒रुर् भ॑वति भवति च॒रु श्च॒रुर् भ॑वति । \newline
48. भ॒व॒ति॒ ब्रह्म॒ ब्रह्म॑ भवति भवति॒ ब्रह्म॑ । \newline
49. ब्रह्म॒ वै वै ब्रह्म॒ ब्रह्म॒ वै । \newline
50. वै दे॒वाना᳚म् दे॒वानां॒ ॅवै वै दे॒वाना᳚म् । \newline
51. दे॒वाना॒म् बृह॒स्पति॒र् बृह॒स्पति॑र् दे॒वाना᳚म् दे॒वाना॒म् बृह॒स्पतिः॑ । \newline
52. बृह॒स्पति॒र् ब्रह्म॑णा॒ ब्रह्म॑णा॒ बृह॒स्पति॒र् बृह॒स्पति॒र् ब्रह्म॑णा । \newline
53. ब्रह्म॑ णै॒वैव ब्रह्म॑णा॒ ब्रह्म॑ णै॒व । \newline
54. ए॒वैन॑ मेन मे॒वैवैन᳚म् । \newline
55. ए॒न॒ म॒भ्या᳚(1॒)भ्ये॑न मेन म॒भि । \newline
56. अ॒भि च॑रति चर त्य॒भ्य॑भि च॑रति । \newline
57. च॒र॒ति॒ प्रति॒ प्रति॑ चरति चरति॒ प्रति॑ । \newline

\textbf{Ghana Paata } \newline

1. आ॒ग्ना॒वै॒ष्ण॒व मेका॑दशकपाल॒ मेका॑दशकपाल माग्नावैष्ण॒व मा᳚ग्नावैष्ण॒व मेका॑दशकपाल॒म् निर् णिरेका॑दशकपाल माग्नावैष्ण॒व मा᳚ग्नावैष्ण॒व मेका॑दशकपाल॒म् निः । \newline
2. आ॒ग्ना॒वै॒ष्ण॒वमित्या᳚ग्ना - वै॒ष्ण॒वम् । \newline
3. एका॑दशकपाल॒म् निर् णिरेका॑दशकपाल॒ मेका॑दशकपाल॒म् निर् व॑पेद् वपे॒न् निरेका॑दशकपाल॒ मेका॑दशकपाल॒म् निर् व॑पेत् । \newline
4. एका॑दशकपाल॒मित्येका॑दश - क॒पा॒ल॒म् । \newline
5. निर् व॑पेद् वपे॒न् निर् णिर् व॑पे दभि॒चर॑न् नभि॒चरन्॑. वपे॒न् निर् णिर् व॑पे दभि॒चरन्न्॑ । \newline
6. व॒पे॒ द॒भि॒चर॑न् नभि॒चरन्॑. वपेद् वपे दभि॒चर॒न् थ्सर॑स्वती॒ सर॑स्व त्यभि॒चरन्॑. वपेद् वपे दभि॒चर॒न् थ्सर॑स्वती । \newline
7. अ॒भि॒चर॒न् थ्सर॑स्वती॒ सर॑स्व त्यभि॒चर॑न् नभि॒चर॒न् थ्सर॑स्व॒ त्याज्य॑भा॒गा ऽऽज्य॑भागा॒ सर॑स्व त्यभि॒चर॑न् नभि॒चर॒न् थ्सर॑स्व॒ त्याज्य॑भागा । \newline
8. अ॒भि॒चर॒न्नित्य॑भि - चरन्न्॑ । \newline
9. सर॑स्व॒ त्याज्य॑भा॒गा ऽऽज्य॑भागा॒ सर॑स्वती॒ सर॑स्व॒ त्याज्य॑भागा॒ स्याथ् स्यादाज्य॑भागा॒ सर॑स्वती॒ सर॑स्व॒ त्याज्य॑भागा॒ स्यात् । \newline
10. आज्य॑भागा॒ स्याथ् स्यादाज्य॑भा॒गा ऽऽज्य॑भागा॒ स्याद् बा॑र्.हस्प॒त्यो बा॑र्.हस्प॒त्यः स्यादाज्य॑भा॒गा ऽऽज्य॑भागा॒ स्याद् बा॑र्.हस्प॒त्यः । \newline
11. आज्य॑भा॒गेत्याज्य॑ - भा॒गा॒ । \newline
12. स्याद् बा॑र्.हस्प॒त्यो बा॑र्.हस्प॒त्यः स्याथ् स्याद् बा॑र्.हस्प॒त्य श्च॒रु श्च॒रुर् बा॑र्.हस्प॒त्यः स्याथ् स्याद् बा॑र्.हस्प॒त्य श्च॒रुः । \newline
13. बा॒र्॒.ह॒स्प॒त्य श्च॒रु श्च॒रुर् बा॑र्.हस्प॒त्यो बा॑र्.हस्प॒त्य श्च॒रुर् यद् यच् च॒रुर् बा॑र्.हस्प॒त्यो बा॑र्.हस्प॒त्य श्च॒रुर् यत् । \newline
14. च॒रुर् यद् यच् च॒रु श्च॒रुर् यदा᳚ग्नावैष्ण॒व आ᳚ग्नावैष्ण॒वो यच् च॒रु श्च॒रुर् यदा᳚ग्नावैष्ण॒वः । \newline
15. यदा᳚ग्नावैष्ण॒व आ᳚ग्नावैष्ण॒वो यद् यदा᳚ग्नावैष्ण॒व एका॑दशकपाल॒ एका॑दशकपाल आग्नावैष्ण॒वो यद् यदा᳚ग्नावैष्ण॒व एका॑दशकपालः । \newline
16. आ॒ग्ना॒वै॒ष्ण॒व एका॑दशकपाल॒ एका॑दशकपाल आग्नावैष्ण॒व आ᳚ग्नावैष्ण॒व एका॑दशकपालो॒ भव॑ति॒ भव॒त्येका॑दशकपाल आग्नावैष्ण॒व आ᳚ग्नावैष्ण॒व एका॑दशकपालो॒ भव॑ति । \newline
17. आ॒ग्ना॒वै॒ष्ण॒व इत्या᳚ग्ना - वै॒ष्ण॒वः । \newline
18. एका॑दशकपालो॒ भव॑ति॒ भव॒त्येका॑दशकपाल॒ एका॑दशकपालो॒ भव॑ त्य॒ग्नि र॒ग्निर् भव॒त्येका॑दशकपाल॒ एका॑दशकपालो॒ भव॑ त्य॒ग्निः । \newline
19. एका॑दशकपाल॒ इत्येका॑दश - क॒पा॒लः॒ । \newline
20. भव॑ त्य॒ग्नि र॒ग्निर् भव॑ति॒ भव॑ त्य॒ग्निः सर्वाः॒ सर्वा॑ अ॒ग्निर् भव॑ति॒ भव॑ त्य॒ग्निः सर्वाः᳚ । \newline
21. अ॒ग्निः सर्वाः॒ सर्वा॑ अ॒ग्नि र॒ग्निः सर्वा॑ दे॒वता॑ दे॒वताः॒ सर्वा॑ अ॒ग्नि र॒ग्निः सर्वा॑ दे॒वताः᳚ । \newline
22. सर्वा॑ दे॒वता॑ दे॒वताः॒ सर्वाः॒ सर्वा॑ दे॒वता॒ विष्णु॒र् विष्णु॑र् दे॒वताः॒ सर्वाः॒ सर्वा॑ दे॒वता॒ विष्णुः॑ । \newline
23. दे॒वता॒ विष्णु॒र् विष्णु॑र् दे॒वता॑ दे॒वता॒ विष्णु॑र् य॒ज्ञो य॒ज्ञो विष्णु॑र् दे॒वता॑ दे॒वता॒ विष्णु॑र् य॒ज्ञ्ः । \newline
24. विष्णु॑र् य॒ज्ञो य॒ज्ञो विष्णु॒र् विष्णु॑र् य॒ज्ञो दे॒वता॑भिर् दे॒वता॑भिर् य॒ज्ञो विष्णु॒र् विष्णु॑र् य॒ज्ञो दे॒वता॑भिः । \newline
25. य॒ज्ञो दे॒वता॑भिर् दे॒वता॑भिर् य॒ज्ञो य॒ज्ञो दे॒वता॑भिश्च च दे॒वता॑भिर् य॒ज्ञो य॒ज्ञो दे॒वता॑भिश्च । \newline
26. दे॒वता॑भिश्च च दे॒वता॑भिर् दे॒वता॑भि श्चै॒वैव च॑ दे॒वता॑भिर् दे॒वता॑भि श्चै॒व । \newline
27. चै॒वैव च॑ चै॒वैन॑ मेन मे॒व च॑ चै॒वैन᳚म् । \newline
28. ए॒वैन॑ मेन मे॒वैवैनं॑ ॅय॒ज्ञेन॑ य॒ज्ञेनै॑न मे॒वैवैनं॑ ॅय॒ज्ञेन॑ । \newline
29. ए॒नं॒ ॅय॒ज्ञेन॑ य॒ज्ञेनै॑न मेनं ॅय॒ज्ञेन॑ च च य॒ज्ञेनै॑न मेनं ॅय॒ज्ञेन॑ च । \newline
30. य॒ज्ञेन॑ च च य॒ज्ञेन॑ य॒ज्ञेन॑ चा॒भ्य॑भि च॑ य॒ज्ञेन॑ य॒ज्ञेन॑ चा॒भि । \newline
31. चा॒भ्य॑भि च॑ चा॒भि च॑रति चर त्य॒भि च॑ चा॒भि च॑रति । \newline
32. अ॒भि च॑रति चर त्य॒भ्य॑भि च॑रति॒ सर॑स्वती॒ सर॑स्वती चर त्य॒भ्य॑भि च॑रति॒ सर॑स्वती । \newline
33. च॒र॒ति॒ सर॑स्वती॒ सर॑स्वती चरति चरति॒ सर॑स्व॒ त्याज्य॑भा॒गा ऽऽज्य॑भागा॒ सर॑स्वती चरति चरति॒ सर॑स्व॒ त्याज्य॑भागा । \newline
34. सर॑स्व॒ त्याज्य॑भा॒गा ऽऽज्य॑भागा॒ सर॑स्वती॒ सर॑स्व॒ त्याज्य॑भागा भवति भव॒ त्याज्य॑भागा॒ सर॑स्वती॒ सर॑स्व॒ त्याज्य॑भागा भवति । \newline
35. आज्य॑भागा भवति भव॒ त्याज्य॑भा॒गा ऽऽज्य॑भागा भवति॒ वाग् वाग् भ॑व॒ त्याज्य॑भा॒गा ऽऽज्य॑भागा भवति॒ वाक् । \newline
36. आज्य॑भा॒गेत्याज्य॑ - भा॒गा॒ । \newline
37. भ॒व॒ति॒ वाग् वाग् भ॑वति भवति॒ वाग् वै वै वाग् भ॑वति भवति॒ वाग् वै । \newline
38. वाग् वै वै वाग् वाग् वै सर॑स्वती॒ सर॑स्वती॒ वै वाग् वाग् वै सर॑स्वती । \newline
39. वै सर॑स्वती॒ सर॑स्वती॒ वै वै सर॑स्वती वा॒चा वा॒चा सर॑स्वती॒ वै वै सर॑स्वती वा॒चा । \newline
40. सर॑स्वती वा॒चा वा॒चा सर॑स्वती॒ सर॑स्वती वा॒चैवैव वा॒चा सर॑स्वती॒ सर॑स्वती वा॒चैव । \newline
41. वा॒चैवैव वा॒चा वा॒चैवैन॑ मेन मे॒व वा॒चा वा॒चैवैन᳚म् । \newline
42. ए॒वैन॑ मेन मे॒वैवैन॑ म॒भ्या᳚(1॒)भ्ये॑न मे॒वैवैन॑ म॒भि । \newline
43. ए॒न॒ म॒भ्या᳚(1॒)भ्ये॑न मेन म॒भि च॑रति चरत्य॒भ्ये॑न मेन म॒भि च॑रति । \newline
44. अ॒भि च॑रति चर त्य॒भ्य॑भि च॑रति बार्.हस्प॒त्यो बा॑र्.हस्प॒त्य श्च॑र त्य॒भ्य॑भि च॑रति बार्.हस्प॒त्यः । \newline
45. च॒र॒ति॒ बा॒र्॒.ह॒स्प॒त्यो बा॑र्.हस्प॒त्य श्च॑रति चरति बार्.हस्प॒त्य श्च॒रु श्च॒रुर् बा॑र्.हस्प॒त्य श्च॑रति चरति बार्.हस्प॒त्य श्च॒रुः । \newline
46. बा॒र्॒.ह॒स्प॒त्य श्च॒रु श्च॒रुर् बा॑र्.हस्प॒त्यो बा॑र्.हस्प॒त्य श्च॒रुर् भ॑वति भवति च॒रुर् बा॑र्.हस्प॒त्यो बा॑र्.हस्प॒त्य श्च॒रुर् भ॑वति । \newline
47. च॒रुर् भ॑वति भवति च॒रु श्च॒रुर् भ॑वति॒ ब्रह्म॒ ब्रह्म॑ भवति च॒रु श्च॒रुर् भ॑वति॒ ब्रह्म॑ । \newline
48. भ॒व॒ति॒ ब्रह्म॒ ब्रह्म॑ भवति भवति॒ ब्रह्म॒ वै वै ब्रह्म॑ भवति भवति॒ ब्रह्म॒ वै । \newline
49. ब्रह्म॒ वै वै ब्रह्म॒ ब्रह्म॒ वै दे॒वाना᳚म् दे॒वानां॒ ॅवै ब्रह्म॒ ब्रह्म॒ वै दे॒वाना᳚म् । \newline
50. वै दे॒वाना᳚म् दे॒वानां॒ ॅवै वै दे॒वाना॒म् बृह॒स्पति॒र् बृह॒स्पति॑र् दे॒वानां॒ ॅवै वै दे॒वाना॒म् बृह॒स्पतिः॑ । \newline
51. दे॒वाना॒म् बृह॒स्पति॒र् बृह॒स्पति॑र् दे॒वाना᳚म् दे॒वाना॒म् बृह॒स्पति॒र् ब्रह्म॑णा॒ ब्रह्म॑णा॒ बृह॒स्पति॑र् दे॒वाना᳚म् दे॒वाना॒म् बृह॒स्पति॒र् ब्रह्म॑णा । \newline
52. बृह॒स्पति॒र् ब्रह्म॑णा॒ ब्रह्म॑णा॒ बृह॒स्पति॒र् बृह॒स्पति॒र् ब्रह्म॑णै॒वैव ब्रह्म॑णा॒ बृह॒स्पति॒र् बृह॒स्पति॒र् ब्रह्म॑णै॒व । \newline
53. ब्रह्म॑णै॒वैव ब्रह्म॑णा॒ ब्रह्म॑णै॒वैन॑ मेन मे॒व ब्रह्म॑णा॒ ब्रह्म॑णै॒वैन᳚म् । \newline
54. ए॒वैन॑ मेन मे॒वैवैन॑ म॒भ्या᳚(1॒)भ्ये॑न मे॒वैवैन॑ म॒भि । \newline
55. ए॒न॒ म॒भ्या᳚(1॒)भ्ये॑न मेन म॒भि च॑रति चर त्य॒भ्ये॑न मेन म॒भि च॑रति । \newline
56. अ॒भि च॑रति चर त्य॒भ्य॑भि च॑रति॒ प्रति॒ प्रति॑ चर त्य॒भ्य॑भि च॑रति॒ प्रति॑ । \newline
57. च॒र॒ति॒ प्रति॒ प्रति॑ चरति चरति॒ प्रति॒ वै वै प्रति॑ चरति चरति॒ प्रति॒ वै । \newline
\pagebreak
\markright{ TS 2.2.9.2  \hfill https://www.vedavms.in \hfill}
\addcontentsline{toc}{section}{ TS 2.2.9.2 }
\section*{ TS 2.2.9.2 }

\textbf{TS 2.2.9.2 } \newline
\textbf{Samhita Paata} \newline

प्रति॒ वै प॒रस्ता॑दभि॒चर॑न्तम॒भि च॑रन्ति॒ द्वेद्वे॑ पुरोऽनुवा॒क्ये॑ कुर्या॒दति॒ प्रयु॑क्त्या ए॒तयै॒व य॑जेताभि च॒र्यमा॑णो दे॒वता॑भिरे॒व दे॒वताः᳚ प्रति॒चर॑ति य॒ज्ञेन॑ य॒ज्ञ्ं ॅवा॒चा वाचं॒ ब्रह्म॑णा॒ ब्रह्म॒ स दे॒वता᳚श्चै॒व य॒ज्ञ्ं च॑ मद्ध्य॒तो व्यव॑सर्पति॒ तस्य॒ न कुत॑श्च॒नोपा᳚व्या॒धो भ॑वति॒ नैन॑मभि॒चरन्᳚-थ्स्तृणुत आग्नावैष्ण॒व-मेका॑दशकपालं॒ निर्व॑पे॒द्यं ॅय॒ज्ञो नो - [  ] \newline

\textbf{Pada Paata} \newline

प्रतीति॑ । वै । प॒रस्ता᳚त् । अ॒भि॒चर॑न्त॒मित्य॑भि - चर॑न्तम् । अ॒भीति॑ । च॒र॒न्ति॒ । द्वेद्वे॒ इति॒ द्वे - द्वे॒ । पु॒रो॒नु॒वा॒क्य॑ इति॑ पुरः - अ॒नु॒वा॒क्ये᳚ । कु॒र्या॒त् । अतीति॑ । प्रयु॑क्त्या॒ इति॒ प्र - यु॒क्त्यै॒ । ए॒तया᳚ । ए॒व । य॒जे॒त॒ । अ॒भि॒च॒र्यमा॑ण॒ इत्य॑भि - च॒र्यमा॑णः । दे॒वता॑भिः । ए॒व । दे॒वताः᳚ । प्र॒ति॒चर॒तीति॑ प्रति - चर॑ति । य॒ज्ञेन॑ । य॒ज्ञ्म् । वा॒चा । वाच᳚म् । ब्रह्म॑णा । ब्रह्म॑ । सः । दे॒वताः᳚ । च॒ । ए॒व । य॒ज्ञ्म् । च॒ । म॒द्ध्य॒तः । व्यव॑सर्प॒तीति॑ वि - अव॑सर्पति । तस्य॑ । न । कुतः॑ । च॒न । उ॒पा॒व्या॒ध इत्यु॑प - आ॒व्या॒धः । भ॒व॒ति॒ । न । ए॒न॒म् । अ॒भि॒चर॒न्नित्य॑भि - चरन्न्॑ । स्तृ॒णु॒ते॒ । आ॒ग्ना॒वै॒ष्ण॒वमित्या᳚ग्ना - वै॒ष्ण॒वम् । एका॑दशकपाल॒मित्येका॑दश - क॒पा॒ल॒म् । निरिति॑ । व॒पे॒त् । यम् । य॒ज्ञ्ः । न ।  \newline


\textbf{Krama Paata} \newline

प्रति॒ वै । वै प॒रस्ता᳚त् । प॒रस्ता॑दभि॒चर॑न्तम् । अ॒भि॒चर॑न्तम॒भि । अ॒भि॒चर॑न्त॒मित्य॑भि - चर॑न्तम् । अ॒भि च॑रन्ति । च॒र॒न्ति॒ द्वेद्वे᳚ । द्वेद्वे॑ पुरोनुवा॒क्ये᳚ । द्वेद्वे॒ इति॒ द्वे - द्वे॒ । पु॒रो॒नु॒वा॒क्ये॑ कुर्यात् । पु॒रो॒नु॒वा॒क्ये॑ इति॑ पुरः - अ॒नु॒वा॒क्ये᳚ । कु॒र्या॒दति॑ । अति॒ प्रयु॑क्त्यै । प्रयु॑क्त्या ए॒तया᳚ । प्रयु॑क्त्या॒ इति॒ प्र - यु॒क्त्यै॒ । ए॒तयै॒व । ए॒व य॑जेत । य॒जे॒ता॒भि॒च॒र्यमा॑णः । अ॒भि॒च॒र्यमा॑णो दे॒वता॑भिः । अ॒बि॒च॒र्यमा॑ण॒ इत्य॑भि - च॒र्यमा॑णः । दे॒वता॑भिरे॒व । ए॒व दे॒वताः᳚ । दे॒वताः᳚ प्रति॒चर॑ति । प्र॒ति॒चर॑ति य॒ज्ञेन॑ । प्र॒ति॒चर॒तीति॑ प्रति - चर॑ति । य॒ज्ञेन॑ य॒ज्ञ्म् । य॒ज्ञ्ं ॅवा॒चा । वा॒चा वाच᳚म् । वाच॒म् ब्रह्म॑णा । ब्रह्म॑णा॒ ब्रह्म॑ । ब्रह्म॒ सः । स दे॒वताः᳚ । दे॒वता᳚श्च । चै॒व । ए॒व य॒ज्ञ्म् । य॒ज्ञ्म् च॑ । च॒ म॒द्ध्य॒तः । म॒द्ध्य॒तो व्यव॑सर्पति । व्यव॑सर्पति॒ तस्य॑ । व्यव॑सर्प॒तीति॑ वि - अव॑सर्पति । तस्य॒ न । न कुतः॑ । कुत॑ श्च॒न । च॒नोपा᳚व्या॒धः । उ॒पा॒व्या॒धो भ॑वति । उ॒पा॒व्या॒ध इत्यु॑प - आ॒व्या॒धः । भ॒व॒ति॒ न । नैन᳚म् । ए॒न॒म॒भि॒चरन्न्॑ । अ॒भि॒चर᳚न्थ् स्तृणुते । अ॒भि॒चर॒न्नित्य॑भि - चरन्न्॑ । स्तृ॒णु॒त॒ आ॒ग्ना॒वै॒ष्ण॒वम् । आ॒ग्ना॒वै॒ष्ण॒वमेका॑दशकपालम् । आ॒ग्ना॒वै॒ष्ण॒वमित्या᳚ग्ना - वै॒ष्ण॒वम् । एका॑दशकपाल॒म् निः । एका॑दशकपाल॒मित्येका॑दश - क॒पा॒ल॒म् । निर् व॑पेत् । व॒पे॒द् यम् । यं ॅय॒ज्ञ्ः । य॒ज्ञो न । 
नोप॒नमे᳚त् \newline

\textbf{Jatai Paata} \newline

1. प्रति॒ वै वै प्रति॒ प्रति॒ वै । \newline
2. वै प॒रस्ता᳚त् प॒रस्ता॒द् वै वै प॒रस्ता᳚त् । \newline
3. प॒रस्ता॑ दभि॒चर॑न्त मभि॒चर॑न्तम् प॒रस्ता᳚त् प॒रस्ता॑ दभि॒चर॑न्तम् । \newline
4. अ॒भि॒चर॑न्त म॒भ्या᳚(1॒)भ्य॑भि॒चर॑न्त मभि॒चर॑न्त म॒भि । \newline
5. अ॒भि॒चर॑न्त॒मित्य॑भि - चर॑न्तम् । \newline
6. अ॒भि च॑रन्ति चर न्त्य॒भ्य॑भि च॑रन्ति । \newline
7. च॒र॒न्ति॒ द्वेद्वे॒ द्वेद्वे॑ चरन्ति चरन्ति॒ द्वेद्वे᳚ । \newline
8. द्वेद्वे॑ पुरोनुवा॒क्ये॑ पुरोनुवा॒क्ये᳚ द्वेद्वे॒ द्वेद्वे॑ पुरोनुवा॒क्ये᳚ । \newline
9. द्वेद्वे॒ इति॒ द्वे - द्वे॒ । \newline
10. पु॒रो॒नु॒वा॒क्ये॑ कुर्यात् कुर्यात् पुरोनुवा॒क्ये॑ पुरोनुवा॒क्ये॑ कुर्यात् । \newline
11. पु॒रो॒नु॒वा॒क्ये॑ इति॑ पुरः - अ॒नु॒वा॒क्ये᳚ । \newline
12. कु॒र्या॒ दत्यति॑ कुर्यात् कुर्या॒ दति॑ । \newline
13. अति॒ प्रयु॑क्त्यै॒ प्रयु॑क्त्या॒ अत्यति॒ प्रयु॑क्त्यै । \newline
14. प्रयु॑क्त्या ए॒तयै॒तया॒ प्रयु॑क्त्यै॒ प्रयु॑क्त्या ए॒तया᳚ । \newline
15. प्रयु॑क्त्या॒ इति॒ प्र - यु॒क्त्यै॒ । \newline
16. ए॒त यै॒वैवै तयै॒त यै॒व । \newline
17. ए॒व य॑जेत यजेतै॒वैव य॑जेत । \newline
18. य॒जे॒ता॒ भि॒च॒र्यमा॑णो ऽभिच॒र्यमा॑णो यजेत यजेता भिच॒र्यमा॑णः । \newline
19. अ॒भि॒च॒र्यमा॑णो दे॒वता॑भिर् दे॒वता॑भि रभिच॒र्यमा॑णो ऽभिच॒र्यमा॑णो दे॒वता॑भिः । \newline
20. अ॒भि॒च॒र्यमा॑ण॒ इत्य॑भि - च॒र्यमा॑णः । \newline
21. दे॒वता॑भि रे॒वैव दे॒वता॑भिर् दे॒वता॑भि रे॒व । \newline
22. ए॒व दे॒वता॑ दे॒वता॑ ए॒वैव दे॒वताः᳚ । \newline
23. दे॒वताः᳚ प्रति॒चर॑ति प्रति॒चर॑ति दे॒वता॑ दे॒वताः᳚ प्रति॒चर॑ति । \newline
24. प्र॒ति॒चर॑ति य॒ज्ञेन॑ य॒ज्ञेन॑ प्रति॒चर॑ति प्रति॒चर॑ति य॒ज्ञेन॑ । \newline
25. प्र॒ति॒चर॒तीति॑ प्रति - चर॑ति । \newline
26. य॒ज्ञेन॑ य॒ज्ञ्ं ॅय॒ज्ञ्ं ॅय॒ज्ञेन॑ य॒ज्ञेन॑ य॒ज्ञ्म् । \newline
27. य॒ज्ञ्ं ॅवा॒चा वा॒चा य॒ज्ञ्ं ॅय॒ज्ञ्ं ॅवा॒चा । \newline
28. वा॒चा वाचं॒ ॅवाचं॑ ॅवा॒चा वा॒चा वाच᳚म् । \newline
29. वाच॒म् ब्रह्म॑णा॒ ब्रह्म॑णा॒ वाचं॒ ॅवाच॒म् ब्रह्म॑णा । \newline
30. ब्रह्म॑णा॒ ब्रह्म॒ ब्रह्म॒ ब्रह्म॑णा॒ ब्रह्म॑णा॒ ब्रह्म॑ । \newline
31. ब्रह्म॒ स स ब्रह्म॒ ब्रह्म॒ सः । \newline
32. स दे॒वता॑ दे॒वताः॒ स स दे॒वताः᳚ । \newline
33. दे॒वता᳚श्च च दे॒वता॑ दे॒वता᳚श्च । \newline
34. चै॒वैव च॑ चै॒व । \newline
35. ए॒व य॒ज्ञ्ं ॅय॒ज्ञ् मे॒वैव य॒ज्ञ्म् । \newline
36. य॒ज्ञ्म् च॑ च य॒ज्ञ्ं ॅय॒ज्ञ्म् च॑ । \newline
37. च॒ म॒द्ध्य॒तो म॑द्ध्य॒तश्च॑ च मद्ध्य॒तः । \newline
38. म॒द्ध्य॒तो व्यव॑सर्पति॒ व्यव॑सर्पति मद्ध्य॒तो म॑द्ध्य॒तो व्यव॑सर्पति । \newline
39. व्यव॑सर्पति॒ तस्य॒ तस्य॒ व्यव॑सर्पति॒ व्यव॑सर्पति॒ तस्य॑ । \newline
40. व्यव॑सर्प॒तीति॑ वि - अव॑सर्पति । \newline
41. तस्य॒ न न तस्य॒ तस्य॒ न । \newline
42. न कुतः॒ कुतो॒ न न कुतः॑ । \newline
43. कुत॑श्च॒न च॒न कुतः॒ कुत॑श्च॒न । \newline
44. च॒नो पा᳚व्या॒ध उ॑पाव्या॒धश्च॒न च॒नो पा᳚व्या॒धः । \newline
45. उ॒पा॒व्या॒धो भ॑वति भव त्युपाव्या॒ध उ॑पाव्या॒धो भ॑वति । \newline
46. उ॒पा॒व्या॒ध इत्यु॑प - आ॒व्या॒धः । \newline
47. भ॒व॒ति॒ न न भ॑वति भवति॒ न । \newline
48. नैन॑ मेन॒म् न नैन᳚म् । \newline
49. ए॒न॒ म॒भि॒चर॑न् नभि॒चर॑न् नेन मेन मभि॒चरन्न्॑ । \newline
50. अ॒भि॒चर᳚न् थ्स्तृणुते स्तृणुते ऽभि॒चर॑न् नभि॒चर᳚न् थ्स्तृणुते । \newline
51. अ॒भि॒चर॒न्नित्य॑भि - चरन्न्॑ । \newline
52. स्तृ॒णु॒त॒ आ॒ग्ना॒वै॒ष्ण॒व मा᳚ग्नावैष्ण॒वꣳ स्तृ॑णुते स्तृणुत आग्नावैष्ण॒वम् । \newline
53. आ॒ग्ना॒वै॒ष्ण॒व मेका॑दशकपाल॒ मेका॑दशकपाल माग्नावैष्ण॒व मा᳚ग्नावैष्ण॒व मेका॑दशकपालम् । \newline
54. आ॒ग्ना॒वै॒ष्ण॒वमित्या᳚ग्ना - वै॒ष्ण॒वम् । \newline
55. एका॑दशकपाल॒म् निर् णिरेका॑दशकपाल॒ मेका॑दशकपाल॒म् निः । \newline
56. एका॑दशकपाल॒मित्येका॑दश - क॒पा॒ल॒म् । \newline
57. निर् व॑पेद् वपे॒न् निर् णिर् व॑पेत् । \newline
58. व॒पे॒द् यं ॅयं ॅव॑पेद् वपे॒द् यम् । \newline
59. यं ॅय॒ज्ञो य॒ज्ञो यं ॅयं ॅय॒ज्ञ्ः । \newline
60. य॒ज्ञो न न य॒ज्ञो य॒ज्ञो न । \newline
61. नोप॒नमे॑ दुप॒नमे॒न् न नोप॒नमे᳚त् । \newline

\textbf{Ghana Paata } \newline

1. प्रति॒ वै वै प्रति॒ प्रति॒ वै प॒रस्ता᳚त् प॒रस्ता॒द् वै प्रति॒ प्रति॒ वै प॒रस्ता᳚त् । \newline
2. वै प॒रस्ता᳚त् प॒रस्ता॒द् वै वै प॒रस्ता॑ दभि॒चर॑न्त मभि॒चर॑न्तम् प॒रस्ता॒द् वै वै प॒रस्ता॑ दभि॒चर॑न्तम् । \newline
3. प॒रस्ता॑ दभि॒चर॑न्त मभि॒चर॑न्तम् प॒रस्ता᳚त् प॒रस्ता॑ दभि॒चर॑न्त म॒भ्या᳚(1॒)भ्य॑भि॒चर॑न्तम् प॒रस्ता᳚त् प॒रस्ता॑ दभि॒चर॑न्त म॒भि । \newline
4. अ॒भि॒चर॑न्त म॒भ्या᳚(1॒)भ्य॑भि॒चर॑न्त मभि॒चर॑न्त म॒भि च॑रन्ति चर न्त्य॒भ्य॑भि॒चर॑न्त मभि॒चर॑न्त म॒भि च॑रन्ति । \newline
5. अ॒भि॒चर॑न्त॒मित्य॑भि - चर॑न्तम् । \newline
6. अ॒भि च॑रन्ति चर न्त्य॒भ्य॑भि च॑रन्ति॒ द्वेद्वे॒ द्वेद्वे॑ चर न्त्य॒भ्य॑भि च॑रन्ति॒ द्वेद्वे᳚ । \newline
7. च॒र॒न्ति॒ द्वेद्वे॒ द्वेद्वे॑ चरन्ति चरन्ति॒ द्वेद्वे॑ पुरोनुवा॒क्ये॑ पुरोनुवा॒क्ये᳚ द्वेद्वे॑ चरन्ति चरन्ति॒ द्वेद्वे॑ पुरोनुवा॒क्ये᳚ । \newline
8. द्वेद्वे॑ पुरोनुवा॒क्ये॑ पुरोनुवा॒क्ये᳚ द्वेद्वे॒ द्वेद्वे॑ पुरोनुवा॒क्ये॑ कुर्यात् कुर्यात् पुरोनुवा॒क्ये᳚ द्वेद्वे॒ द्वेद्वे॑ पुरोनुवा॒क्ये॑ कुर्यात् । \newline
9. द्वेद्वे॒ इति॒ द्वे - द्वे॒ । \newline
10. पु॒रो॒नु॒वा॒क्ये॑ कुर्यात् कुर्यात् पुरोनुवा॒क्ये॑ पुरोनुवा॒क्ये॑ कुर्या॒दत्यति॑ कुर्यात् पुरोनुवा॒क्ये॑ पुरोनुवा॒क्ये॑ कुर्या॒दति॑ । \newline
11. पु॒रो॒नु॒वा॒क्ये॑ इति॑ पुरः - अ॒नु॒वा॒क्ये᳚ । \newline
12. कु॒र्या॒दत्यति॑ कुर्यात् कुर्या॒दति॒ प्रयु॑क्त्यै॒ प्रयु॑क्त्या॒ अति॑ कुर्यात् कुर्या॒दति॒ प्रयु॑क्त्यै । \newline
13. अति॒ प्रयु॑क्त्यै॒ प्रयु॑क्त्या॒ अत्यति॒ प्रयु॑क्त्या ए॒तयै॒तया॒ प्रयु॑क्त्या॒ अत्यति॒ प्रयु॑क्त्या ए॒तया᳚ । \newline
14. प्रयु॑क्त्या ए॒त यै॒तया॒ प्रयु॑क्त्यै॒ प्रयु॑क्त्या ए॒त यै॒वैवैतया॒ प्रयु॑क्त्यै॒ प्रयु॑क्त्या ए॒तयै॒व । \newline
15. प्रयु॑क्त्या॒ इति॒ प्र - यु॒क्त्यै॒ । \newline
16. ए॒त यै॒वै वैत यै॒तयै॒व य॑जेत यजेतै॒वैत यै॒तयै॒व य॑जेत । \newline
17. ए॒व य॑जेत यजेतै॒वैव य॑जेता भिच॒र्यमा॑णो ऽभिच॒र्यमा॑णो यजेतै॒वैव य॑जेता भिच॒र्यमा॑णः । \newline
18. य॒जे॒ता॒ भि॒च॒र्यमा॑णो ऽभिच॒र्यमा॑णो यजेत यजेता भिच॒र्यमा॑णो दे॒वता॑भिर् दे॒वता॑भि रभिच॒र्यमा॑णो यजेत यजेता भिच॒र्यमा॑णो दे॒वता॑भिः । \newline
19. अ॒भि॒च॒र्यमा॑णो दे॒वता॑भिर् दे॒वता॑भि रभिच॒र्यमा॑णो ऽभिच॒र्यमा॑णो दे॒वता॑भि रे॒वैव दे॒वता॑भि रभिच॒र्यमा॑णो ऽभिच॒र्यमा॑णो दे॒वता॑भि रे॒व । \newline
20. अ॒भि॒च॒र्यमा॑ण॒ इत्य॑भि - च॒र्यमा॑णः । \newline
21. दे॒वता॑भि रे॒वैव दे॒वता॑भिर् दे॒वता॑भि रे॒व दे॒वता॑ दे॒वता॑ ए॒व दे॒वता॑भिर् दे॒वता॑भि रे॒व दे॒वताः᳚ । \newline
22. ए॒व दे॒वता॑ दे॒वता॑ ए॒वैव दे॒वताः᳚ प्रति॒चर॑ति प्रति॒चर॑ति दे॒वता॑ ए॒वैव दे॒वताः᳚ प्रति॒चर॑ति । \newline
23. दे॒वताः᳚ प्रति॒चर॑ति प्रति॒चर॑ति दे॒वता॑ दे॒वताः᳚ प्रति॒चर॑ति य॒ज्ञेन॑ य॒ज्ञेन॑ प्रति॒चर॑ति दे॒वता॑ दे॒वताः᳚ प्रति॒चर॑ति य॒ज्ञेन॑ । \newline
24. प्र॒ति॒चर॑ति य॒ज्ञेन॑ य॒ज्ञेन॑ प्रति॒चर॑ति प्रति॒चर॑ति य॒ज्ञेन॑ य॒ज्ञ्ं ॅय॒ज्ञ्ं ॅय॒ज्ञेन॑ प्रति॒चर॑ति प्रति॒चर॑ति य॒ज्ञेन॑ य॒ज्ञ्म् । \newline
25. प्र॒ति॒चर॒तीति॑ प्रति - चर॑ति । \newline
26. य॒ज्ञेन॑ य॒ज्ञ्ं ॅय॒ज्ञ्ं ॅय॒ज्ञेन॑ य॒ज्ञेन॑ य॒ज्ञ्ं ॅवा॒चा वा॒चा य॒ज्ञ्ं ॅय॒ज्ञेन॑ य॒ज्ञेन॑ य॒ज्ञ्ं ॅवा॒चा । \newline
27. य॒ज्ञ्ं ॅवा॒चा वा॒चा य॒ज्ञ्ं ॅय॒ज्ञ्ं ॅवा॒चा वाचं॒ ॅवाचं॑ ॅवा॒चा य॒ज्ञ्ं ॅय॒ज्ञ्ं ॅवा॒चा वाच᳚म् । \newline
28. वा॒चा वाचं॒ ॅवाचं॑ ॅवा॒चा वा॒चा वाच॒म् ब्रह्म॑णा॒ ब्रह्म॑णा॒ वाचं॑ ॅवा॒चा वा॒चा वाच॒म् ब्रह्म॑णा । \newline
29. वाच॒म् ब्रह्म॑णा॒ ब्रह्म॑णा॒ वाचं॒ ॅवाच॒म् ब्रह्म॑णा॒ ब्रह्म॒ ब्रह्म॒ ब्रह्म॑णा॒ वाचं॒ ॅवाच॒म् ब्रह्म॑णा॒ ब्रह्म॑ । \newline
30. ब्रह्म॑णा॒ ब्रह्म॒ ब्रह्म॒ ब्रह्म॑णा॒ ब्रह्म॑णा॒ ब्रह्म॒ स स ब्रह्म॒ ब्रह्म॑णा॒ ब्रह्म॑णा॒ ब्रह्म॒ सः । \newline
31. ब्रह्म॒ स स ब्रह्म॒ ब्रह्म॒ स दे॒वता॑ दे॒वताः॒ स ब्रह्म॒ ब्रह्म॒ स दे॒वताः᳚ । \newline
32. स दे॒वता॑ दे॒वताः॒ स स दे॒वता᳚श्च च दे॒वताः॒ स स दे॒वता᳚श्च । \newline
33. दे॒वता᳚श्च च दे॒वता॑ दे॒वता᳚ श्चै॒वैव च॑ दे॒वता॑ दे॒वता᳚ श्चै॒व । \newline
34. चै॒वैव च॑ चै॒व य॒ज्ञ्ं ॅय॒ज्ञ् मे॒व च॑ चै॒व य॒ज्ञ्म् । \newline
35. ए॒व य॒ज्ञ्ं ॅय॒ज्ञ् मे॒वैव य॒ज्ञ्म् च॑ च य॒ज्ञ् मे॒वैव य॒ज्ञ्म् च॑ । \newline
36. य॒ज्ञ्म् च॑ च य॒ज्ञ्ं ॅय॒ज्ञ्म् च॑ मद्ध्य॒तो म॑द्ध्य॒तश्च॑ य॒ज्ञ्ं ॅय॒ज्ञ्म् च॑ मद्ध्य॒तः । \newline
37. च॒ म॒द्ध्य॒तो म॑द्ध्य॒तश्च॑ च मद्ध्य॒तो व्यव॑सर्पति॒ व्यव॑सर्पति मद्ध्य॒तश्च॑ च मद्ध्य॒तो व्यव॑सर्पति । \newline
38. म॒द्ध्य॒तो व्यव॑सर्पति॒ व्यव॑सर्पति मद्ध्य॒तो म॑द्ध्य॒तो व्यव॑सर्पति॒ तस्य॒ तस्य॒ व्यव॑सर्पति मद्ध्य॒तो म॑द्ध्य॒तो व्यव॑सर्पति॒ तस्य॑ । \newline
39. व्यव॑सर्पति॒ तस्य॒ तस्य॒ व्यव॑सर्पति॒ व्यव॑सर्पति॒ तस्य॒ न न तस्य॒ व्यव॑सर्पति॒ व्यव॑सर्पति॒ तस्य॒ न । \newline
40. व्यव॑सर्प॒तीति॑ वि - अव॑सर्पति । \newline
41. तस्य॒ न न तस्य॒ तस्य॒ न कुतः॒ कुतो॒ न तस्य॒ तस्य॒ न कुतः॑ । \newline
42. न कुतः॒ कुतो॒ न न कुत॑ श्च॒न च॒न कुतो॒ न न कुत॑ श्च॒न । \newline
43. कुत॑श्च॒न च॒न कुतः॒ कुत॑ श्च॒नोपा᳚व्या॒ध उ॑पाव्या॒ध श्च॒न कुतः॒ कुत॑ श्च॒नोपा᳚व्या॒धः । \newline
44. च॒नोपा᳚व्या॒ध उ॑पाव्या॒ध श्च॒न च॒नोपा᳚व्या॒धो भ॑वति भव त्युपाव्या॒ध श्च॒न च॒नोपा᳚व्या॒धो भ॑वति । \newline
45. उ॒पा॒व्या॒धो भ॑वति भव त्युपाव्या॒ध उ॑पाव्या॒धो भ॑वति॒ न न भ॑व त्युपाव्या॒ध उ॑पाव्या॒धो भ॑वति॒ न । \newline
46. उ॒पा॒व्या॒ध इत्यु॑प - आ॒व्या॒धः । \newline
47. भ॒व॒ति॒ न न भ॑वति भवति॒ नैन॑ मेन॒म् न भ॑वति भवति॒ नैन᳚म् । \newline
48. नैन॑ मेन॒म् न नैन॑ मभि॒चर॑न् नभि॒चर॑न् नेन॒म् न नैन॑ मभि॒चरन्न्॑ । \newline
49. ए॒न॒ म॒भि॒चर॑न् नभि॒चर॑न् नेन मेन मभि॒चर᳚न् थ्स्तृणुते स्तृणुते ऽभि॒चर॑न् नेन मेन मभि॒चर᳚न् थ्स्तृणुते । \newline
50. अ॒भि॒चर᳚न् थ्स्तृणुते स्तृणुते ऽभि॒चर॑न् नभि॒चर᳚न् थ्स्तृणुत आग्नावैष्ण॒व मा᳚ग्नावैष्ण॒वꣳ स्तृ॑णुते 
ऽभि॒चर॑न् नभि॒चर᳚न् थ्स्तृणुत आग्नावैष्ण॒वम् । \newline
51. अ॒भि॒चर॒न्नित्य॑भि - चरन्न्॑ । \newline
52. स्तृ॒णु॒त॒ आ॒ग्ना॒वै॒ष्ण॒व मा᳚ग्नावैष्ण॒वꣳ स्तृ॑णुते स्तृणुत आग्नावैष्ण॒व मेका॑दशकपाल॒ मेका॑दशकपाल माग्नावैष्ण॒वꣳ स्तृ॑णुते स्तृणुत आग्नावैष्ण॒व मेका॑दशकपालम् । \newline
53. आ॒ग्ना॒वै॒ष्ण॒व मेका॑दशकपाल॒ मेका॑दशकपाल माग्नावैष्ण॒व मा᳚ग्नावैष्ण॒व मेका॑दशकपाल॒म् निर् णिरेका॑दशकपाल माग्नावैष्ण॒व मा᳚ग्नावैष्ण॒व मेका॑दशकपाल॒म् निः । \newline
54. आ॒ग्ना॒वै॒ष्ण॒वमित्या᳚ग्ना - वै॒ष्ण॒वम् । \newline
55. एका॑दशकपाल॒म् निर् णिरेका॑दशकपाल॒ मेका॑दशकपाल॒म् निर् व॑पेद् वपे॒न् निरेका॑दशकपाल॒ मेका॑दशकपाल॒म् निर् व॑पेत् । \newline
56. एका॑दशकपाल॒मित्येका॑दश - क॒पा॒ल॒म् । \newline
57. निर् व॑पेद् वपे॒न् निर् णिर् व॑पे॒द् यं ॅयं ॅव॑पे॒न् निर् णिर् व॑पे॒द् यम् । \newline
58. व॒पे॒द् यं ॅयं ॅव॑पेद् वपे॒द् यं ॅय॒ज्ञो य॒ज्ञो यं ॅव॑पेद् वपे॒द् यं ॅय॒ज्ञ्ः । \newline
59. यं ॅय॒ज्ञो य॒ज्ञो यं ॅयं ॅय॒ज्ञो न न य॒ज्ञो यं ॅयं ॅय॒ज्ञो न । \newline
60. य॒ज्ञो न न य॒ज्ञो य॒ज्ञो नोप॒नमे॑ दुप॒नमे॒न् न य॒ज्ञो य॒ज्ञो नोप॒नमे᳚त् । \newline
61. नोप॒नमे॑ दुप॒नमे॒न् न नोप॒नमे॑ द॒ग्नि र॒ग्नि रु॑प॒नमे॒न् न नोप॒नमे॑ द॒ग्निः । \newline
\pagebreak
\markright{ TS 2.2.9.3  \hfill https://www.vedavms.in \hfill}
\addcontentsline{toc}{section}{ TS 2.2.9.3 }
\section*{ TS 2.2.9.3 }

\textbf{TS 2.2.9.3 } \newline
\textbf{Samhita Paata} \newline

-प॒नमे॑द॒ग्निः सर्वा॑ दे॒वता॒ विष्णु॑र्य॒ज्ञो᳚ऽग्निं चै॒व विष्णुं॑ च॒ स्वेन॑ भाग॒धेये॒नोप॑ धावति॒ तावे॒वास्मै॑ य॒ज्ञ्ं प्रय॑च्छत॒ उपै॑नं ॅय॒ज्ञो न॑मत्याग्ना- वैष्ण॒वं घृ॒ते च॒रुं निर्व॑पे॒च्चक्षु॑ष्कामो॒ऽग्नेर्वै चक्षु॑षा मनु॒ष्या॑ वि प॑श्यन्ति य॒ज्ञ्स्य॑ दे॒वा अ॒ग्निं चै॒व विष्णुं॑ च॒ स्वेन॑ भाग॒धेये॒नोप॑ धावति॒ तावे॒वा - [  ] \newline

\textbf{Pada Paata} \newline

उ॒प॒नमे॒दित्यु॑प - नमे᳚त् । अ॒ग्निः । सर्वाः᳚ । दे॒वताः᳚ । विष्णुः॑ । य॒ज्ञ्ः । अ॒ग्निम् । च॒ । ए॒व । विष्णु᳚म् । च॒ । स्वेन॑ । भा॒ग॒धेये॒नेति॑ भाग-धेये॑न । उपेति॑ । धा॒व॒ति॒ । तौ । ए॒व । अ॒स्मै॒ । य॒ज्ञ्म् । प्रेति॑ । य॒च्छ॒तः॒ । उपेति॑ । ए॒न॒म् । य॒ज्ञ्ः । न॒म॒ति॒ । आ॒ग्ना॒वै॒ष्ण॒वमित्या᳚ग्ना - वै॒ष्ण॒वम् । घृ॒ते । च॒रुम् । निरिति॑ । व॒पे॒त् । चक्षु॑ष्काम॒ इति॒ चक्षुः॑ - का॒मः॒ । अ॒ग्नेः । वै । चक्षु॑षा । म॒नु॒ष्याः᳚ । वीति॑ । प॒श्य॒न्ति॒ । य॒ज्ञ्स्य॑ । दे॒वाः । अ॒ग्निम् । च॒ । ए॒व । विष्णु᳚म् । च॒ । स्वेन॑ । भा॒ग॒धेये॒नेति॑ भाग - धेये॑न । उपेति॑ । धा॒व॒ति॒ । तौ । ए॒व ।  \newline


\textbf{Krama Paata} \newline

उ॒प॒नमे॑द॒ग्निः । उ॒प॒नमे॒दित्यु॑प - नमे᳚त् । अ॒ग्निः सर्वाः᳚ । सर्वा॑ दे॒वताः᳚ । दे॒वता॒ विष्णुः॑ । विष्णु॑र्,य॒ज्ञ्ः । य॒ज्ञो᳚ ऽग्निम् । अ॒ग्निम् च॑ । चै॒व । ए॒व विष्णु᳚म् । विष्णु॑म् च । च॒ स्वेन॑ । स्वेन॑ भाग॒धेये॑न । भा॒ग॒धेये॒नोप॑ । भा॒ग॒धेये॒नेति॑ भाग - धेये॑न । उप॑ धावति । धा॒व॒ति॒ तौ । तावे॒व । ए॒वास्मै᳚ । अ॒स्मै॒ य॒ज्ञ्म् । य॒ज्ञ्म् प्र । प्र य॑च्छतः । य॒च्छ॒त॒ उप॑ । उपै॑नम् । ए॒नं॒ ॅय॒ज्ञ्ः । य॒ज्ञो न॑मति । न॒म॒त्या॒ग्ना॒वै॒ष्ण॒वम् । आ॒ग्ना॒वै॒ष्ण॒वम् घृ॒ते । आ॒ग्ना॒वै॒ष्ण॒वमित्या᳚ग्ना - वै॒ष्ण॒वम् । घृ॒ते च॒रुम् । च॒रुम् निः । निर् व॑पेत् । व॒पे॒च्चक्षु॑ष्कामः । चक्षु॑ष्कामो॒ऽग्नेः । चक्षु॑ष्काम॒ इति॒ चक्षुः॑ - का॒मः॒ । अ॒ग्नेर् वै । वै चक्षु॑षा । चक्षु॑षा मनु॒ष्याः᳚ । म॒नु॒ष्या॑ वि । वि प॑श्यन्ति । प॒श्य॒न्ति॒ य॒ज्ञ्स्य॑ । य॒ज्ञ्स्य॑ दे॒वाः । दे॒वा अ॒ग्निम् । अ॒ग्निम् च॑ । चै॒व । ए॒व विष्णु᳚म् । विष्णु॑म् च । च॒ स्वेन॑ । स्वेन॑ भाग॒धेये॑न । भा॒ग॒धेये॒नोप॑ । भा॒ग॒धेये॒नेति॑ भाग - धेये॑न । उप॑ धावति । धा॒व॒ति॒ तौ । तावे॒व । ए॒वास्मिन्न्॑ \newline

\textbf{Jatai Paata} \newline

1. उ॒प॒नमे॑ द॒ग्नि र॒ग्नि रु॑प॒नमे॑ दुप॒नमे॑ द॒ग्निः । \newline
2. उ॒प॒नमे॒दित्यु॑प - नमे᳚त् । \newline
3. अ॒ग्निः सर्वाः॒ सर्वा॑ अ॒ग्नि र॒ग्निः सर्वाः᳚ । \newline
4. सर्वा॑ दे॒वता॑ दे॒वताः॒ सर्वाः॒ सर्वा॑ दे॒वताः᳚ । \newline
5. दे॒वता॒ विष्णु॒र् विष्णु॑र् दे॒वता॑ दे॒वता॒ विष्णुः॑ । \newline
6. विष्णु॑र् य॒ज्ञो य॒ज्ञो विष्णु॒र् विष्णु॑र् य॒ज्ञ्ः । \newline
7. य॒ज्ञो᳚ ऽग्नि म॒ग्निं ॅय॒ज्ञो य॒ज्ञो᳚ ऽग्निम् । \newline
8. अ॒ग्निम् च॑ चा॒ग्नि म॒ग्निम् च॑ । \newline
9. चै॒वैव च॑ चै॒व । \newline
10. ए॒व विष्णुं॒ ॅविष्णु॑ मे॒वैव विष्णु᳚म् । \newline
11. विष्णु॑म् च च॒ विष्णुं॒ ॅविष्णु॑म् च । \newline
12. च॒ स्वेन॒ स्वेन॑ च च॒ स्वेन॑ । \newline
13. स्वेन॑ भाग॒धेये॑न भाग॒धेये॑न॒ स्वेन॒ स्वेन॑ भाग॒धेये॑न । \newline
14. भा॒ग॒धेये॒नोपोप॑ भाग॒धेये॑न भाग॒धेये॒नोप॑ । \newline
15. भा॒ग॒धेये॒नेति॑ भाग - धेये॑न । \newline
16. उप॑ धावति धाव॒ त्युपोप॑ धावति । \newline
17. धा॒व॒ति॒ तौ तौ धा॑वति धावति॒ तौ । \newline
18. ता वे॒वैव तौ ता वे॒व । \newline
19. ए॒वास्मा॑ अस्मा ए॒वैवास्मै᳚ । \newline
20. अ॒स्मै॒ य॒ज्ञ्ं ॅय॒ज्ञ् म॑स्मा अस्मै य॒ज्ञ्म् । \newline
21. य॒ज्ञ्म् प्र प्र य॒ज्ञ्ं ॅय॒ज्ञ्म् प्र । \newline
22. प्र य॑च्छतो यच्छतः॒ प्र प्र य॑च्छतः । \newline
23. य॒च्छ॒त॒ उपोप॑ यच्छतो यच्छत॒ उप॑ । \newline
24. उपै॑न मेन॒ मुपोपै॑नम् । \newline
25. ए॒नं॒ ॅय॒ज्ञो य॒ज्ञ् ए॑न मेनं ॅय॒ज्ञ्ः । \newline
26. य॒ज्ञो न॑मति नमति य॒ज्ञो य॒ज्ञो न॑मति । \newline
27. न॒म॒ त्या॒ग्ना॒वै॒ष्ण॒व मा᳚ग्नावैष्ण॒वम् न॑मति नम त्याग्नावैष्ण॒वम् । \newline
28. आ॒ग्ना॒वै॒ष्ण॒वम् घृ॒ते घृ॒त आ᳚ग्नावैष्ण॒व मा᳚ग्नावैष्ण॒वम् घृ॒ते । \newline
29. आ॒ग्ना॒वै॒ष्ण॒वमित्या᳚ग्ना - वै॒ष्ण॒वम् । \newline
30. घृ॒ते च॒रुम् च॒रुम् घृ॒ते घृ॒ते च॒रुम् । \newline
31. च॒रुम् निर् णिश् च॒रुम् च॒रुम् निः । \newline
32. निर् व॑पेद् वपे॒न् निर् णिर् व॑पेत् । \newline
33. व॒पे॒च् चक्षु॑ष्काम॒ श्चक्षु॑ष्कामो वपेद् वपे॒च् चक्षु॑ष्कामः । \newline
34. चक्षु॑ष्कामो॒ ऽग्ने र॒ग्ने श्चक्षु॑ष्काम॒ श्चक्षु॑ष्कामो॒ ऽग्नेः । \newline
35. चक्षु॑ष्काम॒ इति॒ चक्षुः॑ - का॒मः॒ । \newline
36. अ॒ग्नेर् वै वा अ॒ग्ने र॒ग्नेर् वै । \newline
37. वै चक्षु॑षा॒ चक्षु॑षा॒ वै वै चक्षु॑षा । \newline
38. चक्षु॑षा मनु॒ष्या॑ मनु॒ष्या᳚ श्चक्षु॑षा॒ चक्षु॑षा मनु॒ष्याः᳚ । \newline
39. म॒नु॒ष्या॑ वि वि म॑नु॒ष्या॑ मनु॒ष्या॑ वि । \newline
40. वि प॑श्यन्ति पश्यन्ति॒ वि वि प॑श्यन्ति । \newline
41. प॒श्य॒न्ति॒ य॒ज्ञ्स्य॑ य॒ज्ञ्स्य॑ पश्यन्ति पश्यन्ति य॒ज्ञ्स्य॑ । \newline
42. य॒ज्ञ्स्य॑ दे॒वा दे॒वा य॒ज्ञ्स्य॑ य॒ज्ञ्स्य॑ दे॒वाः । \newline
43. दे॒वा अ॒ग्नि म॒ग्निम् दे॒वा दे॒वा अ॒ग्निम् । \newline
44. अ॒ग्निम् च॑ चा॒ग्नि म॒ग्निम् च॑ । \newline
45. चै॒वैव च॑ चै॒व । \newline
46. ए॒व विष्णुं॒ ॅविष्णु॑ मे॒वैव विष्णु᳚म् । \newline
47. विष्णु॑म् च च॒ विष्णुं॒ ॅविष्णु॑म् च । \newline
48. च॒ स्वेन॒ स्वेन॑ च च॒ स्वेन॑ । \newline
49. स्वेन॑ भाग॒धेये॑न भाग॒धेये॑न॒ स्वेन॒ स्वेन॑ भाग॒धेये॑न । \newline
50. भा॒ग॒धेये॒नोपोप॑ भाग॒धेये॑न भाग॒धेये॒नोप॑ । \newline
51. भा॒ग॒धेये॒नेति॑ भाग - धेये॑न । \newline
52. उप॑ धावति धाव॒ त्युपोप॑ धावति । \newline
53. धा॒व॒ति॒ तौ तौ धा॑वति धावति॒ तौ । \newline
54. ता वे॒वैव तौ ता वे॒व । \newline
55. ए॒वास्मि॑न् नस्मिन् ने॒वैवास्मिन्न्॑ । \newline

\textbf{Ghana Paata } \newline

1. उ॒प॒नमे॑ द॒ग्नि र॒ग्नि रु॑प॒नमे॑ दुप॒नमे॑ द॒ग्निः सर्वाः॒ सर्वा॑ अ॒ग्नि रु॑प॒नमे॑ दुप॒नमे॑ द॒ग्निः सर्वाः᳚ । \newline
2. उ॒प॒नमे॒दित्यु॑प - नमे᳚त् । \newline
3. अ॒ग्निः सर्वाः॒ सर्वा॑ अ॒ग्नि र॒ग्निः सर्वा॑ दे॒वता॑ दे॒वताः॒ सर्वा॑ अ॒ग्नि र॒ग्निः सर्वा॑ दे॒वताः᳚ । \newline
4. सर्वा॑ दे॒वता॑ दे॒वताः॒ सर्वाः॒ सर्वा॑ दे॒वता॒ विष्णु॒र् विष्णु॑र् दे॒वताः॒ सर्वाः॒ सर्वा॑ दे॒वता॒ विष्णुः॑ । \newline
5. दे॒वता॒ विष्णु॒र् विष्णु॑र् दे॒वता॑ दे॒वता॒ विष्णु॑र् य॒ज्ञो य॒ज्ञो विष्णु॑र् दे॒वता॑ दे॒वता॒ विष्णु॑र् य॒ज्ञ्ः । \newline
6. विष्णु॑र् य॒ज्ञो य॒ज्ञो विष्णु॒र् विष्णु॑र् य॒ज्ञो᳚ ऽग्नि म॒ग्निं ॅय॒ज्ञो विष्णु॒र् विष्णु॑र् य॒ज्ञो᳚ ऽग्निम् । \newline
7. य॒ज्ञो᳚ ऽग्नि म॒ग्निं ॅय॒ज्ञो य॒ज्ञो᳚ ऽग्निम् च॑ चा॒ग्निं ॅय॒ज्ञो य॒ज्ञो᳚ ऽग्निम् च॑ । \newline
8. अ॒ग्निम् च॑ चा॒ग्नि म॒ग्निम् चै॒वैव चा॒ग्नि म॒ग्निम् चै॒व । \newline
9. चै॒वैव च॑ चै॒व विष्णुं॒ ॅविष्णु॑ मे॒व च॑ चै॒व विष्णु᳚म् । \newline
10. ए॒व विष्णुं॒ ॅविष्णु॑ मे॒वैव विष्णु॑म् च च॒ विष्णु॑ मे॒वैव विष्णु॑म् च । \newline
11. विष्णु॑म् च च॒ विष्णुं॒ ॅविष्णु॑म् च॒ स्वेन॒ स्वेन॑ च॒ विष्णुं॒ ॅविष्णु॑म् च॒ स्वेन॑ । \newline
12. च॒ स्वेन॒ स्वेन॑ च च॒ स्वेन॑ भाग॒धेये॑न भाग॒धेये॑न॒ स्वेन॑ च च॒ स्वेन॑ भाग॒धेये॑न । \newline
13. स्वेन॑ भाग॒धेये॑न भाग॒धेये॑न॒ स्वेन॒ स्वेन॑ भाग॒धेये॒नोपोप॑ भाग॒धेये॑न॒ स्वेन॒ स्वेन॑ भाग॒धेये॒नोप॑ । \newline
14. भा॒ग॒धेये॒नोपोप॑ भाग॒धेये॑न भाग॒धेये॒नोप॑ धावति धाव॒ त्युप॑ भाग॒धेये॑न भाग॒धेये॒नोप॑ धावति । \newline
15. भा॒ग॒धेये॒नेति॑ भाग - धेये॑न । \newline
16. उप॑ धावति धाव॒ त्युपोप॑ धावति॒ तौ तौ धा॑व॒ त्युपोप॑ धावति॒ तौ । \newline
17. धा॒व॒ति॒ तौ तौ धा॑वति धावति॒ ता वे॒वैव तौ धा॑वति धावति॒ ता वे॒व । \newline
18. ता वे॒वैव तौ ता वे॒वास्मा॑ अस्मा ए॒व तौ ता वे॒वास्मै᳚ । \newline
19. ए॒वास्मा॑ अस्मा ए॒वैवास्मै॑ य॒ज्ञ्ं ॅय॒ज्ञ् म॑स्मा ए॒वैवास्मै॑ य॒ज्ञ्म् । \newline
20. अ॒स्मै॒ य॒ज्ञ्ं ॅय॒ज्ञ् म॑स्मा अस्मै य॒ज्ञ्म् प्र प्र य॒ज्ञ् म॑स्मा अस्मै य॒ज्ञ्म् प्र । \newline
21. य॒ज्ञ्म् प्र प्र य॒ज्ञ्ं ॅय॒ज्ञ्म् प्र य॑च्छतो यच्छतः॒ प्र य॒ज्ञ्ं ॅय॒ज्ञ्म् प्र य॑च्छतः । \newline
22. प्र य॑च्छतो यच्छतः॒ प्र प्र य॑च्छत॒ उपोप॑ यच्छतः॒ प्र प्र य॑च्छत॒ उप॑ । \newline
23. य॒च्छ॒त॒ उपोप॑ यच्छतो यच्छत॒ उपै॑न मेन॒ मुप॑ यच्छतो यच्छत॒ उपै॑नम् । \newline
24. उपै॑न मेन॒ मुपोपै॑नं ॅय॒ज्ञो य॒ज्ञ् ए॑न॒ मुपोपै॑नं ॅय॒ज्ञ्ः । \newline
25. ए॒नं॒ ॅय॒ज्ञो य॒ज्ञ् ए॑न मेनं ॅय॒ज्ञो न॑मति नमति य॒ज्ञ् ए॑न मेनं ॅय॒ज्ञो न॑मति । \newline
26. य॒ज्ञो न॑मति नमति य॒ज्ञो य॒ज्ञो न॑म त्याग्नावैष्ण॒व मा᳚ग्नावैष्ण॒वम् न॑मति य॒ज्ञो य॒ज्ञो न॑मत्याग्नावैष्ण॒वम् । \newline
27. न॒म॒ त्या॒ग्ना॒वै॒ष्ण॒व मा᳚ग्नावैष्ण॒वम् न॑मति नम त्याग्नावैष्ण॒वम् घृ॒ते घृ॒त आ᳚ग्नावैष्ण॒वम् न॑मति नम त्याग्नावैष्ण॒वम् घृ॒ते । \newline
28. आ॒ग्ना॒वै॒ष्ण॒वम् घृ॒ते घृ॒त आ᳚ग्नावैष्ण॒व मा᳚ग्नावैष्ण॒वम् घृ॒ते च॒रुम् च॒रुम् घृ॒त आ᳚ग्नावैष्ण॒व मा᳚ग्नावैष्ण॒वम् घृ॒ते च॒रुम् । \newline
29. आ॒ग्ना॒वै॒ष्ण॒वमित्या᳚ग्ना - वै॒ष्ण॒वम् । \newline
30. घृ॒ते च॒रुम् च॒रुम् घृ॒ते घृ॒ते च॒रुम् निर् णिश्च॒रुम् घृ॒ते घृ॒ते च॒रुम् निः । \newline
31. च॒रुम् निर् णिश्च॒रुम् च॒रुम् निर् व॑पेद् वपे॒न् निश्च॒रुम् च॒रुम् निर् व॑पेत् । \newline
32. निर् व॑पेद् वपे॒न् निर् णिर् व॑पे॒च् चक्षु॑ष्काम॒ श्चक्षु॑ष्कामो वपे॒न् निर् णिर् व॑पे॒च् चक्षु॑ष्कामः । \newline
33. व॒पे॒च् चक्षु॑ष्काम॒ श्चक्षु॑ष्कामो वपेद् वपे॒च् चक्षु॑ष्कामो॒ ऽग्ने र॒ग्ने श्चक्षु॑ष्कामो वपेद् वपे॒च् चक्षु॑ष्कामो॒ ऽग्नेः । \newline
34. चक्षु॑ष्कामो॒ ऽग्ने र॒ग्ने श्चक्षु॑ष्काम॒ श्चक्षु॑ष्कामो॒ ऽग्नेर् वै वा अ॒ग्ने श्चक्षु॑ष्काम॒ श्चक्षु॑ष्कामो॒ ऽग्नेर् वै । \newline
35. चक्षु॑ष्काम॒ इति॒ चक्षुः॑ - का॒मः॒ । \newline
36. अ॒ग्नेर् वै वा अ॒ग्ने र॒ग्नेर् वै चक्षु॑षा॒ चक्षु॑षा॒ वा अ॒ग्ने र॒ग्नेर् वै चक्षु॑षा । \newline
37. वै चक्षु॑षा॒ चक्षु॑षा॒ वै वै चक्षु॑षा मनु॒ष्या॑ मनु॒ष्या᳚ श्चक्षु॑षा॒ वै वै चक्षु॑षा मनु॒ष्याः᳚ । \newline
38. चक्षु॑षा मनु॒ष्या॑ मनु॒ष्या᳚ श्चक्षु॑षा॒ चक्षु॑षा मनु॒ष्या॑ वि वि म॑नु॒ष्या᳚ श्चक्षु॑षा॒ चक्षु॑षा मनु॒ष्या॑ वि । \newline
39. म॒नु॒ष्या॑ वि वि म॑नु॒ष्या॑ मनु॒ष्या॑ वि प॑श्यन्ति पश्यन्ति॒ वि म॑नु॒ष्या॑ मनु॒ष्या॑ वि प॑श्यन्ति । \newline
40. वि प॑श्यन्ति पश्यन्ति॒ वि वि प॑श्यन्ति य॒ज्ञ्स्य॑ य॒ज्ञ्स्य॑ पश्यन्ति॒ वि वि प॑श्यन्ति य॒ज्ञ्स्य॑ । \newline
41. प॒श्य॒न्ति॒ य॒ज्ञ्स्य॑ य॒ज्ञ्स्य॑ पश्यन्ति पश्यन्ति य॒ज्ञ्स्य॑ दे॒वा दे॒वा य॒ज्ञ्स्य॑ पश्यन्ति पश्यन्ति य॒ज्ञ्स्य॑ दे॒वाः । \newline
42. य॒ज्ञ्स्य॑ दे॒वा दे॒वा य॒ज्ञ्स्य॑ य॒ज्ञ्स्य॑ दे॒वा अ॒ग्नि म॒ग्निम् दे॒वा य॒ज्ञ्स्य॑ य॒ज्ञ्स्य॑ दे॒वा अ॒ग्निम् । \newline
43. दे॒वा अ॒ग्नि म॒ग्निम् दे॒वा दे॒वा अ॒ग्निम् च॑ चा॒ग्निम् दे॒वा दे॒वा अ॒ग्निम् च॑ । \newline
44. अ॒ग्निम् च॑ चा॒ग्नि म॒ग्निम् चै॒वैव चा॒ग्नि म॒ग्निम् चै॒व । \newline
45. चै॒वैव च॑ चै॒व विष्णुं॒ ॅविष्णु॑ मे॒व च॑ चै॒व विष्णु᳚म् । \newline
46. ए॒व विष्णुं॒ ॅविष्णु॑ मे॒वैव विष्णु॑म् च च॒ विष्णु॑ मे॒वैव विष्णु॑म् च । \newline
47. विष्णु॑म् च च॒ विष्णुं॒ ॅविष्णु॑म् च॒ स्वेन॒ स्वेन॑ च॒ विष्णुं॒ ॅविष्णु॑म् च॒ स्वेन॑ । \newline
48. च॒ स्वेन॒ स्वेन॑ च च॒ स्वेन॑ भाग॒धेये॑न भाग॒धेये॑न॒ स्वेन॑ च च॒ स्वेन॑ भाग॒धेये॑न । \newline
49. स्वेन॑ भाग॒धेये॑न भाग॒धेये॑न॒ स्वेन॒ स्वेन॑ भाग॒धेये॒नोपोप॑ भाग॒धेये॑न॒ स्वेन॒ स्वेन॑ भाग॒धेये॒नोप॑ । \newline
50. भा॒ग॒धेये॒नोपोप॑ भाग॒धेये॑न भाग॒धेये॒नोप॑ धावति धाव॒ त्युप॑ भाग॒धेये॑न भाग॒धेये॒नोप॑ धावति । \newline
51. भा॒ग॒धेये॒नेति॑ भाग - धेये॑न । \newline
52. उप॑ धावति धाव॒ त्युपोप॑ धावति॒ तौ तौ धा॑व॒ त्युपोप॑ धावति॒ तौ । \newline
53. धा॒व॒ति॒ तौ तौ धा॑वति धावति॒ ता वे॒वैव तौ धा॑वति धावति॒ ता वे॒व । \newline
54. ता वे॒वैव तौ ता वे॒वास्मि॑न् नस्मिन् ने॒व तौ ता वे॒वास्मिन्न्॑ । \newline
55. ए॒वास्मि॑न् नस्मिन् ने॒वैवास्मि॒न् चक्षु॒ श्चक्षु॑ रस्मिन् ने॒वैवास्मि॒न् चक्षुः॑ । \newline
\pagebreak
\markright{ TS 2.2.9.4  \hfill https://www.vedavms.in \hfill}
\addcontentsline{toc}{section}{ TS 2.2.9.4 }
\section*{ TS 2.2.9.4 }

\textbf{TS 2.2.9.4 } \newline
\textbf{Samhita Paata} \newline

-ऽस्मि॒न् चक्षु॑र्द्धत्त॒श्चक्षु॑ष्माने॒व भ॑वति धे॒न्वै वा ए॒तद्-रेतो॒ यदाज्य॑-मन॒डुह॑-स्तण्डु॒ला मि॑थु॒नादे॒वास्मै॒ चक्षुः॒ प्रज॑नयति घृ॒ते भ॑वति॒ तेजो॒ वै घृ॒तं तेज॒श्चक्षु॒-स्तेज॑सै॒वास्मै॒ तेज॒श्चक्षु॒रव॑ रुन्ध इन्द्रि॒यं ॅवै वी॒र्यं॑ ॅवृङ्क्ते॒ भ्रातृ॑व्यो॒ यज॑मा॒नोऽय॑जमानस्या-ध्व॒रक॑ल्पां॒ प्रति॒ निर्व॑पे॒द्-भ्रातृ॑व्ये॒ यज॑माने॒ नास्ये᳚न्द्रि॒यं - [  ] \newline

\textbf{Pada Paata} \newline

अ॒स्मि॒न्न् । चक्षुः॑ । ध॒त्तः॒ । चक्षु॑ष्मान् । ए॒व । भ॒व॒ति॒ । धे॒न्वै । वै । ए॒तत् । रेतः॑ । यत् । आज्य᳚म् । अ॒न॒डुहः॑ । त॒ण्डु॒लाः । मि॒थु॒नात् । ए॒व । अ॒स्मै॒ । चक्षुः॑ । प्रेति॑ । ज॒न॒य॒ति॒ । घृ॒ते । भ॒व॒ति॒ । तेजः॑ । वै । घृ॒तम् । तेजः॑ । चक्षुः॑ । तेज॑सा । ए॒व । अ॒स्मै॒ । तेजः॑ । चक्षुः॑ । अवेति॑ । रु॒न्धे॒ । इ॒न्द्रि॒यम् । वै । वी॒र्य᳚म् । वृ॒ङ्क्ते॒ । भ्रातृ॑व्यः । यज॑मानः । अय॑जमानस्य । अ॒द्ध्व॒रक॑ल्पा॒मित्य॑द्ध्व॒र-क॒ल्पा॒म् । प्रति॑ । निरिति॑ । व॒पे॒त् । भ्रातृ॑व्ये । यज॑माने । न । अ॒स्य॒ । इ॒न्द्रि॒यम् ।  \newline


\textbf{Krama Paata} \newline

अ॒स्मि॒न् चक्षुः॑ । चक्षु॑र्,धत्तः । ध॒त्त॒ श्चक्षु॑ष्मान् । चक्षु॑ष्माने॒व । ए॒व भ॑वति । भ॒व॒ति॒ धे॒न्वै । धे॒न्वै वै । वा ए॒तत् । ए॒तद् रेतः॑ । रेतो॒ यत् । यदाज्य᳚म् । आज्य॑मन॒डुहः॑ । अ॒न॒डुह॑ स्तण्डु॒लाः । त॒ण्डु॒ला मि॑थु॒नात् । मि॒थु॒नादे॒व । ए॒वास्मै᳚ । अ॒स्मै॒ चक्षुः॑ । चक्षुः॒ प्र । प्र ज॑नयति । ज॒न॒य॒ति॒ घृ॒ते । घृ॒ते भ॑वति । भ॒व॒ति॒ तेजः॑ । तेजो॒ वै । वै घृ॒तम् । घृ॒तम् तेजः॑ । तेज॒ श्चक्षुः॑ । चक्षु॒स्तेज॑सा । तेज॑सै॒व । ए॒वास्मै᳚ । अ॒स्मै॒ तेजः॑ । तेज॒ श्चक्षुः॑ । चक्षु॒रव॑ । अव॑ रुन्धे । रु॒न्ध॒ इ॒न्द्रि॒यम् । इ॒न्द्रि॒यं ॅवै । वै वी॒र्य᳚म् । वी॒र्यं॑ ॅवृङ्क्ते । वृ॒ङ्क्ते॒ भ्रातृ॑व्यः । भ्रातृ॑व्यो॒ यज॑मानः । यज॑मा॒नो ऽय॑जमानस्य । अय॑जमानस्याद्ध्व॒रक॑ल्पाम् । अ॒द्ध्व॒रक॑ल्पा॒म् प्रति॑ । अ॒द्ध्व॒रक॑ल्पा॒मित्य॑द्ध्व॒र - क॒ल्पा॒म् । प्रति॒ निः । निर् व॑पेत् । व॒पे॒द्,भ्रातृ॑व्ये । भ्रातृ॑व्ये॒ यज॑माने । यज॑माने॒ न । नास्य॑ । अ॒स्ये॒न्द्रि॒यम् । इ॒न्द्रि॒यं ॅवी॒र्य᳚म् \newline

\textbf{Jatai Paata} \newline

1. अ॒स्मि॒न् चक्षु॒ श्चक्षु॑रस्मिन् नस्मि॒न् चक्षुः॑ । \newline
2. चक्षु॑र् धत्तो धत्त॒ श्चक्षु॒ श्चक्षु॑र् धत्तः । \newline
3. ध॒त्त॒ श्चक्षु॑ष्माꣳ॒॒ श्चक्षु॑ष्मान् धत्तो धत्त॒ श्चक्षु॑ष्मान् । \newline
4. चक्षु॑ष्मा ने॒वैव चक्षु॑ष्माꣳ॒॒ श्चक्षु॑ष्मा ने॒व । \newline
5. ए॒व भ॑वति भव त्ये॒वैव भ॑वति । \newline
6. भ॒व॒ति॒ धे॒न्वै धे॒न्वै भ॑वति भवति धे॒न्वै । \newline
7. धे॒न्वै वै वै धे॒न्वै धे॒न्वै वै । \newline
8. वा ए॒त दे॒तद् वै वा ए॒तत् । \newline
9. ए॒तद् रेतो॒ रेत॑ ए॒त दे॒तद् रेतः॑ । \newline
10. रेतो॒ यद् यद् रेतो॒ रेतो॒ यत् । \newline
11. यदाज्य॒ माज्यं॒ ॅयद् यदाज्य᳚म् । \newline
12. आज्य॑ मन॒डुहो॑ ऽन॒डुह॒ आज्य॒ माज्य॑ मन॒डुहः॑ । \newline
13. अ॒न॒डुह॑ स्तण्डु॒ला स्त॑ण्डु॒ला अ॑न॒डुहो॑ ऽन॒डुह॑ स्तण्डु॒लाः । \newline
14. त॒ण्डु॒ला मि॑थु॒नान् मि॑थु॒नात् त॑ण्डु॒ला स्त॑ण्डु॒ला मि॑थु॒नात् । \newline
15. मि॒थु॒ना दे॒वैव मि॑थु॒नान् मि॑थु॒ना दे॒व । \newline
16. ए॒वास्मा॑ अस्मा ए॒वैवास्मै᳚ । \newline
17. अ॒स्मै॒ चक्षु॒ श्चक्षु॑ रस्मा अस्मै॒ चक्षुः॑ । \newline
18. चक्षुः॒ प्र प्र चक्षु॒ श्चक्षुः॒ प्र । \newline
19. प्र ज॑नयति जनयति॒ प्र प्र ज॑नयति । \newline
20. ज॒न॒य॒ति॒ घृ॒ते घृ॒ते ज॑नयति जनयति घृ॒ते । \newline
21. घृ॒ते भ॑वति भवति घृ॒ते घृ॒ते भ॑वति । \newline
22. भ॒व॒ति॒ तेज॒ स्तेजो॑ भवति भवति॒ तेजः॑ । \newline
23. तेजो॒ वै वै तेज॒ स्तेजो॒ वै । \newline
24. वै घृ॒तम् घृ॒तं ॅवै वै घृ॒तम् । \newline
25. घृ॒तम् तेज॒ स्तेजो॑ घृ॒तम् घृ॒तम् तेजः॑ । \newline
26. तेज॒ श्चक्षु॒ श्चक्षु॒ स्तेज॒ स्तेज॒ श्चक्षुः॑ । \newline
27. चक्षु॒ स्तेज॑सा॒ तेज॑सा॒ चक्षु॒ श्चक्षु॒ स्तेज॑सा । \newline
28. तेज॑सै॒वैव तेज॑सा॒ तेज॑सै॒व । \newline
29. ए॒वास्मा॑ अस्मा ए॒वैवास्मै᳚ । \newline
30. अ॒स्मै॒ तेज॒ स्तेजो᳚ ऽस्मा अस्मै॒ तेजः॑ । \newline
31. तेज॒ श्चक्षु॒ श्चक्षु॒ स्तेज॒ स्तेज॒ श्चक्षुः॑ । \newline
32. चक्षु॒ रवाव॒ चक्षु॒ श्चक्षु॒ रव॑ । \newline
33. अव॑ रुन्धे रु॒न्धे ऽवाव॑ रुन्धे । \newline
34. रु॒न्ध॒ इ॒न्द्रि॒य मि॑न्द्रि॒यꣳ रु॑न्धे रुन्ध इन्द्रि॒यम् । \newline
35. इ॒न्द्रि॒यं ॅवै वा इ॑न्द्रि॒य मि॑न्द्रि॒यं ॅवै । \newline
36. वै वी॒र्यं॑ ॅवी॒र्यं॑ ॅवै वै वी॒र्य᳚म् । \newline
37. वी॒र्यं॑ ॅवृङ्क्ते वृङ्क्ते वी॒र्यं॑ ॅवी॒र्यं॑ ॅवृङ्क्ते । \newline
38. वृ॒ङ्क्ते॒ भ्रातृ॑व्यो॒ भ्रातृ॑व्यो वृङ्क्ते वृङ्क्ते॒ भ्रातृ॑व्यः । \newline
39. भ्रातृ॑व्यो॒ यज॑मानो॒ यज॑मानो॒ भ्रातृ॑व्यो॒ भ्रातृ॑व्यो॒ यज॑मानः । \newline
40. यज॑मा॒नो ऽय॑जमान॒स्या य॑जमानस्य॒ यज॑मानो॒ यज॑मा॒नो ऽय॑जमानस्य । \newline
41. अय॑जमानस्या द्ध्व॒रक॑ल्पा मद्ध्व॒रक॑ल्पा॒ मय॑जमान॒स्या य॑जमानस्या द्ध्व॒रक॑ल्पाम् । \newline
42. अ॒द्ध्व॒रक॑ल्पा॒म् प्रति॒ प्रत्य॑ द्ध्व॒रक॑ल्पा मद्ध्व॒रक॑ल्पा॒म् प्रति॑ । \newline
43. अ॒द्ध्व॒रक॑ल्पा॒मित्य॑द्ध्व॒र - क॒ल्पा॒म् । \newline
44. प्रति॒ निर् णिष् प्रति॒ प्रति॒ निः । \newline
45. निर् व॑पेद् वपे॒न् निर् णिर् व॑पेत् । \newline
46. व॒पे॒द् भ्रातृ॑व्ये॒ भ्रातृ॑व्ये वपेद् वपे॒द् भ्रातृ॑व्ये । \newline
47. भ्रातृ॑व्ये॒ यज॑माने॒ यज॑माने॒ भ्रातृ॑व्ये॒ भ्रातृ॑व्ये॒ यज॑माने । \newline
48. यज॑माने॒ न न यज॑माने॒ यज॑माने॒ न । \newline
49. नास्या᳚स्य॒ न नास्य॑ । \newline
50. अ॒स्ये॒ न्द्रि॒य मि॑न्द्रि॒य म॑स्यास्ये न्द्रि॒यम् । \newline
51. इ॒न्द्रि॒यं ॅवी॒र्यं॑ ॅवी॒र्य॑ मिन्द्रि॒य मि॑न्द्रि॒यं ॅवी॒र्य᳚म् । \newline

\textbf{Ghana Paata } \newline

1. अ॒स्मि॒न् चक्षु॒ श्चक्षु॑ रस्मिन् नस्मि॒न् चक्षु॑र् धत्तो धत्त॒ श्चक्षु॑ रस्मिन् नस्मि॒न् चक्षु॑र् धत्तः । \newline
2. चक्षु॑र् धत्तो धत्त॒ श्चक्षु॒ श्चक्षु॑र् धत्त॒ श्चक्षु॑ष्माꣳ॒॒ श्चक्षु॑ष्मान् धत्त॒ श्चक्षु॒ श्चक्षु॑र् धत्त॒ श्चक्षु॑ष्मान् । \newline
3. ध॒त्त॒ श्चक्षु॑ष्माꣳ॒॒ श्चक्षु॑ष्मान् धत्तो धत्त॒ श्चक्षु॑ष्मा ने॒वैव चक्षु॑ष्मान् धत्तो धत्त॒ श्चक्षु॑ष्मा ने॒व । \newline
4. चक्षु॑ष्मा ने॒वैव चक्षु॑ष्माꣳ॒॒ श्चक्षु॑ष्मा ने॒व भ॑वति भवत्ये॒व चक्षु॑ष्माꣳ॒॒ श्चक्षु॑ष्मा ने॒व भ॑वति । \newline
5. ए॒व भ॑वति भव त्ये॒वैव भ॑वति धे॒न्वै धे॒न्वै भ॑व त्ये॒वैव भ॑वति धे॒न्वै । \newline
6. भ॒व॒ति॒ धे॒न्वै धे॒न्वै भ॑वति भवति धे॒न्वै वै वै धे॒न्वै भ॑वति भवति धे॒न्वै वै । \newline
7. धे॒न्वै वै वै धे॒न्वै धे॒न्वै वा ए॒त दे॒तद् वै धे॒न्वै धे॒न्वै वा ए॒तत् । \newline
8. वा ए॒त दे॒तद् वै वा ए॒तद् रेतो॒ रेत॑ ए॒तद् वै वा ए॒तद् रेतः॑ । \newline
9. ए॒तद् रेतो॒ रेत॑ ए॒त दे॒तद् रेतो॒ यद् यद् रेत॑ ए॒त दे॒तद् रेतो॒ यत् । \newline
10. रेतो॒ यद् यद् रेतो॒ रेतो॒ यदाज्य॒ माज्यं॒ ॅयद् रेतो॒ रेतो॒ यदाज्य᳚म् । \newline
11. यदाज्य॒ माज्यं॒ ॅयद् यदाज्य॑ मन॒डुहो॑ ऽन॒डुह॒ आज्यं॒ ॅयद् यदाज्य॑ मन॒डुहः॑ । \newline
12. आज्य॑ मन॒डुहो॑ ऽन॒डुह॒ आज्य॒ माज्य॑ मन॒डुह॑ स्तण्डु॒ला स्त॑ण्डु॒ला अ॑न॒डुह॒ आज्य॒ माज्य॑ मन॒डुह॑ स्तण्डु॒लाः । \newline
13. अ॒न॒डुह॑ स्तण्डु॒ला स्त॑ण्डु॒ला अ॑न॒डुहो॑ ऽन॒डुह॑ स्तण्डु॒ला मि॑थु॒नान् मि॑थु॒नात् त॑ण्डु॒ला अ॑न॒डुहो॑ ऽन॒डुह॑ स्तण्डु॒ला मि॑थु॒नात् । \newline
14. त॒ण्डु॒ला मि॑थु॒नान् मि॑थु॒नात् त॑ण्डु॒ला स्त॑ण्डु॒ला मि॑थु॒ना दे॒वैव मि॑थु॒नात् त॑ण्डु॒ला स्त॑ण्डु॒ला मि॑थु॒ना दे॒व । \newline
15. मि॒थु॒ना दे॒वैव मि॑थु॒नान् मि॑थु॒ना दे॒वास्मा॑ अस्मा ए॒व मि॑थु॒नान् मि॑थु॒ना दे॒वास्मै᳚ । \newline
16. ए॒वास्मा॑ अस्मा ए॒वैवास्मै॒ चक्षु॒ श्चक्षु॑रस्मा ए॒वैवास्मै॒ चक्षुः॑ । \newline
17. अ॒स्मै॒ चक्षु॒ श्चक्षु॑रस्मा अस्मै॒ चक्षुः॒ प्र प्र चक्षु॑ रस्मा अस्मै॒ चक्षुः॒ प्र । \newline
18. चक्षुः॒ प्र प्र चक्षु॒ श्चक्षुः॒ प्र ज॑नयति जनयति॒ प्र चक्षु॒ श्चक्षुः॒ प्र ज॑नयति । \newline
19. प्र ज॑नयति जनयति॒ प्र प्र ज॑नयति घृ॒ते घृ॒ते ज॑नयति॒ प्र प्र ज॑नयति घृ॒ते । \newline
20. ज॒न॒य॒ति॒ घृ॒ते घृ॒ते ज॑नयति जनयति घृ॒ते भ॑वति भवति घृ॒ते ज॑नयति जनयति घृ॒ते भ॑वति । \newline
21. घृ॒ते भ॑वति भवति घृ॒ते घृ॒ते भ॑वति॒ तेज॒ स्तेजो॑ भवति घृ॒ते घृ॒ते भ॑वति॒ तेजः॑ । \newline
22. भ॒व॒ति॒ तेज॒ स्तेजो॑ भवति भवति॒ तेजो॒ वै वै तेजो॑ भवति भवति॒ तेजो॒ वै । \newline
23. तेजो॒ वै वै तेज॒ स्तेजो॒ वै घृ॒तम् घृ॒तं ॅवै तेज॒ स्तेजो॒ वै घृ॒तम् । \newline
24. वै घृ॒तम् घृ॒तं ॅवै वै घृ॒तम् तेज॒ स्तेजो॑ घृ॒तं ॅवै वै घृ॒तम् तेजः॑ । \newline
25. घृ॒तम् तेज॒ स्तेजो॑ घृ॒तम् घृ॒तम् तेज॒ श्चक्षु॒ श्चक्षु॒ स्तेजो॑ घृ॒तम् घृ॒तम् तेज॒ श्चक्षुः॑ । \newline
26. तेज॒ श्चक्षु॒ श्चक्षु॒ स्तेज॒ स्तेज॒ श्चक्षु॒ स्तेज॑सा॒ तेज॑सा॒ चक्षु॒ स्तेज॒ स्तेज॒ श्चक्षु॒ स्तेज॑सा । \newline
27. चक्षु॒ स्तेज॑सा॒ तेज॑सा॒ चक्षु॒ श्चक्षु॒ स्तेज॑सै॒वैव तेज॑सा॒ चक्षु॒ श्चक्षु॒ स्तेज॑सै॒व । \newline
28. तेज॑सै॒वैव तेज॑सा॒ तेज॑सै॒वास्मा॑ अस्मा ए॒व तेज॑सा॒ तेज॑सै॒वास्मै᳚ । \newline
29. ए॒वास्मा॑ अस्मा ए॒वैवास्मै॒ तेज॒ स्तेजो᳚ ऽस्मा ए॒वैवास्मै॒ तेजः॑ । \newline
30. अ॒स्मै॒ तेज॒ स्तेजो᳚ ऽस्मा अस्मै॒ तेज॒ श्चक्षु॒ श्चक्षु॒ स्तेजो᳚ ऽस्मा अस्मै॒ तेज॒ श्चक्षुः॑ । \newline
31. तेज॒ श्चक्षु॒ श्चक्षु॒ स्तेज॒ स्तेज॒ श्चक्षु॒ रवाव॒ चक्षु॒ स्तेज॒ स्तेज॒ श्चक्षु॒ रव॑ । \newline
32. चक्षु॒ रवाव॒ चक्षु॒ श्चक्षु॒ रव॑ रुन्धे रु॒न्धे ऽव॒ चक्षु॒ श्चक्षु॒ रव॑ रुन्धे । \newline
33. अव॑ रुन्धे रु॒न्धे ऽवाव॑ रुन्ध इन्द्रि॒य मि॑न्द्रि॒यꣳ रु॒न्धे ऽवाव॑ रुन्ध इन्द्रि॒यम् । \newline
34. रु॒न्ध॒ इ॒न्द्रि॒य मि॑न्द्रि॒यꣳ रु॑न्धे रुन्ध इन्द्रि॒यं ॅवै वा इ॑न्द्रि॒यꣳ रु॑न्धे रुन्ध इन्द्रि॒यं ॅवै । \newline
35. इ॒न्द्रि॒यं ॅवै वा इ॑न्द्रि॒य मि॑न्द्रि॒यं ॅवै वी॒र्यं॑ ॅवी॒र्यं॑ ॅवा इ॑न्द्रि॒य मि॑न्द्रि॒यं ॅवै वी॒र्य᳚म् । \newline
36. वै वी॒र्यं॑ ॅवी॒र्यं॑ ॅवै वै वी॒र्यं॑ ॅवृङ्क्ते वृङ्क्ते वी॒र्यं॑ ॅवै वै वी॒र्यं॑ ॅवृङ्क्ते । \newline
37. वी॒र्यं॑ ॅवृङ्क्ते वृङ्क्ते वी॒र्यं॑ ॅवी॒र्यं॑ ॅवृङ्क्ते॒ भ्रातृ॑व्यो॒ भ्रातृ॑व्यो वृङ्क्ते वी॒र्यं॑ ॅवी॒र्यं॑ ॅवृङ्क्ते॒ भ्रातृ॑व्यः । \newline
38. वृ॒ङ्क्ते॒ भ्रातृ॑व्यो॒ भ्रातृ॑व्यो वृङ्क्ते वृङ्क्ते॒ भ्रातृ॑व्यो॒ यज॑मानो॒ यज॑मानो॒ भ्रातृ॑व्यो वृङ्क्ते वृङ्क्ते॒ भ्रातृ॑व्यो॒ यज॑मानः । \newline
39. भ्रातृ॑व्यो॒ यज॑मानो॒ यज॑मानो॒ भ्रातृ॑व्यो॒ भ्रातृ॑व्यो॒ यज॑मा॒नो ऽय॑जमान॒स्या य॑जमानस्य॒ यज॑मानो॒ भ्रातृ॑व्यो॒ भ्रातृ॑व्यो॒ यज॑मा॒नो ऽय॑जमानस्य । \newline
40. यज॑मा॒नो ऽय॑जमान॒स्या य॑जमानस्य॒ यज॑मानो॒ यज॑मा॒नो ऽय॑जमानस्या ध्व॒रक॑ल्पा मद्ध्व॒रक॑ल्पा॒ मय॑जमानस्य॒ यज॑मानो॒ यज॑मा॒नो ऽय॑जमानस्या ध्व॒रक॑ल्पाम् । \newline
41. अय॑जमानस्या ध्व॒रक॑ल्पा मद्ध्व॒रक॑ल्पा॒ मय॑जमान॒स्या य॑जमानस्या ध्व॒रक॑ल्पा॒म् प्रति॒ प्रत्य॑द्ध्व॒रक॑ल्पा॒ मय॑जमान॒स्या य॑जमानस्या ध्व॒रक॑ल्पा॒म् प्रति॑ । \newline
42. अ॒द्ध्व॒रक॑ल्पा॒म् प्रति॒ प्रत्य॑द्ध्व॒रक॑ल्पा मद्ध्व॒रक॑ल्पा॒म् प्रति॒ निर् णिष् प्रत्य॑द्ध्व॒रक॑ल्पा मद्ध्व॒रक॑ल्पा॒म् प्रति॒ निः । \newline
43. अ॒द्ध्व॒रक॑ल्पा॒मित्य॑द्ध्व॒र - क॒ल्पा॒म् । \newline
44. प्रति॒ निर् णिष् प्रति॒ प्रति॒ निर् व॑पेद् वपे॒न् निष् प्रति॒ प्रति॒ निर् व॑पेत् । \newline
45. निर् व॑पेद् वपे॒न् निर् णिर् व॑पे॒द् भ्रातृ॑व्ये॒ भ्रातृ॑व्ये वपे॒न् निर् णिर् व॑पे॒द् भ्रातृ॑व्ये । \newline
46. व॒पे॒द् भ्रातृ॑व्ये॒ भ्रातृ॑व्ये वपेद् वपे॒द् भ्रातृ॑व्ये॒ यज॑माने॒ यज॑माने॒ भ्रातृ॑व्ये वपेद् वपे॒द् भ्रातृ॑व्ये॒ यज॑माने । \newline
47. भ्रातृ॑व्ये॒ यज॑माने॒ यज॑माने॒ भ्रातृ॑व्ये॒ भ्रातृ॑व्ये॒ यज॑माने॒ न न यज॑माने॒ भ्रातृ॑व्ये॒ भ्रातृ॑व्ये॒ यज॑माने॒ न । \newline
48. यज॑माने॒ न न यज॑माने॒ यज॑माने॒ नास्या᳚स्य॒ न यज॑माने॒ यज॑माने॒ नास्य॑ । \newline
49. नास्या᳚स्य॒ न नास्ये᳚ न्द्रि॒य मि॑न्द्रि॒य म॑स्य॒ न नास्ये᳚ न्द्रि॒यम् । \newline
50. अ॒स्ये॒ न्द्रि॒य मि॑न्द्रि॒य म॑स्यास्ये न्द्रि॒यं ॅवी॒र्यं॑ ॅवी॒र्य॑ मिन्द्रि॒य म॑स्यास्ये न्द्रि॒यं ॅवी॒र्य᳚म् । \newline
51. इ॒न्द्रि॒यं ॅवी॒र्यं॑ ॅवी॒र्य॑ मिन्द्रि॒य मि॑न्द्रि॒यं ॅवी॒र्यं॑ ॅवृङ्क्ते वृङ्क्ते वी॒र्य॑ मिन्द्रि॒य मि॑न्द्रि॒यं ॅवी॒र्यं॑ ॅवृङ्क्ते । \newline
\pagebreak
\markright{ TS 2.2.9.5  \hfill https://www.vedavms.in \hfill}
\addcontentsline{toc}{section}{ TS 2.2.9.5 }
\section*{ TS 2.2.9.5 }

\textbf{TS 2.2.9.5 } \newline
\textbf{Samhita Paata} \newline

ॅवी॒र्यं॑ ॅवृङ्क्ते पु॒रावा॒चः प्रव॑दितो॒-र्निर्व॑पे॒द्याव॑त्ये॒व वाक् तामप्रो॑दितां॒ भ्रातृ॑व्यस्य वृङ्क्ते॒ ताम॑स्य॒ वाचं॑ प्र॒वद॑न्तीम॒न्या वाचोऽनु॒ प्रव॑दन्ति॒ ता इ॑न्द्रि॒यं ॅवी॒र्यं॑ ॅयज॑माने दधत्याग्ना वैष्ण॒व-म॒ष्टाक॑पालं॒ निर्व॑पेत् प्रातः सवन॒स्या॑ऽऽ* का॒ले सर॑स्व॒त्याज्य॑भागा॒ स्याद् बा॑र्.हस्प॒त्यश्च॒रु- र्यद॒ष्टाक॑पालो॒ भव॑त्य॒ष्टाक्ष॑रा गाय॒त्री गा॑य॒त्रं प्रा॑तः सव॒नं प्रा॑तः सव॒नमे॒व तेना᳚प्नो - [  ] \newline

\textbf{Pada Paata} \newline

वी॒र्य᳚म् । वृ॒ङ्क्ते॒ । पु॒रा । वा॒चः । प्रव॑दितो॒रिति॒ प्र - व॒दि॒तोः॒ । निरिति॑ । व॒पे॒त् । याव॑ती । ए॒व । वाक् । ताम् । अप्रो॑दिता॒मित्यप्र॑ - उ॒दि॒ता॒म् । भ्रातृ॑व्यस्य । वृ॒ङ्क्ते॒ । ताम् । अ॒स्य॒ । वाच᳚म् । प्र॒वद॑न्ती॒मिति॑ प्र - वद॑न्तीम् । अ॒न्याः । वाचः॑ । अनु॑ । प्रेति॑ । व॒द॒न्ति॒ । ताः । इ॒न्द्रि॒यम् । वी॒र्य᳚म् । यज॑माने । द॒ध॒ति॒ । आ॒ग्ना॒वै॒ष्ण॒वमित्या᳚ग्ना - वै॒ष्ण॒वम् । अ॒ष्टाक॑पाल॒मित्य॒ष्टा-क॒पा॒ल॒म् । निरिति॑ । व॒पे॒त् । प्रा॒तः॒ स॒व॒नस्येति॑ प्रातः - स॒व॒नस्य॑ । आ॒का॒ल इत्या᳚ - का॒ले । सर॑स्वती । आज्य॑भा॒गेत्याज्य॑ - भा॒गा॒ । स्यात् । बा॒र्.॒ह॒स्प॒त्यः । च॒रुः । यत् । अ॒ष्टाक॑पाल॒ इत्य॒ष्टा - क॒पा॒लः॒ । भव॑ति । अ॒ष्टाक्ष॒रेत्य॒ष्टा- अ॒क्ष॒रा॒ । गा॒य॒त्री । गा॒य॒त्रम् । प्रा॒तः॒ स॒व॒नमिति॑ प्रातः - स॒व॒नम् । प्रा॒तः॒ स॒व॒नमिति॑ प्रातः - स॒व॒नम् । ए॒व । तेन॑ । आ॒प्नो॒ति॒ ।  \newline


\textbf{Krama Paata} \newline

वी॒र्यं॑ ॅवृङ्क्ते । वृ॒ङ्क्ते॒ पु॒रा । पु॒रा वा॒चः । वा॒चः प्रव॑दितोः । प्रव॑दितो॒र् निः । प्रव॑दितो॒रिति॒ प्र - व॒दि॒तोः॒ । निर् व॑पेत् । व॒पे॒द्याव॑ती । याव॑त्ये॒व । ए॒व वाक् । वाक् ताम् । तामप्रो॑दिताम् । अप्रो॑दिता॒म् भ्रातृ॑व्यस्य । अप्रो॑दिता॒मित्यप्र॑ - उ॒दि॒ता॒म् । भ्रातृ॑व्यस्य वृङ्क्ते । वृ॒ङ्क्ते॒ ताम् । ताम॑स्य । अ॒स्य॒ वाच᳚म् । वाच॑म् प्र॒वद॑न्तीम् । प्र॒वद॑न्तीम॒न्याः । प्र॒वद॑न्ती॒मिति॑ प्र - वद॑न्तीम् । अ॒न्या वाचः॑ । वाचोऽनु॑ । अनु॒ प्र । प्र व॑दन्ति । व॒द॒न्ति॒ ताः । ता इ॑न्द्रि॒यम् । इ॒न्द्रि॒यं ॅवी॒र्य᳚म् । वी॒र्यं॑ ॅयज॑माने । यज॑माने दधति । द॒ध॒त्या॒ग्ना॒वै॒ष्ण॒वम् । आ॒ग्ना॒वै॒ष्ण॒वम॒ष्टाक॑पालम् । आ॒ग्ना॒वै॒ष्ण॒वमित्या᳚ग्ना - वै॒ष्ण॒वम् । अ॒ष्टाक॑पाल॒म् निः । अ॒ष्टाक॑पाल॒मित्य॒ष्टा - क॒पा॒ल॒म् । निर् व॑पेत् । व॒पे॒त् प्रा॒त॒स्स॒व॒नस्य॑ । प्रा॒त॒स्स॒व॒नस्या॑का॒ले । प्रा॒त॒स्स॒व॒नस्येति॑ प्रातः - स॒व॒नस्य॑ । आ॒का॒ले सर॑स्वती । आ॒का॒ल इत्या᳚ - का॒ले । सर॑स्व॒त्याज्य॑भागा । आज्य॑भागा॒ स्यात् । आज्य॑भा॒गेत्याज्य॑ - भा॒गा॒ । स्याद्,बा॑र्.हस्प॒त्यः । बा॒र्॒.ह॒स्प॒त्य श्च॒रुः । च॒रुर् यत् । यद॒ष्टाक॑पालः । अ॒ष्टाक॑पालो॒ भव॑ति । अ॒ष्टाक॑पाल॒ इत्य॒ष्टा - क॒पा॒लः॒ । भव॑त्य॒ष्ठाक्ष॑रा । अ॒ष्टाक्ष॑रा गाय॒त्री । अ॒ष्टाक्ष॒रेत्य॒ष्टा - अ॒क्ष॒रा॒ । गा॒य॒त्री गा॑य॒त्रम् । गा॒य॒त्रम् प्रा॑तस्सव॒नम् । प्रा॒त॒स्स॒व॒नम् प्रा॑तस्सव॒नम् । प्रा॒त॒स्स॒व॒नमिति॑ प्रातः - स॒व॒नम् । प्रा॒त॒स्स॒व॒नमे॒व । प्रा॒त॒स्स॒व॒नमिति॑ प्रातः - स॒व॒नम् । ए॒व तेन॑ । तेना᳚प्नोति । आ॒प्नो॒त्या॒ग्ना॒वै॒ष्ण॒वम् \newline

\textbf{Jatai Paata} \newline

1. वी॒र्यं॑ ॅवृङ्क्ते वृङ्क्ते वी॒र्यं॑ ॅवी॒र्यं॑ ॅवृङ्क्ते । \newline
2. वृ॒ङ्क्ते॒ पु॒रा पु॒रा वृ॑ङ्क्ते वृङ्क्ते पु॒रा । \newline
3. पु॒रा वा॒चो वा॒चः पु॒रा पु॒रा वा॒चः । \newline
4. वा॒चः प्रव॑दितोः॒ प्रव॑दितोर् वा॒चो वा॒चः प्रव॑दितोः । \newline
5. प्रव॑दितो॒र् निर् णिष् प्रव॑दितोः॒ प्रव॑दितो॒र् निः । \newline
6. प्रव॑दितो॒रिति॒ प्र - व॒दि॒तोः॒ । \newline
7. निर् व॑पेद् वपे॒न् निर् णिर् व॑पेत् । \newline
8. व॒पे॒द् याव॑ती॒ याव॑ती वपेद् वपे॒द् याव॑ती । \newline
9. याव॑ त्ये॒वैव याव॑ती॒ याव॑ त्ये॒व । \newline
10. ए॒व वाग् वागे॒वैव वाक् । \newline
11. वाक् ताम् तां ॅवाग् वाक् ताम् । \newline
12. ता मप्रो॑दिता॒ मप्रो॑दिता॒म् ताम् ता मप्रो॑दिताम् । \newline
13. अप्रो॑दिता॒म् भ्रातृ॑व्यस्य॒ भ्रातृ॑व्य॒स्या प्रो॑दिता॒ मप्रो॑दिता॒म् भ्रातृ॑व्यस्य । \newline
14. अप्रो॑दिता॒मित्यप्र॑ - उ॒दि॒ता॒म् । \newline
15. भ्रातृ॑व्यस्य वृङ्क्ते वृङ्क्ते॒ भ्रातृ॑व्यस्य॒ भ्रातृ॑व्यस्य वृङ्क्ते । \newline
16. वृ॒ङ्क्ते॒ ताम् तां ॅवृ॑ङ्क्ते वृङ्क्ते॒ ताम् । \newline
17. ता म॑स्यास्य॒ ताम् ता म॑स्य । \newline
18. अ॒स्य॒ वाचं॒ ॅवाच॑ मस्यास्य॒ वाच᳚म् । \newline
19. वाच॑म् प्र॒वद॑न्तीम् प्र॒वद॑न्तीं॒ ॅवाचं॒ ॅवाच॑म् प्र॒वद॑न्तीम् । \newline
20. प्र॒वद॑न्ती म॒न्या अ॒न्याः प्र॒वद॑न्तीम् प्र॒वद॑न्ती म॒न्याः । \newline
21. प्र॒वद॑न्ती॒मिति॑ प्र - वद॑न्तीम् । \newline
22. अ॒न्या वाचो॒ वाचो॒ ऽन्या अ॒न्या वाचः॑ । \newline
23. वाचो ऽन्वनु॒ वाचो॒ वाचो ऽनु॑ । \newline
24. अनु॒ प्र प्राण्वनु॒ प्र । \newline
25. प्र व॑दन्ति वदन्ति॒ प्र प्र व॑दन्ति । \newline
26. व॒द॒न्ति॒ तास्ता व॑दन्ति वदन्ति॒ ताः । \newline
27. ता इ॑न्द्रि॒य मि॑न्द्रि॒यम् तास्ता इ॑न्द्रि॒यम् । \newline
28. इ॒न्द्रि॒यं ॅवी॒र्यं॑ ॅवी॒र्य॑ मिन्द्रि॒य मि॑न्द्रि॒यं ॅवी॒र्य᳚म् । \newline
29. वी॒र्यं॑ ॅयज॑माने॒ यज॑माने वी॒र्यं॑ ॅवी॒र्यं॑ ॅयज॑माने । \newline
30. यज॑माने दधति दधति॒ यज॑माने॒ यज॑माने दधति । \newline
31. द॒ध॒ त्या॒ग्ना॒वै॒ष्ण॒व मा᳚ग्नावैष्ण॒वम् द॑धति दध त्याग्नावैष्ण॒वम् । \newline
32. आ॒ग्ना॒वै॒ष्ण॒व म॒ष्टाक॑पाल म॒ष्टाक॑पाल माग्नावैष्ण॒व मा᳚ग्नावैष्ण॒व म॒ष्टाक॑पालम् । \newline
33. आ॒ग्ना॒वै॒ष्ण॒वमित्या᳚ग्ना - वै॒ष्ण॒वम् । \newline
34. अ॒ष्टाक॑पाल॒म् निर् णिर॒ष्टाक॑पाल म॒ष्टाक॑पाल॒म् निः । \newline
35. अ॒ष्टाक॑पाल॒मित्य॒ष्टा - क॒पा॒ल॒म् । \newline
36. निर् व॑पेद् वपे॒न् निर् णिर् व॑पेत् । \newline
37. व॒पे॒त् प्रा॒त॒स्स॒व॒नस्य॑ प्रातस्सव॒नस्य॑ वपेद् वपेत् प्रातस्सव॒नस्य॑ । \newline
38. प्रा॒त॒स्स॒व॒नस्या॑ का॒ल आ॑का॒ले प्रा॑तस्सव॒नस्य॑ प्रातस्सव॒नस्या॑ का॒ले । \newline
39. प्रा॒त॒स्स॒व॒नस्येति॑ प्रातः - स॒व॒नस्य॑ । \newline
40. आ॒का॒ले सर॑स्वती॒ सर॑स्व त्याका॒ल आ॑का॒ले सर॑स्वती । \newline
41. आ॒का॒ल इत्या᳚ - का॒ले । \newline
42. सर॑स्व॒ त्याज्य॑भा॒गा ऽऽज्य॑भागा॒ सर॑स्वती॒ सर॑स्व॒ त्याज्य॑भागा । \newline
43. आज्य॑भागा॒ स्याथ् स्यादाज्य॑भा॒गा ऽऽज्य॑भागा॒ स्यात् । \newline
44. आज्य॑भा॒गेत्याज्य॑ - भा॒गा॒ । \newline
45. स्याद् बा॑र्.हस्प॒त्यो बा॑र्.हस्प॒त्यः स्याथ् स्याद् बा॑र्.हस्प॒त्यः । \newline
46. बा॒र्॒.ह॒स्प॒त्य श्च॒रु श्च॒रुर् बा॑र्.हस्प॒त्यो बा॑र्.हस्प॒त्य श्च॒रुः । \newline
47. च॒रुर् यद् यच् च॒रु श्च॒रुर् यत् । \newline
48. यद॒ष्टाक॑पालो॒ ऽष्टाक॑पालो॒ यद् यद॒ष्टाक॑पालः । \newline
49. अ॒ष्टाक॑पालो॒ भव॑ति॒ भव॑त्य॒ष्टाक॑पालो॒ ऽष्टाक॑पालो॒ भव॑ति । \newline
50. अ॒ष्टाक॑पाल॒ इत्य॒ष्टा - क॒पा॒लः॒ । \newline
51. भव॑ त्य॒ष्टाक्ष॑रा॒ ऽष्टाक्ष॑रा॒ भव॑ति॒ भव॑ त्य॒ष्टाक्ष॑रा । \newline
52. अ॒ष्टाक्ष॑रा गाय॒त्री गा॑य॒त्र्य॑ष्टाक्ष॑रा॒ ऽष्टाक्ष॑रा गाय॒त्री । \newline
53. अ॒ष्टाक्ष॒रेत्य॒ष्टा - अ॒क्ष॒रा॒ । \newline
54. गा॒य॒त्री गा॑य॒त्रम् गा॑य॒त्रम् गा॑य॒त्री गा॑य॒त्री गा॑य॒त्रम् । \newline
55. गा॒य॒त्रम् प्रा॑तस्सव॒नम् प्रा॑तस्सव॒नम् गा॑य॒त्रम् गा॑य॒त्रम् प्रा॑तस्सव॒नम् । \newline
56. प्रा॒त॒स्स॒व॒नम् प्रा॑तस्सव॒नम् । \newline
57. प्रा॒त॒स्स॒व॒नमिति॑ प्रातः - स॒व॒नम् । \newline
58. प्रा॒त॒स्स॒व॒न मे॒वैव प्रा॑तस्सव॒नम् प्रा॑तस्सव॒न मे॒व । \newline
59. प्रा॒त॒स्स॒व॒नमिति॑ प्रातः - स॒व॒नम् । \newline
60. ए॒व तेन॒ तेनै॒वैव तेन॑ । \newline
61. तेना᳚प्नो त्याप्नोति॒ तेन॒ तेना᳚प्नोति । \newline
62. आ॒प्नो॒ त्या॒ग्ना॒वै॒ष्ण॒व मा᳚ग्नावैष्ण॒व मा᳚प्नो त्याप्नो त्याग्नावैष्ण॒वम् । \newline

\textbf{Ghana Paata } \newline

1. वी॒र्यं॑ ॅवृङ्क्ते वृङ्क्ते वी॒र्यं॑ ॅवी॒र्यं॑ ॅवृङ्क्ते पु॒रा पु॒रा वृ॑ङ्क्ते वी॒र्यं॑ ॅवी॒र्यं॑ ॅवृङ्क्ते पु॒रा । \newline
2. वृ॒ङ्क्ते॒ पु॒रा पु॒रा वृ॑ङ्क्ते वृङ्क्ते पु॒रा वा॒चो वा॒चः पु॒रा वृ॑ङ्क्ते वृङ्क्ते पु॒रा वा॒चः । \newline
3. पु॒रा वा॒चो वा॒चः पु॒रा पु॒रा वा॒चः प्रव॑दितोः॒ प्रव॑दितोर् वा॒चः पु॒रा पु॒रा वा॒चः प्रव॑दितोः । \newline
4. वा॒चः प्रव॑दितोः॒ प्रव॑दितोर् वा॒चो वा॒चः प्रव॑दितो॒र् निर् णिष् प्रव॑दितोर् वा॒चो वा॒चः प्रव॑दितो॒र् निः । \newline
5. प्रव॑दितो॒र् निर् णिष् प्रव॑दितोः॒ प्रव॑दितो॒र् निर् व॑पेद् वपे॒न् निष् प्रव॑दितोः॒ प्रव॑दितो॒र् निर् व॑पेत् । \newline
6. प्रव॑दितो॒रिति॒ प्र - व॒दि॒तोः॒ । \newline
7. निर् व॑पेद् वपे॒न् निर् णिर् व॑पे॒द् याव॑ती॒ याव॑ती वपे॒न् निर् णिर् व॑पे॒द् याव॑ती । \newline
8. व॒पे॒द् याव॑ती॒ याव॑ती वपेद् वपे॒द् याव॑ त्ये॒वैव याव॑ती वपेद् वपे॒द् याव॑ त्ये॒व । \newline
9. याव॑ त्ये॒वैव याव॑ती॒ याव॑ त्ये॒व वाग् वागे॒व याव॑ती॒ याव॑ त्ये॒व वाक् । \newline
10. ए॒व वाग् वागे॒वैव वाक् ताम् तां ॅवागे॒वैव वाक् ताम् । \newline
11. वाक् ताम् तां ॅवाग् वाक् ता मप्रो॑दिता॒ मप्रो॑दिता॒म् तां ॅवाग् वाक् ता मप्रो॑दिताम् । \newline
12. ता मप्रो॑दिता॒ मप्रो॑दिता॒म् ताम् ता मप्रो॑दिता॒म् भ्रातृ॑व्यस्य॒ भ्रातृ॑व्य॒स्या प्रो॑दिता॒म् ताम् ता मप्रो॑दिता॒म् भ्रातृ॑व्यस्य । \newline
13. अप्रो॑दिता॒म् भ्रातृ॑व्यस्य॒ भ्रातृ॑व्य॒स्या प्रो॑दिता॒ मप्रो॑दिता॒म् भ्रातृ॑व्यस्य वृङ्क्ते वृङ्क्ते॒ भ्रातृ॑व्य॒स्या प्रो॑दिता॒ मप्रो॑दिता॒म् भ्रातृ॑व्यस्य वृङ्क्ते । \newline
14. अप्रो॑दिता॒मित्यप्र॑ - उ॒दि॒ता॒म् । \newline
15. भ्रातृ॑व्यस्य वृङ्क्ते वृङ्क्ते॒ भ्रातृ॑व्यस्य॒ भ्रातृ॑व्यस्य वृङ्क्ते॒ ताम् तां ॅवृ॑ङ्क्ते॒ भ्रातृ॑व्यस्य॒ भ्रातृ॑व्यस्य वृङ्क्ते॒ ताम् । \newline
16. वृ॒ङ्क्ते॒ ताम् तां ॅवृ॑ङ्क्ते वृङ्क्ते॒ ता म॑स्यास्य॒ तां ॅवृ॑ङ्क्ते वृङ्क्ते॒ ता म॑स्य । \newline
17. ता म॑स्यास्य॒ ताम् ता म॑स्य॒ वाचं॒ ॅवाच॑ मस्य॒ ताम् ता म॑स्य॒ वाच᳚म् । \newline
18. अ॒स्य॒ वाचं॒ ॅवाच॑ मस्यास्य॒ वाच॑म् प्र॒वद॑न्तीम् प्र॒वद॑न्तीं॒ ॅवाच॑ मस्यास्य॒ वाच॑म् प्र॒वद॑न्तीम् । \newline
19. वाच॑म् प्र॒वद॑न्तीम् प्र॒वद॑न्तीं॒ ॅवाचं॒ ॅवाच॑म् प्र॒वद॑न्ती म॒न्या अ॒न्याः प्र॒वद॑न्तीं॒ ॅवाचं॒ ॅवाच॑म् प्र॒वद॑न्ती म॒न्याः । \newline
20. प्र॒वद॑न्ती म॒न्या अ॒न्याः प्र॒वद॑न्तीम् प्र॒वद॑न्ती म॒न्या वाचो॒ वाचो॒ ऽन्याः प्र॒वद॑न्तीम् प्र॒वद॑न्ती म॒न्या वाचः॑ । \newline
21. प्र॒वद॑न्ती॒मिति॑ प्र - वद॑न्तीम् । \newline
22. अ॒न्या वाचो॒ वाचो॒ ऽन्या अ॒न्या वाचो ऽन्वनु॒ वाचो॒ ऽन्या अ॒न्या वाचो ऽनु॑ । \newline
23. वाचो ऽन्वनु॒ वाचो॒ वाचो ऽनु॒ प्र प्राणु॒ वाचो॒ वाचो ऽनु॒ प्र । \newline
24. अनु॒ प्र प्राण्वनु॒ प्र व॑दन्ति वदन्ति॒ प्राण्वनु॒ प्र व॑दन्ति । \newline
25. प्र व॑दन्ति वदन्ति॒ प्र प्र व॑दन्ति॒ तास्ता व॑दन्ति॒ प्र प्र व॑दन्ति॒ ताः । \newline
26. व॒द॒न्ति॒ ता स्ता व॑दन्ति वदन्ति॒ ता इ॑न्द्रि॒य मि॑न्द्रि॒यम् ता व॑दन्ति वदन्ति॒ ता इ॑न्द्रि॒यम् । \newline
27. ता इ॑न्द्रि॒य मि॑न्द्रि॒यम् ता स्ता इ॑न्द्रि॒यं ॅवी॒र्यं॑ ॅवी॒र्य॑ मिन्द्रि॒यम् तास्ता इ॑न्द्रि॒यं ॅवी॒र्य᳚म् । \newline
28. इ॒न्द्रि॒यं ॅवी॒र्यं॑ ॅवी॒र्य॑ मिन्द्रि॒य मि॑न्द्रि॒यं ॅवी॒र्यं॑ ॅयज॑माने॒ यज॑माने वी॒र्य॑ मिन्द्रि॒य मि॑न्द्रि॒यं ॅवी॒र्यं॑ ॅयज॑माने । \newline
29. वी॒र्यं॑ ॅयज॑माने॒ यज॑माने वी॒र्यं॑ ॅवी॒र्यं॑ ॅयज॑माने दधति दधति॒ यज॑माने वी॒र्यं॑ ॅवी॒र्यं॑ ॅयज॑माने दधति । \newline
30. यज॑माने दधति दधति॒ यज॑माने॒ यज॑माने दध त्याग्नावैष्ण॒व मा᳚ग्नावैष्ण॒वम् द॑धति॒ यज॑माने॒ यज॑माने दध त्याग्नावैष्ण॒वम् । \newline
31. द॒ध॒ त्या॒ग्ना॒वै॒ष्ण॒व मा᳚ग्नावैष्ण॒वम् द॑धति दध त्याग्नावैष्ण॒व म॒ष्टाक॑पाल म॒ष्टाक॑पाल माग्नावैष्ण॒वम् द॑धति दध त्याग्नावैष्ण॒व म॒ष्टाक॑पालम् । \newline
32. आ॒ग्ना॒वै॒ष्ण॒व म॒ष्टाक॑पाल म॒ष्टाक॑पाल माग्नावैष्ण॒व मा᳚ग्नावैष्ण॒व म॒ष्टाक॑पाल॒म् निर् णिर॒ष्टाक॑पाल माग्नावैष्ण॒व मा᳚ग्नावैष्ण॒व म॒ष्टाक॑पाल॒म् निः । \newline
33. आ॒ग्ना॒वै॒ष्ण॒वमित्या᳚ग्ना - वै॒ष्ण॒वम् । \newline
34. अ॒ष्टाक॑पाल॒म् निर् णिर॒ष्टाक॑पाल म॒ष्टाक॑पाल॒म् निर् व॑पेद् वपे॒न् निर॒ष्टाक॑पाल म॒ष्टाक॑पाल॒म् निर् व॑पेत् । \newline
35. अ॒ष्टाक॑पाल॒मित्य॒ष्टा - क॒पा॒ल॒म् । \newline
36. निर् व॑पेद् वपे॒न् निर् णिर् व॑पेत् प्रातस्सव॒नस्य॑ प्रातस्सव॒नस्य॑ वपे॒न् निर् णिर् व॑पेत् प्रातस्सव॒नस्य॑ । \newline
37. व॒पे॒त् प्रा॒त॒स्स॒व॒नस्य॑ प्रातस्सव॒नस्य॑ वपेद् वपेत् प्रातस्सव॒नस्या॑का॒ल आ॑का॒ले प्रा॑तस्सव॒नस्य॑ वपेद् वपेत् प्रातस्सव॒नस्या॑का॒ले । \newline
38. प्रा॒त॒स्स॒व॒नस्या॑का॒ल आ॑का॒ले प्रा॑तस्सव॒नस्य॑ प्रातस्सव॒नस्या॑का॒ले सर॑स्वती॒ सर॑स्वत्याका॒ले प्रा॑तस्सव॒नस्य॑ प्रातस्सव॒नस्या॑का॒ले सर॑स्वती । \newline
39. प्रा॒त॒स्स॒व॒नस्येति॑ प्रातः - स॒व॒नस्य॑ । \newline
40. आ॒का॒ले सर॑स्वती॒ सर॑स्व त्याका॒ल आ॑का॒ले सर॑स्व॒ त्याज्य॑भा॒गा ऽऽज्य॑भागा॒ सर॑स्व त्याका॒ल आ॑का॒ले सर॑स्व॒ त्याज्य॑भागा । \newline
41. आ॒का॒ल इत्या᳚ - का॒ले । \newline
42. सर॑स्व॒ त्याज्य॑भा॒गा ऽऽज्य॑भागा॒ सर॑स्वती॒ सर॑स्व॒ त्याज्य॑भागा॒ स्याथ् स्या दाज्य॑भागा॒ सर॑स्वती॒ सर॑स्व॒ त्याज्य॑भागा॒ स्यात् । \newline
43. आज्य॑भागा॒ स्याथ् स्या दाज्य॑भा॒गा ऽऽज्य॑भागा॒ स्याद् बा॑र्.हस्प॒त्यो बा॑र्.हस्प॒त्यः स्या दाज्य॑भा॒गा ऽऽज्य॑भागा॒ स्याद् बा॑र्.हस्प॒त्यः । \newline
44. आज्य॑भा॒गेत्याज्य॑ - भा॒गा॒ । \newline
45. स्याद् बा॑र्.हस्प॒त्यो बा॑र्.हस्प॒त्यः स्याथ् स्याद् बा॑र्.हस्प॒त्य श्च॒रु श्च॒रुर् बा॑र्.हस्प॒त्यः स्याथ् स्याद् बा॑र्.हस्प॒त्य श्च॒रुः । \newline
46. बा॒र्॒.ह॒स्प॒त्य श्च॒रु श्च॒रुर् बा॑र्.हस्प॒त्यो बा॑र्.हस्प॒त्य श्च॒रुर् यद् यच् च॒रुर् बा॑र्.हस्प॒त्यो बा॑र्.हस्प॒त्य श्च॒रुर् यत् । \newline
47. च॒रुर् यद् यच् च॒रु श्च॒रुर् यद॒ष्टाक॑पालो॒ ऽष्टाक॑पालो॒ यच् च॒रु श्च॒रुर् यद॒ष्टाक॑पालः । \newline
48. यद॒ष्टाक॑पालो॒ ऽष्टाक॑पालो॒ यद् यद॒ष्टाक॑पालो॒ भव॑ति॒ भव॑ त्य॒ष्टाक॑पालो॒ यद् यद॒ष्टाक॑पालो॒ भव॑ति । \newline
49. अ॒ष्टाक॑पालो॒ भव॑ति॒ भव॑ त्य॒ष्टाक॑पालो॒ ऽष्टाक॑पालो॒ भव॑ त्य॒ष्टाक्ष॑रा॒ ऽष्टाक्ष॑रा॒ भव॑ त्य॒ष्टाक॑पालो॒ ऽष्टाक॑पालो॒ भव॑ त्य॒ष्टाक्ष॑रा । \newline
50. अ॒ष्टाक॑पाल॒ इत्य॒ष्टा - क॒पा॒लः॒ । \newline
51. भव॑ त्य॒ष्टाक्ष॑रा॒ ऽष्टाक्ष॑रा॒ भव॑ति॒ भव॑ त्य॒ष्टाक्ष॑रा गाय॒त्री गा॑य॒ त्र्य॑ष्टाक्ष॑रा॒ भव॑ति॒ भव॑ त्य॒ष्टाक्ष॑रा गाय॒त्री । \newline
52. अ॒ष्टाक्ष॑रा गाय॒त्री गा॑य॒ त्र्य॑ष्टाक्ष॑रा॒ ऽष्टाक्ष॑रा गाय॒त्री गा॑य॒त्रम् गा॑य॒त्रम् गा॑य॒ त्र्य॑ष्टाक्ष॑रा॒ ऽष्टाक्ष॑रा गाय॒त्री गा॑य॒त्रम् । \newline
53. अ॒ष्टाक्ष॒रेत्य॒ष्टा - अ॒क्ष॒रा॒ । \newline
54. गा॒य॒त्री गा॑य॒त्रम् गा॑य॒त्रम् गा॑य॒त्री गा॑य॒त्री गा॑य॒त्रम् प्रा॑तस्सव॒नम् प्रा॑तस्सव॒नम् गा॑य॒त्रम् गा॑य॒त्री गा॑य॒त्री गा॑य॒त्रम् प्रा॑तस्सव॒नम् । \newline
55. गा॒य॒त्रम् प्रा॑तस्सव॒नम् प्रा॑तस्सव॒नम् गा॑य॒त्रम् गा॑य॒त्रम् प्रा॑तस्सव॒नम् । \newline
56. प्रा॒त॒स्स॒व॒नम् प्रा॑तस्सव॒नम् । \newline
57. प्रा॒त॒स्स॒व॒नमिति॑ प्रातः - स॒व॒नम् । \newline
58. प्रा॒त॒स्स॒व॒न मे॒वैव प्रा॑तस्सव॒नम् प्रा॑तस्सव॒न मे॒व तेन॒ तेनै॒व प्रा॑तस्सव॒नम् प्रा॑तस्सव॒न मे॒व तेन॑ । \newline
59. प्रा॒त॒स्स॒व॒नमिति॑ प्रातः - स॒व॒नम् । \newline
60. ए॒व तेन॒ तेनै॒वैव तेना᳚प्नो त्याप्नोति॒ तेनै॒वैव तेना᳚प्नोति । \newline
61. तेना᳚प्नो त्याप्नोति॒ तेन॒ तेना᳚प्नो त्याग्नावैष्ण॒व मा᳚ग्नावैष्ण॒व मा᳚प्नोति॒ तेन॒ तेना᳚प्नो त्याग्नावैष्ण॒वम् । \newline
62. आ॒प्नो॒ त्या॒ग्ना॒वै॒ष्ण॒व मा᳚ग्नावैष्ण॒व मा᳚प्नो त्याप्नो त्याग्नावैष्ण॒व मेका॑दशकपाल॒ मेका॑दशकपाल माग्नावैष्ण॒व मा᳚प्नो त्याप्नो त्याग्नावैष्ण॒व मेका॑दशकपालम् । \newline
\pagebreak
\markright{ TS 2.2.9.6  \hfill https://www.vedavms.in \hfill}
\addcontentsline{toc}{section}{ TS 2.2.9.6 }
\section*{ TS 2.2.9.6 }

\textbf{TS 2.2.9.6 } \newline
\textbf{Samhita Paata} \newline

-त्याग्नावैष्ण॒व-मेका॑दशकपालं॒ निर्व॑पे॒न्माद्ध्य॑न्दिनस्य॒ सव॑नस्याऽऽ *का॒ले सर॑स्व॒त्याज्य॑भागा॒ स्याद् बा॑र्.हस्प॒त्य-श्च॒रु र्यदेका॑दशकपालो॒ भव॒त्येका॑दशाक्षरा त्रि॒ष्टुप् त्रैष्टु॑भं॒ माद्ध्य॑न्दिनꣳ॒॒ सव॑नं॒ माद्ध्य॑दिंनमे॒व सव॑नं॒ तेना᳚ऽऽ*प्नोत्याग्नावैष्ण॒वं द्वाद॑शकपालं॒ निर्व॑पेत् तृतीयसव॒नस्या॑ऽऽ*का॒ले सर॑स्व॒त्याज्य॑भागा॒ स्याद् बा॑र्.हस्प॒त्यश्च॒रुर्यद् द्वाद॑शकपालो॒ भव॑ति॒ द्वाद॑शाक्षरा॒ जग॑ती॒ जाग॑तं तृतीयसव॒नं तृ॑तीय सव॒नमे॒व तेना᳚ऽऽप्नोति दे॒वता॑भिरे॒व दे॒वताः᳚ - [  ] \newline

\textbf{Pada Paata} \newline

आ॒ग्ना॒वै॒ष्ण॒वमित्या᳚ग्ना - वै॒ष्ण॒वम् । एका॑दशकपाल॒मित्येका॑दश - क॒पा॒ल॒म् । निरिति॑ । व॒पे॒त् । माद्ध्य॑न्दिनस्य । सव॑नस्य । आ॒का॒ल इत्या᳚ - का॒ले । सर॑स्वती । आज्य॑भा॒गेत्याज्य॑ - भा॒गा॒ । स्यात् । बा॒र्.॒ह॒स्प॒त्यः । च॒रुः । यत् । एका॑दशकपाल॒ इत्येका॑दश - क॒पा॒लः॒ । भव॑ति । एका॑दशाक्ष॒रेत्येका॑दश - अ॒क्ष॒रा॒ । त्रि॒ष्टुप् । त्रैष्टु॑भम् । माद्ध्य॑न्दिनम् । सव॑नम् । माद्ध्य॑न्दिनम् । ए॒व । सव॑नम् । तेन॑ । आ॒प्नो॒ति॒ । आ॒ग्ना॒वै॒ष्ण॒वमित्या᳚ग्ना - वै॒ष्ण॒वम् । द्वाद॑शकपाल॒मिति॒ द्वाद॑श - क॒पा॒ल॒म् । निरिति॑ । व॒पे॒त् । तृ॒ती॒य॒स॒व॒नस्येति॑ तृतीय- स॒व॒नस्य॑ । आ॒का॒ल इत्या᳚ - का॒ले । सर॑स्वती । आज्य॑भा॒गेत्याज्य॑ - भा॒गा॒ । स्यात् । बा॒र्.॒ह॒स्प॒त्यः । च॒रुः । यत् । द्वाद॑शकपाल॒ इति॒ द्वाद॑श - क॒पा॒लः॒ । भव॑ति । द्वाद॑शाक्ष॒रेति॒ द्वाद॑श - अ॒क्ष॒रा॒ । जग॑ती । जाग॑तम् । तृ॒ती॒य॒स॒व॒नमिति॑ तृतीय - स॒व॒नम् । तृ॒ती॒य॒स॒व॒नमिति॑ तृतीय -स॒व॒नम् । ए॒व । तेन॑ । आ॒प्नो॒ति॒ । दे॒वता॑भिः । ए॒व । दे॒॒वताः᳚ ।  \newline


\textbf{Krama Paata} \newline

आ॒ग्ना॒वै॒ष्ण॒वमेका॑दशकपालम् । आ॒ग्ना॒वै॒ष्ण॒वमित्या᳚ग्ना - वै॒ष्ण॒वम् । एका॑दशकपाल॒म् निः । एका॑दशकपाल॒मित्येका॑दश - क॒पा॒ल॒म् । निर् व॑पेत् । व॒पे॒न् माद्ध्य॑न्दिनस्य । माद्ध्य॑न्दिनस्य॒ सव॑नस्य । सव॑नस्याका॒ले । आ॒का॒ले सर॑स्वती । आ॒का॒ल इत्या᳚ - का॒ले । सर॑स्व॒त्याज्य॑भागा । आज्य॑भागा॒ स्यात् । आज्य॑भा॒गेत्याज्य॑ - भा॒गा॒ । स्याद्,बा॑र्.हस्प॒त्यः । बा॒र्॒.ह॒स्प॒त्य श्च॒रुः । च॒रुर् यत् । यदेका॑दशकपालः । एका॑दशकपालो॒ भव॑ति । एका॑दशकपाल॒ इत्येका॑दश - क॒पा॒लः॒ । भव॒त्येका॑दशाक्षरा । एका॑दशाक्षरा त्रि॒ष्टुप् । एका॑दशाक्ष॒रेत्येका॑दश - अ॒क्ष॒रा॒ । त्रि॒ष्टुप् त्रैष्टु॑भम् । त्रैष्टु॑भ॒म् माद्ध्य॑न्दिनम् । माद्ध्य॑न्दिनꣳ॒॒ सव॑नम् । सव॑न॒म् माद्ध्य॑न्दिनम् । माद्ध्य॑न्दिनमे॒व । ए॒व सव॑नम् । सव॑न॒म् तेन॑ । तेना᳚प्नोति । आ॒प्नो॒त्या॒ग्ना॒वै॒ष्ण॒वम् । आ॒ग्ना॒वै॒ष्ण॒वम् द्वाद॑शकपालम् । आ॒गा॒वै॒ष्ण॒वमित्या᳚ग्ना - वै॒ष्ण॒वम् । द्वाद॑शकपाल॒म् निः । द्वाद॑शकपाल॒मिति॒ द्वाद॑श - क॒पा॒ल॒म् । निर् व॑पेत् । व॒पे॒त्,तृ॒ती॒य॒स॒व॒नस्य॑ । तृ॒ती॒य॒स॒व॒नस्या॑का॒ले । तृ॒ती॒य॒स॒व॒नस्येति॑ तृतीय - स॒व॒नस्य॑ । आ॒का॒ले सर॑स्वती । आ॒का॒ल इत्या᳚ - का॒ले । सर॑स्व॒त्याज्य॑भागा । आज्य॑भागा॒ स्यात् । आज्य॑भा॒गेत्याज्य॑ - भा॒गा॒ । स्याद्,बा॑र्.हस्प॒त्यः । बा॒र्॒.ह॒स्प॒त्य श्च॒रुः । च॒रुर् यत् । यद् द्वाद॑शकपालः । द्वाद॑शकपालो॒ भव॑ति । द्वाद॑शकपाल॒ इति॒ द्वाद॑श - क॒पा॒लः॒ । भव॑ति॒ द्वाद॑शाक्षरा । द्वाद॑शाक्षरा॒ जग॑ती । द्वाद॑शाक्ष॒रेति॒ द्वाद॑श - अ॒क्ष॒रा॒ । जग॑ती॒ जाग॑तम् । जाग॑तं तृतीयसव॒नम् । तृ॒ती॒य॒स॒व॒नम् तृ॑तीयसव॒नम् । तृ॒ती॒य॒स॒व॒नमिति॑ तृतीय - स॒व॒नम् । तृ॒ती॒य॒स॒व॒नमे॒व । तृ॒ती॒य॒स॒व॒नमिति॑ तृतीय - स॒व॒नम् । ए॒व तेन॑ । तेना᳚प्नोति । आ॒प्नो॒ति॒ दे॒वता॑भिः ( ) । दे॒वता॑भिरे॒व । ए॒व दे॒वताः᳚ । दे॒वताः᳚ प्रति॒चर॑ति \newline

\textbf{Jatai Paata} \newline

1. आ॒ग्ना॒वै॒ष्ण॒व मेका॑दशकपाल॒ मेका॑दशकपाल माग्नावैष्ण॒व मा᳚ग्नावैष्ण॒व मेका॑दशकपालम् । \newline
2. आ॒ग्ना॒वै॒ष्ण॒वमित्या᳚ग्ना - वै॒ष्ण॒वम् । \newline
3. एका॑दशकपाल॒म् निर् णिरेका॑दशकपाल॒ मेका॑दशकपाल॒म् निः । \newline
4. एका॑दशकपाल॒मित्येका॑दश - क॒पा॒ल॒म् । \newline
5. निर् व॑पेद् वपे॒न् निर् णिर् व॑पेत् । \newline
6. व॒पे॒न् माद्ध्य॑न्दिनस्य॒ माद्ध्य॑न्दिनस्य वपेद् वपे॒न् माद्ध्य॑न्दिनस्य । \newline
7. माद्ध्य॑न्दिनस्य॒ सव॑नस्य॒ सव॑नस्य॒ माद्ध्य॑न्दिनस्य॒ माद्ध्य॑न्दिनस्य॒ सव॑नस्य । \newline
8. सव॑नस्याका॒ल आ॑का॒ले सव॑नस्य॒ सव॑नस्याका॒ले । \newline
9. आ॒का॒ले सर॑स्वती॒ सर॑स्वत्याका॒ल आ॑का॒ले सर॑स्वती । \newline
10. आ॒का॒ल इत्या᳚ - का॒ले । \newline
11. सर॑स्व॒ त्याज्य॑भा॒गा ऽऽज्य॑भागा॒ सर॑स्वती॒ सर॑स्व॒ त्याज्य॑भागा । \newline
12. आज्य॑भागा॒ स्याथ् स्यादाज्य॑भा॒गा ऽऽज्य॑भागा॒ स्यात् । \newline
13. आज्य॑भा॒गेत्याज्य॑ - भा॒गा॒ । \newline
14. स्याद् बा॑र्.हस्प॒त्यो बा॑र्.हस्प॒त्यः स्याथ् स्याद् बा॑र्.हस्प॒त्यः । \newline
15. बा॒र्॒.ह॒स्प॒त्य श्च॒रु श्च॒रुर् बा॑र्.हस्प॒त्यो बा॑र्.हस्प॒त्य श्च॒रुः । \newline
16. च॒रुर् यद् यच् च॒रु श्च॒रुर् यत् । \newline
17. यदेका॑दशकपाल॒ एका॑दशकपालो॒ यद् यदेका॑दशकपालः । \newline
18. एका॑दशकपालो॒ भव॑ति॒ भव॒त्येका॑दशकपाल॒ एका॑दशकपालो॒ भव॑ति । \newline
19. एका॑दशकपाल॒ इत्येका॑दश - क॒पा॒लः॒ । \newline
20. भव॒ त्येका॑दशाक्ष॒ रैका॑दशाक्षरा॒ भव॑ति॒ भव॒ त्येका॑दशाक्षरा । \newline
21. एका॑दशाक्षरा त्रि॒ष्टुप् त्रि॒ष्टु बेका॑दशाक्ष॒ रैका॑दशाक्षरा त्रि॒ष्टुप् । \newline
22. एका॑दशाक्ष॒रेत्येका॑दश - अ॒क्ष॒रा॒ । \newline
23. त्रि॒ष्टुप् त्रैष्टु॑भ॒म् त्रैष्टु॑भम् त्रि॒ष्टुप् त्रि॒ष्टुप् त्रैष्टु॑भम् । \newline
24. त्रैष्टु॑भ॒म् माद्ध्य॑न्दिन॒म् माद्ध्य॑न्दिन॒म् त्रैष्टु॑भ॒म् त्रैष्टु॑भ॒म् माद्ध्य॑न्दिनम् । \newline
25. माद्ध्य॑न्दिनꣳ॒॒ सव॑नꣳ॒॒ सव॑न॒म् माद्ध्य॑न्दिन॒म् माद्ध्य॑न्दिनꣳ॒॒ सव॑नम् । \newline
26. सव॑न॒म् माद्ध्य॑न्दिन॒म् माद्ध्य॑न्दिनꣳ॒॒ सव॑नꣳ॒॒ सव॑न॒म् माद्ध्य॑न्दिनम् । \newline
27. माद्ध्य॑न्दिन मे॒वैव माद्ध्य॑न्दिन॒म् माद्ध्य॑न्दिन मे॒व । \newline
28. ए॒व सव॑नꣳ॒॒ सव॑न मे॒वैव सव॑नम् । \newline
29. सव॑न॒म् तेन॒ तेन॒ सव॑नꣳ॒॒ सव॑न॒म् तेन॑ । \newline
30. तेना᳚ प्नो त्याप्नोति॒ तेन॒ तेना᳚प्नोति । \newline
31. आ॒प्नो॒ त्या॒ग्ना॒वै॒ष्ण॒व मा᳚ग्नावैष्ण॒व मा᳚प्नो त्याप्नो त्याग्नावैष्ण॒वम् । \newline
32. आ॒ग्ना॒वै॒ष्ण॒वम् द्वाद॑शकपाल॒म् द्वाद॑शकपाल माग्नावैष्ण॒व मा᳚ग्नावैष्ण॒वम् द्वाद॑शकपालम् । \newline
33. आ॒ग्ना॒वै॒ष्ण॒वमित्या᳚ग्ना - वै॒ष्ण॒वम् । \newline
34. द्वाद॑शकपाल॒म् निर् णिर् द्वाद॑शकपाल॒म् द्वाद॑शकपाल॒म् निः । \newline
35. द्वाद॑शकपाल॒मिति॒ द्वाद॑श - क॒पा॒ल॒म् । \newline
36. निर् व॑पेद् वपे॒न् निर् णिर् व॑पेत् । \newline
37. व॒पे॒त् तृ॒ती॒य॒स॒व॒नस्य॑ तृतीयसव॒नस्य॑ वपेद् वपेत् तृतीयसव॒नस्य॑ । \newline
38. तृ॒ती॒य॒स॒व॒नस्या॑ का॒ल आ॑का॒ले तृ॑तीयसव॒नस्य॑ तृतीयसव॒नस्या॑ का॒ले । \newline
39. तृ॒ती॒य॒स॒व॒नस्येति॑ तृतीय - स॒व॒नस्य॑ । \newline
40. आ॒का॒ले सर॑स्वती॒ सर॑स्वत्या का॒ल आ॑का॒ले सर॑स्वती । \newline
41. आ॒का॒ल इत्या᳚ - का॒ले । \newline
42. सर॑स्व॒ त्याज्य॑भा॒गा ऽऽज्य॑भागा॒ सर॑स्वती॒ सर॑स्व॒ त्याज्य॑भागा । \newline
43. आज्य॑भागा॒ स्याथ् स्या दाज्य॑भा॒गा ऽऽज्य॑भागा॒ स्यात् । \newline
44. आज्य॑भा॒गेत्याज्य॑ - भा॒गा॒ । \newline
45. स्याद् बा॑र्.हस्प॒त्यो बा॑र्.हस्प॒त्यः स्याथ् स्याद् बा॑र्.हस्प॒त्यः । \newline
46. बा॒र्॒.ह॒स्प॒त्य श्च॒रु श्च॒रुर् बा॑र्.हस्प॒त्यो बा॑र्.हस्प॒त्य श्च॒रुः । \newline
47. च॒रुर् यद् यच् च॒रु श्च॒रुर् यत् । \newline
48. यद् द्वाद॑शकपालो॒ द्वाद॑शकपालो॒ यद् यद् द्वाद॑शकपालः । \newline
49. द्वाद॑शकपालो॒ भव॑ति॒ भव॑ति॒ द्वाद॑शकपालो॒ द्वाद॑शकपालो॒ भव॑ति । \newline
50. द्वाद॑शकपाल॒ इति॒ द्वाद॑श - क॒पा॒लः॒ । \newline
51. भव॑ति॒ द्वाद॑शाक्षरा॒ द्वाद॑शाक्षरा॒ भव॑ति॒ भव॑ति॒ द्वाद॑शाक्षरा । \newline
52. द्वाद॑शाक्षरा॒ जग॑ती॒ जग॑ती॒ द्वाद॑शाक्षरा॒ द्वाद॑शाक्षरा॒ जग॑ती । \newline
53. द्वाद॑शाक्ष॒रेति॒ द्वाद॑श - अ॒क्ष॒रा॒ । \newline
54. जग॑ती॒ जाग॑त॒म् जाग॑त॒म् जग॑ती॒ जग॑ती॒ जाग॑तम् । \newline
55. जाग॑तम् तृतीयसव॒नम् तृ॑तीयसव॒नम् जाग॑त॒म् जाग॑तम् तृतीयसव॒नम् । \newline
56. तृ॒ती॒य॒स॒व॒नम् तृ॑तीयसव॒नम् । \newline
57. तृ॒ती॒य॒स॒व॒नमिति॑ तृतीय - स॒व॒नम् । \newline
58. तृ॒ती॒य॒स॒व॒न मे॒वैव तृ॑तीयसव॒नम् तृ॑तीयसव॒न मे॒व । \newline
59. तृ॒ती॒य॒स॒व॒नमिति॑ तृतीय - स॒व॒नम् । \newline
60. ए॒व तेन॒ तेनै॒वैव तेन॑ । \newline
61. तेना᳚ प्नोत्या प्नोति॒ तेन॒ तेना᳚प्नोति । \newline
62. आ॒प्नो॒ति॒ दे॒वता॑भिर् दे॒वता॑भि राप्नो त्याप्नोति दे॒वता॑भिः । \newline
63. दे॒वता॑भि रे॒वैव दे॒वता॑भिर् दे॒वता॑भि रे॒व । \newline
64. ए॒व दे॒वता॑ दे॒वता॑ ए॒वैव दे॒वताः᳚ । \newline
65. दे॒वताः᳚ प्रति॒चर॑ति प्रति॒चर॑ति दे॒वता॑ दे॒वताः᳚ प्रति॒चर॑ति । \newline

\textbf{Ghana Paata } \newline

1. आ॒ग्ना॒वै॒ष्ण॒व मेका॑दशकपाल॒ मेका॑दशकपाल माग्नावैष्ण॒व मा᳚ग्नावैष्ण॒व मेका॑दशकपाल॒म् निर् णिरेका॑दशकपाल माग्नावैष्ण॒व मा᳚ग्नावैष्ण॒व मेका॑दशकपाल॒म् निः । \newline
2. आ॒ग्ना॒वै॒ष्ण॒वमित्या᳚ग्ना - वै॒ष्ण॒वम् । \newline
3. एका॑दशकपाल॒म् निर् णिरेका॑दशकपाल॒ मेका॑दशकपाल॒म् निर् व॑पेद् वपे॒न् निरेका॑दशकपाल॒ मेका॑दशकपाल॒म् निर् व॑पेत् । \newline
4. एका॑दशकपाल॒मित्येका॑दश - क॒पा॒ल॒म् । \newline
5. निर् व॑पेद् वपे॒न् निर् णिर् व॑पे॒न् माद्ध्य॑न्दिनस्य॒ माद्ध्य॑न्दिनस्य वपे॒न् निर् णिर् व॑पे॒न् माद्ध्य॑न्दिनस्य । \newline
6. व॒पे॒न् माद्ध्य॑न्दिनस्य॒ माद्ध्य॑न्दिनस्य वपेद् वपे॒न् माद्ध्य॑न्दिनस्य॒ सव॑नस्य॒ सव॑नस्य॒ माद्ध्य॑न्दिनस्य वपेद् वपे॒न् माद्ध्य॑न्दिनस्य॒ सव॑नस्य । \newline
7. माद्ध्य॑न्दिनस्य॒ सव॑नस्य॒ सव॑नस्य॒ माद्ध्य॑न्दिनस्य॒ माद्ध्य॑न्दिनस्य॒ सव॑नस्याका॒ल आ॑का॒ले सव॑नस्य॒ माद्ध्य॑न्दिनस्य॒ माद्ध्य॑न्दिनस्य॒ सव॑नस्याका॒ले । \newline
8. सव॑नस्याका॒ल आ॑का॒ले सव॑नस्य॒ सव॑नस्याका॒ले सर॑स्वती॒ सर॑स्वत्याका॒ले सव॑नस्य॒ सव॑नस्याका॒ले सर॑स्वती । \newline
9. आ॒का॒ले सर॑स्वती॒ सर॑स्व त्याका॒ल आ॑का॒ले सर॑स्व॒ त्याज्य॑भा॒गा ऽऽज्य॑भागा॒ सर॑स्व त्याका॒ल आ॑का॒ले सर॑स्व॒ त्याज्य॑भागा । \newline
10. आ॒का॒ल इत्या᳚ - का॒ले । \newline
11. सर॑स्व॒ त्याज्य॑भा॒गा ऽऽज्य॑भागा॒ सर॑स्वती॒ सर॑स्व॒ त्याज्य॑भागा॒ स्याथ् स्या दाज्य॑भागा॒ सर॑स्वती॒ सर॑स्व॒ त्याज्य॑भागा॒ स्यात् । \newline
12. आज्य॑भागा॒ स्याथ् स्या दाज्य॑भा॒गा ऽऽज्य॑भागा॒ स्याद् बा॑र्.हस्प॒त्यो बा॑र्.हस्प॒त्यः स्या दाज्य॑भा॒गा ऽऽज्य॑भागा॒ स्याद् बा॑र्.हस्प॒त्यः । \newline
13. आज्य॑भा॒गेत्याज्य॑ - भा॒गा॒ । \newline
14. स्याद् बा॑र्.हस्प॒त्यो बा॑र्.हस्प॒त्यः स्याथ् स्याद् बा॑र्.हस्प॒त्य श्च॒रु श्च॒रुर् बा॑र्.हस्प॒त्यः स्याथ् स्याद् बा॑र्.हस्प॒त्य श्च॒रुः । \newline
15. बा॒र्॒.ह॒स्प॒त्य श्च॒रु श्च॒रुर् बा॑र्.हस्प॒त्यो बा॑र्.हस्प॒त्य श्च॒रुर् यद् यच् च॒रुर् बा॑र्.हस्प॒त्यो बा॑र्.हस्प॒त्य श्च॒रुर् यत् । \newline
16. च॒रुर् यद् यच् च॒रु श्च॒रुर् यदेका॑दशकपाल॒ एका॑दशकपालो॒ यच् च॒रु श्च॒रुर् यदेका॑दशकपालः । \newline
17. यदेका॑दशकपाल॒ एका॑दशकपालो॒ यद् यदेका॑दशकपालो॒ भव॑ति॒ भव॒ त्येका॑दशकपालो॒ यद् यदेका॑दशकपालो॒ भव॑ति । \newline
18. एका॑दशकपालो॒ भव॑ति॒ भव॒ त्येका॑दशकपाल॒ एका॑दशकपालो॒ भव॒ त्येका॑दशाक्ष॒ रैका॑दशाक्षरा॒ भव॒ त्येका॑दशकपाल॒ एका॑दशकपालो॒ भव॒ त्येका॑दशाक्षरा । \newline
19. एका॑दशकपाल॒ इत्येका॑दश - क॒पा॒लः॒ । \newline
20. भव॒ त्येका॑दशाक्ष॒ रैका॑दशाक्षरा॒ भव॑ति॒ भव॒ त्येका॑दशाक्षरा त्रि॒ष्टुप् त्रि॒ष्टु बेका॑दशाक्षरा॒ भव॑ति॒ भव॒ त्येका॑दशाक्षरा त्रि॒ष्टुप् । \newline
21. एका॑दशाक्षरा त्रि॒ष्टुप् त्रि॒ष्टु बेका॑दशाक्ष॒ रैका॑दशाक्षरा त्रि॒ष्टुप् त्रैष्टु॑भ॒म् त्रैष्टु॑भम् त्रि॒ष्टु बेका॑दशाक्ष॒ रैका॑दशाक्षरा त्रि॒ष्टुप् त्रैष्टु॑भम् । \newline
22. एका॑दशाक्ष॒रेत्येका॑दश - अ॒क्ष॒रा॒ । \newline
23. त्रि॒ष्टुप् त्रैष्टु॑भ॒म् त्रैष्टु॑भम् त्रि॒ष्टुप् त्रि॒ष्टुप् त्रैष्टु॑भ॒म् माद्ध्य॑न्दिन॒म् माद्ध्य॑न्दिन॒म् त्रैष्टु॑भम् त्रि॒ष्टुप् त्रि॒ष्टुप् त्रैष्टु॑भ॒म् माद्ध्य॑न्दिनम् । \newline
24. त्रैष्टु॑भ॒म् माद्ध्य॑न्दिन॒म् माद्ध्य॑न्दिन॒म् त्रैष्टु॑भ॒म् त्रैष्टु॑भ॒म् माद्ध्य॑न्दिनꣳ॒॒ सव॑नꣳ॒॒ सव॑न॒म् माद्ध्य॑न्दिन॒म् त्रैष्टु॑भ॒म् त्रैष्टु॑भ॒म् माद्ध्य॑न्दिनꣳ॒॒ सव॑नम् । \newline
25. माद्ध्य॑न्दिनꣳ॒॒ सव॑नꣳ॒॒ सव॑न॒म् माद्ध्य॑न्दिन॒म् माद्ध्य॑न्दिनꣳ॒॒ सव॑न॒म् माद्ध्य॑न्दिन॒म् माद्ध्य॑न्दिनꣳ॒॒ सव॑न॒म् माद्ध्य॑न्दिन॒म् माद्ध्य॑न्दिनꣳ॒॒ सव॑न॒म् माद्ध्य॑न्दिनम् । \newline
26. सव॑न॒म् माद्ध्य॑न्दिन॒म् माद्ध्य॑न्दिनꣳ॒॒ सव॑नꣳ॒॒ सव॑न॒म् माद्ध्य॑न्दिन मे॒वैव माद्ध्य॑न्दिनꣳ॒॒ सव॑नꣳ॒॒ सव॑न॒म् माद्ध्य॑न्दिन मे॒व । \newline
27. माद्ध्य॑न्दिन मे॒वैव माद्ध्य॑न्दिन॒म् माद्ध्य॑न्दिन मे॒व सव॑नꣳ॒॒ सव॑न मे॒व माद्ध्य॑न्दिन॒म् माद्ध्य॑न्दिन मे॒व सव॑नम् । \newline
28. ए॒व सव॑नꣳ॒॒ सव॑न मे॒वैव सव॑न॒म् तेन॒ तेन॒ सव॑न मे॒वैव सव॑न॒म् तेन॑ । \newline
29. सव॑न॒म् तेन॒ तेन॒ सव॑नꣳ॒॒ सव॑न॒म् तेना᳚प्नो त्याप्नोति॒ तेन॒ सव॑नꣳ॒॒ सव॑न॒म् तेना᳚प्नोति । \newline
30. तेना᳚प्नो त्याप्नोति॒ तेन॒ तेना᳚प्नो त्याग्नावैष्ण॒व मा᳚ग्नावैष्ण॒व मा᳚प्नोति॒ तेन॒ तेना᳚प्नो त्याग्नावैष्ण॒वम् । \newline
31. आ॒प्नो॒ त्या॒ग्ना॒वै॒ष्ण॒व मा᳚ग्नावैष्ण॒व मा᳚प्नो त्याप्नो त्याग्नावैष्ण॒वम् द्वाद॑शकपाल॒म् द्वाद॑शकपाल माग्नावैष्ण॒व मा᳚प्नो त्याप्नो त्याग्नावैष्ण॒वम् द्वाद॑शकपालम् । \newline
32. आ॒ग्ना॒वै॒ष्ण॒वम् द्वाद॑शकपाल॒म् द्वाद॑शकपाल माग्नावैष्ण॒व मा᳚ग्नावैष्ण॒वम् द्वाद॑शकपाल॒म् निर् णिर् द्वाद॑शकपाल माग्नावैष्ण॒व मा᳚ग्नावैष्ण॒वम् द्वाद॑शकपाल॒म् निः । \newline
33. आ॒ग्ना॒वै॒ष्ण॒वमित्या᳚ग्ना - वै॒ष्ण॒वम् । \newline
34. द्वाद॑शकपाल॒म् निर् णिर् द्वाद॑शकपाल॒म् द्वाद॑शकपाल॒म् निर् व॑पेद् वपे॒न् निर् द्वाद॑शकपाल॒म् द्वाद॑शकपाल॒म् निर् व॑पेत् । \newline
35. द्वाद॑शकपाल॒मिति॒ द्वाद॑श - क॒पा॒ल॒म् । \newline
36. निर् व॑पेद् वपे॒न् निर् णिर् व॑पेत् तृतीयसव॒नस्य॑ तृतीयसव॒नस्य॑ वपे॒न् निर् णिर् व॑पेत् तृतीयसव॒नस्य॑ । \newline
37. व॒पे॒त् तृ॒ती॒य॒स॒व॒नस्य॑ तृतीयसव॒नस्य॑ वपेद् वपेत् तृतीयसव॒नस्या॑का॒ल आ॑का॒ले तृ॑तीयसव॒नस्य॑ वपेद् वपेत् तृतीयसव॒नस्या॑का॒ले । \newline
38. तृ॒ती॒य॒स॒व॒नस्या॑का॒ल आ॑का॒ले तृ॑तीयसव॒नस्य॑ तृतीयसव॒नस्या॑का॒ले सर॑स्वती॒ सर॑स्वत्याका॒ले तृ॑तीयसव॒नस्य॑ तृतीयसव॒नस्या॑का॒ले सर॑स्वती । \newline
39. तृ॒ती॒य॒स॒व॒नस्येति॑ तृतीय - स॒व॒नस्य॑ । \newline
40. आ॒का॒ले सर॑स्वती॒ सर॑स्व त्याका॒ल आ॑का॒ले सर॑स्व॒ त्याज्य॑भा॒गा ऽऽज्य॑भागा॒ सर॑स्व त्याका॒ल आ॑का॒ले सर॑स्व॒ त्याज्य॑भागा । \newline
41. आ॒का॒ल इत्या᳚ - का॒ले । \newline
42. सर॑स्व॒ त्याज्य॑भा॒गा ऽऽज्य॑भागा॒ सर॑स्वती॒ सर॑स्व॒ त्याज्य॑भागा॒ स्याथ् स्या दाज्य॑भागा॒ सर॑स्वती॒ सर॑स्व॒ त्याज्य॑भागा॒ स्यात् । \newline
43. आज्य॑भागा॒ स्याथ् स्या दाज्य॑भा॒गा ऽऽज्य॑भागा॒ स्याद् बा॑र्.हस्प॒त्यो बा॑र्.हस्प॒त्यः स्या दाज्य॑भा॒गा ऽऽज्य॑भागा॒ स्याद् बा॑र्.हस्प॒त्यः । \newline
44. आज्य॑भा॒गेत्याज्य॑ - भा॒गा॒ । \newline
45. स्याद् बा॑र्.हस्प॒त्यो बा॑र्.हस्प॒त्यः स्याथ् स्याद् बा॑र्.हस्प॒त्य श्च॒रु श्च॒रुर् बा॑र्.हस्प॒त्यः स्याथ् स्याद् बा॑र्.हस्प॒त्य श्च॒रुः । \newline
46. बा॒र्॒.ह॒स्प॒त्य श्च॒रु श्च॒रुर् बा॑र्.हस्प॒त्यो बा॑र्.हस्प॒त्य श्च॒रुर् यद् यच् च॒रुर् बा॑र्.हस्प॒त्यो बा॑र्.हस्प॒त्य श्च॒रुर् यत् । \newline
47. च॒रुर् यद् यच् च॒रु श्च॒रुर् यद् द्वाद॑शकपालो॒ द्वाद॑शकपालो॒ यच् च॒रु श्च॒रुर् यद् द्वाद॑शकपालः । \newline
48. यद् द्वाद॑शकपालो॒ द्वाद॑शकपालो॒ यद् यद् द्वाद॑शकपालो॒ भव॑ति॒ भव॑ति॒ द्वाद॑शकपालो॒ यद् यद् द्वाद॑शकपालो॒ भव॑ति । \newline
49. द्वाद॑शकपालो॒ भव॑ति॒ भव॑ति॒ द्वाद॑शकपालो॒ द्वाद॑शकपालो॒ भव॑ति॒ द्वाद॑शाक्षरा॒ द्वाद॑शाक्षरा॒ भव॑ति॒ द्वाद॑शकपालो॒ द्वाद॑शकपालो॒ भव॑ति॒ द्वाद॑शाक्षरा । \newline
50. द्वाद॑शकपाल॒ इति॒ द्वाद॑श - क॒पा॒लः॒ । \newline
51. भव॑ति॒ द्वाद॑शाक्षरा॒ द्वाद॑शाक्षरा॒ भव॑ति॒ भव॑ति॒ द्वाद॑शाक्षरा॒ जग॑ती॒ जग॑ती॒ द्वाद॑शाक्षरा॒ भव॑ति॒ भव॑ति॒ द्वाद॑शाक्षरा॒ जग॑ती । \newline
52. द्वाद॑शाक्षरा॒ जग॑ती॒ जग॑ती॒ द्वाद॑शाक्षरा॒ द्वाद॑शाक्षरा॒ जग॑ती॒ जाग॑त॒म् जाग॑त॒म् जग॑ती॒ द्वाद॑शाक्षरा॒ द्वाद॑शाक्षरा॒ जग॑ती॒ जाग॑तम् । \newline
53. द्वाद॑शाक्ष॒रेति॒ द्वाद॑श - अ॒क्ष॒रा॒ । \newline
54. जग॑ती॒ जाग॑त॒म् जाग॑त॒म् जग॑ती॒ जग॑ती॒ जाग॑तम् तृतीयसव॒नम् तृ॑तीयसव॒नम् जाग॑त॒म् जग॑ती॒ जग॑ती॒ जाग॑तम् तृतीयसव॒नम् । \newline
55. जाग॑तम् तृतीयसव॒नम् तृ॑तीयसव॒नम् जाग॑त॒म् जाग॑तम् तृतीयसव॒नम् । \newline
56. तृ॒ती॒य॒स॒व॒नम् तृ॑तीयसव॒नम् । \newline
57. तृ॒ती॒य॒स॒व॒नमिति॑ तृतीय - स॒व॒नम् । \newline
58. तृ॒ती॒य॒स॒व॒न मे॒वैव तृ॑तीयसव॒नम् तृ॑तीयसव॒न मे॒व तेन॒ तेनै॒व तृ॑तीयसव॒नम् तृ॑तीयसव॒न मे॒व तेन॑ । \newline
59. तृ॒ती॒य॒स॒व॒नमिति॑ तृतीय - स॒व॒नम् । \newline
60. ए॒व तेन॒ तेनै॒वैव तेना᳚प्नो त्याप्नोति॒ तेनै॒वैव तेना᳚प्नोति । \newline
61. तेना᳚प्नो त्याप्नोति॒ तेन॒ तेना᳚प्नोति दे॒वता॑भिर् दे॒वता॑भि राप्नोति॒ तेन॒ तेना᳚प्नोति दे॒वता॑भिः । \newline
62. आ॒प्नो॒ति॒ दे॒वता॑भिर् दे॒वता॑भि राप्नो त्याप्नोति दे॒वता॑भि रे॒वैव दे॒वता॑भि राप्नो त्याप्नोति दे॒वता॑भि रे॒व । \newline
63. दे॒वता॑भि रे॒वैव दे॒वता॑भिर् दे॒वता॑भि रे॒व दे॒वता॑ दे॒वता॑ ए॒व दे॒वता॑भिर् दे॒वता॑भि रे॒व दे॒वताः᳚ । \newline
64. ए॒व दे॒वता॑ दे॒वता॑ ए॒वैव दे॒वताः᳚ प्रति॒चर॑ति प्रति॒चर॑ति दे॒वता॑ ए॒वैव दे॒वताः᳚ प्रति॒चर॑ति । \newline
65. दे॒वताः᳚ प्रति॒चर॑ति प्रति॒चर॑ति दे॒वता॑ दे॒वताः᳚ प्रति॒चर॑ति य॒ज्ञेन॑ य॒ज्ञेन॑ प्रति॒चर॑ति दे॒वता॑ दे॒वताः᳚ प्रति॒चर॑ति य॒ज्ञेन॑ । \newline
\pagebreak
\markright{ TS 2.2.9.7  \hfill https://www.vedavms.in \hfill}
\addcontentsline{toc}{section}{ TS 2.2.9.7 }
\section*{ TS 2.2.9.7 }

\textbf{TS 2.2.9.7 } \newline
\textbf{Samhita Paata} \newline

प्रति॒चर॑ति य॒ज्ञेन॑ य॒ज्ञ्ं ॅवा॒चा वाचं॒ ब्रह्म॑णा॒ ब्रह्म॑ क॒पालै॑रे॒व छन्दाꣳ॑स्या॒प्नोति॑ पुरो॒डाशैः॒ सव॑नानि मैत्रावरु॒णमेक॑कपालं॒ निर्व॑पेद्-व॒शायै॑ का॒ले यैवासौ भ्रातृ॑व्यस्य व॒शाऽनू॑ब॒न्ध्या॑ सो ए॒वैषैतस्यैक॑कपालो भवति॒ न हि क॒पालैः᳚ प॒शुमर्.ह॒त्याप्तुं᳚ ॥ \newline

\textbf{Pada Paata} \newline

प्र॒ति॒चर॒तीति॑ प्रति - चर॑ति । य॒ज्ञेन॑ । य॒ज्ञ्म् । वा॒चा । वाच᳚म् । ब्रह्म॑णा । ब्रह्म॑ । क॒पालैः᳚ । ए॒व । छन्दाꣳ॑सि । आ॒प्नोति॑ । पु॒रो॒डाशैः᳚ । सव॑नानि । मै॒त्रा॒व॒रु॒णमिति॑ मैत्रा - व॒रु॒णम् । एक॑कपाल॒मित्येक॑ - क॒पा॒ल॒म् । निरिति॑ । व॒पे॒त् । व॒शायै᳚ । का॒ले । या । ए॒व । अ॒सौ । भ्रातृ॑व्यस्य । व॒शा । अ॒नू॒ब॒न्ध्येत्य॑नु - ब॒न्ध्या᳚ । सो इति॑ । ए॒व । ए॒षा । ए॒तस्य॑ । एक॑कपाल॒ इत्येक॑-क॒पा॒लः॒ । भ॒व॒ति॒ । न । हि । क॒पालैः᳚ । प॒शुम् । अर्.ह॑ति । आप्तु᳚म् ॥(ब्रह्म॑णै॒वैन॑म॒भि च॑रति - य॒ज्ञो न - तावे॒वा - ऽस्ये᳚न्द्रि॒य - मा᳚प्नोति -दे॒वताः᳚ -  \newline


\textbf{Krama Paata} \newline

प्र॒ति॒चर॑ति य॒ज्ञेन॑ । प्र॒ति॒चर॒तीति॑ प्रति - चर॑ति । य॒ज्ञेन॑ य॒ज्ञ्म् । य॒ज्ञ्ं ॅवा॒चा । वा॒चा वाच᳚म् । वाच॒म् ब्रह्म॑णा । ब्रह्म॑णा॒ ब्रह्म॑ । ब्रह्म॑ क॒पालैः᳚ । क॒पालै॑रे॒व । ए॒व छन्दाꣳ॑सि । छन्दाꣳ॑स्या॒प्नोति॑ । आ॒प्नोति॑ पुरो॒डाशैः᳚ । पु॒रो॒डाशैः॒ सव॑नानि । सव॑नानि मैत्रावरु॒णम् । मै॒त्रा॒व॒रु॒णमेक॑कपालम् । मै॒त्रा॒व॒रु॒णमिति॑ मैत्रा - व॒रु॒णम् । एक॑कपाल॒म् निः । एक॑कपाल॒ मित्येक॑ - क॒पा॒ल॒म् । निर् व॑पेत् । व॒पे॒द् व॒शायै᳚ । व॒शायै॑ का॒ले । का॒ले या । यैव । ए॒वासौ । अ॒सौ भ्रातृ॑व्यस्य । भ्रातृ॑व्यस्य व॒शा । व॒शा ऽनू॑ब॒न्द्ध्या᳚ । अ॒नू॒ब॒न्द्ध्या॑ सो । अ॒नू॒ब॒न्द्ध्येत्य॑नु - ब॒न्द्ध्या᳚ । सो ए॒व । सो इति॒ सो । ए॒वैषा । ए॒षैतस्य॑ । ए॒तसैक॑कपालः । एक॑कपालो भवति । एक॑कपाल॒ इत्येक॑ - क॒पा॒लः॒ । भ॒व॒ति॒ न । न हि । हि क॒पालैः᳚ । क॒पालैः᳚ प॒शुम् । प॒शुमर्.ह॑ति । अर्.ह॒त्याप्तु᳚म् । आप्तु॒मित्याप्तु᳚म् । \newline

\textbf{Jatai Paata} \newline

1. प्र॒ति॒चर॑ति य॒ज्ञेन॑ य॒ज्ञेन॑ प्रति॒चर॑ति प्रति॒चर॑ति य॒ज्ञेन॑ । \newline
2. प्र॒ति॒चर॒तीति॑ प्रति - चर॑ति । \newline
3. य॒ज्ञेन॑ य॒ज्ञ्ं ॅय॒ज्ञ्ं ॅय॒ज्ञेन॑ य॒ज्ञेन॑ य॒ज्ञ्म् । \newline
4. य॒ज्ञ्ं ॅवा॒चा वा॒चा य॒ज्ञ्ं ॅय॒ज्ञ्ं ॅवा॒चा । \newline
5. वा॒चा वाचं॒ ॅवाचं॑ ॅवा॒चा वा॒चा वाच᳚म् । \newline
6. वाच॒म् ब्रह्म॑णा॒ ब्रह्म॑णा॒ वाचं॒ ॅवाच॒म् ब्रह्म॑णा । \newline
7. ब्रह्म॑णा॒ ब्रह्म॒ ब्रह्म॒ ब्रह्म॑णा॒ ब्रह्म॑णा॒ ब्रह्म॑ । \newline
8. ब्रह्म॑ क॒पालैः᳚ क॒पालै॒र् ब्रह्म॒ ब्रह्म॑ क॒पालैः᳚ । \newline
9. क॒पालै॑ रे॒वैव क॒पालैः᳚ क॒पालै॑ रे॒व । \newline
10. ए॒व छन्दाꣳ॑सि॒ छन्दाꣳ॑ स्ये॒वैव छन्दाꣳ॑सि । \newline
11. छन्दाꣳ॑ स्या॒प्नो त्या॒प्नोति॒ छन्दाꣳ॑सि॒ छन्दाꣳ॑ स्या॒प्नोति॑ । \newline
12. आ॒प्नोति॑ पुरो॒डाशैः᳚ पुरो॒डाशै॑ रा॒प्नो त्या॒प्नोति॑ पुरो॒डाशैः᳚ । \newline
13. पु॒रो॒डाशैः॒ सव॑नानि॒ सव॑नानि पुरो॒डाशैः᳚ पुरो॒डाशैः॒ सव॑नानि । \newline
14. सव॑नानि मैत्रावरु॒णम् मै᳚त्रावरु॒णꣳ सव॑नानि॒ सव॑नानि मैत्रावरु॒णम् । \newline
15. मै॒त्रा॒व॒रु॒ण मेक॑कपाल॒ मेक॑कपालम् मैत्रावरु॒णम् मै᳚त्रावरु॒ण मेक॑कपालम् । \newline
16. मै॒त्रा॒व॒रु॒णमिति॑ मैत्रा - व॒रु॒णम् । \newline
17. एक॑कपाल॒म् निर् णिरेक॑कपाल॒ मेक॑कपाल॒म् निः । \newline
18. एक॑कपाल॒मित्येक॑ - क॒पा॒ल॒म् । \newline
19. निर् व॑पेद् वपे॒न् निर् णिर् व॑पेत् । \newline
20. व॒पे॒द् व॒शायै॑ व॒शायै॑ वपेद् वपेद् व॒शायै᳚ । \newline
21. व॒शायै॑ का॒ले का॒ले व॒शायै॑ व॒शायै॑ का॒ले । \newline
22. का॒ले या या का॒ले का॒ले या । \newline
23. यैवैव या यैव । \newline
24. ए॒वासा व॒सा वे॒वैवासौ । \newline
25. अ॒सौ भ्रातृ॑व्यस्य॒ भ्रातृ॑व्यस्या॒सा व॒सौ भ्रातृ॑व्यस्य । \newline
26. भ्रातृ॑व्यस्य व॒शा व॒शा भ्रातृ॑व्यस्य॒ भ्रातृ॑व्यस्य व॒शा । \newline
27. व॒शा ऽनू॑ब॒न्ध्या॑ ऽनूब॒न्ध्या॑ व॒शा व॒शा ऽनू॑ब॒न्ध्या᳚ । \newline
28. अ॒नू॒ब॒न्ध्या॑ सो सो अ॑नूब॒न्ध्या॑ ऽनूब॒न्ध्या॑ सो । \newline
29. अ॒नू॒ब॒न्ध्येत्य॑नु - ब॒न्ध्या᳚ । \newline
30. सो ए॒वैव सो सो ए॒व । \newline
31. सो इति॒ सो । \newline
32. ए॒वै षैषै वैवैषा । \newline
33. ए॒षैतस्यै॒ तस्यै॒ षैषैतस्य॑ । \newline
34. ए॒त स्यैक॑कपाल॒ एक॑कपाल ए॒त स्यै॒त स्यैक॑कपालः । \newline
35. एक॑कपालो भवति भव॒ त्येक॑कपाल॒ एक॑कपालो भवति । \newline
36. एक॑कपाल॒ इत्येक॑ - क॒पा॒लः॒ । \newline
37. भ॒व॒ति॒ न न भ॑वति भवति॒ न । \newline
38. न हि हि न न हि । \newline
39. हि क॒पालैः᳚ क॒पालै॒र्॒. हि हि क॒पालैः᳚ । \newline
40. क॒पालैः᳚ प॒शुम् प॒शुम् क॒पालैः᳚ क॒पालैः᳚ प॒शुम् । \newline
41. प॒शु मर्.ह॒ त्यर्.ह॑ति प॒शुम् प॒शु मर्.ह॑ति । \newline
42. अर्.ह॒ त्याप्तु॒ माप्तु॒ मर्.ह॒ त्यर्.ह॒ त्याप्तु᳚म् । \newline
43. आप्तु॒मित्याप्तु᳚म् । \newline

\textbf{Ghana Paata } \newline

1. प्र॒ति॒चर॑ति य॒ज्ञेन॑ य॒ज्ञेन॑ प्रति॒चर॑ति प्रति॒चर॑ति य॒ज्ञेन॑ य॒ज्ञ्ं ॅय॒ज्ञ्ं ॅय॒ज्ञेन॑ प्रति॒चर॑ति प्रति॒चर॑ति य॒ज्ञेन॑ य॒ज्ञ्म् । \newline
2. प्र॒ति॒चर॒तीति॑ प्रति - चर॑ति । \newline
3. य॒ज्ञेन॑ य॒ज्ञ्ं ॅय॒ज्ञ्ं ॅय॒ज्ञेन॑ य॒ज्ञेन॑ य॒ज्ञ्ं ॅवा॒चा वा॒चा य॒ज्ञ्ं ॅय॒ज्ञेन॑ य॒ज्ञेन॑ य॒ज्ञ्ं ॅवा॒चा । \newline
4. य॒ज्ञ्ं ॅवा॒चा वा॒चा य॒ज्ञ्ं ॅय॒ज्ञ्ं ॅवा॒चा वाचं॒ ॅवाचं॑ ॅवा॒चा य॒ज्ञ्ं ॅय॒ज्ञ्ं ॅवा॒चा वाच᳚म् । \newline
5. वा॒चा वाचं॒ ॅवाचं॑ ॅवा॒चा वा॒चा वाच॒म् ब्रह्म॑णा॒ ब्रह्म॑णा॒ वाचं॑ ॅवा॒चा वा॒चा वाच॒म् ब्रह्म॑णा । \newline
6. वाच॒म् ब्रह्म॑णा॒ ब्रह्म॑णा॒ वाचं॒ ॅवाच॒म् ब्रह्म॑णा॒ ब्रह्म॒ ब्रह्म॒ ब्रह्म॑णा॒ वाचं॒ ॅवाच॒म् ब्रह्म॑णा॒ ब्रह्म॑ । \newline
7. ब्रह्म॑णा॒ ब्रह्म॒ ब्रह्म॒ ब्रह्म॑णा॒ ब्रह्म॑णा॒ ब्रह्म॑ क॒पालैः᳚ क॒पालै॒र् ब्रह्म॒ ब्रह्म॑णा॒ ब्रह्म॑णा॒ ब्रह्म॑ क॒पालैः᳚ । \newline
8. ब्रह्म॑ क॒पालैः᳚ क॒पालै॒र् ब्रह्म॒ ब्रह्म॑ क॒पालै॑ रे॒वैव क॒पालै॒र् ब्रह्म॒ ब्रह्म॑ क॒पालै॑ रे॒व । \newline
9. क॒पालै॑ रे॒वैव क॒पालैः᳚ क॒पालै॑ रे॒व छन्दाꣳ॑सि॒ छन्दाꣳ॑ स्ये॒व क॒पालैः᳚ क॒पालै॑ रे॒व छन्दाꣳ॑सि । \newline
10. ए॒व छन्दाꣳ॑सि॒ छन्दाꣳ॑ स्ये॒वैव छन्दाꣳ॑स्या॒प्नो त्या॒प्नोति॒ छन्दाꣳ॑स्ये॒वैव छन्दाꣳ॑ स्या॒प्नोति॑ । \newline
11. छन्दाꣳ॑ स्या॒प्नो त्या॒प्नोति॒ छन्दाꣳ॑सि॒ छन्दाꣳ॑ स्या॒प्नोति॑ पुरो॒डाशैः᳚ पुरो॒डाशै॑ रा॒प्नोति॒ छन्दाꣳ॑सि॒ छन्दाꣳ॑ स्या॒प्नोति॑ पुरो॒डाशैः᳚ । \newline
12. आ॒प्नोति॑ पुरो॒डाशैः᳚ पुरो॒डाशै॑ रा॒प्नो त्या॒प्नोति॑ पुरो॒डाशैः॒ सव॑नानि॒ सव॑नानि पुरो॒डाशै॑ रा॒प्नो त्या॒प्नोति॑ पुरो॒डाशैः॒ सव॑नानि । \newline
13. पु॒रो॒डाशैः॒ सव॑नानि॒ सव॑नानि पुरो॒डाशैः᳚ पुरो॒डाशैः॒ सव॑नानि मैत्रावरु॒णम् मै᳚त्रावरु॒णꣳ सव॑नानि पुरो॒डाशैः᳚ पुरो॒डाशैः॒ सव॑नानि मैत्रावरु॒णम् । \newline
14. सव॑नानि मैत्रावरु॒णम् मै᳚त्रावरु॒णꣳ सव॑नानि॒ सव॑नानि मैत्रावरु॒ण मेक॑कपाल॒ मेक॑कपालम् मैत्रावरु॒णꣳ सव॑नानि॒ सव॑नानि मैत्रावरु॒ण मेक॑कपालम् । \newline
15. मै॒त्रा॒व॒रु॒ण मेक॑कपाल॒ मेक॑कपालम् मैत्रावरु॒णम् मै᳚त्रावरु॒ण मेक॑कपाल॒म् निर् णिरेक॑कपालम् मैत्रावरु॒णम् 
मै᳚त्रावरु॒ण मेक॑कपाल॒म् निः । \newline
16. मै॒त्रा॒व॒रु॒णमिति॑ मैत्रा - व॒रु॒णम् । \newline
17. एक॑कपाल॒म् निर् णिरेक॑कपाल॒ मेक॑कपाल॒म् निर् व॑पेद् वपे॒न् निरेक॑कपाल॒ मेक॑कपाल॒म् निर् व॑पेत् । \newline
18. एक॑कपाल॒मित्येक॑ - क॒पा॒ल॒म् । \newline
19. निर् व॑पेद् वपे॒न् निर् णिर् व॑पेद् व॒शायै॑ व॒शायै॑ वपे॒न् निर् णिर् व॑पेद् व॒शायै᳚ । \newline
20. व॒पे॒द् व॒शायै॑ व॒शायै॑ वपेद् वपेद् व॒शायै॑ का॒ले का॒ले व॒शायै॑ वपेद् वपेद् व॒शायै॑ का॒ले । \newline
21. व॒शायै॑ का॒ले का॒ले व॒शायै॑ व॒शायै॑ का॒ले या या का॒ले व॒शायै॑ व॒शायै॑ का॒ले या । \newline
22. का॒ले या या का॒ले का॒ले यैवैव या का॒ले का॒ले यैव । \newline
23. यैवैव या यैवासा व॒सा वे॒व या यैवासौ । \newline
24. ए॒वासा व॒सा वे॒वैवासौ भ्रातृ॑व्यस्य॒ भ्रातृ॑व्यस्या॒सा वे॒वैवासौ भ्रातृ॑व्यस्य । \newline
25. अ॒सौ भ्रातृ॑व्यस्य॒ भ्रातृ॑व्यस्या॒सा व॒सौ भ्रातृ॑व्यस्य व॒शा व॒शा भ्रातृ॑व्यस्या॒सा व॒सौ भ्रातृ॑व्यस्य व॒शा । \newline
26. भ्रातृ॑व्यस्य व॒शा व॒शा भ्रातृ॑व्यस्य॒ भ्रातृ॑व्यस्य व॒शा ऽनू॑ब॒न्ध्या॑ ऽनूब॒न्ध्या॑ व॒शा भ्रातृ॑व्यस्य॒ भ्रातृ॑व्यस्य व॒शा ऽनू॑ब॒न्ध्या᳚ । \newline
27. व॒शा ऽनू॑ब॒न्ध्या॑ ऽनूब॒न्ध्या॑ व॒शा व॒शा ऽनू॑ब॒न्ध्या॑ सो सो अ॑नूब॒न्ध्या॑ व॒शा व॒शा ऽनू॑ब॒न्ध्या॑ सो । \newline
28. अ॒नू॒ब॒न्ध्या॑ सो सो अ॑नूब॒न्ध्या॑ ऽनूब॒न्ध्या॑ सो ए॒वैव सो अ॑नूब॒न्ध्या॑ ऽनूब॒न्ध्या॑ सो ए॒व । \newline
29. अ॒नू॒ब॒न्ध्येत्य॑नु - ब॒न्ध्या᳚ । \newline
30. सो ए॒वैव सो सो ए॒वैषैषैव सो सो ए॒वैषा । \newline
31. सो इति॒ सो । \newline
32. ए॒वै षैषै वैवै षैतस्यै॒ तस्यै॒ षैवैवै षैतस्य॑ । \newline
33. ए॒षैतस्यै॒ तस्यै॒ षैषै तस्यैक॑कपाल॒ एक॑कपाल ए॒तस्यै॒ षैषै तस्यैक॑कपालः । \newline
34. ए॒त स्यैक॑कपाल॒ एक॑कपाल ए॒त स्यै॒ तस्यैक॑कपालो भवति भव॒ त्येक॑कपाल ए॒तस्यै॒ तस्यैक॑कपालो भवति । \newline
35. एक॑कपालो भवति भव॒ त्येक॑कपाल॒ एक॑कपालो भवति॒ न न भ॑व॒ त्येक॑कपाल॒ एक॑कपालो भवति॒ न । \newline
36. एक॑कपाल॒ इत्येक॑ - क॒पा॒लः॒ । \newline
37. भ॒व॒ति॒ न न भ॑वति भवति॒ न हि हि न भ॑वति भवति॒ न हि । \newline
38. न हि हि न न हि क॒पालैः᳚ क॒पालै॒र्॒. हि न न हि क॒पालैः᳚ । \newline
39. हि क॒पालैः᳚ क॒पालै॒र्॒. हि हि क॒पालैः᳚ प॒शुम् प॒शुम् क॒पालै॒र्॒. हि हि क॒पालैः᳚ प॒शुम् । \newline
40. क॒पालैः᳚ प॒शुम् प॒शुम् क॒पालैः᳚ क॒पालैः᳚ प॒शु मर्.ह॒ त्यर्.ह॑ति प॒शुम् क॒पालैः᳚ क॒पालैः᳚ प॒शु मर्.ह॑ति । \newline
41. प॒शु मर्.ह॒ त्यर्.ह॑ति प॒शुम् प॒शु मर्.ह॒ त्याप्तु॒ माप्तु॒ मर्.ह॑ति प॒शुम् प॒शु मर्.ह॒ त्याप्तु᳚म् । \newline
42. अर्.ह॒ त्याप्तु॒ माप्तु॒ मर्.ह॒ त्यर्.ह॒ त्याप्तु᳚म् । \newline
43. आप्तु॒मित्याप्तु᳚म् । \newline
\pagebreak
\markright{ TS 2.2.10.1  \hfill https://www.vedavms.in \hfill}
\addcontentsline{toc}{section}{ TS 2.2.10.1 }
\section*{ TS 2.2.10.1 }

\textbf{TS 2.2.10.1 } \newline
\textbf{Samhita Paata} \newline

अ॒सावा॑दि॒त्यो न व्य॑रोचत॒ तस्मै॑ दे॒वाः प्राय॑श्चित्तिमैच्छ॒न् तस्मा॑ ए॒तꣳ सो॑मारौ॒द्रं च॒रुं निर॑वप॒न् तेनै॒वास्मि॒न् रुच॑मदधु॒र्यो ब्र॑ह्मवर्च॒सका॑मः॒ स्यात् तस्मा॑ ए॒तꣳ सो॑मारौ॒द्रं च॒रुं निर्व॑पे॒थ् सोमं॑ चै॒व रु॒द्रं च॒ स्वेन॑ भाग॒धेये॒नोप॑ धावति॒ तावे॒वास्मि॑न् ब्रह्मवर्च॒सं ध॑त्तो ब्रह्मवर्च॒स्ये॑व भ॑वति तिष्यापूर्णमा॒से निर्व॑पेद्-रु॒द्रो - [  ] \newline

\textbf{Pada Paata} \newline

अ॒सौ । आ॒दि॒त्यः । न । वीति॑ । अ॒रो॒च॒त॒ । तस्मै᳚ । दे॒वाः । प्राय॑श्चित्तिम् । ऐ॒च्छ॒न्न् । तस्मै᳚ । ए॒तम् । सो॒मा॒रौ॒द्रमिति॑ सोमा-रौ॒द्रम् । च॒रुम् । निरिति॑ । अ॒व॒प॒न्न् । तेन॑ । ए॒व । अ॒स्मि॒न्न् । रुच᳚म् । अ॒द॒धुः॒ । यः । ब्र॒ह्म॒व॒र्च॒सका॑म॒ इति॑ ब्रह्मवर्च॒स - का॒मः॒ । स्यात् । तस्मै᳚ । ए॒तम् । सो॒मा॒रौ॒द्रमिति॑ सोमा - रौ॒द्रम् । च॒रुम् । निरिति॑ । व॒पे॒त् । सोम᳚म् । च॒ । ए॒व । रु॒द्रम् । च॒ । स्वेन॑ । भा॒ग॒धेये॒नेति॑ भाग-धेये॑न । उपेति॑ । धा॒व॒ति॒ । तौ । ए॒व । अ॒स्मि॒न्न् । ब्र॒ह्म॒व॒र्च॒समिति॑ ब्रह्म - व॒र्च॒सम् । ध॒त्तः॒ । ब्र॒ह्म॒व॒र्च॒सीति॑ ब्रह्म - व॒र्च॒सी । ए॒व । भ॒व॒ति॒ । ति॒ष्या॒पू॒र्ण॒मा॒स इति॑ तिष्या - पू॒र्ण॒मा॒से । निरिति॑ । व॒पे॒त् । रु॒द्रः ।  \newline


\textbf{Krama Paata} \newline

अ॒सावा॑दि॒त्यः । आ॒दि॒त्यो न । न वि । व्य॑रोचत । अ॒रो॒च॒त॒ तस्मै᳚ । तस्मै॑ दे॒वाः । दे॒वाः प्राय॑श्चित्तिम् । प्राय॑श्चित्तिमैच्छन्न् । ऐ॒च्छ॒न् तस्मै᳚ । तस्मा॑ ए॒तम् । ए॒तꣳ सो॑मारौ॒द्रम् । सो॒मा॒रौ॒द्रम् च॒रुम् । सो॒मा॒रौ॒द्रमिति॑ सोमा - रौ॒द्रम् । च॒रुम् निः । निर॑वपन्न् । अ॒व॒प॒न्,तेन॑ । तेनै॒व । ए॒वास्मिन्न्॑ । अ॒स्मि॒न् रुच᳚म् । रुच॑मदधुः । अ॒द॒धु॒र् यः । यो ब्र॑ह्मवर्च॒सका॑मः । ब्र॒ह्म॒व॒र्च॒सका॑मः॒ स्यात् । ब्र॒ह्म॒व॒र्च॒सका॑म॒ इति॑ ब्रह्मवर्च॒स - का॒मः॒ । स्यात् तस्मै᳚ । तस्मा॑ ए॒तम् । ए॒तꣳ सो॑मारौ॒द्रम् । सो॒मा॒रौ॒द्रम् च॒रुम् । सो॒मा॒रौ॒द्रमिति॑ सोमा - रौ॒द्रम् । च॒रुम् निः । निर् व॑पेत् । व॒पे॒थ् सोम᳚म् । सोम॑म् च । चै॒व । ए॒व रु॒द्रम् । रु॒द्रम् च॑ । च॒ स्वेन॑ । स्वेन॑ भाग॒धेये॑न । भा॒ग॒धेये॒नोप॑ । भा॒ग॒धेये॒नेति॑ भाग - धेये॑न । उप॑ धावति । धा॒व॒ति॒ तौ । तावे॒व । ए॒वास्मिन्न्॑ । अ॒स्मि॒न् ब्र॒ह्म॒व॒र्च॒सम् । ब्र॒ह्म॒व॒र्च॒सम् ध॑त्तः । ब्र॒ह्म॒व॒र्च॒समिति॑ ब्रह्म - व॒र्च॒सम् । ध॒त्तो॒ ब्र॒ह्म॒व॒र्च॒सी । ब्र॒ह्म॒व॒र्च॒स्ये॑व । ब्र॒ह्म॒व॒र्च॒सीति॑ ब्रह्म - व॒र्च॒सी । ए॒व भ॑वति । भ॒व॒ति॒ ति॒ष्या॒पू॒र्ण॒मा॒से । ति॒ष्या॒पू॒र्ण॒मा॒से निः । ति॒ष्या॒पू॒र्ण॒मा॒स इति॑ तिष्या - पू॒र्ण॒मा॒से । निर् व॑पेत् । व॒पे॒द् रु॒द्रः । रु॒द्रो वै \newline

\textbf{Jatai Paata} \newline

1. अ॒सा वा॑दि॒त्य आ॑दि॒त्यो॑ ऽसा व॒सा वा॑दि॒त्यः । \newline
2. आ॒दि॒त्यो न नादि॒त्य आ॑दि॒त्यो न । \newline
3. न वि वि न न वि । \newline
4. व्य॑रोचता रोचत॒ वि व्य॑रोचत । \newline
5. अ॒रो॒च॒त॒ तस्मै॒ तस्मा॑ अरोचता रोचत॒ तस्मै᳚ । \newline
6. तस्मै॑ दे॒वा दे॒वा स्तस्मै॒ तस्मै॑ दे॒वाः । \newline
7. दे॒वाः प्राय॑श्चित्ति॒म् प्राय॑श्चित्तिम् दे॒वा दे॒वाः प्राय॑श्चित्तिम् । \newline
8. प्राय॑श्चित्ति मैच्छन् नैच्छ॒न् प्राय॑श्चित्ति॒म् प्राय॑श्चित्ति मैच्छन्न् । \newline
9. ऐ॒च्छ॒न् तस्मै॒ तस्मा॑ ऐच्छन् नैच्छ॒न् तस्मै᳚ । \newline
10. तस्मा॑ ए॒त मे॒तम् तस्मै॒ तस्मा॑ ए॒तम् । \newline
11. ए॒तꣳ सो॑मारौ॒द्रꣳ सो॑मारौ॒द्र मे॒त मे॒तꣳ सो॑मारौ॒द्रम् । \newline
12. सो॒मा॒रौ॒द्रम् च॒रुम् च॒रुꣳ सो॑मारौ॒द्रꣳ सो॑मारौ॒द्रम् च॒रुम् । \newline
13. सो॒मा॒रौ॒द्रमिति॑ सोमा - रौ॒द्रम् । \newline
14. च॒रुम् निर् णिश्च॒रुम् च॒रुम् निः । \newline
15. नि र॑वपन् नवप॒न् निर् णि र॑वपन्न् । \newline
16. अ॒व॒प॒न् तेन॒ तेना॑वपन् नवप॒न् तेन॑ । \newline
17. तेनै॒वैव तेन॒ तेनै॒व । \newline
18. ए॒वास्मि॑न् नस्मिन् ने॒वैवास्मिन्न्॑ । \newline
19. अ॒स्मि॒न् रुचꣳ॒॒ रुच॑ मस्मिन् नस्मि॒न् रुच᳚म् । \newline
20. रुच॑ मदधु रदधू॒ रुचꣳ॒॒ रुच॑ मदधुः । \newline
21. अ॒द॒धु॒र् यो यो॑ ऽदधु रदधु॒र् यः । \newline
22. यो ब्र॑ह्मवर्च॒सका॑मो ब्रह्मवर्च॒सका॑मो॒ यो यो ब्र॑ह्मवर्च॒सका॑मः । \newline
23. ब्र॒ह्म॒व॒र्च॒सका॑मः॒ स्याथ् स्याद् ब्र॑ह्मवर्च॒सका॑मो ब्रह्मवर्च॒सका॑मः॒ स्यात् । \newline
24. ब्र॒ह्म॒व॒र्च॒सका॑म॒ इति॑ ब्रह्मवर्च॒स - का॒मः॒ । \newline
25. स्यात् तस्मै॒ तस्मै॒ स्याथ् स्यात् तस्मै᳚ । \newline
26. तस्मा॑ ए॒त मे॒तम् तस्मै॒ तस्मा॑ ए॒तम् । \newline
27. ए॒तꣳ सो॑मारौ॒द्रꣳ सो॑मारौ॒द्र मे॒त मे॒तꣳ सो॑मारौ॒द्रम् । \newline
28. सो॒मा॒रौ॒द्रम् च॒रुम् च॒रुꣳ सो॑मारौ॒द्रꣳ सो॑मारौ॒द्रम् च॒रुम् । \newline
29. सो॒मा॒रौ॒द्रमिति॑ सोमा - रौ॒द्रम् । \newline
30. च॒रुम् निर् णिश्च॒रुम् च॒रुम् निः । \newline
31. निर् व॑पेद् वपे॒न् निर् णिर् व॑पेत् । \newline
32. व॒पे॒थ् सोमꣳ॒॒ सोमं॑ ॅवपेद् वपे॒थ् सोम᳚म् । \newline
33. सोम॑म् च च॒ सोमꣳ॒॒ सोम॑म् च । \newline
34. चै॒वैव च॑ चै॒व । \newline
35. ए॒व रु॒द्रꣳ रु॒द्र मे॒वैव रु॒द्रम् । \newline
36. रु॒द्रम् च॑ च रु॒द्रꣳ रु॒द्रम् च॑ । \newline
37. च॒ स्वेन॒ स्वेन॑ च च॒ स्वेन॑ । \newline
38. स्वेन॑ भाग॒धेये॑न भाग॒धेये॑न॒ स्वेन॒ स्वेन॑ भाग॒धेये॑न । \newline
39. भा॒ग॒धेये॒नोपोप॑ भाग॒धेये॑न भाग॒धेये॒नोप॑ । \newline
40. भा॒ग॒धेये॒नेति॑ भाग - धेये॑न । \newline
41. उप॑ धावति धाव॒ त्युपोप॑ धावति । \newline
42. धा॒व॒ति॒ तौ तौ धा॑वति धावति॒ तौ । \newline
43. ता वे॒वैव तौ ता वे॒व । \newline
44. ए॒वास्मि॑न् नस्मिन् ने॒वैवास्मिन्न्॑ । \newline
45. अ॒स्मि॒न् ब्र॒ह्म॒व॒र्च॒सम् ब्र॑ह्मवर्च॒स म॑स्मिन् नस्मिन् ब्रह्मवर्च॒सम् । \newline
46. ब्र॒ह्म॒व॒र्च॒सम् ध॑त्तो धत्तो ब्रह्मवर्च॒सम् ब्र॑ह्मवर्च॒सम् ध॑त्तः । \newline
47. ब्र॒ह्म॒व॒र्च॒समिति॑ ब्रह्म - व॒र्च॒सम् । \newline
48. ध॒त्तो॒ ब्र॒ह्म॒व॒र्च॒सी ब्र॑ह्मवर्च॒सी ध॑त्तो धत्तो ब्रह्मवर्च॒सी । \newline
49. ब्र॒ह्म॒व॒र्च॒ स्ये॑वैव ब्र॑ह्मवर्च॒सी ब्र॑ह्मवर्च॒ स्ये॑व । \newline
50. ब्र॒ह्म॒व॒र्च॒सीति॑ ब्रह्म - व॒र्च॒सी । \newline
51. ए॒व भ॑वति भव त्ये॒वैव भ॑वति । \newline
52. भ॒व॒ति॒ ति॒ष्या॒पू॒र्ण॒मा॒से ति॑ष्यापूर्णमा॒से भ॑वति भवति तिष्यापूर्णमा॒से । \newline
53. ति॒ष्या॒पू॒र्ण॒मा॒से निर् णिष् टि॑ष्यापूर्णमा॒से ति॑ष्यापूर्णमा॒से निः । \newline
54. ति॒ष्या॒पू॒र्ण॒मा॒स इति॑ तिष्या - पू॒र्ण॒मा॒से । \newline
55. निर् व॑पेद् वपे॒न् निर् णिर् व॑पेत् । \newline
56. व॒पे॒द् रु॒द्रो रु॒द्रो व॑पेद् वपेद् रु॒द्रः । \newline
57. रु॒द्रो वै वै रु॒द्रो रु॒द्रो वै । \newline

\textbf{Ghana Paata } \newline

1. अ॒सा वा॑दि॒त्य आ॑दि॒त्यो॑ ऽसा व॒सा वा॑दि॒त्यो न नादि॒त्यो॑ ऽसा व॒सा वा॑दि॒त्यो न । \newline
2. आ॒दि॒त्यो न नादि॒त्य आ॑दि॒त्यो न वि वि नादि॒त्य आ॑दि॒त्यो न वि । \newline
3. न वि वि न न व्य॑रोचता रोचत॒ वि न न व्य॑रोचत । \newline
4. व्य॑रोचता रोचत॒ वि व्य॑रोचत॒ तस्मै॒ तस्मा॑ अरोचत॒ वि व्य॑रोचत॒ तस्मै᳚ । \newline
5. अ॒रो॒च॒त॒ तस्मै॒ तस्मा॑ अरोचता रोचत॒ तस्मै॑ दे॒वा दे॒वा स्तस्मा॑ अरोचता रोचत॒ तस्मै॑ दे॒वाः । \newline
6. तस्मै॑ दे॒वा दे॒वा स्तस्मै॒ तस्मै॑ दे॒वाः प्राय॑श्चित्ति॒म् प्राय॑श्चित्तिम् दे॒वा स्तस्मै॒ तस्मै॑ दे॒वाः प्राय॑श्चित्तिम् । \newline
7. दे॒वाः प्राय॑श्चित्ति॒म् प्राय॑श्चित्तिम् दे॒वा दे॒वाः प्राय॑श्चित्ति मैच्छन् नैच्छ॒न् प्राय॑श्चित्तिम् दे॒वा दे॒वाः प्राय॑श्चित्ति मैच्छन्न् । \newline
8. प्राय॑श्चित्ति मैच्छन् नैच्छ॒न् प्राय॑श्चित्ति॒म् प्राय॑श्चित्ति मैच्छ॒न् तस्मै॒ तस्मा॑ ऐच्छ॒न् प्राय॑श्चित्ति॒म् प्राय॑श्चित्ति मैच्छ॒न् तस्मै᳚ । \newline
9. ऐ॒च्छ॒न् तस्मै॒ तस्मा॑ ऐच्छन् नैच्छ॒न् तस्मा॑ ए॒त मे॒तम् तस्मा॑ ऐच्छन् नैच्छ॒न् तस्मा॑ ए॒तम् । \newline
10. तस्मा॑ ए॒त मे॒तम् तस्मै॒ तस्मा॑ ए॒तꣳ सो॑मारौ॒द्रꣳ सो॑मारौ॒द्र मे॒तम् तस्मै॒ तस्मा॑ ए॒तꣳ सो॑मारौ॒द्रम् । \newline
11. ए॒तꣳ सो॑मारौ॒द्रꣳ सो॑मारौ॒द्र मे॒त मे॒तꣳ सो॑मारौ॒द्रम् च॒रुम् च॒रुꣳ सो॑मारौ॒द्र मे॒त मे॒तꣳ सो॑मारौ॒द्रम् च॒रुम् । \newline
12. सो॒मा॒रौ॒द्रम् च॒रुम् च॒रुꣳ सो॑मारौ॒द्रꣳ सो॑मारौ॒द्रम् च॒रुम् निर् णिश्च॒रुꣳ सो॑मारौ॒द्रꣳ सो॑मारौ॒द्रम् च॒रुम् निः । \newline
13. सो॒मा॒रौ॒द्रमिति॑ सोमा - रौ॒द्रम् । \newline
14. च॒रुम् निर् णिश्च॒रुम् च॒रुम् निर॑वपन् नवप॒न् निश्च॒रुम् च॒रुम् निर॑वपन्न् । \newline
15. निर॑वपन् नवप॒न् निर् णिर॑वप॒न् तेन॒ तेना॑वप॒न् निर् णिर॑वप॒न् तेन॑ । \newline
16. अ॒व॒प॒न् तेन॒ तेना॑वपन् नवप॒न् तेनै॒वैव तेना॑वपन् नवप॒न् तेनै॒व । \newline
17. तेनै॒वैव तेन॒ तेनै॒वास्मि॑न् नस्मिन् ने॒व तेन॒ तेनै॒वास्मिन्न्॑ । \newline
18. ए॒वास्मि॑न् नस्मिन् ने॒वैवास्मि॒न् रुचꣳ॒॒ रुच॑ मस्मिन् ने॒वैवास्मि॒न् रुच᳚म् । \newline
19. अ॒स्मि॒न् रुचꣳ॒॒ रुच॑ मस्मिन् नस्मि॒न् रुच॑ मदधु रदधू॒ रुच॑ मस्मिन् नस्मि॒न् रुच॑ मदधुः । \newline
20. रुच॑ मदधु रदधू॒ रुचꣳ॒॒ रुच॑ मदधु॒र् यो यो॑ ऽदधू॒ रुचꣳ॒॒ रुच॑ मदधु॒र् यः । \newline
21. अ॒द॒धु॒र् यो यो॑ ऽदधु रदधु॒र् यो ब्र॑ह्मवर्च॒सका॑मो ब्रह्मवर्च॒सका॑मो॒ यो॑ ऽदधु रदधु॒र् यो ब्र॑ह्मवर्च॒सका॑मः । \newline
22. यो ब्र॑ह्मवर्च॒सका॑मो ब्रह्मवर्च॒सका॑मो॒ यो यो ब्र॑ह्मवर्च॒सका॑मः॒ स्याथ् स्याद् ब्र॑ह्मवर्च॒सका॑मो॒ यो यो ब्र॑ह्मवर्च॒सका॑मः॒ स्यात् । \newline
23. ब्र॒ह्म॒व॒र्च॒सका॑मः॒ स्याथ् स्याद् ब्र॑ह्मवर्च॒सका॑मो ब्रह्मवर्च॒सका॑मः॒ स्यात् तस्मै॒ तस्मै॒ स्याद् ब्र॑ह्मवर्च॒सका॑मो ब्रह्मवर्च॒सका॑मः॒ स्यात् तस्मै᳚ । \newline
24. ब्र॒ह्म॒व॒र्च॒सका॑म॒ इति॑ ब्रह्मवर्च॒स - का॒मः॒ । \newline
25. स्यात् तस्मै॒ तस्मै॒ स्याथ् स्यात् तस्मा॑ ए॒त मे॒तम् तस्मै॒ स्याथ् स्यात् तस्मा॑ ए॒तम् । \newline
26. तस्मा॑ ए॒त मे॒तम् तस्मै॒ तस्मा॑ ए॒तꣳ सो॑मारौ॒द्रꣳ सो॑मारौ॒द्र मे॒तम् तस्मै॒ तस्मा॑ ए॒तꣳ सो॑मारौ॒द्रम् । \newline
27. ए॒तꣳ सो॑मारौ॒द्रꣳ सो॑मारौ॒द्र मे॒त मे॒तꣳ सो॑मारौ॒द्रम् च॒रुम् च॒रुꣳ सो॑मारौ॒द्र मे॒त मे॒तꣳ सो॑मारौ॒द्रम् च॒रुम् । \newline
28. सो॒मा॒रौ॒द्रम् च॒रुम् च॒रुꣳ सो॑मारौ॒द्रꣳ सो॑मारौ॒द्रम् च॒रुम् निर् णिश्च॒रुꣳ सो॑मारौ॒द्रꣳ सो॑मारौ॒द्रम् च॒रुम् निः । \newline
29. सो॒मा॒रौ॒द्रमिति॑ सोमा - रौ॒द्रम् । \newline
30. च॒रुन्निर् णिश्च॒रुम् च॒रुम् निर् व॑पेद् वपे॒न् निश्च॒रुम् च॒रुम् निर् व॑पेत् । \newline
31. निर् व॑पेद् वपे॒न् निर् णिर् व॑पे॒थ् सोमꣳ॒॒ सोमं॑ ॅवपे॒न् निर् णिर् व॑पे॒थ् सोम᳚म् । \newline
32. व॒पे॒थ् सोमꣳ॒॒ सोमं॑ ॅवपेद् वपे॒थ् सोम॑म् च च॒ सोमं॑ ॅवपेद् वपे॒थ् सोम॑म् च । \newline
33. सोम॑म् च च॒ सोमꣳ॒॒ सोम॑म् चै॒वैव च॒ सोमꣳ॒॒ सोम॑म् चै॒व । \newline
34. चै॒वैव च॑ चै॒व रु॒द्रꣳ रु॒द्र मे॒व च॑ चै॒व रु॒द्रम् । \newline
35. ए॒व रु॒द्रꣳ रु॒द्र मे॒वैव रु॒द्रम् च॑ च रु॒द्र मे॒वैव रु॒द्रम् च॑ । \newline
36. रु॒द्रम् च॑ च रु॒द्रꣳ रु॒द्रम् च॒ स्वेन॒ स्वेन॑ च रु॒द्रꣳ रु॒द्रम् च॒ स्वेन॑ । \newline
37. च॒ स्वेन॒ स्वेन॑ च च॒ स्वेन॑ भाग॒धेये॑न भाग॒धेये॑न॒ स्वेन॑ च च॒ स्वेन॑ भाग॒धेये॑न । \newline
38. स्वेन॑ भाग॒धेये॑न भाग॒धेये॑न॒ स्वेन॒ स्वेन॑ भाग॒धेये॒नोपोप॑ भाग॒धेये॑न॒ स्वेन॒ स्वेन॑ भाग॒धेये॒नोप॑ । \newline
39. भा॒ग॒धेये॒नोपोप॑ भाग॒धेये॑न भाग॒धेये॒नोप॑ धावति धाव॒ त्युप॑ भाग॒धेये॑न भाग॒धेये॒नोप॑ धावति । \newline
40. भा॒ग॒धेये॒नेति॑ भाग - धेये॑न । \newline
41. उप॑ धावति धाव॒ त्युपोप॑ धावति॒ तौ तौ धा॑व॒ त्युपोप॑ धावति॒ तौ । \newline
42. धा॒व॒ति॒ तौ तौ धा॑वति धावति॒ ता वे॒वैव तौ धा॑वति धावति॒ ता वे॒व । \newline
43. ता वे॒वैव तौ ता वे॒वास्मि॑न् नस्मिन् ने॒व तौ ता वे॒वास्मिन्न्॑ । \newline
44. ए॒वास्मि॑न् नस्मिन् ने॒वैवास्मि॑न् ब्रह्मवर्च॒सम् ब्र॑ह्मवर्च॒स म॑स्मिन् ने॒वैवास्मि॑न् ब्रह्मवर्च॒सम् । \newline
45. अ॒स्मि॒न् ब्र॒ह्म॒व॒र्च॒सम् ब्र॑ह्मवर्च॒स म॑स्मिन् नस्मिन् ब्रह्मवर्च॒सम् ध॑त्तो धत्तो ब्रह्मवर्च॒स म॑स्मिन् नस्मिन् ब्रह्मवर्च॒सम् ध॑त्तः । \newline
46. ब्र॒ह्म॒व॒र्च॒सम् ध॑त्तो धत्तो ब्रह्मवर्च॒सम् ब्र॑ह्मवर्च॒सम् ध॑त्तो ब्रह्मवर्च॒सी ब्र॑ह्मवर्च॒सी ध॑त्तो ब्रह्मवर्च॒सम् ब्र॑ह्मवर्च॒सम् ध॑त्तो ब्रह्मवर्च॒सी । \newline
47. ब्र॒ह्म॒व॒र्च॒समिति॑ ब्रह्म - व॒र्च॒सम् । \newline
48. ध॒त्तो॒ ब्र॒ह्म॒व॒र्च॒सी ब्र॑ह्मवर्च॒सी ध॑त्तो धत्तो ब्रह्मवर्च॒ स्ये॑वैव ब्र॑ह्मवर्च॒सी ध॑त्तो धत्तो ब्रह्मवर्च॒ स्ये॑व । \newline
49. ब्र॒ह्म॒व॒र्च॒ स्ये॑वैव ब्र॑ह्मवर्च॒सी ब्र॑ह्मवर्च॒ स्ये॑व भ॑वति भवत्ये॒व ब्र॑ह्मवर्च॒सी ब्र॑ह्मवर्च॒ स्ये॑व भ॑वति । \newline
50. ब्र॒ह्म॒व॒र्च॒सीति॑ ब्रह्म - व॒र्च॒सी । \newline
51. ए॒व भ॑वति भवत्ये॒वैव भ॑वति तिष्यापूर्णमा॒से ति॑ष्यापूर्णमा॒से भ॑वत्ये॒वैव भ॑वति तिष्यापूर्णमा॒से । \newline
52. भ॒व॒ति॒ ति॒ष्या॒पू॒र्ण॒मा॒से ति॑ष्यापूर्णमा॒से भ॑वति भवति तिष्यापूर्णमा॒से निर् णिष् टि॑ष्यापूर्णमा॒से भ॑वति भवति तिष्यापूर्णमा॒से निः । \newline
53. ति॒ष्या॒पू॒र्ण॒मा॒से निर् णिष् टि॑ष्यापूर्णमा॒से ति॑ष्यापूर्णमा॒से निर् व॑पेद् वपे॒न् निष् टि॑ष्यापूर्णमा॒से ति॑ष्यापूर्णमा॒से निर् व॑पेत् । \newline
54. ति॒ष्या॒पू॒र्ण॒मा॒स इति॑ तिष्या - पू॒र्ण॒मा॒से । \newline
55. निर् व॑पेद् वपे॒न् निर् णिर् व॑पेद् रु॒द्रो रु॒द्रो व॑पे॒न् निर् णिर् व॑पेद् रु॒द्रः । \newline
56. व॒पे॒द् रु॒द्रो रु॒द्रो व॑पेद् वपेद् रु॒द्रो वै वै रु॒द्रो व॑पेद् वपेद् रु॒द्रो वै । \newline
57. रु॒द्रो वै वै रु॒द्रो रु॒द्रो वै ति॒ष्य॑ स्ति॒ष्यो॑ वै रु॒द्रो रु॒द्रो वै ति॒ष्यः॑ । \newline
\pagebreak
\markright{ TS 2.2.10.2  \hfill https://www.vedavms.in \hfill}
\addcontentsline{toc}{section}{ TS 2.2.10.2 }
\section*{ TS 2.2.10.2 }

\textbf{TS 2.2.10.2 } \newline
\textbf{Samhita Paata} \newline

वै ति॒ष्यः॑ सोमः॑ पू॒र्णमा॑सः सा॒क्षादे॒व ब्र॑ह्मवर्च॒समव॑ रुन्धे॒ परि॑श्रिते याजयति ब्रह्मवर्च॒सस्य॒ परि॑गृहीत्यै श्वे॒तायै᳚ श्वे॒तव॑थ्सायै दु॒ग्धं म॑थि॒तमाज्यं॑ भव॒त्याज्यं॒ प्रोक्ष॑ण॒माज्ये॑न मार्जयन्ते॒ याव॑दे॒व ब्र॑ह्मवर्च॒सं तथ् सर्वं॑ करो॒त्यति॑ ब्रह्मवर्च॒सं क्रि॑यत॒ इत्या॑हुरीश्व॒रो दु॒श्चर्मा॒ भवि॑तो॒रिति॑ मान॒वी ऋचौ॑ धा॒य्ये॑ कुर्या॒द्-यद्वै किञ्च॒ मनु॒रव॑द॒त्-तद्-भ॑ष॒जं - [  ] \newline

\textbf{Pada Paata} \newline

वै । ति॒ष्यः॑ । सोमः॑ । पू॒र्णमा॑स॒ इति॑ पू॒र्ण - मा॒सः॒ । सा॒क्षादिति॑ स - अ॒क्षात् । ए॒व । ब्र॒ह्म॒व॒र्च॒समिति॑ ब्रह्म - व॒र्च॒सम् । अवेति॑ । रु॒न्धे॒ । परि॑श्रित॒ इति॒ परि॑ - श्रि॒ते॒ । या॒ज॒य॒ति॒ । ब्र॒ह्म॒व॒र्च॒सस्येति॑ ब्रह्म - व॒र्च॒सस्य॑ । परि॑गृहीत्या॒ इति॒ परि॑ - गृ॒ही॒त्यै॒ । श्वे॒तायै᳚ । श्वे॒तव॑थ्साया॒ इति॑ श्वे॒त - व॒थ्सा॒यै॒ । दु॒ग्धम् । म॒थि॒तम् । आज्य᳚म् । भ॒व॒ति॒ । आज्य᳚म् । प्रोक्ष॑ण॒मिति॑ प्र - उक्ष॑णम् । आज्ये॑न । मा॒र्ज॒य॒न्ते॒ । याव॑त् । ए॒व । ब्र॒ह्म॒व॒र्च॒समिति॑ ब्रह्म - व॒र्च॒सम् । तत् । सर्व᳚म् । क॒रो॒ति॒ । अतीति॑ । ब्र॒ह्म॒व॒र्च॒समिति॑ ब्रह्म - व॒र्च॒सम् । क्रि॒य॒ते॒ । इति॑ । आ॒हुः॒ । ई॒श्व॒रः । दु॒श्चर्मेति॑ दुः - चर्मा᳚ । भवि॑तोः । इति॑ । मा॒न॒वी इति॑ । ऋचौ᳚ । धा॒य्ये॑ इति॑ । कु॒र्या॒त् । यत् । वै । किम् । च॒ । मनुः॑ । अव॑दत् । तत् । भे॒ष॒जम् ।  \newline


\textbf{Krama Paata} \newline

वै ति॒ष्यः॑ । ति॒ष्यः॑ सोमः॑ । सोमः॑ पू॒र्णमा॑सः । पू॒र्णमा॑सः सा॒क्षात् । पू॒र्णमा॑स॒ इति॑ पू॒र्ण - मा॒सः॒ । सा॒क्षादे॒व । सा॒क्षादिति॑ स - अ॒क्षात् । ए॒व ब्र॑ह्मवर्च॒सम् । ब्र॒ह्म॒व॒र्च॒समव॑ । ब्र॒ह्म॒व॒र्च॒समिति॑ ब्रह्म - व॒र्च॒सम् । अव॑ रुन्धे । रु॒न्धे॒ परि॑श्रिते । परि॑श्रिते याजयति । परि॑श्रित॒ इति॒ परि॑ - श्रि॒ते॒ । या॒ज॒य॒ति॒ ब्र॒ह्म॒व॒र्च॒सस्य॑ । ब्र॒ह्म॒व॒र्च॒सस्य॒ परि॑गृहीत्यै । ब्र॒ह्म॒व॒र्च॒सस्येति॑ ब्रह्म - व॒र्च॒सस्य॑ । परि॑गृहीत्यै श्वे॒तायै᳚ । परि॑गृहीत्या॒ इति॒ परि॑ - गृ॒ही॒त्यै॒ । श्वे॒तायै᳚ श्वे॒तव॑थ्स्यायै । श्वे॒तव॑थ्सायै दु॒ग्धम् । श्वे॒तव॑थ्साया॒ इति॑ श्वे॒त - व॒थ्सा॒यै॒ । दु॒ग्धम् म॑थि॒तम् । म॒थि॒तमाज्य᳚म् । आज्य॑म् भवति । भ॒व॒त्याज्य᳚म् । आज्य॒म् प्रोक्ष॑णम् । प्रोक्ष॑ण॒माज्ये॑न । प्रोक्ष॑ण॒मिति॑ प्र - उक्ष॑णम् । आज्ये॑न मार्जयन्ते । मा॒र्ज॒य॒न्ते॒ याव॑त् । याव॑दे॒व । ए॒व ब्र॑ह्मवर्च॒सम् । ब्र॒ह्म॒व॒र्च॒सम् तत् । ब्र॒ह्म॒व॒र्च॒समिति॑ ब्रह्म - व॒र्च॒सम् । तथ् सर्व᳚म् । सर्व॑म् करोति । क॒रो॒त्यति॑ । अति॑ ब्रह्मवर्च॒सम् । ब्र॒ह्म॒व॒र्च॒सम् क्रि॑यते । ब्र॒ह्म॒व॒र्च॒स॒मिति॑ ब्रह्म - व॒र्च॒सम् । क्रि॒य॒त॒ इति॑ । इत्या॑हुः । आ॒हु॒री॒श्व॒रः । ई॒श्व॒रो दु॒श्चर्मा᳚ । दु॒श्चर्मा॒ भवि॑तोः । दु॒श्चर्मेति॑ दुः - चर्मा᳚ । भवि॑तो॒रिति॑ । इति॑ मान॒वी । मा॒न॒वी ऋचौ᳚ । मा॒न॒वी इति॑ मान॒वी । ऋचौ॑ धा॒य्ये᳚ । धा॒य्ये॑ कुर्यात् । धा॒य्ये॑ इति॑ धा॒य्ये᳚ । कु॒र्या॒द् यत् । 
यद् वै । वै किम् । किम् च॑ । च॒ मनुः॑ । मनु॒रव॑दत् । अव॑द॒त् तत् । तद् भे॑ष॒जम् । भे॒ष॒जम् भे॑ष॒जम् \newline

\textbf{Jatai Paata} \newline

1. वै ति॒ष्य॑ स्ति॒ष्यो॑ वै वै ति॒ष्यः॑ । \newline
2. ति॒ष्यः॑ सोमः॒ सोम॑ स्ति॒ष्य॑ स्ति॒ष्यः॑ सोमः॑ । \newline
3. सोमः॑ पू॒र्णमा॑सः पू॒र्णमा॑सः॒ सोमः॒ सोमः॑ पू॒र्णमा॑सः । \newline
4. पू॒र्णमा॑सः सा॒क्षाथ् सा॒क्षात् पू॒र्णमा॑सः पू॒र्णमा॑सः सा॒क्षात् । \newline
5. पू॒र्णमा॑स॒ इति॑ पू॒र्ण - मा॒सः॒ । \newline
6. सा॒क्षा दे॒वैव सा॒क्षाथ् सा॒क्षा दे॒व । \newline
7. सा॒क्षादिति॑ स - अ॒क्षात् । \newline
8. ए॒व ब्र॑ह्मवर्च॒सम् ब्र॑ह्मवर्च॒स मे॒वैव ब्र॑ह्मवर्च॒सम् । \newline
9. ब्र॒ह्म॒व॒र्च॒स मवाव॑ ब्रह्मवर्च॒सम् ब्र॑ह्मवर्च॒स मव॑ । \newline
10. ब्र॒ह्म॒व॒र्च॒समिति॑ ब्रह्म - व॒र्च॒सम् । \newline
11. अव॑ रुन्धे रु॒न्धे ऽवाव॑ रुन्धे । \newline
12. रु॒न्धे॒ परि॑श्रिते॒ परि॑श्रिते रुन्धे रुन्धे॒ परि॑श्रिते । \newline
13. परि॑श्रिते याजयति याजयति॒ परि॑श्रिते॒ परि॑श्रिते याजयति । \newline
14. परि॑श्रित॒ इति॒ परि॑ - श्रि॒ते॒ । \newline
15. या॒ज॒य॒ति॒ ब्र॒ह्म॒व॒र्च॒सस्य॑ ब्रह्मवर्च॒सस्य॑ याजयति याजयति ब्रह्मवर्च॒सस्य॑ । \newline
16. ब्र॒ह्म॒व॒र्च॒सस्य॒ परि॑गृहीत्यै॒ परि॑गृहीत्यै ब्रह्मवर्च॒सस्य॑ ब्रह्मवर्च॒सस्य॒ परि॑गृहीत्यै । \newline
17. ब्र॒ह्म॒व॒र्च॒सस्येति॑ ब्रह्म - व॒र्च॒सस्य॑ । \newline
18. परि॑गृहीत्यै श्वे॒तायै᳚ श्वे॒तायै॒ परि॑गृहीत्यै॒ परि॑गृहीत्यै श्वे॒तायै᳚ । \newline
19. परि॑गृहीत्या॒ इति॒ परि॑ - गृ॒ही॒त्यै॒ । \newline
20. श्वे॒तायै᳚ श्वे॒तव॑थ्सायै श्वे॒तव॑थ्सायै श्वे॒तायै᳚ श्वे॒तायै᳚ श्वे॒तव॑थ्सायै । \newline
21. श्वे॒तव॑थ्सायै दु॒ग्धम् दु॒ग्धꣳ श्वे॒तव॑थ्सायै श्वे॒तव॑थ्सायै दु॒ग्धम् । \newline
22. श्वे॒तव॑थ्साया॒ इति॑ श्वे॒त - व॒थ्सा॒यै॒ । \newline
23. दु॒ग्धम् म॑थि॒तम् म॑थि॒तम् दु॒ग्धम् दु॒ग्धम् म॑थि॒तम् । \newline
24. म॒थि॒त माज्य॒ माज्य॑म् मथि॒तम् म॑थि॒त माज्य᳚म् । \newline
25. आज्य॑म् भवति भव॒त्याज्य॒ माज्य॑म् भवति । \newline
26. भ॒व॒ त्याज्य॒ माज्य॑म् भवति भव॒ त्याज्य᳚म् । \newline
27. आज्य॒म् प्रोक्ष॑ण॒म् प्रोक्ष॑ण॒ माज्य॒ माज्य॒म् प्रोक्ष॑णम् । \newline
28. प्रोक्ष॑ण॒ माज्ये॒नाज्ये॑न॒ प्रोक्ष॑ण॒म् प्रोक्ष॑ण॒ माज्ये॑न । \newline
29. प्रोक्ष॑ण॒मिति॑ प्र - उक्ष॑णम् । \newline
30. आज्ये॑न मार्जयन्ते मार्जयन्त॒ आज्ये॒ नाज्ये॑न मार्जयन्ते । \newline
31. मा॒र्ज॒य॒न्ते॒ याव॒द् याव॑न् मार्जयन्ते मार्जयन्ते॒ याव॑त् । \newline
32. याव॑ दे॒वैव याव॒द् याव॑ दे॒व । \newline
33. ए॒व ब्र॑ह्मवर्च॒सम् ब्र॑ह्मवर्च॒स मे॒वैव ब्र॑ह्मवर्च॒सम् । \newline
34. ब्र॒ह्म॒व॒र्च॒सम् तत् तद् ब्र॑ह्मवर्च॒सम् ब्र॑ह्मवर्च॒सम् तत् । \newline
35. ब्र॒ह्म॒व॒र्च॒समिति॑ ब्रह्म - व॒र्च॒सम् । \newline
36. तथ् सर्वꣳ॒॒ सर्व॒म् तत् तथ् सर्व᳚म् । \newline
37. सर्व॑म् करोति करोति॒ सर्वꣳ॒॒ सर्व॑म् करोति । \newline
38. क॒रो॒ त्यत्यति॑ करोति करो॒ त्यति॑ । \newline
39. अति॑ ब्रह्मवर्च॒सम् ब्र॑ह्मवर्च॒स मत्यति॑ ब्रह्मवर्च॒सम् । \newline
40. ब्र॒ह्म॒व॒र्च॒सम् क्रि॑यते क्रियते ब्रह्मवर्च॒सम् ब्र॑ह्मवर्च॒सम् क्रि॑यते । \newline
41. ब्र॒ह्म॒व॒र्च॒समिति॑ ब्रह्म - व॒र्च॒सम् । \newline
42. क्रि॒य॒त॒ इतीति॑ क्रियते क्रियत॒ इति॑ । \newline
43. इत्या॑हु राहु॒ रिती त्या॑हुः । \newline
44. आ॒हु॒ री॒श्व॒र ई᳚श्व॒र आ॑हु राहु रीश्व॒रः । \newline
45. ई॒श्व॒रो दु॒श्चर्मा॑ दु॒श्चर्मे᳚श्व॒र ई᳚श्व॒रो दु॒श्चर्मा᳚ । \newline
46. दु॒श्चर्मा॒ भवि॑तो॒र् भवि॑तोर् दु॒श्चर्मा॑ दु॒श्चर्मा॒ भवि॑तोः । \newline
47. दु॒श्चर्मेति॑ दुः - चर्मा᳚ । \newline
48. भवि॑तो॒ रितीति॒ भवि॑तो॒र् भवि॑तो॒ रिति॑ । \newline
49. इति॑ मान॒वी मा॑न॒वी इतीति॑ मान॒वी । \newline
50. मा॒न॒वी ऋचा॒ वृचौ॑ मान॒वी मा॑न॒वी ऋचौ᳚ । \newline
51. मा॒न॒वी इति॑ मान॒वी । \newline
52. ऋचौ॑ धा॒य्ये॑ धा॒य्ये॑ ऋचा॒ वृचौ॑ धा॒य्ये᳚ । \newline
53. धा॒य्ये॑ कुर्यात् कुर्याद् धा॒य्ये॑ धा॒य्ये॑ कुर्यात् । \newline
54. धा॒य्ये॑ इति॑ धा॒य्ये᳚ । \newline
55. कु॒र्या॒द् यद् यत् कु॑र्यात् कुर्या॒द् यत् । \newline
56. यद् वै वै यद् यद् वै । \newline
57. वै किम् किं ॅवै वै किम् । \newline
58. किम् च॑ च॒ किम् किम् च॑ । \newline
59. च॒ मनु॒र् मनु॑श्च च॒ मनुः॑ । \newline
60. मनु॒ रव॑द॒ दव॑द॒न् मनु॒र् मनु॒रव॑दत् । \newline
61. अव॑द॒त् तत् तदव॑ द॒दव॑द॒त् तत् । \newline
62. तद् भे॑ष॒जम् भे॑ष॒जम् तत् तद् भे॑ष॒जम् । \newline
63. भे॒ष॒जम् भे॑ष॒जम् । \newline

\textbf{Ghana Paata } \newline

1. वै ति॒ष्य॑ स्ति॒ष्यो॑ वै वै ति॒ष्यः॑ सोमः॒ सोम॑ स्ति॒ष्यो॑ वै वै ति॒ष्यः॑ सोमः॑ । \newline
2. ति॒ष्यः॑ सोमः॒ सोम॑ स्ति॒ष्य॑ स्ति॒ष्यः॑ सोमः॑ पू॒र्णमा॑सः पू॒र्णमा॑सः॒ सोम॑ स्ति॒ष्य॑ स्ति॒ष्यः॑ सोमः॑ पू॒र्णमा॑सः । \newline
3. सोमः॑ पू॒र्णमा॑सः पू॒र्णमा॑सः॒ सोमः॒ सोमः॑ पू॒र्णमा॑सः सा॒क्षाथ् सा॒क्षात् पू॒र्णमा॑सः॒ सोमः॒ सोमः॑ पू॒र्णमा॑सः सा॒क्षात् । \newline
4. पू॒र्णमा॑सः सा॒क्षाथ् सा॒क्षात् पू॒र्णमा॑सः पू॒र्णमा॑सः सा॒क्षा दे॒वैव सा॒क्षात् पू॒र्णमा॑सः पू॒र्णमा॑सः सा॒क्षादे॒व । \newline
5. पू॒र्णमा॑स॒ इति॑ पू॒र्ण - मा॒सः॒ । \newline
6. सा॒क्षादे॒वैव सा॒क्षाथ् सा॒क्षादे॒व ब्र॑ह्मवर्च॒सम् ब्र॑ह्मवर्च॒स मे॒व सा॒क्षाथ् सा॒क्षादे॒व ब्र॑ह्मवर्च॒सम् । \newline
7. सा॒क्षादिति॑ स - अ॒क्षात् । \newline
8. ए॒व ब्र॑ह्मवर्च॒सम् ब्र॑ह्मवर्च॒स मे॒वैव ब्र॑ह्मवर्च॒स मवाव॑ ब्रह्मवर्च॒स मे॒वैव ब्र॑ह्मवर्च॒स मव॑ । \newline
9. ब्र॒ह्म॒व॒र्च॒स मवाव॑ ब्रह्मवर्च॒सम् ब्र॑ह्मवर्च॒स मव॑ रुन्धे रु॒न्धे ऽव॑ ब्रह्मवर्च॒सम् ब्र॑ह्मवर्च॒स मव॑ रुन्धे । \newline
10. ब्र॒ह्म॒व॒र्च॒समिति॑ ब्रह्म - व॒र्च॒सम् । \newline
11. अव॑ रुन्धे रु॒न्धे ऽवाव॑ रुन्धे॒ परि॑श्रिते॒ परि॑श्रिते रु॒न्धे ऽवाव॑ रुन्धे॒ परि॑श्रिते । \newline
12. रु॒न्धे॒ परि॑श्रिते॒ परि॑श्रिते रुन्धे रुन्धे॒ परि॑श्रिते याजयति याजयति॒ परि॑श्रिते रुन्धे रुन्धे॒ परि॑श्रिते याजयति । \newline
13. परि॑श्रिते याजयति याजयति॒ परि॑श्रिते॒ परि॑श्रिते याजयति ब्रह्मवर्च॒सस्य॑ ब्रह्मवर्च॒सस्य॑ याजयति॒ परि॑श्रिते॒ परि॑श्रिते याजयति ब्रह्मवर्च॒सस्य॑ । \newline
14. परि॑श्रित॒ इति॒ परि॑ - श्रि॒ते॒ । \newline
15. या॒ज॒य॒ति॒ ब्र॒ह्म॒व॒र्च॒सस्य॑ ब्रह्मवर्च॒सस्य॑ याजयति याजयति ब्रह्मवर्च॒सस्य॒ परि॑गृहीत्यै॒ परि॑गृहीत्यै ब्रह्मवर्च॒सस्य॑ याजयति याजयति ब्रह्मवर्च॒सस्य॒ परि॑गृहीत्यै । \newline
16. ब्र॒ह्म॒व॒र्च॒सस्य॒ परि॑गृहीत्यै॒ परि॑गृहीत्यै ब्रह्मवर्च॒सस्य॑ ब्रह्मवर्च॒सस्य॒ परि॑गृहीत्यै श्वे॒तायै᳚ श्वे॒तायै॒ परि॑गृहीत्यै ब्रह्मवर्च॒सस्य॑ ब्रह्मवर्च॒सस्य॒ परि॑गृहीत्यै श्वे॒तायै᳚ । \newline
17. ब्र॒ह्म॒व॒र्च॒सस्येति॑ ब्रह्म - व॒र्च॒सस्य॑ । \newline
18. परि॑गृहीत्यै श्वे॒तायै᳚ श्वे॒तायै॒ परि॑गृहीत्यै॒ परि॑गृहीत्यै श्वे॒तायै᳚ श्वे॒तव॑थ्सायै श्वे॒तव॑थ्सायै श्वे॒तायै॒ परि॑गृहीत्यै॒ परि॑गृहीत्यै श्वे॒तायै᳚ श्वे॒तव॑थ्सायै । \newline
19. परि॑गृहीत्या॒ इति॒ परि॑ - गृ॒ही॒त्यै॒ । \newline
20. श्वे॒तायै᳚ श्वे॒तव॑थ्सायै श्वे॒तव॑थ्सायै श्वे॒तायै᳚ श्वे॒तायै᳚ श्वे॒तव॑थ्सायै दु॒ग्धम् दु॒ग्धꣳ श्वे॒तव॑थ्सायै श्वे॒तायै᳚ श्वे॒तायै᳚ श्वे॒तव॑थ्सायै दु॒ग्धम् । \newline
21. श्वे॒तव॑थ्सायै दु॒ग्धम् दु॒ग्धꣳ श्वे॒तव॑थ्सायै श्वे॒तव॑थ्सायै दु॒ग्धम् म॑थि॒तम् म॑थि॒तम् दु॒ग्धꣳ श्वे॒तव॑थ्सायै श्वे॒तव॑थ्सायै दु॒ग्धम् म॑थि॒तम् । \newline
22. श्वे॒तव॑थ्साया॒ इति॑ श्वे॒त - व॒थ्सा॒यै॒ । \newline
23. दु॒ग्धम् म॑थि॒तम् म॑थि॒तम् दु॒ग्धम् दु॒ग्धम् म॑थि॒त माज्य॒ माज्य॑म् मथि॒तम् दु॒ग्धम् दु॒ग्धम् म॑थि॒त माज्य᳚म् । \newline
24. म॒थि॒त माज्य॒ माज्य॑म् मथि॒तम् म॑थि॒त माज्य॑म् भवति भव॒ त्याज्य॑म् मथि॒तम् म॑थि॒त माज्य॑म् भवति । \newline
25. आज्य॑म् भवति भव॒ त्याज्य॒ माज्य॑म् भव॒ त्याज्य॒ माज्य॑म् भव॒ त्याज्य॒ माज्य॑म् भव॒ त्याज्य᳚म् । \newline
26. भ॒व॒ त्याज्य॒ माज्य॑म् भवति भव॒ त्याज्य॒म् प्रोक्ष॑ण॒म् प्रोक्ष॑ण॒ माज्य॑म् भवति भव॒ त्याज्य॒म् प्रोक्ष॑णम् । \newline
27. आज्य॒म् प्रोक्ष॑ण॒म् प्रोक्ष॑ण॒ माज्य॒ माज्य॒म् प्रोक्ष॑ण॒ माज्ये॒ नाज्ये॑न॒ प्रोक्ष॑ण॒ माज्य॒ माज्य॒म् प्रोक्ष॑ण॒ माज्ये॑न । \newline
28. प्रोक्ष॑ण॒ माज्ये॒ नाज्ये॑न॒ प्रोक्ष॑ण॒म् प्रोक्ष॑ण॒ माज्ये॑न मार्जयन्ते मार्जयन्त॒ आज्ये॑न॒ प्रोक्ष॑ण॒म् प्रोक्ष॑ण॒ माज्ये॑न मार्जयन्ते । \newline
29. प्रोक्ष॑ण॒मिति॑ प्र - उक्ष॑णम् । \newline
30. आज्ये॑न मार्जयन्ते मार्जयन्त॒ आज्ये॒ नाज्ये॑न मार्जयन्ते॒ याव॒द् याव॑न् मार्जयन्त॒ आज्ये॒ नाज्ये॑न मार्जयन्ते॒ याव॑त् । \newline
31. मा॒र्ज॒य॒न्ते॒ याव॒द् याव॑न् मार्जयन्ते मार्जयन्ते॒ याव॑दे॒वैव याव॑न् मार्जयन्ते मार्जयन्ते॒ याव॑दे॒व । \newline
32. याव॑दे॒वैव याव॒द् याव॑दे॒व ब्र॑ह्मवर्च॒सम् ब्र॑ह्मवर्च॒स मे॒व याव॒द् याव॑दे॒व ब्र॑ह्मवर्च॒सम् । \newline
33. ए॒व ब्र॑ह्मवर्च॒सम् ब्र॑ह्मवर्च॒स मे॒वैव ब्र॑ह्मवर्च॒सम् तत् तद् ब्र॑ह्मवर्च॒स मे॒वैव ब्र॑ह्मवर्च॒सम् तत् । \newline
34. ब्र॒ह्म॒व॒र्च॒सम् तत् तद् ब्र॑ह्मवर्च॒सम् ब्र॑ह्मवर्च॒सम् तथ् सर्वꣳ॒॒ सर्व॒म् तद् ब्र॑ह्मवर्च॒सम् ब्र॑ह्मवर्च॒सम् तथ् सर्व᳚म् । \newline
35. ब्र॒ह्म॒व॒र्च॒समिति॑ ब्रह्म - व॒र्च॒सम् । \newline
36. तथ् सर्वꣳ॒॒ सर्व॒म् तत् तथ् सर्व॑म् करोति करोति॒ सर्व॒म् तत् तथ् सर्व॑म् करोति । \newline
37. सर्व॑म् करोति करोति॒ सर्वꣳ॒॒ सर्व॑म् करो॒ त्य त्यति॑ करोति॒ सर्वꣳ॒॒ सर्व॑म् करो॒ त्यति॑ । \newline
38. क॒रो॒ त्य त्यति॑ करोति करो॒ त्यति॑ ब्रह्मवर्च॒सम् ब्र॑ह्मवर्च॒स मति॑ करोति करो॒ त्यति॑ ब्रह्मवर्च॒सम् । \newline
39. अति॑ ब्रह्मवर्च॒सम् ब्र॑ह्मवर्च॒स मत्यति॑ ब्रह्मवर्च॒सम् क्रि॑यते क्रियते ब्रह्मवर्च॒स मत्यति॑ ब्रह्मवर्च॒सम् क्रि॑यते । \newline
40. ब्र॒ह्म॒व॒र्च॒सम् क्रि॑यते क्रियते ब्रह्मवर्च॒सम् ब्र॑ह्मवर्च॒सम् क्रि॑यत॒ इतीति॑ क्रियते ब्रह्मवर्च॒सम् ब्र॑ह्मवर्च॒सम् क्रि॑यत॒ इति॑ । \newline
41. ब्र॒ह्म॒व॒र्च॒समिति॑ ब्रह्म - व॒र्च॒सम् । \newline
42. क्रि॒य॒त॒ इतीति॑ क्रियते क्रियत॒ इत्या॑हु राहु॒ रिति॑ क्रियते क्रियत॒ इत्या॑हुः । \newline
43. इत्या॑हु राहु॒ रिती त्या॑हु रीश्व॒र ई᳚श्व॒र आ॑हु॒ रिती त्या॑हु रीश्व॒रः । \newline
44. आ॒हु॒ री॒श्व॒र ई᳚श्व॒र आ॑हु राहु रीश्व॒रो दु॒श्चर्मा॑ दु॒श्चर्मे᳚श्व॒र आ॑हु राहु रीश्व॒रो दु॒श्चर्मा᳚ । \newline
45. ई॒श्व॒रो दु॒श्चर्मा॑ दु॒श्चर्मे᳚श्व॒र ई᳚श्व॒रो दु॒श्चर्मा॒ भवि॑तो॒र् भवि॑तोर् दु॒श्चर्मे᳚श्व॒र ई᳚श्व॒रो दु॒श्चर्मा॒ भवि॑तोः । \newline
46. दु॒श्चर्मा॒ भवि॑तो॒र् भवि॑तोर् दु॒श्चर्मा॑ दु॒श्चर्मा॒ भवि॑तो॒ रितीति॒ भवि॑तोर् दु॒श्चर्मा॑ दु॒श्चर्मा॒ भवि॑तो॒ रिति॑ । \newline
47. दु॒श्चर्मेति॑ दुः - चर्मा᳚ । \newline
48. भवि॑तो॒ रितीति॒ भवि॑तो॒र् भवि॑तो॒ रिति॑ मान॒वी मा॑न॒वी इति॒ भवि॑तो॒र् भवि॑तो॒ रिति॑ मान॒वी । \newline
49. इति॑ मान॒वी मा॑न॒वी इतीति॑ मान॒वी ऋचा॒ वृचौ॑ मान॒वी इतीति॑ मान॒वी ऋचौ᳚ । \newline
50. मा॒न॒वी ऋचा॒ वृचौ॑ मान॒वी मा॑न॒वी ऋचौ॑ धा॒य्ये॑ धा॒य्ये॑ ऋचौ॑ मान॒वी मा॑न॒वी ऋचौ॑ धा॒य्ये᳚ । \newline
51. मा॒न॒वी इति॑ मान॒वी । \newline
52. ऋचौ॑ धा॒य्ये॑ धा॒य्ये॑ ऋचा॒ वृचौ॑ धा॒य्ये॑ कुर्यात् कुर्याद् धा॒य्ये॑ ऋचा॒ वृचौ॑ धा॒य्ये॑ कुर्यात् । \newline
53. धा॒य्ये॑ कुर्यात् कुर्याद् धा॒य्ये॑ धा॒य्ये॑ कुर्या॒द् यद् यत् कु॑र्याद् धा॒य्ये॑ धा॒य्ये॑ कुर्या॒द् यत् । \newline
54. धा॒य्ये॑ इति॑ धा॒य्ये᳚ । \newline
55. कु॒र्या॒द् यद् यत् कु॑र्यात् कुर्या॒द् यद् वै वै यत् कु॑र्यात् कुर्या॒द् यद् वै । \newline
56. यद् वै वै यद् यद् वै किम् किं ॅवै यद् यद् वै किम् । \newline
57. वै किम् किं ॅवै वै किम् च॑ च॒ किं ॅवै वै किम् च॑ । \newline
58. किम् च॑ च॒ किम् किम् च॒ मनु॒र् मनु॑श्च॒ किम् किम् च॒ मनुः॑ । \newline
59. च॒ मनु॒र् मनु॑श्च च॒ मनु॒ रव॑द॒ दव॑द॒न् मनु॑श्च च॒ मनु॒ रव॑दत् । \newline
60. मनु॒ रव॑द॒ दव॑द॒न् मनु॒र् मनु॒ रव॑द॒त् तत् तदव॑द॒न् मनु॒र् मनु॒ रव॑द॒त् तत् । \newline
61. अव॑द॒त् तत् तदव॑ द॒दव॑ द॒त् तद् भे॑ष॒जम् भे॑ष॒जम् तदव॑ द॒दव॑ द॒त् तद् भे॑ष॒जम् । \newline
62. तद् भे॑ष॒जम् भे॑ष॒जम् तत् तद् भे॑ष॒जम् । \newline
63. भे॒ष॒जम् भे॑ष॒जम् । \newline
\pagebreak
\markright{ TS 2.2.10.3  \hfill https://www.vedavms.in \hfill}
\addcontentsline{toc}{section}{ TS 2.2.10.3 }
\section*{ TS 2.2.10.3 }

\textbf{TS 2.2.10.3 } \newline
\textbf{Samhita Paata} \newline

भे॑ष॒जमे॒वास्मै॑ करोति॒ यदि॑ बिभी॒याद्-दु॒श्चर्मा॑ भविष्या॒मीति॑ सोमापौ॒ष्णं च॒रुं निर्व॑पेथ् सौ॒म्यो वै दे॒वत॑या॒ पुरु॑षः पौ॒ष्णाः प॒शवः॒ स्वयै॒ वास्मै॑ दे॒वत॑या प॒शुभि॒स्त्वचं॑ करोति॒ न दु॒श्चर्मा॑ भवति सोमारौ॒द्रं च॒रुं निर्व॑पेत् प्र॒जाका॑मः॒ सोमो॒ वै रे॑तो॒धा अ॒ग्निः प्र॒जानां᳚ प्रजनयि॒ता सोम॑ ए॒वास्मै॒ रेतो॒ दधा᳚त्य॒ग्निः प्र॒जां प्रज॑नयति वि॒न्दते᳚ - [  ] \newline

\textbf{Pada Paata} \newline

भे॒ष॒जम् । ए॒व । अ॒स्मै॒ । क॒रो॒ति॒ । यदि॑ । बि॒भी॒यात् । दु॒श्चर्मेति॑ दुः - चर्मा᳚ । भ॒वि॒ष्या॒मि॒ । इति॑ । सो॒मा॒पौ॒ष्णमिति॑ सोमा - पौ॒ष्णम् । च॒रुम् । निरिति॑ । व॒पे॒त् । सौ॒म्यः । वै । दे॒वत॑या । पुरु॑षः । पौ॒ष्णाः । प॒शवः॑ । स्वया᳚ । ए॒व । अ॒स्मै॒ । दे॒वत॑या । प॒शुभि॒रिति॑ प॒शु - भिः॒ । त्वच᳚म् । क॒रो॒ति॒ । न । दु॒श्चर्मेति॑ दुः - चर्मा᳚ । भ॒व॒ति॒ । सो॒मा॒रौ॒द्रमिति॑ सोमा - रौ॒द्रम् । च॒रुम् । निरिति॑ । व॒पे॒त् । प्र॒जाका॑म॒ इति॑ प्र॒जा - का॒मः॒ । सोमः॑ । वै । रे॒तो॒धा इति॑ रेतः - धाः । अ॒ग्निः । प्र॒जाना॒मिति॑ प्र - जाना᳚म् । प्र॒ज॒न॒यि॒तेति॑ प्र - ज॒न॒यि॒ता । सोमः॑ । ए॒व । अ॒स्मै॒ । रेतः॑ । दधा॑ति । अ॒ग्निः । प्र॒जामिति॑ प्र - जाम् । प्रेति॑ । ज॒न॒य॒ति॒ । वि॒न्दते᳚ ।  \newline


\textbf{Krama Paata} \newline

भे॒ष॒जमे॒व । ए॒वास्मै᳚ । अ॒स्मै॒ क॒रो॒ति॒ । क॒रो॒ति॒ यदि॑ । यदि॑ बिभी॒यात् । बि॒भी॒याद् दु॒श्चर्मा᳚ । दु॒श्चर्मा॑ भविष्यामि । दु॒श्चर्मेति॑ दुः - चर्मा᳚ । भ॒वि॒ष्या॒मीति॑ । इति॑ सोमापौ॒ष्णम् । सो॒मा॒पौ॒ष्णम् च॒रुम् । सो॒मा॒पौ॒ष्णमिति॑ सोमा - पौ॒ष्णम् । च॒रुम् निः । निर् व॑पेत् । व॒पे॒थ् सौ॒म्यः । सौ॒म्यो वै । वै दे॒वत॑या । दे॒वत॑या॒ पुरु॑षः । पुरु॑षः पौ॒ष्णाः । पौ॒ष्णाः प॒शवः॑ । प॒शवः॒ स्वया᳚ । स्वयै॒व । ए॒वास्मै᳚ । अ॒स्मै॒ दे॒वत॑या । दे॒वत॑या प॒शुभिः॑ । प॒शुभि॒ स्त्वच᳚म् । प॒शुभि॒रिति॑ प॒शु - भिः॒ । त्वच॑म् करोति । क॒रो॒ति॒ न । न दु॒श्चर्मा᳚ । दु॒श्चर्मा॑ भवति । दु॒श्चर्मेति॑ दुः - चर्मा᳚ । भ॒व॒ति॒ सो॒मा॒रौ॒द्रम् । सो॒मा॒रौ॒द्रम् - च॒रुम् । सो॒मा॒रौ॒द्रमिति॑ सोमा - रौ॒द्रम् । च॒रुम् निः । निर् व॑पेत् । व॒पे॒त्,प्र॒जाका॑मः । प्र॒जाका॑मः॒ सोमः॑ । प्र॒जाका॑म॒ इति॑ प्र॒जा - का॒मः॒ । सोमो॒ वै । वै रे॑तो॒धाः । रे॒तो॒धा अ॒ग्निः । रे॒तो॒धा इति॑ रेतः - धाः । अ॒ग्निः प्र॒जाना᳚म् । प्र॒जाना᳚म् प्रजनयि॒ता । प्र॒जाना॒मिति॑ प्र - जाना᳚म् । प्र॒ज॒न॒यि॒ता सोमः॑ । प्र॒ज॒न॒यि॒तेति॑ प्र - ज॒न॒यि॒ता । सोम॑ ए॒व । ए॒वास्मै᳚ । अ॒स्मै॒ रेतः॑ । रेतो॒ दधा॑ति । दधा᳚त्य॒ग्निः । अ॒ग्निः प्र॒जाम् । प्र॒जाम् प्र । प्र॒जामिति॑ प्र - जाम् । प्र ज॑नयति । ज॒न॒य॒ति॒ वि॒न्दते᳚ । वि॒न्दते᳚ प्र॒जाम् \newline

\textbf{Jatai Paata} \newline

1. भे॒ष॒ज मे॒वैव भे॑ष॒जम् भे॑ष॒ज मे॒व । \newline
2. ए॒वास्मा॑ अस्मा ए॒वैवास्मै᳚ । \newline
3. अ॒स्मै॒ क॒रो॒ति॒ क॒रो॒ त्य॒स्मा॒ अ॒स्मै॒ क॒रो॒ति॒ । \newline
4. क॒रो॒ति॒ यदि॒ यदि॑ करोति करोति॒ यदि॑ । \newline
5. यदि॑ बिभी॒याद् बि॑भी॒याद् यदि॒ यदि॑ बिभी॒यात् । \newline
6. बि॒भी॒याद् दु॒श्चर्मा॑ दु॒श्चर्मा॑ बिभी॒याद् बि॑भी॒याद् दु॒श्चर्मा᳚ । \newline
7. दु॒श्चर्मा॑ भविष्यामि भविष्यामि दु॒श्चर्मा॑ दु॒श्चर्मा॑ भविष्यामि । \newline
8. दु॒श्चर्मेति॑ दुः - चर्मा᳚ । \newline
9. भ॒वि॒ष्या॒मीतीति॑ भविष्यामि भविष्या॒मीति॑ । \newline
10. इति॑ सोमापौ॒ष्णꣳ सो॑मापौ॒ष्ण मितीति॑ सोमापौ॒ष्णम् । \newline
11. सो॒मा॒पौ॒ष्णम् च॒रुम् च॒रुꣳ सो॑मापौ॒ष्णꣳ सो॑मापौ॒ष्णम् च॒रुम् । \newline
12. सो॒मा॒पौ॒ष्णमिति॑ सोमा - पौ॒ष्णम् । \newline
13. च॒रुम् निर् णिश्च॒रुम् च॒रुम् निः । \newline
14. निर् व॑पेद् वपे॒न् निर् णिर् व॑पेत् । \newline
15. व॒पे॒थ् सौ॒म्यः सौ॒म्यो व॑पेद् वपेथ् सौ॒म्यः । \newline
16. सौ॒म्यो वै वै सौ॒म्यः सौ॒म्यो वै । \newline
17. वै दे॒वत॑या दे॒वत॑या॒ वै वै दे॒वत॑या । \newline
18. दे॒वत॑या॒ पुरु॑षः॒ पुरु॑षो दे॒वत॑या दे॒वत॑या॒ पुरु॑षः । \newline
19. पुरु॑षः पौ॒ष्णाः पौ॒ष्णाः पुरु॑षः॒ पुरु॑षः पौ॒ष्णाः । \newline
20. पौ॒ष्णाः प॒शवः॑ प॒शवः॑ पौ॒ष्णाः पौ॒ष्णाः प॒शवः॑ । \newline
21. प॒शवः॒ स्वया॒ स्वया॑ प॒शवः॑ प॒शवः॒ स्वया᳚ । \newline
22. स्वयै॒वैव स्वया॒ स्वयै॒व । \newline
23. ए॒वास्मा॑ अस्मा ए॒वैवास्मै᳚ । \newline
24. अ॒स्मै॒ दे॒वत॑या दे॒वत॑या ऽस्मा अस्मै दे॒वत॑या । \newline
25. दे॒वत॑या प॒शुभिः॑ प॒शुभि॑र् दे॒वत॑या दे॒वत॑या प॒शुभिः॑ । \newline
26. प॒शुभि॒ स्त्वच॒म् त्वच॑म् प॒शुभिः॑ प॒शुभि॒ स्त्वच᳚म् । \newline
27. प॒शुभि॒रिति॑ प॒शु - भिः॒ । \newline
28. त्वच॑म् करोति करोति॒ त्वच॒म् त्वच॑म् करोति । \newline
29. क॒रो॒ति॒ न न क॑रोति करोति॒ न । \newline
30. न दु॒श्चर्मा॑ दु॒श्चर्मा॒ न न दु॒श्चर्मा᳚ । \newline
31. दु॒श्चर्मा॑ भवति भवति दु॒श्चर्मा॑ दु॒श्चर्मा॑ भवति । \newline
32. दु॒श्चर्मेति॑ दुः - चर्मा᳚ । \newline
33. भ॒व॒ति॒ सो॒मा॒रौ॒द्रꣳ सो॑मारौ॒द्रम् भ॑वति भवति सोमारौ॒द्रम् । \newline
34. सो॒मा॒रौ॒द्रम् च॒रुम् च॒रुꣳ सो॑मारौ॒द्रꣳ सो॑मारौ॒द्रम् च॒रुम् । \newline
35. सो॒मा॒रौ॒द्रमिति॑ सोमा - रौ॒द्रम् । \newline
36. च॒रुम् निर् णिश्च॒रुम् च॒रुम् निः । \newline
37. निर् व॑पेद् वपे॒न् निर् णिर् व॑पेत् । \newline
38. व॒पे॒त् प्र॒जाका॑मः प्र॒जाका॑मो वपेद् वपेत् प्र॒जाका॑मः । \newline
39. प्र॒जाका॑मः॒ सोमः॒ सोमः॑ प्र॒जाका॑मः प्र॒जाका॑मः॒ सोमः॑ । \newline
40. प्र॒जाका॑म॒ इति॑ प्र॒जा - का॒मः॒ । \newline
41. सोमो॒ वै वै सोमः॒ सोमो॒ वै । \newline
42. वै रे॑तो॒धा रे॑तो॒धा वै वै रे॑तो॒धाः । \newline
43. रे॒तो॒धा अ॒ग्नि र॒ग्नी रे॑तो॒धा रे॑तो॒धा अ॒ग्निः । \newline
44. रे॒तो॒धा इति॑ रेतः - धाः । \newline
45. अ॒ग्निः प्र॒जाना᳚म् प्र॒जाना॑ म॒ग्निर॒ग्निः प्र॒जाना᳚म् । \newline
46. प्र॒जाना᳚म् प्रजनयि॒ता प्र॑जनयि॒ता प्र॒जाना᳚म् प्र॒जाना᳚म् प्रजनयि॒ता । \newline
47. प्र॒जाना॒मिति॑ प्र - जाना᳚म् । \newline
48. प्र॒ज॒न॒यि॒ता सोमः॒ सोमः॑ प्रजनयि॒ता प्र॑जनयि॒ता सोमः॑ । \newline
49. प्र॒ज॒न॒यि॒तेति॑ प्र - ज॒न॒यि॒ता । \newline
50. सोम॑ ए॒वैव सोमः॒ सोम॑ ए॒व । \newline
51. ए॒वास्मा॑ अस्मा ए॒वैवास्मै᳚ । \newline
52. अ॒स्मै॒ रेतो॒ रेतो᳚ ऽस्मा अस्मै॒ रेतः॑ । \newline
53. रेतो॒ दधा॑ति॒ दधा॑ति॒ रेतो॒ रेतो॒ दधा॑ति । \newline
54. दधा᳚ त्य॒ग्नि र॒ग्निर् दधा॑ति॒ दधा᳚ त्य॒ग्निः । \newline
55. अ॒ग्निः प्र॒जाम् प्र॒जा म॒ग्नि र॒ग्निः प्र॒जाम् । \newline
56. प्र॒जाम् प्र प्र प्र॒जाम् प्र॒जाम् प्र । \newline
57. प्र॒जामिति॑ प्र - जाम् । \newline
58. प्र ज॑नयति जनयति॒ प्र प्र ज॑नयति । \newline
59. ज॒न॒य॒ति॒ वि॒न्दते॑ वि॒न्दते॑ जनयति जनयति वि॒न्दते᳚ । \newline
60. वि॒न्दते᳚ प्र॒जाम् प्र॒जां ॅवि॒न्दते॑ वि॒न्दते᳚ प्र॒जाम् । \newline

\textbf{Ghana Paata } \newline

1. भे॒ष॒ज मे॒वैव भे॑ष॒जम् भे॑ष॒ज मे॒वास्मा॑ अस्मा ए॒व भे॑ष॒जम् भे॑ष॒ज मे॒वास्मै᳚ । \newline
2. ए॒वास्मा॑ अस्मा ए॒वैवास्मै॑ करोति करो त्यस्मा ए॒वैवास्मै॑ करोति । \newline
3. अ॒स्मै॒ क॒रो॒ति॒ क॒रो॒ त्य॒स्मा॒ अ॒स्मै॒ क॒रो॒ति॒ यदि॒ यदि॑ करो त्यस्मा अस्मै करोति॒ यदि॑ । \newline
4. क॒रो॒ति॒ यदि॒ यदि॑ करोति करोति॒ यदि॑ बिभी॒याद् बि॑भी॒याद् यदि॑ करोति करोति॒ यदि॑ बिभी॒यात् । \newline
5. यदि॑ बिभी॒याद् बि॑भी॒याद् यदि॒ यदि॑ बिभी॒याद् दु॒श्चर्मा॑ दु॒श्चर्मा॑ बिभी॒याद् यदि॒ यदि॑ बिभी॒याद् दु॒श्चर्मा᳚ । \newline
6. बि॒भी॒याद् दु॒श्चर्मा॑ दु॒श्चर्मा॑ बिभी॒याद् बि॑भी॒याद् दु॒श्चर्मा॑ भविष्यामि भविष्यामि दु॒श्चर्मा॑ बिभी॒याद् बि॑भी॒याद् दु॒श्चर्मा॑ भविष्यामि । \newline
7. दु॒श्चर्मा॑ भविष्यामि भविष्यामि दु॒श्चर्मा॑ दु॒श्चर्मा॑ भविष्या॒मीतीति॑ भविष्यामि दु॒श्चर्मा॑ दु॒श्चर्मा॑ भविष्या॒मीति॑ । \newline
8. दु॒श्चर्मेति॑ दुः - चर्मा᳚ । \newline
9. भ॒वि॒ष्या॒मीतीति॑ भविष्यामि भविष्या॒मीति॑ सोमापौ॒ष्णꣳ सो॑मापौ॒ष्ण मिति॑ भविष्यामि भविष्या॒मीति॑ सोमापौ॒ष्णम् । \newline
10. इति॑ सोमापौ॒ष्णꣳ सो॑मापौ॒ष्ण मितीति॑ सोमापौ॒ष्णम् च॒रुम् च॒रुꣳ सो॑मापौ॒ष्ण मितीति॑ सोमापौ॒ष्णम् च॒रुम् । \newline
11. सो॒मा॒पौ॒ष्णम् च॒रुम् च॒रुꣳ सो॑मापौ॒ष्णꣳ सो॑मापौ॒ष्णम् च॒रुम् निर् णिश्च॒रुꣳ सो॑मापौ॒ष्णꣳ सो॑मापौ॒ष्णम् च॒रुम् निः । \newline
12. सो॒मा॒पौ॒ष्णमिति॑ सोमा - पौ॒ष्णम् । \newline
13. च॒रुम् निर् णिश्च॒रुम् च॒रुम् निर् व॑पेद् वपे॒न् निश्च॒रुम् च॒रुम् निर् व॑पेत् । \newline
14. निर् व॑पेद् वपे॒न् निर् णिर् व॑पेथ् सौ॒म्यः सौ॒म्यो व॑पे॒न् निर् णिर् व॑पेथ् सौ॒म्यः । \newline
15. व॒पे॒थ् सौ॒म्यः सौ॒म्यो व॑पेद् वपेथ् सौ॒म्यो वै वै सौ॒म्यो व॑पेद् वपेथ् सौ॒म्यो वै । \newline
16. सौ॒म्यो वै वै सौ॒म्यः सौ॒म्यो वै दे॒वत॑या दे॒वत॑या॒ वै सौ॒म्यः सौ॒म्यो वै दे॒वत॑या । \newline
17. वै दे॒वत॑या दे॒वत॑या॒ वै वै दे॒वत॑या॒ पुरु॑षः॒ पुरु॑षो दे॒वत॑या॒ वै वै दे॒वत॑या॒ पुरु॑षः । \newline
18. दे॒वत॑या॒ पुरु॑षः॒ पुरु॑षो दे॒वत॑या दे॒वत॑या॒ पुरु॑षः पौ॒ष्णाः पौ॒ष्णाः पुरु॑षो दे॒वत॑या दे॒वत॑या॒ पुरु॑षः पौ॒ष्णाः । \newline
19. पुरु॑षः पौ॒ष्णाः पौ॒ष्णाः पुरु॑षः॒ पुरु॑षः पौ॒ष्णाः प॒शवः॑ प॒शवः॑ पौ॒ष्णाः पुरु॑षः॒ पुरु॑षः पौ॒ष्णाः प॒शवः॑ । \newline
20. पौ॒ष्णाः प॒शवः॑ प॒शवः॑ पौ॒ष्णाः पौ॒ष्णाः प॒शवः॒ स्वया॒ स्वया॑ प॒शवः॑ पौ॒ष्णाः पौ॒ष्णाः प॒शवः॒ स्वया᳚ । \newline
21. प॒शवः॒ स्वया॒ स्वया॑ प॒शवः॑ प॒शवः॒ स्वयै॒वैव स्वया॑ प॒शवः॑ प॒शवः॒ स्वयै॒व । \newline
22. स्वयै॒वैव स्वया॒ स्वयै॒वास्मा॑ अस्मा ए॒व स्वया॒ स्वयै॒वास्मै᳚ । \newline
23. ए॒वास्मा॑ अस्मा ए॒वैवास्मै॑ दे॒वत॑या दे॒वत॑या ऽस्मा ए॒वैवास्मै॑ दे॒वत॑या । \newline
24. अ॒स्मै॒ दे॒वत॑या दे॒वत॑या ऽस्मा अस्मै दे॒वत॑या प॒शुभिः॑ प॒शुभि॑र् दे॒वत॑या ऽस्मा अस्मै दे॒वत॑या प॒शुभिः॑ । \newline
25. दे॒वत॑या प॒शुभिः॑ प॒शुभि॑र् दे॒वत॑या दे॒वत॑या प॒शुभि॒ स्त्वच॒म् त्वच॑म् प॒शुभि॑र् दे॒वत॑या दे॒वत॑या प॒शुभि॒ स्त्वच᳚म् । \newline
26. प॒शुभि॒ स्त्वच॒म् त्वच॑म् प॒शुभिः॑ प॒शुभि॒ स्त्वच॑म् करोति करोति॒ त्वच॑म् प॒शुभिः॑ प॒शुभि॒ स्त्वच॑म् करोति । \newline
27. प॒शुभि॒रिति॑ प॒शु - भिः॒ । \newline
28. त्वच॑म् करोति करोति॒ त्वच॒म् त्वच॑म् करोति॒ न न क॑रोति॒ त्वच॒म् त्वच॑म् करोति॒ न । \newline
29. क॒रो॒ति॒ न न क॑रोति करोति॒ न दु॒श्चर्मा॑ दु॒श्चर्मा॒ न क॑रोति करोति॒ न दु॒श्चर्मा᳚ । \newline
30. न दु॒श्चर्मा॑ दु॒श्चर्मा॒ न न दु॒श्चर्मा॑ भवति भवति दु॒श्चर्मा॒ न न दु॒श्चर्मा॑ भवति । \newline
31. दु॒श्चर्मा॑ भवति भवति दु॒श्चर्मा॑ दु॒श्चर्मा॑ भवति सोमारौ॒द्रꣳ सो॑मारौ॒द्रम् भ॑वति दु॒श्चर्मा॑ दु॒श्चर्मा॑ भवति सोमारौ॒द्रम् । \newline
32. दु॒श्चर्मेति॑ दुः - चर्मा᳚ । \newline
33. भ॒व॒ति॒ सो॒मा॒रौ॒द्रꣳ सो॑मारौ॒द्रम् भ॑वति भवति सोमारौ॒द्रम् च॒रुम् च॒रुꣳ सो॑मारौ॒द्रम् भ॑वति भवति सोमारौ॒द्रम् च॒रुम् । \newline
34. सो॒मा॒रौ॒द्रम् च॒रुम् च॒रुꣳ सो॑मारौ॒द्रꣳ सो॑मारौ॒द्रम् च॒रुम् निर् णिश्च॒रुꣳ सो॑मारौ॒द्रꣳ सो॑मारौ॒द्रम् च॒रुम् निः । \newline
35. सो॒मा॒रौ॒द्रमिति॑ सोमा - रौ॒द्रम् । \newline
36. च॒रुम् निर् णिश्च॒रुम् च॒रुम् निर् व॑पेद् वपे॒न् निश्च॒रुम् च॒रुम् निर् व॑पेत् । \newline
37. निर् व॑पेद् वपे॒न् निर् णिर् व॑पेत् प्र॒जाका॑मः प्र॒जाका॑मो वपे॒न् निर् णिर् व॑पेत् प्र॒जाका॑मः । \newline
38. व॒पे॒त् प्र॒जाका॑मः प्र॒जाका॑मो वपेद् वपेत् प्र॒जाका॑मः॒ सोमः॒ सोमः॑ प्र॒जाका॑मो वपेद् वपेत् प्र॒जाका॑मः॒ सोमः॑ । \newline
39. प्र॒जाका॑मः॒ सोमः॒ सोमः॑ प्र॒जाका॑मः प्र॒जाका॑मः॒ सोमो॒ वै वै सोमः॑ प्र॒जाका॑मः प्र॒जाका॑मः॒ सोमो॒ वै । \newline
40. प्र॒जाका॑म॒ इति॑ प्र॒जा - का॒मः॒ । \newline
41. सोमो॒ वै वै सोमः॒ सोमो॒ वै रे॑तो॒धा रे॑तो॒धा वै सोमः॒ सोमो॒ वै रे॑तो॒धाः । \newline
42. वै रे॑तो॒धा रे॑तो॒धा वै वै रे॑तो॒धा अ॒ग्नि र॒ग्नी रे॑तो॒धा वै वै रे॑तो॒धा अ॒ग्निः । \newline
43. रे॒तो॒धा अ॒ग्नि र॒ग्नी रे॑तो॒धा रे॑तो॒धा अ॒ग्निः प्र॒जाना᳚म् प्र॒जाना॑ म॒ग्नी रे॑तो॒धा रे॑तो॒धा अ॒ग्निः प्र॒जाना᳚म् । \newline
44. रे॒तो॒धा इति॑ रेतः - धाः । \newline
45. अ॒ग्निः प्र॒जाना᳚म् प्र॒जाना॑ म॒ग्नि र॒ग्निः प्र॒जाना᳚म् प्रजनयि॒ता प्र॑जनयि॒ता प्र॒जाना॑ म॒ग्नि र॒ग्निः प्र॒जाना᳚म् प्रजनयि॒ता । \newline
46. प्र॒जाना᳚म् प्रजनयि॒ता प्र॑जनयि॒ता प्र॒जाना᳚म् प्र॒जाना᳚म् प्रजनयि॒ता सोमः॒ सोमः॑ प्रजनयि॒ता प्र॒जाना᳚म् प्र॒जाना᳚म् प्रजनयि॒ता सोमः॑ । \newline
47. प्र॒जाना॒मिति॑ प्र - जाना᳚म् । \newline
48. प्र॒ज॒न॒यि॒ता सोमः॒ सोमः॑ प्रजनयि॒ता प्र॑जनयि॒ता सोम॑ ए॒वैव सोमः॑ प्रजनयि॒ता प्र॑जनयि॒ता सोम॑ ए॒व । \newline
49. प्र॒ज॒न॒यि॒तेति॑ प्र - ज॒न॒यि॒ता । \newline
50. सोम॑ ए॒वैव सोमः॒ सोम॑ ए॒वास्मा॑ अस्मा ए॒व सोमः॒ सोम॑ ए॒वास्मै᳚ । \newline
51. ए॒वास्मा॑ अस्मा ए॒वैवास्मै॒ रेतो॒ रेतो᳚ ऽस्मा ए॒वैवास्मै॒ रेतः॑ । \newline
52. अ॒स्मै॒ रेतो॒ रेतो᳚ ऽस्मा अस्मै॒ रेतो॒ दधा॑ति॒ दधा॑ति॒ रेतो᳚ ऽस्मा अस्मै॒ रेतो॒ दधा॑ति । \newline
53. रेतो॒ दधा॑ति॒ दधा॑ति॒ रेतो॒ रेतो॒ दधा᳚ त्य॒ग्नि र॒ग्निर् दधा॑ति॒ रेतो॒ रेतो॒ दधा᳚ त्य॒ग्निः । \newline
54. दधा᳚ त्य॒ग्नि र॒ग्निर् दधा॑ति॒ दधा᳚ त्य॒ग्निः प्र॒जाम् प्र॒जा म॒ग्निर् दधा॑ति॒ दधा᳚ त्य॒ग्निः प्र॒जाम् । \newline
55. अ॒ग्निः प्र॒जाम् प्र॒जा म॒ग्नि र॒ग्निः प्र॒जाम् प्र प्र प्र॒जा म॒ग्नि र॒ग्निः प्र॒जाम् प्र । \newline
56. प्र॒जाम् प्र प्र प्र॒जाम् प्र॒जाम् प्र ज॑नयति जनयति॒ प्र प्र॒जाम् प्र॒जाम् प्र ज॑नयति । \newline
57. प्र॒जामिति॑ प्र - जाम् । \newline
58. प्र ज॑नयति जनयति॒ प्र प्र ज॑नयति वि॒न्दते॑ वि॒न्दते॑ जनयति॒ प्र प्र ज॑नयति वि॒न्दते᳚ । \newline
59. ज॒न॒य॒ति॒ वि॒न्दते॑ वि॒न्दते॑ जनयति जनयति वि॒न्दते᳚ प्र॒जाम् प्र॒जां ॅवि॒न्दते॑ जनयति जनयति वि॒न्दते᳚ प्र॒जाम् । \newline
60. वि॒न्दते᳚ प्र॒जाम् प्र॒जां ॅवि॒न्दते॑ वि॒न्दते᳚ प्र॒जाꣳ सो॑मारौ॒द्रꣳ सो॑मारौ॒द्रम् प्र॒जां ॅवि॒न्दते॑ वि॒न्दते᳚ प्र॒जाꣳ सो॑मारौ॒द्रम् । \newline
\pagebreak
\markright{ TS 2.2.10.4  \hfill https://www.vedavms.in \hfill}
\addcontentsline{toc}{section}{ TS 2.2.10.4 }
\section*{ TS 2.2.10.4 }

\textbf{TS 2.2.10.4 } \newline
\textbf{Samhita Paata} \newline

प्र॒जाꣳ सो॑मारौ॒द्रं च॒रुं निर्व॑पेदभि॒चरन्᳚-थ्सौ॒म्यो वै दे॒वत॑या॒ पुरु॑ष ए॒ष रु॒द्रो यद॒ग्निः स्वाया॑ ए॒वैनं॑ दे॒वता॑यै नि॒ष्क्रीय॑ रु॒द्रायापि॑ दधाति ता॒जगार्ति॒मार्च्छ॑ति सोमारौ॒द्रं च॒रुं निर्व॑पे॒-ज्ज्योगा॑मयावी॒ सोमं॒ ॅवा ए॒तस्य॒ रसो॑ गच्छत्य॒ग्निꣳ शरी॑रं॒ ॅयस्य॒ ज्योगा॒मय॑ति॒ सोमा॑दे॒वास्य॒ रसं॑ निष्क्री॒णात्य॒ग्नेः शरी॑रमु॒त यदी॒ - [  ] \newline

\textbf{Pada Paata} \newline

प्र॒जामिति॑ प्र -जाम् । सो॒मा॒रौ॒द्रमिति॑ सोमा- रौ॒द्रम् । च॒रुम् । निरिति॑ । व॒पे॒त् । अ॒भि॒चर॒न्नित्य॑भि-चरन्न्॑ । सौ॒म्यः । वै । दे॒वत॑या । पुरु॑षः । ए॒षः । रु॒द्रः । यत् । अ॒ग्निः । स्वायाः᳚ । ए॒व । ए॒न॒म् । दे॒वता॑यै । नि॒ष्क्रीयेति॑ निः - क्रीय॑ । रु॒द्राय॑ । अपीति॑ । द॒धा॒ति॒ । ता॒जक् । आर्ति᳚म् । एति॑ । ऋ॒च्छ॒ति॒ । सो॒मा॒रौ॒द्रमिति॑ सोमा-रौ॒द्रम् । च॒रुम् । निरिति॑ । व॒पे॒त् । ज्योगा॑मया॒वीति॒ ज्योक् - आ॒म॒या॒वी॒ । सोम᳚म् । वै । ए॒तस्य॑ । रसः॑ । ग॒च्छ॒ति॒ । अ॒ग्निम् । शरी॑रम् । यस्य॑ । ज्योक् । आ॒मय॑ति । सोमा᳚त् । ए॒व । अ॒स्य॒ । रस᳚म् । नि॒ष्क्री॒णातीति॑ निः - क्री॒णाति॑ । अ॒ग्नेः । शरी॑रम् । उ॒त । यदि॑ ।  \newline


\textbf{Krama Paata} \newline

प्र॒जाꣳ सो॑मारौ॒द्रम् । प्र॒जामिति॑ प्र - जाम् । सो॒मा॒रौ॒द्रम् च॒रुम् । सो॒मा॒रौ॒द्रमिति॑ सोमा - रौ॒द्रम् । च॒रुम् निः । निर् व॑पेत् । व॒पे॒द॒भि॒चरन्न्॑ । अ॒भि॒चर᳚न्थ् सौ॒म्यः । अ॒भि॒चर॒न्नित्य॑भि - चरन्न्॑ । सौ॒म्यो वै । वै दे॒वत॑या । दे॒वत॑या॒ पुरु॑षः । पुरु॑ष ए॒षः । ए॒ष रु॒द्रः । रु॒द्रो यत् । यद॒ग्निः । अ॒ग्निः स्वायाः᳚ । स्वाया॑ ए॒व । ए॒वैन᳚म् । ए॒न॒म् दे॒वता॑यै । दे॒वता॑यै नि॒ष्क्रीय॑ । नि॒ष्क्रीय॑ रु॒द्राय॑ । नि॒ष्क्रीयेति॑ निः - क्रीय॑ । रु॒द्रायापि॑ । अपि॑ दधाति । द॒धा॒ति॒ ता॒जक् । ता॒जगार्ति᳚म् । आर्ति॒मा । आर्च्छ॑ति । ऋ॒च्छ॒ति॒ सो॒मा॒रौ॒द्रम् । सो॒मा॒रौ॒द्रम् च॒रुम् । सो॒मा॒रौ॒द्र मिति॑ सोमा - रौ॒द्रम् । च॒रुम् निः । निर् व॑पेत् । व॒पे॒ज्ज्योगा॑मयावी । ज्योगा॑मयावी॒ सोम᳚म् । ज्योगा॑मया॒वीति॒ ज्योक् - आ॒म॒या॒वी॒ । सोमं॒ ॅवै । वा ए॒तस्य॑ । ए॒तस्य॒ रसः॑ । रसो॑ गच्छति । ग॒च्छ॒त्य॒ग्निम् । अ॒ग्निꣳ शरी॑रम् । शरी॑रं॒ ॅयस्य॑ । यस्य॒ ज्योक् । ज्योगा॒मय॑ति । आ॒मय॑ति॒ सोमा᳚त् । सोमा॑दे॒व । ए॒वास्य॑ । अ॒स्य॒ रस᳚म् । रस॑म् निष्क्री॒णाति॑ । नि॒ष्क्री॒णात्य॒ग्नेः । नि॒ष्क्री॒णातीति॑ निः - क्री॒णाति॑ । अ॒ग्नेः शरी॑रम् । शरी॑रमु॒त । उ॒त यदि॑ । यदी॒तासुः॑ \newline

\textbf{Jatai Paata} \newline

1. प्र॒जाꣳ सो॑मारौ॒द्रꣳ सो॑मारौ॒द्रम् प्र॒जाम् प्र॒जाꣳ सो॑मारौ॒द्रम् । \newline
2. प्र॒जामिति॑ प्र - जाम् । \newline
3. सो॒मा॒रौ॒द्रम् च॒रुम् च॒रुꣳ सो॑मारौ॒द्रꣳ सो॑मारौ॒द्रम् च॒रुम् । \newline
4. सो॒मा॒रौ॒द्रमिति॑ सोमा - रौ॒द्रम् । \newline
5. च॒रुम् निर् णिश्च॒रुम् च॒रुम् निः । \newline
6. निर् व॑पेद् वपे॒न् निर् णिर् व॑पेत् । \newline
7. व॒पे॒ द॒भि॒चर॑न् नभि॒चरन्॑. वपेद् वपे दभि॒चरन्न्॑ । \newline
8. अ॒भि॒चर᳚न् थ्सौ॒म्यः सौ॒म्यो॑ ऽभिचर॑न् नभि॒चर᳚न् थ्सौ॒म्यः । \newline
9. अ॒भि॒चर॒न्नित्य॑भि - चरन्न्॑ । \newline
10. सौ॒म्यो वै वै सौ॒म्यः सौ॒म्यो वै । \newline
11. वै दे॒वत॑या दे॒वत॑या॒ वै वै दे॒वत॑या । \newline
12. दे॒वत॑या॒ पुरु॑षः॒ पुरु॑षो दे॒वत॑या दे॒वत॑या॒ पुरु॑षः । \newline
13. पुरु॑ष ए॒ष ए॒ष पुरु॑षः॒ पुरु॑ष ए॒षः । \newline
14. ए॒ष रु॒द्रो रु॒द्र ए॒ष ए॒ष रु॒द्रः । \newline
15. रु॒द्रो यद् यद् रु॒द्रो रु॒द्रो यत् । \newline
16. यद॒ग्नि र॒ग्निर् यद् यद॒ग्निः । \newline
17. अ॒ग्निः स्वायाः॒ स्वाया॑ अ॒ग्नि र॒ग्निः स्वायाः᳚ । \newline
18. स्वाया॑ ए॒वैव स्वायाः॒ स्वाया॑ ए॒व । \newline
19. ए॒वैन॑ मेन मे॒वैवैन᳚म् । \newline
20. ए॒न॒म् दे॒वता॑यै दे॒वता॑या एन मेनम् दे॒वता॑यै । \newline
21. दे॒वता॑यै नि॒ष्क्रीय॑ नि॒ष्क्रीय॑ दे॒वता॑यै दे॒वता॑यै नि॒ष्क्रीय॑ । \newline
22. नि॒ष्क्रीय॑ रु॒द्राय॑ रु॒द्राय॑ नि॒ष्क्रीय॑ नि॒ष्क्रीय॑ रु॒द्राय॑ । \newline
23. नि॒ष्क्रीयेति॑ निः - क्रीय॑ । \newline
24. रु॒द्रायाप्यपि॑ रु॒द्राय॑ रु॒द्रायापि॑ । \newline
25. अपि॑ दधाति दधा॒ त्यप्यपि॑ दधाति । \newline
26. द॒धा॒ति॒ ता॒जक् ता॒जग् द॑धाति दधाति ता॒जक् । \newline
27. ता॒जगार्ति॒ मार्ति॑म् ता॒जक् ता॒जगार्ति᳚म् । \newline
28. आर्ति॒ मा ऽऽर्ति॒ मार्ति॒ मा । \newline
29. आर्च्छ॑ त्यृच्छ॒ त्यार्च्छ॑ति । \newline
30. ऋ॒च्छ॒ति॒ सो॒मा॒रौ॒द्रꣳ सो॑मारौ॒द्र मृ॑च्छ त्यृच्छति सोमारौ॒द्रम् । \newline
31. सो॒मा॒रौ॒द्रम् च॒रुम् च॒रुꣳ सो॑मारौ॒द्रꣳ सो॑मारौ॒द्रम् च॒रुम् । \newline
32. सो॒मा॒रौ॒द्रमिति॑ सोमा - रौ॒द्रम् । \newline
33. च॒रुम् निर् णिश्च॒रुम् च॒रुम् निः । \newline
34. निर् व॑पेद् वपे॒न् निर् णिर् व॑पेत् । \newline
35. व॒पे॒ज् ज्योगा॑मयावी॒ ज्योगा॑मयावी वपेद् वपे॒ज् ज्योगा॑मयावी । \newline
36. ज्योगा॑मयावी॒ सोमꣳ॒॒ सोम॒म् ज्योगा॑मयावी॒ ज्योगा॑मयावी॒ सोम᳚म् । \newline
37. ज्योगा॑मया॒वीति॒ ज्योक् - आ॒म॒या॒वी॒ । \newline
38. सोमं॒ ॅवै वै सोमꣳ॒॒ सोमं॒ ॅवै । \newline
39. वा ए॒त स्यै॒तस्य॒ वै वा ए॒तस्य॑ । \newline
40. ए॒तस्य॒ रसो॒ रस॑ ए॒तस्यै॒तस्य॒ रसः॑ । \newline
41. रसो॑ गच्छति गच्छति॒ रसो॒ रसो॑ गच्छति । \newline
42. ग॒च्छ॒ त्य॒ग्नि म॒ग्निम् ग॑च्छति गच्छ त्य॒ग्निम् । \newline
43. अ॒ग्निꣳ शरी॑रꣳ॒॒ शरी॑र म॒ग्नि म॒ग्निꣳ शरी॑रम् । \newline
44. शरी॑रं॒ ॅयस्य॒ यस्य॒ शरी॑रꣳ॒॒ शरी॑रं॒ ॅयस्य॑ । \newline
45. यस्य॒ ज्योग् ज्योग् यस्य॒ यस्य॒ ज्योक् । \newline
46. ज्योगा॒मय॑ त्या॒मय॑ति॒ ज्योग् ज्योगा॒मय॑ति । \newline
47. आ॒मय॑ति॒ सोमा॒थ् सोमा॑ दा॒मय॑ त्या॒मय॑ति॒ सोमा᳚त् । \newline
48. सोमा॑ दे॒वैव सोमा॒थ् सोमा॑ दे॒व । \newline
49. ए॒वास्या᳚ स्यै॒वै वास्य॑ । \newline
50. अ॒स्य॒ रसꣳ॒॒ रस॑ मस्यास्य॒ रस᳚म् । \newline
51. रस॑म् निष्क्री॒णाति॑ निष्क्री॒णाति॒ रसꣳ॒॒ रस॑म् निष्क्री॒णाति॑ । \newline
52. नि॒ष्क्री॒णा त्य॒ग्ने र॒ग्नेर् नि॑ष्क्री॒णाति॑ निष्क्री॒णा त्य॒ग्नेः । \newline
53. नि॒ष्क्री॒णातीति॑ निः - क्री॒णाति॑ । \newline
54. अ॒ग्नेः शरी॑रꣳ॒॒ शरी॑र म॒ग्ने र॒ग्नेः शरी॑रम् । \newline
55. शरी॑र मु॒तोत शरी॑रꣳ॒॒ शरी॑र मु॒त । \newline
56. उ॒त यदि॒ यद्यु॒तोत यदि॑ । \newline
57. यदी॒तासु॑ रि॒तासु॒र् यदि॒ यदी॒तासुः॑ । \newline

\textbf{Ghana Paata } \newline

1. प्र॒जाꣳ सो॑मारौ॒द्रꣳ सो॑मारौ॒द्रम् प्र॒जाम् प्र॒जाꣳ सो॑मारौ॒द्रम् च॒रुम् च॒रुꣳ सो॑मारौ॒द्रम् प्र॒जाम् प्र॒जाꣳ सो॑मारौ॒द्रम् च॒रुम् । \newline
2. प्र॒जामिति॑ प्र - जाम् । \newline
3. सो॒मा॒रौ॒द्रम् च॒रुम् च॒रुꣳ सो॑मारौ॒द्रꣳ सो॑मारौ॒द्रम् च॒रुम् निर् णिश्च॒रुꣳ सो॑मारौ॒द्रꣳ सो॑मारौ॒द्रम् च॒रुम् निः । \newline
4. सो॒मा॒रौ॒द्रमिति॑ सोमा - रौ॒द्रम् । \newline
5. च॒रुम् निर् णिश्च॒रुम् च॒रुम् निर् व॑पेद् वपे॒न् निश्च॒रुम् च॒रुम् निर् व॑पेत् । \newline
6. निर् व॑पेद् वपे॒न् निर् णिर् व॑पे दभि॒चर॑न् नभि॒चरन्॑. वपे॒न् निर् णिर् व॑पे दभि॒चरन्न्॑ । \newline
7. व॒पे॒ द॒भि॒चर॑न् नभि॒चरन्॑. वपेद् वपे दभि॒चर᳚न् थ्सौ॒म्यः सौ॒म्यो॑ ऽभिचरन्॑. वपेद् वपे दभि॒चर᳚न् थ्सौ॒म्यः । \newline
8. अ॒भि॒चर᳚न् थ्सौ॒म्यः सौ॒म्यो॑ ऽभिचर॑न् नभि॒चर᳚न् थ्सौ॒म्यो वै वै सौ॒म्यो॑ ऽभिचर॑न् नभि॒चर᳚न् थ्सौ॒म्यो वै । \newline
9. अ॒भि॒चर॒न्नित्य॑भि - चरन्न्॑ । \newline
10. सौ॒म्यो वै वै सौ॒म्यः सौ॒म्यो वै दे॒वत॑या दे॒वत॑या॒ वै सौ॒म्यः सौ॒म्यो वै दे॒वत॑या । \newline
11. वै दे॒वत॑या दे॒वत॑या॒ वै वै दे॒वत॑या॒ पुरु॑षः॒ पुरु॑षो दे॒वत॑या॒ वै वै दे॒वत॑या॒ पुरु॑षः । \newline
12. दे॒वत॑या॒ पुरु॑षः॒ पुरु॑षो दे॒वत॑या दे॒वत॑या॒ पुरु॑ष ए॒ष ए॒ष पुरु॑षो दे॒वत॑या दे॒वत॑या॒ पुरु॑ष ए॒षः । \newline
13. पुरु॑ष ए॒ष ए॒ष पुरु॑षः॒ पुरु॑ष ए॒ष रु॒द्रो रु॒द्र ए॒ष पुरु॑षः॒ पुरु॑ष ए॒ष रु॒द्रः । \newline
14. ए॒ष रु॒द्रो रु॒द्र ए॒ष ए॒ष रु॒द्रो यद् यद् रु॒द्र ए॒ष ए॒ष रु॒द्रो यत् । \newline
15. रु॒द्रो यद् यद् रु॒द्रो रु॒द्रो यद॒ग्नि र॒ग्निर् यद् रु॒द्रो रु॒द्रो यद॒ग्निः । \newline
16. यद॒ग्नि र॒ग्निर् यद् यद॒ग्निः स्वायाः॒ स्वाया॑ अ॒ग्निर् यद् यद॒ग्निः स्वायाः᳚ । \newline
17. अ॒ग्निः स्वायाः॒ स्वाया॑ अ॒ग्नि र॒ग्निः स्वाया॑ ए॒वैव स्वाया॑ अ॒ग्नि र॒ग्निः स्वाया॑ ए॒व । \newline
18. स्वाया॑ ए॒वैव स्वायाः॒ स्वाया॑ ए॒वैन॑ मेन मे॒व स्वायाः॒ स्वाया॑ ए॒वैन᳚म् । \newline
19. ए॒वैन॑ मेन मे॒वैवैन॑म् दे॒वता॑यै दे॒वता॑या एन मे॒वैवैन॑म् दे॒वता॑यै । \newline
20. ए॒न॒म् दे॒वता॑यै दे॒वता॑या एन मेनम् दे॒वता॑यै नि॒ष्क्रीय॑ नि॒ष्क्रीय॑ दे॒वता॑या एन मेनम् दे॒वता॑यै नि॒ष्क्रीय॑ । \newline
21. दे॒वता॑यै नि॒ष्क्रीय॑ नि॒ष्क्रीय॑ दे॒वता॑यै दे॒वता॑यै नि॒ष्क्रीय॑ रु॒द्राय॑ रु॒द्राय॑ नि॒ष्क्रीय॑ दे॒वता॑यै दे॒वता॑यै नि॒ष्क्रीय॑ रु॒द्राय॑ । \newline
22. नि॒ष्क्रीय॑ रु॒द्राय॑ रु॒द्राय॑ नि॒ष्क्रीय॑ नि॒ष्क्रीय॑ रु॒द्राया प्यपि॑ रु॒द्राय॑ नि॒ष्क्रीय॑ नि॒ष्क्रीय॑ रु॒द्रायापि॑ । \newline
23. नि॒ष्क्रीयेति॑ निः - क्रीय॑ । \newline
24. रु॒द्राया प्यपि॑ रु॒द्राय॑ रु॒द्रायापि॑ दधाति दधा॒ त्यपि॑ रु॒द्राय॑ रु॒द्रायापि॑ दधाति । \newline
25. अपि॑ दधाति दधा॒ त्यप्यपि॑ दधाति ता॒जक् ता॒जग् द॑धा॒ त्यप्यपि॑ दधाति ता॒जक् । \newline
26. द॒धा॒ति॒ ता॒जक् ता॒जग् द॑धाति दधाति ता॒जगार्ति॒ मार्ति॑म् ता॒जग् द॑धाति दधाति ता॒जगार्ति᳚म् । \newline
27. ता॒जगार्ति॒ मार्ति॑म् ता॒जक् ता॒जगार्ति॒ मा ऽऽर्ति॑म् ता॒जक् ता॒जगार्ति॒ मा । \newline
28. आर्ति॒ मा ऽऽर्ति॒ मार्ति॒ मार्च्छ॑ त्यृच्छ॒त्या ऽऽर्ति॒ मार्ति॒ मार्च्छ॑ति । \newline
29. आर्च्छ॑ त्यृच्छ॒ त्यार्च्छ॑ति सोमारौ॒द्रꣳ सो॑मारौ॒द्र मृ॑च्छ॒ त्यार्च्छ॑ति सोमारौ॒द्रम् । \newline
30. ऋ॒च्छ॒ति॒ सो॒मा॒रौ॒द्रꣳ सो॑मारौ॒द्र मृ॑च्छ त्यृच्छति सोमारौ॒द्रम् च॒रुम् च॒रुꣳ सो॑मारौ॒द्र मृ॑च्छ त्यृच्छति सोमारौ॒द्रम् च॒रुम् । \newline
31. सो॒मा॒रौ॒द्रम् च॒रुम् च॒रुꣳ सो॑मारौ॒द्रꣳ सो॑मारौ॒द्रम् च॒रुम् निर् णिश्च॒रुꣳ सो॑मारौ॒द्रꣳ सो॑मारौ॒द्रम् च॒रुम् निः । \newline
32. सो॒मा॒रौ॒द्रमिति॑ सोमा - रौ॒द्रम् । \newline
33. च॒रुम् निर् णिश्च॒रुम् च॒रुम् निर् व॑पेद् वपे॒न् निश्च॒रुम् च॒रुम् निर् व॑पेत् । \newline
34. निर् व॑पेद् वपे॒न् निर् णिर् व॑पे॒ज् ज्योगा॑मयावी॒ ज्योगा॑मयावी वपे॒न् निर् णिर् व॑पे॒ज् ज्योगा॑मयावी । \newline
35. व॒पे॒ज् ज्योगा॑मयावी॒ ज्योगा॑मयावी वपेद् वपे॒ज् ज्योगा॑मयावी॒ सोमꣳ॒॒ सोम॒म् ज्योगा॑मयावी वपेद् वपे॒ज् ज्योगा॑मयावी॒ सोम᳚म् । \newline
36. ज्योगा॑मयावी॒ सोमꣳ॒॒ सोम॒म् ज्योगा॑मयावी॒ ज्योगा॑मयावी॒ सोमं॒ ॅवै वै सोम॒म् ज्योगा॑मयावी॒ ज्योगा॑मयावी॒ सोमं॒ ॅवै । \newline
37. ज्योगा॑मया॒वीति॒ ज्योक् - आ॒म॒या॒वी॒ । \newline
38. सोमं॒ ॅवै वै सोमꣳ॒॒ सोमं॒ ॅवा ए॒त स्यै॒तस्य॒ वै सोमꣳ॒॒ सोमं॒ ॅवा ए॒तस्य॑ । \newline
39. वा ए॒त स्यै॒तस्य॒ वै वा ए॒तस्य॒ रसो॒ रस॑ ए॒तस्य॒ वै वा ए॒तस्य॒ रसः॑ । \newline
40. ए॒तस्य॒ रसो॒ रस॑ ए॒त स्यै॒तस्य॒ रसो॑ गच्छति गच्छति॒ रस॑ ए॒त स्यै॒तस्य॒ रसो॑ गच्छति । \newline
41. रसो॑ गच्छति गच्छति॒ रसो॒ रसो॑ गच्छ त्य॒ग्नि म॒ग्निम् ग॑च्छति॒ रसो॒ रसो॑ गच्छ त्य॒ग्निम् । \newline
42. ग॒च्छ॒ त्य॒ग्नि म॒ग्निम् ग॑च्छति गच्छ त्य॒ग्निꣳ शरी॑रꣳ॒॒ शरी॑र म॒ग्निम् ग॑च्छति गच्छ त्य॒ग्निꣳ शरी॑रम् । \newline
43. अ॒ग्निꣳ शरी॑रꣳ॒॒ शरी॑र म॒ग्नि म॒ग्निꣳ शरी॑रं॒ ॅयस्य॒ यस्य॒ शरी॑र म॒ग्नि म॒ग्निꣳ शरी॑रं॒ ॅयस्य॑ । \newline
44. शरी॑रं॒ ॅयस्य॒ यस्य॒ शरी॑रꣳ॒॒ शरी॑रं॒ ॅयस्य॒ ज्योग् ज्योग् यस्य॒ शरी॑रꣳ॒॒ शरी॑रं॒ ॅयस्य॒ ज्योक् । \newline
45. यस्य॒ ज्योग् ज्योग् यस्य॒ यस्य॒ ज्योगा॒मय॑ त्या॒मय॑ति॒ ज्योग् यस्य॒ यस्य॒ ज्योगा॒मय॑ति । \newline
46. ज्योगा॒मय॑ त्या॒मय॑ति॒ ज्योग् ज्योगा॒मय॑ति॒ सोमा॒थ् सोमा॑ दा॒मय॑ति॒ ज्योग् ज्योगा॒मय॑ति॒ सोमा᳚त् । \newline
47. आ॒मय॑ति॒ सोमा॒थ् सोमा॑ दा॒मय॑ त्या॒मय॑ति॒ सोमा॑दे॒वैव सोमा॑ दा॒मय॑ त्या॒मय॑ति॒ सोमा॑दे॒व । \newline
48. सोमा॑ दे॒वैव सोमा॒थ् सोमा॑ दे॒वास्या᳚स्यै॒व सोमा॒थ् सोमा॑ दे॒वास्य॑ । \newline
49. ए॒वास्या᳚ स्यै॒वैवास्य॒ रसꣳ॒॒ रस॑ मस्यै॒ वैवास्य॒ रस᳚म् । \newline
50. अ॒स्य॒ रसꣳ॒॒ रस॑ मस्यास्य॒ रस॑म् निष्क्री॒णाति॑ निष्क्री॒णाति॒ रस॑ मस्यास्य॒ रस॑म् निष्क्री॒णाति॑ । \newline
51. रस॑म् निष्क्री॒णाति॑ निष्क्री॒णाति॒ रसꣳ॒॒ रस॑म् निष्क्री॒णा त्य॒ग्ने र॒ग्नेर् नि॑ष्क्री॒णाति॒ रसꣳ॒॒ रस॑म् निष्क्री॒णा त्य॒ग्नेः । \newline
52. नि॒ष्क्री॒णा त्य॒ग्ने र॒ग्नेर् नि॑ष्क्री॒णाति॑ निष्क्री॒णा त्य॒ग्नेः शरी॑रꣳ॒॒ शरी॑र म॒ग्नेर् नि॑ष्क्री॒णाति॑ निष्क्री॒णा त्य॒ग्नेः शरी॑रम् । \newline
53. नि॒ष्क्री॒णातीति॑ निः - क्री॒णाति॑ । \newline
54. अ॒ग्नेः शरी॑रꣳ॒॒ शरी॑र म॒ग्ने र॒ग्नेः शरी॑र मु॒तोत शरी॑र म॒ग्ने र॒ग्नेः शरी॑र मु॒त । \newline
55. शरी॑र मु॒तोत शरी॑रꣳ॒॒ शरी॑र मु॒त यदि॒ यद्यु॒त शरी॑रꣳ॒॒ शरी॑र मु॒त यदि॑ । \newline
56. उ॒त यदि॒ यद्यु॒तोत यदी॒ता सु॑ रि॒तासु॒र् यद्यु॒तोत यदी॒तासुः॑ । \newline
57. यदी॒तासु॑ रि॒तासु॒र् यदि॒ यदी॒तासु॒र् भव॑ति॒ भव॑ती॒तासु॒र् यदि॒ यदी॒तासु॒र् भव॑ति । \newline
\pagebreak
\markright{ TS 2.2.10.5  \hfill https://www.vedavms.in \hfill}
\addcontentsline{toc}{section}{ TS 2.2.10.5 }
\section*{ TS 2.2.10.5 }

\textbf{TS 2.2.10.5 } \newline

\textbf{Pada Paata} \newline

इ॒तासु॒रिती॒त - अ॒सुः॒ । भव॑ति । जीव॑ति । ए॒व । सो॒मा॒रु॒द्रयो॒रिति॑ सोमा - रु॒द्रयोः᳚ । वै । ए॒तम् । ग्र॒सि॒तम् । होता᳚ । निरिति॑ । खि॒द॒ति॒ । सः । ई॒श्व॒रः । आर्ति᳚म् । आता॒र्रिया - अ॒र्तोः॒ । अ॒न॒ड्वान् । होत्रा᳚ । देयः॑ । वह्निः॑ । वै । अ॒न॒ड्वान् । वह्निः॑ । होता᳚ । वह्नि॑ना । ए॒व । वह्नि᳚म् । आ॒त्मान᳚म् । स्पृ॒णो॒ति॒ । सो॒मा॒रौ॒द्रमिति॑ सोमा - रौ॒द्रम् । च॒रुम् । निरिति॑ । व॒पे॒त् । यः । का॒मये॑त । स्वे । अ॒स्मै॒ । आ॒यत॑न॒ इत्या᳚ - यत॑ने । भ्रातृ॑व्यम् । ज॒न॒ये॒य॒म् । इति॑ । वेदि᳚म् । प॒रि॒गृह्येति॑ परि - गृह्य॑ । अ॒र्द्धम् । उ॒द्ध॒न्यादित्यु॑त्-ह॒न्यात् । अ॒र्द्धम् । न । अ॒र्द्धम् । ब॒र्॒.हिषः॑ । स्तृ॒णी॒यात् । अ॒र्द्धम् ( ) । न । अ॒र्द्धम् । इ॒॒द्ध्मस्य॑ । अ॒भ्या॒द॒द्ध्यादित्य॑भि - आ॒द॒द्ध्यात् । अ॒र्द्धम् । न । स्वे । ए॒व । अ॒स्मै॒ । आ॒यत॑न॒ इत्या᳚ - यत॑ने । भ्रातृ॑व्यम् । ज॒न॒य॒ति॒ ॥  \newline


\textbf{Krama Paata} \newline

इ॒तासु॒र् भव॑ति । इ॒तासु॒रिती॒त - अ॒सुः॒ । भव॑ति॒ जीव॑ति । जीव॑त्ये॒व । ए॒व सो॑मारु॒द्रयोः᳚ । सो॒मा॒रु॒द्रयो॒र् वै । सो॒मा॒रु॒द्रयो॒रिति॑ सोमा - रु॒द्रयोः᳚ । वा ए॒तम् । ए॒तम् ग्र॑सि॒तम् । ग्र॒सि॒तꣳ होता᳚ । होता॒ निः । निष् खि॑दति । खि॒द॒ति॒ सः । स ई᳚श्व॒रः । ई॒श्व॒र आर्ति᳚म् । आर्ति॒मार्तोः᳚ । आर्तो॑रन॒ड्वान् । आर्तो॒रित्या - अ॒र्तोः॒ । अ॒न॒ड्वान्. होत्रा᳚ । होत्रा॒ देयः॑ । देयो॒ वह्निः॑ । वह्नि॒र् वै । वा अ॑न॒ड्वान् । अ॒न॒ड्वान्. वह्निः॑ । वह्नि॒र्॒. होता᳚ । होता॒ वह्नि॑ना । वह्नि॑नै॒व । ए॒व वह्नि᳚म् । वह्नि॑मा॒त्मान᳚म् । आ॒त्मानꣳ॑ स्पृणोति । स्पृ॒णो॒ति॒ सो॒मा॒रौ॒द्रम् । सो॒मा॒रौ॒द्रम् च॒रुम् । सो॒म॒रौ॒द्रमिति॑ सोमा - रौ॒द्रम् । च॒रुम् निः । निर् व॑पेत् । व॒पे॒द् यः । यः का॒मये॑त । का॒मये॑त॒ स्वे । स्वे᳚ऽस्मै । अ॒स्मा॒ आ॒यत॑ने । आ॒यत॑ने॒ भ्रातृ॑व्यम् । आ॒यत॑न॒ इत्या᳚ - यत॑ने । भातृ॑व्यम् जनयेयम् । ज॒न॒ये॒य॒मिति॑ । इति॒ वेदि᳚म् । वेदि॑म् परि॒गृह्य॑ । प॒रि॒गृह्या॒र्द्धम् । प॒रि॒गृह्येति॑ परि - गृह्य॑ । अ॒र्द्धमु॑द्ध॒न्यात् । उ॒द्ध॒न्याद॒र्द्धम् । उ॒द्ध॒न्यादित्यु॑त् - ह॒न्यात् । अ॒र्द्धम् न । नार्द्धम् । अ॒र्द्धम् ब॒र्.॒हिषः॑ । ब॒र्॒.हिषः॑ स्तृणी॒यात् । स्तृ॒णी॒याद॒र्द्धम् ( ) । अ॒र्द्धम् न । नार्द्धम् । अ॒र्द्धमि॒द्ध्मस्य॑ । इ॒द्ध्मस्या᳚भ्यादद्ध्यात् । अ॒भ्या॒द॒द्ध्याद॒र्द्धम् । अ॒भ्या॒द॒द्ध्यादित्य॑भि - आ॒द॒द्ध्यात् । अ॒र्द्धम् न । न स्वे । स्व ए॒व । ए॒वास्मै᳚ । अ॒स्मा॒ आ॒यत॑ने । आ॒यत॑ने॒ भ्रातृ॑व्यम् । आ॒यत॑न॒ इत्या᳚ - यत॑ने । भ्रातृ॑व्यम् जनयति । ज॒न॒य॒तीति॑ जनयति । \newline

\textbf{Jatai Paata} \newline

1. इ॒तासु॒र् भव॑ति॒ भव॑ती॒तासु॑ रि॒तासु॒र् भव॑ति । \newline
2. इ॒तासु॒रिती॒त - अ॒सुः॒ । \newline
3. भव॑ति॒ जीव॑ति॒ जीव॑ति॒ भव॑ति॒ भव॑ति॒ जीव॑ति । \newline
4. जीव॑ त्ये॒वैव जीव॑ति॒ जीव॑ त्ये॒व । \newline
5. ए॒व सो॑मारु॒द्रयोः᳚ सोमारु॒द्रयो॑ रे॒वैव सो॑मारु॒द्रयोः᳚ । \newline
6. सो॒मा॒रु॒द्रयो॒र् वै वै सो॑मारु॒द्रयोः᳚ सोमारु॒द्रयो॒र् वै । \newline
7. सो॒मा॒रु॒द्रयो॒रिति॑ सोमा - रु॒द्रयोः᳚ । \newline
8. वा ए॒त मे॒तं ॅवै वा ए॒तम् । \newline
9. ए॒तम् ग्र॑सि॒तम् ग्र॑सि॒त मे॒त मे॒तम् ग्र॑सि॒तम् । \newline
10. ग्र॒सि॒तꣳ होता॒ होता᳚ ग्रसि॒तम् ग्र॑सि॒तꣳ होता᳚ । \newline
11. होता॒ निर् णिर् होता॒ होता॒ निः । \newline
12. निष् खि॑दति खिदति॒ निर् णिष् खि॑दति । \newline
13. खि॒द॒ति॒ स स खि॑दति खिदति॒ सः । \newline
14. स ई᳚श्व॒र ई᳚श्व॒रः स स ई᳚श्व॒रः । \newline
15. ई॒श्व॒र आर्ति॒ मार्ति॑ मीश्व॒र ई᳚श्व॒र आर्ति᳚म् । \newline
16. आर्ति॒ मार्तो॒रार्तो॒रार्ति॒ मार्ति॒ मार्तोः᳚ । \newline
17. आर्तो॑ रन॒ड्वा न॑न॒ड्वा नार्तो॒ रार्तो॑ रन॒ड्वान् । \newline
18. आर्तो॒रिया -अर्तोः॒ । \newline
19. अ॒न॒ड्वान्. होत्रा॒ होत्रा॑ ऽन॒ड्वा न॑न॒ड्वान्. होत्रा᳚ । \newline
20. होत्रा॒ देयो॒ देयो॒ होत्रा॒ होत्रा॒ देयः॑ । \newline
21. देयो॒ वह्नि॒र् वह्नि॒र् देयो॒ देयो॒ वह्निः॑ । \newline
22. वह्नि॒र् वै वै वह्नि॒र् वह्नि॒र् वै । \newline
23. वा अ॑न॒ड्वा न॑न॒ड्वान्. वै वा अ॑न॒ड्वान् । \newline
24. अ॒न॒ड्वान्. वह्नि॒र् वह्नि॑ रन॒ड्वा न॑न॒ड्वान्. वह्निः॑ । \newline
25. वह्नि॒र् होता॒ होता॒ वह्नि॒र् वह्नि॒र् होता᳚ । \newline
26. होता॒ वह्नि॑ना॒ वह्नि॑ना॒ होता॒ होता॒ वह्नि॑ना । \newline
27. वह्नि॑नै॒वैव वह्नि॑ना॒ वह्नि॑नै॒व । \newline
28. ए॒व वह्निं॒ ॅवह्नि॑ मे॒वैव वह्नि᳚म् । \newline
29. वह्नि॑ मा॒त्मान॑ मा॒त्मानं॒ ॅवह्निं॒ ॅवह्नि॑ मा॒त्मान᳚म् । \newline
30. आ॒त्मानꣳ॑ स्पृणोति स्पृणो त्या॒त्मान॑ मा॒त्मानꣳ॑ स्पृणोति । \newline
31. स्पृ॒णो॒ति॒ सो॒मा॒रौ॒द्रꣳ सो॑मारौ॒द्रꣳ स्पृ॑णोति स्पृणोति सोमारौ॒द्रम् । \newline
32. सो॒मा॒रौ॒द्रम् च॒रुम् च॒रुꣳ सो॑मारौ॒द्रꣳ सो॑मारौ॒द्रम् च॒रुम् । \newline
33. सो॒मा॒रौ॒द्रमिति॑ सोमा - रौ॒द्रम् । \newline
34. च॒रुम् निर् णिश्च॒रुम् च॒रुम् निः । \newline
35. निर् व॑पेद् वपे॒न् निर् णिर् व॑पेत् । \newline
36. व॒पे॒द् यो यो व॑पेद् वपे॒द् यः । \newline
37. यः का॒मये॑त का॒मये॑त॒ यो यः का॒मये॑त । \newline
38. का॒मये॑त॒ स्वे स्वे का॒मये॑त का॒मये॑त॒ स्वे । \newline
39. स्वे᳚ ऽस्मा अस्मै॒ स्वे स्वे᳚ ऽस्मै । \newline
40. अ॒स्मा॒ आ॒यत॑न आ॒यत॑ने ऽस्मा अस्मा आ॒यत॑ने । \newline
41. आ॒यत॑ने॒ भ्रातृ॑व्य॒म् भ्रातृ॑व्य मा॒यत॑न आ॒यत॑ने॒ भ्रातृ॑व्यम् । \newline
42. आ॒यत॑न॒ इत्या᳚ - यत॑ने । \newline
43. भ्रातृ॑व्यम् जनयेयम् जनयेय॒म् भ्रातृ॑व्य॒म् भ्रातृ॑व्यम् जनयेयम् । \newline
44. ज॒न॒ये॒य॒ मितीति॑ जनयेयम् जनयेय॒ मिति॑ । \newline
45. इति॒ वेदिं॒ ॅवेदि॒ मितीति॒ वेदि᳚म् । \newline
46. वेदि॑म् परि॒गृह्य॑ परि॒गृह्य॒ वेदिं॒ ॅवेदि॑म् परि॒गृह्य॑ । \newline
47. प॒रि॒गृह्या॒र्द्ध म॒र्द्धम् प॑रि॒गृह्य॑ परि॒गृह्या॒र्द्धम् । \newline
48. प॒रि॒गृह्येति॑ परि - गृह्य॑ । \newline
49. अ॒र्द्ध मु॑द्ध॒न्या दु॑द्ध॒न्या द॒र्द्ध म॒र्द्ध मु॑द्ध॒न्यात् । \newline
50. उ॒द्ध॒न्या द॒र्द्ध म॒र्द्ध मु॑द्ध॒न्या दु॑द्ध॒न्या द॒र्द्धम् । \newline
51. उ॒द्ध॒न्यादित्यु॑त् - ह॒न्यात् । \newline
52. अ॒र्द्धम् न नार्द्ध म॒र्द्धम् न । \newline
53. नार्द्ध म॒र्द्धम् न नार्द्धम् । \newline
54. अ॒र्द्धम् ब॒र्॒.हिषो॑ ब॒र्॒.हिषो॒ ऽर्द्ध म॒र्द्धम् ब॒र्॒.हिषः॑ । \newline
55. ब॒र्॒.हिषः॑ स्तृणी॒याथ् स्तृ॑णी॒याद् ब॒र्॒.हिषो॑ ब॒र्॒.हिषः॑ स्तृणी॒यात् । \newline
56. स्तृ॒णी॒या द॒र्द्ध म॒र्द्धꣳ स्तृ॑णी॒याथ् स्तृ॑णी॒या द॒र्द्धम् । \newline
57. अ॒र्द्धम् न नार्द्ध म॒र्द्धम् न । \newline
58. नार्द्ध म॒र्द्धम् न नार्द्धम् । \newline
59. अ॒र्द्ध मि॒द्ध्मस्ये॒ द्ध्मस्या॒र्द्ध म॒र्द्ध मि॒द्ध्मस्य॑ । \newline
60. इ॒द्ध्मस्या᳚ भ्याद॒द्ध्या द॑भ्याद॒द्ध्या दि॒द्ध्मस्ये॒ द्ध्मस्या᳚ भ्याद॒द्ध्यात् । \newline
61. अ॒भ्या॒द॒द्ध्या द॒र्द्ध म॒र्द्ध म॑भ्याद॒द्ध्या द॑भ्याद॒द्ध्या द॒र्द्धम् । \newline
62. अ॒भ्या॒द॒द्ध्यादित्य॑भि - आ॒द॒द्ध्यात् । \newline
63. अ॒र्द्धम् न नार्द्ध म॒र्द्धम् न । \newline
64. न स्वे स्वे न न स्वे । \newline
65. स्व ए॒वैव स्वे स्व ए॒व । \newline
66. ए॒वास्मा॑ अस्मा ए॒वैवास्मै᳚ । \newline
67. अ॒स्मा॒ आ॒यत॑न आ॒यत॑ने ऽस्मा अस्मा आ॒यत॑ने । \newline
68. आ॒यत॑ने॒ भ्रातृ॑व्य॒म् भ्रातृ॑व्य मा॒यत॑न आ॒यत॑ने॒ भ्रातृ॑व्यम् । \newline
69. आ॒यत॑न॒ इत्या᳚ - यत॑ने । \newline
70. भ्रातृ॑व्यम् जनयति जनयति॒ भ्रातृ॑व्य॒म् भ्रातृ॑व्यम् जनयति । \newline
71. ज॒न॒य॒तीति॑ जनयति । \newline

\textbf{Ghana Paata } \newline

1. इ॒तासु॒र् भव॑ति॒ भव॑ती॒ तासु॑ रि॒तासु॒र् भव॑ति॒ जीव॑ति॒ जीव॑ति॒ भव॑ती॒ तासु॑ रि॒तासु॒र् भव॑ति॒ जीव॑ति । \newline
2. इ॒तासु॒रिती॒त - अ॒सुः॒ । \newline
3. भव॑ति॒ जीव॑ति॒ जीव॑ति॒ भव॑ति॒ भव॑ति॒ जीव॑ त्ये॒वैव जीव॑ति॒ भव॑ति॒ भव॑ति॒ जीव॑ त्ये॒व । \newline
4. जीव॑ त्ये॒वैव जीव॑ति॒ जीव॑ त्ये॒व सो॑मारु॒द्रयोः᳚ सोमारु॒द्रयो॑ रे॒व जीव॑ति॒ जीव॑ त्ये॒व सो॑मारु॒द्रयोः᳚ । \newline
5. ए॒व सो॑मारु॒द्रयोः᳚ सोमारु॒द्रयो॑ रे॒वैव सो॑मारु॒द्रयो॒र् वै वै सो॑मारु॒द्रयो॑ रे॒वैव सो॑मारु॒द्रयो॒र् वै । \newline
6. सो॒मा॒रु॒द्रयो॒र् वै वै सो॑मारु॒द्रयोः᳚ सोमारु॒द्रयो॒र् वा ए॒त मे॒तं ॅवै सो॑मारु॒द्रयोः᳚ सोमारु॒द्रयो॒र् वा ए॒तम् । \newline
7. सो॒मा॒रु॒द्रयो॒रिति॑ सोमा - रु॒द्रयोः᳚ । \newline
8. वा ए॒त मे॒तं ॅवै वा ए॒तम् ग्र॑सि॒तम् ग्र॑सि॒त मे॒तं ॅवै वा ए॒तम् ग्र॑सि॒तम् । \newline
9. ए॒तम् ग्र॑सि॒तम् ग्र॑सि॒त मे॒त मे॒तम् ग्र॑सि॒तꣳ होता॒ होता᳚ ग्रसि॒त मे॒त मे॒तम् ग्र॑सि॒तꣳ होता᳚ । \newline
10. ग्र॒सि॒तꣳ होता॒ होता᳚ ग्रसि॒तम् ग्र॑सि॒तꣳ होता॒ निर् णिर् होता᳚ ग्रसि॒तम् ग्र॑सि॒तꣳ होता॒ निः । \newline
11. होता॒ निर् णिर् होता॒ होता॒ निष् खि॑दति खिदति॒ निर् होता॒ होता॒ निष् खि॑दति । \newline
12. निष् खि॑दति खिदति॒ निर् णिष् खि॑दति॒ स स खि॑दति॒ निर् णिष् खि॑दति॒ सः । \newline
13. खि॒द॒ति॒ स स खि॑दति खिदति॒ स ई᳚श्व॒र ई᳚श्व॒रः स खि॑दति खिदति॒ स ई᳚श्व॒रः । \newline
14. स ई᳚श्व॒र ई᳚श्व॒रः स स ई᳚श्व॒र आर्ति॒ मार्ति॑ मीश्व॒रः स स ई᳚श्व॒र आर्ति᳚म् । \newline
15. ई॒श्व॒र आर्ति॒ मार्ति॑ मीश्व॒र ई᳚श्व॒र आर्ति॒ मार्तो॒ रार्तो॒ रार्ति॑ मीश्व॒र ई᳚श्व॒र आर्ति॒ मार्तोः᳚ । \newline
16. आर्ति॒ मार्तो॒ रार्तो॒ रार्ति॒ मार्ति॒ मार्तो॑ रन॒ड्वा न॑न॒ड्वा नार्तो॒ रार्ति॒ मार्ति॒ मार्तो॑ रन॒ड्वान् । \newline
17. आर्तो॑ रन॒ड्वा न॑न॒ड्वा नार्तो॒ रार्तो॑ रन॒ड्वान्. होत्रा॒ होत्रा॑ ऽन॒ड्वा नार्तो॒ रार्तो॑ रन॒ड्वान्. होत्रा᳚ । \newline
18. आर्तो॒रिया -अर्तोः॒ । \newline
19. अ॒न॒ड्वान्. होत्रा॒ होत्रा॑ ऽन॒ड्वा न॑न॒ड्वान्. होत्रा॒ देयो॒ देयो॒ होत्रा॑ ऽन॒ड्वा न॑न॒ड्वान्. होत्रा॒ देयः॑ । \newline
20. होत्रा॒ देयो॒ देयो॒ होत्रा॒ होत्रा॒ देयो॒ वह्नि॒र् वह्नि॒र् देयो॒ होत्रा॒ होत्रा॒ देयो॒ वह्निः॑ । \newline
21. देयो॒ वह्नि॒र् वह्नि॒र् देयो॒ देयो॒ वह्नि॒र् वै वै वह्नि॒र् देयो॒ देयो॒ वह्नि॒र् वै । \newline
22. वह्नि॒र् वै वै वह्नि॒र् वह्नि॒र् वा अ॑न॒ड्वा न॑न॒ड्वान्. वै वह्नि॒र् वह्नि॒र् वा अ॑न॒ड्वान् । \newline
23. वा अ॑न॒ड्वा न॑न॒ड्वान्. वै वा अ॑न॒ड्वान्. वह्नि॒र् वह्नि॑ रन॒ड्वान्. वै वा अ॑न॒ड्वान्. वह्निः॑ । \newline
24. अ॒न॒ड्वान्. वह्नि॒र् वह्नि॑ रन॒ड्वा न॑न॒ड्वान्. वह्नि॒र् होता॒ होता॒ वह्नि॑ रन॒ड्वा न॑न॒ड्वान्. वह्नि॒र् होता᳚ । \newline
25. वह्नि॒र् होता॒ होता॒ वह्नि॒र् वह्नि॒र् होता॒ वह्नि॑ना॒ वह्नि॑ना॒ होता॒ वह्नि॒र् वह्नि॒र् होता॒ वह्नि॑ना । \newline
26. होता॒ वह्नि॑ना॒ वह्नि॑ना॒ होता॒ होता॒ वह्नि॑नै॒वैव वह्नि॑ना॒ होता॒ होता॒ वह्नि॑नै॒व । \newline
27. वह्नि॑नै॒वैव वह्नि॑ना॒ वह्नि॑नै॒व वह्निं॒ ॅवह्नि॑ मे॒व वह्नि॑ना॒ वह्नि॑नै॒व वह्नि᳚म् । \newline
28. ए॒व वह्निं॒ ॅवह्नि॑ मे॒वैव वह्नि॑ मा॒त्मान॑ मा॒त्मानं॒ ॅवह्नि॑ मे॒वैव वह्नि॑ मा॒त्मान᳚म् । \newline
29. वह्नि॑ मा॒त्मान॑ मा॒त्मानं॒ ॅवह्निं॒ ॅवह्नि॑ मा॒त्मानꣳ॑ स्पृणोति स्पृणो त्या॒त्मानं॒ ॅवह्निं॒ ॅवह्नि॑ मा॒त्मानꣳ॑ स्पृणोति । \newline
30. आ॒त्मानꣳ॑ स्पृणोति स्पृणो त्या॒त्मान॑ मा॒त्मानꣳ॑ स्पृणोति सोमारौ॒द्रꣳ सो॑मारौ॒द्रꣳ स्पृ॑णो त्या॒त्मान॑ मा॒त्मानꣳ॑ स्पृणोति सोमारौ॒द्रम् । \newline
31. स्पृ॒णो॒ति॒ सो॒मा॒रौ॒द्रꣳ सो॑मारौ॒द्रꣳ स्पृ॑णोति स्पृणोति सोमारौ॒द्रम् च॒रुम् च॒रुꣳ सो॑मारौ॒द्रꣳ स्पृ॑णोति स्पृणोति सोमारौ॒द्रम् च॒रुम् । \newline
32. सो॒मा॒रौ॒द्रम् च॒रुम् च॒रुꣳ सो॑मारौ॒द्रꣳ सो॑मारौ॒द्रम् च॒रुम् निर् णिश्च॒रुꣳ सो॑मारौ॒द्रꣳ सो॑मारौ॒द्रम् च॒रुम् निः । \newline
33. सो॒मा॒रौ॒द्रमिति॑ सोमा - रौ॒द्रम् । \newline
34. च॒रुम् निर् णिश्च॒रुम् च॒रुम् निर् व॑पेद् वपे॒न् निश्च॒रुम् च॒रुम् निर् व॑पेत् । \newline
35. निर् व॑पेद् वपे॒न् निर् णिर् व॑पे॒द् यो यो व॑पे॒न् निर् णिर् व॑पे॒द् यः । \newline
36. व॒पे॒द् यो यो व॑पेद् वपे॒द् यः का॒मये॑त का॒मये॑त॒ यो व॑पेद् वपे॒द् यः का॒मये॑त । \newline
37. यः का॒मये॑त का॒मये॑त॒ यो यः का॒मये॑त॒ स्वे स्वे का॒मये॑त॒ यो यः का॒मये॑त॒ स्वे । \newline
38. का॒मये॑त॒ स्वे स्वे का॒मये॑त का॒मये॑त॒ स्वे᳚ ऽस्मा अस्मै॒ स्वे का॒मये॑त का॒मये॑त॒ स्वे᳚ ऽस्मै । \newline
39. स्वे᳚ ऽस्मा अस्मै॒ स्वे स्वे᳚ ऽस्मा आ॒यत॑न आ॒यत॑ने ऽस्मै॒ स्वे स्वे᳚ ऽस्मा आ॒यत॑ने । \newline
40. अ॒स्मा॒ आ॒यत॑न आ॒यत॑ने ऽस्मा अस्मा आ॒यत॑ने॒ भ्रातृ॑व्य॒म् भ्रातृ॑व्य मा॒यत॑ने ऽस्मा अस्मा आ॒यत॑ने॒ भ्रातृ॑व्यम् । \newline
41. आ॒यत॑ने॒ भ्रातृ॑व्य॒म् भ्रातृ॑व्य मा॒यत॑न आ॒यत॑ने॒ भ्रातृ॑व्यम् जनयेयम् जनयेय॒म् भ्रातृ॑व्य मा॒यत॑न आ॒यत॑ने॒ भ्रातृ॑व्यम् जनयेयम् । \newline
42. आ॒यत॑न॒ इत्या᳚ - यत॑ने । \newline
43. भ्रातृ॑व्यम् जनयेयम् जनयेय॒म् भ्रातृ॑व्य॒म् भ्रातृ॑व्यम् जनयेय॒ मितीति॑ जनयेय॒म् भ्रातृ॑व्य॒म् भ्रातृ॑व्यम् जनयेय॒ मिति॑ । \newline
44. ज॒न॒ये॒य॒ मितीति॑ जनयेयम् जनयेय॒ मिति॒ वेदिं॒ ॅवेदि॒ मिति॑ जनयेयम् जनयेय॒ मिति॒ वेदि᳚म् । \newline
45. इति॒ वेदिं॒ ॅवेदि॒ मितीति॒ वेदि॑म् परि॒गृह्य॑ परि॒गृह्य॒ वेदि॒ मितीति॒ वेदि॑म् परि॒गृह्य॑ । \newline
46. वेदि॑म् परि॒गृह्य॑ परि॒गृह्य॒ वेदिं॒ ॅवेदि॑म् परि॒गृह्या॒र्द्ध म॒र्द्धम् प॑रि॒गृह्य॒ वेदिं॒ ॅवेदि॑म् परि॒गृह्या॒र्द्धम् । \newline
47. प॒रि॒गृह्या॒र्द्ध म॒र्द्धम् प॑रि॒गृह्य॑ परि॒गृह्या॒र्द्ध मु॑द्ध॒न्या दु॑द्ध॒न्या द॒र्द्धम् प॑रि॒गृह्य॑ परि॒गृह्या॒र्द्ध मु॑द्ध॒न्यात् । \newline
48. प॒रि॒गृह्येति॑ परि - गृह्य॑ । \newline
49. अ॒र्द्ध मु॑द्ध॒न्या दु॑द्ध॒न्या द॒र्द्ध म॒र्द्ध मु॑द्ध॒न्या द॒र्द्ध म॒र्द्ध मु॑द्ध॒न्या द॒र्द्ध म॒र्द्ध मु॑द्ध॒न्या द॒र्द्धम् । \newline
50. उ॒द्ध॒न्या द॒र्द्ध म॒र्द्ध मु॑द्ध॒न्या दु॑द्ध॒न्याद॒र् द्धम् न नार्द्ध मु॑द्ध॒न्या दु॑द्ध॒न्या द॒र्द्धम् न । \newline
51. उ॒द्ध॒न्यादित्यु॑त् - ह॒न्यात् । \newline
52. अ॒र्द्धम् न नार्द्ध म॒र्द्धम् नार्द्ध म॒र्द्धम् नार्द्ध म॒र्द्धम् नार्द्धम् । \newline
53. नार्द्ध म॒र्द्धम् न नार्द्धम् ब॒र्॒.हिषो॑ ब॒र्॒.हिषो॒ ऽर्द्धम् न नार्द्धम् ब॒र्॒.हिषः॑ । \newline
54. अ॒र्द्धम् ब॒र्॒.हिषो॑ ब॒र्॒.हिषो॒ ऽर्द्ध म॒र्द्धम् ब॒र्॒.हिषः॑ स्तृणी॒याथ् स्तृ॑णी॒याद् ब॒र्॒.हिषो॒ ऽर्द्ध म॒र्द्धम् ब॒र्॒.हिषः॑ स्तृणी॒यात् । \newline
55. ब॒र्॒.हिषः॑ स्तृणी॒याथ् स्तृ॑णी॒याद् ब॒र्॒.हिषो॑ ब॒र्॒.हिषः॑ स्तृणी॒या द॒र्द्ध म॒र्द्धꣳ स्तृ॑णी॒याद् ब॒र्॒.हिषो॑ ब॒र्॒.हिषः॑ स्तृणी॒या द॒र्द्धम् । \newline
56. स्तृ॒णी॒या द॒र्द्ध म॒र्द्धꣳ स्तृ॑णी॒याथ् स्तृ॑णी॒या द॒र्द्धम् न नार्द्धꣳ स्तृ॑णी॒याथ् स्तृ॑णी॒या द॒र्द्धम् न । \newline
57. अ॒र्द्धम् न नार्द्ध म॒र्द्धम् नार्द्ध म॒र्द्धम् नार्द्ध म॒र्द्धम् नार्द्धम् । \newline
58. नार्द्ध म॒र्द्धम् न नार्द्ध मि॒द्ध्मस्ये॒ द्ध्मस्या॒र्द्धम् न नार्द्ध मि॒द्ध्मस्य॑ । \newline
59. अ॒र्द्ध मि॒द्ध्मस्ये॒ द्ध्मस्या॒र्द्ध म॒र्द्ध मि॒द्ध्मस्या᳚ भ्याद॒द्ध्या द॑भ्याद॒द्ध्या दि॒द्ध्मस्या॒र्द्ध म॒र्द्ध मि॒द्ध्मस्या᳚ भ्याद॒द्ध्यात् । \newline
60. इ॒द्ध्मस्या᳚ भ्याद॒द्ध्या द॑भ्याद॒द्ध्या दि॒द्ध्मस्ये॒ द्ध्मस्या᳚ भ्याद॒द्ध्याद॒र्द्ध म॒र्द्ध म॑भ्याद॒द्ध्या दि॒द्ध्मस्ये॒ द्ध्मस्या᳚भ्याद॒द्ध्या द॒र्द्धम् । \newline
61. अ॒भ्या॒द॒द्ध्याद॒र्द्ध म॒र्द्ध म॑भ्याद॒द्ध्या द॑भ्याद॒द्ध्याद॒र्द्धम् न नार्द्ध म॑भ्याद॒द्ध्या 
द॑भ्याद॒द्ध्याद॒र्द्धम् न । \newline
62. अ॒भ्या॒द॒द्ध्यादित्य॑भि - आ॒द॒द्ध्यात् । \newline
63. अ॒र्द्धम् न नार्द्ध म॒र्द्धम् न स्वे स्वे नार्द्ध म॒र्द्धम् न स्वे । \newline
64. न स्वे स्वे न न स्व ए॒वैव स्वे न न स्व ए॒व । \newline
65. स्व ए॒वैव स्वे स्व ए॒वास्मा॑ अस्मा ए॒व स्वे स्व ए॒वास्मै᳚ । \newline
66. ए॒वास्मा॑ अस्मा ए॒वैवास्मा॑ आ॒यत॑न आ॒यत॑ने ऽस्मा ए॒वैवास्मा॑ आ॒यत॑ने । \newline
67. अ॒स्मा॒ आ॒यत॑न आ॒यत॑ने ऽस्मा अस्मा आ॒यत॑ने॒ भ्रातृ॑व्य॒म् भ्रातृ॑व्य मा॒यत॑ने ऽस्मा अस्मा आ॒यत॑ने॒ भ्रातृ॑व्यम् । \newline
68. आ॒यत॑ने॒ भ्रातृ॑व्य॒म् भ्रातृ॑व्य मा॒यत॑न आ॒यत॑ने॒ भ्रातृ॑व्यम् जनयति जनयति॒ भ्रातृ॑व्य मा॒यत॑न आ॒यत॑ने॒ भ्रातृ॑व्यम् जनयति । \newline
69. आ॒यत॑न॒ इत्या᳚ - यत॑ने । \newline
70. भ्रातृ॑व्यम् जनयति जनयति॒ भ्रातृ॑व्य॒म् भ्रातृ॑व्यम् जनयति । \newline
71. ज॒न॒य॒तीति॑ जनयति । \newline
\pagebreak
\markright{ TS 2.2.11.1  \hfill https://www.vedavms.in \hfill}
\addcontentsline{toc}{section}{ TS 2.2.11.1 }
\section*{ TS 2.2.11.1 }

\textbf{TS 2.2.11.1 } \newline
\textbf{Samhita Paata} \newline

ऐ॒न्द्रमेका॑दशकपालं॒ निर्व॑पेन्मारु॒तꣳ स॒प्तक॑पालं॒ ग्राम॑काम॒ इन्द्रं॑ चै॒व म॒रुत॑श्च॒ स्वेन॑ भाग॒धेये॒नोप॑ धावति॒ त ए॒वास्मै॑ सजा॒तान् प्रय॑च्छन्ति ग्रा॒म्ये॑व भ॑वत्याहव॒नीय॑ ऐ॒न्द्रमधि॑ श्रयति॒ गार्.ह॑पत्ये मारु॒तं पा॑पवस्य॒सस्य॒ विधृ॑त्यै स॒प्तक॑पालो मारु॒तो भ॑वति स॒प्तग॑णा॒ वै म॒रुतो॑गण॒श ए॒वास्मै॑ सजा॒तानव॑ रुन्धेऽनू॒च्यमा॑न॒ आ सा॑दयति॒ विश॑मे॒वा - [  ] \newline

\textbf{Pada Paata} \newline

ऐ॒न्द्रम् । एका॑दशकपाल॒मित्येका॑दश - क॒पा॒ल॒म् । निरिति॑ । व॒पे॒त् । मा॒रु॒तम् । स॒प्तक॑पाल॒मिति॑ स॒प्त - क॒पा॒ल॒म् । ग्राम॑काम॒ इति॒ ग्राम॑ - का॒मः॒ । इन्द्र᳚म् । च॒ । ए॒व । म॒रुतः॑ । च॒ । स्वेन॑ । भा॒ग॒धेये॒नेति॑ भाग - धेये॑न । उपेति॑ । धा॒व॒ति॒ । ते । ए॒व । अ॒स्मै॒ । स॒जा॒तानिति॑ स-जा॒तान् । प्रेति॑ । य॒च्छ॒न्ति॒ । ग्रा॒मी । ए॒व । भ॒व॒ति॒ । आ॒ह॒व॒नीय॒ इत्या᳚ - ह॒व॒नीये᳚ । ऐ॒न्द्रम् । अधीति॑ । श्र॒य॒ति॒ । गार्.ह॑पत्य॒ इति॒ गार्.ह॑ - प॒त्ये॒ । मा॒रु॒तम् । पा॒प॒व॒स्य॒सस्येति॑ पाप - व॒स्य॒सस्य॑ । विधृ॑त्या॒ इति॒ वि - धृ॒त्यै॒ । स॒प्तक॑पाल॒ इति॑ स॒प्त - क॒पा॒लः॒ । मा॒रु॒तः । भ॒व॒ति॒ । स॒प्तग॑णा॒ इति॑ स॒प्त - ग॒णाः॒ । वै । म॒रुतः॑ । ग॒ण॒श इति॑ गण-शः । ए॒व । अ॒स्मै॒ । स॒जा॒तानिति॑ स - जा॒तान् । अवेति॑ । रु॒न्धे॒ । अ॒नू॒च्यमा॑न॒ इत्य॑नु - उ॒च्यमा॑ने । एति॑ । सा॒द॒य॒ति॒ । विश᳚म् । ए॒व ।  \newline


\textbf{Krama Paata} \newline

ऐ॒न्द्रमेका॑दशकपालम् । एका॑दशकपाल॒म् निः । एका॑दशकपाल॒मित्येका॑दश - क॒पा॒ल॒म् । निर् व॑पेत् । व॒पे॒न् मा॒रु॒तम् । मा॒रु॒तꣳ स॒प्तक॑पालम् । स॒प्तक॑पाल॒म् ग्राम॑कामः । स॒प्तक॑पाल॒मिति॑ स॒प्त - क॒पा॒ल॒म् । ग्राम॑काम॒ इन्द्र᳚म् । ग्राम॑काम॒ इति॒ ग्राम॑ - का॒मः॒ । इन्द्र॑म् च । चै॒व । ए॒व म॒रुतः॑ । म॒रुत॑श्च । च॒ स्वेन॑ । स्वेन॑ भाग॒धेये॑न । भा॒ग॒धेये॒नोप॑ । भा॒ग॒धेये॒नेति॑ भाग - धेये॑न । उप॑ धावति । धा॒व॒ति॒ ते । त ए॒व । ए॒वास्मै᳚ । अ॒स्मै॒ स॒जा॒तान् । स॒जा॒तान् प्र । स॒जा॒तानिति॑ स - जा॒तान् । प्र य॑च्छन्ति । य॒च्छ॒न्ति॒ ग्रा॒मी । ग्रा॒म्ये॑व । ए॒व भ॑वति । भ॒व॒त्या॒ह॒व॒नीये᳚ । आ॒ह॒व॒नीय॑ ऐ॒न्द्रम् । आ॒ह॒व॒नीय॒ इत्या᳚ - ह॒व॒नीये᳚ । ऐ॒न्द्रमधि॑ । अधि॑ श्रयति । श्र॒य॒ति॒ गार्.ह॑पत्ये । गार्.ह॑पत्ये मारु॒तम् । गार्.ह॑पत्य॒ इति॒ गार्.ह॑ - प॒त्ये॒ । मा॒रु॒तम् पा॑पवस्य॒सस्य॑ । पा॒प॒व॒स्य॒सस्य॒ विधृ॑त्यै । पा॒प॒व॒स्य॒सस्येति॑ पाप - व॒स्य॒सस्य॑ । विधृ॑त्यै स॒प्तक॑पालः । विधृ॑त्या॒ इति॒ वि - धृ॒त्यै॒ । स॒प्तक॑पालो मारु॒तः । स॒प्तक॑पाल॒ इति॑ स॒प्त - क॒पा॒लः॒ । मा॒रु॒तो भ॑वति । भ॒व॒ति॒ स॒प्तग॑णाः । स॒प्तग॑णा॒ वै । स॒प्तग॑णा॒ इति॑ स॒प्त - ग॒णाः॒ । वै म॒रुतः॑ । म॒रुतो॑ गण॒शः । ग॒ण॒श ए॒व । ग॒ण॒श इति॑ गण - शः । ए॒वास्मै᳚ । अ॒स्मै॒ स॒जा॒तान् । स॒जा॒तानव॑ । स॒जा॒तानिति॑ स - जा॒तान् । अव॑ रुन्धे । रु॒न्धे॒ऽनू॒च्यमा॑ने । अ॒नू॒च्यमा॑न॒ आ । अ॒नू॒च्यमा॑न॒ इत्य॑नु - उ॒च्यमा॑ने । आ सा॑दयति । सा॒द॒य॒ति॒ विश᳚म् । विश॑मे॒व । ए॒वास्मै᳚ \newline

\textbf{Jatai Paata} \newline

1. ऐ॒न्द्र मेका॑दशकपाल॒ मेका॑दशकपाल मै॒न्द्र मै॒न्द्र मेका॑दशकपालम् । \newline
2. एका॑दशकपाल॒म् निर् णि रेका॑दशकपाल॒ मेका॑दशकपाल॒म् निः । \newline
3. एका॑दशकपाल॒मित्येका॑दश - क॒पा॒ल॒म् । \newline
4. निर् व॑पेद् वपे॒न् निर् णिर् व॑पेत् । \newline
5. व॒पे॒न् मा॒रु॒तम् मा॑रु॒तं ॅव॑पेद् वपेन् मारु॒तम् । \newline
6. मा॒रु॒तꣳ स॒प्तक॑पालꣳ स॒प्तक॑पालम् मारु॒तम् मा॑रु॒तꣳ स॒प्तक॑पालम् । \newline
7. स॒प्तक॑पाल॒म् ग्राम॑कामो॒ ग्राम॑कामः स॒प्तक॑पालꣳ स॒प्तक॑पाल॒म् ग्राम॑कामः । \newline
8. स॒प्तक॑पाल॒मिति॑ स॒प्त - क॒पा॒ल॒म् । \newline
9. ग्राम॑काम॒ इन्द्र॒ मिन्द्र॒म् ग्राम॑कामो॒ ग्राम॑काम॒ इन्द्र᳚म् । \newline
10. ग्राम॑काम॒ इति॒ ग्राम॑ - का॒मः॒ । \newline
11. इन्द्र॑म् च॒ चे न्द्र॒ मिन्द्र॑म् च । \newline
12. चै॒वैव च॑ चै॒व । \newline
13. ए॒व म॒रुतो॑ म॒रुत॑ ए॒वैव म॒रुतः॑ । \newline
14. म॒रुत॑श्च च म॒रुतो॑ म॒रुत॑श्च । \newline
15. च॒ स्वेन॒ स्वेन॑ च च॒ स्वेन॑ । \newline
16. स्वेन॑ भाग॒धेये॑न भाग॒धेये॑न॒ स्वेन॒ स्वेन॑ भाग॒धेये॑न । \newline
17. भा॒ग॒धेये॒नोपोप॑ भाग॒धेये॑न भाग॒धेये॒नोप॑ । \newline
18. भा॒ग॒धेये॒नेति॑ भाग - धेये॑न । \newline
19. उप॑ धावति धाव॒ त्युपोप॑ धावति । \newline
20. धा॒व॒ति॒ ते ते धा॑वति धावति॒ ते । \newline
21. त ए॒वैव ते त ए॒व । \newline
22. ए॒वास्मा॑ अस्मा ए॒वैवास्मै᳚ । \newline
23. अ॒स्मै॒ स॒जा॒तान् थ्स॑जा॒ता न॑स्मा अस्मै सजा॒तान् । \newline
24. स॒जा॒तान् प्र प्र स॑जा॒तान् थ्स॑जा॒तान् प्र । \newline
25. स॒जा॒तानिति॑ स - जा॒तान् । \newline
26. प्र य॑च्छन्ति यच्छन्ति॒ प्र प्र य॑च्छन्ति । \newline
27. य॒च्छ॒न्ति॒ ग्रा॒मी ग्रा॒मी य॑च्छन्ति यच्छन्ति ग्रा॒मी । \newline
28. ग्रा॒म्ये॑वैव ग्रा॒मी ग्रा॒म्ये॑व । \newline
29. ए॒व भ॑वति भव त्ये॒वैव भ॑वति । \newline
30. भ॒व॒ त्या॒ह॒व॒नीय॑ आहव॒नीये॑ भवति भव त्याहव॒नीये᳚ । \newline
31. आ॒ह॒व॒नीय॑ ऐ॒न्द्र मै॒न्द्र मा॑हव॒नीय॑ आहव॒नीय॑ ऐ॒न्द्रम् । \newline
32. आ॒ह॒व॒नीय॒ इत्या᳚ - ह॒व॒नीये᳚ । \newline
33. ऐ॒न्द्र मध्यध्यै॒न्द्र मै॒न्द्र मधि॑ । \newline
34. अधि॑ श्रयति श्रय॒ त्यध्यधि॑ श्रयति । \newline
35. श्र॒य॒ति॒ गार्.ह॑पत्ये॒ गार्.ह॑पत्ये श्रयति श्रयति॒ गार्.ह॑पत्ये । \newline
36. गार्.ह॑पत्ये मारु॒तम् मा॑रु॒तम् गार्.ह॑पत्ये॒ गार्.ह॑पत्ये मारु॒तम् । \newline
37. गार्.ह॑पत्य॒ इति॒ गार्.ह॑ - प॒त्ये॒ । \newline
38. मा॒रु॒तम् पा॑पवस्य॒सस्य॑ पापवस्य॒सस्य॑ मारु॒तम् मा॑रु॒तम् पा॑पवस्य॒सस्य॑ । \newline
39. पा॒प॒व॒स्य॒सस्य॒ विधृ॑त्यै॒ विधृ॑त्यै पापवस्य॒सस्य॑ पापवस्य॒सस्य॒ विधृ॑त्यै । \newline
40. पा॒प॒व॒स्य॒सस्येति॑ पाप - व॒स्य॒सस्य॑ । \newline
41. विधृ॑त्यै स॒प्तक॑पालः स॒प्तक॑पालो॒ विधृ॑त्यै॒ विधृ॑त्यै स॒प्तक॑पालः । \newline
42. विधृ॑त्या॒ इति॒ वि - धृ॒त्यै॒ । \newline
43. स॒प्तक॑पालो मारु॒तो मा॑रु॒तः स॒प्तक॑पालः स॒प्तक॑पालो मारु॒तः । \newline
44. स॒प्तक॑पाल॒ इति॑ स॒प्त - क॒पा॒लः॒ । \newline
45. मा॒रु॒तो भ॑वति भवति मारु॒तो मा॑रु॒तो भ॑वति । \newline
46. भ॒व॒ति॒ स॒प्तग॑णाः स॒प्तग॑णा भवति भवति स॒प्तग॑णाः । \newline
47. स॒प्तग॑णा॒ वै वै स॒प्तग॑णाः स॒प्तग॑णा॒ वै । \newline
48. स॒प्तग॑णा॒ इति॑ स॒प्त - ग॒णाः॒ । \newline
49. वै म॒रुतो॑ म॒रुतो॒ वै वै म॒रुतः॑ । \newline
50. म॒रुतो॑ गण॒शो ग॑ण॒शो म॒रुतो॑ म॒रुतो॑ गण॒शः । \newline
51. ग॒ण॒श ए॒वैव ग॑ण॒शो ग॑ण॒श ए॒व । \newline
52. ग॒ण॒श इति॑ गण - शः । \newline
53. ए॒वास्मा॑ अस्मा ए॒वैवास्मै᳚ । \newline
54. अ॒स्मै॒ स॒जा॒तान् थ्स॑जा॒ता न॑स्मा अस्मै सजा॒तान् । \newline
55. स॒जा॒ता नवाव॑ सजा॒तान् थ्स॑जा॒ता नव॑ । \newline
56. स॒जा॒तानिति॑ स - जा॒तान् । \newline
57. अव॑ रुन्धे रु॒न्धे ऽवाव॑ रुन्धे । \newline
58. रु॒न्धे॒ ऽनू॒च्यमा॑ने ऽनू॒च्यमा॑ने रुन्धे रुन्धे ऽनू॒च्यमा॑ने । \newline
59. अ॒नू॒च्यमा॑न॒ आ ऽनू॒च्यमा॑ने ऽनू॒च्यमा॑न॒ आ । \newline
60. अ॒नू॒च्यमा॑न॒ इत्य॑नु - उ॒च्यमा॑ने । \newline
61. आ सा॑दयति सादय॒त्या सा॑दयति । \newline
62. सा॒द॒य॒ति॒ विशं॒ ॅविशꣳ॑ सादयति सादयति॒ विश᳚म् । \newline
63. विश॑ मे॒वैव विशं॒ ॅविश॑ मे॒व । \newline
64. ए॒वास्मा॑ अस्मा ए॒वैवास्मै᳚ । \newline

\textbf{Ghana Paata } \newline

1. ऐ॒न्द्र मेका॑दशकपाल॒ मेका॑दशकपाल मै॒न्द्र मै॒न्द्र मेका॑दशकपाल॒म् निर् णिरेका॑दशकपाल मै॒न्द्र मै॒न्द्र मेका॑दशकपाल॒म् निः । \newline
2. एका॑दशकपाल॒म् निर् णिरेका॑दशकपाल॒ मेका॑दशकपाल॒म् निर् व॑पेद् वपे॒न् निरेका॑दशकपाल॒ मेका॑दशकपाल॒म् निर् व॑पेत् । \newline
3. एका॑दशकपाल॒मित्येका॑दश - क॒पा॒ल॒म् । \newline
4. निर् व॑पेद् वपे॒न् निर् णिर् व॑पेन् मारु॒तम् मा॑रु॒तं ॅव॑पे॒न् निर् णिर् व॑पेन् मारु॒तम् । \newline
5. व॒पे॒न् मा॒रु॒तम् मा॑रु॒तं ॅव॑पेद् वपेन् मारु॒तꣳ स॒प्तक॑पालꣳ स॒प्तक॑पालम् मारु॒तं ॅव॑पेद् वपेन् मारु॒तꣳ स॒प्तक॑पालम् । \newline
6. मा॒रु॒तꣳ स॒प्तक॑पालꣳ स॒प्तक॑पालम् मारु॒तम् मा॑रु॒तꣳ स॒प्तक॑पाल॒म् ग्राम॑कामो॒ ग्राम॑कामः स॒प्तक॑पालम् मारु॒तम् मा॑रु॒तꣳ स॒प्तक॑पाल॒म् ग्राम॑कामः । \newline
7. स॒प्तक॑पाल॒म् ग्राम॑कामो॒ ग्राम॑कामः स॒प्तक॑पालꣳ स॒प्तक॑पाल॒म् ग्राम॑काम॒ इन्द्र॒ मिन्द्र॒म् ग्राम॑कामः स॒प्तक॑पालꣳ स॒प्तक॑पाल॒म् ग्राम॑काम॒ इन्द्र᳚म् । \newline
8. स॒प्तक॑पाल॒मिति॑ स॒प्त - क॒पा॒ल॒म् । \newline
9. ग्राम॑काम॒ इन्द्र॒ मिन्द्र॒म् ग्राम॑कामो॒ ग्राम॑काम॒ इन्द्र॑म् च॒ चे न्द्र॒म् ग्राम॑कामो॒ ग्राम॑काम॒ इन्द्र॑म् च । \newline
10. ग्राम॑काम॒ इति॒ ग्राम॑ - का॒मः॒ । \newline
11. इन्द्र॑म् च॒ चे न्द्र॒ मिन्द्र॑म् चै॒वैव चे न्द्र॒ मिन्द्र॑म् चै॒व । \newline
12. चै॒वैव च॑ चै॒व म॒रुतो॑ म॒रुत॑ ए॒व च॑ चै॒व म॒रुतः॑ । \newline
13. ए॒व म॒रुतो॑ म॒रुत॑ ए॒वैव म॒रुत॑श्च च म॒रुत॑ ए॒वैव म॒रुत॑श्च । \newline
14. म॒रुत॑श्च च म॒रुतो॑ म॒रुत॑श्च॒ स्वेन॒ स्वेन॑ च म॒रुतो॑ म॒रुत॑श्च॒ स्वेन॑ । \newline
15. च॒ स्वेन॒ स्वेन॑ च च॒ स्वेन॑ भाग॒धेये॑न भाग॒धेये॑न॒ स्वेन॑ च च॒ स्वेन॑ भाग॒धेये॑न । \newline
16. स्वेन॑ भाग॒धेये॑न भाग॒धेये॑न॒ स्वेन॒ स्वेन॑ भाग॒धेये॒नोपोप॑ भाग॒धेये॑न॒ स्वेन॒ स्वेन॑ भाग॒धेये॒नोप॑ । \newline
17. भा॒ग॒धेये॒नोपोप॑ भाग॒धेये॑न भाग॒धेये॒नोप॑ धावति धाव॒ त्युप॑ भाग॒धेये॑न भाग॒धेये॒नोप॑ धावति । \newline
18. भा॒ग॒धेये॒नेति॑ भाग - धेये॑न । \newline
19. उप॑ धावति धाव॒ त्युपोप॑ धावति॒ ते ते धा॑व॒ त्युपोप॑ धावति॒ ते । \newline
20. धा॒व॒ति॒ ते ते धा॑वति धावति॒ त ए॒वैव ते धा॑वति धावति॒ त ए॒व । \newline
21. त ए॒वैव ते त ए॒वास्मा॑ अस्मा ए॒व ते त ए॒वास्मै᳚ । \newline
22. ए॒वास्मा॑ अस्मा ए॒वैवास्मै॑ सजा॒तान् थ्स॑जा॒ता न॑स्मा ए॒वैवास्मै॑ सजा॒तान् । \newline
23. अ॒स्मै॒ स॒जा॒तान् थ्स॑जा॒ता न॑स्मा अस्मै सजा॒तान् प्र प्र स॑जा॒ता न॑स्मा अस्मै सजा॒तान् प्र । \newline
24. स॒जा॒तान् प्र प्र स॑जा॒तान् थ्स॑जा॒तान् प्र य॑च्छन्ति यच्छन्ति॒ प्र स॑जा॒तान् थ्स॑जा॒तान् प्र य॑च्छन्ति । \newline
25. स॒जा॒तानिति॑ स - जा॒तान् । \newline
26. प्र य॑च्छन्ति यच्छन्ति॒ प्र प्र य॑च्छन्ति ग्रा॒मी ग्रा॒मी य॑च्छन्ति॒ प्र प्र य॑च्छन्ति ग्रा॒मी । \newline
27. य॒च्छ॒न्ति॒ ग्रा॒मी ग्रा॒मी य॑च्छन्ति यच्छन्ति ग्रा॒म्ये॑वैव ग्रा॒मी य॑च्छन्ति यच्छन्ति ग्रा॒म्ये॑व । \newline
28. ग्रा॒म्ये॑वैव ग्रा॒मी ग्रा॒म्ये॑व भ॑वति भव त्ये॒व ग्रा॒मी ग्रा॒म्ये॑व भ॑वति । \newline
29. ए॒व भ॑वति भव त्ये॒वैव भ॑व त्याहव॒नीय॑ आहव॒नीये॑ भव त्ये॒वैव भ॑व त्याहव॒नीये᳚ । \newline
30. भ॒व॒ त्या॒ह॒व॒नीय॑ आहव॒नीये॑ भवति भव त्याहव॒नीय॑ ऐ॒न्द्र मै॒न्द्र मा॑हव॒नीये॑ भवति भव त्याहव॒नीय॑ ऐ॒न्द्रम् । \newline
31. आ॒ह॒व॒नीय॑ ऐ॒न्द्र मै॒न्द्र मा॑हव॒नीय॑ आहव॒नीय॑ ऐ॒न्द्र मध्यध्यै॒न्द्र मा॑हव॒नीय॑ आहव॒नीय॑ ऐ॒न्द्र मधि॑ । \newline
32. आ॒ह॒व॒नीय॒ इत्या᳚ - ह॒व॒नीये᳚ । \newline
33. ऐ॒न्द्र मध्यध्यै॒न्द्र मै॒न्द्र मधि॑ श्रयति श्रय॒ त्यध्यै॒न्द्र मै॒न्द्र मधि॑ श्रयति । \newline
34. अधि॑ श्रयति श्रय॒ त्यध्यधि॑ श्रयति॒ गार्.ह॑पत्ये॒ गार्.ह॑पत्ये श्रय॒ त्यध्यधि॑ श्रयति॒ गार्.ह॑पत्ये । \newline
35. श्र॒य॒ति॒ गार्.ह॑पत्ये॒ गार्.ह॑पत्ये श्रयति श्रयति॒ गार्.ह॑पत्ये मारु॒तम् मा॑रु॒तम् गार्.ह॑पत्ये श्रयति श्रयति॒ गार्.ह॑पत्ये मारु॒तम् । \newline
36. गार्.ह॑पत्ये मारु॒तम् मा॑रु॒तम् गार्.ह॑पत्ये॒ गार्.ह॑पत्ये मारु॒तम् पा॑पवस्य॒सस्य॑ पापवस्य॒सस्य॑ मारु॒तम् गार्.ह॑पत्ये॒ गार्.ह॑पत्ये मारु॒तम् पा॑पवस्य॒सस्य॑ । \newline
37. गार्.ह॑पत्य॒ इति॒ गार्.ह॑ - प॒त्ये॒ । \newline
38. मा॒रु॒तम् पा॑पवस्य॒सस्य॑ पापवस्य॒सस्य॑ मारु॒तम् मा॑रु॒तम् पा॑पवस्य॒सस्य॒ विधृ॑त्यै॒ विधृ॑त्यै पापवस्य॒सस्य॑ मारु॒तम् मा॑रु॒तम् पा॑पवस्य॒सस्य॒ विधृ॑त्यै । \newline
39. पा॒प॒व॒स्य॒सस्य॒ विधृ॑त्यै॒ विधृ॑त्यै पापवस्य॒सस्य॑ पापवस्य॒सस्य॒ विधृ॑त्यै स॒प्तक॑पालः स॒प्तक॑पालो॒ विधृ॑त्यै पापवस्य॒सस्य॑ पापवस्य॒सस्य॒ विधृ॑त्यै स॒प्तक॑पालः । \newline
40. पा॒प॒व॒स्य॒सस्येति॑ पाप - व॒स्य॒सस्य॑ । \newline
41. विधृ॑त्यै स॒प्तक॑पालः स॒प्तक॑पालो॒ विधृ॑त्यै॒ विधृ॑त्यै स॒प्तक॑पालो मारु॒तो मा॑रु॒तः स॒प्तक॑पालो॒ विधृ॑त्यै॒ विधृ॑त्यै स॒प्तक॑पालो मारु॒तः । \newline
42. विधृ॑त्या॒ इति॒ वि - धृ॒त्यै॒ । \newline
43. स॒प्तक॑पालो मारु॒तो मा॑रु॒तः स॒प्तक॑पालः स॒प्तक॑पालो मारु॒तो भ॑वति भवति मारु॒तः स॒प्तक॑पालः स॒प्तक॑पालो मारु॒तो भ॑वति । \newline
44. स॒प्तक॑पाल॒ इति॑ स॒प्त - क॒पा॒लः॒ । \newline
45. मा॒रु॒तो भ॑वति भवति मारु॒तो मा॑रु॒तो भ॑वति स॒प्तग॑णाः स॒प्तग॑णा भवति मारु॒तो मा॑रु॒तो भ॑वति स॒प्तग॑णाः । \newline
46. भ॒व॒ति॒ स॒प्तग॑णाः स॒प्तग॑णा भवति भवति स॒प्तग॑णा॒ वै वै स॒प्तग॑णा भवति भवति स॒प्तग॑णा॒ वै । \newline
47. स॒प्तग॑णा॒ वै वै स॒प्तग॑णाः स॒प्तग॑णा॒ वै म॒रुतो॑ म॒रुतो॒ वै स॒प्तग॑णाः स॒प्तग॑णा॒ वै म॒रुतः॑ । \newline
48. स॒प्तग॑णा॒ इति॑ स॒प्त - ग॒णाः॒ । \newline
49. वै म॒रुतो॑ म॒रुतो॒ वै वै म॒रुतो॑ गण॒शो ग॑ण॒शो म॒रुतो॒ वै वै म॒रुतो॑ गण॒शः । \newline
50. म॒रुतो॑ गण॒शो ग॑ण॒शो म॒रुतो॑ म॒रुतो॑ गण॒श ए॒वैव ग॑ण॒शो म॒रुतो॑ म॒रुतो॑ गण॒श ए॒व । \newline
51. ग॒ण॒श ए॒वैव ग॑ण॒शो ग॑ण॒श ए॒वास्मा॑ अस्मा ए॒व ग॑ण॒शो ग॑ण॒श ए॒वास्मै᳚ । \newline
52. ग॒ण॒श इति॑ गण - शः । \newline
53. ए॒वास्मा॑ अस्मा ए॒वैवास्मै॑ सजा॒तान् थ्स॑जा॒ता न॑स्मा ए॒वैवास्मै॑ सजा॒तान् । \newline
54. अ॒स्मै॒ स॒जा॒तान् थ्स॑जा॒ता न॑स्मा अस्मै सजा॒ता नवाव॑ सजा॒ता न॑स्मा अस्मै सजा॒ता नव॑ । \newline
55. स॒जा॒ता नवाव॑ सजा॒तान् थ्स॑जा॒ता नव॑ रुन्धे रु॒न्धे ऽव॑ सजा॒तान् थ्स॑जा॒ता नव॑ रुन्धे । \newline
56. स॒जा॒तानिति॑ स - जा॒तान् । \newline
57. अव॑ रुन्धे रु॒न्धे ऽवाव॑ रुन्धे ऽनू॒च्यमा॑ने ऽनू॒च्यमा॑ने रु॒न्धे ऽवाव॑ रुन्धे ऽनू॒च्यमा॑ने । \newline
58. रु॒न्धे॒ ऽनू॒च्यमा॑ने ऽनू॒च्यमा॑ने रुन्धे रुन्धे ऽनू॒च्यमा॑न॒ आ ऽनू॒च्यमा॑ने रुन्धे रुन्धे ऽनू॒च्यमा॑न॒ आ । \newline
59. अ॒नू॒च्यमा॑न॒ आ ऽनू॒च्यमा॑ने ऽनू॒च्यमा॑न॒ आ सा॑दयति सादय॒त्या ऽनू॒च्यमा॑ने ऽनू॒च्यमा॑न॒ आ सा॑दयति । \newline
60. अ॒नू॒च्यमा॑न॒ इत्य॑नु - उ॒च्यमा॑ने । \newline
61. आ सा॑दयति सादय॒त्या सा॑दयति॒ विशं॒ ॅविशꣳ॑ सादय॒त्या सा॑दयति॒ विश᳚म् । \newline
62. सा॒द॒य॒ति॒ विशं॒ ॅविशꣳ॑ सादयति सादयति॒ विश॑ मे॒वैव विशꣳ॑ सादयति सादयति॒ विश॑ मे॒व । \newline
63. विश॑ मे॒वैव विशं॒ ॅविश॑ मे॒वास्मा॑ अस्मा ए॒व विशं॒ ॅविश॑ मे॒वास्मै᳚ । \newline
64. ए॒वास्मा॑ अस्मा ए॒वैवास्मा॒ अनु॑वर्त्मान॒ मनु॑वर्त्मान मस्मा ए॒वैवास्मा॒ अनु॑वर्त्मानम् । \newline
\pagebreak
\markright{ TS 2.2.11.2  \hfill https://www.vedavms.in \hfill}
\addcontentsline{toc}{section}{ TS 2.2.11.2 }
\section*{ TS 2.2.11.2 }

\textbf{TS 2.2.11.2 } \newline
\textbf{Samhita Paata} \newline

-स्मा॒ अनु॑वर्त्मानं करोत्ये॒तामे॒व निर्व॑पे॒द्यः का॒मये॑त क्ष॒त्राय॑ च वि॒शे च॑ स॒मदं॑ दद्ध्या॒मित्यै॒न्द्रस्या॑-व॒द्यन् ब्रू॑या॒दिन्द्रा॒यानु॑ ब्रू॒हीत्या॒श्राव्य॑ ब्रूयान्-म॒रुतो॑ य॒जेति॑ मारु॒तस्या॑व॒द्यन् ब्रू॑यान्म॒रुद्भ्योऽनु॑ ब्रू॒हीत्या॒श्राव्य॑ ब्रूया॒दिन्द्रं॑ ॅय॒जेति॒ स्व ए॒वैभ्यो॑ भाग॒धेये॑ स॒मदं॑ दधाति वितृꣳहा॒णास्ति॑ष्ठन्त्ये॒ तामे॒व - [  ] \newline

\textbf{Pada Paata} \newline

अ॒स्मै॒ । अनु॑वर्त्मान॒मित्यनु॑ - व॒र्त्मा॒न॒म् । क॒रो॒ति॒ । ए॒ताम् । ए॒व । निरिति॑ । व॒पे॒त् । यः । का॒मये॑त । क्ष॒त्राय॑ । च॒ । वि॒शे । च॒ । स॒मद॒मिति॑ स - मद᳚म् । द॒द्ध्या॒म् । इति॑ । ऐ॒न्द्रस्य॑ । अ॒व॒द्यन्नित्य॑व - द्यन्न् । ब्रू॒या॒त् । इन्द्रा॑य । अन्विति॑॑ । ब्रू॒हि॒ । इति॑ । आ॒श्राव्येत्या᳚ - श्राव्य॑ । ब्रू॒या॒त् । म॒रुतः॑ । य॒ज॒ । इति॑ । मा॒रु॒तस्य॑ । अ॒व॒द्यन्नित्य॑व - द्यन्न् । ब्रू॒या॒त् । म॒रुद्भ्य॒ इति॑ म॒रुत् - भ्यः॒ । अन्विति॑ । ब्रू॒हि॒ । इति॑ । आ॒श्राव्येत्या᳚ - श्राव्य॑ । ब्रू॒या॒त् । इन्द्र᳚म् । य॒ज॒ । इति॑ । स्वे । ए॒व । ए॒भ्यः॒ । भा॒ग॒धेय॒ इति॑ भाग - धेये᳚ । स॒मद॒मिति॑ स - मद᳚म् । द॒धा॒ति॒ । वि॒तृꣳ॒॒हा॒णा इति॑ वि-तृꣳ॒॒हा॒णाः । ति॒ष्ठ॒न्ति॒ । ए॒ताम् । ए॒व ।  \newline


\textbf{Krama Paata} \newline

अ॒स्मा॒ अनु॑वर्त्मानम् । अनु॑वर्त्मानम् करोति । अनु॑वर्त्मान॒मित्यनु॑ - व॒र्त्मा॒न॒म् । क॒रो॒त्ये॒ताम् । ए॒तामे॒व । ए॒व निः । निर् व॑पेत् । व॒पे॒द् यः । यः का॒मये॑त । का॒मये॑त क्ष॒त्राय॑ । क्ष॒त्राय॑ च । च॒ वि॒शे । वि॒शे च॑ । च॒ स॒मद᳚म् । स॒मद॑म् दद्ध्याम् । स॒मद॒मिति॑ स - मद᳚म् । द॒द्ध्या॒मिति॑ । इत्यै॒न्द्रस्य॑ । ऐ॒न्द्रस्या॑व॒द्यन्न् । अ॒व॒द्यन्,ब्रू॑यात् । अ॒व॒द्यन्नित्य॑व - द्यन्न् । ब्रू॒या॒दिन्द्रा॑य । इन्द्रा॒यानु॑ । अनु॑ ब्रूहि । ब्रू॒हीति॑ । इत्या॒श्राव्य॑ । आ॒श्राव्य॑ ब्रूयात् । आ॒श्राव्येत्या᳚ - श्राव्य॑ । ब्रू॒या॒न् म॒रुतः॑ । म॒रुतो॑ यज । य॒जेति॑ । इति॑ मारु॒तस्य॑ । मा॒रु॒तस्या॑व॒द्यन्न् । अ॒व॒द्यन् ब्रू॑यात् । अ॒व॒द्यन्नित्य॑व - द्यन्न् । ब्रू॒या॒न् म॒रुद्भ्यः॑ । म॒रुद्भ्यो ऽनु॑ । म॒रुद्भ्य॒ इति॑ म॒रुत् - भ्यः॒ । अनु॑ ब्रूहि । ब्रू॒हीति॑ । इत्या॒श्राव्य॑ । आ॒शाव्य॑ ब्रूयात् । आ॒श्राव्येत्या᳚ - श्राव्य॑ । ब्रू॒या॒दिन्द्र᳚म् । इन्द्रं॑ ॅयज । य॒जेति॑ । इति॒ स्वे । स्व ए॒व । ए॒वैभ्यः॑ । ए॒भ्यो॒ भा॒ग॒धेये᳚ । भा॒ग॒धेये॑ स॒मद᳚म् । भा॒ग॒धेय॒ इति॑ भाग - धेये᳚ । स॒मद॑म् दधाति । स॒मद॒मिति॑ स - मद᳚म् । द॒धा॒ति॒ वि॒तृꣳ॒॒हा॒णाः । वि॒तृꣳ॒॒हा॒णास्ति॑ष्ठन्ति । वि॒तृꣳ॒॒हा॒णा इति॑ वि - तृꣳ॒॒हा॒णाः । ति॒ष्ठ॒न्त्ये॒ताम् । ए॒तामे॒व । ए॒व निः \newline

\textbf{Jatai Paata} \newline

1. अ॒स्मा॒ अनु॑वर्त्मान॒ मनु॑वर्त्मान मस्मा अस्मा॒ अनु॑वर्त्मानम् । \newline
2. अनु॑वर्त्मानम् करोति करो॒ त्यनु॑वर्त्मान॒ मनु॑वर्त्मानम् करोति । \newline
3. अनु॑वर्त्मान॒मित्यनु॑ - व॒र्त्मा॒न॒म् । \newline
4. क॒रो॒ त्ये॒ता मे॒ताम् क॑रोति करो त्ये॒ताम् । \newline
5. ए॒ता मे॒वैवैता मे॒ता मे॒व । \newline
6. ए॒व निर् णि रे॒वैव निः । \newline
7. निर् व॑पेद् वपे॒न् निर् णिर् व॑पेत् । \newline
8. व॒पे॒द् यो यो व॑पेद् वपे॒द् यः । \newline
9. यः का॒मये॑त का॒मये॑त॒ यो यः का॒मये॑त । \newline
10. का॒मये॑त क्ष॒त्राय॑ क्ष॒त्राय॑ का॒मये॑त का॒मये॑त क्ष॒त्राय॑ । \newline
11. क्ष॒त्राय॑ च च क्ष॒त्राय॑ क्ष॒त्राय॑ च । \newline
12. च॒ वि॒शे वि॒शे च॑ च वि॒शे । \newline
13. वि॒शे च॑ च वि॒शे वि॒शे च॑ । \newline
14. च॒ स॒मदꣳ॑ स॒मद॑म् च च स॒मद᳚म् । \newline
15. स॒मद॑म् दद्ध्याम् दद्ध्याꣳ स॒मदꣳ॑ स॒मद॑म् दद्ध्याम् । \newline
16. स॒मद॒मिति॑ स - मद᳚म् । \newline
17. द॒द्ध्या॒ मितीति॑ दद्ध्याम् दद्ध्या॒ मिति॑ । \newline
18. इत्यै॒न्द्र स्यै॒न्द्रस्ये तीत्यै॒न्द्रस्य॑ । \newline
19. ऐ॒न्द्रस्या॑ व॒द्यन् न॑व॒द्यन् नै॒न्द्र स्यै॒न्द्रस्या॑ व॒द्यन्न् । \newline
20. अ॒व॒द्यन् ब्रू॑याद् ब्रूया दव॒द्यन् न॑व॒द्यन् ब्रू॑यात् । \newline
21. अ॒व॒द्यन्नित्य॑व - द्यन्न् । \newline
22. ब्रू॒या॒ दिन्द्रा॒ये न्द्रा॑य ब्रूयाद् ब्रूया॒ दिन्द्रा॑य । \newline
23. इन्द्रा॒या न्वन्विन्द्रा॒ये न्द्रा॒यानु॑ । \newline
24. अनु॑ ब्रूहि ब्रू॒ह्यन्वनु॑ ब्रूहि । \newline
25. ब्रू॒हीतीति॑ ब्रूहि ब्रू॒हीति॑ । \newline
26. इत्या॒श्राव्या॒ श्राव्ये ती त्या॒श्राव्य॑ । \newline
27. आ॒श्राव्य॑ ब्रूयाद् ब्रूया दा॒श्राव्या॒ श्राव्य॑ ब्रूयात् । \newline
28. आ॒श्राव्येत्या᳚ - श्राव्य॑ । \newline
29. ब्रू॒या॒न् म॒रुतो॑ म॒रुतो᳚ ब्रूयाद् ब्रूयान् म॒रुतः॑ । \newline
30. म॒रुतो॑ यज यज म॒रुतो॑ म॒रुतो॑ यज । \newline
31. य॒जे तीति॑ यज य॒जे ति॑ । \newline
32. इति॑ मारु॒तस्य॑ मारु॒तस्ये तीति॑ मारु॒तस्य॑ । \newline
33. मा॒रु॒तस्या॑ व॒द्यन् न॑व॒द्यन् मा॑रु॒तस्य॑ मारु॒तस्या॑ व॒द्यन्न् । \newline
34. अ॒व॒द्यन् ब्रू॑याद् ब्रूया दव॒द्यन् न॑व॒द्यन् ब्रू॑यात् । \newline
35. अ॒व॒द्यन्नित्य॑व - द्यन्न् । \newline
36. ब्रू॒या॒न् म॒रुद्भ्यो॑ म॒रुद्भ्यो᳚ ब्रूयाद् ब्रूयान् म॒रुद्भ्यः॑ । \newline
37. म॒रुद्भ्यो ऽन्वनु॑ म॒रुद्भ्यो॑ म॒रुद्भ्यो ऽनु॑ । \newline
38. म॒रुद्भ्य॒ इति॑ म॒रुत् - भ्यः॒ । \newline
39. अनु॑ ब्रूहि ब्रू॒ह्यन्वनु॑ ब्रूहि । \newline
40. ब्रू॒हीतीति॑ ब्रूहि ब्रू॒हीति॑ । \newline
41. इत्या॒श्राव्या॒ श्राव्ये तीत्या॒श्राव्य॑ । \newline
42. आ॒श्राव्य॑ ब्रूयाद् ब्रूया दा॒श्राव्या॒ श्राव्य॑ ब्रूयात् । \newline
43. आ॒श्राव्येत्या᳚ - श्राव्य॑ । \newline
44. ब्रू॒या॒ दिन्द्र॒ मिन्द्र॑म् ब्रूयाद् ब्रूया॒ दिन्द्र᳚म् । \newline
45. इन्द्रं॑ ॅयज य॒जे न्द्र॒ मिन्द्रं॑ ॅयज । \newline
46. य॒जे तीति॑ यज य॒जे ति॑ । \newline
47. इति॒ स्वे स्व इतीति॒ स्वे । \newline
48. स्व ए॒वैव स्वे स्व ए॒व । \newline
49. ए॒वैभ्य॑ एभ्य ए॒वैवैभ्यः॑ । \newline
50. ए॒भ्यो॒ भा॒ग॒धेये॑ भाग॒धेय॑ एभ्य एभ्यो भाग॒धेये᳚ । \newline
51. भा॒ग॒धेये॑ स॒मदꣳ॑ स॒मद॑म् भाग॒धेये॑ भाग॒धेये॑ स॒मद᳚म् । \newline
52. भा॒ग॒धेय॒ इति॑ भाग - धेये᳚ । \newline
53. स॒मद॑म् दधाति दधाति स॒मदꣳ॑ स॒मद॑म् दधाति । \newline
54. स॒मद॒मिति॑ स - मद᳚म् । \newline
55. द॒धा॒ति॒ वि॒तृꣳ॒॒हा॒णा वि॑तृꣳहा॒णा द॑धाति दधाति वितृꣳहा॒णाः । \newline
56. वि॒तृꣳ॒॒हा॒णा स्ति॑ष्ठन्ति तिष्ठन्ति वितृꣳहा॒णा वि॑तृꣳहा॒णा स्ति॑ष्ठन्ति । \newline
57. वि॒तृꣳ॒॒हा॒णा इति॑ वि - तृꣳ॒॒हा॒णाः । \newline
58. ति॒ष्ठ॒ न्त्ये॒ता मे॒ताम् ति॑ष्ठन्ति तिष्ठ न्त्ये॒ताम् । \newline
59. ए॒ता मे॒वैवैता मे॒ता मे॒व । \newline
60. ए॒व निर् णि रे॒वैव निः । \newline

\textbf{Ghana Paata } \newline

1. अ॒स्मा॒ अनु॑वर्त्मान॒ मनु॑वर्त्मान मस्मा अस्मा॒ अनु॑वर्त्मानम् करोति करो॒ त्यनु॑वर्त्मान मस्मा अस्मा॒ अनु॑वर्त्मानम् करोति । \newline
2. अनु॑वर्त्मानम् करोति करो॒ त्यनु॑वर्त्मान॒ मनु॑वर्त्मानम् करो त्ये॒ता मे॒ताम् क॑रो॒ त्यनु॑वर्त्मान॒ मनु॑वर्त्मानम् करो त्ये॒ताम् । \newline
3. अनु॑वर्त्मान॒मित्यनु॑ - व॒र्त्मा॒न॒म् । \newline
4. क॒रो॒ त्ये॒ता मे॒ताम् क॑रोति करो त्ये॒ता मे॒वैवैताम् क॑रोति करो त्ये॒ता मे॒व । \newline
5. ए॒ता मे॒वैवैता मे॒ता मे॒व निर् णिरे॒वैता मे॒ता मे॒व निः । \newline
6. ए॒व निर् णिरे॒वैव निर् व॑पेद् वपे॒न् निरे॒वैव निर् व॑पेत् । \newline
7. निर् व॑पेद् वपे॒न् निर् णिर् व॑पे॒द् यो यो व॑पे॒न् निर् णिर् व॑पे॒द् यः । \newline
8. व॒पे॒द् यो यो व॑पेद् वपे॒द् यः का॒मये॑त का॒मये॑त॒ यो व॑पेद् वपे॒द् यः का॒मये॑त । \newline
9. यः का॒मये॑त का॒मये॑त॒ यो यः का॒मये॑त क्ष॒त्राय॑ क्ष॒त्राय॑ का॒मये॑त॒ यो यः का॒मये॑त क्ष॒त्राय॑ । \newline
10. का॒मये॑त क्ष॒त्राय॑ क्ष॒त्राय॑ का॒मये॑त का॒मये॑त क्ष॒त्राय॑ च च क्ष॒त्राय॑ का॒मये॑त का॒मये॑त क्ष॒त्राय॑ च । \newline
11. क्ष॒त्राय॑ च च क्ष॒त्राय॑ क्ष॒त्राय॑ च वि॒शे वि॒शे च॑ क्ष॒त्राय॑ क्ष॒त्राय॑ च वि॒शे । \newline
12. च॒ वि॒शे वि॒शे च॑ च वि॒शे च॑ च वि॒शे च॑ च वि॒शे च॑ । \newline
13. वि॒शे च॑ च वि॒शे वि॒शे च॑ स॒मदꣳ॑ स॒मद॑म् च वि॒शे वि॒शे च॑ स॒मद᳚म् । \newline
14. च॒ स॒मदꣳ॑ स॒मद॑म् च च स॒मद॑म् दद्ध्याम् दद्ध्याꣳ स॒मद॑म् च च स॒मद॑म् दद्ध्याम् । \newline
15. स॒मद॑म् दद्ध्याम् दद्ध्याꣳ स॒मदꣳ॑ स॒मद॑म् दद्ध्या॒ मितीति॑ दद्ध्याꣳ स॒मदꣳ॑ स॒मद॑म् दद्ध्या॒ मिति॑ । \newline
16. स॒मद॒मिति॑ स - मद᳚म् । \newline
17. द॒द्ध्या॒ मितीति॑ दद्ध्याम् दद्ध्या॒ मित्यै॒न्द्र स्यै॒न्द्रस्ये ति॑ दद्ध्याम् दद्ध्या॒ मित्यै॒न्द्रस्य॑ । \newline
18. इत्यै॒न्द्र स्यै॒न्द्रस्ये तीत्यै॒न्द्रस्या॑ व॒द्यन् न॑व॒द्यन् नै॒न्द्रस्ये तीत्यै॒न्द्रस्या॑ व॒द्यन्न् । \newline
19. ऐ॒न्द्रस्या॑ व॒द्यन् न॑व॒द्यन् नै॒न्द्र स्यै॒न्द्रस्या॑ व॒द्यन् ब्रू॑याद् ब्रूया दव॒द्यन् नै॒न्द्र स्यै॒न्द्रस्या॑ व॒द्यन् ब्रू॑यात् । \newline
20. अ॒व॒द्यन् ब्रू॑याद् ब्रूया दव॒द्यन् न॑व॒द्यन् ब्रू॑या॒ दिन्द्रा॒ये न्द्रा॑य ब्रूया दव॒द्यन् न॑व॒द्यन् ब्रू॑या॒ दिन्द्रा॑य । \newline
21. अ॒व॒द्यन्नित्य॑व - द्यन्न् । \newline
22. ब्रू॒या॒ दिन्द्रा॒ये न्द्रा॑य ब्रूयाद् ब्रूया॒ दिन्द्रा॒यान्व न्विन्द्रा॑य ब्रूयाद् ब्रूया॒ दिन्द्रा॒यानु॑ । \newline
23. इन्द्रा॒यान्व न्विन्द्रा॒ये न्द्रा॒यानु॑ ब्रूहि ब्रू॒ह्यन्विन्द्रा॒ये न्द्रा॒यानु॑ ब्रूहि । \newline
24. अनु॑ ब्रूहि ब्रू॒ह्यन्वनु॑ ब्रू॒हीतीति॑ ब्रू॒ह्यन्वनु॑ ब्रू॒हीति॑ । \newline
25. ब्रू॒हीतीति॑ ब्रूहि ब्रू॒ही त्या॒श्राव्या॒ श्राव्ये ति॑ ब्रूहि ब्रू॒ही त्या॒श्राव्य॑ । \newline
26. इत्या॒श्राव्या॒ श्राव्ये ती त्या॒श्राव्य॑ ब्रूयाद् ब्रूया दा॒श्राव्ये ती त्या॒श्राव्य॑ ब्रूयात् । \newline
27. आ॒श्राव्य॑ ब्रूयाद् ब्रूया दा॒श्राव्या॒ श्राव्य॑ ब्रूयान् म॒रुतो॑ म॒रुतो᳚ ब्रूया दा॒श्राव्या॒ श्राव्य॑ ब्रूयान् म॒रुतः॑ । \newline
28. आ॒श्राव्येत्या᳚ - श्राव्य॑ । \newline
29. ब्रू॒या॒न् म॒रुतो॑ म॒रुतो᳚ ब्रूयाद् ब्रूयान् म॒रुतो॑ यज यज म॒रुतो᳚ ब्रूयाद् ब्रूयान् म॒रुतो॑ यज । \newline
30. म॒रुतो॑ यज यज म॒रुतो॑ म॒रुतो॑ य॒जे तीति॑ यज म॒रुतो॑ म॒रुतो॑ य॒जे ति॑ । \newline
31. य॒जे तीति॑ यज य॒जे ति॑ मारु॒तस्य॑ मारु॒तस्ये ति॑ यज य॒जे ति॑ मारु॒तस्य॑ । \newline
32. इति॑ मारु॒तस्य॑ मारु॒तस्ये तीति॑ मारु॒तस्या॑ व॒द्यन् न॑व॒द्यन् मा॑रु॒तस्ये तीति॑ मारु॒तस्या॑ व॒द्यन्न् । \newline
33. मा॒रु॒तस्या॑ व॒द्यन् न॑व॒द्यन् मा॑रु॒तस्य॑ मारु॒तस्या॑ व॒द्यन् ब्रू॑याद् ब्रूया दव॒द्यन् मा॑रु॒तस्य॑ मारु॒तस्या॑ व॒द्यन् ब्रू॑यात् । \newline
34. अ॒व॒द्यन् ब्रू॑याद् ब्रूया दव॒द्यन् न॑व॒द्यन् ब्रू॑यान् म॒रुद्भ्यो॑ म॒रुद्भ्यो᳚ ब्रूया दव॒द्यन् न॑व॒द्यन् ब्रू॑यान् म॒रुद्भ्यः॑ । \newline
35. अ॒व॒द्यन्नित्य॑व - द्यन्न् । \newline
36. ब्रू॒या॒न् म॒रुद्भ्यो॑ म॒रुद्भ्यो᳚ ब्रूयाद् ब्रूयान् म॒रुद्भ्यो ऽन्वनु॑ म॒रुद्भ्यो᳚ ब्रूयाद् ब्रूयान् म॒रुद्भ्यो ऽनु॑ । \newline
37. म॒रुद्भ्यो ऽन्वनु॑ म॒रुद्भ्यो॑ म॒रुद्भ्यो ऽनु॑ ब्रूहि ब्रू॒ह्यनु॑ म॒रुद्भ्यो॑ म॒रुद्भ्यो ऽनु॑ ब्रूहि । \newline
38. म॒रुद्भ्य॒ इति॑ म॒रुत् - भ्यः॒ । \newline
39. अनु॑ ब्रूहि ब्रू॒ह्यन्वनु॑ ब्रू॒हीतीति॑ ब्रू॒ह्यन्वनु॑ ब्रू॒हीति॑ । \newline
40. ब्रू॒हीतीति॑ ब्रूहि ब्रू॒ही त्या॒श्राव्या॒ श्राव्ये ति॑ ब्रूहि ब्रू॒ही त्या॒श्राव्य॑ । \newline
41. इत्या॒श्राव्या॒ श्राव्ये ती त्या॒श्राव्य॑ ब्रूयाद् ब्रूया दा॒श्राव्ये ती त्या॒श्राव्य॑ ब्रूयात् । \newline
42. आ॒श्राव्य॑ ब्रूयाद् ब्रूया दा॒श्राव्या॒ श्राव्य॑ ब्रूया॒ दिन्द्र॒ मिन्द्र॑म् ब्रूया दा॒श्राव्या॒ श्राव्य॑ ब्रूया॒ दिन्द्र᳚म् । \newline
43. आ॒श्राव्येत्या᳚ - श्राव्य॑ । \newline
44. ब्रू॒या॒ दिन्द्र॒ मिन्द्र॑म् ब्रूयाद् ब्रूया॒ दिन्द्रं॑ ॅयज य॒जे न्द्र॑म् ब्रूयाद् ब्रूया॒ दिन्द्रं॑ ॅयज । \newline
45. इन्द्रं॑ ॅयज य॒जे न्द्र॒ मिन्द्रं॑ ॅय॒जे तीति॑ य॒जे न्द्र॒ मिन्द्रं॑ ॅय॒जे ति॑ । \newline
46. य॒जे तीति॑ यज य॒जे ति॒ स्वे स्व इति॑ यज य॒जे ति॒ स्वे । \newline
47. इति॒ स्वे स्व इतीति॒ स्व ए॒वैव स्व इतीति॒ स्व ए॒व । \newline
48. स्व ए॒वैव स्वे स्व ए॒वैभ्य॑ एभ्य ए॒व स्वे स्व ए॒वैभ्यः॑ । \newline
49. ए॒वैभ्य॑ एभ्य ए॒वैवैभ्यो॑ भाग॒धेये॑ भाग॒धेय॑ एभ्य ए॒वैवैभ्यो॑ भाग॒धेये᳚ । \newline
50. ए॒भ्यो॒ भा॒ग॒धेये॑ भाग॒धेय॑ एभ्य एभ्यो भाग॒धेये॑ स॒मदꣳ॑ स॒मद॑म् भाग॒धेय॑ एभ्य एभ्यो भाग॒धेये॑ स॒मद᳚म् । \newline
51. भा॒ग॒धेये॑ स॒मदꣳ॑ स॒मद॑म् भाग॒धेये॑ भाग॒धेये॑ स॒मद॑म् दधाति दधाति स॒मद॑म् भाग॒धेये॑ भाग॒धेये॑ स॒मद॑म् दधाति । \newline
52. भा॒ग॒धेय॒ इति॑ भाग - धेये᳚ । \newline
53. स॒मद॑म् दधाति दधाति स॒मदꣳ॑ स॒मद॑म् दधाति वितृꣳहा॒णा वि॑तृꣳहा॒णा द॑धाति स॒मदꣳ॑ स॒मद॑म् दधाति वितृꣳहा॒णाः । \newline
54. स॒मद॒मिति॑ स - मद᳚म् । \newline
55. द॒धा॒ति॒ वि॒तृꣳ॒॒हा॒णा वि॑तृꣳहा॒णा द॑धाति दधाति वितृꣳहा॒णा स्ति॑ष्ठन्ति तिष्ठन्ति वितृꣳहा॒णा द॑धाति दधाति वितृꣳहा॒णा स्ति॑ष्ठन्ति । \newline
56. वि॒तृꣳ॒॒हा॒णा स्ति॑ष्ठन्ति तिष्ठन्ति वितृꣳहा॒णा वि॑तृꣳहा॒णा स्ति॑ष्ठ न्त्ये॒ता मे॒ताम् ति॑ष्ठन्ति वितृꣳहा॒णा वि॑तृꣳहा॒णा स्ति॑ष्ठ न्त्ये॒ताम् । \newline
57. वि॒तृꣳ॒॒हा॒णा इति॑ वि - तृꣳ॒॒हा॒णाः । \newline
58. ति॒ष्ठ॒ न्त्ये॒ता मे॒ताम् ति॑ष्ठन्ति तिष्ठ न्त्ये॒ता मे॒वैवैताम् ति॑ष्ठन्ति तिष्ठ न्त्ये॒ता मे॒व । \newline
59. ए॒ता मे॒वैवैता मे॒ता मे॒व निर् णिरे॒वैता मे॒ता मे॒व निः । \newline
60. ए॒व निर् णिरे॒वैव निर् व॑पेद् वपे॒न् निरे॒वैव निर् व॑पेत् । \newline
\pagebreak
\markright{ TS 2.2.11.3  \hfill https://www.vedavms.in \hfill}
\addcontentsline{toc}{section}{ TS 2.2.11.3 }
\section*{ TS 2.2.11.3 }

\textbf{TS 2.2.11.3 } \newline
\textbf{Samhita Paata} \newline

निर्व॑पे॒द्यः का॒मये॑त॒ कल्पे॑र॒न्निति॑ यथादेव॒तम॑व॒दाय॑ यथा देव॒तं ॅय॑जेद्-भाग॒धेये॑नै॒वैनान्॑ यथाय॒थं क॑ल्पयति॒ कल्प॑न्त ए॒वैन्द्र-मेका॑दशकपालं॒ निर्व॑पेद्-वैश्वदे॒वं द्वाद॑शकपालं॒ ग्राम॑काम॒ इन्द्रं॑ चै॒व विश्वाꣳ॑श्च दे॒वान्थ्-स्वेन॑ भाग॒धेये॒नोप॑ धावति॒ त ए॒वास्मै॑ सजा॒तान् प्रय॑च्छन्ति ग्रा॒म्ये॑व भ॑वत्यै॒न्द्रस्या॑व॒दाय॑ वैश्वदे॒वस्याव॑ द्ये॒दथै॒न्द्रस्यो॒- [  ] \newline

\textbf{Pada Paata} \newline

निरिति॑ । व॒पे॒त् । यः । का॒मये॑त । कल्पे॑रन्न् । इति॑ । य॒था॒द॒व॒तमिति॑ यथा - दे॒व॒तम् । अ॒व॒दायेत्य॑व - दाय॑ । य॒था॒दे॒व॒तमिति॑ यथा - दे॒व॒तम् । य॒जे॒त् । भा॒ग॒धेये॒नेति॑ भाग-धेये॑न । ए॒व । ए॒ना॒न् । य॒था॒य॒थमिति॑ यथा - य॒थम् । क॒ल्प॒य॒ति॒ । कल्प॑न्ते । ए॒व । ऐ॒न्द्रम् । एका॑दशकपाल॒मित्येका॑दश - क॒पा॒ल॒म् । निरिति॑ । व॒पे॒त् । वै॒श्व॒दे॒वमिति॑ वैश्व - दे॒वम् । द्वाद॑शकपाल॒मिति॒ द्वाद॑श - क॒पा॒ल॒म् । ग्राम॑काम॒ इति॒ ग्राम॑ - का॒मः॒ । इन्द्र᳚म् । च॒ । ए॒व । विश्वान्॑ । च॒ । दे॒वान् । स्वेन॑ । भा॒ग॒धेये॒नेति॑ भाग - धेये॑न । उपेति॑ । धा॒व॒ति॒ । ते । ए॒व । अ॒स्मै॒ । स॒जा॒तानिति॑ स - जा॒तान् । प्रेति॑ । य॒च्छ॒न्ति॒ । ग्रा॒मी । ए॒व । भ॒व॒ति॒ । ऐ॒न्द्रस्य॑ । अ॒व॒दायेत्य॑व - दाय॑ । वै॒श्व॒दे॒वस्येति॑ वैश्व-दे॒वस्य॑ । अवेति॑ । द्ये॒त् । अथ॑ । ऐ॒न्द्रस्यः॑ ।  \newline


\textbf{Krama Paata} \newline

निर् व॑पेत् । व॒पे॒द् यः । यः का॒मये॑त । का॒मये॑त॒ कल्पे॑रन्न् । कल्पे॑र॒न्निति॑ । इति॑ यथादेव॒तम् । य॒था॒दे॒व॒तम॑व॒दाय॑ । य॒था॒दे॒व॒तमिति॑ यथा - दे॒व॒तम् । अ॒व॒दाय॑ यथादेव॒तम् । अ॒व॒दायेत्य॑व - दाय॑ । य॒था॒दे॒व॒तं ॅय॑जेत् । य॒था॒दे॒व॒तमिति॑ यथा - दे॒व॒तम् । य॒जे॒द् भा॒ग॒धेये॑न । भा॒ग॒धेये॑नै॒व । भा॒ग॒धेये॒नेति॑ भाग - धेये॑न । ए॒वैनान्॑ । ए॒ना॒न्॒. य॒था॒य॒थम् । य॒था॒य॒थम् क॑ल्पयति । य॒था॒य॒थमिति॑ यथा - य॒थम् । क॒ल्प॒य॒ति॒ कल्प॑न्ते । कल्प॑न्त ए॒व । ए॒वैन्द्रम् । ऐ॒न्द्रमेका॑दशकपालम् । एका॑दशकपाल॒म् निः । एका॑दशकपाल॒मित्येका॑दश - क॒पा॒ल॒म् । निर् व॑पेत् । व॒पे॒द् वै॒श्व॒दे॒वम् । वै॒श्व॒दे॒वम् द्वाद॑शकपालम् । वै॒श्व॒दे॒वमिति॑ वैश्व - दे॒वम् । द्वाद॑शकपाल॒म् ग्राम॑कामः । द्वाद॑शकपाल॒मिति॒ द्वाद॑श - क॒पा॒ल॒म् । ग्राम॑काम॒ इन्द्र᳚म् । ग्राम॑काम॒ इति॒ ग्राम॑ - का॒मः॒ । इन्द्र॑म् च । चै॒व । ए॒व विश्वान्॑ । विश्वाꣳ॑श्च । च॒ दे॒वान् । दे॒वान्थ् स्वेन॑ । स्वेन॑ भाग॒धेये॑न । भा॒ग॒धेये॒नोप॑ । भा॒ग॒धेये॒नेति॑ भाग - धेये॑न । उप॑ धावति । धा॒व॒ति॒ ते । त ए॒व । ए॒वास्मै᳚ । अ॒स्मे॒ स॒जा॒तान् । स॒जा॒तान् प्र । स॒जा॒तानिति॑ स - जा॒तान् । प्र य॑च्छन्ति । य॒च्छ॒न्ति॒ ग्रा॒मी । ग्रा॒म्ये॑व । ए॒व भ॑वति । भ॒व॒तै॒न्द्रस्य॑ । ऐ॒न्द्रस्या॑व॒दाय॑ । अ॒व॒दाय॑ वैश्वदे॒वस्य॑ । अ॒व॒दायेत्य॑व - दाय॑ । वै॒श्व॒दे॒वस्याव॑ । वै॒श्व॒दे॒वस्येति॑ वैश्व - दे॒वस्य॑ । अव॑ द्येत् । द्ये॒दथ॑ । अथै॒न्द्रस्य॑ । ऐ॒न्द्रस्यो॒परि॑ष्टात् \newline

\textbf{Jatai Paata} \newline

1. निर् व॑पेद् वपे॒न् निर् णिर् व॑पेत् । \newline
2. व॒पे॒द् यो यो व॑पेद् वपे॒द् यः । \newline
3. यः का॒मये॑त का॒मये॑त॒ यो यः का॒मये॑त । \newline
4. का॒मये॑त॒ कल्पे॑र॒न् कल्पे॑रन् का॒मये॑त का॒मये॑त॒ कल्पे॑रन्न् । \newline
5. कल्पे॑र॒न् नितीति॒ कल्पे॑र॒न् कल्पे॑र॒न् निति॑ । \newline
6. इति॑ यथादेव॒तं ॅय॑थादेव॒त मितीति॑ यथादेव॒तम् । \newline
7. य॒था॒दे॒व॒त म॑व॒दाया॑ व॒दाय॑ यथादेव॒तं ॅय॑थादेव॒त म॑व॒दाय॑ । \newline
8. य॒था॒दे॒व॒तमिति॑ यथा - दे॒व॒तम् । \newline
9. अ॒व॒दाय॑ यथादेव॒तं ॅय॑थादेव॒त म॑व॒दाया॑ व॒दाय॑ यथादेव॒तम् । \newline
10. अ॒व॒दायेत्य॑व - दाय॑ । \newline
11. य॒था॒दे॒व॒तं ॅय॑जेद् यजेद् यथादेव॒तं ॅय॑थादेव॒तं ॅय॑जेत् । \newline
12. य॒था॒दे॒व॒तमिति॑ यथा - दे॒व॒तम् । \newline
13. य॒जे॒द् भा॒ग॒धेये॑न भाग॒धेये॑न यजेद् यजेद् भाग॒धेये॑न । \newline
14. भा॒ग॒धेये॑ नै॒वैव भा॑ग॒धेये॑न भाग॒धेये॑ नै॒व । \newline
15. भा॒ग॒धेये॒नेति॑ भाग - धेये॑न । \newline
16. ए॒वैना॑ नेना ने॒वैवैनान्॑ । \newline
17. ए॒ना॒न्॒. य॒था॒य॒थं ॅय॑थाय॒थ मे॑ना नेनान्. यथाय॒थम् । \newline
18. य॒था॒य॒थम् क॑ल्पयति कल्पयति यथाय॒थं ॅय॑थाय॒थम् क॑ल्पयति । \newline
19. य॒था॒य॒थमिति॑ यथा - य॒थम् । \newline
20. क॒ल्प॒य॒ति॒ कल्प॑न्ते॒ कल्प॑न्ते कल्पयति कल्पयति॒ कल्प॑न्ते । \newline
21. कल्प॑न्त ए॒वैव कल्प॑न्ते॒ कल्प॑न्त ए॒व । \newline
22. ए॒वैन्द्र मै॒न्द्र मे॒वैवैन्द्रम् । \newline
23. ऐ॒न्द्र मेका॑दशकपाल॒ मेका॑दशकपाल मै॒न्द्र मै॒न्द्र मेका॑दशकपालम् । \newline
24. एका॑दशकपाल॒म् निर् णिरेका॑दशकपाल॒ मेका॑दशकपाल॒म् निः । \newline
25. एका॑दशकपाल॒मित्येका॑दश - क॒पा॒ल॒म् । \newline
26. निर् व॑पेद् वपे॒न् निर् णिर् व॑पेत् । \newline
27. व॒पे॒द् वै॒श्व॒दे॒वं ॅवै᳚श्वदे॒वं ॅव॑पेद् वपेद् वैश्वदे॒वम् । \newline
28. वै॒श्व॒दे॒वम् द्वाद॑शकपाल॒म् द्वाद॑शकपालं ॅवैश्वदे॒वं ॅवै᳚श्वदे॒वम् द्वाद॑शकपालम् । \newline
29. वै॒श्व॒दे॒वमिति॑ वैश्व - दे॒वम् । \newline
30. द्वाद॑शकपाल॒म् ग्राम॑कामो॒ ग्राम॑कामो॒ द्वाद॑शकपाल॒म् द्वाद॑शकपाल॒म् ग्राम॑कामः । \newline
31. द्वाद॑शकपाल॒मिति॒ द्वाद॑श - क॒पा॒ल॒म् । \newline
32. ग्राम॑काम॒ इन्द्र॒ मिन्द्र॒म् ग्राम॑कामो॒ ग्राम॑काम॒ इन्द्र᳚म् । \newline
33. ग्राम॑काम॒ इति॒ ग्राम॑ - का॒मः॒ । \newline
34. इन्द्र॑म् च॒ चे न्द्र॒ मिन्द्र॑म् च । \newline
35. चै॒वैव च॑ चै॒व । \newline
36. ए॒व विश्वा॒न्॒. विश्वा॑ ने॒वैव विश्वान्॑ । \newline
37. विश्वाꣳ॑श्च च॒ विश्वा॒न्॒. विश्वाꣳ॑श्च । \newline
38. च॒ दे॒वान् दे॒वाꣳश्च॑ च दे॒वान् । \newline
39. दे॒वान् थ्स्वेन॒ स्वेन॑ दे॒वान् दे॒वान् थ्स्वेन॑ । \newline
40. स्वेन॑ भाग॒धेये॑न भाग॒धेये॑न॒ स्वेन॒ स्वेन॑ भाग॒धेये॑न । \newline
41. भा॒ग॒धेये॒नोपोप॑ भाग॒धेये॑न भाग॒धेये॒नोप॑ । \newline
42. भा॒ग॒धेये॒नेति॑ भाग - धेये॑न । \newline
43. उप॑ धावति धाव॒ त्युपोप॑ धावति । \newline
44. धा॒व॒ति॒ ते ते धा॑वति धावति॒ ते । \newline
45. त ए॒वैव ते त ए॒व । \newline
46. ए॒वास्मा॑ अस्मा ए॒वैवास्मै᳚ । \newline
47. अ॒स्मै॒ स॒जा॒तान् थ्स॑जा॒ता न॑स्मा अस्मै सजा॒तान् । \newline
48. स॒जा॒तान् प्र प्र स॑जा॒तान् थ्स॑जा॒तान् प्र । \newline
49. स॒जा॒तानिति॑ स - जा॒तान् । \newline
50. प्र य॑च्छन्ति यच्छन्ति॒ प्र प्र य॑च्छन्ति । \newline
51. य॒च्छ॒न्ति॒ ग्रा॒मी ग्रा॒मी य॑च्छन्ति यच्छन्ति ग्रा॒मी । \newline
52. ग्रा॒म्ये॑वैव ग्रा॒मी ग्रा॒म्ये॑व । \newline
53. ए॒व भ॑वति भव त्ये॒वैव भ॑वति । \newline
54. भ॒व॒ त्यै॒न्द्र स्यै॒न्द्रस्य॑ भवति भव त्यै॒न्द्रस्य॑ । \newline
55. ऐ॒न्द्र स्या॑व॒दाया॑ व॒दा यै॒न्द्र स्यै॒न्द्रस्या॑ व॒दाय॑ । \newline
56. अ॒व॒दाय॑ वैश्वदे॒वस्य॑ वैश्वदे॒वस्या॑ व॒दाया॑ व॒दाय॑ वैश्वदे॒वस्य॑ । \newline
57. अ॒व॒दायेत्य॑व - दाय॑ । \newline
58. वै॒श्व॒दे॒वस्यावाव॑ वैश्वदे॒वस्य॑ वैश्वदे॒वस्याव॑ । \newline
59. वै॒श्व॒दे॒वस्येति॑ वैश्व - दे॒वस्य॑ । \newline
60. अव॑ द्येद् द्ये॒दवाव॑ द्येत् । \newline
61. द्ये॒दथाथ॑ द्येद् द्ये॒दथ॑ । \newline
62. अथै॒ न्द्र स्यै॒न्द्रस्या थाथै॒न्द्रस्य॑ । \newline
63. ऐ॒न्द्र स्यो॒परि॑ष्टा दु॒परि॑ष्टा दै॒न्द्र स्यै॒न्द्र स्यो॒परि॑ष्टात् । \newline

\textbf{Ghana Paata } \newline

1. निर् व॑पेद् वपे॒न् निर् णिर् व॑पे॒द् यो यो व॑पे॒न् निर् णिर् व॑पे॒द् यः । \newline
2. व॒पे॒द् यो यो व॑पेद् वपे॒द् यः का॒मये॑त का॒मये॑त॒ यो व॑पेद् वपे॒द् यः का॒मये॑त । \newline
3. यः का॒मये॑त का॒मये॑त॒ यो यः का॒मये॑त॒ कल्पे॑र॒न् कल्पे॑रन् का॒मये॑त॒ यो यः का॒मये॑त॒ कल्पे॑रन्न् । \newline
4. का॒मये॑त॒ कल्पे॑र॒न् कल्पे॑रन् का॒मये॑त का॒मये॑त॒ कल्पे॑र॒न् नितीति॒ कल्पे॑रन् का॒मये॑त का॒मये॑त॒ कल्पे॑र॒न् निति॑ । \newline
5. कल्पे॑र॒न् नितीति॒ कल्पे॑र॒न् कल्पे॑र॒न् निति॑ यथादेव॒तं ॅय॑थादेव॒त मिति॒ कल्पे॑र॒न् कल्पे॑र॒न् निति॑ यथादेव॒तम् । \newline
6. इति॑ यथादेव॒तं ॅय॑थादेव॒त मितीति॑ यथादेव॒त म॑व॒दाया॑ व॒दाय॑ यथादेव॒त मितीति॑ यथादेव॒त म॑व॒दाय॑ । \newline
7. य॒था॒दे॒व॒त म॑व॒दाया॑ व॒दाय॑ यथादेव॒तं ॅय॑थादेव॒त म॑व॒दाय॑ यथादेव॒तं ॅय॑थादेव॒त म॑व॒दाय॑ यथादेव॒तं ॅय॑थादेव॒त म॑व॒दाय॑ यथादेव॒तम् । \newline
8. य॒था॒दे॒व॒तमिति॑ यथा - दे॒व॒तम् । \newline
9. अ॒व॒दाय॑ यथादेव॒तं ॅय॑थादेव॒त म॑व॒दाया॑ व॒दाय॑ यथादेव॒तं ॅय॑जेद् यजेद् यथादेव॒त म॑व॒दाया॑ व॒दाय॑ यथादेव॒तं ॅय॑जेत् । \newline
10. अ॒व॒दायेत्य॑व - दाय॑ । \newline
11. य॒था॒दे॒व॒तं ॅय॑जेद् यजेद् यथादेव॒तं ॅय॑थादेव॒तं ॅय॑जेद् भाग॒धेये॑न भाग॒धेये॑न यजेद् यथादेव॒तं ॅय॑थादेव॒तं ॅय॑जेद् भाग॒धेये॑न । \newline
12. य॒था॒दे॒व॒तमिति॑ यथा - दे॒व॒तम् । \newline
13. य॒जे॒द् भा॒ग॒धेये॑न भाग॒धेये॑न यजेद् यजेद् भाग॒धेये॑नै॒वैव भा॑ग॒धेये॑न यजेद् यजेद् भाग॒धेये॑नै॒व । \newline
14. भा॒ग॒धेये॑नै॒वैव भा॑ग॒धेये॑न भाग॒धेये॑नै॒वैना॑ नेना ने॒व भा॑ग॒धेये॑न भाग॒धेये॑नै॒वैनान्॑ । \newline
15. भा॒ग॒धेये॒नेति॑ भाग - धेये॑न । \newline
16. ए॒वैना॑ नेना ने॒वैवैनान्॑. यथाय॒थं ॅय॑थाय॒थ मे॑ना ने॒वैवैनान्॑. यथाय॒थम् । \newline
17. ए॒ना॒न्॒꣡. य॒था॒य॒थं ॅय॑थाय॒थ मे॑ना नेनान्. यथाय॒थम् क॑ल्पयति कल्पयति यथाय॒थ मे॑ना नेनान्. यथाय॒थम् क॑ल्पयति । \newline
18. य॒था॒य॒थम् क॑ल्पयति कल्पयति यथाय॒थं ॅय॑थाय॒थम् क॑ल्पयति॒ कल्प॑न्ते॒ कल्प॑न्ते कल्पयति यथाय॒थं ॅय॑थाय॒थम् क॑ल्पयति॒ कल्प॑न्ते । \newline
19. य॒था॒य॒थमिति॑ यथा - य॒थम् । \newline
20. क॒ल्प॒य॒ति॒ कल्प॑न्ते॒ कल्प॑न्ते कल्पयति कल्पयति॒ कल्प॑न्त ए॒वैव कल्प॑न्ते कल्पयति कल्पयति॒ कल्प॑न्त ए॒व । \newline
21. कल्प॑न्त ए॒वैव कल्प॑न्ते॒ कल्प॑न्त ए॒वैन्द्र मै॒न्द्र मे॒व कल्प॑न्ते॒ कल्प॑न्त ए॒वैन्द्रम् । \newline
22. ए॒वैन्द्र मै॒न्द्र मे॒वैवैन्द्र मेका॑दशकपाल॒ मेका॑दशकपाल मै॒न्द्र मे॒वैवैन्द्र मेका॑दशकपालम् । \newline
23. ऐ॒न्द्र मेका॑दशकपाल॒ मेका॑दशकपाल मै॒न्द्र मै॒न्द्र मेका॑दशकपाल॒म् निर् णिरेका॑दशकपाल मै॒न्द्र मै॒न्द्र मेका॑दशकपाल॒म् निः । \newline
24. एका॑दशकपाल॒म् निर् णिरेका॑दशकपाल॒ मेका॑दशकपाल॒म् निर् व॑पेद् वपे॒न् निरेका॑दशकपाल॒ मेका॑दशकपाल॒म् निर् व॑पेत् । \newline
25. एका॑दशकपाल॒मित्येका॑दश - क॒पा॒ल॒म् । \newline
26. निर् व॑पेद् वपे॒न् निर् णिर् व॑पेद् वैश्वदे॒वं ॅवै᳚श्वदे॒वं ॅव॑पे॒न् निर् णिर् व॑पेद् वैश्वदे॒वम् । \newline
27. व॒पे॒द् वै॒श्व॒दे॒वं ॅवै᳚श्वदे॒वं ॅव॑पेद् वपेद् वैश्वदे॒वम् द्वाद॑शकपाल॒म् द्वाद॑शकपालं ॅवैश्वदे॒वं ॅव॑पेद् वपेद् वैश्वदे॒वम् द्वाद॑शकपालम् । \newline
28. वै॒श्व॒दे॒वम् द्वाद॑शकपाल॒म् द्वाद॑शकपालं॒ ॅवैश्वदे॒वं ॅवै᳚श्वदे॒वम् द्वाद॑शकपाल॒म् ग्राम॑कामो॒ ग्राम॑कामो॒ द्वाद॑शकपालं ॅवैश्वदे॒वं ॅवै᳚श्वदे॒वम् द्वाद॑शकपाल॒म् ग्राम॑कामः । \newline
29. वै॒श्व॒दे॒वमिति॑ वैश्व - दे॒वम् । \newline
30. द्वाद॑शकपाल॒म् ग्राम॑कामो॒ ग्राम॑कामो॒ द्वाद॑शकपाल॒म् द्वाद॑शकपाल॒म् ग्राम॑काम॒ इन्द्र॒ मिन्द्र॒म् ग्राम॑कामो॒ द्वाद॑शकपाल॒म् द्वाद॑शकपाल॒म् ग्राम॑काम॒ इन्द्र᳚म् । \newline
31. द्वाद॑शकपाल॒मिति॒ द्वाद॑श - क॒पा॒ल॒म् । \newline
32. ग्राम॑काम॒ इन्द्र॒ मिन्द्र॒म् ग्राम॑कामो॒ ग्राम॑काम॒ इन्द्र॑म् च॒ चे न्द्र॒म् ग्राम॑कामो॒ ग्राम॑काम॒ इन्द्र॑म् च । \newline
33. ग्राम॑काम॒ इति॒ ग्राम॑ - का॒मः॒ । \newline
34. इन्द्र॑म् च॒ चे न्द्र॒ मिन्द्र॑म् चै॒वैव चे न्द्र॒ मिन्द्र॑म् चै॒व । \newline
35. चै॒वैव च॑ चै॒व विश्वा॒न्॒. विश्वा॑ ने॒व च॑ चै॒व विश्वान्॑ । \newline
36. ए॒व विश्वा॒न्॒. विश्वा॑ ने॒वैव विश्वाꣳ॑श्च च॒ विश्वा॑ ने॒वैव विश्वाꣳ॑श्च । \newline
37. विश्वाꣳ॑श्च च॒ विश्वा॒न्॒. विश्वाꣳ॑श्च दे॒वान् दे॒वाꣳश्च॒ विश्वा॒न्॒. विश्वाꣳ॑श्च दे॒वान् । \newline
38. च॒ दे॒वान् दे॒वाꣳश्च॑ च दे॒वान् थ्स्वेन॒ स्वेन॑ दे॒वाꣳश्च॑ च दे॒वान् थ्स्वेन॑ । \newline
39. दे॒वान् थ्स्वेन॒ स्वेन॑ दे॒वान् दे॒वान् थ्स्वेन॑ भाग॒धेये॑न भाग॒धेये॑न॒ स्वेन॑ दे॒वान् दे॒वान् थ्स्वेन॑ भाग॒धेये॑न । \newline
40. स्वेन॑ भाग॒धेये॑न भाग॒धेये॑न॒ स्वेन॒ स्वेन॑ भाग॒धेये॒नोपोप॑ भाग॒धेये॑न॒ स्वेन॒ स्वेन॑ भाग॒धेये॒नोप॑ । \newline
41. भा॒ग॒धेये॒नोपोप॑ भाग॒धेये॑न भाग॒धेये॒नोप॑ धावति धाव॒ त्युप॑ भाग॒धेये॑न भाग॒धेये॒नोप॑ धावति । \newline
42. भा॒ग॒धेये॒नेति॑ भाग - धेये॑न । \newline
43. उप॑ धावति धाव॒ त्युपोप॑ धावति॒ ते ते धा॑व॒ त्युपोप॑ धावति॒ ते । \newline
44. धा॒व॒ति॒ ते ते धा॑वति धावति॒ त ए॒वैव ते धा॑वति धावति॒ त ए॒व । \newline
45. त ए॒वैव ते त ए॒वास्मा॑ अस्मा ए॒व ते त ए॒वास्मै᳚ । \newline
46. ए॒वास्मा॑ अस्मा ए॒वैवास्मै॑ सजा॒तान् थ्स॑जा॒ता न॑स्मा ए॒वैवास्मै॑ सजा॒तान् । \newline
47. अ॒स्मै॒ स॒जा॒तान् थ्स॑जा॒ता न॑स्मा अस्मै सजा॒तान् प्र प्र स॑जा॒ता न॑स्मा अस्मै सजा॒तान् प्र । \newline
48. स॒जा॒तान् प्र प्र स॑जा॒तान् थ्स॑जा॒तान् प्र य॑च्छन्ति यच्छन्ति॒ प्र स॑जा॒तान् थ्स॑जा॒तान् प्र य॑च्छन्ति । \newline
49. स॒जा॒तानिति॑ स - जा॒तान् । \newline
50. प्र य॑च्छन्ति यच्छन्ति॒ प्र प्र य॑च्छन्ति ग्रा॒मी ग्रा॒मी य॑च्छन्ति॒ प्र प्र य॑च्छन्ति ग्रा॒मी । \newline
51. य॒च्छ॒न्ति॒ ग्रा॒मी ग्रा॒मी य॑च्छन्ति यच्छन्ति ग्रा॒म्ये॑वैव ग्रा॒मी य॑च्छन्ति यच्छन्ति ग्रा॒म्ये॑व । \newline
52. ग्रा॒म्ये॑वैव ग्रा॒मी ग्रा॒म्ये॑व भ॑वति भव त्ये॒व ग्रा॒मी ग्रा॒म्ये॑व भ॑वति । \newline
53. ए॒व भ॑वति भव त्ये॒वैव भ॑व त्यै॒न्द्र स्यै॒न्द्रस्य॑ भव त्ये॒वैव भ॑व त्यै॒न्द्रस्य॑ । \newline
54. भ॒व॒ त्यै॒न्द्र स्यै॒न्द्रस्य॑ भवति भव त्यै॒न्द्रस्या॑ व॒दाया॑ व॒दायै॒न्द्रस्य॑ भवति भवत्यै॒न्द्रस्या॑ व॒दाय॑ । \newline
55. ऐ॒न्द्रस्या॑ व॒दाया॑ व॒दायै॒न्द्र स्यै॒न्द्रस्या॑ व॒दाय॑ वैश्वदे॒वस्य॑ वैश्वदे॒वस्या॑ व॒दायै॒न्द्र स्यै॒न्द्रस्या॑ व॒दाय॑ वैश्वदे॒वस्य॑ । \newline
56. अ॒व॒दाय॑ वैश्वदे॒वस्य॑ वैश्वदे॒वस्या॑ व॒दाया॑ व॒दाय॑ वैश्वदे॒वस्या वाव॑ वैश्वदे॒वस्या॑ व॒दाया॑ व॒दाय॑ वैश्वदे॒वस्याव॑ । \newline
57. अ॒व॒दायेत्य॑व - दाय॑ । \newline
58. वै॒श्व॒दे॒वस्या वाव॑ वैश्वदे॒वस्य॑ वैश्वदे॒वस्याव॑ द्येद् द्ये॒दव॑ वैश्वदे॒वस्य॑ वैश्वदे॒वस्याव॑ द्येत् । \newline
59. वै॒श्व॒दे॒वस्येति॑ वैश्व - दे॒वस्य॑ । \newline
60. अव॑ द्येद् द्ये॒दवाव॑ द्ये॒दथाथ॑ द्ये॒दवाव॑ द्ये॒दथ॑ । \newline
61. द्ये॒ दथाथ॑ द्येद् द्ये॒ दथै॒न्द्र स्यै॒न्द्रस्याथ॑ द्येद् द्ये॒ दथै॒न्द्रस्य॑ । \newline
62. अथै॒ न्द्रस्यै॒ न्द्रस्या थाथै॒न्द्र स्यो॒परि॑ष्टा दु॒परि॑ष्टा दै॒न्द्रस्या थाथै॒न्द्र स्यो॒परि॑ष्टात् । \newline
63. ऐ॒न्द्रस्यो॒परि॑ष्टा दु॒परि॑ष्टा दै॒न्द्रस्यै॒न्द्र स्यो॒परि॑ष्टा दिन्द्रि॒येणे᳚ न्द्रि॒येणो॒परि॑ष्टा 
दै॒न्द्र स्यै॒न्द्र स्यो॒परि॑ष्टा दिन्द्रि॒येण॑ । \newline
\pagebreak
\markright{ TS 2.2.11.4  \hfill https://www.vedavms.in \hfill}
\addcontentsline{toc}{section}{ TS 2.2.11.4 }
\section*{ TS 2.2.11.4 }

\textbf{TS 2.2.11.4 } \newline
\textbf{Samhita Paata} \newline

-परि॑ष्टादिन्द्रि॒येणै॒वास्मा॑ उभ॒यतः॑ सजा॒तान् परि॑ गृह्णात्युपाधा॒य्य॑ पूर्वयं॒ ॅवासो॒ दक्षि॑णा सजा॒ताना॒मुप॑हित्यै॒ पृश्नि॑यै दु॒ग्धे प्रैय॑ङ्गवं च॒रुं निर्व॑पेन्म॒रुद्भ्यो॒ ग्राम॑कामः॒ पृश्नि॑यै॒ वै पय॑सो म॒रुतो॑ जा॒ताः पृश्नि॑यै प्रि॒यङ्ग॑वो मारु॒ताः खलु॒ वै दे॒वत॑या सजा॒ता म॒रुत॑ ए॒व स्वेन॑ भाग॒धेये॒नोप॑ धावति॒ त ए॒वास्मै॑ सजा॒तान् प्रय॑च्छन्ति ग्रा॒म्ये॑व भ॑वति प्रि॒यव॑ती याज्यानुवा॒क्ये॑ - [  ] \newline

\textbf{Pada Paata} \newline

उ॒परि॑ष्टात् । इ॒न्द्रि॒येण॑ । ए॒व । अ॒स्मै॒ । उ॒भ॒यतः॑ । स॒जा॒तानिति॑ स - जा॒तान् । परीति॑ । गृ॒ह्णा॒ति॒ । उ॒पा॒धा॒य्य॑ पूर्वय॒मित्यु॑पाधा॒य्य॑ - पू॒र्व॒य॒म् । वासः॑ । दक्षि॑णा । स॒जा॒ताना॒मिति॑ स - जा॒ताना᳚म् । उप॑हित्या॒ इत्युप॑ - हि॒त्यै॒ । पृश्नि॑यै । दु॒ग्धे । प्रैय॑ङ्गवम् । च॒रुम् । निरिति॑ । व॒पे॒त् । म॒रुद्भ्य॒ इति॑ म॒रुत् - भ्यः॒ । ग्राम॑काम॒ इति॒ ग्राम॑ - का॒मः॒ । पृश्नि॑यै । वै । पय॑सः । म॒रुतः॑ । जा॒ताः । पृश्नि॑यै । प्रि॒यङ्ग॑वः । मा॒रु॒ताः । खलु॑ । वै । दे॒वत॑या । स॒जा॒ता इति॑ स - जा॒ताः । म॒रुतः॑ । ए॒व । स्वेन॑ । भा॒ग॒धेये॒नेति॑ भाग - धेये॑न । उपेति॑ । धा॒व॒ति॒ । ते । ए॒व । अ॒स्मै॒ । स॒जा॒तानिति॑ स - जा॒तान् । प्रेति॑ । य॒च्छ॒न्ति॒ । ग्रा॒मी । ए॒व । भ॒व॒ति॒ । प्रि॒यव॑ती॒ इति॑ प्रि॒य - व॒ती॒ । या॒ज्या॒नु॒वा॒क्ये॑ इति॑ याज्या - अ॒नु॒वा॒क्ये᳚ ।  \newline


\textbf{Krama Paata} \newline

उ॒परि॑ष्टादिन्द्रि॒येण॑ । इ॒न्द्रि॒येणै॒व । ए॒वास्मै᳚ । अ॒स्मा॒ उ॒भ॒यतः॑ । उ॒भ॒यतः॑ सजा॒तान् । स॒जा॒तान् परि॑ । स॒जा॒तानिति॑ स - जा॒तान् । परि॑ गृह्णाति । गृ॒ह्णा॒त्यु॒पा॒धा॒य्य॑पूर्वयम् । उ॒पा॒धा॒य्य॑पूर्वयं॒ ॅवासः॑ । उ॒पा॒धा॒य्य॑पूर्वय॒मित्यु॑पाधा॒य्य॑ - पू॒र्व॒य॒म् । वासो॒ दक्षि॑णा । दक्षि॑णा सजा॒ताना᳚म् । स॒जा॒ताना॒मुप॑हित्यै । स॒जा॒ताना॒मिति॑ स - जा॒ताना᳚म् । उप॑हित्यै॒ पृश्ञि॑यै । उप॑हित्या॒ इत्युप॑ - हि॒त्यै॒ । पृश्ञि॑यै दु॒ग्धे । दु॒ग्धे प्रैय॑ङ्गवम् । प्रैय॑ङ्गवम् च॒रुम् । च॒रुम् निः । निर् व॑पेत् । व॒पे॒न् म॒रुद्भ्यः॑ । म॒रुद्भ्यो॒ ग्राम॑कामः । म॒रुद्भ्य॒ इति॑ म॒रुत् - भ्यः॒ । ग्राम॑कामः॒ पृश्ञि॑यै । ग्राम॑काम॒ इति॒ ग्राम॑ - का॒मः॒ । पृश्ञि॑यै॒ वै । वै पय॑सः । पय॑सो म॒रुतः॑ । म॒रुतो॑ जा॒ताः । जा॒ताः पृश्ञि॑यै । पृश्ञि॑यै प्रि॒यङ्ग॑वः । प्रि॒यङ्ग॑वो मारु॒ताः । मा॒रु॒ताः खलु॑ । खलु॒ वै । वै दे॒वत॑या । दे॒वत॑या सजा॒ताः । स॒जा॒ता म॒रुतः॑ । स॒जा॒ता इति॑ स - जा॒ताः । म॒रुत॑ ए॒व । ए॒व स्वेन॑ । स्वेन॑ भाग॒धेये॑न । भा॒ग॒धेये॒नोप॑ । भा॒ग॒धेये॒नेति॑ भाग - धेये॑न । उप॑ धावति । धा॒व॒ति॒ ते । त ए॒व । ए॒वास्मै᳚ । अ॒स्मे॒ स॒जा॒तान् । स॒जा॒तान् प्र । स॒जा॒तानिति॑ स - जा॒तान् । प्र य॑च्छन्ति । य॒च्छ॒न्ति॒ ग्रा॒मी । ग्रा॒म्ये॑व । ए॒व भ॑वति । भ॒व॒ति॒ प्रि॒यव॑ती । प्रि॒यव॑ती याज्यानुवा॒क्ये᳚ । प्रि॒यव॑ती॒ इति॑ प्रि॒य - व॒ती॒ । या॒ज्या॒नु॒वा॒क्ये॑ भवतः । या॒ज्या॒नु॒वा॒क्ये॑ इति॑ याज्या - अ॒नु॒वा॒क्ये᳚ \newline

\textbf{Jatai Paata} \newline

1. उ॒परि॑ष्टा दिन्द्रि॒येणे᳚ न्द्रि॒ये णो॒परि॑ष्टा दु॒परि॑ष्टा दिन्द्रि॒येण॑ । \newline
2. इ॒न्द्रि॒ये णै॒वैवे न्द्रि॒येणे᳚ न्द्रि॒ये णै॒व । \newline
3. ए॒वास्मा॑ अस्मा ए॒वैवास्मै᳚ । \newline
4. अ॒स्मा॒ उ॒भ॒यत॑ उभ॒यतो᳚ ऽस्मा अस्मा उभ॒यतः॑ । \newline
5. उ॒भ॒यतः॑ सजा॒तान् थ्स॑जा॒ता नु॑भ॒यत॑ उभ॒यतः॑ सजा॒तान् । \newline
6. स॒जा॒तान् परि॒ परि॑ सजा॒तान् थ्स॑जा॒तान् परि॑ । \newline
7. स॒जा॒तानिति॑ स - जा॒तान् । \newline
8. परि॑ गृह्णाति गृह्णाति॒ परि॒ परि॑ गृह्णाति । \newline
9. गृ॒ह्णा॒ त्यु॒पा॒धा॒य्य॑पूर्वय मुपाधा॒य्य॑पूर्वयम् गृह्णाति गृह्णा त्युपाधा॒य्य॑पूर्वयम् । \newline
10. उ॒पा॒धा॒य्य॑पूर्वयं॒ ॅवासो॒ वास॑ उपाधा॒य्य॑पूर्वय मुपाधा॒य्य॑पूर्वयं॒ ॅवासः॑ । \newline
11. उ॒पा॒धा॒य्य॑ पूर्वय॒मित्यु॑पाधा॒य्य॑ - पू॒र्व॒य॒म् । \newline
12. वासो॒ दक्षि॑णा॒ दक्षि॑णा॒ वासो॒वासो॒ दक्षि॑णा । \newline
13. दक्षि॑णा सजा॒तानाꣳ॑ सजा॒ताना॒म् दक्षि॑णा॒ दक्षि॑णा सजा॒ताना᳚म् । \newline
14. स॒जा॒ताना॒ मुप॑हित्या॒ उप॑हित्यै सजा॒तानाꣳ॑ सजा॒ताना॒ मुप॑हित्यै । \newline
15. स॒जा॒ताना॒मिति॑ स - जा॒ताना᳚म् । \newline
16. उप॑हित्यै॒ पृश्ञि॑यै॒ पृश्ञि॑या॒ उप॑हित्या॒ उप॑हित्यै॒ पृश्ञि॑यै । \newline
17. उप॑हित्या॒ इत्युप॑ - हि॒त्यै॒ । \newline
18. पृश्ञि॑यै दु॒ग्धे दु॒ग्धे पृश्ञि॑यै॒ पृश्ञि॑यै दु॒ग्धे । \newline
19. दु॒ग्धे प्रैय॑ङ्गव॒म् प्रैय॑ङ्गवम् दु॒ग्धे दु॒ग्धे प्रैय॑ङ्गवम् । \newline
20. प्रैय॑ङ्गवम् च॒रुम् च॒रुम् प्रैय॑ङ्गव॒म् प्रैय॑ङ्गवम् च॒रुम् । \newline
21. च॒रुम् निर् णिश्च॒रुम् च॒रुम् निः । \newline
22. निर् व॑पेद् वपे॒न् निर् णिर् व॑पेत् । \newline
23. व॒पे॒न् म॒रुद्भ्यो॑ म॒रुद्भ्यो॑ वपेद् वपेन् म॒रुद्भ्यः॑ । \newline
24. म॒रुद्भ्यो॒ ग्राम॑कामो॒ ग्राम॑कामो म॒रुद्भ्यो॑ म॒रुद्भ्यो॒ ग्राम॑कामः । \newline
25. म॒रुद्भ्य॒ इति॑ म॒रुत् - भ्यः॒ । \newline
26. ग्राम॑कामः॒ पृश्ञि॑यै॒ पृश्ञि॑यै॒ ग्राम॑कामो॒ ग्राम॑कामः॒ पृश्ञि॑यै । \newline
27. ग्राम॑काम॒ इति॒ ग्राम॑ - का॒मः॒ । \newline
28. पृश्ञि॑यै॒ वै वै पृश्ञि॑यै॒ पृश्ञि॑यै॒ वै । \newline
29. वै पय॑सः॒ पय॑ सो॒ वै वै पय॑सः । \newline
30. पय॑सो म॒रुतो॑ म॒रुतः॒ पय॑सः॒ पय॑सो म॒रुतः॑ । \newline
31. म॒रुतो॑ जा॒ता जा॒ता म॒रुतो॑ म॒रुतो॑ जा॒ताः । \newline
32. जा॒ताः पृश्ञि॑यै॒ पृश्ञि॑यै जा॒ता जा॒ताः पृश्ञि॑यै । \newline
33. पृश्ञि॑यै प्रि॒यङ्ग॑वः प्रि॒यङ्ग॑वः॒ पृश्ञि॑यै॒ पृश्ञि॑यै प्रि॒यङ्ग॑वः । \newline
34. प्रि॒यङ्ग॑वो मारु॒ता मा॑रु॒ताः प्रि॒यङ्ग॑वः प्रि॒यङ्ग॑वो मारु॒ताः । \newline
35. मा॒रु॒ताः खलु॒ खलु॑ मारु॒ता मा॑रु॒ताः खलु॑ । \newline
36. खलु॒ वै वै खलु॒ खलु॒ वै । \newline
37. वै दे॒वत॑या दे॒वत॑या॒ वै वै दे॒वत॑या । \newline
38. दे॒वत॑या सजा॒ताः स॑जा॒ता दे॒वत॑या दे॒वत॑या सजा॒ताः । \newline
39. स॒जा॒ता म॒रुतो॑ म॒रुतः॑ सजा॒ताः स॑जा॒ता म॒रुतः॑ । \newline
40. स॒जा॒ता इति॑ स - जा॒ताः । \newline
41. म॒रुत॑ ए॒वैव म॒रुतो॑ म॒रुत॑ ए॒व । \newline
42. ए॒व स्वेन॒ स्वेनै॒वैव स्वेन॑ । \newline
43. स्वेन॑ भाग॒धेये॑न भाग॒धेये॑न॒ स्वेन॒ स्वेन॑ भाग॒धेये॑न । \newline
44. भा॒ग॒धेये॒नोपोप॑ भाग॒धेये॑न भाग॒धेये॒नोप॑ । \newline
45. भा॒ग॒धेये॒नेति॑ भाग - धेये॑न । \newline
46. उप॑ धावति धाव॒ त्युपोप॑ धावति । \newline
47. धा॒व॒ति॒ ते ते धा॑वति धावति॒ ते । \newline
48. त ए॒वैव ते त ए॒व । \newline
49. ए॒वास्मा॑ अस्मा ए॒वैवास्मै᳚ । \newline
50. अ॒स्मै॒ स॒जा॒तान् थ्स॑जा॒ता न॑स्मा अस्मै सजा॒तान् । \newline
51. स॒जा॒तान् प्र प्र स॑जा॒तान् थ्स॑जा॒तान् प्र । \newline
52. स॒जा॒तानिति॑ स - जा॒तान् । \newline
53. प्र य॑च्छन्ति यच्छन्ति॒ प्र प्र य॑च्छन्ति । \newline
54. य॒च्छ॒न्ति॒ ग्रा॒मी ग्रा॒मी य॑च्छन्ति यच्छन्ति ग्रा॒मी । \newline
55. ग्रा॒म्ये॑वैव ग्रा॒मी ग्रा॒म्ये॑व । \newline
56. ए॒व भ॑वति भव त्ये॒वैव भ॑वति । \newline
57. भ॒व॒ति॒ प्रि॒यव॑ती प्रि॒यव॑ती भवति भवति प्रि॒यव॑ती । \newline
58. प्रि॒यव॑ती याज्यानुवा॒क्ये॑ याज्यानुवा॒क्ये᳚ प्रि॒यव॑ती प्रि॒यव॑ती याज्यानुवा॒क्ये᳚ । \newline
59. प्रि॒यव॑ती॒ इति॑ प्रि॒य - व॒ती॒ । \newline
60. या॒ज्या॒नु॒वा॒क्ये॑ भवतो भवतो याज्यानुवा॒क्ये॑ याज्यानुवा॒क्ये॑ भवतः । \newline
61. या॒ज्या॒नु॒वा॒क्ये॑ इति॑ याज्या - अ॒नु॒वा॒क्ये᳚ । \newline

\textbf{Ghana Paata } \newline

1. उ॒परि॑ष्टा दिन्द्रि॒येणे᳚ न्द्रि॒येणो॒परि॑ष्टा दु॒परि॑ष्टा दिन्द्रि॒ये णै॒वैवे न्द्रि॒येणो॒परि॑ष्टा दु॒परि॑ष्टा दिन्द्रि॒येणै॒व । \newline
2. इ॒न्द्रि॒येणै॒ वैवे न्द्रि॒येणे᳚ न्द्रि॒येणै॒वास्मा॑ अस्मा ए॒वे न्द्रि॒येणे᳚ न्द्रि॒येणै॒वास्मै᳚ । \newline
3. ए॒वास्मा॑ अस्मा ए॒वैवास्मा॑ उभ॒यत॑ उभ॒यतो᳚ ऽस्मा ए॒वैवास्मा॑ उभ॒यतः॑ । \newline
4. अ॒स्मा॒ उ॒भ॒यत॑ उभ॒यतो᳚ ऽस्मा अस्मा उभ॒यतः॑ सजा॒तान् थ्स॑जा॒ता नु॑भ॒यतो᳚ ऽस्मा अस्मा उभ॒यतः॑ सजा॒तान् । \newline
5. उ॒भ॒यतः॑ सजा॒तान् थ्स॑जा॒ता नु॑भ॒यत॑ उभ॒यतः॑ सजा॒तान् परि॒ परि॑ सजा॒ता नु॑भ॒यत॑ उभ॒यतः॑ सजा॒तान् परि॑ । \newline
6. स॒जा॒तान् परि॒ परि॑ सजा॒तान् थ्स॑जा॒तान् परि॑ गृह्णाति गृह्णाति॒ परि॑ सजा॒तान् थ्स॑जा॒तान् परि॑ गृह्णाति । \newline
7. स॒जा॒तानिति॑ स - जा॒तान् । \newline
8. परि॑ गृह्णाति गृह्णाति॒ परि॒ परि॑ गृह्णा त्युपाधा॒य्य॑पूर्वय मुपाधा॒य्य॑पूर्वयम् गृह्णाति॒ परि॒ परि॑ गृह्णा त्युपाधा॒य्य॑पूर्वयम् । \newline
9. गृ॒ह्णा॒ त्यु॒पा॒धा॒य्य॑पूर्वय मुपाधा॒य्य॑पूर्वयम् गृह्णाति गृह्णा त्युपाधा॒य्य॑पूर्वयं॒ ॅवासो॒ वास॑ उपाधा॒य्य॑पूर्वयम् गृह्णाति गृह्णा त्युपाधा॒य्य॑पूर्वयं॒ ॅवासः॑ । \newline
10. उ॒पा॒धा॒य्य॑पूर्वयं॒ ॅवासो॒ वास॑ उपाधा॒य्य॑पूर्वय मुपाधा॒य्य॑पूर्वयं॒ ॅवासो॒ दक्षि॑णा॒ दक्षि॑णा॒ वास॑ उपाधा॒य्य॑पूर्वय मुपाधा॒य्य॑पूर्वयं॒ ॅवासो॒ दक्षि॑णा । \newline
11. उ॒पा॒धा॒य्य॑पूर्वय॒मित्यु॑पाधा॒य्य॑ - पू॒र्व॒य॒म् । \newline
12. वासो॒ दक्षि॑णा॒ दक्षि॑णा॒ वासो॒ वासो॒ दक्षि॑णा सजा॒तानाꣳ॑ सजा॒ताना॒म् दक्षि॑णा॒ वासो॒ वासो॒ दक्षि॑णा सजा॒ताना᳚म् । \newline
13. दक्षि॑णा सजा॒तानाꣳ॑ सजा॒ताना॒म् दक्षि॑णा॒ दक्षि॑णा सजा॒ताना॒ मुप॑हित्या॒ उप॑हित्यै सजा॒ताना॒म् दक्षि॑णा॒ दक्षि॑णा सजा॒ताना॒ मुप॑हित्यै । \newline
14. स॒जा॒ताना॒ मुप॑हित्या॒ उप॑हित्यै सजा॒तानाꣳ॑ सजा॒ताना॒ मुप॑हित्यै॒ पृश्ञि॑यै॒ पृश्ञि॑या॒ उप॑हित्यै सजा॒तानाꣳ॑ सजा॒ताना॒ मुप॑हित्यै॒ पृश्ञि॑यै । \newline
15. स॒जा॒ताना॒मिति॑ स - जा॒ताना᳚म् । \newline
16. उप॑हित्यै॒ पृश्ञि॑यै॒ पृश्ञि॑या॒ उप॑हित्या॒ उप॑हित्यै॒ पृश्ञि॑यै दु॒ग्धे दु॒ग्धे पृश्ञि॑या॒ उप॑हित्या॒ उप॑हित्यै॒ पृश्ञि॑यै दु॒ग्धे । \newline
17. उप॑हित्या॒ इत्युप॑ - हि॒त्यै॒ । \newline
18. पृश्ञि॑यै दु॒ग्धे दु॒ग्धे पृश्ञि॑यै॒ पृश्ञि॑यै दु॒ग्धे प्रैय॑ङ्गव॒म् प्रैय॑ङ्गवम् दु॒ग्धे पृश्ञि॑यै॒ पृश्ञि॑यै दु॒ग्धे प्रैय॑ङ्गवम् । \newline
19. दु॒ग्धे प्रैय॑ङ्गव॒म् प्रैय॑ङ्गवम् दु॒ग्धे दु॒ग्धे प्रैय॑ङ्गवम् च॒रुम् च॒रुम् प्रैय॑ङ्गवम् दु॒ग्धे दु॒ग्धे प्रैय॑ङ्गवम् च॒रुम् । \newline
20. प्रैय॑ङ्गवम् च॒रुम् च॒रुम् प्रैय॑ङ्गव॒म् प्रैय॑ङ्गवम् च॒रुम् निर् णिश्च॒रुम् प्रैय॑ङ्गव॒म् प्रैय॑ङ्गवम् च॒रुम् निः । \newline
21. च॒रुम् निर् णिश्च॒रुम् च॒रुम् निर् व॑पेद् वपे॒न् निश्च॒रुम् च॒रुम् निर् व॑पेत् । \newline
22. निर् व॑पेद् वपे॒न् निर् णिर् व॑पेन् म॒रुद्भ्यो॑ म॒रुद्भ्यो॑ वपे॒न् निर् णिर् व॑पेन् म॒रुद्भ्यः॑ । \newline
23. व॒पे॒न् म॒रुद्भ्यो॑ म॒रुद्भ्यो॑ वपेद् वपेन् म॒रुद्भ्यो॒ ग्राम॑कामो॒ ग्राम॑कामो म॒रुद्भ्यो॑ वपेद् वपेन् म॒रुद्भ्यो॒ ग्राम॑कामः । \newline
24. म॒रुद्भ्यो॒ ग्राम॑कामो॒ ग्राम॑कामो म॒रुद्भ्यो॑ म॒रुद्भ्यो॒ ग्राम॑कामः॒ पृश्ञि॑यै॒ पृश्ञि॑यै॒ ग्राम॑कामो म॒रुद्भ्यो॑ म॒रुद्भ्यो॒ ग्राम॑कामः॒ पृश्ञि॑यै । \newline
25. म॒रुद्भ्य॒ इति॑ म॒रुत् - भ्यः॒ । \newline
26. ग्राम॑कामः॒ पृश्ञि॑यै॒ पृश्ञि॑यै॒ ग्राम॑कामो॒ ग्राम॑कामः॒ पृश्ञि॑यै॒ वै वै पृश्ञि॑यै॒ ग्राम॑कामो॒ ग्राम॑कामः॒ पृश्ञि॑यै॒ वै । \newline
27. ग्राम॑काम॒ इति॒ ग्राम॑ - का॒मः॒ । \newline
28. पृश्ञि॑यै॒ वै वै पृश्ञि॑यै॒ पृश्ञि॑यै॒ वै पय॑सः॒ पय॑सो॒ वै पृश्ञि॑यै॒ पृश्ञि॑यै॒ वै पय॑सः । \newline
29. वै पय॑सः॒ पय॑सो॒ वै वै पय॑सो म॒रुतो॑ म॒रुतः॒ पय॑सो॒ वै वै पय॑सो म॒रुतः॑ । \newline
30. पय॑सो म॒रुतो॑ म॒रुतः॒ पय॑सः॒ पय॑सो म॒रुतो॑ जा॒ता जा॒ता म॒रुतः॒ पय॑सः॒ पय॑सो म॒रुतो॑ जा॒ताः । \newline
31. म॒रुतो॑ जा॒ता जा॒ता म॒रुतो॑ म॒रुतो॑ जा॒ताः पृश्ञि॑यै॒ पृश्ञि॑यै जा॒ता म॒रुतो॑ म॒रुतो॑ जा॒ताः पृश्ञि॑यै । \newline
32. जा॒ताः पृश्ञि॑यै॒ पृश्ञि॑यै जा॒ता जा॒ताः पृश्ञि॑यै प्रि॒यङ्ग॑वः प्रि॒यङ्ग॑वः॒ पृश्ञि॑यै जा॒ता जा॒ताः पृश्ञि॑यै प्रि॒यङ्ग॑वः । \newline
33. पृश्ञि॑यै प्रि॒यङ्ग॑वः प्रि॒यङ्ग॑वः॒ पृश्ञि॑यै॒ पृश्ञि॑यै प्रि॒यङ्ग॑वो मारु॒ता मा॑रु॒ताः प्रि॒यङ्ग॑वः॒ पृश्ञि॑यै॒ पृश्ञि॑यै प्रि॒यङ्ग॑वो मारु॒ताः । \newline
34. प्रि॒यङ्ग॑वो मारु॒ता मा॑रु॒ताः प्रि॒यङ्ग॑वः प्रि॒यङ्ग॑वो मारु॒ताः खलु॒ खलु॑ मारु॒ताः प्रि॒यङ्ग॑वः प्रि॒यङ्ग॑वो मारु॒ताः खलु॑ । \newline
35. मा॒रु॒ताः खलु॒ खलु॑ मारु॒ता मा॑रु॒ताः खलु॒ वै वै खलु॑ मारु॒ता मा॑रु॒ताः खलु॒ वै । \newline
36. खलु॒ वै वै खलु॒ खलु॒ वै दे॒वत॑या दे॒वत॑या॒ वै खलु॒ खलु॒ वै दे॒वत॑या । \newline
37. वै दे॒वत॑या दे॒वत॑या॒ वै वै दे॒वत॑या सजा॒ताः स॑जा॒ता दे॒वत॑या॒ वै वै दे॒वत॑या सजा॒ताः । \newline
38. दे॒वत॑या सजा॒ताः स॑जा॒ता दे॒वत॑या दे॒वत॑या सजा॒ता म॒रुतो॑ म॒रुतः॑ सजा॒ता दे॒वत॑या दे॒वत॑या सजा॒ता म॒रुतः॑ । \newline
39. स॒जा॒ता म॒रुतो॑ म॒रुतः॑ सजा॒ताः स॑जा॒ता म॒रुत॑ ए॒वैव म॒रुतः॑ सजा॒ताः स॑जा॒ता म॒रुत॑ ए॒व । \newline
40. स॒जा॒ता इति॑ स - जा॒ताः । \newline
41. म॒रुत॑ ए॒वैव म॒रुतो॑ म॒रुत॑ ए॒व स्वेन॒ स्वेनै॒व म॒रुतो॑ म॒रुत॑ ए॒व स्वेन॑ । \newline
42. ए॒व स्वेन॒ स्वेनै॒वैव स्वेन॑ भाग॒धेये॑न भाग॒धेये॑न॒ स्वेनै॒वैव स्वेन॑ भाग॒धेये॑न । \newline
43. स्वेन॑ भाग॒धेये॑न भाग॒धेये॑न॒ स्वेन॒ स्वेन॑ भाग॒धेये॒नोपोप॑ भाग॒धेये॑न॒ स्वेन॒ स्वेन॑ भाग॒धेये॒नोप॑ । \newline
44. भा॒ग॒धेये॒नोपोप॑ भाग॒धेये॑न भाग॒धेये॒नोप॑ धावति धाव॒ त्युप॑ भाग॒धेये॑न भाग॒धेये॒नोप॑ धावति । \newline
45. भा॒ग॒धेये॒नेति॑ भाग - धेये॑न । \newline
46. उप॑ धावति धाव॒ त्युपोप॑ धावति॒ ते ते धा॑व॒ त्युपोप॑ धावति॒ ते । \newline
47. धा॒व॒ति॒ ते ते धा॑वति धावति॒ त ए॒वैव ते धा॑वति धावति॒ त ए॒व । \newline
48. त ए॒वैव ते त ए॒वास्मा॑ अस्मा ए॒व ते त ए॒वास्मै᳚ । \newline
49. ए॒वास्मा॑ अस्मा ए॒वैवास्मै॑ सजा॒तान् थ्स॑जा॒ता न॑स्मा ए॒वैवास्मै॑ सजा॒तान् । \newline
50. अ॒स्मै॒ स॒जा॒तान् थ्स॑जा॒ता न॑स्मा अस्मै सजा॒तान् प्र प्र स॑जा॒ता न॑स्मा अस्मै सजा॒तान् प्र । \newline
51. स॒जा॒तान् प्र प्र स॑जा॒तान् थ्स॑जा॒तान् प्र य॑च्छन्ति यच्छन्ति॒ प्र स॑जा॒तान् थ्स॑जा॒तान् प्र य॑च्छन्ति । \newline
52. स॒जा॒तानिति॑ स - जा॒तान् । \newline
53. प्र य॑च्छन्ति यच्छन्ति॒ प्र प्र य॑च्छन्ति ग्रा॒मी ग्रा॒मी य॑च्छन्ति॒ प्र प्र य॑च्छन्ति ग्रा॒मी । \newline
54. य॒च्छ॒न्ति॒ ग्रा॒मी ग्रा॒मी य॑च्छन्ति यच्छन्ति ग्रा॒म्ये॑वैव ग्रा॒मी य॑च्छन्ति यच्छन्ति ग्रा॒म्ये॑व । \newline
55. ग्रा॒म्ये॑वैव ग्रा॒मी ग्रा॒म्ये॑व भ॑वति भवत्ये॒व ग्रा॒मी ग्रा॒म्ये॑व भ॑वति । \newline
56. ए॒व भ॑वति भव त्ये॒वैव भ॑वति प्रि॒यव॑ती प्रि॒यव॑ती भव त्ये॒वैव भ॑वति प्रि॒यव॑ती । \newline
57. भ॒व॒ति॒ प्रि॒यव॑ती प्रि॒यव॑ती भवति भवति प्रि॒यव॑ती याज्यानुवा॒क्ये॑ याज्यानुवा॒क्ये᳚ प्रि॒यव॑ती भवति भवति प्रि॒यव॑ती याज्यानुवा॒क्ये᳚ । \newline
58. प्रि॒यव॑ती याज्यानुवा॒क्ये॑ याज्यानुवा॒क्ये᳚ प्रि॒यव॑ती प्रि॒यव॑ती याज्यानुवा॒क्ये॑ भवतो भवतो याज्यानुवा॒क्ये᳚ प्रि॒यव॑ती प्रि॒यव॑ती याज्यानुवा॒क्ये॑ भवतः । \newline
59. प्रि॒यव॑ती॒ इति॑ प्रि॒य - व॒ती॒ । \newline
60. या॒ज्या॒नु॒वा॒क्ये॑ भवतो भवतो याज्यानुवा॒क्ये॑ याज्यानुवा॒क्ये॑ भवतः प्रि॒यम् प्रि॒यम् भ॑वतो याज्यानुवा॒क्ये॑ याज्यानुवा॒क्ये॑ भवतः प्रि॒यम् । \newline
61. या॒ज्या॒नु॒वा॒क्ये॑ इति॑ याज्या - अ॒नु॒वा॒क्ये᳚ । \newline
\pagebreak
\markright{ TS 2.2.11.5  \hfill https://www.vedavms.in \hfill}
\addcontentsline{toc}{section}{ TS 2.2.11.5 }
\section*{ TS 2.2.11.5 }

\textbf{TS 2.2.11.5 } \newline
\textbf{Samhita Paata} \newline

भवतः प्रि॒यमे॒वैनꣳ॑ समा॒नानां᳚ करोति द्वि॒पदा॑ पुरोऽनुवा॒क्या॑ भवति द्वि॒पद॑ ए॒वाव॑ रुन्धे॒ चतु॑ष्पदा या॒ज्या॑ चतु॑ष्पद ए॒व प॒शूनव॑ रुन्धे देवासु॒राः संॅय॑त्ता आस॒न् ते दे॒वा मि॒थो विप्रि॑या आस॒न् ते᳚ (1॒) ऽन्यो᳚ऽन्यस्मै॒ ज्यैष्‌ठ्या॒या-ति॑ष्ठमानाश्चतु॒र्द्धा व्य॑क्रामन्न॒ग्निर्वसु॑भिः॒ सोमो॑ रु॒द्रैरिन्द्रो॑ म॒रुद्भि॒र्वरु॑ण आदि॒त्यैः स इन्द्रः॑ प्र॒जाप॑ति॒मुपा॑धाव॒त् त - [  ] \newline

\textbf{Pada Paata} \newline

भ॒व॒तः॒ । प्रि॒यम् । ए॒व । ए॒न॒म् । स॒मा॒नाना᳚म् । क॒रो॒ति॒ । द्वि॒पदेति॑ द्वि - पदा᳚ । पु॒रो॒नु॒वा॒क्येति॑ पुरः - अ॒नु॒वा॒क्या᳚ । भ॒व॒ति॒ । द्वि॒पद॒ इति॑ द्वि - पदः॑ । ए॒व । अवेति॑ । रु॒न्धे॒ । चतु॑ष्प॒देति॒ चतुः॑ - प॒दा॒ । या॒ज्या᳚ । चतु॑ष्पद॒ इति॒ चतुः॑ - प॒दः॒ । ए॒व । प॒शून् । अवेति॑ । रु॒न्धे॒ । दे॒वा॒सु॒रा इति॑ देव - अ॒सु॒राः । संॅय॑त्ता॒ इति॒ सं - य॒त्ताः॒ । आ॒स॒न्न् । ते । दे॒वाः । मि॒थः । विप्रि॑या॒ इति॒ वि - प्रि॒याः॒ । आ॒स॒न्न् । ते । अ॒न्यः । अ॒न्यस्मै᳚ । ज्यैष्ठ्या॑य । अति॑ष्ठमानाः । च॒तु॒र्द्धेति॑ चतुः - धा । वीति॑ । अ॒क्रा॒म॒न्न् । अ॒ग्निः । वसु॑भि॒रिति॒ वसु॑ - भिः॒ । सोमः॑ । रु॒द्रैः । इन्द्रः॑ । म॒रुद्भि॒रिति॑ म॒रुत् - भिः॒ । वरु॑णः । आ॒दि॒त्यैः । सः । इन्द्रः॑ । प्र॒जाप॑ति॒मिति॑ प्र॒जा - प॒ति॒म् । उपेति॑ । अ॒धा॒व॒त् । तम् ।  \newline


\textbf{Krama Paata} \newline

भ॒व॒तः॒ प्रि॒यम् । प्रि॒यमे॒व । ए॒वैन᳚म् । ए॒नꣳ॒॒ स॒मा॒नाना᳚म् । स॒मा॒नाना᳚म् करोति । क॒रो॒ति॒ द्वि॒पदा᳚ । द्वि॒पदा॑ पुरोनुवा॒क्या᳚ । द्वि॒पदेति॑ द्वि - पदा᳚ । पु॒रो॒नु॒वा॒क्या॑ भवति । पु॒रो॒नु॒वा॒क्येति॑ पुरः - अ॒नु॒वा॒क्या᳚ । भ॒व॒ति॒ द्वि॒पदः॑ । द्वि॒पद॑ ए॒व । द्वि॒पद॒ इति॑ द्वि - पदः॑ । ए॒वाव॑ । अव॑ रुन्धे । रु॒न्धे॒ चतु॑ष्पदा । चतु॑ष्पदा या॒ज्या᳚ । चतु॑ष्प॒देति॒ चतुः॑ - प॒दा॒ । या॒ज्या॑ चतु॑ष्पदः । चतु॑ष्पद ए॒व । चतु॑ष्पद॒ इति॒ चतुः॑ - प॒दः॒ । ए॒व प॒शून् । प॒शूनव॑ । अव॑ रुन्धे । रु॒न्धे॒ दे॒वा॒सु॒राः । दे॒वा॒सु॒राः सम्ॅय॑त्ताः । दे॒वा॒सु॒रा इति॑ देव - अ॒सु॒राः । सम्ॅय॑त्ता आसन्न् । सम्ॅय॑त्ता॒ इति॒ सं - य॒त्ताः॒ । आ॒स॒न् ते । ते दे॒वाः । दे॒वा मि॒थः । मि॒थो विप्रि॑याः । विप्रि॑या आसन्न् । विप्रि॑या॒ इति॒ वि - प्रि॒याः॒ । आ॒स॒न् ते । ते᳚ऽन्यः । अ॒न्यो᳚ ऽन्यस्मै᳚ । अ॒न्यस्मै॒ जैष्ठ्या॑य । जैष्ठ्या॒याति॑ष्ठमानाः । अति॑ष्ठमाना श्चतु॒र्द्धा । च॒तु॒र्द्धा वि । च॒तु॒र्द्धेति॑ चतुः - धा । व्य॑क्रामन्न् । अ॒क्रा॒म॒न्न॒ग्निः । अ॒ग्निर् वसु॑भिः । वसु॑भिः॒ सोमः॑ । वसु॑भि॒रिति॒ वसु॑ - भिः॒ । सोमो॑ रु॒द्रैः । रु॒द्रैरिन्द्रः॑ । इन्द्रो॑ म॒रुद्भिः॑ । म॒रुद्भि॒र् वरु॑णः । म॒रुद्भि॒रिति॑ म॒रुत् - भिः॒ । वरु॑ण आदि॒त्यैः । आ॒दि॒त्यैः सः । स इन्द्रः॑ । इन्द्रः॑ प्र॒जाप॑तिम् । प्र॒जाप॑ति॒मुप॑ । प्र॒जाप॑ति॒मिति॑ प्र॒जा - प॒ति॒म् । उपा॑धावत् । अ॒धा॒व॒त् तम् । तमे॒तया᳚ \newline

\textbf{Jatai Paata} \newline

1. भ॒व॒तः॒ प्रि॒यम् प्रि॒यम् भ॑वतो भवतः प्रि॒यम् । \newline
2. प्रि॒य मे॒वैव प्रि॒यम् प्रि॒य मे॒व । \newline
3. ए॒वैन॑ मेन मे॒वैवैन᳚म् । \newline
4. ए॒नꣳ॒॒ स॒मा॒नानाꣳ॑ समा॒नाना॑ मेन मेनꣳ समा॒नाना᳚म् । \newline
5. स॒मा॒नाना᳚म् करोति करोति समा॒नानाꣳ॑ समा॒नाना᳚म् करोति । \newline
6. क॒रो॒ति॒ द्वि॒पदा᳚ द्वि॒पदा॑ करोति करोति द्वि॒पदा᳚ । \newline
7. द्वि॒पदा॑ पुरोनुवा॒क्या॑ पुरोनुवा॒क्या᳚ द्वि॒पदा᳚ द्वि॒पदा॑ पुरोनुवा॒क्या᳚ । \newline
8. द्वि॒पदेति॑ द्वि - पदा᳚ । \newline
9. पु॒रो॒नु॒वा॒क्या॑ भवति भवति पुरोनुवा॒क्या॑ पुरोनुवा॒क्या॑ भवति । \newline
10. पु॒रो॒नु॒वा॒क्येति॑ पुरः - अ॒नु॒वा॒क्या᳚ । \newline
11. भ॒व॒ति॒ द्वि॒पदो᳚ द्वि॒पदो॑ भवति भवति द्वि॒पदः॑ । \newline
12. द्वि॒पद॑ ए॒वैव द्वि॒पदो᳚ द्वि॒पद॑ ए॒व । \newline
13. द्वि॒पद॒ इति॑ द्वि - पदः॑ । \newline
14. ए॒वा वावै॒ वैवाव॑ । \newline
15. अव॑ रुन्धे रु॒न्धे ऽवाव॑ रुन्धे । \newline
16. रु॒न्धे॒ चतु॑ष्पदा॒ चतु॑ष्पदा रुन्धे रुन्धे॒ चतु॑ष्पदा । \newline
17. चतु॑ष्पदा या॒ज्या॑ या॒ज्या॑ चतु॑ष्पदा॒ चतु॑ष्पदा या॒ज्या᳚ । \newline
18. चतु॑ष्प॒देति॒ चतुः॑ - प॒दा॒ । \newline
19. या॒ज्या॑ चतु॑ष्पद॒ श्चतु॑ष्पदो या॒ज्या॑ या॒ज्या॑ चतु॑ष्पदः । \newline
20. चतु॑ष्पद ए॒वैव चतु॑ष्पद॒ श्चतु॑ष्पद ए॒व । \newline
21. चतु॑ष्पद॒ इति॒ चतुः॑ - प॒दः॒ । \newline
22. ए॒व प॒शून् प॒शू ने॒वैव प॒शून् । \newline
23. प॒शू नवाव॑ प॒शून् प॒शू नव॑ । \newline
24. अव॑ रुन्धे रु॒न्धे ऽवाव॑ रुन्धे । \newline
25. रु॒न्धे॒ दे॒वा॒सु॒रा दे॑वासु॒रा रु॑न्धे रुन्धे देवासु॒राः । \newline
26. दे॒वा॒सु॒राः संॅय॑त्ताः॒ संॅय॑त्ता देवासु॒रा दे॑वासु॒राः संॅय॑त्ताः । \newline
27. दे॒वा॒सु॒रा इति॑ देव - अ॒सु॒राः । \newline
28. संॅय॑त्ता आसन् नास॒न् थ्संॅय॑त्ताः॒ संॅय॑त्ता आसन्न् । \newline
29. संॅय॑त्ता॒ इति॒ सं - य॒त्ताः॒ । \newline
30. आ॒स॒न् ते त आ॑सन् नास॒न् ते । \newline
31. ते दे॒वा दे॒वा स्ते ते दे॒वाः । \newline
32. दे॒वा मि॒थो मि॒थो दे॒वा दे॒वा मि॒थः । \newline
33. मि॒थो विप्रि॑या॒ विप्रि॑या मि॒थो मि॒थो विप्रि॑याः । \newline
34. विप्रि॑या आसन् नास॒न्॒. विप्रि॑या॒ विप्रि॑या आसन्न् । \newline
35. विप्रि॑या॒ इति॒ वि - प्रि॒याः॒ । \newline
36. आ॒स॒न् ते त आ॑सन् नास॒न् ते । \newline
37. ते᳚(1॒) ऽन्यो᳚ ऽन्यस्ते ते᳚ ऽन्यः । \newline
38. अ॒न्यो᳚ ऽन्यस्मा॑ अ॒न्यस्मा॑ अ॒न्यो᳚(1॒) ऽन्यो᳚ ऽन्यस्मै᳚ । \newline
39. अ॒न्यस्मै॒ ज्यैष्ठ्‍या॑य॒ ज्यैष्ठ्‍या॑या॒न्यस्मा॑ अ॒न्यस्मै॒ ज्यैष्ठ्‍या॑य । \newline
40. ज्यैष्ठ्‍या॒याति॑ष् ठमाना॒ अति॑ष्ठमाना॒ ज्यैष्ठ्‍या॑य॒ ज्यैष्ठ्‍या॒या ति॑ष्ठमानाः । \newline
41. अति॑ष्ठमाना श्चतु॒र्द्धा च॑तु॒र्द्धा ऽति॑ष्ठमाना॒ अति॑ष्ठमाना श्चतु॒र्द्धा । \newline
42. च॒तु॒र्द्धा वि वि च॑तु॒र्द्धा च॑तु॒र्द्धा वि । \newline
43. च॒तु॒र्द्धेति॑ चतुः - धा । \newline
44. व्य॑क्रामन् नक्राम॒न्॒. वि व्य॑क्रामन्न् । \newline
45. अ॒क्रा॒म॒न् न॒ग्नि र॒ग्नि र॑क्रामन् नक्रामन् न॒ग्निः । \newline
46. अ॒ग्निर् वसु॑भि॒र् वसु॑भि र॒ग्नि र॒ग्निर् वसु॑भिः । \newline
47. वसु॑भिः॒ सोमः॒ सोमो॒ वसु॑भि॒र् वसु॑भिः॒ सोमः॑ । \newline
48. वसु॑भि॒रिति॒ वसु॑ - भिः॒ । \newline
49. सोमो॑ रु॒द्रै रु॒द्रैः सोमः॒ सोमो॑ रु॒द्रैः । \newline
50. रु॒द्रै रिन्द्र॒ इन्द्रो॑ रु॒द्रै रु॒द्रै रिन्द्रः॑ । \newline
51. इन्द्रो॑ म॒रुद्भि॑र् म॒रुद्भि॒ रिन्द्र॒ इन्द्रो॑ म॒रुद्भिः॑ । \newline
52. म॒रुद्भि॒र् वरु॑णो॒ वरु॑णो म॒रुद्भि॑र् म॒रुद्भि॒र् वरु॑णः । \newline
53. म॒रुद्भि॒रिति॑ म॒रुत् - भिः॒ । \newline
54. वरु॑ण आदि॒त्यै रा॑दि॒त्यैर् वरु॑णो॒ वरु॑ण आदि॒त्यैः । \newline
55. आ॒दि॒त्यैः स स आ॑दि॒त्यै रा॑दि॒त्यैः सः । \newline
56. स इन्द्र॒ इन्द्रः॒ स स इन्द्रः॑ । \newline
57. इन्द्रः॑ प्र॒जाप॑तिम् प्र॒जाप॑ति॒ मिन्द्र॒ इन्द्रः॑ प्र॒जाप॑तिम् । \newline
58. प्र॒जाप॑ति॒ मुपोप॑ प्र॒जाप॑तिम् प्र॒जाप॑ति॒ मुप॑ । \newline
59. प्र॒जाप॑ति॒मिति॑ प्र॒जा - प॒ति॒म् । \newline
60. उपा॑धाव दधाव॒ दुपोपा॑ धावत् । \newline
61. अ॒धा॒व॒त् तम् त म॑धाव दधाव॒त् तम् । \newline
62. त मे॒तयै॒तया॒ तम् त मे॒तया᳚ । \newline

\textbf{Ghana Paata } \newline

1. भ॒व॒तः॒ प्रि॒यम् प्रि॒यम् भ॑वतो भवतः प्रि॒य मे॒वैव प्रि॒यम् भ॑वतो भवतः प्रि॒य मे॒व । \newline
2. प्रि॒य मे॒वैव प्रि॒यम् प्रि॒य मे॒वैन॑ मेन मे॒व प्रि॒यम् प्रि॒य मे॒वैन᳚म् । \newline
3. ए॒वैन॑ मेन मे॒वैवैनꣳ॑ समा॒नानाꣳ॑ समा॒नाना॑ मेन मे॒वैवैनꣳ॑ समा॒नाना᳚म् । \newline
4. ए॒नꣳ॒॒ स॒मा॒नानाꣳ॑ समा॒नाना॑ मेन मेनꣳ समा॒नाना᳚म् करोति करोति समा॒नाना॑ मेन मेनꣳ समा॒नाना᳚म् करोति । \newline
5. स॒मा॒नाना᳚म् करोति करोति समा॒नानाꣳ॑ समा॒नाना᳚म् करोति द्वि॒पदा᳚ द्वि॒पदा॑ करोति समा॒नानाꣳ॑ समा॒नाना᳚म् करोति द्वि॒पदा᳚ । \newline
6. क॒रो॒ति॒ द्वि॒पदा᳚ द्वि॒पदा॑ करोति करोति द्वि॒पदा॑ पुरोनुवा॒क्या॑ पुरोनुवा॒क्या᳚ द्वि॒पदा॑ करोति करोति द्वि॒पदा॑ पुरोनुवा॒क्या᳚ । \newline
7. द्वि॒पदा॑ पुरोनुवा॒क्या॑ पुरोनुवा॒क्या᳚ द्वि॒पदा᳚ द्वि॒पदा॑ पुरोनुवा॒क्या॑ भवति भवति पुरोनुवा॒क्या᳚ द्वि॒पदा᳚ द्वि॒पदा॑ पुरोनुवा॒क्या॑ भवति । \newline
8. द्वि॒पदेति॑ द्वि - पदा᳚ । \newline
9. पु॒रो॒नु॒वा॒क्या॑ भवति भवति पुरोनुवा॒क्या॑ पुरोनुवा॒क्या॑ भवति द्वि॒पदो᳚ द्वि॒पदो॑ भवति पुरोनुवा॒क्या॑ पुरोनुवा॒क्या॑ भवति द्वि॒पदः॑ । \newline
10. पु॒रो॒नु॒वा॒क्येति॑ पुरः - अ॒नु॒वा॒क्या᳚ । \newline
11. भ॒व॒ति॒ द्वि॒पदो᳚ द्वि॒पदो॑ भवति भवति द्वि॒पद॑ ए॒वैव द्वि॒पदो॑ भवति भवति द्वि॒पद॑ ए॒व । \newline
12. द्वि॒पद॑ ए॒वैव द्वि॒पदो᳚ द्वि॒पद॑ ए॒वावा वै॒व द्वि॒पदो᳚ द्वि॒पद॑ ए॒वाव॑ । \newline
13. द्वि॒पद॒ इति॑ द्वि - पदः॑ । \newline
14. ए॒वावा वै॒वै वाव॑ रुन्धे रु॒न्धे ऽवै॒वै वाव॑ रुन्धे । \newline
15. अव॑ रुन्धे रु॒न्धे ऽवाव॑ रुन्धे॒ चतु॑ष्पदा॒ चतु॑ष्पदा रु॒न्धे ऽवाव॑ रुन्धे॒ चतु॑ष्पदा । \newline
16. रु॒न्धे॒ चतु॑ष्पदा॒ चतु॑ष्पदा रुन्धे रुन्धे॒ चतु॑ष्पदा या॒ज्या॑ या॒ज्या॑ चतु॑ष्पदा रुन्धे रुन्धे॒ चतु॑ष्पदा या॒ज्या᳚ । \newline
17. चतु॑ष्पदा या॒ज्या॑ या॒ज्या॑ चतु॑ष्पदा॒ चतु॑ष्पदा या॒ज्या॑ चतु॑ष्पद॒ श्चतु॑ष्पदो या॒ज्या॑ चतु॑ष्पदा॒ चतु॑ष्पदा या॒ज्या॑ चतु॑ष्पदः । \newline
18. चतु॑ष्प॒देति॒ चतुः॑ - प॒दा॒ । \newline
19. या॒ज्या॑ चतु॑ष्पद॒ श्चतु॑ष्पदो या॒ज्या॑ या॒ज्या॑ चतु॑ष्पद ए॒वैव चतु॑ष्पदो या॒ज्या॑ या॒ज्या॑ चतु॑ष्पद ए॒व । \newline
20. चतु॑ष्पद ए॒वैव चतु॑ष्पद॒ श्चतु॑ष्पद ए॒व प॒शून् प॒शू ने॒व चतु॑ष्पद॒ श्चतु॑ष्पद ए॒व प॒शून् । \newline
21. चतु॑ष्पद॒ इति॒ चतुः॑ - प॒दः॒ । \newline
22. ए॒व प॒शून् प॒शू ने॒वैव प॒शू नवाव॑ प॒शू ने॒वैव प॒शू नव॑ । \newline
23. प॒शू नवाव॑ प॒शून् प॒शू नव॑ रुन्धे रु॒न्धे ऽव॑ प॒शून् प॒शू नव॑ रुन्धे । \newline
24. अव॑ रुन्धे रु॒न्धे ऽवाव॑ रुन्धे देवासु॒रा दे॑वासु॒रा रु॒न्धे ऽवाव॑ रुन्धे देवासु॒राः । \newline
25. रु॒न्धे॒ दे॒वा॒सु॒रा दे॑वासु॒रा रु॑न्धे रुन्धे देवासु॒राः संॅय॑त्ताः॒ संॅय॑त्ता देवासु॒रा रु॑न्धे रुन्धे देवासु॒राः संॅय॑त्ताः । \newline
26. दे॒वा॒सु॒राः संॅय॑त्ताः॒ संॅय॑त्ता देवासु॒रा दे॑वासु॒राः संॅय॑त्ता आसन् नास॒न् थ्संॅय॑त्ता देवासु॒रा दे॑वासु॒राः संॅय॑त्ता आसन्न् । \newline
27. दे॒वा॒सु॒रा इति॑ देव - अ॒सु॒राः । \newline
28. संॅय॑त्ता आसन् नास॒न् थ्संॅय॑त्ताः॒ संॅय॑त्ता आस॒न् ते त आ॑स॒न् थ्संॅय॑त्ताः॒ संॅय॑त्ता आस॒न् ते । \newline
29. संॅय॑त्ता॒ इति॒ सं - य॒त्ताः॒ । \newline
30. आ॒स॒न् ते त आ॑सन् नास॒न् ते दे॒वा दे॒वा स्त आ॑सन् नास॒न् ते दे॒वाः । \newline
31. ते दे॒वा दे॒वा स्ते ते दे॒वा मि॒थो मि॒थो दे॒वा स्ते ते दे॒वा मि॒थः । \newline
32. दे॒वा मि॒थो मि॒थो दे॒वा दे॒वा मि॒थो विप्रि॑या॒ विप्रि॑या मि॒थो दे॒वा दे॒वा मि॒थो विप्रि॑याः । \newline
33. मि॒थो विप्रि॑या॒ विप्रि॑या मि॒थो मि॒थो विप्रि॑या आसन् नास॒न्॒. विप्रि॑या मि॒थो मि॒थो विप्रि॑या आसन्न् । \newline
34. विप्रि॑या आसन् नास॒न्॒. विप्रि॑या॒ विप्रि॑या आस॒न् ते त आ॑स॒न्॒. विप्रि॑या॒ विप्रि॑या आस॒न् ते । \newline
35. विप्रि॑या॒ इति॒ वि - प्रि॒याः॒ । \newline
36. आ॒स॒न् ते त आ॑सन् नास॒न् ते᳚(1॒) ऽन्यो᳚ ऽन्यस्त आ॑सन् नास॒न् ते᳚ ऽन्यः । \newline
37. ते᳚(1॒) ऽन्यो᳚ ऽन्यस्ते ते᳚(1॒) ऽन्यो᳚ ऽन्यस्मा॑ अ॒न्यस्मा॑ अ॒न्यस्ते ते᳚(1॒) ऽन्यो᳚ ऽन्यस्मै᳚ । \newline
38. अ॒न्यो᳚ ऽन्यस्मा॑ अ॒न्यस्मा॑ अ॒न्यो᳚(1॒) ऽन्यो᳚ ऽन्यस्मै॒ ज्यैष्ठ्‍या॑य॒ ज्यैष्ठ्‍या॑या॒न्यस्मा॑ अ॒न्यो᳚(1॒) ऽन्यो᳚ ऽन्यस्मै॒ ज्यैष्ठ्‍या॑य । \newline
39. अ॒न्यस्मै॒ ज्यैष्ठ्‍या॑य॒ ज्यैष्ठ्‍या॑या॒न्यस्मा॑ अ॒न्यस्मै॒ ज्यैष्ठ्‍या॒या ति॑ष्ठमाना॒ अति॑ष्ठमाना॒ ज्यैष्ठ्‍या॑या॒ न्यस्मा॑ अ॒न्यस्मै॒ ज्यैष्ठ्‍या॒या ति॑ष्ठमानाः । \newline
40. ज्यैष्ठ्‍या॒या ति॑ष्ठमाना॒ अति॑ष्ठमाना॒ ज्यैष्ठ्‍या॑य॒ ज्यैष्ठ्या॒या ति॑ष्ठमाना श्चतु॒र्द्धा च॑तु॒र्द्धा ऽति॑ष्ठमाना॒ ज्यैष्ठ्‍या॑य॒ ज्यैष्ठ्‍या॒या ति॑ष्ठमाना श्चतु॒र्द्धा । \newline
41. अति॑ष्ठमाना श्चतु॒र्द्धा च॑तु॒र्द्धा ऽति॑ष्ठमाना॒ अति॑ष्ठमाना श्चतु॒र्द्धा वि वि च॑तु॒र्द्धा ऽति॑ष्ठमाना॒ अति॑ष्ठमाना श्चतु॒र्द्धा वि । \newline
42. च॒तु॒र्द्धा वि वि च॑तु॒र्द्धा च॑तु॒र्द्धा व्य॑क्रामन् नक्राम॒न्॒. वि च॑तु॒र्द्धा च॑तु॒र्द्धा व्य॑क्रामन्न् । \newline
43. च॒तु॒र्द्धेति॑ चतुः - धा । \newline
44. व्य॑क्रामन् नक्राम॒न्॒. वि व्य॑क्रामन् न॒ग्नि र॒ग्नि र॑क्राम॒न्॒. वि व्य॑क्रामन् न॒ग्निः । \newline
45. अ॒क्रा॒म॒न् न॒ग्नि र॒ग्नि र॑क्रामन् नक्रामन् न॒ग्निर् वसु॑भि॒र् वसु॑भि र॒ग्नि र॑क्रामन् नक्रामन् न॒ग्निर् वसु॑भिः । \newline
46. अ॒ग्निर् वसु॑भि॒र् वसु॑भि र॒ग्नि र॒ग्निर् वसु॑भिः॒ सोमः॒ सोमो॒ वसु॑भि र॒ग्नि र॒ग्निर् वसु॑भिः॒ सोमः॑ । \newline
47. वसु॑भिः॒ सोमः॒ सोमो॒ वसु॑भि॒र् वसु॑भिः॒ सोमो॑ रु॒द्रै रु॒द्रैः सोमो॒ वसु॑भि॒र् वसु॑भिः॒ सोमो॑ रु॒द्रैः । \newline
48. वसु॑भि॒रिति॒ वसु॑ - भिः॒ । \newline
49. सोमो॑ रु॒द्रै रु॒द्रैः सोमः॒ सोमो॑ रु॒द्रै रिन्द्र॒ इन्द्रो॑ रु॒द्रैः सोमः॒ सोमो॑ रु॒द्रै रिन्द्रः॑ । \newline
50. रु॒द्रै रिन्द्र॒ इन्द्रो॑ रु॒द्रै रु॒द्रै रिन्द्रो॑ म॒रुद्भि॑र् म॒रुद्भि॒ रिन्द्रो॑ रु॒द्रै रु॒द्रै रिन्द्रो॑ म॒रुद्भिः॑ । \newline
51. इन्द्रो॑ म॒रुद्भि॑र् म॒रुद्भि॒ रिन्द्र॒ इन्द्रो॑ म॒रुद्भि॒र् वरु॑णो॒ वरु॑णो म॒रुद्भि॒ रिन्द्र॒ इन्द्रो॑ म॒रुद्भि॒र् वरु॑णः । \newline
52. म॒रुद्भि॒र् वरु॑णो॒ वरु॑णो म॒रुद्भि॑र् म॒रुद्भि॒र् वरु॑ण आदि॒त्यै रा॑दि॒त्यैर् वरु॑णो म॒रुद्भि॑र् म॒रुद्भि॒र् वरु॑ण आदि॒त्यैः । \newline
53. म॒रुद्भि॒रिति॑ म॒रुत् - भिः॒ । \newline
54. वरु॑ण आदि॒त्यै रा॑दि॒त्यैर् वरु॑णो॒ वरु॑ण आदि॒त्यैः स स आ॑दि॒त्यैर् वरु॑णो॒ वरु॑ण आदि॒त्यैः सः । \newline
55. आ॒दि॒त्यैः स स आ॑दि॒त्यै रा॑दि॒त्यैः स इन्द्र॒ इन्द्रः॒ स आ॑दि॒त्यै रा॑दि॒त्यैः स इन्द्रः॑ । \newline
56. स इन्द्र॒ इन्द्रः॒ स स इन्द्रः॑ प्र॒जाप॑तिम् प्र॒जाप॑ति॒ मिन्द्रः॒ स स इन्द्रः॑ प्र॒जाप॑तिम् । \newline
57. इन्द्रः॑ प्र॒जाप॑तिम् प्र॒जाप॑ति॒ मिन्द्र॒ इन्द्रः॑ प्र॒जाप॑ति॒ मुपोप॑ प्र॒जाप॑ति॒ मिन्द्र॒ इन्द्रः॑ प्र॒जाप॑ति॒ मुप॑ । \newline
58. प्र॒जाप॑ति॒ मुपोप॑ प्र॒जाप॑तिम् प्र॒जाप॑ति॒ मुपा॑धाव दधाव॒दुप॑ प्र॒जाप॑तिम् प्र॒जाप॑ति॒ मुपा॑धावत् । \newline
59. प्र॒जाप॑ति॒मिति॑ प्र॒जा - प॒ति॒म् । \newline
60. उपा॑धाव दधाव॒ दुपोपा॑धाव॒त् तम् त म॑धाव॒ दुपोपा॑ धाव॒त् तम् । \newline
61. अ॒धा॒व॒त् तम् त म॑धाव दधाव॒त् त मे॒तयै॒तया॒ त म॑धाव दधाव॒त् त मे॒तया᳚ । \newline
62. त मे॒तयै॒तया॒ तम् त मे॒तया॑ सं॒(2)ज्ञान्या॑ सं॒(2)ज्ञान्यै॒तया॒ तम् त मे॒तया॑ सं॒(2)ज्ञान्या᳚ । \newline
\pagebreak
\markright{ TS 2.2.11.6  \hfill https://www.vedavms.in \hfill}
\addcontentsline{toc}{section}{ TS 2.2.11.6 }
\section*{ TS 2.2.11.6 }

\textbf{TS 2.2.11.6 } \newline
\textbf{Samhita Paata} \newline

-मे॒तया॑ स॒ज्ञांन्या॑ऽयाजयद॒ग्नये॒ वसु॑मते पुरो॒डाश॑म॒ष्टाक॑पालं॒ निर॑वप॒थ् सोमा॑य रु॒द्रव॑ते च॒रुमिन्द्रा॑य म॒रुत्व॑ते पुरो॒डाश॒ -मेका॑दशकपालं॒ ॅवरु॑णायाऽऽ*दि॒त्यव॑ते च॒रुं ततो॒ वा इन्द्रं॑ दे॒वा ज्यैष्ठ्या॑या॒भि सम॑जानत॒ यः स॑मा॒नैर्मि॒थो विप्रि॑यः॒ स्यात् तमे॒तया॑ स॒ज्ञांन्या॑ याजयेद॒ग्नये॒ वसु॑मते पुरो॒डाश॑म॒ष्टाक॑पालं॒ निर्व॑पे॒थ् सोमा॑य रु॒द्रव॑ते च॒रुमिन्द्रा॑य म॒रुत्व॑ते पुरो॒डाश॒मेका॑दशकपालं॒ ॅवरु॑णाया ( ) ऽऽ*दि॒त्यव॑ते च॒रुमिन्द्र॑मे॒वैनं॑ भू॒तं ज्यैष्ठ्या॑य समा॒ना अ॒भि सं जा॑नते॒ वसि॑ष्ठः समा॒नानां᳚ भवति ॥ \newline

\textbf{Pada Paata} \newline

ए॒तया᳚ । स॒ज्ञांन्येति॑ सं - ज्ञान्या᳚ । अ॒या॒ज॒य॒त् । अ॒ग्नये᳚ । वसु॑मत॒ इति॒ वसु॑ - म॒ते॒ । पु॒रो॒डाश᳚म् । अ॒ष्टाक॑पाल॒मित्य॒ष्टा - क॒पा॒ल॒म् । निरिति॑ । अ॒व॒प॒त् । सोमा॑य । रु॒द्रव॑त॒ इति॑ रु॒द्र - व॒ते॒ । च॒रुम् । इन्द्रा॑य । म॒रुत्व॑ते । पु॒रो॒डाश᳚म् । एका॑दशकपाल॒मित्येका॑दश - क॒पा॒ल॒म् । वरु॑णाय । आ॒दि॒त्यव॑त॒ इत्या॑दि॒त्य - व॒ते॒ । च॒रुम् । ततः॑ । वै । इन्द्र᳚म् । दे॒वाः । ज्यैष्ठ्या॑य । अ॒भि । समिति॑ । अ॒जा॒न॒त॒ । यः । स॒मा॒नैः । मि॒थः । विप्रि॑य॒ इति॒ वि - प्रि॒यः॒ । स्यात् । तम् । ए॒तया᳚ । स॒ज्ञांन्येति॑ सं - ज्ञान्या᳚ । या॒ज॒ये॒त् । अ॒ग्नये᳚ । वसु॑मत॒ इति॒ वसु॑ - म॒ते॒ । पु॒रो॒डाश᳚म् । अ॒ष्टाक॑पाल॒मित्य॒ष्टा - क॒पा॒ल॒म् । निरिति॑ । व॒पे॒त् । सोमा॑य । रु॒द्रव॑त॒ इति॑ रु॒द्र - व॒ते॒ । च॒रुम् । इन्द्रा॑य । म॒रुत्व॑ते । पु॒रो॒डाश᳚म् । एका॑दशकपाल॒मित्येका॑दश - क॒पा॒ल॒म् । वरु॑णाय ( ) । आ॒दि॒त्यव॑त॒ इत्या॑दि॒त्य - व॒ते॒ । च॒रुम् । इन्द्र᳚म् । ए॒व । ए॒न॒म् । भू॒तम् । ज्यैष्ठ्या॑य । स॒मा॒नाः । अ॒भि । समिति॑ । जा॒न॒ते॒ । वसि॑ष्ठः । स॒मा॒नाना᳚म् । भ॒व॒ति॒ ॥  \newline


\textbf{Krama Paata} \newline

ए॒तया॑ सं॒(2)ज्ञान्या᳚ । सं॒(2)ज्ञान्या॑ ऽयाजयत् । सं॒(2)ज्ञान्येति॑ सं. - ज्ञान्या᳚ । अ॒या॒ज॒य॒द॒ग्नये᳚ । अ॒ग्नये॒ वसु॑मते । वसु॑मते पुरो॒डाश᳚म् । वसु॑मत॒ इति॒ वसु॑ - म॒ते॒ । पु॒रो॒डाश॑म॒ष्टाक॑पालम् । अ॒ष्टाक॑पाल॒म् निः । अ॒ष्टाक॑पाल॒मित्य॒ष्टा - क॒पा॒ल॒म् । निर॑वपत् । अ॒व॒प॒थ् सोमा॑य । सोमा॑य रु॒द्रव॑ते । रु॒द्रव॑ते च॒रुम् । रु॒द्रव॑त॒ इति॑ रु॒द्र - व॒ते॒ । च॒रुमिन्द्रा॑य । इन्द्रा॑य म॒रुत्व॑ते । म॒रुत्व॑ते पुरो॒डाश᳚म् । पु॒रो॒डाश॒मेका॑दशकपालम् । एका॑दशकपालं॒ ॅवरु॑णाय । एका॑दशकपाल॒मित्येका॑दश - क॒पा॒ल॒म् । वरु॑णायादि॒त्यव॑ते । आ॒दि॒त्यव॑ते च॒रुम् । आ॒दि॒त्यव॑त॒ इत्या॑दि॒त्य - व॒ते॒ । च॒रुम् ततः॑ । ततो॒ वै । वा इन्द्र᳚म् । इन्द्र॑म् दे॒वाः । दे॒वा ज्यैष्ठ्या॑य । ज्यैष्ठ्या॑या॒भि । अ॒भि सम् । सम॑जानत । अ॒जा॒न॒त॒ यः । यः स॑मा॒नैः । स॒मा॒नैर् मि॒थः । मि॒थो विप्रि॑यः । विप्रि॑यः॒ स्यात् । विप्रि॑य॒ इति॒ वि - प्रि॒यः॒ । स्यात् तम् । तमे॒तया᳚ । ए॒तया॑ 
सं॒(2)ज्ञान्या᳚ । सं॒(2)ज्ञान्या॑ याजयेत् । स॒म्(2)ज्ञान्येति॑ सं - ज्ञान्या᳚ । या॒ज॒ये॒द॒ग्नये᳚ । अ॒ग्नये॒ वसु॑मते । वसु॑मते पुरो॒डाश᳚म् । वसु॑मत॒ इति॒ वसु॑ - म॒ते॒ । पु॒रो॒डाश॑म॒ष्टाक॑पालम् । अ॒ष्टाक॑पाल॒म् निः । अ॒ष्टाक॑पाल॒मित्य॒ष्टा - क॒पा॒ल॒म् । निर् व॑पेत् । व॒पे॒थ् सोमा॑य । सोमा॑य रु॒द्रव॑ते । रु॒द्रव॑ते च॒रुम् । रु॒द्रव॑त॒ इति॑ रु॒द्र - व॒ते॒ । च॒रुमिन्द्रा॑य । इन्द्रा॑य म॒रुत्व॑ते । म॒रुत्व॑ते पुरो॒डाश᳚म् । पु॒रो॒डाश॒मेका॑दशकपालम् । एका॑दशकपालं॒ ॅवरु॑णाय ( ) । एका॑दशकपाल॒मित्येका॑दश - क॒पा॒ल॒म् । वरु॑णायादि॒त्यव॑ते । आ॒दि॒त्यव॑ते च॒रुम् । आ॒दि॒त्यव॑त॒ इत्या॑दि॒त्य - व॒ते॒ । च॒रुमिन्द्र᳚म् । इन्द्र॑मे॒व । ए॒वैन᳚म् । ए॒न॒म् भू॒तम् । भू॒तम् ज्यैष्ठ्या॑य । ज्यैष्ठ्या॑य समा॒नाः । स॒मा॒ना अ॒भि । अ॒भि सम् । सं जा॑नते । जा॒न॒ते॒ वसि॑ष्ठः । वसि॑ष्ठः समा॒नाना᳚म् । स॒मा॒नाना᳚म् भवति । भ॒व॒तीति॑ भवति । \newline

\textbf{Jatai Paata} \newline

1. ए॒तया॑ सं॒(2)ज्ञान्या॑ सं॒(2)ज्ञान् यै॒तयै॒तया॑ सं॒(2)ज्ञान्या᳚ । \newline
2. सं॒(2)ज्ञान्या॑ ऽयाजय दयाजयथ् सं॒(2)ज्ञान्या॑ सं॒(2)ज्ञान्या॑ ऽयाजयत् । \newline
3. सं॒(2)ज्ञान्येति॑ सं - ज्ञान्या᳚ । \newline
4. अ॒या॒ज॒य॒ द॒ग्नये॒ ऽग्नये॑ ऽयाजय दयाजय द॒ग्नये᳚ । \newline
5. अ॒ग्नये॒ वसु॑मते॒ वसु॑मते॒ ऽग्नये॒ ऽग्नये॒ वसु॑मते । \newline
6. वसु॑मते पुरो॒डाश॑म् पुरो॒डाशं॒ ॅवसु॑मते॒ वसु॑मते पुरो॒डाश᳚म् । \newline
7. वसु॑मत॒ इति॒ वसु॑ - म॒ते॒ । \newline
8. पु॒रो॒डाश॑ म॒ष्टाक॑पाल म॒ष्टाक॑पालम् पुरो॒डाश॑म् पुरो॒डाश॑ म॒ष्टाक॑पालम् । \newline
9. अ॒ष्टाक॑पाल॒म् निर् णिर॒ष्टाक॑पाल म॒ष्टाक॑पाल॒म् निः । \newline
10. अ॒ष्टाक॑पाल॒मित्य॒ष्टा - क॒पा॒ल॒म् । \newline
11. निर॑वप दवप॒न् निर् णि र॑वपत् । \newline
12. अ॒व॒प॒थ् सोमा॑य॒ सोमा॑या वपदवप॒थ् सोमा॑य । \newline
13. सोमा॑य रु॒द्रव॑ते रु॒द्रव॑ते॒ सोमा॑य॒ सोमा॑य रु॒द्रव॑ते । \newline
14. रु॒द्रव॑ते च॒रुम् च॒रुꣳ रु॒द्रव॑ते रु॒द्रव॑ते च॒रुम् । \newline
15. रु॒द्रव॑त॒ इति॑ रु॒द्र - व॒ते॒ । \newline
16. च॒रु मिन्द्रा॒ये न्द्रा॑य च॒रुम् च॒रु मिन्द्रा॑य । \newline
17. इन्द्रा॑य म॒रुत्व॑ते म॒रुत्व॑त॒ इन्द्रा॒ये न्द्रा॑य म॒रुत्व॑ते । \newline
18. म॒रुत्व॑ते पुरो॒डाश॑म् पुरो॒डाश॑म् म॒रुत्व॑ते म॒रुत्व॑ते पुरो॒डाश᳚म् । \newline
19. पु॒रो॒डाश॒ मेका॑दशकपाल॒ मेका॑दशकपालम् पुरो॒डाश॑म् पुरो॒डाश॒ मेका॑दशकपालम् । \newline
20. एका॑दशकपालं॒ ॅवरु॑णाय॒ वरु॑णा॒यैका॑दशकपाल॒ मेका॑दशकपालं॒ ॅवरु॑णाय । \newline
21. एका॑दशकपाल॒मित्येका॑दश - क॒पा॒ल॒म् । \newline
22. वरु॑णायादि॒त्यव॑त आदि॒त्यव॑ते॒ वरु॑णाय॒ वरु॑णायादि॒त्यव॑ते । \newline
23. आ॒दि॒त्यव॑ते च॒रुम् च॒रु मा॑दि॒त्यव॑त आदि॒त्यव॑ते च॒रुम् । \newline
24. आ॒दि॒त्यव॑त॒ इत्या॑दि॒त्य - व॒ते॒ । \newline
25. च॒रुम् तत॒ स्तत॑ श्च॒रुम् च॒रुम् ततः॑ । \newline
26. ततो॒ वै वै तत॒ स्ततो॒ वै । \newline
27. वा इन्द्र॒ मिन्द्रं॒ ॅवै वा इन्द्र᳚म् । \newline
28. इन्द्र॑म् दे॒वा दे॒वा इन्द्र॒ मिन्द्र॑म् दे॒वाः । \newline
29. दे॒वा ज्यैष्ठ्‍या॑य॒ ज्यैष्ठ्‍या॑य दे॒वा दे॒वा ज्यैष्ठ्‍या॑य । \newline
30. ज्यैष्ठ्‍या॑या॒ भ्य॑भि ज्यैष्ठ्‍या॑य॒ ज्यैष्ठ्‍या॑या॒भि । \newline
31. अ॒भि सꣳ स म॒भ्य॑भि सम् । \newline
32. स म॑जानता जानत॒ सꣳ स म॑जानत । \newline
33. अ॒जा॒न॒त॒ यो यो॑ ऽजानता जानत॒ यः । \newline
34. यः स॑मा॒नैः स॑मा॒नैर् यो यः स॑मा॒नैः । \newline
35. स॒मा॒नैर् मि॒थो मि॒थः स॑मा॒नैः स॑मा॒नैर् मि॒थः । \newline
36. मि॒थो विप्रि॑यो॒ विप्रि॑यो मि॒थो मि॒थो विप्रि॑यः । \newline
37. विप्रि॑यः॒ स्याथ् स्याद् विप्रि॑यो॒ विप्रि॑यः॒ स्यात् । \newline
38. विप्रि॑य॒ इति॒ वि - प्रि॒यः॒ । \newline
39. स्यात् तम् तꣳ स्याथ् स्यात् तम् । \newline
40. त मे॒तयै॒तया॒ तम् त मे॒तया᳚ । \newline
41. ए॒तया॑ सं॒(2)ज्ञान्या॑ सं॒(2)ज्ञा न्यै॒तयै॒तया॑ सं॒(2)ज्ञान्या᳚ । \newline
42. सं॒(2)ज्ञान्या॑ याजयेद् याजयेथ् सं॒(2)ज्ञान्या॑ सं॒(2)ज्ञान्या॑ याजयेत् । \newline
43. सं॒(2)ज्ञान्येति॑ सं - ज्ञान्या᳚ । \newline
44. या॒ज॒ये॒ द॒ग्नये॒ ऽग्नये॑ याजयेद् याजये द॒ग्नये᳚ । \newline
45. अ॒ग्नये॒ वसु॑मते॒ वसु॑मते॒ ऽग्नये॒ ऽग्नये॒ वसु॑मते । \newline
46. वसु॑मते पुरो॒डाश॑म् पुरो॒डाशं॒ ॅवसु॑मते॒ वसु॑मते पुरो॒डाश᳚म् । \newline
47. वसु॑मत॒ इति॒ वसु॑ - म॒ते॒ । \newline
48. पु॒रो॒डाश॑ म॒ष्टाक॑पाल म॒ष्टाक॑पालम् पुरो॒डाश॑म् पुरो॒डाश॑ म॒ष्टाक॑पालम् । \newline
49. अ॒ष्टाक॑पाल॒म् निर् णिर॒ष्टाक॑पाल म॒ष्टाक॑पाल॒म् निः । \newline
50. अ॒ष्टाक॑पाल॒मित्य॒ष्टा - क॒पा॒ल॒म् । \newline
51. निर् व॑पेद् वपे॒न् निर् णिर् व॑पेत् । \newline
52. व॒पे॒थ् सोमा॑य॒ सोमा॑य वपेद् वपे॒थ् सोमा॑य । \newline
53. सोमा॑य रु॒द्रव॑ते रु॒द्रव॑ते॒ सोमा॑य॒ सोमा॑य रु॒द्रव॑ते । \newline
54. रु॒द्रव॑ते च॒रुम् च॒रुꣳ रु॒द्रव॑ते रु॒द्रव॑ते च॒रुम् । \newline
55. रु॒द्रव॑त॒ इति॑ रु॒द्र - व॒ते॒ । \newline
56. च॒रु मिन्द्रा॒ये न्द्रा॑य च॒रुम् च॒रु मिन्द्रा॑य । \newline
57. इन्द्रा॑य म॒रुत्व॑ते म॒रुत्व॑त॒ इन्द्रा॒ये न्द्रा॑य म॒रुत्व॑ते । \newline
58. म॒रुत्व॑ते पुरो॒डाश॑म् पुरो॒डाश॑म् म॒रुत्व॑ते म॒रुत्व॑ते पुरो॒डाश᳚म् । \newline
59. पु॒रो॒डाश॒ मेका॑दशकपाल॒ मेका॑दशकपालम् पुरो॒डाश॑म् पुरो॒डाश॒ मेका॑दशकपालम् । \newline
60. एका॑दशकपालं॒ ॅवरु॑णाय॒ वरु॑णा॒ यैका॑दशकपाल॒ मेका॑दशकपालं॒ ॅवरु॑णाय । \newline
61. एका॑दशकपाल॒मित्येका॑दश - क॒पा॒ल॒म् । \newline
62. वरु॑णाया दि॒त्यव॑त आदि॒त्यव॑ते॒ वरु॑णाय॒ वरु॑णाया दि॒त्यव॑ते । \newline
63. आ॒दि॒त्यव॑ते च॒रुम् च॒रु मा॑दि॒त्यव॑त आदि॒त्यव॑ते च॒रुम् । \newline
64. आ॒दि॒त्यव॑त॒ इत्या॑दि॒त्य - व॒ते॒ । \newline
65. च॒रु मिन्द्र॒ मिन्द्र॑म् च॒रुम् च॒रु मिन्द्र᳚म् । \newline
66. इन्द्र॑ मे॒वैवे न्द्र॒ मिन्द्र॑ मे॒व । \newline
67. ए॒वैन॑ मेन मे॒वैवैन᳚म् । \newline
68. ए॒न॒म् भू॒तम् भू॒त मे॑न मेनम् भू॒तम् । \newline
69. भू॒तम् ज्यैष्ठ्‍या॑य॒ ज्यैष्ठ्‍या॑य भू॒तम् भू॒तम् ज्यैष्ठ्‍या॑य । \newline
70. ज्यैष्ठ्‍या॑य समा॒नाः स॑मा॒ना ज्यैष्ठ्‍या॑य॒ ज्यैष्ठ्‍या॑य समा॒नाः । \newline
71. स॒मा॒ना अ॒भ्य॑भि स॑मा॒नाः स॑मा॒ना अ॒भि । \newline
72. अ॒भि सꣳ स म॒भ्य॑भि सम् । \newline
73. सम् जा॑नते जानते॒ सꣳ सम् जा॑नते । \newline
74. जा॒न॒ते॒ वसि॑ष्ठो॒ वसि॑ष्ठो जानते जानते॒ वसि॑ष्ठः । \newline
75. वसि॑ष्ठः समा॒नानाꣳ॑ समा॒नानां॒ ॅवसि॑ष्ठो॒ वसि॑ष्ठः समा॒नाना᳚म् । \newline
76. स॒मा॒नाना᳚म् भवति भवति समा॒नानाꣳ॑ समा॒नाना᳚म् भवति । \newline
77. भ॒व॒तीति॑ भवति । \newline

\textbf{Ghana Paata } \newline

1. ए॒तया॑ सं॒(2(ज्ञान्या॑ सं॒(2)ज्ञान्यै॒त यै॒तया॑ सं॒(2)ज्ञान्या॑ ऽयाजय दयाजयथ् सं॒(2)ज्ञान्यै॒त यै॒तया॑ सं॒(2)ज्ञान्या॑ ऽयाजयत् । \newline
2. सं॒(2)ज्ञान्या॑ ऽयाजय दयाजयथ् सं॒(2)ज्ञान्या॑ सं॒(2)ज्ञान्या॑ ऽयाजय द॒ग्नये॒ ऽग्नये॑ ऽयाजयथ् सं॒(2(ज्ञान्या॑ 
सं॒(2)ज्ञान्या॑ ऽयाजय द॒ग्नये᳚ । \newline
3. सं॒(2)ज्ञान्येति॑ सं - ज्ञान्या᳚ । \newline
4. अ॒या॒ज॒य॒ द॒ग्नये॒ ऽग्नये॑ ऽयाजय दयाजय द॒ग्नये॒ वसु॑मते॒ वसु॑मते॒ ऽग्नये॑ ऽयाजय दयाजय द॒ग्नये॒ वसु॑मते । \newline
5. अ॒ग्नये॒ वसु॑मते॒ वसु॑मते॒ ऽग्नये॒ ऽग्नये॒ वसु॑मते पुरो॒डाश॑म् पुरो॒डाशं॒ ॅवसु॑मते॒ ऽग्नये॒ ऽग्नये॒ वसु॑मते पुरो॒डाश᳚म् । \newline
6. वसु॑मते पुरो॒डाश॑म् पुरो॒डाशं॒ ॅवसु॑मते॒ वसु॑मते पुरो॒डाश॑ म॒ष्टाक॑पाल म॒ष्टाक॑पालम् पुरो॒डाशं॒ ॅवसु॑मते॒ वसु॑मते पुरो॒डाश॑ म॒ष्टाक॑पालम् । \newline
7. वसु॑मत॒ इति॒ वसु॑ - म॒ते॒ । \newline
8. पु॒रो॒डाश॑ म॒ष्टाक॑पाल म॒ष्टाक॑पालम् पुरो॒डाश॑म् पुरो॒डाश॑ म॒ष्टाक॑पाल॒म् निर् णिर॒ष्टाक॑पालम् पुरो॒डाश॑म् पुरो॒डाश॑ म॒ष्टाक॑पाल॒म् निः । \newline
9. अ॒ष्टाक॑पाल॒म् निर् णिर॒ष्टाक॑पाल म॒ष्टाक॑पाल॒म् निर॑वप दवप॒न् निर॒ष्टाक॑पाल म॒ष्टाक॑पाल॒म् निर॑वपत् । \newline
10. अ॒ष्टाक॑पाल॒मित्य॒ष्टा - क॒पा॒ल॒म् । \newline
11. निर॑वप दवप॒न् निर् णिर॑वप॒थ् सोमा॑य॒ सोमा॑यावप॒न् निर् णिर॑वप॒थ् सोमा॑य । \newline
12. अ॒व॒प॒थ् सोमा॑य॒ सोमा॑या वप दवप॒थ् सोमा॑य रु॒द्रव॑ते रु॒द्रव॑ते॒ सोमा॑या वप दवप॒थ् सोमा॑य रु॒द्रव॑ते । \newline
13. सोमा॑य रु॒द्रव॑ते रु॒द्रव॑ते॒ सोमा॑य॒ सोमा॑य रु॒द्रव॑ते च॒रुम् च॒रुꣳ रु॒द्रव॑ते॒ सोमा॑य॒ सोमा॑य रु॒द्रव॑ते च॒रुम् । \newline
14. रु॒द्रव॑ते च॒रुम् च॒रुꣳ रु॒द्रव॑ते रु॒द्रव॑ते च॒रु मिन्द्रा॒ये न्द्रा॑य च॒रुꣳ रु॒द्रव॑ते रु॒द्रव॑ते च॒रु मिन्द्रा॑य । \newline
15. रु॒द्रव॑त॒ इति॑ रु॒द्र - व॒ते॒ । \newline
16. च॒रु मिन्द्रा॒ये न्द्रा॑य च॒रुम् च॒रु मिन्द्रा॑य म॒रुत्व॑ते म॒रुत्व॑त॒ इन्द्रा॑य च॒रुम् च॒रु मिन्द्रा॑य म॒रुत्व॑ते । \newline
17. इन्द्रा॑य म॒रुत्व॑ते म॒रुत्व॑त॒ इन्द्रा॒ये न्द्रा॑य म॒रुत्व॑ते पुरो॒डाश॑म् पुरो॒डाश॑म् म॒रुत्व॑त॒ इन्द्रा॒ये न्द्रा॑य म॒रुत्व॑ते पुरो॒डाश᳚म् । \newline
18. म॒रुत्व॑ते पुरो॒डाश॑म् पुरो॒डाश॑म् म॒रुत्व॑ते म॒रुत्व॑ते पुरो॒डाश॒ मेका॑दशकपाल॒ मेका॑दशकपालम् पुरो॒डाश॑म् म॒रुत्व॑ते म॒रुत्व॑ते पुरो॒डाश॒ मेका॑दशकपालम् । \newline
19. पु॒रो॒डाश॒ मेका॑दशकपाल॒ मेका॑दशकपालम् पुरो॒डाश॑म् पुरो॒डाश॒ मेका॑दशकपालं॒ ॅवरु॑णाय॒ वरु॑णा॒ यैका॑दशकपालम् पुरो॒डाश॑म् पुरो॒डाश॒ मेका॑दशकपालं॒ ॅवरु॑णाय । \newline
20. एका॑दशकपालं॒ ॅवरु॑णाय॒ वरु॑णा॒ यैका॑दशकपाल॒ मेका॑दशकपालं॒ ॅवरु॑णाया दि॒त्यव॑त आदि॒त्यव॑ते॒ वरु॑णा॒ यैका॑दशकपाल॒ मेका॑दशकपालं॒ ॅवरु॑णाया दि॒त्यव॑ते । \newline
21. एका॑दशकपाल॒मित्येका॑दश - क॒पा॒ल॒म् । \newline
22. वरु॑णाया दि॒त्यव॑त आदि॒त्यव॑ते॒ वरु॑णाय॒ वरु॑णाया दि॒त्यव॑ते च॒रुम् च॒रु मा॑दि॒त्यव॑ते॒ वरु॑णाय॒ वरु॑णाया दि॒त्यव॑ते च॒रुम् । \newline
23. आ॒दि॒त्यव॑ते च॒रुम् च॒रु मा॑दि॒त्यव॑त आदि॒त्यव॑ते च॒रुम् तत॒ स्तत॑ श्च॒रु मा॑दि॒त्यव॑त आदि॒त्यव॑ते च॒रुम् ततः॑ । \newline
24. आ॒दि॒त्यव॑त॒ इत्या॑दि॒त्य - व॒ते॒ । \newline
25. च॒रुम् तत॒ स्तत॑ श्च॒रुम् च॒रुम् ततो॒ वै वै तत॑ श्च॒रुम् च॒रुम् ततो॒ वै । \newline
26. ततो॒ वै वै तत॒ स्ततो॒ वा इन्द्र॒ मिन्द्रं॒ ॅवै तत॒ स्ततो॒ वा इन्द्र᳚म् । \newline
27. वा इन्द्र॒ मिन्द्रं॒ ॅवै वा इन्द्र॑म् दे॒वा दे॒वा इन्द्रं॒ ॅवै वा इन्द्र॑म् दे॒वाः । \newline
28. इन्द्र॑म् दे॒वा दे॒वा इन्द्र॒ मिन्द्र॑म् दे॒वा ज्यैष्ठ्‍या॑य॒ ज्यैष्ठ्‍या॑य दे॒वा इन्द्र॒ मिन्द्र॑म् दे॒वा ज्यैष्ठ्‍या॑य । \newline
29. दे॒वा ज्यैष्ठ्‍या॑य॒ ज्यैष्ठ्‍या॑य दे॒वा दे॒वा ज्यैष्ठ्‍या॑या॒भ्य॑भि ज्यैष्ठ्‍या॑य दे॒वा दे॒वा ज्यैष्ठ्‍या॑या॒भि । \newline
30. ज्यैष्ठ्‍या॑या॒भ्य॑भि ज्यैष्ठ्‍या॑य॒ ज्यैष्ठ्‍या॑या॒भि सꣳ स म॒भि ज्यैष्ठ्‍या॑य॒ ज्यैष्ठ्‍या॑या॒भि सम् । \newline
31. अ॒भि सꣳ स म॒भ्य॑भि स म॑जानता जानत॒ स म॒भ्य॑भि स म॑जानत । \newline
32. स म॑जानता जानत॒ सꣳ स म॑जानत॒ यो यो॑ ऽजानत॒ सꣳ स म॑जानत॒ यः । \newline
33. अ॒जा॒न॒त॒ यो यो॑ ऽजानता जानत॒ यः स॑मा॒नैः स॑मा॒नैर् यो॑ ऽजानता जानत॒ यः स॑मा॒नैः । \newline
34. यः स॑मा॒नैः स॑मा॒नैर् यो यः स॑मा॒नैर् मि॒थो मि॒थः स॑मा॒नैर् यो यः स॑मा॒नैर् मि॒थः । \newline
35. स॒मा॒नैर् मि॒थो मि॒थः स॑मा॒नैः स॑मा॒नैर् मि॒थो विप्रि॑यो॒ विप्रि॑यो मि॒थः स॑मा॒नैः स॑मा॒नैर् मि॒थो विप्रि॑यः । \newline
36. मि॒थो विप्रि॑यो॒ विप्रि॑यो मि॒थो मि॒थो विप्रि॑यः॒ स्याथ् स्याद् विप्रि॑यो मि॒थो मि॒थो विप्रि॑यः॒ स्यात् । \newline
37. विप्रि॑यः॒ स्याथ् स्याद् विप्रि॑यो॒ विप्रि॑यः॒ स्यात् तम् तꣳ स्याद् विप्रि॑यो॒ विप्रि॑यः॒ स्यात् तम् । \newline
38. विप्रि॑य॒ इति॒ वि - प्रि॒यः॒ । \newline
39. स्यात् तम् तꣳ स्याथ् स्यात् त मे॒तयै॒तया॒ तꣳ स्याथ् स्यात् त मे॒तया᳚ । \newline
40. त मे॒तयै॒तया॒ तम् त मे॒तया॑ सं॒(2)ज्ञान्या॑ सं॒(2)ज्ञान्यै॒तया॒ तम् त मे॒तया॑ सं॒(2)ज्ञान्या᳚ । \newline
41. ए॒तया॑ सं॒(2)ज्ञान्या॑ सं॒(2)ज्ञान्यै॒त यै॒तया॑ सं॒(2)ज्ञान्या॑ याजयेद् याजयेथ् सं॒(2)ज्ञान्यै॒त यै॒तया॑ सं॒(2)ज्ञान्या॑ याजयेत् । \newline
42. सं॒(2)ज्ञान्या॑ याजयेद् याजयेथ् सं॒(2)ज्ञान्या॑ सं॒(2)ज्ञान्या॑ याजये द॒ग्नये॒ ऽग्नये॑ याजयेथ् सं॒(2)ज्ञान्या॑ 
सं॒(2)ज्ञान्या॑ याजये द॒ग्नये᳚ । \newline
43. सं॒(2)ज्ञान्येति॑ सं - ज्ञान्या᳚ । \newline
44. या॒ज॒ये॒ द॒ग्नये॒ ऽग्नये॑ याजयेद् याजये द॒ग्नये॒ वसु॑मते॒ वसु॑मते॒ ऽग्नये॑ याजयेद् याजये द॒ग्नये॒ वसु॑मते । \newline
45. अ॒ग्नये॒ वसु॑मते॒ वसु॑मते॒ ऽग्नये॒ ऽग्नये॒ वसु॑मते पुरो॒डाश॑म् पुरो॒डाशं॒ ॅवसु॑मते॒ ऽग्नये॒ ऽग्नये॒ वसु॑मते पुरो॒डाश᳚म् । \newline
46. वसु॑मते पुरो॒डाश॑म् पुरो॒डाशं॒ ॅवसु॑मते॒ वसु॑मते पुरो॒डाश॑ म॒ष्टाक॑पाल म॒ष्टाक॑पालम् पुरो॒डाशं॒ ॅवसु॑मते॒ वसु॑मते पुरो॒डाश॑ म॒ष्टाक॑पालम् । \newline
47. वसु॑मत॒ इति॒ वसु॑ - म॒ते॒ । \newline
48. पु॒रो॒डाश॑ म॒ष्टाक॑पाल म॒ष्टाक॑पालम् पुरो॒डाश॑म् पुरो॒डाश॑ म॒ष्टाक॑पाल॒म् निर् णिर॒ष्टाक॑पालम् पुरो॒डाश॑म् पुरो॒डाश॑ म॒ष्टाक॑पाल॒म् निः । \newline
49. अ॒ष्टाक॑पाल॒म् निर् णिर॒ष्टाक॑पाल म॒ष्टाक॑पाल॒म् निर् व॑पेद् वपे॒न् निर॒ष्टाक॑पाल म॒ष्टाक॑पाल॒म् निर् व॑पेत् । \newline
50. अ॒ष्टाक॑पाल॒मित्य॒ष्टा - क॒पा॒ल॒म् । \newline
51. निर् व॑पेद् वपे॒न् निर् णिर् व॑पे॒थ् सोमा॑य॒ सोमा॑य वपे॒न् निर् णिर् व॑पे॒थ् सोमा॑य । \newline
52. व॒पे॒थ् सोमा॑य॒ सोमा॑य वपेद् वपे॒थ् सोमा॑य रु॒द्रव॑ते रु॒द्रव॑ते॒ सोमा॑य वपेद् वपे॒थ् सोमा॑य रु॒द्रव॑ते । \newline
53. सोमा॑य रु॒द्रव॑ते रु॒द्रव॑ते॒ सोमा॑य॒ सोमा॑य रु॒द्रव॑ते च॒रुम् च॒रुꣳ रु॒द्रव॑ते॒ सोमा॑य॒ सोमा॑य रु॒द्रव॑ते च॒रुम् । \newline
54. रु॒द्रव॑ते च॒रुम् च॒रुꣳ रु॒द्रव॑ते रु॒द्रव॑ते च॒रु मिन्द्रा॒ये न्द्रा॑य च॒रुꣳ रु॒द्रव॑ते रु॒द्रव॑ते च॒रु मिन्द्रा॑य । \newline
55. रु॒द्रव॑त॒ इति॑ रु॒द्र - व॒ते॒ । \newline
56. च॒रु मिन्द्रा॒ये न्द्रा॑य च॒रुम् च॒रु मिन्द्रा॑य म॒रुत्व॑ते म॒रुत्व॑त॒ इन्द्रा॑य च॒रुम् च॒रु मिन्द्रा॑य म॒रुत्व॑ते । \newline
57. इन्द्रा॑य म॒रुत्व॑ते म॒रुत्व॑त॒ इन्द्रा॒ये न्द्रा॑य म॒रुत्व॑ते पुरो॒डाश॑म् पुरो॒डाश॑म् म॒रुत्व॑त॒ इन्द्रा॒ये न्द्रा॑य म॒रुत्व॑ते पुरो॒डाश᳚म् । \newline
58. म॒रुत्व॑ते पुरो॒डाश॑म् पुरो॒डाश॑म् म॒रुत्व॑ते म॒रुत्व॑ते पुरो॒डाश॒ मेका॑दशकपाल॒ मेका॑दशकपालम् पुरो॒डाश॑म् म॒रुत्व॑ते म॒रुत्व॑ते पुरो॒डाश॒ मेका॑दशकपालम् । \newline
59. पु॒रो॒डाश॒ मेका॑दशकपाल॒ मेका॑दशकपालम् पुरो॒डाश॑म् पुरो॒डाश॒ मेका॑दशकपालं॒ ॅवरु॑णाय॒ वरु॑णा॒ यैका॑दशकपालम् पुरो॒डाश॑म् पुरो॒डाश॒ मेका॑दशकपालं॒ ॅवरु॑णाय । \newline
60. एका॑दशकपालं॒ ॅवरु॑णाय॒ वरु॑णा॒ यैका॑दशकपाल॒ मेका॑दशकपालं॒ ॅवरु॑णायादि॒त्यव॑त आदि॒त्यव॑ते॒ वरु॑णा॒ यैका॑दशकपाल॒ मेका॑दशकपालं॒ ॅवरु॑णायादि॒त्यव॑ते । \newline
61. एका॑दशकपाल॒मित्येका॑दश - क॒पा॒ल॒म् । \newline
62. वरु॑णाया दि॒त्यव॑त आदि॒त्यव॑ते॒ वरु॑णाय॒ वरु॑णाया दि॒त्यव॑ते च॒रुम् च॒रु मा॑दि॒त्यव॑ते॒ वरु॑णाय॒ वरु॑णाया दि॒त्यव॑ते च॒रुम् । \newline
63. आ॒दि॒त्यव॑ते च॒रुम् च॒रु मा॑दि॒त्यव॑त आदि॒त्यव॑ते च॒रु मिन्द्र॒ मिन्द्र॑म् च॒रु मा॑दि॒त्यव॑त आदि॒त्यव॑ते च॒रु मिन्द्र᳚म् । \newline
64. आ॒दि॒त्यव॑त॒ इत्या॑दि॒त्य - व॒ते॒ । \newline
65. च॒रु मिन्द्र॒ मिन्द्र॑म् च॒रुम् च॒रु मिन्द्र॑ मे॒वैवे न्द्र॑म् च॒रुम् च॒रु मिन्द्र॑ मे॒व । \newline
66. इन्द्र॑ मे॒वैवे न्द्र॒ मिन्द्र॑ मे॒वैन॑ मेन मे॒वे न्द्र॒ मिन्द्र॑ मे॒वैन᳚म् । \newline
67. ए॒वैन॑ मेन मे॒वैवैन॑म् भू॒तम् भू॒त मे॑न मे॒वैवैन॑म् भू॒तम् । \newline
68. ए॒न॒म् भू॒तम् भू॒त मे॑न मेनम् भू॒तम् ज्यैष्ठ्‍या॑य॒ ज्यैष्ठ्‍या॑य भू॒त मे॑न मेनम् भू॒तम् ज्यैष्ठ्‍या॑य । \newline
69. भू॒तम् ज्यैष्ठ्‍या॑य॒ ज्यैष्ठ्‍या॑य भू॒तम् भू॒तम् ज्यैष्ठ्‍या॑य समा॒नाः स॑मा॒ना ज्यैष्ठ्‍या॑य भू॒तम् भू॒तम् ज्यैष्ठ्‍या॑य समा॒नाः । \newline
70. ज्यैष्ठ्‍या॑य समा॒नाः स॑मा॒ना ज्यैष्ठ्‍या॑य॒ ज्यैष्ठ्‍या॑य समा॒ना अ॒भ्य॑भि स॑मा॒ना ज्यैष्ठ्‍या॑य॒ ज्यैष्ठ्‍या॑य समा॒ना अ॒भि । \newline
71. स॒मा॒ना अ॒भ्य॑भि स॑मा॒नाः स॑मा॒ना अ॒भि सꣳ स म॒भि स॑मा॒नाः स॑मा॒ना अ॒भि सम् । \newline
72. अ॒भि सꣳ स म॒भ्य॑भि सम् जा॑नते जानते॒ स म॒भ्य॑भि सम् जा॑नते । \newline
73. सम् जा॑नते जानते॒ सꣳ सम् जा॑नते॒ वसि॑ष्ठो॒ वसि॑ष्ठो जानते॒ सꣳ सम् जा॑नते॒ वसि॑ष्ठः । \newline
74. जा॒न॒ते॒ वसि॑ष्ठो॒ वसि॑ष्ठो जानते जानते॒ वसि॑ष्ठः समा॒नानाꣳ॑ समा॒नानां॒ ॅवसि॑ष्ठो जानते जानते॒ वसि॑ष्ठः समा॒नाना᳚म् । \newline
75. वसि॑ष्ठः समा॒नानाꣳ॑ समा॒नानां॒ ॅवसि॑ष्ठो॒ वसि॑ष्ठः समा॒नाना᳚म् भवति भवति समा॒नानां॒ ॅवसि॑ष्ठो॒ वसि॑ष्ठः समा॒नाना᳚म् भवति । \newline
76. स॒मा॒नाना᳚म् भवति भवति समा॒नानाꣳ॑ समा॒नाना᳚म् भवति । \newline
77. भ॒व॒तीति॑ भवति । \newline
\pagebreak
\markright{ TS 2.2.12.1  \hfill https://www.vedavms.in \hfill}
\addcontentsline{toc}{section}{ TS 2.2.12.1 }
\section*{ TS 2.2.12.1 }

\textbf{TS 2.2.12.1 } \newline
\textbf{Samhita Paata} \newline

हि॒र॒ण्य॒ग॒र्भ >1, आपो॑ ह॒ यत् >2, प्रजा॑पते >3 ॥ स वे॑द पु॒त्रः पि॒तरꣳ॒॒ स मा॒तरꣳ॒॒ स सू॒नुभ॑र्व॒थ् स भु॑व॒त् पुन॑र्मघः । स द्यामौर्णो॑द॒न्तरि॑क्षꣳ॒॒ स सुवः॒ स विश्वा॒ भुवो॑ अभव॒थ् स आ*ऽभ॑वत् ॥ उदु॒ त्यं >4, चि॒त्रं >5 ॥ स प्र॑त्न॒वन्नवी॑य॒साऽ*ग्ने᳚ द्यु॒म्नेन॑ सं॒ॅयता᳚ । बृ॒हत् त॑तन्थ भा॒नुना᳚ ॥ नि काव्या॑ वे॒धसः॒ शश्व॑तस्क॒र्॒.हस्ते॒ दधा॑नो॒ - [  ] \newline

\textbf{Pada Paata} \newline

हि॒र॒ण्य॒ग॒र्भ इति॑ हिरण्य - ग॒र्भः । आपः॑ । ह॒ । यत् । प्रजा॑पत॒ इति॒ प्रजा᳚ - प॒ते॒ ॥ सः । वे॒द॒ । पु॒त्रः । पि॒तर᳚म् । सः । मा॒तर᳚म् । सः । सू॒नुः । भु॒व॒त् । सः । भु॒व॒त् । पुन॑र्मघ॒ इति॒ पुनः॑ - म॒घः॒ ॥ सः । द्याम् । और्णो᳚त् । अ॒न्तरि॑क्षम् । सः । सुवः॑ । सः । विश्वाः᳚ । भुवः॑ । अ॒भ॒व॒त् । सः । एति॑ । अ॒भ॒व॒त् ॥ उदिति॑ । उ॒ । त्यम् । चि॒त्रम् ॥ सः । प्र॒त्न॒वदिति॑ प्रत्न-वत् । नवी॑यसा । अग्ने᳚ । द्यु॒म्नेन॑ । सं॒ॅयतेति॑ सं - यता᳚ ॥ बृ॒हत् । त॒त॒न्थ॒ । भा॒नुना᳚ ॥ नीति॑ । काव्या᳚ । वे॒धसः॑ । शश्व॑तः । कः॒ । हस्ते᳚ । दधा॑नः ।  \newline


\textbf{Krama Paata} \newline

हि॒र॒ण्य॒ग॒र्भ आपः॑ । हि॒र॒ण्य॒ग॒र्भ इति॑ हिरण्य - ग॒र्भः । आपो॑ ह । ह॒ यत् । यत् प्रजा॑पते । प्रजा॑पत॒ इति॒ प्रजा᳚ - प॒ते॒ । स वे॑द । वे॒द॒ पु॒त्रः । पु॒त्रः पि॒तर᳚म् । पि॒तरꣳ॒॒ सः । स मा॒तर᳚म् । मा॒तरꣳ॒॒ सः । स सू॒नुः । सू॒नुर् भु॑वत् । भु॒व॒थ् सः । स भु॑वत् । भु॒व॒त् पुन॑र्मघः । पुन॑र्मघ॒ इति॒ पुनः॑ - म॒घः॒ ॥ स द्याम् । द्यामौर्णो᳚त् । और्णो॑द॒न्तरि॑क्षम् । अ॒न्तरि॑क्षꣳ॒॒ सः । स सुवः॑ । सुवः॒ सः । स विश्वाः᳚ । विश्वा॒ भुवः॑ । भुवो॑ अभवत् । अ॒भ॒व॒थ् सः । स आ । आऽभ॑वत् । अ॒भ॒व॒दित्य॑भवत् ॥ उदु॑ । उ॒त्यम् । त्यम् चि॒त्रम् । चि॒त्रमिति॑ चि॒त्रम् ॥ स प्र॑त्न॒वत् । प्र॒त्न॒वन् नवी॑यसा । प्र॒त्न॒वदिति॑ प्रत्न - वत् । नवी॑य॒साऽग्ने᳚ । अग्ने᳚ द्यु॒म्नेन॑ । द्यु॒म्नेन॑ स॒म्ॅयता᳚ । स॒म्ॅयतेति॑ सं - यता᳚ । बृ॒हत् त॑तन्थ । त॒त॒न्थ॒ भा॒नुना᳚ । भा॒नुनेति॑ भा॒नुना᳚ ॥ नि काव्या᳚ । काव्या॑ वे॒धसः॑ । वे॒धसः॒ शश्व॑तः । शश्व॑तः स्कः । क॒र्॒.हस्ते᳚ । हस्ते॒ दधा॑नः । दधा॑नो॒ नर्याः᳚ \newline

\textbf{Jatai Paata} \newline

1. हि॒र॒ण्य॒ग॒र्भ आप॒ आपो॑ हिरण्यग॒र्भो हि॑रण्यग॒र्भ आपः॑ । \newline
2. हि॒र॒ण्य॒ग॒र्भ इति॑ हिरण्य - ग॒र्भः । \newline
3. आपो॑ ह॒ हाप॒ आपो॑ ह । \newline
4. ह॒ यद् यद्ध॑ ह॒ यत् । \newline
5. यत् प्रजा॑पते॒ प्रजा॑पते॒ यद् यत् प्रजा॑पते । \newline
6. प्रजा॑पत॒ इति॒ प्रजा᳚ - प॒ते॒ । \newline
7. स वे॑द वेद॒ स स वे॑द । \newline
8. वे॒द॒ पु॒त्रः पु॒त्रो वे॑द वेद पु॒त्रः । \newline
9. पु॒त्रः पि॒तर॑म् पि॒तर॑म् पु॒त्रः पु॒त्रः पि॒तर᳚म् । \newline
10. पि॒तरꣳ॒॒ स स पि॒तर॑म् पि॒तरꣳ॒॒ सः । \newline
11. स मा॒तर॑म् मा॒तरꣳ॒॒ स स मा॒तर᳚म् । \newline
12. मा॒तरꣳ॒॒ स स मा॒तर॑म् मा॒तरꣳ॒॒ सः । \newline
13. स सू॒नुः सू॒नुः स स सू॒नुः । \newline
14. सू॒नुर् भु॑वद् भुवथ् सू॒नुः सू॒नुर् भु॑वत् । \newline
15. भु॒व॒थ् स स भु॑वद् भुव॒थ् सः । \newline
16. स भु॑वद् भुव॒थ् स स भु॑वत् । \newline
17. भु॒व॒त् पुन॑र्मघः॒ पुन॑र्मघो भुवद् भुव॒त् पुन॑र्मघः । \newline
18. पुन॑र्मघ॒ इति॒ पुनः॑ - म॒घः॒ । \newline
19. स द्याम् द्याꣳ स स द्याम् । \newline
20. द्या मौर्णो॒ दौर्णो॒द् द्याम् द्या मौर्णो᳚त् । \newline
21. और्णो॑ द॒न्तरि॑क्ष म॒न्तरि॑क्ष॒ मौर्णो॒ दौर्णो॑ द॒न्तरि॑क्षम् । \newline
22. अ॒न्तरि॑क्षꣳ॒॒ स सो अ॒न्तरि॑क्ष म॒न्तरि॑क्षꣳ॒॒ सः । \newline
23. स सुवः॒ सुवः॒ स स सुवः॑ । \newline
24. सुवः॒ स स सुवः॒ सुवः॒ सः । \newline
25. स विश्वा॒ विश्वाः॒ स स विश्वाः᳚ । \newline
26. विश्वा॒ भुवो॒ भुवो॒ विश्वा॒ विश्वा॒ भुवः॑ । \newline
27. भुवो॑ अभव दभव॒द् भुवो॒ भुवो॑ अभवत् । \newline
28. अ॒भ॒व॒थ् स सो अ॑भव दभव॒थ् सः । \newline
29. स आ स स आ । \newline
30. आ ऽभ॑व दभव॒दा ऽभ॑वत् । \newline
31. अ॒भ॒व॒दित्य॑भवत् । \newline
32. उदु॑ वु॒ वुदुदु॑ । \newline
33. उ॒ त्यम् त्य मु॑ वु॒ त्यम् । \newline
34. त्यम् चि॒त्रम् चि॒त्रम् त्यम् त्यम् चि॒त्रम् । \newline
35. चि॒त्रमिति॑ चि॒त्रम् । \newline
36. स प्र॑त्न॒वत् प्र॑त्न॒वथ् स स प्र॑त्न॒वत् । \newline
37. प्र॒त्न॒वन् नवी॑यसा॒ नवी॑यसा प्रत्न॒वत् प्र॑त्न॒वन् नवी॑यसा । \newline
38. प्र॒त्न॒वदिति॑ प्रत्न - वत् । \newline
39. नवी॑य॒सा ऽग्ने ऽग्ने॒ नवी॑यसा॒ नवी॑य॒सा ऽग्ने᳚ । \newline
40. अग्ने᳚ द्यु॒म्नेन॑ द्यु॒म्नेनाग्ने ऽग्ने᳚ द्यु॒म्नेन॑ । \newline
41. द्यु॒म्नेन॑ सं॒ॅयता॑ सं॒ॅयता᳚ द्यु॒म्नेन॑ द्यु॒म्नेन॑ सं॒ॅयता᳚ । \newline
42. सं॒ॅयतेति॑ सं - यता᳚ । \newline
43. बृ॒हत् त॑तन्थ ततन्थ बृ॒हद् बृ॒हत् त॑तन्थ । \newline
44. त॒त॒न्थ॒ भा॒नुना॑ भा॒नुना॑ ततन्थ ततन्थ भा॒नुना᳚ । \newline
45. भा॒नुनेति॑ भा॒नुना᳚ । \newline
46. नि काव्या॒ काव्या॒ नि नि काव्या᳚ । \newline
47. काव्या॑ वे॒धसो॑ वे॒धसः॒ काव्या॒ काव्या॑ वे॒धसः॑ । \newline
48. वे॒धसः॒ शश्व॑तः॒ शश्व॑तो वे॒धसो॑ वे॒धसः॒ शश्व॑तः । \newline
49. शश्व॑तः कः कः॒ शश्व॑तः॒ शश्व॑तः कः । \newline
50. क॒र्॒. हस्ते॒ हस्ते॑ कः क॒र्॒. हस्ते᳚ । \newline
51. हस्ते॒ दधा॑नो॒ दधा॑नो॒ हस्ते॒ हस्ते॒ दधा॑नः । \newline
52. दधा॑नो॒ नर्या॒ नर्या॒ दधा॑नो॒ दधा॑नो॒ नर्या᳚ । \newline

\textbf{Ghana Paata } \newline

1. हि॒र॒ण्य॒ग॒र्भ आप॒ आपो॑ हिरण्यग॒र्भो हि॑रण्यग॒र्भ आपो॑ ह॒ हापो॑ हिरण्यग॒र्भो हि॑रण्यग॒र्भ आपो॑ ह । \newline
2. हि॒र॒ण्य॒ग॒र्भ इति॑ हिरण्य - ग॒र्भः । \newline
3. आपो॑ ह॒ हाप॒ आपो॑ ह॒ यद् यद्धाप॒ आपो॑ ह॒ यत् । \newline
4. ह॒ यद् यद्ध॑ ह॒ यत् प्रजा॑पते॒ प्रजा॑पते॒ यद्ध॑ ह॒ यत् प्रजा॑पते । \newline
5. यत् प्रजा॑पते॒ प्रजा॑पते॒ यद् यत् प्रजा॑पते । \newline
6. प्रजा॑पत॒ इति॒ प्रजा᳚ - प॒ते॒ । \newline
7. स वे॑द वेद॒ स स वे॑द पु॒त्रः पु॒त्रो वे॑द॒ स स वे॑द पु॒त्रः । \newline
8. वे॒द॒ पु॒त्रः पु॒त्रो वे॑द वेद पु॒त्रः पि॒तर॑म् पि॒तर॑म् पु॒त्रो वे॑द वेद पु॒त्रः पि॒तर᳚म् । \newline
9. पु॒त्रः पि॒तर॑म् पि॒तर॑म् पु॒त्रः पु॒त्रः पि॒तरꣳ॒॒ स स पि॒तर॑म् पु॒त्रः पु॒त्रः पि॒तरꣳ॒॒ सः । \newline
10. पि॒तरꣳ॒॒ स स पि॒तर॑म् पि॒तरꣳ॒॒ स मा॒तर॑म् मा॒तरꣳ॒॒ स पि॒तर॑म् पि॒तरꣳ॒॒ स मा॒तर᳚म् । \newline
11. स मा॒तर॑म् मा॒तरꣳ॒॒ स स मा॒तरꣳ॒॒ स स मा॒तरꣳ॒॒ स स मा॒तरꣳ॒॒ सः । \newline
12. मा॒तरꣳ॒॒ स स मा॒तर॑म् मा॒तरꣳ॒॒ स सू॒नुः सू॒नुः स मा॒तर॑म् मा॒तरꣳ॒॒ स सू॒नुः । \newline
13. स सू॒नुः सू॒नुः स स सू॒नुर् भु॑वद् भुवथ् सू॒नुः स स सू॒नुर् भु॑वत् । \newline
14. सू॒नुर् भु॑वद् भुवथ् सू॒नुः सू॒नुर् भु॑व॒थ् स स भु॑वथ् सू॒नुः सू॒नुर् भु॑व॒थ् सः । \newline
15. भु॒व॒थ् स स भु॑वद् भुव॒थ् स भु॑वद् भुव॒थ् स भु॑वद् भुव॒थ् स भु॑वत् । \newline
16. स भु॑वद् भुव॒थ् स स भु॑व॒त् पुन॑र्मघः॒ पुन॑र्मघो भुव॒थ् स स भु॑व॒त् पुन॑र्मघः । \newline
17. भु॒व॒त् पुन॑र्मघः॒ पुन॑र्मघो भुवद् भुव॒त् पुन॑र्मघः । \newline
18. पुन॑र्मघ॒ इति॒ पुनः॑ - म॒घः॒ । \newline
19. स द्याम् द्याꣳ स स द्या मौर्णो॒ दौर्णो॒द् द्याꣳ स स द्या मौर्णो᳚त् । \newline
20. द्या मौर्णो॒ दौर्णो॒द् द्याम् द्या मौर्णो॑ द॒न्तरि॑क्ष म॒न्तरि॑क्ष॒ मौर्णो॒द् द्याम् द्या मौर्णो॑ द॒न्तरि॑क्षम् । \newline
21. और्णो॑ द॒न्तरि॑क्ष म॒न्तरि॑क्ष॒ मौर्णो॒ दौर्णो॑ द॒न्तरि॑क्षꣳ॒॒ स सो अ॒न्तरि॑क्ष॒ मौर्णो॒ दौर्णो॑ द॒न्तरि॑क्षꣳ॒॒ सः । \newline
22. अ॒न्तरि॑क्षꣳ॒॒ स सो अ॒न्तरि॑क्ष म॒न्तरि॑क्षꣳ॒॒ स सुवः॒ सुवः॒ सो अ॒न्तरि॑क्ष म॒न्तरि॑क्षꣳ॒॒ स सुवः॑ । \newline
23. स सुवः॒ सुवः॒ स स सुवः॒ स स सुवः॒ स स सुवः॒ सः । \newline
24. सुवः॒ स स सुवः॒ सुवः॒ स विश्वा॒ विश्वाः॒ स सुवः॒ सुवः॒ स विश्वाः᳚ । \newline
25. स विश्वा॒ विश्वाः॒ स स विश्वा॒ भुवो॒ भुवो॒ विश्वाः॒ स स विश्वा॒ भुवः॑ । \newline
26. विश्वा॒ भुवो॒ भुवो॒ विश्वा॒ विश्वा॒ भुवो॑ अभव दभव॒द् भुवो॒ विश्वा॒ विश्वा॒ भुवो॑ अभवत् । \newline
27. भुवो॑ अभव दभव॒द् भुवो॒ भुवो॑ अभव॒थ् स सो अ॑भव॒द् भुवो॒ भुवो॑ अभव॒थ् सः । \newline
28. अ॒भ॒व॒थ् स सो अ॑भव दभव॒थ् स आ सो अ॑भव दभव॒थ् स आ । \newline
29. स आ स स आ ऽभ॑व दभव॒दा स स आ ऽभ॑वत् । \newline
30. आ ऽभ॑व दभव॒दा ऽभ॑वत् । \newline
31. अ॒भ॒व॒दित्य॑भवत् । \newline
32. उदु॑ वु॒ वुदुदु॒ त्यम् त्य मु॒ वुदुदु॒ त्यम् । \newline
33. उ॒ त्यम् त्य मु॑ वु॒ त्यम् चि॒त्रम् चि॒त्रम् त्य मु॑ वु॒ त्यम् चि॒त्रम् । \newline
34. त्यम् चि॒त्रम् चि॒त्रम् त्यम् त्यम् चि॒त्रम् । \newline
35. चि॒त्रमिति॑ चि॒त्रम् । \newline
36. स प्र॑त्न॒वत् प्र॑त्न॒वथ् स स प्र॑त्न॒वन् नवी॑यसा॒ नवी॑यसा प्रत्न॒वथ् स स प्र॑त्न॒वन् नवी॑यसा । \newline
37. प्र॒त्न॒वन् नवी॑यसा॒ नवी॑यसा प्रत्न॒वत् प्र॑त्न॒वन् नवी॑य॒सा ऽग्ने ऽग्ने॒ नवी॑यसा प्रत्न॒वत् प्र॑त्न॒वन् नवी॑य॒सा ऽग्ने᳚ । \newline
38. प्र॒त्न॒वदिति॑ प्रत्न - वत् । \newline
39. नवी॑य॒सा ऽग्ने ऽग्ने॒ नवी॑यसा॒ नवी॑य॒सा ऽग्ने᳚ द्यु॒म्नेन॑ द्यु॒म्नेनाग्ने॒ नवी॑यसा॒ नवी॑य॒सा ऽग्ने᳚ द्यु॒म्नेन॑ । \newline
40. अग्ने᳚ द्यु॒म्नेन॑ द्यु॒म्नेनाग्ने ऽग्ने᳚ द्यु॒म्नेन॑ सं॒ॅयता॑ सं॒ॅयता᳚ द्यु॒म्नेनाग्ने ऽग्ने᳚ द्यु॒म्नेन॑ सं॒ॅयता᳚ । \newline
41. द्यु॒म्नेन॑ सं॒ॅयता॑ सं॒ॅयता᳚ द्यु॒म्नेन॑ द्यु॒म्नेन॑ सं॒ॅयता᳚ । \newline
42. सं॒ॅयतेति॑ सं - यता᳚ । \newline
43. बृ॒हत् त॑तन्थ ततन्थ बृ॒हद् बृ॒हत् त॑तन्थ भा॒नुना॑ भा॒नुना॑ ततन्थ बृ॒हद् बृ॒हत् त॑तन्थ भा॒नुना᳚ । \newline
44. त॒त॒न्थ॒ भा॒नुना॑ भा॒नुना॑ ततन्थ ततन्थ भा॒नुना᳚ । \newline
45. भा॒नुनेति॑ भा॒नुना᳚ । \newline
46. नि काव्या॒ काव्या॒ नि नि काव्या॑ वे॒धसो॑ वे॒धसः॒ काव्या॒ नि नि काव्या॑ वे॒धसः॑ । \newline
47. काव्या॑ वे॒धसो॑ वे॒धसः॒ काव्या॒ काव्या॑ वे॒धसः॒ शश्व॑तः॒ शश्व॑तो वे॒धसः॒ काव्या॒ काव्या॑ वे॒धसः॒ शश्व॑तः । \newline
48. वे॒धसः॒ शश्व॑तः॒ शश्व॑तो वे॒धसो॑ वे॒धसः॒ शश्व॑तः कः कः॒ शश्व॑तो वे॒धसो॑ वे॒धसः॒ शश्व॑तः कः । \newline
49. शश्व॑तः कः कः॒ शश्व॑तः॒ शश्व॑तः क॒र्॒. हस्ते॒ हस्ते॑ कः॒ शश्व॑तः॒ शश्व॑तः क॒र्॒. हस्ते᳚ । \newline
50. क॒र्॒. हस्ते॒ हस्ते॑ कः क॒र्॒. हस्ते॒ दधा॑नो॒ दधा॑नो॒ हस्ते॑ कः क॒र्॒. हस्ते॒ दधा॑नः । \newline
51. हस्ते॒ दधा॑नो॒ दधा॑नो॒ हस्ते॒ हस्ते॒ दधा॑नो॒ नर्या॒ नर्या॒ दधा॑नो॒ हस्ते॒ हस्ते॒ दधा॑नो॒ नर्या᳚ । \newline
52. दधा॑नो॒ नर्या॒ नर्या॒ दधा॑नो॒ दधा॑नो॒ नर्या॑ पु॒रूणि॑ पु॒रूणि॒ नर्या॒ दधा॑नो॒ दधा॑नो॒ नर्या॑ पु॒रूणि॑ । \newline
\pagebreak
\markright{ TS 2.2.12.2  \hfill https://www.vedavms.in \hfill}
\addcontentsline{toc}{section}{ TS 2.2.12.2 }
\section*{ TS 2.2.12.2 }

\textbf{TS 2.2.12.2 } \newline
\textbf{Samhita Paata} \newline

नर्या॑ पु॒रूणि॑ । अ॒ग्निर्भु॑वद्रयि॒पती॑ रयी॒णाꣳ स॒त्रा च॑क्रा॒णो अ॒मृता॑नि॒ विश्वा᳚ ॥ हिर॑ण्यपाणिमू॒तये॑ सवि॒तार॒मुप॑ ह्वये । स चेत्ता॑ दे॒वता॑ प॒दं ॥वा॒मम॒द्य स॑वितर्वा॒ममु॒ श्वो दि॒वेदि॑वे वा॒मम॒स्मभ्यꣳ॑ सावीः । वा॒मस्य॒ हि क्षय॑स्य देव॒ भूरे॑र॒या धि॒या वा॑म॒भाजः॑ स्याम ॥बडि॒त्था पर्व॑तानां खि॒द्रं बि॑भर्.षि पृथिवि । प्र या भू॑मि प्रवत्वति म॒ह्ना जि॒नोषि॑ - [  ] \newline

\textbf{Pada Paata} \newline

नर्या᳚ । पु॒रूणि॑ ॥ अ॒ग्निः । भु॒व॒त् । र॒यि॒पति॒रिति॑ रयि - पतिः॑ । र॒यी॒णाम् । स॒त्रा । च॒क्रा॒णः । अ॒मृता॑नि । विश्वा᳚ ॥ हिर॑ण्यपाणि॒मिति॒ हिर॑ण्य - पा॒णि॒म् । ऊ॒तये᳚ । स॒वि॒तार᳚म् । उपेति॑ । ह्व॒ये॒ ॥ सः । चेत्ता᳚ । दे॒वता᳚ । प॒दम् ॥ वा॒मम् । अ॒द्य । स॒वि॒तः॒ । वा॒मम् । उ॒ । श्वः । दि॒वेदि॑व॒ इति॑ दि॒वे - दि॒वे॒ । वा॒मम् । अ॒स्मभ्य॒मित्य॒स्म - भ्य॒म् । सा॒वीः॒ ॥ वा॒मस्य॑ । हि । क्षय॑स्य । दे॒व॒ । भूरेः᳚ । अ॒या । धि॒या । वा॒म॒भाज॒ इति॑ वाम - भाजः॑ । स्या॒म॒ ॥ बट् । इ॒त्था । पर्व॑तानाम् । खि॒द्रम् । बि॒भ॒र्॒.षि॒ । पृ॒थि॒वि॒ ॥ प्रेति॑ । या । भू॒मि॒ । प्र॒व॒त्व॒ति॒ । म॒ह्ना । जि॒नोषि॑ ।  \newline


\textbf{Krama Paata} \newline

नर्या॑ पु॒रूणि॑ । पु॒रूणीति॑ पु॒रूणि॑ ॥ अ॒ग्निर् भु॑वत् । भु॒व॒द् र॒यि॒पतिः॑ । र॒यि॒पती॑ रयी॒णाम् । र॒यि॒पति॒रिति॑ रयि - पतिः॑ । र॒यी॒णाꣳ स॒त्रा । स॒त्रा च॑क्रा॒णः । च॒क्रा॒णो अ॒मृता॑नि । अ॒मृता॑नि॒ विश्वा᳚ । विश्वेति॒ विश्वा᳚ ॥ हिर॑ण्यपाणिमू॒तये᳚ । हिर॑ण्यपाणि॒मिति॒ हिर॑ण्य - पा॒णि॒म् । ऊ॒तये॑ सवि॒तार᳚म् । स॒वि॒तार॒मुप॑ । उप॑ ह्वये । ह्व॒य॒ इति॑ ह्वये ॥ स चेत्ता᳚ । चेत्ता॑ दे॒वता᳚ । दे॒वता॑ प॒दम् । प॒दमिति॑ प॒दम् ॥ वा॒मम॒द्य । अ॒द्य स॑वितः । स॒वि॒त॒र् वा॒मम् । वा॒ममु॑ । उ॒ श्वः । श्वो दि॒वेदि॑वे । दि॒वेदि॑वे वा॒मम् । दि॒वेदि॑व॒ इति॑ दि॒वे - दि॒वे॒ । वा॒मम॒स्मभ्य᳚म् । अ॒स्मभ्यꣳ॑ सावीः । अ॒स्मभ्य॒मित्य॒स्म - भ्य॒म् । सा॒वी॒रिति॑ सावीः । वा॒मस्य॒ हि । हि क्षय॑स्य । क्षय॑स्य देव । दे॒व॒ भूरेः᳚ । भूरे॑र॒या । अ॒या धि॒या । धि॒या वा॑म॒भाजः॑ । वा॒म॒भाजः॑ स्याम । वा॒म॒भाज॒ इति॑ वाम - भाजः॑ । स्या॒मेति॑ स्याम ॥ बडि॒त्था । इ॒त्था पर्व॑तानाम् । पर्व॑तानाम् खि॒द्रम् । खि॒द्रम् बि॑भर्.षि । बि॒भ॒र्॒.षि॒ पृ॒थि॒वि॒ । पृ॒थि॒वीति॑ पृथिवि ॥ प्र या । या भू॑मि । भू॒मि॒ प्र॒व॒त्व॒ति॒ । प्र॒व॒त्व॒ति॒ म॒ह्ना । म॒ह्ना जि॒नोषि॑ । जि॒नोषि॑ महिनि \newline

\textbf{Jatai Paata} \newline

1. नर्या॑ पु॒रूणि॑ पु॒रूणि॒ नर्या॒ नर्या॑ पु॒रूणि॑ । \newline
2. पु॒रूणीति॑ पु॒रूणि॑ । \newline
3. अ॒ग्निर् भु॑वद् भुव द॒ग्नि र॒ग्निर् भु॑वत् । \newline
4. भु॒व॒द् र॒यि॒पती॑ रयि॒पति॑र् भुवद् भुवद् रयि॒पतिः॑ । \newline
5. र॒यि॒पती॑ रयी॒णाꣳ र॑यी॒णाꣳ र॑यि॒पती॑ रयि॒पती॑ रयी॒णाम् । \newline
6. र॒यि॒पति॒रिति॑ रयि - पतिः॑ । \newline
7. र॒यी॒णाꣳ स॒त्रा स॒त्रा र॑यी॒णाꣳ र॑यी॒णाꣳ स॒त्रा । \newline
8. स॒त्रा च॑क्रा॒ण श्च॑क्रा॒णः स॒त्रा स॒त्रा च॑क्रा॒णः । \newline
9. च॒क्रा॒णो अ॒मृता᳚ न्य॒मृता॑नि चक्रा॒ण श्च॑क्रा॒णो अ॒मृता॑नि । \newline
10. अ॒मृता॑नि॒ विश्वा॒ विश्वा॒ ऽमृता᳚ न्य॒मृता॑नि॒ विश्वा᳚ । \newline
11. विश्वेति॒ विश्वा᳚ । \newline
12. हिर॑ण्यपाणि मू॒तय॑ ऊ॒तये॒ हिर॑ण्यपाणिꣳ॒॒ हिर॑ण्यपाणि मू॒तये᳚ । \newline
13. हिर॑ण्यपाणि॒मिति॒ हिर॑ण्य - पा॒णि॒म् । \newline
14. ऊ॒तये॑ सवि॒तारꣳ॑ सवि॒तार॑ मू॒तय॑ ऊ॒तये॑ सवि॒तार᳚म् । \newline
15. स॒वि॒तार॒ मुपोप॑ सवि॒तारꣳ॑ सवि॒तार॒ मुप॑ । \newline
16. उप॑ ह्वये ह्वय॒ उपोप॑ ह्वये । \newline
17. ह्व॒य॒ इति॑ ह्वये । \newline
18. स चेत्ता॒ चेत्ता॒ स स चेत्ता᳚ । \newline
19. चेत्ता॑ दे॒वता॑ दे॒वता॒ चेत्ता॒ चेत्ता॑ दे॒वता᳚ । \newline
20. दे॒वता॑ प॒दम् प॒दम् दे॒वता॑ दे॒वता॑ प॒दम् । \newline
21. प॒दमिति॑ प॒दम् । \newline
22. वा॒म म॒द्याद्य वा॒मं ॅवा॒म म॒द्य । \newline
23. अ॒द्य स॑वितः सवित-र॒द्याद्य स॑वितः । \newline
24. स॒वि॒त॒र् वा॒मं ॅवा॒मꣳ स॑वितः सवितर् वा॒मम् । \newline
25. वा॒म मु॑ वु वा॒मं ॅवा॒म मु॑ । \newline
26. उ॒ श्वः श्व उ॑ वु॒ श्वः । \newline
27. श्वो दि॒वेदि॑वे दि॒वेदि॑वे॒ श्वः श्वो दि॒वेदि॑वे । \newline
28. दि॒वेदि॑वे वा॒मं ॅवा॒मम् दि॒वेदि॑वे दि॒वेदि॑वे वा॒मम् । \newline
29. दि॒वेदि॑व॒ इति॑ दि॒वे - दि॒वे॒ । \newline
30. वा॒म म॒स्मभ्य॑ म॒स्मभ्यं॑ ॅवा॒मं ॅवा॒म म॒स्मभ्य᳚म् । \newline
31. अ॒स्मभ्यꣳ॑ सावीः सावीर॒स्मभ्य॑ म॒स्मभ्यꣳ॑ सावीः । \newline
32. अ॒स्मभ्य॒मित्य॒स्म - भ्य॒म् । \newline
33. सा॒वी॒रिति॑ सावीः । \newline
34. वा॒मस्य॒ हि हि वा॒मस्य॑ वा॒मस्य॒ हि । \newline
35. हि क्षय॑स्य॒ क्षय॑स्य॒ हि हि क्षय॑स्य । \newline
36. क्षय॑स्य देव देव॒ क्षय॑स्य॒ क्षय॑स्य देव । \newline
37. दे॒व॒ भूरे॒र् भूरे᳚र् देव देव॒ भूरेः᳚ । \newline
38. भूरे॑र॒या ऽया भूरे॒र् भूरे॑र॒या । \newline
39. अ॒या धि॒या धि॒या ऽया ऽया धि॒या । \newline
40. धि॒या वा॑म॒भाजो॑ वाम॒भाजो॑ धि॒या धि॒या वा॑म॒भाजः॑ । \newline
41. वा॒म॒भाजः॑ स्याम स्याम वाम॒भाजो॑ वाम॒भाजः॑ स्याम । \newline
42. वा॒म॒भाज॒ इति॑ वाम - भाजः॑ । \newline
43. स्या॒मेति॑ स्याम । \newline
44. बडि॒त्थेत्था बड् बडि॒त्था । \newline
45. इ॒त्था पर्व॑ताना॒म् पर्व॑ताना मि॒त्थेत्था पर्व॑तानाम् । \newline
46. पर्व॑तानाम् खि॒द्रम् खि॒द्रम् पर्व॑ताना॒म् पर्व॑तानाम् खि॒द्रम् । \newline
47. खि॒द्रम् बि॑भर्.षि बिभर्.षि खि॒द्रम् खि॒द्रम् बि॑भर्.षि । \newline
48. बि॒भ॒र्॒.षि॒ पृ॒थि॒वि॒ पृ॒थि॒वि॒ बि॒भ॒र्॒.षि॒ बि॒भ॒र्॒.षि॒ पृ॒थि॒वि॒ । \newline
49. पृ॒थि॒वीति॑ पृथिवि । \newline
50. प्र या या प्र प्र या । \newline
51. या भू॑मि भूमि॒ या या भू॑मि । \newline
52. भू॒मि॒ प्र॒व॒त्व॒ति॒ प्र॒व॒त्व॒ति॒ भू॒मि॒ भू॒मि॒ प्र॒व॒त्व॒ति॒ । \newline
53. प्र॒व॒त्व॒ति॒ म॒ह्ना म॒ह्ना प्र॑वत्वति प्रवत्वति म॒ह्ना । \newline
54. म॒ह्ना जि॒नोषि॑ जि॒नोषि॑ म॒ह्ना म॒ह्ना जि॒नोषि॑ । \newline
55. जि॒नोषि॑ महिनि महिनि जि॒नोषि॑ जि॒नोषि॑ महिनि । \newline

\textbf{Ghana Paata } \newline

1. नर्या॑ पु॒रूणि॑ पु॒रूणि॒ नर्या॒ नर्या॑ पु॒रूणि॑ । \newline
2. पु॒रूणीति॑ पु॒रूणि॑ । \newline
3. अ॒ग्निर् भु॑वद् भुव द॒ग्नि र॒ग्निर् भु॑वद् रयि॒पती॑ रयि॒पति॑र् भुव द॒ग्नि र॒ग्निर् भु॑वद् रयि॒पतिः॑ । \newline
4. भु॒व॒द् र॒यि॒पती॑ रयि॒पति॑र् भुवद् भुवद् रयि॒पती॑ रयी॒णाꣳ र॑यी॒णाꣳ र॑यि॒पति॑र् भुवद् भुवद् रयि॒पती॑ रयी॒णाम् । \newline
5. र॒यि॒पती॑ रयी॒णाꣳ र॑यी॒णाꣳ र॑यि॒पती॑ रयि॒पती॑ रयी॒णाꣳ स॒त्रा स॒त्रा र॑यी॒णाꣳ र॑यि॒पती॑ रयि॒पती॑ रयी॒णाꣳ स॒त्रा । \newline
6. र॒यि॒पति॒रिति॑ रयि - पतिः॑ । \newline
7. र॒यी॒णाꣳ स॒त्रा स॒त्रा र॑यी॒णाꣳ र॑यी॒णाꣳ स॒त्रा च॑क्रा॒ण श्च॑क्रा॒णः स॒त्रा र॑यी॒णाꣳ र॑यी॒णाꣳ स॒त्रा च॑क्रा॒णः । \newline
8. स॒त्रा च॑क्रा॒ण श्च॑क्रा॒णः स॒त्रा स॒त्रा च॑क्रा॒णो अ॒मृता᳚ न्य॒मृता॑नि चक्रा॒णः स॒त्रा स॒त्रा च॑क्रा॒णो अ॒मृता॑नि । \newline
9. च॒क्रा॒णो अ॒मृता᳚ न्य॒मृता॑नि चक्रा॒ण श्च॑क्रा॒णो अ॒मृता॑नि॒ विश्वा॒ विश्वा॒ ऽमृता॑नि चक्रा॒ण श्च॑क्रा॒णो अ॒मृता॑नि॒ विश्वा᳚ । \newline
10. अ॒मृता॑नि॒ विश्वा॒ विश्वा॒ ऽमृता᳚ न्य॒मृता॑नि॒ विश्वा᳚ । \newline
11. विश्वेति॒ विश्वा᳚ । \newline
12. हिर॑ण्यपाणि मू॒तय॑ ऊ॒तये॒ हिर॑ण्यपाणिꣳ॒॒ हिर॑ण्यपाणि मू॒तये॑ सवि॒तारꣳ॑ सवि॒तार॑ मू॒तये॒ हिर॑ण्यपाणिꣳ॒॒ हिर॑ण्यपाणि मू॒तये॑ सवि॒तार᳚म् । \newline
13. हिर॑ण्यपाणि॒मिति॒ हिर॑ण्य - पा॒णि॒म् । \newline
14. ऊ॒तये॑ सवि॒तारꣳ॑ सवि॒तार॑ मू॒तय॑ ऊ॒तये॑ सवि॒तार॒ मुपोप॑ सवि॒तार॑ मू॒तय॑ ऊ॒तये॑ सवि॒तार॒ मुप॑ । \newline
15. स॒वि॒तार॒ मुपोप॑ सवि॒तारꣳ॑ सवि॒तार॒ मुप॑ ह्वये ह्वय॒ उप॑ सवि॒तारꣳ॑ सवि॒तार॒ मुप॑ ह्वये । \newline
16. उप॑ ह्वये ह्वय॒ उपोप॑ ह्वये । \newline
17. ह्व॒य॒ इति॑ ह्वये । \newline
18. स चेत्ता॒ चेत्ता॒ स स चेत्ता॑ दे॒वता॑ दे॒वता॒ चेत्ता॒ स स चेत्ता॑ दे॒वता᳚ । \newline
19. चेत्ता॑ दे॒वता॑ दे॒वता॒ चेत्ता॒ चेत्ता॑ दे॒वता॑ प॒दम् प॒दम् दे॒वता॒ चेत्ता॒ चेत्ता॑ दे॒वता॑ प॒दम् । \newline
20. दे॒वता॑ प॒दम् प॒दम् दे॒वता॑ दे॒वता॑ प॒दम् । \newline
21. प॒दमिति॑ प॒दम् । \newline
22. वा॒म म॒द्याद्य वा॒मं ॅवा॒म म॒द्य स॑वितः सवित र॒द्य वा॒मं ॅवा॒म म॒द्य स॑वितः । \newline
23. अ॒द्य स॑वितः सवित र॒द्याद्य स॑वितर् वा॒मं ॅवा॒मꣳ स॑वित र॒द्याद्य स॑वितर् वा॒मम् । \newline
24. स॒वि॒त॒र् वा॒मं ॅवा॒मꣳ स॑वितः सवितर् वा॒म मु॑ वु वा॒मꣳ स॑वितः सवितर् वा॒म मु॑ । \newline
25. वा॒म मु॑ वु वा॒मं ॅवा॒म मु॒ श्वः श्व उ॑ वा॒मं ॅवा॒म मु॒ श्वः । \newline
26. उ॒ श्वः श्व उ॑ वु॒ श्वो दि॒वेदि॑वे दि॒वेदि॑वे॒ श्व उ॑ वु॒ श्वो दि॒वेदि॑वे । \newline
27. श्वो दि॒वेदि॑वे दि॒वेदि॑वे॒ श्वः श्वो दि॒वेदि॑वे वा॒मं ॅवा॒मम् दि॒वेदि॑वे॒ श्वः श्वो दि॒वेदि॑वे वा॒मम् । \newline
28. दि॒वेदि॑वे वा॒मं ॅवा॒मम् दि॒वेदि॑वे दि॒वेदि॑वे वा॒म म॒स्मभ्य॑ म॒स्मभ्यं॑ ॅवा॒मम् दि॒वेदि॑वे दि॒वेदि॑वे वा॒म म॒स्मभ्य᳚म् । \newline
29. दि॒वेदि॑व॒ इति॑ दि॒वे - दि॒वे॒ । \newline
30. वा॒म म॒स्मभ्य॑ म॒स्मभ्यं॑ ॅवा॒मं ॅवा॒म म॒स्मभ्यꣳ॑ सावीः सावी र॒स्मभ्यं॑ ॅवा॒मं ॅवा॒म म॒स्मभ्यꣳ॑ सावीः । \newline
31. अ॒स्मभ्यꣳ॑ सावीः सावी र॒स्मभ्य॑ म॒स्मभ्यꣳ॑ सावीः । \newline
32. अ॒स्मभ्य॒मित्य॒स्म - भ्य॒म् । \newline
33. सा॒वी॒रिति॑ सावीः । \newline
34. वा॒मस्य॒ हि हि वा॒मस्य॑ वा॒मस्य॒ हि क्षय॑स्य॒ क्षय॑स्य॒ हि वा॒मस्य॑ वा॒मस्य॒ हि क्षय॑स्य । \newline
35. हि क्षय॑स्य॒ क्षय॑स्य॒ हि हि क्षय॑स्य देव देव॒ क्षय॑स्य॒ हि हि क्षय॑स्य देव । \newline
36. क्षय॑स्य देव देव॒ क्षय॑स्य॒ क्षय॑स्य देव॒ भूरे॒र् भूरे᳚र् देव॒ क्षय॑स्य॒ क्षय॑स्य देव॒ भूरेः᳚ । \newline
37. दे॒व॒ भूरे॒र् भूरे᳚र् देव देव॒ भूरे॑र॒या ऽया भूरे᳚र् देव देव॒ भूरे॑र॒या । \newline
38. भूरे॑र॒या ऽया भूरे॒र् भूरे॑र॒या धि॒या धि॒या ऽया भूरे॒र् भूरे॑ र॒या धि॒या । \newline
39. अ॒या धि॒या धि॒या ऽया ऽया धि॒या वा॑म॒भाजो॑ वाम॒भाजो॑ धि॒या ऽया ऽया धि॒या वा॑म॒भाजः॑ । \newline
40. धि॒या वा॑म॒भाजो॑ वाम॒भाजो॑ धि॒या धि॒या वा॑म॒भाजः॑ स्याम स्याम वाम॒भाजो॑ धि॒या धि॒या वा॑म॒भाजः॑ स्याम । \newline
41. वा॒म॒भाजः॑ स्याम स्याम वाम॒भाजो॑ वाम॒भाजः॑ स्याम । \newline
42. वा॒म॒भाज॒ इति॑ वाम - भाजः॑ । \newline
43. स्या॒मेति॑ स्याम । \newline
44. बडि॒त्थेत्था बड् बडि॒त्था पर्व॑ताना॒म् पर्व॑ताना मि॒त्था बड् बडि॒त्था पर्व॑तानाम् । \newline
45. इ॒त्था पर्व॑ताना॒म् पर्व॑ताना मि॒त्थेत्था पर्व॑तानाम् खि॒द्रम् खि॒द्रम् पर्व॑ताना मि॒त्थेत्था पर्व॑तानाम् खि॒द्रम् । \newline
46. पर्व॑तानाम् खि॒द्रम् खि॒द्रम् पर्व॑ताना॒म् पर्व॑तानाम् खि॒द्रम् बि॑भर्.षि बिभर्.षि खि॒द्रम् पर्व॑ताना॒म् पर्व॑तानाम् खि॒द्रम् बि॑भर्.षि । \newline
47. खि॒द्रम् बि॑भर्.षि बिभर्.षि खि॒द्रम् खि॒द्रम् बि॑भर्.षि पृथिवि पृथिवि बिभर्.षि खि॒द्रम् खि॒द्रम् बि॑भर्.षि पृथिवि । \newline
48. बि॒भ॒र्॒.षि॒ पृ॒थि॒वि॒ पृ॒थि॒वि॒ बि॒भ॒र्॒.षि॒ बि॒भ॒र्॒.षि॒ पृ॒थि॒वि॒ । \newline
49. पृ॒थि॒वीति॑ पृथिवि । \newline
50. प्र या या प्र प्र या भू॑मि भूमि॒ या प्र प्र या भू॑मि । \newline
51. या भू॑मि भूमि॒ या या भू॑मि प्रवत्वति प्रवत्वति भूमि॒ या या भू॑मि प्रवत्वति । \newline
52. भू॒मि॒ प्र॒व॒त्व॒ति॒ प्र॒व॒त्व॒ति॒ भू॒मि॒ भू॒मि॒ प्र॒व॒त्व॒ति॒ म॒ह्ना म॒ह्ना प्र॑वत्वति भूमि भूमि प्रवत्वति म॒ह्ना । \newline
53. प्र॒व॒त्व॒ति॒ म॒ह्ना म॒ह्ना प्र॑वत्वति प्रवत्वति म॒ह्ना जि॒नोषि॑ जि॒नोषि॑ म॒ह्ना प्र॑वत्वति प्रवत्वति म॒ह्ना जि॒नोषि॑ । \newline
54. म॒ह्ना जि॒नोषि॑ जि॒नोषि॑ म॒ह्ना म॒ह्ना जि॒नोषि॑ महिनि महिनि जि॒नोषि॑ म॒ह्ना म॒ह्ना जि॒नोषि॑ महिनि । \newline
55. जि॒नोषि॑ महिनि महिनि जि॒नोषि॑ जि॒नोषि॑ महिनि । \newline
\pagebreak
\markright{ TS 2.2.12.3  \hfill https://www.vedavms.in \hfill}
\addcontentsline{toc}{section}{ TS 2.2.12.3 }
\section*{ TS 2.2.12.3 }

\textbf{TS 2.2.12.3 } \newline
\textbf{Samhita Paata} \newline

महिनि ॥ स्तोमा॑सस्त्वा विचारिणि॒ प्रति॑ष्टोभन्त्य॒क्तुभिः॑ । प्रया वाजं॒ न हेष॑न्तं पे॒रुमस्य॑स्यर्जुनि ॥ ऋ॒दू॒दरे॑ण॒ सख्या॑ सचेय॒ यो मा॒ न रिष्ये᳚द्धर्यश्व पी॒तः । अ॒यं ॅयः सोमो॒ न्यधा᳚य्य॒स्मे तस्मा॒ इन्द्रं॑ प्र॒तिर॑मे॒म्यच्छ॑ ॥ आपा᳚न्तमन्युस्तृ॒पल॑-प्रभर्मा॒ धुनिः॒ शिमी॑ वा॒ञ्छरु॑माꣳ ऋजी॒षी । सोमो॒ विश्वा᳚न्यत॒सा वना॑नि॒ नार्वागिन्द्रं॑ प्रति॒माना॑नि देभुः ॥ प्र - [  ] \newline

\textbf{Pada Paata} \newline

म॒हि॒नि॒ ॥ स्तोमा॑सः । त्वा॒ । वि॒चा॒रि॒णीति॑ वि - चा॒रि॒णि॒ । प्रतीति॑ । स्तो॒भ॒न्ति॒ । अ॒क्तुभि॒रित्य॒क्तु - भिः॒ ॥ प्रेति॑ । या । वाज᳚म् । न । हेष॑न्तम् । पे॒रुम् । अस्य॑सि । अ॒र्जु॒नि॒ ॥ ऋ॒दू॒दरे॑ण । सख्या᳚ । स॒चे॒य॒ । यः । मा॒ । न । रिष्ये᳚त् । ह॒र्य॒श्वेति॑ हरि-अ॒श्व॒ । पी॒तः ॥ अ॒यम् । यः । सोमः॑ । न्यधा॒यीति॑ नि - अधा॑यि । अ॒स्मे इति॑ । तस्मै᳚ । इन्द्र᳚म् । प्र॒तिर॒मिति॑ प्र - तिर᳚म् । ए॒मि॒ । अच्छ॑ ॥ आपा᳚न्तमन्यु॒रित्यापा᳚न्त - म॒न्युः॒ । तृ॒पल॑प्रभ॒र्मेति॑ तृ॒पल॑ - प्र॒भ॒र्मा॒ । धुनिः॑ । शिमी॑वान् । शरु॑मा॒निति॒ शरु॑ - मा॒न् । ऋ॒जी॒षी ॥ सोमः॑ । विश्वा॑नि । अ॒त॒सा । वना॑नि । न । अ॒र्वाक् । इन्द्र᳚म् । प्र॒ति॒माना॒नीति॑ प्रति - माना॑नि । दे॒भुः॒ ॥ प्रेति॑ ।  \newline


\textbf{Krama Paata} \newline

म॒हि॒नीति॑ महिनि ॥ स्तोमा॑सस्त्वा । त्वा॒ वि॒चा॒रि॒णि॒ । वि॒चा॒रि॒णि॒ प्रति॑ । वि॒चा॒रि॒णीति॑ वि - चा॒रि॒णि॒ । प्रति॑ष्टोभन्ति । स्तो॒भ॒न्त्य॒क्तुभिः॑ । अ॒क्तुभि॒रित्य॒क्तु - भिः॒ ॥ प्र या । या वाज᳚म् । वाज॒म् न । न हेष॑न्तम् । हेष॑न्तम् पे॒रुम् । पे॒रुमस्य॑सि । अस्य॑स्यर्जुनि । अ॒र्जु॒नीत्य॑र्जुनि ॥ ऋ॒दू॒दरे॑ण॒ सख्या᳚ । सख्या॑ सचेय । स॒चे॒य॒ यः । यो मा᳚ । मा॒ न । न रिष्ये᳚त् । रिष्ये᳚द्धर्यश्व । ह॒र्य॒श्व॒ पी॒तः । ह॒र्य॒श्वेति॑ हरि - अ॒श्व॒ । पी॒त इति॑ पी॒तः ॥ अ॒यं ॅयः । यः सोमः॑ । सोमो॒ न्यधा॑यि । न्यधा᳚य्य॒स्मे । न्यधा॒यीति॑ नि - अधा॑यि । अ॒स्मे तस्मै᳚ । अ॒स्मे इत्य॒स्मे । तस्मा॒ इन्द्र᳚म् । इन्द्र॑म् प्र॒तिर᳚म् । प्र॒तिर॑मेमि । प्र॒तिर॒मिति॑ प्र - तिर᳚म् । ए॒म्यच्छ॑ । अच्छेत्यच्छ॑ ॥? आपा᳚न्तमन्यु,स्तृ॒पल॑प्रभर्मा । आपा᳚न्तमन्यु॒रित्यापा᳚न्त - म॒न्युः॒ । तृ॒पल॑प्रभर्मा॒ धुनिः॑ । तृ॒पल॑प्रभ॒र्मेति॑ तृ॒पल॑ - प्र॒भ॒र्मा॒ । धुनिः॒ शिमी॑वान् । शिमी॑वा॒ञ्छरु॑मान् । शरु॑माꣳ ऋजी॒षी । शरु॑मा॒निति॒ शरु॑ - मा॒न्॒ । ऋ॒जी॒षीत्यृ॑जी॒षी ॥ सोमो॒ विश्वा॑नि । विश्वा᳚न्यत॒सा । अ॒त॒सा वना॑नि । वना॑नि॒ न । नार्वाक् । अ॒र्वागिन्द्र᳚म् । इन्द्र॑म् प्रति॒माना॑नि । प्र॒ति॒माना॑नि देभुः । प्र॒ति॒माना॒नीति॑ प्रति - माना॑नि । दे॒भु॒रिति॑ देभुः ॥ प्र सु॑वा॒नः \newline

\textbf{Jatai Paata} \newline

1. म॒हि॒नीति॑ महिनि । \newline
2. स्तोमा॑स स्त्वा त्वा॒ स्तोमा॑सः॒ स्तोमा॑ सस्त्वा । \newline
3. त्वा॒ वि॒चा॒रि॒णि॒ वि॒चा॒रि॒णि॒ त्वा॒ त्वा॒ वि॒चा॒रि॒णि॒ । \newline
4. वि॒चा॒रि॒णि॒ प्रति॒ प्रति॑ विचारिणि विचारिणि॒ प्रति॑ । \newline
5. वि॒चा॒रि॒णीति॑ वि - चा॒रि॒णि॒ । \newline
6. प्रति॑ ष्टोभन्ति स्तोभन्ति॒ प्रति॒ प्रति॑ ष्टोभन्ति । \newline
7. स्तो॒भ॒ न्त्य॒क्तुभि॑ र॒क्तुभिः॑ स्तोभन्ति स्तोभ न्त्य॒क्तुभिः॑ । \newline
8. अ॒क्तुभि॒रित्य॒क्तु - भिः॒ । \newline
9. प्र या या प्र प्र या । \newline
10. या वाजं॒ ॅवाजं॒ ॅया या वाज᳚म् । \newline
11. वाज॒म् न न वाजं॒ ॅवाज॒म् न । \newline
12. न हेष॑न्तꣳ॒॒ हेष॑न्त॒म् न न हेष॑न्तम् । \newline
13. हेष॑न्तम् पे॒रुम् पे॒रुꣳ हेष॑न्तꣳ॒॒ हेष॑न्तम् पे॒रुम् । \newline
14. पे॒रु मस्य॒ स्यस्य॑सि पे॒रुम् पे॒रु मस्य॑सि । \newline
15. अस्य॑ स्यर्जु न्यर्जु॒ न्यस्य॒ स्यस्य॑ स्यर्जुनि । \newline
16. अ॒र्जु॒नीत्य॑र्जुनि । \newline
17. ऋ॒दू॒दरे॑ण॒ सख्या॒ सख्य॑ र्‌दू॒दरे॑ण र्‌दू॒दरे॑ण॒ सख्या᳚ । \newline
18. सख्या॑ सचेय सचेय॒ सख्या॒ सख्या॑ सचेय । \newline
19. स॒चे॒य॒ यो यः स॑चेय सचेय॒ यः । \newline
20. यो मा॑ मा॒ यो यो मा᳚ । \newline
21. मा॒ न न मा॑ मा॒ न । \newline
22. न रिष्ये॒द् रिष्ये॒न् न न रिष्ये᳚त् । \newline
23. रिष्ये᳚ द्धर्यश्व हर्यश्व॒ रिष्ये॒द् रिष्ये᳚ द्धर्यश्व । \newline
24. ह॒र्य॒श्व॒ पी॒तः पी॒तो ह॑र्यश्व हर्यश्व पी॒तः । \newline
25. ह॒र्य॒श्वेति॑ हरि - अ॒श्व॒ । \newline
26. पी॒त इति॑ पी॒तः । \newline
27. अ॒यं ॅयो यो॑ ऽय म॒यं ॅयः । \newline
28. यः सोमः॒ सोमो॒ यो यः सोमः॑ । \newline
29. सोमो॒ न्यधा॑यि॒ न्यधा॑यि॒ सोमः॒ सोमो॒ न्यधा॑यि । \newline
30. न्यधा᳚य्य॒स्मे अ॒स्मे न्यधा॑यि॒ न्यधा᳚य्य॒स्मे । \newline
31. न्यधा॒यीति॑ नि - अधा॑यि । \newline
32. अ॒स्मे तस्मै॒ तस्मा॑ अ॒स्मे अ॒स्मे तस्मै᳚ । \newline
33. अ॒स्मे इत्य॒स्मे । \newline
34. तस्मा॒ इन्द्र॒ मिन्द्र॒म् तस्मै॒ तस्मा॒ इन्द्र᳚म् । \newline
35. इन्द्र॑म् प्र॒तिर॑म् प्र॒तिर॒ मिन्द्र॒ मिन्द्र॑म् प्र॒तिर᳚म् । \newline
36. प्र॒तिर॑ मेम्येमि प्र॒तिर॑म् प्र॒तिर॑ मेमि । \newline
37. प्र॒तिर॒मिति॑ प्र - तिर᳚म् । \newline
38. ए॒ म्यच्छा च्छै᳚म्ये॒ म्यच्छ॑ । \newline
39. अच्छेत्यच्छ॑ । \newline
40. आपा᳚न्तमन्यु स्तृ॒पल॑प्रभर्मा तृ॒पल॑प्रभ॒र्मा ऽऽपा᳚न्तमन्यु॒ रापा᳚न्तमन्यु स्तृ॒पल॑प्रभर्मा । \newline
41. आपा᳚न्तमन्यु॒रित्यापा᳚न्त - म॒न्युः॒ । \newline
42. तृ॒पल॑प्रभर्मा॒ धुनि॒र् धुनि॑ स्तृ॒पल॑प्रभर्मा तृ॒पल॑प्रभर्मा॒ धुनिः॑ । \newline
43. तृ॒पल॑प्रभ॒र्मेति॑ तृ॒पल॑ - प्र॒भ॒र्मा॒ । \newline
44. धुनिः॒ शिमी॑वा॒ञ् छिमी॑वा॒न् धुनि॒र् धुनिः॒ शिमी॑वान् । \newline
45. शिमी॑वा॒ञ् छरु॑मा॒ञ् छरु॑मा॒ञ् छिमी॑वा॒ञ् छिमी॑वा॒ञ् छरु॑मान् । \newline
46. शरु॑माꣳ ऋजी॒ष्यृ॑जी॒षी शरु॑मा॒ञ् छरु॑माꣳ ऋजी॒षी । \newline
47. शरु॑मा॒निति॒ शरु॑ - मा॒न् । \newline
48. ऋ॒जी॒षीत्यृ॑जी॒षी । \newline
49. सोमो॒ विश्वा॑नि॒ विश्वा॑नि॒ सोमः॒ सोमो॒ विश्वा॑नि । \newline
50. विश्वा᳚न्यत॒सा ऽत॒सा विश्वा॑नि॒ विश्वा᳚ न्यत॒सा । \newline
51. अ॒त॒सा वना॑नि॒ वना᳚ न्यत॒सा ऽत॒सा वना॑नि । \newline
52. वना॑नि॒ न न वना॑नि॒ वना॑नि॒ न । \newline
53. नार्वा ग॒र्वाङ् न नार्वाक् । \newline
54. अ॒र्वा गिन्द्र॒ मिन्द्र॑ म॒र्वा ग॒र्वागिन्द्र᳚म् । \newline
55. इन्द्र॑म् प्रति॒माना॑नि प्रति॒माना॒नीन्द्र॒ मिन्द्र॑म् प्रति॒माना॑नि । \newline
56. प्र॒ति॒माना॑नि देभुर् देभुः प्रति॒माना॑नि प्रति॒माना॑नि देभुः । \newline
57. प्र॒ति॒माना॒नीति॑ प्रति - माना॑नि । \newline
58. दे॒भु॒रिति॑ देभुः । \newline
59. प्र सु॑वा॒नः सु॑वा॒नः प्र प्र सु॑वा॒नः । \newline

\textbf{Ghana Paata } \newline

1. म॒हि॒नीति॑ महिनि । \newline
2. स्तोमा॑स स्त्वा त्वा॒ स्तोमा॑सः॒ स्तोमा॑ सस्त्वा विचारिणि विचारिणि त्वा॒ स्तोमा॑सः॒ स्तोमा॑ सस्त्वा विचारिणि । \newline
3. त्वा॒ वि॒चा॒रि॒णि॒ वि॒चा॒रि॒णि॒ त्वा॒ त्वा॒ वि॒चा॒रि॒णि॒ प्रति॒ प्रति॑ विचारिणि त्वा त्वा विचारिणि॒ प्रति॑ । \newline
4. वि॒चा॒रि॒णि॒ प्रति॒ प्रति॑ विचारिणि विचारिणि॒ प्रति॑ ष्टोभन्ति स्तोभन्ति॒ प्रति॑ विचारिणि विचारिणि॒ प्रति॑ ष्टोभन्ति । \newline
5. वि॒चा॒रि॒णीति॑ वि - चा॒रि॒णि॒ । \newline
6. प्रति॑ ष्टोभन्ति स्तोभन्ति॒ प्रति॒ प्रति॑ ष्टोभ न्त्य॒क्तुभि॑ र॒क्तुभिः॑ स्तोभन्ति॒ प्रति॒ प्रति॑ ष्टोभ न्त्य॒क्तुभिः॑ । \newline
7. स्तो॒भ॒ न्त्य॒क्तुभि॑ र॒क्तुभिः॑ स्तोभन्ति स्तोभ न्त्य॒क्तुभिः॑ । \newline
8. अ॒क्तुभि॒रित्य॒क्तु - भिः॒ । \newline
9. प्र या या प्र प्र या वाजं॒ ॅवाजं॒ ॅया प्र प्र या वाज᳚म् । \newline
10. या वाजं॒ ॅवाजं॒ ॅया या वाज॒म् न न वाजं॒ ॅया या वाज॒म् न । \newline
11. वाज॒म् न न वाजं॒ ॅवाज॒म् न हेष॑न्तꣳ॒॒ हेष॑न्त॒म् न वाजं॒ ॅवाज॒म् न हेष॑न्तम् । \newline
12. न हेष॑न्तꣳ॒॒ हेष॑न्त॒म् न न हेष॑न्तम् पे॒रुम् पे॒रुꣳ हेष॑न्त॒म् न न हेष॑न्तम् पे॒रुम् । \newline
13. हेष॑न्तम् पे॒रुम् पे॒रुꣳ हेष॑न्तꣳ॒॒ हेष॑न्तम् पे॒रु मस्य॒ स्यस्य॑सि पे॒रुꣳ हेष॑न्तꣳ॒॒ हेष॑न्तम् पे॒रु मस्य॑सि । \newline
14. पे॒रु मस्य॒ स्यस्य॑सि पे॒रुम् पे॒रु मस्य॑ स्यर्जु न्यर्जु॒ न्यस्य॑सि पे॒रुम् पे॒रु मस्य॑ स्यर्जुनि । \newline
15. अस्य॑ स्यर्जु न्यर्जु॒ न्यस्य॒ स्यस्य॑ स्यर्जुनि । \newline
16. अ॒र्जु॒नीत्य॑र्जुनि । \newline
17. ऋ॒दू॒दरे॑ण॒ सख्या॒ सख्य॑ र्‌दू॒दरे॑ण र्दू॒दरे॑ण॒ सख्या॑ सचेय सचेय॒ सख्य॑ र्‌दू॒दरे॑ण र्‌दू॒दरे॑ण॒ सख्या॑ सचेय । \newline
18. सख्या॑ सचेय सचेय॒ सख्या॒ सख्या॑ सचेय॒ यो यः स॑चेय॒ सख्या॒ सख्या॑ सचेय॒ यः । \newline
19. स॒चे॒य॒ यो यः स॑चेय सचेय॒ यो मा॑ मा॒ यः स॑चेय सचेय॒ यो मा᳚ । \newline
20. यो मा॑ मा॒ यो यो मा॒ न न मा॒ यो यो मा॒ न । \newline
21. मा॒ न न मा॑ मा॒ न रिष्ये॒द् रिष्ये॒न् न मा॑ मा॒ न रिष्ये᳚त् । \newline
22. न रिष्ये॒द् रिष्ये॒न् न न रिष्ये᳚ द्धर्यश्व हर्यश्व॒ रिष्ये॒न् न न रिष्ये᳚ द्धर्यश्व । \newline
23. रिष्ये᳚ द्धर्यश्व हर्यश्व॒ रिष्ये॒द् रिष्ये᳚ द्धर्यश्व पी॒तः पी॒तो ह॑र्यश्व॒ रिष्ये॒द् रिष्ये᳚ द्धर्यश्व पी॒तः । \newline
24. ह॒र्य॒श्व॒ पी॒तः पी॒तो ह॑र्यश्व हर्यश्व पी॒तः । \newline
25. ह॒र्य॒श्वेति॑ हरि - अ॒श्व॒ । \newline
26. पी॒त इति॑ पी॒तः । \newline
27. अ॒यं ॅयो यो॑ ऽय म॒यं ॅयः सोमः॒ सोमो॒ यो॑ ऽय म॒यं ॅयः सोमः॑ । \newline
28. यः सोमः॒ सोमो॒ यो यः सोमो॒ न्यधा॑यि॒ न्यधा॑यि॒ सोमो॒ यो यः सोमो॒ न्यधा॑यि । \newline
29. सोमो॒ न्यधा॑यि॒ न्यधा॑यि॒ सोमः॒ सोमो॒ न्यधा᳚य्य॒स्मे अ॒स्मे न्यधा॑यि॒ सोमः॒ सोमो॒ न्यधा᳚य्य॒स्मे । \newline
30. न्यधा᳚य्य॒स्मे अ॒स्मे न्यधा॑यि॒ न्यधा᳚य्य॒स्मे तस्मै॒ तस्मा॑ अ॒स्मे न्यधा॑यि॒ न्यधा᳚य्य॒स्मे तस्मै᳚ । \newline
31. न्यधा॒यीति॑ नि - अधा॑यि । \newline
32. अ॒स्मे तस्मै॒ तस्मा॑ अ॒स्मे अ॒स्मे तस्मा॒ इन्द्र॒ मिन्द्र॒म् तस्मा॑ अ॒स्मे अ॒स्मे तस्मा॒ इन्द्र᳚म् । \newline
33. अ॒स्मे इत्य॒स्मे । \newline
34. तस्मा॒ इन्द्र॒ मिन्द्र॒म् तस्मै॒ तस्मा॒ इन्द्र॑म् प्र॒तिर॑म् प्र॒तिर॒ मिन्द्र॒म् तस्मै॒ तस्मा॒ इन्द्र॑म् प्र॒तिर᳚म् । \newline
35. इन्द्र॑म् प्र॒तिर॑म् प्र॒तिर॒ मिन्द्र॒ मिन्द्र॑म् प्र॒तिर॑ मेम्येमि प्र॒तिर॒ मिन्द्र॒ मिन्द्र॑म् प्र॒तिर॑ मेमि । \newline
36. प्र॒तिर॑ मेम्येमि प्र॒तिर॑म् प्र॒तिर॑ मे॒म्य च्छा च्छै॑मि प्र॒तिर॑म् प्र॒तिर॑ मे॒म्यच्छ॑ । \newline
37. प्र॒तिर॒मिति॑ प्र - तिर᳚म् । \newline
38. ए॒म्य च्छा च्छै᳚म्ये॒ म्यच्छ॑ । \newline
39. अच्छेत्यच्छ॑ । \newline
40. आपा᳚न्तमन्यु स्तृ॒पल॑प्रभर्मा तृ॒पल॑प्रभ॒र्मा ऽऽपा᳚न्तमन्यु॒ रापा᳚न्तमन्यु स्तृ॒पल॑प्रभर्मा॒ धुनि॒र् धुनि॑ स्तृ॒पल॑प्रभ॒र्मा ऽऽपा᳚न्तमन्यु॒ रापा᳚न्तमन्यु स्तृ॒पल॑प्रभर्मा॒ धुनिः॑ । \newline
41. आपा᳚न्तमन्यु॒रित्यापा᳚न्त - म॒न्युः॒ । \newline
42. तृ॒पल॑प्रभर्मा॒ धुनि॒र् धुनि॑ स्तृ॒पल॑प्रभर्मा तृ॒पल॑प्रभर्मा॒ धुनिः॒ शिमी॑वा॒ञ् छिमी॑वा॒न् धुनि॑ स्तृ॒पल॑प्रभर्मा तृ॒पल॑प्रभर्मा॒ धुनिः॒ शिमी॑वान् । \newline
43. तृ॒पल॑प्रभ॒र्मेति॑ तृ॒पल॑ - प्र॒भ॒र्मा॒ । \newline
44. धुनिः॒ शिमी॑वा॒ञ् छिमी॑वा॒न् धुनि॒र् धुनिः॒ शिमी॑वा॒ञ् छरु॑मा॒ञ् छरु॑मा॒ञ् छिमी॑वा॒न् धुनि॒र् धुनिः॒ शिमी॑वा॒ञ् छरु॑मान् । \newline
45. शिमी॑वा॒ञ् छरु॑मा॒ञ् छरु॑मा॒ञ् छिमी॑वा॒ञ् छिमी॑वा॒ञ् छरु॑माꣳ ऋजी॒ ष्यृ॑जी॒षी शरु॑मा॒ञ् छिमी॑वा॒ञ् छिमी॑वा॒ञ् छरु॑माꣳ ऋजी॒षी । \newline
46. शरु॑माꣳ ऋजी॒ ष्यृ॑जी॒षी शरु॑मा॒ञ् छरु॑माꣳ ऋजी॒षी । \newline
47. शरु॑मा॒निति॒ शरु॑ - मा॒न् । \newline
48. ऋ॒जी॒षीत्यृ॑जी॒षी । \newline
49. सोमो॒ विश्वा॑नि॒ विश्वा॑नि॒ सोमः॒ सोमो॒ विश्वा᳚ न्यत॒सा ऽत॒सा विश्वा॑नि॒ सोमः॒ सोमो॒ विश्वा᳚ न्यत॒सा । \newline
50. विश्वा᳚ न्यत॒सा ऽत॒सा विश्वा॑नि॒ विश्वा᳚ न्यत॒सा वना॑नि॒ वना᳚ न्यत॒सा विश्वा॑नि॒ विश्वा᳚ न्यत॒सा वना॑नि । \newline
51. अ॒त॒सा वना॑नि॒ वना᳚ न्यत॒सा ऽत॒सा वना॑नि॒ न न वना᳚ न्यत॒सा ऽत॒सा वना॑नि॒ न । \newline
52. वना॑नि॒ न न वना॑नि॒ वना॑नि॒ नार्वाग॒र्वाङ् न वना॑नि॒ वना॑नि॒ नार्वाक् । \newline
53. नार्वाग॒र्वाङ् न नार्वागिन्द्र॒ मिन्द्र॑ म॒र्वाङ् न नार्वागिन्द्र᳚म् । \newline
54. अ॒र्वागिन्द्र॒ मिन्द्र॑ म॒र्वा ग॒र्वा गिन्द्र॑म् प्रति॒माना॑नि प्रति॒माना॒नीन्द्र॑ म॒र्वा ग॒र्वा गिन्द्र॑म् प्रति॒माना॑नि । \newline
55. इन्द्र॑म् प्रति॒माना॑नि प्रति॒माना॒नीन्द्र॒ मिन्द्र॑म् प्रति॒माना॑नि देभुर् देभुः प्रति॒माना॒नीन्द्र॒ मिन्द्र॑म् प्रति॒माना॑नि देभुः । \newline
56. प्र॒ति॒माना॑नि देभुर् देभुः प्रति॒माना॑नि प्रति॒माना॑नि देभुः । \newline
57. प्र॒ति॒माना॒नीति॑ प्रति - माना॑नि । \newline
58. दे॒भु॒रिति॑ देभुः । \newline
59. प्र सु॑वा॒नः सु॑वा॒नः प्र प्र सु॑वा॒नः सोमः॒ सोमः॑ सुवा॒नः प्र प्र सु॑वा॒नः सोमः॑ । \newline
\pagebreak
\markright{ TS 2.2.12.4  \hfill https://www.vedavms.in \hfill}
\addcontentsline{toc}{section}{ TS 2.2.12.4 }
\section*{ TS 2.2.12.4 }

\textbf{TS 2.2.12.4 } \newline
\textbf{Samhita Paata} \newline

सु॑वा॒नः सोम॑ ऋत॒युश्चि॑के॒तेन्द्रा॑य॒ ब्रह्म॑ ज॒मद॑ग्नि॒रर्चन्न्॑ । वृषा॑ य॒न्ताऽसि॒ शव॑सस्तु॒रस्या॒न्तर्य॑च्छ गृण॒ते ध॒र्त्रं दृꣳ॑ह ॥स॒बाध॑स्ते॒ मदं॑ च शुष्म॒यं च॒ ब्रह्म॒ नरो᳚ ब्रह्म॒कृतः॑ सपर्यन्न् । अ॒र्को वा॒ यत् तु॒रते॒ सोम॑चक्षा॒स्तत्रेदिन्द्रो॑ दधते पृ॒थ्सु तु॒र्यां ॥वष॑ट् ते विष्णवा॒स आ कृ॑णोमि॒ तन्मे॑ जुषस्व शिपिविष्ट ह॒व्यं । \newline

\textbf{Pada Paata} \newline

सु॒वा॒नः । सोमः॑ । ऋ॒त॒युरित्यृ॑त - युः । चि॒के॒त॒ । इन्द्रा॑य । ब्रह्म॑ । ज॒मद॑ग्निः । अर्चन्न्॑ ॥ वृषा᳚ । य॒न्ता । अ॒सि॒ । शव॑सः । तु॒रस्य॑ । अ॒न्तः । य॒च्छ॒ । गृ॒ण॒ते । ध॒र्त्रम् । दृꣳ॒॒ह॒ ॥ स॒बाध॒ इति॑ स - बाधः॑ । ते॒ । मद᳚म् । च॒ । शु॒ष्म॒यम् । च॒ । ब्रह्म॑ । नरः॑ । ब्र॒ह्म॒कृत॒ इति॑ ब्रह्म - कृतः॑ । स॒प॒र्य॒न्न् ॥ अ॒र्कः । वा॒ । यत् । तु॒रते᳚ । सोम॑चक्षा॒ इति॒ सोम॑ - च॒क्षाः॒ । तत्र॑ । इत् । इन्द्रः॑ । द॒ध॒ते॒ । पृ॒थ्स्विति॑ पृत् - सु । तु॒र्याम् ॥ वष॑ट् । ते॒ । वि॒ष्णो॒ । आ॒सः । एति॑ । कृ॒णो॒मि॒ । तत् । मे॒ । जु॒ष॒स्व॒ । शि॒पि॒वि॒ष्टेति॑ शिपि - वि॒ष्ट॒ । ह॒व्यम् ॥  \newline


\textbf{Krama Paata} \newline

सु॒वा॒नः सोमः॑ । सोम॑ ऋत॒युः । ऋ॒त॒युश्चि॑केत । ऋ॒त॒युरित्यृ॑त - युः । चि॒के॒तेन्द्रा॑य । इन्द्रा॑य॒ ब्रह्म॑ । ब्रह्म॑ ज॒मद॑ग्निः । ज॒मद॑ग्नि॒रर्चन्न्॑ । अर्च॒न्नित्यर्चन्न्॑ ॥ वृषा॑ य॒न्ता । य॒न्ता ऽसि॑ । अ॒सि॒ शव॑सः । शव॑सस्तु॒रस्य॑ । तु॒रस्या॒न्तः । अ॒न्तर् य॑च्छ । य॒च्छ॒ गृ॒ण॒ते । गृ॒ण॒ते ध॒र्त्रम् । ध॒र्त्रम् दृꣳ॑ह । दृꣳ॒॒हेति॑ दृꣳह ॥? स॒बाध॑स्ते । स॒बाध॒ इति॑ स - बाधः॑ । ते॒ मद᳚म् । मद॑म् च । च॒ शु॒ष्म॒यम् । शु॒ष्म॒यम् च॑ । च॒ ब्रह्म॑ । ब्रह्म॒ नरः॑ । नरो᳚ ब्रह्म॒कृतः॑ । ब्र॒ह्म॒कृतः॑ सपर्यन्न् । ब्र॒ह्म॒कृत॒ इति॑ ब्रह्म - कृतः॑ । स॒प॒र्य॒न्निति॑ सपर्यन्न् ॥ अ॒र्को वा᳚ । वा॒ यत् । यत् तु॒रते᳚ । तु॒रते॒ सोम॑चक्षाः । सोम॑चक्षा॒स्तत्र॑ । सोम॑चक्षा॒ इति॒ सोम॑ - च॒क्षाः॒ । तत्रेत् । इदिन्द्रः॑ । इन्द्रो॑ दधते । द॒ध॒ते॒ पृ॒थ्सु । पृ॒थ्सु तु॒र्याम् । पृ॒थ्स्विति॑ पृत् - सु । तु॒र्यामिति॑ तु॒र्याम् ॥ वष॑ट् ते । ते॒ वि॒ष्णो॒ । वि॒ष्ण॒वा॒सः । आ॒स आ । आ कृ॑णोमि । कृ॒णो॒मि॒ तत् । तन् मे᳚ । मे॒ जु॒ष॒स्व॒ । जु॒ष॒स्व॒ शि॒पि॒वि॒ष्ट॒ । शि॒पि॒वि॒ष्ट॒ ह॒व्यम् । शि॒पि॒वि॒ष्टेति॑ शिपि - वि॒ष्ट॒ । ह॒व्यमिति॑ ह॒व्यम् । \newline

\textbf{Jatai Paata} \newline

1. सु॒वा॒नः सोमः॒ सोमः॑ सुवा॒नः सु॑वा॒नः सोमः॑ । \newline
2. सोम॑ ऋत॒युर्. ऋ॑त॒युः सोमः॒ सोम॑ ऋत॒युः । \newline
3. ऋ॒त॒यु श्चि॑केत चिकेत र्त॒युर्. ऋ॑त॒यु श्चि॑केत । \newline
4. ऋ॒त॒युरित्यृ॑त - युः । \newline
5. चि॒के॒ते न्द्रा॒ये न्द्रा॑य चिकेत चिके॒ते न्द्रा॑य । \newline
6. इन्द्रा॑य॒ ब्रह्म॒ ब्रह्मे न्द्रा॒ये न्द्रा॑य॒ ब्रह्म॑ । \newline
7. ब्रह्म॑ ज॒मद॑ग्निर् ज॒मद॑ग्नि॒र् ब्रह्म॒ ब्रह्म॑ ज॒मद॑ग्निः । \newline
8. ज॒मद॑ग्नि॒ रर्च॒न् नर्च॑न् ज॒मद॑ग्निर् ज॒मद॑ग्नि॒ रर्चन्न्॑ । \newline
9. अर्च॒न्नित्यर्चन्न्॑ । \newline
10. वृषा॑ य॒न्ता य॒न्ता वृषा॒ वृषा॑ य॒न्ता । \newline
11. य॒न्ता ऽस्य॑सि य॒न्ता य॒न्ता ऽसि॑ । \newline
12. अ॒सि॒ शव॑सः॒ शव॑सो ऽस्यसि॒ शव॑सः । \newline
13. शव॑स स्तु॒रस्य॑ तु॒रस्य॒ शव॑सः॒ शव॑स स्तु॒रस्य॑ । \newline
14. तु॒रस्या॒ न्त र॒न्त स्तु॒रस्य॑ तु॒रस्या॒न्तः । \newline
15. अ॒न्तर् य॑च्छ यच्छा॒न्त र॒न्तर् य॑च्छ । \newline
16. य॒च्छ॒ गृ॒ण॒ते गृ॑ण॒ते य॑च्छ यच्छ गृण॒ते । \newline
17. गृ॒ण॒ते ध॒र्त्रम् ध॒र्त्रम् गृ॑ण॒ते गृ॑ण॒ते ध॒र्त्रम् । \newline
18. ध॒र्त्रम् दृꣳ॑ह दृꣳह ध॒र्त्रम् ध॒र्त्रम् दृꣳ॑ह । \newline
19. दृꣳ॒॒हेति॑ दृꣳह । \newline
20. स॒बाध॑ स्ते ते स॒बाधः॑ स॒बाध॑ स्ते । \newline
21. स॒बाध॒ इति॑ स - बाधः॑ । \newline
22. ते॒ मद॒म् मद॑म् ते ते॒ मद᳚म् । \newline
23. मद॑म् च च॒ मद॒म् मद॑म् च । \newline
24. च॒ शु॒ष्म॒यꣳ शु॑ष्म॒यम् च॑ च शुष्म॒यम् । \newline
25. शु॒ष्म॒यम् च॑ च शुष्म॒यꣳ शु॑ष्म॒यम् च॑ । \newline
26. च॒ ब्रह्म॒ ब्रह्म॑ च च॒ ब्रह्म॑ । \newline
27. ब्रह्म॒ नरो॒ नरो॒ ब्रह्म॒ ब्रह्म॒ नरः॑ । \newline
28. नरो᳚ ब्रह्म॒कृतो᳚ ब्रह्म॒कृतो॒ नरो॒ नरो᳚ ब्रह्म॒कृतः॑ । \newline
29. ब्र॒ह्म॒कृतः॑ सपर्यन् थ्सपर्यन् ब्रह्म॒कृतो᳚ ब्रह्म॒कृतः॑ सपर्यन्न् । \newline
30. ब्र॒ह्म॒कृत॒ इति॑ ब्रह्म - कृतः॑ । \newline
31. स॒प॒र्य॒न्निति॑ सपर्यन्न् । \newline
32. अ॒र्को वा॑ वा॒ ऽर्को॑ अ॒र्को वा᳚ । \newline
33. वा॒ यद् यद् वा॑ वा॒ यत् । \newline
34. यत् तु॒रते॑ तु॒रते॒ यद् यत् तु॒रते᳚ । \newline
35. तु॒रते॒ सोम॑चक्षाः॒ सोम॑चक्षा स्तु॒रते॑ तु॒रते॒ सोम॑चक्षाः । \newline
36. सोम॑चक्षा॒ स्तत्र॒ तत्र॒ सोम॑चक्षाः॒ सोम॑चक्षा॒ स्तत्र॑ । \newline
37. सोम॑चक्षा॒ इति॒ सोम॑ - च॒क्षाः॒ । \newline
38. तत्रे दित् तत्र॒ तत्रे त् । \newline
39. दिन्द्र॒ इन्द्र॒ इदि दिन्द्रः॑ । \newline
40. इन्द्रो॑ दधते दधत॒ इन्द्र॒ इन्द्रो॑ दधते । \newline
41. द॒ध॒ते॒ पृ॒थ्सु पृ॒थ्सु द॑धते दधते पृ॒थ्सु । \newline
42. पृ॒थ्सु तु॒र्याम् तु॒र्याम् पृ॒थ्सु पृ॒थ्सु॑ तु॒र्याम् । \newline
43. पृ॒थ्स्विति॑ पृत् - सु । \newline
44. तु॒र्यामिति॑ तु॒र्याम् । \newline
45. वष॑ट् ते ते॒ वष॒ड् वष॑ट् ते । \newline
46. ते॒ वि॒ष्णो॒ वि॒ष्णो॒ ते॒ ते॒ वि॒ष्णो॒ । \newline
47. वि॒ष्ण॒ वा॒स आ॒सो वि॑ष्णो विष्ण वा॒सः । \newline
48. आ॒स आ ऽऽस आ॒स आ । \newline
49. आ कृ॑णोमि कृणो॒म्या कृ॑णोमि । \newline
50. कृ॒णो॒मि॒ तत् तत् कृ॑णोमि कृणोमि॒ तत् । \newline
51. तन् मे॑ मे॒ तत् तन् मे᳚ । \newline
52. मे॒ जु॒ष॒स्व॒ जु॒ष॒स्व॒ मे॒ मे॒ जु॒ष॒स्व॒ । \newline
53. जु॒ष॒स्व॒ शि॒पि॒वि॒ष्ट॒ शि॒पि॒वि॒ष्ट॒ जु॒ष॒स्व॒ जु॒ष॒स्व॒ शि॒पि॒वि॒ष्ट॒ । \newline
54. शि॒पि॒वि॒ष्ट॒ ह॒व्यꣳ ह॒व्यꣳ शि॑पिविष्ट शिपिविष्ट ह॒व्यम् । \newline
55. शि॒पि॒वि॒ष्टेति॑ शिपि - वि॒ष्ट॒ । \newline
56. ह॒व्यमिति॑ ह॒व्यम् । \newline

\textbf{Ghana Paata } \newline

1. सु॒वा॒नः सोमः॒ सोमः॑ सुवा॒नः सु॑वा॒नः सोम॑ ऋत॒युर्. ऋ॑त॒युः सोमः॑ सुवा॒नः सु॑वा॒नः सोम॑ ऋत॒युः । \newline
2. सोम॑ ऋत॒युर्. ऋ॑त॒युः सोमः॒ सोम॑ ऋत॒यु श्चि॑केत चिकेत र्त॒युः सोमः॒ सोम॑ ऋत॒यु श्चि॑केत । \newline
3. ऋ॒त॒यु श्चि॑केत चिकेत र्त॒युर्. ऋ॑त॒यु श्चि॑के॒ते न्द्रा॒ये न्द्रा॑य चिकेत र्त॒युर्. ऋ॑त॒यु श्चि॑के॒ते न्द्रा॑य । \newline
4. ऋ॒त॒युरित्यृ॑त - युः । \newline
5. चि॒के॒ते न्द्रा॒ये न्द्रा॑य चिकेत चिके॒ते न्द्रा॑य॒ ब्रह्म॒ ब्रह्मे न्द्रा॑य चिकेत चिके॒ते न्द्रा॑य॒ ब्रह्म॑ । \newline
6. इन्द्रा॑य॒ ब्रह्म॒ ब्रह्मे न्द्रा॒ये न्द्रा॑य॒ ब्रह्म॑ ज॒मद॑ग्निर् ज॒मद॑ग्नि॒र् ब्रह्मे न्द्रा॒ये न्द्रा॑य॒ ब्रह्म॑ ज॒मद॑ग्निः । \newline
7. ब्रह्म॑ ज॒मद॑ग्निर् ज॒मद॑ग्नि॒र् ब्रह्म॒ ब्रह्म॑ ज॒मद॑ग्नि॒ रर्च॒न् नर्च॑न् ज॒मद॑ग्नि॒र् ब्रह्म॒ ब्रह्म॑ ज॒मद॑ग्नि॒ रर्चन्न्॑ । \newline
8. ज॒मद॑ग्नि॒ रर्च॒न् नर्च॑न् ज॒मद॑ग्निर् ज॒मद॑ग्नि॒ रर्चन्न्॑ । \newline
9. अर्च॒न्नित्यर्चन्न्॑ । \newline
10. वृषा॑ य॒न्ता य॒न्ता वृषा॒ वृषा॑ य॒न्ता ऽस्य॑सि य॒न्ता वृषा॒ वृषा॑ य॒न्ता ऽसि॑ । \newline
11. य॒न्ता ऽस्य॑सि य॒न्ता य॒न्ता ऽसि॒ शव॑सः॒ शव॑सो ऽसि य॒न्ता य॒न्ता ऽसि॒ शव॑सः । \newline
12. अ॒सि॒ शव॑सः॒ शव॑सो ऽस्यसि॒ शव॑स स्तु॒रस्य॑ तु॒रस्य॒ शव॑सो ऽस्यसि॒ शव॑स स्तु॒रस्य॑ । \newline
13. शव॑स स्तु॒रस्य॑ तु॒रस्य॒ शव॑सः॒ शव॑स स्तु॒रस्या॒न्त र॒न्तस्तु॒रस्य॒ शव॑सः॒ शव॑स स्तु॒रस्या॒न्तः । \newline
14. तु॒रस्या॒न्त र॒न्त स्तु॒रस्य॑ तु॒रस्या॒न्तर् य॑च्छ यच्छा॒न्त स्तु॒रस्य॑ तु॒रस्या॒न्तर् य॑च्छ । \newline
15. अ॒न्तर् य॑च्छ यच्छा॒न्त र॒न्तर् य॑च्छ गृण॒ते गृ॑ण॒ते य॑च्छा॒न्त र॒न्तर् य॑च्छ गृण॒ते । \newline
16. य॒च्छ॒ गृ॒ण॒ते गृ॑ण॒ते य॑च्छ यच्छ गृण॒ते ध॒र्त्रम् ध॒र्त्रम् गृ॑ण॒ते य॑च्छ यच्छ गृण॒ते ध॒र्त्रम् । \newline
17. गृ॒ण॒ते ध॒र्त्रम् ध॒र्त्रम् गृ॑ण॒ते गृ॑ण॒ते ध॒र्त्रम् दृꣳ॑ह दृꣳह ध॒र्त्रम् गृ॑ण॒ते गृ॑ण॒ते ध॒र्त्रम् दृꣳ॑ह । \newline
18. ध॒र्त्रम् दृꣳ॑ह दृꣳह ध॒र्त्रम् ध॒र्त्रम् दृꣳ॑ह । \newline
19. दृꣳ॒॒हेति॑ दृꣳह । \newline
20. स॒बाध॑ स्ते ते स॒बाधः॑ स॒बाध॑ स्ते॒ मद॒म् मद॑म् ते स॒बाधः॑ स॒बाध॑ स्ते॒ मद᳚म् । \newline
21. स॒बाध॒ इति॑ स - बाधः॑ । \newline
22. ते॒ मद॒म् मद॑म् ते ते॒ मद॑म् च च॒ मद॑म् ते ते॒ मद॑म् च । \newline
23. मद॑म् च च॒ मद॒म् मद॑म् च शुष्म॒यꣳ शु॑ष्म॒यम् च॒ मद॒म् मद॑म् च शुष्म॒यम् । \newline
24. च॒ शु॒ष्म॒यꣳ शु॑ष्म॒यम् च॑ च शुष्म॒यम् च॑ च शुष्म॒यम् च॑ च शुष्म॒यम् च॑ । \newline
25. शु॒ष्म॒यम् च॑ च शुष्म॒यꣳ शु॑ष्म॒यम् च॒ ब्रह्म॒ ब्रह्म॑ च शुष्म॒यꣳ शु॑ष्म॒यम् च॒ ब्रह्म॑ । \newline
26. च॒ ब्रह्म॒ ब्रह्म॑ च च॒ ब्रह्म॒ नरो॒ नरो॒ ब्रह्म॑ च च॒ ब्रह्म॒ नरः॑ । \newline
27. ब्रह्म॒ नरो॒ नरो॒ ब्रह्म॒ ब्रह्म॒ नरो᳚ ब्रह्म॒कृतो᳚ ब्रह्म॒कृतो॒ नरो॒ ब्रह्म॒ ब्रह्म॒ नरो᳚ ब्रह्म॒कृतः॑ । \newline
28. नरो᳚ ब्रह्म॒कृतो᳚ ब्रह्म॒कृतो॒ नरो॒ नरो᳚ ब्रह्म॒कृतः॑ सपर्यन् थ्सपर्यन् ब्रह्म॒कृतो॒ नरो॒ नरो᳚ ब्रह्म॒कृतः॑ सपर्यन्न् । \newline
29. ब्र॒ह्म॒कृतः॑ सपर्यन् थ्सपर्यन् ब्रह्म॒कृतो᳚ ब्रह्म॒कृतः॑ सपर्यन्न् । \newline
30. ब्र॒ह्म॒कृत॒ इति॑ ब्रह्म - कृतः॑ । \newline
31. स॒प॒र्य॒न्निति॑ सपर्यन्न् । \newline
32. अ॒र्को वा॑ वा॒ ऽर्को॑ अ॒र्को वा॒ यद् यद् वा॒ ऽर्को॑ अ॒र्को वा॒ यत् । \newline
33. वा॒ यद् यद् वा॑ वा॒ यत् तु॒रते॑ तु॒रते॒ यद् वा॑ वा॒ यत् तु॒रते᳚ । \newline
34. यत् तु॒रते॑ तु॒रते॒ यद् यत् तु॒रते॒ सोम॑चक्षाः॒ सोम॑चक्षा स्तु॒रते॒ यद् यत् तु॒रते॒ सोम॑चक्षाः । \newline
35. तु॒रते॒ सोम॑चक्षाः॒ सोम॑चक्षा स्तु॒रते॑ तु॒रते॒ सोम॑चक्षा॒ स्तत्र॒ तत्र॒ सोम॑चक्षा स्तु॒रते॑ तु॒रते॒ सोम॑चक्षा॒ स्तत्र॑ । \newline
36. सोम॑चक्षा॒ स्तत्र॒ तत्र॒ सोम॑चक्षाः॒ सोम॑चक्षा॒ स्तत्रे दित् तत्र॒ सोम॑चक्षाः॒ सोम॑चक्षा॒ स्तत्रेत् । \newline
37. सोम॑चक्षा॒ इति॒ सोम॑ - च॒क्षाः॒ । \newline
38. तत्रे दित् तत्र॒ तत्रे दिन्द्र॒ इन्द्र॑ इत् तत्र॒ तत्रे दिन्द्रः॑ । \newline
39. इदिन्द्र॒ इन्द्र॒ इदि दिन्द्रो॑ दधते दधत॒ इन्द्र॒ इदि दिन्द्रो॑ दधते । \newline
40. इन्द्रो॑ दधते दधत॒ इन्द्र॒ इन्द्रो॑ दधते पृ॒थ्सु पृ॒थ्सु द॑धत॒ इन्द्र॒ इन्द्रो॑ दधते पृ॒थ्सु । \newline
41. द॒ध॒ते॒ पृ॒थ्सु पृ॒थ्सु द॑धते दधते पृ॒थ्सु तु॒र्याम् तु॒र्याम् पृ॒थ्सु दधते दधते पृ॒थ्सु तु॒र्याम् । \newline
42. पृ॒थ्सु तु॒र्याम् तु॒र्याम् पृ॒थ्सु पृ॒थ्सु तु॒र्याम् । \newline
43. पृ॒थ्स्विति॑ पृत् - सु । \newline
44. तु॒र्यामिति॑ तु॒र्याम् । \newline
45. वष॑ट् ते ते॒ वष॒ड् वष॑ट् ते विष्णो विष्णो ते॒ वष॒ड् वष॑ट् ते विष्णो । \newline
46. ते॒ वि॒ष्णो॒ वि॒ष्णो॒ ते॒ ते॒ वि॒ष्ण॒ वा॒स आ॒सो वि॑ष्णो ते ते विष्ण वा॒सः । \newline
47. वि॒ष्ण॒ वा॒स आ॒सो वि॑ष्णो विष्ण वा॒स आ ऽऽसो वि॑ष्णो विष्ण वा॒स आ । \newline
48. आ॒स आ ऽऽस आ॒स आ कृ॑णोमि कृणो॒म्या ऽऽस आ॒स आ कृ॑णोमि । \newline
49. आ कृ॑णोमि कृणो॒म्या कृ॑णोमि॒ तत् तत् कृ॑णो॒म्या कृ॑णोमि॒ तत् । \newline
50. कृ॒णो॒मि॒ तत् तत् कृ॑णोमि कृणोमि॒ तन् मे॑ मे॒ तत् कृ॑णोमि कृणोमि॒ तन् मे᳚ । \newline
51. तन् मे॑ मे॒ तत् तन् मे॑ जुषस्व जुषस्व मे॒ तत् तन् मे॑ जुषस्व । \newline
52. मे॒ जु॒ष॒स्व॒ जु॒ष॒स्व॒ मे॒ मे॒ जु॒ष॒स्व॒ शि॒पि॒वि॒ष्ट॒ शि॒पि॒वि॒ष्ट॒ जु॒ष॒स्व॒ मे॒ मे॒ जु॒ष॒स्व॒ शि॒पि॒वि॒ष्ट॒ । \newline
53. जु॒ष॒स्व॒ शि॒पि॒वि॒ष्ट॒ शि॒पि॒वि॒ष्ट॒ जु॒ष॒स्व॒ जु॒ष॒स्व॒ शि॒पि॒वि॒ष्ट॒ ह॒व्यꣳ ह॒व्यꣳ शि॑पिविष्ट जुषस्व जुषस्व शिपिविष्ट ह॒व्यम् । \newline
54. शि॒पि॒वि॒ष्ट॒ ह॒व्यꣳ ह॒व्यꣳ शि॑पिविष्ट शिपिविष्ट ह॒व्यम् । \newline
55. शि॒पि॒वि॒ष्टेति॑ शिपि - वि॒ष्ट॒ । \newline
56. ह॒व्यमिति॑ ह॒व्यम् । \newline
\pagebreak
\markright{ TS 2.2.12.5  \hfill https://www.vedavms.in \hfill}
\addcontentsline{toc}{section}{ TS 2.2.12.5 }
\section*{ TS 2.2.12.5 }

\textbf{TS 2.2.12.5 } \newline
\textbf{Samhita Paata} \newline

वर्द्ध॑न्तु त्वा सुष्टु॒तयो॒ गिरो॑ मे यू॒यं पा॑त स्व॒स्तिभिः॒ सदा॑ नः ॥प्र तत् ते॑ अ॒द्य शि॑पिविष्ट॒ नामा॒ऽर्यः शꣳ॑ सामि व॒युना॑नि वि॒द्वान् । तन्त्वा॑ गृणामि त॒वस॒मत॑वीया॒न् क्षय॑न्तम॒स्य रज॑सः परा॒के ॥किमित् ते॑ विष्णो परि॒चक्ष्यं॑ भू॒त् प्र यद्व॑व॒क्षे शि॑पिवि॒ष्टो अ॑स्मि । मा वर्पो॑ अ॒स्मदप॑ गूह ए॒तद्यद॒न्यरू॑पः समि॒थे ब॒भूथ॑ ॥ \newline

\textbf{Pada Paata} \newline

वर्द्ध॑न्तु । त्वा॒ । सु॒ष्टु॒तय॒ इति॑ सु - स्तु॒तयः॑ । गिरः॑ । मे॒ । यू॒यम् । पा॒त॒ । स्व॒स्तिभि॒रिति॑ स्व॒स्ति - भिः॒ । सदा᳚ । नः॒ ॥ प्रेति॑ । तत् । ते॒ । अ॒द्य । शि॒पि॒वि॒ष्टेति॑ शिपि - वि॒ष्ट॒ । नाम॑ । अ॒र्यः । शꣳ॒॒सा॒मि॒ । व॒युना॑नि । वि॒द्वान् ॥ तम् । त्वा॒ । गृ॒णा॒मि॒ । त॒वस᳚म् । अत॑वीयान् । क्षय॑न्तम् । अ॒स्य । रज॑सः । प॒रा॒के ॥ किम् । इत् । ते॒ । वि॒ष्णो॒ इति॑ । प॒रि॒चक्ष्य॒मिति॑ परि - चक्ष्य᳚म् । भू॒त् । प्रेति॑ । यत् । व॒व॒क्षे । शि॒पि॒वि॒ष्ट इति॑ शिपि - वि॒ष्टः । अ॒स्मि॒ ॥ मा । वर्पः॑ । अ॒स्मत् । अपेति॑ । गू॒हः॒ । ए॒तत् । यत् । अ॒न्यरू॑प॒ इत्य॒न्य - रू॒पः॒ । स॒मि॒थ इति॑ सं - इ॒थे । ब॒भूथ॑ ॥  \newline


\textbf{Krama Paata} \newline

वर्द्ध॑न्तु त्वा । त्वा॒ सु॒ष्टु॒तयः॑ । सु॒ष्टु॒तयो॒ गिरः॑ । सु॒ष्टु॒तय॒ इति॑ सु - स्तु॒तयः॑ । गिरो॑ मे । मे॒ यू॒यम् । यू॒यम् पा॑त । पा॒त॒ स्व॒स्तिभिः॑ । स्व॒स्तिभिः॒ सदा᳚ । स्व॒स्तिभि॒रिति॑ स्व॒स्ति - भिः॒ । सदा॑ नः । न॒ इति॑ नः ॥ प्र तत् । तत् ते᳚ । ते॒ अ॒द्य । अ॒द्य शि॑पिविष्ट । शि॒पि॒वि॒ष्ट॒ नाम॑ । शि॒पि॒वि॒ष्टेति॑ शिपि - वि॒ष्ट॒ । नामा॒र्यः । अ॒र्यः शꣳ॑सामि । शꣳ॒॒सा॒मि॒ व॒युना॑नि । व॒युना॑नि वि॒द्वान् । वि॒द्वानिति॑ वि॒द्वान् ॥ तम् त्वा᳚ । त्वा॒ गृ॒णा॒मि॒ । गृ॒णा॒मि॒ त॒वस᳚म् । त॒वस॒मत॑वीयान् । अत॑वीया॒न् क्षय॑न्तम् । क्षय॑न्तम॒स्य । अ॒स्य रज॑सः । रज॑सः परा॒के । प॒रा॒क इति॑ परा॒के ॥ किमित् । इत् ते᳚ । ते॒ वि॒ष्णो॒ । वि॒ष्णो॒ प॒रि॒चक्ष्य᳚म् । वि॒ष्णो॒ इति॑ विष्णो । प॒रि॒चक्ष्य॑म् भूत् । प॒रि॒चक्ष्य॒मिति॑ परि - चक्ष्य᳚म् । भू॒त् प्र । प्र यत् । 
यद् व॑व॒क्षे । व॒व॒क्षे शि॑पिवि॒ष्टः । शि॒पि॒वि॒ष्टो अ॑स्मि । शि॒पि॒वि॒ष्ट इति॑ शिपि - वि॒ष्टः । अ॒स्मीत्य॑स्मि ॥ मा वर्पः॑ । वर्पो॑ अ॒स्मत् । अ॒स्मदप॑ । अप॑ गूहः । गू॒ह॒ ए॒तत् । ए॒तद् यत् । यद॒न्यरू॑पः । अ॒न्यरू॑पः समि॒थे । अ॒न्यरू॑प॒ इत्य॒न्य - रू॒पः॒ । स॒मि॒थे ब॒भूथ॑ । स॒मि॒थ इति॑ सं - इ॒थे । ब॒भूथेति॑ ब॒भूथ॑ । \newline

\textbf{Jatai Paata} \newline

1. वर्द्ध॑न्तु त्वा त्वा॒ वर्द्ध॑न्तु॒ वर्द्ध॑न्तु त्वा । \newline
2. त्वा॒ सु॒ष्टु॒तयः॑ सुष्टु॒तय॑ स्त्वा त्वा सुष्टु॒तयः॑ । \newline
3. सु॒ष्टु॒तयो॒ गिरो॒ गिरः॑ सुष्टु॒तयः॑ सुष्टु॒तयो॒ गिरः॑ । \newline
4. सु॒ष्टु॒तय॒ इति॑ सु - स्तु॒तयः॑ । \newline
5. गिरो॑ मे मे॒ गिरो॒ गिरो॑ मे । \newline
6. मे॒ यू॒यं ॅयू॒यम् मे॑ मे यू॒यम् । \newline
7. यू॒यम् पा॑त पात यू॒यं ॅयू॒यम् पा॑त । \newline
8. पा॒त॒ स्व॒स्तिभिः॑ स्व॒स्तिभिः॑ पात पात स्व॒स्तिभिः॑ । \newline
9. स्व॒स्तिभिः॒ सदा॒ सदा᳚ स्व॒स्तिभिः॑ स्व॒स्तिभिः॒ सदा᳚ । \newline
10. स्व॒स्तिभि॒रिति॑ स्व॒स्ति - भिः॒ । \newline
11. सदा॑ नो नः॒ सदा॒ सदा॑ नः । \newline
12. न॒ इति॑ नः । \newline
13. प्र तत् तत् प्र प्र तत् । \newline
14. तत् ते॑ ते॒ तत् तत् ते᳚ । \newline
15. ते॒ अ॒द्याद्य ते॑ ते अ॒द्य । \newline
16. अ॒द्य शि॑पिविष्ट शिपिविष्टा॒ द्याद्य शि॑पिविष्ट । \newline
17. शि॒पि॒वि॒ष्ट॒ नाम॒ नाम॑ शिपिविष्ट शिपिविष्ट॒ नाम॑ । \newline
18. शि॒पि॒वि॒ष्टेति॑ शिपि - वि॒ष्ट॒ । \newline
19. नामा॒र्यो अ॒र्यो नाम॒ नामा॒र्यः । \newline
20. अ॒र्यः शꣳ॑सामि शꣳसा म्य॒र्यो अ॒र्यः शꣳ॑सामि । \newline
21. शꣳ॒॒सा॒मि॒ व॒युना॑नि व॒युना॑नि शꣳसामि शꣳसामि व॒युना॑नि । \newline
22. व॒युना॑नि वि॒द्वान्. वि॒द्वान्. व॒युना॑नि व॒युना॑नि वि॒द्वान् । \newline
23. वि॒द्वानिति॑ वि॒द्वान् । \newline
24. तम् त्वा᳚ त्वा॒ तम् तम् त्वा᳚ । \newline
25. त्वा॒ गृ॒णा॒मि॒ गृ॒णा॒मि॒ त्वा॒ त्वा॒ गृ॒णा॒मि॒ । \newline
26. गृ॒णा॒मि॒ त॒वस॑म् त॒वस॑म् गृणामि गृणामि त॒वस᳚म् । \newline
27. त॒वस॒ मत॑वीया॒ नत॑वीयान् त॒वस॑म् त॒वस॒ मत॑वीयान् । \newline
28. अत॑वीया॒न् क्षय॑न्त॒म् क्षय॑न्त॒ मत॑वीया॒ नत॑वीया॒न् क्षय॑न्तम् । \newline
29. क्षय॑न्त म॒स्यास्य क्षय॑न्त॒म् क्षय॑न्त म॒स्य । \newline
30. अ॒स्य रज॑सो॒ रज॑सो अ॒स्यास्य रज॑सः । \newline
31. रज॑सः परा॒के प॑रा॒के रज॑सो॒ रज॑सः परा॒के । \newline
32. प॒रा॒क इति॑ परा॒के । \newline
33. कि मिदित् किम् कि मित् । \newline
34. इत् ते॑ त॒ इदित् ते᳚ । \newline
35. ते॒ वि॒ष्णो॒ वि॒ष्णो॒ ते॒ ते॒ वि॒ष्णो॒ । \newline
36. वि॒ष्णो॒ प॒रि॒चक्ष्य॑म् परि॒चक्ष्यं॑ ॅविष्णो विष्णो परि॒चक्ष्य᳚म् । \newline
37. वि॒ष्णो॒ इति॑ विष्णो । \newline
38. प॒रि॒चक्ष्य॑म् भूद् भूत् परि॒चक्ष्य॑म् परि॒चक्ष्य॑म् भूत् । \newline
39. प॒रि॒चक्ष्य॒मिति॑ परि - चक्ष्य᳚म् । \newline
40. भू॒त् प्र प्र भू᳚द् भू॒त् प्र । \newline
41. प्र यद् यत् प्र प्र यत् । \newline
42. यद् व॑व॒क्षे व॑व॒क्षे यद् यद् व॑व॒क्षे । \newline
43. व॒व॒क्षे शि॑पिवि॒ष्टः शि॑पिवि॒ष्टो व॑व॒क्षे व॑व॒क्षे शि॑पिवि॒ष्टः । \newline
44. शि॒पि॒वि॒ष्टो अ॑स्म्यस्मि शिपिवि॒ष्टः शि॑पिवि॒ष्टो अ॑स्मि । \newline
45. शि॒पि॒वि॒ष्ट इति॑ शिपि - वि॒ष्टः । \newline
46. अ॒स्मीत्य॑स्मि । \newline
47. मा वर्पो॒ वर्पो॒ मा मा वर्पः॑ । \newline
48. वर्पो॑ अ॒स्म द॒स्मद् वर्पो॒ वर्पो॑ अ॒स्मत् । \newline
49. अ॒स्म दपा पा॒स्म द॒स्म दप॑ । \newline
50. अप॑ गूहो गूहो॒ अपाप॑ गूहः । \newline
51. गू॒ह॒ ए॒त दे॒तद् गू॑हो गूह ए॒तत् । \newline
52. ए॒तद् यद् यदे॒त दे॒तद् यत् । \newline
53. यद॒न्यरू॑पो अ॒न्यरू॑पो॒ यद् यद॒न्यरू॑पः । \newline
54. अ॒न्यरू॑पः समि॒थे स॑मि॒थे अ॒न्यरू॑पो अ॒न्यरू॑पः समि॒थे । \newline
55. अ॒न्यरू॑प॒ इत्य॒न्य - रू॒पः॒ । \newline
56. स॒मि॒थे ब॒भूथ॑ ब॒भूथ॑ समि॒थे स॑मि॒थे ब॒भूथ॑ । \newline
57. स॒मि॒थ इति॑ सं - इ॒थे । \newline
58. ब॒भूथेति॑ ब॒भूथ॑ । \newline

\textbf{Ghana Paata } \newline

1. वर्द्ध॑न्तु त्वा त्वा॒ वर्द्ध॑न्तु॒ वर्द्ध॑न्तु त्वा सुष्टु॒तयः॑ सुष्टु॒तय॑ स्त्वा॒ वर्द्ध॑न्तु॒ वर्द्ध॑न्तु त्वा सुष्टु॒तयः॑ । \newline
2. त्वा॒ सु॒ष्टु॒तयः॑ सुष्टु॒तय॑ स्त्वा त्वा सुष्टु॒तयो॒ गिरो॒ गिरः॑ सुष्टु॒तय॑ स्त्वा त्वा सुष्टु॒तयो॒ गिरः॑ । \newline
3. सु॒ष्टु॒तयो॒ गिरो॒ गिरः॑ सुष्टु॒तयः॑ सुष्टु॒तयो॒ गिरो॑ मे मे॒ गिरः॑ सुष्टु॒तयः॑ सुष्टु॒तयो॒ गिरो॑ मे । \newline
4. सु॒ष्टु॒तय॒ इति॑ सु - स्तु॒तयः॑ । \newline
5. गिरो॑ मे मे॒ गिरो॒ गिरो॑ मे यू॒यं ॅयू॒यम् मे॒ गिरो॒ गिरो॑ मे यू॒यम् । \newline
6. मे॒ यू॒यं ॅयू॒यम् मे॑ मे यू॒यम् पा॑त पात यू॒यम् मे॑ मे यू॒यम् पा॑त । \newline
7. यू॒यम् पा॑त पात यू॒यं ॅयू॒यम् पा॑त स्व॒स्तिभिः॑ स्व॒स्तिभिः॑ पात यू॒यं ॅयू॒यम् पा॑त स्व॒स्तिभिः॑ । \newline
8. पा॒त॒ स्व॒स्तिभिः॑ स्व॒स्तिभिः॑ पात पात स्व॒स्तिभिः॒ सदा॒ सदा᳚ स्व॒स्तिभिः॑ पात पात स्व॒स्तिभिः॒ सदा᳚ । \newline
9. स्व॒स्तिभिः॒ सदा॒ सदा᳚ स्व॒स्तिभिः॑ स्व॒स्तिभिः॒ सदा॑ नो नः॒ सदा᳚ स्व॒स्तिभिः॑ स्व॒स्तिभिः॒ सदा॑ नः । \newline
10. स्व॒स्तिभि॒रिति॑ स्व॒स्ति - भिः॒ । \newline
11. सदा॑ नो नः॒ सदा॒ सदा॑ नः । \newline
12. न॒ इति॑ नः । \newline
13. प्र तत् तत् प्र प्र तत् ते॑ ते॒ तत् प्र प्र तत् ते᳚ । \newline
14. तत् ते॑ ते॒ तत् तत् ते॑ अ॒द्याद्य ते॒ तत् तत् ते॑ अ॒द्य । \newline
15. ते॒ अ॒द्याद्य ते॑ ते अ॒द्य शि॑पिविष्ट शिपिविष्टा॒द्य ते॑ ते अ॒द्य शि॑पिविष्ट । \newline
16. अ॒द्य शि॑पिविष्ट शिपिविष्टा॒द्याद्य शि॑पिविष्ट॒ नाम॒ नाम॑ शिपिविष्टा॒द्याद्य शि॑पिविष्ट॒ नाम॑ । \newline
17. शि॒पि॒वि॒ष्ट॒ नाम॒ नाम॑ शिपिविष्ट शिपिविष्ट॒ नामा॒र्यो अ॒र्यो नाम॑ शिपिविष्ट शिपिविष्ट॒ नामा॒र्यः । \newline
18. शि॒पि॒वि॒ष्टेति॑ शिपि - वि॒ष्ट॒ । \newline
19. नामा॒र्यो अ॒र्यो नाम॒ नामा॒र्यः शꣳ॑सामि शꣳसाम्य॒र्यो नाम॒ नामा॒र्यः शꣳ॑सामि । \newline
20. अ॒र्यः शꣳ॑सामि शꣳसा म्य॒र्यो अ॒र्यः शꣳ॑सामि व॒युना॑नि व॒युना॑नि शꣳसाम्य॒र्यो अ॒र्यः शꣳ॑सामि व॒युना॑नि । \newline
21. शꣳ॒॒सा॒मि॒ व॒युना॑नि व॒युना॑नि शꣳसामि शꣳसामि व॒युना॑नि वि॒द्वान्. वि॒द्वान्. व॒युना॑नि शꣳसामि शꣳसामि व॒युना॑नि वि॒द्वान् । \newline
22. व॒युना॑नि वि॒द्वान्. वि॒द्वान्. व॒युना॑नि व॒युना॑नि वि॒द्वान् । \newline
23. वि॒द्वानिति॑ वि॒द्वान् । \newline
24. तम् त्वा᳚ त्वा॒ तम् तम् त्वा॑ गृणामि गृणामि त्वा॒ तम् तम् त्वा॑ गृणामि । \newline
25. त्वा॒ गृ॒णा॒मि॒ गृ॒णा॒मि॒ त्वा॒ त्वा॒ गृ॒णा॒मि॒ त॒वस॑म् त॒वस॑म् गृणामि त्वा त्वा गृणामि त॒वस᳚म् । \newline
26. गृ॒णा॒मि॒ त॒वस॑म् त॒वस॑म् गृणामि गृणामि त॒वस॒ मत॑वीया॒ नत॑वीयान् त॒वस॑म् गृणामि गृणामि त॒वस॒ मत॑वीयान् । \newline
27. त॒वस॒ मत॑वीया॒ नत॑वीयान् त॒वस॑म् त॒वस॒ मत॑वीया॒न् क्षय॑न्त॒म् क्षय॑न्त॒ मत॑वीयान् त॒वस॑म् त॒वस॒ मत॑वीया॒न् क्षय॑न्तम् । \newline
28. अत॑वीया॒न् क्षय॑न्त॒म् क्षय॑न्त॒ मत॑वीया॒ नत॑वीया॒न् क्षय॑न्त म॒स्यास्य क्षय॑न्त॒ मत॑वीया॒ नत॑वीया॒न् क्षय॑न्त म॒स्य । \newline
29. क्षय॑न्त म॒स्यास्य क्षय॑न्त॒म् क्षय॑न्त म॒स्य रज॑सो॒ रज॑सो अ॒स्य क्षय॑न्त॒म् क्षय॑न्त म॒स्य रज॑सः । \newline
30. अ॒स्य रज॑सो॒ रज॑सो अ॒स्यास्य रज॑सः परा॒के प॑रा॒के रज॑सो अ॒स्यास्य रज॑सः परा॒के । \newline
31. रज॑सः परा॒के प॑रा॒के रज॑सो॒ रज॑सः परा॒के । \newline
32. प॒रा॒क इति॑ परा॒के । \newline
33. कि मिदित् किम् कि मित् ते॑ त॒ इत् किम् कि मित् ते᳚ । \newline
34. इत् ते॑ त॒ इदित् ते॑ विष्णो विष्णो त॒ इदित् ते॑ विष्णो । \newline
35. ते॒ वि॒ष्णो॒ वि॒ष्णो॒ ते॒ ते॒ वि॒ष्णो॒ प॒रि॒चक्ष्य॑म् परि॒चक्ष्यं॑ ॅविष्णो ते ते विष्णो परि॒चक्ष्य᳚म् । \newline
36. वि॒ष्णो॒ प॒रि॒चक्ष्य॑म् परि॒चक्ष्यं॑ ॅविष्णो विष्णो परि॒चक्ष्य॑म् भूद् भूत् परि॒चक्ष्यं॑ ॅविष्णो विष्णो परि॒चक्ष्य॑म् भूत् । \newline
37. वि॒ष्णो॒ इति॑ विष्णो । \newline
38. प॒रि॒चक्ष्य॑म् भूद् भूत् परि॒चक्ष्य॑म् परि॒चक्ष्य॑म् भू॒त् प्र प्र भू᳚त् परि॒चक्ष्य॑म् परि॒चक्ष्य॑म् भू॒त् प्र । \newline
39. प॒रि॒चक्ष्य॒मिति॑ परि - चक्ष्य᳚म् । \newline
40. भू॒त् प्र प्र भू᳚द् भू॒त् प्र यद् यत् प्र भू᳚द् भू॒त् प्र यत् । \newline
41. प्र यद् यत् प्र प्र यद् व॑व॒क्षे व॑व॒क्षे यत् प्र प्र यद् व॑व॒क्षे । \newline
42. यद् व॑व॒क्षे व॑व॒क्षे यद् यद् व॑व॒क्षे शि॑पिवि॒ष्टः शि॑पिवि॒ष्टो व॑व॒क्षे यद् यद् व॑व॒क्षे शि॑पिवि॒ष्टः । \newline
43. व॒व॒क्षे शि॑पिवि॒ष्टः शि॑पिवि॒ष्टो व॑व॒क्षे व॑व॒क्षे शि॑पिवि॒ष्टो अ॑स्म्यस्मि शिपिवि॒ष्टो व॑व॒क्षे व॑व॒क्षे शि॑पिवि॒ष्टो अ॑स्मि । \newline
44. शि॒पि॒वि॒ष्टो अ॑स्म्यस्मि शिपिवि॒ष्टः शि॑पिवि॒ष्टो अ॑स्मि । \newline
45. शि॒पि॒वि॒ष्ट इति॑ शिपि - वि॒ष्टः । \newline
46. अ॒स्मीत्य॑स्मि । \newline
47. मा वर्पो॒ वर्पो॒ मा मा वर्पो॑ अ॒स्म द॒स्मद् वर्पो॒ मा मा वर्पो॑ अ॒स्मत् । \newline
48. वर्पो॑ अ॒स्म द॒स्मद् वर्पो॒ वर्पो॑ अ॒स्म दपापा॒स्मद् वर्पो॒ वर्पो॑ अ॒स्मदप॑ । \newline
49. अ॒स्म दपापा॒स्म द॒स्मदप॑ गूहो गूहो॒ अपा॒स्म द॒स्मदप॑ गूहः । \newline
50. अप॑ गूहो गूहो॒ अपाप॑ गूह ए॒त दे॒तद् गू॑हो॒ अपाप॑ गूह ए॒तत् । \newline
51. गू॒ह॒ ए॒त दे॒तद् गू॑हो गूह ए॒तद् यद् यदे॒तद् गू॑हो गूह ए॒तद् यत् । \newline
52. ए॒तद् यद् यदे॒ तदे॒तद् यद॒न्यरू॑पो अ॒न्यरू॑पो॒ यदे॒ तदे॒तद् यद॒न्यरू॑पः । \newline
53. यद॒न्यरू॑पो अ॒न्यरू॑पो॒ यद् यद॒न्यरू॑पः समि॒थे स॑मि॒थे अ॒न्यरू॑पो॒ यद् यद॒न्यरू॑पः समि॒थे । \newline
54. अ॒न्यरू॑पः समि॒थे स॑मि॒थे अ॒न्यरू॑पो अ॒न्यरू॑पः समि॒थे ब॒भूथ॑ ब॒भूथ॑ समि॒थे अ॒न्यरू॑पो अ॒न्यरू॑पः समि॒थे ब॒भूथ॑ । \newline
55. अ॒न्यरू॑प॒ इत्य॒न्य - रू॒पः॒ । \newline
56. स॒मि॒थे ब॒भूथ॑ ब॒भूथ॑ समि॒थे स॑मि॒थे ब॒भूथ॑ । \newline
57. स॒मि॒थ इति॑ सं - इ॒थे । \newline
58. ब॒भूथेति॑ ब॒भूथ॑ । \newline
\pagebreak
\markright{ TS 2.2.12.6  \hfill https://www.vedavms.in \hfill}
\addcontentsline{toc}{section}{ TS 2.2.12.6 }
\section*{ TS 2.2.12.6 }

\textbf{TS 2.2.12.6 } \newline
\textbf{Samhita Paata} \newline

अग्ने॒ दा दा॒शुषे॑ र॒यिं ॅवी॒रव॑न्तं॒ परी॑णसं । शि॒शी॒हि नः॑ सूनु॒मतः॑ ॥दा नो॑ अग्ने श॒तिनो॒ दाः स॑ह॒स्रिणो॑ दु॒रो न वाजꣳ॒॒ श्रुत्या॒ अपा॑ वृधि । प्राची॒ द्यावा॑पृथि॒वी ब्रह्म॑णा कृधि॒ सुव॒र्ण शु॒क्रमु॒षसो॒ वि दि॑द्युतुः ॥अ॒ग्निर्दा॒ द्रवि॑णं ॅवी॒रपे॑शा अ॒ग्निर् ऋषिं॒ ॅयः स॒हस्रा॑ स॒नोति॑ । अ॒ग्निर्दि॒वि ह॒व्यमा त॑ताना॒ग्नेर्द्धामा॑नि॒ विभृ॑ता पुरु॒त्रा ॥ मा - [  ] \newline

\textbf{Pada Paata} \newline

अग्ने᳚ । दाः । दा॒शुषे᳚ । र॒यिम् । वी॒रव॑न्त॒मिति॑ वी॒र - व॒न्त॒म् । परी॑णस॒मिति॒ परि॑ - न॒स॒म् । शि॒शी॒हि । नः॒ । सू॒नु॒मत॒ इति॑ सूनु-मतः॑ ॥ दाः । नः॒ । अ॒ग्ने॒ । श॒तिनः॑ । दाः । स॒ह॒स्रिणः॑ । दु॒रः । न । वाज᳚म् । श्रुत्यै᳚ । अपेति॑ । वृ॒धि॒ ॥ प्राची॒ इति॑ । द्यावा॑पृथि॒वी इति॒ द्यावा᳚ - पृ॒थि॒वी । ब्रह्म॑णा । कृ॒धि॒ । सुवः॑ । न । शु॒क्रम् । उ॒षसः॑ । वीति॑ । दि॒द्यु॒तुः॒ ॥ अ॒ग्निः । दाः॒ । द्रवि॑णम् । वी॒रपे॑शा॒ इति॑ वी॒र - पे॒शाः॒ । अ॒ग्निः । ऋषि᳚म् । यः । स॒हस्रा᳚ । स॒नोति॑ ॥ अ॒ग्निः । दि॒वि । ह॒व्यम् । एति॑ । त॒ता॒न॒ । अ॒ग्नेः । धामा॑नि । विभृ॒तेति॒ वि - भृ॒ता॒ । पु॒रु॒त्रेति॑ पुरु - त्रा ॥ मा ।  \newline


\textbf{Krama Paata} \newline

अग्ने॒ दाः । दा दा॒शुषे᳚ । दा॒शुषे॑ र॒यिम् । र॒यिं ॅवी॒रव॑न्तम् । वी॒रव॑न्त॒म् परी॑णसम् । वी॒रव॑न्त॒मिति॑ वी॒र - व॒न्त॒म् । परी॑णस॒मिति॒ परि॑ - न॒स॒म् ॥ शि॒शी॒हि नः॑ । नः॒ सू॒नु॒मतः॑ । सू॒नु॒मत॒ इति॑ सूनु - मतः॑ ॥ दा नः॑ । नो॒ अ॒ग्ने॒ । अ॒ग्ने॒ श॒तिनः॑ । श॒तिनो॒ दाः । दाः स॑ह॒स्रिणः॑ । स॒ह॒स्रिणो॑ दु॒रः । दु॒रो न । न वाज᳚म् । वाजꣳ॒॒ श्रुत्यै᳚ । श्रुत्या॒ अप॑ । अपा॑ वृधि । वृ॒धीति॑ वृधि ॥ प्राची॒ द्यावा॑पृथि॒वी । प्राची॒ इति॒ प्राची᳚ । द्यावा॑पृथि॒वी ब्रह्म॑णा । द्यावा॑पृथि॒वी इति॒ द्यावा᳚ - पृ॒थि॒वी । ब्रह्म॑णा कृधि । कृ॒धि॒ सुवः॑ । सुव॒र् न । न शु॒क्रम् । शु॒क्रमु॒षसः॑ । उ॒षसो॒ वि । वि दि॑द्युतुः । दि॒द्यु॒तु॒रिति॑ दिद्युतुः ॥ अ॒ग्निर् दाः᳚ । दा॒ द्रवि॑णम् । द्रवि॑णं ॅवी॒रपे॑शाः । वी॒रपे॑शा अ॒ग्निः । वी॒रपे॑शा॒ इति॑ वी॒र - पे॒शाः॒ । अ॒ग्निर्. ऋषि᳚म् । ऋषिं॒ ॅयः । यः स॒हस्रा᳚ । स॒हस्रा॑ स॒नोति॑ । स॒नोतीति॑ स॒नोति॑ ॥ अ॒ग्निर् दि॒वि । दि॒वि ह॒व्यम् । ह॒व्यमा । आ त॑तान । त॒ता॒ना॒ग्नेः । अ॒ग्नेर् धामा॑नि । धामा॑नि॒ विभृ॑ता । विभृ॑ता पुरु॒त्रा । विभृ॒तेति॒ वि - भृ॒ता॒ । पु॒रु॒त्रेति॑ पुरु - त्रा ॥ मा नः॑ \newline

\textbf{Jatai Paata} \newline

1. अग्ने॒ दा दा अग्ने ऽग्ने॒ दाः । \newline
2. दा दा॒शुषे॑ दा॒शुषे॒ दा दा दा॒शुषे᳚ । \newline
3. दा॒शुषे॑ र॒यिꣳ र॒यिम् दा॒शुषे॑ दा॒शुषे॑ र॒यिम् । \newline
4. र॒यिं ॅवी॒रव॑न्तं ॅवी॒रव॑न्तꣳ र॒यिꣳ र॒यिं ॅवी॒रव॑न्तम् । \newline
5. वी॒रव॑न्त॒म् परी॑णस॒म् परी॑णसं ॅवी॒रव॑न्तं ॅवी॒रव॑न्त॒म् परी॑णसम् । \newline
6. वी॒रव॑न्त॒मिति॑ वी॒र - व॒न्त॒म् । \newline
7. परी॑णस॒मिति॒ परि॑ - न॒स॒म् । \newline
8. शि॒शी॒हि नो॑ नः शिशी॒हि शि॑शी॒हि नः॑ । \newline
9. नः॒ सू॒नु॒मतः॑ सूनु॒मतो॑ नो नः सूनु॒मतः॑ । \newline
10. सू॒नु॒मत॒ इति॑ सूनु - मतः॑ । \newline
11. दा नो॑ नो॒ दा दा नः॑ । \newline
12. नो॒ अ॒ग्ने॒ अ॒ग्ने॒ नो॒ नो॒ अ॒ग्ने॒ । \newline
13. अ॒ग्ने॒ श॒तिनः॑ श॒तिनो॑ अग्ने अग्ने श॒तिनः॑ । \newline
14. श॒तिनो॒ दा दाः श॒तिनः॑ श॒तिनो॒ दाः । \newline
15. दाः स॑ह॒स्रिणः॑ सह॒स्रिणो॒ दा दाः स॑ह॒स्रिणः॑ । \newline
16. स॒ह॒स्रिणो॑ दु॒रो दु॒रः स॑ह॒स्रिणः॑ सह॒स्रिणो॑ दु॒रः । \newline
17. दु॒रो न न दु॒रो दु॒रो न । \newline
18. न वाजं॒ ॅवाज॒म् न न वाज᳚म् । \newline
19. वाजꣳ॒॒ श्रुत्यै॒ श्रुत्यै॒ वाजं॒ ॅवाजꣳ॒॒ श्रुत्यै᳚ । \newline
20. श्रुत्या॒ अपाप॒ श्रुत्यै॒ श्रुत्या॒ अप॑ । \newline
21. अपा॑ वृधि वृ॒ध्यपापा॑ वृधि । \newline
22. वृ॒धीति॑ वृधि । \newline
23. प्राची॒ द्यावा॑पृथि॒वी द्यावा॑पृथि॒वी प्राची॒ प्राची॒ द्यावा॑पृथि॒वी । \newline
24. प्राची॒ इति॒ प्राची᳚ । \newline
25. द्यावा॑पृथि॒वी ब्रह्म॑णा॒ ब्रह्म॑णा॒ द्यावा॑पृथि॒वी द्यावा॑पृथि॒वी ब्रह्म॑णा । \newline
26. द्यावा॑पृथि॒वी इति॒ द्यावा᳚ - पृ॒थि॒वी । \newline
27. ब्रह्म॑णा कृधि कृधि॒ ब्रह्म॑णा॒ ब्रह्म॑णा कृधि । \newline
28. कृ॒धि॒ सुवः॒ सुव॑ स्कृधि कृधि॒ सुवः॑ । \newline
29. सुव॒र् ण न सुवः॒ सुव॒र् ण । \newline
30. न शु॒क्रꣳ शु॒क्रम् न न शु॒क्रम् । \newline
31. शु॒क्र मु॒षस॑ उ॒षसः॑ शु॒क्रꣳ शु॒क्र मु॒षसः॑ । \newline
32. उ॒षसो॒वि व्यु॑षस॑ उ॒षसो॒वि । \newline
33. वि दि॑द्युतुर् दिद्युतु॒र् वि वि दि॑द्युतुः । \newline
34. दि॒द्यु॒तु॒रिति॑ दिद्युतुः । \newline
35. अ॒ग्निर् दा॑ दा अ॒ग्नि र॒ग्निर् दाः᳚ । \newline
36. दा॒ द्रवि॑ण॒म् द्रवि॑णम् दा दा॒ द्रवि॑णम् । \newline
37. द्रवि॑णं ॅवी॒रपे॑शा वी॒रपे॑शा॒ द्रवि॑ण॒म् द्रवि॑णं ॅवी॒रपे॑शाः । \newline
38. वी॒रपे॑शा अ॒ग्निर॒ग्निर् वी॒रपे॑शा वी॒रपे॑शा अ॒ग्निः । \newline
39. वी॒रपे॑शा॒ इति॑ वी॒र - पे॒शाः॒ । \newline
40. अ॒ग्निर्. ऋषि॒ मृषि॑ म॒ग्नि र॒ग्निर्. ऋषि᳚म् । \newline
41. ऋषिं॒ ॅयो य ऋषि॒ मृषिं॒ ॅयः । \newline
42. यः स॒हस्रा॑ स॒हस्रा॒ यो यः स॒हस्रा᳚ । \newline
43. स॒हस्रा॑ स॒नोति॑ स॒नोति॑ स॒हस्रा॑ स॒हस्रा॑ स॒नोति॑ । \newline
44. स॒नोतीति॑ स॒नोति॑ । \newline
45. अ॒ग्निर् दि॒वि दि॒व्य॑ग्नि र॒ग्निर् दि॒वि । \newline
46. दि॒वि ह॒व्यꣳ ह॒व्यम् दि॒वि दि॒वि ह॒व्यम् । \newline
47. ह॒व्य मा ह॒व्यꣳ ह॒व्य मा । \newline
48. आ त॑तान तता॒ना त॑तान । \newline
49. त॒ता॒ना॒ ग्ने र॒ग्ने स्त॑तान तताना॒ ग्नेः । \newline
50. अ॒ग्नेर् धामा॑नि॒ धामा᳚ न्य॒ग्ने र॒ग्नेर् धामा॑नि । \newline
51. धामा॑नि॒ विभृ॑ता॒ विभृ॑ता॒ धामा॑नि॒ धामा॑नि॒ विभृ॑ता । \newline
52. विभृ॑ता पुरु॒त्रा पु॑रु॒त्रा विभृ॑ता॒ विभृ॑ता पुरु॒त्रा । \newline
53. विभृ॒तेति॒ वि - भृ॒ता॒ । \newline
54. पु॒रु॒त्रेति॑ पुरु - त्रा । \newline
55. मा नो॑ नो॒ मा मा नः॑ । \newline

\textbf{Ghana Paata } \newline

1. अग्ने॒ दा दा अग्ने ऽग्ने॒ दा दा॒शुषे॑ दा॒शुषे॒ दा अग्ने ऽग्ने॒ दा दा॒शुषे᳚ । \newline
2. दा दा॒शुषे॑ दा॒शुषे॒ दा दा दा॒शुषे॑ र॒यिꣳ र॒यिम् दा॒शुषे॒ दा दा दा॒शुषे॑ र॒यिम् । \newline
3. दा॒शुषे॑ र॒यिꣳ र॒यिम् दा॒शुषे॑ दा॒शुषे॑ र॒यिं ॅवी॒रव॑न्तं ॅवी॒रव॑न्तꣳ र॒यिम् दा॒शुषे॑ दा॒शुषे॑ र॒यिं ॅवी॒रव॑न्तम् । \newline
4. र॒यिं ॅवी॒रव॑न्तं ॅवी॒रव॑न्तꣳ र॒यिꣳ र॒यिं ॅवी॒रव॑न्त॒म् परी॑णस॒म् परी॑णसं ॅवी॒रव॑न्तꣳ र॒यिꣳ र॒यिं ॅवी॒रव॑न्त॒म् परी॑णसम् । \newline
5. वी॒रव॑न्त॒म् परी॑णस॒म् परी॑णसं ॅवी॒रव॑न्तं ॅवी॒रव॑न्त॒म् परी॑णसम् । \newline
6. वी॒रव॑न्त॒मिति॑ वी॒र - व॒न्त॒म् । \newline
7. परी॑णस॒मिति॒ परि॑ - न॒स॒म् । \newline
8. शि॒शी॒हि नो॑ नः शिशी॒हि शि॑शी॒हि नः॑ सूनु॒मतः॑ सूनु॒मतो॑ नः शिशी॒हि शि॑शी॒हि नः॑ सूनु॒मतः॑ । \newline
9. नः॒ सू॒नु॒मतः॑ सूनु॒मतो॑ नो नः सूनु॒मतः॑ । \newline
10. सू॒नु॒मत॒ इति॑ सूनु - मतः॑ । \newline
11. दा नो॑ नो॒ दा दा नो॑ अग्ने अग्ने नो॒ दा दा नो॑ अग्ने । \newline
12. नो॒ अ॒ग्ने॒ अ॒ग्ने॒ नो॒ नो॒ अ॒ग्ने॒ श॒तिनः॑ श॒तिनो॑ अग्ने नो नो अग्ने श॒तिनः॑ । \newline
13. अ॒ग्ने॒ श॒तिनः॑ श॒तिनो॑ अग्ने अग्ने श॒तिनो॒ दा दाः श॒तिनो॑ अग्ने अग्ने श॒तिनो॒ दाः । \newline
14. श॒तिनो॒ दा दाः श॒तिनः॑ श॒तिनो॒ दाः स॑ह॒स्रिणः॑ सह॒स्रिणो॒ दाः श॒तिनः॑ श॒तिनो॒ दाः स॑ह॒स्रिणः॑ । \newline
15. दाः स॑ह॒स्रिणः॑ सह॒स्रिणो॒ दा दाः स॑ह॒स्रिणो॑ दु॒रो दु॒रः स॑ह॒स्रिणो॒ दा दाः स॑ह॒स्रिणो॑ दु॒रः । \newline
16. स॒ह॒स्रिणो॑ दु॒रो दु॒रः स॑ह॒स्रिणः॑ सह॒स्रिणो॑ दु॒रो न न दु॒रः स॑ह॒स्रिणः॑ सह॒स्रिणो॑ दु॒रो न । \newline
17. दु॒रो न न दु॒रो दु॒रो न वाजं॒ ॅवाज॒म् न दु॒रो दु॒रो न वाज᳚म् । \newline
18. न वाजं॒ ॅवाज॒म् न न वाजꣳ॒॒ श्रुत्यै॒ श्रुत्यै॒ वाज॒म् न न वाजꣳ॒॒ श्रुत्यै᳚ । \newline
19. वाजꣳ॒॒ श्रुत्यै॒ श्रुत्यै॒ वाजं॒ ॅवाजꣳ॒॒ श्रुत्या॒ अपाप॒ श्रुत्यै॒ वाजं॒ ॅवाजꣳ॒॒ श्रुत्या॒ अप॑ । \newline
20. श्रुत्या॒ अपाप॒ श्रुत्यै॒ श्रुत्या॒ अपा॑ वृधि वृ॒ध्यप॒ श्रुत्यै॒ श्रुत्या॒ अपा॑ वृधि । \newline
21. अपा॑ वृधि वृ॒ध्यपापा॑ वृधि । \newline
22. वृ॒धीति॑ वृधि । \newline
23. प्राची॒ द्यावा॑पृथि॒वी द्यावा॑पृथि॒वी प्राची॒ प्राची॒ द्यावा॑पृथि॒वी ब्रह्म॑णा॒ ब्रह्म॑णा॒ द्यावा॑पृथि॒वी प्राची॒ प्राची॒ द्यावा॑पृथि॒वी ब्रह्म॑णा । \newline
24. प्राची॒ इति॒ प्राची᳚ । \newline
25. द्यावा॑पृथि॒वी ब्रह्म॑णा॒ ब्रह्म॑णा॒ द्यावा॑पृथि॒वी द्यावा॑पृथि॒वी ब्रह्म॑णा कृधि कृधि॒ ब्रह्म॑णा॒ द्यावा॑पृथि॒वी द्यावा॑पृथि॒वी ब्रह्म॑णा कृधि । \newline
26. द्यावा॑पृथि॒वी इति॒ द्यावा᳚ - पृ॒थि॒वी । \newline
27. ब्रह्म॑णा कृधि कृधि॒ ब्रह्म॑णा॒ ब्रह्म॑णा कृधि॒ सुवः॒ सुव॑ स्कृधि॒ ब्रह्म॑णा॒ ब्रह्म॑णा कृधि॒ सुवः॑ । \newline
28. कृ॒धि॒ सुवः॒ सुव॑ स्कृधि कृधि॒ सुव॒र् ण न सुव॑ स्कृधि कृधि॒ सुव॒र् ण । \newline
29. सुव॒र् ण न सुवः॒ सुव॒र् ण शु॒क्रꣳ शु॒क्रम् न सुवः॒ सुव॒र् ण शु॒क्रम् । \newline
30. न शु॒क्रꣳ शु॒क्रम् न न शु॒क्र मु॒षस॑ उ॒षसः॑ शु॒क्रम् न न शु॒क्र मु॒षसः॑ । \newline
31. शु॒क्र मु॒षस॑ उ॒षसः॑ शु॒क्रꣳ शु॒क्र मु॒षसो॒ वि व्यु॑षसः॑ शु॒क्रꣳ शु॒क्र मु॒षसो॒ वि । \newline
32. उ॒षसो॒ वि व्यु॑षस॑ उ॒षसो॒ वि दि॑द्युतुर् दिद्युतुर् व्यु॒षस॑ उ॒षसो॒ वि दि॑द्युतुः । \newline
33. वि दि॑द्युतुर् दिद्युतु॒र् वि वि दि॑द्युतुः । \newline
34. दि॒द्यु॒तु॒रिति॑ दिद्युतुः । \newline
35. अ॒ग्निर् दा॑ दा अ॒ग्नि र॒ग्निर् दा॒ द्रवि॑ण॒म् द्रवि॑णम् दा अ॒ग्नि र॒ग्निर् दा॒ द्रवि॑णम् । \newline
36. दा॒ द्रवि॑ण॒म् द्रवि॑णम् दा दा॒ द्रवि॑णं ॅवी॒रपे॑शा वी॒रपे॑शा॒ द्रवि॑णम् दा दा॒ द्रवि॑णं ॅवी॒रपे॑शाः । \newline
37. द्रवि॑णं ॅवी॒रपे॑शा वी॒रपे॑शा॒ द्रवि॑ण॒म् द्रवि॑णं ॅवी॒रपे॑शा अ॒ग्नि र॒ग्निर् वी॒रपे॑शा॒ द्रवि॑ण॒म् द्रवि॑णं ॅवी॒रपे॑शा अ॒ग्निः । \newline
38. वी॒रपे॑शा अ॒ग्नि र॒ग्निर् वी॒रपे॑शा वी॒रपे॑शा अ॒ग्निर्. ऋषि॒ मृषि॑ म॒ग्निर् वी॒रपे॑शा वी॒रपे॑शा अ॒ग्निर्. ऋषि᳚म् । \newline
39. वी॒रपे॑शा॒ इति॑ वी॒र - पे॒शाः॒ । \newline
40. अ॒ग्निर्. ऋषि॒ मृषि॑ म॒ग्नि र॒ग्निर्. ऋषिं॒ ॅयो य ऋषि॑ म॒ग्नि र॒ग्निर्. ऋषिं॒ ॅयः । \newline
41. ऋषिं॒ ॅयो य ऋषि॒ मृषिं॒ ॅयः स॒हस्रा॑ स॒हस्रा॒ य ऋषि॒ मृषिं॒ ॅयः स॒हस्रा᳚ । \newline
42. यः स॒हस्रा॑ स॒हस्रा॒ यो यः स॒हस्रा॑ स॒नोति॑ स॒नोति॑ स॒हस्रा॒ यो यः स॒हस्रा॑ स॒नोति॑ । \newline
43. स॒हस्रा॑ स॒नोति॑ स॒नोति॑ स॒हस्रा॑ स॒हस्रा॑ स॒नोति॑ । \newline
44. स॒नोतीति॑ स॒नोति॑ । \newline
45. अ॒ग्निर् दि॒वि दि॒व्य॑ग्नि र॒ग्निर् दि॒वि ह॒व्यꣳ ह॒व्यम् दि॒व्य॑ग्नि र॒ग्निर् दि॒वि ह॒व्यम् । \newline
46. दि॒वि ह॒व्यꣳ ह॒व्यम् दि॒वि दि॒वि ह॒व्य मा ह॒व्यम् दि॒वि दि॒वि ह॒व्य मा । \newline
47. ह॒व्य मा ह॒व्यꣳ ह॒व्य मा त॑तान तता॒ना ह॒व्यꣳ ह॒व्य मा त॑तान । \newline
48. आ त॑तान तता॒ना त॑ताना॒ग्ने र॒ग्ने स्त॑ता॒ना त॑ताना॒ग्नेः । \newline
49. त॒ता॒ना॒ग्ने र॒ग्ने स्त॑तान तताना॒ग्नेर् धामा॑नि॒ धामा᳚ न्य॒ग्ने स्त॑तान तताना॒ग्नेर् धामा॑नि । \newline
50. अ॒ग्नेर् धामा॑नि॒ धामा᳚ न्य॒ग्ने र॒ग्नेर् धामा॑नि॒ विभृ॑ता॒ विभृ॑ता॒ धामा᳚ न्य॒ग्ने र॒ग्नेर् धामा॑नि॒ विभृ॑ता । \newline
51. धामा॑नि॒ विभृ॑ता॒ विभृ॑ता॒ धामा॑नि॒ धामा॑नि॒ विभृ॑ता पुरु॒त्रा पु॑रु॒त्रा विभृ॑ता॒ धामा॑नि॒ धामा॑नि॒ विभृ॑ता पुरु॒त्रा । \newline
52. विभृ॑ता पुरु॒त्रा पु॑रु॒त्रा विभृ॑ता॒ विभृ॑ता पुरु॒त्रा । \newline
53. विभृ॒तेति॒ वि - भृ॒ता॒ । \newline
54. पु॒रु॒त्रेति॑ पुरु - त्रा । \newline
55. मा नो॑ नो॒ मा मा नो॑ मर्द्धीर् मर्द्धीर् नो॒ मा मा नो॑ मर्द्धीः । \newline
\pagebreak
\markright{ TS 2.2.12.7  \hfill https://www.vedavms.in \hfill}
\addcontentsline{toc}{section}{ TS 2.2.12.7 }
\section*{ TS 2.2.12.7 }

\textbf{TS 2.2.12.7 } \newline
\textbf{Samhita Paata} \newline

नो॑ मर्धी॒ >6, रा तू भ॑र >7 ॥ घृ॒तं न पू॒तं त॒नूर॑रे॒पाः शुचि॒ हिर॑ण्यं । तत् ते॑ रु॒क्मो न रो॑चत स्वधावः ॥  उ॒भे सु॑श्चन्द्र स॒र्पिषो॒ दर्वी᳚ श्रीणीष आ॒सनि॑ । उ॒तो न॒ उत् पु॑पूर्या उ॒क्थेषु॑ शवसस्पत॒ इषꣳ॑ स्तो॒तृभ्य॒ आ भ॑र ॥ वायो॑ श॒तꣳ हरी॑णां ॅयु॒वस्व॒ पोष्या॑णां । उ॒त वा॑ ते सह॒स्रिणो॒ रथ॒ आ या॑तु॒ पाज॑सा ॥ प्र याभि॒ - [  ] \newline

\textbf{Pada Paata} \newline

नः॒ । म॒र्द्धीः॒ । एति॑ । तु । भ॒र॒ ॥ घृ॒तम् । न । पू॒तम् । त॒नूः । अ॒रे॒पाः । शुचि॑ । हिर॑ण्यम् ॥ तत् । ते॒ । रु॒क्मः । न । रो॒च॒त॒ । स्व॒धा॒व॒ इति॑ स्वधा - वः॒ ॥ उ॒भे इति॑ । सु॒श्च॒न्द्रेति॑ सु - च॒न्द्र॒ । स॒र्पिषः॑ । दर्वी॒ इति॑ । श्री॒णी॒षे॒ । आ॒सनि॑ ॥ उ॒तो इति॑ । नः॒ । उदिति॑ । पु॒पू॒र्याः॒ । उ॒क्थेषु॑ । श॒व॒सः॒ । प॒ते॒ । इष᳚म् । स्तो॒तृभ्य॒ इति॑ स्तो॒तृ-भ्यः॒ । एति॑ । भ॒र॒ ॥ वायो॒ इति॑ । श॒तम् । हरी॑णाम् । यु॒वस्व॑ । पोष्या॑णाम् ॥ उ॒त । वा॒ । ते॒ । स॒ह॒स्रिणः॑ । रथः॑ । एति॑ । या॒तु॒ । पाज॑सा ॥ प्रेति॑ । याभिः॑ ।  \newline


\textbf{Krama Paata} \newline

नो॒ म॒र्द्धीः॒ । म॒र्द्धी॒रा । आ तु । तू भ॑र । भ॒रेति॑ भर ॥ घृ॒तम् न । न पू॒तम् । पू॒तम् त॒नूः । त॒नूर॑रे॒पाः । अ॒रे॒पाः शुचि॑ । शुचि॒ हिर॑ण्यम् । हिर॑ण्य॒मिति॒ हिर॑ण्यम् ॥ तत् ते᳚ । ते॒ रु॒क्मः । रु॒क्मो न । न रो॑चत । रो॒च॒त॒ स्व॒धा॒वः॒ । स्व॒धा॒व॒ इति॑ स्वधावः ॥ उ॒भे सु॑श्चन्द्र । उ॒भे इत्यु॒भे । सु॒श्च॒न्द्र॒ स॒र्पिषः॑ । सु॒श्च॒न्द्रेति॑ सु - च॒न्द्र॒ । स॒र्पिषो॒ दर्वी᳚ । दर्वी᳚ श्रीणीषे । दर्वी॒ इति॒ दर्वी᳚ । श्री॒णी॒ष॒ आ॒सनि॑ । आ॒सनीत्या॒सनि॑ ॥ 
उ॒तो नः॑ । उ॒तो इत्यु॒तो । न॒ उत् । उत् पु॑पूर्याः । पु॒पू॒र्या॒ उ॒क्थेषु॑ । उ॒क्थेषु॑ शवसः । श॒व॒स॒स्प॒ते॒ । प॒त॒ इष᳚म् । इषꣳ॑ स्तो॒तृभ्यः॑ । स्तो॒तृभ्य॒ आ । स्तो॒तृभ्य॒ इति॑ स्तो॒तृ - भ्यः॒ । आ भ॑र । भ॒रेति॑ भर ॥ वायो॑ श॒तम् । वायो॒ इति॒ वायो᳚ । श॒तꣳ हरी॑णाम् । हरी॑णां ॅयु॒वस्व॑ । यु॒वस्व॒ पोष्या॑णाम् । पोष्या॑णा॒मिति॒ पोष्या॑णाम् ॥ उ॒त वा᳚ । वा॒ ते॒ । ते॒ स॒ह॒स्रिणः॑ । स॒ह॒स्रिणो॒ रथः॑ । रथ॒ आ । आ या॑तु । या॒तु॒ पाज॑सा । पाज॒सेति॒ पाज॑सा ॥ प्र याभिः॑ ( ) । याभि॒र् यासि॑ \newline

\textbf{Jatai Paata} \newline

1. नो॒ म॒र्द्धी॒र् म॒र्द्धी॒र् नो॒ नो॒ म॒र्द्धीः॒ । \newline
2. म॒र्द्धी॒रा म॑र्द्धीर् मर्द्धी॒रा । \newline
3. आ तु त्वा तु । \newline
4. तू भ॑र भर॒ तु तू भ॑र । \newline
5. भ॒रेति॑ भर । \newline
6. घृ॒तम् न न घृ॒तम् घृ॒तम् न । \newline
7. न पू॒तम् पू॒तम् न न पू॒तम् । \newline
8. पू॒तम् त॒नू स्त॒नूः पू॒तम् पू॒तम् त॒नूः । \newline
9. त॒नू र॑रे॒पा अ॑रे॒पा स्त॒नू स्त॒नू र॑रे॒पाः । \newline
10. अ॒रे॒पाः शुचि॒ शुच्य॑रे॒पा अ॑रे॒पाः शुचि॑ । \newline
11. शुचि॒ हिर॑ण्यꣳ॒॒ हिर॑ण्यꣳ॒॒ शुचि॒ शुचि॒ हिर॑ण्यम् । \newline
12. हिर॑ण्य॒मिति॒ हिर॑ण्यम् । \newline
13. तत् ते॑ ते॒ तत् तत् ते᳚ । \newline
14. ते॒ रु॒क्मो रु॒क्म स्ते॑ ते रु॒क्मः । \newline
15. रु॒क्मो न न रु॒क्मो रु॒क्मो न । \newline
16. न रो॑चत रोचत॒ न न रो॑चत । \newline
17. रो॒च॒त॒ स्व॒धा॒वः॒ स्व॒धा॒वो॒ रो॒च॒त॒ रो॒च॒त॒ स्व॒धा॒वः॒ । \newline
18. स्व॒धा॒व॒ इति॑ स्वधा - वः॒ । \newline
19. उ॒भे सु॑श्चन्द्र सुश्चन्द्रो॒भे उ॒भे सु॑श्चन्द्र । \newline
20. उ॒भे इत्यु॒भे । \newline
21. सु॒श्च॒न्द्र॒ स॒र्पिषः॑ स॒र्पिषः॑ सुश्चन्द्र सुश्चन्द्र स॒र्पिषः॑ । \newline
22. सु॒श्च॒न्द्रेति॑ सु - च॒न्द्र॒ । \newline
23. स॒र्पिषो॒ दर्वी॒ दर्वी॑ स॒र्पिषः॑ स॒र्पिषो॒ दर्वी᳚ । \newline
24. दर्वी᳚ श्रीणीषे श्रीणीषे॒ दर्वी॒ दर्वी᳚ श्रीणीषे । \newline
25. दर्वी॒ इति॒ दर्वी᳚ । \newline
26. श्री॒णी॒ष॒ आ॒स न्या॒सनि॑ श्रीणीषे श्रीणीष आ॒सनि॑ । \newline
27. आ॒सनीत्या॒सनि॑ । \newline
28. उ॒तो नो॑ न उ॒तो उ॒तो नः॑ । \newline
29. उ॒तो इत्यु॒तो । \newline
30. न॒ उदुन् नो॑ न॒ उत् । \newline
31. उत् पु॑पूर्याः पुपूर्या॒ उदुत् पु॑पूर्याः । \newline
32. पु॒पू॒र्या॒ उ॒क्थे षू॒क्थेषु॑ पुपूर्याः पुपूर्या उ॒क्थेषु॑ । \newline
33. उ॒क्थेषु॑ शवसः शवस उ॒क्थे षू॒क्थेषु॑ शवसः । \newline
34. श॒व॒स॒ स्प॒ते॒ प॒ते॒ श॒व॒सः॒ श॒व॒स॒ स्प॒ते॒ । \newline
35. प॒त॒ इष॒ मिष॑म् पते पत॒ इष᳚म् । \newline
36. इषꣳ॑ स्तो॒तृभ्यः॑ स्तो॒तृभ्य॒ इष॒ मिषꣳ॑ स्तो॒तृभ्यः॑ । \newline
37. स्तो॒तृभ्य॒ आ स्तो॒तृभ्यः॑ स्तो॒तृभ्य॒ आ । \newline
38. स्तो॒तृभ्य॒ इति॑ स्तो॒तृ - भ्यः॒ । \newline
39. आ भ॑र भ॒रा भ॑र । \newline
40. भ॒रेति॑ भर । \newline
41. वायो॑ श॒तꣳ श॒तं ॅवायो॒ वायो॑ श॒तम् । \newline
42. वायो॒ इति॒ वायो᳚ । \newline
43. श॒तꣳ हरी॑णाꣳ॒॒ हरी॑णाꣳ श॒तꣳ श॒तꣳ हरी॑णाम् । \newline
44. हरी॑णां ॅयु॒वस्व॑ यु॒वस्व॒ हरी॑णाꣳ॒॒ हरी॑णां ॅयु॒वस्व॑ । \newline
45. यु॒वस्व॒ पोष्या॑णा॒म् पोष्या॑णां ॅयु॒वस्व॑ यु॒वस्व॒ पोष्या॑णाम् । \newline
46. पोष्या॑णा॒मिति॒ पोष्या॑णाम् । \newline
47. उ॒त वा॑ वो॒तोत वा᳚ । \newline
48. वा॒ ते॒ ते॒ वा॒ वा॒ ते॒ । \newline
49. ते॒ स॒ह॒स्रिणः॑ सह॒स्रिण॑ स्ते ते सह॒स्रिणः॑ । \newline
50. स॒ह॒स्रिणो॒ रथो॒ रथः॑ सह॒स्रिणः॑ सह॒स्रिणो॒ रथः॑ । \newline
51. रथ॒ आ रथो॒ रथ॒ आ । \newline
52. आ या॑तु या॒त्वा या॑तु । \newline
53. या॒तु॒ पाज॑सा॒ पाज॑सा यातु यातु॒ पाज॑सा । \newline
54. पाज॒सेति॒ पाज॑सा । \newline
55. प्र याभि॒र् याभिः॒ प्र प्र याभिः॑ । \newline
56. याभि॒र् यासि॒ यासि॒ याभि॒र् याभि॒र् यासि॑ । \newline

\textbf{Ghana Paata } \newline

1. नो॒ म॒र्द्धी॒र् म॒र्द्धी॒र् नो॒ नो॒ म॒र्द्धी॒रा म॑र्द्धीर् नो नो मर्द्धी॒रा । \newline
2. म॒र्द्धी॒रा म॑र्द्धीर् मर्द्धी॒रा तु त्वा म॑र्द्धीर् मर्द्धी॒रा तु । \newline
3. आ तु त्वा तू भ॑र भर॒ त्वा तू भ॑र । \newline
4. तू भ॑र भर॒ तु तू भ॑र । \newline
5. भ॒रेति॑ भर । \newline
6. घृ॒तम् न न घृ॒तम् घृ॒तम् न पू॒तम् पू॒तम् न घृ॒तम् घृ॒तम् न पू॒तम् । \newline
7. न पू॒तम् पू॒तम् न न पू॒तम् त॒नू स्त॒नूः पू॒तम् न न पू॒तम् त॒नूः । \newline
8. पू॒तम् त॒नू स्त॒नूः पू॒तम् पू॒तम् त॒नू र॑रे॒पा अ॑रे॒पा स्त॒नूः पू॒तम् पू॒तम् त॒नू र॑रे॒पाः । \newline
9. त॒नूर॑रे॒पा अ॑रे॒पा स्त॒नू स्त॒नू र॑रे॒पाः शुचि॒ शुच्य॑रे॒पा स्त॒नू स्त॒नू र॑रे॒पाः शुचि॑ । \newline
10. अ॒रे॒पाः शुचि॒ शुच्य॑रे॒पा अ॑रे॒पाः शुचि॒ हिर॑ण्यꣳ॒॒ हिर॑ण्यꣳ॒॒ शुच्य॑रे॒पा अ॑रे॒पाः शुचि॒ हिर॑ण्यम् । \newline
11. शुचि॒ हिर॑ण्यꣳ॒॒ हिर॑ण्यꣳ॒॒ शुचि॒ शुचि॒ हिर॑ण्यम् । \newline
12. हिर॑ण्य॒मिति॒ हिर॑ण्यम् । \newline
13. तत् ते॑ ते॒ तत् तत् ते॑ रु॒क्मो रु॒क्म स्ते॒ तत् तत् ते॑ रु॒क्मः । \newline
14. ते॒ रु॒क्मो रु॒क्म स्ते॑ ते रु॒क्मो न न रु॒क्म स्ते॑ ते रु॒क्मो न । \newline
15. रु॒क्मो न न रु॒क्मो रु॒क्मो न रो॑चत रोचत॒ न रु॒क्मो रु॒क्मो न रो॑चत । \newline
16. न रो॑चत रोचत॒ न न रो॑चत स्वधावः स्वधावो रोचत॒ न न रो॑चत स्वधावः । \newline
17. रो॒च॒त॒ स्व॒धा॒वः॒ स्व॒धा॒वो॒ रो॒च॒त॒ रो॒च॒त॒ स्व॒धा॒वः॒ । \newline
18. स्व॒धा॒व॒ इति॑ स्वधा - वः॒ । \newline
19. उ॒भे सु॑श्चन्द्र सुश्चन्द्रो॒भे उ॒भे सु॑श्चन्द्र स॒र्पिषः॑ स॒र्पिषः॑ सुश्चन्द्रो॒भे उ॒भे सु॑श्चन्द्र स॒र्पिषः॑ । \newline
20. उ॒भे इत्यु॒भे । \newline
21. सु॒श्च॒न्द्र॒ स॒र्पिषः॑ स॒र्पिषः॑ सुश्चन्द्र सुश्चन्द्र स॒र्पिषो॒ दर्वी॒ दर्वी॑ स॒र्पिषः॑ सुश्चन्द्र सुश्चन्द्र स॒र्पिषो॒ दर्वी᳚ । \newline
22. सु॒श्च॒न्द्रेति॑ सु - च॒न्द्र॒ । \newline
23. स॒र्पिषो॒ दर्वी॒ दर्वी॑ स॒र्पिषः॑ स॒र्पिषो॒ दर्वी᳚ श्रीणीषे श्रीणीषे॒ दर्वी॑ स॒र्पिषः॑ स॒र्पिषो॒ दर्वी᳚ श्रीणीषे । \newline
24. दर्वी᳚ श्रीणीषे श्रीणीषे॒ दर्वी॒ दर्वी᳚ श्रीणीष आ॒स न्या॒सनि॑ श्रीणीषे॒ दर्वी॒ दर्वी᳚ श्रीणीष आ॒सनि॑ । \newline
25. दर्वी॒ इति॒ दर्वी᳚ । \newline
26. श्री॒णी॒ष॒ आ॒स न्या॒सनि॑ श्रीणीषे श्रीणीष आ॒सनि॑ । \newline
27. आ॒सनीत्या॒सनि॑ । \newline
28. उ॒तो नो॑ न उ॒तो उ॒तो न॒ उदुन् न॑ उ॒तो उ॒तो न॒ उत् । \newline
29. उ॒तो इत्यु॒तो । \newline
30. न॒ उदुन् नो॑ न॒ उत् पु॑पूर्याः पुपूर्या॒ उन् नो॑ न॒ उत् पु॑पूर्याः । \newline
31. उत् पु॑पूर्याः पुपूर्या॒ उदुत् पु॑पूर्या उ॒क्थे षू॒क्थेषु॑ पुपूर्या॒ उदुत् पु॑पूर्या उ॒क्थेषु॑ । \newline
32. पु॒पू॒र्या॒ उ॒क्थे षू॒क्थेषु॑ पुपूर्याः पुपूर्या उ॒क्थेषु॑ शवसः शवस उ॒क्थेषु॑ पुपूर्याः पुपूर्या उ॒क्थेषु॑ शवसः । \newline
33. उ॒क्थेषु॑ शवसः शवस उ॒क्थे षू॒क्थेषु॑ शवस स्पते पते शवस उ॒क्थे षू॒क्थेषु॑ शवस स्पते । \newline
34. श॒व॒स॒ स्प॒ते॒ प॒ते॒ श॒व॒सः॒ श॒व॒स॒ स्प॒त॒ इष॒ मिष॑म् पते शवसः शवस स्पत॒ इष᳚म् । \newline
35. प॒त॒ इष॒ मिष॑म् पते पत॒ इषꣳ॑ स्तो॒तृभ्यः॑ स्तो॒तृभ्य॒ इष॑म् पते पत॒ इषꣳ॑ स्तो॒तृभ्यः॑ । \newline
36. इषꣳ॑ स्तो॒तृभ्यः॑ स्तो॒तृभ्य॒ इष॒ मिषꣳ॑ स्तो॒तृभ्य॒ आ स्तो॒तृभ्य॒ इष॒ मिषꣳ॑ स्तो॒तृभ्य॒ आ । \newline
37. स्तो॒तृभ्य॒ आ स्तो॒तृभ्यः॑ स्तो॒तृभ्य॒ आ भ॑र भ॒रा स्तो॒तृभ्यः॑ स्तो॒तृभ्य॒ आ भ॑र । \newline
38. स्तो॒तृभ्य॒ इति॑ स्तो॒तृ - भ्यः॒ । \newline
39. आ भ॑र भ॒रा भ॑र । \newline
40. भ॒रेति॑ भर । \newline
41. वायो॑ श॒तꣳ श॒तं ॅवायो॒ वायो॑ श॒तꣳ हरी॑णाꣳ॒॒ हरी॑णाꣳ श॒तं ॅवायो॒ वायो॑ श॒तꣳ हरी॑णाम् । \newline
42. वायो॒ इति॒ वायो᳚ । \newline
43. श॒तꣳ हरी॑णाꣳ॒॒ हरी॑णाꣳ श॒तꣳ श॒तꣳ हरी॑णां ॅयु॒वस्व॑ यु॒वस्व॒ हरी॑णाꣳ श॒तꣳ श॒तꣳ हरी॑णां ॅयु॒वस्व॑ । \newline
44. हरी॑णां ॅयु॒वस्व॑ यु॒वस्व॒ हरी॑णाꣳ॒॒ हरी॑णां ॅयु॒वस्व॒ पोष्या॑णा॒म् पोष्या॑णां ॅयु॒वस्व॒ हरी॑णाꣳ॒॒ हरी॑णां ॅयु॒वस्व॒ पोष्या॑णाम् । \newline
45. यु॒वस्व॒ पोष्या॑णा॒म् पोष्या॑णां ॅयु॒वस्व॑ यु॒वस्व॒ पोष्या॑णाम् । \newline
46. पोष्या॑णा॒मिति॒ पोष्या॑णाम् । \newline
47. उ॒त वा॑ वो॒तोत वा॑ ते ते वो॒तोत वा॑ ते । \newline
48. वा॒ ते॒ ते॒ वा॒ वा॒ ते॒ स॒ह॒स्रिणः॑ सह॒स्रिण॑ स्ते वा वा ते सह॒स्रिणः॑ । \newline
49. ते॒ स॒ह॒स्रिणः॑ सह॒स्रिण॑ स्ते ते सह॒स्रिणो॒ रथो॒ रथः॑ सह॒स्रिण॑स्ते ते सह॒स्रिणो॒ रथः॑ । \newline
50. स॒ह॒स्रिणो॒ रथो॒ रथः॑ सह॒स्रिणः॑ सह॒स्रिणो॒ रथ॒ आ रथः॑ सह॒स्रिणः॑ सह॒स्रिणो॒ रथ॒ आ । \newline
51. रथ॒ आ रथो॒ रथ॒ आ या॑तु या॒त्वा रथो॒ रथ॒ आ या॑तु । \newline
52. आ या॑तु या॒त्वा या॑तु॒ पाज॑सा॒ पाज॑सा या॒त्वा या॑तु॒ पाज॑सा । \newline
53. या॒तु॒ पाज॑सा॒ पाज॑सा यातु यातु॒ पाज॑सा । \newline
54. पाज॒सेति॒ पाज॑सा । \newline
55. प्र याभि॒र् याभिः॒ प्र प्र याभि॒र् यासि॒ यासि॒ याभिः॒ प्र प्र याभि॒र् यासि॑ । \newline
56. याभि॒र् यासि॒ यासि॒ याभि॒र् याभि॒र् यासि॑ दा॒श्वाꣳस॑म् दा॒श्वाꣳसं॒ ॅयासि॒ याभि॒र् याभि॒र् यासि॑ दा॒श्वाꣳस᳚म् । \newline
\pagebreak
\markright{ TS 2.2.12.8  \hfill https://www.vedavms.in \hfill}
\addcontentsline{toc}{section}{ TS 2.2.12.8 }
\section*{ TS 2.2.12.8 }

\textbf{TS 2.2.12.8 } \newline
\textbf{Samhita Paata} \newline

-र्यासि॑ दा॒श्वाꣳ समच्छा॑ नि॒युद्भि॑-र्वायवि॒ष्टये॑ दुरो॒णे । नि नो॑ र॒यिꣳ सु॒भोज॑सं ॅयुवे॒ह नि वी॒रव॒द्-गव्य॒मश्वि॑यं च॒ राधः॑ ॥रे॒वती᳚र्नः सध॒माद॒ इन्द्रे॑ सन्तु तु॒विवा॑जाः । क्षु॒मन्तो॒ याभि॒र्मदे॑म ॥रे॒वाꣳ इद्रे॒वतः॑ स्तो॒ता स्यात् त्वाव॑तो म॒घोनः॑ । प्रेदु॑ हरिवः श्रु॒तस्य॑ ॥ \newline

\textbf{Pada Paata} \newline

यासि॑ । दा॒श्वाꣳस᳚म् । अच्छ॑ । नि॒युद्भि॒रिति॑ नि॒युत् - भिः॒ । वा॒यो॒ । इ॒ष्टये᳚ । दु॒रो॒ण इति॑ दुः - ओ॒ने ॥ नीति॑ । नः॒ । र॒यिम् । सु॒भोज॑स॒मिति॑ सु - भोज॑सम् । यु॒व॒ । इ॒ह । नीति॑ । वी॒रव॒दिति॑ वी॒र - व॒त् । गव्य᳚म् । अश्वि॑यम् । च॒ । राधः॑ ॥ रे॒वतीः᳚ । नः॒ । स॒ध॒माद॒ इति॑ सध - मादः॑ । इन्द्रे᳚ । स॒न्तु॒ । तु॒विवा॑जा॒ इति॑ तु॒वि - वा॒जाः॒ ॥ क्षु॒मन्तः॑ । याभिः॑ । मदे॑म ॥ रे॒वान् । इत् । रे॒वतः॑ । स्तो॒ता । स्यात् । त्वाव॑त॒ इति॒ त्व - व॒तः॒ । म॒घोनः॑ ॥ प्रेति॑ । इत् । उ॒ । ह॒रि॒व॒ इति॑ हरि - वः॒ । श्रु॒तस्य॑ ॥  \newline


\textbf{Krama Paata} \newline

यासि॑ दा॒श्वाꣳस᳚म् । दा॒श्वाꣳस॒मच्छ॑ । अच्छा॑नि॒युद्भिः॑ । नि॒युद्भि॑र् वायो । नि॒युद्भि॒रिति॑ नि॒युत् - भिः॒ । वा॒य॒ वि॒ष्टये᳚ । इ॒ष्टये॑ दुरो॒णे । दु॒रो॒ण इति॑ दुः - ओ॒ने ॥ नि नः॑ । नो॒ र॒यिम् । र॒यिꣳ सु॒भोज॑सम् । सु॒भोज॑सं ॅयुव । सु॒भोज॑स॒मिति॑ सु - भोज॑सम् । यु॒वे॒ह । इ॒ह नि । नि वी॒रव॑त् । वी॒रव॒द् गव्य᳚म् । वी॒रव॒दिति॑ वी॒र - व॒त्॒ । गव्य॒मश्वि॑यम् । अश्वि॑यम् च । च॒ राधः॑ । राध॒ इति॒ राधः॑ ॥ रे॒वती᳚र् नः । नः॒ स॒ध॒मादः॑ । स॒ध॒माद॒ इन्द्रे᳚ । स॒ध॒माद॒ इति॑ सध - मादः॑ । इन्द्रे॑ सन्तु । स॒न्तु॒ तु॒विवा॑जाः । तु॒विवा॑जा॒ इति॑ तु॒वि - वा॒जाः॒ । क्षु॒मन्तो॒ याभिः॑ । याभि॒र् मदे॑म । मदे॒मेति॒ मदे॑म ॥ रे॒वाꣳ इत् । इद् रे॒वतः॑ । रे॒वतः॑ स्तो॒ता । स्तो॒ता स्यात् । स्यात् त्वाव॑तः । त्वाव॑तो म॒घोनः॑ । त्वाव॑त॒ इति॒ त्व - व॒तः॒ । म॒घोन॒ इति॑ म॒घोनः॑ ॥ प्रेत् । इदु॑ । उ॒ ह॒रि॒वः॒ । ह॒रि॒वः॒ श्रु॒तस्य॑ । ह॒रि॒व॒ इति॑ हरि - वः॒ । श्रु॒तस्येति॑ श्रु॒तस्य॑ । \newline

\textbf{Jatai Paata} \newline

1. यासि॑ दा॒श्वाꣳस॑म् दा॒श्वाꣳसं॒ ॅयासि॒ यासि॑ दा॒श्वाꣳस᳚म् । \newline
2. दा॒श्वाꣳस॒ मच्छाच्छ॑ दा॒श्वाꣳस॑म् दा॒श्वाꣳस॒ मच्छ॑ । \newline
3. अच्छा॑ नि॒युद्भि॑र् नि॒युद्भि॒ रच्छाच्छा॑ नि॒युद्भिः॑ । \newline
4. नि॒युद्भि॑र् वायो वायो नि॒युद्भि॑र् नि॒युद्भि॑र् वायो । \newline
5. नि॒युद्भि॒रिति॑ नि॒युत् - भिः॒ । \newline
6. वा॒यवि॒ष्टय॑ इ॒ष्टये॑ वायो वा॒यवि॒ष्टये᳚ । \newline
7. इ॒ष्टये॑ दुरो॒णे दु॑रो॒ण इ॒ष्टय॑ इ॒ष्टये॑ दुरो॒णे । \newline
8. दु॒रो॒ण इति॑ दुः - ओ॒ने । \newline
9. नि नो॑ नो॒ नि नि नः॑ । \newline
10. नो॒ र॒यिꣳ र॒यिम् नो॑ नो र॒यिम् । \newline
11. र॒यिꣳ सु॒भोज॑सꣳ सु॒भोज॑सꣳ र॒यिꣳ र॒यिꣳ सु॒भोज॑सम् । \newline
12. सु॒भोज॑सं ॅयुव युव सु॒भोज॑सꣳ सु॒भोज॑सं ॅयुव । \newline
13. सु॒भोज॑स॒मिति॑ सु - भोज॑सम् । \newline
14. यु॒वे॒ हे ह यु॑व युवे॒ ह । \newline
15. इ॒ह नि नीहे ह नि । \newline
16. नि वी॒रव॑द् वी॒रव॒न् नि नि वी॒रव॑त् । \newline
17. वी॒रव॒द् गव्य॒म् गव्यं॑ ॅवी॒रव॑द् वी॒रव॒द् गव्य᳚म् । \newline
18. वी॒रव॒दिति॑ वी॒र - व॒त् । \newline
19. गव्य॒ मश्वि॑य॒ मश्वि॑य॒म् गव्य॒म् गव्य॒ मश्वि॑यम् । \newline
20. अश्वि॑यम् च॒ चाश्वि॑य॒ मश्वि॑यम् च । \newline
21. च॒ राधो॒ राध॑श्च च॒ राधः॑ । \newline
22. राध॒ इति॒ राधः॑ । \newline
23. रे॒वती᳚र् नो नो रे॒वती॑ रे॒वती᳚र् नः । \newline
24. नः॒ स॒ध॒मादः॑ सध॒मादो॑ नो नः सध॒मादः॑ । \newline
25. स॒ध॒माद॒ इन्द्र॒ इन्द्रे॑ सध॒मादः॑ सध॒माद॒ इन्द्रे᳚ । \newline
26. स॒ध॒माद॒ इति॑ सध - मादः॑ । \newline
27. इन्द्रे॑ सन्तु स॒न्त्विन्द्र॒ इन्द्रे॑ सन्तु । \newline
28. स॒न्तु॒ तु॒विवा॑जा स्तु॒विवा॑जाः सन्तु सन्तु तु॒विवा॑जाः । \newline
29. तु॒विवा॑जा॒ इति॑ तु॒वि - वा॒जाः॒ । \newline
30. क्षु॒मन्तो॒ याभि॒र् याभिः॑ क्षु॒मन्तः॑ क्षु॒मन्तो॒ याभिः॑ । \newline
31. याभि॒र् मदे॑म॒ मदे॑म॒ याभि॒र् याभि॒र् मदे॑म । \newline
32. मदे॒मेति॒ मदे॑म । \newline
33. रे॒वाꣳ इदिद् रे॒वान् रे॒वाꣳ इत् । \newline
34. इद् रे॒वतो॑ रे॒वत॒ इदिद् रे॒वतः॑ । \newline
35. रे॒वतः॑ स्तो॒ता स्तो॒ता रे॒वतो॑ रे॒वतः॑ स्तो॒ता । \newline
36. स्तो॒ता स्याथ् स्याथ् स्तो॒ता स्तो॒ता स्यात् । \newline
37. स्यात् त्वाव॑त॒ स्त्वाव॑तः॒ स्याथ् स्यात् त्वाव॑तः । \newline
38. त्वाव॑तो म॒घोनो॑ म॒घोन॒ स्त्वाव॑त॒ स्त्वाव॑तो म॒घोनः॑ । \newline
39. त्वाव॑त॒ इति॒ त्व - व॒तः॒ । \newline
40. म॒घोन॒ इति॑ म॒घोनः॑ । \newline
41. प्रे दित् प्र प्रे त् । \newline
42. इदु॑ वु॒ विदिदु॑ । \newline
43. उ॒ ह॒रि॒वो॒ ह॒रि॒व॒ उ॒ वु॒ ह॒रि॒वः॒ । \newline
44. ह॒रि॒वः॒ श्रु॒तस्य॑ श्रु॒तस्य॑ हरिवो हरिवः श्रु॒तस्य॑ । \newline
45. ह॒रि॒व॒ इति॑ हरि - वः॒ । \newline
46. श्रु॒तस्येति॑ श्रु॒तस्य॑ । \newline

\textbf{Ghana Paata } \newline

1. यासि॑ दा॒श्वाꣳस॑म् दा॒श्वाꣳसं॒ ॅयासि॒ यासि॑ दा॒श्वाꣳस॒ मच्छाच्छ॑ दा॒श्वाꣳसं॒ ॅयासि॒ यासि॑ दा॒श्वाꣳस॒ मच्छ॑ । \newline
2. दा॒श्वाꣳस॒ मच्छाच्छ॑ दा॒श्वाꣳस॑म् दा॒श्वाꣳस॒ मच्छा॑ नि॒युद्भि॑र् नि॒युद्भि॒ रच्छ॑ दा॒श्वाꣳस॑म् दा॒श्वाꣳस॒ मच्छा॑ नि॒युद्भिः॑ । \newline
3. अच्छा॑ नि॒युद्भि॑र् नि॒युद्भि॒ रच्छाच्छा॑ नि॒युद्भि॑र् वायो वायो नि॒युद्भि॒ रच्छाच्छा॑ नि॒युद्भि॑र् वायो । \newline
4. नि॒युद्भि॑र् वायो वायो नि॒युद्भि॑र् नि॒युद्भि॑र् वा॒यवि॒ष्टय॑ इ॒ष्टये॑ वायो नि॒युद्भि॑र् नि॒युद्भि॑र् वा॒यवि॒ष्टये᳚ । \newline
5. नि॒युद्भि॒रिति॑ नि॒युत् - भिः॒ । \newline
6. वा॒यवि॒ष्टय॑ इ॒ष्टये॑ वायो वा॒यवि॒ष्टये॑ दुरो॒णे दु॑रो॒ण इ॒ष्टये॑ वायो वा॒यवि॒ष्टये॑ दुरो॒णे । \newline
7. इ॒ष्टये॑ दुरो॒णे दु॑रो॒ण इ॒ष्टय॑ इ॒ष्टये॑ दुरो॒णे । \newline
8. दु॒रो॒ण इति॑ दुः - ओ॒ने । \newline
9. नि नो॑ नो॒ नि नि नो॑ र॒यिꣳ र॒यिम् नो॒ नि नि नो॑ र॒यिम् । \newline
10. नो॒ र॒यिꣳ र॒यिम् नो॑ नो र॒यिꣳ सु॒भोज॑सꣳ सु॒भोज॑सꣳ र॒यिम् नो॑ नो र॒यिꣳ सु॒भोज॑सम् । \newline
11. र॒यिꣳ सु॒भोज॑सꣳ सु॒भोज॑सꣳ र॒यिꣳ र॒यिꣳ सु॒भोज॑सं ॅयुव युव सु॒भोज॑सꣳ र॒यिꣳ र॒यिꣳ सु॒भोज॑सं ॅयुव । \newline
12. सु॒भोज॑सं ॅयुव युव सु॒भोज॑सꣳ सु॒भोज॑सं ॅयुवे॒ हे ह यु॑व सु॒भोज॑सꣳ सु॒भोज॑सं ॅयुवे॒ ह । \newline
13. सु॒भोज॑स॒मिति॑ सु - भोज॑सम् । \newline
14. यु॒वे॒ हे ह यु॑व युवे॒ ह नि नीह यु॑व युवे॒ ह नि । \newline
15. इ॒ह नि नीहे ह नि वी॒रव॑द् वी॒रव॒न् नीहे ह नि वी॒रव॑त् । \newline
16. नि वी॒रव॑द् वी॒रव॒न् नि नि वी॒रव॒द् गव्य॒म् गव्यं॑ ॅवी॒रव॒न् नि नि वी॒रव॒द् गव्य᳚म् । \newline
17. वी॒रव॒द् गव्य॒म् गव्यं॑ ॅवी॒रव॑द् वी॒रव॒द् गव्य॒ मश्वि॑य॒ मश्वि॑य॒म् गव्यं॑ ॅवी॒रव॑द् वी॒रव॒द् गव्य॒ मश्वि॑यम् । \newline
18. वी॒रव॒दिति॑ वी॒र - व॒त् । \newline
19. गव्य॒ मश्वि॑य॒ मश्वि॑य॒म् गव्य॒म् गव्य॒ मश्वि॑यम् च॒ चाश्वि॑य॒म् गव्य॒म् गव्य॒ मश्वि॑यम् च । \newline
20. अश्वि॑यम् च॒ चाश्वि॑य॒ मश्वि॑यम् च॒ राधो॒ राध॒श्चाश्वि॑य॒ मश्वि॑यम् च॒ राधः॑ । \newline
21. च॒ राधो॒ राध॑श्च च॒ राधः॑ । \newline
22. राध॒ इति॒ राधः॑ । \newline
23. रे॒वती᳚र् नो नो रे॒वती॑ रे॒वती᳚र् नः सध॒मादः॑ सध॒मादो॑ नो रे॒वती॑ रे॒वती᳚र् नः सध॒मादः॑ । \newline
24. नः॒ स॒ध॒मादः॑ सध॒मादो॑ नो नः सध॒माद॒ इन्द्र॒ इन्द्रे॑ सध॒मादो॑ नो नः सध॒माद॒ इन्द्रे᳚ । \newline
25. स॒ध॒माद॒ इन्द्र॒ इन्द्रे॑ सध॒मादः॑ सध॒माद॒ इन्द्रे॑ सन्तु स॒न्त्विन्द्रे॑ सध॒मादः॑ सध॒माद॒ इन्द्रे॑ सन्तु । \newline
26. स॒ध॒माद॒ इति॑ सध - मादः॑ । \newline
27. इन्द्रे॑ सन्तु स॒न्त्विन्द्र॒ इन्द्रे॑ सन्तु तु॒विवा॑जा स्तु॒विवा॑जाः स॒न्त्विन्द्र॒ इन्द्रे॑ सन्तु तु॒विवा॑जाः । \newline
28. स॒न्तु॒ तु॒विवा॑जा स्तु॒विवा॑जाः सन्तु सन्तु तु॒विवा॑जाः । \newline
29. तु॒विवा॑जा॒ इति॑ तु॒वि - वा॒जाः॒ । \newline
30. क्षु॒मन्तो॒ याभि॒र् याभिः॑ क्षु॒मन्तः॑ क्षु॒मन्तो॒ याभि॒र् मदे॑म॒ मदे॑म॒ याभिः॑ क्षु॒मन्तः॑ क्षु॒मन्तो॒ याभि॒र् मदे॑म । \newline
31. याभि॒र् मदे॑म॒ मदे॑म॒ याभि॒र् याभि॒र् मदे॑म । \newline
32. मदे॒मेति॒ मदे॑म । \newline
33. रे॒वाꣳ इदिद् रे॒वान् रे॒वाꣳ इद् रे॒वतो॑ रे॒वत॒ इद् रे॒वान् रे॒वाꣳ इद् रे॒वतः॑ । \newline
34. इद् रे॒वतो॑ रे॒वत॒ इदिद् रे॒वतः॑ स्तो॒ता स्तो॒ता रे॒वत॒ इदिद् रे॒वतः॑ स्तो॒ता । \newline
35. रे॒वतः॑ स्तो॒ता स्तो॒ता रे॒वतो॑ रे॒वतः॑ स्तो॒ता स्याथ् स्याथ् स्तो॒ता रे॒वतो॑ रे॒वतः॑ स्तो॒ता स्यात् । \newline
36. स्तो॒ता स्याथ् स्याथ् स्तो॒ता स्तो॒ता स्यात् त्वाव॑त॒ स्त्वाव॑तः॒ स्याथ् स्तो॒ता स्तो॒ता स्यात् त्वाव॑तः । \newline
37. स्यात् त्वाव॑त॒ स्त्वाव॑तः॒ स्याथ् स्यात् त्वाव॑तो म॒घोनो॑ म॒घोन॒ स्त्वाव॑तः॒ स्याथ् स्यात् त्वाव॑तो म॒घोनः॑ । \newline
38. त्वाव॑तो म॒घोनो॑ म॒घोन॒ स्त्वाव॑त॒ स्त्वाव॑तो म॒घोनः॑ । \newline
39. त्वाव॑त॒ इति॒ त्व - व॒तः॒ । \newline
40. म॒घोन॒ इति॑ म॒घोनः॑ । \newline
41. प्रे दित् प्र प्रे दु॑ वु॒ वित् प्र प्रे दु॑ । \newline
42. इदु॑ वु॒ विदिदु॑ हरिवो हरिव उ॒ विदिदु॑ हरिवः । \newline
43. उ॒ ह॒रि॒वो॒ ह॒रि॒व॒ उ॒ वु॒ ह॒रि॒वः॒ श्रु॒तस्य॑ श्रु॒तस्य॑ हरिव उ वु हरिवः श्रु॒तस्य॑ । \newline
44. ह॒रि॒वः॒ श्रु॒तस्य॑ श्रु॒तस्य॑ हरिवो हरिवः श्रु॒तस्य॑ । \newline
45. ह॒रि॒व॒ इति॑ हरि - वः॒ । \newline
46. श्रु॒तस्येति॑ श्रु॒तस्य॑ । \newline
\pagebreak


\end{document}