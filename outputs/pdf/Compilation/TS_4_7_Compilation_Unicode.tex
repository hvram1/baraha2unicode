\documentclass[17pt]{extarticle}
\usepackage{babel}
\usepackage{fontspec}
\usepackage{polyglossia}
\usepackage{extsizes}

\usepackage{color}   %May be necessary if you want to color links
\usepackage{hyperref}
\hypersetup{
    colorlinks=true, %set true if you want colored links
    linktoc=all,     %set to all if you want both sections and subsections linked
    linkcolor=black,  %choose some color if you want links to stand out
}

\setmainlanguage{sanskrit}
\setotherlanguages{english} %% or other languages
\setlength{\parindent}{0pt}
\pagestyle{myheadings}
\newfontfamily\devanagarifont[Script=Devanagari]{AdishilaVedic}
\renewcommand{\theHsection}{\thepart.section.\thesection}

\newcommand{\VAR}[1]{}
\newcommand{\BLOCK}[1]{}




\begin{document}
\begin{titlepage}
    \begin{center}
 
\begin{sanskrit}
    { \Large
    कृष्ण यजुर्वेदीय तैत्तिरीय संहिता,पद,जटा,घन पाठः 
    }
    \\
    \vspace{2.5cm}
    \mbox{ \Large
    4.7     चतुर्थकाण्डे सप्तमः प्रश्नः - वसोर्धारादिशिष्ट संस्काराभिधानं   }
\end{sanskrit}
\end{center}

\end{titlepage}
\tableofcontents
\phantomsection
\pagebreak

\markright{ TS 4.7.1.1  \hfill https://www.vedavms.in \hfill}

\section{ TS 4.7.1.1 }

\textbf{TS 4.7.1.1 } \newline
\textbf{Samhita Paata} \newline

अग्ना॑विष्णू स॒जोष॑से॒मा व॑र्द्धन्तु वां॒ गिरः॑ ।ध्यु॒नैंर्वाजे॑भि॒रा ग॑तं ॥ वाज॑श्च मे प्रस॒वश्च॑ मे॒ प्रय॑तिश्च मे॒ प्रसि॑तिश्च मे धी॒तिश्च॑ मे॒ क्रतु॑श्च मे॒ स्वर॑श्च मे॒ श्लोक॑श्च मे श्रा॒वश्च॑ मे॒ श्रुति॑श्च मे॒ ज्योति॑श्च मे॒ सुव॑श्च मे प्रा॒णश्च॑ मे ऽपा॒न - [  ] \newline

\textbf{Pada Paata} \newline

अग्ना॑विष्णू॒ इत्यग्ना᳚ - वि॒ष्णू॒ । स॒जोष॒सेति॑ स - जोष॑सा । इ॒माः । व॒द्‌र्ध॒न्तु॒ । वा॒म् । गिरः॑ ॥ ध्यु॒म्नैः । वाजे॑भिः । एति॑ । ग॒त॒म् ॥ वाजः॑ । च॒ । मे॒ । प्र॒स॒व इति॑ प्र-स॒वः । च॒ । मे॒ । प्रय॑ति॒रिति॒ प्र-य॒तिः॒ । च॒ । मे॒ । प्रसि॑ति॒रिति॒ प्र-सि॒तिः॒ । च॒ । मे॒ । धी॒तिः । च॒ । मे॒ । क्रतुः॑ । च॒ । मे॒ । स्वरः॑ । च॒ । मे॒ । श्लोकः॑ । च॒ । मे॒ । श्रा॒वः । च॒ । मे॒ । श्रुतिः॑ । च॒ । मे॒ । ज्योतिः॑ । च॒ । मे॒ । सुवः॑ । च॒ । मे॒ । प्रा॒ण इति॑ प्र - अ॒नः । च॒ । मे॒ । अ॒पा॒न इत्य॑प - अ॒नः ।  \newline


\textbf{Krama Paata} \newline

अग्ना॑विष्णू स॒जोष॑सा । अग्ना॑विष्णू॒॒ इत्यग्ना᳚ - वि॒ष्णू॒ । स॒जोष॑से॒माः । स॒॒जोष॒॒सेति॑ स - जोष॑सा । इ॒॒मा व॑र्द्धन्तु । व॒र्द्ध॒न्तु॒ वा॒म् । वा॒ङ्गिरः॑ । गिर॒ इति॒ गिरः॑ ॥ द्यु॒म्नैर् वाजे॑भिः । वाजे॑भि॒रा । आ ग॑तम् । ग॒त॒मिति॑ गतम् ॥ वाज॑श्च । च॒ मे॒ । मे॒ प्र॒स॒वः । प्र॒स॒वश्च॑ । प्र॒स॒व इति॑ प्र - स॒वः । च॒ मे॒ । मे॒ प्रय॑तिः । प्रय॑तिश्च । प्रय॑ति॒रिति॒ प्र - य॒तिः॒ । च॒ मे॒ । मे॒ प्रसि॑तिः । प्रसि॑तिश्च । प्रसि॑ति॒रिति॒ प्र - सि॒तिः॒ । च॒ मे॒ । मे॒ धी॒तिः । धी॒तिश्च॑ । च॒ मे॒ । मे॒ क्रतुः॑ । क्रतु॑श्च । च॒ मे॒ । मे॒ स्वरः॑ । स्वर॑श्च । 
च॒ मे॒ । मे॒ श्लोकः॑ । श्लोक॑श्च । च॒ मे॒ । मे॒ श्रा॒वः । श्रा॒वश्च॑ । च॒ मे॒ । मे॒ श्रुतिः॑ । श्रुति॑श्च । च॒ मे॒ । मे॒ ज्योतिः॑ । ज्योति॑श्च । च॒ मे॒ । मे॒ सुवः॑ । सुव॑श्च । च॒ मे॒ । मे॒ प्रा॒णः । प्रा॒णश्च॑ । प्रा॒ण इति॑ प्र - अ॒नः । च॒ मे॒ । मे॒ऽपा॒नः । अ॒पा॒नश्च॑ । 
अ॒पा॒न इत्य॑प - अ॒नः \newline

\textbf{Jatai Paata} \newline

1. अग्ना॑विष्णू स॒जोष॑सा स॒जोष॒सा ऽग्ना॑विष्णू॒ अग्ना॑विष्णू स॒जोष॑सा । \newline
2. अग्ना॑विष्णू॒ इत्यग्ना᳚ - वि॒ष्णू॒ । \newline
3. स॒जोष॑से॒मा इ॒माः स॒जोष॑सा स॒जोष॑से॒माः । \newline
4. स॒जोष॒सेति॑ स - जोष॑सा । \newline
5. इ॒मा व॑र्द्धन्तु वर्द्ध न्त्वि॒मा इ॒मा व॑र्द्धन्तु । \newline
6. व॒र्द्ध॒न्तु॒ वां॒ ॅवां॒ ॅव॒र्द्ध॒न्तु॒ व॒र्द्ध॒न्तु॒ वा॒म् । \newline
7. वा॒म् गिरो॒ गिरो॑ वां ॅवा॒म् गिरः॑ । \newline
8. गिर॒ इति॒ गिरः॑ । \newline
9. द्यु॒म्नैर् वाजे॑भि॒र् वाजे॑भिर् द्यु॒म्नैर् द्यु॒म्नैर् वाजे॑भिः । \newline
10. वाजे॑भि॒रा वाजे॑भि॒र् वाजे॑भि॒रा । \newline
11. आ ग॑तम् गत॒ मा ग॑तम् । \newline
12. ग॒त॒मिति॑ गतम् । \newline
13. वाज॑श्च च॒ वाजो॒ वाज॑श्च । \newline
14. च॒ मे॒ मे॒ च॒ च॒ मे॒ । \newline
15. मे॒ प्र॒स॒वः प्र॑स॒वो मे॑ मे प्रस॒वः । \newline
16. प्र॒स॒वश्च॑ च प्रस॒वः प्र॑स॒वश्च॑ । \newline
17. प्र॒स॒व इति॑ प्र - स॒वः । \newline
18. च॒ मे॒ मे॒ च॒ च॒ मे॒ । \newline
19. मे॒ प्रय॑तिः॒ प्रय॑तिर् मे मे॒ प्रय॑तिः । \newline
20. प्रय॑तिश्च च॒ प्रय॑तिः॒ प्रय॑तिश्च । \newline
21. प्रय॑ति॒रिति॒ प्र - य॒तिः॒ । \newline
22. च॒ मे॒ मे॒ च॒ च॒ मे॒ । \newline
23. मे॒ प्रसि॑तिः॒ प्रसि॑तिर् मे मे॒ प्रसि॑तिः । \newline
24. प्रसि॑तिश्च च॒ प्रसि॑तिः॒ प्रसि॑तिश्च । \newline
25. प्रसि॑ति॒रिति॒ प्र - सि॒तिः॒ । \newline
26. च॒ मे॒ मे॒ च॒ च॒ मे॒ । \newline
27. मे॒ धी॒तिर् धी॒तिर् मे॑ मे धी॒तिः । \newline
28. धी॒तिश्च॑ च धी॒तिर् धी॒तिश्च॑ । \newline
29. च॒ मे॒ मे॒ च॒ च॒ मे॒ । \newline
30. मे॒ क्रतुः॒ क्रतु॑र् मे मे॒ क्रतुः॑ । \newline
31. क्रतु॑श्च च॒ क्रतुः॒ क्रतु॑श्च । \newline
32. च॒ मे॒ मे॒ च॒ च॒ मे॒ । \newline
33. मे॒ स्वरः॒ स्वरो॑ मे मे॒ स्वरः॑ । \newline
34. स्वर॑श्च च॒ स्वरः॒ स्वर॑श्च । \newline
35. च॒ मे॒ मे॒ च॒ च॒ मे॒ । \newline
36. मे॒ श्लोकः॒ श्लोको॑ मे मे॒ श्लोकः॑ । \newline
37. श्लोक॑श्च च॒ श्लोकः॒ श्लोक॑श्च । \newline
38. च॒ मे॒ मे॒ च॒ च॒ मे॒ । \newline
39. मे॒ श्रा॒वः श्रा॒वो मे॑ मे श्रा॒वः । \newline
40. श्रा॒वश्च॑ च श्रा॒वः श्रा॒वश्च॑ । \newline
41. च॒ मे॒ मे॒ च॒ च॒ मे॒ । \newline
42. मे॒ श्रुतिः॒ श्रुति॑र् मे मे॒ श्रुतिः॑ । \newline
43. श्रुति॑श्च च॒ श्रुतिः॒ श्रुति॑श्च । \newline
44. च॒ मे॒ मे॒ च॒ च॒ मे॒ । \newline
45. मे॒ ज्योति॒र् ज्योति॑र् मे मे॒ ज्योतिः॑ । \newline
46. ज्योति॑श्च च॒ ज्योति॒र् ज्योति॑श्च । \newline
47. च॒ मे॒ मे॒ च॒ च॒ मे॒ । \newline
48. मे॒ सुवः॒ सुव॑र् मे मे॒ सुवः॑ । \newline
49. सुव॑श्च च॒ सुवः॒ सुव॑श्च । \newline
50. च॒ मे॒ मे॒ च॒ च॒ मे॒ । \newline
51. मे॒ प्रा॒णः प्रा॒णो मे॑ मे प्रा॒णः । \newline
52. प्रा॒णश्च॑ च प्रा॒णः प्रा॒णश्च॑ । \newline
53. प्रा॒ण इति॑ प्र - अ॒नः । \newline
54. च॒ मे॒ मे॒ च॒ च॒ मे॒ । \newline
55. मे॒ ऽपा॒नो॑ ऽपा॒नो मे॑ मे ऽपा॒नः । \newline
56. अ॒पा॒नश्च॑ चापा॒नो॑ ऽपा॒नश्च॑ । \newline
57. अ॒पा॒न इत्य॑प - अ॒नः । \newline

\textbf{Ghana Paata } \newline

1. अग्ना॑विष्णू स॒जोष॑सा स॒जोष॒सा ऽग्ना॑विष्णू॒ अग्ना॑विष्णू स॒जोष॑से॒मा इ॒माः स॒जोष॒सा ऽग्ना॑विष्णू॒ अग्ना॑विष्णू स॒जोष॑से॒माः । \newline
2. अग्ना॑विष्णू॒ इत्यग्ना᳚ - वि॒ष्णू॒ । \newline
3. स॒जोष॑से॒मा इ॒माः स॒जोष॑सा स॒जोष॑से॒मा व॑र्द्धन्तु वर्द्धन्त्वि॒माः स॒जोष॑सा स॒जोष॑से॒मा व॑र्द्धन्तु । \newline
4. स॒जोष॒सेति॑ स - जोष॑सा । \newline
5. इ॒मा व॑र्द्धन्तु वर्द्धन्त्वि॒मा इ॒मा व॑र्द्धन्तु वां ॅवां ॅवर्द्धन्त्वि॒मा इ॒मा व॑र्द्धन्तु वाम् । \newline
6. व॒र्द्ध॒न्तु॒ वां॒ ॅवां॒ ॅव॒र्द्ध॒न्तु॒ व॒र्द्ध॒न्तु॒ वा॒म् गिरो॒ गिरो॑ वां ॅवर्द्धन्तु वर्द्धन्तु वा॒म् गिरः॑ । \newline
7. वा॒म् गिरो॒ गिरो॑ वां ॅवा॒म् गिरः॑ । \newline
8. गिर॒ इति॒ गिरः॑ । \newline
9. द्यु॒म्नैर् वाजे॑भि॒र् वाजे॑भिर् द्यु॒म्नैर् द्यु॒म्नैर् वाजे॑भि॒रा वाजे॑भिर् द्यु॒म्नैर् द्यु॒म्नैर् वाजे॑भि॒रा । \newline
10. वाजे॑भि॒रा वाजे॑भि॒र् वाजे॑भि॒रा ग॑तम् गत॒मा वाजे॑भि॒र् वाजे॑भि॒रा ग॑तम् । \newline
11. आ ग॑तम् गत॒मा ग॑तम् । \newline
12. ग॒त॒मिति॑ गतम् । \newline
13. वाज॑श्च च॒ वाजो॒ वाज॑श्च मे मे च॒ वाजो॒ वाज॑श्च मे । \newline
14. च॒ मे॒ मे॒ च॒ च॒ मे॒ प्र॒स॒वः प्र॑स॒वो मे॑ च च मे प्रस॒वः । \newline
15. मे॒ प्र॒स॒वः प्र॑स॒वो मे॑ मे प्रस॒वश्च॑ च प्रस॒वो मे॑ मे प्रस॒वश्च॑ । \newline
16. प्र॒स॒वश्च॑ च प्रस॒वः प्र॑स॒वश्च॑ मे मे च प्रस॒वः प्र॑स॒वश्च॑ मे । \newline
17. प्र॒स॒व इति॑ प्र - स॒वः । \newline
18. च॒ मे॒ मे॒ च॒ च॒ मे॒ प्रय॑तिः॒ प्रय॑तिर् मे च च मे॒ प्रय॑तिः । \newline
19. मे॒ प्रय॑तिः॒ प्रय॑तिर् मे मे॒ प्रय॑तिश्च च॒ प्रय॑तिर् मे मे॒ प्रय॑तिश्च । \newline
20. प्रय॑तिश्च च॒ प्रय॑तिः॒ प्रय॑तिश्च मे मे च॒ प्रय॑तिः॒ प्रय॑तिश्च मे । \newline
21. प्रय॑ति॒रिति॒ प्र - य॒तिः॒ । \newline
22. च॒ मे॒ मे॒ च॒ च॒ मे॒ प्रसि॑तिः॒ प्रसि॑तिर् मे च च मे॒ प्रसि॑तिः । \newline
23. मे॒ प्रसि॑तिः॒ प्रसि॑तिर् मे मे॒ प्रसि॑तिश्च च॒ प्रसि॑तिर् मे मे॒ प्रसि॑तिश्च । \newline
24. प्रसि॑तिश्च च॒ प्रसि॑तिः॒ प्रसि॑तिश्च मे मे च॒ प्रसि॑तिः॒ प्रसि॑तिश्च मे । \newline
25. प्रसि॑ति॒रिति॒ प्र - सि॒तिः॒ । \newline
26. च॒ मे॒ मे॒ च॒ च॒ मे॒ धी॒तिर् धी॒तिर् मे॑ च च मे धी॒तिः । \newline
27. मे॒ धी॒तिर् धी॒तिर् मे॑ मे धी॒तिश्च॑ च धी॒तिर् मे॑ मे धी॒तिश्च॑ । \newline
28. धी॒तिश्च॑ च धी॒तिर् धी॒तिश्च॑ मे मे च धी॒तिर् धी॒तिश्च॑ मे । \newline
29. च॒ मे॒ मे॒ च॒ च॒ मे॒ क्रतुः॒ क्रतु॑र् मे च च मे॒ क्रतुः॑ । \newline
30. मे॒ क्रतुः॒ क्रतु॑र् मे मे॒ क्रतु॑श्च च॒ क्रतु॑र् मे मे॒ क्रतु॑श्च । \newline
31. क्रतु॑श्च च॒ क्रतुः॒ क्रतु॑श्च मे मे च॒ क्रतुः॒ क्रतु॑श्च मे । \newline
32. च॒ मे॒ मे॒ च॒ च॒ मे॒ स्वरः॒ स्वरो॑ मे च च मे॒ स्वरः॑ । \newline
33. मे॒ स्वरः॒ स्वरो॑ मे मे॒ स्वर॑श्च च॒ स्वरो॑ मे मे॒ स्वर॑श्च । \newline
34. स्वर॑श्च च॒ स्वरः॒ स्वर॑श्च मे मे च॒ स्वरः॒ स्वर॑श्च मे । \newline
35. च॒ मे॒ मे॒ च॒ च॒ मे॒ श्लोकः॒ श्लोको॑ मे च च मे॒ श्लोकः॑ । \newline
36. मे॒ श्लोकः॒ श्लोको॑ मे मे॒ श्लोक॑श्च च॒ श्लोको॑ मे मे॒ श्लोक॑श्च । \newline
37. श्लोक॑श्च च॒ श्लोकः॒ श्लोक॑श्च मे मे च॒ श्लोकः॒ श्लोक॑श्च मे । \newline
38. च॒ मे॒ मे॒ च॒ च॒ मे॒ श्रा॒वः श्रा॒वो मे॑ च च मे श्रा॒वः । \newline
39. मे॒ श्रा॒वः श्रा॒वो मे॑ मे श्रा॒वश्च॑ च श्रा॒वो मे॑ मे श्रा॒वश्च॑ । \newline
40. श्रा॒वश्च॑ च श्रा॒वः श्रा॒वश्च॑ मे मे च श्रा॒वः श्रा॒वश्च॑ मे । \newline
41. च॒ मे॒ मे॒ च॒ च॒ मे॒ श्रुतिः॒ श्रुति॑र् मे च च मे॒ श्रुतिः॑ । \newline
42. मे॒ श्रुतिः॒ श्रुति॑र् मे मे॒ श्रुति॑श्च च॒ श्रुति॑र् मे मे॒ श्रुति॑श्च । \newline
43. श्रुति॑श्च च॒ श्रुतिः॒ श्रुति॑श्च मे मे च॒ श्रुतिः॒ श्रुति॑श्च मे । \newline
44. च॒ मे॒ मे॒ च॒ च॒ मे॒ ज्योति॒र् ज्योति॑र् मे च च मे॒ ज्योतिः॑ । \newline
45. मे॒ ज्योति॒र् ज्योति॑र् मे मे॒ ज्योति॑श्च च॒ ज्योति॑र् मे मे॒ ज्योति॑श्च । \newline
46. ज्योति॑श्च च॒ ज्योति॒र् ज्योति॑श्च मे मे च॒ ज्योति॒र् ज्योति॑श्च मे । \newline
47. च॒ मे॒ मे॒ च॒ च॒ मे॒ सुवः॒ सुव॑र् मे च च मे॒ सुवः॑ । \newline
48. मे॒ सुवः॒ सुव॑र् मे मे॒ सुव॑श्च च॒ सुव॑र् मे मे॒ सुव॑श्च । \newline
49. सुव॑श्च च॒ सुवः॒ सुव॑श्च मे मे च॒ सुवः॒ सुव॑श्च मे । \newline
50. च॒ मे॒ मे॒ च॒ च॒ मे॒ प्रा॒णः प्रा॒णो मे॑ च च मे प्रा॒णः । \newline
51. मे॒ प्रा॒णः प्रा॒णो मे॑ मे प्रा॒णश्च॑ च प्रा॒णो मे॑ मे प्रा॒णश्च॑ । \newline
52. प्रा॒णश्च॑ च प्रा॒णः प्रा॒णश्च॑ मे मे च प्रा॒णः प्रा॒णश्च॑ मे । \newline
53. प्रा॒ण इति॑ प्र - अ॒नः । \newline
54. च॒ मे॒ मे॒ च॒ च॒ मे॒ ऽपा॒नो॑ ऽपा॒नो मे॑ च च मे ऽपा॒नः । \newline
55. मे॒ ऽपा॒नो॑ ऽपा॒नो मे॑ मे ऽपा॒नश्च॑ चापा॒नो मे॑ मे ऽपा॒नश्च॑ । \newline
56. अ॒पा॒नश्च॑ चापा॒नो॑ ऽपा॒नश्च॑ मे मे चापा॒नो॑ ऽपा॒नश्च॑ मे । \newline
57. अ॒पा॒न इत्य॑प - अ॒नः । \newline
\pagebreak
\markright{ TS 4.7.1.2  \hfill https://www.vedavms.in \hfill}

\section{ TS 4.7.1.2 }

\textbf{TS 4.7.1.2 } \newline
\textbf{Samhita Paata} \newline

-श्च॑ मे व्या॒नश्च॒ मे ऽसु॑श्च मे चि॒त्तं च॑ म॒ आधी॑तं च मे॒ वाक्च॑ मे॒ मन॑श्च मे॒ चक्षु॑श्च मे॒ श्रोत्रं॑ च मे॒ दक्ष॑श्च मे॒ बलं॑ च म॒ ओज॑श्च मे॒ सह॑श्च म॒ आयु॑श्च मे ज॒रा च॑ म आ॒त्मा च॑ मे त॒नूश्च॑ मे॒ ( ) शर्म॑ च मे॒ वर्म॑ च॒ मे ऽङ्गा॑नि च मे॒ ऽस्थानि॑ च मे॒ परूꣳ॑षि च मे॒ शरी॑राणि च मे ॥ \newline

\textbf{Pada Paata} \newline

च॒ । मे॒ । व्या॒न इति॑ वि-अ॒नः । च॒ । मे॒ । असुः॑ । च॒ । मे॒ । चि॒त्तम् । च॒ । मे॒ । आधी॑त॒मित्या - धी॒त॒म् । च॒ । मे॒ । वाक् । च॒ । मे॒ । मनः॑ । च॒ । मे॒ । चक्षुः॑ । च॒ । मे॒ । श्रोत्र᳚म् । च॒ । मे॒ । दक्षः॑ । च॒ । मे॒ । बल᳚म् । च॒ । मे॒ । ओजः॑ । च॒ । मे॒ । सहः॑ । च॒ । मे॒ । आयुः॑ । च॒ । मे॒ । ज॒रा । च॒ । मे॒ । आ॒त्मा । च॒ । मे॒ । त॒नूः । च॒ । मे॒ ( ) । शर्म॑ । च॒ । मे॒ । वर्म॑ । च॒ । मे॒ । अङ्गा॑नि । च॒ । मे॒ । अ॒स्थानि॑ । च॒ । मे॒ । परूꣳ॑षि । च॒ । मे॒ । शरी॑राणि । च॒ । मे॒ ॥  \newline


\textbf{Krama Paata} \newline

च॒ मे॒ । म॒ व्या॒नः । व्या॒नश्च॑ । व्या॒न इति॑ वि - अ॒नः । च॒ मे॒ । मेऽसुः॑ । असु॑श्च । च॒ मे॒ । मे॒ चि॒त्तम् । चि॒त्तम् च॑ । च॒ मे॒ । म॒ आधी॑तम् । आधी॑तम् च । आधी॑त॒मित्या - धी॒त॒म् । च॒ मे॒ । मे॒ वाक् । वाक् च॑ । च॒ मे॒ । मे॒ मनः॑ । मन॑श्च । च॒ मे॒ । मे॒ चक्षुः॑ । चक्षु॑श्च । च॒ मे॒ । मे॒ श्रोत्र᳚म् । श्रोत्र॑म् च । च॒ मे॒ । मे॒ दक्षः॑ । दक्ष॑श्च । च॒ मे॒ । मे॒ बल᳚म् । बल॑म् च । च॒ मे॒ । म॒ ओजः॑ । ओज॑श्च । च॒ मे॒ । मे॒ सहः॑ । सह॑श्च । च॒ मे॒ । म॒ आयुः॑ । आयु॑श्च । च॒ मे॒ । मे॒ ज॒रा । ज॒रा च॑ । च॒ मे॒ । म॒ आ॒त्मा । आ॒त्मा च॑ । च॒ मे॒ । मे॒ त॒नूः । त॒नूश्च॑ । च॒ मे॒ ( ) । मे॒ शर्म॑ । शर्म॑ च । च॒ मे॒ । मे॒ वर्म॑ । वर्म॑ च । च॒ मे॒ । मेऽङ्गा॑नि । अङ्गा॑नि च । च॒ मे॒ । मे॒ऽस्थानि॑ । अ॒स्थानि॑ च । च॒ मे॒ । मे॒ परूꣳ॑षि । परूꣳ॑षि च । च॒ मे॒ । मे॒ शरी॑राणि । शरी॑राणि च । च॒ मे॒ । 
म॒ इति॑ मे । \newline

\textbf{Jatai Paata} \newline

1. च॒ मे॒ मे॒ च॒ च॒ मे॒ । \newline
2. मे॒ व्या॒नो व्या॒नो मे॑ मे व्या॒नः । \newline
3. व्या॒नश्च॑ च व्या॒नो व्या॒नश्च॑ । \newline
4. व्या॒न इति॑ वि - अ॒नः । \newline
5. च॒ मे॒ मे॒ च॒ च॒ मे॒ । \newline
6. मे ऽसु॒ रसु॑र् मे॒ मे ऽसुः॑ । \newline
7. असु॑श्च॒ चासु॒ रसु॑श्च । \newline
8. च॒ मे॒ मे॒ च॒ च॒ मे॒ । \newline
9. मे॒ चि॒त्तम् चि॒त्तम् मे॑ मे चि॒त्तम् । \newline
10. चि॒त्तम् च॑ च चि॒त्तम् चि॒त्तम् च॑ । \newline
11. च॒ मे॒ मे॒ च॒ च॒ मे॒ । \newline
12. म॒ आधी॑त॒ माधी॑तम् मे म॒ आधी॑तम् । \newline
13. आधी॑तम् च॒ चाधी॑त॒ माधी॑तम् च । \newline
14. आधी॑त॒मित्या - धी॒त॒म् । \newline
15. च॒ मे॒ मे॒ च॒ च॒ मे॒ । \newline
16. मे॒ वाग् वाङ् मे॑ मे॒ वाक् । \newline
17. वाक् च॑ च॒ वाग् वाक् च॑ । \newline
18. च॒ मे॒ मे॒ च॒ च॒ मे॒ । \newline
19. मे॒ मनो॒ मनो॑ मे मे॒ मनः॑ । \newline
20. मन॑श्च च॒ मनो॒ मन॑श्च । \newline
21. च॒ मे॒ मे॒ च॒ च॒ मे॒ । \newline
22. मे॒ चक्षु॒ श्चक्षु॑र् मे मे॒ चक्षुः॑ । \newline
23. चक्षु॑श्च च॒ चक्षु॒ श्चक्षु॑श्च । \newline
24. च॒ मे॒ मे॒ च॒ च॒ मे॒ । \newline
25. मे॒ श्रोत्र॒(ग्ग्॒) श्रोत्र॑म् मे मे॒ श्रोत्र᳚म् । \newline
26. श्रोत्र॑म् च च॒ श्रोत्र॒(ग्ग्॒) श्रोत्र॑म् च । \newline
27. च॒ मे॒ मे॒ च॒ च॒ मे॒ । \newline
28. मे॒ दक्षो॒ दक्षो॑ मे मे॒ दक्षः॑ । \newline
29. दक्ष॑श्च च॒ दक्षो॒ दक्ष॑श्च । \newline
30. च॒ मे॒ मे॒ च॒ च॒ मे॒ । \newline
31. मे॒ बल॒म् बल॑म् मे मे॒ बल᳚म् । \newline
32. बल॑म् च च॒ बल॒म् बल॑म् च । \newline
33. च॒ मे॒ मे॒ च॒ च॒ मे॒ । \newline
34. म॒ ओज॒ ओजो॑ मे म॒ ओजः॑ । \newline
35. ओज॑श्च॒ चौज॒ ओज॑श्च । \newline
36. च॒ मे॒ मे॒ च॒ च॒ मे॒ । \newline
37. मे॒ सहः॒ सहो॑ मे मे॒ सहः॑ । \newline
38. सह॑श्च च॒ सहः॒ सह॑श्च । \newline
39. च॒ मे॒ मे॒ च॒ च॒ मे॒ । \newline
40. म॒ आयु॒ रायु॑र् मे म॒ आयुः॑ । \newline
41. आयु॑श्च॒ चायु॒ रायु॑श्च । \newline
42. च॒ मे॒ मे॒ च॒ च॒ मे॒ । \newline
43. मे॒ ज॒रा ज॒रा मे॑ मे ज॒रा । \newline
44. ज॒रा च॑ च ज॒रा ज॒रा च॑ । \newline
45. च॒ मे॒ मे॒ च॒ च॒ मे॒ । \newline
46. म॒ आ॒त्मा ऽऽत्मा मे॑ म आ॒त्मा । \newline
47. आ॒त्मा च॑ चा॒त्मा ऽऽत्मा च॑ । \newline
48. च॒ मे॒ मे॒ च॒ च॒ मे॒ । \newline
49. मे॒ त॒नू स्त॒नूर् मे॑ मे त॒नूः । \newline
50. त॒नूश्च॑ च त॒नू स्त॒नूश्च॑ । \newline
51. च॒ मे॒ मे॒ च॒ च॒ मे॒ । \newline
52. मे॒ शर्म॒ शर्म॑ मे मे॒ शर्म॑ । \newline
53. शर्म॑ च च॒ शर्म॒ शर्म॑ च । \newline
54. च॒ मे॒ मे॒ च॒ च॒ मे॒ । \newline
55. मे॒ वर्म॒ वर्म॑ मे मे॒ वर्म॑ । \newline
56. वर्म॑ च च॒ वर्म॒ वर्म॑ च । \newline
57. च॒ मे॒ मे॒ च॒ च॒ मे॒ । \newline
58. मे ऽङ्गा॒ न्यङ्गा॑नि मे॒ मे ऽङ्गा॑नि । \newline
59. अङ्गा॑नि च॒ चाङ्गा॒ न्यङ्गा॑नि च । \newline
60. च॒ मे॒ मे॒ च॒ च॒ मे॒ । \newline
61. मे॒ ऽस्था न्य॒स्थानि॑ मे मे॒ ऽस्थानि॑ । \newline
62. अ॒स्थानि॑ च चा॒स्था न्य॒स्थानि॑ च । \newline
63. च॒ मे॒ मे॒ च॒ च॒ मे॒ । \newline
64. मे॒ परू(ग्म्॑)षि॒ परू(ग्म्॑)षि मे मे॒ परू(ग्म्॑)षि । \newline
65. परू(ग्म्॑)षि च च॒ परू(ग्म्॑)षि॒ परू(ग्म्॑)षि च । \newline
66. च॒ मे॒ मे॒ च॒ च॒ मे॒ । \newline
67. मे॒ शरी॑राणि॒ शरी॑राणि मे मे॒ शरी॑राणि । \newline
68. शरी॑राणि च च॒ शरी॑राणि॒ शरी॑राणि च । \newline
69. च॒ मे॒ मे॒ च॒ च॒ मे॒ । \newline
70. म॒ इति॑ मे । \newline

\textbf{Ghana Paata } \newline

1. च॒ मे॒ मे॒ च॒ च॒ मे॒ व्या॒नो व्या॒नो मे॑ च च मे व्या॒नः । \newline
2. मे॒ व्या॒नो व्या॒नो मे॑ मे व्या॒नश्च॑ च व्या॒नो मे॑ मे व्या॒नश्च॑ । \newline
3. व्या॒नश्च॑ च व्या॒नो व्या॒नश्च॑ मे मे च व्या॒नो व्या॒नश्च॑ मे । \newline
4. व्या॒न इति॑ वि - अ॒नः । \newline
5. च॒ मे॒ मे॒ च॒ च॒ मे ऽसु॒ रसु॑र् मे च च॒ मे ऽसुः॑ । \newline
6. मे ऽसु॒ रसु॑र् मे॒ मे ऽसु॑श्च॒ चासु॑र् मे॒ मे ऽसु॑श्च । \newline
7. असु॑श्च॒ चासु॒ रसु॑श्च मे मे॒ चासु॒ रसु॑श्च मे । \newline
8. च॒ मे॒ मे॒ च॒ च॒ मे॒ चि॒त्तम् चि॒त्तम् मे॑ च च मे चि॒त्तम् । \newline
9. मे॒ चि॒त्तम् चि॒त्तम् मे॑ मे चि॒त्तम् च॑ च चि॒त्तम् मे॑ मे चि॒त्तम् च॑ । \newline
10. चि॒त्तम् च॑ च चि॒त्तम् चि॒त्तम् च॑ मे मे च चि॒त्तम् चि॒त्तम् च॑ मे । \newline
11. च॒ मे॒ मे॒ च॒ च॒ म॒ आधी॑त॒ माधी॑तम् मे च च म॒ आधी॑तम् । \newline
12. म॒ आधी॑त॒ माधी॑तम् मे म॒ आधी॑तम् च॒ चाधी॑तम् मे म॒ आधी॑तम् च । \newline
13. आधी॑तम् च॒ चाधी॑त॒ माधी॑तम् च मे मे॒ चाधी॑त॒ माधी॑तम् च मे । \newline
14. आधी॑त॒मित्या - धी॒त॒म् । \newline
15. च॒ मे॒ मे॒ च॒ च॒ मे॒ वाग् वाङ् मे॑ च च मे॒ वाक् । \newline
16. मे॒ वाग् वाङ् मे॑ मे॒ वाक् च॑ च॒ वाङ् मे॑ मे॒ वाक् च॑ । \newline
17. वाक् च॑ च॒ वाग् वाक् च॑ मे मे च॒ वाग् वाक् च॑ मे । \newline
18. च॒ मे॒ मे॒ च॒ च॒ मे॒ मनो॒ मनो॑ मे च च मे॒ मनः॑ । \newline
19. मे॒ मनो॒ मनो॑ मे मे॒ मन॑श्च च॒ मनो॑ मे मे॒ मन॑श्च । \newline
20. मन॑श्च च॒ मनो॒ मन॑श्च मे मे च॒ मनो॒ मन॑श्च मे । \newline
21. च॒ मे॒ मे॒ च॒ च॒ मे॒ चक्षु॒ श्चक्षु॑र् मे च च मे॒ चक्षुः॑ । \newline
22. मे॒ चक्षु॒ श्चक्षु॑र् मे मे॒ चक्षु॑श्च च॒ चक्षु॑र् मे मे॒ चक्षु॑श्च । \newline
23. चक्षु॑श्च च॒ चक्षु॒ श्चक्षु॑श्च मे मे च॒ चक्षु॒ श्चक्षु॑श्च मे । \newline
24. च॒ मे॒ मे॒ च॒ च॒ मे॒ श्रोत्र॒(ग्ग्॒) श्रोत्र॑म् मे च च मे॒ श्रोत्र᳚म् । \newline
25. मे॒ श्रोत्र॒(ग्ग्॒) श्रोत्र॑म् मे मे॒ श्रोत्र॑म् च च॒ श्रोत्र॑म् मे मे॒ श्रोत्र॑म् च । \newline
26. श्रोत्र॑म् च च॒ श्रोत्र॒(ग्ग्॒) श्रोत्र॑म् च मे मे च॒ श्रोत्र॒(ग्ग्॒) श्रोत्र॑म् च मे । \newline
27. च॒ मे॒ मे॒ च॒ च॒ मे॒ दक्षो॒ दक्षो॑ मे च च मे॒ दक्षः॑ । \newline
28. मे॒ दक्षो॒ दक्षो॑ मे मे॒ दक्ष॑श्च च॒ दक्षो॑ मे मे॒ दक्ष॑श्च । \newline
29. दक्ष॑श्च च॒ दक्षो॒ दक्ष॑श्च मे मे च॒ दक्षो॒ दक्ष॑श्च मे । \newline
30. च॒ मे॒ मे॒ च॒ च॒ मे॒ बल॒म् बल॑म् मे च च मे॒ बल᳚म् । \newline
31. मे॒ बल॒म् बल॑म् मे मे॒ बल॑म् च च॒ बल॑म् मे मे॒ बल॑म् च । \newline
32. बल॑म् च च॒ बल॒म् बल॑म् च मे मे च॒ बल॒म् बल॑म् च मे । \newline
33. च॒ मे॒ मे॒ च॒ च॒ म॒ ओज॒ ओजो॑ मे च च म॒ ओजः॑ । \newline
34. म॒ ओज॒ ओजो॑ मे म॒ ओज॑श्च॒ चौजो॑ मे म॒ ओज॑श्च । \newline
35. ओज॑श्च॒ चौज॒ ओज॑श्च मे मे॒ चौज॒ ओज॑श्च मे । \newline
36. च॒ मे॒ मे॒ च॒ च॒ मे॒ सहः॒ सहो॑ मे च च मे॒ सहः॑ । \newline
37. मे॒ सहः॒ सहो॑ मे मे॒ सह॑श्च च॒ सहो॑ मे मे॒ सह॑श्च । \newline
38. सह॑श्च च॒ सहः॒ सह॑श्च मे मे च॒ सहः॒ सह॑श्च मे । \newline
39. च॒ मे॒ मे॒ च॒ च॒ म॒ आयु॒ रायु॑र् मे च च म॒ आयुः॑ । \newline
40. म॒ आयु॒ रायु॑र् मे म॒ आयु॑श्च॒ चायु॑र् मे म॒ आयु॑श्च । \newline
41. आयु॑श्च॒ चायु॒ रायु॑श्च मे मे॒ चायु॒ रायु॑श्च मे । \newline
42. च॒ मे॒ मे॒ च॒ च॒ मे॒ ज॒रा ज॒रा मे॑ च च मे ज॒रा । \newline
43. मे॒ ज॒रा ज॒रा मे॑ मे ज॒रा च॑ च ज॒रा मे॑ मे ज॒रा च॑ । \newline
44. ज॒रा च॑ च ज॒रा ज॒रा च॑ मे मे च ज॒रा ज॒रा च॑ मे । \newline
45. च॒ मे॒ मे॒ च॒ च॒ म॒ आ॒त्मा ऽऽत्मा मे॑ च च म आ॒त्मा । \newline
46. म॒ आ॒त्मा ऽऽत्मा मे॑ म आ॒त्मा च॑ चा॒त्मा मे॑ म आ॒त्मा च॑ । \newline
47. आ॒त्मा च॑ चा॒त्मा ऽऽत्मा च॑ मे मे चा॒त्मा ऽऽत्मा च॑ मे । \newline
48. च॒ मे॒ मे॒ च॒ च॒ मे॒ त॒नू स्त॒नूर् मे॑ च च मे त॒नूः । \newline
49. मे॒ त॒नू स्त॒नूर् मे॑ मे त॒नूश्च॑ च त॒नूर् मे॑ मे त॒नूश्च॑ । \newline
50. त॒नूश्च॑ च त॒नू स्त॒नूश्च॑ मे मे च त॒नू स्त॒नूश्च॑ मे । \newline
51. च॒ मे॒ मे॒ च॒ च॒ मे॒ शर्म॒ शर्म॑ मे च च मे॒ शर्म॑ । \newline
52. मे॒ शर्म॒ शर्म॑ मे मे॒ शर्म॑ च च॒ शर्म॑ मे मे॒ शर्म॑ च । \newline
53. शर्म॑ च च॒ शर्म॒ शर्म॑ च मे मे च॒ शर्म॒ शर्म॑ च मे । \newline
54. च॒ मे॒ मे॒ च॒ च॒ मे॒ वर्म॒ वर्म॑ मे च च मे॒ वर्म॑ । \newline
55. मे॒ वर्म॒ वर्म॑ मे मे॒ वर्म॑ च च॒ वर्म॑ मे मे॒ वर्म॑ च । \newline
56. वर्म॑ च च॒ वर्म॒ वर्म॑ च मे मे च॒ वर्म॒ वर्म॑ च मे । \newline
57. च॒ मे॒ मे॒ च॒ च॒ मे ऽङ्गा॒ न्यङ्गा॑नि मे च च॒ मे ऽङ्गा॑नि । \newline
58. मे ऽङ्गा॒ न्यङ्गा॑नि मे॒ मे ऽङ्गा॑नि च॒ चाङ्गा॑नि मे॒ मे ऽङ्गा॑नि च । \newline
59. अङ्गा॑नि च॒ चाङ्गा॒ न्यङ्गा॑नि च मे मे॒ चाङ्गा॒ न्यङ्गा॑नि च मे । \newline
60. च॒ मे॒ मे॒ च॒ च॒ मे॒ ऽस्था न्य॒स्थानि॑ मे च च मे॒ ऽस्थानि॑ । \newline
61. मे॒ ऽस्था न्य॒स्थानि॑ मे मे॒ ऽस्थानि॑ च चा॒स्थानि॑ मे मे॒ ऽस्थानि॑ च । \newline
62. अ॒स्थानि॑ च चा॒स्था न्य॒स्थानि॑ च मे मे चा॒स्था न्य॒स्थानि॑ च मे । \newline
63. च॒ मे॒ मे॒ च॒ च॒ मे॒ परू(ग्म्॑)षि॒ परू(ग्म्॑)षि मे च च मे॒ परू(ग्म्॑)षि । \newline
64. मे॒ परू(ग्म्॑)षि॒ परू(ग्म्॑)षि मे मे॒ परू(ग्म्॑)षि च च॒ परू(ग्म्॑)षि मे मे॒ परू(ग्म्॑)षि च । \newline
65. परू(ग्म्॑)षि च च॒ परू(ग्म्॑)षि॒ परू(ग्म्॑)षि च मे मे च॒ परू(ग्म्॑)षि॒ परू(ग्म्॑)षि च मे । \newline
66. च॒ मे॒ मे॒ च॒ च॒ मे॒ शरी॑राणि॒ शरी॑राणि मे च च मे॒ शरी॑राणि । \newline
67. मे॒ शरी॑राणि॒ शरी॑राणि मे मे॒ शरी॑राणि च च॒ शरी॑राणि मे मे॒ शरी॑राणि च । \newline
68. शरी॑राणि च च॒ शरी॑राणि॒ शरी॑राणि च मे मे च॒ शरी॑राणि॒ शरी॑राणि च मे । \newline
69. च॒ मे॒ मे॒ च॒ च॒ मे॒ । \newline
70. म॒ इति॑ मे । \newline
\pagebreak
\markright{ TS 4.7.2.1  \hfill https://www.vedavms.in \hfill}

\section{ TS 4.7.2.1 }

\textbf{TS 4.7.2.1 } \newline
\textbf{Samhita Paata} \newline

ज्यैष्ठ्यं॑ च म॒ आधि॑पत्यं च मे म॒न्युश्च॑ मे॒ भाम॑श्च॒ मेऽम॑श्च॒ मे ऽभं॑श्च मे जे॒मा च॑ मे महि॒मा च॑ मे वरि॒मा च॑ मे      प्रथि॒मा च॑ मे व॒र्ष्मा च॑ मे द्राघु॒या च॑ मे वृ॒द्धं च॑ मे॒      वृद्धि॑श्च मे स॒त्यं च॑ मे श्र॒द्धा च॑ मे॒ जग॑च्च - [  ] \newline

\textbf{Pada Paata} \newline

ज्यैष्ठ्य᳚म् । च॒ । मे॒ । आधि॑पत्य॒मित्याधि॑-प॒त्य॒म् । च॒ । मे॒ । म॒न्युः । च॒ । मे॒ । भामः॑ । च॒ । मे॒ । अमः॑ । च॒ । मे॒ । अभंः॑ । च॒ । मे॒ । जे॒मा । च॒ । मे॒ । म॒हि॒मा । च॒ । मे॒ । व॒रि॒मा । च॒ । मे॒ । प्र॒थि॒मा । च॒ । मे॒ । व॒र्ष्मा । च॒ । मे॒ । द्रा॒घु॒या । च॒ । मे॒ । वृ॒द्धम् । च॒ । मे॒ । वृद्धिः॑ । च॒ । मे॒ । स॒त्यम् । च॒ । मे॒ । श्र॒द्धेति॑ श्रत् - धा । च॒ । मे॒ । जग॑त् । च॒ ।  \newline


\textbf{Krama Paata} \newline

ज्यैष्ठ्य॑म् च । च॒ मे॒ । म॒ आधि॑पत्यम् । आधि॑पत्यम् च । आधि॑पत्य॒मित्याधि॑ - प॒त्य॒म् । च॒ मे॒ । मे॒ म॒न्युः । म॒न्युश्च॑ । च॒ मे॒ । मे॒ भामः॑ । भाम॑श्च । च॒ मे॒ । मेऽमः॑ । अम॑श्च । च॒ मे॒ । मेऽम्भः॑ । अम्भ॑श्च । च॒ मे॒ । मे॒ जे॒मा । जे॒मा च॑ । च॒ मे॒ । मे॒ म॒हि॒मा । म॒हि॒मा च॑ । च॒ मे॒ । मे॒ व॒रि॒मा । व॒रि॒मा च॑ । च॒ मे॒ । मे॒ प्र॒थि॒मा । प्र॒थि॒मा च॑ । च॒ मे॒ । मे॒ व॒र्ष्मा । व॒र्ष्मा च॑ । च॒ मे॒ । मे॒ द्रा॒घु॒या । द्रा॒घु॒या च॑ । च॒ मे॒ । मे॒ वृ॒द्धम् । वृ॒द्धम् च॑ । च॒ मे॒ । मे॒ वृद्धिः॑ । वृद्धि॑श्च । च॒ मे॒ । मे॒ स॒त्यम् । स॒त्यम् च॑ । च॒ मे॒ । मे॒ श्र॒द्धा । श्र॒द्धा च॑ । श्र॒द्धेति॑ श्रत् - धा । च॒ मे॒ । मे॒ जग॑त् । जग॑च् च । च॒ मे॒ \newline

\textbf{Jatai Paata} \newline

1. ज्यैष्ठ्य॑म् च च॒ ज्यैष्ठ्य॒म् ज्यैष्ठ्य॑म् च । \newline
2. च॒ मे॒ मे॒ च॒ च॒ मे॒ । \newline
3. म॒ आधि॑पत्य॒ माधि॑पत्यम् मे म॒ आधि॑पत्यम् । \newline
4. आधि॑पत्यम् च॒ चाधि॑पत्य॒ माधि॑पत्यम् च । \newline
5. आधि॑पत्य॒मित्याधि॑ - प॒त्य॒म् । \newline
6. च॒ मे॒ मे॒ च॒ च॒ मे॒ । \newline
7. मे॒ म॒न्युर् म॒न्युर् मे॑ मे म॒न्युः । \newline
8. म॒न्युश्च॑ च म॒न्युर् म॒न्युश्च॑ । \newline
9. च॒ मे॒ मे॒ च॒ च॒ मे॒ । \newline
10. मे॒ भामो॒ भामो॑ मे मे॒ भामः॑ । \newline
11. भाम॑श्च च॒ भामो॒ भाम॑श्च । \newline
12. च॒ मे॒ मे॒ च॒ च॒ मे॒ । \newline
13. मे ऽमो ऽमो॑ मे॒ मे ऽमः॑ । \newline
14. अम॑श्च॒ चामो ऽम॑श्च । \newline
15. च॒ मे॒ मे॒ च॒ च॒ मे॒ । \newline
16. मे ऽम्भो ऽम्भो॑ मे॒ मे ऽम्भः॑ । \newline
17. अम्भ॑श्च॒ चाम्भो ऽम्भ॑श्च । \newline
18. च॒ मे॒ मे॒ च॒ च॒ मे॒ । \newline
19. मे॒ जे॒मा जे॒मा मे॑ मे जे॒मा । \newline
20. जे॒मा च॑ च जे॒मा जे॒मा च॑ । \newline
21. च॒ मे॒ मे॒ च॒ च॒ मे॒ । \newline
22. मे॒ म॒हि॒मा म॑हि॒मा मे॑ मे महि॒मा । \newline
23. म॒हि॒मा च॑ च महि॒मा म॑हि॒मा च॑ । \newline
24. च॒ मे॒ मे॒ च॒ च॒ मे॒ । \newline
25. मे॒ व॒रि॒मा व॑रि॒मा मे॑ मे वरि॒मा । \newline
26. व॒रि॒मा च॑ च वरि॒मा व॑रि॒मा च॑ । \newline
27. च॒ मे॒ मे॒ च॒ च॒ मे॒ । \newline
28. मे॒ प्र॒थि॒मा प्र॑थि॒मा मे॑ मे प्रथि॒मा । \newline
29. प्र॒थि॒मा च॑ च प्रथि॒मा प्र॑थि॒मा च॑ । \newline
30. च॒ मे॒ मे॒ च॒ च॒ मे॒ । \newline
31. मे॒ व॒र्ष्मा व॒र्ष्मा मे॑ मे व॒र्ष्मा । \newline
32. व॒र्ष्मा च॑ च व॒र्ष्मा व॒र्ष्मा च॑ । \newline
33. च॒ मे॒ मे॒ च॒ च॒ मे॒ । \newline
34. मे॒ द्रा॒घु॒या द्रा॑घु॒या मे॑ मे द्राघु॒या । \newline
35. द्रा॒घु॒या च॑ च द्राघु॒या द्रा॑घु॒या च॑ । \newline
36. च॒ मे॒ मे॒ च॒ च॒ मे॒ । \newline
37. मे॒ वृ॒द्धं ॅवृ॒द्धम् मे॑ मे वृ॒द्धम् । \newline
38. वृ॒द्धम् च॑ च वृ॒द्धं ॅवृ॒द्धम् च॑ । \newline
39. च॒ मे॒ मे॒ च॒ च॒ मे॒ । \newline
40. मे॒ वृद्धि॒र् वृद्धि॑र् मे मे॒ वृद्धिः॑ । \newline
41. वृद्धि॑श्च च॒ वृद्धि॒र् वृद्धि॑श्च । \newline
42. च॒ मे॒ मे॒ च॒ च॒ मे॒ । \newline
43. मे॒ स॒त्यꣳ स॒त्यम् मे॑ मे स॒त्यम् । \newline
44. स॒त्यम् च॑ च स॒त्यꣳ स॒त्यम् च॑ । \newline
45. च॒ मे॒ मे॒ च॒ च॒ मे॒ । \newline
46. मे॒ श्र॒द्धा श्र॒द्धा मे॑ मे श्र॒द्धा । \newline
47. श्र॒द्धा च॑ च श्र॒द्धा श्र॒द्धा च॑ । \newline
48. श्र॒द्धेति॑ श्रत् - धा । \newline
49. च॒ मे॒ मे॒ च॒ च॒ मे॒ । \newline
50. मे॒ जग॒ज् जग॑न् मे मे॒ जग॑त् । \newline
51. जग॑च् च च॒ जग॒ज् जग॑च् च । \newline
52. च॒ मे॒ मे॒ च॒ च॒ मे॒ । \newline

\textbf{Ghana Paata } \newline

1. ज्यैष्ठ्य॑म् च च॒ ज्यैष्ठ्य॒म् ज्यैष्ठ्य॑म् च मे मे च॒ ज्यैष्ठ्य॒म् ज्यैष्ठ्य॑म् च मे । \newline
2. च॒ मे॒ मे॒ च॒ च॒ म॒ आधि॑पत्य॒ माधि॑पत्यम् मे च च म॒ आधि॑पत्यम् । \newline
3. म॒ आधि॑पत्य॒ माधि॑पत्यम् मे म॒ आधि॑पत्यम् च॒ चाधि॑पत्यम् मे म॒ आधि॑पत्यम् च । \newline
4. आधि॑पत्यम् च॒ चाधि॑पत्य॒ माधि॑पत्यम् च मे मे॒ चाधि॑पत्य॒ माधि॑पत्यम् च मे । \newline
5. आधि॑पत्य॒मित्याधि॑ - प॒त्य॒म् । \newline
6. च॒ मे॒ मे॒ च॒ च॒ मे॒ म॒न्युर् म॒न्युर् मे॑ च च मे म॒न्युः । \newline
7. मे॒ म॒न्युर् म॒न्युर् मे॑ मे म॒न्युश्च॑ च म॒न्युर् मे॑ मे म॒न्युश्च॑ । \newline
8. म॒न्युश्च॑ च म॒न्युर् म॒न्युश्च॑ मे मे च म॒न्युर् म॒न्युश्च॑ मे । \newline
9. च॒ मे॒ मे॒ च॒ च॒ मे॒ भामो॒ भामो॑ मे च च मे॒ भामः॑ । \newline
10. मे॒ भामो॒ भामो॑ मे मे॒ भाम॑श्च च॒ भामो॑ मे मे॒ भाम॑श्च । \newline
11. भाम॑श्च च॒ भामो॒ भाम॑श्च मे मे च॒ भामो॒ भाम॑श्च मे । \newline
12. च॒ मे॒ मे॒ च॒ च॒ मे ऽमो ऽमो॑ मे च च॒ मे ऽमः॑ । \newline
13. मे ऽमो ऽमो॑ मे॒ मे ऽम॑श्च॒ चामो॑ मे॒ मे ऽम॑श्च । \newline
14. अम॑श्च॒ चामो ऽम॑श्च मे मे॒ चामो ऽम॑श्च मे । \newline
15. च॒ मे॒ मे॒ च॒ च॒ मे ऽम्भो ऽम्भो॑ मे च च॒ मे ऽम्भः॑ । \newline
16. मे ऽम्भो ऽम्भो॑ मे॒ मे ऽम्भ॑श्च॒ चाम्भो॑ मे॒ मे ऽम्भ॑श्च । \newline
17. अम्भ॑श्च॒ चाम्भो ऽम्भ॑श्च मे मे॒ चाम्भो ऽम्भ॑श्च मे । \newline
18. च॒ मे॒ मे॒ च॒ च॒ मे॒ जे॒मा जे॒मा मे॑ च च मे जे॒मा । \newline
19. मे॒ जे॒मा जे॒मा मे॑ मे जे॒मा च॑ च जे॒मा मे॑ मे जे॒मा च॑ । \newline
20. जे॒मा च॑ च जे॒मा जे॒मा च॑ मे मे च जे॒मा जे॒मा च॑ मे । \newline
21. च॒ मे॒ मे॒ च॒ च॒ मे॒ म॒हि॒मा म॑हि॒मा मे॑ च च मे महि॒मा । \newline
22. मे॒ म॒हि॒मा म॑हि॒मा मे॑ मे महि॒मा च॑ च महि॒मा मे॑ मे महि॒मा च॑ । \newline
23. म॒हि॒मा च॑ च महि॒मा म॑हि॒मा च॑ मे मे च महि॒मा म॑हि॒मा च॑ मे । \newline
24. च॒ मे॒ मे॒ च॒ च॒ मे॒ व॒रि॒मा व॑रि॒मा मे॑ च च मे वरि॒मा । \newline
25. मे॒ व॒रि॒मा व॑रि॒मा मे॑ मे वरि॒मा च॑ च वरि॒मा मे॑ मे वरि॒मा च॑ । \newline
26. व॒रि॒मा च॑ च वरि॒मा व॑रि॒मा च॑ मे मे च वरि॒मा व॑रि॒मा च॑ मे । \newline
27. च॒ मे॒ मे॒ च॒ च॒ मे॒ प्र॒थि॒मा प्र॑थि॒मा मे॑ च च मे प्रथि॒मा । \newline
28. मे॒ प्र॒थि॒मा प्र॑थि॒मा मे॑ मे प्रथि॒मा च॑ च प्रथि॒मा मे॑ मे प्रथि॒मा च॑ । \newline
29. प्र॒थि॒मा च॑ च प्रथि॒मा प्र॑थि॒मा च॑ मे मे च प्रथि॒मा प्र॑थि॒मा च॑ मे । \newline
30. च॒ मे॒ मे॒ च॒ च॒ मे॒ व॒र्ष्मा व॒र्ष्मा मे॑ च च मे व॒र्ष्मा । \newline
31. मे॒ व॒र्ष्मा व॒र्ष्मा मे॑ मे व॒र्ष्मा च॑ च व॒र्ष्मा मे॑ मे व॒र्ष्मा च॑ । \newline
32. व॒र्ष्मा च॑ च व॒र्ष्मा व॒र्ष्मा च॑ मे मे च व॒र्ष्मा व॒र्ष्मा च॑ मे । \newline
33. च॒ मे॒ मे॒ च॒ च॒ मे॒ द्रा॒घु॒या द्रा॑घु॒या मे॑ च च मे द्राघु॒या । \newline
34. मे॒ द्रा॒घु॒या द्रा॑घु॒या मे॑ मे द्राघु॒या च॑ च द्राघु॒या मे॑ मे द्राघु॒या च॑ । \newline
35. द्रा॒घु॒या च॑ च द्राघु॒या द्रा॑घु॒या च॑ मे मे च द्राघु॒या द्रा॑घु॒या च॑ मे । \newline
36. च॒ मे॒ मे॒ च॒ च॒ मे॒ वृ॒द्धं ॅवृ॒द्धम् मे॑ च च मे वृ॒द्धम् । \newline
37. मे॒ वृ॒द्धं ॅवृ॒द्धम् मे॑ मे वृ॒द्धम् च॑ च वृ॒द्धम् मे॑ मे वृ॒द्धम् च॑ । \newline
38. वृ॒द्धम् च॑ च वृ॒द्धं ॅवृ॒द्धम् च॑ मे मे च वृ॒द्धं ॅवृ॒द्धम् च॑ मे । \newline
39. च॒ मे॒ मे॒ च॒ च॒ मे॒ वृद्धि॒र् वृद्धि॑र् मे च च मे॒ वृद्धिः॑ । \newline
40. मे॒ वृद्धि॒र् वृद्धि॑र् मे मे॒ वृद्धि॑श्च च॒ वृद्धि॑र् मे मे॒ वृद्धि॑श्च । \newline
41. वृद्धि॑श्च च॒ वृद्धि॒र् वृद्धि॑श्च मे मे च॒ वृद्धि॒र् वृद्धि॑श्च मे । \newline
42. च॒ मे॒ मे॒ च॒ च॒ मे॒ स॒त्यꣳ स॒त्यम् मे॑ च च मे स॒त्यम् । \newline
43. मे॒ स॒त्यꣳ स॒त्यम् मे॑ मे स॒त्यम् च॑ च स॒त्यम् मे॑ मे स॒त्यम् च॑ । \newline
44. स॒त्यम् च॑ च स॒त्यꣳ स॒त्यम् च॑ मे मे च स॒त्यꣳ स॒त्यम् च॑ मे । \newline
45. च॒ मे॒ मे॒ च॒ च॒ मे॒ श्र॒द्धा श्र॒द्धा मे॑ च च मे श्र॒द्धा । \newline
46. मे॒ श्र॒द्धा श्र॒द्धा मे॑ मे श्र॒द्धा च॑ च श्र॒द्धा मे॑ मे श्र॒द्धा च॑ । \newline
47. श्र॒द्धा च॑ च श्र॒द्धा श्र॒द्धा च॑ मे मे च श्र॒द्धा श्र॒द्धा च॑ मे । \newline
48. श्र॒द्धेति॑ श्रत् - धा । \newline
49. च॒ मे॒ मे॒ च॒ च॒ मे॒ जग॒ज् जग॑न् मे च च मे॒ जग॑त् । \newline
50. मे॒ जग॒ज् जग॑न् मे मे॒ जग॑च् च च॒ जग॑न् मे मे॒ जग॑च् च । \newline
51. जग॑च् च च॒ जग॒ज् जग॑च् च मे मे च॒ जग॒ज् जग॑च् च मे । \newline
52. च॒ मे॒ मे॒ च॒ च॒ मे॒ धन॒म् धन॑म् मे च च मे॒ धन᳚म् । \newline
\pagebreak
\markright{ TS 4.7.2.2  \hfill https://www.vedavms.in \hfill}

\section{ TS 4.7.2.2 }

\textbf{TS 4.7.2.2 } \newline
\textbf{Samhita Paata} \newline

-मे॒ धनं॑ च मे॒ वश॑श्च मे॒ त्विषि॑श्च मे क्री॒डा च॑ मे॒मोद॑श्च मे जा॒तं च॑ मे जनि॒ष्यमा॑णं च मे सू॒क्तं च॑ मे सुकृ॒तं च॑ मे वि॒त्तं च॑ मे॒ वेद्यं॑ च मे भू॒तंच॑ मे भवि॒ष्यच्च॑ मेसु॒गं च॑ मे सु॒पथं॑ च म ऋ॒द्धं च॑ म॒ ऋद्धि॑ ( ) श्च मे क्लृ॒प्तं च॑ मे॒    क्लृप्ति॑श्च मे म॒तिश्च॑ मे सुम॒तिश्च॑ मे ॥ \newline

\textbf{Pada Paata} \newline

मे॒ । धन᳚म् । च॒ । मे॒ । वशः॑ । च॒ । मे॒ । त्विषिः॑ । च॒ । मे॒ । क्री॒डा । च॒ । मे॒ । मोदः॑ । च॒ । मे॒ । जा॒तम् । च॒ । मे॒ । ज॒नि॒ष्यमा॑णम् । च॒ । मे॒ । सू॒क्तमिति॑ सु - उ॒क्तम् । च॒ । मे॒ । सु॒कृ॒तमिति॑ सु - कृ॒तम् । च॒ । मे॒ । वि॒त्तम् । च॒ । मे॒ । वेद्य᳚म् । च॒ । मे॒ । भू॒तम् । च॒ । मे॒ । भ॒वि॒ष्यत् । च॒ । मे॒ । सु॒गमिति॑ सु - गम् । च॒ । मे॒ । सु॒पथ॒मिति॑ सु - पथ᳚म् । च॒ । मे॒ । ऋ॒द्धम् । च॒ । मे॒ । ऋद्धिः॑ ( ) । च॒ । मे॒ । क्लृ॒प्तम् । च॒ । मे॒ । क्लृप्तिः॑ । च॒ । मे॒ । म॒तिः । च॒ । मे॒ । सु॒म॒तिरिति॑ सु - म॒तिः । च॒ । मे॒ ॥  \newline


\textbf{Krama Paata} \newline

मे॒ धन᳚म् । धन॑म् च । च॒ मे॒ । मे॒ वशः॑ । वश॑श्च । च॒ मे॒ । मे॒ त्विषिः॑ । त्विषि॑श्च । च॒ मे॒ । मे॒ क्री॒डा । क्री॒डा च॑ । च॒ मे॒ । मे॒ मोदः॑ । मोद॑श्च । च॒ मे॒ । मे॒ जा॒तम् । जा॒तम् च॑ । च॒ मे॒ । मे॒ ज॒नि॒ष्यमा॑णम् । ज॒नि॒ष्यमा॑णम् च । च॒ मे॒ । मे॒ सू॒क्तम् । सू॒क्तम् च॑ । सू॒क्तमिति॑ सु - उ॒क्तम् । च॒ मे॒ । मे॒ सु॒कृ॒तम् । सु॒कृ॒तम् च॑ । सु॒कृ॒तमिति॑ सु - कृ॒तम् । च॒ मे॒ । म॒ वि॒त्तम् । वि॒त्तम् च॑ । च॒ मे॒ । मे॒ वेद्य᳚म् । वेद्य॑म् च । च॒ मे॒ । मे॒ भू॒तम् । भू॒तम् च॑ । च॒ मे॒ । मे॒ भ॒वि॒ष्यत् । भ॒वि॒ष्यच् च॑ । च॒ मे॒ । मे॒ सु॒गम् । सु॒गम् च॑ । सु॒गमिति॑ सु - गम् । च॒ मे॒ । मे॒ सु॒पथ᳚म् । सु॒पथ॑म् च । सु॒पथ॒मिति॑ सु - पथ᳚म् । च॒ मे॒ । म॒ ऋ॒द्धम् । ऋ॒द्धम् च॑ । च॒ मे॒ । म॒ ऋद्धिः॑ ( ) । ऋद्धि॑श्च । च॒ मे॒ । मे॒ क्लृ॒प्तम् । क्लृ॒प्तम् च॑ । च॒ मे॒ । मे॒ क्लृप्तिः॑ । क्लृप्ति॑श्च । च॒ मे॒ । मे॒ म॒तिः । म॒तिश्च॑ । च॒ मे॒ । मे॒ सु॒म॒तिः । सु॒म॒तिश्च॑ । सु॒म॒तिरिति॑ सु - म॒तिः । च॒ मे॒ । म॒ इति॑ मे । \newline

\textbf{Jatai Paata} \newline

1. मे॒ धन॒म् धन॑म् मे मे॒ धन᳚म् । \newline
2. धन॑म् च च॒ धन॒म् धन॑म् च । \newline
3. च॒ मे॒ मे॒ च॒ च॒ मे॒ । \newline
4. मे॒ वशो॒ वशो॑ मे मे॒ वशः॑ । \newline
5. वश॑श्च च॒ वशो॒ वश॑श्च । \newline
6. च॒ मे॒ मे॒ च॒ च॒ मे॒ । \newline
7. मे॒ त्विषि॒ स्त्विषि॑र् मे मे॒ त्विषिः॑ । \newline
8. त्विषि॑श्च च॒ त्विषि॒ स्त्विषि॑श्च । \newline
9. च॒ मे॒ मे॒ च॒ च॒ मे॒ । \newline
10. मे॒ क्री॒डा क्री॒डा मे॑ मे क्री॒डा । \newline
11. क्री॒डा च॑ च क्री॒डा क्री॒डा च॑ । \newline
12. च॒ मे॒ मे॒ च॒ च॒ मे॒ । \newline
13. मे॒ मोदो॒ मोदो॑ मे मे॒ मोदः॑ । \newline
14. मोद॑श्च च॒ मोदो॒ मोद॑श्च । \newline
15. च॒ मे॒ मे॒ च॒ च॒ मे॒ । \newline
16. मे॒ जा॒तम् जा॒तम् मे॑ मे जा॒तम् । \newline
17. जा॒तम् च॑ च जा॒तम् जा॒तम् च॑ । \newline
18. च॒ मे॒ मे॒ च॒ च॒ मे॒ । \newline
19. मे॒ ज॒नि॒ष्यमा॑णम् जनि॒ष्यमा॑णम् मे मे जनि॒ष्यमा॑णम् । \newline
20. ज॒नि॒ष्यमा॑णम् च च जनि॒ष्यमा॑णम् जनि॒ष्यमा॑णम् च । \newline
21. च॒ मे॒ मे॒ च॒ च॒ मे॒ । \newline
22. मे॒ सू॒क्तꣳ सू॒क्तम् मे॑ मे सू॒क्तम् । \newline
23. सू॒क्तम् च॑ च सू॒क्तꣳ सू॒क्तम् च॑ । \newline
24. सू॒क्तमिति॑ सु - उ॒क्तम् । \newline
25. च॒ मे॒ मे॒ च॒ च॒ मे॒ । \newline
26. मे॒ सु॒कृ॒तꣳ सु॑कृ॒तम् मे॑ मे सुकृ॒तम् । \newline
27. सु॒कृ॒तम् च॑ च सुकृ॒तꣳ सु॑कृ॒तम् च॑ । \newline
28. सु॒कृ॒तमिति॑ सु - कृ॒तम् । \newline
29. च॒ मे॒ मे॒ च॒ च॒ मे॒ । \newline
30. मे॒ वि॒त्तं ॅवि॒त्तम् मे॑ मे वि॒त्तम् । \newline
31. वि॒त्तम् च॑ च वि॒त्तं ॅवि॒त्तम् च॑ । \newline
32. च॒ मे॒ मे॒ च॒ च॒ मे॒ । \newline
33. मे॒ वेद्यं॒ ॅवेद्य॑म् मे मे॒ वेद्य᳚म् । \newline
34. वेद्य॑म् च च॒ वेद्यं॒ ॅवेद्य॑म् च । \newline
35. च॒ मे॒ मे॒ च॒ च॒ मे॒ । \newline
36. मे॒ भू॒तम् भू॒तम् मे॑ मे भू॒तम् । \newline
37. भू॒तम् च॑ च भू॒तम् भू॒तम् च॑ । \newline
38. च॒ मे॒ मे॒ च॒ च॒ मे॒ । \newline
39. मे॒ भ॒वि॒ष्यद् भ॑वि॒ष्यन् मे॑ मे भवि॒ष्यत् । \newline
40. भ॒वि॒ष्यच् च॑ च भवि॒ष्यद् भ॑वि॒ष्यच् च॑ । \newline
41. च॒ मे॒ मे॒ च॒ च॒ मे॒ । \newline
42. मे॒ सु॒गꣳ सु॒गम् मे॑ मे सु॒गम् । \newline
43. सु॒गम् च॑ च सु॒गꣳ सु॒गम् च॑ । \newline
44. सु॒गमिति॑ सु - गम् । \newline
45. च॒ मे॒ मे॒ च॒ च॒ मे॒ । \newline
46. मे॒ सु॒पथ(ग्म्॑) सु॒पथ॑म् मे मे सु॒पथ᳚म् । \newline
47. सु॒पथ॑म् च च सु॒पथ(ग्म्॑) सु॒पथ॑म् च । \newline
48. सु॒पथ॒मिति॑ सु - पथ᳚म् । \newline
49. च॒ मे॒ मे॒ च॒ च॒ मे॒ । \newline
50. म॒ ऋ॒द्ध मृ॒द्धम् मे॑ म ऋ॒द्धम् । \newline
51. ऋ॒द्धम् च॑ च॒ र्‌द्ध मृ॒द्धम् च॑ । \newline
52. च॒ मे॒ मे॒ च॒ च॒ मे॒ । \newline
53. म॒ ऋद्धि॒र्॒. ऋद्धि॑र् मे म॒ ऋद्धिः॑ । \newline
54. ऋद्धि॑श्च॒ चर्‌द्धि॒र्॒. ऋद्धि॑श्च । \newline
55. च॒ मे॒ मे॒ च॒ च॒ मे॒ । \newline
56. मे॒ क्लृ॒प्तम् क्लृ॒प्तम् मे॑ मे क्लृ॒प्तम् । \newline
57. क्लृ॒प्तम् च॑ च क्लृ॒प्तम् क्लृ॒प्तम् च॑ । \newline
58. च॒ मे॒ मे॒ च॒ च॒ मे॒ । \newline
59. मे॒ क्लृप्तिः॒ क्लृप्ति॑र् मे मे॒ क्लृप्तिः॑ । \newline
60. क्लृप्ति॑श्च च॒ क्लृप्तिः॒ क्लृप्ति॑श्च । \newline
61. च॒ मे॒ मे॒ च॒ च॒ मे॒ । \newline
62. मे॒ म॒तिर् म॒तिर् मे॑ मे म॒तिः । \newline
63. म॒तिश्च॑ च म॒तिर् म॒तिश्च॑ । \newline
64. च॒ मे॒ मे॒ च॒ च॒ मे॒ । \newline
65. मे॒ सु॒म॒तिः सु॑म॒तिर् मे॑ मे सुम॒तिः । \newline
66. सु॒म॒तिश्च॑ च सुम॒तिः सु॑म॒तिश्च॑ । \newline
67. सु॒म॒तिरिति॑ सु - म॒तिः । \newline
68. च॒ मे॒ मे॒ च॒ च॒ मे॒ । \newline
69. म॒ इति॑ मे । \newline

\textbf{Ghana Paata } \newline

1. मे॒ धन॒म् धन॑म् मे मे॒ धन॑म् च च॒ धन॑म् मे मे॒ धन॑म् च । \newline
2. धन॑म् च च॒ धन॒म् धन॑म् च मे मे च॒ धन॒म् धन॑म् च मे । \newline
3. च॒ मे॒ मे॒ च॒ च॒ मे॒ वशो॒ वशो॑ मे च च मे॒ वशः॑ । \newline
4. मे॒ वशो॒ वशो॑ मे मे॒ वश॑श्च च॒ वशो॑ मे मे॒ वश॑श्च । \newline
5. वश॑श्च च॒ वशो॒ वश॑श्च मे मे च॒ वशो॒ वश॑श्च मे । \newline
6. च॒ मे॒ मे॒ च॒ च॒ मे॒ त्विषि॒ स्त्विषि॑र् मे च च मे॒ त्विषिः॑ । \newline
7. मे॒ त्विषि॒ स्त्विषि॑र् मे मे॒ त्विषि॑श्च च॒ त्विषि॑र् मे मे॒ त्विषि॑श्च । \newline
8. त्विषि॑श्च च॒ त्विषि॒ स्त्विषि॑श्च मे मे च॒ त्विषि॒ स्त्विषि॑श्च मे । \newline
9. च॒ मे॒ मे॒ च॒ च॒ मे॒ क्री॒डा क्री॒डा मे॑ च च मे क्री॒डा । \newline
10. मे॒ क्री॒डा क्री॒डा मे॑ मे क्री॒डा च॑ च क्री॒डा मे॑ मे क्री॒डा च॑ । \newline
11. क्री॒डा च॑ च क्री॒डा क्री॒डा च॑ मे मे च क्री॒डा क्री॒डा च॑ मे । \newline
12. च॒ मे॒ मे॒ च॒ च॒ मे॒ मोदो॒ मोदो॑ मे च च मे॒ मोदः॑ । \newline
13. मे॒ मोदो॒ मोदो॑ मे मे॒ मोद॑श्च च॒ मोदो॑ मे मे॒ मोद॑श्च । \newline
14. मोद॑श्च च॒ मोदो॒ मोद॑श्च मे मे च॒ मोदो॒ मोद॑श्च मे । \newline
15. च॒ मे॒ मे॒ च॒ च॒ मे॒ जा॒तम् जा॒तम् मे॑ च च मे जा॒तम् । \newline
16. मे॒ जा॒तम् जा॒तम् मे॑ मे जा॒तम् च॑ च जा॒तम् मे॑ मे जा॒तम् च॑ । \newline
17. जा॒तम् च॑ च जा॒तम् जा॒तम् च॑ मे मे च जा॒तम् जा॒तम् च॑ मे । \newline
18. च॒ मे॒ मे॒ च॒ च॒ मे॒ ज॒नि॒ष्यमा॑णम् जनि॒ष्यमा॑णम् मे च च मे जनि॒ष्यमा॑णम् । \newline
19. मे॒ ज॒नि॒ष्यमा॑णम् जनि॒ष्यमा॑णम् मे मे जनि॒ष्यमा॑णम् च च जनि॒ष्यमा॑णम् मे मे जनि॒ष्यमा॑णम् च । \newline
20. ज॒नि॒ष्यमा॑णम् च च जनि॒ष्यमा॑णम् जनि॒ष्यमा॑णम् च मे मे च जनि॒ष्यमा॑णम् जनि॒ष्यमा॑णम् च मे । \newline
21. च॒ मे॒ मे॒ च॒ च॒ मे॒ सू॒क्तꣳ सू॒क्तम् मे॑ च च मे सू॒क्तम् । \newline
22. मे॒ सू॒क्तꣳ सू॒क्तम् मे॑ मे सू॒क्तम् च॑ च सू॒क्तम् मे॑ मे सू॒क्तम् च॑ । \newline
23. सू॒क्तम् च॑ च सू॒क्तꣳ सू॒क्तम् च॑ मे मे च सू॒क्तꣳ सू॒क्तम् च॑ मे । \newline
24. सू॒क्तमिति॑ सु - उ॒क्तम् । \newline
25. च॒ मे॒ मे॒ च॒ च॒ मे॒ सु॒कृ॒तꣳ सु॑कृ॒तम् मे॑ च च मे सुकृ॒तम् । \newline
26. मे॒ सु॒कृ॒तꣳ सु॑कृ॒तम् मे॑ मे सुकृ॒तम् च॑ च सुकृ॒तम् मे॑ मे सुकृ॒तम् च॑ । \newline
27. सु॒कृ॒तम् च॑ च सुकृ॒तꣳ सु॑कृ॒तम् च॑ मे मे च सुकृ॒तꣳ सु॑कृ॒तम् च॑ मे । \newline
28. सु॒कृ॒तमिति॑ सु - कृ॒तम् । \newline
29. च॒ मे॒ मे॒ च॒ च॒ मे॒ वि॒त्तं ॅवि॒त्तम् मे॑ च च मे वि॒त्तम् । \newline
30. मे॒ वि॒त्तं ॅवि॒त्तम् मे॑ मे वि॒त्तम् च॑ च वि॒त्तम् मे॑ मे वि॒त्तम् च॑ । \newline
31. वि॒त्तम् च॑ च वि॒त्तं ॅवि॒त्तम् च॑ मे मे च वि॒त्तं ॅवि॒त्तम् च॑ मे । \newline
32. च॒ मे॒ मे॒ च॒ च॒ मे॒ वेद्यं॒ ॅवेद्य॑म् मे च च मे॒ वेद्य᳚म् । \newline
33. मे॒ वेद्यं॒ ॅवेद्य॑म् मे मे॒ वेद्य॑म् च च॒ वेद्य॑म् मे मे॒ वेद्य॑म् च । \newline
34. वेद्य॑म् च च॒ वेद्यं॒ ॅवेद्य॑म् च मे मे च॒ वेद्यं॒ ॅवेद्य॑म् च मे । \newline
35. च॒ मे॒ मे॒ च॒ च॒ मे॒ भू॒तम् भू॒तम् मे॑ च च मे भू॒तम् । \newline
36. मे॒ भू॒तम् भू॒तम् मे॑ मे भू॒तम् च॑ च भू॒तम् मे॑ मे भू॒तम् च॑ । \newline
37. भू॒तम् च॑ च भू॒तम् भू॒तम् च॑ मे मे च भू॒तम् भू॒तम् च॑ मे । \newline
38. च॒ मे॒ मे॒ च॒ च॒ मे॒ भ॒वि॒ष्यद् भ॑वि॒ष्यन् मे॑ च च मे भवि॒ष्यत् । \newline
39. मे॒ भ॒वि॒ष्यद् भ॑वि॒ष्यन् मे॑ मे भवि॒ष्यच् च॑ च भवि॒ष्यन् मे॑ मे भवि॒ष्यच् च॑ । \newline
40. भ॒वि॒ष्यच् च॑ च भवि॒ष्यद् भ॑वि॒ष्यच् च॑ मे मे च भवि॒ष्यद् भ॑वि॒ष्यच् च॑ मे । \newline
41. च॒ मे॒ मे॒ च॒ च॒ मे॒ सु॒गꣳ सु॒गम् मे॑ च च मे सु॒गम् । \newline
42. मे॒ सु॒गꣳ सु॒गम् मे॑ मे सु॒गम् च॑ च सु॒गम् मे॑ मे सु॒गम् च॑ । \newline
43. सु॒गम् च॑ च सु॒गꣳ सु॒गम् च॑ मे मे च सु॒गꣳ सु॒गम् च॑ मे । \newline
44. सु॒गमिति॑ सु - गम् । \newline
45. च॒ मे॒ मे॒ च॒ च॒ मे॒ सु॒पथ(ग्म्॑) सु॒पथ॑म् मे च च मे सु॒पथ᳚म् । \newline
46. मे॒ सु॒पथ(ग्म्॑) सु॒पथ॑म् मे मे सु॒पथ॑म् च च सु॒पथ॑म् मे मे सु॒पथ॑म् च । \newline
47. सु॒पथ॑म् च च सु॒पथ(ग्म्॑) सु॒पथ॑म् च मे मे च सु॒पथ(ग्म्॑) सु॒पथ॑म् च मे । \newline
48. सु॒पथ॒मिति॑ सु - पथ᳚म् । \newline
49. च॒ मे॒ मे॒ च॒ च॒ म॒ ऋ॒द्ध मृ॒द्धम् मे॑ च च म ऋ॒द्धम् । \newline
50. म॒ ऋ॒द्ध मृ॒द्धम् मे॑ म ऋ॒द्धम् च॑ च॒ र्‌द्धम् मे॑ म ऋ॒द्धम् च॑ । \newline
51. ऋ॒द्धम् च॑ च॒ र्‌द्ध मृ॒द्धम् च॑ मे मे च॒ र्‌द्ध मृ॒द्धम् च॑ मे । \newline
52. च॒ मे॒ मे॒ च॒ च॒ म॒ ऋद्धि॒र्॒. ऋद्धि॑र् मे च च म॒ ऋद्धिः॑ । \newline
53. म॒ ऋद्धि॒र्॒. ऋद्धि॑र् मे म॒ ऋद्धि॑श्च॒ च र्‌द्धि॑र्मे म॒ ऋद्धि॑श्च । \newline
54. ऋद्धि॑श्च॒ च र्‌द्धि॒र्॒. ऋद्धि॑श्च मे मे॒ च र्‌द्धि॒र्॒. ऋद्धि॑श्च मे । \newline
55. च॒ मे॒ मे॒ च॒ च॒ मे॒ क्लृ॒प्तम् क्लृ॒प्तम् मे॑ च च मे क्लृ॒प्तम् । \newline
56. मे॒ क्लृ॒प्तम् क्लृ॒प्तम् मे॑ मे क्लृ॒प्तम् च॑ च क्लृ॒प्तम् मे॑ मे क्लृ॒प्तम् च॑ । \newline
57. क्लृ॒प्तम् च॑ च क्लृ॒प्तम् क्लृ॒प्तम् च॑ मे मे च क्लृ॒प्तम् क्लृ॒प्तम् च॑ मे । \newline
58. च॒ मे॒ मे॒ च॒ च॒ मे॒ क्लृप्तिः॒ क्लृप्ति॑र् मे च च मे॒ क्लृप्तिः॑ । \newline
59. मे॒ क्लृप्तिः॒ क्लृप्ति॑र् मे मे॒ क्लृप्ति॑श्च च॒ क्लृप्ति॑र् मे मे॒ क्लृप्ति॑श्च । \newline
60. क्लृप्ति॑श्च च॒ क्लृप्तिः॒ क्लृप्ति॑श्च मे मे च॒ क्लृप्तिः॒ क्लृप्ति॑श्च मे । \newline
61. च॒ मे॒ मे॒ च॒ च॒ मे॒ म॒तिर् म॒तिर् मे॑ च च मे म॒तिः । \newline
62. मे॒ म॒तिर् म॒तिर् मे॑ मे म॒तिश्च॑ च म॒तिर् मे॑ मे म॒तिश्च॑ । \newline
63. म॒तिश्च॑ च म॒तिर् म॒तिश्च॑ मे मे च म॒तिर् म॒तिश्च॑ मे । \newline
64. च॒ मे॒ मे॒ च॒ च॒ मे॒ सु॒म॒तिः सु॑म॒तिर् मे॑ च च मे सुम॒तिः । \newline
65. मे॒ सु॒म॒तिः सु॑म॒तिर् मे॑ मे सुम॒तिश्च॑ च सुम॒तिर् मे॑ मे सुम॒तिश्च॑ । \newline
66. सु॒म॒तिश्च॑ च सुम॒तिः सु॑म॒तिश्च॑ मे मे च सुम॒तिः सु॑म॒तिश्च॑ मे । \newline
67. सु॒म॒तिरिति॑ सु - म॒तिः । \newline
68. च॒ मे॒ मे॒ च॒ च॒ मे॒ । \newline
69. म॒ इति॑ मे । \newline
\pagebreak
\markright{ TS 4.7.3.1  \hfill https://www.vedavms.in \hfill}

\section{ TS 4.7.3.1 }

\textbf{TS 4.7.3.1 } \newline
\textbf{Samhita Paata} \newline

शं च॑ मे॒ मय॑श्च मे प्रि॒यं च॑ मे ऽनुका॒मश्च॑ मे॒ काम॑श्च मे     सौमन॒सश्च॑ मे भ॒द्रं च॑ मे॒ श्रेय॑श्च मे॒ वस्य॑श्च मे॒   यश॑श्च मे॒ भग॑श्च मे॒ द्रवि॑णं च मे य॒न्ता च॑ मे ध॒र्ता च॑ मे॒क्षेम॑श्च मे॒ धृति॑श्च मे॒ विश्वं॑ च - [  ] \newline

\textbf{Pada Paata} \newline

शम् । च॒ । मे॒ । मयः॑ । च॒ । मे॒ । प्रि॒यम् । च॒ । मे॒ । अ॒नु॒का॒म इत्य॑नु - का॒मः । च॒ । मे॒ । कामः॑ । च॒ । मे॒ । सौ॒म॒न॒सः । च॒ । मे॒ । भ॒द्रम् । च॒ । मे॒ । श्रेयः॑ । च॒ । मे॒ । वस्यः॑ । च॒ । मे॒ । यशः॑ । च॒ । मे॒ । भगः॑ । च॒ । मे॒ । द्रवि॑णम् । च॒ । मे॒ । य॒न्ता । च॒ । मे॒ । ध॒र्ता । च॒ । मे॒ । क्षेमः॑ । च॒ । मे॒ । धृतिः॑ । च॒ । मे॒ । विश्व᳚म् । च॒ ।  \newline


\textbf{Krama Paata} \newline

शम् च॑ । च॒ मे॒ । मे॒ मयः॑ । मय॑श्च । च॒ मे॒ । मे॒ प्रि॒यम् । प्रि॒यम् च॑ । च॒ मे॒ । मे॒ऽनु॒का॒मः । अ॒नु॒का॒मश्च॑ । अ॒नु॒का॒म इत्य॑नु - का॒मः । च॒ मे॒ । मे॒ कामः॑ । काम॑श्च । च॒ मे॒ । मे॒ सौ॒म॒न॒सः । सौ॒म॒न॒सश्च॑ । च॒ मे॒ । मे॒ भ॒द्रम् । भ॒द्रम् च॑ । च॒ मे॒ । मे॒ श्रेयः॑ । श्रेय॑श्च । च॒ मे॒ । मे॒ वस्यः॑ । वस्य॑श्च । च॒ मे॒ । मे॒ यशः॑ । यश॑श्च । च॒ मे॒ । मे॒ भगः॑ । भग॑श्च । च॒ मे॒ । मे॒ द्रवि॑णम् । द्रवि॑णम् च । च॒ मे॒ । मे॒ य॒न्ता । य॒न्ता च॑ । च॒ मे॒ । मे॒ ध॒र्त्ता । ध॒र्त्ता च॑ । च॒ मे॒ । मे॒ क्षेमः॑ । क्षेम॑श्च । च॒ मे॒ । मे॒ धृतिः॑ । धृति॑श्च । च॒ मे॒ । मे॒ विश्व᳚म् । विश्व॑म् च । च॒ मे॒ \newline

\textbf{Jatai Paata} \newline

1. शम् च॑ च॒ शꣳ शम् च॑ । \newline
2. च॒ मे॒ मे॒ च॒ च॒ मे॒ । \newline
3. मे॒ मयो॒ मयो॑ मे मे॒ मयः॑ । \newline
4. मय॑श्च च॒ मयो॒ मय॑श्च । \newline
5. च॒ मे॒ मे॒ च॒ च॒ मे॒ । \newline
6. मे॒ प्रि॒यम् प्रि॒यम् मे॑ मे प्रि॒यम् । \newline
7. प्रि॒यम् च॑ च प्रि॒यम् प्रि॒यम् च॑ । \newline
8. च॒ मे॒ मे॒ च॒ च॒ मे॒ । \newline
9. मे॒ ऽनु॒का॒मो॑ ऽनुका॒मो मे॑ मे ऽनुका॒मः । \newline
10. अ॒नु॒का॒मश्च॑ चानुका॒मो॑ ऽनुका॒मश्च॑ । \newline
11. अ॒नु॒का॒म इत्य॑नु - का॒मः । \newline
12. च॒ मे॒ मे॒ च॒ च॒ मे॒ । \newline
13. मे॒ कामः॒ कामो॑ मे मे॒ कामः॑ । \newline
14. काम॑श्च च॒ कामः॒ काम॑श्च । \newline
15. च॒ मे॒ मे॒ च॒ च॒ मे॒ । \newline
16. मे॒ सौ॒म॒न॒सः सौ॑मन॒सो मे॑ मे सौमन॒सः । \newline
17. सौ॒म॒न॒सश्च॑ च सौमन॒सः सौ॑मन॒सश्च॑ । \newline
18. च॒ मे॒ मे॒ च॒ च॒ मे॒ । \newline
19. मे॒ भ॒द्रम् भ॒द्रम् मे॑ मे भ॒द्रम् । \newline
20. भ॒द्रम् च॑ च भ॒द्रम् भ॒द्रम् च॑ । \newline
21. च॒ मे॒ मे॒ च॒ च॒ मे॒ । \newline
22. मे॒ श्रेयः॒ श्रेयो॑ मे मे॒ श्रेयः॑ । \newline
23. श्रेय॑श्च च॒ श्रेयः॒ श्रेय॑श्च । \newline
24. च॒ मे॒ मे॒ च॒ च॒ मे॒ । \newline
25. मे॒ वस्यो॒ वस्यो॑ मे मे॒ वस्यः॑ । \newline
26. वस्य॑श्च च॒ वस्यो॒ वस्य॑श्च । \newline
27. च॒ मे॒ मे॒ च॒ च॒ मे॒ । \newline
28. मे॒ यशो॒ यशो॑ मे मे॒ यशः॑ । \newline
29. यश॑श्च च॒ यशो॒ यश॑श्च । \newline
30. च॒ मे॒ मे॒ च॒ च॒ मे॒ । \newline
31. मे॒ भगो॒ भगो॑ मे मे॒ भगः॑ । \newline
32. भग॑श्च च॒ भगो॒ भग॑श्च । \newline
33. च॒ मे॒ मे॒ च॒ च॒ मे॒ । \newline
34. मे॒ द्रवि॑ण॒म् द्रवि॑णम् मे मे॒ द्रवि॑णम् । \newline
35. द्रवि॑णम् च च॒ द्रवि॑ण॒म् द्रवि॑णम् च । \newline
36. च॒ मे॒ मे॒ च॒ च॒ मे॒ । \newline
37. मे॒ य॒न्ता य॒न्ता मे॑ मे य॒न्ता । \newline
38. य॒न्ता च॑ च य॒न्ता य॒न्ता च॑ । \newline
39. च॒ मे॒ मे॒ च॒ च॒ मे॒ । \newline
40. मे॒ ध॒र्ता ध॒र्ता मे॑ मे ध॒र्ता । \newline
41. ध॒र्ता च॑ च ध॒र्ता ध॒र्ता च॑ । \newline
42. च॒ मे॒ मे॒ च॒ च॒ मे॒ । \newline
43. मे॒ क्षेमः॒ क्षेमो॑ मे मे॒ क्षेमः॑ । \newline
44. क्षेम॑श्च च॒ क्षेमः॒ क्षेम॑श्च । \newline
45. च॒ मे॒ मे॒ च॒ च॒ मे॒ । \newline
46. मे॒ धृति॒र् धृति॑र् मे मे॒ धृतिः॑ । \newline
47. धृति॑श्च च॒ धृति॒र् धृति॑श्च । \newline
48. च॒ मे॒ मे॒ च॒ च॒ मे॒ । \newline
49. मे॒ विश्वं॒ ॅविश्व॑म् मे मे॒ विश्व᳚म् । \newline
50. विश्व॑म् च च॒ विश्वं॒ ॅविश्व॑म् च । \newline
51. च॒ मे॒ मे॒ च॒ च॒ मे॒ । \newline

\textbf{Ghana Paata } \newline

1. शम् च॑ च॒ शꣳ शम् च॑ मे मे च॒ शꣳ शम् च॑ मे । \newline
2. च॒ मे॒ मे॒ च॒ च॒ मे॒ मयो॒ मयो॑ मे च च मे॒ मयः॑ । \newline
3. मे॒ मयो॒ मयो॑ मे मे॒ मय॑श्च च॒ मयो॑ मे मे॒ मय॑श्च । \newline
4. मय॑श्च च॒ मयो॒ मय॑श्च मे मे च॒ मयो॒ मय॑श्च मे । \newline
5. च॒ मे॒ मे॒ च॒ च॒ मे॒ प्रि॒यम् प्रि॒यम् मे॑ च च मे प्रि॒यम् । \newline
6. मे॒ प्रि॒यम् प्रि॒यम् मे॑ मे प्रि॒यम् च॑ च प्रि॒यम् मे॑ मे प्रि॒यम् च॑ । \newline
7. प्रि॒यम् च॑ च प्रि॒यम् प्रि॒यम् च॑ मे मे च प्रि॒यम् प्रि॒यम् च॑ मे । \newline
8. च॒ मे॒ मे॒ च॒ च॒ मे॒ ऽनु॒का॒मो॑ ऽनुका॒मो मे॑ च च मे ऽनुका॒मः । \newline
9. मे॒ ऽनु॒का॒मो॑ ऽनुका॒मो मे॑ मे ऽनुका॒मश्च॑ चानुका॒मो मे॑ मे ऽनुका॒मश्च॑ । \newline
10. अ॒नु॒का॒मश्च॑ चानुका॒मो॑ ऽनुका॒मश्च॑ मे मे चानुका॒मो॑ ऽनुका॒मश्च॑ मे । \newline
11. अ॒नु॒का॒म इत्य॑नु - का॒मः । \newline
12. च॒ मे॒ मे॒ च॒ च॒ मे॒ कामः॒ कामो॑ मे च च मे॒ कामः॑ । \newline
13. मे॒ कामः॒ कामो॑ मे मे॒ काम॑श्च च॒ कामो॑ मे मे॒ काम॑श्च । \newline
14. काम॑श्च च॒ कामः॒ काम॑श्च मे मे च॒ कामः॒ काम॑श्च मे । \newline
15. च॒ मे॒ मे॒ च॒ च॒ मे॒ सौ॒म॒न॒सः सौ॑मन॒सो मे॑ च च मे सौमन॒सः । \newline
16. मे॒ सौ॒म॒न॒सः सौ॑मन॒सो मे॑ मे सौमन॒सश्च॑ च सौमन॒सो मे॑ मे सौमन॒सश्च॑ । \newline
17. सौ॒म॒न॒सश्च॑ च सौमन॒सः सौ॑मन॒सश्च॑ मे मे च सौमन॒सः सौ॑मन॒सश्च॑ मे । \newline
18. च॒ मे॒ मे॒ च॒ च॒ मे॒ भ॒द्रम् भ॒द्रम् मे॑ च च मे भ॒द्रम् । \newline
19. मे॒ भ॒द्रम् भ॒द्रम् मे॑ मे भ॒द्रम् च॑ च भ॒द्रम् मे॑ मे भ॒द्रम् च॑ । \newline
20. भ॒द्रम् च॑ च भ॒द्रम् भ॒द्रम् च॑ मे मे च भ॒द्रम् भ॒द्रम् च॑ मे । \newline
21. च॒ मे॒ मे॒ च॒ च॒ मे॒ श्रेयः॒ श्रेयो॑ मे च च मे॒ श्रेयः॑ । \newline
22. मे॒ श्रेयः॒ श्रेयो॑ मे मे॒ श्रेय॑श्च च॒ श्रेयो॑ मे मे॒ श्रेय॑श्च । \newline
23. श्रेय॑श्च च॒ श्रेयः॒ श्रेय॑श्च मे मे च॒ श्रेयः॒ श्रेय॑श्च मे । \newline
24. च॒ मे॒ मे॒ च॒ च॒ मे॒ वस्यो॒ वस्यो॑ मे च च मे॒ वस्यः॑ । \newline
25. मे॒ वस्यो॒ वस्यो॑ मे मे॒ वस्य॑श्च च॒ वस्यो॑ मे मे॒ वस्य॑श्च । \newline
26. वस्य॑श्च च॒ वस्यो॒ वस्य॑श्च मे मे च॒ वस्यो॒ वस्य॑श्च मे । \newline
27. च॒ मे॒ मे॒ च॒ च॒ मे॒ यशो॒ यशो॑ मे च च मे॒ यशः॑ । \newline
28. मे॒ यशो॒ यशो॑ मे मे॒ यश॑श्च च॒ यशो॑ मे मे॒ यश॑श्च । \newline
29. यश॑श्च च॒ यशो॒ यश॑श्च मे मे च॒ यशो॒ यश॑श्च मे । \newline
30. च॒ मे॒ मे॒ च॒ च॒ मे॒ भगो॒ भगो॑ मे च च मे॒ भगः॑ । \newline
31. मे॒ भगो॒ भगो॑ मे मे॒ भग॑श्च च॒ भगो॑ मे मे॒ भग॑श्च । \newline
32. भग॑श्च च॒ भगो॒ भग॑श्च मे मे च॒ भगो॒ भग॑श्च मे । \newline
33. च॒ मे॒ मे॒ च॒ च॒ मे॒ द्रवि॑ण॒म् द्रवि॑णम् मे च च मे॒ द्रवि॑णम् । \newline
34. मे॒ द्रवि॑ण॒म् द्रवि॑णम् मे मे॒ द्रवि॑णम् च च॒ द्रवि॑णम् मे मे॒ द्रवि॑णम् च । \newline
35. द्रवि॑णम् च च॒ द्रवि॑ण॒म् द्रवि॑णम् च मे मे च॒ द्रवि॑ण॒म् द्रवि॑णम् च मे । \newline
36. च॒ मे॒ मे॒ च॒ च॒ मे॒ य॒न्ता य॒न्ता मे॑ च च मे य॒न्ता । \newline
37. मे॒ य॒न्ता य॒न्ता मे॑ मे य॒न्ता च॑ च य॒न्ता मे॑ मे य॒न्ता च॑ । \newline
38. य॒न्ता च॑ च य॒न्ता य॒न्ता च॑ मे मे च य॒न्ता य॒न्ता च॑ मे । \newline
39. च॒ मे॒ मे॒ च॒ च॒ मे॒ ध॒र्ता ध॒र्ता मे॑ च च मे ध॒र्ता । \newline
40. मे॒ ध॒र्ता ध॒र्ता मे॑ मे ध॒र्ता च॑ च ध॒र्ता मे॑ मे ध॒र्ता च॑ । \newline
41. ध॒र्ता च॑ च ध॒र्ता ध॒र्ता च॑ मे मे च ध॒र्ता ध॒र्ता च॑ मे । \newline
42. च॒ मे॒ मे॒ च॒ च॒ मे॒ क्षेमः॒ क्षेमो॑ मे च च मे॒ क्षेमः॑ । \newline
43. मे॒ क्षेमः॒ क्षेमो॑ मे मे॒ क्षेम॑श्च च॒ क्षेमो॑ मे मे॒ क्षेम॑श्च । \newline
44. क्षेम॑श्च च॒ क्षेमः॒ क्षेम॑श्च मे मे च॒ क्षेमः॒ क्षेम॑श्च मे । \newline
45. च॒ मे॒ मे॒ च॒ च॒ मे॒ धृति॒र् धृति॑र् मे च च मे॒ धृतिः॑ । \newline
46. मे॒ धृति॒र् धृति॑र् मे मे॒ धृति॑श्च च॒ धृति॑र् मे मे॒ धृति॑श्च । \newline
47. धृति॑श्च च॒ धृति॒र् धृति॑श्च मे मे च॒ धृति॒र् धृति॑श्च मे । \newline
48. च॒ मे॒ मे॒ च॒ च॒ मे॒ विश्वं॒ ॅविश्व॑म् मे च च मे॒ विश्व᳚म् । \newline
49. मे॒ विश्वं॒ ॅविश्व॑म् मे मे॒ विश्व॑म् च च॒ विश्व॑म् मे मे॒ विश्व॑म् च । \newline
50. विश्व॑म् च च॒ विश्वं॒ ॅविश्व॑म् च मे मे च॒ विश्वं॒ ॅविश्व॑म् च मे । \newline
51. च॒ मे॒ मे॒ च॒ च॒ मे॒ महो॒ महो॑ मे च च मे॒ महः॑ । \newline
\pagebreak
\markright{ TS 4.7.3.2  \hfill https://www.vedavms.in \hfill}

\section{ TS 4.7.3.2 }

\textbf{TS 4.7.3.2 } \newline
\textbf{Samhita Paata} \newline

मे॒ मह॑श्च मे सं॒ॅविच्च॑ मे॒ ज्ञात्रं॑ च मे॒ सूश्च॑ मे प्र॒सूश्च॑ मे॒ सीरं॑ च मे ल॒यश्च॑ म ऋ॒तं च॑ मे॒ ऽमृतं॑ च मेऽय॒क्ष्मं च॒ मे ऽना॑मयच्च मे जी॒वातु॑श्च मे दीर्घायु॒त्वं च॑ मे ऽनमि॒त्रं च॒ मे ऽभ॑यं च मे सु॒गं च॑ मे॒ शय॑नं ( ) च मे सू॒षा च॑ मे सु॒दिनं॑ च मे ॥ \newline

\textbf{Pada Paata} \newline

मे॒ । महः॑ । च॒ । मे॒ । सं॒ॅविदिति॑ सम्-वित् । च॒ । मे॒ । ज्ञात्र᳚म् । च॒ । मे॒ । सूः । च॒ । मे॒ । प्र॒सूरिति॑ प्र - सूः । च॒ । मे॒ । सीरं᳚ । च॒ । मे॒ । ल॒यः । च॒ । मे॒ । ऋ॒तम् । च॒ । मे॒ । अ॒मृत᳚म् । च॒ । मे॒ । अ॒य॒क्ष्मम् । च॒ । मे॒ । अना॑मयत् । च॒ । मे॒ । जी॒वातुः॑ । च॒ । मे॒ । दी॒र्घा॒यु॒त्वमिति॑ दीर्घायु - त्वम् । च॒ । मे॒ । अ॒न॒मि॒त्रम् । च॒ । मे॒ । अभ॑यम् । च॒ । मे॒ । सु॒गमिति॑ सु-गम् । च॒ । मे॒ । शय॑नम् ( ) । च॒ । मे॒ । सू॒षेति॑ सु - उ॒षा । च॒ । मे॒ । सु॒दिन॒मिति॑ सु - दिन᳚म् । च॒ । मे॒ ॥  \newline


\textbf{Krama Paata} \newline

मे॒ महः॑ । मह॑श्च । च॒ मे॒ । मे॒ स॒म्ॅवित् । स॒म्ॅविच् च॑ । स॒म्ॅविदिति॑ सम् - वित् । च॒ मे॒ । मे॒ ज्ञात्र᳚म् । ज्ञात्र॑म् च । च॒ मे॒ । मे॒ सूः । सूश्च॑ । च॒ मे॒ । मे॒ प्र॒सूः । प्र॒सूश्च॑ । प्र॒सूरिति॑ प्र - सूः । च॒ मे॒ । मे॒ सीर᳚म् । सीर॑म् च । च॒ मे॒ । मे॒ ल॒यः । ल॒यश्च॑ । च॒ मे॒ । म॒ ऋ॒तम् । ऋ॒तम् च॑ । च॒ मे॒ । मे॒ऽमृत᳚म् । अ॒मृत॑म् च । च॒ मे॒ । मे॒ऽय॒क्ष्मम् । अ॒य॒क्ष्मम् च॑ । च॒ मे॒ । मेऽना॑मयत् । अना॑मयच्च । च॒ मे॒ । मे॒ जी॒वातुः॑ । जी॒वातु॑श्च । च॒ मे॒ । मे॒ दी॒र्घा॒यु॒त्वम् । दी॒र्घा॒यु॒त्वम् च॑ । दी॒र्घा॒यु॒त्वमिति॑ दीर्घायु - त्वम् । च॒ मे॒ । मे॒ऽन॒मि॒त्रम् । अ॒न॒मि॒त्रम् च॑ । च॒ मे॒ । मेऽभ॑यम् । अभ॑यम् च । च॒ मे॒ । मे॒ सु॒गम् । सु॒गम् च॑ । सु॒गमिति॑ सु - गम् । च॒ मे॒ । मे॒ शय॑नम् ( ) । शय॑नम् च । च॒ मे॒ । मे॒ सू॒षा । सू॒षा च॑ । सू॒षेति॑ सु - उ॒षा । च॒ मे॒ । मे॒ सु॒दिन᳚म् । सु॒दिन॑म् च । सु॒दिन॒मिति॑ सु - दिन᳚म् । च॒ मे॒ । म॒ इति॑ मे । \newline

\textbf{Jatai Paata} \newline

1. मे॒ महो॒ महो॑ मे मे॒ महः॑ । \newline
2. मह॑श्च च॒ महो॒ मह॑श्च । \newline
3. च॒ मे॒ मे॒ च॒ च॒ मे॒ । \newline
4. मे॒ स॒म्ॅविथ् स॒म्ॅविन् मे॑ मे स॒म्ॅवित् । \newline
5. स॒म्ॅविच् च॑ च स॒म्ॅविथ् स॒म्ॅविच् च॑ । \newline
6. स॒म्ॅविदिति॑ सम् - वित् । \newline
7. च॒ मे॒ मे॒ च॒ च॒ मे॒ । \newline
8. मे॒ ज्ञात्र॒म् ज्ञात्र॑म् मे मे॒ ज्ञात्र᳚म् । \newline
9. ज्ञात्र॑म् च च॒ ज्ञात्र॒म् ज्ञात्र॑म् च । \newline
10. च॒ मे॒ मे॒ च॒ च॒ मे॒ । \newline
11. मे॒ सूः सूर् मे॑ मे॒ सूः । \newline
12. सूश्च॑ च॒ सूः सूश्च॑ । \newline
13. च॒ मे॒ मे॒ च॒ च॒ मे॒ । \newline
14. मे॒ प्र॒सूः प्र॒सूर् मे॑ मे प्र॒सूः । \newline
15. प्र॒सूश्च॑ च प्र॒सूः प्र॒सूश्च॑ । \newline
16. प्र॒सूरिति॑ प्र - सूः । \newline
17. च॒ मे॒ मे॒ च॒ च॒ मे॒ । \newline
18. मे॒ सीर॒(ग्म्॒) सीर॑म् मे मे॒ सीर᳚म् । \newline
19. सीर॑म् च च॒ सीर॒(ग्म्॒) सीर॑म् च । \newline
20. च॒ मे॒ मे॒ च॒ च॒ मे॒ । \newline
21. मे॒ ल॒यो ल॒यो मे॑ मे ल॒यः । \newline
22. ल॒यश्च॑ च ल॒यो ल॒यश्च॑ । \newline
23. च॒ मे॒ मे॒ च॒ च॒ मे॒ । \newline
24. म॒ ऋ॒त मृ॒तम् मे॑ म ऋ॒तम् । \newline
25. ऋ॒तम् च॑ च॒ र्‌त मृ॒तम् च॑ । \newline
26. च॒ मे॒ मे॒ च॒ च॒ मे॒ । \newline
27. मे॒ ऽमृत॑ म॒मृत॑म् मे मे॒ ऽमृत᳚म् । \newline
28. अ॒मृत॑म् च चा॒मृत॑ म॒मृत॑म् च । \newline
29. च॒ मे॒ मे॒ च॒ च॒ मे॒ । \newline
30. मे॒ ऽय॒क्ष्म म॑य॒क्ष्मम् मे॑ मे ऽय॒क्ष्मम् । \newline
31. अ॒य॒क्ष्मम् च॑ चाय॒क्ष्म म॑य॒क्ष्मम् च॑ । \newline
32. च॒ मे॒ मे॒ च॒ च॒ मे॒ । \newline
33. मे ऽना॑मय॒ दना॑मयन् मे॒ मे ऽना॑मयत् । \newline
34. अना॑मयच् च॒ चाना॑मय॒ दना॑मयच् च । \newline
35. च॒ मे॒ मे॒ च॒ च॒ मे॒ । \newline
36. मे॒ जी॒वातु॑र् जी॒वातु॑र् मे मे जी॒वातुः॑ । \newline
37. जी॒वातु॑श्च च जी॒वातु॑र् जी॒वातु॑श्च । \newline
38. च॒ मे॒ मे॒ च॒ च॒ मे॒ । \newline
39. मे॒ दी॒र्घा॒यु॒त्वम् दी᳚र्घायु॒त्वम् मे॑ मे दीर्घायु॒त्वम् । \newline
40. दी॒र्घा॒यु॒त्वम् च॑ च दीर्घायु॒त्वम् दी᳚र्घायु॒त्वम् च॑ । \newline
41. दी॒र्घा॒यु॒त्वमिति॑ दीर्घायु - त्वम् । \newline
42. च॒ मे॒ मे॒ च॒ च॒ मे॒ । \newline
43. मे॒ ऽन॒मि॒त्र म॑नमि॒त्रम् मे॑ मे ऽनमि॒त्रम् । \newline
44. अ॒न॒मि॒त्रम् च॑ चानमि॒त्र म॑नमि॒त्रम् च॑ । \newline
45. च॒ मे॒ मे॒ च॒ च॒ मे॒ । \newline
46. मे ऽभ॑य॒ मभ॑यम् मे॒ मे ऽभ॑यम् । \newline
47. अभ॑यम् च॒ चाभ॑य॒ मभ॑यम् च । \newline
48. च॒ मे॒ मे॒ च॒ च॒ मे॒ । \newline
49. मे॒ सु॒गꣳ सु॒गम् मे॑ मे सु॒गम् । \newline
50. सु॒गम् च॑ च सु॒गꣳ सु॒गम् च॑ । \newline
51. सु॒गमिति॑ सु - गम् । \newline
52. च॒ मे॒ मे॒ च॒ च॒ मे॒ । \newline
53. मे॒ शय॑न॒(ग्म्॒) शय॑नम् मे मे॒ शय॑नम् । \newline
54. शय॑नम् च च॒ शय॑न॒(ग्म्॒) शय॑नम् च । \newline
55. च॒ मे॒ मे॒ च॒ च॒ मे॒ । \newline
56. मे॒ सू॒षा सू॒षा मे॑ मे सू॒षा । \newline
57. सू॒षा च॑ च सू॒षा सू॒षा च॑ । \newline
58. सू॒षेति॑ सु - उ॒षा । \newline
59. च॒ मे॒ मे॒ च॒ च॒ मे॒ । \newline
60. मे॒ सु॒दिन(ग्म्॑) सु॒दिन॑म् मे मे सु॒दिन᳚म् । \newline
61. सु॒दिन॑म् च च सु॒दिन(ग्म्॑) सु॒दिन॑म् च । \newline
62. सु॒दिन॒मिति॑ सु - दिन᳚म् । \newline
63. च॒ मे॒ मे॒ च॒ च॒ मे॒ । \newline
64. म॒ इति॑ मे । \newline

\textbf{Ghana Paata } \newline

1. मे॒ महो॒ महो॑ मे मे॒ मह॑श्च च॒ महो॑ मे मे॒ मह॑श्च । \newline
2. मह॑श्च च॒ महो॒ मह॑श्च मे मे च॒ महो॒ मह॑श्च मे । \newline
3. च॒ मे॒ मे॒ च॒ च॒ मे॒ स॒म्ॅविथ् स॒म्ॅविन् मे॑ च च मे स॒म्ॅवित् । \newline
4. मे॒ स॒म्ॅविथ् स॒म्ॅविन् मे॑ मे स॒म्ॅविच् च॑ च स॒म्ॅविन् मे॑ मे स॒म्ॅविच् च॑ । \newline
5. स॒म्ॅविच् च॑ च स॒म्ॅविथ् स॒म्ॅविच् च॑ मे मे च स॒म्ॅविथ् स॒म्ॅविच् च॑ मे । \newline
6. स॒म्ॅविदिति॑ सम् - वित् । \newline
7. च॒ मे॒ मे॒ च॒ च॒ मे॒ ज्ञात्र॒म् ज्ञात्र॑म् मे च च मे॒ ज्ञात्र᳚म् । \newline
8. मे॒ ज्ञात्र॒म् ज्ञात्र॑म् मे मे॒ ज्ञात्र॑म् च च॒ ज्ञात्र॑म् मे मे॒ ज्ञात्र॑म् च । \newline
9. ज्ञात्र॑म् च च॒ ज्ञात्र॒म् ज्ञात्र॑म् च मे मे च॒ ज्ञात्र॒म् ज्ञात्र॑म् च मे । \newline
10. च॒ मे॒ मे॒ च॒ च॒ मे॒ सूः सूर् मे॑ च च मे॒ सूः । \newline
11. मे॒ सूः सूर् मे॑ मे॒ सूश्च॑ च॒ सूर् मे॑ मे॒ सूश्च॑ । \newline
12. सूश्च॑ च॒ सूः सूश्च॑ मे मे च॒ सूः सूश्च॑ मे । \newline
13. च॒ मे॒ मे॒ च॒ च॒ मे॒ प्र॒सूः प्र॒सूर् मे॑ च च मे प्र॒सूः । \newline
14. मे॒ प्र॒सूः प्र॒सूर् मे॑ मे प्र॒सूश्च॑ च प्र॒सूर् मे॑ मे प्र॒सूश्च॑ । \newline
15. प्र॒सूश्च॑ च प्र॒सूः प्र॒सूश्च॑ मे मे च प्र॒सूः प्र॒सूश्च॑ मे । \newline
16. प्र॒सूरिति॑ प्र - सूः । \newline
17. च॒ मे॒ मे॒ च॒ च॒ मे॒ सीर॒(ग्म्॒) सीर॑म् मे च च मे॒ सीर᳚म् । \newline
18. मे॒ सीर॒(ग्म्॒) सीर॑म् मे मे॒ सीर॑म् च च॒ सीर॑म् मे मे॒ सीर॑म् च । \newline
19. सीर॑म् च च॒ सीर॒(ग्म्॒) सीर॑म् च मे मे च॒ सीर॒(ग्म्॒) सीर॑म् च मे । \newline
20. च॒ मे॒ मे॒ च॒ च॒ मे॒ ल॒यो ल॒यो मे॑ च च मे ल॒यः । \newline
21. मे॒ ल॒यो ल॒यो मे॑ मे ल॒यश्च॑ च ल॒यो मे॑ मे ल॒यश्च॑ । \newline
22. ल॒यश्च॑ च ल॒यो ल॒यश्च॑ मे मे च ल॒यो ल॒यश्च॑ मे । \newline
23. च॒ मे॒ मे॒ च॒ च॒ म॒ ऋ॒त मृ॒तम् मे॑ च च म ऋ॒तम् । \newline
24. म॒ ऋ॒त मृ॒तम् मे॑ म ऋ॒तम् च॑ च॒ र्‌तम् मे॑ म ऋ॒तम् च॑ । \newline
25. ऋ॒तम् च॑ च॒ र्‌त मृ॒तम् च॑ मे मे च॒ र्‌त मृ॒तम् च॑ मे । \newline
26. च॒ मे॒ मे॒ च॒ च॒ मे॒ ऽमृत॑ म॒मृत॑म् मे च च मे॒ ऽमृत᳚म् । \newline
27. मे॒ ऽमृत॑ म॒मृत॑म् मे मे॒ ऽमृत॑म् च चा॒मृत॑म् मे मे॒ ऽमृत॑म् च । \newline
28. अ॒मृत॑म् च चा॒मृत॑ म॒मृत॑म् च मे मे चा॒मृत॑ म॒मृत॑म् च मे । \newline
29. च॒ मे॒ मे॒ च॒ च॒ मे॒ ऽय॒क्ष्म म॑य॒क्ष्मम् मे॑ च च मे ऽय॒क्ष्मम् । \newline
30. मे॒ ऽय॒क्ष्म म॑य॒क्ष्मम् मे॑ मे ऽय॒क्ष्मम् च॑ चाय॒क्ष्मम् मे॑ मे ऽय॒क्ष्मम् च॑ । \newline
31. अ॒य॒क्ष्मम् च॑ चाय॒क्ष्म म॑य॒क्ष्मम् च॑ मे मे चाय॒क्ष्म म॑य॒क्ष्मम् च॑ मे । \newline
32. च॒ मे॒ मे॒ च॒ च॒ मे ऽना॑मय॒ दना॑मयन् मे च च॒ मे ऽना॑मयत् । \newline
33. मे ऽना॑मय॒ दना॑मयन् मे॒ मे ऽना॑मयच् च॒ चाना॑मयन् मे॒ मे ऽना॑मयच् च । \newline
34. अना॑मयच् च॒ चाना॑मय॒ दना॑मयच् च मे मे॒ चाना॑मय॒ दना॑मयच् च मे । \newline
35. च॒ मे॒ मे॒ च॒ च॒ मे॒ जी॒वातु॑र् जी॒वातु॑र् मे च च मे जी॒वातुः॑ । \newline
36. मे॒ जी॒वातु॑र् जी॒वातु॑र् मे मे जी॒वातु॑श्च च जी॒वातु॑र् मे मे जी॒वातु॑श्च । \newline
37. जी॒वातु॑श्च च जी॒वातु॑र् जी॒वातु॑श्च मे मे च जी॒वातु॑र् जी॒वातु॑श्च मे । \newline
38. च॒ मे॒ मे॒ च॒ च॒ मे॒ दी॒र्घा॒यु॒त्वम् दी᳚र्घायु॒त्वम् मे॑ च च मे दीर्घायु॒त्वम् । \newline
39. मे॒ दी॒र्घा॒यु॒त्वम् दी᳚र्घायु॒त्वम् मे॑ मे दीर्घायु॒त्वम् च॑ च दीर्घायु॒त्वम् मे॑ मे दीर्घायु॒त्वम् च॑ । \newline
40. दी॒र्घा॒यु॒त्वम् च॑ च दीर्घायु॒त्वम् दी᳚र्घायु॒त्वम् च॑ मे मे च दीर्घायु॒त्वम् दी᳚र्घायु॒त्वम् च॑ मे । \newline
41. दी॒र्घा॒यु॒त्वमिति॑ दीर्घायु - त्वम् । \newline
42. च॒ मे॒ मे॒ च॒ च॒ मे॒ ऽन॒मि॒त्र म॑नमि॒त्रम् मे॑ च च मे ऽनमि॒त्रम् । \newline
43. मे॒ ऽन॒मि॒त्र म॑नमि॒त्रम् मे॑ मे ऽनमि॒त्रम् च॑ चानमि॒त्रम् मे॑ मे ऽनमि॒त्रम् च॑ । \newline
44. अ॒न॒मि॒त्रम् च॑ चानमि॒त्र म॑नमि॒त्रम् च॑ मे मे चानमि॒त्र म॑नमि॒त्रम् च॑ मे । \newline
45. च॒ मे॒ मे॒ च॒ च॒ मे ऽभ॑य॒ मभ॑यम् मे च च॒ मे ऽभ॑यम् । \newline
46. मे ऽभ॑य॒ मभ॑यम् मे॒ मे ऽभ॑यम् च॒ चाभ॑यम् मे॒ मे ऽभ॑यम् च । \newline
47. अभ॑यम् च॒ चाभ॑य॒ मभ॑यम् च मे मे॒ चाभ॑य॒ मभ॑यम् च मे । \newline
48. च॒ मे॒ मे॒ च॒ च॒ मे॒ सु॒गꣳ सु॒गम् मे॑ च च मे सु॒गम् । \newline
49. मे॒ सु॒गꣳ सु॒गम् मे॑ मे सु॒गम् च॑ च सु॒गम् मे॑ मे सु॒गम् च॑ । \newline
50. सु॒गम् च॑ च सु॒गꣳ सु॒गम् च॑ मे मे च सु॒गꣳ सु॒गम् च॑ मे । \newline
51. सु॒गमिति॑ सु - गम् । \newline
52. च॒ मे॒ मे॒ च॒ च॒ मे॒ शय॑न॒(ग्म्॒) शय॑नम् मे च च मे॒ शय॑नम् । \newline
53. मे॒ शय॑न॒(ग्म्॒) शय॑नम् मे मे॒ शय॑नम् च च॒ शय॑नम् मे मे॒ शय॑नम् च । \newline
54. शय॑नम् च च॒ शय॑न॒(ग्म्॒) शय॑नम् च मे मे च॒ शय॑न॒(ग्म्॒) शय॑नम् च मे । \newline
55. च॒ मे॒ मे॒ च॒ च॒ मे॒ सू॒षा सू॒षा मे॑ च च मे सू॒षा । \newline
56. मे॒ सू॒षा सू॒षा मे॑ मे सू॒षा च॑ च सू॒षा मे॑ मे सू॒षा च॑ । \newline
57. सू॒षा च॑ च सू॒षा सू॒षा च॑ मे मे च सू॒षा सू॒षा च॑ मे । \newline
58. सू॒षेति॑ सु - उ॒षा । \newline
59. च॒ मे॒ मे॒ च॒ च॒ मे॒ सु॒दिन(ग्म्॑) सु॒दिन॑म् मे च च मे सु॒दिन᳚म् । \newline
60. मे॒ सु॒दिन(ग्म्॑) सु॒दिन॑म् मे मे सु॒दिन॑म् च च सु॒दिन॑म् मे मे सु॒दिन॑म् च । \newline
61. सु॒दिन॑म् च च सु॒दिन(ग्म्॑) सु॒दिन॑म् च मे मे च सु॒दिन(ग्म्॑) सु॒दिन॑म् च मे । \newline
62. सु॒दिन॒मिति॑ सु - दिन᳚म् । \newline
63. च॒ मे॒ मे॒ च॒ च॒ मे॒ । \newline
64. म॒ इति॑ मे । \newline
\pagebreak
\markright{ TS 4.7.4.1  \hfill https://www.vedavms.in \hfill}

\section{ TS 4.7.4.1 }

\textbf{TS 4.7.4.1 } \newline
\textbf{Samhita Paata} \newline

ऊर्क्च॑ मे सू॒नृता॑ च मे॒ पय॑श्च मे॒ रस॑श्च मे घृ॒तं च॑ मे॒ मधु॑ च मे॒ सग्धि॑श्च मे॒ सपी॑तिश्च मे कृ॒षिश्च॑ मे॒ वृष्टि॑श्च मे॒ जैत्रं॑ च म॒ औद्भि॑द्यं च मे र॒यिश्च॑ मे॒ राय॑श्च मे पु॒ष्टं च॑ मे॒ पुष्टि॑श्च मे वि॒भु च॑ - [  ] \newline

\textbf{Pada Paata} \newline

ऊर्क् । च॒ । मे॒ । सू॒नृता᳚ । च॒ । मे॒ । पयः॑ । च॒ । मे॒ । रसः॑ । च॒ । मे॒ । घृ॒तम् । च॒ । मे॒ । मधु॑ । च॒ । मे॒ । सग्धिः॑ । च॒ । मे॒ । सपी॑ति॒रिति॒ स - पी॒तिः॒ । च॒ । मे॒ । कृ॒षिः । च॒ । मे॒ । वृष्टिः॑ । च॒ । मे॒ । जैत्र᳚म् । च॒ । मे॒ । औद्भि॑द्य॒मित्यौत्-भि॒द्य॒म् । च॒ । मे॒ । र॒यिः । च॒ । मे॒ । रायः॑ । च॒ । मे॒ । पु॒ष्टम् । च॒ । मे॒ । पुष्टिः॑ । च॒ । मे॒ । वि॒भ्विति॑ वि - भु । च॒ ।  \newline


\textbf{Krama Paata} \newline

ऊर्क् च॑ । च॒ मे॒ । मे॒ सू॒नृता᳚ । सू॒नृता॑ च । च॒ मे॒ । मे॒ पयः॑ । पय॑श्च । च॒ मे॒ । मे॒ रसः॑ । रस॑श्च । च॒ मे॒ । मे॒ घृ॒तम् । घृ॒तम् च॑ । च॒ मे॒ । मे॒ मधु॑ । मधु॑ च । च॒ मे॒ । मे॒ सग्धिः॑ । सग्धि॑श्च । च॒ मे॒ । मे॒ सपी॑तिः । सपी॑तिश्च । सपी॑ति॒रिति॒ स - पी॒तिः॒ । च॒ मे॒ । मे॒ कृ॒षिः । कृ॒षिश्च॑ । च॒ मे॒ । मे॒ वृष्टिः॑ । वृष्टि॑श्च । च॒ मे॒ । मे॒ जैत्र᳚म् । जैत्र॑म् च । च॒ मे॒ । म॒ औद्भि॑द्यम् । औद्भि॑द्यम् च । औद्भि॑द्य॒मित्यौत् - भि॒द्य॒म् । च॒ मे॒ । मे॒ र॒यिः । र॒यिश्च॑ । च॒ मे॒ । मे॒ रायः॑ । राय॑श्च । च॒ मे॒ । मे॒ पु॒ष्टम् । पु॒ष्टम् च॑ । च॒ मे॒ । मे॒ पुष्टिः॑ । पुष्टि॑श्च । च॒ मे॒ । मे॒ वि॒भु । वि॒भु च॑ । वि॒भ्विति॑ वि - भु । च॒ मे॒ \newline

\textbf{Jatai Paata} \newline

1. ऊर्क् च॒ चोर्गूर्क् च॑ । \newline
2. च॒ मे॒ मे॒ च॒ च॒ मे॒ । \newline
3. मे॒ सू॒नृता॑ सू॒नृता॑ मे मे सू॒नृता᳚ । \newline
4. सू॒नृता॑ च च सू॒नृता॑ सू॒नृता॑ च । \newline
5. च॒ मे॒ मे॒ च॒ च॒ मे॒ । \newline
6. मे॒ पयः॒ पयो॑ मे मे॒ पयः॑ । \newline
7. पय॑श्च च॒ पयः॒ पय॑श्च । \newline
8. च॒ मे॒ मे॒ च॒ च॒ मे॒ । \newline
9. मे॒ रसो॒ रसो॑ मे मे॒ रसः॑ । \newline
10. रस॑श्च च॒ रसो॒ रस॑श्च । \newline
11. च॒ मे॒ मे॒ च॒ च॒ मे॒ । \newline
12. मे॒ घृ॒तम् घृ॒तम् मे॑ मे घृ॒तम् । \newline
13. घृ॒तम् च॑ च घृ॒तम् घृ॒तम् च॑ । \newline
14. च॒ मे॒ मे॒ च॒ च॒ मे॒ । \newline
15. मे॒ मधु॒ मधु॑ मे मे॒ मधु॑ । \newline
16. मधु॑ च च॒ मधु॒ मधु॑ च । \newline
17. च॒ मे॒ मे॒ च॒ च॒ मे॒ । \newline
18. मे॒ सग्धिः॒ सग्धि॑र् मे मे॒ सग्धिः॑ । \newline
19. सग्धि॑श्च च॒ सग्धिः॒ सग्धि॑श्च । \newline
20. च॒ मे॒ मे॒ च॒ च॒ मे॒ । \newline
21. मे॒ सपी॑तिः॒ सपी॑तिर् मे मे॒ सपी॑तिः । \newline
22. सपी॑तिश्च च॒ सपी॑तिः॒ सपी॑तिश्च । \newline
23. सपी॑ति॒रिति॒ स - पी॒तिः॒ । \newline
24. च॒ मे॒ मे॒ च॒ च॒ मे॒ । \newline
25. मे॒ कृ॒षिः कृ॒षिर् मे॑ मे कृ॒षिः । \newline
26. कृ॒षिश्च॑ च कृ॒षिः कृ॒षिश्च॑ । \newline
27. च॒ मे॒ मे॒ च॒ च॒ मे॒ । \newline
28. मे॒ वृष्टि॒र् वृष्टि॑र् मे मे॒ वृष्टिः॑ । \newline
29. वृष्टि॑श्च च॒ वृष्टि॒र् वृष्टि॑श्च । \newline
30. च॒ मे॒ मे॒ च॒ च॒ मे॒ । \newline
31. मे॒ जैत्र॒म् जैत्र॑म् मे मे॒ जैत्र᳚म् । \newline
32. जैत्र॑म् च च॒ जैत्र॒म् जैत्र॑म् च । \newline
33. च॒ मे॒ मे॒ च॒ च॒ मे॒ । \newline
34. म॒ औद्भि॑द्य॒ मौद्भि॑द्यम् मे म॒ औद्भि॑द्यम् । \newline
35. औद्भि॑द्यम् च॒ चौद्भि॑द्य॒ मौद्भि॑द्यम् च । \newline
36. औद्भि॑द्य॒मित्यौत् - भि॒द्य॒म् । \newline
37. च॒ मे॒ मे॒ च॒ च॒ मे॒ । \newline
38. मे॒ र॒यी र॒यिर् मे॑ मे र॒यिः । \newline
39. र॒यिश्च॑ च र॒यी र॒यिश्च॑ । \newline
40. च॒ मे॒ मे॒ च॒ च॒ मे॒ । \newline
41. मे॒ रायो॒ रायो॑ मे मे॒ रायः॑ । \newline
42. राय॑श्च च॒ रायो॒ राय॑श्च । \newline
43. च॒ मे॒ मे॒ च॒ च॒ मे॒ । \newline
44. मे॒ पु॒ष्टम् पु॒ष्टम् मे॑ मे पु॒ष्टम् । \newline
45. पु॒ष्टम् च॑ च पु॒ष्टम् पु॒ष्टम् च॑ । \newline
46. च॒ मे॒ मे॒ च॒ च॒ मे॒ । \newline
47. मे॒ पुष्टिः॒ पुष्टि॑र् मे मे॒ पुष्टिः॑ । \newline
48. पुष्टि॑श्च च॒ पुष्टिः॒ पुष्टि॑श्च । \newline
49. च॒ मे॒ मे॒ च॒ च॒ मे॒ । \newline
50. मे॒ वि॒भु वि॒भु मे॑ मे वि॒भु । \newline
51. वि॒भु च॑ च वि॒भु वि॒भु च॑ । \newline
52. वि॒भ्विति॑ वि - भु । \newline
53. च॒ मे॒ मे॒ च॒ च॒ मे॒ । \newline

\textbf{Ghana Paata } \newline

1. ऊर्क् च॒ चोर् गूर्क् च॑ मे मे॒ चोर् गूर्क् च॑ मे । \newline
2. च॒ मे॒ मे॒ च॒ च॒ मे॒ सू॒नृता॑ सू॒नृता॑ मे च च मे सू॒नृता᳚ । \newline
3. मे॒ सू॒नृता॑ सू॒नृता॑ मे मे सू॒नृता॑ च च सू॒नृता॑ मे मे सू॒नृता॑ च । \newline
4. सू॒नृता॑ च च सू॒नृता॑ सू॒नृता॑ च मे मे च सू॒नृता॑ सू॒नृता॑ च मे । \newline
5. च॒ मे॒ मे॒ च॒ च॒ मे॒ पयः॒ पयो॑ मे च च मे॒ पयः॑ । \newline
6. मे॒ पयः॒ पयो॑ मे मे॒ पय॑श्च च॒ पयो॑ मे मे॒ पय॑श्च । \newline
7. पय॑श्च च॒ पयः॒ पय॑श्च मे मे च॒ पयः॒ पय॑श्च मे । \newline
8. च॒ मे॒ मे॒ च॒ च॒ मे॒ रसो॒ रसो॑ मे च च मे॒ रसः॑ । \newline
9. मे॒ रसो॒ रसो॑ मे मे॒ रस॑श्च च॒ रसो॑ मे मे॒ रस॑श्च । \newline
10. रस॑श्च च॒ रसो॒ रस॑श्च मे मे च॒ रसो॒ रस॑श्च मे । \newline
11. च॒ मे॒ मे॒ च॒ च॒ मे॒ घृ॒तम् घृ॒तम् मे॑ च च मे घृ॒तम् । \newline
12. मे॒ घृ॒तम् घृ॒तम् मे॑ मे घृ॒तम् च॑ च घृ॒तम् मे॑ मे घृ॒तम् च॑ । \newline
13. घृ॒तम् च॑ च घृ॒तम् घृ॒तम् च॑ मे मे च घृ॒तम् घृ॒तम् च॑ मे । \newline
14. च॒ मे॒ मे॒ च॒ च॒ मे॒ मधु॒ मधु॑ मे च च मे॒ मधु॑ । \newline
15. मे॒ मधु॒ मधु॑ मे मे॒ मधु॑ च च॒ मधु॑ मे मे॒ मधु॑ च । \newline
16. मधु॑ च च॒ मधु॒ मधु॑ च मे मे च॒ मधु॒ मधु॑ च मे । \newline
17. च॒ मे॒ मे॒ च॒ च॒ मे॒ सग्धिः॒ सग्धि॑र् मे च च मे॒ सग्धिः॑ । \newline
18. मे॒ सग्धिः॒ सग्धि॑र् मे मे॒ सग्धि॑श्च च॒ सग्धि॑र् मे मे॒ सग्धि॑श्च । \newline
19. सग्धि॑श्च च॒ सग्धिः॒ सग्धि॑श्च मे मे च॒ सग्धिः॒ सग्धि॑श्च मे । \newline
20. च॒ मे॒ मे॒ च॒ च॒ मे॒ सपी॑तिः॒ सपी॑तिर् मे च च मे॒ सपी॑तिः । \newline
21. मे॒ सपी॑तिः॒ सपी॑तिर् मे मे॒ सपी॑तिश्च च॒ सपी॑तिर् मे मे॒ सपी॑तिश्च । \newline
22. सपी॑तिश्च च॒ सपी॑तिः॒ सपी॑तिश्च मे मे च॒ सपी॑तिः॒ सपी॑तिश्च मे । \newline
23. सपी॑ति॒रिति॒ स - पी॒तिः॒ । \newline
24. च॒ मे॒ मे॒ च॒ च॒ मे॒ कृ॒षिः कृ॒षिर् मे॑ च च मे कृ॒षिः । \newline
25. मे॒ कृ॒षिः कृ॒षिर् मे॑ मे कृ॒षिश्च॑ च कृ॒षिर् मे॑ मे कृ॒षिश्च॑ । \newline
26. कृ॒षिश्च॑ च कृ॒षिः कृ॒षिश्च॑ मे मे च कृ॒षिः कृ॒षिश्च॑ मे । \newline
27. च॒ मे॒ मे॒ च॒ च॒ मे॒ वृष्टि॒र् वृष्टि॑र् मे च च मे॒ वृष्टिः॑ । \newline
28. मे॒ वृष्टि॒र् वृष्टि॑र् मे मे॒ वृष्टि॑श्च च॒ वृष्टि॑र् मे मे॒ वृष्टि॑श्च । \newline
29. वृष्टि॑श्च च॒ वृष्टि॒र् वृष्टि॑श्च मे मे च॒ वृष्टि॒र् वृष्टि॑श्च मे । \newline
30. च॒ मे॒ मे॒ च॒ च॒ मे॒ जैत्र॒म् जैत्र॑म् मे च च मे॒ जैत्र᳚म् । \newline
31. मे॒ जैत्र॒म् जैत्र॑म् मे मे॒ जैत्र॑म् च च॒ जैत्र॑म् मे मे॒ जैत्र॑म् च । \newline
32. जैत्र॑म् च च॒ जैत्र॒म् जैत्र॑म् च मे मे च॒ जैत्र॒म् जैत्र॑म् च मे । \newline
33. च॒ मे॒ मे॒ च॒ च॒ म॒ औद्भि॑द्य॒ मौद्भि॑द्यम् मे च च म॒ औद्भि॑द्यम् । \newline
34. म॒ औद्भि॑द्य॒ मौद्भि॑द्यम् मे म॒ औद्भि॑द्यम् च॒ चौद्भि॑द्यम् मे म॒ औद्भि॑द्यम् च । \newline
35. औद्भि॑द्यम् च॒ चौद्भि॑द्य॒ मौद्भि॑द्यम् च मे मे॒ चौद्भि॑द्य॒ मौद्भि॑द्यम् च मे । \newline
36. औद्भि॑द्य॒मित्यौत् - भि॒द्य॒म् । \newline
37. च॒ मे॒ मे॒ च॒ च॒ मे॒ र॒यी र॒यिर् मे॑ च च मे र॒यिः । \newline
38. मे॒ र॒यी र॒यिर् मे॑ मे र॒यिश्च॑ च र॒यिर् मे॑ मे र॒यिश्च॑ । \newline
39. र॒यिश्च॑ च र॒यी र॒यिश्च॑ मे मे च र॒यी र॒यिश्च॑ मे । \newline
40. च॒ मे॒ मे॒ च॒ च॒ मे॒ रायो॒ रायो॑ मे च च मे॒ रायः॑ । \newline
41. मे॒ रायो॒ रायो॑ मे मे॒ राय॑श्च च॒ रायो॑ मे मे॒ राय॑श्च । \newline
42. राय॑श्च च॒ रायो॒ राय॑श्च मे मे च॒ रायो॒ राय॑श्च मे । \newline
43. च॒ मे॒ मे॒ च॒ च॒ मे॒ पु॒ष्टम् पु॒ष्टम् मे॑ च च मे पु॒ष्टम् । \newline
44. मे॒ पु॒ष्टम् पु॒ष्टम् मे॑ मे पु॒ष्टम् च॑ च पु॒ष्टम् मे॑ मे पु॒ष्टम् च॑ । \newline
45. पु॒ष्टम् च॑ च पु॒ष्टम् पु॒ष्टम् च॑ मे मे च पु॒ष्टम् पु॒ष्टम् च॑ मे । \newline
46. च॒ मे॒ मे॒ च॒ च॒ मे॒ पुष्टिः॒ पुष्टि॑र् मे च च मे॒ पुष्टिः॑ । \newline
47. मे॒ पुष्टिः॒ पुष्टि॑र् मे मे॒ पुष्टि॑श्च च॒ पुष्टि॑र् मे मे॒ पुष्टि॑श्च । \newline
48. पुष्टि॑श्च च॒ पुष्टिः॒ पुष्टि॑श्च मे मे च॒ पुष्टिः॒ पुष्टि॑श्च मे । \newline
49. च॒ मे॒ मे॒ च॒ च॒ मे॒ वि॒भु वि॒भु मे॑ च च मे वि॒भु । \newline
50. मे॒ वि॒भु वि॒भु मे॑ मे वि॒भु च॑ च वि॒भु मे॑ मे वि॒भु च॑ । \newline
51. वि॒भु च॑ च वि॒भु वि॒भु च॑ मे मे च वि॒भु वि॒भु च॑ मे । \newline
52. वि॒भ्विति॑ वि - भु । \newline
53. च॒ मे॒ मे॒ च॒ च॒ मे॒ प्र॒भु प्र॒भु मे॑ च च मे प्र॒भु । \newline
\pagebreak
\markright{ TS 4.7.4.2  \hfill https://www.vedavms.in \hfill}

\section{ TS 4.7.4.2 }

\textbf{TS 4.7.4.2 } \newline
\textbf{Samhita Paata} \newline

मे प्र॒भु च॑ मे ब॒हु च॑ मे॒ भूय॑श्च मे पू॒र्णं च॑ मे पू॒र्णत॑रं च॒ मे   ऽक्षि॑तिश्च मे॒ कूय॑वाश्च॒ मेऽन्नं॑ च॒ मे ऽक्षु॑च्च मे व्री॒हय॑श्च मे॒       यवा᳚श्च मे॒ माषा᳚श्च मे॒ तिला᳚श्च मे मु॒द्गाश्च॑ मे ख॒ल्वा᳚श्च मे गो॒धूमा᳚श्च मे म॒सुरा᳚- ( ) -श्च मे प्रि॒यंग॑वश्च॒ मे ऽण॑वश्च मेश्या॒माका᳚श्च मे नी॒वारा᳚श्च मे ॥ \newline

\textbf{Pada Paata} \newline

मे॒ । प्र॒भ्विति॑ प्र - भु । च॒ । मे॒ । ब॒हु । च॒ । मे॒ । भूयः॑ । च॒ । मे॒ । पू॒र्णम् । च॒ । मे॒ । पू॒र्णत॑र॒मिति॑ पू॒र्ण - त॒र॒म् । च॒ । मे॒ । अक्षि॑तिः । च॒ । मे॒ । कूय॑वाः । च॒ । मे॒ । अन्न᳚म् । च॒ । मे॒ । अक्षु॑त् । च॒ । मे॒ । व्री॒हयः॑ । च॒ । मे॒ । यवाः᳚ । च॒ । मे॒ । माषाः᳚ । च॒ । मे॒ । तिलाः᳚ । च॒ । मे॒ । मु॒द्गाः । च॒ । मे॒ । ख॒ल्वाः᳚ । च॒ । मे॒ । गो॒धूमाः᳚ । च॒ । मे॒ । म॒सुराः᳚ ( ) । च॒ । मे॒ । प्रि॒यंग॑वः । च॒ । मे॒ । अण॑वः । च॒ । मे॒ । श्या॒माकाः᳚ । च॒ । मे॒ । नी॒वाराः᳚ । च॒ । मे॒ ॥  \newline


\textbf{Krama Paata} \newline

मे॒ प्र॒भु । प्र॒भु च॑ । प्र॒भ्विति॑ प्र - भु । च॒ मे॒ । मे॒ ब॒हु । ब॒हु च॑ । च॒ मे॒ । मे॒ भूयः॑ । भूय॑श्च । च॒ मे॒ । मे॒ पू॒र्णम् । पू॒र्णम् च॑ । च॒ मे॒ । मे॒ पू॒र्णत॑रम् । पू॒र्णत॑रम् च । पू॒र्णत॑र॒मिति॑ पू॒र्ण - त॒र॒म् । च॒ मे॒ । मेऽक्षि॑तिः । अक्षि॑तिश्च । च॒ मे॒ । मे॒ कूय॑वाः । कूय॑वाश्च । च॒ मे॒ । मेऽन्न᳚म् । अन्न॑म् च । च॒ मे॒ । मेऽक्षु॑त् । अक्षु॑च् च । च॒ मे॒ । मे॒ व्री॒हयः॑ । व्री॒हय॑श्च । च॒ मे॒ । मे॒ यवाः᳚ । यवा᳚श्च । च॒ मे॒ । मे॒ माषाः᳚ । माषा᳚श्च । च॒ मे॒ । मे॒ तिलाः᳚ । तिला᳚श्च । च॒ मे॒ । मे॒ मु॒द्गाः । मु॒द्गाश्च॑ । च॒ मे॒ । मे॒ ख॒ल्वाः᳚ । ख॒ल्वा᳚श्च । च॒ मे॒ । मे॒ गो॒धूमाः᳚ । गो॒धूमा᳚श्च । च॒ मे॒ । मे॒ म॒सुराः᳚ ( ) । म॒सुरा᳚श्च । च॒ मे॒ । मे॒ प्रि॒यङ्ग॑वः । प्रि॒यङ्ग॑वश्च । च॒ मे॒ । मेऽण॑वः । अण॑वश्च । च॒ मे॒ । मे॒ श्या॒माकाः᳚ । श्या॒माका᳚श्च । च॒ मे॒ । मे॒ नी॒वाराः᳚ । नी॒वारा᳚श्च । च॒ मे॒ । म॒ इति॑ मे । \newline

\textbf{Jatai Paata} \newline

1. मे॒ प्र॒भु प्र॒भु मे॑ मे प्र॒भु । \newline
2. प्र॒भु च॑ च प्र॒भु प्र॒भु च॑ । \newline
3. प्र॒भ्विति॑ प्र - भु । \newline
4. च॒ मे॒ मे॒ च॒ च॒ मे॒ । \newline
5. मे॒ ब॒हु ब॒हु मे॑ मे ब॒हु । \newline
6. ब॒हु च॑ च ब॒हु ब॒हु च॑ । \newline
7. च॒ मे॒ मे॒ च॒ च॒ मे॒ । \newline
8. मे॒ भूयो॒ भूयो॑ मे मे॒ भूयः॑ । \newline
9. भूय॑श्च च॒ भूयो॒ भूय॑श्च । \newline
10. च॒ मे॒ मे॒ च॒ च॒ मे॒ । \newline
11. मे॒ पू॒र्णम् पू॒र्णम् मे॑ मे पू॒र्णम् । \newline
12. पू॒र्णम् च॑ च पू॒र्णम् पू॒र्णम् च॑ । \newline
13. च॒ मे॒ मे॒ च॒ च॒ मे॒ । \newline
14. मे॒ पू॒र्णत॑रम् पू॒र्णत॑रम् मे मे पू॒र्णत॑रम् । \newline
15. पू॒र्णत॑रम् च च पू॒र्णत॑रम् पू॒र्णत॑रम् च । \newline
16. पू॒र्णत॑र॒मिति॑ पू॒र्ण - त॒र॒म् । \newline
17. च॒ मे॒ मे॒ च॒ च॒ मे॒ । \newline
18. मे ऽक्षि॑ति॒ रक्षि॑तिर् मे॒ मे ऽक्षि॑तिः । \newline
19. अक्षि॑तिश्च॒ चाक्षि॑ति॒ रक्षि॑तिश्च । \newline
20. च॒ मे॒ मे॒ च॒ च॒ मे॒ । \newline
21. मे॒ कूय॑वाः॒ कूय॑वा मे मे॒ कूय॑वाः । \newline
22. कूय॑वाश्च च॒ कूय॑वाः॒ कूय॑वाश्च । \newline
23. च॒ मे॒ मे॒ च॒ च॒ मे॒ । \newline
24. मे ऽन्न॒ मन्न॑म् मे॒ मे ऽन्न᳚म् । \newline
25. अन्न॑म् च॒ चान्न॒ मन्न॑म् च । \newline
26. च॒ मे॒ मे॒ च॒ च॒ मे॒ । \newline
27. मे ऽक्षु॒ दक्षु॑न् मे॒ मे ऽक्षु॑त् । \newline
28. अक्षु॑च् च॒ चाक्षु॒ दक्षु॑च् च । \newline
29. च॒ मे॒ मे॒ च॒ च॒ मे॒ । \newline
30. मे॒ व्री॒हयो᳚ व्री॒हयो॑ मे मे व्री॒हयः॑ । \newline
31. व्री॒हय॑श्च च व्री॒हयो᳚ व्री॒हय॑श्च । \newline
32. च॒ मे॒ मे॒ च॒ च॒ मे॒ । \newline
33. मे॒ यवा॒ यवा॑ मे मे॒ यवाः᳚ । \newline
34. यवा᳚श्च च॒ यवा॒ यवा᳚श्च । \newline
35. च॒ मे॒ मे॒ च॒ च॒ मे॒ । \newline
36. मे॒ माषा॒ माषा॑ मे मे॒ माषाः᳚ । \newline
37. माषा᳚श्च च॒ माषा॒ माषा᳚श्च । \newline
38. च॒ मे॒ मे॒ च॒ च॒ मे॒ । \newline
39. मे॒ तिला॒ स्तिला॑ मे मे॒ तिलाः᳚ । \newline
40. तिला᳚श्च च॒ तिला॒ स्तिला᳚श्च । \newline
41. च॒ मे॒ मे॒ च॒ च॒ मे॒ । \newline
42. मे॒ मु॒द्‍गा मु॒द्‍गा मे॑ मे मु॒द्‍गाः । \newline
43. मु॒द्‍गाश्च॑ च मु॒द्‍गा मु॒द्‍गाश्च॑ । \newline
44. च॒ मे॒ मे॒ च॒ च॒ मे॒ । \newline
45. मे॒ ख॒ल्वाः᳚ ख॒ल्वा॑ मे मे ख॒ल्वाः᳚ । \newline
46. ख॒ल्वा᳚श्च च ख॒ल्वाः᳚ ख॒ल्वा᳚श्च । \newline
47. च॒ मे॒ मे॒ च॒ च॒ मे॒ । \newline
48. मे॒ गो॒धूमा॑ गो॒धूमा॑ मे मे गो॒धूमाः᳚ । \newline
49. गो॒धूमा᳚श्च च गो॒धूमा॑ गो॒धूमा᳚श्च । \newline
50. च॒ मे॒ मे॒ च॒ च॒ मे॒ । \newline
51. मे॒ म॒सुरा॑ म॒सुरा॑ मे मे म॒सुराः᳚ । \newline
52. म॒सुरा᳚श्च च म॒सुरा॑ म॒सुरा᳚श्च । \newline
53. च॒ मे॒ मे॒ च॒ च॒ मे॒ । \newline
54. मे॒ प्रि॒यङ्ग॑वः प्रि॒यङ्ग॑वो मे मे प्रि॒यङ्ग॑वः । \newline
55. प्रि॒यङ्ग॑वश्च च प्रि॒यङ्ग॑वः प्रि॒यङ्ग॑वश्च । \newline
56. च॒ मे॒ मे॒ च॒ च॒ मे॒ । \newline
57. मे ऽण॒वो ऽण॑वो मे॒ मे ऽण॑वः । \newline
58. अण॑वश्च॒ चाण॒वो ऽण॑वश्च । \newline
59. च॒ मे॒ मे॒ च॒ च॒ मे॒ । \newline
60. मे॒ श्या॒माकाः᳚ श्या॒माका॑ मे मे श्या॒माकाः᳚ । \newline
61. श्या॒माका᳚श्च च श्या॒माकाः᳚ श्या॒माका᳚श्च । \newline
62. च॒ मे॒ मे॒ च॒ च॒ मे॒ । \newline
63. मे॒ नी॒वारा॑ नी॒वारा॑ मे मे नी॒वाराः᳚ । \newline
64. नी॒वारा᳚श्च च नी॒वारा॑ नी॒वारा᳚श्च । \newline
65. च॒ मे॒ मे॒ च॒ च॒ मे॒ । \newline
66. म॒ इति॑ मे । \newline

\textbf{Ghana Paata } \newline

1. मे॒ प्र॒भु प्र॒भु मे॑ मे प्र॒भु च॑ च प्र॒भु मे॑ मे प्र॒भु च॑ । \newline
2. प्र॒भु च॑ च प्र॒भु प्र॒भु च॑ मे मे च प्र॒भु प्र॒भु च॑ मे । \newline
3. प्र॒भ्विति॑ प्र - भु । \newline
4. च॒ मे॒ मे॒ च॒ च॒ मे॒ ब॒हु ब॒हु मे॑ च च मे ब॒हु । \newline
5. मे॒ ब॒हु ब॒हु मे॑ मे ब॒हु च॑ च ब॒हु मे॑ मे ब॒हु च॑ । \newline
6. ब॒हु च॑ च ब॒हु ब॒हु च॑ मे मे च ब॒हु ब॒हु च॑ मे । \newline
7. च॒ मे॒ मे॒ च॒ च॒ मे॒ भूयो॒ भूयो॑ मे च च मे॒ भूयः॑ । \newline
8. मे॒ भूयो॒ भूयो॑ मे मे॒ भूय॑श्च च॒ भूयो॑ मे मे॒ भूय॑श्च । \newline
9. भूय॑श्च च॒ भूयो॒ भूय॑श्च मे मे च॒ भूयो॒ भूय॑श्च मे । \newline
10. च॒ मे॒ मे॒ च॒ च॒ मे॒ पू॒र्णम् पू॒र्णम् मे॑ च च मे पू॒र्णम् । \newline
11. मे॒ पू॒र्णम् पू॒र्णम् मे॑ मे पू॒र्णम् च॑ च पू॒र्णम् मे॑ मे पू॒र्णम् च॑ । \newline
12. पू॒र्णम् च॑ च पू॒र्णम् पू॒र्णम् च॑ मे मे च पू॒र्णम् पू॒र्णम् च॑ मे । \newline
13. च॒ मे॒ मे॒ च॒ च॒ मे॒ पू॒र्णत॑रम् पू॒र्णत॑रम् मे च च मे पू॒र्णत॑रम् । \newline
14. मे॒ पू॒र्णत॑रम् पू॒र्णत॑रम् मे मे पू॒र्णत॑रम् च च पू॒र्णत॑रम् मे मे पू॒र्णत॑रम् च । \newline
15. पू॒र्णत॑रम् च च पू॒र्णत॑रम् पू॒र्णत॑रम् च मे मे च पू॒र्णत॑रम् पू॒र्णत॑रम् च मे । \newline
16. पू॒र्णत॑र॒मिति॑ पू॒र्ण - त॒र॒म् । \newline
17. च॒ मे॒ मे॒ च॒ च॒ मे ऽक्षि॑ति॒ रक्षि॑तिर् मे च च॒ मे ऽक्षि॑तिः । \newline
18. मे ऽक्षि॑ति॒ रक्षि॑तिर् मे॒ मे ऽक्षि॑तिश्च॒ चाक्षि॑तिर् मे॒ मे ऽक्षि॑तिश्च । \newline
19. अक्षि॑तिश्च॒ चाक्षि॑ति॒ रक्षि॑तिश्च मे मे॒ चाक्षि॑ति॒ रक्षि॑तिश्च मे । \newline
20. च॒ मे॒ मे॒ च॒ च॒ मे॒ कूय॑वाः॒ कूय॑वा मे च च मे॒ कूय॑वाः । \newline
21. मे॒ कूय॑वाः॒ कूय॑वा मे मे॒ कूय॑वाश्च च॒ कूय॑वा मे मे॒ कूय॑वाश्च । \newline
22. कूय॑वाश्च च॒ कूय॑वाः॒ कूय॑वाश्च मे मे च॒ कूय॑वाः॒ कूय॑वाश्च मे । \newline
23. च॒ मे॒ मे॒ च॒ च॒ मे ऽन्न॒ मन्न॑म् मे च च॒ मे ऽन्न᳚म् । \newline
24. मे ऽन्न॒ मन्न॑म् मे॒ मे ऽन्न॑म् च॒ चान्न॑म् मे॒ मे ऽन्न॑म् च । \newline
25. अन्न॑म् च॒ चान्न॒ मन्न॑म् च मे मे॒ चान्न॒ मन्न॑म् च मे । \newline
26. च॒ मे॒ मे॒ च॒ च॒ मे ऽक्षु॒ दक्षु॑न् मे च च॒ मे ऽक्षु॑त् । \newline
27. मे ऽक्षु॒ दक्षु॑न् मे॒ मे ऽक्षु॑च् च॒ चाक्षु॑न् मे॒ मे ऽक्षु॑च् च । \newline
28. अक्षु॑च् च॒ चाक्षु॒ दक्षु॑च् च मे मे॒ चाक्षु॒ दक्षु॑च् च मे । \newline
29. च॒ मे॒ मे॒ च॒ च॒ मे॒ व्री॒हयो᳚ व्री॒हयो॑ मे च च मे व्री॒हयः॑ । \newline
30. मे॒ व्री॒हयो᳚ व्री॒हयो॑ मे मे व्री॒हय॑श्च च व्री॒हयो॑ मे मे व्री॒हय॑श्च । \newline
31. व्री॒हय॑श्च च व्री॒हयो᳚ व्री॒हय॑श्च मे मे च व्री॒हयो᳚ व्री॒हय॑श्च मे । \newline
32. च॒ मे॒ मे॒ च॒ च॒ मे॒ यवा॒ यवा॑ मे च च मे॒ यवाः᳚ । \newline
33. मे॒ यवा॒ यवा॑ मे मे॒ यवा᳚श्च च॒ यवा॑ मे मे॒ यवा᳚श्च । \newline
34. यवा᳚श्च च॒ यवा॒ यवा᳚श्च मे मे च॒ यवा॒ यवा᳚श्च मे । \newline
35. च॒ मे॒ मे॒ च॒ च॒ मे॒ माषा॒ माषा॑ मे च च मे॒ माषाः᳚ । \newline
36. मे॒ माषा॒ माषा॑ मे मे॒ माषा᳚श्च च॒ माषा॑ मे मे॒ माषा᳚श्च । \newline
37. माषा᳚श्च च॒ माषा॒ माषा᳚श्च मे मे च॒ माषा॒ माषा᳚श्च मे । \newline
38. च॒ मे॒ मे॒ च॒ च॒ मे॒ तिला॒ स्तिला॑ मे च च मे॒ तिलाः᳚ । \newline
39. मे॒ तिला॒ स्तिला॑ मे मे॒ तिला᳚श्च च॒ तिला॑ मे मे॒ तिला᳚श्च । \newline
40. तिला᳚श्च च॒ तिला॒ स्तिला᳚श्च मे मे च॒ तिला॒ स्तिला᳚श्च मे । \newline
41. च॒ मे॒ मे॒ च॒ च॒ मे॒ मु॒द्‍गा मु॒द्‍गा मे॑ च च मे मु॒द्‍गाः । \newline
42. मे॒ मु॒द्‍गा मु॒द्‍गा मे॑ मे मु॒द्‍गाश्च॑ च मु॒द्‍गा मे॑ मे मु॒द्‍गाश्च॑ । \newline
43. मु॒द्‍गाश्च॑ च मु॒द्‍गा मु॒द्‍गाश्च॑ मे मे च मु॒द्‍गा मु॒द्‍गाश्च॑ मे । \newline
44. च॒ मे॒ मे॒ च॒ च॒ मे॒ ख॒ल्वाः᳚ ख॒ल्वा॑ मे च च मे ख॒ल्वाः᳚ । \newline
45. मे॒ ख॒ल्वाः᳚ ख॒ल्वा॑ मे मे ख॒ल्वा᳚श्च च ख॒ल्वा॑ मे मे ख॒ल्वा᳚श्च । \newline
46. ख॒ल्वा᳚श्च च ख॒ल्वाः᳚ ख॒ल्वा᳚श्च मे मे च ख॒ल्वाः᳚ ख॒ल्वा᳚श्च मे । \newline
47. च॒ मे॒ मे॒ च॒ च॒ मे॒ गो॒धूमा॑ गो॒धूमा॑ मे च च मे गो॒धूमाः᳚ । \newline
48. मे॒ गो॒धूमा॑ गो॒धूमा॑ मे मे गो॒धूमा᳚श्च च गो॒धूमा॑ मे मे गो॒धूमा᳚श्च । \newline
49. गो॒धूमा᳚श्च च गो॒धूमा॑ गो॒धूमा᳚श्च मे मे च गो॒धूमा॑ गो॒धूमा᳚श्च मे । \newline
50. च॒ मे॒ मे॒ च॒ च॒ मे॒ म॒सुरा॑ म॒सुरा॑ मे च च मे म॒सुराः᳚ । \newline
51. मे॒ म॒सुरा॑ म॒सुरा॑ मे मे म॒सुरा᳚श्च च म॒सुरा॑ मे मे म॒सुरा᳚श्च । \newline
52. म॒सुरा᳚श्च च म॒सुरा॑ म॒सुरा᳚श्च मे मे च म॒सुरा॑ म॒सुरा᳚श्च मे । \newline
53. च॒ मे॒ मे॒ च॒ च॒ मे॒ प्रि॒यङ्ग॑वः प्रि॒यङ्ग॑वो मे च च मे प्रि॒यङ्ग॑वः । \newline
54. मे॒ प्रि॒यङ्ग॑वः प्रि॒यङ्ग॑वो मे मे प्रि॒यङ्ग॑वश्च च प्रि॒यङ्ग॑वो मे मे प्रि॒यङ्ग॑वश्च । \newline
55. प्रि॒यङ्ग॑वश्च च प्रि॒यङ्ग॑वः प्रि॒यङ्ग॑वश्च मे मे च प्रि॒यङ्ग॑वः प्रि॒यङ्ग॑वश्च मे । \newline
56. च॒ मे॒ मे॒ च॒ च॒ मे ऽण॒वो ऽण॑वो मे च च॒ मे ऽण॑वः । \newline
57. मे ऽण॒वो ऽण॑वो मे॒ मे ऽण॑वश्च॒ चाण॑वो मे॒ मे ऽण॑वश्च । \newline
58. अण॑वश्च॒ चाण॒वो ऽण॑वश्च मे मे॒ चाण॒वो ऽण॑वश्च मे । \newline
59. च॒ मे॒ मे॒ च॒ च॒ मे॒ श्या॒माकाः᳚ श्या॒माका॑ मे च च मे श्या॒माकाः᳚ । \newline
60. मे॒ श्या॒माकाः᳚ श्या॒माका॑ मे मे श्या॒माका᳚श्च च श्या॒माका॑ मे मे श्या॒माका᳚श्च । \newline
61. श्या॒माका᳚श्च च श्या॒माकाः᳚ श्या॒माका᳚श्च मे मे च श्या॒माकाः᳚ श्या॒माका᳚श्च मे । \newline
62. च॒ मे॒ मे॒ च॒ च॒ मे॒ नी॒वारा॑ नी॒वारा॑ मे च च मे नी॒वाराः᳚ । \newline
63. मे॒ नी॒वारा॑ नी॒वारा॑ मे मे नी॒वारा᳚श्च च नी॒वारा॑ मे मे नी॒वारा᳚श्च । \newline
64. नी॒वारा᳚श्च च नी॒वारा॑ नी॒वारा᳚श्च मे मे च नी॒वारा॑ नी॒वारा᳚श्च मे । \newline
65. च॒ मे॒ मे॒ च॒ च॒ मे॒ । \newline
66. म॒ इति॑ मे । \newline
\pagebreak
\markright{ TS 4.7.5.1  \hfill https://www.vedavms.in \hfill}

\section{ TS 4.7.5.1 }

\textbf{TS 4.7.5.1 } \newline
\textbf{Samhita Paata} \newline

अश्मा॑ च मे॒ मृत्ति॑का च मे गि॒रय॑श्च मे॒ पर्व॑ताश्च मे॒ सिक॑ताश्च मे॒     वन॒स्पत॑यश्च मे॒ हिर॑ण्यं च॒ मे ऽय॑श्च मे॒ सीसं॑ च मे॒ त्रपु॑श्च मे    श्या॒मं च॑ मे लो॒हं च॑ मे॒ऽग्निश्च॑ म॒ आप॑श्च मेवी॒रुध॑श्च म॒ ओष॑धयश्च मे कृष्टप॒च्यं च॑ - [  ] \newline

\textbf{Pada Paata} \newline

अश्मा᳚ । च॒ । मे॒ । मृत्ति॑का । च॒ । मे॒ । गि॒रयः॑ । च॒ । मे॒ । पर्व॑ताः । च॒ । मे॒ । सिक॑ताः । च॒ । मे॒ । वन॒स्पत॑यः । च॒ । मे॒ । हिर॑ण्यम् । च॒ । मे॒ । अयः॑ । च॒ । मे॒ । सीस᳚म् । च॒ । मे॒ । त्रपु॑ । च॒ । मे॒ । श्या॒मम् । च॒ । मे॒ । लो॒हम् । च॒ । मे॒ । अ॒ग्निः । च॒ । मे॒ । आपः॑ । च॒ । मे॒ । वी॒रुधः॑ । च॒ । मे॒ । ओष॑धयः । च॒ । मे॒ । कृ॒ष्ट॒प॒च्यमिति॑ कृष्ट - प॒च्यम् । च॒ ।  \newline


\textbf{Krama Paata} \newline

अश्मा॑ च । च॒ मे॒ । मे॒ मृत्ति॑का । मृत्ति॑का च । च॒ मे॒ । मे॒॒ गि॒रयः॑ । गि॒रय॑श्च । च॒ मे॒ । मे॒ पर्व॑ताः । पर्व॑ताश्च । च॒ मे॒ । मे॒ सिक॑ताः । सिक॑ताश्च । च॒ मे॒ । मे॒ वन॒स्पत॑यः । वन॒स्पत॑यश्च । च॒ मे॒ । मे॒ हिर॑ण्यम् । हिर॑ण्यम् च । च॒ मे॒ । मेऽयः॑ । अय॑श्च । च॒ मे॒ । मे॒ सीस᳚म् । सीस॑म् च । च॒ मे॒ । मे॒ त्रपु॑ । त्रपु॑श्च । च॒ मे॒ । मे॒ श्या॒मम् । श्या॒मम् च॑ । च॒ मे॒ । मे॒ लो॒हम् । लो॒हम् च॑ । च॒ मे॒ । मे॒ऽग्निः । अ॒ग्निश्च॑ । च॒ मे॒ । म॒ आपः॑ । आप॑श्च । च॒ मे॒ । मे॒ वी॒रुधः॑ । वी॒रुध॑श्च । च॒ मे॒ । म॒ ओष॑धयः । ओष॑धयश्च । च॒ मे॒ । मे॒ कृ॒ष्ट॒प॒च्यम् । कृ॒ष्ट॒प॒च्यम् च॑ ( ) । कृ॒ष्ट॒प॒च्यमिति॑ कृष्ट - प॒च्यम् । च॒ मे॒ \newline

\textbf{Jatai Paata} \newline

1. अश्मा॑ च॒ चाश्मा ऽश्मा॑ च । \newline
2. च॒ मे॒ मे॒ च॒ च॒ मे॒ । \newline
3. मे॒ मृत्ति॑का॒ मृत्ति॑का मे मे॒ मृत्ति॑का । \newline
4. मृत्ति॑का च च॒ मृत्ति॑का॒ मृत्ति॑का च । \newline
5. च॒ मे॒ मे॒ च॒ च॒ मे॒ । \newline
6. मे॒ गि॒रयो॑ गि॒रयो॑ मे मे गि॒रयः॑ । \newline
7. गि॒रय॑श्च च गि॒रयो॑ गि॒रय॑श्च । \newline
8. च॒ मे॒ मे॒ च॒ च॒ मे॒ । \newline
9. मे॒ पर्व॑ताः॒ पर्व॑ता मे मे॒ पर्व॑ताः । \newline
10. पर्व॑ताश्च च॒ पर्व॑ताः॒ पर्व॑ताश्च । \newline
11. च॒ मे॒ मे॒ च॒ च॒ मे॒ । \newline
12. मे॒ सिक॑ताः॒ सिक॑ता मे मे॒ सिक॑ताः । \newline
13. सिक॑ताश्च च॒ सिक॑ताः॒ सिक॑ताश्च । \newline
14. च॒ मे॒ मे॒ च॒ च॒ मे॒ । \newline
15. मे॒ वन॒स्पत॑यो॒ वन॒स्पत॑यो मे मे॒ वन॒स्पत॑यः । \newline
16. वन॒स्पत॑यश्च च॒ वन॒स्पत॑यो॒ वन॒स्पत॑यश्च । \newline
17. च॒ मे॒ मे॒ च॒ च॒ मे॒ । \newline
18. मे॒ हिर॑ण्य॒(ग्म्॒) हिर॑ण्यम् मे मे॒ हिर॑ण्यम् । \newline
19. हिर॑ण्यम् च च॒ हिर॑ण्य॒(ग्म्॒) हिर॑ण्यम् च । \newline
20. च॒ मे॒ मे॒ च॒ च॒ मे॒ । \newline
21. मे ऽयो ऽयो॑ मे॒ मे ऽयः॑ । \newline
22. अय॑श्च॒ चायो ऽय॑श्च । \newline
23. च॒ मे॒ मे॒ च॒ च॒ मे॒ । \newline
24. मे॒ सीस॒(ग्म्॒) सीस॑म् मे मे॒ सीस᳚म् । \newline
25. सीस॑म् च च॒ सीस॒(ग्म्॒) सीस॑म् च । \newline
26. च॒ मे॒ मे॒ च॒ च॒ मे॒ । \newline
27. मे॒ त्रपु॒ त्रपु॑ मे मे॒ त्रपु॑ । \newline
28. त्रपु॑श् च च॒ त्रपु॒ त्रपु॑श् च । \newline
29. च॒ मे॒ मे॒ च॒ च॒ मे॒ । \newline
30. मे॒ श्या॒मꣳ श्या॒मम् मे॑ मे श्या॒मम् । \newline
31. श्या॒मम् च॑ च श्या॒मꣳ श्या॒मम् च॑ । \newline
32. च॒ मे॒ मे॒ च॒ च॒ मे॒ । \newline
33. मे॒ लो॒हम् ॅलो॒हम् मे॑ मे लो॒हम् । \newline
34. लो॒हम् च॑ च लो॒हम् ॅलो॒हम् च॑ । \newline
35. च॒ मे॒ मे॒ च॒ च॒ मे॒ । \newline
36. मे॒ ऽग्नि र॒ग्निर् मे॑ मे॒ ऽग्निः । \newline
37. अ॒ग्निश्च॑ चा॒ग्नि र॒ग्निश्च॑ । \newline
38. च॒ मे॒ मे॒ च॒ च॒ मे॒ । \newline
39. म॒ आप॒ आपो॑ मे म॒ आपः॑ । \newline
40. आप॑श्च॒ चाप॒ आप॑श्च । \newline
41. च॒ मे॒ मे॒ च॒ च॒ मे॒ । \newline
42. मे॒ वी॒रुधो॑ वी॒रुधो॑ मे मे वी॒रुधः॑ । \newline
43. वी॒रुध॑श्च च वी॒रुधो॑ वी॒रुध॑श्च । \newline
44. च॒ मे॒ मे॒ च॒ च॒ मे॒ । \newline
45. म॒ ओष॑धय॒ ओष॑धयो मे म॒ ओष॑धयः । \newline
46. ओष॑धयश्च॒ चौष॑धय॒ ओष॑धयश्च । \newline
47. च॒ मे॒ मे॒ च॒ च॒ मे॒ । \newline
48. मे॒ कृ॒ष्ट॒प॒च्यम् कृ॑ष्टप॒च्यम् मे॑ मे कृष्टप॒च्यम् । \newline
49. कृ॒ष्ट॒प॒च्यम् च॑ च कृष्टप॒च्यम् कृ॑ष्टप॒च्यम् च॑ । \newline
50. कृ॒ष्ट॒प॒च्यमिति॑ कृष्ट - प॒च्यम् । \newline
51. च॒ मे॒ मे॒ च॒ च॒ मे॒ । \newline

\textbf{Ghana Paata } \newline

1. अश्मा॑ च॒ चाश्मा ऽश्मा॑ च मे मे॒ चाश्मा ऽश्मा॑ च मे । \newline
2. च॒ मे॒ मे॒ च॒ च॒ मे॒ मृत्ति॑का॒ मृत्ति॑का मे च च मे॒ मृत्ति॑का । \newline
3. मे॒ मृत्ति॑का॒ मृत्ति॑का मे मे॒ मृत्ति॑का च च॒ मृत्ति॑का मे मे॒ मृत्ति॑का च । \newline
4. मृत्ति॑का च च॒ मृत्ति॑का॒ मृत्ति॑का च मे मे च॒ मृत्ति॑का॒ मृत्ति॑का च मे । \newline
5. च॒ मे॒ मे॒ च॒ च॒ मे॒ गि॒रयो॑ गि॒रयो॑ मे च च मे गि॒रयः॑ । \newline
6. मे॒ गि॒रयो॑ गि॒रयो॑ मे मे गि॒रय॑श्च च गि॒रयो॑ मे मे गि॒रय॑श्च । \newline
7. गि॒रय॑श्च च गि॒रयो॑ गि॒रय॑श्च मे मे च गि॒रयो॑ गि॒रय॑श्च मे । \newline
8. च॒ मे॒ मे॒ च॒ च॒ मे॒ पर्व॑ताः॒ पर्व॑ता मे च च मे॒ पर्व॑ताः । \newline
9. मे॒ पर्व॑ताः॒ पर्व॑ता मे मे॒ पर्व॑ताश्च च॒ पर्व॑ता मे मे॒ पर्व॑ताश्च । \newline
10. पर्व॑ताश्च च॒ पर्व॑ताः॒ पर्व॑ताश्च मे मे च॒ पर्व॑ताः॒ पर्व॑ताश्च मे । \newline
11. च॒ मे॒ मे॒ च॒ च॒ मे॒ सिक॑ताः॒ सिक॑ता मे च च मे॒ सिक॑ताः । \newline
12. मे॒ सिक॑ताः॒ सिक॑ता मे मे॒ सिक॑ताश्च च॒ सिक॑ता मे मे॒ सिक॑ताश्च । \newline
13. सिक॑ताश्च च॒ सिक॑ताः॒ सिक॑ताश्च मे मे च॒ सिक॑ताः॒ सिक॑ताश्च मे । \newline
14. च॒ मे॒ मे॒ च॒ च॒ मे॒ वन॒स्पत॑यो॒ वन॒स्पत॑यो मे च च मे॒ वन॒स्पत॑यः । \newline
15. मे॒ वन॒स्पत॑यो॒ वन॒स्पत॑यो मे मे॒ वन॒स्पत॑यश्च च॒ वन॒स्पत॑यो मे मे॒ वन॒स्पत॑यश्च । \newline
16. वन॒स्पत॑यश्च च॒ वन॒स्पत॑यो॒ वन॒स्पत॑यश्च मे मे च॒ वन॒स्पत॑यो॒ वन॒स्पत॑यश्च मे । \newline
17. च॒ मे॒ मे॒ च॒ च॒ मे॒ हिर॑ण्य॒(ग्म्॒) हिर॑ण्यम् मे च च मे॒ हिर॑ण्यम् । \newline
18. मे॒ हिर॑ण्य॒(ग्म्॒) हिर॑ण्यम् मे मे॒ हिर॑ण्यम् च च॒ हिर॑ण्यम् मे मे॒ हिर॑ण्यम् च । \newline
19. हिर॑ण्यम् च च॒ हिर॑ण्य॒(ग्म्॒) हिर॑ण्यम् च मे मे च॒ हिर॑ण्य॒(ग्म्॒) हिर॑ण्यम् च मे । \newline
20. च॒ मे॒ मे॒ च॒ च॒ मे ऽयो ऽयो॑ मे च च॒ मे ऽयः॑ । \newline
21. मे ऽयो ऽयो॑ मे॒ मे ऽय॑श्च॒ चायो॑ मे॒ मे ऽय॑श्च । \newline
22. अय॑श्च॒ चायो ऽय॑श्च मे मे॒ चायो ऽय॑श्च मे । \newline
23. च॒ मे॒ मे॒ च॒ च॒ मे॒ सीस॒(ग्म्॒) सीस॑म् मे च च मे॒ सीस᳚म् । \newline
24. मे॒ सीस॒(ग्म्॒) सीस॑म् मे मे॒ सीस॑म् च च॒ सीस॑म् मे मे॒ सीस॑म् च । \newline
25. सीस॑म् च च॒ सीस॒(ग्म्॒) सीस॑म् च मे मे च॒ सीस॒(ग्म्॒) सीस॑म् च मे । \newline
26. च॒ मे॒ मे॒ च॒ च॒ मे॒ त्रपु॒ त्रपु॑ मे च च मे॒ त्रपु॑ । \newline
27. मे॒ त्रपु॒ त्रपु॑ मे मे॒ त्रपु॑श् च च॒ त्रपु॑ मे मे॒ त्रपु॑श् च । \newline
28. त्रपु॑श् च च॒ त्रपु॒ त्रपु॑श् च मे मे च॒ त्रपु॒ त्रपु॑श् च मे । \newline
29. च॒ मे॒ मे॒ च॒ च॒ मे॒ श्या॒मꣳ श्या॒मम् मे॑ च च मे श्या॒मम् । \newline
30. मे॒ श्या॒मꣳ श्या॒मम् मे॑ मे श्या॒मम् च॑ च श्या॒मम् मे॑ मे श्या॒मम् च॑ । \newline
31. श्या॒मम् च॑ च श्या॒मꣳ श्या॒मम् च॑ मे मे च श्या॒मꣳ श्या॒मम् च॑ मे । \newline
32. च॒ मे॒ मे॒ च॒ च॒ मे॒ लो॒हम् ॅलो॒हम् मे॑ च च मे लो॒हम् । \newline
33. मे॒ लो॒हम् ॅलो॒हम् मे॑ मे लो॒हम् च॑ च लो॒हम् मे॑ मे लो॒हम् च॑ । \newline
34. लो॒हम् च॑ च लो॒हम् ॅलो॒हम् च॑ मे मे च लो॒हम् ॅलो॒हम् च॑ मे । \newline
35. च॒ मे॒ मे॒ च॒ च॒ मे॒ ऽग्नि र॒ग्निर् मे॑ च च मे॒ ऽग्निः । \newline
36. मे॒ ऽग्नि र॒ग्निर् मे॑ मे॒ ऽग्निश्च॑ चा॒ग्निर् मे॑ मे॒ ऽग्निश्च॑ । \newline
37. अ॒ग्निश्च॑ चा॒ग्नि र॒ग्निश्च॑ मे मे चा॒ग्नि र॒ग्निश्च॑ मे । \newline
38. च॒ मे॒ मे॒ च॒ च॒ म॒ आप॒ आपो॑ मे च च म॒ आपः॑ । \newline
39. म॒ आप॒ आपो॑ मे म॒ आप॑श्च॒ चापो॑ मे म॒ आप॑श्च । \newline
40. आप॑श्च॒ चाप॒ आप॑श्च मे मे॒ चाप॒ आप॑श्च मे । \newline
41. च॒ मे॒ मे॒ च॒ च॒ मे॒ वी॒रुधो॑ वी॒रुधो॑ मे च च मे वी॒रुधः॑ । \newline
42. मे॒ वी॒रुधो॑ वी॒रुधो॑ मे मे वी॒रुध॑श्च च वी॒रुधो॑ मे मे वी॒रुध॑श्च । \newline
43. वी॒रुध॑श्च च वी॒रुधो॑ वी॒रुध॑श्च मे मे च वी॒रुधो॑ वी॒रुध॑श्च मे । \newline
44. च॒ मे॒ मे॒ च॒ च॒ म॒ ओष॑धय॒ ओष॑धयो मे च च म॒ ओष॑धयः । \newline
45. म॒ ओष॑धय॒ ओष॑धयो मे म॒ ओष॑धयश्च॒ चौष॑धयो मे म॒ ओष॑धयश्च । \newline
46. ओष॑धयश्च॒ चौष॑धय॒ ओष॑धयश्च मे मे॒ चौष॑धय॒ ओष॑धयश्च मे । \newline
47. च॒ मे॒ मे॒ च॒ च॒ मे॒ कृ॒ष्ट॒प॒च्यम् कृ॑ष्टप॒च्यम् मे॑ च च मे कृष्टप॒च्यम् । \newline
48. मे॒ कृ॒ष्ट॒प॒च्यम् कृ॑ष्टप॒च्यम् मे॑ मे कृष्टप॒च्यम् च॑ च कृष्टप॒च्यम् मे॑ मे कृष्टप॒च्यम् च॑ । \newline
49. कृ॒ष्ट॒प॒च्यम् च॑ च कृष्टप॒च्यम् कृ॑ष्टप॒च्यम् च॑ मे मे च कृष्टप॒च्यम् कृ॑ष्टप॒च्यम् च॑ मे । \newline
50. कृ॒ष्ट॒प॒च्यमिति॑ कृष्ट - प॒च्यम् । \newline
51. च॒ मे॒ मे॒ च॒ च॒ मे॒ ऽकृ॒ष्ट॒प॒च्य म॑कृष्टप॒च्यम् मे॑ च च मे ऽकृष्टप॒च्यम् । \newline
\pagebreak
\markright{ TS 4.7.5.2  \hfill https://www.vedavms.in \hfill}

\section{ TS 4.7.5.2 }

\textbf{TS 4.7.5.2 } \newline
\textbf{Samhita Paata} \newline

मे ऽकृष्टप॒च्यं च॑ मे ग्रा॒यांश्च॑ मे प॒शव॑ आर॒ण्याश्च॑ य॒ज्ञेन॑ कल्पन्तां ॅवि॒त्तं च॑ मे॒ वित्ति॑श्च मे भू॒तं च॑ मे॒ भूति॑श्च मे॒ वसु॑ च मे वस॒तिश्च॑ मे॒ कर्म॑ च मे॒ शक्ति॑श्च॒ मेऽर्थ॑श्च म॒ एम॑श्च म॒ इति॑श्च मे॒ गति॑श्च मे ॥ \newline

\textbf{Pada Paata} \newline

मे॒ । अ॒कृ॒ष्ट॒प॒च्यमित्य॑कृष्ट - प॒च्यम् । च॒ । मे॒ । ग्रा॒म्याः । च॒ । मे॒ । प॒शवः॑ । आ॒र॒ण्याः । च॒ । य॒ज्ञेन॑ । क॒ल्प॒न्ता॒म् । वि॒त्तम् । च॒ । मे॒ । वित्तिः॑ । च॒ । मे॒ । भू॒तम् । च॒ । मे॒ । भूतिः॑ । च॒ । मे॒ । वसु॑ । च॒ । मे॒ । व॒स॒तिः । च॒ । मे॒ । कर्म॑ । च॒ । मे॒ । शक्तिः॑ । च॒ । मे॒ । अर्थः॑ । च॒ । मे॒ । एमः॑ । च॒ । मे॒ । इतिः॑ । च॒ । मे॒ । गतिः॑ । च॒ । मे॒ ॥  \newline


\textbf{Krama Paata} \newline

मे॒ऽकृ॒ष्ट॒प॒च्यम् । अ॒कृ॒ष्ट॒प॒च्यम् च॑ । अ॒कृ॒ष्ट॒प॒च्यमित्य॑कृष्ट - प॒च्यम् । च॒ मे॒ । मे॒ ग्रा॒म्याः । ग्रा॒म्याश्च॑ । च॒ मे॒ । मे॒ प॒शवः॑ । प॒शव॑ आर॒ण्याः । आ॒र॒ण्याश्च॑ । च॒ य॒ज्ञेन॑ । य॒ज्ञेन॑ कल्पन्ताम् । क॒ल्प॒न्ता॒म् ॅवि॒त्तम् । वि॒त्तम् च॑ । च॒ मे॒ । मे॒ वित्तिः॑ । वित्ति॑श्च । च॒ मे॒ । मे॒ भू॒तम् । भू॒तम् च॑ । च॒ मे॒ । मे॒ भूतिः॑ । भूति॑श्च । च॒ मे॒ । मे॒ वसु॑ । वसु॑ च । च॒ मे॒ । मे॒ व॒स॒तिः । व॒स॒तिश्च॑ । च॒ मे॒ । मे॒ कर्म॑ । कर्म॑ च । च॒ मे॒ । मे॒ शक्तिः॑ । शक्ति॑श्च । च॒ मे॒ । मेऽर्त्थः॑ । अर्त्थ॑श्च । च॒ मे॒ । म॒ एमः॑ । एम॑श्च । च॒ मे॒ । म॒ इतिः॑ । इति॑श्च । च॒ मे॒ । मे॒ गतिः॑ । गति॑श्च । च॒ मे॒ । म॒ इति॑ मे । \newline

\textbf{Jatai Paata} \newline

1. मे॒ ऽकृ॒ष्ट॒प॒च्य म॑कृष्टप॒च्यम् मे॑ मे ऽकृष्टप॒च्यम् । \newline
2. अ॒कृ॒ष्ट॒प॒च्यम् च॑ चाकृष्टप॒च्य म॑कृष्टप॒च्यम् च॑ । \newline
3. अ॒कृ॒ष्ट॒प॒च्यमित्य॑कृष्ट - प॒च्यम् । \newline
4. च॒ मे॒ मे॒ च॒ च॒ मे॒ । \newline
5. मे॒ ग्रा॒म्या ग्रा॒म्या मे॑ मे ग्रा॒म्याः । \newline
6. ग्रा॒म्याश्च॑ च ग्रा॒म्या ग्रा॒म्याश्च॑ । \newline
7. च॒ मे॒ मे॒ च॒ च॒ मे॒ । \newline
8. मे॒ प॒शवः॑ प॒शवो॑ मे मे प॒शवः॑ । \newline
9. प॒शव॑ आर॒ण्या आ॑र॒ण्याः प॒शवः॑ प॒शव॑ आर॒ण्याः । \newline
10. आ॒र॒ण्याश्च॑ चार॒ण्या आ॑र॒ण्याश्च॑ । \newline
11. च॒ य॒ज्ञेन॑ य॒ज्ञेन॑ च च य॒ज्ञेन॑ । \newline
12. य॒ज्ञेन॑ कल्पन्ताम् कल्पन्तां ॅय॒ज्ञेन॑ य॒ज्ञेन॑ कल्पन्ताम् । \newline
13. क॒ल्प॒न्तां॒ ॅवि॒त्तं ॅवि॒त्तम् क॑ल्पन्ताम् कल्पन्तां ॅवि॒त्तम् । \newline
14. वि॒त्तम् च॑ च वि॒त्तं ॅवि॒त्तम् च॑ । \newline
15. च॒ मे॒ मे॒ च॒ च॒ मे॒ । \newline
16. मे॒ वित्ति॒र् वित्ति॑र् मे मे॒ वित्तिः॑ । \newline
17. वित्ति॑श्च च॒ वित्ति॒र् वित्ति॑श्च । \newline
18. च॒ मे॒ मे॒ च॒ च॒ मे॒ । \newline
19. मे॒ भू॒तम् भू॒तम् मे॑ मे भू॒तम् । \newline
20. भू॒तम् च॑ च भू॒तम् भू॒तम् च॑ । \newline
21. च॒ मे॒ मे॒ च॒ च॒ मे॒ । \newline
22. मे॒ भूति॒र् भूति॑र् मे मे॒ भूतिः॑ । \newline
23. भूति॑श्च च॒ भूति॒र् भूति॑श्च । \newline
24. च॒ मे॒ मे॒ च॒ च॒ मे॒ । \newline
25. मे॒ वसु॒ वसु॑ मे मे॒ वसु॑ । \newline
26. वसु॑ च च॒ वसु॒ वसु॑ च । \newline
27. च॒ मे॒ मे॒ च॒ च॒ मे॒ । \newline
28. मे॒ व॒स॒तिर् व॑स॒तिर् मे॑ मे वस॒तिः । \newline
29. व॒स॒तिश्च॑ च वस॒तिर् व॑स॒तिश्च॑ । \newline
30. च॒ मे॒ मे॒ च॒ च॒ मे॒ । \newline
31. मे॒ कर्म॒ कर्म॑ मे मे॒ कर्म॑ । \newline
32. कर्म॑ च च॒ कर्म॒ कर्म॑ च । \newline
33. च॒ मे॒ मे॒ च॒ च॒ मे॒ । \newline
34. मे॒ शक्तिः॒ शक्ति॑र् मे मे॒ शक्तिः॑ । \newline
35. शक्ति॑श्च च॒ शक्तिः॒ शक्ति॑श्च । \newline
36. च॒ मे॒ मे॒ च॒ च॒ मे॒ । \newline
37. मे ऽर्थो ऽर्थो॑ मे॒ मे ऽर्थः॑ । \newline
38. अर्थ॑श्च॒ चार्थो ऽर्थ॑श्च । \newline
39. च॒ मे॒ मे॒ च॒ च॒ मे॒ । \newline
40. म॒ एम॒ एमो॑ मे म॒ एमः॑ । \newline
41. एम॑श्च॒ चैम॒ एम॑श्च । \newline
42. च॒ मे॒ मे॒ च॒ च॒ मे॒ । \newline
43. म॒ इति॒ रिति॑र् मे म॒ इतिः॑ । \newline
44. इति॑श्च॒ चेति॒ रिति॑श्च । \newline
45. च॒ मे॒ मे॒ च॒ च॒ मे॒ । \newline
46. मे॒ गति॒र् गति॑र् मे मे॒ गतिः॑ । \newline
47. गति॑श्च च॒ गति॒र् गति॑श्च । \newline
48. च॒ मे॒ मे॒ च॒ च॒ मे॒ । \newline
49. म॒ इति॑ मे । \newline

\textbf{Ghana Paata } \newline

1. मे॒ ऽकृ॒ष्ट॒प॒च्य म॑कृष्टप॒च्यम् मे॑ मे ऽकृष्टप॒च्यम् च॑ चाकृष्टप॒च्यम् मे॑ मे ऽकृष्टप॒च्यम् च॑ । \newline
2. अ॒कृ॒ष्ट॒प॒च्यम् च॑ चाकृष्टप॒च्य म॑कृष्टप॒च्यम् च॑ मे मे चाकृष्टप॒च्य म॑कृष्टप॒च्यम् च॑ मे । \newline
3. अ॒कृ॒ष्ट॒प॒च्यमित्य॑कृष्ट - प॒च्यम् । \newline
4. च॒ मे॒ मे॒ च॒ च॒ मे॒ ग्रा॒म्या ग्रा॒म्या मे॑ च च मे ग्रा॒म्याः । \newline
5. मे॒ ग्रा॒म्या ग्रा॒म्या मे॑ मे ग्रा॒म्याश्च॑ च ग्रा॒म्या मे॑ मे ग्रा॒म्याश्च॑ । \newline
6. ग्रा॒म्याश्च॑ च ग्रा॒म्या ग्रा॒म्याश्च॑ मे मे च ग्रा॒म्या ग्रा॒म्याश्च॑ मे । \newline
7. च॒ मे॒ मे॒ च॒ च॒ मे॒ प॒शवः॑ प॒शवो॑ मे च च मे प॒शवः॑ । \newline
8. मे॒ प॒शवः॑ प॒शवो॑ मे मे प॒शव॑ आर॒ण्या आ॑र॒ण्याः प॒शवो॑ मे मे प॒शव॑ आर॒ण्याः । \newline
9. प॒शव॑ आर॒ण्या आ॑र॒ण्याः प॒शवः॑ प॒शव॑ आर॒ण्याश्च॑ चार॒ण्याः प॒शवः॑ प॒शव॑ आर॒ण्याश्च॑ । \newline
10. आ॒र॒ण्याश्च॑ चार॒ण्या आ॑र॒ण्याश्च॑ य॒ज्ञेन॑ य॒ज्ञेन॑ चार॒ण्या आ॑र॒ण्याश्च॑ य॒ज्ञेन॑ । \newline
11. च॒ य॒ज्ञेन॑ य॒ज्ञेन॑ च च य॒ज्ञेन॑ कल्पन्ताम् कल्पन्तां ॅय॒ज्ञेन॑ च च य॒ज्ञेन॑ कल्पन्ताम् । \newline
12. य॒ज्ञेन॑ कल्पन्ताम् कल्पन्तां ॅय॒ज्ञेन॑ य॒ज्ञेन॑ कल्पन्तां ॅवि॒त्तं ॅवि॒त्तम् क॑ल्पन्तां ॅय॒ज्ञेन॑ य॒ज्ञेन॑ कल्पन्तां ॅवि॒त्तम् । \newline
13. क॒ल्प॒न्तां॒ ॅवि॒त्तं ॅवि॒त्तम् क॑ल्पन्ताम् कल्पन्तां ॅवि॒त्तम् च॑ च वि॒त्तम् क॑ल्पन्ताम् कल्पन्तां ॅवि॒त्तम् च॑ । \newline
14. वि॒त्तम् च॑ च वि॒त्तं ॅवि॒त्तम् च॑ मे मे च वि॒त्तं ॅवि॒त्तम् च॑ मे । \newline
15. च॒ मे॒ मे॒ च॒ च॒ मे॒ वित्ति॒र् वित्ति॑र् मे च च मे॒ वित्तिः॑ । \newline
16. मे॒ वित्ति॒र् वित्ति॑र् मे मे॒ वित्ति॑श्च च॒ वित्ति॑र् मे मे॒ वित्ति॑श्च । \newline
17. वित्ति॑श्च च॒ वित्ति॒र् वित्ति॑श्च मे मे च॒ वित्ति॒र् वित्ति॑श्च मे । \newline
18. च॒ मे॒ मे॒ च॒ च॒ मे॒ भू॒तम् भू॒तम् मे॑ च च मे भू॒तम् । \newline
19. मे॒ भू॒तम् भू॒तम् मे॑ मे भू॒तम् च॑ च भू॒तम् मे॑ मे भू॒तम् च॑ । \newline
20. भू॒तम् च॑ च भू॒तम् भू॒तम् च॑ मे मे च भू॒तम् भू॒तम् च॑ मे । \newline
21. च॒ मे॒ मे॒ च॒ च॒ मे॒ भूति॒र् भूति॑र् मे च च मे॒ भूतिः॑ । \newline
22. मे॒ भूति॒र् भूति॑र् मे मे॒ भूति॑श्च च॒ भूति॑र् मे मे॒ भूति॑श्च । \newline
23. भूति॑श्च च॒ भूति॒र् भूति॑श्च मे मे च॒ भूति॒र् भूति॑श्च मे । \newline
24. च॒ मे॒ मे॒ च॒ च॒ मे॒ वसु॒ वसु॑ मे च च मे॒ वसु॑ । \newline
25. मे॒ वसु॒ वसु॑ मे मे॒ वसु॑ च च॒ वसु॑ मे मे॒ वसु॑ च । \newline
26. वसु॑ च च॒ वसु॒ वसु॑ च मे मे च॒ वसु॒ वसु॑ च मे । \newline
27. च॒ मे॒ मे॒ च॒ च॒ मे॒ व॒स॒तिर् व॑स॒तिर् मे॑ च च मे वस॒तिः । \newline
28. मे॒ व॒स॒तिर् व॑स॒तिर् मे॑ मे वस॒तिश्च॑ च वस॒तिर् मे॑ मे वस॒तिश्च॑ । \newline
29. व॒स॒तिश्च॑ च वस॒तिर् व॑स॒तिश्च॑ मे मे च वस॒तिर् व॑स॒तिश्च॑ मे । \newline
30. च॒ मे॒ मे॒ च॒ च॒ मे॒ कर्म॒ कर्म॑ मे च च मे॒ कर्म॑ । \newline
31. मे॒ कर्म॒ कर्म॑ मे मे॒ कर्म॑ च च॒ कर्म॑ मे मे॒ कर्म॑ च । \newline
32. कर्म॑ च च॒ कर्म॒ कर्म॑ च मे मे च॒ कर्म॒ कर्म॑ च मे । \newline
33. च॒ मे॒ मे॒ च॒ च॒ मे॒ शक्तिः॒ शक्ति॑र् मे च च मे॒ शक्तिः॑ । \newline
34. मे॒ शक्तिः॒ शक्ति॑र् मे मे॒ शक्ति॑श्च च॒ शक्ति॑र् मे मे॒ शक्ति॑श्च । \newline
35. शक्ति॑श्च च॒ शक्तिः॒ शक्ति॑श्च मे मे च॒ शक्तिः॒ शक्ति॑श्च मे । \newline
36. च॒ मे॒ मे॒ च॒ च॒ मे ऽर्थो ऽर्थो॑ मे च च॒ मे ऽर्थः॑ । \newline
37. मे ऽर्थो ऽर्थो॑ मे॒ मे ऽर्थ॑श्च॒ चार्थो॑ मे॒ मे ऽर्थ॑श्च । \newline
38. अर्थ॑श्च॒ चार्थो ऽर्थ॑श्च मे मे॒ चार्थो ऽर्थ॑श्च मे । \newline
39. च॒ मे॒ मे॒ च॒ च॒ म॒ एम॒ एमो॑ मे च च म॒ एमः॑ । \newline
40. म॒ एम॒ एमो॑ मे म॒ एम॑श्च॒ चैमो॑ मे म॒ एम॑श्च । \newline
41. एम॑श्च॒ चैम॒ एम॑श्च मे मे॒ चैम॒ एम॑श्च मे । \newline
42. च॒ मे॒ मे॒ च॒ च॒ म॒ इति॒ रिति॑र् मे च च म॒ इतिः॑ । \newline
43. म॒ इति॒ रिति॑र् मे म॒ इति॑श्च॒ चेति॑र् मे म॒ इति॑श्च । \newline
44. इति॑श्च॒ चेति॒ रिति॑श्च मे मे॒ चेति॒ रिति॑श्च मे । \newline
45. च॒ मे॒ मे॒ च॒ च॒ मे॒ गति॒र् गति॑र् मे च च मे॒ गतिः॑ । \newline
46. मे॒ गति॒र् गति॑र् मे मे॒ गति॑श्च च॒ गति॑र् मे मे॒ गति॑श्च । \newline
47. गति॑श्च च॒ गति॒र् गति॑श्च मे मे च॒ गति॒र् गति॑श्च मे । \newline
48. च॒ मे॒ मे॒ च॒ च॒ मे॒ । \newline
49. म॒ इति॑ मे । \newline
\pagebreak
\markright{ TS 4.7.6.1  \hfill https://www.vedavms.in \hfill}

\section{ TS 4.7.6.1 }

\textbf{TS 4.7.6.1 } \newline
\textbf{Samhita Paata} \newline

अ॒ग्निश्च॑ म॒ इन्द्र॑श्च मे॒ सोम॑श्च म॒ इन्द्र॑श्च मे सवि॒ता च॑ म॒ इन्द्र॑श्च मे॒ सर॑स्वती च म॒ इन्द्र॑श्च मे पू॒षा च॑ म॒ इन्द्र॑श्च मे॒ बृह॒स्पति॑श्च म॒ इन्द्र॑श्च मे मि॒त्रश्च॑ म॒ इन्द्र॑श्च मे॒ वरु॑णश्च म॒ इन्द्र॑श्च मे॒ त्वष्टा॑ च - [  ] \newline

\textbf{Pada Paata} \newline

अ॒ग्निः । च॒ । मे॒ । इन्द्रः॑ । च॒ । मे॒ । सोमः॑ । च॒ । मे॒ । इन्द्रः॑ । च॒ । मे॒ । स॒वि॒ता । च॒ । मे॒ । इन्द्रः॑ । च॒ । मे॒ । सर॑स्वती । च॒ । मे॒ । इन्द्रः॑ । च॒ । मे॒ । पू॒षा । च॒ । मे॒ । इन्द्रः॑ । च॒ । मे॒ । बृह॒स्पतिः॑ । च॒ । मे॒ । इन्द्रः॑ । च॒ । मे॒ । मि॒त्रः । च॒ । मे॒ । इन्द्रः॑ । च॒ । मे॒ । वरु॑णः । च॒ । मे॒ । इन्द्रः॑ । च॒ । मे॒ । त्वष्टा᳚ । च॒ ।  \newline


\textbf{Krama Paata} \newline

अ॒ग्निश्च॑ । च॒ मे॒ । म॒ इन्द्रः॑ । इन्द्र॑श्च । च॒ मे॒ । मे॒ सोमः॑ । सोम॑श्च । च॒ मे॒ । म॒ इन्द्रः॑ । इन्द्र॑श्च । च॒ मे॒ । मे॒ स॒वि॒ता । स॒वि॒ता च॑ । च॒ मे॒ । म॒ इन्द्रः॑ । इन्द्र॑श्च । च॒ मे॒ । मे॒ सर॑स्वती । सर॑स्वती च । च॒ मे॒ । म॒ इन्द्रः॑ । इन्द्र॑श्च । च॒ मे॒ । मे॒ पू॒षा । पू॒षा च॑ । च॒ मे॒ । म॒ इन्द्रः॑ । इन्द्र॑श्च । च॒ मे॒ । मे॒ बृह॒स्पतिः॑ । बृह॒स्पति॑श्च । च॒ मे॒ । म॒ इन्द्रः॑ । इन्द्र॑श्च । च॒ मे॒ । मे॒ मि॒त्रः । मि॒त्रश्च॑ । च॒ मे॒ । म॒ इन्द्रः॑ । इन्द्र॑श्च । च॒ मे॒ । मे॒ वरु॑णः । वरु॑णश्च । च॒ मे॒ । म॒ इन्द्रः॑ । इन्द्र॑श्च । च॒ मे॒ । मे॒ त्वष्टा᳚ । त्वष्टा॑ च । च॒ मे॒ \newline

\textbf{Jatai Paata} \newline

1. अ॒ग्निश्च॑ चा॒ग्नि र॒ग्निश्च॑ । \newline
2. च॒ मे॒ मे॒ च॒ च॒ मे॒ । \newline
3. म॒ इन्द्र॒ इन्द्रो॑ मे म॒ इन्द्रः॑ । \newline
4. इन्द्र॑श्च॒ चेन्द्र॒ इन्द्र॑श्च । \newline
5. च॒ मे॒ मे॒ च॒ च॒ मे॒ । \newline
6. मे॒ सोमः॒ सोमो॑ मे मे॒ सोमः॑ । \newline
7. सोम॑श्च च॒ सोमः॒ सोम॑श्च । \newline
8. च॒ मे॒ मे॒ च॒ च॒ मे॒ । \newline
9. म॒ इन्द्र॒ इन्द्रो॑ मे म॒ इन्द्रः॑ । \newline
10. इन्द्र॑श्च॒ चेन्द्र॒ इन्द्र॑श्च । \newline
11. च॒ मे॒ मे॒ च॒ च॒ मे॒ । \newline
12. मे॒ स॒वि॒ता स॑वि॒ता मे॑ मे सवि॒ता । \newline
13. स॒वि॒ता च॑ च सवि॒ता स॑वि॒ता च॑ । \newline
14. च॒ मे॒ मे॒ च॒ च॒ मे॒ । \newline
15. म॒ इन्द्र॒ इन्द्रो॑ मे म॒ इन्द्रः॑ । \newline
16. इन्द्र॑श्च॒ चेन्द्र॒ इन्द्र॑श्च । \newline
17. च॒ मे॒ मे॒ च॒ च॒ मे॒ । \newline
18. मे॒ सर॑स्वती॒ सर॑स्वती मे मे॒ सर॑स्वती । \newline
19. सर॑स्वती च च॒ सर॑स्वती॒ सर॑स्वती च । \newline
20. च॒ मे॒ मे॒ च॒ च॒ मे॒ । \newline
21. म॒ इन्द्र॒ इन्द्रो॑ मे म॒ इन्द्रः॑ । \newline
22. इन्द्र॑श्च॒ चेन्द्र॒ इन्द्र॑श्च । \newline
23. च॒ मे॒ मे॒ च॒ च॒ मे॒ । \newline
24. मे॒ पू॒षा पू॒षा मे॑ मे पू॒षा । \newline
25. पू॒षा च॑ च पू॒षा पू॒षा च॑ । \newline
26. च॒ मे॒ मे॒ च॒ च॒ मे॒ । \newline
27. म॒ इन्द्र॒ इन्द्रो॑ मे म॒ इन्द्रः॑ । \newline
28. इन्द्र॑श्च॒ चेन्द्र॒ इन्द्र॑श्च । \newline
29. च॒ मे॒ मे॒ च॒ च॒ मे॒ । \newline
30. मे॒ बृह॒स्पति॒र् बृह॒स्पति॑र् मे मे॒ बृह॒स्पतिः॑ । \newline
31. बृह॒स्पति॑श्च च॒ बृह॒स्पति॒र् बृह॒स्पति॑श्च । \newline
32. च॒ मे॒ मे॒ च॒ च॒ मे॒ । \newline
33. म॒ इन्द्र॒ इन्द्रो॑ मे म॒ इन्द्रः॑ । \newline
34. इन्द्र॑श्च॒ चेन्द्र॒ इन्द्र॑श्च । \newline
35. च॒ मे॒ मे॒ च॒ च॒ मे॒ । \newline
36. मे॒ मि॒त्रो मि॒त्रो मे॑ मे मि॒त्रः । \newline
37. मि॒त्रश्च॑ च मि॒त्रो मि॒त्रश्च॑ । \newline
38. च॒ मे॒ मे॒ च॒ च॒ मे॒ । \newline
39. म॒ इन्द्र॒ इन्द्रो॑ मे म॒ इन्द्रः॑ । \newline
40. इन्द्र॑श्च॒ चेन्द्र॒ इन्द्र॑श्च । \newline
41. च॒ मे॒ मे॒ च॒ च॒ मे॒ । \newline
42. मे॒ वरु॑णो॒ वरु॑णो मे मे॒ वरु॑णः । \newline
43. वरु॑णश्च च॒ वरु॑णो॒ वरु॑णश्च । \newline
44. च॒ मे॒ मे॒ च॒ च॒ मे॒ । \newline
45. म॒ इन्द्र॒ इन्द्रो॑ मे म॒ इन्द्रः॑ । \newline
46. इन्द्र॑श्च॒ चे न्द्र॒ इन्द्र॑श्च । \newline
47. च॒ मे॒ मे॒ च॒ च॒ मे॒ । \newline
48. मे॒ त्वष्टा॒ त्वष्टा॑ मे मे॒ त्वष्टा᳚ । \newline
49. त्वष्टा॑ च च॒ त्वष्टा॒ त्वष्टा॑ च । \newline
50. च॒ मे॒ मे॒ च॒ च॒ मे॒ । \newline

\textbf{Ghana Paata } \newline

1. अ॒ग्निश्च॑ चा॒ग्नि र॒ग्निश्च॑ मे मे चा॒ग्नि र॒ग्निश्च॑ मे । \newline
2. च॒ मे॒ मे॒ च॒ च॒ म॒ इन्द्र॒ इन्द्रो॑ मे च च म॒ इन्द्रः॑ । \newline
3. म॒ इन्द्र॒ इन्द्रो॑ मे म॒ इन्द्र॑श्च॒ चेन्द्रो॑ मे म॒ इन्द्र॑श्च । \newline
4. इन्द्र॑श्च॒ चेन्द्र॒ इन्द्र॑श्च मे मे॒ चेन्द्र॒ इन्द्र॑श्च मे । \newline
5. च॒ मे॒ मे॒ च॒ च॒ मे॒ सोमः॒ सोमो॑ मे च च मे॒ सोमः॑ । \newline
6. मे॒ सोमः॒ सोमो॑ मे मे॒ सोम॑श्च च॒ सोमो॑ मे मे॒ सोम॑श्च । \newline
7. सोम॑श्च च॒ सोमः॒ सोम॑श्च मे मे च॒ सोमः॒ सोम॑श्च मे । \newline
8. च॒ मे॒ मे॒ च॒ च॒ म॒ इन्द्र॒ इन्द्रो॑ मे च च म॒ इन्द्रः॑ । \newline
9. म॒ इन्द्र॒ इन्द्रो॑ मे म॒ इन्द्र॑श्च॒ चेन्द्रो॑ मे म॒ इन्द्र॑श्च । \newline
10. इन्द्र॑श्च॒ चेन्द्र॒ इन्द्र॑श्च मे मे॒ चेन्द्र॒ इन्द्र॑श्च मे । \newline
11. च॒ मे॒ मे॒ च॒ च॒ मे॒ स॒वि॒ता स॑वि॒ता मे॑ च च मे सवि॒ता । \newline
12. मे॒ स॒वि॒ता स॑वि॒ता मे॑ मे सवि॒ता च॑ च सवि॒ता मे॑ मे सवि॒ता च॑ । \newline
13. स॒वि॒ता च॑ च सवि॒ता स॑वि॒ता च॑ मे मे च सवि॒ता स॑वि॒ता च॑ मे । \newline
14. च॒ मे॒ मे॒ च॒ च॒ म॒ इन्द्र॒ इन्द्रो॑ मे च च म॒ इन्द्रः॑ । \newline
15. म॒ इन्द्र॒ इन्द्रो॑ मे म॒ इन्द्र॑श्च॒ चेन्द्रो॑ मे म॒ इन्द्र॑श्च । \newline
16. इन्द्र॑श्च॒ चेन्द्र॒ इन्द्र॑श्च मे मे॒ चेन्द्र॒ इन्द्र॑श्च मे । \newline
17. च॒ मे॒ मे॒ च॒ च॒ मे॒ सर॑स्वती॒ सर॑स्वती मे च च मे॒ सर॑स्वती । \newline
18. मे॒ सर॑स्वती॒ सर॑स्वती मे मे॒ सर॑स्वती च च॒ सर॑स्वती मे मे॒ सर॑स्वती च । \newline
19. सर॑स्वती च च॒ सर॑स्वती॒ सर॑स्वती च मे मे च॒ सर॑स्वती॒ सर॑स्वती च मे । \newline
20. च॒ मे॒ मे॒ च॒ च॒ म॒ इन्द्र॒ इन्द्रो॑ मे च च म॒ इन्द्रः॑ । \newline
21. म॒ इन्द्र॒ इन्द्रो॑ मे म॒ इन्द्र॑श्च॒ चेन्द्रो॑ मे म॒ इन्द्र॑श्च । \newline
22. इन्द्र॑श्च॒ चेन्द्र॒ इन्द्र॑श्च मे मे॒ चेन्द्र॒ इन्द्र॑श्च मे । \newline
23. च॒ मे॒ मे॒ च॒ च॒ मे॒ पू॒षा पू॒षा मे॑ च च मे पू॒षा । \newline
24. मे॒ पू॒षा पू॒षा मे॑ मे पू॒षा च॑ च पू॒षा मे॑ मे पू॒षा च॑ । \newline
25. पू॒षा च॑ च पू॒षा पू॒षा च॑ मे मे च पू॒षा पू॒षा च॑ मे । \newline
26. च॒ मे॒ मे॒ च॒ च॒ म॒ इन्द्र॒ इन्द्रो॑ मे च च म॒ इन्द्रः॑ । \newline
27. म॒ इन्द्र॒ इन्द्रो॑ मे म॒ इन्द्र॑श्च॒ चेन्द्रो॑ मे म॒ इन्द्र॑श्च । \newline
28. इन्द्र॑श्च॒ चेन्द्र॒ इन्द्र॑श्च मे मे॒ चेन्द्र॒ इन्द्र॑श्च मे । \newline
29. च॒ मे॒ मे॒ च॒ च॒ मे॒ बृह॒स्पति॒र् बृह॒स्पति॑र् मे च च मे॒ बृह॒स्पतिः॑ । \newline
30. मे॒ बृह॒स्पति॒र् बृह॒स्पति॑र् मे मे॒ बृह॒स्पति॑श्च च॒ बृह॒स्पति॑र् मे मे॒ बृह॒स्पति॑श्च । \newline
31. बृह॒स्पति॑श्च च॒ बृह॒स्पति॒र् बृह॒स्पति॑श्च मे मे च॒ बृह॒स्पति॒र् बृह॒स्पति॑श्च मे । \newline
32. च॒ मे॒ मे॒ च॒ च॒ म॒ इन्द्र॒ इन्द्रो॑ मे च च म॒ इन्द्रः॑ । \newline
33. म॒ इन्द्र॒ इन्द्रो॑ मे म॒ इन्द्र॑श्च॒ चेन्द्रो॑ मे म॒ इन्द्र॑श्च । \newline
34. इन्द्र॑श्च॒ चेन्द्र॒ इन्द्र॑श्च मे मे॒ चेन्द्र॒ इन्द्र॑श्च मे । \newline
35. च॒ मे॒ मे॒ च॒ च॒ मे॒ मि॒त्रो मि॒त्रो मे॑ च च मे मि॒त्रः । \newline
36. मे॒ मि॒त्रो मि॒त्रो मे॑ मे मि॒त्रश्च॑ च मि॒त्रो मे॑ मे मि॒त्रश्च॑ । \newline
37. मि॒त्रश्च॑ च मि॒त्रो मि॒त्रश्च॑ मे मे च मि॒त्रो मि॒त्रश्च॑ मे । \newline
38. च॒ मे॒ मे॒ च॒ च॒ म॒ इन्द्र॒ इन्द्रो॑ मे च च म॒ इन्द्रः॑ । \newline
39. म॒ इन्द्र॒ इन्द्रो॑ मे म॒ इन्द्र॑श्च॒ चेन्द्रो॑ मे म॒ इन्द्र॑श्च । \newline
40. इन्द्र॑श्च॒ चेन्द्र॒ इन्द्र॑श्च मे मे॒ चेन्द्र॒ इन्द्र॑श्च मे । \newline
41. च॒ मे॒ मे॒ च॒ च॒ मे॒ वरु॑णो॒ वरु॑णो मे च च मे॒ वरु॑णः । \newline
42. मे॒ वरु॑णो॒ वरु॑णो मे मे॒ वरु॑णश्च च॒ वरु॑णो मे मे॒ वरु॑णश्च । \newline
43. वरु॑णश्च च॒ वरु॑णो॒ वरु॑णश्च मे मे च॒ वरु॑णो॒ वरु॑णश्च मे । \newline
44. च॒ मे॒ मे॒ च॒ च॒ म॒ इन्द्र॒ इन्द्रो॑ मे च च म॒ इन्द्रः॑ । \newline
45. म॒ इन्द्र॒ इन्द्रो॑ मे म॒ इन्द्र॑श्च॒ चेन्द्रो॑ मे म॒ इन्द्र॑श्च । \newline
46. इन्द्र॑श्च॒ चेन्द्र॒ इन्द्र॑श्च मे मे॒ चेन्द्र॒ इन्द्र॑श्च मे । \newline
47. च॒ मे॒ मे॒ च॒ च॒ मे॒ त्वष्टा॒ त्वष्टा॑ मे च च मे॒ त्वष्टा᳚ । \newline
48. मे॒ त्वष्टा॒ त्वष्टा॑ मे मे॒ त्वष्टा॑ च च॒ त्वष्टा॑ मे मे॒ त्वष्टा॑ च । \newline
49. त्वष्टा॑ च च॒ त्वष्टा॒ त्वष्टा॑ च मे मे च॒ त्वष्टा॒ त्वष्टा॑ च मे । \newline
50. च॒ मे॒ मे॒ च॒ च॒ म॒ इन्द्र॒ इन्द्रो॑ मे च च म॒ इन्द्रः॑ । \newline
\pagebreak
\markright{ TS 4.7.6.2  \hfill https://www.vedavms.in \hfill}

\section{ TS 4.7.6.2 }

\textbf{TS 4.7.6.2 } \newline
\textbf{Samhita Paata} \newline

म॒ इन्द्र॑श्च मे धा॒ता च॑ म॒ इन्द्र॑श्च मे॒ विष्णु॑श्च म॒ इन्द्र॑श्च मे॒ ऽश्विनौ॑ च म॒ इन्द्र॑श्च मे म॒रुत॑श्च म॒ इन्द्र॑श्च मे॒ विश्वे॑ च मे दे॒वा इन्द्र॑श्च मे   पृथि॒वी च॑ म॒ इन्द्र॑श्च मे॒ऽन्तरि॑क्षं च म॒ इन्द्र॑श्च मे॒ द्यौश्च॑ म॒ ( ) इन्द्र॑श्च मे॒ दिश॑श्च म॒ इन्द्र॑श्च मे                मू॒र्द्धा च॑ म॒ इन्द्र॑श्च मे प्र॒जाप॑तिश्च म॒ इन्द्र॑श्च मे ॥ \newline

\textbf{Pada Paata} \newline

मे॒ । इन्द्रः॑ । च॒ । मे॒ । धा॒ता । च॒ । मे॒ । इन्द्रः॑ । च॒ । मे॒ । विष्णुः॑ । च॒ । मे॒ । इन्द्रः॑ । च॒ । मे॒ । अ॒श्विनौ᳚ । च॒ । मे॒ । इन्द्रः॑ । च॒ । मे॒ । म॒रुतः॑ । च॒ । मे॒ । इन्द्रः॑ । च॒ । मे॒ । विश्वे᳚ । च॒ । मे॒ । दे॒वाः । इन्द्रः॑ । च॒ । मे॒ । पृ॒थि॒वी । च॒ । मे॒ । इन्द्रः॑ । च॒ । मे॒ । अ॒न्तरि॑क्षम् । च॒ । मे॒ । इन्द्रः॑ । च॒ । मे॒ । द्यौः । च॒ । मे॒ ( ) । इन्द्रः॑ । च॒ । मे॒ । दिशः॑ । च॒ । मे॒ । इन्द्रः॑ । च॒ । मे॒ । मू॒र्द्धा । च॒ । मे॒ । इन्द्रः॑ । च॒ । मे॒ । प्र॒जाप॑ति॒रिति॑ प्र॒जा - प॒तिः॒ । च॒ । मे॒ । इन्द्रः॑ । च॒ । मे॒ ॥  \newline


\textbf{Krama Paata} \newline

म॒ इन्द्रः॑ । इन्द्र॑श्च । च॒ मे॒ । मे॒ धा॒ता । धा॒ता च॑ । च॒ मे॒ । म॒ इन्द्रः॑ । इन्द्र॑श्च । च॒ मे॒ । मे॒ विष्णुः॑ । विष्णु॑श्च । च॒ मे॒ । म॒ इन्द्रः॑ । इन्द्र॑श्च । च॒ मे॒ । मे॒ऽश्विनौ᳚ । अ॒श्विनौ॑ च । च॒ मे॒ । म॒ इन्द्रः॑ । इन्द्र॑श्च । च॒ मे॒ । मे॒ म॒रुतः॑ । म॒रुत॑श्च । च॒ मे॒ । म॒ इन्द्रः॑ । इन्द्र॑श्च । च॒ मे॒ । मे॒ विश्वे᳚ । विश्वे॑ च । च॒ मे॒ । मे॒ दे॒वाः । दे॒वा इन्द्रः॑ । इन्द्र॑श्च । च॒ मे॒ । मे॒ पृ॒थि॒वी । पृ॒थि॒वी च॑ । च॒ मे॒ । म॒ इन्द्रः॑ । इन्द्र॑श्च । च॒ मे॒ । मे॒ऽन्तरि॑क्षम् । अ॒न्तरि॑क्षम् च । च॒ मे॒ । म॒ इन्द्रः॑ । इन्द्र॑श्च । च॒ मे॒ । मे॒ द्यौः । द्यौश्च॑ । च॒ मे॒ ( ) । म॒ इन्द्रः॑ । इन्द्र॑श्च । च॒ मे॒ । मे॒ दिशः॑ । दिश॑श्च । च॒ मे॒ । म॒ इन्द्रः॑ । इन्द्र॑श्च । च॒ मे॒ । मे॒ मू॒र्द्धा । मू॒र्द्धा च॑ । च॒ मे॒ । म॒ इन्द्रः॑ । इन्द्र॑श्च । च॒ मे॒ । मे॒ प्र॒जाप॑तिः । प्र॒जाप॑तिश्च । प्र॒जाप॑ति॒रिति॑ प्र॒जा - प॒तिः॒ । च॒ मे॒ । म॒ इन्द्रः॑ । इन्द्र॑श्च । च॒ मे॒ । म॒ इति॑ मे । \newline

\textbf{Jatai Paata} \newline

1. म॒ इन्द्र॒ इन्द्रो॑ मे म॒ इन्द्रः॑ । \newline
2. इन्द्र॑श्च॒ चेन्द्र॒ इन्द्र॑श्च । \newline
3. च॒ मे॒ मे॒ च॒ च॒ मे॒ । \newline
4. मे॒ धा॒ता धा॒ता मे॑ मे धा॒ता । \newline
5. धा॒ता च॑ च धा॒ता धा॒ता च॑ । \newline
6. च॒ मे॒ मे॒ च॒ च॒ मे॒ । \newline
7. म॒ इन्द्र॒ इन्द्रो॑ मे म॒ इन्द्रः॑ । \newline
8. इन्द्र॑श्च॒ चेन्द्र॒ इन्द्र॑श्च । \newline
9. च॒ मे॒ मे॒ च॒ च॒ मे॒ । \newline
10. मे॒ विष्णु॒र् विष्णु॑र् मे मे॒ विष्णुः॑ । \newline
11. विष्णु॑श्च च॒ विष्णु॒र् विष्णु॑श्च । \newline
12. च॒ मे॒ मे॒ च॒ च॒ मे॒ । \newline
13. म॒ इन्द्र॒ इन्द्रो॑ मे म॒ इन्द्रः॑ । \newline
14. इन्द्र॑श्च॒ चेन्द्र॒ इन्द्र॑श्च । \newline
15. च॒ मे॒ मे॒ च॒ च॒ मे॒ । \newline
16. मे॒ ऽश्विना॑ व॒श्विनौ॑ मे मे॒ ऽश्विनौ᳚ । \newline
17. अ॒श्विनौ॑ च चा॒श्विना॑ व॒श्विनौ॑ च । \newline
18. च॒ मे॒ मे॒ च॒ च॒ मे॒ । \newline
19. म॒ इन्द्र॒ इन्द्रो॑ मे म॒ इन्द्रः॑ । \newline
20. इन्द्र॑श्च॒ चेन्द्र॒ इन्द्र॑श्च । \newline
21. च॒ मे॒ मे॒ च॒ च॒ मे॒ । \newline
22. मे॒ म॒रुतो॑ म॒रुतो॑ मे मे म॒रुतः॑ । \newline
23. म॒रुत॑श्च च म॒रुतो॑ म॒रुत॑श्च । \newline
24. च॒ मे॒ मे॒ च॒ च॒ मे॒ । \newline
25. म॒ इन्द्र॒ इन्द्रो॑ मे म॒ इन्द्रः॑ । \newline
26. इन्द्र॑श्च॒ चेन्द्र॒ इन्द्र॑श्च । \newline
27. च॒ मे॒ मे॒ च॒ च॒ मे॒ । \newline
28. मे॒ विश्वे॒ विश्वे॑ मे मे॒ विश्वे᳚ । \newline
29. विश्वे॑ च च॒ विश्वे॒ विश्वे॑ च । \newline
30. च॒ मे॒ मे॒ च॒ च॒ मे॒ । \newline
31. मे॒ दे॒वा दे॒वा मे॑ मे दे॒वाः । \newline
32. दे॒वा इन्द्र॒ इन्द्रो॑ दे॒वा दे॒वा इन्द्रः॑ । \newline
33. इन्द्र॑श्च॒ चेन्द्र॒ इन्द्र॑श्च । \newline
34. च॒ मे॒ मे॒ च॒ च॒ मे॒ । \newline
35. मे॒ पृ॒थि॒वी पृ॑थि॒वी मे॑ मे पृथि॒वी । \newline
36. पृ॒थि॒वी च॑ च पृथि॒वी पृ॑थि॒वी च॑ । \newline
37. च॒ मे॒ मे॒ च॒ च॒ मे॒ । \newline
38. म॒ इन्द्र॒ इन्द्रो॑ मे म॒ इन्द्रः॑ । \newline
39. इन्द्र॑श्च॒ चेन्द्र॒ इन्द्र॑श्च । \newline
40. च॒ मे॒ मे॒ च॒ च॒ मे॒ । \newline
41. मे॒ ऽन्तरि॑क्ष म॒न्तरि॑क्षम् मे मे॒ ऽन्तरि॑क्षम् । \newline
42. अ॒न्तरि॑क्षम् च चा॒न्तरि॑क्ष म॒न्तरि॑क्षम् च । \newline
43. च॒ मे॒ मे॒ च॒ च॒ मे॒ । \newline
44. म॒ इन्द्र॒ इन्द्रो॑ मे म॒ इन्द्रः॑ । \newline
45. इन्द्र॑श्च॒ चेन्द्र॒ इन्द्र॑श्च । \newline
46. च॒ मे॒ मे॒ च॒ च॒ मे॒ । \newline
47. मे॒ द्यौर् द्यौर् मे॑ मे॒ द्यौः । \newline
48. द्यौश्च॑ च॒ द्यौर् द्यौश्च॑ । \newline
49. च॒ मे॒ मे॒ च॒ च॒ मे॒ । \newline
50. म॒ इन्द्र॒ इन्द्रो॑ मे म॒ इन्द्रः॑ । \newline
51. इन्द्र॑श्च॒ चेन्द्र॒ इन्द्र॑श्च । \newline
52. च॒ मे॒ मे॒ च॒ च॒ मे॒ । \newline
53. मे॒ दिशो॒ दिशो॑ मे मे॒ दिशः॑ । \newline
54. दिश॑श्च च॒ दिशो॒ दिश॑श्च । \newline
55. च॒ मे॒ मे॒ च॒ च॒ मे॒ । \newline
56. म॒ इन्द्र॒ इन्द्रो॑ मे म॒ इन्द्रः॑ । \newline
57. इन्द्र॑श्च॒ चे न्द्र॒ इन्द्र॑श्च । \newline
58. च॒ मे॒ मे॒ च॒ च॒ मे॒ । \newline
59. मे॒ मू॒र्द्धा मू॒र्द्धा मे॑ मे मू॒र्द्धा । \newline
60. मू॒र्द्धा च॑ च मू॒र्द्धा मू॒र्द्धा च॑ । \newline
61. च॒ मे॒ मे॒ च॒ च॒ मे॒ । \newline
62. म॒ इन्द्र॒ इन्द्रो॑ मे म॒ इन्द्रः॑ । \newline
63. इन्द्र॑श्च॒ चेन्द्र॒ इन्द्र॑श्च । \newline
64. च॒ मे॒ मे॒ च॒ च॒ मे॒ । \newline
65. मे॒ प्र॒जाप॑तिः प्र॒जाप॑तिर् मे मे प्र॒जाप॑तिः । \newline
66. प्र॒जाप॑तिश्च च प्र॒जाप॑तिः प्र॒जाप॑तिश्च । \newline
67. प्र॒जाप॑ति॒रिति॑ प्र॒जा - प॒तिः॒ । \newline
68. च॒ मे॒ मे॒ च॒ च॒ मे॒ । \newline
69. म॒ इन्द्र॒ इन्द्रो॑ मे म॒ इन्द्रः॑ । \newline
70. इन्द्र॑श्च॒ चेन्द्र॒ इन्द्र॑श्च । \newline
71. च॒ मे॒ मे॒ च॒ च॒ मे॒ । \newline
72. म॒ इति॑ मे । \newline

\textbf{Ghana Paata } \newline

1. म॒ इन्द्र॒ इन्द्रो॑ मे म॒ इन्द्र॑श्च॒ चेन्द्रो॑ मे म॒ इन्द्र॑श्च । \newline
2. इन्द्र॑श्च॒ चेन्द्र॒ इन्द्र॑श्च मे मे॒ चेन्द्र॒ इन्द्र॑श्च मे । \newline
3. च॒ मे॒ मे॒ च॒ च॒ मे॒ धा॒ता धा॒ता मे॑ च च मे धा॒ता । \newline
4. मे॒ धा॒ता धा॒ता मे॑ मे धा॒ता च॑ च धा॒ता मे॑ मे धा॒ता च॑ । \newline
5. धा॒ता च॑ च धा॒ता धा॒ता च॑ मे मे च धा॒ता धा॒ता च॑ मे । \newline
6. च॒ मे॒ मे॒ च॒ च॒ म॒ इन्द्र॒ इन्द्रो॑ मे च च म॒ इन्द्रः॑ । \newline
7. म॒ इन्द्र॒ इन्द्रो॑ मे म॒ इन्द्र॑श्च॒ चेन्द्रो॑ मे म॒ इन्द्र॑श्च । \newline
8. इन्द्र॑श्च॒ चेन्द्र॒ इन्द्र॑श्च मे मे॒ चेन्द्र॒ इन्द्र॑श्च मे । \newline
9. च॒ मे॒ मे॒ च॒ च॒ मे॒ विष्णु॒र् विष्णु॑र् मे च च मे॒ विष्णुः॑ । \newline
10. मे॒ विष्णु॒र् विष्णु॑र् मे मे॒ विष्णु॑श्च च॒ विष्णु॑र् मे मे॒ विष्णु॑श्च । \newline
11. विष्णु॑श्च च॒ विष्णु॒र् विष्णु॑श्च मे मे च॒ विष्णु॒र् विष्णु॑श्च मे । \newline
12. च॒ मे॒ मे॒ च॒ च॒ म॒ इन्द्र॒ इन्द्रो॑ मे च च म॒ इन्द्रः॑ । \newline
13. म॒ इन्द्र॒ इन्द्रो॑ मे म॒ इन्द्र॑श्च॒ चेन्द्रो॑ मे म॒ इन्द्र॑श्च । \newline
14. इन्द्र॑श्च॒ चेन्द्र॒ इन्द्र॑श्च मे मे॒ चेन्द्र॒ इन्द्र॑श्च मे । \newline
15. च॒ मे॒ मे॒ च॒ च॒ मे॒ ऽश्विना॑ व॒श्विनौ॑ मे च च मे॒ ऽश्विनौ᳚ । \newline
16. मे॒ ऽश्विना॑ व॒श्विनौ॑ मे मे॒ ऽश्विनौ॑ च चा॒श्विनौ॑ मे मे॒ ऽश्विनौ॑ च । \newline
17. अ॒श्विनौ॑ च चा॒श्विना॑ व॒श्विनौ॑ च मे मे चा॒श्विना॑ व॒श्विनौ॑ च मे । \newline
18. च॒ मे॒ मे॒ च॒ च॒ म॒ इन्द्र॒ इन्द्रो॑ मे च च म॒ इन्द्रः॑ । \newline
19. म॒ इन्द्र॒ इन्द्रो॑ मे म॒ इन्द्र॑श्च॒ चेन्द्रो॑ मे म॒ इन्द्र॑श्च । \newline
20. इन्द्र॑श्च॒ चेन्द्र॒ इन्द्र॑श्च मे मे॒ चेन्द्र॒ इन्द्र॑श्च मे । \newline
21. च॒ मे॒ मे॒ च॒ च॒ मे॒ म॒रुतो॑ म॒रुतो॑ मे च च मे म॒रुतः॑ । \newline
22. मे॒ म॒रुतो॑ म॒रुतो॑ मे मे म॒रुत॑श्च च म॒रुतो॑ मे मे म॒रुत॑श्च । \newline
23. म॒रुत॑श्च च म॒रुतो॑ म॒रुत॑श्च मे मे च म॒रुतो॑ म॒रुत॑श्च मे । \newline
24. च॒ मे॒ मे॒ च॒ च॒ म॒ इन्द्र॒ इन्द्रो॑ मे च च म॒ इन्द्रः॑ । \newline
25. म॒ इन्द्र॒ इन्द्रो॑ मे म॒ इन्द्र॑श्च॒ चेन्द्रो॑ मे म॒ इन्द्र॑श्च । \newline
26. इन्द्र॑श्च॒ चेन्द्र॒ इन्द्र॑श्च मे मे॒ चेन्द्र॒ इन्द्र॑श्च मे । \newline
27. च॒ मे॒ मे॒ च॒ च॒ मे॒ विश्वे॒ विश्वे॑ मे च च मे॒ विश्वे᳚ । \newline
28. मे॒ विश्वे॒ विश्वे॑ मे मे॒ विश्वे॑ च च॒ विश्वे॑ मे मे॒ विश्वे॑ च । \newline
29. विश्वे॑ च च॒ विश्वे॒ विश्वे॑ च मे मे च॒ विश्वे॒ विश्वे॑ च मे । \newline
30. च॒ मे॒ मे॒ च॒ च॒ मे॒ दे॒वा दे॒वा मे॑ च च मे दे॒वाः । \newline
31. मे॒ दे॒वा दे॒वा मे॑ मे दे॒वा इन्द्र॒ इन्द्रो॑ दे॒वा मे॑ मे दे॒वा इन्द्रः॑ । \newline
32. दे॒वा इन्द्र॒ इन्द्रो॑ दे॒वा दे॒वा इन्द्र॑श्च॒ चेन्द्रो॑ दे॒वा दे॒वा इन्द्र॑श्च । \newline
33. इन्द्र॑श्च॒ चेन्द्र॒ इन्द्र॑श्च मे मे॒ चेन्द्र॒ इन्द्र॑श्च मे । \newline
34. च॒ मे॒ मे॒ च॒ च॒ मे॒ पृ॒थि॒वी पृ॑थि॒वी मे॑ च च मे पृथि॒वी । \newline
35. मे॒ पृ॒थि॒वी पृ॑थि॒वी मे॑ मे पृथि॒वी च॑ च पृथि॒वी मे॑ मे पृथि॒वी च॑ । \newline
36. पृ॒थि॒वी च॑ च पृथि॒वी पृ॑थि॒वी च॑ मे मे च पृथि॒वी पृ॑थि॒वी च॑ मे । \newline
37. च॒ मे॒ मे॒ च॒ च॒ म॒ इन्द्र॒ इन्द्रो॑ मे च च म॒ इन्द्रः॑ । \newline
38. म॒ इन्द्र॒ इन्द्रो॑ मे म॒ इन्द्र॑श्च॒ चेन्द्रो॑ मे म॒ इन्द्र॑श्च । \newline
39. इन्द्र॑श्च॒ चेन्द्र॒ इन्द्र॑श्च मे मे॒ चेन्द्र॒ इन्द्र॑श्च मे । \newline
40. च॒ मे॒ मे॒ च॒ च॒ मे॒ ऽन्तरि॑क्ष म॒न्तरि॑क्षम् मे च च मे॒ ऽन्तरि॑क्षम् । \newline
41. मे॒ ऽन्तरि॑क्ष म॒न्तरि॑क्षम् मे मे॒ ऽन्तरि॑क्षम् च चा॒न्तरि॑क्षम् मे मे॒ ऽन्तरि॑क्षम् च । \newline
42. अ॒न्तरि॑क्षम् च चा॒न्तरि॑क्ष म॒न्तरि॑क्षम् च मे मे चा॒न्तरि॑क्ष म॒न्तरि॑क्षम् च मे । \newline
43. च॒ मे॒ मे॒ च॒ च॒ म॒ इन्द्र॒ इन्द्रो॑ मे च च म॒ इन्द्रः॑ । \newline
44. म॒ इन्द्र॒ इन्द्रो॑ मे म॒ इन्द्र॑श्च॒ चेन्द्रो॑ मे म॒ इन्द्र॑श्च । \newline
45. इन्द्र॑श्च॒ चेन्द्र॒ इन्द्र॑श्च मे मे॒ चेन्द्र॒ इन्द्र॑श्च मे । \newline
46. च॒ मे॒ मे॒ च॒ च॒ मे॒ द्यौर् द्यौर् मे॑ च च मे॒ द्यौः । \newline
47. मे॒ द्यौर् द्यौर् मे॑ मे॒ द्यौश्च॑ च॒ द्यौर् मे॑ मे॒ द्यौश्च॑ । \newline
48. द्यौश्च॑ च॒ द्यौर् द्यौश्च॑ मे मे च॒ द्यौर् द्यौश्च॑ मे । \newline
49. च॒ मे॒ मे॒ च॒ च॒ म॒ इन्द्र॒ इन्द्रो॑ मे च च म॒ इन्द्रः॑ । \newline
50. म॒ इन्द्र॒ इन्द्रो॑ मे म॒ इन्द्र॑श्च॒ चेन्द्रो॑ मे म॒ इन्द्र॑श्च । \newline
51. इन्द्र॑श्च॒ चेन्द्र॒ इन्द्र॑श्च मे मे॒ चेन्द्र॒ इन्द्र॑श्च मे । \newline
52. च॒ मे॒ मे॒ च॒ च॒ मे॒ दिशो॒ दिशो॑ मे च च मे॒ दिशः॑ । \newline
53. मे॒ दिशो॒ दिशो॑ मे मे॒ दिश॑श्च च॒ दिशो॑ मे मे॒ दिश॑श्च । \newline
54. दिश॑श्च च॒ दिशो॒ दिश॑श्च मे मे च॒ दिशो॒ दिश॑श्च मे । \newline
55. च॒ मे॒ मे॒ च॒ च॒ म॒ इन्द्र॒ इन्द्रो॑ मे च च म॒ इन्द्रः॑ । \newline
56. म॒ इन्द्र॒ इन्द्रो॑ मे म॒ इन्द्र॑श्च॒ चेन्द्रो॑ मे म॒ इन्द्र॑श्च । \newline
57. इन्द्र॑श्च॒ चेन्द्र॒ इन्द्र॑श्च मे मे॒ चेन्द्र॒ इन्द्र॑श्च मे । \newline
58. च॒ मे॒ मे॒ च॒ च॒ मे॒ मू॒र्द्धा मू॒र्द्धा मे॑ च च मे मू॒र्द्धा । \newline
59. मे॒ मू॒र्द्धा मू॒र्द्धा मे॑ मे मू॒र्द्धा च॑ च मू॒र्द्धा मे॑ मे मू॒र्द्धा च॑ । \newline
60. मू॒र्द्धा च॑ च मू॒र्द्धा मू॒र्द्धा च॑ मे मे च मू॒र्द्धा मू॒र्द्धा च॑ मे । \newline
61. च॒ मे॒ मे॒ च॒ च॒ म॒ इन्द्र॒ इन्द्रो॑ मे च च म॒ इन्द्रः॑ । \newline
62. म॒ इन्द्र॒ इन्द्रो॑ मे म॒ इन्द्र॑श्च॒ चेन्द्रो॑ मे म॒ इन्द्र॑श्च । \newline
63. इन्द्र॑श्च॒ चेन्द्र॒ इन्द्र॑श्च मे मे॒ चेन्द्र॒ इन्द्र॑श्च मे । \newline
64. च॒ मे॒ मे॒ च॒ च॒ मे॒ प्र॒जाप॑तिः प्र॒जाप॑तिर् मे च च मे प्र॒जाप॑तिः । \newline
65. मे॒ प्र॒जाप॑तिः प्र॒जाप॑तिर् मे मे प्र॒जाप॑तिश्च च प्र॒जाप॑तिर् मे मे प्र॒जाप॑तिश्च । \newline
66. प्र॒जाप॑तिश्च च प्र॒जाप॑तिः प्र॒जाप॑तिश्च मे मे च प्र॒जाप॑तिः प्र॒जाप॑तिश्च मे । \newline
67. प्र॒जाप॑ति॒रिति॑ प्र॒जा - प॒तिः॒ । \newline
68. च॒ मे॒ मे॒ च॒ च॒ म॒ इन्द्र॒ इन्द्रो॑ मे च च म॒ इन्द्रः॑ । \newline
69. म॒ इन्द्र॒ इन्द्रो॑ मे म॒ इन्द्र॑श्च॒ चेन्द्रो॑ मे म॒ इन्द्र॑श्च । \newline
70. इन्द्र॑श्च॒ चेन्द्र॒ इन्द्र॑श्च मे मे॒ चेन्द्र॒ इन्द्र॑श्च मे । \newline
71. च॒ मे॒ मे॒ च॒ च॒ मे॒ । \newline
72. म॒ इति॑ मे । \newline
\pagebreak
\markright{ TS 4.7.7.1  \hfill https://www.vedavms.in \hfill}

\section{ TS 4.7.7.1 }

\textbf{TS 4.7.7.1 } \newline
\textbf{Samhita Paata} \newline

अꣳ॒॒शुश्च॑ मे र॒श्मिश्च॒ मे ऽदा᳚भ्यश्च॒ मेऽधि॑पतिश्च मउपाꣳ॒॒शुश्च॑ मे ऽन्तर्या॒मश्च॑ म ऐन्द्रवाय॒वश्च॑ मे मैत्रावरु॒णश्च॑ म आश्वि॒नश्च॑ मे प्रतिप्र॒स्थान॑श्च मे शु॒क्रश्च॑ मे म॒न्थी च॑ म आग्रय॒णश्च॑ मे वैश्वदे॒वश्च॑ मे ध्रु॒वश्च॑ मे वैश्वान॒रश्च॑ म ऋतुग्र॒हाश्च॑- [  ] \newline

\textbf{Pada Paata} \newline

अꣳ॒॒शुः । च॒ । मे॒ । र॒श्मिः । च॒ । मे॒ । अदा᳚भ्यः । च॒ । मे॒ । अधि॑पति॒रित्यधि॑ - प॒तिः॒ । च॒ । मे॒ । उ॒पाꣳ॒॒शुरित्यु॑प-अꣳ॒॒शुः । च॒ । मे॒ । अ॒न्त॒र्या॒म इत्य॑न्तः-या॒मः । च॒ । मे॒ । ऐ॒न्द्र॒वा॒य॒व इत्यै᳚न्द्र-वा॒य॒वः । च॒ । मे॒ । मै॒त्रा॒व॒रु॒ण इति॑ मैत्रा - व॒रु॒णः । च॒ । मे॒ । आ॒श्वि॒नः । च॒ । मे॒ । प्र॒ति॒प्र॒स्थान॒ इति॑ प्रति - प्र॒स्थानः॑ । च॒ । मे॒ । शु॒क्रः । च॒ । मे॒ । म॒न्थी । च॒ । मे॒ । आ॒ग्र॒य॒णः । च॒ । मे॒ । वै॒श्व॒दे॒व इति॑ वैश्व - दे॒वः । च॒ । मे॒ । ध्रु॒वः । च॒ । मे॒ । वै॒श्वा॒न॒रः । च॒ । मे॒ । ऋ॒तु॒ग्र॒हा इत्यृ॑तु - ग्र॒हाः । च॒ ।  \newline


\textbf{Krama Paata} \newline

अꣳ॒॒शुश्च॑ । च॒ मे॒ । मे॒ र॒श्मिः । र॒श्मिश्च॑ । च॒ मे॒ । मेऽदा᳚भ्यः । अदा᳚भ्यश्च । च॒ मे॒ । मेऽधि॑पतिः । अधि॑पतिश्च । अधि॑पति॒रित्यधि॑ - प॒तिः॒ । च॒ मे॒ । म॒ उ॒पाꣳ॒॒शुः । उ॒पाꣳ॒॒शुश्च॑ । उ॒पाꣳ॒॒शुरित्यु॑प - अꣳ॒॒शुः । च॒ मे॒ । मे॒ऽन्त॒र्या॒मः । अ॒न्त॒र्या॒मश्च॑ । अ॒न्त॒र्या॒म इत्य॑न्तः - या॒मः । च॒ मे॒ । म॒ ऐ॒न्द्र॒वा॒य॒वः । ऐ॒न्द्र॒वा॒य॒वश्च॑ । 
ऐ॒न्द्र॒वा॒य॒व इत्यै᳚न्द्र - वा॒य॒वः । च॒ मे॒ । मे॒ मै॒त्रा॒व॒रु॒णः । मै॒त्रा॒व॒रु॒णश्च॑ । मै॒त्रा॒व॒रु॒ण इति॑ मैत्रा - व॒रु॒णः । 
च॒ मे॒ । म॒ आ॒श्वि॒नः । आ॒श्वि॒नश्च॑ । च॒ मे॒ । 
मे॒ प्र॒ति॒प्र॒स्थानः॑ । प्र॒ति॒प्र॒स्थान॑श्च । प्र॒ति॒प्र॒स्थान॒ इति॑ प्रति - प्र॒स्थानः॑ । च॒ मे॒ । मे॒ शु॒क्रः । शु॒क्रश्च॑ । च॒ मे॒ । मे॒ म॒न्थी । म॒न्थी च॑ । च॒ मे॒ । म॒ आ॒ग्र॒य॒णः । आ॒ग्र॒य॒णश्च॑ । च॒ मे॒ । मे॒ वै॒श्व॒दे॒वः । वै॒श्व॒दे॒वश्च॑ । वै॒श्व॒दे॒व इति॑ वैश्व - दे॒वः । च॒ मे॒ । मे॒ ध्रु॒वः । ध्रु॒वश्च॑ । च॒ मे॒ । मे॒ वै॒श्वा॒न॒रः । वै॒श्वा॒न॒रश्च॑ । च॒ मे॒ । म॒ ऋ॒तु॒ग्र॒हाः । ऋ॒तु॒ग्र॒हाश्च॑ ( ) । ऋ॒तु॒ग्र॒हा इत्यृ॑तु - ग्र॒हाः । च॒ मे॒ \newline

\textbf{Jatai Paata} \newline

1. अ॒(ग्म्॒)शुश्च॑ चा॒(ग्म्॒)शु र॒(ग्म्॒)शुश्च॑ । \newline
2. च॒ मे॒ मे॒ च॒ च॒ मे॒ । \newline
3. मे॒ र॒श्मी र॒श्मिर् मे॑ मे र॒श्मिः । \newline
4. र॒श्मिश्च॑ च र॒श्मी र॒श्मिश्च॑ । \newline
5. च॒ मे॒ मे॒ च॒ च॒ मे॒ । \newline
6. मे ऽदा॒भ्यो ऽदा᳚भ्यो मे॒ मे ऽदा᳚भ्यः । \newline
7. अदा᳚भ्यश्च॒ चादा॒भ्यो ऽदा᳚भ्यश्च । \newline
8. च॒ मे॒ मे॒ च॒ च॒ मे॒ । \newline
9. मे ऽधि॑पति॒ रधि॑पतिर् मे॒ मे ऽधि॑पतिः । \newline
10. अधि॑पतिश्च॒ चाधि॑पति॒ रधि॑पतिश्च । \newline
11. अधि॑पति॒रित्यधि॑ - प॒तिः॒ । \newline
12. च॒ मे॒ मे॒ च॒ च॒ मे॒ । \newline
13. म॒ उ॒पा॒(ग्म्॒)शु रु॑पा॒(ग्म्॒)शुर् मे॑ म उपा॒(ग्म्॒)शुः । \newline
14. उ॒पा॒(ग्म्॒)शुश्च॑ चोपा॒(ग्म्॒)शु रु॑पा॒(ग्म्॒)शुश्च॑ । \newline
15. उ॒पा॒(ग्म्॒)शुरित्यु॑प - अ॒(ग्म्॒)शुः । \newline
16. च॒ मे॒ मे॒ च॒ च॒ मे॒ । \newline
17. मे॒ ऽन्त॒र्या॒मो᳚ ऽन्तर्या॒मो मे॑ मे ऽन्तर्या॒मः । \newline
18. अ॒न्त॒र्या॒मश्च॑ चान्तर्या॒मो᳚ ऽन्तर्या॒मश्च॑ । \newline
19. अ॒न्त॒र्या॒म इत्य॑न्तः - या॒मः । \newline
20. च॒ मे॒ मे॒ च॒ च॒ मे॒ । \newline
21. म॒ ऐ॒न्द्र॒वा॒य॒व ऐ᳚न्द्रवाय॒वो मे॑ म ऐन्द्रवाय॒वः । \newline
22. ऐ॒न्द्र॒वा॒य॒वश्च॑ चैन्द्रवाय॒व ऐ᳚न्द्रवाय॒वश्च॑ । \newline
23. ऐ॒न्द्र॒वा॒य॒व इत्यै᳚न्द्र - वा॒य॒वः । \newline
24. च॒ मे॒ मे॒ च॒ च॒ मे॒ । \newline
25. मे॒ मै॒त्रा॒व॒रु॒णो मै᳚त्रावरु॒णो मे॑ मे मैत्रावरु॒णः । \newline
26. मै॒त्रा॒व॒रु॒णश्च॑ च मैत्रावरु॒णो मै᳚त्रावरु॒णश्च॑ । \newline
27. मै॒त्रा॒व॒रु॒ण इति॑ मैत्रा - व॒रु॒णः । \newline
28. च॒ मे॒ मे॒ च॒ च॒ मे॒ । \newline
29. म॒ आ॒श्वि॒न आ᳚श्वि॒नो मे॑ म आश्वि॒नः । \newline
30. आ॒श्वि॒नश्च॑ चाश्वि॒न आ᳚श्वि॒नश्च॑ । \newline
31. च॒ मे॒ मे॒ च॒ च॒ मे॒ । \newline
32. मे॒ प्र॒ति॒प्र॒स्थानः॑ प्रतिप्र॒स्थानो॑ मे मे प्रतिप्र॒स्थानः॑ । \newline
33. प्र॒ति॒प्र॒स्थान॑श्च च प्रतिप्र॒स्थानः॑ प्रतिप्र॒स्थान॑श्च । \newline
34. प्र॒ति॒प्र॒स्थान॒ इति॑ प्रति - प्र॒स्थानः॑ । \newline
35. च॒ मे॒ मे॒ च॒ च॒ मे॒ । \newline
36. मे॒ शु॒क्रः शु॒क्रो मे॑ मे शु॒क्रः । \newline
37. शु॒क्रश्च॑ च शु॒क्रः शु॒क्रश्च॑ । \newline
38. च॒ मे॒ मे॒ च॒ च॒ मे॒ । \newline
39. मे॒ म॒न्थी म॒न्थी मे॑ मे म॒न्थी । \newline
40. म॒न्थी च॑ च म॒न्थी म॒न्थी च॑ । \newline
41. च॒ मे॒ मे॒ च॒ च॒ मे॒ । \newline
42. म॒ आ॒ग्र॒य॒ण आ᳚ग्रय॒णो मे॑ म आग्रय॒णः । \newline
43. आ॒ग्र॒य॒णश्च॑ चाग्रय॒ण आ᳚ग्रय॒णश्च॑ । \newline
44. च॒ मे॒ मे॒ च॒ च॒ मे॒ । \newline
45. मे॒ वै॒श्व॒दे॒वो वै᳚श्वदे॒वो मे॑ मे वैश्वदे॒वः । \newline
46. वै॒श्व॒दे॒वश्च॑ च वैश्वदे॒वो वै᳚श्वदे॒वश्च॑ । \newline
47. वै॒श्व॒दे॒व इति॑ वैश्व - दे॒वः । \newline
48. च॒ मे॒ मे॒ च॒ च॒ मे॒ । \newline
49. मे॒ ध्रु॒वो ध्रु॒वो मे॑ मे ध्रु॒वः । \newline
50. ध्रु॒वश्च॑ च ध्रु॒वो ध्रु॒वश्च॑ । \newline
51. च॒ मे॒ मे॒ च॒ च॒ मे॒ । \newline
52. मे॒ वै॒श्वा॒न॒रो वै᳚श्वान॒रो मे॑ मे वैश्वान॒रः । \newline
53. वै॒श्वा॒न॒रश्च॑ च वैश्वान॒रो वै᳚श्वान॒रश्च॑ । \newline
54. च॒ मे॒ मे॒ च॒ च॒ मे॒ । \newline
55. म॒ ऋ॒तु॒ग्र॒हा ऋ॑तुग्र॒हा मे॑ म ऋतुग्र॒हाः । \newline
56. ऋ॒तु॒ग्र॒हाश्च॑ च र्‌तुग्र॒हा ऋ॑तुग्र॒हाश्च॑ । \newline
57. ऋ॒तु॒ग्र॒हा इत्यृ॑तु - ग्र॒हाः । \newline
58. च॒ मे॒ मे॒ च॒ च॒ मे॒ । \newline

\textbf{Ghana Paata } \newline

1. अ॒(ग्म्॒)शुश्च॑ चा॒(ग्म्॒)शु र॒(ग्म्॒)शुश्च॑ मे मे चा॒(ग्म्॒)शु र॒(ग्म्॒)शुश्च॑ मे । \newline
2. च॒ मे॒ मे॒ च॒ च॒ मे॒ र॒श्मी र॒श्मिर् मे॑ च च मे र॒श्मिः । \newline
3. मे॒ र॒श्मी र॒श्मिर् मे॑ मे र॒श्मिश्च॑ च र॒श्मिर् मे॑ मे र॒श्मिश्च॑ । \newline
4. र॒श्मिश्च॑ च र॒श्मी र॒श्मिश्च॑ मे मे च र॒श्मी र॒श्मिश्च॑ मे । \newline
5. च॒ मे॒ मे॒ च॒ च॒ मे ऽदा॒भ्यो ऽदा᳚भ्यो मे च च॒ मे ऽदा᳚भ्यः । \newline
6. मे ऽदा॒भ्यो ऽदा᳚भ्यो मे॒ मे ऽदा᳚भ्यश्च॒ चादा᳚भ्यो मे॒ मे ऽदा᳚भ्यश्च । \newline
7. अदा᳚भ्यश्च॒ चादा॒भ्यो ऽदा᳚भ्यश्च मे मे॒ चादा॒भ्यो ऽदा᳚भ्यश्च मे । \newline
8. च॒ मे॒ मे॒ च॒ च॒ मे ऽधि॑पति॒ रधि॑पतिर् मे च च॒ मे ऽधि॑पतिः । \newline
9. मे ऽधि॑पति॒ रधि॑पतिर् मे॒ मे ऽधि॑पतिश्च॒ चाधि॑पतिर् मे॒ मे ऽधि॑पतिश्च । \newline
10. अधि॑पतिश्च॒ चाधि॑पति॒ रधि॑पतिश्च मे मे॒ चाधि॑पति॒ रधि॑पतिश्च मे । \newline
11. अधि॑पति॒रित्यधि॑ - प॒तिः॒ । \newline
12. च॒ मे॒ मे॒ च॒ च॒ म॒ उ॒पा॒(ग्म्॒)शु रु॑पा॒(ग्म्॒)शुर् मे॑ च च म उपा॒(ग्म्॒)शुः । \newline
13. म॒ उ॒पा॒(ग्म्॒)शु रु॑पा॒(ग्म्॒)शुर् मे॑ म उपा॒(ग्म्॒)शुश्च॑ चोपा॒(ग्म्॒)शुर् मे॑ म उपा॒(ग्म्॒)शुश्च॑ । \newline
14. उ॒पा॒(ग्म्॒)शुश्च॑ चोपा॒(ग्म्॒)शु रु॑पा॒(ग्म्॒)शुश्च॑ मे मे चोपा॒(ग्म्॒)शु रु॑पा॒(ग्म्॒)शुश्च॑ मे । \newline
15. उ॒पा॒(ग्म्॒)शुरित्यु॑प - अ॒(ग्म्॒)शुः । \newline
16. च॒ मे॒ मे॒ च॒ च॒ मे॒ ऽन्त॒र्या॒मो᳚ ऽन्तर्या॒मो मे॑ च च मे ऽन्तर्या॒मः । \newline
17. मे॒ ऽन्त॒र्या॒मो᳚ ऽन्तर्या॒मो मे॑ मे ऽन्तर्या॒मश्च॑ चान्तर्या॒मो मे॑ मे ऽन्तर्या॒मश्च॑ । \newline
18. अ॒न्त॒र्या॒मश्च॑ चान्तर्या॒मो᳚ ऽन्तर्या॒मश्च॑ मे मे चान्तर्या॒मो᳚ ऽन्तर्या॒मश्च॑ मे । \newline
19. अ॒न्त॒र्या॒म इत्य॑न्तः - या॒मः । \newline
20. च॒ मे॒ मे॒ च॒ च॒ म॒ ऐ॒न्द्र॒वा॒य॒व ऐ᳚न्द्रवाय॒वो मे॑ च च म ऐन्द्रवाय॒वः । \newline
21. म॒ ऐ॒न्द्र॒वा॒य॒व ऐ᳚न्द्रवाय॒वो मे॑ म ऐन्द्रवाय॒वश्च॑ चैन्द्रवाय॒वो मे॑ म ऐन्द्रवाय॒वश्च॑ । \newline
22. ऐ॒न्द्र॒वा॒य॒वश्च॑ चैन्द्रवाय॒व ऐ᳚न्द्रवाय॒वश्च॑ मे मे चैन्द्रवाय॒व ऐ᳚न्द्रवाय॒वश्च॑ मे । \newline
23. ऐ॒न्द्र॒वा॒य॒व इत्यै᳚न्द्र - वा॒य॒वः । \newline
24. च॒ मे॒ मे॒ च॒ च॒ मे॒ मै॒त्रा॒व॒रु॒णो मै᳚त्रावरु॒णो मे॑ च च मे मैत्रावरु॒णः । \newline
25. मे॒ मै॒त्रा॒व॒रु॒णो मै᳚त्रावरु॒णो मे॑ मे मैत्रावरु॒णश्च॑ च मैत्रावरु॒णो मे॑ मे मैत्रावरु॒णश्च॑ । \newline
26. मै॒त्रा॒व॒रु॒णश्च॑ च मैत्रावरु॒णो मै᳚त्रावरु॒णश्च॑ मे मे च मैत्रावरु॒णो मै᳚त्रावरु॒णश्च॑ मे । \newline
27. मै॒त्रा॒व॒रु॒ण इति॑ मैत्रा - व॒रु॒णः । \newline
28. च॒ मे॒ मे॒ च॒ च॒ म॒ आ॒श्वि॒न आ᳚श्वि॒नो मे॑ च च म आश्वि॒नः । \newline
29. म॒ आ॒श्वि॒न आ᳚श्वि॒नो मे॑ म आश्वि॒नश्च॑ चाश्वि॒नो मे॑ म आश्वि॒नश्च॑ । \newline
30. आ॒श्वि॒नश्च॑ चाश्वि॒न आ᳚श्वि॒नश्च॑ मे मे चाश्वि॒न आ᳚श्वि॒नश्च॑ मे । \newline
31. च॒ मे॒ मे॒ च॒ च॒ मे॒ प्र॒ति॒प्र॒स्थानः॑ प्रतिप्र॒स्थानो॑ मे च च मे प्रतिप्र॒स्थानः॑ । \newline
32. मे॒ प्र॒ति॒प्र॒स्थानः॑ प्रतिप्र॒स्थानो॑ मे मे प्रतिप्र॒स्थान॑श्च च प्रतिप्र॒स्थानो॑ मे मे प्रतिप्र॒स्थान॑श्च । \newline
33. प्र॒ति॒प्र॒स्थान॑श्च च प्रतिप्र॒स्थानः॑ प्रतिप्र॒स्थान॑श्च मे मे च प्रतिप्र॒स्थानः॑ प्रतिप्र॒स्थान॑श्च मे । \newline
34. प्र॒ति॒प्र॒स्थान॒ इति॑ प्रति - प्र॒स्थानः॑ । \newline
35. च॒ मे॒ मे॒ च॒ च॒ मे॒ शु॒क्रः शु॒क्रो मे॑ च च मे शु॒क्रः । \newline
36. मे॒ शु॒क्रः शु॒क्रो मे॑ मे शु॒क्रश्च॑ च शु॒क्रो मे॑ मे शु॒क्रश्च॑ । \newline
37. शु॒क्रश्च॑ च शु॒क्रः शु॒क्रश्च॑ मे मे च शु॒क्रः शु॒क्रश्च॑ मे । \newline
38. च॒ मे॒ मे॒ च॒ च॒ मे॒ म॒न्थी म॒न्थी मे॑ च च मे म॒न्थी । \newline
39. मे॒ म॒न्थी म॒न्थी मे॑ मे म॒न्थी च॑ च म॒न्थी मे॑ मे म॒न्थी च॑ । \newline
40. म॒न्थी च॑ च म॒न्थी म॒न्थी च॑ मे मे च म॒न्थी म॒न्थी च॑ मे । \newline
41. च॒ मे॒ मे॒ च॒ च॒ म॒ आ॒ग्र॒य॒ण आ᳚ग्रय॒णो मे॑ च च म आग्रय॒णः । \newline
42. म॒ आ॒ग्र॒य॒ण आ᳚ग्रय॒णो मे॑ म आग्रय॒णश्च॑ चाग्रय॒णो मे॑ म आग्रय॒णश्च॑ । \newline
43. आ॒ग्र॒य॒णश्च॑ चाग्रय॒ण आ᳚ग्रय॒णश्च॑ मे मे चाग्रय॒ण आ᳚ग्रय॒णश्च॑ मे । \newline
44. च॒ मे॒ मे॒ च॒ च॒ मे॒ वै॒श्व॒दे॒वो वै᳚श्वदे॒वो मे॑ च च मे वैश्वदे॒वः । \newline
45. मे॒ वै॒श्व॒दे॒वो वै᳚श्वदे॒वो मे॑ मे वैश्वदे॒वश्च॑ च वैश्वदे॒वो मे॑ मे वैश्वदे॒वश्च॑ । \newline
46. वै॒श्व॒दे॒वश्च॑ च वैश्वदे॒वो वै᳚श्वदे॒वश्च॑ मे मे च वैश्वदे॒वो वै᳚श्वदे॒वश्च॑ मे । \newline
47. वै॒श्व॒दे॒व इति॑ वैश्व - दे॒वः । \newline
48. च॒ मे॒ मे॒ च॒ च॒ मे॒ ध्रु॒वो ध्रु॒वो मे॑ च च मे ध्रु॒वः । \newline
49. मे॒ ध्रु॒वो ध्रु॒वो मे॑ मे ध्रु॒वश्च॑ च ध्रु॒वो मे॑ मे ध्रु॒वश्च॑ । \newline
50. ध्रु॒वश्च॑ च ध्रु॒वो ध्रु॒वश्च॑ मे मे च ध्रु॒वो ध्रु॒वश्च॑ मे । \newline
51. च॒ मे॒ मे॒ च॒ च॒ मे॒ वै॒श्वा॒न॒रो वै᳚श्वान॒रो मे॑ च च मे वैश्वान॒रः । \newline
52. मे॒ वै॒श्वा॒न॒रो वै᳚श्वान॒रो मे॑ मे वैश्वान॒रश्च॑ च वैश्वान॒रो मे॑ मे वैश्वान॒रश्च॑ । \newline
53. वै॒श्वा॒न॒रश्च॑ च वैश्वान॒रो वै᳚श्वान॒रश्च॑ मे मे च वैश्वान॒रो वै᳚श्वान॒रश्च॑ मे । \newline
54. च॒ मे॒ मे॒ च॒ च॒ म॒ ऋ॒तु॒ग्र॒हा ऋ॑तुग्र॒हा मे॑ च च म ऋतुग्र॒हाः । \newline
55. म॒ ऋ॒तु॒ग्र॒हा ऋ॑तुग्र॒हा मे॑ म ऋतुग्र॒हाश्च॑ च र्‌तुग्र॒हा मे॑ म ऋतुग्र॒हाश्च॑ । \newline
56. ऋ॒तु॒ग्र॒हाश्च॑ च र्‌तुग्र॒हा ऋ॑तुग्र॒हाश्च॑ मे मे च र्‌तुग्र॒हा ऋ॑तुग्र॒हाश्च॑ मे । \newline
57. ऋ॒तु॒ग्र॒हा इत्यृ॑तु - ग्र॒हाः । \newline
58. च॒ मे॒ मे॒ च॒ च॒ मे॒ ऽति॒ग्रा॒ह्या॑ अतिग्रा॒ह्या॑ मे च च मे ऽतिग्रा॒ह्याः᳚ । \newline
\pagebreak
\markright{ TS 4.7.7.2  \hfill https://www.vedavms.in \hfill}

\section{ TS 4.7.7.2 }

\textbf{TS 4.7.7.2 } \newline
\textbf{Samhita Paata} \newline

मे ऽतिग्रा॒ह्या᳚श्च म ऐन्द्रा॒ग्नश्च॑ मे वैश्वदे॒वश्च॑ मेमरुत्व॒तीया᳚श्च मे माहे॒न्द्रश्च॑ म आदि॒त्यश्च॑ मे सावि॒त्रश्च॑ मेसारस्व॒तश्च॑ मे पौ॒ष्णश्च॑ मे पात्नीव॒तश्च॑ मे हारियोज॒नश्च॑ मे ॥ \newline

\textbf{Pada Paata} \newline

मे॒ । अ॒ति॒ग्रा॒ह्या᳚ इत्य॑ति-ग्रा॒ह्याः᳚ । च॒ । मे॒ । ऐ॒न्द्रा॒ग्न इत्यै᳚न्द्र-अ॒ग्नः । च॒ । मे॒ । वै॒श्व॒दे॒व इति॑ वैश्व - दे॒वः । च॒ । मे॒ । म॒रु॒त्व॒तीयाः᳚ । च॒ । मे॒ । मा॒हे॒न्द्र इति॑ माहा - इ॒न्द्रः । च॒ । मे॒ । आ॒दि॒त्यः । च॒ । मे॒ । सा॒वि॒त्रः । च॒ । मे॒ । सा॒र॒स्व॒तः । च॒ । मे॒ । पौ॒ष्णः । च॒ । मे॒ । पा॒त्नी॒व॒त इति॑ पात्नी-व॒तः । च॒ । मे॒ । हा॒रि॒यो॒ज॒न इति॑ हारि-यो॒ज॒नः । च॒ । मे॒ ॥  \newline


\textbf{Krama Paata} \newline

मे॒ऽति॒ग्रा॒ह्याः᳚ । अ॒ति॒ग्रा॒ह्या᳚श्च । अ॒ति॒ग्रा॒ह्या॑ इत्य॑ति - ग्रा॒ह्याः᳚ । च॒ मे॒ । म॒ ऐ॒न्द्रा॒ग्नः । ऐ॒न्द्रा॒ग्नश्च॑ । ऐ॒न्द्रा॒ग्न इत्यै᳚न्द्र - अ॒ग्नः । च॒ मे॒ । मे॒ वै॒श्व॒दे॒वः । वै॒श्व॒दे॒वश्च॑ । वै॒श्व॒दे॒व इति॑ वैश्व - दे॒वः । च॒ मे॒ । मे॒ म॒रु॒त्व॒तीयाः᳚ । म॒रु॒त्व॒तीया᳚श्च । च॒ मे॒ । मे॒ मा॒हे॒न्द्रः । मा॒हे॒न्द्रश्च॑ । मा॒हे॒न्द्र इति॑ माहा - इ॒न्द्रः । च॒ मे॒ । म॒ आ॒दि॒त्यः । आ॒दि॒त्यश्च॑ । च॒ मे॒ । मे॒ सा॒वि॒त्रः । सा॒वि॒त्रश्च॑ । च॒ मे॒ । मे॒ सा॒र॒स्व॒तः । सा॒र॒स्व॒तश्च॑ । च॒ मे॒ । मे॒ पौ॒ष्णः । पौ॒ष्णश्च॑ । च॒ मे॒ । मे॒ पा॒त्नी॒व॒तः । पा॒त्नी॒व॒तश्च॑ । पा॒त्नी॒व॒त इति॑ पात्नी - व॒तः । 
च॒ मे॒ । मे॒ हा॒रि॒यो॒ज॒नः । हा॒रि॒यो॒ज॒नश्च॑ । हा॒रि॒यो॒ज॒न इति॑ हारि - यो॒ज॒नः । च॒ मे॒ । म॒ इति॑ मे । \newline

\textbf{Jatai Paata} \newline

1. मे॒ ऽति॒ग्रा॒ह्या॑ अतिग्रा॒ह्या॑ मे मे ऽतिग्रा॒ह्याः᳚ । \newline
2. अ॒ति॒ग्रा॒ह्या᳚श्च चातिग्रा॒ह्या॑ अतिग्रा॒ह्या᳚श्च । \newline
3. अ॒ति॒ग्रा॒ह्या॑ इत्य॑ति - ग्रा॒ह्याः᳚ । \newline
4. च॒ मे॒ मे॒ च॒ च॒ मे॒ । \newline
5. म॒ ऐ॒न्द्रा॒ग्न ऐ᳚न्द्रा॒ग्नो मे॑ म ऐन्द्रा॒ग्नः । \newline
6. ऐ॒न्द्रा॒ग्नश्च॑ चैन्द्रा॒ग्न ऐ᳚न्द्रा॒ग्नश्च॑ । \newline
7. ऐ॒न्द्रा॒ग्न इत्यै᳚न्द्र - अ॒ग्नः । \newline
8. च॒ मे॒ मे॒ च॒ च॒ मे॒ । \newline
9. मे॒ वै॒श्व॒दे॒वो वै᳚श्वदे॒वो मे॑ मे वैश्वदे॒वः । \newline
10. वै॒श्व॒दे॒वश्च॑ च वैश्वदे॒वो वै᳚श्वदे॒वश्च॑ । \newline
11. वै॒श्व॒दे॒व इति॑ वैश्व - दे॒वः । \newline
12. च॒ मे॒ मे॒ च॒ च॒ मे॒ । \newline
13. मे॒ म॒रु॒त्व॒तीया॑ मरुत्व॒तीया॑ मे मे मरुत्व॒तीयाः᳚ । \newline
14. म॒रु॒त्व॒तीया᳚श्च च मरुत्व॒तीया॑ मरुत्व॒तीया᳚श्च । \newline
15. च॒ मे॒ मे॒ च॒ च॒ मे॒ । \newline
16. मे॒ मा॒हे॒न्द्रो मा॑हे॒न्द्रो मे॑ मे माहे॒न्द्रः । \newline
17. मा॒हे॒न्द्रश्च॑ च माहे॒न्द्रो मा॑हे॒न्द्रश्च॑ । \newline
18. मा॒हे॒न्द्र इति॑ माहा - इ॒न्द्रः । \newline
19. च॒ मे॒ मे॒ च॒ च॒ मे॒ । \newline
20. म॒ आ॒दि॒त्य आ॑दि॒त्यो मे॑ म आदि॒त्यः । \newline
21. आ॒दि॒त्यश्च॑ चादि॒त्य आ॑दि॒त्यश्च॑ । \newline
22. च॒ मे॒ मे॒ च॒ च॒ मे॒ । \newline
23. मे॒ सा॒वि॒त्रः सा॑वि॒त्रो मे॑ मे सावि॒त्रः । \newline
24. सा॒वि॒त्रश्च॑ च सावि॒त्रः सा॑वि॒त्रश्च॑ । \newline
25. च॒ मे॒ मे॒ च॒ च॒ मे॒ । \newline
26. मे॒ सा॒र॒स्व॒तः सा॑रस्व॒तो मे॑ मे सारस्व॒तः । \newline
27. सा॒र॒स्व॒तश्च॑ च सारस्व॒तः सा॑रस्व॒तश्च॑ । \newline
28. च॒ मे॒ मे॒ च॒ च॒ मे॒ । \newline
29. मे॒ पौ॒ष्णः पौ॒ष्णो मे॑ मे पौ॒ष्णः । \newline
30. पौ॒ष्णश्च॑ च पौ॒ष्णः पौ॒ष्णश्च॑ । \newline
31. च॒ मे॒ मे॒ च॒ च॒ मे॒ । \newline
32. मे॒ पा॒त्नी॒व॒तः पा᳚त्नीव॒तो मे॑ मे पात्नीव॒तः । \newline
33. पा॒त्नी॒व॒तश्च॑ च पात्नीव॒तः पा᳚त्नीव॒तश्च॑ । \newline
34. पा॒त्नी॒व॒त इति॑ पात्नी - व॒तः । \newline
35. च॒ मे॒ मे॒ च॒ च॒ मे॒ । \newline
36. मे॒ हा॒रि॒यो॒ज॒नो हा॑रियोज॒नो मे॑ मे हारियोज॒नः । \newline
37. हा॒रि॒यो॒ज॒नश्च॑ च हारियोज॒नो हा॑रियोज॒नश्च॑ । \newline
38. हा॒रि॒यो॒ज॒न इति॑ हारि - यो॒ज॒नः । \newline
39. च॒ मे॒ मे॒ च॒ च॒ मे॒ । \newline
40. म॒ इति॑ मे । \newline

\textbf{Ghana Paata } \newline

1. मे॒ ऽति॒ग्रा॒ह्या॑ अतिग्रा॒ह्या॑ मे मे ऽतिग्रा॒ह्या᳚श्च चातिग्रा॒ह्या॑ मे मे ऽतिग्रा॒ह्या᳚श्च । \newline
2. अ॒ति॒ग्रा॒ह्या᳚श्च चातिग्रा॒ह्या॑ अतिग्रा॒ह्या᳚श्च मे मे चातिग्रा॒ह्या॑ अतिग्रा॒ह्या᳚श्च मे । \newline
3. अ॒ति॒ग्रा॒ह्या॑ इत्य॑ति - ग्रा॒ह्याः᳚ । \newline
4. च॒ मे॒ मे॒ च॒ च॒ म॒ ऐ॒न्द्रा॒ग्न ऐ᳚न्द्रा॒ग्नो मे॑ च च म ऐन्द्रा॒ग्नः । \newline
5. म॒ ऐ॒न्द्रा॒ग्न ऐ᳚न्द्रा॒ग्नो मे॑ म ऐन्द्रा॒ग्नश्च॑ चैन्द्रा॒ग्नो मे॑ म ऐन्द्रा॒ग्नश्च॑ । \newline
6. ऐ॒न्द्रा॒ग्नश्च॑ चैन्द्रा॒ग्न ऐ᳚न्द्रा॒ग्नश्च॑ मे मे चैन्द्रा॒ग्न ऐ᳚न्द्रा॒ग्नश्च॑ मे । \newline
7. ऐ॒न्द्रा॒ग्न इत्यै᳚न्द्र - अ॒ग्नः । \newline
8. च॒ मे॒ मे॒ च॒ च॒ मे॒ वै॒श्व॒दे॒वो वै᳚श्वदे॒वो मे॑ च च मे वैश्वदे॒वः । \newline
9. मे॒ वै॒श्व॒दे॒वो वै᳚श्वदे॒वो मे॑ मे वैश्वदे॒वश्च॑ च वैश्वदे॒वो मे॑ मे वैश्वदे॒वश्च॑ । \newline
10. वै॒श्व॒दे॒वश्च॑ च वैश्वदे॒वो वै᳚श्वदे॒वश्च॑ मे मे च वैश्वदे॒वो वै᳚श्वदे॒वश्च॑ मे । \newline
11. वै॒श्व॒दे॒व इति॑ वैश्व - दे॒वः । \newline
12. च॒ मे॒ मे॒ च॒ च॒ मे॒ म॒रु॒त्व॒तीया॑ मरुत्व॒तीया॑ मे च च मे मरुत्व॒तीयाः᳚ । \newline
13. मे॒ म॒रु॒त्व॒तीया॑ मरुत्व॒तीया॑ मे मे मरुत्व॒तीया᳚श्च च मरुत्व॒तीया॑ मे मे मरुत्व॒तीया᳚श्च । \newline
14. म॒रु॒त्व॒तीया᳚श्च च मरुत्व॒तीया॑ मरुत्व॒तीया᳚श्च मे मे च मरुत्व॒तीया॑ मरुत्व॒तीया᳚श्च मे । \newline
15. च॒ मे॒ मे॒ च॒ च॒ मे॒ मा॒हे॒न्द्रो मा॑हे॒न्द्रो मे॑ च च मे माहे॒न्द्रः । \newline
16. मे॒ मा॒हे॒न्द्रो मा॑हे॒न्द्रो मे॑ मे माहे॒न्द्रश्च॑ च माहे॒न्द्रो मे॑ मे माहे॒न्द्रश्च॑ । \newline
17. मा॒हे॒न्द्रश्च॑ च माहे॒न्द्रो मा॑हे॒न्द्रश्च॑ मे मे च माहे॒न्द्रो मा॑हे॒न्द्रश्च॑ मे । \newline
18. मा॒हे॒न्द्र इति॑ माहा - इ॒न्द्रः । \newline
19. च॒ मे॒ मे॒ च॒ च॒ म॒ आ॒दि॒त्य आ॑दि॒त्यो मे॑ च च म आदि॒त्यः । \newline
20. म॒ आ॒दि॒त्य आ॑दि॒त्यो मे॑ म आदि॒त्यश्च॑ चादि॒त्यो मे॑ म आदि॒त्यश्च॑ । \newline
21. आ॒दि॒त्यश्च॑ चादि॒त्य आ॑दि॒त्यश्च॑ मे मे चादि॒त्य आ॑दि॒त्यश्च॑ मे । \newline
22. च॒ मे॒ मे॒ च॒ च॒ मे॒ सा॒वि॒त्रः सा॑वि॒त्रो मे॑ च च मे सावि॒त्रः । \newline
23. मे॒ सा॒वि॒त्रः सा॑वि॒त्रो मे॑ मे सावि॒त्रश्च॑ च सावि॒त्रो मे॑ मे सावि॒त्रश्च॑ । \newline
24. सा॒वि॒त्रश्च॑ च सावि॒त्रः सा॑वि॒त्रश्च॑ मे मे च सावि॒त्रः सा॑वि॒त्रश्च॑ मे । \newline
25. च॒ मे॒ मे॒ च॒ च॒ मे॒ सा॒र॒स्व॒तः सा॑रस्व॒तो मे॑ च च मे सारस्व॒तः । \newline
26. मे॒ सा॒र॒स्व॒तः सा॑रस्व॒तो मे॑ मे सारस्व॒तश्च॑ च सारस्व॒तो मे॑ मे सारस्व॒तश्च॑ । \newline
27. सा॒र॒स्व॒तश्च॑ च सारस्व॒तः सा॑रस्व॒तश्च॑ मे मे च सारस्व॒तः सा॑रस्व॒तश्च॑ मे । \newline
28. च॒ मे॒ मे॒ च॒ च॒ मे॒ पौ॒ष्णः पौ॒ष्णो मे॑ च च मे पौ॒ष्णः । \newline
29. मे॒ पौ॒ष्णः पौ॒ष्णो मे॑ मे पौ॒ष्णश्च॑ च पौ॒ष्णो मे॑ मे पौ॒ष्णश्च॑ । \newline
30. पौ॒ष्णश्च॑ च पौ॒ष्णः पौ॒ष्णश्च॑ मे मे च पौ॒ष्णः पौ॒ष्णश्च॑ मे । \newline
31. च॒ मे॒ मे॒ च॒ च॒ मे॒ पा॒त्नी॒व॒तः पा᳚त्नीव॒तो मे॑ च च मे पात्नीव॒तः । \newline
32. मे॒ पा॒त्नी॒व॒तः पा᳚त्नीव॒तो मे॑ मे पात्नीव॒तश्च॑ च पात्नीव॒तो मे॑ मे पात्नीव॒तश्च॑ । \newline
33. पा॒त्नी॒व॒तश्च॑ च पात्नीव॒तः पा᳚त्नीव॒तश्च॑ मे मे च पात्नीव॒तः पा᳚त्नीव॒तश्च॑ मे । \newline
34. पा॒त्नी॒व॒त इति॑ पात्नी - व॒तः । \newline
35. च॒ मे॒ मे॒ च॒ च॒ मे॒ हा॒रि॒यो॒ज॒नो हा॑रियोज॒नो मे॑ च च मे हारियोज॒नः । \newline
36. मे॒ हा॒रि॒यो॒ज॒नो हा॑रियोज॒नो मे॑ मे हारियोज॒नश्च॑ च हारियोज॒नो मे॑ मे हारियोज॒नश्च॑ । \newline
37. हा॒रि॒यो॒ज॒नश्च॑ च हारियोज॒नो हा॑रियोज॒नश्च॑ मे मे च हारियोज॒नो हा॑रियोज॒नश्च॑ मे । \newline
38. हा॒रि॒यो॒ज॒न इति॑ हारि - यो॒ज॒नः । \newline
39. च॒ मे॒ मे॒ च॒ च॒ मे॒ । \newline
40. म॒ इति॑ मे । \newline
\pagebreak
\markright{ TS 4.7.8.1  \hfill https://www.vedavms.in \hfill}

\section{ TS 4.7.8.1 }

\textbf{TS 4.7.8.1 } \newline
\textbf{Samhita Paata} \newline

इ॒द्ध्मश्च॑ मे ब॒र्॒.हिश्च॑ मे॒ वेदि॑श्च मे॒ धिष्णि॑याश्च मे॒ स्रुच॑श्च मे चम॒साश्च॑ मे॒ ग्रावा॑णश्च मे॒ स्वर॑वश्च मउपर॒वाश्च॑ मे ऽधि॒षव॑णे च मे द्रोणकल॒शश्च॑ मे वाय॒व्या॑नि च मे पूत॒भृच्च॑ म आधव॒नीय॑श्च म॒ आग्नी᳚द्ध्रं च मे हवि॒र्द्धानं॑ च मे गृ॒हाश्च॑ ( ) मे॒ सद॑श्च मे पुरो॒डाशा᳚श्च मे पच॒ताश्च॑ मेऽवभृ॒थश्च॑ मे स्वगाका॒रश्च॑ मे ॥ \newline

\textbf{Pada Paata} \newline

इ॒द्ध्मः । च॒ । मे॒ । ब॒र्॒.हिः । च॒ । मे॒ । वेदिः॑ । च॒ । मे॒ । धिष्णि॑याः । च॒ । मे॒ । स्रुचः॑ । च॒ । मे॒ । च॒म॒साः । च॒ । मे॒ । ग्रावा॑णः । च॒ । मे॒ । स्वर॑वः । च॒ । मे॒ । उ॒प॒र॒वा इत्यु॑प - र॒वाः । च॒ । मे॒ । अ॒धि॒षव॑णे॒ इत्य॑धि - सव॑ने । च॒ । मे॒ । द्रो॒ण॒क॒ल॒श इति॑ द्रोण-क॒ल॒शः । च॒ । मे॒ । वा॒य॒व्या॑नि । च॒ । मे॒ । पू॒त॒भृदिति॑ पूत - भृत् । च॒ । मे॒ । आ॒ध॒व॒नीय॒ इत्या᳚ - ध॒व॒नीयः॑ । च॒ । मे॒ । आग्नी᳚द्ध्र॒मित्याग्नि॑-इ॒द्ध्र॒म् । च॒ । मे॒ । ह॒वि॒द्‌र्धान॒मिति॑ हविः - धान᳚म् । च॒ । मे॒ । गृ॒हाः । च॒ ( ) । मे॒ । सदः॑ । च॒ । मे॒ । पु॒रो॒डाशाः᳚ । च॒ । मे॒ । प॒च॒ताः । च॒ । मे॒ । अ॒व॒भृ॒थ इत्य॑व - भृ॒थः । च॒ । मे॒ । स्व॒गा॒का॒र इति॑ स्वगा - का॒रः । च॒ । मे॒ ॥  \newline


\textbf{Krama Paata} \newline

इ॒द्ध्मश्च॑ । च॒ मे॒ । मे॒ ब॒र्.॒हिः । ब॒र्.॒हिश्च॑ । च॒ मे॒ । मे॒ वेदिः॑ । वेदि॑श्च । च॒ मे॒ । मे॒ धिष्णि॑याः । धिष्णि॑याश्च । च॒ मे॒ । मे॒ स्रुचः॑ । स्रुच॑श्च । च॒ मे॒ । मे॒ च॒म॒साः । च॒म॒साश्च॑ । च॒ मे॒ । मे॒ ग्रावा॑णः । ग्रावा॑णश्च । च॒ मे॒ । मे॒ स्वर॑वः । स्वर॑वश्च । च॒ मे॒ । म॒ उ॒प॒र॒वाः । उ॒प॒र॒वाश्च॑ । उ॒प॒र॒वा इत्यु॑प - र॒वाः । च॒ मे॒ । मे॒ऽधि॒षव॑णे । अ॒धि॒षव॑णे च । अ॒धि॒षव॑णे॒ इत्य॑धि - सव॑ने । च॒ मे॒ । मे॒ द्रो॒ण॒क॒ल॒शः । द्रो॒ण॒क॒ल॒शश्च॑ । द्रो॒ण॒क॒ल॒श इति॑ द्रोण - क॒ल॒शः । च॒ मे॒ । मे॒ वा॒य॒व्या॑नि । वा॒य॒व्या॑नि च । च॒ मे॒ । मे॒ पू॒त॒भृत् । पू॒त॒भृच् च॑ । पू॒त॒भृदिति॑ पूत - भृत् । च॒ मे॒ । म॒ आ॒ध॒व॒नीयः॑ । आ॒ध॒व॒नीय॑श्च । आ॒ध॒व॒नीय॒ इत्या᳚ - ध॒व॒नीयः॑ । च॒ मे॒ । म॒ आग्नी᳚द्ध्रम् । आग्नी᳚द्ध्रम् च । आग्नी᳚द्ध्र॒मित्याग्नि॑ - इ॒द्ध्र॒म् । च॒ मे॒ । मे॒ ह॒वि॒र्द्धान᳚म् । ह॒वि॒र्द्धान॑म् च । ह॒वि॒र्द्धान॒मिति॑ हविः - धान᳚म् । च॒ मे॒ । मे॒ गृ॒हाः । गृ॒हाश्च॑ ( ) । च॒ मे॒ । मे॒ सदः॑ । सद॑श्च । च॒ मे॒ । मे॒ पु॒रो॒डाशाः᳚ । पु॒रो॒डाशा᳚श्च । च॒ मे॒ । मे॒ प॒च॒ताः । प॒च॒ताश्च॑ । च॒ मे॒ । मे॒ऽव॒भृ॒थः । अ॒व॒भृ॒थश्च॑ । अ॒व॒भृ॒थ इत्य॑व - भृ॒थः । च॒ मे॒ । मे॒ स्व॒गा॒का॒रः । स्व॒गा॒का॒रश्च॑ । स्व॒गा॒का॒र इति॑ स्वगा - का॒रः । च॒ मे॒ । म॒ इति॑ मे । \newline

\textbf{Jatai Paata} \newline

1. इ॒द्ध्मश्च॑ चे॒द्ध्म इ॒द्ध्मश्च॑ । \newline
2. च॒ मे॒ मे॒ च॒ च॒ मे॒ । \newline
3. मे॒ ब॒र्॒.हिर् ब॒र्॒.हिर् मे॑ मे ब॒र्॒.हिः । \newline
4. ब॒र्॒.हिश्च॑ च ब॒र्॒.हिर् ब॒र्॒.हिश्च॑ । \newline
5. च॒ मे॒ मे॒ च॒ च॒ मे॒ । \newline
6. मे॒ वेदि॒र् वेदि॑र् मे मे॒ वेदिः॑ । \newline
7. वेदि॑श्च च॒ वेदि॒र् वेदि॑श्च । \newline
8. च॒ मे॒ मे॒ च॒ च॒ मे॒ । \newline
9. मे॒ धिष्णि॑या॒ धिष्णि॑या मे मे॒ धिष्णि॑याः । \newline
10. धिष्णि॑याश्च च॒ धिष्णि॑या॒ धिष्णि॑याश्च । \newline
11. च॒ मे॒ मे॒ च॒ च॒ मे॒ । \newline
12. मे॒ स्रुचः॒ स्रुचो॑ मे मे॒ स्रुचः॑ । \newline
13. स्रुच॑श्च च॒ स्रुचः॒ स्रुच॑श्च । \newline
14. च॒ मे॒ मे॒ च॒ च॒ मे॒ । \newline
15. मे॒ च॒म॒सा श्च॑म॒सा मे॑ मे चम॒साः । \newline
16. च॒म॒साश्च॑ च चम॒सा श्च॑म॒साश्च॑ । \newline
17. च॒ मे॒ मे॒ च॒ च॒ मे॒ । \newline
18. मे॒ ग्रावा॑णो॒ ग्रावा॑णो मे मे॒ ग्रावा॑णः । \newline
19. ग्रावा॑णश्च च॒ ग्रावा॑णो॒ ग्रावा॑णश्च । \newline
20. च॒ मे॒ मे॒ च॒ च॒ मे॒ । \newline
21. मे॒ स्वर॑वः॒ स्वर॑वो मे मे॒ स्वर॑वः । \newline
22. स्वर॑वश्च च॒ स्वर॑वः॒ स्वर॑वश्च । \newline
23. च॒ मे॒ मे॒ च॒ च॒ मे॒ । \newline
24. म॒ उ॒प॒र॒वा उ॑पर॒वा मे॑ म उपर॒वाः । \newline
25. उ॒प॒र॒वाश्च॑ चोपर॒वा उ॑पर॒वाश्च॑ । \newline
26. उ॒प॒र॒वा इत्यु॑प - र॒वाः । \newline
27. च॒ मे॒ मे॒ च॒ च॒ मे॒ । \newline
28. मे॒ ऽधि॒षव॑णे अधि॒षव॑णे मे मे ऽधि॒षव॑णे । \newline
29. अ॒धि॒षव॑णे च चाधि॒षव॑णे अधि॒षव॑णे च । \newline
30. अ॒धि॒षव॑णे॒ इत्य॑धि - सव॑ने । \newline
31. च॒ मे॒ मे॒ च॒ च॒ मे॒ । \newline
32. मे॒ द्रो॒ण॒क॒ल॒शो द्रो॑णकल॒शो मे॑ मे द्रोणकल॒शः । \newline
33. द्रो॒ण॒क॒ल॒शश्च॑ च द्रोणकल॒शो द्रो॑णकल॒शश्च॑ । \newline
34. द्रो॒ण॒क॒ल॒श इति॑ द्रोण - क॒ल॒शः । \newline
35. च॒ मे॒ मे॒ च॒ च॒ मे॒ । \newline
36. मे॒ वा॒य॒व्या॑नि वाय॒व्या॑नि मे मे वाय॒व्या॑नि । \newline
37. वा॒य॒व्या॑नि च च वाय॒व्या॑नि वाय॒व्या॑नि च । \newline
38. च॒ मे॒ मे॒ च॒ च॒ मे॒ । \newline
39. मे॒ पू॒त॒भृत् पू॑त॒भृन् मे॑ मे पूत॒भृत् । \newline
40. पू॒त॒भृच् च॑ च पूत॒भृत् पू॑त॒भृच् च॑ । \newline
41. पू॒त॒भृदिति॑ पूत - भृत् । \newline
42. च॒ मे॒ मे॒ च॒ च॒ मे॒ । \newline
43. म॒ आ॒ध॒व॒नीय॑ आधव॒नीयो॑ मे म आधव॒नीयः॑ । \newline
44. आ॒ध॒व॒नीय॑श्च चाधव॒नीय॑ आधव॒नीय॑श्च । \newline
45. आ॒ध॒व॒नीय॒इत्या᳚ - ध॒व॒नीयः॑ । \newline
46. च॒ मे॒ मे॒ च॒ च॒ मे॒ । \newline
47. म॒ आग्नी᳚द्ध्र॒ माग्नी᳚द्ध्रम् मे म॒ आग्नी᳚द्ध्रम् । \newline
48. आग्नी᳚द्ध्रम् च॒ चाग्नी᳚द्ध्र॒ माग्नी᳚द्ध्रम् च । \newline
49. आग्नी᳚द्ध्र॒मित्याग्नि॑ - इ॒द्ध्र॒म् । \newline
50. च॒ मे॒ मे॒ च॒ च॒ मे॒ । \newline
51. मे॒ ह॒वि॒र्द्धान(ग्म्॑) हवि॒र्द्धान॑म् मे मे हवि॒र्द्धान᳚म् । \newline
52. ह॒वि॒र्द्धान॑म् च च हवि॒र्द्धान(ग्म्॑) हवि॒र्द्धान॑म् च । \newline
53. ह॒वि॒र्द्धान॒मिति॑ हविः - धान᳚म् । \newline
54. च॒ मे॒ मे॒ च॒ च॒ मे॒ । \newline
55. मे॒ गृ॒हा गृ॒हा मे॑ मे गृ॒हाः । \newline
56. गृ॒हाश्च॑ च गृ॒हा गृ॒हाश्च॑ । \newline
57. च॒ मे॒ मे॒ च॒ च॒ मे॒ । \newline
58. मे॒ सदः॒ सदो॑ मे मे॒ सदः॑ । \newline
59. सद॑श्च च॒ सदः॒ सद॑श्च । \newline
60. च॒ मे॒ मे॒ च॒ च॒ मे॒ । \newline
61. मे॒ पु॒रो॒डाशाः᳚ पुरो॒डाशा॑ मे मे पुरो॒डाशाः᳚ । \newline
62. पु॒रो॒डाशा᳚श्च च पुरो॒डाशाः᳚ पुरो॒डाशा᳚श्च । \newline
63. च॒ मे॒ मे॒ च॒ च॒ मे॒ । \newline
64. मे॒ प॒च॒ताः प॑च॒ता मे॑ मे पच॒ताः । \newline
65. प॒च॒ताश्च॑ च पच॒ताः प॑च॒ताश्च॑ । \newline
66. च॒ मे॒ मे॒ च॒ च॒ मे॒ । \newline
67. मे॒ ऽव॒भृ॒थो॑ ऽवभृ॒थो मे॑ मे ऽवभृ॒थः । \newline
68. अ॒व॒भृ॒थश्च॑ चावभृ॒थो॑ ऽवभृ॒थश्च॑ । \newline
69. अ॒व॒भृ॒थ इत्य॑व - भृ॒थः । \newline
70. च॒ मे॒ मे॒ च॒ च॒ मे॒ । \newline
71. मे॒ स्व॒गा॒का॒रः स्व॑गाका॒रो मे॑ मे स्वगाका॒रः । \newline
72. स्व॒गा॒का॒रश्च॑ च स्वगाका॒रः स्व॑गाका॒रश्च॑ । \newline
73. स्व॒गा॒का॒र इति॑ स्वगा - का॒रः । \newline
74. च॒ मे॒ मे॒ च॒ च॒ मे॒ । \newline
75. म॒ इति॑ मे । \newline

\textbf{Ghana Paata } \newline

1. इ॒द्ध्मश्च॑ चे॒द्ध्म इ॒द्ध्मश्च॑ मे मे चे॒द्ध्म इ॒द्ध्मश्च॑ मे । \newline
2. च॒ मे॒ मे॒ च॒ च॒ मे॒ ब॒र्॒.हिर् ब॒र्॒.हिर् मे॑ च च मे ब॒र्॒.हिः । \newline
3. मे॒ ब॒र्॒.हिर् ब॒र्॒.हिर् मे॑ मे ब॒र्॒.हिश्च॑ च ब॒र्॒.हिर् मे॑ मे ब॒र्॒.हिश्च॑ । \newline
4. ब॒र्॒.हिश्च॑ च ब॒र्॒.हिर् ब॒र्॒.हिश्च॑ मे मे च ब॒र्॒.हिर् ब॒र्॒.हिश्च॑ मे । \newline
5. च॒ मे॒ मे॒ च॒ च॒ मे॒ वेदि॒र् वेदि॑र् मे च च मे॒ वेदिः॑ । \newline
6. मे॒ वेदि॒र् वेदि॑र् मे मे॒ वेदि॑श्च च॒ वेदि॑र् मे मे॒ वेदि॑श्च । \newline
7. वेदि॑श्च च॒ वेदि॒र् वेदि॑श्च मे मे च॒ वेदि॒र् वेदि॑श्च मे । \newline
8. च॒ मे॒ मे॒ च॒ च॒ मे॒ धिष्णि॑या॒ धिष्णि॑या मे च च मे॒ धिष्णि॑याः । \newline
9. मे॒ धिष्णि॑या॒ धिष्णि॑या मे मे॒ धिष्णि॑याश्च च॒ धिष्णि॑या मे मे॒ धिष्णि॑याश्च । \newline
10. धिष्णि॑याश्च च॒ धिष्णि॑या॒ धिष्णि॑याश्च मे मे च॒ धिष्णि॑या॒ धिष्णि॑याश्च मे । \newline
11. च॒ मे॒ मे॒ च॒ च॒ मे॒ स्रुचः॒ स्रुचो॑ मे च च मे॒ स्रुचः॑ । \newline
12. मे॒ स्रुचः॒ स्रुचो॑ मे मे॒ स्रुच॑श्च च॒ स्रुचो॑ मे मे॒ स्रुच॑श्च । \newline
13. स्रुच॑श्च च॒ स्रुचः॒ स्रुच॑श्च मे मे च॒ स्रुचः॒ स्रुच॑श्च मे । \newline
14. च॒ मे॒ मे॒ च॒ च॒ मे॒ च॒म॒सा श्च॑म॒सा मे॑ च च मे चम॒साः । \newline
15. मे॒ च॒म॒सा श्च॑म॒सा मे॑ मे चम॒साश्च॑ च चम॒सा मे॑ मे चम॒साश्च॑ । \newline
16. च॒म॒साश्च॑ च चम॒सा श्च॑म॒साश्च॑ मे मे च चम॒सा श्च॑म॒साश्च॑ मे । \newline
17. च॒ मे॒ मे॒ च॒ च॒ मे॒ ग्रावा॑णो॒ ग्रावा॑णो मे च च मे॒ ग्रावा॑णः । \newline
18. मे॒ ग्रावा॑णो॒ ग्रावा॑णो मे मे॒ ग्रावा॑णश्च च॒ ग्रावा॑णो मे मे॒ ग्रावा॑णश्च । \newline
19. ग्रावा॑णश्च च॒ ग्रावा॑णो॒ ग्रावा॑णश्च मे मे च॒ ग्रावा॑णो॒ ग्रावा॑णश्च मे । \newline
20. च॒ मे॒ मे॒ च॒ च॒ मे॒ स्वर॑वः॒ स्वर॑वो मे च च मे॒ स्वर॑वः । \newline
21. मे॒ स्वर॑वः॒ स्वर॑वो मे मे॒ स्वर॑वश्च च॒ स्वर॑वो मे मे॒ स्वर॑वश्च । \newline
22. स्वर॑वश्च च॒ स्वर॑वः॒ स्वर॑वश्च मे मे च॒ स्वर॑वः॒ स्वर॑वश्च मे । \newline
23. च॒ मे॒ मे॒ च॒ च॒ म॒ उ॒प॒र॒वा उ॑पर॒वा मे॑ च च म उपर॒वाः । \newline
24. म॒ उ॒प॒र॒वा उ॑पर॒वा मे॑ म उपर॒वाश्च॑ चोपर॒वा मे॑ म उपर॒वाश्च॑ । \newline
25. उ॒प॒र॒वाश्च॑ चोपर॒वा उ॑पर॒वाश्च॑ मे मे चोपर॒वा उ॑पर॒वाश्च॑ मे । \newline
26. उ॒प॒र॒वा इत्यु॑प - र॒वाः । \newline
27. च॒ मे॒ मे॒ च॒ च॒ मे॒ ऽधि॒षव॑णे अधि॒षव॑णे मे च च मे ऽधि॒षव॑णे । \newline
28. मे॒ ऽधि॒षव॑णे अधि॒षव॑णे मे मे ऽधि॒षव॑णे च चाधि॒षव॑णे मे मे ऽधि॒षव॑णे च । \newline
29. अ॒धि॒षव॑णे च चाधि॒षव॑णे अधि॒षव॑णे च मे मे चाधि॒षव॑णे अधि॒षव॑णे च मे । \newline
30. अ॒धि॒षव॑णे॒ इत्य॑धि - सव॑ने । \newline
31. च॒ मे॒ मे॒ च॒ च॒ मे॒ द्रो॒ण॒क॒ल॒शो द्रो॑णकल॒शो मे॑ च च मे द्रोणकल॒शः । \newline
32. मे॒ द्रो॒ण॒क॒ल॒शो द्रो॑णकल॒शो मे॑ मे द्रोणकल॒शश्च॑ च द्रोणकल॒शो मे॑ मे द्रोणकल॒शश्च॑ । \newline
33. द्रो॒ण॒क॒ल॒शश्च॑ च द्रोणकल॒शो द्रो॑णकल॒शश्च॑ मे मे च द्रोणकल॒शो द्रो॑णकल॒शश्च॑ मे । \newline
34. द्रो॒ण॒क॒ल॒श इति॑ द्रोण - क॒ल॒शः । \newline
35. च॒ मे॒ मे॒ च॒ च॒ मे॒ वा॒य॒व्या॑नि वाय॒व्या॑नि मे च च मे वाय॒व्या॑नि । \newline
36. मे॒ वा॒य॒व्या॑नि वाय॒व्या॑नि मे मे वाय॒व्या॑नि च च वाय॒व्या॑नि मे मे वाय॒व्या॑नि च । \newline
37. वा॒य॒व्या॑नि च च वाय॒व्या॑नि वाय॒व्या॑नि च मे मे च वाय॒व्या॑नि वाय॒व्या॑नि च मे । \newline
38. च॒ मे॒ मे॒ च॒ च॒ मे॒ पू॒त॒भृत् पू॑त॒भृन् मे॑ च च मे पूत॒भृत् । \newline
39. मे॒ पू॒त॒भृत् पू॑त॒भृन् मे॑ मे पूत॒भृच् च॑ च पूत॒भृन् मे॑ मे पूत॒भृच् च॑ । \newline
40. पू॒त॒भृच् च॑ च पूत॒भृत् पू॑त॒भृच् च॑ मे मे च पूत॒भृत् पू॑त॒भृच् च॑ मे । \newline
41. पू॒त॒भृदिति॑ पूत - भृत् । \newline
42. च॒ मे॒ मे॒ च॒ च॒ म॒ आ॒ध॒व॒नीय॑ आधव॒नीयो॑ मे च च म आधव॒नीयः॑ । \newline
43. म॒ आ॒ध॒व॒नीय॑ आधव॒नीयो॑ मे म आधव॒नीय॑श्च चाधव॒नीयो॑ मे म आधव॒नीय॑श्च । \newline
44. आ॒ध॒व॒नीय॑श्च चाधव॒नीय॑ आधव॒नीय॑श्च मे मे चाधव॒नीय॑ आधव॒नीय॑श्च मे । \newline
45. आ॒ध॒व॒नीय॒ इत्या᳚ - ध॒व॒नीयः॑ । \newline
46. च॒ मे॒ मे॒ च॒ च॒ म॒ आग्नी᳚द्ध्र॒ माग्नी᳚द्ध्रम् मे च च म॒ आग्नी᳚द्ध्रम् । \newline
47. म॒ आग्नी᳚द्ध्र॒ माग्नी᳚द्ध्रम् मे म॒ आग्नी᳚द्ध्रम् च॒ चाग्नी᳚द्ध्रम् मे म॒ आग्नी᳚द्ध्रम् च । \newline
48. आग्नी᳚द्ध्रम् च॒ चाग्नी᳚द्ध्र॒ माग्नी᳚द्ध्रम् च मे मे॒ चाग्नी᳚द्ध्र॒ माग्नी᳚द्ध्रम् च मे । \newline
49. आग्नी᳚द्ध्र॒मित्याग्नि॑ - इ॒द्ध्र॒म् । \newline
50. च॒ मे॒ मे॒ च॒ च॒ मे॒ ह॒वि॒र्द्धान(ग्म्॑) हवि॒र्द्धान॑म् मे च च मे हवि॒र्द्धान᳚म् । \newline
51. मे॒ ह॒वि॒र्द्धान(ग्म्॑) हवि॒र्द्धान॑म् मे मे हवि॒र्द्धान॑म् च च हवि॒र्द्धान॑म् मे मे हवि॒र्द्धान॑म् च । \newline
52. ह॒वि॒र्द्धान॑म् च च हवि॒र्द्धान(ग्म्॑) हवि॒र्द्धान॑म् च मे मे च हवि॒र्द्धान(ग्म्॑) हवि॒र्द्धान॑म् च मे । \newline
53. ह॒वि॒र्द्धान॒मिति॑ हविः - धान᳚म् । \newline
54. च॒ मे॒ मे॒ च॒ च॒ मे॒ गृ॒हा गृ॒हा मे॑ च च मे गृ॒हाः । \newline
55. मे॒ गृ॒हा गृ॒हा मे॑ मे गृ॒हाश्च॑ च गृ॒हा मे॑ मे गृ॒हाश्च॑ । \newline
56. गृ॒हाश्च॑ च गृ॒हा गृ॒हाश्च॑ मे मे च गृ॒हा गृ॒हाश्च॑ मे । \newline
57. च॒ मे॒ मे॒ च॒ च॒ मे॒ सदः॒ सदो॑ मे च च मे॒ सदः॑ । \newline
58. मे॒ सदः॒ सदो॑ मे मे॒ सद॑श्च च॒ सदो॑ मे मे॒ सद॑श्च । \newline
59. सद॑श्च च॒ सदः॒ सद॑श्च मे मे च॒ सदः॒ सद॑श्च मे । \newline
60. च॒ मे॒ मे॒ च॒ च॒ मे॒ पु॒रो॒डाशाः᳚ पुरो॒डाशा॑ मे च च मे पुरो॒डाशाः᳚ । \newline
61. मे॒ पु॒रो॒डाशाः᳚ पुरो॒डाशा॑ मे मे पुरो॒डाशा᳚श्च च पुरो॒डाशा॑ मे मे पुरो॒डाशा᳚श्च । \newline
62. पु॒रो॒डाशा᳚श्च च पुरो॒डाशाः᳚ पुरो॒डाशा᳚श्च मे मे च पुरो॒डाशाः᳚ पुरो॒डाशा᳚श्च मे । \newline
63. च॒ मे॒ मे॒ च॒ च॒ मे॒ प॒च॒ताः प॑च॒ता मे॑ च च मे पच॒ताः । \newline
64. मे॒ प॒च॒ताः प॑च॒ता मे॑ मे पच॒ताश्च॑ च पच॒ता मे॑ मे पच॒ताश्च॑ । \newline
65. प॒च॒ताश्च॑ च पच॒ताः प॑च॒ताश्च॑ मे मे च पच॒ताः प॑च॒ताश्च॑ मे । \newline
66. च॒ मे॒ मे॒ च॒ च॒ मे॒ ऽव॒भृ॒थो॑ ऽवभृ॒थो मे॑ च च मे ऽवभृ॒थः । \newline
67. मे॒ ऽव॒भृ॒थो॑ ऽवभृ॒थो मे॑ मे ऽवभृ॒थश्च॑ चावभृ॒थो मे॑ मे ऽवभृ॒थश्च॑ । \newline
68. अ॒व॒भृ॒थश्च॑ चावभृ॒थो॑ ऽवभृ॒थश्च॑ मे मे चावभृ॒थो॑ ऽवभृ॒थश्च॑ मे । \newline
69. अ॒व॒भृ॒थ इत्य॑व - भृ॒थः । \newline
70. च॒ मे॒ मे॒ च॒ च॒ मे॒ स्व॒गा॒का॒रः स्व॑गाका॒रो मे॑ च च मे स्वगाका॒रः । \newline
71. मे॒ स्व॒गा॒का॒रः स्व॑गाका॒रो मे॑ मे स्वगाका॒रश्च॑ च स्वगाका॒रो मे॑ मे स्वगाका॒रश्च॑ । \newline
72. स्व॒गा॒का॒रश्च॑ च स्वगाका॒रः स्व॑गाका॒रश्च॑ मे मे च स्वगाका॒रः स्व॑गाका॒रश्च॑ मे । \newline
73. स्व॒गा॒का॒र इति॑ स्वगा - का॒रः । \newline
74. च॒ मे॒ मे॒ च॒ च॒ मे॒ । \newline
75. म॒ इति॑ मे । \newline
\pagebreak
\markright{ TS 4.7.9.1  \hfill https://www.vedavms.in \hfill}

\section{ TS 4.7.9.1 }

\textbf{TS 4.7.9.1 } \newline
\textbf{Samhita Paata} \newline

अ॒ग्निश्च॑ मे घ॒र्मश्च॑ मे॒ऽर्कश्च॑ मे॒ सूर्य॑श्च मे प्रा॒णश्च॑ मे ऽश्वमे॒धश्च॑ मे पृथि॒वी च॒ मे ऽदि॑तिश्च मे॒ दिति॑श्च मे॒ द्यौश्च॑ मे॒ शक्व॑रीर॒ङ्गुल॑यो॒ दिश॑श्च मे     य॒ज्ञेन॑ कल्पन्ता॒- मृक्च॑ मे॒ साम॑ च मे॒ स्तोम॑श्च मे॒ यजु॑श्च मे दी॒क्षा ( ) च॑ मे॒ तप॑श्च म ऋ॒तुश्च॑ मे व्र॒तं च॑ मे ऽहोरा॒त्रयो᳚ र्वृ॒ष्ट्या बृ॑हद्रथन्त॒रे च॑ मे य॒ज्ञेन॑ कल्पेतां ॥ \newline

\textbf{Pada Paata} \newline

अ॒ग्निः । च॒ । मे॒ । घ॒र्मः । च॒ । मे॒ । अ॒र्कः । च॒ । मे॒ । सूर्यः॑ । च॒ । मे॒ । प्रा॒ण इति॑ प्र - अ॒नः । च॒ । मे॒ । अ॒श्व॒मे॒ध इत्य॑श्व-मे॒धः । च॒ । मे॒ । पृ॒थि॒वी । च॒ । मे॒ । अदि॑तिः । च॒ । मे॒ । दितिः॑ । च॒ । मे॒ । द्यौः । च॒ । मे॒ । शक्व॑रीः । अ॒ङ्गुल॑यः । दिशः॑ । च॒ । मे॒ । य॒ज्ञेन॑ । क॒ल्प॒न्ता॒म् । ऋक् । च॒ । मे॒ । साम॑ । च॒ । मे॒ । स्तोमः॑ । च॒ । मे॒ । यजुः॑ । च॒ । मे॒ । दी॒क्षा ( ) । च॒ । मे॒ । तपः॑ । च॒ । मे॒ । ऋ॒तुः । च॒ । मे॒ । व्र॒तम् । च॒ । मे॒ । अ॒हो॒रा॒त्रयो॒रित्य॑हः - रा॒त्रयोः᳚ । वृ॒ष्ट्या । बृ॒ह॒द्र॒थ॒न्त॒रे इति॑ बृहत् - र॒थ॒न्त॒रे । च॒ । मे॒ । य॒ज्ञेन॑ । क॒ल्पे॒ता॒म् ॥  \newline


\textbf{Krama Paata} \newline

अ॒ग्निश्च॑ । च॒ मे॒ । मे॒ घ॒र्मः । घ॒र्मश्च॑ । च॒ मे॒ । मे॒ऽर्कः । अ॒र्कश्च॑ । च॒ मे॒ । मे॒ सूर्यः॑ । सूर्य॑श्च । च॒ मे॒ । मे॒ प्रा॒णः । प्रा॒णश्च॑ । प्रा॒॒ण इति॑ प्र - अ॒नः । च॒ मे॒ । मे॒ऽश्व॒मे॒धः । अ॒श्व॒मे॒धश्च॑ । अ॒श्व॒मे॒ध इत्य॑श्व - मे॒धः । च॒ मे॒ । मे॒ पृ॒थि॒वी । पृ॒थि॒वी च॑ । च॒ मे॒ । मेऽदि॑तिः । अदि॑तिश्च । च॒ मे॒ । मे॒ दितिः॑ । दिति॑श्च । च॒ मे॒ । मे॒ द्यौः । द्यौश्च॑ । च॒ मे॒ । मे॒ शक्व॑रीः । शक्व॑रीर॒ङ्गुल॑यः । अ॒ङ्गुल॑यो॒ दिशः॑ । दिश॑श्च । च॒ मे॒ । मे॒ य॒ज्ञेन॑ । य॒ज्ञेन॑ कल्पन्ताम् । क॒ल्प॒न्ता॒मृक् । ऋक् च॑ । च॒ मे॒ । मे॒ साम॑ । साम॑ च । च॒ मे॒ । मे॒ स्तोमः॑ । स्तोम॑श्च । च॒ मे॒ । मे॒ यजुः॑ । यजु॑श्च । च॒ मे॒ । मे॒ दी॒क्षा ( ) । दी॒क्षा च॑ । च॒ मे॒ । मे॒ तपः॑ । तप॑श्च । च॒ मे॒ । म॒ ऋ॒तुः । ऋ॒तुश्च॑ । च॒ मे॒ । 
मे॒ व्र॒तम् । व्र॒तम् च॑ । च॒ मे॒ । मे॒ऽहो॒रा॒त्रयोः᳚ । अ॒हो॒रा॒त्रयो᳚र् वृ॒ष्ट्या । अ॒हो॒रा॒त्रयो॒रित्य॑हः - रा॒त्रयोः᳚ । वृ॒ष्ट्या बृ॑हद्रथन्त॒रे । बृ॒ह॒द्र॒थ॒न्त॒रे च॑ । बृ॒ह॒द्र॒थ॒न्त॒रे इति॑ बृहत् - र॒थ॒न्त॒रे । च॒ मे॒ । मे॒ य॒ज्ञेन॑ । य॒ज्ञेन॑ कल्पेताम् । क॒ल्पे॒ता॒मिति॑ कल्पेताम् । \newline

\textbf{Jatai Paata} \newline

1. अ॒ग्निश्च॑ चा॒ग्नि र॒ग्निश्च॑ । \newline
2. च॒ मे॒ मे॒ च॒ च॒ मे॒ । \newline
3. मे॒ घ॒र्मो घ॒र्मो मे॑ मे घ॒र्मः । \newline
4. घ॒र्मश्च॑ च घ॒र्मो घ॒र्मश्च॑ । \newline
5. च॒ मे॒ मे॒ च॒ च॒ मे॒ । \newline
6. मे॒ ऽर्को᳚ ऽर्को मे॑ मे॒ ऽर्कः । \newline
7. अ॒र्कश्च॑ चा॒र्को᳚ ऽर्कश्च॑ । \newline
8. च॒ मे॒ मे॒ च॒ च॒ मे॒ । \newline
9. मे॒ सूर्यः॒ सूर्यो॑ मे मे॒ सूर्यः॑ । \newline
10. सूर्य॑श्च च॒ सूर्यः॒ सूर्य॑श्च । \newline
11. च॒ मे॒ मे॒ च॒ च॒ मे॒ । \newline
12. मे॒ प्रा॒णः प्रा॒णो मे॑ मे प्रा॒णः । \newline
13. प्रा॒णश्च॑ च प्रा॒णः प्रा॒णश्च॑ । \newline
14. प्रा॒ण इति॑ प्र - अ॒नः । \newline
15. च॒ मे॒ मे॒ च॒ च॒ मे॒ । \newline
16. मे॒ ऽश्व॒मे॒धो᳚ ऽश्वमे॒धो मे॑ मे ऽश्वमे॒धः । \newline
17. अ॒श्व॒मे॒धश्च॑ चाश्वमे॒धो᳚ ऽश्वमे॒धश्च॑ । \newline
18. अ॒श्व॒मे॒ध इत्य॑श्व - मे॒धः । \newline
19. च॒ मे॒ मे॒ च॒ च॒ मे॒ । \newline
20. मे॒ पृ॒थि॒वी पृ॑थि॒वी मे॑ मे पृथि॒वी । \newline
21. पृ॒थि॒वी च॑ च पृथि॒वी पृ॑थि॒वी च॑ । \newline
22. च॒ मे॒ मे॒ च॒ च॒ मे॒ । \newline
23. मे ऽदि॑ति॒ रदि॑तिर् मे॒ मे ऽदि॑तिः । \newline
24. अदि॑तिश्च॒ चादि॑ति॒ रदि॑तिश्च । \newline
25. च॒ मे॒ मे॒ च॒ च॒ मे॒ । \newline
26. मे॒ दिति॒र् दिति॑र् मे मे॒ दितिः॑ । \newline
27. दिति॑श्च च॒ दिति॒र् दिति॑श्च । \newline
28. च॒ मे॒ मे॒ च॒ च॒ मे॒ । \newline
29. मे॒ द्यौर् द्यौर् मे॑ मे॒ द्यौः । \newline
30. द्यौश्च॑ च॒ द्यौर् द्यौश्च॑ । \newline
31. च॒ मे॒ मे॒ च॒ च॒ मे॒ । \newline
32. मे॒ शक्व॑रीः॒ शक्व॑रीर् मे मे॒ शक्व॑रीः । \newline
33. शक्व॑री र॒ङ्गुल॑यो॒ ऽङ्गुल॑यः॒ शक्व॑रीः॒ शक्व॑री र॒ङ्गुल॑यः । \newline
34. अ॒ङ्गुल॑यो॒ दिशो॒ दिशो॒ ऽङ्गुल॑यो॒ ऽङ्गुल॑यो॒ दिशः॑ । \newline
35. दिश॑श्च च॒ दिशो॒ दिश॑श्च । \newline
36. च॒ मे॒ मे॒ च॒ च॒ मे॒ । \newline
37. मे॒ य॒ज्ञेन॑ य॒ज्ञेन॑ मे मे य॒ज्ञेन॑ । \newline
38. य॒ज्ञेन॑ कल्पन्ताम् कल्पन्तां ॅय॒ज्ञेन॑ य॒ज्ञेन॑ कल्पन्ताम् । \newline
39. क॒ल्प॒न्ता॒ मृ गृक् क॑ल्पन्ताम् कल्पन्ता॒ मृक् । \newline
40. ऋक् च॒ चर्‌ गृक् च॑ । \newline
41. च॒ मे॒ मे॒ च॒ च॒ मे॒ । \newline
42. मे॒ साम॒ साम॑ मे मे॒ साम॑ । \newline
43. साम॑ च च॒ साम॒ साम॑ च । \newline
44. च॒ मे॒ मे॒ च॒ च॒ मे॒ । \newline
45. मे॒ स्तोमः॒ स्तोमो॑ मे मे॒ स्तोमः॑ । \newline
46. स्तोम॑श्च च॒ स्तोमः॒ स्तोम॑श्च । \newline
47. च॒ मे॒ मे॒ च॒ च॒ मे॒ । \newline
48. मे॒ यजु॒र् यजु॑र् मे मे॒ यजुः॑ । \newline
49. यजु॑श्च च॒ यजु॒र् यजु॑श्च । \newline
50. च॒ मे॒ मे॒ च॒ च॒ मे॒ । \newline
51. मे॒ दी॒क्षा दी॒क्षा मे॑ मे दी॒क्षा । \newline
52. दी॒क्षा च॑ च दी॒क्षा दी॒क्षा च॑ । \newline
53. च॒ मे॒ मे॒ च॒ च॒ मे॒ । \newline
54. मे॒ तप॒ स्तपो॑ मे मे॒ तपः॑ । \newline
55. तप॑श्च च॒ तप॒ स्तप॑श्च । \newline
56. च॒ मे॒ मे॒ च॒ च॒ मे॒ । \newline
57. म॒ ऋ॒तुर्. ऋ॒तुर् मे॑ म ऋ॒तुः । \newline
58. ऋ॒तुश्च॑ च॒ र्‌तुर्. ऋ॒तुश्च॑ । \newline
59. च॒ मे॒ मे॒ च॒ च॒ मे॒ । \newline
60. मे॒ व्र॒तं ॅव्र॒तम् मे॑ मे व्र॒तम् । \newline
61. व्र॒तम् च॑ च व्र॒तं ॅव्र॒तम् च॑ । \newline
62. च॒ मे॒ मे॒ च॒ च॒ मे॒ । \newline
63. मे॒ ऽहो॒रा॒त्रयो॑ रहोरा॒त्रयो᳚र् मे मे ऽहोरा॒त्रयोः᳚ । \newline
64. अ॒हो॒रा॒त्रयो᳚र् वृ॒ष्ट्या वृ॒ष्ट्या ऽहो॑रा॒त्रयो॑रहो रा॒त्रयो᳚र् वृ॒ष्ट्या । \newline
65. अ॒हो॒रा॒त्रयो॒रित्य॑हः - रा॒त्रयोः᳚ । \newline
66. वृ॒ष्ट्या बृ॑हद्रथन्त॒रे बृ॑हद्रथन्त॒रे वृ॒ष्ट्या वृ॒ष्ट्या बृ॑हद्रथन्त॒रे । \newline
67. बृ॒ह॒द्र॒थ॒न्त॒रे च॑ च बृहद्रथन्त॒रे बृ॑हद्रथन्त॒रे च॑ । \newline
68. बृ॒ह॒द्र॒थ॒न्त॒रे इति॑ बृहत् - र॒थ॒न्त॒रे । \newline
69. च॒ मे॒ मे॒ च॒ च॒ मे॒ । \newline
70. मे॒ य॒ज्ञेन॑ य॒ज्ञेन॑ मे मे य॒ज्ञेन॑ । \newline
71. य॒ज्ञेन॑ कल्पेताम् कल्पेतां ॅय॒ज्ञेन॑ य॒ज्ञेन॑ कल्पेताम् । \newline
72. क॒ल्पे॒ता॒मिति॑ कल्पेताम् । \newline

\textbf{Ghana Paata } \newline

1. अ॒ग्निश्च॑ चा॒ग्नि र॒ग्निश्च॑ मे मे चा॒ग्नि र॒ग्निश्च॑ मे । \newline
2. च॒ मे॒ मे॒ च॒ च॒ मे॒ घ॒र्मो घ॒र्मो मे॑ च च मे घ॒र्मः । \newline
3. मे॒ घ॒र्मो घ॒र्मो मे॑ मे घ॒र्मश्च॑ च घ॒र्मो मे॑ मे घ॒र्मश्च॑ । \newline
4. घ॒र्मश्च॑ च घ॒र्मो घ॒र्मश्च॑ मे मे च घ॒र्मो घ॒र्मश्च॑ मे । \newline
5. च॒ मे॒ मे॒ च॒ च॒ मे॒ ऽर्को᳚ ऽर्को मे॑ च च मे॒ ऽर्कः । \newline
6. मे॒ ऽर्को᳚ ऽर्को मे॑ मे॒ ऽर्कश्च॑ चा॒र्को मे॑ मे॒ ऽर्कश्च॑ । \newline
7. अ॒र्कश्च॑ चा॒र्को᳚ ऽर्कश्च॑ मे मे चा॒र्को᳚ ऽर्कश्च॑ मे । \newline
8. च॒ मे॒ मे॒ च॒ च॒ मे॒ सूर्यः॒ सूर्यो॑ मे च च मे॒ सूर्यः॑ । \newline
9. मे॒ सूर्यः॒ सूर्यो॑ मे मे॒ सूर्य॑श्च च॒ सूर्यो॑ मे मे॒ सूर्य॑श्च । \newline
10. सूर्य॑श्च च॒ सूर्यः॒ सूर्य॑श्च मे मे च॒ सूर्यः॒ सूर्य॑श्च मे । \newline
11. च॒ मे॒ मे॒ च॒ च॒ मे॒ प्रा॒णः प्रा॒णो मे॑ च च मे प्रा॒णः । \newline
12. मे॒ प्रा॒णः प्रा॒णो मे॑ मे प्रा॒णश्च॑ च प्रा॒णो मे॑ मे प्रा॒णश्च॑ । \newline
13. प्रा॒णश्च॑ च प्रा॒णः प्रा॒णश्च॑ मे मे च प्रा॒णः प्रा॒णश्च॑ मे । \newline
14. प्रा॒ण इति॑ प्र - अ॒नः । \newline
15. च॒ मे॒ मे॒ च॒ च॒ मे॒ ऽश्व॒मे॒धो᳚ ऽश्वमे॒धो मे॑ च च मे ऽश्वमे॒धः । \newline
16. मे॒ ऽश्व॒मे॒धो᳚ ऽश्वमे॒धो मे॑ मे ऽश्वमे॒धश्च॑ चाश्वमे॒धो मे॑ मे ऽश्वमे॒धश्च॑ । \newline
17. अ॒श्व॒मे॒धश्च॑ चाश्वमे॒धो᳚ ऽश्वमे॒धश्च॑ मे मे चाश्वमे॒धो᳚ ऽश्वमे॒धश्च॑ मे । \newline
18. अ॒श्व॒मे॒ध इत्य॑श्व - मे॒धः । \newline
19. च॒ मे॒ मे॒ च॒ च॒ मे॒ पृ॒थि॒वी पृ॑थि॒वी मे॑ च च मे पृथि॒वी । \newline
20. मे॒ पृ॒थि॒वी पृ॑थि॒वी मे॑ मे पृथि॒वी च॑ च पृथि॒वी मे॑ मे पृथि॒वी च॑ । \newline
21. पृ॒थि॒वी च॑ च पृथि॒वी पृ॑थि॒वी च॑ मे मे च पृथि॒वी पृ॑थि॒वी च॑ मे । \newline
22. च॒ मे॒ मे॒ च॒ च॒ मे ऽदि॑ति॒ रदि॑तिर् मे च च॒ मे ऽदि॑तिः । \newline
23. मे ऽदि॑ति॒ रदि॑तिर् मे॒ मे ऽदि॑तिश्च॒ चादि॑तिर् मे॒ मे ऽदि॑तिश्च । \newline
24. अदि॑तिश्च॒ चादि॑ति॒ रदि॑तिश्च मे मे॒ चादि॑ति॒ रदि॑तिश्च मे । \newline
25. च॒ मे॒ मे॒ च॒ च॒ मे॒ दिति॒र् दिति॑र् मे च च मे॒ दितिः॑ । \newline
26. मे॒ दिति॒र् दिति॑र् मे मे॒ दिति॑श्च च॒ दिति॑र् मे मे॒ दिति॑श्च । \newline
27. दिति॑श्च च॒ दिति॒र् दिति॑श्च मे मे च॒ दिति॒र् दिति॑श्च मे । \newline
28. च॒ मे॒ मे॒ च॒ च॒ मे॒ द्यौर् द्यौर् मे॑ च च मे॒ द्यौः । \newline
29. मे॒ द्यौर् द्यौर् मे॑ मे॒ द्यौश्च॑ च॒ द्यौर् मे॑ मे॒ द्यौश्च॑ । \newline
30. द्यौश्च॑ च॒ द्यौर् द्यौश्च॑ मे मे च॒ द्यौर् द्यौश्च॑ मे । \newline
31. च॒ मे॒ मे॒ च॒ च॒ मे॒ शक्व॑रीः॒ शक्व॑रीर् मे च च मे॒ शक्व॑रीः । \newline
32. मे॒ शक्व॑रीः॒ शक्व॑रीर् मे मे॒ शक्व॑री र॒ङ्गुल॑यो॒ ऽङ्गुल॑यः॒ शक्व॑रीर् मे मे॒ शक्व॑री र॒ङ्गुल॑यः । \newline
33. शक्व॑री र॒ङ्गुल॑यो॒ ऽङ्गुल॑यः॒ शक्व॑रीः॒ शक्व॑री र॒ङ्गुल॑यो॒ दिशो॒ दिशो॒ ऽङ्गुल॑यः॒ शक्व॑रीः॒ शक्व॑री र॒ङ्गुल॑यो॒ दिशः॑ । \newline
34. अ॒ङ्गुल॑यो॒ दिशो॒ दिशो॒ ऽङ्गुल॑यो॒ ऽङ्गुल॑यो॒ दिश॑श्च च॒ दिशो॒ ऽङ्गुल॑यो॒ ऽङ्गुल॑यो॒ दिश॑श्च । \newline
35. दिश॑श्च च॒ दिशो॒ दिश॑श्च मे मे च॒ दिशो॒ दिश॑श्च मे । \newline
36. च॒ मे॒ मे॒ च॒ च॒ मे॒ य॒ज्ञेन॑ य॒ज्ञेन॑ मे च च मे य॒ज्ञेन॑ । \newline
37. मे॒ य॒ज्ञेन॑ य॒ज्ञेन॑ मे मे य॒ज्ञेन॑ कल्पन्ताम् कल्पन्तां ॅय॒ज्ञेन॑ मे मे य॒ज्ञेन॑ कल्पन्ताम् । \newline
38. य॒ज्ञेन॑ कल्पन्ताम् कल्पन्तां ॅय॒ज्ञेन॑ य॒ज्ञेन॑ कल्पन्ता॒ मृगृक् क॑ल्पन्तां ॅय॒ज्ञेन॑ य॒ज्ञेन॑ कल्पन्ता॒ मृक् । \newline
39. क॒ल्प॒न्ता॒ मृगृक् क॑ल्पन्ताम् कल्पन्ता॒ मृक् च॒ चर्‌क् क॑ल्पन्ताम् कल्पन्ता॒ मृक् च॑ । \newline
40. ऋक् च॒ चर्‌ गृक् च॑ मे मे॒ चर्‌ गृक् च॑ मे । \newline
41. च॒ मे॒ मे॒ च॒ च॒ मे॒ साम॒ साम॑ मे च च मे॒ साम॑ । \newline
42. मे॒ साम॒ साम॑ मे मे॒ साम॑ च च॒ साम॑ मे मे॒ साम॑ च । \newline
43. साम॑ च च॒ साम॒ साम॑ च मे मे च॒ साम॒ साम॑ च मे । \newline
44. च॒ मे॒ मे॒ च॒ च॒ मे॒ स्तोमः॒ स्तोमो॑ मे च च मे॒ स्तोमः॑ । \newline
45. मे॒ स्तोमः॒ स्तोमो॑ मे मे॒ स्तोम॑श्च च॒ स्तोमो॑ मे मे॒ स्तोम॑श्च । \newline
46. स्तोम॑श्च च॒ स्तोमः॒ स्तोम॑श्च मे मे च॒ स्तोमः॒ स्तोम॑श्च मे । \newline
47. च॒ मे॒ मे॒ च॒ च॒ मे॒ यजु॒र् यजु॑र् मे च च मे॒ यजुः॑ । \newline
48. मे॒ यजु॒र् यजु॑र् मे मे॒ यजु॑श्च च॒ यजु॑र् मे मे॒ यजु॑श्च । \newline
49. यजु॑श्च च॒ यजु॒र् यजु॑श्च मे मे च॒ यजु॒र् यजु॑श्च मे । \newline
50. च॒ मे॒ मे॒ च॒ च॒ मे॒ दी॒क्षा दी॒क्षा मे॑ च च मे दी॒क्षा । \newline
51. मे॒ दी॒क्षा दी॒क्षा मे॑ मे दी॒क्षा च॑ च दी॒क्षा मे॑ मे दी॒क्षा च॑ । \newline
52. दी॒क्षा च॑ च दी॒क्षा दी॒क्षा च॑ मे मे च दी॒क्षा दी॒क्षा च॑ मे । \newline
53. च॒ मे॒ मे॒ च॒ च॒ मे॒ तप॒ स्तपो॑ मे च च मे॒ तपः॑ । \newline
54. मे॒ तप॒ स्तपो॑ मे मे॒ तप॑श्च च॒ तपो॑ मे मे॒ तप॑श्च । \newline
55. तप॑श्च च॒ तप॒ स्तप॑श्च मे मे च॒ तप॒ स्तप॑श्च मे । \newline
56. च॒ मे॒ मे॒ च॒ च॒ म॒ ऋ॒तुर्. ऋ॒तुर् मे॑ च च म ऋ॒तुः । \newline
57. म॒ ऋ॒तुर्. ऋ॒तुर् मे॑ म ऋ॒तुश्च॑ च॒ र्‌तुर् मे॑ म ऋ॒तुश्च॑ । \newline
58. ऋ॒तुश्च॑ च॒ र्‌तुर्. ऋ॒तुश्च॑ मे मे च॒ र्‌तुर्. ऋ॒तुश्च॑ मे । \newline
59. च॒ मे॒ मे॒ च॒ च॒ मे॒ व्र॒तं ॅव्र॒तम् मे॑ च च मे व्र॒तम् । \newline
60. मे॒ व्र॒तं ॅव्र॒तम् मे॑ मे व्र॒तम् च॑ च व्र॒तम् मे॑ मे व्र॒तम् च॑ । \newline
61. व्र॒तम् च॑ च व्र॒तं ॅव्र॒तम् च॑ मे मे च व्र॒तं ॅव्र॒तम् च॑ मे । \newline
62. च॒ मे॒ मे॒ च॒ च॒ मे॒ ऽहो॒रा॒त्रयो॑ रहोरा॒त्रयो᳚र् मे च च मे ऽहोरा॒त्रयोः᳚ । \newline
63. मे॒ ऽहो॒रा॒त्रयो॑ रहोरा॒त्रयो᳚र् मे मे ऽहोरा॒त्रयो᳚र् वृ॒ष्ट्या वृ॒ष्ट्या ऽहो॑रा॒त्रयो᳚र् मे मे ऽहोरा॒त्रयो᳚र् वृ॒ष्ट्या । \newline
64. अ॒हो॒रा॒त्रयो᳚र् वृ॒ष्ट्या वृ॒ष्ट्या ऽहो॑रा॒त्रयो॑ रहोरा॒त्रयो᳚र् वृ॒ष्ट्या बृ॑हद्रथन्त॒रे बृ॑हद्रथन्त॒रे वृ॒ष्ट्या ऽहो॑रा॒त्रयो॑ रहोरा॒त्रयो᳚र् वृ॒ष्ट्या बृ॑हद्रथन्त॒रे । \newline
65. अ॒हो॒रा॒त्रयो॒रित्य॑हः - रा॒त्रयोः᳚ । \newline
66. वृ॒ष्ट्या बृ॑हद्रथन्त॒रे बृ॑हद्रथन्त॒रे वृ॒ष्ट्या वृ॒ष्ट्या बृ॑हद्रथन्त॒रे च॑ च बृहद्रथन्त॒रे वृ॒ष्ट्या वृ॒ष्ट्या बृ॑हद्रथन्त॒रे च॑ । \newline
67. बृ॒ह॒द्र॒थ॒न्त॒रे च॑ च बृहद्रथन्त॒रे बृ॑हद्रथन्त॒रे च॑ मे मे च बृहद्रथन्त॒रे बृ॑हद्रथन्त॒रे च॑ मे । \newline
68. बृ॒ह॒द्र॒थ॒न्त॒रे इति॑ बृहत् - र॒थ॒न्त॒रे । \newline
69. च॒ मे॒ मे॒ च॒ च॒ मे॒ य॒ज्ञेन॑ य॒ज्ञेन॑ मे च च मे य॒ज्ञेन॑ । \newline
70. मे॒ य॒ज्ञेन॑ य॒ज्ञेन॑ मे मे य॒ज्ञेन॑ कल्पेताम् कल्पेतां ॅय॒ज्ञेन॑ मे मे य॒ज्ञेन॑ कल्पेताम् । \newline
71. य॒ज्ञेन॑ कल्पेताम् कल्पेतां ॅय॒ज्ञेन॑ य॒ज्ञेन॑ कल्पेताम् । \newline
72. क॒ल्पे॒ता॒मिति॑ कल्पेताम् । \newline
\pagebreak
\markright{ TS 4.7.10.1  \hfill https://www.vedavms.in \hfill}

\section{ TS 4.7.10.1 }

\textbf{TS 4.7.10.1 } \newline
\textbf{Samhita Paata} \newline

गर्भा᳚श्च मे व॒थ्साश्च॑ मे॒ त्र्यवि॑श्च मे त्र्य॒वी च॑ मे दित्य॒वाट् च॑ मे दित्यौ॒ही च॑ मे॒ पञ्चा॑विश्च मे पञ्चा॒वी च॑ मे त्रिव॒थ्सश्च॑ मे त्रिव॒थ्सा च॑ मे तुर्य॒वाट् च॑ मे तुर्यौ॒ही च॑ मे पष्ठ॒वाच् च॑ मे पष्ठौ॒ही च॑ म उ॒क्षा च॑ मे व॒शा च॑ म ऋष॒भश्च ॑- [  ] \newline

\textbf{Pada Paata} \newline

गर्भाः᳚ । च॒ । मे॒ । व॒थ्साः । च॒ । मे॒ । त्र्यवि॒रिति॑ त्रि - अविः॑ । च॒ । मे॒ । त्र्य॒वीति॑ त्रि - अ॒वी । च॒ । मे॒ । दि॒त्य॒वाडिति॑ दित्य - वाट् । च॒ । मे॒ । दि॒त्यौ॒ही । च॒ । मे॒ । पञ्चा॑वि॒रिति॒ पञ्च॑ - अ॒विः॒ । च॒ । मे॒ । प॒ञ्चा॒वीति॑ पञ्च-अ॒वी । च॒ । मे॒ । त्रि॒व॒थ्स इति॑ त्रि-व॒थ्सः । च॒ । मे॒ । त्रि॒व॒थ्सेति॑ त्रि - व॒थ्सा । च॒ । मे॒ । तु॒र्य॒वाडिति॑ तुर्य - वाट् । च॒ । मे॒ । तु॒र्यौ॒ही । च॒ । मे॒ । प॒ष्ठ॒वादिति॑ पष्ठ - वात् । च॒ । मे॒ । प॒ष्ठौ॒ही । च॒ । मे॒ । उ॒क्षा । च॒ । मे॒ । व॒शा । च॒ । मे॒ । ऋ॒ष॒भः । च॒ ।  \newline


\textbf{Krama Paata} \newline

गर्भा᳚श्च । च॒ मे॒ । मे॒ व॒थ्साः । व॒थ्साश्च॑ । च॒ मे॒ । मे॒ त्र्यविः॑ । त्र्यवि॑श्च । त्र्यवि॒रिति॑ त्रि - अविः॑ । च॒ मे॒ । मे॒ त्र्य॒वी । त्र्य॒वी च॑ । त्र्य॒वीति॑ त्रि - अ॒वी । च॒ मे॒ । मे॒ दि॒त्य॒वाट् । दि॒त्य॒वाट् च॑ । दि॒त्य॒वाडिति॑ दित्य - वाट् । च॒ मे॒ । मे॒ दि॒त्यौ॒ही । दि॒त्यौ॒ही च॑ । च॒ मे॒ । मे॒ पञ्चा॑विः । पञ्चा॑विश्च । पञ्चा॑वि॒रिति॒ पञ्च॑ - अ॒विः॒ । च॒ मे॒ । मे॒ प॒ञ्चा॒वी । प॒ञ्चा॒वी च॑ । प॒ञ्चा॒वीति॑ पञ्च - अ॒वी । च॒ मे॒ । मे॒ त्रि॒व॒थ्सः । त्रि॒व॒थ्सश्च॑ । त्रि॒व॒थ्स इति॑ त्रि - व॒थ्सः । च॒ मे॒ । मे॒ त्रि॒व॒थ्सा । त्रि॒व॒थ्सा च॑ । त्रि॒व॒थ्सेति॑ त्रि - व॒थ्सा । च॒ मे॒ । मे॒ तु॒र्य॒वाट् । तु॒र्य॒वाट् च॑ । तु॒र्य॒वाडिति॑ तुर्य - वाट् । च॒ मे॒ । मे॒ तु॒र्यौ॒ही । तु॒र्यौ॒ही च॑ । च॒ मे॒ । मे॒ प॒ष्ठ॒वात् । प॒ष्ठ॒वाच् च॑ । प॒ष्ठ॒वादिति॑ पष्ठ - वात् । च॒ मे॒ । मे॒ प॒ष्ठौ॒ही । प॒ष्ठौ॒ही च॑ । च॒ मे॒ । म॒ उ॒क्षा । उ॒क्षा च॑ । च॒ मे॒ । मे॒ व॒शा । व॒शा च॑ । च॒ मे॒ । म॒ ऋ॒ष॒भः । ऋ॒ष॒भश्च॑ ( ) । च॒ मे॒ \newline

\textbf{Jatai Paata} \newline

1. गर्भा᳚श्च च॒ गर्भा॒ गर्भा᳚श्च । \newline
2. च॒ मे॒ मे॒ च॒ च॒ मे॒ । \newline
3. मे॒ व॒थ्सा व॒थ्सा मे॑ मे व॒थ्साः । \newline
4. व॒थ्साश्च॑ च व॒थ्सा व॒थ्साश्च॑ । \newline
5. च॒ मे॒ मे॒ च॒ च॒ मे॒ । \newline
6. मे॒ त्र्यवि॒ स्त्र्यवि॑र् मे मे॒ त्र्यविः॑ । \newline
7. त्र्यवि॑श्च च॒ त्र्यवि॒ स्त्र्यवि॑श्च । \newline
8. त्र्यवि॒रिति॑ त्रि - अविः॑ । \newline
9. च॒ मे॒ मे॒ च॒ च॒ मे॒ । \newline
10. मे॒ त्र्य॒वी त्र्य॒वी मे॑ मे त्र्य॒वी । \newline
11. त्र्य॒वी च॑ च त्र्य॒वी त्र्य॒वी च॑ । \newline
12. त्र्य॒वीति॑ त्रि - अ॒वी । \newline
13. च॒ मे॒ मे॒ च॒ च॒ मे॒ । \newline
14. मे॒ दि॒त्य॒वाड् दि॑त्य॒वाण् मे॑ मे दित्य॒वाट् । \newline
15. दि॒त्य॒वाट् च॑ च दित्य॒वाड् दि॑त्य॒वाट् च॑ । \newline
16. दि॒त्य॒वाडिति॑ दित्य - वाट् । \newline
17. च॒ मे॒ मे॒ च॒ च॒ मे॒ । \newline
18. मे॒ दि॒त्यौ॒ही दि॑त्यौ॒ही मे॑ मे दित्यौ॒ही । \newline
19. दि॒त्यौ॒ही च॑ च दित्यौ॒ही दि॑त्यौ॒ही च॑ । \newline
20. च॒ मे॒ मे॒ च॒ च॒ मे॒ । \newline
21. मे॒ पञ्चा॑विः॒ पञ्चा॑विर् मे मे॒ पञ्चा॑विः । \newline
22. पञ्चा॑विश्च च॒ पञ्चा॑विः॒ पञ्चा॑विश्च । \newline
23. पञ्चा॑वि॒रिति॒ पञ्च॑ - अ॒विः॒ । \newline
24. च॒ मे॒ मे॒ च॒ च॒ मे॒ । \newline
25. मे॒ प॒ञ्चा॒वी प॑ञ्चा॒वी मे॑ मे पञ्चा॒वी । \newline
26. प॒ञ्चा॒वी च॑ च पञ्चा॒वी प॑ञ्चा॒वी च॑ । \newline
27. प॒ञ्चा॒वीति॑ पञ्च - अ॒वी । \newline
28. च॒ मे॒ मे॒ च॒ च॒ मे॒ । \newline
29. मे॒ त्रि॒व॒थ्स स्त्रि॑व॒थ्सो मे॑ मे त्रिव॒थ्सः । \newline
30. त्रि॒व॒थ्सश्च॑ च त्रिव॒थ्स स्त्रि॑व॒थ्सश्च॑ । \newline
31. त्रि॒व॒थ्स इति॑ त्रि - व॒थ्सः । \newline
32. च॒ मे॒ मे॒ च॒ च॒ मे॒ । \newline
33. मे॒ त्रि॒व॒थ्सा त्रि॑व॒थ्सा मे॑ मे त्रिव॒थ्सा । \newline
34. त्रि॒व॒थ्सा च॑ च त्रिव॒थ्सा त्रि॑व॒थ्सा च॑ । \newline
35. त्रि॒व॒थ्सेति॑ त्रि - व॒थ्सा । \newline
36. च॒ मे॒ मे॒ च॒ च॒ मे॒ । \newline
37. मे॒ तु॒र्य॒वाट् तु॑र्य॒वाण् मे॑ मे तुर्य॒वाट् । \newline
38. तु॒र्य॒वाट् च॑ च तुर्य॒वाट् तु॑र्य॒वाट् च॑ । \newline
39. तु॒र्य॒वाडिति॑ तुर्य - वाट् । \newline
40. च॒ मे॒ मे॒ च॒ च॒ मे॒ । \newline
41. मे॒ तु॒र्यौ॒ही तु॑र्यौ॒ही मे॑ मे तुर्यौ॒ही । \newline
42. तु॒र्यौ॒ही च॑ च तुर्यौ॒ही तु॑र्यौ॒ही च॑ । \newline
43. च॒ मे॒ मे॒ च॒ च॒ मे॒ । \newline
44. मे॒ प॒ष्ठ॒वात् प॑ष्ठ॒वान् मे॑ मे पष्ठ॒वात् । \newline
45. प॒ष्ठ॒वाच् च॑ च पष्ठ॒वात् प॑ष्ठ॒वाच् च॑ । \newline
46. प॒ष्ठ॒वादिति॑ पष्ठ - वात् । \newline
47. च॒ मे॒ मे॒ च॒ च॒ मे॒ । \newline
48. मे॒ प॒ष्ठौ॒ही प॑ष्ठौ॒ही मे॑ मे पष्ठौ॒ही । \newline
49. प॒ष्ठौ॒ही च॑ च पष्ठौ॒ही प॑ष्ठौ॒ही च॑ । \newline
50. च॒ मे॒ मे॒ च॒ च॒ मे॒ । \newline
51. म॒ उ॒क्षोक्षा मे॑ म उ॒क्षा । \newline
52. उ॒क्षा च॑ चो॒क्षोक्षा च॑ । \newline
53. च॒ मे॒ मे॒ च॒ च॒ मे॒ । \newline
54. मे॒ व॒शा व॒शा मे॑ मे व॒शा । \newline
55. व॒शा च॑ च व॒शा व॒शा च॑ । \newline
56. च॒ मे॒ मे॒ च॒ च॒ मे॒ । \newline
57. म॒ ऋ॒ष॒भ ऋ॑ष॒भो मे॑ म ऋष॒भः । \newline
58. ऋ॒ष॒भश्च॑ चर्.ष॒भ ऋ॑ष॒भश्च॑ । \newline
59. च॒ मे॒ मे॒ च॒ च॒ मे॒ । \newline

\textbf{Ghana Paata } \newline

1. गर्भा᳚श्च च॒ गर्भा॒ गर्भा᳚श्च मे मे च॒ गर्भा॒ गर्भा᳚श्च मे । \newline
2. च॒ मे॒ मे॒ च॒ च॒ मे॒ व॒थ्सा व॒थ्सा मे॑ च च मे व॒थ्साः । \newline
3. मे॒ व॒थ्सा व॒थ्सा मे॑ मे व॒थ्साश्च॑ च व॒थ्सा मे॑ मे व॒थ्साश्च॑ । \newline
4. व॒थ्साश्च॑ च व॒थ्सा व॒थ्साश्च॑ मे मे च व॒थ्सा व॒थ्साश्च॑ मे । \newline
5. च॒ मे॒ मे॒ च॒ च॒ मे॒ त्र्यवि॒ स्त्र्यवि॑र् मे च च मे॒ त्र्यविः॑ । \newline
6. मे॒ त्र्यवि॒ स्त्र्यवि॑र् मे मे॒ त्र्यवि॑श्च च॒ त्र्यवि॑र् मे मे॒ त्र्यवि॑श्च । \newline
7. त्र्यवि॑श्च च॒ त्र्यवि॒ स्त्र्यवि॑श्च मे मे च॒ त्र्यवि॒ स्त्र्यवि॑श्च मे । \newline
8. त्र्यवि॒रिति॑ त्रि - अविः॑ । \newline
9. च॒ मे॒ मे॒ च॒ च॒ मे॒ त्र्य॒वी त्र्य॒वी मे॑ च च मे त्र्य॒वी । \newline
10. मे॒ त्र्य॒वी त्र्य॒वी मे॑ मे त्र्य॒वी च॑ च त्र्य॒वी मे॑ मे त्र्य॒वी च॑ । \newline
11. त्र्य॒वी च॑ च त्र्य॒वी त्र्य॒वी च॑ मे मे च त्र्य॒वी त्र्य॒वी च॑ मे । \newline
12. त्र्य॒वीति॑ त्रि - अ॒वी । \newline
13. च॒ मे॒ मे॒ च॒ च॒ मे॒ दि॒त्य॒वाड् दि॑त्य॒वाण् मे॑ च च मे दित्य॒वाट् । \newline
14. मे॒ दि॒त्य॒वाड् दि॑त्य॒वाण् मे॑ मे दित्य॒वाट् च॑ च दित्य॒वाण् मे॑ मे दित्य॒वाट् च॑ । \newline
15. दि॒त्य॒वाट् च॑ च दित्य॒वाड् दि॑त्य॒वाट् च॑ मे मे च दित्य॒वाड् दि॑त्य॒वाट् च॑ मे । \newline
16. दि॒त्य॒वाडिति॑ दित्य - वाट् । \newline
17. च॒ मे॒ मे॒ च॒ च॒ मे॒ दि॒त्यौ॒ही दि॑त्यौ॒ही मे॑ च च मे दित्यौ॒ही । \newline
18. मे॒ दि॒त्यौ॒ही दि॑त्यौ॒ही मे॑ मे दित्यौ॒ही च॑ च दित्यौ॒ही मे॑ मे दित्यौ॒ही च॑ । \newline
19. दि॒त्यौ॒ही च॑ च दित्यौ॒ही दि॑त्यौ॒ही च॑ मे मे च दित्यौ॒ही दि॑त्यौ॒ही च॑ मे । \newline
20. च॒ मे॒ मे॒ च॒ च॒ मे॒ पञ्चा॑विः॒ पञ्चा॑विर् मे च च मे॒ पञ्चा॑विः । \newline
21. मे॒ पञ्चा॑विः॒ पञ्चा॑विर् मे मे॒ पञ्चा॑विश्च च॒ पञ्चा॑विर् मे मे॒ पञ्चा॑विश्च । \newline
22. पञ्चा॑विश्च च॒ पञ्चा॑विः॒ पञ्चा॑विश्च मे मे च॒ पञ्चा॑विः॒ पञ्चा॑विश्च मे । \newline
23. पञ्चा॑वि॒रिति॒ पञ्च॑ - अ॒विः॒ । \newline
24. च॒ मे॒ मे॒ च॒ च॒ मे॒ प॒ञ्चा॒वी प॑ञ्चा॒वी मे॑ च च मे पञ्चा॒वी । \newline
25. मे॒ प॒ञ्चा॒वी प॑ञ्चा॒वी मे॑ मे पञ्चा॒वी च॑ च पञ्चा॒वी मे॑ मे पञ्चा॒वी च॑ । \newline
26. प॒ञ्चा॒वी च॑ च पञ्चा॒वी प॑ञ्चा॒वी च॑ मे मे च पञ्चा॒वी प॑ञ्चा॒वी च॑ मे । \newline
27. प॒ञ्चा॒वीति॑ पञ्च - अ॒वी । \newline
28. च॒ मे॒ मे॒ च॒ च॒ मे॒ त्रि॒व॒थ्स स्त्रि॑व॒थ्सो मे॑ च च मे त्रिव॒थ्सः । \newline
29. मे॒ त्रि॒व॒थ्स स्त्रि॑व॒थ्सो मे॑ मे त्रिव॒थ्सश्च॑ च त्रिव॒थ्सो मे॑ मे त्रिव॒थ्सश्च॑ । \newline
30. त्रि॒व॒थ्सश्च॑ च त्रिव॒थ्स स्त्रि॑व॒थ्सश्च॑ मे मे च त्रिव॒थ्स स्त्रि॑व॒थ्सश्च॑ मे । \newline
31. त्रि॒व॒थ्स इति॑ त्रि - व॒थ्सः । \newline
32. च॒ मे॒ मे॒ च॒ च॒ मे॒ त्रि॒व॒थ्सा त्रि॑व॒थ्सा मे॑ च च मे त्रिव॒थ्सा । \newline
33. मे॒ त्रि॒व॒थ्सा त्रि॑व॒थ्सा मे॑ मे त्रिव॒थ्सा च॑ च त्रिव॒थ्सा मे॑ मे त्रिव॒थ्सा च॑ । \newline
34. त्रि॒व॒थ्सा च॑ च त्रिव॒थ्सा त्रि॑व॒थ्सा च॑ मे मे च त्रिव॒थ्सा त्रि॑व॒थ्सा च॑ मे । \newline
35. त्रि॒व॒थ्सेति॑ त्रि - व॒थ्सा । \newline
36. च॒ मे॒ मे॒ च॒ च॒ मे॒ तु॒र्य॒वाट् तु॑र्य॒वाण् मे॑ च च मे तुर्य॒वाट् । \newline
37. मे॒ तु॒र्य॒वाट् तु॑र्य॒वाण् मे॑ मे तुर्य॒वाट् च॑ च तुर्य॒वाण् मे॑ मे तुर्य॒वाट् च॑ । \newline
38. तु॒र्य॒वाट् च॑ च तुर्य॒वाट् तु॑र्य॒वाट् च॑ मे मे च तुर्य॒वाट् तु॑र्य॒वाट् च॑ मे । \newline
39. तु॒र्य॒वाडिति॑ तुर्य - वाट् । \newline
40. च॒ मे॒ मे॒ च॒ च॒ मे॒ तु॒र्यौ॒ही तु॑र्यौ॒ही मे॑ च च मे तुर्यौ॒ही । \newline
41. मे॒ तु॒र्यौ॒ही तु॑र्यौ॒ही मे॑ मे तुर्यौ॒ही च॑ च तुर्यौ॒ही मे॑ मे तुर्यौ॒ही च॑ । \newline
42. तु॒र्यौ॒ही च॑ च तुर्यौ॒ही तु॑र्यौ॒ही च॑ मे मे च तुर्यौ॒ही तु॑र्यौ॒ही च॑ मे । \newline
43. च॒ मे॒ मे॒ च॒ च॒ मे॒ प॒ष्ठ॒वात् प॑ष्ठ॒वान् मे॑ च च मे पष्ठ॒वात् । \newline
44. मे॒ प॒ष्ठ॒वात् प॑ष्ठ॒वान् मे॑ मे पष्ठ॒वाच् च॑ च पष्ठ॒वान् मे॑ मे पष्ठ॒वाच् च॑ । \newline
45. प॒ष्ठ॒वाच् च॑ च पष्ठ॒वात् प॑ष्ठ॒वाच् च॑ मे मे च पष्ठ॒वात् प॑ष्ठ॒वाच् च॑ मे । \newline
46. प॒ष्ठ॒वादिति॑ पष्ठ - वात् । \newline
47. च॒ मे॒ मे॒ च॒ च॒ मे॒ प॒ष्ठौ॒ही प॑ष्ठौ॒ही मे॑ च च मे पष्ठौ॒ही । \newline
48. मे॒ प॒ष्ठौ॒ही प॑ष्ठौ॒ही मे॑ मे पष्ठौ॒ही च॑ च पष्ठौ॒ही मे॑ मे पष्ठौ॒ही च॑ । \newline
49. प॒ष्ठौ॒ही च॑ च पष्ठौ॒ही प॑ष्ठौ॒ही च॑ मे मे च पष्ठौ॒ही प॑ष्ठौ॒ही च॑ मे । \newline
50. च॒ मे॒ मे॒ च॒ च॒ म॒ उ॒क्षोक्षा मे॑ च च म उ॒क्षा । \newline
51. म॒ उ॒क्षोक्षा मे॑ म उ॒क्षा च॑ चो॒क्षा मे॑ म उ॒क्षा च॑ । \newline
52. उ॒क्षा च॑ चो॒क्षोक्षा च॑ मे मे चो॒क्षोक्षा च॑ मे । \newline
53. च॒ मे॒ मे॒ च॒ च॒ मे॒ व॒शा व॒शा मे॑ च च मे व॒शा । \newline
54. मे॒ व॒शा व॒शा मे॑ मे व॒शा च॑ च व॒शा मे॑ मे व॒शा च॑ । \newline
55. व॒शा च॑ च व॒शा व॒शा च॑ मे मे च व॒शा व॒शा च॑ मे । \newline
56. च॒ मे॒ मे॒ च॒ च॒ म॒ ऋ॒ष॒भ ऋ॑ष॒भो मे॑ च च म ऋष॒भः । \newline
57. म॒ ऋ॒ष॒भ ऋ॑ष॒भो मे॑ म ऋष॒भश्च॑ च र्.ष॒भो मे॑ म ऋष॒भश्च॑ । \newline
58. ऋ॒ष॒भश्च॑ च र्.ष॒भ ऋ॑ष॒भश्च॑ मे मे च र्.ष॒भ ऋ॑ष॒भश्च॑ मे । \newline
59. च॒ मे॒ मे॒ च॒ च॒ मे॒ वे॒हद् वे॒हन् मे॑ च च मे वे॒हत् । \newline
\pagebreak
\markright{ TS 4.7.10.2  \hfill https://www.vedavms.in \hfill}

\section{ TS 4.7.10.2 }

\textbf{TS 4.7.10.2 } \newline
\textbf{Samhita Paata} \newline

मे वे॒हच्च॑ मे ऽन॒ड्वान् च॑ मे धे॒नुश्च॑ म॒ आयु॑र्य॒ज्ञेन॑ कल्पतां प्रा॒णो य॒ज्ञेन॑ कल्पता-मपा॒नो य॒ज्ञेन॑ कल्पतां ॅव्या॒नो य॒ज्ञेन॑ कल्पतां॒ चक्षु॑-र्य॒ज्ञेन॑ कल्पताꣳ॒॒ श्रोत्रं॑ ॅय॒ज्ञेन॑ कल्पतां॒ मनो॑ य॒ज्ञेन॑ कल्पतां॒ ॅवाग् य॒ज्ञेन॑ कल्पता-मा॒त्मा य॒ज्ञेन॑ कल्पतां ॅय॒ज्ञो य॒ज्ञेन॑ कल्पतां ॥ \newline

\textbf{Pada Paata} \newline

म॒ । वे॒हत् । च॒ । मे॒ । अ॒न॒ड्वान् । च॒ । मे॒ । धे॒नुः । च॒ । मे॒ । आयुः॑ । य॒ज्ञेन॑ । क॒ल्प॒ता॒म् । प्रा॒ण इति॑ प्र - अ॒नः । य॒ज्ञेन॑ । क॒ल्प॒ता॒म् । अ॒पा॒न इत्य॑प - अ॒नः । य॒ज्ञेन॑ । क॒ल्प॒ता॒म् । व्या॒न इति॑ वि - अ॒नः । य॒ज्ञेन॑ । क॒ल्प॒ता॒म् । चक्षुः॑ । य॒ज्ञेन॑ । क॒ल्प॒ता॒म् । श्रोत्र᳚म् । य॒ज्ञेन॑ । क॒ल्प॒ता॒म् । मनः॑ । य॒ज्ञेन॑ । क॒ल्प॒ता॒म् । वाक् । य॒ज्ञेन॑ । क॒ल्प॒ता॒म् । आ॒त्मा । य॒ज्ञेन॑ । क॒ल्प॒ता॒म् । य॒ज्ञ्ः । य॒ज्ञेन॑ । क॒ल्प॒ता॒म् ॥  \newline


\textbf{Krama Paata} \newline

मे॒ वे॒हत् । वे॒हच् च॑ । च॒ मे॒ । मे॒ऽन॒ड्वान् । अ॒न॒ड्वान् च॑ । च॒ मे॒ । मे॒ धे॒नुः । धे॒नुश्च॑ । च॒ मे॒ । म॒ आयुः॑ । आयु॑र् य॒ज्ञेन॑ । य॒ज्ञेन॑ कल्पताम् । क॒ल्प॒ता॒म् प्रा॒णः । प्रा॒णो य॒ज्ञेन॑ । प्रा॒ण इति॑ प्र - अ॒नः । य॒ज्ञेन॑ कल्पताम् । क॒ल्प॒ता॒म॒पा॒नः । अ॒पा॒नो य॒ज्ञेन॑ । अ॒पा॒न इत्य॑प - अ॒नः । य॒ज्ञेन॑ कल्पताम् । क॒ल्प॒ता॒म् ॅव्या॒नः । व्या॒नो य॒ज्ञेन॑ । व्या॒न इति॑ वि - अ॒नः । य॒ज्ञेन॑ कल्पताम् । क॒ल्प॒ता॒म् चक्षुः॑ । चक्षु॑र् य॒ज्ञेन॑ । य॒ज्ञेन॑ कल्पताम् । क॒ल्प॒ताꣳ॒॒ श्रोत्र᳚म् । श्रोत्र॑म् ॅय॒ज्ञेन॑ । य॒ज्ञेन॑ कल्पताम् । क॒ल्प॒ता॒म् मनः॑ । मनो॑ य॒ज्ञेन॑ । य॒ज्ञेन॑ कल्पताम् । क॒ल्प॒ता॒म् ॅवाक् । वाग् य॒ज्ञेन॑ । य॒ज्ञेन॑ कल्पताम् । क॒ल्प॒ता॒मा॒त्मा । आ॒त्मा य॒ज्ञेन॑ । य॒ज्ञेन॑ कल्पताम् । क॒ल्प॒ता॒म् ॅय॒ज्ञ्ः । य॒ज्ञो य॒ज्ञेन॑ । य॒ज्ञेन॑ कल्पताम् । क॒ल्प॒ता॒मिति॑ कल्पताम् । \newline

\textbf{Jatai Paata} \newline

1. मे॒ वे॒हद् वे॒हन् मे॑ मे वे॒हत् । \newline
2. वे॒हच् च॑ च वे॒हद् वे॒हच् च॑ । \newline
3. च॒ मे॒ मे॒ च॒ च॒ मे॒ । \newline
4. मे॒ ऽन॒ड्वा न॑न॒ड्वान् मे॑ मे ऽन॒ड्वान् । \newline
5. अ॒न॒ड्वान् च॑ चान॒ड्वा न॑न॒ड्वान् च॑ । \newline
6. च॒ मे॒ मे॒ च॒ च॒ मे॒ । \newline
7. मे॒ धे॒नुर् धे॒नुर् मे॑ मे धे॒नुः । \newline
8. धे॒नुश्च॑ च धे॒नुर् धे॒नुश्च॑ । \newline
9. च॒ मे॒ मे॒ च॒ च॒ मे॒ । \newline
10. म॒ आयु॒ रायु॑र् मे म॒ आयुः॑ । \newline
11. आयु॑र् य॒ज्ञेन॑ य॒ज्ञे नायु॒ रायु॑र् य॒ज्ञेन॑ । \newline
12. य॒ज्ञेन॑ कल्पताम् कल्पतां ॅय॒ज्ञेन॑ य॒ज्ञेन॑ कल्पताम् । \newline
13. क॒ल्प॒ता॒म् प्रा॒णः प्रा॒णः क॑ल्पताम् कल्पताम् प्रा॒णः । \newline
14. प्रा॒णो य॒ज्ञेन॑ य॒ज्ञेन॑ प्रा॒णः प्रा॒णो य॒ज्ञेन॑ । \newline
15. प्रा॒ण इति॑ प्र - अ॒नः । \newline
16. य॒ज्ञेन॑ कल्पताम् कल्पतां ॅय॒ज्ञेन॑ य॒ज्ञेन॑ कल्पताम् । \newline
17. क॒ल्प॒ता॒ म॒पा॒नो अ॑पा॒नः क॑ल्पताम् कल्पता मपा॒नः । \newline
18. अ॒पा॒नो य॒ज्ञेन॑ य॒ज्ञेना॑ पा॒नो अ॑पा॒नो य॒ज्ञेन॑ । \newline
19. अ॒पा॒न इत्य॑प - अ॒नः । \newline
20. य॒ज्ञेन॑ कल्पताम् कल्पतां ॅय॒ज्ञेन॑ य॒ज्ञेन॑ कल्पताम् । \newline
21. क॒ल्प॒तां॒ ॅव्या॒नो व्या॒नः क॑ल्पताम् कल्पतां ॅव्या॒नः । \newline
22. व्या॒नो य॒ज्ञेन॑ य॒ज्ञेन॑ व्या॒नो व्या॒नो य॒ज्ञेन॑ । \newline
23. व्या॒न इति॑ वि - अ॒नः । \newline
24. य॒ज्ञेन॑ कल्पताम् कल्पतां ॅय॒ज्ञेन॑ य॒ज्ञेन॑ कल्पताम् । \newline
25. क॒ल्प॒ता॒म् चक्षु॒ श्चक्षुः॑ कल्पताम् कल्पता॒म् चक्षुः॑ । \newline
26. चक्षु॑र् य॒ज्ञेन॑ य॒ज्ञेन॒ चक्षु॒ श्चक्षु॑र् य॒ज्ञेन॑ । \newline
27. य॒ज्ञेन॑ कल्पताम् कल्पतां ॅय॒ज्ञेन॑ य॒ज्ञेन॑ कल्पताम् । \newline
28. क॒ल्प॒ता॒(ग्ग्॒) श्रोत्र॒(ग्ग्॒) श्रोत्र॑म् कल्पताम् कल्पता॒(ग्ग्॒) श्रोत्र᳚म् । \newline
29. श्रोत्रं॑ ॅय॒ज्ञेन॑ य॒ज्ञेन॒ श्रोत्र॒(ग्ग्॒) श्रोत्रं॑ ॅय॒ज्ञेन॑ । \newline
30. य॒ज्ञेन॑ कल्पताम् कल्पतां ॅय॒ज्ञेन॑ य॒ज्ञेन॑ कल्पताम् । \newline
31. क॒ल्प॒ता॒म् मनो॒ मनः॑ कल्पताम् कल्पता॒म् मनः॑ । \newline
32. मनो॑ य॒ज्ञेन॑ य॒ज्ञेन॒ मनो॒ मनो॑ य॒ज्ञेन॑ । \newline
33. य॒ज्ञेन॑ कल्पताम् कल्पतां ॅय॒ज्ञेन॑ य॒ज्ञेन॑ कल्पताम् । \newline
34. क॒ल्प॒तां॒ ॅवाग् वाक् क॑ल्पताम् कल्पतां॒ ॅवाक् । \newline
35. वाग् य॒ज्ञेन॑ य॒ज्ञेन॒ वाग् वाग् य॒ज्ञेन॑ । \newline
36. य॒ज्ञेन॑ कल्पताम् कल्पतां ॅय॒ज्ञेन॑ य॒ज्ञेन॑ कल्पताम् । \newline
37. क॒ल्प॒ता॒ मा॒त्मा ऽऽत्मा क॑ल्पताम् कल्पता मा॒त्मा । \newline
38. आ॒त्मा य॒ज्ञेन॑ य॒ज्ञेना॒त्मा ऽऽत्मा य॒ज्ञेन॑ । \newline
39. य॒ज्ञेन॑ कल्पताम् कल्पतां ॅय॒ज्ञेन॑ य॒ज्ञेन॑ कल्पताम् । \newline
40. क॒ल्प॒तां॒ ॅय॒ज्ञो य॒ज्ञ्ः क॑ल्पताम् कल्पतां ॅय॒ज्ञ्ः । \newline
41. य॒ज्ञो य॒ज्ञेन॑ य॒ज्ञेन॑ य॒ज्ञो य॒ज्ञो य॒ज्ञेन॑ । \newline
42. य॒ज्ञेन॑ कल्पताम् कल्पतां ॅय॒ज्ञेन॑ य॒ज्ञेन॑ कल्पताम् । \newline
43. क॒ल्प॒ता॒मिति॑ कल्पताम् । \newline

\textbf{Ghana Paata } \newline

1. मे॒ वे॒हद् वे॒हन् मे॑ मे वे॒हच् च॑ च वे॒हन् मे॑ मे वे॒हच् च॑ । \newline
2. वे॒हच् च॑ च वे॒हद् वे॒हच् च॑ मे मे च वे॒हद् वे॒हच् च॑ मे । \newline
3. च॒ मे॒ मे॒ च॒ च॒ मे॒ ऽन॒ड्वा न॑न॒ड्वान् मे॑ च च मे ऽन॒ड्वान् । \newline
4. मे॒ ऽन॒ड्वा न॑न॒ड्वान् मे॑ मे ऽन॒ड्वान् च॑ चान॒ड्वान् मे॑ मे ऽन॒ड्वान् च॑ । \newline
5. अ॒न॒ड्वान् च॑ चान॒ड्वा न॑न॒ड्वान् च॑ मे मे चान॒ड्वा न॑न॒ड्वान् च॑ मे । \newline
6. च॒ मे॒ मे॒ च॒ च॒ मे॒ धे॒नुर् धे॒नुर् मे॑ च च मे धे॒नुः । \newline
7. मे॒ धे॒नुर् धे॒नुर् मे॑ मे धे॒नुश्च॑ च धे॒नुर् मे॑ मे धे॒नुश्च॑ । \newline
8. धे॒नुश्च॑ च धे॒नुर् धे॒नुश्च॑ मे मे च धे॒नुर् धे॒नुश्च॑ मे । \newline
9. च॒ मे॒ मे॒ च॒ च॒ म॒ आयु॒ रायु॑र् मे च च म॒ आयुः॑ । \newline
10. म॒ आयु॒ रायु॑र् मे म॒ आयु॑र् य॒ज्ञेन॑ य॒ज्ञे नायु॑र् मे म॒ आयु॑र् य॒ज्ञेन॑ । \newline
11. आयु॑र् य॒ज्ञेन॑ य॒ज्ञेनायु॒ रायु॑र् य॒ज्ञेन॑ कल्पताम् कल्पतां ॅय॒ज्ञेनायु॒ रायु॑र् य॒ज्ञेन॑ कल्पताम् । \newline
12. य॒ज्ञेन॑ कल्पताम् कल्पतां ॅय॒ज्ञेन॑ य॒ज्ञेन॑ कल्पताम् प्रा॒णः प्रा॒णः क॑ल्पतां ॅय॒ज्ञेन॑ य॒ज्ञेन॑ कल्पताम् प्रा॒णः । \newline
13. क॒ल्प॒ता॒म् प्रा॒णः प्रा॒णः क॑ल्पताम् कल्पताम् प्रा॒णो य॒ज्ञेन॑ य॒ज्ञेन॑ प्रा॒णः क॑ल्पताम् कल्पताम् प्रा॒णो य॒ज्ञेन॑ । \newline
14. प्रा॒णो य॒ज्ञेन॑ य॒ज्ञेन॑ प्रा॒णः प्रा॒णो य॒ज्ञेन॑ कल्पताम् कल्पतां ॅय॒ज्ञेन॑ प्रा॒णः प्रा॒णो य॒ज्ञेन॑ कल्पताम् । \newline
15. प्रा॒ण इति॑ प्र - अ॒नः । \newline
16. य॒ज्ञेन॑ कल्पताम् कल्पतां ॅय॒ज्ञेन॑ य॒ज्ञेन॑ कल्पता मपा॒नो अ॑पा॒नः क॑ल्पतां ॅय॒ज्ञेन॑ य॒ज्ञेन॑ कल्पता मपा॒नः । \newline
17. क॒ल्प॒ता॒ म॒पा॒नो अ॑पा॒नः क॑ल्पताम् कल्पता मपा॒नो य॒ज्ञेन॑ य॒ज्ञेना॑पा॒नः क॑ल्पताम् कल्पता मपा॒नो य॒ज्ञेन॑ । \newline
18. अ॒पा॒नो य॒ज्ञेन॑ य॒ज्ञेना॑पा॒नो अ॑पा॒नो य॒ज्ञेन॑ कल्पताम् कल्पतां ॅय॒ज्ञेना॑पा॒नो अ॑पा॒नो य॒ज्ञेन॑ कल्पताम् । \newline
19. अ॒पा॒न इत्य॑प - अ॒नः । \newline
20. य॒ज्ञेन॑ कल्पताम् कल्पतां ॅय॒ज्ञेन॑ य॒ज्ञेन॑ कल्पतां ॅव्या॒नो व्या॒नः क॑ल्पतां ॅय॒ज्ञेन॑ य॒ज्ञेन॑ कल्पतां ॅव्या॒नः । \newline
21. क॒ल्प॒तां॒ ॅव्या॒नो व्या॒नः क॑ल्पताम् कल्पतां ॅव्या॒नो य॒ज्ञेन॑ य॒ज्ञेन॑ व्या॒नः क॑ल्पताम् कल्पतां ॅव्या॒नो य॒ज्ञेन॑ । \newline
22. व्या॒नो य॒ज्ञेन॑ य॒ज्ञेन॑ व्या॒नो व्या॒नो य॒ज्ञेन॑ कल्पताम् कल्पतां ॅय॒ज्ञेन॑ व्या॒नो व्या॒नो य॒ज्ञेन॑ कल्पताम् । \newline
23. व्या॒न इति॑ वि - अ॒नः । \newline
24. य॒ज्ञेन॑ कल्पताम् कल्पतां ॅय॒ज्ञेन॑ य॒ज्ञेन॑ कल्पता॒म् चक्षु॒ श्चक्षुः॑ कल्पतां ॅय॒ज्ञेन॑ य॒ज्ञेन॑ कल्पता॒म् चक्षुः॑ । \newline
25. क॒ल्प॒ता॒म् चक्षु॒ श्चक्षुः॑ कल्पताम् कल्पता॒म् चक्षु॑र् य॒ज्ञेन॑ य॒ज्ञेन॒ चक्षुः॑ कल्पताम् कल्पता॒म् चक्षु॑र् य॒ज्ञेन॑ । \newline
26. चक्षु॑र् य॒ज्ञेन॑ य॒ज्ञेन॒ चक्षु॒ श्चक्षु॑र् य॒ज्ञेन॑ कल्पताम् कल्पतां ॅय॒ज्ञेन॒ चक्षु॒ श्चक्षु॑र् य॒ज्ञेन॑ कल्पताम् । \newline
27. य॒ज्ञेन॑ कल्पताम् कल्पतां ॅय॒ज्ञेन॑ य॒ज्ञेन॑ कल्पता॒(ग्ग्॒) श्रोत्र॒(ग्ग्॒) श्रोत्र॑म् कल्पतां ॅय॒ज्ञेन॑ य॒ज्ञेन॑ कल्पता॒(ग्ग्॒) श्रोत्र᳚म् । \newline
28. क॒ल्प॒ता॒(ग्ग्॒) श्रोत्र॒(ग्ग्॒) श्रोत्र॑म् कल्पताम् कल्पता॒(ग्ग्॒) श्रोत्रं॑ ॅय॒ज्ञेन॑ य॒ज्ञेन॒ श्रोत्र॑म् कल्पताम् कल्पता॒(ग्ग्॒) श्रोत्रं॑ ॅय॒ज्ञेन॑ । \newline
29. श्रोत्रं॑ ॅय॒ज्ञेन॑ य॒ज्ञेन॒ श्रोत्र॒(ग्ग्॒) श्रोत्रं॑ ॅय॒ज्ञेन॑ कल्पताम् कल्पतां ॅय॒ज्ञेन॒ श्रोत्र॒(ग्ग्॒) श्रोत्रं॑ ॅय॒ज्ञेन॑ कल्पताम् । \newline
30. य॒ज्ञेन॑ कल्पताम् कल्पतां ॅय॒ज्ञेन॑ य॒ज्ञेन॑ कल्पता॒म् मनो॒ मनः॑ कल्पतां ॅय॒ज्ञेन॑ य॒ज्ञेन॑ कल्पता॒म् मनः॑ । \newline
31. क॒ल्प॒ता॒म् मनो॒ मनः॑ कल्पताम् कल्पता॒म् मनो॑ य॒ज्ञेन॑ य॒ज्ञेन॒ मनः॑ कल्पताम् कल्पता॒म् मनो॑ य॒ज्ञेन॑ । \newline
32. मनो॑ य॒ज्ञेन॑ य॒ज्ञेन॒ मनो॒ मनो॑ य॒ज्ञेन॑ कल्पताम् कल्पतां ॅय॒ज्ञेन॒ मनो॒ मनो॑ य॒ज्ञेन॑ कल्पताम् । \newline
33. य॒ज्ञेन॑ कल्पताम् कल्पतां ॅय॒ज्ञेन॑ य॒ज्ञेन॑ कल्पतां॒ ॅवाग् वाक् क॑ल्पतां ॅय॒ज्ञेन॑ य॒ज्ञेन॑ कल्पतां॒ ॅवाक् । \newline
34. क॒ल्प॒तां॒ ॅवाग् वाक् क॑ल्पताम् कल्पतां॒ ॅवाग् य॒ज्ञेन॑ य॒ज्ञेन॒ वाक् क॑ल्पताम् कल्पतां॒ ॅवाग् य॒ज्ञेन॑ । \newline
35. वाग् य॒ज्ञेन॑ य॒ज्ञेन॒ वाग् वाग् य॒ज्ञेन॑ कल्पताम् कल्पतां ॅय॒ज्ञेन॒ वाग् वाग् य॒ज्ञेन॑ कल्पताम् । \newline
36. य॒ज्ञेन॑ कल्पताम् कल्पतां ॅय॒ज्ञेन॑ य॒ज्ञेन॑ कल्पता मा॒त्मा ऽऽत्मा क॑ल्पतां ॅय॒ज्ञेन॑ य॒ज्ञेन॑ कल्पता मा॒त्मा । \newline
37. क॒ल्प॒ता॒ मा॒त्मा ऽऽत्मा क॑ल्पताम् कल्पता मा॒त्मा य॒ज्ञेन॑ य॒ज्ञेना॒त्मा क॑ल्पताम् कल्पता मा॒त्मा य॒ज्ञेन॑ । \newline
38. आ॒त्मा य॒ज्ञेन॑ य॒ज्ञेना॒त्मा ऽऽत्मा य॒ज्ञेन॑ कल्पताम् कल्पतां ॅय॒ज्ञेना॒त्मा ऽऽत्मा य॒ज्ञेन॑ कल्पताम् । \newline
39. य॒ज्ञेन॑ कल्पताम् कल्पतां ॅय॒ज्ञेन॑ य॒ज्ञेन॑ कल्पतां ॅय॒ज्ञो य॒ज्ञ्ः क॑ल्पतां ॅय॒ज्ञेन॑ य॒ज्ञेन॑ कल्पतां ॅय॒ज्ञ्ः । \newline
40. क॒ल्प॒तां॒ ॅय॒ज्ञो य॒ज्ञ्ः क॑ल्पताम् कल्पतां ॅय॒ज्ञो य॒ज्ञेन॑ य॒ज्ञेन॑ य॒ज्ञ्ः क॑ल्पताम् कल्पतां ॅय॒ज्ञो य॒ज्ञेन॑ । \newline
41. य॒ज्ञो य॒ज्ञेन॑ य॒ज्ञेन॑ य॒ज्ञो य॒ज्ञो य॒ज्ञेन॑ कल्पताम् कल्पतां ॅय॒ज्ञेन॑ य॒ज्ञो य॒ज्ञो य॒ज्ञेन॑ कल्पताम् । \newline
42. य॒ज्ञेन॑ कल्पताम् कल्पतां ॅय॒ज्ञेन॑ य॒ज्ञेन॑ कल्पताम् । \newline
43. क॒ल्प॒ता॒मिति॑ कल्पताम् । \newline
\pagebreak
\markright{ TS 4.7.11.1  \hfill https://www.vedavms.in \hfill}

\section{ TS 4.7.11.1 }

\textbf{TS 4.7.11.1 } \newline
\textbf{Samhita Paata} \newline

एका॑ च मे ति॒स्रश्च॑ मे॒ पञ्च॑ च मे स॒प्त च॑ मे॒ नव॑ च म॒ एका॑दश च मे॒ त्रयो॑दश च मे॒ पञ्च॑दश च मे स॒प्तद॑श च मे॒ नव॑दश च म॒ एक॑विꣳशतिश्च मे॒ त्रयो॑विꣳशतिश्च मे॒ पञ्च॑विꣳशतिश्च मे स॒प्तविꣳ॑शतिश्च मे॒ नव॑विꣳशतिश्च म॒ एक॑त्रिꣳशच्च मे॒ त्रय॑स्त्रिꣳशच्च- [  ] \newline

\textbf{Pada Paata} \newline

एका᳚ । च॒ । मे॒ । ति॒स्रः । च॒ । मे॒ । पञ्च॑ । च॒ । मे॒ । स॒प्त । च॒ । मे॒ । नव॑ । च॒ । मे॒ । एका॑दश । च॒ । मे॒ । त्रयो॑द॒शेति॒ त्रयः॑ - द॒श॒ । च॒ । मे॒ । पञ्च॑द॒शेति॒ पञ्च॑-द॒श॒ । च॒ । मे॒ । स॒प्तद॒शेति॑ स॒प्त - द॒श॒ । च॒ । मे॒ । नव॑द॒शेति॒ नव॑ - द॒श॒ । च॒ । मे॒ । एक॑विꣳशति॒रित्येक॑ - विꣳ॒॒श॒तिः॒ । च॒ । मे॒ । त्रयो॑विꣳशति॒रिति॒ त्रयः॑-विꣳ॒॒श॒तिः॒ । च॒ । मे॒ । पञ्च॑विꣳशति॒रिति॒ पञ्च॑-विꣳ॒॒श॒तिः॒ । च॒ । मे॒ । स॒प्तविꣳ॑शति॒रिति॑ स॒प्त -विꣳ॒॒श॒तिः॒ । च॒ । मे॒ । नव॑विꣳशति॒रिति॒ नव॑ - विꣳ॒॒श॒तिः॒ । च॒ । मे॒ । एक॑त्रिꣳश॒दित्येक॑ - त्रिꣳ॒॒श॒त् । च॒ । मे॒ । त्रय॑स्त्रिꣳश॒दिति॒ त्रयः॑ - त्रिꣳ॒॒श॒त् । च॒ ।  \newline


\textbf{Krama Paata} \newline

एका॑ च । च॒ मे॒ । मे॒ ति॒स्रः । ति॒स्रश्च॑ । च॒ मे॒ । मे॒ पञ्च॑ । पञ्च॑ च । च॒ मे॒ । म॒ स॒प्त । स॒प्त च॑ । च॒ मे॒ । मे॒ नव॑ । नव॑ च । च॒ मे॒ । म॒ एका॑दश । एका॑दश च । च॒ मे॒ । मे॒ त्रयो॑दश । त्रयो॑दश च । त्रयो॑द॒शेति॒ त्रयः॑ - द॒श॒ । च॒ मे॒ । मे॒ पञ्च॑दश । पञ्च॑दश च । पञ्च॑द॒शेति॒ पञ्च॑ - द॒श॒ । च॒ मे॒ । मे॒ स॒प्तद॑श । स॒प्तद॑श च । स॒प्तद॒शेति॑ स॒प्त - द॒श॒ । च॒ मे॒ । मे॒ नव॑दश । नव॑दश च । नव॑द॒शेति॒ नव॑ - द॒श॒ । च॒ मे॒ । म॒ एक॑विꣳशतिः । एक॑विꣳशतिश्च । एक॑विꣳशति॒रित्येक॑ - विꣳ॒॒श॒तिः॒ । च॒ मे॒ । मे॒ त्रयो॑विꣳशतिः । त्रयो॑विꣳशतिश्च । त्रयो॑विꣳशति॒रिति॒ त्रयः॑ - विꣳ॒॒श॒तिः॒ । च॒ मे॒ । मे॒ पञ्च॑विꣳशतिः । पञ्च॑विꣳशतिश्च । पञ्च॑विꣳशति॒रिति॒ पञ्च॑ - विꣳ॒॒श॒तिः॒ । च॒ मे॒ । मे॒ स॒प्तविꣳ॑शतिः । स॒प्तविꣳ॑शतिश्च । स॒प्तविꣳ॑शति॒रिति॑ स॒प्त - विꣳ॒॒श॒तिः॒ । च॒ मे॒ । मे॒ नव॑विꣳशतिः । नव॑विꣳशतिश्च । नव॑विꣳशति॒रिति॒ नव॑ - विꣳ॒॒श॒तिः॒ । च॒ मे॒ । म॒ एक॑त्रिꣳशत् । एक॑त्रिꣳशच् च । एक॑त्रिꣳश॒दित्येक॑ - त्रिꣳ॒॒श॒त्॒ । च॒ मे॒ । मे॒ त्रय॑स्त्रिꣳशत् । त्रय॑स्त्रिꣳशच् च ( ) । त्रय॑स्त्रिꣳश॒दिति॒ त्रयः॑ - त्रिꣳ॒॒श॒त्॒ । च॒ मे॒ \newline

\textbf{Jatai Paata} \newline

1. एका॑ च॒ चैकैका॑ च । \newline
2. च॒ मे॒ मे॒ च॒ च॒ मे॒ । \newline
3. मे॒ ति॒स्र स्ति॒स्रो मे॑ मे ति॒स्रः । \newline
4. ति॒स्रश्च॑ च ति॒स्र स्ति॒स्रश्च॑ । \newline
5. च॒ मे॒ मे॒ च॒ च॒ मे॒ । \newline
6. मे॒ पञ्च॒ पञ्च॑ मे मे॒ पञ्च॑ । \newline
7. पञ्च॑ च च॒ पञ्च॒ पञ्च॑ च । \newline
8. च॒ मे॒ मे॒ च॒ च॒ मे॒ । \newline
9. मे॒ स॒प्त स॒प्त मे॑ मे स॒प्त । \newline
10. स॒प्त च॑ च स॒प्त स॒प्त च॑ । \newline
11. च॒ मे॒ मे॒ च॒ च॒ मे॒ । \newline
12. मे॒ नव॒ नव॑ मे मे॒ नव॑ । \newline
13. नव॑ च च॒ नव॒ नव॑ च । \newline
14. च॒ मे॒ मे॒ च॒ च॒ मे॒ । \newline
15. म॒ एका॑द॒ शैका॑दश मे म॒ एका॑दश । \newline
16. एका॑दश च॒ चैका॑द॒ शैका॑दश च । \newline
17. च॒ मे॒ मे॒ च॒ च॒ मे॒ । \newline
18. मे॒ त्रयो॑दश॒ त्रयो॑दश मे मे॒ त्रयो॑दश । \newline
19. त्रयो॑दश च च॒ त्रयो॑दश॒ त्रयो॑दश च । \newline
20. त्रयो॑द॒शेति॒ त्रयः॑ - द॒श॒ । \newline
21. च॒ मे॒ मे॒ च॒ च॒ मे॒ । \newline
22. मे॒ पञ्च॑दश॒ पञ्च॑दश मे मे॒ पञ्च॑दश । \newline
23. पञ्च॑दश च च॒ पञ्च॑दश॒ पञ्च॑दश च । \newline
24. पञ्च॑द॒शेति॒ पञ्च॑ - द॒श॒ । \newline
25. च॒ मे॒ मे॒ च॒ च॒ मे॒ । \newline
26. मे॒ स॒प्तद॑श स॒प्तद॑श मे मे स॒प्तद॑श । \newline
27. स॒प्तद॑श च च स॒प्तद॑श स॒प्तद॑श च । \newline
28. स॒प्तद॒शेति॑ स॒प्त - द॒श॒ । \newline
29. च॒ मे॒ मे॒ च॒ च॒ मे॒ । \newline
30. मे॒ नव॑दश॒ नव॑दश मे मे॒ नव॑दश । \newline
31. नव॑दश च च॒ नव॑दश॒ नव॑दश च । \newline
32. नव॑द॒शेति॒ नव॑ - द॒श॒ । \newline
33. च॒ मे॒ मे॒ च॒ च॒ मे॒ । \newline
34. म॒ एक॑विꣳशति॒ रेक॑विꣳशतिर् मे म॒ एक॑विꣳशतिः । \newline
35. एक॑विꣳशतिश्च॒ चैक॑विꣳशति॒ रेक॑विꣳशतिश्च । \newline
36. एक॑विꣳशति॒रित्येक॑ - वि॒(ग्म्॒)श॒तिः॒ । \newline
37. च॒ मे॒ मे॒ च॒ च॒ मे॒ । \newline
38. मे॒ त्रयो॑विꣳशति॒ स्त्रयो॑विꣳशतिर् मे मे॒ त्रयो॑विꣳशतिः । \newline
39. त्रयो॑विꣳशतिश्च च॒ त्रयो॑विꣳशति॒ स्त्रयो॑विꣳशतिश्च । \newline
40. त्रयो॑विꣳशति॒रिति॒ त्रयः॑ - वि॒(ग्म्॒)श॒तिः॒ । \newline
41. च॒ मे॒ मे॒ च॒ च॒ मे॒ । \newline
42. मे॒ पञ्च॑विꣳशतिः॒ पञ्च॑विꣳशतिर् मे मे॒ पञ्च॑विꣳशतिः । \newline
43. पञ्च॑विꣳशतिश्च च॒ पञ्च॑विꣳशतिः॒ पञ्च॑विꣳशतिश्च । \newline
44. पञ्च॑विꣳशति॒रिति॒ पञ्च॑ - वि॒(ग्म्॒)श॒तिः॒ । \newline
45. च॒ मे॒ मे॒ च॒ च॒ मे॒ । \newline
46. मे॒ स॒प्तवि(ग्म्॑)शतिः स॒प्तवि(ग्म्॑)शतिर् मे मे स॒प्तवि(ग्म्॑)शतिः । \newline
47. स॒प्तवि(ग्म्॑)शतिश्च च स॒प्तवि(ग्म्॑)शतिः स॒प्तवि(ग्म्॑)शतिश्च । \newline
48. स॒प्तवि(ग्म्॑)शति॒रिति॑ स॒प्त - वि॒(ग्म्॒)श॒तिः॒ । \newline
49. च॒ मे॒ मे॒ च॒ च॒ मे॒ । \newline
50. मे॒ नव॑विꣳशति॒र् नव॑विꣳशतिर् मे मे॒ नव॑विꣳशतिः । \newline
51. नव॑विꣳशतिश्च च॒ नव॑विꣳशति॒र् नव॑विꣳशतिश्च । \newline
52. नव॑विꣳशति॒रिति॒ नव॑ - वि॒(ग्म्॒)श॒तिः॒ । \newline
53. च॒ मे॒ मे॒ च॒ च॒ मे॒ । \newline
54. म॒ एक॑त्रिꣳश॒ देक॑त्रिꣳशन् मे म॒ एक॑त्रिꣳशत् । \newline
55. एक॑त्रिꣳशच् च॒ चैक॑त्रिꣳश॒ देक॑त्रिꣳशच् च । \newline
56. एक॑त्रिꣳश॒दित्येक॑ - त्रि॒(ग्म्॒)श॒त् । \newline
57. च॒ मे॒ मे॒ च॒ च॒ मे॒ । \newline
58. मे॒ त्रय॑स्त्रिꣳश॒त् त्रय॑स्त्रिꣳशन् मे मे॒ त्रय॑स्त्रिꣳशत् । \newline
59. त्रय॑स्त्रिꣳशच् च च॒ त्रय॑स्त्रिꣳश॒त् त्रय॑स्त्रिꣳशच् च । \newline
60. त्रय॑स्त्रिꣳश॒दिति॒ त्रयः॑ - त्रि॒(ग्म्॒)श॒त् । \newline
61. च॒ मे॒ मे॒ च॒ च॒ मे॒ । \newline

\textbf{Ghana Paata } \newline

1. एका॑ च॒ चैकैका॑ च मे मे॒ चैकैका॑ च मे । \newline
2. च॒ मे॒ मे॒ च॒ च॒ मे॒ ति॒स्र स्ति॒स्रो मे॑ च च मे ति॒स्रः । \newline
3. मे॒ ति॒स्र स्ति॒स्रो मे॑ मे ति॒स्रश्च॑ च ति॒स्रो मे॑ मे ति॒स्रश्च॑ । \newline
4. ति॒स्रश्च॑ च ति॒स्र स्ति॒स्रश्च॑ मे मे च ति॒स्र स्ति॒स्रश्च॑ मे । \newline
5. च॒ मे॒ मे॒ च॒ च॒ मे॒ पञ्च॒ पञ्च॑ मे च च मे॒ पञ्च॑ । \newline
6. मे॒ पञ्च॒ पञ्च॑ मे मे॒ पञ्च॑ च च॒ पञ्च॑ मे मे॒ पञ्च॑ च । \newline
7. पञ्च॑ च च॒ पञ्च॒ पञ्च॑ च मे मे च॒ पञ्च॒ पञ्च॑ च मे । \newline
8. च॒ मे॒ मे॒ च॒ च॒ मे॒ स॒प्त स॒प्त मे॑ च च मे स॒प्त । \newline
9. मे॒ स॒प्त स॒प्त मे॑ मे स॒प्त च॑ च स॒प्त मे॑ मे स॒प्त च॑ । \newline
10. स॒प्त च॑ च स॒प्त स॒प्त च॑ मे मे च स॒प्त स॒प्त च॑ मे । \newline
11. च॒ मे॒ मे॒ च॒ च॒ मे॒ नव॒ नव॑ मे च च मे॒ नव॑ । \newline
12. मे॒ नव॒ नव॑ मे मे॒ नव॑ च च॒ नव॑ मे मे॒ नव॑ च । \newline
13. नव॑ च च॒ नव॒ नव॑ च मे मे च॒ नव॒ नव॑ च मे । \newline
14. च॒ मे॒ मे॒ च॒ च॒ म॒ एका॑द॒ शैका॑दश मे च च म॒ एका॑दश । \newline
15. म॒ एका॑द॒ शैका॑दश मे म॒ एका॑दश च॒ चैका॑दश मे म॒ एका॑दश च । \newline
16. एका॑दश च॒ चैका॑द॒ शैका॑दश च मे मे॒ चैका॑द॒ शैका॑दश च मे । \newline
17. च॒ मे॒ मे॒ च॒ च॒ मे॒ त्रयो॑दश॒ त्रयो॑दश मे च च मे॒ त्रयो॑दश । \newline
18. मे॒ त्रयो॑दश॒ त्रयो॑दश मे मे॒ त्रयो॑दश च च॒ त्रयो॑दश मे मे॒ त्रयो॑दश च । \newline
19. त्रयो॑दश च च॒ त्रयो॑दश॒ त्रयो॑दश च मे मे च॒ त्रयो॑दश॒ त्रयो॑दश च मे । \newline
20. त्रयो॑द॒शेति॒ त्रयः॑ - द॒श॒ । \newline
21. च॒ मे॒ मे॒ च॒ च॒ मे॒ पञ्च॑दश॒ पञ्च॑दश मे च च मे॒ पञ्च॑दश । \newline
22. मे॒ पञ्च॑दश॒ पञ्च॑दश मे मे॒ पञ्च॑दश च च॒ पञ्च॑दश मे मे॒ पञ्च॑दश च । \newline
23. पञ्च॑दश च च॒ पञ्च॑दश॒ पञ्च॑दश च मे मे च॒ पञ्च॑दश॒ पञ्च॑दश च मे । \newline
24. पञ्च॑द॒शेति॒ पञ्च॑ - द॒श॒ । \newline
25. च॒ मे॒ मे॒ च॒ च॒ मे॒ स॒प्तद॑श स॒प्तद॑श मे च च मे स॒प्तद॑श । \newline
26. मे॒ स॒प्तद॑श स॒प्तद॑श मे मे स॒प्तद॑श च च स॒प्तद॑श मे मे स॒प्तद॑श च । \newline
27. स॒प्तद॑श च च स॒प्तद॑श स॒प्तद॑श च मे मे च स॒प्तद॑श स॒प्तद॑श च मे । \newline
28. स॒प्तद॒शेति॑ स॒प्त - द॒श॒ । \newline
29. च॒ मे॒ मे॒ च॒ च॒ मे॒ नव॑दश॒ नव॑दश मे च च मे॒ नव॑दश । \newline
30. मे॒ नव॑दश॒ नव॑दश मे मे॒ नव॑दश च च॒ नव॑दश मे मे॒ नव॑दश च । \newline
31. नव॑दश च च॒ नव॑दश॒ नव॑दश च मे मे च॒ नव॑दश॒ नव॑दश च मे । \newline
32. नव॑द॒शेति॒ नव॑ - द॒श॒ । \newline
33. च॒ मे॒ मे॒ च॒ च॒ म॒ एक॑विꣳशति॒ रेक॑विꣳशतिर् मे च च म॒ एक॑विꣳशतिः । \newline
34. म॒ एक॑विꣳशति॒ रेक॑विꣳशतिर् मे म॒ एक॑विꣳशतिश्च॒ चैक॑विꣳशतिर् मे म॒ एक॑विꣳशतिश्च । \newline
35. एक॑विꣳशतिश्च॒ चैक॑विꣳशति॒ रेक॑विꣳशतिश्च मे मे॒ चैक॑विꣳशति॒ रेक॑विꣳशतिश्च मे । \newline
36. एक॑विꣳशति॒रित्येक॑ - वि॒(ग्म्॒)श॒तिः॒ । \newline
37. च॒ मे॒ मे॒ च॒ च॒ मे॒ त्रयो॑विꣳशति॒ स्त्रयो॑विꣳशतिर् मे च च मे॒ त्रयो॑विꣳशतिः । \newline
38. मे॒ त्रयो॑विꣳशति॒ स्त्रयो॑विꣳशतिर् मे मे॒ त्रयो॑विꣳशतिश्च च॒ त्रयो॑विꣳशतिर् मे मे॒ त्रयो॑विꣳशतिश्च । \newline
39. त्रयो॑विꣳशतिश्च च॒ त्रयो॑विꣳशति॒ स्त्रयो॑विꣳशतिश्च मे मे च॒ त्रयो॑विꣳशति॒ स्त्रयो॑विꣳशतिश्च मे । \newline
40. त्रयो॑विꣳशति॒रिति॒ त्रयः॑ - वि॒(ग्म्॒)श॒तिः॒ । \newline
41. च॒ मे॒ मे॒ च॒ च॒ मे॒ पञ्च॑विꣳशतिः॒ पञ्च॑विꣳशतिर् मे च च मे॒ पञ्च॑विꣳशतिः । \newline
42. मे॒ पञ्च॑विꣳशतिः॒ पञ्च॑विꣳशतिर् मे मे॒ पञ्च॑विꣳशतिश्च च॒ पञ्च॑विꣳशतिर् मे मे॒ पञ्च॑विꣳशतिश्च । \newline
43. पञ्च॑विꣳशतिश्च च॒ पञ्च॑विꣳशतिः॒ पञ्च॑विꣳशतिश्च मे मे च॒ पञ्च॑विꣳशतिः॒ पञ्च॑विꣳशतिश्च मे । \newline
44. पञ्च॑विꣳशति॒रिति॒ पञ्च॑ - वि॒(ग्म्॒)श॒तिः॒ । \newline
45. च॒ मे॒ मे॒ च॒ च॒ मे॒ स॒प्तवि(ग्म्॑)शतिः स॒प्तवि(ग्म्॑)शतिर् मे च च मे स॒प्तवि(ग्म्॑)शतिः । \newline
46. मे॒ स॒प्तवि(ग्म्॑)शतिः स॒प्तवि(ग्म्॑)शतिर् मे मे स॒प्तवि(ग्म्॑)शतिश्च च स॒प्तवि(ग्म्॑)शतिर् मे मे स॒प्तवि(ग्म्॑)शतिश्च । \newline
47. स॒प्तवि(ग्म्॑)शतिश्च च स॒प्तवि(ग्म्॑)शतिः स॒प्तवि(ग्म्॑)शतिश्च मे मे च स॒प्तवि(ग्म्॑)शतिः स॒प्तवि(ग्म्॑)शतिश्च मे । \newline
48. स॒प्तवि(ग्म्॑)शति॒रिति॑ स॒प्त - वि॒(ग्म्॒)श॒तिः॒ । \newline
49. च॒ मे॒ मे॒ च॒ च॒ मे॒ नव॑विꣳशति॒र् नव॑विꣳशतिर् मे च च मे॒ नव॑विꣳशतिः । \newline
50. मे॒ नव॑विꣳशति॒र् नव॑विꣳशतिर् मे मे॒ नव॑विꣳशतिश्च च॒ नव॑विꣳशतिर् मे मे॒ नव॑विꣳशतिश्च । \newline
51. नव॑विꣳशतिश्च च॒ नव॑विꣳशति॒र् नव॑विꣳशतिश्च मे मे च॒ नव॑विꣳशति॒र् नव॑विꣳशतिश्च मे । \newline
52. नव॑विꣳशति॒रिति॒ नव॑ - वि॒(ग्म्॒)श॒तिः॒ । \newline
53. च॒ मे॒ मे॒ च॒ च॒ म॒ एक॑त्रिꣳश॒ देक॑त्रिꣳशन् मे च च म॒ एक॑त्रिꣳशत् । \newline
54. म॒ एक॑त्रिꣳश॒ देक॑त्रिꣳशन् मे म॒ एक॑त्रिꣳशच् च॒ चैक॑त्रिꣳशन् मे म॒ एक॑त्रिꣳशच् च । \newline
55. एक॑त्रिꣳशच् च॒ चैक॑त्रिꣳश॒ देक॑त्रिꣳशच् च मे मे॒ चैक॑त्रिꣳश॒ देक॑त्रिꣳशच् च मे । \newline
56. एक॑त्रिꣳश॒दित्येक॑ - त्रि॒(ग्म्॒)श॒त् । \newline
57. च॒ मे॒ मे॒ च॒ च॒ मे॒ त्रय॑स्त्रिꣳश॒त् त्रय॑स्त्रिꣳशन् मे च च मे॒ त्रय॑स्त्रिꣳशत् । \newline
58. मे॒ त्रय॑स्त्रिꣳश॒त् त्रय॑स्त्रिꣳशन् मे मे॒ त्रय॑स्त्रिꣳशच् च च॒ त्रय॑स्त्रिꣳशन् मे मे॒ त्रय॑स्त्रिꣳशच् च । \newline
59. त्रय॑स्त्रिꣳशच् च च॒ त्रय॑स्त्रिꣳश॒त् त्रय॑स्त्रिꣳशच् च मे मे च॒ त्रय॑स्त्रिꣳश॒त् त्रय॑स्त्रिꣳशच् च मे । \newline
60. त्रय॑स्त्रिꣳश॒दिति॒ त्रयः॑ - त्रि॒(ग्म्॒)श॒त् । \newline
61. च॒ मे॒ मे॒ च॒ च॒ मे॒ चत॑स्र॒ श्चत॑स्रो मे च च मे॒ चत॑स्रः । \newline
\pagebreak
\markright{ TS 4.7.11.2  \hfill https://www.vedavms.in \hfill}

\section{ TS 4.7.11.2 }

\textbf{TS 4.7.11.2 } \newline
\textbf{Samhita Paata} \newline

मे॒ चत॑स्रश्च मे॒ ऽष्टौ च॑ मे॒ द्वाद॑श च मे॒ षोड॑श च मे विꣳश॒तिश्च॑ मे॒ चतु॑र्विꣳशतिश्च मे॒ ऽष्टाविꣳ॑शतिश्च मे॒ द्वात्रिꣳ॑शच्च मे॒ षट्-त्रिꣳ॑शच्च मे चत्वारिꣳ॒॒शच्च॑ मे॒ चतु॑श्चत्वारिꣳशच्च मे॒ ऽष्टाच॑त्वारिꣳशच्च मे॒ वाज॑श्च प्रस॒वश्चा॑-पि॒जश्च॒ क्रतु॑श्च॒ सुव॑श्च मू॒र्द्धा च॒ व्यश्ञि॑यश्चा- ( ) -न्त्याय॒नश्चा- न्त्य॑श्च भौव॒नश्च॒ भुव॑न॒श्चा-धि॑पतिश्च ॥ \newline

\textbf{Pada Paata} \newline

मे॒ । चत॑स्रः । च॒ । मे॒ । अ॒ष्टौ । च॒ । मे॒ । द्वाद॑श । च॒ । मे॒ । षोड॑श । च॒ । मे॒ । विꣳ॒श॒तिः । च॒ । मे॒ । चतु॑र्विꣳशति॒रिति॒ चतुः॑ - विꣳ॒॒श॒तिः॒ । च॒ । मे॒ । अ॒ष्टाविꣳ॑शति॒रित्य॒ष्टा - विꣳ॒॒श॒तिः॒ । च॒ । मे॒ । द्वात्रिꣳ॑शत् । च॒ । मे॒ । षट्त्रिꣳ॑श॒दिति॒ षट् - त्रिꣳ॒॒श॒त् । च॒ । मे॒ । च॒त्वा॒रिꣳ॒॒शत् । च॒ । मे॒ । चतु॑श्चत्वारिꣳश॒दिति॒ चतुः॑ - च॒त्वा॒रिꣳ॒॒श॒त् । च॒ । मे॒ । अ॒ष्टाच॑त्वारिꣳश॒दित्य॒ष्टा - च॒त्वा॒रिꣳ॒॒श॒त् । च॒ । मे॒ । वाजः॑ । च॒ । प्र॒स॒व इति॑ प्र - स॒वः । च॒ । अ॒पि॒ज इत्य॑पि-जः । च॒ । क्रतुः॑ । च॒ । सुवः॑ । च॒ । मू॒द्‌र्धा । च॒ । व्यश्नि॑य॒ इति॑ वि - अश्नि॑यः ( ) । च॒ । अ॒न्त्या॒य॒नः । च॒ । अन्त्यः॑ । च॒ । भौ॒व॒नः । च॒ । भुव॑नः । च॒ । अधि॑पति॒रित्यधि॑ - प॒तिः॒ । च॒ ॥  \newline


\textbf{Krama Paata} \newline

मे॒ चत॑स्रः । चत॑स्रश्च । च॒ मे॒ । मे॒ऽष्टौ । अ॒ष्टौ च॑ । च॒ मे॒ । मे॒ द्वाद॑श । द्वाद॑श च । च॒ मे॒ । मे॒ षोड॑श । षोड॑श च । च॒ मे॒ । मे॒ विꣳ॒॒श॒तिः । विꣳ॒॒श॒तिश्च॑ । च॒ मे॒ । मे॒ चतु॑र्विꣳशतिः । चतु॑र्विꣳशतिश्च । चतु॑र्विꣳशति॒रिति॒ चतुः॑ - विꣳ॒॒श॒तिः॒ । च॒ मे॒ । मे॒ऽष्टाविꣳ॑शतिः । अ॒ष्टाविꣳ॑शतिश्च । अ॒ष्टाविꣳ॑शति॒रित्य॒ष्टा - विꣳ॒॒श॒तिः॒ । च॒ मे॒ । मे॒ द्वात्रिꣳ॑शत् । द्वात्रिꣳ॑शच् च । च॒ मे॒ । मे॒ षट्त्रिꣳ॑शत् । षट्त्रिꣳ॑शच्च । षट्त्रिꣳ॑श॒दिति॒ षट् - त्रिꣳ॒॒श॒त्॒ । च॒ मे॒ । मे॒ च॒त्वा॒रिꣳ॒॒शत् । च॒त्वा॒रिꣳ॒॒शच् च॑ । च॒ मे॒ । मे॒ चतु॑श्चत्वारिꣳशत् । चतु॑श्चत्वारिꣳशच् च । चतु॑श्चत्वारिꣳश॒दिति॒ चतुः॑ - च॒त्वा॒रिꣳ॒॒श॒त्॒ । च॒ मे॒ । मे॒ऽष्टाच॑त्वारिꣳशत् । अ॒ष्टाच॑त्वारिꣳशच् च । अ॒ष्टाच॑त्वारिꣳश॒दित्य॒ष्टा - च॒त्वा॒रिꣳ॒॒श॒त्॒ । च॒ मे॒ । मे॒ वाजः॑ । वाज॑श्च । च॒ प्र॒स॒वः । प्र॒स॒वश्च॑ । प्र॒स॒व इति॑ प्र - स॒वः । चा॒पि॒जः । अ॒पि॒जश्च॑ । अ॒पि॒ज इत्य॑पि - जः । च॒ क्रतुः॑ । क्रतु॑श्च । च॒ सुवः॑ । सुव॑श्च । च॒ मू॒र्द्धा । मू॒र्द्धा च॑ । च॒ व्यश्ञि॑यः ( ) । व्यश्ञि॑यश्च । व्यश्ञि॑य॒ इति॑ वि - अश्ञि॑यः । चा॒न्त्या॒य॒नः । आ॒न्त्या॒य॒नश्च॑ । चान्त्यः॑ । अन्त्य॑श्च । च॒ भौ॒व॒नः । भौ॒व॒नश्च॑ । च॒ भुव॑नः । भुव॑नश्च । चाधि॑पतिः । अधि॑पतिश्च । अधि॑पति॒रित्यधि॑ - प॒तिः॒ । चेति॑ च । \newline

\textbf{Jatai Paata} \newline

1. मे॒ चत॑स्र॒ श्चत॑स्रो मे मे॒ चत॑स्रः । \newline
2. चत॑स्रश्च च॒ चत॑स्र॒ श्चत॑स्रश्च । \newline
3. च॒ मे॒ मे॒ च॒ च॒ मे॒ । \newline
4. मे॒ ऽष्टा व॒ष्टौ मे॑ मे॒ ऽष्टौ । \newline
5. अ॒ष्टौ च॑ चा॒ष्टा व॒ष्टौ च॑ । \newline
6. च॒ मे॒ मे॒ च॒ च॒ मे॒ । \newline
7. मे॒ द्वाद॑श॒ द्वाद॑श मे मे॒ द्वाद॑श । \newline
8. द्वाद॑श च च॒ द्वाद॑श॒ द्वाद॑श च । \newline
9. च॒ मे॒ मे॒ च॒ च॒ मे॒ । \newline
10. मे॒ षोड॑श॒ षोड॑श मे मे॒ षोड॑श । \newline
11. षोड॑श च च॒ षोड॑श॒ षोड॑श च । \newline
12. च॒ मे॒ मे॒ च॒ च॒ मे॒ । \newline
13. मे॒ वि॒(ग्म्॒)श॒तिर् वि(ग्म्॑)श॒तिर् मे॑ मे विꣳश॒तिः । \newline
14. वि॒(ग्म्॒)श॒तिश्च॑ च विꣳश॒तिर् वि(ग्म्॑)श॒तिश्च॑ । \newline
15. च॒ मे॒ मे॒ च॒ च॒ मे॒ । \newline
16. मे॒ चतु॑र्विꣳशति॒ श्चतु॑र्विꣳशतिर् मे मे॒ चतु॑र्विꣳशतिः । \newline
17. चतु॑र्विꣳशतिश्च च॒ चतु॑र्विꣳशति॒ श्चतु॑र्विꣳशतिश्च । \newline
18. चतु॑र्विꣳशति॒रिति॒ चतुः॑ - वि॒(ग्म्॒)श॒तिः॒ । \newline
19. च॒ मे॒ मे॒ च॒ च॒ मे॒ । \newline
20. मे॒ ऽष्टावि(ग्म्॑)शति र॒ष्टावि(ग्म्॑)शतिर् मे मे॒ ऽष्टावि(ग्म्॑)शतिः । \newline
21. अ॒ष्टावि(ग्म्॑)शतिश्च चा॒ष्टावि(ग्म्॑)शति र॒ष्टावि(ग्म्॑)शतिश्च । \newline
22. अ॒ष्टावि(ग्म्॑)शति॒रित्य॒ष्टा - वि॒(ग्म्॒)श॒तिः॒ । \newline
23. च॒ मे॒ मे॒ च॒ च॒ मे॒ । \newline
24. मे॒ द्वात्रि(ग्म्॑)श॒द् द्वात्रि(ग्म्॑)शन् मे मे॒ द्वात्रि(ग्म्॑)शत् । \newline
25. द्वात्रि(ग्म्॑)शच् च च॒ द्वात्रि(ग्म्॑)श॒द् द्वात्रि(ग्म्॑)शच् च । \newline
26. च॒ मे॒ मे॒ च॒ च॒ मे॒ । \newline
27. मे॒ षट्त्रि(ग्म्॑)श॒थ् षट्त्रि(ग्म्॑)शन् मे मे॒ षट्त्रि(ग्म्॑)शत् । \newline
28. षट्त्रि(ग्म्॑)शच् च च॒ षट्त्रि(ग्म्॑)श॒थ् षट्त्रि(ग्म्॑)शच् च । \newline
29. षट्त्रि(ग्म्॑)श॒दिति॒ षट् - त्रि॒(ग्म्॒)श॒त् । \newline
30. च॒ मे॒ मे॒ च॒ च॒ मे॒ । \newline
31. मे॒ च॒त्वा॒रि॒(ग्म्॒)शच् च॑त्वारि॒(ग्म्॒)शन् मे॑ मे चत्वारि॒(ग्म्॒)शत् । \newline
32. च॒त्वा॒रि॒(ग्म्॒)शच् च॑ च चत्वारि॒(ग्म्॒)शच् च॑त्वारि॒(ग्म्॒)शच् च॑ । \newline
33. च॒ मे॒ मे॒ च॒ च॒ मे॒ । \newline
34. मे॒ चतु॑श्चत्वारिꣳश॒च् चतु॑श्चत्वारिꣳशन् मे मे॒ चतु॑श्चत्वारिꣳशत् । \newline
35. चतु॑श्चत्वारिꣳशच् च च॒ चतु॑श्चत्वारिꣳश॒च् चतु॑श्चत्वारिꣳशच् च । \newline
36. चतु॑श्चत्वारिꣳश॒दिति॒ चतुः॑ - च॒त्वा॒रि॒(ग्म्॒)श॒त् । \newline
37. च॒ मे॒ मे॒ च॒ च॒ मे॒ । \newline
38. मे॒ ऽष्टाच॑त्वारिꣳश द॒ष्टाच॑त्वारिꣳशन् मे मे॒ ऽष्टाच॑त्वारिꣳशत् । \newline
39. अ॒ष्टाच॑त्वारिꣳशच् च चा॒ष्टाच॑त्वारिꣳश द॒ष्टाच॑त्वारिꣳशच् च । \newline
40. अ॒ष्टाच॑त्वारिꣳश॒दित्य॒ष्टा - च॒त्वा॒रि॒(ग्म्॒)श॒त् । \newline
41. च॒ मे॒ मे॒ च॒ च॒ मे॒ । \newline
42. मे॒ वाजो॒ वाजो॑ मे मे॒ वाजः॑ । \newline
43. वाज॑श्च च॒ वाजो॒ वाज॑श्च । \newline
44. च॒ प्र॒स॒वः प्र॑स॒वश्च॑ च प्रस॒वः । \newline
45. प्र॒स॒वश्च॑ च प्रस॒वः प्र॑स॒वश्च॑ । \newline
46. प्र॒स॒व इति॑ प्र - स॒वः । \newline
47. चा॒पि॒जो अ॑पि॒जश्च॑ चापि॒जः । \newline
48. अ॒पि॒जश्च॑ चापि॒जो अ॑पि॒जश्च॑ । \newline
49. अ॒पि॒ज इत्य॑पि - जः । \newline
50. च॒ क्रतुः॒ क्रतु॑श्च च॒ क्रतुः॑ । \newline
51. क्रतु॑श्च च॒ क्रतुः॒ क्रतु॑श्च । \newline
52. च॒ सुवः॒ सुव॑श्च च॒ सुवः॑ । \newline
53. सुव॑श्च च॒ सुवः॒ सुव॑श्च । \newline
54. च॒ मू॒र्द्धा मू॒र्द्धा च॑ च मू॒र्द्धा । \newline
55. मू॒र्द्धा च॑ च मू॒र्द्धा मू॒र्द्धा च॑ । \newline
56. च॒ व्यश्ञि॑यो॒ व्यश्ञि॑यश्च च॒ व्यश्ञि॑यः । \newline
57. व्यश्ञि॑यश्च च॒ व्यश्ञि॑यो॒ व्यश्ञि॑यश्च । \newline
58. व्यश्ञि॑य॒ इति॑ वि - अश्ञि॑यः । \newline
59. चा॒न्त्या॒य॒न आ᳚न्त्याय॒नश्च॑ चान्त्याय॒नः । \newline
60. आ॒न्त्या॒य॒नश्च॑ चान्त्याय॒न आ᳚न्त्याय॒नश्च॑ । \newline
61. चान्त्यो॒ अन्त्य॑श्च॒ चान्त्यः॑ । \newline
62. अन्त्य॑श्च॒ चान्त्यो॒ अन्त्य॑श्च । \newline
63. च॒ भौ॒व॒नो भौ॑व॒नश्च॑ च भौव॒नः । \newline
64. भौ॒व॒नश्च॑ च भौव॒नो भौ॑व॒नश्च॑ । \newline
65. च॒ भुव॑नो॒ भुव॑नश्च च॒ भुव॑नः । \newline
66. भुव॑नश्च च॒ भुव॑नो॒ भुव॑नश्च । \newline
67. चाधि॑पति॒ रधि॑पतिश्च॒ चाधि॑पतिः । \newline
68. अधि॑पतिश्च॒ चाधि॑पति॒ रधि॑पतिश्च । \newline
69. अधि॑पति॒रित्यधि॑ - प॒तिः॒ । \newline
70. चेति॑ च । \newline

\textbf{Ghana Paata } \newline

1. मे॒ चत॑स्र॒ श्चत॑स्रो मे मे॒ चत॑स्रश्च च॒ चत॑स्रो मे मे॒ चत॑स्रश्च । \newline
2. चत॑स्रश्च च॒ चत॑स्र॒ श्चत॑स्रश्च मे मे च॒ चत॑स्र॒ श्चत॑स्रश्च मे । \newline
3. च॒ मे॒ मे॒ च॒ च॒ मे॒ ऽष्टा व॒ष्टौ मे॑ च च मे॒ ऽष्टौ । \newline
4. मे॒ ऽष्टा व॒ष्टौ मे॑ मे॒ ऽष्टौ च॑ चा॒ष्टौ मे॑ मे॒ ऽष्टौ च॑ । \newline
5. अ॒ष्टौ च॑ चा॒ष्टा व॒ष्टौ च॑ मे मे चा॒ष्टा व॒ष्टौ च॑ मे । \newline
6. च॒ मे॒ मे॒ च॒ च॒ मे॒ द्वाद॑श॒ द्वाद॑श मे च च मे॒ द्वाद॑श । \newline
7. मे॒ द्वाद॑श॒ द्वाद॑श मे मे॒ द्वाद॑श च च॒ द्वाद॑श मे मे॒ द्वाद॑श च । \newline
8. द्वाद॑श च च॒ द्वाद॑श॒ द्वाद॑श च मे मे च॒ द्वाद॑श॒ द्वाद॑श च मे । \newline
9. च॒ मे॒ मे॒ च॒ च॒ मे॒ षोड॑श॒ षोड॑श मे च च मे॒ षोड॑श । \newline
10. मे॒ षोड॑श॒ षोड॑श मे मे॒ षोड॑श च च॒ षोड॑श मे मे॒ षोड॑श च । \newline
11. षोड॑श च च॒ षोड॑श॒ षोड॑श च मे मे च॒ षोड॑श॒ षोड॑श च मे । \newline
12. च॒ मे॒ मे॒ च॒ च॒ मे॒ वि॒(ग्म्॒)श॒तिर् वि(ग्म्॑)श॒तिर् मे॑ च च मे विꣳश॒तिः । \newline
13. मे॒ वि॒(ग्म्॒)श॒तिर् वि(ग्म्॑)श॒तिर् मे॑ मे विꣳश॒तिश्च॑ च विꣳश॒तिर् मे॑ मे विꣳश॒तिश्च॑ । \newline
14. वि॒(ग्म्॒)श॒तिश्च॑ च विꣳश॒तिर् वि(ग्म्॑)श॒तिश्च॑ मे मे च विꣳश॒तिर् वि(ग्म्॑)श॒तिश्च॑ मे । \newline
15. च॒ मे॒ मे॒ च॒ च॒ मे॒ चतु॑र्विꣳशति॒ श्चतु॑र्विꣳशतिर् मे च च मे॒ चतु॑र्विꣳशतिः । \newline
16. मे॒ चतु॑र्विꣳशति॒ श्चतु॑र्विꣳशतिर् मे मे॒ चतु॑र्विꣳशतिश्च च॒ चतु॑र्विꣳशतिर् मे मे॒ चतु॑र्विꣳशतिश्च । \newline
17. चतु॑र्विꣳशतिश्च च॒ चतु॑र्विꣳशति॒ श्चतु॑र्विꣳशतिश्च मे मे च॒ चतु॑र्विꣳशति॒ श्चतु॑र्विꣳशतिश्च मे । \newline
18. चतु॑र्विꣳशति॒रिति॒ चतुः॑ - वि॒(ग्म्॒)श॒तिः॒ । \newline
19. च॒ मे॒ मे॒ च॒ च॒ मे॒ ऽष्टावि(ग्म्॑)शति र॒ष्टावि(ग्म्॑)शतिर् मे च च मे॒ ऽष्टावि(ग्म्॑)शतिः । \newline
20. मे॒ ऽष्टावि(ग्म्॑)शति र॒ष्टावि(ग्म्॑)शतिर् मे मे॒ ऽष्टावि(ग्म्॑)शतिश्च चा॒ष्टावि(ग्म्॑)शतिर् मे मे॒ ऽष्टावि(ग्म्॑)शतिश्च । \newline
21. अ॒ष्टावि(ग्म्॑)शतिश्च चा॒ष्टावि(ग्म्॑)शति र॒ष्टावि(ग्म्॑)शतिश्च मे मे चा॒ष्टावि(ग्म्॑)शति र॒ष्टावि(ग्म्॑)शतिश्च मे । \newline
22. अ॒ष्टावि(ग्म्॑)शति॒रित्य॒ष्टा - वि॒(ग्म्॒)श॒तिः॒ । \newline
23. च॒ मे॒ मे॒ च॒ च॒ मे॒ द्वात्रि(ग्म्॑)श॒द् द्वात्रि(ग्म्॑)शन् मे च च मे॒ द्वात्रि(ग्म्॑)शत् । \newline
24. मे॒ द्वात्रि(ग्म्॑)श॒द् द्वात्रि(ग्म्॑)शन् मे मे॒ द्वात्रि(ग्म्॑)शच् च च॒ द्वात्रि(ग्म्॑)शन् मे मे॒ द्वात्रि(ग्म्॑)शच् च । \newline
25. द्वात्रि(ग्म्॑)शच् च च॒ द्वात्रि(ग्म्॑)श॒द् द्वात्रि(ग्म्॑)शच् च मे मे च॒ द्वात्रि(ग्म्॑)श॒द् द्वात्रि(ग्म्॑)शच् च मे । \newline
26. च॒ मे॒ मे॒ च॒ च॒ मे॒ षट्त्रि(ग्म्॑)श॒थ् षट्त्रि(ग्म्॑)शन् मे च च मे॒ षट्त्रि(ग्म्॑)शत् । \newline
27. मे॒ षट्त्रि(ग्म्॑)श॒थ् षट्त्रि(ग्म्॑)शन् मे मे॒ षट्त्रि(ग्म्॑)शच् च च॒ षट्त्रि(ग्म्॑)शन् मे मे॒ षट्त्रि(ग्म्॑)शच् च । \newline
28. षट्त्रि(ग्म्॑)शच् च च॒ षट्त्रि(ग्म्॑)श॒थ् षट्त्रि(ग्म्॑)शच् च मे मे च॒ षट्त्रि(ग्म्॑)श॒थ् षट्त्रि(ग्म्॑)शच् च मे । \newline
29. षट्त्रि(ग्म्॑)श॒दिति॒ षट् - त्रि॒(ग्म्॒)श॒त् । \newline
30. च॒ मे॒ मे॒ च॒ च॒ मे॒ च॒त्वा॒रि॒(ग्म्॒)शच् च॑त्वारि॒(ग्म्॒)शन् मे॑ च च मे चत्वारि॒(ग्म्॒)शत् । \newline
31. मे॒ च॒त्वा॒रि॒(ग्म्॒)शच् च॑त्वारि॒(ग्म्॒)शन् मे॑ मे चत्वारि॒(ग्म्॒)शच् च॑ च चत्वारि॒(ग्म्॒)शन् मे॑ मे चत्वारि॒(ग्म्॒)शच् च॑ । \newline
32. च॒त्वा॒रि॒(ग्म्॒)शच् च॑ च चत्वारि॒(ग्म्॒)शच् च॑त्वारि॒(ग्म्॒)शच् च॑ मे मे च चत्वारि॒(ग्म्॒)शच् च॑त्वारि॒(ग्म्॒)शच् च॑ मे । \newline
33. च॒ मे॒ मे॒ च॒ च॒ मे॒ चतु॑श्चत्वारिꣳश॒च् चतु॑श्चत्वारिꣳशन् मे च च मे॒ चतु॑श्चत्वारिꣳशत् । \newline
34. मे॒ चतु॑श्चत्वारिꣳश॒च् चतु॑श्चत्वारिꣳशन् मे मे॒ चतु॑श्चत्वारिꣳशच् च च॒ चतु॑श्चत्वारिꣳशन् मे मे॒ चतु॑श्चत्वारिꣳशच् च । \newline
35. चतु॑श्चत्वारिꣳशच् च च॒ चतु॑श्चत्वारिꣳश॒च् चतु॑श्चत्वारिꣳशच् च मे मे च॒ चतु॑श्चत्वारिꣳश॒च् चतु॑श्चत्वारिꣳशच् च मे । \newline
36. चतु॑श्चत्वारिꣳश॒दिति॒ चतुः॑ - च॒त्वा॒रि॒(ग्म्॒)श॒त् । \newline
37. च॒ मे॒ मे॒ च॒ च॒ मे॒ ऽष्टाच॑त्वारिꣳश द॒ष्टाच॑त्वारिꣳशन् मे च च मे॒ ऽष्टाच॑त्वारिꣳशत् । \newline
38. मे॒ ऽष्टाच॑त्वारिꣳश द॒ष्टाच॑त्वारिꣳशन् मे मे॒ ऽष्टाच॑त्वारिꣳशच् च चा॒ष्टाच॑त्वारिꣳशन् मे मे॒ ऽष्टाच॑त्वारिꣳशच् च । \newline
39. अ॒ष्टाच॑त्वारिꣳशच् च चा॒ष्टाच॑त्वारिꣳश द॒ष्टाच॑त्वारिꣳशच् च मे मे चा॒ष्टाच॑त्वारिꣳश द॒ष्टाच॑त्वारिꣳशच् च मे । \newline
40. अ॒ष्टाच॑त्वारिꣳश॒दित्य॒ष्टा - च॒त्वा॒रि॒(ग्म्॒)श॒त् । \newline
41. च॒ मे॒ मे॒ च॒ च॒ मे॒ वाजो॒ वाजो॑ मे च च मे॒ वाजः॑ । \newline
42. मे॒ वाजो॒ वाजो॑ मे मे॒ वाज॑श्च च॒ वाजो॑ मे मे॒ वाज॑श्च । \newline
43. वाज॑श्च च॒ वाजो॒ वाज॑श्च प्रस॒वः प्र॑स॒वश्च॒ वाजो॒ वाज॑श्च प्रस॒वः । \newline
44. च॒ प्र॒स॒वः प्र॑स॒वश्च॑ च प्रस॒वश्च॑ च प्रस॒वश्च॑ च प्रस॒वश्च॑ । \newline
45. प्र॒स॒वश्च॑ च प्रस॒वः प्र॑स॒व श्चा॑पि॒जो अ॑पि॒जश्च॑ प्रस॒वः प्र॑स॒व श्चा॑पि॒जः । \newline
46. प्र॒स॒व इति॑ प्र - स॒वः । \newline
47. चा॒पि॒जो अ॑पि॒जश्च॑ चापि॒जश्च॑ चापि॒जश्च॑ चापि॒जश्च॑ । \newline
48. अ॒पि॒जश्च॑ चापि॒जो अ॑पि॒जश्च॒ क्रतुः॒ क्रतु॑ श्चापि॒जो अ॑पि॒जश्च॒ क्रतुः॑ । \newline
49. अ॒पि॒ज इत्य॑पि - जः । \newline
50. च॒ क्रतुः॒ क्रतु॑श्च च॒ क्रतु॑श्च च॒ क्रतु॑श्च च॒ क्रतु॑श्च । \newline
51. क्रतु॑श्च च॒ क्रतुः॒ क्रतु॑श्च॒ सुवः॒ सुव॑श्च॒ क्रतुः॒ क्रतु॑श्च॒ सुवः॑ । \newline
52. च॒ सुवः॒ सुव॑श्च च॒ सुव॑श्च च॒ सुव॑श्च च॒ सुव॑श्च । \newline
53. सुव॑श्च च॒ सुवः॒ सुव॑श्च मू॒र्द्धा मू॒र्द्धा च॒ सुवः॒ सुव॑श्च मू॒र्द्धा । \newline
54. च॒ मू॒र्द्धा मू॒र्द्धा च॑ च मू॒र्द्धा च॑ च मू॒र्द्धा च॑ च मू॒र्द्धा च॑ । \newline
55. मू॒र्द्धा च॑ च मू॒र्द्धा मू॒र्द्धा च॒ व्यश्ञि॑यो॒ व्यश्ञि॑यश्च मू॒र्द्धा मू॒र्द्धा च॒ व्यश्ञि॑यः । \newline
56. च॒ व्यश्ञि॑यो॒ व्यश्ञि॑यश्च च॒ व्यश्ञि॑यश्च च॒ व्यश्ञि॑यश्च च॒ व्यश्ञि॑यश्च । \newline
57. व्यश्ञि॑यश्च च॒ व्यश्ञि॑यो॒ व्यश्ञि॑य श्चान्त्याय॒न आ᳚न्त्याय॒नश्च॒ व्यश्ञि॑यो॒ व्यश्ञि॑य श्चान्त्याय॒नः । \newline
58. व्यश्ञि॑य॒ इति॑ वि - अश्ञि॑यः । \newline
59. चा॒न्त्या॒य॒न आ᳚न्त्याय॒नश्च॑ चान्त्याय॒नश्च॑ चान्त्याय॒नश्च॑ चान्त्याय॒नश्च॑ । \newline
60. आ॒न्त्या॒य॒नश्च॑ चान्त्याय॒न आ᳚न्त्याय॒न श्चान्त्यो॒ अन्त्य॑ श्चान्त्याय॒न आ᳚न्त्याय॒न श्चान्त्यः॑ । \newline
61. चान्त्यो॒ अन्त्य॑श्च॒ चान्त्य॑श्च॒ चान्त्य॑श्च॒ चान्त्य॑श्च । \newline
62. अन्त्य॑श्च॒ चान्त्यो॒ अन्त्य॑श्च भौव॒नो भौ॑व॒न श्चान्त्यो॒ अन्त्य॑श्च भौव॒नः । \newline
63. च॒ भौ॒व॒नो भौ॑व॒नश्च॑ च भौव॒नश्च॑ च भौव॒नश्च॑ च भौव॒नश्च॑ । \newline
64. भौ॒व॒नश्च॑ च भौव॒नो भौ॑व॒नश्च॒ भुव॑नो॒ भुव॑नश्च भौव॒नो भौ॑व॒नश्च॒ भुव॑नः । \newline
65. च॒ भुव॑नो॒ भुव॑नश्च च॒ भुव॑नश्च च॒ भुव॑नश्च च॒ भुव॑नश्च । \newline
66. भुव॑नश्च च॒ भुव॑नो॒ भुव॑न॒ श्चाधि॑पति॒ रधि॑पतिश्च॒ भुव॑नो॒ भुव॑न॒ श्चाधि॑पतिः । \newline
67. चाधि॑पति॒ रधि॑पतिश्च॒ चाधि॑पतिश्च॒ चाधि॑पतिश्च॒ चाधि॑पतिश्च । \newline
68. अधि॑पतिश्च॒ चाधि॑पति॒ रधि॑पतिश्च । \newline
69. अधि॑पति॒रित्यधि॑ - प॒तिः॒ । \newline
70. चेति॑ च । \newline
\pagebreak
\markright{ TS 4.7.12.1  \hfill https://www.vedavms.in \hfill}

\section{ TS 4.7.12.1 }

\textbf{TS 4.7.12.1 } \newline
\textbf{Samhita Paata} \newline

वाजो॑ नः स॒प्त प्र॒दिश॒श्चत॑स्रो वा परा॒वतः॑ । वाजो॑ नो॒ विश्वै᳚र्दे॒वै-र्धन॑सातावि॒हाव॑तु ॥ विश्वे॑ अ॒द्य म॒रुतो॒ विश्व॑ ऊ॒ती विश्वे॑ भवन्त्व॒ग्नयः॒ समि॑द्धाः । विश्वे॑ नो दे॒वा अव॒साऽऽ ग॑मन्तु॒ विश्व॑मस्तु॒ द्रवि॑णं॒ ॅवाजो॑ अ॒स्मे ॥ वाज॑स्य प्रस॒वं दे॑वा॒ रथै᳚र्याता हिर॒ण्ययैः᳚ । अ॒ग्निरिन्द्रो॒ बृह॒स्पति॑र्म॒रुतः॒ सोम॑पीतये ॥ वाजे॑वाजे ऽवत वाजिनो नो॒ धने॑षु - [  ] \newline

\textbf{Pada Paata} \newline

वाजः॑ । नः॒ । स॒प्त । प्र॒दिश॒ इति॑ प्र - दिशः॑ । चत॑स्रः । वा॒ । प॒रा॒वत॒ इति॑ परा - वतः॑ ॥ वाजः॑ । नः॒ । विश्वैः᳚ । दे॒वैः । धन॑साता॒विति॒ धन॑ - सा॒तौ॒ । इ॒ह । अ॒व॒तु॒ ॥ विश्वे᳚ । अ॒द्य । म॒रुतः॑ । विश्वे᳚ । ऊ॒ती । विश्वे᳚ । भ॒व॒न्तु॒ । अ॒ग्नयः॑ । समि॑द्धा॒ इति॒ सं - इ॒द्धाः॒ ॥ विश्वे᳚ । नः॒ । दे॒वाः । अव॑सा । एति॑ । ग॒म॒न्तु॒ । विश्व᳚म् । अ॒स्तु॒ । द्रवि॑णम् । वाजः॑ । अ॒स्मे इति॑ ॥ वाज॑स्य । प्र॒स॒वमिति॑ प्र - स॒वम् । दे॒वाः॒ । रथैः᳚ । या॒त॒ । हि॒र॒ण्ययैः᳚ ॥ अ॒ग्निः । इन्द्रः॑ । बृह॒स्पतिः॑ । म॒रुतः॑ । सोम॑पीतय॒ इति॒ सोम॑ - पी॒त॒ये॒ ॥ वाजे॑वाज॒ इति॒ वाजे᳚ - वा॒जे॒ । अ॒व॒त॒ । वा॒जि॒नः॒ । नः॒ । धने॑षु ।  \newline


\textbf{Krama Paata} \newline

वाजो॑ नः । नः॒ स॒प्त । स॒प्त प्र॒दिशः॑ । प्र॒दिश॒श्चत॑स्रः । प्र॒दिश॒ इति॑ प्र - दिशः॑ । चत॑स्रो वा । वा॒ प॒रा॒वतः॑ । प॒रा॒वत॒ इति॑ परा - वतः॑ ॥ वाजो॑ नः । नो॒ विश्वैः᳚ । विश्वै᳚र् दे॒वैः । दे॒वैर् धन॑सातौ । धन॑सातावि॒ह । धन॑साता॒विति॒ धन॑ - सा॒तौ॒ । इ॒हाव॑तु । अ॒व॒त्वित्य॑वतु ॥ विश्वे॑ अ॒द्य । अ॒द्य म॒रुतः॑ । म॒रुतो॒ विश्वे᳚ । विश्व॑ ऊ॒ती । ऊ॒ती विश्वे᳚ । विश्वे॑ भवन्तु । भ॒व॒न्त्व॒ग्नयः॑ । अ॒ग्नयः॒ समि॑द्धाः । समि॑द्धा॒ इति॒ सम् - इ॒द्धाः॒ ॥ विश्वे॑ नः । नो॒ दे॒वाः । दे॒वा अव॑सा । अव॒सा । आ ग॑मन्तु । ग॒म॒न्तु॒ विश्व᳚म् । विश्व॑मस्तु । अ॒स्तु॒ द्रवि॑णम् । द्रवि॑ण॒म् ॅवाजः॑ । वाजो॑ अ॒स्मे । अ॒स्मे इत्य॒स्मे ॥ वाज॑स्य प्रस॒वम् । प्र॒स॒वम् दे॑वाः । प्र॒स॒वमिति॑ प्र - स॒वम् । दे॒वा॒ रथैः᳚ । रथै᳚र् यात । या॒ता॒ हि॒र॒ण्ययैः᳚ । हि॒र॒ण्ययै॒रिति॑ हिर॒ण्ययैः᳚ ॥ अ॒ग्निरिन्द्रः॑ । इन्द्रो॒ बृह॒स्पतिः॑ । बृह॒स्पति॑र् म॒रुतः॑ । म॒रुतः॒ सोम॑पीतये । सोम॑पीतय॒ इति॒ सोम॑ - पी॒त॒ये॒ ॥ वाजे॑वाजेऽवत । वाजे॑वाज॒ इति॒ वाजे᳚ - वा॒जे॒ । अ॒व॒त॒ वा॒जि॒नः॒ । वा॒जि॒नो॒ नः॒ । नो॒ धने॑षु ( ) । धने॑षु विप्राः \newline

\textbf{Jatai Paata} \newline

1. वाजो॑ नो नो॒ वाजो॒ वाजो॑ नः । \newline
2. नः॒ स॒प्त स॒प्त नो॑ नः स॒प्त । \newline
3. स॒प्त प्र॒दिशः॑ प्र॒दिशः॑ स॒प्त स॒प्त प्र॒दिशः॑ । \newline
4. प्र॒दिश॒ श्चत॑स्र॒ श्चत॑स्रः प्र॒दिशः॑ प्र॒दिश॒ श्चत॑स्रः । \newline
5. प्र॒दिश॒ इति॑ प्र - दिशः॑ । \newline
6. चत॑स्रो वा वा॒ चत॑स्र॒ श्चत॑स्रो वा । \newline
7. वा॒ प॒रा॒वतः॑ परा॒वतो॑ वा वा परा॒वतः॑ । \newline
8. प॒रा॒वत॒ इति॑ परा - वतः॑ । \newline
9. वाजो॑ नो नो॒ वाजो॒ वाजो॑ नः । \newline
10. नो॒ विश्वै॒र् विश्वै᳚र् नो नो॒ विश्वैः᳚ । \newline
11. विश्वै᳚र् दे॒वैर् दे॒वैर् विश्वै॒र् विश्वै᳚र् दे॒वैः । \newline
12. दे॒वैर् धन॑सातौ॒ धन॑सातौ दे॒वैर् दे॒वैर् धन॑सातौ । \newline
13. धन॑साता वि॒हेह धन॑सातौ॒ धन॑साता वि॒ह । \newline
14. धन॑साता॒विति॒ धन॑ - सा॒तौ॒ । \newline
15. इ॒हा व॑त्व वत्वि॒ हेहा व॑तु । \newline
16. अ॒व॒त्वित्य॑वतु । \newline
17. विश्वे॑ अ॒द्याद्य विश्वे॒ विश्वे॑ अ॒द्य । \newline
18. अ॒द्य म॒रुतो॑ म॒रुतो॑ अ॒द्याद्य म॒रुतः॑ । \newline
19. म॒रुतो॒ विश्वे॒ विश्वे॑ म॒रुतो॑ म॒रुतो॒ विश्वे᳚ । \newline
20. विश्व॑ ऊ॒त्यू॑ती विश्वे॒ विश्व॑ ऊ॒ती । \newline
21. ऊ॒ती विश्वे॒ विश्व॑ ऊ॒त्यू॑ती विश्वे᳚ । \newline
22. विश्वे॑ भवन्तु भवन्तु॒ विश्वे॒ विश्वे॑ भवन्तु । \newline
23. भ॒व॒ न्त्व॒ग्नयो॑ अ॒ग्नयो॑ भवन्तु भव न्त्व॒ग्नयः॑ । \newline
24. अ॒ग्नयः॒ समि॑द्धाः॒ समि॑द्धा अ॒ग्नयो॑ अ॒ग्नयः॒ समि॑द्धाः । \newline
25. समि॑द्धा॒ इति॒ सं - इ॒द्धाः॒ । \newline
26. विश्वे॑ नो नो॒ विश्वे॒ विश्वे॑ नः । \newline
27. नो॒ दे॒वा दे॒वा नो॑ नो दे॒वाः । \newline
28. दे॒वा अव॒सा ऽव॑सा दे॒वा दे॒वा अव॑सा । \newline
29. अव॒सा ऽऽव॒सा ऽव॒सा । \newline
30. आ ग॑मन्तु गम॒न्त्वा ग॑मन्तु । \newline
31. ग॒म॒न्तु॒ विश्वं॒ ॅविश्व॑म् गमन्तु गमन्तु॒ विश्व᳚म् । \newline
32. विश्व॑ मस्त्वस्तु॒ विश्वं॒ ॅविश्व॑ मस्तु । \newline
33. अ॒स्तु॒ द्रवि॑ण॒म् द्रवि॑ण मस्त्वस्तु॒ द्रवि॑णम् । \newline
34. द्रवि॑णं॒ ॅवाजो॒ वाजो॒ द्रवि॑ण॒म् द्रवि॑णं॒ ॅवाजः॑ । \newline
35. वाजो॑ अ॒स्मे अ॒स्मे वाजो॒ वाजो॑ अ॒स्मे । \newline
36. अ॒स्मे इत्य॒स्मे । \newline
37. वाज॑स्य प्रस॒वम् प्र॑स॒वं ॅवाज॑स्य॒ वाज॑स्य प्रस॒वम् । \newline
38. प्र॒स॒वम् दे॑वा देवाः प्रस॒वम् प्र॑स॒वम् दे॑वाः । \newline
39. प्र॒स॒वमिति॑ प्र - स॒वम् । \newline
40. दे॒वा॒ रथै॒र् रथै᳚र् देवा देवा॒ रथैः᳚ । \newline
41. रथै᳚र् यात यात॒ रथै॒र् रथै᳚र् यात । \newline
42. या॒ता॒ हि॒र॒ण्ययैर्॑. हिर॒ण्ययै᳚र् यात याता हिर॒ण्ययैः᳚ । \newline
43. हि॒र॒ण्ययै॒रिति॑ हिर॒ण्ययैः᳚ । \newline
44. अ॒ग्नि रिन्द्र॒ इन्द्रो॑ अ॒ग्नि र॒ग्नि रिन्द्रः॑ । \newline
45. इन्द्रो॒ बृह॒स्पति॒र् बृह॒स्पति॒ रिन्द्र॒ इन्द्रो॒ बृह॒स्पतिः॑ । \newline
46. बृह॒स्पति॑र् म॒रुतो॑ म॒रुतो॒ बृह॒स्पति॒र् बृह॒स्पति॑र् म॒रुतः॑ । \newline
47. म॒रुतः॒ सोम॑पीतये॒ सोम॑पीतये म॒रुतो॑ म॒रुतः॒ सोम॑पीतये । \newline
48. सोम॑पीतय॒ इति॒ सोम॑ - पी॒त॒ये॒ । \newline
49. वाजे॑वाजे ऽवतावत॒ वाजे॑वाजे॒ वाजे॑वाजे ऽवत । \newline
50. वाजे॑वाज॒ इति॒ वाजे᳚ - वा॒जे॒ । \newline
51. अ॒व॒त॒ वा॒जि॒नो॒ वा॒जि॒नो॒ ऽव॒ता॒ व॒त॒ वा॒जि॒नः॒ । \newline
52. वा॒जि॒नो॒ नो॒ नो॒ वा॒जि॒नो॒ वा॒जि॒नो॒ नः॒ । \newline
53. नो॒ धने॑षु॒ धने॑षु नो नो॒ धने॑षु । \newline
54. धने॑षु विप्रा विप्रा॒ धने॑षु॒ धने॑षु विप्राः । \newline

\textbf{Ghana Paata } \newline

1. वाजो॑ नो नो॒ वाजो॒ वाजो॑ नः स॒प्त स॒प्त नो॒ वाजो॒ वाजो॑ नः स॒प्त । \newline
2. नः॒ स॒प्त स॒प्त नो॑ नः स॒प्त प्र॒दिशः॑ प्र॒दिशः॑ स॒प्त नो॑ नः स॒प्त प्र॒दिशः॑ । \newline
3. स॒प्त प्र॒दिशः॑ प्र॒दिशः॑ स॒प्त स॒प्त प्र॒दिश॒ श्चत॑स्र॒ श्चत॑स्रः प्र॒दिशः॑ स॒प्त स॒प्त प्र॒दिश॒ श्चत॑स्रः । \newline
4. प्र॒दिश॒ श्चत॑स्र॒ श्चत॑स्रः प्र॒दिशः॑ प्र॒दिश॒ श्चत॑स्रो वा वा॒ चत॑स्रः प्र॒दिशः॑ प्र॒दिश॒ श्चत॑स्रो वा । \newline
5. प्र॒दिश॒ इति॑ प्र - दिशः॑ । \newline
6. चत॑स्रो वा वा॒ चत॑स्र॒ श्चत॑स्रो वा परा॒वतः॑ परा॒वतो॑ वा॒ चत॑स्र॒ श्चत॑स्रो वा परा॒वतः॑ । \newline
7. वा॒ प॒रा॒वतः॑ परा॒वतो॑ वा वा परा॒वतः॑ । \newline
8. प॒रा॒वत॒ इति॑ परा - वतः॑ । \newline
9. वाजो॑ नो नो॒ वाजो॒ वाजो॑ नो॒ विश्वै॒र् विश्वै᳚र् नो॒ वाजो॒ वाजो॑ नो॒ विश्वैः᳚ । \newline
10. नो॒ विश्वै॒र् विश्वै᳚र् नो नो॒ विश्वै᳚र् दे॒वैर् दे॒वैर् विश्वै᳚र् नो नो॒ विश्वै᳚र् दे॒वैः । \newline
11. विश्वै᳚र् दे॒वैर् दे॒वैर् विश्वै॒र् विश्वै᳚र् दे॒वैर् धन॑सातौ॒ धन॑सातौ दे॒वैर् विश्वै॒र् विश्वै᳚र् दे॒वैर् धन॑सातौ । \newline
12. दे॒वैर् धन॑सातौ॒ धन॑सातौ दे॒वैर् दे॒वैर् धन॑साता वि॒हेह धन॑सातौ दे॒वैर् दे॒वैर् धन॑साता वि॒ह । \newline
13. धन॑साता वि॒हेह धन॑सातौ॒ धन॑साता वि॒हाव॑ त्वव त्वि॒ह धन॑सातौ॒ धन॑साता वि॒हाव॑तु । \newline
14. धन॑साता॒विति॒ धन॑ - सा॒तौ॒ । \newline
15. इ॒हाव॑ त्ववत्वि॒ हेहाव॑तु । \newline
16. अ॒व॒त्वित्य॑वतु । \newline
17. विश्वे॑ अ॒द्याद्य विश्वे॒ विश्वे॑ अ॒द्य म॒रुतो॑ म॒रुतो॑ अ॒द्य विश्वे॒ विश्वे॑ अ॒द्य म॒रुतः॑ । \newline
18. अ॒द्य म॒रुतो॑ म॒रुतो॑ अ॒द्याद्य म॒रुतो॒ विश्वे॒ विश्वे॑ म॒रुतो॑ अ॒द्याद्य म॒रुतो॒ विश्वे᳚ । \newline
19. म॒रुतो॒ विश्वे॒ विश्वे॑ म॒रुतो॑ म॒रुतो॒ विश्व॑ ऊ॒त् यू॑ती विश्वे॑ म॒रुतो॑ म॒रुतो॒ विश्व॑ ऊ॒ती । \newline
20. विश्व॑ ऊ॒त् यू॑ती विश्वे॒ विश्व॑ ऊ॒ती विश्वे॒ विश्व॑ ऊ॒ती विश्वे॒ विश्व॑ ऊ॒ती विश्वे᳚ । \newline
21. ऊ॒ती विश्वे॒ विश्व॑ ऊ॒त् यू॑ती विश्वे॑ भवन्तु भवन्तु॒ विश्व॑ ऊ॒त् यू॑ती विश्वे॑ भवन्तु । \newline
22. विश्वे॑ भवन्तु भवन्तु॒ विश्वे॒ विश्वे॑ भवन् त्व॒ग्नयो॑ अ॒ग्नयो॑ भवन्तु॒ विश्वे॒ विश्वे॑ भवन् त्व॒ग्नयः॑ । \newline
23. भ॒व॒न् त्व॒ग्नयो॑ अ॒ग्नयो॑ भवन्तु भवन् त्व॒ग्नयः॒ समि॑द्धाः॒ समि॑द्धा अ॒ग्नयो॑ भवन्तु भवन् त्व॒ग्नयः॒ समि॑द्धाः । \newline
24. अ॒ग्नयः॒ समि॑द्धाः॒ समि॑द्धा अ॒ग्नयो॑ अ॒ग्नयः॒ समि॑द्धाः । \newline
25. समि॑द्धा॒ इति॒ सं - इ॒द्धाः॒ । \newline
26. विश्वे॑ नो नो॒ विश्वे॒ विश्वे॑ नो दे॒वा दे॒वा नो॒ विश्वे॒ विश्वे॑ नो दे॒वाः । \newline
27. नो॒ दे॒वा दे॒वा नो॑ नो दे॒वा अव॒सा ऽव॑सा दे॒वा नो॑ नो दे॒वा अव॑सा । \newline
28. दे॒वा अव॒सा ऽव॑सा दे॒वा दे॒वा अव॒सा ऽऽव॑सा दे॒वा दे॒वा अव॒सा । \newline
29. अव॒सा ऽऽव॒सा ऽव॒सा ऽऽग॑मन्तु गम॒न्त्वा ऽव॒सा ऽव॒सा ऽऽग॑मन्तु । \newline
30. आ ग॑मन्तु गम॒न्त्वा ग॑मन्तु॒ विश्वं॒ ॅविश्व॑म् गम॒न्त्वा ग॑मन्तु॒ विश्व᳚म् । \newline
31. ग॒म॒न्तु॒ विश्वं॒ ॅविश्व॑म् गमन्तु गमन्तु॒ विश्व॑ मस्त्वस्तु॒ विश्व॑म् गमन्तु गमन्तु॒ विश्व॑ मस्तु । \newline
32. विश्व॑ मस्त्वस्तु॒ विश्वं॒ ॅविश्व॑ मस्तु॒ द्रवि॑ण॒म् द्रवि॑ण मस्तु॒ विश्वं॒ ॅविश्व॑ मस्तु॒ द्रवि॑णम् । \newline
33. अ॒स्तु॒ द्रवि॑ण॒म् द्रवि॑ण मस्त्वस्तु॒ द्रवि॑णं॒ ॅवाजो॒ वाजो॒ द्रवि॑ण मस्त्वस्तु॒ द्रवि॑णं॒ ॅवाजः॑ । \newline
34. द्रवि॑णं॒ ॅवाजो॒ वाजो॒ द्रवि॑ण॒म् द्रवि॑णं॒ ॅवाजो॑ अ॒स्मे अ॒स्मे वाजो॒ द्रवि॑ण॒म् द्रवि॑णं॒ ॅवाजो॑ अ॒स्मे । \newline
35. वाजो॑ अ॒स्मे अ॒स्मे वाजो॒ वाजो॑ अ॒स्मे । \newline
36. अ॒स्मे इत्य॒स्मे । \newline
37. वाज॑स्य प्रस॒वम् प्र॑स॒वं ॅवाज॑स्य॒ वाज॑स्य प्रस॒वम् दे॑वा देवाः प्रस॒वं ॅवाज॑स्य॒ वाज॑स्य प्रस॒वम् दे॑वाः । \newline
38. प्र॒स॒वम् दे॑वा देवाः प्रस॒वम् प्र॑स॒वम् दे॑वा॒ रथै॒र् रथै᳚र् देवाः प्रस॒वम् प्र॑स॒वम् दे॑वा॒ रथैः᳚ । \newline
39. प्र॒स॒वमिति॑ प्र - स॒वम् । \newline
40. दे॒वा॒ रथै॒र् रथै᳚र् देवा देवा॒ रथै᳚र् यात यात॒ रथै᳚र् देवा देवा॒ रथै᳚र् यात । \newline
41. रथै᳚र् यात यात॒ रथै॒र् रथै᳚र् याता हिर॒ण्ययैर्॑. हिर॒ण्ययै᳚र् यात॒ रथै॒र् रथै᳚र् याता हिर॒ण्ययैः᳚ । \newline
42. या॒ता॒ हि॒र॒ण्ययैर्॑. हिर॒ण्ययै᳚र् यात याता हिर॒ण्ययैः᳚ । \newline
43. हि॒र॒ण्ययै॒रिति॑ हिर॒ण्ययैः᳚ । \newline
44. अ॒ग्निरिन्द्र॒ इन्द्रो॑ अ॒ग्नि र॒ग्नि रिन्द्रो॒ बृह॒स्पति॒र् बृह॒स्पति॒ रिन्द्रो॑ अ॒ग्नि र॒ग्नि रिन्द्रो॒ बृह॒स्पतिः॑ । \newline
45. इन्द्रो॒ बृह॒स्पति॒र् बृह॒स्पति॒ रिन्द्र॒ इन्द्रो॒ बृह॒स्पति॑र् म॒रुतो॑ म॒रुतो॒ बृह॒स्पति॒ रिन्द्र॒ इन्द्रो॒ बृह॒स्पति॑र् म॒रुतः॑ । \newline
46. बृह॒स्पति॑र् म॒रुतो॑ म॒रुतो॒ बृह॒स्पति॒र् बृह॒स्पति॑र् म॒रुतः॒ सोम॑पीतये॒ सोम॑पीतये म॒रुतो॒ बृह॒स्पति॒र् बृह॒स्पति॑र् म॒रुतः॒ सोम॑पीतये । \newline
47. म॒रुतः॒ सोम॑पीतये॒ सोम॑पीतये म॒रुतो॑ म॒रुतः॒ सोम॑पीतये । \newline
48. सोम॑पीतय॒ इति॒ सोम॑ - पी॒त॒ये॒ । \newline
49. वाजे॑वाजे ऽवतावत॒ वाजे॑वाजे॒ वाजे॑वाजे ऽवत वाजिनो वाजिनो ऽवत॒ वाजे॑वाजे॒ वाजे॑वाजे ऽवत वाजिनः । \newline
50. वाजे॑वाज॒ इति॒ वाजे᳚ - वा॒जे॒ । \newline
51. अ॒व॒त॒ वा॒जि॒नो॒ वा॒जि॒नो॒ ऽव॒ता॒व॒त॒ वा॒जि॒नो॒ नो॒ नो॒ वा॒जि॒नो॒ ऽव॒ता॒व॒त॒ वा॒जि॒नो॒ नः॒ । \newline
52. वा॒जि॒नो॒ नो॒ नो॒ वा॒जि॒नो॒ वा॒जि॒नो॒ नो॒ धने॑षु॒ धने॑षु नो वाजिनो वाजिनो नो॒ धने॑षु । \newline
53. नो॒ धने॑षु॒ धने॑षु नो नो॒ धने॑षु विप्रा विप्रा॒ धने॑षु नो नो॒ धने॑षु विप्राः । \newline
54. धने॑षु विप्रा विप्रा॒ धने॑षु॒ धने॑षु विप्रा अमृता अमृता विप्रा॒ धने॑षु॒ धने॑षु विप्रा अमृताः । \newline
\pagebreak
\markright{ TS 4.7.12.2  \hfill https://www.vedavms.in \hfill}

\section{ TS 4.7.12.2 }

\textbf{TS 4.7.12.2 } \newline
\textbf{Samhita Paata} \newline

विप्रा अमृता ऋतज्ञाः । अ॒स्य मद्ध्वः॑ पिबत मा॒दय॑द्ध्वं तृ॒प्ता या॑त प॒थिभि॑र्देव॒यानैः᳚ ॥ वाजः॑ पु॒रस्ता॑दु॒त म॑द्ध्य॒तो नो॒ वाजो॑ दे॒वाꣳ ऋ॒तुभिः॑ कल्पयाति । वाज॑स्य॒ हि प्र॑स॒वो नन्न॑मीति॒ विश्वा॒ आशा॒ वाज॑पतिर्भवेयं ॥ पयः॑ पृथि॒व्यां पय॒ ओष॑धीषु॒ पयो॑ दि॒व्य॑न्तरि॑क्षे॒ पयो॑ धां । पय॑स्वतीः प्र॒दिशः॑ सन्तु॒ मह्यं᳚ ॥ सं मा॑ सृजामि॒ पय॑सा घृ॒तेन॒ सं मा॑ सृजाम्य॒प - [  ] \newline

\textbf{Pada Paata} \newline

वि॒प्राः॒ । अ॒मृ॒ताः॒ । ऋ॒त॒ज्ञा॒ इत्यृ॑त - ज्ञाः॒ ॥ अ॒स्य । मद्ध्वः॑ । पि॒ब॒त॒ । मा॒दय॑द्ध्वम् । तृ॒प्ताः । या॒त॒ । प॒थिभि॒रिति॑ प॒थि - भिः॒ । दे॒व॒यानै॒रिति॑ देव - यानैः᳚ ॥ वाजः॑ । पु॒रस्ता᳚त् । उ॒त । म॒द्ध्य॒तः । नः॒ । वाजः॑ । दे॒वान् । ऋ॒तुभि॒रित्यृ॒तु - भिः॒ । क॒ल्प॒या॒ति॒ ॥ वाज॑स्य । हि । प्र॒स॒व इति॑ प्र - स॒वः । नन्न॑मीति । विश्वाः᳚ । आशाः᳚ । वाज॑पति॒रिति॒ वाज॑ - प॒तिः॒ । भ॒वे॒य॒म् ॥ पयः॑ । पृ॒थि॒व्याम् । पयः॑ । ओष॑धीषु । पयः॑ । दि॒वि । अ॒न्तरि॑क्षे । पयः॑ । धा॒म् ॥ पय॑स्वतीः । प्र॒दिश॒ इति॑ प्र - दिशः॑ । स॒न्तु॒ । मह्य᳚म् ॥ समिति॑ । मा॒ । सृ॒जा॒मि॒ । पय॑सा । घृ॒तेन॑ । समिति॑ । मा॒ । सृ॒जा॒मि॒ । अ॒पः ।  \newline


\textbf{Krama Paata} \newline

वि॒प्रा॒ अ॒मृ॒ताः॒ । अ॒मृ॒ता॒ ऋ॒त॒ज्ञाः॒ । ऋ॒त॒ज्ञा॒ इत्यृ॑त - ज्ञाः॒ ॥ अ॒स्य मद्ध्वः॑ । मद्ध्वः॑ पिबत । पि॒ब॒त॒ मा॒दय॑द्ध्वम् । मा॒दय॑द्ध्वम् तृ॒प्ताः । तृ॒प्ता या॑त । या॒त॒ प॒थिभिः॑ । प॒थिभि॑र् देव॒यानैः᳚ । प॒थिभि॒रिति॑ प॒थि - भिः॒ । दे॒व॒यानै॒रिति॑ देव - यानैः᳚ । वाजः॑ पु॒रस्ता᳚त् । पु॒रस्ता॑दु॒त । उ॒त म॑द्ध्य॒तः । म॒द्ध्य॒तो नः॑ । नो॒ वाजः॑ । वाजो॑ दे॒वान् । दे॒वाꣳ ऋ॒तुभिः॑ । ऋ॒तुभिः॑ कल्पयाति । ऋ॒तुभि॒रित्यृ॒तु - भिः॒ । क॒ल्प॒या॒तीति॑ कल्पयाति ॥ वाज॑स्य॒ हि । हि प्र॑स॒वः । प्र॒स॒वो नन्न॑मीति । प्र॒स॒व इति॑ प्र - स॒वः । नन्न॑मीति॒ विश्वाः᳚ । विश्वा॒ आशाः᳚ । आशा॒ वाज॑पतिः । वाज॑पतिर् भवेयम् । वाज॑पति॒रिति॒ वाज॑ - प॒तिः॒ । भ॒वे॒य॒मिति॑ भवेयम् ॥ पयः॑ पृथि॒व्याम् । पृ॒थि॒व्याम् पयः॑ । पय॒ ओष॑धीषु । ओष॑धीषु॒ पयः॑ । पयो॑ दि॒वि । दि॒व्य॑न्तरि॑क्षे । अ॒न्तरि॑क्षे॒ पयः॑ । पयो॑ धाम् । धा॒मिति॑ धाम् ॥ पय॑स्वतीः प्र॒दिशः॑ । प्र॒दिशः॑ सन्तु । प्र॒दिश॒ इति॑ प्र - दिशः॑ । स॒न्तु॒ मह्य᳚म् । मह्य॒मिति॒ मह्य᳚म् ॥ सम् मा᳚ । मा॒ सृ॒जा॒मि॒ । सृ॒जा॒मि॒ पय॑सा । पय॑सा घृ॒तेन॑ । घृ॒तेन॒ सम् । सम् मा᳚ । मा॒ सृ॒जा॒मि॒ । सृ॒जा॒म्य॒पः ( ) । अ॒प ओष॑धीभिः \newline

\textbf{Jatai Paata} \newline

1. वि॒प्रा॒ अ॒मृ॒ता॒ अ॒मृ॒ता॒ वि॒प्रा॒ वि॒प्रा॒ अ॒मृ॒ताः॒ । \newline
2. अ॒मृ॒ता॒ ऋ॒त॒ज्ञा॒ ऋ॒त॒ज्ञा॒ अ॒मृ॒ता॒ अ॒मृ॒ता॒ ऋ॒त॒ज्ञाः॒ । \newline
3. ऋ॒त॒ज्ञा॒ इत्यृ॑त - ज्ञाः॒ । \newline
4. अ॒स्य मद्ध्वो॒ मद्ध्वो॑ अ॒स्यास्य मद्ध्वः॑ । \newline
5. मद्ध्वः॑ पिबत पिबत॒ मद्ध्वो॒ मद्ध्वः॑ पिबत । \newline
6. पि॒ब॒त॒ मा॒दय॑द्ध्वम् मा॒दय॑द्ध्वम् पिबत पिबत मा॒दय॑द्ध्वम् । \newline
7. मा॒दय॑द्ध्वम् तृ॒प्ता स्तृ॒प्ता मा॒दय॑द्ध्वम् मा॒दय॑द्ध्वम् तृ॒प्ताः । \newline
8. तृ॒प्ता या॑त यात तृ॒प्ता स्तृ॒प्ता या॑त । \newline
9. या॒त॒ प॒थिभिः॑ प॒थिभि॑र् यात यात प॒थिभिः॑ । \newline
10. प॒थिभि॑र् देव॒यानै᳚र् देव॒यानैः᳚ प॒थिभिः॑ प॒थिभि॑र् देव॒यानैः᳚ । \newline
11. प॒थिभि॒रिति॑ प॒थि - भिः॒ । \newline
12. दे॒व॒यानै॒रिति॑ देव - यानैः᳚ । \newline
13. वाजः॑ पु॒रस्ता᳚त् पु॒रस्ता॒द् वाजो॒ वाजः॑ पु॒रस्ता᳚त् । \newline
14. पु॒रस्ता॑ दु॒तोत पु॒रस्ता᳚त् पु॒रस्ता॑ दु॒त । \newline
15. उ॒त म॑द्ध्य॒तो म॑द्ध्य॒त उ॒तोत म॑द्ध्य॒तः । \newline
16. म॒द्ध्य॒तो नो॑ नो मद्ध्य॒तो म॑द्ध्य॒तो नः॑ । \newline
17. नो॒ वाजो॒ वाजो॑ नो नो॒ वाजः॑ । \newline
18. वाजो॑ दे॒वान् दे॒वान्. वाजो॒ वाजो॑ दे॒वान् । \newline
19. दे॒वाꣳ ऋ॒तुभि॑र्. ऋ॒तुभि॑र् दे॒वान् दे॒वाꣳ ऋ॒तुभिः॑ । \newline
20. ऋ॒तुभिः॑ कल्पयाति कल्पया त्यृ॒तुभि॑र्. ऋ॒तुभिः॑ कल्पयाति । \newline
21. ऋ॒तुभि॒रित्यृ॒तु - भिः॒ । \newline
22. क॒ल्प॒या॒तीति॑ कल्पयाति । \newline
23. वाज॑स्य॒ हि हि वाज॑स्य॒ वाज॑स्य॒ हि । \newline
24. हि प्र॑स॒वः प्र॑स॒वो हि हि प्र॑स॒वः । \newline
25. प्र॒स॒वो नन्न॑मीति॒ नन्न॑मीति प्रस॒वः प्र॑स॒वो नन्न॑मीति । \newline
26. प्र॒स॒व इति॑ प्र - स॒वः । \newline
27. नन्न॑मीति॒ विश्वा॒ विश्वा॒ नन्न॑मीति॒ नन्न॑मीति॒ विश्वाः᳚ । \newline
28. विश्वा॒ आशा॒ आशा॒ विश्वा॒ विश्वा॒ आशाः᳚ । \newline
29. आशा॒ वाज॑पति॒र् वाज॑पति॒ राशा॒ आशा॒ वाज॑पतिः । \newline
30. वाज॑पतिर् भवेयम् भवेयं॒ ॅवाज॑पति॒र् वाज॑पतिर् भवेयम् । \newline
31. वाज॑पति॒रिति॒ वाज॑ - प॒तिः॒ । \newline
32. भ॒वे॒य॒मिति॑ भवेयम् । \newline
33. पयः॑ पृथि॒व्याम् पृ॑थि॒व्याम् पयः॒ पयः॑ पृथि॒व्याम् । \newline
34. पृ॒थि॒व्याम् पयः॒ पयः॑ पृथि॒व्याम् पृ॑थि॒व्याम् पयः॑ । \newline
35. पय॒ ओष॑धी॒ ष्वोष॑धीषु॒ पयः॒ पय॒ ओष॑धीषु । \newline
36. ओष॑धीषु॒ पयः॒ पय॒ ओष॑धी॒ ष्वोष॑धीषु॒ पयः॑ । \newline
37. पयो॑ दि॒वि दि॒वि पयः॒ पयो॑ दि॒वि । \newline
38. दि॒व्य॑न्तरि॑क्षे अ॒न्तरि॑क्षे दि॒वि दि॒व्य॑न्तरि॑क्षे । \newline
39. अ॒न्तरि॑क्षे॒ पयः॒ पयो॑ अ॒न्तरि॑क्षे अ॒न्तरि॑क्षे॒ पयः॑ । \newline
40. पयो॑ धाम् धा॒म् पयः॒ पयो॑ धाम् । \newline
41. धा॒मिति॑ धाम् । \newline
42. पय॑स्वतीः प्र॒दिशः॑ प्र॒दिशः॒ पय॑स्वतीः॒ पय॑स्वतीः प्र॒दिशः॑ । \newline
43. प्र॒दिशः॑ सन्तु सन्तु प्र॒दिशः॑ प्र॒दिशः॑ सन्तु । \newline
44. प्र॒दिश॒ इति॑ प्र - दिशः॑ । \newline
45. स॒न्तु॒ मह्य॒म् मह्य(ग्म्॑) सन्तु सन्तु॒ मह्य᳚म् । \newline
46. मह्य॒मिति॒ मह्य᳚म् । \newline
47. सम् मा॑ मा॒ सꣳ सम् मा᳚ । \newline
48. मा॒ सृ॒जा॒मि॒ सृ॒जा॒मि॒ मा॒ मा॒ सृ॒जा॒मि॒ । \newline
49. सृ॒जा॒मि॒ पय॑सा॒ पय॑सा सृजामि सृजामि॒ पय॑सा । \newline
50. पय॑सा घृ॒तेन॑ घृ॒तेन॒ पय॑सा॒ पय॑सा घृ॒तेन॑ । \newline
51. घृ॒तेन॒ सꣳ सम् घृ॒तेन॑ घृ॒तेन॒ सम् । \newline
52. सम् मा॑ मा॒ सꣳ सम् मा᳚ । \newline
53. मा॒ सृ॒जा॒मि॒ सृ॒जा॒मि॒ मा॒ मा॒ सृ॒जा॒मि॒ । \newline
54. सृ॒जा॒ म्य॒पो अ॒पः सृ॑जामि सृजा म्य॒पः । \newline
55. अ॒प ओष॑धीभि॒ रोष॑धीभि र॒पो अ॒प ओष॑धीभिः । \newline

\textbf{Ghana Paata } \newline

1. वि॒प्रा॒ अ॒मृ॒ता॒ अ॒मृ॒ता॒ वि॒प्रा॒ वि॒प्रा॒ अ॒मृ॒ता॒ ऋ॒त॒ज्ञा॒ ऋ॒त॒ज्ञा॒ अ॒मृ॒ता॒ वि॒प्रा॒ वि॒प्रा॒ अ॒मृ॒ता॒ ऋ॒त॒ज्ञाः॒ । \newline
2. अ॒मृ॒ता॒ ऋ॒त॒ज्ञा॒ ऋ॒त॒ज्ञा॒ अ॒मृ॒ता॒ अ॒मृ॒ता॒ ऋ॒त॒ज्ञाः॒ । \newline
3. ऋ॒त॒ज्ञा॒ इत्यृ॑त - ज्ञाः॒ । \newline
4. अ॒स्य मद्ध्वो॒ मद्ध्वो॑ अ॒स्यास्य मद्ध्वः॑ पिबत पिबत॒ मद्ध्वो॑ अ॒स्यास्य मद्ध्वः॑ पिबत । \newline
5. मद्ध्वः॑ पिबत पिबत॒ मद्ध्वो॒ मद्ध्वः॑ पिबत मा॒दय॑द्ध्वम् मा॒दय॑द्ध्वम् पिबत॒ मद्ध्वो॒ मद्ध्वः॑ पिबत मा॒दय॑द्ध्वम् । \newline
6. पि॒ब॒त॒ मा॒दय॑द्ध्वम् मा॒दय॑द्ध्वम् पिबत पिबत मा॒दय॑द्ध्वम् तृ॒प्ता स्तृ॒प्ता मा॒दय॑द्ध्वम् पिबत पिबत मा॒दय॑द्ध्वम् तृ॒प्ताः । \newline
7. मा॒दय॑द्ध्वम् तृ॒प्ता स्तृ॒प्ता मा॒दय॑द्ध्वम् मा॒दय॑द्ध्वम् तृ॒प्ता या॑त यात तृ॒प्ता मा॒दय॑द्ध्वम् मा॒दय॑द्ध्वम् तृ॒प्ता या॑त । \newline
8. तृ॒प्ता या॑त यात तृ॒प्ता स्तृ॒प्ता या॑त प॒थिभिः॑ प॒थिभि॑र् यात तृ॒प्ता स्तृ॒प्ता या॑त प॒थिभिः॑ । \newline
9. या॒त॒ प॒थिभिः॑ प॒थिभि॑र् यात यात प॒थिभि॑र् देव॒यानै᳚र् देव॒यानैः᳚ प॒थिभि॑र् यात यात प॒थिभि॑र् देव॒यानैः᳚ । \newline
10. प॒थिभि॑र् देव॒यानै᳚र् देव॒यानैः᳚ प॒थिभिः॑ प॒थिभि॑र् देव॒यानैः᳚ । \newline
11. प॒थिभि॒रिति॑ प॒थि - भिः॒ । \newline
12. दे॒व॒यानै॒रिति॑ देव - यानैः᳚ । \newline
13. वाजः॑ पु॒रस्ता᳚त् पु॒रस्ता॒द् वाजो॒ वाजः॑ पु॒रस्ता॑ दु॒तोत पु॒रस्ता॒द् वाजो॒ वाजः॑ पु॒रस्ता॑ दु॒त । \newline
14. पु॒रस्ता॑ दु॒तोत पु॒रस्ता᳚त् पु॒रस्ता॑ दु॒त म॑द्ध्य॒तो म॑द्ध्य॒त उ॒त पु॒रस्ता᳚त् पु॒रस्ता॑ दु॒त म॑द्ध्य॒तः । \newline
15. उ॒त म॑द्ध्य॒तो म॑द्ध्य॒त उ॒तोत म॑द्ध्य॒तो नो॑ नो मद्ध्य॒त उ॒तोत म॑द्ध्य॒तो नः॑ । \newline
16. म॒द्ध्य॒तो नो॑ नो मद्ध्य॒तो म॑द्ध्य॒तो नो॒ वाजो॒ वाजो॑ नो मद्ध्य॒तो म॑द्ध्य॒तो नो॒ वाजः॑ । \newline
17. नो॒ वाजो॒ वाजो॑ नो नो॒ वाजो॑ दे॒वान् दे॒वान्. वाजो॑ नो नो॒ वाजो॑ दे॒वान् । \newline
18. वाजो॑ दे॒वान् दे॒वान्. वाजो॒ वाजो॑ दे॒वाꣳ ऋ॒तुभि॑र्. ऋ॒तुभि॑र् दे॒वान्. वाजो॒ वाजो॑ दे॒वाꣳ ऋ॒तुभिः॑ । \newline
19. दे॒वान् ऋ॒तुभि॑र्. ऋ॒तुभि॑र् दे॒वान् दे॒वाꣳ ऋ॒तुभिः॑ कल्पयाति कल्पया त्यृ॒तुभि॑र् दे॒वान् दे॒वाꣳ ऋ॒तुभिः॑ कल्पयाति । \newline
20. ऋ॒तुभिः॑ कल्पयाति कल्पया त्यृ॒तुभि॑र्. ऋ॒तुभिः॑ कल्पयाति । \newline
21. ऋ॒तुभि॒रित्यृ॒तु - भिः॒ । \newline
22. क॒ल्प॒या॒तीति॑ कल्पयाति । \newline
23. वाज॑स्य॒ हि हि वाज॑स्य॒ वाज॑स्य॒ हि प्र॑स॒वः प्र॑स॒वो हि वाज॑स्य॒ वाज॑स्य॒ हि प्र॑स॒वः । \newline
24. हि प्र॑स॒वः प्र॑स॒वो हि हि प्र॑स॒वो नन्न॑मीति॒ नन्न॑मीति प्रस॒वो हि हि प्र॑स॒वो नन्न॑मीति । \newline
25. प्र॒स॒वो नन्न॑मीति॒ नन्न॑मीति प्रस॒वः प्र॑स॒वो नन्न॑मीति॒ विश्वा॒ विश्वा॒ नन्न॑मीति प्रस॒वः प्र॑स॒वो नन्न॑मीति॒ विश्वाः᳚ । \newline
26. प्र॒स॒व इति॑ प्र - स॒वः । \newline
27. नन्न॑मीति॒ विश्वा॒ विश्वा॒ नन्न॑मीति॒ नन्न॑मीति॒ विश्वा॒ आशा॒ आशा॒ विश्वा॒ नन्न॑मीति॒ नन्न॑मीति॒ विश्वा॒ आशाः᳚ । \newline
28. विश्वा॒ आशा॒ आशा॒ विश्वा॒ विश्वा॒ आशा॒ वाज॑पति॒र् वाज॑पति॒ राशा॒ विश्वा॒ विश्वा॒ आशा॒ वाज॑पतिः । \newline
29. आशा॒ वाज॑पति॒र् वाज॑पति॒ राशा॒ आशा॒ वाज॑पतिर् भवेयम् भवेयं॒ ॅवाज॑पति॒ राशा॒ आशा॒ वाज॑पतिर् भवेयम् । \newline
30. वाज॑पतिर् भवेयम् भवेयं॒ ॅवाज॑पति॒र् वाज॑पतिर् भवेयम् । \newline
31. वाज॑पति॒रिति॒ वाज॑ - प॒तिः॒ । \newline
32. भ॒वे॒य॒मिति॑ भवेयम् । \newline
33. पयः॑ पृथि॒व्याम् पृ॑थि॒व्याम् पयः॒ पयः॑ पृथि॒व्याम् पयः॒ पयः॑ पृथि॒व्याम् पयः॒ पयः॑ पृथि॒व्याम् पयः॑ । \newline
34. पृ॒थि॒व्याम् पयः॒ पयः॑ पृथि॒व्याम् पृ॑थि॒व्याम् पय॒ ओष॑धी॒ ष्वोष॑धीषु॒ पयः॑ पृथि॒व्याम् पृ॑थि॒व्याम् पय॒ ओष॑धीषु । \newline
35. पय॒ ओष॑धी॒ ष्वोष॑धीषु॒ पयः॒ पय॒ ओष॑धीषु॒ पयः॒ पय॒ ओष॑धीषु॒ पयः॒ पय॒ ओष॑धीषु॒ पयः॑ । \newline
36. ओष॑धीषु॒ पयः॒ पय॒ ओष॑धी॒ ष्वोष॑धीषु॒ पयो॑ दि॒वि दि॒वि पय॒ ओष॑धी॒ ष्वोष॑धीषु॒ पयो॑ दि॒वि । \newline
37. पयो॑ दि॒वि दि॒वि पयः॒ पयो॑ दि॒व्य॑न्तरि॑क्षे अ॒न्तरि॑क्षे दि॒वि पयः॒ पयो॑ दि॒व्य॑न्तरि॑क्षे । \newline
38. दि॒व्य॑न्तरि॑क्षे अ॒न्तरि॑क्षे दि॒वि दि॒व्य॑न्तरि॑क्षे॒ पयः॒ पयो॑ अ॒न्तरि॑क्षे दि॒वि दि॒व्य॑न्तरि॑क्षे॒ पयः॑ । \newline
39. अ॒न्तरि॑क्षे॒ पयः॒ पयो॑ अ॒न्तरि॑क्षे अ॒न्तरि॑क्षे॒ पयो॑ धाम् धा॒म् पयो॑ अ॒न्तरि॑क्षे अ॒न्तरि॑क्षे॒ पयो॑ धाम् । \newline
40. पयो॑ धाम् धा॒म् पयः॒ पयो॑ धाम् । \newline
41. धा॒मिति॑ धाम् । \newline
42. पय॑स्वतीः प्र॒दिशः॑ प्र॒दिशः॒ पय॑स्वतीः॒ पय॑स्वतीः प्र॒दिशः॑ सन्तु सन्तु प्र॒दिशः॒ पय॑स्वतीः॒ पय॑स्वतीः प्र॒दिशः॑ सन्तु । \newline
43. प्र॒दिशः॑ सन्तु सन्तु प्र॒दिशः॑ प्र॒दिशः॑ सन्तु॒ मह्य॒म् मह्य(ग्म्॑) सन्तु प्र॒दिशः॑ प्र॒दिशः॑ सन्तु॒ मह्य᳚म् । \newline
44. प्र॒दिश॒ इति॑ प्र - दिशः॑ । \newline
45. स॒न्तु॒ मह्य॒म् मह्य(ग्म्॑) सन्तु सन्तु॒ मह्य᳚म् । \newline
46. मह्य॒मिति॒ मह्य᳚म् । \newline
47. सम् मा॑ मा॒ सꣳ सम् मा॑ सृजामि सृजामि मा॒ सꣳ सम् मा॑ सृजामि । \newline
48. मा॒ सृ॒जा॒मि॒ सृ॒जा॒मि॒ मा॒ मा॒ सृ॒जा॒मि॒ पय॑सा॒ पय॑सा सृजामि मा मा सृजामि॒ पय॑सा । \newline
49. सृ॒जा॒मि॒ पय॑सा॒ पय॑सा सृजामि सृजामि॒ पय॑सा घृ॒तेन॑ घृ॒तेन॒ पय॑सा सृजामि सृजामि॒ पय॑सा घृ॒तेन॑ । \newline
50. पय॑सा घृ॒तेन॑ घृ॒तेन॒ पय॑सा॒ पय॑सा घृ॒तेन॒ सꣳ सम् घृ॒तेन॒ पय॑सा॒ पय॑सा घृ॒तेन॒ सम् । \newline
51. घृ॒तेन॒ सꣳ सम् घृ॒तेन॑ घृ॒तेन॒ सम् मा॑ मा॒ सम् घृ॒तेन॑ घृ॒तेन॒ सम् मा᳚ । \newline
52. सम् मा॑ मा॒ सꣳ सम् मा॑ सृजामि सृजामि मा॒ सꣳ सम् मा॑ सृजामि । \newline
53. मा॒ सृ॒जा॒मि॒ सृ॒जा॒मि॒ मा॒ मा॒ सृ॒जा॒ म्य॒पो अ॒पः सृ॑जामि मा मा सृजाम्य॒पः । \newline
54. सृ॒जा॒ म्य॒पो अ॒पः सृ॑जामि सृजा म्य॒प ओष॑धीभि॒ रोष॑धीभि र॒पः सृ॑जामि सृजा म्य॒प ओष॑धीभिः । \newline
55. अ॒प ओष॑धीभि॒ रोष॑धीभि र॒पो अ॒प ओष॑धीभिः । \newline
\pagebreak
\markright{ TS 4.7.12.3  \hfill https://www.vedavms.in \hfill}

\section{ TS 4.7.12.3 }

\textbf{TS 4.7.12.3 } \newline
\textbf{Samhita Paata} \newline

ओष॑धीभिः । सो॑ऽहं ॅवाजꣳ॑ सनेयमग्ने ॥ नक्तो॒षासा॒ सम॑नसा॒ विरू॑पे धा॒पये॑ते॒ शिशु॒मेकꣳ॑ समी॒ची । द्यावा॒ क्षामा॑ रु॒क्मो अ॒न्तर्वि भा॑ति दे॒वा अ॒ग्निं धा॑रयन् द्रविणो॒दाः ॥ स॒मु॒द्रो॑ऽसि॒ नभ॑स्वाना॒र्द्रदा॑नुः श॒भूंर्म॑यो॒भूर॒भि मा॑ वाहि॒ स्वाहा॑ मारु॒तो॑ऽसि म॒रुतां᳚ ग॒णः श॒भूंर्म॑यो॒भूर॒भि मा॑ वाहि॒ स्वाहा॑ऽव॒स्युर॑सि॒ दुव॑स्वाञ्छ॒भूंर्म॑यो॒भूरभि मा॑ ( ) वाहि॒ स्वाहा᳚ ॥ \newline

\textbf{Pada Paata} \newline

ओष॑धीभि॒रित्योष॑धि - भिः॒ ॥ सः । अ॒हम् । वाज᳚म् । स॒ने॒य॒म् । अ॒ग्ने॒ ॥ नक्तो॒षासा᳚ । सम॑न॒सेति॒ स-म॒न॒सा॒ । विरू॑पे॒ इति॒ वि-रू॒पे॒ । धा॒पये॑ते॒ इति॑ । शिशु᳚म् । एक᳚म् । स॒मी॒ची इति॑ ॥ द्यावा᳚ । क्षाम॑ । रु॒क्मः । अ॒न्तः । वीति॑ । भा॒ति॒ । दे॒वाः । अ॒ग्निम् । धा॒र॒य॒न्न् । द्र॒वि॒णो॒दा इति॑ द्रविणः - दाः ॥ स॒मु॒द्रः । अ॒सि॒ । नभ॑स्वान् । आ॒र्द्रदा॑नु॒रित्या॒र्द्र-दा॒नुः॒ । श॒भूंरिति॑ शं-भूः । म॒यो॒भूरिति॑ मयः-भूः । अ॒भीति॑ । मा॒ । वा॒हि॒ । स्वाहा᳚ । मा॒रु॒तः । अ॒सि॒ । म॒रुता᳚म् । ग॒णः । श॒भूंरिति॑ शं - भूः । म॒यो॒भूरिति॑ मयः - भूः । अ॒भीति॑ । मा॒ । वा॒हि॒ । स्वाहा᳚ । अ॒व॒स्युः । अ॒सि॒ । दुव॑स्वान् । श॒भूंरिति॑ शं - भूः । म॒यो॒भूरिति॑ मयः - भूः । अ॒भीति॑ । मा॒ ( ) । वा॒हि॒ । स्वाहा᳚ ॥  \newline


\textbf{Krama Paata} \newline

ओष॑धीभि॒रित्योष॑धि - भिः॒ ॥ सो॑ऽहम् । अ॒हम् ॅवाज᳚म् । वाजꣳ॑ सनेयम् । स॒ने॒य॒म॒ग्ने॒ । अ॒ग्न॒ इत्य॑ग्ने ॥ नक्तो॒षासा॒ सम॑नसा । सम॑नसा॒ विरू॑पे । सम॑न॒सेति॒ स - म॒न॒सा॒ । विरू॑पे धा॒पये॑ते । विरू॑पे॒ इति॒ वि - रू॒पे॒ । धा॒पये॑ते॒ शिशु᳚म् । धा॒पये॑ते॒ इति॑ धा॒पये॑ते । शिशु॒मेक᳚म् । एकꣳ॑ समी॒ची । स॒मी॒ची इति॑ समी॒ची ॥ द्यावा॒ क्षाम॑ । क्षामा॑ रु॒क्मः । रु॒क्मो अ॒न्तः । अ॒न्तर् वि । वि भा॑ति । भा॒ति॒ दे॒वाः । दे॒वा अ॒ग्निम् । अ॒ग्निम् धा॑रयन्न् । धा॒र॒य॒न् द्र॒वि॒णो॒दाः । द्र॒वि॒णो॒दा इति॑ द्रविणः - दाः ॥ स॒मु॒द्रो॑ऽसि । अ॒सि॒ नभ॑स्वान् । नभ॑स्वाना॒र्द्रदा॑नुः । आ॒र्द्रदा॑नुः श॒म्भूः । आ॒र्द्रदा॑नु॒रित्या॒र्द्र - दा॒नुः॒ । श॒म्भूर् म॑यो॒भूः । श॒म्भूरिति॑ शम् - भूः । म॒यो॒भूर॒भि । म॒यो॒भूरिति॑ मयः - भूः । अ॒भि मा᳚ । मा॒ वा॒हि॒ । वा॒हि॒ स्वाहा᳚ । स्वाहा॑ मारु॒तः । मा॒रु॒तो॑ऽसि । अ॒सि॒ म॒रुता᳚म् । म॒रुता᳚म् ग॒णः । ग॒णः श॒म्भूः । श॒म्भूर् म॑यो॒भूः । श॒म्भूरिति॑ शम् - भूः । म॒यो॒भूर॒भि । म॒यो॒भूरिति॑ मयः - भूः । अ॒भि मा᳚ । मा॒ वा॒हि॒ । वा॒हि॒ स्वाहा᳚ । स्वाहा॑ऽव॒स्युः । अ॒व॒स्युर॑सि । अ॒सि॒ दुव॑स्वान् । दुव॑स्वाञ्छ॒म्भूः । श॒म्भूर् म॑यो॒भूः । श॒म्भूरिति॑ शम् - भूः । म॒यो॒भूर॒भि । म॒यो॒भूरिति॑ मयः - भूः । अ॒भि मा᳚ ( ) । मा॒ वा॒हि॒ । वा॒हि॒ स्वाहा᳚ । स्वाहेति॒ स्वाहा᳚ । \newline

\textbf{Jatai Paata} \newline

1. ओष॑धीभि॒रित्योष॑धि - भिः॒ । \newline
2. सो॑ ऽह म॒हꣳ स सो॑ ऽहम् । \newline
3. अ॒हं ॅवाजं॒ ॅवाज॑ म॒ह म॒हं ॅवाज᳚म् । \newline
4. वाज(ग्म्॑) सनेयꣳ सनेयं॒ ॅवाजं॒ ॅवाज(ग्म्॑) सनेयम् । \newline
5. स॒ने॒य॒ म॒ग्ने॒ अ॒ग्ने॒ स॒ने॒य॒(ग्म्॒) स॒ने॒य॒ म॒ग्ने॒ । \newline
6. अ॒ग्न॒ इत्य॑ग्ने । \newline
7. नक्तो॒षासा॒ सम॑नसा॒ सम॑नसा॒ नक्तो॒षासा॒ नक्तो॒षासा॒ सम॑नसा । \newline
8. सम॑नसा॒ विरू॑पे॒ विरू॑पे॒ सम॑नसा॒ सम॑नसा॒ विरू॑पे । \newline
9. सम॑न॒सेति॒ स - म॒न॒सा॒ । \newline
10. विरू॑पे धा॒पये॑ते धा॒पये॑ते॒ विरू॑पे॒ विरू॑पे धा॒पये॑ते । \newline
11. विरू॑पे॒ इति॒ वि - रू॒पे॒ । \newline
12. धा॒पये॑ते॒ शिशु॒(ग्म्॒) शिशु॑म् धा॒पये॑ते धा॒पये॑ते॒ शिशु᳚म् । \newline
13. धा॒पये॑ते॒ इति॑ धा॒पये॑ते । \newline
14. शिशु॒ मेक॒ मेक॒(ग्म्॒) शिशु॒(ग्म्॒) शिशु॒ मेक᳚म् । \newline
15. एक(ग्म्॑) समी॒ची स॑मी॒ची एक॒ मेक(ग्म्॑) समी॒ची । \newline
16. स॒मी॒ची इति॑ समी॒ची । \newline
17. द्यावा॒ क्षाम॒ क्षाम॒ द्यावा॒ द्यावा॒ क्षाम॑ । \newline
18. क्षामा॑ रु॒क्मो रु॒क्मः क्षाम॒ क्षामा॑ रु॒क्मः । \newline
19. रु॒क्मो अ॒न्त र॒न्ता रु॒क्मो रु॒क्मो अ॒न्तः । \newline
20. अ॒न्तर् वि व्य॑न्त र॒न्तर् वि । \newline
21. वि भा॑ति भाति॒ वि वि भा॑ति । \newline
22. भा॒ति॒ दे॒वा दे॒वा भा॑ति भाति दे॒वाः । \newline
23. दे॒वा अ॒ग्नि म॒ग्निम् दे॒वा दे॒वा अ॒ग्निम् । \newline
24. अ॒ग्निम् धा॑रयन् धारयन् न॒ग्नि म॒ग्निम् धा॑रयन्न् । \newline
25. धा॒र॒य॒न् द्र॒वि॒णो॒दा द्र॑विणो॒दा धा॑रयन् धारयन् द्रविणो॒दाः । \newline
26. द्र॒वि॒णो॒दा इति॑ द्रविणः - दाः । \newline
27. स॒मु॒द्रो᳚ ऽस्यसि समु॒द्रः स॑मु॒द्रो॑ ऽसि । \newline
28. अ॒सि॒ नभ॑स्वा॒न् नभ॑स्वा नस्यसि॒ नभ॑स्वान् । \newline
29. नभ॑स्वा ना॒र्द्रदा॑नु रा॒र्द्रदा॑नु॒र् नभ॑स्वा॒न् नभ॑स्वा ना॒र्द्रदा॑नुः । \newline
30. आ॒र्द्रदा॑नुः शं॒भूः शं॒भू रा॒र्द्रदा॑नु रा॒र्द्रदा॑नुः शं॒भूः । \newline
31. आ॒र्द्रदा॑नु॒रित्या॒र्द्र - दा॒नुः॒ । \newline
32. शं॒भूर् म॑यो॒भूर् म॑यो॒भूः शं॒भूः शं॒भूर् म॑यो॒भूः । \newline
33. शं॒भूरिति॑ शं - भूः । \newline
34. म॒यो॒भू र॒भ्य॑भि म॑यो॒भूर् म॑यो॒भू र॒भि । \newline
35. म॒यो॒भूरिति॑ मयः - भूः । \newline
36. अ॒भि मा॑ मा॒ ऽभ्य॑भि मा᳚ । \newline
37. मा॒ वा॒हि॒ वा॒हि॒ मा॒ मा॒ वा॒हि॒ । \newline
38. वा॒हि॒ स्वाहा॒ स्वाहा॑ वाहि वाहि॒ स्वाहा᳚ । \newline
39. स्वाहा॑ मारु॒तो मा॑रु॒तः स्वाहा॒ स्वाहा॑ मारु॒तः । \newline
40. मा॒रु॒तो᳚ ऽस्यसि मारु॒तो मा॑रु॒तो॑ ऽसि । \newline
41. अ॒सि॒ म॒रुता᳚म् म॒रुता॑ मस्यसि म॒रुता᳚म् । \newline
42. म॒रुता᳚म् ग॒णो ग॒णो म॒रुता᳚म् म॒रुता᳚म् ग॒णः । \newline
43. ग॒णः शं॒भूः शं॒भूर् ग॒णो ग॒णः शं॒भूः । \newline
44. शं॒भूर् म॑यो॒भूर् म॑यो॒भूः शं॒भूः शं॒भूर् म॑यो॒भूः । \newline
45. शं॒भूरिति॑ शं - भूः । \newline
46. म॒यो॒भू र॒भ्य॑भि म॑यो॒भूर् म॑यो॒भू र॒भि । \newline
47. म॒यो॒भूरिति॑ मयः - भूः । \newline
48. अ॒भि मा॑ मा॒ ऽभ्य॑भि मा᳚ । \newline
49. मा॒ वा॒हि॒ वा॒हि॒ मा॒ मा॒ वा॒हि॒ । \newline
50. वा॒हि॒ स्वाहा॒ स्वाहा॑ वाहि वाहि॒ स्वाहा᳚ । \newline
51. स्वाहा॑ ऽव॒स्यु र॑व॒स्युः स्वाहा॒ स्वाहा॑ ऽव॒स्युः । \newline
52. अ॒व॒स्यु र॑स्य स्यव॒स्यु र॑व॒स्यु र॑सि । \newline
53. अ॒सि॒ दुव॑स्वा॒न् दुव॑स्वा नस्यसि॒ दुव॑स्वान् । \newline
54. दुव॑स्वा ञ्छं॒भूः शं॒भूर् दुव॑स्वा॒न् दुव॑स्वा ञ्छं॒भूः । \newline
55. शं॒भूर् म॑यो॒भूर् म॑यो॒भूः शं॒भूः शं॒भूर् म॑यो॒भूः । \newline
56. शं॒भूरिति॑ शं - भूः । \newline
57. म॒यो॒भू र॒भ्य॑भि म॑यो॒भूर् म॑यो॒भू र॒भि । \newline
58. म॒यो॒भूरिति॑ मयः - भूः । \newline
59. अ॒भि मा॑ मा॒ ऽभ्य॑भि मा᳚ । \newline
60. मा॒ वा॒हि॒ वा॒हि॒ मा॒ मा॒ वा॒हि॒ । \newline
61. वा॒हि॒ स्वाहा॒ स्वाहा॑ वाहि वाहि॒ स्वाहा᳚ । \newline
62. स्वाहेति॒ स्वाहा᳚ । \newline

\textbf{Ghana Paata } \newline

1. ओष॑धीभि॒रित्योष॑धि - भिः॒ । \newline
2. सो॑ ऽह म॒हꣳ स सो॑ ऽहं ॅवाजं॒ ॅवाज॑ म॒हꣳ स सो॑ ऽहं ॅवाज᳚म् । \newline
3. अ॒हं ॅवाजं॒ ॅवाज॑ म॒ह म॒हं ॅवाज(ग्म्॑) सनेयꣳ सनेयं॒ ॅवाज॑ म॒ह म॒हं ॅवाज(ग्म्॑) सनेयम् । \newline
4. वाज(ग्म्॑) सनेयꣳ सनेयं॒ ॅवाजं॒ ॅवाज(ग्म्॑) सनेय मग्ने अग्ने सनेयं॒ ॅवाजं॒ ॅवाज(ग्म्॑) सनेय मग्ने । \newline
5. स॒ने॒य॒ म॒ग्ने॒ अ॒ग्ने॒ स॒ने॒य॒(ग्म्॒) स॒ने॒य॒ म॒ग्ने॒ । \newline
6. अ॒ग्न॒ इत्य॑ग्ने । \newline
7. नक्तो॒षासा॒ सम॑नसा॒ सम॑नसा॒ नक्तो॒षासा॒ नक्तो॒षासा॒ सम॑नसा॒ विरू॑पे॒ विरू॑पे॒ सम॑नसा॒ नक्तो॒षासा॒ नक्तो॒षासा॒ सम॑नसा॒ विरू॑पे । \newline
8. सम॑नसा॒ विरू॑पे॒ विरू॑पे॒ सम॑नसा॒ सम॑नसा॒ विरू॑पे धा॒पये॑ते धा॒पये॑ते॒ विरू॑पे॒ सम॑नसा॒ सम॑नसा॒ विरू॑पे धा॒पये॑ते । \newline
9. सम॑न॒सेति॒ स - म॒न॒सा॒ । \newline
10. विरू॑पे धा॒पये॑ते धा॒पये॑ते॒ विरू॑पे॒ विरू॑पे धा॒पये॑ते॒ शिशु॒(ग्म्॒) शिशु॑म् धा॒पये॑ते॒ विरू॑पे॒ विरू॑पे धा॒पये॑ते॒ शिशु᳚म् । \newline
11. विरू॑पे॒ इति॒ वि - रू॒पे॒ । \newline
12. धा॒पये॑ते॒ शिशु॒(ग्म्॒) शिशु॑म् धा॒पये॑ते धा॒पये॑ते॒ शिशु॒ मेक॒ मेक॒(ग्म्॒) शिशु॑म् धा॒पये॑ते धा॒पये॑ते॒ शिशु॒ मेक᳚म् । \newline
13. धा॒पये॑ते॒ इति॑ धा॒पये॑ते । \newline
14. शिशु॒ मेक॒ मेक॒(ग्म्॒) शिशु॒(ग्म्॒) शिशु॒ मेक(ग्म्॑) समी॒ची स॑मी॒ची एक॒(ग्म्॒) शिशु॒(ग्म्॒) शिशु॒ मेक(ग्म्॑) समी॒ची । \newline
15. एक(ग्म्॑) समी॒ची स॑मी॒ची एक॒ मेक(ग्म्॑) समी॒ची । \newline
16. स॒मी॒ची इति॑ समी॒ची । \newline
17. द्यावा॒ क्षाम॒ क्षाम॒ द्यावा॒ द्यावा॒ क्षामा॑ रु॒क्मो रु॒क्मः क्षाम॒ द्यावा॒ द्यावा॒ क्षामा॑ रु॒क्मः । \newline
18. क्षामा॑ रु॒क्मो रु॒क्मः क्षाम॒ क्षामा॑ रु॒क्मो अ॒न्त र॒न्ता रु॒क्मः क्षाम॒ क्षामा॑ रु॒क्मो अ॒न्तः । \newline
19. रु॒क्मो अ॒न्ता र॒न्ता रु॒क्मो रु॒क्मो अ॒न्तर् वि व्य॑न्ता रु॒क्मो रु॒क्मो अ॒न्तर् वि । \newline
20. अ॒न्तर् वि व्य॑न्त र॒न्तर् वि भा॑ति भाति॒ व्य॑न्त र॒न्तर् वि भा॑ति । \newline
21. वि भा॑ति भाति॒ वि वि भा॑ति दे॒वा दे॒वा भा॑ति॒ वि वि भा॑ति दे॒वाः । \newline
22. भा॒ति॒ दे॒वा दे॒वा भा॑ति भाति दे॒वा अ॒ग्नि म॒ग्निम् दे॒वा भा॑ति भाति दे॒वा अ॒ग्निम् । \newline
23. दे॒वा अ॒ग्नि म॒ग्निम् दे॒वा दे॒वा अ॒ग्निम् धा॑रयन् धारयन् न॒ग्निम् दे॒वा दे॒वा अ॒ग्निम् धा॑रयन्न् । \newline
24. अ॒ग्निम् धा॑रयन् धारयन् न॒ग्नि म॒ग्निम् धा॑रयन् द्रविणो॒दा द्र॑विणो॒दा धा॑रयन् न॒ग्नि म॒ग्निम् धा॑रयन् द्रविणो॒दाः । \newline
25. धा॒र॒य॒न् द्र॒वि॒णो॒दा द्र॑विणो॒दा धा॑रयन् धारयन् द्रविणो॒दाः । \newline
26. द्र॒वि॒णो॒दा इति॑ द्रविणः - दाः । \newline
27. स॒मु॒द्रो᳚ ऽस्यसि समु॒द्रः स॑मु॒द्रो॑ ऽसि॒ नभ॑स्वा॒न् नभ॑स्वा नसि समु॒द्रः स॑मु॒द्रो॑ ऽसि॒ नभ॑स्वान् । \newline
28. अ॒सि॒ नभ॑स्वा॒न् नभ॑स्वा नस्यसि॒ नभ॑स्वा ना॒र्द्रदा॑नु रा॒र्द्रदा॑नु॒र् नभ॑स्वा नस्यसि॒ नभ॑स्वा ना॒र्द्रदा॑नुः । \newline
29. नभ॑स्वा ना॒र्द्रदा॑नु रा॒र्द्रदा॑नु॒र् नभ॑स्वा॒न् नभ॑स्वा ना॒र्द्रदा॑नुः शं॒भूः शं॒भू रा॒र्द्रदा॑नु॒र् नभ॑स्वा॒न् नभ॑स्वा ना॒र्द्रदा॑नुः शं॒भूः । \newline
30. आ॒र्द्रदा॑नुः शं॒भूः शं॒भू रा॒र्द्रदा॑नु रा॒र्द्रदा॑नुः शं॒भूर् म॑यो॒भूर् म॑यो॒भूः शं॒भू रा॒र्द्रदा॑नु रा॒र्द्रदा॑नुः शं॒भूर् म॑यो॒भूः । \newline
31. आ॒र्द्रदा॑नु॒रित्या॒र्द्र - दा॒नुः॒ । \newline
32. शं॒भूर् म॑यो॒भूर् म॑यो॒भूः शं॒भूः शं॒भूर् म॑यो॒भू र॒भ्य॑भि म॑यो॒भूः शं॒भूः शं॒भूर् म॑यो॒भूर॒भि । \newline
33. शं॒भूरिति॑ शं - भूः । \newline
34. म॒यो॒भू र॒भ्य॑भि म॑यो॒भूर् म॑यो॒भू र॒भि मा॑ मा॒ ऽभि म॑यो॒भूर् म॑यो॒भू र॒भि मा᳚ । \newline
35. म॒यो॒भूरिति॑ मयः - भूः । \newline
36. अ॒भि मा॑ मा॒ ऽभ्य॑भि मा॑ वाहि वाहि मा॒ ऽभ्य॑भि मा॑ वाहि । \newline
37. मा॒ वा॒हि॒ वा॒हि॒ मा॒ मा॒ वा॒हि॒ स्वाहा॒ स्वाहा॑ वाहि मा मा वाहि॒ स्वाहा᳚ । \newline
38. वा॒हि॒ स्वाहा॒ स्वाहा॑ वाहि वाहि॒ स्वाहा॑ मारु॒तो मा॑रु॒तः स्वाहा॑ वाहि वाहि॒ स्वाहा॑ मारु॒तः । \newline
39. स्वाहा॑ मारु॒तो मा॑रु॒तः स्वाहा॒ स्वाहा॑ मारु॒तो᳚ ऽस्यसि मारु॒तः स्वाहा॒ स्वाहा॑ मारु॒तो॑ ऽसि । \newline
40. मा॒रु॒तो᳚ ऽस्यसि मारु॒तो मा॑रु॒तो॑ ऽसि म॒रुता᳚म् म॒रुता॑ मसि मारु॒तो मा॑रु॒तो॑ ऽसि म॒रुता᳚म् । \newline
41. अ॒सि॒ म॒रुता᳚म् म॒रुता॑ मस्यसि म॒रुता᳚म् ग॒णो ग॒णो म॒रुता॑ मस्यसि म॒रुता᳚म् ग॒णः । \newline
42. म॒रुता᳚म् ग॒णो ग॒णो म॒रुता᳚म् म॒रुता᳚म् ग॒णः शं॒भूः शं॒भूर् ग॒णो म॒रुता᳚म् म॒रुता᳚म् ग॒णः शं॒भूः । \newline
43. ग॒णः शं॒भूः शं॒भूर् ग॒णो ग॒णः शं॒भूर् म॑यो॒भूर् म॑यो॒भूः शं॒भूर् ग॒णो ग॒णः शं॒भूर् म॑यो॒भूः । \newline
44. शं॒भूर् म॑यो॒भूर् म॑यो॒भूः शं॒भूः शं॒भूर् म॑यो॒भू र॒भ्य॑भि म॑यो॒भूः शं॒भूः शं॒भूर् म॑यो॒भूर॒भि । \newline
45. शं॒भूरिति॑ शं - भूः । \newline
46. म॒यो॒भू र॒भ्य॑भि म॑यो॒भूर् म॑यो॒भू र॒भि मा॑ मा॒ ऽभि म॑यो॒भूर् म॑यो॒भू र॒भि मा᳚ । \newline
47. म॒यो॒भूरिति॑ मयः - भूः । \newline
48. अ॒भि मा॑ मा॒ ऽभ्य॑भि मा॑ वाहि वाहि मा॒ ऽभ्य॑भि मा॑ वाहि । \newline
49. मा॒ वा॒हि॒ वा॒हि॒ मा॒ मा॒ वा॒हि॒ स्वाहा॒ स्वाहा॑ वाहि मा मा वाहि॒ स्वाहा᳚ । \newline
50. वा॒हि॒ स्वाहा॒ स्वाहा॑ वाहि वाहि॒ स्वाहा॑ ऽव॒स्यु र॑व॒स्युः स्वाहा॑ वाहि वाहि॒ स्वाहा॑ ऽव॒स्युः । \newline
51. स्वाहा॑ ऽव॒स्यु र॑व॒स्युः स्वाहा॒ स्वाहा॑ ऽव॒स्यु र॑स्य स्यव॒स्युः स्वाहा॒ स्वाहा॑ ऽव॒स्युर॑सि । \newline
52. अ॒व॒स्यु र॑स्य स्यव॒स्यु र॑व॒स्यु र॑सि॒ दुव॑स्वा॒न् दुव॑स्वा नस्यव॒स्यु र॑व॒स्यु र॑सि॒ दुव॑स्वान् । \newline
53. अ॒सि॒ दुव॑स्वा॒न् दुव॑स्वा नस्यसि॒ दुव॑स्वा ञ्छं॒भूः शं॒भूर् दुव॑स्वा नस्यसि॒ दुव॑स्वा ञ्छं॒भूः । \newline
54. दुव॑स्वा ञ्छं॒भूः शं॒भूर् दुव॑स्वा॒न् दुव॑स्वा ञ्छं॒भूर् म॑यो॒भूर् म॑यो॒भूः शं॒भूर् दुव॑स्वा॒न् दुव॑स्वा ञ्छं॒भूर् म॑यो॒भूः । \newline
55. शं॒भूर् म॑यो॒भूर् म॑यो॒भूः शं॒भूः शं॒भूर् म॑यो॒भू र॒भ्य॑भि म॑यो॒भूः शं॒भूः शं॒भूर् म॑यो॒भूर॒भि । \newline
56. शं॒भूरिति॑ शं - भूः । \newline
57. म॒यो॒भू र॒भ्य॑भि म॑यो॒भूर् म॑यो॒भू र॒भि मा॑ मा॒ ऽभि म॑यो॒भूर् म॑यो॒भू र॒भि मा᳚ । \newline
58. म॒यो॒भूरिति॑ मयः - भूः । \newline
59. अ॒भि मा॑ मा॒ ऽभ्य॑भि मा॑ वाहि वाहि मा॒ ऽभ्य॑भि मा॑ वाहि । \newline
60. मा॒ वा॒हि॒ वा॒हि॒ मा॒ मा॒ वा॒हि॒ स्वाहा॒ स्वाहा॑ वाहि मा मा वाहि॒ स्वाहा᳚ । \newline
61. वा॒हि॒ स्वाहा॒ स्वाहा॑ वाहि वाहि॒ स्वाहा᳚ । \newline
62. स्वाहेति॒ स्वाहा᳚ । \newline
\pagebreak
\markright{ TS 4.7.13.1  \hfill https://www.vedavms.in \hfill}

\section{ TS 4.7.13.1 }

\textbf{TS 4.7.13.1 } \newline
\textbf{Samhita Paata} \newline

अ॒ग्निं ॅयु॑नज्मि॒ शव॑सा घृ॒तेन॑ दि॒व्यꣳ सु॑प॒र्णं ॅवय॑सा बृ॒हन्तं᳚ । तेन॑ व॒यं प॑तेम ब्र॒द्ध्नस्य॑ वि॒ष्टपꣳ॒॒ सुवो॒ रुहा॑णा॒ अधि॒ नाक॑ उत्त॒मे ॥ इ॒मौ ते॑ प॒क्षाव॒जरौ॑ पत॒त्रिणो॒ याभ्याꣳ॒॒ रक्षाꣳ॑-स्यप॒हꣳ-स्य॑ग्ने । ताभ्यां᳚ पतेम सु॒कृता॑मु लो॒कं ॅयत्रर्.ष॑यः प्रथम॒जा ये पु॑रा॒णाः ॥ चिद॑सि समु॒द्रयो॑नि॒रिन्दु॒र्दक्षः॑ श्ये॒न ऋ॒तावा᳚ । हिर॑ण्यपक्षः शकु॒नो भु॑र॒ण्यु-र्म॒हान्थ् स॒धस्थे᳚ ध्रु॒व - [  ] \newline

\textbf{Pada Paata} \newline

अ॒ग्निम् । यु॒न॒ज्मि॒ । शव॑सा । घृ॒तेन॑ । दि॒व्यम् । सु॒प॒र्णमिति॑ सु - प॒र्णम् । वय॑सा । बृ॒हन्त᳚म् ॥ तेन॑ । व॒यम् । प॒ते॒म॒ । ब्र॒द्ध्नस्य॑ । वि॒ष्टप᳚म् । सुवः॑ । रुहा॑णाः । अधीति॑ । नाके᳚ । उ॒त्त॒म इत्यु॑त् - त॒मे ॥ इ॒मौ । ते॒ । प॒क्षौ । अ॒जरौ᳚ । प॒त॒त्रिणः॑ । याभ्या᳚म् । रक्षाꣳ॑सि । अ॒प॒हꣳसीत्य॑प - हꣳसि॑ । अ॒ग्ने॒ ॥ ताभ्या᳚म् । प॒ते॒म॒ । सु॒कृता॒मिति॑ सु - कृता᳚म् । उ॒ । लो॒कम् । यत्र॑ । ऋष॑यः । प्र॒थ॒म॒जा इति॑ प्रथम - जाः । ये । पु॒रा॒णाः ॥ चित् । अ॒सि॒ । स॒मु॒द्रयो॑नि॒रिति॑ समु॒द्र - यो॒निः॒ । इन्दुः॑ । दक्षः॑ । श्ये॒नः । ऋ॒तावेत्यृ॒त - वा॒ ॥ हिर॑ण्यपक्ष॒ इति॒ हिर॑ण्य - प॒क्षः॒ । श॒कु॒नः । भु॒र॒ण्युः । म॒हान् । स॒धस्थ॒ इति॑ स॒ध - स्थे॒ । ध्रु॒वः ।  \newline


\textbf{Krama Paata} \newline

अ॒ग्निम् ॅयु॑नज्मि । यु॒न॒ज्मि॒ शव॑सा । शव॑सा घृ॒तेन॑ । घृ॒तेन॑ दि॒व्यम् । दि॒व्यꣳ सु॑प॒र्णम् । सु॒प॒र्णम् ॅवय॑सा । सु॒प॒र्णमिति॑ सु - प॒र्णम् । वय॑सा बृ॒हन्त᳚म् । बृ॒हन्त॒मिति॑ बृ॒हन्त᳚म् ॥ तेन॑ व॒यम् । व॒यम् प॑तेम । प॒ते॒म॒ ब्र॒द्ध्नस्य॑ । ब्र॒द्ध्नस्य॑ वि॒ष्टप᳚म् । वि॒ष्टपꣳ॒॒ सुवः॑ । सुवो॒ रुहा॑णाः । रुहा॑णा॒ अधि॑ । अधि॒ नाके᳚ । नाक॑ उत्त॒मे । उ॒त्त॒म इत्यु॑त् - त॒मे ॥ इ॒मौ ते᳚ । ते॒ प॒क्षौ । प॒क्षाव॒जरौ᳚ । अ॒जरौ॑ पत॒त्रिणः॑ । प॒त॒त्रिणो॒ याभ्या᳚म् । याभ्याꣳ॒॒ रक्षाꣳ॑सि । रक्षाꣳ॑स्यप॒हꣳसि॑ । अ॒प॒हꣳस्य॑ग्ने । अ॒प॒हꣳसीत्य॑प - हꣳसि॑ । अ॒ग्न॒ इत्य॑ग्ने ॥ ताभ्या᳚म् पतेम । प॒ते॒म॒ सु॒कृता᳚म् । सु॒कृता॑मु । सु॒कृता॒मिति॑ सु - कृता᳚म् । उ॒ लो॒कम् । लो॒कम् ॅयत्र॑ । यत्रर्.ष॑यः । ऋष॑यः प्रथम॒जाः । प्र॒थ॒म॒जा ये । प्र॒थ॒म॒जा इति॑ प्रथम - जाः । ये पु॑रा॒णाः । पु॒रा॒णा इति॑ पुरा॒णाः ॥ चिद॑सि । अ॒सि॒ स॒मु॒द्रयो॑निः । स॒मु॒द्रयो॑नि॒रिन्दुः॑ । स॒मु॒द्रयो॑नि॒रिति॑ समु॒द्र - यो॒निः॒ । इन्दु॒र् दक्षः॑ । दक्षः॑ श्ये॒नः । श्ये॒न ऋ॒तावा᳚ । ऋ॒तावेत्यृ॒त - वा॒ ॥ हिर॑ण्यपक्षः शकु॒नः । हिर॑ण्यपक्ष॒ इति॒ हिर॑ण्य - प॒क्षः॒ । श॒कु॒नो भु॑र॒ण्युः । भु॒र॒ण्युर् म॒हान् । म॒हान्थ् स॒धस्थे᳚ । स॒धस्थे᳚ ध्रु॒वः । स॒धस्थ॒ इति॑ स॒ध - स्थे॒ । ध्रु॒व आ \newline

\textbf{Jatai Paata} \newline

1. अ॒ग्निं ॅयु॑नज्मि युनज् म्य॒ग्नि म॒ग्निं ॅयु॑नज्मि । \newline
2. यु॒न॒ज्मि॒ शव॑सा॒ शव॑सा युनज्मि युनज्मि॒ शव॑सा । \newline
3. शव॑सा घृ॒तेन॑ घृ॒तेन॒ शव॑सा॒ शव॑सा घृ॒तेन॑ । \newline
4. घृ॒तेन॑ दि॒व्यम् दि॒व्यम् घृ॒तेन॑ घृ॒तेन॑ दि॒व्यम् । \newline
5. दि॒व्यꣳ सु॑प॒र्णꣳ सु॑प॒र्णम् दि॒व्यम् दि॒व्यꣳ सु॑प॒र्णम् । \newline
6. सु॒प॒र्णं ॅवय॑सा॒ वय॑सा सुप॒र्णꣳ सु॑प॒र्णं ॅवय॑सा । \newline
7. सु॒प॒र्णमिति॑ सु - प॒र्णम् । \newline
8. वय॑सा बृ॒हन्त॑म् बृ॒हन्तं॒ ॅवय॑सा॒ वय॑सा बृ॒हन्त᳚म् । \newline
9. बृ॒हन्त॒मिति॑ बृ॒हन्त᳚म् । \newline
10. तेन॑ व॒यं ॅव॒यम् तेन॒ तेन॑ व॒यम् । \newline
11. व॒यम् प॑तेम पतेम व॒यं ॅव॒यम् प॑तेम । \newline
12. प॒ते॒म॒ ब्र॒द्ध्नस्य॑ ब्र॒द्ध्नस्य॑ पतेम पतेम ब्र॒द्ध्नस्य॑ । \newline
13. ब्र॒द्ध्नस्य॑ वि॒ष्टपं॑ ॅवि॒ष्टप॑म् ब्र॒द्ध्नस्य॑ ब्र॒द्ध्नस्य॑ वि॒ष्टप᳚म् । \newline
14. वि॒ष्टप॒(ग्म्॒) सुवः॒ सुव॑र् वि॒ष्टपं॑ ॅवि॒ष्टप॒(ग्म्॒) सुवः॑ । \newline
15. सुवो॒ रुहा॑णा॒ रुहा॑णाः॒ सुवः॒ सुवो॒ रुहा॑णाः । \newline
16. रुहा॑णा॒ अध्यधि॒ रुहा॑णा॒ रुहा॑णा॒ अधि॑ । \newline
17. अधि॒ नाके॒ नाके॒ अध्यधि॒ नाके᳚ । \newline
18. नाक॑ उत्त॒म उ॑त्त॒मे नाके॒ नाक॑ उत्त॒मे । \newline
19. उ॒त्त॒म इत्यु॑त् - त॒मे । \newline
20. इ॒मौ ते॑ त इ॒मा वि॒मौ ते᳚ । \newline
21. ते॒ प॒क्षौ प॒क्षौ ते॑ ते प॒क्षौ । \newline
22. प॒क्षा व॒जरा॑ व॒जरौ॑ प॒क्षौ प॒क्षा व॒जरौ᳚ । \newline
23. अ॒जरौ॑ पत॒त्रिणः॑ पत॒त्रिणो॑ अ॒जरा॑ व॒जरौ॑ पत॒त्रिणः॑ । \newline
24. प॒त॒त्रिणो॒ याभ्यां॒ ॅयाभ्या᳚म् पत॒त्रिणः॑ पत॒त्रिणो॒ याभ्या᳚म् । \newline
25. याभ्या॒(ग्म्॒) रक्षा(ग्म्॑)सि॒ रक्षा(ग्म्॑)सि॒ याभ्यां॒ ॅयाभ्या॒(ग्म्॒) रक्षा(ग्म्॑)सि । \newline
26. रक्षा(ग्ग्॑) स्यप॒हꣳ स्य॑प॒हꣳसि॒ रक्षा(ग्म्॑)सि॒ रक्षा(ग्ग्॑) स्यप॒हꣳसि॑ । \newline
27. अ॒प॒हꣳ स्य॑ग्ने अग्ने अप॒हꣳ स्य॑प॒हꣳ स्य॑ग्ने । \newline
28. अ॒प॒हꣳसीत्य॑प - हꣳसि॑ । \newline
29. अ॒ग्न॒ इत्य॑ग्ने । \newline
30. ताभ्या᳚म् पतेम पतेम॒ ताभ्या॒म् ताभ्या᳚म् पतेम । \newline
31. प॒ते॒म॒ सु॒कृता(ग्म्॑) सु॒कृता᳚म् पतेम पतेम सु॒कृता᳚म् । \newline
32. सु॒कृता॑ मु वु सु॒कृता(ग्म्॑) सु॒कृता॑ मु । \newline
33. सु॒कृता॒मिति॑ सु - कृता᳚म् । \newline
34. उ॒ लो॒कम् ॅलो॒क मु॑ वु लो॒कम् । \newline
35. लो॒कं ॅयत्र॒ यत्र॑ लो॒कम् ॅलो॒कं ॅयत्र॑ । \newline
36. यत्र र्.ष॑य॒ ऋष॑यो॒ यत्र॒ यत्र र्.ष॑यः । \newline
37. ऋष॑यः प्रथम॒जाः प्र॑थम॒जा ऋष॑य॒ ऋष॑यः प्रथम॒जाः । \newline
38. प्र॒थ॒म॒जा ये ये प्र॑थम॒जाः प्र॑थम॒जा ये । \newline
39. प्र॒थ॒म॒जा इति॑ प्रथम - जाः । \newline
40. ये पु॑रा॒णाः पु॑रा॒णा ये ये पु॑रा॒णाः । \newline
41. पु॒रा॒णा इति॑ पुरा॒णाः । \newline
42. चिद॑स्यसि॒ चिच् चिद॑सि । \newline
43. अ॒सि॒ स॒मु॒द्रयो॑निः समु॒द्रयो॑नि रस्यसि समु॒द्रयो॑निः । \newline
44. स॒मु॒द्रयो॑नि॒ रिन्दु॒ रिन्दुः॑ समु॒द्रयो॑निः समु॒द्रयो॑नि॒ रिन्दुः॑ । \newline
45. स॒मु॒द्रयो॑नि॒रिति॑समु॒द्र - यो॒निः॒ । \newline
46. इन्दु॒र् दक्षो॒ दक्ष॒ इन्दु॒ रिन्दु॒र् दक्षः॑ । \newline
47. दक्षः॑ श्ये॒नः श्ये॒नो दक्षो॒ दक्षः॑ श्ये॒नः । \newline
48. श्ये॒न ऋ॒ताव॒ र्‌तावा᳚ श्ये॒नः श्ये॒न ऋ॒तावा᳚ । \newline
49. ऋ॒तावेत्यृ॒त - वा॒ । \newline
50. हिर॑ण्यपक्षः शकु॒नः श॑कु॒नो हिर॑ण्यपक्षो॒ हिर॑ण्यपक्षः शकु॒नः । \newline
51. हिर॑ण्यपक्ष॒ इति॒ हिर॑ण्य - प॒क्षः॒ । \newline
52. श॒कु॒नो भु॑र॒ण्युर् भु॑र॒ण्युः श॑कु॒नः श॑कु॒नो भु॑र॒ण्युः । \newline
53. भु॒र॒ण्युर् म॒हान् म॒हान् भु॑र॒ण्युर् भु॑र॒ण्युर् म॒हान् । \newline
54. म॒हान् थ्स॒धस्थे॑ स॒धस्थे॑ म॒हान् म॒हान् थ्स॒धस्थे᳚ । \newline
55. स॒धस्थे᳚ ध्रु॒वो ध्रु॒वः स॒धस्थे॑ स॒धस्थे᳚ ध्रु॒वः । \newline
56. स॒धस्थ॒ इति॑ स॒ध - स्थे॒ । \newline
57. ध्रु॒व आ ध्रु॒वो ध्रु॒व आ । \newline

\textbf{Ghana Paata } \newline

1. अ॒ग्निं ॅयु॑नज्मि युनज् म्य॒ग्नि म॒ग्निं ॅयु॑नज्मि॒ शव॑सा॒ शव॑सा युनज्‌म्य॒ग्नि म॒ग्निं ॅयु॑नज्मि॒ शव॑सा । \newline
2. यु॒न॒ज्मि॒ शव॑सा॒ शव॑सा युनज्मि युनज्मि॒ शव॑सा घृ॒तेन॑ घृ॒तेन॒ शव॑सा युनज्मि युनज्मि॒ शव॑सा घृ॒तेन॑ । \newline
3. शव॑सा घृ॒तेन॑ घृ॒तेन॒ शव॑सा॒ शव॑सा घृ॒तेन॑ दि॒व्यम् दि॒व्यम् घृ॒तेन॒ शव॑सा॒ शव॑सा घृ॒तेन॑ दि॒व्यम् । \newline
4. घृ॒तेन॑ दि॒व्यम् दि॒व्यम् घृ॒तेन॑ घृ॒तेन॑ दि॒व्यꣳ सु॑प॒र्णꣳ सु॑प॒र्णम् दि॒व्यम् घृ॒तेन॑ घृ॒तेन॑ दि॒व्यꣳ सु॑प॒र्णम् । \newline
5. दि॒व्यꣳ सु॑प॒र्णꣳ सु॑प॒र्णम् दि॒व्यम् दि॒व्यꣳ सु॑प॒र्णं ॅवय॑सा॒ वय॑सा सुप॒र्णम् दि॒व्यम् दि॒व्यꣳ सु॑प॒र्णं ॅवय॑सा । \newline
6. सु॒प॒र्णं ॅवय॑सा॒ वय॑सा सुप॒र्णꣳ सु॑प॒र्णं ॅवय॑सा बृ॒हन्त॑म् बृ॒हन्तं॒ ॅवय॑सा सुप॒र्णꣳ सु॑प॒र्णं ॅवय॑सा बृ॒हन्त᳚म् । \newline
7. सु॒प॒र्णमिति॑ सु - प॒र्णम् । \newline
8. वय॑सा बृ॒हन्त॑म् बृ॒हन्तं॒ ॅवय॑सा॒ वय॑सा बृ॒हन्त᳚म् । \newline
9. बृ॒हन्त॒मिति॑ बृ॒हन्त᳚म् । \newline
10. तेन॑ व॒यं ॅव॒यम् तेन॒ तेन॑ व॒यम् प॑तेम पतेम व॒यम् तेन॒ तेन॑ व॒यम् प॑तेम । \newline
11. व॒यम् प॑तेम पतेम व॒यं ॅव॒यम् प॑तेम ब्र॒द्ध्नस्य॑ ब्र॒द्ध्नस्य॑ पतेम व॒यं ॅव॒यम् प॑तेम ब्र॒द्ध्नस्य॑ । \newline
12. प॒ते॒म॒ ब्र॒द्ध्नस्य॑ ब्र॒द्ध्नस्य॑ पतेम पतेम ब्र॒द्ध्नस्य॑ वि॒ष्टपं॑ ॅवि॒ष्टप॑म् ब्र॒द्ध्नस्य॑ पतेम पतेम ब्र॒द्ध्नस्य॑ वि॒ष्टप᳚म् । \newline
13. ब्र॒द्ध्नस्य॑ वि॒ष्टपं॑ ॅवि॒ष्टप॑म् ब्र॒द्ध्नस्य॑ ब्र॒द्ध्नस्य॑ वि॒ष्टप॒(ग्म्॒) सुवः॒ सुव॑र् वि॒ष्टप॑म् ब्र॒द्ध्नस्य॑ ब्र॒द्ध्नस्य॑ वि॒ष्टप॒(ग्म्॒) सुवः॑ । \newline
14. वि॒ष्टप॒(ग्म्॒) सुवः॒ सुव॑र् वि॒ष्टपं॑ ॅवि॒ष्टप॒(ग्म्॒) सुवो॒ रुहा॑णा॒ रुहा॑णाः॒ सुव॑र् वि॒ष्टपं॑ ॅवि॒ष्टप॒(ग्म्॒) सुवो॒ रुहा॑णाः । \newline
15. सुवो॒ रुहा॑णा॒ रुहा॑णाः॒ सुवः॒ सुवो॒ रुहा॑णा॒ अध्यधि॒ रुहा॑णाः॒ सुवः॒ सुवो॒ रुहा॑णा॒ अधि॑ । \newline
16. रुहा॑णा॒ अध्यधि॒ रुहा॑णा॒ रुहा॑णा॒ अधि॒ नाके॒ नाके॒ अधि॒ रुहा॑णा॒ रुहा॑णा॒ अधि॒ नाके᳚ । \newline
17. अधि॒ नाके॒ नाके॒ अध्यधि॒ नाक॑ उत्त॒म उ॑त्त॒मे नाके॒ अध्यधि॒ नाक॑ उत्त॒मे । \newline
18. नाक॑ उत्त॒म उ॑त्त॒मे नाके॒ नाक॑ उत्त॒मे । \newline
19. उ॒त्त॒म इत्यु॑त् - त॒मे । \newline
20. इ॒मौ ते॑ त इ॒मा वि॒मौ ते॑ प॒क्षौ प॒क्षौ त॑ इ॒मा वि॒मौ ते॑ प॒क्षौ । \newline
21. ते॒ प॒क्षौ प॒क्षौ ते॑ ते प॒क्षा व॒जरा॑ व॒जरौ॑ प॒क्षौ ते॑ ते प॒क्षा व॒जरौ᳚ । \newline
22. प॒क्षा व॒जरा॑ व॒जरौ॑ प॒क्षौ प॒क्षा व॒जरौ॑ पत॒त्रिणः॑ पत॒त्रिणो॑ अ॒जरौ॑ प॒क्षौ प॒क्षा व॒जरौ॑ पत॒त्रिणः॑ । \newline
23. अ॒जरौ॑ पत॒त्रिणः॑ पत॒त्रिणो॑ अ॒जरा॑ व॒जरौ॑ पत॒त्रिणो॒ याभ्यां॒ ॅयाभ्या᳚म् पत॒त्रिणो॑ अ॒जरा॑ व॒जरौ॑ पत॒त्रिणो॒ याभ्या᳚म् । \newline
24. प॒त॒त्रिणो॒ याभ्यां॒ ॅयाभ्या᳚म् पत॒त्रिणः॑ पत॒त्रिणो॒ याभ्या॒(ग्म्॒) रक्षा(ग्म्॑)सि॒ रक्षा(ग्म्॑)सि॒ याभ्या᳚म् पत॒त्रिणः॑ पत॒त्रिणो॒ याभ्या॒(ग्म्॒) रक्षा(ग्म्॑)सि । \newline
25. याभ्या॒(ग्म्॒) रक्षा(ग्म्॑)सि॒ रक्षा(ग्म्॑)सि॒ याभ्यां॒ ॅयाभ्या॒(ग्म्॒) रक्षा(ग्ग्॑) स्यप॒हꣳ स्य॑प॒हꣳसि॒ रक्षा(ग्म्॑)सि॒ याभ्यां॒ ॅयाभ्या॒(ग्म्॒) रक्षा(ग्ग्॑) स्यप॒हꣳसि॑ । \newline
26. रक्षा(ग्ग्॑) स्यप॒हꣳ स्य॑प॒हꣳसि॒ रक्षा(ग्म्॑)सि॒ रक्षा(ग्ग्॑) स्यप॒हꣳ स्य॑ग्ने अग्ने अप॒हꣳसि॒ रक्षा(ग्म्॑)सि॒ रक्षा(ग्ग्॑) स्यप॒हꣳ स्य॑ग्ने । \newline
27. अ॒प॒हꣳ स्य॑ग्ने अग्ने अप॒हꣳ स्य॑प॒हꣳ स्य॑ग्ने । \newline
28. अ॒प॒हꣳसीत्य॑प - हꣳसि॑ । \newline
29. अ॒ग्न॒ इत्य॑ग्ने । \newline
30. ताभ्या᳚म् पतेम पतेम॒ ताभ्या॒म् ताभ्या᳚म् पतेम सु॒कृता(ग्म्॑) सु॒कृता᳚म् पतेम॒ ताभ्या॒म् ताभ्या᳚म् पतेम सु॒कृता᳚म् । \newline
31. प॒ते॒म॒ सु॒कृता(ग्म्॑) सु॒कृता᳚म् पतेम पतेम सु॒कृता॑ मु वु सु॒कृता᳚म् पतेम पतेम सु॒कृता॑ मु । \newline
32. सु॒कृता॑ मु वु सु॒कृता(ग्म्॑) सु॒कृता॑ मु लो॒कम् ॅलो॒क मु॑ सु॒कृता(ग्म्॑) सु॒कृता॑ मु लो॒कम् । \newline
33. सु॒कृता॒मिति॑ सु - कृता᳚म् । \newline
34. उ॒ लो॒कम् ॅलो॒क मु॑ वु लो॒कं ॅयत्र॒ यत्र॑ लो॒क मु॑ वु लो॒कं ॅयत्र॑ । \newline
35. लो॒कं ॅयत्र॒ यत्र॑ लो॒कम् ॅलो॒कं ॅयत्र र्.ष॑य॒ ऋष॑यो॒ यत्र॑ लो॒कम् ॅलो॒कं ॅयत्र र्.ष॑यः । \newline
36. यत्र र्.ष॑य॒ ऋष॑यो॒ यत्र॒ यत्र र्.ष॑यः प्रथम॒जाः प्र॑थम॒जा ऋष॑यो॒ यत्र॒ यत्र र्.ष॑यः प्रथम॒जाः । \newline
37. ऋष॑यः प्रथम॒जाः प्र॑थम॒जा ऋष॑य॒ ऋष॑यः प्रथम॒जा ये ये प्र॑थम॒जा ऋष॑य॒ ऋष॑यः प्रथम॒जा ये । \newline
38. प्र॒थ॒म॒जा ये ये प्र॑थम॒जाः प्र॑थम॒जा ये पु॑रा॒णाः पु॑रा॒णा ये प्र॑थम॒जाः प्र॑थम॒जा ये पु॑रा॒णाः । \newline
39. प्र॒थ॒म॒जा इति॑ प्रथम - जाः । \newline
40. ये पु॑रा॒णाः पु॑रा॒णा ये ये पु॑रा॒णाः । \newline
41. पु॒रा॒णा इति॑ पुरा॒णाः । \newline
42. चिद॑स्यसि॒ चिच् चिद॑सि समु॒द्रयो॑निः समु॒द्रयो॑नि रसि॒ चिच् चिद॑सि समु॒द्रयो॑निः । \newline
43. अ॒सि॒ स॒मु॒द्रयो॑निः समु॒द्रयो॑नि रस्यसि समु॒द्रयो॑नि॒ रिन्दु॒ रिन्दुः॑ समु॒द्रयो॑नि रस्यसि समु॒द्रयो॑नि॒ रिन्दुः॑ । \newline
44. स॒मु॒द्रयो॑नि॒ रिन्दु॒ रिन्दुः॑ समु॒द्रयो॑निः समु॒द्रयो॑नि॒ रिन्दु॒र् दक्षो॒ दक्ष॒ इन्दुः॑ समु॒द्रयो॑निः समु॒द्रयो॑नि॒ रिन्दु॒र् दक्षः॑ । \newline
45. स॒मु॒द्रयो॑नि॒रिति॑समु॒द्र - यो॒निः॒ । \newline
46. इन्दु॒र् दक्षो॒ दक्ष॒ इन्दु॒ रिन्दु॒र् दक्षः॑ श्ये॒नः श्ये॒नो दक्ष॒ इन्दु॒ रिन्दु॒र् दक्षः॑ श्ये॒नः । \newline
47. दक्षः॑ श्ये॒नः श्ये॒नो दक्षो॒ दक्षः॑ श्ये॒न ऋ॒ताव॒ र्‌तावा᳚ श्ये॒नो दक्षो॒ दक्षः॑ श्ये॒न ऋ॒तावा᳚ । \newline
48. श्ये॒न ऋ॒ताव॒ र्‌तावा᳚ श्ये॒नः श्ये॒न ऋ॒तावा᳚ । \newline
49. ऋ॒तावेत्यृ॒त - वा॒ । \newline
50. हिर॑ण्यपक्षः शकु॒नः श॑कु॒नो हिर॑ण्यपक्षो॒ हिर॑ण्यपक्षः शकु॒नो भु॑र॒ण्युर् भु॑र॒ण्युः श॑कु॒नो हिर॑ण्यपक्षो॒ हिर॑ण्यपक्षः शकु॒नो भु॑र॒ण्युः । \newline
51. हिर॑ण्यपक्ष॒ इति॒ हिर॑ण्य - प॒क्षः॒ । \newline
52. श॒कु॒नो भु॑र॒ण्युर् भु॑र॒ण्युः श॑कु॒नः श॑कु॒नो भु॑र॒ण्युर् म॒हान् म॒हान् भु॑र॒ण्युः श॑कु॒नः श॑कु॒नो भु॑र॒ण्युर् म॒हान् । \newline
53. भु॒र॒ण्युर् म॒हान् म॒हान् भु॑र॒ण्युर् भु॑र॒ण्युर् म॒हान् थ्स॒धस्थे॑ स॒धस्थे॑ म॒हान् भु॑र॒ण्युर् भु॑र॒ण्युर् म॒हान् थ्स॒धस्थे᳚ । \newline
54. म॒हान् थ्स॒धस्थे॑ स॒धस्थे॑ म॒हान् म॒हान् थ्स॒धस्थे᳚ ध्रु॒वो ध्रु॒वः स॒धस्थे॑ म॒हान् म॒हान् थ्स॒धस्थे᳚ ध्रु॒वः । \newline
55. स॒धस्थे᳚ ध्रु॒वो ध्रु॒वः स॒धस्थे॑ स॒धस्थे᳚ ध्रु॒व आ ध्रु॒वः स॒धस्थे॑ स॒धस्थे᳚ ध्रु॒व आ । \newline
56. स॒धस्थ॒ इति॑ स॒ध - स्थे॒ । \newline
57. ध्रु॒व आ ध्रु॒वो ध्रु॒व आ निष॑त्तो॒ निष॑त्त॒ आ ध्रु॒वो ध्रु॒व आ निष॑त्तः । \newline
\pagebreak
\markright{ TS 4.7.13.2  \hfill https://www.vedavms.in \hfill}

\section{ TS 4.7.13.2 }

\textbf{TS 4.7.13.2 } \newline
\textbf{Samhita Paata} \newline

आ निष॑त्तः ॥ नम॑स्ते अस्तु॒ मा मा॑ हिꣳसी॒र्विश्व॑स्य मू॒र्द्धन्नधि॑ तिष्ठसि श्रि॒तः । स॒मु॒द्रे ते॒ हृद॑य-म॒न्तरायु॒-र्द्यावा॑पृथि॒वी भुव॑ने॒ष्वर्पि॑ते ॥ उ॒द्नो द॑त्तोद॒धिं भि॑न्त्त दि॒वः प॒र्जन्या॑द॒न्तरि॑क्षात् पृथि॒व्यास्ततो॑ नो॒ वृष्ट्या॑वत । दि॒वो मू॒र्द्धाऽसि॑ पृथि॒व्या नाभि॒रूर्ग॒पामोष॑धीनां । वि॒श्वायुः॒ शर्म॑ स॒प्रथा॒ नम॑स्प॒थे ॥ येनर्.ष॑य॒स्तप॑सा स॒त्र - [  ] \newline

\textbf{Pada Paata} \newline

एति॑ । निष॑त्त॒ इति॒ नि - स॒त्तः॒ ॥ नमः॑ । ते॒ । अ॒स्तु॒ । मा । मा॒ । हिꣳ॒॒सीः॒ । विश्व॑स्य । मू॒द्‌र्धन्न् । अधीति॑ । ति॒ष्ठ॒सि॒ । श्रि॒तः ॥ स॒मु॒द्रे । ते॒ । हृद॑यम् । अ॒न्तः । आयुः॑ । द्यावा॑पृथि॒वी इति॒ द्यावा᳚ - पृ॒थि॒वी । भुव॑नेषु । अर्पि॑ते॒ इति॑ ॥ उ॒द्नः । द॒त्त॒ । उ॒द॒धिमित्यु॑द - धिम् । भि॒न्त॒ । दि॒वः । प॒र्जन्या᳚त् । अ॒न्तरि॑क्षात् । पृ॒थि॒व्याः । ततः॑ । नः॒ । वृष्ट्या᳚ । अ॒व॒त॒ ॥ दि॒वः । मू॒द्‌र्धा । अ॒सि॒ । पृ॒थि॒व्याः । नाभिः॑ । ऊर्क् । अ॒पाम् । ओष॑धीनाम् ॥ वि॒श्वायु॒रिति॑ वि॒श्व - आ॒युः॒ । शर्म॑ । स॒प्रथा॒ इति॑ स - प्रथाः᳚ । नमः॑ । प॒थे ॥ येन॑ । ऋष॑यः । तप॑सा । स॒त्रम् ।  \newline


\textbf{Krama Paata} \newline

आ निष॑त्तः । निष॑त्त॒ इति॒ नि - स॒त्तः॒ ॥ नम॑स्ते । ते॒ अ॒स्तु॒ । अ॒स्तु॒ मा । मा मा᳚ । मा॒ हिꣳ॒॒सीः॒ । हिꣳ॒॒सी॒र् विश्व॑स्य । विश्व॑स्य मू॒र्द्धन्न् । मू॒र्द्धन्नधि॑ । अधि॑ तिष्ठसि । ति॒ष्ठ॒सि॒ श्रि॒तः । श्रि॒त इति॑ श्रि॒तः ॥ स॒मु॒द्रे ते᳚ । ते॒ हृद॑यम् । हृद॑यम॒न्तः । अ॒न्तरायुः॑ । आयु॒र् द्यावा॑पृथि॒वी । द्यावा॑पृथि॒वी भुव॑नेषु । द्यावा॑पृथि॒वी इति॒ द्यावा᳚ - पृ॒थि॒वी । भुव॑ने॒ष्वर्पि॑ते । अर्पि॑ते॒ इत्यर्पि॑ते ॥ उ॒द्नो द॑त्त । द॒त्तो॒द॒धिम् । उ॒द॒धिम् भि॑न्त । उ॒द॒धिमित्यु॑द - धिम् । भि॒न्त॒ दि॒वः । दि॒वः प॒र्जन्या᳚त् । प॒र्जन्या॑द॒न्तरि॑क्षात् । अ॒न्तरि॑क्षात् पृथि॒व्याः । पृ॒थि॒व्यास्ततः॑ । ततो॑ नः । नो॒ वृष्ट्या᳚ । वृष्ट्या॑ऽवत । अ॒व॒तेत्य॑वत ॥ दि॒वो मू॒र्द्धा । मू॒र्द्धाऽसि॑ । अ॒सि॒ पृ॒थि॒व्या । पृ॒थि॒व्या नाभिः॑ । 
नाभि॒रूर्क् । ऊर्ग॒पाम् । अ॒पामोष॑धीनाम् । ओष॑धीना॒मित्योष॑धीनाम् ॥ वि॒श्वायुः॒ शर्म॑ । वि॒श्वायु॒रिति॑ वि॒श्व - आ॒युः॒ । शर्म॑ स॒प्रथाः᳚ । स॒प्रथा॒ नमः॑ । स॒प्रथा॒ इति॑ स - प्रथाः᳚ । नम॑स्प॒थे । प॒थ इति॑ पथे ॥ येनर्.ष॑यः । ऋष॑य॒स्तप॑सा । तप॑सा स॒त्रम् । स॒त्रमास॑त \newline

\textbf{Jatai Paata} \newline

1. आ निष॑त्तो॒ निष॑त्त॒ आ निष॑त्तः । \newline
2. निष॑त्त॒ इति॒ नि - स॒त्तः॒ । \newline
3. नम॑ स्ते ते॒ नमो॒ नम॑ स्ते । \newline
4. ते॒ अ॒स्त्व॒स्तु॒ ते॒ ते॒ अ॒स्तु॒ । \newline
5. अ॒स्तु॒ मा मा ऽस्त्व॑स्तु॒ मा । \newline
6. मा मा॑ मा॒ मा मा मा᳚ । \newline
7. मा॒ हि॒(ग्म्॒)सी॒र्॒. हि॒(ग्म्॒)सी॒र् मा॒ मा॒ हि॒(ग्म्॒)सीः॒ । \newline
8. हि॒(ग्म्॒)सी॒र् विश्व॑स्य॒ विश्व॑स्य हिꣳसीर्. हिꣳसी॒र् विश्व॑स्य । \newline
9. विश्व॑स्य मू॒र्द्धन् मू॒र्द्धन्. विश्व॑स्य॒ विश्व॑स्य मू॒र्द्धन्न् । \newline
10. मू॒र्द्धन् नध्यधि॑ मू॒र्द्धन् मू॒र्द्धन् नधि॑ । \newline
11. अधि॑ तिष्ठसि तिष्ठ॒ स्यध्यधि॑ तिष्ठसि । \newline
12. ति॒ष्ठ॒सि॒ श्रि॒तः श्रि॒त स्ति॑ष्ठसि तिष्ठसि श्रि॒तः । \newline
13. श्रि॒त इति॑ श्रि॒तः । \newline
14. स॒मु॒द्रे ते॑ ते समु॒द्रे स॑मु॒द्रे ते᳚ । \newline
15. ते॒ हृद॑य॒(ग्म्॒) हृद॑यम् ते ते॒ हृद॑यम् । \newline
16. हृद॑य म॒न्त र॒न्तर्. हृद॑य॒(ग्म्॒) हृद॑य म॒न्तः । \newline
17. अ॒न्त रायु॒ रायु॑ र॒न्त र॒न्त रायुः॑ । \newline
18. आयु॒र् द्यावा॑पृथि॒वी द्यावा॑पृथि॒वी आयु॒ रायु॒र् द्यावा॑पृथि॒वी । \newline
19. द्यावा॑पृथि॒वी भुव॑नेषु॒ भुव॑नेषु॒ द्यावा॑पृथि॒वी द्यावा॑पृथि॒वी भुव॑नेषु । \newline
20. द्यावा॑पृथि॒वी इति॒ द्यावा᳚ - पृ॒थि॒वी । \newline
21. भुव॑ने॒ ष्वर्पि॑ते॒ अर्पि॑ते॒ भुव॑नेषु॒ भुव॑ने॒ ष्वर्पि॑ते । \newline
22. अर्पि॑ते॒ इत्यर्पि॑ते । \newline
23. उ॒द्नो द॑त्त दत्तो॒द्न उ॒द्नो द॑त्त । \newline
24. द॒त्तो॒द॒धि मु॑द॒धिम् द॑त्त दत्तोद॒धिम् । \newline
25. उ॒द॒धिम् भि॑न्त भिन्तोद॒धि मु॑द॒धिम् भि॑न्त । \newline
26. उ॒द॒धिमित्यु॑द - धिम् । \newline
27. भि॒न्त॒ दि॒वो दि॒वो भि॑न्त भिन्त दि॒वः । \newline
28. दि॒वः प॒र्जन्या᳚त् प॒र्जन्या᳚द् दि॒वो दि॒वः प॒र्जन्या᳚त् । \newline
29. प॒र्जन्या॑ द॒न्तरि॑क्षा द॒न्तरि॑क्षात् प॒र्जन्या᳚त् प॒र्जन्या॑ द॒न्तरि॑क्षात् । \newline
30. अ॒न्तरि॑क्षात् पृथि॒व्याः पृ॑थि॒व्या अ॒न्तरि॑क्षा द॒न्तरि॑क्षात् पृथि॒व्याः । \newline
31. पृ॒थि॒व्या स्तत॒ स्ततः॑ पृथि॒व्याः पृ॑थि॒व्या स्ततः॑ । \newline
32. ततो॑ नो न॒ स्तत॒ स्ततो॑ नः । \newline
33. नो॒ वृष्ट्या॒ वृष्ट्या॑ नो नो॒ वृष्ट्या᳚ । \newline
34. वृष्ट्या॑ ऽवता वत॒ वृष्ट्या॒ वृष्ट्या॑ ऽवत । \newline
35. अ॒व॒तेत्य॑वत । \newline
36. दि॒वो मू॒र्द्धा मू॒र्द्धा दि॒वो दि॒वो मू॒र्द्धा । \newline
37. मू॒र्द्धा ऽस्य॑सि मू॒र्द्धा मू॒र्द्धा ऽसि॑ । \newline
38. अ॒सि॒ पृ॒थि॒व्याः पृ॑थि॒व्या अ॑स्यसि पृथि॒व्याः । \newline
39. पृ॒थि॒व्या नाभि॒र् नाभिः॑ पृथि॒व्याः पृ॑थि॒व्या नाभिः॑ । \newline
40. नाभि॒ रूर्गूर्ङ् नाभि॒र् नाभि॒ रूर्क् । \newline
41. ऊर्ग॒पा म॒पा मूर्गूर्ग॒पाम् । \newline
42. अ॒पा मोष॑धीना॒ मोष॑धीना म॒पा म॒पा मोष॑धीनाम् । \newline
43. ओष॑धीना॒मित्योष॑धीनाम् । \newline
44. वि॒श्वायुः॒ शर्म॒ शर्म॑ वि॒श्वायु॑र् वि॒श्वायुः॒ शर्म॑ । \newline
45. वि॒श्वायु॒रिति॑ वि॒श्व - आ॒युः॒ । \newline
46. शर्म॑ स॒प्रथाः᳚ स॒प्रथाः॒ शर्म॒ शर्म॑ स॒प्रथाः᳚ । \newline
47. स॒प्रथा॒ नमो॒ नमः॑ स॒प्रथाः᳚ स॒प्रथा॒ नमः॑ । \newline
48. स॒प्रथा॒ इति॑ स - प्रथाः᳚ । \newline
49. नम॑ स्प॒थे प॒थे नमो॒ नम॑ स्प॒थे । \newline
50. प॒थ इति॑ प॒थे । \newline
51. येन र्.ष॑य॒ ऋष॑यो॒ येन॒ येन र्.ष॑यः । \newline
52. ऋष॑य॒ स्तप॑सा॒ तप॒सर्.ष॑य॒ ऋष॑य॒ स्तप॑सा । \newline
53. तप॑सा स॒त्रꣳ स॒त्रम् तप॑सा॒ तप॑सा स॒त्रम् । \newline
54. स॒त्र मास॒ता स॑त स॒त्रꣳ स॒त्र मास॑त । \newline

\textbf{Ghana Paata } \newline

1. आ निष॑त्तो॒ निष॑त्त॒ आ निष॑त्तः । \newline
2. निष॑त्त॒ इति॒ नि - स॒त्तः॒ । \newline
3. नम॑स्ते ते॒ नमो॒ नम॑स्ते अस्त्वस्तु ते॒ नमो॒ नम॑स्ते अस्तु । \newline
4. ते॒ अ॒स्त्व॒स्तु॒ ते॒ ते॒ अ॒स्तु॒ मा मा ऽस्तु॑ ते ते अस्तु॒ मा । \newline
5. अ॒स्तु॒ मा मा ऽस्त्व॑स्तु॒ मा मा॑ मा॒ मा ऽस्त्व॑स्तु॒ मा मा᳚ । \newline
6. मा मा॑ मा॒ मा मा मा॑ हिꣳसीर्. हिꣳसीर् मा॒ मा मा मा॑ हिꣳसीः । \newline
7. मा॒ हि॒(ग्म्॒)सी॒र्॒. हि॒(ग्म्॒)सी॒र् मा॒ मा॒ हि॒(ग्म्॒)सी॒र् विश्व॑स्य॒ विश्व॑स्य हिꣳसीर् मा मा हिꣳसी॒र् विश्व॑स्य । \newline
8. हि॒(ग्म्॒)सी॒र् विश्व॑स्य॒ विश्व॑स्य हिꣳसीर्. हिꣳसी॒र् विश्व॑स्य मू॒र्द्धन् मू॒र्द्धन्. विश्व॑स्य हिꣳसीर्. हिꣳसी॒र् विश्व॑स्य मू॒र्द्धन्न् । \newline
9. विश्व॑स्य मू॒र्द्धन् मू॒र्द्धन्. विश्व॑स्य॒ विश्व॑स्य मू॒र्द्धन् नध्यधि॑ मू॒र्द्धन्. विश्व॑स्य॒ विश्व॑स्य मू॒र्द्धन् नधि॑ । \newline
10. मू॒र्द्धन् नध्यधि॑ मू॒र्द्धन् मू॒र्द्धन् नधि॑ तिष्ठसि तिष्ठ॒स्यधि॑ मू॒र्द्धन् मू॒र्द्धन् नधि॑ तिष्ठसि । \newline
11. अधि॑ तिष्ठसि तिष्ठ॒ स्यध्यधि॑ तिष्ठसि श्रि॒तः श्रि॒त स्ति॑ष्ठ॒ स्यध्यधि॑ तिष्ठसि श्रि॒तः । \newline
12. ति॒ष्ठ॒सि॒ श्रि॒तः श्रि॒त स्ति॑ष्ठसि तिष्ठसि श्रि॒तः । \newline
13. श्रि॒त इति॑ श्रि॒तः । \newline
14. स॒मु॒द्रे ते॑ ते समु॒द्रे स॑मु॒द्रे ते॒ हृद॑य॒(ग्म्॒) हृद॑यम् ते समु॒द्रे स॑मु॒द्रे ते॒ हृद॑यम् । \newline
15. ते॒ हृद॑य॒(ग्म्॒) हृद॑यम् ते ते॒ हृद॑य म॒न्तर् अ॒न्तर्. हृद॑यम् ते ते॒ हृद॑य म॒न्तः । \newline
16. हृद॑य म॒न्तर् अ॒न्तर्. हृद॑य॒(ग्म्॒) हृद॑य म॒न्तर् आयु॒ रायु॑ र॒न्तर्. हृद॑य॒(ग्म्॒) हृद॑य म॒न्तर् आयुः॑ । \newline
17. अ॒न्त रायु॒ रायु॑ र॒न्तर् अ॒न्त रायु॒र् द्यावा॑पृथि॒वी द्यावा॑पृथि॒वी आयु॑र॒न्त र॒न्तर् आयु॒र् द्यावा॑पृथि॒वी । \newline
18. आयु॒र् द्यावा॑पृथि॒वी द्यावा॑पृथि॒वी आयु॒ रायु॒र् द्यावा॑पृथि॒वी भुव॑नेषु॒ भुव॑नेषु॒ द्यावा॑पृथि॒वी आयु॒ रायु॒र् द्यावा॑पृथि॒वी भुव॑नेषु । \newline
19. द्यावा॑पृथि॒वी भुव॑नेषु॒ भुव॑नेषु॒ द्यावा॑पृथि॒वी द्यावा॑पृथि॒वी भुव॑ने॒ ष्वर्पि॑ते॒ अर्पि॑ते॒ भुव॑नेषु॒ द्यावा॑पृथि॒वी द्यावा॑पृथि॒वी भुव॑ने॒ ष्वर्पि॑ते । \newline
20. द्यावा॑पृथि॒वी इति॒ द्यावा᳚ - पृ॒थि॒वी । \newline
21. भुव॑ने॒ ष्वर्पि॑ते॒ अर्पि॑ते॒ भुव॑नेषु॒ भुव॑ने॒ ष्वर्पि॑ते । \newline
22. अर्पि॑ते॒ इत्यर्पि॑ते । \newline
23. उ॒द्नो द॑त्त दत्तो॒द्न उ॒द्नो द॑त्तोद॒धि मु॑द॒धिम् द॑त्तो॒द्न उ॒द्नो द॑त्तोद॒धिम् । \newline
24. द॒त्तो॒द॒धि मु॑द॒धिम् द॑त्त दत्तोद॒धिम् भि॑न्त भिन्तोद॒धिम् द॑त्त दत्तोद॒धिम् भि॑न्त । \newline
25. उ॒द॒धिम् भि॑न्त भिन्तोद॒धि मु॑द॒धिम् भि॑न्त दि॒वो दि॒वो भि॑न्तोद॒धि मु॑द॒धिम् भि॑न्त दि॒वः । \newline
26. उ॒द॒धिमित्यु॑द - धिम् । \newline
27. भि॒न्त॒ दि॒वो दि॒वो भि॑न्त भिन्त दि॒वः प॒र्जन्या᳚त् प॒र्जन्या᳚द् दि॒वो भि॑न्त भिन्त दि॒वः प॒र्जन्या᳚त् । \newline
28. दि॒वः प॒र्जन्या᳚त् प॒र्जन्या᳚द् दि॒वो दि॒वः प॒र्जन्या॑ द॒न्तरि॑क्षा द॒न्तरि॑क्षात् प॒र्जन्या᳚द् दि॒वो दि॒वः प॒र्जन्या॑ द॒न्तरि॑क्षात् । \newline
29. प॒र्जन्या॑ द॒न्तरि॑क्षा द॒न्तरि॑क्षात् प॒र्जन्या᳚त् प॒र्जन्या॑ द॒न्तरि॑क्षात् पृथि॒व्याः पृ॑थि॒व्या अ॒न्तरि॑क्षात् प॒र्जन्या᳚त् प॒र्जन्या॑ द॒न्तरि॑क्षात् पृथि॒व्याः । \newline
30. अ॒न्तरि॑क्षात् पृथि॒व्याः पृ॑थि॒व्या अ॒न्तरि॑क्षा द॒न्तरि॑क्षात् पृथि॒व्या स्तत॒ स्ततः॑ पृथि॒व्या अ॒न्तरि॑क्षा द॒न्तरि॑क्षात् पृथि॒व्या स्ततः॑ । \newline
31. पृ॒थि॒व्या स्तत॒ स्ततः॑ पृथि॒व्याः पृ॑थि॒व्या स्ततो॑ नो न॒ स्ततः॑ पृथि॒व्याः पृ॑थि॒व्या स्ततो॑ नः । \newline
32. ततो॑ नो न॒ स्तत॒ स्ततो॑ नो॒ वृष्ट्या॒ वृष्ट्या॑ न॒ स्तत॒ स्ततो॑ नो॒ वृष्ट्या᳚ । \newline
33. नो॒ वृष्ट्या॒ वृष्ट्या॑ नो नो॒ वृष्ट्या॑ ऽवतावत॒ वृष्ट्या॑ नो नो॒ वृष्ट्या॑ ऽवत । \newline
34. वृष्ट्या॑ ऽवता वत॒ वृष्ट्या॒ वृष्ट्या॑ ऽवत । \newline
35. अ॒व॒तेत्य॑वत । \newline
36. दि॒वो मू॒र्द्धा मू॒र्द्धा दि॒वो दि॒वो मू॒र्द्धा ऽस्य॑सि मू॒र्द्धा दि॒वो दि॒वो मू॒र्द्धा ऽसि॑ । \newline
37. मू॒र्द्धा ऽस्य॑सि मू॒र्द्धा मू॒र्द्धा ऽसि॑ पृथि॒व्याः पृ॑थि॒व्या अ॑सि मू॒र्द्धा मू॒र्द्धा ऽसि॑ पृथि॒व्याः । \newline
38. अ॒सि॒ पृ॒थि॒व्याः पृ॑थि॒व्या अ॑स्यसि पृथि॒व्या नाभि॒र् नाभिः॑ पृथि॒व्या अ॑स्यसि पृथि॒व्या नाभिः॑ । \newline
39. पृ॒थि॒व्या नाभि॒र् नाभिः॑ पृथि॒व्याः पृ॑थि॒व्या नाभि॒ रूर् गूर्ङ् नाभिः॑ पृथि॒व्याः पृ॑थि॒व्या नाभि॒ रूर्क् । \newline
40. नाभि॒ रूर् गूर्ङ् नाभि॒र् नाभि॒ रूर् ग॒पा म॒पा मूर्ङ् नाभि॒र् नाभि॒ रूर्ग॒पाम् । \newline
41. ऊर्ग॒पा म॒पा मूर् गूर् ग॒पा मोष॑धीना॒ मोष॑धीना म॒पा मूर् गूर्ग॒पा मोष॑धीनाम् । \newline
42. अ॒पा मोष॑धीना॒ मोष॑धीना म॒पा म॒पा मोष॑धीनाम् । \newline
43. ओष॑धीना॒मित्योष॑धीनाम् । \newline
44. वि॒श्वायुः॒ शर्म॒ शर्म॑ वि॒श्वायु॑र् वि॒श्वायुः॒ शर्म॑ स॒प्रथाः᳚ स॒प्रथाः॒ शर्म॑ वि॒श्वायु॑र् वि॒श्वायुः॒ शर्म॑ स॒प्रथाः᳚ । \newline
45. वि॒श्वायु॒रिति॑ वि॒श्व - आ॒युः॒ । \newline
46. शर्म॑ स॒प्रथाः᳚ स॒प्रथाः॒ शर्म॒ शर्म॑ स॒प्रथा॒ नमो॒ नमः॑ स॒प्रथाः॒ शर्म॒ शर्म॑ स॒प्रथा॒ नमः॑ । \newline
47. स॒प्रथा॒ नमो॒ नमः॑ स॒प्रथाः᳚ स॒प्रथा॒ नम॑ स्प॒थे प॒थे नमः॑ स॒प्रथाः᳚ स॒प्रथा॒ नम॑ स्प॒थे । \newline
48. स॒प्रथा॒ इति॑ स - प्रथाः᳚ । \newline
49. नम॑ स्प॒थे प॒थे नमो॒ नम॑ स्प॒थे । \newline
50. प॒थ इति॑ प॒थे । \newline
51. येन र्.ष॑य॒ ऋष॑यो॒ येन॒ येन र्.ष॑य॒ स्तप॑सा॒ तप॒स र्.ष॑यो॒ येन॒ येन र्.ष॑य॒ स्तप॑सा । \newline
52. ऋष॑य॒ स्तप॑सा॒ तप॒स र्.ष॑य॒ ऋष॑य॒ स्तप॑सा स॒त्रꣳ स॒त्रम् तप॒स र्.ष॑य॒ ऋष॑य॒ स्तप॑सा स॒त्रम् । \newline
53. तप॑सा स॒त्रꣳ स॒त्रम् तप॑सा॒ तप॑सा स॒त्र मास॒ता स॑त स॒त्रम् तप॑सा॒ तप॑सा स॒त्र मास॑त । \newline
54. स॒त्र मास॒ता स॑त स॒त्रꣳ स॒त्र मास॒तेन्धा॑ना॒ इन्धा॑ना॒ आस॑त स॒त्रꣳ स॒त्र मास॒तेन्धा॑नाः । \newline
\pagebreak
\markright{ TS 4.7.13.3  \hfill https://www.vedavms.in \hfill}

\section{ TS 4.7.13.3 }

\textbf{TS 4.7.13.3 } \newline
\textbf{Samhita Paata} \newline

मास॒तेन्धा॑ना अ॒ग्निꣳ सुव॑रा॒भर॑न्तः । तस्मि॑न्न॒हं नि द॑धे॒ नाके॑ अ॒ग्निमे॒तं ॅयमा॒हुर्मन॑व स्ती॒र्णब॑र्.हिषं ॥ तं पत्नी॑भि॒रनु॑ गच्छेम देवाः पु॒त्रैर्भ्रातृ॑भिरु॒त वा॒ हिर॑ण्यैः । नाकं॑ गृह्णा॒नाः सु॑कृ॒तस्य॑ लो॒के तृ॒तीये॑ पृ॒ष्ठे अधि॑ रोच॒ने दि॒वः ॥ आ वा॒चो मद्ध्य॑-मरुहद्-भुर॒ण्युर॒य-म॒ग्निः सत्प॑ति॒श्चेकि॑तानः ।पृ॒ष्ठे पृ॑थि॒व्या निहि॑तो॒ दवि॑द्युतदधस्प॒दं कृ॑णुते॒ - [  ] \newline

\textbf{Pada Paata} \newline

आस॑त । इन्धा॑नाः । अ॒ग्निम् । सुवः॑ । आ॒भर॑न्त॒ इत्या᳚ - भर॑न्तः ॥ तस्मिन्न्॑ । अ॒हम् । नीति॑ । द॒धे॒ । नाके᳚ । अ॒ग्निम् । ए॒तम् । यम् । आ॒हुः । मन॑वः । स्ती॒र्णब॑र्.हिष॒मिति॑ स्ती॒र्ण - ब॒र्॒.हि॒ष॒म् ॥ तम् । पत्नी॑भिः । अन्विति॑ । ग॒च्छे॒म॒ । दे॒वाः॒ । पु॒त्रैः । भ्रातृ॑भि॒रिति॒ भ्रातृ॑ - भिः॒ । उ॒त । वा॒ । हिर॑ण्यैः ॥ नाक᳚म् । गृ॒ह्णा॒नाः । सु॒कृ॒तस्येति॑ सु - कृ॒तस्य॑ । लो॒के । तृ॒तीये᳚ । पृ॒ष्ठे । अधीति॑ । रो॒च॒ने । दि॒वः ॥ एति॑ । व॒चः । मद्ध्य᳚म् । अ॒रु॒ह॒त् । भु॒र॒ण्युः । अ॒यम् । अ॒ग्निः । सत्प॑ति॒रिति॒ सत् - प॒तिः॒ । चेकि॑तानः ॥ पृ॒ष्ठे । पृ॒थि॒व्याः । निहि॑त॒ इति॒ नि - हि॒तः॒ । दवि॑द्युतत् । अ॒ध॒स्प॒दमित्य॑धः - प॒दम् । कृ॒णु॒ते॒ ।  \newline


\textbf{Krama Paata} \newline

आस॒तेन्धा॑नाः । इन्धा॑ना अ॒ग्निम् । अ॒ग्निꣳ सुवः॑ । सुव॑रा॒भर॑न्तः । आ॒भर॑न्त॒ इत्या᳚ - भर॑न्तः ॥ तस्मि॑न्न॒हम् । अ॒हम् नि । नि द॑धे । द॒धे॒ नाके᳚ । नाके॑ अ॒ग्निम् । अ॒ग्निमे॒तम् । ए॒तम् ॅयम् । यमा॒हुः । आ॒हुर् मन॑वः । मन॑वः स्ती॒र्णब॑र्.हिषिम् । स्ती॒र्णब॑र्.हिष॒मिति॑ स्ती॒र्ण - ब॒र्॒.हि॒ष॒॒म् ॥ तम् पत्नी॑भिः । पत्नी॑भि॒रनु॑ । अनु॑ गच्छेम । ग॒च्छे॒म॒ दे॒वाः॒ । दे॒वाः॒ पु॒त्रैः । पु॒त्रैर् भ्रातृ॑भिः । भ्रातृ॑भिरु॒त । भ्रातृ॑भि॒रिति॒ भ्रातृ॑ - भिः॒ । उ॒त वा᳚ । वा॒ हिर॑ण्यैः । हिर॑ण्यै॒रिति॒ हिर᳚ण्यैः ॥ नाक॑म् गृह्णा॒नाः । गृ॒ह्णा॒नाः सु॑कृ॒तस्य॑ । सु॒कृ॒तस्य॑ लो॒के । सु॒कृ॒तस्येति॑ सु - कृ॒तस्य॑ । लो॒के तृ॒तीये᳚ । तृ॒तीये॑ पृ॒ष्ठे । पृ॒ष्ठे अधि॑ । अधि॑ रोच॒ने । रो॒च॒ने दि॒वः । दि॒व इति॑ दि॒वः ॥ आ वा॒चः । वा॒चो मद्ध्य᳚म् । मद्ध्य॑मरुहत् । अ॒रु॒ह॒द् भु॒र॒ण्युः । भु॒र॒ण्युर॒यम् । अ॒यम॒ग्निः । अ॒ग्निः सत्प॑तिः । सत्प॑ति॒श्चेकि॑तानः । सत्प॑ति॒रिति॑ सत् - प॒तिः॒ । चेकि॑तान॒ इति॒ चेकि॑तानः ॥ पृ॒ष्ठे पृ॑थि॒व्याः । पृ॒थि॒व्या निहि॑तः । निहि॑तो॒ दवि॑द्युतत् । निहि॑त॒ इति॒ नि - हि॒तः॒ । दवि॑द्युतदधस्प॒दम् । अ॒ध॒स्प॒दम् कृ॑णुते । अ॒ध॒स्प॒दमित्य॑धः - प॒दम् । कृ॒णु॒ते॒ ये \newline

\textbf{Jatai Paata} \newline

1. आस॒ते न्धा॑ना॒ इन्धा॑ना॒ आस॒ता स॒ते न्धा॑नाः । \newline
2. इन्धा॑ना अ॒ग्नि म॒ग्नि मिन्धा॑ना॒ इन्धा॑ना अ॒ग्निम् । \newline
3. अ॒ग्निꣳ सुवः॒ सुव॑ र॒ग्नि म॒ग्निꣳ सुवः॑ । \newline
4. सुव॑ रा॒भर॑न्त आ॒भर॑न्तः॒ सुवः॒ सुव॑ रा॒भर॑न्तः । \newline
5. आ॒भर॑न्त॒ इत्या᳚ - भर॑न्तः । \newline
6. तस्मि॑न् न॒ह म॒हम् तस्मिꣳ॒॒ स्तस्मि॑न् न॒हम् । \newline
7. अ॒हम् नि न्य॑ह म॒हम् नि । \newline
8. नि द॑धे दधे॒ नि नि द॑धे । \newline
9. द॒धे॒ नाके॒ नाके॑ दधे दधे॒ नाके᳚ । \newline
10. नाके॑ अ॒ग्नि म॒ग्निम् नाके॒ नाके॑ अ॒ग्निम् । \newline
11. अ॒ग्नि मे॒त मे॒त म॒ग्नि म॒ग्नि मे॒तम् । \newline
12. ए॒तं ॅयं ॅय मे॒त मे॒तं ॅयम् । \newline
13. य मा॒हु रा॒हुर् यं ॅय मा॒हुः । \newline
14. आ॒हुर् मन॑वो॒ मन॑व आ॒हु रा॒हुर् मन॑वः । \newline
15. मन॑वः स्ती॒र्णब॑र्.हिषꣳ स्ती॒र्णब॑र्.हिष॒म् मन॑वो॒ मन॑वः स्ती॒र्णब॑र्.हिषम् । \newline
16. स्ती॒र्णब॑र्.हिष॒मिति॑ स्ती॒र्ण - ब॒र्.॒हि॒ष॒म् । \newline
17. तम् पत्नी॑भिः॒ पत्नी॑भि॒ स्तम् तम् पत्नी॑भिः । \newline
18. पत्नी॑भि॒ रन्वनु॒ पत्नी॑भिः॒ पत्नी॑भि॒ रनु॑ । \newline
19. अनु॑ गच्छेम गच्छे॒मा न्वनु॑ गच्छेम । \newline
20. ग॒च्छे॒म॒ दे॒वा॒ दे॒वा॒ ग॒च्छे॒म॒ ग॒च्छे॒म॒ दे॒वाः॒ । \newline
21. दे॒वाः॒ पु॒त्रैः पु॒त्रैर् दे॑वा देवाः पु॒त्रैः । \newline
22. पु॒त्रैर् भ्रातृ॑भि॒र् भ्रातृ॑भिः पु॒त्रैः पु॒त्रैर् भ्रातृ॑भिः । \newline
23. भ्रातृ॑भि रु॒तोत भ्रातृ॑भि॒र् भ्रातृ॑भि रु॒त । \newline
24. भ्रातृ॑भि॒रिति॒ भ्रातृ॑ - भिः॒ । \newline
25. उ॒त वा॑ वो॒तोत वा᳚ । \newline
26. वा॒ हिर॑ण्यै॒र्॒. हिर॑ण्यैर् वा वा॒ हिर॑ण्यैः । \newline
27. हिर॑ण्यै॒ रिति॒ हिर॑ण्यैः । \newline
28. नाक॑म् गृह्णा॒ना गृ॑ह्णा॒ना नाक॒म् नाक॑म् गृह्णा॒नाः । \newline
29. गृ॒ह्णा॒नाः सु॑कृ॒तस्य॑ सुकृ॒तस्य॑ गृह्णा॒ना गृ॑ह्णा॒नाः सु॑कृ॒तस्य॑ । \newline
30. सु॒कृ॒तस्य॑ लो॒के लो॒के सु॑कृ॒तस्य॑ सुकृ॒तस्य॑ लो॒के । \newline
31. सु॒कृ॒तस्येति॑ सु - कृ॒तस्य॑ । \newline
32. लो॒के तृ॒तीये॑ तृ॒तीये॑ लो॒के लो॒के तृ॒तीये᳚ । \newline
33. तृ॒तीये॑ पृ॒ष्ठे पृ॒ष्ठे तृ॒तीये॑ तृ॒तीये॑ पृ॒ष्ठे । \newline
34. पृ॒ष्ठे अध्यधि॑ पृ॒ष्ठे पृ॒ष्ठे अधि॑ । \newline
35. अधि॑ रोच॒ने रो॑च॒ने ऽध्यधि॑ रोच॒ने । \newline
36. रो॒च॒ने दि॒वो दि॒वो रो॑च॒ने रो॑च॒ने दि॒वः । \newline
37. दि॒व इति॑ दि॒वः । \newline
38. आ वा॒चो वा॒च आ वा॒चः । \newline
39. वा॒चो मद्ध्य॒म् मद्ध्यं॑ ॅवा॒चो वा॒चो मद्ध्य᳚म् । \newline
40. मद्ध्य॑ मरुह दरुह॒न् मद्ध्य॒म् मद्ध्य॑ मरुहत् । \newline
41. अ॒रु॒ह॒द् भु॒र॒ण्युर् भु॑र॒ण्यु र॑रुह दरुहद् भुर॒ण्युः । \newline
42. भु॒र॒ण्यु र॒य म॒यम् भु॑र॒ण्युर् भु॑र॒ण्यु र॒यम् । \newline
43. अ॒य म॒ग्नि र॒ग्नि र॒य म॒य म॒ग्निः । \newline
44. अ॒ग्निः सत्प॑तिः॒ सत्प॑ति र॒ग्नि र॒ग्निः सत्प॑तिः । \newline
45. सत्प॑ति॒ श्चेकि॑तान॒ श्चेकि॑तानः॒ सत्प॑तिः॒ सत्प॑ति॒ श्चेकि॑तानः । \newline
46. सत्प॑ति॒रिति॒ सत् - प॒तिः॒ । \newline
47. चेकि॑तान॒ इति॒ चेकि॑तानः । \newline
48. पृ॒ष्ठे पृ॑थि॒व्याः पृ॑थि॒व्याः पृ॒ष्ठे पृ॒ष्ठे पृ॑थि॒व्याः । \newline
49. पृ॒थि॒व्या निहि॑तो॒ निहि॑तः पृथि॒व्याः पृ॑थि॒व्या निहि॑तः । \newline
50. निहि॑तो॒ दवि॑द्युत॒द् दवि॑द्युत॒न् निहि॑तो॒ निहि॑तो॒ दवि॑द्युतत् । \newline
51. निहि॑त॒ इति॒ नि - हि॒तः॒ । \newline
52. दवि॑द्युत दधस्प॒द म॑धस्प॒दम् दवि॑द्युत॒द् दवि॑द्युत दधस्प॒दम् । \newline
53. अ॒ध॒स्प॒दम् कृ॑णुते कृणुते अधस्प॒द म॑धस्प॒दम् कृ॑णुते । \newline
54. अ॒ध॒स्प॒दमित्य॑धः - प॒दम् । \newline
55. कृ॒णु॒ते॒ ये ये कृ॑णुते कृणुते॒ ये । \newline

\textbf{Ghana Paata } \newline

1. आस॒तेन्धा॑ना॒ इन्धा॑ना॒ आस॒ता स॒तेन्धा॑ना अ॒ग्नि म॒ग्नि मिन्धा॑ना॒ आस॒ता स॒तेन्धा॑ना अ॒ग्निम् । \newline
2. इन्धा॑ना अ॒ग्नि म॒ग्नि मिन्धा॑ना॒ इन्धा॑ना अ॒ग्निꣳ सुवः॒ सुव॑ र॒ग्नि मिन्धा॑ना॒ इन्धा॑ना अ॒ग्निꣳ सुवः॑ । \newline
3. अ॒ग्निꣳ सुवः॒ सुव॑ र॒ग्नि म॒ग्निꣳ सुव॑ रा॒भर॑न्त आ॒भर॑न्तः॒ सुव॑ र॒ग्नि म॒ग्निꣳ सुव॑ रा॒भर॑न्तः । \newline
4. सुव॑ रा॒भर॑न्त आ॒भर॑न्तः॒ सुवः॒ सुव॑ रा॒भर॑न्तः । \newline
5. आ॒भर॑न्त॒ इत्या᳚ - भर॑न्तः । \newline
6. तस्मि॑न् न॒ह म॒हम् तस्मिꣳ॒॒ स्तस्मि॑न् न॒हम् नि न्य॑हम् तस्मिꣳ॒॒ स्तस्मि॑न् न॒हम् नि । \newline
7. अ॒हम् नि न्य॑ह म॒हम् नि द॑धे दधे॒ न्य॑ह म॒हम् नि द॑धे । \newline
8. नि द॑धे दधे॒ नि नि द॑धे॒ नाके॒ नाके॑ दधे॒ नि नि द॑धे॒ नाके᳚ । \newline
9. द॒धे॒ नाके॒ नाके॑ दधे दधे॒ नाके॑ अ॒ग्नि म॒ग्निम् नाके॑ दधे दधे॒ नाके॑ अ॒ग्निम् । \newline
10. नाके॑ अ॒ग्नि म॒ग्निम् नाके॒ नाके॑ अ॒ग्नि मे॒त मे॒त म॒ग्निम् नाके॒ नाके॑ अ॒ग्नि मे॒तम् । \newline
11. अ॒ग्नि मे॒त मे॒त म॒ग्नि म॒ग्नि मे॒तं ॅयं ॅय मे॒त म॒ग्नि म॒ग्नि मे॒तं ॅयम् । \newline
12. ए॒तं ॅयं ॅय मे॒त मे॒तं ॅय मा॒हु रा॒हुर् य मे॒त मे॒तं ॅय मा॒हुः । \newline
13. य मा॒हु रा॒हुर् यं ॅय मा॒हुर् मन॑वो॒ मन॑व आ॒हुर् यं ॅय मा॒हुर् मन॑वः । \newline
14. आ॒हुर् मन॑वो॒ मन॑व आ॒हु रा॒हुर् मन॑वः स्ती॒र्णब॑र्.हिषꣳ स्ती॒र्णब॑र्.हिष॒म् मन॑व आ॒हु रा॒हुर् मन॑वः स्ती॒र्णब॑र्.हिषम् । \newline
15. मन॑वः स्ती॒र्णब॑र्.हिषꣳ स्ती॒र्णब॑र्.हिष॒म् मन॑वो॒ मन॑वः स्ती॒र्णब॑र्.हिषम् । \newline
16. स्ती॒र्णब॑र्.हिष॒मिति॑ स्ती॒र्ण - ब॒र्.॒हि॒ष॒म् । \newline
17. तम् पत्नी॑भिः॒ पत्नी॑भि॒ स्तम् तम् पत्नी॑भि॒ रन्वनु॒ पत्नी॑भि॒ स्तम् तम् पत्नी॑भि॒ रनु॑ । \newline
18. पत्नी॑भि॒ रन्वनु॒ पत्नी॑भिः॒ पत्नी॑भि॒ रनु॑ गच्छेम गच्छे॒मानु॒ पत्नी॑भिः॒ पत्नी॑भि॒ रनु॑ गच्छेम । \newline
19. अनु॑ गच्छेम गच्छे॒मान्वनु॑ गच्छेम देवा देवा गच्छे॒मान्वनु॑ गच्छेम देवाः । \newline
20. ग॒च्छे॒म॒ दे॒वा॒ दे॒वा॒ ग॒च्छे॒म॒ ग॒च्छे॒म॒ दे॒वाः॒ पु॒त्रैः पु॒त्रैर् दे॑वा गच्छेम गच्छेम देवाः पु॒त्रैः । \newline
21. दे॒वाः॒ पु॒त्रैः पु॒त्रैर् दे॑वा देवाः पु॒त्रैर् भ्रातृ॑भि॒र् भ्रातृ॑भिः पु॒त्रैर् दे॑वा देवाः पु॒त्रैर् भ्रातृ॑भिः । \newline
22. पु॒त्रैर् भ्रातृ॑भि॒र् भ्रातृ॑भिः पु॒त्रैः पु॒त्रैर् भ्रातृ॑भि रु॒तोत भ्रातृ॑भिः पु॒त्रैः पु॒त्रैर् भ्रातृ॑भि रु॒त । \newline
23. भ्रातृ॑भि रु॒तोत भ्रातृ॑भि॒र् भ्रातृ॑भि रु॒त वा॑ वो॒त भ्रातृ॑भि॒र् भ्रातृ॑भि रु॒त वा᳚ । \newline
24. भ्रातृ॑भि॒रिति॒ भ्रातृ॑ - भिः॒ । \newline
25. उ॒त वा॑ वो॒तोत वा॒ हिर॑ण्यै॒र्॒. हिर॑ण्यैर् वो॒तोत वा॒ हिर॑ण्यैः । \newline
26. वा॒ हिर॑ण्यै॒र्॒. हिर॑ण्यैर् वा वा॒ हिर॑ण्यैः । \newline
27. हिर॑ण्यै॒रिति॒ हिर॑ण्यैः । \newline
28. नाक॑म् गृह्णा॒ना गृ॑ह्णा॒ना नाक॒म् नाक॑म् गृह्णा॒नाः सु॑कृ॒तस्य॑ सुकृ॒तस्य॑ गृह्णा॒ना नाक॒म् नाक॑म् गृह्णा॒नाः सु॑कृ॒तस्य॑ । \newline
29. गृ॒ह्णा॒नाः सु॑कृ॒तस्य॑ सुकृ॒तस्य॑ गृह्णा॒ना गृ॑ह्णा॒नाः सु॑कृ॒तस्य॑ लो॒के लो॒के सु॑कृ॒तस्य॑ गृह्णा॒ना गृ॑ह्णा॒नाः सु॑कृ॒तस्य॑ लो॒के । \newline
30. सु॒कृ॒तस्य॑ लो॒के लो॒के सु॑कृ॒तस्य॑ सुकृ॒तस्य॑ लो॒के तृ॒तीये॑ तृ॒तीये॑ लो॒के सु॑कृ॒तस्य॑ सुकृ॒तस्य॑ लो॒के तृ॒तीये᳚ । \newline
31. सु॒कृ॒तस्येति॑ सु - कृ॒तस्य॑ । \newline
32. लो॒के तृ॒तीये॑ तृ॒तीये॑ लो॒के लो॒के तृ॒तीये॑ पृ॒ष्ठे पृ॒ष्ठे तृ॒तीये॑ लो॒के लो॒के तृ॒तीये॑ पृ॒ष्ठे । \newline
33. तृ॒तीये॑ पृ॒ष्ठे पृ॒ष्ठे तृ॒तीये॑ तृ॒तीये॑ पृ॒ष्ठे अध्यधि॑ पृ॒ष्ठे तृ॒तीये॑ तृ॒तीये॑ पृ॒ष्ठे अधि॑ । \newline
34. पृ॒ष्ठे अध्यधि॑ पृ॒ष्ठे पृ॒ष्ठे अधि॑ रोच॒ने रो॑च॒ने ऽधि॑ पृ॒ष्ठे पृ॒ष्ठे अधि॑ रोच॒ने । \newline
35. अधि॑ रोच॒ने रो॑च॒ने ऽध्यधि॑ रोच॒ने दि॒वो दि॒वो रो॑च॒ने ऽध्यधि॑ रोच॒ने दि॒वः । \newline
36. रो॒च॒ने दि॒वो दि॒वो रो॑च॒ने रो॑च॒ने दि॒वः । \newline
37. दि॒व इति॑ दि॒वः । \newline
38. आ वा॒चो वा॒च आ वा॒चो मद्ध्य॒म् मद्ध्यं॑ ॅवा॒च आ वा॒चो मद्ध्य᳚म् । \newline
39. वा॒चो मद्ध्य॒म् मद्ध्यं॑ ॅवा॒चो वा॒चो मद्ध्य॑ मरुह दरुह॒न् मद्ध्यं॑ ॅवा॒चो वा॒चो मद्ध्य॑ मरुहत् । \newline
40. मद्ध्य॑ मरुह दरुह॒न् मद्ध्य॒म् मद्ध्य॑ मरुहद् भुर॒ण्युर् भु॑र॒ण्यु र॑रुह॒न् मद्ध्य॒म् मद्ध्य॑ मरुहद् भुर॒ण्युः । \newline
41. अ॒रु॒ह॒द् भु॒र॒ण्युर् भु॑र॒ण्यु र॑रुह दरुहद् भुर॒ण्यु र॒य म॒यम् भु॑र॒ण्यु र॑रुह दरुहद् भुर॒ण्यु र॒यम् । \newline
42. भु॒र॒ण्यु र॒य म॒यम् भु॑र॒ण्युर् भु॑र॒ण्यु र॒य म॒ग्नि र॒ग्नि र॒यम् भु॑र॒ण्युर् भु॑र॒ण्यु र॒य म॒ग्निः । \newline
43. अ॒य म॒ग्नि र॒ग्नि र॒य म॒य म॒ग्निः सत्प॑तिः॒ सत्प॑ति र॒ग्नि र॒य म॒य म॒ग्निः सत्प॑तिः । \newline
44. अ॒ग्निः सत्प॑तिः॒ सत्प॑ति र॒ग्नि र॒ग्निः सत्प॑ति॒ श्चेकि॑तान॒ श्चेकि॑तानः॒ सत्प॑ति र॒ग्नि र॒ग्निः सत्प॑ति॒ श्चेकि॑तानः । \newline
45. सत्प॑ति॒ श्चेकि॑तान॒ श्चेकि॑तानः॒ सत्प॑तिः॒ सत्प॑ति॒ श्चेकि॑तानः । \newline
46. सत्प॑ति॒रिति॒ सत् - प॒तिः॒ । \newline
47. चेकि॑तान॒ इति॒ चेकि॑तानः । \newline
48. पृ॒ष्ठे पृ॑थि॒व्याः पृ॑थि॒व्याः पृ॒ष्ठे पृ॒ष्ठे पृ॑थि॒व्या निहि॑तो॒ निहि॑तः पृथि॒व्याः पृ॒ष्ठे पृ॒ष्ठे पृ॑थि॒व्या निहि॑तः । \newline
49. पृ॒थि॒व्या निहि॑तो॒ निहि॑तः पृथि॒व्याः पृ॑थि॒व्या निहि॑तो॒ दवि॑द्युत॒द् दवि॑द्युत॒न् निहि॑तः पृथि॒व्याः पृ॑थि॒व्या निहि॑तो॒ दवि॑द्युतत् । \newline
50. निहि॑तो॒ दवि॑द्युत॒द् दवि॑द्युत॒न् निहि॑तो॒ निहि॑तो॒ दवि॑द्युत दधस्प॒द म॑धस्प॒दम् दवि॑द्युत॒न् निहि॑तो॒ निहि॑तो॒ दवि॑द्युत दधस्प॒दम् । \newline
51. निहि॑त॒ इति॒ नि - हि॒तः॒ । \newline
52. दवि॑द्युत दधस्प॒द म॑धस्प॒दम् दवि॑द्युत॒द् दवि॑द्युत दधस्प॒दम् कृ॑णुते कृणुते अधस्प॒दम् दवि॑द्युत॒द् दवि॑द्युत दधस्प॒दम् कृ॑णुते । \newline
53. अ॒ध॒स्प॒दम् कृ॑णुते कृणुते अधस्प॒द म॑धस्प॒दम् कृ॑णुते॒ ये ये कृ॑णुते अधस्प॒द म॑धस्प॒दम् कृ॑णुते॒ ये । \newline
54. अ॒ध॒स्प॒दमित्य॑धः - प॒दम् । \newline
55. कृ॒णु॒ते॒ ये ये कृ॑णुते कृणुते॒ ये पृ॑त॒न्यवः॑ पृत॒न्यवो॒ ये कृ॑णुते कृणुते॒ ये पृ॑त॒न्यवः॑ । \newline
\pagebreak
\markright{ TS 4.7.13.4  \hfill https://www.vedavms.in \hfill}

\section{ TS 4.7.13.4 }

\textbf{TS 4.7.13.4 } \newline
\textbf{Samhita Paata} \newline

ये पृ॑त॒न्यवः॑ ॥ अ॒यम॒ग्निर्वी॒रत॑मो वयो॒धाः स॑ह॒स्रियो॑ दीप्यता॒मप्र॑युच्छन्न् । वि॒भ्राज॑मानः सरि॒रस्य॒ मद्ध्य॒ उप॒ प्र या॑त दि॒व्यानि॒ धाम॑ ॥ सं प्र च्य॑वद्ध्व॒मनु॒ सं प्र या॒ताग्ने॑ प॒थो दे॑व॒याना᳚न् कृणुद्ध्वं । अ॒स्मिन्थ् स॒धस्थे॒ अद्ध्युत्त॑रस्मि॒न् विश्वे॑ देवा॒ यज॑मानश्च सीदत ॥ येना॑ स॒हस्रं॒ ॅवह॑सि॒ येना᳚ग्ने सर्ववेद॒सं । तेने॒मं ॅय॒ज्ञ्ं नो॑ वह देव॒यानो॒ य - [  ] \newline

\textbf{Pada Paata} \newline

ये । पृ॒त॒न्यवः॑ ॥ अ॒यम् । अ॒ग्निः । वी॒रत॑म॒ इति॑ वी॒र - त॒मः॒ । व॒यो॒धा इति॑ वयः - धाः । स॒ह॒स्रियः॑ । दी॒प्य॒ता॒म् । अप्र॑युच्छ॒न्नित्यप्र॑ - यु॒च्छ॒न्न् ॥ वि॒भ्राज॑मान॒ इति॑ वि - भ्राज॑मानः । स॒रि॒रस्य॑ । मद्ध्ये᳚ । उप॑ । प्रेति॑ । या॒त॒ । दि॒व्यानि॑ । धाम॑ ॥ सम् । प्रेति॑ । च्य॒व॒द्ध्व॒म् । अन्विति॑ । सम् । प्रेति॑ । या॒त॒ । अग्ने᳚ । प॒थः । दे॒व॒याना॒निति॑ दे॒व - यानान्॑ । कृ॒णु॒द्ध्व॒म् ॥ अ॒स्मिन्न् । स॒धस्थ॒ इति॑ स॒ध - स्थे॒ । अधीति॑ । उत्त॑रस्मि॒न्नित्युत्-त॒र॒स्मि॒न्न् । विश्वे᳚ । दे॒वाः॒ । यज॑मानः । च॒ । सी॒द॒त॒ ॥ येन॑ । स॒हस्र᳚म् । वह॑सि । येन॑ । अ॒ग्ने॒ । स॒र्व॒वे॒द॒समिति॑ सर्व - वे॒द॒सम् ॥ तेन॑ । इ॒मम् । य॒ज्ञ्म् । नः॒ । व॒ह॒ । दे॒व॒यान॒ इति॑ देव - यानः॑ । यः ।  \newline


\textbf{Krama Paata} \newline

ये पृ॑त॒न्यवः॑ । पृ॒त॒न्यव॒ इति॑ पृत॒न्यवः॑ ॥ अ॒यम॒ग्निः । अ॒ग्निर् वी॒रत॑मः । वी॒रत॑मो वयो॒धाः । वी॒रत॑म॒ इति॑ वी॒र - त॒मः॒ । व॒यो॒धाः स॑ह॒स्रियः॑ । व॒यो॒धा इति॑ वयः - धाः । स॒ह॒स्रियो॑ दीप्यताम् । दी॒प्य॒ता॒मप्र॑युच्छन्न् । अप्र॑युच्छ॒न्नित्यप्र॑ - यु॒च्छ॒न्न्॒ ॥ वि॒भ्राज॑मानः सरि॒रस्य॑ । वि॒भ्राज॑मान॒ इति॑ वि - भ्राज॑मानः । स॒रि॒रस्य॒ मद्ध्ये᳚ । मद्ध्य॒ उप॑ । उप॒ प्र । प्र या॑त । या॒त॒ दि॒व्यानि॑ । दि॒व्यानि॒ धाम॑ । धामेति॒ धाम॑ ॥ सम् प्र । प्र च्य॑वद्धम् । च्य॒व॒द्ध॒मनु॑ । अनु॒ सम् । सम् प्र । प्र या॑त । या॒ताग्ने᳚ । अग्ने॑ प॒थः । प॒थो दे॑व॒यानान्॑ । दे॒व॒याना᳚न् कृणुद्ध्वम् । दे॒व॒याना॒निति॑ देव - यानान्॑ । कृ॒णु॒द्ध्व॒मिति॑ कृणुद्ध्वम् ॥ अ॒स्मिन्थ् स॒धस्थे᳚ । स॒धस्थे॒ अधि॑ । स॒धस्थ॒ इति॑ स॒ध - स्थे॒ । अद्ध्युत्त॑रस्मिन्न् । उत्त॑रस्मि॒न् विश्वे᳚ । उत्त॑रस्मि॒न्नित्युत् - त॒र॒स्मि॒न्न्॒ । विश्वे॑ देवाः । दे॒वा॒ यज॑मानः । यज॑मानश्च । च॒ सी॒द॒त॒ । सी॒द॒तेति॑ सीदत ॥ येना॑ स॒हस्र᳚म् । स॒हस्र॒म् ॅवह॑सि । वह॑सि॒ येन॑ । येना᳚ग्ने । अ॒ग्ने॒ स॒र्व॒वे॒द॒सम् । स॒र्व॒वे॒द॒समिति॑ सर्व - वे॒द॒सम् । तेने॒मम् । इ॒मम् ॅय॒ज्ञ्म् । य॒ज्ञ्म् नः॑ । नो॒ व॒ह॒ । व॒ह॒ दे॒व॒यानः॑ । दे॒व॒यानो॒ यः ( ) । दे॒व॒यान॒ इति॑ देव - यानः॑ । य उ॑त्त॒मः \newline

\textbf{Jatai Paata} \newline

1. ये पृ॑त॒न्यवः॑ पृत॒न्यवो॒ ये ये पृ॑त॒न्यवः॑ । \newline
2. पृ॒त॒न्यव॒ इति॑ पृत॒न्यवः॑ । \newline
3. अ॒य म॒ग्नि र॒ग्नि र॒य म॒य म॒ग्निः । \newline
4. अ॒ग्निर् वी॒रत॑मो वी॒रत॑मो॒ ऽग्नि र॒ग्निर् वी॒रत॑मः । \newline
5. वी॒रत॑मो वयो॒धा व॑यो॒धा वी॒रत॑मो वी॒रत॑मो वयो॒धाः । \newline
6. वी॒रत॑म॒ इति॑ वी॒र - त॒मः॒ । \newline
7. व॒यो॒धाः स॑ह॒स्रियः॑ सह॒स्रियो॑ वयो॒धा व॑यो॒धाः स॑ह॒स्रियः॑ । \newline
8. व॒यो॒धा इति॑ वयः - धाः । \newline
9. स॒ह॒स्रियो॑ दीप्यताम् दीप्यताꣳ सह॒स्रियः॑ सह॒स्रियो॑ दीप्यताम् । \newline
10. दी॒प्य॒ता॒ मप्र॑युच्छ॒न् नप्र॑युच्छन् दीप्यताम् दीप्यता॒ मप्र॑युच्छन्न् । \newline
11. अप्र॑युच्छ॒न्नित्यप्र॑ - यु॒च्छ॒न्न् । \newline
12. वि॒भ्राज॑मानः सरि॒रस्य॑ सरि॒रस्य॑ वि॒भ्राज॑मानो वि॒भ्राज॑मानः सरि॒रस्य॑ । \newline
13. वि॒भ्राज॑मान॒ इति॑ वि - भ्राज॑मानः । \newline
14. स॒रि॒रस्य॒ मद्ध्ये॒ मद्ध्ये॑ सरि॒रस्य॑ सरि॒रस्य॒ मद्ध्ये᳚ । \newline
15. मद्ध्य॒ उपोप॒ मद्ध्ये॒ मद्ध्य॒ उप॑ । \newline
16. उप॒ प्र प्रोपोप॒ प्र । \newline
17. प्र या॑त यात॒ प्र प्र या॑त । \newline
18. या॒त॒ दि॒व्यानि॑ दि॒व्यानि॑ यात यात दि॒व्यानि॑ । \newline
19. दि॒व्यानि॒ धाम॒ धाम॑ दि॒व्यानि॑ दि॒व्यानि॒ धाम॑ । \newline
20. धामेति॒ धाम॑ । \newline
21. सम् प्र प्र सꣳ सम् प्र । \newline
22. प्र च्य॑वद्ध्वम् च्यवद्ध्व॒म् प्र प्र च्य॑वद्ध्वम् । \newline
23. च्य॒व॒द्ध्व॒ मन्वनु॑ च्यवद्ध्वम् च्यवद्ध्व॒ मनु॑ । \newline
24. अनु॒ सꣳ स मन्वनु॒ सम् । \newline
25. सम् प्र प्र सꣳ सम् प्र । \newline
26. प्र या॑त यात॒ प्र प्र या॑त । \newline
27. या॒ताग्ने ऽग्ने॑ यात या॒ताग्ने᳚ । \newline
28. अग्ने॑ प॒थः प॒थो ऽग्ने ऽग्ने॑ प॒थः । \newline
29. प॒थो दे॑व॒याना᳚न् देव॒याना᳚न् प॒थः प॒थो दे॑व॒यानान्॑ । \newline
30. दे॒व॒याना᳚न् कृणुद्ध्वम् कृणुद्ध्वम् देव॒याना᳚न् देव॒याना᳚न् कृणुद्ध्वम् । \newline
31. दे॒व॒याना॒निति॑ देव - यानान्॑ । \newline
32. कृ॒णु॒द्ध्व॒मिति॑ कृणुद्ध्वम् । \newline
33. अ॒स्मिन् थ्स॒धस्थे॑ स॒धस्थे॑ अ॒स्मिन् न॒स्मिन् थ्स॒धस्थे᳚ । \newline
34. स॒धस्थे॒ अध्यधि॑ स॒धस्थे॑ स॒धस्थे॒ अधि॑ । \newline
35. स॒धस्थ॒ इति॑ स॒ध - स्थे॒ । \newline
36. अध् युत्त॑रस्मि॒न् नुत्त॑रस्मि॒न् नध्यध् युत्त॑रस्मिन्न् । \newline
37. उत्त॑रस्मि॒न्॒. विश्वे॒ विश्व॒ उत्त॑रस्मि॒न् नुत्त॑रस्मि॒न्॒. विश्वे᳚ । \newline
38. उत्त॑रस्मि॒न्नित्युत् - त॒र॒स्मि॒न् । \newline
39. विश्वे॑ देवा देवा॒ विश्वे॒ विश्वे॑ देवाः । \newline
40. दे॒वा॒ यज॑मानो॒ यज॑मानो देवा देवा॒ यज॑मानः । \newline
41. यज॑मानश्च च॒ यज॑मानो॒ यज॑मानश्च । \newline
42. च॒ सी॒द॒त॒ सी॒द॒त॒ च॒ च॒ सी॒द॒त॒ । \newline
43. सी॒द॒तेति॑ सीदत । \newline
44. येना॑ स॒हस्र(ग्म्॑) स॒हस्रं॒ ॅयेन॒ येना॑ स॒हस्र᳚म् । \newline
45. स॒हस्रं॒ ॅवह॑सि॒ वह॑सि स॒हस्र(ग्म्॑) स॒हस्रं॒ ॅवह॑सि । \newline
46. वह॑सि॒ येन॒ येन॒ वह॑सि॒ वह॑सि॒ येन॑ । \newline
47. येना᳚ग्ने अग्ने॒ येन॒ येना᳚ग्ने । \newline
48. अ॒ग्ने॒ स॒र्व॒वे॒द॒सꣳ स॑र्ववेद॒स म॑ग्ने अग्ने सर्ववेद॒सम् । \newline
49. स॒र्व॒वे॒द॒समिति॑ सर्व - वे॒द॒सम् । \newline
50. तेने॒ ममि॒मम् तेन॒ तेने॒मम् । \newline
51. इ॒मं ॅय॒ज्ञ्ं ॅय॒ज्ञ् मि॒म मि॒मं ॅय॒ज्ञ्म् । \newline
52. य॒ज्ञ्म् नो॑ नो य॒ज्ञ्ं ॅय॒ज्ञ्म् नः॑ । \newline
53. नो॒ व॒ह॒ व॒ह॒ नो॒ नो॒ व॒ह॒ । \newline
54. व॒ह॒ दे॒व॒यानो॑ देव॒यानो॑ वह वह देव॒यानः॑ । \newline
55. दे॒व॒यानो॒ यो यो दे॑व॒यानो॑ देव॒यानो॒ यः । \newline
56. दे॒व॒यान॒ इति॑ देव - यानः॑ । \newline
57. य उ॑त्त॒म उ॑त्त॒मो यो य उ॑त्त॒मः । \newline

\textbf{Ghana Paata } \newline

1. ये पृ॑त॒न्यवः॑ पृत॒न्यवो॒ ये ये पृ॑त॒न्यवः॑ । \newline
2. पृ॒त॒न्यव॒ इति॑ पृत॒न्यवः॑ । \newline
3. अ॒य म॒ग्नि र॒ग्नि र॒य म॒य म॒ग्निर् वी॒रत॑मो वी॒रत॑मो॒ ऽग्नि र॒य म॒य म॒ग्निर् वी॒रत॑मः । \newline
4. अ॒ग्निर् वी॒रत॑मो वी॒रत॑मो॒ ऽग्नि र॒ग्निर् वी॒रत॑मो वयो॒धा व॑यो॒धा वी॒रत॑मो॒ ऽग्नि र॒ग्निर् वी॒रत॑मो वयो॒धाः । \newline
5. वी॒रत॑मो वयो॒धा व॑यो॒धा वी॒रत॑मो वी॒रत॑मो वयो॒धाः स॑ह॒स्रियः॑ सह॒स्रियो॑ वयो॒धा वी॒रत॑मो वी॒रत॑मो वयो॒धाः स॑ह॒स्रियः॑ । \newline
6. वी॒रत॑म॒ इति॑ वी॒र - त॒मः॒ । \newline
7. व॒यो॒धाः स॑ह॒स्रियः॑ सह॒स्रियो॑ वयो॒धा व॑यो॒धाः स॑ह॒स्रियो॑ दीप्यताम् दीप्यताꣳ सह॒स्रियो॑ वयो॒धा व॑यो॒धाः स॑ह॒स्रियो॑ दीप्यताम् । \newline
8. व॒यो॒धा इति॑ वयः - धाः । \newline
9. स॒ह॒स्रियो॑ दीप्यताम् दीप्यताꣳ सह॒स्रियः॑ सह॒स्रियो॑ दीप्यता॒ मप्र॑युच्छ॒न् नप्र॑युच्छन् दीप्यताꣳ सह॒स्रियः॑ सह॒स्रियो॑ दीप्यता॒ मप्र॑युच्छन्न् । \newline
10. दी॒प्य॒ता॒ मप्र॑युच्छ॒न् नप्र॑युच्छन् दीप्यताम् दीप्यता॒ मप्र॑युच्छन्न् । \newline
11. अप्र॑युच्छ॒न्नित्यप्र॑ - यु॒च्छ॒न्न् । \newline
12. वि॒भ्राज॑मानः सरि॒रस्य॑ सरि॒रस्य॑ वि॒भ्राज॑मानो वि॒भ्राज॑मानः सरि॒रस्य॒ मद्ध्ये॒ मद्ध्ये॑ सरि॒रस्य॑ वि॒भ्राज॑मानो वि॒भ्राज॑मानः सरि॒रस्य॒ मद्ध्ये᳚ । \newline
13. वि॒भ्राज॑मान॒ इति॑ वि - भ्राज॑मानः । \newline
14. स॒रि॒रस्य॒ मद्ध्ये॒ मद्ध्ये॑ सरि॒रस्य॑ सरि॒रस्य॒ मद्ध्य॒ उपोप॒ मद्ध्ये॑ सरि॒रस्य॑ सरि॒रस्य॒ मद्ध्य॒ उप॑ । \newline
15. मद्ध्य॒ उपोप॒ मद्ध्ये॒ मद्ध्य॒ उप॒ प्र प्रोप॒ मद्ध्ये॒ मद्ध्य॒ उप॒ प्र । \newline
16. उप॒ प्र प्रोपोप॒ प्र या॑त यात॒ प्रोपोप॒ प्र या॑त । \newline
17. प्र या॑त यात॒ प्र प्र या॑त दि॒व्यानि॑ दि॒व्यानि॑ यात॒ प्र प्र या॑त दि॒व्यानि॑ । \newline
18. या॒त॒ दि॒व्यानि॑ दि॒व्यानि॑ यात यात दि॒व्यानि॒ धाम॒ धाम॑ दि॒व्यानि॑ यात यात दि॒व्यानि॒ धाम॑ । \newline
19. दि॒व्यानि॒ धाम॒ धाम॑ दि॒व्यानि॑ दि॒व्यानि॒ धाम॑ । \newline
20. धामेति॒ धाम॑ । \newline
21. सम् प्र प्र सꣳ सम् प्र च्य॑वद्ध्वम् च्यवद्ध्व॒म् प्र सꣳ सम् प्र च्य॑वद्ध्वम् । \newline
22. प्र च्य॑वद्ध्वम् च्यवद्ध्व॒म् प्र प्र च्य॑वद्ध्व॒ मन्वनु॑ च्यवद्ध्व॒म् प्र प्र च्य॑वद्ध्व॒ मनु॑ । \newline
23. च्य॒व॒द्ध्व॒ मन्वनु॑ च्यवद्ध्वम् च्यवद्ध्व॒ मनु॒ सꣳ स मनु॑ च्यवद्ध्वम् च्यवद्ध्व॒ मनु॒ सम् । \newline
24. अनु॒ सꣳ स मन्वनु॒ सम् प्र प्र स मन्वनु॒ सम् प्र । \newline
25. सम् प्र प्र सꣳ सम् प्र या॑त यात॒ प्र सꣳ सम् प्र या॑त । \newline
26. प्र या॑त यात॒ प्र प्र या॒ताग्ने ऽग्ने॑ यात॒ प्र प्र या॒ताग्ने᳚ । \newline
27. या॒ताग्ने ऽग्ने॑ यात या॒ताग्ने॑ प॒थः प॒थो ऽग्ने॑ यात या॒ताग्ने॑ प॒थः । \newline
28. अग्ने॑ प॒थः प॒थो ऽग्ने ऽग्ने॑ प॒थो दे॑व॒याना᳚न् देव॒याना᳚न् प॒थो ऽग्ने ऽग्ने॑ प॒थो दे॑व॒यानान्॑ । \newline
29. प॒थो दे॑व॒याना᳚न् देव॒याना᳚न् प॒थः प॒थो दे॑व॒याना᳚न् कृणुद्ध्वम् कृणुद्ध्वम् देव॒याना᳚न् प॒थः प॒थो दे॑व॒याना᳚न् कृणुद्ध्वम् । \newline
30. दे॒व॒याना᳚न् कृणुद्ध्वम् कृणुद्ध्वम् देव॒याना᳚न् देव॒याना᳚न् कृणुद्ध्वम् । \newline
31. दे॒व॒याना॒निति॑ देव - यानान्॑ । \newline
32. कृ॒णु॒द्ध्व॒मिति॑ कृणुद्ध्वम् । \newline
33. अ॒स्मिन् थ्स॒धस्थे॑ स॒धस्थे॑ अ॒स्मिन् न॒स्मिन् थ्स॒धस्थे॒ अध्यधि॑ स॒धस्थे॑ अ॒स्मिन् न॒स्मिन् थ्स॒धस्थे॒ अधि॑ । \newline
34. स॒धस्थे॒ अध्यधि॑ स॒धस्थे॑ स॒धस्थे॒ अध्युत्त॑रस्मि॒न् नुत्त॑रस्मि॒न् नधि॑ स॒धस्थे॑ स॒धस्थे॒ 
अध्युत्त॑रस्मिन्न् । \newline
35. स॒धस्थ॒ इति॑ स॒ध - स्थे॒ । \newline
36. अध् युत्त॑रस्मि॒न् नुत्त॑रस्मि॒न् नध्यध् युत्त॑रस्मि॒न्॒. विश्वे॒ विश्व॒ उत्त॑रस्मि॒न् नध्य ध्युत्त॑रस्मि॒न्॒. विश्वे᳚ । \newline
37. उत्त॑रस्मि॒न्॒. विश्वे॒ विश्व॒ उत्त॑रस्मि॒न् नुत्त॑रस्मि॒न्॒. विश्वे॑ देवा देवा॒ विश्व॒ उत्त॑रस्मि॒न् नुत्त॑रस्मि॒न्॒. विश्वे॑ देवाः । \newline
38. उत्त॑रस्मि॒न्नित्युत् - त॒र॒स्मि॒न् । \newline
39. विश्वे॑ देवा देवा॒ विश्वे॒ विश्वे॑ देवा॒ यज॑मानो॒ यज॑मानो देवा॒ विश्वे॒ विश्वे॑ देवा॒ यज॑मानः । \newline
40. दे॒वा॒ यज॑मानो॒ यज॑मानो देवा देवा॒ यज॑मानश्च च॒ यज॑मानो देवा देवा॒ यज॑मानश्च । \newline
41. यज॑मानश्च च॒ यज॑मानो॒ यज॑मानश्च सीदत सीदत च॒ यज॑मानो॒ यज॑मानश्च सीदत । \newline
42. च॒ सी॒द॒त॒ सी॒द॒त॒ च॒ च॒ सी॒द॒त॒ । \newline
43. सी॒द॒तेति॑ सीदत । \newline
44. येना॑ स॒हस्र(ग्म्॑) स॒हस्रं॒ ॅयेन॒ येना॑ स॒हस्रं॒ ॅवह॑सि॒ वह॑सि स॒हस्रं॒ ॅयेन॒ येना॑ स॒हस्रं॒ ॅवह॑सि । \newline
45. स॒हस्रं॒ ॅवह॑सि॒ वह॑सि स॒हस्र(ग्म्॑) स॒हस्रं॒ ॅवह॑सि॒ येन॒ येन॒ वह॑सि स॒हस्र(ग्म्॑) स॒हस्रं॒ ॅवह॑सि॒ येन॑ । \newline
46. वह॑सि॒ येन॒ येन॒ वह॑सि॒ वह॑सि॒ येना᳚ग्ने अग्ने॒ येन॒ वह॑सि॒ वह॑सि॒ येना᳚ग्ने । \newline
47. येना᳚ग्ने अग्ने॒ येन॒ येना᳚ग्ने सर्ववेद॒सꣳ स॑र्ववेद॒स म॑ग्ने॒ येन॒ येना᳚ग्ने सर्ववेद॒सम् । \newline
48. अ॒ग्ने॒ स॒र्व॒वे॒द॒सꣳ स॑र्ववेद॒स म॑ग्ने अग्ने सर्ववेद॒सम् । \newline
49. स॒र्व॒वे॒द॒समिति॑ सर्व - वे॒द॒सम् । \newline
50. तेने॒म मि॒मम् तेन॒ तेने॒मं ॅय॒ज्ञ्ं ॅय॒ज्ञ् मि॒मम् तेन॒ तेने॒मं ॅय॒ज्ञ्म् । \newline
51. इ॒मं ॅय॒ज्ञ्ं ॅय॒ज्ञ् मि॒म मि॒मं ॅय॒ज्ञ्म् नो॑ नो य॒ज्ञ् मि॒म मि॒मं ॅय॒ज्ञ्म् नः॑ । \newline
52. य॒ज्ञ्म् नो॑ नो य॒ज्ञ्ं ॅय॒ज्ञ्म् नो॑ वह वह नो य॒ज्ञ्ं ॅय॒ज्ञ्म् नो॑ वह । \newline
53. नो॒ व॒ह॒ व॒ह॒ नो॒ नो॒ व॒ह॒ दे॒व॒यानो॑ देव॒यानो॑ वह नो नो वह देव॒यानः॑ । \newline
54. व॒ह॒ दे॒व॒यानो॑ देव॒यानो॑ वह वह देव॒यानो॒ यो यो दे॑व॒यानो॑ वह वह देव॒यानो॒ यः । \newline
55. दे॒व॒यानो॒ यो यो दे॑व॒यानो॑ देव॒यानो॒ य उ॑त्त॒म उ॑त्त॒मो यो दे॑व॒यानो॑ देव॒यानो॒ य उ॑त्त॒मः । \newline
56. दे॒व॒यान॒ इति॑ देव - यानः॑ । \newline
57. य उ॑त्त॒म उ॑त्त॒मो यो य उ॑त्त॒मः । \newline
\pagebreak
\markright{ TS 4.7.13.5  \hfill https://www.vedavms.in \hfill}

\section{ TS 4.7.13.5 }

\textbf{TS 4.7.13.5 } \newline
\textbf{Samhita Paata} \newline

उ॑त्त॒मः ॥ उद्-बु॑द्ध्यस्वाग्ने॒ प्रति॑ जागृह्येन मिष्टापू॒र्ते सꣳ सृ॑जेथाम॒यं च॑ । पुनः॑ कृ॒ण्वꣳस्त्वा॑ पि॒तरं॒ ॅयुवा॑न-म॒न्वाताꣳ॑सी॒त् त्वयि॒ तन्तु॑मे॒तं ॥ अ॒यं ते॒ योनि॑र्.ऋ॒त्वियो॒ यतो॑ जा॒तो अरो॑चथाः । तं जा॒नन्न॑ग्न॒ आ रो॒हाथा॑ नो वर्द्धया र॒यिं ॥ \newline

\textbf{Pada Paata} \newline

उ॒त्त॒म इत्यु॑त्-त॒मः ॥ उदिति॑ । बु॒द्ध्य॒स्व॒ । अ॒ग्ने॒ । प्रतीति॑ । जा॒गृ॒हि॒ । ए॒न॒म् । इ॒ष्टा॒पू॒र्ते इती᳚ष्टा - पू॒र्ते । समिति॑ । सृ॒जे॒था॒म् । अ॒यम् । च॒ ॥ पुनः॑ । कृ॒ण्वन्न् । त्वा॒ । पि॒तर᳚म् । युवा॑नम् । अ॒न्वाताꣳ॑सी॒दित्य॑नु-आताꣳ॑सीत् । त्वयि॑ । तन्तु᳚म् । ए॒तम् ॥ अ॒यम् । ते॒ । योनिः॑ । ऋ॒त्वियः॑ । यतः॑ । जा॒तः । अरो॑चथाः ॥ तम् । जा॒नन्न् । अ॒ग्ने॒ । एति॑ । रो॒ह॒ । अथ॑ । नः॒ । व॒द्‌र्ध॒य॒ । र॒यिम् ॥  \newline


\textbf{Krama Paata} \newline

उ॒त्त॒म इत्यु॑त् - त॒मः । उद् बु॑द्ध्यस्व । बु॒द्ध्य॒स्वा॒ग्ने॒ । अ॒ग्ने॒ प्रति॑ । प्रति॑ जागृहि । जा॒गृ॒ह्ये॒न॒म् । ए॒न॒मि॒ष्टा॒पू॒र्ते । इ॒ष्टा॒पू॒र्ते सम् । इ॒ष्टा॒पू॒र्ते इती᳚ष्टा - पू॒र्ते । सꣳ सृ॑जेथाम् । सृ॒जे॒था॒म॒यम् । अ॒यम् च॑ । चेति॑ च ॥ पुनः॑ कृ॒ण्वन्न् । कृ॒ण्वꣳस्त्वा᳚ । त्वा॒ पि॒तर᳚म् । पि॒तर॒म् ॅयुवा॑नम् । युवा॑नम॒न्वाताꣳ॑सीत् । अ॒न्वाताꣳ॑सी॒त् त्वयि॑ । अ॒न्वाताꣳ॑सी॒दित्य॑नु - आताꣳ॑सीत् । त्वयि॒ तन्तु᳚म् । तन्तु॑मे॒तम् । ए॒तमित्ये॒तम् ॥ अ॒यम् ते᳚ । ते॒ योनिः॑ । योनि॑र्. ऋ॒त्वियः॑ । ऋ॒त्वियो॒ यतः॑ । यतो॑ जा॒तः । जा॒तो अरो॑चथाः । अरो॑चथा॒ इत्यरो॑चथाः ॥ तम् जा॒नन्न् । जा॒नन्न॑ग्ने । अ॒ग्न॒ आ । आ रो॑ह । रो॒हाथ॑ । अथा॑ नः । नो॒ व॒र्द्ध॒य॒ । व॒र्द्ध॒या॒ र॒यिम् । र॒यिमिति॑ र॒यिम् । \newline

\textbf{Jatai Paata} \newline

1. उ॒त्त॒म इत्यु॑त् - त॒मः । \newline
2. उद् बु॑द्ध्यस्व बुद्ध्य॒स्वो दुद् बु॑द्ध्यस्व । \newline
3. बु॒द्ध्य॒स्वा॒ग्ने॒ अ॒ग्ने॒ बु॒द्ध्य॒स्व॒ बु॒द्ध्य॒स्वा॒ग्ने॒ । \newline
4. अ॒ग्ने॒ प्रति॒ प्रत्य॑ग्ने अग्ने॒ प्रति॑ । \newline
5. प्रति॑ जागृहि जागृहि॒ प्रति॒ प्रति॑ जागृहि । \newline
6. जा॒गृ॒ह्ये॒न॒ मे॒न॒म् जा॒गृ॒हि॒ जा॒गृ॒ह्ये॒न॒म् । \newline
7. ए॒न॒ मि॒ष्टा॒पू॒र्ते इ॑ष्टापू॒र्ते ए॑न मेन मिष्टापू॒र्ते । \newline
8. इ॒ष्टा॒पू॒र्ते सꣳ स मि॑ष्टापू॒र्ते इ॑ष्टापू॒र्ते सम् । \newline
9. इ॒ष्टा॒पू॒र्ते इती᳚ष्टा - पू॒र्ते । \newline
10. सꣳ सृ॑जेथाꣳ सृजेथा॒(ग्म्॒) सꣳ सꣳ सृ॑जेथाम् । \newline
11. सृ॒जे॒था॒ म॒य म॒यꣳ सृ॑जेथाꣳ सृजेथा म॒यम् । \newline
12. अ॒यम् च॑ चा॒य म॒यम् च॑ । \newline
13. चेति॑ च । \newline
14. पुनः॑ कृ॒ण्वन् कृ॒ण्वन् पुनः॒ पुनः॑ कृ॒ण्वन्न् । \newline
15. कृ॒ण्वꣳ त्वा᳚ त्वा कृ॒ण्वन् कृ॒ण्वꣳ त्वा᳚ । \newline
16. त्वा॒ पि॒तर॑म् पि॒तर॑म् त्वा त्वा पि॒तर᳚म् । \newline
17. पि॒तरं॒ ॅयुवा॑नं॒ ॅयुवा॑नम् पि॒तर॑म् पि॒तरं॒ ॅयुवा॑नम् । \newline
18. युवा॑न म॒न्वाता(ग्म्॑)सी द॒न्वाता(ग्म्॑)सी॒द् युवा॑नं॒ ॅयुवा॑न म॒न्वाता(ग्म्॑)सीत् । \newline
19. अ॒न्वाता(ग्म्॑)सी॒त् त्वयि॒ त्वय्य॒न्वाता(ग्म्॑)सी द॒न्वाता(ग्म्॑)सी॒त् त्वयि॑ । \newline
20. अ॒न्वाता(ग्म्॑)सी॒दित्य॑नु - आता(ग्म्॑)सीत् । \newline
21. त्वयि॒ तन्तु॒म् तन्तु॒म् त्वयि॒ त्वयि॒ तन्तु᳚म् । \newline
22. तन्तु॑ मे॒त मे॒तम् तन्तु॒म् तन्तु॑ मे॒तम् । \newline
23. ए॒तमित्ये॒तम् । \newline
24. अ॒यम् ते॑ ते॒ ऽय म॒यम् ते᳚ । \newline
25. ते॒ योनि॒र् योनि॑ स्ते ते॒ योनिः॑ । \newline
26. योनि॑र्. ऋ॒त्विय॑ ऋ॒त्वियो॒ योनि॒र् योनि॑र्. ऋ॒त्वियः॑ । \newline
27. ऋ॒त्वियो॒ यतो॒ यत॑ ऋ॒त्विय॑ ऋ॒त्वियो॒ यतः॑ । \newline
28. यतो॑ जा॒तो जा॒तो यतो॒ यतो॑ जा॒तः । \newline
29. जा॒तो अरो॑चथा॒ अरो॑चथा जा॒तो जा॒तो अरो॑चथाः । \newline
30. अरो॑चथा॒ इत्यरो॑चथाः । \newline
31. तम् जा॒नन् जा॒नन् तम् तम् जा॒नन्न् । \newline
32. जा॒नन् न॑ग्ने अग्ने जा॒नन् जा॒नन् न॑ग्ने । \newline
33. अ॒ग्न॒ आ ऽग्ने॑ अग्न॒ आ । \newline
34. आ रो॑ह रो॒हा रो॑ह । \newline
35. रो॒हाथाथ॑ रोह रो॒हाथ॑ । \newline
36. अथा॑ नो नो॒ अथाथा॑ नः । \newline
37. नो॒ व॒र्द्ध॒य॒ व॒र्द्ध॒य॒ नो॒ नो॒ व॒र्द्ध॒य॒ । \newline
38. व॒र्द्ध॒या॒ र॒यिꣳ र॒यिं ॅव॑र्द्धय वर्द्धया र॒यिम् । \newline
39. र॒यिमिति॑ र॒यिम् । \newline

\textbf{Ghana Paata } \newline

1. उ॒त्त॒म इत्यु॑त् - त॒मः । \newline
2. उद् बु॑द्ध्यस्व बुद्ध्य॒स्वो दुद् बु॑द्ध्यस्वाग्ने अग्ने बुद्ध्य॒स्वो दुद् बु॑द्ध्यस्वाग्ने । \newline
3. बु॒द्ध्य॒स्वा॒ग्ने॒ अ॒ग्ने॒ बु॒द्ध्य॒स्व॒ बु॒द्ध्य॒स्वा॒ग्ने॒ प्रति॒ प्रत्य॑ग्ने बुद्ध्यस्व बुद्ध्यस्वाग्ने॒ प्रति॑ । \newline
4. अ॒ग्ने॒ प्रति॒ प्रत्य॑ग्ने अग्ने॒ प्रति॑ जागृहि जागृहि॒ प्रत्य॑ग्ने अग्ने॒ प्रति॑ जागृहि । \newline
5. प्रति॑ जागृहि जागृहि॒ प्रति॒ प्रति॑ जागृह्येन मेनम् जागृहि॒ प्रति॒ प्रति॑ जागृह्येनम् । \newline
6. जा॒गृ॒ह्ये॒न॒ मे॒न॒म् जा॒गृ॒हि॒ जा॒गृ॒ह्ये॒न॒ मि॒ष्टा॒पू॒र्ते इ॑ष्टापू॒र्ते ए॑नम् जागृहि जागृह्येन मिष्टापू॒र्ते । \newline
7. ए॒न॒ मि॒ष्टा॒पू॒र्ते इ॑ष्टापू॒र्ते ए॑न मेन मिष्टापू॒र्ते सꣳ स मि॑ष्टापू॒र्ते ए॑न मेन मिष्टापू॒र्ते सम् । \newline
8. इ॒ष्टा॒पू॒र्ते सꣳ स मि॑ष्टापू॒र्ते इ॑ष्टापू॒र्ते सꣳ सृ॑जेथाꣳ सृजेथा॒(ग्म्॒) स मि॑ष्टापू॒र्ते इ॑ष्टापू॒र्ते सꣳ सृ॑जेथाम् । \newline
9. इ॒ष्टा॒पू॒र्ते इती᳚ष्टा - पू॒र्ते । \newline
10. सꣳ सृ॑जेथाꣳ सृजेथा॒(ग्म्॒) सꣳ सꣳ सृ॑जेथा म॒य म॒यꣳ सृ॑जेथा॒(ग्म्॒) सꣳ सꣳ सृ॑जेथा म॒यम् । \newline
11. सृ॒जे॒था॒ म॒य म॒यꣳ सृ॑जेथाꣳ सृजेथा म॒यम् च॑ चा॒यꣳ सृ॑जेथाꣳ सृजेथा म॒यम् च॑ । \newline
12. अ॒यम् च॑ चा॒य म॒यम् च॑ । \newline
13. चेति॑ च । \newline
14. पुनः॑ कृ॒ण्वन् कृ॒ण्वन् पुनः॒ पुनः॑ कृ॒ण्वꣳ स्त्वा᳚ त्वा कृ॒ण्वन् पुनः॒ पुनः॑ कृ॒ण्वꣳ स्त्वा᳚ । \newline
15. कृ॒ण्वꣳ स्त्वा᳚ त्वा कृ॒ण्वन् कृ॒ण्वꣳ स्त्वा॑ पि॒तर॑म् पि॒तर॑म् त्वा कृ॒ण्वन् कृ॒ण्वꣳ स्त्वा॑ पि॒तर᳚म् । \newline
16. त्वा॒ पि॒तर॑म् पि॒तर॑म् त्वा त्वा पि॒तरं॒ ॅयुवा॑नं॒ ॅयुवा॑नम् पि॒तर॑म् त्वा त्वा पि॒तरं॒ ॅयुवा॑नम् । \newline
17. पि॒तरं॒ ॅयुवा॑नं॒ ॅयुवा॑नम् पि॒तर॑म् पि॒तरं॒ ॅयुवा॑न म॒न्वाता(ग्म्॑)सी द॒न्वाता(ग्म्॑)सी॒द् युवा॑नम् पि॒तर॑म् पि॒तरं॒ ॅयुवा॑न म॒न्वाता(ग्म्॑)सीत् । \newline
18. युवा॑न म॒न्वाता(ग्म्॑)सी द॒न्वाता(ग्म्॑)सी॒द् युवा॑नं॒ ॅयुवा॑न म॒न्वाता(ग्म्॑)सी॒त् त्वयि॒ त्वय्य॒न्वाता(ग्म्॑)सी॒द् युवा॑नं॒ ॅयुवा॑न म॒न्वाता(ग्म्॑)सी॒त् त्वयि॑ । \newline
19. अ॒न्वाता(ग्म्॑)सी॒त् त्वयि॒ त्वय्य॒न्वाता(ग्म्॑)सी द॒न्वाता(ग्म्॑)सी॒त् त्वयि॒ तन्तु॒म् तन्तु॒म् त्वय्य॒न्वाता(ग्म्॑)सी द॒न्वाता(ग्म्॑)सी॒त् त्वयि॒ तन्तु᳚म् । \newline
20. अ॒न्वाता(ग्म्॑)सी॒दित्य॑नु - आता(ग्म्॑)सीत् । \newline
21. त्वयि॒ तन्तु॒म् तन्तु॒म् त्वयि॒ त्वयि॒ तन्तु॑ मे॒त मे॒तम् तन्तु॒म् त्वयि॒ त्वयि॒ तन्तु॑ मे॒तम् । \newline
22. तन्तु॑ मे॒त मे॒तम् तन्तु॒म् तन्तु॑ मे॒तम् । \newline
23. ए॒तमित्ये॒तम् । \newline
24. अ॒यम् ते॑ ते॒ ऽय म॒यम् ते॒ योनि॒र् योनि॑ स्ते॒ ऽय म॒यम् ते॒ योनिः॑ । \newline
25. ते॒ योनि॒र् योनि॑ स्ते ते॒ योनि॑र्. ऋ॒त्विय॑ ऋ॒त्वियो॒ योनि॑ स्ते ते॒ योनि॑र्. ऋ॒त्वियः॑ । \newline
26. योनि॑र्. ऋ॒त्विय॑ ऋ॒त्वियो॒ योनि॒र् योनि॑र्. ऋ॒त्वियो॒ यतो॒ यत॑ ऋ॒त्वियो॒ योनि॒र् योनि॑र्. ऋ॒त्वियो॒ यतः॑ । \newline
27. ऋ॒त्वियो॒ यतो॒ यत॑ ऋ॒त्विय॑ ऋ॒त्वियो॒ यतो॑ जा॒तो जा॒तो यत॑ ऋ॒त्विय॑ ऋ॒त्वियो॒ यतो॑ जा॒तः । \newline
28. यतो॑ जा॒तो जा॒तो यतो॒ यतो॑ जा॒तो अरो॑चथा॒ अरो॑चथा जा॒तो यतो॒ यतो॑ जा॒तो अरो॑चथाः । \newline
29. जा॒तो अरो॑चथा॒ अरो॑चथा जा॒तो जा॒तो अरो॑चथाः । \newline
30. अरो॑चथा॒ इत्यरो॑चथाः । \newline
31. तम् जा॒नन् जा॒नन् तम् तम् जा॒नन् न॑ग्ने अग्ने जा॒नन् तम् तम् जा॒नन् न॑ग्ने । \newline
32. जा॒नन् न॑ग्ने अग्ने जा॒नन् जा॒नन् न॑ग्न॒ आ ऽग्ने॑ जा॒नन् जा॒नन् न॑ग्न॒ आ । \newline
33. अ॒ग्न॒ आ ऽग्ने॑ अग्न॒ आ रो॑ह रो॒हा ऽग्ने॑ अग्न॒ आ रो॑ह । \newline
34. आ रो॑ह रो॒हा रो॒हाथाथ॑ रो॒हा रो॒हाथ॑ । \newline
35. रो॒हाथाथ॑ रोह रो॒हाथा॑ नो नो॒ अथ॑ रोह रो॒हाथा॑ नः । \newline
36. अथा॑ नो नो॒ अथाथा॑ नो वर्द्धय वर्द्धय नो॒ अथाथा॑ नो वर्द्धय । \newline
37. नो॒ व॒र्द्ध॒य॒ व॒र्द्ध॒य॒ नो॒ नो॒ व॒र्द्ध॒या॒ र॒यिꣳ र॒यिं ॅव॑र्द्धय नो नो वर्द्धया र॒यिम् । \newline
38. व॒र्द्ध॒या॒ र॒यिꣳ र॒यिं ॅव॑र्द्धय वर्द्धया र॒यिम् । \newline
39. र॒यिमिति॑ र॒यिम् । \newline
\pagebreak
\markright{ TS 4.7.14.1  \hfill https://www.vedavms.in \hfill}

\section{ TS 4.7.14.1 }

\textbf{TS 4.7.14.1 } \newline
\textbf{Samhita Paata} \newline

ममा᳚ग्ने॒ वर्चो॑ विह॒वेष्व॑स्तु व॒यं त्वेन्धा॑ना स्त॒नुवं॑ पुषेम । मह्यं॑ नमन्तां प्र॒दिश॒श्चत॑स्र॒ स्त्वया-ऽद्ध्य॑क्षेण॒ पृत॑ना जयेम ॥ मम॑ दे॒वा वि॑ह॒वे स॑न्तु॒ सर्व॒ इन्द्रा॑वन्तो म॒रुतो॒ विष्णु॑र॒ग्निः । ममा॒न्तरि॑क्ष मु॒रु गो॒पम॑स्तु॒ मह्यं॒ ॅवातः॑ पवतां॒ कामे॑ अ॒स्मिन्न् ॥मयि॑ दे॒वा द्रवि॑ण॒ माय॑जन्तां॒ मय्या॒ शीर॑स्तु॒ मयि॑ दे॒वहू॑तिः । दैव्या॒ होता॑रा वनिषन्त॒ - [  ] \newline

\textbf{Pada Paata} \newline

मम॑ । अ॒ग्ने॒ । वर्चः॑ । वि॒ह॒वेष्विति॑ वि - ह॒वेषु॑ । अ॒स्तु॒ । व॒यम् । त्वा॒ । इन्धा॑नाः । त॒नुव᳚म् । पु॒षे॒म॒ ॥ मह्य᳚म् । न॒म॒न्ता॒म् । प्र॒दिश॒ इति॑ प्र - दिशः॑ । चत॑स्रः । त्वया᳚ । अद्ध्य॑क्षे॒णेत्यधि॑ - अ॒क्षे॒ण॒ । पृत॑नाः । ज॒ये॒म॒ ॥ मम॑ । दे॒वाः । वि॒ह॒व इति॑ वि - ह॒वे । स॒न्तु॒ । सर्वे᳚ । इन्द्रा॑वन्त॒ इतीन्द्र॑ - व॒न्तः॒ । म॒रुतः॑ । विष्णुः॑ । अ॒ग्निः ॥ मम॑ । अ॒न्तरि॑क्षम् । उ॒रु । गो॒पम् । अ॒स्तु॒ । मह्य᳚म् । वातः॑ । प॒व॒ता॒म् । कामे᳚ । अ॒स्मिन्न् ॥ मयि॑ । दे॒वाः । द्रवि॑णम् । एति॑ । य॒ज॒न्ता॒म् । मयि॑ । आ॒शीरित्या᳚ - शीः । अ॒स्तु॒ । मयि॑ । दे॒वहू॑ति॒रिति॑ दे॒व - हू॒तिः॒ ॥ दैव्या᳚ । होता॑रा । व॒नि॒ष॒न्त॒ ।  \newline


\textbf{Krama Paata} \newline

ममा᳚ग्ने । अ॒ग्ने॒ वर्चः॑ । वर्चो॑ विह॒वेषु॑ । वि॒ह॒वेष्व॑स्तु । वि॒ह॒वेष्विति॑ वि - ह॒वेषु॑ । अ॒स्तु॒ व॒यम् । व॒यम् त्वा᳚ । त्वेन्धा॑नाः । इन्धा॑ना स्त॒नुव᳚म् । त॒नुव॑म् पुषेम । पु॒षे॒मेति॑ पुषेम ॥ मह्य॑म् नमन्ताम् । न॒म॒न्ता॒म् प्र॒दिशः॑ । प्र॒दिश॒श्चत॑स्रः । प्र॒दिश॒ इति॑ प्र - दिशः॑ । चत॑स्र॒स्त्वया᳚ । त्वयाऽद्ध्य॑क्षेण । अद्ध्य॑क्षेण॒ पृत॑नाः । अद्ध्य॑क्षे॒णेत्यधि॑ - अ॒क्षे॒ण॒ । पृत॑ना जयेम । ज॒ये॒मेति॑ जयेम ॥ मम॑ दे॒वाः । दे॒वा वि॑ह॒वे । वि॒ह॒वे स॑न्तु । वि॒ह॒व इति॑ वि - ह॒वे । स॒न्तु॒ सर्वे᳚ । सर्व॒ इन्द्रा॑वन्तः । इन्द्रा॑वन्तो म॒रुतः॑ । इन्द्रा॑वन्त॒ इतीन्द्र॑ - व॒न्तः॒ । म॒रुतो॒ विष्णुः॑ । विष्णु॑र॒ग्निः । अ॒ग्निरित्य॒ग्निः ॥ ममा॒न्तरि॑क्षम् । अ॒न्तरि॑क्षमु॒रु । उ॒रु गो॒पम् । गो॒पम॑स्तु । अ॒स्तु॒ मह्य᳚म् । मह्य॒म् ॅवातः॑ । वातः॑ पवताम् । प॒व॒ता॒म् कामे᳚ । कामे॑ अ॒स्मिन्न् । अ॒स्मिन्नित्य॒स्मिन्न् ॥ मयि॑ दे॒वाः । दे॒वा द्रवि॑णम् । द्रवि॑ण॒मा । आ य॑जन्ताम् । य॒ज॒न्ता॒म् मयि॑ । मय्या॒शीः । आ॒शीर॑स्तु । आ॒शीरित्या᳚ - शीः । अ॒स्तु॒ मयि॑ । मयि॑ दे॒वहू॑तिः । दे॒वहू॑ति॒रिति॑ दे॒व - हू॒तिः॒ ॥ दैव्या॒ होता॑रा । होता॑रा वनिषन्त । व॒नि॒ष॒न्त॒ पूर्वे᳚ \newline

\textbf{Jatai Paata} \newline

1. ममा᳚ग्ने अग्ने॒ मम॒ ममा᳚ग्ने । \newline
2. अ॒ग्ने॒ वर्चो॒ वर्चो॑ अग्ने अग्ने॒ वर्चः॑ । \newline
3. वर्चो॑ विह॒वेषु॑ विह॒वेषु॒ वर्चो॒ वर्चो॑ विह॒वेषु॑ । \newline
4. वि॒ह॒वे ष्व॑स्त्वस्तु विह॒वेषु॑ विह॒वे ष्व॑स्तु । \newline
5. वि॒ह॒वेष्विति॑ वि - ह॒वेषु॑ । \newline
6. अ॒स्तु॒ व॒यं ॅव॒य म॑स्त्वस्तु व॒यम् । \newline
7. व॒यम् त्वा᳚ त्वा व॒यं ॅव॒यम् त्वा᳚ । \newline
8. त्वेन्धा॑ना॒ इन्धा॑ना स्त्वा॒ त्वेन्धा॑नाः । \newline
9. इन्धा॑ना स्त॒नुव॑म् त॒नुव॒ मिन्धा॑ना॒ इन्धा॑ना स्त॒नुव᳚म् । \newline
10. त॒नुव॑म् पुषेम पुषेम त॒नुव॑म् त॒नुव॑म् पुषेम । \newline
11. पु॒षे॒मेति॑ पुषेम । \newline
12. मह्य॑म् नमन्ताम् नमन्ता॒म् मह्य॒म् मह्य॑म् नमन्ताम् । \newline
13. न॒म॒न्ता॒म् प्र॒दिशः॑ प्र॒दिशो॑ नमन्ताम् नमन्ताम् प्र॒दिशः॑ । \newline
14. प्र॒दिश॒ श्चत॑स्र॒ श्चत॑स्रः प्र॒दिशः॑ प्र॒दिश॒ श्चत॑स्रः । \newline
15. प्र॒दिश॒ इति॑ प्र - दिशः॑ । \newline
16. चत॑स्र॒ स्त्वया॒ त्वया॒ चत॑स्र॒ श्चत॑स्र॒ स्त्वया᳚ । \newline
17. त्वया ऽद्ध्य॑क्षे॒णा द्ध्य॑क्षेण॒ त्वया॒ त्वया ऽद्ध्य॑क्षेण । \newline
18. अद्ध्य॑क्षेण॒ पृत॑नाः॒ पृत॑ना॒ अद्ध्य॑क्षे॒णा द्ध्य॑क्षेण॒ पृत॑नाः । \newline
19. अद्ध्य॑क्षे॒णेत्यधि॑ - अ॒क्षे॒ण॒ । \newline
20. पृत॑ना जयेम जयेम॒ पृत॑नाः॒ पृत॑ना जयेम । \newline
21. ज॒ये॒मेति॑ जयेम । \newline
22. मम॑ दे॒वा दे॒वा मम॒ मम॑ दे॒वाः । \newline
23. दे॒वा वि॑ह॒वे वि॑ह॒वे दे॒वा दे॒वा वि॑ह॒वे । \newline
24. वि॒ह॒वे स॑न्तु सन्तु विह॒वे वि॑ह॒वे स॑न्तु । \newline
25. वि॒ह॒व इति॑ वि - ह॒वे । \newline
26. स॒न्तु॒ सर्वे॒ सर्वे॑ सन्तु सन्तु॒ सर्वे᳚ । \newline
27. सर्व॒ इन्द्रा॑वन्त॒ इन्द्रा॑वन्तः॒ सर्वे॒ सर्व॒ इन्द्रा॑वन्तः । \newline
28. इन्द्रा॑वन्तो म॒रुतो॑ म॒रुत॒ इन्द्रा॑वन्त॒ इन्द्रा॑वन्तो म॒रुतः॑ । \newline
29. इन्द्रा॑वन्त॒ इतीन्द्र॑ - व॒न्तः॒ । \newline
30. म॒रुतो॒ विष्णु॒र् विष्णु॑र् म॒रुतो॑ म॒रुतो॒ विष्णुः॑ । \newline
31. विष्णु॑ र॒ग्नि र॒ग्निर् विष्णु॒र् विष्णु॑ र॒ग्निः । \newline
32. अ॒ग्निरित्य॒ग्निः । \newline
33. ममा॒न्तरि॑क्ष म॒न्तरि॑क्ष॒म् मम॒ ममा॒न्तरि॑क्षम् । \newline
34. अ॒न्तरि॑क्ष मु॒रू᳚(1॒)र्व॑न्तरि॑क्ष म॒न्तरि॑क्ष मु॒रु । \newline
35. उ॒रु गो॒पम् गो॒प मु॒रू॑रु गो॒पम् । \newline
36. गो॒प म॑स्त्वस्तु गो॒पम् गो॒प म॑स्तु । \newline
37. अ॒स्तु॒ मह्य॒म् मह्य॑ मस्त्वस्तु॒ मह्य᳚म् । \newline
38. मह्यं॒ ॅवातो॒ वातो॒ मह्य॒म् मह्यं॒ ॅवातः॑ । \newline
39. वातः॑ पवताम् पवतां॒ ॅवातो॒ वातः॑ पवताम् । \newline
40. प॒व॒ता॒म् कामे॒ कामे॑ पवताम् पवता॒म् कामे᳚ । \newline
41. कामे॑ अ॒स्मिन् न॒स्मिन् कामे॒ कामे॑ अ॒स्मिन्न् । \newline
42. अ॒स्मिन्नित्य॒स्मिन्न् । \newline
43. मयि॑ दे॒वा दे॒वा मयि॒ मयि॑ दे॒वाः । \newline
44. दे॒वा द्रवि॑ण॒म् द्रवि॑णम् दे॒वा दे॒वा द्रवि॑णम् । \newline
45. द्रवि॑ण॒ मा द्रवि॑ण॒म् द्रवि॑ण॒ मा । \newline
46. आ य॑जन्तां ॅयजन्ता॒ मा य॑जन्ताम् । \newline
47. य॒ज॒न्ता॒म् मयि॒ मयि॑ यजन्तां ॅयजन्ता॒म् मयि॑ । \newline
48. मय्या॒शी रा॒शीर् मयि॒ मय्या॒शीः । \newline
49. आ॒शी र॑स्त्व स्त्वा॒शी रा॒शी र॑स्तु । \newline
50. आ॒शीरित्या᳚ - शीः । \newline
51. अ॒स्तु॒ मयि॒ मय्य॑ स्त्वस्तु॒ मयि॑ । \newline
52. मयि॑ दे॒वहू॑तिर् दे॒वहू॑ति॒र् मयि॒ मयि॑ दे॒वहू॑तिः । \newline
53. दे॒वहू॑ति॒रिति॑ दे॒व - हू॒तिः॒ । \newline
54. दैव्या॒ होता॑रा॒ होता॑रा॒ दैव्या॒ दैव्या॒ होता॑रा । \newline
55. होता॑रा वनिषन्त वनिषन्त॒ होता॑रा॒ होता॑रा वनिषन्त । \newline
56. व॒नि॒ष॒न्त॒ पूर्वे॒ पूर्वे॑ वनिषन्त वनिषन्त॒ पूर्वे᳚ । \newline

\textbf{Ghana Paata } \newline

1. ममा᳚ग्ने अग्ने॒ मम॒ ममा᳚ग्ने॒ वर्चो॒ वर्चो॑ अग्ने॒ मम॒ ममा᳚ग्ने॒ वर्चः॑ । \newline
2. अ॒ग्ने॒ वर्चो॒ वर्चो॑ अग्ने अग्ने॒ वर्चो॑ विह॒वेषु॑ विह॒वेषु॒ वर्चो॑ अग्ने अग्ने॒ वर्चो॑ विह॒वेषु॑ । \newline
3. वर्चो॑ विह॒वेषु॑ विह॒वेषु॒ वर्चो॒ वर्चो॑ विह॒वे ष्व॑स्त्वस्तु विह॒वेषु॒ वर्चो॒ वर्चो॑ विह॒वे ष्व॑स्तु । \newline
4. वि॒ह॒वे ष्व॑स्त्वस्तु विह॒वेषु॑ विह॒वे ष्व॑स्तु व॒यं ॅव॒य म॑स्तु विह॒वेषु॑ विह॒वे ष्व॑स्तु व॒यम् । \newline
5. वि॒ह॒वेष्विति॑ वि - ह॒वेषु॑ । \newline
6. अ॒स्तु॒ व॒यं ॅव॒य म॑स्त्वस्तु व॒यम् त्वा᳚ त्वा व॒य म॑स्त्वस्तु व॒यम् त्वा᳚ । \newline
7. व॒यम् त्वा᳚ त्वा व॒यं ॅव॒यम् त्वेन्धा॑ना॒ इन्धा॑ना स्त्वा व॒यं ॅव॒यम् त्वेन्धा॑नाः । \newline
8. त्वेन्धा॑ना॒ इन्धा॑ना स्त्वा॒ त्वेन्धा॑ना स्त॒नुव॑म् त॒नुव॒ मिन्धा॑ना स्त्वा॒ त्वेन्धा॑ना स्त॒नुव᳚म् । \newline
9. इन्धा॑ना स्त॒नुव॑म् त॒नुव॒ मिन्धा॑ना॒ इन्धा॑ना स्त॒नुव॑म् पुषेम पुषेम त॒नुव॒ मिन्धा॑ना॒ इन्धा॑ना स्त॒नुव॑म् पुषेम । \newline
10. त॒नुव॑म् पुषेम पुषेम त॒नुव॑म् त॒नुव॑म् पुषेम । \newline
11. पु॒षे॒मेति॑ पुषेम । \newline
12. मह्य॑म् नमन्ताम् नमन्ता॒म् मह्य॒म् मह्य॑म् नमन्ताम् प्र॒दिशः॑ प्र॒दिशो॑ नमन्ता॒म् मह्य॒म् मह्य॑म् नमन्ताम् प्र॒दिशः॑ । \newline
13. न॒म॒न्ता॒म् प्र॒दिशः॑ प्र॒दिशो॑ नमन्ताम् नमन्ताम् प्र॒दिश॒ श्चत॑स्र॒ श्चत॑स्रः प्र॒दिशो॑ नमन्ताम् नमन्ताम् प्र॒दिश॒ श्चत॑स्रः । \newline
14. प्र॒दिश॒ श्चत॑स्र॒ श्चत॑स्रः प्र॒दिशः॑ प्र॒दिश॒ श्चत॑स्र॒ स्त्वया॒ त्वया॒ चत॑स्रः प्र॒दिशः॑ प्र॒दिश॒ श्चत॑स्र॒ स्त्वया᳚ । \newline
15. प्र॒दिश॒ इति॑ प्र - दिशः॑ । \newline
16. चत॑स्र॒ स्त्वया॒ त्वया॒ चत॑स्र॒ श्चत॑स्र॒ स्त्वया ऽद्ध्य॑क्षे॒णा द्ध्य॑क्षेण॒ त्वया॒ चत॑स्र॒ श्चत॑स्र॒ स्त्वया ऽद्ध्य॑क्षेण । \newline
17. त्वया ऽद्ध्य॑क्षे॒णा द्ध्य॑क्षेण॒ त्वया॒ त्वया ऽद्ध्य॑क्षेण॒ पृत॑नाः॒ पृत॑ना॒ अद्ध्य॑क्षेण॒ त्वया॒ त्वया ऽद्ध्य॑क्षेण॒ पृत॑नाः । \newline
18. अद्ध्य॑क्षेण॒ पृत॑नाः॒ पृत॑ना॒ अद्ध्य॑क्षे॒ णाद्ध्य॑क्षेण॒ पृत॑ना जयेम जयेम॒ पृत॑ना॒ अद्ध्य॑क्षे॒ णाद्ध्य॑क्षेण॒ पृत॑ना जयेम । \newline
19. अद्ध्य॑क्षे॒णेत्यधि॑ - अ॒क्षे॒ण॒ । \newline
20. पृत॑ना जयेम जयेम॒ पृत॑नाः॒ पृत॑ना जयेम । \newline
21. ज॒ये॒मेति॑ जयेम । \newline
22. मम॑ दे॒वा दे॒वा मम॒ मम॑ दे॒वा वि॑ह॒वे वि॑ह॒वे दे॒वा मम॒ मम॑ दे॒वा वि॑ह॒वे । \newline
23. दे॒वा वि॑ह॒वे वि॑ह॒वे दे॒वा दे॒वा वि॑ह॒वे स॑न्तु सन्तु विह॒वे दे॒वा दे॒वा वि॑ह॒वे स॑न्तु । \newline
24. वि॒ह॒वे स॑न्तु सन्तु विह॒वे वि॑ह॒वे स॑न्तु॒ सर्वे॒ सर्वे॑ सन्तु विह॒वे वि॑ह॒वे स॑न्तु॒ सर्वे᳚ । \newline
25. वि॒ह॒व इति॑ वि - ह॒वे । \newline
26. स॒न्तु॒ सर्वे॒ सर्वे॑ सन्तु सन्तु॒ सर्व॒ इन्द्रा॑वन्त॒ इन्द्रा॑वन्तः॒ सर्वे॑ सन्तु सन्तु॒ सर्व॒ इन्द्रा॑वन्तः । \newline
27. सर्व॒ इन्द्रा॑वन्त॒ इन्द्रा॑वन्तः॒ सर्वे॒ सर्व॒ इन्द्रा॑वन्तो म॒रुतो॑ म॒रुत॒ इन्द्रा॑वन्तः॒ सर्वे॒ सर्व॒ इन्द्रा॑वन्तो म॒रुतः॑ । \newline
28. इन्द्रा॑वन्तो म॒रुतो॑ म॒रुत॒ इन्द्रा॑वन्त॒ इन्द्रा॑वन्तो म॒रुतो॒ विष्णु॒र् विष्णु॑र् म॒रुत॒ इन्द्रा॑वन्त॒ इन्द्रा॑वन्तो म॒रुतो॒ विष्णुः॑ । \newline
29. इन्द्रा॑वन्त॒ इतीन्द्र॑ - व॒न्तः॒ । \newline
30. म॒रुतो॒ विष्णु॒र् विष्णु॑र् म॒रुतो॑ म॒रुतो॒ विष्णु॑ र॒ग्नि र॒ग्निर् विष्णु॑र् म॒रुतो॑ म॒रुतो॒ विष्णु॑ र॒ग्निः । \newline
31. विष्णु॑ र॒ग्नि र॒ग्निर् विष्णु॒र् विष्णु॑ र॒ग्निः । \newline
32. अ॒ग्निरित्य॒ग्निः । \newline
33. ममा॒न्तरि॑क्ष म॒न्तरि॑क्ष॒म् मम॒ ममा॒न्तरि॑क्ष मु॒रू᳚(1॒)र्व॑न्तरि॑क्ष॒म् मम॒ ममा॒न्तरि॑क्ष मु॒रु । \newline
34. अ॒न्तरि॑क्ष मु॒रू᳚(1॒)र्व॑न्तरि॑क्ष म॒न्तरि॑क्ष मु॒रु गो॒पम् गो॒प मु॒र्व॑न्तरि॑क्ष म॒न्तरि॑क्ष मु॒रु गो॒पम् । \newline
35. उ॒रु गो॒पम् गो॒प मु॒रू॑रु गो॒प म॑स्त्वस्तु गो॒प मु॒रू॑रु गो॒प म॑स्तु । \newline
36. गो॒प म॑स्त्वस्तु गो॒पम् गो॒प म॑स्तु॒ मह्य॒म् मह्य॑ मस्तु गो॒पम् गो॒प म॑स्तु॒ मह्य᳚म् । \newline
37. अ॒स्तु॒ मह्य॒म् मह्य॑ मस्त्वस्तु॒ मह्यं॒ ॅवातो॒ वातो॒ मह्य॑ मस्त्वस्तु॒ मह्यं॒ ॅवातः॑ । \newline
38. मह्यं॒ ॅवातो॒ वातो॒ मह्य॒म् मह्यं॒ ॅवातः॑ पवताम् पवतां॒ ॅवातो॒ मह्य॒म् मह्यं॒ ॅवातः॑ पवताम् । \newline
39. वातः॑ पवताम् पवतां॒ ॅवातो॒ वातः॑ पवता॒म् कामे॒ कामे॑ पवतां॒ ॅवातो॒ वातः॑ पवता॒म् कामे᳚ । \newline
40. प॒व॒ता॒म् कामे॒ कामे॑ पवताम् पवता॒म् कामे॑ अ॒स्मिन् न॒स्मिन् कामे॑ पवताम् पवता॒म् कामे॑ अ॒स्मिन्न् । \newline
41. कामे॑ अ॒स्मिन् न॒स्मिन् कामे॒ कामे॑ अ॒स्मिन्न् । \newline
42. अ॒स्मिन्नित्य॒स्मिन्न् । \newline
43. मयि॑ दे॒वा दे॒वा मयि॒ मयि॑ दे॒वा द्रवि॑ण॒म् द्रवि॑णम् दे॒वा मयि॒ मयि॑ दे॒वा द्रवि॑णम् । \newline
44. दे॒वा द्रवि॑ण॒म् द्रवि॑णम् दे॒वा दे॒वा द्रवि॑ण॒ मा द्रवि॑णम् दे॒वा दे॒वा द्रवि॑ण॒ मा । \newline
45. द्रवि॑ण॒ मा द्रवि॑ण॒म् द्रवि॑ण॒ मा य॑जन्तां ॅयजन्ता॒ मा द्रवि॑ण॒म् द्रवि॑ण॒ मा य॑जन्ताम् । \newline
46. आ य॑जन्तां ॅयजन्ता॒ मा य॑जन्ता॒म् मयि॒ मयि॑ यजन्ता॒ मा य॑जन्ता॒म् मयि॑ । \newline
47. य॒ज॒न्ता॒म् मयि॒ मयि॑ यजन्तां ॅयजन्ता॒म् मय्या॒शी रा॒शीर् मयि॑ यजन्तां ॅयजन्ता॒म् मय्या॒शीः । \newline
48. मय्या॒शी रा॒शीर् मयि॒ मय्या॒शी र॑स्त्व स्त्वा॒शीर् मयि॒ मय्या॒शी र॑स्तु । \newline
49. आ॒शी र॑स्त्व स्त्वा॒शी रा॒शी र॑स्तु॒ मयि॒ मय्य॑ स्त्वा॒शी रा॒शी र॑स्तु॒ मयि॑ । \newline
50. आ॒शीरित्या᳚ - शीः । \newline
51. अ॒स्तु॒ मयि॒ मय्य॑स्त्वस्तु॒ मयि॑ दे॒वहू॑तिर् दे॒वहू॑ति॒र् मय्य॑स्त्वस्तु॒ मयि॑ दे॒वहू॑तिः । \newline
52. मयि॑ दे॒वहू॑तिर् दे॒वहू॑ति॒र् मयि॒ मयि॑ दे॒वहू॑तिः । \newline
53. दे॒वहू॑ति॒रिति॑ दे॒व - हू॒तिः॒ । \newline
54. दैव्या॒ होता॑रा॒ होता॑रा॒ दैव्या॒ दैव्या॒ होता॑रा वनिषन्त वनिषन्त॒ होता॑रा॒ दैव्या॒ दैव्या॒ होता॑रा वनिषन्त । \newline
55. होता॑रा वनिषन्त वनिषन्त॒ होता॑रा॒ होता॑रा वनिषन्त॒ पूर्वे॒ पूर्वे॑ वनिषन्त॒ होता॑रा॒ होता॑रा वनिषन्त॒ पूर्वे᳚ । \newline
56. व॒नि॒ष॒न्त॒ पूर्वे॒ पूर्वे॑ वनिषन्त वनिषन्त॒ पूर्वे ऽरि॑ष्टा॒ अरि॑ष्टाः॒ पूर्वे॑ वनिषन्त वनिषन्त॒ पूर्वे ऽरि॑ष्टाः । \newline
\pagebreak
\markright{ TS 4.7.14.2  \hfill https://www.vedavms.in \hfill}

\section{ TS 4.7.14.2 }

\textbf{TS 4.7.14.2 } \newline
\textbf{Samhita Paata} \newline

पूर्वे ऽरि॑ष्टाः स्याम त॒नुवा॑ सु॒वीराः᳚ ॥मह्यं॑ ॅयजन्तु॒ मम॒ यानि॑ ह॒व्याऽऽकू॑तिः स॒त्या मन॑सो मे अस्तु ।एनो॒ मानिगां᳚ कत॒मच्च॒नाहं ॅविश्वे॑ देवासो॒ अधि॑वोच ता मे ॥देवीः᳚ षडुर्वीरु॒रुणः॑ कृणोत॒ विश्वे॑ देवा स इ॒ह वी॑रयद्ध्वं ।माहा᳚स्महि प्र॒जया॒ मा त॒नूभि॒र्मा र॑धाम द्विष॒ते सो॑म राजन्न् ॥अ॒ग्निर्म॒न्युं प्र॑तिनु॒दन् पु॒रस्ता॒- [  ] \newline

\textbf{Pada Paata} \newline

पूर्वे᳚ । अरि॑ष्टाः । स्या॒म॒ । त॒नुवा᳚ । सु॒वीरा॒ इति॑ सु - वीराः᳚ ॥ मह्य᳚म् । य॒ज॒न्तु॒ । मम॑ । यानि॑ । ह॒व्या । आकू॑ति॒रित्या - कू॒तिः॒ । स॒त्या । मन॑सः । मे॒ । अ॒स्तु॒ ॥ एनः॑ । मा । नीति॑ । गा॒म् । क॒त॒मत् । च॒न । अ॒हम् । विश्वे᳚ । दे॒वा॒सः॒ । अधीति॑ । वो॒च॒त॒ । मे॒ ॥ देवीः᳚ । ष॒डु॒र्वी॒रिति॑ षट् - उ॒र्वीः॒ । उ॒रु । नः॒ । कृ॒णो॒त॒ । विश्वे᳚ । दे॒वा॒सः॒ । इ॒ह । वी॒र॒य॒द्ध्व॒म् ॥ मा । हा॒स्म॒हि॒ । प्र॒जयेति॑ प्र - जया᳚ । मा । त॒नूभिः॑ । मा । र॒धा॒म॒ । द्वि॒ष॒ते । सो॒म॒ । रा॒ज॒न्न् ॥ अ॒ग्निः । म॒न्युम् । प्र॒ति॒नु॒दन्निति॑ प्रति - नु॒दन्न् । पु॒रस्ता᳚त् ।  \newline


\textbf{Krama Paata} \newline

पूर्वेऽरि॑ष्टाः । अरि॑ष्टाः स्याम । स्या॒म॒ त॒नुवा᳚ । त॒नुवा॑ सु॒वीराः᳚ । सु॒वीरा॒ इति॑ सु - वीराः᳚ ॥ मह्य॑म् ॅयजन्तु । य॒ज॒न्तु॒ मम॑ । मम॒ यानि॑ । यानि॑ ह॒व्या । ह॒व्याऽऽकू॑तिः । आकू॑तिः स॒त्या । आकू॑ति॒रित्या - कू॒तिः॒ । स॒त्या मन॑सः । मन॑सो मे । मे॒ अ॒स्तु॒ । अ॒स्त्वित्य॑स्तु ॥ एनो॒ मा । मा नि । नि गा᳚म् । गा॒म् क॒त॒मत् । क॒त॒मच् च॒न । च॒नाहम् । अ॒हम् ॅविश्वे᳚ । विश्वे॑ देवासः । दे॒वा॒सो॒ अधि॑ । अधि॑ वोचत । वो॒च॒ता॒ मे॒ । म॒ इति॑ मे ॥ देवीः᳚ षडुर्वीः । ष॒डु॒र्वी॒रु॒रु । ष॒डु॒र्वी॒रिति॑ षट् - उ॒र्वीः॒ । उ॒रु णः॑ । नः॒ कृ॒णो॒त॒ । कृ॒णो॒त॒ विश्वे᳚ । विश्वे॑ देवासः । दे॒वा॒स॒ इ॒ह । इ॒ह वी॑रयद्ध्वम् । वी॒र॒य॒द्ध्व॒मिति॑ वीरयद्ध्वम् ॥ मा हा᳚स्महि । हा॒स्म॒हि॒ प्र॒जया᳚ । प्र॒जया॒ मा । प्र॒जयेति॑ प्र - जया᳚ । मा त॒नूभिः॑ । त॒नूभि॒र् मा । मा र॑धाम । र॒धा॒म॒ द्वि॒ष॒ते । द्वि॒ष॒ते सो॑म । सो॒म॒ रा॒ज॒न्न्॒ । रा॒ज॒न्निति॑ राजन्न् ॥ अ॒ग्निर् म॒न्युम् । म॒न्युम् प्र॑तिनु॒दन्न् । प्र॒ति॒नु॒दन् पु॒रस्ता᳚त् । प्र॒ति॒नु॒दन्निति॑ प्रति - नु॒दन्न् । पु॒रस्ता॒दद॑ब्धः \newline

\textbf{Jatai Paata} \newline

1. पूर्वे ऽरि॑ष्टा॒ अरि॑ष्टाः॒ पूर्वे॒ पूर्वे ऽरि॑ष्टाः । \newline
2. अरि॑ष्टाः स्याम स्या॒मा रि॑ष्टा॒ अरि॑ष्टाः स्याम । \newline
3. स्या॒म॒ त॒नुवा॑ त॒नुवा᳚ स्याम स्याम त॒नुवा᳚ । \newline
4. त॒नुवा॑ सु॒वीराः᳚ सु॒वीरा᳚ स्त॒नुवा॑ त॒नुवा॑ सु॒वीराः᳚ । \newline
5. सु॒वीरा॒ इति॑ सु - वीराः᳚ । \newline
6. मह्यं॑ ॅयजन्तु यजन्तु॒ मह्य॒म् मह्यं॑ ॅयजन्तु । \newline
7. य॒ज॒न्तु॒ मम॒ मम॑ यजन्तु यजन्तु॒ मम॑ । \newline
8. मम॒ यानि॒ यानि॒ मम॒ मम॒ यानि॑ । \newline
9. यानि॑ ह॒व्या ह॒व्या यानि॒ यानि॑ ह॒व्या । \newline
10. ह॒व्या ऽऽकू॑ति॒ राकू॑तिर्. ह॒व्या ह॒व्या ऽऽकू॑तिः । \newline
11. आकू॑तिः स॒त्या स॒त्या ऽऽकू॑ति॒ राकू॑तिः स॒त्या । \newline
12. आकू॑ति॒रित्या - कू॒तिः॒ । \newline
13. स॒त्या मन॑सो॒ मन॑सः स॒त्या स॒त्या मन॑सः । \newline
14. मन॑सो मे मे॒ मन॑सो॒ मन॑सो मे । \newline
15. मे॒ अ॒स्त्व॒स्तु॒ मे॒ मे॒ अ॒स्तु॒ । \newline
16. अ॒स्त्वित्य॑स्तु । \newline
17. एनो॒ मा मैन॒ एनो॒ मा । \newline
18. मा नि नि मा मा नि । \newline
19. नि गा᳚म् गा॒म् नि नि गा᳚म् । \newline
20. गा॒म् क॒त॒मत् क॑त॒मद् गा᳚म् गाम् कत॒मत् । \newline
21. क॒त॒मच् च॒न च॒न क॑त॒मत् क॑त॒मच् च॒न । \newline
22. च॒नाह म॒हम् च॒न च॒नाहम् । \newline
23. अ॒हं ॅविश्वे॒ विश्वे॒ ऽह म॒हं ॅविश्वे᳚ । \newline
24. विश्वे॑ देवासो देवासो॒ विश्वे॒ विश्वे॑ देवासः । \newline
25. दे॒वा॒सो॒ अध्यधि॑ देवासो देवासो॒ अधि॑ । \newline
26. अधि॑ वोचत वोच॒ता ध्यधि॑ वोचत । \newline
27. वो॒च॒ता॒ मे॒ मे॒ वो॒च॒त॒ वो॒च॒ता॒ मे॒ । \newline
28. म॒ इति॑ मे । \newline
29. देवी᳚ ष्षडुर्वी ष्षडुर्वी॒र् देवी॒र् देवी᳚ ष्षडुर्वीः । \newline
30. ष॒डु॒र्वी॒ रु॒रू॑रु ष॑डुर्वी ष्षडुर्वी रु॒रु । \newline
31. ष॒डु॒र्वी॒रिति॑ षट् - उ॒र्वीः॒ । \newline
32. उ॒रु णो॑ न उ॒रू॑रु णः॑ । \newline
33. नः॒ कृ॒णो॒त॒ कृ॒णो॒त॒ नो॒ नः॒ कृ॒णो॒त॒ । \newline
34. कृ॒णो॒त॒ विश्वे॒ विश्वे॑ कृणोत कृणोत॒ विश्वे᳚ । \newline
35. विश्वे॑ देवासो देवासो॒ विश्वे॒ विश्वे॑ देवासः । \newline
36. दे॒वा॒स॒ इ॒हे ह दे॑वासो देवास इ॒ह । \newline
37. इ॒ह वी॑रयद्ध्वं ॅवीरयद्ध्व मि॒हे ह वी॑रयद्ध्वम् । \newline
38. वी॒र॒य॒द्ध्व॒मिति॑ वीरयद्ध्वम् । \newline
39. मा हा᳚स्महि हास्महि॒ मा मा हा᳚स्महि । \newline
40. हा॒स्म॒हि॒ प्र॒जया᳚ प्र॒जया॑ हास्महि हास्महि प्र॒जया᳚ । \newline
41. प्र॒जया॒ मा मा प्र॒जया᳚ प्र॒जया॒ मा । \newline
42. प्र॒जयेति॑ प्र - जया᳚ । \newline
43. मा त॒नूभि॑ स्त॒नूभि॒र् मा मा त॒नूभिः॑ । \newline
44. त॒नूभि॒र् मा मा त॒नूभि॑ स्त॒नूभि॒र् मा । \newline
45. मा र॑धाम रधाम॒ मा मा र॑धाम । \newline
46. र॒धा॒म॒ द्वि॒ष॒ते द्वि॑ष॒ते र॑धाम रधाम द्विष॒ते । \newline
47. द्वि॒ष॒ते सो॑म सोम द्विष॒ते द्वि॑ष॒ते सो॑म । \newline
48. सो॒म॒ रा॒ज॒न् रा॒ज॒न् थ्सो॒म॒ सो॒म॒ रा॒ज॒न्न् । \newline
49. रा॒ज॒न्निति॑ राजन्न् । \newline
50. अ॒ग्निर् म॒न्युम् म॒न्यु म॒ग्नि र॒ग्निर् म॒न्युम् । \newline
51. म॒न्युम् प्र॑तिनु॒दन् प्र॑तिनु॒दन् म॒न्युम् म॒न्युम् प्र॑तिनु॒दन्न् । \newline
52. प्र॒ति॒नु॒दन् पु॒रस्ता᳚त् पु॒रस्ता᳚त् प्रतिनु॒दन् प्र॑तिनु॒दन् पु॒रस्ता᳚त् । \newline
53. प्र॒ति॒नु॒दन्निति॑ प्रति - नु॒दन्न् । \newline
54. पु॒रस्ता॒ दद॑ब्धो॒ अद॑ब्धः पु॒रस्ता᳚त् पु॒रस्ता॒ दद॑ब्धः । \newline

\textbf{Ghana Paata } \newline

1. पूर्वे ऽरि॑ष्टा॒ अरि॑ष्टाः॒ पूर्वे॒ पूर्वे ऽरि॑ष्टाः स्याम स्या॒मा रि॑ष्टाः॒ पूर्वे॒ पूर्वे ऽरि॑ष्टाः स्याम । \newline
2. अरि॑ष्टाः स्याम स्या॒मा रि॑ष्टा॒ अरि॑ष्टाः स्याम त॒नुवा॑ त॒नुवा᳚ स्या॒मा रि॑ष्टा॒ अरि॑ष्टाः स्याम त॒नुवा᳚ । \newline
3. स्या॒म॒ त॒नुवा॑ त॒नुवा᳚ स्याम स्याम त॒नुवा॑ सु॒वीराः᳚ सु॒वीरा᳚ स्त॒नुवा᳚ स्याम स्याम त॒नुवा॑ सु॒वीराः᳚ । \newline
4. त॒नुवा॑ सु॒वीराः᳚ सु॒वीरा᳚ स्त॒नुवा॑ त॒नुवा॑ सु॒वीराः᳚ । \newline
5. सु॒वीरा॒ इति॑ सु - वीराः᳚ । \newline
6. मह्यं॑ ॅयजन्तु यजन्तु॒ मह्य॒म् मह्यं॑ ॅयजन्तु॒ मम॒ मम॑ यजन्तु॒ मह्य॒म् मह्यं॑ ॅयजन्तु॒ मम॑ । \newline
7. य॒ज॒न्तु॒ मम॒ मम॑ यजन्तु यजन्तु॒ मम॒ यानि॒ यानि॒ मम॑ यजन्तु यजन्तु॒ मम॒ यानि॑ । \newline
8. मम॒ यानि॒ यानि॒ मम॒ मम॒ यानि॑ ह॒व्या ह॒व्या यानि॒ मम॒ मम॒ यानि॑ ह॒व्या । \newline
9. यानि॑ ह॒व्या ह॒व्या यानि॒ यानि॑ ह॒व्या ऽऽकू॑ति॒ राकू॑तिर्. ह॒व्या यानि॒ यानि॑ ह॒व्या ऽऽकू॑तिः । \newline
10. ह॒व्या ऽऽकू॑ति॒ राकू॑तिर्. ह॒व्या ह॒व्या ऽऽकू॑तिः स॒त्या स॒त्या ऽऽकू॑तिर्. ह॒व्या ह॒व्या ऽऽकू॑तिः स॒त्या । \newline
11. आकू॑तिः स॒त्या स॒त्या ऽऽकू॑ति॒ राकू॑तिः स॒त्या मन॑सो॒ मन॑सः स॒त्या ऽऽकू॑ति॒ राकू॑तिः स॒त्या मन॑सः । \newline
12. आकू॑ति॒रित्या - कू॒तिः॒ । \newline
13. स॒त्या मन॑सो॒ मन॑सः स॒त्या स॒त्या मन॑सो मे मे॒ मन॑सः स॒त्या स॒त्या मन॑सो मे । \newline
14. मन॑सो मे मे॒ मन॑सो॒ मन॑सो मे अस्त्वस्तु मे॒ मन॑सो॒ मन॑सो मे अस्तु । \newline
15. मे॒ अ॒स्त्व॒स्तु॒ मे॒ मे॒ अ॒स्तु॒ । \newline
16. अ॒स्त्वित्य॑स्तु । \newline
17. एनो॒ मा मैन॒ एनो॒ मा नि नि मैन॒ एनो॒ मा नि । \newline
18. मा नि नि मा मा नि गा᳚म् गा॒म् नि मा मा नि गा᳚म् । \newline
19. नि गा᳚म् गा॒म् नि नि गा᳚म् कत॒मत् क॑त॒मद् गा॒म् नि नि गा᳚म् कत॒मत् । \newline
20. गा॒म् क॒त॒मत् क॑त॒मद् गा᳚म् गाम् कत॒मच् च॒न च॒न क॑त॒मद् गा᳚म् गाम् कत॒मच् च॒न । \newline
21. क॒त॒मच् च॒न च॒न क॑त॒मत् क॑त॒मच् च॒नाह म॒हम् च॒न क॑त॒मत् क॑त॒मच् च॒नाहम् । \newline
22. च॒नाह म॒हम् च॒न च॒नाहं ॅविश्वे॒ विश्वे॒ ऽहम् च॒न च॒नाहं ॅविश्वे᳚ । \newline
23. अ॒हं ॅविश्वे॒ विश्वे॒ ऽह म॒हं ॅविश्वे॑ देवासो देवासो॒ विश्वे॒ ऽह म॒हं ॅविश्वे॑ देवासः । \newline
24. विश्वे॑ देवासो देवासो॒ विश्वे॒ विश्वे॑ देवासो॒ अध्यधि॑ देवासो॒ विश्वे॒ विश्वे॑ देवासो॒ अधि॑ । \newline
25. दे॒वा॒सो॒ अध्यधि॑ देवासो देवासो॒ अधि॑ वोचत वोच॒ताधि॑ देवासो देवासो॒ अधि॑ वोचत । \newline
26. अधि॑ वोचत वोच॒ताध्यधि॑ वोचता मे मे वोच॒ताध्यधि॑ वोचता मे । \newline
27. वो॒च॒ता॒ मे॒ मे॒ वो॒च॒त॒ वो॒च॒ता॒ मे॒ । \newline
28. म॒ इति॑ मे । \newline
29. देवी᳚ ष्षडुर्वी ष्षडुर्वी॒र् देवी॒र् देवी᳚ ष्षडुर्वी रु॒रू॑रु ष॑डुर्वी॒र् देवी॒र् देवी᳚ ष्षडुर्वी रु॒रु । \newline
30. ष॒डु॒र्वी॒ रु॒रू॑रु ष॑डुर्वी ष्षडुर्वी रु॒रु णो॑ न उ॒रु ष॑डुर्वी ष्षडुर्वी रु॒रु णः॑ । \newline
31. ष॒डु॒र्वी॒रिति॑ षट् - उ॒र्वीः॒ । \newline
32. उ॒रु णो॑ न उ॒रू॑रु णः॑ कृणोत कृणोत न उ॒रू॑रु णः॑ कृणोत । \newline
33. नः॒ कृ॒णो॒त॒ कृ॒णो॒त॒ नो॒ नः॒ कृ॒णो॒त॒ विश्वे॒ विश्वे॑ कृणोत नो नः कृणोत॒ विश्वे᳚ । \newline
34. कृ॒णो॒त॒ विश्वे॒ विश्वे॑ कृणोत कृणोत॒ विश्वे॑ देवासो देवासो॒ विश्वे॑ कृणोत कृणोत॒ विश्वे॑ देवासः । \newline
35. विश्वे॑ देवासो देवासो॒ विश्वे॒ विश्वे॑ देवास इ॒हेह दे॑वासो॒ विश्वे॒ विश्वे॑ देवास इ॒ह । \newline
36. दे॒वा॒स॒ इ॒हेह दे॑वासो देवास इ॒ह वी॑रयद्ध्वं ॅवीरयद्ध्व मि॒ह दे॑वासो देवास इ॒ह वी॑रयद्ध्वम् । \newline
37. इ॒ह वी॑रयद्ध्वं ॅवीरयद्ध्व मि॒हेह वी॑रयद्ध्वम् । \newline
38. वी॒र॒य॒द्ध्व॒मिति॑ वीरयद्ध्वम् । \newline
39. मा हा᳚स्महि हास्महि॒ मा मा हा᳚स्महि प्र॒जया᳚ प्र॒जया॑ हास्महि॒ मा मा हा᳚स्महि प्र॒जया᳚ । \newline
40. हा॒स्म॒हि॒ प्र॒जया᳚ प्र॒जया॑ हास्महि हास्महि प्र॒जया॒ मा मा प्र॒जया॑ हास्महि हास्महि प्र॒जया॒ मा । \newline
41. प्र॒जया॒ मा मा प्र॒जया᳚ प्र॒जया॒ मा त॒नूभि॑ स्त॒नूभि॒र् मा प्र॒जया᳚ प्र॒जया॒ मा त॒नूभिः॑ । \newline
42. प्र॒जयेति॑ प्र - जया᳚ । \newline
43. मा त॒नूभि॑ स्त॒नूभि॒र् मा मा त॒नूभि॒र् मा मा त॒नूभि॒र् मा मा त॒नूभि॒र् मा । \newline
44. त॒नूभि॒र् मा मा त॒नूभि॑ स्त॒नूभि॒र् मा र॑धाम रधाम॒ मा त॒नूभि॑ स्त॒नूभि॒र् मा र॑धाम । \newline
45. मा र॑धाम रधाम॒ मा मा र॑धाम द्विष॒ते द्वि॑ष॒ते र॑धाम॒ मा मा र॑धाम द्विष॒ते । \newline
46. र॒धा॒म॒ द्वि॒ष॒ते द्वि॑ष॒ते र॑धाम रधाम द्विष॒ते सो॑म सोम द्विष॒ते र॑धाम रधाम द्विष॒ते सो॑म । \newline
47. द्वि॒ष॒ते सो॑म सोम द्विष॒ते द्वि॑ष॒ते सो॑म राजन् राजन् थ्सोम द्विष॒ते द्वि॑ष॒ते सो॑म राजन्न् । \newline
48. सो॒म॒ रा॒ज॒न् रा॒ज॒न् थ्सो॒म॒ सो॒म॒ रा॒ज॒न्न् । \newline
49. रा॒ज॒न्निति॑ राजन्न् । \newline
50. अ॒ग्निर् म॒न्युम् म॒न्यु म॒ग्नि र॒ग्निर् म॒न्युम् प्र॑तिनु॒दन् प्र॑तिनु॒दन् म॒न्यु म॒ग्नि र॒ग्निर् म॒न्युम् प्र॑तिनु॒दन्न् । \newline
51. म॒न्युम् प्र॑तिनु॒दन् प्र॑तिनु॒दन् म॒न्युम् म॒न्युम् प्र॑तिनु॒दन् पु॒रस्ता᳚त् पु॒रस्ता᳚त् प्रतिनु॒दन् म॒न्युम् म॒न्युम् प्र॑तिनु॒दन् पु॒रस्ता᳚त् । \newline
52. प्र॒ति॒नु॒दन् पु॒रस्ता᳚त् पु॒रस्ता᳚त् प्रतिनु॒दन् प्र॑तिनु॒दन् पु॒रस्ता॒ दद॑ब्धो॒ अद॑ब्धः पु॒रस्ता᳚त् प्रतिनु॒दन् प्र॑तिनु॒दन् पु॒रस्ता॒ दद॑ब्धः । \newline
53. प्र॒ति॒नु॒दन्निति॑ प्रति - नु॒दन्न् । \newline
54. पु॒रस्ता॒ दद॑ब्धो॒ अद॑ब्धः पु॒रस्ता᳚त् पु॒रस्ता॒ दद॑ब्धो गो॒पा गो॒पा अद॑ब्धः पु॒रस्ता᳚त् पु॒रस्ता॒ दद॑ब्धो गो॒पाः । \newline
\pagebreak
\markright{ TS 4.7.14.3  \hfill https://www.vedavms.in \hfill}

\section{ TS 4.7.14.3 }

\textbf{TS 4.7.14.3 } \newline
\textbf{Samhita Paata} \newline

दद॑ब्धो गो॒पाः परि॑पाहि न॒स्त्वं । प्र॒त्यञ्चो॑ यन्तु नि॒गुतः॒ पुन॒स्ते॑ ऽमैषां᳚ चि॒त्तं प्र॒बुधा॒ विने॑शत् ॥ धा॒ता धा॑तृ॒णां भुव॑नस्य॒ यस्पति॑ र्दे॒वꣳ स॑वि॒तार॑मभि माति॒षाहं᳚ । इ॒मं ॅय॒ज्ञ् म॒श्विनो॒भा बृह॒स्पति॑ र्दे॒वाः पा᳚न्तु॒ यज॑मानं न्य॒र्त्थात् ॥ उ॒रु॒व्यचा॑ नो महि॒षः शर्म॑ यꣳ सद॒स्मिन्. हवे॑ पुरुहू॒तः पु॑रु॒क्षु । स नः॑ प्र॒जायै॑ हर्यश्व मृड॒येन्द्र॒ मा - [  ] \newline

\textbf{Pada Paata} \newline

अद॑ब्धः । गो॒पा इति॑ गो - पाः । परीति॑ । पा॒हि॒ । नः॒ । त्वम् ॥ प्र॒त्यञ्चः॑ । य॒न्तु॒ । नि॒गुत॒ इति॑ नि - गुतः॑ । पुनः॑ । ते । अ॒मा । ए॒षा॒म् । चि॒त्तम् । प्र॒बुधेति॑ प्र - बुधा᳚ । वीति॑ । ने॒श॒त् ॥ धा॒ता । धा॒तृ॒णाम् । भुव॑नस्य । यः । पतिः॑ । दे॒वम् । स॒वि॒तार᳚म् । अ॒भि॒मा॒ति॒षाह॒मित्य॑भिमाति - साह᳚म् ॥ इ॒मम् । य॒ज्ञ्म् । अ॒श्विना᳚ । उ॒भा । बृह॒स्पतिः॑ । दे॒वाः । पा॒न्तु॒ । यज॑मानम् । न्य॒र्थादिति॑ नि- अ॒र्थात् ॥ उ॒रु॒व्यचा॒ इत्यु॑रु-व्यचाः᳚ । नः॒ । म॒हि॒षः । शर्म॑ । यꣳ॒॒स॒त् । अ॒स्मिन्न् । हवे᳚ । पु॒रु॒हू॒त इति॑ पुरु - हू॒तः । पु॒रु॒क्षु ॥ सः । नः॒ । प्र॒जाय॒ इति॑ प्र - जायै᳚ । ह॒र्य॒श्वेति॑ हरि - अ॒श्व॒ । मृ॒ड॒य॒ । इन्द्र॑ । मा ।  \newline


\textbf{Krama Paata} \newline

अद॑ब्धो गो॒पाः । गो॒पाः परि॑ । गो॒पा इति॑ गो - पाः । परि॑ पाहि । पा॒हि॒ नः॒ । न॒स्त्वम् । त्वमिति॒ त्वम् ॥ प्र॒त्यञ्चो॑ यन्तु । य॒न्तु॒ नि॒गुतः॑ । नि॒गुतः॒ पुनः॑ । नि॒गुत॒ इति॑ नि - गुतः॑ । पुन॒स्ते । ते॑ऽमा । अ॒मैषा᳚म् । ए॒षा॒म् चि॒त्तम् । चि॒त्तम् प्र॒बुधा᳚ । प्र॒बुधा॒ वि । प्र॒बुधेति॑ प्र - बुधा᳚ । वि ने॑शत् । ने॒श॒दिति॑ नेशत् ॥ धा॒ता धा॑तृ॒णाम् । धा॒तृ॒णाम् भुव॑नस्य । भुव॑नस्य॒ यः । यस्पतिः॑ । पति॑र् दे॒वम् । दे॒वꣳ स॑वि॒तार᳚म् । स॒वि॒तार॑मभिमाति॒षाह᳚म् । अ॒भि॒मा॒ति॒षाह॒मित्य॑भिमाति - साह᳚म् ॥ इ॒मम् ॅय॒ज्ञ्म् । य॒ज्ञ्म॒श्विना᳚ । अ॒श्विनो॒भा । उ॒भा बृह॒स्पतिः॑ । बृह॒स्पति॑र् दे॒वाः । दे॒वाः पा᳚न्तु । पा॒न्तु॒ यज॑मानम् । यज॑मानम् न्य॒र्त्थात् । न्य॒र्त्थादिति॑ नि - अ॒र्त्थात् ॥ उ॒रु॒व्यचा॑ नः । उ॒रु॒व्यचा॒ इत्यु॑रु - व्यचाः᳚ । नो॒ म॒हि॒षः । म॒हि॒षः शर्म॑ । शर्म॑ यꣳसत् । यꣳ॒॒स॒द॒स्मिन्न् । अ॒स्मिन्. हवे᳚ । हवे॑ पुरुहू॒तः । पु॒रु॒हू॒तः पु॑रु॒क्षु । पु॒रु॒हू॒त इति॑ पुरु - हू॒तः । पु॒रु॒क्ष्विति॑ पुरु॒क्षु ॥ स नः॑ । नः॒ प्र॒जायै᳚ । प्र॒जायै॑ हर्यश्व । प्र॒जाया॒ इति॑ प्र - जायै᳚ । ह॒र्य॒श्व॒ मृ॒ड॒य॒ । ह॒र्य॒श्वेति॑ हरि - अ॒श्व॒ । मृ॒ड॒येन्द्र॑ । इन्द्र॒ मा ( ) । मा नः॑ \newline

\textbf{Jatai Paata} \newline

1. अद॑ब्धो गो॒पा गो॒पा अद॑ब्धो॒ अद॑ब्धो गो॒पाः । \newline
2. गो॒पाः परि॒ परि॑ गो॒पा गो॒पाः परि॑ । \newline
3. गो॒पा इति॑ गो - पाः । \newline
4. परि॑ पाहि पाहि॒ परि॒ परि॑ पाहि । \newline
5. पा॒हि॒ नो॒ नः॒ पा॒हि॒ पा॒हि॒ नः॒ । \newline
6. न॒ स्त्वम् त्वम् नो॑ न॒स्त्वम् । \newline
7. त्वमिति॒ त्वम् । \newline
8. प्र॒त्यञ्चो॑ यन्तु यन्तु प्र॒त्यञ्चः॑ प्र॒त्यञ्चो॑ यन्तु । \newline
9. य॒न्तु॒ नि॒गुतो॑ नि॒गुतो॑ यन्तु यन्तु नि॒गुतः॑ । \newline
10. नि॒गुतः॒ पुनः॒ पुन॑र् नि॒गुतो॑ नि॒गुतः॒ पुनः॑ । \newline
11. नि॒गुत॒ इति॑ नि - गुतः॑ । \newline
12. पुन॒ स्ते ते पुनः॒ पुन॒ स्ते । \newline
13. ते॑ ऽमा ऽमा ते ते॑ ऽमा । \newline
14. अ॒मैषा॑ मेषा म॒मा ऽमैषा᳚म् । \newline
15. ए॒षा॒म् चि॒त्तम् चि॒त्त मे॑षा मेषाम् चि॒त्तम् । \newline
16. चि॒त्तम् प्र॒बुधा᳚ प्र॒बुधा॑ चि॒त्तम् चि॒त्तम् प्र॒बुधा᳚ । \newline
17. प्र॒बुधा॒ वि वि प्र॒बुधा᳚ प्र॒बुधा॒ वि । \newline
18. प्र॒बुधेति॑ प्र - बुधा᳚ । \newline
19. वि ने॑शन् नेश॒द् वि वि ने॑शत् । \newline
20. ने॒श॒दिति॑ नेशत् । \newline
21. धा॒ता धा॑तृ॒णाम् धा॑तृ॒णाम् धा॒ता धा॒ता धा॑तृ॒णाम् । \newline
22. धा॒तृ॒णाम् भुव॑नस्य॒ भुव॑नस्य धातृ॒णाम् धा॑तृ॒णाम् भुव॑नस्य । \newline
23. भुव॑नस्य॒ यो यो भुव॑नस्य॒ भुव॑नस्य॒ यः । \newline
24. य स्पति॒ष् पति॒र् यो य स्पतिः॑ । \newline
25. पति॑र् दे॒वम् दे॒वम् पति॒ष् पति॑र् दे॒वम् । \newline
26. दे॒वꣳ स॑वि॒तार(ग्म्॑) सवि॒तार॑म् दे॒वम् दे॒वꣳ स॑वि॒तार᳚म् । \newline
27. स॒वि॒तार॑ मभिमाति॒षाह॑ मभिमाति॒षाह(ग्म्॑) सवि॒तार(ग्म्॑) सवि॒तार॑ मभिमाति॒षाह᳚म् । \newline
28. अ॒भि॒मा॒ति॒षाह॒मित्य॑भिमाति - साह᳚म् । \newline
29. इ॒मं ॅय॒ज्ञ्ं ॅय॒ज्ञ् मि॒म मि॒मं ॅय॒ज्ञ्म् । \newline
30. य॒ज्ञ् म॒श्विना॒ ऽश्विना॑ य॒ज्ञ्ं ॅय॒ज्ञ् म॒श्विना᳚ । \newline
31. अ॒श्वि नो॒भोभा ऽश्विना॒ ऽश्विनो॒भा । \newline
32. उ॒भा बृह॒स्पति॒र् बृह॒स्पति॑ रु॒भोभा बृह॒स्पतिः॑ । \newline
33. बृह॒स्पति॑र् दे॒वा दे॒वा बृह॒स्पति॒र् बृह॒स्पति॑र् दे॒वाः । \newline
34. दे॒वाः पा᳚न्तु पान्तु दे॒वा दे॒वाः पा᳚न्तु । \newline
35. पा॒न्तु॒ यज॑मानं॒ ॅयज॑मानम् पान्तु पान्तु॒ यज॑मानम् । \newline
36. यज॑मानन् न्य॒र्त्थान् न्य॒र्त्थाद् यज॑मानं॒ ॅयज॑मानन् न्य॒र्त्थात् । \newline
37. न्य॒र्त्थादिति॑ नि - अ॒र्त्थात् । \newline
38. उ॒रु॒व्यचा॑ नो न उरु॒व्यचा॑ उरु॒व्यचा॑ नः । \newline
39. उ॒रु॒व्यचा॒इत्यु॑रु - व्यचाः᳚ । \newline
40. नो॒ म॒हि॒षो म॑हि॒षो नो॑ नो महि॒षः । \newline
41. म॒हि॒षः शर्म॒ शर्म॑ महि॒षो म॑हि॒षः शर्म॑ । \newline
42. शर्म॑ यꣳसद् यꣳस॒च्छर्म॒ शर्म॑ यꣳसत् । \newline
43. य॒(ग्म्॒)स॒ द॒स्मिन् न॒स्मिन्. य(ग्म्॑)सद् यꣳस द॒स्मिन्न् । \newline
44. अ॒स्मिन्. हवे॒ हवे॑ अ॒स्मिन् न॒स्मिन्. हवे᳚ । \newline
45. हवे॑ पुरुहू॒तः पु॑रुहू॒तो हवे॒ हवे॑ पुरुहू॒तः । \newline
46. पु॒रु॒हू॒तः पु॑रु॒क्षु पु॑रु॒क्षु पु॑रुहू॒तः पु॑रुहू॒तः पु॑रु॒क्षु । \newline
47. पु॒रु॒हू॒त इति॑ पुरु - हू॒तः । \newline
48. पु॒रु॒क्ष्विति॑ पुरु॒क्षु । \newline
49. स नो॑ नः॒ स स नः॑ । \newline
50. नः॒ प्र॒जायै᳚ प्र॒जायै॑ नो नः प्र॒जायै᳚ । \newline
51. प्र॒जायै॑ हर्यश्व हर्यश्व प्र॒जायै᳚ प्र॒जायै॑ हर्यश्व । \newline
52. प्र॒जाया॒ इति॑ प्र - जायै᳚ । \newline
53. ह॒र्य॒श्व॒ मृ॒ड॒य॒ मृ॒ड॒य॒ ह॒र्य॒श्व॒ ह॒र्य॒श्व॒ मृ॒ड॒य॒ । \newline
54. ह॒र्य॒श्वेति॑ हरि - अ॒श्व॒ । \newline
55. मृ॒ड॒ येन्द्रेन्द्र॑ मृडय मृड॒येन्द्र॑ । \newline
56. इन्द्र॒ मा मेन्द्रेन्द्र॒ मा । \newline
57. मा नो॑ नो॒ मा मा नः॑ । \newline

\textbf{Ghana Paata } \newline

1. अद॑ब्धो गो॒पा गो॒पा अद॑ब्धो॒ अद॑ब्धो गो॒पाः परि॒ परि॑ गो॒पा अद॑ब्धो॒ अद॑ब्धो गो॒पाः परि॑ । \newline
2. गो॒पाः परि॒ परि॑ गो॒पा गो॒पाः परि॑ पाहि पाहि॒ परि॑ गो॒पा गो॒पाः परि॑ पाहि । \newline
3. गो॒पा इति॑ गो - पाः । \newline
4. परि॑ पाहि पाहि॒ परि॒ परि॑ पाहि नो नः पाहि॒ परि॒ परि॑ पाहि नः । \newline
5. पा॒हि॒ नो॒ नः॒ पा॒हि॒ पा॒हि॒ न॒ स्त्वम् त्वम् नः॑ पाहि पाहि न॒ स्त्वम् । \newline
6. न॒ स्त्वम् त्वम् नो॑ न॒ स्त्वम् । \newline
7. त्वमिति॒ त्वम् । \newline
8. प्र॒त्यञ्चो॑ यन्तु यन्तु प्र॒त्यञ्चः॑ प्र॒त्यञ्चो॑ यन्तु नि॒गुतो॑ नि॒गुतो॑ यन्तु प्र॒त्यञ्चः॑ प्र॒त्यञ्चो॑ यन्तु नि॒गुतः॑ । \newline
9. य॒न्तु॒ नि॒गुतो॑ नि॒गुतो॑ यन्तु यन्तु नि॒गुतः॒ पुनः॒ पुन॑र् नि॒गुतो॑ यन्तु यन्तु नि॒गुतः॒ पुनः॑ । \newline
10. नि॒गुतः॒ पुनः॒ पुन॑र् नि॒गुतो॑ नि॒गुतः॒ पुन॒ स्ते ते पुन॑र् नि॒गुतो॑ नि॒गुतः॒ पुन॒ स्ते । \newline
11. नि॒गुत॒ इति॑ नि - गुतः॑ । \newline
12. पुन॒ स्ते ते पुनः॒ पुन॒ स्ते॑ ऽमा ऽमा ते पुनः॒ पुन॒ स्ते॑ ऽमा । \newline
13. ते॑ ऽमा ऽमा ते ते॑ ऽमैषा॑ मेषा म॒मा ते ते॑ ऽमैषा᳚म् । \newline
14. अ॒मैषा॑ मेषा म॒मा ऽमैषा᳚म् चि॒त्तम् चि॒त्त मे॑षा म॒मा ऽमैषा᳚म् चि॒त्तम् । \newline
15. ए॒षा॒म् चि॒त्तम् चि॒त्त मे॑षा मेषाम् चि॒त्तम् प्र॒बुधा᳚ प्र॒बुधा॑ चि॒त्त मे॑षा मेषाम् चि॒त्तम् प्र॒बुधा᳚ । \newline
16. चि॒त्तम् प्र॒बुधा᳚ प्र॒बुधा॑ चि॒त्तम् चि॒त्तम् प्र॒बुधा॒ वि वि प्र॒बुधा॑ चि॒त्तम् चि॒त्तम् प्र॒बुधा॒ वि । \newline
17. प्र॒बुधा॒ वि वि प्र॒बुधा᳚ प्र॒बुधा॒ वि ने॑शन् नेश॒द् वि प्र॒बुधा᳚ प्र॒बुधा॒ वि ने॑शत् । \newline
18. प्र॒बुधेति॑ प्र - बुधा᳚ । \newline
19. वि ने॑शन् नेश॒द् वि वि ने॑शत् । \newline
20. ने॒श॒दिति॑ नेशत् । \newline
21. धा॒ता धा॑तृ॒णाम् धा॑तृ॒णाम् धा॒ता धा॒ता धा॑तृ॒णाम् भुव॑नस्य॒ भुव॑नस्य धातृ॒णाम् धा॒ता धा॒ता धा॑तृ॒णाम् भुव॑नस्य । \newline
22. धा॒तृ॒णाम् भुव॑नस्य॒ भुव॑नस्य धातृ॒णाम् धा॑तृ॒णाम् भुव॑नस्य॒ यो यो भुव॑नस्य धातृ॒णाम् धा॑तृ॒णाम् भुव॑नस्य॒ यः । \newline
23. भुव॑नस्य॒ यो यो भुव॑नस्य॒ भुव॑नस्य॒ य स्पति॒ष् पति॒र् यो भुव॑नस्य॒ भुव॑नस्य॒ य स्पतिः॑ । \newline
24. यस्पति॒ष् पति॒र् यो य स्पति॑र् दे॒वम् दे॒वम् पति॒र् यो य स्पति॑र् दे॒वम् । \newline
25. पति॑र् दे॒वम् दे॒वम् पति॒ष् पति॑र् दे॒वꣳ स॑वि॒तार(ग्म्॑) सवि॒तार॑म् दे॒वम् पति॒ष् पति॑र् दे॒वꣳ स॑वि॒तार᳚म् । \newline
26. दे॒वꣳ स॑वि॒तार(ग्म्॑) सवि॒तार॑म् दे॒वम् दे॒वꣳ स॑वि॒तार॑ मभिमाति॒षाह॑ मभिमाति॒षाह(ग्म्॑) सवि॒तार॑म् दे॒वम् दे॒वꣳ स॑वि॒तार॑ मभिमाति॒षाह᳚म् । \newline
27. स॒वि॒तार॑ मभिमाति॒षाह॑ मभिमाति॒षाह(ग्म्॑) सवि॒तार(ग्म्॑) सवि॒तार॑ मभिमाति॒षाह᳚म् । \newline
28. अ॒भि॒मा॒ति॒षाह॒मित्य॑भिमाति - साह᳚म् । \newline
29. इ॒मं ॅय॒ज्ञ्ं ॅय॒ज्ञ् मि॒म मि॒मं ॅय॒ज्ञ् म॒श्विना॒ ऽश्विना॑ य॒ज्ञ् मि॒म मि॒मं ॅय॒ज्ञ् म॒श्विना᳚ । \newline
30. य॒ज्ञ् म॒श्विना॒ ऽश्विना॑ य॒ज्ञ्ं ॅय॒ज्ञ् म॒श्विनो॒भोभा ऽश्विना॑ य॒ज्ञ्ं ॅय॒ज्ञ् म॒श्विनो॒भा । \newline
31. अ॒श्विनो॒भोभा ऽश्विना॒ ऽश्विनो॒भा बृह॒स्पति॒र् बृह॒स्पति॑ रु॒भा ऽश्विना॒ ऽश्विनो॒भा बृह॒स्पतिः॑ । \newline
32. उ॒भा बृह॒स्पति॒र् बृह॒स्पति॑ रु॒भोभा बृह॒स्पति॑र् दे॒वा दे॒वा बृह॒स्पति॑ रु॒भोभा बृह॒स्पति॑र् दे॒वाः । \newline
33. बृह॒स्पति॑र् दे॒वा दे॒वा बृह॒स्पति॒र् बृह॒स्पति॑र् दे॒वाः पा᳚न्तु पान्तु दे॒वा बृह॒स्पति॒र् बृह॒स्पति॑र् दे॒वाः पा᳚न्तु । \newline
34. दे॒वाः पा᳚न्तु पान्तु दे॒वा दे॒वाः पा᳚न्तु॒ यज॑मानं॒ ॅयज॑मानम् पान्तु दे॒वा दे॒वाः पा᳚न्तु॒ यज॑मानम् । \newline
35. पा॒न्तु॒ यज॑मानं॒ ॅयज॑मानम् पान्तु पान्तु॒ यज॑मानं न्य॒र्त्थान् न्य॒र्त्थाद् यज॑मानम् पान्तु पान्तु॒ यज॑मानं न्य॒र्त्थात् । \newline
36. यज॑मानं न्य॒र्त्थान् न्य॒र्त्थाद् यज॑मानं॒ ॅयज॑मानं न्य॒र्त्थात् । \newline
37. न्य॒र्त्थादिति॑ नि - अ॒र्त्थात् । \newline
38. उ॒रु॒व्यचा॑ नो न उरु॒व्यचा॑ उरु॒व्यचा॑ नो महि॒षो म॑हि॒षो न॑ उरु॒व्यचा॑ उरु॒व्यचा॑ नो महि॒षः । \newline
39. उ॒रु॒व्यचा॒ इत्यु॑रु - व्यचाः᳚ । \newline
40. नो॒ म॒हि॒षो म॑हि॒षो नो॑ नो महि॒षः शर्म॒ शर्म॑ महि॒षो नो॑ नो महि॒षः शर्म॑ । \newline
41. म॒हि॒षः शर्म॒ शर्म॑ महि॒षो म॑हि॒षः शर्म॑ यꣳसद् यꣳस॒ च्छर्म॑ महि॒षो म॑हि॒षः शर्म॑ यꣳसत् । \newline
42. शर्म॑ यꣳसद् यꣳस॒ च्छर्म॒ शर्म॑ यꣳस द॒स्मिन् न॒स्मिन्. य(ग्म्॑)स॒ च्छर्म॒ शर्म॑ यꣳस द॒स्मिन्न् । \newline
43. य॒(ग्म्॒)स॒ द॒स्मिन् न॒स्मिन्. य(ग्म्॑)सद् यꣳस द॒स्मिन्. हवे॒ हवे॑ अ॒स्मिन्. य(ग्म्॑)सद् यꣳस द॒स्मिन्. हवे᳚ । \newline
44. अ॒स्मिन्. हवे॒ हवे॑ अ॒स्मिन् न॒स्मिन्. हवे॑ पुरुहू॒तः पु॑रुहू॒तो हवे॑ अ॒स्मिन् न॒स्मिन्. हवे॑ पुरुहू॒तः । \newline
45. हवे॑ पुरुहू॒तः पु॑रुहू॒तो हवे॒ हवे॑ पुरुहू॒तः पु॑रु॒क्षु पु॑रु॒क्षु पु॑रुहू॒तो हवे॒ हवे॑ पुरुहू॒तः पु॑रु॒क्षु । \newline
46. पु॒रु॒हू॒तः पु॑रु॒क्षु पु॑रु॒क्षु पु॑रुहू॒तः पु॑रुहू॒तः पु॑रु॒क्षु । \newline
47. पु॒रु॒हू॒त इति॑ पुरु - हू॒तः । \newline
48. पु॒रु॒क्ष्विति॑ पुरु॒क्षु । \newline
49. स नो॑ नः॒ स स नः॑ प्र॒जायै᳚ प्र॒जायै॑ नः॒ स स नः॑ प्र॒जायै᳚ । \newline
50. नः॒ प्र॒जायै᳚ प्र॒जायै॑ नो नः प्र॒जायै॑ हर्यश्व हर्यश्व प्र॒जायै॑ नो नः प्र॒जायै॑ हर्यश्व । \newline
51. प्र॒जायै॑ हर्यश्व हर्यश्व प्र॒जायै᳚ प्र॒जायै॑ हर्यश्व मृडय मृडय हर्यश्व प्र॒जायै᳚ प्र॒जायै॑ हर्यश्व मृडय । \newline
52. प्र॒जाया॒ इति॑ प्र - जायै᳚ । \newline
53. ह॒र्य॒श्व॒ मृ॒ड॒य॒ मृ॒ड॒य॒ ह॒र्य॒श्व॒ ह॒र्य॒श्व॒ मृ॒ड॒ येन्द्रेन्द्र॑ मृडय हर्यश्व हर्यश्व मृड॒येन्द्र॑ । \newline
54. ह॒र्य॒श्वेति॑ हरि - अ॒श्व॒ । \newline
55. मृ॒ड॒ येन्द्रेन्द्र॑ मृडय मृड॒येन्द्र॒ मा मेन्द्र॑ मृडय मृड॒येन्द्र॒ मा । \newline
56. इन्द्र॒ मा मेन्द्रेन्द्र॒ मा नो॑ नो॒ मेन्द्रेन्द्र॒ मा नः॑ । \newline
57. मा नो॑ नो॒ मा मा नो॑ रीरिषो रीरिषो नो॒ मा मा नो॑ रीरिषः । \newline
\pagebreak
\markright{ TS 4.7.14.4  \hfill https://www.vedavms.in \hfill}

\section{ TS 4.7.14.4 }

\textbf{TS 4.7.14.4 } \newline
\textbf{Samhita Paata} \newline

नो॑ रीरिषो॒ मा परा॑ दाः ॥ ये नः॑ स॒पत्ना॒ अप॒ते भ॑वन्त्विन्द्रा॒-ग्निभ्या॒मव॑ बाधामहे॒ तान् ।वस॑वो रु॒द्रा आ॑दि॒त्या उ॑परि॒ स्पृशं॑ मो॒ग्रं चेत्ता॑रमधि रा॒जम॑क्रन्न् ॥ अ॒र्वाञ्च॒ मिन्द्र॑म॒मुतो॑ हवामहे॒ यो गो॒जिद्-ध॑न॒-जिद॑श्व॒-जिद्यः । इ॒मन्नो॑ य॒ज्ञ्ं ॅवि॑ह॒वे जु॑षस्वा॒स्य कु॑र्मो हरिवो मे॒दिनं॑ त्वा ॥ \newline

\textbf{Pada Paata} \newline

नः॒ । री॒रि॒षः॒ । मा । परेति॑ । दाः॒ ॥ ये । नः॒ । स॒पत्नाः᳚ । अपेति॑ । ते । भ॒व॒न्तु॒ । इ॒न्द्रा॒ग्निभ्या॒मिती᳚न्द्रा॒ग्नि-भ्या॒म् । अवेति॑ । बा॒धा॒म॒हे॒ । तान् ॥ वस॑वः । रु॒द्राः । आ॒दि॒त्याः । उ॒प॒रि॒स्पृश॒मित्यु॑परि - स्पृश᳚म् । मा॒ । उ॒ग्रम् । चेत्ता॑रम् । अ॒धि॒रा॒जमित्य॑धि - रा॒जम् । अ॒क्र॒न्न् ॥ अ॒र्वाञ्च᳚म् । इन्द्र᳚म् । अ॒मुतः॑ । ह॒वा॒म॒हे॒ । यः । गो॒जिदिति॑ गो - जित् । ध॒न॒जिदिति॑ धन - जित् । अ॒श्व॒जिदित्य॑श्व - जित् । यः ॥ इ॒मम् । नः॒ । य॒ज्ञ्म् । वि॒ह॒व इति॑ वि - ह॒वे । जु॒ष॒स्व॒ । अ॒स्य । कु॒र्मः॒ । ह॒रि॒व॒ इति॑ हरि - वः॒ । मे॒दिन᳚म् । त्वा॒ ॥  \newline


\textbf{Krama Paata} \newline

नो॒ री॒रि॒षः॒ । री॒रि॒षो॒ मा । मा परा᳚ । परा॑ दाः । दा॒ इति॑ दाः ॥ ये नः॑ । नः॒ स॒पत्नाः᳚ । स॒पत्ना॒ अप॑ । अप॒ ते । ते भ॑वन्तु । भ॒व॒न्त्वि॒न्द्रा॒ग्निभ्या᳚म् । इ॒न्द्रा॒ग्निभ्या॒मव॑ । इ॒न्द्रा॒ग्निभ्या॒मिती᳚न्द्रा॒ग्नि - भ्या॒म् । अव॑ बाधामहे । बा॒धा॒म॒हे॒ तान् । तानिति॒ तान् ॥ वस॑वो रु॒द्राः । रु॒द्रा आ॑दि॒त्याः । आ॒दि॒त्या उ॑परि॒स्पृश᳚म् । उ॒प॒रि॒स्पृश॑म् मा । उ॒प॒रि॒स्पृश॒मित्यु॑परि - स्पृश᳚म् । मो॒ग्रम् । उ॒ग्रम् चेत्ता॑रम् । चेत्ता॑रमधिरा॒जम् । अ॒धि॒रा॒जम॑क्रन्न् । अ॒धि॒रा॒जमित्य॑धि - रा॒जम् । अ॒क्र॒न्नित्य॑क्रन्न् ॥ अ॒र्वाञ्च॒मिन्द्र᳚म् । इन्द्र॑म॒मुतः॑ । अ॒मुतो॑ हवामहे । ह॒वा॒म॒हे॒ यः । यो गो॒जित् । गो॒जिद् ध॑न॒जित् । गो॒जिदिति॑ गो - जित् । ध॒न॒जिद॑श्व॒जित् । ध॒न॒जिदिति॑ धन - जित् । अ॒श्व॒जिद् यः । अ॒श्व॒जिदित्य॑श्व - जित् । य इति॒ यः ॥ इ॒मम् नः॑ । नो॒ य॒ज्ञ्म् । य॒ज्ञ्म् ॅवि॑ह॒वे । वि॒ह॒वे जु॑षस्व । वि॒ह॒व इति॑ वि - ह॒वे । जु॒ष॒स्वा॒स्य । अ॒स्य कु॑र्मः । कु॒र्मो॒ ह॒रि॒वः॒ । ह॒रि॒वो॒ मे॒दिन᳚म् । ह॒रि॒व॒ इति॑ हरि - वः॒ । मे॒दिन॑म् त्वा । त्वेति॑ त्वा । \newline

\textbf{Jatai Paata} \newline

1. नो॒ री॒रि॒षो॒ री॒रि॒षो॒ नो॒ नो॒ री॒रि॒षः॒ । \newline
2. री॒रि॒षो॒ मा मा री॑रिषो रीरिषो॒ मा । \newline
3. मा परा॒ परा॒ मा मा परा᳚ । \newline
4. परा॑ दा दाः॒ परा॒ परा॑ दाः । \newline
5. दा॒ इति॑ दाः । \newline
6. ये नो॑ नो॒ ये ये नः॑ । \newline
7. नः॒ स॒पत्नाः᳚ स॒पत्ना॑ नो नः स॒पत्नाः᳚ । \newline
8. स॒पत्ना॒ अपाप॑ स॒पत्नाः᳚ स॒पत्ना॒ अप॑ । \newline
9. अप॒ ते ते अपाप॒ ते । \newline
10. ते भ॑वन्तु भवन्तु॒ ते ते भ॑वन्तु । \newline
11. भ॒व॒ न्त्वि॒न्द्रा॒ग्निभ्या॑ मिन्द्रा॒ग्निभ्या᳚म् भवन्तु भव न्त्विन्द्रा॒ग्निभ्या᳚म् । \newline
12. इ॒न्द्रा॒ग्निभ्या॒ मवा वे᳚न्द्रा॒ग्निभ्या॑ मिन्द्रा॒ग्निभ्या॒ मव॑ । \newline
13. इ॒न्द्रा॒ग्निभ्या॒मिती᳚न्द्रा॒ग्नि - भ्या॒म् । \newline
14. अव॑ बाधामहे बाधाम॒हे ऽवाव॑ बाधामहे । \newline
15. बा॒धा॒म॒हे॒ ताꣳ तान् बा॑धामहे बाधामहे॒ तान् । \newline
16. तानिति॒ तान् । \newline
17. वस॑वो रु॒द्रा रु॒द्रा वस॑वो॒ वस॑वो रु॒द्राः । \newline
18. रु॒द्रा आ॑दि॒त्या आ॑दि॒त्या रु॒द्रा रु॒द्रा आ॑दि॒त्याः । \newline
19. आ॒दि॒त्या उ॑परि॒स्पृश॑ मुपरि॒स्पृश॑ मादि॒त्या आ॑दि॒त्या उ॑परि॒स्पृश᳚म् । \newline
20. उ॒प॒रि॒स्पृश॑म् मा मोपरि॒स्पृश॑ मुपरि॒स्पृश॑म् मा । \newline
21. उ॒प॒रि॒स्पृश॒मित्यु॑परि - स्पृश᳚म् । \newline
22. मो॒ग्र मु॒ग्रम् मा॑ मो॒ग्रम् । \newline
23. उ॒ग्रम् चेत्ता॑र॒म् चेत्ता॑र मु॒ग्र मु॒ग्रम् चेत्ता॑रम् । \newline
24. चेत्ता॑र मधिरा॒ज म॑धिरा॒जम् चेत्ता॑र॒म् चेत्ता॑र मधिरा॒जम् । \newline
25. अ॒धि॒रा॒ज म॑क्रन् नक्रन् नधिरा॒ज म॑धिरा॒ज म॑क्रन्न् । \newline
26. अ॒धि॒रा॒जमित्य॑धि - रा॒जम् । \newline
27. अ॒क्र॒न्नित्य॑क्रन्न् । \newline
28. अ॒र्वाञ्च॒ मिन्द्र॒ मिन्द्र॑ म॒र्वाञ्च॑ म॒र्वाञ्च॒ मिन्द्र᳚म् । \newline
29. इन्द्र॑ म॒मुतो॑ अ॒मुत॒ इन्द्र॒ मिन्द्र॑ म॒मुतः॑ । \newline
30. अ॒मुतो॑ हवामहे हवामहे अ॒मुतो॑ अ॒मुतो॑ हवामहे । \newline
31. ह॒वा॒म॒हे॒ यो यो ह॑वामहे हवामहे॒ यः । \newline
32. यो गो॒जिद् गो॒जिद् यो यो गो॒जित् । \newline
33. गो॒जिद् ध॑न॒जिद् ध॑न॒जिद् गो॒जिद् गो॒जिद् ध॑न॒जित् । \newline
34. गो॒जिदिति॑ गो - जित् । \newline
35. ध॒न॒जि द॑श्व॒जि द॑श्व॒जिद् ध॑न॒जिद् ध॑न॒जि द॑श्व॒जित् । \newline
36. ध॒न॒जिदिति॑ धन - जित् । \newline
37. अ॒श्व॒जिद् यो यो अ॑श्व॒जि द॑श्व॒जिद् यः । \newline
38. अ॒श्व॒जिदित्य॑श्व - जित् । \newline
39. य इति॒ यः । \newline
40. इ॒मम् नो॑ न इ॒म मि॒मम् नः॑ । \newline
41. नो॒ य॒ज्ञ्ं ॅय॒ज्ञ्म् नो॑ नो य॒ज्ञ्म् । \newline
42. य॒ज्ञ्ं ॅवि॑ह॒वे वि॑ह॒वे य॒ज्ञ्ं ॅय॒ज्ञ्ं ॅवि॑ह॒वे । \newline
43. वि॒ह॒वे जु॑षस्व जुषस्व विह॒वे वि॑ह॒वे जु॑षस्व । \newline
44. वि॒ह॒व इति॑ वि - ह॒वे । \newline
45. जु॒ष॒स्वा॒ स्यास्य जु॑षस्व जुषस्वा॒स्य । \newline
46. अ॒स्य कु॑र्मः कुर्मो अ॒स्यास्य कु॑र्मः । \newline
47. कु॒र्मो॒ ह॒रि॒वो॒ ह॒रि॒वः॒ कु॒र्मः॒ कु॒र्मो॒ ह॒रि॒वः॒ । \newline
48. ह॒रि॒वो॒ मे॒दिन॑म् मे॒दिन(ग्म्॑) हरिवो हरिवो मे॒दिन᳚म् । \newline
49. ह॒रि॒व॒ इति॑ हरि - वः॒ । \newline
50. मे॒दिन॑म् त्वा त्वा मे॒दिन॑म् मे॒दिन॑म् त्वा । \newline
51. त्वेति॑ त्वा । \newline

\textbf{Ghana Paata } \newline

1. नो॒ री॒रि॒षो॒ री॒रि॒षो॒ नो॒ नो॒ री॒रि॒षो॒ मा मा री॑रिषो नो नो रीरिषो॒ मा । \newline
2. री॒रि॒षो॒ मा मा री॑रिषो रीरिषो॒ मा परा॒ परा॒ मा री॑रिषो रीरिषो॒ मा परा᳚ । \newline
3. मा परा॒ परा॒ मा मा परा॑ दा दाः॒ परा॒ मा मा परा॑ दाः । \newline
4. परा॑ दा दाः॒ परा॒ परा॑ दाः । \newline
5. दा॒ इति॑ दाः । \newline
6. ये नो॑ नो॒ ये ये नः॑ स॒पत्नाः᳚ स॒पत्ना॑ नो॒ ये ये नः॑ स॒पत्नाः᳚ । \newline
7. नः॒ स॒पत्नाः᳚ स॒पत्ना॑ नो नः स॒पत्ना॒ अपाप॑ स॒पत्ना॑ नो नः स॒पत्ना॒ अप॑ । \newline
8. स॒पत्ना॒ अपाप॑ स॒पत्नाः᳚ स॒पत्ना॒ अप॒ ते ते अप॑ स॒पत्नाः᳚ स॒पत्ना॒ अप॒ ते । \newline
9. अप॒ ते ते अपाप॒ ते भ॑वन्तु भवन्तु॒ ते अपाप॒ ते भ॑वन्तु । \newline
10. ते भ॑वन्तु भवन्तु॒ ते ते भ॑वन् त्विन्द्रा॒ग्निभ्या॑ मिन्द्रा॒ग्निभ्या᳚म् भवन्तु॒ ते ते भ॑वन् त्विन्द्रा॒ग्निभ्या᳚म् । \newline
11. भ॒व॒न् त्वि॒न्द्रा॒ग्निभ्या॑ मिन्द्रा॒ग्निभ्या᳚म् भवन्तु भवन् त्विन्द्रा॒ग्निभ्या॒ मवावे᳚न्द्रा॒ग्निभ्या᳚म् भवन्तु भवन् त्विन्द्रा॒ग्निभ्या॒ मव॑ । \newline
12. इ॒न्द्रा॒ग्निभ्या॒ मवावे᳚न्द्रा॒ग्निभ्या॑ मिन्द्रा॒ग्निभ्या॒ मव॑ बाधामहे बाधाम॒हे ऽवे᳚न्द्रा॒ग्निभ्या॑ मिन्द्रा॒ग्निभ्या॒ मव॑ बाधामहे । \newline
13. इ॒न्द्रा॒ग्निभ्या॒मिती᳚न्द्रा॒ग्नि - भ्या॒म् । \newline
14. अव॑ बाधामहे बाधाम॒हे ऽवाव॑ बाधामहे॒ ताꣳ स्तान् बा॑धाम॒हे ऽवाव॑ बाधामहे॒ तान् । \newline
15. बा॒धा॒म॒हे॒ ताꣳ स्तान् बा॑धामहे बाधामहे॒ तान् । \newline
16. तानिति॒ तान् । \newline
17. वस॑वो रु॒द्रा रु॒द्रा वस॑वो॒ वस॑वो रु॒द्रा आ॑दि॒त्या आ॑दि॒त्या रु॒द्रा वस॑वो॒ वस॑वो रु॒द्रा आ॑दि॒त्याः । \newline
18. रु॒द्रा आ॑दि॒त्या आ॑दि॒त्या रु॒द्रा रु॒द्रा आ॑दि॒त्या उ॑परि॒स्पृश॑ मुपरि॒स्पृश॑ मादि॒त्या रु॒द्रा रु॒द्रा आ॑दि॒त्या उ॑परि॒स्पृश᳚म् । \newline
19. आ॒दि॒त्या उ॑परि॒स्पृश॑ मुपरि॒स्पृश॑ मादि॒त्या आ॑दि॒त्या उ॑परि॒स्पृश॑म् मा मोपरि॒स्पृश॑ मादि॒त्या आ॑दि॒त्या उ॑परि॒स्पृश॑म् मा । \newline
20. उ॒प॒रि॒स्पृश॑म् मा मोपरि॒स्पृश॑ मुपरि॒स्पृश॑म् मो॒ग्र मु॒ग्रम् मो॑परि॒स्पृश॑ मुपरि॒स्पृश॑म् मो॒ग्रम् । \newline
21. उ॒प॒रि॒स्पृश॒मित्यु॑परि - स्पृश᳚म् । \newline
22. मो॒ग्र मु॒ग्रम् मा॑ मो॒ग्रम् चेत्ता॑र॒म् चेत्ता॑र मु॒ग्रम् मा॑ मो॒ग्रम् चेत्ता॑रम् । \newline
23. उ॒ग्रम् चेत्ता॑र॒म् चेत्ता॑र मु॒ग्र मु॒ग्रम् चेत्ता॑र मधिरा॒ज म॑धिरा॒जम् चेत्ता॑र मु॒ग्र मु॒ग्रम् चेत्ता॑र मधिरा॒जम् । \newline
24. चेत्ता॑र मधिरा॒ज म॑धिरा॒जम् चेत्ता॑र॒म् चेत्ता॑र मधिरा॒ज म॑क्रन् नक्रन् नधिरा॒जम् चेत्ता॑र॒म् चेत्ता॑र मधिरा॒ज म॑क्रन्न् । \newline
25. अ॒धि॒रा॒ज म॑क्रन् नक्रन् नधिरा॒ज म॑धिरा॒ज म॑क्रन्न् । \newline
26. अ॒धि॒रा॒जमित्य॑धि - रा॒जम् । \newline
27. अ॒क्र॒न्नित्य॑क्रन्न् । \newline
28. अ॒र्वाञ्च॒ मिन्द्र॒ मिन्द्र॑ म॒र्वाञ्च॑ म॒र्वाञ्च॒ मिन्द्र॑ म॒मुतो॑ अ॒मुत॒ इन्द्र॑ म॒र्वाञ्च॑ म॒र्वाञ्च॒ मिन्द्र॑ म॒मुतः॑ । \newline
29. इन्द्र॑ म॒मुतो॑ अ॒मुत॒ इन्द्र॒ मिन्द्र॑ म॒मुतो॑ हवामहे हवामहे अ॒मुत॒ इन्द्र॒ मिन्द्र॑ म॒मुतो॑ हवामहे । \newline
30. अ॒मुतो॑ हवामहे हवामहे अ॒मुतो॑ अ॒मुतो॑ हवामहे॒ यो यो ह॑वामहे अ॒मुतो॑ अ॒मुतो॑ हवामहे॒ यः । \newline
31. ह॒वा॒म॒हे॒ यो यो ह॑वामहे हवामहे॒ यो गो॒जिद् गो॒जिद् यो ह॑वामहे हवामहे॒ यो गो॒जित् । \newline
32. यो गो॒जिद् गो॒जिद् यो यो गो॒जिद् ध॑न॒जिद् ध॑न॒जिद् गो॒जिद् यो यो गो॒जिद् ध॑न॒जित् । \newline
33. गो॒जिद् ध॑न॒जिद् ध॑न॒जिद् गो॒जिद् गो॒जिद् ध॑न॒जि द॑श्व॒जि द॑श्व॒जिद् ध॑न॒जिद् गो॒जिद् गो॒जिद् ध॑न॒जि द॑श्व॒जित् । \newline
34. गो॒जिदिति॑ गो - जित् । \newline
35. ध॒न॒जि द॑श्व॒जि द॑श्व॒जिद् ध॑न॒जिद् ध॑न॒जि द॑श्व॒जिद् यो यो अ॑श्व॒जिद् ध॑न॒जिद् ध॑न॒जि द॑श्व॒जिद् यः । \newline
36. ध॒न॒जिदिति॑ धन - जित् । \newline
37. अ॒श्व॒जिद् यो यो अ॑श्व॒जि द॑श्व॒जिद् यः । \newline
38. अ॒श्व॒जिदित्य॑श्व - जित् । \newline
39. य इति॒ यः । \newline
40. इ॒मम् नो॑ न इ॒म मि॒मम् नो॑ य॒ज्ञ्ं ॅय॒ज्ञ्म् न॑ इ॒म मि॒मम् नो॑ य॒ज्ञ्म् । \newline
41. नो॒ य॒ज्ञ्ं ॅय॒ज्ञ्म् नो॑ नो य॒ज्ञ्ं ॅवि॑ह॒वे वि॑ह॒वे य॒ज्ञ्म् नो॑ नो य॒ज्ञ्ं ॅवि॑ह॒वे । \newline
42. य॒ज्ञ्ं ॅवि॑ह॒वे वि॑ह॒वे य॒ज्ञ्ं ॅय॒ज्ञ्ं ॅवि॑ह॒वे जु॑षस्व जुषस्व विह॒वे य॒ज्ञ्ं ॅय॒ज्ञ्ं ॅवि॑ह॒वे जु॑षस्व । \newline
43. वि॒ह॒वे जु॑षस्व जुषस्व विह॒वे वि॑ह॒वे जु॑षस्वा॒ स्यास्य जु॑षस्व विह॒वे वि॑ह॒वे जु॑षस्वा॒स्य । \newline
44. वि॒ह॒व इति॑ वि - ह॒वे । \newline
45. जु॒ष॒स्वा॒ स्यास्य जु॑षस्व जुषस्वा॒स्य कु॑र्मः कुर्मो अ॒स्य जु॑षस्व जुषस्वा॒स्य कु॑र्मः । \newline
46. अ॒स्य कु॑र्मः कुर्मो अ॒स्यास्य कु॑र्मो हरिवो हरिवः कुर्मो अ॒स्यास्य कु॑र्मो हरिवः । \newline
47. कु॒र्मो॒ ह॒रि॒वो॒ ह॒रि॒वः॒ कु॒र्मः॒ कु॒र्मो॒ ह॒रि॒वो॒ मे॒दिन॑म् मे॒दिन(ग्म्॑) हरिवः कुर्मः कुर्मो हरिवो मे॒दिन᳚म् । \newline
48. ह॒रि॒वो॒ मे॒दिन॑म् मे॒दिन(ग्म्॑) हरिवो हरिवो मे॒दिन॑म् त्वा त्वा मे॒दिन(ग्म्॑) हरिवो हरिवो मे॒दिन॑म् त्वा । \newline
49. ह॒रि॒व॒ इति॑ हरि - वः॒ । \newline
50. मे॒दिन॑म् त्वा त्वा मे॒दिन॑म् मे॒दिन॑म् त्वा । \newline
51. त्वेति॑ त्वा । \newline
\pagebreak
\markright{ TS 4.7.15.1  \hfill https://www.vedavms.in \hfill}

\section{ TS 4.7.15.1 }

\textbf{TS 4.7.15.1 } \newline
\textbf{Samhita Paata} \newline

अ॒ग्नेर्म॑न्वे प्रथ॒मस्य॒ प्रचे॑तसो॒ यं पाञ्च॑जन्यं ब॒हवः॑ समि॒न्धते᳚ । विश्व॑स्यां ॅवि॒शि प्र॑विविशि॒वाꣳ स॑मीमहे॒ स नो॑ मुञ्च॒त्वꣳ ह॑सः ॥ यस्ये॒दं प्रा॒णन्नि॑मि॒षद् यदेज॑ति॒ यस्य॑ जा॒तं जन॑मानं च॒ केव॑लं । स्तौम्य॒ग्निं ना॑थि॒तो जो॑हवीमि॒ स नो॑ मुञ्च॒त्वꣳ ह॑सः ॥ इन्द्र॑स्य मन्ये प्रथ॒मस्य॒ प्रचे॑तसो वृत्र॒घ्नः स्तोमा॒ उप॒ मामु॒पागुः॑ । यो दा॒शुषः॑ सु॒कृतो॒ हव॒मुप॒ गन्ता॒ - [  ] \newline

\textbf{Pada Paata} \newline

अ॒ग्नेः । म॒न्वे॒ । प्र॒थ॒मस्य॑ । प्रचे॑तस॒ इति॒ प्र - चे॒त॒सः॒ । यम् । पाञ्च॑जन्य॒मिति॒ पाञ्च॑ - ज॒न्य॒म् । ब॒हवः॑ । स॒मि॒न्धत॒ इति॑ सम् - इ॒न्धते᳚ ॥ विश्व॑स्याम् । वि॒शि । प्र॒वि॒वि॒शि॒वाꣳस॒मिति॑ प्र - वि॒वि॒शि॒वाꣳस᳚म् । ई॒म॒हे॒ । सः । नः॒ । मु॒ञ्च॒तु॒ । अꣳह॑सः ॥ यस्य॑ । इ॒दम् । प्रा॒णदिति॑ प्र - अ॒नत् । नि॒मि॒षदिति॑ नि - मि॒षत् । यत् । एज॑ति । यस्य॑ । जा॒तम् । जन॑मानम् । च॒ । केव॑लम् ॥ स्तौमि॑ । अ॒ग्निम् । ना॒थि॒तः । जो॒ह॒वी॒मि॒ । सः । नः॒ । मु॒ञ्च॒तु॒ । अꣳह॑सः ॥ इन्द्र॑स्य । म॒न्ये॒ । प्र॒थ॒मस्य॑ । प्रचे॑तस॒ इति॒ प्र - चे॒त॒सः॒ । वृ॒त्र॒घ्न इति॑ वृत्र-घ्नः । स्तोमाः᳚ । उपेति॑ । माम् । उ॒पागु॒रित्यु॑प - आगुः॑ ॥ यः । दा॒शुषः॑ । सु॒कृत॒ इति॑ सु - कृतः॑ । हव᳚म् । उपेति॑ । गन्ता᳚ ।  \newline


\textbf{Krama Paata} \newline

अ॒ग्नेर् म॑न्वे । म॒न्वे॒ प्र॒थ॒मस्य॑ । प्र॒थ॒मस्य॒ प्रचे॑तसः । प्रचे॑तसो॒ यम् । प्रचे॑तस॒ इति॒ प्र - चे॒त॒सः॒ । यम् पाञ्च॑जन्यम् । पाञ्च॑जन्यम् ब॒हवः॑ । पाञ्च॑जन्य॒मिति॒ पाञ्च॑ - ज॒न्य॒॒म् । ब॒हवः॑ समि॒न्धते᳚ । स॒मि॒न्धत॒ इति॑ सम् - इ॒न्धते᳚ ॥ विश्व॑स्याम् ॅवि॒शि । वि॒शि प्र॑विविशि॒वाꣳस᳚म् । प्र॒वि॒वि॒शि॒वाꣳस॑मीमहे । प्र॒वि॒वि॒शि॒वाꣳस॒मिति॑ प्र - वि॒वि॒शि॒वाꣳस᳚म् । ई॒म॒हे॒ सः । स नः॑ । नो॒ मु॒ञ्च॒तु॒ । मु॒ञ्च॒त्वꣳह॑सः । अꣳह॑स॒ इत्यꣳह॑सः ॥ यस्ये॒दम् । इ॒दम् प्रा॒णत् । प्रा॒णन्नि॑मि॒षत् । प्रा॒णदिति॑ प्र - अ॒नत् । नि॒मि॒षद् यत् । नि॒मि॒षदिति॑ नि - मि॒षत् । यदेज॑ति । एज॑ति॒ यस्य॑ । यस्य॑ जा॒तम् । जा॒तम् जन॑मानम् । जन॑मानम् च । च॒ केव॑लम् । केव॑ल॒मिति॒ केव॑लम् ॥ स्तौम्य॒ग्निम् । अ॒ग्निम् ना॑थि॒तः । ना॒थि॒तो जो॑हवीमि । जो॒ह॒वी॒मि॒ सः । स नः॑ । नो॒ मु॒ञ्च॒तु॒ । मु॒ञ्च॒त्वꣳह॑सः । अꣳह॑स॒ इत्यꣳह॑सः ॥ इन्द्र॑स्य मन्ये । म॒न्ये॒ प्र॒थ॒मस्य॑ । प्र॒थ॒मस्य॒ प्रचे॑तसः । प्रचे॑तसो वृत्र॒घ्नः । प्रचे॑तस॒ इति॒ प्र - चे॒त॒सः॒ । वृ॒त्र॒घ्नः स्तोमाः᳚ । वृ॒त्र॒घ्न इति॑ वृत्र - घ्नः । स्तोमा॒ उप॑ । उप॒ माम् । मामु॒पागुः॑ । उ॒पागु॒रित्यु॑प - आगुः॑ । यो दा॒शुषः॑ । दा॒शुषः॑ सु॒कृतः॑ । सु॒कृतो॒ हव᳚म् । सु॒कृत॒ इति॑ सु - कृतः॑ । हव॒मुप॑ । उप॒ गन्ता᳚ । गन्ता॒ सः \newline

\textbf{Jatai Paata} \newline

1. अ॒ग्नेर् म॑न्वे मन्वे अ॒ग्ने र॒ग्नेर् म॑न्वे । \newline
2. म॒न्वे॒ प्र॒थ॒मस्य॑ प्रथ॒मस्य॑ मन्वे मन्वे प्रथ॒मस्य॑ । \newline
3. प्र॒थ॒मस्य॒ प्रचे॑तसः॒ प्रचे॑तसः प्रथ॒मस्य॑ प्रथ॒मस्य॒ प्रचे॑तसः । \newline
4. प्रचे॑तसो॒ यं ॅयम् प्रचे॑तसः॒ प्रचे॑तसो॒ यम् । \newline
5. प्रचे॑तस॒ इति॒ प्र - चे॒त॒सः॒ । \newline
6. यम् पाञ्च॑जन्य॒म् पाञ्च॑जन्यं॒ ॅयं ॅयम् पाञ्च॑जन्यम् । \newline
7. पाञ्च॑जन्यम् ब॒हवो॑ ब॒हवः॒ पाञ्च॑जन्य॒म् पाञ्च॑जन्यम् ब॒हवः॑ । \newline
8. पाञ्च॑जन्य॒मिति॒ पाञ्च॑ - ज॒न्य॒म् । \newline
9. ब॒हवः॑ समि॒न्धते॑ समि॒न्धते॑ ब॒हवो॑ ब॒हवः॑ समि॒न्धते᳚ । \newline
10. स॒मि॒न्धत॒ इति॑ सम् - इ॒न्धते᳚ । \newline
11. विश्व॑स्यां ॅवि॒शि वि॒शि विश्व॑स्यां॒ ॅविश्व॑स्यां ॅवि॒शि । \newline
12. वि॒शि प्र॑विविशि॒वाꣳस॑म् प्रविविशि॒वाꣳसं॑ ॅवि॒शि वि॒शि प्र॑विविशि॒वाꣳस᳚म् । \newline
13. प्र॒वि॒वि॒शि॒वाꣳस॑ मीमह ईमहे प्रविविशि॒वाꣳस॑म् प्रविविशि॒वाꣳस॑ मीमहे । \newline
14. प्र॒वि॒वि॒शि॒वाꣳस॒मिति॑ प्र - वि॒वि॒शि॒वाꣳस᳚म् । \newline
15. ई॒म॒हे॒ स स ई॑मह ईमहे॒ सः । \newline
16. स नो॑ नः॒ स स नः॑ । \newline
17. नो॒ मु॒ञ्च॒तु॒ मु॒ञ्च॒तु॒ नो॒ नो॒ मु॒ञ्च॒तु॒ । \newline
18. मु॒ञ्च॒ त्वꣳह॑सो॒ अꣳह॑सो मुञ्चतु मुञ्च॒ त्वꣳह॑सः । \newline
19. अꣳह॑स॒ इत्यꣳह॑सः । \newline
20. यस्ये॒द मि॒दं ॅयस्य॒ यस्ये॒दम् । \newline
21. इ॒दम् प्रा॒णत् प्रा॒ण दि॒द मि॒दम् प्रा॒णत् । \newline
22. प्रा॒णन् नि॑मि॒षन् नि॑मि॒षत् प्रा॒णत् प्रा॒णन् नि॑मि॒षत् । \newline
23. प्रा॒णदिति॑ प्र - अ॒नत् । \newline
24. नि॒मि॒षद् यद् यन् नि॑मि॒षन् नि॑मि॒षद् यत् । \newline
25. नि॒मि॒षदिति॑ नि - मि॒षत् । \newline
26. यदेज॒ त्येज॑ति॒ यद् यदेज॑ति । \newline
27. एज॑ति॒ यस्य॒ यस्यैज॒ त्येज॑ति॒ यस्य॑ । \newline
28. यस्य॑ जा॒तम् जा॒तं ॅयस्य॒ यस्य॑ जा॒तम् । \newline
29. जा॒तम् जन॑मान॒म् जन॑मानम् जा॒तम् जा॒तम् जन॑मानम् । \newline
30. जन॑मानम् च च॒ जन॑मान॒म् जन॑मानम् च । \newline
31. च॒ केव॑ल॒म् केव॑लम् च च॒ केव॑लम् । \newline
32. केव॑ल॒मिति॒ केव॑लम् । \newline
33. स्तौ म्य॒ग्नि म॒ग्निꣳ स्तौमि॒ स्तौ म्य॒ग्निम् । \newline
34. अ॒ग्निम् ना॑थि॒तो ना॑थि॒तो अ॒ग्नि म॒ग्निम् ना॑थि॒तः । \newline
35. ना॒थि॒तो जो॑हवीमि जोहवीमि नाथि॒तो ना॑थि॒तो जो॑हवीमि । \newline
36. जो॒ह॒वी॒मि॒ स स जो॑हवीमि जोहवीमि॒ सः । \newline
37. स नो॑ नः॒ स स नः॑ । \newline
38. नो॒ मु॒ञ्च॒तु॒ मु॒ञ्च॒तु॒ नो॒ नो॒ मु॒ञ्च॒तु॒ । \newline
39. मु॒ञ्च॒ त्वꣳह॑सो॒ अꣳह॑सो मुञ्चतु मुञ्च॒ त्वꣳह॑सः । \newline
40. अꣳह॑स॒ इत्यꣳह॑सः । \newline
41. इन्द्र॑स्य मन्ये मन्य॒ इन्द्र॒ स्येन्द्र॑स्य मन्ये । \newline
42. म॒न्ये॒ प्र॒थ॒मस्य॑ प्रथ॒मस्य॑ मन्ये मन्ये प्रथ॒मस्य॑ । \newline
43. प्र॒थ॒मस्य॒ प्रचे॑तसः॒ प्रचे॑तसः प्रथ॒मस्य॑ प्रथ॒मस्य॒ प्रचे॑तसः । \newline
44. प्रचे॑तसो वृत्र॒घ्नो वृ॑त्र॒घ्नः प्रचे॑तसः॒ प्रचे॑तसो वृत्र॒घ्नः । \newline
45. प्रचे॑तस॒ इति॒ प्र - चे॒त॒सः॒ । \newline
46. वृ॒त्र॒घ्नः स्तोमाः॒ स्तोमा॑ वृत्र॒घ्नो वृ॑त्र॒घ्नः स्तोमाः᳚ । \newline
47. वृ॒त्र॒घ्न इति॑ वृत्र - घ्नः । \newline
48. स्तोमा॒ उपोप॒ स्तोमाः॒ स्तोमा॒ उप॑ । \newline
49. उप॒ माम् मा मुपोप॒ माम् । \newline
50. मा मु॒पागु॑ रु॒पागु॒र् माम् मा मु॒पागुः॑ । \newline
51. उ॒पागु॒रित्यु॑प - आगुः॑ । \newline
52. यो दा॒शुषो॑ दा॒शुषो॒ यो यो दा॒शुषः॑ । \newline
53. दा॒शुषः॑ सु॒कृतः॑ सु॒कृतो॑ दा॒शुषो॑ दा॒शुषः॑ सु॒कृतः॑ । \newline
54. सु॒कृतो॒ हव॒(ग्म्॒) हव(ग्म्॑) सु॒कृतः॑ सु॒कृतो॒ हव᳚म् । \newline
55. सु॒कृत॒ इति॑ सु - कृतः॑ । \newline
56. हव॒ मुपोप॒ हव॒(ग्म्॒) हव॒ मुप॑ । \newline
57. उप॒ गन्ता॒ गन्तोपोप॒ गन्ता᳚ । \newline
58. गन्ता॒ स स गन्ता॒ गन्ता॒ सः । \newline

\textbf{Ghana Paata } \newline

1. अ॒ग्नेर् म॑न्वे मन्वे अ॒ग्ने र॒ग्नेर् म॑न्वे प्रथ॒मस्य॑ प्रथ॒मस्य॑ मन्वे अ॒ग्ने र॒ग्नेर् म॑न्वे प्रथ॒मस्य॑ । \newline
2. म॒न्वे॒ प्र॒थ॒मस्य॑ प्रथ॒मस्य॑ मन्वे मन्वे प्रथ॒मस्य॒ प्रचे॑तसः॒ प्रचे॑तसः प्रथ॒मस्य॑ मन्वे मन्वे प्रथ॒मस्य॒ प्रचे॑तसः । \newline
3. प्र॒थ॒मस्य॒ प्रचे॑तसः॒ प्रचे॑तसः प्रथ॒मस्य॑ प्रथ॒मस्य॒ प्रचे॑तसो॒ यं ॅयम् प्रचे॑तसः प्रथ॒मस्य॑ प्रथ॒मस्य॒ प्रचे॑तसो॒ यम् । \newline
4. प्रचे॑तसो॒ यं ॅयम् प्रचे॑तसः॒ प्रचे॑तसो॒ यम् पाञ्च॑जन्य॒म् पाञ्च॑जन्यं॒ ॅयम् प्रचे॑तसः॒ प्रचे॑तसो॒ यम् पाञ्च॑जन्यम् । \newline
5. प्रचे॑तस॒ इति॒ प्र - चे॒त॒सः॒ । \newline
6. यम् पाञ्च॑जन्य॒म् पाञ्च॑जन्यं॒ ॅयं ॅयम् पाञ्च॑जन्यम् ब॒हवो॑ ब॒हवः॒ पाञ्च॑जन्यं॒ ॅयं ॅयम् पाञ्च॑जन्यम् ब॒हवः॑ । \newline
7. पाञ्च॑जन्यम् ब॒हवो॑ ब॒हवः॒ पाञ्च॑जन्य॒म् पाञ्च॑जन्यम् ब॒हवः॑ समि॒न्धते॑ समि॒न्धते॑ ब॒हवः॒ पाञ्च॑जन्य॒म् पाञ्च॑जन्यम् ब॒हवः॑ समि॒न्धते᳚ । \newline
8. पाञ्च॑जन्य॒मिति॒ पाञ्च॑ - ज॒न्य॒म् । \newline
9. ब॒हवः॑ समि॒न्धते॑ समि॒न्धते॑ ब॒हवो॑ ब॒हवः॑ समि॒न्धते᳚ । \newline
10. स॒मि॒न्धत॒ इति॑ सम् - इ॒न्धते᳚ । \newline
11. विश्व॑स्यां ॅवि॒शि वि॒शि विश्व॑स्यां॒ ॅविश्व॑स्यां ॅवि॒शि प्र॑विविशि॒वाꣳस॑म् प्रविविशि॒वाꣳसं॑ ॅवि॒शि विश्व॑स्यां॒ ॅविश्व॑स्यां ॅवि॒शि प्र॑विविशि॒वाꣳस᳚म् । \newline
12. वि॒शि प्र॑विविशि॒वाꣳस॑म् प्रविविशि॒वाꣳसं॑ ॅवि॒शि वि॒शि प्र॑विविशि॒वाꣳस॑ मीमह ईमहे प्रविविशि॒वाꣳसं॑ ॅवि॒शि वि॒शि प्र॑विविशि॒वाꣳस॑ मीमहे । \newline
13. प्र॒वि॒वि॒शि॒वाꣳस॑ मीमह ईमहे प्रविविशि॒वाꣳस॑म् प्रविविशि॒वाꣳस॑ मीमहे॒ स स ई॑महे प्रविविशि॒वाꣳस॑म् प्रविविशि॒वाꣳस॑ मीमहे॒ सः । \newline
14. प्र॒वि॒वि॒शि॒वाꣳस॒मिति॑ प्र - वि॒वि॒शि॒वाꣳस᳚म् । \newline
15. ई॒म॒हे॒ स स ई॑मह ईमहे॒ स नो॑ नः॒ स ई॑मह ईमहे॒ स नः॑ । \newline
16. स नो॑ नः॒ स स नो॑ मुञ्चतु मुञ्चतु नः॒ स स नो॑ मुञ्चतु । \newline
17. नो॒ मु॒ञ्च॒तु॒ मु॒ञ्च॒तु॒ नो॒ नो॒ मु॒ञ्च॒ त्वꣳह॑सो॒ अꣳह॑सो मुञ्चतु नो नो मुञ्च॒ त्वꣳह॑सः । \newline
18. मु॒ञ्च॒ त्वꣳह॑सो॒ अꣳह॑सो मुञ्चतु मुञ्च॒ त्वꣳह॑सः । \newline
19. अꣳह॑स॒ इत्यꣳह॑सः । \newline
20. यस्ये॒द मि॒दं ॅयस्य॒ यस्ये॒ दम् प्रा॒णत् प्रा॒णदि॒दं ॅयस्य॒ यस्ये॒दम् प्रा॒णत् । \newline
21. इ॒दम् प्रा॒णत् प्रा॒णदि॒द मि॒दम् प्रा॒णन् नि॑मि॒षन् नि॑मि॒षत् प्रा॒णदि॒द मि॒दम् प्रा॒णन् नि॑मि॒षत् । \newline
22. प्रा॒णन् नि॑मि॒षन् नि॑मि॒षत् प्रा॒णत् प्रा॒णन् नि॑मि॒षद् यद् यन् नि॑मि॒षत् प्रा॒णत् प्रा॒णन् नि॑मि॒षद् यत् । \newline
23. प्रा॒णदिति॑ प्र - अ॒नत् । \newline
24. नि॒मि॒षद् यद् यन् नि॑मि॒षन् नि॑मि॒षद् य देज॒ त्येज॑ति॒ यन् नि॑मि॒षन् नि॑मि॒षद् य देज॑ति । \newline
25. नि॒मि॒षदिति॑ नि - मि॒षत् । \newline
26. य देज॒ त्येज॑ति॒ यद् यदेज॑ति॒ यस्य॒ यस्यैज॑ति॒ यद् यदेज॑ति॒ यस्य॑ । \newline
27. एज॑ति॒ यस्य॒ यस्यैज॒ त्येज॑ति॒ यस्य॑ जा॒तम् जा॒तं ॅयस्यैज॒ त्येज॑ति॒ यस्य॑ जा॒तम् । \newline
28. यस्य॑ जा॒तम् जा॒तं ॅयस्य॒ यस्य॑ जा॒तम् जन॑मान॒म् जन॑मानम् जा॒तं ॅयस्य॒ यस्य॑ जा॒तम् जन॑मानम् । \newline
29. जा॒तम् जन॑मान॒म् जन॑मानम् जा॒तम् जा॒तम् जन॑मानम् च च॒ जन॑मानम् जा॒तम् जा॒तम् जन॑मानम् च । \newline
30. जन॑मानम् च च॒ जन॑मान॒म् जन॑मानम् च॒ केव॑ल॒म् केव॑लम् च॒ जन॑मान॒म् जन॑मानम् च॒ केव॑लम् । \newline
31. च॒ केव॑ल॒म् केव॑लम् च च॒ केव॑लम् । \newline
32. केव॑ल॒मिति॒ केव॑लम् । \newline
33. स्तौम्य॒ग्नि म॒ग्निꣳ स्तौमि॒ स्तौम्य॒ग्निम् ना॑थि॒तो ना॑थि॒तो अ॒ग्निꣳ स्तौमि॒ स्तौम्य॒ग्निम् ना॑थि॒तः । \newline
34. अ॒ग्निम् ना॑थि॒तो ना॑थि॒तो अ॒ग्नि म॒ग्निम् ना॑थि॒तो जो॑हवीमि जोहवीमि नाथि॒तो अ॒ग्नि म॒ग्निम् ना॑थि॒तो जो॑हवीमि । \newline
35. ना॒थि॒तो जो॑हवीमि जोहवीमि नाथि॒तो ना॑थि॒तो जो॑हवीमि॒ स स जो॑हवीमि नाथि॒तो ना॑थि॒तो जो॑हवीमि॒ सः । \newline
36. जो॒ह॒वी॒मि॒ स स जो॑हवीमि जोहवीमि॒ स नो॑ नः॒ स जो॑हवीमि जोहवीमि॒ स नः॑ । \newline
37. स नो॑ नः॒ स स नो॑ मुञ्चतु मुञ्चतु नः॒ स स नो॑ मुञ्चतु । \newline
38. नो॒ मु॒ञ्च॒तु॒ मु॒ञ्च॒तु॒ नो॒ नो॒ मु॒ञ्च॒ त्वꣳह॑सो॒ अꣳह॑सो मुञ्चतु नो नो मुञ्च॒ त्वꣳह॑सः । \newline
39. मु॒ञ्च॒ त्वꣳह॑सो॒ अꣳह॑सो मुञ्चतु मुञ्च॒ त्वꣳह॑सः । \newline
40. अꣳह॑स॒ इत्यꣳह॑सः । \newline
41. इन्द्र॑स्य मन्ये मन्य॒ इन्द्र॒ स्येन्द्र॑स्य मन्ये प्रथ॒मस्य॑ प्रथ॒मस्य॑ मन्य॒ इन्द्र॒ स्येन्द्र॑स्य मन्ये प्रथ॒मस्य॑ । \newline
42. म॒न्ये॒ प्र॒थ॒मस्य॑ प्रथ॒मस्य॑ मन्ये मन्ये प्रथ॒मस्य॒ प्रचे॑तसः॒ प्रचे॑तसः प्रथ॒मस्य॑ मन्ये मन्ये प्रथ॒मस्य॒ प्रचे॑तसः । \newline
43. प्र॒थ॒मस्य॒ प्रचे॑तसः॒ प्रचे॑तसः प्रथ॒मस्य॑ प्रथ॒मस्य॒ प्रचे॑तसो वृत्र॒घ्नो वृ॑त्र॒घ्नः प्रचे॑तसः प्रथ॒मस्य॑ प्रथ॒मस्य॒ प्रचे॑तसो वृत्र॒घ्नः । \newline
44. प्रचे॑तसो वृत्र॒घ्नो वृ॑त्र॒घ्नः प्रचे॑तसः॒ प्रचे॑तसो वृत्र॒घ्नः स्तोमाः॒ स्तोमा॑ वृत्र॒घ्नः प्रचे॑तसः॒ प्रचे॑तसो वृत्र॒घ्नः स्तोमाः᳚ । \newline
45. प्रचे॑तस॒ इति॒ प्र - चे॒त॒सः॒ । \newline
46. वृ॒त्र॒घ्नः स्तोमाः॒ स्तोमा॑ वृत्र॒घ्नो वृ॑त्र॒घ्नः स्तोमा॒ उपोप॒ स्तोमा॑ वृत्र॒घ्नो वृ॑त्र॒घ्नः स्तोमा॒ उप॑ । \newline
47. वृ॒त्र॒घ्न इति॑ वृत्र - घ्नः । \newline
48. स्तोमा॒ उपोप॒ स्तोमाः॒ स्तोमा॒ उप॒ माम् मा मुप॒ स्तोमाः॒ स्तोमा॒ उप॒ माम् । \newline
49. उप॒ माम् मा मुपोप॒ मा मु॒पागु॑ रु॒पागु॒र् मा मुपोप॒ मा मु॒पागुः॑ । \newline
50. मा मु॒पागु॑ रु॒पागु॒र् माम् मा मु॒पागुः॑ । \newline
51. उ॒पागु॒रित्यु॑प - आगुः॑ । \newline
52. यो दा॒शुषो॑ दा॒शुषो॒ यो यो दा॒शुषः॑ सु॒कृतः॑ सु॒कृतो॑ दा॒शुषो॒ यो यो दा॒शुषः॑ सु॒कृतः॑ । \newline
53. दा॒शुषः॑ सु॒कृतः॑ सु॒कृतो॑ दा॒शुषो॑ दा॒शुषः॑ सु॒कृतो॒ हव॒(ग्म्॒) हव(ग्म्॑) सु॒कृतो॑ दा॒शुषो॑ दा॒शुषः॑ सु॒कृतो॒ हव᳚म् । \newline
54. सु॒कृतो॒ हव॒(ग्म्॒) हव(ग्म्॑) सु॒कृतः॑ सु॒कृतो॒ हव॒ मुपोप॒ हव(ग्म्॑) सु॒कृतः॑ सु॒कृतो॒ हव॒ मुप॑ । \newline
55. सु॒कृत॒ इति॑ सु - कृतः॑ । \newline
56. हव॒ मुपोप॒ हव॒(ग्म्॒) हव॒ मुप॒ गन्ता॒ गन्तोप॒ हव॒(ग्म्॒) हव॒ मुप॒ गन्ता᳚ । \newline
57. उप॒ गन्ता॒ गन्तोपोप॒ गन्ता॒ स स गन्तोपोप॒ गन्ता॒ सः । \newline
58. गन्ता॒ स स गन्ता॒ गन्ता॒ स नो॑ नः॒ स गन्ता॒ गन्ता॒ स नः॑ । \newline
\pagebreak
\markright{ TS 4.7.15.2  \hfill https://www.vedavms.in \hfill}

\section{ TS 4.7.15.2 }

\textbf{TS 4.7.15.2 } \newline
\textbf{Samhita Paata} \newline

स नो॑ मुञ्च॒त्वꣳ ह॑सः ॥ यः सं॑ग्रा॒मं नय॑ति॒ संॅ ॅव॒शी यु॒धे यः पु॒ष्टानि॑ सꣳसृ॒जति॑ त्र॒याणि॑ । स्तौमीन्द्रं॑ नाथि॒तो जो॑हवीमि॒ स नो॑ मुञ्च॒त्वꣳ ह॑सः ॥म॒न्वे वां᳚ मित्रा वरुणा॒ तस्य॑ वित्तꣳ॒॒ सत्यौ॑जसा दृꣳहणा॒ यं नु॒देथे᳚ । या राजा॑नꣳ स॒रथं॑ ॅया॒थ उ॑ग्रा॒ ता नो॑ मुञ्चत॒माग॑सः ॥ यो वाꣳ॒॒ रथ॑ ऋ॒जुर॑श्मिः स॒त्यध॑र्मा॒ मिथु॒ श्चर॑न्त-मुप॒याति॑ दू॒षयन्न्॑ । स्तौमि॑ - [  ] \newline

\textbf{Pada Paata} \newline

सः । नः॒ । मु॒ञ्च॒तु॒ । अꣳह॑सः ॥ यः । स॒ग्रां॒ममिति॑ सम् - ग्रा॒मम् । नय॑ति । समिति॑ । व॒शी । यु॒धे । यः । पु॒ष्टानि॑ । सꣳ॒॒सृ॒जतीति॑ सं-सृ॒जति॑ । त्र॒याणि॑ ॥ स्तौमि॑ । इन्द्र᳚म् । ना॒थि॒तः । जो॒ह॒वी॒मि॒ । सः । नः॒ । मु॒ञ्च॒तु॒ । अꣳह॑सः ॥ म॒न्वे । वा॒म् । मि॒त्रा॒व॒रु॒णेति॑ मित्रा - व॒रु॒णा॒ । तस्य॑ । वि॒त्त॒म् । सत्यौ॑ज॒सेति॒ सत्य॑ - ओ॒ज॒सा॒ । दृꣳ॒॒ह॒णा॒ । यम् । नु॒देथे॒ इति॑ ॥ या । राजा॑नम् । स॒रथ॒मिति॑ स-रथ᳚म् । या॒थः । उ॒ग्रा॒ । ता । नः॒ । मु॒ञ्च॒त॒म् । आग॑सः ॥ यः । वा॒म् । रथः॑ । ऋ॒जुर॑श्मि॒रित्यृ॒जु - र॒श्मिः॒ । स॒त्यध॒र्मेति॑ स॒त्य - ध॒र्मा॒ । मिथु॑ । चर॑न्तम् । उ॒प॒यातीत्यु॑प - याति॑ । दू॒षयन्न्॑ ॥ स्तौमि॑ ।  \newline


\textbf{Krama Paata} \newline

स नः॑ । नो॒ मु॒ञ्च॒तु॒ । मु॒ञ्च॒त्वꣳह॑सः । अꣳह॑स॒ इत्यꣳह॑सः ॥ यः स॑ङ्ग्रा॒मम् । स॒ङ्ग्रा॒मम् नय॑ति । स॒ङ्ग्रा॒ममिति॑ सम् - ग्रा॒मम् । नय॑ति॒ सम् । सम् ॅव॒शी । व॒शी यु॒धे । यु॒धे यः । यः पु॒ष्टानि॑ । पु॒ष्टानि॑ सꣳसृ॒जति॑ । सꣳ॒॒सृ॒जति॑ त्र॒याणि॑ । सꣳ॒॒सृ॒जतीति॑ सम् - सृ॒जति॑ । त्र॒याणीति॑ त्र॒याणि॑ ॥ स्तौमीन्द्र᳚म् । इन्द्र॑म् नाथि॒तः । ना॒थि॒तो जो॑हवीमि । जो॒ह॒वी॒मि॒ सः । स नः॑ । नो॒ मु॒ञ्च॒तु॒ । मु॒ञ्च॒त्वꣳह॑सः । अꣳह॑स॒ इत्यꣳह॑सः ॥ म॒न्वे वा᳚म् । वा॒म् मि॒त्रा॒व॒रु॒णा॒ । मि॒त्रा॒व॒रु॒णा॒ तस्य॑ । मि॒त्रा॒व॒रु॒णेति॑ मित्रा - व॒रु॒णा॒ । तस्य॑ वित्तम् । वि॒त्तꣳ॒॒ सत्यौ॑जसा । सत्यौ॑जसा दृꣳहणा । सत्यौ॑ज॒सेति॒ सत्य॑ - ओ॒ज॒सा॒ । दृꣳ॒॒ह॒णा॒ यम् । यम् नु॒देथे᳚ । नु॒देथे॒ इति॑ नु॒देथे᳚ ॥ या राजा॑नम् । राजा॑नꣳ स॒रथ᳚म् । स॒रथ॑म् ॅया॒थः । स॒रथ॒मिति॑ स - रथ᳚म् । या॒थ उ॑ग्रा । उ॒ग्रा॒ ता । ता नः॑ । नो॒ मु॒ञ्च॒त॒म् । मु॒ञ्च॒त॒माग॑सः । आग॑स॒ इत्याग॑सः ॥ यो वा᳚म् । वाꣳ॒॒ रथः॑ । रथ॑ ऋ॒जुर॑श्मिः । ऋ॒जुर॑श्मिः स॒त्यध॑र्मा । ऋ॒जुर॑श्मि॒रित्यृ॒जु - र॒श्मिः॒ । स॒त्यध॑र्मा॒ मिथु॑ । स॒त्यध॒र्मेति॑ स॒त्य - ध॒र्मा॒ । मिथु॒श्चर॑न्तम् । चर॑न्तमुप॒याति॑ । उ॒प॒याति॑ दू॒षयन्न्॑ । उ॒प॒यातीत्यु॑प - याति॑ । दू॒षयन्निति॑ दू॒षयन्न्॑ ॥ स्तौमि॑ मि॒त्रावरु॑णा \newline

\textbf{Jatai Paata} \newline

1. स नो॑ नः॒ स स नः॑ । \newline
2. नो॒ मु॒ञ्च॒तु॒ मु॒ञ्च॒तु॒ नो॒ नो॒ मु॒ञ्च॒तु॒ । \newline
3. मु॒ञ्च॒ त्वꣳह॑सो॒ अꣳह॑सो मुञ्चतु मुञ्च॒ त्वꣳह॑सः । \newline
4. अꣳह॑स॒ इत्यꣳह॑सः । \newline
5. यः स॑ङ्ग्रा॒मꣳ स॑ङ्ग्रा॒मं ॅयो यः स॑ङ्ग्रा॒मम् । \newline
6. स॒ङ्ग्रा॒मम् नय॑ति॒ नय॑ति सङ्ग्रा॒मꣳ स॑ङ्ग्रा॒मम् नय॑ति । \newline
7. स॒ङ्ग्रा॒ममिति॑ सम् - ग्रा॒मम् । \newline
8. नय॑ति॒ सꣳ सम् नय॑ति॒ नय॑ति॒ सम् । \newline
9. सं ॅव॒शी व॒शी सꣳ सं ॅव॒शी । \newline
10. व॒शी यु॒धे यु॒धे व॒शी व॒शी यु॒धे । \newline
11. यु॒धे यो यो यु॒धे यु॒धे यः । \newline
12. यः पु॒ष्टानि॑ पु॒ष्टानि॒ यो यः पु॒ष्टानि॑ । \newline
13. पु॒ष्टानि॑ सꣳसृ॒जति॑ सꣳसृ॒जति॑ पु॒ष्टानि॑ पु॒ष्टानि॑ सꣳसृ॒जति॑ । \newline
14. स॒(ग्म्॒)सृ॒जति॑ त्र॒याणि॑ त्र॒याणि॑ सꣳसृ॒जति॑ सꣳसृ॒जति॑ त्र॒याणि॑ । \newline
15. स॒(ग्म्॒)सृ॒जतीति॑ सं - सृ॒जति॑ । \newline
16. त्र॒याणीति॑ त्र॒याणि॑ । \newline
17. स्तौमीन्द्र॒ मिन्द्र॒(ग्ग्॒) स्तौमि॒ स्तौमीन्द्र᳚म् । \newline
18. इन्द्र॑म् नाथि॒तो ना॑थि॒त इन्द्र॒ मिन्द्र॑म् नाथि॒तः । \newline
19. ना॒थि॒तो जो॑हवीमि जोहवीमि नाथि॒तो ना॑थि॒तो जो॑हवीमि । \newline
20. जो॒ह॒वी॒मि॒ स स जो॑हवीमि जोहवीमि॒ सः । \newline
21. स नो॑ नः॒ स स नः॑ । \newline
22. नो॒ मु॒ञ्च॒तु॒ मु॒ञ्च॒तु॒ नो॒ नो॒ मु॒ञ्च॒तु॒ । \newline
23. मु॒ञ्च॒त्वꣳह॑सो॒ अꣳह॑सो मुञ्चतु मुञ्च॒त्वꣳह॑सः । \newline
24. अꣳह॑स॒ इत्यꣳह॑सः । \newline
25. म॒न्वे वां᳚ ॅवाम् म॒न्वे म॒न्वे वा᳚म् । \newline
26. वा॒म् मि॒त्रा॒व॒रु॒णा॒ मि॒त्रा॒व॒रु॒णा॒ वां॒ ॅवा॒म् मि॒त्रा॒व॒रु॒णा॒ । \newline
27. मि॒त्रा॒व॒रु॒णा॒ तस्य॒ तस्य॑ मित्रावरुणा मित्रावरुणा॒ तस्य॑ । \newline
28. मि॒त्रा॒व॒रु॒णेति॑ मित्रा - व॒रु॒णा॒ । \newline
29. तस्य॑ वित्तं ॅवित्त॒म् तस्य॒ तस्य॑ वित्तम् । \newline
30. वि॒त्त॒(ग्म्॒) सत्यौ॑जसा॒ सत्यौ॑जसा वित्तं ॅवित्त॒(ग्म्॒) सत्यौ॑जसा । \newline
31. सत्यौ॑जसा दृꣳहणा दृꣳहणा॒ सत्यौ॑जसा॒ सत्यौ॑जसा दृꣳहणा । \newline
32. सत्यौ॑ज॒सेति॒ सत्य॑ - ओ॒ज॒सा॒ । \newline
33. दृ॒(ग्म्॒)ह॒णा॒ यं ॅयम् दृ(ग्म्॑)हणा दृꣳहणा॒ यम् । \newline
34. यम् नु॒देथे॑ नु॒देथे॒ यं ॅयम् नु॒देथे᳚ । \newline
35. नु॒देथे॒ इति॑ नु॒देथे᳚ । \newline
36. या राजा॑न॒(ग्म्॒) राजा॑नं॒ ॅया या राजा॑नम् । \newline
37. राजा॑नꣳ स॒रथ(ग्म्॑) स॒रथ॒(ग्म्॒) राजा॑न॒(ग्म्॒) राजा॑नꣳ स॒रथ᳚म् । \newline
38. स॒रथं॑ ॅया॒थो या॒थः स॒रथ(ग्म्॑) स॒रथं॑ ॅया॒थः । \newline
39. स॒रथ॒मिति॑ स - रथ᳚म् । \newline
40. या॒थ उ॑ग्रोग्रा या॒थो या॒थ उ॑ग्रा । \newline
41. उ॒ग्रा॒ ता तोग्रो᳚ग्रा॒ ता । \newline
42. ता नो॑ न॒ स्ता ता नः॑ । \newline
43. नो॒ मु॒ञ्च॒त॒म् मु॒ञ्च॒त॒म् नो॒ नो॒ मु॒ञ्च॒त॒म् । \newline
44. मु॒ञ्च॒त॒ माग॑स॒ आग॑सो मुञ्चतम् मुञ्चत॒ माग॑सः । \newline
45. आग॑स॒ इत्याग॑सः । \newline
46. यो वां᳚ ॅवां॒ ॅयो यो वा᳚म् । \newline
47. वा॒(ग्म्॒) रथो॒ रथो॑ वां ॅवा॒(ग्म्॒) रथः॑ । \newline
48. रथ॑ ऋ॒जुर॑श्मिर्. ऋ॒जुर॑श्मी॒ रथो॒ रथ॑ ऋ॒जुर॑श्मिः । \newline
49. ऋ॒जुर॑श्मिः स॒त्यध॑र्मा स॒त्यध॑र्म॒ र्‌जुर॑श्मिर्. ऋ॒जुर॑श्मिः स॒त्यध॑र्मा । \newline
50. ऋ॒जुर॑श्मि॒रित्यृ॒जु - र॒श्मिः॒ । \newline
51. स॒त्यध॑र्मा॒ मिथु॒ मिथु॑ स॒त्यध॑र्मा स॒त्यध॑र्मा॒ मिथु॑ । \newline
52. स॒त्यध॒र्मेति॑ स॒त्य - ध॒र्मा॒ । \newline
53. मिथु॒श् चर॑न्त॒म् चर॑न्त॒म् मिथु॒ मिथु॒श् चर॑न्तम् । \newline
54. चर॑न्त मुप॒यात् यु॑प॒याति॒ चर॑न्त॒म् चर॑न्त मुप॒याति॑ । \newline
55. उ॒प॒याति॑ दू॒षय॑न् दू॒षय॑न् नुप॒यात् यु॑प॒याति॑ दू॒षयन्न्॑ । \newline
56. उ॒प॒यातीत्यु॑प - याति॑ । \newline
57. दू॒षय॒न्निति॑ दू॒षयन्न्॑ । \newline
58. स्तौमि॑ मि॒त्रावरु॑णा मि॒त्रावरु॑णा॒ स्तौमि॒ स्तौमि॑ मि॒त्रावरु॑णा । \newline

\textbf{Ghana Paata } \newline

1. स नो॑ नः॒ स स नो॑ मुञ्चतु मुञ्चतु नः॒ स स नो॑ मुञ्चतु । \newline
2. नो॒ मु॒ञ्च॒तु॒ मु॒ञ्च॒तु॒ नो॒ नो॒ मु॒ञ्च॒ त्वꣳह॑सो॒ अꣳह॑सो मुञ्चतु नो नो मुञ्च॒ त्वꣳह॑सः । \newline
3. मु॒ञ्च॒ त्वꣳह॑सो॒ अꣳह॑सो मुञ्चतु मुञ्च॒ त्वꣳह॑सः । \newline
4. अꣳह॑स॒ इत्यꣳह॑सः । \newline
5. यः स॑ङ्ग्रा॒मꣳ स॑ङ्ग्रा॒मं ॅयो यः स॑ङ्ग्रा॒मम् नय॑ति॒ नय॑ति सङ्ग्रा॒मं ॅयो यः स॑ङ्ग्रा॒मम् नय॑ति । \newline
6. स॒ङ्ग्रा॒मम् नय॑ति॒ नय॑ति सङ्ग्रा॒मꣳ स॑ङ्ग्रा॒मम् नय॑ति॒ सꣳ सम् नय॑ति सङ्ग्रा॒मꣳ स॑ङ्ग्रा॒मम् नय॑ति॒ सम् । \newline
7. स॒ङ्ग्रा॒ममिति॑ सम् - ग्रा॒मम् । \newline
8. नय॑ति॒ सꣳ सम् नय॑ति॒ नय॑ति॒ सं ॅव॒शी व॒शी सम् नय॑ति॒ नय॑ति॒ सं ॅव॒शी । \newline
9. सं ॅव॒शी व॒शी सꣳ सं ॅव॒शी यु॒धे यु॒धे व॒शी सꣳ सं ॅव॒शी यु॒धे । \newline
10. व॒शी यु॒धे यु॒धे व॒शी व॒शी यु॒धे यो यो यु॒धे व॒शी व॒शी यु॒धे यः । \newline
11. यु॒धे यो यो यु॒धे यु॒धे यः पु॒ष्टानि॑ पु॒ष्टानि॒ यो यु॒धे यु॒धे यः पु॒ष्टानि॑ । \newline
12. यः पु॒ष्टानि॑ पु॒ष्टानि॒ यो यः पु॒ष्टानि॑ सꣳसृ॒जति॑ सꣳसृ॒जति॑ पु॒ष्टानि॒ यो यः पु॒ष्टानि॑ सꣳसृ॒जति॑ । \newline
13. पु॒ष्टानि॑ सꣳसृ॒जति॑ सꣳसृ॒जति॑ पु॒ष्टानि॑ पु॒ष्टानि॑ सꣳसृ॒जति॑ त्र॒याणि॑ त्र॒याणि॑ सꣳसृ॒जति॑ पु॒ष्टानि॑ पु॒ष्टानि॑ सꣳसृ॒जति॑ त्र॒याणि॑ । \newline
14. स॒(ग्म्॒)सृ॒जति॑ त्र॒याणि॑ त्र॒याणि॑ सꣳसृ॒जति॑ सꣳसृ॒जति॑ त्र॒याणि॑ । \newline
15. स॒(ग्म्॒)सृ॒जतीति॑ सं - सृ॒जति॑ । \newline
16. त्र॒याणीति॑ त्र॒याणि॑ । \newline
17. स्तौमीन्द्र॒ मिन्द्र॒(ग्ग्॒) स्तौमि॒ स्तौमीन्द्र॑म् नाथि॒तो ना॑थि॒त इन्द्र॒(ग्ग्॒) स्तौमि॒ स्तौमीन्द्र॑म् नाथि॒तः । \newline
18. इन्द्र॑म् नाथि॒तो ना॑थि॒त इन्द्र॒ मिन्द्र॑म् नाथि॒तो जो॑हवीमि जोहवीमि नाथि॒त इन्द्र॒ मिन्द्र॑म् नाथि॒तो जो॑हवीमि । \newline
19. ना॒थि॒तो जो॑हवीमि जोहवीमि नाथि॒तो ना॑थि॒तो जो॑हवीमि॒ स स जो॑हवीमि नाथि॒तो ना॑थि॒तो जो॑हवीमि॒ सः । \newline
20. जो॒ह॒वी॒मि॒ स स जो॑हवीमि जोहवीमि॒ स नो॑ नः॒ स जो॑हवीमि जोहवीमि॒ स नः॑ । \newline
21. स नो॑ नः॒ स स नो॑ मुञ्चतु मुञ्चतु नः॒ स स नो॑ मुञ्चतु । \newline
22. नो॒ मु॒ञ्च॒तु॒ मु॒ञ्च॒तु॒ नो॒ नो॒ मु॒ञ्च॒ त्वꣳह॑सो॒ अꣳह॑सो मुञ्चतु नो नो मुञ्च॒ त्वꣳह॑सः । \newline
23. मु॒ञ्च॒ त्वꣳह॑सो॒ अꣳह॑सो मुञ्चतु मुञ्च॒ त्वꣳह॑सः । \newline
24. अꣳह॑स॒ इत्यꣳह॑सः । \newline
25. म॒न्वे वां᳚ ॅवाम् म॒न्वे म॒न्वे वा᳚म् मित्रावरुणा मित्रावरुणा वाम् म॒न्वे म॒न्वे वा᳚म् मित्रावरुणा । \newline
26. वा॒म् मि॒त्रा॒व॒रु॒णा॒ मि॒त्रा॒व॒रु॒णा॒ वां॒ ॅवा॒म् मि॒त्रा॒व॒रु॒णा॒ तस्य॒ तस्य॑ मित्रावरुणा वां ॅवाम् मित्रावरुणा॒ तस्य॑ । \newline
27. मि॒त्रा॒व॒रु॒णा॒ तस्य॒ तस्य॑ मित्रावरुणा मित्रावरुणा॒ तस्य॑ वित्तं ॅवित्त॒म् तस्य॑ मित्रावरुणा मित्रावरुणा॒ तस्य॑ वित्तम् । \newline
28. मि॒त्रा॒व॒रु॒णेति॑ मित्रा - व॒रु॒णा॒ । \newline
29. तस्य॑ वित्तं ॅवित्त॒म् तस्य॒ तस्य॑ वित्त॒(ग्म्॒) सत्यौ॑जसा॒ सत्यौ॑जसा वित्त॒म् तस्य॒ तस्य॑ वित्त॒(ग्म्॒) सत्यौ॑जसा । \newline
30. वि॒त्त॒(ग्म्॒) सत्यौ॑जसा॒ सत्यौ॑जसा वित्तं ॅवित्त॒(ग्म्॒) सत्यौ॑जसा दृꣳहणा दृꣳहणा॒ सत्यौ॑जसा वित्तं ॅवित्त॒(ग्म्॒) सत्यौ॑जसा दृꣳहणा । \newline
31. सत्यौ॑जसा दृꣳहणा दृꣳहणा॒ सत्यौ॑जसा॒ सत्यौ॑जसा दृꣳहणा॒ यं ॅयम् दृ(ग्म्॑)हणा॒ सत्यौ॑जसा॒ सत्यौ॑जसा दृꣳहणा॒ यम् । \newline
32. सत्यौ॑ज॒सेति॒ सत्य॑ - ओ॒ज॒सा॒ । \newline
33. दृ॒(ग्म्॒)ह॒णा॒ यं ॅयम् दृ(ग्म्॑)हणा दृꣳहणा॒ यम् नु॒देथे॑ नु॒देथे॒ यम् दृ(ग्म्॑)हणा दृꣳहणा॒ यम् नु॒देथे᳚ । \newline
34. यम् नु॒देथे॑ नु॒देथे॒ यं ॅयम् नु॒देथे᳚ । \newline
35. नु॒देथे॒ इति॑ नु॒देथे᳚ । \newline
36. या राजा॑न॒(ग्म्॒) राजा॑नं॒ ॅया या राजा॑नꣳ स॒रथ(ग्म्॑) स॒रथ॒(ग्म्॒) राजा॑नं॒ ॅया या राजा॑नꣳ स॒रथ᳚म् । \newline
37. राजा॑नꣳ स॒रथ(ग्म्॑) स॒रथ॒(ग्म्॒) राजा॑न॒(ग्म्॒) राजा॑नꣳ स॒रथं॑ ॅया॒थो या॒थः स॒रथ॒(ग्म्॒) राजा॑न॒(ग्म्॒) राजा॑नꣳ स॒रथं॑ ॅया॒थः । \newline
38. स॒रथं॑ ॅया॒थो या॒थः स॒रथ(ग्म्॑) स॒रथं॑ ॅया॒थ उ॑ग्रोग्रा या॒थः स॒रथ(ग्म्॑) स॒रथं॑ ॅया॒थ उ॑ग्रा । \newline
39. स॒रथ॒मिति॑ स - रथ᳚म् । \newline
40. या॒थ उ॑ग्रोग्रा या॒थो या॒थ उ॑ग्रा॒ ता तोग्रा॑ या॒थो या॒थ उ॑ग्रा॒ ता । \newline
41. उ॒ग्रा॒ ता तोग्रो᳚ग्रा॒ ता नो॑ न॒ स्तोग्रो᳚ग्रा॒ ता नः॑ । \newline
42. ता नो॑ न॒ स्ता ता नो॑ मुञ्चतम् मुञ्चतम् न॒ स्ता ता नो॑ मुञ्चतम् । \newline
43. नो॒ मु॒ञ्च॒त॒म् मु॒ञ्च॒त॒म् नो॒ नो॒ मु॒ञ्च॒त॒ माग॑स॒ आग॑सो मुञ्चतम् नो नो मुञ्चत॒ माग॑सः । \newline
44. मु॒ञ्च॒त॒ माग॑स॒ आग॑सो मुञ्चतम् मुञ्चत॒ माग॑सः । \newline
45. आग॑स॒ इत्याग॑सः । \newline
46. यो वां᳚ ॅवां॒ ॅयो यो वा॒(ग्म्॒) रथो॒ रथो॑ वां॒ ॅयो यो वा॒(ग्म्॒) रथः॑ । \newline
47. वा॒(ग्म्॒) रथो॒ रथो॑ वां ॅवा॒(ग्म्॒) रथ॑ ऋ॒जुर॑श्मिर्. ऋ॒जुर॑श्मी॒ रथो॑ वां ॅवा॒(ग्म्॒) रथ॑ ऋ॒जुर॑श्मिः । \newline
48. रथ॑ ऋ॒जुर॑श्मिर्. ऋ॒जुर॑श्मी॒ रथो॒ रथ॑ ऋ॒जुर॑श्मिः स॒त्यध॑र्मा स॒त्यध॑र्म॒ र्‌जुर॑श्मी॒ रथो॒ रथ॑ ऋ॒जुर॑श्मिः स॒त्यध॑र्मा । \newline
49. ऋ॒जुर॑श्मिः स॒त्यध॑र्मा स॒त्यध॑र्म॒ र्‌जुर॑श्मिर्. ऋ॒जुर॑श्मिः स॒त्यध॑र्मा॒ मिथु॒ मिथु॑ स॒त्यध॑र्म॒ र्‌जुर॑श्मिर्. ऋ॒जुर॑श्मिः स॒त्यध॑र्मा॒ मिथु॑ । \newline
50. ऋ॒जुर॑श्मि॒रित्यृ॒जु - र॒श्मिः॒ । \newline
51. स॒त्यध॑र्मा॒ मिथु॒ मिथु॑ स॒त्यध॑र्मा स॒त्यध॑र्मा॒ मिथु॒श् चर॑न्त॒म् चर॑न्त॒म् मिथु॑ स॒त्यध॑र्मा स॒त्यध॑र्मा॒ मिथु॒श् चर॑न्तम् । \newline
52. स॒त्यध॒र्मेति॑ स॒त्य - ध॒र्मा॒ । \newline
53. मिथु॒श् चर॑न्त॒म् चर॑न्त॒म् मिथु॒ मिथु॒श् चर॑न्त मुप॒या त्यु॑प॒याति॒ चर॑न्त॒म् मिथु॒ मिथु॒श् चर॑न्त मुप॒याति॑ । \newline
54. चर॑न्त मुप॒या त्यु॑प॒याति॒ चर॑न्त॒म् चर॑न्त मुप॒याति॑ दू॒षय॑न् दू॒षय॑न् नुप॒याति॒ चर॑न्त॒म् चर॑न्त मुप॒याति॑ दू॒षयन्न्॑ । \newline
55. उ॒प॒याति॑ दू॒षय॑न् दू॒षय॑न् नुप॒या त्यु॑प॒याति॑ दू॒षयन्न्॑ । \newline
56. उ॒प॒यातीत्यु॑प - याति॑ । \newline
57. दू॒षय॒न्निति॑ दू॒षयन्न्॑ । \newline
58. स्तौमि॑ मि॒त्रावरु॑णा मि॒त्रावरु॑णा॒ स्तौमि॒ स्तौमि॑ मि॒त्रावरु॑णा नाथि॒तो ना॑थि॒तो मि॒त्रावरु॑णा॒ स्तौमि॒ स्तौमि॑ मि॒त्रावरु॑णा नाथि॒तः । \newline
\pagebreak
\markright{ TS 4.7.15.3  \hfill https://www.vedavms.in \hfill}

\section{ TS 4.7.15.3 }

\textbf{TS 4.7.15.3 } \newline
\textbf{Samhita Paata} \newline

मि॒त्रावरु॑णा नाथि॒तो जो॑हवीमि॒ तौ नो॑ मुञ्चत॒माग॑सः ॥वा॒योः स॑वि॒तु र्वि॒दथा॑नि मन्महे॒ यावा᳚त्म॒न्वद्-बि॑भृ॒तो यौ च॒ रक्ष॑तः । यौ विश्व॑स्य परि॒भू ब॑भू॒वतु॒स्तौ नो॑ मुञ्चत॒माग॑सः ॥उप॒ श्रेष्ठा॑न आ॒शिषो॑ दे॒वयो॒र्द्धर्मे॑ अस्थिरन्न् ।स्तौमि॑ वा॒युꣳ स॑वि॒तारं॑ नाथि॒तो जो॑हवीमि॒ तौ नो॑ मुञ्चत॒माग॑सः ॥ र॒थीत॑मौ रथी॒नाम॑ह्व ऊ॒तये॒ शुभं॒ गमि॑ष्ठौ सु॒यमे॑भि॒रश्वैः᳚ । ययो᳚ -[  ] \newline

\textbf{Pada Paata} \newline

मि॒त्रावरु॒णेति॑ मि॒त्रा - वरु॑णा । ना॒थि॒तः । जो॒ह॒वी॒मि॒ । तौ । नः॒ । मु॒ञ्च॒त॒म् । आग॑सः ॥ वा॒योः । स॒वि॒तुः । वि॒दथा॑नि । म॒न्म॒हे॒ । यौ । आ॒त्म॒न्वदित्या᳚त्मन्न् - वत् । बि॒भृ॒तः । यौ । च॒ । रक्ष॑तः ॥ यौ । विश्व॑स्य । प॒रि॒भू इति॑ परि - भूः । ब॒भू॒वतुः॑ । तौ । नः॒ । मु॒ञ्च॒त॒म् । आग॑सः ॥ उपेति॑ । श्रेष्ठाः᳚ । नः॒ । आ॒शिष॒ इत्या᳚ - शिषः॑ । दे॒वयोः᳚ । धर्मे᳚ । अ॒स्थि॒र॒न्न् ॥ स्तौमि॑ । वा॒युम् । स॒वि॒तार᳚म् । ना॒थि॒तः । जो॒ह॒वी॒मि॒ । तौ । नः॒ । मु॒ञ्च॒त॒म् । आग॑सः ॥ र॒थीत॑मा॒विति॑ र॒थि - त॒मौ॒ । र॒थी॒नाम् । अ॒ह्वे॒ । ऊ॒तये᳚ । शुभं᳚ । गमि॑ष्ठौ । सु॒यमे॑भि॒रिति॑ सु - यमे॑भिः । अश्वैः᳚ ॥ ययोः᳚ ।  \newline


\textbf{Krama Paata} \newline

मि॒त्रावरु॑णा नाथि॒तः । मि॒त्रावरु॒णेति॑ मि॒त्रा - वरु॑णा । ना॒थि॒तो जो॑हवीमि । जो॒ह॒वी॒मि॒ तौ । तौ नः॑ । नो॒ मु॒ञ्च॒त॒म् । मु॒ञ्च॒त॒माग॑सः । आग॑स॒ इत्याग॑सः ॥ वा॒योः स॑वि॒तुः । स॒वि॒तुर् वि॒दथा॑नि । वि॒दथा॑नि मन्महे । म॒न्म॒हे॒ यौ । यावा᳚त्म॒न्वत् । आ॒त्म॒न्वद् बि॑भृ॒तः । आ॒त्म॒न्वदित्या᳚त्मन्न् - वत् । बि॒भृ॒तो यौ । यौ च॑ । च॒ रक्ष॑तः । रक्ष॑त॒ इति॒ रक्ष॑तः ॥ यौ विश्व॑स्य । विश्व॑स्य परि॒भू । प॒रि॒भू ब॑भू॒वतुः॑ । प॒रि॒भू इति॑ परि - भू । ब॒भू॒वतु॒स्तौ । तौ नः॑ । नो॒ मु॒ञ्च॒त॒॒म् । मु॒ञ्च॒त॒माग॑सः । आग॑स॒ इत्याग॑सः ॥ उप॒ श्रेष्ठाः᳚ । श्रेष्ठा॑ नः । न॒ आ॒शिषः॑ । आ॒शिषो॑ दे॒वयोः᳚ । आ॒शिष॒ इत्या᳚ - शिषः॑ । दे॒वयो॒र् धर्मे᳚ । धर्मे॑ अस्थिरन्न् । अ॒स्थि॒र॒न्नित्य॑स्थिरन्न् ॥ स्तौमि॑ वा॒युम् । वा॒युꣳ स॑वि॒तार᳚म् । स॒वि॒तार॑म् नाथि॒तः । ना॒थि॒तो जो॑हवीमि । जो॒ह॒वी॒मि॒ तौ । तौ नः॑ । नो॒ मु॒ञ्च॒त॒॒म् । मु॒ञ्च॒त॒माग॑सः । आग॑स॒ इत्याग॑सः ॥ र॒थीत॑मौ रथी॒नाम् । र॒थीत॑मा॒विति॑ र॒थि- त॒मौ॒ । र॒थी॒नाम॑ह्वे । अ॒ह्व॒ ऊ॒तये᳚ । ऊ॒तये॒ शुभ᳚म् । शुभ॒म् गमि॑ष्ठौ । गमि॑ष्ठौ सु॒यमे॑भिः । सु॒यमे॑भि॒रश्वैः᳚ । सु॒यमे॑भि॒रिति॑ सु - यमे॑भिः । अश्वै॒रित्यश्वैः᳚ ॥ ययो᳚र् वाम् \newline

\textbf{Jatai Paata} \newline

1. मि॒त्रावरु॑णा नाथि॒तो ना॑थि॒तो मि॒त्रावरु॑णा मि॒त्रावरु॑णा नाथि॒तः । \newline
2. मि॒त्रावरु॒णेति॑ मि॒त्रा - वरु॑णा । \newline
3. ना॒थि॒तो जो॑हवीमि जोहवीमि नाथि॒तो ना॑थि॒तो जो॑हवीमि । \newline
4. जो॒ह॒वी॒मि॒ तौ तौ जो॑हवीमि जोहवीमि॒ तौ । \newline
5. तौ नो॑ न॒ स्तौ तौ नः॑ । \newline
6. नो॒ मु॒ञ्च॒त॒म् मु॒ञ्च॒त॒म् नो॒ नो॒ मु॒ञ्च॒त॒म् । \newline
7. मु॒ञ्च॒त॒ माग॑स॒ आग॑सो मुञ्चतम् मुञ्चत॒ माग॑सः । \newline
8. आग॑स॒ इत्याग॑सः । \newline
9. वा॒योः स॑वि॒तुः स॑वि॒तुर् वा॒योर् वा॒योः स॑वि॒तुः । \newline
10. स॒वि॒तुर् वि॒दथा॑नि वि॒दथा॑नि सवि॒तुः स॑वि॒तुर् वि॒दथा॑नि । \newline
11. वि॒दथा॑नि मन्महे मन्महे वि॒दथा॑नि वि॒दथा॑नि मन्महे । \newline
12. म॒न्म॒हे॒ यौ यौ म॑न्महे मन्महे॒ यौ । \newline
13. या वा᳚त्म॒न् वदा᳚त्म॒न् वद् यौ या वा᳚त्म॒न्वत् । \newline
14. आ॒त्म॒न्वद् बि॑भृ॒तो बि॑भृ॒त आ᳚त्म॒न् वदा᳚त्म॒न्वद् बि॑भृ॒तः । \newline
15. आ॒त्म॒न्वदित्या᳚त्मन्न् - वत् । \newline
16. बि॒भृ॒तो यौ यौ बि॑भृ॒तो बि॑भृ॒तो यौ । \newline
17. यौ च॑ च॒ यौ यौ च॑ । \newline
18. च॒ रक्ष॑तो॒ रक्ष॑तश्च च॒ रक्ष॑तः । \newline
19. रक्ष॑त॒ इति॒ रक्ष॑तः । \newline
20. यौ विश्व॑स्य॒ विश्व॑स्य॒ यौ यौ विश्व॑स्य । \newline
21. विश्व॑स्य परि॒भू प॑रि॒भू विश्व॑स्य॒ विश्व॑स्य परि॒भू । \newline
22. प॒रि॒भू ब॑भू॒वतु॑र् बभू॒वतुः॑ परि॒भू प॑रि॒भू ब॑भू॒वतुः॑ । \newline
23. प॒रि॒भू इति॑ परि - भूः । \newline
24. ब॒भू॒वतु॒ स्तौ तौ ब॑भू॒वतु॑र् बभू॒वतु॒ स्तौ । \newline
25. तौ नो॑ न॒ स्तौ तौ नः॑ । \newline
26. नो॒ मु॒ञ्च॒त॒म् मु॒ञ्च॒त॒म् नो॒ नो॒ मु॒ञ्च॒त॒म् । \newline
27. मु॒ञ्च॒त॒ माग॑स॒ आग॑सो मुञ्चतम् मुञ्चत॒ माग॑सः । \newline
28. आग॑स॒ इत्याग॑सः । \newline
29. उप॒ श्रेष्ठाः॒ श्रेष्ठा॒ उपोप॒ श्रेष्ठाः᳚ । \newline
30. श्रेष्ठा॑ नो नः॒ श्रेष्ठाः॒ श्रेष्ठा॑ नः । \newline
31. न॒ आ॒शिष॑ आ॒शिषो॑ नो न आ॒शिषः॑ । \newline
32. आ॒शिषो॑ दे॒वयो᳚र् दे॒वयो॑ रा॒शिष॑ आ॒शिषो॑ दे॒वयोः᳚ । \newline
33. आ॒शिष॒इत्या᳚ - शिषः॑ । \newline
34. दे॒वयो॒र् धर्मे॒ धर्मे॑ दे॒वयो᳚र् दे॒वयो॒र् धर्मे᳚ । \newline
35. धर्मे॑ अस्थिरन् नस्थिर॒न् धर्मे॒ धर्मे॑ अस्थिरन्न् । \newline
36. अ॒स्थि॒र॒न्नित्य॑स्थिरन्न् । \newline
37. स्तौमि॑ वा॒युं ॅवा॒युꣳ स्तौमि॒ स्तौमि॑ वा॒युम् । \newline
38. वा॒युꣳ स॑वि॒तार(ग्म्॑) सवि॒तारं॑ ॅवा॒युं ॅवा॒युꣳ स॑वि॒तार᳚म् । \newline
39. स॒वि॒तार॑म् नाथि॒तो ना॑थि॒तः स॑वि॒तार(ग्म्॑) सवि॒तार॑म् नाथि॒तः । \newline
40. ना॒थि॒तो जो॑हवीमि जोहवीमि नाथि॒तो ना॑थि॒तो जो॑हवीमि । \newline
41. जो॒ह॒वी॒मि॒ तौ तौ जो॑हवीमि जोहवीमि॒ तौ । \newline
42. तौ नो॑ न॒ स्तौ तौ नः॑ । \newline
43. नो॒ मु॒ञ्च॒त॒म् मु॒ञ्च॒त॒म् नो॒ नो॒ मु॒ञ्च॒त॒म् । \newline
44. मु॒ञ्च॒त॒ माग॑स॒ आग॑सो मुञ्चतम् मुञ्चत॒ माग॑सः । \newline
45. आग॑स॒ इत्याग॑सः । \newline
46. र॒थीत॑मौ रथी॒नाꣳ र॑थी॒नाꣳ र॒थीत॑मौ र॒थीत॑मौ रथी॒नाम् । \newline
47. र॒थीत॑मा॒विति॑ र॒थि - त॒मौ॒ । \newline
48. र॒थी॒ना म॑ह्वे अह्वे रथी॒नाꣳ र॑थी॒ना म॑ह्वे । \newline
49. अ॒ह्व॒ ऊ॒तय॑ ऊ॒तये॑ अह्वे अह्व ऊ॒तये᳚ । \newline
50. ऊ॒तये॒ शुभꣳ॒॒ शुभ॑ मू॒तय॑ ऊ॒तये॒ शुभं᳚ । \newline
51. शुभं॒ गमि॑ष्ठौ॒ गमि॑ष्ठौ॒ शुभꣳ॒॒ शुभं॒ गमि॑ष्ठौ । \newline
52. गमि॑ष्ठौ सु॒यमे॑भिः सु॒यमे॑भि॒र् गमि॑ष्ठौ॒ गमि॑ष्ठौ सु॒यमे॑भिः । \newline
53. सु॒यमे॑भि॒ रश्वै॒ रश्वैः᳚ सु॒यमे॑भिः सु॒यमे॑भि॒ रश्वैः᳚ । \newline
54. सु॒यमे॑भि॒रिति॑ सु - यमे॑भिः । \newline
55. अश्वै॒रित्यश्वैः᳚ । \newline
56. ययो᳚र् वां ॅवां॒ ॅययो॒र् ययो᳚र् वाम् । \newline

\textbf{Ghana Paata } \newline

1. मि॒त्रावरु॑णा नाथि॒तो ना॑थि॒तो मि॒त्रावरु॑णा मि॒त्रावरु॑णा नाथि॒तो जो॑हवीमि जोहवीमि नाथि॒तो मि॒त्रावरु॑णा मि॒त्रावरु॑णा नाथि॒तो जो॑हवीमि । \newline
2. मि॒त्रावरु॒णेति॑ मि॒त्रा - वरु॑णा । \newline
3. ना॒थि॒तो जो॑हवीमि जोहवीमि नाथि॒तो ना॑थि॒तो जो॑हवीमि॒ तौ तौ जो॑हवीमि नाथि॒तो ना॑थि॒तो जो॑हवीमि॒ तौ । \newline
4. जो॒ह॒वी॒मि॒ तौ तौ जो॑हवीमि जोहवीमि॒ तौ नो॑ न॒ स्तौ जो॑हवीमि जोहवीमि॒ तौ नः॑ । \newline
5. तौ नो॑ न॒ स्तौ तौ नो॑ मुञ्चतम् मुञ्चतम् न॒ स्तौ तौ नो॑ मुञ्चतम् । \newline
6. नो॒ मु॒ञ्च॒त॒म् मु॒ञ्च॒त॒म् नो॒ नो॒ मु॒ञ्च॒त॒ माग॑स॒ आग॑सो मुञ्चतम् नो नो मुञ्चत॒ माग॑सः । \newline
7. मु॒ञ्च॒त॒ माग॑स॒ आग॑सो मुञ्चतम् मुञ्चत॒ माग॑सः । \newline
8. आग॑स॒ इत्याग॑सः । \newline
9. वा॒योः स॑वि॒तुः स॑वि॒तुर् वा॒योर् वा॒योः स॑वि॒तुर् वि॒दथा॑नि वि॒दथा॑नि सवि॒तुर् वा॒योर् वा॒योः स॑वि॒तुर् वि॒दथा॑नि । \newline
10. स॒वि॒तुर् वि॒दथा॑नि वि॒दथा॑नि सवि॒तुः स॑वि॒तुर् वि॒दथा॑नि मन्महे मन्महे वि॒दथा॑नि सवि॒तुः स॑वि॒तुर् वि॒दथा॑नि मन्महे । \newline
11. वि॒दथा॑नि मन्महे मन्महे वि॒दथा॑नि वि॒दथा॑नि मन्महे॒ यौ यौ म॑न्महे वि॒दथा॑नि वि॒दथा॑नि मन्महे॒ यौ । \newline
12. म॒न्म॒हे॒ यौ यौ म॑न्महे मन्महे॒ या वा᳚त्म॒न्व दा᳚त्म॒न्वद् यौ म॑न्महे मन्महे॒ या वा᳚त्म॒न्वत् । \newline
13. या वा᳚त्म॒न्व दा᳚त्म॒न्वद् यौ या वा᳚त्म॒न्वद् बि॑भृ॒तो बि॑भृ॒त आ᳚त्म॒न्वद् यौ या वा᳚त्म॒न्वद् बि॑भृ॒तः । \newline
14. आ॒त्म॒न्वद् बि॑भृ॒तो बि॑भृ॒त आ᳚त्म॒न्व दा᳚त्म॒न्वद् बि॑भृ॒तो यौ यौ बि॑भृ॒त आ᳚त्म॒न्व दा᳚त्म॒न्वद् बि॑भृ॒तो यौ । \newline
15. आ॒त्म॒न्वदित्या᳚त्मन्न् - वत् । \newline
16. बि॒भृ॒तो यौ यौ बि॑भृ॒तो बि॑भृ॒तो यौ च॑ च॒ यौ बि॑भृ॒तो बि॑भृ॒तो यौ च॑ । \newline
17. यौ च॑ च॒ यौ यौ च॒ रक्ष॑तो॒ रक्ष॑तश्च॒ यौ यौ च॒ रक्ष॑तः । \newline
18. च॒ रक्ष॑तो॒ रक्ष॑तश्च च॒ रक्ष॑तः । \newline
19. रक्ष॑त॒ इति॒ रक्ष॑तः । \newline
20. यौ विश्व॑स्य॒ विश्व॑स्य॒ यौ यौ विश्व॑स्य परि॒भू प॑रि॒भू विश्व॑स्य॒ यौ यौ विश्व॑स्य परि॒भू । \newline
21. विश्व॑स्य परि॒भू प॑रि॒भू विश्व॑स्य॒ विश्व॑स्य परि॒भू ब॑भू॒वतु॑र् बभू॒वतुः॑ परि॒भू विश्व॑स्य॒ विश्व॑स्य परि॒भू ब॑भू॒वतुः॑ । \newline
22. प॒रि॒भू ब॑भू॒वतु॑र् बभू॒वतुः॑ परि॒भू प॑रि॒भू ब॑भू॒वतु॒ स्तौ तौ ब॑भू॒वतुः॑ परि॒भू प॑रि॒भू ब॑भू॒वतु॒ स्तौ । \newline
23. प॒रि॒भू इति॑ परि - भूः । \newline
24. ब॒भू॒वतु॒ स्तौ तौ ब॑भू॒वतु॑र् बभू॒वतु॒ स्तौ नो॑ न॒ स्तौ ब॑भू॒वतु॑र् बभू॒वतु॒ स्तौ नः॑ । \newline
25. तौ नो॑ न॒ स्तौ तौ नो॑ मुञ्चतम् मुञ्चतम् न॒ स्तौ तौ नो॑ मुञ्चतम् । \newline
26. नो॒ मु॒ञ्च॒त॒म् मु॒ञ्च॒त॒म् नो॒ नो॒ मु॒ञ्च॒त॒ माग॑स॒ आग॑सो मुञ्चतम् नो नो मुञ्चत॒ माग॑सः । \newline
27. मु॒ञ्च॒त॒ माग॑स॒ आग॑सो मुञ्चतम् मुञ्चत॒ माग॑सः । \newline
28. आग॑स॒ इत्याग॑सः । \newline
29. उप॒ श्रेष्ठाः॒ श्रेष्ठा॒ उपोप॒ श्रेष्ठा॑ नो नः॒ श्रेष्ठा॒ उपोप॒ श्रेष्ठा॑ नः । \newline
30. श्रेष्ठा॑ नो नः॒ श्रेष्ठाः॒ श्रेष्ठा॑ न आ॒शिष॑ आ॒शिषो॑ नः॒ श्रेष्ठाः॒ श्रेष्ठा॑ न आ॒शिषः॑ । \newline
31. न॒ आ॒शिष॑ आ॒शिषो॑ नो न आ॒शिषो॑ दे॒वयो᳚र् दे॒वयो॑ रा॒शिषो॑ नो न आ॒शिषो॑ दे॒वयोः᳚ । \newline
32. आ॒शिषो॑ दे॒वयो᳚र् दे॒वयो॑ रा॒शिष॑ आ॒शिषो॑ दे॒वयो॒र् धर्मे॒ धर्मे॑ दे॒वयो॑ रा॒शिष॑ आ॒शिषो॑ दे॒वयो॒र् धर्मे᳚ । \newline
33. आ॒शिष॒ इत्या᳚ - शिषः॑ । \newline
34. दे॒वयो॒र् धर्मे॒ धर्मे॑ दे॒वयो᳚र् दे॒वयो॒र् धर्मे॑ अस्थिरन् नस्थिर॒न् धर्मे॑ दे॒वयो᳚र् दे॒वयो॒र् धर्मे॑ अस्थिरन्न् । \newline
35. धर्मे॑ अस्थिरन् नस्थिर॒न् धर्मे॒ धर्मे॑ अस्थिरन्न् । \newline
36. अ॒स्थि॒र॒न्नित्य॑स्थिरन्न् । \newline
37. स्तौमि॑ वा॒युं ॅवा॒युꣳ स्तौमि॒ स्तौमि॑ वा॒युꣳ स॑वि॒तार(ग्म्॑) सवि॒तारं॑ ॅवा॒युꣳ स्तौमि॒ स्तौमि॑ वा॒युꣳ स॑वि॒तार᳚म् । \newline
38. वा॒युꣳ स॑वि॒तार(ग्म्॑) सवि॒तारं॑ ॅवा॒युं ॅवा॒युꣳ स॑वि॒तार॑म् नाथि॒तो ना॑थि॒तः स॑वि॒तारं॑ ॅवा॒युं ॅवा॒युꣳ स॑वि॒तार॑म् नाथि॒तः । \newline
39. स॒वि॒तार॑म् नाथि॒तो ना॑थि॒तः स॑वि॒तार(ग्म्॑) सवि॒तार॑म् नाथि॒तो जो॑हवीमि जोहवीमि नाथि॒तः स॑वि॒तार(ग्म्॑) सवि॒तार॑म् नाथि॒तो जो॑हवीमि । \newline
40. ना॒थि॒तो जो॑हवीमि जोहवीमि नाथि॒तो ना॑थि॒तो जो॑हवीमि॒ तौ तौ जो॑हवीमि नाथि॒तो ना॑थि॒तो जो॑हवीमि॒ तौ । \newline
41. जो॒ह॒वी॒मि॒ तौ तौ जो॑हवीमि जोहवीमि॒ तौ नो॑ न॒ स्तौ जो॑हवीमि जोहवीमि॒ तौ नः॑ । \newline
42. तौ नो॑ न॒ स्तौ तौ नो॑ मुञ्चतम् मुञ्चतम् न॒ स्तौ तौ नो॑ मुञ्चतम् । \newline
43. नो॒ मु॒ञ्च॒त॒म् मु॒ञ्च॒त॒म् नो॒ नो॒ मु॒ञ्च॒त॒ माग॑स॒ आग॑सो मुञ्चतम् नो नो मुञ्चत॒ माग॑सः । \newline
44. मु॒ञ्च॒त॒ माग॑स॒ आग॑सो मुञ्चतम् मुञ्चत॒ माग॑सः । \newline
45. आग॑स॒ इत्याग॑सः । \newline
46. र॒थीत॑मौ रथी॒नाꣳ र॑थी॒नाꣳ र॒थीत॑मौ र॒थीत॑मौ रथी॒ना म॑ह्वे अह्वे रथी॒नाꣳ र॒थीत॑मौ र॒थीत॑मौ रथी॒ना म॑ह्वे । \newline
47. र॒थीत॑मा॒विति॑ र॒थि - त॒मौ॒ । \newline
48. र॒थी॒ना म॑ह्वे अह्वे रथी॒नाꣳ र॑थी॒ना म॑ह्व ऊ॒तय॑ ऊ॒तये॑ अह्वे रथी॒नाꣳ र॑थी॒ना म॑ह्व ऊ॒तये᳚ । \newline
49. अ॒ह्व॒ ऊ॒तय॑ ऊ॒तये॑ अह्वे अह्व ऊ॒तये॒ शुभꣳ॒॒ शुभ॑ मू॒तये॑ अह्वे अह्व ऊ॒तये॒ शुभं᳚ । \newline
50. ऊ॒तये॒ शुभꣳ॒॒ शुभ॑ मू॒तय॑ ऊ॒तये॒ शुभं॒ गमि॑ष्ठौ॒ गमि॑ष्ठौ॒ शुभ॑ मू॒तय॑ ऊ॒तये॒ शुभं॒ गमि॑ष्ठौ । \newline
51. शुभं॒ गमि॑ष्ठौ॒ गमि॑ष्ठौ॒ शुभꣳ॒॒ शुभं॒ गमि॑ष्ठौ सु॒यमे॑भिः सु॒यमे॑भि॒र् गमि॑ष्ठौ॒ शुभꣳ॒॒ शुभं॒ गमि॑ष्ठौ सु॒यमे॑भिः । \newline
52. गमि॑ष्ठौ सु॒यमे॑भिः सु॒यमे॑भि॒र् गमि॑ष्ठौ॒ गमि॑ष्ठौ सु॒यमे॑भि॒ रश्वै॒ रश्वैः᳚ सु॒यमे॑भि॒र् गमि॑ष्ठौ॒ गमि॑ष्ठौ सु॒यमे॑भि॒ रश्वैः᳚ । \newline
53. सु॒यमे॑भि॒ रश्वै॒ रश्वैः᳚ सु॒यमे॑भिः सु॒यमे॑भि॒ रश्वैः᳚ । \newline
54. सु॒यमे॑भि॒रिति॑ सु - यमे॑भिः । \newline
55. अश्वै॒रित्यश्वैः᳚ । \newline
56. ययो᳚र् वां ॅवां॒ ॅययो॒र् ययो᳚र् वाम् देवौ देवौ वां॒ ॅययो॒र् ययो᳚र् वाम् देवौ । \newline
\pagebreak
\markright{ TS 4.7.15.4  \hfill https://www.vedavms.in \hfill}

\section{ TS 4.7.15.4 }

\textbf{TS 4.7.15.4 } \newline
\textbf{Samhita Paata} \newline

-र्वां देवौ दे॒वेष्व-नि॑शित॒-मोज॒स्तौ नो॑ मुञ्चत॒माग॑सः ॥यदया॑तं ॅवॅह॒तुꣳ सू॒र्याया᳚-स्त्रिच॒क्रेण॑ सꣳ॒॒ सद॑मि॒च्छमा॑नौ ।स्तौमि॑ दे॒वा व॒श्विनौ॑ नाथि॒तो जो॑हवीमि॒ तौ नो॑ मुञ्चत॒माग॑सः ॥म॒रुतां᳚ मन्वे॒ अधि॑नो ब्रुवन्तु॒ प्रेमां ॅवाचं॒ ॅविश्वा॑ मवन्तु॒ विश्वे᳚ । आ॒शून्. हु॑वे सु॒यमा॑नू॒तये॒ ते नो॑ मुञ्च॒न्त्वेन॑सः ॥ति॒ग्ममायु॑धं ॅवीडि॒तꣳ सह॑स्वद् दि॒व्यꣳ शर्द्धः॒ - [  ] \newline

\textbf{Pada Paata} \newline

वा॒म् । दे॒वौ॒ । दे॒वेषु॑ । अनि॑शित॒मित्यनि॑-शि॒त॒म् । ओजः॑ । तौ । नः॒ । मु॒ञ्च॒त॒म् । आग॑सः ॥ यत् । अया॑तम् । व॒ह॒तुम् । सू॒र्यायाः᳚ । त्रि॒च॒क्रेणेति॑ त्रि - च॒क्रेण॑ । सꣳ॒॒सद॒मिति॑ सम् - सद᳚म् । इ॒च्छमा॑नौ ॥ स्तौमि॑ । दे॒वौ । अ॒श्विनौ᳚ । ना॒थि॒तः । जो॒ह॒वी॒मि॒ । तौ । नः॒ । मु॒ञ्च॒त॒म् । आग॑सः ॥ म॒रुता᳚म् । म॒न्वे॒ । अधीति॑ । नः॒ । ब्रु॒व॒न्तु॒ । प्रेति॑ । इ॒माम् । वाच᳚म् । विश्वा᳚म् । अ॒व॒न्तु॒ । विश्वे᳚ ॥ आ॒शून् । हु॒वे॒ । सु॒यमा॒निति॑ सु - यमान्॑ । ऊ॒तये᳚ । ते । नः॒ । मु॒ञ्च॒न्तु॒ । एन॑सः ॥ ति॒ग्मम् । आयु॑धम् । वी॒डि॒तम् । सह॑स्वत् । दि॒व्यम् । शर्द्धः॑ ।  \newline


\textbf{Krama Paata} \newline

वा॒म् दे॒वौ॒ । दे॒वौ॒ दे॒वेषु॑ । दे॒वेष्वनि॑शितम् । अनि॑शित॒मोजः॑ । अनि॑शित॒मित्यनि॑ - शि॒त॒॒म् । ओज॒स्तौ । 
तौ नः॑ । नो॒ मु॒ञ्च॒त॒म् । मु॒ञ्च॒त॒माग॑सः । आग॑स॒ इत्याग॑सः ॥ यदया॑तम् । अया॑तम् ॅवह॒तुम् । व॒ह॒तुꣳ सू॒र्यायाः᳚ । सू॒र्याया᳚स्त्रिच॒क्रेण॑ । त्रि॒च॒क्रेण॑ सꣳ॒॒सद᳚म् । त्रि॒च॒क्रेणेति॑ त्रि - च॒क्रेण॑ । सꣳ॒॒सद॑मि॒च्छमा॑नौ । सꣳ॒॒सद॒मिति॑ सम् - सद᳚म् । इ॒च्छमा॑ना॒विती॒च्छमा॑नौ ॥ स्तौमि॑ दे॒वौ । दे॒वाव॒श्विनौ᳚ । अ॒श्विनौ॑ नाथि॒तः । ना॒थि॒तो जो॑हवीमि । जो॒ह॒वी॒मि॒ तौ । तौ नः॑ । नो॒ मु॒ञ्च॒त॒म् । मु॒ञ्च॒त॒माग॑सः । आग॑स॒ इत्याग॑सः ॥ म॒रुता᳚म् मन्वे । म॒न्वे॒ अधि॑ । अधि॑ नः । नो॒ ब्रु॒व॒न्तु॒ । बु॒व॒न्तु॒ प्र । प्रेमाम् । इ॒माम् ॅवाच᳚म् । वाच॒म् ॅविश्वा᳚म् । विश्वा॑मवन्तु । अ॒व॒न्तु॒ विश्वे᳚ । विश्व॒ इति॒ विश्वे᳚ ॥ आ॒शून्. हु॑वे । हु॒वे॒ सु॒यमान्॑ । सु॒यमा॑नू॒तये᳚ । सु॒यमानिति॑ सु - यमान्॑ । ऊ॒तये॒ ते । ते नः॑ । नो॒ मु॒ञ्च॒न्तु॒ । मु॒ञ्च॒त्वेन॑सः । एन॑स॒ इत्येन॑सः ॥ ति॒ग्ममायु॑धम् । आयु॑धम् ॅवीडि॒तम् । वी॒डि॒तꣳ सह॑स्वत् । सह॑स्वद् दि॒व्यम् । दि॒व्यꣳ शर्द्धः॑ । शर्द्धः॒ पृत॑नासु \newline

\textbf{Jatai Paata} \newline

1. वा॒म् दे॒वौ॒ दे॒वौ॒ वां॒ ॅवा॒म् दे॒वौ॒ । \newline
2. दे॒वौ॒ दे॒वेषु॑ दे॒वेषु॑ देवौ देवौ दे॒वेषु॑ । \newline
3. दे॒वे ष्वनि॑शित॒ मनि॑शितम् दे॒वेषु॑ दे॒वे ष्वनि॑शितम् । \newline
4. अनि॑शित॒ मोज॒ ओजो ऽनि॑शित॒ मनि॑शित॒ मोजः॑ । \newline
5. अनि॑शित॒मित्यनि॑ - शि॒त॒म् । \newline
6. ओज॒ स्तौ ता वोज॒ ओज॒ स्तौ । \newline
7. तौ नो॑ न॒ स्तौ तौ नः॑ । \newline
8. नो॒ मु॒ञ्च॒त॒म् मु॒ञ्च॒त॒म् नो॒ नो॒ मु॒ञ्च॒त॒म् । \newline
9. मु॒ञ्च॒त॒ माग॑स॒ आग॑सो मुञ्चतम् मुञ्चत॒ माग॑सः । \newline
10. आग॑स॒ इत्याग॑सः । \newline
11. यदया॑त॒ मया॑तं॒ ॅयद् यदया॑तम् । \newline
12. अया॑तं ॅवह॒तुं ॅव॑ह॒तु मया॑त॒ मया॑तं ॅवह॒तुम् । \newline
13. व॒ह॒तुꣳ सू॒र्यायाः᳚ सू॒र्याया॑ वह॒तुं ॅव॑ह॒तुꣳ सू॒र्यायाः᳚ । \newline
14. सू॒र्याया᳚ स्त्रिच॒क्रेण॑ त्रिच॒क्रेण॑ सू॒र्यायाः᳚ सू॒र्याया᳚ स्त्रिच॒क्रेण॑ । \newline
15. त्रि॒च॒क्रेण॑ स॒(ग्म्॒)सद(ग्म्॑) स॒(ग्म्॒)सद॑म् त्रिच॒क्रेण॑ त्रिच॒क्रेण॑ स॒(ग्म्॒)सद᳚म् । \newline
16. त्रि॒च॒क्रेणेति॑ त्रि - च॒क्रेण॑ । \newline
17. स॒(ग्म्॒)सद॑ मि॒च्छमा॑ना वि॒च्छमा॑नौ स॒(ग्म्॒)सद(ग्म्॑) स॒(ग्म्॒)सद॑ मि॒च्छमा॑नौ । \newline
18. स॒(ग्म्॒)सद॒मिति॑ सम् - सद᳚म् । \newline
19. इ॒च्छमा॑ना॒विती॒च्छमा॑नौ । \newline
20. स्तौमि॑ दे॒वौ दे॒वौ स्तौमि॒ स्तौमि॑ दे॒वौ । \newline
21. दे॒वा व॒श्विना॑ व॒श्विनौ॑ दे॒वौ दे॒वा व॒श्विनौ᳚ । \newline
22. अ॒श्विनौ॑ नाथि॒तो ना॑थि॒तो अ॒श्विना॑ व॒श्विनौ॑ नाथि॒तः । \newline
23. ना॒थि॒तो जो॑हवीमि जोहवीमि नाथि॒तो ना॑थि॒तो जो॑हवीमि । \newline
24. जो॒ह॒वी॒मि॒ तौ तौ जो॑हवीमि जोहवीमि॒ तौ । \newline
25. तौ नो॑ न॒ स्तौ तौ नः॑ । \newline
26. नो॒ मु॒ञ्च॒त॒म् मु॒ञ्च॒त॒म् नो॒ नो॒ मु॒ञ्च॒त॒म् । \newline
27. मु॒ञ्च॒त॒ माग॑स॒ आग॑सो मुञ्चतम् मुञ्चत॒ माग॑सः । \newline
28. आग॑स॒ इत्याग॑सः । \newline
29. म॒रुता᳚म् मन्वे मन्वे म॒रुता᳚म् म॒रुता᳚म् मन्वे । \newline
30. म॒न्वे॒ अध्यधि॑ मन्वे मन्वे॒ अधि॑ । \newline
31. अधि॑ नो नो॒ अध्यधि॑ नः । \newline
32. नो॒ ब्रु॒व॒न्तु॒ ब्रु॒व॒न्तु॒ नो॒ नो॒ ब्रु॒व॒न्तु॒ । \newline
33. ब्रु॒व॒न्तु॒ प्र प्र ब्रु॑वन्तु ब्रुवन्तु॒ प्र । \newline
34. प्रेमा मि॒माम् प्र प्रेमाम् । \newline
35. इ॒मां ॅवाचं॒ ॅवाच॑ मि॒मा मि॒मां ॅवाच᳚म् । \newline
36. वाचं॒ ॅविश्वां॒ ॅविश्वां॒ ॅवाचं॒ ॅवाचं॒ ॅविश्वा᳚म् । \newline
37. विश्वा॑ मव न्त्ववन्तु॒ विश्वां॒ ॅविश्वा॑ मवन्तु । \newline
38. अ॒व॒न्तु॒ विश्वे॒ विश्वे॑ ऽवन्त्ववन्तु॒ विश्वे᳚ । \newline
39. विश्व॒ इति॒ विश्वे᳚ । \newline
40. आ॒शून्. हु॑वे हुव आ॒शू ना॒शून्. हु॑वे । \newline
41. हु॒वे॒ सु॒यमा᳚न् थ्सु॒यमान्॑. हुवे हुवे सु॒यमान्॑ । \newline
42. सु॒यमा॑ नू॒तय॑ ऊ॒तये॑ सु॒यमा᳚न् थ्सु॒यमा॑ नू॒तये᳚ । \newline
43. सु॒यमा॒निति॑ सु - यमान्॑ । \newline
44. ऊ॒तये॒ ते त ऊ॒तय॑ ऊ॒तये॒ ते । \newline
45. ते नो॑ न॒ स्ते ते नः॑ । \newline
46. नो॒ मु॒ञ्च॒न्तु॒ मु॒ञ्च॒न्तु॒ नो॒ नो॒ मु॒ञ्च॒न्तु॒ । \newline
47. मु॒ञ्च॒ न्त्वेन॑स॒ एन॑सो मुञ्चन्तु मुञ्च॒ न्त्वेन॑सः । \newline
48. एन॑स॒ इत्येन॑सः । \newline
49. ति॒ग्म मायु॑ध॒ मायु॑धम् ति॒ग्मम् ति॒ग्म मायु॑धम् । \newline
50. आयु॑धं ॅवीडि॒तं ॅवी॑डि॒त मायु॑ध॒ मायु॑धं ॅवीडि॒तम् । \newline
51. वी॒डि॒तꣳ सह॑स्व॒थ् सह॑स्वद् वीडि॒तं ॅवी॑डि॒तꣳ सह॑स्वत् । \newline
52. सह॑स्वद् दि॒व्यम् दि॒व्यꣳ सह॑स्व॒थ् सह॑स्वद् दि॒व्यम् । \newline
53. दि॒व्यꣳ शर्द्धः॒ शर्द्धो॑ दि॒व्यम् दि॒व्यꣳ शर्द्धः॑ । \newline
54. शर्द्धः॒ पृत॑नासु॒ पृत॑नासु॒ शर्द्धः॒ शर्द्धः॒ पृत॑नासु । \newline

\textbf{Ghana Paata } \newline

1. वा॒म् दे॒वौ॒ दे॒वौ॒ वां॒ ॅवा॒म् दे॒वौ॒ दे॒वेषु॑ दे॒वेषु॑ देवौ वां ॅवाम् देवौ दे॒वेषु॑ । \newline
2. दे॒वौ॒ दे॒वेषु॑ दे॒वेषु॑ देवौ देवौ दे॒वे ष्वनि॑शित॒ मनि॑शितम् दे॒वेषु॑ देवौ देवौ दे॒वे ष्वनि॑शितम् । \newline
3. दे॒वे ष्वनि॑शित॒ मनि॑शितम् दे॒वेषु॑ दे॒वे ष्वनि॑शित॒ मोज॒ ओजो ऽनि॑शितम् दे॒वेषु॑ दे॒वे ष्वनि॑शित॒ मोजः॑ । \newline
4. अनि॑शित॒ मोज॒ ओजो ऽनि॑शित॒ मनि॑शित॒ मोज॒ स्तौ ता वोजो ऽनि॑शित॒ मनि॑शित॒ मोज॒ स्तौ । \newline
5. अनि॑शित॒मित्यनि॑ - शि॒त॒म् । \newline
6. ओज॒ स्तौ ता वोज॒ ओज॒ स्तौ नो॑ न॒ स्ता वोज॒ ओज॒ स्तौ नः॑ । \newline
7. तौ नो॑ न॒ स्तौ तौ नो॑ मुञ्चतम् मुञ्चतम् न॒ स्तौ तौ नो॑ मुञ्चतम् । \newline
8. नो॒ मु॒ञ्च॒त॒म् मु॒ञ्च॒त॒म् नो॒ नो॒ मु॒ञ्च॒त॒ माग॑स॒ आग॑सो मुञ्चतम् नो नो मुञ्चत॒ माग॑सः । \newline
9. मु॒ञ्च॒त॒ माग॑स॒ आग॑सो मुञ्चतम् मुञ्चत॒ माग॑सः । \newline
10. आग॑स॒ इत्याग॑सः । \newline
11. यदया॑त॒ मया॑तं॒ ॅयद् यदया॑तं ॅवह॒तुं ॅव॑ह॒तु मया॑तं॒ ॅयद् यदया॑तं ॅवह॒तुम् । \newline
12. अया॑तं ॅवह॒तुं ॅव॑ह॒तु मया॑त॒ मया॑तं ॅवह॒तुꣳ सू॒र्यायाः᳚ सू॒र्याया॑ वह॒तु मया॑त॒ मया॑तं ॅवह॒तुꣳ सू॒र्यायाः᳚ । \newline
13. व॒ह॒तुꣳ सू॒र्यायाः᳚ सू॒र्याया॑ वह॒तुं ॅव॑ह॒तुꣳ सू॒र्याया᳚ स्त्रिच॒क्रेण॑ त्रिच॒क्रेण॑ सू॒र्याया॑ वह॒तुं ॅव॑ह॒तुꣳ सू॒र्याया᳚ स्त्रिच॒क्रेण॑ । \newline
14. सू॒र्याया᳚ स्त्रिच॒क्रेण॑ त्रिच॒क्रेण॑ सू॒र्यायाः᳚ सू॒र्याया᳚ स्त्रिच॒क्रेण॑ स॒(ग्म्॒)सद(ग्म्॑) स॒(ग्म्॒)सद॑म् त्रिच॒क्रेण॑ सू॒र्यायाः᳚ सू॒र्याया᳚ स्त्रिच॒क्रेण॑ स॒(ग्म्॒)सद᳚म् । \newline
15. त्रि॒च॒क्रेण॑ स॒(ग्म्॒)सद(ग्म्॑) स॒(ग्म्॒)सद॑म् त्रिच॒क्रेण॑ त्रिच॒क्रेण॑ स॒(ग्म्॒)सद॑ मि॒च्छमा॑ना वि॒च्छमा॑नौ स॒(ग्म्॒)सद॑म् त्रिच॒क्रेण॑ त्रिच॒क्रेण॑ स॒(ग्म्॒)सद॑ मि॒च्छमा॑नौ । \newline
16. त्रि॒च॒क्रेणेति॑ त्रि - च॒क्रेण॑ । \newline
17. स॒(ग्म्॒)सद॑ मि॒च्छमा॑ना वि॒च्छमा॑नौ स॒(ग्म्॒)सद(ग्म्॑) स॒(ग्म्॒)सद॑ मि॒च्छमा॑नौ । \newline
18. स॒(ग्म्॒)सद॒मिति॑ सम् - सद᳚म् । \newline
19. इ॒च्छमा॑ना॒ विती॒च्छमा॑नौ । \newline
20. स्तौमि॑ दे॒वौ दे॒वौ स्तौमि॒ स्तौमि॑ दे॒वा व॒श्विना॑ व॒श्विनौ॑ दे॒वौ स्तौमि॒ स्तौमि॑ दे॒वा व॒श्विनौ᳚ । \newline
21. दे॒वा व॒श्विना॑ व॒श्विनौ॑ दे॒वौ दे॒वा व॒श्विनौ॑ नाथि॒तो ना॑थि॒तो अ॒श्विनौ॑ दे॒वौ दे॒वा व॒श्विनौ॑ नाथि॒तः । \newline
22. अ॒श्विनौ॑ नाथि॒तो ना॑थि॒तो अ॒श्विना॑ व॒श्विनौ॑ नाथि॒तो जो॑हवीमि जोहवीमि नाथि॒तो अ॒श्विना॑ व॒श्विनौ॑ नाथि॒तो जो॑हवीमि । \newline
23. ना॒थि॒तो जो॑हवीमि जोहवीमि नाथि॒तो ना॑थि॒तो जो॑हवीमि॒ तौ तौ जो॑हवीमि नाथि॒तो ना॑थि॒तो जो॑हवीमि॒ तौ । \newline
24. जो॒ह॒वी॒मि॒ तौ तौ जो॑हवीमि जोहवीमि॒ तौ नो॑ न॒ स्तौ जो॑हवीमि जोहवीमि॒ तौ नः॑ । \newline
25. तौ नो॑ न॒ स्तौ तौ नो॑ मुञ्चतम् मुञ्चतम् न॒ स्तौ तौ नो॑ मुञ्चतम् । \newline
26. नो॒ मु॒ञ्च॒त॒म् मु॒ञ्च॒त॒म् नो॒ नो॒ मु॒ञ्च॒त॒ माग॑स॒ आग॑सो मुञ्चतम् नो नो मुञ्चत॒ माग॑सः । \newline
27. मु॒ञ्च॒त॒ माग॑स॒ आग॑सो मुञ्चतम् मुञ्चत॒ माग॑सः । \newline
28. आग॑स॒ इत्याग॑सः । \newline
29. म॒रुता᳚म् मन्वे मन्वे म॒रुता᳚म् म॒रुता᳚म् मन्वे॒ अध्यधि॑ मन्वे म॒रुता᳚म् म॒रुता᳚म् मन्वे॒ अधि॑ । \newline
30. म॒न्वे॒ अध्यधि॑ मन्वे मन्वे॒ अधि॑ नो नो॒ अधि॑ मन्वे मन्वे॒ अधि॑ नः । \newline
31. अधि॑ नो नो॒ अध्यधि॑ नो ब्रुवन्तु ब्रुवन्तु नो॒ अध्यधि॑ नो ब्रुवन्तु । \newline
32. नो॒ ब्रु॒व॒न्तु॒ ब्रु॒व॒न्तु॒ नो॒ नो॒ ब्रु॒व॒न्तु॒ प्र प्र ब्रु॑वन्तु नो नो ब्रुवन्तु॒ प्र । \newline
33. ब्रु॒व॒न्तु॒ प्र प्र ब्रु॑वन्तु ब्रुवन्तु॒ प्रेमा मि॒माम् प्र ब्रु॑वन्तु ब्रुवन्तु॒ प्रेमाम् । \newline
34. प्रेमा मि॒माम् प्र प्रेमां ॅवाचं॒ ॅवाच॑ मि॒माम् प्र प्रेमां ॅवाच᳚म् । \newline
35. इ॒मां ॅवाचं॒ ॅवाच॑ मि॒मा मि॒मां ॅवाचं॒ ॅविश्वां॒ ॅविश्वां॒ ॅवाच॑ मि॒मा मि॒मां ॅवाचं॒ ॅविश्वा᳚म् । \newline
36. वाचं॒ ॅविश्वां॒ ॅविश्वां॒ ॅवाचं॒ ॅवाचं॒ ॅविश्वा॑ मवन् त्ववन्तु॒ विश्वां॒ ॅवाचं॒ ॅवाचं॒ ॅविश्वा॑ मवन्तु । \newline
37. विश्वा॑ मवन् त्ववन्तु॒ विश्वां॒ ॅविश्वा॑ मवन्तु॒ विश्वे॒ विश्वे॑ ऽवन्तु॒ विश्वां॒ ॅविश्वा॑ मवन्तु॒ विश्वे᳚ । \newline
38. अ॒व॒न्तु॒ विश्वे॒ विश्वे॑ ऽवन्त्ववन्तु॒ विश्वे᳚ । \newline
39. विश्व॒ इति॒ विश्वे᳚ । \newline
40. आ॒शून्. हु॑वे हुव आ॒शू ना॒शून्. हु॑वे सु॒यमा᳚न् थ्सु॒यमान्॑. हुव आ॒शू ना॒शून्. हु॑वे सु॒यमान्॑ । \newline
41. हु॒वे॒ सु॒यमा᳚न् थ्सु॒यमान्॑. हुवे हुवे सु॒यमा॑ नू॒तय॑ ऊ॒तये॑ सु॒यमान्॑. हुवे हुवे सु॒यमा॑ नू॒तये᳚ । \newline
42. सु॒यमा॑ नू॒तय॑ ऊ॒तये॑ सु॒यमा᳚न् थ्सु॒यमा॑ नू॒तये॒ ते त ऊ॒तये॑ सु॒यमा᳚न् थ्सु॒यमा॑ नू॒तये॒ ते । \newline
43. सु॒यमा॒निति॑ सु - यमान्॑ । \newline
44. ऊ॒तये॒ ते त ऊ॒तय॑ ऊ॒तये॒ ते नो॑ न॒ स्त ऊ॒तय॑ ऊ॒तये॒ ते नः॑ । \newline
45. ते नो॑ न॒ स्ते ते नो॑ मुञ्चन्तु मुञ्चन्तु न॒ स्ते ते नो॑ मुञ्चन्तु । \newline
46. नो॒ मु॒ञ्च॒न्तु॒ मु॒ञ्च॒न्तु॒ नो॒ नो॒ मु॒ञ्च॒ न्त्वेन॑स॒ एन॑सो मुञ्चन्तु नो नो मुञ्च॒न् त्वेन॑सः । \newline
47. मु॒ञ्च॒न् त्वेन॑स॒ एन॑सो मुञ्चन्तु मुञ्च॒न् त्वेन॑सः । \newline
48. एन॑स॒ इत्येन॑सः । \newline
49. ति॒ग्म मायु॑ध॒ मायु॑धम् ति॒ग्मम् ति॒ग्म मायु॑धं ॅवीडि॒तं ॅवी॑डि॒त मायु॑धम् ति॒ग्मम् ति॒ग्म मायु॑धं ॅवीडि॒तम् । \newline
50. आयु॑धं ॅवीडि॒तं ॅवी॑डि॒त मायु॑ध॒ मायु॑धं ॅवीडि॒तꣳ सह॑स्व॒थ् सह॑स्वद् वीडि॒त मायु॑ध॒ मायु॑धं ॅवीडि॒तꣳ सह॑स्वत् । \newline
51. वी॒डि॒तꣳ सह॑स्व॒थ् सह॑स्वद् वीडि॒तं ॅवी॑डि॒तꣳ सह॑स्वद् दि॒व्यम् दि॒व्यꣳ सह॑स्वद् वीडि॒तं ॅवी॑डि॒तꣳ सह॑स्वद् दि॒व्यम् । \newline
52. सह॑स्वद् दि॒व्यम् दि॒व्यꣳ सह॑स्व॒थ् सह॑स्वद् दि॒व्यꣳ शर्द्धः॒ शर्द्धो॑ दि॒व्यꣳ सह॑स्व॒थ् सह॑स्वद् दि॒व्यꣳ शर्द्धः॑ । \newline
53. दि॒व्यꣳ शर्द्धः॒ शर्द्धो॑ दि॒व्यम् दि॒व्यꣳ शर्द्धः॒ पृत॑नासु॒ पृत॑नासु॒ शर्द्धो॑ दि॒व्यम् दि॒व्यꣳ शर्द्धः॒ पृत॑नासु । \newline
54. शर्द्धः॒ पृत॑नासु॒ पृत॑नासु॒ शर्द्धः॒ शर्द्धः॒ पृत॑नासु जि॒ष्णु जि॒ष्णु पृत॑नासु॒ शर्द्धः॒ शर्द्धः॒ पृत॑नासु जि॒ष्णु । \newline
\pagebreak
\markright{ TS 4.7.15.5  \hfill https://www.vedavms.in \hfill}

\section{ TS 4.7.15.5 }

\textbf{TS 4.7.15.5 } \newline
\textbf{Samhita Paata} \newline

पृत॑नासु जि॒ष्णु । स्तौमि॑ दे॒वान् म॒रुतो॑ नाथि॒तो जो॑हवीमि॒ ते नो॑ मुञ्च॒न्त्वेन॑सः ॥ दे॒वानां᳚ मन्वे॒ अधि॑ नो ब्रुवन्तु॒ प्रेमां ॅवाचं॒ ॅविश्वा॑मवन्तु॒ विश्वे᳚ । आ॒शून्. हु॑वे सु॒यमा॑नू॒तये॒ ते नो॑ मुञ्च॒न्त्वेन॑सः ॥ यदि॒दं मा॑ऽभि॒शोच॑ति॒ पौरु॑षेयेण॒ दैव्ये॑न ।स्तौमि॒ विश्वा᳚न् दे॒वान् ना॑थि॒तो जो॑हवीमि॒ ते नो॑ मुञ्च॒न्त्वेन॑सः ॥अनु॑नो॒ऽद्यानु॑मति॒ >1, रन्वि - [  ] \newline

\textbf{Pada Paata} \newline

पृत॑नासु । जि॒ष्णु ॥ स्तौमि॑ । दे॒वान् । म॒रुतः॑ । ना॒थि॒तः । जो॒ह॒वी॒मि॒ । ते । नः॒ । मु॒ञ्च॒न्तु॒ । एन॑सः ॥ दे॒वाना᳚म् । म॒न्वे॒ । अधीति॑ । नः॒ । ब्रु॒व॒न्तु॒ । प्रेति॑ । इ॒माम् । वाच᳚म् । विश्वा᳚म् । अ॒व॒न्तु॒ । विश्वे᳚ ॥ आ॒शून् । हु॒वे॒ । सु॒यमा॒निति॑ सु - यमान्॑ । ऊ॒तये᳚ । ते । नः॒ । मु॒ञ्च॒न्तु॒ । एन॑सः ॥ यत् । इ॒दम् । मा॒ । अ॒भि॒शोच॒तीत्य॑भि-शोच॑ति । पौरु॑षेयेण । दैव्ये॑न ॥ स्तौमि॑ । विश्वान्॑ । दे॒वान् । ना॒थि॒तः । जो॒ह॒वी॒मि॒ । ते । नः॒ । मु॒ञ्च॒न्तु॒ । एन॑सः ॥ अन्विति॑ । नः॒ । अ॒द्य । अनु॑मति॒रियनु॑ - म॒तिः॒ । अन्विति॑ ।  \newline


\textbf{Krama Paata} \newline

पृत॑नासु जि॒ष्णु । जि॒ष्ण्विति॑ जि॒ष्णु ॥ स्तौमि॑ दे॒वान् । दे॒वान् म॒रुतः॑ । म॒रुतो॑ नाथि॒तः । ना॒थि॒तो जो॑हवीमि । जो॒ह॒वी॒मि॒ ते । ते नः॑ । नो॒ मु॒ञ्च॒न्तु॒ । मु॒ञ्च॒न्त्वेन॑सः । एन॑स॒ इत्येन॑सः । दे॒वाना᳚म् मन्वे । म॒न्वे॒ अधि॑ । अधि॑ नः । नो॒ ब्रु॒व॒न्तु॒ । ब्रु॒व॒न्तु॒ प्र । प्रेमाम् । इ॒माम् ॅवाच᳚म् । वाच॒म् ॅविश्वा᳚म् । विश्वा॑मवन्तु । अ॒व॒न्तु॒ विश्वे᳚ । विश्व॒ इति॒ विश्वे᳚ ॥ आ॒शून्. हु॑वे । हु॒वे॒ सु॒यमान्॑ । सु॒यमा॑नू॒तये᳚ । सु॒यमा॒निति॑ सु - यमान्॑ । ऊ॒तये॒ ते । ते नः॑ । नो॒ मु॒ञ्च॒न्तु॒ । मु॒ञ्च॒न्त्वेन॑सः । एन॑स॒ इत्येन॑सः ॥ यदि॒दम् । इ॒दम् मा᳚ । मा॒ऽभि॒शोच॑ति । अ॒भि॒शोच॑ति॒ पौरु॑षेयेण । अ॒भि॒शोच॒तीत्य॑भि - शोच॑ति । पौरु॑षेयेण॒ दैव्ये॑न । दैव्ये॒नेति॒ दैव्ये॑न ॥ स्तौमि॒ विश्वान्॑ । विश्वा᳚न् दे॒वान् । दे॒वान् ना॑थि॒तः । ना॒थि॒तो जो॑हवीमि । जो॒ह॒वी॒मि॒ ते । ते नः॑ । नो॒ मु॒ञ्च॒न्तु॒ । मु॒ञ्च॒न्त्वेन॑सः । एन॑स॒ इत्येन॑सः । अनु॑ नः । नो॒ऽद्य । अ॒द्यानु॑मतिः । अनु॑मति॒रनु॑ । अनु॑मति॒रित्यनु॑ - म॒तिः॒ । अन्वित् \newline

\textbf{Jatai Paata} \newline

1. पृत॑नासु जि॒ष्णु जि॒ष्णु पृत॑नासु॒ पृत॑नासु जि॒ष्णु । \newline
2. जि॒ष्ण्विति॑ जि॒ष्णु । \newline
3. स्तौमि॑ दे॒वान् दे॒वान् थ्स्तौमि॒ स्तौमि॑ दे॒वान् । \newline
4. दे॒वान् म॒रुतो॑ म॒रुतो॑ दे॒वान् दे॒वान् म॒रुतः॑ । \newline
5. म॒रुतो॑ नाथि॒तो ना॑थि॒तो म॒रुतो॑ म॒रुतो॑ नाथि॒तः । \newline
6. ना॒थि॒तो जो॑हवीमि जोहवीमि नाथि॒तो ना॑थि॒तो जो॑हवीमि । \newline
7. जो॒ह॒वी॒मि॒ ते ते जो॑हवीमि जोहवीमि॒ ते । \newline
8. ते नो॑ न॒ स्ते ते नः॑ । \newline
9. नो॒ मु॒ञ्च॒न्तु॒ मु॒ञ्च॒न्तु॒ नो॒ नो॒ मु॒ञ्च॒न्तु॒ । \newline
10. मु॒ञ्च॒ न्त्वेन॑स॒ एन॑सो मुञ्चन्तु मुञ्च॒ न्त्वेन॑सः । \newline
11. एन॑स॒ इत्येन॑सः । \newline
12. दे॒वाना᳚म् मन्वे मन्वे दे॒वाना᳚म् दे॒वाना᳚म् मन्वे । \newline
13. म॒न्वे॒ अध्यधि॑ मन्वे मन्वे॒ अधि॑ । \newline
14. अधि॑ नो नो॒ अध्यधि॑ नः । \newline
15. नो॒ ब्रु॒व॒न्तु॒ ब्रु॒व॒न्तु॒ नो॒ नो॒ ब्रु॒व॒न्तु॒ । \newline
16. ब्रु॒व॒न्तु॒ प्र प्र ब्रु॑वन्तु ब्रुवन्तु॒ प्र । \newline
17. प्रेमा मि॒माम् प्र प्रेमाम् । \newline
18. इ॒मां ॅवाचं॒ ॅवाच॑ मि॒मा मि॒मां ॅवाच᳚म् । \newline
19. वाचं॒ ॅविश्वां॒ ॅविश्वां॒ ॅवाचं॒ ॅवाचं॒ ॅविश्वा᳚म् । \newline
20. विश्वा॑ मव न्त्ववन्तु॒ विश्वां॒ ॅविश्वा॑ मवन्तु । \newline
21. अ॒व॒न्तु॒ विश्वे॒ विश्वे॑ ऽवन् त्ववन्तु॒ विश्वे᳚ । \newline
22. विश्व॒ इति॒ विश्वे᳚ । \newline
23. आ॒शून्. हु॑वे हुव आ॒शू ना॒शून्. हु॑वे । \newline
24. हु॒वे॒ सु॒यमा᳚न् थ्सु॒यमान्॑. हुवे हुवे सु॒यमान्॑ । \newline
25. सु॒यमा॑ नू॒तय॑ ऊ॒तये॑ सु॒यमा᳚न् थ्सु॒यमा॑ नू॒तये᳚ । \newline
26. सु॒यमा॒निति॑ सु - यमान्॑ । \newline
27. ऊ॒तये॒ ते त ऊ॒तय॑ ऊ॒तये॒ ते । \newline
28. ते नो॑ न॒ स्ते ते नः॑ । \newline
29. नो॒ मु॒ञ्च॒न्तु॒ मु॒ञ्च॒न्तु॒ नो॒ नो॒ मु॒ञ्च॒न्तु॒ । \newline
30. मु॒ञ्च॒ न्त्वेन॑स॒ एन॑सो मुञ्चन्तु मुञ्च॒ न्त्वेन॑सः । \newline
31. एन॑स॒ इत्येन॑सः । \newline
32. यदि॒द मि॒दं ॅयद् यदि॒दम् । \newline
33. इ॒दम् मा॑ मे॒द मि॒दम् मा᳚ । \newline
34. मा॒ ऽभि॒शोच॑ त्यभि॒शोच॑ति मा मा ऽभि॒शोच॑ति । \newline
35. अ॒भि॒शोच॑ति॒ पौरु॑षेयेण॒ पौरु॑षेयेणा भि॒शोच॑त्य भि॒शोच॑ति॒ पौरु॑षेयेण । \newline
36. अ॒भि॒शोच॒तीत्य॑भि - शोच॑ति । \newline
37. पौरु॑षेयेण॒ दैव्ये॑न॒ दैव्ये॑न॒ पौरु॑षेयेण॒ पौरु॑षेयेण॒ दैव्ये॑न । \newline
38. दैव्ये॒नेति॒ दैव्ये॑न । \newline
39. स्तौमि॒ विश्वा॒न्॒. विश्वा॒न् थ्स्तौमि॒ स्तौमि॒ विश्वान्॑ । \newline
40. विश्वा᳚न् दे॒वान् दे॒वान्. विश्वा॒न्॒. विश्वा᳚न् दे॒वान् । \newline
41. दे॒वान् ना॑थि॒तो ना॑थि॒तो दे॒वान् दे॒वान् ना॑थि॒तः । \newline
42. ना॒थि॒तो जो॑हवीमि जोहवीमि नाथि॒तो ना॑थि॒तो जो॑हवीमि । \newline
43. जो॒ह॒वी॒मि॒ ते ते जो॑हवीमि जोहवीमि॒ ते । \newline
44. ते नो॑ न॒ स्ते ते नः॑ । \newline
45. नो॒ मु॒ञ्च॒न्तु॒ मु॒ञ्च॒न्तु॒ नो॒ नो॒ मु॒ञ्च॒न्तु॒ । \newline
46. मु॒ञ्च॒ न्त्वेन॑स॒ एन॑सो मुञ्चन्तु मुञ्च॒ न्त्वेन॑सः । \newline
47. एन॑स॒ इत्येन॑सः । \newline
48. अनु॑ नो नो॒ अन्वनु॑ नः । \newline
49. नो॒ ऽद्याद्य नो॑ नो॒ ऽद्य । \newline
50. अ॒द्या नु॑मति॒ रनु॑मति र॒द्याद्या नु॑मतिः । \newline
51. अनु॑मति॒ रन्वन् वनु॑मति॒ रनु॑मति॒ रनु॑ । \newline
52. अनु॑मति॒रियनु॑ - म॒तिः॒ । \newline
53. अन्विदि दन् वन् वित् । \newline

\textbf{Ghana Paata } \newline

1. पृत॑नासु जि॒ष्णु जि॒ष्णु पृत॑नासु॒ पृत॑नासु जि॒ष्णु । \newline
2. जि॒ष्ण्विति॑ जि॒ष्णु । \newline
3. स्तौमि॑ दे॒वान् दे॒वान् थ्स्तौमि॒ स्तौमि॑ दे॒वान् म॒रुतो॑ म॒रुतो॑ दे॒वान् थ्स्तौमि॒ स्तौमि॑ दे॒वान् म॒रुतः॑ । \newline
4. दे॒वान् म॒रुतो॑ म॒रुतो॑ दे॒वान् दे॒वान् म॒रुतो॑ नाथि॒तो ना॑थि॒तो म॒रुतो॑ दे॒वान् दे॒वान् म॒रुतो॑ नाथि॒तः । \newline
5. म॒रुतो॑ नाथि॒तो ना॑थि॒तो म॒रुतो॑ म॒रुतो॑ नाथि॒तो जो॑हवीमि जोहवीमि नाथि॒तो म॒रुतो॑ म॒रुतो॑ नाथि॒तो जो॑हवीमि । \newline
6. ना॒थि॒तो जो॑हवीमि जोहवीमि नाथि॒तो ना॑थि॒तो जो॑हवीमि॒ ते ते जो॑हवीमि नाथि॒तो ना॑थि॒तो जो॑हवीमि॒ ते । \newline
7. जो॒ह॒वी॒मि॒ ते ते जो॑हवीमि जोहवीमि॒ ते नो॑ न॒ स्ते जो॑हवीमि जोहवीमि॒ ते नः॑ । \newline
8. ते नो॑ न॒ स्ते ते नो॑ मुञ्चन्तु मुञ्चन्तु न॒ स्ते ते नो॑ मुञ्चन्तु । \newline
9. नो॒ मु॒ञ्च॒न्तु॒ मु॒ञ्च॒न्तु॒ नो॒ नो॒ मु॒ञ्च॒न् त्वेन॑स॒ एन॑सो मुञ्चन्तु नो नो मुञ्च॒न् त्वेन॑सः । \newline
10. मु॒ञ्च॒न् त्वेन॑स॒ एन॑सो मुञ्चन्तु मुञ्च॒न् त्वेन॑सः । \newline
11. एन॑स॒ इत्येन॑सः । \newline
12. दे॒वाना᳚म् मन्वे मन्वे दे॒वाना᳚म् दे॒वाना᳚म् मन्वे॒ अध्यधि॑ मन्वे दे॒वाना᳚म् दे॒वाना᳚म् मन्वे॒ अधि॑ । \newline
13. म॒न्वे॒ अध्यधि॑ मन्वे मन्वे॒ अधि॑ नो नो॒ अधि॑ मन्वे मन्वे॒ अधि॑ नः । \newline
14. अधि॑ नो नो॒ अध्यधि॑ नो ब्रुवन्तु ब्रुवन्तु नो॒ अध्यधि॑ नो ब्रुवन्तु । \newline
15. नो॒ ब्रु॒व॒न्तु॒ ब्रु॒व॒न्तु॒ नो॒ नो॒ ब्रु॒व॒न्तु॒ प्र प्र ब्रु॑वन्तु नो नो ब्रुवन्तु॒ प्र । \newline
16. ब्रु॒व॒न्तु॒ प्र प्र ब्रु॑वन्तु ब्रुवन्तु॒ प्रेमा मि॒माम् प्र ब्रु॑वन्तु ब्रुवन्तु॒ प्रेमाम् । \newline
17. प्रेमा मि॒माम् प्र प्रेमां ॅवाचं॒ ॅवाच॑ मि॒माम् प्र प्रेमां ॅवाच᳚म् । \newline
18. इ॒मां ॅवाचं॒ ॅवाच॑ मि॒मा मि॒मां ॅवाचं॒ ॅविश्वां॒ ॅविश्वां॒ ॅवाच॑ मि॒मा मि॒मां ॅवाचं॒ ॅविश्वा᳚म् । \newline
19. वाचं॒ ॅविश्वां॒ ॅविश्वां॒ ॅवाचं॒ ॅवाचं॒ ॅविश्वा॑ मवन् त्ववन्तु॒ विश्वां॒ ॅवाचं॒ ॅवाचं॒ ॅविश्वा॑ मवन्तु । \newline
20. विश्वा॑ मवन् त्ववन्तु॒ विश्वां॒ ॅविश्वा॑ मवन्तु॒ विश्वे॒ विश्वे॑ ऽवन्तु॒ विश्वां॒ ॅविश्वा॑ मवन्तु॒ विश्वे᳚ । \newline
21. अ॒व॒न्तु॒ विश्वे॒ विश्वे॑ ऽवन् त्ववन्तु॒ विश्वे᳚ । \newline
22. विश्व॒ इति॒ विश्वे᳚ । \newline
23. आ॒शून्. हु॑वे हुव आ॒शू ना॒शून्. हु॑वे सु॒यमा᳚न् थ्सु॒यमान्॑. हुव आ॒शू ना॒शून्. हु॑वे सु॒यमान्॑ । \newline
24. हु॒वे॒ सु॒यमा᳚न् थ्सु॒यमान्॑. हुवे हुवे सु॒यमा॑ नू॒तय॑ ऊ॒तये॑ सु॒यमान्॑. हुवे हुवे सु॒यमा॑ नू॒तये᳚ । \newline
25. सु॒यमा॑ नू॒तय॑ ऊ॒तये॑ सु॒यमा᳚न् थ्सु॒यमा॑ नू॒तये॒ ते त ऊ॒तये॑ सु॒यमा᳚न् थ्सु॒यमा॑ नू॒तये॒ ते । \newline
26. सु॒यमा॒निति॑ सु - यमान्॑ । \newline
27. ऊ॒तये॒ ते त ऊ॒तय॑ ऊ॒तये॒ ते नो॑ न॒ स्त ऊ॒तय॑ ऊ॒तये॒ ते नः॑ । \newline
28. ते नो॑ न॒ स्ते ते नो॑ मुञ्चन्तु मुञ्चन्तु न॒ स्ते ते नो॑ मुञ्चन्तु । \newline
29. नो॒ मु॒ञ्च॒न्तु॒ मु॒ञ्च॒न्तु॒ नो॒ नो॒ मु॒ञ्च॒न् त्वेन॑स॒ एन॑सो मुञ्चन्तु नो नो मुञ्च॒न् त्वेन॑सः । \newline
30. मु॒ञ्च॒न् त्वेन॑स॒ एन॑सो मुञ्चन्तु मुञ्च॒न् त्वेन॑सः । \newline
31. एन॑स॒ इत्येन॑सः । \newline
32. यदि॒द मि॒दं ॅयद् यदि॒दम् मा॑ मे॒दं ॅयद् यदि॒दम् मा᳚ । \newline
33. इ॒दम् मा॑ मे॒द मि॒दम् मा॑ ऽभि॒शोच॑ त्यभि॒शोच॑ति मे॒द मि॒दम् मा॑ ऽभि॒शोच॑ति । \newline
34. मा॒ ऽभि॒शोच॑ त्यभि॒शोच॑ति मा मा ऽभि॒शोच॑ति॒ पौरु॑षेयेण॒ पौरु॑षेयेणा भि॒शोच॑ति मा मा ऽभि॒शोच॑ति॒ पौरु॑षेयेण । \newline
35. अ॒भि॒शोच॑ति॒ पौरु॑षेयेण॒ पौरु॑षेयेणा भि॒शोच॑ त्यभि॒शोच॑ति॒ पौरु॑षेयेण॒ दैव्ये॑न॒ दैव्ये॑न॒ पौरु॑षेयेणा भि॒शोच॑ त्यभि॒शोच॑ति॒ पौरु॑षेयेण॒ दैव्ये॑न । \newline
36. अ॒भि॒शोच॒तीत्य॑भि - शोच॑ति । \newline
37. पौरु॑षेयेण॒ दैव्ये॑न॒ दैव्ये॑न॒ पौरु॑षेयेण॒ पौरु॑षेयेण॒ दैव्ये॑न । \newline
38. दैव्ये॒नेति॒ दैव्ये॑न । \newline
39. स्तौमि॒ विश्वा॒न्॒. विश्वा॒न् थ्स्तौमि॒ स्तौमि॒ विश्वा᳚न् दे॒वान् दे॒वान्. विश्वा॒न् थ्स्तौमि॒ स्तौमि॒ विश्वा᳚न् दे॒वान् । \newline
40. विश्वा᳚न् दे॒वान् दे॒वान्. विश्वा॒न्॒. विश्वा᳚न् दे॒वान् ना॑थि॒तो ना॑थि॒तो दे॒वान्. विश्वा॒न्॒. विश्वा᳚न् दे॒वान् ना॑थि॒तः । \newline
41. दे॒वान् ना॑थि॒तो ना॑थि॒तो दे॒वान् दे॒वान् ना॑थि॒तो जो॑हवीमि जोहवीमि नाथि॒तो दे॒वान् दे॒वान् ना॑थि॒तो जो॑हवीमि । \newline
42. ना॒थि॒तो जो॑हवीमि जोहवीमि नाथि॒तो ना॑थि॒तो जो॑हवीमि॒ ते ते जो॑हवीमि नाथि॒तो ना॑थि॒तो जो॑हवीमि॒ ते । \newline
43. जो॒ह॒वी॒मि॒ ते ते जो॑हवीमि जोहवीमि॒ ते नो॑ न॒ स्ते जो॑हवीमि जोहवीमि॒ ते नः॑ । \newline
44. ते नो॑ न॒ स्ते ते नो॑ मुञ्चन्तु मुञ्चन्तु न॒ स्ते ते नो॑ मुञ्चन्तु । \newline
45. नो॒ मु॒ञ्च॒न्तु॒ मु॒ञ्च॒न्तु॒ नो॒ नो॒ मु॒ञ्च॒न् त्वेन॑स॒ एन॑सो मुञ्चन्तु नो नो मुञ्च॒न् त्वेन॑सः । \newline
46. मु॒ञ्च॒न् त्वेन॑स॒ एन॑सो मुञ्चन्तु मुञ्च॒न् त्वेन॑सः । \newline
47. एन॑स॒ इत्येन॑सः । \newline
48. अनु॑ नो नो॒ अन्वनु॑ नो॒ ऽद्याद्य नो॒ अन्वनु॑ नो॒ ऽद्य । \newline
49. नो॒ ऽद्याद्य नो॑ नो॒ ऽद्यानु॑मति॒ रनु॑मति र॒द्य नो॑ नो॒ ऽद्यानु॑मतिः । \newline
50. अ॒द्यानु॑मति॒ रनु॑मति र॒द्याद्या नु॑मति॒ रन्वन् वनु॑मति र॒द्याद्या नु॑मति॒ रनु॑ । \newline
51. अनु॑मति॒ रन्वन् वनु॑मति॒ रनु॑मति॒ रन्विदि दन् वनु॑मति॒ रनु॑मति॒ रन्वित् । \newline
52. अनु॑मति॒रियनु॑ - म॒तिः॒ । \newline
53. अन्वि दिदन् वन् विद॑नुमते ऽनुमत॒ इदन् वन् विद॑नुमते । \newline
\pagebreak
\markright{ TS 4.7.15.6  \hfill https://www.vedavms.in \hfill}

\section{ TS 4.7.15.6 }

\textbf{TS 4.7.15.6 } \newline
\textbf{Samhita Paata} \newline

द॑नुमते॒ त्वं >2, ॅवै᳚श्वान॒रो न॑ ऊ॒त्या>3, पृ॒ष्टो दि॒वि>ये अप्र॑थेता॒-ममि॑तेभि॒ रोजो॑भि॒ र्ये प्र॑ति॒ष्ठे अभ॑वतां॒ ॅवसू॑नां ।स्तौमि॒ द्यावा॑ पृथि॒वी ना॑थि॒तो जो॑हवीमि॒ ते नो॑ मुञ्चत॒मꣳ ह॑सः ॥उर्वी॑ रोदसी॒ वरि॑वः कृणोतं॒ क्षेत्र॑स्य पत्नी॒ अधि॑ नो ब्रूयातं ।स्तौमि॒ द्यावा॑ पृथि॒वी ना॑थि॒तो जो॑हवीमि॒ ते नो॑ मुञ्चत॒मꣳ ह॑सः ॥यत् ते॑ व॒यं पु॑रुष॒त्रा य॑वि॒ष्ठा वि॑द्वाꣳसश्चकृ॒मा कच्च॒ ना - [  ] \newline

\textbf{Pada Paata} \newline

इत् । अ॒नु॒म॒त॒ इत्य॑नु - म॒ते॒ । त्वम् । वै॒श्वा॒न॒रः । नः॒ । ऊ॒त्या । पृ॒ष्टः । दि॒वि ॥ ये इति॑ । अप्र॑थेताम् । अमि॑तेभिः । ओजो॑भि॒रित्योजः॑ - भिः॒ । ये इति॑ । प्र॒ति॒ष्ठे इति॑ प्रति - स्थे । अभ॑वताम् । वसू॑नाम् ॥ स्तौमि॑ । द्यावा॑पृथि॒वी इति॒ द्यावा᳚ - पृ॒थि॒वी । ना॒थि॒तः । जो॒ह॒वी॒मि॒ । ते इति॑ । नः॒ । मु॒ञ्च॒त॒म् । अꣳह॑सः ॥ उर्वी॒ इति॑ । रो॒द॒सी॒ इति॑ । वरि॑वः । कृ॒णो॒त॒म् । क्षेत्र॑स्य । प॒त्नी॒ इति॑ । अधीति॑ । नः॒ । ब्रू॒या॒त॒म् ॥ स्तौमि॑ । द्यावा॑पृथि॒वी इति॒ द्यावा᳚ - पृ॒थि॒वी । ना॒थि॒तः । जो॒ह॒वी॒मि॒ । ते इति॑ । नः॒ । मु॒ञ्च॒त॒म् । अꣳह॑सः ॥ यत् । ते॒ । व॒यम् । पु॒रु॒ष॒त्रेति॑ पुरुष-त्रा । य॒वि॒ष्ठ॒ । अवि॑द्वाꣳसः । च॒कृ॒म । कत् । च॒न ।  \newline


\textbf{Krama Paata} \newline

इद॑नुमते । अ॒नु॒म॒ते॒ त्वम् । अ॒नु॒म॒त॒ इत्य॑नु - म॒ते॒ । त्वम् ॅवै᳚श्वान॒रः । वै॒श्वा॒न॒रो नः॑ । न॒ ऊ॒त्या । ऊ॒त्या पृ॒ष्टः । पृ॒ष्टो दि॒वि । दि॒वीति॑ दि॒वि ॥ ये अप्र॑थेताम् । ये इति॒ ये । अप्र॑थेता॒ममि॑तेभिः । अमि॑तेभि॒रोजो॑भिः । ओजो॑भि॒र् ये । ओजो॑भि॒रित्योजः॑ - भिः॒ । ये प्र॑ति॒ष्ठे । ये इति॒ ये । प्र॒ति॒ष्ठे अभ॑वताम् । प्र॒ति॒ष्ठे इति॑ प्रति - स्थे । अभ॑वता॒म् ॅवसू॑नाम् । वसू॑ना॒मिति॒ वसू॑नाम् ॥ स्तौमि॒ द्यावा॑पृथि॒वी । द्यावा॑पृथि॒वी ना॑थि॒तः । द्यावा॑पृथि॒वी इति॒ द्यावा᳚ - पृ॒थि॒वी । ना॒थि॒तो जो॑हविमी । जो॒ह॒वी॒मि॒ ते । ते नः॑ । ते इति॒ ते । नो॒ मु॒ञ्च॒त॒॒म् । मु॒ञ्च॒त॒मꣳह॑सः । अꣳह॑स॒ इत्यꣳह॑सः ॥ उर्वी॑ रोदसी । उर्वी॒ इत्युर्वी᳚ । रो॒द॒सी॒ वरि॑वः । रो॒द॒सी॒ इति॑ रोदसी । वरि॑वः कृणोतम् । कृ॒णो॒त॒म् क्षेत्र॑स्य । क्षेत्र॑स्य पत्नी । प॒त्नी॒ अधि॑ । प॒त्नी॒ इति॑ पत्नी । अधि॑ नः । नो॒ ब्रू॒या॒त॒॒म् । ब्रू॒या॒त॒मिति॑ ब्रूयातम् ॥ स्तौमि॒ द्यावा॑पृथि॒वी । द्यावा॑पृथि॒वी ना॑थि॒तः । द्यावा॑पृथि॒वी इति॒ द्यावा᳚ - पृ॒थि॒वी । ना॒थि॒तो जो॑हवीमि । जो॒ह॒वी॒मि॒ ते । ते नः॑ । ते इति॒ ते । नो॒ मु॒ञ्च॒त॒म् । मु॒ञ्च॒त॒मꣳह॑सः । अꣳह॑स॒ इत्यꣳह॑सः ॥ यत् ते᳚ । ते॒ व॒यम् । व॒यम् पु॑रुष॒त्रा । पु॒रु॒ष॒त्रा य॑विष्ठ । पु॒रु॒ष॒त्रेति॑ पुरुष - त्रा । य॒वि॒ष्ठावि॑द्वाꣳसः । अवि॑द्वाꣳसश्चकृ॒म । च॒कृ॒मा कत् । कच् च॒न ( ) । च॒नाऽऽगः॑ \newline

\textbf{Jatai Paata} \newline

1. इद॑नुमते ऽनुमत॒ इदि द॑नुमते । \newline
2. अ॒नु॒म॒ते॒ त्वम् त्व म॑नुमते ऽनुमते॒ त्वम् । \newline
3. अ॒नु॒म॒त॒ इत्य॑नु - म॒ते॒ । \newline
4. त्वं ॅवै᳚श्वान॒रो वै᳚श्वान॒र स्त्वम् त्वं ॅवै᳚श्वान॒रः । \newline
5. वै॒श्वा॒न॒रो नो॑ नो वैश्वान॒रो वै᳚श्वान॒रो नः॑ । \newline
6. न॒ ऊ॒त्योत्या नो॑ न ऊ॒त्या । \newline
7. ऊ॒त्या पृ॒ष्टः पृ॒ष्ट ऊ॒त्योत्या पृ॒ष्टः । \newline
8. पृ॒ष्टो दि॒वि दि॒वि पृ॒ष्टः पृ॒ष्टो दि॒वि । \newline
9. दि॒वीति॑ दि॒वि । \newline
10. ये अप्र॑थेता॒ मप्र॑थेतां॒ ॅये ये अप्र॑थेताम् । \newline
11. ये इति॒ ये । \newline
12. अप्र॑थेता॒ ममि॑तेभि॒ रमि॑तेभि॒ रप्र॑थेता॒ मप्र॑थेता॒ ममि॑तेभिः । \newline
13. अमि॑तेभि॒ रोजो॑भि॒ रोजो॑भि॒ रमि॑तेभि॒ रमि॑तेभि॒ रोजो॑भिः । \newline
14. ओजो॑भि॒र् ये ये ओजो॑भि॒ रोजो॑भि॒र् ये । \newline
15. ओजो॑भि॒रित्योजः॑ - भिः॒ । \newline
16. ये प्र॑ति॒ष्ठे प्र॑ति॒ष्ठे ये ये प्र॑ति॒ष्ठे । \newline
17. ये इति॒ ये । \newline
18. प्र॒ति॒ष्ठे अभ॑वता॒ मभ॑वताम् प्रति॒ष्ठे प्र॑ति॒ष्ठे अभ॑वताम् । \newline
19. प्र॒ति॒ष्ठे इति॑ प्रति - स्थे । \newline
20. अभ॑वतां॒ ॅवसू॑नां॒ ॅवसू॑ना॒ मभ॑वता॒ मभ॑वतां॒ ॅवसू॑नाम् । \newline
21. वसू॑ना॒मिति॒ वसू॑नाम् । \newline
22. स्तौमि॒ द्यावा॑पृथि॒वी द्यावा॑पृथि॒वी स्तौमि॒ स्तौमि॒ द्यावा॑पृथि॒वी । \newline
23. द्यावा॑पृथि॒वी ना॑थि॒तो ना॑थि॒तो द्यावा॑पृथि॒वी द्यावा॑पृथि॒वी ना॑थि॒तः । \newline
24. द्यावा॑पृथि॒वी इति॒ द्यावा᳚ - पृ॒थि॒वी । \newline
25. ना॒थि॒तो जो॑हवीमि जोहवीमि नाथि॒तो ना॑थि॒तो जो॑हवीमि । \newline
26. जो॒ह॒वी॒मि॒ ते ते जो॑हवीमि जोहवीमि॒ ते । \newline
27. ते नो॑ न॒ स्ते ते नः॑ । \newline
28. ते इति॒ ते । \newline
29. नो॒ मु॒ञ्च॒त॒म् मु॒ञ्च॒त॒म् नो॒ नो॒ मु॒ञ्च॒त॒म् । \newline
30. मु॒ञ्च॒त॒ मꣳह॑सो॒ अꣳह॑सो मुञ्चतम् मुञ्चत॒ मꣳह॑सः । \newline
31. अꣳह॑स॒ इत्यꣳह॑सः । \newline
32. उर्वी॑ रोदसी रोदसी॒ उर्वी॒ उर्वी॑ रोदसी । \newline
33. उर्वी॒ इत्युर्वी᳚ । \newline
34. रो॒द॒सी॒ वरि॑वो॒ वरि॑वो रोदसी रोदसी॒ वरि॑वः । \newline
35. रो॒द॒सी॒ इति॑ रोदसी । \newline
36. वरि॑वः कृणोतम् कृणोतं॒ ॅवरि॑वो॒ वरि॑वः कृणोतम् । \newline
37. कृ॒णो॒त॒म् क्षेत्र॑स्य॒ क्षेत्र॑स्य कृणोतम् कृणोत॒म् क्षेत्र॑स्य । \newline
38. क्षेत्र॑स्य पत्नी पत्नी॒ क्षेत्र॑स्य॒ क्षेत्र॑स्य पत्नी । \newline
39. प॒त्नी॒ अध्यधि॑ पत्नी पत्नी॒ अधि॑ । \newline
40. प॒त्नी॒ इति॑ पत्नी । \newline
41. अधि॑ नो नो॒ अध्यधि॑ नः । \newline
42. नो॒ ब्रू॒या॒त॒म् ब्रू॒या॒त॒म् नो॒ नो॒ ब्रू॒या॒त॒म् । \newline
43. ब्रू॒या॒त॒मिति॑ ब्रूयातम् । \newline
44. स्तौमि॒ द्यावा॑पृथि॒वी द्यावा॑पृथि॒वी स्तौमि॒ स्तौमि॒ द्यावा॑पृथि॒वी । \newline
45. द्यावा॑पृथि॒वी ना॑थि॒तो ना॑थि॒तो द्यावा॑पृथि॒वी द्यावा॑पृथि॒वी ना॑थि॒तः । \newline
46. द्यावा॑पृथि॒वी इति॒ द्यावा᳚ - पृ॒थि॒वी । \newline
47. ना॒थि॒तो जो॑हवीमि जोहवीमि नाथि॒तो ना॑थि॒तो जो॑हवीमि । \newline
48. जो॒ह॒वी॒मि॒ ते ते जो॑हवीमि जोहवीमि॒ ते । \newline
49. ते नो॑ न॒ स्ते ते नः॑ । \newline
50. ते इति॒ ते । \newline
51. नो॒ मु॒ञ्च॒त॒म् मु॒ञ्च॒त॒म् नो॒ नो॒ मु॒ञ्च॒त॒म् । \newline
52. मु॒ञ्च॒त॒ मꣳह॑सो॒ अꣳह॑सो मुञ्चतम् मुञ्चत॒ मꣳह॑सः । \newline
53. अꣳह॑स॒ इत्यꣳह॑सः । \newline
54. यत् ते॑ ते॒ यद् यत् ते᳚ । \newline
55. ते॒ व॒यं ॅव॒यम् ते॑ ते व॒यम् । \newline
56. व॒यम् पु॑रुष॒त्रा पु॑रुष॒त्रा व॒यं ॅव॒यम् पु॑रुष॒त्रा । \newline
57. पु॒रु॒ष॒त्रा य॑विष्ठ यविष्ठ पुरुष॒त्रा पु॑रुष॒त्रा य॑विष्ठ । \newline
58. पु॒रु॒ष॒त्रेति॑ पुरुष - त्रा । \newline
59. य॒वि॒ष्ठावि॑द्वा॒(ग्म्॒)सो ऽवि॑द्वाꣳसो यविष्ठ यवि॒ष्ठावि॑द्वाꣳसः । \newline
60. अवि॑द्वाꣳस श्चकृ॒म च॑कृ॒मावि॑द्वा॒(ग्म्॒)सो ऽवि॑द्वाꣳस श्चकृ॒म । \newline
61. च॒कृ॒मा कत् कच् च॑कृ॒म च॑कृ॒मा कत् । \newline
62. कच् च॒न च॒न कत् कच् च॒न । \newline
63. च॒नाग॒ आग॑ श्च॒न च॒नागः॑ । \newline

\textbf{Ghana Paata } \newline

1. इद॑नुमते ऽनुमत॒ इदि द॑नुमते॒ त्वम् त्व म॑नुमत॒ इदि द॑नुमते॒ त्वम् । \newline
2. अ॒नु॒म॒ते॒ त्वम् त्व म॑नुमते ऽनुमते॒ त्वं ॅवै᳚श्वान॒रो वै᳚श्वान॒ रस्त्व म॑नुमते ऽनुमते॒ त्वं ॅवै᳚श्वान॒रः । \newline
3. अ॒नु॒म॒त॒ इत्य॑नु - म॒ते॒ । \newline
4. त्वं ॅवै᳚श्वान॒रो वै᳚श्वान॒र स्त्वम् त्वं ॅवै᳚श्वान॒रो नो॑ नो वैश्वान॒र स्त्वम् त्वं ॅवै᳚श्वान॒रो नः॑ । \newline
5. वै॒श्वा॒न॒रो नो॑ नो वैश्वान॒रो वै᳚श्वान॒रो न॑ ऊ॒त्योत्या नो॑ वैश्वान॒रो वै᳚श्वान॒रो न॑ ऊ॒त्या । \newline
6. न॒ ऊ॒त्योत्या नो॑ न ऊ॒त्या पृ॒ष्टः पृ॒ष्ट ऊ॒त्या नो॑ न ऊ॒त्या पृ॒ष्टः । \newline
7. ऊ॒त्या पृ॒ष्टः पृ॒ष्ट ऊ॒त्योत्या पृ॒ष्टो दि॒वि दि॒वि पृ॒ष्ट ऊ॒त्योत्या पृ॒ष्टो दि॒वि । \newline
8. पृ॒ष्टो दि॒वि दि॒वि पृ॒ष्टः पृ॒ष्टो दि॒वि । \newline
9. दि॒वीति॑ दि॒वि । \newline
10. ये अप्र॑थेता॒ मप्र॑थेतां॒ ॅये ये अप्र॑थेता॒ ममि॑तेभि॒ रमि॑तेभि॒ रप्र॑थेतां॒ ॅये ये अप्र॑थेता॒ ममि॑तेभिः । \newline
11. ये इति॒ ये । \newline
12. अप्र॑थेता॒ ममि॑तेभि॒ रमि॑तेभि॒ रप्र॑थेता॒ मप्र॑थेता॒ ममि॑तेभि॒ रोजो॑भि॒ रोजो॑भि॒ रमि॑तेभि॒ रप्र॑थेता॒ मप्र॑थेता॒ ममि॑तेभि॒ रोजो॑भिः । \newline
13. अमि॑तेभि॒ रोजो॑भि॒ रोजो॑भि॒ रमि॑तेभि॒ रमि॑तेभि॒ रोजो॑भि॒र् ये ये ओजो॑भि॒ रमि॑तेभि॒ रमि॑तेभि॒ रोजो॑भि॒र् ये । \newline
14. ओजो॑भि॒र् ये ये ओजो॑भि॒ रोजो॑भि॒र् ये प्र॑ति॒ष्ठे प्र॑ति॒ष्ठे ये ओजो॑भि॒ रोजो॑भि॒र् ये प्र॑ति॒ष्ठे । \newline
15. ओजो॑भि॒रित्योजः॑ - भिः॒ । \newline
16. ये प्र॑ति॒ष्ठे प्र॑ति॒ष्ठे ये ये प्र॑ति॒ष्ठे अभ॑वता॒ मभ॑वताम् प्रति॒ष्ठे ये ये प्र॑ति॒ष्ठे अभ॑वताम् । \newline
17. ये इति॒ ये । \newline
18. प्र॒ति॒ष्ठे अभ॑वता॒ मभ॑वताम् प्रति॒ष्ठे प्र॑ति॒ष्ठे अभ॑वतां॒ ॅवसू॑नां॒ ॅवसू॑ना॒ मभ॑वताम् प्रति॒ष्ठे प्र॑ति॒ष्ठे अभ॑वतां॒ ॅवसू॑नाम् । \newline
19. प्र॒ति॒ष्ठे इति॑ प्रति - स्थे । \newline
20. अभ॑वतां॒ ॅवसू॑नां॒ ॅवसू॑ना॒ मभ॑वता॒ मभ॑वतां॒ ॅवसू॑नाम् । \newline
21. वसू॑ना॒मिति॒ वसू॑नाम् । \newline
22. स्तौमि॒ द्यावा॑पृथि॒वी द्यावा॑पृथि॒वी स्तौमि॒ स्तौमि॒ द्यावा॑पृथि॒वी ना॑थि॒तो ना॑थि॒तो द्यावा॑पृथि॒वी स्तौमि॒ स्तौमि॒ द्यावा॑पृथि॒वी ना॑थि॒तः । \newline
23. द्यावा॑पृथि॒वी ना॑थि॒तो ना॑थि॒तो द्यावा॑पृथि॒वी द्यावा॑पृथि॒वी ना॑थि॒तो जो॑हवीमि जोहवीमि नाथि॒तो द्यावा॑पृथि॒वी द्यावा॑पृथि॒वी ना॑थि॒तो जो॑हवीमि । \newline
24. द्यावा॑पृथि॒वी इति॒ द्यावा᳚ - पृ॒थि॒वी । \newline
25. ना॒थि॒तो जो॑हवीमि जोहवीमि नाथि॒तो ना॑थि॒तो जो॑हवीमि॒ ते ते जो॑हवीमि नाथि॒तो ना॑थि॒तो जो॑हवीमि॒ ते । \newline
26. जो॒ह॒वी॒मि॒ ते ते जो॑हवीमि जोहवीमि॒ ते नो॑ न॒ स्ते जो॑हवीमि जोहवीमि॒ ते नः॑ । \newline
27. ते नो॑ न॒ स्ते ते नो॑ मुञ्चतम् मुञ्चतम् न॒ स्ते ते नो॑ मुञ्चतम् । \newline
28. ते इति॒ ते । \newline
29. नो॒ मु॒ञ्च॒त॒म् मु॒ञ्च॒त॒म् नो॒ नो॒ मु॒ञ्च॒त॒ मꣳह॑सो॒ अꣳह॑सो मुञ्चतम् नो नो मुञ्चत॒ 
मꣳह॑सः । \newline
30. मु॒ञ्च॒त॒ मꣳह॑सो॒ अꣳह॑सो मुञ्चतम् मुञ्चत॒ मꣳह॑सः । \newline
31. अꣳह॑स॒ इत्यꣳह॑सः । \newline
32. उर्वी॑ रोदसी रोदसी॒ उर्वी॒ उर्वी॑ रोदसी॒ वरि॑वो॒ वरि॑वो रोदसी॒ उर्वी॒ उर्वी॑ रोदसी॒ वरि॑वः । \newline
33. उर्वी॒ इत्युर्वी᳚ । \newline
34. रो॒द॒सी॒ वरि॑वो॒ वरि॑वो रोदसी रोदसी॒ वरि॑वः कृणोतम् कृणोतं॒ ॅवरि॑वो रोदसी रोदसी॒ वरि॑वः कृणोतम् । \newline
35. रो॒द॒सी॒ इति॑ रोदसी । \newline
36. वरि॑वः कृणोतम् कृणोतं॒ ॅवरि॑वो॒ वरि॑वः कृणोत॒म् क्षेत्र॑स्य॒ क्षेत्र॑स्य कृणोतं॒ ॅवरि॑वो॒ वरि॑वः कृणोत॒म् क्षेत्र॑स्य । \newline
37. कृ॒णो॒त॒म् क्षेत्र॑स्य॒ क्षेत्र॑स्य कृणोतम् कृणोत॒म् क्षेत्र॑स्य पत्नी पत्नी॒ क्षेत्र॑स्य कृणोतम् कृणोत॒म् क्षेत्र॑स्य पत्नी । \newline
38. क्षेत्र॑स्य पत्नी पत्नी॒ क्षेत्र॑स्य॒ क्षेत्र॑स्य पत्नी॒ अध्यधि॑ पत्नी॒ क्षेत्र॑स्य॒ क्षेत्र॑स्य पत्नी॒ अधि॑ । \newline
39. प॒त्नी॒ अध्यधि॑ पत्नी पत्नी॒ अधि॑ नो नो॒ अधि॑ पत्नी पत्नी॒ अधि॑ नः । \newline
40. प॒त्नी॒ इति॑ पत्नी । \newline
41. अधि॑ नो नो॒ अध्यधि॑ नो ब्रूयातम् ब्रूयातम् नो॒ अध्यधि॑ नो ब्रूयातम् । \newline
42. नो॒ ब्रू॒या॒त॒म् ब्रू॒या॒त॒म् नो॒ नो॒ ब्रू॒या॒त॒म् । \newline
43. ब्रू॒या॒त॒मिति॑ ब्रूयातम् । \newline
44. स्तौमि॒ द्यावा॑पृथि॒वी द्यावा॑पृथि॒वी स्तौमि॒ स्तौमि॒ द्यावा॑पृथि॒वी ना॑थि॒तो ना॑थि॒तो द्यावा॑पृथि॒वी स्तौमि॒ स्तौमि॒ द्यावा॑पृथि॒वी ना॑थि॒तः । \newline
45. द्यावा॑पृथि॒वी ना॑थि॒तो ना॑थि॒तो द्यावा॑पृथि॒वी द्यावा॑पृथि॒वी ना॑थि॒तो जो॑हवीमि जोहवीमि नाथि॒तो द्यावा॑पृथि॒वी द्यावा॑पृथि॒वी ना॑थि॒तो जो॑हवीमि । \newline
46. द्यावा॑पृथि॒वी इति॒ द्यावा᳚ - पृ॒थि॒वी । \newline
47. ना॒थि॒तो जो॑हवीमि जोहवीमि नाथि॒तो ना॑थि॒तो जो॑हवीमि॒ ते ते जो॑हवीमि नाथि॒तो ना॑थि॒तो जो॑हवीमि॒ ते । \newline
48. जो॒ह॒वी॒मि॒ ते ते जो॑हवीमि जोहवीमि॒ ते नो॑ न॒ स्ते जो॑हवीमि जोहवीमि॒ ते नः॑ । \newline
49. ते नो॑ न॒ स्ते ते नो॑ मुञ्चतम् मुञ्चतम् न॒ स्ते ते नो॑ मुञ्चतम् । \newline
50. ते इति॒ ते । \newline
51. नो॒ मु॒ञ्च॒त॒म् मु॒ञ्च॒त॒म् नो॒ नो॒ मु॒ञ्च॒त॒ मꣳह॑सो॒ अꣳह॑सो मुञ्चतम् नो नो मुञ्चत॒ मꣳह॑सः । \newline
52. मु॒ञ्च॒त॒ मꣳह॑सो॒ अꣳह॑सो मुञ्चतम् मुञ्चत॒ मꣳह॑सः । \newline
53. अꣳह॑स॒ इत्यꣳह॑सः । \newline
54. यत् ते॑ ते॒ यद् यत् ते॑ व॒यं ॅव॒यम् ते॒ यद् यत् ते॑ व॒यम् । \newline
55. ते॒ व॒यं ॅव॒यम् ते॑ ते व॒यम् पु॑रुष॒त्रा पु॑रुष॒त्रा व॒यम् ते॑ ते व॒यम् पु॑रुष॒त्रा । \newline
56. व॒यम् पु॑रुष॒त्रा पु॑रुष॒त्रा व॒यं ॅव॒यम् पु॑रुष॒त्रा य॑विष्ठ यविष्ठ पुरुष॒त्रा व॒यं ॅव॒यम् पु॑रुष॒त्रा य॑विष्ठ । \newline
57. पु॒रु॒ष॒त्रा य॑विष्ठ यविष्ठ पुरुष॒त्रा पु॑रुष॒त्रा य॑वि॒ष्ठा वि॑द्वा॒(ग्म्॒)सो ऽवि॑द्वाꣳसो यविष्ठ पुरुष॒त्रा पु॑रुष॒त्रा य॑वि॒ष्ठा वि॑द्वाꣳसः । \newline
58. पु॒रु॒ष॒त्रेति॑ पुरुष - त्रा । \newline
59. य॒वि॒ष्ठा वि॑द्वा॒(ग्म्॒)सो ऽवि॑द्वाꣳसो यविष्ठ यवि॒ष्ठा वि॑द्वाꣳस श्चकृ॒म च॑कृ॒मा वि॑द्वाꣳसो यविष्ठ यवि॒ष्ठा वि॑द्वाꣳस श्चकृ॒म । \newline
60. अवि॑द्वाꣳस श्चकृ॒म च॑कृ॒मा वि॑द्वा॒(ग्म्॒)सो ऽवि॑द्वाꣳस श्चकृ॒मा कत् कच् च॑कृ॒मा वि॑द्वा॒(ग्म्॒)सो ऽवि॑द्वाꣳस श्चकृ॒मा कत् । \newline
61. च॒कृ॒मा कत् कच् च॑कृ॒म च॑कृ॒मा कच् च॒न च॒न कच् च॑कृ॒म च॑कृ॒मा कच् च॒न । \newline
62. कच् च॒न च॒न कत् कच् च॒नाग॒ आग॑ श्च॒न कत् कच् च॒नागः॑ । \newline
63. च॒नाग॒ आग॑ श्च॒न च॒नागः॑ । \newline
\pagebreak
\markright{ TS 4.7.15.7  \hfill https://www.vedavms.in \hfill}

\section{ TS 4.7.15.7 }

\textbf{TS 4.7.15.7 } \newline
\textbf{Samhita Paata} \newline

ऽऽ*गः॑ । कृ॒धी स्व॑स्माꣳ अदि॑ते॒रना॑गा॒ व्येनाꣳ॑सि शिश्रथो॒ विष्व॑गग्ने ॥यथा॑ ह॒ तद् व॑सवो गौ॒र्यं॑ चित् प॒दिषि॒ता ममु॑ञ्चता यजत्राः ।ए॒वा त्वम॒स्मत् प्रमु॑ञ्चा॒ व्यꣳहः॒ प्राता᳚र्यग्ने प्रत॒रान्न॒ आयुः॑ ॥ \newline

\textbf{Pada Paata} \newline

आगः॑ ॥ कृ॒धि । स्विति॑ । अ॒स्मान् । अदि॑तेः । अना॑गाः । वीति॑ । एनाꣳ॑सि । शि॒श्र॒थः॒ । विष्व॑क् । अ॒ग्ने॒ ॥ यथा᳚ । ह॒ । तत् । व॒स॒वः॒ । गौ॒र्य᳚म् । चि॒त् । प॒दि । सि॒ताम् । अमु॑ञ्चत । य॒ज॒त्राः॒ ॥ ए॒वा । त्वम् । अ॒स्मत् । प्रेति॑ । मु॒ञ्च॒ । वीति॑ । अꣳहः॑ । प्रेति॑ । अ॒ता॒रि॒ । अ॒ग्ने॒ । प्र॒त॒रामिति॑ प्र - त॒राम् । नः॒ । आयुः॑ ॥(अ॒ग्नेर्म॑न्वे॒ - यस्ये॒द- मिन्द्र॑स्य॒ - यः सं॑ ग्रा॒ममिन्द्रꣳ॒॒ - स नो॑ मुञ्च॒त्वꣳ ह॑सः । म॒न्वे वा॒न्ता नो॑ मुञ्चत॒माग॑सः । यो वां᳚ - वा॒यो- रुप॑ - र॒थीत॑मौ॒ - यदया॑त-म॒श्विनौ॒ - तौ नो॑ मुञ्चत॒माग॑सः । म॒रुता᳚न्- ति॒ग्मं - म॒रुतो॑ - दे॒वानां॒ - ॅयदि॒दं ॅविश्वा॒न् - ते नो॑ मुञ्च॒न्त्वेन॑सः । अनु॑ न॒ - उर्वी॒ - द्यावा॑पृथि॒वी - ते नो॑ मुञ्चत॒मꣳह॑सो॒ यत्तै᳚ । च॒तुरꣳ ह॑सः॒ षाडाग॑सश्च॒तुरेन॑सो॒ द्विरꣳह॑सः ।  \newline


\textbf{Krama Paata} \newline

आग॒ इत्यागः॑ ॥ कृ॒धी सु । स्व॑स्मान् । अ॒स्माꣳ अदि॑तेः । अदि॑ते॒रना॑गाः । अना॑गा॒ वि । व्येनाꣳ॑सि । एनाꣳ॑सि शिश्रथः । शि॒श्र॒थो॒ विष्व॑क् । विष्व॑गग्ने । अ॒ग्न॒ इत्य॑ग्ने ॥ यथा॑ ह । ह॒ तत् । तद् व॑सवः । व॒स॒वो॒ गौ॒र्य᳚म् । गौ॒र्य॑म् चित् । चि॒त् प॒दि । प॒दिषि॒ताम् । सि॒ताममु॑ञ्चत । अमु॑ञ्चता यजत्राः । य॒ज॒त्रा॒ इति॑ यजत्राः ॥ ए॒वा त्वम् । त्वम॒स्मत् । अ॒स्मत् प्र । प्र मु॑ञ्च । मु॒ञ्चा॒ वि । व्यꣳहः॑ । अꣳहः॒ प्र । प्राता॑रि । अ॒ता॒र्य॒ग्ने॒ । अ॒ग्ने॒ प्र॒त॒राम् । प्र॒त॒राम् नः॑ । प्र॒त॒रामिति॑ प्र - त॒राम् । न॒ आयुः॑ । आयु॒रित्यायुः॑ । \newline

\textbf{Jatai Paata} \newline

1. आग॒ इत्यागः॑ । \newline
2. कृ॒धी सु सु कृ॒धि कृ॒धी सु । \newline
3. स्व॑स्माꣳ अ॒स्मान् थ्सु स्व॑स्मान् । \newline
4. अ॒स्माꣳ अदि॑ते॒ रदि॑ते र॒स्माꣳ अ॒स्माꣳ अदि॑तेः । \newline
5. अदि॑ते॒ रना॑गा॒ अना॑गा॒ अदि॑ते॒ रदि॑ते॒ रना॑गाः । \newline
6. अना॑गा॒ वि व्यना॑गा॒ अना॑गा॒ वि । \newline
7. व्येना॒(ग्ग्॒) स्येना(ग्म्॑)सि॒ वि व्येना(ग्म्॑)सि । \newline
8. एना(ग्म्॑)सि शिश्रथः शिश्रथ॒ एना॒(ग्ग्॒) स्येना(ग्म्॑)सि शिश्रथः । \newline
9. शि॒श्र॒थो॒ विष्व॒ग् विष्व॑क् छिश्रथः शिश्रथो॒ विष्व॑क् । \newline
10. विष्व॑गग्ने अग्ने॒ विष्व॒ग् विष्व॑गग्ने । \newline
11. अ॒ग्न॒ इत्य॑ग्ने । \newline
12. यथा॑ ह ह॒ यथा॒ यथा॑ ह । \newline
13. ह॒ तत् त द्ध॑ ह॒ तत् । \newline
14. तद् व॑सवो वसव॒ स्तत् तद् व॑सवः । \newline
15. व॒स॒वो॒ गौ॒र्य॑म् गौ॒र्यं॑ ॅवसवो वसवो गौ॒र्य᳚म् । \newline
16. गौ॒र्य॑म् चिच् चिद् गौ॒र्य॑म् गौ॒र्य॑म् चित् । \newline
17. चि॒त् प॒दि प॒दि चि॑च् चित् प॒दि । \newline
18. प॒दि षि॒ताꣳ सि॒ताम् प॒दि प॒दि षि॒ताम् । \newline
19. सि॒ता ममु॑ञ्च॒ता मु॑ञ्चत सि॒ताꣳ सि॒ता ममु॑ञ्चत । \newline
20. अमु॑ञ्चता यजत्रा यजत्रा॒ अमु॑ञ्च॒ता मु॑ञ्चता यजत्राः । \newline
21. य॒ज॒त्रा॒ इति॑ यजत्राः । \newline
22. ए॒वा त्वम् त्व मे॒वैवा त्वम् । \newline
23. त्व म॒स्म द॒स्मत् त्वम् त्व म॒स्मत् । \newline
24. अ॒स्मत् प्र प्रास्म द॒स्मत् प्र । \newline
25. प्र मु॑ञ्च मुञ्च॒ प्र प्र मु॑ञ्च । \newline
26. मु॒ञ्चा॒ वि वि मु॑ञ्च मुञ्चा॒ वि । \newline
27. व्यꣳहो ऽꣳहो॒ वि व्यꣳहः॑ । \newline
28. अꣳहः॒ प्र प्राꣳहो ऽꣳहः॒ प्र । \newline
29. प्राता᳚र्यतारि॒ प्र प्राता॑रि । \newline
30. अ॒ता॒ र्य॒ग्ने॒ अ॒ग्ने॒ अ॒ता॒ र्य॒ता॒ र्य॒ग्ने॒ । \newline
31. अ॒ग्ने॒ प्र॒त॒राम् प्र॑त॒रा म॑ग्ने अग्ने प्रत॒राम् । \newline
32. प्र॒त॒राम् नो॑ नः प्रत॒राम् प्र॑त॒राम् नः॑ । \newline
33. प्र॒त॒रामिति॑ प्र - त॒राम् । \newline
34. न॒ आयु॒ रायु॑र् नो न॒ आयुः॑ । \newline
35. आयु॒रित्यायुः॑ । \newline

\textbf{Ghana Paata } \newline

1. आग॒ इत्यागः॑ । \newline
2. कृ॒धी सु सु कृ॒धि कृ॒धी स्व॑स्माꣳ अ॒स्मान् थ्सु कृ॒धि कृ॒धी स्व॑स्मान् । \newline
3. स्व॑स्माꣳ अ॒स्मान् थ्सु स्व॑स्माꣳ अदि॑ते॒ रदि॑ते र॒स्मान् थ्सु स्व॑स्माꣳ अदि॑तेः । \newline
4. अ॒स्माꣳ अदि॑ते॒ रदि॑ते र॒स्माꣳ अ॒स्माꣳ अदि॑ते॒ रना॑गा॒ अना॑गा॒ अदि॑ते र॒स्माꣳ अ॒स्माꣳ अदि॑ते॒ रना॑गाः । \newline
5. अदि॑ते॒ रना॑गा॒ अना॑गा॒ अदि॑ते॒ रदि॑ते॒ रना॑गा॒ वि व्यना॑गा॒ अदि॑ते॒ रदि॑ते॒ रना॑गा॒ वि । \newline
6. अना॑गा॒ वि व्यना॑गा॒ अना॑गा॒ व्येना॒(ग्ग्॒) स्येना(ग्म्॑)सि॒ व्यना॑गा॒ अना॑गा॒ व्येना(ग्म्॑)सि । \newline
7. व्येना॒(ग्ग्॒) स्येना(ग्म्॑)सि॒ वि व्येना(ग्म्॑)सि शिश्रथः शिश्रथ॒ एना(ग्म्॑)सि॒ वि व्येना(ग्म्॑)सि शिश्रथः । \newline
8. एना(ग्म्॑)सि शिश्रथः शिश्रथ॒ एना॒(ग्ग्॒) स्येना(ग्म्॑)सि शिश्रथो॒ विष्व॒ग् विष्व॑क् छिश्रथ॒ एना॒(ग्ग्॒) स्येना(ग्म्॑)सि शिश्रथो॒ विष्व॑क् । \newline
9. शि॒श्र॒थो॒ विष्व॒ग् विष्व॑क् छिश्रथः शिश्रथो॒ विष्व॑गग्ने अग्ने॒ विष्व॑क् छिश्रथः शिश्रथो॒ विष्व॑गग्ने । \newline
10. विष्व॑गग्ने अग्ने॒ विष्व॒ग् विष्व॑गग्ने । \newline
11. अ॒ग्न॒ इत्य॑ग्ने । \newline
12. यथा॑ ह ह॒ यथा॒ यथा॑ ह॒ तत् तद्ध॒ यथा॒ यथा॑ ह॒ तत् । \newline
13. ह॒ तत् तद्ध॑ ह॒ तद् व॑सवो वसव॒ स्तद्ध॑ ह॒ तद् व॑सवः । \newline
14. तद् व॑सवो वसव॒ स्तत् तद् व॑सवो गौ॒र्य॑म् गौ॒र्यं॑ ॅवसव॒ स्तत् तद् व॑सवो गौ॒र्य᳚म् । \newline
15. व॒स॒वो॒ गौ॒र्य॑म् गौ॒र्यं॑ ॅवसवो वसवो गौ॒र्य॑म् चिच् चिद् गौ॒र्यं॑ ॅवसवो वसवो गौ॒र्य॑म् चित् । \newline
16. गौ॒र्य॑म् चिच् चिद् गौ॒र्य॑म् गौ॒र्य॑म् चित् प॒दि प॒दि चि॑द् गौ॒र्य॑म् गौ॒र्य॑म् चित् प॒दि । \newline
17. चि॒त् प॒दि प॒दि चि॑च् चित् प॒दि षि॒ताꣳ सि॒ताम् प॒दि चि॑च् चित् प॒दि षि॒ताम् । \newline
18. प॒दि षि॒ताꣳ सि॒ताम् प॒दि प॒दि षि॒ता ममु॑ञ्च॒ता मु॑ञ्चत सि॒ताम् प॒दि प॒दि षि॒ता ममु॑ञ्चत । \newline
19. सि॒ता ममु॑ञ्च॒ता मु॑ञ्चत सि॒ताꣳ सि॒ता ममु॑ञ्चता यजत्रा यजत्रा॒ अमु॑ञ्चत सि॒ताꣳ सि॒ता ममु॑ञ्चता यजत्राः । \newline
20. अमु॑ञ्चता यजत्रा यजत्रा॒ अमु॑ञ्च॒ता मु॑ञ्चता यजत्राः । \newline
21. य॒ज॒त्रा॒ इति॑ यजत्राः । \newline
22. ए॒वा त्वम् त्व मे॒वैवा त्व म॒स्म द॒स्मत् त्व मे॒वैवा त्व म॒स्मत् । \newline
23. त्व म॒स्म द॒स्मत् त्वम् त्व म॒स्मत् प्र प्रास्मत् त्वम् त्व म॒स्मत् प्र । \newline
24. अ॒स्मत् प्र प्रास्म द॒स्मत् प्र मु॑ञ्च मुञ्च॒ प्रास्म द॒स्मत् प्र मु॑ञ्च । \newline
25. प्र मु॑ञ्च मुञ्च॒ प्र प्र मु॑ञ्चा॒ वि वि मु॑ञ्च॒ प्र प्र मु॑ञ्चा॒ वि । \newline
26. मु॒ञ्चा॒ वि वि मु॑ञ्च मुञ्चा॒ व्यꣳहो ऽꣳहो॒ वि मु॑ञ्च मुञ्चा॒ व्यꣳहः॑ । \newline
27. व्यꣳहो ऽꣳहो॒ वि व्यꣳहः॒ प्र प्राꣳहो॒ वि व्यꣳहः॒ प्र । \newline
28. अꣳहः॒ प्र प्राꣳहो ऽꣳहः॒ प्राता᳚र् यतारि॒ प्राꣳहो ऽꣳहः॒ प्राता॑रि । \newline
29. प्राता᳚र् यतारि॒ प्र प्राता᳚र् यग्ने अग्ने अतारि॒ प्र प्राता᳚र् यग्ने । \newline
30. अ॒ता॒र् य॒ग्ने॒ अ॒ग्ने॒ अ॒ता॒र् य॒ता॒र् य॒ग्ने॒ प्र॒त॒राम् प्र॑त॒रा म॑ग्ने अतार् यतार् यग्ने प्रत॒राम् । \newline
31. अ॒ग्ने॒ प्र॒त॒राम् प्र॑त॒रा म॑ग्ने अग्ने प्रत॒राम् नो॑ नः प्रत॒रा म॑ग्ने अग्ने प्रत॒राम् नः॑ । \newline
32. प्र॒त॒राम् नो॑ नः प्रत॒राम् प्र॑त॒राम् न॒ आयु॒ रायु॑र् नः प्रत॒राम् प्र॑त॒राम् न॒ आयुः॑ । \newline
33. प्र॒त॒रामिति॑ प्र - त॒राम् । \newline
34. न॒ आयु॒ रायु॑र् नो न॒ आयुः॑ । \newline
35. आयु॒रित्यायुः॑ । \newline
\pagebreak


\end{document}