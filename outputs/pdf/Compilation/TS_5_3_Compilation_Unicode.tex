\documentclass[17pt]{extarticle}
\usepackage{babel}
\usepackage{fontspec}
\usepackage{polyglossia}
\usepackage{extsizes}

\usepackage{color}   %May be necessary if you want to color links
\usepackage{hyperref}
\hypersetup{
    colorlinks=true, %set true if you want colored links
    linktoc=all,     %set to all if you want both sections and subsections linked
    linkcolor=black,  %choose some color if you want links to stand out
}

\setmainlanguage{sanskrit}
\setotherlanguages{english} %% or other languages
\setlength{\parindent}{0pt}
\pagestyle{myheadings}
\newfontfamily\devanagarifont[Script=Devanagari]{AdishilaVedic}
\renewcommand{\theHsection}{\thepart.section.\thesection}

\newcommand{\VAR}[1]{}
\newcommand{\BLOCK}[1]{}




\begin{document}
\begin{titlepage}
    \begin{center}
 
\begin{sanskrit}
    { \Large
    कृष्ण यजुर्वेदीय तैत्तिरीय संहिता,पद,जटा,घन पाठः 
    }
    \\
    \vspace{2.5cm}
    \mbox{ \Large
    5.3      पञ्चमकाण्डे तृतीयः प्रश्नः - चितीनां निरूपणं   }
\end{sanskrit}
\end{center}

\end{titlepage}
\tableofcontents
\phantomsection
\pagebreak

\markright{ TS 5.3.1.1  \hfill https://www.vedavms.in \hfill}

\section{ TS 5.3.1.1 }

\textbf{TS 5.3.1.1 } \newline
\textbf{Samhita Paata} \newline

उ॒थ्स॒न्न॒ य॒ज्ञो वा ए॒ष यद॒ग्निः किं ॅवाऽहै॒तस्य॑ क्रि॒यते॒ किं ॅवा॒ न यद्वै य॒ज्ञ्स्य॑ क्रि॒यमा॑णस्या-न्त॒र्यन्ति॒ पूय॑ति॒ वा अ॑स्य॒ तदा᳚श्वि॒नीरुप॑ दधात्य॒श्विनौ॒ वै दे॒वानां᳚ भि॒षजौ॒ ताभ्या॑मे॒वास्मै॑ भेष॒जं क॑रोति॒ पञ्चोप॑ दधाति॒ पाङ्क्तो॑ य॒ज्ञो यावा॑ने॒व य॒ज्ञ्स्तस्मै॑ भेष॒जं क॑रोत्यृत॒व्या॑ उप॑ दधात्यृतू॒नां क्लृप्त्यै॒ - [  ] \newline

\textbf{Pada Paata} \newline

उ॒थ्स॒न्न॒य॒ज्ञ् इत्यु॑थ्सन्न - य॒ज्ञ्ः । वै । ए॒षः । यत् । अ॒ग्निः । किम् । वा॒ । अह॑ । ए॒तस्य॑ । क्रि॒यते᳚ । किम् । वा॒ । न । यत् । वै । य॒ज्ञ्स्य॑ । क्रि॒यमा॑णस्य । अ॒न्त॒र्यन्तीत्य॑न्तः - यन्ति॑ । पूय॑ति । वै । अ॒स्य॒ । तत् । आ॒श्वि॒नीः । उपेति॑ । द॒धा॒ति॒ । अ॒श्विनौ᳚ । वै । दे॒वाना᳚म् । भि॒षजौ᳚ । ताभ्या᳚म् । ए॒व । अ॒स्मै॒ । भे॒ष॒जम् । क॒रो॒ति॒ । पञ्च॑ । उपेति॑ । द॒धा॒ति॒ । पाङ्क्तः॑ । य॒ज्ञ्ः । यावान्॑ । ए॒व । य॒ज्ञ्ः । तस्मै᳚ । भे॒ष॒जम् । क॒रो॒ति॒ । ऋ॒त॒व्याः᳚ । उपेति॑ । द॒धा॒ति॒ । ऋ॒तू॒नाम् । क्लृप्त्यै᳚ ।  \newline


\textbf{Krama Paata} \newline

उ॒थ्स॒न्न॒य॒ज्ञो वै । उ॒थ्स॒न्न॒य॒ज्ञ् इत्यु॑थ्सन्न - य॒ज्ञ्ः । 
वा ए॒षः । ए॒ष यत् । यद॒ग्निः । अ॒ग्निः किम् । किम् ॅवा᳚ । वाऽह॑ । अहै॒तस्य॑ । ए॒तस्य॑ क्रि॒यते᳚ । क्रि॒यते॒ किम् । किम् ॅवा᳚ । वा॒ न । न यत् । यद् वै । वै य॒ज्ञ्स्य॑ । य॒ज्ञ्स्य॑ क्रि॒यमा॑णस्य । क्रि॒यमा॑णस्यान्त॒र्यन्ति॑ । अ॒न्त॒र्यन्ति॒ पूय॑ति । अ॒न्त॒र्यन्तीत्य॑न्तः - यन्ति॑ । पूय॑ति॒ वै । वा अ॑स्य । अ॒स्य॒ तत् । तदा᳚श्वि॒नीः । आ॒श्वि॒नीरुप॑ । उप॑ दधाति । द॒धा॒त्य॒श्विनौ᳚ । अ॒श्विनौ॒ वै । वै दे॒वाना᳚म् । दे॒वाना᳚म् भि॒षजौ᳚ । भि॒षजौ॒ ताभ्या᳚म् । ताभ्या॑मे॒व । ए॒वास्मै᳚ । अ॒स्मै॒ भे॒ष॒जम् । भे॒ष॒जम् क॑रोति । क॒रो॒ति॒ पञ्च॑ । पञ्चोप॑ । उप॑ दधाति । द॒धा॒ति॒ पाङ्क्तः॑ । पाङ्क्तो॑ य॒ज्ञ्ः । य॒ज्ञो यावान्॑ । यावा॑ने॒व । ए॒व य॒ज्ञ्ः । य॒ज्ञ्स्तस्मै᳚ । तस्मै॑ भेष॒जम् । भे॒ष॒जम् क॑रोति । क॒रो॒त्यृ॒त॒व्याः᳚ । ऋ॒त॒व्या॑ उप॑ । उप॑ दधाति । द॒धा॒त्यृ॒तू॒नाम् । ऋ॒तू॒नाम् क्लृप्त्यै᳚ । क्लृप्त्यै॒ पञ्च॑ \newline

\textbf{Jatai Paata} \newline

1. उ॒थ्स॒न्न॒य॒ज्ञो वै वा उ॑थ्सन्नय॒ज्ञ् उ॑थ्सन्नय॒ज्ञो वै । \newline
2. उ॒थ्स॒न्न॒य॒ज्ञ् इत्यु॑थ्सन्न - य॒ज्ञ्ः । \newline
3. वा ए॒ष ए॒ष वै वा ए॒षः । \newline
4. ए॒ष यद् यदे॒ष ए॒ष यत् । \newline
5. यद॒ग्नि र॒ग्निर् यद् यद॒ग्निः । \newline
6. अ॒ग्निः किम् कि म॒ग्नि र॒ग्निः किम् । \newline
7. किं ॅवा॑ वा॒ किम् किं ॅवा᳚ । \newline
8. वा ऽहाह॑ वा॒ वा ऽह॑ । \newline
9. अहै॒त स्यै॒त स्याहा है॒तस्य॑ । \newline
10. ए॒तस्य॑ क्रि॒यते᳚ क्रि॒यत॑ ए॒त स्यै॒तस्य॑ क्रि॒यते᳚ । \newline
11. क्रि॒यते॒ किम् किम् क्रि॒यते᳚ क्रि॒यते॒ किम् । \newline
12. किं ॅवा॑ वा॒ किम् किं ॅवा᳚ । \newline
13. वा॒ न न वा॑ वा॒ न । \newline
14. न यद् यन् न न यत् । \newline
15. यद् वै वै यद् यद् वै । \newline
16. वै य॒ज्ञ्स्य॑ य॒ज्ञ्स्य॒ वै वै य॒ज्ञ्स्य॑ । \newline
17. य॒ज्ञ्स्य॑ क्रि॒यमा॑णस्य क्रि॒यमा॑णस्य य॒ज्ञ्स्य॑ य॒ज्ञ्स्य॑ क्रि॒यमा॑णस्य । \newline
18. क्रि॒यमा॑णस्या न्त॒र्यन् त्य॑न्त॒र्यन्ति॑ क्रि॒यमा॑णस्य क्रि॒यमा॑णस्या न्त॒र्यन्ति॑ । \newline
19. अ॒न्त॒र्यन्ति॒ पूय॑ति॒ पूय॑ त्यन्त॒र्यन् त्य॑न्त॒र्यन्ति॒ पूय॑ति । \newline
20. अ॒न्त॒र्यन्तीत्य॑न्तः - यन्ति॑ । \newline
21. पूय॑ति॒ वै वै पूय॑ति॒ पूय॑ति॒ वै । \newline
22. वा अ॑स्यास्य॒ वै वा अ॑स्य । \newline
23. अ॒स्य॒ तत् तद॑ स्यास्य॒ तत् । \newline
24. तदा᳚श्वि॒नी रा᳚श्वि॒नी स्तत् तदा᳚श्वि॒नीः । \newline
25. आ॒श्वि॒नी रुपोपा᳚ श्वि॒नीरा᳚ श्वि॒नी रुप॑ । \newline
26. उप॑ दधाति दधा॒ त्युपोप॑ दधाति । \newline
27. द॒धा॒ त्य॒श्विना॑ व॒श्विनौ॑ दधाति दधा त्य॒श्विनौ᳚ । \newline
28. अ॒श्विनौ॒ वै वा अ॒श्विना॑ व॒श्विनौ॒ वै । \newline
29. वै दे॒वाना᳚म् दे॒वानां॒ ॅवै वै दे॒वाना᳚म् । \newline
30. दे॒वाना᳚म् भि॒षजौ॑ भि॒षजौ॑ दे॒वाना᳚म् दे॒वाना᳚म् भि॒षजौ᳚ । \newline
31. भि॒षजौ॒ ताभ्या॒म् ताभ्या᳚म् भि॒षजौ॑ भि॒षजौ॒ ताभ्या᳚म् । \newline
32. ताभ्या॑ मे॒वैव ताभ्या॒म् ताभ्या॑ मे॒व । \newline
33. ए॒वास्मा॑ अस्मा ए॒वै वास्मै᳚ । \newline
34. अ॒स्मै॒ भे॒ष॒जम् भे॑ष॒ज म॑स्मा अस्मै भेष॒जम् । \newline
35. भे॒ष॒जम् क॑रोति करोति भेष॒जम् भे॑ष॒जम् क॑रोति । \newline
36. क॒रो॒ति॒ पञ्च॒ पञ्च॑ करोति करोति॒ पञ्च॑ । \newline
37. पञ्चोपोप॒ पञ्च॒ पञ्चोप॑ । \newline
38. उप॑ दधाति दधा॒ त्युपोप॑ दधाति । \newline
39. द॒धा॒ति॒ पाङ्क्तः॒ पाङ्क्तो॑ दधाति दधाति॒ पाङ्क्तः॑ । \newline
40. पाङ्क्तो॑ य॒ज्ञो य॒ज्ञ्ः पाङ्क्तः॒ पाङ्क्तो॑ य॒ज्ञ्ः । \newline
41. य॒ज्ञो यावा॒न्॒. यावान्॑. य॒ज्ञो य॒ज्ञो यावान्॑ । \newline
42. यावा॑ ने॒वैव यावा॒न्॒. यावा॑ ने॒व । \newline
43. ए॒व य॒ज्ञो य॒ज्ञ् ए॒वैव य॒ज्ञ्ः । \newline
44. य॒ज्ञ् स्तस्मै॒ तस्मै॑ य॒ज्ञो य॒ज्ञ् स्तस्मै᳚ । \newline
45. तस्मै॑ भेष॒जम् भे॑ष॒जम् तस्मै॒ तस्मै॑ भेष॒जम् । \newline
46. भे॒ष॒जम् क॑रोति करोति भेष॒जम् भे॑ष॒जम् क॑रोति । \newline
47. क॒रो॒ त्यृ॒त॒व्या॑ ऋत॒व्याः᳚ करोति करो त्यृत॒व्याः᳚ । \newline
48. ऋ॒त॒व्या॑ उपोपा᳚ र्‌त॒व्या॑ ऋत॒व्या॑ उप॑ । \newline
49. उप॑ दधाति दधा॒ त्युपोप॑ दधाति । \newline
50. द॒धा॒ त्यृ॒तू॒ना मृ॑तू॒नाम् द॑धाति दधा त्यृतू॒नाम् । \newline
51. ऋ॒तू॒नाम् क्लृप्त्यै॒ क्लृप्त्या॑ ऋतू॒ना मृ॑तू॒नाम् क्लृप्त्यै᳚ । \newline
52. क्लृप्त्यै॒ पञ्च॒ पञ्च॒ क्लृप्त्यै॒ क्लृप्त्यै॒ पञ्च॑ । \newline

\textbf{Ghana Paata } \newline

1. उ॒थ्स॒न्न॒य॒ज्ञो वै वा उ॑थ्सन्नय॒ज्ञ् उ॑थ्सन्नय॒ज्ञो वा ए॒ष ए॒ष वा उ॑थ्सन्नय॒ज्ञ् उ॑थ्सन्नय॒ज्ञो वा ए॒षः । \newline
2. उ॒थ्स॒न्न॒य॒ज्ञ् इत्यु॑थ्सन्न - य॒ज्ञ्ः । \newline
3. वा ए॒ष ए॒ष वै वा ए॒ष यद् यदे॒ष वै वा ए॒ष यत् । \newline
4. ए॒ष यद् यदे॒ष ए॒ष यद॒ग्नि र॒ग्निर् यदे॒ष ए॒ष यद॒ग्निः । \newline
5. यद॒ग्नि र॒ग्निर् यद् यद॒ग्निः किम् कि म॒ग्निर् यद् यद॒ग्निः किम् । \newline
6. अ॒ग्निः किम् कि म॒ग्नि र॒ग्निः किं ॅवा॑ वा॒ कि म॒ग्नि र॒ग्निः किं ॅवा᳚ । \newline
7. किं ॅवा॑ वा॒ किम् किं ॅवा ऽहाह॑ वा॒ किम् किं ॅवा ऽह॑ । \newline
8. वा ऽहाह॑ वा॒ वा ऽहै॒त स्यै॒त स्याह॑ वा॒ वा ऽहै॒तस्य॑ । \newline
9. अहै॒त स्यै॒तस्या हा है॒तस्य॑ क्रि॒यते᳚ क्रि॒यत॑ ए॒तस्याहा है॒तस्य॑ क्रि॒यते᳚ । \newline
10. ए॒तस्य॑ क्रि॒यते᳚ क्रि॒यत॑ ए॒त स्यै॒तस्य॑ क्रि॒यते॒ किम् किम् क्रि॒यत॑ ए॒त स्यै॒तस्य॑ क्रि॒यते॒ किम् । \newline
11. क्रि॒यते॒ किम् किम् क्रि॒यते᳚ क्रि॒यते॒ किं ॅवा॑ वा॒ किम् क्रि॒यते᳚ क्रि॒यते॒ किं ॅवा᳚ । \newline
12. किं ॅवा॑ वा॒ किम् किं ॅवा॒ न न वा॒ किम् किं ॅवा॒ न । \newline
13. वा॒ न न वा॑ वा॒ न यद् यन् न वा॑ वा॒ न यत् । \newline
14. न यद् यन् न न यद् वै वै यन् न न यद् वै । \newline
15. यद् वै वै यद् यद् वै य॒ज्ञ्स्य॑ य॒ज्ञ्स्य॒ वै यद् यद् वै य॒ज्ञ्स्य॑ । \newline
16. वै य॒ज्ञ्स्य॑ य॒ज्ञ्स्य॒ वै वै य॒ज्ञ्स्य॑ क्रि॒यमा॑णस्य क्रि॒यमा॑णस्य य॒ज्ञ्स्य॒ वै वै य॒ज्ञ्स्य॑ क्रि॒यमा॑णस्य । \newline
17. य॒ज्ञ्स्य॑ क्रि॒यमा॑णस्य क्रि॒यमा॑णस्य य॒ज्ञ्स्य॑ य॒ज्ञ्स्य॑ क्रि॒यमा॑णस्या न्त॒र्यन्त्य॑ न्त॒र्यन्ति॑ क्रि॒यमा॑णस्य य॒ज्ञ्स्य॑ य॒ज्ञ्स्य॑ क्रि॒यमा॑णस्या न्त॒र्यन्ति॑ । \newline
18. क्रि॒यमा॑णस्या न्त॒र्यन् त्य॑न्त॒र्यन्ति॑ क्रि॒यमा॑णस्य क्रि॒यमा॑णस्या न्त॒र्यन्ति॒ पूय॑ति॒ पूय॑ त्यन्त॒र्यन्ति॑ क्रि॒यमा॑णस्य क्रि॒यमा॑णस्या न्त॒र्यन्ति॒ पूय॑ति । \newline
19. अ॒न्त॒र्यन्ति॒ पूय॑ति॒ पूय॑ त्यन्त॒र्यन्त्य॑ न्त॒र्यन्ति॒ पूय॑ति॒ वै वै पूय॑त्य न्त॒र्य न्त्य॑न्त॒र्यन्ति॒ पूय॑ति॒ वै । \newline
20. अ॒न्त॒र्यन्तीत्य॑न्तः - यन्ति॑ । \newline
21. पूय॑ति॒ वै वै पूय॑ति॒ पूय॑ति॒ वा अ॑स्यास्य॒ वै पूय॑ति॒ पूय॑ति॒ वा अ॑स्य । \newline
22. वा अ॑स्यास्य॒ वै वा अ॑स्य॒ तत् तद॑स्य॒ वै वा अ॑स्य॒ तत् । \newline
23. अ॒स्य॒ तत् तद॑ स्यास्य॒ तदा᳚ श्वि॒नी रा᳚श्वि॒नी स्तद॑ स्यास्य॒ तदा᳚श्वि॒नीः । \newline
24. तदा᳚श्वि॒नी रा᳚श्वि॒नी स्तत् तदा᳚श्वि॒नी रुपो पा᳚श्वि॒नी स्तत् तदा᳚ श्वि॒नीरुप॑ । \newline
25. आ॒श्वि॒नी रुपोपा᳚ श्वि॒नीरा᳚ श्वि॒नी रुप॑ दधाति दधा॒ त्युपा᳚ श्वि॒नी रा᳚श्वि॒नी रुप॑ दधाति । \newline
26. उप॑ दधाति दधा॒ त्युपोप॑ दधा त्य॒श्विना॑ व॒श्विनौ॑ दधा॒ त्युपोप॑ दधा त्य॒श्विनौ᳚ । \newline
27. द॒धा॒ त्य॒श्विना॑ व॒श्विनौ॑ दधाति दधात्य॒ श्विनौ॒ वै वा अ॒श्विनौ॑ दधाति दधा त्य॒श्विनौ॒ वै । \newline
28. अ॒श्विनौ॒ वै वा अ॒श्विना॑ व॒श्विनौ॒ वै दे॒वाना᳚म् दे॒वानां॒ ॅवा अ॒श्विना॑ व॒श्विनौ॒ वै दे॒वाना᳚म् । \newline
29. वै दे॒वाना᳚म् दे॒वानां॒ ॅवै वै दे॒वाना᳚म् भि॒षजौ॑ भि॒षजौ॑ दे॒वानां॒ ॅवै वै दे॒वाना᳚म् भि॒षजौ᳚ । \newline
30. दे॒वाना᳚म् भि॒षजौ॑ भि॒षजौ॑ दे॒वाना᳚म् दे॒वाना᳚म् भि॒षजौ॒ ताभ्या॒म् ताभ्या᳚म् भि॒षजौ॑ दे॒वाना᳚म् दे॒वाना᳚म् भि॒षजौ॒ ताभ्या᳚म् । \newline
31. भि॒षजौ॒ ताभ्या॒म् ताभ्या᳚म् भि॒षजौ॑ भि॒षजौ॒ ताभ्या॑ मे॒वैव ताभ्या᳚म् भि॒षजौ॑ भि॒षजौ॒ ताभ्या॑ मे॒व । \newline
32. ताभ्या॑ मे॒वैव ताभ्या॒म् ताभ्या॑ मे॒वास्मा॑ अस्मा ए॒व ताभ्या॒म् ताभ्या॑ मे॒वास्मै᳚ । \newline
33. ए॒वास्मा॑ अस्मा ए॒वै वास्मै॑ भेष॒जम् भे॑ष॒ज म॑स्मा ए॒वै वास्मै॑ भेष॒जम् । \newline
34. अ॒स्मै॒ भे॒ष॒जम् भे॑ष॒ज म॑स्मा अस्मै भेष॒जम् क॑रोति करोति भेष॒ज म॑स्मा अस्मै भेष॒जम् क॑रोति । \newline
35. भे॒ष॒जम् क॑रोति करोति भेष॒जम् भे॑ष॒जम् क॑रोति॒ पञ्च॒ पञ्च॑ करोति भेष॒जम् भे॑ष॒जम् क॑रोति॒ पञ्च॑ । \newline
36. क॒रो॒ति॒ पञ्च॒ पञ्च॑ करोति करोति॒ पञ्चोपोप॒ पञ्च॑ करोति करोति॒ पञ्चोप॑ । \newline
37. पञ्चो पोप॒ पञ्च॒ पञ्चोप॑ दधाति दधा॒ त्युप॒ पञ्च॒ पञ्चोप॑ दधाति । \newline
38. उप॑ दधाति दधा॒ त्युपोप॑ दधाति॒ पाङ्क्तः॒ पाङ्क्तो॑ दधा॒ त्युपोप॑ दधाति॒ पाङ्क्तः॑ । \newline
39. द॒धा॒ति॒ पाङ्क्तः॒ पाङ्क्तो॑ दधाति दधाति॒ पाङ्क्तो॑ य॒ज्ञो य॒ज्ञ्ः पाङ्क्तो॑ दधाति दधाति॒ पाङ्क्तो॑ य॒ज्ञ्ः । \newline
40. पाङ्क्तो॑ य॒ज्ञो य॒ज्ञ्ः पाङ्क्तः॒ पाङ्क्तो॑ य॒ज्ञो यावा॒न्॒. यावान्॑. य॒ज्ञ्ः पाङ्क्तः॒ पाङ्क्तो॑ य॒ज्ञो यावान्॑ । \newline
41. य॒ज्ञो यावा॒न्॒. यावान्॑. य॒ज्ञो य॒ज्ञो यावा॑ ने॒वैव यावान्॑. य॒ज्ञो य॒ज्ञो यावा॑ ने॒व । \newline
42. यावा॑ ने॒वैव यावा॒न्॒. यावा॑ ने॒व य॒ज्ञो य॒ज्ञ् ए॒व यावा॒न्॒. यावा॑ ने॒व य॒ज्ञ्ः । \newline
43. ए॒व य॒ज्ञो य॒ज्ञ् ए॒वैव य॒ज्ञ् स्तस्मै॒ तस्मै॑ य॒ज्ञ् ए॒वैव य॒ज्ञ् स्तस्मै᳚ । \newline
44. य॒ज्ञ् स्तस्मै॒ तस्मै॑ य॒ज्ञो य॒ज्ञ् स्तस्मै॑ भेष॒जम् भे॑ष॒जम् तस्मै॑ य॒ज्ञो य॒ज्ञ् स्तस्मै॑ भेष॒जम् । \newline
45. तस्मै॑ भेष॒जम् भे॑ष॒जम् तस्मै॒ तस्मै॑ भेष॒जम् क॑रोति करोति भेष॒जम् तस्मै॒ तस्मै॑ भेष॒जम् क॑रोति । \newline
46. भे॒ष॒जम् क॑रोति करोति भेष॒जम् भे॑ष॒जम् क॑रो त्यृत॒व्या॑ ऋत॒व्याः᳚ करोति भेष॒जम् भे॑ष॒जम् क॑रो त्यृत॒व्याः᳚ । \newline
47. क॒रो॒ त्यृ॒त॒व्या॑ ऋत॒व्याः᳚ करोति करो त्यृत॒व्या॑ उपोपा᳚ र्‌त॒व्याः᳚ करोति करो त्यृत॒व्या॑ उप॑ । \newline
48. ऋ॒त॒व्या॑ उपोपा᳚ र्‌त॒व्या॑ ऋत॒व्या॑ उप॑ दधाति दधा॒ त्युपा᳚ र्‌त॒व्या॑ ऋत॒व्या॑ उप॑ दधाति । \newline
49. उप॑ दधाति दधा॒ त्युपोप॑ दधा त्यृतू॒ना मृ॑तू॒नाम् द॑धा॒ त्युपोप॑ दधा त्यृतू॒नाम् । \newline
50. द॒धा॒ त्यृ॒तू॒ना मृ॑तू॒नाम् द॑धाति दधा त्यृतू॒नाम् क्लृप्त्यै॒ क्लृप्त्या॑ ऋतू॒नाम् द॑धाति दधा त्यृतू॒नाम् क्लृप्त्यै᳚ । \newline
51. ऋ॒तू॒नाम् क्लृप्त्यै॒ क्लृप्त्या॑ ऋतू॒ना मृ॑तू॒नाम् क्लृप्त्यै॒ पञ्च॒ पञ्च॒ क्लृप्त्या॑ ऋतू॒ना मृ॑तू॒नाम् क्लृप्त्यै॒ पञ्च॑ । \newline
52. क्लृप्त्यै॒ पञ्च॒ पञ्च॒ क्लृप्त्यै॒ क्लृप्त्यै॒ पञ्चोपोप॒ पञ्च॒ क्लृप्त्यै॒ क्लृप्त्यै॒ पञ्चोप॑ । \newline
\pagebreak
\markright{ TS 5.3.1.2  \hfill https://www.vedavms.in \hfill}

\section{ TS 5.3.1.2 }

\textbf{TS 5.3.1.2 } \newline
\textbf{Samhita Paata} \newline

पञ्चोप॑ दधाति॒ पञ्च॒ वा ऋ॒तवो॒ याव॑न्त ए॒वर्तव॒स्तान् क॑ल्पयति समा॒नप्र॑भृतयो भवन्ति समा॒नोद॑र्का॒स्तस्मा᳚थ् समा॒ना ऋ॒तव॒ एके॑न प॒देन॒ व्याव॑र्तन्ते॒ तस्मा॑द्-ऋ॒तवो॒ व्याव॑र्तन्ते प्राण॒भृत॒ उप॑ दधात्यृ॒तुष्वे॒व प्रा॒णान् द॑धाति॒ तस्मा᳚थ् समा॒नाः सन्त॑ ऋ॒तवो॒ न जी᳚र्यन्त्यथो॒ प्रज॑नयत्ये॒वैना॑ने॒ष वै वा॒युर्यत् प्रा॒णो यद्-ऋ॑त॒व्या॑ उप॒धाय॑ प्राण॒भृत॑ - [  ] \newline

\textbf{Pada Paata} \newline

पञ्च॑ । उपेति॑ । द॒धा॒ति॒ । पञ्च॑ । वै । ऋ॒तवः॑ । याव॑न्तः । ए॒व । ऋ॒तवः॑ । तान् । क॒ल्प॒य॒ति॒ । स॒मा॒नप्र॑भृतय॒ इति॑ समा॒न - प्र॒भृ॒त॒यः॒ । भ॒व॒न्ति॒ । स॒मा॒नोद॑र्का॒ इति॑ समा॒न - उ॒द॒र्काः॒ । तस्मा᳚त् । स॒मा॒नाः । ऋ॒तवः॑ । एके॑न । प॒देन॑ । व्याव॑र्तन्त॒ इति॑ वि - आव॑र्तन्ते । तस्मा᳚त् । ऋ॒तवः॑ । व्याव॑र्तन्त॒ इति॑ वि-आव॑र्तन्ते । प्रा॒ण॒भृत॒ इति॑ प्राण-भृतः॑ । उपेति॑ । द॒धा॒ति॒ । ऋ॒तुषु॑ । ए॒व । प्रा॒णानिति॑ प्र - अ॒नान् । द॒धा॒ति॒ । तस्मा᳚त् । स॒मा॒नाः । सन्तः॑ । ऋ॒तवः॑ । न । जी॒र्य॒न्ति॒ । अथो॒ इति॑ । प्रेति॑ । ज॒न॒य॒ति॒ । ए॒व । ए॒ना॒न् । ए॒षः । वै । वा॒युः । यत् । प्रा॒ण इति॑ प्र - अ॒नः । यत् । ऋ॒त॒व्याः᳚ । उ॒प॒धायेत्यु॑प - धाय॑ । प्रा॒ण॒भृत॒ इति॑ प्राण - भृतः॑ ।  \newline


\textbf{Krama Paata} \newline

पञ्चोप॑ । उप॑ दधाति । द॒धा॒ति॒ पञ्च॑ । पञ्च॒ वै । वा ऋ॒तवः॑ । ऋ॒तवो॒ याव॑न्तः । याव॑न्त ए॒व । ए॒वर्तवः॑ । ऋ॒तव॒स्तान् । तान् क॑ल्पयति । क॒ल्प॒य॒ति॒ स॒मा॒नप्र॑भृतयः । स॒मा॒नप्र॑भृतयो भवन्ति । स॒मा॒नप्र॑भृतय॒ इति॑ समा॒न - प्र॒भृ॒त॒यः॒ । भ॒व॒न्ति॒ स॒मा॒नोद॑र्काः । स॒मा॒नोद॑र्का॒स्तस्मा᳚त् । स॒मा॒नोद॑र्का॒ इति॑ समा॒न - उ॒द॒र्काः॒ । तस्मा᳚थ् समा॒नाः । स॒मा॒ना ऋ॒तवः॑ । ऋ॒तव॒ एके॑न । एके॑न प॒देन॑ । प॒देन॒ व्याव॑र्तन्ते । व्याव॑र्तन्ते॒ तस्मा᳚त् । व्याव॑र्तन्त॒ इति॑ वि - आव॑र्तन्ते । तस्मा॑दृ॒तवः॑ । ऋ॒तवो॒ व्याव॑र्तन्ते । व्याव॑र्तन्ते प्राण॒भृतः॑ । व्याव॑र्तन्त॒ इति॑ वि - आव॑र्तन्ते । प्रा॒ण॒भृत॒ उप॑ । प्रा॒ण॒भृत॒ इति॑ प्राण - भृतः॑ । उप॑ दधाति । द॒धा॒त्यृ॒तुषु॑ । ऋ॒तुष्वे॒व । ए॒व प्रा॒णान् । प्रा॒णान् द॑धाति । प्रा॒णानिति॑ प्र - अ॒नान् । द॒धा॒ति॒ तस्मा᳚त् । तस्मा᳚थ् समा॒नाः । स॒मा॒नाः सन्तः॑ । सन्त॑ ऋ॒तवः॑ । ऋ॒तवो॒ न । न जी᳚र्यन्ति । जी॒र्य॒न्त्यथो᳚ । अथो॒ प्र । अथो॒ इत्यथो᳚ । प्र ज॑नयति । ज॒न॒य॒त्ये॒व । ए॒वैनान्॑ । ए॒ना॒ने॒षः । ए॒ष वै । वै वा॒युः । वा॒युर् यत् । यत् प्रा॒णः । प्रा॒णो यत् । प्रा॒ण इति॑ प्र - अ॒नः । यदृ॑त॒व्याः᳚ । ऋ॒त॒व्या॑ उप॒धाय॑ । उ॒प॒धाय॑ प्राण॒भृतः॑ । उ॒पा॒धायेत्यु॑प - धाय॑ । प्रा॒ण॒भृत॑ उप॒दधा॑ति । प्रा॒ण॒भृत॒ इति॑ प्राण - भृतः॑ \newline

\textbf{Jatai Paata} \newline

1. पञ्चोपोप॒ पञ्च॒ पञ्चोप॑ । \newline
2. उप॑ दधाति दधा॒ त्युपोप॑ दधाति । \newline
3. द॒धा॒ति॒ पञ्च॒ पञ्च॑ दधाति दधाति॒ पञ्च॑ । \newline
4. पञ्च॒ वै वै पञ्च॒ पञ्च॒ वै । \newline
5. वा ऋ॒तव॑ ऋ॒तवो॒ वै वा ऋ॒तवः॑ । \newline
6. ऋ॒तवो॒ याव॑न्तो॒ याव॑न्त ऋ॒तव॑ ऋ॒तवो॒ याव॑न्तः । \newline
7. याव॑न्त ए॒वैव याव॑न्तो॒ याव॑न्त ए॒व । \newline
8. ए॒व र्‌तव॑ ऋ॒तव॑ ए॒वैव र्‌तवः॑ । \newline
9. ऋ॒तव॒ स्ताꣳ स्ता नृ॒तव॑ ऋ॒तव॒ स्तान् । \newline
10. तान् क॑ल्पयति कल्पयति॒ ताꣳ स्तान् क॑ल्पयति । \newline
11. क॒ल्प॒य॒ति॒ स॒मा॒नप्र॑भृतयः समा॒नप्र॑भृतयः कल्पयति कल्पयति समा॒नप्र॑भृतयः । \newline
12. स॒मा॒नप्र॑भृतयो भवन्ति भवन्ति समा॒नप्र॑भृतयः समा॒नप्र॑भृतयो भवन्ति । \newline
13. स॒मा॒नप्र॑भृतय॒ इति॑ समा॒न - प्र॒भृ॒त॒यः॒ । \newline
14. भ॒व॒न्ति॒ स॒मा॒नोद॑र्काः समा॒नोद॑र्का भवन्ति भवन्ति समा॒नोद॑र्काः । \newline
15. स॒मा॒नोद॑र्का॒ स्तस्मा॒त् तस्मा᳚थ् समा॒नोद॑र्काः समा॒नोद॑र्का॒ स्तस्मा᳚त् । \newline
16. स॒मा॒नोद॑र्का॒ इति॑ समा॒न - उ॒द॒र्काः॒ । \newline
17. तस्मा᳚थ् समा॒नाः स॑मा॒ना स्तस्मा॒त् तस्मा᳚थ् समा॒नाः । \newline
18. स॒मा॒ना ऋ॒तव॑ ऋ॒तवः॑ समा॒नाः स॑मा॒ना ऋ॒तवः॑ । \newline
19. ऋ॒तव॒ एके॒नैके॑न॒ र्तव॑ ऋ॒तव॒ एके॑न । \newline
20. एके॑न प॒देन॑ प॒देनैके॒ नैके॑न प॒देन॑ । \newline
21. प॒देन॒ व्याव॑र्तन्ते॒ व्याव॑र्तन्ते प॒देन॑ प॒देन॒ व्याव॑र्तन्ते । \newline
22. व्याव॑र्तन्ते॒ तस्मा॒त् तस्मा॒द् व्याव॑र्तन्ते॒ व्याव॑र्तन्ते॒ तस्मा᳚त् । \newline
23. व्याव॑र्तन्त॒ इति॑ वि - आव॑र्तन्ते । \newline
24. तस्मा॑ दृ॒तव॑ ऋ॒तव॒ स्तस्मा॒त् तस्मा॑ दृ॒तवः॑ । \newline
25. ऋ॒तवो॒ व्याव॑र्तन्ते॒ व्याव॑र्तन्त ऋ॒तव॑ ऋ॒तवो॒ व्याव॑र्तन्ते । \newline
26. व्याव॑र्तन्ते प्राण॒भृतः॑ प्राण॒भृतो॒ व्याव॑र्तन्ते॒ व्याव॑र्तन्ते प्राण॒भृतः॑ । \newline
27. व्याव॑र्तन्त॒ इति॑ वि - आव॑र्तन्ते । \newline
28. प्रा॒ण॒भृत॒ उपोप॑ प्राण॒भृतः॑ प्राण॒भृत॒ उप॑ । \newline
29. प्रा॒ण॒भृत॒ इति॑ प्राण - भृतः॑ । \newline
30. उप॑ दधाति दधा॒ त्युपोप॑ दधाति । \newline
31. द॒धा॒ त्यृ॒तु ष्वृ॒तुषु॑ दधाति दधा त्यृ॒तुषु॑ । \newline
32. ऋ॒तु ष्वे॒वैव र्‌तुष्वृ॒तु ष्वे॒व । \newline
33. ए॒व प्रा॒णान् प्रा॒णा ने॒वैव प्रा॒णान् । \newline
34. प्रा॒णान् द॑धाति दधाति प्रा॒णान् प्रा॒णान् द॑धाति । \newline
35. प्रा॒णानिति॑ प्र - अ॒नान् । \newline
36. द॒धा॒ति॒ तस्मा॒त् तस्मा᳚द् दधाति दधाति॒ तस्मा᳚त् । \newline
37. तस्मा᳚थ् समा॒नाः स॑मा॒ना स्तस्मा॒त् तस्मा᳚थ् समा॒नाः । \newline
38. स॒मा॒नाः सन्तः॒ सन्तः॑ समा॒नाः स॑मा॒नाः सन्तः॑ । \newline
39. सन्त॑ ऋ॒तव॑ ऋ॒तवः॒ सन्तः॒ सन्त॑ ऋ॒तवः॑ । \newline
40. ऋ॒तवो॒ न न र्‌तव॑ ऋ॒तवो॒ न । \newline
41. न जी᳚र्यन्ति जीर्यन्ति॒ न न जी᳚र्यन्ति । \newline
42. जी॒र्य॒न् त्यथो॒ अथो॑ जीर्यन्ति जीर्य॒न् त्यथो᳚ । \newline
43. अथो॒ प्र प्राथो॒ अथो॒ प्र । \newline
44. अथो॒ इत्यथो᳚ । \newline
45. प्र ज॑नयति जनयति॒ प्र प्र ज॑नयति । \newline
46. ज॒न॒य॒ त्ये॒वैव ज॑नयति जनय त्ये॒व । \newline
47. ए॒वैना॑ नेना ने॒वै वैनान्॑ । \newline
48. ए॒ना॒ ने॒ष ए॒ष ए॑ना नेना ने॒षः । \newline
49. ए॒ष वै वा ए॒ष ए॒ष वै । \newline
50. वै वा॒युर् वा॒युर् वै वै वा॒युः । \newline
51. वा॒युर् यद् यद् वा॒युर् वा॒युर् यत् । \newline
52. यत् प्रा॒णः प्रा॒णो यद् यत् प्रा॒णः । \newline
53. प्रा॒णो यद् यत् प्रा॒णः प्रा॒णो यत् । \newline
54. प्रा॒ण इति॑ प्र - अ॒नः । \newline
55. यदृ॑त॒व्या॑ ऋत॒व्या॑ यद् यदृ॑त॒व्याः᳚ । \newline
56. ऋ॒त॒व्या॑ उप॒धायो॑ प॒धाय॑ र्‌त॒व्या॑ ऋत॒व्या॑ उप॒धाय॑ । \newline
57. उ॒प॒धाय॑ प्राण॒भृतः॑ प्राण॒भृत॑ उप॒धायो॑ प॒धाय॑ प्राण॒भृतः॑ । \newline
58. उ॒प॒धायेत्यु॑प - धाय॑ । \newline
59. प्रा॒ण॒भृत॑ उप॒दधा᳚ त्युप॒दधा॑ति प्राण॒भृतः॑ प्राण॒भृत॑ उप॒दधा॑ति । \newline
60. प्रा॒ण॒भृत॒ इति॑ प्राण - भृतः॑ । \newline

\textbf{Ghana Paata } \newline

1. पञ्चोपोप॒ पञ्च॒ पञ्चोप॑ दधाति दधा॒ त्युप॒ पञ्च॒ पञ्चोप॑ दधाति । \newline
2. उप॑ दधाति दधा॒ त्युपोप॑ दधाति॒ पञ्च॒ पञ्च॑ दधा॒ त्युपोप॑ दधाति॒ पञ्च॑ । \newline
3. द॒धा॒ति॒ पञ्च॒ पञ्च॑ दधाति दधाति॒ पञ्च॒ वै वै पञ्च॑ दधाति दधाति॒ पञ्च॒ वै । \newline
4. पञ्च॒ वै वै पञ्च॒ पञ्च॒ वा ऋ॒तव॑ ऋ॒तवो॒ वै पञ्च॒ पञ्च॒ वा ऋ॒तवः॑ । \newline
5. वा ऋ॒तव॑ ऋ॒तवो॒ वै वा ऋ॒तवो॒ याव॑न्तो॒ याव॑न्त ऋ॒तवो॒ वै वा ऋ॒तवो॒ याव॑न्तः । \newline
6. ऋ॒तवो॒ याव॑न्तो॒ याव॑न्त ऋ॒तव॑ ऋ॒तवो॒ याव॑न्त ए॒वैव याव॑न्त ऋ॒तव॑ ऋ॒तवो॒ याव॑न्त ए॒व । \newline
7. याव॑न्त ए॒वैव याव॑न्तो॒ याव॑न्त ए॒व र्‌तव॑ ऋ॒तव॑ ए॒व याव॑न्तो॒ याव॑न्त ए॒व र्‌तवः॑ । \newline
8. ए॒व र्तव॑ ऋ॒तव॑ ए॒वैव र्‌तव॒ स्ताꣳ स्ता नृ॒तव॑ ए॒वैव र्‌तव॒ स्तान् । \newline
9. ऋ॒तव॒ स्ताꣳ स्ता नृ॒तव॑ ऋ॒तव॒ स्तान् क॑ल्पयति कल्पयति॒ ता नृ॒तव॑ ऋ॒तव॒ स्तान् क॑ल्पयति । \newline
10. तान् क॑ल्पयति कल्पयति॒ ताꣳ स्तान् क॑ल्पयति समा॒नप्र॑भृतयः समा॒नप्र॑भृतयः कल्पयति॒ ताꣳ स्तान् क॑ल्पयति समा॒नप्र॑भृतयः । \newline
11. क॒ल्प॒य॒ति॒ स॒मा॒नप्र॑भृतयः समा॒नप्र॑भृतयः कल्पयति कल्पयति समा॒नप्र॑भृतयो भवन्ति भवन्ति समा॒नप्र॑भृतयः कल्पयति कल्पयति समा॒नप्र॑भृतयो भवन्ति । \newline
12. स॒मा॒नप्र॑भृतयो भवन्ति भवन्ति समा॒नप्र॑भृतयः समा॒नप्र॑भृतयो भवन्ति समा॒नोद॑र्काः समा॒नोद॑र्का भवन्ति समा॒नप्र॑भृतयः समा॒नप्र॑भृतयो भवन्ति समा॒नोद॑र्काः । \newline
13. स॒मा॒नप्र॑भृतय॒ इति॑ समा॒न - प्र॒भृ॒त॒यः॒ । \newline
14. भ॒व॒न्ति॒ स॒मा॒नोद॑र्काः समा॒नोद॑र्का भवन्ति भवन्ति समा॒नोद॑र्का॒ स्तस्मा॒त् तस्मा᳚थ् समा॒नोद॑र्का भवन्ति भवन्ति समा॒नोद॑र्का॒ स्तस्मा᳚त् । \newline
15. स॒मा॒नोद॑र्का॒ स्तस्मा॒त् तस्मा᳚थ् समा॒नोद॑र्काः समा॒नोद॑र्का॒ स्तस्मा᳚थ् समा॒नाः स॑मा॒ना स्तस्मा᳚थ् समा॒नोद॑र्काः समा॒नोद॑र्का॒ स्तस्मा᳚थ् समा॒नाः । \newline
16. स॒मा॒नोद॑र्का॒ इति॑ समा॒न - उ॒द॒र्काः॒ । \newline
17. तस्मा᳚थ् समा॒नाः स॑मा॒ना स्तस्मा॒त् तस्मा᳚थ् समा॒ना ऋ॒तव॑ ऋ॒तवः॑ समा॒ना स्तस्मा॒त् तस्मा᳚थ् समा॒ना ऋ॒तवः॑ । \newline
18. स॒मा॒ना ऋ॒तव॑ ऋ॒तवः॑ समा॒नाः स॑मा॒ना ऋ॒तव॒ एके॒ नैके॑न॒ र्‌तवः॑ समा॒नाः स॑मा॒ना ऋ॒तव॒ एके॑न । \newline
19. ऋ॒तव॒ एके॒नै के॑न॒ र्‌तव॑ ऋ॒तव॒ एके॑न प॒देन॑ प॒दे नैके॑न॒ र्‌तव॑ ऋ॒तव॒ एके॑न प॒देन॑ । \newline
20. एके॑न प॒देन॑ प॒दे नैके॒ नैके॑न प॒देन॒ व्याव॑र्तन्ते॒ व्याव॑र्तन्ते प॒दे नैके॒नै के॑न प॒देन॒ व्याव॑र्तन्ते । \newline
21. प॒देन॒ व्याव॑र्तन्ते॒ व्याव॑र्तन्ते प॒देन॑ प॒देन॒ व्याव॑र्तन्ते॒ तस्मा॒त् तस्मा॒द् व्याव॑र्तन्ते प॒देन॑ प॒देन॒ व्याव॑र्तन्ते॒ तस्मा᳚त् । \newline
22. व्याव॑र्तन्ते॒ तस्मा॒त् तस्मा॒द् व्याव॑र्तन्ते॒ व्याव॑र्तन्ते॒ तस्मा॑ दृ॒तव॑ ऋ॒तव॒ स्तस्मा॒द् व्याव॑र्तन्ते॒ व्याव॑र्तन्ते॒ तस्मा॑ दृ॒तवः॑ । \newline
23. व्याव॑र्तन्त॒ इति॑ वि - आव॑र्तन्ते । \newline
24. तस्मा॑ दृ॒तव॑ ऋ॒तव॒ स्तस्मा॒त् तस्मा॑ दृ॒तवो॒ व्याव॑र्तन्ते॒ व्याव॑र्तन्त ऋ॒तव॒ स्तस्मा॒त् तस्मा॑ दृ॒तवो॒ व्याव॑र्तन्ते । \newline
25. ऋ॒तवो॒ व्याव॑र्तन्ते॒ व्याव॑र्तन्त ऋ॒तव॑ ऋ॒तवो॒ व्याव॑र्तन्ते प्राण॒भृतः॑ प्राण॒भृतो॒ व्याव॑र्तन्त ऋ॒तव॑ ऋ॒तवो॒ व्याव॑र्तन्ते प्राण॒भृतः॑ । \newline
26. व्याव॑र्तन्ते प्राण॒भृतः॑ प्राण॒भृतो॒ व्याव॑र्तन्ते॒ व्याव॑र्तन्ते प्राण॒भृत॒ उपोप॑ प्राण॒भृतो॒ व्याव॑र्तन्ते॒ व्याव॑र्तन्ते प्राण॒भृत॒ उप॑ । \newline
27. व्याव॑र्तन्त॒ इति॑ वि - आव॑र्तन्ते । \newline
28. प्रा॒ण॒भृत॒ उपोप॑ प्राण॒भृतः॑ प्राण॒भृत॒ उप॑ दधाति दधा॒ त्युप॑ प्राण॒भृतः॑ प्राण॒भृत॒ उप॑ दधाति । \newline
29. प्रा॒ण॒भृत॒ इति॑ प्राण - भृतः॑ । \newline
30. उप॑ दधाति दधा॒ त्युपोप॑ दधा त्यृ॒तु ष्वृ॒तुषु॑ दधा॒ त्युपोप॑ दधा त्यृ॒तुषु॑ । \newline
31. द॒धा॒ त्यृ॒तु ष्वृ॒तुषु॑ दधाति दधा त्यृ॒तु ष्वे॒वैव र्‌तुषु॑ दधाति दधा त्यृ॒तु ष्वे॒व । \newline
32. ऋ॒तु ष्वे॒वैव र्‌तु ष्वृ॒तु ष्वे॒व प्रा॒णान् प्रा॒णा ने॒व र्‌तु ष्वृ॒तु ष्वे॒व प्रा॒णान् । \newline
33. ए॒व प्रा॒णान् प्रा॒णा ने॒वैव प्रा॒णान् द॑धाति दधाति प्रा॒णा ने॒वैव प्रा॒णान् द॑धाति । \newline
34. प्रा॒णान् द॑धाति दधाति प्रा॒णान् प्रा॒णान् द॑धाति॒ तस्मा॒त् तस्मा᳚द् दधाति प्रा॒णान् प्रा॒णान् द॑धाति॒ तस्मा᳚त् । \newline
35. प्रा॒णानिति॑ प्र - अ॒नान् । \newline
36. द॒धा॒ति॒ तस्मा॒त् तस्मा᳚द् दधाति दधाति॒ तस्मा᳚थ् समा॒नाः स॑मा॒ना स्तस्मा᳚द् दधाति दधाति॒ तस्मा᳚थ् समा॒नाः । \newline
37. तस्मा᳚थ् समा॒नाः स॑मा॒ना स्तस्मा॒त् तस्मा᳚थ् समा॒नाः सन्तः॒ सन्तः॑ समा॒ना स्तस्मा॒त् तस्मा᳚थ् समा॒नाः सन्तः॑ । \newline
38. स॒मा॒नाः सन्तः॒ सन्तः॑ समा॒नाः स॑मा॒नाः सन्त॑ ऋ॒तव॑ ऋ॒तवः॒ सन्तः॑ समा॒नाः स॑मा॒नाः सन्त॑ ऋ॒तवः॑ । \newline
39. सन्त॑ ऋ॒तव॑ ऋ॒तवः॒ सन्तः॒ सन्त॑ ऋ॒तवो॒ न न र्‌तवः॒ सन्तः॒ सन्त॑ ऋ॒तवो॒ न । \newline
40. ऋ॒तवो॒ न न र्‌तव॑ ऋ॒तवो॒ न जी᳚र्यन्ति जीर्यन्ति॒ न र्‌तव॑ ऋ॒तवो॒ न जी᳚र्यन्ति । \newline
41. न जी᳚र्यन्ति जीर्यन्ति॒ न न जी᳚र्य॒न् त्यथो॒ अथो॑ जीर्यन्ति॒ न न जी᳚र्य॒न् त्यथो᳚ । \newline
42. जी॒र्य॒न् त्यथो॒ अथो॑ जीर्यन्ति जीर्य॒न् त्यथो॒ प्र प्राथो॑ जीर्यन्ति जीर्य॒न् त्यथो॒ प्र । \newline
43. अथो॒ प्र प्राथो॒ अथो॒ प्र ज॑नयति जनयति॒ प्राथो॒ अथो॒ प्र ज॑नयति । \newline
44. अथो॒ इत्यथो᳚ । \newline
45. प्र ज॑नयति जनयति॒ प्र प्र ज॑नय त्ये॒वैव ज॑नयति॒ प्र प्र ज॑नय त्ये॒व । \newline
46. ज॒न॒य॒ त्ये॒वैव ज॑नयति जनय त्ये॒वैना॑ नेना ने॒व ज॑नयति जनय त्ये॒वैनान्॑ । \newline
47. ए॒वैना॑ नेना ने॒वै वैना॑ ने॒ष ए॒ष ए॑ना ने॒वै वैना॑ ने॒षः । \newline
48. ए॒ना॒ ने॒ष ए॒ष ए॑ना नेना ने॒ष वै वा ए॒ष ए॑ना नेना ने॒ष वै । \newline
49. ए॒ष वै वा ए॒ष ए॒ष वै वा॒युर् वा॒युर् वा ए॒ष ए॒ष वै वा॒युः । \newline
50. वै वा॒युर् वा॒युर् वै वै वा॒युर् यद् यद् वा॒युर् वै वै वा॒युर् यत् । \newline
51. वा॒युर् यद् यद् वा॒युर् वा॒युर् यत् प्रा॒णः प्रा॒णो यद् वा॒युर् वा॒युर् यत् प्रा॒णः । \newline
52. यत् प्रा॒णः प्रा॒णो यद् यत् प्रा॒णो यद् यत् प्रा॒णो यद् यत् प्रा॒णो यत् । \newline
53. प्रा॒णो यद् यत् प्रा॒णः प्रा॒णो यदृ॑त॒व्या॑ ऋत॒व्या॑ यत् प्रा॒णः प्रा॒णो यदृ॑त॒व्याः᳚ । \newline
54. प्रा॒ण इति॑ प्र - अ॒नः । \newline
55. यदृ॑त॒व्या॑ ऋत॒व्या॑ यद् यदृ॑त॒व्या॑ उप॒धा यो॑प॒धाय॑ र्‌त॒व्या॑ यद् यदृ॑त॒व्या॑ उप॒धाय॑ । \newline
56. ऋ॒त॒व्या॑ उप॒धा यो॑प॒धाय॑ र्‌त॒व्या॑ ऋत॒व्या॑ उप॒धाय॑ प्राण॒भृतः॑ प्राण॒भृत॑ उप॒धाय॑ र्‌त॒व्या॑ ऋत॒व्या॑ उप॒धाय॑ प्राण॒भृतः॑ । \newline
57. उ॒प॒धाय॑ प्राण॒भृतः॑ प्राण॒भृत॑ उप॒धायो॑ प॒धाय॑ प्राण॒भृत॑ उप॒दधा᳚ त्युप॒दधा॑ति प्राण॒भृत॑ उप॒धायो॑ प॒धाय॑ प्राण॒भृत॑ उप॒दधा॑ति । \newline
58. उ॒प॒धायेत्यु॑प - धाय॑ । \newline
59. प्रा॒ण॒भृत॑ उप॒दधा᳚ त्युप॒दधा॑ति प्राण॒भृतः॑ प्राण॒भृत॑ उप॒दधा॑ति॒ तस्मा॒त् तस्मा॑ दुप॒दधा॑ति प्राण॒भृतः॑ प्राण॒भृत॑ उप॒दधा॑ति॒ तस्मा᳚त् । \newline
60. प्रा॒ण॒भृत॒ इति॑ प्राण - भृतः॑ । \newline
\pagebreak
\markright{ TS 5.3.1.3  \hfill https://www.vedavms.in \hfill}

\section{ TS 5.3.1.3 }

\textbf{TS 5.3.1.3 } \newline
\textbf{Samhita Paata} \newline

उप॒दधा॑ति॒ तस्मा॒थ् सर्वा॑नृ॒तूननु॑ वा॒युरा व॑रीवर्त्ति वृष्टि॒सनी॒रुप॑ दधाति॒ वृष्टि॑मे॒वाव॑ रुन्धे॒ यदे॑क॒धोप॑द॒द्ध्या-देक॑मृ॒तुं ॅव॑र्.षेदनुपरि॒हारꣳ॑ सादयति॒ तस्मा॒थ् सर्वा॑नृ॒तून्. व॑र्.षति॒ यत् प्रा॑ण॒भृत॑ उप॒धाय॑ वृष्टि॒सनी॑रुप॒दधा॑ति॒ तस्मा᳚द्-वा॒युप्र॑च्युता दि॒वो वृष्टि॑रीर्ते प॒शवो॒ वै व॑य॒स्या॑ नाना॑मनसः॒ खलु॒ वै प॒शवो॒ नाना᳚व्रता॒स्ते॑ऽप ए॒वाभि सम॑नसो॒ - [  ] \newline

\textbf{Pada Paata} \newline

उ॒प॒दधा॒तीत्यु॑प - दधा॑ति । तस्मा᳚त् । सर्वान्॑ । ऋ॒तून् । अन्विति॑ । वा॒युः । एति॑ । व॒री॒व॒र्ति॒ । वृ॒ष्टि॒सनी॒रिति॑ वृष्टि - सनीः᳚ । उपेति॑ । द॒धा॒ति॒ । वृष्टि᳚म् । ए॒व । अवेति॑ । रु॒न्धे॒ । यत् । ए॒क॒धेत्ये॑क - धा । उ॒प॒द॒द्ध्यादित्यु॑प - द॒द्ध्यात् । एक᳚म् । ऋ॒तुम् । व॒र्.॒षे॒त् । अ॒नु॒प॒रि॒हार॒मित्य॑नु-प॒रि॒हार᳚म् । सा॒द॒य॒ति॒ । तस्मा᳚त् । सर्वान्॑ । ऋ॒तून् । व॒र्.॒ष॒ति॒ । यत् । प्रा॒ण॒भृत॒ इति॑ प्राण - भृतः॑ । उ॒प॒धायेत्यु॑प - धाय॑ । वृ॒ष्टि॒सनी॒रिति॑ वृष्टि - सनीः᳚ । उ॒प॒दधा॒तीत्यु॑प - दधा॑ति । तस्मा᳚त् । वा॒युप्र॑च्यु॒तेति॑ वा॒यु - प्र॒च्यु॒ता॒ । दि॒वः । वृष्टिः॑ । ई॒र्ते॒ । प॒शवः॑ । वै । व॒य॒स्याः᳚ । नाना॑मनस॒ इति॒ नाना᳚ - म॒न॒सः॒ । खलु॑ । वै । प॒शवः॑ । नाना᳚व्रता॒ इति॒ नाना᳚ - व्र॒ताः॒ । ते । अ॒पः । ए॒व । अ॒भीति॑ । सम॑नस॒ इति॒ स - म॒न॒सः॒ ।  \newline


\textbf{Krama Paata} \newline

उ॒प॒दधा॑ति॒ तस्मा᳚त् । उ॒प॒दधा॒तीत्यु॑प - दधा॑ति । तस्मा॒थ् सर्वान्॑ । सर्वा॑नृ॒तून् । ऋ॒तूननु॑ । अनु॑ वा॒युः । वा॒युरा । आ व॑रीवर्ति । व॒री॒व॒र्ति॒ वृ॒ष्टि॒सनीः᳚ । वृ॒ष्टि॒सनी॒रुप॑ । वृ॒ष्टि॒सनी॒रिति॑ वृष्टि - सनीः᳚ । उप॑ दधाति । द॒धा॒ति॒ वृष्टि᳚म् । वृष्टि॑मे॒व । ए॒वाव॑ । अव॑ रुन्धे । रु॒न्धे॒ यत् । यदे॑क॒धा । ए॒क॒धोप॑द॒द्ध्यात् । ए॒क॒धेत्ये॑क - धा । उ॒प॒द॒द्ध्यादेक᳚म् । उ॒प॒द॒द्ध्यादित्यु॑प - द॒द्ध्यात् । एक॑मृ॒तुम् । ऋ॒तुम् ॅव॑र्.षेत् । व॒र्॒.षे॒द॒नु॒प॒रि॒हार᳚म् । अ॒नु॒प॒रि॒हारꣳ॑ सादयति । अ॒नु॒प॒रि॒हार॒मित्य॑नु - प॒रि॒हार᳚म् । सा॒द॒य॒ति॒ तस्मा᳚त् । तस्मा॒थ् सर्वान्॑ । सर्वा॑नृ॒तून् । ऋ॒तून्. व॑र्.षति । व॒र्॒.ष॒ति॒ यत् । यत् प्रा॑ण॒भृतः॑ । प्रा॒ण॒भृत॑ उप॒धाय॑ । प्रा॒ण॒भृत॒ इति॑ प्राण - भृतः॑ । उ॒प॒धाय॑ वृष्टि॒सनीः᳚ । उ॒प॒धायेत्यु॑प - धाय॑ । वृ॒ष्टि॒सनी॑रुप॒दधा॑ति । वृ॒ष्टि॒सनी॒रिति॑ वृष्टि - सनीः᳚ । उ॒प॒दधा॑ति॒ तस्मा᳚त् । उ॒प॒दधा॒तीत्यु॑प - दधा॑ति । तस्मा᳚द् वा॒युप्र॑च्युता । वा॒युप्र॑च्युता दि॒वः । वा॒युप्र॑च्यु॒तेति॑ वा॒यु - प्र॒च्यु॒ता॒ । दि॒वो वृष्टिः॑ । वृष्टि॑रीर्ते । ई॒र्ते॒ प॒शवः॑ । प॒शवो॒ वै । वै व॑य॒स्याः᳚ । व॒य॒स्या॑ नाना॑मनसः । नाना॑मनसः॒ खलु॑ । नाना॑मनस॒ इति॒ नाना᳚ - म॒न॒सः॒ । खलु॒ वै । वै प॒शवः॑ । प॒शवो॒ नाना᳚व्रताः । नाना᳚व्रता॒स्ते । नाना᳚व्रता॒ इति॒ नाना᳚ - व्र॒ताः॒ । ते॑ऽपः । अ॒प ए॒व । ए॒वाभि । अ॒भि सम॑नसः । सम॑नसो॒ यम् । सम॑नस॒ इति॒ स - म॒न॒सः॒ \newline

\textbf{Jatai Paata} \newline

1. उ॒प॒दधा॑ति॒ तस्मा॒त् तस्मा॑ दुप॒दधा᳚ त्युप॒दधा॑ति॒ तस्मा᳚त् । \newline
2. उ॒प॒दधा॒तीत्यु॑प - दधा॑ति । \newline
3. तस्मा॒थ् सर्वा॒न् थ्सर्वा॒न् तस्मा॒त् तस्मा॒थ् सर्वान्॑ । \newline
4. सर्वा॑ नृ॒तू नृ॒तून् थ्सर्वा॒न् थ्सर्वा॑ नृ॒तून् । \newline
5. ऋ॒तू नन्वन् वृ॒तू नृ॒तू ननु॑ । \newline
6. अनु॑ वा॒युर् वा॒यु रन्वनु॑ वा॒युः । \newline
7. वा॒युरा वा॒युर् वा॒युरा । \newline
8. आ व॑रीवर्ति वरीव॒र्त्या व॑रीवर्ति । \newline
9. व॒री॒व॒र्ति॒ वृ॒ष्टि॒सनी᳚र् वृष्टि॒सनी᳚र् वरीवर्ति वरीवर्ति वृष्टि॒सनीः᳚ । \newline
10. वृ॒ष्टि॒सनी॒ रुपोप॑ वृष्टि॒सनी᳚र् वृष्टि॒सनी॒ रुप॑ । \newline
11. वृ॒ष्टि॒सनी॒रिति॑ वृष्टि - सनीः᳚ । \newline
12. उप॑ दधाति दधा॒ त्युपोप॑ दधाति । \newline
13. द॒धा॒ति॒ वृष्टिं॒ ॅवृष्टि॑म् दधाति दधाति॒ वृष्टि᳚म् । \newline
14. वृष्टि॑ मे॒वैव वृष्टिं॒ ॅवृष्टि॑ मे॒व । \newline
15. ए॒वावा वै॒वै वाव॑ । \newline
16. अव॑ रुन्धे रु॒न्धे ऽवाव॑ रुन्धे । \newline
17. रु॒न्धे॒ यद् यद् रु॑न्धे रुन्धे॒ यत् । \newline
18. यदे॑क॒ धैक॒धा यद् यदे॑क॒धा । \newline
19. ए॒क॒धो प॑द॒द्ध्या दु॑पद॒द्ध्या दे॑क॒ धैक॒धो प॑द॒द्ध्यात् । \newline
20. ए॒क॒धेत्ये॑क - धा । \newline
21. उ॒प॒द॒द्ध्या देक॒ मेक॑ मुपद॒द्ध्या दु॑पद॒द्ध्या देक᳚म् । \newline
22. उ॒प॒द॒द्ध्यादित्यु॑प - द॒द्ध्यात् । \newline
23. एक॑ मृ॒तु मृ॒तु मेक॒ मेक॑ मृ॒तुम् । \newline
24. ऋ॒तुं ॅव॑र्.षेद् वर्.षेदृ॒तु मृ॒तुं ॅव॑र्.षेत् । \newline
25. व॒र्॒.षे॒ द॒नु॒प॒रि॒हार॑ मनुपरि॒हारं॑ ॅवर्.षेद् वर्.षे दनुपरि॒हार᳚म् । \newline
26. अ॒नु॒प॒रि॒हारꣳ॑ सादयति सादय त्यनुपरि॒हार॑ मनुपरि॒हारꣳ॑ सादयति । \newline
27. अ॒नु॒प॒रि॒हार॒मित्य॑नु - प॒रि॒हार᳚म् । \newline
28. सा॒द॒य॒ति॒ तस्मा॒त् तस्मा᳚थ् सादयति सादयति॒ तस्मा᳚त् । \newline
29. तस्मा॒थ् सर्वा॒न् थ्सर्वा॒न् तस्मा॒त् तस्मा॒थ् सर्वान्॑ । \newline
30. सर्वा॑ नृ॒तू नृ॒तून् थ्सर्वा॒न् थ्सर्वा॑ नृ॒तून् । \newline
31. ऋ॒तून्. व॑र्.षति वर्.ष त्यृ॒तू नृ॒तून्. व॑र्.षति । \newline
32. व॒र्॒.ष॒ति॒ यद् यद् व॑र्.षति वर्.षति॒ यत् । \newline
33. यत् प्रा॑ण॒भृतः॑ प्राण॒भृतो॒ यद् यत् प्रा॑ण॒भृतः॑ । \newline
34. प्रा॒ण॒भृत॑ उप॒धा यो॑प॒धाय॑ प्राण॒भृतः॑ प्राण॒भृत॑ उप॒धाय॑ । \newline
35. प्रा॒ण॒भृत॒ इति॑ प्राण - भृतः॑ । \newline
36. उ॒प॒धाय॑ वृष्टि॒सनी᳚र् वृष्टि॒सनी॑ रुप॒धा यो॑प॒धाय॑ वृष्टि॒सनीः᳚ । \newline
37. उ॒प॒धायेत्यु॑प - धाय॑ । \newline
38. वृ॒ष्टि॒सनी॑ रुप॒दधा᳚ त्युप॒दधा॑ति वृष्टि॒सनी᳚र् वृष्टि॒सनी॑ रुप॒दधा॑ति । \newline
39. वृ॒ष्टि॒सनी॒रिति॑ वृष्टि - सनीः᳚ । \newline
40. उ॒प॒दधा॑ति॒ तस्मा॒त् तस्मा॑ दुप॒दधा᳚ त्युप॒दधा॑ति॒ तस्मा᳚त् । \newline
41. उ॒प॒दधा॒तीत्यु॑प - दधा॑ति । \newline
42. तस्मा᳚द् वा॒युप्र॑च्युता वा॒युप्र॑च्युता॒ तस्मा॒त् तस्मा᳚द् वा॒युप्र॑च्युता । \newline
43. वा॒युप्र॑च्युता दि॒वो दि॒वो वा॒युप्र॑च्युता वा॒युप्र॑च्युता दि॒वः । \newline
44. वा॒युप्र॑च्यु॒तेति॑ वा॒यु - प्र॒च्यु॒ता॒ । \newline
45. दि॒वो वृष्टि॒र् वृष्टि॑र् दि॒वो दि॒वो वृष्टिः॑ । \newline
46. वृष्टि॑ रीर्त ईर्ते॒ वृष्टि॒र् वृष्टि॑ रीर्ते । \newline
47. ई॒र्ते॒ प॒शवः॑ प॒शव॑ ईर्त ईर्ते प॒शवः॑ । \newline
48. प॒शवो॒ वै वै प॒शवः॑ प॒शवो॒ वै । \newline
49. वै व॑य॒स्या॑ वय॒स्या॑ वै वै व॑य॒स्याः᳚ । \newline
50. व॒य॒स्या॑ नाना॑मनसो॒ नाना॑मनसो वय॒स्या॑ वय॒स्या॑ नाना॑मनसः । \newline
51. नाना॑मनसः॒ खलु॒ खलु॒ नाना॑मनसो॒ नाना॑मनसः॒ खलु॑ । \newline
52. नाना॑मनस॒ इति॒ नाना᳚ - म॒न॒सः॒ । \newline
53. खलु॒ वै वै खलु॒ खलु॒ वै । \newline
54. वै प॒शवः॑ प॒शवो॒ वै वै प॒शवः॑ । \newline
55. प॒शवो॒ नाना᳚व्रता॒ नाना᳚व्रताः प॒शवः॑ प॒शवो॒ नाना᳚व्रताः । \newline
56. नाना᳚व्रता॒ स्ते ते नाना᳚व्रता॒ नाना᳚व्रता॒ स्ते । \newline
57. नाना᳚व्रता॒ इति॒ नाना᳚ - व्र॒ताः॒ । \newline
58. ते᳚(1॒) ऽपो॑ ऽप स्ते ते॑ ऽपः । \newline
59. अ॒प ए॒वैवापो॑ ऽप ए॒व । \newline
60. ए॒वाभ्या᳚(1॒)भ्ये॑ वैवाभि । \newline
61. अ॒भि सम॑नसः॒ सम॑नसो॒ ऽभ्य॑भि सम॑नसः । \newline
62. सम॑नसो॒ यं ॅयꣳ सम॑नसः॒ सम॑नसो॒ यम् । \newline
63. सम॑नस॒ इति॒ स - म॒न॒सः॒ । \newline

\textbf{Ghana Paata } \newline

1. उ॒प॒दधा॑ति॒ तस्मा॒त् तस्मा॑ दुप॒दधा᳚ त्युप॒दधा॑ति॒ तस्मा॒थ् सर्वा॒न् थ्सर्वा॒न् तस्मा॑ दुप॒दधा᳚ त्युप॒दधा॑ति॒ तस्मा॒थ् सर्वान्॑ । \newline
2. उ॒प॒दधा॒तीत्यु॑प - दधा॑ति । \newline
3. तस्मा॒थ् सर्वा॒न् थ्सर्वा॒न् तस्मा॒त् तस्मा॒थ् सर्वा॑ नृ॒तू नृ॒तून् थ्सर्वा॒न् तस्मा॒त् तस्मा॒थ् सर्वा॑ नृ॒तून् । \newline
4. सर्वा॑ नृ॒तू नृ॒तून् थ्सर्वा॒न् थ्सर्वा॑ नृ॒तू नन्वन् वृ॒तून् थ्सर्वा॒न् थ्सर्वा॑ नृ॒तू ननु॑ । \newline
5. ऋ॒तू नन्वन् वृ॒तू नृ॒तू ननु॑ वा॒युर् वा॒यु रन्वृ॒तू नृ॒तू ननु॑ वा॒युः । \newline
6. अनु॑ वा॒युर् वा॒यु रन्वनु॑ वा॒युरा वा॒यु रन्वनु॑ वा॒युरा । \newline
7. वा॒युरा वा॒युर् वा॒युरा व॑रीवर्ति वरीव॒र्त्या वा॒युर् वा॒युरा व॑रीवर्ति । \newline
8. आ व॑रीवर्ति वरीव॒र्त्या व॑रीवर्ति वृष्टि॒सनी᳚र् वृष्टि॒सनी᳚र् वरीव॒र्त्या व॑रीवर्ति वृष्टि॒सनीः᳚ । \newline
9. व॒री॒व॒र्ति॒ वृ॒ष्टि॒सनी᳚र् वृष्टि॒सनी᳚र् वरीवर्ति वरीवर्ति वृष्टि॒सनी॒ रुपोप॑ वृष्टि॒सनी᳚र् वरीवर्ति वरीवर्ति वृष्टि॒सनी॒ रुप॑ । \newline
10. वृ॒ष्टि॒सनी॒ रुपोप॑ वृष्टि॒सनी᳚र् वृष्टि॒सनी॒ रुप॑ दधाति दधा॒ त्युप॑ वृष्टि॒सनी᳚र् वृष्टि॒सनी॒ रुप॑ दधाति । \newline
11. वृ॒ष्टि॒सनी॒रिति॑ वृष्टि - सनीः᳚ । \newline
12. उप॑ दधाति दधा॒ त्युपोप॑ दधाति॒ वृष्टिं॒ ॅवृष्टि॑म् दधा॒ त्युपोप॑ दधाति॒ वृष्टि᳚म् । \newline
13. द॒धा॒ति॒ वृष्टिं॒ ॅवृष्टि॑म् दधाति दधाति॒ वृष्टि॑ मे॒वैव वृष्टि॑म् दधाति दधाति॒ वृष्टि॑ मे॒व । \newline
14. वृष्टि॑ मे॒वैव वृष्टिं॒ ॅवृष्टि॑ मे॒वावा वै॒व वृष्टिं॒ ॅवृष्टि॑ मे॒वाव॑ । \newline
15. ए॒वावा वै॒वै वाव॑ रुन्धे रु॒न्धे ऽवै॒वै वाव॑ रुन्धे । \newline
16. अव॑ रुन्धे रु॒न्धे ऽवाव॑ रुन्धे॒ यद् यद् रु॒न्धे ऽवाव॑ रुन्धे॒ यत् । \newline
17. रु॒न्धे॒ यद् यद् रु॑न्धे रुन्धे॒ यदे॑क॒ धैक॒धा यद् रु॑न्धे रुन्धे॒ यदे॑क॒धा । \newline
18. यदे॑क॒ धैक॒धा यद् यदे॑क॒धो प॑द॒द्ध्या दु॑पद॒द्ध्या दे॑क॒धा यद् यदे॑क॒ धोप॑द॒द्ध्यात् । \newline
19. ए॒क॒ धोप॑द॒द्ध्या दु॑पद॒द्ध्या दे॑क॒ धैक॒ धोप॑द॒द्ध्या देक॒ मेक॑ मुपद॒द्ध्या दे॑क॒
धैक॒ धोप॑द॒द्ध्या देक᳚म् । \newline
20. ए॒क॒धेत्ये॑क - धा । \newline
21. उ॒प॒द॒द्ध्या देक॒ मेक॑ मुपद॒द्ध्या दु॑पद॒द्ध्या देक॑ मृ॒तु मृ॒तु मेक॑ मुपद॒द्ध्या दु॑पद॒द्ध्या देक॑ मृ॒तुम् । \newline
22. उ॒प॒द॒द्ध्यादित्यु॑प - द॒द्ध्यात् । \newline
23. एक॑ मृ॒तु मृ॒तु मेक॒ मेक॑ मृ॒तुं ॅव॑र्.षेद् वर्.षे दृ॒तु मेक॒ मेक॑ मृ॒तुं ॅव॑र्.षेत् । \newline
24. ऋ॒तुं ॅव॑र्.षेद् वर्.षे दृ॒तु मृ॒तुं ॅव॑र्.षे दनुपरि॒हार॑ मनुपरि॒हारं॑ ॅवर्.षेदृ॒तु मृ॒तुं ॅव॑र्.षे दनुपरि॒हार᳚म् । \newline
25. व॒र्॒.षे॒ द॒नु॒प॒रि॒हार॑ मनुपरि॒हारं॑ ॅवर्.षेद् वर्.षे दनुपरि॒हारꣳ॑ सादयति सादयत्य नुपरि॒हारं॑ ॅवर्.षेद् वर्.षे दनुपरि॒हारꣳ॑ सादयति । \newline
26. अ॒नु॒प॒रि॒हारꣳ॑ सादयति सादय त्यनुपरि॒हार॑ मनुपरि॒हारꣳ॑ सादयति॒ तस्मा॒त् तस्मा᳚थ् 
सादय त्यनुपरि॒हार॑ मनुपरि॒हारꣳ॑ सादयति॒ तस्मा᳚त् । \newline
27. अ॒नु॒प॒रि॒हार॒मित्य॑नु - प॒रि॒हार᳚म् । \newline
28. सा॒द॒य॒ति॒ तस्मा॒त् तस्मा᳚थ् सादयति सादयति॒ तस्मा॒थ् सर्वा॒न् थ्सर्वा॒न् तस्मा᳚थ् सादयति सादयति॒ तस्मा॒थ् सर्वान्॑ । \newline
29. तस्मा॒थ् सर्वा॒न् थ्सर्वा॒न् तस्मा॒त् तस्मा॒थ् सर्वा॑ नृ॒तू नृ॒तून् थ्सर्वा॒न् तस्मा॒त् तस्मा॒थ् सर्वा॑ नृ॒तून् । \newline
30. सर्वा॑ नृ॒तू नृ॒तून् थ्सर्वा॒न् थ्सर्वा॑ नृ॒तून्. व॑र्.षति वर्.ष त्यृ॒तून् थ्सर्वा॒न् थ्सर्वा॑ नृ॒तून्. व॑र्.षति । \newline
31. ऋ॒तून्. व॑र्.षति वर्.ष त्यृ॒तू नृ॒तून्. व॑र्.षति॒ यद् यद् व॑र्.ष त्यृ॒तू नृ॒तून्. व॑र्.षति॒ यत् । \newline
32. व॒र्॒.ष॒ति॒ यद् यद् व॑र्.षति वर्.षति॒ यत् प्रा॑ण॒भृतः॑ प्राण॒भृतो॒ यद् व॑र्.षति वर्.षति॒ यत् प्रा॑ण॒भृतः॑ । \newline
33. यत् प्रा॑ण॒भृतः॑ प्राण॒भृतो॒ यद् यत् प्रा॑ण॒भृत॑ उप॒धा यो॑प॒धाय॑ प्राण॒भृतो॒ यद् यत् प्रा॑ण॒भृत॑ उप॒धाय॑ । \newline
34. प्रा॒ण॒भृत॑ उप॒धा यो॑प॒धाय॑ प्राण॒भृतः॑ प्राण॒भृत॑ उप॒धाय॑ वृष्टि॒सनी᳚र् वृष्टि॒सनी॑ रुप॒धाय॑ प्राण॒भृतः॑ प्राण॒भृत॑ उप॒धाय॑ वृष्टि॒सनीः᳚ । \newline
35. प्रा॒ण॒भृत॒ इति॑ प्राण - भृतः॑ । \newline
36. उ॒प॒धाय॑ वृष्टि॒सनी᳚र् वृष्टि॒सनी॑ रुप॒धा यो॑प॒धाय॑ वृष्टि॒सनी॑ रुप॒दधा᳚ त्युप॒दधा॑ति वृष्टि॒सनी॑ रुप॒धा यो॑प॒धाय॑ वृष्टि॒सनी॑ रुप॒दधा॑ति । \newline
37. उ॒प॒धायेत्यु॑प - धाय॑ । \newline
38. वृ॒ष्टि॒सनी॑ रुप॒दधा᳚ त्युप॒दधा॑ति वृष्टि॒सनी᳚र् वृष्टि॒सनी॑ रुप॒दधा॑ति॒ तस्मा॒त् तस्मा॑ दुप॒दधा॑ति वृष्टि॒सनी᳚र् वृष्टि॒सनी॑ रुप॒दधा॑ति॒ तस्मा᳚त् । \newline
39. वृ॒ष्टि॒सनी॒रिति॑ वृष्टि - सनीः᳚ । \newline
40. उ॒प॒दधा॑ति॒ तस्मा॒त् तस्मा॑ दुप॒दधा᳚ त्युप॒दधा॑ति॒ तस्मा᳚द् वा॒युप्र॑च्युता वा॒युप्र॑च्युता॒ तस्मा॑ दुप॒दधा᳚ त्युप॒दधा॑ति॒ तस्मा᳚द् वा॒युप्र॑च्युता । \newline
41. उ॒प॒दधा॒तीत्यु॑प - दधा॑ति । \newline
42. तस्मा᳚द् वा॒युप्र॑च्युता वा॒युप्र॑च्युता॒ तस्मा॒त् तस्मा᳚द् वा॒युप्र॑च्युता दि॒वो दि॒वो वा॒युप्र॑च्युता॒ तस्मा॒त् तस्मा᳚द् वा॒युप्र॑च्युता दि॒वः । \newline
43. वा॒युप्र॑च्युता दि॒वो दि॒वो वा॒युप्र॑च्युता वा॒युप्र॑च्युता दि॒वो वृष्टि॒र् वृष्टि॑र् दि॒वो वा॒युप्र॑च्युता वा॒युप्र॑च्युता दि॒वो वृष्टिः॑ । \newline
44. वा॒युप्र॑च्यु॒तेति॑ वा॒यु - प्र॒च्यु॒ता॒ । \newline
45. दि॒वो वृष्टि॒र् वृष्टि॑र् दि॒वो दि॒वो वृष्टि॑ रीर्त ईर्ते॒ वृष्टि॑र् दि॒वो दि॒वो वृष्टि॑ रीर्ते । \newline
46. वृष्टि॑ रीर्त ईर्ते॒ वृष्टि॒र् वृष्टि॑ रीर्ते प॒शवः॑ प॒शव॑ ईर्ते॒ वृष्टि॒र् वृष्टि॑ रीर्ते प॒शवः॑ । \newline
47. ई॒र्ते॒ प॒शवः॑ प॒शव॑ ईर्त ईर्ते प॒शवो॒ वै वै प॒शव॑ ईर्त ईर्ते प॒शवो॒ वै । \newline
48. प॒शवो॒ वै वै प॒शवः॑ प॒शवो॒ वै व॑य॒स्या॑ वय॒स्या॑ वै प॒शवः॑ प॒शवो॒ वै व॑य॒स्याः᳚ । \newline
49. वै व॑य॒स्या॑ वय॒स्या॑ वै वै व॑य॒स्या॑ नाना॑मनसो॒ नाना॑मनसो वय॒स्या॑ वै वै व॑य॒स्या॑ नाना॑मनसः । \newline
50. व॒य॒स्या॑ नाना॑मनसो॒ नाना॑मनसो वय॒स्या॑ वय॒स्या॑ नाना॑मनसः॒ खलु॒ खलु॒ नाना॑मनसो वय॒स्या॑ वय॒स्या॑ नाना॑मनसः॒ खलु॑ । \newline
51. नाना॑मनसः॒ खलु॒ खलु॒ नाना॑मनसो॒ नाना॑मनसः॒ खलु॒ वै वै खलु॒ नाना॑मनसो॒ नाना॑मनसः॒ खलु॒ वै । \newline
52. नाना॑मनस॒ इति॒ नाना᳚ - म॒न॒सः॒ । \newline
53. खलु॒ वै वै खलु॒ खलु॒ वै प॒शवः॑ प॒शवो॒ वै खलु॒ खलु॒ वै प॒शवः॑ । \newline
54. वै प॒शवः॑ प॒शवो॒ वै वै प॒शवो॒ नाना᳚व्रता॒ नाना᳚व्रताः प॒शवो॒ वै वै प॒शवो॒ नाना᳚व्रताः । \newline
55. प॒शवो॒ नाना᳚व्रता॒ नाना᳚व्रताः प॒शवः॑ प॒शवो॒ नाना᳚व्रता॒ स्ते ते नाना᳚व्रताः प॒शवः॑ प॒शवो॒ नाना᳚व्रता॒ स्ते । \newline
56. नाना᳚व्रता॒ स्ते ते नाना᳚व्रता॒ नाना᳚व्रता॒ स्ते᳚(1॒) ऽपो॑ ऽपस्ते नाना᳚व्रता॒ नाना᳚व्रता॒ स्ते॑ ऽपः । \newline
57. नाना᳚व्रता॒ इति॒ नाना᳚ - व्र॒ताः॒ । \newline
58. ते᳚(1॒) ऽपो॑ ऽप स्ते ते॑ ऽप ए॒वै वाप स्ते ते॑ ऽप ए॒व । \newline
59. अ॒प ए॒वै वापो॑ ऽप ए॒वाभ्या᳚(1॒)भ्ये॑ वापो॑ ऽप ए॒वाभि । \newline
60. ए॒वाभ्या᳚(1॒)भ्ये॑ वैवाभि सम॑नसः॒ सम॑नसो॒ ऽभ्ये॑ वैवाभि सम॑नसः । \newline
61. अ॒भि सम॑नसः॒ सम॑नसो॒ ऽभ्य॑भि सम॑नसो॒ यं ॅयꣳ सम॑नसो॒ ऽभ्य॑भि सम॑नसो॒ यम् । \newline
62. सम॑नसो॒ यं ॅयꣳ सम॑नसः॒ सम॑नसो॒ यम् का॒मये॑त का॒मये॑त॒ यꣳ सम॑नसः॒ सम॑नसो॒ यम् का॒मये॑त । \newline
63. सम॑नस॒ इति॒ स - म॒न॒सः॒ । \newline
\pagebreak
\markright{ TS 5.3.1.4  \hfill https://www.vedavms.in \hfill}

\section{ TS 5.3.1.4 }

\textbf{TS 5.3.1.4 } \newline
\textbf{Samhita Paata} \newline

यं का॒मये॑ताप॒शुः स्या॒दिति॑ वय॒स्या᳚स्तस्यो॑-प॒धाया॑प॒स्या॑ उप॑ दद्ध्या॒द स᳚ज्ञांनमे॒वास्मै॑ प॒शुभिः॑ करोत्यप॒शुरे॒व भ॑वति॒ यं का॒मये॑त पशु॒मान्थ्-स्या॒दित्य॑-प॒स्या᳚स्तस्यो॑प॒धाय॑ वय॒स्या॑ उप॑ दद्ध्याथ् स॒ज्ञांन॑मे॒वास्मै॑ प॒शुभिः॑ करोति पशु॒माने॒व भ॑वति॒ चत॑स्रः पु॒रस्ता॒दुप॑ दधाति॒ तस्मा᳚च्च॒त्वारि॒ चक्षु॑षो रू॒पाणि॒ द्वे शु॒क्ले द्वे कृ॒ष्णे - [  ] \newline

\textbf{Pada Paata} \newline

यम् । का॒मये॑त । अ॒प॒शुः । स्या॒त् । इति॑ । व॒य॒स्याः᳚ । तस्य॑ । उ॒प॒धायेत्यु॑प - धाय॑ । अ॒प॒स्याः᳚ । उपेति॑ । द॒द्ध्या॒त् । अस᳚ज्ञांन॒मित्यसं᳚ - ज्ञा॒न॒म् । ए॒व । अ॒स्मै॒ । प॒शुभि॒रिति॑ प॒शु-भिः॒ । क॒रो॒ति॒ । अ॒प॒शुः । ए॒व । भ॒व॒ति॒ । यम् । का॒मये॑त । प॒शु॒मानिति॑ पशु - मान् । स्या॒त् । इति॑ । अ॒प॒स्याः᳚ । तस्य॑ । उ॒प॒धायेत्यु॑प - धाय॑ । व॒य॒स्याः᳚ । उपेति॑ । द॒द्ध्या॒त् । स॒ज्ञांन॒मिति॑ सं - ज्ञान᳚म् । ए॒व । अ॒स्मै॒ । प॒शुभि॒रिति॑ प॒शु - भिः॒ । क॒रो॒ति॒ । प॒शु॒मानिति॑ पशु - मान् । ए॒व । भ॒व॒ति॒ । चत॑स्रः । पु॒रस्ता᳚त् । उपेति॑ । द॒धा॒ति॒ । तस्मा᳚त् । च॒त्वारि॑ । चक्षु॑षः । रू॒पाणि॑ । द्वे इति॑ । शु॒क्ले इति॑ । द्वे इति॑ । कृ॒ष्णे इति॑ ।  \newline


\textbf{Krama Paata} \newline

यम् का॒मये॑त । का॒मये॑ताप॒शुः । अ॒प॒शुः स्या᳚त् । स्या॒दिति॑ । इति॑ वय॒स्याः᳚ । व॒य॒स्या᳚स्तस्य॑ । तस्यो॑प॒धाय॑ । उ॒प॒धाया॑प॒स्याः᳚ । उ॒प॒धायेत्यु॑प - धाय॑ । अ॒प॒स्या॑ उप॑ । उप॑ दद्ध्यात् । द॒द्ध्या॒दस᳚म्(2)ज्ञानम् । अस᳚म्(2)ज्ञानमे॒व । अस᳚म्(2)ज्ञान॒मित्यस᳚म् - ज्ञा॒न॒म् । ए॒वास्मै᳚ । अ॒स्मै॒ प॒शुभिः॑ । प॒शुभिः॑ करोति । प॒शुभि॒रिति॑ प॒शु - भिः॒ । क॒रो॒त्य॒प॒शुः । अ॒प॒शुरे॒व । ए॒व भ॑वति । भ॒व॒ति॒ यम् । यम् का॒मये॑त । का॒मये॑त पशु॒मान् । प॒शु॒मान्थ् स्या᳚त् । प॒शु॒मानिति॑ पशु - मान् । स्या॒दिति॑ । इत्य॑प॒स्याः᳚ । अ॒प॒स्या᳚स्तस्य॑ । तस्यो॑प॒धाय॑ । उ॒प॒धाय॑ व॒यस्याः᳚ । उ॒प॒धायेत्यु॑प - धाय॑ । व॒य॒स्या॑ उप॑ । उप॑ दद्ध्यात् । द॒द्ध्या॒थ् स॒म्(2)ज्ञान᳚म् । स॒म्(2)ज्ञान॑मे॒व । स॒म्(2)ज्ञान॒मिति॑ सम् - ज्ञान᳚म् । ए॒वास्मै᳚ । अ॒स्मै॒ प॒शुभिः॑ । प॒शुभिः॑ करोति । प॒शुभि॒रिति॑ प॒शु - भिः॒ । क॒रो॒ति॒ प॒शु॒मान् । प॒शु॒माने॒व । प॒शु॒मानिति॑ पशु - मान् । ए॒व भ॑वति । भ॒व॒ति॒ चत॑स्रः । चत॑स्रः पु॒रस्ता᳚त् । पु॒रस्ता॒दुप॑ । उप॑ दधाति । द॒धा॒ति॒ तस्मा᳚त् । तस्मा᳚च्च॒त्वारि॑ । च॒त्वारि॒ चक्षु॑षः । चक्षु॑षो रू॒पाणि॑ । रू॒पाणि॒ द्वे । द्वे शु॒क्ले । द्वे इति॒ द्वे । शु॒क्ले द्वे । शु॒क्ले इति॑ शु॒क्ले । द्वे कृ॒ष्णे । द्वे इति॒ द्वे । कृ॒ष्णे मू᳚र्द्ध॒न्वतीः᳚ । कृ॒ष्णे इति॑ कृ॒ष्णे \newline

\textbf{Jatai Paata} \newline

1. यम् का॒मये॑त का॒मये॑त॒ यं ॅयम् का॒मये॑त । \newline
2. का॒मये॑ता प॒शु र॑प॒शुः का॒मये॑त का॒मये॑ता प॒शुः । \newline
3. अ॒प॒शुः स्या᳚थ् स्या दप॒शु र॑प॒शुः स्या᳚त् । \newline
4. स्या॒ दितीति॑ स्याथ् स्या॒ दिति॑ । \newline
5. इति॑ वय॒स्या॑ वय॒स्या॑ इतीति॑ वय॒स्याः᳚ । \newline
6. व॒य॒स्या᳚ स्तस्य॒ तस्य॑ वय॒स्या॑ वय॒स्या᳚ स्तस्य॑ । \newline
7. तस्यो॑प॒धा यो॑प॒धाय॒ तस्य॒ तस्यो॑प॒धाय॑ । \newline
8. उ॒प॒धाया॑ प॒स्या॑ अप॒स्या॑ उप॒धा यो॑प॒धाया॑ प॒स्याः᳚ । \newline
9. उ॒प॒धायेत्यु॑प - धाय॑ । \newline
10. अ॒प॒स्या॑ उपोपा॑ प॒स्या॑ अप॒स्या॑ उप॑ । \newline
11. उप॑ दद्ध्याद् दद्ध्या॒ दुपोप॑ दद्ध्यात् । \newline
12. द॒द्ध्या॒ दसं᳚.ज्ञान॒ मसं᳚.ज्ञानम् दद्ध्याद् दद्ध्या॒ दसं᳚.ज्ञानम् । \newline
13. असं᳚.ज्ञान मे॒वै वासं᳚.ज्ञान॒ मसं᳚.ज्ञान मे॒व । \newline
14. असं᳚.ज्ञान॒मित्यसं᳚ - ज्ञा॒न॒म् । \newline
15. ए॒वास्मा॑ अस्मा ए॒वै वास्मै᳚ । \newline
16. अ॒स्मै॒ प॒शुभिः॑ प॒शुभि॑ रस्मा अस्मै प॒शुभिः॑ । \newline
17. प॒शुभिः॑ करोति करोति प॒शुभिः॑ प॒शुभिः॑ करोति । \newline
18. प॒शुभि॒रिति॑ प॒शु - भिः॒ । \newline
19. क॒रो॒ त्य॒प॒शु र॑प॒शुः क॑रोति करो त्यप॒शुः । \newline
20. अ॒प॒शु रे॒वैवा प॒शुर॑ प॒शु रे॒व । \newline
21. ए॒व भ॑वति भव त्ये॒वैव भ॑वति । \newline
22. भ॒व॒ति॒ यं ॅयम् भ॑वति भवति॒ यम् । \newline
23. यम् का॒मये॑त का॒मये॑त॒ यं ॅयम् का॒मये॑त । \newline
24. का॒मये॑त पशु॒मान् प॑शु॒मान् का॒मये॑त का॒मये॑त पशु॒मान् । \newline
25. प॒शु॒मान् थ्स्या᳚थ् स्यात् पशु॒मान् प॑शु॒मान् थ्स्या᳚त् । \newline
26. प॒शु॒मानिति॑ पशु - मान् । \newline
27. स्या॒ दितीति॑ स्याथ् स्या॒ दिति॑ । \newline
28. इत्य॑प॒स्या॑ अप॒स्या॑ इतीत्य॑ प॒स्याः᳚ । \newline
29. अ॒प॒स्या᳚ स्तस्य॒ तस्या॑ प॒स्या॑ अप॒स्या᳚ स्तस्य॑ । \newline
30. तस्यो॑प॒धा यो॑प॒धाय॒ तस्य॒ तस्यो॑प॒धाय॑ । \newline
31. उ॒प॒धाय॑ वय॒स्या॑ वय॒स्या॑ उप॒धा यो॑प॒धाय॑ वय॒स्याः᳚ । \newline
32. उ॒प॒धायेत्यु॑प - धाय॑ । \newline
33. व॒य॒स्या॑ उपोप॑ वय॒स्या॑ वय॒स्या॑ उप॑ । \newline
34. उप॑ दद्ध्याद् दद्ध्या॒ दुपोप॑ दद्ध्यात् । \newline
35. द॒द्ध्या॒थ् सं॒.ज्ञानꣳ॑ सं॒.ज्ञान॑म् दद्ध्याद् दद्ध्याथ् सं॒.ज्ञान᳚म् । \newline
36. सं॒.ज्ञान॑ मे॒वैव सं॒.ज्ञानꣳ॑ सं॒.ज्ञान॑ मे॒व । \newline
37. सं॒.ज्ञान॒मिति॑ सं - ज्ञान᳚म् । \newline
38. ए॒वास्मा॑ अस्मा ए॒वै वास्मै᳚ । \newline
39. अ॒स्मै॒ प॒शुभिः॑ प॒शुभि॑ रस्मा अस्मै प॒शुभिः॑ । \newline
40. प॒शुभिः॑ करोति करोति प॒शुभिः॑ प॒शुभिः॑ करोति । \newline
41. प॒शुभि॒रिति॑ प॒शु - भिः॒ । \newline
42. क॒रो॒ति॒ प॒शु॒मान् प॑शु॒मान् क॑रोति करोति पशु॒मान् । \newline
43. प॒शु॒मा ने॒वैव प॑शु॒मान् प॑शु॒मा ने॒व । \newline
44. प॒शु॒मानिति॑ पशु - मान् । \newline
45. ए॒व भ॑वति भव त्ये॒वैव भ॑वति । \newline
46. भ॒व॒ति॒ चत॑स्र॒ श्चत॑स्रो भवति भवति॒ चत॑स्रः । \newline
47. चत॑स्रः पु॒रस्ता᳚त् पु॒रस्ता॒च् चत॑स्र॒ श्चत॑स्रः पु॒रस्ता᳚त् । \newline
48. पु॒रस्ता॒ दुपोप॑ पु॒रस्ता᳚त् पु॒रस्ता॒ दुप॑ । \newline
49. उप॑ दधाति दधा॒ त्युपोप॑ दधाति । \newline
50. द॒धा॒ति॒ तस्मा॒त् तस्मा᳚द् दधाति दधाति॒ तस्मा᳚त् । \newline
51. तस्मा᳚च् च॒त्वारि॑ च॒त्वारि॒ तस्मा॒त् तस्मा᳚च् च॒त्वारि॑ । \newline
52. च॒त्वारि॒ चक्षु॑ष॒ श्चक्षु॑ष श्च॒त्वारि॑ च॒त्वारि॒ चक्षु॑षः । \newline
53. चक्षु॑षो रू॒पाणि॑ रू॒पाणि॒ चक्षु॑ष॒ श्चक्षु॑षो रू॒पाणि॑ । \newline
54. रू॒पाणि॒ द्वे द्वे रू॒पाणि॑ रू॒पाणि॒ द्वे । \newline
55. द्वे शु॒क्ले शु॒क्ले द्वे द्वे शु॒क्ले । \newline
56. द्वे इति॒ द्वे । \newline
57. शु॒क्ले द्वे द्वे शु॒क्ले शु॒क्ले द्वे । \newline
58. शु॒क्ले इति॑ शु॒क्ले । \newline
59. द्वे कृ॒ष्णे कृ॒ष्णे द्वे द्वे कृ॒ष्णे । \newline
60. द्वे इति॒ द्वे । \newline
61. कृ॒ष्णे मू᳚र्द्ध॒न्वती᳚र् मूर्द्ध॒न्वतीः᳚ कृ॒ष्णे कृ॒ष्णे मू᳚र्द्ध॒न्वतीः᳚ । \newline
62. कृ॒ष्णे इति॑ कृ॒ष्णे । \newline

\textbf{Ghana Paata } \newline

1. यम् का॒मये॑त का॒मये॑त॒ यं ॅयम् का॒मये॑ता प॒शु र॑प॒शुः का॒मये॑त॒ यं ॅयम् का॒मये॑ता प॒शुः । \newline
2. का॒मये॑ता प॒शु र॑प॒शुः का॒मये॑त का॒मये॑ता प॒शुः स्या᳚थ् स्या दप॒शुः का॒मये॑त का॒मये॑ता प॒शुः स्या᳚त् । \newline
3. अ॒प॒शुः स्या᳚थ् स्या दप॒शु र॑प॒शुः स्या॒ दितीति॑ स्या दप॒शु र॑प॒शुः स्या॒ दिति॑ । \newline
4. स्या॒ दितीति॑ स्याथ् स्या॒ दिति॑ वय॒स्या॑ वय॒स्या॑ इति॑ स्याथ् स्या॒ दिति॑ वय॒स्याः᳚ । \newline
5. इति॑ वय॒स्या॑ वय॒स्या॑ इतीति॑ वय॒स्या᳚ स्तस्य॒ तस्य॑ वय॒स्या॑ इतीति॑ वय॒स्या᳚ स्तस्य॑ । \newline
6. व॒य॒स्या᳚ स्तस्य॒ तस्य॑ वय॒स्या॑ वय॒स्या᳚ स्तस्यो॑प॒धा यो॑प॒धाय॒ तस्य॑ वय॒स्या॑ वय॒स्या᳚ स्तस्यो॑ प॒धाय॑ । \newline
7. तस्यो॑ प॒धायो॑ प॒धाय॒ तस्य॒ तस्यो॑प॒धाया॑ प॒स्या॑ अप॒स्या॑ उप॒धाय॒ तस्य॒ तस्यो॑ प॒धाया॑ प॒स्याः᳚ । \newline
8. उ॒प॒धाया॑ प॒स्या॑ अप॒स्या॑ उप॒धा यो॑प॒धाया॑ प॒स्या॑ उपोपा॑ प॒स्या॑ उप॒धा यो॑प॒धाया॑ प॒स्या॑ उप॑ । \newline
9. उ॒प॒धायेत्यु॑प - धाय॑ । \newline
10. अ॒प॒स्या॑ उपोपा॑ प॒स्या॑ अप॒स्या॑ उप॑ दद्ध्याद् दद्ध्या॒ दुपा॑प॒स्या॑ अप॒स्या॑ उप॑ दद्ध्यात् । \newline
11. उप॑ दद्ध्याद् दद्ध्या॒ दुपोप॑ दद्ध्या॒ दसं᳚.ज्ञान॒ मसं᳚.ज्ञानम् दद्ध्या॒ दुपोप॑ दद्ध्या॒ 
दसं᳚.ज्ञानम् । \newline
12. द॒द्ध्या॒ दसं᳚.ज्ञान॒ मसं᳚.ज्ञानम् दद्ध्याद् दद्ध्या॒ दसं᳚.ज्ञान मे॒वै वासं᳚.ज्ञानम् दद्ध्याद् दद्ध्या॒ दसं᳚.ज्ञान मे॒व । \newline
13. असं᳚.ज्ञान मे॒वै वासं᳚.ज्ञान॒ मसं᳚.ज्ञान मे॒वास्मा॑ अस्मा ए॒वासं᳚.ज्ञान॒ मसं᳚.ज्ञान मे॒वास्मै᳚ । \newline
14. असं᳚.ज्ञान॒मित्यसं᳚ - ज्ञा॒न॒म् । \newline
15. ए॒वास्मा॑ अस्मा ए॒वै वास्मै॑ प॒शुभिः॑ प॒शुभि॑ रस्मा ए॒वै वास्मै॑ प॒शुभिः॑ । \newline
16. अ॒स्मै॒ प॒शुभिः॑ प॒शुभि॑ रस्मा अस्मै प॒शुभिः॑ करोति करोति प॒शुभि॑ रस्मा अस्मै प॒शुभिः॑ करोति । \newline
17. प॒शुभिः॑ करोति करोति प॒शुभिः॑ प॒शुभिः॑ करो त्यप॒शु र॑प॒शुः क॑रोति प॒शुभिः॑ प॒शुभिः॑ करोत्यप॒शुः । \newline
18. प॒शुभि॒रिति॑ प॒शु - भिः॒ । \newline
19. क॒रो॒ त्य॒प॒शु र॑प॒शुः क॑रोति करो त्यप॒शु रे॒वैवा प॒शुः क॑रोति करो त्यप॒शु रे॒व । \newline
20. अ॒प॒शु रे॒वैवा प॒शु र॑प॒शु रे॒व भ॑वति भव त्ये॒वा प॒शु र॑प॒शु रे॒व भ॑वति । \newline
21. ए॒व भ॑वति भव त्ये॒वैव भ॑वति॒ यं ॅयम् भ॑व त्ये॒वैव भ॑वति॒ यम् । \newline
22. भ॒व॒ति॒ यं ॅयम् भ॑वति भवति॒ यम् का॒मये॑त का॒मये॑त॒ यम् भ॑वति भवति॒ यम् का॒मये॑त । \newline
23. यम् का॒मये॑त का॒मये॑त॒ यं ॅयम् का॒मये॑त पशु॒मान् प॑शु॒मान् का॒मये॑त॒ यं ॅयम् का॒मये॑त पशु॒मान् । \newline
24. का॒मये॑त पशु॒मान् प॑शु॒मान् का॒मये॑त का॒मये॑त पशु॒मान् थ्स्या᳚थ् स्यात् पशु॒मान् का॒मये॑त का॒मये॑त पशु॒मान् थ्स्या᳚त् । \newline
25. प॒शु॒मान् थ्स्या᳚थ् स्यात् पशु॒मान् प॑शु॒मान् थ्स्या॒ दितीति॑ स्यात् पशु॒मान् प॑शु॒मान् थ्स्या॒ दिति॑ । \newline
26. प॒शु॒मानिति॑ पशु - मान् । \newline
27. स्या॒ दितीति॑ स्याथ् स्या॒ दित्य॑प॒स्या॑ अप॒स्या॑ इति॑ स्याथ् स्या॒ दित्य॑प॒स्याः᳚ । \newline
28. इत्य॑प॒स्या॑ अप॒स्या॑ इतीत्य॑प॒स्या᳚ स्तस्य॒ तस्या॑ प॒स्या॑ इतीत्य॑ प॒स्या᳚ स्तस्य॑ । \newline
29. अ॒प॒स्या᳚ स्तस्य॒ तस्या॑ प॒स्या॑ अप॒स्या᳚ स्तस्यो॑ प॒धायो॑ प॒धाय॒ तस्या॑ प॒स्या॑ अप॒स्या᳚ स्तस्यो॑ प॒धाय॑ । \newline
30. तस्यो॑ प॒धायो॑ प॒धाय॒ तस्य॒ तस्यो॑ प॒धाय॑ वय॒स्या॑ वय॒स्या॑ उप॒धाय॒ तस्य॒ तस्यो॑ प॒धाय॑ वय॒स्याः᳚ । \newline
31. उ॒प॒धाय॑ वय॒स्या॑ वय॒स्या॑ उप॒धायो॑ प॒धाय॑ वय॒स्या॑ उपोप॑ वय॒स्या॑ उप॒धायो॑ प॒धाय॑ वय॒स्या॑ उप॑ । \newline
32. उ॒प॒धायेत्यु॑प - धाय॑ । \newline
33. व॒य॒स्या॑ उपोप॑ वय॒स्या॑ वय॒स्या॑ उप॑ दद्ध्याद् दद्ध्या॒ दुप॑ वय॒स्या॑ वय॒स्या॑ उप॑ दद्ध्यात् । \newline
34. उप॑ दद्ध्याद् दद्ध्या॒ दुपोप॑ दद्ध्याथ् सं॒.ज्ञानꣳ॑ सं॒.ज्ञान॑म् दद्ध्या॒ दुपोप॑ दद्ध्याथ् सं॒.ज्ञान᳚म् । \newline
35. द॒द्ध्या॒थ् सं॒.ज्ञानꣳ॑ सं॒.ज्ञान॑म् दद्ध्याद् दद्ध्याथ् सं॒.ज्ञान॑ मे॒वैव सं॒.ज्ञान॑म् दद्ध्याद् दद्ध्याथ् सं॒.ज्ञान॑ मे॒व । \newline
36. सं॒.ज्ञान॑ मे॒वैव सं॒.ज्ञानꣳ॑ सं॒.ज्ञान॑ मे॒वास्मा॑ अस्मा ए॒व सं॒.ज्ञानꣳ॑ सं॒.ज्ञान॑ मे॒वास्मै᳚ । \newline
37. सं॒.ज्ञान॒मिति॑ सं - ज्ञान᳚म् । \newline
38. ए॒वास्मा॑ अस्मा ए॒वै वास्मै॑ प॒शुभिः॑ प॒शुभि॑ रस्मा ए॒वै वास्मै॑ प॒शुभिः॑ । \newline
39. अ॒स्मै॒ प॒शुभिः॑ प॒शुभि॑ रस्मा अस्मै प॒शुभिः॑ करोति करोति प॒शुभि॑ रस्मा अस्मै प॒शुभिः॑ करोति । \newline
40. प॒शुभिः॑ करोति करोति प॒शुभिः॑ प॒शुभिः॑ करोति पशु॒मान् प॑शु॒मान् क॑रोति प॒शुभिः॑ प॒शुभिः॑ करोति पशु॒मान् । \newline
41. प॒शुभि॒रिति॑ प॒शु - भिः॒ । \newline
42. क॒रो॒ति॒ प॒शु॒मान् प॑शु॒मान् क॑रोति करोति पशु॒मा ने॒वैव प॑शु॒मान् क॑रोति करोति पशु॒मा ने॒व । \newline
43. प॒शु॒मा ने॒वैव प॑शु॒मान् प॑शु॒मा ने॒व भ॑वति भव त्ये॒व प॑शु॒मान् प॑शु॒मा ने॒व भ॑वति । \newline
44. प॒शु॒मानिति॑ पशु - मान् । \newline
45. ए॒व भ॑वति भव त्ये॒वैव भ॑वति॒ चत॑स्र॒ श्चत॑स्रो भव त्ये॒वैव भ॑वति॒ चत॑स्रः । \newline
46. भ॒व॒ति॒ चत॑स्र॒ श्चत॑स्रो भवति भवति॒ चत॑स्रः पु॒रस्ता᳚त् पु॒रस्ता॒च् चत॑स्रो भवति भवति॒ चत॑स्रः पु॒रस्ता᳚त् । \newline
47. चत॑स्रः पु॒रस्ता᳚त् पु॒रस्ता॒च् चत॑स्र॒ श्चत॑स्रः पु॒रस्ता॒ दुपोप॑ पु॒रस्ता॒च् चत॑स्र॒ श्चत॑स्रः पु॒रस्ता॒ दुप॑ । \newline
48. पु॒रस्ता॒ दुपोप॑ पु॒रस्ता᳚त् पु॒रस्ता॒ दुप॑ दधाति दधा॒ त्युप॑ पु॒रस्ता᳚त् पु॒रस्ता॒ दुप॑ दधाति । \newline
49. उप॑ दधाति दधा॒ त्युपोप॑ दधाति॒ तस्मा॒त् तस्मा᳚द् दधा॒ त्युपोप॑ दधाति॒ तस्मा᳚त् । \newline
50. द॒धा॒ति॒ तस्मा॒त् तस्मा᳚द् दधाति दधाति॒ तस्मा᳚च् च॒त्वारि॑ च॒त्वारि॒ तस्मा᳚द् दधाति दधाति॒ तस्मा᳚च् च॒त्वारि॑ । \newline
51. तस्मा᳚च् च॒त्वारि॑ च॒त्वारि॒ तस्मा॒त् तस्मा᳚च् च॒त्वारि॒ चक्षु॑ष॒ श्चक्षु॑ष श्च॒त्वारि॒ तस्मा॒त् तस्मा᳚च् च॒त्वारि॒ चक्षु॑षः । \newline
52. च॒त्वारि॒ चक्षु॑ष॒ श्चक्षु॑ष श्च॒त्वारि॑ च॒त्वारि॒ चक्षु॑षो रू॒पाणि॑ रू॒पाणि॒ चक्षु॑ष श्च॒त्वारि॑ च॒त्वारि॒ चक्षु॑षो रू॒पाणि॑ । \newline
53. चक्षु॑षो रू॒पाणि॑ रू॒पाणि॒ चक्षु॑ष॒ श्चक्षु॑षो रू॒पाणि॒ द्वे द्वे रू॒पाणि॒ चक्षु॑ष॒ श्चक्षु॑षो रू॒पाणि॒ द्वे । \newline
54. रू॒पाणि॒ द्वे द्वे रू॒पाणि॑ रू॒पाणि॒ द्वे शु॒क्ले शु॒क्ले द्वे रू॒पाणि॑ रू॒पाणि॒ द्वे शु॒क्ले । \newline
55. द्वे शु॒क्ले शु॒क्ले द्वे द्वे शु॒क्ले द्वे द्वे शु॒क्ले द्वे द्वे शु॒क्ले द्वे । \newline
56. द्वे इति॒ द्वे । \newline
57. शु॒क्ले द्वे द्वे शु॒क्ले शु॒क्ले द्वे कृ॒ष्णे कृ॒ष्णे द्वे शु॒क्ले शु॒क्ले द्वे कृ॒ष्णे । \newline
58. शु॒क्ले इति॑ शु॒क्ले । \newline
59. द्वे कृ॒ष्णे कृ॒ष्णे द्वे द्वे कृ॒ष्णे मू᳚र्द्ध॒न्वती᳚र् मूर्द्ध॒न्वतीः᳚ कृ॒ष्णे द्वे द्वे कृ॒ष्णे मू᳚र्द्ध॒न्वतीः᳚ । \newline
60. द्वे इति॒ द्वे । \newline
61. कृ॒ष्णे मू᳚र्द्ध॒न्वती᳚र् मूर्द्ध॒न्वतीः᳚ कृ॒ष्णे कृ॒ष्णे मू᳚र्द्ध॒न्वती᳚र् भवन्ति भवन्ति मूर्द्ध॒न्वतीः᳚ कृ॒ष्णे कृ॒ष्णे मू᳚र्द्ध॒न्वती᳚र् भवन्ति । \newline
62. कृ॒ष्णे इति॑ कृ॒ष्णे । \newline
\pagebreak
\markright{ TS 5.3.1.5  \hfill https://www.vedavms.in \hfill}

\section{ TS 5.3.1.5 }

\textbf{TS 5.3.1.5 } \newline
\textbf{Samhita Paata} \newline

मू᳚र्द्ध॒न्वती᳚र्भवन्ति॒ तस्मा᳚त् पु॒रस्ता᳚न्मू॒र्द्धा पञ्च॒ दक्षि॑णायाꣳ॒॒ श्रोण्या॒मुप॑ दधाति॒ पञ्चोत्त॑रस्यां॒ तस्मा᳚त् प॒श्चाद्-वर्.षी॑यान् पु॒रस्ता᳚त् प्रवणः प॒शुर्ब॒स्तो वय॒ इति॒ दक्षि॒णेऽꣳस॒ उप॑ दधाति वृ॒ष्णिर्वय॒ इत्युत्त॒रे ऽꣳसा॑वे॒व प्रति॑ दधाति व्या॒घ्रो वय॒ इति॒ दक्षि॑णे प॒क्ष उप॑ दधाति सिꣳ॒॒हो वय॒ इत्युत्त॑रे प॒क्षयो॑रे॒व वी॒र्यं॑ दधाति॒ पुरु॑षो॒ वय॒ इति॒ ( ) मद्ध्ये॒ तस्मा॒त् पुरु॑षः पशू॒नामधि॑पतिः ॥ \newline

\textbf{Pada Paata} \newline

मू॒द्‌र्ध॒न्वती॒रिति॑ मूद्‌र्धन्न्-वतीः᳚ । भ॒व॒न्ति॒ । तस्मा᳚त् । पु॒रस्ता᳚त् । मू॒द्‌र्धा । पञ्च॑ । दक्षि॑णायाम् । श्रोण्या᳚म् । उपेति॑ । द॒धा॒ति॒ । पञ्च॑ । उत्त॑रस्या॒मित्युत्-त॒र॒स्या॒म् । तस्मा᳚त् । प॒श्चात् । वर्.षी॑यान् । पु॒रस्ता᳚त् प्रवण॒ इति॑ पु॒रस्ता᳚त् - प्र॒व॒णः॒ । प॒शुः । ब॒स्तः । वयः॑ । इति॑ । दक्षि॑णे । अꣳसे᳚ । उपेति॑ । द॒धा॒ति॒ । वृ॒ष्णिः । वयः॑ । इति॑ । उत्त॑र॒ इत्युत् - त॒रे॒ । अꣳसौ᳚ । ए॒व । प्रतीति॑ । द॒धा॒ति॒ । व्या॒घ्रः । वयः॑ । इति॑ । दक्षि॑णे । प॒क्षे । उपेति॑ । द॒धा॒ति॒ । सिꣳ॒॒हः । वयः॑ । इति॑ । उत्त॑र॒ इत्युत् - त॒रे॒ । प॒क्षयोः᳚ । ए॒व । वी॒र्य᳚म् । द॒धा॒ति॒ । पुरु॑षः । वयः॑ । इति॑ ( ) । मद्ध्ये᳚ । तस्मा᳚त् । पुरु॑षः । प॒शू॒नाम् । अधि॑पति॒रित्यधि॑ - प॒तिः॒ ॥  \newline


\textbf{Krama Paata} \newline

मू॒र्द्ध॒न्वती᳚र् भवन्ति । मू॒र्द्ध॒न्वती॒रिति॑ मूर्द्धन्न् - वतीः᳚ । भ॒व॒न्ति॒ तस्मा᳚त् । तस्मा᳚त् पु॒रस्ता᳚त् । पु॒रस्ता᳚न् मू॒र्द्धा । मू॒र्द्धा पञ्च॑ । पञ्च॒ दक्षि॑णायाम् । दक्षि॑णायाꣳ॒॒ श्रोण्या᳚म् । श्रोण्या॒मुप॑ । उप॑ दधाति । द॒धा॒ति॒ पञ्च॑ । पञ्चोत्त॑रस्याम् । उत्त॑रस्या॒म् तस्मा᳚त् । उत्त॑रस्या॒मित्युत् - त॒र॒स्या॒म् । तस्मा᳚त् प॒श्चात् । प॒श्चाद् वर्.षी॑यान् । वर्.षी॑यान् पु॒रस्ता᳚त्प्रवणः । पु॒रस्ता᳚त्प्रवणः प॒शुः । पु॒रस्ता᳚त्प्रवण॒ इति॑ पु॒रस्ता᳚त् - प्र॒व॒णः॒ । प॒शुर् ब॒स्तः । ब॒स्तो वयः॑ । वय॒ इति॑ । इति॒ दक्षि॑णे । दक्षि॒णेऽꣳसे᳚ । अꣳस॒ उप॑ । उप॑ दधाति । द॒धा॒ति॒ वृ॒ष्णिः । वृ॒ष्णिर् वयः॑ । वय॒ इति॑ । इत्युत्त॑रे । उत्त॒रेऽꣳसौ᳚ । उत्त॑र॒ इत्युत् - त॒रे॒ । अꣳसा॑वे॒व । ए॒व प्रति॑ । प्रति॑ दधाति । द॒धा॒ति॒ व्या॒घ्रः । व्या॒घ्रो वयः॑ । वय॒ इति॑ । इति॒ दक्षि॑णे । दक्षि॑णे प॒क्षे । प॒क्ष उप॑ । उप॑ दधाति । द॒धा॒ति॒ सिꣳ॒॒हः । सिꣳ॒॒हो वयः॑ । वय॒ इति॑ । इत्युत्त॑रे । उत्त॑रे प॒क्षयोः᳚ । उत्त॑र॒ इत्युत् - त॒रे॒ । प॒क्षयो॑रे॒व । ए॒व वी॒र्य᳚म् । वी॒र्य॑म् दधाति । द॒धा॒ति॒ पुरु॑षः । पुरु॑षो॒ वयः॑ । वय॒ इति॑ ( ) । इति॒ मद्ध्ये᳚ । मद्ध्ये॒ तस्मा᳚त् । तस्मा॒त् पुरु॑षः । पुरु॑षः पशू॒नाम् । प॒शू॒नामधि॑पतिः । अधि॑पति॒रित्यधि॑ - प॒तिः॒ । \newline

\textbf{Jatai Paata} \newline

1. मू॒र्द्ध॒न्वती᳚र् भवन्ति भवन्ति मूर्द्ध॒न्वती᳚र् मूर्द्ध॒न्वती᳚र् भवन्ति । \newline
2. मू॒र्द्ध॒न्वती॒रिति॑ मूर्द्धन्न् - वतीः᳚ । \newline
3. भ॒व॒न्ति॒ तस्मा॒त् तस्मा᳚द् भवन्ति भवन्ति॒ तस्मा᳚त् । \newline
4. तस्मा᳚त् पु॒रस्ता᳚त् पु॒रस्ता॒त् तस्मा॒त् तस्मा᳚त् पु॒रस्ता᳚त् । \newline
5. पु॒रस्ता᳚न् मू॒र्द्धा मू॒र्द्धा पु॒रस्ता᳚त् पु॒रस्ता᳚न् मू॒र्द्धा । \newline
6. मू॒र्द्धा पञ्च॒ पञ्च॑ मू॒र्द्धा मू॒र्द्धा पञ्च॑ । \newline
7. पञ्च॒ दक्षि॑णाया॒म् दक्षि॑णाया॒म् पञ्च॒ पञ्च॒ दक्षि॑णायाम् । \newline
8. दक्षि॑णायाꣳ॒॒ श्रोण्याꣳ॒॒ श्रोण्या॒म् दक्षि॑णाया॒म् दक्षि॑णायाꣳ॒॒ श्रोण्या᳚म् । \newline
9. श्रोण्या॒ मुपोप॒ श्रोण्याꣳ॒॒ श्रोण्या॒ मुप॑ । \newline
10. उप॑ दधाति दधा॒ त्युपोप॑ दधाति । \newline
11. द॒धा॒ति॒ पञ्च॒ पञ्च॑ दधाति दधाति॒ पञ्च॑ । \newline
12. पञ्चो त्त॑रस्या॒ मुत्त॑रस्या॒म् पञ्च॒ पञ्चो त्त॑रस्याम् । \newline
13. उत्त॑रस्या॒म् तस्मा॒त् तस्मा॒ दुत्त॑रस्या॒ मुत्त॑रस्या॒म् तस्मा᳚त् । \newline
14. उत्त॑रस्या॒मित्युत् - त॒र॒स्या॒म् । \newline
15. तस्मा᳚त् प॒श्चात् प॒श्चात् तस्मा॒त् तस्मा᳚त् प॒श्चात् । \newline
16. प॒श्चाद् वर्.षी॑या॒न्॒. वर्.षी॑यान् प॒श्चात् प॒श्चाद् वर्.षी॑यान् । \newline
17. वर्.षी॑यान् पु॒रस्ता᳚त्प्रवणः पु॒रस्ता᳚त्प्रवणो॒ वर्.षी॑या॒न्॒. वर्.षी॑यान् पु॒रस्ता᳚त्प्रवणः । \newline
18. पु॒रस्ता᳚त्प्रवणः प॒शुः प॒शुः पु॒रस्ता᳚त्प्रवणः पु॒रस्ता᳚त्प्रवणः प॒शुः । \newline
19. पु॒रस्ता᳚त्प्रवण॒ इति॑ पु॒रस्ता᳚त् - प्र॒व॒णः॒ । \newline
20. प॒शुर् ब॒स्तो ब॒स्तः प॒शुः प॒शुर् ब॒स्तः । \newline
21. ब॒स्तो वयो॒ वयो॑ ब॒स्तो ब॒स्तो वयः॑ । \newline
22. वय॒ इतीति॒ वयो॒ वय॒ इति॑ । \newline
23. इति॒ दक्षि॑णे॒ दक्षि॑ण॒ इतीति॒ दक्षि॑णे । \newline
24. दक्षि॒णे ऽꣳसे ऽꣳसे॒ दक्षि॑णे॒ दक्षि॒णे ऽꣳसे᳚ । \newline
25. अꣳस॒ उपोपाꣳसे ऽꣳस॒ उप॑ । \newline
26. उप॑ दधाति दधा॒ त्युपोप॑ दधाति । \newline
27. द॒धा॒ति॒ वृ॒ष्णिर् वृ॒ष्णिर् द॑धाति दधाति वृ॒ष्णिः । \newline
28. वृ॒ष्णिर् वयो॒ वयो॑ वृ॒ष्णिर् वृ॒ष्णिर् वयः॑ । \newline
29. वय॒ इतीति॒ वयो॒ वय॒ इति॑ । \newline
30. इत्युत्त॑र॒ उत्त॑र॒ इती त्युत्त॑रे । \newline
31. उत्त॒रे ऽꣳसा॒ वꣳसा॒ वुत्त॑र॒ उत्त॒रे ऽꣳसौ᳚ । \newline
32. उत्त॑र॒ इत्युत् - त॒रे॒ । \newline
33. अꣳसा॑ वे॒वै वाꣳसा॒ वꣳसा॑ वे॒व । \newline
34. ए॒व प्रति॒ प्रत्ये॒वैव प्रति॑ । \newline
35. प्रति॑ दधाति दधाति॒ प्रति॒ प्रति॑ दधाति । \newline
36. द॒धा॒ति॒ व्या॒घ्रो व्या॒घ्रो द॑धाति दधाति व्या॒घ्रः । \newline
37. व्या॒घ्रो वयो॒ वयो᳚ व्या॒घ्रो व्या॒घ्रो वयः॑ । \newline
38. वय॒ इतीति॒ वयो॒ वय॒ इति॑ । \newline
39. इति॒ दक्षि॑णे॒ दक्षि॑ण॒ इतीति॒ दक्षि॑णे । \newline
40. दक्षि॑णे प॒क्षे प॒क्षे दक्षि॑णे॒ दक्षि॑णे प॒क्षे । \newline
41. प॒क्ष उपोप॑ प॒क्षे प॒क्ष उप॑ । \newline
42. उप॑ दधाति दधा॒ त्युपोप॑ दधाति । \newline
43. द॒धा॒ति॒ सिꣳ॒॒हः सिꣳ॒॒हो द॑धाति दधाति सिꣳ॒॒हः । \newline
44. सिꣳ॒॒हो वयो॒ वयः॑ सिꣳ॒॒हः सिꣳ॒॒हो वयः॑ । \newline
45. वय॒ इतीति॒ वयो॒ वय॒ इति॑ । \newline
46. इत्युत्त॑र॒ उत्त॑र॒ इतीत्युत्त॑रे । \newline
47. उत्त॑रे प॒क्षयोः᳚ प॒क्षयो॒ रुत्त॑र॒ उत्त॑रे प॒क्षयोः᳚ । \newline
48. उत्त॑र॒ इत्युत् - त॒रे॒ । \newline
49. प॒क्षयो॑ रे॒वैव प॒क्षयोः᳚ प॒क्षयो॑ रे॒व । \newline
50. ए॒व वी॒र्यं॑ ॅवी॒र्य॑ मे॒वैव वी॒र्य᳚म् । \newline
51. वी॒र्य॑म् दधाति दधाति वी॒र्यं॑ ॅवी॒र्य॑म् दधाति । \newline
52. द॒धा॒ति॒ पुरु॑षः॒ पुरु॑षो दधाति दधाति॒ पुरु॑षः । \newline
53. पुरु॑षो॒ वयो॒ वयः॒ पुरु॑षः॒ पुरु॑षो॒ वयः॑ । \newline
54. वय॒ इतीति॒ वयो॒ वय॒ इति॑ । \newline
55. इति॒ मद्ध्ये॒ मद्ध्य॒ इतीति॒ मद्ध्ये᳚ । \newline
56. मद्ध्ये॒ तस्मा॒त् तस्मा॒न् मद्ध्ये॒ मद्ध्ये॒ तस्मा᳚त् । \newline
57. तस्मा॒त् पुरु॑षः॒ पुरु॑ष॒ स्तस्मा॒त् तस्मा॒त् पुरु॑षः । \newline
58. पुरु॑षः पशू॒नाम् प॑शू॒नाम् पुरु॑षः॒ पुरु॑षः पशू॒नाम् । \newline
59. प॒शू॒ना मधि॑पति॒ रधि॑पतिः पशू॒नाम् प॑शू॒ना मधि॑पतिः । \newline
60. अधि॑पति॒रित्यधि॑ - प॒तिः॒ । \newline

\textbf{Ghana Paata } \newline

1. मू॒र्द्ध॒न्वती᳚र् भवन्ति भवन्ति मूर्द्ध॒न्वती᳚र् मूर्द्ध॒न्वती᳚र् भवन्ति॒ तस्मा॒त् तस्मा᳚द् भवन्ति मूर्द्ध॒न्वती᳚र् मूर्द्ध॒न्वती᳚र् भवन्ति॒ तस्मा᳚त् । \newline
2. मू॒र्द्ध॒न्वती॒रिति॑ मूर्द्धन्न् - वतीः᳚ । \newline
3. भ॒व॒न्ति॒ तस्मा॒त् तस्मा᳚द् भवन्ति भवन्ति॒ तस्मा᳚त् पु॒रस्ता᳚त् पु॒रस्ता॒त् तस्मा᳚द् भवन्ति भवन्ति॒ तस्मा᳚त् पु॒रस्ता᳚त् । \newline
4. तस्मा᳚त् पु॒रस्ता᳚त् पु॒रस्ता॒त् तस्मा॒त् तस्मा᳚त् पु॒रस्ता᳚न् मू॒र्द्धा मू॒र्द्धा पु॒रस्ता॒त् तस्मा॒त् तस्मा᳚त् पु॒रस्ता᳚न् मू॒र्द्धा । \newline
5. पु॒रस्ता᳚न् मू॒र्द्धा मू॒र्द्धा पु॒रस्ता᳚त् पु॒रस्ता᳚न् मू॒र्द्धा पञ्च॒ पञ्च॑ मू॒र्द्धा पु॒रस्ता᳚त् पु॒रस्ता᳚न् मू॒र्द्धा पञ्च॑ । \newline
6. मू॒र्द्धा पञ्च॒ पञ्च॑ मू॒र्द्धा मू॒र्द्धा पञ्च॒ दक्षि॑णाया॒म् दक्षि॑णाया॒म् पञ्च॑ मू॒र्द्धा मू॒र्द्धा पञ्च॒ दक्षि॑णायाम् । \newline
7. पञ्च॒ दक्षि॑णाया॒म् दक्षि॑णाया॒म् पञ्च॒ पञ्च॒ दक्षि॑णायाꣳ॒॒ श्रोण्याꣳ॒॒ श्रोण्या॒म् दक्षि॑णाया॒म् पञ्च॒ पञ्च॒ दक्षि॑णायाꣳ॒॒ श्रोण्या᳚म् । \newline
8. दक्षि॑णायाꣳ॒॒ श्रोण्याꣳ॒॒ श्रोण्या॒म् दक्षि॑णाया॒म् दक्षि॑णायाꣳ॒॒ श्रोण्या॒ मुपोप॒ श्रोण्या॒म् दक्षि॑णाया॒म् दक्षि॑णायाꣳ॒॒ श्रोण्या॒ मुप॑ । \newline
9. श्रोण्या॒ मुपोप॒ श्रोण्याꣳ॒॒ श्रोण्या॒ मुप॑ दधाति दधा॒ त्युप॒ श्रोण्याꣳ॒॒ श्रोण्या॒ मुप॑ दधाति । \newline
10. उप॑ दधाति दधा॒ त्युपोप॑ दधाति॒ पञ्च॒ पञ्च॑ दधा॒ त्युपोप॑ दधाति॒ पञ्च॑ । \newline
11. द॒धा॒ति॒ पञ्च॒ पञ्च॑ दधाति दधाति॒ पञ्चोत्त॑रस्या॒ मुत्त॑रस्या॒म् पञ्च॑ दधाति दधाति॒ पञ्चोत्त॑रस्याम् । \newline
12. पञ्चोत्त॑रस्या॒ मुत्त॑रस्या॒म् पञ्च॒ पञ्चोत्त॑रस्या॒म् तस्मा॒त् तस्मा॒ दुत्त॑रस्या॒म् पञ्च॒ पञ्चोत्त॑रस्या॒म् तस्मा᳚त् । \newline
13. उत्त॑रस्या॒म् तस्मा॒त् तस्मा॒ दुत्त॑रस्या॒ मुत्त॑रस्या॒म् तस्मा᳚त् प॒श्चात् प॒श्चात् तस्मा॒ दुत्त॑रस्या॒ मुत्त॑रस्या॒म् तस्मा᳚त् प॒श्चात् । \newline
14. उत्त॑रस्या॒मित्युत् - त॒र॒स्या॒म् । \newline
15. तस्मा᳚त् प॒श्चात् प॒श्चात् तस्मा॒त् तस्मा᳚त् प॒श्चाद् वर्.षी॑या॒न्॒. वर्.षी॑यान् प॒श्चात् तस्मा॒त् तस्मा᳚त् प॒श्चाद् वर्.षी॑यान् । \newline
16. प॒श्चाद् वर्.षी॑या॒न्॒. वर्.षी॑यान् प॒श्चात् प॒श्चाद् वर्.षी॑यान् पु॒रस्ता᳚त्प्रवणः पु॒रस्ता᳚त्प्रवणो॒ वर्.षी॑यान् प॒श्चात् प॒श्चाद् वर्.षी॑यान् पु॒रस्ता᳚त्प्रवणः । \newline
17. वर्.षी॑यान् पु॒रस्ता᳚त्प्रवणः पु॒रस्ता᳚त्प्रवणो॒ वर्.षी॑या॒न्॒. वर्.षी॑यान् पु॒रस्ता᳚त्प्रवणः प॒शुः प॒शुः पु॒रस्ता᳚त्प्रवणो॒ वर्.षी॑या॒न्॒. वर्.षी॑यान् पु॒रस्ता᳚त्प्रवणः प॒शुः । \newline
18. पु॒रस्ता᳚त्प्रवणः प॒शुः प॒शुः पु॒रस्ता᳚त्प्रवणः पु॒रस्ता᳚त्प्रवणः प॒शुर् ब॒स्तो ब॒स्तः प॒शुः पु॒रस्ता᳚त्प्रवणः पु॒रस्ता᳚त्प्रवणः प॒शुर् ब॒स्तः । \newline
19. पु॒रस्ता᳚त्प्रवण॒ इति॑ पु॒रस्ता᳚त् - प्र॒व॒णः॒ । \newline
20. प॒शुर् ब॒स्तो ब॒स्तः प॒शुः प॒शुर् ब॒स्तो वयो॒ वयो॑ ब॒स्तः प॒शुः प॒शुर् ब॒स्तो वयः॑ । \newline
21. ब॒स्तो वयो॒ वयो॑ ब॒स्तो ब॒स्तो वय॒ इतीति॒ वयो॑ ब॒स्तो ब॒स्तो वय॒ इति॑ । \newline
22. वय॒ इतीति॒ वयो॒ वय॒ इति॒ दक्षि॑णे॒ दक्षि॑ण॒ इति॒ वयो॒ वय॒ इति॒ दक्षि॑णे । \newline
23. इति॒ दक्षि॑णे॒ दक्षि॑ण॒ इतीति॒ दक्षि॒णे ऽꣳसे ऽꣳसे॒ दक्षि॑ण॒ इतीति॒ दक्षि॒णे ऽꣳसे᳚ । \newline
24. दक्षि॒णे ऽꣳसे ऽꣳसे॒ दक्षि॑णे॒ दक्षि॒णे ऽꣳस॒ उपोपाꣳसे॒ दक्षि॑णे॒ दक्षि॒णे ऽꣳस॒ उप॑ । \newline
25. अꣳस॒ उपोपाꣳसे ऽꣳस॒ उप॑ दधाति दधा॒ त्युपाꣳसे ऽꣳस॒ उप॑ दधाति । \newline
26. उप॑ दधाति दधा॒ त्युपोप॑ दधाति वृ॒ष्णिर् वृ॒ष्णिर् द॑धा॒ त्युपोप॑ दधाति वृ॒ष्णिः । \newline
27. द॒धा॒ति॒ वृ॒ष्णिर् वृ॒ष्णिर् द॑धाति दधाति वृ॒ष्णिर् वयो॒ वयो॑ वृ॒ष्णिर् द॑धाति दधाति वृ॒ष्णिर् वयः॑ । \newline
28. वृ॒ष्णिर् वयो॒ वयो॑ वृ॒ष्णिर् वृ॒ष्णिर् वय॒ इतीति॒ वयो॑ वृ॒ष्णिर् वृ॒ष्णिर् वय॒ इति॑ । \newline
29. वय॒ इतीति॒ वयो॒ वय॒ इत्युत्त॑र॒ उत्त॑र॒ इति॒ वयो॒ वय॒ इत्युत्त॑रे । \newline
30. इत्युत्त॑र॒ उत्त॑र॒ इतीत्युत्त॒रे ऽꣳसा॒ वꣳसा॒ वुत्त॑र॒ इतीत्युत्त॒रे ऽꣳसौ᳚ । \newline
31. उत्त॒रे ऽꣳसा॒ वꣳसा॒ वुत्त॑र॒ उत्त॒रे ऽꣳसा॑ वे॒वै वाꣳसा॒ वुत्त॑र॒ उत्त॒रे ऽꣳसा॑ वे॒व । \newline
32. उत्त॑र॒ इत्युत् - त॒रे॒ । \newline
33. अꣳसा॑ वे॒वै वाꣳसा॒ वꣳसा॑ वे॒व प्रति॒ प्रत्ये॒वाꣳसा॒ वꣳसा॑ वे॒व प्रति॑ । \newline
34. ए॒व प्रति॒ प्रत्ये॒वैव प्रति॑ दधाति दधाति॒ प्रत्ये॒वैव प्रति॑ दधाति । \newline
35. प्रति॑ दधाति दधाति॒ प्रति॒ प्रति॑ दधाति व्या॒घ्रो व्या॒घ्रो द॑धाति॒ प्रति॒ प्रति॑ दधाति व्या॒घ्रः । \newline
36. द॒धा॒ति॒ व्या॒घ्रो व्या॒घ्रो द॑धाति दधाति व्या॒घ्रो वयो॒ वयो᳚ व्या॒घ्रो द॑धाति दधाति व्या॒घ्रो वयः॑ । \newline
37. व्या॒घ्रो वयो॒ वयो᳚ व्या॒घ्रो व्या॒घ्रो वय॒ इतीति॒ वयो᳚ व्या॒घ्रो व्या॒घ्रो वय॒ इति॑ । \newline
38. वय॒ इतीति॒ वयो॒ वय॒ इति॒ दक्षि॑णे॒ दक्षि॑ण॒ इति॒ वयो॒ वय॒ इति॒ दक्षि॑णे । \newline
39. इति॒ दक्षि॑णे॒ दक्षि॑ण॒ इतीति॒ दक्षि॑णे प॒क्षे प॒क्षे दक्षि॑ण॒ इतीति॒ दक्षि॑णे प॒क्षे । \newline
40. दक्षि॑णे प॒क्षे प॒क्षे दक्षि॑णे॒ दक्षि॑णे प॒क्ष उपोप॑ प॒क्षे दक्षि॑णे॒ दक्षि॑णे प॒क्ष उप॑ । \newline
41. प॒क्ष उपोप॑ प॒क्षे प॒क्ष उप॑ दधाति दधा॒ त्युप॑ प॒क्षे प॒क्ष उप॑ दधाति । \newline
42. उप॑ दधाति दधा॒ त्युपोप॑ दधाति सिꣳ॒॒हः सिꣳ॒॒हो द॑धा॒ त्युपोप॑ दधाति सिꣳ॒॒हः । \newline
43. द॒धा॒ति॒ सिꣳ॒॒हः सिꣳ॒॒हो द॑धाति दधाति सिꣳ॒॒हो वयो॒ वयः॑ सिꣳ॒॒हो द॑धाति दधाति सिꣳ॒॒हो वयः॑ । \newline
44. सिꣳ॒॒हो वयो॒ वयः॑ सिꣳ॒॒हः सिꣳ॒॒हो वय॒ इतीति॒ वयः॑ सिꣳ॒॒हः सिꣳ॒॒हो वय॒ इति॑ । \newline
45. वय॒ इतीति॒ वयो॒ वय॒ इत्युत्त॑र॒ उत्त॑र॒ इति॒ वयो॒ वय॒ इत्युत्त॑रे । \newline
46. इत्युत्त॑र॒ उत्त॑र॒ इतीत्युत्त॑रे प॒क्षयोः᳚ प॒क्षयो॒ रुत्त॑र॒ इतीत्युत्त॑रे प॒क्षयोः᳚ । \newline
47. उत्त॑रे प॒क्षयोः᳚ प॒क्षयो॒ रुत्त॑र॒ उत्त॑रे प॒क्षयो॑ रे॒वैव प॒क्षयो॒ रुत्त॑र॒ उत्त॑रे प॒क्षयो॑ रे॒व । \newline
48. उत्त॑र॒ इत्युत् - त॒रे॒ । \newline
49. प॒क्षयो॑ रे॒वैव प॒क्षयोः᳚ प॒क्षयो॑ रे॒व वी॒र्यं॑ ॅवी॒र्य॑ मे॒व प॒क्षयोः᳚ प॒क्षयो॑ रे॒व वी॒र्य᳚म् । \newline
50. ए॒व वी॒र्यं॑ ॅवी॒र्य॑ मे॒वैव वी॒र्य॑म् दधाति दधाति वी॒र्य॑ मे॒वैव वी॒र्य॑म् दधाति । \newline
51. वी॒र्य॑म् दधाति दधाति वी॒र्यं॑ ॅवी॒र्य॑म् दधाति॒ पुरु॑षः॒ पुरु॑षो दधाति वी॒र्यं॑ ॅवी॒र्य॑म् दधाति॒ पुरु॑षः । \newline
52. द॒धा॒ति॒ पुरु॑षः॒ पुरु॑षो दधाति दधाति॒ पुरु॑षो॒ वयो॒ वयः॒ पुरु॑षो दधाति दधाति॒ पुरु॑षो॒ वयः॑ । \newline
53. पुरु॑षो॒ वयो॒ वयः॒ पुरु॑षः॒ पुरु॑षो॒ वय॒ इतीति॒ वयः॒ पुरु॑षः॒ पुरु॑षो॒ वय॒ इति॑ । \newline
54. वय॒ इतीति॒ वयो॒ वय॒ इति॒ मद्ध्ये॒ मद्ध्य॒ इति॒ वयो॒ वय॒ इति॒ मद्ध्ये᳚ । \newline
55. इति॒ मद्ध्ये॒ मद्ध्य॒ इतीति॒ मद्ध्ये॒ तस्मा॒त् तस्मा॒न् मद्ध्य॒ इतीति॒ मद्ध्ये॒ तस्मा᳚त् । \newline
56. मद्ध्ये॒ तस्मा॒त् तस्मा॒न् मद्ध्ये॒ मद्ध्ये॒ तस्मा॒त् पुरु॑षः॒ पुरु॑ष॒ स्तस्मा॒न् मद्ध्ये॒ मद्ध्ये॒ तस्मा॒त् पुरु॑षः । \newline
57. तस्मा॒त् पुरु॑षः॒ पुरु॑ष॒ स्तस्मा॒त् तस्मा॒त् पुरु॑षः पशू॒नाम् प॑शू॒नाम् पुरु॑ष॒ स्तस्मा॒त् तस्मा॒त् पुरु॑षः पशू॒नाम् । \newline
58. पुरु॑षः पशू॒नाम् प॑शू॒नाम् पुरु॑षः॒ पुरु॑षः पशू॒ना मधि॑पति॒ रधि॑पतिः पशू॒नाम् पुरु॑षः॒ पुरु॑षः पशू॒ना मधि॑पतिः । \newline
59. प॒शू॒ना मधि॑पति॒ रधि॑पतिः पशू॒नाम् प॑शू॒ना मधि॑पतिः । \newline
60. अधि॑पति॒रित्यधि॑ - प॒तिः॒ । \newline
\pagebreak
\markright{ TS 5.3.2.1  \hfill https://www.vedavms.in \hfill}

\section{ TS 5.3.2.1 }

\textbf{TS 5.3.2.1 } \newline
\textbf{Samhita Paata} \newline

इन्द्रा᳚ग्नी॒ अव्य॑थमाना॒मिति॑ स्वयमातृ॒ण्णामुप॑ दधातीन्द्रा॒ग्निभ्यां॒ ॅवा इ॒मौ लो॒कौ विधृ॑ताव॒नयो᳚-र्लो॒कयो॒-र्विधृ॑त्या॒ अधृ॑तेव॒ वा ए॒षा यन्म॑द्ध्य॒मा चिति॑र॒न्तरि॑क्षमिव॒ वा ए॒षेन्द्रा᳚ग्नी॒ इत्या॑हेन्द्रा॒ग्नी वै दे॒वाना॑मोजो॒ भृता॒वोज॑सै॒वैना॑-म॒न्तरि॑क्षे चिनुते॒ धृत्यै᳚ स्वयमातृ॒ण्णामुप॑ दधात्य॒न्तरि॑क्षं॒ ॅवै स्व॑यमातृ॒ण्णा ऽन्तरि॑क्षमे॒वोप॑ ध॒त्ते ऽश्व॒मुप॑ - [  ] \newline

\textbf{Pada Paata} \newline

इन्द्रा᳚ग्नी॒ इतीन्द्र॑ - अ॒ग्नी॒ । अव्य॑थमानाम् । इति॑ । स्व॒य॒मा॒तृ॒ण्णामिति॑ स्वयं - आ॒तृ॒ण्णाम् । उपेति॑ । द॒धा॒ति॒ । इ॒न्द्रा॒ग्निभ्या॒मिती᳚न्द्रा॒ग्नि - भ्या॒म् । वै । इ॒मौ । लो॒कौ । विधृ॑ता॒विति॒ वि - धृ॒तौ॒ । अ॒नयोः᳚ । लो॒कयोः᳚ । विधृ॑त्या॒ इति॒ वि - धृ॒त्यै॒ । अधृ॑ता । इ॒व॒ । वै । ए॒षा । यत् । म॒द्ध्य॒मा । चितिः॑ । अ॒न्तरि॑क्षम् । इ॒व॒ । वै । ए॒षा । इन्द्रा᳚ग्नी॒ इतीन्द्र॑ - अ॒ग्नी॒ । इति॑ । आ॒ह॒ । इ॒न्द्रा॒ग्नी इती᳚न्द्र - अ॒ग्नी । वै । दे॒वाना᳚म् । ओ॒जो॒भृता॒वित्यो॑जः - भृतौ᳚ । ओज॑सा । ए॒व । ए॒ना॒म् । अ॒न्तरि॑क्षे । चि॒नु॒ते॒ । धृत्यै᳚ । स्व॒य॒मा॒तृ॒ण्णामिति॑ स्वयं - आ॒तृ॒ण्णाम् । उपेति॑ । द॒धा॒ति॒ । अ॒न्तरि॑क्षम् । वै । स्व॒य॒मा॒तृ॒ण्णेति॑ स्वयं-आ॒तृ॒ण्णा । अ॒न्तरि॑क्षम् । ए॒व । उपेति॑ । ध॒त्ते॒ । अश्व᳚म् । उपेति॑ ।  \newline


\textbf{Krama Paata} \newline

इन्द्रा᳚ग्नी॒ अव्य॑थमानाम् । इन्द्रा॑ग्नी॒ इतीन्द्र॑ - अ॒ग्नी॒ । अव्य॑थमाना॒मिति॑ । इति॑ स्वयमातृ॒ण्णाम् । स्व॒य॒मा॒तृ॒ण्णामुप॑ । स्व॒य॒मा॒तृ॒ण्णामिति॑ स्वयम् - आ॒तृ॒ण्णाम् । उप॑ दधाति । द॒धा॒ती॒न्द्रा॒ग्निभ्या᳚म् । इ॒न्द्रा॒ग्निभ्या॒म् ॅवै । इ॒न्द्रा॒ग्निभ्या॒मिती᳚न्द्रा॒ग्नि - भ्या॒म् । वा इ॒मौ । इ॒मौ लो॒कौ । लो॒कौ विधृ॑तौ । विधृ॑ताव॒नयोः᳚ । विधृ॑ता॒विति॒ वि - धृ॒तौ॒ । अ॒नयो᳚र् लो॒कयोः᳚ । लो॒कयो॒र् विधृ॑त्यै । विधृ॑त्या॒ अधृ॑ता । विधृ॑त्या॒ इति॒ वि - धृ॒त्यै॒ । अधृ॑तेव । इ॒व॒ वै । वा ए॒षा । ए॒षा यत् । यन् म॑द्ध्य॒मा । म॒द्ध्य॒मा चितिः॑ । चिति॑र॒न्तरि॑क्षम् । अ॒न्तरि॑क्षमिव । इ॒व॒ वै । वा ए॒षा । ए॒षेन्द्रा᳚ग्नी । इन्द्रा᳚ग्नी॒ इति॑ । इन्द्रा᳚ग्नी॒ इतीन्द्र॑ - अ॒ग्नी॒ । इत्या॑ह । आ॒हे॒न्द्रा॒ग्नी । इ॒न्द्रा॒ग्नी वै । इ॒न्द्रा॒ग्नी इतीन्द्र॑ - अ॒ग्नी । वै दे॒वाना᳚म् । दे॒वाना॑मोजो॒भृतौ᳚ । ओ॒जो॒भृता॒वोज॑सा । ओ॒जो॒भृता॒वित्यो॑जः - भृतौ᳚ । ओज॑सै॒व । ए॒वैना᳚म् । ए॒ना॒म॒न्तरि॑क्षे । अ॒न्तरि॑क्षे चिनुते । चि॒नु॒ते॒ धृत्यै᳚ । धृत्यै᳚ स्वयमातृ॒ण्णाम् । स्व॒य॒मा॒तृ॒ण्णामुप॑ । स्व॒य॒मा॒तृ॒ण्णामिति॑ स्वयम् - आ॒तृ॒ण्णाम् । उप॑ दधाति । द॒धा॒त्य॒न्तरि॑क्षम् । अ॒न्तरि॑क्ष॒म् ॅवै । वै स्व॑यमातृ॒ण्णा । स्व॒य॒मा॒तृ॒ण्णाऽन्तरि॑क्षम् । स्व॒य॒मा॒तृ॒ण्णेति॑ स्वयम् - आ॒तृ॒ण्णा । अ॒न्तरि॑क्षमे॒व । ए॒वोप॑ । उप॑ धत्ते । ध॒त्तेऽश्व᳚म् । अश्व॒मुप॑ । उप॑ घ्रापयति \newline

\textbf{Jatai Paata} \newline

1. इन्द्रा᳚ग्नी॒ अव्य॑थमाना॒ मव्य॑थमाना॒ मिन्द्रा᳚ग्नी॒ इन्द्रा᳚ग्नी॒ अव्य॑थमानाम् । \newline
2. इन्द्रा᳚ग्नी॒ इतीन्द्र॑ - अ॒ग्नी॒ । \newline
3. अव्य॑थमाना॒ मितीत्य व्य॑थमाना॒ मव्य॑थमाना॒ मिति॑ । \newline
4. इति॑ स्वयमातृ॒ण्णाꣳ स्व॑यमातृ॒ण्णा मितीति॑ स्वयमातृ॒ण्णाम् । \newline
5. स्व॒य॒मा॒तृ॒ण्णा मुपोप॑ स्वयमातृ॒ण्णाꣳ स्व॑यमातृ॒ण्णा मुप॑ । \newline
6. स्व॒य॒मा॒तृ॒ण्णामिति॑ स्वयं - आ॒तृ॒ण्णाम् । \newline
7. उप॑ दधाति दधा॒ त्युपोप॑ दधाति । \newline
8. द॒धा॒ती॒ न्द्रा॒ग्निभ्या॑ मिन्द्रा॒ग्निभ्या᳚म् दधाति दधाती न्द्रा॒ग्निभ्या᳚म् । \newline
9. इ॒न्द्रा॒ग्निभ्यां॒ ॅवै वा इ॑न्द्रा॒ग्निभ्या॑ मिन्द्रा॒ग्निभ्यां॒ ॅवै । \newline
10. इ॒न्द्रा॒ग्निभ्या॒मिती᳚न्द्रा॒ग्नि - भ्या॒म् । \newline
11. वा इ॒मा वि॒मौ वै वा इ॒मौ । \newline
12. इ॒मौ लो॒कौ लो॒का वि॒मा वि॒मौ लो॒कौ । \newline
13. लो॒कौ विधृ॑तौ॒ विधृ॑तौ लो॒कौ लो॒कौ विधृ॑तौ । \newline
14. विधृ॑ता व॒नयो॑ र॒नयो॒र् विधृ॑तौ॒ विधृ॑ता व॒नयोः᳚ । \newline
15. विधृ॑ता॒विति॒ वि - धृ॒तौ॒ । \newline
16. अ॒नयो᳚र् लो॒कयो᳚र् लो॒कयो॑ र॒नयो॑ र॒नयो᳚र् लो॒कयोः᳚ । \newline
17. लो॒कयो॒र् विधृ॑त्यै॒ विधृ॑त्यै लो॒कयो᳚र् लो॒कयो॒र् विधृ॑त्यै । \newline
18. विधृ॑त्या॒ अधृ॒ता ऽधृ॑ता॒ विधृ॑त्यै॒ विधृ॑त्या॒ अधृ॑ता । \newline
19. विधृ॑त्या॒ इति॒ वि - धृ॒त्यै॒ । \newline
20. अधृ॑तेवे॒ वाधृ॒ता ऽधृ॑तेव । \newline
21. इ॒व॒वै वा इ॑वेव॒ वै । \newline
22. वा ए॒षैषा वै वा ए॒षा । \newline
23. ए॒षा यद् यदे॒ षैषा यत् । \newline
24. यन् म॑द्ध्य॒मा म॑द्ध्य॒मा यद् यन् म॑द्ध्य॒मा । \newline
25. म॒द्ध्य॒मा चिति॒ श्चिति॑र् मद्ध्य॒मा म॑द्ध्य॒मा चितिः॑ । \newline
26. चिति॑ र॒न्तरि॑क्ष म॒न्तरि॑क्ष॒म् चिति॒ श्चिति॑ र॒न्तरि॑क्षम् । \newline
27. अ॒न्तरि॑क्ष मिवे वा॒न्तरि॑क्ष म॒न्तरि॑क्ष मिव । \newline
28. इ॒व॒ वै वा इ॑वेव॒ वै । \newline
29. वा ए॒षैषा वै वा ए॒षा । \newline
30. ए॒षे न्द्रा᳚ग्नी॒ इन्द्रा᳚ग्नी ए॒षै षेन्द्रा᳚ग्नी । \newline
31. इन्द्रा᳚ग्नी॒ इतीती न्द्रा᳚ग्नी॒ इन्द्रा᳚ग्नी॒ इति॑ । \newline
32. इन्द्रा᳚ग्नी॒ इतीन्द्र॑ - अ॒ग्नी॒ । \newline
33. इत्या॑ हा॒हे तीत्या॑ह । \newline
34. आ॒हे॒न्द्रा॒ग्नी इ॑न्द्रा॒ग्नी आ॑हा हेन्द्रा॒ग्नी । \newline
35. इ॒न्द्रा॒ग्नी वै वा इ॑न्द्रा॒ग्नी इ॑न्द्रा॒ग्नी वै । \newline
36. इ॒न्द्रा॒ग्नी इती᳚न्द्र - अ॒ग्नी । \newline
37. वै दे॒वाना᳚म् दे॒वानां॒ ॅवै वै दे॒वाना᳚म् । \newline
38. दे॒वाना॑ मोजो॒भृता॑ वोजो॒भृतौ॑ दे॒वाना᳚म् दे॒वाना॑ मोजो॒भृतौ᳚ । \newline
39. ओ॒जो॒भृता॒ वोज॒सौ ज॑सौजो॒ भृता॑ वोजो॒भृता॒ वोज॑सा । \newline
40. ओ॒जो॒भृता॒वित्यो॑जः - भृतौ᳚ । \newline
41. ओज॑ सै॒वैवौज॒ सौज॑सै॒व । \newline
42. ए॒वैना॑ मेना मे॒वै वैना᳚म् । \newline
43. ए॒ना॒ म॒न्तरि॑क्षे॒ ऽन्तरि॑क्ष एना मेना म॒न्तरि॑क्षे । \newline
44. अ॒न्तरि॑क्षे चिनुते चिनुते॒ ऽन्तरि॑क्षे॒ ऽन्तरि॑क्षे चिनुते । \newline
45. चि॒नु॒ते॒ धृत्यै॒ धृत्यै॑ चिनुते चिनुते॒ धृत्यै᳚ । \newline
46. धृत्यै᳚ स्वयमातृ॒ण्णाꣳ स्व॑यमातृ॒ण्णाम् धृत्यै॒ धृत्यै᳚ स्वयमातृ॒ण्णाम् । \newline
47. स्व॒य॒मा॒तृ॒ण्णा मुपोप॑ स्वयमातृ॒ण्णाꣳ स्व॑यमातृ॒ण्णा मुप॑ । \newline
48. स्व॒य॒मा॒तृ॒ण्णामिति॑ स्वयं - आ॒तृ॒ण्णाम् । \newline
49. उप॑ दधाति दधा॒ त्युपोप॑ दधाति । \newline
50. द॒धा॒ त्य॒न्तरि॑क्ष म॒न्तरि॑क्षम् दधाति दधा त्य॒न्तरि॑क्षम् । \newline
51. अ॒न्तरि॑क्षं॒ ॅवै वा अ॒न्तरि॑क्ष म॒न्तरि॑क्षं॒ ॅवै । \newline
52. वै स्व॑यमातृ॒ण्णा स्व॑यमातृ॒ण्णा वै वै स्व॑यमातृ॒ण्णा । \newline
53. स्व॒य॒मा॒तृ॒ण्णा ऽन्तरि॑क्ष म॒न्तरि॑क्षꣳ स्वयमातृ॒ण्णा स्व॑यमातृ॒ण्णा ऽन्तरि॑क्षम् । \newline
54. स्व॒य॒मा॒तृ॒ण्णेति॑ स्वयं - आ॒तृ॒ण्णा । \newline
55. अ॒न्तरि॑क्ष मे॒वै वान्तरि॑क्ष म॒न्तरि॑क्ष मे॒व । \newline
56. ए॒वोपो पै॒वै वोप॑ । \newline
57. उप॑ धत्ते धत्त॒ उपोप॑ धत्ते । \newline
58. ध॒त्ते ऽश्व॒ मश्व॑म् धत्ते ध॒त्ते ऽश्व᳚म् । \newline
59. अश्व॒ मुपोपाश्व॒ मश्व॒ मुप॑ । \newline
60. उप॑ घ्रापयति घ्रापय॒ त्युपोप॑ घ्रापयति । \newline

\textbf{Ghana Paata } \newline

1. इन्द्रा᳚ग्नी॒ अव्य॑थमाना॒ मव्य॑थमाना॒ मिन्द्रा᳚ग्नी॒ इन्द्रा᳚ग्नी॒ अव्य॑थमाना॒ मिती त्यव्य॑थमाना॒ मिन्द्रा᳚ग्नी॒ इन्द्रा᳚ग्नी॒ अव्य॑थमाना॒ मिति॑ । \newline
2. इन्द्रा᳚ग्नी॒ इतीन्द्र॑ - अ॒ग्नी॒ । \newline
3. अव्य॑थमाना॒ मिती त्यव्य॑थमाना॒ मव्य॑थमाना॒ मिति॑ स्वयमातृ॒ण्णाꣳ स्व॑यमातृ॒ण्णा मित्यव्य॑थमाना॒ मव्य॑थमाना॒ मिति॑ स्वयमातृ॒ण्णाम् । \newline
4. इति॑ स्वयमातृ॒ण्णाꣳ स्व॑यमातृ॒ण्णा मितीति॑ स्वयमातृ॒ण्णा मुपोप॑ स्वयमातृ॒ण्णा मितीति॑ स्वयमातृ॒ण्णा मुप॑ । \newline
5. स्व॒य॒मा॒तृ॒ण्णा मुपोप॑ स्वयमातृ॒ण्णाꣳ स्व॑यमातृ॒ण्णा मुप॑ दधाति दधा॒ त्युप॑ स्वयमातृ॒ण्णाꣳ स्व॑यमातृ॒ण्णा मुप॑ दधाति । \newline
6. स्व॒य॒मा॒तृ॒ण्णामिति॑ स्वयं - आ॒तृ॒ण्णाम् । \newline
7. उप॑ दधाति दधा॒ त्युपोप॑ दधातीन्द्रा॒ग्निभ्या॑ मिन्द्रा॒ग्निभ्या᳚म् दधा॒ त्युपोप॑ दधातीन्द्रा॒ग्निभ्या᳚म् । \newline
8. द॒धा॒ती॒न्द्रा॒ग्निभ्या॑ मिन्द्रा॒ग्निभ्या᳚म् दधाति दधातीन्द्रा॒ग्निभ्यां॒ ॅवै वा इ॑न्द्रा॒ग्निभ्या᳚म् दधाति 
दधातीन्द्रा॒ग्निभ्यां॒ ॅवै । \newline
9. इ॒न्द्रा॒ग्निभ्यां॒ ॅवै वा इ॑न्द्रा॒ग्निभ्या॑ मिन्द्रा॒ग्निभ्यां॒ ॅवा इ॒मा वि॒मौ वा इ॑न्द्रा॒ग्निभ्या॑ मिन्द्रा॒ग्निभ्यां॒ ॅवा इ॒मौ । \newline
10. इ॒न्द्रा॒ग्निभ्या॒मिती᳚न्द्रा॒ग्नि - भ्या॒म् । \newline
11. वा इ॒मा वि॒मौ वै वा इ॒मौ लो॒कौ लो॒का वि॒मौ वै वा इ॒मौ लो॒कौ । \newline
12. इ॒मौ लो॒कौ लो॒का वि॒मा वि॒मौ लो॒कौ विधृ॑तौ॒ विधृ॑तौ लो॒का वि॒मा वि॒मौ लो॒कौ विधृ॑तौ । \newline
13. लो॒कौ विधृ॑तौ॒ विधृ॑तौ लो॒कौ लो॒कौ विधृ॑ता व॒नयो॑ र॒नयो॒र् विधृ॑तौ लो॒कौ लो॒कौ विधृ॑ता व॒नयोः᳚ । \newline
14. विधृ॑ता व॒नयो॑ र॒नयो॒र् विधृ॑तौ॒ विधृ॑ता व॒नयो᳚र् लो॒कयो᳚र् लो॒कयो॑ र॒नयो॒र् विधृ॑तौ॒ विधृ॑ता व॒नयो᳚र् लो॒कयोः᳚ । \newline
15. विधृ॑ता॒विति॒ वि - धृ॒तौ॒ । \newline
16. अ॒नयो᳚र् लो॒कयो᳚र् लो॒कयो॑ र॒नयो॑ र॒नयो᳚र् लो॒कयो॒र् विधृ॑त्यै॒ विधृ॑त्यै लो॒कयो॑ र॒नयो॑ र॒नयो᳚र् लो॒कयो॒र् विधृ॑त्यै । \newline
17. लो॒कयो॒र् विधृ॑त्यै॒ विधृ॑त्यै लो॒कयो᳚र् लो॒कयो॒र् विधृ॑त्या॒ अधृ॒ता ऽधृ॑ता॒ विधृ॑त्यै लो॒कयो᳚र् लो॒कयो॒र् विधृ॑त्या॒ अधृ॑ता । \newline
18. विधृ॑त्या॒ अधृ॒ता ऽधृ॑ता॒ विधृ॑त्यै॒ विधृ॑त्या॒ अधृ॑तेवे॒ वाधृ॑ता॒ विधृ॑त्यै॒ विधृ॑त्या॒ अधृ॑तेव । \newline
19. विधृ॑त्या॒ इति॒ वि - धृ॒त्यै॒ । \newline
20. अधृ॑ तेवे॒वा धृ॒ता ऽधृ॑ तेव॒ वै वा इ॒वा धृ॒ता ऽधृ॑ तेव॒ वै । \newline
21. इ॒व॒ वै वा इ॑वेव॒ वा ए॒षैषा वा इ॑वेव॒ वा ए॒षा । \newline
22. वा ए॒षैषा वै वा ए॒षा यद् यदे॒षा वै वा ए॒षा यत् । \newline
23. ए॒षा यद् यदे॒ षैषा यन् म॑द्ध्य॒मा म॑द्ध्य॒मा यदे॒ षैषा यन् म॑द्ध्य॒मा । \newline
24. यन् म॑द्ध्य॒मा म॑द्ध्य॒मा यद् यन् म॑द्ध्य॒मा चिति॒ श्चिति॑र् मद्ध्य॒मा यद् यन् म॑द्ध्य॒मा चितिः॑ । \newline
25. म॒द्ध्य॒मा चिति॒ श्चिति॑र् मद्ध्य॒मा म॑द्ध्य॒मा चिति॑ र॒न्तरि॑क्ष म॒न्तरि॑क्ष॒म् चिति॑र् मद्ध्य॒मा म॑द्ध्य॒मा चिति॑ र॒न्तरि॑क्षम् । \newline
26. चिति॑ र॒न्तरि॑क्ष म॒न्तरि॑क्ष॒म् चिति॒ श्चिति॑ र॒न्तरि॑क्ष मिवे वा॒न्तरि॑क्ष॒म् चिति॒ श्चिति॑ र॒न्तरि॑क्ष मिव । \newline
27. अ॒न्तरि॑क्ष मिवे वा॒न्तरि॑क्ष म॒न्तरि॑क्ष मिव॒ वै वा इ॑वा॒न्तरि॑क्ष म॒न्तरि॑क्ष मिव॒ वै । \newline
28. इ॒व॒ वै वा इ॑वेव॒ वा ए॒षैषा वा इ॑वेव॒ वा ए॒षा । \newline
29. वा ए॒षैषा वै वा ए॒षेन्द्रा᳚ग्नी॒ इन्द्रा᳚ग्नी ए॒षा वै वा ए॒षेन्द्रा᳚ग्नी । \newline
30. ए॒षेन्द्रा᳚ग्नी॒ इन्द्रा᳚ग्नी ए॒षैषेन्द्रा᳚ग्नी॒ इतीतीन्द्रा᳚ग्नी ए॒षैषेन्द्रा᳚ग्नी॒ इति॑ । \newline
31. इन्द्रा᳚ग्नी॒ इतीतीन्द्रा᳚ग्नी॒ इन्द्रा᳚ग्नी॒ इत्या॑हा॒हे तीन्द्रा᳚ग्नी॒ इन्द्रा᳚ग्नी॒ इत्या॑ह । \newline
32. इन्द्रा᳚ग्नी॒ इतीन्द्र॑ - अ॒ग्नी॒ । \newline
33. इत्या॑हा॒हे तीत्या॑हेन्द्रा॒ग्नी इ॑न्द्रा॒ग्नी आ॒हे तीत्या॑हेन्द्रा॒ग्नी । \newline
34. आ॒हे॒न्द्रा॒ग्नी इ॑न्द्रा॒ग्नी आ॑हाहेन्द्रा॒ग्नी वै वा इ॑न्द्रा॒ग्नी आ॑हाहेन्द्रा॒ग्नी वै । \newline
35. इ॒न्द्रा॒ग्नी वै वा इ॑न्द्रा॒ग्नी इ॑न्द्रा॒ग्नी वै दे॒वाना᳚म् दे॒वानां॒ ॅवा इ॑न्द्रा॒ग्नी इ॑न्द्रा॒ग्नी वै दे॒वाना᳚म् । \newline
36. इ॒न्द्रा॒ग्नी इती᳚न्द्र - अ॒ग्नी । \newline
37. वै दे॒वाना᳚म् दे॒वानां॒ ॅवै वै दे॒वाना॑ मोजो॒भृता॑ वोजो॒भृतौ॑ दे॒वानां॒ ॅवै वै दे॒वाना॑ मोजो॒भृतौ᳚ । \newline
38. दे॒वाना॑ मोजो॒भृता॑ वोजो॒भृतौ॑ दे॒वाना᳚म् दे॒वाना॑ मोजो॒भृता॒ वोज॒सौ ज॑सौजो॒ भृतौ॑ दे॒वाना᳚म् दे॒वाना॑ मोजो॒भृता॒ वोज॑सा । \newline
39. ओ॒जो॒भृता॒ वोज॒सौज॑ सौजो॒भृता॑ वोजो॒भृता॒ वोज॑ सै॒वैवौज॑ सौजो॒भृता॑ वोजो॒भृता॒ वोज॑सै॒व । \newline
40. ओ॒जो॒भृता॒वित्यो॑जः - भृतौ᳚ । \newline
41. ओज॑ सै॒वै वौज॒ सौज॑ सै॒वैना॑ मेना मे॒वौज॒ सौज॑ सै॒वैना᳚म् । \newline
42. ए॒वैना॑ मेना मे॒वै वैना॑ म॒न्तरि॑क्षे॒ ऽन्तरि॑क्ष एना मे॒वै वैना॑ म॒न्तरि॑क्षे । \newline
43. ए॒ना॒ म॒न्तरि॑क्षे॒ ऽन्तरि॑क्ष एना मेना म॒न्तरि॑क्षे चिनुते चिनुते॒ ऽन्तरि॑क्ष एना मेना म॒न्तरि॑क्षे चिनुते । \newline
44. अ॒न्तरि॑क्षे चिनुते चिनुते॒ ऽन्तरि॑क्षे॒ ऽन्तरि॑क्षे चिनुते॒ धृत्यै॒ धृत्यै॑ चिनुते॒ ऽन्तरि॑क्षे॒ ऽन्तरि॑क्षे चिनुते॒ धृत्यै᳚ । \newline
45. चि॒नु॒ते॒ धृत्यै॒ धृत्यै॑ चिनुते चिनुते॒ धृत्यै᳚ स्वयमातृ॒ण्णाꣳ स्व॑यमातृ॒ण्णाम् धृत्यै॑ चिनुते चिनुते॒ धृत्यै᳚ स्वयमातृ॒ण्णाम् । \newline
46. धृत्यै᳚ स्वयमातृ॒ण्णाꣳ स्व॑यमातृ॒ण्णाम् धृत्यै॒ धृत्यै᳚ स्वयमातृ॒ण्णा मुपोप॑ स्वयमातृ॒ण्णाम् धृत्यै॒ धृत्यै᳚ स्वयमातृ॒ण्णा मुप॑ । \newline
47. स्व॒य॒मा॒तृ॒ण्णा मुपोप॑ स्वयमातृ॒ण्णाꣳ स्व॑यमातृ॒ण्णा मुप॑ दधाति दधा॒ त्युप॑ स्वयमातृ॒ण्णाꣳ स्व॑यमातृ॒ण्णा मुप॑ दधाति । \newline
48. स्व॒य॒मा॒तृ॒ण्णामिति॑ स्वयं - आ॒तृ॒ण्णाम् । \newline
49. उप॑ दधाति दधा॒ त्युपोप॑ दधा त्य॒न्तरि॑क्ष म॒न्तरि॑क्षम् दधा॒ त्युपोप॑ दधा त्य॒न्तरि॑क्षम् । \newline
50. द॒धा॒ त्य॒न्तरि॑क्ष म॒न्तरि॑क्षम् दधाति दधा त्य॒न्तरि॑क्षं॒ ॅवै वा अ॒न्तरि॑क्षम् दधाति दधा
त्य॒न्तरि॑क्षं॒ ॅवै । \newline
51. अ॒न्तरि॑क्षं॒ ॅवै वा अ॒न्तरि॑क्ष म॒न्तरि॑क्षं॒ ॅवै स्व॑यमातृ॒ण्णा स्व॑यमातृ॒ण्णा वा अ॒न्तरि॑क्ष म॒न्तरि॑क्षं॒ ॅवै स्व॑यमातृ॒ण्णा । \newline
52. वै स्व॑यमातृ॒ण्णा स्व॑यमातृ॒ण्णा वै वै स्व॑यमातृ॒ण्णा ऽन्तरि॑क्ष म॒न्तरि॑क्षꣳ स्वयमातृ॒ण्णा वै वै स्व॑यमातृ॒ण्णा ऽन्तरि॑क्षम् । \newline
53. स्व॒य॒मा॒तृ॒ण्णा ऽन्तरि॑क्ष म॒न्तरि॑क्षꣳ स्वयमातृ॒ण्णा स्व॑यमातृ॒ण्णा ऽन्तरि॑क्ष 
मे॒वै वान्तरि॑क्षꣳ स्वयमातृ॒ण्णा स्व॑यमातृ॒ण्णा ऽन्तरि॑क्ष मे॒व । \newline
54. स्व॒य॒मा॒तृ॒ण्णेति॑ स्वयं - आ॒तृ॒ण्णा । \newline
55. अ॒न्तरि॑क्ष मे॒वै वान्तरि॑क्ष म॒न्तरि॑क्ष मे॒वोपोपै॒ वान्तरि॑क्ष म॒न्तरि॑क्ष मे॒वोप॑ । \newline
56. ए॒वोपो पै॒वैवोप॑ धत्ते धत्त॒ उपै॒वै वोप॑ धत्ते । \newline
57. उप॑ धत्ते धत्त॒ उपोप॑ ध॒त्ते ऽश्व॒ मश्व॑म् धत्त॒ उपोप॑ ध॒त्ते ऽश्व᳚म् । \newline
58. ध॒त्ते ऽश्व॒ मश्व॑म् धत्ते ध॒त्ते ऽश्व॒ मुपोपा श्व॑म् धत्ते ध॒त्ते ऽश्व॒ मुप॑ । \newline
59. अश्व॒ मुपोपा श्व॒ मश्व॒ मुप॑ घ्रापयति घ्रापय॒ त्युपा श्व॒ मश्व॒ मुप॑ घ्रापयति । \newline
60. उप॑ घ्रापयति घ्रापय॒ त्युपोप॑ घ्रापयति प्रा॒णम् प्रा॒णम् घ्रा॑पय॒ त्युपोप॑ घ्रापयति प्रा॒णम् । \newline
\pagebreak
\markright{ TS 5.3.2.2  \hfill https://www.vedavms.in \hfill}

\section{ TS 5.3.2.2 }

\textbf{TS 5.3.2.2 } \newline
\textbf{Samhita Paata} \newline

घ्रापयति प्रा॒णमे॒वास्यां᳚ दधा॒त्यथो᳚ प्राजाप॒त्यो वा अश्वः॑ प्र॒जाप॑तिनै॒वाग्निं चि॑नुते स्वयमातृ॒ण्णा भ॑वति प्रा॒णाना॒मुथ्सृ॑ष्ट्या॒ अथो॑ सुव॒र्गस्य॑ लो॒कस्यानु॑ख्यात्यै दे॒वानां॒ ॅवै सु॑व॒र्गं ॅलो॒कं ॅय॒तां दिशः॒ सम॑व्लीयन्त॒ त ए॒ता दिश्या॑ अपश्य॒न् ता उपा॑दधत॒ ताभि॒र्वै ते दिशो॑ऽदृꣳह॒न्॒ यद्दिश्या॑ उप॒दधा॑ति दि॒शां ॅविधृ॑त्यै॒ दश॑ प्राण॒भृतः॑ पु॒रस्ता॒दुप॑ - [  ] \newline

\textbf{Pada Paata} \newline

घ्रा॒प॒य॒ति॒ । प्रा॒णमिति॑ प्र - अ॒नम् । ए॒व । अ॒स्या॒म् । द॒धा॒ति॒ । अथो॒ इति॑ । प्रा॒जा॒प॒त्य इति॑ प्राजा - प॒त्यः । वै । अश्वः॑ । प्र॒जाप॑ति॒नेति॑ प्र॒जा - प॒ति॒ना॒ । ए॒व । अ॒ग्निम् । चि॒नु॒ते॒ । स्व॒य॒मा॒तृ॒ण्णेति॑ स्वयं - आ॒तृ॒ण्णा । भ॒व॒ति॒ । प्रा॒णाना॒मिति॑ प्र - अ॒नाना᳚म् । उथ्सृ॑ष्ट्य॒ इत्युत् - सृ॒ष्ट्यै॒ । अथो॒ इति॑ । सु॒व॒र्गस्येति॑ सुवः - गस्य॑ । लो॒कस्य॑ । अनु॑ख्यात्या॒ इत्यनु॑ - ख्या॒त्यै॒ । दे॒वाना᳚म् । वै । सु॒व॒र्गमिति॑ सुवः - गम् । लो॒कम् । य॒ताम् । दिशः॑ । समिति॑ । अ॒व्ली॒य॒न्त॒ । ते । ए॒ताः । दिश्याः᳚ । अ॒प॒श्य॒न्न् । ताः । उपेति॑ । अ॒द॒ध॒त॒ । ताभिः॑ । वै । ते । दिशः॑ । अ॒दृꣳ॒॒ह॒न्न् । यत् । दिश्याः᳚ । उ॒प॒दधा॒तीत्यु॑प - दधा॑ति । दि॒शाम् । विधृ॑त्या॒ इति॒ वि - धृ॒त्यै॒ । दश॑ । प्रा॒ण॒भृत॒ इति॑ प्राण - भृतः॑ । पु॒रस्ता᳚त् । उपेति॑ ।  \newline


\textbf{Krama Paata} \newline

घ्रा॒प॒य॒ति॒ प्रा॒णम् । प्रा॒णमे॒व । प्रा॒णमिति॑ प्र - अ॒नम् । ए॒वास्या᳚म् । अ॒स्या॒म् द॒धा॒ति॒ । द॒धा॒त्यथो᳚ । अथो᳚ प्राजाप॒त्यः । अथो॒ इत्यथो᳚ । प्रा॒जा॒प॒त्यो वै । प्रा॒जा॒प॒त्य इति॑ प्राजा - प॒त्यः । वा अश्वः॑ । अश्वः॑ प्र॒जाप॑तिना । प्र॒जाप॑तिनै॒व । प्र॒जाप॑ति॒नेति॑ प्र॒जा - प॒ति॒ना॒ । ए॒वाग्निम् । अ॒ग्निम् चि॑नुते । चि॒नु॒ते॒ स्व॒य॒मा॒तृ॒ण्णा । स्व॒य॒मा॒तृ॒ण्णा भ॑वति । स्व॒य॒मा॒तृ॒ण्णेति॑ स्वयम् - आ॒तृ॒ण्णा । भ॒व॒ति॒ प्रा॒णाना᳚म् । प्रा॒णाना॒मुथ्सृ॑ष्ट्यै । प्रा॒णाना॒मिति॑ प्र - अ॒नाना᳚म् । उथ्सृ॑ष्ट्या॒ अथो᳚ । उथ्सृ॑ष्ट्या॒ इत्युत् - सृ॒ष्ट्यै॒ । अथो॑ सुव॒र्गस्य॑ । अथो॒ इत्यथो᳚ । सु॒व॒र्गस्य॑ लो॒कस्य॑ । सु॒व॒र्गस्येति॑ सुवः - गस्य॑ । लो॒कस्यानु॑ख्यात्यै । अनु॑ख्यात्यै दे॒वाना᳚म् । अनु॑ख्यात्या॒ इत्यनु॑ - ख्या॒त्यै॒ । दे॒वाना॒म् ॅवै । वै सु॑व॒र्गम् । सु॒व॒र्गम् ॅलो॒कम् । सु॒व॒र्गमिति॑ सुवः - गम् । लो॒कम् ॅय॒ताम् । य॒ताम् दिशः॑ । दिशः॒ सम् । सम॑व्लीयन्त । अ॒व्ली॒य॒न्त॒ ते । त ए॒ताः । ए॒ता दिश्याः᳚ । दिश्या॑ अपश्यन्न् । अ॒प॒श्य॒न् ताः । ता उप॑ । उपा॑दधत । अ॒द॒ध॒त॒ ताभिः॑ । ताभि॒र् वै । वै ते । ते दिशः॑ । दिशो॑ऽदृꣳहन्न् । अ॒दृꣳ॒॒ह॒न्॒. यत् । यद् दिश्याः᳚ । दिश्या॑ उप॒दधा॑ति । उ॒प॒दधा॑ति दि॒शाम् । उ॒प॒दधा॒तीत्यु॑प - दधा॑ति । दि॒शाम् ॅविधृ॑त्यै । विधृ॑त्यै॒ दश॑ । विधृ॑त्या॒ इति॒ वि - धृ॒त्यै॒ । दश॑ प्राण॒भृतः॑ । प्रा॒ण॒भृतः॑ पु॒रस्ता᳚त् । प्रा॒ण॒भृत॒ इति॑ प्राण - भृतः॑ । पु॒रस्ता॒दुप॑ । उप॑ दधाति \newline

\textbf{Jatai Paata} \newline

1. घ्रा॒प॒य॒ति॒ प्रा॒णम् प्रा॒णम् घ्रा॑पयति घ्रापयति प्रा॒णम् । \newline
2. प्रा॒ण मे॒वैव प्रा॒णम् प्रा॒ण मे॒व । \newline
3. प्रा॒णमिति॑ प्र - अ॒नम् । \newline
4. ए॒वास्या॑ मस्या मे॒वै वास्या᳚म् । \newline
5. अ॒स्या॒म् द॒धा॒ति॒ द॒धा॒ त्य॒स्या॒ म॒स्या॒म् द॒धा॒ति॒ । \newline
6. द॒धा॒ त्यथो॒ अथो॑ दधाति दधा॒ त्यथो᳚ । \newline
7. अथो᳚ प्राजाप॒त्यः प्रा॑जाप॒त्यो ऽथो॒ अथो᳚ प्राजाप॒त्यः । \newline
8. अथो॒ इत्यथो᳚ । \newline
9. प्रा॒जा॒प॒त्यो वै वै प्रा॑जाप॒त्यः प्रा॑जाप॒त्यो वै । \newline
10. प्रा॒जा॒प॒त्य इति॑ प्राजा - प॒त्यः । \newline
11. वा अश्वो ऽश्वो॒ वै वा अश्वः॑ । \newline
12. अश्वः॑ प्र॒जाप॑तिना प्र॒जाप॑ति॒ना ऽश्वो ऽश्वः॑ प्र॒जाप॑तिना । \newline
13. प्र॒जाप॑ति नै॒वैव प्र॒जाप॑तिना प्र॒जाप॑ति नै॒व । \newline
14. प्र॒जाप॑ति॒नेति॑ प्र॒जा - प॒ति॒ना॒ । \newline
15. ए॒वाग्नि म॒ग्नि मे॒वै वाग्निम् । \newline
16. अ॒ग्निम् चि॑नुते चिनुते॒ ऽग्नि म॒ग्निम् चि॑नुते । \newline
17. चि॒नु॒ते॒ स्व॒य॒मा॒तृ॒ण्णा स्व॑यमातृ॒ण्णा चि॑नुते चिनुते स्वयमातृ॒ण्णा । \newline
18. स्व॒य॒मा॒तृ॒ण्णा भ॑वति भवति स्वयमातृ॒ण्णा स्व॑यमातृ॒ण्णा भ॑वति । \newline
19. स्व॒य॒मा॒तृ॒ण्णेति॑ स्वयं - आ॒तृ॒ण्णा । \newline
20. भ॒व॒ति॒ प्रा॒णाना᳚म् प्रा॒णाना᳚म् भवति भवति प्रा॒णाना᳚म् । \newline
21. प्रा॒णाना॒ मुथ्सृ॑ष्ट्या॒ उथ्सृ॑ष्ट्यै प्रा॒णाना᳚म् प्रा॒णाना॒ मुथ्सृ॑ष्ट्यै । \newline
22. प्रा॒णाना॒मिति॑ प्र - अ॒नाना᳚म् । \newline
23. उथ्सृ॑ष्ट्या॒ अथो॒ अथो॒ उथ्सृ॑ष्ट्या॒ उथ्सृ॑ष्ट्या॒ अथो᳚ । \newline
24. उथ्सृ॑ष्ट्या॒ इत्युत् - सृ॒ष्ट्यै॒ । \newline
25. अथो॑ सुव॒र्गस्य॑ सुव॒र्ग स्याथो॒ अथो॑ सुव॒र्गस्य॑ । \newline
26. अथो॒ इत्यथो᳚ । \newline
27. सु॒व॒र्गस्य॑ लो॒कस्य॑ लो॒कस्य॑ सुव॒र्गस्य॑ सुव॒र्गस्य॑ लो॒कस्य॑ । \newline
28. सु॒व॒र्गस्येति॑ सुवः - गस्य॑ । \newline
29. लो॒कस्या नु॑ख्यात्या॒ अनु॑ख्यात्यै लो॒कस्य॑ लो॒कस्या नु॑ख्यात्यै । \newline
30. अनु॑ख्यात्यै दे॒वाना᳚म् दे॒वाना॒ मनु॑ख्यात्या॒ अनु॑ख्यात्यै दे॒वाना᳚म् । \newline
31. अनु॑ख्यात्या॒ इत्यनु॑ - ख्या॒त्यै॒ । \newline
32. दे॒वानां॒ ॅवै वै दे॒वाना᳚म् दे॒वानां॒ ॅवै । \newline
33. वै सु॑व॒र्गꣳ सु॑व॒र्गं ॅवै वै सु॑व॒र्गम् । \newline
34. सु॒व॒र्गम् ॅलो॒कम् ॅलो॒कꣳ सु॑व॒र्गꣳ सु॑व॒र्गम् ॅलो॒कम् । \newline
35. सु॒व॒र्गमिति॑ सुवः - गम् । \newline
36. लो॒कं ॅय॒तां ॅय॒ताम् ॅलो॒कम् ॅलो॒कं ॅय॒ताम् । \newline
37. य॒ताम् दिशो॒ दिशो॑ य॒तां ॅय॒ताम् दिशः॑ । \newline
38. दिशः॒ सꣳ सम् दिशो॒ दिशः॒ सम् । \newline
39. स म॑व्लीयन्ता व्लीयन्त॒ सꣳ स म॑व्लीयन्त । \newline
40. अ॒व्ली॒य॒न्त॒ ते ते᳚ ऽव्लीयन्ता व्लीयन्त॒ ते । \newline
41. त ए॒ता ए॒ता स्ते त ए॒ताः । \newline
42. ए॒ता दिश्या॒ दिश्या॑ ए॒ता ए॒ता दिश्याः᳚ । \newline
43. दिश्या॑ अपश्यन् नपश्य॒न् दिश्या॒ दिश्या॑ अपश्यन्न् । \newline
44. अ॒प॒श्य॒न् ता स्ता अ॑पश्यन् नपश्य॒न् ताः । \newline
45. ता उपोप॒ ता स्ता उप॑ । \newline
46. उपा॑ दधता दध॒तो पोपा॑ दधत । \newline
47. अ॒द॒ध॒त॒ ताभि॒ स्ताभि॑ रदधता दधत॒ ताभिः॑ । \newline
48. ताभि॒र् वै वै ताभि॒ स्ताभि॒र् वै । \newline
49. वै ते ते वै वै ते । \newline
50. ते दिशो॒ दिश॒ स्ते ते दिशः॑ । \newline
51. दिशो॑ ऽदृꣳहन् नदृꣳह॒न् दिशो॒ दिशो॑ ऽदृꣳहन्न् । \newline
52. अ॒दृꣳ॒॒ह॒न्॒. यद् यद॑दृꣳहन् नदृꣳह॒न्॒. यत् । \newline
53. यद् दिश्या॒ दिश्या॒ यद् यद् दिश्याः᳚ । \newline
54. दिश्या॑ उप॒दधा᳚ त्युप॒दधा॑ति॒ दिश्या॒ दिश्या॑ उप॒दधा॑ति । \newline
55. उ॒प॒दधा॑ति दि॒शाम् दि॒शा मु॑प॒दधा᳚ त्युप॒दधा॑ति दि॒शाम् । \newline
56. उ॒प॒दधा॒तीत्यु॑प - दधा॑ति । \newline
57. दि॒शां ॅविधृ॑त्यै॒ विधृ॑त्यै दि॒शाम् दि॒शां ॅविधृ॑त्यै । \newline
58. विधृ॑त्यै॒ दश॒ दश॒ विधृ॑त्यै॒ विधृ॑त्यै॒ दश॑ । \newline
59. विधृ॑त्या॒ इति॒ वि - धृ॒त्यै॒ । \newline
60. दश॑ प्राण॒भृतः॑ प्राण॒भृतो॒ दश॒ दश॑ प्राण॒भृतः॑ । \newline
61. प्रा॒ण॒भृतः॑ पु॒रस्ता᳚त् पु॒रस्ता᳚त् प्राण॒भृतः॑ प्राण॒भृतः॑ पु॒रस्ता᳚त् । \newline
62. प्रा॒ण॒भृत॒ इति॑ प्राण - भृतः॑ । \newline
63. पु॒रस्ता॒ दुपोप॑ पु॒रस्ता᳚त् पु॒रस्ता॒ दुप॑ । \newline
64. उप॑ दधाति दधा॒ त्युपोप॑ दधाति । \newline

\textbf{Ghana Paata } \newline

1. घ्रा॒प॒य॒ति॒ प्रा॒णम् प्रा॒णम् घ्रा॑पयति घ्रापयति प्रा॒ण मे॒वैव प्रा॒णम् घ्रा॑पयति घ्रापयति प्रा॒ण मे॒व । \newline
2. प्रा॒ण मे॒वैव प्रा॒णम् प्रा॒ण मे॒वास्या॑ मस्या मे॒व प्रा॒णम् प्रा॒ण मे॒वास्या᳚म् । \newline
3. प्रा॒णमिति॑ प्र - अ॒नम् । \newline
4. ए॒वास्या॑ मस्या मे॒वै वास्या᳚म् दधाति दधा त्यस्या मे॒वै वास्या᳚म् दधाति । \newline
5. अ॒स्या॒म् द॒धा॒ति॒ द॒धा॒ त्य॒स्या॒ म॒स्या॒म् द॒धा॒ त्यथो॒ अथो॑ दधा त्यस्या मस्याम् दधा॒ त्यथो᳚ । \newline
6. द॒धा॒ त्यथो॒ अथो॑ दधाति दधा॒ त्यथो᳚ प्राजाप॒त्यः प्रा॑जाप॒त्यो ऽथो॑ दधाति दधा॒ त्यथो᳚ प्राजाप॒त्यः । \newline
7. अथो᳚ प्राजाप॒त्यः प्रा॑जाप॒त्यो ऽथो॒ अथो᳚ प्राजाप॒त्यो वै वै प्रा॑जाप॒त्यो ऽथो॒ अथो᳚ प्राजाप॒त्यो वै । \newline
8. अथो॒ इत्यथो᳚ । \newline
9. प्रा॒जा॒प॒त्यो वै वै प्रा॑जाप॒त्यः प्रा॑जाप॒त्यो वा अश्वो ऽश्वो॒ वै प्रा॑जाप॒त्यः प्रा॑जाप॒त्यो वा अश्वः॑ । \newline
10. प्रा॒जा॒प॒त्य इति॑ प्राजा - प॒त्यः । \newline
11. वा अश्वो ऽश्वो॒ वै वा अश्वः॑ प्र॒जाप॑तिना प्र॒जाप॑ति॒ना ऽश्वो॒ वै वा अश्वः॑ प्र॒जाप॑तिना । \newline
12. अश्वः॑ प्र॒जाप॑तिना प्र॒जाप॑ति॒ना ऽश्वो ऽश्वः॑ प्र॒जाप॑ति नै॒वैव प्र॒जाप॑ति॒ना ऽश्वो ऽश्वः॑ प्र॒जाप॑ति नै॒व । \newline
13. प्र॒जाप॑ति नै॒वैव प्र॒जाप॑तिना प्र॒जाप॑ति नै॒वाग्नि म॒ग्नि मे॒व प्र॒जाप॑तिना प्र॒जाप॑ति नै॒वाग्निम् । \newline
14. प्र॒जाप॑ति॒नेति॑ प्र॒जा - प॒ति॒ना॒ । \newline
15. ए॒वाग्नि म॒ग्नि मे॒वै वाग्निम् चि॑नुते चिनुते॒ ऽग्नि मे॒वै वाग्निम् चि॑नुते । \newline
16. अ॒ग्निम् चि॑नुते चिनुते॒ ऽग्नि म॒ग्निम् चि॑नुते स्वयमातृ॒ण्णा स्व॑यमातृ॒ण्णा चि॑नुते॒ ऽग्नि म॒ग्निम् चि॑नुते स्वयमातृ॒ण्णा । \newline
17. चि॒नु॒ते॒ स्व॒य॒मा॒तृ॒ण्णा स्व॑यमातृ॒ण्णा चि॑नुते चिनुते स्वयमातृ॒ण्णा भ॑वति भवति स्वयमातृ॒ण्णा चि॑नुते चिनुते स्वयमातृ॒ण्णा भ॑वति । \newline
18. स्व॒य॒मा॒तृ॒ण्णा भ॑वति भवति स्वयमातृ॒ण्णा स्व॑यमातृ॒ण्णा भ॑वति प्रा॒णाना᳚म् प्रा॒णाना᳚म् भवति स्वयमातृ॒ण्णा स्व॑यमातृ॒ण्णा भ॑वति प्रा॒णाना᳚म् । \newline
19. स्व॒य॒मा॒तृ॒ण्णेति॑ स्वयं - आ॒तृ॒ण्णा । \newline
20. भ॒व॒ति॒ प्रा॒णाना᳚म् प्रा॒णाना᳚म् भवति भवति प्रा॒णाना॒ मुथ्सृ॑ष्ट्या॒ उथ्सृ॑ष्ट्यै प्रा॒णाना᳚म् भवति भवति प्रा॒णाना॒ मुथ्सृ॑ष्ट्यै । \newline
21. प्रा॒णाना॒ मुथ्सृ॑ष्ट्या॒ उथ्सृ॑ष्ट्यै प्रा॒णाना᳚म् प्रा॒णाना॒ मुथ्सृ॑ष्ट्या॒ अथो॒ अथो॒ उथ्सृ॑ष्ट्यै प्रा॒णाना᳚म् प्रा॒णाना॒ मुथ्सृ॑ष्ट्या॒ अथो᳚ । \newline
22. प्रा॒णाना॒मिति॑ प्र - अ॒नाना᳚म् । \newline
23. उथ्सृ॑ष्ट्या॒ अथो॒ अथो॒ उथ्सृ॑ष्ट्या॒ उथ्सृ॑ष्ट्या॒ अथो॑ सुव॒र्गस्य॑ सुव॒र्ग स्याथो॒ उथ्सृ॑ष्ट्या॒ उथ्सृ॑ष्ट्या॒ अथो॑ सुव॒र्गस्य॑ । \newline
24. उथ्सृ॑ष्ट्या॒ इत्युत् - सृ॒ष्ट्यै॒ । \newline
25. अथो॑ सुव॒र्गस्य॑ सुव॒र्ग स्याथो॒ अथो॑ सुव॒र्गस्य॑ लो॒कस्य॑ लो॒कस्य॑ सुव॒र्ग स्याथो॒ अथो॑ सुव॒र्गस्य॑ लो॒कस्य॑ । \newline
26. अथो॒ इत्यथो᳚ । \newline
27. सु॒व॒र्गस्य॑ लो॒कस्य॑ लो॒कस्य॑ सुव॒र्गस्य॑ सुव॒र्गस्य॑ लो॒कस्या नु॑ख्यात्या॒ अनु॑ख्यात्यै लो॒कस्य॑ सुव॒र्गस्य॑ सुव॒र्गस्य॑ लो॒कस्या नु॑ख्यात्यै । \newline
28. सु॒व॒र्गस्येति॑ सुवः - गस्य॑ । \newline
29. लो॒कस्यानु॑ ख्यात्या॒ अनु॑ख्यात्यै लो॒कस्य॑ लो॒कस्या नु॑ख्यात्यै दे॒वाना᳚म् दे॒वाना॒ मनु॑ख्यात्यै लो॒कस्य॑ लो॒कस्या नु॑ख्यात्यै दे॒वाना᳚म् । \newline
30. अनु॑ख्यात्यै दे॒वाना᳚म् दे॒वाना॒ मनु॑ख्यात्या॒ अनु॑ख्यात्यै दे॒वानां॒ ॅवै वै दे॒वाना॒ मनु॑ख्यात्या॒ अनु॑ख्यात्यै दे॒वानां॒ ॅवै । \newline
31. अनु॑ख्यात्या॒ इत्यनु॑ - ख्या॒त्यै॒ । \newline
32. दे॒वानां॒ ॅवै वै दे॒वाना᳚म् दे॒वानां॒ ॅवै सु॑व॒र्गꣳ सु॑व॒र्गं ॅवै दे॒वाना᳚म् दे॒वानां॒ ॅवै सु॑व॒र्गम् । \newline
33. वै सु॑व॒र्गꣳ सु॑व॒र्गं ॅवै वै सु॑व॒र्गम् ॅलो॒कम् ॅलो॒कꣳ सु॑व॒र्गं ॅवै वै सु॑व॒र्गम् ॅलो॒कम् । \newline
34. सु॒व॒र्गम् ॅलो॒कम् ॅलो॒कꣳ सु॑व॒र्गꣳ सु॑व॒र्गम् ॅलो॒कं ॅय॒तां ॅय॒ताम् ॅलो॒कꣳ सु॑व॒र्गꣳ सु॑व॒र्गम् ॅलो॒कं ॅय॒ताम् । \newline
35. सु॒व॒र्गमिति॑ सुवः - गम् । \newline
36. लो॒कं ॅय॒तां ॅय॒ताम् ॅलो॒कम् ॅलो॒कं ॅय॒ताम् दिशो॒ दिशो॑ य॒ताम् ॅलो॒कम् ॅलो॒कं ॅय॒ताम् दिशः॑ । \newline
37. य॒ताम् दिशो॒ दिशो॑ य॒तां ॅय॒ताम् दिशः॒ सꣳ सम् दिशो॑ य॒तां ॅय॒ताम् दिशः॒ सम् । \newline
38. दिशः॒ सꣳ सम् दिशो॒ दिशः॒ स म॑व्लीयन्ता व्लीयन्त॒ सम् दिशो॒ दिशः॒ स म॑व्लीयन्त । \newline
39. स म॑व्लीयन्ता व्लीयन्त॒ सꣳ स म॑व्लीयन्त॒ ते ते᳚ ऽव्लीयन्त॒ सꣳ स म॑व्लीयन्त॒ ते । \newline
40. अ॒व्ली॒य॒न्त॒ ते ते᳚ ऽव्लीयन्ता व्लीयन्त॒ त ए॒ता ए॒ता स्ते᳚ ऽव्लीयन्ता व्लीयन्त॒ त ए॒ताः । \newline
41. त ए॒ता ए॒ता स्ते त ए॒ता दिश्या॒ दिश्या॑ ए॒ता स्ते त ए॒ता दिश्याः᳚ । \newline
42. ए॒ता दिश्या॒ दिश्या॑ ए॒ता ए॒ता दिश्या॑ अपश्यन् नपश्य॒न् दिश्या॑ ए॒ता ए॒ता दिश्या॑ अपश्यन्न् । \newline
43. दिश्या॑ अपश्यन् नपश्य॒न् दिश्या॒ दिश्या॑ अपश्य॒न् ता स्ता अ॑पश्य॒न् दिश्या॒ दिश्या॑ अपश्य॒न् ताः । \newline
44. अ॒प॒श्य॒न् ता स्ता अ॑पश्यन् नपश्य॒न् ता उपोप॒ ता अ॑पश्यन् नपश्य॒न् ता उप॑ । \newline
45. ता उपोप॒ ता स्ता उपा॑दधता दध॒तोप॒ ता स्ता उपा॑दधत । \newline
46. उपा॑दधता दध॒ तोपोपा॑ दधत॒ ताभि॒ स्ताभि॑ रदध॒ तोपोपा॑ दधत॒ ताभिः॑ । \newline
47. अ॒द॒ध॒त॒ ताभि॒ स्ताभि॑ रदधता दधत॒ ताभि॒र् वै वै ताभि॑ रदधता दधत॒ ताभि॒र् वै । \newline
48. ताभि॒र् वै वै ताभि॒ स्ताभि॒र् वै ते ते वै ताभि॒ स्ताभि॒र् वै ते । \newline
49. वै ते ते वै वै ते दिशो॒ दिश॒ स्ते वै वै ते दिशः॑ । \newline
50. ते दिशो॒ दिश॒ स्ते ते दिशो॑ ऽदृꣳहन् नदृꣳह॒न् दिश॒ स्ते ते दिशो॑ ऽदृꣳहन्न् । \newline
51. दिशो॑ ऽदृꣳहन् नदृꣳह॒न् दिशो॒ दिशो॑ ऽदृꣳह॒न्॒. यद् यद॑दृꣳह॒न् दिशो॒ दिशो॑ ऽदृꣳह॒न्॒. यत् । \newline
52. अ॒दृꣳ॒॒ह॒न्॒. यद् यद॑दृꣳहन् नदृꣳह॒न्॒. यद् दिश्या॒ दिश्या॒ यद॑दृꣳहन् नदृꣳह॒न्॒. यद् दिश्याः᳚ । \newline
53. यद् दिश्या॒ दिश्या॒ यद् यद् दिश्या॑ उप॒दधा᳚ त्युप॒दधा॑ति॒ दिश्या॒ यद् यद् दिश्या॑ उप॒दधा॑ति । \newline
54. दिश्या॑ उप॒दधा᳚ त्युप॒दधा॑ति॒ दिश्या॒ दिश्या॑ उप॒दधा॑ति दि॒शाम् दि॒शा मु॑प॒दधा॑ति॒ दिश्या॒ दिश्या॑ उप॒दधा॑ति दि॒शाम् । \newline
55. उ॒प॒दधा॑ति दि॒शाम् दि॒शा मु॑प॒दधा᳚ त्युप॒दधा॑ति दि॒शां ॅविधृ॑त्यै॒ विधृ॑त्यै दि॒शा मु॑प॒दधा᳚ त्युप॒दधा॑ति दि॒शां ॅविधृ॑त्यै । \newline
56. उ॒प॒दधा॒तीत्यु॑प - दधा॑ति । \newline
57. दि॒शां ॅविधृ॑त्यै॒ विधृ॑त्यै दि॒शाम् दि॒शां ॅविधृ॑त्यै॒ दश॒ दश॒ विधृ॑त्यै दि॒शाम् दि॒शां ॅविधृ॑त्यै॒ दश॑ । \newline
58. विधृ॑त्यै॒ दश॒ दश॒ विधृ॑त्यै॒ विधृ॑त्यै॒ दश॑ प्राण॒भृतः॑ प्राण॒भृतो॒ दश॒ विधृ॑त्यै॒ विधृ॑त्यै॒ दश॑ प्राण॒भृतः॑ । \newline
59. विधृ॑त्या॒ इति॒ वि - धृ॒त्यै॒ । \newline
60. दश॑ प्राण॒भृतः॑ प्राण॒भृतो॒ दश॒ दश॑ प्राण॒भृतः॑ पु॒रस्ता᳚त् पु॒रस्ता᳚त् प्राण॒भृतो॒ दश॒ दश॑ प्राण॒भृतः॑ पु॒रस्ता᳚त् । \newline
61. प्रा॒ण॒भृतः॑ पु॒रस्ता᳚त् पु॒रस्ता᳚त् प्राण॒भृतः॑ प्राण॒भृतः॑ पु॒रस्ता॒ दुपोप॑ पु॒रस्ता᳚त् प्राण॒भृतः॑ प्राण॒भृतः॑ पु॒रस्ता॒ दुप॑ । \newline
62. प्रा॒ण॒भृत॒ इति॑ प्राण - भृतः॑ । \newline
63. पु॒रस्ता॒ दुपोप॑ पु॒रस्ता᳚त् पु॒रस्ता॒ दुप॑ दधाति दधा॒ त्युप॑ पु॒रस्ता᳚त् पु॒रस्ता॒ दुप॑ दधाति । \newline
64. उप॑ दधाति दधा॒ त्युपोप॑ दधाति॒ नव॒ नव॑ दधा॒ त्युपोप॑ दधाति॒ नव॑ । \newline
\pagebreak
\markright{ TS 5.3.2.3  \hfill https://www.vedavms.in \hfill}

\section{ TS 5.3.2.3 }

\textbf{TS 5.3.2.3 } \newline
\textbf{Samhita Paata} \newline

दधाति॒ नव॒ वै पुरु॑षे प्रा॒णा नाभि॑र्दश॒मी प्रा॒णाने॒व पु॒रस्ता᳚द्धत्ते॒ तस्मा᳚त् पु॒रस्ता᳚त् प्रा॒णा ज्योति॑ष्मती-मुत्त॒मामुप॑ दधाति॒ तस्मा᳚त् प्रा॒णानां॒ ॅवाग्ज्योति॑रुत्त॒मा दशोप॑ दधाति॒ दशा᳚क्षरा वि॒राड् वि॒राट् छन्द॑सां॒ ज्योति॒र्ज्योति॑रे॒व पु॒रस्ता᳚द्धत्ते॒ तस्मा᳚त् पु॒रस्ता॒ज्ज्योति॒रुपा᳚ ऽऽ*स्महे॒ छन्दाꣳ॑सि प॒शुष्वा॒जिम॑यु॒स्तान् बृ॑ह॒त्युद॑जय॒त् तस्मा॒द्-बार्.ह॑ताः - [  ] \newline

\textbf{Pada Paata} \newline

द॒धा॒ति॒ । नव॑ । वै । पुरु॑षे । प्रा॒णा इति॑ प्र - अ॒नाः । नाभिः॑ । द॒श॒मी । प्रा॒णानिति॑ प्र - अ॒नान् । ए॒व । पु॒रस्ता᳚त् । ध॒त्ते॒ । तस्मा᳚त् । पु॒रस्ता᳚त् । प्रा॒णा इति॑ प्र - अ॒नाः । ज्योति॑ष्मतीम् । उ॒त्त॒मामित्यु॑त् - त॒माम् । उपेति॑ । द॒धा॒ति॒ । तस्मा᳚त् । प्रा॒णाना॒मिति॑ प्र - अ॒नाना᳚म् । वाक् । ज्योतिः॑ । उ॒त्त॒मेत्यु॑त् - त॒मा । दश॑ । उपेति॑ । द॒धा॒ति॒ । दशा᳚क्ष॒रेति॒ दश॑ - अ॒क्ष॒रा॒ । वि॒राडिति॑ वि - राट् । वि॒राडिति॑ वि - राट् । छन्द॑साम् । ज्योतिः॑ । ज्योतिः॑ । ए॒व । पु॒रस्ता᳚त् । ध॒त्ते॒ । तस्मा᳚त् । पु॒रस्ता᳚त् । ज्योतिः॑ । उपेति॑ । आ॒स्म॒हे॒ । छन्दाꣳ॑सि । प॒शुषु॑ । आ॒जिम् । अ॒युः॒ । तान् । बृ॒ह॒ती । उदिति॑ । अ॒ज॒य॒त् । तस्मा᳚त् । बार्.ह॑ताः ।  \newline


\textbf{Krama Paata} \newline

द॒धा॒ति॒ नव॑ । नव॒ वै । वै पुरु॑षे । पुरु॑षे प्रा॒णाः । प्रा॒णा नाभिः॑ । प्रा॒णा इति॑ प्र - अ॒नाः । नाभि॑र् दश॒मी । द॒श॒मी प्रा॒णान् । प्रा॒णाने॒व । प्रा॒णानिति॑ प्र - अ॒नान् । ए॒व पु॒रस्ता᳚त् । पु॒रस्ता᳚द् धत्ते । ध॒त्ते॒ तस्मा᳚त् । तस्मा᳚त् पु॒रस्ता᳚त् । पु॒रस्ता᳚त् प्रा॒णाः । प्रा॒णा ज्योति॑ष्मतीम् । प्रा॒णा इति॑ प्र - अ॒नाः । ज्योति॑ष्मतीमुत्त॒माम् । उ॒त्त॒मामुप॑ । उ॒त्त॒मामित्यु॑त् - त॒माम् । उप॑ दधाति । द॒धा॒ति॒ तस्मा᳚त् । तस्मा᳚त् प्रा॒णाना᳚म् । प्रा॒णाना॒म् ॅवाक् । प्रा॒णाना॒मिति॑ प्र - अ॒नाना᳚म् । वाग् ज्योतिः॑ । ज्योति॑रुत्त॒मा । उ॒त्त॒मा दश॑ । उ॒त्त॒मेत्यु॑त् - त॒मा । दशोप॑ । उप॑ दधाति । द॒धा॒ति॒ दशा᳚क्षरा । दशा᳚क्षरा वि॒राट् । दशा᳚क्ष॒रेति॒ दश॑ - अ॒क्ष॒रा॒ । वि॒राड् वि॒राट् । वि॒राडिति॑ वि - राट् । वि॒राट् छन्द॑साम् । वि॒राडिति॑ वि - राट् । छन्द॑सा॒म् ज्योतिः॑ । ज्योति॒र् ज्योतिः॑ । ज्योति॑रे॒व । ए॒व पु॒रस्ता᳚त् । पु॒रस्ता᳚द् धत्ते । ध॒त्ते॒ तस्मा᳚त् । तस्मा᳚त् पु॒रस्ता᳚त् । पु॒रस्ता॒ज् ज्योतिः॑ । ज्योति॒रुप॑ । उपा᳚स्महे । आ॒स्म॒हे॒ छन्दाꣳ॑सि । छन्दाꣳ॑सि प॒शुषु॑ । प॒शुष्वा॒जिम् । आ॒जिम॑युः । अ॒यु॒स्तान् । तान् बृ॑ह॒ती । बृ॒ह॒त्युत् । उद॑जयत् । अ॒ज॒य॒त् तस्मा᳚त् । तस्मा॒द् बार्.ह॑ताः । बार्.ह॑ताः प॒शवः॑ \newline

\textbf{Jatai Paata} \newline

1. द॒धा॒ति॒ नव॒ नव॑ दधाति दधाति॒ नव॑ । \newline
2. नव॒ वै वै नव॒ नव॒ वै । \newline
3. वै पुरु॑षे॒ पुरु॑षे॒ वै वै पुरु॑षे । \newline
4. पुरु॑षे प्रा॒णाः प्रा॒णाः पुरु॑षे॒ पुरु॑षे प्रा॒णाः । \newline
5. प्रा॒णा नाभि॒र् नाभिः॑ प्रा॒णाः प्रा॒णा नाभिः॑ । \newline
6. प्रा॒णा इति॑ प्र - अ॒नाः । \newline
7. नाभि॑र् दश॒मी द॑श॒मी नाभि॒र् नाभि॑र् दश॒मी । \newline
8. द॒श॒मी प्रा॒णान् प्रा॒णान् द॑श॒मी द॑श॒मी प्रा॒णान् । \newline
9. प्रा॒णा ने॒वैव प्रा॒णान् प्रा॒णा ने॒व । \newline
10. प्रा॒णानिति॑ प्र - अ॒नान् । \newline
11. ए॒व पु॒रस्ता᳚त् पु॒रस्ता॑ दे॒वैव पु॒रस्ता᳚त् । \newline
12. पु॒रस्ता᳚द् धत्ते धत्ते पु॒रस्ता᳚त् पु॒रस्ता᳚द् धत्ते । \newline
13. ध॒त्ते॒ तस्मा॒त् तस्मा᳚द् धत्ते धत्ते॒ तस्मा᳚त् । \newline
14. तस्मा᳚त् पु॒रस्ता᳚त् पु॒रस्ता॒त् तस्मा॒त् तस्मा᳚त् पु॒रस्ता᳚त् । \newline
15. पु॒रस्ता᳚त् प्रा॒णाः प्रा॒णाः पु॒रस्ता᳚त् पु॒रस्ता᳚त् प्रा॒णाः । \newline
16. प्रा॒णा ज्योति॑ष्मती॒म् ज्योति॑ष्मतीम् प्रा॒णाः प्रा॒णा ज्योति॑ष्मतीम् । \newline
17. प्रा॒णा इति॑ प्र - अ॒नाः । \newline
18. ज्योति॑ष्मती मुत्त॒मा मु॑त्त॒माम् ज्योति॑ष्मती॒म् ज्योति॑ष्मती मुत्त॒माम् । \newline
19. उ॒त्त॒मा मुपोपो᳚ त्त॒मा मु॑त्त॒मा मुप॑ । \newline
20. उ॒त्त॒मामित्यु॑त् - त॒माम् । \newline
21. उप॑ दधाति दधा॒ त्युपोप॑ दधाति । \newline
22. द॒धा॒ति॒ तस्मा॒त् तस्मा᳚द् दधाति दधाति॒ तस्मा᳚त् । \newline
23. तस्मा᳚त् प्रा॒णाना᳚म् प्रा॒णाना॒म् तस्मा॒त् तस्मा᳚त् प्रा॒णाना᳚म् । \newline
24. प्रा॒णानां॒ ॅवाग् वाक् प्रा॒णाना᳚म् प्रा॒णानां॒ ॅवाक् । \newline
25. प्रा॒णाना॒मिति॑ प्र - अ॒नाना᳚म् । \newline
26. वाग् ज्योति॒र् ज्योति॒र् वाग् वाग् ज्योतिः॑ । \newline
27. ज्योति॑ रुत्त॒मो त्त॒मा ज्योति॒र् ज्योति॑ रुत्त॒मा । \newline
28. उ॒त्त॒मा दश॒ दशो᳚ त्त॒मो त्त॒मा दश॑ । \newline
29. उ॒त्त॒मेत्यु॑त् - त॒मा । \newline
30. दशो पोप॒ दश॒ दशोप॑ । \newline
31. उप॑ दधाति दधा॒ त्युपोप॑ दधाति । \newline
32. द॒धा॒ति॒ दशा᳚क्षरा॒ दशा᳚क्षरा दधाति दधाति॒ दशा᳚क्षरा । \newline
33. दशा᳚क्षरा वि॒राड् वि॒राड् दशा᳚क्षरा॒ दशा᳚क्षरा वि॒राट् । \newline
34. दशा᳚क्ष॒रेति॒ दश॑ - अ॒क्ष॒रा॒ । \newline
35. वि॒राड् वि॒राट् । \newline
36. वि॒राडिति॑ वि - राट् । \newline
37. वि॒राट् छन्द॑सा॒म् छन्द॑सां ॅवि॒राड् वि॒राट् छन्द॑साम् । \newline
38. वि॒राडिति॑ वि - राट् । \newline
39. छन्द॑सा॒म् ज्योति॒र् ज्योति॒ श्छन्द॑सा॒म् छन्द॑सा॒म् ज्योतिः॑ । \newline
40. ज्योति॒र् ज्योतिः॑ । \newline
41. ज्योति॑ रे॒वैव ज्योति॒र् ज्योति॑ रे॒व । \newline
42. ए॒व पु॒रस्ता᳚त् पु॒रस्ता॑ दे॒वैव पु॒रस्ता᳚त् । \newline
43. पु॒रस्ता᳚द् धत्ते धत्ते पु॒रस्ता᳚त् पु॒रस्ता᳚द् धत्ते । \newline
44. ध॒त्ते॒ तस्मा॒त् तस्मा᳚द् धत्ते धत्ते॒ तस्मा᳚त् । \newline
45. तस्मा᳚त् पु॒रस्ता᳚त् पु॒रस्ता॒त् तस्मा॒त् तस्मा᳚त् पु॒रस्ता᳚त् । \newline
46. पु॒रस्ता॒ज् ज्योति॒र् ज्योतिः॑ पु॒रस्ता᳚त् पु॒रस्ता॒ज् ज्योतिः॑ । \newline
47. ज्योति॒ रुपोप॒ ज्योति॒र् ज्योति॒ रुप॑ । \newline
48. उपा᳚स्मह आस्मह॒ उपोपा᳚ स्महे । \newline
49. आ॒स्म॒हे॒ छन्दाꣳ॑सि॒ छन्दाꣳ॑ स्यास्मह आस्महे॒ छन्दाꣳ॑सि । \newline
50. छन्दाꣳ॑सि प॒शुषु॑ प॒शुषु॒ छन्दाꣳ॑सि॒ छन्दाꣳ॑सि प॒शुषु॑ । \newline
51. प॒शु ष्वा॒जि मा॒जिम् प॒शुषु॑ प॒शु ष्वा॒जिम् । \newline
52. आ॒जि म॑यु रयु रा॒जि मा॒जि म॑युः । \newline
53. अ॒यु॒ स्ताꣳ स्ता न॑यु रयु॒स्तान् । \newline
54. तान् बृ॑ह॒ती बृ॑ह॒ती ताꣳ स्तान् बृ॑ह॒ती । \newline
55. बृ॒ह॒ त्युदुद् बृ॑ह॒ती बृ॑ह॒ त्युत् । \newline
56. उद॑जय दजय॒ दुदु द॑जयत् । \newline
57. अ॒ज॒य॒त् तस्मा॒त् तस्मा॑ दजय दजय॒त् तस्मा᳚त् । \newline
58. तस्मा॒द् बार्.ह॑ता॒ बार्.ह॑ता॒ स्तस्मा॒त् तस्मा॒द् बार्.ह॑ताः । \newline
59. बार्.ह॑ताः प॒शवः॑ प॒शवो॒ बार्.ह॑ता॒ बार्.ह॑ताः प॒शवः॑ । \newline

\textbf{Ghana Paata } \newline

1. द॒धा॒ति॒ नव॒ नव॑ दधाति दधाति॒ नव॒ वै वै नव॑ दधाति दधाति॒ नव॒ वै । \newline
2. नव॒ वै वै नव॒ नव॒ वै पुरु॑षे॒ पुरु॑षे॒ वै नव॒ नव॒ वै पुरु॑षे । \newline
3. वै पुरु॑षे॒ पुरु॑षे॒ वै वै पुरु॑षे प्रा॒णाः प्रा॒णाः पुरु॑षे॒ वै वै पुरु॑षे प्रा॒णाः । \newline
4. पुरु॑षे प्रा॒णाः प्रा॒णाः पुरु॑षे॒ पुरु॑षे प्रा॒णा नाभि॒र् नाभिः॑ प्रा॒णाः पुरु॑षे॒ पुरु॑षे प्रा॒णा नाभिः॑ । \newline
5. प्रा॒णा नाभि॒र् नाभिः॑ प्रा॒णाः प्रा॒णा नाभि॑र् दश॒मी द॑श॒मी नाभिः॑ प्रा॒णाः प्रा॒णा नाभि॑र् दश॒मी । \newline
6. प्रा॒णा इति॑ प्र - अ॒नाः । \newline
7. नाभि॑र् दश॒मी द॑श॒मी नाभि॒र् नाभि॑र् दश॒मी प्रा॒णान् प्रा॒णान् द॑श॒मी नाभि॒र् नाभि॑र् दश॒मी प्रा॒णान् । \newline
8. द॒श॒मी प्रा॒णान् प्रा॒णान् द॑श॒मी द॑श॒मी प्रा॒णा ने॒वैव प्रा॒णान् द॑श॒मी द॑श॒मी प्रा॒णा ने॒व । \newline
9. प्रा॒णा ने॒वैव प्रा॒णान् प्रा॒णा ने॒व पु॒रस्ता᳚त् पु॒रस्ता॑ दे॒व प्रा॒णान् प्रा॒णा ने॒व पु॒रस्ता᳚त् । \newline
10. प्रा॒णानिति॑ प्र - अ॒नान् । \newline
11. ए॒व पु॒रस्ता᳚त् पु॒रस्ता॑ दे॒वैव पु॒रस्ता᳚द् धत्ते धत्ते पु॒रस्ता॑ दे॒वैव पु॒रस्ता᳚द् धत्ते । \newline
12. पु॒रस्ता᳚द् धत्ते धत्ते पु॒रस्ता᳚त् पु॒रस्ता᳚द् धत्ते॒ तस्मा॒त् तस्मा᳚द् धत्ते पु॒रस्ता᳚त् पु॒रस्ता᳚द् धत्ते॒ तस्मा᳚त् । \newline
13. ध॒त्ते॒ तस्मा॒त् तस्मा᳚द् धत्ते धत्ते॒ तस्मा᳚त् पु॒रस्ता᳚त् पु॒रस्ता॒त् तस्मा᳚द् धत्ते धत्ते॒ तस्मा᳚त् पु॒रस्ता᳚त् । \newline
14. तस्मा᳚त् पु॒रस्ता᳚त् पु॒रस्ता॒त् तस्मा॒त् तस्मा᳚त् पु॒रस्ता᳚त् प्रा॒णाः प्रा॒णाः पु॒रस्ता॒त् तस्मा॒त् तस्मा᳚त् पु॒रस्ता᳚त् प्रा॒णाः । \newline
15. पु॒रस्ता᳚त् प्रा॒णाः प्रा॒णाः पु॒रस्ता᳚त् पु॒रस्ता᳚त् प्रा॒णा ज्योति॑ष्मती॒म् ज्योति॑ष्मतीम् प्रा॒णाः पु॒रस्ता᳚त् पु॒रस्ता᳚त् प्रा॒णा ज्योति॑ष्मतीम् । \newline
16. प्रा॒णा ज्योति॑ष्मती॒म् ज्योति॑ष्मतीम् प्रा॒णाः प्रा॒णा ज्योति॑ष्मती मुत्त॒मा मु॑त्त॒माम् ज्योति॑ष्मतीम् प्रा॒णाः प्रा॒णा ज्योति॑ष्मती मुत्त॒माम् । \newline
17. प्रा॒णा इति॑ प्र - अ॒नाः । \newline
18. ज्योति॑ष्मती मुत्त॒मा मु॑त्त॒माम् ज्योति॑ष्मती॒म् ज्योति॑ष्मती मुत्त॒मा मुपोपो᳚ त्त॒माम् ज्योति॑ष्मती॒म् ज्योति॑ष्मती मुत्त॒मा मुप॑ । \newline
19. उ॒त्त॒मा मुपोपो᳚ त्त॒मा मु॑त्त॒मा मुप॑ दधाति दधा॒ त्युपो᳚ त्त॒मा मु॑त्त॒मा मुप॑ दधाति । \newline
20. उ॒त्त॒मामित्यु॑त् - त॒माम् । \newline
21. उप॑ दधाति दधा॒ त्युपोप॑ दधाति॒ तस्मा॒त् तस्मा᳚द् दधा॒ त्युपोप॑ दधाति॒ तस्मा᳚त् । \newline
22. द॒धा॒ति॒ तस्मा॒त् तस्मा᳚द् दधाति दधाति॒ तस्मा᳚त् प्रा॒णाना᳚म् प्रा॒णाना॒म् तस्मा᳚द् दधाति दधाति॒ तस्मा᳚त् प्रा॒णाना᳚म् । \newline
23. तस्मा᳚त् प्रा॒णाना᳚म् प्रा॒णाना॒म् तस्मा॒त् तस्मा᳚त् प्रा॒णानां॒ ॅवाग् वाक् प्रा॒णाना॒म् तस्मा॒त् तस्मा᳚त् प्रा॒णानां॒ ॅवाक् । \newline
24. प्रा॒णानां॒ ॅवाग् वाक् प्रा॒णाना᳚म् प्रा॒णानां॒ ॅवाग् ज्योति॒र् ज्योति॒र् वाक् प्रा॒णाना᳚म् प्रा॒णानां॒ ॅवाग् ज्योतिः॑ । \newline
25. प्रा॒णाना॒मिति॑ प्र - अ॒नाना᳚म् । \newline
26. वाग् ज्योति॒र् ज्योति॒र् वाग् वाग् ज्योति॑ रुत्त॒ मोत्त॒मा ज्योति॒र् वाग् वाग् ज्योति॑ रुत्त॒मा । \newline
27. ज्योति॑ रुत्त॒मो त्त॒मा ज्योति॒र् ज्योति॑ रुत्त॒मा दश॒ दशो᳚त्त॒मा ज्योति॒र् ज्योति॑ रुत्त॒मा दश॑ । \newline
28. उ॒त्त॒मा दश॒ दशो᳚त्त॒ मोत्त॒मा दशोपोप॒ दशो᳚त्त॒ मोत्त॒मा दशोप॑ । \newline
29. उ॒त्त॒मेत्यु॑त् - त॒मा । \newline
30. दशोपोप॒ दश॒ दशोप॑ दधाति दधा॒ त्युप॒ दश॒ दशोप॑ दधाति । \newline
31. उप॑ दधाति दधा॒ त्युपोप॑ दधाति॒ दशा᳚क्षरा॒ दशा᳚क्षरा दधा॒ त्युपोप॑ दधाति॒ दशा᳚क्षरा । \newline
32. द॒धा॒ति॒ दशा᳚क्षरा॒ दशा᳚क्षरा दधाति दधाति॒ दशा᳚क्षरा वि॒राड् वि॒राड् दशा᳚क्षरा दधाति दधाति॒ दशा᳚क्षरा वि॒राट् । \newline
33. दशा᳚क्षरा वि॒राड् वि॒राड् दशा᳚क्षरा॒ दशा᳚क्षरा वि॒राट् । \newline
34. दशा᳚क्ष॒रेति॒ दश॑ - अ॒क्ष॒रा॒ । \newline
35. वि॒राड् वि॒राट् । \newline
36. वि॒राडिति॑ वि - राट् । \newline
37. वि॒राट् छन्द॑सा॒म् छन्द॑सां ॅवि॒राड् वि॒राट् छन्द॑सा॒म् ज्योति॒र् ज्योति॒ श्छन्द॑सां ॅवि॒राड् वि॒राट् छन्द॑सा॒म् ज्योतिः॑ । \newline
38. वि॒राडिति॑ वि - राट् । \newline
39. छन्द॑सा॒म् ज्योति॒र् ज्योति॒ श्छन्द॑सा॒म् छन्द॑सा॒म् ज्योतिः॑ । \newline
40. ज्योति॒र् ज्योतिः॑ । \newline
41. ज्योति॑रे॒ वैव ज्योति॒र् ज्योति॑ रे॒व पु॒रस्ता᳚त् पु॒रस्ता॑ दे॒व ज्योति॒र् ज्योति॑ रे॒व पु॒रस्ता᳚त् । \newline
42. ए॒व पु॒रस्ता᳚त् पु॒रस्ता॑ दे॒वैव पु॒रस्ता᳚द् धत्ते धत्ते पु॒रस्ता॑ दे॒वैव पु॒रस्ता᳚द् धत्ते । \newline
43. पु॒रस्ता᳚द् धत्ते धत्ते पु॒रस्ता᳚त् पु॒रस्ता᳚द् धत्ते॒ तस्मा॒त् तस्मा᳚द् धत्ते पु॒रस्ता᳚त् पु॒रस्ता᳚द् धत्ते॒ तस्मा᳚त् । \newline
44. ध॒त्ते॒ तस्मा॒त् तस्मा᳚द् धत्ते धत्ते॒ तस्मा᳚त् पु॒रस्ता᳚त् पु॒रस्ता॒त् तस्मा᳚द् धत्ते धत्ते॒ तस्मा᳚त् पु॒रस्ता᳚त् । \newline
45. तस्मा᳚त् पु॒रस्ता᳚त् पु॒रस्ता॒त् तस्मा॒त् तस्मा᳚त् पु॒रस्ता॒ज् ज्योति॒र् ज्योतिः॑ पु॒रस्ता॒त् तस्मा॒त् तस्मा᳚त् पु॒रस्ता॒ज् ज्योतिः॑ । \newline
46. पु॒रस्ता॒ज् ज्योति॒र् ज्योतिः॑ पु॒रस्ता᳚त् पु॒रस्ता॒ज् ज्योति॒ रुपोप॒ ज्योतिः॑ पु॒रस्ता᳚त् पु॒रस्ता॒ज् ज्योति॒ रुप॑ । \newline
47. ज्योति॒ रुपोप॒ ज्योति॒र् ज्योति॒ रुपा᳚ स्मह आस्मह॒ उप॒ ज्योति॒र् ज्योति॒रुपा᳚ स्महे । \newline
48. उपा᳚स्मह आस्मह॒ उपोपा᳚ स्महे॒ छन्दाꣳ॑सि॒ छन्दाꣳ॑ स्यास्मह॒ उपोपा᳚ स्महे॒ छन्दाꣳ॑सि । \newline
49. आ॒स्म॒हे॒ छन्दाꣳ॑सि॒ छन्दाꣳ॑ स्यास्मह आस्महे॒ छन्दाꣳ॑सि प॒शुषु॑ प॒शुषु॒ छन्दाꣳ॑ स्यास्मह आस्महे॒ छन्दाꣳ॑सि प॒शुषु॑ । \newline
50. छन्दाꣳ॑सि प॒शुषु॑ प॒शुषु॒ छन्दाꣳ॑सि॒ छन्दाꣳ॑सि प॒शुष्वा॒जि मा॒जिम् प॒शुषु॒ छन्दाꣳ॑सि॒ छन्दाꣳ॑सि प॒शुष्वा॒जिम् । \newline
51. प॒शु ष्वा॒जि मा॒जिम् प॒शुषु॑ प॒शुष्वा॒जि म॑यु रयु रा॒जिम् प॒शुषु॑ प॒शु ष्वा॒जि म॑युः । \newline
52. आ॒जि म॑यु रयु रा॒जि मा॒जि म॑यु॒ स्ताꣳ स्ता न॑यु रा॒जि मा॒जि म॑यु॒स्तान् । \newline
53. अ॒यु॒ स्ताꣳ स्ता न॑युरयु॒ स्तान् बृ॑ह॒ती बृ॑ह॒ती ता न॑यु रयु॒स्तान् बृ॑ह॒ती । \newline
54. तान् बृ॑ह॒ती बृ॑ह॒ती ताꣳ स्तान् बृ॑ह॒ त्युदुद् बृ॑ह॒ती ताꣳ स्तान् बृ॑ह॒त्युत् । \newline
55. बृ॒ह॒ त्युदुद् बृ॑ह॒ती बृ॑ह॒त्यु द॑जय दजय॒दुद् बृ॑ह॒ती बृ॑ह॒ त्युद॑जयत् । \newline
56. उद॑जय दजय॒ दुदु द॑जय॒त् तस्मा॒त् तस्मा॑ दजय॒दुदु द॑जय॒त् तस्मा᳚त् । \newline
57. अ॒ज॒य॒त् तस्मा॒त् तस्मा॑ दजय दजय॒त् तस्मा॒द् बार्.ह॑ता॒ बार्.ह॑ता॒ स्तस्मा॑ दजय दजय॒त् तस्मा॒द् बार्.ह॑ताः । \newline
58. तस्मा॒द् बार्.ह॑ता॒ बार्.ह॑ता॒ स्तस्मा॒त् तस्मा॒द् बार्.ह॑ताः प॒शवः॑ प॒शवो॒ बार्.ह॑ता॒ स्तस्मा॒त् तस्मा॒द् बार्.ह॑ताः प॒शवः॑ । \newline
59. बार्.ह॑ताः प॒शवः॑ प॒शवो॒ बार्.ह॑ता॒ बार्.ह॑ताः प॒शव॑ उच्यन्त उच्यन्ते प॒शवो॒ बार्.ह॑ता॒ बार्.ह॑ताः प॒शव॑ उच्यन्ते । \newline
\pagebreak
\markright{ TS 5.3.2.4  \hfill https://www.vedavms.in \hfill}

\section{ TS 5.3.2.4 }

\textbf{TS 5.3.2.4 } \newline
\textbf{Samhita Paata} \newline

प॒शव॑ उच्यन्ते॒ मा छन्द॒ इति॑ दक्षिण॒त उप॑ दधाति॒ तस्मा᳚द्-दक्षि॒णा वृ॑तो॒ मासाः᳚ पृथि॒वी छन्द॒ इति॑ प॒श्चात् प्रति॑ष्ठित्या अ॒ग्निर्दे॒वतेत्यु॑त्तर॒त ओजो॒ वा अ॒ग्निरोज॑ ए॒वोत्त॑र॒तो ध॑त्ते॒ तस्मा॑दुत्तरतो ऽभिप्रया॒यी ज॑यति॒ षट्त्रिꣳ॑श॒थ् संप॑द्यन्ते॒ षट्त्रिꣳ॑शदक्षरा बृह॒ती बार्.ह॑ताः प॒शवो॑ बृह॒त्यैवास्मै॑ प॒शूनव॑ रुन्धे बृह॒ती छन्द॑साꣳ॒॒ स्वारा᳚ज्यं॒ परी॑याय॒ यस्यै॒ता - [  ] \newline

\textbf{Pada Paata} \newline

प॒शवः॑ । उ॒च्य॒न्ते॒ । मा । छन्दः॑ । इति॑ । द॒क्षि॒ण॒तः । उपेति॑ । द॒धा॒ति॒ । तस्मा᳚त् । द॒क्षि॒णावृ॑त॒ इति॑ दक्षि॒णा - आ॒वृ॒तः॒ । मासाः᳚ । पृ॒थि॒वी । छन्दः॑ । इति॑ । प॒श्चात् । प्रति॑ष्ठित्या॒ इति॒ प्रति॑ - स्थि॒त्यै॒ । अ॒ग्निः । दे॒वता᳚ । इति॑ । उ॒त्त॒र॒त इत्यु॑त् - त॒र॒तः । ओजः॑ । वै । अ॒ग्निः । ओजः॑ । ए॒व । उ॒त्त॒र॒त इत्यु॑त्-त॒र॒तः । ध॒त्ते॒ । तस्मा᳚त् । उ॒त्त॒र॒तो॒ऽभि॒प्र॒या॒यीत्यु॑त्तरतः - अ॒भि॒प्र॒या॒यी । ज॒य॒ति॒ । षट्त्रिꣳ॑श॒दिति॒ षट् - त्रिꣳ॒॒श॒त् । समिति॑ । प॒द्य॒न्ते॒ । षट्त्रिꣳ॑शदक्ष॒रेति॒ षट्त्रिꣳ॑शत् -   अ॒क्ष॒रा॒ । बृ॒ह॒ती । बार्.ह॑ताः । प॒शवः॑ । बृ॒ह॒त्या । ए॒व । अ॒स्मै॒ । प॒शून् । अवेति॑ । रु॒न्धे॒ । बृ॒ह॒ती । छन्द॑साम् । स्वारा᳚ज्य॒मिति॒ स्व - रा॒ज्य॒म् । परीति॑ । इ॒या॒य॒ । यस्य॑ । ए॒ताः ।  \newline


\textbf{Krama Paata} \newline

प॒शव॑ उच्यन्ते । उ॒च्य॒न्ते॒ मा । मा छन्दः॑ । छन्द॒ इति॑ । इति॑ दक्षिण॒तः । द॒क्षि॒ण॒त उप॑ । उप॑ दधाति । द॒धा॒ति॒ तस्मा᳚त् । तस्मा᳚द् दक्षि॒णावृ॑तः । द॒क्षि॒णावृ॑तो॒ मासाः᳚ । द॒क्षि॒णावृ॑त॒ इति॑ दक्षि॒णा - आ॒वृ॒तः॒ । मासाः᳚ पृथि॒वी । पृ॒थि॒वी छन्दः॑ । छन्द॒ इति॑ । इति॑ प॒श्चात् । प॒श्चात् प्रति॑ष्ठित्यै । प्रति॑ष्ठित्या अ॒ग्निः । प्रति॑ष्ठित्या॒ इति॒ प्रति॑ - स्थि॒त्यै॒ । अ॒ग्निर् दे॒वता᳚ । दे॒वतेति॑ । इत्यु॑त्तर॒तः । उ॒त्त॒र॒त ओजः॑ । उ॒त्त॒र॒त इत्यु॑त् - त॒र॒तः । ओजो॒ वै । वा अ॒ग्निः । अ॒ग्निरोजः॑ । ओज॑ ए॒व । ए॒वोत्त॑र॒तः । उ॒त्त॒र॒तो ध॑त्ते । उ॒त्त॒र॒त इत्यु॑त् - त॒र॒तः । ध॒त्ते॒ तस्मा᳚त् । तस्मा॑दुत्तरतोभिप्रया॒यी । उ॒त्त॒र॒तो॒भि॒प्र॒या॒यी ज॑यति । उ॒त्त॒र॒तो॒भि॒प्र॒या॒यीत्यु॑त्तरतः - अ॒भि॒प्र॒या॒यी । ज॒य॒ति॒ षट्त्रिꣳ॑शत् । षट्त्रिꣳ॑श॒थ् सम् । षट्त्रिꣳ॑श॒दिति॒ षट् - त्रिꣳ॒॒श॒त्॒ । सम्प॑द्यन्ते । प॒द्य॒न्ते॒ षट्त्रिꣳ॑शदक्षरा । षट्त्रिꣳ॑शदक्षरा बृह॒ती । षट्त्रꣳ॑शदक्ष॒रेति॒ षट्त्रिꣳ॑शत् - अ॒क्ष॒रा॒ । बृ॒ह॒ती बार्.ह॑ताः । बार्.ह॑ताः प॒शवः॑ । प॒शवो॑ बृह॒त्या । बृ॒ह॒त्यैव । ए॒वास्मै᳚ । अ॒स्मै॒ प॒शून् । प॒शूनव॑ । अव॑ रुन्धे । रु॒न्धे॒ बृ॒ह॒ती । बृ॒ह॒ती छन्द॑साम् । छन्द॑साꣳ॒॒ स्वारा᳚ज्यम् । स्वारा᳚ज्य॒म् परि॑ । स्वारा᳚ज्य॒मिति॒ स्व - रा॒ज्य॒म् । परी॑याय । इ॒या॒य॒ यस्य॑ । यस्यै॒ताः ( ) । ए॒ता उ॑पधी॒यन्ते᳚ \newline

\textbf{Jatai Paata} \newline

1. प॒शव॑ उच्यन्त उच्यन्ते प॒शवः॑ प॒शव॑ उच्यन्ते । \newline
2. उ॒च्य॒न्ते॒ मा मोच्य॑न्त उच्यन्ते॒ मा । \newline
3. मा छन्द॒ श्छन्दो॒ मा मा छन्दः॑ । \newline
4. छन्द॒ इतीति॒ छन्द॒ श्छन्द॒ इति॑ । \newline
5. इति॑ दक्षिण॒तो द॑क्षिण॒त इतीति॑ दक्षिण॒तः । \newline
6. द॒क्षि॒ण॒त उपोप॑ दक्षिण॒तो द॑क्षिण॒त उप॑ । \newline
7. उप॑ दधाति दधा॒ त्युपोप॑ दधाति । \newline
8. द॒धा॒ति॒ तस्मा॒त् तस्मा᳚द् दधाति दधाति॒ तस्मा᳚त् । \newline
9. तस्मा᳚द् दक्षि॒णावृ॑तो दक्षि॒णावृ॑त॒ स्तस्मा॒त् तस्मा᳚द् दक्षि॒णावृ॑तः । \newline
10. द॒क्षि॒णावृ॑तो॒ मासा॒ मासा॑ दक्षि॒णावृ॑तो दक्षि॒णावृ॑तो॒ मासाः᳚ । \newline
11. द॒क्षि॒णावृ॑त॒ इति॑ दक्षि॒णा - आ॒वृ॒तः॒ । \newline
12. मासाः᳚ पृथि॒वी पृ॑थि॒वी मासा॒ मासाः᳚ पृथि॒वी । \newline
13. पृ॒थि॒वी छन्द॒ श्छन्दः॑ पृथि॒वी पृ॑थि॒वी छन्दः॑ । \newline
14. छन्द॒ इतीति॒ छन्द॒ श्छन्द॒ इति॑ । \newline
15. इति॑ प॒श्चात् प॒श्चा दितीति॑ प॒श्चात् । \newline
16. प॒श्चात् प्रति॑ष्ठित्यै॒ प्रति॑ष्ठित्यै प॒श्चात् प॒श्चात् प्रति॑ष्ठित्यै । \newline
17. प्रति॑ष्ठित्या अ॒ग्नि र॒ग्निः प्रति॑ष्ठित्यै॒ प्रति॑ष्ठित्या अ॒ग्निः । \newline
18. प्रति॑ष्ठित्या॒ इति॒ प्रति॑ - स्थि॒त्यै॒ । \newline
19. अ॒ग्निर् दे॒वता॑ दे॒वता॒ ऽग्नि र॒ग्निर् दे॒वता᳚ । \newline
20. दे॒व तेतीति॑ दे॒वता॑ दे॒व तेति॑ । \newline
21. इत्यु॑त्तर॒त उ॑त्तर॒त इती त्यु॑त्तर॒तः । \newline
22. उ॒त्त॒र॒त ओज॒ ओज॑ उत्तर॒त उ॑त्तर॒त ओजः॑ । \newline
23. उ॒त्त॒र॒त इत्यु॑त् - त॒र॒तः । \newline
24. ओजो॒ वै वा ओज॒ ओजो॒ वै । \newline
25. वा अ॒ग्नि र॒ग्निर् वै वा अ॒ग्निः । \newline
26. अ॒ग्नि रोज॒ ओजो॒ ऽग्नि र॒ग्नि रोजः॑ । \newline
27. ओज॑ ए॒वै वौज॒ ओज॑ ए॒व । \newline
28. ए॒वोत्त॑र॒त उ॑त्तर॒त ए॒वै वोत्त॑र॒तः । \newline
29. उ॒त्त॒र॒तो ध॑त्ते धत्त उत्तर॒त उ॑त्तर॒तो ध॑त्ते । \newline
30. उ॒त्त॒र॒त इत्यु॑त् - त॒र॒तः । \newline
31. ध॒त्ते॒ तस्मा॒त् तस्मा᳚द् धत्ते धत्ते॒ तस्मा᳚त् । \newline
32. तस्मा॑ दुत्तरतोभिप्रया॒ य्यु॑त्तरतोभिप्रया॒यी तस्मा॒त् तस्मा॑ दुत्तरतोभिप्रया॒यी । \newline
33. उ॒त्त॒र॒तो॒भि॒प्र॒या॒यी ज॑यति जय त्युत्तरतोभिप्रया॒ य्यु॑त्तरतोभिप्रया॒यी ज॑यति । \newline
34. उ॒त्त॒र॒तो॒भि॒प्र॒या॒यीत्यु॑त्तरतः - अ॒भि॒प्र॒या॒यी । \newline
35. ज॒य॒ति॒ षट्त्रिꣳ॑श॒ थ्षट्त्रिꣳ॑शज् जयति जयति॒ षट्त्रिꣳ॑शत् । \newline
36. षट्त्रिꣳ॑श॒थ् सꣳ सꣳ षट्त्रिꣳ॑श॒ थ्षट्त्रिꣳ॑श॒थ् सम् । \newline
37. षट्त्रिꣳ॑श॒दिति॒ षट् - त्रिꣳ॒॒श॒त् । \newline
38. सम् प॑द्यन्ते पद्यन्ते॒ सꣳ सम् प॑द्यन्ते । \newline
39. प॒द्य॒न्ते॒ षट्त्रिꣳ॑शदक्षरा॒ षट्त्रिꣳ॑शदक्षरा पद्यन्ते पद्यन्ते॒ षट्त्रिꣳ॑शदक्षरा । \newline
40. षट्त्रिꣳ॑शदक्षरा बृह॒ती बृ॑ह॒ती षट्त्रिꣳ॑शदक्षरा॒ षट्त्रिꣳ॑शदक्षरा बृह॒ती । \newline
41. षट्त्रिꣳ॑शदक्ष॒रेति॒ षट्त्रिꣳ॑शत् - अ॒क्ष॒रा॒ । \newline
42. बृ॒ह॒ती बार्.ह॑ता॒ बार्.ह॑ता बृह॒ती बृ॑ह॒ती बार्.ह॑ताः । \newline
43. बार्.ह॑ताः प॒शवः॑ प॒शवो॒ बार्.ह॑ता॒ बार्.ह॑ताः प॒शवः॑ । \newline
44. प॒शवो॑ बृह॒त्या बृ॑ह॒त्या प॒शवः॑ प॒शवो॑ बृह॒त्या । \newline
45. बृ॒ह॒त्यैवैव बृ॑ह॒त्या बृ॑ह॒त्यैव । \newline
46. ए॒वास्मा॑ अस्मा ए॒वै वास्मै᳚ । \newline
47. अ॒स्मै॒ प॒शून् प॒शू न॑स्मा अस्मै प॒शून् । \newline
48. प॒शू नवाव॑ प॒शून् प॒शू नव॑ । \newline
49. अव॑ रुन्धे रु॒न्धे ऽवाव॑ रुन्धे । \newline
50. रु॒न्धे॒ बृ॒ह॒ती बृ॑ह॒ती रु॑न्धे रुन्धे बृह॒ती । \newline
51. बृ॒ह॒ती छन्द॑सा॒म् छन्द॑साम् बृह॒ती बृ॑ह॒ती छन्द॑साम् । \newline
52. छन्द॑साꣳ॒॒ स्वारा᳚ज्यꣳ॒॒ स्वारा᳚ज्य॒म् छन्द॑सा॒म् छन्द॑साꣳ॒॒ स्वारा᳚ज्यम् । \newline
53. स्वारा᳚ज्य॒म् परि॒ परि॒ स्वारा᳚ज्यꣳ॒॒ स्वारा᳚ज्य॒म् परि॑ । \newline
54. स्वारा᳚ज्य॒मिति॒ स्व - रा॒ज्य॒म् । \newline
55. परी॑ यायेयाय॒ परि॒ परी॑याय । \newline
56. इ॒या॒य॒ यस्य॒ यस्ये॑याये याय॒ यस्य॑ । \newline
57. यस्यै॒ता ए॒ता यस्य॒ यस्यै॒ताः । \newline
58. ए॒ता उ॑पधी॒यन्त॑ उपधी॒यन्त॑ ए॒ता ए॒ता उ॑पधी॒यन्ते᳚ । \newline

\textbf{Ghana Paata } \newline

1. प॒शव॑ उच्यन्त उच्यन्ते प॒शवः॑ प॒शव॑ उच्यन्ते॒ मा मोच्य॑न्ते प॒शवः॑ प॒शव॑ उच्यन्ते॒ मा । \newline
2. उ॒च्य॒न्ते॒ मा मोच्य॑न्त उच्यन्ते॒ मा छन्द॒ श्छन्दो॒ मोच्य॑न्त उच्यन्ते॒ मा छन्दः॑ । \newline
3. मा छन्द॒ श्छन्दो॒ मा मा छन्द॒ इतीति॒ छन्दो॒ मा मा छन्द॒ इति॑ । \newline
4. छन्द॒ इतीति॒ छन्द॒ श्छन्द॒ इति॑ दक्षिण॒तो द॑क्षिण॒त इति॒ छन्द॒ श्छन्द॒ इति॑ दक्षिण॒तः । \newline
5. इति॑ दक्षिण॒तो द॑क्षिण॒त इतीति॑ दक्षिण॒त उपोप॑ दक्षिण॒त इतीति॑ दक्षिण॒त उप॑ । \newline
6. द॒क्षि॒ण॒त उपोप॑ दक्षिण॒तो द॑क्षिण॒त उप॑ दधाति दधा॒ त्युप॑ दक्षिण॒तो द॑क्षिण॒त उप॑ दधाति । \newline
7. उप॑ दधाति दधा॒ त्युपोप॑ दधाति॒ तस्मा॒त् तस्मा᳚द् दधा॒ त्युपोप॑ दधाति॒ तस्मा᳚त् । \newline
8. द॒धा॒ति॒ तस्मा॒त् तस्मा᳚द् दधाति दधाति॒ तस्मा᳚द् दक्षि॒णावृ॑तो दक्षि॒णावृ॑त॒ स्तस्मा᳚द् दधाति दधाति॒ तस्मा᳚द् दक्षि॒णावृ॑तः । \newline
9. तस्मा᳚द् दक्षि॒णावृ॑तो दक्षि॒णावृ॑त॒ स्तस्मा॒त् तस्मा᳚द् दक्षि॒णावृ॑तो॒ मासा॒ मासा॑ दक्षि॒णावृ॑त॒ स्तस्मा॒त् तस्मा᳚द् दक्षि॒णावृ॑तो॒ मासाः᳚ । \newline
10. द॒क्षि॒णावृ॑तो॒ मासा॒ मासा॑ दक्षि॒णावृ॑तो दक्षि॒णावृ॑तो॒ मासाः᳚ पृथि॒वी पृ॑थि॒वी मासा॑ दक्षि॒णावृ॑तो दक्षि॒णावृ॑तो॒ मासाः᳚ पृथि॒वी । \newline
11. द॒क्षि॒णावृ॑त॒ इति॑ दक्षि॒णा - आ॒वृ॒तः॒ । \newline
12. मासाः᳚ पृथि॒वी पृ॑थि॒वी मासा॒ मासाः᳚ पृथि॒वी छन्द॒ श्छन्दः॑ पृथि॒वी मासा॒ मासाः᳚ पृथि॒वी छन्दः॑ । \newline
13. पृ॒थि॒वी छन्द॒ श्छन्दः॑ पृथि॒वी पृ॑थि॒वी छन्द॒ इतीति॒ छन्दः॑ पृथि॒वी पृ॑थि॒वी छन्द॒ इति॑ । \newline
14. छन्द॒ इतीति॒ छन्द॒ श्छन्द॒ इति॑ प॒श्चात् प॒श्चा दिति॒ छन्द॒ श्छन्द॒ इति॑ प॒श्चात् । \newline
15. इति॑ प॒श्चात् प॒श्चा दितीति॑ प॒श्चात् प्रति॑ष्ठित्यै॒ प्रति॑ष्ठित्यै प॒श्चा दितीति॑ प॒श्चात् प्रति॑ष्ठित्यै । \newline
16. प॒श्चात् प्रति॑ष्ठित्यै॒ प्रति॑ष्ठित्यै प॒श्चात् प॒श्चात् प्रति॑ष्ठित्या अ॒ग्नि र॒ग्निः प्रति॑ष्ठित्यै प॒श्चात् प॒श्चात् प्रति॑ष्ठित्या अ॒ग्निः । \newline
17. प्रति॑ष्ठित्या अ॒ग्नि र॒ग्निः प्रति॑ष्ठित्यै॒ प्रति॑ष्ठित्या अ॒ग्निर् दे॒वता॑ दे॒वता॒ ऽग्निः प्रति॑ष्ठित्यै॒ प्रति॑ष्ठित्या अ॒ग्निर् दे॒वता᳚ । \newline
18. प्रति॑ष्ठित्या॒ इति॒ प्रति॑ - स्थि॒त्यै॒ । \newline
19. अ॒ग्निर् दे॒वता॑ दे॒वता॒ ऽग्नि र॒ग्निर् दे॒व तेतीति॑ दे॒वता॒ ऽग्नि र॒ग्निर् दे॒व तेति॑ । \newline
20. दे॒व तेतीति॑ दे॒वता॑ दे॒वते त्यु॑त्तर॒त उ॑त्तर॒त इति॑ दे॒वता॑ दे॒वते त्यु॑त्तर॒तः । \newline
21. इत्यु॑ त्तर॒त उ॑त्तर॒त इती त्यु॑त्तर॒त ओज॒ ओज॑ उत्तर॒त इती त्यु॑त्तर॒त ओजः॑ । \newline
22. उ॒त्त॒र॒त ओज॒ ओज॑ उत्तर॒त उ॑त्तर॒त ओजो॒ वै वा ओज॑ उत्तर॒त उ॑त्तर॒त ओजो॒ वै । \newline
23. उ॒त्त॒र॒त इत्यु॑त् - त॒र॒तः । \newline
24. ओजो॒ वै वा ओज॒ ओजो॒ वा अ॒ग्नि र॒ग्निर् वा ओज॒ ओजो॒ वा अ॒ग्निः । \newline
25. वा अ॒ग्नि र॒ग्निर् वै वा अ॒ग्नि रोज॒ ओजो॒ ऽग्निर् वै वा अ॒ग्नि रोजः॑ । \newline
26. अ॒ग्नि रोज॒ ओजो॒ ऽग्नि र॒ग्नि रोज॑ ए॒वै वौजो॒ ऽग्नि र॒ग्नि रोज॑ ए॒व । \newline
27. ओज॑ ए॒वै वौज॒ ओज॑ ए॒वोत्त॑र॒त उ॑त्तर॒त ए॒वौज॒ ओज॑ ए॒वोत्त॑र॒तः । \newline
28. ए॒वोत्त॑र॒त उ॑त्तर॒त ए॒वै वोत्त॑र॒तो ध॑त्ते धत्त उत्तर॒त ए॒वै वोत्त॑र॒तो ध॑त्ते । \newline
29. उ॒त्त॒र॒तो ध॑त्ते धत्त उत्तर॒त उ॑त्तर॒तो ध॑त्ते॒ तस्मा॒त् तस्मा᳚द् धत्त उत्तर॒त उ॑त्तर॒तो ध॑त्ते॒ तस्मा᳚त् । \newline
30. उ॒त्त॒र॒त इत्यु॑त् - त॒र॒तः । \newline
31. ध॒त्ते॒ तस्मा॒त् तस्मा᳚द् धत्ते धत्ते॒ तस्मा॑ दुत्तरतोभिप्रया॒य्यु॑ त्तरतोभिप्रया॒यी तस्मा᳚द् धत्ते धत्ते॒ तस्मा॑ दुत्तरतोभिप्रया॒यी । \newline
32. तस्मा॑ दुत्तरतोभिप्रया॒ य्यु॑त्तरतोभिप्रया॒यी तस्मा॒त् तस्मा॑ दुत्तरतोभिप्रया॒यी ज॑यति जयत्यु त्तरतोभिप्रया॒यी तस्मा॒त् तस्मा॑ दुत्तरतोभिप्रया॒यी ज॑यति । \newline
33. उ॒त्त॒र॒तो॒भि॒प्र॒या॒यी ज॑यति जयत्युत्तरतोभिप्रया॒ य्यु॑त्तरतोभिप्रया॒यी ज॑यति॒ षट्त्रिꣳ॑श॒ थ्षट्त्रिꣳ॑शज् जयत्यु त्तरतोभिप्रया॒ य्यु॑त्तरतोभिप्रया॒यी ज॑यति॒ षट्त्रिꣳ॑शत् । \newline
34. उ॒त्त॒र॒तो॒भि॒प्र॒या॒यीत्यु॑त्तरतः - अ॒भि॒प्र॒या॒यी । \newline
35. ज॒य॒ति॒ षट्त्रिꣳ॑श॒थ् षट्त्रिꣳ॑शज् जयति जयति॒ षट्त्रिꣳ॑श॒थ् सꣳ सꣳ षट्त्रिꣳ॑शज् जयति जयति॒ षट्त्रिꣳ॑श॒थ् सम् । \newline
36. षट्त्रिꣳ॑श॒थ् सꣳ सꣳ षट्त्रिꣳ॑श॒ थ्षट्त्रिꣳ॑श॒थ् सम् प॑द्यन्ते पद्यन्ते॒ सꣳ षट्त्रिꣳ॑श॒ थ्षट्त्रिꣳ॑श॒थ् सम् प॑द्यन्ते । \newline
37. षट्त्रिꣳ॑श॒दिति॒ षट् - त्रिꣳ॒॒श॒त् । \newline
38. सम् प॑द्यन्ते पद्यन्ते॒ सꣳ सम् प॑द्यन्ते॒ षट्त्रिꣳ॑शदक्षरा॒ षट्त्रिꣳ॑शदक्षरा पद्यन्ते॒ सꣳ सम् प॑द्यन्ते॒ षट्त्रिꣳ॑शदक्षरा । \newline
39. प॒द्य॒न्ते॒ षट्त्रिꣳ॑शदक्षरा॒ षट्त्रिꣳ॑शदक्षरा पद्यन्ते पद्यन्ते॒ षट्त्रिꣳ॑शदक्षरा बृह॒ती बृ॑ह॒ती षट्त्रिꣳ॑शदक्षरा पद्यन्ते पद्यन्ते॒ षट्त्रिꣳ॑शदक्षरा बृह॒ती । \newline
40. षट्त्रिꣳ॑शदक्षरा बृह॒ती बृ॑ह॒ती षट्त्रिꣳ॑शदक्षरा॒ षट्त्रिꣳ॑शदक्षरा बृह॒ती बार्.ह॑ता॒ बार्.ह॑ता बृह॒ती षट्त्रिꣳ॑शदक्षरा॒ षट्त्रिꣳ॑शदक्षरा बृह॒ती बार्.ह॑ताः । \newline
41. षट्त्रिꣳ॑शदक्ष॒रेति॒ षट्त्रिꣳ॑शत् - अ॒क्ष॒रा॒ । \newline
42. बृ॒ह॒ती बार्.ह॑ता॒ बार्.ह॑ता बृह॒ती बृ॑ह॒ती बार्.ह॑ताः प॒शवः॑ प॒शवो॒ बार्.ह॑ता बृह॒ती बृ॑ह॒ती बार्.ह॑ताः प॒शवः॑ । \newline
43. बार्.ह॑ताः प॒शवः॑ प॒शवो॒ बार्.ह॑ता॒ बार्.ह॑ताः प॒शवो॑ बृह॒त्या बृ॑ह॒त्या प॒शवो॒ बार्.ह॑ता॒ बार्.ह॑ताः प॒शवो॑ बृह॒त्या । \newline
44. प॒शवो॑ बृह॒त्या बृ॑ह॒त्या प॒शवः॑ प॒शवो॑ बृह॒त्यै वैव बृ॑ह॒त्या प॒शवः॑ प॒शवो॑ बृह॒त्यैव । \newline
45. बृ॒ह॒त्यै वैव बृ॑ह॒त्या बृ॑ह॒त्यै वास्मा॑ अस्मा ए॒व बृ॑ह॒त्या बृ॑ह॒त्यै वास्मै᳚ । \newline
46. ए॒वास्मा॑ अस्मा ए॒वै वास्मै॑ प॒शून् प॒शू न॑स्मा ए॒वै वास्मै॑ प॒शून् । \newline
47. अ॒स्मै॒ प॒शून् प॒शू न॑स्मा अस्मै प॒शू नवाव॑ प॒शू न॑स्मा अस्मै प॒शू नव॑ । \newline
48. प॒शू नवाव॑ प॒शून् प॒शू नव॑ रुन्धे रु॒न्धे ऽव॑ प॒शून् प॒शू नव॑ रुन्धे । \newline
49. अव॑ रुन्धे रु॒न्धे ऽवाव॑ रुन्धे बृह॒ती बृ॑ह॒ती रु॒न्धे ऽवाव॑ रुन्धे बृह॒ती । \newline
50. रु॒न्धे॒ बृ॒ह॒ती बृ॑ह॒ती रु॑न्धे रुन्धे बृह॒ती छन्द॑सा॒म् छन्द॑साम् बृह॒ती रु॑न्धे रुन्धे बृह॒ती छन्द॑साम् । \newline
51. बृ॒ह॒ती छन्द॑सा॒म् छन्द॑साम् बृह॒ती बृ॑ह॒ती छन्द॑साꣳ॒॒ स्वारा᳚ज्यꣳ॒॒ स्वारा᳚ज्य॒म् छन्द॑साम् बृह॒ती बृ॑ह॒ती छन्द॑साꣳ॒॒ स्वारा᳚ज्यम् । \newline
52. छन्द॑साꣳ॒॒ स्वारा᳚ज्यꣳ॒॒ स्वारा᳚ज्य॒म् छन्द॑सा॒म् छन्द॑साꣳ॒॒ स्वारा᳚ज्य॒म् परि॒ परि॒ स्वारा᳚ज्य॒म् छन्द॑सा॒म् छन्द॑साꣳ॒॒ स्वारा᳚ज्य॒म् परि॑ । \newline
53. स्वारा᳚ज्य॒म् परि॒ परि॒ स्वारा᳚ज्यꣳ॒॒ स्वारा᳚ज्य॒म् परी॑ याये याय॒ परि॒ स्वारा᳚ज्यꣳ॒॒ स्वारा᳚ज्य॒म् परी॑ याय । \newline
54. स्वारा᳚ज्य॒मिति॒ स्व - रा॒ज्य॒म् । \newline
55. परी॑याये याय॒ परि॒ परी॑याय॒ यस्य॒ यस्ये॑ याय॒ परि॒ परी॑याय॒ यस्य॑ । \newline
56. इ॒या॒य॒ यस्य॒ यस्ये॑ याये याय॒ यस्यै॒ता ए॒ता यस्ये॑ याये याय॒ यस्यै॒ताः । \newline
57. यस्यै॒ता ए॒ता यस्य॒ यस्यै॒ता उ॑पधी॒यन्त॑ उपधी॒यन्त॑ ए॒ता यस्य॒ यस्यै॒ता उ॑पधी॒यन्ते᳚ । \newline
58. ए॒ता उ॑पधी॒यन्त॑ उपधी॒यन्त॑ ए॒ता ए॒ता उ॑पधी॒यन्ते॒ गच्छ॑ति॒ गच्छ॑ त्युपधी॒यन्त॑ ए॒ता ए॒ता उ॑पधी॒यन्ते॒ गच्छ॑ति । \newline
\pagebreak
\markright{ TS 5.3.2.5  \hfill https://www.vedavms.in \hfill}

\section{ TS 5.3.2.5 }

\textbf{TS 5.3.2.5 } \newline
\textbf{Samhita Paata} \newline

उ॑पधी॒यन्ते॒ गच्छ॑ति॒ स्वारा᳚ज्यꣳ स॒प्त वाल॑खिल्याः पु॒रस्ता॒दुप॑ दधाति स॒प्त प॒श्चाथ् स॒प्त वै शी॑र्.ष॒ण्याः᳚ प्रा॒णा द्वाववा᳚ञ्चौ प्रा॒णानाꣳ॑ सवीर्य॒त्वाय॑ मू॒र्द्धाऽसि॒ राडिति॑ पु॒रस्ता॒दुप॑ दधाति॒ यन्त्री॒ राडिति॑ प॒श्चात् प्रा॒णाने॒वास्मै॑ स॒मीचो॑ दधाति ॥ \newline

\textbf{Pada Paata} \newline

उ॒प॒धी॒यन्त॒ इत्यु॑प-धी॒यन्ते᳚ । गच्छ॑ति । स्वारा᳚ज्य॒मिति॒ स्व-रा॒ज्य॒म् । स॒प्त । वाल॑खिल्या॒ इति॒ वाल॑ - खि॒ल्याः॒ । पु॒रस्ता᳚त् । उपेति॑ । द॒धा॒ति॒ । स॒प्त । प॒श्चात् । स॒प्त । वै । शी॒र्.॒ष॒ण्याः᳚ । प्रा॒णा इति॑ प्र - अ॒नाः । द्वौ । अवा᳚ञ्चौ । प्रा॒णाना॒मिति॑ प्र - अ॒नाना᳚म् । स॒वी॒र्य॒त्वायेति॑ सवीर्य - त्वाय॑ । मू॒द्‌र्धा । अ॒सि॒ । राट् । इति॑ । पु॒रस्ता᳚त् । उपेति॑ । द॒धा॒ति॒ । यन्त्री᳚ । राट् । इति॑ । प॒श्चात् । प्रा॒णानिति॑ प्र - अ॒नान् । ए॒व । अ॒स्मै॒ । स॒मीचः॑ । द॒धा॒ति॒ ॥  \newline


\textbf{Krama Paata} \newline

उ॒प॒धी॒यन्ते॒ गच्छ॑ति । उ॒प॒धी॒यन्त॒ इत्यु॑प - धी॒यन्ते᳚ । गच्छ॑ति॒ स्वारा᳚ज्यम् । स्वारा᳚ज्यꣳ स॒प्त । स्वारा᳚ज्य॒मिति॒ स्व - रा॒ज्य॒म् । स॒प्त वाल॑खिल्याः । वाल॑खिल्याः पु॒रस्ता᳚त् । वाल॑खिल्या॒ इति॒ वाल॑ - खि॒ल्याः॒ । पु॒रस्ता॒दुप॑ । उप॑ दधाति । द॒धा॒ति॒ स॒प्त । स॒प्त प॒श्चात् । प॒श्चाथ् स॒प्त । स॒प्त वै । वै शी॑र्.ष॒ण्याः᳚ । शी॒र्॒.ष॒ण्याः᳚ प्रा॒णाः । प्रा॒णा द्वौ । प्रा॒णा इति॑ प्र - अ॒नाः । द्वाववा᳚ञ्चौ । अवा᳚ञ्चौ प्रा॒णाना᳚म् । प्रा॒णानाꣳ॑ सवीर्य॒त्वाय॑ । प्रा॒णाना॒मिति॑ प्र - अ॒नाना᳚म् । स॒वी॒र्य॒त्वाय॑ मू॒र्द्धा । स॒वी॒र्य॒त्वायेति॑ सवीर्य - त्वाय॑ । मू॒र्द्धा असि॑ । अ॒सि॒ राट् । राडिति॑ । इति॑ पु॒रस्ता᳚त् । पु॒रस्ता॒दुप॑ । उप॑ दधाति । द॒धा॒ति॒ यन्त्री᳚ । यन्त्री॒ राट् । राडिति॑ । इति॑ प॒श्चात् । प॒श्चात् प्रा॒णान् । प्रा॒णाने॒व । प्रा॒णानिति॑ प्र - अ॒नान् । ए॒वास्मै᳚ । अ॒स्मै॒ स॒मीचः॑ । स॒मीचो॑ दधाति । द॒धा॒तीति॑ दधाति । \newline

\textbf{Jatai Paata} \newline

1. उ॒प॒धी॒यन्ते॒ गच्छ॑ति॒ गच्छ॑ त्युपधी॒यन्त॑ उपधी॒यन्ते॒ गच्छ॑ति । \newline
2. उ॒प॒धी॒यन्त॒ इत्यु॑प - धी॒यन्ते᳚ । \newline
3. गच्छ॑ति॒ स्वारा᳚ज्यꣳ॒॒ स्वारा᳚ज्य॒म् गच्छ॑ति॒ गच्छ॑ति॒ स्वारा᳚ज्यम् । \newline
4. स्वारा᳚ज्यꣳ स॒प्त स॒प्त स्वारा᳚ज्यꣳ॒॒ स्वारा᳚ज्यꣳ स॒प्त । \newline
5. स्वारा᳚ज्य॒मिति॒ स्व - रा॒ज्य॒म् । \newline
6. स॒प्त वाल॑खिल्या॒ वाल॑खिल्याः स॒प्त स॒प्त वाल॑खिल्याः । \newline
7. वाल॑खिल्याः पु॒रस्ता᳚त् पु॒रस्ता॒द् वाल॑खिल्या॒ वाल॑खिल्याः पु॒रस्ता᳚त् । \newline
8. वाल॑खिल्या॒ इति॒ वाल॑ - खि॒ल्याः॒ । \newline
9. पु॒रस्ता॒ दुपोप॑ पु॒रस्ता᳚त् पु॒रस्ता॒ दुप॑ । \newline
10. उप॑ दधाति दधा॒ त्युपोप॑ दधाति । \newline
11. द॒धा॒ति॒ स॒प्त स॒प्त द॑धाति दधाति स॒प्त । \newline
12. स॒प्त प॒श्चात् प॒श्चाथ् स॒प्त स॒प्त प॒श्चात् । \newline
13. प॒श्चाथ् स॒प्त स॒प्त प॒श्चात् प॒श्चाथ् स॒प्त । \newline
14. स॒प्त वै वै स॒प्त स॒प्त वै । \newline
15. वै शी॑र्.ष॒ण्याः᳚ शीर्.ष॒ण्या॑ वै वै शी॑र्.ष॒ण्याः᳚ । \newline
16. शी॒र्॒.ष॒ण्याः᳚ प्रा॒णाः प्रा॒णाः शी॑र्.ष॒ण्याः᳚ शीर्.ष॒ण्याः᳚ प्रा॒णाः । \newline
17. प्रा॒णा द्वौ द्वौ प्रा॒णाः प्रा॒णा द्वौ । \newline
18. प्रा॒णा इति॑ प्र - अ॒नाः । \newline
19. द्वा ववा᳚ञ्चा॒ ववा᳚ञ्चौ॒ द्वौ द्वा ववा᳚ञ्चौ । \newline
20. अवा᳚ञ्चौ प्रा॒णाना᳚म् प्रा॒णाना॒ मवा᳚ञ्चा॒ ववा᳚ञ्चौ प्रा॒णाना᳚म् । \newline
21. प्रा॒णानाꣳ॑ सवीर्य॒त्वाय॑ सवीर्य॒त्वाय॑ प्रा॒णाना᳚म् प्रा॒णानाꣳ॑ सवीर्य॒त्वाय॑ । \newline
22. प्रा॒णाना॒मिति॑ प्र - अ॒नाना᳚म् । \newline
23. स॒वी॒र्य॒त्वाय॑ मू॒र्द्धा मू॒र्द्धा स॑वीर्य॒त्वाय॑ सवीर्य॒त्वाय॑ मू॒र्द्धा । \newline
24. स॒वी॒र्य॒त्वायेति॑ सवीर्य - त्वाय॑ । \newline
25. मू॒र्द्धा ऽस्य॑सि मू॒र्द्धा मू॒र्द्धा ऽसि॑ । \newline
26. अ॒सि॒ राड् राड॑ स्यसि॒ राट् । \newline
27. राडितीति॒ राड् राडिति॑ । \newline
28. इति॑ पु॒रस्ता᳚त् पु॒रस्ता॒ दितीति॑ पु॒रस्ता᳚त् । \newline
29. पु॒रस्ता॒ दुपोप॑ पु॒रस्ता᳚त् पु॒रस्ता॒ दुप॑ । \newline
30. उप॑ दधाति दधा॒ त्युपोप॑ दधाति । \newline
31. द॒धा॒ति॒ यन्त्री॒ यन्त्री॑ दधाति दधाति॒ यन्त्री᳚ । \newline
32. यन्त्री॒ राड् राड् यन्त्री॒ यन्त्री॒ राट् । \newline
33. राडितीति॒ राड् राडिति॑ । \newline
34. इति॑ प॒श्चात् प॒श्चा दितीति॑ प॒श्चात् । \newline
35. प॒श्चात् प्रा॒णान् प्रा॒णान् प॒श्चात् प॒श्चात् प्रा॒णान् । \newline
36. प्रा॒णा ने॒वैव प्रा॒णान् प्रा॒णा ने॒व । \newline
37. प्रा॒णानिति॑ प्र - अ॒नान् । \newline
38. ए॒वास्मा॑ अस्मा ए॒वै वास्मै᳚ । \newline
39. अ॒स्मै॒ स॒मीचः॑ स॒मीचो᳚ ऽस्मा अस्मै स॒मीचः॑ । \newline
40. स॒मीचो॑ दधाति दधाति स॒मीचः॑ स॒मीचो॑ दधाति । \newline
41. द॒धा॒तीति॑ दधाति । \newline

\textbf{Ghana Paata } \newline

1. उ॒प॒धी॒यन्ते॒ गच्छ॑ति॒ गच्छ॑ त्युपधी॒यन्त॑ उपधी॒यन्ते॒ गच्छ॑ति॒ स्वारा᳚ज्यꣳ॒॒ स्वारा᳚ज्य॒म् गच्छ॑ त्युपधी॒यन्त॑ उपधी॒यन्ते॒ गच्छ॑ति॒ स्वारा᳚ज्यम् । \newline
2. उ॒प॒धी॒यन्त॒ इत्यु॑प - धी॒यन्ते᳚ । \newline
3. गच्छ॑ति॒ स्वारा᳚ज्यꣳ॒॒ स्वारा᳚ज्य॒म् गच्छ॑ति॒ गच्छ॑ति॒ स्वारा᳚ज्यꣳ स॒प्त स॒प्त स्वारा᳚ज्य॒म् गच्छ॑ति॒ गच्छ॑ति॒ स्वारा᳚ज्यꣳ स॒प्त । \newline
4. स्वारा᳚ज्यꣳ स॒प्त स॒प्त स्वारा᳚ज्यꣳ॒॒ स्वारा᳚ज्यꣳ स॒प्त वाल॑खिल्या॒ वाल॑खिल्याः स॒प्त स्वारा᳚ज्यꣳ॒॒ स्वारा᳚ज्यꣳ स॒प्त वाल॑खिल्याः । \newline
5. स्वारा᳚ज्य॒मिति॒ स्व - रा॒ज्य॒म् । \newline
6. स॒प्त वाल॑खिल्या॒ वाल॑खिल्याः स॒प्त स॒प्त वाल॑खिल्याः पु॒रस्ता᳚त् पु॒रस्ता॒द् वाल॑खिल्याः स॒प्त स॒प्त वाल॑खिल्याः पु॒रस्ता᳚त् । \newline
7. वाल॑खिल्याः पु॒रस्ता᳚त् पु॒रस्ता॒द् वाल॑खिल्या॒ वाल॑खिल्याः पु॒रस्ता॒ दुपोप॑ पु॒रस्ता॒द् वाल॑खिल्या॒ वाल॑खिल्याः पु॒रस्ता॒ दुप॑ । \newline
8. वाल॑खिल्या॒ इति॒ वाल॑ - खि॒ल्याः॒ । \newline
9. पु॒रस्ता॒ दुपोप॑ पु॒रस्ता᳚त् पु॒रस्ता॒ दुप॑ दधाति दधा॒ त्युप॑ पु॒रस्ता᳚त् पु॒रस्ता॒ दुप॑ दधाति । \newline
10. उप॑ दधाति दधा॒ त्युपोप॑ दधाति स॒प्त स॒प्त द॑धा॒ त्युपोप॑ दधाति स॒प्त । \newline
11. द॒धा॒ति॒ स॒प्त स॒प्त द॑धाति दधाति स॒प्त प॒श्चात् प॒श्चाथ् स॒प्त द॑धाति दधाति स॒प्त प॒श्चात् । \newline
12. स॒प्त प॒श्चात् प॒श्चाथ् स॒प्त स॒प्त प॒श्चाथ् स॒प्त स॒प्त प॒श्चाथ् स॒प्त स॒प्त प॒श्चाथ् स॒प्त । \newline
13. प॒श्चाथ् स॒प्त स॒प्त प॒श्चात् प॒श्चाथ् स॒प्त वै वै स॒प्त प॒श्चात् प॒श्चाथ् स॒प्त वै । \newline
14. स॒प्त वै वै स॒प्त स॒प्त वै शी॑र्.ष॒ण्याः᳚ शीर्.ष॒ण्या॑ वै स॒प्त स॒प्त वै शी॑र्.ष॒ण्याः᳚ । \newline
15. वै शी॑र्.ष॒ण्याः᳚ शीर्.ष॒ण्या॑ वै वै शी॑र्.ष॒ण्याः᳚ प्रा॒णाः प्रा॒णाः शी॑र्.ष॒ण्या॑ वै वै शी॑र्.ष॒ण्याः᳚ प्रा॒णाः । \newline
16. शी॒र्॒.ष॒ण्याः᳚ प्रा॒णाः प्रा॒णाः शी॑र्.ष॒ण्याः᳚ शीर्.ष॒ण्याः᳚ प्रा॒णा द्वौ द्वौ प्रा॒णाः शी॑र्.ष॒ण्याः᳚ शीर्.ष॒ण्याः᳚ प्रा॒णा द्वौ । \newline
17. प्रा॒णा द्वौ द्वौ प्रा॒णाः प्रा॒णा द्वा ववा᳚ञ्चा॒ ववा᳚ञ्चौ॒ द्वौ प्रा॒णाः प्रा॒णा द्वा ववा᳚ञ्चौ । \newline
18. प्रा॒णा इति॑ प्र - अ॒नाः । \newline
19. द्वा ववा᳚ञ्चा॒ ववा᳚ञ्चौ॒ द्वौ द्वा ववा᳚ञ्चौ प्रा॒णाना᳚म् प्रा॒णाना॒ मवा᳚ञ्चौ॒ द्वौ द्वा ववा᳚ञ्चौ प्रा॒णाना᳚म् । \newline
20. अवा᳚ञ्चौ प्रा॒णाना᳚म् प्रा॒णाना॒ मवा᳚ञ्चा॒ ववा᳚ञ्चौ प्रा॒णानाꣳ॑ सवीर्य॒त्वाय॑ सवीर्य॒त्वाय॑ प्रा॒णाना॒ मवा᳚ञ्चा॒ ववा᳚ञ्चौ प्रा॒णानाꣳ॑ सवीर्य॒त्वाय॑ । \newline
21. प्रा॒णानाꣳ॑ सवीर्य॒त्वाय॑ सवीर्य॒त्वाय॑ प्रा॒णाना᳚म् प्रा॒णानाꣳ॑ सवीर्य॒त्वाय॑ मू॒र्द्धा मू॒र्द्धा स॑वीर्य॒त्वाय॑ प्रा॒णाना᳚म् प्रा॒णानाꣳ॑ सवीर्य॒त्वाय॑ मू॒र्द्धा । \newline
22. प्रा॒णाना॒मिति॑ प्र - अ॒नाना᳚म् । \newline
23. स॒वी॒र्य॒त्वाय॑ मू॒र्द्धा मू॒र्द्धा स॑वीर्य॒त्वाय॑ सवीर्य॒त्वाय॑ मू॒र्द्धा ऽस्य॑सि मू॒र्द्धा स॑वीर्य॒त्वाय॑ सवीर्य॒त्वाय॑ मू॒र्द्धा ऽसि॑ । \newline
24. स॒वी॒र्य॒त्वायेति॑ सवीर्य - त्वाय॑ । \newline
25. मू॒र्द्धा ऽस्य॑सि मू॒र्द्धा मू॒र्द्धा ऽसि॒ राड् राड॑सि मू॒र्द्धा मू॒र्द्धा ऽसि॒ राट् । \newline
26. अ॒सि॒ राड् राड॑ स्यसि॒ राडि तीति॒ राड॑ स्यसि॒ राडिति॑ । \newline
27. राडि तीति॒ राड् राडिति॑ पु॒रस्ता᳚त् पु॒रस्ता॒ दिति॒ राड् राडिति॑ पु॒रस्ता᳚त् । \newline
28. इति॑ पु॒रस्ता᳚त् पु॒रस्ता॒ दितीति॑ पु॒रस्ता॒ दुपोप॑ पु॒रस्ता॒ दितीति॑ पु॒रस्ता॒ दुप॑ । \newline
29. पु॒रस्ता॒ दुपोप॑ पु॒रस्ता᳚त् पु॒रस्ता॒ दुप॑ दधाति दधा॒ त्युप॑ पु॒रस्ता᳚त् पु॒रस्ता॒ दुप॑ दधाति । \newline
30. उप॑ दधाति दधा॒ त्युपोप॑ दधाति॒ यन्त्री॒ यन्त्री॑ दधा॒ त्युपोप॑ दधाति॒ यन्त्री᳚ । \newline
31. द॒धा॒ति॒ यन्त्री॒ यन्त्री॑ दधाति दधाति॒ यन्त्री॒ राड् राड् यन्त्री॑ दधाति दधाति॒ यन्त्री॒ राट् । \newline
32. यन्त्री॒ राड् राड् यन्त्री॒ यन्त्री॒ राडि तीति॒ राड् यन्त्री॒ यन्त्री॒ राडिति॑ । \newline
33. राडितीति॒ राड् राडिति॑ प॒श्चात् प॒श्चा दिति॒ राड् राडिति॑ प॒श्चात् । \newline
34. इति॑ प॒श्चात् प॒श्चा दितीति॑ प॒श्चात् प्रा॒णान् प्रा॒णान् प॒श्चा दितीति॑ प॒श्चात् प्रा॒णान् । \newline
35. प॒श्चात् प्रा॒णान् प्रा॒णान् प॒श्चात् प॒श्चात् प्रा॒णा ने॒वैव प्रा॒णान् प॒श्चात् प॒श्चात् प्रा॒णा ने॒व । \newline
36. प्रा॒णा ने॒वैव प्रा॒णान् प्रा॒णा ने॒वास्मा॑ अस्मा ए॒व प्रा॒णान् प्रा॒णा ने॒वास्मै᳚ । \newline
37. प्रा॒णानिति॑ प्र - अ॒नान् । \newline
38. ए॒वास्मा॑ अस्मा ए॒वै वास्मै॑ स॒मीचः॑ स॒मीचो᳚ ऽस्मा ए॒वै वास्मै॑ स॒मीचः॑ । \newline
39. अ॒स्मै॒ स॒मीचः॑ स॒मीचो᳚ ऽस्मा अस्मै स॒मीचो॑ दधाति दधाति स॒मीचो᳚ ऽस्मा अस्मै स॒मीचो॑ दधाति । \newline
40. स॒मीचो॑ दधाति दधाति स॒मीचः॑ स॒मीचो॑ दधाति । \newline
41. द॒धा॒तीति॑ दधाति । \newline
\pagebreak
\markright{ TS 5.3.3.1  \hfill https://www.vedavms.in \hfill}

\section{ TS 5.3.3.1 }

\textbf{TS 5.3.3.1 } \newline
\textbf{Samhita Paata} \newline

दे॒वा वै यद्-य॒ज्ञे ऽकु॑र्वत॒ तदसु॑रा अकुर्वत॒ ते दे॒वा ए॒ता अ॑क्ष्णयास्तो॒मीया॑ अपश्य॒न् ता अ॒न्यथा॒ ऽनूच्या॒-न्यथोपा॑दधत॒ तदसु॑रा॒ नान्ववा॑य॒न् ततो॑ दे॒वा अभ॑व॒न् पराऽसु॑रा॒ यद॑क्ष्णयास्तो॒मीया॑ अ॒न्यथा॒ ऽनूच्या॒न्यथो॑प॒ दधा॑ति॒ भ्रातृ॑व्याभिभूत्यै॒ भव॑त्या॒त्मना॒ परा᳚ऽस्य॒ भ्रातृ॑व्यो भवत्या॒-शुस्त्रि॒वृदिति॑ पु॒रस्ता॒दुप॑ दधाति यज्ञ्मु॒खं ॅवै त्रि॒वृ - [  ] \newline

\textbf{Pada Paata} \newline

दे॒वाः । वै । यत् । य॒ज्ञे । अकु॑र्वत । तत् । असु॑राः । अ॒कु॒र्व॒त॒ । ते । दे॒वाः । ए॒ताः । अ॒क्ष्ण॒या॒स्तो॒मीया॒ इत्य॑क्ष्णया-स्तो॒मीयाः᳚ । अ॒प॒श्य॒न्न् । ताः । अ॒न्यथा᳚ । अ॒नूच्येत्य॑नु - उच्य॑ । अ॒न्यथा᳚ । उपेति॑ । अ॒द॒ध॒त॒ । तत् । असु॑राः । न । अ॒न्ववा॑य॒न्नित्य॑नु - अवा॑यन्न् । ततः॑ । दे॒वाः । अभ॑वन्न् । परेति॑ । असु॑राः । यत् । अ॒क्ष्ण॒या॒स्तो॒मीया॒ इत्य॑क्ष्णया - स्तो॒मीयाः᳚ । अ॒न्यथा᳚ । अ॒नूच्येत्य॑नु - उच्य॑ । अ॒न्यथा᳚ । उ॒प॒दधा॒तीत्यु॑प-दधा॑ति । भ्रातृ॑व्याभिभूत्या॒ इति॒ भ्रातृ॑व्य - अ॒भि॒भू॒त्यै॒ । भव॑ति । आ॒त्मना᳚ । परेति॑ । अ॒स्य॒ । भ्रातृ॑व्यः । भ॒व॒ति॒ । आ॒शुः । त्रि॒वृदिति॑ त्रि - वृत् । इति॑ । पु॒रस्ता᳚त् । उपेति॑ । द॒धा॒ति॒ । य॒ज्ञ्॒मु॒खमिति॑ यज्ञ् - मु॒खम् । वै । त्रि॒वृदिति॑ त्रि - वृत् ।  \newline


\textbf{Krama Paata} \newline

दे॒वा वै । वै यत् । यद् य॒ज्ञे । य॒ज्ञेऽकु॑र्वत । अकु॑र्वत॒ तत् । तदसु॑राः । असु॑रा अकुर्वत । अ॒कु॒र्व॒त॒ ते । ते दे॒वाः । दे॒वा ए॒ताः । ए॒ता अ॑क्ष्णयास्तो॒मीयाः᳚ । अ॒क्ष्ण॒या॒स्तो॒मीया॑ अपश्यन्न् । अ॒क्ष्ण॒या॒स्तो॒मीया॒ इत्य॑क्ष्णया - स्तो॒मीयाः᳚ । अ॒प॒श्य॒न् ताः । ता अ॒न्यथा᳚ । अ॒न्यथा॒ऽनूच्य॑ । अ॒नूच्या॒न्यथा᳚ । अ॒नूच्येत्य॑नु - उच्य॑ । अ॒न्यथोप॑ । उपा॑दधत । अ॒द॒ध॒त॒ तत् । तदसु॑राः । असु॑रा॒ न । नान्ववा॑यन्न् । अ॒न्ववा॑य॒न् ततः॑ । अ॒न्ववा॑य॒न्नित्य॑नु - अवा॑यन्न् । ततो॑ दे॒वाः । दे॒वा अभ॑वन्न् । अभ॑व॒न् परा᳚ । पराऽसु॑राः । असु॑रा॒ यत् । यद॑क्ष्णयास्तो॒मीयाः᳚ । अ॒क्ष्ण॒या॒स्तो॒मीया॑ अ॒न्यथा᳚ । अ॒क्ष्ण॒या॒स्तो॒मीया॒ इत्य॑क्ष्णया - स्तो॒मीयाः᳚ । अ॒न्यथा॒ऽनूच्य॑ । अ॒नूच्या॒न्यथा᳚ । अ॒नूचेत्य॑नु - उच्य॑ । अ॒न्यथो॑प॒दधा॑ति । उ॒प॒दधा॑ति॒ भ्रातृ॑व्याभिभूत्यै । उ॒प॒दधा॒तीत्यु॑प - दधा॑ति । भ्रातृ॑व्याभिभूत्यै॒ भव॑ति । भ्रातृ॑व्याभिभूत्या॒ इति॒ भ्रातृ॑व्य - अ॒भि॒भू॒त्यै॒ । भव॑त्या॒त्मना᳚ । आ॒त्मना॒ परा᳚ । परा᳚ऽस्य । अ॒स्य॒ भ्रातृ॑व्यः । भ्रातृ॑व्यो भवति । भ॒व॒त्या॒शुः । आ॒शुस्त्रि॒वृत् । त्रि॒वृदिति॑ । त्रि॒वृदिति॑ त्रि - वृत् । इति॑ पु॒रस्ता᳚त् । पु॒रस्ता॒दुप॑ । उप॑ दधाति । द॒धा॒ति॒ य॒ज्ञ्॒मु॒खम् । य॒ज्ञ्॒मु॒खम् ॅवै । य॒ज्ञ्॒मु॒खमिति॑ यज्ञ् - मु॒खम् । वै त्रि॒वृत् । त्रि॒वृद् य॑ज्ञ्मु॒खम् । त्रि॒वृदिति॑ त्रि - वृत् \newline

\textbf{Jatai Paata} \newline

1. दे॒वा वै वै दे॒वा दे॒वा वै । \newline
2. वै यद् यद् वै वै यत् । \newline
3. यद् य॒ज्ञे य॒ज्ञे यद् यद् य॒ज्ञे । \newline
4. य॒ज्ञे ऽकु॑र्व॒ता कु॑र्वत य॒ज्ञे य॒ज्ञे ऽकु॑र्वत । \newline
5. अकु॑र्वत॒ तत् तदकु॑र्व॒ता कु॑र्वत॒ तत् । \newline
6. तदसु॑रा॒ असु॑रा॒ स्तत् तदसु॑राः । \newline
7. असु॑रा अकुर्वता कुर्व॒ता सु॑रा॒ असु॑रा अकुर्वत । \newline
8. अ॒कु॒र्व॒त॒ ते ते॑ ऽकुर्वता कुर्वत॒ ते । \newline
9. ते दे॒वा दे॒वा स्ते ते दे॒वाः । \newline
10. दे॒वा ए॒ता ए॒ता दे॒वा दे॒वा ए॒ताः । \newline
11. ए॒ता अ॑क्ष्णयास्तो॒मीया॑ अक्ष्णयास्तो॒मीया॑ ए॒ता ए॒ता अ॑क्ष्णयास्तो॒मीयाः᳚ । \newline
12. अ॒क्ष्ण॒या॒स्तो॒मीया॑ अपश्यन् नपश्यन् नक्ष्णयास्तो॒मीया॑ अक्ष्णयास्तो॒मीया॑ अपश्यन्न् । \newline
13. अ॒क्ष्ण॒या॒स्तो॒मीया॒ इत्य॑क्ष्णया - स्तो॒मीयाः᳚ । \newline
14. अ॒प॒श्य॒न् ता स्ता अ॑पश्यन् नपश्य॒न् ताः । \newline
15. ता अ॒न्यथा॒ ऽन्यथा॒ ता स्ता अ॒न्यथा᳚ । \newline
16. अ॒न्यथा॒ ऽनूच्या॒ नूच्या॒ न्यथा॒ ऽन्यथा॒ ऽनूच्य॑ । \newline
17. अ॒नूच्या॒ न्यथा॒ ऽन्यथा॒ ऽनूच्या॒ नूच्या॒ न्यथा᳚ । \newline
18. अ॒नूच्येत्य॑नु - उच्य॑ । \newline
19. अ॒न्यथोपोपा॒ न्यथा॒ ऽन्यथोप॑ । \newline
20. उपा॑ दधता दध॒तोपोपा॑ दधत । \newline
21. अ॒द॒ध॒त॒ तत् तद॑दधता दधत॒ तत् । \newline
22. तदसु॑रा॒ असु॑रा॒ स्तत् तदसु॑राः । \newline
23. असु॑रा॒ न नासु॑रा॒ असु॑रा॒ न । \newline
24. नान्ववा॑यन् न॒न्ववा॑य॒न् न नान्ववा॑यन्न् । \newline
25. अ॒न्ववा॑य॒न् तत॒ स्ततो॒ ऽन्ववा॑यन् न॒न्ववा॑य॒न् ततः॑ । \newline
26. अ॒न्ववा॑य॒न्नित्य॑नु - अवा॑यन्न् । \newline
27. ततो॑ दे॒वा दे॒वा स्तत॒ स्ततो॑ दे॒वाः । \newline
28. दे॒वा अभ॑व॒न् नभ॑वन् दे॒वा दे॒वा अभ॑वन्न् । \newline
29. अभ॑व॒न् परा॒ परा ऽभ॑व॒न् नभ॑व॒न् परा᳚ । \newline
30. परा ऽसु॑रा॒ असु॑राः॒ परा॒ परा ऽसु॑राः । \newline
31. असु॑रा॒ यद् यदसु॑रा॒ असु॑रा॒ यत् । \newline
32. यद॑क्ष्णयास्तो॒मीया॑ अक्ष्णयास्तो॒मीया॒ यद् यद॑क्ष्णयास्तो॒मीयाः᳚ । \newline
33. अ॒क्ष्ण॒या॒स्तो॒मीया॑ अ॒न्यथा॒ ऽन्यथा᳚ ऽक्ष्णयास्तो॒मीया॑ अक्ष्णयास्तो॒मीया॑ अ॒न्यथा᳚ । \newline
34. अ॒क्ष्ण॒या॒स्तो॒मीया॒ इत्य॑क्ष्णया - स्तो॒मीयाः᳚ । \newline
35. अ॒न्यथा॒ ऽनूच्या॒ नूच्या॒ न्यथा॒ ऽन्यथा॒ ऽनूच्य॑ । \newline
36. अ॒नूच्या॒ न्यथा॒ ऽन्यथा॒ ऽनूच्या॒ नूच्या॒ न्यथा᳚ । \newline
37. अ॒नूच्येत्य॑नु - उच्य॑ । \newline
38. अ॒न्यथो॑ प॒दधा᳚ त्युप॒दधा᳚ त्य॒न्यथा॒ ऽन्यथो॑ प॒दधा॑ति । \newline
39. उ॒प॒दधा॑ति॒ भ्रातृ॑व्याभिभूत्यै॒ भ्रातृ॑व्याभिभूत्या उप॒दधा᳚ त्युप॒दधा॑ति॒ भ्रातृ॑व्याभिभूत्यै । \newline
40. उ॒प॒दधा॒तीत्यु॑प - दधा॑ति । \newline
41. भ्रातृ॑व्याभिभूत्यै॒ भव॑ति॒ भव॑ति॒ भ्रातृ॑व्याभिभूत्यै॒ भ्रातृ॑व्याभिभूत्यै॒ भव॑ति । \newline
42. भ्रातृ॑व्याभिभूत्या॒ इति॒ भ्रातृ॑व्य - अ॒भि॒भू॒त्यै॒ । \newline
43. भव॑ त्या॒त्मना॒ ऽऽत्मना॒ भव॑ति॒ भव॑ त्या॒त्मना᳚ । \newline
44. आ॒त्मना॒ परा॒ परा॒ ऽऽत्मना॒ ऽऽत्मना॒ परा᳚ । \newline
45. परा᳚ ऽस्यास्य॒ परा॒ परा᳚ ऽस्य । \newline
46. अ॒स्य॒ भ्रातृ॑व्यो॒ भ्रातृ॑व्यो ऽस्यास्य॒ भ्रातृ॑व्यः । \newline
47. भ्रातृ॑व्यो भवति भवति॒ भ्रातृ॑व्यो॒ भ्रातृ॑व्यो भवति । \newline
48. भ॒व॒ त्या॒शु रा॒शुर् भ॑वति भव त्या॒शुः । \newline
49. आ॒शु स्त्रि॒वृत् त्रि॒वृ दा॒शु रा॒शु स्त्रि॒वृत् । \newline
50. त्रि॒वृ दितीति॑ त्रि॒वृत् त्रि॒वृ दिति॑ । \newline
51. त्रि॒वृदिति॑ त्रि - वृत् । \newline
52. इति॑ पु॒रस्ता᳚त् पु॒रस्ता॒ दितीति॑ पु॒रस्ता᳚त् । \newline
53. पु॒रस्ता॒ दुपोप॑ पु॒रस्ता᳚त् पु॒रस्ता॒ दुप॑ । \newline
54. उप॑ दधाति दधा॒ त्युपोप॑ दधाति । \newline
55. द॒धा॒ति॒ य॒ज्ञ्॒मु॒खं ॅय॑ज्ञ्मु॒खम् द॑धाति दधाति यज्ञ्मु॒खम् । \newline
56. य॒ज्ञ्॒मु॒खं ॅवै वै य॑ज्ञ्मु॒खं ॅय॑ज्ञ्मु॒खं ॅवै । \newline
57. य॒ज्ञ्॒मु॒खमिति॑ यज्ञ् - मु॒खम् । \newline
58. वै त्रि॒वृत् त्रि॒वृद् वै वै त्रि॒वृत् । \newline
59. त्रि॒वृद् य॑ज्ञ्मु॒खं ॅय॑ज्ञ्मु॒खम् त्रि॒वृत् त्रि॒वृद् य॑ज्ञ्मु॒खम् । \newline
60. त्रि॒वृदिति॑ त्रि - वृत् । \newline

\textbf{Ghana Paata } \newline

1. दे॒वा वै वै दे॒वा दे॒वा वै यद् यद् वै दे॒वा दे॒वा वै यत् । \newline
2. वै यद् यद् वै वै यद् य॒ज्ञे य॒ज्ञे यद् वै वै यद् य॒ज्ञे । \newline
3. यद् य॒ज्ञे य॒ज्ञे यद् यद् य॒ज्ञे ऽकु॑र्व॒ता कु॑र्वत य॒ज्ञे यद् यद् य॒ज्ञे ऽकु॑र्वत । \newline
4. य॒ज्ञे ऽकु॑र्व॒ता कु॑र्वत य॒ज्ञे य॒ज्ञे ऽकु॑र्वत॒ तत् तदकु॑र्वत य॒ज्ञे य॒ज्ञे ऽकु॑र्वत॒ तत् । \newline
5. अकु॑र्वत॒ तत् तदकु॑र्व॒ता कु॑र्वत॒ तदसु॑रा॒ असु॑रा॒ स्तदकु॑र्व॒ता कु॑र्वत॒ तदसु॑राः । \newline
6. तदसु॑रा॒ असु॑रा॒ स्तत् तदसु॑रा अकुर्वता कुर्व॒ता सु॑रा॒ स्तत् तदसु॑रा अकुर्वत । \newline
7. असु॑रा अकुर्वता कुर्व॒ता सु॑रा॒ असु॑रा अकुर्वत॒ ते ते॑ ऽकुर्व॒ता सु॑रा॒ असु॑रा अकुर्वत॒ ते । \newline
8. अ॒कु॒र्व॒त॒ ते ते॑ ऽकुर्वता कुर्वत॒ ते दे॒वा दे॒वा स्ते॑ ऽकुर्वता कुर्वत॒ ते दे॒वाः । \newline
9. ते दे॒वा दे॒वा स्ते ते दे॒वा ए॒ता ए॒ता दे॒वा स्ते ते दे॒वा ए॒ताः । \newline
10. दे॒वा ए॒ता ए॒ता दे॒वा दे॒वा ए॒ता अ॑क्ष्णयास्तो॒मीया॑ अक्ष्णयास्तो॒मीया॑ ए॒ता दे॒वा दे॒वा ए॒ता अ॑क्ष्णयास्तो॒मीयाः᳚ । \newline
11. ए॒ता अ॑क्ष्णयास्तो॒मीया॑ अक्ष्णयास्तो॒मीया॑ ए॒ता ए॒ता अ॑क्ष्णयास्तो॒मीया॑ अपश्यन् नपश्यन् नक्ष्णयास्तो॒मीया॑ ए॒ता ए॒ता अ॑क्ष्णयास्तो॒मीया॑ अपश्यन्न् । \newline
12. अ॒क्ष्ण॒या॒स्तो॒मीया॑ अपश्यन् नपश्यन् नक्ष्णयास्तो॒मीया॑ अक्ष्णयास्तो॒मीया॑ अपश्य॒न् ता स्ता अ॑पश्यन् नक्ष्णयास्तो॒मीया॑ अक्ष्णयास्तो॒मीया॑ अपश्य॒न् ताः । \newline
13. अ॒क्ष्ण॒या॒स्तो॒मीया॒ इत्य॑क्ष्णया - स्तो॒मीयाः᳚ । \newline
14. अ॒प॒श्य॒न् ता स्ता अ॑पश्यन् नपश्य॒न् ता अ॒न्यथा॒ ऽन्यथा॒ ता अ॑पश्यन् नपश्य॒न् ता अ॒न्यथा᳚ । \newline
15. ता अ॒न्यथा॒ ऽन्यथा॒ ता स्ता अ॒न्यथा॒ ऽनूच्या॒ नूच्या॒ न्यथा॒ तास्ता अ॒न्यथा॒ ऽनूच्य॑ । \newline
16. अ॒न्यथा॒ ऽनूच्या॒ नूच्या॒न्यथा॒ ऽन्यथा॒ ऽनूच्या॒ न्यथा॒ ऽन्यथा॒ ऽनूच्या॒न्यथा॒ ऽन्यथा॒ ऽनूच्या॒ न्यथा᳚ । \newline
17. अ॒नूच्या॒ न्यथा॒ ऽन्यथा॒ ऽनूच्या॒ नूच्या॒न्य थोपोपा॒ न्यथा॒ ऽनूच्या॒ नूच्या॒ न्यथोप॑ । \newline
18. अ॒नूच्येत्य॑नु - उच्य॑ । \newline
19. अ॒न्यथोपोपा॒ न्यथा॒ ऽन्यथोपा॑ दधता दध॒तोपा॒ न्यथा॒ ऽन्यथोपा॑ दधत । \newline
20. उपा॑ दधता दध॒तोपोपा॑ दधत॒ तत् तद॑दध॒ तोपोपा॑ दधत॒ तत् । \newline
21. अ॒द॒ध॒त॒ तत् तद॑दधता दधत॒ तदसु॑रा॒ असु॑रा॒ स्तद॑ दधता दधत॒ तदसु॑राः । \newline
22. तदसु॑रा॒ असु॑रा॒ स्तत् तदसु॑रा॒ न नासु॑रा॒ स्तत् तदसु॑रा॒ न । \newline
23. असु॑रा॒ न नासु॑रा॒ असु॑रा॒ नान्ववा॑यन् न॒न्ववा॑य॒न् नासु॑रा॒ असु॑रा॒ नान्ववा॑यन्न् । \newline
24. नान्ववा॑यन् न॒न्ववा॑य॒न् न नान्ववा॑य॒न् तत॒ स्ततो॒ ऽन्ववा॑य॒न् न नान्ववा॑य॒न् ततः॑ । \newline
25. अ॒न्ववा॑य॒न् तत॒ स्ततो॒ ऽन्ववा॑यन् न॒न्ववा॑य॒न् ततो॑ दे॒वा दे॒वा स्ततो॒ ऽन्ववा॑यन् न॒न्ववा॑य॒न् ततो॑ दे॒वाः । \newline
26. अ॒न्ववा॑य॒न्नित्य॑नु - अवा॑यन्न् । \newline
27. ततो॑ दे॒वा दे॒वा स्तत॒ स्ततो॑ दे॒वा अभ॑व॒न् नभ॑वन् दे॒वा स्तत॒ स्ततो॑ दे॒वा अभ॑वन्न् । \newline
28. दे॒वा अभ॑व॒न् नभ॑वन् दे॒वा दे॒वा अभ॑व॒न् परा॒ परा ऽभ॑वन् दे॒वा दे॒वा अभ॑व॒न् परा᳚ । \newline
29. अभ॑व॒न् परा॒ परा ऽभ॑व॒न् नभ॑व॒न् परा ऽसु॑रा॒ असु॑राः॒ परा ऽभ॑व॒न् नभ॑व॒न् परा ऽसु॑राः । \newline
30. परा ऽसु॑रा॒ असु॑राः॒ परा॒ परा ऽसु॑रा॒ यद् यदसु॑राः॒ परा॒ परा ऽसु॑रा॒ यत् । \newline
31. असु॑रा॒ यद् यदसु॑रा॒ असु॑रा॒ यद॑क्ष्णयास्तो॒मीया॑ अक्ष्णयास्तो॒मीया॒ यदसु॑रा॒ असु॑रा॒ यद॑क्ष्णयास्तो॒मीयाः᳚ । \newline
32. यद॑क्ष्णयास्तो॒मीया॑ अक्ष्णयास्तो॒मीया॒ यद् यद॑क्ष्णयास्तो॒मीया॑ अ॒न्यथा॒ ऽन्यथा᳚ ऽक्ष्णयास्तो॒मीया॒ यद् यद॑क्ष्णयास्तो॒मीया॑ अ॒न्यथा᳚ । \newline
33. अ॒क्ष्ण॒या॒स्तो॒मीया॑ अ॒न्यथा॒ ऽन्यथा᳚ ऽक्ष्णयास्तो॒मीया॑ अक्ष्णयास्तो॒मीया॑ अ॒न्यथा॒ ऽनूच्या॒ नूच्या॒ न्यथा᳚ ऽक्ष्णयास्तो॒मीया॑ अक्ष्णयास्तो॒मीया॑ अ॒न्यथा॒ ऽनूच्य॑ । \newline
34. अ॒क्ष्ण॒या॒स्तो॒मीया॒ इत्य॑क्ष्णया - स्तो॒मीयाः᳚ । \newline
35. अ॒न्यथा॒ ऽनूच्या॒ नूच्या॒ न्यथा॒ ऽन्यथा॒ ऽनूच्या॒ न्यथा॒ ऽन्यथा॒ ऽनूच्या॒ न्यथा॒ ऽन्यथा॒ ऽनूच्या॒ न्यथा᳚ । \newline
36. अ॒नूच्या॒ न्यथा॒ ऽन्यथा॒ ऽनूच्या॒ नूच्या॒ न्यथो॑ प॒दधा᳚ त्युप॒दधा᳚ त्य॒न्यथा॒ ऽनूच्या॒ नूच्या॒न्य थो॑प॒दधा॑ति । \newline
37. अ॒नूच्येत्य॑नु - उच्य॑ । \newline
38. अ॒न्यथो॑ प॒दधा᳚ त्युप॒दधा᳚ त्य॒न्यथा॒ ऽन्यथो॑ प॒दधा॑ति॒ भ्रातृ॑व्याभिभूत्यै॒ भ्रातृ॑व्याभिभूत्या उप॒दधा᳚ त्य॒न्यथा॒ ऽन्यथो॑ प॒दधा॑ति॒ भ्रातृ॑व्याभिभूत्यै । \newline
39. उ॒प॒दधा॑ति॒ भ्रातृ॑व्याभिभूत्यै॒ भ्रातृ॑व्याभिभूत्या उप॒दधा᳚ त्युप॒दधा॑ति॒ भ्रातृ॑व्याभिभूत्यै॒ भव॑ति॒ भव॑ति॒ भ्रातृ॑व्याभिभूत्या उप॒दधा᳚ त्युप॒दधा॑ति॒ भ्रातृ॑व्याभिभूत्यै॒ भव॑ति । \newline
40. उ॒प॒दधा॒तीत्यु॑प - दधा॑ति । \newline
41. भ्रातृ॑व्याभिभूत्यै॒ भव॑ति॒ भव॑ति॒ भ्रातृ॑व्याभिभूत्यै॒ भ्रातृ॑व्याभिभूत्यै॒ भव॑त्या॒त्मना॒ ऽऽत्मना॒ भव॑ति॒ भ्रातृ॑व्याभिभूत्यै॒ भ्रातृ॑व्याभिभूत्यै॒ भव॑ त्या॒त्मना᳚ । \newline
42. भ्रातृ॑व्याभिभूत्या॒ इति॒ भ्रातृ॑व्य - अ॒भि॒भू॒त्यै॒ । \newline
43. भव॑त्या॒त्मना॒ ऽऽत्मना॒ भव॑ति॒ भव॑त्या॒त्मना॒ परा॒ परा॒ ऽऽत्मना॒ भव॑ति॒ भव॑त्या॒त्मना॒ परा᳚ । \newline
44. आ॒त्मना॒ परा॒ परा॒ ऽऽत्मना॒ ऽऽत्मना॒ परा᳚ ऽस्यास्य॒ परा॒ ऽऽत्मना॒ ऽऽत्मना॒ परा᳚ ऽस्य । \newline
45. परा᳚ ऽस्यास्य॒ परा॒ परा᳚ ऽस्य॒ भ्रातृ॑व्यो॒ भ्रातृ॑व्यो ऽस्य॒ परा॒ परा᳚ ऽस्य॒ भ्रातृ॑व्यः । \newline
46. अ॒स्य॒ भ्रातृ॑व्यो॒ भ्रातृ॑व्यो ऽस्यास्य॒ भ्रातृ॑व्यो भवति भवति॒ भ्रातृ॑व्यो ऽस्यास्य॒ भ्रातृ॑व्यो भवति । \newline
47. भ्रातृ॑व्यो भवति भवति॒ भ्रातृ॑व्यो॒ भ्रातृ॑व्यो भव त्या॒शु रा॒शुर् भ॑वति॒ भ्रातृ॑व्यो॒ भ्रातृ॑व्यो भवत्या॒शुः । \newline
48. भ॒व॒त्या॒ शुरा॒शुर् भ॑वति भवत्या॒शु स्त्रि॒वृत् त्रि॒वृ दा॒शुर् भ॑वति भव त्या॒शु स्त्रि॒वृत् । \newline
49. आ॒शु स्त्रि॒वृत् त्रि॒वृ दा॒शु रा॒शु स्त्रि॒वृ दितीति॑ त्रि॒वृ दा॒शु रा॒शु स्त्रि॒वृ दिति॑ । \newline
50. त्रि॒वृ दितीति॑ त्रि॒वृत् त्रि॒वृदिति॑ पु॒रस्ता᳚त् पु॒रस्ता॒ दिति॑ त्रि॒वृत् त्रि॒वृदिति॑ पु॒रस्ता᳚त् । \newline
51. त्रि॒वृदिति॑ त्रि - वृत् । \newline
52. इति॑ पु॒रस्ता᳚त् पु॒रस्ता॒ दितीति॑ पु॒रस्ता॒ दुपोप॑ पु॒रस्ता॒ दितीति॑ पु॒रस्ता॒ दुप॑ । \newline
53. पु॒रस्ता॒ दुपोप॑ पु॒रस्ता᳚त् पु॒रस्ता॒ दुप॑ दधाति दधा॒ त्युप॑ पु॒रस्ता᳚त् पु॒रस्ता॒ दुप॑ दधाति । \newline
54. उप॑ दधाति दधा॒ त्युपोप॑ दधाति यज्ञ्मु॒खं ॅय॑ज्ञ्मु॒खम् द॑धा॒ त्युपोप॑ दधाति यज्ञ्मु॒खम् । \newline
55. द॒धा॒ति॒ य॒ज्ञ्॒मु॒खं ॅय॑ज्ञ्मु॒खम् द॑धाति दधाति यज्ञ्मु॒खं ॅवै वै य॑ज्ञ्मु॒खम् द॑धाति दधाति यज्ञ्मु॒खं ॅवै । \newline
56. य॒ज्ञ्॒मु॒खं ॅवै वै य॑ज्ञ्मु॒खं ॅय॑ज्ञ्मु॒खं ॅवै त्रि॒वृत् त्रि॒वृद् वै य॑ज्ञ्मु॒खं ॅय॑ज्ञ्मु॒खं ॅवै त्रि॒वृत् । \newline
57. य॒ज्ञ्॒मु॒खमिति॑ यज्ञ् - मु॒खम् । \newline
58. वै त्रि॒वृत् त्रि॒वृद् वै वै त्रि॒वृद् य॑ज्ञ्मु॒खं ॅय॑ज्ञ्मु॒खम् त्रि॒वृद् वै वै त्रि॒वृद् य॑ज्ञ्मु॒खम् । \newline
59. त्रि॒वृद् य॑ज्ञ्मु॒खं ॅय॑ज्ञ्मु॒खम् त्रि॒वृत् त्रि॒वृद् य॑ज्ञ्मु॒ख मे॒वैव य॑ज्ञ्मु॒खम् त्रि॒वृत् त्रि॒वृद् य॑ज्ञ्मु॒ख मे॒व । \newline
60. त्रि॒वृदिति॑ त्रि - वृत् । \newline
\pagebreak
\markright{ TS 5.3.3.2  \hfill https://www.vedavms.in \hfill}

\section{ TS 5.3.3.2 }

\textbf{TS 5.3.3.2 } \newline
\textbf{Samhita Paata} \newline

-द्य॑ज्ञ्मु॒खमे॒व पु॒रस्ता॒द्वि या॑तयति॒ व्यो॑म सप्तद॒श इति॑ दक्षिण॒तो ऽन्नं॒ ॅवै व्यो॑माऽन्नꣳ॑ सप्तद॒शोऽन्न॑मे॒व द॑क्षिण॒तो ध॑त्ते॒ तस्मा॒द्-दक्षि॑णे॒नान्न॑मद्यते ध॒रुण॑ एकविꣳ॒॒श इति॑ प॒श्चात् प्र॑ति॒ष्ठा वा ए॑कविꣳ॒॒शः प्रति॑ष्ठित्यै भा॒न्तः प॑ञ्चद॒श इत्यु॑त्तर॒त ओजो॒ वै भा॒न्त ओजः॑ पञ्चद॒श ओज॑ ए॒वोत्त॑र॒तो ध॑त्ते॒ तस्मा॑दुत्तरतो ऽभिप्रया॒यी ज॑यति॒ प्रतू᳚र्तिरष्टाद॒श इति॑ पु॒रस्ता॒- [  ] \newline

\textbf{Pada Paata} \newline

य॒ज्ञ्॒मु॒खमिति॑ यज्ञ् - मु॒खम् । ए॒व । पु॒रस्ता᳚त् । वीति॑ । या॒त॒य॒ति॒ । व्यो॑मेति॒ वि - ओ॒म॒ । स॒प्त॒द॒श इति॑ सप्त - द॒शः । इति॑ । द॒क्षि॒ण॒तः । अन्न᳚म् । वै । व्यो॑मेति॒ वि - ओ॒म॒ । अन्न᳚म् । स॒प्त॒द॒श इति॑ सप्त - द॒शः । अन्न᳚म् । ए॒व । द॒क्षि॒ण॒तः । ध॒त्ते॒ । तस्मा᳚त् । दक्षि॑णेन । अन्न᳚म् । अ॒द्य॒ते॒ । ध॒रुणः॑ । ए॒क॒विꣳ॒॒श इत्ये॑क-विꣳ॒॒शः । इति॑ । प॒श्चात् । प्र॒ति॒ष्ठेति॑ प्रति - स्था । वै । ए॒क॒विꣳ॒॒श इत्ये॑क - विꣳ॒॒शः । प्रति॑ष्ठित्या॒ इति॒ प्रति॑ - स्थि॒त्यै॒ । भा॒न्तः । प॒ञ्च॒द॒श इति॑ पञ्च - द॒शः । इति॑ । उ॒त्त॒र॒त इत्यु॑त् - त॒र॒तः । ओजः॑ । वै । भा॒न्तः । ओजः॑ । प॒ञ्च॒द॒श इति॑ पञ्च - द॒शः । ओजः॑ । ए॒व । उ॒त्त॒र॒त इत्यु॑त् - त॒र॒तः । ध॒त्ते॒ । तस्मा᳚त् । उ॒त्त॒र॒तो॒ऽभि॒प्र॒या॒यीत्यु॑त्तरतः - अ॒भि॒प्र॒या॒यी । ज॒य॒ति॒ । प्रतू᳚र्ति॒रिति॒ प्र - तू॒र्तिः॒ । अ॒ष्टा॒द॒श इत्य॑ष्टा - द॒शः । इति॑ । पु॒रस्ता᳚त् ।  \newline


\textbf{Krama Paata} \newline

य॒ज्ञ्॒मु॒खमे॒व । य॒ज्ञ्॒मु॒खमिति॑ यज्ञ् - मु॒खम् । ए॒व पु॒रस्ता᳚त् । पु॒रस्ता॒द् वि । वि या॑तयति । या॒त॒य॒ति॒ व्यो॑म । व्यो॑म सप्तद॒शः । व्यो॑मेति॒ वि - ओ॒म॒ । स॒प्त॒द॒श इति॑ । स॒प्त॒द॒श इति॑ सप्त - द॒शः । इति॑ दक्षिण॒तः । द॒क्षि॒ण॒तोऽन्न᳚म् । अन्न॒म् ॅवै । वै व्यो॑म । व्यो॑मान्न᳚म् । व्यो॑मेति॒ वि - ओ॒म॒ । अन्नꣳ॑ सप्तद॒शः । स॒प्त॒द॒शोऽन्न᳚म् । स॒प्त॒द॒श इति॑ सप्त - द॒शः । अन्न॑मे॒व । ए॒व द॑क्षिण॒तः । द॒क्षि॒ण॒तो ध॑त्ते । ध॒त्ते॒ तस्मा᳚त् । तस्मा॒द् दक्षि॑णेन । दक्षि॑णे॒नान्न᳚म् । अन्न॑मद्यते । अ॒द्य॒ते॒ ध॒रुणः॑ । ध॒रुण॑ एकविꣳ॒॒शः । ए॒क॒विꣳ॒॒श इति॑ । ए॒क॒विꣳ॒॒श इत्ये॑क - विꣳ॒॒शः । इति॑ प॒श्चात् । प॒श्चात् प्र॑ति॒ष्ठा । प्र॒ति॒ष्ठा वै । प्र॒ति॒ष्ठेति॑ प्रति - स्था । वा ए॑कविꣳ॒॒शः । ए॒क॒विꣳ॒॒शः प्रति॑ष्ठित्यै । ए॒क॒विꣳ॒॒श इत्ये॑क - विꣳ॒॒शः । प्रति॑ष्ठित्यै भा॒न्तः । प्रति॑ष्ठित्या॒ इति॒ प्रति॑ - स्थि॒त्यै॒ । भा॒न्तः प॑ञ्चद॒शः । प॒ञ्च॒द॒श इति॑ । प॒ञ्च॒द॒श इति॑ पञ्च - द॒शः । इत्यु॑त्तर॒तः । उ॒त्त॒र॒त ओजः॑ । उ॒त्त॒र॒त इत्यु॑त् - त॒र॒तः । ओजो॒ वै । वै भा॒न्तः । भा॒न्त ओजः॑ । ओजः॑ पञ्चद॒शः । प॒ञ्च॒द॒श ओजः॑ । प॒ञ्च॒द॒श इति॑ पञ्च - द॒शः । ओज॑ ए॒व । ए॒वोत्त॑र॒तः । उ॒त्त॒र॒तो ध॑त्ते । उ॒त्त॒र॒त इत्यु॑त् - त॒र॒तः । ध॒त्ते॒ तस्मा᳚त् । तस्मा॑दुत्तरतोभिप्रया॒यी । उ॒त्त॒र॒तो॒भि॒प्र॒या॒यी ज॑यति । उ॒त्त॒र॒तो॒भि॒प्र॒या॒यीत्यु॑त्तरतः - अ॒भि॒प्र॒या॒यी । ज॒य॒ति॒ प्रतू᳚र्तिः । प्रतू᳚र्तिरष्टाद॒शः । प्रतू᳚र्ति॒रिति॒ प्र - तू॒र्तिः॒ । अ॒ष्टा॒द॒श इति॑ । अ॒ष्टा॒द॒श इत्य॑ष्टा - द॒शः । इति॑ पु॒रस्ता᳚त् । पु॒रस्ता॒दुप॑ \newline

\textbf{Jatai Paata} \newline

1. य॒ज्ञ्॒मु॒ख मे॒वैव य॑ज्ञ्मु॒खं ॅय॑ज्ञ्मु॒ख मे॒व । \newline
2. य॒ज्ञ्॒मु॒खमिति॑ यज्ञ् - मु॒खम् । \newline
3. ए॒व पु॒रस्ता᳚त् पु॒रस्ता॑ दे॒वैव पु॒रस्ता᳚त् । \newline
4. पु॒रस्ता॒द् वि वि पु॒रस्ता᳚त् पु॒रस्ता॒द् वि । \newline
5. वि या॑तयति यातयति॒ वि वि या॑तयति । \newline
6. या॒त॒य॒ति॒ व्यो॑म॒ व्यो॑म यातयति यातयति॒ व्यो॑म । \newline
7. व्यो॑म सप्तद॒शः स॑प्तद॒शो व्यो॑म॒ व्यो॑म सप्तद॒शः । \newline
8. व्यो॑मेति॒ वि - ओ॒म॒ । \newline
9. स॒प्त॒द॒श इतीति॑ सप्तद॒शः स॑प्तद॒श इति॑ । \newline
10. स॒प्त॒द॒श इति॑ सप्त - द॒शः । \newline
11. इति॑ दक्षिण॒तो द॑क्षिण॒त इतीति॑ दक्षिण॒तः । \newline
12. द॒क्षि॒ण॒तो ऽन्न॒ मन्न॑म् दक्षिण॒तो द॑क्षिण॒तो ऽन्न᳚म् । \newline
13. अन्नं॒ ॅवै वा अन्न॒ मन्नं॒ ॅवै । \newline
14. वै व्यो॑म॒ व्यो॑म॒ वै वै व्यो॑म । \newline
15. व्यो॑मान्न॒ मन्नं॒ ॅव्यो॑म॒ व्यो॑मान्न᳚म् । \newline
16. व्यो॑मेति॒ वि - ओ॒म॒ । \newline
17. अन्नꣳ॑ सप्तद॒शः स॑प्तद॒शो ऽन्न॒ मन्नꣳ॑ सप्तद॒शः । \newline
18. स॒प्त॒द॒शो ऽन्न॒ मन्नꣳ॑ सप्तद॒शः स॑प्तद॒शो ऽन्न᳚म् । \newline
19. स॒प्त॒द॒श इति॑ सप्त - द॒शः । \newline
20. अन्न॑ मे॒वैवान्न॒ मन्न॑ मे॒व । \newline
21. ए॒व द॑क्षिण॒तो द॑क्षिण॒त ए॒वैव द॑क्षिण॒तः । \newline
22. द॒क्षि॒ण॒तो ध॑त्ते धत्ते दक्षिण॒तो द॑क्षिण॒तो ध॑त्ते । \newline
23. ध॒त्ते॒ तस्मा॒त् तस्मा᳚द् धत्ते धत्ते॒ तस्मा᳚त् । \newline
24. तस्मा॒द् दक्षि॑णेन॒ दक्षि॑णेन॒ तस्मा॒त् तस्मा॒द् दक्षि॑णेन । \newline
25. दक्षि॑णे॒ नान्न॒ मन्न॒म् दक्षि॑णेन॒ दक्षि॑णे॒ नान्न᳚म् । \newline
26. अन्न॑ मद्यते ऽद्य॒ते ऽन्न॒ मन्न॑ मद्यते । \newline
27. अ॒द्य॒ते॒ ध॒रुणो॑ ध॒रुणो᳚ ऽद्यते ऽद्यते ध॒रुणः॑ । \newline
28. ध॒रुण॑ एकविꣳ॒॒श ए॑कविꣳ॒॒शो ध॒रुणो॑ ध॒रुण॑ एकविꣳ॒॒शः । \newline
29. ए॒क॒विꣳ॒॒श इतीत्ये॑कविꣳ॒॒श ए॑कविꣳ॒॒श इति॑ । \newline
30. ए॒क॒विꣳ॒॒श इत्ये॑क - विꣳ॒॒शः । \newline
31. इति॑ प॒श्चात् प॒श्चा दितीति॑ प॒श्चात् । \newline
32. प॒श्चात् प्र॑ति॒ष्ठा प्र॑ति॒ष्ठा प॒श्चात् प॒श्चात् प्र॑ति॒ष्ठा । \newline
33. प्र॒ति॒ष्ठा वै वै प्र॑ति॒ष्ठा प्र॑ति॒ष्ठा वै । \newline
34. प्र॒ति॒ष्ठेति॑ प्रति - स्था । \newline
35. वा ए॑कविꣳ॒॒श ए॑कविꣳ॒॒शो वै वा ए॑कविꣳ॒॒शः । \newline
36. ए॒क॒विꣳ॒॒शः प्रति॑ष्ठित्यै॒ प्रति॑ष्ठित्या एकविꣳ॒॒श ए॑कविꣳ॒॒शः प्रति॑ष्ठित्यै । \newline
37. ए॒क॒विꣳ॒॒श इत्ये॑क - विꣳ॒॒शः । \newline
38. प्रति॑ष्ठित्यै भा॒न्तो भा॒न्तः प्रति॑ष्ठित्यै॒ प्रति॑ष्ठित्यै भा॒न्तः । \newline
39. प्रति॑ष्ठित्या॒ इति॒ प्रति॑ - स्थि॒त्यै॒ । \newline
40. भा॒न्तः प॑ञ्चद॒शः प॑ञ्चद॒शो भा॒न्तो भा॒न्तः प॑ञ्चद॒शः । \newline
41. प॒ञ्च॒द॒श इतीति॑ पञ्चद॒शः प॑ञ्चद॒श इति॑ । \newline
42. प॒ञ्च॒द॒श इति॑ पञ्च - द॒शः । \newline
43. इत्यु॑त्तर॒त उ॑त्तर॒त इतीत्यु॑ त्तर॒तः । \newline
44. उ॒त्त॒र॒त ओज॒ ओज॑ उत्तर॒त उ॑त्तर॒त ओजः॑ । \newline
45. उ॒त्त॒र॒त इत्यु॑त् - त॒र॒तः । \newline
46. ओजो॒ वै वा ओज॒ ओजो॒ वै । \newline
47. वै भा॒न्तो भा॒न्तो वै वै भा॒न्तः । \newline
48. भा॒न्त ओज॒ ओजो॑ भा॒न्तो भा॒न्त ओजः॑ । \newline
49. ओजः॑ पञ्चद॒शः प॑ञ्चद॒श ओज॒ ओजः॑ पञ्चद॒शः । \newline
50. प॒ञ्च॒द॒श ओज॒ ओजः॑ पञ्चद॒शः प॑ञ्चद॒श ओजः॑ । \newline
51. प॒ञ्च॒द॒श इति॑ पञ्च - द॒शः । \newline
52. ओज॑ ए॒वैवौज॒ ओज॑ ए॒व । \newline
53. ए॒वोत्त॑र॒त उ॑त्तर॒त ए॒वैवोत्त॑र॒तः । \newline
54. उ॒त्त॒र॒तो ध॑त्ते धत्त उत्तर॒त उ॑त्तर॒तो ध॑त्ते । \newline
55. उ॒त्त॒र॒त इत्यु॑त् - त॒र॒तः । \newline
56. ध॒त्ते॒ तस्मा॒त् तस्मा᳚द् धत्ते धत्ते॒ तस्मा᳚त् । \newline
57. तस्मा॑ दुत्तरतोभिप्रया॒ य्यु॑त्तरतोभिप्रया॒यी तस्मा॒त् तस्मा॑ दुत्तरतोभिप्रया॒यी । \newline
58. उ॒त्त॒र॒तो॒भि॒प्र॒या॒यी ज॑यति जयत्युत्तरतोभिप्रया॒ य्यु॑त्तरतोभिप्रया॒यी ज॑यति । \newline
59. उ॒त्त॒र॒तो॒भि॒प्र॒या॒यीत्यु॑त्तरतः - अ॒भि॒प्र॒या॒यी । \newline
60. ज॒य॒ति॒ प्रतू᳚र्तिः॒ प्रतू᳚र्तिर् जयति जयति॒ प्रतू᳚र्तिः । \newline
61. प्रतू᳚र्ति रष्टाद॒शो᳚ ऽष्टाद॒शः प्रतू᳚र्तिः॒ प्रतू᳚र्ति रष्टाद॒शः । \newline
62. प्रतू᳚र्ति॒रिति॒ प्र - तू॒र्तिः॒ । \newline
63. अ॒ष्टा॒द॒श इतीत्य॑ ष्टाद॒शो᳚ ऽष्टाद॒श इति॑ । \newline
64. अ॒ष्टा॒द॒श इत्य॑ष्टा - द॒शः । \newline
65. इति॑ पु॒रस्ता᳚त् पु॒रस्ता॒ दितीति॑ पु॒रस्ता᳚त् । \newline
66. पु॒रस्ता॒ दुपोप॑ पु॒रस्ता᳚त् पु॒रस्ता॒ दुप॑ । \newline

\textbf{Ghana Paata } \newline

1. य॒ज्ञ्॒मु॒ख मे॒वैव य॑ज्ञ्मु॒खं ॅय॑ज्ञ्मु॒ख मे॒व पु॒रस्ता᳚त् पु॒रस्ता॑ दे॒व य॑ज्ञ्मु॒खं ॅय॑ज्ञ्मु॒ख मे॒व पु॒रस्ता᳚त् । \newline
2. य॒ज्ञ्॒मु॒खमिति॑ यज्ञ् - मु॒खम् । \newline
3. ए॒व पु॒रस्ता᳚त् पु॒रस्ता॑ दे॒वैव पु॒रस्ता॒द् वि वि पु॒रस्ता॑ दे॒वैव पु॒रस्ता॒द् वि । \newline
4. पु॒रस्ता॒द् वि वि पु॒रस्ता᳚त् पु॒रस्ता॒द् वि या॑तयति यातयति॒ वि पु॒रस्ता᳚त् पु॒रस्ता॒द् वि या॑तयति । \newline
5. वि या॑तयति यातयति॒ वि वि या॑तयति॒ व्यो॑म॒ व्यो॑म यातयति॒ वि वि या॑तयति॒ व्यो॑म । \newline
6. या॒त॒य॒ति॒ व्यो॑म॒ व्यो॑म यातयति यातयति॒ व्यो॑म सप्तद॒शः स॑प्तद॒शो व्यो॑म यातयति यातयति॒ व्यो॑म सप्तद॒शः । \newline
7. व्यो॑म सप्तद॒शः स॑प्तद॒शो व्यो॑म॒ व्यो॑म सप्तद॒श इतीति॑ सप्तद॒शो व्यो॑म॒ व्यो॑म सप्तद॒श इति॑ । \newline
8. व्यो॑मेति॒ वि - ओ॒म॒ । \newline
9. स॒प्त॒द॒श इतीति॑ सप्तद॒शः स॑प्तद॒श इति॑ दक्षिण॒तो द॑क्षिण॒त इति॑ सप्तद॒शः स॑प्तद॒श इति॑ दक्षिण॒तः । \newline
10. स॒प्त॒द॒श इति॑ सप्त - द॒शः । \newline
11. इति॑ दक्षिण॒तो द॑क्षिण॒त इतीति॑ दक्षिण॒तो ऽन्न॒ मन्न॑म् दक्षिण॒त इतीति॑ दक्षिण॒तो ऽन्न᳚म् । \newline
12. द॒क्षि॒ण॒तो ऽन्न॒ मन्न॑म् दक्षिण॒तो द॑क्षिण॒तो ऽन्नं॒ ॅवै वा अन्न॑म् दक्षिण॒तो द॑क्षिण॒तो ऽन्नं॒ ॅवै । \newline
13. अन्नं॒ ॅवै वा अन्न॒ मन्नं॒ ॅवै व्यो॑म॒ व्यो॑म॒ वा अन्न॒ मन्नं॒ ॅवै व्यो॑म । \newline
14. वै व्यो॑म॒ व्यो॑म॒ वै वै व्यो॑मान्न॒ मन्नं॒ ॅव्यो॑म॒ वै वै व्यो॑मान्न᳚म् । \newline
15. व्यो॑मान्न॒ मन्नं॒ ॅव्यो॑म॒ व्यो॑मान्नꣳ॑ सप्तद॒शः स॑प्तद॒शो ऽन्नं॒ ॅव्यो॑म॒ व्यो॑मान्नꣳ॑ सप्तद॒शः । \newline
16. व्यो॑मेति॒ वि - ओ॒म॒ । \newline
17. अन्नꣳ॑ सप्तद॒शः स॑प्तद॒शो ऽन्न॒ मन्नꣳ॑ सप्तद॒शो ऽन्न॒ मन्नꣳ॑ सप्तद॒शो ऽन्न॒ मन्नꣳ॑ सप्तद॒शो ऽन्न᳚म् । \newline
18. स॒प्त॒द॒शो ऽन्न॒ मन्नꣳ॑ सप्तद॒शः स॑प्तद॒शो ऽन्न॑ मे॒वैवान्नꣳ॑ सप्तद॒शः स॑प्तद॒शो ऽन्न॑ मे॒व । \newline
19. स॒प्त॒द॒श इति॑ सप्त - द॒शः । \newline
20. अन्न॑ मे॒वैवान्न॒ मन्न॑ मे॒व द॑क्षिण॒तो द॑क्षिण॒त ए॒वान्न॒ मन्न॑ मे॒व द॑क्षिण॒तः । \newline
21. ए॒व द॑क्षिण॒तो द॑क्षिण॒त ए॒वैव द॑क्षिण॒तो ध॑त्ते धत्ते दक्षिण॒त ए॒वैव द॑क्षिण॒तो ध॑त्ते । \newline
22. द॒क्षि॒ण॒तो ध॑त्ते धत्ते दक्षिण॒तो द॑क्षिण॒तो ध॑त्ते॒ तस्मा॒त् तस्मा᳚द् धत्ते दक्षिण॒तो द॑क्षिण॒तो ध॑त्ते॒ तस्मा᳚त् । \newline
23. ध॒त्ते॒ तस्मा॒त् तस्मा᳚द् धत्ते धत्ते॒ तस्मा॒द् दक्षि॑णेन॒ दक्षि॑णेन॒ तस्मा᳚द् धत्ते धत्ते॒ तस्मा॒द् दक्षि॑णेन । \newline
24. तस्मा॒द् दक्षि॑णेन॒ दक्षि॑णेन॒ तस्मा॒त् तस्मा॒द् दक्षि॑णे॒ नान्न॒ मन्न॒म् दक्षि॑णेन॒ तस्मा॒त् तस्मा॒द् दक्षि॑णे॒ नान्न᳚म् । \newline
25. दक्षि॑णे॒ नान्न॒ मन्न॒म् दक्षि॑णेन॒ दक्षि॑णे॒ नान्न॑ मद्यते ऽद्य॒ते ऽन्न॒म् दक्षि॑णेन॒ दक्षि॑णे॒ नान्न॑ मद्यते । \newline
26. अन्न॑ मद्यते ऽद्य॒ते ऽन्न॒ मन्न॑ मद्यते ध॒रुणो॑ ध॒रुणो᳚ ऽद्य॒ते ऽन्न॒ मन्न॑ मद्यते ध॒रुणः॑ । \newline
27. अ॒द्य॒ते॒ ध॒रुणो॑ ध॒रुणो᳚ ऽद्यते ऽद्यते ध॒रुण॑ एकविꣳ॒॒श ए॑कविꣳ॒॒शो ध॒रुणो᳚ ऽद्यते ऽद्यते ध॒रुण॑ एकविꣳ॒॒शः । \newline
28. ध॒रुण॑ एकविꣳ॒॒श ए॑कविꣳ॒॒शो ध॒रुणो॑ ध॒रुण॑ एकविꣳ॒॒श इती त्ये॑कविꣳ॒॒शो ध॒रुणो॑ ध॒रुण॑ एकविꣳ॒॒श इति॑ । \newline
29. ए॒क॒विꣳ॒॒श इती त्ये॑कविꣳ॒॒श ए॑कविꣳ॒॒श इति॑ प॒श्चात् प॒श्चा दित्ये॑कविꣳ॒॒श ए॑कविꣳ॒॒श इति॑ प॒श्चात् । \newline
30. ए॒क॒विꣳ॒॒श इत्ये॑क - विꣳ॒॒शः । \newline
31. इति॑ प॒श्चात् प॒श्चा दितीति॑ प॒श्चात् प्र॑ति॒ष्ठा प्र॑ति॒ष्ठा प॒श्चा दितीति॑ प॒श्चात् प्र॑ति॒ष्ठा । \newline
32. प॒श्चात् प्र॑ति॒ष्ठा प्र॑ति॒ष्ठा प॒श्चात् प॒श्चात् प्र॑ति॒ष्ठा वै वै प्र॑ति॒ष्ठा प॒श्चात् प॒श्चात् प्र॑ति॒ष्ठा वै । \newline
33. प्र॒ति॒ष्ठा वै वै प्र॑ति॒ष्ठा प्र॑ति॒ष्ठा वा ए॑कविꣳ॒॒श ए॑कविꣳ॒॒शो वै प्र॑ति॒ष्ठा प्र॑ति॒ष्ठा वा ए॑कविꣳ॒॒शः । \newline
34. प्र॒ति॒ष्ठेति॑ प्रति - स्था । \newline
35. वा ए॑कविꣳ॒॒श ए॑कविꣳ॒॒शो वै वा ए॑कविꣳ॒॒शः प्रति॑ष्ठित्यै॒ प्रति॑ष्ठित्या एकविꣳ॒॒शो वै वा ए॑कविꣳ॒॒शः प्रति॑ष्ठित्यै । \newline
36. ए॒क॒विꣳ॒॒शः प्रति॑ष्ठित्यै॒ प्रति॑ष्ठित्या एकविꣳ॒॒श ए॑कविꣳ॒॒शः प्रति॑ष्ठित्यै भा॒न्तो भा॒न्तः प्रति॑ष्ठित्या एकविꣳ॒॒श ए॑कविꣳ॒॒शः प्रति॑ष्ठित्यै भा॒न्तः । \newline
37. ए॒क॒विꣳ॒॒श इत्ये॑क - विꣳ॒॒शः । \newline
38. प्रति॑ष्ठित्यै भा॒न्तो भा॒न्तः प्रति॑ष्ठित्यै॒ प्रति॑ष्ठित्यै भा॒न्तः प॑ञ्चद॒शः प॑ञ्चद॒शो भा॒न्तः प्रति॑ष्ठित्यै॒ प्रति॑ष्ठित्यै भा॒न्तः प॑ञ्चद॒शः । \newline
39. प्रति॑ष्ठित्या॒ इति॒ प्रति॑ - स्थि॒त्यै॒ । \newline
40. भा॒न्तः प॑ञ्चद॒शः प॑ञ्चद॒शो भा॒न्तो भा॒न्तः प॑ञ्चद॒श इतीति॑ पञ्चद॒शो भा॒न्तो भा॒न्तः प॑ञ्चद॒श इति॑ । \newline
41. प॒ञ्च॒द॒श इतीति॑ पञ्चद॒शः प॑ञ्चद॒श इत्यु॑त्तर॒त उ॑त्तर॒त इति॑ पञ्चद॒शः प॑ञ्चद॒श इत्यु॑त्तर॒तः । \newline
42. प॒ञ्च॒द॒श इति॑ पञ्च - द॒शः । \newline
43. इत्यु॑त्तर॒त उ॑त्तर॒त इती त्यु॑त्तर॒त ओज॒ ओज॑ उत्तर॒त इती त्यु॑त्तर॒त ओजः॑ । \newline
44. उ॒त्त॒र॒त ओज॒ ओज॑ उत्तर॒त उ॑त्तर॒त ओजो॒ वै वा ओज॑ उत्तर॒त उ॑त्तर॒त ओजो॒ वै । \newline
45. उ॒त्त॒र॒त इत्यु॑त् - त॒र॒तः । \newline
46. ओजो॒ वै वा ओज॒ ओजो॒ वै भा॒न्तो भा॒न्तो वा ओज॒ ओजो॒ वै भा॒न्तः । \newline
47. वै भा॒न्तो भा॒न्तो वै वै भा॒न्त ओज॒ ओजो॑ भा॒न्तो वै वै भा॒न्त ओजः॑ । \newline
48. भा॒न्त ओज॒ ओजो॑ भा॒न्तो भा॒न्त ओजः॑ पञ्चद॒शः प॑ञ्चद॒श ओजो॑ भा॒न्तो भा॒न्त ओजः॑ पञ्चद॒शः । \newline
49. ओजः॑ पञ्चद॒शः प॑ञ्चद॒श ओज॒ ओजः॑ पञ्चद॒श ओज॒ ओजः॑ पञ्चद॒श ओज॒ ओजः॑ पञ्चद॒श ओजः॑ । \newline
50. प॒ञ्च॒द॒श ओज॒ ओजः॑ पञ्चद॒शः प॑ञ्चद॒श ओज॑ ए॒वै वौजः॑ पञ्चद॒शः प॑ञ्चद॒श ओज॑ ए॒व । \newline
51. प॒ञ्च॒द॒श इति॑ पञ्च - द॒शः । \newline
52. ओज॑ ए॒वैवौज॒ ओज॑ ए॒वोत्त॑र॒त उ॑त्तर॒त ए॒वौज॒ ओज॑ ए॒वोत्त॑र॒तः । \newline
53. ए॒वोत्त॑र॒त उ॑त्तर॒त ए॒वैवोत्त॑र॒तो ध॑त्ते धत्त उत्तर॒त ए॒वैवोत्त॑र॒तो ध॑त्ते । \newline
54. उ॒त्त॒र॒तो ध॑त्ते धत्त उत्तर॒त उ॑त्तर॒तो ध॑त्ते॒ तस्मा॒त् तस्मा᳚द् धत्त उत्तर॒त उ॑त्तर॒तो ध॑त्ते॒ तस्मा᳚त् । \newline
55. उ॒त्त॒र॒त इत्यु॑त् - त॒र॒तः । \newline
56. ध॒त्ते॒ तस्मा॒त् तस्मा᳚द् धत्ते धत्ते॒ तस्मा॑ दुत्तरतोभिप्रया॒ य्यु॑त्तरतोभिप्रया॒यी तस्मा᳚द् धत्ते धत्ते॒ तस्मा॑ दुत्तरतोभिप्रया॒यी । \newline
57. तस्मा॑ दुत्तरतोभिप्रया॒ य्यु॑त्तरतोभिप्रया॒यी तस्मा॒त् तस्मा॑ दुत्तरतोभिप्रया॒यी ज॑यति जय त्युत्तरतोभिप्रया॒यी तस्मा॒त् तस्मा॑ दुत्तरतोभिप्रया॒यी ज॑यति । \newline
58. उ॒त्त॒र॒तो॒भि॒प्र॒या॒यी ज॑यति जय त्युत्तरतोभिप्रया॒ य्यु॑त्तरतोभिप्रया॒यी ज॑यति॒ प्रतू᳚र्तिः॒ प्रतू᳚र्तिर् जय 
त्युत्तरतोभिप्रया॒ य्यु॑त्तरतोभिप्रया॒यी ज॑यति॒ प्रतू᳚र्तिः । \newline
59. उ॒त्त॒र॒तो॒भि॒प्र॒या॒यीत्यु॑त्तरतः - अ॒भि॒प्र॒या॒यी । \newline
60. ज॒य॒ति॒ प्रतू᳚र्तिः॒ प्रतू᳚र्तिर् जयति जयति॒ प्रतू᳚र्ति रष्टाद॒शो᳚ ऽष्टाद॒शः प्रतू᳚र्तिर् जयति जयति॒ प्रतू᳚र्ति रष्टाद॒शः । \newline
61. प्रतू᳚र्ति रष्टाद॒शो᳚ ऽष्टाद॒शः प्रतू᳚र्तिः॒ प्रतू᳚र्ति रष्टाद॒श इती त्य॑ष्टाद॒शः प्रतू᳚र्तिः॒ प्रतू᳚र्ति रष्टाद॒श इति॑ । \newline
62. प्रतू᳚र्ति॒रिति॒ प्र - तू॒र्तिः॒ । \newline
63. अ॒ष्टा॒द॒श इती त्य॑ष्टाद॒शो᳚ ऽष्टाद॒श इति॑ पु॒रस्ता᳚त् पु॒रस्ता॒दि त्य॑ष्टाद॒शो᳚ ऽष्टाद॒श इति॑ पु॒रस्ता᳚त् । \newline
64. अ॒ष्टा॒द॒श इत्य॑ष्टा - द॒शः । \newline
65. इति॑ पु॒रस्ता᳚त् पु॒रस्ता॒ दितीति॑ पु॒रस्ता॒ दुपोप॑ पु॒रस्ता॒ दितीति॑ पु॒रस्ता॒ दुप॑ । \newline
66. पु॒रस्ता॒ दुपोप॑ पु॒रस्ता᳚त् पु॒रस्ता॒ दुप॑ दधाति दधा॒ त्युप॑ पु॒रस्ता᳚त् पु॒रस्ता॒ दुप॑ दधाति । \newline
\pagebreak
\markright{ TS 5.3.3.3  \hfill https://www.vedavms.in \hfill}

\section{ TS 5.3.3.3 }

\textbf{TS 5.3.3.3 } \newline
\textbf{Samhita Paata} \newline

-दुप॑ दधाति॒ द्वौ त्रि॒वृता॑वभिपू॒र्वं ॅय॑ज्ञ्मु॒खे वि या॑तयत्यभिव॒र्तः स॑विꣳ॒॒श इति॑ दक्षिण॒तोऽन्नं॒ ॅवा अ॑भिव॒र्तोऽन्नꣳ॑ सविꣳ॒॒शोऽन्न॑मे॒व द॑क्षिण॒तो ध॑त्ते॒ तस्मा॒द्-दक्षि॑णे॒नान्न॑मद्यते॒ वर्चो᳚ द्वाविꣳ॒॒श इति॑ प॒श्चाद्-यद्-विꣳ॑श॒तिर्द्वे तेन॑ वि॒राजौ॒ यद् द्वे प्र॑ति॒ष्ठा तेन॑ वि॒राजो॑रे॒वा-भि॑पू॒र्वम॒न्नाद्ये॒ प्रति॑तिष्ठति॒ तपो॑ नवद॒श इत्यु॑त्तर॒ तस्मा᳚थ् स॒व्यो - [  ] \newline

\textbf{Pada Paata} \newline

उपेति॑ । द॒धा॒ति॒ । द्वौ । त्रि॒वृता॒विति॑ त्रि - वृतौ᳚ । अ॒भि॒पू॒र्वमित्य॑भि - पू॒र्वम् । य॒ज्ञ्॒मु॒ख इति॑ यज्ञ् - मु॒खे । वीति॑ । या॒त॒य॒ति॒ । अ॒भि॒व॒र्त इत्य॑भि - व॒र्तः । स॒विꣳ॒॒श इति॑ स-विꣳ॒॒शः । इति॑ । द॒क्षि॒ण॒तः । अन्न᳚म् । वै । अ॒भि॒व॒र्त इत्य॑भि - व॒र्तः । अन्न᳚म् । स॒विꣳ॒॒श इति॑ स - विꣳ॒॒शः । अन्न᳚म् । ए॒व । द॒क्षि॒ण॒तः । ध॒त्ते॒ । तस्मा᳚त् । दक्षि॑णेन । अन्न᳚म् । अ॒द्य॒ते॒ । वर्चः॑ । द्वा॒विꣳ॒॒शः । इति॑ । प॒श्चात् । यत् । विꣳ॒॒श॒तिः । द्वे इति॑ । तेन॑ । वि॒राजा॒विति॑ वि-राजौ᳚ । यत् । द्वे इति॑ । प्र॒ति॒ष्ठेति॑ प्रति - स्था । तेन॑ । वि॒राजो॒रिति॑ वि - राजोः᳚ । ए॒व । अ॒भि॒पू॒र्वमित्य॑भि - पू॒र्वम् । अ॒न्नाद्य॒ इत्य॑न्न - अद्ये᳚ । प्रतीति॑ । ति॒ष्ठ॒ति॒ । तपः॑ । न॒व॒द॒श इति॑ नव - द॒शः । इति॑ । उ॒त्त॒र॒त इत्यु॑त्- त॒र॒तः । तस्मा᳚त् । स॒व्यः ।  \newline


\textbf{Krama Paata} \newline

उप॑ दधाति । द॒धा॒ति॒ द्वौ । द्वौ त्रि॒वृतौ᳚ । त्रि॒वृता॑वभिपू॒र्वम् । त्रि॒वृता॒विति॑ त्रि - वृतौ᳚ । अ॒भि॒पू॒र्वम् ॅय॑ज्ञ्मु॒खे । अ॒भि॒पू॒र्वमित्य॑भि - पू॒र्वम् । य॒ज्ञ्॒मु॒खे वि । य॒ज्ञ्॒मु॒ख इति॑ यज्ञ् - मु॒खे । वि या॑तयति । या॒त॒य॒त्य॒भि॒व॒र्तः । अ॒भि॒व॒र्तः स॑विꣳ॒॒शः । अ॒भि॒व॒र्त इत्य॑भि - व॒र्तः । स॒विꣳ॒॒श इति॑ । स॒विꣳ॒॒श इति॑ स - विꣳ॒॒शः । इति॑ दक्षिण॒तः । द॒क्षि॒ण॒तोऽन्न᳚म् । अन्न॒म् ॅवै । वा अ॑भिव॒र्तः । अ॒भि॒व॒र्तोऽन्न᳚म् । अ॒भि॒व॒र्त इत्य॑भि - व॒र्तः । अन्नꣳ॑ सविꣳ॒॒शः । स॒विꣳ॒॒शोऽन्न᳚म् । स॒विꣳ॒॒श इति॑ स - विꣳ॒॒शः । अन्न॑मे॒व । ए॒व द॑क्षिण॒तः । द॒क्षि॒ण॒तो ध॑त्ते । ध॒त्ते॒ तस्मा᳚त् । तस्मा॒द् दक्षि॑णेन । दक्षि॑णे॒नान्न᳚म् । अन्न॑मद्यते । अ॒द्य॒ते॒ वर्चः॑ । वर्चो᳚ द्वाविꣳ॒॒शः । द्वा॒विꣳ॒॒श इति॑ । इति॑ प॒श्चात् । प॒श्चाद् यत् । यद् विꣳ॑श॒तिः । विꣳ॒॒श॒तिर् द्वे । द्वे तेन॑ । द्वे इति॒ द्वे । तेन॑ वि॒राजौ᳚ । वि॒राजौ॒ यत् । वि॒राजा॒विति॑ वि - राजौ᳚ । यद् द्वे । द्वे प्र॑ति॒ष्ठा । द्वे इति॒ द्वे । प्र॒ति॒ष्ठा तेन॑ । प्र॒ति॒ष्ठेति॑ प्रति - स्था । तेन॑ वि॒राजोः᳚ । वि॒राजो॑रे॒व । वि॒राजो॒रिति॑ वि - राजोः᳚ । ए॒वाभि॑पू॒र्वम् । अ॒भि॒पू॒र्वम॒न्नाद्ये᳚ । अ॒भि॒पू॒र्वमित्य॑भि - पू॒र्वम् । अ॒न्नाद्ये॒ प्रति॑ । अ॒न्नाद्य॒ इत्य॑न्न - अद्ये᳚ । प्रति॑ तिष्ठति । ति॒ष्ठ॒ति॒ तपः॑ । तपो॑ नवद॒शः । न॒व॒द॒श इति॑ । न॒व॒द॒श इति॑ नव - द॒शः । इत्यु॑त्तर॒तः । उ॒त्त॒र॒त स्तस्मा᳚त् । उ॒त्त॒र॒त इत्यु॑त् - त॒र॒तः । तस्मा᳚थ् स॒व्यः । स॒व्यो हस्त॑योः \newline

\textbf{Jatai Paata} \newline

1. उप॑ दधाति दधा॒ त्युपोप॑ दधाति । \newline
2. द॒धा॒ति॒ द्वौ द्वौ द॑धाति दधाति॒ द्वौ । \newline
3. द्वौ त्रि॒वृतौ᳚ त्रि॒वृतौ॒ द्वौ द्वौ त्रि॒वृतौ᳚ । \newline
4. त्रि॒वृता॑ वभिपू॒र्व म॑भिपू॒र्वम् त्रि॒वृतौ᳚ त्रि॒वृता॑ वभिपू॒र्वम् । \newline
5. त्रि॒वृता॒विति॑ त्रि - वृतौ᳚ । \newline
6. अ॒भि॒पू॒र्वं ॅय॑ज्ञ्मु॒खे य॑ज्ञ्मु॒खे॑ ऽभिपू॒र्व म॑भिपू॒र्वं ॅय॑ज्ञ्मु॒खे । \newline
7. अ॒भि॒पू॒र्वमित्य॑भि - पू॒र्वम् । \newline
8. य॒ज्ञ्॒मु॒खे वि वि य॑ज्ञ्मु॒खे य॑ज्ञ्मु॒खे वि । \newline
9. य॒ज्ञ्॒मु॒ख इति॑ यज्ञ् - मु॒खे । \newline
10. वि या॑तयति यातयति॒ वि वि या॑तयति । \newline
11. या॒त॒य॒ त्य॒भि॒व॒र्तो॑ ऽभिव॒र्तो या॑तयति यातय त्यभिव॒र्तः । \newline
12. अ॒भि॒व॒र्तः स॑विꣳ॒॒शः स॑विꣳ॒॒शो॑ ऽभिव॒र्तो॑ ऽभिव॒र्तः स॑विꣳ॒॒शः । \newline
13. अ॒भि॒व॒र्त इत्य॑भि - व॒र्तः । \newline
14. स॒विꣳ॒॒श इतीति॑ सविꣳ॒॒शः स॑विꣳ॒॒श इति॑ । \newline
15. स॒विꣳ॒॒श इति॑ स - विꣳ॒॒शः । \newline
16. इति॑ दक्षिण॒तो द॑क्षिण॒त इतीति॑ दक्षिण॒तः । \newline
17. द॒क्षि॒ण॒तो ऽन्न॒ मन्न॑म् दक्षिण॒तो द॑क्षिण॒तो ऽन्न᳚म् । \newline
18. अन्नं॒ ॅवै वा अन्न॒ मन्नं॒ ॅवै । \newline
19. वा अ॑भिव॒र्तो॑ ऽभिव॒र्तो वै वा अ॑भिव॒र्तः । \newline
20. अ॒भि॒व॒र्तो ऽन्न॒ मन्न॑ मभिव॒र्तो॑ ऽभिव॒र्तो ऽन्न᳚म् । \newline
21. अ॒भि॒व॒र्त इत्य॑भि - व॒र्तः । \newline
22. अन्नꣳ॑ सविꣳ॒॒शः स॑विꣳ॒॒शो ऽन्न॒ मन्नꣳ॑ सविꣳ॒॒शः । \newline
23. स॒विꣳ॒॒शो ऽन्न॒ मन्नꣳ॑ सविꣳ॒॒शः स॑विꣳ॒॒शो ऽन्न᳚म् । \newline
24. स॒विꣳ॒॒श इति॑ स - विꣳ॒॒शः । \newline
25. अन्न॑ मे॒वै वान्न॒ मन्न॑ मे॒व । \newline
26. ए॒व द॑क्षिण॒तो द॑क्षिण॒त ए॒वैव द॑क्षिण॒तः । \newline
27. द॒क्षि॒ण॒तो ध॑त्ते धत्ते दक्षिण॒तो द॑क्षिण॒तो ध॑त्ते । \newline
28. ध॒त्ते॒ तस्मा॒त् तस्मा᳚द् धत्ते धत्ते॒ तस्मा᳚त् । \newline
29. तस्मा॒द् दक्षि॑णेन॒ दक्षि॑णेन॒ तस्मा॒त् तस्मा॒द् दक्षि॑णेन । \newline
30. दक्षि॑णे॒ नान्न॒ मन्न॒म् दक्षि॑णेन॒ दक्षि॑णे॒ नान्न᳚म् । \newline
31. अन्न॑ मद्यते ऽद्य॒ते ऽन्न॒ मन्न॑ मद्यते । \newline
32. अ॒द्य॒ते॒ वर्चो॒ वर्चो᳚ ऽद्यते ऽद्यते॒ वर्चः॑ । \newline
33. वर्चो᳚ द्वाविꣳ॒॒शो द्वा॑विꣳ॒॒शो वर्चो॒ वर्चो᳚ द्वाविꣳ॒॒शः । \newline
34. द्वा॒विꣳ॒॒श इतीति॑ द्वाविꣳ॒॒शो द्वा॑विꣳ॒॒श इति॑ । \newline
35. इति॑ प॒श्चात् प॒श्चा दितीति॑ प॒श्चात् । \newline
36. प॒श्चाद् यद् यत् प॒श्चात् प॒श्चाद् यत् । \newline
37. यद् विꣳ॑श॒तिर् विꣳ॑श॒तिर् यद् यद् विꣳ॑श॒तिः । \newline
38. विꣳ॒॒श॒तिर् द्वे द्वे विꣳ॑श॒तिर् विꣳ॑श॒तिर् द्वे । \newline
39. द्वे तेन॒ तेन॒ द्वे द्वे तेन॑ । \newline
40. द्वे इति॒ द्वे । \newline
41. तेन॑ वि॒राजौ॑ वि॒राजौ॒ तेन॒ तेन॑ वि॒राजौ᳚ । \newline
42. वि॒राजौ॒ यद् यद् वि॒राजौ॑ वि॒राजौ॒ यत् । \newline
43. वि॒राजा॒विति॑ वि - राजौ᳚ । \newline
44. यद् द्वे द्वे यद् यद् द्वे । \newline
45. द्वे प्र॑ति॒ष्ठा प्र॑ति॒ष्ठा द्वे द्वे प्र॑ति॒ष्ठा । \newline
46. द्वे इति॒ द्वे । \newline
47. प्र॒ति॒ष्ठा तेन॒ तेन॑ प्रति॒ष्ठा प्र॑ति॒ष्ठा तेन॑ । \newline
48. प्र॒ति॒ष्ठेति॑ प्रति - स्था । \newline
49. तेन॑ वि॒राजो᳚र् वि॒राजो॒ स्तेन॒ तेन॑ वि॒राजोः᳚ । \newline
50. वि॒राजो॑ रे॒वैव वि॒राजो᳚र् वि॒राजो॑ रे॒व । \newline
51. वि॒राजो॒रिति॑ वि - राजोः᳚ । \newline
52. ए॒वाभि॑पू॒र्व म॑भिपू॒र्व मे॒वैवा भि॑पू॒र्वम् । \newline
53. अ॒भि॒पू॒र्व म॒न्नाद्ये॒ ऽन्नाद्ये॑ ऽभिपू॒र्व म॑भिपू॒र्व म॒न्नाद्ये᳚ । \newline
54. अ॒भि॒पू॒र्वमित्य॑भि - पू॒र्वम् । \newline
55. अ॒न्नाद्ये॒ प्रति॒ प्रत्य॒ न्नाद्ये॒ ऽन्नाद्ये॒ प्रति॑ । \newline
56. अ॒न्नाद्य॒ इत्य॑न्न - अद्ये᳚ । \newline
57. प्रति॑ तिष्ठति तिष्ठति॒ प्रति॒ प्रति॑ तिष्ठति । \newline
58. ति॒ष्ठ॒ति॒ तप॒ स्तप॑ स्तिष्ठति तिष्ठति॒ तपः॑ । \newline
59. तपो॑ नवद॒शो न॑वद॒श स्तप॒ स्तपो॑ नवद॒शः । \newline
60. न॒व॒द॒श इतीति॑ नवद॒शो न॑वद॒श इति॑ । \newline
61. न॒व॒द॒श इति॑ नव - द॒शः । \newline
62. इत्यु॑त्तर॒त उ॑त्तर॒त इती त्यु॑त्तर॒तः । \newline
63. उ॒त्त॒र॒त स्तस्मा॒त् तस्मा॑ दुत्तर॒त उ॑त्तर॒त स्तस्मा᳚त् । \newline
64. उ॒त्त॒र॒त इत्यु॑त् - त॒र॒तः । \newline
65. तस्मा᳚थ् स॒व्यः स॒व्य स्तस्मा॒त् तस्मा᳚थ् स॒व्यः । \newline
66. स॒व्यो हस्त॑यो॒र्॒. हस्त॑योः स॒व्यः स॒व्यो हस्त॑योः । \newline

\textbf{Ghana Paata } \newline

1. उप॑ दधाति दधा॒ त्युपोप॑ दधाति॒ द्वौ द्वौ द॑धा॒ त्युपोप॑ दधाति॒ द्वौ । \newline
2. द॒धा॒ति॒ द्वौ द्वौ द॑धाति दधाति॒ द्वौ त्रि॒वृतौ᳚ त्रि॒वृतौ॒ द्वौ द॑धाति दधाति॒ द्वौ त्रि॒वृतौ᳚ । \newline
3. द्वौ त्रि॒वृतौ᳚ त्रि॒वृतौ॒ द्वौ द्वौ त्रि॒वृता॑ वभिपू॒र्व म॑भिपू॒र्वम् त्रि॒वृतौ॒ द्वौ द्वौ त्रि॒वृता॑ वभिपू॒र्वम् । \newline
4. त्रि॒वृता॑ वभिपू॒र्व म॑भिपू॒र्वम् त्रि॒वृतौ᳚ त्रि॒वृता॑ वभिपू॒र्वं ॅय॑ज्ञ्मु॒खे य॑ज्ञ्मु॒खे॑ ऽभिपू॒र्वम् त्रि॒वृतौ᳚ त्रि॒वृता॑ वभिपू॒र्वं ॅय॑ज्ञ्मु॒खे । \newline
5. त्रि॒वृता॒विति॑ त्रि - वृतौ᳚ । \newline
6. अ॒भि॒पू॒र्वं ॅय॑ज्ञ्मु॒खे य॑ज्ञ्मु॒खे॑ ऽभिपू॒र्व म॑भिपू॒र्वं ॅय॑ज्ञ्मु॒खे वि वि य॑ज्ञ्मु॒खे॑ ऽभिपू॒र्व म॑भिपू॒र्वं ॅय॑ज्ञ्मु॒खे वि । \newline
7. अ॒भि॒पू॒र्वमित्य॑भि - पू॒र्वम् । \newline
8. य॒ज्ञ्॒मु॒खे वि वि य॑ज्ञ्मु॒खे य॑ज्ञ्मु॒खे वि या॑तयति यातयति॒ वि य॑ज्ञ्मु॒खे य॑ज्ञ्मु॒खे वि या॑तयति । \newline
9. य॒ज्ञ्॒मु॒ख इति॑ यज्ञ् - मु॒खे । \newline
10. वि या॑तयति यातयति॒ वि वि या॑तय त्यभिव॒र्तो॑ ऽभिव॒र्तो या॑तयति॒ वि वि या॑तय त्यभिव॒र्तः । \newline
11. या॒त॒य॒ त्य॒भि॒व॒र्तो॑ ऽभिव॒र्तो या॑तयति यातय त्यभिव॒र्तः स॑विꣳ॒॒शः स॑विꣳ॒॒शो॑ ऽभिव॒र्तो या॑तयति यातय त्यभिव॒र्तः स॑विꣳ॒॒शः । \newline
12. अ॒भि॒व॒र्तः स॑विꣳ॒॒शः स॑विꣳ॒॒शो॑ ऽभिव॒र्तो॑ ऽभिव॒र्तः स॑विꣳ॒॒श इतीति॑ सविꣳ॒॒शो॑ ऽभिव॒र्तो॑ ऽभिव॒र्तः स॑विꣳ॒॒श इति॑ । \newline
13. अ॒भि॒व॒र्त इत्य॑भि - व॒र्तः । \newline
14. स॒विꣳ॒॒श इतीति॑ सविꣳ॒॒शः स॑विꣳ॒॒श इति॑ दक्षिण॒तो द॑क्षिण॒त इति॑ सविꣳ॒॒शः स॑विꣳ॒॒श इति॑ दक्षिण॒तः । \newline
15. स॒विꣳ॒॒श इति॑ स - विꣳ॒॒शः । \newline
16. इति॑ दक्षिण॒तो द॑क्षिण॒त इतीति॑ दक्षिण॒तो ऽन्न॒ मन्न॑म् दक्षिण॒त इतीति॑ दक्षिण॒तो ऽन्न᳚म् । \newline
17. द॒क्षि॒ण॒तो ऽन्न॒ मन्न॑म् दक्षिण॒तो द॑क्षिण॒तो ऽन्नं॒ ॅवै वा अन्न॑म् दक्षिण॒तो द॑क्षिण॒तो ऽन्नं॒ ॅवै । \newline
18. अन्नं॒ ॅवै वा अन्न॒ मन्नं॒ ॅवा अ॑भिव॒र्तो॑ ऽभिव॒र्तो वा अन्न॒ मन्नं॒ ॅवा अ॑भिव॒र्तः । \newline
19. वा अ॑भिव॒र्तो॑ ऽभिव॒र्तो वै वा अ॑भिव॒र्तो ऽन्न॒ मन्न॑ मभिव॒र्तो वै वा अ॑भिव॒र्तो ऽन्न᳚म् । \newline
20. अ॒भि॒व॒र्तो ऽन्न॒ मन्न॑ मभिव॒र्तो॑ ऽभिव॒र्तो ऽन्नꣳ॑ सविꣳ॒॒शः स॑विꣳ॒॒शो ऽन्न॑ मभिव॒र्तो॑ ऽभिव॒र्तो ऽन्नꣳ॑ सविꣳ॒॒शः । \newline
21. अ॒भि॒व॒र्त इत्य॑भि - व॒र्तः । \newline
22. अन्नꣳ॑ सविꣳ॒॒शः स॑विꣳ॒॒शो ऽन्न॒ मन्नꣳ॑ सविꣳ॒॒शो ऽन्न॒ मन्नꣳ॑ सविꣳ॒॒शो ऽन्न॒ मन्नꣳ॑ सविꣳ॒॒शो ऽन्न᳚म् । \newline
23. स॒विꣳ॒॒शो ऽन्न॒ मन्नꣳ॑ सविꣳ॒॒शः स॑विꣳ॒॒शो ऽन्न॑ मे॒वै वान्नꣳ॑ सविꣳ॒॒शः स॑विꣳ॒॒शो ऽन्न॑ मे॒व । \newline
24. स॒विꣳ॒॒श इति॑ स - विꣳ॒॒शः । \newline
25. अन्न॑ मे॒वै वान्न॒ मन्न॑ मे॒व द॑क्षिण॒तो द॑क्षिण॒त ए॒वान्न॒ मन्न॑ मे॒व द॑क्षिण॒तः । \newline
26. ए॒व द॑क्षिण॒तो द॑क्षिण॒त ए॒वैव द॑क्षिण॒तो ध॑त्ते धत्ते दक्षिण॒त ए॒वैव द॑क्षिण॒तो ध॑त्ते । \newline
27. द॒क्षि॒ण॒तो ध॑त्ते धत्ते दक्षिण॒तो द॑क्षिण॒तो ध॑त्ते॒ तस्मा॒त् तस्मा᳚द् धत्ते दक्षिण॒तो द॑क्षिण॒तो ध॑त्ते॒ तस्मा᳚त् । \newline
28. ध॒त्ते॒ तस्मा॒त् तस्मा᳚द् धत्ते धत्ते॒ तस्मा॒द् दक्षि॑णेन॒ दक्षि॑णेन॒ तस्मा᳚द् धत्ते धत्ते॒ तस्मा॒द् दक्षि॑णेन । \newline
29. तस्मा॒द् दक्षि॑णेन॒ दक्षि॑णेन॒ तस्मा॒त् तस्मा॒द् दक्षि॑णे॒ नान्न॒ मन्न॒म् दक्षि॑णेन॒ तस्मा॒त् तस्मा॒द् दक्षि॑णे॒ नान्न᳚म् । \newline
30. दक्षि॑णे॒ नान्न॒ मन्न॒म् दक्षि॑णेन॒ दक्षि॑णे॒ नान्न॑ मद्यते ऽद्य॒ते ऽन्न॒म् दक्षि॑णेन॒ दक्षि॑णे॒ नान्न॑ मद्यते । \newline
31. अन्न॑ मद्यते ऽद्य॒ते ऽन्न॒ मन्न॑ मद्यते॒ वर्चो॒ वर्चो᳚ ऽद्य॒ते ऽन्न॒ मन्न॑ मद्यते॒ वर्चः॑ । \newline
32. अ॒द्य॒ते॒ वर्चो॒ वर्चो᳚ ऽद्यते ऽद्यते॒ वर्चो᳚ द्वाविꣳ॒॒शो द्वा॑विꣳ॒॒शो वर्चो᳚ ऽद्यते ऽद्यते॒ वर्चो᳚ द्वाविꣳ॒॒शः । \newline
33. वर्चो᳚ द्वाविꣳ॒॒शो द्वा॑विꣳ॒॒शो वर्चो॒ वर्चो᳚ द्वाविꣳ॒॒श इतीति॑ द्वाविꣳ॒॒शो वर्चो॒ वर्चो᳚ द्वाविꣳ॒॒श इति॑ । \newline
34. द्वा॒विꣳ॒॒श इतीति॑ द्वाविꣳ॒॒शो द्वा॑विꣳ॒॒श इति॑ प॒श्चात् प॒श्चादिति॑ द्वाविꣳ॒॒शो द्वा॑विꣳ॒॒श इति॑ प॒श्चात् । \newline
35. इति॑ प॒श्चात् प॒श्चा दितीति॑ प॒श्चाद् यद् यत् प॒श्चा दितीति॑ प॒श्चाद् यत् । \newline
36. प॒श्चाद् यद् यत् प॒श्चात् प॒श्चाद् यद् विꣳ॑श॒तिर् विꣳ॑श॒तिर् यत् प॒श्चात् प॒श्चाद् यद् विꣳ॑श॒तिः । \newline
37. यद् विꣳ॑श॒तिर् विꣳ॑श॒तिर् यद् यद् विꣳ॑श॒तिर् द्वे द्वे विꣳ॑श॒तिर् यद् यद् विꣳ॑श॒तिर् द्वे । \newline
38. विꣳ॒॒श॒तिर् द्वे द्वे विꣳ॑श॒तिर् विꣳ॑श॒तिर् द्वे तेन॒ तेन॒ द्वे विꣳ॑श॒तिर् विꣳ॑श॒तिर् द्वे तेन॑ । \newline
39. द्वे तेन॒ तेन॒ द्वे द्वे तेन॑ वि॒राजौ॑ वि॒राजौ॒ तेन॒ द्वे द्वे तेन॑ वि॒राजौ᳚ । \newline
40. द्वे इति॒ द्वे । \newline
41. तेन॑ वि॒राजौ॑ वि॒राजौ॒ तेन॒ तेन॑ वि॒राजौ॒ यद् यद् वि॒राजौ॒ तेन॒ तेन॑ वि॒राजौ॒ यत् । \newline
42. वि॒राजौ॒ यद् यद् वि॒राजौ॑ वि॒राजौ॒ यद् द्वे द्वे यद् वि॒राजौ॑ वि॒राजौ॒ यद् द्वे । \newline
43. वि॒राजा॒विति॑ वि - राजौ᳚ । \newline
44. यद् द्वे द्वे यद् यद् द्वे प्र॑ति॒ष्ठा प्र॑ति॒ष्ठा द्वे यद् यद् द्वे प्र॑ति॒ष्ठा । \newline
45. द्वे प्र॑ति॒ष्ठा प्र॑ति॒ष्ठा द्वे द्वे प्र॑ति॒ष्ठा तेन॒ तेन॑ प्रति॒ष्ठा द्वे द्वे प्र॑ति॒ष्ठा तेन॑ । \newline
46. द्वे इति॒ द्वे । \newline
47. प्र॒ति॒ष्ठा तेन॒ तेन॑ प्रति॒ष्ठा प्र॑ति॒ष्ठा तेन॑ वि॒राजो᳚र् वि॒राजो॒ स्तेन॑ प्रति॒ष्ठा प्र॑ति॒ष्ठा तेन॑ वि॒राजोः᳚ । \newline
48. प्र॒ति॒ष्ठेति॑ प्रति - स्था । \newline
49. तेन॑ वि॒राजो᳚र् वि॒राजो॒ स्तेन॒ तेन॑ वि॒राजो॑ रे॒वैव वि॒राजो॒ स्तेन॒ तेन॑ वि॒राजो॑ रे॒व । \newline
50. वि॒राजो॑ रे॒वैव वि॒राजो᳚र् वि॒राजो॑ रे॒वाभि॑पू॒र्व म॑भिपू॒र्व मे॒व वि॒राजो᳚र् वि॒राजो॑ रे॒वाभि॑पू॒र्वम् । \newline
51. वि॒राजो॒रिति॑ वि - राजोः᳚ । \newline
52. ए॒वाभि॑पू॒र्व म॑भिपू॒र्व मे॒वैवा भि॑पू॒र्व म॒न्नाद्ये॒ ऽन्नाद्ये॑ ऽभिपू॒र्व मे॒वैवा भि॑पू॒र्व म॒न्नाद्ये᳚ । \newline
53. अ॒भि॒पू॒र्व म॒न्नाद्ये॒ ऽन्नाद्ये॑ ऽभिपू॒र्व म॑भिपू॒र्व म॒न्नाद्ये॒ प्रति॒ प्रत्य॒न्नाद्ये॑ ऽभिपू॒र्व म॑भिपू॒र्व म॒न्नाद्ये॒ प्रति॑ । \newline
54. अ॒भि॒पू॒र्वमित्य॑भि - पू॒र्वम् । \newline
55. अ॒न्नाद्ये॒ प्रति॒ प्रत्य॒न्नाद्ये॒ ऽन्नाद्ये॒ प्रति॑ तिष्ठति तिष्ठति॒ प्रत्य॒न्नाद्ये॒ ऽन्नाद्ये॒ प्रति॑ तिष्ठति । \newline
56. अ॒न्नाद्य॒ इत्य॑न्न - अद्ये᳚ । \newline
57. प्रति॑ तिष्ठति तिष्ठति॒ प्रति॒ प्रति॑ तिष्ठति॒ तप॒ स्तप॑ स्तिष्ठति॒ प्रति॒ प्रति॑ तिष्ठति॒ तपः॑ । \newline
58. ति॒ष्ठ॒ति॒ तप॒ स्तप॑ स्तिष्ठति तिष्ठति॒ तपो॑ नवद॒शो न॑वद॒श स्तप॑ स्तिष्ठति तिष्ठति॒ तपो॑ नवद॒शः । \newline
59. तपो॑ नवद॒शो न॑वद॒श स्तप॒ स्तपो॑ नवद॒श इतीति॑ नवद॒श स्तप॒ स्तपो॑ नवद॒श इति॑ । \newline
60. न॒व॒द॒श इतीति॑ नवद॒शो न॑वद॒श इत्यु॑त्तर॒त उ॑त्तर॒त इति॑ नवद॒शो न॑वद॒श इत्यु॑त्तर॒तः । \newline
61. न॒व॒द॒श इति॑ नव - द॒शः । \newline
62. इत्यु॑त्तर॒त उ॑त्तर॒त इती त्यु॑त्तर॒त स्तस्मा॒त् तस्मा॑ दुत्तर॒त इती त्यु॑त्तर॒त स्तस्मा᳚त् । \newline
63. उ॒त्त॒र॒त स्तस्मा॒त् तस्मा॑ दुत्तर॒त उ॑त्तर॒त स्तस्मा᳚थ् स॒व्यः स॒व्य स्तस्मा॑ दुत्तर॒त उ॑त्तर॒त स्तस्मा᳚थ् स॒व्यः । \newline
64. उ॒त्त॒र॒त इत्यु॑त् - त॒र॒तः । \newline
65. तस्मा᳚थ् स॒व्यः स॒व्य स्तस्मा॒त् तस्मा᳚थ् स॒व्यो हस्त॑यो॒र्॒. हस्त॑योः स॒व्य स्तस्मा॒त् तस्मा᳚थ् स॒व्यो हस्त॑योः । \newline
66. स॒व्यो हस्त॑यो॒र्॒. हस्त॑योः स॒व्यः स॒व्यो हस्त॑यो स्तप॒स्वित॑र स्तप॒स्वित॑रो॒ हस्त॑योः स॒व्यः स॒व्यो हस्त॑यो स्तप॒स्वित॑रः । \newline
\pagebreak
\markright{ TS 5.3.3.4  \hfill https://www.vedavms.in \hfill}

\section{ TS 5.3.3.4 }

\textbf{TS 5.3.3.4 } \newline
\textbf{Samhita Paata} \newline

हस्त॑योस्तप॒स्वित॑रो॒ योनि॑श्चतुर्विꣳ॒॒श इति॑ पु॒रस्ता॒दुप॑ दधाति॒ चतु॑र्विꣳशत्यक्षरा गाय॒त्री गा॑य॒त्री य॑ज्ञ्मु॒खं ॅय॑ज्ञ्मु॒खमे॒व पु॒रस्ता॒द्-विया॑तयति॒ गर्भाः᳚ पञ्चविꣳ॒॒श इति॑ दक्षिण॒तोऽन्नं॒ ॅवै गर्भा॒ अन्नं॑ पञ्चविꣳ॒॒शोन्न॑मे॒व द॑क्षिण॒तो ध॑त्ते॒ तस्मा॒द् दक्षि॑णे॒नान्न॑मद्यत॒ ओज॑स्त्रिण॒व इति॑ प॒श्चादि॒मे वै लो॒कास्त्रि॑ण॒व ए॒ष्वे॑व लो॒केषु॒ प्रति॑तिष्ठति स॒भंर॑णस्त्रयोविꣳ॒॒श इत्यु॑ - [  ] \newline

\textbf{Pada Paata} \newline

हस्त॑योः । त॒प॒स्वित॑र॒ इति॑ तप॒स्वि - त॒रः॒ । योनिः॑ । च॒तु॒र्विꣳ॒॒श इति॑ चतुः - विꣳ॒॒शः । इति॑ । पु॒रस्ता᳚त् । उपेति॑ । द॒धा॒ति॒ । चतु॑र्विꣳशत्यक्ष॒रेति॒ चतु॑र्विꣳशति - अ॒क्ष॒रा॒ । गा॒य॒त्री । गा॒य॒त्री । य॒ज्ञ्॒मु॒खमिति॑ यज्ञ् - मु॒खम् । य॒ज्ञ्॒मु॒खमिति॑ यज्ञ् - मु॒खम् । ए॒व । पु॒रस्ता᳚त् । वीति॑ । या॒त॒य॒ति॒ । गर्भाः᳚ । प॒ञ्च॒विꣳ॒॒श इति॑ पञ्च - विꣳ॒॒शः । इति॑ । द॒क्षि॒ण॒तः । अन्न᳚म् । वै । गर्भाः᳚ । अन्न᳚म् । प॒ञ्च॒विꣳ॒॒श इति॑ पञ्च - विꣳ॒॒शः । अन्न᳚म् । ए॒व । द॒क्षि॒ण॒तः । ध॒त्ते॒ । तस्मा᳚त् । दक्षि॑णेन । अन्न᳚म् । अ॒द्य॒ते॒ । ओजः॑ । त्रि॒ण॒व इति॑ त्रि - न॒वः । इति॑ । प॒श्चात् । इ॒मे । वै । लो॒काः । त्रि॒ण॒व इति॑ त्रि - न॒वः । ए॒षु । ए॒व । लो॒केषु॑ । प्रतीति॑ । ति॒ष्ठ॒ति॒ । स॒भंर॑ण॒ इति॑ सं - भर॑णः । त्र॒यो॒विꣳ॒॒श इति॑ त्रयः - विꣳ॒॒शः । इति॑ ।  \newline


\textbf{Krama Paata} \newline

हस्त॑योस्तप॒स्वित॑रः । त॒प॒स्वित॑रो॒ योनिः॑ । त॒प॒स्वित॑र॒ इति॑ तप॒स्वि - त॒रः॒ । योनि॑श्चतुर्विꣳ॒॒शः । च॒तु॒र्विꣳ॒॒श इति॑ । च॒तु॒र्विꣳ॒॒श इति॑ चतुः - विꣳ॒॒शः । इति॑ पु॒रस्ता᳚त् । पु॒रस्ता॒दुप॑ । उप॑ दधाति । द॒धा॒ति॒ चतु॑र्विꣳशत्यक्षरा । चतु॑र्विꣳशत्यक्षरा गाय॒त्री । चतु॑र्विꣳशत्यक्ष॒रेति॒ चतु॑र्विꣳशति - अ॒क्ष॒रा॒ । गा॒य॒त्री गा॑य॒त्री । गा॒य॒त्री य॑ज्ञ्मु॒खम् । य॒ज्ञ्॒मु॒खम् ॅय॑ज्ञ्मु॒खम् । य॒ज्ञ्॒मु॒खमिति॑ यज्ञ् - मु॒खम् । य॒ज्ञ्॒मु॒खमे॒व । य॒ज्ञ्॒मु॒खमिति॑ यज्ञ् - मु॒खम् । ए॒व पु॒रस्ता᳚त् । पु॒रस्ता॒द् वि । वि या॑तयति । या॒त॒य॒ति॒ गर्भाः᳚ । गर्भाः᳚ पञ्चविꣳ॒॒शः । प॒ञ्च॒विꣳ॒॒श इति॑ । प॒ञ्च॒विꣳ॒॒श इति॑ पञ्च - विꣳ॒॒शः । इति॑ दक्षिण॒तः । द॒क्षि॒ण॒तोऽन्न᳚म् । अन्न॒म् ॅवै । वै गर्भाः᳚ । गर्भा॒ अन्न᳚म् । अन्न॑म् पञ्चविꣳ॒॒शः । प॒ञ्च॒विꣳ॒॒शोऽन्न᳚म् । प॒ञ्च॒विꣳ॒॒श इति॑ पञ्च - विꣳ॒॒शः । अन्न॑मे॒व । ए॒व द॑क्षिण॒तः । द॒क्षि॒ण॒तो ध॑त्ते । ध॒त्ते॒ तस्मा᳚त् । तस्मा॒द् दक्षि॑णेन । दक्षि॑णे॒नान्न᳚म् । अन्न॑मद्यते । अ॒द्य॒त॒ ओजः॑ । ओज॑स्त्रिण॒वः । त्रि॒ण॒व इति॑ । त्रि॒ण॒व इति॑ त्रि - न॒वः । इति॑ प॒श्चात् । प॒श्चादि॒मे । इ॒मे वै । वै लो॒काः । लो॒कास्त्रि॑ण॒वः । त्रि॒ण॒व ए॒षु । त्रि॒ण॒व इति॑ त्रि - न॒वः । ए॒ष्वे॑व । ए॒व लो॒केषु॑ । लो॒केषु॒ प्रति॑ । प्रति॑ तिष्ठति । ति॒ष्ठ॒ति॒ स॒म्भर॑णः । स॒म्भर॑ण स्त्रयोविꣳ॒॒शः । स॒म्भर॑ण॒ इति॑ सम् - भर॑णः । त्र॒यो॒विꣳ॒॒श इति॑ । त्र॒यो॒विꣳ॒॒श इति॑ त्रयः - विꣳ॒॒शः । इत्यु॑त्तर॒तः \newline

\textbf{Jatai Paata} \newline

1. हस्त॑यो स्तप॒स्वित॑र स्तप॒स्वित॑रो॒ हस्त॑यो॒र्॒. हस्त॑यो स्तप॒स्वित॑रः । \newline
2. त॒प॒स्वित॑रो॒ योनि॒र् योनि॑ स्तप॒स्वित॑ रस्तप॒स्वित॑रो॒ योनिः॑ । \newline
3. त॒प॒स्वित॑र॒ इति॑ तप॒स्वि - त॒रः॒ । \newline
4. योनि॑ श्चतुर्विꣳ॒॒श श्च॑तुर्विꣳ॒॒शो योनि॒र् योनि॑ श्चतुर्विꣳ॒॒शः । \newline
5. च॒तु॒र्विꣳ॒॒श इतीति॑ चतुर्विꣳ॒॒श श्च॑तुर्विꣳ॒॒श इति॑ । \newline
6. च॒तु॒र्विꣳ॒॒श इति॑ चतुः - विꣳ॒॒शः । \newline
7. इति॑ पु॒रस्ता᳚त् पु॒रस्ता॒ दितीति॑ पु॒रस्ता᳚त् । \newline
8. पु॒रस्ता॒ दुपोप॑ पु॒रस्ता᳚त् पु॒रस्ता॒ दुप॑ । \newline
9. उप॑ दधाति दधा॒ त्युपोप॑ दधाति । \newline
10. द॒धा॒ति॒ चतु॑र्विꣳशत्यक्षरा॒ चतु॑र्विꣳशत्यक्षरा दधाति दधाति॒ चतु॑र्विꣳशत्यक्षरा । \newline
11. चतु॑र्विꣳशत्यक्षरा गाय॒त्री गा॑य॒त्री चतु॑र्विꣳशत्यक्षरा॒ चतु॑र्विꣳशत्यक्षरा गाय॒त्री । \newline
12. चतु॑र्विꣳशत्यक्ष॒रेति॒ चतु॑र्विꣳशति - अ॒क्ष॒रा॒ । \newline
13. गा॒य॒त्री गा॑य॒त्री । \newline
14. गा॒य॒त्री य॑ज्ञ्मु॒खं ॅय॑ज्ञ्मु॒खम् गा॑य॒त्री गा॑य॒त्री य॑ज्ञ्मु॒खम् । \newline
15. य॒ज्ञ्॒मु॒खं ॅय॑ज्ञ्मु॒खम् । \newline
16. य॒ज्ञ्॒मु॒खमिति॑ यज्ञ् - मु॒खम् । \newline
17. य॒ज्ञ्॒मु॒ख मे॒वैव य॑ज्ञ्मु॒खं ॅय॑ज्ञ्मु॒ख मे॒व । \newline
18. य॒ज्ञ्॒मु॒खमिति॑ यज्ञ् - मु॒खम् । \newline
19. ए॒व पु॒रस्ता᳚त् पु॒रस्ता॑ दे॒वैव पु॒रस्ता᳚त् । \newline
20. पु॒रस्ता॒द् वि वि पु॒रस्ता᳚त् पु॒रस्ता॒द् वि । \newline
21. वि या॑तयति यातयति॒ वि वि या॑तयति । \newline
22. या॒त॒य॒ति॒ गर्भा॒ गर्भा॑ यातयति यातयति॒ गर्भाः᳚ । \newline
23. गर्भाः᳚ पञ्चविꣳ॒॒शः प॑ञ्चविꣳ॒॒शो गर्भा॒ गर्भाः᳚ पञ्चविꣳ॒॒शः । \newline
24. प॒ञ्च॒विꣳ॒॒श इतीति॑ पञ्चविꣳ॒॒शः प॑ञ्चविꣳ॒॒श इति॑ । \newline
25. प॒ञ्च॒विꣳ॒॒श इति॑ पञ्च - विꣳ॒॒शः । \newline
26. इति॑ दक्षिण॒तो द॑क्षिण॒त इतीति॑ दक्षिण॒तः । \newline
27. द॒क्षि॒ण॒तो ऽन्न॒ मन्न॑म् दक्षिण॒तो द॑क्षिण॒तो ऽन्न᳚म् । \newline
28. अन्नं॒ ॅवै वा अन्न॒ मन्नं॒ ॅवै । \newline
29. वै गर्भा॒ गर्भा॒ वै वै गर्भाः᳚ । \newline
30. गर्भा॒ अन्न॒ मन्न॒म् गर्भा॒ गर्भा॒ अन्न᳚म् । \newline
31. अन्न॑म् पञ्चविꣳ॒॒शः प॑ञ्चविꣳ॒॒शो ऽन्न॒ मन्न॑म् पञ्चविꣳ॒॒शः । \newline
32. प॒ञ्च॒विꣳ॒॒शो ऽन्न॒ मन्न॑म् पञ्चविꣳ॒॒शः प॑ञ्चविꣳ॒॒शो ऽन्न᳚म् । \newline
33. प॒ञ्च॒विꣳ॒॒श इति॑ पञ्च - विꣳ॒॒शः । \newline
34. अन्न॑ मे॒वै वान्न॒ मन्न॑ मे॒व । \newline
35. ए॒व द॑क्षिण॒तो द॑क्षिण॒त ए॒वैव द॑क्षिण॒तः । \newline
36. द॒क्षि॒ण॒तो ध॑त्ते धत्ते दक्षिण॒तो द॑क्षिण॒तो ध॑त्ते । \newline
37. ध॒त्ते॒ तस्मा॒त् तस्मा᳚द् धत्ते धत्ते॒ तस्मा᳚त् । \newline
38. तस्मा॒द् दक्षि॑णेन॒ दक्षि॑णेन॒ तस्मा॒त् तस्मा॒द् दक्षि॑णेन । \newline
39. दक्षि॑णे॒ नान्न॒ मन्न॒म् दक्षि॑णेन॒ दक्षि॑णे॒ नान्न᳚म् । \newline
40. अन्न॑ मद्यते ऽद्य॒ते ऽन्न॒ मन्न॑ मद्यते । \newline
41. अ॒द्य॒त॒ ओज॒ ओजो᳚ ऽद्यते ऽद्यत॒ ओजः॑ । \newline
42. ओज॑ स्त्रिण॒व स्त्रि॑ण॒व ओज॒ ओज॑ स्त्रिण॒वः । \newline
43. त्रि॒ण॒व इतीति॑ त्रिण॒व स्त्रि॑ण॒व इति॑ । \newline
44. त्रि॒ण॒व इति॑ त्रि - न॒वः । \newline
45. इति॑ प॒श्चात् प॒श्चा दितीति॑ प॒श्चात् । \newline
46. प॒श्चा दि॒म इ॒मे प॒श्चात् प॒श्चा दि॒मे । \newline
47. इ॒मे वै वा इ॒म इ॒मे वै । \newline
48. वै लो॒का लो॒का वै वै लो॒काः । \newline
49. लो॒का स्त्रि॑ण॒व स्त्रि॑ण॒वो लो॒का लो॒का स्त्रि॑ण॒वः । \newline
50. त्रि॒ण॒व ए॒ष्वे॑षु त्रि॑ण॒व स्त्रि॑ण॒व ए॒षु । \newline
51. त्रि॒ण॒व इति॑ त्रि - न॒वः । \newline
52. ए॒ष्वे॑ वैवै ष्वे᳚(1॒) ष्वे॑व । \newline
53. ए॒व लो॒केषु॑ लो॒केष्वे॒ वैव लो॒केषु॑ । \newline
54. लो॒केषु॒ प्रति॒ प्रति॑ लो॒केषु॑ लो॒केषु॒ प्रति॑ । \newline
55. प्रति॑ तिष्ठति तिष्ठति॒ प्रति॒ प्रति॑ तिष्ठति । \newline
56. ति॒ष्ठ॒ति॒ सं॒भर॑णः सं॒भर॑ण स्तिष्ठति तिष्ठति सं॒भर॑णः । \newline
57. सं॒भर॑ण स्त्रयोविꣳ॒॒श स्त्र॑योविꣳ॒॒शः सं॒भर॑णः सं॒भर॑ण स्त्रयोविꣳ॒॒शः । \newline
58. सं॒भर॑ण॒ इति॑ सं - भर॑णः । \newline
59. त्र॒यो॒विꣳ॒॒श इतीति॑ त्रयोविꣳ॒॒श स्त्र॑योविꣳ॒॒श इति॑ । \newline
60. त्र॒यो॒विꣳ॒॒श इति॑ त्रयः - विꣳ॒॒शः । \newline
61. इत्यु॑त्तर॒त उ॑त्तर॒त इती त्यु॑त्तर॒तः । \newline

\textbf{Ghana Paata } \newline

1. हस्त॑यो स्तप॒स्वित॑र स्तप॒स्वित॑रो॒ हस्त॑यो॒र्॒. हस्त॑यो स्तप॒स्वित॑रो॒ योनि॒र् योनि॑ स्तप॒स्वित॑रो॒ हस्त॑यो॒र्॒. हस्त॑यो स्तप॒स्वित॑रो॒ योनिः॑ । \newline
2. त॒प॒स्वित॑रो॒ योनि॒र् योनि॑ स्तप॒स्वित॑र स्तप॒स्वित॑रो॒ योनि॑ श्चतुर्विꣳ॒॒श श्च॑तुर्विꣳ॒॒शो योनि॑ स्तप॒स्वित॑र स्तप॒स्वित॑रो॒ योनि॑ श्चतुर्विꣳ॒॒शः । \newline
3. त॒प॒स्वित॑र॒ इति॑ तप॒स्वि - त॒रः॒ । \newline
4. योनि॑ श्चतुर्विꣳ॒॒श श्च॑तुर्विꣳ॒॒शो योनि॒र् योनि॑ श्चतुर्विꣳ॒॒श इतीति॑ चतुर्विꣳ॒॒शो योनि॒र् योनि॑ श्चतुर्विꣳ॒॒श इति॑ । \newline
5. च॒तु॒र्विꣳ॒॒श इतीति॑ चतुर्विꣳ॒॒श श्च॑तुर्विꣳ॒॒श इति॑ पु॒रस्ता᳚त् पु॒रस्ता॒ दिति॑ चतुर्विꣳ॒॒श श्च॑तुर्विꣳ॒॒श इति॑ पु॒रस्ता᳚त् । \newline
6. च॒तु॒र्विꣳ॒॒श इति॑ चतुः - विꣳ॒॒शः । \newline
7. इति॑ पु॒रस्ता᳚त् पु॒रस्ता॒ दितीति॑ पु॒रस्ता॒ दुपोप॑ पु॒रस्ता॒ दितीति॑ पु॒रस्ता॒ दुप॑ । \newline
8. पु॒रस्ता॒ दुपोप॑ पु॒रस्ता᳚त् पु॒रस्ता॒ दुप॑ दधाति दधा॒ त्युप॑ पु॒रस्ता᳚त् पु॒रस्ता॒ दुप॑ दधाति । \newline
9. उप॑ दधाति दधा॒ त्युपोप॑ दधाति॒ चतु॑र्विꣳशत्यक्षरा॒ चतु॑र्विꣳशत्यक्षरा दधा॒ त्युपोप॑ दधाति॒ चतु॑र्विꣳशत्यक्षरा । \newline
10. द॒धा॒ति॒ चतु॑र्विꣳशत्यक्षरा॒ चतु॑र्विꣳशत्यक्षरा दधाति दधाति॒ चतु॑र्विꣳशत्यक्षरा गाय॒त्री गा॑य॒त्री चतु॑र्विꣳशत्यक्षरा दधाति दधाति॒ चतु॑र्विꣳशत्यक्षरा गाय॒त्री । \newline
11. चतु॑र्विꣳशत्यक्षरा गाय॒त्री गा॑य॒त्री चतु॑र्विꣳशत्यक्षरा॒ चतु॑र्विꣳशत्यक्षरा गाय॒त्री । \newline
12. चतु॑र्विꣳशत्यक्ष॒रेति॒ चतु॑र्विꣳशति - अ॒क्ष॒रा॒ । \newline
13. गा॒य॒त्री गा॑य॒त्री । \newline
14. गा॒य॒त्री य॑ज्ञ्मु॒खं ॅय॑ज्ञ्मु॒खम् गा॑य॒त्री गा॑य॒त्री य॑ज्ञ्मु॒खम् । \newline
15. य॒ज्ञ्॒मु॒खं ॅय॑ज्ञ्मु॒खम् । \newline
16. य॒ज्ञ्॒मु॒खमिति॑ यज्ञ् - मु॒खम् । \newline
17. य॒ज्ञ्॒मु॒ख मे॒वैव य॑ज्ञ्मु॒खं ॅय॑ज्ञ्मु॒ख मे॒व पु॒रस्ता᳚त् पु॒रस्ता॑ दे॒व य॑ज्ञ्मु॒खं ॅय॑ज्ञ्मु॒ख मे॒व पु॒रस्ता᳚त् । \newline
18. य॒ज्ञ्॒मु॒खमिति॑ यज्ञ् - मु॒खम् । \newline
19. ए॒व पु॒रस्ता᳚त् पु॒रस्ता॑ दे॒वैव पु॒रस्ता॒द् वि वि पु॒रस्ता॑ दे॒वैव पु॒रस्ता॒द् वि । \newline
20. पु॒रस्ता॒द् वि वि पु॒रस्ता᳚त् पु॒रस्ता॒द् वि या॑तयति यातयति॒ वि पु॒रस्ता᳚त् पु॒रस्ता॒द् वि या॑तयति । \newline
21. वि या॑तयति यातयति॒ वि वि या॑तयति॒ गर्भा॒ गर्भा॑ यातयति॒ वि वि या॑तयति॒ गर्भाः᳚ । \newline
22. या॒त॒य॒ति॒ गर्भा॒ गर्भा॑ यातयति यातयति॒ गर्भाः᳚ पञ्चविꣳ॒॒शः प॑ञ्चविꣳ॒॒शो गर्भा॑ यातयति यातयति॒ गर्भाः᳚ पञ्चविꣳ॒॒शः । \newline
23. गर्भाः᳚ पञ्चविꣳ॒॒शः प॑ञ्चविꣳ॒॒शो गर्भा॒ गर्भाः᳚ पञ्चविꣳ॒॒श इतीति॑ पञ्चविꣳ॒॒शो गर्भा॒ गर्भाः᳚ पञ्चविꣳ॒॒श इति॑ । \newline
24. प॒ञ्च॒विꣳ॒॒श इतीति॑ पञ्चविꣳ॒॒शः प॑ञ्चविꣳ॒॒श इति॑ दक्षिण॒तो द॑क्षिण॒त इति॑ पञ्चविꣳ॒॒शः प॑ञ्चविꣳ॒॒श इति॑ दक्षिण॒तः । \newline
25. प॒ञ्च॒विꣳ॒॒श इति॑ पञ्च - विꣳ॒॒शः । \newline
26. इति॑ दक्षिण॒तो द॑क्षिण॒त इतीति॑ दक्षिण॒तो ऽन्न॒ मन्न॑म् दक्षिण॒त इतीति॑ दक्षिण॒तो ऽन्न᳚म् । \newline
27. द॒क्षि॒ण॒तो ऽन्न॒ मन्न॑म् दक्षिण॒तो द॑क्षिण॒तो ऽन्नं॒ ॅवै वा अन्न॑म् दक्षिण॒तो द॑क्षिण॒तो ऽन्नं॒ ॅवै । \newline
28. अन्नं॒ ॅवै वा अन्न॒ मन्नं॒ ॅवै गर्भा॒ गर्भा॒ वा अन्न॒ मन्नं॒ ॅवै गर्भाः᳚ । \newline
29. वै गर्भा॒ गर्भा॒ वै वै गर्भा॒ अन्न॒ मन्न॒म् गर्भा॒ वै वै गर्भा॒ अन्न᳚म् । \newline
30. गर्भा॒ अन्न॒ मन्न॒म् गर्भा॒ गर्भा॒ अन्न॑म् पञ्चविꣳ॒॒शः प॑ञ्चविꣳ॒॒शो ऽन्न॒म् गर्भा॒ गर्भा॒ अन्न॑म् पञ्चविꣳ॒॒शः । \newline
31. अन्न॑म् पञ्चविꣳ॒॒शः प॑ञ्चविꣳ॒॒शो ऽन्न॒ मन्न॑म् पञ्चविꣳ॒॒शो ऽन्न॒ मन्न॑म् पञ्चविꣳ॒॒शो ऽन्न॒ मन्न॑म् पञ्चविꣳ॒॒शो ऽन्न᳚म् । \newline
32. प॒ञ्च॒विꣳ॒॒शो ऽन्न॒ मन्न॑म् पञ्चविꣳ॒॒शः प॑ञ्चविꣳ॒॒शो ऽन्न॑ मे॒वैवान्न॑म् पञ्चविꣳ॒॒शः प॑ञ्चविꣳ॒॒शो ऽन्न॑ मे॒व । \newline
33. प॒ञ्च॒विꣳ॒॒श इति॑ पञ्च - विꣳ॒॒शः । \newline
34. अन्न॑ मे॒वै वान्न॒ मन्न॑ मे॒व द॑क्षिण॒तो द॑क्षिण॒त ए॒वान्न॒ मन्न॑ मे॒व द॑क्षिण॒तः । \newline
35. ए॒व द॑क्षिण॒तो द॑क्षिण॒त ए॒वैव द॑क्षिण॒तो ध॑त्ते धत्ते दक्षिण॒त ए॒वैव द॑क्षिण॒तो ध॑त्ते । \newline
36. द॒क्षि॒ण॒तो ध॑त्ते धत्ते दक्षिण॒तो द॑क्षिण॒तो ध॑त्ते॒ तस्मा॒त् तस्मा᳚द् धत्ते दक्षिण॒तो द॑क्षिण॒तो ध॑त्ते॒ तस्मा᳚त् । \newline
37. ध॒त्ते॒ तस्मा॒त् तस्मा᳚द् धत्ते धत्ते॒ तस्मा॒द् दक्षि॑णेन॒ दक्षि॑णेन॒ तस्मा᳚द् धत्ते धत्ते॒ तस्मा॒द् दक्षि॑णेन । \newline
38. तस्मा॒द् दक्षि॑णेन॒ दक्षि॑णेन॒ तस्मा॒त् तस्मा॒द् दक्षि॑णे॒नान्न॒ मन्न॒म् दक्षि॑णेन॒ तस्मा॒त् तस्मा॒द् दक्षि॑णे॒नान्न᳚म् । \newline
39. दक्षि॑णे॒नान्न॒ मन्न॒म् दक्षि॑णेन॒ दक्षि॑णे॒नान्न॑ मद्यते ऽद्य॒ते ऽन्न॒म् दक्षि॑णेन॒ दक्षि॑णे॒ नान्न॑ मद्यते । \newline
40. अन्न॑ मद्यते ऽद्य॒ते ऽन्न॒ मन्न॑ मद्यत॒ ओज॒ ओजो᳚ ऽद्य॒ते ऽन्न॒ मन्न॑ मद्यत॒ ओजः॑ । \newline
41. अ॒द्य॒त॒ ओज॒ ओजो᳚ ऽद्यते ऽद्यत॒ ओज॑ स्त्रिण॒व स्त्रि॑ण॒व ओजो᳚ ऽद्यते ऽद्यत॒ ओज॑ स्त्रिण॒वः । \newline
42. ओज॑ स्त्रिण॒व स्त्रि॑ण॒व ओज॒ ओज॑ स्त्रिण॒व इतीति॑ त्रिण॒व ओज॒ ओज॑ स्त्रिण॒व इति॑ । \newline
43. त्रि॒ण॒व इतीति॑ त्रिण॒व स्त्रि॑ण॒व इति॑ प॒श्चात् प॒श्चा दिति॑ त्रिण॒व स्त्रि॑ण॒व इति॑ प॒श्चात् । \newline
44. त्रि॒ण॒व इति॑ त्रि - न॒वः । \newline
45. इति॑ प॒श्चात् प॒श्चा दितीति॑ प॒श्चाद् इ॒म इ॒मे प॒श्चा दितीति॑ प॒श्चा दि॒मे । \newline
46. प॒श्चादि॒म इ॒मे प॒श्चात् प॒श्चादि॒मे वै वा इ॒मे प॒श्चात् प॒श्चादि॒मे वै । \newline
47. इ॒मे वै वा इ॒म इ॒मे वै लो॒का लो॒का वा इ॒म इ॒मे वै लो॒काः । \newline
48. वै लो॒का लो॒का वै वै लो॒का स्त्रि॑ण॒व स्त्रि॑ण॒वो लो॒का वै वै लो॒का स्त्रि॑ण॒वः । \newline
49. लो॒का स्त्रि॑ण॒व स्त्रि॑ण॒वो लो॒का लो॒का स्त्रि॑ण॒व ए॒ष्वे॑षु त्रि॑ण॒वो लो॒का लो॒का स्त्रि॑ण॒व ए॒षु । \newline
50. त्रि॒ण॒व ए॒ष्वे॑षु त्रि॑ण॒व स्त्रि॑ण॒व ए॒ष्वे॑ वैवैषु त्रि॑ण॒व स्त्रि॑ण॒व ए॒ष्वे॑व । \newline
51. त्रि॒ण॒व इति॑ त्रि - न॒वः । \newline
52. ए॒ष्वे॑ वैवैष्वे᳚(1॒)ष्वे॑व लो॒केषु॑ लो॒के ष्वे॒वैष्वे᳚(1॒) ष्वे॑व लो॒केषु॑ । \newline
53. ए॒व लो॒केषु॑ लो॒के ष्वे॒वैव लो॒केषु॒ प्रति॒ प्रति॑ लो॒के ष्वे॒वैव लो॒केषु॒ प्रति॑ । \newline
54. लो॒केषु॒ प्रति॒ प्रति॑ लो॒केषु॑ लो॒केषु॒ प्रति॑ तिष्ठति तिष्ठति॒ प्रति॑ लो॒केषु॑ लो॒केषु॒ प्रति॑ तिष्ठति । \newline
55. प्रति॑ तिष्ठति तिष्ठति॒ प्रति॒ प्रति॑ तिष्ठति सं॒भर॑णः सं॒भर॑ण स्तिष्ठति॒ प्रति॒ प्रति॑ तिष्ठति सं॒भर॑णः । \newline
56. ति॒ष्ठ॒ति॒ सं॒भर॑णः सं॒भर॑ण स्तिष्ठति तिष्ठति सं॒भर॑ण स्त्रयोविꣳ॒॒श 
स्त्र॑योविꣳ॒॒शः सं॒भर॑णस्तिष्ठति तिष्ठति सं॒भर॑ण स्त्रयोविꣳ॒॒शः । \newline
57. सं॒भर॑ण स्त्रयोविꣳ॒॒श स्त्र॑योविꣳ॒॒शः सं॒भर॑णः सं॒भर॑ण स्त्रयोविꣳ॒॒श इतीति॑ त्रयोविꣳ॒॒शः सं॒भर॑णः सं॒भर॑ण स्त्रयोविꣳ॒॒श इति॑ । \newline
58. सं॒भर॑ण॒ इति॑ सं - भर॑णः । \newline
59. त्र॒यो॒विꣳ॒॒श इतीति॑ त्रयोविꣳ॒॒श स्त्र॑योविꣳ॒॒श इत्यु॑त्तर॒त उ॑त्तर॒त इति॑ त्रयोविꣳ॒॒श स्त्र॑योविꣳ॒॒श इत्यु॑त्तर॒तः । \newline
60. त्र॒यो॒विꣳ॒॒श इति॑ त्रयः - विꣳ॒॒शः । \newline
61. इत्यु॑त्तर॒त उ॑त्तर॒त इतीत्यु॑त्तर॒त स्तस्मा॒त् तस्मा॑ दुत्तर॒त इतीत्यु॑त्तर॒त स्तस्मा᳚त् । \newline
\pagebreak
\markright{ TS 5.3.3.5  \hfill https://www.vedavms.in \hfill}

\section{ TS 5.3.3.5 }

\textbf{TS 5.3.3.5 } \newline
\textbf{Samhita Paata} \newline

-त्तर॒तस्तस्मा᳚थ् स॒व्यो हस्त॑योः सम्भा॒र्य॑तरः॒ क्रतु॑रेकत्रिꣳ॒॒श इति॑ पु॒रस्ता॒दुप॑ दधाति॒ वाग्वै क्रतु॑र्यज्ञ्मु॒खं ॅवाग्य॑ज्ञ्मु॒खमे॒व पु॒रस्ता॒द्वि या॑तयति ब्र॒द्ध्नस्य॑ वि॒ष्टपं॑ चतुस्त्रिꣳ॒॒श इति॑ दक्षिण॒तो॑ऽसौ वा आ॑दि॒त्यो ब्र॒द्ध्नस्य॑ वि॒ष्टपं॑ ब्रह्मवर्च॒समे॒व द॑क्षिण॒तो ध॑त्ते॒ तस्मा॒द् दक्षि॒णोऽर्द्धो᳚ ब्रह्मवर्च॒सित॑रः प्रति॒ष्ठा त्र॑यस्त्रिꣳ॒॒श इति॑ प॒श्चात् प्रति॑ष्ठित्यै॒ नाकः॑ षट्त्रिꣳ॒॒श इत्यु॑त्तर॒तः सु॑व॒र्गो वै ( ) लो॒को नाकः॑ सुव॒र्गस्य॑ लो॒कस्य॒ सम॑ष्ट्यै ॥श्पेचिअल् खोर्वै fओर् अनुवाकम्आ॒शु - र्व्यो॑म - ध॒रुणो॑ - भा॒न्तः - प्रतू᳚र्तिर -भिव॒र्तो - वर्च॒ - स्तपो॒ - योनि॒ - र्गर्भा॒ - ओजः॑ - स॒भंर॑णः॒ - क्रतु॑ - र्ब्र॒द्ध्रस्य॑ - प्रति॒ष्ठा - नाकः॒ - षोड॑श) \newline

\textbf{Pada Paata} \newline

उ॒त्त॒र॒त इत्यु॑त् - त॒र॒तः । तस्मा᳚त् । स॒व्यः । हस्त॑योः । स॒भां॒र्य॑तर॒ इति॑ संभा॒र्य॑ - त॒रः॒ । क्रतुः॑ । ए॒क॒त्रिꣳ॒॒श इत्ये॑क - त्रिꣳ॒॒शः । इति॑ । पु॒रस्ता᳚त् । उपेति॑ । द॒धा॒ति॒ । वाक् । वै । क्रतुः॑ । य॒ज्ञ्॒मु॒खमिति॑ यज्ञ् - मु॒खम् । वाक् । य॒ज्ञ्॒मु॒खमिति॑ यज्ञ् - मु॒खम् । ए॒व । पु॒रस्ता᳚त् । वीति॑ । या॒त॒य॒ति॒ । ब्र॒द्ध्नस्य॑ । वि॒ष्टप᳚म् । च॒तु॒स्त्रिꣳ॒॒श इति॑ चतुः - त्रिꣳ॒॒शः । इति॑ । द॒क्षि॒ण॒तः । अ॒सौ । वै । आ॒दि॒त्यः । ब्र॒द्ध्नस्य॑ । वि॒ष्टप᳚म् । ब्र॒ह्म॒व॒र्च॒समिति॑ ब्रह्म - व॒र्च॒सम् । ए॒व । द॒क्षि॒ण॒तः । ध॒त्ते॒ । तस्मा᳚त् । दक्षि॑णः । अद्‌र्धः॑ । ब्र॒ह्म॒व॒र्च॒सित॑र॒ इति॑ ब्रह्मवर्च॒सि - त॒रः॒ । प्र॒ति॒ष्ठेति॑ प्रति - स्था । त्र॒य॒स्त्रिꣳ॒॒श इति॑ त्रयः-त्रिꣳ॒॒शः । इति॑ । प॒श्चात् । प्रति॑ष्ठित्या॒ इति॒ प्रति॑-स्थि॒त्यै॒ । नाकः॑ । ष॒ट्त्रिꣳ॒॒श इति॑ षट्-त्रिꣳ॒॒शः । इति॑ । उ॒त्त॒र॒त इत्यु॑त्-त॒र॒तः । सु॒व॒र्ग इति॑ सुवः - गः । वै ( ) । लो॒कः । नाकः॑ । सु॒व॒र्गस्येति॑ सुवः - गस्य॑ । लो॒कस्य॑ । सम॑ष्ट्या॒ इति॒ सं - अ॒ष्ट्यै॒ ॥आ॒शु - र्व्यो॑म - ध॒रुणो॑ - भा॒न्तः - प्रतू᳚र्तिर -भिव॒र्तो - वर्च॒ - स्तपो॒ - योनि॒ - र्गर्भा॒ - ओजः॑ - स॒भंर॑णः॒ - क्रतु॑ - र्ब्र॒द्ध्रस्य॑ - प्रति॒ष्ठा - नाकः॒ - षोड॑श)  \newline


\textbf{Krama Paata} \newline

उ॒त्त॒र॒तस्तस्मा᳚त् । उ॒त्त॒र॒त इत्यु॑त् - त॒र॒तः । तस्मा᳚थ् स॒व्यः । स॒व्यो हस्त॑योः । हस्त॑योः सम्भा॒र्य॑तरः । स॒म्भा॒र्य॑तरः॒ क्रतुः॑ । स॒म्भा॒र्य॑तर॒ इति॑ सम्भा॒र्य॑ - त॒रः॒ । क्रतु॑रेकत्रिꣳ॒॒शः । ए॒क॒त्रिꣳ॒॒श इति॑ । ए॒क॒त्रिꣳ॒॒श इत्ये॑क - त्रिꣳ॒॒शः । इति॑ पु॒रस्ता᳚त् । पु॒रस्ता॒दुप॑ । उप॑ दधाति । द॒धा॒ति॒ वाक् । वाग् वै । वै क्रतुः॑ । क्रतु॑र् यज्ञ्मु॒खम् । य॒ज्ञ्॒मु॒खम् ॅवाक् । य॒ज्ञ्॒मु॒खमिति॑ यज्ञ् - मु॒खम् । वाग् य॑ज्ञ्मु॒खम् । य॒ज्ञ्॒मु॒खमे॒व । य॒ज्ञ्॒मु॒खमिति॑ यज्ञ् - मु॒खम् । ए॒व पु॒रस्ता᳚त् । पु॒रस्ता॒द् वि । वि या॑तयति । या॒त॒य॒ति॒ ब्र॒द्ध्नस्य॑ । ब्र॒द्ध्नस्य॑ वि॒ष्टप᳚म् । वि॒ष्टप॑म् चतुस्त्रिꣳ॒॒शः । च॒तु॒स्त्रिꣳ॒॒श इति॑ । च॒तु॒स्त्रिꣳ॒॒श इति॑ चतुः - त्रिꣳ॒॒शः । इति॑ दक्षिण॒तः । द॒क्षि॒ण॒तो॑ऽसौ । अ॒सौ वै । वा आ॑दि॒त्यः । आ॒दि॒त्यो ब्र॒द्ध्नस्य॑ । ब्र॒द्ध्नस्य॑ वि॒ष्टप᳚म् । वि॒ष्टप॑म् ब्रह्मवर्च॒सम् । ब्र॒ह्म॒व॒र्च॒समे॒व । ब्र॒ह्म॒व॒र्च॒समिति॑ ब्रह्म - व॒र्च॒सम् । ए॒व द॑क्षिण॒तः । द॒क्षि॒ण॒तो ध॑त्ते । ध॒त्ते॒ तस्मा᳚त् । तस्मा॒द् दक्षि॑णः । दक्षि॒णोऽर्द्धः॑ । अर्द्धो᳚ ब्रह्मवर्च॒सित॑रः । ब्र॒ह्म॒व॒र्च॒सित॑रः प्रति॒ष्ठा । ब्र॒ह्म॒व॒र्च॒सित॑र॒ इति॑ ब्रह्मवर्च॒सि - त॒रः॒ । प्र॒ति॒ष्ठा त्र॑यस्त्रिꣳ॒॒शः । प्र॒ति॒ष्ठेति॑ प्रति - स्था । त्र॒य॒स्त्रिꣳ॒॒श इति॑ । त्र॒य॒स्त्रिꣳ॒॒श इति॑ त्रयः - त्रिꣳ॒॒शः । इति॑ प॒श्चात् । प॒श्चात् प्रति॑ष्ठित्यै । प्रति॑ष्ठित्यै॒ नाकः॑ । प्रति॑ष्ठित्या॒ इति॒ प्रति॑ - स्थि॒त्यै॒ । नाक॑ष्षट्त्रिꣳ॒॒शः । ष॒ट्त्रिꣳ॒॒श इति॑ । ष॒ट्त्रिꣳ॒॒श इति॑ षट् - त्रिꣳ॒॒शः । इत्यु॑त्तर॒तः । उ॒त्त॒र॒तः सु॑व॒र्गः । उ॒त्त॒र॒त इत्यु॑त् - त॒र॒तः । सु॒व॒र्गो वै ( ) । सु॒व॒र्ग इति॑ सुवः - गः । वै लो॒कः । लो॒को नाकः॑ । नाकः॑ सुव॒र्गस्य॑ । सु॒व॒र्गस्य॑ लो॒कस्य॑ । सु॒व॒र्गस्येति॑ सुवः - गस्य॑ । लो॒कस्य॒ सम॑ष्ट्यै । सम॑ष्ट्या॒ इति॒ सम् - अ॒ष्ट्यै॒ । \newline

\textbf{Jatai Paata} \newline

1. उ॒त्त॒र॒त स्तस्मा॒त् तस्मा॑ दुत्तर॒त उ॑त्तर॒त स्तस्मा᳚त् । \newline
2. उ॒त्त॒र॒त इत्यु॑त् - त॒र॒तः । \newline
3. तस्मा᳚थ् स॒व्यः स॒व्य स्तस्मा॒त् तस्मा᳚थ् स॒व्यः । \newline
4. स॒व्यो हस्त॑यो॒र्॒. हस्त॑योः स॒व्यः स॒व्यो हस्त॑योः । \newline
5. हस्त॑योः संभा॒र्य॑तरः संभा॒र्य॑तरो॒ हस्त॑यो॒र्॒. हस्त॑योः संभा॒र्य॑तरः । \newline
6. सं॒भा॒र्य॑तरः॒ क्रतुः॒ क्रतुः॑ संभा॒र्य॑तरः संभा॒र्य॑तरः॒ क्रतुः॑ । \newline
7. सं॒भा॒र्य॑तर॒ इति॑ संभा॒र्य॑ - त॒रः॒ । \newline
8. क्रतु॑ रेकत्रिꣳ॒॒श ए॑कत्रिꣳ॒॒शः क्रतुः॒ क्रतु॑ रेकत्रिꣳ॒॒शः । \newline
9. ए॒क॒त्रिꣳ॒॒श इती त्ये॑कत्रिꣳ॒॒श ए॑कत्रिꣳ॒॒श इति॑ । \newline
10. ए॒क॒त्रिꣳ॒॒श इत्ये॑क - त्रिꣳ॒॒शः । \newline
11. इति॑ पु॒रस्ता᳚त् पु॒रस्ता॒ दितीति॑ पु॒रस्ता᳚त् । \newline
12. पु॒रस्ता॒ दुपोप॑ पु॒रस्ता᳚त् पु॒रस्ता॒ दुप॑ । \newline
13. उप॑ दधाति दधा॒ त्युपोप॑ दधाति । \newline
14. द॒धा॒ति॒ वाग् वाग् द॑धाति दधाति॒ वाक् । \newline
15. वाग् वै वै वाग् वाग् वै । \newline
16. वै क्रतुः॒ क्रतु॒र् वै वै क्रतुः॑ । \newline
17. क्रतु॑र् यज्ञ्मु॒खं ॅय॑ज्ञ्मु॒खम् क्रतुः॒ क्रतु॑र् यज्ञ्मु॒खम् । \newline
18. य॒ज्ञ्॒मु॒खं ॅवाग् वाग् य॑ज्ञ्मु॒खं ॅय॑ज्ञ्मु॒खं ॅवाक् । \newline
19. य॒ज्ञ्॒मु॒खमिति॑ यज्ञ् - मु॒खम् । \newline
20. वाग् य॑ज्ञ्मु॒खं ॅय॑ज्ञ्मु॒खं ॅवाग् वाग् य॑ज्ञ्मु॒खम् । \newline
21. य॒ज्ञ्॒मु॒ख मे॒वैव य॑ज्ञ्मु॒खं ॅय॑ज्ञ्मु॒ख मे॒व । \newline
22. य॒ज्ञ्॒मु॒खमिति॑ यज्ञ् - मु॒खम् । \newline
23. ए॒व पु॒रस्ता᳚त् पु॒रस्ता॑ दे॒वैव पु॒रस्ता᳚त् । \newline
24. पु॒रस्ता॒द् वि वि पु॒रस्ता᳚त् पु॒रस्ता॒द् वि । \newline
25. वि या॑तयति यातयति॒ वि वि या॑तयति । \newline
26. या॒त॒य॒ति॒ ब्र॒द्ध्नस्य॑ ब्र॒द्ध्नस्य॑ यातयति यातयति ब्र॒द्ध्नस्य॑ । \newline
27. ब्र॒द्ध्नस्य॑ वि॒ष्टपं॑ ॅवि॒ष्टप॑म् ब्र॒द्ध्नस्य॑ ब्र॒द्ध्नस्य॑ वि॒ष्टप᳚म् । \newline
28. वि॒ष्टप॑म् चतुस्त्रिꣳ॒॒श श्च॑तुस्त्रिꣳ॒॒शो वि॒ष्टपं॑ ॅवि॒ष्टप॑म् चतुस्त्रिꣳ॒॒शः । \newline
29. च॒तु॒स्त्रिꣳ॒॒श इतीति॑ चतुस्त्रिꣳ॒॒श श्च॑तुस्त्रिꣳ॒॒श इति॑ । \newline
30. च॒तु॒स्त्रिꣳ॒॒श इति॑ चतुः - त्रिꣳ॒॒शः । \newline
31. इति॑ दक्षिण॒तो द॑क्षिण॒त इतीति॑ दक्षिण॒तः । \newline
32. द॒क्षि॒ण॒तो॑ ऽसा व॒सौ द॑क्षिण॒तो द॑क्षिण॒तो॑ ऽसौ । \newline
33. अ॒सौ वै वा अ॒सा व॒सौ वै । \newline
34. वा आ॑दि॒त्य आ॑दि॒त्यो वै वा आ॑दि॒त्यः । \newline
35. आ॒दि॒त्यो ब्र॒द्ध्नस्य॑ ब्र॒द्ध्न स्या॑दि॒त्य आ॑दि॒त्यो ब्र॒द्ध्नस्य॑ । \newline
36. ब्र॒द्ध्नस्य॑ वि॒ष्टपं॑ ॅवि॒ष्टप॑म् ब्र॒द्ध्नस्य॑ ब्र॒द्ध्नस्य॑ वि॒ष्टप᳚म् । \newline
37. वि॒ष्टप॑म् ब्रह्मवर्च॒सम् ब्र॑ह्मवर्च॒सं ॅवि॒ष्टपं॑ ॅवि॒ष्टप॑म् ब्रह्मवर्च॒सम् । \newline
38. ब्र॒ह्म॒व॒र्च॒स मे॒वैव ब्र॑ह्मवर्च॒सम् ब्र॑ह्मवर्च॒स मे॒व । \newline
39. ब्र॒ह्म॒व॒र्च॒समिति॑ ब्रह्म - व॒र्च॒सम् । \newline
40. ए॒व द॑क्षिण॒तो द॑क्षिण॒त ए॒वैव द॑क्षिण॒तः । \newline
41. द॒क्षि॒ण॒तो ध॑त्ते धत्ते दक्षिण॒तो द॑क्षिण॒तो ध॑त्ते । \newline
42. ध॒त्ते॒ तस्मा॒त् तस्मा᳚द् धत्ते धत्ते॒ तस्मा᳚त् । \newline
43. तस्मा॒द् दक्षि॑णो॒ दक्षि॑ण॒ स्तस्मा॒त् तस्मा॒द् दक्षि॑णः । \newline
44. दक्षि॒णो ऽर्द्धो ऽर्द्धो॒ दक्षि॑णो॒ दक्षि॒णो ऽर्द्धः॑ । \newline
45. अर्द्धो᳚ ब्रह्मवर्च॒सित॑रो ब्रह्मवर्च॒सित॒रो ऽर्द्धो ऽर्द्धो᳚ ब्रह्मवर्च॒सित॑रः । \newline
46. ब्र॒ह्म॒व॒र्च॒सित॑रः प्रति॒ष्ठा प्र॑ति॒ष्ठा ब्र॑ह्मवर्च॒सित॑रो ब्रह्मवर्च॒सित॑रः प्रति॒ष्ठा । \newline
47. ब्र॒ह्म॒व॒र्च॒सित॑र॒ इति॑ ब्रह्मवर्च॒सि - त॒रः॒ । \newline
48. प्र॒ति॒ष्ठा त्र॑यस्त्रिꣳ॒॒श स्त्र॑यस्त्रिꣳ॒॒शः प्र॑ति॒ष्ठा प्र॑ति॒ष्ठा त्र॑यस्त्रिꣳ॒॒शः । \newline
49. प्र॒ति॒ष्ठेति॑ प्रति - स्था । \newline
50. त्र॒य॒स्त्रिꣳ॒॒श इतीति॑ त्रयस्त्रिꣳ॒॒श स्त्र॑यस्त्रिꣳ॒॒श इति॑ । \newline
51. त्र॒य॒स्त्रिꣳ॒॒श इति॑ त्रयः - त्रिꣳ॒॒शः । \newline
52. इति॑ प॒श्चात् प॒श्चा दितीति॑ प॒श्चात् । \newline
53. प॒श्चात् प्रति॑ष्ठित्यै॒ प्रति॑ष्ठित्यै प॒श्चात् प॒श्चात् प्रति॑ष्ठित्यै । \newline
54. प्रति॑ष्ठित्यै॒ नाको॒ नाकः॒ प्रति॑ष्ठित्यै॒ प्रति॑ष्ठित्यै॒ नाकः॑ । \newline
55. प्रति॑ष्ठित्या॒ इति॒ प्रति॑ - स्थि॒त्यै॒ । \newline
56. नाक॑ ष्षट्त्रिꣳ॒॒श ष्ष॑ट्त्रिꣳ॒॒शो नाको॒ नाक॑ ष्षट्त्रिꣳ॒॒शः । \newline
57. ष॒ट्त्रिꣳ॒॒श इतीति॑ षट्त्रिꣳ॒॒श ष्ष॑ट्त्रिꣳ॒॒श इति॑ । \newline
58. ष॒ट्त्रिꣳ॒॒श इति॑ षट् - त्रिꣳ॒॒शः । \newline
59. इत्यु॑त्तर॒त उ॑त्तर॒त इतीत्यु॑ त्तर॒तः । \newline
60. उ॒त्त॒र॒तः सु॑व॒र्गः सु॑व॒र्ग उ॑त्तर॒त उ॑त्तर॒तः सु॑व॒र्गः । \newline
61. उ॒त्त॒र॒त इत्यु॑त् - त॒र॒तः । \newline
62. सु॒व॒र्गो वै वै सु॑व॒र्गः सु॑व॒र्गो वै । \newline
63. सु॒व॒र्ग इति॑ सुवः - गः । \newline
64. वै लो॒को लो॒को वै वै लो॒कः । \newline
65. लो॒को नाको॒ नाको॑ लो॒को लो॒को नाकः॑ । \newline
66. नाकः॑ सुव॒र्गस्य॑ सुव॒र्गस्य॒ नाको॒ नाकः॑ सुव॒र्गस्य॑ । \newline
67. सु॒व॒र्गस्य॑ लो॒कस्य॑ लो॒कस्य॑ सुव॒र्गस्य॑ सुव॒र्गस्य॑ लो॒कस्य॑ । \newline
68. सु॒व॒र्गस्येति॑ सुवः - गस्य॑ । \newline
69. लो॒कस्य॒ सम॑ष्ट्यै॒ सम॑ष्ट्यै लो॒कस्य॑ लो॒कस्य॒ सम॑ष्ट्यै । \newline
70. सम॑ष्ट्या॒ इति॒ सं - अ॒ष्ट्यै॒ । \newline

\textbf{Ghana Paata } \newline

1. उ॒त्त॒र॒त स्तस्मा॒त् तस्मा॑ दुत्तर॒त उ॑त्तर॒त स्तस्मा᳚थ् स॒व्यः स॒व्य स्तस्मा॑ दुत्तर॒त उ॑त्तर॒त स्तस्मा᳚थ् स॒व्यः । \newline
2. उ॒त्त॒र॒त इत्यु॑त् - त॒र॒तः । \newline
3. तस्मा᳚थ् स॒व्यः स॒व्य स्तस्मा॒त् तस्मा᳚थ् स॒व्यो हस्त॑यो॒र्॒. हस्त॑योः स॒व्य स्तस्मा॒त् तस्मा᳚थ् स॒व्यो हस्त॑योः । \newline
4. स॒व्यो हस्त॑यो॒र्॒. हस्त॑योः स॒व्यः स॒व्यो हस्त॑योः संभा॒र्य॑तरः संभा॒र्य॑तरो॒ हस्त॑योः स॒व्यः स॒व्यो हस्त॑योः संभा॒र्य॑तरः । \newline
5. हस्त॑योः संभा॒र्य॑तरः संभा॒र्य॑तरो॒ हस्त॑यो॒र्॒. हस्त॑योः संभा॒र्य॑तरः॒ क्रतुः॒ क्रतुः॑ संभा॒र्य॑तरो॒ हस्त॑यो॒र्॒. हस्त॑योः संभा॒र्य॑तरः॒ क्रतुः॑ । \newline
6. सं॒भा॒र्य॑तरः॒ क्रतुः॒ क्रतुः॑ संभा॒र्य॑तरः संभा॒र्य॑तरः॒ क्रतु॑ रेकत्रिꣳ॒॒श ए॑कत्रिꣳ॒॒शः क्रतुः॑ संभा॒र्य॑तरः संभा॒र्य॑तरः॒ क्रतु॑ रेकत्रिꣳ॒॒शः । \newline
7. सं॒भा॒र्य॑तर॒ इति॑ संभा॒र्य॑ - त॒रः॒ । \newline
8. क्रतु॑ रेकत्रिꣳ॒॒श ए॑कत्रिꣳ॒॒शः क्रतुः॒ क्रतु॑ रेकत्रिꣳ॒॒श इती त्ये॑कत्रिꣳ॒॒शः क्रतुः॒ क्रतु॑ रेकत्रिꣳ॒॒श इति॑ । \newline
9. ए॒क॒त्रिꣳ॒॒श इती त्ये॑कत्रिꣳ॒॒श ए॑कत्रिꣳ॒॒श इति॑ पु॒रस्ता᳚त् पु॒रस्ता॒दि त्ये॑कत्रिꣳ॒॒श ए॑कत्रिꣳ॒॒श इति॑ पु॒रस्ता᳚त् । \newline
10. ए॒क॒त्रिꣳ॒॒श इत्ये॑क - त्रिꣳ॒॒शः । \newline
11. इति॑ पु॒रस्ता᳚त् पु॒रस्ता॒ दितीति॑ पु॒रस्ता॒ दुपोप॑ पु॒रस्ता॒ दितीति॑ पु॒रस्ता॒ दुप॑ । \newline
12. पु॒रस्ता॒ दुपोप॑ पु॒रस्ता᳚त् पु॒रस्ता॒ दुप॑ दधाति दधा॒ त्युप॑ पु॒रस्ता᳚त् पु॒रस्ता॒ दुप॑ दधाति । \newline
13. उप॑ दधाति दधा॒ त्युपोप॑ दधाति॒ वाग् वाग् द॑धा॒ त्युपोप॑ दधाति॒ वाक् । \newline
14. द॒धा॒ति॒ वाग् वाग् द॑धाति दधाति॒ वाग् वै वै वाग् द॑धाति दधाति॒ वाग् वै । \newline
15. वाग् वै वै वाग् वाग् वै क्रतुः॒ क्रतु॒र् वै वाग् वाग् वै क्रतुः॑ । \newline
16. वै क्रतुः॒ क्रतु॒र् वै वै क्रतु॑र् यज्ञ्मु॒खं ॅय॑ज्ञ्मु॒खम् क्रतु॒र् वै वै क्रतु॑र् यज्ञ्मु॒खम् । \newline
17. क्रतु॑र् यज्ञ्मु॒खं ॅय॑ज्ञ्मु॒खम् क्रतुः॒ क्रतु॑र् यज्ञ्मु॒खं ॅवाग् वाग् य॑ज्ञ्मु॒खम् क्रतुः॒ क्रतु॑र् यज्ञ्मु॒खं ॅवाक् । \newline
18. य॒ज्ञ्॒मु॒खं ॅवाग् वाग् य॑ज्ञ्मु॒खं ॅय॑ज्ञ्मु॒खं ॅवाग् य॑ज्ञ्मु॒खं ॅय॑ज्ञ्मु॒खं ॅवाग् य॑ज्ञ्मु॒खं ॅय॑ज्ञ्मु॒खं ॅवाग् य॑ज्ञ्मु॒खम् । \newline
19. य॒ज्ञ्॒मु॒खमिति॑ यज्ञ् - मु॒खम् । \newline
20. वाग् य॑ज्ञ्मु॒खं ॅय॑ज्ञ्मु॒खं ॅवाग् वाग् य॑ज्ञ्मु॒ख मे॒वैव य॑ज्ञ्मु॒खं ॅवाग् वाग् य॑ज्ञ्मु॒ख मे॒व । \newline
21. य॒ज्ञ्॒मु॒ख मे॒वैव य॑ज्ञ्मु॒खं ॅय॑ज्ञ्मु॒ख मे॒व पु॒रस्ता᳚त् पु॒रस्ता॑ दे॒व य॑ज्ञ्मु॒खं ॅय॑ज्ञ्मु॒ख मे॒व पु॒रस्ता᳚त् । \newline
22. य॒ज्ञ्॒मु॒खमिति॑ यज्ञ् - मु॒खम् । \newline
23. ए॒व पु॒रस्ता᳚त् पु॒रस्ता॑ दे॒वैव पु॒रस्ता॒द् वि वि पु॒रस्ता॑ दे॒वैव पु॒रस्ता॒द् वि । \newline
24. पु॒रस्ता॒द् वि वि पु॒रस्ता᳚त् पु॒रस्ता॒द् वि या॑तयति यातयति॒ वि पु॒रस्ता᳚त् पु॒रस्ता॒द् वि या॑तयति । \newline
25. वि या॑तयति यातयति॒ वि वि या॑तयति ब्र॒द्ध्नस्य॑ ब्र॒द्ध्नस्य॑ यातयति॒ वि वि या॑तयति ब्र॒द्ध्नस्य॑ । \newline
26. या॒त॒य॒ति॒ ब्र॒द्ध्नस्य॑ ब्र॒द्ध्नस्य॑ यातयति यातयति ब्र॒द्ध्नस्य॑ वि॒ष्टपं॑ ॅवि॒ष्टप॑म् ब्र॒द्ध्नस्य॑ यातयति यातयति ब्र॒द्ध्नस्य॑ वि॒ष्टप᳚म् । \newline
27. ब्र॒द्ध्नस्य॑ वि॒ष्टपं॑ ॅवि॒ष्टप॑म् ब्र॒द्ध्नस्य॑ ब्र॒द्ध्नस्य॑ वि॒ष्टप॑म् चतुस्त्रिꣳ॒॒श श्च॑तुस्त्रिꣳ॒॒शो वि॒ष्टप॑म् ब्र॒द्ध्नस्य॑ ब्र॒द्ध्नस्य॑ वि॒ष्टप॑म् चतुस्त्रिꣳ॒॒शः । \newline
28. वि॒ष्टप॑म् चतुस्त्रिꣳ॒॒श श्च॑तुस्त्रिꣳ॒॒शो वि॒ष्टपं॑ ॅवि॒ष्टप॑म् चतुस्त्रिꣳ॒॒श इतीति॑ चतुस्त्रिꣳ॒॒शो वि॒ष्टपं॑ ॅवि॒ष्टप॑म् चतुस्त्रिꣳ॒॒श इति॑ । \newline
29. च॒तु॒स्त्रिꣳ॒॒श इतीति॑ चतुस्त्रिꣳ॒॒श श्च॑तुस्त्रिꣳ॒॒श इति॑ दक्षिण॒तो द॑क्षिण॒त इति॑ चतुस्त्रिꣳ॒॒श श्च॑तुस्त्रिꣳ॒॒श इति॑ दक्षिण॒तः । \newline
30. च॒तु॒स्त्रिꣳ॒॒श इति॑ चतुः - त्रिꣳ॒॒शः । \newline
31. इति॑ दक्षिण॒तो द॑क्षिण॒त इतीति॑ दक्षिण॒तो॑ ऽसा व॒सौ द॑क्षिण॒त इतीति॑ दक्षिण॒तो॑ ऽसौ । \newline
32. द॒क्षि॒ण॒तो॑ ऽसा व॒सौ द॑क्षिण॒तो द॑क्षिण॒तो॑ ऽसौ वै वा अ॒सौ द॑क्षिण॒तो द॑क्षिण॒तो॑ ऽसौ वै । \newline
33. अ॒सौ वै वा अ॒सा व॒सौ वा आ॑दि॒त्य आ॑दि॒त्यो वा अ॒सा व॒सौ वा आ॑दि॒त्यः । \newline
34. वा आ॑दि॒त्य आ॑दि॒त्यो वै वा आ॑दि॒त्यो ब्र॒द्ध्नस्य॑ ब्र॒द्ध्न स्या॑दि॒त्यो वै वा आ॑दि॒त्यो ब्र॒द्ध्नस्य॑ । \newline
35. आ॒दि॒त्यो ब्र॒द्ध्नस्य॑ ब्र॒द्ध्न स्या॑दि॒त्य आ॑दि॒त्यो ब्र॒द्ध्नस्य॑ वि॒ष्टपं॑ ॅवि॒ष्टप॑म् ब्र॒द्ध्न स्या॑दि॒त्य आ॑दि॒त्यो ब्र॒द्ध्नस्य॑ वि॒ष्टप᳚म् । \newline
36. ब्र॒द्ध्नस्य॑ वि॒ष्टपं॑ ॅवि॒ष्टप॑म् ब्र॒द्ध्नस्य॑ ब्र॒द्ध्नस्य॑ वि॒ष्टप॑म् ब्रह्मवर्च॒सम् ब्र॑ह्मवर्च॒सं ॅवि॒ष्टप॑म् ब्र॒द्ध्नस्य॑ ब्र॒द्ध्नस्य॑ वि॒ष्टप॑म् ब्रह्मवर्च॒सम् । \newline
37. वि॒ष्टप॑म् ब्रह्मवर्च॒सम् ब्र॑ह्मवर्च॒सं ॅवि॒ष्टपं॑ ॅवि॒ष्टप॑म् ब्रह्मवर्च॒स मे॒वैव ब्र॑ह्मवर्च॒सं ॅवि॒ष्टपं॑ ॅवि॒ष्टप॑म् ब्रह्मवर्च॒स मे॒व । \newline
38. ब्र॒ह्म॒व॒र्च॒स मे॒वैव ब्र॑ह्मवर्च॒सम् ब्र॑ह्मवर्च॒स मे॒व द॑क्षिण॒तो द॑क्षिण॒त ए॒व ब्र॑ह्मवर्च॒सम् ब्र॑ह्मवर्च॒स मे॒व द॑क्षिण॒तः । \newline
39. ब्र॒ह्म॒व॒र्च॒समिति॑ ब्रह्म - व॒र्च॒सम् । \newline
40. ए॒व द॑क्षिण॒तो द॑क्षिण॒त ए॒वैव द॑क्षिण॒तो ध॑त्ते धत्ते दक्षिण॒त ए॒वैव द॑क्षिण॒तो ध॑त्ते । \newline
41. द॒क्षि॒ण॒तो ध॑त्ते धत्ते दक्षिण॒तो द॑क्षिण॒तो ध॑त्ते॒ तस्मा॒त् तस्मा᳚द् धत्ते दक्षिण॒तो द॑क्षिण॒तो ध॑त्ते॒ तस्मा᳚त् । \newline
42. ध॒त्ते॒ तस्मा॒त् तस्मा᳚द् धत्ते धत्ते॒ तस्मा॒द् दक्षि॑णो॒ दक्षि॑ण॒ स्तस्मा᳚द् धत्ते धत्ते॒ तस्मा॒द् दक्षि॑णः । \newline
43. तस्मा॒द् दक्षि॑णो॒ दक्षि॑ण॒ स्तस्मा॒त् तस्मा॒द् दक्षि॒णो ऽर्द्धो ऽर्द्धो॒ दक्षि॑ण॒ स्तस्मा॒त् तस्मा॒द् दक्षि॒णो ऽर्द्धः॑ । \newline
44. दक्षि॒णो ऽर्द्धो ऽर्द्धो॒ दक्षि॑णो॒ दक्षि॒णो ऽर्द्धो᳚ ब्रह्मवर्च॒सित॑रो ब्रह्मवर्च॒सित॒रो ऽर्द्धो॒ दक्षि॑णो॒ दक्षि॒णो ऽर्द्धो᳚ ब्रह्मवर्च॒सित॑रः । \newline
45. अर्द्धो᳚ ब्रह्मवर्च॒सित॑रो ब्रह्मवर्च॒सित॒रो ऽर्द्धो ऽर्द्धो᳚ ब्रह्मवर्च॒सित॑रः प्रति॒ष्ठा प्र॑ति॒ष्ठा ब्र॑ह्मवर्च॒सित॒रो ऽर्द्धो ऽर्द्धो᳚ ब्रह्मवर्च॒सित॑रः प्रति॒ष्ठा । \newline
46. ब्र॒ह्म॒व॒र्च॒सित॑रः प्रति॒ष्ठा प्र॑ति॒ष्ठा ब्र॑ह्मवर्च॒सित॑रो ब्रह्मवर्च॒सित॑रः प्रति॒ष्ठा त्र॑यस्त्रिꣳ॒॒श स्त्र॑यस्त्रिꣳ॒॒शः प्र॑ति॒ष्ठा ब्र॑ह्मवर्च॒सित॑रो ब्रह्मवर्च॒सित॑रः प्रति॒ष्ठा त्र॑यस्त्रिꣳ॒॒शः । \newline
47. ब्र॒ह्म॒व॒र्च॒सित॑र॒ इति॑ ब्रह्मवर्च॒सि - त॒रः॒ । \newline
48. प्र॒ति॒ष्ठा त्र॑यस्त्रिꣳ॒॒श स्त्र॑यस्त्रिꣳ॒॒शः प्र॑ति॒ष्ठा प्र॑ति॒ष्ठा त्र॑यस्त्रिꣳ॒॒श इतीति॑ त्रयस्त्रिꣳ॒॒शः प्र॑ति॒ष्ठा प्र॑ति॒ष्ठा त्र॑यस्त्रिꣳ॒॒श इति॑ । \newline
49. प्र॒ति॒ष्ठेति॑ प्रति - स्था । \newline
50. त्र॒य॒स्त्रिꣳ॒॒श इतीति॑ त्रयस्त्रिꣳ॒॒श स्त्र॑यस्त्रिꣳ॒॒श इति॑ प॒श्चात् प॒श्चादिति॑ त्रयस्त्रिꣳ॒॒श स्त्र॑यस्त्रिꣳ॒॒श इति॑ प॒श्चात् । \newline
51. त्र॒य॒स्त्रिꣳ॒॒श इति॑ त्रयः - त्रिꣳ॒॒शः । \newline
52. इति॑ प॒श्चात् प॒श्चा दितीति॑ प॒श्चात् प्रति॑ष्ठित्यै॒ प्रति॑ष्ठित्यै प॒श्चा दितीति॑ प॒श्चात् प्रति॑ष्ठित्यै । \newline
53. प॒श्चात् प्रति॑ष्ठित्यै॒ प्रति॑ष्ठित्यै प॒श्चात् प॒श्चात् प्रति॑ष्ठित्यै॒ नाको॒ नाकः॒ प्रति॑ष्ठित्यै प॒श्चात् प॒श्चात् प्रति॑ष्ठित्यै॒ नाकः॑ । \newline
54. प्रति॑ष्ठित्यै॒ नाको॒ नाकः॒ प्रति॑ष्ठित्यै॒ प्रति॑ष्ठित्यै॒ नाक॑ ष्षट्त्रिꣳ॒॒श ष्ष॑ट्त्रिꣳ॒॒शो नाकः॒ प्रति॑ष्ठित्यै॒ प्रति॑ष्ठित्यै॒ नाक॑ ष्षट्त्रिꣳ॒॒शः । \newline
55. प्रति॑ष्ठित्या॒ इति॒ प्रति॑ - स्थि॒त्यै॒ । \newline
56. नाक॑ ष्षट्त्रिꣳ॒॒श ष्ष॑ट्त्रिꣳ॒॒शो नाको॒ नाक॑ ष्षट्त्रिꣳ॒॒श इतीति॑ षट्त्रिꣳ॒॒शो नाको॒ नाक॑ ष्षट्त्रिꣳ॒॒श इति॑ । \newline
57. ष॒ट्त्रिꣳ॒॒श इतीति॑ षट्त्रिꣳ॒॒श ष्ष॑ट्त्रिꣳ॒॒श इत्यु॑त्तर॒त उ॑त्तर॒त इति॑ षट्त्रिꣳ॒॒श ष्ष॑ट्त्रिꣳ॒॒श इत्यु॑त्तर॒तः । \newline
58. ष॒ट्त्रिꣳ॒॒श इति॑ षट् - त्रिꣳ॒॒शः । \newline
59. इत्यु॑त्तर॒त उ॑त्तर॒त इती त्यु॑त्तर॒तः सु॑व॒र्गः सु॑व॒र्ग उ॑त्तर॒त इती त्यु॑त्तर॒तः सु॑व॒र्गः । \newline
60. उ॒त्त॒र॒तः सु॑व॒र्गः सु॑व॒र्ग उ॑त्तर॒त उ॑त्तर॒तः सु॑व॒र्गो वै वै सु॑व॒र्ग उ॑त्तर॒त उ॑त्तर॒तः सु॑व॒र्गो वै । \newline
61. उ॒त्त॒र॒त इत्यु॑त् - त॒र॒तः । \newline
62. सु॒व॒र्गो वै वै सु॑व॒र्गः सु॑व॒र्गो वै लो॒को लो॒को वै सु॑व॒र्गः सु॑व॒र्गो वै लो॒कः । \newline
63. सु॒व॒र्ग इति॑ सुवः - गः । \newline
64. वै लो॒को लो॒को वै वै लो॒को नाको॒ नाको॑ लो॒को वै वै लो॒को नाकः॑ । \newline
65. लो॒को नाको॒ नाको॑ लो॒को लो॒को नाकः॑ सुव॒र्गस्य॑ सुव॒र्गस्य॒ नाको॑ लो॒को लो॒को नाकः॑ सुव॒र्गस्य॑ । \newline
66. नाकः॑ सुव॒र्गस्य॑ सुव॒र्गस्य॒ नाको॒ नाकः॑ सुव॒र्गस्य॑ लो॒कस्य॑ लो॒कस्य॑ सुव॒र्गस्य॒ नाको॒ नाकः॑ सुव॒र्गस्य॑ लो॒कस्य॑ । \newline
67. सु॒व॒र्गस्य॑ लो॒कस्य॑ लो॒कस्य॑ सुव॒र्गस्य॑ सुव॒र्गस्य॑ लो॒कस्य॒ सम॑ष्ट्यै॒ सम॑ष्ट्यै लो॒कस्य॑ सुव॒र्गस्य॑ सुव॒र्गस्य॑ लो॒कस्य॒ सम॑ष्ट्यै । \newline
68. सु॒व॒र्गस्येति॑ सुवः - गस्य॑ । \newline
69. लो॒कस्य॒ सम॑ष्ट्यै॒ सम॑ष्ट्यै लो॒कस्य॑ लो॒कस्य॒ सम॑ष्ट्यै । \newline
70. सम॑ष्ट्या॒ इति॒ सं - अ॒ष्ट्यै॒ । \newline
\pagebreak
\markright{ TS 5.3.4.1  \hfill https://www.vedavms.in \hfill}

\section{ TS 5.3.4.1 }

\textbf{TS 5.3.4.1 } \newline
\textbf{Samhita Paata} \newline

अ॒ग्नेर्भा॒गो॑ऽसीति॑ पु॒रस्ता॒दुप॑ दधाति यज्ञ्मु॒खं ॅवा अ॒ग्निर्य॑ज्ञ्मु॒खं दी॒क्षा य॑ज्ञ्मु॒खं ब्रह्म॑ यज्ञ्मु॒खं त्रि॒वृद्-य॑ज्ञ्मु॒खमे॒व पु॒रस्ता॒द्वि या॑तयति नृ॒चक्ष॑सां भा॒गो॑ऽसीति॑ दक्षिण॒तः शु॑श्रु॒वाꣳसो॒ वै नृ॒चक्ष॒सोऽन्नं॑ धा॒ता जा॒तायै॒वास्मा॒ अन्न॒मपि॑ दधाति॒ तस्मा᳚ज्जा॒तोऽन्न॑मत्ति ज॒नित्रꣳ॑ स्पृ॒तꣳ स॑प्तद॒शः स्तोम॒ इत्या॒हाऽन्नं॒ ॅवै ज॒नित्र॒ - [  ] \newline

\textbf{Pada Paata} \newline

अ॒ग्नेः । भा॒गः । अ॒सि॒ । इति॑ । पु॒रस्ता᳚त् । उपेति॑ । द॒धा॒ति॒ । य॒ज्ञ्॒मु॒खमिति॑ यज्ञ् - मु॒खम् । वै । अ॒ग्निः । य॒ज्ञ्॒मु॒खमिति॑ यज्ञ्-मु॒खम् । दी॒क्षा । य॒ज्ञ्॒मु॒खमिति॑ यज्ञ् - मु॒खम् । ब्रह्म॑ । य॒ज्ञ्॒मु॒खमिति॑ यज्ञ् - मु॒खम् । त्रि॒वृदिति॑ त्रि - वृत् । य॒ज्ञ्॒मु॒खमिति॑ यज्ञ्-मु॒खम् । ए॒व । पु॒रस्ता᳚त् । वीति॑ । या॒त॒य॒ति॒ । नृ॒चक्ष॑सा॒मिति॑ नृ - चक्ष॑साम् । भा॒गः । अ॒सि॒ । इति॑ । द॒क्षि॒ण॒तः । शु॒श्रु॒वाꣳसः॑ । वै । नृ॒चक्ष॑स॒ इति॑ नृ - चक्ष॑सः । अन्न᳚म् । धा॒ता । जा॒ताय॑ । ए॒व । अ॒स्मै॒ । अन्न᳚म् । अपीति॑ । द॒धा॒ति॒ । तस्मा᳚त् । जा॒तः । अन्न᳚म् । अ॒त्ति॒ । ज॒नित्र᳚म् । स्पृ॒तम् । स॒प्त॒द॒श इति॑ सप्त - द॒शः । स्तोमः॑ । इति॑ । आ॒ह॒ । अन्न᳚म् । वै । ज॒नित्र᳚म् ।  \newline


\textbf{Krama Paata} \newline

अ॒ग्नेर् भा॒गः । भा॒गो॑ऽसि । अ॒सीति॑ । इति॑ पु॒रस्ता᳚त् । पु॒रस्ता॒दुप॑ । उप॑ दधाति । द॒धा॒ति॒ य॒ज्ञ्॒मु॒खम् । य॒ज्ञ्॒मु॒खम् ॅवै । य॒ज्ञ्॒मु॒खमिति॑ यज्ञ् - मु॒खम् । वा अ॒ग्निः । अ॒ग्निर् य॑ज्ञ्मु॒खम् । य॒ज्ञ्॒मु॒खम् दी॒क्षा । य॒ज्ञ्॒मु॒खमिति॑ यज्ञ् - मु॒खम् । दी॒क्षा य॑ज्ञ्मु॒खम् । य॒ज्ञ्॒मु॒खम् ब्रह्म॑ । य॒ज्ञ्॒मु॒खमिति॑ यज्ञ् - मु॒खम् । ब्रह्म॑ यज्ञ्मु॒खम् । य॒ज्ञ्॒मु॒खम् त्रि॒वृत् । य॒ज्ञ्॒मु॒खमिति॑ यज्ञ् - मु॒खम् । त्रि॒वृद् य॑ज्ञ्मु॒खम् । त्रि॒वृदिति॑ त्रि - वृत् । य॒ज्ञ्॒मु॒खमे॒व । य॒ज्ञ्॒मु॒खमिति॑ यज्ञ् - मु॒खम् । ए॒व पु॒रस्ता᳚त् । पु॒रस्ता॒द् वि । वि या॑तयति । या॒त॒य॒ति॒ नृ॒चक्ष॑साम् । नृ॒चक्ष॑साम् भा॒गः । नृ॒चक्ष॑सा॒मिति॑ नृ - चक्ष॑साम् । भा॒गो॑ऽसि । अ॒सीति॑ । इति॑ दक्षिण॒तः । द॒क्षि॒ण॒तः शु॑श्रु॒वाꣳसः॑ । शु॒श्रु॒वाꣳसो॒ वै । वै नृ॒चक्ष॑सः । नृ॒चक्ष॒सोऽन्न᳚म् । नृ॒चक्ष॑स॒ इति॑ नृ - चक्ष॑सः । अन्न॑म् धा॒ता । धा॒ता जा॒ताय॑ । जा॒तायै॒व । ए॒वास्मै᳚ । अ॒स्मा॒ अन्न᳚म् । अन्न॒मपि॑ । अपि॑ दधाति । द॒धा॒ति॒ तस्मा᳚त् । तस्मा᳚ज् जा॒तः । जा॒तोऽन्न᳚म् । अन्न॑मत्ति । अ॒त्ति॒ ज॒नित्र᳚म् । ज॒नित्रꣳ॑ स्पृ॒तम् । स्पृ॒तꣳ स॑प्तद॒शः । स॒प्त॒द॒शः स्तोमः॑ । स॒प्त॒द॒श इति॑ सप्त - द॒शः । स्तोम॒ इति॑ । इत्या॑ह । आ॒हान्न᳚म् । अन्न॒म् ॅवै । वै ज॒नित्र᳚म् । ज॒नित्र॒मन्न᳚म् \newline

\textbf{Jatai Paata} \newline

1. अ॒ग्नेर् भा॒गो भा॒गो᳚ ऽग्ने र॒ग्नेर् भा॒गः । \newline
2. भा॒गो᳚ ऽस्यसि भा॒गो भा॒गो॑ ऽसि । \newline
3. अ॒सी तीत्य॑ स्य॒सीति॑ । \newline
4. इति॑ पु॒रस्ता᳚त् पु॒रस्ता॒ दितीति॑ पु॒रस्ता᳚त् । \newline
5. पु॒रस्ता॒ दुपोप॑ पु॒रस्ता᳚त् पु॒रस्ता॒ दुप॑ । \newline
6. उप॑ दधाति दधा॒ त्युपोप॑ दधाति । \newline
7. द॒धा॒ति॒ य॒ज्ञ्॒मु॒खं ॅय॑ज्ञ्मु॒खम् द॑धाति दधाति यज्ञ्मु॒खम् । \newline
8. य॒ज्ञ्॒मु॒खं ॅवै वै य॑ज्ञ्मु॒खं ॅय॑ज्ञ्मु॒खं ॅवै । \newline
9. य॒ज्ञ्॒मु॒खमिति॑ यज्ञ् - मु॒खम् । \newline
10. वा अ॒ग्नि र॒ग्निर् वै वा अ॒ग्निः । \newline
11. अ॒ग्निर् य॑ज्ञ्मु॒खं ॅय॑ज्ञ्मु॒ख म॒ग्नि र॒ग्निर् य॑ज्ञ्मु॒खम् । \newline
12. य॒ज्ञ्॒मु॒खम् दी॒क्षा दी॒क्षा य॑ज्ञ्मु॒खं ॅय॑ज्ञ्मु॒खम् दी॒क्षा । \newline
13. य॒ज्ञ्॒मु॒खमिति॑ यज्ञ् - मु॒खम् । \newline
14. दी॒क्षा य॑ज्ञ्मु॒खं ॅय॑ज्ञ्मु॒खम् दी॒क्षा दी॒क्षा य॑ज्ञ्मु॒खम् । \newline
15. य॒ज्ञ्॒मु॒खम् ब्रह्म॒ ब्रह्म॑ यज्ञ्मु॒खं ॅय॑ज्ञ्मु॒खम् ब्रह्म॑ । \newline
16. य॒ज्ञ्॒मु॒खमिति॑ यज्ञ् - मु॒खम् । \newline
17. ब्रह्म॑ यज्ञ्मु॒खं ॅय॑ज्ञ्मु॒खम् ब्रह्म॒ ब्रह्म॑ यज्ञ्मु॒खम् । \newline
18. य॒ज्ञ्॒मु॒खम् त्रि॒वृत् त्रि॒वृद् य॑ज्ञ्मु॒खं ॅय॑ज्ञ्मु॒खम् त्रि॒वृत् । \newline
19. य॒ज्ञ्॒मु॒खमिति॑ यज्ञ् - मु॒खम् । \newline
20. त्रि॒वृद् य॑ज्ञ्मु॒खं ॅय॑ज्ञ्मु॒खम् त्रि॒वृत् त्रि॒वृद् य॑ज्ञ्मु॒खम् । \newline
21. त्रि॒वृदिति॑ त्रि - वृत् । \newline
22. य॒ज्ञ्॒मु॒ख मे॒वैव य॑ज्ञ्मु॒खं ॅय॑ज्ञ्मु॒ख मे॒व । \newline
23. य॒ज्ञ्॒मु॒खमिति॑ यज्ञ् - मु॒खम् । \newline
24. ए॒व पु॒रस्ता᳚त् पु॒रस्ता॑ दे॒वैव पु॒रस्ता᳚त् । \newline
25. पु॒रस्ता॒द् वि वि पु॒रस्ता᳚त् पु॒रस्ता॒द् वि । \newline
26. वि या॑तयति यातयति॒ वि वि या॑तयति । \newline
27. या॒त॒य॒ति॒ नृ॒चक्ष॑साम् नृ॒चक्ष॑सां ॅयातयति यातयति नृ॒चक्ष॑साम् । \newline
28. नृ॒चक्ष॑साम् भा॒गो भा॒गो नृ॒चक्ष॑साम् नृ॒चक्ष॑साम् भा॒गः । \newline
29. नृ॒चक्ष॑सा॒मिति॑ नृ - चक्ष॑साम् । \newline
30. भा॒गो᳚ ऽस्यसि भा॒गो भा॒गो॑ ऽसि । \newline
31. अ॒सीती त्य॑स्य॒सीति॑ । \newline
32. इति॑ दक्षिण॒तो द॑क्षिण॒त इतीति॑ दक्षिण॒तः । \newline
33. द॒क्षि॒ण॒तः शु॑श्रु॒वाꣳसः॑ शुश्रु॒वाꣳसो॑ दक्षिण॒तो द॑क्षिण॒तः शु॑श्रु॒वाꣳसः॑ । \newline
34. शु॒श्रु॒वाꣳसो॒ वै वै शु॑श्रु॒वाꣳसः॑ शुश्रु॒वाꣳसो॒ वै । \newline
35. वै नृ॒चक्ष॑सो नृ॒चक्ष॑सो॒ वै वै नृ॒चक्ष॑सः । \newline
36. नृ॒चक्ष॒सो ऽन्न॒ मन्न॑न् नृ॒चक्ष॑सो नृ॒चक्ष॒सो ऽन्न᳚म् । \newline
37. नृ॒चक्ष॑स॒ इति॑ नृ - चक्ष॑सः । \newline
38. अन्न॑म् धा॒ता धा॒ता ऽन्न॒ मन्न॑म् धा॒ता । \newline
39. धा॒ता जा॒ताय॑ जा॒ताय॑ धा॒ता धा॒ता जा॒ताय॑ । \newline
40. जा॒ता यै॒वैव जा॒ताय॑ जा॒ता यै॒व । \newline
41. ए॒वास्मा॑ अस्मा ए॒वै वास्मै᳚ । \newline
42. अ॒स्मा॒ अन्न॒ मन्न॑ मस्मा अस्मा॒ अन्न᳚म् । \newline
43. अन्न॒ मप्य प्यन्न॒ मन्न॒ मपि॑ । \newline
44. अपि॑ दधाति दधा॒ त्यप्यपि॑ दधाति । \newline
45. द॒धा॒ति॒ तस्मा॒त् तस्मा᳚द् दधाति दधाति॒ तस्मा᳚त् । \newline
46. तस्मा᳚ज् जा॒तो जा॒त स्तस्मा॒त् तस्मा᳚ज् जा॒तः । \newline
47. जा॒तो ऽन्न॒ मन्न॑म् जा॒तो जा॒तो ऽन्न᳚म् । \newline
48. अन्न॑ मत्त्य॒ त्त्यन्न॒ मन्न॑ मत्ति । \newline
49. अ॒त्ति॒ ज॒नित्र॑म् ज॒नित्र॑ मत्त्यत्ति ज॒नित्र᳚म् । \newline
50. ज॒नित्रꣳ॑ स्पृ॒तꣳ स्पृ॒तम् ज॒नित्र॑म् ज॒नित्रꣳ॑ स्पृ॒तम् । \newline
51. स्पृ॒तꣳ स॑प्तद॒शः स॑प्तद॒शः स्पृ॒तꣳ स्पृ॒तꣳ स॑प्तद॒शः । \newline
52. स॒प्त॒द॒शः स्तोमः॒ स्तोमः॑ सप्तद॒शः स॑प्तद॒शः स्तोमः॑ । \newline
53. स॒प्त॒द॒श इति॑ सप्त - द॒शः । \newline
54. स्तोम॒ इतीति॒ स्तोमः॒ स्तोम॒ इति॑ । \newline
55. इत्या॑हा॒हे तीत्या॑ह । \newline
56. आ॒हान्न॒ मन्न॑ माहा॒ हान्न᳚म् । \newline
57. अन्नं॒ ॅवै वा अन्न॒ मन्नं॒ ॅवै । \newline
58. वै ज॒नित्र॑म् ज॒नित्रं॒ ॅवै वै ज॒नित्र᳚म् । \newline
59. ज॒नित्र॒ मन्न॒ मन्न॑म् ज॒नित्र॑म् ज॒नित्र॒ मन्न᳚म् । \newline

\textbf{Ghana Paata } \newline

1. अ॒ग्नेर् भा॒गो भा॒गो᳚ ऽग्ने र॒ग्नेर् भा॒गो᳚ ऽस्यसि भा॒गो᳚ ऽग्ने र॒ग्नेर् भा॒गो॑ ऽसि । \newline
2. भा॒गो᳚ ऽस्यसि भा॒गो भा॒गो॑ ऽसीती त्य॑सि भा॒गो भा॒गो॑ ऽसीति॑ । \newline
3. अ॒सीती त्य॑स्य॒ सीति॑ पु॒रस्ता᳚त् पु॒रस्ता॒ दित्य॑स्य॒ सीति॑ पु॒रस्ता᳚त् । \newline
4. इति॑ पु॒रस्ता᳚त् पु॒रस्ता॒ दितीति॑ पु॒रस्ता॒ दुपोप॑ पु॒रस्ता॒ दितीति॑ पु॒रस्ता॒ दुप॑ । \newline
5. पु॒रस्ता॒ दुपोप॑ पु॒रस्ता᳚त् पु॒रस्ता॒ दुप॑ दधाति दधा॒ त्युप॑ पु॒रस्ता᳚त् पु॒रस्ता॒ दुप॑ दधाति । \newline
6. उप॑ दधाति दधा॒ त्युपोप॑ दधाति यज्ञ्मु॒खं ॅय॑ज्ञ्मु॒खम् द॑धा॒ त्युपोप॑ दधाति यज्ञ्मु॒खम् । \newline
7. द॒धा॒ति॒ य॒ज्ञ्॒मु॒खं ॅय॑ज्ञ्मु॒खम् द॑धाति दधाति यज्ञ्मु॒खं ॅवै वै य॑ज्ञ्मु॒खम् द॑धाति दधाति यज्ञ्मु॒खं ॅवै । \newline
8. य॒ज्ञ्॒मु॒खं ॅवै वै य॑ज्ञ्मु॒खं ॅय॑ज्ञ्मु॒खं ॅवा अ॒ग्नि र॒ग्निर् वै य॑ज्ञ्मु॒खं ॅय॑ज्ञ्मु॒खं ॅवा अ॒ग्निः । \newline
9. य॒ज्ञ्॒मु॒खमिति॑ यज्ञ् - मु॒खम् । \newline
10. वा अ॒ग्नि र॒ग्निर् वै वा अ॒ग्निर् य॑ज्ञ्मु॒खं ॅय॑ज्ञ्मु॒ख म॒ग्निर् वै वा अ॒ग्निर् य॑ज्ञ्मु॒खम् । \newline
11. अ॒ग्निर् य॑ज्ञ्मु॒खं ॅय॑ज्ञ्मु॒ख म॒ग्नि र॒ग्निर् य॑ज्ञ्मु॒खम् दी॒क्षा दी॒क्षा य॑ज्ञ्मु॒ख म॒ग्नि र॒ग्निर् य॑ज्ञ्मु॒खम् दी॒क्षा । \newline
12. य॒ज्ञ्॒मु॒खम् दी॒क्षा दी॒क्षा य॑ज्ञ्मु॒खं ॅय॑ज्ञ्मु॒खम् दी॒क्षा य॑ज्ञ्मु॒खं ॅय॑ज्ञ्मु॒खम् दी॒क्षा य॑ज्ञ्मु॒खं ॅय॑ज्ञ्मु॒खम् दी॒क्षा य॑ज्ञ्मु॒खम् । \newline
13. य॒ज्ञ्॒मु॒खमिति॑ यज्ञ् - मु॒खम् । \newline
14. दी॒क्षा य॑ज्ञ्मु॒खं ॅय॑ज्ञ्मु॒खम् दी॒क्षा दी॒क्षा य॑ज्ञ्मु॒खम् ब्रह्म॒ ब्रह्म॑ यज्ञ्मु॒खम् दी॒क्षा दी॒क्षा य॑ज्ञ्मु॒खम् ब्रह्म॑ । \newline
15. य॒ज्ञ्॒मु॒खम् ब्रह्म॒ ब्रह्म॑ यज्ञ्मु॒खं ॅय॑ज्ञ्मु॒खम् ब्रह्म॑ यज्ञ्मु॒खं ॅय॑ज्ञ्मु॒खम् ब्रह्म॑ यज्ञ्मु॒खं ॅय॑ज्ञ्मु॒खम् ब्रह्म॑ यज्ञ्मु॒खम् । \newline
16. य॒ज्ञ्॒मु॒खमिति॑ यज्ञ् - मु॒खम् । \newline
17. ब्रह्म॑ यज्ञ्मु॒खं ॅय॑ज्ञ्मु॒खम् ब्रह्म॒ ब्रह्म॑ यज्ञ्मु॒खम् त्रि॒वृत् त्रि॒वृद् य॑ज्ञ्मु॒खम् ब्रह्म॒ ब्रह्म॑ यज्ञ्मु॒खम् त्रि॒वृत् । \newline
18. य॒ज्ञ्॒मु॒खम् त्रि॒वृत् त्रि॒वृद् य॑ज्ञ्मु॒खं ॅय॑ज्ञ्मु॒खम् त्रि॒वृद् य॑ज्ञ्मु॒खं ॅय॑ज्ञ्मु॒खम् त्रि॒वृद् य॑ज्ञ्मु॒खं ॅय॑ज्ञ्मु॒खम् त्रि॒वृद् य॑ज्ञ्मु॒खम् । \newline
19. य॒ज्ञ्॒मु॒खमिति॑ यज्ञ् - मु॒खम् । \newline
20. त्रि॒वृद् य॑ज्ञ्मु॒खं ॅय॑ज्ञ्मु॒खम् त्रि॒वृत् त्रि॒वृद् य॑ज्ञ्मु॒ख मे॒वैव य॑ज्ञ्मु॒खम् त्रि॒वृत् त्रि॒वृद् य॑ज्ञ्मु॒ख मे॒व । \newline
21. त्रि॒वृदिति॑ त्रि - वृत् । \newline
22. य॒ज्ञ्॒मु॒ख मे॒वैव य॑ज्ञ्मु॒खं ॅय॑ज्ञ्मु॒ख मे॒व पु॒रस्ता᳚त् पु॒रस्ता॑ दे॒व य॑ज्ञ्मु॒खं ॅय॑ज्ञ्मु॒ख मे॒व पु॒रस्ता᳚त् । \newline
23. य॒ज्ञ्॒मु॒खमिति॑ यज्ञ् - मु॒खम् । \newline
24. ए॒व पु॒रस्ता᳚त् पु॒रस्ता॑ दे॒वैव पु॒रस्ता॒द् वि वि पु॒रस्ता॑ दे॒वैव पु॒रस्ता॒द् वि । \newline
25. पु॒रस्ता॒द् वि वि पु॒रस्ता᳚त् पु॒रस्ता॒द् वि या॑तयति यातयति॒ वि पु॒रस्ता᳚त् पु॒रस्ता॒द् वि या॑तयति । \newline
26. वि या॑तयति यातयति॒ वि वि या॑तयति नृ॒चक्ष॑साम् नृ॒चक्ष॑सां ॅयातयति॒ वि वि या॑तयति नृ॒चक्ष॑साम् । \newline
27. या॒त॒य॒ति॒ नृ॒चक्ष॑साम् नृ॒चक्ष॑सां ॅयातयति यातयति नृ॒चक्ष॑साम् भा॒गो भा॒गो नृ॒चक्ष॑सां ॅयातयति यातयति नृ॒चक्ष॑साम् भा॒गः । \newline
28. नृ॒चक्ष॑साम् भा॒गो भा॒गो नृ॒चक्ष॑साम् नृ॒चक्ष॑साम् भा॒गो᳚ ऽस्यसि भा॒गो नृ॒चक्ष॑साम् नृ॒चक्ष॑साम् भा॒गो॑ ऽसि । \newline
29. नृ॒चक्ष॑सा॒मिति॑ नृ - चक्ष॑साम् । \newline
30. भा॒गो᳚ ऽस्यसि भा॒गो भा॒गो॑ ऽसीती त्य॑सि भा॒गो भा॒गो॑ ऽसीति॑ । \newline
31. अ॒सीती त्य॑स्य॒ सीति॑ दक्षिण॒तो द॑क्षिण॒त इत्य॑स्य॒ सीति॑ दक्षिण॒तः । \newline
32. इति॑ दक्षिण॒तो द॑क्षिण॒त इतीति॑ दक्षिण॒तः शु॑श्रु॒वाꣳसः॑ शुश्रु॒वाꣳसो॑ दक्षिण॒त इतीति॑ दक्षिण॒तः शु॑श्रु॒वाꣳसः॑ । \newline
33. द॒क्षि॒ण॒तः शु॑श्रु॒वाꣳसः॑ शुश्रु॒वाꣳसो॑ दक्षिण॒तो द॑क्षिण॒तः शु॑श्रु॒वाꣳसो॒ वै वै शु॑श्रु॒वाꣳसो॑ दक्षिण॒तो द॑क्षिण॒तः शु॑श्रु॒वाꣳसो॒ वै । \newline
34. शु॒श्रु॒वाꣳसो॒ वै वै शु॑श्रु॒वाꣳसः॑ शुश्रु॒वाꣳसो॒ वै नृ॒चक्ष॑सो नृ॒चक्ष॑सो॒ वै शु॑श्रु॒वाꣳसः॑ शुश्रु॒वाꣳसो॒ वै नृ॒चक्ष॑सः । \newline
35. वै नृ॒चक्ष॑सो नृ॒चक्ष॑सो॒ वै वै नृ॒चक्ष॒सो ऽन्न॒ मन्न॑म् नृ॒चक्ष॑सो॒ वै वै नृ॒चक्ष॒सो ऽन्न᳚म् । \newline
36. नृ॒चक्ष॒सो ऽन्न॒ मन्न॑न् नृ॒चक्ष॑सो नृ॒चक्ष॒सो ऽन्न॑म् धा॒ता धा॒ता ऽन्न॑म् नृ॒चक्ष॑सो नृ॒चक्ष॒सो ऽन्न॑म् धा॒ता । \newline
37. नृ॒चक्ष॑स॒ इति॑ नृ - चक्ष॑सः । \newline
38. अन्न॑म् धा॒ता धा॒ता ऽन्न॒ मन्न॑म् धा॒ता जा॒ताय॑ जा॒ताय॑ धा॒ता ऽन्न॒ मन्न॑म् धा॒ता जा॒ताय॑ । \newline
39. धा॒ता जा॒ताय॑ जा॒ताय॑ धा॒ता धा॒ता जा॒तायै॒वैव जा॒ताय॑ धा॒ता धा॒ता जा॒तायै॒व । \newline
40. जा॒तायै॒वैव जा॒ताय॑ जा॒ता यै॒वास्मा॑ अस्मा ए॒व जा॒ताय॑ जा॒तायै॒वास्मै᳚ । \newline
41. ए॒वास्मा॑ अस्मा ए॒वै वास्मा॒ अन्न॒ मन्न॑ मस्मा ए॒वै वास्मा॒ अन्न᳚म् । \newline
42. अ॒स्मा॒ अन्न॒ मन्न॑ मस्मा अस्मा॒ अन्न॒ मप्य प्यन्न॑ मस्मा अस्मा॒ अन्न॒ मपि॑ । \newline
43. अन्न॒ मप्य प्यन्न॒ मन्न॒ मपि॑ दधाति दधा॒ त्यप्यन्न॒ मन्न॒ मपि॑ दधाति । \newline
44. अपि॑ दधाति दधा॒ त्यप्यपि॑ दधाति॒ तस्मा॒त् तस्मा᳚द् दधा॒ त्यप्यपि॑ दधाति॒ तस्मा᳚त् । \newline
45. द॒धा॒ति॒ तस्मा॒त् तस्मा᳚द् दधाति दधाति॒ तस्मा᳚ज् जा॒तो जा॒त स्तस्मा᳚द् दधाति दधाति॒ तस्मा᳚ज् जा॒तः । \newline
46. तस्मा᳚ज् जा॒तो जा॒त स्तस्मा॒त् तस्मा᳚ज् जा॒तो ऽन्न॒ मन्न॑म् जा॒त स्तस्मा॒त् तस्मा᳚ज् जा॒तो ऽन्न᳚म् । \newline
47. जा॒तो ऽन्न॒ मन्न॑म् जा॒तो जा॒तो ऽन्न॑ मत्त्य॒ त्त्यन्न॑म् जा॒तो जा॒तो ऽन्न॑ मत्ति । \newline
48. अन्न॑ मत्त्य॒ त्त्यन्न॒ मन्न॑ मत्ति ज॒नित्र॑म् ज॒नित्र॑ म॒त्त्यन्न॒ मन्न॑ मत्ति ज॒नित्र᳚म् । \newline
49. अ॒त्ति॒ ज॒नित्र॑म् ज॒नित्र॑ मत्त्यत्ति ज॒नित्रꣳ॑ स्पृ॒तꣳ स्पृ॒तम् ज॒नित्र॑ मत्त्यत्ति ज॒नित्रꣳ॑ स्पृ॒तम् । \newline
50. ज॒नित्रꣳ॑ स्पृ॒तꣳ स्पृ॒तम् ज॒नित्र॑म् ज॒नित्रꣳ॑ स्पृ॒तꣳ स॑प्तद॒शः स॑प्तद॒शः स्पृ॒तम् ज॒नित्र॑म् ज॒नित्रꣳ॑ स्पृ॒तꣳ स॑प्तद॒शः । \newline
51. स्पृ॒तꣳ स॑प्तद॒शः स॑प्तद॒शः स्पृ॒तꣳ स्पृ॒तꣳ स॑प्तद॒शः स्तोमः॒ स्तोमः॑ सप्तद॒शः स्पृ॒तꣳ स्पृ॒तꣳ स॑प्तद॒शः स्तोमः॑ । \newline
52. स॒प्त॒द॒शः स्तोमः॒ स्तोमः॑ सप्तद॒शः स॑प्तद॒शः स्तोम॒ इतीति॒ स्तोमः॑ सप्तद॒शः स॑प्तद॒शः स्तोम॒ इति॑ । \newline
53. स॒प्त॒द॒श इति॑ सप्त - द॒शः । \newline
54. स्तोम॒ इतीति॒ स्तोमः॒ स्तोम॒ इत्या॑ हा॒हेति॒ स्तोमः॒ स्तोम॒ इत्या॑ह । \newline
55. इत्या॑ हा॒हेतीत्या॒ हान्न॒ मन्न॑ मा॒हेती त्या॒हान्न᳚म् । \newline
56. आ॒हान्न॒ मन्न॑ माहा॒हान्नं॒ ॅवै वा अन्न॑ माहा॒हान्नं॒ ॅवै । \newline
57. अन्नं॒ ॅवै वा अन्न॒ मन्नं॒ ॅवै ज॒नित्र॑म् ज॒नित्रं॒ ॅवा अन्न॒ मन्नं॒ ॅवै ज॒नित्र᳚म् । \newline
58. वै ज॒नित्र॑म् ज॒नित्रं॒ ॅवै वै ज॒नित्र॒ मन्न॒ मन्न॑म् ज॒नित्रं॒ ॅवै वै ज॒नित्र॒ मन्न᳚म् । \newline
59. ज॒नित्र॒ मन्न॒ मन्न॑म् ज॒नित्र॑म् ज॒नित्र॒ मन्नꣳ॑ सप्तद॒शः स॑प्तद॒शो ऽन्न॑म् ज॒नित्र॑म् ज॒नित्र॒ मन्नꣳ॑ सप्तद॒शः । \newline
\pagebreak
\markright{ TS 5.3.4.2  \hfill https://www.vedavms.in \hfill}

\section{ TS 5.3.4.2 }

\textbf{TS 5.3.4.2 } \newline
\textbf{Samhita Paata} \newline

-मन्नꣳ॑ सप्तद॒शो-ऽन्न॑मे॒व द॑क्षिण॒तो ध॑त्ते॒ तस्मा॒द् दक्षि॑णे॒ना-न्न॑मद्यते मि॒त्रस्य॑ भा॒गो॑ऽसीति॑ प॒श्चात् प्रा॒णो वै मि॒त्रो॑ऽपा॒नो वरु॑णः प्राणापा॒नावे॒वास्मि॑न् दधाति दि॒वो वृ॒ष्टिर्वाताः᳚ स्पृ॒ता ए॑कविꣳ॒॒शः स्तोम॒ इत्या॑ह प्रति॒ष्ठा वा ए॑कविꣳ॒॒शः प्रति॑ष्ठित्या॒ इन्द्र॑स्य भा॒गो॑ऽसीत्यु॑त्तर॒त ओजो॒ वा इन्द्र॒ ओजो॒ विष्णु॒रोजः॑ क्ष॒त्रमोजः॑ पञ्चद॒श - [  ] \newline

\textbf{Pada Paata} \newline

अन्न᳚म् । स॒प्त॒द॒श इति॑ सप्त - द॒शः । अन्न᳚म् । ए॒व । द॒क्षि॒ण॒तः । ध॒त्ते॒ । तस्मा᳚त् । दक्षि॑णेन । अन्न᳚म् । अ॒द्य॒ते॒ । मि॒त्रस्य॑ । भा॒गः । अ॒सि॒ । इति॑ । प॒श्चात् । प्रा॒ण इति॑ प्र - अ॒नः । वै । मि॒त्रः । अ॒पा॒न इत्य॑प - अ॒नः । वरु॑णः । प्रा॒णा॒पा॒नाविति॑ प्राण - अ॒पा॒नौ । ए॒व । अ॒स्मि॒न्न् । द॒धा॒ति॒ । दि॒वः । वृ॒ष्टिः । वाताः᳚ । स्पृ॒ताः । ए॒क॒विꣳ॒॒श इत्ये॑क - विꣳ॒॒शः । स्तोमः॑ । इति॑ । आ॒ह॒ । प्र॒ति॒ष्ठेति॑ प्रति - स्था । वै । ए॒क॒विꣳ॒॒श इत्ये॑क - विꣳ॒॒शः । प्रति॑ष्ठित्या॒ इति॒ प्रति॑-स्थि॒त्यै॒ । इन्द्र॑स्य । भा॒गः । अ॒सि॒ । इति॑ । उ॒त्त॒र॒त इत्यु॑त् - त॒र॒तः । ओजः॑ । वै । इन्द्रः॑ । ओजः॑ । विष्णुः॑ । ओजः॑ । क्ष॒त्रम् । ओजः॑ । प॒ञ्च॒द॒श इति॑ पञ्च - द॒शः ।  \newline


\textbf{Krama Paata} \newline

अन्नꣳ॑ सप्तद॒शः । स॒प्त॒द॒शोऽन्न᳚म् । स॒प्त॒द॒श इति॑ सप्त - द॒शः । अन्न॑मे॒व । ए॒व द॑क्षिण॒तः । द॒क्षि॒ण॒तो ध॑त्ते । ध॒त्ते॒ तस्मा᳚त् । तस्मा॒द् दक्षि॑णेन । दक्षि॑णे॒नान्न᳚म् । अन्न॑मद्यते । अ॒द्य॒ते॒ मि॒त्रस्य॑ । मि॒त्रस्य॑ भा॒गः । भा॒गो॑ऽसि । अ॒सीति॑ । इति॑ प॒श्चात् । प॒श्चात् प्रा॒णः । प्रा॒णो वै । प्रा॒ण इति॑ प्र - अ॒नः । वै मि॒त्रः । मि॒त्रो॑ऽपा॒नः । अ॒पा॒नो वरु॑णः । अ॒पा॒न इत्य॑प - अ॒नः । वरु॑णः प्राणापा॒नौ । प्रा॒णा॒पा॒नावे॒व । प्रा॒णा॒पा॒नाविति॑ प्राण - अ॒पा॒नौ । ए॒वास्मिन्न्॑ । अ॒स्मि॒न् द॒धा॒ति॒ । द॒धा॒ति॒ दि॒वः । दि॒वो वृ॒ष्टिः । वृ॒ष्टिर् वाताः᳚ । वाताः᳚ स्पृ॒ताः । स्पृ॒ता ए॑कविꣳ॒॒शः । ए॒क॒विꣳ॒॒शः स्तोमः॑ । ए॒क॒विꣳ॒॒श इत्ये॑क - विꣳ॒॒शः । स्तोम॒ इति॑ । इत्या॑ह । आ॒ह॒ प्र॒ति॒ष्ठा । प्र॒ति॒ष्ठा वै । प्र॒ति॒ष्ठेति॑ प्रति - स्था । वा ए॑कविꣳ॒॒शः । ए॒क॒विꣳ॒॒शः प्रति॑ष्ठित्यै । ए॒क॒विꣳ॒॒श इत्ये॑क - विꣳ॒॒शः । प्रति॑ष्ठित्या॒ इन्द्र॑स्य । प्रति॑ष्ठित्या॒ इति॒ प्रति॑ - स्थि॒त्यै॒ । इन्द्र॑स्य भा॒गः । भा॒गो॑ऽसि । अ॒सीति॑ । इत्यु॑त्तर॒तः । उ॒त्त॒र॒त ओजः॑ । उ॒त्त॒र॒त इत्यु॑त् - त॒र॒तः । ओजो॒ वै । वा इन्द्रः॑ । इन्द्र॒ ओजः॑ । ओजो॒ विष्णुः॑ । विष्णु॒रोजः॑ । ओजः॑ क्ष॒त्रम् । क्ष॒त्रमोजः॑ । ओजः॑ पञ्चद॒शः । प॒ञ्च॒द॒श ओजः॑ । प॒ञ्च॒द॒श इति॑ पञ्च - द॒शः \newline

\textbf{Jatai Paata} \newline

1. अन्नꣳ॑ सप्तद॒शः स॑प्तद॒शो ऽन्न॒ मन्नꣳ॑ सप्तद॒शः । \newline
2. स॒प्त॒द॒शो ऽन्न॒ मन्नꣳ॑ सप्तद॒शः स॑प्तद॒शो ऽन्न᳚म् । \newline
3. स॒प्त॒द॒श इति॑ सप्त - द॒शः । \newline
4. अन्न॑ मे॒वै वान्न॒ मन्न॑ मे॒व । \newline
5. ए॒व द॑क्षिण॒तो द॑क्षिण॒त ए॒वैव द॑क्षिण॒तः । \newline
6. द॒क्षि॒ण॒तो ध॑त्ते धत्ते दक्षिण॒तो द॑क्षिण॒तो ध॑त्ते । \newline
7. ध॒त्ते॒ तस्मा॒त् तस्मा᳚द् धत्ते धत्ते॒ तस्मा᳚त् । \newline
8. तस्मा॒द् दक्षि॑णेन॒ दक्षि॑णेन॒ तस्मा॒त् तस्मा॒द् दक्षि॑णेन । \newline
9. दक्षि॑णे॒ नान्न॒ मन्न॒म् दक्षि॑णेन॒ दक्षि॑णे॒ नान्न᳚म् । \newline
10. अन्न॑ मद्यते ऽद्य॒ते ऽन्न॒ मन्न॑ मद्यते । \newline
11. अ॒द्य॒ते॒ मि॒त्रस्य॑ मि॒त्रस्या᳚ द्यते ऽद्यते मि॒त्रस्य॑ । \newline
12. मि॒त्रस्य॑ भा॒गो भा॒गो मि॒त्रस्य॑ मि॒त्रस्य॑ भा॒गः । \newline
13. भा॒गो᳚ ऽस्यसि भा॒गो भा॒गो॑ ऽसि । \newline
14. अ॒सीती त्य॑स्य॒ सीति॑ । \newline
15. इति॑ प॒श्चात् प॒श्चा दितीति॑ प॒श्चात् । \newline
16. प॒श्चात् प्रा॒णः प्रा॒णः प॒श्चात् प॒श्चात् प्रा॒णः । \newline
17. प्रा॒णो वै वै प्रा॒णः प्रा॒णो वै । \newline
18. प्रा॒ण इति॑ प्र - अ॒नः । \newline
19. वै मि॒त्रो मि॒त्रो वै वै मि॒त्रः । \newline
20. मि॒त्रो॑ ऽपा॒नो॑ ऽपा॒नो मि॒त्रो मि॒त्रो॑ ऽपा॒नः । \newline
21. अ॒पा॒नो वरु॑णो॒ वरु॑णो ऽपा॒नो॑ ऽपा॒नो वरु॑णः । \newline
22. अ॒पा॒न इत्य॑प - अ॒नः । \newline
23. वरु॑णः प्राणापा॒नौ प्रा॑णापा॒नौ वरु॑णो॒ वरु॑णः प्राणापा॒नौ । \newline
24. प्रा॒णा॒पा॒ना वे॒वैव प्रा॑णापा॒नौ प्रा॑णापा॒ना वे॒व । \newline
25. प्रा॒णा॒पा॒नाविति॑ प्राण - अ॒पा॒नौ । \newline
26. ए॒वास्मि॑न् नस्मिन् ने॒वै वास्मिन्न्॑ । \newline
27. अ॒स्मि॒न् द॒धा॒ति॒ द॒धा॒ त्य॒स्मि॒न् न॒स्मि॒न् द॒धा॒ति॒ । \newline
28. द॒धा॒ति॒ दि॒वो दि॒वो द॑धाति दधाति दि॒वः । \newline
29. दि॒वो वृ॒ष्टिर् वृ॒ष्टिर् दि॒वो दि॒वो वृ॒ष्टिः । \newline
30. वृ॒ष्टिर् वाता॒ वाता॑ वृ॒ष्टिर् वृ॒ष्टिर् वाताः᳚ । \newline
31. वाताः᳚ स्पृ॒ताः स्पृ॒ता वाता॒ वाताः᳚ स्पृ॒ताः । \newline
32. स्पृ॒ता ए॑कविꣳ॒॒श ए॑कविꣳ॒॒शः स्पृ॒ताः स्पृ॒ता ए॑कविꣳ॒॒शः । \newline
33. ए॒क॒विꣳ॒॒शः स्तोमः॒ स्तोम॑ एकविꣳ॒॒श ए॑कविꣳ॒॒शः स्तोमः॑ । \newline
34. ए॒क॒विꣳ॒॒श इत्ये॑क - विꣳ॒॒शः । \newline
35. स्तोम॒ इतीति॒ स्तोमः॒ स्तोम॒ इति॑ । \newline
36. इत्या॑ हा॒हे तीत्या॑ह । \newline
37. आ॒ह॒ प्र॒ति॒ष्ठा प्र॑ति॒ष्ठा ऽऽहा॑ह प्रति॒ष्ठा । \newline
38. प्र॒ति॒ष्ठा वै वै प्र॑ति॒ष्ठा प्र॑ति॒ष्ठा वै । \newline
39. प्र॒ति॒ष्ठेति॑ प्रति - स्था । \newline
40. वा ए॑कविꣳ॒॒श ए॑कविꣳ॒॒शो वै वा ए॑कविꣳ॒॒शः । \newline
41. ए॒क॒विꣳ॒॒शः प्रति॑ष्ठित्यै॒ प्रति॑ष्ठित्या एकविꣳ॒॒श ए॑कविꣳ॒॒शः प्रति॑ष्ठित्यै । \newline
42. ए॒क॒विꣳ॒॒श इत्ये॑क - विꣳ॒॒शः । \newline
43. प्रति॑ष्ठित्या॒ इन्द्र॒ स्येन्द्र॑स्य॒ प्रति॑ष्ठित्यै॒ प्रति॑ष्ठित्या॒ इन्द्र॑स्य । \newline
44. प्रति॑ष्ठित्या॒ इति॒ प्रति॑ - स्थि॒त्यै॒ । \newline
45. इन्द्र॑स्य भा॒गो भा॒ग इन्द्र॒ स्येन्द्र॑स्य भा॒गः । \newline
46. भा॒गो᳚ ऽस्यसि भा॒गो भा॒गो॑ ऽसि । \newline
47. अ॒सीती त्य॑स्य॒ सीति॑ । \newline
48. इत्यु॑त्तर॒त उ॑त्तर॒त इती त्यु॑त्तर॒तः । \newline
49. उ॒त्त॒र॒त ओज॒ ओज॑ उत्तर॒त उ॑त्तर॒त ओजः॑ । \newline
50. उ॒त्त॒र॒त इत्यु॑त् - त॒र॒तः । \newline
51. ओजो॒ वै वा ओज॒ ओजो॒ वै । \newline
52. वा इन्द्र॒ इन्द्रो॒ वै वा इन्द्रः॑ । \newline
53. इन्द्र॒ ओज॒ ओज॒ इन्द्र॒ इन्द्र॒ ओजः॑ । \newline
54. ओजो॒ विष्णु॒र् विष्णु॒ रोज॒ ओजो॒ विष्णुः॑ । \newline
55. विष्णु॒ रोज॒ ओजो॒ विष्णु॒र् विष्णु॒ रोजः॑ । \newline
56. ओजः॑ क्ष॒त्रम् क्ष॒त्र मोज॒ ओजः॑ क्ष॒त्रम् । \newline
57. क्ष॒त्र मोज॒ ओजः॑ क्ष॒त्रम् क्ष॒त्र मोजः॑ । \newline
58. ओजः॑ पञ्चद॒शः प॑ञ्चद॒श ओज॒ ओजः॑ पञ्चद॒शः । \newline
59. प॒ञ्च॒द॒श ओज॒ ओजः॑ पञ्चद॒शः प॑ञ्चद॒श ओजः॑ । \newline
60. प॒ञ्च॒द॒श इति॑ पञ्च - द॒शः । \newline

\textbf{Ghana Paata } \newline

1. अन्नꣳ॑ सप्तद॒शः स॑प्तद॒शो ऽन्न॒ मन्नꣳ॑ सप्तद॒शो ऽन्न॒ मन्नꣳ॑ सप्तद॒शो ऽन्न॒ मन्नꣳ॑ सप्तद॒शो ऽन्न᳚म् । \newline
2. स॒प्त॒द॒शो ऽन्न॒ मन्नꣳ॑ सप्तद॒शः स॑प्तद॒शो ऽन्न॑ मे॒वै वान्नꣳ॑ सप्तद॒शः स॑प्तद॒शो ऽन्न॑ मे॒व । \newline
3. स॒प्त॒द॒श इति॑ सप्त - द॒शः । \newline
4. अन्न॑ मे॒वै वान्न॒ मन्न॑ मे॒व द॑क्षिण॒तो द॑क्षिण॒त ए॒वान्न॒ मन्न॑ मे॒व द॑क्षिण॒तः । \newline
5. ए॒व द॑क्षिण॒तो द॑क्षिण॒त ए॒वैव द॑क्षिण॒तो ध॑त्ते धत्ते दक्षिण॒त ए॒वैव द॑क्षिण॒तो ध॑त्ते । \newline
6. द॒क्षि॒ण॒तो ध॑त्ते धत्ते दक्षिण॒तो द॑क्षिण॒तो ध॑त्ते॒ तस्मा॒त् तस्मा᳚द् धत्ते दक्षिण॒तो द॑क्षिण॒तो ध॑त्ते॒ तस्मा᳚त् । \newline
7. ध॒त्ते॒ तस्मा॒त् तस्मा᳚द् धत्ते धत्ते॒ तस्मा॒द् दक्षि॑णेन॒ दक्षि॑णेन॒ तस्मा᳚द् धत्ते धत्ते॒ तस्मा॒द् दक्षि॑णेन । \newline
8. तस्मा॒द् दक्षि॑णेन॒ दक्षि॑णेन॒ तस्मा॒त् तस्मा॒द् दक्षि॑णे॒ नान्न॒ मन्न॒म् दक्षि॑णेन॒ तस्मा॒त् तस्मा॒द् दक्षि॑णे॒ नान्न᳚म् । \newline
9. दक्षि॑णे॒ नान्न॒ मन्न॒म् दक्षि॑णेन॒ दक्षि॑णे॒ नान्न॑ मद्यते ऽद्य॒ते ऽन्न॒म् दक्षि॑णेन॒ दक्षि॑णे॒ नान्न॑ मद्यते । \newline
10. अन्न॑ मद्यते ऽद्य॒ते ऽन्न॒ मन्न॑ मद्यते मि॒त्रस्य॑ मि॒त्रस्या᳚ द्य॒ते ऽन्न॒ मन्न॑ मद्यते मि॒त्रस्य॑ । \newline
11. अ॒द्य॒ते॒ मि॒त्रस्य॑ मि॒त्रस्या᳚ द्यते ऽद्यते मि॒त्रस्य॑ भा॒गो भा॒गो मि॒त्रस्या᳚ द्यते ऽद्यते मि॒त्रस्य॑ भा॒गः । \newline
12. मि॒त्रस्य॑ भा॒गो भा॒गो मि॒त्रस्य॑ मि॒त्रस्य॑ भा॒गो᳚ ऽस्यसि भा॒गो मि॒त्रस्य॑ मि॒त्रस्य॑ भा॒गो॑ ऽसि । \newline
13. भा॒गो᳚ ऽस्यसि भा॒गो भा॒गो॑ ऽसीती त्य॑सि भा॒गो भा॒गो॑ ऽसीति॑ । \newline
14. अ॒सीती त्य॑स्य॒सीति॑ प॒श्चात् प॒श्चा दित्य॑स्य॒ सीति॑ प॒श्चात् । \newline
15. इति॑ प॒श्चात् प॒श्चा दितीति॑ प॒श्चात् प्रा॒णः प्रा॒णः प॒श्चा दितीति॑ प॒श्चात् प्रा॒णः । \newline
16. प॒श्चात् प्रा॒णः प्रा॒णः प॒श्चात् प॒श्चात् प्रा॒णो वै वै प्रा॒णः प॒श्चात् प॒श्चात् प्रा॒णो वै । \newline
17. प्रा॒णो वै वै प्रा॒णः प्रा॒णो वै मि॒त्रो मि॒त्रो वै प्रा॒णः प्रा॒णो वै मि॒त्रः । \newline
18. प्रा॒ण इति॑ प्र - अ॒नः । \newline
19. वै मि॒त्रो मि॒त्रो वै वै मि॒त्रो॑ ऽपा॒नो॑ ऽपा॒नो मि॒त्रो वै वै मि॒त्रो॑ ऽपा॒नः । \newline
20. मि॒त्रो॑ ऽपा॒नो॑ ऽपा॒नो मि॒त्रो मि॒त्रो॑ ऽपा॒नो वरु॑णो॒ वरु॑णो ऽपा॒नो मि॒त्रो मि॒त्रो॑ ऽपा॒नो वरु॑णः । \newline
21. अ॒पा॒नो वरु॑णो॒ वरु॑णो ऽपा॒नो॑ ऽपा॒नो वरु॑णः प्राणापा॒नौ प्रा॑णापा॒नौ वरु॑णो ऽपा॒नो॑ ऽपा॒नो वरु॑णः प्राणापा॒नौ । \newline
22. अ॒पा॒न इत्य॑प - अ॒नः । \newline
23. वरु॑णः प्राणापा॒नौ प्रा॑णापा॒नौ वरु॑णो॒ वरु॑णः प्राणापा॒ना वे॒वैव प्रा॑णापा॒नौ वरु॑णो॒ वरु॑णः प्राणापा॒ना वे॒व । \newline
24. प्रा॒णा॒पा॒ना वे॒वैव प्रा॑णापा॒नौ प्रा॑णापा॒ना वे॒वास्मि॑न् नस्मिन् ने॒व प्रा॑णापा॒नौ प्रा॑णापा॒ना वे॒वास्मिन्न्॑ । \newline
25. प्रा॒णा॒पा॒नाविति॑ प्राण - अ॒पा॒नौ । \newline
26. ए॒वास्मि॑न् नस्मिन् ने॒वै वास्मि॑न् दधाति दधा त्यस्मिन् ने॒वै वास्मि॑न् दधाति । \newline
27. अ॒स्मि॒न् द॒धा॒ति॒ द॒धा॒ त्य॒स्मि॒न् न॒स्मि॒न् द॒धा॒ति॒ दि॒वो दि॒वो द॑धा त्यस्मिन् नस्मिन् दधाति दि॒वः । \newline
28. द॒धा॒ति॒ दि॒वो दि॒वो द॑धाति दधाति दि॒वो वृ॒ष्टिर् वृ॒ष्टिर् दि॒वो द॑धाति दधाति दि॒वो वृ॒ष्टिः । \newline
29. दि॒वो वृ॒ष्टिर् वृ॒ष्टिर् दि॒वो दि॒वो वृ॒ष्टिर् वाता॒ वाता॑ वृ॒ष्टिर् दि॒वो दि॒वो वृ॒ष्टिर् वाताः᳚ । \newline
30. वृ॒ष्टिर् वाता॒ वाता॑ वृ॒ष्टिर् वृ॒ष्टिर् वाताः᳚ स्पृ॒ताः स्पृ॒ता वाता॑ वृ॒ष्टिर् वृ॒ष्टिर् वाताः᳚ स्पृ॒ताः । \newline
31. वाताः᳚ स्पृ॒ताः स्पृ॒ता वाता॒ वाताः᳚ स्पृ॒ता ए॑कविꣳ॒॒श ए॑कविꣳ॒॒शः स्पृ॒ता वाता॒ वाताः᳚ स्पृ॒ता ए॑कविꣳ॒॒शः । \newline
32. स्पृ॒ता ए॑कविꣳ॒॒श ए॑कविꣳ॒॒शः स्पृ॒ताः स्पृ॒ता ए॑कविꣳ॒॒शः स्तोमः॒ स्तोम॑ एकविꣳ॒॒शः स्पृ॒ताः स्पृ॒ता ए॑कविꣳ॒॒शः स्तोमः॑ । \newline
33. ए॒क॒विꣳ॒॒शः स्तोमः॒ स्तोम॑ एकविꣳ॒॒श ए॑कविꣳ॒॒शः स्तोम॒ इतीति॒ स्तोम॑ एकविꣳ॒॒श ए॑कविꣳ॒॒शः स्तोम॒ इति॑ । \newline
34. ए॒क॒विꣳ॒॒श इत्ये॑क - विꣳ॒॒शः । \newline
35. स्तोम॒ इतीति॒ स्तोमः॒ स्तोम॒ इत्या॑हा॒हेति॒ स्तोमः॒ स्तोम॒ इत्या॑ह । \newline
36. इत्या॑ हा॒हेतीत्या॑ह प्रति॒ष्ठा प्र॑ति॒ष्ठा ऽऽहेतीत्या॑ह प्रति॒ष्ठा । \newline
37. आ॒ह॒ प्र॒ति॒ष्ठा प्र॑ति॒ष्ठा ऽऽहा॑ह प्रति॒ष्ठा वै वै प्र॑ति॒ष्ठा ऽऽहा॑ह प्रति॒ष्ठा वै । \newline
38. प्र॒ति॒ष्ठा वै वै प्र॑ति॒ष्ठा प्र॑ति॒ष्ठा वा ए॑कविꣳ॒॒श ए॑कविꣳ॒॒शो वै प्र॑ति॒ष्ठा प्र॑ति॒ष्ठा वा ए॑कविꣳ॒॒शः । \newline
39. प्र॒ति॒ष्ठेति॑ प्रति - स्था । \newline
40. वा ए॑कविꣳ॒॒श ए॑कविꣳ॒॒शो वै वा ए॑कविꣳ॒॒शः प्रति॑ष्ठित्यै॒ प्रति॑ष्ठित्या एकविꣳ॒॒शो वै वा ए॑कविꣳ॒॒शः प्रति॑ष्ठित्यै । \newline
41. ए॒क॒विꣳ॒॒शः प्रति॑ष्ठित्यै॒ प्रति॑ष्ठित्या एकविꣳ॒॒श ए॑कविꣳ॒॒शः प्रति॑ष्ठित्या॒ इन्द्र॒ स्येन्द्र॑स्य॒ प्रति॑ष्ठित्या एकविꣳ॒॒श ए॑कविꣳ॒॒शः प्रति॑ष्ठित्या॒ इन्द्र॑स्य । \newline
42. ए॒क॒विꣳ॒॒श इत्ये॑क - विꣳ॒॒शः । \newline
43. प्रति॑ष्ठित्या॒ इन्द्र॒ स्येन्द्र॑स्य॒ प्रति॑ष्ठित्यै॒ प्रति॑ष्ठित्या॒ इन्द्र॑स्य भा॒गो भा॒ग इन्द्र॑स्य॒ प्रति॑ष्ठित्यै॒ प्रति॑ष्ठित्या॒ इन्द्र॑स्य भा॒गः । \newline
44. प्रति॑ष्ठित्या॒ इति॒ प्रति॑ - स्थि॒त्यै॒ । \newline
45. इन्द्र॑स्य भा॒गो भा॒ग इन्द्र॒ स्येन्द्र॑स्य भा॒गो᳚ ऽस्यसि भा॒ग इन्द्र॒ स्येन्द्र॑स्य भा॒गो॑ ऽसि । \newline
46. भा॒गो᳚ ऽस्यसि भा॒गो भा॒गो॑ ऽसीती त्य॑सि भा॒गो भा॒गो॑ ऽसीति॑ । \newline
47. अ॒सीती त्य॑स्य॒सी त्यु॑त्तर॒त उ॑त्तर॒त इत्य॑स्य॒सी त्यु॑त्तर॒तः । \newline
48. इत्यु॑त्तर॒त उ॑त्तर॒त इती त्यु॑त्तर॒त ओज॒ ओज॑ उत्तर॒त इती त्यु॑त्तर॒त ओजः॑ । \newline
49. उ॒त्त॒र॒त ओज॒ ओज॑ उत्तर॒त उ॑त्तर॒त ओजो॒ वै वा ओज॑ उत्तर॒त उ॑त्तर॒त ओजो॒ वै । \newline
50. उ॒त्त॒र॒त इत्यु॑त् - त॒र॒तः । \newline
51. ओजो॒ वै वा ओज॒ ओजो॒ वा इन्द्र॒ इन्द्रो॒ वा ओज॒ ओजो॒ वा इन्द्रः॑ । \newline
52. वा इन्द्र॒ इन्द्रो॒ वै वा इन्द्र॒ ओज॒ ओज॒ इन्द्रो॒ वै वा इन्द्र॒ ओजः॑ । \newline
53. इन्द्र॒ ओज॒ ओज॒ इन्द्र॒ इन्द्र॒ ओजो॒ विष्णु॒र् विष्णु॒ रोज॒ इन्द्र॒ इन्द्र॒ ओजो॒ विष्णुः॑ । \newline
54. ओजो॒ विष्णु॒र् विष्णु॒ रोज॒ ओजो॒ विष्णु॒ रोज॒ ओजो॒ विष्णु॒ रोज॒ ओजो॒ विष्णु॒ रोजः॑ । \newline
55. विष्णु॒ रोज॒ ओजो॒ विष्णु॒र् विष्णु॒ रोजः॑ क्ष॒त्रम् क्ष॒त्र मोजो॒ विष्णु॒र् विष्णु॒ रोजः॑ क्ष॒त्रम् । \newline
56. ओजः॑ क्ष॒त्रम् क्ष॒त्र मोज॒ ओजः॑ क्ष॒त्र मोज॒ ओजः॑ क्ष॒त्र मोज॒ ओजः॑ क्ष॒त्र मोजः॑ । \newline
57. क्ष॒त्र मोज॒ ओजः॑ क्ष॒त्रम् क्ष॒त्र मोजः॑ पञ्चद॒शः प॑ञ्चद॒श ओजः॑ क्ष॒त्रम् क्ष॒त्र मोजः॑ पञ्चद॒शः । \newline
58. ओजः॑ पञ्चद॒शः प॑ञ्चद॒श ओज॒ ओजः॑ पञ्चद॒श ओज॒ ओजः॑ पञ्चद॒श ओज॒ ओजः॑ पञ्चद॒श ओजः॑ । \newline
59. प॒ञ्च॒द॒श ओज॒ ओजः॑ पञ्चद॒शः प॑ञ्चद॒श ओज॑ ए॒वै वौजः॑ पञ्चद॒शः प॑ञ्चद॒श ओज॑ ए॒व । \newline
60. प॒ञ्च॒द॒श इति॑ पञ्च - द॒शः । \newline
\pagebreak
\markright{ TS 5.3.4.3  \hfill https://www.vedavms.in \hfill}

\section{ TS 5.3.4.3 }

\textbf{TS 5.3.4.3 } \newline
\textbf{Samhita Paata} \newline

ओज॑ ए॒वोत्त॑र॒तो ध॑त्ते॒ तस्मा॑दुत्तरतो-ऽभिप्रया॒यी ज॑यति॒ वसू॑नां भा॒गो॑ऽसीति॑ पु॒रस्ता॒दुप॑ दधाति यज्ञ्मु॒खं ॅवै वस॑वो यज्ञ्मु॒खꣳ रु॒द्रा य॑ज्ञ्मु॒खं च॑तुर्विꣳ॒॒शो य॑ज्ञ्मु॒खमे॒व पु॒रस्ता॒द्वि या॑तयत्यादि॒त्यानां᳚ भा॒गो॑ऽसीति॑ दक्षिण॒तोऽन्नं॒ ॅवा आ॑दि॒त्या अन्नं॑ म॒रुतोऽन्नं॒ गर्भा॒ अन्नं॑ पञ्चविꣳ॒॒शोऽन्न॑मे॒व द॑क्षिण॒तो ध॑त्ते॒ तस्मा॒द् दक्षि॑णे॒नाऽन्न॑मद्य॒ते ऽदि॑त्यै भा॒गो॑ - [  ] \newline

\textbf{Pada Paata} \newline

ओजः॑ । ए॒व । उ॒त्त॒र॒त इत्यु॑त् - त॒र॒तः । ध॒त्ते॒ । तस्मा᳚त् । उ॒त्त॒र॒तो॒ऽभि॒प्र॒या॒यीत्यु॑त्तरतः-अ॒भि॒प्र॒या॒यी । ज॒य॒ति॒ । वसू॑नाम् । भा॒गः । अ॒सि॒ । इति॑ । पु॒रस्ता᳚त् । उपेति॑ । द॒धा॒ति॒ । य॒ज्ञ्॒मु॒खमिति॑ यज्ञ्-मु॒खम् । वै । वस॑वः । य॒ज्ञ्॒मु॒खमिति॑ यज्ञ् - मु॒खम् । रु॒द्राः । य॒ज्ञ्॒मु॒खमिति॑ यज्ञ् - मु॒खम् । च॒तु॒र्विꣳ॒॒श इति॑ चतुः - विꣳ॒॒शः । य॒ज्ञ्॒मु॒खमिति॑ यज्ञ् - मु॒खम् । ए॒व । पु॒रस्ता᳚त् । वीति॑ । या॒त॒य॒ति॒ । आ॒दि॒त्याना᳚म् । भा॒गः । अ॒सि॒ । इति॑ । द॒क्षि॒ण॒तः । अन्न᳚म् । वै । आ॒दि॒त्याः । अन्न᳚म् । म॒रुतः॑ । अन्न᳚म् । गर्भाः᳚ । अन्न᳚म् । प॒ञ्च॒विꣳ॒॒श इति॑ पञ्च - विꣳ॒॒शः । अन्न᳚म् । ए॒व । द॒क्षि॒ण॒तः । ध॒त्ते॒ । तस्मा᳚त् । दक्षि॑णेन । अन्न᳚म् । अ॒द्य॒ते॒ । अदि॑त्यै । भा॒गः ।  \newline


\textbf{Krama Paata} \newline

ओज॑ ए॒व । ए॒वोत्त॑र॒तः । उ॒त्त॒र॒तो ध॑त्ते । उ॒त्त॒र॒त इत्यु॑त् - त॒र॒तः । ध॒त्ते॒ तस्मा᳚त् । तस्मा॑दुत्तरतोभिप्रया॒यी । उ॒त्त॒र॒तो॒भि॒प्र॒या॒यी ज॑यति । उ॒त्त॒र॒तो॒भि॒प्र॒या॒यीत्यु॑त्तरतः - अ॒भि॒प्र॒या॒यी । ज॒य॒ति॒ वसू॑नाम् । वसू॑नाम् भा॒गः । भा॒गो॑ऽसि । अ॒सीति॑ । इति॑ पु॒रस्ता᳚त् । पु॒रस्ता॒दुप॑ । उप॑ दधाति । द॒धा॒ति॒ य॒ज्ञ्॒मु॒खम् । य॒ज्ञ्॒मु॒खम् ॅवै । य॒ज्ञ्॒मु॒खमिति॑ यज्ञ् - मु॒खम् । वै वस॑वः । वस॑वो यज्ञ्मु॒खम् । य॒ज्ञ्॒मु॒खꣳ रु॒द्राः । य॒ज्ञ्॒मु॒खमिति॑ यज्ञ् - मु॒खम् । रु॒द्रा य॑ज्ञ्मु॒खम् । य॒ज्ञ्॒मु॒खम् च॑तुर्विꣳ॒॒शः । य॒ज्ञ्॒मु॒खमिति॑ यज्ञ् - मु॒खम् । च॒तु॒र्विꣳ॒॒शो य॑ज्ञ्मु॒खम् । च॒तु॒र्विꣳ॒॒श इति॑ चतुः - विꣳ॒॒शः । य॒ज्ञ्॒मु॒खमे॒व । य॒ज्ञ्॒मु॒खमिति॑ यज्ञ् - मु॒खम् । ए॒व पु॒रस्ता᳚त् । पु॒रस्ता॒द् वि । वि या॑तयति । या॒त॒य॒त्या॒दि॒त्याना᳚म् । आ॒दि॒त्याना᳚म् भा॒गः । भा॒गो॑ऽसि । अ॒सीति॑ । इति॑ दक्षिण॒तः । द॒क्षि॒ण॒तोऽन्न᳚म् । अन्न॒म् ॅवै । वा आ॑दि॒त्याः । आ॒दि॒त्या अन्न᳚म् । अन्न॑म् म॒रुतः॑ । म॒रुतोऽन्न᳚म् । अन्न॒म् गर्भाः᳚ । गर्भा॒ अन्न᳚म् । अन्न॑म् पञ्चविꣳ॒॒शः । प॒ञ्च॒विꣳ॒॒शोऽन्न᳚म् । प॒ञ्च॒विꣳ॒॒श इति॑ पञ्च - विꣳ॒॒शः । अन्न॑मे॒व । ए॒व द॑क्षिण॒तः । द॒क्षि॒ण॒तो ध॑त्ते । ध॒त्ते॒ तस्मा᳚त् । तस्मा॒द् दक्षि॑णेन । दक्षि॑णे॒नान्न᳚म् । अन्न॑मद्यते । अ॒द्य॒तेऽदि॑त्यै । अदि॑त्यै भा॒गः । भा॒गो॑ऽसि \newline

\textbf{Jatai Paata} \newline

1. ओज॑ ए॒वै वौज॒ ओज॑ ए॒व । \newline
2. ए॒वोत्त॑र॒त उ॑त्तर॒त ए॒वैवोत्त॑र॒तः । \newline
3. उ॒त्त॒र॒तो ध॑त्ते धत्त उत्तर॒त उ॑त्तर॒तो ध॑त्ते । \newline
4. उ॒त्त॒र॒त इत्यु॑त् - त॒र॒तः । \newline
5. ध॒त्ते॒ तस्मा॒त् तस्मा᳚द् धत्ते धत्ते॒ तस्मा᳚त् । \newline
6. तस्मा॑ दुत्तरतोभिप्रया॒ य्यु॑त्तरतोभिप्रया॒यी तस्मा॒त् तस्मा॑ दुत्तरतोभिप्रया॒यी । \newline
7. उ॒त्त॒र॒तो॒भि॒प्र॒या॒यी ज॑यति जय त्युत्तरतोभिप्रया॒ य्यु॑त्तरतोभिप्रया॒यी ज॑यति । \newline
8. उ॒त्त॒र॒तो॒भि॒प्र॒या॒यीत्यु॑त्तरतः - अ॒भि॒प्र॒या॒यी । \newline
9. ज॒य॒ति॒ वसू॑नां॒ ॅवसू॑नाम् जयति जयति॒ वसू॑नाम् । \newline
10. वसू॑नाम् भा॒गो भा॒गो वसू॑नां॒ ॅवसू॑नाम् भा॒गः । \newline
11. भा॒गो᳚ ऽस्यसि भा॒गो भा॒गो॑ ऽसि । \newline
12. अ॒सीती त्य॑स्य॒ सीति॑ । \newline
13. इति॑ पु॒रस्ता᳚त् पु॒रस्ता॒ दितीति॑ पु॒रस्ता᳚त् । \newline
14. पु॒रस्ता॒ दुपोप॑ पु॒रस्ता᳚त् पु॒रस्ता॒ दुप॑ । \newline
15. उप॑ दधाति दधा॒ त्युपोप॑ दधाति । \newline
16. द॒धा॒ति॒ य॒ज्ञ्॒मु॒खं ॅय॑ज्ञ्मु॒खम् द॑धाति दधाति यज्ञ्मु॒खम् । \newline
17. य॒ज्ञ्॒मु॒खं ॅवै वै य॑ज्ञ्मु॒खं ॅय॑ज्ञ्मु॒खं ॅवै । \newline
18. य॒ज्ञ्॒मु॒खमिति॑ यज्ञ् - मु॒खम् । \newline
19. वै वस॑वो॒ वस॑वो॒ वै वै वस॑वः । \newline
20. वस॑वो यज्ञ्मु॒खं ॅय॑ज्ञ्मु॒खं ॅवस॑वो॒ वस॑वो यज्ञ्मु॒खम् । \newline
21. य॒ज्ञ्॒मु॒खꣳ रु॒द्रा रु॒द्रा य॑ज्ञ्मु॒खं ॅय॑ज्ञ्मु॒खꣳ रु॒द्राः । \newline
22. य॒ज्ञ्॒मु॒खमिति॑ यज्ञ् - मु॒खम् । \newline
23. रु॒द्रा य॑ज्ञ्मु॒खं ॅय॑ज्ञ्मु॒खꣳ रु॒द्रा रु॒द्रा य॑ज्ञ्मु॒खम् । \newline
24. य॒ज्ञ्॒मु॒खम् च॑तुर्विꣳ॒॒श श्च॑तुर्विꣳ॒॒शो य॑ज्ञ्मु॒खं ॅय॑ज्ञ्मु॒खम् च॑तुर्विꣳ॒॒शः । \newline
25. य॒ज्ञ्॒मु॒खमिति॑ यज्ञ् - मु॒खम् । \newline
26. च॒तु॒र्विꣳ॒॒शो य॑ज्ञ्मु॒खं ॅय॑ज्ञ्मु॒खम् च॑तुर्विꣳ॒॒श श्च॑तुर्विꣳ॒॒शो य॑ज्ञ्मु॒खम् । \newline
27. च॒तु॒र्विꣳ॒॒श इति॑ चतुः - विꣳ॒॒शः । \newline
28. य॒ज्ञ्॒मु॒ख मे॒वैव य॑ज्ञ्मु॒खं ॅय॑ज्ञ्मु॒ख मे॒व । \newline
29. य॒ज्ञ्॒मु॒खमिति॑ यज्ञ् - मु॒खम् । \newline
30. ए॒व पु॒रस्ता᳚त् पु॒रस्ता॑ दे॒वैव पु॒रस्ता᳚त् । \newline
31. पु॒रस्ता॒द् वि वि पु॒रस्ता᳚त् पु॒रस्ता॒द् वि । \newline
32. वि या॑तयति यातयति॒ वि वि या॑तयति । \newline
33. या॒त॒य॒ त्या॒दि॒त्याना॑ मादि॒त्यानां᳚ ॅयातयति यातय त्यादि॒त्याना᳚म् । \newline
34. आ॒दि॒त्याना᳚म् भा॒गो भा॒ग आ॑दि॒त्याना॑ मादि॒त्याना᳚म् भा॒गः । \newline
35. भा॒गो᳚ ऽस्यसि भा॒गो भा॒गो॑ ऽसि । \newline
36. अ॒सीती त्य॑स्य॒ सीति॑ । \newline
37. इति॑ दक्षिण॒तो द॑क्षिण॒त इतीति॑ दक्षिण॒तः । \newline
38. द॒क्षि॒ण॒तो ऽन्न॒ मन्न॑म् दक्षिण॒तो द॑क्षिण॒तो ऽन्न᳚म् । \newline
39. अन्नं॒ ॅवै वा अन्न॒ मन्नं॒ ॅवै । \newline
40. वा आ॑दि॒त्या आ॑दि॒त्या वै वा आ॑दि॒त्याः । \newline
41. आ॒दि॒त्या अन्न॒ मन्न॑ मादि॒त्या आ॑दि॒त्या अन्न᳚म् । \newline
42. अन्न॑म् म॒रुतो॑ म॒रुतो ऽन्न॒ मन्न॑म् म॒रुतः॑ । \newline
43. म॒रुतो ऽन्न॒ मन्न॑म् म॒रुतो॑ म॒रुतो ऽन्न᳚म् । \newline
44. अन्न॒म् गर्भा॒ गर्भा॒ अन्न॒ मन्न॒म् गर्भाः᳚ । \newline
45. गर्भा॒ अन्न॒ मन्न॒म् गर्भा॒ गर्भा॒ अन्न᳚म् । \newline
46. अन्न॑म् पञ्चविꣳ॒॒शः प॑ञ्चविꣳ॒॒शो ऽन्न॒ मन्न॑म् पञ्चविꣳ॒॒शः । \newline
47. प॒ञ्च॒विꣳ॒॒शो ऽन्न॒ मन्न॑म् पञ्चविꣳ॒॒शः प॑ञ्चविꣳ॒॒शो ऽन्न᳚म् । \newline
48. प॒ञ्च॒विꣳ॒॒श इति॑ पञ्च - विꣳ॒॒शः । \newline
49. अन्न॑ मे॒वै वान्न॒ मन्न॑ मे॒व । \newline
50. ए॒व द॑क्षिण॒तो द॑क्षिण॒त ए॒वैव द॑क्षिण॒तः । \newline
51. द॒क्षि॒ण॒तो ध॑त्ते धत्ते दक्षिण॒तो द॑क्षिण॒तो ध॑त्ते । \newline
52. ध॒त्ते॒ तस्मा॒त् तस्मा᳚द् धत्ते धत्ते॒ तस्मा᳚त् । \newline
53. तस्मा॒द् दक्षि॑णेन॒ दक्षि॑णेन॒ तस्मा॒त् तस्मा॒द् दक्षि॑णेन । \newline
54. दक्षि॑णे॒ नान्न॒ मन्न॒म् दक्षि॑णेन॒ दक्षि॑णे॒ नान्न᳚म् । \newline
55. अन्न॑ मद्यते ऽद्य॒ते ऽन्न॒ मन्न॑ मद्यते । \newline
56. अ॒द्य॒ते ऽदि॑त्या॒ अदि॑त्या अद्यते ऽद्य॒ते ऽदि॑त्यै । \newline
57. अदि॑त्यै भा॒गो भा॒गो ऽदि॑त्या॒ अदि॑त्यै भा॒गः । \newline
58. भा॒गो᳚ ऽस्यसि भा॒गो भा॒गो॑ ऽसि । \newline

\textbf{Ghana Paata } \newline

1. ओज॑ ए॒वै वौज॒ ओज॑ ए॒वो त्त॑र॒त उ॑त्तर॒त ए॒वौज॒ ओज॑ ए॒वो त्त॑र॒तः । \newline
2. ए॒वोत्त॑र॒त उ॑त्तर॒त ए॒वैवोत्त॑र॒तो ध॑त्ते धत्त उत्तर॒त ए॒वैवोत्त॑र॒तो ध॑त्ते । \newline
3. उ॒त्त॒र॒तो ध॑त्ते धत्त उत्तर॒त उ॑त्तर॒तो ध॑त्ते॒ तस्मा॒त् तस्मा᳚द् धत्त उत्तर॒त उ॑त्तर॒तो ध॑त्ते॒ तस्मा᳚त् । \newline
4. उ॒त्त॒र॒त इत्यु॑त् - त॒र॒तः । \newline
5. ध॒त्ते॒ तस्मा॒त् तस्मा᳚द् धत्ते धत्ते॒ तस्मा॑ दुत्तरतोभिप्रया॒ य्यु॑त्तरतोभिप्रया॒यी तस्मा᳚द् धत्ते धत्ते॒ तस्मा॑ दुत्तरतोभिप्रया॒यी । \newline
6. तस्मा॑ दुत्तरतोभिप्रया॒ य्यु॑त्तरतोभिप्रया॒यी तस्मा॒त् तस्मा॑ दुत्तरतोभिप्रया॒यी ज॑यति जय त्युत्तरतोभिप्रया॒यी तस्मा॒त् तस्मा॑ दुत्तरतोभिप्रया॒यी ज॑यति । \newline
7. उ॒त्त॒र॒तो॒भि॒प्र॒या॒यी ज॑यति जय त्युत्तरतोभिप्रया॒ य्यु॑त्तरतोभिप्रया॒यी ज॑यति॒ वसू॑नां॒ ॅवसू॑नाम् जय त्युत्तरतोभिप्रया॒ य्यु॑त्तरतोभिप्रया॒यी ज॑यति॒ वसू॑नाम् । \newline
8. उ॒त्त॒र॒तो॒भि॒प्र॒या॒यीत्यु॑त्तरतः - अ॒भि॒प्र॒या॒यी । \newline
9. ज॒य॒ति॒ वसू॑नां॒ ॅवसू॑नाम् जयति जयति॒ वसू॑नाम् भा॒गो भा॒गो वसू॑नाम् जयति जयति॒ वसू॑नाम् भा॒गः । \newline
10. वसू॑नाम् भा॒गो भा॒गो वसू॑नां॒ ॅवसू॑नाम् भा॒गो᳚ ऽस्यसि भा॒गो वसू॑नां॒ ॅवसू॑नाम् भा॒गो॑ ऽसि । \newline
11. भा॒गो᳚ ऽस्यसि भा॒गो भा॒गो॑ ऽसीती त्य॑सि भा॒गो भा॒गो॑ ऽसीति॑ । \newline
12. अ॒सीती त्य॑स्य॒ सीति॑ पु॒रस्ता᳚त् पु॒रस्ता॒ दित्य॑स्य॒ सीति॑ पु॒रस्ता᳚त् । \newline
13. इति॑ पु॒रस्ता᳚त् पु॒रस्ता॒ दितीति॑ पु॒रस्ता॒ दुपोप॑ पु॒रस्ता॒ दितीति॑ पु॒रस्ता॒ दुप॑ । \newline
14. पु॒रस्ता॒ दुपोप॑ पु॒रस्ता᳚त् पु॒रस्ता॒ दुप॑ दधाति दधा॒ त्युप॑ पु॒रस्ता᳚त् पु॒रस्ता॒ दुप॑ दधाति । \newline
15. उप॑ दधाति दधा॒ त्युपोप॑ दधाति यज्ञ्मु॒खं ॅय॑ज्ञ्मु॒खम् द॑धा॒ त्युपोप॑ दधाति यज्ञ्मु॒खम् । \newline
16. द॒धा॒ति॒ य॒ज्ञ्॒मु॒खं ॅय॑ज्ञ्मु॒खम् द॑धाति दधाति यज्ञ्मु॒खं ॅवै वै य॑ज्ञ्मु॒खम् द॑धाति दधाति यज्ञ्मु॒खं ॅवै । \newline
17. य॒ज्ञ्॒मु॒खं ॅवै वै य॑ज्ञ्मु॒खं ॅय॑ज्ञ्मु॒खं ॅवै वस॑वो॒ वस॑वो॒ वै य॑ज्ञ्मु॒खं ॅय॑ज्ञ्मु॒खं ॅवै वस॑वः । \newline
18. य॒ज्ञ्॒मु॒खमिति॑ यज्ञ् - मु॒खम् । \newline
19. वै वस॑वो॒ वस॑वो॒ वै वै वस॑वो यज्ञ्मु॒खं ॅय॑ज्ञ्मु॒खं ॅवस॑वो॒ वै वै वस॑वो यज्ञ्मु॒खम् । \newline
20. वस॑वो यज्ञ्मु॒खं ॅय॑ज्ञ्मु॒खं ॅवस॑वो॒ वस॑वो यज्ञ्मु॒खꣳ रु॒द्रा रु॒द्रा य॑ज्ञ्मु॒खं ॅवस॑वो॒ वस॑वो यज्ञ्मु॒खꣳ रु॒द्राः । \newline
21. य॒ज्ञ्॒मु॒खꣳ रु॒द्रा रु॒द्रा य॑ज्ञ्मु॒खं ॅय॑ज्ञ्मु॒खꣳ रु॒द्रा य॑ज्ञ्मु॒खं ॅय॑ज्ञ्मु॒खꣳ रु॒द्रा य॑ज्ञ्मु॒खं ॅय॑ज्ञ्मु॒खꣳ रु॒द्रा य॑ज्ञ्मु॒खम् । \newline
22. य॒ज्ञ्॒मु॒खमिति॑ यज्ञ् - मु॒खम् । \newline
23. रु॒द्रा य॑ज्ञ्मु॒खं ॅय॑ज्ञ्मु॒खꣳ रु॒द्रा रु॒द्रा य॑ज्ञ्मु॒खम् च॑तुर्विꣳ॒॒श श्च॑तुर्विꣳ॒॒शो य॑ज्ञ्मु॒खꣳ रु॒द्रा रु॒द्रा य॑ज्ञ्मु॒खम् च॑तुर्विꣳ॒॒शः । \newline
24. य॒ज्ञ्॒मु॒खम् च॑तुर्विꣳ॒॒श श्च॑तुर्विꣳ॒॒शो य॑ज्ञ्मु॒खं ॅय॑ज्ञ्मु॒खम् च॑तुर्विꣳ॒॒शो य॑ज्ञ्मु॒खं ॅय॑ज्ञ्मु॒खम् च॑तुर्विꣳ॒॒शो य॑ज्ञ्मु॒खं ॅय॑ज्ञ्मु॒खम् च॑तुर्विꣳ॒॒शो य॑ज्ञ्मु॒खम् । \newline
25. य॒ज्ञ्॒मु॒खमिति॑ यज्ञ् - मु॒खम् । \newline
26. च॒तु॒र्विꣳ॒॒शो य॑ज्ञ्मु॒खं ॅय॑ज्ञ्मु॒खम् च॑तुर्विꣳ॒॒श श्च॑तुर्विꣳ॒॒शो य॑ज्ञ्मु॒ख मे॒वैव य॑ज्ञ्मु॒खम् च॑तुर्विꣳ॒॒श श्च॑तुर्विꣳ॒॒शो य॑ज्ञ्मु॒ख मे॒व । \newline
27. च॒तु॒र्विꣳ॒॒श इति॑ चतुः - विꣳ॒॒शः । \newline
28. य॒ज्ञ्॒मु॒ख मे॒वैव य॑ज्ञ्मु॒खं ॅय॑ज्ञ्मु॒ख मे॒व पु॒रस्ता᳚त् पु॒रस्ता॑ दे॒व य॑ज्ञ्मु॒खं ॅय॑ज्ञ्मु॒ख मे॒व पु॒रस्ता᳚त् । \newline
29. य॒ज्ञ्॒मु॒खमिति॑ यज्ञ् - मु॒खम् । \newline
30. ए॒व पु॒रस्ता᳚त् पु॒रस्ता॑ दे॒वैव पु॒रस्ता॒द् वि वि पु॒रस्ता॑ दे॒वैव पु॒रस्ता॒द् वि । \newline
31. पु॒रस्ता॒द् वि वि पु॒रस्ता᳚त् पु॒रस्ता॒द् वि या॑तयति यातयति॒ वि पु॒रस्ता᳚त् पु॒रस्ता॒द् वि या॑तयति । \newline
32. वि या॑तयति यातयति॒ वि वि या॑तय त्यादि॒त्याना॑ मादि॒त्यानां᳚ ॅयातयति॒ वि वि या॑तय त्यादि॒त्याना᳚म् । \newline
33. या॒त॒य॒ त्या॒दि॒त्याना॑ मादि॒त्यानां᳚ ॅयातयति यातय त्यादि॒त्याना᳚म् भा॒गो भा॒ग आ॑दि॒त्यानां᳚ ॅयातयति यातय त्यादि॒त्याना᳚म् भा॒गः । \newline
34. आ॒दि॒त्याना᳚म् भा॒गो भा॒ग आ॑दि॒त्याना॑ मादि॒त्याना᳚म् भा॒गो᳚ ऽस्यसि भा॒ग आ॑दि॒त्याना॑ मादि॒त्याना᳚म् भा॒गो॑ ऽसि । \newline
35. भा॒गो᳚ ऽस्यसि भा॒गो भा॒गो॑ ऽसीती त्य॑सि भा॒गो भा॒गो॑ ऽसीति॑ । \newline
36. अ॒सीती त्य॑स्य॒ सीति॑ दक्षिण॒तो द॑क्षिण॒त इत्य॑स्य॒ सीति॑ दक्षिण॒तः । \newline
37. इति॑ दक्षिण॒तो द॑क्षिण॒त इतीति॑ दक्षिण॒तो ऽन्न॒ मन्न॑म् दक्षिण॒त इतीति॑ दक्षिण॒तो ऽन्न᳚म् । \newline
38. द॒क्षि॒ण॒तो ऽन्न॒ मन्न॑म् दक्षिण॒तो द॑क्षिण॒तो ऽन्नं॒ ॅवै वा अन्न॑म् दक्षिण॒तो द॑क्षिण॒तो ऽन्नं॒ ॅवै । \newline
39. अन्नं॒ ॅवै वा अन्न॒ मन्नं॒ ॅवा आ॑दि॒त्या आ॑दि॒त्या वा अन्न॒ मन्नं॒ ॅवा आ॑दि॒त्याः । \newline
40. वा आ॑दि॒त्या आ॑दि॒त्या वै वा आ॑दि॒त्या अन्न॒ मन्न॑ मादि॒त्या वै वा आ॑दि॒त्या अन्न᳚म् । \newline
41. आ॒दि॒त्या अन्न॒ मन्न॑ मादि॒त्या आ॑दि॒त्या अन्न॑म् म॒रुतो॑ म॒रुतो ऽन्न॑ मादि॒त्या आ॑दि॒त्या अन्न॑म् म॒रुतः॑ । \newline
42. अन्न॑म् म॒रुतो॑ म॒रुतो ऽन्न॒ मन्न॑म् म॒रुतो ऽन्न॒ मन्न॑म् म॒रुतो ऽन्न॒ मन्न॑म् म॒रुतो ऽन्न᳚म् । \newline
43. म॒रुतो ऽन्न॒ मन्न॑म् म॒रुतो॑ म॒रुतो ऽन्न॒म् गर्भा॒ गर्भा॒ अन्न॑म् म॒रुतो॑ म॒रुतो ऽन्न॒म् गर्भाः᳚ । \newline
44. अन्न॒म् गर्भा॒ गर्भा॒ अन्न॒ मन्न॒म् गर्भा॒ अन्न॒ मन्न॒म् गर्भा॒ अन्न॒ मन्न॒म् गर्भा॒ अन्न᳚म् । \newline
45. गर्भा॒ अन्न॒ मन्न॒म् गर्भा॒ गर्भा॒ अन्न॑म् पञ्चविꣳ॒॒शः प॑ञ्चविꣳ॒॒शो ऽन्न॒म् गर्भा॒ गर्भा॒ अन्न॑म् पञ्चविꣳ॒॒शः । \newline
46. अन्न॑म् पञ्चविꣳ॒॒शः प॑ञ्चविꣳ॒॒शो ऽन्न॒ मन्न॑म् पञ्चविꣳ॒॒शो ऽन्न॒ मन्न॑म् पञ्चविꣳ॒॒शो ऽन्न॒ मन्न॑म् पञ्चविꣳ॒॒शो ऽन्न᳚म् । \newline
47. प॒ञ्च॒विꣳ॒॒शो ऽन्न॒ मन्न॑म् पञ्चविꣳ॒॒शः प॑ञ्चविꣳ॒॒शो ऽन्न॑ मे॒वै वान्न॑म् पञ्चविꣳ॒॒शः प॑ञ्चविꣳ॒॒शो ऽन्न॑मे॒व । \newline
48. प॒ञ्च॒विꣳ॒॒श इति॑ पञ्च - विꣳ॒॒शः । \newline
49. अन्न॑ मे॒वै वान्न॒ मन्न॑ मे॒व द॑क्षिण॒तो द॑क्षिण॒त ए॒वान्न॒ मन्न॑ मे॒व द॑क्षिण॒तः । \newline
50. ए॒व द॑क्षिण॒तो द॑क्षिण॒त ए॒वैव द॑क्षिण॒तो ध॑त्ते धत्ते दक्षिण॒त ए॒वैव द॑क्षिण॒तो ध॑त्ते । \newline
51. द॒क्षि॒ण॒तो ध॑त्ते धत्ते दक्षिण॒तो द॑क्षिण॒तो ध॑त्ते॒ तस्मा॒त् तस्मा᳚द् धत्ते दक्षिण॒तो द॑क्षिण॒तो ध॑त्ते॒ तस्मा᳚त् । \newline
52. ध॒त्ते॒ तस्मा॒त् तस्मा᳚द् धत्ते धत्ते॒ तस्मा॒द् दक्षि॑णेन॒ दक्षि॑णेन॒ तस्मा᳚द् धत्ते धत्ते॒ तस्मा॒द् दक्षि॑णेन । \newline
53. तस्मा॒द् दक्षि॑णेन॒ दक्षि॑णेन॒ तस्मा॒त् तस्मा॒द् दक्षि॑णे॒ नान्न॒ मन्न॒म् दक्षि॑णेन॒ तस्मा॒त् तस्मा॒द् दक्षि॑णे॒ नान्न᳚म् । \newline
54. दक्षि॑णे॒ नान्न॒ मन्न॒म् दक्षि॑णेन॒ दक्षि॑णे॒ नान्न॑ मद्यते ऽद्य॒ते ऽन्न॒म् दक्षि॑णेन॒ दक्षि॑णे॒ नान्न॑ मद्यते । \newline
55. अन्न॑ मद्यते ऽद्य॒ते ऽन्न॒ मन्न॑ मद्य॒ते ऽदि॑त्या॒ अदि॑त्या अद्य॒ते ऽन्न॒ मन्न॑ मद्य॒ते ऽदि॑त्यै । \newline
56. अ॒द्य॒ते ऽदि॑त्या॒ अदि॑त्या अद्यते ऽद्य॒ते ऽदि॑त्यै भा॒गो भा॒गो ऽदि॑त्या अद्यते ऽद्य॒ते ऽदि॑त्यै भा॒गः । \newline
57. अदि॑त्यै भा॒गो भा॒गो ऽदि॑त्या॒ अदि॑त्यै भा॒गो᳚ ऽस्यसि भा॒गो ऽदि॑त्या॒ अदि॑त्यै भा॒गो॑ ऽसि । \newline
58. भा॒गो᳚ ऽस्यसि भा॒गो भा॒गो॑ ऽसीती त्य॑सि भा॒गो भा॒गो॑ ऽसीति॑ । \newline
\pagebreak
\markright{ TS 5.3.4.4  \hfill https://www.vedavms.in \hfill}

\section{ TS 5.3.4.4 }

\textbf{TS 5.3.4.4 } \newline
\textbf{Samhita Paata} \newline

ऽसीति॑ प॒श्चात् प्र॑ति॒ष्ठा वा अदि॑तिः प्रति॒ष्ठा पू॒षा प्र॑ति॒ष्ठा त्रि॑ण॒वः प्रति॑ष्ठित्यै दे॒वस्य॑ सवि॒तुर्भा॒गो॑ऽ सीत्यु॑त्तर॒तो ब्रह्म॒ वै दे॒वः स॑वि॒ता ब्रह्म॒ बृह॒स्पति॒र्ब्रह्म॑ चतुष्टो॒मो ब्र॑ह्मवर्च॒समे॒वोत्त॑र॒तो ध॑त्ते॒ तस्मा॒दुत्त॒रोऽर्द्धो᳚ ब्रह्मवर्च॒सित॑रः सावि॒त्रव॑ती भवति॒ प्रसू᳚त्यै॒ तस्मा᳚द् ब्राह्म॒णाना॒मुदी॑ची स॒निः प्रसू॑ता ध॒र्त्रश्च॑तुष्टो॒म इति॑ पु॒रस्ता॒दुप॑ दधाति यज्ञ्मु॒खं ॅवै ध॒र्त्रो - [  ] \newline

\textbf{Pada Paata} \newline

अ॒सि॒ । इति॑ । प॒श्चात् । प्र॒ति॒ष्ठेति॑ प्रति - स्था । वै । अदि॑तिः । प्र॒ति॒ष्ठेति॑ प्रति - स्था । पू॒षा । प्र॒ति॒ष्ठेति॑ प्रति - स्था । त्रि॒ण॒व इति॑ त्रि - न॒वः । प्रति॑ष्ठित्या॒ इति॒ प्रति॑-स्थि॒त्यै॒ । दे॒वस्य॑ । स॒वि॒तुः । भा॒गः । अ॒सि॒ । इति॑ । उ॒त्त॒र॒त इत्यु॑त् - त॒र॒तः । ब्रह्म॑ । वै । दे॒वः । स॒वि॒ता । ब्रह्म॑ । बृह॒स्पतिः॑ । ब्रह्म॑ । च॒तु॒ष्टो॒म इति॑ चतुः - स्तो॒मः । ब्र॒ह्म॒व॒र्च॒समिति॑ ब्रह्म - व॒र्च॒सम् । ए॒व । उ॒त्त॒र॒त इत्यु॑त् - त॒र॒तः । ध॒त्ते॒ । तस्मा᳚त् । उत्त॑र॒ इत्युत् - त॒रः॒ । अद्‌र्धः॑ । ब्र॒ह्म॒व॒र्च॒सित॑र॒ इति॑ ब्रह्मवर्च॒सि - त॒रः॒ । सा॒वि॒त्रव॒तीति॑ सावि॒त्र - व॒ती॒ । भ॒व॒ति॒ । प्रसू᳚त्या॒ इति॒ प्र - सू॒त्यै॒ । तस्मा᳚त् । ब्रा॒ह्म॒णाना᳚म् । उदी॑ची । स॒निः । प्रसू॒तेति॒ प्र - सू॒ता॒ । ध॒र्त्रः । च॒तु॒ष्टो॒म इति॑ चतुः - स्तो॒मः । इति॑ । पु॒रस्ता᳚त् । उपेति॑ । द॒धा॒ति॒ । य॒ज्ञ्॒मु॒खमिति॑ यज्ञ् - मु॒खम् । वै । ध॒र्त्रः ।  \newline


\textbf{Krama Paata} \newline

अ॒सीति॑ । इति॑ प॒श्चात् । प॒श्चात् प्र॑ति॒ष्ठा । प्र॒ति॒ष्ठा वै । प्र॒ति॒ष्ठेति॑ प्रति - स्था । वा अदि॑तिः । अदि॑तिः प्रति॒ष्ठा । प्र॒ति॒ष्ठा पू॒षा । प्र॒ति॒ष्ठेति॑ प्रति - स्था । पू॒षा प्र॑ति॒ष्ठा । प्र॒ति॒ष्ठा त्रि॑ण॒वः । प्र॒ति॒ष्ठेति॑ प्रति - स्था । त्रि॒ण॒वः प्रति॑ष्ठित्यै । त्रि॒ण॒व इति॑ त्रि - न॒वः । प्रति॑ष्ठित्यै दे॒वस्य॑ । प्रति॑ष्ठित्या॒ इति॒ प्रति॑ - स्थि॒त्यै॒ । दे॒वस्य॑ सवि॒तुः । स॒वि॒तुर् भा॒गः । भा॒गो॑ऽसि । अ॒सीति॑ । इत्यु॑त्तर॒तः । उ॒त्त॒र॒तो ब्रह्म॑ । उ॒त्त॒र॒त इत्यु॑त् - त॒र॒तः । ब्रह्म॒ वै । वै दे॒वः । दे॒वः स॑वि॒ता । स॒वि॒ता ब्रह्म॑ । ब्रह्म॒ बृह॒स्पतिः॑ । बृह॒स्पति॒र् ब्रह्म॑ । ब्रह्म॑ चतुष्टो॒मः । च॒तु॒ष्टो॒मो ब्र॑ह्मवर्च॒सम् । च॒तु॒ष्टो॒म इति॑ चतुः - स्तो॒मः । ब्र॒ह्म॒व॒र्च॒समे॒व । ब्र॒ह्म॒व॒र्च॒समिति॑ ब्रह्म - व॒र्च॒सम् । ए॒वोत्त॑र॒तः । उ॒त्त॒र॒तो ध॑त्ते । उ॒त्त॒र॒त इत्यु॑त् - त॒र॒तः । ध॒त्ते॒ तस्मा᳚त् । तस्मा॒दुत्त॑रः । उत्त॒रोऽर्द्धः॑ । उत्त॑र॒ इत्युत् - त॒रः॒ । अर्द्धो᳚ ब्रह्मवर्च॒सित॑रः । ब्र॒ह्म॒व॒र्च॒सित॑रः सावि॒त्रव॑ती । ब्र॒ह्म॒व॒र्च॒सित॑र॒ इति॑ ब्रह्मवर्च॒सि - त॒रः॒ । सा॒वि॒त्रव॑ती भवति । सा॒वि॒त्रव॒तीति॑ सावि॒त्र - व॒ती॒ । भ॒व॒ति॒ प्रसू᳚त्यै । प्रसू᳚त्यै॒ तस्मा᳚त् । प्रसू᳚त्या॒ इति॒ प्र - सू॒त्यै॒ । तस्मा᳚द् ब्राह्म॒णाना᳚म् । ब्रा॒ह्म॒णाना॒मुदी॑ची । उदी॑ची स॒निः । स॒निः प्रसू॑ता । प्रसू॑ता ध॒र्त्रः । प्रसू॒तेति॒ प्र - सू॒ता॒ । ध॒र्त्रश्च॑तुष्टो॒मः । च॒तु॒ष्टो॒म इति॑ । च॒तु॒ष्टो॒म इति॑ चतुः - स्तो॒मः । इति॑ पु॒रस्ता᳚त् । पु॒रस्ता॒दुप॑ । उप॑ दधाति । द॒धा॒ति॒ य॒ज्ञ्॒मु॒खम् । य॒ज्ञ्॒मु॒खम् ॅवै । य॒ज्ञ्॒मु॒खमिति॑ यज्ञ् - मु॒खम् । वै ध॒र्त्रः । ध॒र्त्रो य॑ज्ञ्मु॒खम् \newline

\textbf{Jatai Paata} \newline

1. अ॒सीती त्य॑स्य॒ सीति॑ । \newline
2. इति॑ प॒श्चात् प॒श्चा दितीति॑ प॒श्चात् । \newline
3. प॒श्चात् प्र॑ति॒ष्ठा प्र॑ति॒ष्ठा प॒श्चात् प॒श्चात् प्र॑ति॒ष्ठा । \newline
4. प्र॒ति॒ष्ठा वै वै प्र॑ति॒ष्ठा प्र॑ति॒ष्ठा वै । \newline
5. प्र॒ति॒ष्ठेति॑ प्रति - स्था । \newline
6. वा अदि॑ति॒ रदि॑ति॒र् वै वा अदि॑तिः । \newline
7. अदि॑तिः प्रति॒ष्ठा प्र॑ति॒ष्ठा ऽदि॑ति॒ रदि॑तिः प्रति॒ष्ठा । \newline
8. प्र॒ति॒ष्ठा पू॒षा पू॒षा प्र॑ति॒ष्ठा प्र॑ति॒ष्ठा पू॒षा । \newline
9. प्र॒ति॒ष्ठेति॑ प्रति - स्था । \newline
10. पू॒षा प्र॑ति॒ष्ठा प्र॑ति॒ष्ठा पू॒षा पू॒षा प्र॑ति॒ष्ठा । \newline
11. प्र॒ति॒ष्ठा त्रि॑ण॒व स्त्रि॑ण॒वः प्र॑ति॒ष्ठा प्र॑ति॒ष्ठा त्रि॑ण॒वः । \newline
12. प्र॒ति॒ष्ठेति॑ प्रति - स्था । \newline
13. त्रि॒ण॒वः प्रति॑ष्ठित्यै॒ प्रति॑ष्ठित्यै त्रिण॒व स्त्रि॑ण॒वः प्रति॑ष्ठित्यै । \newline
14. त्रि॒ण॒व इति॑ त्रि - न॒वः । \newline
15. प्रति॑ष्ठित्यै दे॒वस्य॑ दे॒वस्य॒ प्रति॑ष्ठित्यै॒ प्रति॑ष्ठित्यै दे॒वस्य॑ । \newline
16. प्रति॑ष्ठित्या॒ इति॒ प्रति॑ - स्थि॒त्यै॒ । \newline
17. दे॒वस्य॑ सवि॒तुः स॑वि॒तुर् दे॒वस्य॑ दे॒वस्य॑ सवि॒तुः । \newline
18. स॒वि॒तुर् भा॒गो भा॒गः स॑वि॒तुः स॑वि॒तुर् भा॒गः । \newline
19. भा॒गो᳚ ऽस्यसि भा॒गो भा॒गो॑ ऽसि । \newline
20. अ॒सीती त्य॑स्य॒ सीति॑ । \newline
21. इत्यु॑त्तर॒त उ॑त्तर॒त इती त्यु॑त्तर॒तः । \newline
22. उ॒त्त॒र॒तो ब्रह्म॒ ब्रह्मो᳚ त्तर॒त उ॑त्तर॒तो ब्रह्म॑ । \newline
23. उ॒त्त॒र॒त इत्यु॑त् - त॒र॒तः । \newline
24. ब्रह्म॒ वै वै ब्रह्म॒ ब्रह्म॒ वै । \newline
25. वै दे॒वो दे॒वो वै वै दे॒वः । \newline
26. दे॒वः स॑वि॒ता स॑वि॒ता दे॒वो दे॒वः स॑वि॒ता । \newline
27. स॒वि॒ता ब्रह्म॒ ब्रह्म॑ सवि॒ता स॑वि॒ता ब्रह्म॑ । \newline
28. ब्रह्म॒ बृह॒स्पति॒र् बृह॒स्पति॒र् ब्रह्म॒ ब्रह्म॒ बृह॒स्पतिः॑ । \newline
29. बृह॒स्पति॒र् ब्रह्म॒ ब्रह्म॒ बृह॒स्पति॒र् बृह॒स्पति॒र् ब्रह्म॑ । \newline
30. ब्रह्म॑ चतुष्टो॒म श्च॑तुष्टो॒मो ब्रह्म॒ ब्रह्म॑ चतुष्टो॒मः । \newline
31. च॒तु॒ष्टो॒मो ब्र॑ह्मवर्च॒सम् ब्र॑ह्मवर्च॒सम् च॑तुष्टो॒म श्च॑तुष्टो॒मो ब्र॑ह्मवर्च॒सम् । \newline
32. च॒तु॒ष्टो॒म इति॑ चतुः - स्तो॒मः । \newline
33. ब्र॒ह्म॒व॒र्च॒स मे॒वैव ब्र॑ह्मवर्च॒सम् ब्र॑ह्मवर्च॒स मे॒व । \newline
34. ब्र॒ह्म॒व॒र्च॒समिति॑ ब्रह्म - व॒र्च॒सम् । \newline
35. ए॒वोत्त॑र॒त उ॑त्तर॒त ए॒वैवो त्त॑र॒तः । \newline
36. उ॒त्त॒र॒तो ध॑त्ते धत्त उत्तर॒त उ॑त्तर॒तो ध॑त्ते । \newline
37. उ॒त्त॒र॒त इत्यु॑त् - त॒र॒तः । \newline
38. ध॒त्ते॒ तस्मा॒त् तस्मा᳚द् धत्ते धत्ते॒ तस्मा᳚त् । \newline
39. तस्मा॒ दुत्त॑र॒ उत्त॑र॒ स्तस्मा॒त् तस्मा॒ दुत्त॑रः । \newline
40. उत्त॒रो ऽर्द्धो ऽर्द्ध॒ उत्त॑र॒ उत्त॒रो ऽर्द्धः॑ । \newline
41. उत्त॑र॒ इत्युत् - त॒रः॒ । \newline
42. अर्द्धो᳚ ब्रह्मवर्च॒सित॑रो ब्रह्मवर्च॒सित॒रो ऽर्द्धो ऽर्द्धो᳚ ब्रह्मवर्च॒सित॑रः । \newline
43. ब्र॒ह्म॒व॒र्च॒सित॑रः सावि॒त्रव॑ती सावि॒त्रव॑ती ब्रह्मवर्च॒सित॑रो ब्रह्मवर्च॒सित॑रः सावि॒त्रव॑ती । \newline
44. ब्र॒ह्म॒व॒र्च॒सित॑र॒ इति॑ ब्रह्मवर्च॒सि - त॒रः॒ । \newline
45. सा॒वि॒त्रव॑ती भवति भवति सावि॒त्रव॑ती सावि॒त्रव॑ती भवति । \newline
46. सा॒वि॒त्रव॒तीति॑ सावि॒त्र - व॒ती॒ । \newline
47. भ॒व॒ति॒ प्रसू᳚त्यै॒ प्रसू᳚त्यै भवति भवति॒ प्रसू᳚त्यै । \newline
48. प्रसू᳚त्यै॒ तस्मा॒त् तस्मा॒त् प्रसू᳚त्यै॒ प्रसू᳚त्यै॒ तस्मा᳚त् । \newline
49. प्रसू᳚त्या॒ इति॒ प्र - सू॒त्यै॒ । \newline
50. तस्मा᳚द् ब्राह्म॒णाना᳚म् ब्राह्म॒णाना॒म् तस्मा॒त् तस्मा᳚द् ब्राह्म॒णाना᳚म् । \newline
51. ब्रा॒ह्म॒णाना॒ मुदी॒ च्युदी॑ची ब्राह्म॒णाना᳚म् ब्राह्म॒णाना॒ मुदी॑ची । \newline
52. उदी॑ची स॒निः स॒नि रुदी॒ च्युदी॑ची स॒निः । \newline
53. स॒निः प्रसू॑ता॒ प्रसू॑ता स॒निः स॒निः प्रसू॑ता । \newline
54. प्रसू॑ता ध॒र्त्रो ध॒र्त्रः प्रसू॑ता॒ प्रसू॑ता ध॒र्त्रः । \newline
55. प्रसू॒तेति॒ प्र - सू॒ता॒ । \newline
56. ध॒र्त्र श्च॑तुष्टो॒म श्च॑तुष्टो॒मो ध॒र्त्रो ध॒र्त्र श्च॑तुष्टो॒मः । \newline
57. च॒तु॒ष्टो॒म इतीति॑ चतुष्टो॒म श्च॑तुष्टो॒म इति॑ । \newline
58. च॒तु॒ष्टो॒म इति॑ चतुः - स्तो॒मः । \newline
59. इति॑ पु॒रस्ता᳚त् पु॒रस्ता॒ दितीति॑ पु॒रस्ता᳚त् । \newline
60. पु॒रस्ता॒ दुपोप॑ पु॒रस्ता᳚त् पु॒रस्ता॒ दुप॑ । \newline
61. उप॑ दधाति दधा॒ त्युपोप॑ दधाति । \newline
62. द॒धा॒ति॒ य॒ज्ञ्॒मु॒खं ॅय॑ज्ञ्मु॒खम् द॑धाति दधाति यज्ञ्मु॒खम् । \newline
63. य॒ज्ञ्॒मु॒खं ॅवै वै य॑ज्ञ्मु॒खं ॅय॑ज्ञ्मु॒खं ॅवै । \newline
64. य॒ज्ञ्॒मु॒खमिति॑ यज्ञ् - मु॒खम् । \newline
65. वै ध॒र्त्रो ध॒र्त्रो वै वै ध॒र्त्रः । \newline
66. ध॒र्त्रो य॑ज्ञ्मु॒खं ॅय॑ज्ञ्मु॒खम् ध॒र्त्रो ध॒र्त्रो य॑ज्ञ्मु॒खम् । \newline

\textbf{Ghana Paata } \newline

1. अ॒सीती त्य॑स्य॒ सीति॑ प॒श्चात् प॒श्चा दित्य॑स्य॒ सीति॑ प॒श्चात् । \newline
2. इति॑ प॒श्चात् प॒श्चा दितीति॑ प॒श्चात् प्र॑ति॒ष्ठा प्र॑ति॒ष्ठा प॒श्चा दितीति॑ प॒श्चात् प्र॑ति॒ष्ठा । \newline
3. प॒श्चात् प्र॑ति॒ष्ठा प्र॑ति॒ष्ठा प॒श्चात् प॒श्चात् प्र॑ति॒ष्ठा वै वै प्र॑ति॒ष्ठा प॒श्चात् प॒श्चात् प्र॑ति॒ष्ठा वै । \newline
4. प्र॒ति॒ष्ठा वै वै प्र॑ति॒ष्ठा प्र॑ति॒ष्ठा वा अदि॑ति॒ रदि॑ति॒र् वै प्र॑ति॒ष्ठा प्र॑ति॒ष्ठा वा अदि॑तिः । \newline
5. प्र॒ति॒ष्ठेति॑ प्रति - स्था । \newline
6. वा अदि॑ति॒ रदि॑ति॒र् वै वा अदि॑तिः प्रति॒ष्ठा प्र॑ति॒ष्ठा ऽदि॑ति॒र् वै वा अदि॑तिः प्रति॒ष्ठा । \newline
7. अदि॑तिः प्रति॒ष्ठा प्र॑ति॒ष्ठा ऽदि॑ति॒ रदि॑तिः प्रति॒ष्ठा पू॒षा पू॒षा प्र॑ति॒ष्ठा ऽदि॑ति॒ रदि॑तिः प्रति॒ष्ठा पू॒षा । \newline
8. प्र॒ति॒ष्ठा पू॒षा पू॒षा प्र॑ति॒ष्ठा प्र॑ति॒ष्ठा पू॒षा प्र॑ति॒ष्ठा प्र॑ति॒ष्ठा पू॒षा प्र॑ति॒ष्ठा प्र॑ति॒ष्ठा पू॒षा प्र॑ति॒ष्ठा । \newline
9. प्र॒ति॒ष्ठेति॑ प्रति - स्था । \newline
10. पू॒षा प्र॑ति॒ष्ठा प्र॑ति॒ष्ठा पू॒षा पू॒षा प्र॑ति॒ष्ठा त्रि॑ण॒व स्त्रि॑ण॒वः प्र॑ति॒ष्ठा पू॒षा पू॒षा प्र॑ति॒ष्ठा त्रि॑ण॒वः । \newline
11. प्र॒ति॒ष्ठा त्रि॑ण॒व स्त्रि॑ण॒वः प्र॑ति॒ष्ठा प्र॑ति॒ष्ठा त्रि॑ण॒वः प्रति॑ष्ठित्यै॒ प्रति॑ष्ठित्यै त्रिण॒वः प्र॑ति॒ष्ठा प्र॑ति॒ष्ठा त्रि॑ण॒वः प्रति॑ष्ठित्यै । \newline
12. प्र॒ति॒ष्ठेति॑ प्रति - स्था । \newline
13. त्रि॒ण॒वः प्रति॑ष्ठित्यै॒ प्रति॑ष्ठित्यै त्रिण॒व स्त्रि॑ण॒वः प्रति॑ष्ठित्यै दे॒वस्य॑ दे॒वस्य॒ प्रति॑ष्ठित्यै त्रिण॒व स्त्रि॑ण॒वः प्रति॑ष्ठित्यै दे॒वस्य॑ । \newline
14. त्रि॒ण॒व इति॑ त्रि - न॒वः । \newline
15. प्रति॑ष्ठित्यै दे॒वस्य॑ दे॒वस्य॒ प्रति॑ष्ठित्यै॒ प्रति॑ष्ठित्यै दे॒वस्य॑ सवि॒तुः स॑वि॒तुर् दे॒वस्य॒ प्रति॑ष्ठित्यै॒ प्रति॑ष्ठित्यै दे॒वस्य॑ सवि॒तुः । \newline
16. प्रति॑ष्ठित्या॒ इति॒ प्रति॑ - स्थि॒त्यै॒ । \newline
17. दे॒वस्य॑ सवि॒तुः स॑वि॒तुर् दे॒वस्य॑ दे॒वस्य॑ सवि॒तुर् भा॒गो भा॒गः स॑वि॒तुर् दे॒वस्य॑ दे॒वस्य॑ सवि॒तुर् भा॒गः । \newline
18. स॒वि॒तुर् भा॒गो भा॒गः स॑वि॒तुः स॑वि॒तुर् भा॒गो᳚ ऽस्यसि भा॒गः स॑वि॒तुः स॑वि॒तुर् भा॒गो॑ ऽसि । \newline
19. भा॒गो᳚ ऽस्यसि भा॒गो भा॒गो॑ ऽसीतीत्य॑सि भा॒गो भा॒गो॑ ऽसीति॑ । \newline
20. अ॒सीती त्य॑स्य॒सी त्यु॑त्तर॒त उ॑त्तर॒त इत्य॑स्य॒सी त्यु॑त्तर॒तः । \newline
21. इत्यु॑त्तर॒त उ॑त्तर॒त इती त्यु॑त्तर॒तो ब्रह्म॒ ब्रह्मो᳚ त्तर॒त इती त्यु॑त्तर॒तो ब्रह्म॑ । \newline
22. उ॒त्त॒र॒तो ब्रह्म॒ ब्रह्मो᳚ त्तर॒त उ॑त्तर॒तो ब्रह्म॒ वै वै ब्रह्मो᳚ त्तर॒त उ॑त्तर॒तो ब्रह्म॒ वै । \newline
23. उ॒त्त॒र॒त इत्यु॑त् - त॒र॒तः । \newline
24. ब्रह्म॒ वै वै ब्रह्म॒ ब्रह्म॒ वै दे॒वो दे॒वो वै ब्रह्म॒ ब्रह्म॒ वै दे॒वः । \newline
25. वै दे॒वो दे॒वो वै वै दे॒वः स॑वि॒ता स॑वि॒ता दे॒वो वै वै दे॒वः स॑वि॒ता । \newline
26. दे॒वः स॑वि॒ता स॑वि॒ता दे॒वो दे॒वः स॑वि॒ता ब्रह्म॒ ब्रह्म॑ सवि॒ता दे॒वो दे॒वः स॑वि॒ता ब्रह्म॑ । \newline
27. स॒वि॒ता ब्रह्म॒ ब्रह्म॑ सवि॒ता स॑वि॒ता ब्रह्म॒ बृह॒स्पति॒र् बृह॒स्पति॒र् ब्रह्म॑ सवि॒ता स॑वि॒ता ब्रह्म॒ बृह॒स्पतिः॑ । \newline
28. ब्रह्म॒ बृह॒स्पति॒र् बृह॒स्पति॒र् ब्रह्म॒ ब्रह्म॒ बृह॒स्पति॒र् ब्रह्म॒ ब्रह्म॒ बृह॒स्पति॒र् ब्रह्म॒ ब्रह्म॒ बृह॒स्पति॒र् ब्रह्म॑ । \newline
29. बृह॒स्पति॒र् ब्रह्म॒ ब्रह्म॒ बृह॒स्पति॒र् बृह॒स्पति॒र् ब्रह्म॑ चतुष्टो॒म श्च॑तुष्टो॒मो ब्रह्म॒ बृह॒स्पति॒र् बृह॒स्पति॒र् ब्रह्म॑ चतुष्टो॒मः । \newline
30. ब्रह्म॑ चतुष्टो॒म श्च॑तुष्टो॒मो ब्रह्म॒ ब्रह्म॑ चतुष्टो॒मो ब्र॑ह्मवर्च॒सम् ब्र॑ह्मवर्च॒सम् च॑तुष्टो॒मो ब्रह्म॒ ब्रह्म॑ चतुष्टो॒मो ब्र॑ह्मवर्च॒सम् । \newline
31. च॒तु॒ष्टो॒मो ब्र॑ह्मवर्च॒सम् ब्र॑ह्मवर्च॒सम् च॑तुष्टो॒म श्च॑तुष्टो॒मो ब्र॑ह्मवर्च॒स मे॒वैव ब्र॑ह्मवर्च॒सम् च॑तुष्टो॒म श्च॑तुष्टो॒मो ब्र॑ह्मवर्च॒स मे॒व । \newline
32. च॒तु॒ष्टो॒म इति॑ चतुः - स्तो॒मः । \newline
33. ब्र॒ह्म॒व॒र्च॒स मे॒वैव ब्र॑ह्मवर्च॒सम् ब्र॑ह्मवर्च॒स मे॒वोत्त॑र॒त उ॑त्तर॒त ए॒व ब्र॑ह्मवर्च॒सम् ब्र॑ह्मवर्च॒स मे॒वोत्त॑र॒तः । \newline
34. ब्र॒ह्म॒व॒र्च॒समिति॑ ब्रह्म - व॒र्च॒सम् । \newline
35. ए॒वोत्त॑र॒त उ॑त्तर॒त ए॒वै वोत्त॑र॒तो ध॑त्ते धत्त उत्तर॒त ए॒वै वोत्त॑र॒तो ध॑त्ते । \newline
36. उ॒त्त॒र॒तो ध॑त्ते धत्त उत्तर॒त उ॑त्तर॒तो ध॑त्ते॒ तस्मा॒त् तस्मा᳚द् धत्त उत्तर॒त उ॑त्तर॒तो ध॑त्ते॒ तस्मा᳚त् । \newline
37. उ॒त्त॒र॒त इत्यु॑त् - त॒र॒तः । \newline
38. ध॒त्ते॒ तस्मा॒त् तस्मा᳚द् धत्ते धत्ते॒ तस्मा॒ दुत्त॑र॒ उत्त॑र॒ स्तस्मा᳚द् धत्ते धत्ते॒ तस्मा॒ दुत्त॑रः । \newline
39. तस्मा॒ दुत्त॑र॒ उत्त॑र॒ स्तस्मा॒त् तस्मा॒ दुत्त॒रो ऽर्द्धो ऽर्द्ध॒ उत्त॑र॒ स्तस्मा॒त् तस्मा॒ दुत्त॒रो ऽर्द्धः॑ । \newline
40. उत्त॒रो ऽर्द्धो ऽर्द्ध॒ उत्त॑र॒ उत्त॒रो ऽर्द्धो᳚ ब्रह्मवर्च॒सित॑रो ब्रह्मवर्च॒सित॒रो ऽर्द्ध॒ उत्त॑र॒ उत्त॒रो ऽर्द्धो᳚ ब्रह्मवर्च॒सित॑रः । \newline
41. उत्त॑र॒ इत्युत् - त॒रः॒ । \newline
42. अर्द्धो᳚ ब्रह्मवर्च॒सित॑रो ब्रह्मवर्च॒सित॒रो ऽर्द्धो ऽर्द्धो᳚ ब्रह्मवर्च॒सित॑रः सावि॒त्रव॑ती सावि॒त्रव॑ती ब्रह्मवर्च॒सित॒रो ऽर्द्धो ऽर्द्धो᳚ ब्रह्मवर्च॒सित॑रः सावि॒त्रव॑ती । \newline
43. ब्र॒ह्म॒व॒र्च॒सित॑रः सावि॒त्रव॑ती सावि॒त्रव॑ती ब्रह्मवर्च॒सित॑रो ब्रह्मवर्च॒सित॑रः सावि॒त्रव॑ती भवति भवति सावि॒त्रव॑ती ब्रह्मवर्च॒सित॑रो ब्रह्मवर्च॒सित॑रः सावि॒त्रव॑ती भवति । \newline
44. ब्र॒ह्म॒व॒र्च॒सित॑र॒ इति॑ ब्रह्मवर्च॒सि - त॒रः॒ । \newline
45. सा॒वि॒त्रव॑ती भवति भवति सावि॒त्रव॑ती सावि॒त्रव॑ती भवति॒ प्रसू᳚त्यै॒ प्रसू᳚त्यै भवति सावि॒त्रव॑ती सावि॒त्रव॑ती भवति॒ प्रसू᳚त्यै । \newline
46. सा॒वि॒त्रव॒तीति॑ सावि॒त्र - व॒ती॒ । \newline
47. भ॒व॒ति॒ प्रसू᳚त्यै॒ प्रसू᳚त्यै भवति भवति॒ प्रसू᳚त्यै॒ तस्मा॒त् तस्मा॒त् प्रसू᳚त्यै भवति भवति॒ प्रसू᳚त्यै॒ तस्मा᳚त् । \newline
48. प्रसू᳚त्यै॒ तस्मा॒त् तस्मा॒त् प्रसू᳚त्यै॒ प्रसू᳚त्यै॒ तस्मा᳚द् ब्राह्म॒णाना᳚म् ब्राह्म॒णाना॒म् तस्मा॒त् प्रसू᳚त्यै॒ प्रसू᳚त्यै॒ तस्मा᳚द् ब्राह्म॒णाना᳚म् । \newline
49. प्रसू᳚त्या॒ इति॒ प्र - सू॒त्यै॒ । \newline
50. तस्मा᳚द् ब्राह्म॒णाना᳚म् ब्राह्म॒णाना॒म् तस्मा॒त् तस्मा᳚द् ब्राह्म॒णाना॒ मुदी॒ च्युदी॑ची ब्राह्म॒णाना॒म् तस्मा॒त् तस्मा᳚द् ब्राह्म॒णाना॒ मुदी॑ची । \newline
51. ब्रा॒ह्म॒णाना॒ मुदी॒ च्युदी॑ची ब्राह्म॒णाना᳚म् ब्राह्म॒णाना॒ मुदी॑ची स॒निः स॒नि रुदी॑ची ब्राह्म॒णाना᳚म् ब्राह्म॒णाना॒ मुदी॑ची स॒निः । \newline
52. उदी॑ची स॒निः स॒नि रुदी॒ च्युदी॑ची स॒निः प्रसू॑ता॒ प्रसू॑ता स॒नि रुदी॒ च्युदी॑ची स॒निः प्रसू॑ता । \newline
53. स॒निः प्रसू॑ता॒ प्रसू॑ता स॒निः स॒निः प्रसू॑ता ध॒र्त्रो ध॒र्त्रः प्रसू॑ता स॒निः स॒निः प्रसू॑ता ध॒र्त्रः । \newline
54. प्रसू॑ता ध॒र्त्रो ध॒र्त्रः प्रसू॑ता॒ प्रसू॑ता ध॒र्त्र श्च॑तुष्टो॒म श्च॑तुष्टो॒मो ध॒र्त्रः प्रसू॑ता॒ प्रसू॑ता ध॒र्त्र श्च॑तुष्टो॒मः । \newline
55. प्रसू॒तेति॒ प्र - सू॒ता॒ । \newline
56. ध॒र्त्र श्च॑तुष्टो॒म श्च॑तुष्टो॒मो ध॒र्त्रो ध॒र्त्र श्च॑तुष्टो॒म इतीति॑ चतुष्टो॒मो ध॒र्त्रो ध॒र्त्र श्च॑तुष्टो॒म इति॑ । \newline
57. च॒तु॒ष्टो॒म इतीति॑ चतुष्टो॒म श्च॑तुष्टो॒म इति॑ पु॒रस्ता᳚त् पु॒रस्ता॒ दिति॑ चतुष्टो॒म श्च॑तुष्टो॒म इति॑ पु॒रस्ता᳚त् । \newline
58. च॒तु॒ष्टो॒म इति॑ चतुः - स्तो॒मः । \newline
59. इति॑ पु॒रस्ता᳚त् पु॒रस्ता॒ दितीति॑ पु॒रस्ता॒ दुपोप॑ पु॒रस्ता॒ दितीति॑ पु॒रस्ता॒ दुप॑ । \newline
60. पु॒रस्ता॒ दुपोप॑ पु॒रस्ता᳚त् पु॒रस्ता॒ दुप॑ दधाति दधा॒ त्युप॑ पु॒रस्ता᳚त् पु॒रस्ता॒ दुप॑ दधाति । \newline
61. उप॑ दधाति दधा॒ त्युपोप॑ दधाति यज्ञ्मु॒खं ॅय॑ज्ञ्मु॒खम् द॑धा॒ त्युपोप॑ दधाति यज्ञ्मु॒खम् । \newline
62. द॒धा॒ति॒ य॒ज्ञ्॒मु॒खं ॅय॑ज्ञ्मु॒खम् द॑धाति दधाति यज्ञ्मु॒खं ॅवै वै य॑ज्ञ्मु॒खम् द॑धाति दधाति यज्ञ्मु॒खं ॅवै । \newline
63. य॒ज्ञ्॒मु॒खं ॅवै वै य॑ज्ञ्मु॒खं ॅय॑ज्ञ्मु॒खं ॅवै ध॒र्त्रो ध॒र्त्रो वै य॑ज्ञ्मु॒खं ॅय॑ज्ञ्मु॒खं ॅवै ध॒र्त्रः । \newline
64. य॒ज्ञ्॒मु॒खमिति॑ यज्ञ् - मु॒खम् । \newline
65. वै ध॒र्त्रो ध॒र्त्रो वै वै ध॒र्त्रो य॑ज्ञ्मु॒खं ॅय॑ज्ञ्मु॒खम् ध॒र्त्रो वै वै ध॒र्त्रो य॑ज्ञ्मु॒खम् । \newline
66. ध॒र्त्रो य॑ज्ञ्मु॒खं ॅय॑ज्ञ्मु॒खम् ध॒र्त्रो ध॒र्त्रो य॑ज्ञ्मु॒खम् च॑तुष्टो॒म श्च॑तुष्टो॒मो य॑ज्ञ्मु॒खम् ध॒र्त्रो ध॒र्त्रो य॑ज्ञ्मु॒खम् च॑तुष्टो॒मः । \newline
\pagebreak
\markright{ TS 5.3.4.5  \hfill https://www.vedavms.in \hfill}

\section{ TS 5.3.4.5 }

\textbf{TS 5.3.4.5 } \newline
\textbf{Samhita Paata} \newline

य॑ज्ञ्मु॒खं च॑तुष्टो॒मो य॑ज्ञ्मु॒खमे॒व पु॒रस्ता॒द्वि या॑तयति॒ यावा॑नां भा॒गो॑ऽसीति॑ दक्षिण॒तो मासा॒ वै यावा॑ अर्द्धमा॒सा अया॑वा॒-स्तस्मा᳚द्-दक्षि॒णावृ॑तो॒ मासा॒ अन्नं॒ ॅवै यावा॒ अन्नं॑ प्र॒जा अन्न॑मे॒व द॑क्षिण॒तो ध॑त्ते॒ तस्मा॒द् दक्षि॑णे॒ना-न्न॑मद्यत ऋभू॒णां भा॒गो॑ऽसीति॑ प॒श्चात् प्रति॑ष्ठित्यै विव॒र्तो᳚ ऽष्टाचत्वारिꣳ॒॒श इत्यु॑त्तर॒तो॑ऽनयो᳚र्लो॒कयोः᳚ सवीर्य॒त्वाय॒ तस्मा॑दि॒मौ लो॒कौ स॒माव॑द्-वीर्यौ॒ - [  ] \newline

\textbf{Pada Paata} \newline

य॒ज्ञ्॒मु॒खमिति॑ यज्ञ् - मु॒खम् । च॒तु॒ष्टो॒म इति॑ चतुः - स्तो॒मः । य॒ज्ञ्॒मु॒खमिति॑ यज्ञ् - मु॒खम् । ए॒व । पु॒रस्ता᳚त् । वीति॑ । या॒त॒य॒ति॒ । यावा॑नाम् । भा॒गः । अ॒सि॒ । इति॑ । द॒क्षि॒ण॒तः । मासाः᳚ । वै । यावाः᳚ । अ॒द्‌र्ध॒मा॒सा इत्य॑द्‌र्ध - मा॒साः । अया॑वाः । तस्मा᳚त् । द॒क्षि॒णावृ॑त॒ इति॑ दक्षि॒णा - आ॒वृ॒तः॒ । मासाः᳚ । अन्न᳚म् । वै । यावाः᳚ । अन्न᳚म् । प्र॒जा इति॑ प्र - जाः । अन्न᳚म् । ए॒व । द॒क्षि॒ण॒तः । ध॒त्ते॒ । तस्मा᳚त् । दक्षि॑णेन । अन्न᳚म् । अ॒द्य॒ते॒ । ऋ॒भू॒णाम् । भा॒गः । अ॒सि॒ । इति॑ । प॒श्चात् । प्रति॑ष्ठित्या॒ इति॒ प्रति॑ - स्थि॒त्यै॒ । वि॒व॒र्त इति॑ वि - व॒र्तः । अ॒ष्टा॒च॒त्वा॒रिꣳ॒॒श इत्य॑ष्टा-च॒त्वा॒रिꣳ॒॒शः । इति॑ । उ॒त्त॒र॒त इत्यु॑त्-त॒र॒तः । अ॒नयोः᳚ । लो॒कयोः᳚ । स॒वी॒र्य॒त्वायेति॑ सवीर्य - त्वाय॑ । तस्मा᳚त् । इ॒मौ । लो॒कौ । स॒माव॑द्वीर्या॒विति॑ स॒माव॑त् - वी॒र्यौ॒ ।  \newline


\textbf{Krama Paata} \newline

य॒ज्ञ्॒मु॒खम् च॑तुष्टो॒मः । य॒ज्ञ्॒मु॒खमिति॑ यज्ञ् - मु॒खम् । च॒तु॒ष्टो॒मो य॑ज्ञ्मु॒खम् । च॒तु॒ष्टो॒म इति॑ चतुः - स्तो॒मः । य॒ज्ञ्॒मु॒खमे॒व । य॒ज्ञ्॒मु॒खमिति॑ यज्ञ् - मु॒खम् । 
ए॒व पु॒रस्ता᳚त् । पु॒रस्ता॒द् वि । वि या॑तयति । या॒त॒य॒ति॒ यावा॑नाम् । यावा॑नाम् भा॒गः । भा॒गो॑ऽसि । अ॒सीति॑ । इति॑ दक्षिण॒तः । द॒क्षि॒ण॒तो मासाः᳚ । मासा॒ वै । वै यावाः᳚ । यावा॑ अर्द्धमा॒साः । अ॒र्द्ध॒मा॒सा अया॑वाः । अ॒र्द्ध॒मा॒सा इत्य॑र्द्ध - मा॒साः । अया॑वा॒स्तस्मा᳚त् । तस्मा᳚द् दक्षि॒णावृ॑तः । द॒क्षि॒णावृ॑तो॒ मासाः᳚ । द॒क्षि॒णावृ॑त॒ इति॑ दक्षि॒णा - आ॒वृ॒तः॒ । मासा॒ अन्न᳚म् । अन्न॒म् ॅवै । वै यावाः᳚ । यावा॒ अन्न᳚म् । अन्न॑म् प्र॒जाः । प्र॒जा अन्न᳚म् । प्र॒जा इति॑ प्र - जाः । अन्न॑मे॒व । ए॒व द॑क्षिण॒तः । द॒क्षि॒ण॒तो ध॑त्ते । ध॒त्ते॒ तस्मा᳚त् । तस्मा॒द् दक्षि॑णेन । दक्षि॑णे॒नान्न᳚म् । अन्न॑मद्यते । अ॒द्य॒त॒ ऋ॒भू॒णाम् । ऋ॒भू॒णाम् भा॒गः । भा॒गो॑ऽसि । अ॒सीति॑ । इति॑ प॒श्चात् । प॒श्चात् प्रति॑ष्ठित्यै । प्रति॑ष्ठित्यै विव॒र्तः । प्रति॑ष्ठित्या॒ इति॒ प्रति॑ - स्थि॒त्यै॒ । वि॒व॒र्तो᳚ऽष्टाचत्वारिꣳ॒॒शः । वि॒व॒र्त इति॑ वि - व॒र्तः । अ॒ष्टा॒च॒त्वा॒रिꣳ॒॒श इति॑ । अ॒ष्टा॒च॒त्वा॒रिꣳ॒॒श इत्य॑ष्टा - च॒त्वा॒रिꣳ॒॒शः । इत्यु॑त्तर॒तः । उ॒त्त॒र॒तो॑ऽनयोः᳚ । उ॒त्त॒र॒त इत्यु॑त् - त॒र॒तः । अ॒नयो᳚र् लो॒कयोः᳚ । लो॒कयोः᳚ सवीर्य॒त्वाय॑ । स॒वी॒र्य॒त्वाय॒ तस्मा᳚त् । स॒वी॒र्य॒त्वायेति॑ सवीर्य - त्वाय॑ । तस्मा॑दि॒मौ । इ॒मौ लो॒कौ । लो॒कौ स॒माव॑द्वीर्यौ । स॒माव॑द्विर्यौ॒ यस्य॑ । स॒माव॑द्वीर्या॒विति॑ स॒माव॑त् - वी॒र्यौ॒ \newline

\textbf{Jatai Paata} \newline

1. य॒ज्ञ्॒मु॒खम् च॑तुष्टो॒म श्च॑तुष्टो॒मो य॑ज्ञ्मु॒खं ॅय॑ज्ञ्मु॒खम् च॑तुष्टो॒मः । \newline
2. य॒ज्ञ्॒मु॒खमिति॑ यज्ञ् - मु॒खम् । \newline
3. च॒तु॒ष्टो॒मो य॑ज्ञ्मु॒खं ॅय॑ज्ञ्मु॒खम् च॑तुष्टो॒म श्च॑तुष्टो॒मो य॑ज्ञ्मु॒खम् । \newline
4. च॒तु॒ष्टो॒म इति॑ चतुः - स्तो॒मः । \newline
5. य॒ज्ञ्॒मु॒ख मे॒वैव य॑ज्ञ्मु॒खं ॅय॑ज्ञ्मु॒ख मे॒व । \newline
6. य॒ज्ञ्॒मु॒खमिति॑ यज्ञ् - मु॒खम् । \newline
7. ए॒व पु॒रस्ता᳚त् पु॒रस्ता॑ दे॒वैव पु॒रस्ता᳚त् । \newline
8. पु॒रस्ता॒द् वि वि पु॒रस्ता᳚त् पु॒रस्ता॒द् वि । \newline
9. वि या॑तयति यातयति॒ वि वि या॑तयति । \newline
10. या॒त॒य॒ति॒ यावा॑नां॒ ॅयावा॑नां ॅयातयति यातयति॒ यावा॑नाम् । \newline
11. यावा॑नाम् भा॒गो भा॒गो यावा॑नां॒ ॅयावा॑नाम् भा॒गः । \newline
12. भा॒गो᳚ ऽस्यसि भा॒गो भा॒गो॑ ऽसि । \newline
13. अ॒सीती त्य॑स्य॒ सीति॑ । \newline
14. इति॑ दक्षिण॒तो द॑क्षिण॒त इतीति॑ दक्षिण॒तः । \newline
15. द॒क्षि॒ण॒तो मासा॒ मासा॑ दक्षिण॒तो द॑क्षिण॒तो मासाः᳚ । \newline
16. मासा॒ वै वै मासा॒ मासा॒ वै । \newline
17. वै यावा॒ यावा॒ वै वै यावाः᳚ । \newline
18. यावा॑ अर्द्धमा॒सा अ॑र्द्धमा॒सा यावा॒ यावा॑ अर्द्धमा॒साः । \newline
19. अ॒र्द्ध॒मा॒सा अया॑वा॒ अया॑वा अर्द्धमा॒सा अ॑र्द्धमा॒सा अया॑वाः । \newline
20. अ॒र्द्ध॒मा॒सा इत्य॑र्द्ध - मा॒साः । \newline
21. अया॑वा॒ स्तस्मा॒त् तस्मा॒ दया॑वा॒ अया॑वा॒ स्तस्मा᳚त् । \newline
22. तस्मा᳚द् दक्षि॒णावृ॑तो दक्षि॒णावृ॑त॒ स्तस्मा॒त् तस्मा᳚द् दक्षि॒णावृ॑तः । \newline
23. द॒क्षि॒णावृ॑तो॒ मासा॒ मासा॑ दक्षि॒णावृ॑तो दक्षि॒णावृ॑तो॒ मासाः᳚ । \newline
24. द॒क्षि॒णावृ॑त॒ इति॑ दक्षि॒णा - आ॒वृ॒तः॒ । \newline
25. मासा॒ अन्न॒ मन्न॒म् मासा॒ मासा॒ अन्न᳚म् । \newline
26. अन्नं॒ ॅवै वा अन्न॒ मन्नं॒ ॅवै । \newline
27. वै यावा॒ यावा॒ वै वै यावाः᳚ । \newline
28. यावा॒ अन्न॒ मन्नं॒ ॅयावा॒ यावा॒ अन्न᳚म् । \newline
29. अन्न॑म् प्र॒जाः प्र॒जा अन्न॒ मन्न॑म् प्र॒जाः । \newline
30. प्र॒जा अन्न॒ मन्न॑म् प्र॒जाः प्र॒जा अन्न᳚म् । \newline
31. प्र॒जा इति॑ प्र - जाः । \newline
32. अन्न॑ मे॒वै वान्न॒ मन्न॑ मे॒व । \newline
33. ए॒व द॑क्षिण॒तो द॑क्षिण॒त ए॒वैव द॑क्षिण॒तः । \newline
34. द॒क्षि॒ण॒तो ध॑त्ते धत्ते दक्षिण॒तो द॑क्षिण॒तो ध॑त्ते । \newline
35. ध॒त्ते॒ तस्मा॒त् तस्मा᳚द् धत्ते धत्ते॒ तस्मा᳚त् । \newline
36. तस्मा॒द् दक्षि॑णेन॒ दक्षि॑णेन॒ तस्मा॒त् तस्मा॒द् दक्षि॑णेन । \newline
37. दक्षि॑णे॒ नान्न॒ मन्न॒म् दक्षि॑णेन॒ दक्षि॑णे॒ नान्न᳚म् । \newline
38. अन्न॑ मद्यते ऽद्य॒ते ऽन्न॒ मन्न॑ मद्यते । \newline
39. अ॒द्य॒त॒ ऋ॒भू॒णा मृ॑भू॒णा म॑द्यते ऽद्यत ऋभू॒णाम् । \newline
40. ऋ॒भू॒णाम् भा॒गो भा॒ग ऋ॑भू॒णा मृ॑भू॒णाम् भा॒गः । \newline
41. भा॒गो᳚ ऽस्यसि भा॒गो भा॒गो॑ ऽसि । \newline
42. अ॒सीती त्य॑स्य॒ सीति॑ । \newline
43. इति॑ प॒श्चात् प॒श्चा दितीति॑ प॒श्चात् । \newline
44. प॒श्चात् प्रति॑ष्ठित्यै॒ प्रति॑ष्ठित्यै प॒श्चात् प॒श्चात् प्रति॑ष्ठित्यै । \newline
45. प्रति॑ष्ठित्यै विव॒र्तो वि॑व॒र्तः प्रति॑ष्ठित्यै॒ प्रति॑ष्ठित्यै विव॒र्तः । \newline
46. प्रति॑ष्ठित्या॒ इति॒ प्रति॑ - स्थि॒त्यै॒ । \newline
47. वि॒व॒र्तो᳚ ऽष्टाचत्वारिꣳ॒॒शो᳚ ऽष्टाचत्वारिꣳ॒॒शो वि॑व॒र्तो वि॑व॒र्तो᳚ ऽष्टाचत्वारिꣳ॒॒शः । \newline
48. वि॒व॒र्त इति॑ वि - व॒र्तः । \newline
49. अ॒ष्टा॒च॒त्वा॒रिꣳ॒॒श इतीत्य॑ ष्टाचत्वारिꣳ॒॒शो᳚ ऽष्टाचत्वारिꣳ॒॒श इति॑ । \newline
50. अ॒ष्टा॒च॒त्वा॒रिꣳ॒॒श इत्य॑ष्टा - च॒त्वा॒रिꣳ॒॒शः । \newline
51. इत्यु॑त्तर॒त उ॑त्तर॒त इती त्यु॑त्तर॒तः । \newline
52. उ॒त्त॒र॒तो॑ ऽनयो॑ र॒नयो॑ रुत्तर॒त उ॑त्तर॒तो॑ ऽनयोः᳚ । \newline
53. उ॒त्त॒र॒त इत्यु॑त् - त॒र॒तः । \newline
54. अ॒नयो᳚र् लो॒कयो᳚र् लो॒कयो॑ र॒नयो॑ र॒नयो᳚र् लो॒कयोः᳚ । \newline
55. लो॒कयोः᳚ सवीर्य॒त्वाय॑ सवीर्य॒त्वाय॑ लो॒कयो᳚र् लो॒कयोः᳚ सवीर्य॒त्वाय॑ । \newline
56. स॒वी॒र्य॒त्वाय॒ तस्मा॒त् तस्मा᳚थ् सवीर्य॒त्वाय॑ सवीर्य॒त्वाय॒ तस्मा᳚त् । \newline
57. स॒वी॒र्य॒त्वायेति॑ सवीर्य - त्वाय॑ । \newline
58. तस्मा॑ दि॒मा वि॒मौ तस्मा॒त् तस्मा॑ दि॒मौ । \newline
59. इ॒मौ लो॒कौ लो॒का वि॒मा वि॒मौ लो॒कौ । \newline
60. लो॒कौ स॒माव॑द्वीर्यौ स॒माव॑द्वीर्यौ लो॒कौ लो॒कौ स॒माव॑द्वीर्यौ । \newline
61. स॒माव॑द्वीर्यौ॒ यस्य॒ यस्य॑ स॒माव॑द्वीर्यौ स॒माव॑द्वीर्यौ॒ यस्य॑ । \newline
62. स॒माव॑द्वीर्या॒विति॑ स॒माव॑त् - वी॒र्यौ॒ । \newline

\textbf{Ghana Paata } \newline

1. य॒ज्ञ्॒मु॒खम् च॑तुष्टो॒म श्च॑तुष्टो॒मो य॑ज्ञ्मु॒खं ॅय॑ज्ञ्मु॒खम् च॑तुष्टो॒मो य॑ज्ञ्मु॒खं ॅय॑ज्ञ्मु॒खम् च॑तुष्टो॒मो य॑ज्ञ्मु॒खं ॅय॑ज्ञ्मु॒खम् च॑तुष्टो॒मो य॑ज्ञ्मु॒खम् । \newline
2. य॒ज्ञ्॒मु॒खमिति॑ यज्ञ् - मु॒खम् । \newline
3. च॒तु॒ष्टो॒मो य॑ज्ञ्मु॒खं ॅय॑ज्ञ्मु॒खम् च॑तुष्टो॒म श्च॑तुष्टो॒मो य॑ज्ञ्मु॒ख मे॒वैव य॑ज्ञ्मु॒खम् च॑तुष्टो॒म श्च॑तुष्टो॒मो य॑ज्ञ्मु॒ख मे॒व । \newline
4. च॒तु॒ष्टो॒म इति॑ चतुः - स्तो॒मः । \newline
5. य॒ज्ञ्॒मु॒ख मे॒वैव य॑ज्ञ्मु॒खं ॅय॑ज्ञ्मु॒ख मे॒व पु॒रस्ता᳚त् पु॒रस्ता॑ दे॒व य॑ज्ञ्मु॒खं ॅय॑ज्ञ्मु॒ख मे॒व पु॒रस्ता᳚त् । \newline
6. य॒ज्ञ्॒मु॒खमिति॑ यज्ञ् - मु॒खम् । \newline
7. ए॒व पु॒रस्ता᳚त् पु॒रस्ता॑ दे॒वैव पु॒रस्ता॒द् वि वि पु॒रस्ता॑ दे॒वैव पु॒रस्ता॒द् वि । \newline
8. पु॒रस्ता॒द् वि वि पु॒रस्ता᳚त् पु॒रस्ता॒द् वि या॑तयति यातयति॒ वि पु॒रस्ता᳚त् पु॒रस्ता॒द् वि या॑तयति । \newline
9. वि या॑तयति यातयति॒ वि वि या॑तयति॒ यावा॑नां॒ ॅयावा॑नां ॅयातयति॒ वि वि या॑तयति॒ यावा॑नाम् । \newline
10. या॒त॒य॒ति॒ यावा॑नां॒ ॅयावा॑नां ॅयातयति यातयति॒ यावा॑नाम् भा॒गो भा॒गो यावा॑नां ॅयातयति यातयति॒ यावा॑नाम् भा॒गः । \newline
11. यावा॑नाम् भा॒गो भा॒गो यावा॑नां॒ ॅयावा॑नाम् भा॒गो᳚ ऽस्यसि भा॒गो यावा॑नां॒ ॅयावा॑नाम् भा॒गो॑ ऽसि । \newline
12. भा॒गो᳚ ऽस्यसि भा॒गो भा॒गो॑ ऽसीती त्य॑सि भा॒गो भा॒गो॑ ऽसीति॑ । \newline
13. अ॒सीती त्य॑स्य॒ सीति॑ दक्षिण॒तो द॑क्षिण॒त इत्य॑स्य॒सीति॑ दक्षिण॒तः । \newline
14. इति॑ दक्षिण॒तो द॑क्षिण॒त इतीति॑ दक्षिण॒तो मासा॒ मासा॑ दक्षिण॒त इतीति॑ दक्षिण॒तो मासाः᳚ । \newline
15. द॒क्षि॒ण॒तो मासा॒ मासा॑ दक्षिण॒तो द॑क्षिण॒तो मासा॒ वै वै मासा॑ दक्षिण॒तो द॑क्षिण॒तो मासा॒ वै । \newline
16. मासा॒ वै वै मासा॒ मासा॒ वै यावा॒ यावा॒ वै मासा॒ मासा॒ वै यावाः᳚ । \newline
17. वै यावा॒ यावा॒ वै वै यावा॑ अर्द्धमा॒सा अ॑र्द्धमा॒सा यावा॒ वै वै यावा॑ अर्द्धमा॒साः । \newline
18. यावा॑ अर्द्धमा॒सा अ॑र्द्धमा॒सा यावा॒ यावा॑ अर्द्धमा॒सा अया॑वा॒ अया॑वा अर्द्धमा॒सा यावा॒ यावा॑ अर्द्धमा॒सा अया॑वाः । \newline
19. अ॒र्द्ध॒मा॒सा अया॑वा॒ अया॑वा अर्द्धमा॒सा अ॑र्द्धमा॒सा अया॑वा॒ स्तस्मा॒त् तस्मा॒ दया॑वा अर्द्धमा॒सा अ॑र्द्धमा॒सा अया॑वा॒ स्तस्मा᳚त् । \newline
20. अ॒र्द्ध॒मा॒सा इत्य॑र्द्ध - मा॒साः । \newline
21. अया॑वा॒ स्तस्मा॒त् तस्मा॒ दया॑वा॒ अया॑वा॒ स्तस्मा᳚द् दक्षि॒णावृ॑तो दक्षि॒णावृ॑त॒ स्तस्मा॒ दया॑वा॒ अया॑वा॒ स्तस्मा᳚द् दक्षि॒णावृ॑तः । \newline
22. तस्मा᳚द् दक्षि॒णावृ॑तो दक्षि॒णावृ॑त॒ स्तस्मा॒त् तस्मा᳚द् दक्षि॒णावृ॑तो॒ मासा॒ मासा॑ दक्षि॒णावृ॑त॒ स्तस्मा॒त् तस्मा᳚द् दक्षि॒णावृ॑तो॒ मासाः᳚ । \newline
23. द॒क्षि॒णावृ॑तो॒ मासा॒ मासा॑ दक्षि॒णावृ॑तो दक्षि॒णावृ॑तो॒ मासा॒ अन्न॒ मन्न॒म् मासा॑ दक्षि॒णावृ॑तो दक्षि॒णावृ॑तो॒ मासा॒ अन्न᳚म् । \newline
24. द॒क्षि॒णावृ॑त॒ इति॑ दक्षि॒णा - आ॒वृ॒तः॒ । \newline
25. मासा॒ अन्न॒ मन्न॒म् मासा॒ मासा॒ अन्नं॒ ॅवै वा अन्न॒म् मासा॒ मासा॒ अन्नं॒ ॅवै । \newline
26. अन्नं॒ ॅवै वा अन्न॒ मन्नं॒ ॅवै यावा॒ यावा॒ वा अन्न॒ मन्नं॒ ॅवै यावाः᳚ । \newline
27. वै यावा॒ यावा॒ वै वै यावा॒ अन्न॒ मन्नं॒ ॅयावा॒ वै वै यावा॒ अन्न᳚म् । \newline
28. यावा॒ अन्न॒ मन्नं॒ ॅयावा॒ यावा॒ अन्न॑म् प्र॒जाः प्र॒जा अन्नं॒ ॅयावा॒ यावा॒ अन्न॑म् प्र॒जाः । \newline
29. अन्न॑म् प्र॒जाः प्र॒जा अन्न॒ मन्न॑म् प्र॒जा अन्न॒ मन्न॑म् प्र॒जा अन्न॒ मन्न॑म् प्र॒जा अन्न᳚म् । \newline
30. प्र॒जा अन्न॒ मन्न॑म् प्र॒जाः प्र॒जा अन्न॑ मे॒वै वान्न॑म् प्र॒जाः प्र॒जा अन्न॑ मे॒व । \newline
31. प्र॒जा इति॑ प्र - जाः । \newline
32. अन्न॑ मे॒वै वान्न॒ मन्न॑ मे॒व द॑क्षिण॒तो द॑क्षिण॒त ए॒वान्न॒ मन्न॑ मे॒व द॑क्षिण॒तः । \newline
33. ए॒व द॑क्षिण॒तो द॑क्षिण॒त ए॒वैव द॑क्षिण॒तो ध॑त्ते धत्ते दक्षिण॒त ए॒वैव द॑क्षिण॒तो ध॑त्ते । \newline
34. द॒क्षि॒ण॒तो ध॑त्ते धत्ते दक्षिण॒तो द॑क्षिण॒तो ध॑त्ते॒ तस्मा॒त् तस्मा᳚द् धत्ते दक्षिण॒तो द॑क्षिण॒तो ध॑त्ते॒ तस्मा᳚त् । \newline
35. ध॒त्ते॒ तस्मा॒त् तस्मा᳚द् धत्ते धत्ते॒ तस्मा॒द् दक्षि॑णेन॒ दक्षि॑णेन॒ तस्मा᳚द् धत्ते धत्ते॒ तस्मा॒द् दक्षि॑णेन । \newline
36. तस्मा॒द् दक्षि॑णेन॒ दक्षि॑णेन॒ तस्मा॒त् तस्मा॒द् दक्षि॑णे॒ नान्न॒ मन्न॒म् दक्षि॑णेन॒ तस्मा॒त् तस्मा॒द् दक्षि॑णे॒ नान्न᳚म् । \newline
37. दक्षि॑णे॒ नान्न॒ मन्न॒म् दक्षि॑णेन॒ दक्षि॑णे॒ नान्न॑ मद्यते ऽद्य॒ते ऽन्न॒म् दक्षि॑णेन॒ दक्षि॑णे॒नान्न॑ मद्यते । \newline
38. अन्न॑ मद्यते ऽद्य॒ते ऽन्न॒ मन्न॑ मद्यत ऋभू॒णा मृ॑भू॒णा म॑द्य॒ते ऽन्न॒ मन्न॑ मद्यत ऋभू॒णाम् । \newline
39. अ॒द्य॒त॒ ऋ॒भू॒णा मृ॑भू॒णा म॑द्यते ऽद्यत ऋभू॒णाम् भा॒गो भा॒ग ऋ॑भू॒णा म॑द्यते ऽद्यत ऋभू॒णाम् भा॒गः । \newline
40. ऋ॒भू॒णाम् भा॒गो भा॒ग ऋ॑भू॒णा मृ॑भू॒णाम् भा॒गो᳚ ऽस्यसि भा॒ग ऋ॑भू॒णा मृ॑भू॒णाम् भा॒गो॑ ऽसि । \newline
41. भा॒गो᳚ ऽस्यसि भा॒गो भा॒गो॑ ऽसीती त्य॑सि भा॒गो भा॒गो॑ ऽसीति॑ । \newline
42. अ॒सीती त्य॑स्य॒ सीति॑ प॒श्चात् प॒श्चा दित्य॑स्य॒ सीति॑ प॒श्चात् । \newline
43. इति॑ प॒श्चात् प॒श्चा दितीति॑ प॒श्चात् प्रति॑ष्ठित्यै॒ प्रति॑ष्ठित्यै प॒श्चा दितीति॑ प॒श्चात् प्रति॑ष्ठित्यै । \newline
44. प॒श्चात् प्रति॑ष्ठित्यै॒ प्रति॑ष्ठित्यै प॒श्चात् प॒श्चात् प्रति॑ष्ठित्यै विव॒र्तो वि॑व॒र्तः प्रति॑ष्ठित्यै प॒श्चात् प॒श्चात् प्रति॑ष्ठित्यै विव॒र्तः । \newline
45. प्रति॑ष्ठित्यै विव॒र्तो वि॑व॒र्तः प्रति॑ष्ठित्यै॒ प्रति॑ष्ठित्यै विव॒र्तो᳚ ऽष्टाचत्वारिꣳ॒॒शो᳚ ऽष्टाचत्वारिꣳ॒॒शो वि॑व॒र्तः प्रति॑ष्ठित्यै॒ प्रति॑ष्ठित्यै विव॒र्तो᳚ ऽष्टाचत्वारिꣳ॒॒शः । \newline
46. प्रति॑ष्ठित्या॒ इति॒ प्रति॑ - स्थि॒त्यै॒ । \newline
47. वि॒व॒र्तो᳚ ऽष्टाचत्वारिꣳ॒॒शो᳚ ऽष्टाचत्वारिꣳ॒॒शो वि॑व॒र्तो वि॑व॒र्तो᳚ ऽष्टाचत्वारिꣳ॒॒श इतीत्य॑ष्टाचत्वारिꣳ॒॒शो वि॑व॒र्तो वि॑व॒र्तो᳚ ऽष्टाचत्वारिꣳ॒॒श इति॑ । \newline
48. वि॒व॒र्त इति॑ वि - व॒र्तः । \newline
49. अ॒ष्टा॒च॒त्वा॒रिꣳ॒॒श इतीत्य॑ ष्टाचत्वारिꣳ॒॒शो᳚ ऽष्टाचत्वारिꣳ॒॒श इत्यु॑त्तर॒त उ॑त्तर॒त इत्य॑ष्टाचत्वारिꣳ॒॒शो᳚ ऽष्टाचत्वारिꣳ॒॒श इत्यु॑त्तर॒तः । \newline
50. अ॒ष्टा॒च॒त्वा॒रिꣳ॒॒श इत्य॑ष्टा - च॒त्वा॒रिꣳ॒॒शः । \newline
51. इत्यु॑त्तर॒त उ॑त्तर॒त इती त्यु॑त्तर॒तो॑ ऽनयो॑ र॒नयो॑ रुत्तर॒त इती त्यु॑त्तर॒तो॑ ऽनयोः᳚ । \newline
52. उ॒त्त॒र॒तो॑ ऽनयो॑ र॒नयो॑ रुत्तर॒त उ॑त्तर॒तो॑ ऽनयो᳚र् लो॒कयो᳚र् लो॒कयो॑ र॒नयो॑ रुत्तर॒त उ॑त्तर॒तो॑ ऽनयो᳚र् लो॒कयोः᳚ । \newline
53. उ॒त्त॒र॒त इत्यु॑त् - त॒र॒तः । \newline
54. अ॒नयो᳚र् लो॒कयो᳚र् लो॒कयो॑ र॒नयो॑ र॒नयो᳚र् लो॒कयोः᳚ सवीर्य॒त्वाय॑ सवीर्य॒त्वाय॑ लो॒कयो॑ र॒नयो॑ र॒नयो᳚र् लो॒कयोः᳚ सवीर्य॒त्वाय॑ । \newline
55. लो॒कयोः᳚ सवीर्य॒त्वाय॑ सवीर्य॒त्वाय॑ लो॒कयो᳚र् लो॒कयोः᳚ सवीर्य॒त्वाय॒ तस्मा॒त् तस्मा᳚थ् सवीर्य॒त्वाय॑ लो॒कयो᳚र् लो॒कयोः᳚ सवीर्य॒त्वाय॒ तस्मा᳚त् । \newline
56. स॒वी॒र्य॒त्वाय॒ तस्मा॒त् तस्मा᳚थ् सवीर्य॒त्वाय॑ सवीर्य॒त्वाय॒ तस्मा॑दि॒मा वि॒मौ तस्मा᳚थ् सवीर्य॒त्वाय॑ सवीर्य॒त्वाय॒ तस्मा॑दि॒मौ । \newline
57. स॒वी॒र्य॒त्वायेति॑ सवीर्य - त्वाय॑ । \newline
58. तस्मा॑दि॒मा वि॒मौ तस्मा॒त् तस्मा॑ दि॒मौ लो॒कौ लो॒का वि॒मौ तस्मा॒त् तस्मा॑ दि॒मौ लो॒कौ । \newline
59. इ॒मौ लो॒कौ लो॒का वि॒मा वि॒मौ लो॒कौ स॒माव॑द्वीर्यौ स॒माव॑द्वीर्यौ लो॒का वि॒मा वि॒मौ लो॒कौ स॒माव॑द्वीर्यौ । \newline
60. लो॒कौ स॒माव॑द्वीर्यौ स॒माव॑द्वीर्यौ लो॒कौ लो॒कौ स॒माव॑द्वीर्यौ॒ यस्य॒ यस्य॑ स॒माव॑द्वीर्यौ लो॒कौ लो॒कौ स॒माव॑द्वीर्यौ॒ यस्य॑ । \newline
61. स॒माव॑द्वीर्यौ॒ यस्य॒ यस्य॑ स॒माव॑द्वीर्यौ स॒माव॑द्वीर्यौ॒ यस्य॒ मुख्य॑वती॒र् मुख्य॑वती॒र् यस्य॑ स॒माव॑द्वीर्यौ स॒माव॑द्वीर्यौ॒ यस्य॒ मुख्य॑वतीः । \newline
62. स॒माव॑द्वीर्या॒विति॑ स॒माव॑त् - वी॒र्यौ॒ । \newline
\pagebreak
\markright{ TS 5.3.4.6  \hfill https://www.vedavms.in \hfill}

\section{ TS 5.3.4.6 }

\textbf{TS 5.3.4.6 } \newline
\textbf{Samhita Paata} \newline

यस्य॒ मुख्य॑वतीः पु॒रस्ता॑दुपधी॒यन्ते॒ मुख्य॑ ए॒व भ॑व॒त्याऽस्य॒ मुख्यो॑ जायते॒ यस्या-न्न॑वती - र्दक्षिण॒तो-ऽत्त्यन्न॒माऽस्या᳚न्ना॒दो जा॑यते॒ यस्य॑ प्रति॒ष्ठाव॑तीः प॒श्चात् प्रत्ये॒व ति॑ष्ठति॒ यस्यौज॑स्वतीरुत्तर॒त ओ॑ज॒स्व्ये॑व भ॑व॒त्याऽस्यौ॑ज॒स्वी जा॑यते॒ ऽर्को वा ए॒ष यद॒ग्निस्तस्यै॒तदे॒व स्तो॒त्रमे॒तच्छ॒स्त्रं ॅयदे॒षा वि॒धा - [  ] \newline

\textbf{Pada Paata} \newline

यस्य॑ । मुख्य॑वती॒रिति॒ मुख्य॑ - व॒तीः॒ । पु॒रस्ता᳚त् । उ॒प॒धी॒यन्त॒ इत्यु॑प - धी॒यन्ते᳚ । मुख्यः॑ । ए॒व । भ॒व॒ति॒ । एति॑ । अ॒स्य॒ । मुख्यः॑ । जा॒य॒ते॒ । यस्य॑ । अन्न॑वती॒रित्यन्न॑ - व॒तीः॒ । द॒क्षि॒ण॒तः । अत्ति॑ । अन्न᳚म् । एति॑ । अ॒स्य॒ । अ॒न्ना॒द इत्य॑न्न - अ॒दः । जा॒य॒ते॒ । यस्य॑ । प्र॒ति॒ष्ठाव॑ती॒रिति॑ प्रति॒ष्ठा - व॒तीः॒ । प॒श्चात् । प्रतीति॑ । ए॒व । ति॒ष्ठ॒ति॒ । यस्य॑ । ओज॑स्वतीः । उ॒त्त॒र॒त इत्यु॑त् - त॒र॒तः । ओ॒ज॒स्वी । ए॒व । भ॒व॒ति॒ । एति॑ । अ॒स्य॒ । ओ॒ज॒स्वी । जा॒य॒ते॒ । अ॒र्कः । वै । ए॒षः । यत् । अ॒ग्निः । तस्य॑ । ए॒तत् । ए॒व । स्तो॒त्रम् । ए॒तत् । श॒स्त्रम् । यत् । ए॒षा । वि॒धेति॑ वि - धा ।  \newline


\textbf{Krama Paata} \newline

यस्य॒ मुख्य॑वतीः । मुख्य॑वतीः पु॒रस्ता᳚त् । मुख्य॑वती॒रिति॒ मुख्य॑ - व॒तीः॒ । पु॒रस्ता॑दुपधी॒यन्ते᳚ । उ॒प॒धी॒यन्ते॒ मुख्यः॑ । उ॒प॒धी॒यन्त॒ इत्यु॑प - धी॒यन्ते᳚ । मुख्य॑ ए॒व । ए॒व भ॑वति । भ॒व॒त्या । आऽस्य॑ । अ॒स्य॒ मुख्यः॑ । मुख्यो॑ जायते । जा॒य॒ते॒ यस्य॑ । यस्यान्न॑वतीः । अन्न॑वतीर् दक्षिण॒तः । अन्न॑वती॒रित्यन्न॑ - व॒तीः॒ । द॒क्षि॒ण॒तोऽत्ति॑ । अत्यन्न᳚म् । अन्न॒मा । आऽस्य॑ । अ॒स्या॒न्ना॒दः । अ॒न्ना॒दो जा॑यते । अ॒न्ना॒द इत्य॑न्न - अ॒दः । जा॒य॒ते॒ यस्य॑ । यस्य॑ प्रति॒ष्ठाव॑तीः । प्र॒ति॒ष्ठाव॑तीः प॒श्चात् । प्र॒ति॒ष्ठाव॑ती॒रिति॑ प्रति॒ष्ठा - व॒तीः॒ । प॒श्चात् प्रति॑ । प्रत्ये॒व । ए॒व ति॑ष्ठति । ति॒ष्ठ॒ति॒ यस्य॑ । यस्यौज॑स्वतीः । ओज॑स्वतीरुत्तर॒तः । उ॒त्त॒र॒त ओ॑ज॒स्वी । उ॒त्त॒र॒त इत्यु॑त् - त॒र॒तः । ओ॒ज॒स्व्ये॑व । ए॒व भ॑वति । भ॒व॒त्या । आऽस्य॑ । अ॒स्यौ॒ज॒स्वी । ओ॒ज॒स्वी जा॑यते । जा॒य॒ते॒ऽर्कः । अ॒र्को वै । वा ए॒षः । ए॒ष यत् । यद॒ग्निः । अ॒ग्निस्तस्य॑ । तस्यै॒तत् । ए॒तदे॒व । ए॒व स्तो॒त्रम् । स्तो॒त्रमे॒तत् । ए॒तच्छ॒स्त्रम् । श॒स्त्रम् ॅयत् । यदे॒षा । ए॒षा वि॒धा । वि॒धा वि॑धी॒यते᳚ । वि॒धेति॑ वि - धा \newline

\textbf{Jatai Paata} \newline

1. यस्य॒ मुख्य॑वती॒र् मुख्य॑वती॒र् यस्य॒ यस्य॒ मुख्य॑वतीः । \newline
2. मुख्य॑वतीः पु॒रस्ता᳚त् पु॒रस्ता॒न् मुख्य॑वती॒र् मुख्य॑वतीः पु॒रस्ता᳚त् । \newline
3. मुख्य॑वती॒रिति॒ मुख्य॑ - व॒तीः॒ । \newline
4. पु॒रस्ता॑ दुपधी॒यन्त॑ उपधी॒यन्ते॑ पु॒रस्ता᳚त् पु॒रस्ता॑ दुपधी॒यन्ते᳚ । \newline
5. उ॒प॒धी॒यन्ते॒ मुख्यो॒ मुख्य॑ उपधी॒यन्त॑ उपधी॒यन्ते॒ मुख्यः॑ । \newline
6. उ॒प॒धी॒यन्त॒ इत्यु॑प - धी॒यन्ते᳚ । \newline
7. मुख्य॑ ए॒वैव मुख्यो॒ मुख्य॑ ए॒व । \newline
8. ए॒व भ॑वति भव त्ये॒वैव भ॑वति । \newline
9. भ॒व॒त्या भ॑वति भव॒त्या । \newline
10. आ ऽस्या॒स्या ऽस्य॑ । \newline
11. अ॒स्य॒ मुख्यो॒ मुख्यो᳚ ऽस्यास्य॒ मुख्यः॑ । \newline
12. मुख्यो॑ जायते जायते॒ मुख्यो॒ मुख्यो॑ जायते । \newline
13. जा॒य॒ते॒ यस्य॒ यस्य॑ जायते जायते॒ यस्य॑ । \newline
14. यस्या न्न॑वती॒ रन्न॑वती॒र् यस्य॒ यस्या न्न॑वतीः । \newline
15. अन्न॑वतीर् दक्षिण॒तो द॑क्षिण॒तो ऽन्न॑वती॒ रन्न॑वतीर् दक्षिण॒तः । \newline
16. अन्न॑वती॒रित्यन्न॑ - व॒तीः॒ । \newline
17. द॒क्षि॒ण॒तो ऽत्त्यत्ति॑ दक्षिण॒तो द॑क्षिण॒तो ऽत्ति॑ । \newline
18. अत्त्यन्न॒ मन्न॒ मत्त्य त्त्यन्न᳚म् । \newline
19. अन्न॒ मा ऽन्न॒ मन्न॒ मा । \newline
20. आ ऽस्या॒स्या ऽस्य॑ । \newline
21. अ॒स्या॒ न्ना॒दो᳚ ऽन्ना॒दो᳚ ऽस्यास्या न्ना॒दः । \newline
22. अ॒न्ना॒दो जा॑यते जायते ऽन्ना॒दो᳚ ऽन्ना॒दो जा॑यते । \newline
23. अ॒न्ना॒द इत्य॑न्न - अ॒दः । \newline
24. जा॒य॒ते॒ यस्य॒ यस्य॑ जायते जायते॒ यस्य॑ । \newline
25. यस्य॑ प्रति॒ष्ठाव॑तीः प्रति॒ष्ठाव॑ती॒र् यस्य॒ यस्य॑ प्रति॒ष्ठाव॑तीः । \newline
26. प्र॒ति॒ष्ठाव॑तीः प॒श्चात् प॒श्चात् प्र॑ति॒ष्ठाव॑तीः प्रति॒ष्ठाव॑तीः प॒श्चात् । \newline
27. प्र॒ति॒ष्ठाव॑ ती॒रिति॑ प्रति॒ष्ठा - व॒तीः॒ । \newline
28. प॒श्चात् प्रति॒ प्रति॑ प॒श्चात् प॒श्चात् प्रति॑ । \newline
29. प्रत्ये॒ वैव प्रति॒ प्रत्ये॒व । \newline
30. ए॒व ति॑ष्ठति तिष्ठ त्ये॒वैव ति॑ष्ठति । \newline
31. ति॒ष्ठ॒ति॒ यस्य॒ यस्य॑ तिष्ठति तिष्ठति॒ यस्य॑ । \newline
32. यस्यौ ज॑स्वती॒ रोज॑स्वती॒र् यस्य॒ यस्यौज॑स्वतीः । \newline
33. ओज॑स्वती रुत्तर॒त उ॑त्तर॒त ओज॑स्वती॒ रोज॑स्वती रुत्तर॒तः । \newline
34. उ॒त्त॒र॒त ओ॑ज॒ स्व्यो॑ज॒ स्व्यु॑त्तर॒त उ॑त्तर॒त ओ॑ज॒स्वी । \newline
35. उ॒त्त॒र॒त इत्यु॑त् - त॒र॒तः । \newline
36. ओ॒ज॒ स्व्ये॑वै वौज॒ स्व्यो॑ज॒ स्व्ये॑व । \newline
37. ए॒व भ॑वति भव त्ये॒वैव भ॑वति । \newline
38. भ॒व॒त्या भ॑वति भव॒त्या । \newline
39. आ ऽस्या॒स्या ऽस्य॑ । \newline
40. अ॒स्यौ॒ज॒ स्व्यो॑ज॒ स्व्य॑स्या स्यौज॒स्वी । \newline
41. ओ॒ज॒स्वी जा॑यते जायत ओज॒ स्व्यो॑ज॒स्वी जा॑यते । \newline
42. जा॒य॒ते॒ ऽर्को᳚ ऽर्को जा॑यते जायते॒ ऽर्कः । \newline
43. अ॒र्को वै वा अ॒र्को᳚ ऽर्को वै । \newline
44. वा ए॒ष ए॒ष वै वा ए॒षः । \newline
45. ए॒ष यद् यदे॒ष ए॒ष यत् । \newline
46. यद॒ग्नि र॒ग्निर् यद् यद॒ग्निः । \newline
47. अ॒ग्नि स्तस्य॒ तस्या॒ ग्नि र॒ग्नि स्तस्य॑ । \newline
48. तस्यै॒त दे॒तत् तस्य॒ तस्यै॒तत् । \newline
49. ए॒त दे॒वैवैत दे॒त दे॒व । \newline
50. ए॒व स्तो॒त्रꣳ स्तो॒त्र मे॒वैव स्तो॒त्रम् । \newline
51. स्तो॒त्र मे॒तदे॒तथ् स्तो॒त्रꣳ स्तो॒त्र मे॒तत् । \newline
52. ए॒त च्छ॒स्त्रꣳ श॒स्त्र मे॒त दे॒त च्छ॒स्त्रम् । \newline
53. श॒स्त्रं ॅयद् यच्छ॒ स्त्रꣳ श॒स्त्रं ॅयत् । \newline
54. यदे॒षैषा यद् यदे॒षा । \newline
55. ए॒षा वि॒धा वि॒धैषैषा वि॒धा । \newline
56. वि॒धा वि॑धी॒यते॑ विधी॒यते॑ वि॒धा वि॒धा वि॑धी॒यते᳚ । \newline
57. वि॒धेति॑ वि - धा । \newline

\textbf{Ghana Paata } \newline

1. यस्य॒ मुख्य॑वती॒र् मुख्य॑वती॒र् यस्य॒ यस्य॒ मुख्य॑वतीः पु॒रस्ता᳚त् पु॒रस्ता॒न् मुख्य॑वती॒र् यस्य॒ यस्य॒ मुख्य॑वतीः पु॒रस्ता᳚त् । \newline
2. मुख्य॑वतीः पु॒रस्ता᳚त् पु॒रस्ता॒न् मुख्य॑वती॒र् मुख्य॑वतीः पु॒रस्ता॑ दुपधी॒यन्त॑ उपधी॒यन्ते॑ पु॒रस्ता॒न् मुख्य॑वती॒र् मुख्य॑वतीः पु॒रस्ता॑ दुपधी॒यन्ते᳚ । \newline
3. मुख्य॑वती॒रिति॒ मुख्य॑ - व॒तीः॒ । \newline
4. पु॒रस्ता॑ दुपधी॒यन्त॑ उपधी॒यन्ते॑ पु॒रस्ता᳚त् पु॒रस्ता॑ दुपधी॒यन्ते॒ मुख्यो॒ मुख्य॑ उपधी॒यन्ते॑ पु॒रस्ता᳚त् पु॒रस्ता॑ दुपधी॒यन्ते॒ मुख्यः॑ । \newline
5. उ॒प॒धी॒यन्ते॒ मुख्यो॒ मुख्य॑ उपधी॒यन्त॑ उपधी॒यन्ते॒ मुख्य॑ ए॒वैव मुख्य॑ उपधी॒यन्त॑ उपधी॒यन्ते॒ मुख्य॑ ए॒व । \newline
6. उ॒प॒धी॒यन्त॒ इत्यु॑प - धी॒यन्ते᳚ । \newline
7. मुख्य॑ ए॒वैव मुख्यो॒ मुख्य॑ ए॒व भ॑वति भव त्ये॒व मुख्यो॒ मुख्य॑ ए॒व भ॑वति । \newline
8. ए॒व भ॑वति भव त्ये॒वैव भ॑व॒त्या भ॑व त्ये॒वैव भ॑व॒त्या । \newline
9. भ॒व॒त्या भ॑वति भव॒त्या ऽस्या॒स्या भ॑वति भव॒त्या ऽस्य॑ । \newline
10. आ ऽस्या॒स्या ऽस्य॒ मुख्यो॒ मुख्यो॒ ऽस्या ऽस्य॒ मुख्यः॑ । \newline
11. अ॒स्य॒ मुख्यो॒ मुख्यो᳚ ऽस्यास्य॒ मुख्यो॑ जायते जायते॒ मुख्यो᳚ ऽस्यास्य॒ मुख्यो॑ जायते । \newline
12. मुख्यो॑ जायते जायते॒ मुख्यो॒ मुख्यो॑ जायते॒ यस्य॒ यस्य॑ जायते॒ मुख्यो॒ मुख्यो॑ जायते॒ यस्य॑ । \newline
13. जा॒य॒ते॒ यस्य॒ यस्य॑ जायते जायते॒ यस्या न्न॑वती॒ रन्न॑वती॒र् यस्य॑ जायते जायते॒ यस्या न्न॑वतीः । \newline
14. यस्या न्न॑वती॒ रन्न॑वती॒र् यस्य॒ यस्या न्न॑वतीर् दक्षिण॒तो द॑क्षिण॒तो ऽन्न॑वती॒र् यस्य॒ यस्या न्न॑वतीर् दक्षिण॒तः । \newline
15. अन्न॑वतीर् दक्षिण॒तो द॑क्षिण॒तो ऽन्न॑वती॒ रन्न॑वतीर् दक्षिण॒तो ऽत्त्यत्ति॑ दक्षिण॒तो ऽन्न॑वती॒ रन्न॑वतीर् दक्षिण॒तो ऽत्ति॑ । \newline
16. अन्न॑वती॒रित्यन्न॑ - व॒तीः॒ । \newline
17. द॒क्षि॒ण॒तो ऽत्त्यत्ति॑ दक्षिण॒तो द॑क्षिण॒तो ऽत्त्यन्न॒ मन्न॒ मत्ति॑ दक्षिण॒तो द॑क्षिण॒तो ऽत्त्यन्न᳚म् । \newline
18. अत्त्यन्न॒ मन्न॒ मत्त्य त्त्यन्न॒ मा ऽन्न॒ मत्त्य त्त्यन्न॒ मा । \newline
19. अन्न॒ मा ऽन्न॒ मन्न॒ मा ऽस्या॒स्या ऽन्न॒ मन्न॒ मा ऽस्य॑ । \newline
20. आ ऽस्या॒स्या ऽस्या᳚ न्ना॒दो᳚ ऽन्ना॒दो᳚ ऽस्या ऽस्या᳚ न्ना॒दः । \newline
21. अ॒स्या॒ न्ना॒दो᳚ ऽन्ना॒दो᳚ ऽस्यास्या न्ना॒दो जा॑यते जायते ऽन्ना॒दो᳚ ऽस्यास्या न्ना॒दो जा॑यते । \newline
22. अ॒न्ना॒दो जा॑यते जायते ऽन्ना॒दो᳚ ऽन्ना॒दो जा॑यते॒ यस्य॒ यस्य॑ जायते ऽन्ना॒दो᳚ ऽन्ना॒दो जा॑यते॒ यस्य॑ । \newline
23. अ॒न्ना॒द इत्य॑न्न - अ॒दः । \newline
24. जा॒य॒ते॒ यस्य॒ यस्य॑ जायते जायते॒ यस्य॑ प्रति॒ष्ठाव॑तीः प्रति॒ष्ठाव॑ती॒र् यस्य॑ जायते जायते॒ यस्य॑ प्रति॒ष्ठाव॑तीः । \newline
25. यस्य॑ प्रति॒ष्ठाव॑तीः प्रति॒ष्ठाव॑ती॒र् यस्य॒ यस्य॑ प्रति॒ष्ठाव॑तीः प॒श्चात् प॒श्चात् प्र॑ति॒ष्ठाव॑ती॒र् यस्य॒ यस्य॑ प्रति॒ष्ठाव॑तीः प॒श्चात् । \newline
26. प्र॒ति॒ष्ठाव॑तीः प॒श्चात् प॒श्चात् प्र॑ति॒ष्ठाव॑तीः प्रति॒ष्ठाव॑तीः प॒श्चात् प्रति॒ प्रति॑ प॒श्चात् प्र॑ति॒ष्ठाव॑तीः प्रति॒ष्ठाव॑तीः प॒श्चात् प्रति॑ । \newline
27. प्र॒ति॒ष्ठाव॑ती॒रिति॑ प्रति॒ष्ठा - व॒तीः॒ । \newline
28. प॒श्चात् प्रति॒ प्रति॑ प॒श्चात् प॒श्चात् प्रत्ये॒वैव प्रति॑ प॒श्चात् प॒श्चात् प्रत्ये॒व । \newline
29. प्रत्ये॒वैव प्रति॒ प्रत्ये॒व ति॑ष्ठति तिष्ठ त्ये॒व प्रति॒ प्रत्ये॒व ति॑ष्ठति । \newline
30. ए॒व ति॑ष्ठति तिष्ठ त्ये॒वैव ति॑ष्ठति॒ यस्य॒ यस्य॑ तिष्ठ त्ये॒वैव ति॑ष्ठति॒ यस्य॑ । \newline
31. ति॒ष्ठ॒ति॒ यस्य॒ यस्य॑ तिष्ठति तिष्ठति॒ यस्यौज॑स्वती॒ रोज॑स्वती॒र् यस्य॑ तिष्ठति तिष्ठति॒ यस्यौज॑स्वतीः । \newline
32. यस्यौज॑स्वती॒ रोज॑स्वती॒र् यस्य॒ यस्यौज॑स्वती रुत्तर॒त उ॑त्तर॒त ओज॑स्वती॒र् यस्य॒ यस्यौज॑स्वती रुत्तर॒तः । \newline
33. ओज॑स्वती रुत्तर॒त उ॑त्तर॒त ओज॑स्वती॒ रोज॑स्वती रुत्तर॒त ओ॑ज॒ स्व्यो॑ज॒ स्व्यु॑त्तर॒त ओज॑स्वती॒ रोज॑स्वती रुत्तर॒त ओ॑ज॒स्वी । \newline
34. उ॒त्त॒र॒त ओ॑ज॒ स्व्यो॑ज॒ स्व्यु॑त्तर॒त उ॑त्तर॒त ओ॑ज॒ स्व्ये॑वैवौज॒ स्व्यु॑त्तर॒त उ॑त्तर॒त ओ॑ज॒ स्व्ये॑व । \newline
35. उ॒त्त॒र॒त इत्यु॑त् - त॒र॒तः । \newline
36. ओ॒ज॒ स्व्ये॑वैवौज॒ स्व्यो॑ज॒ स्व्ये॑व भ॑वति भव त्ये॒वौज॒ स्व्यो॑ज॒ स्व्ये॑व भ॑वति । \newline
37. ए॒व भ॑वति भव त्ये॒वैव भ॑व॒त्या भ॑व त्ये॒वैव भ॑व॒त्या । \newline
38. भ॒व॒त्या भ॑वति भव॒त्या ऽस्या॒स्या भ॑वति भव॒त्या ऽस्य॑ । \newline
39. आ ऽस्या॒स्या ऽस्यौ॑ज॒ स्व्यो॑ज॒ स्व्य॑स्या ऽस्यौ॑ज॒स्वी । \newline
40. अ॒स्यौ॒ज॒ स्व्यो॑ज॒ स्व्य॑स्या स्यौज॒स्वी जा॑यते जायत ओज॒ स्व्य॑स्या स्यौज॒स्वी जा॑यते । \newline
41. ओ॒ज॒स्वी जा॑यते जायत ओज॒ स्व्यो॑ज॒स्वी जा॑यते॒ ऽर्को᳚ ऽर्को जा॑यत ओज॒ स्व्यो॑ज॒स्वी जा॑यते॒ ऽर्कः । \newline
42. जा॒य॒ते॒ ऽर्को᳚ ऽर्को जा॑यते जायते॒ ऽर्को वै वा अ॒र्को जा॑यते जायते॒ ऽर्को वै । \newline
43. अ॒र्को वै वा अ॒र्को᳚ ऽर्को वा ए॒ष ए॒ष वा अ॒र्को᳚ ऽर्को वा ए॒षः । \newline
44. वा ए॒ष ए॒ष वै वा ए॒ष यद् यदे॒ष वै वा ए॒ष यत् । \newline
45. ए॒ष यद् यदे॒ष ए॒ष यद॒ग्नि र॒ग्निर् यदे॒ष ए॒ष यद॒ग्निः । \newline
46. यद॒ग्नि र॒ग्निर् यद् यद॒ग्नि स्तस्य॒ तस्या॒ग्निर् यद् यद॒ग्नि स्तस्य॑ । \newline
47. अ॒ग्नि स्तस्य॒ तस्या॒ ग्नि र॒ग्नि स्तस्यै॒त दे॒तत् तस्या॒ ग्नि र॒ग्नि स्तस्यै॒तत् । \newline
48. तस्यै॒त दे॒तत् तस्य॒ तस्यै॒त दे॒वै वैतत् तस्य॒ तस्यै॒त दे॒व । \newline
49. ए॒त दे॒वै वैत दे॒त दे॒व स्तो॒त्रꣳ स्तो॒त्र मे॒वैत दे॒त दे॒व स्तो॒त्रम् । \newline
50. ए॒व स्तो॒त्रꣳ स्तो॒त्र मे॒वैव स्तो॒त्र मे॒त दे॒तथ् स्तो॒त्र मे॒वैव स्तो॒त्र मे॒तत् । \newline
51. स्तो॒त्र मे॒त दे॒तथ् स्तो॒त्रꣳ स्तो॒त्र मे॒त च्छ॒स्त्रꣳ श॒स्त्र मे॒तथ् स्तो॒त्रꣳ स्तो॒त्र मे॒त च्छ॒स्त्रम् । \newline
52. ए॒त च्छ॒स्त्रꣳ श॒स्त्र मे॒त दे॒त च्छ॒स्त्रं ॅयद् यच्छ॒स्त्र मे॒त दे॒त च्छ॒स्त्रं ॅयत् । \newline
53. श॒स्त्रं ॅयद् यच्छ॒स्त्रꣳ श॒स्त्रं ॅयदे॒ षैषा यच्छ॒स्त्रꣳ श॒स्त्रं ॅयदे॒षा । \newline
54. यदे॒षैषा यद् यदे॒षा वि॒धा वि॒धैषा यद् यदे॒षा वि॒धा । \newline
55. ए॒षा वि॒धा वि॒धैषैषा वि॒धा वि॑धी॒यते॑ विधी॒यते॑ वि॒धैषैषा वि॒धा वि॑धी॒यते᳚ । \newline
56. वि॒धा वि॑धी॒यते॑ विधी॒यते॑ वि॒धा वि॒धा वि॑धी॒यते॒ ऽर्के᳚ ऽर्के वि॑धी॒यते॑ वि॒धा वि॒धा वि॑धी॒यते॒ ऽर्के । \newline
57. वि॒धेति॑ वि - धा । \newline
\pagebreak
\markright{ TS 5.3.4.7  \hfill https://www.vedavms.in \hfill}

\section{ TS 5.3.4.7 }

\textbf{TS 5.3.4.7 } \newline
\textbf{Samhita Paata} \newline

वि॑धी॒यते॒ऽर्क ए॒व तद॒र्क्य॑मनु॒ वि धी॑य॒ते ऽत्त्यन्न॒माऽस्या᳚न्ना॒दो जा॑यते॒ यस्यै॒षा वि॒धा वि॑धी॒यते॒ य उ॑ चैनामे॒वं ॅवेद॒ सृष्टी॒रुप॑ दधाति यथासृ॒ष्टमे॒वाव॑ रुन्धे॒ न वा इ॒दं दिवा॒ न नक्त॑मासी॒दव्या॑वृत्तं॒ ते दे॒वा ए॒ता व्यु॑ष्टीरपश्य॒न् ता उपा॑दधत॒ ततो॒ वा इ॒दं ( ) ॅव्यौ᳚च्छ॒द्-यस्यै॒ता उ॑पधी॒यन्ते॒ व्ये॑वास्मा॑ उच्छ॒त्यथो॒ तम॑ ए॒वाप॑हते ॥श्पेचिअल् खोर्वै fओर् अनुवाकम्(अ॒ग्ने - र्नृ॒चक्ष॑सां - ज॒नित्रं॑ - मि॒त्र - स्येन्द्र॑स्य॒ -वसू॑ना - मादि॒त्याना॒ - मदि॑त्यै - दे॒वस्य॑ सवि॒तुः - सा॑वि॒त्रव॑ती - ध॒र्त्रो - यावा॑ना-मृभू॒णां - ॅवि॑व॒र्त - श्चतु॑र्दश) \newline

\textbf{Pada Paata} \newline

वि॒धी॒यत॒ इति॑ वि - धी॒यते᳚ । अ॒र्के । ए॒व । तत् । अ॒र्क्य᳚म् । अनु॑ । वीति॑ । धी॒य॒ते॒ । अत्ति॑ । अन्न᳚म् । एति॑ । अ॒स्य॒ । अ॒न्ना॒द इत्य॑न्न - अ॒दः । जा॒य॒ते॒ । यस्य॑ । ए॒षा । वि॒धेति॑ वि - धा । वि॒धी॒यत॒ इति॑ वि - धी॒यते᳚ । यः । उ॒ । च॒ । ए॒ना॒म् । ए॒वम् । वेद॑ । सृष्टीः᳚ । उपेति॑ । द॒धा॒ति॒ । य॒था॒सृ॒ष्टमिति॑ यथा - सृ॒ष्टम् । ए॒व । अवेति॑ । रु॒न्धे॒ । न । वै । इ॒दम् । दिवा᳚ । न । नक्त᳚म् । आ॒सी॒त् । अव्या॑वृत्त॒मित्यवि॑ - आ॒वृ॒त्त॒म् । ते । दे॒वाः । ए॒ताः । व्यु॑ष्टी॒रिति॒ वि - उ॒ष्टीः॒ । अ॒प॒श्य॒न्न् । ताः । उपेति॑ । अ॒द॒ध॒त॒ । ततः॑ । वै । इ॒दम् ( ) । वीति॑ । औ॒च्छ॒त् । यस्य॑ । ए॒ताः । उ॒प॒धी॒यन्त॒ इत्यु॑प - धी॒यन्ते᳚ । वीति॑ । ए॒व । अ॒स्मै॒ । उ॒च्छ॒ति॒ । अथो॒ इति॑ । तमः॑ । ए॒व । अपेति॑ । ह॒ते॒ ॥(अ॒ग्ने - र्नृ॒चक्ष॑सां - ज॒नित्रं॑ - मि॒त्र - स्येन्द्र॑स्य॒ -वसू॑ना - मादि॒त्याना॒ - मदि॑त्यै - दे॒वस्य॑ सवि॒तुः - सा॑वि॒त्रव॑ती - ध॒र्त्रो - यावा॑ना-मृभू॒णां - ॅवि॑व॒र्त - श्चतु॑र्दश)  \newline


\textbf{Krama Paata} \newline

वि॒धी॒यते॒ऽर्के । वि॒धी॒यत॒ इति॑ वि - धी॒यते᳚ । अ॒र्क ए॒व । ए॒व तत् । तद॒र्क्य᳚म् । अ॒र्क्य॑मनु॑ । अनु॒ वि । वि धी॑यते । धी॒य॒तेऽत्ति॑ । अत्यन्न᳚म् । अन्न॒मा । आऽस्य॑ । अ॒स्या॒न्ना॒दः । अ॒न्ना॒दो जा॑यते । अ॒न्ना॒द इत्य॑न्न - अ॒दः । जा॒य॒ते॒ यस्य॑ । यस्यै॒षा । ए॒षा वि॒धा । वि॒धा वि॑धी॒यते᳚ । वि॒धेति॑ वि - धा । वि॒धि॒यते॒ यः । वि॒धी॒यत॒ इति॑ वि - धी॒यते᳚ । य उ॑ । उ॒ च॒ । चै॒ना॒म् । ए॒ना॒मे॒वम् । ए॒वम् ॅवेद॑ । वेद॒ सृष्टीः᳚ । सृष्टी॒रुप॑ । उप॑ दधाति । द॒धा॒ति॒ य॒था॒सृ॒ष्टम् । य॒था॒सृ॒ष्टमे॒व । य॒था॒सृ॒ष्टमिति॑ यथा - सृ॒ष्टम् । ए॒वाव॑ । अव॑ रुन्धे । रु॒न्धे॒ न । न वै । वा इ॒दम् । इ॒दम् दिवा᳚ । दिवा॒ न । न नक्त᳚म् । नक्त॑मासीत् । आ॒सी॒दव्या॑वृत्तम् । अव्या॑वृत्त॒म् ते । अव्या॑वृत्त॒मित्यवि॑ - आ॒वृ॒त्त॒म् । ते दे॒वाः । दे॒वा ए॒ताः । ए॒ता व्यु॑ष्टीः । व्यु॑ष्टीरपश्यन्न् । व्यु॑ष्टी॒रिति॒ वि - उ॒ष्टीः॒ । अ॒प॒श्य॒न् ताः । ता उप॑ । उपा॑दधत । अ॒द॒ध॒त॒ ततः॑ । ततो॒ वै । वा इ॒दम् ( ) । इ॒दम् ॅवि । व्यौ᳚च्छत् । औ॒च्छ॒द् यस्य॑ । यस्यै॒ताः । ए॒ता उ॑पधी॒यन्ते᳚ । उ॒प॒धी॒यन्ते॒ वि । उ॒प॒धी॒यन्त॒ इत्यु॑प - धी॒यन्ते᳚ । व्ये॑व । ए॒वास्मै᳚ । अ॒स्मा॒ उ॒च्छ॒ति॒ । उ॒च्छ॒त्यथो᳚ । अथो॒ तमः॑ । अथो॒ इत्यथो᳚ । तम॑ ए॒व । ए॒वाप॑ । अप॑ हते । ह॒त॒ इति॑ हते । \newline

\textbf{Jatai Paata} \newline

1. वि॒धी॒यते॒ ऽर्के᳚ ऽर्के वि॑धी॒यते॑ विधी॒यते॒ ऽर्के । \newline
2. वि॒धी॒यत॒ इति॑ वि - धी॒यते᳚ । \newline
3. अ॒र्क ए॒वैवार्के᳚ ऽर्क ए॒व । \newline
4. ए॒व तत् तदे॒ वैव तत् । \newline
5. तद॒र्क्य॑ म॒र्क्य॑म् तत् तद॒र्क्य᳚म् । \newline
6. अ॒र्क्य॑ मन्वन् व॒र्क्य॑ म॒र्क्य॑ मनु॑ । \newline
7. अनु॒ वि व्यन् वनु॒ वि । \newline
8. वि धी॑यते धीयते॒ वि वि धी॑यते । \newline
9. धी॒य॒ते ऽत्त्यत्ति॑ धीयते धीय॒ते ऽत्ति॑ । \newline
10. अत्त्यन्न॒ मन्न॒ मत्त्य त्त्यन्न᳚म् । \newline
11. अन्न॒ मा ऽन्न॒ मन्न॒ मा । \newline
12. आ ऽस्या॒स्या ऽस्य॑ । \newline
13. अ॒स्या॒ न्ना॒दो᳚ ऽन्ना॒दो᳚ ऽस्यास्या न्ना॒दः । \newline
14. अ॒न्ना॒दो जा॑यते जायते ऽन्ना॒दो᳚ ऽन्ना॒दो जा॑यते । \newline
15. अ॒न्ना॒द इत्य॑न्न - अ॒दः । \newline
16. जा॒य॒ते॒ यस्य॒ यस्य॑ जायते जायते॒ यस्य॑ । \newline
17. यस्यै॒षैषा यस्य॒ यस्यै॒षा । \newline
18. ए॒षा वि॒धा वि॒धैषैषा वि॒धा । \newline
19. वि॒धा वि॑धी॒यते॑ विधी॒यते॑ वि॒धा वि॒धा वि॑धी॒यते᳚ । \newline
20. वि॒धेति॑ वि - धा । \newline
21. वि॒धी॒यते॒ यो यो वि॑धी॒यते॑ विधी॒यते॒ यः । \newline
22. वि॒धी॒यत॒ इति॑ वि - धी॒यते᳚ । \newline
23. य उ॑ वु॒ यो य उ॑ । \newline
24. उ॒ च॒ च॒ वु॒ च॒ । \newline
25. चै॒ना॒ मे॒ना॒म् च॒ चै॒ना॒म् । \newline
26. ए॒ना॒ मे॒व मे॒व मे॑ना मेना मे॒वम् । \newline
27. ए॒वं ॅवेद॒ वेदै॒व मे॒वं ॅवेद॑ । \newline
28. वेद॒ सृष्टीः॒ सृष्टी॒र् वेद॒ वेद॒ सृष्टीः᳚ । \newline
29. सृष्टी॒ रुपोप॒ सृष्टीः॒ सृष्टी॒ रुप॑ । \newline
30. उप॑ दधाति दधा॒ त्युपोप॑ दधाति । \newline
31. द॒धा॒ति॒ य॒था॒सृ॒ष्टं ॅय॑थासृ॒ष्टम् द॑धाति दधाति यथासृ॒ष्टम् । \newline
32. य॒था॒सृ॒ष्ट मे॒वैव य॑थासृ॒ष्टं ॅय॑थासृ॒ष्ट मे॒व । \newline
33. य॒था॒सृ॒ष्टमिति॑ यथा - सृ॒ष्टम् । \newline
34. ए॒वावा वै॒वै वाव॑ । \newline
35. अव॑ रुन्धे रु॒न्धे ऽवाव॑ रुन्धे । \newline
36. रु॒न्धे॒ न न रु॑न्धे रुन्धे॒ न । \newline
37. न वै वै न न वै । \newline
38. वा इ॒द मि॒दं ॅवै वा इ॒दम् । \newline
39. इ॒दम् दिवा॒ दिवे॒द मि॒दम् दिवा᳚ । \newline
40. दिवा॒ न न दिवा॒ दिवा॒ न । \newline
41. न नक्त॒म् नक्त॒म् न न नक्त᳚म् । \newline
42. नक्त॑ मासीदासी॒न् नक्त॒म् नक्त॑ मासीत् । \newline
43. आ॒सी॒ दव्या॑वृत्त॒ मव्या॑वृत्त मासी दासी॒ दव्या॑वृत्तम् । \newline
44. अव्या॑वृत्त॒म् ते ते ऽव्या॑वृत्त॒ मव्या॑वृत्त॒म् ते । \newline
45. अव्या॑वृत्त॒मित्यवि॑ - आ॒वृ॒त्त॒म् । \newline
46. ते दे॒वा दे॒वा स्ते ते दे॒वाः । \newline
47. दे॒वा ए॒ता ए॒ता दे॒वा दे॒वा ए॒ताः । \newline
48. ए॒ता व्यु॑ष्टी॒र् व्यु॑ष्टी रे॒ता ए॒ता व्यु॑ष्टीः । \newline
49. व्यु॑ष्टी रपश्यन् नपश्य॒न् व्यु॑ष्टी॒र् व्यु॑ष्टी रपश्यन्न् । \newline
50. व्यु॑ष्टी॒रिति॒ वि - उ॒ष्टीः॒ । \newline
51. अ॒प॒श्य॒न् ता स्ता अ॑पश्यन् नपश्य॒न् ताः । \newline
52. ता उपोप॒ ता स्ता उप॑ । \newline
53. उपा॑ दधता दध॒तोपोपा॑ दधत । \newline
54. अ॒द॒ध॒त॒ तत॒ स्ततो॑ ऽदधता दधत॒ ततः॑ । \newline
55. ततो॒ वै वै तत॒ स्ततो॒ वै । \newline
56. वा इ॒द मि॒दं ॅवै वा इ॒दम् । \newline
57. इ॒दं ॅवि वीद मि॒दं ॅवि । \newline
58. व्यौ᳚च्छ दौच्छ॒द् वि व्यौ᳚च्छत् । \newline
59. औ॒च्छ॒द् यस्य॒ यस्यौ᳚च्छ दौच्छ॒द् यस्य॑ । \newline
60. यस्यै॒ता ए॒ता यस्य॒ यस्यै॒ताः । \newline
61. ए॒ता उ॑पधी॒यन्त॑ उपधी॒यन्त॑ ए॒ता ए॒ता उ॑पधी॒यन्ते᳚ । \newline
62. उ॒प॒धी॒यन्ते॒ वि व्यु॑पधी॒यन्त॑ उपधी॒यन्ते॒ वि । \newline
63. उ॒प॒धी॒यन्त॒ इत्यु॑प - धी॒यन्ते᳚ । \newline
64. व्ये॑वैव वि व्ये॑व । \newline
65. ए॒वास्मा॑ अस्मा ए॒वै वास्मै᳚ । \newline
66. अ॒स्मा॒ उ॒च्छ॒ त्यु॒च्छ॒ त्य॒स्मा॒ अ॒स्मा॒ उ॒च्छ॒ति॒ । \newline
67. उ॒च्छ॒ त्यथो॒ अथो॑ उच्छ त्युच्छ॒ त्यथो᳚ । \newline
68. अथो॒ तम॒ स्तमो ऽथो॒ अथो॒ तमः॑ । \newline
69. अथो॒ इत्यथो᳚ । \newline
70. तम॑ ए॒वैव तम॒ स्तम॑ ए॒व । \newline
71. ए॒वापा पै॒वै वाप॑ । \newline
72. अप॑ हते ह॒ते ऽपाप॑ हते । \newline
73. ह॒त॒ इति॑ हते । \newline

\textbf{Ghana Paata } \newline

1. वि॒धी॒यते॒ ऽर्के᳚ ऽर्के वि॑धी॒यते॑ विधी॒यते॒ ऽर्क ए॒वैवार्के वि॑धी॒यते॑ विधी॒यते॒ ऽर्क ए॒व । \newline
2. वि॒धी॒यत॒ इति॑ वि - धी॒यते᳚ । \newline
3. अ॒र्क ए॒वै वार्के᳚ ऽर्क ए॒व तत् तदे॒ वार्के᳚ ऽर्क ए॒व तत् । \newline
4. ए॒व तत् तदे॒ वैव तद॒र्क्य॑ म॒र्क्य॑म् तदे॒वैव तद॒र्क्य᳚म् । \newline
5. तद॒र्क्य॑ म॒र्क्य॑म् तत् तद॒र्क्य॑ मन्वन् व॒र्क्य॑म् तत् तद॒र्क्य॑ मनु॑ । \newline
6. अ॒र्क्य॑ मन्वन् व॒र्क्य॑ म॒र्क्य॑ मनु॒ वि व्यन् व॒र्क्य॑ म॒र्क्य॑ मनु॒ वि । \newline
7. अनु॒ वि व्यन् वनु॒ वि धी॑यते धीयते॒ व्यन् वनु॒ वि धी॑यते । \newline
8. वि धी॑यते धीयते॒ वि वि धी॑य॒ते ऽत्त्यत्ति॑ धीयते॒ वि वि धी॑य॒ते ऽत्ति॑ । \newline
9. धी॒य॒ते ऽत्त्यत्ति॑ धीयते धीय॒ते ऽत्त्यन्न॒ मन्न॒ मत्ति॑ धीयते धीय॒ते ऽत्त्यन्न᳚म् । \newline
10. अत्त्यन्न॒ मन्न॒ मत्त्य त्त्यन्न॒ मा ऽन्न॒ मत्त्य त्त्यन्न॒ मा । \newline
11. अन्न॒ मा ऽन्न॒ मन्न॒ मा ऽस्या॒स्या ऽन्न॒ मन्न॒ मा ऽस्य॑ । \newline
12. आ ऽस्या॒स्या ऽस्या᳚ न्ना॒दो᳚ ऽन्ना॒दो᳚ ऽस्या ऽस्या᳚ न्ना॒दः । \newline
13. अ॒स्या॒ न्ना॒दो᳚ ऽन्ना॒दो᳚ ऽस्यास्या न्ना॒दो जा॑यते जायते ऽन्ना॒दो᳚ ऽस्यास्या न्ना॒दो जा॑यते । \newline
14. अ॒न्ना॒दो जा॑यते जायते ऽन्ना॒दो᳚ ऽन्ना॒दो जा॑यते॒ यस्य॒ यस्य॑ जायते ऽन्ना॒दो᳚ ऽन्ना॒दो जा॑यते॒ यस्य॑ । \newline
15. अ॒न्ना॒द इत्य॑न्न - अ॒दः । \newline
16. जा॒य॒ते॒ यस्य॒ यस्य॑ जायते जायते॒ यस्यै॒षैषा यस्य॑ जायते जायते॒ यस्यै॒षा । \newline
17. यस्यै॒षैषा यस्य॒ यस्यै॒षा वि॒धा वि॒धैषा यस्य॒ यस्यै॒षा वि॒धा । \newline
18. ए॒षा वि॒धा वि॒धैषैषा वि॒धा वि॑धी॒यते॑ विधी॒यते॑ वि॒धैषैषा वि॒धा वि॑धी॒यते᳚ । \newline
19. वि॒धा वि॑धी॒यते॑ विधी॒यते॑ वि॒धा वि॒धा वि॑धी॒यते॒ यो यो वि॑धी॒यते॑ वि॒धा वि॒धा वि॑धी॒यते॒ यः । \newline
20. वि॒धेति॑ वि - धा । \newline
21. वि॒धी॒यते॒ यो यो वि॑धी॒यते॑ विधी॒यते॒ य उ॑ वु॒ यो वि॑धी॒यते॑ विधी॒यते॒ य उ॑ । \newline
22. वि॒धी॒यत॒ इति॑ वि - धी॒यते᳚ । \newline
23. य उ॑ वु॒ यो य उ॑ च चो॒ यो य उ॑ च । \newline
24. उ॒ च॒ च॒ वु॒ चै॒ना॒ मे॒ना॒म् च॒ वु॒ चै॒ना॒म् । \newline
25. चै॒ना॒ मे॒ना॒म् च॒ चै॒ना॒ मे॒व मे॒व मे॑नाम् च चैना मे॒वम् । \newline
26. ए॒ना॒ मे॒व मे॒व मे॑ना मेना मे॒वं ॅवेद॒ वेदै॒व मे॑ना मेना मे॒वं ॅवेद॑ । \newline
27. ए॒वं ॅवेद॒ वेदै॒व मे॒वं ॅवेद॒ सृष्टीः॒ सृष्टी॒र् वेदै॒व मे॒वं ॅवेद॒ सृष्टीः᳚ । \newline
28. वेद॒ सृष्टीः॒ सृष्टी॒र् वेद॒ वेद॒ सृष्टी॒ रुपोप॒ सृष्टी॒र् वेद॒ वेद॒ सृष्टी॒ रुप॑ । \newline
29. सृष्टी॒ रुपोप॒ सृष्टीः॒ सृष्टी॒ रुप॑ दधाति दधा॒ त्युप॒ सृष्टीः॒ सृष्टी॒ रुप॑ दधाति । \newline
30. उप॑ दधाति दधा॒ त्युपोप॑ दधाति यथासृ॒ष्टं ॅय॑थासृ॒ष्टम् द॑धा॒ त्युपोप॑ दधाति यथासृ॒ष्टम् । \newline
31. द॒धा॒ति॒ य॒था॒सृ॒ष्टं ॅय॑थासृ॒ष्टम् द॑धाति दधाति यथासृ॒ष्ट मे॒वैव य॑थासृ॒ष्टम् द॑धाति दधाति यथासृ॒ष्ट मे॒व । \newline
32. य॒था॒सृ॒ष्ट मे॒वैव य॑थासृ॒ष्टं ॅय॑थासृ॒ष्ट मे॒वावा वै॒व य॑थासृ॒ष्टं ॅय॑थासृ॒ष्ट मे॒वाव॑ । \newline
33. य॒था॒सृ॒ष्टमिति॑ यथा - सृ॒ष्टम् । \newline
34. ए॒वावा वै॒वै वाव॑ रुन्धे रु॒न्धे ऽवै॒वै वाव॑ रुन्धे । \newline
35. अव॑ रुन्धे रु॒न्धे ऽवाव॑ रुन्धे॒ न न रु॒न्धे ऽवाव॑ रुन्धे॒ न । \newline
36. रु॒न्धे॒ न न रु॑न्धे रुन्धे॒ न वै वै न रु॑न्धे रुन्धे॒ न वै । \newline
37. न वै वै न न वा इ॒द मि॒दं ॅवै न न वा इ॒दम् । \newline
38. वा इ॒द मि॒दं ॅवै वा इ॒दम् दिवा॒ दिवे॒दं ॅवै वा इ॒दम् दिवा᳚ । \newline
39. इ॒दम् दिवा॒ दिवे॒द मि॒दम् दिवा॒ न न दिवे॒द मि॒दम् दिवा॒ न । \newline
40. दिवा॒ न न दिवा॒ दिवा॒ न नक्त॒म् नक्त॒म् न दिवा॒ दिवा॒ न नक्त᳚म् । \newline
41. न नक्त॒म् नक्त॒म् न न नक्त॑ मासी दासी॒न् नक्त॒म् न न नक्त॑ मासीत् । \newline
42. नक्त॑ मासी दासी॒न् नक्त॒न् नक्त॑ मासी॒ दव्या॑वृत्त॒ मव्या॑वृत्त मासी॒न् नक्त॒म् नक्त॑ मासी॒ दव्या॑वृत्तम् । \newline
43. आ॒सी॒ दव्या॑वृत्त॒ मव्या॑वृत्त मासी दासी॒ दव्या॑वृत्त॒म् ते ते ऽव्या॑वृत्त मासी
दासी॒ दव्या॑वृत्त॒म् ते । \newline
44. अव्या॑वृत्त॒म् ते ते ऽव्या॑वृत्त॒ मव्या॑वृत्त॒म् ते दे॒वा दे॒वा स्ते ऽव्या॑वृत्त॒ मव्या॑वृत्त॒म् ते दे॒वाः । \newline
45. अव्या॑वृत्त॒मित्यवि॑ - आ॒वृ॒त्त॒म् । \newline
46. ते दे॒वा दे॒वा स्ते ते दे॒वा ए॒ता ए॒ता दे॒वा स्ते ते दे॒वा ए॒ताः । \newline
47. दे॒वा ए॒ता ए॒ता दे॒वा दे॒वा ए॒ता व्यु॑ष्टी॒र् व्यु॑ष्टी रे॒ता दे॒वा दे॒वा ए॒ता व्यु॑ष्टीः । \newline
48. ए॒ता व्यु॑ष्टी॒र् व्यु॑ष्टी रे॒ता ए॒ता व्यु॑ष्टी रपश्यन् नपश्य॒न् व्यु॑ष्टी रे॒ता ए॒ता व्यु॑ष्टी रपश्यन्न् । \newline
49. व्यु॑ष्टी रपश्यन् नपश्य॒न् व्यु॑ष्टी॒र् व्यु॑ष्टी रपश्य॒न् ता स्ता अ॑पश्य॒न् व्यु॑ष्टी॒र् व्यु॑ष्टी रपश्य॒न् ताः । \newline
50. व्यु॑ष्टी॒रिति॒ वि - उ॒ष्टीः॒ । \newline
51. अ॒प॒श्य॒न् ता स्ता अ॑पश्यन् नपश्य॒न् ता उपोप॒ ता अ॑पश्यन् नपश्य॒न् ता उप॑ । \newline
52. ता उपोप॒ ता स्ता उपा॑दधता दध॒तोप॒ ता स्ता उपा॑दधत । \newline
53. उपा॑दधता दध॒ तोपोपा॑ दधत॒ तत॒ स्ततो॑ ऽदध॒ तोपोपा॑ दधत॒ ततः॑ । \newline
54. अ॒द॒ध॒त॒ तत॒ स्ततो॑ ऽदधता दधत॒ ततो॒ वै वै ततो॑ ऽदधता दधत॒ ततो॒ वै । \newline
55. ततो॒ वै वै तत॒ स्ततो॒ वा इ॒द मि॒दं ॅवै तत॒ स्ततो॒ वा इ॒दम् । \newline
56. वा इ॒द मि॒दं ॅवै वा इ॒दं ॅवि वीदं ॅवै वा इ॒दं ॅवि । \newline
57. इ॒दं ॅवि वीद मि॒दं ॅव्यौ᳚च्छ दौच्छ॒द् वीद मि॒दं ॅव्यौ᳚च्छत् । \newline
58. व्यौ᳚च्छ दौच्छ॒द् वि व्यौ᳚च्छ॒द् यस्य॒ यस्यौ᳚च्छ॒द् वि व्यौ᳚च्छ॒द् यस्य॑ । \newline
59. औ॒च्छ॒द् यस्य॒ यस्यौ᳚च्छ दौच्छ॒द् यस्यै॒ता ए॒ता यस्यौ᳚च्छ दौच्छ॒द् यस्यै॒ताः । \newline
60. यस्यै॒ता ए॒ता यस्य॒ यस्यै॒ता उ॑पधी॒यन्त॑ उपधी॒यन्त॑ ए॒ता यस्य॒ यस्यै॒ता उ॑पधी॒यन्ते᳚ । \newline
61. ए॒ता उ॑पधी॒यन्त॑ उपधी॒यन्त॑ ए॒ता ए॒ता उ॑पधी॒यन्ते॒ वि व्यु॑पधी॒यन्त॑ ए॒ता ए॒ता उ॑पधी॒यन्ते॒ वि । \newline
62. उ॒प॒धी॒यन्ते॒ वि व्यु॑पधी॒यन्त॑ उपधी॒यन्ते॒ व्ये॑वैव व्यु॑पधी॒यन्त॑ उपधी॒यन्ते॒ व्ये॑व । \newline
63. उ॒प॒धी॒यन्त॒ इत्यु॑प - धी॒यन्ते᳚ । \newline
64. व्ये॑वैव वि व्ये॑वास्मा॑ अस्मा ए॒व वि व्ये॑वास्मै᳚ । \newline
65. ए॒वास्मा॑ अस्मा ए॒वै वास्मा॑ उच्छ त्युच्छ त्यस्मा ए॒वै वास्मा॑ उच्छति । \newline
66. अ॒स्मा॒ उ॒च्छ॒ त्यु॒च्छ॒ त्य॒स्मा॒ अ॒स्मा॒ उ॒च्छ॒ त्यथो॒ अथो॑ उच्छ त्यस्मा अस्मा उच्छ॒ त्यथो᳚ । \newline
67. उ॒च्छ॒ त्यथो॒ अथो॑ उच्छ त्युच्छ॒ त्यथो॒ तम॒ स्तमो ऽथो॑ उच्छ त्युच्छ॒ त्यथो॒ तमः॑ । \newline
68. अथो॒ तम॒ स्तमो ऽथो॒ अथो॒ तम॑ ए॒वैव तमो ऽथो॒ अथो॒ तम॑ ए॒व । \newline
69. अथो॒ इत्यथो᳚ । \newline
70. तम॑ ए॒वैव तम॒ स्तम॑ ए॒वापापै॒व तम॒ स्तम॑ ए॒वाप॑ । \newline
71. ए॒वापा पै॒वैवाप॑ हते ह॒ते ऽपै॒वैवाप॑ हते । \newline
72. अप॑ हते ह॒ते ऽपाप॑ हते । \newline
73. ह॒त॒ इति॑ हते । \newline
\pagebreak
\markright{ TS 5.3.5.1  \hfill https://www.vedavms.in \hfill}

\section{ TS 5.3.5.1 }

\textbf{TS 5.3.5.1 } \newline
\textbf{Samhita Paata} \newline

अग्ने॑ जा॒तान् प्रणु॑दा नः स॒पत्ना॒निति॑ पु॒रस्ता॒दुप॑ दधाति जा॒ताने॒व भ्रातृ॑व्या॒न् प्रणु॑दते॒ सह॑सा जा॒तानिति॑ प॒श्चाज्ज॑नि॒ष्यमा॑णाने॒व प्रति॑ नुदते चतुश्चत्वारिꣳ॒॒शः स्तोम॒ इति॑ दक्षिण॒तो ब्र॑ह्मवर्च॒सं ॅवै च॑तुश्चत्वारिꣳ॒॒शो ब्र॑ह्मवर्च॒समे॒व द॑क्षिण॒तो ध॑त्ते॒ तस्मा॒द् दक्षि॒णोऽर्द्धो᳚ ब्रह्मवर्च॒सित॑रः षोड॒शः स्तोम॒ इत्यु॑त्तर॒त ओजो॒ वै षो॑ड॒श ओज॑ ए॒वोत्त॑र॒तो ध॑त्ते॒ तस्मा॑ - [  ] \newline

\textbf{Pada Paata} \newline

अग्ने᳚ । जा॒तान् । प्रेति॑ । नु॒द॒ । नः॒ । स॒पत्नान्॑ । इति॑ । पु॒रस्ता᳚त् । उपेति॑ । द॒धा॒ति॒ । जा॒तान् । ए॒व । भ्रातृ॑व्यान् । प्रेति॑ । नु॒द॒ते॒ । सह॑सा । जा॒तान् । इति॑ । प॒श्चात् । ज॒नि॒ष्यमा॑णान् । ए॒व । प्रतीति॑ । नु॒द॒ते॒ । च॒तु॒श्च॒त्वा॒रिꣳ॒॒श इति॑ चतुः - च॒त्वा॒रिꣳ॒॒शः । स्तोमः॑ । इति॑ । द॒क्षि॒ण॒तः । ब्र॒ह्म॒व॒र्च॒समिति॑ ब्रह्म - व॒र्च॒सम् । वै । च॒तु॒श्च॒त्वा॒रिꣳ॒॒श इति॑ चतुः - च॒त्वा॒रिꣳ॒॒शः । ब्र॒ह्म॒व॒र्च॒समिति॑ ब्रह्म - व॒र्च॒सम् । ए॒व । द॒क्षि॒ण॒तः । ध॒त्ते॒ । तस्मा᳚त् । दक्षि॑णः । अद्‌र्धः॑ । ब्र॒ह्म॒व॒र्च॒सित॑र॒ इति॑ ब्रह्मवर्च॒सि - त॒रः॒ । षो॒ड॒शः । स्तोमः॑ । इति॑ । उ॒त्त॒र॒त इत्यु॑त् - त॒र॒तः । ओजः॑ । वै । षो॒ड॒शः । ओजः॑ । ए॒व । उ॒त्त॒र॒त इत्यु॑त् - त॒र॒तः । ध॒त्ते॒ । तस्मा᳚त् ।  \newline


\textbf{Krama Paata} \newline

अग्ने॑ जा॒तान् । जा॒तान् प्र । प्र णु॑द । नु॒दा॒ नः॒ । नः॒ स॒पत्नान्॑ । स॒पत्ना॒निति॑ । इति॑ पु॒रस्ता᳚त् । पु॒रस्ता॒दुप॑ । उप॑ दधाति । द॒धा॒ति॒ जा॒तान् । जा॒ताने॒व । ए॒व भ्रातृ॑व्यान् । भ्रातृ॑व्या॒न् प्र । प्र णु॑दते । नु॒द॒ते॒ सह॑सा । सह॑सा जा॒तान् । जा॒तानिति॑ । इति॑ प॒श्चात् । प॒श्चाज् ज॑नि॒ष्यमा॑णान् । ज॒नि॒ष्यमा॑णाने॒व । ए॒व प्रति॑ । प्रति॑ नुदते । नु॒द॒ते॒ च॒तु॒श्च॒त्वा॒रिꣳ॒॒शः । च॒तु॒श्च॒त्वा॒रिꣳ॒॒शः स्तोमः॑ । च॒तु॒श्च॒त्वा॒रिꣳ॒॒श इति॑ चतुः - च॒त्वा॒रिꣳ॒॒शः । स्तोम॒ इति॑ । इति॑ दक्षिण॒तः । द॒क्षि॒ण॒तो ब्र॑ह्मवर्च॒सम् । ब्र॒ह्म॒व॒र्च॒सम् ॅवै । ब्र॒ह्म॒व॒र्च॒समिति॑ ब्रह्म - व॒र्च॒सम् । वै च॑तुश्चत्वारिꣳ॒॒शः । च॒तु॒श्च॒त्वा॒रिꣳ॒॒शो ब्र॑ह्मवर्च॒सम् । च॒तु॒श्च॒त्वा॒रिꣳ॒॒श इति॑ चतुः - च॒त्वा॒रिꣳ॒॒शः । ब्र॒ह्म॒व॒र्च॒समे॒व । ब्र॒ह्म॒व॒र्च॒समिति॑ ब्रह्म - व॒र्च॒सम् । ए॒व द॑क्षिण॒तः । द॒क्षि॒ण॒तो ध॑त्ते । ध॒त्ते॒ तस्मा᳚त् । तस्मा॒द् दक्षि॑णः । दक्षि॒णोऽर्द्धः॑ । अर्द्धो᳚ ब्रह्मवर्च॒सित॑रः । ब्र॒ह्म॒व॒र्च॒सित॑रः षोड॒शः । ब्र॒ह्म॒व॒र्च॒सित॑र॒ इति॑ ब्रह्मवर्च॒सि - त॒रः॒ । षो॒ड॒शः स्तोमः॑ । स्तोम॒ इति॑ । इत्यु॑त्तर॒तः । उ॒त्त॒र॒त ओजः॑ । उ॒त्त॒र॒त इत्यु॑त् - त॒र॒तः । ओजो॒ वै । वै षो॑ड॒शः । षो॒ड॒श ओजः॑ । ओज॑ ए॒व । ए॒वोत्त॑र॒तः । उ॒त्त॒र॒तो ध॑त्ते । उ॒त्त॒र॒त इत्यु॑त् - त॒र॒तः । ध॒त्ते॒ तस्मा᳚त् । तस्मा॑दुत्तरतोभिप्रया॒यी \newline

\textbf{Jatai Paata} \newline

1. अग्ने॑ जा॒तान् जा॒ता नग्ने ऽग्ने॑ जा॒तान् । \newline
2. जा॒तान् प्र प्र जा॒तान् जा॒तान् प्र । \newline
3. प्र णु॑द नुद॒ प्र प्र णु॑द । \newline
4. नु॒दा॒ नो॒ नो॒ नु॒द॒ नु॒दा॒ नः॒ । \newline
5. नः॒ स॒पत्ना᳚न् थ्स॒पत्ना᳚न् नो नः स॒पत्नान्॑ । \newline
6. स॒पत्ना॒ नितीति॑ स॒पत्ना᳚न् थ्स॒पत्ना॒ निति॑ । \newline
7. इति॑ पु॒रस्ता᳚त् पु॒रस्ता॒ दितीति॑ पु॒रस्ता᳚त् । \newline
8. पु॒रस्ता॒ दुपोप॑ पु॒रस्ता᳚त् पु॒रस्ता॒ दुप॑ । \newline
9. उप॑ दधाति दधा॒ त्युपोप॑ दधाति । \newline
10. द॒धा॒ति॒ जा॒तान् जा॒तान् द॑धाति दधाति जा॒तान् । \newline
11. जा॒ता ने॒वैव जा॒तान् जा॒ता ने॒व । \newline
12. ए॒व भ्रातृ॑व्या॒न् भ्रातृ॑व्या ने॒वैव भ्रातृ॑व्यान् । \newline
13. भ्रातृ॑व्या॒न् प्र प्र भ्रातृ॑व्या॒न् भ्रातृ॑व्या॒न् प्र । \newline
14. प्र णु॑दते नुदते॒ प्र प्र णु॑दते । \newline
15. नु॒द॒ते॒ सह॑सा॒ सह॑सा नुदते नुदते॒ सह॑सा । \newline
16. सह॑सा जा॒तान् जा॒तान् थ्सह॑सा॒ सह॑सा जा॒तान् । \newline
17. जा॒ता नितीति॑ जा॒तान् जा॒ता निति॑ । \newline
18. इति॑ प॒श्चात् प॒श्चा दितीति॑ प॒श्चात् । \newline
19. प॒श्चाज् ज॑नि॒ष्यमा॑णान् जनि॒ष्यमा॑णान् प॒श्चात् प॒श्चाज् ज॑नि॒ष्यमा॑णान् । \newline
20. ज॒नि॒ष्यमा॑णा ने॒वैव ज॑नि॒ष्यमा॑णान् जनि॒ष्यमा॑णा ने॒व । \newline
21. ए॒व प्रति॒ प्रत्ये॒ वैव प्रति॑ । \newline
22. प्रति॑ नुदते नुदते॒ प्रति॒ प्रति॑ नुदते । \newline
23. नु॒द॒ते॒ च॒तु॒श्च॒त्वा॒रिꣳ॒॒श श्च॑तुश्चत्वारिꣳ॒॒शो नु॑दते नुदते चतुश्चत्वारिꣳ॒॒शः । \newline
24. च॒तु॒श्च॒त्वा॒रिꣳ॒॒शः स्तोमः॒ स्तोम॑ श्चतुश्चत्वारिꣳ॒॒श श्च॑तुश्चत्वारिꣳ॒॒शः स्तोमः॑ । \newline
25. च॒तु॒श्च॒त्वा॒रिꣳ॒॒श इति॑ चतुः - च॒त्वा॒रिꣳ॒॒शः । \newline
26. स्तोम॒ इतीति॒ स्तोमः॒ स्तोम॒ इति॑ । \newline
27. इति॑ दक्षिण॒तो द॑क्षिण॒त इतीति॑ दक्षिण॒तः । \newline
28. द॒क्षि॒ण॒तो ब्र॑ह्मवर्च॒सम् ब्र॑ह्मवर्च॒सम् द॑क्षिण॒तो द॑क्षिण॒तो ब्र॑ह्मवर्च॒सम् । \newline
29. ब्र॒ह्म॒व॒र्च॒सं ॅवै वै ब्र॑ह्मवर्च॒सम् ब्र॑ह्मवर्च॒सं ॅवै । \newline
30. ब्र॒ह्म॒व॒र्च॒समिति॑ ब्रह्म - व॒र्च॒सम् । \newline
31. वै च॑तुश्चत्वारिꣳ॒॒श श्च॑तुश्चत्वारिꣳ॒॒शो वै वै च॑तुश्चत्वारिꣳ॒॒शः । \newline
32. च॒तु॒श्च॒त्वा॒रिꣳ॒॒शो ब्र॑ह्मवर्च॒सम् ब्र॑ह्मवर्च॒सम् च॑तुश्चत्वारिꣳ॒॒श श्च॑तुश्चत्वारिꣳ॒॒शो ब्र॑ह्मवर्च॒सम् । \newline
33. च॒तु॒श्च॒त्वा॒रिꣳ॒॒श इति॑ चतुः - च॒त्वा॒रिꣳ॒॒शः । \newline
34. ब्र॒ह्म॒व॒र्च॒स मे॒वैव ब्र॑ह्मवर्च॒सम् ब्र॑ह्मवर्च॒स मे॒व । \newline
35. ब्र॒ह्म॒व॒र्च॒समिति॑ ब्रह्म - व॒र्च॒सम् । \newline
36. ए॒व द॑क्षिण॒तो द॑क्षिण॒त ए॒वैव द॑क्षिण॒तः । \newline
37. द॒क्षि॒ण॒तो ध॑त्ते धत्ते दक्षिण॒तो द॑क्षिण॒तो ध॑त्ते । \newline
38. ध॒त्ते॒ तस्मा॒त् तस्मा᳚द् धत्ते धत्ते॒ तस्मा᳚त् । \newline
39. तस्मा॒द् दक्षि॑णो॒ दक्षि॑ण॒ स्तस्मा॒त् तस्मा॒द् दक्षि॑णः । \newline
40. दक्षि॒णो ऽर्द्धो ऽर्द्धो॒ दक्षि॑णो॒ दक्षि॒णो ऽर्द्धः॑ । \newline
41. अर्द्धो᳚ ब्रह्मवर्च॒सित॑रो ब्रह्मवर्च॒सित॒रो ऽर्द्धो ऽर्द्धो᳚ ब्रह्मवर्च॒सित॑रः । \newline
42. ब्र॒ह्म॒व॒र्च॒सित॑र ष्षोड॒श ष्षो॑ड॒शो ब्र॑ह्मवर्च॒सित॑रो ब्रह्मवर्च॒सित॑र ष्षोड॒शः । \newline
43. ब्र॒ह्म॒व॒र्च॒सित॑र॒ इति॑ ब्रह्मवर्च॒सि - त॒रः॒ । \newline
44. षो॒ड॒शः स्तोमः॒ स्तोम॑ ष्षोड॒श ष्षो॑ड॒शः स्तोमः॑ । \newline
45. स्तोम॒ इतीति॒ स्तोमः॒ स्तोम॒ इति॑ । \newline
46. इत्यु॑त्तर॒त उ॑त्तर॒त इती त्यु॑त्तर॒तः । \newline
47. उ॒त्त॒र॒त ओज॒ ओज॑ उत्तर॒त उ॑त्तर॒त ओजः॑ । \newline
48. उ॒त्त॒र॒त इत्यु॑त् - त॒र॒तः । \newline
49. ओजो॒ वै वा ओज॒ ओजो॒ वै । \newline
50. वै षो॑ड॒श ष्षो॑ड॒शो वै वै षो॑ड॒शः । \newline
51. षो॒ड॒श ओज॒ ओज॑ ष्षोड॒श ष्षो॑ड॒श ओजः॑ । \newline
52. ओज॑ ए॒वै वौज॒ ओज॑ ए॒व । \newline
53. ए॒वोत्त॑र॒त उ॑त्तर॒त ए॒वैवो त्त॑र॒तः । \newline
54. उ॒त्त॒र॒तो ध॑त्ते धत्त उत्तर॒त उ॑त्तर॒तो ध॑त्ते । \newline
55. उ॒त्त॒र॒त इत्यु॑त् - त॒र॒तः । \newline
56. ध॒त्ते॒ तस्मा॒त् तस्मा᳚द् धत्ते धत्ते॒ तस्मा᳚त् । \newline
57. तस्मा॑ दुत्तरतोभिप्रया॒य्यु॑ त्तरतोभिप्रया॒यी तस्मा॒त् तस्मा॑ दुत्तरतोभिप्रया॒यी । \newline

\textbf{Ghana Paata } \newline

1. अग्ने॑ जा॒तान् जा॒ता नग्ने ऽग्ने॑ जा॒तान् प्र प्र जा॒ता नग्ने ऽग्ने॑ जा॒तान् प्र । \newline
2. जा॒तान् प्र प्र जा॒तान् जा॒तान् प्र णु॑द नुद॒ प्र जा॒तान् जा॒तान् प्र णु॑द । \newline
3. प्र णु॑द नुद॒ प्र प्र णु॑दा नो नो नुद॒ प्र प्र णु॑दा नः । \newline
4. नु॒दा॒ नो॒ नो॒ नु॒द॒ नु॒दा॒ नः॒ स॒पत्ना᳚न् थ्स॒पत्ना᳚न् नो नुद नुदा नः स॒पत्नान्॑ । \newline
5. नः॒ स॒पत्ना᳚न् थ्स॒पत्ना᳚न् नो नः स॒पत्ना॒ नितीति॑ स॒पत्ना᳚न् नो नः स॒पत्ना॒ निति॑ । \newline
6. स॒पत्ना॒ नितीति॑ स॒पत्ना᳚न् थ्स॒पत्ना॒ निति॑ पु॒रस्ता᳚त् पु॒रस्ता॒ दिति॑ स॒पत्ना᳚न् थ्स॒पत्ना॒ निति॑ पु॒रस्ता᳚त् । \newline
7. इति॑ पु॒रस्ता᳚त् पु॒रस्ता॒ दितीति॑ पु॒रस्ता॒ दुपोप॑ पु॒रस्ता॒ दितीति॑ पु॒रस्ता॒ दुप॑ । \newline
8. पु॒रस्ता॒ दुपोप॑ पु॒रस्ता᳚त् पु॒रस्ता॒ दुप॑ दधाति दधा॒ त्युप॑ पु॒रस्ता᳚त् पु॒रस्ता॒ दुप॑ दधाति । \newline
9. उप॑ दधाति दधा॒ त्युपोप॑ दधाति जा॒तान् जा॒तान् द॑धा॒ त्युपोप॑ दधाति जा॒तान् । \newline
10. द॒धा॒ति॒ जा॒तान् जा॒तान् द॑धाति दधाति जा॒ता ने॒वैव जा॒तान् द॑धाति दधाति जा॒ता ने॒व । \newline
11. जा॒ता ने॒वैव जा॒तान् जा॒ता ने॒व भ्रातृ॑व्या॒न् भ्रातृ॑व्या ने॒व जा॒तान् जा॒ता ने॒व भ्रातृ॑व्यान् । \newline
12. ए॒व भ्रातृ॑व्या॒न् भ्रातृ॑व्या ने॒वैव भ्रातृ॑व्या॒न् प्र प्र भ्रातृ॑व्या ने॒वैव भ्रातृ॑व्या॒न् प्र । \newline
13. भ्रातृ॑व्या॒न् प्र प्र भ्रातृ॑व्या॒न् भ्रातृ॑व्या॒न् प्र णु॑दते नुदते॒ प्र भ्रातृ॑व्या॒न् भ्रातृ॑व्या॒न् प्र णु॑दते । \newline
14. प्र णु॑दते नुदते॒ प्र प्र णु॑दते॒ सह॑सा॒ सह॑सा नुदते॒ प्र प्र णु॑दते॒ सह॑सा । \newline
15. नु॒द॒ते॒ सह॑सा॒ सह॑सा नुदते नुदते॒ सह॑सा जा॒तान् जा॒तान् थ्सह॑सा नुदते नुदते॒ सह॑सा जा॒तान् । \newline
16. सह॑सा जा॒तान् जा॒तान् थ्सह॑सा॒ सह॑सा जा॒ता नितीति॑ जा॒तान् थ्सह॑सा॒ सह॑सा जा॒ता निति॑ । \newline
17. जा॒ता नितीति॑ जा॒तान् जा॒ता निति॑ प॒श्चात् प॒श्चादिति॑ जा॒तान् जा॒ता निति॑ प॒श्चात् । \newline
18. इति॑ प॒श्चात् प॒श्चा दितीति॑ प॒श्चाज् ज॑नि॒ष्यमा॑णान् जनि॒ष्यमा॑णान् प॒श्चा दितीति॑ प॒श्चाज् ज॑नि॒ष्यमा॑णान् । \newline
19. प॒श्चाज् ज॑नि॒ष्यमा॑णान् जनि॒ष्यमा॑णान् प॒श्चात् प॒श्चाज् ज॑नि॒ष्यमा॑णा ने॒वैव ज॑नि॒ष्यमा॑णान् प॒श्चात् प॒श्चाज् ज॑नि॒ष्यमा॑णा ने॒व । \newline
20. ज॒नि॒ष्यमा॑णा ने॒वैव ज॑नि॒ष्यमा॑णान् जनि॒ष्यमा॑णा ने॒व प्रति॒ प्रत्ये॒व ज॑नि॒ष्यमा॑णान् जनि॒ष्यमा॑णा ने॒व प्रति॑ । \newline
21. ए॒व प्रति॒ प्रत्ये॒ वैव प्रति॑ नुदते नुदते॒ प्रत्ये॒वैव प्रति॑ नुदते । \newline
22. प्रति॑ नुदते नुदते॒ प्रति॒ प्रति॑ नुदते चतुश्चत्वारिꣳ॒॒श श्च॑तुश्चत्वारिꣳ॒॒शो नु॑दते॒ प्रति॒ प्रति॑ नुदते चतुश्चत्वारिꣳ॒॒शः । \newline
23. नु॒द॒ते॒ च॒तु॒श्च॒त्वा॒रिꣳ॒॒श श्च॑तुश्चत्वारिꣳ॒॒शो नु॑दते नुदते चतुश्चत्वारिꣳ॒॒शः स्तोमः॒ स्तोम॑ श्चतुश्चत्वारिꣳ॒॒शो नु॑दते नुदते चतुश्चत्वारिꣳ॒॒शः स्तोमः॑ । \newline
24. च॒तु॒श्च॒त्वा॒रिꣳ॒॒शः स्तोमः॒ स्तोम॑ श्चतुश्चत्वारिꣳ॒॒श श्च॑तुश्चत्वारिꣳ॒॒शः स्तोम॒ इतीति॒ स्तोम॑ श्चतुश्चत्वारिꣳ॒॒श श्च॑तुश्चत्वारिꣳ॒॒शः स्तोम॒ इति॑ । \newline
25. च॒तु॒श्च॒त्वा॒रिꣳ॒॒श इति॑ चतुः - च॒त्वा॒रिꣳ॒॒शः । \newline
26. स्तोम॒ इतीति॒ स्तोमः॒ स्तोम॒ इति॑ दक्षिण॒तो द॑क्षिण॒त इति॒ स्तोमः॒ स्तोम॒ इति॑ दक्षिण॒तः । \newline
27. इति॑ दक्षिण॒तो द॑क्षिण॒त इतीति॑ दक्षिण॒तो ब्र॑ह्मवर्च॒सम् ब्र॑ह्मवर्च॒सम् द॑क्षिण॒त इतीति॑ दक्षिण॒तो ब्र॑ह्मवर्च॒सम् । \newline
28. द॒क्षि॒ण॒तो ब्र॑ह्मवर्च॒सम् ब्र॑ह्मवर्च॒सम् द॑क्षिण॒तो द॑क्षिण॒तो ब्र॑ह्मवर्च॒सं ॅवै वै ब्र॑ह्मवर्च॒सम् द॑क्षिण॒तो द॑क्षिण॒तो ब्र॑ह्मवर्च॒सं ॅवै । \newline
29. ब्र॒ह्म॒व॒र्च॒सं ॅवै वै ब्र॑ह्मवर्च॒सम् ब्र॑ह्मवर्च॒सं ॅवै च॑तुश्चत्वारिꣳ॒॒श श्च॑तुश्चत्वारिꣳ॒॒शो वै ब्र॑ह्मवर्च॒सम् ब्र॑ह्मवर्च॒सं ॅवै च॑तुश्चत्वारिꣳ॒॒शः । \newline
30. ब्र॒ह्म॒व॒र्च॒समिति॑ ब्रह्म - व॒र्च॒सम् । \newline
31. वै च॑तुश्चत्वारिꣳ॒॒श श्च॑तुश्चत्वारिꣳ॒॒शो वै वै च॑तुश्चत्वारिꣳ॒॒शो ब्र॑ह्मवर्च॒सम् ब्र॑ह्मवर्च॒सम् च॑तुश्चत्वारिꣳ॒॒शो वै वै च॑तुश्चत्वारिꣳ॒॒शो ब्र॑ह्मवर्च॒सम् । \newline
32. च॒तु॒श्च॒त्वा॒रिꣳ॒॒शो ब्र॑ह्मवर्च॒सम् ब्र॑ह्मवर्च॒सम् च॑तुश्चत्वारिꣳ॒॒श श्च॑तुश्चत्वारिꣳ॒॒शो ब्र॑ह्मवर्च॒स मे॒वैव ब्र॑ह्मवर्च॒सम् च॑तुश्चत्वारिꣳ॒॒श श्च॑तुश्चत्वारिꣳ॒॒शो ब्र॑ह्मवर्च॒स मे॒व । \newline
33. च॒तु॒श्च॒त्वा॒रिꣳ॒॒श इति॑ चतुः - च॒त्वा॒रिꣳ॒॒शः । \newline
34. ब्र॒ह्म॒व॒र्च॒स मे॒वैव ब्र॑ह्मवर्च॒सम् ब्र॑ह्मवर्च॒स मे॒व द॑क्षिण॒तो द॑क्षिण॒त ए॒व ब्र॑ह्मवर्च॒सम् ब्र॑ह्मवर्च॒स मे॒व द॑क्षिण॒तः । \newline
35. ब्र॒ह्म॒व॒र्च॒समिति॑ ब्रह्म - व॒र्च॒सम् । \newline
36. ए॒व द॑क्षिण॒तो द॑क्षिण॒त ए॒वैव द॑क्षिण॒तो ध॑त्ते धत्ते दक्षिण॒त ए॒वैव द॑क्षिण॒तो ध॑त्ते । \newline
37. द॒क्षि॒ण॒तो ध॑त्ते धत्ते दक्षिण॒तो द॑क्षिण॒तो ध॑त्ते॒ तस्मा॒त् तस्मा᳚द् धत्ते दक्षिण॒तो द॑क्षिण॒तो ध॑त्ते॒ तस्मा᳚त् । \newline
38. ध॒त्ते॒ तस्मा॒त् तस्मा᳚द् धत्ते धत्ते॒ तस्मा॒द् दक्षि॑णो॒ दक्षि॑ण॒ स्तस्मा᳚द् धत्ते धत्ते॒ तस्मा॒द् दक्षि॑णः । \newline
39. तस्मा॒द् दक्षि॑णो॒ दक्षि॑ण॒ स्तस्मा॒त् तस्मा॒द् दक्षि॒णो ऽर्द्धो ऽर्द्धो॒ दक्षि॑ण॒ स्तस्मा॒त् तस्मा॒द् दक्षि॒णो ऽर्द्धः॑ । \newline
40. दक्षि॒णो ऽर्द्धो ऽर्द्धो॒ दक्षि॑णो॒ दक्षि॒णो ऽर्द्धो᳚ ब्रह्मवर्च॒सित॑रो ब्रह्मवर्च॒सित॒रो ऽर्द्धो॒ दक्षि॑णो॒ दक्षि॒णो ऽर्द्धो᳚ ब्रह्मवर्च॒सित॑रः । \newline
41. अर्द्धो᳚ ब्रह्मवर्च॒सित॑रो ब्रह्मवर्च॒सित॒रो ऽर्द्धो ऽर्द्धो᳚ ब्रह्मवर्च॒सित॑र ष्षोड॒श ष्षो॑ड॒शो ब्र॑ह्मवर्च॒सित॒रो ऽर्द्धो ऽर्द्धो᳚ ब्रह्मवर्च॒सित॑र ष्षोड॒शः । \newline
42. ब्र॒ह्म॒व॒र्च॒सित॑र ष्षोड॒श ष्षो॑ड॒शो ब्र॑ह्मवर्च॒सित॑रो ब्रह्मवर्च॒सित॑र ष्षोड॒शः स्तोमः॒ स्तोम॑ ष्षोड॒शो ब्र॑ह्मवर्च॒सित॑रो ब्रह्मवर्च॒सित॑र ष्षोड॒शः स्तोमः॑ । \newline
43. ब्र॒ह्म॒व॒र्च॒सित॑र॒ इति॑ ब्रह्मवर्च॒सि - त॒रः॒ । \newline
44. षो॒ड॒शः स्तोमः॒ स्तोम॑ ष्षोड॒श ष्षो॑ड॒शः स्तोम॒ इतीति॒ स्तोम॑ ष्षोड॒श ष्षो॑ड॒शः स्तोम॒ इति॑ । \newline
45. स्तोम॒ इतीति॒ स्तोमः॒ स्तोम॒ इत्यु॑त्तर॒त उ॑त्तर॒त इति॒ स्तोमः॒ स्तोम॒ इत्यु॑त्तर॒तः । \newline
46. इत्यु॑त्तर॒त उ॑त्तर॒त इती त्यु॑त्तर॒त ओज॒ ओज॑ उत्तर॒त इती त्यु॑त्तर॒त ओजः॑ । \newline
47. उ॒त्त॒र॒त ओज॒ ओज॑ उत्तर॒त उ॑त्तर॒त ओजो॒ वै वा ओज॑ उत्तर॒त उ॑त्तर॒त ओजो॒ वै । \newline
48. उ॒त्त॒र॒त इत्यु॑त् - त॒र॒तः । \newline
49. ओजो॒ वै वा ओज॒ ओजो॒ वै षो॑ड॒श ष्षो॑ड॒शो वा ओज॒ ओजो॒ वै षो॑ड॒शः । \newline
50. वै षो॑ड॒श ष्षो॑ड॒शो वै वै षो॑ड॒श ओज॒ ओज॑ ष्षोड॒शो वै वै षो॑ड॒श ओजः॑ । \newline
51. षो॒ड॒श ओज॒ ओज॑ ष्षोड॒श ष्षो॑ड॒श ओज॑ ए॒वैवौज॑ ष्षोड॒श ष्षो॑ड॒श ओज॑ ए॒व । \newline
52. ओज॑ ए॒वैवौज॒ ओज॑ ए॒वोत्त॑र॒त उ॑त्तर॒त ए॒वौज॒ ओज॑ ए॒वोत्त॑र॒तः । \newline
53. ए॒वोत्त॑र॒त उ॑त्तर॒त ए॒वैवोत्त॑र॒तो ध॑त्ते धत्त उत्तर॒त ए॒वैवोत्त॑र॒तो ध॑त्ते । \newline
54. उ॒त्त॒र॒तो ध॑त्ते धत्त उत्तर॒त उ॑त्तर॒तो ध॑त्ते॒ तस्मा॒त् तस्मा᳚द् धत्त उत्तर॒त उ॑त्तर॒तो ध॑त्ते॒ तस्मा᳚त् । \newline
55. उ॒त्त॒र॒त इत्यु॑त् - त॒र॒तः । \newline
56. ध॒त्ते॒ तस्मा॒त् तस्मा᳚द् धत्ते धत्ते॒ तस्मा॑ दुत्तरतोभिप्रया॒ य्यु॑त्तरतोभिप्रया॒यी तस्मा᳚द् धत्ते धत्ते॒ तस्मा॑ दुत्तरतोभिप्रया॒यी । \newline
57. तस्मा॑ दुत्तरतोभिप्रया॒ य्यु॑त्तरतोभिप्रया॒यी तस्मा॒त् तस्मा॑ दुत्तरतोभिप्रया॒यी ज॑यति जय त्युत्तरतोभिप्रया॒यी तस्मा॒त् तस्मा॑ दुत्तरतोभिप्रया॒यी ज॑यति । \newline
\pagebreak
\markright{ TS 5.3.5.2  \hfill https://www.vedavms.in \hfill}

\section{ TS 5.3.5.2 }

\textbf{TS 5.3.5.2 } \newline
\textbf{Samhita Paata} \newline

दुत्तरतोऽभिप्रया॒यी ज॑यति॒ वज्रो॒ वै च॑तुश्चत्वारिꣳ॒॒शो वज्रः॑ षोड॒शो यदे॒ते इष्ट॑के उप॒दधा॑ति जा॒ताꣳश्चै॒व ज॑नि॒ष्यमा॑णाꣳश्च॒ भ्रातृ॑व्यान् प्र॒णुद्य॒ वज्र॒मनु॒ प्रह॑रति॒ स्तृत्यै॒ पुरी॑षवतीं॒ मद्ध्य॒ उप॑दधाति॒ पुरी॑षं॒ ॅवै मद्ध्य॑मा॒त्मनः॒ सात्मा॑नमे॒वाग्निं चि॑नुते॒ सात्मा॒ऽमुष्मि॑न् ॅलो॒के भ॑वति॒ य ए॒वं ॅवेदै॒ता वा अ॑सप॒त्ना नामेष्ट॑का॒ यस्यै॒ता उ॑पधी॒यन्ते॒ - [  ] \newline

\textbf{Pada Paata} \newline

उ॒त्त॒र॒तो॒ऽभि॒प्र॒या॒यीत्यु॑त्तरतः - अ॒भि॒प्र॒या॒यी । ज॒य॒ति॒ । वज्रः॑ । वै । च॒तु॒श्च॒त्वा॒रिꣳ॒॒श इति॑ चतुः-च॒त्वा॒रिꣳ॒॒शः । वज्रः॑ । षो॒ड॒शः । यत् । ए॒ते इति॑ । इष्ट॑के॒ इति॑ । उ॒प॒दधा॒तीत्यु॑प - दधा॑ति । जा॒तान् । च॒ । ए॒व । ज॒नि॒ष्यमा॑णान् । च॒ । भ्रातृ॑व्यान् । प्र॒णुद्येति॑ प्र-नुद्य॑ । वज्र᳚म् । अनु॑ । प्रेति॑ । ह॒र॒ति॒ । स्तृत्यै᳚ । पुरी॑षवती॒मिति॒ पुरी॑ष-व॒ती॒म् । मद्ध्ये᳚ । उपेति॑ । द॒धा॒ति॒ । पुरी॑षम् । वै । मद्ध्य᳚म् । आ॒त्मनः॑ । सात्मा॑न॒मिति॒ स - आ॒त्मा॒न॒म् । ए॒व । अ॒ग्निम् । चि॒नु॒ते॒ । सात्मेति॒ स॒ - आ॒त्मा॒ । अ॒मुष्मिन्न्॑ । लो॒के । भ॒व॒ति॒ । यः । ए॒वम् । वेद॑ । ए॒ताः । वै । अ॒स॒प॒त्नाः । नाम॑ । इष्ट॑काः । यस्य॑ । ए॒ताः । उ॒प॒धी॒यन्त॒ इत्यु॑प - धी॒यन्ते᳚ ।  \newline


\textbf{Krama Paata} \newline

उ॒त्त॒र॒तो॒भि॒प्र॒या॒यी ज॑यति । उ॒त्त॒र॒तो॒भि॒प्र॒या॒यीत्यु॑त्तरतः - अ॒भि॒प्र॒या॒यी । ज॒य॒ति॒ वज्रः॑ । वज्रो॒ वै । वै च॑तुश्चत्वारिꣳ॒॒शः । च॒तु॒श्च॒त्वा॒रिꣳ॒॒शो वज्रः॑ । च॒तु॒श्च॒त्वा॒रिꣳ॒॒श इति॑ चतुः - च॒त्वा॒रिꣳ॒॒शः । वज्रः॑ षोड॒शः । षो॒ड॒शो यत् । यदे॒ते । ए॒ते इष्ट॑के । ए॒ते इत्ये॒ते । इष्ट॑के उप॒दधा॑ति । इष्ट॑के॒ इतीष्ट॑के । उ॒प॒दधा॑ति जा॒तान् । उ॒प॒दधा॒तीत्यु॑प - दधा॑ति । जा॒ताꣳश्च॑ । चै॒व । ए॒व ज॑नि॒ष्यमा॑णान् । ज॒नि॒ष्यमा॑णाꣳश्च । च॒ भ्रातृ॑व्यान् । भ्रातृ॑व्यान् प्र॒णुद्य॑ । प्र॒णुद्य॒ वज्र᳚म् । प्र॒णुद्येति॑ प्र - नुद्य॑ । वज्र॒मनु॑ । अनु॒ प्र । प्र ह॑रति । ह॒र॒ति॒ स्तृत्यै᳚ । स्तृत्यै॒ पुरी॑षवतीम् । पुरी॑षवती॒म् मद्ध्ये᳚ । पुरी॑षवती॒मिति॒ पुरी॑ष - व॒ती॒म् । मद्ध्य॒ उप॑ । उप॑ दधाति । द॒धा॒ति॒ पुरी॑षम् । पुरी॑ष॒म् वै । वै मद्ध्य᳚म् । मद्ध्य॑मा॒त्मनः॑ । आ॒त्मनः॒ सात्मा॑नम् । सात्मा॑नमे॒व । सात्मा॑न॒मिति॒ स - आ॒त्मा॒न॒म् । ए॒वाग्निम् । अ॒ग्निम् चि॑नुते । चि॒नु॒ते॒ सात्मा᳚ । सात्मा॒ऽमुष्मिन्न्॑ । सात्मेति॒ स - आ॒त्मा॒ । अ॒मुष्मि॑न् ॅलो॒के । लो॒के भ॑वति । भ॒व॒ति॒ यः । य ए॒वम् । ए॒वम् ॅवेद॑ । वेदै॒ताः । ए॒ता वै । वा अ॑सप॒त्नाः । अ॒स॒प॒त्ना नाम॑ । नामेष्ट॑काः । इष्ट॑का॒ यस्य॑ । यस्यै॒ताः । ए॒ता उ॑पधी॒यन्ते᳚ । उ॒प॒धी॒यन्ते॒ न । उ॒प॒धी॒यन्त॒ इत्यु॑प - धी॒यन्ते᳚ \newline

\textbf{Jatai Paata} \newline

1. उ॒त्त॒र॒तो॒भि॒प्र॒या॒यी ज॑यति जय त्युत्तरतोभिप्रया॒ य्यु॑त्तरतोभिप्रया॒यी ज॑यति । \newline
2. उ॒त्त॒र॒तो॒भि॒प्र॒या॒यीत्यु॑त्तरतः - अ॒भि॒प्र॒या॒यी । \newline
3. ज॒य॒ति॒ वज्रो॒ वज्रो॑ जयति जयति॒ वज्रः॑ । \newline
4. वज्रो॒ वै वै वज्रो॒ वज्रो॒ वै । \newline
5. वै च॑तुश्चत्वारिꣳ॒॒श श्च॑तुश्चत्वारिꣳ॒॒शो वै वै च॑तुश्चत्वारिꣳ॒॒शः । \newline
6. च॒तु॒श्च॒त्वा॒रिꣳ॒॒शो वज्रो॒ वज्र॑श्चतुश्चत्वारिꣳ॒॒श श्च॑तुश्चत्वारिꣳ॒॒शो वज्रः॑ । \newline
7. च॒तु॒श्च॒त्वा॒रिꣳ॒॒श इति॑ चतुः - च॒त्वा॒रिꣳ॒॒शः । \newline
8. वज्र॑ ष्षोड॒श ष्षो॑ड॒शो वज्रो॒ वज्र॑ ष्षोड॒शः । \newline
9. षो॒ड॒शो यद् यथ् षो॑ड॒श ष्षो॑ड॒शो यत् । \newline
10. यदे॒ते ए॒ते यद् यदे॒ते । \newline
11. ए॒ते इष्ट॑के॒ इष्ट॑के ए॒ते ए॒ते इष्ट॑के । \newline
12. ए॒ते इत्ये॒ते । \newline
13. इष्ट॑के उप॒दधा᳚ त्युप॒दधा॒तीष्ट॑के॒ इष्ट॑के उप॒दधा॑ति । \newline
14. इष्ट॑के॒ इतीष्ट॑के । \newline
15. उ॒प॒दधा॑ति जा॒तान् जा॒ता नु॑प॒दधा᳚ त्युप॒दधा॑ति जा॒तान् । \newline
16. उ॒प॒दधा॒तीत्यु॑प - दधा॑ति । \newline
17. जा॒ताꣳ श्च॑ च जा॒तान् जा॒ताꣳ श्च॑ । \newline
18. चै॒वैव च॑ चै॒व । \newline
19. ए॒व ज॑नि॒ष्यमा॑णान् जनि॒ष्यमा॑णा ने॒वैव ज॑नि॒ष्यमा॑णान् । \newline
20. ज॒नि॒ष्यमा॑णाꣳ श्च च जनि॒ष्यमा॑णान् जनि॒ष्यमा॑णाꣳ श्च । \newline
21. च॒ भ्रातृ॑व्या॒न् भ्रातृ॑व्याꣳ श्च च॒ भ्रातृ॑व्यान् । \newline
22. भ्रातृ॑व्यान् प्र॒णुद्य॑ प्र॒णुद्य॒ भ्रातृ॑व्या॒न् भ्रातृ॑व्यान् प्र॒णुद्य॑ । \newline
23. प्र॒णुद्य॒ वज्रं॒ ॅवज्र॑म् प्र॒णुद्य॑ प्र॒णुद्य॒ वज्र᳚म् । \newline
24. प्र॒णुद्येति॑ प्र - नुद्य॑ । \newline
25. वज्र॒ मन्वनु॒ वज्रं॒ ॅवज्र॒ मनु॑ । \newline
26. अनु॒ प्र प्राण्वनु॒ प्र । \newline
27. प्र ह॑रति हरति॒ प्र प्र ह॑रति । \newline
28. ह॒र॒ति॒ स्तृत्यै॒ स्तृत्यै॑ हरति हरति॒ स्तृत्यै᳚ । \newline
29. स्तृत्यै॒ पुरी॑षवती॒म् पुरी॑षवतीꣳ॒॒ स्तृत्यै॒ स्तृत्यै॒ पुरी॑षवतीम् । \newline
30. पुरी॑षवती॒म् मद्ध्ये॒ मद्ध्ये॒ पुरी॑षवती॒म् पुरी॑षवती॒म् मद्ध्ये᳚ । \newline
31. पुरी॑षवती॒मिति॒ पुरी॑ष - व॒ती॒म् । \newline
32. मद्ध्य॒ उपोप॒ मद्ध्ये॒ मद्ध्य॒ उप॑ । \newline
33. उप॑ दधाति दधा॒ त्युपोप॑ दधाति । \newline
34. द॒धा॒ति॒ पुरी॑ष॒म् पुरी॑षम् दधाति दधाति॒ पुरी॑षम् । \newline
35. पुरी॑षं॒ ॅवै वै पुरी॑ष॒म् पुरी॑षं॒ ॅवै । \newline
36. वै मद्ध्य॒म् मद्ध्यं॒ ॅवै वै मद्ध्य᳚म् । \newline
37. मद्ध्य॑ मा॒त्मन॑ आ॒त्मनो॒ मद्ध्य॒म् मद्ध्य॑ मा॒त्मनः॑ । \newline
38. आ॒त्मनः॒ सात्मा॑नꣳ॒॒ सात्मा॑न मा॒त्मन॑ आ॒त्मनः॒ सात्मा॑नम् । \newline
39. सात्मा॑न मे॒वैव सात्मा॑नꣳ॒॒ सात्मा॑न मे॒व । \newline
40. सात्मा॑न॒मिति॒ स - आ॒त्मा॒न॒म् । \newline
41. ए॒वाग्नि म॒ग्नि मे॒वै वाग्निम् । \newline
42. अ॒ग्निम् चि॑नुते चिनुते॒ ऽग्नि म॒ग्निम् चि॑नुते । \newline
43. चि॒नु॒ते॒ सात्मा॒ सात्मा॑ चिनुते चिनुते॒ सात्मा᳚ । \newline
44. सात्मा॒ ऽमुष्मि॑न् न॒मुष्मि॒न् थ्सात्मा॒ सात्मा॒ ऽमुष्मिन्न्॑ । \newline
45. सात्मेति॒ स - आ॒त्मा॒ । \newline
46. अ॒मुष्मि॑न् ॅलो॒के लो॒के॑ ऽमुष्मि॑न् न॒मुष्मि॑न् ॅलो॒के । \newline
47. लो॒के भ॑वति भवति लो॒के लो॒के भ॑वति । \newline
48. भ॒व॒ति॒ यो यो भ॑वति भवति॒ यः । \newline
49. य ए॒व मे॒वं ॅयो य ए॒वम् । \newline
50. ए॒वं ॅवेद॒ वेदै॒व मे॒वं ॅवेद॑ । \newline
51. वेदै॒ता ए॒ता वेद॒ वेदै॒ताः । \newline
52. ए॒ता वै वा ए॒ता ए॒ता वै । \newline
53. वा अ॑सप॒त्ना अ॑सप॒त्ना वै वा अ॑सप॒त्नाः । \newline
54. अ॒स॒प॒त्ना नाम॒ नामा॑ सप॒त्ना अ॑सप॒त्ना नाम॑ । \newline
55. नामेष्ट॑का॒ इष्ट॑का॒ नाम॒ नामेष्ट॑काः । \newline
56. इष्ट॑का॒ यस्य॒ यस्ये ष्ट॑का॒ इष्ट॑का॒ यस्य॑ । \newline
57. यस्यै॒ता ए॒ता यस्य॒ यस्यै॒ताः । \newline
58. ए॒ता उ॑पधी॒यन्त॑ उपधी॒यन्त॑ ए॒ता ए॒ता उ॑पधी॒यन्ते᳚ । \newline
59. उ॒प॒धी॒यन्ते॒ न नोप॑धी॒यन्त॑ उपधी॒यन्ते॒ न । \newline
60. उ॒प॒धी॒यन्त॒ इत्यु॑प - धी॒यन्ते᳚ । \newline

\textbf{Ghana Paata } \newline

1. उ॒त्त॒र॒तो॒भि॒प्र॒या॒यी ज॑यति जय त्युत्तरतोभिप्रया॒ य्यु॑त्तरतोभिप्रया॒यी ज॑यति॒ वज्रो॒ वज्रो॑ जय
त्युत्तरतोभिप्रया॒ य्यु॑त्तरतोभिप्रया॒यी ज॑यति॒ वज्रः॑ । \newline
2. उ॒त्त॒र॒तो॒भि॒प्र॒या॒यीत्यु॑त्तरतः - अ॒भि॒प्र॒या॒यी । \newline
3. ज॒य॒ति॒ वज्रो॒ वज्रो॑ जयति जयति॒ वज्रो॒ वै वै वज्रो॑ जयति जयति॒ वज्रो॒ वै । \newline
4. वज्रो॒ वै वै वज्रो॒ वज्रो॒ वै च॑तुश्चत्वारिꣳ॒॒श श्च॑तुश्चत्वारिꣳ॒॒शो वै वज्रो॒ वज्रो॒ वै च॑तुश्चत्वारिꣳ॒॒शः । \newline
5. वै च॑तुश्चत्वारिꣳ॒॒श श्च॑तुश्चत्वारिꣳ॒॒शो वै वै च॑तुश्चत्वारिꣳ॒॒शो वज्रो॒ वज्र॑ श्चतुश्चत्वारिꣳ॒॒शो वै वै च॑तुश्चत्वारिꣳ॒॒शो वज्रः॑ । \newline
6. च॒तु॒श्च॒त्वा॒रिꣳ॒॒शो वज्रो॒ वज्र॑ श्चतुश्चत्वारिꣳ॒॒श श्च॑तुश्चत्वारिꣳ॒॒शो वज्र॑ ष्षोड॒श ष्षो॑ड॒शो वज्र॑ श्चतुश्चत्वारिꣳ॒॒श श्च॑तुश्चत्वारिꣳ॒॒शो वज्र॑ ष्षोड॒शः । \newline
7. च॒तु॒श्च॒त्वा॒रिꣳ॒॒श इति॑ चतुः - च॒त्वा॒रिꣳ॒॒शः । \newline
8. वज्र॑ ष्षोड॒श ष्षो॑ड॒शो वज्रो॒ वज्र॑ ष्षोड॒शो यद् यथ् षो॑ड॒शो वज्रो॒ वज्र॑ ष्षोड॒शो यत् । \newline
9. षो॒ड॒शो यद् यथ् षो॑ड॒श ष्षो॑ड॒शो यदे॒ते ए॒ते यथ् षो॑ड॒श ष्षो॑ड॒शो यदे॒ते । \newline
10. यदे॒ते ए॒ते यद् यदे॒ते इष्ट॑के॒ इष्ट॑के ए॒ते यद् यदे॒ते इष्ट॑के । \newline
11. ए॒ते इष्ट॑के॒ इष्ट॑के ए॒ते ए॒ते इष्ट॑के उप॒दधा᳚ त्युप॒दधा॒तीष्ट॑के ए॒ते ए॒ते इष्ट॑के उप॒दधा॑ति । \newline
12. ए॒ते इत्ये॒ते । \newline
13. इष्ट॑के उप॒दधा᳚ त्युप॒दधा॒तीष्ट॑के॒ इष्ट॑के उप॒दधा॑ति जा॒तान् जा॒ता नु॑प॒दधा॒तीष्ट॑के॒ इष्ट॑के उप॒दधा॑ति जा॒तान् । \newline
14. इष्ट॑के॒ इतीष्ट॑के । \newline
15. उ॒प॒दधा॑ति जा॒तान् जा॒ता नु॑प॒दधा᳚ त्युप॒दधा॑ति जा॒ताꣳश्च॑ च जा॒ता नु॑प॒दधा᳚ त्युप॒दधा॑ति जा॒ताꣳश्च॑ । \newline
16. उ॒प॒दधा॒तीत्यु॑प - दधा॑ति । \newline
17. जा॒ताꣳश्च॑ च जा॒तान् जा॒ताꣳ श्चै॒वैव च॑ जा॒तान् जा॒ताꣳ श्चै॒व । \newline
18. चै॒वैव च॑ चै॒व ज॑नि॒ष्यमा॑णान् जनि॒ष्यमा॑णा ने॒व च॑ चै॒व ज॑नि॒ष्यमा॑णान् । \newline
19. ए॒व ज॑नि॒ष्यमा॑णान् जनि॒ष्यमा॑णा ने॒वैव ज॑नि॒ष्यमा॑णाꣳश्च च जनि॒ष्यमा॑णा ने॒वैव ज॑नि॒ष्यमा॑णाꣳश्च । \newline
20. ज॒नि॒ष्यमा॑णाꣳश्च च जनि॒ष्यमा॑णान् जनि॒ष्यमा॑णाꣳश्च॒ भ्रातृ॑व्या॒न् भ्रातृ॑व्याꣳश्च जनि॒ष्यमा॑णान् जनि॒ष्यमा॑णाꣳश्च॒ भ्रातृ॑व्यान् । \newline
21. च॒ भ्रातृ॑व्या॒न् भ्रातृ॑व्याꣳश्च च॒ भ्रातृ॑व्यान् प्र॒णुद्य॑ प्र॒णुद्य॒ भ्रातृ॑व्याꣳश्च च॒ भ्रातृ॑व्यान् प्र॒णुद्य॑ । \newline
22. भ्रातृ॑व्यान् प्र॒णुद्य॑ प्र॒णुद्य॒ भ्रातृ॑व्या॒न् भ्रातृ॑व्यान् प्र॒णुद्य॒ वज्रं॒ ॅवज्र॑म् प्र॒णुद्य॒ भ्रातृ॑व्या॒न् भ्रातृ॑व्यान् प्र॒णुद्य॒ वज्र᳚म् । \newline
23. प्र॒णुद्य॒ वज्रं॒ ॅवज्र॑म् प्र॒णुद्य॑ प्र॒णुद्य॒ वज्र॒ मन्वनु॒ वज्र॑म् प्र॒णुद्य॑ प्र॒णुद्य॒ वज्र॒ मनु॑ । \newline
24. प्र॒णुद्येति॑ प्र - नुद्य॑ । \newline
25. वज्र॒ मन्वनु॒ वज्रं॒ ॅवज्र॒ मनु॒ प्र प्राणु॒ वज्रं॒ ॅवज्र॒ मनु॒ प्र । \newline
26. अनु॒ प्र प्राण्वनु॒ प्र ह॑रति हरति॒ प्राण्वनु॒ प्र ह॑रति । \newline
27. प्र ह॑रति हरति॒ प्र प्र ह॑रति॒ स्तृत्यै॒ स्तृत्यै॑ हरति॒ प्र प्र ह॑रति॒ स्तृत्यै᳚ । \newline
28. ह॒र॒ति॒ स्तृत्यै॒ स्तृत्यै॑ हरति हरति॒ स्तृत्यै॒ पुरी॑षवती॒म् पुरी॑षवतीꣳ॒॒ स्तृत्यै॑ हरति हरति॒ स्तृत्यै॒ पुरी॑षवतीम् । \newline
29. स्तृत्यै॒ पुरी॑षवती॒म् पुरी॑षवतीꣳ॒॒ स्तृत्यै॒ स्तृत्यै॒ पुरी॑षवती॒म् मद्ध्ये॒ मद्ध्ये॒ पुरी॑षवतीꣳ॒॒ स्तृत्यै॒ स्तृत्यै॒ पुरी॑षवती॒म् मद्ध्ये᳚ । \newline
30. पुरी॑षवती॒म् मद्ध्ये॒ मद्ध्ये॒ पुरी॑षवती॒म् पुरी॑षवती॒म् मद्ध्य॒ उपोप॒ मद्ध्ये॒ पुरी॑षवती॒म् पुरी॑षवती॒म् मद्ध्य॒ उप॑ । \newline
31. पुरी॑षवती॒मिति॒ पुरी॑ष - व॒ती॒म् । \newline
32. मद्ध्य॒ उपोप॒ मद्ध्ये॒ मद्ध्य॒ उप॑ दधाति दधा॒ त्युप॒ मद्ध्ये॒ मद्ध्य॒ उप॑ दधाति । \newline
33. उप॑ दधाति दधा॒ त्युपोप॑ दधाति॒ पुरी॑ष॒म् पुरी॑षम् दधा॒ त्युपोप॑ दधाति॒ पुरी॑षम् । \newline
34. द॒धा॒ति॒ पुरी॑ष॒म् पुरी॑षम् दधाति दधाति॒ पुरी॑षं॒ ॅवै वै पुरी॑षम् दधाति दधाति॒ पुरी॑षं॒ ॅवै । \newline
35. पुरी॑षं॒ ॅवै वै पुरी॑ष॒म् पुरी॑षं॒ ॅवै मद्ध्य॒म् मद्ध्यं॒ ॅवै पुरी॑ष॒म् पुरी॑षं॒ ॅवै मद्ध्य᳚म् । \newline
36. वै मद्ध्य॒म् मद्ध्यं॒ ॅवै वै मद्ध्य॑ मा॒त्मन॑ आ॒त्मनो॒ मद्ध्यं॒ ॅवै वै मद्ध्य॑ मा॒त्मनः॑ । \newline
37. मद्ध्य॑ मा॒त्मन॑ आ॒त्मनो॒ मद्ध्य॒म् मद्ध्य॑ मा॒त्मनः॒ सात्मा॑नꣳ॒॒ सात्मा॑न मा॒त्मनो॒ मद्ध्य॒म् मद्ध्य॑ मा॒त्मनः॒ सात्मा॑नम् । \newline
38. आ॒त्मनः॒ सात्मा॑नꣳ॒॒ सात्मा॑न मा॒त्मन॑ आ॒त्मनः॒ सात्मा॑न मे॒वैव सात्मा॑न मा॒त्मन॑ आ॒त्मनः॒ सात्मा॑न मे॒व । \newline
39. सात्मा॑न मे॒वैव सात्मा॑नꣳ॒॒ सात्मा॑न मे॒वाग्नि म॒ग्नि मे॒व सात्मा॑नꣳ॒॒ सात्मा॑न मे॒वाग्निम् । \newline
40. सात्मा॑न॒मिति॒ स - आ॒त्मा॒न॒म् । \newline
41. ए॒वाग्नि म॒ग्नि मे॒वैवाग्निम् चि॑नुते चिनुते॒ ऽग्नि मे॒वैवाग्निम् चि॑नुते । \newline
42. अ॒ग्निम् चि॑नुते चिनुते॒ ऽग्नि म॒ग्निम् चि॑नुते॒ सात्मा॒ सात्मा॑ चिनुते॒ ऽग्नि म॒ग्निम् चि॑नुते॒ सात्मा᳚ । \newline
43. चि॒नु॒ते॒ सात्मा॒ सात्मा॑ चिनुते चिनुते॒ सात्मा॒ ऽमुष्मि॑न् न॒मुष्मि॒न् थ्सात्मा॑ चिनुते चिनुते॒ सात्मा॒ ऽमुष्मिन्न्॑ । \newline
44. सात्मा॒ ऽमुष्मि॑न् न॒मुष्मि॒न् थ्सात्मा॒ सात्मा॒ ऽमुष्मि॑न् ॅलो॒के लो॒के॑ ऽमुष्मि॒न् थ्सात्मा॒ सात्मा॒ ऽमुष्मि॑न् ॅलो॒के । \newline
45. सात्मेति॒ स - आ॒त्मा॒ । \newline
46. अ॒मुष्मि॑न् ॅलो॒के लो॒के॑ ऽमुष्मि॑न् न॒मुष्मि॑न् ॅलो॒के भ॑वति भवति लो॒के॑ ऽमुष्मि॑न् न॒मुष्मि॑न् ॅलो॒के भ॑वति । \newline
47. लो॒के भ॑वति भवति लो॒के लो॒के भ॑वति॒ यो यो भ॑वति लो॒के लो॒के भ॑वति॒ यः । \newline
48. भ॒व॒ति॒ यो यो भ॑वति भवति॒ य ए॒व मे॒वं ॅयो भ॑वति भवति॒ य ए॒वम् । \newline
49. य ए॒व मे॒वं ॅयो य ए॒वं ॅवेद॒ वेदै॒वं ॅयो य ए॒वं ॅवेद॑ । \newline
50. ए॒वं ॅवेद॒ वेदै॒व मे॒वं ॅवेदै॒ता ए॒ता वेदै॒व मे॒वं ॅवेदै॒ताः । \newline
51. वेदै॒ता ए॒ता वेद॒ वेदै॒ता वै वा ए॒ता वेद॒ वेदै॒ता वै । \newline
52. ए॒ता वै वा ए॒ता ए॒ता वा अ॑सप॒त्ना अ॑सप॒त्ना वा ए॒ता ए॒ता वा अ॑सप॒त्नाः । \newline
53. वा अ॑सप॒त्ना अ॑सप॒त्ना वै वा अ॑सप॒त्ना नाम॒ नामा॑ सप॒त्ना वै वा अ॑सप॒त्ना नाम॑ । \newline
54. अ॒स॒प॒त्ना नाम॒ नामा॑ सप॒त्ना अ॑सप॒त्ना नामेष्ट॑का॒ इष्ट॑का॒ नामा॑ सप॒त्ना अ॑सप॒त्ना नामे ष्ट॑काः । \newline
55. नामेष्ट॑का॒ इष्ट॑का॒ नाम॒ नामेष्ट॑का॒ यस्य॒ यस्येष्ट॑का॒ नाम॒ नामेष्ट॑का॒ यस्य॑ । \newline
56. इष्ट॑का॒ यस्य॒ यस्येष्ट॑का॒ इष्ट॑का॒ यस्यै॒ता ए॒ता यस्येष्ट॑का॒ इष्ट॑का॒ यस्यै॒ताः । \newline
57. यस्यै॒ता ए॒ता यस्य॒ यस्यै॒ता उ॑पधी॒यन्त॑ उपधी॒यन्त॑ ए॒ता यस्य॒ यस्यै॒ता उ॑पधी॒यन्ते᳚ । \newline
58. ए॒ता उ॑पधी॒यन्त॑ उपधी॒यन्त॑ ए॒ता ए॒ता उ॑पधी॒यन्ते॒ न नोप॑धी॒यन्त॑ ए॒ता ए॒ता उ॑पधी॒यन्ते॒ न । \newline
59. उ॒प॒धी॒यन्ते॒ न नोप॑धी॒यन्त॑ उपधी॒यन्ते॒ नास्या᳚स्य॒ नोप॑धी॒यन्त॑ उपधी॒यन्ते॒ नास्य॑ । \newline
60. उ॒प॒धी॒यन्त॒ इत्यु॑प - धी॒यन्ते᳚ । \newline
\pagebreak
\markright{ TS 5.3.5.3  \hfill https://www.vedavms.in \hfill}

\section{ TS 5.3.5.3 }

\textbf{TS 5.3.5.3 } \newline
\textbf{Samhita Paata} \newline

नास्य॑ स॒पत्नो॑ भवति प॒शुर्वा ए॒ष यद॒ग्निर्वि॒राज॑ उत्त॒मायां॒ चित्या॒मुप॑ दधाति वि॒राज॑मे॒वोत्त॒मां प॒शुषु॑ दधाति॒ तस्मा᳚त् पशु॒मानु॑त्त॒मां ॅवाचं॑ ॅवदति॒ दश॑द॒शोप॑ दधाति सवीर्य॒त्वाया᳚ऽक्ष्ण॒योप॑ दधाति॒ तस्मा॑दक्ष्ण॒या प॒शवोऽङ्गा॑नि॒ प्रह॑रन्ति॒ प्रति॑ष्ठित्यै॒ यानि॒ वै छन्दाꣳ॑सि सुव॒र्ग्या᳚ण्यास॒न् तैर्दे॒वाः सु॑व॒र्गं ॅलो॒कमा॑य॒न् तेनर्.ष॑यो - [  ] \newline

\textbf{Pada Paata} \newline

न । अ॒स्य॒ । स॒पत्नः॑ । भ॒व॒ति॒ । प॒शुः । वै । ए॒षः । यत् । अ॒ग्निः । वि॒राज॒ इति॑ वि - राजः॑ । उ॒त्त॒माया॒मित्यु॑त् - त॒माया᳚म् । चित्या᳚म् । उपेति॑ । द॒धा॒ति॒ । वि॒राज॒मिति॑ वि - राज᳚म् । ए॒व । उ॒त्त॒मामित्यु॑त् - त॒माम् । प॒शुषु॑ । द॒धा॒ति॒ । तस्मा᳚त् । प॒शु॒मानिति॑ पशु - मान् । उ॒त्त॒मामित्यु॑त् - त॒माम् । वाच᳚म् । व॒द॒ति॒ । दश॑द॒शेति॒ दश॑ - द॒श॒ । उपेति॑ । द॒धा॒ति॒ । स॒वी॒र्य॒त्वायेति॑ सवीर्य-त्वाय॑ । अ॒क्ष्ण॒या । उपेति॑ । द॒धा॒ति॒ । तस्मा᳚त् । अ॒क्ष्ण॒या । प॒शवः॑ । अङ्गा॑नि । प्रेति॑ । ह॒र॒न्ति॒ । प्रति॑ष्ठित्या॒ इति॒ प्रति॑ - स्थि॒त्यै॒ । यानि॑ । वै । छन्दाꣳ॑सि । सु॒व॒र्ग्या॑णीति॑ सुवः - ग्या॑नि । आसन्न्॑ । तैः । दे॒वाः । सु॒व॒र्गमिति॑ सुवः - गम् । लो॒कम् । आ॒य॒न्न् । तेन॑ । ऋष॑यः ।  \newline


\textbf{Krama Paata} \newline

नास्य॑ । अ॒स्य॒ स॒पत्नः॑ । स॒पत्नो॑ भवति । भ॒व॒ति॒ प॒शुः । 
प॒शुर् वै । वा ए॒षः । ए॒ष यत् । यद॒ग्निः । अ॒ग्निर् वि॒राजः॑ । वि॒राज॑ उत्त॒माया᳚म् । वि॒राज॒ इति॑ वि - राजः॑ । उ॒त्त॒माया॒म् चित्या᳚म् । उ॒त्त॒माया॒मित्यु॑त् - त॒माया᳚म् । 
चित्या॒मुप॑ । उप॑ दधाति । द॒धा॒ति॒ वि॒राज᳚म् । वि॒राज॑मे॒व । वि॒राज॒मिति॑ वि - राज᳚म् । ए॒वोत्त॒माम् । उ॒त्त॒माम् प॒शुषु॑ । उ॒त्त॒मामित्यु॑त् - त॒माम् । प॒शुषु॑ दधाति । द॒धा॒ति॒ तस्मा᳚त् । तस्मा᳚त् पशु॒मान् । प॒शु॒मानु॑त्त॒माम् । प॒शु॒मानिति॑ पशु - मान् । उ॒त्त॒माम् ॅवाच᳚म् । उ॒त्त॒मामित्यु॑त् - त॒माम् । वाच॑म् ॅवदति । व॒द॒ति॒ दश॑दश । दश॑द॒शोप॑ । दश॑द॒शेति॒ दश॑ - द॒श॒ । उप॑ दधाति । द॒धा॒ति॒ स॒वी॒र्य॒त्वाय॑ । स॒वी॒र्य॒त्वाया᳚क्ष्ण॒या । स॒वी॒र्य॒त्वायेति॑ सवीर्य - त्वाय॑ । अ॒क्ष्ण॒योप॑ । उप॑ दधाति । द॒धा॒ति॒ तस्मा᳚त् । तस्मा॑दक्ष्ण॒या । अ॒क्ष्ण॒या प॒शवः॑ । प॒शवोऽङ्गा॑नि । अङ्गा॑नि॒ प्र । प्र ह॑रन्ति । ह॒र॒न्ति॒ प्रति॑ष्ठित्यै । प्रति॑ष्ठित्यै॒ यानि॑ । प्रति॑ष्ठित्या॒ इति॒ प्रति॑ - स्थि॒त्यै॒ । यानि॒ वै । वै छन्दाꣳ॑सि । छन्दाꣳ॑सि सुव॒र्ग्या॑णि । सु॒व॒र्ग्या᳚ण्यासन्न्॑ । सु॒व॒र्ग्या॑णीति॑ सुवः - ग्या॑नि । आस॒न् तैः । तैर् दे॒वाः । दे॒वाः सु॑व॒र्गम् । सु॒व॒र्गम् ॅलो॒कम् । सु॒व॒र्गमिति॑ सुवः - गम् । लो॒कमा॑यन्न् । आ॒य॒न् तेन॑ । तेनर्.ष॑यः । ऋष॑योऽश्राम्यन्न् \newline

\textbf{Jatai Paata} \newline

1. नास्या᳚स्य॒ न नास्य॑ । \newline
2. अ॒स्य॒ स॒पत्नः॑ स॒पत्नो᳚ ऽस्यास्य स॒पत्नः॑ । \newline
3. स॒पत्नो॑ भवति भवति स॒पत्नः॑ स॒पत्नो॑ भवति । \newline
4. भ॒व॒ति॒ प॒शुः प॒शुर् भ॑वति भवति प॒शुः । \newline
5. प॒शुर् वै वै प॒शुः प॒शुर् वै । \newline
6. वा ए॒ष ए॒ष वै वा ए॒षः । \newline
7. ए॒ष यद् यदे॒ष ए॒ष यत् । \newline
8. यद॒ग्नि र॒ग्निर् यद् यद॒ग्निः । \newline
9. अ॒ग्निर् वि॒राजो॑ वि॒राजो॒ ऽग्नि र॒ग्निर् वि॒राजः॑ । \newline
10. वि॒राज॑ उत्त॒माया॑ मुत्त॒मायां᳚ ॅवि॒राजो॑ वि॒राज॑ उत्त॒माया᳚म् । \newline
11. वि॒राज॒ इति॑ वि - राजः॑ । \newline
12. उ॒त्त॒माया॒म् चित्या॒म् चित्या॑ मुत्त॒माया॑ मुत्त॒माया॒म् चित्या᳚म् । \newline
13. उ॒त्त॒माया॒मित्यु॑त् - त॒माया᳚म् । \newline
14. चित्या॒ मुपोप॒ चित्या॒म् चित्या॒ मुप॑ । \newline
15. उप॑ दधाति दधा॒ त्युपोप॑ दधाति । \newline
16. द॒धा॒ति॒ वि॒राजं॑ ॅवि॒राज॑म् दधाति दधाति वि॒राज᳚म् । \newline
17. वि॒राज॑ मे॒वैव वि॒राजं॑ ॅवि॒राज॑ मे॒व । \newline
18. वि॒राज॒मिति॑ वि - राज᳚म् । \newline
19. ए॒वोत्त॒मा मु॑त्त॒मा मे॒वै वोत्त॒माम् । \newline
20. उ॒त्त॒माम् प॒शुषु॑ प॒शुषू᳚त्त॒मा मु॑त्त॒माम् प॒शुषु॑ । \newline
21. उ॒त्त॒मामित्यु॑त् - त॒माम् । \newline
22. प॒शुषु॑ दधाति दधाति प॒शुषु॑ प॒शुषु॑ दधाति । \newline
23. द॒धा॒ति॒ तस्मा॒त् तस्मा᳚द् दधाति दधाति॒ तस्मा᳚त् । \newline
24. तस्मा᳚त् पशु॒मान् प॑शु॒मान् तस्मा॒त् तस्मा᳚त् पशु॒मान् । \newline
25. प॒शु॒मा नु॑त्त॒मा मु॑त्त॒माम् प॑शु॒मान् प॑शु॒मा नु॑त्त॒माम् । \newline
26. प॒शु॒मानिति॑ पशु - मान् । \newline
27. उ॒त्त॒मां ॅवाचं॒ ॅवाच॑ मुत्त॒मा मु॑त्त॒मां ॅवाच᳚म् । \newline
28. उ॒त्त॒मामित्यु॑त् - त॒माम् । \newline
29. वाचं॑ ॅवदति वदति॒ वाचं॒ ॅवाचं॑ ॅवदति । \newline
30. व॒द॒ति॒ दश॑दश॒ दश॑दश वदति वदति॒ दश॑दश । \newline
31. दश॑द॒शो पोप॒ दश॑दश॒ दश॑द॒शोप॑ । \newline
32. दश॑द॒शेति॒ दश॑ - द॒श॒ । \newline
33. उप॑ दधाति दधा॒ त्युपोप॑ दधाति । \newline
34. द॒धा॒ति॒ स॒वी॒र्य॒त्वाय॑ सवीर्य॒त्वाय॑ दधाति दधाति सवीर्य॒त्वाय॑ । \newline
35. स॒वी॒र्य॒त्वाया᳚ क्ष्ण॒या ऽक्ष्ण॒या स॑वीर्य॒त्वाय॑ सवीर्य॒त्वाया᳚ क्ष्ण॒या । \newline
36. स॒वी॒र्य॒त्वायेति॑ सवीर्य - त्वाय॑ । \newline
37. अ॒क्ष्ण॒यो पोपा᳚ क्ष्ण॒या ऽक्ष्ण॒योप॑ । \newline
38. उप॑ दधाति दधा॒ त्युपोप॑ दधाति । \newline
39. द॒धा॒ति॒ तस्मा॒त् तस्मा᳚द् दधाति दधाति॒ तस्मा᳚त् । \newline
40. तस्मा॑ दक्ष्ण॒या ऽक्ष्ण॒या तस्मा॒त् तस्मा॑ दक्ष्ण॒या । \newline
41. अ॒क्ष्ण॒या प॒शवः॑ प॒शवो᳚ ऽक्ष्ण॒या ऽक्ष्ण॒या प॒शवः॑ । \newline
42. प॒शवो ऽङ्गा॒ न्यङ्गा॑नि प॒शवः॑ प॒शवो ऽङ्गा॑नि । \newline
43. अङ्गा॑नि॒ प्र प्राङ्गा॒ न्यङ्गा॑नि॒ प्र । \newline
44. प्र ह॑रन्ति हरन्ति॒ प्र प्र ह॑रन्ति । \newline
45. ह॒र॒न्ति॒ प्रति॑ष्ठित्यै॒ प्रति॑ष्ठित्यै हरन्ति हरन्ति॒ प्रति॑ष्ठित्यै । \newline
46. प्रति॑ष्ठित्यै॒ यानि॒ यानि॒ प्रति॑ष्ठित्यै॒ प्रति॑ष्ठित्यै॒ यानि॑ । \newline
47. प्रति॑ष्ठित्या॒ इति॒ प्रति॑ - स्थि॒त्यै॒ । \newline
48. यानि॒ वै वै यानि॒ यानि॒ वै । \newline
49. वै छन्दाꣳ॑सि॒ छन्दाꣳ॑सि॒ वै वै छन्दाꣳ॑सि । \newline
50. छन्दाꣳ॑सि सुव॒र्ग्या॑णि सुव॒र्ग्या॑णि॒ छन्दाꣳ॑सि॒ छन्दाꣳ॑सि सुव॒र्ग्या॑णि । \newline
51. सु॒व॒र्ग्या᳚ ण्यास॒न् नासन्᳚ थ्सुव॒र्ग्या॑णि सुव॒र्ग्या᳚ ण्यासन्न्॑ । \newline
52. सु॒व॒र्ग्या॑णीति॑ सुवः - ग्या॑नि । \newline
53. आस॒न् तै स्तै रास॒न् नास॒न् तैः । \newline
54. तैर् दे॒वा दे॒वा स्तै स्तैर् दे॒वाः । \newline
55. दे॒वाः सु॑व॒र्गꣳ सु॑व॒र्गम् दे॒वा दे॒वाः सु॑व॒र्गम् । \newline
56. सु॒व॒र्गम् ॅलो॒कम् ॅलो॒कꣳ सु॑व॒र्गꣳ सु॑व॒र्गम् ॅलो॒कम् । \newline
57. सु॒व॒र्गमिति॑ सुवः - गम् । \newline
58. लो॒क मा॑यन् नायन् ॅलो॒कम् ॅलो॒क मा॑यन्न् । \newline
59. आ॒य॒न् तेन॒ तेना॑यन् नाय॒न् तेन॑ । \newline
60. तेन र्.ष॑य॒ ऋष॑य॒ स्तेन॒ तेन र्.ष॑यः । \newline
61. ऋष॑यो ऽश्राम्यन् नश्राम्य॒न् नृष॑य॒ ऋष॑यो ऽश्राम्यन्न् । \newline

\textbf{Ghana Paata } \newline

1. नास्या᳚स्य॒ न नास्य॑ स॒पत्नः॑ स॒पत्नो᳚ ऽस्य॒ न नास्य॑ स॒पत्नः॑ । \newline
2. अ॒स्य॒ स॒पत्नः॑ स॒पत्नो᳚ ऽस्यास्य स॒पत्नो॑ भवति भवति स॒पत्नो᳚ ऽस्यास्य स॒पत्नो॑ भवति । \newline
3. स॒पत्नो॑ भवति भवति स॒पत्नः॑ स॒पत्नो॑ भवति प॒शुः प॒शुर् भ॑वति स॒पत्नः॑ स॒पत्नो॑ भवति प॒शुः । \newline
4. भ॒व॒ति॒ प॒शुः प॒शुर् भ॑वति भवति प॒शुर् वै वै प॒शुर् भ॑वति भवति प॒शुर् वै । \newline
5. प॒शुर् वै वै प॒शुः प॒शुर् वा ए॒ष ए॒ष वै प॒शुः प॒शुर् वा ए॒षः । \newline
6. वा ए॒ष ए॒ष वै वा ए॒ष यद् यदे॒ष वै वा ए॒ष यत् । \newline
7. ए॒ष यद् यदे॒ष ए॒ष यद॒ग्नि र॒ग्निर् यदे॒ष ए॒ष यद॒ग्निः । \newline
8. यद॒ग्नि र॒ग्निर् यद् यद॒ग्निर् वि॒राजो॑ वि॒राजो॒ ऽग्निर् यद् यद॒ग्निर् वि॒राजः॑ । \newline
9. अ॒ग्निर् वि॒राजो॑ वि॒राजो॒ ऽग्नि र॒ग्निर् वि॒राज॑ उत्त॒माया॑ मुत्त॒मायां᳚ ॅवि॒राजो॒ ऽग्नि र॒ग्निर् वि॒राज॑ उत्त॒माया᳚म् । \newline
10. वि॒राज॑ उत्त॒माया॑ मुत्त॒मायां᳚ ॅवि॒राजो॑ वि॒राज॑ उत्त॒माया॒म् चित्या॒म् चित्या॑ मुत्त॒मायां᳚ ॅवि॒राजो॑ वि॒राज॑ उत्त॒माया॒म् चित्या᳚म् । \newline
11. वि॒राज॒ इति॑ वि - राजः॑ । \newline
12. उ॒त्त॒माया॒म् चित्या॒म् चित्या॑ मुत्त॒माया॑ मुत्त॒माया॒म् चित्या॒ मुपोप॒ चित्या॑ मुत्त॒माया॑ मुत्त॒माया॒म् चित्या॒ मुप॑ । \newline
13. उ॒त्त॒माया॒मित्यु॑त् - त॒माया᳚म् । \newline
14. चित्या॒ मुपोप॒ चित्या॒म् चित्या॒ मुप॑ दधाति दधा॒ त्युप॒ चित्या॒म् चित्या॒ मुप॑ दधाति । \newline
15. उप॑ दधाति दधा॒ त्युपोप॑ दधाति वि॒राजं॑ ॅवि॒राज॑म् दधा॒ त्युपोप॑ दधाति वि॒राज᳚म् । \newline
16. द॒धा॒ति॒ वि॒राजं॑ ॅवि॒राज॑म् दधाति दधाति वि॒राज॑ मे॒वैव वि॒राज॑म् दधाति दधाति वि॒राज॑ मे॒व । \newline
17. वि॒राज॑ मे॒वैव वि॒राजं॑ ॅवि॒राज॑ मे॒वोत्त॒मा मु॑त्त॒मा मे॒व वि॒राजं॑ ॅवि॒राज॑ मे॒वोत्त॒माम् । \newline
18. वि॒राज॒मिति॑ वि - राज᳚म् । \newline
19. ए॒वोत्त॒मा मु॑त्त॒मा मे॒वैवोत्त॒माम् प॒शुषु॑ प॒शुषू᳚ त्त॒मा मे॒वैवोत्त॒माम् प॒शुषु॑ । \newline
20. उ॒त्त॒माम् प॒शुषु॑ प॒शुषू᳚ त्त॒मा मु॑त्त॒माम् प॒शुषु॑ दधाति दधाति प॒शुषू᳚ त्त॒मा मु॑त्त॒माम् प॒शुषु॑ दधाति । \newline
21. उ॒त्त॒मामित्यु॑त् - त॒माम् । \newline
22. प॒शुषु॑ दधाति दधाति प॒शुषु॑ प॒शुषु॑ दधाति॒ तस्मा॒त् तस्मा᳚द् दधाति प॒शुषु॑ प॒शुषु॑ दधाति॒ तस्मा᳚त् । \newline
23. द॒धा॒ति॒ तस्मा॒त् तस्मा᳚द् दधाति दधाति॒ तस्मा᳚त् पशु॒मान् प॑शु॒मान् तस्मा᳚द् दधाति दधाति॒ तस्मा᳚त् पशु॒मान् । \newline
24. तस्मा᳚त् पशु॒मान् प॑शु॒मान् तस्मा॒त् तस्मा᳚त् पशु॒मा नु॑त्त॒मा मु॑त्त॒माम् प॑शु॒मान् तस्मा॒त् तस्मा᳚त् पशु॒मा नु॑त्त॒माम् । \newline
25. प॒शु॒मा नु॑त्त॒मा मु॑त्त॒माम् प॑शु॒मान् प॑शु॒मा नु॑त्त॒मां ॅवाचं॒ ॅवाच॑ मुत्त॒माम् प॑शु॒मान् प॑शु॒मा नु॑त्त॒मां ॅवाच᳚म् । \newline
26. प॒शु॒मानिति॑ पशु - मान् । \newline
27. उ॒त्त॒मां ॅवाचं॒ ॅवाच॑ मुत्त॒मा मु॑त्त॒मां ॅवाचं॑ ॅवदति वदति॒ वाच॑ मुत्त॒मा मु॑त्त॒मां ॅवाचं॑ ॅवदति । \newline
28. उ॒त्त॒मामित्यु॑त् - त॒माम् । \newline
29. वाचं॑ ॅवदति वदति॒ वाचं॒ ॅवाचं॑ ॅवदति॒ दश॑दश॒ दश॑दश वदति॒ वाचं॒ ॅवाचं॑ ॅवदति॒ दश॑दश । \newline
30. व॒द॒ति॒ दश॑दश॒ दश॑दश वदति वदति॒ दश॑द॒शोपोप॒ दश॑दश वदति वदति॒ दश॑द॒शोप॑ । \newline
31. दश॑द॒शोपोप॒ दश॑दश॒ दश॑द॒शोप॑ दधाति दधा॒ त्युप॒ दश॑दश॒ दश॑द॒शोप॑ दधाति । \newline
32. दश॑द॒शेति॒ दश॑ - द॒श॒ । \newline
33. उप॑ दधाति दधा॒ त्युपोप॑ दधाति सवीर्य॒त्वाय॑ सवीर्य॒त्वाय॑ दधा॒ त्युपोप॑ दधाति सवीर्य॒त्वाय॑ । \newline
34. द॒धा॒ति॒ स॒वी॒र्य॒त्वाय॑ सवीर्य॒त्वाय॑ दधाति दधाति सवीर्य॒त्वाया᳚ क्ष्ण॒या ऽक्ष्ण॒या स॑वीर्य॒त्वाय॑ दधाति दधाति सवीर्य॒त्वाया᳚ क्ष्ण॒या । \newline
35. स॒वी॒र्य॒त्वाया᳚ क्ष्ण॒या ऽक्ष्ण॒या स॑वीर्य॒त्वाय॑ सवीर्य॒त्वाया᳚ क्ष्ण॒योपोपा᳚ क्ष्ण॒या स॑वीर्य॒त्वाय॑ सवीर्य॒त्वाया᳚ क्ष्ण॒योप॑ । \newline
36. स॒वी॒र्य॒त्वायेति॑ सवीर्य - त्वाय॑ । \newline
37. अ॒क्ष्ण॒योपोपा᳚ क्ष्ण॒या ऽक्ष्ण॒योप॑ दधाति दधा॒ त्युपा᳚ क्ष्ण॒या ऽक्ष्ण॒योप॑ दधाति । \newline
38. उप॑ दधाति दधा॒ त्युपोप॑ दधाति॒ तस्मा॒त् तस्मा᳚द् दधा॒ त्युपोप॑ दधाति॒ तस्मा᳚त् । \newline
39. द॒धा॒ति॒ तस्मा॒त् तस्मा᳚द् दधाति दधाति॒ तस्मा॑ दक्ष्ण॒या ऽक्ष्ण॒या तस्मा᳚द् दधाति दधाति॒ तस्मा॑ दक्ष्ण॒या । \newline
40. तस्मा॑ दक्ष्ण॒या ऽक्ष्ण॒या तस्मा॒त् तस्मा॑ दक्ष्ण॒या प॒शवः॑ प॒शवो᳚ ऽक्ष्ण॒या तस्मा॒त् तस्मा॑ दक्ष्ण॒या प॒शवः॑ । \newline
41. अ॒क्ष्ण॒या प॒शवः॑ प॒शवो᳚ ऽक्ष्ण॒या ऽक्ष्ण॒या प॒शवो ऽङ्गा॒ न्यङ्गा॑नि प॒शवो᳚ ऽक्ष्ण॒या ऽक्ष्ण॒या प॒शवो ऽङ्गा॑नि । \newline
42. प॒शवो ऽङ्गा॒ न्यङ्गा॑नि प॒शवः॑ प॒शवो ऽङ्गा॑नि॒ प्र प्राङ्गा॑नि प॒शवः॑ प॒शवो ऽङ्गा॑नि॒ प्र । \newline
43. अङ्गा॑नि॒ प्र प्राङ्गा॒ न्यङ्गा॑नि॒ प्र ह॑रन्ति हरन्ति॒ प्राङ्गा॒ न्यङ्गा॑नि॒ प्र ह॑रन्ति । \newline
44. प्र ह॑रन्ति हरन्ति॒ प्र प्र ह॑रन्ति॒ प्रति॑ष्ठित्यै॒ प्रति॑ष्ठित्यै हरन्ति॒ प्र प्र ह॑रन्ति॒ प्रति॑ष्ठित्यै । \newline
45. ह॒र॒न्ति॒ प्रति॑ष्ठित्यै॒ प्रति॑ष्ठित्यै हरन्ति हरन्ति॒ प्रति॑ष्ठित्यै॒ यानि॒ यानि॒ प्रति॑ष्ठित्यै हरन्ति हरन्ति॒ प्रति॑ष्ठित्यै॒ यानि॑ । \newline
46. प्रति॑ष्ठित्यै॒ यानि॒ यानि॒ प्रति॑ष्ठित्यै॒ प्रति॑ष्ठित्यै॒ यानि॒ वै वै यानि॒ प्रति॑ष्ठित्यै॒ प्रति॑ष्ठित्यै॒ यानि॒ वै । \newline
47. प्रति॑ष्ठित्या॒ इति॒ प्रति॑ - स्थि॒त्यै॒ । \newline
48. यानि॒ वै वै यानि॒ यानि॒ वै छन्दाꣳ॑सि॒ छन्दाꣳ॑सि॒ वै यानि॒ यानि॒ वै छन्दाꣳ॑सि । \newline
49. वै छन्दाꣳ॑सि॒ छन्दाꣳ॑सि॒ वै वै छन्दाꣳ॑सि सुव॒र्ग्या॑णि सुव॒र्ग्या॑णि॒ छन्दाꣳ॑सि॒ वै वै छन्दाꣳ॑सि सुव॒र्ग्या॑णि । \newline
50. छन्दाꣳ॑सि सुव॒र्ग्या॑णि सुव॒र्ग्या॑णि॒ छन्दाꣳ॑सि॒ छन्दाꣳ॑सि सुव॒र्ग्या᳚ ण्यास॒न् नासन्᳚ थ्सुव॒र्ग्या॑णि॒ छन्दाꣳ॑सि॒ छन्दाꣳ॑सि सुव॒र्ग्या᳚ ण्यासन्न्॑ । \newline
51. सु॒व॒र्ग्या᳚ ण्यास॒न् नासन्᳚ थ्सुव॒र्ग्या॑णि सुव॒र्ग्या᳚ ण्यास॒न् तै स्तै रासन्᳚ थ्सुव॒र्ग्या॑णि सुव॒र्ग्या᳚ ण्यास॒न् तैः । \newline
52. सु॒व॒र्ग्या॑णीति॑ सुवः - ग्या॑नि । \newline
53. आस॒न् तै स्तै रास॒न् नास॒न् तैर् दे॒वा दे॒वा स्तै रास॒न् नास॒न् तैर् दे॒वाः । \newline
54. तैर् दे॒वा दे॒वा स्तै स्तैर् दे॒वाः सु॑व॒र्गꣳ सु॑व॒र्गम् दे॒वा स्तै स्तैर् दे॒वाः सु॑व॒र्गम् । \newline
55. दे॒वाः सु॑व॒र्गꣳ सु॑व॒र्गम् दे॒वा दे॒वाः सु॑व॒र्गम् ॅलो॒कम् ॅलो॒कꣳ सु॑व॒र्गम् दे॒वा दे॒वाः सु॑व॒र्गम् ॅलो॒कम् । \newline
56. सु॒व॒र्गम् ॅलो॒कम् ॅलो॒कꣳ सु॑व॒र्गꣳ सु॑व॒र्गम् ॅलो॒क मा॑यन् नायन् ॅलो॒कꣳ सु॑व॒र्गꣳ सु॑व॒र्गम् ॅलो॒क मा॑यन्न् । \newline
57. सु॒व॒र्गमिति॑ सुवः - गम् । \newline
58. लो॒क मा॑यन् नायन् ॅलो॒कम् ॅलो॒क मा॑य॒न् तेन॒ तेना॑यन् ॅलो॒कम् ॅलो॒क मा॑य॒न् तेन॑ । \newline
59. आ॒य॒न् तेन॒ तेना॑यन् नाय॒न् तेन र्.ष॑य॒ ऋष॑य॒ स्तेना॑यन् नाय॒न् तेन र्.ष॑यः । \newline
60. तेन र्.ष॑य॒ ऋष॑य॒ स्तेन॒ तेन र्.ष॑यो ऽश्राम्यन् नश्राम्य॒न् नृष॑य॒ स्तेन॒ तेन र्.ष॑यो ऽश्राम्यन्न् । \newline
61. ऋष॑यो ऽश्राम्यन् नश्राम्य॒न् नृष॑य॒ ऋष॑यो ऽश्राम्य॒न् ते ते᳚ ऽश्राम्य॒न् नृष॑य॒ ऋष॑यो ऽश्राम्य॒न् ते । \newline
\pagebreak
\markright{ TS 5.3.5.4  \hfill https://www.vedavms.in \hfill}

\section{ TS 5.3.5.4 }

\textbf{TS 5.3.5.4 } \newline
\textbf{Samhita Paata} \newline

ऽश्राम्य॒न् ते तपो॑ऽतप्यन्त॒ तानि॒ तप॑साऽपश्य॒न् तेभ्य॑ ए॒ता इष्ट॑का॒ निर॑मिम॒तेव॒श्छन्दो॒ वरि॑व॒श्छन्द॒ इति॒ ता उपा॑दधत॒ ताभि॒र्वै ते सु॑व॒र्गं ॅलो॒कमा॑य॒न॒. यदे॒ता इष्ट॑का उप॒दधा॑ति॒ यान्ये॒व छन्दाꣳ॑सि सुव॒र्ग्या॑णि॒ तैरे॒व यज॑मानः सुव॒र्गं ॅलो॒कमे॑ति य॒ज्ञेन॒ वै प्र॒जाप॑तिः प्र॒जा अ॑सृजत॒ ताः स्तोम॑ भागैरे॒वाऽसृ॑जत॒ यथ् - [  ] \newline

\textbf{Pada Paata} \newline

अ॒श्रा॒म्य॒न्न् । ते । तपः॑ । अ॒त॒प्य॒न्त॒ । तानि॑ । तप॑सा । अ॒प॒श्य॒न्न् । तेभ्यः॑ । ए॒ताः । इष्ट॑काः । निरिति॑ । अ॒मि॒म॒त॒ । एवः॑ । छन्दः॑ । वरि॑वः । छन्दः॑ । इति॑ । ताः । उपेति॑ । अ॒द॒ध॒त॒ । ताभिः॑ । वै । ते । सु॒व॒र्गमिति॑ सुवः-गम् । लो॒कम् । आ॒य॒न्न् । यत् । ए॒ताः । इष्ट॑काः । उ॒प॒दधा॒तीत्यु॑प - दधा॑ति । यानि॑ । ए॒व । छन्दाꣳ॑सि । सु॒व॒र्ग्या॑णीति॑ सुवः-ग्या॑नि । तैः । ए॒व । यज॑मानः । सु॒व॒र्गमिति॑ सुवः - गम् । लो॒कम् । ए॒ति॒ । य॒ज्ञेन॑ । वै । प्र॒जाप॑ति॒रिति॑ प्र॒जा - प॒तिः॒ । प्र॒जा इति॑ प्र - जाः । अ॒सृ॒ज॒त॒ । ताः । स्तोम॑भागै॒रिति॒ स्तोम॑ - भा॒गैः॒ । ए॒व । अ॒सृ॒ज॒त॒ । यत् ।  \newline


\textbf{Krama Paata} \newline

अ॒श्रा॒म्य॒न् ते । ते तपः॑ । तपो॑ऽतप्यन्न् । अ॒त॒प्य॒न् तानि॑ । तानि॒ तप॑सा । तप॑साऽपश्यन्न् । अ॒प॒श्य॒न् तेभ्यः॑ । तेभ्य॑ ए॒ताः । ए॒ता इष्ट॑काः । इष्ट॑का॒ निः । निर॑मिमत । अ॒मि॒म॒तेवः॑ । एव॒श्छन्दः॑ । छन्दो॒ वरि॑वः । वरि॑व॒श्छन्दः॑ । छन्द॒ इति॑ । इति॒ ताः । ता उप॑ । उपा॑दधत । अ॒द॒ध॒त॒ ताभिः॑ । ताभि॒र् वै । वै ते । ते सु॑व॒र्गम् । सु॒व॒र्गम् ॅलो॒कम् । सु॒व॒र्गमिति॑ सुवः - गम् । लो॒कमा॑यन्न् । आ॒य॒न्॒. यत् । यदे॒ताः । ए॒ता इष्ट॑काः । इष्ट॑का उप॒दधा॑ति । उ॒प॒दधा॑ति॒ यानि॑ । उ॒प॒दधा॒तीत्यु॑प - दधा॑ति । यान्ये॒व । ए॒व छन्दाꣳ॑सि । छन्दाꣳ॑सि सुव॒र्ग्या॑णि । सु॒व॒र्ग्या॑णि॒ तैः । सु॒व॒र्ग्या॑णीति॑ सुवः - ग्या॑नि । तैरे॒व । ए॒व यज॑मानः । यज॑मानः सुव॒र्गम् । सु॒व॒र्गम् ॅलो॒कम् । सु॒व॒र्गमिति॑ सुवः - गम् । लो॒कमे॑ति । ए॒ति॒ य॒ज्ञेन॑ । य॒ज्ञेन॒ वै । वै प्र॒जाप॑तिः । प्र॒जाप॑तिः प्र॒जाः । प्र॒जाप॑ति॒रिति॑ प्र॒जा - प॒तिः॒ । प्र॒जा अ॑सृजत । प्र॒जा इति॑ प्र - जाः । अ॒सृ॒ज॒त॒ ताः । ताः स्तोम॑भागैः । स्तोम॑भागैरे॒व । स्तोम॑भागै॒रिति॒ स्तोम॑ - भा॒गैः॒ । ए॒वासृ॑जत । अ॒सृ॒ज॒त॒ यत् ( ) । यथ् स्तोम॑भागाः \newline

\textbf{Jatai Paata} \newline

1. अ॒श्रा॒म्य॒न् ते ते᳚ ऽश्राम्यन् नश्राम्य॒न् ते । \newline
2. ते तप॒ स्तप॒ स्ते ते तपः॑ । \newline
3. तपो॑ ऽतप्यन्ता तप्यन्त॒ तप॒ स्तपो॑ ऽतप्यन्त । \newline
4. अ॒त॒प्य॒न्त॒ तानि॒ तान्य॑तप्यन्ता तप्यन्त॒ तानि॑ । \newline
5. तानि॒ तप॑सा॒ तप॑सा॒ तानि॒ तानि॒ तप॑सा । \newline
6. तप॑सा ऽपश्यन् नपश्य॒न् तप॑सा॒ तप॑सा ऽपश्यन्न् । \newline
7. अ॒प॒श्य॒न् तेभ्य॒ स्तेभ्यो॑ ऽपश्यन् नपश्य॒न् तेभ्यः॑ । \newline
8. तेभ्य॑ ए॒ता ए॒ता स्तेभ्य॒ स्तेभ्य॑ ए॒ताः । \newline
9. ए॒ता इष्ट॑का॒ इष्ट॑का ए॒ता ए॒ता इष्ट॑काः । \newline
10. इष्ट॑का॒ निर् णिरिष्ट॑का॒ इष्ट॑का॒ निः । \newline
11. निर॑मिमता मिमत॒ निर् णिर॑मिमत । \newline
12. अ॒मि॒म॒ तेव॒ एवो॑ ऽमिमता मिम॒तेवः॑ । \newline
13. एव॒ श्छन्द॒ श्छन्द॒ एव॒ एव॒ श्छन्दः॑ । \newline
14. छन्दो॒ वरि॑वो॒ वरि॑व॒ श्छन्द॒ श्छन्दो॒ वरि॑वः । \newline
15. वरि॑व॒ श्छन्द॒ श्छन्दो॒ वरि॑वो॒ वरि॑व॒ श्छन्दः॑ । \newline
16. छन्द॒ इतीति॒ छन्द॒ श्छन्द॒ इति॑ । \newline
17. इति॒ ता स्ता इतीति॒ ताः । \newline
18. ता उपोप॒ ता स्ता उप॑ । \newline
19. उपा॑ दधता दध॒तोपोपा॑ दधत । \newline
20. अ॒द॒ध॒त॒ ताभि॒ स्ताभि॑ रदधता दधत॒ ताभिः॑ । \newline
21. ताभि॒र् वै वै ताभि॒ स्ताभि॒र् वै । \newline
22. वै ते ते वै वै ते । \newline
23. ते सु॑व॒र्गꣳ सु॑व॒र्गम् ते ते सु॑व॒र्गम् । \newline
24. सु॒व॒र्गम् ॅलो॒कम् ॅलो॒कꣳ सु॑व॒र्गꣳ सु॑व॒र्गम् ॅलो॒कम् । \newline
25. सु॒व॒र्गमिति॑ सुवः - गम् । \newline
26. लो॒क मा॑यन् नायन् ॅलो॒कम् ॅलो॒क मा॑यन्न् । \newline
27. आ॒य॒न्॒. यद् यदा॑यन् नाय॒न्॒. यत् । \newline
28. यदे॒ता ए॒ता यद् यदे॒ताः । \newline
29. ए॒ता इष्ट॑का॒ इष्ट॑का ए॒ता ए॒ता इष्ट॑काः । \newline
30. इष्ट॑का उप॒दधा᳚त्यु प॒दधा॒ती ष्ट॑का॒ इष्ट॑का उप॒दधा॑ति । \newline
31. उ॒प॒दधा॑ति॒ यानि॒ यान्यु॑प॒दधा᳚ त्युप॒दधा॑ति॒ यानि॑ । \newline
32. उ॒प॒दधा॒तीत्यु॑प - दधा॑ति । \newline
33. यान्ये॒ वैव यानि॒ यान्ये॒व । \newline
34. ए॒व छन्दाꣳ॑सि॒ छन्दाꣳ॑ स्ये॒वैव छन्दाꣳ॑सि । \newline
35. छन्दाꣳ॑सि सुव॒र्ग्या॑णि सुव॒र्ग्या॑णि॒ छन्दाꣳ॑सि॒ छन्दाꣳ॑सि सुव॒र्ग्या॑णि । \newline
36. सु॒व॒र्ग्या॑णि॒ तै स्तैः सु॑व॒र्ग्या॑णि सुव॒र्ग्या॑णि॒ तैः । \newline
37. सु॒व॒र्ग्या॑णीति॑ सुवः - ग्या॑नि । \newline
38. तै रे॒वैव तै स्तै रे॒व । \newline
39. ए॒व यज॑मानो॒ यज॑मान ए॒वैव यज॑मानः । \newline
40. यज॑मानः सुव॒र्गꣳ सु॑व॒र्गं ॅयज॑मानो॒ यज॑मानः सुव॒र्गम् । \newline
41. सु॒व॒र्गम् ॅलो॒कम् ॅलो॒कꣳ सु॑व॒र्गꣳ सु॑व॒र्गम् ॅलो॒कम् । \newline
42. सु॒व॒र्गमिति॑ सुवः - गम् । \newline
43. लो॒क मे᳚त्येति लो॒कम् ॅलो॒क मे॑ति । \newline
44. ए॒ति॒ य॒ज्ञेन॑ य॒ज्ञेनै᳚ त्येति य॒ज्ञेन॑ । \newline
45. य॒ज्ञेन॒ वै वै य॒ज्ञेन॑ य॒ज्ञेन॒ वै । \newline
46. वै प्र॒जाप॑तिः प्र॒जाप॑ति॒र् वै वै प्र॒जाप॑तिः । \newline
47. प्र॒जाप॑तिः प्र॒जाः प्र॒जाः प्र॒जाप॑तिः प्र॒जाप॑तिः प्र॒जाः । \newline
48. प्र॒जाप॑ति॒रिति॑ प्र॒जा - प॒तिः॒ । \newline
49. प्र॒जा अ॑सृजता सृजत प्र॒जाः प्र॒जा अ॑सृजत । \newline
50. प्र॒जा इति॑ प्र - जाः । \newline
51. अ॒सृ॒ज॒त॒ तास्ता अ॑सृजता सृजत॒ ताः । \newline
52. ताः स्तोम॑भागैः॒ स्तोम॑भागै॒ स्तास्ताः स्तोम॑भागैः । \newline
53. स्तोम॑भागै रे॒वैव स्तोम॑भागैः॒ स्तोम॑भागै रे॒व । \newline
54. स्तोम॑भागै॒रिति॒ स्तोम॑ - भा॒गैः॒ । \newline
55. ए॒वा सृ॑जता सृजतै॒ वैवा सृ॑जत । \newline
56. अ॒सृ॒ज॒त॒ यद् यद॑सृजता सृजत॒ यत् । \newline
57. यथ् स्तोम॑भागाः॒ स्तोम॑भागा॒ यद् यथ् स्तोम॑भागाः । \newline

\textbf{Ghana Paata } \newline

1. अ॒श्रा॒म्य॒न् ते ते᳚ ऽश्राम्यन् नश्राम्य॒न् ते तप॒ स्तप॒ स्ते᳚ ऽश्राम्यन् नश्राम्य॒न् ते तपः॑ । \newline
2. ते तप॒ स्तप॒ स्ते ते तपो॑ ऽतप्यन्ता तप्यन्त॒ तप॒ स्ते ते तपो॑ ऽतप्यन्त । \newline
3. तपो॑ ऽतप्यन्ता तप्यन्त॒ तप॒ स्तपो॑ ऽतप्यन्त॒ तानि॒ तान्य॑ तप्यन्त॒ तप॒ स्तपो॑ ऽतप्यन्त॒ तानि॑ । \newline
4. अ॒त॒प्य॒न्त॒ तानि॒ ता न्य॑तप्यन्ता तप्यन्त॒ तानि॒ तप॑सा॒ तप॑सा॒ ता न्य॑तप्यन्ता तप्यन्त॒ तानि॒ तप॑सा । \newline
5. तानि॒ तप॑सा॒ तप॑सा॒ तानि॒ तानि॒ तप॑सा ऽपश्यन् नपश्य॒न् तप॑सा॒ तानि॒ तानि॒ तप॑सा ऽपश्यन्न् । \newline
6. तप॑सा ऽपश्यन् नपश्य॒न् तप॑सा॒ तप॑सा ऽपश्य॒न् तेभ्य॒ स्तेभ्यो॑ ऽपश्य॒न् तप॑सा॒ तप॑सा ऽपश्य॒न् तेभ्यः॑ । \newline
7. अ॒प॒श्य॒न् तेभ्य॒ स्तेभ्यो॑ ऽपश्यन् नपश्य॒न् तेभ्य॑ ए॒ता ए॒ता स्तेभ्यो॑ ऽपश्यन् नपश्य॒न् तेभ्य॑ ए॒ताः । \newline
8. तेभ्य॑ ए॒ता ए॒ता स्तेभ्य॒ स्तेभ्य॑ ए॒ता इष्ट॑का॒ इष्ट॑का ए॒ता स्तेभ्य॒ स्तेभ्य॑ ए॒ता इष्ट॑काः । \newline
9. ए॒ता इष्ट॑का॒ इष्ट॑का ए॒ता ए॒ता इष्ट॑का॒ निर् णिरिष्ट॑का ए॒ता ए॒ता इष्ट॑का॒ निः । \newline
10. इष्ट॑का॒ निर् णिरिष्ट॑का॒ इष्ट॑का॒ निर॑मिमता मिमत॒ निरिष्ट॑का॒ इष्ट॑का॒ निर॑मिमत । \newline
11. निर॑मिमता मिमत॒ निर् णि र॑मिम॒तेव॒ एवो॑ ऽमिमत॒ निर् णि र॑मिम॒तेवः॑ । \newline
12. अ॒मि॒म॒तेव॒ एवो॑ ऽमिमता मिम॒तेव॒ श्छन्द॒ श्छन्द॒ एवो॑ ऽमिमता मिम॒तेव॒ श्छन्दः॑ । \newline
13. एव॒ श्छन्द॒ श्छन्द॒ एव॒ एव॒ श्छन्दो॒ वरि॑वो॒ वरि॑व॒ श्छन्द॒ एव॒ एव॒ श्छन्दो॒ वरि॑वः । \newline
14. छन्दो॒ वरि॑वो॒ वरि॑व॒ श्छन्द॒ श्छन्दो॒ वरि॑व॒ श्छन्द॒ श्छन्दो॒ वरि॑व॒ श्छन्द॒ श्छन्दो॒ वरि॑व॒ श्छन्दः॑ । \newline
15. वरि॑व॒ श्छन्द॒ श्छन्दो॒ वरि॑वो॒ वरि॑व॒ श्छन्द॒ इतीति॒ छन्दो॒ वरि॑वो॒ वरि॑व॒ श्छन्द॒ इति॑ । \newline
16. छन्द॒ इतीति॒ छन्द॒ श्छन्द॒ इति॒ ता स्ता इति॒ छन्द॒ श्छन्द॒ इति॒ ताः । \newline
17. इति॒ ता स्ता इतीति॒ ता उपोप॒ ता इतीति॒ ता उप॑ । \newline
18. ता उपोप॒ ता स्ता उपा॑दधता दध॒तोप॒ ता स्ता उपा॑दधत । \newline
19. उपा॑दधता दध॒तोपोपा॑ दधत॒ ताभि॒ स्ताभि॑ रदध॒तोपोपा॑ दधत॒ ताभिः॑ । \newline
20. अ॒द॒ध॒त॒ ताभि॒ स्ताभि॑ रदधता दधत॒ ताभि॒र् वै वै ताभि॑ रदधता दधत॒ ताभि॒र् वै । \newline
21. ताभि॒र् वै वै ताभि॒ स्ताभि॒र् वै ते ते वै ताभि॒ स्ताभि॒र् वै ते । \newline
22. वै ते ते वै वै ते सु॑व॒र्गꣳ सु॑व॒र्गम् ते वै वै ते सु॑व॒र्गम् । \newline
23. ते सु॑व॒र्गꣳ सु॑व॒र्गम् ते ते सु॑व॒र्गम् ॅलो॒कम् ॅलो॒कꣳ सु॑व॒र्गम् ते ते सु॑व॒र्गम् ॅलो॒कम् । \newline
24. सु॒व॒र्गम् ॅलो॒कम् ॅलो॒कꣳ सु॑व॒र्गꣳ सु॑व॒र्गम् ॅलो॒क मा॑यन् नायन् ॅलो॒कꣳ सु॑व॒र्गꣳ सु॑व॒र्गम् ॅलो॒क मा॑यन्न् । \newline
25. सु॒व॒र्गमिति॑ सुवः - गम् । \newline
26. लो॒क मा॑यन् नायन् ॅलो॒कम् ॅलो॒क मा॑य॒न्॒. यद् यदा॑यन् ॅलो॒कम् ॅलो॒क मा॑य॒न्॒. यत् । \newline
27. आ॒य॒न्॒. यद् यदा॑यन् नाय॒न्॒. यदे॒ता ए॒ता यदा॑यन् नाय॒न्॒. यदे॒ताः । \newline
28. यदे॒ता ए॒ता यद् यदे॒ता इष्ट॑का॒ इष्ट॑का ए॒ता यद् यदे॒ता इष्ट॑काः । \newline
29. ए॒ता इष्ट॑का॒ इष्ट॑का ए॒ता ए॒ता इष्ट॑का उप॒दधा᳚ त्युप॒दधा॒ती ष्ट॑का ए॒ता ए॒ता इष्ट॑का उप॒दधा॑ति । \newline
30. इष्ट॑का उप॒दधा᳚ त्युप॒दधा॒ती ष्ट॑का॒ इष्ट॑का उप॒दधा॑ति॒ यानि॒ यान्यु॑प॒दधा॒ती ष्ट॑का॒ इष्ट॑का उप॒दधा॑ति॒ यानि॑ । \newline
31. उ॒प॒दधा॑ति॒ यानि॒ यान्यु॑प॒दधा᳚ त्युप॒दधा॑ति॒ यान्ये॒वैव यान्यु॑प॒दधा᳚ त्युप॒दधा॑ति॒ यान्ये॒व । \newline
32. उ॒प॒दधा॒तीत्यु॑प - दधा॑ति । \newline
33. यान्ये॒वैव यानि॒ यान्ये॒व छन्दाꣳ॑सि॒ छन्दाꣳ॑स्ये॒व यानि॒ यान्ये॒व छन्दाꣳ॑सि । \newline
34. ए॒व छन्दाꣳ॑सि॒ छन्दाꣳ॑ स्ये॒वैव छन्दाꣳ॑सि सुव॒र्ग्या॑णि सुव॒र्ग्या॑णि॒ छन्दाꣳ॑ स्ये॒वैव छन्दाꣳ॑सि सुव॒र्ग्या॑णि । \newline
35. छन्दाꣳ॑सि सुव॒र्ग्या॑णि सुव॒र्ग्या॑णि॒ छन्दाꣳ॑सि॒ छन्दाꣳ॑सि सुव॒र्ग्या॑णि॒ तै स्तैः सु॑व॒र्ग्या॑णि॒ छन्दाꣳ॑सि॒ छन्दाꣳ॑सि सुव॒र्ग्या॑णि॒ तैः । \newline
36. सु॒व॒र्ग्या॑णि॒ तै स्तैः सु॑व॒र्ग्या॑णि सुव॒र्ग्या॑णि॒ तै रे॒वैव तैः सु॑व॒र्ग्या॑णि सुव॒र्ग्या॑णि॒ तै रे॒व । \newline
37. सु॒व॒र्ग्या॑णीति॑ सुवः - ग्या॑नि । \newline
38. तै रे॒वैव तै स्तै रे॒व यज॑मानो॒ यज॑मान ए॒व तै स्तै रे॒व यज॑मानः । \newline
39. ए॒व यज॑मानो॒ यज॑मान ए॒वैव यज॑मानः सुव॒र्गꣳ सु॑व॒र्गं ॅयज॑मान ए॒वैव यज॑मानः सुव॒र्गम् । \newline
40. यज॑मानः सुव॒र्गꣳ सु॑व॒र्गं ॅयज॑मानो॒ यज॑मानः सुव॒र्गम् ॅलो॒कम् ॅलो॒कꣳ सु॑व॒र्गं ॅयज॑मानो॒ यज॑मानः सुव॒र्गम् ॅलो॒कम् । \newline
41. सु॒व॒र्गम् ॅलो॒कम् ॅलो॒कꣳ सु॑व॒र्गꣳ सु॑व॒र्गम् ॅलो॒क मे᳚त्येति लो॒कꣳ सु॑व॒र्गꣳ सु॑व॒र्गम् ॅलो॒क मे॑ति । \newline
42. सु॒व॒र्गमिति॑ सुवः - गम् । \newline
43. लो॒क मे᳚त्येति लो॒कम् ॅलो॒क मे॑ति य॒ज्ञेन॑ य॒ज्ञेनै॑ति लो॒कम् ॅलो॒क मे॑ति य॒ज्ञेन॑ । \newline
44. ए॒ति॒ य॒ज्ञेन॑ य॒ज्ञेनै᳚ त्येति य॒ज्ञेन॒ वै वै य॒ज्ञेनै᳚ त्येति य॒ज्ञेन॒ वै । \newline
45. य॒ज्ञेन॒ वै वै य॒ज्ञेन॑ य॒ज्ञेन॒ वै प्र॒जाप॑तिः प्र॒जाप॑ति॒र् वै य॒ज्ञेन॑ य॒ज्ञेन॒ वै प्र॒जाप॑तिः । \newline
46. वै प्र॒जाप॑तिः प्र॒जाप॑ति॒र् वै वै प्र॒जाप॑तिः प्र॒जाः प्र॒जाः प्र॒जाप॑ति॒र् वै वै प्र॒जाप॑तिः प्र॒जाः । \newline
47. प्र॒जाप॑तिः प्र॒जाः प्र॒जाः प्र॒जाप॑तिः प्र॒जाप॑तिः प्र॒जा अ॑सृजता सृजत प्र॒जाः प्र॒जाप॑तिः प्र॒जाप॑तिः प्र॒जा अ॑सृजत । \newline
48. प्र॒जाप॑ति॒रिति॑ प्र॒जा - प॒तिः॒ । \newline
49. प्र॒जा अ॑सृजता सृजत प्र॒जाः प्र॒जा अ॑सृजत॒ ता स्ता अ॑सृजत प्र॒जाः प्र॒जा अ॑सृजत॒ ताः । \newline
50. प्र॒जा इति॑ प्र - जाः । \newline
51. अ॒सृ॒ज॒त॒ ता स्ता अ॑सृजता सृजत॒ ताः स्तोम॑भागैः॒ स्तोम॑भागै॒ स्ता अ॑सृजता सृजत॒ ताः स्तोम॑भागैः । \newline
52. ताः स्तोम॑भागैः॒ स्तोम॑भागै॒ स्ता स्ताः स्तोम॑भागै रे॒वैव स्तोम॑भागै॒ स्ता स्ताः स्तोम॑भागै रे॒व । \newline
53. स्तोम॑भागै रे॒वैव स्तोम॑भागैः॒ स्तोम॑भागै रे॒वासृ॑जता सृजतै॒व स्तोम॑भागैः॒ स्तोम॑भागै रे॒वासृ॑जत । \newline
54. स्तोम॑भागै॒रिति॒ स्तोम॑ - भा॒गैः॒ । \newline
55. ए॒वा सृ॑जता सृजतै॒ वैवा सृ॑जत॒ यद् य द॑सृजतै॒ वैवा सृ॑जत॒ यत् । \newline
56. अ॒सृ॒ज॒त॒ यद् यद॑सृजता सृजत॒ यथ् स्तोम॑भागाः॒ स्तोम॑भागा॒ यद॑सृजता सृजत॒ यथ् स्तोम॑भागाः । \newline
57. यथ् स्तोम॑भागाः॒ स्तोम॑भागा॒ यद् यथ् स्तोम॑भागा उप॒दधा᳚ त्युप॒दधा॑ति॒ स्तोम॑भागा॒ यद् यथ् स्तोम॑भागा उप॒दधा॑ति । \newline
\pagebreak
\markright{ TS 5.3.5.5  \hfill https://www.vedavms.in \hfill}

\section{ TS 5.3.5.5 }

\textbf{TS 5.3.5.5 } \newline
\textbf{Samhita Paata} \newline

स्तोम॑ भागा उप॒दधा॑ति प्र॒जा ए॒व तद्-यज॑मानः सृजते॒ बृह॒स्पति॒र्वा ए॒तद्-य॒ज्ञ्स्य॒ तेजः॒ सम॑भर॒द्यथ् स्तोम॑भागा॒ यथ् स्तोम॑भागा उप॒दधा॑ति॒ सते॑जसमे॒वाग्निं चि॑नुते॒ बृह॒स्पति॒र्वा ए॒तां ॅय॒ज्ञ्स्य॑ प्रति॒ष्ठाम॑पश्य॒द्यथ् स्तोम॑भागा॒ यथ् स्तोम॑भागा उप॒दधा॑ति य॒ज्ञ्स्य॒ प्रति॑ष्ठित्यै स॒प्तस॒प्तोप॑ दधाति सवीर्य॒त्वाय॑ ति॒स्रो मद्ध्ये॒ प्रति॑ष्ठित्यै ॥ \newline

\textbf{Pada Paata} \newline

स्तोम॑भागा॒ इति॒ स्तोम॑ - भा॒गाः॒ । उ॒प॒दधा॒तीत्यु॑प - दधा॑ति । प्र॒जा इति॑ प्र - जाः । ए॒व । तत् । यज॑मानः । सृ॒ज॒ते॒ । बृह॒स्पतिः॑ । वै । ए॒तत् । य॒ज्ञ्स्य॑ । तेजः॑ । समिति॑ । अ॒भ॒र॒त् । यत् । स्तोम॑भागा॒ इति॒ स्तोम॑ - भा॒गाः॒ । यत् । स्तोम॑भागा॒ इति॒ स्तोम॑ - भा॒गाः॒ । उ॒प॒दधा॒तीत्यु॑प - दधा॑ति । सते॑जस॒मिति॒ स - ते॒ज॒स॒म् । ए॒व । अ॒ग्निम् । चि॒नु॒ते॒ । बृह॒स्पतिः॑ । वै । ए॒ताम् । य॒ज्ञ्स्य॑ । प्र॒ति॒ष्ठामिति॑ प्रति - स्थाम् । अ॒प॒श्य॒त् । यत् । स्तोम॑भागा॒ इति॒ स्तोम॑ - भा॒गाः॒ । यत् । स्तोम॑भागा॒ इति॒ स्तोम॑ - भा॒गाः॒ । उ॒प॒दधा॒तीत्यु॑प - दधा॑ति । य॒ज्ञ्स्य॑ । प्रति॑ष्ठित्या॒ इति॒ प्रति॑-स्थि॒त्यै॒ । स॒प्तस॒प्तेति॑ स॒प्त - स॒प्त॒ । उपेति॑ । द॒धा॒ति॒ । स॒वी॒र्य॒त्वायेति॑ सवीर्य - त्वाय॑ । ति॒स्रः । मद्ध्ये᳚ । प्रति॑ष्ठित्या॒ इति॒ प्रति॑ - स्थि॒त्यै॒ ॥  \newline


\textbf{Krama Paata} \newline

स्तोम॑भागा उप॒दधा॑ति । स्तोम॑भागा॒ इति॒ स्तोम॑ - भा॒गाः॒ । उ॒प॒दधा॑ति प्र॒जाः । उ॒प॒दधा॒तीत्यु॑प - दधा॑ति । प्र॒जा ए॒व । प्र॒जा इति॑ प्र - जाः । ए॒व तत् । तद् यज॑मानः । यज॑मानः सृजते । सृ॒ज॒ते॒ बृह॒स्पतिः॑ । बृह॒स्पति॒र् वै । वा ए॒तत् । ए॒तद् य॒ज्ञ्स्य॑ । य॒ज्ञ्स्य॒ तेजः॑ । तेजः॒ सम् । सम॑भरत् । अ॒भ॒र॒द् यत् । यथ् स्तोम॑भागाः । स्तोम॑भागा॒ यत् । स्तोम॑भागा॒ इति॒ स्तोम॑ - भा॒गाः॒ । यथ् स्तोम॑भागाः । स्तोम॑भागा उप॒दधा॑ति । स्तोम॑भागा॒ इति॒ स्तोम॑ - भा॒गाः॒ । उ॒प॒दधा॑ति॒ सते॑जसम् । उ॒प॒दधा॒तीत्यु॑प - दधा॑ति । सते॑जसमे॒व । सते॑जस॒मिति॒ स - ते॒ज॒स॒म् । ए॒वाग्निम् । अ॒ग्निम् चि॑नुते । चि॒नु॒ते॒ बृह॒स्पतिः॑ । बृह॒स्पति॒र् वै । वा ए॒ताम् । ए॒ताम् ॅय॒ज्ञ्स्य॑ । य॒ज्ञ्स्य॑ प्रति॒ष्ठाम् । प्र॒ति॒ष्ठाम॑पश्यत् । प्र॒ति॒ष्ठामिति॑ प्रति - स्थाम् । अ॒प॒श्य॒द् यत् । यथ् स्तोम॑भागाः । स्तोम॑भागा॒ यत् । स्तोम॑भागा॒ इति॒ स्तोम॑ - भा॒गाः॒ । यथ् स्तोम॑भागाः । स्तोम॑भागा उप॒दधा॑ति । स्तोम॑भागा॒ इति॒ स्तोम॑ - भा॒गाः॒ । उ॒प॒दधा॑ति य॒ज्ञ्स्य॑ । उ॒प॒दधा॒तीत्यु॑प - दधा॑ति । य॒ज्ञ्स्य॒ प्रति॑ष्ठित्यै । प्रति॑ष्ठित्यै स॒प्तस॑प्त । प्रति॑ष्ठित्या॒ इति॒ प्रति॑ - स्थि॒त्यै॒ । स॒प्तस॒प्तोप॑ । स॒प्तस॒प्तेति॑ स॒प्त - स॒प्त॒ । उप॑ दधाति । द॒धा॒ति॒ स॒वी॒र्य॒त्वाय॑ । स॒वी॒र्य॒त्वाय॑ ति॒स्रः । स॒वी॒र्य॒त्वायेति॑ सवीर्य - त्वाय॑ । ति॒स्रो मद्ध्ये᳚ । मद्ध्ये॒ प्रति॑ष्ठित्यै । प्रति॑ष्ठित्या॒ इति॒ प्रति॑ - स्थि॒त्यै॒ । \newline

\textbf{Jatai Paata} \newline

1. स्तोम॑भागा उप॒दधा᳚ त्युप॒दधा॑ति॒ स्तोम॑भागाः॒ स्तोम॑भागा उप॒दधा॑ति । \newline
2. स्तोम॑भागा॒ इति॒ स्तोम॑ - भा॒गाः॒ । \newline
3. उ॒प॒दधा॑ति प्र॒जाः प्र॒जा उ॑प॒दधा᳚ त्युप॒दधा॑ति प्र॒जाः । \newline
4. उ॒प॒दधा॒तीत्यु॑प - दधा॑ति । \newline
5. प्र॒जा ए॒वैव प्र॒जाः प्र॒जा ए॒व । \newline
6. प्र॒जा इति॑ प्र - जाः । \newline
7. ए॒व तत् तदे॒ वैव तत् । \newline
8. तद् यज॑मानो॒ यज॑मान॒ स्तत् तद् यज॑मानः । \newline
9. यज॑मानः सृजते सृजते॒ यज॑मानो॒ यज॑मानः सृजते । \newline
10. सृ॒ज॒ते॒ बृह॒स्पति॒र् बृह॒स्पतिः॑ सृजते सृजते॒ बृह॒स्पतिः॑ । \newline
11. बृह॒स्पति॒र् वै वै बृह॒स्पति॒र् बृह॒स्पति॒र् वै । \newline
12. वा ए॒त दे॒तद् वै वा ए॒तत् । \newline
13. ए॒तद् य॒ज्ञ्स्य॑ य॒ज्ञ् स्यै॒त दे॒तद् य॒ज्ञ्स्य॑ । \newline
14. य॒ज्ञ्स्य॒ तेज॒ स्तेजो॑ य॒ज्ञ्स्य॑ य॒ज्ञ्स्य॒ तेजः॑ । \newline
15. तेजः॒ सꣳ सम् तेज॒ स्तेजः॒ सम् । \newline
16. स म॑भर दभर॒थ् सꣳ स म॑भरत् । \newline
17. अ॒भ॒र॒द् यद् यद॑भर दभर॒द् यत् । \newline
18. यथ् स्तोम॑भागाः॒ स्तोम॑भागा॒ यद् यथ् स्तोम॑भागाः । \newline
19. स्तोम॑भागा॒ यद् यथ् स्तोम॑भागाः॒ स्तोम॑भागा॒ यत् । \newline
20. स्तोम॑भागा॒ इति॒ स्तोम॑ - भा॒गाः॒ । \newline
21. यथ् स्तोम॑भागाः॒ स्तोम॑भागा॒ यद् यथ् स्तोम॑भागाः । \newline
22. स्तोम॑भागा उप॒दधा᳚ त्युप॒दधा॑ति॒ स्तोम॑भागाः॒ स्तोम॑भागा उप॒दधा॑ति । \newline
23. स्तोम॑भागा॒ इति॒ स्तोम॑ - भा॒गाः॒ । \newline
24. उ॒प॒दधा॑ति॒ सते॑जसꣳ॒॒ सते॑जस मुप॒दधा᳚ त्युप॒दधा॑ति॒ सते॑जसम् । \newline
25. उ॒प॒दधा॒तीत्यु॑प - दधा॑ति । \newline
26. सते॑जस मे॒वैव सते॑जसꣳ॒॒ सते॑जस मे॒व । \newline
27. सते॑जस॒मिति॒ स - ते॒ज॒स॒म् । \newline
28. ए॒वाग्नि म॒ग्नि मे॒वैवाग्निम् । \newline
29. अ॒ग्निम् चि॑नुते चिनुते॒ ऽग्नि म॒ग्निम् चि॑नुते । \newline
30. चि॒नु॒ते॒ बृह॒स्पति॒र् बृह॒स्पति॑ श्चिनुते चिनुते॒ बृह॒स्पतिः॑ । \newline
31. बृह॒स्पति॒र् वै वै बृह॒स्पति॒र् बृह॒स्पति॒र् वै । \newline
32. वा ए॒ता मे॒तां ॅवै वा ए॒ताम् । \newline
33. ए॒तां ॅय॒ज्ञ्स्य॑ य॒ज्ञ् स्यै॒ता मे॒तां ॅय॒ज्ञ्स्य॑ । \newline
34. य॒ज्ञ्स्य॑ प्रति॒ष्ठाम् प्र॑ति॒ष्ठां ॅय॒ज्ञ्स्य॑ य॒ज्ञ्स्य॑ प्रति॒ष्ठाम् । \newline
35. प्र॒ति॒ष्ठा म॑पश्य दपश्यत् प्रति॒ष्ठाम् प्र॑ति॒ष्ठा म॑पश्यत् । \newline
36. प्र॒ति॒ष्ठामिति॑ प्रति - स्थाम् । \newline
37. अ॒प॒श्य॒द् यद् यद॑पश्य दपश्य॒द् यत् । \newline
38. यथ् स्तोम॑भागाः॒ स्तोम॑भागा॒ यद् यथ् स्तोम॑भागाः । \newline
39. स्तोम॑भागा॒ यद् यथ् स्तोम॑भागाः॒ स्तोम॑भागा॒ यत् । \newline
40. स्तोम॑भागा॒ इति॒ स्तोम॑ - भा॒गाः॒ । \newline
41. यथ् स्तोम॑भागाः॒ स्तोम॑भागा॒ यद् यथ् स्तोम॑भागाः । \newline
42. स्तोम॑भागा उप॒दधा᳚ त्युप॒दधा॑ति॒ स्तोम॑भागाः॒ स्तोम॑भागा उप॒दधा॑ति । \newline
43. स्तोम॑भागा॒ इति॒ स्तोम॑ - भा॒गाः॒ । \newline
44. उ॒प॒दधा॑ति य॒ज्ञ्स्य॑ य॒ज्ञ्स्यो॑ प॒दधा᳚ त्युप॒दधा॑ति य॒ज्ञ्स्य॑ । \newline
45. उ॒प॒दधा॒तीत्यु॑प - दधा॑ति । \newline
46. य॒ज्ञ्स्य॒ प्रति॑ष्ठित्यै॒ प्रति॑ष्ठित्यै य॒ज्ञ्स्य॑ य॒ज्ञ्स्य॒ प्रति॑ष्ठित्यै । \newline
47. प्रति॑ष्ठित्यै स॒प्तस॑प्त स॒प्तस॑प्त॒ प्रति॑ष्ठित्यै॒ प्रति॑ष्ठित्यै स॒प्तस॑प्त । \newline
48. प्रति॑ष्ठित्या॒ इति॒ प्रति॑ - स्थि॒त्यै॒ । \newline
49. स॒प्तस॒प्तो पोप॑ स॒प्तस॑प्त स॒प्तस॒प्तोप॑ । \newline
50. स॒प्तस॒प्तेति॑ स॒प्त - स॒प्त॒ । \newline
51. उप॑ दधाति दधा॒ त्युपोप॑ दधाति । \newline
52. द॒धा॒ति॒ स॒वी॒र्य॒त्वाय॑ सवीर्य॒त्वाय॑ दधाति दधाति सवीर्य॒त्वाय॑ । \newline
53. स॒वी॒र्य॒त्वाय॑ ति॒स्र स्ति॒स्रः स॑वीर्य॒त्वाय॑ सवीर्य॒त्वाय॑ ति॒स्रः । \newline
54. स॒वी॒र्य॒त्वायेति॑ सवीर्य - त्वाय॑ । \newline
55. ति॒स्रो मद्ध्ये॒ मद्ध्ये॑ ति॒स्र स्ति॒स्रो मद्ध्ये᳚ । \newline
56. मद्ध्ये॒ प्रति॑ष्ठित्यै॒ प्रति॑ष्ठित्यै॒ मद्ध्ये॒ मद्ध्ये॒ प्रति॑ष्ठित्यै । \newline
57. प्रति॑ष्ठित्या॒ इति॒ प्रति॑ - स्थि॒त्यै॒ । \newline

\textbf{Ghana Paata } \newline

1. स्तोम॑भागा उप॒दधा᳚ त्युप॒दधा॑ति॒ स्तोम॑भागाः॒ स्तोम॑भागा उप॒दधा॑ति प्र॒जाः प्र॒जा उ॑प॒दधा॑ति॒ स्तोम॑भागाः॒ स्तोम॑भागा उप॒दधा॑ति प्र॒जाः । \newline
2. स्तोम॑भागा॒ इति॒ स्तोम॑ - भा॒गाः॒ । \newline
3. उ॒प॒दधा॑ति प्र॒जाः प्र॒जा उ॑प॒दधा᳚ त्युप॒दधा॑ति प्र॒जा ए॒वैव प्र॒जा उ॑प॒दधा᳚ त्युप॒दधा॑ति प्र॒जा ए॒व । \newline
4. उ॒प॒दधा॒तीत्यु॑प - दधा॑ति । \newline
5. प्र॒जा ए॒वैव प्र॒जाः प्र॒जा ए॒व तत् तदे॒व प्र॒जाः प्र॒जा ए॒व तत् । \newline
6. प्र॒जा इति॑ प्र - जाः । \newline
7. ए॒व तत् तदे॒ वैव तद् यज॑मानो॒ यज॑मान॒ स्तदे॒वैव तद् यज॑मानः । \newline
8. तद् यज॑मानो॒ यज॑मान॒ स्तत् तद् यज॑मानः सृजते सृजते॒ यज॑मान॒ स्तत् तद् यज॑मानः सृजते । \newline
9. यज॑मानः सृजते सृजते॒ यज॑मानो॒ यज॑मानः सृजते॒ बृह॒स्पति॒र् बृह॒स्पतिः॑ सृजते॒ यज॑मानो॒ यज॑मानः सृजते॒ बृह॒स्पतिः॑ । \newline
10. सृ॒ज॒ते॒ बृह॒स्पति॒र् बृह॒स्पतिः॑ सृजते सृजते॒ बृह॒स्पति॒र् वै वै बृह॒स्पतिः॑ सृजते सृजते॒ बृह॒स्पति॒र् वै । \newline
11. बृह॒स्पति॒र् वै वै बृह॒स्पति॒र् बृह॒स्पति॒र् वा ए॒त दे॒तद् वै बृह॒स्पति॒र् बृह॒स्पति॒र् वा ए॒तत् । \newline
12. वा ए॒त दे॒तद् वै वा ए॒तद् य॒ज्ञ्स्य॑ य॒ज्ञ् स्यै॒तद् वै वा ए॒तद् य॒ज्ञ्स्य॑ । \newline
13. ए॒तद् य॒ज्ञ्स्य॑ य॒ज्ञ् स्यै॒त दे॒तद् य॒ज्ञ्स्य॒ तेज॒ स्तेजो॑ य॒ज्ञ् स्यै॒त दे॒तद् य॒ज्ञ्स्य॒ तेजः॑ । \newline
14. य॒ज्ञ्स्य॒ तेज॒ स्तेजो॑ य॒ज्ञ्स्य॑ य॒ज्ञ्स्य॒ तेजः॒ सꣳ सम् तेजो॑ य॒ज्ञ्स्य॑ य॒ज्ञ्स्य॒ तेजः॒ सम् । \newline
15. तेजः॒ सꣳ सम् तेज॒ स्तेजः॒ स म॑भर दभर॒थ् सम् तेज॒ स्तेजः॒ स म॑भरत् । \newline
16. स म॑भर दभर॒थ् सꣳ स म॑भर॒द् यद् यद॑भर॒थ् सꣳ स म॑भर॒द् यत् । \newline
17. अ॒भ॒र॒द् यद् यद॑भर दभर॒द् यथ् स्तोम॑भागाः॒ स्तोम॑भागा॒ यद॑भर दभर॒द् यथ् स्तोम॑भागाः । \newline
18. यथ् स्तोम॑भागाः॒ स्तोम॑भागा॒ यद् यथ् स्तोम॑भागा॒ यद् यथ् स्तोम॑भागा॒ यद् यथ् स्तोम॑भागा॒ यत् । \newline
19. स्तोम॑भागा॒ यद् यथ् स्तोम॑भागाः॒ स्तोम॑भागा॒ यथ् स्तोम॑भागाः॒ स्तोम॑भागा॒ यथ् स्तोम॑भागाः॒ स्तोम॑भागा॒ यथ् स्तोम॑भागाः । \newline
20. स्तोम॑भागा॒ इति॒ स्तोम॑ - भा॒गाः॒ । \newline
21. यथ् स्तोम॑भागाः॒ स्तोम॑भागा॒ यद् यथ् स्तोम॑भागा उप॒दधा᳚ त्युप॒दधा॑ति॒ स्तोम॑भागा॒ यद् यथ् स्तोम॑भागा उप॒दधा॑ति । \newline
22. स्तोम॑भागा उप॒दधा᳚ त्युप॒दधा॑ति॒ स्तोम॑भागाः॒ स्तोम॑भागा उप॒दधा॑ति॒ सते॑जसꣳ॒॒ सते॑जस मुप॒दधा॑ति॒ स्तोम॑भागाः॒ स्तोम॑भागा उप॒दधा॑ति॒ सते॑जसम् । \newline
23. स्तोम॑भागा॒ इति॒ स्तोम॑ - भा॒गाः॒ । \newline
24. उ॒प॒दधा॑ति॒ सते॑जसꣳ॒॒ सते॑जस मुप॒दधा᳚ त्युप॒दधा॑ति॒ सते॑जस मे॒वैव सते॑जस मुप॒दधा᳚ त्युप॒दधा॑ति॒ सते॑जस मे॒व । \newline
25. उ॒प॒दधा॒तीत्यु॑प - दधा॑ति । \newline
26. सते॑जस मे॒वैव सते॑जसꣳ॒॒ सते॑जस मे॒वाग्नि म॒ग्नि मे॒व सते॑जसꣳ॒॒ सते॑जस मे॒वाग्निम् । \newline
27. सते॑जस॒मिति॒ स - ते॒ज॒स॒म् । \newline
28. ए॒वाग्नि म॒ग्नि मे॒वै वाग्निम् चि॑नुते चिनुते॒ ऽग्नि मे॒वै वाग्निम् चि॑नुते । \newline
29. अ॒ग्निम् चि॑नुते चिनुते॒ ऽग्नि म॒ग्निम् चि॑नुते॒ बृह॒स्पति॒र् बृह॒स्पति॑ श्चिनुते॒ ऽग्नि म॒ग्निम् चि॑नुते॒ बृह॒स्पतिः॑ । \newline
30. चि॒नु॒ते॒ बृह॒स्पति॒र् बृह॒स्पति॑ श्चिनुते चिनुते॒ बृह॒स्पति॒र् वै वै बृह॒स्पति॑ श्चिनुते चिनुते॒ बृह॒स्पति॒र् वै । \newline
31. बृह॒स्पति॒र् वै वै बृह॒स्पति॒र् बृह॒स्पति॒र् वा ए॒ता मे॒तां ॅवै बृह॒स्पति॒र् बृह॒स्पति॒र् वा ए॒ताम् । \newline
32. वा ए॒ता मे॒तां ॅवै वा ए॒तां ॅय॒ज्ञ्स्य॑ य॒ज्ञ्स्यै॒तां ॅवै वा ए॒तां ॅय॒ज्ञ्स्य॑ । \newline
33. ए॒तां ॅय॒ज्ञ्स्य॑ य॒ज्ञ्स्यै॒ता मे॒तां ॅय॒ज्ञ्स्य॑ प्रति॒ष्ठाम् प्र॑ति॒ष्ठां ॅय॒ज्ञ्स्यै॒ता मे॒तां ॅय॒ज्ञ्स्य॑ प्रति॒ष्ठाम् । \newline
34. य॒ज्ञ्स्य॑ प्रति॒ष्ठाम् प्र॑ति॒ष्ठां ॅय॒ज्ञ्स्य॑ य॒ज्ञ्स्य॑ प्रति॒ष्ठा म॑पश्य दपश्यत् प्रति॒ष्ठां ॅय॒ज्ञ्स्य॑ य॒ज्ञ्स्य॑ प्रति॒ष्ठा म॑पश्यत् । \newline
35. प्र॒ति॒ष्ठा म॑पश्य दपश्यत् प्रति॒ष्ठाम् प्र॑ति॒ष्ठा म॑पश्य॒द् यद् यद॑पश्यत् प्रति॒ष्ठाम् प्र॑ति॒ष्ठा म॑पश्य॒द् यत् । \newline
36. प्र॒ति॒ष्ठामिति॑ प्रति - स्थाम् । \newline
37. अ॒प॒श्य॒द् यद् यद॑पश्य दपश्य॒द् यथ् स्तोम॑भागाः॒ स्तोम॑भागा॒ यद॑पश्य दपश्य॒द् यथ् स्तोम॑भागाः । \newline
38. यथ् स्तोम॑भागाः॒ स्तोम॑भागा॒ यद् यथ् स्तोम॑भागा॒ यद् यथ् स्तोम॑भागा॒ यद् यथ् स्तोम॑भागा॒ यत् । \newline
39. स्तोम॑भागा॒ यद् यथ् स्तोम॑भागाः॒ स्तोम॑भागा॒ यथ् स्तोम॑भागाः॒ स्तोम॑भागा॒ यथ् स्तोम॑भागाः॒ स्तोम॑भागा॒ यथ् स्तोम॑भागाः । \newline
40. स्तोम॑भागा॒ इति॒ स्तोम॑ - भा॒गाः॒ । \newline
41. यथ् स्तोम॑भागाः॒ स्तोम॑भागा॒ यद् यथ् स्तोम॑भागा उप॒दधा᳚ त्युप॒दधा॑ति॒ स्तोम॑भागा॒ यद् यथ् स्तोम॑भागा उप॒दधा॑ति । \newline
42. स्तोम॑भागा उप॒दधा᳚ त्युप॒दधा॑ति॒ स्तोम॑भागाः॒ स्तोम॑भागा उप॒दधा॑ति य॒ज्ञ्स्य॑ य॒ज्ञ् स्यो॑प॒दधा॑ति॒ स्तोम॑भागाः॒ स्तोम॑भागा उप॒दधा॑ति य॒ज्ञ्स्य॑ । \newline
43. स्तोम॑भागा॒ इति॒ स्तोम॑ - भा॒गाः॒ । \newline
44. उ॒प॒दधा॑ति य॒ज्ञ्स्य॑ य॒ज्ञ् स्यो॑प॒दधा᳚ त्युप॒दधा॑ति य॒ज्ञ्स्य॒ प्रति॑ष्ठित्यै॒ प्रति॑ष्ठित्यै य॒ज्ञ्स्यो॑ प॒दधा᳚ त्युप॒दधा॑ति य॒ज्ञ्स्य॒ प्रति॑ष्ठित्यै । \newline
45. उ॒प॒दधा॒तीत्यु॑प - दधा॑ति । \newline
46. य॒ज्ञ्स्य॒ प्रति॑ष्ठित्यै॒ प्रति॑ष्ठित्यै य॒ज्ञ्स्य॑ य॒ज्ञ्स्य॒ प्रति॑ष्ठित्यै स॒प्तस॑प्त स॒प्तस॑प्त॒ प्रति॑ष्ठित्यै य॒ज्ञ्स्य॑ य॒ज्ञ्स्य॒ प्रति॑ष्ठित्यै स॒प्तस॑प्त । \newline
47. प्रति॑ष्ठित्यै स॒प्तस॑प्त स॒प्तस॑प्त॒ प्रति॑ष्ठित्यै॒ प्रति॑ष्ठित्यै स॒प्तस॒प्तोपोप॑ स॒प्तस॑प्त॒ प्रति॑ष्ठित्यै॒ प्रति॑ष्ठित्यै स॒प्तस॒प्तोप॑ । \newline
48. प्रति॑ष्ठित्या॒ इति॒ प्रति॑ - स्थि॒त्यै॒ । \newline
49. स॒प्तस॒प्तोपोप॑ स॒प्तस॑प्त स॒प्तस॒प्तोप॑ दधाति दधा॒ त्युप॑ स॒प्तस॑प्त स॒प्तस॒प्तोप॑ दधाति । \newline
50. स॒प्तस॒प्तेति॑ स॒प्त - स॒प्त॒ । \newline
51. उप॑ दधाति दधा॒ त्युपोप॑ दधाति सवीर्य॒त्वाय॑ सवीर्य॒त्वाय॑ दधा॒ त्युपोप॑ दधाति सवीर्य॒त्वाय॑ । \newline
52. द॒धा॒ति॒ स॒वी॒र्य॒त्वाय॑ सवीर्य॒त्वाय॑ दधाति दधाति सवीर्य॒त्वाय॑ ति॒स्र स्ति॒स्रः स॑वीर्य॒त्वाय॑ दधाति दधाति सवीर्य॒त्वाय॑ ति॒स्रः । \newline
53. स॒वी॒र्य॒त्वाय॑ ति॒स्र स्ति॒स्रः स॑वीर्य॒त्वाय॑ सवीर्य॒त्वाय॑ ति॒स्रो मद्ध्ये॒ मद्ध्ये॑ ति॒स्रः स॑वीर्य॒त्वाय॑ सवीर्य॒त्वाय॑ ति॒स्रो मद्ध्ये᳚ । \newline
54. स॒वी॒र्य॒त्वायेति॑ सवीर्य - त्वाय॑ । \newline
55. ति॒स्रो मद्ध्ये॒ मद्ध्ये॑ ति॒स्र स्ति॒स्रो मद्ध्ये॒ प्रति॑ष्ठित्यै॒ प्रति॑ष्ठित्यै॒ मद्ध्ये॑ ति॒स्र स्ति॒स्रो मद्ध्ये॒ प्रति॑ष्ठित्यै । \newline
56. मद्ध्ये॒ प्रति॑ष्ठित्यै॒ प्रति॑ष्ठित्यै॒ मद्ध्ये॒ मद्ध्ये॒ प्रति॑ष्ठित्यै । \newline
57. प्रति॑ष्ठित्या॒ इति॒ प्रति॑ - स्थि॒त्यै॒ । \newline
\pagebreak
\markright{ TS 5.3.6.1  \hfill https://www.vedavms.in \hfill}

\section{ TS 5.3.6.1 }

\textbf{TS 5.3.6.1 } \newline
\textbf{Samhita Paata} \newline

र॒श्मिरित्ये॒वा ऽऽदि॒त्यम॑सृजत॒ प्रेति॒रिति॒ धर्म॒मन्वि॑ति॒रिति॒ दिवꣳ॑ स॒धिंरित्य॒न्तरि॑क्षं प्रति॒धिरिति॑ पृथि॒वीं ॅवि॑ष्ट॒म्भ इति॒ वृष्टिं॑ प्र॒वेत्यह॑रनु॒वेति॒ रात्रि॑मु॒शिगिति॒ वसू᳚न् प्रके॒त इति॑ रु॒द्रान्थ् सु॑दी॒तिरित्या॑दि॒त्यानोज॒ इति॑ पि॒तॄꣳस्तन्तु॒रिति॑ प्र॒जाः पृ॑तना॒षाडिति॑ प॒शून् रे॒वदित्यो-ष॑धीरभि॒जिद॑सि यु॒क्तग्रा॒वे - [  ] \newline

\textbf{Pada Paata} \newline

र॒श्मिः । इति॑ । ए॒व । आ॒दि॒त्यम् । अ॒सृ॒ज॒त॒ । प्रेति॒रिति॒ प्र - इ॒तिः॒ । इति॑ । धर्म᳚म् । अन्वि॑ति॒रित्यनु॑ - इ॒तिः॒ । इति॑ । दिव᳚म् । स॒न्धिरिति॑ सं - धिः । इति॑ । अ॒न्तरि॑क्षम् । प्र॒ति॒धिरिति॑ प्रति - धिः । इति॑ । पृ॒थि॒वीम् । वि॒ष्ट॒भं इति॑ वि - स्त॒भंः । इति॑ । वृष्टि᳚म् । प्र॒वेति॑ प्र - वा । इति॑ । अहः॑ । अ॒नु॒वेत्य॑नु-वा । इति॑ । रात्रि᳚म् । उ॒शिक् । इति॑ । वसून्॑ । प्र॒के॒त इति॑ प्र - के॒तः । इति॑ । रु॒द्रान् । सु॒दी॒तिरिति॑ सु-दी॒तिः । इति॑ । आ॒दि॒त्यान् । ओजः॑ । इति॑ । पि॒तॄन् । तन्तुः॑ । इति॑ । प्र॒जा इति॑ प्र - जाः । पृ॒त॒ना॒षाट् । इति॑ । प॒शून् । रे॒वत् । इति॑ । ओष॑धीः । अ॒भि॒जिदित्य॑भि - जित् । अ॒सि॒ । यु॒क्तग्रा॒वेति॑ यु॒क्त - ग्रा॒वा॒ ।  \newline


\textbf{Krama Paata} \newline

र॒श्मिरिति॑ । इत्ये॒व । ए॒वादि॒त्यम् । आ॒दि॒त्यम॑सृजत । अ॒सृ॒ज॒त॒ प्रेतिः॑ । प्रेति॒रिति॑ । प्रेति॒रिति॒ प्र - इ॒तिः॒ । इति॒ धर्म᳚म् । धर्म॒मन्वि॑तिः । अन्वि॑ति॒रिति॑ । अन्वि॑ति॒रित्यनु॑ - इ॒तिः॒ । इति॒ दिव᳚म् । दिवꣳ॑ स॒न्धिः । स॒न्धिरिति॑ । स॒न्धिरिति॑ सम् - धिः । इत्य॒न्तरि॑क्षम् । अ॒न्तरि॑क्षम् प्रति॒धिः । प्र॒ति॒धिरिति॑ । प्र॒ति॒धिरिति॑ प्रति - धिः । इति॑ पृथि॒वीम् । पृ॒थि॒वीम् ॅवि॑ष्ट॒म्भः । वि॒ष्ट॒म्भ इति॑ । वि॒ष्ट॒म्भ इति॑ वि - स्त॒म्भः । इति॒ वृष्टि᳚म् । वृष्टि॑म् प्र॒वा । प्र॒वेति॑ । प्र॒वेति॑ प्र - वा । इत्यहः॑ । अह॑रनु॒वा । अ॒नु॒वेति॑ । अ॒नु॒वेत्य॑नु - वा । इति॒ रात्रि᳚म् । रात्रि॑मु॒शिक् । उ॒शिगिति॑ । इति॒ वसून्॑ । वसू᳚न् प्रके॒तः । प्र॒के॒त इति॑ । प्र॒के॒त इति॑ प्र - के॒तः । इति॑ रु॒द्रान् । रु॒द्रान्थ् सु॑दी॒तिः । सु॒दी॒तिरिति॑ । सु॒दी॒तिरिति॑ सु - दी॒तिः । इत्या॑दि॒त्यान् । आ॒दि॒त्यानोजः॑ । ओज॒ इति॑ । इति॑ पि॒तॄन् । पि॒तॄꣳस्तन्तुः॑ । तन्तु॒रिति॑ । इति॑ प्र॒जाः । प्र॒जाः पृ॑तना॒षाट् । प्र॒जा इति॑ प्र - जाः । पृ॒त॒ना॒षाडिति॑ । इति॑ प॒शून् । प॒शून् रे॒वत् । रे॒वदिति॑ । इत्योष॑धीः । ओष॑धीरभि॒जित् । अ॒भि॒जिद॑सि । अ॒भि॒जिदित्य॑भि - जित् । अ॒सि॒ यु॒क्तग्रा॑वा । यु॒क्तग्रा॒वेन्द्रा॑य । यु॒क्तग्रा॒वेति॑ यु॒क्त - ग्रा॒वा॒ \newline

\textbf{Jatai Paata} \newline

1. र॒श्मि रितीति॑ र॒श्मी र॒श्मि रिति॑ । \newline
2. इत्ये॒वैवे तीत्ये॒व । \newline
3. ए॒वादि॒त्य मा॑दि॒त्य मे॒वै वादि॒त्यम् । \newline
4. आ॒दि॒त्य म॑सृजता सृजतादि॒त्य मा॑दि॒त्य म॑सृजत । \newline
5. अ॒सृ॒ज॒त॒ प्रेतिः॒ प्रेति॑ रसृजता सृजत॒ प्रेतिः॑ । \newline
6. प्रेति॒ रितीति॒ प्रेतिः॒ प्रेति॒ रिति॑ । \newline
7. प्रेति॒रिति॒ प्र - इ॒तिः॒ । \newline
8. इति॒ धर्म॒म् धर्म॒ मितीति॒ धर्म᳚म् । \newline
9. धर्म॒ मन्वि॑ति॒ रन्वि॑ति॒र् धर्म॒म् धर्म॒ मन्वि॑तिः । \newline
10. अन्वि॑ति॒रिती त्यन्वि॑ति॒ रन्वि॑ति॒ रिति॑ । \newline
11. अन्वि॑ति॒रित्यनु॑ - इ॒तिः॒ । \newline
12. इति॒ दिव॒म् दिव॒ मितीति॒ दिव᳚म् । \newline
13. दिवꣳ॑ स॒न्धिः स॒न्धिर् दिव॒म् दिवꣳ॑ स॒न्धिः । \newline
14. स॒न्धि रितीति॑ स॒न्धिः स॒न्धि रिति॑ । \newline
15. स॒न्धिरिति॑ सं - धिः । \newline
16. इत्य॒न्तरि॑क्ष म॒न्तरि॑क्ष॒ मिती त्य॒न्तरि॑क्षम् । \newline
17. अ॒न्तरि॑क्षम् प्रति॒धिः प्र॑ति॒धि र॒न्तरि॑क्ष म॒न्तरि॑क्षम् प्रति॒धिः । \newline
18. प्र॒ति॒धि रितीति॑ प्रति॒धिः प्र॑ति॒धि रिति॑ । \newline
19. प्र॒ति॒धिरिति॑ प्रति - धिः । \newline
20. इति॑ पृथि॒वीम् पृ॑थि॒वी मितीति॑ पृथि॒वीम् । \newline
21. पृ॒थि॒वीं ॅवि॑ष्टं॒भो वि॑ष्टं॒भः पृ॑थि॒वीम् पृ॑थि॒वीं ॅवि॑ष्टं॒भः । \newline
22. वि॒ष्टं॒भ इतीति॑ विष्टं॒भो वि॑ष्टं॒भ इति॑ । \newline
23. वि॒ष्टं॒भ इति॑ वि - स्तं॒भः । \newline
24. इति॒ वृष्टिं॒ ॅवृष्टि॒ मितीति॒ वृष्टि᳚म् । \newline
25. वृष्टि॑म् प्र॒वा प्र॒वा वृष्टिं॒ ॅवृष्टि॑म् प्र॒वा । \newline
26. प्र॒वे तीति॑ प्र॒वा प्र॒वेति॑ । \newline
27. प्र॒वेति॑ प्र - वा । \newline
28. इत्यह॒ रह॒ रिती त्यहः॑ । \newline
29. अह॑ रनु॒वा ऽनु॒वा ऽह॒ रह॑ रनु॒वा । \newline
30. अ॒नु॒वेती त्य॑नु॒वा ऽनु॒वेति॑ । \newline
31. अ॒नु॒वेत्य॑नु - वा । \newline
32. इति॒ रात्रिꣳ॒॒ रात्रि॒ मितीति॒ रात्रि᳚म् । \newline
33. रात्रि॑ मु॒शि गु॒शिग् रात्रिꣳ॒॒ रात्रि॑ मु॒शिक् । \newline
34. उ॒शि गिती त्यु॒शि गु॒शि गिति॑ । \newline
35. इति॒ वसू॒न्॒. वसू॒ नितीति॒ वसून्॑ । \newline
36. वसू᳚न् प्रके॒तः प्र॑के॒तो वसू॒न्॒. वसू᳚न् प्रके॒तः । \newline
37. प्र॒के॒त इतीति॑ प्रके॒तः प्र॑के॒त इति॑ । \newline
38. प्र॒के॒त इति॑ प्र - के॒तः । \newline
39. इति॑ रु॒द्रान् रु॒द्रा नितीति॑ रु॒द्रान् । \newline
40. रु॒द्रान् थ्सु॑दी॒तिः सु॑दी॒ती रु॒द्रान् रु॒द्रान् थ्सु॑दी॒तिः । \newline
41. सु॒दी॒ति रितीति॑ सुदी॒तिः सु॑दी॒ति रिति॑ । \newline
42. सु॒दी॒तिरिति॑ सु - दी॒तिः । \newline
43. इत्या॑दि॒त्या ना॑दि॒त्या निती त्या॑दि॒त्यान् । \newline
44. आ॒दि॒त्या नोज॒ ओज॑ आदि॒त्या ना॑दि॒त्या नोजः॑ । \newline
45. ओज॒ इतीत्योज॒ ओज॒ इति॑ । \newline
46. इति॑ पि॒तॄन् पि॒तॄ नितीति॑ पि॒तॄन् । \newline
47. पि॒तॄन् तन्तु॒ स्तन्तुः॑ पि॒तॄन् पि॒तॄन् तन्तुः॑ । \newline
48. तन्तु॒ रितीति॒ तन्तु॒ स्तन्तु॒ रिति॑ । \newline
49. इति॑ प्र॒जाः प्र॒जा इतीति॑ प्र॒जाः । \newline
50. प्र॒जाः पृ॑तना॒षाट् पृ॑तना॒षाट् प्र॒जाः प्र॒जाः पृ॑तना॒षाट् । \newline
51. प्र॒जा इति॑ प्र - जाः । \newline
52. पृ॒त॒ना॒षा डितीति॑ पृतना॒षाट् पृ॑तना॒षा डिति॑ । \newline
53. इति॑ प॒शून् प॒शू नितीति॑ प॒शून् । \newline
54. प॒शून् रे॒वद् रे॒वत् प॒शून् प॒शून् रे॒वत् । \newline
55. रे॒व दितीति॑ रे॒वद् रे॒व दिति॑ । \newline
56. इत्योष॑धी॒ रोष॑धी॒ रिती त्योष॑धीः । \newline
57. ओष॑धी रभि॒जि द॑भि॒जि दोष॑धी॒ रोष॑धी रभि॒जित् । \newline
58. अ॒भि॒जि द॑स्यस्य भि॒जि द॑भि॒जि द॑सि । \newline
59. अ॒भि॒जिदित्य॑भि - जित् । \newline
60. अ॒सि॒ यु॒क्तग्रा॑वा यु॒क्तग्रा॑वा ऽस्यसि यु॒क्तग्रा॑वा । \newline
61. यु॒क्तग्रा॒ वेन्द्रा॒ येन्द्रा॑य यु॒क्तग्रा॑वा यु॒क्तग्रा॒ वेन्द्रा॑य । \newline
62. यु॒क्तग्रा॒वेति॑ यु॒क्त - ग्रा॒वा॒ । \newline

\textbf{Ghana Paata } \newline

1. र॒श्मि रितीति॑ र॒श्मी र॒श्मि रित्ये॒वैवेति॑ र॒श्मी र॒श्मि रित्ये॒व । \newline
2. इत्ये॒ वैवेती त्ये॒वादि॒त्य मा॑दि॒त्य मे॒वेती त्ये॒वादि॒त्यम् । \newline
3. ए॒वादि॒त्य मा॑दि॒त्य मे॒वै वादि॒त्य म॑सृजता सृजतादि॒त्य मे॒वै वादि॒त्य म॑सृजत । \newline
4. आ॒दि॒त्य म॑सृजता सृजतादि॒त्य मा॑दि॒त्य म॑सृजत॒ प्रेतिः॒ प्रेति॑ रसृजता दि॒त्य मा॑दि॒त्य म॑सृजत॒ प्रेतिः॑ । \newline
5. अ॒सृ॒ज॒त॒ प्रेतिः॒ प्रेति॑ रसृजता सृजत॒ प्रेति॒ रितीति॒ प्रेति॑ रसृजता सृजत॒ प्रेति॒ रिति॑ । \newline
6. प्रेति॒ रितीति॒ प्रेतिः॒ प्रेति॒ रिति॒ धर्म॒म् धर्म॒ मिति॒ प्रेतिः॒ प्रेति॒ रिति॒ धर्म᳚म् । \newline
7. प्रेति॒रिति॒ प्र - इ॒तिः॒ । \newline
8. इति॒ धर्म॒म् धर्म॒ मितीति॒ धर्म॒ मन्वि॑ति॒ रन्वि॑ति॒र् धर्म॒ मितीति॒ धर्म॒ मन्वि॑तिः । \newline
9. धर्म॒ मन्वि॑ति॒ रन्वि॑ति॒र् धर्म॒म् धर्म॒ मन्वि॑ति॒रिती त्यन्वि॑ति॒र् धर्म॒म् धर्म॒ मन्वि॑ति॒ रिति॑ । \newline
10. अन्वि॑ति॒रिती त्यन्वि॑ति॒ रन्वि॑ति॒ रिति॒ दिव॒म् दिव॒ मित्यन्वि॑ति॒ रन्वि॑ति॒ रिति॒ दिव᳚म् । \newline
11. अन्वि॑ति॒रित्यनु॑ - इ॒तिः॒ । \newline
12. इति॒ दिव॒म् दिव॒ मितीति॒ दिवꣳ॑ स॒न्धिः स॒न्धिर् दिव॒ मितीति॒ दिवꣳ॑ स॒न्धिः । \newline
13. दिवꣳ॑ स॒न्धिः स॒न्धिर् दिव॒म् दिवꣳ॑ स॒न्धि रितीति॑ स॒न्धिर् दिव॒म् दिवꣳ॑ स॒न्धि रिति॑ । \newline
14. स॒न्धि रितीति॑ स॒न्धिः स॒न्धिरि त्य॒न्तरि॑क्ष म॒न्तरि॑क्ष॒ मिति॑ स॒न्धिः स॒न्धिरि त्य॒न्तरि॑क्षम् । \newline
15. स॒न्धिरिति॑ सं - धिः । \newline
16. इत्य॒न्तरि॑क्ष म॒न्तरि॑क्ष॒ मिती त्य॒न्तरि॑क्षम् प्रति॒धिः प्र॑ति॒धि र॒न्तरि॑क्ष॒ मिती त्य॒न्तरि॑क्षम् प्रति॒धिः । \newline
17. अ॒न्तरि॑क्षम् प्रति॒धिः प्र॑ति॒ धिर॒न्तरि॑क्ष म॒न्तरि॑क्षम् प्रति॒धि रितीति॑ प्रति॒धि र॒न्तरि॑क्ष म॒न्तरि॑क्षम् प्रति॒धि रिति॑ । \newline
18. प्र॒ति॒धि रितीति॑ प्रति॒धिः प्र॑ति॒धि रिति॑ पृथि॒वीम् पृ॑थि॒वी मिति॑ प्रति॒धिः प्र॑ति॒धि रिति॑ पृथि॒वीम् । \newline
19. प्र॒ति॒धिरिति॑ प्रति - धिः । \newline
20. इति॑ पृथि॒वीम् पृ॑थि॒वी मितीति॑ पृथि॒वीं ॅवि॑ष्टं॒भो वि॑ष्टं॒भः पृ॑थि॒वी मितीति॑ पृथि॒वीं ॅवि॑ष्टं॒भः । \newline
21. पृ॒थि॒वीं ॅवि॑ष्टं॒भो वि॑ष्टं॒भः पृ॑थि॒वीम् पृ॑थि॒वीं ॅवि॑ष्टं॒भ इतीति॑ विष्टं॒भः पृ॑थि॒वीम् पृ॑थि॒वीं ॅवि॑ष्टं॒भ इति॑ । \newline
22. वि॒ष्टं॒भ इतीति॑ विष्टं॒भो वि॑ष्टं॒भ इति॒ वृष्टिं॒ ॅवृष्टि॒ मिति॑ विष्टं॒भो वि॑ष्टं॒भ इति॒ वृष्टि᳚म् । \newline
23. वि॒ष्टं॒भ इति॑ वि - स्तं॒भः । \newline
24. इति॒ वृष्टिं॒ ॅवृष्टि॒ मितीति॒ वृष्टि॑म् प्र॒वा प्र॒वा वृष्टि॒ मितीति॒ वृष्टि॑म् प्र॒वा । \newline
25. वृष्टि॑म् प्र॒वा प्र॒वा वृष्टिं॒ ॅवृष्टि॑म् प्र॒वे तीति॑ प्र॒वा वृष्टिं॒ ॅवृष्टि॑म् प्र॒वेति॑ । \newline
26. प्र॒वे तीति॑ प्र॒वा प्र॒वे त्यह॒ रह॒ रिति॑ प्र॒वा प्र॒वे त्यहः॑ । \newline
27. प्र॒वेति॑ प्र - वा । \newline
28. इत्यह॒ रह॒ रिती त्यह॑ रनु॒वा ऽनु॒वा ऽह॒ रिती त्यह॑ रनु॒वा । \newline
29. अह॑ रनु॒वा ऽनु॒वा ऽह॒ रह॑ रनु॒वेती त्य॑नु॒वा ऽह॒ रह॑ रनु॒वेति॑ । \newline
30. अ॒नु॒वेती त्य॑नु॒वा ऽनु॒वेति॒ रात्रिꣳ॒॒ रात्रि॒ मित्य॑नु॒वा ऽनु॒वेति॒ रात्रि᳚म् । \newline
31. अ॒नु॒वेत्य॑नु - वा । \newline
32. इति॒ रात्रिꣳ॒॒ रात्रि॒ मितीति॒ रात्रि॑ मु॒शि गु॒शिग् रात्रि॒ मितीति॒ रात्रि॑ मु॒शिक् । \newline
33. रात्रि॑ मु॒शि गु॒शिग् रात्रिꣳ॒॒ रात्रि॑ मु॒शि गिती त्यु॒शिग् रात्रिꣳ॒॒ रात्रि॑ मु॒शि गिति॑ । \newline
34. उ॒शिगिती त्यु॒शि गु॒शि गिति॒ वसू॒न्॒. वसू॒ नित्यु॒शि गु॒शि गिति॒ वसून्॑ । \newline
35. इति॒ वसू॒न्॒. वसू॒ नितीति॒ वसू᳚न् प्रके॒तः प्र॑के॒तो वसू॒ नितीति॒ वसू᳚न् प्रके॒तः । \newline
36. वसू᳚न् प्रके॒तः प्र॑के॒तो वसू॒न्॒. वसू᳚न् प्रके॒त इतीति॑ प्रके॒तो वसू॒न्॒. वसू᳚न् प्रके॒त इति॑ । \newline
37. प्र॒के॒त इतीति॑ प्रके॒तः प्र॑के॒त इति॑ रु॒द्रान् रु॒द्रा निति॑ प्रके॒तः प्र॑के॒त इति॑ रु॒द्रान् । \newline
38. प्र॒के॒त इति॑ प्र - के॒तः । \newline
39. इति॑ रु॒द्रान् रु॒द्रा नितीति॑ रु॒द्रान् थ्सु॑दी॒तिः सु॑दी॒ती रु॒द्रा नितीति॑ रु॒द्रान् थ्सु॑दी॒तिः । \newline
40. रु॒द्रान् थ्सु॑दी॒तिः सु॑दी॒ती रु॒द्रान् रु॒द्रान् थ्सु॑दी॒ति रितीति॑ सुदी॒ती रु॒द्रान् रु॒द्रान् थ्सु॑दी॒ति रिति॑ । \newline
41. सु॒दी॒ति रितीति॑ सुदी॒तिः सु॑दी॒ति रित्या॑दि॒त्या ना॑दि॒त्या निति॑ सुदी॒तिः सु॑दी॒ति रित्या॑दि॒त्यान् । \newline
42. सु॒दी॒तिरिति॑ सु - दी॒तिः । \newline
43. इत्या॑दि॒त्या ना॑दि॒त्या नितीत्या॑दि॒त्या नोज॒ ओज॑ आदि॒त्या नितीत्या॑दि॒त्या नोजः॑ । \newline
44. आ॒दि॒त्या नोज॒ ओज॑ आदि॒त्या ना॑दि॒त्या नोज॒ इतीत्योज॑ आदि॒त्या ना॑दि॒त्या नोज॒ इति॑ । \newline
45. ओज॒ इतीत्योज॒ ओज॒ इति॑ पि॒तॄन् पि॒तॄ नित्योज॒ ओज॒ इति॑ पि॒तॄन् । \newline
46. इति॑ पि॒तॄन् पि॒तॄ नितीति॑ पि॒तॄन् तन्तु॒ स्तन्तुः॑ पि॒तॄ नितीति॑ पि॒तॄन् तन्तुः॑ । \newline
47. पि॒तॄन् तन्तु॒ स्तन्तुः॑ पि॒तॄन् पि॒तॄन् तन्तु॒ रितीति॒ तन्तुः॑ पि॒तॄन् पि॒तॄन् तन्तु॒ रिति॑ । \newline
48. तन्तु॒ रितीति॒ तन्तु॒ स्तन्तु॒ रिति॑ प्र॒जाः प्र॒जा इति॒ तन्तु॒ स्तन्तु॒ रिति॑ प्र॒जाः । \newline
49. इति॑ प्र॒जाः प्र॒जा इतीति॑ प्र॒जाः पृ॑तना॒षाट् पृ॑तना॒षाट् प्र॒जा इतीति॑ प्र॒जाः पृ॑तना॒षाट् । \newline
50. प्र॒जाः पृ॑तना॒षाट् पृ॑तना॒षाट् प्र॒जाः प्र॒जाः पृ॑तना॒षा डितीति॑ पृतना॒षाट् प्र॒जाः प्र॒जाः पृ॑तना॒षा डिति॑ । \newline
51. प्र॒जा इति॑ प्र - जाः । \newline
52. पृ॒त॒ना॒षा डितीति॑ पृतना॒षाट् पृ॑तना॒षा डिति॑ प॒शून् प॒शू निति॑ पृतना॒षाट् पृ॑तना॒षा डिति॑ प॒शून् । \newline
53. इति॑ प॒शून् प॒शू नितीति॑ प॒शून् रे॒वद् रे॒वत् प॒शू नितीति॑ प॒शून् रे॒वत् । \newline
54. प॒शून् रे॒वद् रे॒वत् प॒शून् प॒शून् रे॒व दितीति॑ रे॒वत् प॒शून् प॒शून् रे॒व दिति॑ । \newline
55. रे॒व दितीति॑ रे॒वद् रे॒वदि त्योष॑धी॒ रोष॑धी॒ रिति॑ रे॒वद् रे॒वदि त्योष॑धीः । \newline
56. इत्योष॑धी॒ रोष॑धी॒ रिती त्योष॑धी रभि॒जि द॑भि॒जि दोष॑धी॒ रिती त्योष॑धी रभि॒जित् । \newline
57. ओष॑धी रभि॒जि द॑भि॒जि दोष॑धी॒ रोष॑धी रभि॒जि द॑स्य स्यभि॒जि दोष॑धी॒ रोष॑धी रभि॒जि द॑सि । \newline
58. अ॒भि॒जि द॑स्य स्यभि॒जि द॑भि॒जि द॑सि यु॒क्तग्रा॑वा यु॒क्तग्रा॑वा ऽस्यभि॒जि द॑भि॒जि द॑सि यु॒क्तग्रा॑वा । \newline
59. अ॒भि॒जिदित्य॑भि - जित् । \newline
60. अ॒सि॒ यु॒क्तग्रा॑वा यु॒क्तग्रा॑वा ऽस्यसि यु॒क्तग्रा॒ वेन्द्रा॒ येन्द्रा॑य यु॒क्तग्रा॑वा ऽस्यसि यु॒क्तग्रा॒ वेन्द्रा॑य । \newline
61. यु॒क्तग्रा॒ वेन्द्रा॒ येन्द्रा॑य यु॒क्तग्रा॑वा यु॒क्तग्रा॒ वेन्द्रा॑य त्वा॒ त्वेन्द्रा॑य यु॒क्तग्रा॑वा यु॒क्तग्रा॒ वेन्द्रा॑य त्वा । \newline
62. यु॒क्तग्रा॒वेति॑ यु॒क्त - ग्रा॒वा॒ । \newline
\pagebreak
\markright{ TS 5.3.6.2  \hfill https://www.vedavms.in \hfill}

\section{ TS 5.3.6.2 }

\textbf{TS 5.3.6.2 } \newline
\textbf{Samhita Paata} \newline

-न्द्रा॑य॒ त्वेन्द्रं॑ जि॒न्वेत्ये॒व द॑क्षिण॒तो वज्रं॒ पर्यौ॑हद॒भिजि॑त्यै॒ ताः प्र॒जा अप॑प्राणा असृजत॒ तास्वधि॑पतिर॒सीत्ये॒व प्रा॒णम॑दधा-द्य॒न्तेत्य॑पा॒नꣳ सꣳ॒॒सर्प॒ इति॒ चक्षु॑र्वयो॒धा इति॒ श्रोत्रं॒ ताः प्र॒जाः प्रा॑ण॒तीर॑पान॒तीः पश्य॑न्तीः शृण्व॒तीर्न मि॑थु॒नी अ॑भव॒न् तासु॑ त्रि॒वृद॒सीत्ये॒व मि॑थु॒नम॑दधा॒त् ताः प्र॒जा मि॑थु॒नी - [  ] \newline

\textbf{Pada Paata} \newline

इन्द्रा॑य । त्वा॒ । इन्द्र᳚म् । जि॒न्व॒ । इति॑ । ए॒व । द॒क्षि॒ण॒तः । वज्र᳚म् । परीति॑ । औ॒ह॒त् । अ॒भिजि॑त्या॒ इत्य॒भि - जि॒त्यै॒ । ताः । प्र॒जा इति॑ प्र - जाः । अप॑प्राणा॒ इत्यप॑ - प्रा॒णाः॒ । अ॒सृ॒ज॒त॒ । तासु॑ । अधि॑पति॒रित्यधि॑-प॒तिः॒ । अ॒सि॒ । इति॑ । ए॒व । प्रा॒णमिति॑ प्र-अ॒नम् । अ॒द॒धा॒त् । य॒न्ता । इति॑ । अ॒पा॒नमित्य॑प - अ॒नम् । सꣳ॒॒सर्प॒ इति॑ सं - सर्पः॑ । इति॑ । चक्षुः॑ । व॒यो॒धा इति॑ वयः - धाः । इति॑ । श्रोत्र᳚म् । ताः । प्र॒जा इति॑ प्र - जाः । प्रा॒ण॒तीरिति॑ प्र - अ॒न॒तीः । अ॒पा॒न॒तीरित्य॑प - अ॒न॒तीः । पश्य॑न्तीः । शृ॒ण्व॒तीः । न । मि॒थु॒नी । अ॒भ॒व॒न्न् । तासु॑ । त्रि॒वृदिति॑ त्रि - वृत् । अ॒सि॒ । इति॑ । ए॒व । मि॒थु॒नम् । अ॒द॒धा॒त् । ताः । प्र॒जा इति॑ प्र-जाः । मि॒थु॒नी ।  \newline


\textbf{Krama Paata} \newline

इन्द्रा॑य त्वा । त्वेन्द्र᳚म् । इन्द्र॑म् जिन्व । जि॒न्वेति॑ । इत्ये॒व । ए॒व द॑क्षिण॒तः । द॒क्षि॒ण॒तो वज्र᳚म् । वज्र॒म् परि॑ । पर्यौ॑हत् । औ॒ह॒द॒भिजि॑त्यै । अ॒भिजि॑त्यै॒ ताः । अ॒भिजि॑त्या॒ इत्य॒भि - जि॒त्यै॒ । ताः प्र॒जाः । प्र॒जा अप॑प्राणाः । प्र॒जा इति॑ प्र - जाः । अप॑प्राणा असृजत । अप॑प्राणा॒ इत्यप॑ - प्रा॒णाः॒ । अ॒सृ॒ज॒त॒ तासु॑ । तास्वधि॑पतिः । अधि॑पतिरसि । अधि॑पति॒रित्यधि॑ - प॒तिः॒ । अ॒सीति॑ । इत्ये॒व । ए॒व प्रा॒णम् । प्रा॒णम॑दधात् । प्रा॒णमिति॑ प्र - अ॒नम् । अ॒द॒धा॒द् य॒न्ता । य॒न्तेति॑ । इत्य॑पा॒नम् । अ॒पा॒नꣳ सꣳ॒॒सर्पः॑ । अ॒पा॒नमित्य॑प - अ॒नम् । सꣳ॒॒सर्प॒ इति॑ । सꣳ॒॒सर्प॒ इति॑ सम् - सर्पः॑ । इति॒ चक्षुः॑ । चक्षु॑र् वयो॒धाः । व॒यो॒धा इति॑ । व॒यो॒धा इति॑ वयः - धाः । इति॒ श्रोत्र᳚म् । श्रोत्र॒म् ताः । ताः प्र॒जाः । प्र॒जाः प्रा॑ण॒तीः । प्र॒जा इति॑ प्र - जाः । प्रा॒ण॒तीर॑पान॒तीः । प्रा॒ण॒तीरिति॑ प्र - अ॒न॒तीः । अ॒पा॒न॒तीः पश्य॑न्तीः । अ॒पा॒न॒तीरित्य॑प - अ॒न॒तीः । पश्य॑न्तीः शृण्व॒तीः । शृ॒ण्व॒तीर् न । न मि॑थु॒नी । मि॒थु॒नी अ॑भवन्न् । अ॒भ॒व॒न् तासु॑ । तासु॑ त्रि॒वृत् । त्रि॒वृद॑सि । त्रि॒वृदिति॑ त्रि - वृत् । अ॒सीति॑ । इत्ये॒व । ए॒व मि॑थु॒नम् । मि॒थु॒नम॑दधात् । अ॒द॒धा॒त् ताः । ताः प्र॒जाः । प्र॒जा मि॑थु॒नी । प्र॒जा इति॑ प्र - जाः । मि॒थु॒नी भव॑न्तीः \newline

\textbf{Jatai Paata} \newline

1. इन्द्रा॑य त्वा॒ त्वेन्द्रा॒ येन्द्रा॑य त्वा । \newline
2. त्वेन्द्र॒ मिन्द्र॑म् त्वा॒ त्वेन्द्र᳚म् । \newline
3. इन्द्र॑म् जिन्व जि॒न्वेन्द्र॒ मिन्द्र॑म् जिन्व । \newline
4. जि॒न्वे तीति॑ जिन्व जि॒न्वे ति॑ । \newline
5. इत्ये॒वैवे तीत्ये॒व । \newline
6. ए॒व द॑क्षिण॒तो द॑क्षिण॒त ए॒वैव द॑क्षिण॒तः । \newline
7. द॒क्षि॒ण॒तो वज्रं॒ ॅवज्र॑म् दक्षिण॒तो द॑क्षिण॒तो वज्र᳚म् । \newline
8. वज्र॒म् परि॒ परि॒ वज्रं॒ ॅवज्र॒म् परि॑ । \newline
9. पर्यौ॑ह दौह॒त् परि॒ पर्यौ॑हत् । \newline
10. औ॒ह॒ द॒भिजि॑त्या अ॒भिजि॑त्या औह दौह द॒भिजि॑त्यै । \newline
11. अ॒भिजि॑त्यै॒ ता स्ता अ॒भिजि॑त्या अ॒भिजि॑त्यै॒ ताः । \newline
12. अ॒भिजि॑त्या॒ इत्य॒भि - जि॒त्यै॒ । \newline
13. ताः प्र॒जाः प्र॒जा स्ता स्ताः प्र॒जाः । \newline
14. प्र॒जा अप॑प्राणा॒ अप॑प्राणाः प्र॒जाः प्र॒जा अप॑प्राणाः । \newline
15. प्र॒जा इति॑ प्र - जाः । \newline
16. अप॑प्राणा असृजता सृज॒ता प॑प्राणा॒ अप॑प्राणा असृजत । \newline
17. अप॑प्राणा॒ इत्यप॑ - प्रा॒णाः॒ । \newline
18. अ॒सृ॒ज॒त॒ तासु॒ तास्व॑ सृजता सृजत॒ तासु॑ । \newline
19. तास्व धि॑पति॒ रधि॑पति॒ स्तासु॒ तास्व धि॑पतिः । \newline
20. अधि॑पति रस्य॒स्य धि॑पति॒ रधि॑पति रसि । \newline
21. अधि॑पति॒रित्यधि॑ - प॒तिः॒ । \newline
22. अ॒सीती त्य॑स्य॒ सीति॑ । \newline
23. इत्ये॒ वैवेती त्ये॒व । \newline
24. ए॒व प्रा॒णम् प्रा॒ण मे॒वैव प्रा॒णम् । \newline
25. प्रा॒ण म॑दधा ददधात् प्रा॒णम् प्रा॒ण म॑दधात् । \newline
26. प्रा॒णमिति॑ प्र - अ॒नम् । \newline
27. अ॒द॒धा॒द् य॒न्ता य॒न्ता ऽद॑धा ददधाद् य॒न्ता । \newline
28. य॒न्तेतीति॑ य॒न्ता य॒न्तेति॑ । \newline
29. इत्य॑पा॒न म॑पा॒न मिती त्य॑पा॒नम् । \newline
30. अ॒पा॒नꣳ सꣳ॒॒सर्पः॑ सꣳ॒॒सर्पो॑ ऽपा॒न म॑पा॒नꣳ सꣳ॒॒सर्पः॑ । \newline
31. अ॒पा॒नमित्य॑प - अ॒नम् । \newline
32. सꣳ॒॒सर्प॒ इतीति॑ सꣳ॒॒सर्पः॑ सꣳ॒॒सर्प॒ इति॑ । \newline
33. सꣳ॒॒सर्प॒ इति॑ सं - सर्पः॑ । \newline
34. इति॒ चक्षु॒ श्चक्षु॒ रितीति॒ चक्षुः॑ । \newline
35. चक्षु॑र् वयो॒धा व॑यो॒धा श्चक्षु॒ श्चक्षु॑र् वयो॒धाः । \newline
36. व॒यो॒धा इतीति॑ वयो॒धा व॑यो॒धा इति॑ । \newline
37. व॒यो॒धा इति॑ वयः - धाः । \newline
38. इति॒ श्रोत्रꣳ॒॒ श्रोत्र॒ मितीति॒ श्रोत्र᳚म् । \newline
39. श्रोत्र॒म् ता स्ताः श्रोत्रꣳ॒॒ श्रोत्र॒म् ताः । \newline
40. ताः प्र॒जाः प्र॒जा स्ता स्ताः प्र॒जाः । \newline
41. प्र॒जाः प्रा॑ण॒तीः प्रा॑ण॒तीः प्र॒जाः प्र॒जाः प्रा॑ण॒तीः । \newline
42. प्र॒जा इति॑ प्र - जाः । \newline
43. प्रा॒ण॒ती र॑पान॒ती र॑पान॒तीः प्रा॑ण॒तीः प्रा॑ण॒ती र॑पान॒तीः । \newline
44. प्रा॒ण॒तीरिति॑ प्र - अ॒न॒तीः । \newline
45. अ॒पा॒न॒तीः पश्य॑न्तीः॒ पश्य॑न्ती रपान॒ती र॑पान॒तीः पश्य॑न्तीः । \newline
46. अ॒पा॒न॒तीरित्य॑प - अ॒न॒तीः । \newline
47. पश्य॑न्तीः शृण्व॒तीः शृ॑ण्व॒तीः पश्य॑न्तीः॒ पश्य॑न्तीः शृण्व॒तीः । \newline
48. शृ॒ण्व॒तीर् न न शृ॑ण्व॒तीः शृ॑ण्व॒तीर् न । \newline
49. न मि॑थु॒नी मि॑थु॒नी न न मि॑थु॒नी । \newline
50. मि॒थु॒नी अ॑भवन् नभवन् मिथु॒नी मि॑थु॒नी अ॑भवन्न् । \newline
51. अ॒भ॒व॒न् तासु॒ तास्व॑भवन् नभव॒न् तासु॑ । \newline
52. तासु॑ त्रि॒वृत् त्रि॒वृत् तासु॒ तासु॑ त्रि॒वृत् । \newline
53. त्रि॒वृ द॑स्यसि त्रि॒वृत् त्रि॒वृ द॑सि । \newline
54. त्रि॒वृदिति॑ त्रि - वृत् । \newline
55. अ॒सीत्ती य॑स्य॒ सीति॑ । \newline
56. इत्ये॒वैवेती त्ये॒व । \newline
57. ए॒व मि॑थु॒नम् मि॑थु॒न मे॒वैव मि॑थु॒नम् । \newline
58. मि॒थु॒न म॑दधा ददधान् मिथु॒नम् मि॑थु॒न म॑दधात् । \newline
59. अ॒द॒धा॒त् ता स्ता अ॑दधा ददधा॒त् ताः । \newline
60. ताः प्र॒जाः प्र॒जा स्ता स्ताः प्र॒जाः । \newline
61. प्र॒जा मि॑थु॒नी मि॑थु॒नी प्र॒जाः प्र॒जा मि॑थु॒नी । \newline
62. प्र॒जा इति॑ प्र - जाः । \newline
63. मि॒थु॒नी भव॑न्ती॒र् भव॑न्तीर् मिथु॒नी मि॑थु॒नी भव॑न्तीः । \newline

\textbf{Ghana Paata } \newline

1. इन्द्रा॑य त्वा॒ त्वेन्द्रा॒ येन्द्रा॑य॒ त्वेन्द्र॒ मिन्द्र॒म् त्वेन्द्रा॒ येन्द्रा॑य॒ त्वेन्द्र᳚म् । \newline
2. त्वेन्द्र॒ मिन्द्र॑म् त्वा॒ त्वेन्द्र॑म् जिन्व जि॒न्वेन्द्र॑म् त्वा॒ त्वेन्द्र॑म् जिन्व । \newline
3. इन्द्र॑म् जिन्व जि॒न्वेन्द्र॒ मिन्द्र॑म् जि॒न्वेतीति॑ जि॒न्वेन्द्र॒ मिन्द्र॑म् जि॒न्वेति॑ । \newline
4. जि॒न्वेतीति॑ जिन्व जि॒न्वे त्ये॒वैवेति॑ जिन्व जि॒न्वे त्ये॒व । \newline
5. इत्ये॒वैवे तीत्ये॒व द॑क्षिण॒तो द॑क्षिण॒त ए॒वे तीत्ये॒व द॑क्षिण॒तः । \newline
6. ए॒व द॑क्षिण॒तो द॑क्षिण॒त ए॒वैव द॑क्षिण॒तो वज्रं॒ ॅवज्र॑म् दक्षिण॒त ए॒वैव द॑क्षिण॒तो वज्र᳚म् । \newline
7. द॒क्षि॒ण॒तो वज्रं॒ ॅवज्र॑म् दक्षिण॒तो द॑क्षिण॒तो वज्र॒म् परि॒ परि॒ वज्र॑म् दक्षिण॒तो द॑क्षिण॒तो वज्र॒म् परि॑ । \newline
8. वज्र॒म् परि॒ परि॒ वज्रं॒ ॅवज्र॒म् पर्यौ॑ह दौह॒त् परि॒ वज्रं॒ ॅवज्र॒म् पर्यौ॑हत् । \newline
9. पर्यौ॑ह दौह॒त् परि॒ पर्यौ॑ह द॒भिजि॑त्या अ॒भिजि॑त्या औह॒त् परि॒ पर्यौ॑ह द॒भिजि॑त्यै । \newline
10. औ॒ह॒ द॒भिजि॑त्या अ॒भिजि॑त्या औह दौह द॒भिजि॑त्यै॒ ता स्ता अ॒भिजि॑त्या औह दौह द॒भिजि॑त्यै॒ ताः । \newline
11. अ॒भिजि॑त्यै॒ ता स्ता अ॒भिजि॑त्या अ॒भिजि॑त्यै॒ ताः प्र॒जाः प्र॒जा स्ता अ॒भिजि॑त्या अ॒भिजि॑त्यै॒ ताः प्र॒जाः । \newline
12. अ॒भिजि॑त्या॒ इत्य॒भि - जि॒त्यै॒ । \newline
13. ताः प्र॒जाः प्र॒जा स्ता स्ताः प्र॒जा अप॑प्राणा॒ अप॑प्राणाः प्र॒जा स्ता स्ताः प्र॒जा अप॑प्राणाः । \newline
14. प्र॒जा अप॑प्राणा॒ अप॑प्राणाः प्र॒जाः प्र॒जा अप॑प्राणा असृजता सृज॒ता प॑प्राणाः प्र॒जाः प्र॒जा अप॑प्राणा असृजत । \newline
15. प्र॒जा इति॑ प्र - जाः । \newline
16. अप॑प्राणा असृजता सृज॒ता प॑प्राणा॒ अप॑प्राणा असृजत॒ तासु॒ तास्व॑सृज॒ता प॑प्राणा॒ अप॑प्राणा असृजत॒ तासु॑ । \newline
17. अप॑प्राणा॒ इत्यप॑ - प्रा॒णाः॒ । \newline
18. अ॒सृ॒ज॒त॒ तासु॒ तास्व॑सृजता सृजत॒ तास्वधि॑पति॒ रधि॑पति॒ स्तास्व॑सृजता सृजत॒ तास्वधि॑पतिः । \newline
19. तास्वधि॑पति॒ रधि॑पति॒ स्तासु॒ तास्वधि॑पति रस्य॒ स्यधि॑पति॒ स्तासु॒ तास्वधि॑पति रसि । \newline
20. अधि॑पति रस्य॒ स्यधि॑पति॒ रधि॑पति र॒सीती त्य॒स्यधि॑पति॒ रधि॑पति र॒सीति॑ । \newline
21. अधि॑पति॒रित्यधि॑ - प॒तिः॒ । \newline
22. अ॒सीती त्य॑स्य॒ सीत्ये॒ वैवेत्य॑स्य॒ सीत्ये॒व । \newline
23. इत्ये॒वैवे तीत्ये॒व प्रा॒णम् प्रा॒ण मे॒वे तीत्ये॒व प्रा॒णम् । \newline
24. ए॒व प्रा॒णम् प्रा॒ण मे॒वैव प्रा॒ण म॑दधा ददधात् प्रा॒ण मे॒वैव प्रा॒ण म॑दधात् । \newline
25. प्रा॒ण म॑दधा ददधात् प्रा॒णम् प्रा॒ण म॑दधाद् य॒न्ता य॒न्ता ऽद॑धात् प्रा॒णम् प्रा॒ण म॑दधाद् य॒न्ता । \newline
26. प्रा॒णमिति॑ प्र - अ॒नम् । \newline
27. अ॒द॒धा॒द् य॒न्ता य॒न्ता ऽद॑धा ददधाद् य॒न्तेतीति॑ य॒न्ता ऽद॑धा ददधाद् य॒न्तेति॑ । \newline
28. य॒न्तेतीति॑ य॒न्ता य॒न्ते त्य॑पा॒न म॑पा॒न मिति॑ य॒न्ता य॒न्ते त्य॑पा॒नम् । \newline
29. इत्य॑पा॒न म॑पा॒न मिती त्य॑पा॒नꣳ सꣳ॒॒सर्पः॑ सꣳ॒॒सर्पो॑ ऽपा॒न मिती त्य॑पा॒नꣳ सꣳ॒॒सर्पः॑ । \newline
30. अ॒पा॒नꣳ सꣳ॒॒सर्पः॑ सꣳ॒॒सर्पो॑ ऽपा॒न म॑पा॒नꣳ सꣳ॒॒सर्प॒ इतीति॑ सꣳ॒॒सर्पो॑ ऽपा॒न म॑पा॒नꣳ सꣳ॒॒सर्प॒ इति॑ । \newline
31. अ॒पा॒नमित्य॑प - अ॒नम् । \newline
32. सꣳ॒॒सर्प॒ इतीति॑ सꣳ॒॒सर्पः॑ सꣳ॒॒सर्प॒ इति॒ चक्षु॒ श्चक्षु॒ रिति॑ सꣳ॒॒सर्पः॑ सꣳ॒॒सर्प॒ इति॒ चक्षुः॑ । \newline
33. सꣳ॒॒सर्प॒ इति॑ सं - सर्पः॑ । \newline
34. इति॒ चक्षु॒ श्चक्षु॒ रितीति॒ चक्षु॑र् वयो॒धा व॑यो॒धा श्चक्षु॒ रितीति॒ चक्षु॑र् वयो॒धाः । \newline
35. चक्षु॑र् वयो॒धा व॑यो॒धा श्चक्षु॒ श्चक्षु॑र् वयो॒धा इतीति॑ वयो॒धा श्चक्षु॒ श्चक्षु॑र् वयो॒धा इति॑ । \newline
36. व॒यो॒धा इतीति॑ वयो॒धा व॑यो॒धा इति॒ श्रोत्रꣳ॒॒ श्रोत्र॒ मिति॑ वयो॒धा व॑यो॒धा इति॒ श्रोत्र᳚म् । \newline
37. व॒यो॒धा इति॑ वयः - धाः । \newline
38. इति॒ श्रोत्रꣳ॒॒ श्रोत्र॒ मितीति॒ श्रोत्र॒म् तास्ताः श्रोत्र॒ मितीति॒ श्रोत्र॒म् ताः । \newline
39. श्रोत्र॒म् ता स्ताः श्रोत्रꣳ॒॒ श्रोत्र॒म् ताः प्र॒जाः प्र॒जा स्ताः श्रोत्रꣳ॒॒ श्रोत्र॒म् ताः प्र॒जाः । \newline
40. ताः प्र॒जाः प्र॒जा स्ता स्ताः प्र॒जाः प्रा॑ण॒तीः प्रा॑ण॒तीः प्र॒जा स्ता स्ताः प्र॒जाः प्रा॑ण॒तीः । \newline
41. प्र॒जाः प्रा॑ण॒तीः प्रा॑ण॒तीः प्र॒जाः प्र॒जाः प्रा॑ण॒ती र॑पान॒ती र॑पान॒तीः प्रा॑ण॒तीः प्र॒जाः प्र॒जाः प्रा॑ण॒ती र॑पान॒तीः । \newline
42. प्र॒जा इति॑ प्र - जाः । \newline
43. प्रा॒ण॒ती र॑पान॒ती र॑पान॒तीः प्रा॑ण॒तीः प्रा॑ण॒ती र॑पान॒तीः पश्य॑न्तीः॒ पश्य॑न्ती रपान॒तीः प्रा॑ण॒तीः प्रा॑ण॒ती र॑पान॒तीः पश्य॑न्तीः । \newline
44. प्रा॒ण॒तीरिति॑ प्र - अ॒न॒तीः । \newline
45. अ॒पा॒न॒तीः पश्य॑न्तीः॒ पश्य॑न्ती रपान॒ती र॑पान॒तीः पश्य॑न्तीः शृण्व॒तीः शृ॑ण्व॒तीः पश्य॑न्ती रपान॒ती र॑पान॒तीः पश्य॑न्तीः शृण्व॒तीः । \newline
46. अ॒पा॒न॒तीरित्य॑प - अ॒न॒तीः । \newline
47. पश्य॑न्तीः शृण्व॒तीः शृ॑ण्व॒तीः पश्य॑न्तीः॒ पश्य॑न्तीः शृण्व॒तीर् न न शृ॑ण्व॒तीः पश्य॑न्तीः॒ पश्य॑न्तीः शृण्व॒तीर् न । \newline
48. शृ॒ण्व॒तीर् न न शृ॑ण्व॒तीः शृ॑ण्व॒तीर् न मि॑थु॒नी मि॑थु॒नी न शृ॑ण्व॒तीः शृ॑ण्व॒तीर् न मि॑थु॒नी । \newline
49. न मि॑थु॒नी मि॑थु॒नी न न मि॑थु॒नी अ॑भवन् नभवन् मिथु॒नी न न मि॑थु॒नी अ॑भवन्न् । \newline
50. मि॒थु॒नी अ॑भवन् नभवन् मिथु॒नी मि॑थु॒नी अ॑भव॒न् तासु॒ तास्व॑भवन् मिथु॒नी मि॑थु॒नी अ॑भव॒न् तासु॑ । \newline
51. अ॒भ॒व॒न् तासु॒ तास्व॑भवन् नभव॒न् तासु॑ त्रि॒वृत् त्रि॒वृत् तास्व॑भवन् नभव॒न् तासु॑ त्रि॒वृत् । \newline
52. तासु॑ त्रि॒वृत् त्रि॒वृत् तासु॒ तासु॑ त्रि॒वृद॑ स्यसि त्रि॒वृत् तासु॒ तासु॑ त्रि॒वृद॑सि । \newline
53. त्रि॒वृ द॑स्यसि त्रि॒वृत् त्रि॒वृ द॒सीती त्य॑सि त्रि॒वृत् त्रि॒वृ द॒सीति॑ । \newline
54. त्रि॒वृदिति॑ त्रि - वृत् । \newline
55. अ॒सीती त्य॑स्य॒ सीत्ये॒ वैवे त्य॑स्य॒ सीत्ये॒व । \newline
56. इत्ये॒ वैवेतीत्ये॒व मि॑थु॒नम् मि॑थु॒न मे॒वे तीत्ये॒व मि॑थु॒नम् । \newline
57. ए॒व मि॑थु॒नम् मि॑थु॒न मे॒वैव मि॑थु॒न म॑दधा ददधान् मिथु॒न मे॒वैव मि॑थु॒न म॑दधात् । \newline
58. मि॒थु॒न म॑दधा ददधान् मिथु॒नम् मि॑थु॒न म॑दधा॒त् ता स्ता अ॑दधान् मिथु॒नम् मि॑थु॒न म॑दधा॒त् ताः । \newline
59. अ॒द॒धा॒त् ता स्ता अ॑दधा ददधा॒त् ताः प्र॒जाः प्र॒जा स्ता अ॑दधा ददधा॒त् ताः प्र॒जाः । \newline
60. ताः प्र॒जाः प्र॒जा स्ता स्ताः प्र॒जा मि॑थु॒नी मि॑थु॒नी प्र॒जा स्ता स्ताः प्र॒जा मि॑थु॒नी । \newline
61. प्र॒जा मि॑थु॒नी मि॑थु॒नी प्र॒जाः प्र॒जा मि॑थु॒नी भव॑न्ती॒र् भव॑न्तीर् मिथु॒नी प्र॒जाः प्र॒जा मि॑थु॒नी भव॑न्तीः । \newline
62. प्र॒जा इति॑ प्र - जाः । \newline
63. मि॒थु॒नी भव॑न्ती॒र् भव॑न्तीर् मिथु॒नी मि॑थु॒नी भव॑न्ती॒र् न न भव॑न्तीर् मिथु॒नी मि॑थु॒नी भव॑न्ती॒र् न । \newline
\pagebreak
\markright{ TS 5.3.6.3  \hfill https://www.vedavms.in \hfill}

\section{ TS 5.3.6.3 }

\textbf{TS 5.3.6.3 } \newline
\textbf{Samhita Paata} \newline

भव॑न्ती॒र्न प्राजा॑यन्त॒ ताः सꣳ॑रो॒हो॑ऽसि नीरो॒हो॑ऽसीत्ये॒व प्राऽज॑नय॒त् ताः प्र॒जाः प्रजा॑ता॒ न प्रत्य॑तिष्ठ॒न् ता व॑सु॒को॑ऽसि॒ वेष॑श्रिरसि॒ वस्य॑ष्टिर॒सीत्ये॒वैषु लो॒केषु॒ प्रत्य॑स्थापय॒द्यदाह॑ वसु॒को॑ऽसि॒ वेष॑श्रिरसि॒ वस्य॑ष्टिर॒सीति॑ प्र॒जा ए॒व प्रजा॑ता ए॒षु लो॒केषु॒ प्रति॑ष्ठापयति॒ सात्मा॒ऽन्तरि॑क्षꣳ ( ) रोहति॒ सप्रा॑णो॒ऽमुष्मि॑न् ॅलो॒के प्रति॑ तिष्ठ॒त्यव्य॑र्द्धुकः प्राणापा॒नाभ्यां᳚ भवति॒ य ए॒वं ॅवेद॑ ॥ \newline

\textbf{Pada Paata} \newline

भव॑न्तीः । न । प्रेति॑ । अ॒जा॒य॒न्त॒ । ताः । सꣳ॒॒रो॒ह इति॑ सं - रो॒हः । अ॒सि॒ । नी॒रो॒ह इति॑ निः - रो॒हः । अ॒सि॒ । इति॑ । ए॒व । प्रेति॑ । अ॒ज॒न॒य॒त् । ताः । प्र॒जा इति॑ प्र - जाः । प्रजा॑ता॒ इति॒ प्र-जा॒ताः॒ । न । प्रतीति॑ । अ॒ति॒ष्ठ॒न्न् । ताः । व॒सु॒कः । अ॒सि॒ । वेष॑श्रि॒रिति॒ वेष॑-श्रिः॒ । अ॒सि॒ । वस्य॑ष्टिः । अ॒सि॒ । इति॑ । ए॒व । ए॒षु । लो॒केषु॑ । प्रतीति॑ । अ॒स्था॒प॒य॒त् । यत् । आह॑ । व॒सु॒कः । अ॒सि॒ । वेष॑श्रि॒रिति॒ वेष॑ - श्रिः॒ । अ॒सि॒ । वस्य॑ष्टिः । अ॒सि॒ । इति॑ । प्र॒जा इति॑ प्र - जाः । ए॒व । प्रजा॑ता॒ इति॒ प्र - जा॒ताः॒ । ए॒षु । लो॒केषु॑ । प्रतीति॑ । स्था॒प॒य॒ति॒ । सात्मेति॒ स - आ॒त्मा॒ । अ॒न्तरि॑क्षम् ( ) । रो॒ह॒ति॒ । सप्रा॑ण॒ इति॒ स - प्रा॒णः॒ । अ॒मुष्मिन्न्॑ । लो॒के । प्रतीति॑ । ति॒ष्ठ॒ति॒ । अव्य॑द्‌र्धुक॒ इत्यवि॑ - अ॒द्‌र्धु॒कः॒ । प्रा॒णा॒पा॒नाभ्या॒मिति॑ प्राण - अ॒पा॒नाभ्या᳚म् । भ॒व॒ति॒ । यः । ए॒वम् । वेद॑ ॥  \newline


\textbf{Krama Paata} \newline

भव॑न्ती॒र् न । न प्र । प्राजा॑यन्त । अ॒जा॒य॒न्त॒ ताः । ताः सꣳ॑रो॒हः । सꣳ॒॒रो॒हो॑ऽसि । सꣳ॒॒रो॒ह इति॑ सम् - रो॒हः । अ॒सि॒ नी॒रो॒हः । नी॒रो॒हो॑ऽसि । नी॒रो॒ह इति॑ निः - रो॒हः । अ॒सीति॑ । इत्ये॒व । ए॒व प्र । प्राज॑नयत् । अ॒ज॒न॒य॒त् ताः । ताः प्र॒जाः । प्र॒जाः प्रजा॑ताः । प्र॒जा इति॑ प्र - जाः । प्रजा॑ता॒ न । प्रजा॑ता॒ इति॒ प्र - जा॒ताः॒ । न प्रति॑ । प्रत्य॑तिष्ठन्न् । अ॒ति॒ष्ठ॒न् ताः । ता व॑सु॒कः । व॒सु॒को॑ऽसि । अ॒सि॒ वेष॑श्रिः । वेष॑श्रिरसि । वेष॑श्रि॒रिति॒ वेष॑ - श्रिः॒ । अ॒सि॒ वस्य॑ष्टिः । वस्य॑ष्टिरसि । अ॒सीति॑ । इत्ये॒व । ए॒वैषु । ए॒षु लो॒केषु॑ । लो॒केषु॒ प्रति॑ । प्रत्य॑स्थापयत् । अ॒स्था॒प॒य॒द् यत् । यदाह॑ । आह॑ वसु॒कः । व॒सु॒को॑ऽसि । अ॒सि॒ वेष॑श्रिः । वेष॑श्रिरसि । वेष॑श्रि॒रिति॒ वेष॑ - श्रिः॒ । अ॒सि॒ वस्य॑ष्टिः । वस्य॑ष्टिरसि । अ॒सीति॑ । इति॑ प्र॒जाः । प्र॒जा ए॒व । प्र॒जा इति॑ प्र - जाः । ए॒व प्रजा॑ताः । प्रजा॑ता ए॒षु । प्रजा॑ता॒ इति॒ प्र - जा॒ताः॒ । ए॒षु लो॒केषु॑ । लो॒केषु॒ प्रति॑ । प्रति॑ ष्ठापयति । स्था॒प॒य॒ति॒ सात्मा᳚ । सात्मा॒ऽन्तरि॑क्षम् ( ) । सात्मेति॒ स - आ॒त्मा॒ । अ॒न्तरि॑क्षꣳ रोहति । रो॒ह॒ति॒ सप्रा॑णः । सप्रा॑णो॒ऽमुष्मिन्न्॑ । सप्रा॑ण॒ इति॒ स - प्रा॒णः॒ । अ॒मुष्मि॑न् ॅलो॒के । लो॒के प्रति॑ । प्रति॑ तिष्ठति । ति॒ष्ठ॒त्यव्य॑र्द्धुकः । अव्य॑र्द्धुकः प्राणापा॒नाभ्या᳚म् । अव्य॑र्द्धुक॒ इत्यवि॑ - अ॒र्द्धु॒कः॒ । 
प्रा॒णा॒पा॒नाभ्या᳚म् भवति । प्रा॒णा॒पा॒नाभ्या॒मिति॑ प्राण - अ॒पा॒नाभ्या᳚म् । भ॒व॒ति॒ यः । य ए॒वम् । ए॒वम् ॅवेद॑ । वेदेति॒ वेद॑ । \newline

\textbf{Jatai Paata} \newline

1. भव॑न्ती॒र् न न भव॑न्ती॒र् भव॑न्ती॒र् न । \newline
2. न प्र प्र ण न प्र । \newline
3. प्रा जा॑यन्ता जायन्त॒ प्र प्रा जा॑यन्त । \newline
4. अ॒जा॒य॒न्त॒ ता स्ता अ॑जायन्ता जायन्त॒ ताः । \newline
5. ताः सꣳ॑रो॒हः सꣳ॑रो॒ह स्ता स्ताः सꣳ॑रो॒हः । \newline
6. सꣳ॒॒रो॒हो᳚ ऽस्यसि सꣳरो॒हः सꣳ॑रो॒हो॑ ऽसि । \newline
7. सꣳ॒॒रो॒ह इति॑ सं - रो॒हः । \newline
8. अ॒सि॒ नी॒रो॒हो नी॑रो॒हो᳚ ऽस्यसि नीरो॒हः । \newline
9. नी॒रो॒हो᳚ ऽस्यसि नीरो॒हो नी॑रो॒हो॑ ऽसि । \newline
10. नी॒रो॒ह इति॑ निः - रो॒हः । \newline
11. अ॒सीती त्य॑स्य॒ सीति॑ । \newline
12. इत्ये॒वैवेती त्ये॒व । \newline
13. ए॒व प्र प्रैवैव प्र । \newline
14. प्रा ज॑नय दजनय॒त् प्र प्रा ज॑नयत् । \newline
15. अ॒ज॒न॒य॒त् ता स्ता अ॑जनय दजनय॒त् ताः । \newline
16. ताः प्र॒जाः प्र॒जा स्ता स्ताः प्र॒जाः । \newline
17. प्र॒जाः प्रजा॑ताः॒ प्रजा॑ताः प्र॒जाः प्र॒जाः प्रजा॑ताः । \newline
18. प्र॒जा इति॑ प्र - जाः । \newline
19. प्रजा॑ता॒ न न प्रजा॑ताः॒ प्रजा॑ता॒ न । \newline
20. प्रजा॑ता॒ इति॒ प्र - जा॒ताः॒ । \newline
21. न प्रति॒ प्रति॒ न न प्रति॑ । \newline
22. प्रत्य॑तिष्ठन् नतिष्ठ॒न् प्रति॒ प्रत्य॑तिष्ठन्न् । \newline
23. अ॒ति॒ष्ठ॒न् ता स्ता अ॑तिष्ठन् नतिष्ठ॒न् ताः । \newline
24. ता व॑सु॒को व॑सु॒क स्ता स्ता व॑सु॒कः । \newline
25. व॒सु॒को᳚ ऽस्यसि वसु॒को व॑सु॒को॑ ऽसि । \newline
26. अ॒सि॒ वेष॑श्रि॒र् वेष॑श्रि रस्यसि॒ वेष॑श्रिः । \newline
27. वेष॑श्रि रस्यसि॒ वेष॑श्रि॒र् वेष॑श्रि रसि । \newline
28. वेष॑श्रि॒रिति॒ वेष॑ - श्रिः॒ । \newline
29. अ॒सि॒ वस्य॑ष्टि॒र् वस्य॑ष्टि रस्यसि॒ वस्य॑ष्टिः । \newline
30. वस्य॑ष्टि रस्यसि॒ वस्य॑ष्टि॒र् वस्य॑ष्टि रसि । \newline
31. अ॒सीती त्य॑स्य॒ सीति॑ । \newline
32. इत्ये॒वैवेती त्ये॒व । \newline
33. ए॒वैष्वे᳚(1॒) ष्वे॑वै वैषु । \newline
34. ए॒षु लो॒केषु॑ लो॒के ष्वे॒ ष्वे॑षु लो॒केषु॑ । \newline
35. लो॒केषु॒ प्रति॒ प्रति॑ लो॒केषु॑ लो॒केषु॒ प्रति॑ । \newline
36. प्रत्य॑स्थापय दस्थापय॒त् प्रति॒ प्रत्य॑स्थापयत् । \newline
37. अ॒स्था॒प॒य॒द् यद् यद॑स्थापय दस्थापय॒द् यत् । \newline
38. यदाहाह॒ यद् यदाह॑ । \newline
39. आह॑ वसु॒को व॑सु॒क आहाह॑ वसु॒कः । \newline
40. व॒सु॒को᳚ ऽस्यसि वसु॒को व॑सु॒को॑ ऽसि । \newline
41. अ॒सि॒ वेष॑श्रि॒र् वेष॑श्रि रस्यसि॒ वेष॑श्रिः । \newline
42. वेष॑श्रि रस्यसि॒ वेष॑श्रि॒र् वेष॑श्रि रसि । \newline
43. वेष॑श्रि॒रिति॒ वेष॑ - श्रिः॒ । \newline
44. अ॒सि॒ वस्य॑ष्टि॒र् वस्य॑ष्टि रस्यसि॒ वस्य॑ष्टिः । \newline
45. वस्य॑ष्टि रस्यसि॒ वस्य॑ष्टि॒र् वस्य॑ष्टि रसि । \newline
46. अ॒सीती त्य॑स्य॒ सीति॑ । \newline
47. इति॑ प्र॒जाः प्र॒जा इतीति॑ प्र॒जाः । \newline
48. प्र॒जा ए॒वैव प्र॒जाः प्र॒जा ए॒व । \newline
49. प्र॒जा इति॑ प्र - जाः । \newline
50. ए॒व प्रजा॑ताः॒ प्रजा॑ता ए॒वैव प्रजा॑ताः । \newline
51. प्रजा॑ता ए॒ष्वे॑षु प्रजा॑ताः॒ प्रजा॑ता ए॒षु । \newline
52. प्रजा॑ता॒ इति॒ प्र - जा॒ताः॒ । \newline
53. ए॒षु लो॒केषु॑ लो॒के ष्वे॒ ष्वे॑षु लो॒केषु॑ । \newline
54. लो॒केषु॒ प्रति॒ प्रति॑ लो॒केषु॑ लो॒केषु॒ प्रति॑ । \newline
55. प्रति॑ ष्ठापयति स्थापयति॒ प्रति॒ प्रति॑ ष्ठापयति । \newline
56. स्था॒प॒य॒ति॒ सात्मा॒ सात्मा᳚ स्थापयति स्थापयति॒ सात्मा᳚ । \newline
57. सात्मा॒ ऽन्तरि॑क्ष म॒न्तरि॑क्षꣳ॒॒ सात्मा॒ सात्मा॒ ऽन्तरि॑क्षम् । \newline
58. सात्मेति॒ स - आ॒त्मा॒ । \newline
59. अ॒न्तरि॑क्षꣳ रोहति रोहत्य॒ न्तरि॑क्ष म॒न्तरि॑क्षꣳ रोहति । \newline
60. रो॒ह॒ति॒ सप्रा॑णः॒ सप्रा॑णो रोहति रोहति॒ सप्रा॑णः । \newline
61. सप्रा॑णो॒ ऽमुष्मि॑न् न॒मुष्मि॒न् थ्सप्रा॑णः॒ सप्रा॑णो॒ ऽमुष्मिन्न्॑ । \newline
62. सप्रा॑ण॒ इति॒ स - प्रा॒णः॒ । \newline
63. अ॒मुष्मि॑न् ॅलो॒के लो॒के॑ ऽमुष्मि॑न् न॒मुष्मि॑न् ॅलो॒के । \newline
64. लो॒के प्रति॒ प्रति॑ लो॒के लो॒के प्रति॑ । \newline
65. प्रति॑ तिष्ठति तिष्ठति॒ प्रति॒ प्रति॑ तिष्ठति । \newline
66. ति॒ष्ठ॒त्य व्य॑र्द्धु॒को ऽव्य॑र्द्धुक स्तिष्ठति तिष्ठ॒त्य व्य॑र्द्धुकः । \newline
67. अव्य॑र्द्धुकः प्राणापा॒नाभ्या᳚म् प्राणापा॒नाभ्या॒ मव्य॑र्द्धु॒को ऽव्य॑र्द्धुकः प्राणापा॒नाभ्या᳚म् । \newline
68. अव्य॑र्द्धुक॒ इत्यवि॑ - अ॒र्द्धु॒कः॒ । \newline
69. प्रा॒णा॒पा॒नाभ्या᳚म् भवति भवति प्राणापा॒नाभ्या᳚म् प्राणापा॒नाभ्या᳚म् भवति । \newline
70. प्रा॒णा॒पा॒नाभ्या॒मिति॑ प्राण - अ॒पा॒नाभ्या᳚म् । \newline
71. भ॒व॒ति॒ यो यो भ॑वति भवति॒ यः । \newline
72. य ए॒व मे॒वं ॅयो य ए॒वम् । \newline
73. ए॒वं ॅवेद॒ वेदै॒व मे॒वं ॅवेद॑ । \newline
74. वेदेति॒ वेद॑ । \newline

\textbf{Ghana Paata } \newline

1. भव॑न्ती॒र् न न भव॑न्ती॒र् भव॑न्ती॒र् न प्र प्र ण भव॑न्ती॒र् भव॑न्ती॒र् न प्र । \newline
2. न प्र प्र ण न प्रा जा॑यन्ता जायन्त॒ प्र ण न प्राजा॑यन्त । \newline
3. प्रा जा॑यन्ता जायन्त॒ प्र प्राजा॑यन्त॒ ता स्ता अ॑जायन्त॒ प्र प्राजा॑यन्त॒ ताः । \newline
4. अ॒जा॒य॒न्त॒ ता स्ता अ॑जायन्ता जायन्त॒ ताः सꣳ॑रो॒हः सꣳ॑रो॒ह स्ता अ॑जायन्ता जायन्त॒ ताः सꣳ॑रो॒हः । \newline
5. ताः सꣳ॑रो॒हः सꣳ॑रो॒ह स्ता स्ताः सꣳ॑रो॒हो᳚ ऽस्यसि सꣳरो॒ह स्ता स्ताः सꣳ॑रो॒हो॑ ऽसि । \newline
6. सꣳ॒॒रो॒हो᳚ ऽस्यसि सꣳरो॒हः सꣳ॑रो॒हो॑ ऽसि नीरो॒हो नी॑रो॒हो॑ ऽसि सꣳरो॒हः सꣳ॑रो॒हो॑ ऽसि नीरो॒हः । \newline
7. सꣳ॒॒रो॒ह इति॑ सं - रो॒हः । \newline
8. अ॒सि॒ नी॒रो॒हो नी॑रो॒हो᳚ ऽस्यसि नीरो॒हो᳚ ऽस्यसि नीरो॒हो᳚ ऽस्यसि नीरो॒हो॑ ऽसि । \newline
9. नी॒रो॒हो᳚ ऽस्यसि नीरो॒हो नी॑रो॒हो॑ ऽसीती त्य॑सि नीरो॒हो नी॑रो॒हो॑ ऽसीति॑ । \newline
10. नी॒रो॒ह इति॑ निः - रो॒हः । \newline
11. अ॒सीती त्य॑स्य॒ सीत्ये॒वैवे त्य॑स्य॒ सीत्ये॒व । \newline
12. इत्ये॒ वैवे तीत्ये॒व प्र प्रैवे तीत्ये॒व प्र । \newline
13. ए॒व प्र प्रैवैव प्रा ज॑नय दजनय॒त् प्रैवैव प्रा ज॑नयत् । \newline
14. प्रा ज॑नय दजनय॒त् प्र प्रा ज॑नय॒त् ता स्ता अ॑जनय॒त् प्र प्रा ज॑नय॒त् ताः । \newline
15. अ॒ज॒न॒य॒त् ता स्ता अ॑जनय दजनय॒त् ताः प्र॒जाः प्र॒जा स्ता अ॑जनय दजनय॒त् ताः प्र॒जाः । \newline
16. ताः प्र॒जाः प्र॒जा स्ता स्ताः प्र॒जाः प्रजा॑ताः॒ प्रजा॑ताः प्र॒जा स्ता स्ताः प्र॒जाः प्रजा॑ताः । \newline
17. प्र॒जाः प्रजा॑ताः॒ प्रजा॑ताः प्र॒जाः प्र॒जाः प्रजा॑ता॒ न न प्रजा॑ताः प्र॒जाः प्र॒जाः प्रजा॑ता॒ न । \newline
18. प्र॒जा इति॑ प्र - जाः । \newline
19. प्रजा॑ता॒ न न प्रजा॑ताः॒ प्रजा॑ता॒ न प्रति॒ प्रति॒ न प्रजा॑ताः॒ प्रजा॑ता॒ न प्रति॑ । \newline
20. प्रजा॑ता॒ इति॒ प्र - जा॒ताः॒ । \newline
21. न प्रति॒ प्रति॒ न न प्रत्य॑तिष्ठन् नतिष्ठ॒न् प्रति॒ न न प्रत्य॑तिष्ठन्न् । \newline
22. प्रत्य॑तिष्ठन् नतिष्ठ॒न् प्रति॒ प्रत्य॑तिष्ठ॒न् ता स्ता अ॑तिष्ठ॒न् प्रति॒ प्रत्य॑तिष्ठ॒न् ताः । \newline
23. अ॒ति॒ष्ठ॒न् ता स्ता अ॑तिष्ठन् नतिष्ठ॒न् ता व॑सु॒को व॑सु॒क स्ता अ॑तिष्ठन् नतिष्ठ॒न् ता व॑सु॒कः । \newline
24. ता व॑सु॒को व॑सु॒क स्ता स्ता व॑सु॒को᳚ ऽस्यसि वसु॒क स्ता स्ता व॑सु॒को॑ ऽसि । \newline
25. व॒सु॒को᳚ ऽस्यसि वसु॒को व॑सु॒को॑ ऽसि॒ वेष॑श्रि॒र् वेष॑श्रि रसि वसु॒को व॑सु॒को॑ ऽसि॒ वेष॑श्रिः । \newline
26. अ॒सि॒ वेष॑श्रि॒र् वेष॑श्रि रस्यसि॒ वेष॑श्रि रस्यसि॒ वेष॑श्रि रस्यसि॒ वेष॑श्रिरसि । \newline
27. वेष॑श्रि रस्यसि॒ वेष॑श्रि॒र् वेष॑श्रि रसि॒ वस्य॑ष्टि॒र् वस्य॑ष्टि रसि॒ वेष॑श्रि॒र् वेष॑श्रि रसि॒ वस्य॑ष्टिः । \newline
28. वेष॑श्रि॒रिति॒ वेष॑ - श्रिः॒ । \newline
29. अ॒सि॒ वस्य॑ष्टि॒र् वस्य॑ष्टि रस्यसि॒ वस्य॑ष्टि रस्यसि॒ वस्य॑ष्टि रस्यसि॒ वस्य॑ष्टि रसि । \newline
30. वस्य॑ष्टि रस्यसि॒ वस्य॑ष्टि॒र् वस्य॑ष्टि र॒सीती त्य॑सि॒ वस्य॑ष्टि॒र् वस्य॑ष्टि र॒सीति॑ । \newline
31. अ॒सीती त्य॑स्य॒ सीत्ये॒ वैवे त्य॑स्य॒ सीत्ये॒व । \newline
32. इत्ये॒ वैवे तीत्ये॒ वैष्वे᳚(1॒)ष्वे॑वे तीत्ये॒ वैषु । \newline
33. ए॒वैष्वे᳚(1॒)ष्वे॑वै वैषु लो॒केषु॑ लो॒केष्वे॒ ष्वे॑वै वैषु लो॒केषु॑ । \newline
34. ए॒षु लो॒केषु॑ लो॒के ष्वे॒ ष्वे॑षु लो॒केषु॒ प्रति॒ प्रति॑ लो॒के ष्वे॒ ष्वे॑षु लो॒केषु॒ प्रति॑ । \newline
35. लो॒केषु॒ प्रति॒ प्रति॑ लो॒केषु॑ लो॒केषु॒ प्रत्य॑स्थापय दस्थापय॒त् प्रति॑ लो॒केषु॑ लो॒केषु॒ प्रत्य॑स्थापयत् । \newline
36. प्रत्य॑स्थापय दस्थापय॒त् प्रति॒ प्रत्य॑स्थापय॒द् यद् यद॑ स्थापय॒त् प्रति॒ प्रत्य॑स्थापय॒द् यत् । \newline
37. अ॒स्था॒प॒य॒द् यद् यद॑स्थापय दस्थापय॒द् यदाहाह॒ यद॑स्थापय दस्थापय॒द् यदाह॑ । \newline
38. यदाहाह॒ यद् यदाह॑ वसु॒को व॑सु॒क आह॒ यद् यदाह॑ वसु॒कः । \newline
39. आह॑ वसु॒को व॑सु॒क आहाह॑ वसु॒को᳚ ऽस्यसि वसु॒क आहाह॑ वसु॒को॑ ऽसि । \newline
40. व॒सु॒को᳚ ऽस्यसि वसु॒को व॑सु॒को॑ ऽसि॒ वेष॑श्रि॒र् वेष॑श्रि रसि वसु॒को व॑सु॒को॑ ऽसि॒ वेष॑श्रिः । \newline
41. अ॒सि॒ वेष॑श्रि॒र् वेष॑श्रि रस्यसि॒ वेष॑श्रि रस्यसि॒ वेष॑श्रि रस्यसि॒ वेष॑श्रि रसि । \newline
42. वेष॑श्रि रस्यसि॒ वेष॑श्रि॒र् वेष॑श्रि रसि॒ वस्य॑ष्टि॒र् वस्य॑ष्टि रसि॒ वेष॑श्रि॒र् वेष॑श्रि रसि॒ वस्य॑ष्टिः । \newline
43. वेष॑श्रि॒रिति॒ वेष॑ - श्रिः॒ । \newline
44. अ॒सि॒ वस्य॑ष्टि॒र् वस्य॑ष्टि रस्यसि॒ वस्य॑ष्टि रस्यसि॒ वस्य॑ष्टि रस्यसि॒ वस्य॑ष्टि रसि । \newline
45. वस्य॑ष्टि रस्यसि॒ वस्य॑ष्टि॒र् वस्य॑ष्टि र॒सीती त्य॑सि॒ वस्य॑ष्टि॒र् वस्य॑ष्टि र॒सीति॑ । \newline
46. अ॒सीती त्य॑स्य॒ सीति॑ प्र॒जाः प्र॒जा इत्य॑स्य॒ सीति॑ प्र॒जाः । \newline
47. इति॑ प्र॒जाः प्र॒जा इतीति॑ प्र॒जा ए॒वैव प्र॒जा इतीति॑ प्र॒जा ए॒व । \newline
48. प्र॒जा ए॒वैव प्र॒जाः प्र॒जा ए॒व प्रजा॑ताः॒ प्रजा॑ता ए॒व प्र॒जाः प्र॒जा ए॒व प्रजा॑ताः । \newline
49. प्र॒जा इति॑ प्र - जाः । \newline
50. ए॒व प्रजा॑ताः॒ प्रजा॑ता ए॒वैव प्रजा॑ता ए॒ष्वे॑षु प्रजा॑ता ए॒वैव प्रजा॑ता ए॒षु । \newline
51. प्रजा॑ता ए॒ष्वे॑षु प्रजा॑ताः॒ प्रजा॑ता ए॒षु लो॒केषु॑ लो॒के ष्वे॒षु प्रजा॑ताः॒ प्रजा॑ता ए॒षु लो॒केषु॑ । \newline
52. प्रजा॑ता॒ इति॒ प्र - जा॒ताः॒ । \newline
53. ए॒षु लो॒केषु॑ लो॒के ष्वे॒ ष्वे॑षु लो॒केषु॒ प्रति॒ प्रति॑ लो॒के ष्वे॒ ष्वे॑षु लो॒केषु॒ प्रति॑ । \newline
54. लो॒केषु॒ प्रति॒ प्रति॑ लो॒केषु॑ लो॒केषु॒ प्रति॑ ष्ठापयति स्थापयति॒ प्रति॑ लो॒केषु॑ लो॒केषु॒ प्रति॑ ष्ठापयति । \newline
55. प्रति॑ ष्ठापयति स्थापयति॒ प्रति॒ प्रति॑ ष्ठापयति॒ सात्मा॒ सात्मा᳚ स्थापयति॒ प्रति॒ प्रति॑ ष्ठापयति॒ सात्मा᳚ । \newline
56. स्था॒प॒य॒ति॒ सात्मा॒ सात्मा᳚ स्थापयति स्थापयति॒ सात्मा॒ ऽन्तरि॑क्ष म॒न्तरि॑क्षꣳ॒॒ सात्मा᳚ स्थापयति स्थापयति॒ सात्मा॒ ऽन्तरि॑क्षम् । \newline
57. सात्मा॒ ऽन्तरि॑क्ष म॒न्तरि॑क्षꣳ॒॒ सात्मा॒ सात्मा॒ ऽन्तरि॑क्षꣳ रोहति रोह त्य॒न्तरि॑क्षꣳ॒॒ सात्मा॒ सात्मा॒ ऽन्तरि॑क्षꣳ रोहति । \newline
58. सात्मेति॒ स - आ॒त्मा॒ । \newline
59. अ॒न्तरि॑क्षꣳ रोहति रोह त्य॒न्तरि॑क्ष म॒न्तरि॑क्षꣳ रोहति॒ सप्रा॑णः॒ सप्रा॑णो रोह त्य॒न्तरि॑क्ष म॒न्तरि॑क्षꣳ रोहति॒ सप्रा॑णः । \newline
60. रो॒ह॒ति॒ सप्रा॑णः॒ सप्रा॑णो रोहति रोहति॒ सप्रा॑णो॒ ऽमुष्मि॑न् न॒मुष्मि॒न् थ्सप्रा॑णो रोहति रोहति॒ सप्रा॑णो॒ ऽमुष्मिन्न्॑ । \newline
61. सप्रा॑णो॒ ऽमुष्मि॑न् न॒मुष्मि॒न् थ्सप्रा॑णः॒ सप्रा॑णो॒ ऽमुष्मि॑न् ॅलो॒के लो॒के॑ ऽमुष्मि॒न् थ्सप्रा॑णः॒ सप्रा॑णो॒ ऽमुष्मि॑न् ॅलो॒के । \newline
62. सप्रा॑ण॒ इति॒ स - प्रा॒णः॒ । \newline
63. अ॒मुष्मि॑न् ॅलो॒के लो॒के॑ ऽमुष्मि॑न् न॒मुष्मि॑न् ॅलो॒के प्रति॒ प्रति॑ लो॒के॑ ऽमुष्मि॑न् न॒मुष्मि॑न् ॅलो॒के प्रति॑ । \newline
64. लो॒के प्रति॒ प्रति॑ लो॒के लो॒के प्रति॑ तिष्ठति तिष्ठति॒ प्रति॑ लो॒के लो॒के प्रति॑ तिष्ठति । \newline
65. प्रति॑ तिष्ठति तिष्ठति॒ प्रति॒ प्रति॑ तिष्ठ॒ त्यव्य॑र्द्धु॒को ऽव्य॑र्द्धुक स्तिष्ठति॒ प्रति॒ प्रति॑ तिष्ठ॒ त्यव्य॑र्द्धुकः । \newline
66. ति॒ष्ठ॒ त्यव्य॑र्द्धु॒को ऽव्य॑र्द्धुक स्तिष्ठति तिष्ठ॒ त्यव्य॑र्द्धुकः प्राणापा॒नाभ्या᳚म् प्राणापा॒नाभ्या॒ मव्य॑र्द्धुक स्तिष्ठति तिष्ठ॒ त्यव्य॑र्द्धुकः प्राणापा॒नाभ्या᳚म् । \newline
67. अव्य॑र्द्धुकः प्राणापा॒नाभ्या᳚म् प्राणापा॒नाभ्या॒ मव्य॑र्द्धु॒को ऽव्य॑र्द्धुकः प्राणापा॒नाभ्या᳚म् भवति भवति प्राणापा॒नाभ्या॒ मव्य॑र्द्धु॒को ऽव्य॑र्द्धुकः प्राणापा॒नाभ्या᳚म् भवति । \newline
68. अव्य॑र्द्धुक॒ इत्यवि॑ - अ॒र्द्धु॒कः॒ । \newline
69. प्रा॒णा॒पा॒नाभ्या᳚म् भवति भवति प्राणापा॒नाभ्या᳚म् प्राणापा॒नाभ्या᳚म् भवति॒ यो यो भ॑वति प्राणापा॒नाभ्या᳚म् प्राणापा॒नाभ्या᳚म् भवति॒ यः । \newline
70. प्रा॒णा॒पा॒नाभ्या॒मिति॑ प्राण - अ॒पा॒नाभ्या᳚म् । \newline
71. भ॒व॒ति॒ यो यो भ॑वति भवति॒ य ए॒व मे॒वं ॅयो भ॑वति भवति॒ य ए॒वम् । \newline
72. य ए॒व मे॒वं ॅयो य ए॒वं ॅवेद॒ वेदै॒वं ॅयो य ए॒वं ॅवेद॑ । \newline
73. ए॒वं ॅवेद॒ वेदै॒व मे॒वं ॅवेद॑ । \newline
74. वेदेति॒ वेद॑ । \newline
\pagebreak
\markright{ TS 5.3.7.1  \hfill https://www.vedavms.in \hfill}

\section{ TS 5.3.7.1 }

\textbf{TS 5.3.7.1 } \newline
\textbf{Samhita Paata} \newline

ना॒क॒सद्भि॒र्वै दे॒वाः सु॑व॒र्गं ॅलो॒कमा॑य॒न् तन्ना॑क॒सदां᳚ नाकस॒त्त्वं ॅयन्ना॑क॒सद॑ उप॒दधा॑ति नाक॒सद्भि॑रे॒व तद्-यज॑मानः सुव॒र्गं ॅलो॒कमे॑ति सुव॒र्गो वै लो॒को नाको॒ यस्यै॒ता उ॑पधी॒यन्ते॒ नास्मा॒ अकं॑ भवति यजमानायत॒नं ॅवै ना॑क॒सदो॒ यन्ना॑क॒सद॑ उप॒दधा᳚त्या॒यत॑नमे॒व तद्-यज॑मानः कुरुते पृ॒ष्ठानां॒ ॅवा ए॒तत् तेजः॒ संभृ॑तं॒ ॅयन्ना॑क॒सदो॒ यन्ना॑क॒सद॑ - [  ] \newline

\textbf{Pada Paata} \newline

ना॒क॒सद्भि॒रिति॑ नाक॒सत् - भिः॒ । वै । दे॒वाः । सु॒व॒र्गमिति॑ सुवः - गम् । लो॒कम् । आ॒य॒न्न् । तत् । ना॒क॒सदा॒मिति॑ नाक - सदा᳚म् । ना॒क॒स॒त्त्वमिति॑ नाकसत् - त्वम् । यत् । ना॒क॒सद॒ इति॑ नाक - सदः॑ । उ॒प॒दधा॒तीत्यु॑प - दधा॑ति । ना॒क॒सद्भि॒रिति॑ नाक॒सत् - भिः॒ । ए॒व । तत् । यज॑मानः । सु॒व॒र्गमिति॑ सुवः-गम् । लो॒कम् । ए॒ति॒ । सु॒व॒र्ग इति॑ सुवः-गः । वै । लो॒कः । नाकः॑ । यस्य॑ । ए॒ताः । उ॒प॒धी॒यन्त॒ इत्यु॑प - धी॒यन्ते᳚ । न । अ॒स्मै॒ । अक᳚म् । भ॒व॒ति॒ । य॒ज॒मा॒ना॒य॒त॒नमिति॑ यजमान - आ॒य॒त॒नम् । वै । ना॒क॒सद॒ इति॑ नाक - सदः॑ । यत् । ना॒क॒सद॒ इति॑ नाक - सदः॑ । उ॒प॒दधा॒तीत्यु॑प - दधा॑ति । आ॒यत॑न॒मित्या᳚ - यत॑नम् । ए॒व । तत् । यज॑मानः । कु॒रु॒ते॒ । पृ॒ष्ठाना᳚म् । वै । ए॒तत् । तेजः॑ । संभृ॑त॒मिति॒ सं - भृ॒त॒म् । यत् । ना॒क॒सद॒ इति॑ नाक - सदः॑ । यत् । ना॒क॒सद॒ इति॑ नाक - सदः॑ ।  \newline


\textbf{Krama Paata} \newline

ना॒क॒सद्भि॒र् वै । ना॒क॒सद्भि॒रिति॑ नाक॒सत् - भिः॒ । वै दे॒वाः । दे॒वाः सु॑व॒र्गम् । सु॒व॒र्गम् ॅलो॒कम् । सु॒व॒र्गमिति॑ सुवः - गम् । लो॒कमा॑यन्न् । आ॒य॒न् तत् । तन् ना॑क॒सदा᳚म् । ना॒क॒सदा᳚म् नाकस॒त्त्वम् । ना॒क॒सदा॒मिति॑ नाक - सदा᳚म् । ना॒क॒स॒त्त्वम् ॅयत् । ना॒क॒स॒त्त्वमिति॑ नाकसत् - त्वम् । यन् ना॑क॒सदः॑ । ना॒क॒सद॑ उप॒दधा॑ति । ना॒क॒सद॒ इति॑ नाक - सदः॑ । उ॒प॒दधा॑ति नाक॒सद्भिः॑ । उ॒प॒दधा॒तीत्यु॑प - दधा॑ति । ना॒क॒सद्भि॑रे॒व । ना॒क॒सद्भि॒रिति॑ नाक॒सत् - भिः॒ । ए॒व तत् । तद् यज॑मानः । यज॑मानः सुव॒र्गम् । सु॒व॒र्गम् ॅलो॒कम् । सु॒व॒र्गमिति॑ सुवः - गम् । लो॒कमे॑ति । ए॒ति॒ सु॒व॒र्गः । सु॒व॒र्गो वै । सु॒व॒र्ग इति॑ सुवः - गः । वै लो॒कः । लो॒को नाकः॑ । नाको॒ यस्य॑ । यस्ये॒ताः । ए॒ता उ॑पधी॒यन्ते᳚ । उ॒प॒धी॒यन्ते॒ न । उ॒प॒धी॒यन्त॒ इत्यु॑प - धी॒यन्ते᳚ । नास्मै᳚ । अ॒स्मा॒ अक᳚म् । अक॑म् भवति । भ॒व॒ति॒ य॒ज॒मा॒ना॒य॒त॒नम् । य॒ज॒मा॒ना॒य॒त॒नम् ॅवै । य॒ज॒मा॒ना॒य॒त॒नमिति॑ यजमान - आ॒य॒त॒नम् । वै ना॑क॒सदः॑ । ना॒क॒सदो॒ यत् । ना॒क॒सद॒ इति॑ नाक - सदः॑ । यन् ना॑क॒सदः॑ । ना॒क॒सद॑ उप॒दधा॑ति । ना॒क॒सद॒ इति॑ नाक - सदः॑ । उ॒प॒दधा᳚त्या॒यत॑नम् । उ॒प॒दधा॒तीत्यु॑प - दधा॑ति । आ॒यत॑नमे॒व । आ॒यत॑न॒मित्या᳚ - यत॑नम् । ए॒व तत् । तद् यज॑मानः । यज॑मानः कुरुते । कु॒रु॒ते॒ पृ॒ष्ठाना᳚म् । पृ॒ष्ठाना॒म् ॅवै । वा ए॒तत् । ए॒तत् तेजः॑ । तेजः॒ सम्भृ॑तम् । सम्भृ॑त॒म् ॅयत् । सम्भृ॑त॒मिति॒ सम् - भृ॒त॒म् । यन् ना॑क॒सदः॑ । ना॒क॒सदो॒ यत् । ना॒क॒सद॒ इति॑ नाक - सदः॑ । यन् ना॑क॒सदः॑ । ना॒क॒सद॑ उप॒दधा॑ति । ना॒क॒सद॒ इति॑ नाक - सदः॑ \newline

\textbf{Jatai Paata} \newline

1. ना॒क॒सद्भि॒र् वै वै ना॑क॒सद्भि॑र् नाक॒सद्भि॒र् वै । \newline
2. ना॒क॒सद्भि॒रिति॑ नाक॒सत् - भिः॒ । \newline
3. वै दे॒वा दे॒वा वै वै दे॒वाः । \newline
4. दे॒वाः सु॑व॒र्गꣳ सु॑व॒र्गम् दे॒वा दे॒वाः सु॑व॒र्गम् । \newline
5. सु॒व॒र्गम् ॅलो॒कम् ॅलो॒कꣳ सु॑व॒र्गꣳ सु॑व॒र्गम् ॅलो॒कम् । \newline
6. सु॒व॒र्गमिति॑ सुवः - गम् । \newline
7. लो॒क मा॑यन् नायन् ॅलो॒कम् ॅलो॒क मा॑यन्न् । \newline
8. आ॒य॒न् तत् तदा॑यन् नाय॒न् तत् । \newline
9. तन् ना॑क॒सदा᳚न् नाक॒सदा॒म् तत् तन् ना॑क॒सदा᳚म् । \newline
10. ना॒क॒सदा᳚न् नाकस॒त्त्वन् ना॑कस॒त्त्वन् ना॑क॒सदा᳚न् नाक॒सदा᳚न् नाकस॒त्त्वम् । \newline
11. ना॒क॒सदा॒मिति॑ नाक - सदा᳚म् । \newline
12. ना॒क॒स॒त्त्वं ॅयद् यन् ना॑कस॒त्त्वन् ना॑कस॒त्त्वं ॅयत् । \newline
13. ना॒क॒स॒त्त्वमिति॑ नाकसत् - त्वम् । \newline
14. यन् ना॑क॒सदो॑ नाक॒सदो॒ यद् यन् ना॑क॒सदः॑ । \newline
15. ना॒क॒सद॑ उप॒दधा᳚ त्युप॒दधा॑ति नाक॒सदो॑ नाक॒सद॑ उप॒दधा॑ति । \newline
16. ना॒क॒सद॒ इति॑ नाक - सदः॑ । \newline
17. उ॒प॒दधा॑ति नाक॒सद्भि॑र् नाक॒सद्भि॑ रुप॒दधा᳚ त्युप॒दधा॑ति नाक॒सद्भिः॑ । \newline
18. उ॒प॒दधा॒तीत्यु॑प - दधा॑ति । \newline
19. ना॒क॒सद्भि॑ रे॒वैव ना॑क॒सद्भि॑र् नाक॒सद्भि॑ रे॒व । \newline
20. ना॒क॒सद्भि॒रिति॑ नाक॒सत् - भिः॒ । \newline
21. ए॒व तत् तदे॒ वैव तत् । \newline
22. तद् यज॑मानो॒ यज॑मान॒ स्तत् तद् यज॑मानः । \newline
23. यज॑मानः सुव॒र्गꣳ सु॑व॒र्गं ॅयज॑मानो॒ यज॑मानः सुव॒र्गम् । \newline
24. सु॒व॒र्गम् ॅलो॒कम् ॅलो॒कꣳ सु॑व॒र्गꣳ सु॑व॒र्गम् ॅलो॒कम् । \newline
25. सु॒व॒र्गमिति॑ सुवः - गम् । \newline
26. लो॒क मे᳚त्येति लो॒कम् ॅलो॒क मे॑ति । \newline
27. ए॒ति॒ सु॒व॒र्गः सु॑व॒र्ग ए᳚त्येति सुव॒र्गः । \newline
28. सु॒व॒र्गो वै वै सु॑व॒र्गः सु॑व॒र्गो वै । \newline
29. सु॒व॒र्ग इति॑ सुवः - गः । \newline
30. वै लो॒को लो॒को वै वै लो॒कः । \newline
31. लो॒को नाको॒ नाको॑ लो॒को लो॒को नाकः॑ । \newline
32. नाको॒ यस्य॒ यस्य॒ नाको॒ नाको॒ यस्य॑ । \newline
33. यस्यै॒ता ए॒ता यस्य॒ यस्यै॒ताः । \newline
34. ए॒ता उ॑पधी॒यन्त॑ उपधी॒यन्त॑ ए॒ता ए॒ता उ॑पधी॒यन्ते᳚ । \newline
35. उ॒प॒धी॒यन्ते॒ न नोप॑धी॒यन्त॑ उपधी॒यन्ते॒ न । \newline
36. उ॒प॒धी॒यन्त॒ इत्यु॑प - धी॒यन्ते᳚ । \newline
37. नास्मा॑ अस्मै॒ न नास्मै᳚ । \newline
38. अ॒स्मा॒ अक॒ मक॑ मस्मा अस्मा॒ अक᳚म् । \newline
39. अक॑म् भवति भव॒ त्यक॒ मक॑म् भवति । \newline
40. भ॒व॒ति॒ य॒ज॒मा॒ना॒य॒त॒नं ॅय॑जमानायत॒नम् भ॑वति भवति यजमानायत॒नम् । \newline
41. य॒ज॒मा॒ना॒य॒त॒नं ॅवै वै य॑जमानायत॒नं ॅय॑जमानायत॒नं ॅवै । \newline
42. य॒ज॒मा॒ना॒य॒त॒नमिति॑ यजमान - आ॒य॒त॒नम् । \newline
43. वै ना॑क॒सदो॑ नाक॒सदो॒ वै वै ना॑क॒सदः॑ । \newline
44. ना॒क॒सदो॒ यद् यन् ना॑क॒सदो॑ नाक॒सदो॒ यत् । \newline
45. ना॒क॒सद॒ इति॑ नाक - सदः॑ । \newline
46. यन् ना॑क॒सदो॑ नाक॒सदो॒ यद् यन् ना॑क॒सदः॑ । \newline
47. ना॒क॒सद॑ उप॒दधा᳚ त्युप॒दधा॑ति नाक॒सदो॑ नाक॒सद॑ उप॒दधा॑ति । \newline
48. ना॒क॒सद॒ इति॑ नाक - सदः॑ । \newline
49. उ॒प॒दधा᳚ त्या॒यत॑न मा॒यत॑न मुप॒दधा᳚ त्युप॒दधा᳚ त्या॒यत॑नम् । \newline
50. उ॒प॒दधा॒तीत्यु॑प - दधा॑ति । \newline
51. आ॒यत॑न मे॒वै वायत॑न मा॒यत॑न मे॒व । \newline
52. आ॒यत॑न॒मित्या᳚ - यत॑नम् । \newline
53. ए॒व तत् तदे॒ वैव तत् । \newline
54. तद् यज॑मानो॒ यज॑मान॒ स्तत् तद् यज॑मानः । \newline
55. यज॑मानः कुरुते कुरुते॒ यज॑मानो॒ यज॑मानः कुरुते । \newline
56. कु॒रु॒ते॒ पृ॒ष्ठाना᳚म् पृ॒ष्ठाना᳚म् कुरुते कुरुते पृ॒ष्ठाना᳚म् । \newline
57. पृ॒ष्ठानां॒ ॅवै वै पृ॒ष्ठाना᳚म् पृ॒ष्ठानां॒ ॅवै । \newline
58. वा ए॒त दे॒तद् वै वा ए॒तत् । \newline
59. ए॒तत् तेज॒ स्तेज॑ ए॒त दे॒तत् तेजः॑ । \newline
60. तेजः॒ संभृ॑तꣳ॒॒ संभृ॑त॒म् तेज॒ स्तेजः॒ संभृ॑तम् । \newline
61. संभृ॑तं॒ ॅयद् यथ् संभृ॑तꣳ॒॒ संभृ॑तं॒ ॅयत् । \newline
62. संभृ॑त॒मिति॒ सं - भृ॒त॒म् । \newline
63. यन् ना॑क॒सदो॑ नाक॒सदो॒ यद् यन् ना॑क॒सदः॑ । \newline
64. ना॒क॒सदो॒ यद् यन् ना॑क॒सदो॑ नाक॒सदो॒ यत् । \newline
65. ना॒क॒सद॒ इति॑ नाक - सदः॑ । \newline
66. यन् ना॑क॒सदो॑ नाक॒सदो॒ यद् यन् ना॑क॒सदः॑ । \newline
67. ना॒क॒सद॑ उप॒दधा᳚ त्युप॒दधा॑ति नाक॒सदो॑ नाक॒सद॑ उप॒दधा॑ति । \newline
68. ना॒क॒सद॒ इति॑ नाक - सदः॑ । \newline

\textbf{Ghana Paata } \newline

1. ना॒क॒सद्भि॒र् वै वै ना॑क॒सद्भि॑र् नाक॒सद्भि॒र् वै दे॒वा दे॒वा वै ना॑क॒सद्भि॑र् नाक॒सद्भि॒र् वै दे॒वाः । \newline
2. ना॒क॒सद्भि॒रिति॑ नाक॒सत् - भिः॒ । \newline
3. वै दे॒वा दे॒वा वै वै दे॒वाः सु॑व॒र्गꣳ सु॑व॒र्गम् दे॒वा वै वै दे॒वाः सु॑व॒र्गम् । \newline
4. दे॒वाः सु॑व॒र्गꣳ सु॑व॒र्गम् दे॒वा दे॒वाः सु॑व॒र्गम् ॅलो॒कम् ॅलो॒कꣳ सु॑व॒र्गम् दे॒वा दे॒वाः सु॑व॒र्गम् ॅलो॒कम् । \newline
5. सु॒व॒र्गम् ॅलो॒कम् ॅलो॒कꣳ सु॑व॒र्गꣳ सु॑व॒र्गम् ॅलो॒क मा॑यन् नायन् ॅलो॒कꣳ सु॑व॒र्गꣳ सु॑व॒र्गम् ॅलो॒क मा॑यन्न् । \newline
6. सु॒व॒र्गमिति॑ सुवः - गम् । \newline
7. लो॒क मा॑यन् नायन् ॅलो॒कम् ॅलो॒क मा॑य॒न् तत् तदा॑यन् ॅलो॒कम् ॅलो॒क मा॑य॒न् तत् । \newline
8. आ॒य॒न् तत् तदा॑यन् नाय॒न् तन् ना॑क॒सदा᳚म् नाक॒सदा॒म् तदा॑यन् नाय॒न् तन् ना॑क॒सदा᳚म् । \newline
9. तन् ना॑क॒सदा᳚म् नाक॒सदा॒म् तत् तन् ना॑क॒सदा᳚म् नाकस॒त्त्वम् ना॑कस॒त्त्वम् ना॑क॒सदा॒म् तत् तन् ना॑क॒सदा᳚म् नाकस॒त्त्वम् । \newline
10. ना॒क॒सदा᳚म् नाकस॒त्त्वम् ना॑कस॒त्त्वम् ना॑क॒सदा᳚म् नाक॒सदा᳚म् नाकस॒त्त्वं ॅयद् यन् ना॑कस॒त्त्वम् ना॑क॒सदा᳚म् नाक॒सदा᳚म् नाकस॒त्त्वं ॅयत् । \newline
11. ना॒क॒सदा॒मिति॑ नाक - सदा᳚म् । \newline
12. ना॒क॒स॒त्त्वं ॅयद् यन् ना॑कस॒त्त्वम् ना॑कस॒त्त्वं ॅयन् ना॑क॒सदो॑ नाक॒सदो॒ यन् ना॑कस॒त्त्वम् ना॑कस॒त्त्वं ॅयन् ना॑क॒सदः॑ । \newline
13. ना॒क॒स॒त्त्वमिति॑ नाकसत् - त्वम् । \newline
14. यन् ना॑क॒सदो॑ नाक॒सदो॒ यद् यन् ना॑क॒सद॑ उप॒दधा᳚ त्युप॒दधा॑ति नाक॒सदो॒ यद् यन् ना॑क॒सद॑ उप॒दधा॑ति । \newline
15. ना॒क॒सद॑ उप॒दधा᳚ त्युप॒दधा॑ति नाक॒सदो॑ नाक॒सद॑ उप॒दधा॑ति नाक॒सद्भि॑र् नाक॒सद्भि॑ रुप॒दधा॑ति नाक॒सदो॑ नाक॒सद॑ उप॒दधा॑ति नाक॒सद्भिः॑ । \newline
16. ना॒क॒सद॒ इति॑ नाक - सदः॑ । \newline
17. उ॒प॒दधा॑ति नाक॒सद्भि॑र् नाक॒सद्भि॑ रुप॒दधा᳚ त्युप॒दधा॑ति नाक॒सद्भि॑ रे॒वैव ना॑क॒सद्भि॑ रुप॒दधा᳚ त्युप॒दधा॑ति नाक॒सद्भि॑ रे॒व । \newline
18. उ॒प॒दधा॒तीत्यु॑प - दधा॑ति । \newline
19. ना॒क॒सद्भि॑ रे॒वैव ना॑क॒सद्भि॑र् नाक॒सद्भि॑ रे॒व तत् तदे॒व ना॑क॒सद्भि॑र् नाक॒सद्भि॑ रे॒व तत् । \newline
20. ना॒क॒सद्भि॒रिति॑ नाक॒सत् - भिः॒ । \newline
21. ए॒व तत् तदे॒ वैव तद् यज॑मानो॒ यज॑मान॒ स्तदे॒वैव तद् यज॑मानः । \newline
22. तद् यज॑मानो॒ यज॑मान॒ स्तत् तद् यज॑मानः सुव॒र्गꣳ सु॑व॒र्गं ॅयज॑मान॒ स्तत् तद् यज॑मानः सुव॒र्गम् । \newline
23. यज॑मानः सुव॒र्गꣳ सु॑व॒र्गं ॅयज॑मानो॒ यज॑मानः सुव॒र्गम् ॅलो॒कम् ॅलो॒कꣳ सु॑व॒र्गं ॅयज॑मानो॒ यज॑मानः सुव॒र्गम् ॅलो॒कम् । \newline
24. सु॒व॒र्गम् ॅलो॒कम् ॅलो॒कꣳ सु॑व॒र्गꣳ सु॑व॒र्गम् ॅलो॒क मे᳚त्येति लो॒कꣳ सु॑व॒र्गꣳ सु॑व॒र्गम् ॅलो॒क मे॑ति । \newline
25. सु॒व॒र्गमिति॑ सुवः - गम् । \newline
26. लो॒क मे᳚त्येति लो॒कम् ॅलो॒क मे॑ति सुव॒र्गः सु॑व॒र्ग ए॑ति लो॒कम् ॅलो॒क मे॑ति सुव॒र्गः । \newline
27. ए॒ति॒ सु॒व॒र्गः सु॑व॒र्ग ए᳚त्येति सुव॒र्गो वै वै सु॑व॒र्ग ए᳚त्येति सुव॒र्गो वै । \newline
28. सु॒व॒र्गो वै वै सु॑व॒र्गः सु॑व॒र्गो वै लो॒को लो॒को वै सु॑व॒र्गः सु॑व॒र्गो वै लो॒कः । \newline
29. सु॒व॒र्ग इति॑ सुवः - गः । \newline
30. वै लो॒को लो॒को वै वै लो॒को नाको॒ नाको॑ लो॒को वै वै लो॒को नाकः॑ । \newline
31. लो॒को नाको॒ नाको॑ लो॒को लो॒को नाको॒ यस्य॒ यस्य॒ नाको॑ लो॒को लो॒को नाको॒ यस्य॑ । \newline
32. नाको॒ यस्य॒ यस्य॒ नाको॒ नाको॒ यस्यै॒ता ए॒ता यस्य॒ नाको॒ नाको॒ यस्यै॒ताः । \newline
33. यस्यै॒ता ए॒ता यस्य॒ यस्यै॒ता उ॑पधी॒यन्त॑ उपधी॒यन्त॑ ए॒ता यस्य॒ यस्यै॒ता उ॑पधी॒यन्ते᳚ । \newline
34. ए॒ता उ॑पधी॒यन्त॑ उपधी॒यन्त॑ ए॒ता ए॒ता उ॑पधी॒यन्ते॒ न नोप॑धी॒यन्त॑ ए॒ता ए॒ता उ॑पधी॒यन्ते॒ न । \newline
35. उ॒प॒धी॒यन्ते॒ न नोप॑धी॒यन्त॑ उपधी॒यन्ते॒ नास्मा॑ अस्मै॒ नोप॑धी॒यन्त॑ उपधी॒यन्ते॒ नास्मै᳚ । \newline
36. उ॒प॒धी॒यन्त॒ इत्यु॑प - धी॒यन्ते᳚ । \newline
37. नास्मा॑ अस्मै॒ न नास्मा॒ अक॒ मक॑ मस्मै॒ न नास्मा॒ अक᳚म् । \newline
38. अ॒स्मा॒ अक॒ मक॑ मस्मा अस्मा॒ अक॑म् भवति भव॒ त्यक॑ मस्मा अस्मा॒ अक॑म् भवति । \newline
39. अक॑म् भवति भव॒ त्यक॒ मक॑म् भवति यजमानायत॒नं ॅय॑जमानायत॒नम् भ॑व॒ त्यक॒ मक॑म् भवति यजमानायत॒नम् । \newline
40. भ॒व॒ति॒ य॒ज॒मा॒ना॒य॒त॒नं ॅय॑जमानायत॒नम् भ॑वति भवति यजमानायत॒नं ॅवै वै य॑जमानायत॒नम् भ॑वति भवति यजमानायत॒नं ॅवै । \newline
41. य॒ज॒मा॒ना॒य॒त॒नं ॅवै वै य॑जमानायत॒नं ॅय॑जमानायत॒नं ॅवै ना॑क॒सदो॑ नाक॒सदो॒ वै य॑जमानायत॒नं ॅय॑जमानायत॒नं ॅवै ना॑क॒सदः॑ । \newline
42. य॒ज॒मा॒ना॒य॒त॒नमिति॑ यजमान - आ॒य॒त॒नम् । \newline
43. वै ना॑क॒सदो॑ नाक॒सदो॒ वै वै ना॑क॒सदो॒ यद् यन् ना॑क॒सदो॒ वै वै ना॑क॒सदो॒ यत् । \newline
44. ना॒क॒सदो॒ यद् यन् ना॑क॒सदो॑ नाक॒सदो॒ यन् ना॑क॒सदो॑ नाक॒सदो॒ यन् ना॑क॒सदो॑ नाक॒सदो॒ यन् ना॑क॒सदः॑ । \newline
45. ना॒क॒सद॒ इति॑ नाक - सदः॑ । \newline
46. यन् ना॑क॒सदो॑ नाक॒सदो॒ यद् यन् ना॑क॒सद॑ उप॒दधा᳚ त्युप॒दधा॑ति नाक॒सदो॒ यद् यन् ना॑क॒सद॑ उप॒दधा॑ति । \newline
47. ना॒क॒सद॑ उप॒दधा᳚ त्युप॒दधा॑ति नाक॒सदो॑ नाक॒सद॑ उप॒दधा᳚ त्या॒यत॑न मा॒यत॑न मुप॒दधा॑ति नाक॒सदो॑ नाक॒सद॑ उप॒दधा᳚ त्या॒यत॑नम् । \newline
48. ना॒क॒सद॒ इति॑ नाक - सदः॑ । \newline
49. उ॒प॒दधा᳚ त्या॒यत॑न मा॒यत॑न मुप॒दधा᳚ त्युप॒दधा᳚ त्या॒यत॑न मे॒वै वायत॑न मुप॒दधा᳚ त्युप॒दधा᳚ त्या॒यत॑न मे॒व । \newline
50. उ॒प॒दधा॒तीत्यु॑प - दधा॑ति । \newline
51. आ॒यत॑न मे॒वैवा यत॑न मा॒यत॑न मे॒व तत् तदे॒वा यत॑न मा॒यत॑न मे॒व तत् । \newline
52. आ॒यत॑न॒मित्या᳚ - यत॑नम् । \newline
53. ए॒व तत् तदे॒ वैव तद् यज॑मानो॒ यज॑मान॒ स्तदे॒वैव तद् यज॑मानः । \newline
54. तद् यज॑मानो॒ यज॑मान॒ स्तत् तद् यज॑मानः कुरुते कुरुते॒ यज॑मान॒ स्तत् तद् यज॑मानः कुरुते । \newline
55. यज॑मानः कुरुते कुरुते॒ यज॑मानो॒ यज॑मानः कुरुते पृ॒ष्ठाना᳚म् पृ॒ष्ठाना᳚म् कुरुते॒ यज॑मानो॒ यज॑मानः कुरुते पृ॒ष्ठाना᳚म् । \newline
56. कु॒रु॒ते॒ पृ॒ष्ठाना᳚म् पृ॒ष्ठाना᳚म् कुरुते कुरुते पृ॒ष्ठानां॒ ॅवै वै पृ॒ष्ठाना᳚म् कुरुते कुरुते पृ॒ष्ठानां॒ ॅवै । \newline
57. पृ॒ष्ठानां॒ ॅवै वै पृ॒ष्ठाना᳚म् पृ॒ष्ठानां॒ ॅवा ए॒त दे॒तद् वै पृ॒ष्ठाना᳚म् पृ॒ष्ठानां॒ ॅवा ए॒तत् । \newline
58. वा ए॒त दे॒तद् वै वा ए॒तत् तेज॒ स्तेज॑ ए॒तद् वै वा ए॒तत् तेजः॑ । \newline
59. ए॒तत् तेज॒ स्तेज॑ ए॒त दे॒तत् तेजः॒ संभृ॑तꣳ॒॒ संभृ॑त॒म् तेज॑ ए॒त दे॒तत् तेजः॒ संभृ॑तम् । \newline
60. तेजः॒ संभृ॑तꣳ॒॒ संभृ॑त॒म् तेज॒ स्तेजः॒ संभृ॑तं॒ ॅयद् यथ् संभृ॑त॒म् तेज॒ स्तेजः॒ संभृ॑तं॒ ॅयत् । \newline
61. संभृ॑तं॒ ॅयद् यथ् संभृ॑तꣳ॒॒ संभृ॑तं॒ ॅयन् ना॑क॒सदो॑ नाक॒सदो॒ यथ् संभृ॑तꣳ॒॒ संभृ॑तं॒ ॅयन् ना॑क॒सदः॑ । \newline
62. संभृ॑त॒मिति॒ सं - भृ॒त॒म् । \newline
63. यन् ना॑क॒सदो॑ नाक॒सदो॒ यद् यन् ना॑क॒सदो॒ यद् यन् ना॑क॒सदो॒ यद् यन् ना॑क॒सदो॒ यत् । \newline
64. ना॒क॒सदो॒ यद् यन् ना॑क॒सदो॑ नाक॒सदो॒ यन् ना॑क॒सदो॑ नाक॒सदो॒ यन् ना॑क॒सदो॑ नाक॒सदो॒ यन् ना॑क॒सदः॑ । \newline
65. ना॒क॒सद॒ इति॑ नाक - सदः॑ । \newline
66. यन् ना॑क॒सदो॑ नाक॒सदो॒ यद् यन् ना॑क॒सद॑ उप॒दधा᳚ त्युप॒दधा॑ति नाक॒सदो॒ यद् यन् ना॑क॒सद॑ उप॒दधा॑ति । \newline
67. ना॒क॒सद॑ उप॒दधा᳚ त्युप॒दधा॑ति नाक॒सदो॑ नाक॒सद॑ उप॒दधा॑ति पृ॒ष्ठाना᳚म् पृ॒ष्ठाना॑ मुप॒दधा॑ति नाक॒सदो॑ नाक॒सद॑ उप॒दधा॑ति पृ॒ष्ठाना᳚म् । \newline
68. ना॒क॒सद॒ इति॑ नाक - सदः॑ । \newline
\pagebreak
\markright{ TS 5.3.7.2  \hfill https://www.vedavms.in \hfill}

\section{ TS 5.3.7.2 }

\textbf{TS 5.3.7.2 } \newline
\textbf{Samhita Paata} \newline

उप॒दधा॑ति पृ॒ष्ठाना॑मे॒व तेजोऽव॑ रुन्धे पञ्च॒चोडा॒ उप॑ दधात्यफ्स॒रस॑ ए॒वैन॑मे॒ता भू॒ता अ॒मुष्मि॑न् ॅलो॒क उप॑ शे॒रेऽथो॑ तनू॒पानी॑रे॒वैता यज॑मानस्य॒ यं द्वि॒ष्यात् तमु॑प॒दध॑द्ध्यायेदे॒ताभ्य॑ ए॒वैनं॑ दे॒वता᳚भ्य॒ आ वृ॑श्चति ता॒जगार्ति॒मार्च्छ॒त्युत्त॑रा नाक॒सद्भ्य॒ उप॑दधाति॒ यथा॑ जा॒यामा॒नीय॑ गृ॒हेषु॑ निषा॒दय॑ति ता॒दृगे॒व तत् - [  ] \newline

\textbf{Pada Paata} \newline

उ॒प॒दधा॒तीत्यु॑प - दधा॑ति । पृ॒ष्ठाना᳚म् । ए॒व । तेजः॑ । अवेति॑ । रु॒न्धे॒ । प॒ञ्च॒चोडा॒ इति॑ पञ्च - चोडाः᳚ । उपेति॑ । द॒धा॒ति॒ । अ॒फ्स॒रसः॑ । ए॒व । ए॒न॒म् । ए॒ताः । भू॒ताः । अ॒मुष्मिन्न्॑ । लो॒के । उपेति॑ । शे॒रे॒ । अथो॒ इति॑ । त॒नू॒पानी॒रिति॑ तनू - पानीः᳚ । ए॒व । ए॒ताः । यज॑मानस्य । यम् । द्वि॒ष्यात् । तम् । उ॒प॒दध॒दित्यु॑प - दध॑त् । ध्या॒ये॒त् । ए॒ताभ्यः॑ । ए॒व । ए॒न॒म् । दे॒वता᳚भ्यः । एति॑ । वृ॒श्च॒ति॒ । ता॒जक् । आर्ति᳚म् । एति॑ । ऋ॒च्छ॒ति॒ । उत्त॑रा॒ इत्युत् - त॒राः । ना॒क॒सद्भ्य॒ इति॑ नाक॒सत्-भ्यः॒ । उपेति॑ । द॒धा॒ति॒ । यथा᳚ । जा॒याम् । आ॒नीयेत्या᳚ - नीय॑ । गृ॒हेषु॑ । नि॒षा॒दय॒तीति॑ नि - सा॒दय॑ति । ता॒दृक् । ए॒व । तत् ।  \newline


\textbf{Krama Paata} \newline

उ॒प॒दधा॑ति पृ॒ष्ठाना᳚म् । उ॒प॒दधा॒तीत्यु॑प - दधा॑ति । पृ॒ष्ठाना॑मे॒व । ए॒व तेजः॑ । तेजोऽव॑ । अव॑ रुन्धे । रु॒न्धे॒ प॒ञ्च॒चोडाः᳚ । प॒ञ्च॒चोडा॒ उप॑ । प॒ञ्च॒चोडा॒ इति॑ पञ्च - चोडाः᳚ । उप॑ दधाति । द॒धा॒त्य॒फ्स॒रसः॑ । अ॒फ्स॒रस॑ ए॒व । ए॒वैन᳚म् । ए॒न॒मे॒ताः । ए॒ता भू॒ताः । भू॒ता अ॒मुष्मिन्न्॑ । अ॒मुष्मि॑न् ॅलो॒के । लो॒क उप॑ । उप॑ शेरे । शे॒रेऽथो᳚ । अथो॑ तनू॒पानीः᳚ । अथो॒ इत्यथो᳚ । त॒नू॒पानी॑रे॒व । त॒नू॒पानी॒रिति॑ तनू - पानीः᳚ । ए॒वैताः । ए॒ता यज॑मानस्य । यज॑मानस्य॒ यम् । यम् द्वि॒ष्यात् । द्वि॒ष्यात् तम् । तमु॑प॒दध॑त् । उ॒प॒दध॑द् ध्यायेत् । उ॒प॒दध॒दित्यु॑प - दध॑त् । ध्या॒ये॒दे॒ताभ्यः॑ । ए॒ताभ्य॑ ए॒व । ए॒वैन᳚म् । ए॒न॒म् दे॒वता᳚भ्यः । दे॒वता᳚भ्य॒ आ । आ वृ॑श्चति । वृ॒श्च॒ति॒ ता॒जक् । ता॒जगार्ति᳚म् । आर्ति॒मा । आर्च्छ॑ति । ऋ॒च्छ॒त्युत्त॑राः । उत्त॑रा नाक॒सद्भ्यः॑ । उत्त॑रा॒ इत्युत् - त॒राः॒ । ना॒क॒सद्भ्य॒ उप॑ । ना॒क॒सद्भ्य॒ इति॑ नाक॒सत् - भ्यः॒ । उप॑ दधाति । द॒धा॒ति॒ यथा᳚ । यथा॑ जा॒याम् । जा॒यामा॒नीय॑ । आ॒नीय॑ गृ॒हेषु॑ । आ॒नीयेत्या᳚ - नीय॑ । गृ॒हेषु॑ निषा॒दय॑ति । नि॒षा॒दय॑ति ता॒दृक् । नि॒षा॒दय॒तीति॑ नि - सा॒दय॑ति । ता॒दृगे॒व । ए॒व तत् । तत् प॒श्चात् \newline

\textbf{Jatai Paata} \newline

1. उ॒प॒दधा॑ति पृ॒ष्ठाना᳚म् पृ॒ष्ठाना॑ मुप॒दधा᳚ त्युप॒दधा॑ति पृ॒ष्ठाना᳚म् । \newline
2. उ॒प॒दधा॒तीत्यु॑प - दधा॑ति । \newline
3. पृ॒ष्ठाना॑ मे॒वैव पृ॒ष्ठाना᳚म् पृ॒ष्ठाना॑ मे॒व । \newline
4. ए॒व तेज॒ स्तेज॑ ए॒वैव तेजः॑ । \newline
5. तेजो ऽवाव॒ तेज॒ स्तेजो ऽव॑ । \newline
6. अव॑ रुन्धे रु॒न्धे ऽवाव॑ रुन्धे । \newline
7. रु॒न्धे॒ प॒ञ्च॒चोडाः᳚ पञ्च॒चोडा॑ रुन्धे रुन्धे पञ्च॒चोडाः᳚ । \newline
8. प॒ञ्च॒चोडा॒ उपोप॑ पञ्च॒चोडाः᳚ पञ्च॒चोडा॒ उप॑ । \newline
9. प॒ञ्च॒चोडा॒ इति॑ पञ्च - चोडाः᳚ । \newline
10. उप॑ दधाति दधा॒ त्युपोप॑ दधाति । \newline
11. द॒धा॒त्य॒ फ्स॒रसो᳚ ऽफ्स॒रसो॑ दधाति दधात्य फ्स॒रसः॑ । \newline
12. अ॒फ्स॒रस॑ ए॒वैवा फ्स॒रसो᳚ ऽफ्स॒रस॑ ए॒व । \newline
13. ए॒वैन॑ मेन मे॒वै वैन᳚म् । \newline
14. ए॒न॒ मे॒ता ए॒ता ए॑न मेन मे॒ताः । \newline
15. ए॒ता भू॒ता भू॒ता ए॒ता ए॒ता भू॒ताः । \newline
16. भू॒ता अ॒मुष्मि॑न् न॒मुष्मि॑न् भू॒ता भू॒ता अ॒मुष्मिन्न्॑ । \newline
17. अ॒मुष्मि॑न् ॅलो॒के लो॒के॑ ऽमुष्मि॑न् न॒मुष्मि॑न् ॅलो॒के । \newline
18. लो॒क उपोप॑ लो॒के लो॒क उप॑ । \newline
19. उप॑ शेरे शेर॒ उपोप॑ शेरे । \newline
20. शे॒रे ऽथो॒ अथो॑ शेरे शे॒रे ऽथो᳚ । \newline
21. अथो॑ तनू॒पानी᳚ स्तनू॒पानी॒ रथो॒ अथो॑ तनू॒पानीः᳚ । \newline
22. अथो॒ इत्यथो᳚ । \newline
23. त॒नू॒पानी॑ रे॒वैव त॑नू॒पानी᳚ स्तनू॒पानी॑ रे॒व । \newline
24. त॒नू॒पानी॒रिति॑ तनू - पानीः᳚ । \newline
25. ए॒वैता ए॒ता ए॒वैवैताः । \newline
26. ए॒ता यज॑मानस्य॒ यज॑मान स्यै॒ता ए॒ता यज॑मानस्य । \newline
27. यज॑मानस्य॒ यं ॅयं ॅयज॑मानस्य॒ यज॑मानस्य॒ यम् । \newline
28. यम् द्वि॒ष्याद् द्वि॒ष्याद् यं ॅयम् द्वि॒ष्यात् । \newline
29. द्वि॒ष्यात् तम् तम् द्वि॒ष्याद् द्वि॒ष्यात् तम् । \newline
30. त मु॑प॒दध॑ दुप॒दध॒त् तम् त मु॑प॒दध॑त् । \newline
31. उ॒प॒दध॑द् ध्यायेद् ध्याये दुप॒दध॑ दुप॒दध॑द् ध्यायेत् । \newline
32. उ॒प॒दध॒दित्यु॑प - दध॑त् । \newline
33. ध्या॒ये॒ दे॒ताभ्य॑ ए॒ताभ्यो᳚ ध्यायेद् ध्याये दे॒ताभ्यः॑ । \newline
34. ए॒ताभ्य॑ ए॒वै वैताभ्य॑ ए॒ताभ्य॑ ए॒व । \newline
35. ए॒वैन॑ मेन मे॒वै वैन᳚म् । \newline
36. ए॒न॒म् दे॒वता᳚भ्यो दे॒वता᳚भ्य एन मेनम् दे॒वता᳚भ्यः । \newline
37. दे॒वता᳚भ्य॒ आ दे॒वता᳚भ्यो दे॒वता᳚भ्य॒ आ । \newline
38. आ वृ॑श्चति वृश्च॒त्या वृ॑श्चति । \newline
39. वृ॒श्च॒ति॒ ता॒जक् ता॒जग् वृ॑श्चति वृश्चति ता॒जक् । \newline
40. ता॒जगार्ति॒ मार्ति॑म् ता॒जक् ता॒जगार्ति᳚म् । \newline
41. आर्ति॒ मा ऽऽर्ति॒ मार्ति॒ मा । \newline
42. आर्च्छ॑ त्यृच्छ त्यार्च्छति । \newline
43. ऋ॒च्छ॒ त्युत्त॑रा॒ उत्त॑रा ऋच्छ त्यृच्छ॒ त्युत्त॑राः । \newline
44. उत्त॑रा नाक॒सद्भ्यो॑ नाक॒सद्भ्य॒ उत्त॑रा॒ उत्त॑रा नाक॒सद्भ्यः॑ । \newline
45. उत्त॑रा॒ इत्युत् - त॒राः॒ । \newline
46. ना॒क॒सद्भ्य॒ उपोप॑ नाक॒सद्भ्यो॑ नाक॒सद्भ्य॒ उप॑ । \newline
47. ना॒क॒सद्भ्य॒ इति॑ नाक॒सत् - भ्यः॒ । \newline
48. उप॑ दधाति दधा॒ त्युपोप॑ दधाति । \newline
49. द॒धा॒ति॒ यथा॒ यथा॑ दधाति दधाति॒ यथा᳚ । \newline
50. यथा॑ जा॒याम् जा॒यां ॅयथा॒ यथा॑ जा॒याम् । \newline
51. जा॒या मा॒नीया॒नीय॑ जा॒याम् जा॒या मा॒नीय॑ । \newline
52. आ॒नीय॑ गृ॒हेषु॑ गृ॒हे ष्वा॒नीया॒ नीय॑ गृ॒हेषु॑ । \newline
53. आ॒नीयेत्या᳚ - नीय॑ । \newline
54. गृ॒हेषु॑ निषा॒दय॑ति निषा॒दय॑ति गृ॒हेषु॑ गृ॒हेषु॑ निषा॒दय॑ति । \newline
55. नि॒षा॒दय॑ति ता॒दृक् ता॒दृङ् नि॑षा॒दय॑ति निषा॒दय॑ति ता॒दृक् । \newline
56. नि॒षा॒दय॒तीति॑ नि - सा॒दय॑ति । \newline
57. ता॒दृ गे॒वैव ता॒दृक् ता॒दृगे॒व । \newline
58. ए॒व तत् तदे॒ वैव तत् । \newline
59. तत् प॒श्चात् प॒श्चात् तत् तत् प॒श्चात् । \newline

\textbf{Ghana Paata } \newline

1. उ॒प॒दधा॑ति पृ॒ष्ठाना᳚म् पृ॒ष्ठाना॑ मुप॒दधा᳚ त्युप॒दधा॑ति पृ॒ष्ठाना॑ मे॒वैव पृ॒ष्ठाना॑ मुप॒दधा᳚ त्युप॒दधा॑ति पृ॒ष्ठाना॑ मे॒व । \newline
2. उ॒प॒दधा॒तीत्यु॑प - दधा॑ति । \newline
3. पृ॒ष्ठाना॑ मे॒वैव पृ॒ष्ठाना᳚म् पृ॒ष्ठाना॑ मे॒व तेज॒ स्तेज॑ ए॒व पृ॒ष्ठाना᳚म् पृ॒ष्ठाना॑ मे॒व तेजः॑ । \newline
4. ए॒व तेज॒ स्तेज॑ ए॒वैव तेजो ऽवाव॒ तेज॑ ए॒वैव तेजो ऽव॑ । \newline
5. तेजो ऽवाव॒ तेज॒ स्तेजो ऽव॑ रुन्धे रु॒न्धे ऽव॒ तेज॒ स्तेजो ऽव॑ रुन्धे । \newline
6. अव॑ रुन्धे रु॒न्धे ऽवाव॑ रुन्धे पञ्च॒चोडाः᳚ पञ्च॒चोडा॑ रु॒न्धे ऽवाव॑ रुन्धे पञ्च॒चोडाः᳚ । \newline
7. रु॒न्धे॒ प॒ञ्च॒चोडाः᳚ पञ्च॒चोडा॑ रुन्धे रुन्धे पञ्च॒चोडा॒ उपोप॑ पञ्च॒चोडा॑ रुन्धे रुन्धे पञ्च॒चोडा॒ उप॑ । \newline
8. प॒ञ्च॒चोडा॒ उपोप॑ पञ्च॒चोडाः᳚ पञ्च॒चोडा॒ उप॑ दधाति दधा॒ त्युप॑ पञ्च॒चोडाः᳚ पञ्च॒चोडा॒ उप॑ दधाति । \newline
9. प॒ञ्च॒चोडा॒ इति॑ पञ्च - चोडाः᳚ । \newline
10. उप॑ दधाति दधा॒ त्युपोप॑ दधा त्यफ्स॒रसो᳚ ऽफ्स॒रसो॑ दधा॒ त्युपोप॑ दधा त्यफ्स॒रसः॑ । \newline
11. द॒धा॒ त्य॒फ्स॒रसो᳚ ऽफ्स॒रसो॑ दधाति दधा त्यफ्स॒रस॑ ए॒वै वाफ्स॒रसो॑ दधाति दधा त्यफ्स॒रस॑ ए॒व । \newline
12. अ॒फ्स॒रस॑ ए॒वै वाफ्स॒रसो᳚ ऽफ्स॒रस॑ ए॒वैन॑ मेन मे॒वा फ्स॒रसो᳚ ऽफ्स॒रस॑ ए॒वैन᳚म् । \newline
13. ए॒वैन॑ मेन मे॒वै वैन॑ मे॒ता ए॒ता ए॑न मे॒वै वैन॑ मे॒ताः । \newline
14. ए॒न॒ मे॒ता ए॒ता ए॑न मेन मे॒ता भू॒ता भू॒ता ए॒ता ए॑न मेन मे॒ता भू॒ताः । \newline
15. ए॒ता भू॒ता भू॒ता ए॒ता ए॒ता भू॒ता अ॒मुष्मि॑न् न॒मुष्मि॑न् भू॒ता ए॒ता ए॒ता भू॒ता अ॒मुष्मिन्न्॑ । \newline
16. भू॒ता अ॒मुष्मि॑न् न॒मुष्मि॑न् भू॒ता भू॒ता अ॒मुष्मि॑न् ॅलो॒के लो॒के॑ ऽमुष्मि॑न् भू॒ता भू॒ता अ॒मुष्मि॑न् ॅलो॒के । \newline
17. अ॒मुष्मि॑न् ॅलो॒के लो॒के॑ ऽमुष्मि॑न् न॒मुष्मि॑न् ॅलो॒क उपोप॑ लो॒के॑ ऽमुष्मि॑न् न॒मुष्मि॑न् ॅलो॒क उप॑ । \newline
18. लो॒क उपोप॑ लो॒के लो॒क उप॑ शेरे शेर॒ उप॑ लो॒के लो॒क उप॑ शेरे । \newline
19. उप॑ शेरे शेर॒ उपोप॑ शे॒रे ऽथो॒ अथो॑ शेर॒ उपोप॑ शे॒रे ऽथो᳚ । \newline
20. शे॒रे ऽथो॒ अथो॑ शेरे शे॒रे ऽथो॑ तनू॒पानी᳚ स्तनू॒पानी॒ रथो॑ शेरे शे॒रे ऽथो॑ तनू॒पानीः᳚ । \newline
21. अथो॑ तनू॒पानी᳚ स्तनू॒पानी॒ रथो॒ अथो॑ तनू॒पानी॑ रे॒वैव त॑नू॒पानी॒ रथो॒ अथो॑ तनू॒पानी॑ रे॒व । \newline
22. अथो॒ इत्यथो᳚ । \newline
23. त॒नू॒पानी॑ रे॒वैव त॑नू॒पानी᳚ स्तनू॒पानी॑ रे॒वैता ए॒ता ए॒व त॑नू॒पानी᳚ स्तनू॒पानी॑ रे॒वैताः । \newline
24. त॒नू॒पानी॒रिति॑ तनू - पानीः᳚ । \newline
25. ए॒वैता ए॒ता ए॒वै वैता यज॑मानस्य॒ यज॑मान स्यै॒ता ए॒वै वैता यज॑मानस्य । \newline
26. ए॒ता यज॑मानस्य॒ यज॑मान स्यै॒ता ए॒ता यज॑मानस्य॒ यं ॅयं ॅयज॑मान स्यै॒ता ए॒ता यज॑मानस्य॒ यम् । \newline
27. यज॑मानस्य॒ यं ॅयं ॅयज॑मानस्य॒ यज॑मानस्य॒ यम् द्वि॒ष्याद् द्वि॒ष्याद् यं ॅयज॑मानस्य॒ यज॑मानस्य॒ यम् द्वि॒ष्यात् । \newline
28. यम् द्वि॒ष्याद् द्वि॒ष्याद् यं ॅयम् द्वि॒ष्यात् तम् तम् द्वि॒ष्याद् यं ॅयम् द्वि॒ष्यात् तम् । \newline
29. द्वि॒ष्यात् तम् तम् द्वि॒ष्याद् द्वि॒ष्यात् त मु॑प॒दध॑ दुप॒दध॒त् तम् द्वि॒ष्याद् द्वि॒ष्यात् त मु॑प॒दध॑त् । \newline
30. त मु॑प॒दध॑ दुप॒दध॒त् तम् त मु॑प॒दध॑द् ध्यायेद् ध्याये दुप॒दध॒त् तम् त मु॑प॒दध॑द् ध्यायेत् । \newline
31. उ॒प॒दध॑द् ध्यायेद् ध्याये दुप॒दध॑ दुप॒दध॑द् ध्याये दे॒ताभ्य॑ ए॒ताभ्यो᳚ ध्याये दुप॒दध॑ दुप॒दध॑द् ध्याये दे॒ताभ्यः॑ । \newline
32. उ॒प॒दध॒दित्यु॑प - दध॑त् । \newline
33. ध्या॒ये॒ दे॒ताभ्य॑ ए॒ताभ्यो᳚ ध्यायेद् ध्याये दे॒ताभ्य॑ ए॒वैवैताभ्यो᳚ ध्यायेद् ध्याये दे॒ताभ्य॑ ए॒व । \newline
34. ए॒ताभ्य॑ ए॒वैवैताभ्य॑ ए॒ताभ्य॑ ए॒वैन॑ मेन मे॒वै ताभ्य॑ ए॒ताभ्य॑ ए॒वैन᳚म् । \newline
35. ए॒वैन॑ मेन मे॒वै वैन॑म् दे॒वता᳚भ्यो दे॒वता᳚भ्य एन मे॒वै वैन॑म् दे॒वता᳚भ्यः । \newline
36. ए॒न॒म् दे॒वता᳚भ्यो दे॒वता᳚भ्य एन मेनम् दे॒वता᳚भ्य॒ आ दे॒वता᳚भ्य एन मेनम् दे॒वता᳚भ्य॒ आ । \newline
37. दे॒वता᳚भ्य॒ आ दे॒वता᳚भ्यो दे॒वता᳚भ्य॒ आ वृ॑श्चति वृश्च॒त्या दे॒वता᳚भ्यो दे॒वता᳚भ्य॒ आ वृ॑श्चति । \newline
38. आ वृ॑श्चति वृश्च॒त्या वृ॑श्चति ता॒जक् ता॒जग् वृ॑श्च॒त्या वृ॑श्चति ता॒जक् । \newline
39. वृ॒श्च॒ति॒ ता॒जक् ता॒जग् वृ॑श्चति वृश्चति ता॒जगार्ति॒ मार्ति॑म् ता॒जग् वृ॑श्चति वृश्चति ता॒जगार्ति᳚म् । \newline
40. ता॒जगार्ति॒ मार्ति॑म् ता॒जक् ता॒जगार्ति॒ मा ऽऽर्ति॑म् ता॒जक् ता॒जगार्ति॒ मा । \newline
41. आर्ति॒ मा ऽऽर्ति॒ मार्ति॒ मार्च्छ॑ त्यृच्छ॒ त्या ऽऽर्ति॒ मार्ति॒ मार्च्छ॑ति । \newline
42. आर्च्छ॑ त्यृच्छ त्यार्च्छ॒ त्युत्त॑रा॒ उत्त॑रा ऋच्छ त्यार्च्छ॒ त्युत्त॑राः । \newline
43. ऋ॒च्छ॒ त्युत्त॑रा॒ उत्त॑रा ऋच्छ त्यृच्छ॒ त्युत्त॑रा नाक॒सद्भ्यो॑ नाक॒सद्भ्य॒ उत्त॑रा ऋच्छ त्यृच्छ॒ त्युत्त॑रा नाक॒सद्भ्यः॑ । \newline
44. उत्त॑रा नाक॒सद्भ्यो॑ नाक॒सद्भ्य॒ उत्त॑रा॒ उत्त॑रा नाक॒सद्भ्य॒ उपोप॑ नाक॒सद्भ्य॒ उत्त॑रा॒ उत्त॑रा नाक॒सद्भ्य॒ उप॑ । \newline
45. उत्त॑रा॒ इत्युत् - त॒राः॒ । \newline
46. ना॒क॒सद्भ्य॒ उपोप॑ नाक॒सद्भ्यो॑ नाक॒सद्भ्य॒ उप॑ दधाति दधा॒ त्युप॑ नाक॒सद्भ्यो॑ नाक॒सद्भ्य॒ उप॑ दधाति । \newline
47. ना॒क॒सद्भ्य॒ इति॑ नाक॒सत् - भ्यः॒ । \newline
48. उप॑ दधाति दधा॒ त्युपोप॑ दधाति॒ यथा॒ यथा॑ दधा॒ त्युपोप॑ दधाति॒ यथा᳚ । \newline
49. द॒धा॒ति॒ यथा॒ यथा॑ दधाति दधाति॒ यथा॑ जा॒याम् जा॒यां ॅयथा॑ दधाति दधाति॒ यथा॑ जा॒याम् । \newline
50. यथा॑ जा॒याम् जा॒यां ॅयथा॒ यथा॑ जा॒या मा॒नीया॒ नीय॑ जा॒यां ॅयथा॒ यथा॑ जा॒या मा॒नीय॑ । \newline
51. जा॒या मा॒नीया॒ नीय॑ जा॒याम् जा॒या मा॒नीय॑ गृ॒हेषु॑ गृ॒हे ष्वा॒नीय॑ जा॒याम् जा॒या मा॒नीय॑ गृ॒हेषु॑ । \newline
52. आ॒नीय॑ गृ॒हेषु॑ गृ॒हे ष्वा॒नीया॒ नीय॑ गृ॒हेषु॑ निषा॒दय॑ति निषा॒दय॑ति गृ॒हे ष्वा॒नीया॒ नीय॑ गृ॒हेषु॑ निषा॒दय॑ति । \newline
53. आ॒नीयेत्या᳚ - नीय॑ । \newline
54. गृ॒हेषु॑ निषा॒दय॑ति निषा॒दय॑ति गृ॒हेषु॑ गृ॒हेषु॑ निषा॒दय॑ति ता॒दृक् ता॒दृङ् नि॑षा॒दय॑ति गृ॒हेषु॑ गृ॒हेषु॑ निषा॒दय॑ति ता॒दृक् । \newline
55. नि॒षा॒दय॑ति ता॒दृक् ता॒दृङ् नि॑षा॒दय॑ति निषा॒दय॑ति ता॒दृगे॒ वैव ता॒दृङ् नि॑षा॒दय॑ति निषा॒दय॑ति ता॒दृ गे॒व । \newline
56. नि॒षा॒दय॒तीति॑ नि - सा॒दय॑ति । \newline
57. ता॒दृगे॒ वैव ता॒दृक् ता॒दृ गे॒व तत् तदे॒व ता॒दृक् ता॒दृ गे॒व तत् । \newline
58. ए॒व तत् तदे॒ वैव तत् प॒श्चात् प॒श्चात् तदे॒ वैव तत् प॒श्चात् । \newline
59. तत् प॒श्चात् प॒श्चात् तत् तत् प॒श्चात् प्राची॒म् प्राची᳚म् प॒श्चात् तत् तत् प॒श्चात् प्राची᳚म् । \newline
\pagebreak
\markright{ TS 5.3.7.3  \hfill https://www.vedavms.in \hfill}

\section{ TS 5.3.7.3 }

\textbf{TS 5.3.7.3 } \newline
\textbf{Samhita Paata} \newline

प॒श्चात् प्राची॑मुत्त॒मामुप॑ दधाति॒ तस्मा᳚त् प॒श्चात् प्राची॒ पत्न्यन्वा᳚स्ते स्वयमातृ॒ण्णां च॑ विक॒र्णीं चो᳚त्त॒मे उप॑ दधाति प्रा॒णो वै स्व॑यमातृ॒ण्णाऽऽयु॑र्विक॒र्णी प्रा॒णं चै॒वाऽऽ*यु॑श्च प्रा॒णाना॑मुत्त॒मौ ध॑त्ते॒ तस्मा᳚त् प्रा॒णश्चाऽऽ*यु॑श्च प्रा॒णाना॑मुत्त॒मौ नान्यामुत्त॑रा॒मिष्ट॑का॒मुप॑ दद्ध्या॒द्-यद॒न्यामुत्त॑रा॒-मिष्ट॑का-मुपद॒द्ध्यात् प॑शू॒नां - [  ] \newline

\textbf{Pada Paata} \newline

प॒श्चात् । प्राची᳚म् । उ॒त्त॒मामित्यु॑त् - त॒माम् । उपेति॑ । द॒धा॒ति॒ । तस्मा᳚त् । प॒श्चात् । प्राची᳚ । पत्नी᳚ । अन्विति॑ । आ॒स्ते॒ । स्व॒य॒मा॒तृ॒ण्णामिति॑ स्वयं - आ॒तृ॒ण्णाम् । च॒ । वि॒क॒र्णीमिति॑ वि - क॒र्णीम् । च॒ । उ॒त्त॒मे इत्यु॑त् - त॒मे । उपेति॑ । द॒धा॒ति॒ । प्रा॒ण इति॑ प्र - अ॒नः । वै । स्व॒य॒मा॒तृ॒ण्णेति॑ स्वयं - आ॒तृ॒ण्णा । आयुः॑ । वि॒क॒र्णीति॑ वि - क॒र्णी । प्रा॒णमिति॑ प्र - अ॒नम् । च॒ । ए॒व । आयुः॑ । च॒ । प्रा॒णाना॒मिति॑ प्र - अ॒नाना᳚म् । उ॒त्त॒मावित्यु॑त् - त॒मौ । ध॒त्ते॒ । तस्मा᳚त् । प्रा॒ण इति॑ प्र - अ॒नः । च॒ । आयुः॑ । च॒ । प्रा॒णाना॒मिति॑ प्र - अ॒नाना᳚म् । उ॒त्त॒मावित्यु॑त् - त॒मौ । न । अ॒न्याम् । उत्त॑रा॒मित्युत् - त॒रा॒म् । इष्ट॑काम् । उपेति॑ । द॒द्ध्या॒त् । यत् । अ॒न्याम् । उत्त॑रा॒मित्युत् - त॒रा॒म् । इष्ट॑काम् । उ॒प॒द॒द्ध्यादित्यु॑प - द॒द्ध्यात् । प॒शू॒नाम् ।  \newline


\textbf{Krama Paata} \newline

प॒श्चात् प्राची᳚म् । प्राची॑मुत्त॒माम् । उ॒त्त॒मामुप॑ । उ॒त्त॒मामित्यु॑त् - त॒माम् । उप॑ दधाति । द॒धा॒ति॒ तस्मा᳚त् । तस्मा᳚त् प॒श्चात् । प॒श्चात् प्राची᳚ । प्राची॒ पत्नी᳚ । पत्न्यनु॑ । अन्वा᳚स्ते । आ॒स्ते॒ स्व॒य॒मा॒तृ॒ण्णाम् । स्व॒य॒मा॒तृ॒ण्णाम् च॑ । स्व॒य॒मा॒तृ॒ण्णामिति॑ स्वयम् - आ॒तृ॒ण्णाम् । च॒ वि॒क॒र्णीम् । वि॒क॒र्णीम् च॑ । वि॒क॒र्णीमिति॑ वि - क॒र्णीम् । चो॒त्त॒मे । उ॒त्त॒मे उप॑ । उ॒त्त॒मे इत्यु॑त् - त॒मे । उप॑ दधाति । द॒धा॒ति॒ प्रा॒णः । प्रा॒णो वै । प्रा॒ण इति॑ प्र - अ॒नः । वै स्व॑यमातृ॒ण्णा । स्व॒य॒मा॒तृ॒ण्णाऽऽयुः॑ । स्व॒य॒मा॒तृ॒ण्णेति॑ स्वयम् - आ॒तृ॒ण्णा । आयु॑र् विक॒र्णी । वि॒क॒र्णी प्रा॒णम् । वि॒क॒र्णीति॑ वि - क॒र्णी । प्रा॒णम् च॑ । प्रा॒णमिति॑ प्र - अ॒नम् । चै॒व । ए॒वायुः॑ । आयु॑श्च । च॒ प्रा॒णाना᳚म् । प्रा॒णाना॑मुत्त॒मौ । प्रा॒णाना॒मिति॑ प्र - अ॒नाना᳚म् । उ॒त्त॒मौ ध॑त्ते । उ॒त्त॒मावित्यु॑त् - त॒मौ । ध॒त्ते॒ तस्मा᳚त् । तस्मा᳚त् प्रा॒णः । प्रा॒णश्च॑ । प्रा॒ण इति॑ प्र - अ॒नः । चायुः॑ । आयु॑श्च । च॒ प्रा॒णाना᳚म् । प्रा॒णाना॑मुत्त॒मौ । प्रा॒णाना॒मिति॑ प्र - अ॒नाना᳚म् । उ॒त्त॒मौ न । उ॒त्त॒मावित्यु॑त् - त॒मौ । नान्याम् । अ॒न्यामुत्त॑राम् । उत्त॑रा॒मिष्ट॑काम् । उत्त॑रा॒मित्युत् - त॒रा॒म् । इष्ट॑का॒मुप॑ । उप॑ दद्ध्यात् । द॒द्ध्या॒द् यत् । यद॒न्याम् । अ॒न्यामुत्त॑राम् । उत्त॑रा॒मिष्ट॑काम् । उत्त॑रा॒मित्युत् - त॒रा॒म् । इष्ट॑कामुपद॒द्ध्यात् । उ॒प॒द॒द्ध्यात् प॑शू॒नाम् । उ॒प॒द॒द्ध्यादित्यु॑प - द॒द्ध्यात् । प॒शू॒नाम् च॑ \newline

\textbf{Jatai Paata} \newline

1. प॒श्चात् प्राची॒म् प्राची᳚म् प॒श्चात् प॒श्चात् प्राची᳚म् । \newline
2. प्राची॑ मुत्त॒मा मु॑त्त॒माम् प्राची॒म् प्राची॑ मुत्त॒माम् । \newline
3. उ॒त्त॒मा मुपोपो᳚ त्त॒मा मु॑त्त॒मा मुप॑ । \newline
4. उ॒त्त॒मामित्यु॑त् - त॒माम् । \newline
5. उप॑ दधाति दधा॒ त्युपोप॑ दधाति । \newline
6. द॒धा॒ति॒ तस्मा॒त् तस्मा᳚द् दधाति दधाति॒ तस्मा᳚त् । \newline
7. तस्मा᳚त् प॒श्चात् प॒श्चात् तस्मा॒त् तस्मा᳚त् प॒श्चात् । \newline
8. प॒श्चात् प्राची॒ प्राची॑ प॒श्चात् प॒श्चात् प्राची᳚ । \newline
9. प्राची॒ पत्नी॒ पत्नी॒ प्राची॒ प्राची॒ पत्नी᳚ । \newline
10. पत्न्य न्वनु॒ पत्नी॒ पत्न्यनु॑ । \newline
11. अन्वा᳚स्त आस्ते॒ ऽन्वन् वा᳚स्ते । \newline
12. आ॒स्ते॒ स्व॒य॒मा॒तृ॒ण्णाꣳ स्व॑यमातृ॒ण्णा मा᳚स्त आस्ते स्वयमातृ॒ण्णाम् । \newline
13. स्व॒य॒मा॒तृ॒ण्णाम् च॑ च स्वयमातृ॒ण्णाꣳ स्व॑यमातृ॒ण्णाम् च॑ । \newline
14. स्व॒य॒मा॒तृ॒ण्णामिति॑ स्वयं - आ॒तृ॒ण्णाम् । \newline
15. च॒ वि॒क॒र्णीं ॅवि॑क॒र्णीम् च॑ च विक॒र्णीम् । \newline
16. वि॒क॒र्णीम् च॑ च विक॒र्णीं ॅवि॑क॒र्णीम् च॑ । \newline
17. वि॒क॒र्णीमिति॑ वि - क॒र्णीम् । \newline
18. चो॒त्त॒मे उ॑त्त॒मे च॑ चोत्त॒मे । \newline
19. उ॒त्त॒मे उपोपो᳚ त्त॒मे उ॑त्त॒मे उप॑ । \newline
20. उ॒त्त॒मे इत्यु॑त् - त॒मे । \newline
21. उप॑ दधाति दधा॒ त्युपोप॑ दधाति । \newline
22. द॒धा॒ति॒ प्रा॒णः प्रा॒णो द॑धाति दधाति प्रा॒णः । \newline
23. प्रा॒णो वै वै प्रा॒णः प्रा॒णो वै । \newline
24. प्रा॒ण इति॑ प्र - अ॒नः । \newline
25. वै स्व॑यमातृ॒ण्णा स्व॑यमातृ॒ण्णा वै वै स्व॑यमातृ॒ण्णा । \newline
26. स्व॒य॒मा॒तृ॒ण्णा ऽऽयु॒ रायुः॑ स्वयमातृ॒ण्णा स्व॑यमातृ॒ण्णा ऽऽयुः॑ । \newline
27. स्व॒य॒मा॒तृ॒ण्णेति॑ स्वयं - आ॒तृ॒ण्णा । \newline
28. आयु॑र् विक॒र्णी वि॑क॒र्ण्यायु॒ रायु॑र् विक॒र्णी । \newline
29. वि॒क॒र्णी प्रा॒णम् प्रा॒णं ॅवि॑क॒र्णी वि॑क॒र्णी प्रा॒णम् । \newline
30. वि॒क॒र्णीति॑ वि - क॒र्णी । \newline
31. प्रा॒णम् च॑ च प्रा॒णम् प्रा॒णम् च॑ । \newline
32. प्रा॒णमिति॑ प्र - अ॒नम् । \newline
33. चै॒वैव च॑ चै॒व । \newline
34. ए॒वायु॒ रायु॑ रे॒वै वायुः॑ । \newline
35. आयु॑श्च॒ चायु॒ रायु॑श्च । \newline
36. च॒ प्रा॒णाना᳚म् प्रा॒णाना᳚म् च च प्रा॒णाना᳚म् । \newline
37. प्रा॒णाना॑ मुत्त॒मा वु॑त्त॒मौ प्रा॒णाना᳚म् प्रा॒णाना॑ मुत्त॒मौ । \newline
38. प्रा॒णाना॒मिति॑ प्र - अ॒नाना᳚म् । \newline
39. उ॒त्त॒मौ ध॑त्ते धत्त उत्त॒मा वु॑त्त॒मौ ध॑त्ते । \newline
40. उ॒त्त॒मावित्यु॑त् - त॒मौ । \newline
41. ध॒त्ते॒ तस्मा॒त् तस्मा᳚द् धत्ते धत्ते॒ तस्मा᳚त् । \newline
42. तस्मा᳚त् प्रा॒णः प्रा॒ण स्तस्मा॒त् तस्मा᳚त् प्रा॒णः । \newline
43. प्रा॒णश्च॑ च प्रा॒णः प्रा॒णश्च॑ । \newline
44. प्रा॒ण इति॑ प्र - अ॒नः । \newline
45. चायु॒ रायु॑श्च॒ चायुः॑ । \newline
46. आयु॑श्च॒ चायु॒ रायु॑श्च । \newline
47. च॒ प्रा॒णाना᳚म् प्रा॒णाना᳚म् च च प्रा॒णाना᳚म् । \newline
48. प्रा॒णाना॑ मुत्त॒मा वु॑त्त॒मौ प्रा॒णाना᳚म् प्रा॒णाना॑ मुत्त॒मौ । \newline
49. प्रा॒णाना॒मिति॑ प्र - अ॒नाना᳚म् । \newline
50. उ॒त्त॒मौ न नोत्त॒मा वु॑त्त॒मौ न । \newline
51. उ॒त्त॒मावित्यु॑त् - त॒मौ । \newline
52. नान्या म॒न्याम् न नान्याम् । \newline
53. अ॒न्या मुत्त॑रा॒ मुत्त॑रा म॒न्या म॒न्या मुत्त॑राम् । \newline
54. उत्त॑रा॒ मिष्ट॑का॒ मिष्ट॑का॒ मुत्त॑रा॒ मुत्त॑रा॒ मिष्ट॑काम् । \newline
55. उत्त॑रा॒मित्युत् - त॒रा॒म् । \newline
56. इष्ट॑का॒ मुपोपे ष्ट॑का॒ मिष्ट॑का॒ मुप॑ । \newline
57. उप॑ दद्ध्याद् दद्ध्या॒ दुपोप॑ दद्ध्यात् । \newline
58. द॒द्ध्या॒द् यद् यद् द॑द्ध्याद् दद्ध्या॒द् यत् । \newline
59. यद॒न्या म॒न्यां ॅयद् यद॒न्याम् । \newline
60. अ॒न्या मुत्त॑रा॒ मुत्त॑रा म॒न्या म॒न्या मुत्त॑राम् । \newline
61. उत्त॑रा॒ मिष्ट॑का॒ मिष्ट॑का॒ मुत्त॑रा॒ मुत्त॑रा॒ मिष्ट॑काम् । \newline
62. उत्त॑रा॒मित्युत् - त॒रा॒म् । \newline
63. इष्ट॑का मुपद॒द्ध्या दु॑पद॒द्ध्या दिष्ट॑का॒ मिष्ट॑का मुपद॒द्ध्यात् । \newline
64. उ॒प॒द॒द्ध्यात् प॑शू॒नाम् प॑शू॒ना मु॑पद॒द्ध्या दु॑पद॒द्ध्यात् प॑शू॒नाम् । \newline
65. उ॒प॒द॒द्ध्यादित्यु॑प - द॒द्ध्यात् । \newline
66. प॒शू॒नाम् च॑ च पशू॒नाम् प॑शू॒नाम् च॑ । \newline

\textbf{Ghana Paata } \newline

1. प॒श्चात् प्राची॒म् प्राची᳚म् प॒श्चात् प॒श्चात् प्राची॑ मुत्त॒मा मु॑त्त॒माम् प्राची᳚म् प॒श्चात् प॒श्चात् प्राची॑ मुत्त॒माम् । \newline
2. प्राची॑ मुत्त॒मा मु॑त्त॒माम् प्राची॒म् प्राची॑ मुत्त॒मा मुपोपो᳚ त्त॒माम् प्राची॒म् प्राची॑ मुत्त॒मा मुप॑ । \newline
3. उ॒त्त॒मा मुपोपो᳚ त्त॒मा मु॑त्त॒मा मुप॑ दधाति दधा॒ त्युपो᳚त्त॒मा मु॑त्त॒मा मुप॑ दधाति । \newline
4. उ॒त्त॒मामित्यु॑त् - त॒माम् । \newline
5. उप॑ दधाति दधा॒ त्युपोप॑ दधाति॒ तस्मा॒त् तस्मा᳚द् दधा॒ त्युपोप॑ दधाति॒ तस्मा᳚त् । \newline
6. द॒धा॒ति॒ तस्मा॒त् तस्मा᳚द् दधाति दधाति॒ तस्मा᳚त् प॒श्चात् प॒श्चात् तस्मा᳚द् दधाति दधाति॒ तस्मा᳚त् प॒श्चात् । \newline
7. तस्मा᳚त् प॒श्चात् प॒श्चात् तस्मा॒त् तस्मा᳚त् प॒श्चात् प्राची॒ प्राची॑ प॒श्चात् तस्मा॒त् तस्मा᳚त् प॒श्चात् प्राची᳚ । \newline
8. प॒श्चात् प्राची॒ प्राची॑ प॒श्चात् प॒श्चात् प्राची॒ पत्नी॒ पत्नी॒ प्राची॑ प॒श्चात् प॒श्चात् प्राची॒ पत्नी᳚ । \newline
9. प्राची॒ पत्नी॒ पत्नी॒ प्राची॒ प्राची॒ पत्न्यन्वनु॒ पत्नी॒ प्राची॒ प्राची॒ पत्न्यनु॑ । \newline
10. पत्न्यन्वनु॒ पत्नी॒ पत्न्यन् वा᳚स्त आस्ते ऽनु॒ पत्नी॒ पत्न्यन् वा᳚स्ते । \newline
11. अन्वा᳚स्त आस्ते॒ ऽन्वन् वा᳚स्ते स्वयमातृ॒ण्णाꣳ स्व॑यमातृ॒ण्णा मा᳚स्ते॒ ऽन्वन् वा᳚स्ते स्वयमातृ॒ण्णाम् । \newline
12. आ॒स्ते॒ स्व॒य॒मा॒तृ॒ण्णाꣳ स्व॑यमातृ॒ण्णा मा᳚स्त आस्ते स्वयमातृ॒ण्णाम् च॑ च स्वयमातृ॒ण्णा मा᳚स्त आस्ते स्वयमातृ॒ण्णाम् च॑ । \newline
13. स्व॒य॒मा॒तृ॒ण्णाम् च॑ च स्वयमातृ॒ण्णाꣳ स्व॑यमातृ॒ण्णाम् च॑ विक॒र्णीं ॅवि॑क॒र्णीम् च॑ स्वयमातृ॒ण्णाꣳ स्व॑यमातृ॒ण्णाम् च॑ विक॒र्णीम् । \newline
14. स्व॒य॒मा॒तृ॒ण्णामिति॑ स्वयं - आ॒तृ॒ण्णाम् । \newline
15. च॒ वि॒क॒र्णीं ॅवि॑क॒र्णीम् च॑ च विक॒र्णीम् च॑ च विक॒र्णीम् च॑ च विक॒र्णीम् च॑ । \newline
16. वि॒क॒र्णीम् च॑ च विक॒र्णीं ॅवि॑क॒र्णीम् चो᳚त्त॒मे उ॑त्त॒मे च॑ विक॒र्णीं ॅवि॑क॒र्णीम् चो᳚त्त॒मे । \newline
17. वि॒क॒र्णीमिति॑ वि - क॒र्णीम् । \newline
18. चो॒त्त॒मे उ॑त्त॒मे च॑ चोत्त॒मे उपोपो᳚त्त॒मे च॑ चोत्त॒मे उप॑ । \newline
19. उ॒त्त॒मे उपोपो᳚त्त॒मे उ॑त्त॒मे उप॑ दधाति दधा॒ त्युपो᳚त्त॒मे उ॑त्त॒मे उप॑ दधाति । \newline
20. उ॒त्त॒मे इत्यु॑त् - त॒मे । \newline
21. उप॑ दधाति दधा॒ त्युपोप॑ दधाति प्रा॒णः प्रा॒णो द॑धा॒ त्युपोप॑ दधाति प्रा॒णः । \newline
22. द॒धा॒ति॒ प्रा॒णः प्रा॒णो द॑धाति दधाति प्रा॒णो वै वै प्रा॒णो द॑धाति दधाति प्रा॒णो वै । \newline
23. प्रा॒णो वै वै प्रा॒णः प्रा॒णो वै स्व॑यमातृ॒ण्णा स्व॑यमातृ॒ण्णा वै प्रा॒णः प्रा॒णो वै स्व॑यमातृ॒ण्णा । \newline
24. प्रा॒ण इति॑ प्र - अ॒नः । \newline
25. वै स्व॑यमातृ॒ण्णा स्व॑यमातृ॒ण्णा वै वै स्व॑यमातृ॒ण्णा ऽऽयु॒ रायुः॑ स्वयमातृ॒ण्णा वै वै स्व॑यमातृ॒ण्णा ऽऽयुः॑ । \newline
26. स्व॒य॒मा॒तृ॒ण्णा ऽऽयु॒ रायुः॑ स्वयमातृ॒ण्णा स्व॑यमातृ॒ण्णा ऽऽयु॑र् विक॒र्णी वि॑क॒र्ण्यायुः॑ स्वयमातृ॒ण्णा स्व॑यमातृ॒ण्णा ऽऽयु॑र् विक॒र्णी । \newline
27. स्व॒य॒मा॒तृ॒ण्णेति॑ स्वयं - आ॒तृ॒ण्णा । \newline
28. आयु॑र् विक॒र्णी वि॑क॒र्ण्या यु॒रायु॑र् विक॒र्णी प्रा॒णम् प्रा॒णं ॅवि॑क॒र्ण्या यु॒रायु॑र् विक॒र्णी प्रा॒णम् । \newline
29. वि॒क॒र्णी प्रा॒णम् प्रा॒णं ॅवि॑क॒र्णी वि॑क॒र्णी प्रा॒णम् च॑ च प्रा॒णं ॅवि॑क॒र्णी वि॑क॒र्णी प्रा॒णम् च॑ । \newline
30. वि॒क॒र्णीति॑ वि - क॒र्णी । \newline
31. प्रा॒णम् च॑ च प्रा॒णम् प्रा॒णम् चै॒वैव च॑ प्रा॒णम् प्रा॒णम् चै॒व । \newline
32. प्रा॒णमिति॑ प्र - अ॒नम् । \newline
33. चै॒वैव च॑ चै॒वायु॒ रायु॑ रे॒व च॑ चै॒वायुः॑ । \newline
34. ए॒वायु॒ रायु॑ रे॒वैवायु॑ श्च॒ चायु॑ रे॒वै वायु॑श्च । \newline
35. आयु॑श्च॒ चायु॒ रायु॑श्च प्रा॒णाना᳚म् प्रा॒णाना॒म् चायु॒ रायु॑श्च प्रा॒णाना᳚म् । \newline
36. च॒ प्रा॒णाना᳚म् प्रा॒णाना᳚म् च च प्रा॒णाना॑ मुत्त॒मा वु॑त्त॒मौ प्रा॒णाना᳚म् च च प्रा॒णाना॑ मुत्त॒मौ । \newline
37. प्रा॒णाना॑ मुत्त॒मा वु॑त्त॒मौ प्रा॒णाना᳚म् प्रा॒णाना॑ मुत्त॒मौ ध॑त्ते धत्त उत्त॒मौ प्रा॒णाना᳚म् प्रा॒णाना॑ मुत्त॒मौ ध॑त्ते । \newline
38. प्रा॒णाना॒मिति॑ प्र - अ॒नाना᳚म् । \newline
39. उ॒त्त॒मौ ध॑त्ते धत्त उत्त॒मा वु॑त्त॒मौ ध॑त्ते॒ तस्मा॒त् तस्मा᳚द् धत्त उत्त॒मा वु॑त्त॒मौ ध॑त्ते॒ तस्मा᳚त् । \newline
40. उ॒त्त॒मावित्यु॑त् - त॒मौ । \newline
41. ध॒त्ते॒ तस्मा॒त् तस्मा᳚द् धत्ते धत्ते॒ तस्मा᳚त् प्रा॒णः प्रा॒ण स्तस्मा᳚द् धत्ते धत्ते॒ तस्मा᳚त् प्रा॒णः । \newline
42. तस्मा᳚त् प्रा॒णः प्रा॒ण स्तस्मा॒त् तस्मा᳚त् प्रा॒णश्च॑ च प्रा॒ण स्तस्मा॒त् तस्मा᳚त् प्रा॒णश्च॑ । \newline
43. प्रा॒णश्च॑ च प्रा॒णः प्रा॒ण श्चायु॒ रायु॑श्च प्रा॒णः प्रा॒ण श्चायुः॑ । \newline
44. प्रा॒ण इति॑ प्र - अ॒नः । \newline
45. चायु॒ रायु॑श्च॒ चायु॑श्च॒ चायु॑श्च॒ चायु॑श्च । \newline
46. आयु॑श्च॒ चायु॒ रायु॑श्च प्रा॒णाना᳚म् प्रा॒णाना॒म् चायु॒ रायु॑श्च प्रा॒णाना᳚म् । \newline
47. च॒ प्रा॒णाना᳚म् प्रा॒णाना᳚म् च च प्रा॒णाना॑ मुत्त॒मा वु॑त्त॒मौ प्रा॒णाना᳚म् च च प्रा॒णाना॑ मुत्त॒मौ । \newline
48. प्रा॒णाना॑ मुत्त॒मा वु॑त्त॒मौ प्रा॒णाना᳚म् प्रा॒णाना॑ मुत्त॒मौ न नोत्त॒मौ प्रा॒णाना᳚म् प्रा॒णाना॑ मुत्त॒मौ न । \newline
49. प्रा॒णाना॒मिति॑ प्र - अ॒नाना᳚म् । \newline
50. उ॒त्त॒मौ न नोत्त॒मा वु॑त्त॒मौ नान्या म॒न्यान् नोत्त॒मा वु॑त्त॒मौ नान्याम् । \newline
51. उ॒त्त॒मावित्यु॑त् - त॒मौ । \newline
52. नान्या म॒न्याम् न नान्या मुत्त॑रा॒ मुत्त॑रा म॒न्याम् न नान्या मुत्त॑राम् । \newline
53. अ॒न्या मुत्त॑रा॒ मुत्त॑रा म॒न्या म॒न्या मुत्त॑रा॒ मिष्ट॑का॒ मिष्ट॑का॒ मुत्त॑रा म॒न्या म॒न्या मुत्त॑रा॒ मिष्ट॑काम् । \newline
54. उत्त॑रा॒ मिष्ट॑का॒ मिष्ट॑का॒ मुत्त॑रा॒ मुत्त॑रा॒ मिष्ट॑का॒ मुपोपेष्ट॑का॒ मुत्त॑रा॒ मुत्त॑रा॒ मिष्ट॑का॒ मुप॑ । \newline
55. उत्त॑रा॒मित्युत् - त॒रा॒म् । \newline
56. इष्ट॑का॒ मुपोपेष्ट॑का॒ मिष्ट॑का॒ मुप॑ दद्ध्याद् दद्ध्या॒ दुपेष्ट॑का॒ मिष्ट॑का॒ मुप॑ दद्ध्यात् । \newline
57. उप॑ दद्ध्याद् दद्ध्या॒ दुपोप॑ दद्ध्या॒द् यद् यद् द॑द्ध्या॒ दुपोप॑ दद्ध्या॒द् यत् । \newline
58. द॒द्ध्या॒द् यद् यद् द॑द्ध्याद् दद्ध्या॒द् यद॒न्या म॒न्यां ॅयद् द॑द्ध्याद् दद्ध्या॒द् यद॒न्याम् । \newline
59. यद॒न्या म॒न्यां ॅयद् यद॒न्या मुत्त॑रा॒ मुत्त॑रा म॒न्यां ॅयद् यद॒न्या मुत्त॑राम् । \newline
60. अ॒न्या मुत्त॑रा॒ मुत्त॑रा म॒न्या म॒न्या मुत्त॑रा॒ मिष्ट॑का॒ मिष्ट॑का॒ मुत्त॑रा म॒न्या म॒न्या मुत्त॑रा॒ मिष्ट॑काम् । \newline
61. उत्त॑रा॒ मिष्ट॑का॒ मिष्ट॑का॒ मुत्त॑रा॒ मुत्त॑रा॒ मिष्ट॑का मुपद॒द्ध्या दु॑पद॒द्ध्या दिष्ट॑का॒ मुत्त॑रा॒ मुत्त॑रा॒ मिष्ट॑का मुपद॒द्ध्यात् । \newline
62. उत्त॑रा॒मित्युत् - त॒रा॒म् । \newline
63. इष्ट॑का मुपद॒द्ध्या दु॑पद॒द्ध्या दिष्ट॑का॒ मिष्ट॑का मुपद॒द्ध्यात् प॑शू॒नाम् प॑शू॒ना मु॑पद॒द्ध्या दिष्ट॑का॒ मिष्ट॑का मुपद॒द्ध्यात् प॑शू॒नाम् । \newline
64. उ॒प॒द॒द्ध्यात् प॑शू॒नाम् प॑शू॒ना मु॑पद॒द्ध्या दु॑पद॒द्ध्यात् प॑शू॒नाम् च॑ च पशू॒ना मु॑पद॒द्ध्या दु॑पद॒द्ध्यात् प॑शू॒नाम् च॑ । \newline
65. उ॒प॒द॒द्ध्यादित्यु॑प - द॒द्ध्यात् । \newline
66. प॒शू॒नाम् च॑ च पशू॒नाम् प॑शू॒नाम् च॒ यज॑मानस्य॒ यज॑मानस्य च पशू॒नाम् प॑शू॒नाम् च॒ यज॑मानस्य । \newline
\pagebreak
\markright{ TS 5.3.7.4  \hfill https://www.vedavms.in \hfill}

\section{ TS 5.3.7.4 }

\textbf{TS 5.3.7.4 } \newline
\textbf{Samhita Paata} \newline

च॒ यज॑मानस्य च प्रा॒णं चाऽऽ*यु॒श्चापि॑ दद्ध्या॒त् तस्मा॒न्ना-न्योत्त॒रेष्ट॑कोप॒धेया᳚ स्वयमातृ॒ण्णामुप॑ दधात्य॒सौ वै स्व॑यमातृ॒ण्णा- ऽमूमे॒वोप॑ ध॒त्ते ऽश्व॒मुप॑ घ्रापयति प्रा॒णमे॒वास्यां᳚ दधा॒त्यथो᳚ प्राजाप॒त्यो वा अश्वः॑ प्र॒जाप॑तिनै॒वाग्निं चि॑नुते स्वयमातृ॒ण्णा भ॑वति प्रा॒णाना॒मुथ्सृ॑ष्ट्या॒ अथो॑ सुव॒र्गस्य॑ लो॒कस्याऽनु॑ख्यात्या ए॒षा वै ( ) दे॒वानां॒ ॅविक्रा᳚न्ति॒र्यद्-वि॑क॒र्णी यद्-वि॑क॒र्णीमु॑प॒दधा॑ति दे॒वाना॑मे॒व विक्रा᳚न्ति॒मनु॒ विक्र॑मत उत्तर॒त उप॑दधाति॒ तस्मा॑दुत्तर॒त उ॑पचारो॒ऽग्नि र्वा॑यु॒मती॑ भवति॒ समि॑द्ध्यै ॥ \newline

\textbf{Pada Paata} \newline

च॒ । यज॑मानस्य । च॒ । प्रा॒णमिति॑ प्र - अ॒नम् । च॒ । आयुः॑ । च॒ । अपीति॑ । द॒द्ध्या॒त् । तस्मा᳚त् । न । अ॒न्या । उत्त॒रेत्युत्-त॒रा॒ । इष्ट॑का । उ॒प॒धेयेत्यु॑प - धेया᳚ । स्व॒य॒मा॒तृ॒ण्णामिति॑ स्वयं - आ॒तृ॒ण्णाम् । उपेति॑ । द॒धा॒ति॒ । अ॒सौ । वै । स्व॒य॒मा॒तृ॒ण्णेति॑ स्वयं - आ॒तृ॒ण्णा । अ॒मूम् । ए॒व । उपेति॑ । ध॒त्ते॒ । अश्व᳚म् । उपेति॑ । घ्रा॒प॒य॒ति॒ । प्रा॒णमिति॑ प्र - अ॒नम् । ए॒व । अ॒स्या॒म् । द॒धा॒ति॒ । अथो॒ इति॑ । प्रा॒जा॒प॒त्य इति॑ प्राजा - प॒त्यः । वै । अश्वः॑ । प्र॒जाप॑ति॒नेति॑ प्र॒जा - प॒ति॒ना॒ । ए॒व । अ॒ग्निम् । चि॒नु॒ते॒ । स्व॒य॒मा॒तृ॒ण्णेति॑ स्वयं - आ॒तृ॒ण्णा । भ॒व॒ति॒ । प्रा॒णाना॒मिति॑ प्र - अ॒नाना᳚म् । उथ्सृ॑ष्ट्या॒ इत्युत् - सृ॒ष्ट्यै॒ । अथो॒ इति॑ । सु॒व॒र्गस्येति॑ सुवः - गस्य॑ । लो॒कस्य॑ । अनु॑ख्यात्या॒ इत्यनु॑ - ख्या॒त्यै॒ । ए॒षा । वै ( ) । दे॒वाना᳚म् । विक्रा᳚न्ति॒रिति॒ वि - क्रा॒न्तिः॒ । यत् । वि॒क॒र्णीति॑ वि - क॒र्णी । यत् । वि॒क॒र्णीमिति॑ वि - क॒र्णीम् । उ॒प॒दधा॒तीत्यु॑प - दधा॑ति । दे॒वाना᳚म् । ए॒व । विक्रा᳚न्ति॒मिति॒ वि- क्रा॒न्ति॒म् । अनु॑ । वीति॑ । क्र॒म॒ते॒ । उ॒त्त॒र॒त इत्यु॑त् - त॒र॒तः । उपेति॑ । द॒धा॒ति॒ । तस्मा᳚त् । उ॒त्त॒र॒त उ॑पचार॒ इत्यु॑त्तर॒तः - उ॒प॒चा॒रः॒ । अ॒ग्निः । वा॒यु॒मतीति॑ वायु - मती᳚ । भ॒व॒ति॒ । समि॑द्ध्या॒ इति॒ सं - इ॒द्ध्यै॒ ॥  \newline


\textbf{Krama Paata} \newline

च॒ यज॑मानस्य । यज॑मानस्य च । च॒ प्रा॒णम् । प्रा॒णम् च॑ । प्रा॒णमिति॑ प्र - अ॒नम् । चायुः॑ । आयु॑श्च । चापि॑ । अपि॑ दद्ध्यात् । द॒द्ध्या॒त् तस्मा᳚त् । तस्मा॒न् न । नान्या । अ॒न्योत्त॑रा । उत्त॒रेष्ट॑का । उत्त॒रेत्युत् - त॒रा॒ । इष्ट॑कोप॒धेया᳚ । उ॒प॒धेया᳚ स्वयमातृ॒ण्णाम् । उ॒प॒धेयेत्यु॑प - धेया᳚ । स्व॒य॒मा॒तृ॒ण्णामुप॑ । स्व॒य॒मा॒तृ॒ण्णामिति॑ स्वयम् - आ॒तृ॒ण्णाम् । उप॑ दधाति । द॒धा॒त्य॒सौ । अ॒सौ वै । वै स्व॑यमातृ॒ण्णा । स्व॒य॒मा॒तृ॒ण्णाऽमूम् । स्व॒य॒मा॒तृ॒ण्णेति॑ स्वयम् - आ॒तृ॒ण्णा । अ॒मूमे॒व । ए॒वोप॑ । उप॑ धत्ते । ध॒त्तेऽश्व᳚म् । अश्व॒मुप॑ । उप॑ घ्रापयति । घ्रा॒प॒य॒ति॒ प्रा॒णम् । प्रा॒णमे॒व । प्रा॒णमिति॑ प्र - अ॒नम् । ए॒वास्या᳚म् । अ॒स्या॒म् द॒धा॒ति॒ । द॒धा॒त्यथो᳚ । अथो᳚ प्राजाप॒त्यः । अथो॒ इत्यथो᳚ । प्रा॒जा॒प॒त्यो वै । प्रा॒जा॒प॒त्य इति॑ प्राजा - प॒त्यः । वा अश्वः॑ । अश्वः॑ प्र॒जाप॑तिना । प्र॒जाप॑तिनै॒व । प्र॒जाप॑ति॒नेति॑ प्र॒जा - प॒ति॒ना॒ । ए॒वाग्निम् । अ॒ग्निम् चि॑नुते । चि॒नु॒ते॒ स्व॒य॒मा॒तृ॒ण्णा । स्व॒य॒मा॒तृ॒ण्णा भ॑वति । स्व॒य॒मा॒तृ॒ण्णेति॑ स्वयम् - आ॒तृ॒ण्णा । भ॒व॒ति॒ प्रा॒णाना᳚म् । प्रा॒णाना॒मुथ्सृ॑ष्ट्यै । प्रा॒णाना॒मिति॑ प्र - अ॒नाना᳚म् । उथ्सृ॑ष्ट्या॒ अथो᳚ । उथ्सृ॑ष्ट्या॒ इत्युत् - सृ॒ष्ट्यै॒ । अथो॑ सुव॒र्गस्य॑ । अथो॒ इत्यथो᳚ । सु॒व॒र्गस्य॑ लो॒कस्य॑ । सु॒व॒र्गस्येति॑ सुवः - गस्य॑ । लो॒कस्यानु॑ख्यात्यै । अनु॑ख्यात्या ए॒षा । अनु॑ख्यात्या॒ इत्यनु॑ - ख्या॒त्यै॒ । ए॒षा वै ( ) । वै दे॒वाना᳚म् । दे॒वाना॒म् ॅविक्रा᳚न्तिः । विक्रा᳚न्ति॒र् यत् । विक्रा᳚न्ति॒रिति॒ वि - क्रा॒न्तिः॒ । यद् वि॑क॒र्णी । वि॒क॒र्णी यत् । वि॒क॒र्णीति॑ वि - क॒र्णी । यद् वि॑क॒र्णीम् । वि॒क॒र्णीमु॑प॒दधा॑ति । वि॒क॒र्णीमिति॑ वि - क॒र्णीम् । उ॒प॒दधा॑ति दे॒वाना᳚म् । उ॒प॒दधा॒तीत्यु॑प दधा॑ति । दे॒वाना॑मे॒व । ए॒व विक्रा᳚न्तिम् । विक्रा᳚न्ति॒मनु॑ । विक्रा᳚न्ति॒मिति॒ वि - क्रा॒न्ति॒म् । अनु॒ वि । वि क्र॑मते । क्र॒म॒त॒ उ॒त्त॒र॒तः । उ॒त्त॒र॒त उप॑ । उ॒त्त॒र॒त इत्यु॑त् - त॒र॒तः । उप॑ दधाति । द॒धा॒ति॒ तस्मा᳚त् । तस्मा॑दुत्तर॒तौ॑पचारः । उ॒त्त॒र॒त उ॑पचारो॒ऽग्निः । उ॒त्त॒र॒त उ॑पचार॒ इत्यु॑त्तर॒तः - उ॒प॒चा॒रः॒ । अ॒ग्निर् वा॑यु॒मती᳚ । वा॒यु॒मती॑ भवति । वा॒यु॒मतीति॑ वायु - मती᳚ । भ॒व॒ति॒ समि॑द्ध्यै । समि॑द्ध्या॒ इति॒ सम् - इ॒द्ध्यै॒ । \newline

\textbf{Jatai Paata} \newline

1. च॒ यज॑मानस्य॒ यज॑मानस्य च च॒ यज॑मानस्य । \newline
2. यज॑मानस्य च च॒ यज॑मानस्य॒ यज॑मानस्य च । \newline
3. च॒ प्रा॒णम् प्रा॒णम् च॑ च प्रा॒णम् । \newline
4. प्रा॒णम् च॑ च प्रा॒णम् प्रा॒णम् च॑ । \newline
5. प्रा॒णमिति॑ प्र - अ॒नम् । \newline
6. चायु॒ रायु॑श्च॒ चायुः॑ । \newline
7. आयु॑श्च॒ चायु॒ रायु॑श्च । \newline
8. चाप्यपि॑ च॒ चापि॑ । \newline
9. अपि॑ दद्ध्याद् दद्ध्या॒ दप्यपि॑ दद्ध्यात् । \newline
10. द॒द्ध्या॒त् तस्मा॒त् तस्मा᳚द् दद्ध्याद् दद्ध्या॒त् तस्मा᳚त् । \newline
11. तस्मा॒न् न न तस्मा॒त् तस्मा॒न् न । \newline
12. नान्या ऽन्या न नान्या । \newline
13. अ॒न्यो त्त॒रोत्त॑रा॒ ऽन्या ऽन्योत्त॑रा । \newline
14. उत्त॒रेष्ट॒ केष्ट॒ कोत्त॒ रोत्त॒ रेष्ट॑का । \newline
15. उत्त॒रेत्युत् - त॒रा॒ । \newline
16. इष्ट॑कोप॒धे यो॑प॒धे येष्ट॒ केष्ट॑ कोप॒धेया᳚ । \newline
17. उ॒प॒धेया᳚ स्वयमातृ॒ण्णाꣳ स्व॑यमातृ॒ण्णा मु॑प॒धेयो॑ प॒धेया᳚ स्वयमातृ॒ण्णाम् । \newline
18. उ॒प॒धेयेत्यु॑प - धेया᳚ । \newline
19. स्व॒य॒मा॒तृ॒ण्णा मुपोप॑ स्वयमातृ॒ण्णाꣳ स्व॑यमातृ॒ण्णा मुप॑ । \newline
20. स्व॒य॒मा॒तृ॒ण्णामिति॑ स्वयं - आ॒तृ॒ण्णाम् । \newline
21. उप॑ दधाति दधा॒ त्युपोप॑ दधाति । \newline
22. द॒धा॒ त्य॒सा व॒सौ द॑धाति दधा त्य॒सौ । \newline
23. अ॒सौ वै वा अ॒सा व॒सौ वै । \newline
24. वै स्व॑यमातृ॒ण्णा स्व॑यमातृ॒ण्णा वै वै स्व॑यमातृ॒ण्णा । \newline
25. स्व॒य॒मा॒तृ॒ण्णा ऽमू म॒मूꣳ स्व॑यमातृ॒ण्णा स्व॑यमातृ॒ण्णा ऽमूम् । \newline
26. स्व॒य॒मा॒तृ॒ण्णेति॑ स्वयं - आ॒तृ॒ण्णा । \newline
27. अ॒मू मे॒वै वामू म॒मू मे॒व । \newline
28. ए॒वो पोपै॒ वैवोप॑ । \newline
29. उप॑ धत्ते धत्त॒ उपोप॑ धत्ते । \newline
30. ध॒त्ते ऽश्व॒ मश्व॑म् धत्ते ध॒त्ते ऽश्व᳚म् । \newline
31. अश्व॒ मुपो पाश्व॒ मश्व॒ मुप॑ । \newline
32. उप॑ घ्रापयति घ्रापय॒ त्युपोप॑ घ्रापयति । \newline
33. घ्रा॒प॒य॒ति॒ प्रा॒णम् प्रा॒णम् घ्रा॑पयति घ्रापयति प्रा॒णम् । \newline
34. प्रा॒ण मे॒वैव प्रा॒णम् प्रा॒ण मे॒व । \newline
35. प्रा॒णमिति॑ प्र - अ॒नम् । \newline
36. ए॒वास्या॑ मस्या मे॒वै वास्या᳚म् । \newline
37. अ॒स्या॒म् द॒धा॒ति॒ द॒धा॒ त्य॒स्या॒ म॒स्या॒म् द॒धा॒ति॒ । \newline
38. द॒धा॒ त्यथो॒ अथो॑ दधाति दधा॒ त्यथो᳚ । \newline
39. अथो᳚ प्राजाप॒त्यः प्रा॑जाप॒त्यो ऽथो॒ अथो᳚ प्राजाप॒त्यः । \newline
40. अथो॒ इत्यथो᳚ । \newline
41. प्रा॒जा॒प॒त्यो वै वै प्रा॑जाप॒त्यः प्रा॑जाप॒त्यो वै । \newline
42. प्रा॒जा॒प॒त्य इति॑ प्राजा - प॒त्यः । \newline
43. वा अश्वो ऽश्वो॒ वै वा अश्वः॑ । \newline
44. अश्वः॑ प्र॒जाप॑तिना प्र॒जाप॑ति॒ना ऽश्वो ऽश्वः॑ प्र॒जाप॑तिना । \newline
45. प्र॒जाप॑तिनै॒ वैव प्र॒जाप॑तिना प्र॒जाप॑ति नै॒व । \newline
46. प्र॒जाप॑ति॒नेति॑ प्र॒जा - प॒ति॒ना॒ । \newline
47. ए॒वाग्नि म॒ग्नि मे॒वै वाग्निम् । \newline
48. अ॒ग्निम् चि॑नुते चिनुते॒ ऽग्नि म॒ग्निम् चि॑नुते । \newline
49. चि॒नु॒ते॒ स्व॒य॒मा॒तृ॒ण्णा स्व॑यमातृ॒ण्णा चि॑नुते चिनुते स्वयमातृ॒ण्णा । \newline
50. स्व॒य॒मा॒तृ॒ण्णा भ॑वति भवति स्वयमातृ॒ण्णा स्व॑यमातृ॒ण्णा भ॑वति । \newline
51. स्व॒य॒मा॒तृ॒ण्णेति॑ स्वयं - आ॒तृ॒ण्णा । \newline
52. भ॒व॒ति॒ प्रा॒णाना᳚म् प्रा॒णाना᳚म् भवति भवति प्रा॒णाना᳚म् । \newline
53. प्रा॒णाना॒ मुथ्सृ॑ष्ट्या॒ उथ्सृ॑ष्ट्यै प्रा॒णाना᳚म् प्रा॒णाना॒ मुथ्सृ॑ष्ट्यै । \newline
54. प्रा॒णाना॒मिति॑ प्र - अ॒नाना᳚म् । \newline
55. उथ्सृ॑ष्ट्या॒ अथो॒ अथो॒ उथ्सृ॑ष्ट्या॒ उथ्सृ॑ष्ट्या॒ अथो᳚ । \newline
56. उथ्सृ॑ष्ट्या॒ इत्युत् - सृ॒ष्ट्यै॒ । \newline
57. अथो॑ सुव॒र्गस्य॑ सुव॒र्गस्याथो॒ अथो॑ सुव॒र्गस्य॑ । \newline
58. अथो॒ इत्यथो᳚ । \newline
59. सु॒व॒र्गस्य॑ लो॒कस्य॑ लो॒कस्य॑ सुव॒र्गस्य॑ सुव॒र्गस्य॑ लो॒कस्य॑ । \newline
60. सु॒व॒र्गस्येति॑ सुवः - गस्य॑ । \newline
61. लो॒कस्या नु॑ख्यात्या॒ अनु॑ख्यात्यै लो॒कस्य॑ लो॒कस्या नु॑ख्यात्यै । \newline
62. अनु॑ख्यात्या ए॒षैषा ऽनु॑ख्यात्या॒ अनु॑ख्यात्या ए॒षा । \newline
63. अनु॑ख्यात्या॒ इत्यनु॑ - ख्या॒त्यै॒ । \newline
64. ए॒षा वै वा ए॒षैषा वै । \newline
65. वै दे॒वाना᳚म् दे॒वानां॒ ॅवै वै दे॒वाना᳚म् । \newline
66. दे॒वानां॒ ॅविक्रा᳚न्ति॒र् विक्रा᳚न्तिर् दे॒वाना᳚म् दे॒वानां॒ ॅविक्रा᳚न्तिः । \newline
67. विक्रा᳚न्ति॒र् यद् यद् विक्रा᳚न्ति॒र् विक्रा᳚न्ति॒र् यत् । \newline
68. विक्रा᳚न्ति॒रिति॒ वि - क्रा॒न्तिः॒ । \newline
69. यद् वि॑क॒र्णी वि॑क॒र्णी यद् यद् वि॑क॒र्णी । \newline
70. वि॒क॒र्णी यद् यद् वि॑क॒र्णी वि॑क॒र्णी यत् । \newline
71. वि॒क॒र्णीति॑ वि - क॒र्णी । \newline
72. यद् वि॑क॒र्णीं ॅवि॑क॒र्णीं ॅयद् यद् वि॑क॒र्णीम् । \newline
73. वि॒क॒र्णी मु॑प॒दधा᳚ त्युप॒दधा॑ति विक॒र्णीं ॅवि॑क॒र्णी मु॑प॒दधा॑ति । \newline
74. वि॒क॒र्णीमिति॑ वि - क॒र्णीम् । \newline
75. उ॒प॒दधा॑ति दे॒वाना᳚म् दे॒वाना॑ मुप॒दधा᳚ त्युप॒दधा॑ति दे॒वाना᳚म् । \newline
76. उ॒प॒दधा॒तीत्यु॑प - दधा॑ति । \newline
77. दे॒वाना॑ मे॒वैव दे॒वाना᳚म् दे॒वाना॑ मे॒व । \newline
78. ए॒व विक्रा᳚न्तिं॒ ॅविक्रा᳚न्ति मे॒वैव विक्रा᳚न्तिम् । \newline
79. विक्रा᳚न्ति॒ मन्वनु॒ विक्रा᳚न्तिं॒ ॅविक्रा᳚न्ति॒ मनु॑ । \newline
80. विक्रा᳚न्ति॒मिति॒ वि - क्रा॒न्ति॒म् । \newline
81. अनु॒ वि व्यन्वनु॒ वि । \newline
82. वि क्र॑मते क्रमते॒ वि वि क्र॑मते । \newline
83. क्र॒म॒त॒ उ॒त्त॒र॒त उ॑त्तर॒तः क्र॑मते क्रमत उत्तर॒तः । \newline
84. उ॒त्त॒र॒त उपोपो᳚त्तर॒त उ॑त्तर॒त उप॑ । \newline
85. उ॒त्त॒र॒त इत्यु॑त् - त॒र॒तः । \newline
86. उप॑ दधाति दधा॒ त्युपोप॑ दधाति । \newline
87. द॒धा॒ति॒ तस्मा॒त् तस्मा᳚द् दधाति दधाति॒ तस्मा᳚त् । \newline
88. तस्मा॑ दुत्तर॒त‌उ॑पचार उत्तर॒त‌उ॑पचार॒ स्तस्मा॒त् तस्मा॑ दुत्तर॒त‌उ॑पचारः । \newline
89. उ॒त्त॒र॒त‌उ॑पचारो॒ ऽग्नि र॒ग्नि रु॑त्तर॒त‌उ॑पचार उत्तर॒त‌उ॑पचारो॒ ऽग्निः । \newline
90. उ॒त्त॒र॒त‌उ॑पचार॒ इत्यु॑त्तर॒तः - उ॒प॒चा॒रः॒ । \newline
91. अ॒ग्निर् वा॑यु॒मती॑ वायु॒म त्य॒ग्नि र॒ग्निर् वा॑यु॒मती᳚ । \newline
92. वा॒यु॒मती॑ भवति भवति वायु॒मती॑ वायु॒मती॑ भवति । \newline
93. वा॒यु॒मतीति॑ वायु - मती᳚ । \newline
94. भ॒व॒ति॒ समि॑द्ध्यै॒ समि॑द्ध्यै भवति भवति॒ समि॑द्ध्यै । \newline
95. समि॑द्ध्या॒ इति॒ सं - इ॒द्ध्यै॒ । \newline

\textbf{Ghana Paata } \newline

1. च॒ यज॑मानस्य॒ यज॑मानस्य च च॒ यज॑मानस्य च च॒ यज॑मानस्य च च॒ यज॑मानस्य च । \newline
2. यज॑मानस्य च च॒ यज॑मानस्य॒ यज॑मानस्य च प्रा॒णम् प्रा॒णम् च॒ यज॑मानस्य॒ यज॑मानस्य च प्रा॒णम् । \newline
3. च॒ प्रा॒णम् प्रा॒णम् च॑ च प्रा॒णम् च॑ च प्रा॒णम् च॑ च प्रा॒णम् च॑ । \newline
4. प्रा॒णम् च॑ च प्रा॒णम् प्रा॒णम् चायु॒ रायु॑श्च प्रा॒णम् प्रा॒णम् चायुः॑ । \newline
5. प्रा॒णमिति॑ प्र - अ॒नम् । \newline
6. चायु॒ रायु॑श्च॒ चायु॑श्च॒ चायु॑श्च॒ चायु॑श्च । \newline
7. आयु॑श्च॒ चायु॒ रायु॒श्चा प्यपि॒ चायु॒ रायु॒श्चापि॑ । \newline
8. चाप्यपि॑ च॒ चापि॑ दद्ध्याद् दद्ध्या॒ दपि॑ च॒ चापि॑ दद्ध्यात् । \newline
9. अपि॑ दद्ध्याद् दद्ध्या॒ दप्यपि॑ दद्ध्या॒त् तस्मा॒त् तस्मा᳚द् दद्ध्या॒ दप्यपि॑ दद्ध्या॒त् तस्मा᳚त् । \newline
10. द॒द्ध्या॒त् तस्मा॒त् तस्मा᳚द् दद्ध्याद् दद्ध्या॒त् तस्मा॒न् न न तस्मा᳚द् दद्ध्याद् दद्ध्या॒त् तस्मा॒न् न । \newline
11. तस्मा॒न् न न तस्मा॒त् तस्मा॒न् नान्या ऽन्या न तस्मा॒त् तस्मा॒न् नान्या । \newline
12. नान्या ऽन्या न नान्योत्त॒ रोत्त॑रा॒ ऽन्या न नान्योत्त॑रा । \newline
13. अ॒न्योत्त॒ रोत्त॑रा॒ ऽन्या ऽन्योत्त॒ रेष्ट॒केष्ट॒ कोत्त॑रा॒ ऽन्या ऽन्योत्त॒ रेष्ट॑का । \newline
14. उत्त॒रेष्ट॒केष्ट॒ कोत्त॒ रोत्त॒ रेष्ट॑कोप॒धे यो॑प॒धे येष्ट॒कोत्त॒ रोत्त॒ रेष्ट॑कोप॒धेया᳚ । \newline
15. उत्त॒रेत्युत् - त॒रा॒ । \newline
16. इष्ट॑कोप॒धे यो॑प॒धे येष्ट॒केष्ट॑ कोप॒धेया᳚ स्वयमातृ॒ण्णाꣳ स्व॑यमातृ॒ण्णा मु॑प॒धे येष्ट॒ केष्ट॑ कोप॒धेया᳚ स्वयमातृ॒ण्णाम् । \newline
17. उ॒प॒धेया᳚ स्वयमातृ॒ण्णाꣳ स्व॑यमातृ॒ण्णा मु॑प॒धे यो॑प॒धेया᳚ स्वयमातृ॒ण्णा मुपोप॑ स्वयमातृ॒ण्णा मु॑प॒धे यो॑प॒धेया᳚ स्वयमातृ॒ण्णा मुप॑ । \newline
18. उ॒प॒धेयेत्यु॑प - धेया᳚ । \newline
19. स्व॒य॒मा॒तृ॒ण्णा मुपोप॑ स्वयमातृ॒ण्णाꣳ स्व॑यमातृ॒ण्णा मुप॑ दधाति दधा॒ त्युप॑ स्वयमातृ॒ण्णाꣳ स्व॑यमातृ॒ण्णा मुप॑ दधाति । \newline
20. स्व॒य॒मा॒तृ॒ण्णामिति॑ स्वयं - आ॒तृ॒ण्णाम् । \newline
21. उप॑ दधाति दधा॒ त्युपोप॑ दधा त्य॒सा व॒सौ द॑धा॒ त्युपोप॑ दधा त्य॒सौ । \newline
22. द॒धा॒ त्य॒सा व॒सौ द॑धाति दधा त्य॒सौ वै वा अ॒सौ द॑धाति दधा त्य॒सौ वै । \newline
23. अ॒सौ वै वा अ॒सा व॒सौ वै स्व॑यमातृ॒ण्णा स्व॑यमातृ॒ण्णा वा अ॒सा व॒सौ वै स्व॑यमातृ॒ण्णा । \newline
24. वै स्व॑यमातृ॒ण्णा स्व॑यमातृ॒ण्णा वै वै स्व॑यमातृ॒ण्णा ऽमू म॒मूꣳ स्व॑यमातृ॒ण्णा वै वै स्व॑यमातृ॒ण्णा ऽमूम् । \newline
25. स्व॒य॒मा॒तृ॒ण्णा ऽमू म॒मूꣳ स्व॑यमातृ॒ण्णा स्व॑यमातृ॒ण्णा ऽमू मे॒वैवामूꣳ स्व॑यमातृ॒ण्णा स्व॑यमातृ॒ण्णा ऽमू मे॒व । \newline
26. स्व॒य॒मा॒तृ॒ण्णेति॑ स्वयं - आ॒तृ॒ण्णा । \newline
27. अ॒मू मे॒वै वामू म॒मू मे॒वोपोपै॒ वामू म॒मू मे॒वोप॑ । \newline
28. ए॒वो पोपै॒ वैवोप॑ धत्ते धत्त॒ उपै॒वै वोप॑ धत्ते । \newline
29. उप॑ धत्ते धत्त॒ उपोप॑ ध॒त्ते ऽश्व॒ मश्व॑म् धत्त॒ उपोप॑ ध॒त्ते ऽश्व᳚म् । \newline
30. ध॒त्ते ऽश्व॒ मश्व॑म् धत्ते ध॒त्ते ऽश्व॒ मुपो पाश्व॑म् धत्ते ध॒त्ते ऽश्व॒ मुप॑ । \newline
31. अश्व॒ मुपो पाश्व॒ मश्व॒ मुप॑ घ्रापयति घ्रापय॒ त्युपाश्व॒ मश्व॒ मुप॑ घ्रापयति । \newline
32. उप॑ घ्रापयति घ्रापय॒ त्युपोप॑ घ्रापयति प्रा॒णम् प्रा॒णम् घ्रा॑पय॒ त्युपोप॑ घ्रापयति प्रा॒णम् । \newline
33. घ्रा॒प॒य॒ति॒ प्रा॒णम् प्रा॒णम् घ्रा॑पयति घ्रापयति प्रा॒ण मे॒वैव प्रा॒णम् घ्रा॑पयति घ्रापयति प्रा॒ण मे॒व । \newline
34. प्रा॒ण मे॒वैव प्रा॒णम् प्रा॒ण मे॒वास्या॑ मस्या मे॒व प्रा॒णम् प्रा॒ण मे॒वास्या᳚म् । \newline
35. प्रा॒णमिति॑ प्र - अ॒नम् । \newline
36. ए॒वास्या॑ मस्या मे॒वै वास्या᳚म् दधाति दधा त्यस्या मे॒वै वास्या᳚म् दधाति । \newline
37. अ॒स्या॒म् द॒धा॒ति॒ द॒धा॒ त्य॒स्या॒ म॒स्या॒म् द॒धा॒ त्यथो॒ अथो॑ दधा त्यस्या मस्याम् दधा॒ त्यथो᳚ । \newline
38. द॒धा॒ त्यथो॒ अथो॑ दधाति दधा॒ त्यथो᳚ प्राजाप॒त्यः प्रा॑जाप॒त्यो ऽथो॑ दधाति दधा॒ त्यथो᳚ प्राजाप॒त्यः । \newline
39. अथो᳚ प्राजाप॒त्यः प्रा॑जाप॒त्यो ऽथो॒ अथो᳚ प्राजाप॒त्यो वै वै प्रा॑जाप॒त्यो ऽथो॒ अथो᳚ प्राजाप॒त्यो वै । \newline
40. अथो॒ इत्यथो᳚ । \newline
41. प्रा॒जा॒प॒त्यो वै वै प्रा॑जाप॒त्यः प्रा॑जाप॒त्यो वा अश्वो ऽश्वो॒ वै प्रा॑जाप॒त्यः प्रा॑जाप॒त्यो वा अश्वः॑ । \newline
42. प्रा॒जा॒प॒त्य इति॑ प्राजा - प॒त्यः । \newline
43. वा अश्वो ऽश्वो॒ वै वा अश्वः॑ प्र॒जाप॑तिना प्र॒जाप॑ति॒ना ऽश्वो॒ वै वा अश्वः॑ प्र॒जाप॑तिना । \newline
44. अश्वः॑ प्र॒जाप॑तिना प्र॒जाप॑ति॒ना ऽश्वो ऽश्वः॑ प्र॒जाप॑ति नै॒वैव प्र॒जाप॑ति॒ना ऽश्वो ऽश्वः॑ प्र॒जाप॑ति नै॒व । \newline
45. प्र॒जाप॑ति नै॒वैव प्र॒जाप॑तिना प्र॒जाप॑ति नै॒वाग्नि म॒ग्नि मे॒व प्र॒जाप॑तिना प्र॒जाप॑ति नै॒वाग्निम् । \newline
46. प्र॒जाप॑ति॒नेति॑ प्र॒जा - प॒ति॒ना॒ । \newline
47. ए॒वाग्नि म॒ग्नि मे॒वैवाग्निम् चि॑नुते चिनुते॒ ऽग्नि मे॒वैवाग्निम् चि॑नुते । \newline
48. अ॒ग्निम् चि॑नुते चिनुते॒ ऽग्नि म॒ग्निम् चि॑नुते स्वयमातृ॒ण्णा स्व॑यमातृ॒ण्णा चि॑नुते॒ ऽग्नि म॒ग्निम् चि॑नुते स्वयमातृ॒ण्णा । \newline
49. चि॒नु॒ते॒ स्व॒य॒मा॒तृ॒ण्णा स्व॑यमातृ॒ण्णा चि॑नुते चिनुते स्वयमातृ॒ण्णा भ॑वति भवति स्वयमातृ॒ण्णा चि॑नुते चिनुते स्वयमातृ॒ण्णा भ॑वति । \newline
50. स्व॒य॒मा॒तृ॒ण्णा भ॑वति भवति स्वयमातृ॒ण्णा स्व॑यमातृ॒ण्णा भ॑वति प्रा॒णाना᳚म् प्रा॒णाना᳚म् भवति स्वयमातृ॒ण्णा स्व॑यमातृ॒ण्णा भ॑वति प्रा॒णाना᳚म् । \newline
51. स्व॒य॒मा॒तृ॒ण्णेति॑ स्वयं - आ॒तृ॒ण्णा । \newline
52. भ॒व॒ति॒ प्रा॒णाना᳚म् प्रा॒णाना᳚म् भवति भवति प्रा॒णाना॒ मुथ्सृ॑ष्ट्या॒ उथ्सृ॑ष्ट्यै प्रा॒णाना᳚म् भवति भवति प्रा॒णाना॒ मुथ्सृ॑ष्ट्यै । \newline
53. प्रा॒णाना॒ मुथ्सृ॑ष्ट्या॒ उथ्सृ॑ष्ट्यै प्रा॒णाना᳚म् प्रा॒णाना॒ मुथ्सृ॑ष्ट्या॒ अथो॒ अथो॒ उथ्सृ॑ष्ट्यै प्रा॒णाना᳚म् प्रा॒णाना॒ मुथ्सृ॑ष्ट्या॒ अथो᳚ । \newline
54. प्रा॒णाना॒मिति॑ प्र - अ॒नाना᳚म् । \newline
55. उथ्सृ॑ष्ट्या॒ अथो॒ अथो॒ उथ्सृ॑ष्ट्या॒ उथ्सृ॑ष्ट्या॒ अथो॑ सुव॒र्गस्य॑ सुव॒र्गस्याथो॒ उथ्सृ॑ष्ट्या॒ उथ्सृ॑ष्ट्या॒ अथो॑ सुव॒र्गस्य॑ । \newline
56. उथ्सृ॑ष्ट्या॒ इत्युत् - सृ॒ष्ट्यै॒ । \newline
57. अथो॑ सुव॒र्गस्य॑ सुव॒र्गस्याथो॒ अथो॑ सुव॒र्गस्य॑ लो॒कस्य॑ लो॒कस्य॑ सुव॒र्गस्याथो॒ अथो॑ सुव॒र्गस्य॑ लो॒कस्य॑ । \newline
58. अथो॒ इत्यथो᳚ । \newline
59. सु॒व॒र्गस्य॑ लो॒कस्य॑ लो॒कस्य॑ सुव॒र्गस्य॑ सुव॒र्गस्य॑ लो॒कस्या नु॑ख्यात्या॒ अनु॑ख्यात्यै लो॒कस्य॑ सुव॒र्गस्य॑ सुव॒र्गस्य॑ लो॒कस्या नु॑ख्यात्यै । \newline
60. सु॒व॒र्गस्येति॑ सुवः - गस्य॑ । \newline
61. लो॒कस्या नु॑ख्यात्या॒ अनु॑ख्यात्यै लो॒कस्य॑ लो॒कस्या नु॑ख्यात्या ए॒षैषा ऽनु॑ख्यात्यै लो॒कस्य॑ लो॒कस्या नु॑ख्यात्या ए॒षा । \newline
62. अनु॑ख्यात्या ए॒षैषा ऽनु॑ख्यात्या॒ अनु॑ख्यात्या ए॒षा वै वा ए॒षा ऽनु॑ख्यात्या॒ अनु॑ख्यात्या ए॒षा वै । \newline
63. अनु॑ख्यात्या॒ इत्यनु॑ - ख्या॒त्यै॒ । \newline
64. ए॒षा वै वा ए॒षैषा वै दे॒वाना᳚म् दे॒वानां॒ ॅवा ए॒षैषा वै दे॒वाना᳚म् । \newline
65. वै दे॒वाना᳚म् दे॒वानां॒ ॅवै वै दे॒वानां॒ ॅविक्रा᳚न्ति॒र् विक्रा᳚न्तिर् दे॒वानां॒ ॅवै वै दे॒वानां॒ ॅविक्रा᳚न्तिः । \newline
66. दे॒वानां॒ ॅविक्रा᳚न्ति॒र् विक्रा᳚न्तिर् दे॒वाना᳚म् दे॒वानां॒ ॅविक्रा᳚न्ति॒र् यद् यद् विक्रा᳚न्तिर् दे॒वाना᳚म् दे॒वानां॒ ॅविक्रा᳚न्ति॒र् यत् । \newline
67. विक्रा᳚न्ति॒र् यद् यद् विक्रा᳚न्ति॒र् विक्रा᳚न्ति॒र् यद् वि॑क॒र्णी वि॑क॒र्णी यद् विक्रा᳚न्ति॒र् विक्रा᳚न्ति॒र् यद् वि॑क॒र्णी । \newline
68. विक्रा᳚न्ति॒रिति॒ वि - क्रा॒न्तिः॒ । \newline
69. यद् वि॑क॒र्णी वि॑क॒र्णी यद् यद् वि॑क॒र्णी यद् यद् वि॑क॒र्णी यद् यद् वि॑क॒र्णी यत् । \newline
70. वि॒क॒र्णी यद् यद् वि॑क॒र्णी वि॑क॒र्णी यद् वि॑क॒र्णीं ॅवि॑क॒र्णीं ॅयद् वि॑क॒र्णी वि॑क॒र्णी यद् वि॑क॒र्णीम् । \newline
71. वि॒क॒र्णीति॑ वि - क॒र्णी । \newline
72. यद् वि॑क॒र्णीं ॅवि॑क॒र्णीं ॅयद् यद् वि॑क॒र्णी मु॑प॒दधा᳚ त्युप॒दधा॑ति विक॒र्णीं ॅयद् यद् वि॑क॒र्णी मु॑प॒दधा॑ति । \newline
73. वि॒क॒र्णी मु॑प॒दधा᳚ त्युप॒दधा॑ति विक॒र्णीं ॅवि॑क॒र्णी मु॑प॒दधा॑ति दे॒वाना᳚म् दे॒वाना॑ मुप॒दधा॑ति विक॒र्णीं ॅवि॑क॒र्णी मु॑प॒दधा॑ति दे॒वाना᳚म् । \newline
74. वि॒क॒र्णीमिति॑ वि - क॒र्णीम् । \newline
75. उ॒प॒दधा॑ति दे॒वाना᳚म् दे॒वाना॑ मुप॒दधा᳚ त्युप॒दधा॑ति दे॒वाना॑ मे॒वैव दे॒वाना॑ मुप॒दधा᳚ त्युप॒दधा॑ति दे॒वाना॑ मे॒व । \newline
76. उ॒प॒दधा॒तीत्यु॑प - दधा॑ति । \newline
77. दे॒वाना॑ मे॒वैव दे॒वाना᳚म् दे॒वाना॑ मे॒व विक्रा᳚न्तिं॒ ॅविक्रा᳚न्ति मे॒व दे॒वाना᳚म् दे॒वाना॑ मे॒व विक्रा᳚न्तिम् । \newline
78. ए॒व विक्रा᳚न्तिं॒ ॅविक्रा᳚न्ति मे॒वैव विक्रा᳚न्ति॒ मन्वनु॒ विक्रा᳚न्ति मे॒वैव विक्रा᳚न्ति॒ मनु॑ । \newline
79. विक्रा᳚न्ति॒ मन्वनु॒ विक्रा᳚न्तिं॒ ॅविक्रा᳚न्ति॒ मनु॒ वि व्यनु॒ विक्रा᳚न्तिं॒ ॅविक्रा᳚न्ति॒ मनु॒ वि । \newline
80. विक्रा᳚न्ति॒मिति॒ वि - क्रा॒न्ति॒म् । \newline
81. अनु॒ वि व्यन्वनु॒ वि क्र॑मते क्रमते॒ व्यन्वनु॒ वि क्र॑मते । \newline
82. वि क्र॑मते क्रमते॒ वि वि क्र॑मत उत्तर॒त उ॑त्तर॒तः क्र॑मते॒ वि वि क्र॑मत उत्तर॒तः । \newline
83. क्र॒म॒त॒ उ॒त्त॒र॒त उ॑त्तर॒तः क्र॑मते क्रमत उत्तर॒त उपोपो᳚त्तर॒तः क्र॑मते क्रमत उत्तर॒त उप॑ । \newline
84. उ॒त्त॒र॒त उपोपो᳚ त्तर॒त उ॑त्तर॒त उप॑ दधाति दधा॒ त्युपो᳚त्तर॒त उ॑त्तर॒त उप॑ दधाति । \newline
85. उ॒त्त॒र॒त इत्यु॑त् - त॒र॒तः । \newline
86. उप॑ दधाति दधा॒ त्युपोप॑ दधाति॒ तस्मा॒त् तस्मा᳚द् दधा॒ त्युपोप॑ दधाति॒ तस्मा᳚त् । \newline
87. द॒धा॒ति॒ तस्मा॒त् तस्मा᳚द् दधाति दधाति॒ तस्मा॑ दुत्तर॒त‌उ॑पचार उत्तर॒त‌उ॑पचार॒ स्तस्मा᳚द् दधाति दधाति॒ तस्मा॑ दुत्तर॒त‌उ॑पचारः । \newline
88. तस्मा॑ दुत्तर॒त‍उ॑पचार उत्तर॒त‍उ॑पचार॒ स्तस्मा॒त् तस्मा॑ दुत्तर॒त‍उ॑पचारो॒ ऽग्नि र॒ग्नि रु॑त्तर॒त‍उ॑पचार॒ स्तस्मा॒त् तस्मा॑ दुत्तर॒त‍उ॑पचारो॒ ऽग्निः । \newline
89. उ॒त्त॒र॒त‍उ॑पचारो॒ ऽग्नि र॒ग्नि रु॑त्तर॒त‍उ॑पचार उत्तर॒त‍उ॑पचारो॒ ऽग्निर् वा॑यु॒मती॑ वायु॒म त्य॒ग्नि 
रु॑त्तर॒त‍उ॑पचार उत्तर॒त‍उ॑पचारो॒ ऽग्निर् वा॑यु॒मती᳚ । \newline
90. उ॒त्त॒र॒त‍उ॑पचार॒ इत्यु॑त्तर॒तः - उ॒प॒चा॒रः॒ । \newline
91. अ॒ग्निर् वा॑यु॒मती॑ वायु॒म त्य॒ग्नि र॒ग्निर् वा॑यु॒मती॑ भवति भवति वायु॒म त्य॒ग्नि र॒ग्निर् वा॑यु॒मती॑ भवति । \newline
92. वा॒यु॒मती॑ भवति भवति वायु॒मती॑ वायु॒मती॑ भवति॒ समि॑द्ध्यै॒ समि॑द्ध्यै भवति वायु॒मती॑ वायु॒मती॑ भवति॒ समि॑द्ध्यै । \newline
93. वा॒यु॒मतीति॑ वायु - मती᳚ । \newline
94. भ॒व॒ति॒ समि॑द्ध्यै॒ समि॑द्ध्यै भवति भवति॒ समि॑द्ध्यै । \newline
95. समि॑द्ध्या॒ इति॒ सं - इ॒द्ध्यै॒ । \newline
\pagebreak
\markright{ TS 5.3.8.1  \hfill https://www.vedavms.in \hfill}

\section{ TS 5.3.8.1 }

\textbf{TS 5.3.8.1 } \newline
\textbf{Samhita Paata} \newline

छन्दाꣳ॒॒स्युप॑ दधाति प॒शवो॒ वै छन्दाꣳ॑सि प॒शूने॒वाव॑ रुन्धे॒ छन्दाꣳ॑सि॒ वै दे॒वानां᳚ ॅवा॒मं प॒शवो॑ वा॒ममे॒व प॒शूनव॑ रुन्ध ए॒ताꣳ ह॒ वै य॒ज्ञ्से॑न-श्चैत्रियाय॒ण-श्चितिं॑ ॅवि॒दां च॑कार॒ तया॒ वै स प॒शूनवा॑रुन्ध॒ यदे॒तामु॑प॒दधा॑ति प॒शूने॒वाव॑ रुन्धे गाय॒त्रीः पु॒रस्ता॒दुप॑ दधाति॒ तेजो॒ वै गा॑य॒त्री तेज॑ ए॒व - [  ] \newline

\textbf{Pada Paata} \newline

छन्दाꣳ॑सि । उपेति॑ । द॒धा॒ति॒ । प॒शवः॑ । वै । छन्दाꣳ॑सि । प॒शून् । ए॒व । अवेति॑ । रु॒न्धे॒ । छन्दाꣳ॑सि । वै । दे॒वाना᳚म् । वा॒मम् । प॒शवः॑ । वा॒मम् । ए॒व । प॒शून् । अवेति॑ । रु॒न्धे॒ । ए॒ताम् । ह॒ । वै । य॒ज्ञ्से॑न॒ इति॑ य॒ज्ञ् - से॒नः॒ । चै॒त्रि॒या॒य॒णः । चिति᳚म् । वि॒दाम् । च॒का॒र॒ । तया᳚ । वै । सः । प॒शून् । अवेति॑ । अ॒रु॒न्ध॒ । यत् । ए॒ताम् । उ॒प॒दधा॒तीत्यु॑प - द॒धा॑ति । प॒शून् । ए॒व । अवेति॑ । रु॒न्धे॒ । गा॒य॒त्रीः । पु॒रस्ता᳚त् । उपेति॑ । द॒धा॒ति॒ । तेजः॑ । वै । गा॒य॒त्री । तेजः॑ । ए॒व ।  \newline


\textbf{Krama Paata} \newline

छन्दाꣳ॒॒स्युप॑ । उप॑ दधाति । द॒धा॒ति॒ प॒शवः॑ । प॒शवो॒ वै । वै छन्दाꣳ॑सि । छन्दाꣳ॑सि प॒शून् । प॒शूने॒व । ए॒वाव॑ । अव॑ रुन्धे । रु॒न्धे॒ छन्दाꣳ॑सि । छन्दाꣳ॑सि॒ वै । वै दे॒वाना᳚म् । दे॒वाना᳚म् ॅवा॒मम् । वा॒मम् प॒शवः॑ । प॒शवो॑ वा॒मम् । वा॒ममे॒व । ए॒व प॒शून् । प॒शूनव॑ । अव॑ रुन्धे । रु॒न्ध॒ ए॒ताम् । ए॒ताꣳ ह॑ । ह॒ वै । वै य॒ज्ञ्से॑नः । य॒ज्ञ्से॑नश्चैत्रियाय॒णः । य॒ज्ञ्से॑न॒ इति॑ य॒ज्ञ् - से॒नः॒ । चै॒त्रि॒या॒य॒णश्चिति᳚म् । चिति॑म् ॅवि॒दाम् । वि॒दाम् च॑कार । च॒का॒र॒ तया᳚ । तया॒ वै । वै सः । स प॒शून् । प॒शूनव॑ । अवा॑रुन्ध । अ॒रु॒न्ध॒ यत् । यदे॒ताम् । ए॒तामु॑प॒दधा॑ति । उ॒प॒दधा॑ति प॒शून् । उ॒प॒दधा॒तीत्यु॑प - दधा॑ति । प॒शूने॒व । ए॒वाव॑ । अव॑ रुन्धे । रु॒न्धे॒ गा॒य॒त्रीः । गा॒य॒त्रीः पु॒रस्ता᳚त् । पु॒रस्ता॒दुप॑ । उप॑ दधाति । द॒धा॒ति॒ तेजः॑ । तेजो॒ वै । वै गा॑य॒त्री । गा॒य॒त्री तेजः॑ । तेज॑ ए॒व । ए॒व मु॑ख॒तः \newline

\textbf{Jatai Paata} \newline

1. छन्दाꣳ॒॒ स्युपोप॒ छन्दाꣳ॑सि॒ छन्दाꣳ॒॒ स्युप॑ । \newline
2. उप॑ दधाति दधा॒ त्युपोप॑ दधाति । \newline
3. द॒धा॒ति॒ प॒शवः॑ प॒शवो॑ दधाति दधाति प॒शवः॑ । \newline
4. प॒शवो॒ वै वै प॒शवः॑ प॒शवो॒ वै । \newline
5. वै छन्दाꣳ॑सि॒ छन्दाꣳ॑सि॒ वै वै छन्दाꣳ॑सि । \newline
6. छन्दाꣳ॑सि प॒शून् प॒शून् छन्दाꣳ॑सि॒ छन्दाꣳ॑सि प॒शून् । \newline
7. प॒शू ने॒वैव प॒शून् प॒शू ने॒व । \newline
8. ए॒वावा वै॒वै वाव॑ । \newline
9. अव॑ रुन्धे रु॒न्धे ऽवाव॑ रुन्धे । \newline
10. रु॒न्धे॒ छन्दाꣳ॑सि॒ छन्दाꣳ॑सि रुन्धे रुन्धे॒ छन्दाꣳ॑सि । \newline
11. छन्दाꣳ॑सि॒ वै वै छन्दाꣳ॑सि॒ छन्दाꣳ॑सि॒ वै । \newline
12. वै दे॒वाना᳚म् दे॒वानां॒ ॅवै वै दे॒वाना᳚म् । \newline
13. दे॒वानां᳚ ॅवा॒मं ॅवा॒मम् दे॒वाना᳚म् दे॒वानां᳚ ॅवा॒मम् । \newline
14. वा॒मम् प॒शवः॑ प॒शवो॑ वा॒मं ॅवा॒मम् प॒शवः॑ । \newline
15. प॒शवो॑ वा॒मं ॅवा॒मम् प॒शवः॑ प॒शवो॑ वा॒मम् । \newline
16. वा॒म मे॒वैव वा॒मं ॅवा॒म मे॒व । \newline
17. ए॒व प॒शून् प॒शू ने॒वैव प॒शून् । \newline
18. प॒शू नवाव॑ प॒शून् प॒शू नव॑ । \newline
19. अव॑ रुन्धे रु॒न्धे ऽवाव॑ रुन्धे । \newline
20. रु॒न्ध॒ ए॒ता मे॒ताꣳ रु॑न्धे रुन्ध ए॒ताम् । \newline
21. ए॒ताꣳ ह॑ है॒ता मे॒ताꣳ ह॑ । \newline
22. ह॒ वै वै ह॑ ह॒ वै । \newline
23. वै य॒ज्ञ्से॑नो य॒ज्ञ्से॑नो॒ वै वै य॒ज्ञ्से॑नः । \newline
24. य॒ज्ञ्से॑न श्चैत्रियाय॒ण श्चै᳚त्रियाय॒णो य॒ज्ञ्से॑नो य॒ज्ञ्से॑न श्चैत्रियाय॒णः । \newline
25. य॒ज्ञ्से॑न॒ इति॑ य॒ज्ञ् - से॒नः॒ । \newline
26. चै॒त्रि॒या॒य॒ण श्चिति॒म् चिति॑म् चैत्रियाय॒ण श्चै᳚त्रियाय॒ण श्चिति᳚म् । \newline
27. चितिं॑ ॅवि॒दां ॅवि॒दाम् चिति॒म् चितिं॑ ॅवि॒दाम् । \newline
28. वि॒दाम् च॑कार चकार वि॒दां ॅवि॒दाम् च॑कार । \newline
29. च॒का॒र॒ तया॒ तया॑ चकार चकार॒ तया᳚ । \newline
30. तया॒ वै वै तया॒ तया॒ वै । \newline
31. वै स स वै वै सः । \newline
32. स प॒शून् प॒शून् थ्स स प॒शून् । \newline
33. प॒शू नवाव॑ प॒शून् प॒शू नव॑ । \newline
34. अवा॑रुन्धा रु॒न्धा वावा॑ रुन्ध । \newline
35. अ॒रु॒न्ध॒ यद् यद॑रुन्धा रुन्ध॒ यत् । \newline
36. यदे॒ता मे॒तां ॅयद् यदे॒ताम् । \newline
37. ए॒ता मु॑प॒दधा᳚ त्युप॒दधा᳚ त्ये॒ता मे॒ता मु॑प॒दधा॑ति । \newline
38. उ॒प॒दधा॑ति प॒शून् प॒शू नु॑प॒दधा᳚ त्युप॒दधा॑ति प॒शून् । \newline
39. उ॒प॒दधा॒तीत्यु॑प - दधा॑ति । \newline
40. प॒शू ने॒वैव प॒शून् प॒शू ने॒व । \newline
41. ए॒वावा वै॒वै वाव॑ । \newline
42. अव॑ रुन्धे रु॒न्धे ऽवाव॑ रुन्धे । \newline
43. रु॒न्धे॒ गा॒य॒त्रीर् गा॑य॒त्री रु॑न्धे रुन्धे गाय॒त्रीः । \newline
44. गा॒य॒त्रीः पु॒रस्ता᳚त् पु॒रस्ता᳚द् गाय॒त्रीर् गा॑य॒त्रीः पु॒रस्ता᳚त् । \newline
45. पु॒रस्ता॒ दुपोप॑ पु॒रस्ता᳚त् पु॒रस्ता॒ दुप॑ । \newline
46. उप॑ दधाति दधा॒ त्युपोप॑ दधाति । \newline
47. द॒धा॒ति॒ तेज॒ स्तेजो॑ दधाति दधाति॒ तेजः॑ । \newline
48. तेजो॒ वै वै तेज॒ स्तेजो॒ वै । \newline
49. वै गा॑य॒त्री गा॑य॒त्री वै वै गा॑य॒त्री । \newline
50. गा॒य॒त्री तेज॒ स्तेजो॑ गाय॒त्री गा॑य॒त्री तेजः॑ । \newline
51. तेज॑ ए॒वैव तेज॒ स्तेज॑ ए॒व । \newline
52. ए॒व मु॑ख॒तो मु॑ख॒त ए॒वैव मु॑ख॒तः । \newline

\textbf{Ghana Paata } \newline

1. छन्दाꣳ॒॒ स्युपोप॒ च्छन्दाꣳ॑सि॒ छन्दाꣳ॒॒स्युप॑ दधाति दधा॒ त्युप॒ च्छन्दाꣳ॑सि॒ छन्दाꣳ॒॒स्युप॑ दधाति । \newline
2. उप॑ दधाति दधा॒ त्युपोप॑ दधाति प॒शवः॑ प॒शवो॑ दधा॒ त्युपोप॑ दधाति प॒शवः॑ । \newline
3. द॒धा॒ति॒ प॒शवः॑ प॒शवो॑ दधाति दधाति प॒शवो॒ वै वै प॒शवो॑ दधाति दधाति प॒शवो॒ वै । \newline
4. प॒शवो॒ वै वै प॒शवः॑ प॒शवो॒ वै छन्दाꣳ॑सि॒ छन्दाꣳ॑सि॒ वै प॒शवः॑ प॒शवो॒ वै छन्दाꣳ॑सि । \newline
5. वै छन्दाꣳ॑सि॒ छन्दाꣳ॑सि॒ वै वै छन्दाꣳ॑सि प॒शून् प॒शून् छन्दाꣳ॑सि॒ वै वै छन्दाꣳ॑सि प॒शून् । \newline
6. छन्दाꣳ॑सि प॒शून् प॒शून् छन्दाꣳ॑सि॒ छन्दाꣳ॑सि प॒शू ने॒वैव प॒शून् छन्दाꣳ॑सि॒ छन्दाꣳ॑सि प॒शू ने॒व । \newline
7. प॒शू ने॒वैव प॒शून् प॒शू ने॒वा वावै॒व प॒शून् प॒शू ने॒वाव॑ । \newline
8. ए॒वावा वै॒वै वाव॑ रुन्धे रु॒न्धे ऽवै॒वै वाव॑ रुन्धे । \newline
9. अव॑ रुन्धे रु॒न्धे ऽवाव॑ रुन्धे॒ छन्दाꣳ॑सि॒ छन्दाꣳ॑सि रु॒न्धे ऽवाव॑ रुन्धे॒ छन्दाꣳ॑सि । \newline
10. रु॒न्धे॒ छन्दाꣳ॑सि॒ छन्दाꣳ॑सि रुन्धे रुन्धे॒ छन्दाꣳ॑सि॒ वै वै छन्दाꣳ॑सि रुन्धे रुन्धे॒ छन्दाꣳ॑सि॒ वै । \newline
11. छन्दाꣳ॑सि॒ वै वै छन्दाꣳ॑सि॒ छन्दाꣳ॑सि॒ वै दे॒वाना᳚म् दे॒वानां॒ ॅवै छन्दाꣳ॑सि॒ छन्दाꣳ॑सि॒ वै दे॒वाना᳚म् । \newline
12. वै दे॒वाना᳚म् दे॒वानां॒ ॅवै वै दे॒वानां᳚ ॅवा॒मं ॅवा॒मम् दे॒वानां॒ ॅवै वै दे॒वानां᳚ ॅवा॒मम् । \newline
13. दे॒वानां᳚ ॅवा॒मं ॅवा॒मम् दे॒वाना᳚म् दे॒वानां᳚ ॅवा॒मम् प॒शवः॑ प॒शवो॑ वा॒मम् दे॒वाना᳚म् दे॒वानां᳚ ॅवा॒मम् प॒शवः॑ । \newline
14. वा॒मम् प॒शवः॑ प॒शवो॑ वा॒मं ॅवा॒मम् प॒शवो॑ वा॒मं ॅवा॒मम् प॒शवो॑ वा॒मं ॅवा॒मम् प॒शवो॑ वा॒मम् । \newline
15. प॒शवो॑ वा॒मं ॅवा॒मम् प॒शवः॑ प॒शवो॑ वा॒म मे॒वैव वा॒मम् प॒शवः॑ प॒शवो॑ वा॒म मे॒व । \newline
16. वा॒म मे॒वैव वा॒मं ॅवा॒म मे॒व प॒शून् प॒शू ने॒व वा॒मं ॅवा॒म मे॒व प॒शून् । \newline
17. ए॒व प॒शून् प॒शू ने॒वैव प॒शू नवाव॑ प॒शू ने॒वैव प॒शू नव॑ । \newline
18. प॒शू नवाव॑ प॒शून् प॒शू नव॑ रुन्धे रु॒न्धे ऽव॑ प॒शून् प॒शू नव॑ रुन्धे । \newline
19. अव॑ रुन्धे रु॒न्धे ऽवाव॑ रुन्ध ए॒ता मे॒ताꣳ रु॒न्धे ऽवाव॑ रुन्ध ए॒ताम् । \newline
20. रु॒न्ध॒ ए॒ता मे॒ताꣳ रु॑न्धे रुन्ध ए॒ताꣳ ह॑ है॒ताꣳ रु॑न्धे रुन्ध ए॒ताꣳ ह॑ । \newline
21. ए॒ताꣳ ह॑ है॒ता मे॒ताꣳ ह॒ वै वै है॒ता मे॒ताꣳ ह॒ वै । \newline
22. ह॒ वै वै ह॑ ह॒ वै य॒ज्ञ्से॑नो य॒ज्ञ्से॑नो॒ वै ह॑ ह॒ वै य॒ज्ञ्से॑नः । \newline
23. वै य॒ज्ञ्से॑नो य॒ज्ञ्से॑नो॒ वै वै य॒ज्ञ्से॑न श्चैत्रियाय॒ण श्चै᳚त्रियाय॒णो य॒ज्ञ्से॑नो॒ वै वै य॒ज्ञ्से॑न श्चैत्रियाय॒णः । \newline
24. य॒ज्ञ्से॑न श्चैत्रियाय॒ण श्चै᳚त्रियाय॒णो य॒ज्ञ्से॑नो य॒ज्ञ्से॑न श्चैत्रियाय॒ण श्चिति॒म् चिति॑म् चैत्रियाय॒णो य॒ज्ञ्से॑नो य॒ज्ञ्से॑न श्चैत्रियाय॒ण श्चिति᳚म् । \newline
25. य॒ज्ञ्से॑न॒ इति॑ य॒ज्ञ् - से॒नः॒ । \newline
26. चै॒त्रि॒या॒य॒ण श्चिति॒म् चिति॑म् चैत्रियाय॒ण श्चै᳚त्रियाय॒ण श्चितिं॑ ॅवि॒दां ॅवि॒दाम् चिति॑म् चैत्रियाय॒ण श्चै᳚त्रियाय॒ण श्चितिं॑ ॅवि॒दाम् । \newline
27. चितिं॑ ॅवि॒दां ॅवि॒दाम् चिति॒म् चितिं॑ ॅवि॒दाम् च॑कार चकार वि॒दाम् चिति॒म् चितिं॑ ॅवि॒दाम् च॑कार । \newline
28. वि॒दाम् च॑कार चकार वि॒दां ॅवि॒दाम् च॑कार॒ तया॒ तया॑ चकार वि॒दां ॅवि॒दाम् च॑कार॒ तया᳚ । \newline
29. च॒का॒र॒ तया॒ तया॑ चकार चकार॒ तया॒ वै वै तया॑ चकार चकार॒ तया॒ वै । \newline
30. तया॒ वै वै तया॒ तया॒ वै स स वै तया॒ तया॒ वै सः । \newline
31. वै स स वै वै स प॒शून् प॒शून् थ्स वै वै स प॒शून् । \newline
32. स प॒शून् प॒शून् थ्स स प॒शू नवाव॑ प॒शून् थ्स स प॒शू नव॑ । \newline
33. प॒शू नवाव॑ प॒शून् प॒शू नवा॑रुन्धा रु॒न्धाव॑ प॒शून् प॒शू नवा॑रुन्ध । \newline
34. अवा॑रुन्धा रु॒न्धा वावा॑ रुन्ध॒ यद् यद॑रु॒न्धा वावा॑ रुन्ध॒ यत् । \newline
35. अ॒रु॒न्ध॒ यद् यद॑रुन्धा रुन्ध॒ यदे॒ता मे॒तां ॅयद॑रुन्धा रुन्ध॒ यदे॒ताम् । \newline
36. यदे॒ता मे॒तां ॅयद् यदे॒ता मु॑प॒दधा᳚ त्युप॒दधा᳚ त्ये॒तां ॅयद् यदे॒ता मु॑प॒दधा॑ति । \newline
37. ए॒ता मु॑प॒दधा᳚ त्युप॒दधा᳚ त्ये॒ता मे॒ता मु॑प॒दधा॑ति प॒शून् प॒शू नु॑प॒दधा᳚ त्ये॒ता मे॒ता मु॑प॒दधा॑ति प॒शून् । \newline
38. उ॒प॒दधा॑ति प॒शून् प॒शू नु॑प॒दधा᳚ त्युप॒दधा॑ति प॒शू ने॒वैव प॒शू नु॑प॒दधा᳚ त्युप॒दधा॑ति प॒शू ने॒व । \newline
39. उ॒प॒दधा॒तीत्यु॑प - दधा॑ति । \newline
40. प॒शू ने॒वैव प॒शून् प॒शू ने॒वावा वै॒व प॒शून् प॒शू ने॒वाव॑ । \newline
41. ए॒वावा वै॒वै वाव॑ रुन्धे रु॒न्धे ऽवै॒वै वाव॑ रुन्धे । \newline
42. अव॑ रुन्धे रु॒न्धे ऽवाव॑ रुन्धे गाय॒त्रीर् गा॑य॒त्री रु॒न्धे ऽवाव॑ रुन्धे गाय॒त्रीः । \newline
43. रु॒न्धे॒ गा॒य॒त्रीर् गा॑य॒त्री रु॑न्धे रुन्धे गाय॒त्रीः पु॒रस्ता᳚त् पु॒रस्ता᳚द् गाय॒त्री रु॑न्धे रुन्धे गाय॒त्रीः पु॒रस्ता᳚त् । \newline
44. गा॒य॒त्रीः पु॒रस्ता᳚त् पु॒रस्ता᳚द् गाय॒त्रीर् गा॑य॒त्रीः पु॒रस्ता॒ दुपोप॑ पु॒रस्ता᳚द् गाय॒त्रीर् गा॑य॒त्रीः पु॒रस्ता॒ दुप॑ । \newline
45. पु॒रस्ता॒ दुपोप॑ पु॒रस्ता᳚त् पु॒रस्ता॒ दुप॑ दधाति दधा॒ त्युप॑ पु॒रस्ता᳚त् पु॒रस्ता॒ दुप॑ दधाति । \newline
46. उप॑ दधाति दधा॒ त्युपोप॑ दधाति॒ तेज॒ स्तेजो॑ दधा॒ त्युपोप॑ दधाति॒ तेजः॑ । \newline
47. द॒धा॒ति॒ तेज॒ स्तेजो॑ दधाति दधाति॒ तेजो॒ वै वै तेजो॑ दधाति दधाति॒ तेजो॒ वै । \newline
48. तेजो॒ वै वै तेज॒ स्तेजो॒ वै गा॑य॒त्री गा॑य॒त्री वै तेज॒ स्तेजो॒ वै गा॑य॒त्री । \newline
49. वै गा॑य॒त्री गा॑य॒त्री वै वै गा॑य॒त्री तेज॒ स्तेजो॑ गाय॒त्री वै वै गा॑य॒त्री तेजः॑ । \newline
50. गा॒य॒त्री तेज॒ स्तेजो॑ गाय॒त्री गा॑य॒त्री तेज॑ ए॒वैव तेजो॑ गाय॒त्री गा॑य॒त्री तेज॑ ए॒व । \newline
51. तेज॑ ए॒वैव तेज॒ स्तेज॑ ए॒व मु॑ख॒तो मु॑ख॒त ए॒व तेज॒ स्तेज॑ ए॒व मु॑ख॒तः । \newline
52. ए॒व मु॑ख॒तो मु॑ख॒त ए॒वैव मु॑ख॒तो ध॑त्ते धत्ते मुख॒त ए॒वैव मु॑ख॒तो ध॑त्ते । \newline
\pagebreak
\markright{ TS 5.3.8.2  \hfill https://www.vedavms.in \hfill}

\section{ TS 5.3.8.2 }

\textbf{TS 5.3.8.2 } \newline
\textbf{Samhita Paata} \newline

मु॑ख॒तो ध॑त्ते मूर्द्ध॒न्वती᳚र्भवन्ति मू॒र्द्धान॑मे॒वैनꣳ॑ समा॒नानां᳚ करोति त्रि॒ष्टुभ॒ उप॑ दधातीन्द्रि॒यं ॅवै त्रि॒ष्टुगि॑न्द्रि॒यमे॒व म॑द्ध्य॒तो ध॑त्ते॒ जग॑ती॒रुप॑ दधाति॒ जाग॑ता॒ वै प॒शवः॑ प॒शूने॒वाव॑ रुन्धे ऽनु॒ष्टुभ॒ उप॑ दधाति प्रा॒णा वा अ॑नु॒ष्टुप् प्रा॒णाना॒मुथ्सृ॑ष्ट्यै बृह॒तीरु॒ष्णिहाः᳚ प॒ङ्क्तीर॒क्षर॑पङ्क्ती॒रिति॒ विषु॑रूपाणि॒ छन्दाꣳ॒॒स्युप॑ दधाति॒ विषु॑रूपा॒ वै प॒शवः॑ प॒शवः॒ - [  ] \newline

\textbf{Pada Paata} \newline

मु॒ख॒तः । ध॒त्ते॒ । मू॒द्‌र्ध॒न्वती॒रिति॑ मूर्धन्न् - वतीः᳚ । भ॒व॒न्ति॒ । मू॒द्‌र्धान᳚म् । ए॒व । ए॒न॒म् । स॒मा॒नाना᳚म् । क॒रो॒ति॒ । त्रि॒ष्टुभः॑ । उपेति॑ । द॒धा॒ति॒ । इ॒न्द्रि॒यम् । वै । त्रि॒ष्टुक् । इ॒न्द्रि॒यम् । ए॒व । म॒द्ध्य॒तः । ध॒त्ते॒ । जग॑तीः । उपेति॑ । द॒धा॒ति॒ । जाग॑ताः । वै । प॒शवः॑ । प॒शून् । ए॒व । अवेति॑ । रु॒न्धे॒ । अ॒नु॒ष्टुभ॒ इत्य॑नु - स्तुभः॑ । उपेति॑ । द॒धा॒ति॒ । प्रा॒णा इति॑ प्र - अ॒नाः । वै । अ॒नु॒ष्टुबित्य॑नु - स्तुप् । प्रा॒णाना॒मिति॑ प्र -  अ॒नाना᳚म् । उथ्सृ॑ष्ट्या॒ इत्युत् - सृ॒ष्ट्यै॒ । बृ॒ह॒तीः । उ॒ष्णिहाः᳚ । प॒ङ्क्तीः । अ॒क्षर॑पङ्क्ती॒रित्य॒क्षर॑ - प॒ङ्क्तीः॒ । इति॑ । विषु॑रूपा॒णीति॒ विषु॑ - रू॒पा॒णि॒ । छन्दाꣳ॑सि । उपेति॑ । द॒धा॒ति॒ । विषु॑रूपा॒ इति॒ विषु॑ - रू॒पाः॒ । वै । प॒शवः॑ । प॒शवः॑ ।  \newline


\textbf{Krama Paata} \newline

मु॒ख॒तो ध॑त्ते । ध॒त्ते॒ मू॒र्द्ध॒न्वतीः᳚ । मू॒र्द्ध॒न्वती᳚र् भवन्ति । मू॒र्द्ध॒न्वती॒रिति॑ मूर्द्धन्न् - वतीः᳚ । भ॒व॒न्ति॒ मू॒र्द्धान᳚म् । मू॒र्द्धान॑मे॒व । ए॒वैन᳚म् । ए॒नꣳ॒॒ स॒मा॒नाना᳚म् । स॒मा॒नाना᳚म् करोति । क॒रो॒ति॒ त्रि॒ष्टुभः॑ । त्रि॒ष्टुभ॒ उप॑ । उप॑ दधाति । द॒धा॒ती॒न्द्रि॒यम् । इ॒न्द्रि॒य ॅवै । वै त्रि॒ष्टुक् । त्रि॒ष्टुगि॑न्द्रि॒यम् । इ॒न्द्रि॒यमे॒व । ए॒व म॑द्ध्य॒तः । म॒द्ध्य॒तो ध॑त्ते । ध॒त्ते॒ जग॑तीः । जग॑ती॒रुप॑ । उप॑ दधाति । द॒धा॒ति॒ जाग॑ताः । जाग॑ता॒ वै । वै प॒शवः॑ । प॒शवः॑ प॒शून् । प॒शूने॒व । ए॒वाव॑ । अव॑ रुन्धे । रु॒न्धे॒ऽनु॒ष्टुभः॑ । अ॒नु॒ष्टुभ॒ उप॑ । अ॒नु॒ष्टुभ॒ इत्य॑नु - स्तुभः॑ । उप॑ दधाति । द॒धा॒ति॒ प्रा॒णाः । प्रा॒णा वै । प्रा॒णा इति॑ प्र - अ॒नाः । वा अ॑नु॒ष्टुप् । अ॒नु॒ष्टुप् प्रा॒णाना᳚म् । अ॒नु॒ष्टुबित्य॑नु - स्तुप् । प्रा॒णाना॒मुथ्सृ॑ष्ट्यै । प्रा॒णाना॒मिति॑ प्र - अ॒नाना᳚म् । उथ्सृ॑ष्ट्यै बृह॒तीः । उथ्सृ॑ष्ट्या॒ इत्युत् - सृ॒ष्ट्यै॒ । बृ॒ह॒तीरु॒ष्णिहाः᳚ । उ॒ष्णिहाः᳚ प॒ङ्क्तीः । प॒ङ्क्तीर॒क्षर॑पङ्क्तीः । अ॒क्षर॑पङ्क्ती॒रिति॑ । अ॒क्षर॑पङ्क्ती॒रित्य॒क्षर॑ - प॒ङ्क्तीः॒ । इति॒ विषु॑रूपाणि । विषु॑रूपाणि॒ छन्दाꣳ॑सि । विषु॑रूपा॒णीति॒ विषु॑ - रू॒पा॒णि॒ । छन्दाꣳ॒॒स्युप॑ । उप॑ दधाति । द॒धा॒ति॒ विषु॑रूपाः । विषु॑रूपा॒ वै । विषु॑रूपा॒ इति॒ विषु॑ - रू॒पाः॒ । वै प॒शवः॑ । प॒शवः॑ प॒शवः॑ । प॒शव॒ श्छन्दाꣳ॑सि \newline

\textbf{Jatai Paata} \newline

1. मु॒ख॒तो ध॑त्ते धत्ते मुख॒तो मु॑ख॒तो ध॑त्ते । \newline
2. ध॒त्ते॒ मू॒र्द्ध॒न्वती᳚र् मूर्द्ध॒न्वती᳚र् धत्ते धत्ते मूर्द्ध॒न्वतीः᳚ । \newline
3. मू॒र्द्ध॒न्वती᳚र् भवन्ति भवन्ति मूर्द्ध॒न्वती᳚र् मूर्द्ध॒न्वती᳚र् भवन्ति । \newline
4. मू॒र्द्ध॒न्वती॒रिति॑ मूर्धन्न् - वतीः᳚ । \newline
5. भ॒व॒न्ति॒ मू॒र्द्धान॑म् मू॒र्द्धान॑म् भवन्ति भवन्ति मू॒र्द्धान᳚म् । \newline
6. मू॒र्द्धान॑ मे॒वैव मू॒र्द्धान॑म् मू॒र्द्धान॑ मे॒व । \newline
7. ए॒वैन॑ मेन मे॒वै वैन᳚म् । \newline
8. ए॒नꣳ॒॒ स॒मा॒नानाꣳ॑ समा॒नाना॑ मेन मेनꣳ समा॒नाना᳚म् । \newline
9. स॒मा॒नाना᳚म् करोति करोति समा॒नानाꣳ॑ समा॒नाना᳚म् करोति । \newline
10. क॒रो॒ति॒ त्रि॒ष्टुभ॑ स्त्रि॒ष्टुभः॑ करोति करोति त्रि॒ष्टुभः॑ । \newline
11. त्रि॒ष्टुभ॒ उपोप॑ त्रि॒ष्टुभ॑ स्त्रि॒ष्टुभ॒ उप॑ । \newline
12. उप॑ दधाति दधा॒ त्युपोप॑ दधाति । \newline
13. द॒धा॒ती॒ न्द्रि॒य मि॑न्द्रि॒यम् द॑धाति दधाती न्द्रि॒यम् । \newline
14. इ॒न्द्रि॒यं ॅवै वा इ॑न्द्रि॒य मि॑न्द्रि॒यं ॅवै । \newline
15. वै त्रि॒ष्टुक् त्रि॒ष्टुग् वै वै त्रि॒ष्टुक् । \newline
16. त्रि॒ष्टु गि॑न्द्रि॒य मि॑न्द्रि॒यम् त्रि॒ष्टुक् त्रि॒ष्टु गि॑न्द्रि॒यम् । \newline
17. इ॒न्द्रि॒य मे॒वैवेन्द्रि॒य मि॑न्द्रि॒य मे॒व । \newline
18. ए॒व म॑द्ध्य॒तो म॑द्ध्य॒त ए॒वैव म॑द्ध्य॒तः । \newline
19. म॒द्ध्य॒तो ध॑त्ते धत्ते मद्ध्य॒तो म॑द्ध्य॒तो ध॑त्ते । \newline
20. ध॒त्ते॒ जग॑ती॒र् जग॑तीर् धत्ते धत्ते॒ जग॑तीः । \newline
21. जग॑ती॒ रुपोप॒ जग॑ती॒र् जग॑ती॒ रुप॑ । \newline
22. उप॑ दधाति दधा॒ त्युपोप॑ दधाति । \newline
23. द॒धा॒ति॒ जाग॑ता॒ जाग॑ता दधाति दधाति॒ जाग॑ताः । \newline
24. जाग॑ता॒ वै वै जाग॑ता॒ जाग॑ता॒ वै । \newline
25. वै प॒शवः॑ प॒शवो॒ वै वै प॒शवः॑ । \newline
26. प॒शवः॑ प॒शून् प॒शून् प॒शवः॑ प॒शवः॑ प॒शून् । \newline
27. प॒शू ने॒वैव प॒शून् प॒शू ने॒व । \newline
28. ए॒वावा वै॒वै वाव॑ । \newline
29. अव॑ रुन्धे रु॒न्धे ऽवाव॑ रुन्धे । \newline
30. रु॒न्धे॒ ऽनु॒ष्टुभो॑ ऽनु॒ष्टुभो॑ रुन्धे रुन्धे ऽनु॒ष्टुभः॑ । \newline
31. अ॒नु॒ष्टुभ॒ उपोपा॑ नु॒ष्टुभो॑ ऽनु॒ष्टुभ॒ उप॑ । \newline
32. अ॒नु॒ष्टुभ॒ इत्य॑नु - स्तुभः॑ । \newline
33. उप॑ दधाति दधा॒ त्युपोप॑ दधाति । \newline
34. द॒धा॒ति॒ प्रा॒णाः प्रा॒णा द॑धाति दधाति प्रा॒णाः । \newline
35. प्रा॒णा वै वै प्रा॒णाः प्रा॒णा वै । \newline
36. प्रा॒णा इति॑ प्र - अ॒नाः । \newline
37. वा अ॑नु॒ष्टु ब॑नु॒ष्टुब् वै वा अ॑नु॒ष्टुप् । \newline
38. अ॒नु॒ष्टुप् प्रा॒णाना᳚म् प्रा॒णाना॑ मनु॒ष्टु ब॑नु॒ष्टुप् प्रा॒णाना᳚म् । \newline
39. अ॒नु॒ष्टुबित्य॑नु - स्तुप् । \newline
40. प्रा॒णाना॒ मुथ्सृ॑ष्ट्या॒ उथ्सृ॑ष्ट्यै प्रा॒णाना᳚म् प्रा॒णाना॒ मुथ्सृ॑ष्ट्यै । \newline
41. प्रा॒णाना॒मिति॑ प्र - अ॒नाना᳚म् । \newline
42. उथ्सृ॑ष्ट्यै बृह॒तीर् बृ॑ह॒ती रुथ्सृ॑ष्ट्या॒ उथ्सृ॑ष्ट्यै बृह॒तीः । \newline
43. उथ्सृ॑ष्ट्या॒ इत्युत् - सृ॒ष्ट्यै॒ । \newline
44. बृ॒ह॒ती रु॒ष्णिहा॑ उ॒ष्णिहा॑ बृह॒तीर् बृ॑ह॒ती रु॒ष्णिहाः᳚ । \newline
45. उ॒ष्णिहाः᳚ प॒ङ्क्तीः प॒ङ्क्ती रु॒ष्णिहा॑ उ॒ष्णिहाः᳚ प॒ङ्क्तीः । \newline
46. प॒ङ्क्ती र॒क्षर॑पङ्क्ती र॒क्षर॑पङ्क्तीः प॒ङ्क्तीः प॒ङ्क्ती र॒क्षर॑पङ्क्तीः । \newline
47. अ॒क्षर॑पङ्क्ती॒ रिती त्य॒क्षर॑पङ्क्ती र॒क्षर॑पङ्क्ती॒ रिति॑ । \newline
48. अ॒क्षर॑पङ्क्ती॒रित्य॒क्षर॑ - प॒ङ्क्तीः॒ । \newline
49. इति॒ विषु॑रूपाणि॒ विषु॑रूपा॒णीतीति॒ विषु॑रूपाणि । \newline
50. विषु॑रूपाणि॒ छन्दाꣳ॑सि॒ छन्दाꣳ॑सि॒ विषु॑रूपाणि॒ विषु॑रूपाणि॒ छन्दाꣳ॑सि । \newline
51. विषु॑रूपा॒णीति॒ विषु॑ - रू॒पा॒णि॒ । \newline
52. छन्दाꣳ॒॒ स्युपोप॒ छन्दाꣳ॑सि॒ छन्दाꣳ॒॒ स्युप॑ । \newline
53. उप॑ दधाति दधा॒ त्युपोप॑ दधाति । \newline
54. द॒धा॒ति॒ विषु॑रूपा॒ विषु॑रूपा दधाति दधाति॒ विषु॑रूपाः । \newline
55. विषु॑रूपा॒ वै वै विषु॑रूपा॒ विषु॑रूपा॒ वै । \newline
56. विषु॑रूपा॒ इति॒ विषु॑ - रू॒पाः॒ । \newline
57. वै प॒शवः॑ प॒शवो॒ वै वै प॒शवः॑ । \newline
58. प॒शवः॑ प॒शवः॑ । \newline
59. प॒शव॒ श्छन्दाꣳ॑सि॒ छन्दाꣳ॑सि प॒शवः॑ प॒शव॒ श्छन्दाꣳ॑सि । \newline

\textbf{Ghana Paata } \newline

1. मु॒ख॒तो ध॑त्ते धत्ते मुख॒तो मु॑ख॒तो ध॑त्ते मूर्द्ध॒न्वती᳚र् मूर्द्ध॒न्वती᳚र् धत्ते मुख॒तो मु॑ख॒तो ध॑त्ते मूर्द्ध॒न्वतीः᳚ । \newline
2. ध॒त्ते॒ मू॒र्द्ध॒न्वती᳚र् मूर्द्ध॒न्वती᳚र् धत्ते धत्ते मूर्द्ध॒न्वती᳚र् भवन्ति भवन्ति मूर्द्ध॒न्वती᳚र् धत्ते धत्ते मूर्द्ध॒न्वती᳚र् भवन्ति । \newline
3. मू॒र्द्ध॒न्वती᳚र् भवन्ति भवन्ति मूर्द्ध॒न्वती᳚र् मूर्द्ध॒न्वती᳚र् भवन्ति मू॒र्द्धान॑म् मू॒र्द्धान॑म् भवन्ति मूर्द्ध॒न्वती᳚र् मूर्द्ध॒न्वती᳚र् भवन्ति मू॒र्द्धान᳚म् । \newline
4. मू॒र्द्ध॒न्वती॒रिति॑ मूर्धन्न् - वतीः᳚ । \newline
5. भ॒व॒न्ति॒ मू॒र्द्धान॑म् मू॒र्द्धान॑म् भवन्ति भवन्ति मू॒र्द्धान॑ मे॒वैव मू॒र्द्धान॑म् भवन्ति भवन्ति मू॒र्द्धान॑ मे॒व । \newline
6. मू॒र्द्धान॑ मे॒वैव मू॒र्द्धान॑म् मू॒र्द्धान॑ मे॒वैन॑ मेन मे॒व मू॒र्द्धान॑म् मू॒र्द्धान॑ मे॒वैन᳚म् । \newline
7. ए॒वैन॑ मेन मे॒वैवैनꣳ॑ समा॒नानाꣳ॑ समा॒नाना॑ मेन मे॒वैवैनꣳ॑ समा॒नाना᳚म् । \newline
8. ए॒नꣳ॒॒ स॒मा॒नानाꣳ॑ समा॒नाना॑ मेन मेनꣳ समा॒नाना᳚म् करोति करोति समा॒नाना॑ मेन मेनꣳ समा॒नाना᳚म् करोति । \newline
9. स॒मा॒नाना᳚म् करोति करोति समा॒नानाꣳ॑ समा॒नाना᳚म् करोति त्रि॒ष्टुभ॑ स्त्रि॒ष्टुभः॑ करोति समा॒नानाꣳ॑ समा॒नाना᳚म् करोति त्रि॒ष्टुभः॑ । \newline
10. क॒रो॒ति॒ त्रि॒ष्टुभ॑ स्त्रि॒ष्टुभः॑ करोति करोति त्रि॒ष्टुभ॒ उपोप॑ त्रि॒ष्टुभः॑ करोति करोति त्रि॒ष्टुभ॒ उप॑ । \newline
11. त्रि॒ष्टुभ॒ उपोप॑ त्रि॒ष्टुभ॑ स्त्रि॒ष्टुभ॒ उप॑ दधाति दधा॒ त्युप॑ त्रि॒ष्टुभ॑ स्त्रि॒ष्टुभ॒ उप॑ दधाति । \newline
12. उप॑ दधाति दधा॒ त्युपोप॑ दधाती न्द्रि॒य मि॑न्द्रि॒यम् द॑धा॒ त्युपोप॑ दधाती न्द्रि॒यम् । \newline
13. द॒धा॒ती॒ न्द्रि॒य मि॑न्द्रि॒यम् द॑धाति दधाती न्द्रि॒यं ॅवै वा इ॑न्द्रि॒यम् द॑धाति दधाती न्द्रि॒यं ॅवै । \newline
14. इ॒न्द्रि॒यं ॅवै वा इ॑न्द्रि॒य मि॑न्द्रि॒यं ॅवै त्रि॒ष्टुक् त्रि॒ष्टुग् वा इ॑न्द्रि॒य मि॑न्द्रि॒यं ॅवै त्रि॒ष्टुक् । \newline
15. वै त्रि॒ष्टुक् त्रि॒ष्टुग् वै वै त्रि॒ष्टु गि॑न्द्रि॒य मि॑न्द्रि॒यम् त्रि॒ष्टुग् वै वै त्रि॒ष्टु गि॑न्द्रि॒यम् । \newline
16. त्रि॒ष्टु गि॑न्द्रि॒य मि॑न्द्रि॒यम् त्रि॒ष्टुक् त्रि॒ष्टु गि॑न्द्रि॒य मे॒वैवेन्द्रि॒यम् त्रि॒ष्टुक् त्रि॒ष्टु गि॑न्द्रि॒य मे॒व । \newline
17. इ॒न्द्रि॒य मे॒वैवेन्द्रि॒य मि॑न्द्रि॒य मे॒व म॑द्ध्य॒तो म॑द्ध्य॒त ए॒वेन्द्रि॒य मि॑न्द्रि॒य मे॒व म॑द्ध्य॒तः । \newline
18. ए॒व म॑द्ध्य॒तो म॑द्ध्य॒त ए॒वैव म॑द्ध्य॒तो ध॑त्ते धत्ते मद्ध्य॒त ए॒वैव म॑द्ध्य॒तो ध॑त्ते । \newline
19. म॒द्ध्य॒तो ध॑त्ते धत्ते मद्ध्य॒तो म॑द्ध्य॒तो ध॑त्ते॒ जग॑ती॒र् जग॑तीर् धत्ते मद्ध्य॒तो म॑द्ध्य॒तो ध॑त्ते॒ जग॑तीः । \newline
20. ध॒त्ते॒ जग॑ती॒र् जग॑तीर् धत्ते धत्ते॒ जग॑ती॒ रुपोप॒ जग॑तीर् धत्ते धत्ते॒ जग॑ती॒ रुप॑ । \newline
21. जग॑ती॒ रुपोप॒ जग॑ती॒र् जग॑ती॒ रुप॑ दधाति दधा॒ त्युप॒ जग॑ती॒र् जग॑ती॒ रुप॑ दधाति । \newline
22. उप॑ दधाति दधा॒ त्युपोप॑ दधाति॒ जाग॑ता॒ जाग॑ता दधा॒ त्युपोप॑ दधाति॒ जाग॑ताः । \newline
23. द॒धा॒ति॒ जाग॑ता॒ जाग॑ता दधाति दधाति॒ जाग॑ता॒ वै वै जाग॑ता दधाति दधाति॒ जाग॑ता॒ वै । \newline
24. जाग॑ता॒ वै वै जाग॑ता॒ जाग॑ता॒ वै प॒शवः॑ प॒शवो॒ वै जाग॑ता॒ जाग॑ता॒ वै प॒शवः॑ । \newline
25. वै प॒शवः॑ प॒शवो॒ वै वै प॒शवः॑ प॒शून् प॒शून् प॒शवो॒ वै वै प॒शवः॑ प॒शून् । \newline
26. प॒शवः॑ प॒शून् प॒शून् प॒शवः॑ प॒शवः॑ प॒शू ने॒वैव प॒शून् प॒शवः॑ प॒शवः॑ प॒शू ने॒व । \newline
27. प॒शू ने॒वैव प॒शून् प॒शू ने॒वावा वै॒व प॒शून् प॒शू ने॒वाव॑ । \newline
28. ए॒वावा वै॒वै वाव॑ रुन्धे रु॒न्धे ऽवै॒वै वाव॑ रुन्धे । \newline
29. अव॑ रुन्धे रु॒न्धे ऽवाव॑ रुन्धे ऽनु॒ष्टुभो॑ ऽनु॒ष्टुभो॑ रु॒न्धे ऽवाव॑ रुन्धे ऽनु॒ष्टुभः॑ । \newline
30. रु॒न्धे॒ ऽनु॒ष्टुभो॑ ऽनु॒ष्टुभो॑ रुन्धे रुन्धे ऽनु॒ष्टुभ॒ उपोपा॑ नु॒ष्टुभो॑ रुन्धे रुन्धे ऽनु॒ष्टुभ॒ उप॑ । \newline
31. अ॒नु॒ष्टुभ॒ उपोपा॑ नु॒ष्टुभो॑ ऽनु॒ष्टुभ॒ उप॑ दधाति दधा॒ त्युपा॑ नु॒ष्टुभो॑ ऽनु॒ष्टुभ॒ उप॑ दधाति । \newline
32. अ॒नु॒ष्टुभ॒ इत्य॑नु - स्तुभः॑ । \newline
33. उप॑ दधाति दधा॒ त्युपोप॑ दधाति प्रा॒णाः प्रा॒णा द॑धा॒ त्युपोप॑ दधाति प्रा॒णाः । \newline
34. द॒धा॒ति॒ प्रा॒णाः प्रा॒णा द॑धाति दधाति प्रा॒णा वै वै प्रा॒णा द॑धाति दधाति प्रा॒णा वै । \newline
35. प्रा॒णा वै वै प्रा॒णाः प्रा॒णा वा अ॑नु॒ष्टु ब॑नु॒ष्टुब् वै प्रा॒णाः प्रा॒णा वा अ॑नु॒ष्टुप् । \newline
36. प्रा॒णा इति॑ प्र - अ॒नाः । \newline
37. वा अ॑नु॒ष्टु ब॑नु॒ष्टुब् वै वा अ॑नु॒ष्टुप् प्रा॒णाना᳚म् प्रा॒णाना॑ मनु॒ष्टुब् वै वा अ॑नु॒ष्टुप् प्रा॒णाना᳚म् । \newline
38. अ॒नु॒ष्टुप् प्रा॒णाना᳚म् प्रा॒णाना॑ मनु॒ष्टु ब॑नु॒ष्टुप् प्रा॒णाना॒ मुथ्सृ॑ष्ट्या॒ उथ्सृ॑ष्ट्यै प्रा॒णाना॑ मनु॒ष्टु ब॑नु॒ष्टुप् प्रा॒णाना॒ मुथ्सृ॑ष्ट्यै । \newline
39. अ॒नु॒ष्टुबित्य॑नु - स्तुप् । \newline
40. प्रा॒णाना॒ मुथ्सृ॑ष्ट्या॒ उथ्सृ॑ष्ट्यै प्रा॒णाना᳚म् प्रा॒णाना॒ मुथ्सृ॑ष्ट्यै बृह॒तीर् बृ॑ह॒ती रुथ्सृ॑ष्ट्यै प्रा॒णाना᳚म् प्रा॒णाना॒ मुथ्सृ॑ष्ट्यै बृह॒तीः । \newline
41. प्रा॒णाना॒मिति॑ प्र - अ॒नाना᳚म् । \newline
42. उथ्सृ॑ष्ट्यै बृह॒तीर् बृ॑ह॒ती रुथ्सृ॑ष्ट्या॒ उथ्सृ॑ष्ट्यै बृह॒ती रु॒ष्णिहा॑ उ॒ष्णिहा॑ बृह॒ती रुथ्सृ॑ष्ट्या॒ उथ्सृ॑ष्ट्यै बृह॒ती रु॒ष्णिहाः᳚ । \newline
43. उथ्सृ॑ष्ट्या॒ इत्युत् - सृ॒ष्ट्यै॒ । \newline
44. बृ॒ह॒ती रु॒ष्णिहा॑ उ॒ष्णिहा॑ बृह॒तीर् बृ॑ह॒ती रु॒ष्णिहाः᳚ प॒ङ्क्तीः प॒ङ्क्ती रु॒ष्णिहा॑ बृह॒तीर् बृ॑ह॒ती रु॒ष्णिहाः᳚ प॒ङ्क्तीः । \newline
45. उ॒ष्णिहाः᳚ प॒ङ्क्तीः प॒ङ्क्ती रु॒ष्णिहा॑ उ॒ष्णिहाः᳚ प॒ङ्क्ती र॒क्षर॑पङ्क्ती र॒क्षर॑पङ्क्तीः प॒ङ्क्ती रु॒ष्णिहा॑ उ॒ष्णिहाः᳚ प॒ङ्क्ती र॒क्षर॑पङ्क्तीः । \newline
46. प॒ङ्क्ती र॒क्षर॑पङ्क्ती र॒क्षर॑पङ्क्तीः प॒ङ्क्तीः प॒ङ्क्ती र॒क्षर॑पङ्क्ती॒ रिती त्य॒क्षर॑पङ्क्तीः प॒ङ्क्तीः प॒ङ्क्ती र॒क्षर॑पङ्क्ती॒ रिति॑ । \newline
47. अ॒क्षर॑पङ्क्ती॒ रिती त्य॒क्षर॑पङ्क्ती र॒क्षर॑पङ्क्ती॒ रिति॒ विषु॑रूपाणि॒ विषु॑रूपा॒णी त्य॒क्षर॑पङ्क्ती र॒क्षर॑पङ्क्ती॒ रिति॒ विषु॑रूपाणि । \newline
48. अ॒क्षर॑पङ्क्ती॒रित्य॒क्षर॑ - प॒ङ्क्तीः॒ । \newline
49. इति॒ विषु॑रूपाणि॒ विषु॑रूपा॒णीतीति॒ विषु॑रूपाणि॒ छन्दाꣳ॑सि॒ छन्दाꣳ॑सि॒ विषु॑रूपा॒णीतीति॒ विषु॑रूपाणि॒ छन्दाꣳ॑सि । \newline
50. विषु॑रूपाणि॒ छन्दाꣳ॑सि॒ छन्दाꣳ॑सि॒ विषु॑रूपाणि॒ विषु॑रूपाणि॒ छन्दाꣳ॒॒ स्युपोप॒ च्छन्दाꣳ॑सि॒ विषु॑रूपाणि॒ विषु॑रूपाणि॒ छन्दाꣳ॒॒स्युप॑ । \newline
51. विषु॑रूपा॒णीति॒ विषु॑ - रू॒पा॒णि॒ । \newline
52. छन्दाꣳ॒॒ स्युपोप॒ च्छन्दाꣳ॑सि॒ छन्दाꣳ॒॒ स्युप॑ दधाति दधा॒त्युप॒ च्छन्दाꣳ॑सि॒ छन्दाꣳ॒॒ स्युप॑ दधाति । \newline
53. उप॑ दधाति दधा॒ त्युपोप॑ दधाति॒ विषु॑रूपा॒ विषु॑रूपा दधा॒ त्युपोप॑ दधाति॒ विषु॑रूपाः । \newline
54. द॒धा॒ति॒ विषु॑रूपा॒ विषु॑रूपा दधाति दधाति॒ विषु॑रूपा॒ वै वै विषु॑रूपा दधाति दधाति॒ विषु॑रूपा॒ वै । \newline
55. विषु॑रूपा॒ वै वै विषु॑रूपा॒ विषु॑रूपा॒ वै प॒शवः॑ प॒शवो॒ वै विषु॑रूपा॒ विषु॑रूपा॒ वै प॒शवः॑ । \newline
56. विषु॑रूपा॒ इति॒ विषु॑ - रू॒पाः॒ । \newline
57. वै प॒शवः॑ प॒शवो॒ वै वै प॒शवः॑ । \newline
58. प॒शवः॑ प॒शवः॑ । \newline
59. प॒शव॒ श्छन्दाꣳ॑सि॒ छन्दाꣳ॑सि प॒शवः॑ प॒शव॒ श्छन्दाꣳ॑सि॒ विषु॑रूपा॒न्॒. विषु॑रूपा॒न् छन्दाꣳ॑सि प॒शवः॑ प॒शव॒ श्छन्दाꣳ॑सि॒ विषु॑रूपान् । \newline
\pagebreak
\markright{ TS 5.3.8.3  \hfill https://www.vedavms.in \hfill}

\section{ TS 5.3.8.3 }

\textbf{TS 5.3.8.3 } \newline
\textbf{Samhita Paata} \newline

-छन्दाꣳ॑सि॒ विषु॑रूपाने॒व प॒शूनव॑ रुन्धे॒ विषु॑रूपमस्य गृ॒हे दृ॑श्यते॒ यस्यै॒ता उ॑पधी॒यन्ते॒ य उ॑ चैना ए॒वं ॅवेदाऽ*ति॑च्छन्दस॒मुप॑ दधा॒त्यति॑च्छन्दा॒ वै सर्वा॑णि॒ छन्दाꣳ॑सि॒ सर्वे॑भिरे॒वैनं॒ छन्दो॑भिश्चिनुते॒ वर्ष्म॒ वा ए॒षा छन्द॑सां॒ ॅयदति॑च्छन्दा॒ यदति॑च्छन्दस-मुप॒दधा॑ति॒ वर्ष्मै॒वैनꣳ॑ समा॒नानां᳚ करोति द्वि॒पदा॒ उप॑ दधाति द्वि॒पाद्-यज॑मानः॒ ( ) प्रति॑ष्ठित्यै ॥ \newline

\textbf{Pada Paata} \newline

छन्दाꣳ॑सि । विषु॑रूपा॒निति॒ विषु॑ - रू॒पा॒न् । ए॒व । प॒शून् । अवेति॑ । रु॒न्धे॒ । विषु॑रूप॒मिति॒ विषु॑-रू॒प॒म् । अ॒स्य॒ । गृ॒हे । दृ॒श्य॒ते॒ । यस्य॑ । ए॒ताः । उ॒प॒धी॒यन्त॒ इत्यु॑प - धी॒यन्ते᳚ । यः । उ॒ । च॒ । ए॒नाः॒ । ए॒वम् । वेद॑ । अति॑च्छन्दस॒मित्यति॑ - छ॒न्द॒स॒म् । उपेति॑ । द॒धा॒ति॒ । अति॑च्छन्दा॒ इत्यति॑ - छ॒न्दाः॒ । वै । सर्वा॑णि । छन्दाꣳ॑सि । सर्वे॑भिः । ए॒व । ए॒न॒म् । छन्दो॑भि॒रिति॒ छन्दः॑-भिः॒ । चि॒नु॒ते॒ । वर्ष्म॑ । वै । ए॒षा । छन्द॑साम् । यत् । अति॑च्छन्दा॒ इत्यति॑ - छ॒न्दाः॒ । यत् । अति॑च्छन्दस॒मित्यति॑ - छ॒न्द॒स॒म् । उ॒प॒दधा॒तीत्यु॑प - दधा॑ति । वर्ष्म॑ । ए॒व । ए॒न॒म् । स॒मा॒नाना᳚म् । क॒रो॒ति॒ । द्वि॒पदा॒ इति॑ द्वि - पदाः᳚ । उपेति॑ । द॒धा॒ति॒ । द्वि॒पादिति॑ द्वि - पात् । यज॑मानः ( ) । प्रति॑ष्ठित्या॒ इति॒ प्रति॑ - स्थि॒त्यै॒ ॥  \newline


\textbf{Krama Paata} \newline

छन्दाꣳ॑सि॒ विषु॑रूपान् । विषु॑रूपाने॒व । विषु॑रूपा॒निति॒ विषु॑ - रू॒पा॒न्॒ । ए॒व प॒शून् । प॒शूनव॑ । अव॑ रुन्धे । रु॒न्धे॒ विषु॑रूपम् । विषु॑रूपमस्य । विषु॑रूप॒मिति॒ विषु॑ - रू॒प॒म् । अ॒स्य॒ गृ॒हे । गृ॒हे दृ॑श्यते । दृ॒श्य॒ते॒ यस्य॑ । यस्यै॒ताः । ए॒ता उ॑पधी॒यन्ते᳚ । उ॒प॒धी॒यन्ते॒ यः । उ॒प॒धी॒यन्त॒ इत्यु॑प - धी॒यन्ते᳚ । य उ॑ । उ॒ च॒ । चै॒नाः॒ । ए॒ना॒ ए॒वम् । ए॒वम् ॅवेद॑ । वेदाति॑च्छन्दसम् । अति॑च्छन्दस॒मुप॑ । अति॑च्छन्दस॒मित्यति॑ - छ॒न्द॒स॒म् । उप॑ दधाति । द॒धा॒त्यति॑च्छन्दाः । अति॑च्छन्दा॒ वै । अति॑च्छन्दा॒ इत्यति॑ - छ॒न्दाः॒ । वै सर्वा॑णि । सर्वा॑णि॒ छन्दाꣳ॑सि । छन्दाꣳ॑सि॒ सर्वे॑भिः । सर्वे॑भिरे॒व । ए॒वैन᳚म् । ए॒न॒म् छन्दो॑भिः । छन्दो॑भिश्चिनुते । छन्दो॑भि॒रिति॒ छन्दः॑ - भिः॒ । चि॒नु॒ते॒ वर्ष्म॑ । वर्ष्म॒ वै । वा ए॒षा । ए॒षा छन्द॑साम् । छन्द॑सा॒म् ॅयत् । यदति॑च्छन्दाः । अति॑च्छन्दा॒ यत् । अति॑च्छन्दा॒ इत्यति॑ - छ॒न्दाः॒ । यदति॑च्छन्दसम् । अति॑च्छन्दसमुप॒दधा॑ति । अति॑च्छन्दस॒मित्यति॑ - छ॒न्द॒स॒म् । उ॒प॒दधा॑ति॒ वर्ष्म॑ । उ॒प॒दधा॒तीत्यु॑प - दधा॑ति । वर्ष्मै॒व । ए॒वैन᳚म् । ए॒नꣳ॒॒ स॒मा॒नाना᳚म् । स॒मा॒नाना᳚म् करोति । क॒रो॒ति॒ द्वि॒पदाः᳚ । द्वि॒पदा॒ उप॑ । द्वि॒पदा॒ इति॑ द्वि - पदाः᳚ । उप॑ दधाति । द॒धा॒ति॒ द्वि॒पात् । द्वि॒पाद् यज॑मानः ( ) । द्वि॒पादिति॑ द्वि - पात् । यज॑मानः॒ प्रति॑ष्ठित्यै । प्रति॑ष्ठित्या॒ इति॒ प्रति॑ - स्थि॒त्यै॒ । \newline

\textbf{Jatai Paata} \newline

1. छन्दाꣳ॑सि॒ विषु॑रूपा॒न्॒. विषु॑रूपा॒न् छन्दाꣳ॑सि॒ छन्दाꣳ॑सि॒ विषु॑रूपान् । \newline
2. विषु॑रूपा ने॒वैव विषु॑रूपा॒न्॒. विषु॑रूपा ने॒व । \newline
3. विषु॑रूपा॒निति॒ विषु॑ - रू॒पा॒न् । \newline
4. ए॒व प॒शून् प॒शू ने॒वैव प॒शून् । \newline
5. प॒शू नवाव॑ प॒शून् प॒शू नव॑ । \newline
6. अव॑ रुन्धे रु॒न्धे ऽवाव॑ रुन्धे । \newline
7. रु॒न्धे॒ विषु॑रूपं॒ ॅविषु॑रूपꣳ रुन्धे रुन्धे॒ विषु॑रूपम् । \newline
8. विषु॑रूप मस्यास्य॒ विषु॑रूपं॒ ॅविषु॑रूप मस्य । \newline
9. विषु॑रूप॒मिति॒ विषु॑ - रू॒प॒म् । \newline
10. अ॒स्य॒ गृ॒हे गृ॒हे᳚ ऽस्यास्य गृ॒हे । \newline
11. गृ॒हे दृ॑श्यते दृश्यते गृ॒हे गृ॒हे दृ॑श्यते । \newline
12. दृ॒श्य॒ते॒ यस्य॒ यस्य॑ दृश्यते दृश्यते॒ यस्य॑ । \newline
13. यस्यै॒ता ए॒ता यस्य॒ यस्यै॒ताः । \newline
14. ए॒ता उ॑पधी॒यन्त॑ उपधी॒यन्त॑ ए॒ता ए॒ता उ॑पधी॒यन्ते᳚ । \newline
15. उ॒प॒धी॒यन्ते॒ यो य उ॑पधी॒यन्त॑ उपधी॒यन्ते॒ यः । \newline
16. उ॒प॒धी॒यन्त॒ इत्यु॑प - धी॒यन्ते᳚ । \newline
17. य उ॑ वु॒ यो य उ॑ । \newline
18. उ॒ च॒ च॒ वु॒ च॒ । \newline
19. चै॒ना॒ ए॒ना॒श्च॒ चै॒नाः॒ । \newline
20. ए॒ना॒ ए॒व मे॒व मे॑ना एना ए॒वम् । \newline
21. ए॒वं ॅवेद॒ वेदै॒व मे॒वं ॅवेद॑ । \newline
22. वेदा ति॑च्छन्दस॒ मति॑च्छन्दसं॒ ॅवेद॒ वेदा ति॑च्छन्दसम् । \newline
23. अति॑च्छन्दस॒ मुपोपा ति॑च्छन्दस॒ मति॑च्छन्दस॒ मुप॑ । \newline
24. अति॑च्छन्दस॒मित्यति॑ - छ॒न्द॒स॒म् । \newline
25. उप॑ दधाति दधा॒ त्युपोप॑ दधाति । \newline
26. द॒धा॒त्य ति॑च्छन्दा॒ अति॑च्छन्दा दधाति दधा॒त्य ति॑च्छन्दाः । \newline
27. अति॑च्छन्दा॒ वै वा अति॑च्छन्दा॒ अति॑च्छन्दा॒ वै । \newline
28. अति॑च्छन्दा॒ इत्यति॑ - छ॒न्दाः॒ । \newline
29. वै सर्वा॑णि॒ सर्वा॑णि॒ वै वै सर्वा॑णि । \newline
30. सर्वा॑णि॒ छन्दाꣳ॑सि॒ छन्दाꣳ॑सि॒ सर्वा॑णि॒ सर्वा॑णि॒ छन्दाꣳ॑सि । \newline
31. छन्दाꣳ॑सि॒ सर्वे॑भिः॒ सर्वे॑भि॒ श्छन्दाꣳ॑सि॒ छन्दाꣳ॑सि॒ सर्वे॑भिः । \newline
32. सर्वे॑भि रे॒वैव सर्वे॑भिः॒ सर्वे॑भि रे॒व । \newline
33. ए॒वैन॑ मेन मे॒वै वैन᳚म् । \newline
34. ए॒न॒म् छन्दो॑भि॒ श्छन्दो॑भि रेन मेन॒म् छन्दो॑भिः । \newline
35. छन्दो॑भि श्चिनुते चिनुते॒ छन्दो॑भि॒ श्छन्दो॑भि श्चिनुते । \newline
36. छन्दो॑भि॒रिति॒ छन्दः॑ - भिः॒ । \newline
37. चि॒नु॒ते॒ वर्ष्म॒ वर्ष्म॑ चिनुते चिनुते॒ वर्ष्म॑ । \newline
38. वर्ष्म॒ वै वै वर्ष्म॒ वर्ष्म॒ वै । \newline
39. वा ए॒षैषा वै वा ए॒षा । \newline
40. ए॒षा छन्द॑सा॒म् छन्द॑सा मे॒षैषा छन्द॑साम् । \newline
41. छन्द॑सां॒ ॅयद् यच् छन्द॑सा॒म् छन्द॑सां॒ ॅयत् । \newline
42. यदति॑च्छन्दा॒ अति॑च्छन्दा॒ यद् यदति॑च्छन्दाः । \newline
43. अति॑च्छन्दा॒ यद् यदति॑च्छन्दा॒ अति॑च्छन्दा॒ यत् । \newline
44. अति॑च्छन्दा॒ इत्यति॑ - छ॒न्दाः॒ । \newline
45. यदति॑च्छन्दस॒ मति॑च्छन्दसं॒ ॅयद् यदति॑च्छन्दसम् । \newline
46. अति॑च्छन्दस मुप॒दधा᳚ त्युप॒दधा॒ त्यति॑च्छन्दस॒ मति॑च्छन्दस मुप॒दधा॑ति । \newline
47. अति॑च्छन्दस॒मित्यति॑ - छ॒न्द॒स॒म् । \newline
48. उ॒प॒दधा॑ति॒ वर्ष्म॒ वर्ष्मो॑ प॒दधा᳚ त्युप॒दधा॑ति॒ वर्ष्म॑ । \newline
49. उ॒प॒दधा॒तीत्यु॑प - दधा॑ति । \newline
50. वर्ष्मै॒ वैव वर्ष्म॒ वर्ष्मै॒व । \newline
51. ए॒वैन॑ मेन मे॒वै वैन᳚म् । \newline
52. ए॒नꣳ॒॒ स॒मा॒नानाꣳ॑ समा॒नाना॑ मेन मेनꣳ समा॒नाना᳚म् । \newline
53. स॒मा॒नाना᳚म् करोति करोति समा॒नानाꣳ॑ समा॒नाना᳚म् करोति । \newline
54. क॒रो॒ति॒ द्वि॒पदा᳚ द्वि॒पदाः᳚ करोति करोति द्वि॒पदाः᳚ । \newline
55. द्वि॒पदा॒ उपोप॑ द्वि॒पदा᳚ द्वि॒पदा॒ उप॑ । \newline
56. द्वि॒पदा॒ इति॑ द्वि - पदाः᳚ । \newline
57. उप॑ दधाति दधा॒ त्युपोप॑ दधाति । \newline
58. द॒धा॒ति॒ द्वि॒पाद् द्वि॒पाद् द॑धाति दधाति द्वि॒पात् । \newline
59. द्वि॒पाद् यज॑मानो॒ यज॑मानो द्वि॒पाद् द्वि॒पाद् यज॑मानः । \newline
60. द्वि॒पादिति॑ द्वि - पात् । \newline
61. यज॑मानः॒ प्रति॑ष्ठित्यै॒ प्रति॑ष्ठित्यै॒ यज॑मानो॒ यज॑मानः॒ प्रति॑ष्ठित्यै । \newline
62. प्रति॑ष्ठित्या॒ इति॒ प्रति॑ - स्थि॒त्यै॒ । \newline

\textbf{Ghana Paata } \newline

1. छन्दाꣳ॑सि॒ विषु॑रूपा॒न्॒. विषु॑रूपा॒न् छन्दाꣳ॑सि॒ छन्दाꣳ॑सि॒ विषु॑रूपा ने॒वैव विषु॑रूपा॒न् छन्दाꣳ॑सि॒ छन्दाꣳ॑सि॒ विषु॑रूपा ने॒व । \newline
2. विषु॑रूपा ने॒वैव विषु॑रूपा॒न्॒. विषु॑रूपा ने॒व प॒शून् प॒शू ने॒व विषु॑रूपा॒न्॒. विषु॑रूपा ने॒व प॒शून् । \newline
3. विषु॑रूपा॒निति॒ विषु॑ - रू॒पा॒न् । \newline
4. ए॒व प॒शून् प॒शू ने॒वैव प॒शू नवाव॑ प॒शू ने॒वैव प॒शू नव॑ । \newline
5. प॒शू नवाव॑ प॒शून् प॒शू नव॑ रुन्धे रु॒न्धे ऽव॑ प॒शून् प॒शू नव॑ रुन्धे । \newline
6. अव॑ रुन्धे रु॒न्धे ऽवाव॑ रुन्धे॒ विषु॑रूपं॒ ॅविषु॑रूपꣳ रु॒न्धे ऽवाव॑ रुन्धे॒ विषु॑रूपम् । \newline
7. रु॒न्धे॒ विषु॑रूपं॒ ॅविषु॑रूपꣳ रुन्धे रुन्धे॒ विषु॑रूप मस्यास्य॒ विषु॑रूपꣳ रुन्धे रुन्धे॒ विषु॑रूप मस्य । \newline
8. विषु॑रूप मस्यास्य॒ विषु॑रूपं॒ ॅविषु॑रूप मस्य गृ॒हे गृ॒हे᳚ ऽस्य॒ विषु॑रूपं॒ ॅविषु॑रूप मस्य गृ॒हे । \newline
9. विषु॑रूप॒मिति॒ विषु॑ - रू॒प॒म् । \newline
10. अ॒स्य॒ गृ॒हे गृ॒हे᳚ ऽस्यास्य गृ॒हे दृ॑श्यते दृश्यते गृ॒हे᳚ ऽस्यास्य गृ॒हे दृ॑श्यते । \newline
11. गृ॒हे दृ॑श्यते दृश्यते गृ॒हे गृ॒हे दृ॑श्यते॒ यस्य॒ यस्य॑ दृश्यते गृ॒हे गृ॒हे दृ॑श्यते॒ यस्य॑ । \newline
12. दृ॒श्य॒ते॒ यस्य॒ यस्य॑ दृश्यते दृश्यते॒ यस्यै॒ता ए॒ता यस्य॑ दृश्यते दृश्यते॒ यस्यै॒ताः । \newline
13. यस्यै॒ता ए॒ता यस्य॒ यस्यै॒ता उ॑पधी॒यन्त॑ उपधी॒यन्त॑ ए॒ता यस्य॒ यस्यै॒ता उ॑पधी॒यन्ते᳚ । \newline
14. ए॒ता उ॑पधी॒यन्त॑ उपधी॒यन्त॑ ए॒ता ए॒ता उ॑पधी॒यन्ते॒ यो य उ॑पधी॒यन्त॑ ए॒ता ए॒ता उ॑पधी॒यन्ते॒ यः । \newline
15. उ॒प॒धी॒यन्ते॒ यो य उ॑पधी॒यन्त॑ उपधी॒यन्ते॒ य उ॑ वु॒ य उ॑पधी॒यन्त॑ उपधी॒यन्ते॒ य उ॑ । \newline
16. उ॒प॒धी॒यन्त॒ इत्यु॑प - धी॒यन्ते᳚ । \newline
17. य उ॑ वु॒ यो य उ॑ च चो॒ यो य उ॑ च । \newline
18. उ॒ च॒ च॒ वु॒ चै॒ना॒ ए॒ना॒श्च॒ वु॒ चै॒नाः॒ । \newline
19. चै॒ना॒ ए॒ना॒श्च॒ चै॒ना॒ ए॒व मे॒व मे॑नाश्च चैना ए॒वम् । \newline
20. ए॒ना॒ ए॒व मे॒व मे॑ना एना ए॒वं ॅवेद॒ वेदै॒व मे॑ना एना ए॒वं ॅवेद॑ । \newline
21. ए॒वं ॅवेद॒ वेदै॒व मे॒वं ॅवेदाति॑च्छन्दस॒ मति॑च्छन्दसं॒ ॅवेदै॒व मे॒वं ॅवेदाति॑च्छन्दसम् । \newline
22. वेदाति॑च्छन्दस॒ मति॑च्छन्दसं॒ ॅवेद॒ वेदाति॑च्छन्दस॒ मुपोपाति॑च्छन्दसं॒ ॅवेद॒ वेदाति॑च्छन्दस॒ मुप॑ । \newline
23. अति॑च्छन्दस॒ मुपोपाति॑च्छन्दस॒ मति॑च्छन्दस॒ मुप॑ दधाति दधा॒ त्युपाति॑च्छन्दस॒ मति॑च्छन्दस॒ मुप॑ दधाति । \newline
24. अति॑च्छन्दस॒मित्यति॑ - छ॒न्द॒स॒म् । \newline
25. उप॑ दधाति दधा॒ त्युपोप॑ दधा॒ त्यति॑च्छन्दा॒ अति॑च्छन्दा दधा॒ त्युपोप॑ दधा॒ त्यति॑च्छन्दाः । \newline
26. द॒धा॒ त्यति॑च्छन्दा॒ अति॑च्छन्दा दधाति दधा॒ त्यति॑च्छन्दा॒ वै वा अति॑च्छन्दा दधाति दधा॒ त्यति॑च्छन्दा॒ वै । \newline
27. अति॑च्छन्दा॒ वै वा अति॑च्छन्दा॒ अति॑च्छन्दा॒ वै सर्वा॑णि॒ सर्वा॑णि॒ वा अति॑च्छन्दा॒ अति॑च्छन्दा॒ वै सर्वा॑णि । \newline
28. अति॑च्छन्दा॒ इत्यति॑ - छ॒न्दाः॒ । \newline
29. वै सर्वा॑णि॒ सर्वा॑णि॒ वै वै सर्वा॑णि॒ छन्दाꣳ॑सि॒ छन्दाꣳ॑सि॒ सर्वा॑णि॒ वै वै सर्वा॑णि॒ छन्दाꣳ॑सि । \newline
30. सर्वा॑णि॒ छन्दाꣳ॑सि॒ छन्दाꣳ॑सि॒ सर्वा॑णि॒ सर्वा॑णि॒ छन्दाꣳ॑सि॒ सर्वे॑भिः॒ सर्वे॑भि॒ श्छन्दाꣳ॑सि॒ सर्वा॑णि॒ सर्वा॑णि॒ छन्दाꣳ॑सि॒ सर्वे॑भिः । \newline
31. छन्दाꣳ॑सि॒ सर्वे॑भिः॒ सर्वे॑भि॒ श्छन्दाꣳ॑सि॒ छन्दाꣳ॑सि॒ सर्वे॑भि रे॒वैव सर्वे॑भि॒ श्छन्दाꣳ॑सि॒ छन्दाꣳ॑सि॒ सर्वे॑भिरे॒व । \newline
32. सर्वे॑भि रे॒वैव सर्वे॑भिः॒ सर्वे॑भि रे॒वैन॑ मेन मे॒व सर्वे॑भिः॒ सर्वे॑भि रे॒वैन᳚म् । \newline
33. ए॒वैन॑ मेन मे॒वै वैन॒म् छन्दो॑भि॒ श्छन्दो॑भि रेन मे॒वै वैन॒म् छन्दो॑भिः । \newline
34. ए॒न॒म् छन्दो॑भि॒ श्छन्दो॑भि रेन मेन॒म् छन्दो॑भि श्चिनुते चिनुते॒ छन्दो॑भि रेन मेन॒म् छन्दो॑भि श्चिनुते । \newline
35. छन्दो॑भि श्चिनुते चिनुते॒ छन्दो॑भि॒ श्छन्दो॑भि श्चिनुते॒ वर्ष्म॒ वर्ष्म॑ चिनुते॒ छन्दो॑भि॒ श्छन्दो॑भि श्चिनुते॒ वर्ष्म॑ । \newline
36. छन्दो॑भि॒रिति॒ छन्दः॑ - भिः॒ । \newline
37. चि॒नु॒ते॒ वर्ष्म॒ वर्ष्म॑ चिनुते चिनुते॒ वर्ष्म॒ वै वै वर्ष्म॑ चिनुते चिनुते॒ वर्ष्म॒ वै । \newline
38. वर्ष्म॒ वै वै वर्ष्म॒ वर्ष्म॒ वा ए॒षैषा वै वर्ष्म॒ वर्ष्म॒ वा ए॒षा । \newline
39. वा ए॒षैषा वै वा ए॒षा छन्द॑सा॒म् छन्द॑सा मे॒षा वै वा ए॒षा छन्द॑साम् । \newline
40. ए॒षा छन्द॑सा॒म् छन्द॑सा मे॒षैषा छन्द॑सां॒ ॅयद् यच् छन्द॑सा मे॒षैषा छन्द॑सां॒ ॅयत् । \newline
41. छन्द॑सां॒ ॅयद् यच् छन्द॑सा॒म् छन्द॑सां॒ ॅयदति॑च्छन्दा॒ अति॑च्छन्दा॒ यच् छन्द॑सा॒म् छन्द॑सां॒ ॅयदति॑च्छन्दाः । \newline
42. यदति॑च्छन्दा॒ अति॑च्छन्दा॒ यद् यदति॑च्छन्दा॒ यद् यदति॑च्छन्दा॒ यद् यदति॑च्छन्दा॒ यत् । \newline
43. अति॑च्छन्दा॒ यद् यदति॑च्छन्दा॒ अति॑च्छन्दा॒ यदति॑च्छन्दस॒ मति॑च्छन्दसं॒ ॅयदति॑च्छन्दा॒ अति॑च्छन्दा॒ यदति॑च्छन्दसम् । \newline
44. अति॑च्छन्दा॒ इत्यति॑ - छ॒न्दाः॒ । \newline
45. यदति॑च्छन्दस॒ मति॑च्छन्दसं॒ ॅयद् यदति॑च्छन्दस मुप॒दधा᳚ त्युप॒दधा॒ त्यति॑च्छन्दसं॒ ॅयद् यदति॑च्छन्दस मुप॒दधा॑ति । \newline
46. अति॑च्छन्दस मुप॒दधा᳚ त्युप॒दधा॒ त्यति॑च्छन्दस॒ मति॑च्छन्दस मुप॒दधा॑ति॒ वर्ष्म॒ वर्ष्मो॑ प॒दधा॒ त्यति॑च्छन्दस॒ मति॑च्छन्दस मुप॒दधा॑ति॒ वर्ष्म॑ । \newline
47. अति॑च्छन्दस॒मित्यति॑ - छ॒न्द॒स॒म् । \newline
48. उ॒प॒दधा॑ति॒ वर्ष्म॒ वर्ष्मो॑ प॒दधा᳚ त्युप॒दधा॑ति॒ वर्ष्मै॒ वैव वर्ष्मो॑ प॒दधा᳚ त्युप॒दधा॑ति॒ वर्ष्मै॒व । \newline
49. उ॒प॒दधा॒तीत्यु॑प - दधा॑ति । \newline
50. वर्ष्मै॒ वैव वर्ष्म॒ वर्ष्मै॒ वैन॑ मेन मे॒व वर्ष्म॒ वर्ष्मै॒ वैन᳚म् । \newline
51. ए॒वैन॑ मेन मे॒वै वैनꣳ॑ समा॒नानाꣳ॑ समा॒नाना॑ मेन मे॒वै वैनꣳ॑ समा॒नाना᳚म् । \newline
52. ए॒नꣳ॒॒ स॒मा॒नानाꣳ॑ समा॒नाना॑ मेन मेनꣳ समा॒नाना᳚म् करोति करोति समा॒नाना॑ मेन मेनꣳ समा॒नाना᳚म् करोति । \newline
53. स॒मा॒नाना᳚म् करोति करोति समा॒नानाꣳ॑ समा॒नाना᳚म् करोति द्वि॒पदा᳚ द्वि॒पदाः᳚ करोति समा॒नानाꣳ॑ समा॒नाना᳚म् करोति द्वि॒पदाः᳚ । \newline
54. क॒रो॒ति॒ द्वि॒पदा᳚ द्वि॒पदाः᳚ करोति करोति द्वि॒पदा॒ उपोप॑ द्वि॒पदाः᳚ करोति करोति द्वि॒पदा॒ उप॑ । \newline
55. द्वि॒पदा॒ उपोप॑ द्वि॒पदा᳚ द्वि॒पदा॒ उप॑ दधाति दधा॒ त्युप॑ द्वि॒पदा᳚ द्वि॒पदा॒ उप॑ दधाति । \newline
56. द्वि॒पदा॒ इति॑ द्वि - पदाः᳚ । \newline
57. उप॑ दधाति दधा॒ त्युपोप॑ दधाति द्वि॒पाद् द्वि॒पाद् द॑धा॒ त्युपोप॑ दधाति द्वि॒पात् । \newline
58. द॒धा॒ति॒ द्वि॒पाद् द्वि॒पाद् द॑धाति दधाति द्वि॒पाद् यज॑मानो॒ यज॑मानो द्वि॒पाद् द॑धाति दधाति द्वि॒पाद् यज॑मानः । \newline
59. द्वि॒पाद् यज॑मानो॒ यज॑मानो द्वि॒पाद् द्वि॒पाद् यज॑मानः॒ प्रति॑ष्ठित्यै॒ प्रति॑ष्ठित्यै॒ यज॑मानो द्वि॒पाद् द्वि॒पाद् यज॑मानः॒ प्रति॑ष्ठित्यै । \newline
60. द्वि॒पादिति॑ द्वि - पात् । \newline
61. यज॑मानः॒ प्रति॑ष्ठित्यै॒ प्रति॑ष्ठित्यै॒ यज॑मानो॒ यज॑मानः॒ प्रति॑ष्ठित्यै । \newline
62. प्रति॑ष्ठित्या॒ इति॒ प्रति॑ - स्थि॒त्यै॒ । \newline
\pagebreak
\markright{ TS 5.3.9.1  \hfill https://www.vedavms.in \hfill}

\section{ TS 5.3.9.1 }

\textbf{TS 5.3.9.1 } \newline
\textbf{Samhita Paata} \newline

सर्वा᳚भ्यो॒ वै दे॒वता᳚भ्यो॒ऽग्निश्ची॑यते॒ यथ् स॒युजो॒ नोप॑द॒द्ध्याद् दे॒वता॑ अस्या॒ग्निं ॅवृ॑ञ्जीर॒न्॒. यथ् स॒युज॑ उप॒दधा᳚त्या॒त्मनै॒वैनꣳ॑ स॒युजं॑ चिनुते॒ नाग्निना॒ व्यृ॑द्ध्य॒तेऽथो॒ यथा॒ पुरु॑षः॒ स्नाव॑भिः॒ संत॑त ए॒वमे॒वैताभि॑र॒ग्निः संत॑तो॒ ऽग्निना॒ वै दे॒वाः सु॑व॒र्गं ॅलो॒कमा॑य॒न् ता अ॒मूः कृत्ति॑का अभव॒न्॒ यस्यै॒ता उ॑प धी॒यन्ते॑ सुव॒र्गमे॒व - [  ] \newline

\textbf{Pada Paata} \newline

सर्वा᳚भ्यः । वै । दे॒वता᳚भ्यः । अ॒ग्निः । ची॒य॒ते॒ । यत् । स॒युज॒ इति॑ स - युजः॑ । न । उ॒प॒द॒द्ध्यादित्यु॑प - द॒ध्यात् । दे॒वताः᳚ । अ॒स्य॒ । अ॒ग्निम् । वृ॒ञ्जी॒र॒न्न् । यत् । स॒युज॒ इति॑ स - युजः॑ । उ॒प॒दधा॒तीत्यु॑प - दधा॑ति । आ॒त्मना᳚ । ए॒व । ए॒न॒म् । स॒युज॒मिति॑ स-युज᳚म् । चि॒नु॒ते॒ । न । अ॒ग्निना᳚ । वीति॑ । ऋ॒द्ध्य॒ते॒ । अथो॒ इति॑ । यथा᳚ । पुरु॑षः । स्नाव॑भि॒रिति॒ स्नाव॑ - भिः॒ । संत॑त॒ इति॒ सं-त॒तः॒ । ए॒वम् । ए॒व । ए॒ताभिः॑ । अ॒ग्निः । संत॑त॒ इति॒ सं-त॒तः॒ । अ॒ग्निना᳚ । वै । दे॒वाः । सु॒व॒र्गमिति॑ सुवः - गम् । लो॒कम् । आ॒य॒न्न् । ताः । अ॒मूः । कृत्ति॑काः । अ॒भ॒व॒न्न् । यस्य॑ । ए॒ताः । उ॒प॒धी॒यन्त॒ इत्यु॑प - धी॒यन्ते᳚ । सु॒व॒र्गमिति॑ सुवः - गम् । ए॒व ।  \newline


\textbf{Krama Paata} \newline

सर्वा᳚भ्यो॒ वै । वै दे॒वता᳚भ्यः । दे॒वता᳚भ्यो॒ऽग्निः । अ॒ग्निश्ची॑यते । ची॒य॒ते॒ यत् । यथ् स॒युजः॑ । स॒युजो॒ न । स॒युज॒ इति॑ स - युजः॑ । नोप॑द॒द्ध्यात् । उ॒प॒द॒द्ध्याद् दे॒वताः᳚ । उ॒प॒द॒द्ध्यादित्यु॑प - द॒द्ध्यात् । दे॒वता॑ अस्य । अ॒स्या॒ग्निम् । अ॒ग्निम् ॅवृ॑ञ्जीरन्न् । वृ॒ञ्जी॒र॒न्॒. यत् । यथ् स॒युजः॑ । स॒युज॑ उप॒दधा॑ति । स॒युज॒ इति॑ स - युजः॑ । उ॒प॒दधा᳚त्या॒त्मना᳚ । उ॒प॒दधा॒तीत्यु॑प - दधा॑ति । आ॒त्मनै॒व । ए॒वैन᳚म् । ए॒नꣳ॒॒ स॒युज᳚म् । स॒युज॑म् चिनुते । स॒युज॒मिति॑ स - युज᳚म् । चि॒नु॒ते॒ न । नाग्निना᳚ । अ॒ग्निना॒ वि । व्यृ॑द्ध्यते । ऋ॒द्ध्य॒तेऽथो᳚ । अथो॒ यथा᳚ । अथो॒ इत्यथो᳚ । यथा॒ पुरु॑षः । पुरु॑षः॒ स्नाव॑भिः । स्नाव॑भिः॒ सन्त॑तः । स्नाव॑भि॒रिति॒ स्नाव॑ - भिः॒ । सन्त॑त ए॒वम् । सन्त॑त॒ इति॒ सम् - त॒तः॒ । ए॒वमे॒व । ए॒वैताभिः॑ । ए॒ताभि॑र॒ग्निः । अ॒ग्निः सन्त॑तः । सन्त॑तो॒ऽग्निना᳚ । सन्त॑त॒ इति॒ सम् - त॒तः॒ । अ॒ग्निना॒ वै । वै दे॒वाः । दे॒वाः सु॑व॒र्गम् । सु॒व॒र्गम् ॅलो॒कम् । सु॒व॒र्गमिति॑ सुवः - गम् । लो॒कमा॑यन्न् । आ॒य॒न् ताः । ता अ॒मूः । अ॒मूः कृत्ति॑काः । कृत्ति॑का अभवन्न् । अ॒भ॒व॒न्॒. यस्य॑ । यस्यै॒ताः । ए॒ता उ॑पधी॒यन्ते᳚ । उ॒प॒धी॒यन्ते॑ सुव॒र्गम् । उ॒प॒धी॒यन्त॒ इत्यु॑प - धी॒यन्ते᳚ । सु॒व॒र्गमे॒व । सु॒व॒र्गमिति॑ सुवः - गम् । ए॒व लो॒कम् \newline

\textbf{Jatai Paata} \newline

1. सर्वा᳚भ्यो॒ वै वै सर्वा᳚भ्यः॒ सर्वा᳚भ्यो॒ वै । \newline
2. वै दे॒वता᳚भ्यो दे॒वता᳚भ्यो॒ वै वै दे॒वता᳚भ्यः । \newline
3. दे॒वता᳚भ्यो॒ ऽग्नि र॒ग्निर् दे॒वता᳚भ्यो दे॒वता᳚भ्यो॒ ऽग्निः । \newline
4. अ॒ग्नि श्ची॑यते चीयते॒ ऽग्नि र॒ग्नि श्ची॑यते । \newline
5. ची॒य॒ते॒ यद् यच् ची॑यते चीयते॒ यत् । \newline
6. यथ् स॒युजः॑ स॒युजो॒ यद् यथ् स॒युजः॑ । \newline
7. स॒युजो॒ न न स॒युजः॑ स॒युजो॒ न । \newline
8. स॒युज॒ इति॑ स - युजः॑ । \newline
9. नोप॑द॒द्ध्या दु॑पद॒द्ध्यान् न नोप॑द॒द्ध्यात् । \newline
10. उ॒प॒द॒द्ध्याद् दे॒वता॑ दे॒वता॑ उपद॒द्ध्या दु॑पद॒द्ध्याद् दे॒वताः᳚ । \newline
11. उ॒प॒द॒द्ध्यादित्यु॑प - द॒ध्यात् । \newline
12. दे॒वता॑ अस्यास्य दे॒वता॑ दे॒वता॑ अस्य । \newline
13. अ॒स्या॒ग्नि म॒ग्नि म॑स्या स्या॒ग्निम् । \newline
14. अ॒ग्निं ॅवृ॑ञ्जीरन् वृञ्जीरन् न॒ग्नि म॒ग्निं ॅवृ॑ञ्जीरन्न् । \newline
15. वृ॒ञ्जी॒र॒न्॒. यद् यद् वृ॑ञ्जीरन् वृञ्जीर॒न्॒. यत् । \newline
16. यथ् स॒युजः॑ स॒युजो॒ यद् यथ् स॒युजः॑ । \newline
17. स॒युज॑ उप॒दधा᳚ त्युप॒दधा॑ति स॒युजः॑ स॒युज॑ उप॒दधा॑ति । \newline
18. स॒युज॒ इति॑ स - युजः॑ । \newline
19. उ॒प॒दधा᳚ त्या॒त्मना॒ ऽऽत्मनो॑ प॒दधा᳚ त्युप॒दधा᳚ त्या॒त्मना᳚ । \newline
20. उ॒प॒दधा॒तीत्यु॑प - दधा॑ति । \newline
21. आ॒त्म नै॒वै वात्मना॒ ऽऽत्मनै॒व । \newline
22. ए॒वैन॑ मेन मे॒वै वैन᳚म् । \newline
23. ए॒नꣳ॒॒ स॒युजꣳ॑ स॒युज॑ मेन मेनꣳ स॒युज᳚म् । \newline
24. स॒युज॑म् चिनुते चिनुते स॒युजꣳ॑ स॒युज॑म् चिनुते । \newline
25. स॒युज॒मिति॑ स - युज᳚म् । \newline
26. चि॒नु॒ते॒ न न चि॑नुते चिनुते॒ न । \newline
27. नाग्निना॒ ऽग्निना॒ न नाग्निना᳚ । \newline
28. अ॒ग्निना॒ वि व्य॑ग्निना॒ ऽग्निना॒ वि । \newline
29. व्यृ॑द्ध्यत ऋद्ध्यते॒ वि व्यृ॑द्ध्यते । \newline
30. ऋ॒द्ध्य॒ते ऽथो॒ अथो॑ ऋद्ध्यत ऋद्ध्य॒ते ऽथो᳚ । \newline
31. अथो॒ यथा॒ यथा ऽथो॒ अथो॒ यथा᳚ । \newline
32. अथो॒ इत्यथो᳚ । \newline
33. यथा॒ पुरु॑षः॒ पुरु॑षो॒ यथा॒ यथा॒ पुरु॑षः । \newline
34. पुरु॑षः॒ स्नाव॑भिः॒ स्नाव॑भिः॒ पुरु॑षः॒ पुरु॑षः॒ स्नाव॑भिः । \newline
35. स्नाव॑भिः॒ सन्त॑तः॒ सन्त॑तः॒ स्नाव॑भिः॒ स्नाव॑भिः॒ सन्त॑तः । \newline
36. स्नाव॑भि॒रिति॒ स्नाव॑ - भिः॒ । \newline
37. सन्त॑त ए॒व मे॒वꣳ सन्त॑तः॒ सन्त॑त ए॒वम् । \newline
38. सन्त॑त॒ इति॒ सं - त॒तः॒ । \newline
39. ए॒व मे॒वै वैव मे॒व मे॒व । \newline
40. ए॒वै ताभि॑ रे॒ताभि॑ रे॒वै वैताभिः॑ । \newline
41. ए॒ताभि॑ र॒ग्नि र॒ग्नि रे॒ताभि॑ रे॒ताभि॑ र॒ग्निः । \newline
42. अ॒ग्निः सन्त॑तः॒ सन्त॑तो॒ ऽग्नि र॒ग्निः सन्त॑तः । \newline
43. सन्त॑तो॒ ऽग्निना॒ ऽग्निना॒ सन्त॑तः॒ सन्त॑तो॒ ऽग्निना᳚ । \newline
44. सन्त॑त॒ इति॒ सं - त॒तः॒ । \newline
45. अ॒ग्निना॒ वै वा अ॒ग्निना॒ ऽग्निना॒ वै । \newline
46. वै दे॒वा दे॒वा वै वै दे॒वाः । \newline
47. दे॒वाः सु॑व॒र्गꣳ सु॑व॒र्गम् दे॒वा दे॒वाः सु॑व॒र्गम् । \newline
48. सु॒व॒र्गम् ॅलो॒कम् ॅलो॒कꣳ सु॑व॒र्गꣳ सु॑व॒र्गम् ॅलो॒कम् । \newline
49. सु॒व॒र्गमिति॑ सुवः - गम् । \newline
50. लो॒क मा॑यन् नायन् ॅलो॒कम् ॅलो॒क मा॑यन्न् । \newline
51. आ॒य॒न् तास्ता आ॑यन् नाय॒न् ताः । \newline
52. ता अ॒मू र॒मू स्ता स्ता अ॒मूः । \newline
53. अ॒मूः कृत्ति॑काः॒ कृत्ति॑का अ॒मू र॒मूः कृत्ति॑काः । \newline
54. कृत्ति॑का अभवन् नभव॒न् कृत्ति॑काः॒ कृत्ति॑का अभवन्न् । \newline
55. अ॒भ॒व॒न्॒. यस्य॒ यस्या॑ भवन् नभव॒न्॒. यस्य॑ । \newline
56. यस्यै॒ता ए॒ता यस्य॒ यस्यै॒ताः । \newline
57. ए॒ता उ॑पधी॒यन्त॑ उपधी॒यन्त॑ ए॒ता ए॒ता उ॑पधी॒यन्ते᳚ । \newline
58. उ॒प॒धी॒यन्ते॑ सुव॒र्गꣳ सु॑व॒र्ग मु॑पधी॒यन्त॑ उपधी॒यन्ते॑ सुव॒र्गम् । \newline
59. उ॒प॒धी॒यन्त॒ इत्यु॑प - धी॒यन्ते᳚ । \newline
60. सु॒व॒र्ग मे॒वैव सु॑व॒र्गꣳ सु॑व॒र्ग मे॒व । \newline
61. सु॒व॒र्गमिति॑ सुवः - गम् । \newline
62. ए॒व लो॒कम् ॅलो॒क मे॒वैव लो॒कम् । \newline

\textbf{Ghana Paata } \newline

1. सर्वा᳚भ्यो॒ वै वै सर्वा᳚भ्यः॒ सर्वा᳚भ्यो॒ वै दे॒वता᳚भ्यो दे॒वता᳚भ्यो॒ वै सर्वा᳚भ्यः॒ सर्वा᳚भ्यो॒ वै दे॒वता᳚भ्यः । \newline
2. वै दे॒वता᳚भ्यो दे॒वता᳚भ्यो॒ वै वै दे॒वता᳚भ्यो॒ ऽग्नि र॒ग्निर् दे॒वता᳚भ्यो॒ वै वै दे॒वता᳚भ्यो॒ ऽग्निः । \newline
3. दे॒वता᳚भ्यो॒ ऽग्नि र॒ग्निर् दे॒वता᳚भ्यो दे॒वता᳚भ्यो॒ ऽग्नि श्ची॑यते चीयते॒ ऽग्निर् दे॒वता᳚भ्यो दे॒वता᳚भ्यो॒ ऽग्नि श्ची॑यते । \newline
4. अ॒ग्नि श्ची॑यते चीयते॒ ऽग्नि र॒ग्नि श्ची॑यते॒ यद् यच् ची॑यते॒ ऽग्नि र॒ग्नि श्ची॑यते॒ यत् । \newline
5. ची॒य॒ते॒ यद् यच् ची॑यते चीयते॒ यथ् स॒युजः॑ स॒युजो॒ यच् ची॑यते चीयते॒ यथ् स॒युजः॑ । \newline
6. यथ् स॒युजः॑ स॒युजो॒ यद् यथ् स॒युजो॒ न न स॒युजो॒ यद् यथ् स॒युजो॒ न । \newline
7. स॒युजो॒ न न स॒युजः॑ स॒युजो॒ नोप॑द॒द्ध्या दु॑पद॒द्ध्यान् न स॒युजः॑ स॒युजो॒ नोप॑द॒द्ध्यात् । \newline
8. स॒युज॒ इति॑ स - युजः॑ । \newline
9. नोप॑द॒द्ध्या दु॑पद॒द्ध्यान् न नोप॑द॒द्ध्याद् दे॒वता॑ दे॒वता॑ उपद॒द्ध्यान् न नोप॑द॒द्ध्याद् दे॒वताः᳚ । \newline
10. उ॒प॒द॒द्ध्याद् दे॒वता॑ दे॒वता॑ उपद॒द्ध्या दु॑पद॒द्ध्याद् दे॒वता॑ अस्यास्य दे॒वता॑ उपद॒द्ध्या दु॑पद॒द्ध्याद् दे॒वता॑ अस्य । \newline
11. उ॒प॒द॒द्ध्यादित्यु॑प - द॒ध्यात् । \newline
12. दे॒वता॑ अस्यास्य दे॒वता॑ दे॒वता॑ अस्या॒ग्नि म॒ग्नि म॑स्य दे॒वता॑ दे॒वता॑ अस्या॒ग्निम् । \newline
13. अ॒स्या॒ग्नि म॒ग्नि म॑स्या स्या॒ग्निं ॅवृ॑ञ्जीरन् वृञ्जीरन् न॒ग्नि म॑स्या स्या॒ग्निं ॅवृ॑ञ्जीरन्न् । \newline
14. अ॒ग्निं ॅवृ॑ञ्जीरन् वृञ्जीरन् न॒ग्नि म॒ग्निं ॅवृ॑ञ्जीर॒न्॒. यद् यद् वृ॑ञ्जीरन् न॒ग्नि म॒ग्निं ॅवृ॑ञ्जीर॒न्॒. यत् । \newline
15. वृ॒ञ्जी॒र॒न्॒. यद् यद् वृ॑ञ्जीरन् वृञ्जीर॒न्॒. यथ् स॒युजः॑ स॒युजो॒ यद् वृ॑ञ्जीरन् वृञ्जीर॒न्॒. यथ् स॒युजः॑ । \newline
16. यथ् स॒युजः॑ स॒युजो॒ यद् यथ् स॒युज॑ उप॒दधा᳚ त्युप॒दधा॑ति स॒युजो॒ यद् यथ् स॒युज॑ उप॒दधा॑ति । \newline
17. स॒युज॑ उप॒दधा᳚ त्युप॒दधा॑ति स॒युजः॑ स॒युज॑ उप॒दधा᳚ त्या॒त्मना॒ ऽऽत्म नो॑प॒दधा॑ति स॒युजः॑ स॒युज॑ उप॒दधा᳚ त्या॒त्मना᳚ । \newline
18. स॒युज॒ इति॑ स - युजः॑ । \newline
19. उ॒प॒दधा᳚ त्या॒त्मना॒ ऽऽत्म नो॑प॒दधा᳚ त्युप॒दधा᳚ त्या॒त्म नै॒वै वात्म नो॑प॒दधा᳚ त्युप॒दधा᳚ त्या॒त्मनै॒व । \newline
20. उ॒प॒दधा॒तीत्यु॑प - दधा॑ति । \newline
21. आ॒त्म नै॒वै वात्मना॒ ऽऽत्मनै॒ वैन॑ मेन मे॒वात्मना॒ ऽऽत्मनै॒ वैन᳚म् । \newline
22. ए॒वैन॑ मेन मे॒वै वैनꣳ॑ स॒युजꣳ॑ स॒युज॑ मेन मे॒वै वैनꣳ॑ स॒युज᳚म् । \newline
23. ए॒नꣳ॒॒ स॒युजꣳ॑ स॒युज॑ मेन मेनꣳ स॒युज॑म् चिनुते चिनुते स॒युज॑ मेन मेनꣳ स॒युज॑म् चिनुते । \newline
24. स॒युज॑म् चिनुते चिनुते स॒युजꣳ॑ स॒युज॑म् चिनुते॒ न न चि॑नुते स॒युजꣳ॑ स॒युज॑म् चिनुते॒ न । \newline
25. स॒युज॒मिति॑ स - युज᳚म् । \newline
26. चि॒नु॒ते॒ न न चि॑नुते चिनुते॒ नाग्निना॒ ऽग्निना॒ न चि॑नुते चिनुते॒ नाग्निना᳚ । \newline
27. नाग्निना॒ ऽग्निना॒ न नाग्निना॒ वि व्य॑ग्निना॒ न नाग्निना॒ वि । \newline
28. अ॒ग्निना॒ वि व्य॑ग्निना॒ ऽग्निना॒ व्यृ॑द्ध्यत ऋद्ध्यते॒ व्य॑ग्निना॒ ऽग्निना॒ व्यृ॑द्ध्यते । \newline
29. व्यृ॑द्ध्यत ऋद्ध्यते॒ वि व्यृ॑द्ध्य॒ते ऽथो॒ अथो॑ ऋद्ध्यते॒ वि व्यृ॑द्ध्य॒ते ऽथो᳚ । \newline
30. ऋ॒द्ध्य॒ते ऽथो॒ अथो॑ ऋद्ध्यत ऋद्ध्य॒ते ऽथो॒ यथा॒ यथा ऽथो॑ ऋद्ध्यत ऋद्ध्य॒ते ऽथो॒ यथा᳚ । \newline
31. अथो॒ यथा॒ यथा ऽथो॒ अथो॒ यथा॒ पुरु॑षः॒ पुरु॑षो॒ यथा ऽथो॒ अथो॒ यथा॒ पुरु॑षः । \newline
32. अथो॒ इत्यथो᳚ । \newline
33. यथा॒ पुरु॑षः॒ पुरु॑षो॒ यथा॒ यथा॒ पुरु॑षः॒ स्नाव॑भिः॒ स्नाव॑भिः॒ पुरु॑षो॒ यथा॒ यथा॒ पुरु॑षः॒ स्नाव॑भिः । \newline
34. पुरु॑षः॒ स्नाव॑भिः॒ स्नाव॑भिः॒ पुरु॑षः॒ पुरु॑षः॒ स्नाव॑भिः॒ सन्त॑तः॒ सन्त॑तः॒ स्नाव॑भिः॒ पुरु॑षः॒ पुरु॑षः॒ स्नाव॑भिः॒ सन्त॑तः । \newline
35. स्नाव॑भिः॒ सन्त॑तः॒ सन्त॑तः॒ स्नाव॑भिः॒ स्नाव॑भिः॒ सन्त॑त ए॒व मे॒वꣳ सन्त॑तः॒ स्नाव॑भिः॒ स्नाव॑भिः॒ सन्त॑त ए॒वम् । \newline
36. स्नाव॑भि॒रिति॒ स्नाव॑ - भिः॒ । \newline
37. सन्त॑त ए॒व मे॒वꣳ सन्त॑तः॒ सन्त॑त ए॒व मे॒वै वैवꣳ सन्त॑तः॒ सन्त॑त ए॒व मे॒व । \newline
38. सन्त॑त॒ इति॒ सं - त॒तः॒ । \newline
39. ए॒व मे॒वै वैव मे॒व मे॒वैताभि॑ रे॒ताभि॑ रे॒वैव मे॒व मे॒वै ताभिः॑ । \newline
40. ए॒वैताभि॑ रे॒ताभि॑ रे॒वैवैताभि॑ र॒ग्नि र॒ग्नि रे॒ताभि॑ रे॒वैवैताभि॑ र॒ग्निः । \newline
41. ए॒ताभि॑ र॒ग्नि र॒ग्नि रे॒ताभि॑ रे॒ताभि॑ र॒ग्निः सन्त॑तः॒ सन्त॑तो॒ ऽग्नि रे॒ताभि॑ रे॒ताभि॑ र॒ग्निः सन्त॑तः । \newline
42. अ॒ग्निः सन्त॑तः॒ सन्त॑तो॒ ऽग्नि र॒ग्निः सन्त॑तो॒ ऽग्निना॒ ऽग्निना॒ सन्त॑तो॒ ऽग्नि र॒ग्निः सन्त॑तो॒ ऽग्निना᳚ । \newline
43. सन्त॑तो॒ ऽग्निना॒ ऽग्निना॒ सन्त॑तः॒ सन्त॑तो॒ ऽग्निना॒ वै वा अ॒ग्निना॒ सन्त॑तः॒ सन्त॑तो॒ ऽग्निना॒ वै । \newline
44. सन्त॑त॒ इति॒ सं - त॒तः॒ । \newline
45. अ॒ग्निना॒ वै वा अ॒ग्निना॒ ऽग्निना॒ वै दे॒वा दे॒वा वा अ॒ग्निना॒ ऽग्निना॒ वै दे॒वाः । \newline
46. वै दे॒वा दे॒वा वै वै दे॒वाः सु॑व॒र्गꣳ सु॑व॒र्गम् दे॒वा वै वै दे॒वाः सु॑व॒र्गम् । \newline
47. दे॒वाः सु॑व॒र्गꣳ सु॑व॒र्गम् दे॒वा दे॒वाः सु॑व॒र्गम् ॅलो॒कम् ॅलो॒कꣳ सु॑व॒र्गम् दे॒वा दे॒वाः सु॑व॒र्गम् ॅलो॒कम् । \newline
48. सु॒व॒र्गम् ॅलो॒कम् ॅलो॒कꣳ सु॑व॒र्गꣳ सु॑व॒र्गम् ॅलो॒क मा॑यन् नायन् ॅलो॒कꣳ सु॑व॒र्गꣳ सु॑व॒र्गम् ॅलो॒क मा॑यन्न् । \newline
49. सु॒व॒र्गमिति॑ सुवः - गम् । \newline
50. लो॒क मा॑यन् नायन् ॅलो॒कम् ॅलो॒क मा॑य॒न् ता स्ता आ॑यन् ॅलो॒कम् ॅलो॒क मा॑य॒न् ताः । \newline
51. आ॒य॒न् ता स्ता आ॑यन् नाय॒न् ता अ॒मू र॒मू स्ता आ॑यन् नाय॒न् ता अ॒मूः । \newline
52. ता अ॒मू र॒मू स्ता स्ता अ॒मूः कृत्ति॑काः॒ कृत्ति॑का अ॒मू स्ता स्ता अ॒मूः कृत्ति॑काः । \newline
53. अ॒मूः कृत्ति॑काः॒ कृत्ति॑का अ॒मू र॒मूः कृत्ति॑का अभवन् नभव॒न् कृत्ति॑का अ॒मू र॒मूः कृत्ति॑का अभवन्न् । \newline
54. कृत्ति॑का अभवन् नभव॒न् कृत्ति॑काः॒ कृत्ति॑का अभव॒न्॒. यस्य॒ यस्या॑भव॒न् कृत्ति॑काः॒ कृत्ति॑का अभव॒न्॒. यस्य॑ । \newline
55. अ॒भ॒व॒न्॒. यस्य॒ यस्या॑भवन् नभव॒न्॒. यस्यै॒ता ए॒ता यस्या॑भवन् नभव॒न्॒. यस्यै॒ताः । \newline
56. यस्यै॒ता ए॒ता यस्य॒ यस्यै॒ता उ॑पधी॒यन्त॑ उपधी॒यन्त॑ ए॒ता यस्य॒ यस्यै॒ता उ॑पधी॒यन्ते᳚ । \newline
57. ए॒ता उ॑पधी॒यन्त॑ उपधी॒यन्त॑ ए॒ता ए॒ता उ॑पधी॒यन्ते॑ सुव॒र्गꣳ सु॑व॒र्ग मु॑पधी॒यन्त॑ ए॒ता ए॒ता उ॑पधी॒यन्ते॑ सुव॒र्गम् । \newline
58. उ॒प॒धी॒यन्ते॑ सुव॒र्गꣳ सु॑व॒र्ग मु॑पधी॒यन्त॑ उपधी॒यन्ते॑ सुव॒र्ग मे॒वैव सु॑व॒र्ग मु॑पधी॒यन्त॑ उपधी॒यन्ते॑ सुव॒र्ग मे॒व । \newline
59. उ॒प॒धी॒यन्त॒ इत्यु॑प - धी॒यन्ते᳚ । \newline
60. सु॒व॒र्ग मे॒वैव सु॑व॒र्गꣳ सु॑व॒र्ग मे॒व लो॒कम् ॅलो॒क मे॒व सु॑व॒र्गꣳ सु॑व॒र्ग मे॒व लो॒कम् । \newline
61. सु॒व॒र्गमिति॑ सुवः - गम् । \newline
62. ए॒व लो॒कम् ॅलो॒क मे॒वैव लो॒क मे᳚त्येति लो॒क मे॒वैव लो॒क मे॑ति । \newline
\pagebreak
\markright{ TS 5.3.9.2  \hfill https://www.vedavms.in \hfill}

\section{ TS 5.3.9.2 }

\textbf{TS 5.3.9.2 } \newline
\textbf{Samhita Paata} \newline

लो॒कमे॑ति॒ गच्छ॑ति प्रका॒शं चि॒त्रमे॒व भ॑वति मण्डलेष्ट॒का उप॑ दधाती॒मे वै लो॒का म॑ण्डलेष्ट॒का इ॒मे खलु॒ वै लो॒का दे॑वपु॒रा दे॑वपु॒रा ए॒व प्रवि॑शति॒ नाऽऽ*र्ति॒मार्च्छ॑त्य॒ग्निं चि॑क्या॒नो वि॒श्वज्यो॑तिष॒ उप॑ दधाती॒माने॒वैताभि-॑र्लो॒कान् ज्योति॑ष्मतः कुरु॒तेऽथो᳚ प्रा॒णाने॒वैता यज॑मानस्य दाद्ध्रत्ये॒ता वै दे॒वताः᳚ सुव॒र्ग्या᳚स्ता ए॒वा- ( ) -न्वा॒रभ्य॑ सुव॒र्गं ॅलो॒कमे॑ति ॥ \newline

\textbf{Pada Paata} \newline

लो॒कम् । ए॒ति॒ । गच्छ॑ति । प्र॒का॒शमिति॑ प्र-का॒शम् । चि॒त्रम् । ए॒व । भ॒व॒ति॒ । म॒ण्ड॒ले॒ष्ट॒का इति॑ मण्डल - इ॒ष्ट॒काः । उपेति॑ । द॒धा॒ति॒ । इ॒मे । वै । लो॒काः । म॒ण्ड॒ले॒ष्ट॒का इति॑ मण्डल - इ॒ष्ट॒काः । इ॒मे । खलु॑ । वै । लो॒काः । दे॒व॒पु॒रा इति॑ देव-पु॒राः । दे॒व॒पु॒रा इति॑-पु॒राः । ए॒व । प्रेति॑ । वि॒श॒ति॒ । न । आर्ति᳚म् । एति॑ । ऋ॒च्छ॒ति॒ । अ॒ग्निम् । चि॒क्या॒नः । वि॒श्वज्यो॑तिष॒ इति॑ वि॒श्व - ज्यो॒ति॒षः॒ । उपेति॑ । द॒धा॒ति॒ । इ॒मान् । ए॒व । एताभिः॑ । लो॒कान् । ज्योति॑ष्मतः । कु॒रु॒ते॒ । अथो॒ इति॑ । प्रा॒णानिति॑ प्र - अ॒नान् । ए॒व । ए॒ताः । यज॑मानस्य । दा॒द्ध्र॒ति॒ । ए॒ताः । वै । दे॒वताः᳚ । सु॒व॒र्ग्या॑ इति॑ सुवः - ग्याः᳚ । ताः । ए॒व ( ) । अ॒न्वा॒रभ्येत्य॑नु - आ॒रभ्य॑ । सु॒व॒र्गमिति॑ सुवः - गम् । लो॒कम् । ए॒ति॒ ॥  \newline


\textbf{Krama Paata} \newline

लो॒कमे॑ति । ए॒ति॒ गच्छ॑ति । गच्छ॑ति प्रका॒शम् । प्र॒का॒शम् चि॒त्रम् । प्र॒का॒शमिति॑ प्र - का॒शम् । चि॒त्रमे॒व । ए॒व भ॑वति । भ॒व॒ति॒ म॒ण्ड॒ले॒ष्ट॒काः । म॒ण्ड॒ले॒ष्ट॒का उप॑ । म॒ण्ड॒ले॒ष्ट॒का इति॑ मण्डल - इ॒ष्ट॒काः । उप॑ दधाति । द॒धा॒ती॒मे । इ॒मे वै । वै लो॒काः । लो॒का म॑ण्डलेष्ट॒काः । म॒ण्ड॒ले॒ष्ट॒का इ॒मे । म॒ण्ड॒ले॒ष्ट॒का इति॑ मण्डल - इ॒ष्ट॒काः । इ॒मे खलु॑ । खलु॒ वै । वै लो॒काः । लो॒का दे॑वपु॒राः । दे॒व॒पु॒रा दे॑वपु॒राः । दे॒व॒पु॒रा इति॑ देव - पु॒राः । दे॒व॒पु॒रा ए॒व । दे॒व॒पु॒रा इति॑ देव - पु॒राः । ए॒व प्र । प्र वि॑शति । वि॒श॒ति॒ न । नार्ति᳚म् । आर्ति॒मा । आर्च्छ॑ति । ऋ॒च्छ॒त्य॒ग्निम् । अ॒ग्निम् चि॑क्या॒नः । चि॒क्या॒नो वि॒श्वज्यो॑तिषः । वि॒श्वज्यो॑तिष॒ उप॑ । वि॒श्वज्यो॑तिष॒ इति॑ वि॒श्व - ज्यो॒ति॒षः॒ । उप॑ दधाति । द॒धा॒ती॒मान् । इ॒माने॒व । ए॒वैताभिः॑ । ए॒ताभि॑र् लो॒कान् । लो॒कान् जोति॑ष्मतः । ज्योति॑ष्मतः कुरुते । कु॒रु॒तेऽथो᳚ । अथो᳚ प्रा॒णान् । अथो॒ इत्यथो᳚ । प्रा॒णाने॒व । प्रा॒णानिति॑ प्र - अ॒नान् । ए॒वैताः । ए॒ता यज॑मानस्य । यज॑मानस्य दाद्ध्रति । दा॒द्ध्र॒त्ये॒ताः । ए॒ता वै । वै दे॒वताः᳚ । दे॒वताः᳚ सुव॒र्ग्याः᳚ । सु॒व॒र्ग्या᳚स्ताः । सु॒व॒र्ग्या॑ इति॑ सुवः - ग्याः᳚ । ता ए॒व ( ) । ए॒वान्वा॒रभ्य॑ । अ॒न्वा॒रभ्य॑ सुव॒र्गम् । अ॒न्वा॒रभ्येत्य॑नु - आ॒रभ्य॑ । सु॒व॒र्गम् ॅलो॒कम् । सु॒व॒र्गमिति॑ सुवः - गम् । लो॒कमे॑ति । ए॒तीत्ये॑ति । \newline

\textbf{Jatai Paata} \newline

1. लो॒क मे᳚त्येति लो॒कम् ॅलो॒क मे॑ति । \newline
2. ए॒ति॒ गच्छ॑ति॒ गच्छ॑ त्ये त्येति॒ गच्छ॑ति । \newline
3. गच्छ॑ति प्रका॒शम् प्र॑का॒शम् गच्छ॑ति॒ गच्छ॑ति प्रका॒शम् । \newline
4. प्र॒का॒शम् चि॒त्रम् चि॒त्रम् प्र॑का॒शम् प्र॑का॒शम् चि॒त्रम् । \newline
5. प्र॒का॒शमिति॑ प्र - का॒शम् । \newline
6. चि॒त्र मे॒वैव चि॒त्रम् चि॒त्र मे॒व । \newline
7. ए॒व भ॑वति भव त्ये॒वैव भ॑वति । \newline
8. भ॒व॒ति॒ म॒ण्ड॒ले॒ष्ट॒का म॑ण्डलेष्ट॒का भ॑वति भवति मण्डलेष्ट॒काः । \newline
9. म॒ण्ड॒ले॒ष्ट॒का उपोप॑ मण्डलेष्ट॒का म॑ण्डलेष्ट॒का उप॑ । \newline
10. म॒ण्ड॒ले॒ष्ट॒का इति॑ मण्डल - इ॒ष्ट॒काः । \newline
11. उप॑ दधाति दधा॒ त्युपोप॑ दधाति । \newline
12. द॒धा॒ती॒म इ॒मे द॑धाति दधाती॒मे । \newline
13. इ॒मे वै वा इ॒म इ॒मे वै । \newline
14. वै लो॒का लो॒का वै वै लो॒काः । \newline
15. लो॒का म॑ण्डलेष्ट॒का म॑ण्डलेष्ट॒का लो॒का लो॒का म॑ण्डलेष्ट॒काः । \newline
16. म॒ण्ड॒ले॒ष्ट॒का इ॒म इ॒मे म॑ण्डलेष्ट॒का म॑ण्डलेष्ट॒का इ॒मे । \newline
17. म॒ण्ड॒ले॒ष्ट॒का इति॑ मण्डल - इ॒ष्ट॒काः । \newline
18. इ॒मे खलु॒ खल्वि॒म इ॒मे खलु॑ । \newline
19. खलु॒ वै वै खलु॒ खलु॒ वै । \newline
20. वै लो॒का लो॒का वै वै लो॒काः । \newline
21. लो॒का दे॑वपु॒रा दे॑वपु॒रा लो॒का लो॒का दे॑वपु॒राः । \newline
22. दे॒व॒पु॒रा दे॑वपु॒राः । \newline
23. दे॒व॒पु॒रा इति॑ देव - पु॒राः । \newline
24. दे॒व॒पु॒रा ए॒वैव दे॑वपु॒रा दे॑वपु॒रा ए॒व । \newline
25. दे॒व॒पु॒रा इति॑ देव - पु॒राः । \newline
26. ए॒व प्र प्रैवैव प्र । \newline
27. प्र वि॑शति विशति॒ प्र प्र वि॑शति । \newline
28. वि॒श॒ति॒ न न वि॑शति विशति॒ न । \newline
29. नार्ति॒ मार्ति॒म् न नार्ति᳚म् । \newline
30. आर्ति॒ मा ऽऽर्ति॒ मार्ति॒ मा । \newline
31. आर्च्छ॑ त्यृच्छ त्यार्च्छति । \newline
32. ऋ॒च्छ॒ त्य॒ग्नि म॒ग्नि मृ॑च्छ त्यृच्छ त्य॒ग्निम् । \newline
33. अ॒ग्निम् चि॑क्या॒न श्चि॑क्या॒नो᳚ ऽग्नि म॒ग्निम् चि॑क्या॒नः । \newline
34. चि॒क्या॒नो वि॒श्वज्यो॑तिषो वि॒श्वज्यो॑तिष श्चिक्या॒न श्चि॑क्या॒नो वि॒श्वज्यो॑तिषः । \newline
35. वि॒श्वज्यो॑तिष॒ उपोप॑ वि॒श्वज्यो॑तिषो वि॒श्वज्यो॑तिष॒ उप॑ । \newline
36. वि॒श्वज्यो॑तिष॒ इति॑ वि॒श्व - ज्यो॒ति॒षः॒ । \newline
37. उप॑ दधाति दधा॒ त्युपोप॑ दधाति । \newline
38. द॒धा॒ती॒मा नि॒मान् द॑धाति दधाती॒मान् । \newline
39. इ॒मा ने॒वैवे मा नि॒मा ने॒व । \newline
40. ए॒वैताभि॑ रे॒ताभि॑ रे॒वै वैताभिः॑ । \newline
41. ए॒ताभि॑र् लो॒कान् ॅलो॒का ने॒ताभि॑ रे॒ताभि॑र् लो॒कान् । \newline
42. लो॒कान् ज्योति॑ष्मतो॒ ज्योति॑ष्मतो लो॒कान् ॅलो॒कान् ज्योति॑ष्मतः । \newline
43. ज्योति॑ष्मतः कुरुते कुरुते॒ ज्योति॑ष्मतो॒ ज्योति॑ष्मतः कुरुते । \newline
44. कु॒रु॒ते ऽथो॒ अथो॑ कुरुते कुरु॒ते ऽथो᳚ । \newline
45. अथो᳚ प्रा॒णान् प्रा॒णा नथो॒ अथो᳚ प्रा॒णान् । \newline
46. अथो॒ इत्यथो᳚ । \newline
47. प्रा॒णा ने॒वैव प्रा॒णान् प्रा॒णा ने॒व । \newline
48. प्रा॒णानिति॑ प्र - अ॒नान् । \newline
49. ए॒वैता ए॒ता ए॒वै वैताः । \newline
50. ए॒ता यज॑मानस्य॒ यज॑मान स्यै॒ता ए॒ता यज॑मानस्य । \newline
51. यज॑मानस्य दाद्ध्रति दाद्ध्रति॒ यज॑मानस्य॒ यज॑मानस्य दाद्ध्रति । \newline
52. दा॒द्ध्र॒ त्ये॒ता ए॒ता दा᳚द्ध्रति दाद्ध्र त्ये॒ताः । \newline
53. ए॒ता वै वा ए॒ता ए॒ता वै । \newline
54. वै दे॒वता॑ दे॒वता॒ वै वै दे॒वताः᳚ । \newline
55. दे॒वताः᳚ सुव॒र्ग्याः᳚ सुव॒र्ग्या॑ दे॒वता॑ दे॒वताः᳚ सुव॒र्ग्याः᳚ । \newline
56. सु॒व॒र्ग्या᳚ स्ता स्ताः सु॑व॒र्ग्याः᳚ सुव॒र्ग्या᳚स्ताः । \newline
57. सु॒व॒र्ग्या॑ इति॑ सुवः - ग्याः᳚ । \newline
58. ता ए॒वैव ता स्ता ए॒व । \newline
59. ए॒वा न्वा॒रभ्या᳚ न्वा॒रभ्यै॒ वैवा न्वा॒रभ्य॑ । \newline
60. अ॒न्वा॒रभ्य॑ सुव॒र्गꣳ सु॑व॒र्ग म॑न्वा॒रभ्या᳚ न्वा॒रभ्य॑ सुव॒र्गम् । \newline
61. अ॒न्वा॒रभ्येत्य॑नु - आ॒रभ्य॑ । \newline
62. सु॒व॒र्गम् ॅलो॒कम् ॅलो॒कꣳ सु॑व॒र्गꣳ सु॑व॒र्गम् ॅलो॒कम् । \newline
63. सु॒व॒र्गमिति॑ सुवः - गम् । \newline
64. लो॒क मे᳚त्येति लो॒कम् ॅलो॒क मे॑ति । \newline
65. ए॒तीत्ये॑ति । \newline

\textbf{Ghana Paata } \newline

1. लो॒क मे᳚त्येति लो॒कम् ॅलो॒क मे॑ति॒ गच्छ॑ति॒ गच्छ॑ त्येति लो॒कम् ॅलो॒क मे॑ति॒ गच्छ॑ति । \newline
2. ए॒ति॒ गच्छ॑ति॒ गच्छ॑ त्येत्येति॒ गच्छ॑ति प्रका॒शम् प्र॑का॒शम् गच्छ॑ त्येत्येति॒ गच्छ॑ति प्रका॒शम् । \newline
3. गच्छ॑ति प्रका॒शम् प्र॑का॒शम् गच्छ॑ति॒ गच्छ॑ति प्रका॒शम् चि॒त्रम् चि॒त्रम् प्र॑का॒शम् गच्छ॑ति॒ गच्छ॑ति प्रका॒शम् चि॒त्रम् । \newline
4. प्र॒का॒शम् चि॒त्रम् चि॒त्रम् प्र॑का॒शम् प्र॑का॒शम् चि॒त्र मे॒वैव चि॒त्रम् प्र॑का॒शम् प्र॑का॒शम् चि॒त्र मे॒व । \newline
5. प्र॒का॒शमिति॑ प्र - का॒शम् । \newline
6. चि॒त्र मे॒वैव चि॒त्रम् चि॒त्र मे॒व भ॑वति भव त्ये॒व चि॒त्रम् चि॒त्र मे॒व भ॑वति । \newline
7. ए॒व भ॑वति भव त्ये॒वैव भ॑वति मण्डलेष्ट॒का म॑ण्डलेष्ट॒का भ॑व त्ये॒वैव भ॑वति मण्डलेष्ट॒काः । \newline
8. भ॒व॒ति॒ म॒ण्ड॒ले॒ष्ट॒का म॑ण्डलेष्ट॒का भ॑वति भवति मण्डलेष्ट॒का उपोप॑ मण्डलेष्ट॒का भ॑वति भवति मण्डलेष्ट॒का उप॑ । \newline
9. म॒ण्ड॒ले॒ष्ट॒का उपोप॑ मण्डलेष्ट॒का म॑ण्डलेष्ट॒का उप॑ दधाति दधा॒ त्युप॑ मण्डलेष्ट॒का म॑ण्डलेष्ट॒का उप॑ दधाति । \newline
10. म॒ण्ड॒ले॒ष्ट॒का इति॑ मण्डल - इ॒ष्ट॒काः । \newline
11. उप॑ दधाति दधा॒ त्युपोप॑ दधाती॒म इ॒मे द॑धा॒ त्युपोप॑ दधाती॒मे । \newline
12. द॒धा॒ती॒म इ॒मे द॑धाति दधाती॒मे वै वा इ॒मे द॑धाति दधाती॒मे वै । \newline
13. इ॒मे वै वा इ॒म इ॒मे वै लो॒का लो॒का वा इ॒म इ॒मे वै लो॒काः । \newline
14. वै लो॒का लो॒का वै वै लो॒का म॑ण्डलेष्ट॒का म॑ण्डलेष्ट॒का लो॒का वै वै लो॒का म॑ण्डलेष्ट॒काः । \newline
15. लो॒का म॑ण्डलेष्ट॒का म॑ण्डलेष्ट॒का लो॒का लो॒का म॑ण्डलेष्ट॒का इ॒म इ॒मे म॑ण्डलेष्ट॒का लो॒का लो॒का म॑ण्डलेष्ट॒का इ॒मे । \newline
16. म॒ण्ड॒ले॒ष्ट॒का इ॒म इ॒मे म॑ण्डलेष्ट॒का म॑ण्डलेष्ट॒का इ॒मे खलु॒ खल्वि॒मे म॑ण्डलेष्ट॒का म॑ण्डलेष्ट॒का इ॒मे खलु॑ । \newline
17. म॒ण्ड॒ले॒ष्ट॒का इति॑ मण्डल - इ॒ष्ट॒काः । \newline
18. इ॒मे खलु॒ खल्वि॒म इ॒मे खलु॒ वै वै खल्वि॒म इ॒मे खलु॒ वै । \newline
19. खलु॒ वै वै खलु॒ खलु॒ वै लो॒का लो॒का वै खलु॒ खलु॒ वै लो॒काः । \newline
20. वै लो॒का लो॒का वै वै लो॒का दे॑वपु॒रा दे॑वपु॒रा लो॒का वै वै लो॒का दे॑वपु॒राः । \newline
21. लो॒का दे॑वपु॒रा दे॑वपु॒रा लो॒का लो॒का दे॑वपु॒राः । \newline
22. दे॒व॒पु॒रा दे॑वपु॒राः । \newline
23. दे॒व॒पु॒रा इति॑ देव - पु॒राः । \newline
24. दे॒व॒पु॒रा ए॒वैव दे॑वपु॒रा दे॑वपु॒रा ए॒व प्र प्रैव दे॑वपु॒रा दे॑वपु॒रा ए॒व प्र । \newline
25. दे॒व॒पु॒रा इति॑ देव - पु॒राः । \newline
26. ए॒व प्र प्रैवैव प्र वि॑शति विशति॒ प्रैवैव प्र वि॑शति । \newline
27. प्र वि॑शति विशति॒ प्र प्र वि॑शति॒ न न वि॑शति॒ प्र प्र वि॑शति॒ न । \newline
28. वि॒श॒ति॒ न न वि॑शति विशति॒ नार्ति॒ मार्ति॒म् न वि॑शति विशति॒ नार्ति᳚म् । \newline
29. नार्ति॒ मार्ति॒म् न नार्ति॒ मा ऽऽर्ति॒म् न नार्ति॒ मा । \newline
30. आर्ति॒ मा ऽऽर्ति॒ मार्ति॒ मार्च्छ॑ त्यृच्छ॒ त्याऽऽर्ति॒ मार्ति॒ मार्च्छ॑ति । \newline
31. आर्च्छ॑ त्यृच्छ त्यार्च्छ त्य॒ग्नि म॒ग्नि मृ॑च्छ त्यार्च्छ त्य॒ग्निम् । \newline
32. ऋ॒च्छ॒ त्य॒ग्नि म॒ग्नि मृ॑च्छ त्यृच्छ त्य॒ग्निम् चि॑क्या॒न श्चि॑क्या॒नो᳚ ऽग्नि मृ॑च्छ त्यृच्छ त्य॒ग्निम् चि॑क्या॒नः । \newline
33. अ॒ग्निम् चि॑क्या॒न श्चि॑क्या॒नो᳚ ऽग्नि म॒ग्निम् चि॑क्या॒नो वि॒श्वज्यो॑तिषो वि॒श्वज्यो॑तिष श्चिक्या॒नो᳚ ऽग्नि म॒ग्निम् चि॑क्या॒नो वि॒श्वज्यो॑तिषः । \newline
34. चि॒क्या॒नो वि॒श्वज्यो॑तिषो वि॒श्वज्यो॑तिष श्चिक्या॒न श्चि॑क्या॒नो वि॒श्वज्यो॑तिष॒ उपोप॑ वि॒श्वज्यो॑तिष श्चिक्या॒न श्चि॑क्या॒नो वि॒श्वज्यो॑तिष॒ उप॑ । \newline
35. वि॒श्वज्यो॑तिष॒ उपोप॑ वि॒श्वज्यो॑तिषो वि॒श्वज्यो॑तिष॒ उप॑ दधाति दधा॒ त्युप॑ वि॒श्वज्यो॑तिषो वि॒श्वज्यो॑तिष॒ उप॑ दधाति । \newline
36. वि॒श्वज्यो॑तिष॒ इति॑ वि॒श्व - ज्यो॒ति॒षः॒ । \newline
37. उप॑ दधाति दधा॒ त्युपोप॑ दधाती॒मा नि॒मान् द॑धा॒ त्युपोप॑ दधाती॒मान् । \newline
38. द॒धा॒ती॒मा नि॒मान् द॑धाति दधाती॒मा ने॒वैवेमान् द॑धाति दधाती॒मा ने॒व । \newline
39. इ॒मा ने॒वैवे मा नि॒मा ने॒वै ताभि॑ रे॒ताभि॑ रे॒वे मा नि॒मा ने॒वै ताभिः॑ । \newline
40. ए॒वैताभि॑ रे॒ताभि॑ रे॒वै वैताभि॑र् लो॒कान् ॅलो॒का ने॒ताभि॑ रे॒वै वैताभि॑र् लो॒कान् । \newline
41. ए॒ताभि॑र् लो॒कान् ॅलो॒का ने॒ताभि॑ रे॒ताभि॑र् लो॒कान् ज्योति॑ष्मतो॒ ज्योति॑ष्मतो लो॒का ने॒ताभि॑ रे॒ताभि॑र् लो॒कान् ज्योति॑ष्मतः । \newline
42. लो॒कान् ज्योति॑ष्मतो॒ ज्योति॑ष्मतो लो॒कान् ॅलो॒कान् ज्योति॑ष्मतः कुरुते कुरुते॒ ज्योति॑ष्मतो लो॒कान् ॅलो॒कान् ज्योति॑ष्मतः कुरुते । \newline
43. ज्योति॑ष्मतः कुरुते कुरुते॒ ज्योति॑ष्मतो॒ ज्योति॑ष्मतः कुरु॒ते ऽथो॒ अथो॑ कुरुते॒ ज्योति॑ष्मतो॒ ज्योति॑ष्मतः कुरु॒ते ऽथो᳚ । \newline
44. कु॒रु॒ते ऽथो॒ अथो॑ कुरुते कुरु॒ते ऽथो᳚ प्रा॒णान् प्रा॒णा नथो॑ कुरुते कुरु॒ते ऽथो᳚ प्रा॒णान् । \newline
45. अथो᳚ प्रा॒णान् प्रा॒णा नथो॒ अथो᳚ प्रा॒णा ने॒वैव प्रा॒णा नथो॒ अथो᳚ प्रा॒णा ने॒व । \newline
46. अथो॒ इत्यथो᳚ । \newline
47. प्रा॒णा ने॒वैव प्रा॒णान् प्रा॒णा ने॒वैता ए॒ता ए॒व प्रा॒णान् प्रा॒णा ने॒वैताः । \newline
48. प्रा॒णानिति॑ प्र - अ॒नान् । \newline
49. ए॒वैता ए॒ता ए॒वै वैता यज॑मानस्य॒ यज॑मानस्यै॒ता ए॒वै वैता यज॑मानस्य । \newline
50. ए॒ता यज॑मानस्य॒ यज॑मानस्यै॒ता ए॒ता यज॑मानस्य दाद्ध्रति दाद्ध्रति॒ यज॑मानस्यै॒ता ए॒ता यज॑मानस्य दाद्ध्रति । \newline
51. यज॑मानस्य दाद्ध्रति दाद्ध्रति॒ यज॑मानस्य॒ यज॑मानस्य दाद्ध्र त्ये॒ता ए॒ता दा᳚द्ध्रति॒ यज॑मानस्य॒ यज॑मानस्य दाद्ध्र त्ये॒ताः । \newline
52. दा॒द्ध्र॒त्ये॒ता ए॒ता दा᳚द्ध्रति दाद्ध्र त्ये॒ता वै वा ए॒ता दा᳚द्ध्रति दाद्ध्र त्ये॒ता वै । \newline
53. ए॒ता वै वा ए॒ता ए॒ता वै दे॒वता॑ दे॒वता॒ वा ए॒ता ए॒ता वै दे॒वताः᳚ । \newline
54. वै दे॒वता॑ दे॒वता॒ वै वै दे॒वताः᳚ सुव॒र्ग्याः᳚ सुव॒र्ग्या॑ दे॒वता॒ वै वै दे॒वताः᳚ सुव॒र्ग्याः᳚ । \newline
55. दे॒वताः᳚ सुव॒र्ग्याः᳚ सुव॒र्ग्या॑ दे॒वता॑ दे॒वताः᳚ सुव॒र्ग्या᳚ स्ता स्ताः सु॑व॒र्ग्या॑ दे॒वता॑ दे॒वताः᳚ सुव॒र्ग्या᳚ स्ताः । \newline
56. सु॒व॒र्ग्या᳚ स्ता स्ताः सु॑व॒र्ग्याः᳚ सुव॒र्ग्या᳚ स्ता ए॒वैव ताः सु॑व॒र्ग्याः᳚ सुव॒र्ग्या᳚ स्ता ए॒व । \newline
57. सु॒व॒र्ग्या॑ इति॑ सुवः - ग्याः᳚ । \newline
58. ता ए॒वैव ता स्ता ए॒वा न्वा॒रभ्या᳚ न्वा॒रभ्यै॒व ता स्ता ए॒वान्वा॒रभ्य॑ । \newline
59. ए॒वा न्वा॒रभ्या᳚ न्वा॒रभ्यै॒ वैवा न्वा॒रभ्य॑ सुव॒र्गꣳ सु॑व॒र्ग म॑न्वा॒रभ्यै॒ वैवान्वा॒रभ्य॑ सुव॒र्गम् । \newline
60. अ॒न्वा॒रभ्य॑ सुव॒र्गꣳ सु॑व॒र्ग म॑न्वा॒रभ्या᳚ न्वा॒रभ्य॑ सुव॒र्गम् ॅलो॒कम् ॅलो॒कꣳ सु॑व॒र्ग म॑न्वा॒रभ्या᳚ न्वा॒रभ्य॑ सुव॒र्गम् ॅलो॒कम् । \newline
61. अ॒न्वा॒रभ्येत्य॑नु - आ॒रभ्य॑ । \newline
62. सु॒व॒र्गम् ॅलो॒कम् ॅलो॒कꣳ सु॑व॒र्गꣳ सु॑व॒र्गम् ॅलो॒क मे᳚त्येति लो॒कꣳ सु॑व॒र्गꣳ सु॑व॒र्गम् ॅलो॒क मे॑ति । \newline
63. सु॒व॒र्गमिति॑ सुवः - गम् । \newline
64. लो॒क मे᳚त्येति लो॒कम् ॅलो॒क मे॑ति । \newline
65. ए॒तीत्ये॑ति । \newline
\pagebreak
\markright{ TS 5.3.10.1  \hfill https://www.vedavms.in \hfill}

\section{ TS 5.3.10.1 }

\textbf{TS 5.3.10.1 } \newline
\textbf{Samhita Paata} \newline

वृ॒ष्टि॒सनी॒रुप॑ दधाति॒ वृष्टि॑मे॒वाव॑ रुन्धे॒ यदे॑क॒धोप॑द॒द्ध्यादेक॑मृ॒तुं व॑र्.षेदनुपरि॒हारꣳ॑ सादयति॒ तस्मा॒थ् सर्वा॑नृ॒तून्. व॑र्.षति पुरोवात॒सनि॑-र॒सीत्या॑है॒तद्वै वृष्ट्यै॑ रू॒पꣳ रू॒पेणै॒व वृष्टि॒मव॑ रुन्धे सं॒ॅयानी॑भि॒र्वै दे॒वा इ॒मान् ॅलो॒कान्थ् सम॑यु॒स्तथ् सं॒ॅयानी॑नाꣳ संॅयानि॒त्वं ॅयथ् सं॒ॅयानी॑रुप॒दधा॑ति॒ यथा॒ऽफ्सु ना॒वा सं॒ॅयात्ये॒व - [  ] \newline

\textbf{Pada Paata} \newline

वृ॒ष्टि॒सनी॒रिति॑ वृष्टि - सनीः᳚ । उपेति॑ । द॒धा॒ति॒ । वृष्टि᳚म् । ए॒व । अवेति॑ । रु॒न्धे॒ । यत् । ए॒क॒धेत्ये॑क - धा । उ॒प॒द॒द्ध्यादित्यु॑प - द॒द्ध्यात् । एक᳚म् । ऋ॒तुम् । व॒र्.॒षे॒त् । अ॒नु॒प॒रि॒हार॒मित्य॑नु - प॒रि॒हार᳚म् । सा॒द॒य॒ति॒ । तस्मा᳚त् । सर्वान्॑ । ऋ॒तून् । व॒र्.॒ष॒ति॒ । पु॒रो॒वा॒त॒सनि॒रिति॑ पुरोवात - सनिः॑ । अ॒सि॒ । इति॑ । आ॒ह॒ । ए॒तत् । वै । वृष्ट्यै᳚ । रू॒पम् । रू॒पेण॑ । ए॒व । वृष्टि᳚म् । अवेति॑ । रु॒न्धे॒ । सं॒ॅयानी॑भि॒रिति॑ सं - यानी॑भिः । वै । दे॒वाः । इ॒मान् । लो॒कान् । समिति॑ । अ॒युः॒ । तत् । सं॒ॅयानी॑ना॒मिति॑ सं - यानी॑नाम् । सं॒ॅया॒नि॒त्वमिति॑ संॅयानि - त्वम् । यत् । सं॒ॅयानी॒रिति॑ सं - यानीः᳚ । उ॒प॒दधा॒तीत्यु॑प - दधा॑ति । यथा᳚ । अ॒फ्स्वित्य॑प् - सु । ना॒वा । सं॒ॅयातीति॑ सं - याति॑ । ए॒वम् ।  \newline


\textbf{Krama Paata} \newline

वृ॒ष्टि॒सनी॒रुप॑ । वृ॒ष्टि॒सनी॒रिति॑ वृष्टि - सनीः᳚ । उप॑ दधाति । द॒धा॒ति॒ वृष्टि᳚म् । वृष्टि॑मे॒व । ए॒वाव॑ । अव॑ रुन्धे । रु॒न्धे॒ यत् । यदे॑क॒धा । ए॒क॒धोप॑द॒द्ध्यात् । ए॒क॒धेत्ये॑क - धा । उ॒प॒द॒द्ध्यादेक᳚म् । उ॒प॒द॒द्ध्यादित्यु॑प - द॒द्ध्यात् । एक॑मृ॒तुम् । ऋ॒तुम् ॅव॑र्.षेत् । व॒र्॒.षे॒द॒नु॒प॒रि॒हार᳚म् । अ॒नु॒प॒रि॒हारꣳ॑ सादयति । अ॒नु॒प॒रि॒हार॒मित्य॑नु - प॒रि॒हार᳚म् । सा॒द॒य॒ति॒ तस्मा᳚त् । तस्मा॒थ् सर्वान्॑ । सर्वा॑नृ॒तून् । ऋ॒तून्. व॑र्.षति । व॒र्॒.ष॒ति॒ पु॒रो॒वा॒त॒सनिः॑ । पु॒रो॒वा॒त॒सनि॑रसि । पु॒रो॒वा॒त॒सनि॒रिति॑ पुरोवात - सनिः॑ । अ॒सीति॑ । इत्या॑ह । आ॒है॒तत् । ए॒तद् वै । वै वृष्ट्यै᳚ । वृष्ट्यै॑ रू॒पम् । रू॒पꣳ रू॒पेण॑ । रू॒पेणै॒व । ए॒व वृष्टि᳚म् । वृष्टि॒मव॑ । अव॑ रुन्धे । रु॒न्धे॒ स॒म्ॅयानी॑भिः । स॒म्ॅयानी॑भि॒र् वै । स॒म्ॅयानी॑भि॒रिति॑ सम् - यानी॑भिः । वै दे॒वाः । दे॒वा इ॒मान् । इ॒मान् ॅलो॒कान् । लो॒कान्थ् सम् । सम॑युः । अ॒यु॒स्तत् । तथ् स॒म्ॅयानी॑नाम् । स॒म्ॅयानी॑नाꣳ सम्ॅयानि॒त्वम् । स॒म्ॅयानी॑ना॒मिति॑ सम् - यानी॑नाम् । स॒म्ॅया॒नि॒त्वम् ॅयत् । स॒म्ॅया॒नि॒त्वमिति॑ सम्ॅयानि - त्वम् । यथ् स॒म्ॅयानीः᳚ । स॒म्ॅयानी॑रुप॒दधा॑ति । स॒म्ॅयानी॒रिति॑ सम् - यानीः᳚ । उ॒प॒दधा॑ति॒ यथा᳚ । उ॒प॒दधा॒तीत्यु॑प - दधा॑ति । यथा॒ऽफ्सु । अ॒फ्सु ना॒वा । अ॒फ्स्वित्य॑प् - सु । ना॒वा स॒म्ॅयाति॑ । स॒म्ॅयात्ये॒वम् । स॒म्ॅयातीति॑ सम् - याति॑ । ए॒वमे॒व \newline

\textbf{Jatai Paata} \newline

1. वृ॒ष्टि॒सनी॒ रुपोप॑ वृष्टि॒सनी᳚र् वृष्टि॒सनी॒ रुप॑ । \newline
2. वृ॒ष्टि॒सनी॒रिति॑ वृष्टि - सनीः᳚ । \newline
3. उप॑ दधाति दधा॒ त्युपोप॑ दधाति । \newline
4. द॒धा॒ति॒ वृष्टिं॒ ॅवृष्टि॑म् दधाति दधाति॒ वृष्टि᳚म् । \newline
5. वृष्टि॑ मे॒वैव वृष्टिं॒ ॅवृष्टि॑ मे॒व । \newline
6. ए॒वावा वै॒वै वाव॑ । \newline
7. अव॑ रुन्धे रु॒न्धे ऽवाव॑ रुन्धे । \newline
8. रु॒न्धे॒ यद् यद् रु॑न्धे रुन्धे॒ यत् । \newline
9. यदे॑क॒ धैक॒धा यद् यदे॑क॒धा । \newline
10. ए॒क॒धो प॑द॒द्ध्या दु॑पद॒द्ध्या दे॑क॒धैक॒धो प॑द॒द्ध्यात् । \newline
11. ए॒क॒धेत्ये॑क - धा । \newline
12. उ॒प॒द॒द्ध्या देक॒ मेक॑ मुपद॒द्ध्या दु॑पद॒द्ध्या देक᳚म् । \newline
13. उ॒प॒द॒द्ध्यादित्यु॑प - द॒द्ध्यात् । \newline
14. एक॑ मृ॒तु मृ॒तु मेक॒ मेक॑ मृ॒तुम् । \newline
15. ऋ॒तुं ॅव॑र्.षेद् वर्.षेदृ॒तु मृ॒तुं ॅव॑र्.षेत् । \newline
16. व॒र्॒.षे॒ द॒नु॒प॒रि॒हार॑ मनुपरि॒हारं॑ ॅवर्.षेद् वर्.षे दनुपरि॒हार᳚म् । \newline
17. अ॒नु॒प॒रि॒हारꣳ॑ सादयति सादय त्यनुपरि॒हार॑ मनुपरि॒हारꣳ॑ सादयति । \newline
18. अ॒नु॒प॒रि॒हार॒मित्य॑नु - प॒रि॒हार᳚म् । \newline
19. सा॒द॒य॒ति॒ तस्मा॒त् तस्मा᳚थ् सादयति सादयति॒ तस्मा᳚त् । \newline
20. तस्मा॒थ् सर्वा॒न् थ्सर्वा॒न् तस्मा॒त् तस्मा॒थ् सर्वान्॑ । \newline
21. सर्वा॑ नृ॒तू नृ॒तून् थ्सर्वा॒न् थ्सर्वा॑ नृ॒तून् । \newline
22. ऋ॒तून्. व॑र्.षति वर्.ष त्यृ॒तू नृ॒तून्. व॑र्.षति । \newline
23. व॒र्॒.ष॒ति॒ पु॒रो॒वा॒त॒सनिः॑ पुरोवात॒सनि॑र् वर्.षति वर्.षति पुरोवात॒सनिः॑ । \newline
24. पु॒रो॒वा॒त॒सनि॑ रस्यसि पुरोवात॒सनिः॑ पुरोवात॒सनि॑ रसि । \newline
25. पु॒रो॒वा॒त॒सनि॒रिति॑ पुरोवात - सनिः॑ । \newline
26. अ॒सीती त्य॑स्य॒ सीति॑ । \newline
27. इत्या॑हा॒हे तीत्या॑ह । \newline
28. आ॒है॒त दे॒त दा॑हाहै॒तत् । \newline
29. ए॒तद् वै वा ए॒त दे॒तद् वै । \newline
30. वै वृष्ट्यै॒ वृष्ट्यै॒ वै वै वृष्ट्यै᳚ । \newline
31. वृष्ट्यै॑ रू॒पꣳ रू॒पं ॅवृष्ट्यै॒ वृष्ट्यै॑ रू॒पम् । \newline
32. रू॒पꣳ रू॒पेण॑ रू॒पेण॑ रू॒पꣳ रू॒पꣳ रू॒पेण॑ । \newline
33. रू॒पे णै॒वैव रू॒पेण॑ रू॒पेणै॒व । \newline
34. ए॒व वृष्टिं॒ ॅवृष्टि॑ मे॒वैव वृष्टि᳚म् । \newline
35. वृष्टि॒ मवाव॒ वृष्टिं॒ ॅवृष्टि॒ मव॑ । \newline
36. अव॑ रुन्धे रु॒न्धे ऽवाव॑ रुन्धे । \newline
37. रु॒न्धे॒ सं॒ॅयानी॑भिः सं॒ॅयानी॑भी रुन्धे रुन्धे सं॒ॅयानी॑भिः । \newline
38. सं॒ॅयानी॑भि॒र् वै वै सं॒ॅयानी॑भिः सं॒ॅयानी॑भि॒र् वै । \newline
39. सं॒ॅयानी॑भि॒रिति॑ सं - यानी॑भिः । \newline
40. वै दे॒वा दे॒वा वै वै दे॒वाः । \newline
41. दे॒वा इ॒मा नि॒मान् दे॒वा दे॒वा इ॒मान् । \newline
42. इ॒मान् ॅलो॒कान् ॅलो॒का नि॒मा नि॒मान् ॅलो॒कान् । \newline
43. लो॒कान् थ्सꣳ सम् ॅलो॒कान् ॅलो॒कान् थ्सम् । \newline
44. स म॑यु रयुः॒ सꣳ स म॑युः । \newline
45. अ॒यु॒ स्तत् तद॑यु रयु॒ स्तत् । \newline
46. तथ् सं॒ॅयानी॑नाꣳ सं॒ॅयानी॑ना॒म् तत् तथ् सं॒ॅयानी॑नाम् । \newline
47. सं॒ॅयानी॑नाꣳ संॅयानि॒त्वꣳ सं॑ॅयानि॒त्वꣳ सं॒ॅयानी॑नाꣳ सं॒ॅयानी॑नाꣳ संॅयानि॒त्वम् । \newline
48. सं॒ॅयानी॑ना॒मिति॑ सं - यानी॑नाम् । \newline
49. सं॒ॅया॒नि॒त्वं ॅयद् यथ् सं॑ॅयानि॒त्वꣳ सं॑ॅयानि॒त्वं ॅयत् । \newline
50. सं॒ॅया॒नि॒त्वमिति॑ संॅयानि - त्वम् । \newline
51. यथ् सं॒ॅयानीः᳚ सं॒ॅयानी॒र् यद् यथ् सं॒ॅयानीः᳚ । \newline
52. सं॒ॅयानी॑ रुप॒दधा᳚ त्युप॒दधा॑ति सं॒ॅयानीः᳚ सं॒ॅयानी॑ रुप॒दधा॑ति । \newline
53. सं॒ॅयानी॒रिति॑ सं - यानीः᳚ । \newline
54. उ॒प॒दधा॑ति॒ यथा॒ यथो॑ प॒दधा᳚ त्युप॒दधा॑ति॒ यथा᳚ । \newline
55. उ॒प॒दधा॒तीत्यु॑प - दधा॑ति । \newline
56. यथा॒ ऽफ्स्व॑फ्सु यथा॒ यथा॒ ऽफ्सु । \newline
57. अ॒फ्सु ना॒वा ना॒वा ऽफ्स्व॑फ्सु ना॒वा । \newline
58. अ॒फ्स्वित्य॑प् - सु । \newline
59. ना॒वा सं॒ॅयाति॑ सं॒ॅयाति॑ ना॒वा ना॒वा सं॒ॅयाति॑ । \newline
60. सं॒ॅयात्ये॒व मे॒वꣳ सं॒ॅयाति॑ सं॒ॅयात्ये॒वम् । \newline
61. सं॒ॅयातीति॑ सं - याति॑ । \newline
62. ए॒व मे॒वै वैव मे॒व मे॒व । \newline

\textbf{Ghana Paata } \newline

1. वृ॒ष्टि॒सनी॒ रुपोप॑ वृष्टि॒सनी᳚र् वृष्टि॒सनी॒ रुप॑ दधाति दधा॒ त्युप॑ वृष्टि॒सनी᳚र् वृष्टि॒सनी॒ रुप॑ दधाति । \newline
2. वृ॒ष्टि॒सनी॒रिति॑ वृष्टि - सनीः᳚ । \newline
3. उप॑ दधाति दधा॒ त्युपोप॑ दधाति॒ वृष्टिं॒ ॅवृष्टि॑म् दधा॒ त्युपोप॑ दधाति॒ वृष्टि᳚म् । \newline
4. द॒धा॒ति॒ वृष्टिं॒ ॅवृष्टि॑म् दधाति दधाति॒ वृष्टि॑ मे॒वैव वृष्टि॑म् दधाति दधाति॒ वृष्टि॑ मे॒व । \newline
5. वृष्टि॑ मे॒वैव वृष्टिं॒ ॅवृष्टि॑ मे॒वावा वै॒व वृष्टिं॒ ॅवृष्टि॑ मे॒वाव॑ । \newline
6. ए॒वावा वै॒वै वाव॑ रुन्धे रु॒न्धे ऽवै॒वै वाव॑ रुन्धे । \newline
7. अव॑ रुन्धे रु॒न्धे ऽवाव॑ रुन्धे॒ यद् यद् रु॒न्धे ऽवाव॑ रुन्धे॒ यत् । \newline
8. रु॒न्धे॒ यद् यद् रु॑न्धे रुन्धे॒ यदे॑क॒ धैक॒धा यद् रु॑न्धे रुन्धे॒ यदे॑क॒धा । \newline
9. यदे॑क॒ धैक॒धा यद् यदे॑क॒ धोप॑द॒द्ध्या दु॑पद॒द्ध्या दे॑क॒धा यद् यदे॑क॒ धोप॑द॒द्ध्यात् । \newline
10. ए॒क॒ धोप॑द॒द्ध्या दु॑पद॒द्ध्या दे॑क॒ धैक॒धोप॑ द॒द्ध्या देक॒ मेक॑ मुपद॒द्ध्या दे॑क॒ धैक॒धो प॑द॒द्ध्या देक᳚म् । \newline
11. ए॒क॒धेत्ये॑क - धा । \newline
12. उ॒प॒द॒द्ध्या देक॒ मेक॑ मुपद॒द्ध्या दु॑पद॒द्ध्या देक॑ मृ॒तु मृ॒तु मेक॑ मुपद॒द्ध्या दु॑पद॒द्ध्या देक॑ मृ॒तुम् । \newline
13. उ॒प॒द॒द्ध्यादित्यु॑प - द॒द्ध्यात् । \newline
14. एक॑ मृ॒तु मृ॒तु मेक॒ मेक॑ मृ॒तुं ॅव॑र्.षेद् वर्.षे दृ॒तु मेक॒ मेक॑ मृ॒तुं ॅव॑र्.षेत् । \newline
15. ऋ॒तुं ॅव॑र्.षेद् वर्.षे दृ॒तु मृ॒तुं ॅव॑र्.षे दनुपरि॒हार॑ मनुपरि॒हारं॑ ॅवर्.षे दृ॒तु मृ॒तुं ॅव॑र्.षे दनुपरि॒हार᳚म् । \newline
16. व॒र्॒.षे॒ द॒नु॒प॒रि॒हार॑ मनुपरि॒हारं॑ ॅवर्.षेद् वर्.षेद् अनुपरि॒हारꣳ॑ सादयति सादय त्यनुपरि॒हारं॑ ॅवर्.षेद् वर्.षे दनुपरि॒हारꣳ॑ सादयति । \newline
17. अ॒नु॒प॒रि॒हारꣳ॑ सादयति सादय त्यनुपरि॒हार॑ मनुपरि॒हारꣳ॑ सादयति॒ तस्मा॒त् तस्मा᳚थ् सादय त्यनुपरि॒हार॑ मनुपरि॒हारꣳ॑ सादयति॒ तस्मा᳚त् । \newline
18. अ॒नु॒प॒रि॒हार॒मित्य॑नु - प॒रि॒हार᳚म् । \newline
19. सा॒द॒य॒ति॒ तस्मा॒त् तस्मा᳚थ् सादयति सादयति॒ तस्मा॒थ् सर्वा॒न् थ्सर्वा॒न् तस्मा᳚थ् सादयति सादयति॒ तस्मा॒थ् सर्वान्॑ । \newline
20. तस्मा॒थ् सर्वा॒न् थ्सर्वा॒न् तस्मा॒त् तस्मा॒थ् सर्वा॑ नृ॒तू नृ॒तून् थ्सर्वा॒न् तस्मा॒त् तस्मा॒थ् सर्वा॑ नृ॒तून् । \newline
21. सर्वा॑ नृ॒तू नृ॒तून् थ्सर्वा॒न् थ्सर्वा॑ नृ॒तून्. व॑र्.षति वर्.ष त्यृ॒तून् थ्सर्वा॒न् थ्सर्वा॑ नृ॒तून्. व॑र्.षति । \newline
22. ऋ॒तून्. व॑र्.षति वर्.ष त्यृ॒तू नृ॒तून्. व॑र्.षति पुरोवात॒सनिः॑ पुरोवात॒सनि॑र् वर्.ष त्यृ॒तू नृ॒तून्. व॑र्.षति पुरोवात॒सनिः॑ । \newline
23. व॒र्॒.ष॒ति॒ पु॒रो॒वा॒त॒सनिः॑ पुरोवात॒सनि॑र् वर्.षति वर्.षति पुरोवात॒सनि॑ रस्यसि पुरोवात॒सनि॑र् वर्.षति वर्.षति पुरोवात॒सनि॑ रसि । \newline
24. पु॒रो॒वा॒त॒सनि॑ रस्यसि पुरोवात॒सनिः॑ पुरोवात॒सनि॑ र॒सीती त्य॑सि पुरोवात॒सनिः॑ पुरोवात॒सनि॑ र॒सीति॑ । \newline
25. पु॒रो॒वा॒त॒सनि॒रिति॑ पुरोवात - सनिः॑ । \newline
26. अ॒सीती त्य॑स्य॒सी त्या॑हा॒हे त्य॑स्य॒सीत्या॑ह । \newline
27. इत्या॑हा॒हे तीत्या॑है॒त दे॒त दा॒हे तीत्या॑ है॒तत् । \newline
28. आ॒है॒त दे॒तदा॑हा है॒तद् वै वा ए॒त दा॑हा है॒तद् वै । \newline
29. ए॒तद् वै वा ए॒त दे॒तद् वै वृष्ट्यै॒ वृष्ट्यै॒ वा ए॒त दे॒तद् वै वृष्ट्यै᳚ । \newline
30. वै वृष्ट्यै॒ वृष्ट्यै॒ वै वै वृष्ट्यै॑ रू॒पꣳ रू॒पं ॅवृष्ट्यै॒ वै वै वृष्ट्यै॑ रू॒पम् । \newline
31. वृष्ट्यै॑ रू॒पꣳ रू॒पं ॅवृष्ट्यै॒ वृष्ट्यै॑ रू॒पꣳ रू॒पेण॑ रू॒पेण॑ रू॒पं ॅवृष्ट्यै॒ वृष्ट्यै॑ रू॒पꣳ रू॒पेण॑ । \newline
32. रू॒पꣳ रू॒पेण॑ रू॒पेण॑ रू॒पꣳ रू॒पꣳ रू॒पेणै॒वैव रू॒पेण॑ रू॒पꣳ रू॒पꣳ रू॒पेणै॒व । \newline
33. रू॒पेणै॒वैव रू॒पेण॑ रू॒पेणै॒व वृष्टिं॒ ॅवृष्टि॑ मे॒व रू॒पेण॑ रू॒पेणै॒व वृष्टि᳚म् । \newline
34. ए॒व वृष्टिं॒ ॅवृष्टि॑ मे॒वैव वृष्टि॒ मवाव॒ वृष्टि॑ मे॒वैव वृष्टि॒ मव॑ । \newline
35. वृष्टि॒ मवाव॒ वृष्टिं॒ ॅवृष्टि॒ मव॑ रुन्धे रु॒न्धे ऽव॒ वृष्टिं॒ ॅवृष्टि॒ मव॑ रुन्धे । \newline
36. अव॑ रुन्धे रु॒न्धे ऽवाव॑ रुन्धे सं॒ॅयानी॑भिः सं॒ॅयानी॑भी रु॒न्धे ऽवाव॑ रुन्धे सं॒ॅयानी॑भिः । \newline
37. रु॒न्धे॒ सं॒ॅयानी॑भिः सं॒ॅयानी॑भी रुन्धे रुन्धे सं॒ॅयानी॑भि॒र् वै वै सं॒ॅयानी॑भी रुन्धे रुन्धे सं॒ॅयानी॑भि॒र् वै । \newline
38. सं॒ॅयानी॑भि॒र् वै वै सं॒ॅयानी॑भिः सं॒ॅयानी॑भि॒र् वै दे॒वा दे॒वा वै सं॒ॅयानी॑भिः सं॒ॅयानी॑भि॒र् वै दे॒वाः । \newline
39. सं॒ॅयानी॑भि॒रिति॑ सं - यानी॑भिः । \newline
40. वै दे॒वा दे॒वा वै वै दे॒वा इ॒मा नि॒मान् दे॒वा वै वै दे॒वा इ॒मान् । \newline
41. दे॒वा इ॒मा नि॒मान् दे॒वा दे॒वा इ॒मान् ॅलो॒कान् ॅलो॒का नि॒मान् दे॒वा दे॒वा इ॒मान् ॅलो॒कान् । \newline
42. इ॒मान् ॅलो॒कान् ॅलो॒का नि॒मा नि॒मान् ॅलो॒कान् थ्सꣳ सम् ॅलो॒का नि॒मा नि॒मान् ॅलो॒कान् थ्सम् । \newline
43. लो॒कान् थ्सꣳ सम् ॅलो॒कान् ॅलो॒कान् थ्स म॑यु रयुः॒ सम् ॅलो॒कान् ॅलो॒कान् थ्स म॑युः । \newline
44. स म॑यु रयुः॒ सꣳ स म॑यु॒ स्तत् तद॑युः॒ सꣳ स म॑यु॒ स्तत् । \newline
45. अ॒यु॒स्तत् तद॑यु रयु॒ स्तथ् सं॒ॅयानी॑नाꣳ सं॒ॅयानी॑ना॒म् तद॑यु रयु॒ स्तथ् सं॒ॅयानी॑नाम् । \newline
46. तथ् सं॒ॅयानी॑नाꣳ सं॒ॅयानी॑ना॒म् तत् तथ् सं॒ॅयानी॑नाꣳ संॅयानि॒त्वꣳ सं॑ॅयानि॒त्वꣳ सं॒ॅयानी॑ना॒म् तत् तथ् सं॒ॅयानी॑नाꣳ संॅयानि॒त्वम् । \newline
47. सं॒ॅयानी॑नाꣳ संॅयानि॒त्वꣳ सं॑ॅयानि॒त्वꣳ सं॒ॅयानी॑नाꣳ सं॒ॅयानी॑नाꣳ संॅयानि॒त्वं ॅयद् यथ् सं॑ॅयानि॒त्वꣳ सं॒ॅयानी॑नाꣳ सं॒ॅयानी॑नाꣳ संॅयानि॒त्वं ॅयत् । \newline
48. सं॒ॅयानी॑ना॒मिति॑ सं - यानी॑नाम् । \newline
49. सं॒ॅया॒नि॒त्वं ॅयद् यथ् सं॑ॅयानि॒त्वꣳ सं॑ॅयानि॒त्वं ॅयथ् सं॒ॅयानीः᳚ सं॒ॅयानी॒र् यथ् सं॑ॅयानि॒त्वꣳ सं॑ॅयानि॒त्वं ॅयथ् सं॒ॅयानीः᳚ । \newline
50. सं॒ॅया॒नि॒त्वमिति॑ संॅयानि - त्वम् । \newline
51. यथ् सं॒ॅयानीः᳚ सं॒ॅयानी॒र् यद् यथ् सं॒ॅयानी॑ रुप॒दधा᳚ त्युप॒दधा॑ति सं॒ॅयानी॒र् यद् यथ् सं॒ॅयानी॑ रुप॒दधा॑ति । \newline
52. सं॒ॅयानी॑ रुप॒दधा᳚ त्युप॒दधा॑ति सं॒ॅयानीः᳚ सं॒ॅयानी॑ रुप॒दधा॑ति॒ यथा॒ यथो॑ प॒दधा॑ति सं॒ॅयानीः᳚ सं॒ॅयानी॑ रुप॒दधा॑ति॒ यथा᳚ । \newline
53. सं॒ॅयानी॒रिति॑ सं - यानीः᳚ । \newline
54. उ॒प॒दधा॑ति॒ यथा॒ यथो॑ प॒दधा᳚ त्युप॒दधा॑ति॒ यथा॒ ऽफ्स्व॑फ्सु यथो॑ प॒दधा᳚ त्युप॒दधा॑ति॒ यथा॒ ऽफ्सु । \newline
55. उ॒प॒दधा॒तीत्यु॑प - दधा॑ति । \newline
56. यथा॒ ऽफ्स्व॑फ्सु यथा॒ यथा॒ ऽफ्सु ना॒वा ना॒वा ऽफ्सु यथा॒ यथा॒ ऽफ्सु ना॒वा । \newline
57. अ॒फ्सु ना॒वा ना॒वा ऽफ्स्व॑फ्सु ना॒वा सं॒ॅयाति॑ सं॒ॅयाति॑ ना॒वा ऽफ्स्व॑फ्सु ना॒वा सं॒ॅयाति॑ । \newline
58. अ॒फ्स्वित्य॑प् - सु । \newline
59. ना॒वा सं॒ॅयाति॑ सं॒ॅयाति॑ ना॒वा ना॒वा सं॒ॅया त्ये॒व मे॒वꣳ सं॒ॅयाति॑ ना॒वा ना॒वा सं॒ॅया त्ये॒वम् । \newline
60. सं॒ॅया त्ये॒व मे॒वꣳ सं॒ॅयाति॑ सं॒ॅया त्ये॒व मे॒वै वैवꣳ सं॒ॅयाति॑ सं॒ॅया त्ये॒व मे॒व । \newline
61. सं॒ॅयातीति॑ सं - याति॑ । \newline
62. ए॒व मे॒वै वैव मे॒व मे॒वै ताभि॑ रे॒ताभि॑ रे॒वैव मे॒व मे॒वै ताभिः॑ । \newline
\pagebreak
\markright{ TS 5.3.10.2  \hfill https://www.vedavms.in \hfill}

\section{ TS 5.3.10.2 }

\textbf{TS 5.3.10.2 } \newline
\textbf{Samhita Paata} \newline

-मे॒वैताभि॒ र्यज॑मान इ॒मान् ॅलो॒कान्थ् सं ॅया॑ति प्ल॒वो वा ए॒षो᳚ऽग्नेर्यथ् सं॒ॅयानी॒र्यथ् सं॒ॅयानी॑रुप॒दधा॑ति प्ल॒वमे॒वैतम॒ग्नय॒ उप॑दधात्यु॒त यस्यै॒तासूप॑हिता॒स्वापो॒ऽग्निꣳ हर॒न्त्यहृ॑त ए॒वास्या॒-ग्निरा॑दित्येष्ट॒का उप॑ दधात्यादि॒त्या वा ए॒तं भूत्यै॒ प्रति॑नुदन्ते॒ योऽलं॒ भूत्यै॒ सन् भूतिं॒ न प्रा॒प्नोत्या॑दि॒त्या - [  ] \newline

\textbf{Pada Paata} \newline

ए॒व । ए॒ताभिः॑ । यज॑मानः । इ॒मान् । लो॒कान् । समिति॑ । या॒ति॒ । प्ल॒वः । वै । ए॒षः । अ॒ग्नेः । यत् । सं॒ॅयानी॒रिति॑ सं - यानीः᳚ । यत् । सं॒ॅयानी॒रिति॑ सं - यानीः᳚ । उ॒प॒दधा॒तीत्यु॑प-दधा॑ति । प्ल॒वम् । ए॒व । ए॒तम् । अ॒ग्नये᳚ । उपेति॑ । द॒धा॒ति॒ । उ॒त । यस्य॑ । ए॒तासु॑ । उप॑हिता॒स्वित्युप॑ - हि॒ता॒सु॒ । आपः॑ । अ॒ग्निम् । हर॑न्ति । अहृ॑तः । ए॒व । अ॒स्य॒ । अ॒ग्निः । आ॒दि॒त्ये॒ष्ट॒का इत्या॑दित्य - इ॒ष्ट॒काः । उपेति॑ । द॒धा॒ति॒ । आ॒दि॒त्याः । वै । ए॒तम् । भूत्यै᳚ । प्रतीति॑ । नु॒द॒न्ते॒ । यः । अल᳚म् । भूत्यै᳚ । सन्न् । भूति᳚म् । न । प्रा॒प्नोतीति॑ प्र - आ॒प्नोति॑ । आ॒दि॒त्याः ।  \newline


\textbf{Krama Paata} \newline

ए॒वैताभिः॑ । ए॒ताभि॒र् यज॑मानः । यज॑मान इ॒मान् । इ॒मान् ॅलो॒कान् । लो॒कान्थ् सम् । सम् ॅया॑ति । या॒ति॒ प्ल॒वः । प्ल॒वो वै । वा ए॒षः । ए॒षो᳚ऽग्नेः । अ॒ग्नेर् यत् । यथ् स॒म्ॅयानीः᳚ । स॒म्ॅयानी॒र् यत् । स॒म्ॅया॒नीरिति॑ सम् - यानीः᳚ । यथ् स॒म्ॅयानीः᳚ । स॒म्ॅयानी॑रुप॒दधा॑ति । स॒म्ॅयानी॒रिति॑ सम् - यानीः᳚ । उ॒प॒दधा॑ति प्ल॒वम् । उ॒प॒दधा॒तीत्यु॑प - दधा॑ति । प्ल॒वमे॒व । ए॒वैतम् । ए॒तम॒ग्नये᳚ । अ॒ग्नय॒ उप॑ । उप॑ दधाति । द॒धा॒त्यु॒त । उ॒त यस्य॑ । यस्यै॒तासु॑ । ए॒तासूप॑हितासु । उप॑हिता॒स्वापः॑ । उप॑हिता॒स्वित्युप॑ - हि॒ता॒सु॒ । आपो॒ऽग्निम् । अ॒ग्निꣳ हर॑न्ति । हर॒न्त्यहृ॑तः । अहृ॑त ए॒व । ए॒वास्य॑ । अ॒स्या॒ग्निः । अ॒ग्निरा॑दित्येष्ट॒काः । आ॒दि॒त्ये॒ष्ट॒का उप॑ । आ॒दि॒त्ये॒ष्ट॒का इत्या॑दित्य - इ॒ष्ट॒काः । उप॑ दधाति । द॒धा॒त्या॒दि॒त्याः । आ॒दि॒त्या वै । वा ए॒तम् । ए॒तम् भूत्यै᳚ । भूत्यै॒ प्रति॑ । प्रति॑ नुदन्ते । नु॒द॒न्ते॒ यः । योऽल᳚म् । अल॒म् भूत्यै᳚ । भूत्यै॒ सन्न् । सन् भूति᳚म् । भूति॒म् न । न प्रा॒प्नोति॑ । प्रा॒प्नोत्या॑दि॒त्याः । प्रा॒प्नोतीति॑ प्र - आ॒प्नोति॑ । आ॒दि॒त्या ए॒व \newline

\textbf{Jatai Paata} \newline

1. ए॒वैताभि॑ रे॒ताभि॑ रे॒वैवैताभिः॑ । \newline
2. ए॒ताभि॒र् यज॑मानो॒ यज॑मान ए॒ताभि॑ रे॒ताभि॒र् यज॑मानः । \newline
3. यज॑मान इ॒मा नि॒मान्. यज॑मानो॒ यज॑मान इ॒मान् । \newline
4. इ॒मान् ॅलो॒कान् ॅलो॒का नि॒मा नि॒मान् ॅलो॒कान् । \newline
5. लो॒कान् थ्सꣳ सम् ॅलो॒कान् ॅलो॒कान् थ्सम् । \newline
6. सं ॅया॑ति याति॒ सꣳ सं ॅया॑ति । \newline
7. या॒ति॒ प्ल॒वः प्ल॒वो या॑ति याति प्ल॒वः । \newline
8. प्ल॒वो वै वै प्ल॒वः प्ल॒वो वै । \newline
9. वा ए॒ष ए॒ष वै वा ए॒षः । \newline
10. ए॒षो᳚ ऽग्ने र॒ग्ने रे॒ष ए॒षो᳚ ऽग्नेः । \newline
11. अ॒ग्नेर् यद् यद॒ग्ने र॒ग्नेर् यत् । \newline
12. यथ् सं॒ॅयानीः᳚ सं॒ॅयानी॒र् यद् यथ् सं॒ॅयानीः᳚ । \newline
13. सं॒ॅयानी॒र् यद् यथ् सं॒ॅयानीः᳚ सं॒ॅयानी॒र् यत् । \newline
14. सं॒ॅयानी॒रिति॑ सं - यानीः᳚ । \newline
15. यथ् सं॒ॅयानीः᳚ सं॒ॅयानी॒र् यद् यथ् सं॒ॅयानीः᳚ । \newline
16. सं॒ॅयानी॑ रुप॒दधा᳚ त्युप॒दधा॑ति सं॒ॅयानीः᳚ सं॒ॅयानी॑ रुप॒दधा॑ति । \newline
17. सं॒ॅयानी॒रिति॑ सं - यानीः᳚ । \newline
18. उ॒प॒दधा॑ति प्ल॒वम् प्ल॒व मु॑प॒दधा᳚ त्युप॒दधा॑ति प्ल॒वम् । \newline
19. उ॒प॒दधा॒तीत्यु॑प - दधा॑ति । \newline
20. प्ल॒व मे॒वैव प्ल॒वम् प्ल॒व मे॒व । \newline
21. ए॒वैत मे॒त मे॒वै वैतम् । \newline
22. ए॒त म॒ग्नये॒ ऽग्नय॑ ए॒त मे॒त म॒ग्नये᳚ । \newline
23. अ॒ग्नय॒ उपोपा॒ ग्नये॒ ऽग्नय॒ उप॑ । \newline
24. उप॑ दधाति दधा॒ त्युपोप॑ दधाति । \newline
25. द॒धा॒ त्यु॒तोत द॑धाति दधा त्यु॒त । \newline
26. उ॒त यस्य॒ यस्यो॒तोत यस्य॑ । \newline
27. यस्यै॒ता स्वे॒तासु॒ यस्य॒ यस्यै॒तासु॑ । \newline
28. ए॒तासू प॑हिता॒सू प॑हिता स्वे॒ता स्वे॒तासू प॑हितासु । \newline
29. उप॑हिता॒ स्वाप॒ आप॒ उप॑हिता॒सू प॑हिता॒ स्वापः॑ । \newline
30. उप॑हिता॒स्वित्युप॑ - हि॒ता॒सु॒ । \newline
31. आपो॒ ऽग्नि म॒ग्नि माप॒ आपो॒ ऽग्निम् । \newline
32. अ॒ग्निꣳ हर॑न्ति॒ हर॑न्त्य॒ग्नि म॒ग्निꣳ हर॑न्ति । \newline
33. हर॒न् त्यहृ॒तो ऽहृ॑तो॒ हर॑न्ति॒ हर॒न् त्यहृ॑तः । \newline
34. अहृ॑त ए॒वैवा हृ॒तो ऽहृ॑त ए॒व । \newline
35. ए॒वास्या᳚ स्यै॒वैवास्य॑ । \newline
36. अ॒स्या॒ग्नि र॒ग्नि र॑स्या स्या॒ग्निः । \newline
37. अ॒ग्नि रा॑दित्येष्ट॒का आ॑दित्येष्ट॒का अ॒ग्नि र॒ग्नि रा॑दित्येष्ट॒काः । \newline
38. आ॒दि॒त्ये॒ष्ट॒का उपोपा॑ दित्येष्ट॒का आ॑दित्येष्ट॒का उप॑ । \newline
39. आ॒दि॒त्ये॒ष्ट॒का इत्या॑दित्य - इ॒ष्ट॒काः । \newline
40. उप॑ दधाति दधा॒ त्युपोप॑ दधाति । \newline
41. द॒धा॒ त्या॒दि॒त्या आ॑दि॒त्या द॑धाति दधा त्यादि॒त्याः । \newline
42. आ॒दि॒त्या वै वा आ॑दि॒त्या आ॑दि॒त्या वै । \newline
43. वा ए॒त मे॒तं ॅवै वा ए॒तम् । \newline
44. ए॒तम् भूत्यै॒ भूत्या॑ ए॒त मे॒तम् भूत्यै᳚ । \newline
45. भूत्यै॒ प्रति॒ प्रति॒ भूत्यै॒ भूत्यै॒ प्रति॑ । \newline
46. प्रति॑ नुदन्ते नुदन्ते॒ प्रति॒ प्रति॑ नुदन्ते । \newline
47. नु॒द॒न्ते॒ यो यो नु॑दन्ते नुदन्ते॒ यः । \newline
48. यो ऽल॒ मलं॒ ॅयो यो ऽल᳚म् । \newline
49. अल॒म् भूत्यै॒ भूत्या॒ अल॒ मल॒म् भूत्यै᳚ । \newline
50. भूत्यै॒ सन् थ्सन् भूत्यै॒ भूत्यै॒ सन्न् । \newline
51. सन् भूति॒म् भूतिꣳ॒॒ सन् थ्सन् भूति᳚म् । \newline
52. भूति॒म् न न भूति॒म् भूति॒म् न । \newline
53. न प्रा॒प्नोति॑ प्रा॒प्नोति॒ न न प्रा॒प्नोति॑ । \newline
54. प्रा॒प्नो त्या॑दि॒त्या आ॑दि॒त्याः प्रा॒प्नोति॑ प्रा॒प्नो त्या॑दि॒त्याः । \newline
55. प्रा॒प्नोतीति॑ प्र - आ॒प्नोति॑ । \newline
56. आ॒दि॒त्या ए॒वै वादि॒त्या आ॑दि॒त्या ए॒व । \newline

\textbf{Ghana Paata } \newline

1. ए॒वैताभि॑ रे॒ताभि॑ रे॒वै वैताभि॒र् यज॑मानो॒ यज॑मान ए॒ताभि॑ रे॒वै वैताभि॒र् यज॑मानः । \newline
2. ए॒ताभि॒र् यज॑मानो॒ यज॑मान ए॒ताभि॑ रे॒ताभि॒र् यज॑मान इ॒मा नि॒मान्. यज॑मान ए॒ताभि॑ रे॒ताभि॒र् यज॑मान इ॒मान् । \newline
3. यज॑मान इ॒मा नि॒मान्. यज॑मानो॒ यज॑मान इ॒मान् ॅलो॒कान् ॅलो॒का नि॒मान्. यज॑मानो॒ यज॑मान इ॒मान् ॅलो॒कान् । \newline
4. इ॒मान् ॅलो॒कान् ॅलो॒का नि॒मा नि॒मान् ॅलो॒कान् थ्सꣳ सम् ॅलो॒का नि॒मा नि॒मान् ॅलो॒कान् थ्सम् । \newline
5. लो॒कान् थ्सꣳ सम् ॅलो॒कान् ॅलो॒कान् थ्सं ॅया॑ति याति॒ सम् ॅलो॒कान् ॅलो॒कान् थ्सं ॅया॑ति । \newline
6. सं ॅया॑ति याति॒ सꣳ सं ॅया॑ति प्ल॒वः प्ल॒वो या॑ति॒ सꣳ सं ॅया॑ति प्ल॒वः । \newline
7. या॒ति॒ प्ल॒वः प्ल॒वो या॑ति याति प्ल॒वो वै वै प्ल॒वो या॑ति याति प्ल॒वो वै । \newline
8. प्ल॒वो वै वै प्ल॒वः प्ल॒वो वा ए॒ष ए॒ष वै प्ल॒वः प्ल॒वो वा ए॒षः । \newline
9. वा ए॒ष ए॒ष वै वा ए॒षो᳚ ऽग्ने र॒ग्ने रे॒ष वै वा ए॒षो᳚ ऽग्नेः । \newline
10. ए॒षो᳚ ऽग्ने र॒ग्ने रे॒ष ए॒षो᳚ ऽग्नेर् यद् यद॒ग्ने रे॒ष ए॒षो᳚ ऽग्नेर् यत् । \newline
11. अ॒ग्नेर् यद् यद॒ग्ने र॒ग्नेर् यथ् सं॒ॅयानीः᳚ सं॒ॅयानी॒र् यद॒ग्ने र॒ग्नेर् यथ् सं॒ॅयानीः᳚ । \newline
12. यथ् सं॒ॅयानीः᳚ सं॒ॅयानी॒र् यद् यथ् सं॒ॅयानी॒र् यद् यथ् सं॒ॅयानी॒र् यद् यथ् सं॒ॅयानी॒र् यत् । \newline
13. सं॒ॅयानी॒र् यद् यथ् सं॒ॅयानीः᳚ सं॒ॅयानी॒र् यथ् सं॒ॅयानीः᳚ सं॒ॅयानी॒र् यथ् सं॒ॅयानीः᳚ सं॒ॅयानी॒र् यथ् सं॒ॅयानीः᳚ । \newline
14. सं॒ॅयानी॒रिति॑ सं - यानीः᳚ । \newline
15. यथ् सं॒ॅयानीः᳚ सं॒ॅयानी॒र् यद् यथ् सं॒ॅयानी॑ रुप॒दधा᳚ त्युप॒दधा॑ति सं॒ॅयानी॒र् यद् यथ् सं॒ॅयानी॑ रुप॒दधा॑ति । \newline
16. सं॒ॅयानी॑ रुप॒दधा᳚ त्युप॒दधा॑ति सं॒ॅयानीः᳚ सं॒ॅयानी॑ रुप॒दधा॑ति प्ल॒वम् प्ल॒व मु॑प॒दधा॑ति सं॒ॅयानीः᳚ सं॒ॅयानी॑ रुप॒दधा॑ति प्ल॒वम् । \newline
17. सं॒ॅयानी॒रिति॑ सं - यानीः᳚ । \newline
18. उ॒प॒दधा॑ति प्ल॒वम् प्ल॒व मु॑प॒दधा᳚ त्युप॒दधा॑ति प्ल॒व मे॒वैव प्ल॒व मु॑प॒दधा᳚ त्युप॒दधा॑ति प्ल॒व मे॒व । \newline
19. उ॒प॒दधा॒तीत्यु॑प - दधा॑ति । \newline
20. प्ल॒व मे॒वैव प्ल॒वम् प्ल॒व मे॒वैत मे॒त मे॒व प्ल॒वम् प्ल॒व मे॒वैतम् । \newline
21. ए॒वैत मे॒त मे॒वै वैत म॒ग्नये॒ ऽग्नय॑ ए॒त मे॒वै वैत म॒ग्नये᳚ । \newline
22. ए॒त म॒ग्नये॒ ऽग्नय॑ ए॒त मे॒त म॒ग्नय॒ उपोपा॒ग्नय॑ ए॒त मे॒त म॒ग्नय॒ उप॑ । \newline
23. अ॒ग्नय॒ उपोपा॒ग्नये॒ ऽग्नय॒ उप॑ दधाति दधा॒ त्युपा॒ग्नये॒ ऽग्नय॒ उप॑ दधाति । \newline
24. उप॑ दधाति दधा॒ त्युपोप॑ दधा त्यु॒तोत द॑धा॒ त्युपोप॑ दधा त्यु॒त । \newline
25. द॒धा॒ त्यु॒तोत द॑धाति दधा त्यु॒त यस्य॒ यस्यो॒त द॑धाति दधा त्यु॒त यस्य॑ । \newline
26. उ॒त यस्य॒ यस्यो॒तोत यस्यै॒ता स्वे॒तासु॒ यस्यो॒तोत यस्यै॒तासु॑ । \newline
27. यस्यै॒ तास्वे॒तासु॒ यस्य॒ यस्यै॒ तासू प॑हिता॒सू प॑हिता स्वे॒तासु॒ यस्य॒ यस्यै॒तासू प॑हितासु । \newline
28. ए॒तासू प॑हिता॒सू प॑हिता स्वे॒ता स्वे॒तासू प॑हिता॒ स्वाप॒ आप॒ उप॑हिता स्वे॒ता स्वे॒तासू प॑हिता॒ स्वापः॑ । \newline
29. उप॑हिता॒ स्वाप॒ आप॒ उप॑हिता॒सू प॑हिता॒ स्वापो॒ ऽग्नि म॒ग्नि माप॒ उप॑हिता॒सू प॑हिता॒ स्वापो॒ ऽग्निम् । \newline
30. उप॑हिता॒स्वित्युप॑ - हि॒ता॒सु॒ । \newline
31. आपो॒ ऽग्नि म॒ग्नि माप॒ आपो॒ ऽग्निꣳ हर॑न्ति॒ हर॑न्त्य॒ग्नि माप॒ आपो॒ ऽग्निꣳ हर॑न्ति । \newline
32. अ॒ग्निꣳ हर॑न्ति॒ हर॑न् त्य॒ग्नि म॒ग्निꣳ हर॒न् त्यहृ॒तो ऽहृ॑तो॒ हर॑न् त्य॒ग्नि म॒ग्निꣳ हर॒न् त्यहृ॑तः । \newline
33. हर॒न् त्यहृ॒तो ऽहृ॑तो॒ हर॑न्ति॒ हर॒न् त्यहृ॑त ए॒वै वाहृ॑तो॒ हर॑न्ति॒ हर॒न् त्यहृ॑त ए॒व । \newline
34. अहृ॑त ए॒वै वाहृ॒तो ऽहृ॑त ए॒वास्या᳚ स्यै॒वाहृ॒तो ऽहृ॑त ए॒वास्य॑ । \newline
35. ए॒वास्या᳚ स्यै॒वैवास्या॒ ग्नि र॒ग्नि र॑स्यै॒वैवा स्या॒ग्निः । \newline
36. अ॒स्या॒ ग्नि र॒ग्नि र॑स्या स्या॒ग्नि रा॑दित्येष्ट॒का आ॑दित्येष्ट॒का अ॒ग्नि र॑स्यास्या॒ ग्नि रा॑दित्येष्ट॒काः । \newline
37. अ॒ग्नि रा॑दित्येष्ट॒का आ॑दित्येष्ट॒का अ॒ग्नि र॒ग्नि रा॑दित्येष्ट॒का उपोपा॑ दित्येष्ट॒का अ॒ग्नि र॒ग्नि रा॑दित्येष्ट॒का उप॑ । \newline
38. आ॒दि॒त्ये॒ष्ट॒का उपोपा॑ दित्येष्ट॒का आ॑दित्येष्ट॒का उप॑ दधाति दधा॒ त्युपा॑ दित्येष्ट॒का आ॑दित्येष्ट॒का उप॑ दधाति । \newline
39. आ॒दि॒त्ये॒ष्ट॒का इत्या॑दित्य - इ॒ष्ट॒काः । \newline
40. उप॑ दधाति दधा॒ त्युपोप॑ दधा त्यादि॒त्या आ॑दि॒त्या द॑धा॒ त्युपोप॑ दधा त्यादि॒त्याः । \newline
41. द॒धा॒ त्या॒दि॒त्या आ॑दि॒त्या द॑धाति दधा त्यादि॒त्या वै वा आ॑दि॒त्या द॑धाति दधा त्यादि॒त्या वै । \newline
42. आ॒दि॒त्या वै वा आ॑दि॒त्या आ॑दि॒त्या वा ए॒त मे॒तं ॅवा आ॑दि॒त्या आ॑दि॒त्या वा ए॒तम् । \newline
43. वा ए॒त मे॒तं ॅवै वा ए॒तम् भूत्यै॒ भूत्या॑ ए॒तं ॅवै वा ए॒तम् भूत्यै᳚ । \newline
44. ए॒तम् भूत्यै॒ भूत्या॑ ए॒त मे॒तम् भूत्यै॒ प्रति॒ प्रति॒ भूत्या॑ ए॒त मे॒तम् भूत्यै॒ प्रति॑ । \newline
45. भूत्यै॒ प्रति॒ प्रति॒ भूत्यै॒ भूत्यै॒ प्रति॑ नुदन्ते नुदन्ते॒ प्रति॒ भूत्यै॒ भूत्यै॒ प्रति॑ नुदन्ते । \newline
46. प्रति॑ नुदन्ते नुदन्ते॒ प्रति॒ प्रति॑ नुदन्ते॒ यो यो नु॑दन्ते॒ प्रति॒ प्रति॑ नुदन्ते॒ यः । \newline
47. नु॒द॒न्ते॒ यो यो नु॑दन्ते नुदन्ते॒ यो ऽल॒ मलं॒ ॅयो नु॑दन्ते नुदन्ते॒ यो ऽल᳚म् । \newline
48. यो ऽल॒ मलं॒ ॅयो यो ऽल॒म् भूत्यै॒ भूत्या॒ अलं॒ ॅयो यो ऽल॒म् भूत्यै᳚ । \newline
49. अल॒म् भूत्यै॒ भूत्या॒ अल॒ मल॒म् भूत्यै॒ सन् थ्सन् भूत्या॒ अल॒ मल॒म् भूत्यै॒ सन्न् । \newline
50. भूत्यै॒ सन् थ्सन् भूत्यै॒ भूत्यै॒ सन् भूति॒म् भूतिꣳ॒॒ सन् भूत्यै॒ भूत्यै॒ सन् भूति᳚म् । \newline
51. सन् भूति॒म् भूतिꣳ॒॒ सन् थ्सन् भूति॒म् न न भूतिꣳ॒॒ सन् थ्सन् भूति॒म् न । \newline
52. भूति॒म् न न भूति॒म् भूति॒म् न प्रा॒प्नोति॑ प्रा॒प्नोति॒ न भूति॒म् भूति॒म् न प्रा॒प्नोति॑ । \newline
53. न प्रा॒प्नोति॑ प्रा॒प्नोति॒ न न प्रा॒प्नो त्या॑दि॒त्या आ॑दि॒त्याः प्रा॒प्नोति॒ न न प्रा॒प्नो त्या॑दि॒त्याः । \newline
54. प्रा॒प्नो त्या॑दि॒त्या आ॑दि॒त्याः प्रा॒प्नोति॑ प्रा॒प्नो त्या॑दि॒त्या ए॒वैवा दि॒त्याः प्रा॒प्नोति॑ प्रा॒प्नो त्या॑दि॒त्या ए॒व । \newline
55. प्रा॒प्नोतीति॑ प्र - आ॒प्नोति॑ । \newline
56. आ॒दि॒त्या ए॒वै वादि॒त्या आ॑दि॒त्या ए॒वैन॑ मेन मे॒वादि॒त्या आ॑दि॒त्या ए॒वैन᳚म् । \newline
\pagebreak
\markright{ TS 5.3.10.3  \hfill https://www.vedavms.in \hfill}

\section{ TS 5.3.10.3 }

\textbf{TS 5.3.10.3 } \newline
\textbf{Samhita Paata} \newline

ए॒वैनं॒ भूतिं॑ गमयन्त्य॒सौ वा ए॒तस्या॑ऽऽ*दि॒त्यो रुच॒मा द॑त्ते॒ यो᳚ऽग्निं चि॒त्वा न रोच॑ते॒ यदा॑दित्येष्ट॒का उ॑प॒दधा᳚त्य॒सावे॒-वास्मि॑न्नादि॒त्यो रुचं॑ दधाति॒ यथा॒ऽसौ दे॒वानाꣳ॒॒ रोच॑त ए॒वमे॒वैष म॑नु॒ष्या॑णाꣳ रोचते घृतेष्ट॒का उप॑ दधात्ये॒तद्वा अ॒ग्नेः प्रि॒यं धाम॒ यद्-घृ॒तं प्रि॒येणै॒वैनं॒ धाम्ना॒ सम॑र्द्धय॒त्य - [  ] \newline

\textbf{Pada Paata} \newline

ए॒व । ए॒न॒म् । भूति᳚म् । ग॒म॒य॒न्ति॒ । अ॒सौ । वै । ए॒तस्य॑ । आ॒दि॒त्यः । रुच᳚म् । एति॑ । द॒त्ते॒ । यः । अ॒ग्निम् । चि॒त्वा । न । रोच॑ते । यत् । आ॒दि॒त्ये॒ष्ट॒का इत्या॑दित्य - इ॒ष्ट॒काः । उ॒प॒दधा॒तीत्यु॑प - दधा॑ति । अ॒सौ । ए॒व । अ॒स्मि॒न्न् । आ॒दि॒त्यः । रुच᳚म् । द॒धा॒ति॒ । यथा᳚ । अ॒सौ । दे॒वाना᳚म् । रोच॑ते । ए॒वम् । ए॒व । ए॒षः । म॒नु॒ष्या॑णाम् । रो॒च॒ते॒ । घृ॒ते॒ष्ट॒का इति॑ घृत - इ॒ष्ट॒काः । उपेति॑ । द॒धा॒ति॒ । ए॒तत् । वै । अ॒ग्नेः । प्रि॒यम् । धाम॑ । यत् । घृ॒तम् । प्रि॒येण॑ । ए॒व । ए॒न॒म् । धाम्ना᳚ । समिति॑ । अ॒द्‌र्ध॒य॒ति॒ ।  \newline


\textbf{Krama Paata} \newline

ए॒वैन᳚म् । ए॒न॒म् भूति᳚म् । भूति॑म् गमयन्ति । ग॒म॒य॒न्त्य॒सौ । अ॒सौ वै । वा ए॒तस्य॑ । ए॒तस्या॑दि॒त्यः । आ॒दि॒त्यो रुच᳚म् । रुच॒मा । आ द॑त्ते । द॒त्ते॒ यः । यो᳚ऽग्निम् । अ॒ग्निम् चि॒त्वा । चि॒त्वा न । न रोच॑ते । रोच॑ते॒ यत् । यदा॑दित्येष्ट॒काः । आ॒दि॒त्ये॒ष्ट॒का उ॑प॒दधा॑ति । आ॒दि॒त्ये॒ष्ट॒का इत्या॑दित्य - इ॒ष्ट॒काः । उ॒प॒दधा᳚त्य॒सौ । उ॒प॒दधा॒तीत्यु॑प - दधा॑ति । अ॒सावे॒व । ए॒वास्मिन्न्॑ । अ॒स्मि॒न्ना॒दि॒त्यः । आ॒दि॒त्यो रुच᳚म् । रुच॑म् दधाति । द॒धा॒ति॒ यथा᳚ । यथा॒ऽसौ । अ॒सौ दे॒वाना᳚म् । दे॒वानाꣳ॒॒ रोच॑ते । रोच॑त ए॒वम् । ए॒वमे॒व । ए॒वैषः । ए॒ष म॑नु॒ष्या॑णाम् । म॒नु॒ष्या॑णाꣳ रोचते । रो॒च॒ते॒ घृ॒ते॒ष्ट॒काः । घृ॒ते॒ष्ट॒का उप॑ । घृ॒ते॒ष्ट॒का इति॑ घृत - इ॒ष्ट॒काः । उप॑ दधाति । द॒धा॒त्ये॒तत् । ए॒तद् वै । वा अ॒ग्नेः । अ॒ग्नेः प्रि॒यम् । प्रि॒यम् धाम॑ । धाम॒ यत् । यद् घृ॒तम् । घृ॒तम् प्रि॒येण॑ । प्रि॒येणै॒व । ए॒वैन᳚म् । ए॒न॒म् धाम्ना᳚ । धाम्ना॒ सम् । सम॑र्द्धयति ( ) । अ॒र्द्ध॒य॒त्यथो᳚ \newline

\textbf{Jatai Paata} \newline

1. ए॒वैन॑ मेन मे॒वै वैन᳚म् । \newline
2. ए॒न॒म् भूति॒म् भूति॑ मेन मेन॒म् भूति᳚म् । \newline
3. भूति॑म् गमयन्ति गमयन्ति॒ भूति॒म् भूति॑म् गमयन्ति । \newline
4. ग॒म॒य॒न् त्य॒सा व॒सौ ग॑मयन्ति गमयन् त्य॒सौ । \newline
5. अ॒सौ वै वा अ॒सा व॒सौ वै । \newline
6. वा ए॒त स्यै॒तस्य॒ वै वा ए॒तस्य॑ । \newline
7. ए॒त स्या॑दि॒त्य आ॑दि॒त्य ए॒त स्यै॒त स्या॑दि॒त्यः । \newline
8. आ॒दि॒त्यो रुचꣳ॒॒ रुच॑ मादि॒त्य आ॑दि॒त्यो रुच᳚म् । \newline
9. रुच॒ मा रुचꣳ॒॒ रुच॒ मा । \newline
10. आ द॑त्ते दत्त॒ आ द॑त्ते । \newline
11. द॒त्ते॒ यो यो द॑त्ते दत्ते॒ यः । \newline
12. यो᳚ ऽग्नि म॒ग्निं ॅयो यो᳚ ऽग्निम् । \newline
13. अ॒ग्निम् चि॒त्वा चि॒त्वा ऽग्नि म॒ग्निम् चि॒त्वा । \newline
14. चि॒त्वा न न चि॒त्वा चि॒त्वा न । \newline
15. न रोच॑ते॒ रोच॑ते॒ न न रोच॑ते । \newline
16. रोच॑ते॒ यद् यद् रोच॑ते॒ रोच॑ते॒ यत् । \newline
17. यदा॑दित्येष्ट॒का आ॑दित्येष्ट॒का यद् यदा॑दित्येष्ट॒काः । \newline
18. आ॒दि॒त्ये॒ष्ट॒का उ॑प॒दधा᳚ त्युप॒दधा᳚ त्यादित्येष्ट॒का आ॑दित्येष्ट॒का उ॑प॒दधा॑ति । \newline
19. आ॒दि॒त्ये॒ष्ट॒का इत्या॑दित्य - इ॒ष्ट॒काः । \newline
20. उ॒प॒दधा᳚ त्य॒सा व॒सा वु॑प॒दधा᳚ त्युप॒दधा᳚ त्य॒सौ । \newline
21. उ॒प॒दधा॒तीत्यु॑प - दधा॑ति । \newline
22. अ॒सा वे॒वै वासा व॒सा वे॒व । \newline
23. ए॒वास्मि॑न् नस्मिन् ने॒वै वास्मिन्न्॑ । \newline
24. अ॒स्मि॒न् ना॒दि॒त्य आ॑दि॒त्यो᳚ ऽस्मिन् नस्मिन् नादि॒त्यः । \newline
25. आ॒दि॒त्यो रुचꣳ॒॒ रुच॑ मादि॒त्य आ॑दि॒त्यो रुच᳚म् । \newline
26. रुच॑म् दधाति दधाति॒ रुचꣳ॒॒ रुच॑म् दधाति । \newline
27. द॒धा॒ति॒ यथा॒ यथा॑ दधाति दधाति॒ यथा᳚ । \newline
28. यथा॒ ऽसा व॒सौ यथा॒ यथा॒ ऽसौ । \newline
29. अ॒सौ दे॒वाना᳚म् दे॒वाना॑ म॒सा व॒सौ दे॒वाना᳚म् । \newline
30. दे॒वानाꣳ॒॒ रोच॑ते॒ रोच॑ते दे॒वाना᳚म् दे॒वानाꣳ॒॒ रोच॑ते । \newline
31. रोच॑त ए॒व मे॒वꣳ रोच॑ते॒ रोच॑त ए॒वम् । \newline
32. ए॒व मे॒वै वैव मे॒व मे॒व । \newline
33. ए॒वैष ए॒ष ए॒वै वैषः । \newline
34. ए॒ष म॑नु॒ष्या॑णाम् मनु॒ष्या॑णा मे॒ष ए॒ष म॑नु॒ष्या॑णाम् । \newline
35. म॒नु॒ष्या॑णाꣳ रोचते रोचते मनु॒ष्या॑णाम् मनु॒ष्या॑णाꣳ रोचते । \newline
36. रो॒च॒ते॒ घृ॒ते॒ष्ट॒का घृ॑तेष्ट॒का रो॑चते रोचते घृतेष्ट॒काः । \newline
37. घृ॒ते॒ष्ट॒का उपोप॑ घृतेष्ट॒का घृ॑तेष्ट॒का उप॑ । \newline
38. घृ॒ते॒ष्ट॒का इति॑ घृत - इ॒ष्ट॒काः । \newline
39. उप॑ दधाति दधा॒ त्युपोप॑ दधाति । \newline
40. द॒धा॒ त्ये॒त दे॒तद् द॑धाति दधा त्ये॒तत् । \newline
41. ए॒तद् वै वा ए॒त दे॒तद् वै । \newline
42. वा अ॒ग्ने र॒ग्नेर् वै वा अ॒ग्नेः । \newline
43. अ॒ग्नेः प्रि॒यम् प्रि॒य म॒ग्ने र॒ग्नेः प्रि॒यम् । \newline
44. प्रि॒यम् धाम॒ धाम॑ प्रि॒यम् प्रि॒यम् धाम॑ । \newline
45. धाम॒ यद् यद् धाम॒ धाम॒ यत् । \newline
46. यद् घृ॒तम् घृ॒तं ॅयद् यद् घृ॒तम् । \newline
47. घृ॒तम् प्रि॒येण॑ प्रि॒येण॑ घृ॒तम् घृ॒तम् प्रि॒येण॑ । \newline
48. प्रि॒ये णै॒वैव प्रि॒येण॑ प्रि॒ये णै॒व । \newline
49. ए॒वैन॑ मेन मे॒वै वैन᳚म् । \newline
50. ए॒न॒म् धाम्ना॒ धाम्नै॑न मेन॒म् धाम्ना᳚ । \newline
51. धाम्ना॒ सꣳ सम् धाम्ना॒ धाम्ना॒ सम् । \newline
52. स म॑र्द्धय त्यर्द्धयति॒ सꣳ स म॑र्द्धयति । \newline
53. अ॒र्द्ध॒य॒ त्यथो॒ अथो॑ अर्द्धय त्यर्द्धय॒ त्यथो᳚ । \newline

\textbf{Ghana Paata } \newline

1. ए॒वैन॑ मेन मे॒वै वैन॒म् भूति॒म् भूति॑ मेन मे॒वै वैन॒म् भूति᳚म् । \newline
2. ए॒न॒म् भूति॒म् भूति॑ मेन मेन॒म् भूति॑म् गमयन्ति गमयन्ति॒ भूति॑ मेन मेन॒म् भूति॑म् गमयन्ति । \newline
3. भूति॑म् गमयन्ति गमयन्ति॒ भूति॒म् भूति॑म् गमयन् त्य॒सा व॒सौ ग॑मयन्ति॒ भूति॒म् भूति॑म् गमयन् त्य॒सौ । \newline
4. ग॒म॒ य॒न्त्य॒सा व॒सौ ग॑मयन्ति गमयन् त्य॒सौ वै वा अ॒सौ ग॑मयन्ति गमयन् त्य॒सौ वै । \newline
5. अ॒सौ वै वा अ॒सा व॒सौ वा ए॒त स्यै॒तस्य॒ वा अ॒सा व॒सौ वा ए॒तस्य॑ । \newline
6. वा ए॒त स्यै॒तस्य॒ वै वा ए॒तस्या॑ दि॒त्य आ॑दि॒त्य ए॒तस्य॒ वै वा ए॒तस्या॑ दि॒त्यः । \newline
7. ए॒तस्या॑ दि॒त्य आ॑दि॒त्य ए॒त स्यै॒तस्या॑दि॒त्यो रुचꣳ॒॒ रुच॑ मादि॒त्य ए॒त स्यै॒तस्या॑दि॒त्यो रुच᳚म् । \newline
8. आ॒दि॒त्यो रुचꣳ॒॒ रुच॑ मादि॒त्य आ॑दि॒त्यो रुच॒ मा रुच॑ मादि॒त्य आ॑दि॒त्यो रुच॒ मा । \newline
9. रुच॒ मा रुचꣳ॒॒ रुच॒ मा द॑त्ते दत्त॒ आ रुचꣳ॒॒ रुच॒ मा द॑त्ते । \newline
10. आ द॑त्ते दत्त॒ आ द॑त्ते॒ यो यो द॑त्त॒ आ द॑त्ते॒ यः । \newline
11. द॒त्ते॒ यो यो द॑त्ते दत्ते॒ यो᳚ ऽग्नि म॒ग्निं ॅयो द॑त्ते दत्ते॒ यो᳚ ऽग्निम् । \newline
12. यो᳚ ऽग्नि म॒ग्निं ॅयो यो᳚ ऽग्निम् चि॒त्वा चि॒त्वा ऽग्निं ॅयो यो᳚ ऽग्निम् चि॒त्वा । \newline
13. अ॒ग्निम् चि॒त्वा चि॒त्वा ऽग्नि म॒ग्निम् चि॒त्वा न न चि॒त्वा ऽग्नि म॒ग्निम् चि॒त्वा न । \newline
14. चि॒त्वा न न चि॒त्वा चि॒त्वा न रोच॑ते॒ रोच॑ते॒ न चि॒त्वा चि॒त्वा न रोच॑ते । \newline
15. न रोच॑ते॒ रोच॑ते॒ न न रोच॑ते॒ यद् यद् रोच॑ते॒ न न रोच॑ते॒ यत् । \newline
16. रोच॑ते॒ यद् यद् रोच॑ते॒ रोच॑ते॒ यदा॑दित्येष्ट॒का आ॑दित्येष्ट॒का यद् रोच॑ते॒ रोच॑ते॒ यदा॑दित्येष्ट॒काः । \newline
17. यदा॑दित्येष्ट॒का आ॑दित्येष्ट॒का यद् यदा॑दित्येष्ट॒का उ॑प॒दधा᳚ त्युप॒दधा᳚ त्यादित्येष्ट॒का यद् यदा॑दित्येष्ट॒का उ॑प॒दधा॑ति । \newline
18. आ॒दि॒त्ये॒ष्ट॒का उ॑प॒दधा᳚ त्युप॒दधा᳚त्या दित्येष्ट॒का आ॑दित्येष्ट॒का उ॑प॒दधा᳚ त्य॒सा व॒सा वु॑प॒दधा᳚त्या दित्येष्ट॒का आ॑दित्येष्ट॒का उ॑प॒दधा᳚ त्य॒सौ । \newline
19. आ॒दि॒त्ये॒ष्ट॒का इत्या॑दित्य - इ॒ष्ट॒काः । \newline
20. उ॒प॒दधा᳚ त्य॒सा व॒सा वु॑प॒दधा᳚ त्युप॒दधा᳚ त्य॒सा वे॒वै वासा वु॑प॒दधा᳚ त्युप॒दधा᳚ त्य॒सा वे॒व । \newline
21. उ॒प॒दधा॒तीत्यु॑प - दधा॑ति । \newline
22. अ॒सा वे॒वै वासा व॒सा वे॒वास्मि॑न् नस्मिन् ने॒वासा व॒सा वे॒वास्मिन्न्॑ । \newline
23. ए॒वास्मि॑न् नस्मिन् ने॒वै वास्मि॑न् नादि॒त्य आ॑दि॒त्यो᳚ ऽस्मिन् ने॒वै वास्मि॑न् नादि॒त्यः । \newline
24. अ॒स्मि॒न् ना॒दि॒त्य आ॑दि॒त्यो᳚ ऽस्मिन् नस्मिन् नादि॒त्यो रुचꣳ॒॒ रुच॑ मादि॒त्यो᳚ ऽस्मिन् नस्मिन् नादि॒त्यो रुच᳚म् । \newline
25. आ॒दि॒त्यो रुचꣳ॒॒ रुच॑ मादि॒त्य आ॑दि॒त्यो रुच॑म् दधाति दधाति॒ रुच॑ मादि॒त्य आ॑दि॒त्यो रुच॑म् दधाति । \newline
26. रुच॑म् दधाति दधाति॒ रुचꣳ॒॒ रुच॑म् दधाति॒ यथा॒ यथा॑ दधाति॒ रुचꣳ॒॒ रुच॑म् दधाति॒ यथा᳚ । \newline
27. द॒धा॒ति॒ यथा॒ यथा॑ दधाति दधाति॒ यथा॒ ऽसा व॒सौ यथा॑ दधाति दधाति॒ यथा॒ ऽसौ । \newline
28. यथा॒ ऽसा व॒सौ यथा॒ यथा॒ ऽसौ दे॒वाना᳚म् दे॒वाना॑ म॒सौ यथा॒ यथा॒ ऽसौ दे॒वाना᳚म् । \newline
29. अ॒सौ दे॒वाना᳚म् दे॒वाना॑ म॒सा व॒सौ दे॒वानाꣳ॒॒ रोच॑ते॒ रोच॑ते दे॒वाना॑ म॒सा व॒सौ दे॒वानाꣳ॒॒ रोच॑ते । \newline
30. दे॒वानाꣳ॒॒ रोच॑ते॒ रोच॑ते दे॒वाना᳚म् दे॒वानाꣳ॒॒ रोच॑त ए॒व मे॒वꣳ रोच॑ते दे॒वाना᳚म् दे॒वानाꣳ॒॒ रोच॑त ए॒वम् । \newline
31. रोच॑त ए॒व मे॒वꣳ रोच॑ते॒ रोच॑त ए॒व मे॒वै वैवꣳ रोच॑ते॒ रोच॑त ए॒व मे॒व । \newline
32. ए॒व मे॒वै वैव मे॒व मे॒वैष ए॒ष ए॒वैव मे॒व मे॒वैषः । \newline
33. ए॒वैष ए॒ष ए॒वै वैष म॑नु॒ष्या॑णाम् मनु॒ष्या॑णा मे॒ष ए॒वै वैष म॑नु॒ष्या॑णाम् । \newline
34. ए॒ष म॑नु॒ष्या॑णाम् मनु॒ष्या॑णा मे॒ष ए॒ष म॑नु॒ष्या॑णाꣳ रोचते रोचते मनु॒ष्या॑णा मे॒ष ए॒ष म॑नु॒ष्या॑णाꣳ रोचते । \newline
35. म॒नु॒ष्या॑णाꣳ रोचते रोचते मनु॒ष्या॑णाम् मनु॒ष्या॑णाꣳ रोचते घृतेष्ट॒का घृ॑तेष्ट॒का रो॑चते मनु॒ष्या॑णाम् मनु॒ष्या॑णाꣳ रोचते घृतेष्ट॒काः । \newline
36. रो॒च॒ते॒ घृ॒ते॒ष्ट॒का घृ॑तेष्ट॒का रो॑चते रोचते घृतेष्ट॒का उपोप॑ घृतेष्ट॒का रो॑चते रोचते घृतेष्ट॒का उप॑ । \newline
37. घृ॒ते॒ष्ट॒का उपोप॑ घृतेष्ट॒का घृ॑तेष्ट॒का उप॑ दधाति दधा॒ त्युप॑ घृतेष्ट॒का घृ॑तेष्ट॒का उप॑ दधाति । \newline
38. घृ॒ते॒ष्ट॒का इति॑ घृत - इ॒ष्ट॒काः । \newline
39. उप॑ दधाति दधा॒ त्युपोप॑ दधा त्ये॒त दे॒तद् द॑धा॒ त्युपोप॑ दधा त्ये॒तत् । \newline
40. द॒धा॒ त्ये॒त दे॒तद् द॑धाति दधा त्ये॒तद् वै वा ए॒तद् द॑धाति दधा त्ये॒तद् वै । \newline
41. ए॒तद् वै वा ए॒त दे॒तद् वा अ॒ग्ने र॒ग्नेर् वा ए॒त दे॒तद् वा अ॒ग्नेः । \newline
42. वा अ॒ग्ने र॒ग्नेर् वै वा अ॒ग्नेः प्रि॒यम् प्रि॒य म॒ग्नेर् वै वा अ॒ग्नेः प्रि॒यम् । \newline
43. अ॒ग्नेः प्रि॒यम् प्रि॒य म॒ग्ने र॒ग्नेः प्रि॒यम् धाम॒ धाम॑ प्रि॒य म॒ग्ने र॒ग्नेः प्रि॒यम् धाम॑ । \newline
44. प्रि॒यम् धाम॒ धाम॑ प्रि॒यम् प्रि॒यम् धाम॒ यद् यद् धाम॑ प्रि॒यम् प्रि॒यम् धाम॒ यत् । \newline
45. धाम॒ यद् यद् धाम॒ धाम॒ यद् घृ॒तम् घृ॒तं ॅयद् धाम॒ धाम॒ यद् घृ॒तम् । \newline
46. यद् घृ॒तम् घृ॒तं ॅयद् यद् घृ॒तम् प्रि॒येण॑ प्रि॒येण॑ घृ॒तं ॅयद् यद् घृ॒तम् प्रि॒येण॑ । \newline
47. घृ॒तम् प्रि॒येण॑ प्रि॒येण॑ घृ॒तम् घृ॒तम् प्रि॒येणै॒ वैव प्रि॒येण॑ घृ॒तम् घृ॒तम् प्रि॒येणै॒व । \newline
48. प्रि॒येणै॒ वैव प्रि॒येण॑ प्रि॒येणै॒वैन॑ मेन मे॒व प्रि॒येण॑ प्रि॒येणै॒ वैन᳚म् । \newline
49. ए॒वैन॑ मेन मे॒वैवैन॒म् धाम्ना॒ धाम्नै॑न मे॒वैवैन॒म् धाम्ना᳚ । \newline
50. ए॒न॒म् धाम्ना॒ धाम्नै॑न मेन॒म् धाम्ना॒ सꣳ सम् धाम्नै॑न मेन॒म् धाम्ना॒ सम् । \newline
51. धाम्ना॒ सꣳ सम् धाम्ना॒ धाम्ना॒ स म॑र्द्धय त्यर्द्धयति॒ सम् धाम्ना॒ धाम्ना॒ स म॑र्द्धयति । \newline
52. स म॑र्द्धय त्यर्द्धयति॒ सꣳ स म॑र्द्धय॒ त्यथो॒ अथो॑ अर्द्धयति॒ सꣳ स म॑र्द्धय॒ त्यथो᳚ । \newline
53. अ॒र्द्ध॒य॒ त्यथो॒ अथो॑ अर्द्धय त्यर्द्धय॒ त्यथो॒ तेज॑सा॒ तेज॒सा ऽथो॑ अर्द्धय त्यर्द्धय॒ त्यथो॒ तेज॑सा । \newline
\pagebreak
\markright{ TS 5.3.10.4  \hfill https://www.vedavms.in \hfill}

\section{ TS 5.3.10.4 }

\textbf{TS 5.3.10.4 } \newline
\textbf{Samhita Paata} \newline

-थो॒ तेज॑सा ऽनुपरि॒हारꣳ॑ सादय॒-त्यप॑रिवर्ग-मे॒वास्मि॒न् तेजो॑ दधाति प्र॒जाप॑तिर॒ग्निम॑चिनुत॒ स यश॑सा॒ व्या᳚र्द्ध्यत॒ स ए॒ता य॑शो॒दा अ॑पश्य॒त् ता उपा॑धत्त॒ ताभि॒र्वै स यश॑ आ॒त्मन्न॑धत्त॒ यद्य॑शो॒दा उ॑प॒दधा॑ति॒ यश॑ ए॒व ताभि॒र्यज॑मान आ॒त्मन् ध॑त्ते॒ पञ्चोप॑ दधाति॒ पाङ्क्तः॒ पुरु॑षो॒ यावा॑ने॒व पुरु॑ष॒स्तस्मि॒न्॒ यशो॑ दधाति ॥ \newline

\textbf{Pada Paata} \newline

अथो॒ इति॑ । तेज॑सा । अ॒नु॒प॒रि॒हार॒मित्य॑नु - प॒रि॒हार᳚म् । सा॒द॒य॒ति॒ । अप॑रिवर्ग॒मित्यप॑रि - व॒र्ग॒म् । ए॒व । अ॒स्मि॒न्न् । तेजः॑ । द॒धा॒ति॒ । प्र॒जाप॑ति॒रिति॑ प्र॒जा - प॒तिः॒ । अ॒ग्निम् । अ॒चि॒नु॒त॒ । सः । यश॑सा । वीति॑ । आ॒द्‌र्ध्य॒त॒ । सः । ए॒ताः । य॒शो॒दा इति॑ यशः - दाः । अ॒प॒श्य॒त् । ताः । उपेति॑ । अ॒ध॒त्त॒ । ताभिः॑ । वै । सः । यशः॑ । आ॒त्मन्न् । अ॒ध॒त्त॒ । यत् । य॒शो॒दा इति॑ यशः - दाः । उ॒प॒दधा॒तीत्यु॑प - दधा॑ति । यशः॑ । ए॒व । ताभिः॑ । यज॑मानः । आ॒त्मन्न् । ध॒त्ते॒ । पञ्च॑ । उपेति॑ । द॒धा॒ति॒ । पाङ्क्तः॑ । पुरु॑षः । यावान्॑ । ए॒व । पुरु॑षः । तस्मिन्न्॑ । यशः॑ । द॒धा॒ति॒ ॥  \newline


\textbf{Krama Paata} \newline

अथो॒ तेज॑सा । अथो॒ इत्यथो᳚ । तेज॑साऽनुपरि॒हार᳚म् । अ॒नु॒प॒रि॒हारꣳ॑ सादयति । अ॒नु॒प॒रि॒हार॒मित्य॑नु - प॒रि॒हार᳚म् । सा॒द॒य॒त्यप॑रिवर्गम् । अप॑रिवर्गमे॒व । अप॑रिवर्ग॒मित्यप॑रि - व॒र्ग॒म् । ए॒वास्मिन्न्॑ । अ॒स्मि॒न् तेजः॑ । तेजो॑ दधाति । द॒धा॒ति॒ प्र॒जाप॑तिः । प्र॒जाप॑तिर॒ग्निम् । प्र॒जाप॑ति॒रिति॑ प्र॒जा - प॒तिः॒ । अ॒ग्निम॑चिनुत । अ॒चि॒नु॒त॒ सः । स यश॑सा । यश॑सा॒ वि । व्या᳚र्द्ध्यत । आ॒र्द्ध्य॒त॒ सः । स ए॒ताः । ए॒ता य॑शो॒दाः । य॒शो॒दा अ॑पश्यत् । य॒शो॒दा इति॑ यशः - दाः । अ॒प॒श्य॒त् ताः । ता उप॑ । उपा॑धत्त । अ॒ध॒त्त॒ ताभिः॑ । ताभि॒र् वै । वै सः । स यशः॑ । यश॑ आ॒त्मन्न् । आ॒त्मन्न॑धत्त । अ॒ध॒त्त॒ यत् । यद् य॑शो॒दाः । य॒शो॒दा उ॑प॒दधा॑ति । य॒शो॒दा इति॑ यशः - दाः । उ॒प॒दधा॑ति॒ यशः॑ । उ॒प॒दधा॒तीत्यु॑प - दधा॑ति । यश॑ ए॒व । ए॒व ताभिः॑ । ताभि॒र् यज॑मानः । यज॑मान आ॒त्मन्न् । आ॒त्मन् ध॑त्ते । ध॒त्ते॒ पञ्च॑ । पञ्चोप॑ । उप॑ दधाति । द॒धा॒ति॒ पाङ्क्तः॑ । पाङ्क्तः॒ पुरु॑षः । पुरु॑षो॒ यावान्॑ । यावा॑ने॒व । ए॒व पुरु॑षः । पुरु॑ष॒स्तस्मिन्न्॑ । तस्मि॒न्॒. यशः॑ । यशो॑ दधाति । द॒धा॒तीति॑ दधाति । \newline

\textbf{Jatai Paata} \newline

1. अथो॒ तेज॑सा॒ तेज॒सा ऽथो॒ अथो॒ तेज॑सा । \newline
2. अथो॒ इत्यथो᳚ । \newline
3. तेज॑सा ऽनुपरि॒हार॑ मनुपरि॒हार॒म् तेज॑सा॒ तेज॑सा ऽनुपरि॒हार᳚म् । \newline
4. अ॒नु॒प॒रि॒हारꣳ॑ सादयति सादयत्य नुपरि॒हार॑ मनुपरि॒हारꣳ॑ सादयति । \newline
5. अ॒नु॒प॒रि॒हार॒मित्य॑नु - प॒रि॒हार᳚म् । \newline
6. सा॒द॒य॒ त्यप॑रिवर्ग॒ मप॑रिवर्गꣳ सादयति सादय॒ त्यप॑रिवर्गम् । \newline
7. अप॑रिवर्ग मे॒वैवा प॑रिवर्ग॒ मप॑रिवर्ग मे॒व । \newline
8. अप॑रिवर्ग॒मित्यप॑रि - व॒र्ग॒म् । \newline
9. ए॒वास्मि॑न् नस्मिन् ने॒वै वास्मिन्न्॑ । \newline
10. अ॒स्मि॒न् तेज॒ स्तेजो᳚ ऽस्मिन् नस्मि॒न् तेजः॑ । \newline
11. तेजो॑ दधाति दधाति॒ तेज॒ स्तेजो॑ दधाति । \newline
12. द॒धा॒ति॒ प्र॒जाप॑तिः प्र॒जाप॑तिर् दधाति दधाति प्र॒जाप॑तिः । \newline
13. प्र॒जाप॑ति र॒ग्नि म॒ग्निम् प्र॒जाप॑तिः प्र॒जाप॑ति र॒ग्निम् । \newline
14. प्र॒जाप॑ति॒रिति॑ प्र॒जा - प॒तिः॒ । \newline
15. अ॒ग्नि म॑चिनुता चिनुता॒ग्नि म॒ग्नि म॑चिनुत । \newline
16. अ॒चि॒नु॒त॒ स सो॑ ऽचिनुता चिनुत॒ सः । \newline
17. स यश॑सा॒ यश॑सा॒ स स यश॑सा । \newline
18. यश॑सा॒ वि वि यश॑सा॒ यश॑सा॒ वि । \newline
19. व्या᳚र्द्ध्यता र्द्ध्यत॒ वि व्या᳚र्द्ध्यत । \newline
20. आ॒र्द्ध्य॒त॒ स स आ᳚र्द्ध्यता र्द्ध्यत॒ सः । \newline
21. स ए॒ता ए॒ताः स स ए॒ताः । \newline
22. ए॒ता य॑शो॒दा य॑शो॒दा ए॒ता ए॒ता य॑शो॒दाः । \newline
23. य॒शो॒दा अ॑पश्य दपश्यद् यशो॒दा य॑शो॒दा अ॑पश्यत् । \newline
24. य॒शो॒दा इति॑ यशः - दाः । \newline
25. अ॒प॒श्य॒त् तास्ता अ॑पश्य दपश्य॒त् ताः । \newline
26. ता उपोप॒ ता स्ता उप॑ । \newline
27. उपा॑ धत्ता ध॒त्तोपोपा॑ धत्त । \newline
28. अ॒ध॒त्त॒ ताभि॒ स्ताभि॑ रधत्ता धत्त॒ ताभिः॑ । \newline
29. ताभि॒र् वै वै ताभि॒ स्ताभि॒र् वै । \newline
30. वै स स वै वै सः । \newline
31. स यशो॒ यशः॒ स स यशः॑ । \newline
32. यश॑ आ॒त्मन् ना॒त्मन्. यशो॒ यश॑ आ॒त्मन्न् । \newline
33. आ॒त्मन् न॑धत्ता धत्ता॒त्मन् ना॒त्मन् न॑धत्त । \newline
34. अ॒ध॒त्त॒ यद् यद॑धत्ता धत्त॒ यत् । \newline
35. यद् य॑शो॒दा य॑शो॒दा यद् यद् य॑शो॒दाः । \newline
36. य॒शो॒दा उ॑प॒दधा᳚ त्युप॒दधा॑ति यशो॒दा य॑शो॒दा उ॑प॒दधा॑ति । \newline
37. य॒शो॒दा इति॑ यशः - दाः । \newline
38. उ॒प॒दधा॑ति॒ यशो॒ यश॑ उप॒दधा᳚ त्युप॒दधा॑ति॒ यशः॑ । \newline
39. उ॒प॒दधा॒तीत्यु॑प - दधा॑ति । \newline
40. यश॑ ए॒वैव यशो॒ यश॑ ए॒व । \newline
41. ए॒व ताभि॒ स्ताभि॑ रे॒वैव ताभिः॑ । \newline
42. ताभि॒र् यज॑मानो॒ यज॑मान॒ स्ताभि॒ स्ताभि॒र् यज॑मानः । \newline
43. यज॑मान आ॒त्मन् ना॒त्मन्. यज॑मानो॒ यज॑मान आ॒त्मन्न् । \newline
44. आ॒त्मन् ध॑त्ते धत्त आ॒त्मन् ना॒त्मन् ध॑त्ते । \newline
45. ध॒त्ते॒ पञ्च॒ पञ्च॑ धत्ते धत्ते॒ पञ्च॑ । \newline
46. पञ्चोपोप॒ पञ्च॒ पञ्चोप॑ । \newline
47. उप॑ दधाति दधा॒ त्युपोप॑ दधाति । \newline
48. द॒धा॒ति॒ पाङ्क्तः॒ पाङ्क्तो॑ दधाति दधाति॒ पाङ्क्तः॑ । \newline
49. पाङ्क्तः॒ पुरु॑षः॒ पुरु॑षः॒ पाङ्क्तः॒ पाङ्क्तः॒ पुरु॑षः । \newline
50. पुरु॑षो॒ यावा॒न्॒. यावा॒न् पुरु॑षः॒ पुरु॑षो॒ यावान्॑ । \newline
51. यावा॑ ने॒वैव यावा॒न्॒. यावा॑ ने॒व । \newline
52. ए॒व पुरु॑षः॒ पुरु॑ष ए॒वैव पुरु॑षः । \newline
53. पुरु॑ष॒ स्तस्मिꣳ॒॒ स्तस्मि॒न् पुरु॑षः॒ पुरु॑ष॒ स्तस्मिन्न्॑ । \newline
54. तस्मि॒न्॒. यशो॒ यश॒ स्तस्मिꣳ॒॒ स्तस्मि॒न्॒. यशः॑ । \newline
55. यशो॑ दधाति दधाति॒ यशो॒ यशो॑ दधाति । \newline
56. द॒धा॒तीति॑ दधाति । \newline

\textbf{Ghana Paata } \newline

1. अथो॒ तेज॑सा॒ तेज॒सा ऽथो॒ अथो॒ तेज॑सा ऽनुपरि॒हार॑ मनुपरि॒हार॒म् तेज॒सा ऽथो॒ अथो॒ तेज॑सा ऽनुपरि॒हार᳚म् । \newline
2. अथो॒ इत्यथो᳚ । \newline
3. तेज॑सा ऽनुपरि॒हार॑ मनुपरि॒हार॒म् तेज॑सा॒ तेज॑सा ऽनुपरि॒हारꣳ॑ सादयति सादय त्यनुपरि॒हार॒म् तेज॑सा॒ तेज॑सा ऽनुपरि॒हारꣳ॑ सादयति । \newline
4. अ॒नु॒प॒रि॒हारꣳ॑ सादयति सादय त्यनुपरि॒हार॑ मनुपरि॒हारꣳ॑ सादय॒ त्यप॑रिवर्ग॒ मप॑रिवर्गꣳ सादय त्यनुपरि॒हार॑ मनुपरि॒हारꣳ॑ सादय॒ त्यप॑रिवर्गम् । \newline
5. अ॒नु॒प॒रि॒हार॒मित्य॑नु - प॒रि॒हार᳚म् । \newline
6. सा॒द॒य॒ त्यप॑रिवर्ग॒ मप॑रिवर्गꣳ सादयति सादय॒ त्यप॑रिवर्ग मे॒वैवा प॑रिवर्गꣳ सादयति सादय॒ त्यप॑रिवर्ग मे॒व । \newline
7. अप॑रिवर्ग मे॒वैवा प॑रिवर्ग॒ मप॑रिवर्ग मे॒वास्मि॑न् नस्मिन् ने॒वा प॑रिवर्ग॒ मप॑रिवर्ग मे॒वास्मिन्न्॑ । \newline
8. अप॑रिवर्ग॒मित्यप॑रि - व॒र्ग॒म् । \newline
9. ए॒वास्मि॑न् नस्मिन् ने॒वैवास्मि॒न् तेज॒ स्तेजो᳚ ऽस्मिन् ने॒वैवास्मि॒न् तेजः॑ । \newline
10. अ॒स्मि॒न् तेज॒ स्तेजो᳚ ऽस्मिन् नस्मि॒न् तेजो॑ दधाति दधाति॒ तेजो᳚ ऽस्मिन् नस्मि॒न् तेजो॑ दधाति । \newline
11. तेजो॑ दधाति दधाति॒ तेज॒ स्तेजो॑ दधाति प्र॒जाप॑तिः प्र॒जाप॑तिर् दधाति॒ तेज॒ स्तेजो॑ दधाति प्र॒जाप॑तिः । \newline
12. द॒धा॒ति॒ प्र॒जाप॑तिः प्र॒जाप॑तिर् दधाति दधाति प्र॒जाप॑ति र॒ग्नि म॒ग्निम् प्र॒जाप॑तिर् दधाति दधाति प्र॒जाप॑ति र॒ग्निम् । \newline
13. प्र॒जाप॑ति र॒ग्नि म॒ग्निम् प्र॒जाप॑तिः प्र॒जाप॑ति र॒ग्नि म॑चिनुता चिनुता॒ग्निम् प्र॒जाप॑तिः प्र॒जाप॑ति र॒ग्नि म॑चिनुत । \newline
14. प्र॒जाप॑ति॒रिति॑ प्र॒जा - प॒तिः॒ । \newline
15. अ॒ग्नि म॑चिनुता चिनुता॒ ग्नि म॒ग्नि म॑चिनुत॒ स सो॑ ऽचिनुता॒ ग्नि म॒ग्नि म॑चिनुत॒ सः । \newline
16. अ॒चि॒नु॒त॒ स सो॑ ऽचिनुता चिनुत॒ स यश॑सा॒ यश॑सा॒ सो॑ ऽचिनुता चिनुत॒ स यश॑सा । \newline
17. स यश॑सा॒ यश॑सा॒ स स यश॑सा॒ वि वि यश॑सा॒ स स यश॑सा॒ वि । \newline
18. यश॑सा॒ वि वि यश॑सा॒ यश॑सा॒ व्या᳚र्द्ध्यता र्द्ध्यत॒ वि यश॑सा॒ यश॑सा॒ व्या᳚र्द्ध्यत । \newline
19. व्या᳚र्द्ध्यता र्द्ध्यत॒ वि व्या᳚र्द्ध्यत॒ स स आ᳚र्द्ध्यत॒ वि व्या᳚र्द्ध्यत॒ सः । \newline
20. आ॒र्द्ध्य॒त॒ स स आ᳚र्द्ध्यता र्द्ध्यत॒ स ए॒ता ए॒ताः स आ᳚र्द्ध्यता र्द्ध्यत॒ स ए॒ताः । \newline
21. स ए॒ता ए॒ताः स स ए॒ता य॑शो॒दा य॑शो॒दा ए॒ताः स स ए॒ता य॑शो॒दाः । \newline
22. ए॒ता य॑शो॒दा य॑शो॒दा ए॒ता ए॒ता य॑शो॒दा अ॑पश्य दपश्यद् यशो॒दा ए॒ता ए॒ता य॑शो॒दा अ॑पश्यत् । \newline
23. य॒शो॒दा अ॑पश्य दपश्यद् यशो॒दा य॑शो॒दा अ॑पश्य॒त् ता स्ता अ॑पश्यद् यशो॒दा य॑शो॒दा अ॑पश्य॒त् ताः । \newline
24. य॒शो॒दा इति॑ यशः - दाः । \newline
25. अ॒प॒श्य॒त् ता स्ता अ॑पश्य दपश्य॒त् ता उपोप॒ ता अ॑पश्य दपश्य॒त् ता उप॑ । \newline
26. ता उपोप॒ ता स्ता उपा॑धत्ता ध॒त्तोप॒ ता स्ता उपा॑धत्त । \newline
27. उपा॑धत्ता ध॒त्तोपोपा॑ धत्त॒ ताभि॒ स्ताभि॑ रध॒त्तोपोपा॑ धत्त॒ ताभिः॑ । \newline
28. अ॒ध॒त्त॒ ताभि॒ स्ताभि॑ रधत्ता धत्त॒ ताभि॒र् वै वै ताभि॑ रधत्ता धत्त॒ ताभि॒र् वै । \newline
29. ताभि॒र् वै वै ताभि॒ स्ताभि॒र् वै स स वै ताभि॒ स्ताभि॒र् वै सः । \newline
30. वै स स वै वै स यशो॒ यशः॒ स वै वै स यशः॑ । \newline
31. स यशो॒ यशः॒ स स यश॑ आ॒त्मन् ना॒त्मन्. यशः॒ स स यश॑ आ॒त्मन्न् । \newline
32. यश॑ आ॒त्मन् ना॒त्मन्. यशो॒ यश॑ आ॒त्मन् न॑धत्ता धत्ता॒त्मन्. यशो॒ यश॑ आ॒त्मन् न॑धत्त । \newline
33. आ॒त्मन् न॑धत्ता धत्ता॒त्मन् ना॒त्मन् न॑धत्त॒ यद् यद॑धत्ता॒त्मन् ना॒त्मन् न॑धत्त॒ यत् । \newline
34. अ॒ध॒त्त॒ यद् यद॑धत्ता धत्त॒ यद् य॑शो॒दा य॑शो॒दा यद॑धत्ता धत्त॒ यद् य॑शो॒दाः । \newline
35. यद् य॑शो॒दा य॑शो॒दा यद् यद् य॑शो॒दा उ॑प॒दधा᳚ त्युप॒दधा॑ति यशो॒दा यद् यद् य॑शो॒दा उ॑प॒दधा॑ति । \newline
36. य॒शो॒दा उ॑प॒दधा᳚ त्युप॒दधा॑ति यशो॒दा य॑शो॒दा उ॑प॒दधा॑ति॒ यशो॒ यश॑ उप॒दधा॑ति यशो॒दा य॑शो॒दा उ॑प॒दधा॑ति॒ यशः॑ । \newline
37. य॒शो॒दा इति॑ यशः - दाः । \newline
38. उ॒प॒दधा॑ति॒ यशो॒ यश॑ उप॒दधा᳚ त्युप॒दधा॑ति॒ यश॑ ए॒वैव यश॑ उप॒दधा᳚ त्युप॒दधा॑ति॒ यश॑ ए॒व । \newline
39. उ॒प॒दधा॒तीत्यु॑प - दधा॑ति । \newline
40. यश॑ ए॒वैव यशो॒ यश॑ ए॒व ताभि॒ स्ताभि॑ रे॒व यशो॒ यश॑ ए॒व ताभिः॑ । \newline
41. ए॒व ताभि॒ स्ताभि॑ रे॒वैव ताभि॒र् यज॑मानो॒ यज॑मान॒ स्ताभि॑ रे॒वैव ताभि॒र् यज॑मानः । \newline
42. ताभि॒र् यज॑मानो॒ यज॑मान॒ स्ताभि॒ स्ताभि॒र् यज॑मान आ॒त्मन् ना॒त्मन्. यज॑मान॒ स्ताभि॒ स्ताभि॒र् यज॑मान आ॒त्मन्न् । \newline
43. यज॑मान आ॒त्मन् ना॒त्मन्. यज॑मानो॒ यज॑मान आ॒त्मन् ध॑त्ते धत्त आ॒त्मन्. यज॑मानो॒ यज॑मान आ॒त्मन् ध॑त्ते । \newline
44. आ॒त्मन् ध॑त्ते धत्त आ॒त्मन् ना॒त्मन् ध॑त्ते॒ पञ्च॒ पञ्च॑ धत्त आ॒त्मन् ना॒त्मन् ध॑त्ते॒ पञ्च॑ । \newline
45. ध॒त्ते॒ पञ्च॒ पञ्च॑ धत्ते धत्ते॒ पञ्चो पोप॒ पञ्च॑ धत्ते धत्ते॒ पञ्चोप॑ । \newline
46. पञ्चोपोप॒ पञ्च॒ पञ्चोप॑ दधाति दधा॒ त्युप॒ पञ्च॒ पञ्चोप॑ दधाति । \newline
47. उप॑ दधाति दधा॒ त्युपोप॑ दधाति॒ पाङ्क्तः॒ पाङ्क्तो॑ दधा॒ त्युपोप॑ दधाति॒ पाङ्क्तः॑ । \newline
48. द॒धा॒ति॒ पाङ्क्तः॒ पाङ्क्तो॑ दधाति दधाति॒ पाङ्क्तः॒ पुरु॑षः॒ पुरु॑षः॒ पाङ्क्तो॑ दधाति दधाति॒ पाङ्क्तः॒ पुरु॑षः । \newline
49. पाङ्क्तः॒ पुरु॑षः॒ पुरु॑षः॒ पाङ्क्तः॒ पाङ्क्तः॒ पुरु॑षो॒ यावा॒न्॒. यावा॒न् पुरु॑षः॒ पाङ्क्तः॒ पाङ्क्तः॒ पुरु॑षो॒ यावान्॑ । \newline
50. पुरु॑षो॒ यावा॒न्॒. यावा॒न् पुरु॑षः॒ पुरु॑षो॒ यावा॑ ने॒वैव यावा॒न् पुरु॑षः॒ पुरु॑षो॒ यावा॑ ने॒व । \newline
51. यावा॑ ने॒वैव यावा॒न्॒. यावा॑ ने॒व पुरु॑षः॒ पुरु॑ष ए॒व यावा॒न्॒. यावा॑ ने॒व पुरु॑षः । \newline
52. ए॒व पुरु॑षः॒ पुरु॑ष ए॒वैव पुरु॑ष॒ स्तस्मिꣳ॒॒ स्तस्मि॒न् पुरु॑ष ए॒वैव पुरु॑ष॒ स्तस्मिन्न्॑ । \newline
53. पुरु॑ष॒ स्तस्मिꣳ॒॒ स्तस्मि॒न् पुरु॑षः॒ पुरु॑ष॒ स्तस्मि॒न्॒. यशो॒ यश॒ स्तस्मि॒न् पुरु॑षः॒ पुरु॑ष॒ स्तस्मि॒न्॒. यशः॑ । \newline
54. तस्मि॒न्॒. यशो॒ यश॒ स्तस्मिꣳ॒॒ स्तस्मि॒न्॒. यशो॑ दधाति दधाति॒ यश॒ स्तस्मिꣳ॒॒ स्तस्मि॒न्॒. यशो॑ दधाति । \newline
55. यशो॑ दधाति दधाति॒ यशो॒ यशो॑ दधाति । \newline
56. द॒धा॒तीति॑ दधाति । \newline
\pagebreak
\markright{ TS 5.3.11.1  \hfill https://www.vedavms.in \hfill}

\section{ TS 5.3.11.1 }

\textbf{TS 5.3.11.1 } \newline
\textbf{Samhita Paata} \newline

दे॒वा॒सु॒राः संॅय॑त्ता आस॒न् कनी॑याꣳसो दे॒वा आस॒न् भूयाꣳ॒॒सो-ऽसु॑रा॒स्ते दे॒वा ए॒ता इष्ट॑का अपश्य॒न् ता उपा॑दधत भूय॒स्कृद॒सीत्ये॒व भूयाꣳ॑सोऽभव॒न् वन॒स्पति॑भि॒-रोष॑धीभि-र्वरिव॒स्कृद॒सीती॒-माम॑जय॒न् प्राच्य॒सीति॒ प्राचीं॒ दिश॑मजयन्नू॒र्द्ध्वा ऽसीत्य॒मूम॑जयन्-नन्तरिक्ष॒सद॑स्य॒न्तरि॑क्षे सी॒देत्य॒-न्तरि॑क्षमजय॒न् ततो॑ दे॒वा अभ॑व॒न् - [  ] \newline

\textbf{Pada Paata} \newline

दे॒वा॒सु॒रा इति॑ देव-अ॒सु॒राः । संॅय॑त्ता॒ इति॒ सं - य॒त्ताः॒ । आ॒स॒न्न् । कनी॑याꣳसः । दे॒वाः । आसन्न्॑ । भूयाꣳ॑सः । असु॑राः । ते । दे॒वाः । ए॒ताः । इष्ट॑काः । अ॒प॒श्य॒न्न् । ताः । उपेति॑ । अ॒द॒ध॒त॒ । भू॒य॒स्कृदिति॑ भूयः - कृत् । अ॒सि॒ । इति॑ । ए॒व । भूयाꣳ॑सः । अ॒भ॒व॒न्न् । वन॒स्पति॑भि॒रिति॒ वन॒स्पति॑ - भिः॒ । ओष॑धीभि॒रित्योष॑धि - भिः॒ । व॒रि॒व॒स्कृदिति॑ वरिवः - कृत् । अ॒सि॒ । इति॑ । इ॒माम् । अ॒ज॒य॒न्न् । प्राची᳚ । अ॒सि॒ । इति॑ । प्राची᳚म् । दिश᳚म् । अ॒ज॒य॒न्न् । ऊ॒द्‌र्ध्वा । अ॒सि॒ । इति॑ । अ॒मूम् । अ॒ज॒य॒न्न् । अ॒न्त॒रि॒क्ष॒सदित्य॑न्तरिक्ष - सत् । अ॒सि॒ । अ॒न्तरि॑क्षे । सी॒द॒ । इति॑ । अ॒न्तरि॑क्षम् । अ॒ज॒य॒न्न् । ततः॑ । दे॒वाः । अभ॑वन्न् ।  \newline


\textbf{Krama Paata} \newline

दे॒वा॒सु॒राः सम्ॅय॑त्ताः । दे॒वा॒सु॒रा इति॑ देव - अ॒सु॒राः । सम्ॅय॑त्ता आसन्न् । सम्ॅय॑त्ता॒ इति॒ सम् - य॒त्ताः॒ । आ॒स॒न् कनी॑याꣳसः । कनी॑याꣳसो दे॒वाः । दे॒वा आसन्न्॑ । आस॒न् भूयाꣳ॑सः । भूयाꣳ॒॒सोऽसु॑राः । असु॑रा॒स्ते । ते दे॒वाः । दे॒वा ए॒ताः । ए॒ता इष्ट॑काः । इष्ट॑का अपश्यन्न् । अ॒प॒श्य॒न् ताः । ता उप॑ । उपा॑दधत । अ॒द॒ध॒त॒ भू॒य॒स्कृत् । भू॒य॒स्कृद॑सि । भू॒य॒स्कृदिति॑ भूयः - कृत् । अ॒सीति॑ । इत्ये॒व । ए॒व भूयाꣳ॑सः । भूयाꣳ॑सोऽभवन्न् । अ॒भ॒व॒न् वन॒स्पति॑भिः । वन॒स्पति॑भि॒रोष॑धीभिः । वन॒स्पति॑भि॒रिति॒ वन॒स्पति॑ - भिः॒ । ओष॑धीभिर् वरिव॒स्कृत् । ओष॑धीभि॒रित्योष॑धि - भिः॒ । व॒रि॒व॒स्कृद॑सि । व॒रि॒व॒स्कृदिति॑ वरिवः - कृत् । अ॒सीति॑ । इती॒माम् । इ॒माम॑जयन्न् । अ॒ज॒य॒न् प्राची᳚ । प्राच्य॑सि । अ॒सीति॑ । इति॒ प्राची᳚म् । प्राची॒म् दिश᳚म् । दिश॑मजयन्न् । अ॒ज॒य॒न्नू॒र्द्ध्वा । ऊ॒र्द्ध्वाऽसि॑ । अ॒सीति॑ । इत्य॒मूम् । अ॒मूम॑जयन्न् । अ॒ज॒य॒न्न॒न्त॒रि॒क्ष॒सत् । अ॒न्त॒रि॒क्ष॒सद॑सि । अ॒न्त॒रि॒क्ष॒सदित्य॑न्तरिक्ष - सत् । अ॒स्य॒न्तरि॑क्षे । अ॒न्तरि॑क्षे सीद । सी॒देति॑ । इत्य॒न्तरि॑क्षम् । अ॒न्तरि॑क्षमजयन्न् । अ॒ज॒य॒न् ततः॑ । ततो॑ दे॒वाः । दे॒वा अभ॑वन्न् । अभ॑व॒न् परा᳚ \newline

\textbf{Jatai Paata} \newline

1. दे॒वा॒सु॒राः संॅय॑त्ताः॒ संॅय॑त्ता देवासु॒रा दे॑वासु॒राः संॅय॑त्ताः । \newline
2. दे॒वा॒सु॒रा इति॑ देव - अ॒सु॒राः । \newline
3. संॅय॑त्ता आसन् नास॒न् थ्संॅय॑त्ताः॒ संॅय॑त्ता आसन्न् । \newline
4. संॅय॑त्ता॒ इति॒ सं - य॒त्ताः॒ । \newline
5. आ॒स॒न् कनी॑याꣳसः॒ कनी॑याꣳस आसन् नास॒न् कनी॑याꣳसः । \newline
6. कनी॑याꣳसो दे॒वा दे॒वाः कनी॑याꣳसः॒ कनी॑याꣳसो दे॒वाः । \newline
7. दे॒वा आस॒न् नास॑न् दे॒वा दे॒वा आसन्न्॑ । \newline
8. आस॒न् भूयाꣳ॑सो॒ भूयाꣳ॑स॒ आस॒न् नास॒न् भूयाꣳ॑सः । \newline
9. भूयाꣳ॒॒सो ऽसु॑रा॒ असु॑रा॒ भूयाꣳ॑सो॒ भूयाꣳ॒॒सो ऽसु॑राः । \newline
10. असु॑रा॒ स्ते ते ऽसु॑रा॒ असु॑रा॒ स्ते । \newline
11. ते दे॒वा दे॒वा स्ते ते दे॒वाः । \newline
12. दे॒वा ए॒ता ए॒ता दे॒वा दे॒वा ए॒ताः । \newline
13. ए॒ता इष्ट॑का॒ इष्ट॑का ए॒ता ए॒ता इष्ट॑काः । \newline
14. इष्ट॑का अपश्यन् नपश्य॒न् निष्ट॑का॒ इष्ट॑का अपश्यन्न् । \newline
15. अ॒प॒श्य॒न् ता स्ता अ॑पश्यन् नपश्य॒न् ताः । \newline
16. ता उपोप॒ ता स्ता उप॑ । \newline
17. उपा॑ दधता दध॒तो पोपा॑ दधत । \newline
18. अ॒द॒ध॒त॒ भू॒य॒स्कृद् भू॑य॒स्कृ द॑दधता दधत भूय॒स्कृत् । \newline
19. भू॒य॒स्कृ द॑स्यसि भूय॒स्कृद् भू॑य॒स्कृ द॑सि । \newline
20. भू॒य॒स्कृदिति॑ भूयः - कृत् । \newline
21. अ॒सीती त्य॑स्य॒ सीति॑ । \newline
22. इत्ये॒ वैवे तीत्ये॒व । \newline
23. ए॒व भूयाꣳ॑सो॒ भूयाꣳ॑स ए॒वैव भूयाꣳ॑सः । \newline
24. भूयाꣳ॑सो ऽभवन् नभव॒न् भूयाꣳ॑सो॒ भूयाꣳ॑सो ऽभवन्न् । \newline
25. अ॒भ॒व॒न्॒. वन॒स्पति॑भि॒र् वन॒स्पति॑भि रभवन् नभव॒न्॒. वन॒स्पति॑भिः । \newline
26. वन॒स्पति॑भि॒ रोष॑धीभि॒ रोष॑धीभि॒र् वन॒स्पति॑भि॒र् वन॒स्पति॑भि॒ रोष॑धीभिः । \newline
27. वन॒स्पति॑भि॒रिति॒ वन॒स्पति॑ - भिः॒ । \newline
28. ओष॑धीभिर् वरिव॒स्कृद् व॑रिव॒स्कृ दोष॑धीभि॒ रोष॑धीभिर् वरिव॒स्कृत् । \newline
29. ओष॑धीभि॒रित्योष॑धि - भिः॒ । \newline
30. व॒रि॒व॒स्कृ द॑स्यसि वरिव॒स्कृद् व॑रिव॒स्कृ द॑सि । \newline
31. व॒रि॒व॒स्कृदिति॑ वरिवः - कृत् । \newline
32. अ॒सीती त्य॑स्य॒ सीति॑ । \newline
33. इती॒मा मि॒मा मितीती॒माम् । \newline
34. इ॒मा म॑जयन् नजयन् नि॒मा मि॒मा म॑जयन्न् । \newline
35. अ॒ज॒य॒न् प्राची॒ प्राच्य॑जयन् नजय॒न् प्राची᳚ । \newline
36. प्राच्य॑स्यसि॒ प्राची॒ प्राच्य॑सि । \newline
37. अ॒सीती त्य॑स्य॒ सीति॑ । \newline
38. इति॒ प्राची॒म् प्राची॒ मितीति॒ प्राची᳚म् । \newline
39. प्राची॒म् दिश॒म् दिश॒म् प्राची॒म् प्राची॒म् दिश᳚म् । \newline
40. दिश॑ मजयन् नजय॒न् दिश॒म् दिश॑ मजयन्न् । \newline
41. अ॒ज॒य॒न् नू॒र्द्ध्वो र्द्ध्वा ऽज॑यन् नजयन् नू॒र्द्ध्वा । \newline
42. ऊ॒र्द्ध्वा ऽस्य॑ स्यू॒र्द्ध्वो र्द्ध्वा ऽसि॑ । \newline
43. अ॒सीती त्य॑स्य॒ सीति॑ । \newline
44. इत्य॒मू म॒मू मिती त्य॒मूम् । \newline
45. अ॒मू म॑जयन् नजयन् न॒मू म॒मू म॑जयन्न् । \newline
46. अ॒ज॒य॒न् न॒न्त॒रि॒क्ष॒स द॑न्तरिक्ष॒स द॑जयन् नजयन् नन्तरिक्ष॒सत् । \newline
47. अ॒न्त॒रि॒क्ष॒स द॑स्यस्य न्तरिक्ष॒स द॑न्तरिक्ष॒स द॑सि । \newline
48. अ॒न्त॒रि॒क्ष॒सदित्य॑न्तरिक्ष - सत् । \newline
49. अ॒स्य॒न्तरि॑क्षे॒ ऽन्तरि॑क्षे ऽस्यस्य॒ न्तरि॑क्षे । \newline
50. अ॒न्तरि॑क्षे सीद सीदा॒ न्तरि॑क्षे॒ ऽन्तरि॑क्षे सीद । \newline
51. सी॒दे तीति॑ सीद सी॒देति॑ । \newline
52. इत्य॒न्तरि॑क्ष म॒न्तरि॑क्ष॒ मितीत्य॒ न्तरि॑क्षम् । \newline
53. अ॒न्तरि॑क्ष मजयन् नजयन् न॒न्तरि॑क्ष म॒न्तरि॑क्ष मजयन्न् । \newline
54. अ॒ज॒य॒न् तत॒ स्ततो॑ ऽजयन् नजय॒न् ततः॑ । \newline
55. ततो॑ दे॒वा दे॒वा स्तत॒ स्ततो॑ दे॒वाः । \newline
56. दे॒वा अभ॑व॒न् नभ॑वन् दे॒वा दे॒वा अभ॑वन्न् । \newline
57. अभ॑व॒न् परा॒ परा ऽभ॑व॒न् नभ॑व॒न् परा᳚ । \newline

\textbf{Ghana Paata } \newline

1. दे॒वा॒सु॒राः संॅय॑त्ताः॒ संॅय॑त्ता देवासु॒रा दे॑वासु॒राः संॅय॑त्ता आसन् नास॒न् थ्संॅय॑त्ता देवासु॒रा दे॑वासु॒राः संॅय॑त्ता आसन्न् । \newline
2. दे॒वा॒सु॒रा इति॑ देव - अ॒सु॒राः । \newline
3. संॅय॑त्ता आसन् नास॒न् थ्संॅय॑त्ताः॒ संॅय॑त्ता आस॒न् कनी॑याꣳसः॒ कनी॑याꣳस आस॒न् थ्संॅय॑त्ताः॒ संॅय॑त्ता आस॒न् कनी॑याꣳसः । \newline
4. संॅय॑त्ता॒ इति॒ सं - य॒त्ताः॒ । \newline
5. आ॒स॒न् कनी॑याꣳसः॒ कनी॑याꣳस आसन् नास॒न् कनी॑याꣳसो दे॒वा दे॒वाः कनी॑याꣳस आसन् नास॒न् कनी॑याꣳसो दे॒वाः । \newline
6. कनी॑याꣳसो दे॒वा दे॒वाः कनी॑याꣳसः॒ कनी॑याꣳसो दे॒वा आस॒न् नास॑न् दे॒वाः कनी॑याꣳसः॒ कनी॑याꣳसो दे॒वा आसन्न्॑ । \newline
7. दे॒वा आस॒न् नास॑न् दे॒वा दे॒वा आस॒न् भूयाꣳ॑सो॒ भूयाꣳ॑स॒ आस॑न् दे॒वा दे॒वा आस॒न् भूयाꣳ॑सः । \newline
8. आस॒न् भूयाꣳ॑सो॒ भूयाꣳ॑स॒ आस॒न् नास॒न् भूयाꣳ॒॒सो ऽसु॑रा॒ असु॑रा॒ भूयाꣳ॑स॒ आस॒न् नास॒न् भूयाꣳ॒॒सो ऽसु॑राः । \newline
9. भूयाꣳ॒॒सो ऽसु॑रा॒ असु॑रा॒ भूयाꣳ॑सो॒ भूयाꣳ॒॒सो ऽसु॑रा॒ स्ते ते ऽसु॑रा॒ भूयाꣳ॑सो॒ भूयाꣳ॒॒सो ऽसु॑रा॒ स्ते । \newline
10. असु॑रा॒ स्ते ते ऽसु॑रा॒ असु॑रा॒ स्ते दे॒वा दे॒वा स्ते ऽसु॑रा॒ असु॑रा॒ स्ते दे॒वाः । \newline
11. ते दे॒वा दे॒वा स्ते ते दे॒वा ए॒ता ए॒ता दे॒वा स्ते ते दे॒वा ए॒ताः । \newline
12. दे॒वा ए॒ता ए॒ता दे॒वा दे॒वा ए॒ता इष्ट॑का॒ इष्ट॑का ए॒ता दे॒वा दे॒वा ए॒ता इष्ट॑काः । \newline
13. ए॒ता इष्ट॑का॒ इष्ट॑का ए॒ता ए॒ता इष्ट॑का अपश्यन् नपश्य॒न् निष्ट॑का ए॒ता ए॒ता इष्ट॑का अपश्यन्न् । \newline
14. इष्ट॑का अपश्यन् नपश्य॒न् निष्ट॑का॒ इष्ट॑का अपश्य॒न् ता स्ता अ॑पश्य॒न् निष्ट॑का॒ इष्ट॑का अपश्य॒न् ताः । \newline
15. अ॒प॒श्य॒न् ता स्ता अ॑पश्यन् नपश्य॒न् ता उपोप॒ ता अ॑पश्यन् नपश्य॒न् ता उप॑ । \newline
16. ता उपोप॒ ता स्ता उपा॑दधता दध॒तोप॒ ता स्ता उपा॑दधत । \newline
17. उपा॑दधता दध॒तोपोपा॑ दधत भूय॒स्कृद् भू॑य॒स्कृ द॑दध॒तोपोपा॑ दधत भूय॒स्कृत् । \newline
18. अ॒द॒ध॒त॒ भू॒य॒स्कृद् भू॑य॒स्कृ द॑दधता दधत भूय॒स्कृ द॑स्यसि भूय॒स्कृ द॑दधता दधत भूय॒स्कृ द॑सि । \newline
19. भू॒य॒स्कृ द॑स्यसि भूय॒स्कृद् भू॑य॒स्कृ द॒सीती त्य॑सि भूय॒स्कृद् भू॑य॒स्कृद॒ सीति॑ । \newline
20. भू॒य॒स्कृदिति॑ भूयः - कृत् । \newline
21. अ॒सीती त्य॑स्य॒सी त्ये॒वैवे त्य॑स्य॒सी त्ये॒व । \newline
22. इत्ये॒ वैवेती त्ये॒व भूयाꣳ॑सो॒ भूयाꣳ॑स ए॒वेती त्ये॒व भूयाꣳ॑सः । \newline
23. ए॒व भूयाꣳ॑सो॒ भूयाꣳ॑स ए॒वैव भूयाꣳ॑सो ऽभवन् नभव॒न् भूयाꣳ॑स ए॒वैव भूयाꣳ॑सो ऽभवन्न् । \newline
24. भूयाꣳ॑सो ऽभवन् नभव॒न् भूयाꣳ॑सो॒ भूयाꣳ॑सो ऽभव॒न्॒. वन॒स्पति॑भि॒र् वन॒स्पति॑भि रभव॒न् भूयाꣳ॑सो॒ भूयाꣳ॑सो ऽभव॒न्॒. वन॒स्पति॑भिः । \newline
25. अ॒भ॒व॒न्॒. वन॒स्पति॑भि॒र् वन॒स्पति॑भि रभवन् नभव॒न्॒. वन॒स्पति॑भि॒ रोष॑धीभि॒ रोष॑धीभि॒र् वन॒स्पति॑भि रभवन् नभव॒न्॒. वन॒स्पति॑भि॒ रोष॑धीभिः । \newline
26. वन॒स्पति॑भि॒ रोष॑धीभि॒ रोष॑धीभि॒र् वन॒स्पति॑भि॒र् वन॒स्पति॑भि॒ रोष॑धीभिर् वरिव॒स्कृद् व॑रिव॒स्कृ दोष॑धीभि॒र् वन॒स्पति॑भि॒र् वन॒स्पति॑भि॒ रोष॑धीभिर् वरिव॒स्कृत् । \newline
27. वन॒स्पति॑भि॒रिति॒ वन॒स्पति॑ - भिः॒ । \newline
28. ओष॑धीभिर् वरिव॒स्कृद् व॑रिव॒स्कृ दोष॑धीभि॒ रोष॑धीभिर् वरिव॒स्कृ द॑स्यसि वरिव॒स्कृ दोष॑धीभि॒ रोष॑धीभिर् वरिव॒स्कृ द॑सि । \newline
29. ओष॑धीभि॒रित्योष॑धि - भिः॒ । \newline
30. व॒रि॒व॒स्कृ द॑स्यसि वरिव॒स्कृद् व॑रिव॒स्कृ द॒सीती त्य॑सि वरिव॒स्कृद् व॑रिव॒स्कृ द॒सीति॑ । \newline
31. व॒रि॒व॒स्कृदिति॑ वरिवः - कृत् । \newline
32. अ॒सीती त्य॑स्य॒ सीती॒मा मि॒मा मित्य॑स्य॒ सीती॒माम् । \newline
33. इती॒मा मि॒मा मितीती॒मा म॑जयन् नजयन् नि॒मा मिती ती॒मा म॑जयन्न् । \newline
34. इ॒मा म॑जयन् नजयन् नि॒मा मि॒मा म॑जय॒न् प्राची॒ प्राच्य॑जयन् नि॒मा मि॒मा म॑जय॒न् प्राची᳚ । \newline
35. अ॒ज॒य॒न् प्राची॒ प्राच्य॑जयन् नजय॒न् प्राच्य॑स्यसि॒ प्राच्य॑जयन् नजय॒न् प्राच्य॑सि । \newline
36. प्राच्य॑ स्यसि॒ प्राची॒ प्राच्य॒ सीती त्य॑सि॒ प्राची॒ प्राच्य॒ सीति॑ । \newline
37. अ॒सीती त्य॑स्य॒ सीति॒ प्राची॒म् प्राची॒ मित्य॑स्य॒सीति॒ प्राची᳚म् । \newline
38. इति॒ प्राची॒म् प्राची॒ मितीति॒ प्राची॒म् दिश॒म् दिश॒म् प्राची॒ मितीति॒ प्राची॒म् दिश᳚म् । \newline
39. प्राची॒म् दिश॒म् दिश॒म् प्राची॒म् प्राची॒म् दिश॑ मजयन् नजय॒न् दिश॒म् प्राची॒म् प्राची॒म् दिश॑ मजयन्न् । \newline
40. दिश॑ मजयन् नजय॒न् दिश॒म् दिश॑ मजयन् नू॒र्द्ध्वो र्द्ध्वा ऽज॑य॒न् दिश॒म् दिश॑ मजयन् नू॒र्द्ध्वा । \newline
41. अ॒ज॒य॒न् नू॒र्द्ध्वो र्द्ध्वा ऽज॑यन् नजयन् नू॒र्द्ध्वा ऽस्य॑ स्यू॒र्द्ध्वा ऽज॑यन् नजयन् नू॒र्द्ध्वा ऽसि॑ । \newline
42. ऊ॒र्द्ध्वा ऽस्य॑स्यू॒र्द्ध्वो र्द्ध्वा ऽसीतीत्य॑ स्यू॒र्द्ध्वो र्द्ध्वा ऽसीति॑ । \newline
43. अ॒सीती त्य॑स्य॒सी त्य॒मू म॒मू मित्य॑स्य॒सी त्य॒मूम् । \newline
44. इत्य॒मू म॒मू मितीत्य॒मू म॑जयन् नजयन् न॒मू मितीत्य॒मू म॑जयन्न् । \newline
45. अ॒मू म॑जयन् नजयन् न॒मू म॒मू म॑जयन् नन्तरिक्ष॒स द॑न्तरिक्ष॒स द॑जयन् न॒मू म॒मू म॑जयन् नन्तरिक्ष॒सत् । \newline
46. अ॒ज॒य॒न् न॒न्त॒रि॒क्ष॒स द॑न्तरिक्ष॒स द॑जयन् नजयन् नन्तरिक्ष॒स द॑स्यस्य न्तरिक्ष॒स द॑जयन् नजयन् नन्तरिक्ष॒स द॑सि । \newline
47. अ॒न्त॒रि॒क्ष॒स द॑स्य स्यन्तरिक्ष॒स द॑न्तरिक्ष॒स द॑स्य॒न्तरि॑क्षे॒ ऽन्तरि॑क्षे ऽस्यन्तरिक्ष॒ सद॑न्तरिक्ष॒स द॑स्य॒न्तरि॑क्षे । \newline
48. अ॒न्त॒रि॒क्ष॒सदित्य॑न्तरिक्ष - सत् । \newline
49. अ॒स्य॒न्तरि॑क्षे॒ ऽन्तरि॑क्षे ऽस्यस्य॒न्तरि॑क्षे सीद सीदा॒न्तरि॑क्षे ऽस्यस्य॒न्तरि॑क्षे सीद । \newline
50. अ॒न्तरि॑क्षे सीद सीदा॒ न्तरि॑क्षे॒ ऽन्तरि॑क्षे सी॒दे तीति॑ सीदा॒ न्तरि॑क्षे॒ ऽन्तरि॑क्षे सी॒देति॑ । \newline
51. सी॒देतीति॑ सीद सी॒दे त्य॒न्तरि॑क्ष म॒न्तरि॑क्ष॒ मिति॑ सीद सी॒दे त्य॒न्तरि॑क्षम् । \newline
52. इत्य॒न्तरि॑क्ष म॒न्तरि॑क्ष॒ मिती त्य॒न्तरि॑क्ष मजयन् नजयन् न॒न्तरि॑क्ष॒ मिती त्य॒न्तरि॑क्ष मजयन्न् । \newline
53. अ॒न्तरि॑क्ष मजयन् नजयन् न॒न्तरि॑क्ष म॒न्तरि॑क्ष मजय॒न् तत॒ स्ततो॑ ऽजयन् न॒न्तरि॑क्ष म॒न्तरि॑क्ष मजय॒न् ततः॑ । \newline
54. अ॒ज॒य॒न् तत॒ स्ततो॑ ऽजयन् नजय॒न् ततो॑ दे॒वा दे॒वा स्ततो॑ ऽजयन् नजय॒न् ततो॑ दे॒वाः । \newline
55. ततो॑ दे॒वा दे॒वा स्तत॒ स्ततो॑ दे॒वा अभ॑व॒न् नभ॑वन् दे॒वा स्तत॒ स्ततो॑ दे॒वा अभ॑वन्न् । \newline
56. दे॒वा अभ॑व॒न् नभ॑वन् दे॒वा दे॒वा अभ॑व॒न् परा॒ परा ऽभ॑वन् दे॒वा दे॒वा अभ॑व॒न् परा᳚ । \newline
57. अभ॑व॒न् परा॒ परा ऽभ॑व॒न् नभ॑व॒न् परा ऽसु॑रा॒ असु॑राः॒ परा ऽभ॑व॒न् नभ॑व॒न् परा ऽसु॑राः । \newline
\pagebreak
\markright{ TS 5.3.11.2  \hfill https://www.vedavms.in \hfill}

\section{ TS 5.3.11.2 }

\textbf{TS 5.3.11.2 } \newline
\textbf{Samhita Paata} \newline

पराऽसु॑रा॒ यस्यै॒ता उ॑पधी॒यन्ते॒ भूया॑ने॒व भ॑वत्य॒भीमान् ॅलो॒कान् ज॑यति॒ भव॑त्या॒त्मना॒ परा᳚ऽस्य॒ भ्रातृ॑व्यो भवत्यफ्सु॒षद॑सि श्येन॒सद॒सीत्या॑है॒तद्वा अ॒ग्ने रू॒पꣳ रू॒पेणै॒वाग्निमव॑ रुन्धे पृथि॒व्यास्त्वा॒ द्रवि॑णे सादया॒मी-त्या॑हे॒माने॒वैताभि॑-र्लो॒कान् द्रवि॑णावतः कुरुत आयु॒ष्या॑ उप॑ दधा॒त्यायु॑रे॒वा - [  ] \newline

\textbf{Pada Paata} \newline

परेति॑ । असु॑राः । यस्य॑ । ए॒ताः । उ॒प॒धी॒यन्त॒ इत्यु॑प - धी॒यन्ते᳚ । भूयान्॑ । ए॒व । भ॒व॒ति॒ । अ॒भीति॑ । इ॒मान् । लो॒कान् । ज॒य॒ति॒ । भव॑ति । आ॒त्मना᳚ । परेति॑ । अ॒स्य॒ । भ्रातृ॑व्यः । भ॒व॒ति॒ । अ॒फ्सु॒षदित्य॑फ्सु - सत् । अ॒सि॒ । श्ये॒न॒सदिति॑ श्येन - सत् । अ॒सि॒ । इति॑ । आ॒ह॒ । ए॒तत् । वै । अ॒ग्नेः । रू॒पम् । रू॒पेण॑ । ए॒व । अ॒ग्निम् । अवेति॑ । रु॒न्धे॒ । पृ॒थि॒व्याः । त्वा॒ । द्रवि॑णे । सा॒द॒या॒मि॒ । इति॑ । आ॒ह॒ । इ॒मान् । ए॒व । ए॒ताभिः॑ । लो॒कान् । द्रवि॑णावत॒ इति॒ द्रवि॑ण - व॒तः॒ । कु॒रु॒ते॒ । आ॒यु॒ष्याः᳚ । उपेति॑ । द॒धा॒ति॒ । आयुः॑ । ए॒व ।  \newline


\textbf{Krama Paata} \newline

पराऽसु॑राः । असु॑रा॒ यस्य॑ । यस्यै॒ताः । ए॒ता उ॑पधी॒यन्ते᳚ । उ॒प॒धी॒यन्ते॒ भूयान्॑ । उ॒प॒धी॒यन्त॒ इत्यु॑प - धी॒यन्ते᳚ । भूया॑ने॒व । ए॒व भ॑वति । भ॒व॒त्य॒भि । अ॒भीमान् । इ॒मान् ॅलो॒कान् । लो॒कान् ज॑यति । ज॒य॒ति॒ भव॑ति । भव॑त्या॒त्मना᳚ । आ॒त्मना॒ परा᳚ । परा᳚ऽस्य । अ॒स्य॒ भ्रातृ॑व्यः । भ्रातृ॑व्यो भवति । भ॒व॒त्य॒फ्सु॒षत् । अ॒फ्सु॒षद॑सि । अ॒फ्सु॒षदित्य॑फ्सु - सत् । अ॒सि॒ श्ये॒न॒सत् । श्ये॒न॒सद॑सि । श्ये॒न॒सदिति॑ श्येन - सत् । अ॒सीति॑ । इत्या॑ह । आ॒है॒तत् । ए॒तद् वै । वा अ॒ग्नेः । अ॒ग्ने रू॒पम् । रू॒पꣳ रू॒पेण॑ । रू॒पेणै॒व । ए॒वाग्निम् । अ॒ग्निमव॑ । अव॑ रुन्धे । रु॒न्धे॒ पृ॒थि॒व्याः । पृ॒थि॒व्यास्त्वा᳚ । त्वा॒ द्रवि॑णे । द्रवि॑णे सादयामि । सा॒द॒या॒मीति॑ । इत्या॑ह । आ॒हे॒मान् । इ॒माने॒व । ए॒वैताभिः॑ । ए॒ताभि॑र् लो॒कान् । लो॒कान् द्रवि॑णावतः । द्रवि॑णावतः कुरुते । द्रवि॑णावत॒ इति॒ द्रवि॑ण - व॒तः॒ । कु॒रु॒त॒ आ॒यु॒ष्याः᳚ । आ॒यु॒ष्या॑ उप॑ । उप॑ दधाति । द॒धा॒त्यायुः॑ । आयु॑रे॒व । ए॒वास्मिन्न्॑ \newline

\textbf{Jatai Paata} \newline

1. परा ऽसु॑रा॒ असु॑राः॒ परा॒ परा ऽसु॑राः । \newline
2. असु॑रा॒ यस्य॒ यस्यासु॑रा॒ असु॑रा॒ यस्य॑ । \newline
3. यस्यै॒ता ए॒ता यस्य॒ यस्यै॒ताः । \newline
4. ए॒ता उ॑पधी॒यन्त॑ उपधी॒यन्त॑ ए॒ता ए॒ता उ॑पधी॒यन्ते᳚ । \newline
5. उ॒प॒धी॒यन्ते॒ भूया॒न् भूया॑ नुपधी॒यन्त॑ उपधी॒यन्ते॒ भूयान्॑ । \newline
6. उ॒प॒धी॒यन्त॒ इत्यु॑प - धी॒यन्ते᳚ । \newline
7. भूया॑ ने॒वैव भूया॒न् भूया॑ ने॒व । \newline
8. ए॒व भ॑वति भव त्ये॒वैव भ॑वति । \newline
9. भ॒व॒ त्य॒भ्य॑भि भ॑वति भव त्य॒भि । \newline
10. अ॒भीमा नि॒मा न॒भ्य॑ भीमान् । \newline
11. इ॒मान् ॅलो॒कान् ॅलो॒का नि॒मा नि॒मान् ॅलो॒कान् । \newline
12. लो॒कान् ज॑यति जयति लो॒कान् ॅलो॒कान् ज॑यति । \newline
13. ज॒य॒ति॒ भव॑ति॒ भव॑ति जयति जयति॒ भव॑ति । \newline
14. भव॑ त्या॒त्मना॒ ऽऽत्मना॒ भव॑ति॒ भव॑ त्या॒त्मना᳚ । \newline
15. आ॒त्मना॒ परा॒ परा॒ ऽऽत्मना॒ ऽऽत्मना॒ परा᳚ । \newline
16. परा᳚ ऽस्यास्य॒ परा॒ परा᳚ ऽस्य । \newline
17. अ॒स्य॒ भ्रातृ॑व्यो॒ भ्रातृ॑व्यो ऽस्यास्य॒ भ्रातृ॑व्यः । \newline
18. भ्रातृ॑व्यो भवति भवति॒ भ्रातृ॑व्यो॒ भ्रातृ॑व्यो भवति । \newline
19. भ॒व॒त्य॒ फ्सु॒ष द॑फ्सु॒षद् भ॑वति भवत्य फ्सु॒षत् । \newline
20. अ॒फ्सु॒ष द॑स्यस्य फ्सु॒ष द॑फ्सु॒ष द॑सि । \newline
21. अ॒फ्सु॒षदित्य॑फ्सु - सत् । \newline
22. अ॒सि॒ श्ये॒न॒स च्छ्ये॑न॒स द॑स्यसि श्येन॒सत् । \newline
23. श्ये॒न॒स द॑स्यसि श्येन॒स च्छ्ये॑न॒स द॑सि । \newline
24. श्ये॒न॒सदिति॑ श्येन - सत् । \newline
25. अ॒सीती त्य॑स्य॒ सीति॑ । \newline
26. इत्या॑हा॒हे तीत्या॑ह । \newline
27. आ॒है॒त दे॒त दा॑हा है॒तत् । \newline
28. ए॒तद् वै वा ए॒त दे॒तद् वै । \newline
29. वा अ॒ग्ने र॒ग्नेर् वै वा अ॒ग्नेः । \newline
30. अ॒ग्ने रू॒पꣳ रू॒प म॒ग्ने र॒ग्ने रू॒पम् । \newline
31. रू॒पꣳ रू॒पेण॑ रू॒पेण॑ रू॒पꣳ रू॒पꣳ रू॒पेण॑ । \newline
32. रू॒पे णै॒वैव रू॒पेण॑ रू॒पेणै॒व । \newline
33. ए॒वाग्नि म॒ग्नि मे॒वैवाग्निम् । \newline
34. अ॒ग्नि मवा वा॒ग्नि म॒ग्नि मव॑ । \newline
35. अव॑ रुन्धे रु॒न्धे ऽवाव॑ रुन्धे । \newline
36. रु॒न्धे॒ पृ॒थि॒व्याः पृ॑थि॒व्या रु॑न्धे रुन्धे पृथि॒व्याः । \newline
37. पृ॒थि॒व्या स्त्वा᳚ त्वा पृथि॒व्याः पृ॑थि॒व्या स्त्वा᳚ । \newline
38. त्वा॒ द्रवि॑णे॒ द्रवि॑णे त्वा त्वा॒ द्रवि॑णे । \newline
39. द्रवि॑णे सादयामि सादयामि॒ द्रवि॑णे॒ द्रवि॑णे सादयामि । \newline
40. सा॒द॒या॒ मीतीति॑ सादयामि सादया॒ मीति॑ । \newline
41. इत्या॑ हा॒हे तीत्या॑ह । \newline
42. आ॒हे॒ मा नि॒मा ना॑हाहे॒मान् । \newline
43. इ॒मा ने॒वैवेमा नि॒मा ने॒व । \newline
44. ए॒वै ताभि॑ रे॒ताभि॑ रे॒वैवैताभिः॑ । \newline
45. ए॒ताभि॑र् लो॒कान् ॅलो॒का ने॒ताभि॑ रे॒ताभि॑र् लो॒कान् । \newline
46. लो॒कान् द्रवि॑णावतो॒ द्रवि॑णावतो लो॒कान् ॅलो॒कान् द्रवि॑णावतः । \newline
47. द्रवि॑णावतः कुरुते कुरुते॒ द्रवि॑णावतो॒ द्रवि॑णावतः कुरुते । \newline
48. द्रवि॑णावत॒ इति॒ द्रवि॑ण - व॒तः॒ । \newline
49. कु॒रु॒त॒ आ॒यु॒ष्या॑ आयु॒ष्याः᳚ कुरुते कुरुत आयु॒ष्याः᳚ । \newline
50. आ॒यु॒ष्या॑ उपोपा॑ यु॒ष्या॑ आयु॒ष्या॑ उप॑ । \newline
51. उप॑ दधाति दधा॒ त्युपोप॑ दधाति । \newline
52. द॒धा॒ त्यायु॒ रायु॑र् दधाति दधा॒ त्यायुः॑ । \newline
53. आयु॑ रे॒वै वायु॒ रायु॑ रे॒व । \newline
54. ए॒वास्मि॑न् नस्मिन् ने॒वै वास्मिन्न्॑ । \newline

\textbf{Ghana Paata } \newline

1. परा ऽसु॑रा॒ असु॑राः॒ परा॒ परा ऽसु॑रा॒ यस्य॒ यस्या सु॑राः॒ परा॒ परा ऽसु॑रा॒ यस्य॑ । \newline
2. असु॑रा॒ यस्य॒ यस्या सु॑रा॒ असु॑रा॒ यस्यै॒ता ए॒ता यस्या सु॑रा॒ असु॑रा॒ यस्यै॒ताः । \newline
3. यस्यै॒ता ए॒ता यस्य॒ यस्यै॒ता उ॑पधी॒यन्त॑ उपधी॒यन्त॑ ए॒ता यस्य॒ यस्यै॒ता उ॑पधी॒यन्ते᳚ । \newline
4. ए॒ता उ॑पधी॒यन्त॑ उपधी॒यन्त॑ ए॒ता ए॒ता उ॑पधी॒यन्ते॒ भूया॒न् भूया॑ नुपधी॒यन्त॑ ए॒ता ए॒ता उ॑पधी॒यन्ते॒ भूयान्॑ । \newline
5. उ॒प॒धी॒यन्ते॒ भूया॒न् भूया॑ नुपधी॒यन्त॑ उपधी॒यन्ते॒ भूया॑ ने॒वैव भूया॑ नुपधी॒यन्त॑ उपधी॒यन्ते॒ भूया॑ ने॒व । \newline
6. उ॒प॒धी॒यन्त॒ इत्यु॑प - धी॒यन्ते᳚ । \newline
7. भूया॑ ने॒वैव भूया॒न् भूया॑ ने॒व भ॑वति भव त्ये॒व भूया॒न् भूया॑ ने॒व भ॑वति । \newline
8. ए॒व भ॑वति भव त्ये॒वैव भ॑व त्य॒भ्य॑भि भ॑व त्ये॒वैव भ॑व त्य॒भि । \newline
9. भ॒व॒ त्य॒भ्य॑भि भ॑वति भव त्य॒भीमा नि॒मा न॒भि भ॑वति भव त्य॒भीमान् । \newline
10. अ॒भीमा नि॒मा न॒भ्य॑भीमान् ॅलो॒कान् ॅलो॒का नि॒मा न॒भ्य॑भीमान् ॅलो॒कान् । \newline
11. इ॒मान् ॅलो॒कान् ॅलो॒का नि॒मा नि॒मान् ॅलो॒कान् ज॑यति जयति लो॒का नि॒मा नि॒मान् ॅलो॒कान् ज॑यति । \newline
12. लो॒कान् ज॑यति जयति लो॒कान् ॅलो॒कान् ज॑यति॒ भव॑ति॒ भव॑ति जयति लो॒कान् ॅलो॒कान् ज॑यति॒ भव॑ति । \newline
13. ज॒य॒ति॒ भव॑ति॒ भव॑ति जयति जयति॒ भव॑ त्या॒त्मना॒ ऽऽत्मना॒ भव॑ति जयति जयति॒ भव॑ त्या॒त्मना᳚ । \newline
14. भव॑ त्या॒त्मना॒ ऽऽत्मना॒ भव॑ति॒ भव॑ त्या॒त्मना॒ परा॒ परा॒ ऽऽत्मना॒ भव॑ति॒ भव॑ त्या॒त्मना॒ परा᳚ । \newline
15. आ॒त्मना॒ परा॒ परा॒ ऽऽत्मना॒ ऽऽत्मना॒ परा᳚ ऽस्यास्य॒ परा॒ ऽऽत्मना॒ ऽऽत्मना॒ परा᳚ ऽस्य । \newline
16. परा᳚ ऽस्यास्य॒ परा॒ परा᳚ ऽस्य॒ भ्रातृ॑व्यो॒ भ्रातृ॑व्यो ऽस्य॒ परा॒ परा᳚ ऽस्य॒ भ्रातृ॑व्यः । \newline
17. अ॒स्य॒ भ्रातृ॑व्यो॒ भ्रातृ॑व्यो ऽस्यास्य॒ भ्रातृ॑व्यो भवति भवति॒ भ्रातृ॑व्यो ऽस्यास्य॒ भ्रातृ॑व्यो भवति । \newline
18. भ्रातृ॑व्यो भवति भवति॒ भ्रातृ॑व्यो॒ भ्रातृ॑व्यो भव त्यफ्सु॒ष द॑फ्सु॒षद् भ॑वति॒ भ्रातृ॑व्यो॒ भ्रातृ॑व्यो भव त्यफ्सु॒षत् । \newline
19. भ॒व॒ त्य॒फ्सु॒ष द॑फ्सु॒षद् भ॑वति भव त्यफ्सु॒ष द॑स्यस्य फ्सु॒षद् भ॑वति भव त्यफ्सु॒ष द॑सि । \newline
20. अ॒फ्सु॒ष द॑स्य स्यफ्सु॒ष द॑फ्सु॒षद॑सि श्येन॒स च्छ्ये॑न॒स द॑स्यफ्सु॒ष द॑फ्सु॒ष द॑सि श्येन॒सत् । \newline
21. अ॒फ्सु॒षदित्य॑फ्सु - सत् । \newline
22. अ॒सि॒ श्ये॒न॒स च्छ्ये॑न॒स द॑स्यसि श्येन॒स द॑स्यसि श्येन॒स द॑स्यसि श्येन॒स द॑सि । \newline
23. श्ये॒न॒स द॑स्यसि श्येन॒स च्छ्ये॑न॒स द॒सीती त्य॑सि श्येन॒स च्छ्ये॑न॒स द॒सीति॑ । \newline
24. श्ये॒न॒सदिति॑ श्येन - सत् । \newline
25. अ॒सीती त्य॑स्य॒सी त्या॑हा॒हे त्य॑स्य॒सी त्या॑ह । \newline
26. इत्या॑ हा॒हेती त्या॑है॒त दे॒त दा॒हेती त्या॑है॒तत् । \newline
27. आ॒है॒त दे॒त दा॑हा है॒तद् वै वा ए॒त दा॑हा है॒तद् वै । \newline
28. ए॒तद् वै वा ए॒त दे॒तद् वा अ॒ग्ने र॒ग्नेर् वा ए॒त दे॒तद् वा अ॒ग्नेः । \newline
29. वा अ॒ग्ने र॒ग्नेर् वै वा अ॒ग्ने रू॒पꣳ रू॒प म॒ग्नेर् वै वा अ॒ग्ने रू॒पम् । \newline
30. अ॒ग्ने रू॒पꣳ रू॒प म॒ग्ने र॒ग्ने रू॒पꣳ रू॒पेण॑ रू॒पेण॑ रू॒प म॒ग्ने र॒ग्ने रू॒पꣳ रू॒पेण॑ । \newline
31. रू॒पꣳ रू॒पेण॑ रू॒पेण॑ रू॒पꣳ रू॒पꣳ रू॒पेणै॒वैव रू॒पेण॑ रू॒पꣳ रू॒पꣳ रू॒पेणै॒व । \newline
32. रू॒पेणै॒वैव रू॒पेण॑ रू॒पे णै॒वाग्नि म॒ग्नि मे॒व रू॒पेण॑ रू॒पे णै॒वाग्निम् । \newline
33. ए॒वाग्नि म॒ग्नि मे॒वै वाग्नि मवा वा॒ग्नि मे॒वै वाग्नि मव॑ । \newline
34. अ॒ग्नि मवा वा॒ग्नि म॒ग्नि मव॑ रुन्धे रु॒न्धे ऽवा॒ग्नि म॒ग्नि मव॑ रुन्धे । \newline
35. अव॑ रुन्धे रु॒न्धे ऽवाव॑ रुन्धे पृथि॒व्याः पृ॑थि॒व्या रु॒न्धे ऽवाव॑ रुन्धे पृथि॒व्याः । \newline
36. रु॒न्धे॒ पृ॒थि॒व्याः पृ॑थि॒व्या रु॑न्धे रुन्धे पृथि॒व्या स्त्वा᳚ त्वा पृथि॒व्या रु॑न्धे रुन्धे पृथि॒व्या स्त्वा᳚ । \newline
37. पृ॒थि॒व्या स्त्वा᳚ त्वा पृथि॒व्याः पृ॑थि॒व्या स्त्वा॒ द्रवि॑णे॒ द्रवि॑णे त्वा पृथि॒व्याः पृ॑थि॒व्या स्त्वा॒ द्रवि॑णे । \newline
38. त्वा॒ द्रवि॑णे॒ द्रवि॑णे त्वा त्वा॒ द्रवि॑णे सादयामि सादयामि॒ द्रवि॑णे त्वा त्वा॒ द्रवि॑णे सादयामि । \newline
39. द्रवि॑णे सादयामि सादयामि॒ द्रवि॑णे॒ द्रवि॑णे सादया॒ मीतीति॑ सादयामि॒ द्रवि॑णे॒ द्रवि॑णे सादया॒ मीति॑ । \newline
40. सा॒द॒या॒ मीतीति॑ सादयामि सादया॒ मीत्या॑ हा॒हेति॑ सादयामि सादया॒मी त्या॑ह । \newline
41. इत्या॑हा॒हेती त्या॑हे॒मा नि॒मा ना॒हेती त्या॑हे॒ मान् । \newline
42. आ॒हे॒ मा नि॒मा ना॑हाहे॒मा ने॒वैवेमा ना॑हाहे॒मा ने॒व । \newline
43. इ॒मा ने॒वैवेमा नि॒मा ने॒वैताभि॑ रे॒ताभि॑ रे॒वेमा नि॒मा ने॒वैताभिः॑ । \newline
44. ए॒वैताभि॑ रे॒ताभि॑ रे॒वैवै ताभि॑र् लो॒कान् ॅलो॒का ने॒ताभि॑ रे॒वैवै ताभि॑र् लो॒कान् । \newline
45. ए॒ताभि॑र् लो॒कान् ॅलो॒का ने॒ताभि॑ रे॒ताभि॑र् लो॒कान् द्रवि॑णावतो॒ द्रवि॑णावतो लो॒का ने॒ताभि॑ रे॒ताभि॑र् लो॒कान् द्रवि॑णावतः । \newline
46. लो॒कान् द्रवि॑णावतो॒ द्रवि॑णावतो लो॒कान् ॅलो॒कान् द्रवि॑णावतः कुरुते कुरुते॒ द्रवि॑णावतो लो॒कान् ॅलो॒कान् द्रवि॑णावतः कुरुते । \newline
47. द्रवि॑णावतः कुरुते कुरुते॒ द्रवि॑णावतो॒ द्रवि॑णावतः कुरुत आयु॒ष्या॑ आयु॒ष्याः᳚ कुरुते॒ द्रवि॑णावतो॒ द्रवि॑णावतः कुरुत आयु॒ष्याः᳚ । \newline
48. द्रवि॑णावत॒ इति॒ द्रवि॑ण - व॒तः॒ । \newline
49. कु॒रु॒त॒ आ॒यु॒ष्या॑ आयु॒ष्याः᳚ कुरुते कुरुत आयु॒ष्या॑ उपोपा॑यु॒ष्याः᳚ कुरुते कुरुत आयु॒ष्या॑ उप॑ । \newline
50. आ॒यु॒ष्या॑ उपोपा॑यु॒ष्या॑ आयु॒ष्या॑ उप॑ दधाति दधा॒ त्युपा॑यु॒ष्या॑ आयु॒ष्या॑ उप॑ दधाति । \newline
51. उप॑ दधाति दधा॒ त्युपोप॑ दधा॒ त्यायु॒ रायु॑र् दधा॒ त्युपोप॑ दधा॒ त्यायुः॑ । \newline
52. द॒धा॒ त्यायु॒ रायु॑र् दधाति दधा॒ त्यायु॑ रे॒वै वायु॑र् दधाति दधा॒ त्यायु॑ रे॒व । \newline
53. आयु॑ रे॒वैवायु॒ रायु॑ रे॒वास्मि॑न् नस्मिन् ने॒वायु॒ रायु॑ रे॒वास्मिन्न्॑ । \newline
54. ए॒वास्मि॑न् नस्मिन् ने॒वै वास्मि॑न् दधाति दधा त्यस्मिन् ने॒वै वास्मि॑न् दधाति । \newline
\pagebreak
\markright{ TS 5.3.11.3  \hfill https://www.vedavms.in \hfill}

\section{ TS 5.3.11.3 }

\textbf{TS 5.3.11.3 } \newline
\textbf{Samhita Paata} \newline

-स्मि॑न् दधा॒त्यग्ने॒ यत्ते॒ परꣳ॒॒ हृन्नामेत्या॑है॒तद्वा अ॒ग्नेः प्रि॒यं धाम॑ प्रि॒यमे॒वास्य॒ धामोपा᳚ऽऽ*प्नोति॒ तावेहि॒ सꣳ र॑भावहा॒ इत्या॑ह॒ व्ये॑वैने॑न॒ परि॑ धत्ते॒ पाञ्च॑जन्ये॒ष्वप्ये᳚द्ध्यग्न॒ इत्या॑है॒ष वा अ॒ग्निः पाञ्च॑जन्यो॒ यः पञ्च॑चितीक॒-स्तस्मा॑दे॒वमा॑हर्त॒व्या॑ उप॑ ( ) दधात्ये॒तद्वा ऋ॑तू॒नां प्रि॒यं धाम॒ यदृ॑त॒व्या॑ ऋतू॒नामे॒व प्रि॒यं धामाव॑ रुन्धे सु॒मेक॒ इत्या॑ह संॅवथ्स॒रो वै सु॒मेकः॑ संॅवथ्स॒रस्यै॒व प्रि॒यं धामोपा᳚ऽऽ*प्नोति ॥ \newline

\textbf{Pada Paata} \newline

अ॒स्मि॒न्न् । द॒धा॒ति॒ । अग्ने᳚ । यत् । ते॒ । पर᳚म् । हृत् । नाम॑ । इति॑ । आ॒ह॒ । ए॒तत् । वै । अ॒ग्नेः । प्रि॒यम् । धाम॑ । प्रि॒यम् । ए॒व । अ॒स्य॒ । धाम॑ । उपेति॑ । आ॒प्नो॒ति॒ । तौ । एति॑ । इ॒हि॒ । समिति॑ । र॒भा॒व॒है॒ । इति॑ । आ॒ह॒ । वीति॑ । ए॒व । ए॒ने॒न॒ । परीति॑ । ध॒त्ते॒ । पाञ्च॑जन्ये॒ष्विति॒ पाञ्च॑ - ज॒न्ये॒षु॒ । अपीति॑ । ए॒धि॒ । अ॒ग्ने॒ । इति॑ । आ॒ह॒ । ए॒षः । वै । अ॒ग्निः । पाञ्च॑जन्य॒ इति॒ पाञ्च॑ - ज॒न्यः॒ । यः । पञ्च॑चितीक॒ इति॒ पञ्च॑-चि॒ती॒कः॒ । तस्मा᳚त् । ए॒वम् । आ॒ह॒ । ऋ॒त॒व्याः᳚ । उपेति॑ ( ) । द॒धा॒ति॒ । ए॒तत् । वै । ऋ॒तू॒नाम् । प्रि॒यम् । धाम॑ । यत् । ऋ॒त॒व्याः᳚ । ऋ॒तू॒नाम् । ए॒व । प्रि॒यम् । धाम॑ । अवेति॑ । रु॒न्धे॒ । सु॒मेक॒ इति॑ सु - मेकः॑ । इति॑ । आ॒ह॒ । सं॒ॅव॒थ्स॒र इति॑ सं - व॒थ्स॒रः । वै । सु॒मेक॒ इति॑ सु - मेकः॑ । सं॒ॅव॒थ्स॒रस्येति॑ सं - व॒थ्स॒रस्य॑ । ए॒व । प्रि॒यम् । धाम॑ । उपेति॑ । आ॒प्नो॒ति॒ ॥  \newline


\textbf{Krama Paata} \newline

अ॒स्मि॒न् द॒धा॒ति॒ । द॒धा॒त्यग्ने᳚ । अग्ने॒ यत् । यत् ते᳚ । ते॒ पर᳚म् । परꣳ॒॒ हृत् । हृन् नाम॑ । नामेति॑ । इत्या॑ह । आ॒है॒तत् । ए॒तद् वै । वा अ॒ग्नेः । अ॒ग्नेः प्रि॒यम् । प्रि॒यम् धाम॑ । धाम॑ प्रि॒यम् । प्रि॒यमे॒व । ए॒वास्य॑ । अ॒स्य॒ धाम॑ । धामोप॑ । उपा᳚प्नोति । आ॒प्नो॒ति॒ तौ । तावा । एहि॑ । इ॒हि॒ सम् । सꣳ र॑भावहै । र॒भा॒व॒हा॒ इति॑ । इत्या॑ह । आ॒ह॒ वि । व्ये॑व । ए॒वैने॑न । ए॒ने॒न॒ परि॑ । परि॑ धत्ते । ध॒त्ते॒ पाञ्च॑जन्येषु । पाञ्च॑जन्ये॒ष्वपि॑ । पाञ्च॑जन्ये॒ष्विति॒ पाञ्च॑ - ज॒न्ये॒षु॒ । अप्ये॑धि । ए॒द्ध्य॒ग्ने॒ । अ॒ग्न॒ इति॑ । इत्या॑ह । आ॒है॒षः । ए॒ष वै । वा अ॒ग्निः । अ॒ग्निः पाञ्च॑जन्यः । पाञ्च॑न्यो॒ यः । पाञ्च॑जन्य॒ इति॒ पाञ्च॑ - ज॒न्यः॒ । यः पञ्च॑चितीकः । पञ्च॑चितीक॒स्तस्मा᳚त् । पञ्च॑चितीक॒ इति॒ पञ्च॑ - चि॒ती॒कः॒ । तस्मा॑दे॒वम् । ए॒वमा॑ह । आ॒ह॒र्त॒व्याः᳚ । ऋ॒त॒व्या॑ उप॑ ( ) । उप॑ दधाति । द॒धा॒त्ये॒तत् । ए॒तद् वै । वा ऋ॑तू॒नाम् । ऋ॒तू॒नाम् प्रि॒यम् । प्रि॒यम् धाम॑ । धाम॒ यत् । यदृ॑त॒व्याः᳚ । ऋ॒त॒व्या॑ ऋतू॒नाम् । ऋ॒तू॒नामे॒व । ए॒व प्रि॒यम् । प्रि॒यम् धाम॑ । धामाव॑ । अव॑ रुन्धे । रु॒न्धे॒ सु॒मेकः॑ । सु॒मेक॒ इति॑ । सु॒मेक॒ इति॑ सु - मेकः॑ । इत्या॑ह । आ॒ह॒ स॒म्ॅव॒थ्स॒रः । स॒म्ॅव॒थ्स॒रो वै । स॒म्ॅव॒थ्स॒र इति॑ सम् - व॒थ्स॒रः । वै सु॒मेकः॑ । सु॒मेकः॑ सम्ॅवथ्स॒रस्य॑ । सु॒मेक॒ इति॑ सु - मेकः॑ । स॒म्ॅव॒थ्स॒रस्यै॒व । स॒म्ॅव॒थ्स॒रस्येति॑ सम् - व॒थ्स॒रस्य॑ । ए॒व प्रि॒यम् । प्रि॒यम् धाम॑ । धामोप॑ । उपा᳚प्नोति । आ॒प्नो॒तीत्या᳚प्नोति । \newline

\textbf{Jatai Paata} \newline

1. अ॒स्मि॒न् द॒धा॒ति॒ द॒धा॒ त्य॒स्मि॒न् न॒स्मि॒न् द॒धा॒ति॒ । \newline
2. द॒धा॒ त्यग्ने ऽग्ने॑ दधाति दधा॒ त्यग्ने᳚ । \newline
3. अग्ने॒ यद् यदग्ने ऽग्ने॒ यत् । \newline
4. यत् ते॑ ते॒ यद् यत् ते᳚ । \newline
5. ते॒ पर॒म् पर॑म् ते ते॒ पर᳚म् । \newline
6. परꣳ॒॒ हृ द्धृत् पर॒म् परꣳ॒॒ हृत् । \newline
7. हृन् नाम॒ नाम॒ हृ द्धृन् नाम॑ । \newline
8. नामे तीति॒ नाम॒ नामे ति॑ । \newline
9. इत्या॑ हा॒हे तीत्या॑ह । \newline
10. आ॒है॒त दे॒त दा॑हा है॒तत् । \newline
11. ए॒तद् वै वा ए॒त दे॒तद् वै । \newline
12. वा अ॒ग्ने र॒ग्नेर् वै वा अ॒ग्नेः । \newline
13. अ॒ग्नेः प्रि॒यम् प्रि॒य म॒ग्ने र॒ग्नेः प्रि॒यम् । \newline
14. प्रि॒यम् धाम॒ धाम॑ प्रि॒यम् प्रि॒यम् धाम॑ । \newline
15. धाम॑ प्रि॒यम् प्रि॒यम् धाम॒ धाम॑ प्रि॒यम् । \newline
16. प्रि॒य मे॒वैव प्रि॒यम् प्रि॒य मे॒व । \newline
17. ए॒वास्या᳚ स्यै॒वै वास्य॑ । \newline
18. अ॒स्य॒ धाम॒ धामा᳚ स्यास्य॒ धाम॑ । \newline
19. धामो पोप॒ धाम॒ धामोप॑ । \newline
20. उपा᳚प्नो त्याप्नो॒ त्युपोपा᳚ प्नोति । \newline
21. आ॒प्नो॒ति॒ तौ ता वा᳚प्नो त्याप्नोति॒ तौ । \newline
22. ता वा तौ ता वा । \newline
23. एही॒ह्येहि॑ । \newline
24. इ॒हि॒ सꣳ स मि॑हीहि॒ सम् । \newline
25. सꣳ र॑भावहै रभावहै॒ सꣳ सꣳ र॑भावहै । \newline
26. र॒भा॒व॒हा॒ इतीति॑ रभावहै रभावहा॒ इति॑ । \newline
27. इत्या॑ हा॒हे तीत्या॑ह । \newline
28. आ॒ह॒ वि व्या॑हाह॒ वि । \newline
29. व्ये॑वैव वि व्ये॑व । \newline
30. ए॒वैने॑ नैने नै॒वै वैने॑न । \newline
31. ए॒ने॒न॒ परि॒ पर्ये॑ने नैनेन॒ परि॑ । \newline
32. परि॑ धत्ते धत्ते॒ परि॒ परि॑ धत्ते । \newline
33. ध॒त्ते॒ पाञ्च॑जन्येषु॒ पाञ्च॑जन्येषु धत्ते धत्ते॒ पाञ्च॑जन्येषु । \newline
34. पाञ्च॑जन्ये॒ ष्वप्यपि॒ पाञ्च॑जन्येषु॒ पाञ्च॑जन्ये॒ ष्वपि॑ । \newline
35. पाञ्च॑जन्ये॒ष्विति॒ पाञ्च॑ - ज॒न्ये॒षु॒ । \newline
36. अप्ये᳚ध्ये॒ ध्यप्य प्ये॑धि । \newline
37. ए॒ध्य॒ग्ने॒ ऽग्न॒ ए॒ध्ये॒ ध्य॒ग्ने॒ । \newline
38. अ॒ग्न॒ इती त्य॑ग्ने ऽग्न॒ इति॑ । \newline
39. इत्या॑ हा॒हे तीत्या॑ह । \newline
40. आ॒है॒ष ए॒ष आ॑हा है॒षः । \newline
41. ए॒ष वै वा ए॒ष ए॒ष वै । \newline
42. वा अ॒ग्नि र॒ग्निर् वै वा अ॒ग्निः । \newline
43. अ॒ग्निः पाञ्च॑जन्यः॒ पाञ्च॑जन्यो॒ ऽग्निर॒ग्निः पाञ्च॑जन्यः । \newline
44. पाञ्च॑जन्यो॒ यो यः पाञ्च॑जन्यः॒ पाञ्च॑जन्यो॒ यः । \newline
45. पाञ्च॑जन्य॒ इति॒ पाञ्च॑ - ज॒न्यः॒ । \newline
46. यः पञ्च॑चितीकः॒ पञ्च॑चितीको॒ यो यः पञ्च॑चितीकः । \newline
47. पञ्च॑चितीक॒ स्तस्मा॒त् तस्मा॒त् पञ्च॑चितीकः॒ पञ्च॑चितीक॒ स्तस्मा᳚त् । \newline
48. पञ्च॑चितीक॒ इति॒ पञ्च॑ - चि॒ती॒कः॒ । \newline
49. तस्मा॑दे॒व मे॒वम् तस्मा॒त् तस्मा॑ दे॒वम् । \newline
50. ए॒व मा॑हा है॒व मे॒व मा॑ह । \newline
51. आ॒ह॒ र्‌त॒व्या॑ ऋत॒व्या॑ आहाह र्‌त॒व्याः᳚ । \newline
52. ऋ॒त॒व्या॑ उपोपा᳚ र्‌त॒व्या॑ ऋत॒व्या॑ उप॑ । \newline
53. उप॑ दधाति दधा॒ त्युपोप॑ दधाति । \newline
54. द॒धा॒ त्ये॒त दे॒तद् द॑धाति दधा त्ये॒तत् । \newline
55. ए॒तद् वै वा ए॒त दे॒तद् वै । \newline
56. वा ऋ॑तू॒ना मृ॑तू॒नां ॅवै वा ऋ॑तू॒नाम् । \newline
57. ऋ॒तू॒नाम् प्रि॒यम् प्रि॒य मृ॑तू॒ना मृ॑तू॒नाम् प्रि॒यम् । \newline
58. प्रि॒यम् धाम॒ धाम॑ प्रि॒यम् प्रि॒यम् धाम॑ । \newline
59. धाम॒ यद् यद् धाम॒ धाम॒ यत् । \newline
60. यदृ॑त॒व्या॑ ऋत॒व्या॑ यद् यदृ॑त॒व्याः᳚ । \newline
61. ऋ॒त॒व्या॑ ऋतू॒ना मृ॑तू॒ना मृ॑त॒व्या॑ ऋत॒व्या॑ ऋतू॒नाम् । \newline
62. ऋ॒तू॒ना मे॒वैव र्‌तू॒ना मृ॑तू॒ना मे॒व । \newline
63. ए॒व प्रि॒यम् प्रि॒य मे॒वैव प्रि॒यम् । \newline
64. प्रि॒यम् धाम॒ धाम॑ प्रि॒यम् प्रि॒यम् धाम॑ । \newline
65. धामावाव॒ धाम॒ धामाव॑ । \newline
66. अव॑ रुन्धे रु॒न्धे ऽवाव॑ रुन्धे । \newline
67. रु॒न्धे॒ सु॒मेकः॑ सु॒मेको॑ रुन्धे रुन्धे सु॒मेकः॑ । \newline
68. सु॒मेक॒ इतीति॑ सु॒मेकः॑ सु॒मेक॒ इति॑ । \newline
69. सु॒मेक॒ इति॑ सु - मेकः॑ । \newline
70. इत्या॑ हा॒हे तीत्या॑ह । \newline
71. आ॒ह॒ सं॒ॅव॒थ्स॒रः सं॑ॅवथ्स॒र आ॑हाह संॅवथ्स॒रः । \newline
72. सं॒ॅव॒थ्स॒रो वै वै सं॑ॅवथ्स॒रः सं॑ॅवथ्स॒रो वै । \newline
73. सं॒ॅव॒थ्स॒र इति॑ सं - व॒थ्स॒रः । \newline
74. वै सु॒मेकः॑ सु॒मेको॒ वै वै सु॒मेकः॑ । \newline
75. सु॒मेकः॑ संॅवथ्स॒रस्य॑ संॅवथ्स॒रस्य॑ सु॒मेकः॑ सु॒मेकः॑ संॅवथ्स॒रस्य॑ । \newline
76. सु॒मेक॒ इति॑ सु - मेकः॑ । \newline
77. सं॒ॅव॒थ्स॒र स्यै॒वैव सं॑ॅवथ्स॒रस्य॑ संॅवथ्स॒र स्यै॒व । \newline
78. सं॒ॅव॒थ्स॒रस्येति॑ सं - व॒थ्स॒रस्य॑ । \newline
79. ए॒व प्रि॒यम् प्रि॒य मे॒वैव प्रि॒यम् । \newline
80. प्रि॒यम् धाम॒ धाम॑ प्रि॒यम् प्रि॒यम् धाम॑ । \newline
81. धामो पोप॒ धाम॒ धामोप॑ । \newline
82. उपा᳚प्नो त्याप्नो॒ त्युपोपा᳚ प्नोति । \newline
83. आ॒प्नो॒तीत्या᳚प्नोति । \newline

\textbf{Ghana Paata } \newline

1. अ॒स्मि॒न् द॒धा॒ति॒ द॒धा॒ त्य॒स्मि॒न् न॒स्मि॒न् द॒धा॒ त्यग्ने ऽग्ने॑ दधा त्यस्मिन् नस्मिन् दधा॒ त्यग्ने᳚ । \newline
2. द॒धा॒ त्यग्ने ऽग्ने॑ दधाति दधा॒ त्यग्ने॒ यद् यदग्ने॑ दधाति दधा॒ त्यग्ने॒ यत् । \newline
3. अग्ने॒ यद् यदग्ने ऽग्ने॒ यत् ते॑ ते॒ यदग्ने ऽग्ने॒ यत् ते᳚ । \newline
4. यत् ते॑ ते॒ यद् यत् ते॒ पर॒म् पर॑म् ते॒ यद् यत् ते॒ पर᳚म् । \newline
5. ते॒ पर॒म् पर॑म् ते ते॒ परꣳ॒॒ हृ द्धृत् पर॑म् ते ते॒ परꣳ॒॒ हृत् । \newline
6. परꣳ॒॒ हृ द्धृत् पर॒म् परꣳ॒॒ हृन् नाम॒ नाम॒ हृत् पर॒म् परꣳ॒॒ हृन् नाम॑ । \newline
7. हृन् नाम॒ नाम॒ हृ द्धृन् नामे तीति॒ नाम॒ हृ द्धृन् नामे ति॑ । \newline
8. नामे तीति॒ नाम॒ नामे त्या॑हा॒हेति॒ नाम॒ नामे त्या॑ह । \newline
9. इत्या॑ हा॒हेती त्या॑है॒त दे॒त दा॒हेती त्या॑है॒तत् । \newline
10. आ॒है॒त दे॒त दा॑हाहै॒तद् वै वा ए॒त दा॑हा है॒तद् वै । \newline
11. ए॒तद् वै वा ए॒त दे॒तद् वा अ॒ग्ने र॒ग्नेर् वा ए॒त दे॒तद् वा अ॒ग्नेः । \newline
12. वा अ॒ग्ने र॒ग्नेर् वै वा अ॒ग्नेः प्रि॒यम् प्रि॒य म॒ग्नेर् वै वा अ॒ग्नेः प्रि॒यम् । \newline
13. अ॒ग्नेः प्रि॒यम् प्रि॒य म॒ग्ने र॒ग्नेः प्रि॒यम् धाम॒ धाम॑ प्रि॒य म॒ग्ने र॒ग्नेः प्रि॒यम् धाम॑ । \newline
14. प्रि॒यम् धाम॒ धाम॑ प्रि॒यम् प्रि॒यम् धाम॑ प्रि॒यम् प्रि॒यम् धाम॑ प्रि॒यम् प्रि॒यम् धाम॑ प्रि॒यम् । \newline
15. धाम॑ प्रि॒यम् प्रि॒यम् धाम॒ धाम॑ प्रि॒य मे॒वैव प्रि॒यम् धाम॒ धाम॑ प्रि॒य मे॒व । \newline
16. प्रि॒य मे॒वैव प्रि॒यम् प्रि॒य मे॒वास्या᳚ स्यै॒व प्रि॒यम् प्रि॒य मे॒वास्य॑ । \newline
17. ए॒वास्या᳚ स्यै॒वै वास्य॒ धाम॒ धामा᳚ स्यै॒वै वास्य॒ धाम॑ । \newline
18. अ॒स्य॒ धाम॒ धामा᳚ स्यास्य॒ धामोपोप॒ धामा᳚ स्यास्य॒ धामोप॑ । \newline
19. धामोपोप॒ धाम॒ धामोपा᳚प्नो त्याप्नो॒ त्युप॒ धाम॒ धामोपा᳚प्नोति । \newline
20. उपा᳚प्नो त्याप्नो॒ त्युपोपा᳚ प्नोति॒ तौ ता वा᳚प्नो॒ त्युपोपा᳚प्नोति॒ तौ । \newline
21. आ॒प्नो॒ति॒ तौ ता वा᳚प्नो त्याप्नोति॒ ता वा ता वा᳚प्नो त्याप्नोति॒ ता वा । \newline
22. ता वा तौ ता वेही॒ह्या तौ ता वेहि॑ । \newline
23. एही॒ह्येहि॒ सꣳ स मि॒ह्येहि॒ सम् । \newline
24. इ॒हि॒ सꣳ स मि॑हीहि॒ सꣳ र॑भावहै रभावहै॒ स मि॑हीहि॒ सꣳ र॑भावहै । \newline
25. सꣳ र॑भावहै रभावहै॒ सꣳ सꣳ र॑भावहा॒ इतीति॑ रभावहै॒ सꣳ सꣳ र॑भावहा॒ इति॑ । \newline
26. र॒भा॒व॒हा॒ इतीति॑ रभावहै रभावहा॒ इत्या॑हा॒हेति॑ रभावहै रभावहा॒ इत्या॑ह । \newline
27. इत्या॑ हा॒हेती त्या॑ह॒ वि व्या॑हेती त्या॑ह॒ वि । \newline
28. आ॒ह॒ वि व्या॑हाह॒ व्ये॑वैव व्या॑हाह॒ व्ये॑व । \newline
29. व्ये॑वैव वि व्ये॑वैने॑ नैने नै॒व वि व्ये॑वै ने॑न । \newline
30. ए॒वैने॑ नैनेनै॒वै वैने॑न॒ परि॒ पर्ये॑ने नै॒वै वैने॑न॒ परि॑ । \newline
31. ए॒ने॒न॒ परि॒ पर्ये॑ने नैनेन॒ परि॑ धत्ते धत्ते॒ पर्ये॑ने नैनेन॒ परि॑ धत्ते । \newline
32. परि॑ धत्ते धत्ते॒ परि॒ परि॑ धत्ते॒ पाञ्च॑जन्येषु॒ पाञ्च॑जन्येषु धत्ते॒ परि॒ परि॑ धत्ते॒ पाञ्च॑जन्येषु । \newline
33. ध॒त्ते॒ पाञ्च॑जन्येषु॒ पाञ्च॑जन्येषु धत्ते धत्ते॒ पाञ्च॑जन्ये॒ ष्वप्यपि॒ पाञ्च॑जन्येषु धत्ते धत्ते॒ पाञ्च॑जन्ये॒ ष्वपि॑ । \newline
34. पाञ्च॑जन्ये॒ ष्वप्यपि॒ पाञ्च॑जन्येषु॒ पाञ्च॑जन्ये॒ ष्वप्ये᳚ ध्ये॒ध्यपि॒ पाञ्च॑जन्येषु॒ पाञ्च॑जन्ये॒ ष्वप्ये॑धि । \newline
35. पाञ्च॑जन्ये॒ष्विति॒ पाञ्च॑ - ज॒न्ये॒षु॒ । \newline
36. अप्ये᳚ ध्ये॒ ध्यप्यप्ये᳚ ध्यग्ने ऽग्न ए॒ध्य प्यप्ये᳚ ध्यग्ने । \newline
37. ए॒ध्य॒ग्ने॒ ऽग्न॒ ए॒ध्ये॒ ध्य॒ग्न॒ इती त्य॑ग्न एध्ये ध्यग्न॒ इति॑ । \newline
38. अ॒ग्न॒ इती त्य॑ग्ने ऽग्न॒ इत्या॑हा॒हे त्य॑ग्ने ऽग्न॒ इत्या॑ह । \newline
39. इत्या॑हा॒हेती त्या॑है॒ष ए॒ष आ॒हेती त्या॑है॒षः । \newline
40. आ॒है॒ष ए॒ष आ॑हा है॒ष वै वा ए॒ष आ॑हा है॒ष वै । \newline
41. ए॒ष वै वा ए॒ष ए॒ष वा अ॒ग्नि र॒ग्निर् वा ए॒ष ए॒ष वा अ॒ग्निः । \newline
42. वा अ॒ग्नि र॒ग्निर् वै वा अ॒ग्निः पाञ्च॑जन्यः॒ पाञ्च॑जन्यो॒ ऽग्निर् वै वा अ॒ग्निः पाञ्च॑जन्यः । \newline
43. अ॒ग्निः पाञ्च॑जन्यः॒ पाञ्च॑जन्यो॒ ऽग्नि र॒ग्निः पाञ्च॑जन्यो॒ यो यः पाञ्च॑जन्यो॒ ऽग्नि र॒ग्निः पाञ्च॑जन्यो॒ यः । \newline
44. पाञ्च॑जन्यो॒ यो यः पाञ्च॑जन्यः॒ पाञ्च॑जन्यो॒ यः पञ्च॑चितीकः॒ पञ्च॑चितीको॒ यः पाञ्च॑जन्यः॒ पाञ्च॑जन्यो॒ यः पञ्च॑चितीकः । \newline
45. पाञ्च॑जन्य॒ इति॒ पाञ्च॑ - ज॒न्यः॒ । \newline
46. यः पञ्च॑चितीकः॒ पञ्च॑चितीको॒ यो यः पञ्च॑चितीक॒ स्तस्मा॒त् तस्मा॒त् पञ्च॑चितीको॒ यो यः पञ्च॑चितीक॒ स्तस्मा᳚त् । \newline
47. पञ्च॑चितीक॒ स्तस्मा॒त् तस्मा॒त् पञ्च॑चितीकः॒ पञ्च॑चितीक॒ स्तस्मा॑ दे॒व मे॒वम् तस्मा॒त् पञ्च॑चितीकः॒ पञ्च॑चितीक॒ स्तस्मा॑ दे॒वम् । \newline
48. पञ्च॑चितीक॒ इति॒ पञ्च॑ - चि॒ती॒कः॒ । \newline
49. तस्मा॑ दे॒व मे॒वम् तस्मा॒त् तस्मा॑ दे॒व मा॑हा है॒वम् तस्मा॒त् तस्मा॑ दे॒व मा॑ह । \newline
50. ए॒व मा॑हा है॒व मे॒व मा॑ह र्‌त॒व्या॑ ऋत॒व्या॑ आहै॒व मे॒व मा॑ह र्‌त॒व्याः᳚ । \newline
51. आ॒ह॒ र्‌त॒व्या॑ ऋत॒व्या॑ आहाह र्‌त॒व्या॑ उपोपा᳚ र्‌त॒व्या॑ आहाह र्‌त॒व्या॑ उप॑ । \newline
52. ऋ॒त॒व्या॑ उपोपा᳚ र्‌त॒व्या॑ ऋत॒व्या॑ उप॑ दधाति दधा॒ त्युपा᳚ र्‌त॒व्या॑ ऋत॒व्या॑ उप॑ दधाति । \newline
53. उप॑ दधाति दधा॒ त्युपोप॑ दधा त्ये॒त दे॒तद् द॑धा॒ त्युपोप॑ दधा त्ये॒तत् । \newline
54. द॒धा॒ त्ये॒त दे॒तद् द॑धाति दधा त्ये॒तद् वै वा ए॒तद् द॑धाति दधा त्ये॒तद् वै । \newline
55. ए॒तद् वै वा ए॒त दे॒तद् वा ऋ॑तू॒ना मृ॑तू॒नां ॅवा ए॒त दे॒तद् वा ऋ॑तू॒नाम् । \newline
56. वा ऋ॑तू॒ना मृ॑तू॒नां ॅवै वा ऋ॑तू॒नाम् प्रि॒यम् प्रि॒य मृ॑तू॒नां ॅवै वा ऋ॑तू॒नाम् प्रि॒यम् । \newline
57. ऋ॒तू॒नाम् प्रि॒यम् प्रि॒य मृ॑तू॒ना मृ॑तू॒नाम् प्रि॒यम् धाम॒ धाम॑ प्रि॒य मृ॑तू॒ना मृ॑तू॒नाम् प्रि॒यम् धाम॑ । \newline
58. प्रि॒यम् धाम॒ धाम॑ प्रि॒यम् प्रि॒यम् धाम॒ यद् यद् धाम॑ प्रि॒यम् प्रि॒यम् धाम॒ यत् । \newline
59. धाम॒ यद् यद् धाम॒ धाम॒ यदृ॑त॒व्या॑ ऋत॒व्या॑ यद् धाम॒ धाम॒ यदृ॑त॒व्याः᳚ । \newline
60. यदृ॑त॒व्या॑ ऋत॒व्या॑ यद् यदृ॑त॒व्या॑ ऋतू॒ना मृ॑तू॒ना मृ॑त॒व्या॑ यद् यदृ॑त॒व्या॑ ऋतू॒नाम् । \newline
61. ऋ॒त॒व्या॑ ऋतू॒ना मृ॑तू॒ना मृ॑त॒व्या॑ ऋत॒व्या॑ ऋतू॒ना मे॒वैव र्‌तू॒ना मृ॑त॒व्या॑ ऋत॒व्या॑ ऋतू॒ना मे॒व । \newline
62. ऋ॒तू॒ना मे॒वैव र्‌तू॒ना मृ॑तू॒ना मे॒व प्रि॒यम् प्रि॒य मे॒व र्‌तू॒ना मृ॑तू॒ना मे॒व प्रि॒यम् । \newline
63. ए॒व प्रि॒यम् प्रि॒य मे॒वैव प्रि॒यम् धाम॒ धाम॑ प्रि॒य मे॒वैव प्रि॒यम् धाम॑ । \newline
64. प्रि॒यम् धाम॒ धाम॑ प्रि॒यम् प्रि॒यम् धामावाव॒ धाम॑ प्रि॒यम् प्रि॒यम् धामाव॑ । \newline
65. धामावाव॒ धाम॒ धामाव॑ रुन्धे रु॒न्धे ऽव॒ धाम॒ धामाव॑ रुन्धे । \newline
66. अव॑ रुन्धे रु॒न्धे ऽवाव॑ रुन्धे सु॒मेकः॑ सु॒मेको॑ रु॒न्धे ऽवाव॑ रुन्धे सु॒मेकः॑ । \newline
67. रु॒न्धे॒ सु॒मेकः॑ सु॒मेको॑ रुन्धे रुन्धे सु॒मेक॒ इतीति॑ सु॒मेको॑ रुन्धे रुन्धे सु॒मेक॒ इति॑ । \newline
68. सु॒मेक॒ इतीति॑ सु॒मेकः॑ सु॒मेक॒ इत्या॑ हा॒हेति॑ सु॒मेकः॑ सु॒मेक॒ इत्या॑ह । \newline
69. सु॒मेक॒ इति॑ सु - मेकः॑ । \newline
70. इत्या॑ हा॒हे तीत्या॑ह संॅवथ्स॒रः सं॑ॅवथ्स॒र आ॒हे तीत्या॑ह संॅवथ्स॒रः । \newline
71. आ॒ह॒ सं॒ॅव॒थ्स॒रः सं॑ॅवथ्स॒र आ॑हाह संॅवथ्स॒रो वै वै सं॑ॅवथ्स॒र आ॑हाह संॅवथ्स॒रो वै । \newline
72. सं॒ॅव॒थ्स॒रो वै वै सं॑ॅवथ्स॒रः सं॑ॅवथ्स॒रो वै सु॒मेकः॑ सु॒मेको॒ वै सं॑ॅवथ्स॒रः सं॑ॅवथ्स॒रो वै सु॒मेकः॑ । \newline
73. सं॒ॅव॒थ्स॒र इति॑ सं - व॒थ्स॒रः । \newline
74. वै सु॒मेकः॑ सु॒मेको॒ वै वै सु॒मेकः॑ संॅवथ्स॒रस्य॑ संॅवथ्स॒रस्य॑ सु॒मेको॒ वै वै सु॒मेकः॑ संॅवथ्स॒रस्य॑ । \newline
75. सु॒मेकः॑ संॅवथ्स॒रस्य॑ संॅवथ्स॒रस्य॑ सु॒मेकः॑ सु॒मेकः॑ संॅवथ्स॒र स्यै॒वैव सं॑ॅवथ्स॒रस्य॑ सु॒मेकः॑ सु॒मेकः॑ संॅवथ्स॒र स्यै॒व । \newline
76. सु॒मेक॒ इति॑ सु - मेकः॑ । \newline
77. सं॒ॅव॒थ्स॒र स्यै॒वैव सं॑ॅवथ्स॒रस्य॑ संॅवथ्स॒र स्यै॒व प्रि॒यम् प्रि॒य मे॒व सं॑ॅवथ्स॒रस्य॑ संॅवथ्स॒र स्यै॒व प्रि॒यम् । \newline
78. सं॒ॅव॒थ्स॒रस्येति॑ सं - व॒थ्स॒रस्य॑ । \newline
79. ए॒व प्रि॒यम् प्रि॒य मे॒वैव प्रि॒यम् धाम॒ धाम॑ प्रि॒य मे॒वैव प्रि॒यम् धाम॑ । \newline
80. प्रि॒यम् धाम॒ धाम॑ प्रि॒यम् प्रि॒यम् धामोपोप॒ धाम॑ प्रि॒यम् प्रि॒यम् धामोप॑ । \newline
81. धामोपोप॒ धाम॒ धामोपा᳚प्नो त्याप्नो॒ त्युप॒ धाम॒ धामोपा᳚प्नोति । \newline
82. उपा᳚प्नो त्याप्नो॒ त्युपो पा᳚प्नोति । \newline
83. आ॒प्नो॒तीत्या᳚प्नोति । \newline
\pagebreak
\markright{ TS 5.3.12.1  \hfill https://www.vedavms.in \hfill}

\section{ TS 5.3.12.1 }

\textbf{TS 5.3.12.1 } \newline
\textbf{Samhita Paata} \newline

प्र॒जाप॑ते॒रक्ष्य॑श्वय॒त् तत् परा॑ऽपत॒त् तदश्वो॑ऽभव॒द्-यदश्व॑य॒त् तदश्व॑स्याश्व॒त्वं तद्दे॒वा अ॑श्वमे॒धेनै॒व प्रत्य॑दधुरे॒ष वै प्र॒जाप॑तिꣳ॒॒ सर्वं॑ करोति॒ यो᳚ऽश्वमे॒धेन॒ यज॑ते॒ सर्व॑ ए॒व भ॑वति॒ सर्व॑स्य॒ वा ए॒षा प्राय॑श्चित्तिः॒ सर्व॑स्य भेष॒जꣳ सर्वं॒ ॅवा ए॒तेन॑ पा॒प्मानं॑ दे॒वा अ॑तर॒न्नपि॒ वा ए॒तेन॑ ब्रह्मह॒त्याम॑तर॒न्थ् सर्वं॑ पा॒प्मानं॑ - [  ] \newline

\textbf{Pada Paata} \newline

प्र॒जाप॑ते॒रिति॑ प्र॒जा - प॒तेः॒ । अक्षि॑ । अ॒श्व॒य॒त् । तत् । परेति॑ । अ॒प॒त॒त् । तत् । अश्वः॑ । अ॒भ॒व॒त् । यत् । अश्व॑यत् । तत् । अश्व॑स्य । अ॒श्व॒त्वमित्य॑श्व- त्वम् । तत् । दे॒वाः । अ॒श्व॒मे॒धेनेत्य॑श्व - मे॒धेन॑ । ए॒व । प्रतीति॑ । अ॒द॒धुः॒ । ए॒षः । वै । प्र॒जाप॑ति॒मिति॑ प्र॒जा - प॒ति॒म् । सर्व᳚म् । क॒रो॒ति॒ । यः । अ॒श्व॒मे॒धेनेत्य॑श्व - मे॒धेन॑ । यज॑ते । सर्वः॑ । ए॒व । भ॒व॒ति॒ । सर्व॑स्य । वै । ए॒षा । प्राय॑श्चित्तिः । सर्व॑स्य । भे॒ष॒जम् । सर्व᳚म् । वै । ए॒तेन॑ । पा॒प्मान᳚म् । दे॒वाः । अ॒त॒र॒न्न् । अपीति॑ । वै । ए॒तेन॑ । ब्र॒ह्म॒ह॒त्यामिति॑ ब्रह्म - ह॒त्याम् । अ॒त॒र॒न्न् । सर्व᳚म् । पा॒प्मान᳚म् ।  \newline


\textbf{Krama Paata} \newline

प्र॒जाप॑ते॒रक्षि॑ । प्र॒जाप॑ते॒रिति॑ प्र॒जा - प॒तेः॒ । अक्ष्य॑श्वयत् । अ॒श्व॒य॒त् तत् । तत् परा᳚ । परा॑ऽपतत् । अ॒प॒त॒त् तत् । तदश्वः॑ । अश्वो॑ऽभवत् । अ॒भ॒व॒द् यत् । यदश्व॑यत् । अश्व॑य॒त् तत् । तदश्व॑स्य । अश्व॑स्याश्व॒त्वम् । अ॒श्व॒त्वम् तत् । अ॒श्व॒त्वमित्य॑श्व - त्वम् । तद् दे॒वाः । दे॒वा अ॑श्वमे॒धेन॑ । अ॒श्व॒मे॒धेनै॒व । अ॒श्व॒मे॒धेनेत्य॑श्व - मे॒धेन॑ । ए॒व प्रति॑ । प्रत्य॑दधुः । अ॒द॒धु॒रे॒षः । ए॒ष वै । वै प्र॒जाप॑तिम् । प्र॒जाप॑तिꣳ॒॒ सर्व᳚म् । प्र॒जाप॑ति॒मिति॑ प्र॒जा - प॒ति॒म् । सर्व॑म् करोति । क॒रो॒ति॒ यः । यो᳚ऽश्वमे॒धेन॑ । अ॒श्व॒मे॒धेन॒ यज॑ते । अ॒श्व॒मे॒धेनेत्य॑श्व - मे॒धेन॑ । यज॑ते॒ सर्वः॑ । सर्व॑ ए॒व । ए॒व भ॑वति । भ॒व॒ति॒ सर्व॑स्य । सर्व॑स्य॒ वै । वा ए॒षा । ए॒षा प्राय॑श्चित्तिः । प्राय॑श्चित्तिः॒ सर्व॑स्य । सर्व॑स्य भेष॒जम् । भे॒ष॒जꣳ सर्व᳚म् । सर्व॒म् ॅवै । वा ए॒तेन॑ । ए॒तेन॑ पा॒प्मान᳚म् । पा॒प्मान॑म् दे॒वाः । दे॒वा अ॑तरन्न् । अ॒त॒र॒न्नपि॑ । अपि॒ वै । वा ए॒तेन॑ । ए॒तेन॑ ब्रह्मह॒त्याम् । ब्र॒ह्म॒ह॒त्याम॑तरन्न् । ब्र॒ह्म॒ह॒त्यामिति॑ ब्रह्म - ह॒त्याम् । अ॒त॒र॒न्थ् सर्व᳚म् । सर्व॑म् पा॒प्मान᳚म् । पा॒प्मान॑म् तरति \newline

\textbf{Jatai Paata} \newline

1. प्र॒जाप॑ते॒ रक्ष्यक्षि॑ प्र॒जाप॑तेः प्र॒जाप॑ते॒ रक्षि॑ । \newline
2. प्र॒जाप॑ते॒रिति॑ प्र॒जा - प॒तेः॒ । \newline
3. अक्ष्य॑ श्वयद श्वय॒ दक्ष्य क्ष्य॑ श्वयत् । \newline
4. अ॒श्व॒य॒त् तत् तद॑श्वय दश्वय॒त् तत् । \newline
5. तत् परा॒ परा॒ तत् तत् परा᳚ । \newline
6. परा॑ ऽपत दपत॒त् परा॒ परा॑ ऽपतत् । \newline
7. अ॒प॒त॒त् तत् तद॑पत दपत॒त् तत् । \newline
8. तदश्वो ऽश्व॒ स्तत् तदश्वः॑ । \newline
9. अश्वो॑ ऽभव दभव॒ दश्वो ऽश्वो॑ ऽभवत् । \newline
10. अ॒भ॒व॒द् यद् यद॑भव दभव॒द् यत् । \newline
11. यदश्व॑य॒ दश्व॑य॒द् यद् यदश्व॑यत् । \newline
12. अश्व॑य॒त् तत् तदश्व॑य॒ दश्व॑य॒त् तत् । \newline
13. तदश्व॒स्या श्व॑स्य॒ तत् तदश्व॑स्य । \newline
14. अश्व॑स्या श्व॒त्व म॑श्व॒त्व मश्व॒स्या श्व॑स्या श्व॒त्वम् । \newline
15. अ॒श्व॒त्वम् तत् तद॑श्व॒त्व म॑श्व॒त्वम् तत् । \newline
16. अ॒श्व॒त्वमित्य॑श्व - त्वम् । \newline
17. तद् दे॒वा दे॒वा स्तत् तद् दे॒वाः । \newline
18. दे॒वा अ॑श्वमे॒धेना᳚ श्वमे॒धेन॑ दे॒वा दे॒वा अ॑श्वमे॒धेन॑ । \newline
19. अ॒श्व॒मे॒धे नै॒वै वाश्व॑मे॒धेना᳚ श्वमे॒धेनै॒व । \newline
20. अ॒श्व॒मे॒धेनेत्य॑श्व - मे॒धेन॑ । \newline
21. ए॒व प्रति॒ प्रत्ये॒ वैव प्रति॑ । \newline
22. प्रत्य॑ दधु रदधुः॒ प्रति॒ प्रत्य॑ दधुः । \newline
23. अ॒द॒धु॒ रे॒ष ए॒षो॑ ऽदधु रदधु रे॒षः । \newline
24. ए॒ष वै वा ए॒ष ए॒ष वै । \newline
25. वै प्र॒जाप॑तिम् प्र॒जाप॑तिं॒ ॅवै वै प्र॒जाप॑तिम् । \newline
26. प्र॒जाप॑तिꣳ॒॒ सर्वꣳ॒॒ सर्व॑म् प्र॒जाप॑तिम् प्र॒जाप॑तिꣳ॒॒ सर्व᳚म् । \newline
27. प्र॒जाप॑ति॒मिति॑ प्र॒जा - प॒ति॒म् । \newline
28. सर्व॑म् करोति करोति॒ सर्वꣳ॒॒ सर्व॑म् करोति । \newline
29. क॒रो॒ति॒ यो यः क॑रोति करोति॒ यः । \newline
30. यो᳚ ऽश्वमे॒धेना᳚ श्वमे॒धेन॒ यो यो᳚ ऽश्वमे॒धेन॑ । \newline
31. अ॒श्व॒मे॒धेन॒ यज॑ते॒ यज॑ते ऽश्वमे॒धेना᳚ श्वमे॒धेन॒ यज॑ते । \newline
32. अ॒श्व॒मे॒धेनेत्य॑श्व - मे॒धेन॑ । \newline
33. यज॑ते॒ सर्वः॒ सर्वो॒ यज॑ते॒ यज॑ते॒ सर्वः॑ । \newline
34. सर्व॑ ए॒वैव सर्वः॒ सर्व॑ ए॒व । \newline
35. ए॒व भ॑वति भव त्ये॒वैव भ॑वति । \newline
36. भ॒व॒ति॒ सर्व॑स्य॒ सर्व॑स्य भवति भवति॒ सर्व॑स्य । \newline
37. सर्व॑स्य॒ वै वै सर्व॑स्य॒ सर्व॑स्य॒ वै । \newline
38. वा ए॒षैषा वै वा ए॒षा । \newline
39. ए॒षा प्राय॑श्चित्तिः॒ प्राय॑श्चित्ति रे॒षैषा प्राय॑श्चित्तिः । \newline
40. प्राय॑श्चित्तिः॒ सर्व॑स्य॒ सर्व॑स्य॒ प्राय॑श्चित्तिः॒ प्राय॑श्चित्तिः॒ सर्व॑स्य । \newline
41. सर्व॑स्य भेष॒जम् भे॑ष॒जꣳ सर्व॑स्य॒ सर्व॑स्य भेष॒जम् । \newline
42. भे॒ष॒जꣳ सर्वꣳ॒॒ सर्व॑म् भेष॒जम् भे॑ष॒जꣳ सर्व᳚म् । \newline
43. सर्वं॒ ॅवै वै सर्वꣳ॒॒ सर्वं॒ ॅवै । \newline
44. वा ए॒ते नै॒तेन॒ वै वा ए॒तेन॑ । \newline
45. ए॒तेन॑ पा॒प्मान॑म् पा॒प्मान॑ मे॒ते नै॒तेन॑ पा॒प्मान᳚म् । \newline
46. पा॒प्मान॑म् दे॒वा दे॒वाः पा॒प्मान॑म् पा॒प्मान॑म् दे॒वाः । \newline
47. दे॒वा अ॑तरन् नतरन् दे॒वा दे॒वा अ॑तरन्न् । \newline
48. अ॒त॒र॒न् नप्य प्य॑तरन् नतर॒न् नपि॑ । \newline
49. अपि॒ वै वा अप्यपि॒ वै । \newline
50. वा ए॒ते नै॒तेन॒ वै वा ए॒तेन॑ । \newline
51. ए॒तेन॑ ब्रह्मह॒त्याम् ब्र॑ह्मह॒त्या मे॒ते नै॒तेन॑ ब्रह्मह॒त्याम् । \newline
52. ब्र॒ह्म॒ह॒त्या म॑तरन् नतरन् ब्रह्मह॒त्याम् ब्र॑ह्मह॒त्या म॑तरन्न् । \newline
53. ब्र॒ह्म॒ह॒त्यामिति॑ ब्रह्म - ह॒त्याम् । \newline
54. अ॒त॒र॒न् थ्सर्वꣳ॒॒ सर्व॑ मतरन् नतर॒न् थ्सर्व᳚म् । \newline
55. सर्व॑म् पा॒प्मान॑म् पा॒प्मानꣳ॒॒ सर्वꣳ॒॒ सर्व॑म् पा॒प्मान᳚म् । \newline
56. पा॒प्मान॑म् तरति तरति पा॒प्मान॑म् पा॒प्मान॑म् तरति । \newline

\textbf{Ghana Paata } \newline

1. प्र॒जाप॑ते॒ रक्ष्यक्षि॑ प्र॒जाप॑तेः प्र॒जाप॑ते॒ रक्ष्य॑ श्वय दश्वय॒ दक्षि॑ प्र॒जाप॑तेः प्र॒जाप॑ते॒ रक्ष्य॑श्वयत् । \newline
2. प्र॒जाप॑ते॒रिति॑ प्र॒जा - प॒तेः॒ । \newline
3. अक्ष्य॑श्वय दश्वय॒ दक्ष्य क्ष्य॑श्वय॒त् तत् तद॑श्वय॒ दक्ष्य क्ष्य॑श्वय॒त् तत् । \newline
4. अ॒श्व॒य॒त् तत् तद॑श्वय दश्वय॒त् तत् परा॒ परा॒ तद॑श्वय दश्वय॒त् तत् परा᳚ । \newline
5. तत् परा॒ परा॒ तत् तत् परा॑ ऽपत दपत॒त् परा॒ तत् तत् परा॑ ऽपतत् । \newline
6. परा॑ ऽपत दपत॒त् परा॒ परा॑ ऽपत॒त् तत् तद॑ पत॒त् परा॒ परा॑ ऽपत॒त् तत् । \newline
7. अ॒प॒त॒त् तत् तद॑पत दपत॒त् तदश्वो ऽश्व॒ स्तद॑पत दपत॒त् तदश्वः॑ । \newline
8. तदश्वो ऽश्व॒ स्तत् तदश्वो॑ ऽभव दभव॒ दश्व॒ स्तत् तदश्वो॑ ऽभवत् । \newline
9. अश्वो॑ ऽभव दभव॒ दश्वो ऽश्वो॑ ऽभव॒द् यद् यद॑भव॒ दश्वो ऽश्वो॑ ऽभव॒द् यत् । \newline
10. अ॒भ॒व॒द् यद् यद॑भव दभव॒द् यदश्व॑य॒ दश्व॑य॒द् यद॑भव दभव॒द् यदश्व॑यत् । \newline
11. यदश्व॑य॒ दश्व॑य॒द् यद् यदश्व॑य॒त् तत् तदश्व॑य॒द् यद् यदश्व॑य॒त् तत् । \newline
12. अश्व॑य॒त् तत् तदश्व॑य॒ दश्व॑य॒त् तदश्व॒स्या श्व॑स्य॒ तदश्व॑य॒ दश्व॑य॒त् तदश्व॑स्य । \newline
13. तदश्व॒स्या श्व॑स्य॒ तत् तदश्व॑ स्याश्व॒त्व म॑श्व॒त्व मश्व॑स्य॒ तत् तदश्व॑स्या श्व॒त्वम् । \newline
14. अश्व॑स्या श्व॒त्व म॑श्व॒त्व मश्व॒स्या श्व॑स्या श्व॒त्वम् तत् तद॑श्व॒त्व मश्व॒स्या श्व॑स्या श्व॒त्वम् तत् । \newline
15. अ॒श्व॒त्वम् तत् तद॑श्व॒त्व म॑श्व॒त्वम् तद् दे॒वा दे॒वा स्त द॑श्व॒त्व म॑श्व॒त्वम् तद् दे॒वाः । \newline
16. अ॒श्व॒त्वमित्य॑श्व - त्वम् । \newline
17. तद् दे॒वा दे॒वा स्तत् तद् दे॒वा अ॑श्वमे॒धेना᳚ श्वमे॒धेन॑ दे॒वा स्तत् तद् दे॒वा अ॑श्वमे॒धेन॑ । \newline
18. दे॒वा अ॑श्वमे॒धेना᳚ श्वमे॒धेन॑ दे॒वा दे॒वा अ॑श्वमे॒धे नै॒वैवा श्व॑मे॒धेन॑ दे॒वा दे॒वा अ॑श्वमे॒धे नै॒व । \newline
19. अ॒श्व॒मे॒धे नै॒वै वाश्व॑मे॒धेना᳚ श्वमे॒धे नै॒व प्रति॒ प्रत्ये॒ वाश्व॑मे॒धेना᳚ श्वमे॒धे नै॒व प्रति॑ । \newline
20. अ॒श्व॒मे॒धेनेत्य॑श्व - मे॒धेन॑ । \newline
21. ए॒व प्रति॒ प्रत्ये॒ वैव प्रत्य॑दधु रदधुः॒ प्रत्ये॒ वैव प्रत्य॑ दधुः । \newline
22. प्रत्य॑ दधु रदधुः॒ प्रति॒ प्रत्य॑ दधु रे॒ष ए॒षो॑ ऽदधुः॒ प्रति॒ प्रत्य॑ दधु रे॒षः । \newline
23. अ॒द॒धु॒ रे॒ष ए॒षो॑ ऽदधु रदधु रे॒ष वै वा ए॒षो॑ ऽदधु रदधु रे॒ष वै । \newline
24. ए॒ष वै वा ए॒ष ए॒ष वै प्र॒जाप॑तिम् प्र॒जाप॑तिं॒ ॅवा ए॒ष ए॒ष वै प्र॒जाप॑तिम् । \newline
25. वै प्र॒जाप॑तिम् प्र॒जाप॑तिं॒ ॅवै वै प्र॒जाप॑तिꣳ॒॒ सर्वꣳ॒॒ सर्व॑म् प्र॒जाप॑तिं॒ ॅवै वै प्र॒जाप॑तिꣳ॒॒ सर्व᳚म् । \newline
26. प्र॒जाप॑तिꣳ॒॒ सर्वꣳ॒॒ सर्व॑म् प्र॒जाप॑तिम् प्र॒जाप॑तिꣳ॒॒ सर्व॑म् करोति करोति॒ सर्व॑म् प्र॒जाप॑तिम् प्र॒जाप॑तिꣳ॒॒ सर्व॑म् करोति । \newline
27. प्र॒जाप॑ति॒मिति॑ प्र॒जा - प॒ति॒म् । \newline
28. सर्व॑म् करोति करोति॒ सर्वꣳ॒॒ सर्व॑म् करोति॒ यो यः क॑रोति॒ सर्वꣳ॒॒ सर्व॑म् करोति॒ यः । \newline
29. क॒रो॒ति॒ यो यः क॑रोति करोति॒ यो᳚ ऽश्वमे॒धेना᳚ श्वमे॒धेन॒ यः क॑रोति करोति॒ यो᳚  ऽश्वमे॒धेन॑ । \newline
30. यो᳚ ऽश्वमे॒धेना᳚ श्वमे॒धेन॒ यो यो᳚ ऽश्वमे॒धेन॒ यज॑ते॒ यज॑ते ऽश्वमे॒धेन॒ यो यो᳚ ऽश्वमे॒धेन॒ यज॑ते । \newline
31. अ॒श्व॒मे॒धेन॒ यज॑ते॒ यज॑ते ऽश्वमे॒धेना᳚ श्वमे॒धेन॒ यज॑ते॒ सर्वः॒ सर्वो॒ यज॑ते ऽश्वमे॒धेना᳚ श्वमे॒धेन॒ यज॑ते॒ सर्वः॑ । \newline
32. अ॒श्व॒मे॒धेनेत्य॑श्व - मे॒धेन॑ । \newline
33. यज॑ते॒ सर्वः॒ सर्वो॒ यज॑ते॒ यज॑ते॒ सर्व॑ ए॒वैव सर्वो॒ यज॑ते॒ यज॑ते॒ सर्व॑ ए॒व । \newline
34. सर्व॑ ए॒वैव सर्वः॒ सर्व॑ ए॒व भ॑वति भव त्ये॒व सर्वः॒ सर्व॑ ए॒व भ॑वति । \newline
35. ए॒व भ॑वति भव त्ये॒वैव भ॑वति॒ सर्व॑स्य॒ सर्व॑स्य भव त्ये॒वैव भ॑वति॒ सर्व॑स्य । \newline
36. भ॒व॒ति॒ सर्व॑स्य॒ सर्व॑स्य भवति भवति॒ सर्व॑स्य॒ वै वै सर्व॑स्य भवति भवति॒ सर्व॑स्य॒ वै । \newline
37. सर्व॑स्य॒ वै वै सर्व॑स्य॒ सर्व॑स्य॒ वा ए॒षैषा वै सर्व॑स्य॒ सर्व॑स्य॒ वा ए॒षा । \newline
38. वा ए॒षैषा वै वा ए॒षा प्राय॑श्चित्तिः॒ प्राय॑श्चित्ति रे॒षा वै वा ए॒षा प्राय॑श्चित्तिः । \newline
39. ए॒षा प्राय॑श्चित्तिः॒ प्राय॑श्चित्ति रे॒षैषा प्राय॑श्चित्तिः॒ सर्व॑स्य॒ सर्व॑स्य॒ प्राय॑श्चित्ति रे॒षैषा प्राय॑श्चित्तिः॒ सर्व॑स्य । \newline
40. प्राय॑श्चित्तिः॒ सर्व॑स्य॒ सर्व॑स्य॒ प्राय॑श्चित्तिः॒ प्राय॑श्चित्तिः॒ सर्व॑स्य भेष॒जम् भे॑ष॒जꣳ सर्व॑स्य॒ प्राय॑श्चित्तिः॒ प्राय॑श्चित्तिः॒ सर्व॑स्य भेष॒जम् । \newline
41. सर्व॑स्य भेष॒जम् भे॑ष॒जꣳ सर्व॑स्य॒ सर्व॑स्य भेष॒जꣳ सर्वꣳ॒॒ सर्व॑म् भेष॒जꣳ सर्व॑स्य॒ सर्व॑स्य भेष॒जꣳ सर्व᳚म् । \newline
42. भे॒ष॒जꣳ सर्वꣳ॒॒ सर्व॑म् भेष॒जम् भे॑ष॒जꣳ सर्वं॒ ॅवै वै सर्व॑म् भेष॒जम् भे॑ष॒जꣳ सर्वं॒ ॅवै । \newline
43. सर्वं॒ ॅवै वै सर्वꣳ॒॒ सर्वं॒ ॅवा ए॒तेनै॒तेन॒ वै सर्वꣳ॒॒ सर्वं॒ ॅवा ए॒तेन॑ । \newline
44. वा ए॒तेनै॒तेन॒ वै वा ए॒तेन॑ पा॒प्मान॑म् पा॒प्मान॑ मे॒तेन॒ वै वा ए॒तेन॑ पा॒प्मान᳚म् । \newline
45. ए॒तेन॑ पा॒प्मान॑म् पा॒प्मान॑ मे॒तेनै॒तेन॑ पा॒प्मान॑म् दे॒वा दे॒वाः पा॒प्मान॑ मे॒तेनै॒तेन॑ पा॒प्मान॑म् दे॒वाः । \newline
46. पा॒प्मान॑म् दे॒वा दे॒वाः पा॒प्मान॑म् पा॒प्मान॑म् दे॒वा अ॑तरन् नतरन् दे॒वाः पा॒प्मान॑म् पा॒प्मान॑म् दे॒वा अ॑तरन्न् । \newline
47. दे॒वा अ॑तरन् नतरन् दे॒वा दे॒वा अ॑तर॒न् नप्य प्य॑तरन् दे॒वा दे॒वा अ॑तर॒न् नपि॑ । \newline
48. अ॒त॒र॒न् नप्य प्य॑तरन् नतर॒न् नपि॒ वै वा अप्य॑तरन् नतर॒न् नपि॒ वै । \newline
49. अपि॒ वै वा अप्यपि॒ वा ए॒ते नै॒तेन॒ वा अप्यपि॒ वा ए॒तेन॑ । \newline
50. वा ए॒ते नै॒तेन॒ वै वा ए॒तेन॑ ब्रह्मह॒त्याम् ब्र॑ह्मह॒त्या मे॒तेन॒ वै वा ए॒तेन॑ ब्रह्मह॒त्याम् । \newline
51. ए॒तेन॑ ब्रह्मह॒त्याम् ब्र॑ह्मह॒त्या मे॒ते नै॒तेन॑ ब्रह्मह॒त्या म॑तरन् नतरन् ब्रह्मह॒त्या मे॒ते नै॒तेन॑ ब्रह्मह॒त्या म॑तरन्न् । \newline
52. ब्र॒ह्म॒ह॒त्या म॑तरन् नतरन् ब्रह्मह॒त्याम् ब्र॑ह्मह॒त्या म॑तर॒न् थ्सर्वꣳ॒॒ सर्व॑ मतरन् ब्रह्मह॒त्याम् ब्र॑ह्मह॒त्या म॑तर॒न् थ्सर्व᳚म् । \newline
53. ब्र॒ह्म॒ह॒त्यामिति॑ ब्रह्म - ह॒त्याम् । \newline
54. अ॒त॒र॒न् थ्सर्वꣳ॒॒ सर्व॑ मतरन् नतर॒न् थ्सर्व॑म् पा॒प्मान॑म् पा॒प्मानꣳ॒॒ सर्व॑ मतरन् नतर॒न् थ्सर्व॑म् पा॒प्मान᳚म् । \newline
55. सर्व॑म् पा॒प्मान॑म् पा॒प्मानꣳ॒॒ सर्वꣳ॒॒ सर्व॑म् पा॒प्मान॑म् तरति तरति पा॒प्मानꣳ॒॒ सर्वꣳ॒॒ सर्व॑म् पा॒प्मान॑म् तरति । \newline
56. पा॒प्मान॑म् तरति तरति पा॒प्मान॑म् पा॒प्मान॑म् तरति॒ तर॑ति॒ तर॑ति तरति पा॒प्मान॑म् पा॒प्मान॑म् तरति॒ तर॑ति । \newline
\pagebreak
\markright{ TS 5.3.12.2  \hfill https://www.vedavms.in \hfill}

\section{ TS 5.3.12.2 }

\textbf{TS 5.3.12.2 } \newline
\textbf{Samhita Paata} \newline

तरति॒ तर॑ति ब्रह्मह॒त्यां ॅयो᳚ऽश्वमे॒धेन॒ यज॑ते॒ य उ॑ चैनमे॒वं ॅवेदोत्त॑रं॒ ॅवै तत् प्र॒जाप॑ते॒रक्ष्य॑श्वय॒त् तस्मा॒दश्व॑स्योत्तर॒तोऽव॑ द्यन्ति दक्षिण॒तो᳚ऽन्येषां᳚ पशू॒नां ॅवै॑त॒सः कटो॑ भवत्य॒फ्सुयो॑नि॒र्वा अश्वो᳚ऽफ्सु॒जो वे॑त॒सः स्व ए॒वैनं॒ ॅयोनौ॒ प्रति॑ष्ठापयति चतुष्टो॒मः स्तोमो॑ भवति स॒रड्ढ॒ वा अश्व॑स्य॒ सक्थ्याऽवृ॑ह॒त् ( ) तद्-दे॒वाश्च॑तुष्टो॒मेनै॒व प्रत्य॑दधु॒र्यच्च॑तुष्टो॒मः स्तोमो॒ भव॒त्यश्व॑स्य सर्व॒त्वाय॑ ॥ \newline

\textbf{Pada Paata} \newline

त॒र॒ति॒ । तर॑ति । ब्र॒ह्म॒ह॒त्यामिति॑ ब्रह्म - ह॒त्याम् । यः । अ॒श्व॒मे॒धेनेत्य॑श्व - मे॒धेन॑ । यज॑ते । यः । उ॒ । च॒ । ए॒न॒म् । ए॒वम् । वेद॑ । उत्त॑र॒मित्युत् - त॒र॒म् । वै । तत् । प्र॒जाप॑ते॒रिति॑ प्र॒जा - प॒तेः॒ । अक्षि॑ । अ॒श्व॒य॒त् । तस्मा᳚त् । अश्व॑स्य । उ॒त्त॒र॒त इत्यु॑त् - त॒र॒तः । अवेति॑ । द्य॒न्ति॒ । द॒क्षि॒ण॒तः । अ॒न्येषा᳚म् । प॒शू॒नाम् । वै॒त॒सः । कटः॑ । भ॒व॒ति॒ । अ॒फ्सुयो॑नि॒रित्य॒फ्सु - यो॒निः॒ । वै । अश्वः॑ । अ॒फ्सु॒ज इत्य॑फ्सु - जः । वे॒त॒सः । स्वे । ए॒व । ए॒न॒म् । योनौ᳚ । प्रतीति॑ । स्था॒प॒य॒ति॒ । च॒तु॒ष्टो॒म इति॑ चतुः - स्तो॒मः । स्तोमः॑ । भ॒व॒ति॒ । स॒रट् । ह॒ । वै । अश्व॑स्य । सक्थि॑ । एति॑ । अ॒वृ॒ह॒त् ( ) । तत् । दे॒वाः । च॒तु॒ष्टो॒मेनेति॑ चतुः-स्तो॒मेन॑ । ए॒व । प्रतीति॑ । अ॒द॒धुः॒ । यत् । च॒तु॒ष्टो॒म इति॑ चतुः - स्तो॒मः । स्तोमः॑ । भव॑ति । अश्व॑स्य । स॒र्व॒त्वायेति॑ सर्व - त्वाय॑ ॥  \newline


\textbf{Krama Paata} \newline

त॒र॒ति॒ तर॑ति । तर॑ति ब्रह्मह॒त्याम् । ब्र॒ह्म॒ह॒त्याम् ॅयः । ब्र॒ह्म॒ह॒त्यामिति॑ ब्रह्म - ह॒त्याम् । यो᳚ऽश्वमे॒धेन॑ । अ॒श्व॒मे॒धेन॒ यज॑ते । अ॒श्व॒मे॒धेनेत्य॑श्व - मे॒धेन॑ । यज॑ते॒ यः । य उ॑ । उ॒ च॒ । चै॒न॒म् । ए॒न॒मे॒वम् । ए॒वम् ॅवेद॑ । वेदोत्त॑रम् । उत्त॑र॒म् ॅवै । उत्त॑र॒मित्युत् - त॒र॒म् । वै तत् । तत् प्र॒जाप॑तेः । प्र॒जाप॑ते॒रक्षि॑ । प्र॒जाप॑ते॒रिति॑ प्र॒जा - प॒तेः॒ । अक्ष्य॑श्वयत् । अ॒श्व॒य॒त् तस्मा᳚त् । तस्मा॒दश्व॑स्य । अश्व॑स्योत्तर॒तः । उ॒त्त॒र॒तोऽव॑ । उ॒त्त॒र॒त इत्यु॑त् - त॒र॒तः॒ । अव॑ द्यन्ति । द्य॒न्ति॒ द॒क्षि॒ण॒तः । द॒क्षि॒ण॒तो᳚ऽन्येषा᳚म् । अ॒न्येषा᳚म् पशू॒नाम् । प॒शू॒नाम् ॅवै॑त॒सः । वै॒त॒सः कटः॑ । कटो॑ भवति । भ॒व॒त्य॒फ्सुयो॑निः । अ॒फ्सुयो॑नि॒र् वै । अ॒फ्सुयो॑नि॒रित्य॒फ्सु - यो॒निः॒ । वा अश्वः॑ । अश्वो᳚ऽफ्सु॒जः । अ॒फ्सु॒जो वे॑त॒सः । अ॒फ्सु॒ज इत्य॑फ्सु - जः । वे॒त॒सः स्वे । स्व ए॒व । ए॒वैन᳚म् । ए॒न॒म् ॅयोनौ᳚ । योनौ॒ प्रति॑ । प्रति॑ ष्ठापयति । स्था॒प॒य॒ति॒ च॒तु॒ष्टो॒मः । च॒तु॒ष्टो॒मः स्तोमः॑ । च॒तु॒ष्टो॒म इति॑ चतुः - स्तो॒मः । स्तोमो॑ भवति । भ॒व॒ति॒ स॒रट् । स॒रड्ढ॑ । ह॒ वै । वा अश्व॑स्य । अश्व॑स्य॒ सक्थि॑ । सक्थ्या । आऽवृ॑हत् ( ) । अ॒वृ॒ह॒त् तत् । तद् दे॒वाः । दे॒वाश्च॑तुष्टो॒मेन॑ । च॒तु॒ष्टो॒मेनै॒व । च॒तु॒ष्टो॒मेनेति॑ चतुः - स्तो॒मेन॑ । ए॒व प्रति॑ । प्रत्य॑दधुः । अ॒द॒धु॒र् यत् । यच् च॑तुष्टो॒मः । च॒तु॒ष्टो॒मः स्तोमः॑ । च॒तु॒ष्टो॒म इति॑ चतुः - स्तो॒मः । स्तोमो॒ भव॑ति । भव॒त्यश्व॑स्य । अश्व॑स्य सर्व॒त्वाय॑ । स॒र्वा॒त्वायेति॑ सर्व - त्वाय॑ । \newline

\textbf{Jatai Paata} \newline

1. त॒र॒ति॒ तर॑ति॒ तर॑ति तरति तरति॒ तर॑ति । \newline
2. तर॑ति ब्रह्मह॒त्याम् ब्र॑ह्मह॒त्याम् तर॑ति॒ तर॑ति ब्रह्मह॒त्याम् । \newline
3. ब्र॒ह्म॒ह॒त्यां ॅयो यो ब्र॑ह्मह॒त्याम् ब्र॑ह्मह॒त्यां ॅयः । \newline
4. ब्र॒ह्म॒ह॒त्यामिति॑ ब्रह्म - ह॒त्याम् । \newline
5. यो᳚ ऽश्वमे॒धेना᳚ श्वमे॒धेन॒ यो यो᳚ ऽश्वमे॒धेन॑ । \newline
6. अ॒श्व॒मे॒धेन॒ यज॑ते॒ यज॑ते ऽश्वमे॒धेना᳚ श्वमे॒धेन॒ यज॑ते । \newline
7. अ॒श्व॒मे॒धेनेत्य॑श्व - मे॒धेन॑ । \newline
8. यज॑ते॒ यो यो यज॑ते॒ यज॑ते॒ यः । \newline
9. य उ॑ वु॒ यो य उ॑ । \newline
10. उ॒ च॒ च॒ वु॒ च॒ । \newline
11. चै॒न॒ मे॒न॒म् च॒ चै॒न॒म् । \newline
12. ए॒न॒ मे॒व मे॒व मे॑न मेन मे॒वम् । \newline
13. ए॒वं ॅवेद॒ वेदै॒व मे॒वं ॅवेद॑ । \newline
14. वेदोत्त॑र॒ मुत्त॑रं॒ ॅवेद॒ वेदोत्त॑रम् । \newline
15. उत्त॑रं॒ ॅवै वा उत्त॑र॒ मुत्त॑रं॒ ॅवै । \newline
16. उत्त॑र॒मित्युत् - त॒र॒म् । \newline
17. वै तत् तद् वै वै तत् । \newline
18. तत् प्र॒जाप॑तेः प्र॒जाप॑ते॒ स्तत् तत् प्र॒जाप॑तेः । \newline
19. प्र॒जाप॑ते॒ रक्ष्यक्षि॑ प्र॒जाप॑तेः प्र॒जाप॑ते॒ रक्षि॑ । \newline
20. प्र॒जाप॑ते॒रिति॑ प्र॒जा - प॒तेः॒ । \newline
21. अक्ष्य॑ श्वय दश्वय॒ दक्ष्य क्ष्य॑ श्वयत् । \newline
22. अ॒श्व॒य॒त् तस्मा॒त् तस्मा॑ दश्वय दश्वय॒त् तस्मा᳚त् । \newline
23. तस्मा॒ दश्व॒स्या श्व॑स्य॒ तस्मा॒त् तस्मा॒ दश्व॑स्य । \newline
24. अश्व॑स्यो त्तर॒त उ॑त्तर॒तो ऽश्व॒स्या श्व॑स्यो त्तर॒तः । \newline
25. उ॒त्त॒र॒तो ऽवावो᳚ त्तर॒त उ॑त्तर॒तो ऽव॑ । \newline
26. उ॒त्त॒र॒त इत्यु॑त् - त॒र॒तः । \newline
27. अव॑ द्यन्ति द्य॒न् त्यवाव॑ द्यन्ति । \newline
28. द्य॒न्ति॒ द॒क्षि॒ण॒तो द॑क्षिण॒तो द्य॑न्ति द्यन्ति दक्षिण॒तः । \newline
29. द॒क्षि॒ण॒तो᳚ ऽन्येषा॑ म॒न्येषा᳚म् दक्षिण॒तो द॑क्षिण॒तो᳚ ऽन्येषा᳚म् । \newline
30. अ॒न्येषा᳚म् पशू॒नाम् प॑शू॒ना म॒न्येषा॑ म॒न्येषा᳚म् पशू॒नाम् । \newline
31. प॒शू॒नां ॅवै॑त॒सो वै॑त॒सः प॑शू॒नाम् प॑शू॒नां ॅवै॑त॒सः । \newline
32. वै॒त॒सः कटः॒ कटो॑ वैत॒सो वै॑त॒सः कटः॑ । \newline
33. कटो॑ भवति भवति॒ कटः॒ कटो॑ भवति । \newline
34. भ॒व॒ त्य॒फ्सुयो॑नि र॒फ्सुयो॑निर् भवति भव त्य॒फ्सुयो॑निः । \newline
35. अ॒फ्सुयो॑नि॒र् वै वा अ॒फ्सुयो॑नि र॒फ्सुयो॑नि॒र् वै । \newline
36. अ॒फ्सुयो॑नि॒रित्य॒फ्सु - यो॒निः॒ । \newline
37. वा अश्वो ऽश्वो॒ वै वा अश्वः॑ । \newline
38. अश्वो᳚ ऽफ्सु॒जो᳚ ऽफ्सु॒जो ऽश्वो ऽश्वो᳚ ऽफ्सु॒जः । \newline
39. अ॒फ्सु॒जो वे॑त॒सो वे॑त॒सो᳚ ऽफ्सु॒जो᳚ ऽफ्सु॒जो वे॑त॒सः । \newline
40. अ॒फ्सु॒ज इत्य॑फ्सु - जः । \newline
41. वे॒त॒सः स्वे स्वे वे॑त॒सो वे॑त॒सः स्वे । \newline
42. स्व ए॒वैव स्वे स्व ए॒व । \newline
43. ए॒वैन॑ मेन मे॒वै वैन᳚म् । \newline
44. ए॒नं॒ ॅयोनौ॒ योना॑ वेन मेनं॒ ॅयोनौ᳚ । \newline
45. योनौ॒ प्रति॒ प्रति॒ योनौ॒ योनौ॒ प्रति॑ । \newline
46. प्रति॑ ष्ठापयति स्थापयति॒ प्रति॒ प्रति॑ ष्ठापयति । \newline
47. स्था॒प॒य॒ति॒ च॒तु॒ष्टो॒म श्च॑तुष्टो॒मः स्था॑पयति स्थापयति चतुष्टो॒मः । \newline
48. च॒तु॒ष्टो॒मः स्तोमः॒ स्तोम॑ श्चतुष्टो॒म श्च॑तुष्टो॒मः स्तोमः॑ । \newline
49. च॒तु॒ष्टो॒म इति॑ चतुः - स्तो॒मः । \newline
50. स्तोमो॑ भवति भवति॒ स्तोमः॒ स्तोमो॑ भवति । \newline
51. भ॒व॒ति॒ स॒रट् थ्स॒रड् भ॑वति भवति स॒रट् । \newline
52. स॒रड्ढ॑ ह स॒रट् थ्स॒रड्ढ॑ । \newline
53. ह॒ वै वै ह॑ ह॒ वै । \newline
54. वा अश्व॒स्या श्व॑स्य॒ वै वा अश्व॑स्य । \newline
55. अश्व॑स्य॒ सक्थि॒ सक्थ्य श्व॒स्या श्व॑स्य॒ सक्थि॑ । \newline
56. सक्थ्या सक्थि॒ सक्थ्या । \newline
57. आ ऽवृ॑ह दवृह॒दा ऽवृ॑हत् । \newline
58. अ॒वृ॒ह॒त् तत् तद॑वृह दवृह॒त् तत् । \newline
59. तद् दे॒वा दे॒वा स्तत् तद् दे॒वाः । \newline
60. दे॒वा श्च॑तुष्टो॒मेन॑ चतुष्टो॒मेन॑ दे॒वा दे॒वा श्च॑तुष्टो॒मेन॑ । \newline
61. च॒तु॒ष्टो॒मेनै॒ वैव च॑तुष्टो॒मेन॑ चतुष्टो॒मे नै॒व । \newline
62. च॒तु॒ष्टो॒मेनेति॑ चतुः - स्तो॒मेन॑ । \newline
63. ए॒व प्रति॒ प्रत्ये॒वैव प्रति॑ । \newline
64. प्रत्य॑दधु रदधुः॒ प्रति॒ प्रत्य॑दधुः । \newline
65. अ॒द॒धु॒र् यद् यद॑दधु रदधु॒र् यत् । \newline
66. यच् च॑तुष्टो॒म श्च॑तुष्टो॒मो यद् यच् च॑तुष्टो॒मः । \newline
67. च॒तु॒ष्टो॒मः स्तोमः॒ स्तोम॑ श्चतुष्टो॒म श्च॑तुष्टो॒मः स्तोमः॑ । \newline
68. च॒तु॒ष्टो॒म इति॑ चतुः - स्तो॒मः । \newline
69. स्तोमो॒ भव॑ति॒ भव॑ति॒ स्तोमः॒ स्तोमो॒ भव॑ति । \newline
70. भव॒त्य श्व॒स्या श्व॑स्य॒ भव॑ति॒ भव॒त्य श्व॑स्य । \newline
71. अश्व॑स्य सर्व॒त्वाय॑ सर्व॒त्वाया श्व॒स्या श्व॑स्य सर्व॒त्वाय॑ । \newline
72. स॒र्व॒त्वायेति॑ सर्व - त्वाय॑ । \newline

\textbf{Ghana Paata } \newline

1. त॒र॒ति॒ तर॑ति॒ तर॑ति तरति तरति॒ तर॑ति ब्रह्मह॒त्याम् ब्र॑ह्मह॒त्याम् तर॑ति तरति तरति॒ तर॑ति ब्रह्मह॒त्याम् । \newline
2. तर॑ति ब्रह्मह॒त्याम् ब्र॑ह्मह॒त्याम् तर॑ति॒ तर॑ति ब्रह्मह॒त्यां ॅयो यो ब्र॑ह्मह॒त्याम् तर॑ति॒ तर॑ति ब्रह्मह॒त्यां ॅयः । \newline
3. ब्र॒ह्म॒ह॒त्यां ॅयो यो ब्र॑ह्मह॒त्याम् ब्र॑ह्मह॒त्यां ॅयो᳚ ऽश्वमे॒धेना᳚ श्वमे॒धेन॒ यो ब्र॑ह्मह॒त्याम् ब्र॑ह्मह॒त्यां ॅयो᳚ ऽश्वमे॒धेन॑ । \newline
4. ब्र॒ह्म॒ह॒त्यामिति॑ ब्रह्म - ह॒त्याम् । \newline
5. यो᳚ ऽश्वमे॒धेना᳚ श्वमे॒धेन॒ यो यो᳚ ऽश्वमे॒धेन॒ यज॑ते॒ यज॑ते ऽश्वमे॒धेन॒ यो यो᳚ ऽश्वमे॒धेन॒ यज॑ते । \newline
6. अ॒श्व॒मे॒धेन॒ यज॑ते॒ यज॑ते ऽश्वमे॒धेना᳚ श्वमे॒धेन॒ यज॑ते॒ यो यो यज॑ते ऽश्वमे॒धेना᳚ श्वमे॒धेन॒ यज॑ते॒ यः । \newline
7. अ॒श्व॒मे॒धेनेत्य॑श्व - मे॒धेन॑ । \newline
8. यज॑ते॒ यो यो यज॑ते॒ यज॑ते॒ य उ॑ वु॒ यो यज॑ते॒ यज॑ते॒ य उ॑ । \newline
9. य उ॑ वु॒ यो य उ॑ च चो॒ यो य उ॑ च । \newline
10. उ॒ च॒ च॒ वु॒ चै॒न॒ मे॒न॒म् च॒ वु॒ चै॒न॒म् । \newline
11. चै॒न॒ मे॒न॒म् च॒ चै॒न॒ मे॒व मे॒व मे॑नम् च चैन मे॒वम् । \newline
12. ए॒न॒ मे॒व मे॒व मे॑न मेन मे॒वं ॅवेद॒ वेदै॒व मे॑न मेन मे॒वं ॅवेद॑ । \newline
13. ए॒वं ॅवेद॒ वेदै॒व मे॒वं ॅवेदोत्त॑र॒ मुत्त॑रं॒ ॅवेदै॒व मे॒वं ॅवेदोत्त॑रम् । \newline
14. वेदोत्त॑र॒ मुत्त॑रं॒ ॅवेद॒ वेदोत्त॑रं॒ ॅवै वा उत्त॑रं॒ ॅवेद॒ वेदोत्त॑रं॒ ॅवै । \newline
15. उत्त॑रं॒ ॅवै वा उत्त॑र॒ मुत्त॑रं॒ ॅवै तत् तद् वा उत्त॑र॒ मुत्त॑रं॒ ॅवै तत् । \newline
16. उत्त॑र॒मित्युत् - त॒र॒म् । \newline
17. वै तत् तद् वै वै तत् प्र॒जाप॑तेः प्र॒जाप॑ते॒ स्तद् वै वै तत् प्र॒जाप॑तेः । \newline
18. तत् प्र॒जाप॑तेः प्र॒जाप॑ते॒ स्तत् तत् प्र॒जाप॑ते॒ रक्ष्यक्षि॑ प्र॒जाप॑ते॒ स्तत् तत् प्र॒जाप॑ते॒ रक्षि॑ । \newline
19. प्र॒जाप॑ते॒ रक्ष्यक्षि॑ प्र॒जाप॑तेः प्र॒जाप॑ते॒ रक्ष्य॑ श्वय दश्वय॒ दक्षि॑ प्र॒जाप॑तेः प्र॒जाप॑ते॒ रक्ष्य॑ श्वयत् । \newline
20. प्र॒जाप॑ते॒रिति॑ प्र॒जा - प॒तेः॒ । \newline
21. अक्ष्य॑ श्वय दश्वय॒ दक्ष्य क्ष्य॑ श्वय॒त् तस्मा॒त् तस्मा॑ दश्वय॒ दक्ष्य क्ष्य॑ श्वय॒त् तस्मा᳚त् । \newline
22. अ॒श्व॒य॒त् तस्मा॒त् तस्मा॑ दश्वय दश्वय॒त् तस्मा॒ दश्व॒स्या श्व॑स्य॒ तस्मा॑ दश्वय दश्वय॒त् तस्मा॒ दश्व॑स्य । \newline
23. तस्मा॒ दश्व॒स्या श्व॑स्य॒ तस्मा॒त् तस्मा॒ दश्व॑स्योत्तर॒त उ॑त्तर॒तो ऽश्व॑स्य॒ तस्मा॒त् तस्मा॒ दश्व॑स्यो त्तर॒तः । \newline
24. अश्व॑स्योत्तर॒त उ॑त्तर॒तो ऽश्व॒स्या श्व॑स्यो त्तर॒तो ऽवावो᳚त्तर॒तो ऽश्व॒स्या श्व॑स्योत्तर॒तो ऽव॑ । \newline
25. उ॒त्त॒र॒तो ऽवावो᳚ त्तर॒त उ॑त्तर॒तो ऽव॑ द्यन्ति द्य॒न्त्यवो ᳚त्तर॒त उ॑त्तर॒तो ऽव॑ द्यन्ति । \newline
26. उ॒त्त॒र॒त इत्यु॑त् - त॒र॒तः । \newline
27. अव॑ द्यन्ति द्य॒न्त्यवाव॑ द्यन्ति दक्षिण॒तो द॑क्षिण॒तो द्य॒न्त्यवाव॑ द्यन्ति दक्षिण॒तः । \newline
28. द्य॒न्ति॒ द॒क्षि॒ण॒तो द॑क्षिण॒तो द्य॑न्ति द्यन्ति दक्षिण॒तो᳚ ऽन्येषा॑ म॒न्येषा᳚म् दक्षिण॒तो द्य॑न्ति द्यन्ति दक्षिण॒तो᳚ ऽन्येषा᳚म् । \newline
29. द॒क्षि॒ण॒तो᳚ ऽन्येषा॑ म॒न्येषा᳚म् दक्षिण॒तो द॑क्षिण॒तो᳚ ऽन्येषा᳚म् पशू॒नाम् प॑शू॒ना म॒न्येषा᳚म् दक्षिण॒तो द॑क्षिण॒तो᳚ ऽन्येषा᳚म् पशू॒नाम् । \newline
30. अ॒न्येषा᳚म् पशू॒नाम् प॑शू॒ना म॒न्येषा॑ म॒न्येषा᳚म् पशू॒नां ॅवै॑त॒सो वै॑त॒सः प॑शू॒ना म॒न्येषा॑ म॒न्येषा᳚म् पशू॒नां ॅवै॑त॒सः । \newline
31. प॒शू॒नां ॅवै॑त॒सो वै॑त॒सः प॑शू॒नाम् प॑शू॒नां ॅवै॑त॒सः कटः॒ कटो॑ वैत॒सः प॑शू॒नाम् प॑शू॒नां ॅवै॑त॒सः कटः॑ । \newline
32. वै॒त॒सः कटः॒ कटो॑ वैत॒सो वै॑त॒सः कटो॑ भवति भवति॒ कटो॑ वैत॒सो वै॑त॒सः कटो॑ भवति । \newline
33. कटो॑ भवति भवति॒ कटः॒ कटो॑ भव त्य॒फ्सुयो॑नि र॒फ्सुयो॑निर् भवति॒ कटः॒ कटो॑ भव त्य॒फ्सुयो॑निः । \newline
34. भ॒व॒त्य॒ फ्सुयो॑नि र॒फ्सुयो॑निर् भवति भव त्य॒फ्सुयो॑नि॒र् वै वा अ॒फ्सुयो॑निर् भवति भव त्य॒फ्सुयो॑नि॒र् वै । \newline
35. अ॒फ्सुयो॑नि॒र् वै वा अ॒फ्सुयो॑नि र॒फ्सुयो॑नि॒र् वा अश्वो ऽश्वो॒ वा अ॒फ्सुयो॑नि र॒फ्सुयो॑नि॒र् वा अश्वः॑ । \newline
36. अ॒फ्सुयो॑नि॒रित्य॒फ्सु - यो॒निः॒ । \newline
37. वा अश्वो ऽश्वो॒ वै वा अश्वो᳚ ऽफ्सु॒जो᳚ ऽफ्सु॒जो ऽश्वो॒ वै वा अश्वो᳚ ऽफ्सु॒जः । \newline
38. अश्वो᳚ ऽफ्सु॒जो᳚ ऽफ्सु॒जो ऽश्वो ऽश्वो᳚ ऽफ्सु॒जो वे॑त॒सो वे॑त॒सो᳚ ऽफ्सु॒जो ऽश्वो ऽश्वो᳚ ऽफ्सु॒जो वे॑त॒सः । \newline
39. अ॒फ्सु॒जो वे॑त॒सो वे॑त॒सो᳚ ऽफ्सु॒जो᳚ ऽफ्सु॒जो वे॑त॒सः स्वे स्वे वे॑त॒सो᳚ ऽफ्सु॒जो᳚ ऽफ्सु॒जो वे॑त॒सः स्वे । \newline
40. अ॒फ्सु॒ज इत्य॑फ्सु - जः । \newline
41. वे॒त॒सः स्वे स्वे वे॑त॒सो वे॑त॒सः स्व ए॒वैव स्वे वे॑त॒सो वे॑त॒सः स्व ए॒व । \newline
42. स्व ए॒वैव स्वे स्व ए॒वैन॑ मेन मे॒व स्वे स्व ए॒वैन᳚म् । \newline
43. ए॒वैन॑ मेन मे॒वै वैनं॒ ॅयोनौ॒ योना॑ वेन मे॒वै वैनं॒ ॅयोनौ᳚ । \newline
44. ए॒नं॒ ॅयोनौ॒ योना॑ वेन मेनं॒ ॅयोनौ॒ प्रति॒ प्रति॒ योना॑ वेन मेनं॒ ॅयोनौ॒ प्रति॑ । \newline
45. योनौ॒ प्रति॒ प्रति॒ योनौ॒ योनौ॒ प्रति॑ ष्ठापयति स्थापयति॒ प्रति॒ योनौ॒ योनौ॒ प्रति॑ ष्ठापयति । \newline
46. प्रति॑ ष्ठापयति स्थापयति॒ प्रति॒ प्रति॑ ष्ठापयति चतुष्टो॒म श्च॑तुष्टो॒मः स्था॑पयति॒ प्रति॒ प्रति॑ ष्ठापयति चतुष्टो॒मः । \newline
47. स्था॒प॒य॒ति॒ च॒तु॒ष्टो॒म श्च॑तुष्टो॒मः स्था॑पयति स्थापयति चतुष्टो॒मः स्तोमः॒ स्तोम॑ श्चतुष्टो॒मः स्था॑पयति स्थापयति चतुष्टो॒मः स्तोमः॑ । \newline
48. च॒तु॒ष्टो॒मः स्तोमः॒ स्तोम॑ श्चतुष्टो॒म श्च॑तुष्टो॒मः स्तोमो॑ भवति भवति॒ स्तोम॑ श्चतुष्टो॒म श्च॑तुष्टो॒मः स्तोमो॑ भवति । \newline
49. च॒तु॒ष्टो॒म इति॑ चतुः - स्तो॒मः । \newline
50. स्तोमो॑ भवति भवति॒ स्तोमः॒ स्तोमो॑ भवति स॒रट् थ्स॒रड् भ॑वति॒ स्तोमः॒ स्तोमो॑ भवति स॒रट् । \newline
51. भ॒व॒ति॒ स॒रट् थ्स॒रड् भ॑वति भवति स॒रड्ढ॑ ह स॒रड् भ॑वति भवति स॒रड्ढ॑ । \newline
52. स॒रड्ढ॑ ह स॒रट् थ्स॒रड्ढ॒ वै वै ह॑ स॒रट् थ्स॒रड्ढ॒ वै । \newline
53. ह॒ वै वै ह॑ ह॒ वा अश्व॒स्या श्व॑स्य॒ वै ह॑ ह॒ वा अश्व॑स्य । \newline
54. वा अश्व॒स्या श्व॑स्य॒ वै वा अश्व॑स्य॒ सक्थि॒ सक्थ्य श्व॑स्य॒ वै वा अश्व॑स्य॒ सक्थि॑ । \newline
55. अश्व॑स्य॒ सक्थि॒ सक्थ्य श्व॒स्या श्व॑स्य॒ सक्थ्या सक्थ्य श्व॒स्या श्व॑स्य॒ सक्थ्या । \newline
56. सक्थ्या सक्थि॒ सक्थ्या ऽवृ॑ह दवृह॒दा सक्थि॒ सक्थ्या ऽवृ॑हत् । \newline
57. आ ऽवृ॑ह दवृह॒दा ऽवृ॑ह॒त् तत् तद॑वृह॒दा ऽवृ॑ह॒त् तत् । \newline
58. अ॒वृ॒ह॒त् तत् तद॑वृह दवृह॒त् तद् दे॒वा दे॒वा स्तद॑वृह दवृह॒त् तद् दे॒वाः । \newline
59. तद् दे॒वा दे॒वा स्तत् तद् दे॒वा श्च॑तुष्टो॒मेन॑ चतुष्टो॒मेन॑ दे॒वा स्तत् तद् दे॒वा श्च॑तुष्टो॒मेन॑ । \newline
60. दे॒वा श्च॑तुष्टो॒मेन॑ चतुष्टो॒मेन॑ दे॒वा दे॒वा श्च॑तुष्टो॒मे नै॒वैव च॑तुष्टो॒मेन॑ दे॒वा दे॒वा श्च॑तुष्टो॒ मेनै॒व । \newline
61. च॒तु॒ष्टो॒मे नै॒वैव च॑तुष्टो॒मेन॑ चतुष्टो॒मे नै॒व प्रति॒ प्रत्ये॒व च॑तुष्टो॒मेन॑ चतुष्टो॒मे नै॒व प्रति॑ । \newline
62. च॒तु॒ष्टो॒मेनेति॑ चतुः - स्तो॒मेन॑ । \newline
63. ए॒व प्रति॒ प्रत्ये॒वैव प्रत्य॑दधु रदधुः॒ प्रत्ये॒वैव प्रत्य॑दधुः । \newline
64. प्रत्य॑दधु रदधुः॒ प्रति॒ प्रत्य॑दधु॒र् यद् यद॑दधुः॒ प्रति॒ प्रत्य॑दधु॒र् यत् । \newline
65. अ॒द॒धु॒र् यद् यद॑दधु रदधु॒र् यच् च॑तुष्टो॒म श्च॑तुष्टो॒मो यद॑दधु रदधु॒र् यच् च॑तुष्टो॒मः । \newline
66. यच् च॑तुष्टो॒म श्च॑तुष्टो॒मो यद् यच् च॑तुष्टो॒मः स्तोमः॒ स्तोम॑ श्चतुष्टो॒मो यद् यच् च॑तुष्टो॒मः स्तोमः॑ । \newline
67. च॒तु॒ष्टो॒मः स्तोमः॒ स्तोम॑ श्चतुष्टो॒म श्च॑तुष्टो॒मः स्तोमो॒ भव॑ति॒ भव॑ति॒ स्तोम॑ श्चतुष्टो॒म श्च॑तुष्टो॒मः स्तोमो॒ भव॑ति । \newline
68. च॒तु॒ष्टो॒म इति॑ चतुः - स्तो॒मः । \newline
69. स्तोमो॒ भव॑ति॒ भव॑ति॒ स्तोमः॒ स्तोमो॒ भव॒ त्यश्व॒स्या श्व॑स्य॒ भव॑ति॒ स्तोमः॒ स्तोमो॒ भव॒ त्यश्व॑स्य । \newline
70. भव॒ त्यश्व॒स्या श्व॑स्य॒ भव॑ति॒ भव॒ त्यश्व॑स्य सर्व॒त्वाय॑ सर्व॒त्वाया श्व॑स्य॒ भव॑ति॒ भव॒ त्यश्व॑स्य सर्व॒त्वाय॑ । \newline
71. अश्व॑स्य सर्व॒त्वाय॑ सर्व॒त्वाया श्व॒स्याश्व॑स्य सर्व॒त्वाय॑ । \newline
72. स॒र्व॒त्वायेति॑ सर्व - त्वाय॑ । \newline
\pagebreak


\end{document}