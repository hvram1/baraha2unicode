\documentclass[17pt]{extarticle}
\usepackage{babel}
\usepackage{fontspec}
\usepackage{polyglossia}
\usepackage{extsizes}



\setmainlanguage{sanskrit}
\setotherlanguages{english} %% or other languages
\setlength{\parindent}{0pt}
\pagestyle{myheadings}
\newfontfamily\devanagarifont[Script=Devanagari]{AdishilaVedic}


\newcommand{\VAR}[1]{}
\newcommand{\BLOCK}[1]{}




\begin{document}
\begin{titlepage}
    \begin{center}
 
\begin{sanskrit}
    { \Huge
    कृष्ण यजुर्वेदीय तैत्तिरीय संहिता,पद,जटा,घन पाठः 
    }
    \\
    \vspace{2.5cm}
    \mbox{ \Huge
    5.3      पञ्चमकाण्डे तृतीयः प्रश्नः - चितीनां निरूपणं   }
\end{sanskrit}
\end{center}

\end{titlepage}
\tableofcontents
\pagebreak

\markright{ TS 5.3.1.1  \hfill https://www.vedavms.in \hfill}
\addcontentsline{toc}{section}{ TS 5.3.1.1 }
\section*{ TS 5.3.1.1 }

\textbf{TS 5.3.1.1 } \newline
\textbf{Samhita Paata} \newline

उ॒थ्स॒न्न॒ य॒ज्ञो वा ए॒ष यद॒ग्निः किं ॅवाऽहै॒तस्य॑ क्रि॒यते॒ किं ॅवा॒ न यद्वै य॒ज्ञ्स्य॑ क्रि॒यमा॑णस्या-न्त॒र्यन्ति॒ पूय॑ति॒ वा अ॑स्य॒ तदा᳚श्वि॒नीरुप॑ दधात्य॒श्विनौ॒ वै दे॒वानां᳚ भि॒षजौ॒ ताभ्या॑मे॒वास्मै॑ भेष॒जं क॑रोति॒ पञ्चोप॑ दधाति॒ पाङ्क्तो॑ य॒ज्ञो यावा॑ने॒व य॒ज्ञ्स्तस्मै॑ भेष॒जं क॑रोत्यृत॒व्या॑ उप॑ दधात्यृतू॒नां क्लृप्त्यै॒ - [  ] \newline

\textbf{Pada Paata} \newline

उ॒थ्स॒न्न॒य॒ज्ञ् इत्यु॑थ्सन्न - य॒ज्ञ्ः । वै । ए॒षः । यत् । अ॒ग्निः । किम् । वा॒ । अह॑ । ए॒तस्य॑ । क्रि॒यते᳚ । किम् । वा॒ । न । यत् । वै । य॒ज्ञ्स्य॑ । क्रि॒यमा॑णस्य । अ॒न्त॒र्यन्तीत्य॑न्तः - यन्ति॑ । पूय॑ति । वै । अ॒स्य॒ । तत् । आ॒श्वि॒नीः । उपेति॑ । द॒धा॒ति॒ । अ॒श्विनौ᳚ । वै । दे॒वाना᳚म् । भि॒षजौ᳚ । ताभ्या᳚म् । ए॒व । अ॒स्मै॒ । भे॒ष॒जम् । क॒रो॒ति॒ । पञ्च॑ । उपेति॑ । द॒धा॒ति॒ । पाङ्क्तः॑ । य॒ज्ञ्ः । यावान्॑ । ए॒व । य॒ज्ञ्ः । तस्मै᳚ । भे॒ष॒जम् । क॒रो॒ति॒ । ऋ॒त॒व्याः᳚ । उपेति॑ । द॒धा॒ति॒ । ऋ॒तू॒नाम् । क्लृप्त्यै᳚ ।  \newline




\markright{ TS 5.3.1.2  \hfill https://www.vedavms.in \hfill}
\addcontentsline{toc}{section}{ TS 5.3.1.2 }
\section*{ TS 5.3.1.2 }

\textbf{TS 5.3.1.2 } \newline
\textbf{Samhita Paata} \newline

पञ्चोप॑ दधाति॒ पञ्च॒ वा ऋ॒तवो॒ याव॑न्त ए॒वर्तव॒स्तान् क॑ल्पयति समा॒नप्र॑भृतयो भवन्ति समा॒नोद॑र्का॒स्तस्मा᳚थ् समा॒ना ऋ॒तव॒ एके॑न प॒देन॒ व्याव॑र्तन्ते॒ तस्मा॑द्-ऋ॒तवो॒ व्याव॑र्तन्ते प्राण॒भृत॒ उप॑ दधात्यृ॒तुष्वे॒व प्रा॒णान् द॑धाति॒ तस्मा᳚थ् समा॒नाः सन्त॑ ऋ॒तवो॒ न जी᳚र्यन्त्यथो॒ प्रज॑नयत्ये॒वैना॑ने॒ष वै वा॒युर्यत् प्रा॒णो यद्-ऋ॑त॒व्या॑ उप॒धाय॑ प्राण॒भृत॑ - [  ] \newline

\textbf{Pada Paata} \newline

पञ्च॑ । उपेति॑ । द॒धा॒ति॒ । पञ्च॑ । वै । ऋ॒तवः॑ । याव॑न्तः । ए॒व । ऋ॒तवः॑ । तान् । क॒ल्प॒य॒ति॒ । स॒मा॒नप्र॑भृतय॒ इति॑ समा॒न - प्र॒भृ॒त॒यः॒ । भ॒व॒न्ति॒ । स॒मा॒नोद॑र्का॒ इति॑ समा॒न - उ॒द॒र्काः॒ । तस्मा᳚त् । स॒मा॒नाः । ऋ॒तवः॑ । एके॑न । प॒देन॑ । व्याव॑र्तन्त॒ इति॑ वि - आव॑र्तन्ते । तस्मा᳚त् । ऋ॒तवः॑ । व्याव॑र्तन्त॒ इति॑ वि-आव॑र्तन्ते । प्रा॒ण॒भृत॒ इति॑ प्राण-भृतः॑ । उपेति॑ । द॒धा॒ति॒ । ऋ॒तुषु॑ । ए॒व । प्रा॒णानिति॑ प्र - अ॒नान् । द॒धा॒ति॒ । तस्मा᳚त् । स॒मा॒नाः । सन्तः॑ । ऋ॒तवः॑ । न । जी॒र्य॒न्ति॒ । अथो॒ इति॑ । प्रेति॑ । ज॒न॒य॒ति॒ । ए॒व । ए॒ना॒न् । ए॒षः । वै । वा॒युः । यत् । प्रा॒ण इति॑ प्र - अ॒नः । यत् । ऋ॒त॒व्याः᳚ । उ॒प॒धायेत्यु॑प - धाय॑ । प्रा॒ण॒भृत॒ इति॑ प्राण - भृतः॑ ।  \newline




\markright{ TS 5.3.1.3  \hfill https://www.vedavms.in \hfill}
\addcontentsline{toc}{section}{ TS 5.3.1.3 }
\section*{ TS 5.3.1.3 }

\textbf{TS 5.3.1.3 } \newline
\textbf{Samhita Paata} \newline

उप॒दधा॑ति॒ तस्मा॒थ् सर्वा॑नृ॒तूननु॑ वा॒युरा व॑रीवर्त्ति वृष्टि॒सनी॒रुप॑ दधाति॒ वृष्टि॑मे॒वाव॑ रुन्धे॒ यदे॑क॒धोप॑द॒द्ध्या-देक॑मृ॒तुं ॅव॑र्.षेदनुपरि॒हारꣳ॑ सादयति॒ तस्मा॒थ् सर्वा॑नृ॒तून्. व॑र्.षति॒ यत् प्रा॑ण॒भृत॑ उप॒धाय॑ वृष्टि॒सनी॑रुप॒दधा॑ति॒ तस्मा᳚द्-वा॒युप्र॑च्युता दि॒वो वृष्टि॑रीर्ते प॒शवो॒ वै व॑य॒स्या॑ नाना॑मनसः॒ खलु॒ वै प॒शवो॒ नाना᳚व्रता॒स्ते॑ऽप ए॒वाभि सम॑नसो॒ - [  ] \newline

\textbf{Pada Paata} \newline

उ॒प॒दधा॒तीत्यु॑प - दधा॑ति । तस्मा᳚त् । सर्वान्॑ । ऋ॒तून् । अन्विति॑ । वा॒युः । एति॑ । व॒री॒व॒र्ति॒ । वृ॒ष्टि॒सनी॒रिति॑ वृष्टि - सनीः᳚ । उपेति॑ । द॒धा॒ति॒ । वृष्टि᳚म् । ए॒व । अवेति॑ । रु॒न्धे॒ । यत् । ए॒क॒धेत्ये॑क - धा । उ॒प॒द॒द्ध्यादित्यु॑प - द॒द्ध्यात् । एक᳚म् । ऋ॒तुम् । व॒र्.॒षे॒त् । अ॒नु॒प॒रि॒हार॒मित्य॑नु-प॒रि॒हार᳚म् । सा॒द॒य॒ति॒ । तस्मा᳚त् । सर्वान्॑ । ऋ॒तून् । व॒र्.॒ष॒ति॒ । यत् । प्रा॒ण॒भृत॒ इति॑ प्राण - भृतः॑ । उ॒प॒धायेत्यु॑प - धाय॑ । वृ॒ष्टि॒सनी॒रिति॑ वृष्टि - सनीः᳚ । उ॒प॒दधा॒तीत्यु॑प - दधा॑ति । तस्मा᳚त् । वा॒युप्र॑च्यु॒तेति॑ वा॒यु - प्र॒च्यु॒ता॒ । दि॒वः । वृष्टिः॑ । ई॒र्ते॒ । प॒शवः॑ । वै । व॒य॒स्याः᳚ । नाना॑मनस॒ इति॒ नाना᳚ - म॒न॒सः॒ । खलु॑ । वै । प॒शवः॑ । नाना᳚व्रता॒ इति॒ नाना᳚ - व्र॒ताः॒ । ते । अ॒पः । ए॒व । अ॒भीति॑ । सम॑नस॒ इति॒ स - म॒न॒सः॒ ।  \newline




\markright{ TS 5.3.1.4  \hfill https://www.vedavms.in \hfill}
\addcontentsline{toc}{section}{ TS 5.3.1.4 }
\section*{ TS 5.3.1.4 }

\textbf{TS 5.3.1.4 } \newline
\textbf{Samhita Paata} \newline

यं का॒मये॑ताप॒शुः स्या॒दिति॑ वय॒स्या᳚स्तस्यो॑-प॒धाया॑प॒स्या॑ उप॑ दद्ध्या॒द स᳚ज्ञांनमे॒वास्मै॑ प॒शुभिः॑ करोत्यप॒शुरे॒व भ॑वति॒ यं का॒मये॑त पशु॒मान्थ्-स्या॒दित्य॑-प॒स्या᳚स्तस्यो॑प॒धाय॑ वय॒स्या॑ उप॑ दद्ध्याथ् स॒ज्ञांन॑मे॒वास्मै॑ प॒शुभिः॑ करोति पशु॒माने॒व भ॑वति॒ चत॑स्रः पु॒रस्ता॒दुप॑ दधाति॒ तस्मा᳚च्च॒त्वारि॒ चक्षु॑षो रू॒पाणि॒ द्वे शु॒क्ले द्वे कृ॒ष्णे - [  ] \newline

\textbf{Pada Paata} \newline

यम् । का॒मये॑त । अ॒प॒शुः । स्या॒त् । इति॑ । व॒य॒स्याः᳚ । तस्य॑ । उ॒प॒धायेत्यु॑प - धाय॑ । अ॒प॒स्याः᳚ । उपेति॑ । द॒द्ध्या॒त् । अस᳚ज्ञांन॒मित्यसं᳚ - ज्ञा॒न॒म् । ए॒व । अ॒स्मै॒ । प॒शुभि॒रिति॑ प॒शु-भिः॒ । क॒रो॒ति॒ । अ॒प॒शुः । ए॒व । भ॒व॒ति॒ । यम् । का॒मये॑त । प॒शु॒मानिति॑ पशु - मान् । स्या॒त् । इति॑ । अ॒प॒स्याः᳚ । तस्य॑ । उ॒प॒धायेत्यु॑प - धाय॑ । व॒य॒स्याः᳚ । उपेति॑ । द॒द्ध्या॒त् । स॒ज्ञांन॒मिति॑ सं - ज्ञान᳚म् । ए॒व । अ॒स्मै॒ । प॒शुभि॒रिति॑ प॒शु - भिः॒ । क॒रो॒ति॒ । प॒शु॒मानिति॑ पशु - मान् । ए॒व । भ॒व॒ति॒ । चत॑स्रः । पु॒रस्ता᳚त् । उपेति॑ । द॒धा॒ति॒ । तस्मा᳚त् । च॒त्वारि॑ । चक्षु॑षः । रू॒पाणि॑ । द्वे इति॑ । शु॒क्ले इति॑ । द्वे इति॑ । कृ॒ष्णे इति॑ ।  \newline




\markright{ TS 5.3.1.5  \hfill https://www.vedavms.in \hfill}
\addcontentsline{toc}{section}{ TS 5.3.1.5 }
\section*{ TS 5.3.1.5 }

\textbf{TS 5.3.1.5 } \newline
\textbf{Samhita Paata} \newline

मू᳚र्द्ध॒न्वती᳚र्भवन्ति॒ तस्मा᳚त् पु॒रस्ता᳚न्मू॒र्द्धा पञ्च॒ दक्षि॑णायाꣳ॒॒ श्रोण्या॒मुप॑ दधाति॒ पञ्चोत्त॑रस्यां॒ तस्मा᳚त् प॒श्चाद्-वर्.षी॑यान् पु॒रस्ता᳚त् प्रवणः प॒शुर्ब॒स्तो वय॒ इति॒ दक्षि॒णेऽꣳस॒ उप॑ दधाति वृ॒ष्णिर्वय॒ इत्युत्त॒रे ऽꣳसा॑वे॒व प्रति॑ दधाति व्या॒घ्रो वय॒ इति॒ दक्षि॑णे प॒क्ष उप॑ दधाति सिꣳ॒॒हो वय॒ इत्युत्त॑रे प॒क्षयो॑रे॒व वी॒र्यं॑ दधाति॒ पुरु॑षो॒ वय॒ इति॒ ( ) मद्ध्ये॒ तस्मा॒त् पुरु॑षः पशू॒नामधि॑पतिः ॥ \newline

\textbf{Pada Paata} \newline

मू॒द्‌र्ध॒न्वती॒रिति॑ मूद्‌र्धन्न्-वतीः᳚ । भ॒व॒न्ति॒ । तस्मा᳚त् । पु॒रस्ता᳚त् । मू॒द्‌र्धा । पञ्च॑ । दक्षि॑णायाम् । श्रोण्या᳚म् । उपेति॑ । द॒धा॒ति॒ । पञ्च॑ । उत्त॑रस्या॒मित्युत्-त॒र॒स्या॒म् । तस्मा᳚त् । प॒श्चात् । वर्.षी॑यान् । पु॒रस्ता᳚त् प्रवण॒ इति॑ पु॒रस्ता᳚त् - प्र॒व॒णः॒ । प॒शुः । ब॒स्तः । वयः॑ । इति॑ । दक्षि॑णे । अꣳसे᳚ । उपेति॑ । द॒धा॒ति॒ । वृ॒ष्णिः । वयः॑ । इति॑ । उत्त॑र॒ इत्युत् - त॒रे॒ । अꣳसौ᳚ । ए॒व । प्रतीति॑ । द॒धा॒ति॒ । व्या॒घ्रः । वयः॑ । इति॑ । दक्षि॑णे । प॒क्षे । उपेति॑ । द॒धा॒ति॒ । सिꣳ॒॒हः । वयः॑ । इति॑ । उत्त॑र॒ इत्युत् - त॒रे॒ । प॒क्षयोः᳚ । ए॒व । वी॒र्य᳚म् । द॒धा॒ति॒ । पुरु॑षः । वयः॑ । इति॑ ( ) । मद्ध्ये᳚ । तस्मा᳚त् । पुरु॑षः । प॒शू॒नाम् । अधि॑पति॒रित्यधि॑ - प॒तिः॒ ॥  \newline




\markright{ TS 5.3.2.1  \hfill https://www.vedavms.in \hfill}
\addcontentsline{toc}{section}{ TS 5.3.2.1 }
\section*{ TS 5.3.2.1 }

\textbf{TS 5.3.2.1 } \newline
\textbf{Samhita Paata} \newline

इन्द्रा᳚ग्नी॒ अव्य॑थमाना॒मिति॑ स्वयमातृ॒ण्णामुप॑ दधातीन्द्रा॒ग्निभ्यां॒ ॅवा इ॒मौ लो॒कौ विधृ॑ताव॒नयो᳚-र्लो॒कयो॒-र्विधृ॑त्या॒ अधृ॑तेव॒ वा ए॒षा यन्म॑द्ध्य॒मा चिति॑र॒न्तरि॑क्षमिव॒ वा ए॒षेन्द्रा᳚ग्नी॒ इत्या॑हेन्द्रा॒ग्नी वै दे॒वाना॑मोजो॒ भृता॒वोज॑सै॒वैना॑-म॒न्तरि॑क्षे चिनुते॒ धृत्यै᳚ स्वयमातृ॒ण्णामुप॑ दधात्य॒न्तरि॑क्षं॒ ॅवै स्व॑यमातृ॒ण्णा ऽन्तरि॑क्षमे॒वोप॑ ध॒त्ते ऽश्व॒मुप॑ - [  ] \newline

\textbf{Pada Paata} \newline

इन्द्रा᳚ग्नी॒ इतीन्द्र॑ - अ॒ग्नी॒ । अव्य॑थमानाम् । इति॑ । स्व॒य॒मा॒तृ॒ण्णामिति॑ स्वयं - आ॒तृ॒ण्णाम् । उपेति॑ । द॒धा॒ति॒ । इ॒न्द्रा॒ग्निभ्या॒मिती᳚न्द्रा॒ग्नि - भ्या॒म् । वै । इ॒मौ । लो॒कौ । विधृ॑ता॒विति॒ वि - धृ॒तौ॒ । अ॒नयोः᳚ । लो॒कयोः᳚ । विधृ॑त्या॒ इति॒ वि - धृ॒त्यै॒ । अधृ॑ता । इ॒व॒ । वै । ए॒षा । यत् । म॒द्ध्य॒मा । चितिः॑ । अ॒न्तरि॑क्षम् । इ॒व॒ । वै । ए॒षा । इन्द्रा᳚ग्नी॒ इतीन्द्र॑ - अ॒ग्नी॒ । इति॑ । आ॒ह॒ । इ॒न्द्रा॒ग्नी इती᳚न्द्र - अ॒ग्नी । वै । दे॒वाना᳚म् । ओ॒जो॒भृता॒वित्यो॑जः - भृतौ᳚ । ओज॑सा । ए॒व । ए॒ना॒म् । अ॒न्तरि॑क्षे । चि॒नु॒ते॒ । धृत्यै᳚ । स्व॒य॒मा॒तृ॒ण्णामिति॑ स्वयं - आ॒तृ॒ण्णाम् । उपेति॑ । द॒धा॒ति॒ । अ॒न्तरि॑क्षम् । वै । स्व॒य॒मा॒तृ॒ण्णेति॑ स्वयं-आ॒तृ॒ण्णा । अ॒न्तरि॑क्षम् । ए॒व । उपेति॑ । ध॒त्ते॒ । अश्व᳚म् । उपेति॑ ।  \newline




\markright{ TS 5.3.2.2  \hfill https://www.vedavms.in \hfill}
\addcontentsline{toc}{section}{ TS 5.3.2.2 }
\section*{ TS 5.3.2.2 }

\textbf{TS 5.3.2.2 } \newline
\textbf{Samhita Paata} \newline

घ्रापयति प्रा॒णमे॒वास्यां᳚ दधा॒त्यथो᳚ प्राजाप॒त्यो वा अश्वः॑ प्र॒जाप॑तिनै॒वाग्निं चि॑नुते स्वयमातृ॒ण्णा भ॑वति प्रा॒णाना॒मुथ्सृ॑ष्ट्या॒ अथो॑ सुव॒र्गस्य॑ लो॒कस्यानु॑ख्यात्यै दे॒वानां॒ ॅवै सु॑व॒र्गं ॅलो॒कं ॅय॒तां दिशः॒ सम॑व्लीयन्त॒ त ए॒ता दिश्या॑ अपश्य॒न् ता उपा॑दधत॒ ताभि॒र्वै ते दिशो॑ऽदृꣳह॒न्॒ यद्दिश्या॑ उप॒दधा॑ति दि॒शां ॅविधृ॑त्यै॒ दश॑ प्राण॒भृतः॑ पु॒रस्ता॒दुप॑ - [  ] \newline

\textbf{Pada Paata} \newline

घ्रा॒प॒य॒ति॒ । प्रा॒णमिति॑ प्र - अ॒नम् । ए॒व । अ॒स्या॒म् । द॒धा॒ति॒ । अथो॒ इति॑ । प्रा॒जा॒प॒त्य इति॑ प्राजा - प॒त्यः । वै । अश्वः॑ । प्र॒जाप॑ति॒नेति॑ प्र॒जा - प॒ति॒ना॒ । ए॒व । अ॒ग्निम् । चि॒नु॒ते॒ । स्व॒य॒मा॒तृ॒ण्णेति॑ स्वयं - आ॒तृ॒ण्णा । भ॒व॒ति॒ । प्रा॒णाना॒मिति॑ प्र - अ॒नाना᳚म् । उथ्सृ॑ष्ट्य॒ इत्युत् - सृ॒ष्ट्यै॒ । अथो॒ इति॑ । सु॒व॒र्गस्येति॑ सुवः - गस्य॑ । लो॒कस्य॑ । अनु॑ख्यात्या॒ इत्यनु॑ - ख्या॒त्यै॒ । दे॒वाना᳚म् । वै । सु॒व॒र्गमिति॑ सुवः - गम् । लो॒कम् । य॒ताम् । दिशः॑ । समिति॑ । अ॒व्ली॒य॒न्त॒ । ते । ए॒ताः । दिश्याः᳚ । अ॒प॒श्य॒न्न् । ताः । उपेति॑ । अ॒द॒ध॒त॒ । ताभिः॑ । वै । ते । दिशः॑ । अ॒दृꣳ॒॒ह॒न्न् । यत् । दिश्याः᳚ । उ॒प॒दधा॒तीत्यु॑प - दधा॑ति । दि॒शाम् । विधृ॑त्या॒ इति॒ वि - धृ॒त्यै॒ । दश॑ । प्रा॒ण॒भृत॒ इति॑ प्राण - भृतः॑ । पु॒रस्ता᳚त् । उपेति॑ ।  \newline




\markright{ TS 5.3.2.3  \hfill https://www.vedavms.in \hfill}
\addcontentsline{toc}{section}{ TS 5.3.2.3 }
\section*{ TS 5.3.2.3 }

\textbf{TS 5.3.2.3 } \newline
\textbf{Samhita Paata} \newline

दधाति॒ नव॒ वै पुरु॑षे प्रा॒णा नाभि॑र्दश॒मी प्रा॒णाने॒व पु॒रस्ता᳚द्धत्ते॒ तस्मा᳚त् पु॒रस्ता᳚त् प्रा॒णा ज्योति॑ष्मती-मुत्त॒मामुप॑ दधाति॒ तस्मा᳚त् प्रा॒णानां॒ ॅवाग्ज्योति॑रुत्त॒मा दशोप॑ दधाति॒ दशा᳚क्षरा वि॒राड् वि॒राट् छन्द॑सां॒ ज्योति॒र्ज्योति॑रे॒व पु॒रस्ता᳚द्धत्ते॒ तस्मा᳚त् पु॒रस्ता॒ज्ज्योति॒रुपा᳚ ऽऽ*स्महे॒ छन्दाꣳ॑सि प॒शुष्वा॒जिम॑यु॒स्तान् बृ॑ह॒त्युद॑जय॒त् तस्मा॒द्-बार्.ह॑ताः - [  ] \newline

\textbf{Pada Paata} \newline

द॒धा॒ति॒ । नव॑ । वै । पुरु॑षे । प्रा॒णा इति॑ प्र - अ॒नाः । नाभिः॑ । द॒श॒मी । प्रा॒णानिति॑ प्र - अ॒नान् । ए॒व । पु॒रस्ता᳚त् । ध॒त्ते॒ । तस्मा᳚त् । पु॒रस्ता᳚त् । प्रा॒णा इति॑ प्र - अ॒नाः । ज्योति॑ष्मतीम् । उ॒त्त॒मामित्यु॑त् - त॒माम् । उपेति॑ । द॒धा॒ति॒ । तस्मा᳚त् । प्रा॒णाना॒मिति॑ प्र - अ॒नाना᳚म् । वाक् । ज्योतिः॑ । उ॒त्त॒मेत्यु॑त् - त॒मा । दश॑ । उपेति॑ । द॒धा॒ति॒ । दशा᳚क्ष॒रेति॒ दश॑ - अ॒क्ष॒रा॒ । वि॒राडिति॑ वि - राट् । वि॒राडिति॑ वि - राट् । छन्द॑साम् । ज्योतिः॑ । ज्योतिः॑ । ए॒व । पु॒रस्ता᳚त् । ध॒त्ते॒ । तस्मा᳚त् । पु॒रस्ता᳚त् । ज्योतिः॑ । उपेति॑ । आ॒स्म॒हे॒ । छन्दाꣳ॑सि । प॒शुषु॑ । आ॒जिम् । अ॒युः॒ । तान् । बृ॒ह॒ती । उदिति॑ । अ॒ज॒य॒त् । तस्मा᳚त् । बार्.ह॑ताः ।  \newline




\markright{ TS 5.3.2.4  \hfill https://www.vedavms.in \hfill}
\addcontentsline{toc}{section}{ TS 5.3.2.4 }
\section*{ TS 5.3.2.4 }

\textbf{TS 5.3.2.4 } \newline
\textbf{Samhita Paata} \newline

प॒शव॑ उच्यन्ते॒ मा छन्द॒ इति॑ दक्षिण॒त उप॑ दधाति॒ तस्मा᳚द्-दक्षि॒णा वृ॑तो॒ मासाः᳚ पृथि॒वी छन्द॒ इति॑ प॒श्चात् प्रति॑ष्ठित्या अ॒ग्निर्दे॒वतेत्यु॑त्तर॒त ओजो॒ वा अ॒ग्निरोज॑ ए॒वोत्त॑र॒तो ध॑त्ते॒ तस्मा॑दुत्तरतो ऽभिप्रया॒यी ज॑यति॒ षट्त्रिꣳ॑श॒थ् संप॑द्यन्ते॒ षट्त्रिꣳ॑शदक्षरा बृह॒ती बार्.ह॑ताः प॒शवो॑ बृह॒त्यैवास्मै॑ प॒शूनव॑ रुन्धे बृह॒ती छन्द॑साꣳ॒॒ स्वारा᳚ज्यं॒ परी॑याय॒ यस्यै॒ता - [  ] \newline

\textbf{Pada Paata} \newline

प॒शवः॑ । उ॒च्य॒न्ते॒ । मा । छन्दः॑ । इति॑ । द॒क्षि॒ण॒तः । उपेति॑ । द॒धा॒ति॒ । तस्मा᳚त् । द॒क्षि॒णावृ॑त॒ इति॑ दक्षि॒णा - आ॒वृ॒तः॒ । मासाः᳚ । पृ॒थि॒वी । छन्दः॑ । इति॑ । प॒श्चात् । प्रति॑ष्ठित्या॒ इति॒ प्रति॑ - स्थि॒त्यै॒ । अ॒ग्निः । दे॒वता᳚ । इति॑ । उ॒त्त॒र॒त इत्यु॑त् - त॒र॒तः । ओजः॑ । वै । अ॒ग्निः । ओजः॑ । ए॒व । उ॒त्त॒र॒त इत्यु॑त्-त॒र॒तः । ध॒त्ते॒ । तस्मा᳚त् । उ॒त्त॒र॒तो॒ऽभि॒प्र॒या॒यीत्यु॑त्तरतः - अ॒भि॒प्र॒या॒यी । ज॒य॒ति॒ । षट्त्रिꣳ॑श॒दिति॒ षट् - त्रिꣳ॒॒श॒त् । समिति॑ । प॒द्य॒न्ते॒ । षट्त्रिꣳ॑शदक्ष॒रेति॒ षट्त्रिꣳ॑शत् -   अ॒क्ष॒रा॒ । बृ॒ह॒ती । बार्.ह॑ताः । प॒शवः॑ । बृ॒ह॒त्या । ए॒व । अ॒स्मै॒ । प॒शून् । अवेति॑ । रु॒न्धे॒ । बृ॒ह॒ती । छन्द॑साम् । स्वारा᳚ज्य॒मिति॒ स्व - रा॒ज्य॒म् । परीति॑ । इ॒या॒य॒ । यस्य॑ । ए॒ताः ।  \newline




\markright{ TS 5.3.2.5  \hfill https://www.vedavms.in \hfill}
\addcontentsline{toc}{section}{ TS 5.3.2.5 }
\section*{ TS 5.3.2.5 }

\textbf{TS 5.3.2.5 } \newline
\textbf{Samhita Paata} \newline

उ॑पधी॒यन्ते॒ गच्छ॑ति॒ स्वारा᳚ज्यꣳ स॒प्त वाल॑खिल्याः पु॒रस्ता॒दुप॑ दधाति स॒प्त प॒श्चाथ् स॒प्त वै शी॑र्.ष॒ण्याः᳚ प्रा॒णा द्वाववा᳚ञ्चौ प्रा॒णानाꣳ॑ सवीर्य॒त्वाय॑ मू॒र्द्धाऽसि॒ राडिति॑ पु॒रस्ता॒दुप॑ दधाति॒ यन्त्री॒ राडिति॑ प॒श्चात् प्रा॒णाने॒वास्मै॑ स॒मीचो॑ दधाति ॥ \newline

\textbf{Pada Paata} \newline

उ॒प॒धी॒यन्त॒ इत्यु॑प-धी॒यन्ते᳚ । गच्छ॑ति । स्वारा᳚ज्य॒मिति॒ स्व-रा॒ज्य॒म् । स॒प्त । वाल॑खिल्या॒ इति॒ वाल॑ - खि॒ल्याः॒ । पु॒रस्ता᳚त् । उपेति॑ । द॒धा॒ति॒ । स॒प्त । प॒श्चात् । स॒प्त । वै । शी॒र्.॒ष॒ण्याः᳚ । प्रा॒णा इति॑ प्र - अ॒नाः । द्वौ । अवा᳚ञ्चौ । प्रा॒णाना॒मिति॑ प्र - अ॒नाना᳚म् । स॒वी॒र्य॒त्वायेति॑ सवीर्य - त्वाय॑ । मू॒द्‌र्धा । अ॒सि॒ । राट् । इति॑ । पु॒रस्ता᳚त् । उपेति॑ । द॒धा॒ति॒ । यन्त्री᳚ । राट् । इति॑ । प॒श्चात् । प्रा॒णानिति॑ प्र - अ॒नान् । ए॒व । अ॒स्मै॒ । स॒मीचः॑ । द॒धा॒ति॒ ॥  \newline




\markright{ TS 5.3.3.1  \hfill https://www.vedavms.in \hfill}
\addcontentsline{toc}{section}{ TS 5.3.3.1 }
\section*{ TS 5.3.3.1 }

\textbf{TS 5.3.3.1 } \newline
\textbf{Samhita Paata} \newline

दे॒वा वै यद्-य॒ज्ञे ऽकु॑र्वत॒ तदसु॑रा अकुर्वत॒ ते दे॒वा ए॒ता अ॑क्ष्णयास्तो॒मीया॑ अपश्य॒न् ता अ॒न्यथा॒ ऽनूच्या॒-न्यथोपा॑दधत॒ तदसु॑रा॒ नान्ववा॑य॒न् ततो॑ दे॒वा अभ॑व॒न् पराऽसु॑रा॒ यद॑क्ष्णयास्तो॒मीया॑ अ॒न्यथा॒ ऽनूच्या॒न्यथो॑प॒ दधा॑ति॒ भ्रातृ॑व्याभिभूत्यै॒ भव॑त्या॒त्मना॒ परा᳚ऽस्य॒ भ्रातृ॑व्यो भवत्या॒-शुस्त्रि॒वृदिति॑ पु॒रस्ता॒दुप॑ दधाति यज्ञ्मु॒खं ॅवै त्रि॒वृ - [  ] \newline

\textbf{Pada Paata} \newline

दे॒वाः । वै । यत् । य॒ज्ञे । अकु॑र्वत । तत् । असु॑राः । अ॒कु॒र्व॒त॒ । ते । दे॒वाः । ए॒ताः । अ॒क्ष्ण॒या॒स्तो॒मीया॒ इत्य॑क्ष्णया-स्तो॒मीयाः᳚ । अ॒प॒श्य॒न्न् । ताः । अ॒न्यथा᳚ । अ॒नूच्येत्य॑नु - उच्य॑ । अ॒न्यथा᳚ । उपेति॑ । अ॒द॒ध॒त॒ । तत् । असु॑राः । न । अ॒न्ववा॑य॒न्नित्य॑नु - अवा॑यन्न् । ततः॑ । दे॒वाः । अभ॑वन्न् । परेति॑ । असु॑राः । यत् । अ॒क्ष्ण॒या॒स्तो॒मीया॒ इत्य॑क्ष्णया - स्तो॒मीयाः᳚ । अ॒न्यथा᳚ । अ॒नूच्येत्य॑नु - उच्य॑ । अ॒न्यथा᳚ । उ॒प॒दधा॒तीत्यु॑प-दधा॑ति । भ्रातृ॑व्याभिभूत्या॒ इति॒ भ्रातृ॑व्य - अ॒भि॒भू॒त्यै॒ । भव॑ति । आ॒त्मना᳚ । परेति॑ । अ॒स्य॒ । भ्रातृ॑व्यः । भ॒व॒ति॒ । आ॒शुः । त्रि॒वृदिति॑ त्रि - वृत् । इति॑ । पु॒रस्ता᳚त् । उपेति॑ । द॒धा॒ति॒ । य॒ज्ञ्॒मु॒खमिति॑ यज्ञ् - मु॒खम् । वै । त्रि॒वृदिति॑ त्रि - वृत् ।  \newline




\markright{ TS 5.3.3.2  \hfill https://www.vedavms.in \hfill}
\addcontentsline{toc}{section}{ TS 5.3.3.2 }
\section*{ TS 5.3.3.2 }

\textbf{TS 5.3.3.2 } \newline
\textbf{Samhita Paata} \newline

-द्य॑ज्ञ्मु॒खमे॒व पु॒रस्ता॒द्वि या॑तयति॒ व्यो॑म सप्तद॒श इति॑ दक्षिण॒तो ऽन्नं॒ ॅवै व्यो॑माऽन्नꣳ॑ सप्तद॒शोऽन्न॑मे॒व द॑क्षिण॒तो ध॑त्ते॒ तस्मा॒द्-दक्षि॑णे॒नान्न॑मद्यते ध॒रुण॑ एकविꣳ॒॒श इति॑ प॒श्चात् प्र॑ति॒ष्ठा वा ए॑कविꣳ॒॒शः प्रति॑ष्ठित्यै भा॒न्तः प॑ञ्चद॒श इत्यु॑त्तर॒त ओजो॒ वै भा॒न्त ओजः॑ पञ्चद॒श ओज॑ ए॒वोत्त॑र॒तो ध॑त्ते॒ तस्मा॑दुत्तरतो ऽभिप्रया॒यी ज॑यति॒ प्रतू᳚र्तिरष्टाद॒श इति॑ पु॒रस्ता॒- [  ] \newline

\textbf{Pada Paata} \newline

य॒ज्ञ्॒मु॒खमिति॑ यज्ञ् - मु॒खम् । ए॒व । पु॒रस्ता᳚त् । वीति॑ । या॒त॒य॒ति॒ । व्यो॑मेति॒ वि - ओ॒म॒ । स॒प्त॒द॒श इति॑ सप्त - द॒शः । इति॑ । द॒क्षि॒ण॒तः । अन्न᳚म् । वै । व्यो॑मेति॒ वि - ओ॒म॒ । अन्न᳚म् । स॒प्त॒द॒श इति॑ सप्त - द॒शः । अन्न᳚म् । ए॒व । द॒क्षि॒ण॒तः । ध॒त्ते॒ । तस्मा᳚त् । दक्षि॑णेन । अन्न᳚म् । अ॒द्य॒ते॒ । ध॒रुणः॑ । ए॒क॒विꣳ॒॒श इत्ये॑क-विꣳ॒॒शः । इति॑ । प॒श्चात् । प्र॒ति॒ष्ठेति॑ प्रति - स्था । वै । ए॒क॒विꣳ॒॒श इत्ये॑क - विꣳ॒॒शः । प्रति॑ष्ठित्या॒ इति॒ प्रति॑ - स्थि॒त्यै॒ । भा॒न्तः । प॒ञ्च॒द॒श इति॑ पञ्च - द॒शः । इति॑ । उ॒त्त॒र॒त इत्यु॑त् - त॒र॒तः । ओजः॑ । वै । भा॒न्तः । ओजः॑ । प॒ञ्च॒द॒श इति॑ पञ्च - द॒शः । ओजः॑ । ए॒व । उ॒त्त॒र॒त इत्यु॑त् - त॒र॒तः । ध॒त्ते॒ । तस्मा᳚त् । उ॒त्त॒र॒तो॒ऽभि॒प्र॒या॒यीत्यु॑त्तरतः - अ॒भि॒प्र॒या॒यी । ज॒य॒ति॒ । प्रतू᳚र्ति॒रिति॒ प्र - तू॒र्तिः॒ । अ॒ष्टा॒द॒श इत्य॑ष्टा - द॒शः । इति॑ । पु॒रस्ता᳚त् ।  \newline




\markright{ TS 5.3.3.3  \hfill https://www.vedavms.in \hfill}
\addcontentsline{toc}{section}{ TS 5.3.3.3 }
\section*{ TS 5.3.3.3 }

\textbf{TS 5.3.3.3 } \newline
\textbf{Samhita Paata} \newline

-दुप॑ दधाति॒ द्वौ त्रि॒वृता॑वभिपू॒र्वं ॅय॑ज्ञ्मु॒खे वि या॑तयत्यभिव॒र्तः स॑विꣳ॒॒श इति॑ दक्षिण॒तोऽन्नं॒ ॅवा अ॑भिव॒र्तोऽन्नꣳ॑ सविꣳ॒॒शोऽन्न॑मे॒व द॑क्षिण॒तो ध॑त्ते॒ तस्मा॒द्-दक्षि॑णे॒नान्न॑मद्यते॒ वर्चो᳚ द्वाविꣳ॒॒श इति॑ प॒श्चाद्-यद्-विꣳ॑श॒तिर्द्वे तेन॑ वि॒राजौ॒ यद् द्वे प्र॑ति॒ष्ठा तेन॑ वि॒राजो॑रे॒वा-भि॑पू॒र्वम॒न्नाद्ये॒ प्रति॑तिष्ठति॒ तपो॑ नवद॒श इत्यु॑त्तर॒ तस्मा᳚थ् स॒व्यो - [  ] \newline

\textbf{Pada Paata} \newline

उपेति॑ । द॒धा॒ति॒ । द्वौ । त्रि॒वृता॒विति॑ त्रि - वृतौ᳚ । अ॒भि॒पू॒र्वमित्य॑भि - पू॒र्वम् । य॒ज्ञ्॒मु॒ख इति॑ यज्ञ् - मु॒खे । वीति॑ । या॒त॒य॒ति॒ । अ॒भि॒व॒र्त इत्य॑भि - व॒र्तः । स॒विꣳ॒॒श इति॑ स-विꣳ॒॒शः । इति॑ । द॒क्षि॒ण॒तः । अन्न᳚म् । वै । अ॒भि॒व॒र्त इत्य॑भि - व॒र्तः । अन्न᳚म् । स॒विꣳ॒॒श इति॑ स - विꣳ॒॒शः । अन्न᳚म् । ए॒व । द॒क्षि॒ण॒तः । ध॒त्ते॒ । तस्मा᳚त् । दक्षि॑णेन । अन्न᳚म् । अ॒द्य॒ते॒ । वर्चः॑ । द्वा॒विꣳ॒॒शः । इति॑ । प॒श्चात् । यत् । विꣳ॒॒श॒तिः । द्वे इति॑ । तेन॑ । वि॒राजा॒विति॑ वि-राजौ᳚ । यत् । द्वे इति॑ । प्र॒ति॒ष्ठेति॑ प्रति - स्था । तेन॑ । वि॒राजो॒रिति॑ वि - राजोः᳚ । ए॒व । अ॒भि॒पू॒र्वमित्य॑भि - पू॒र्वम् । अ॒न्नाद्य॒ इत्य॑न्न - अद्ये᳚ । प्रतीति॑ । ति॒ष्ठ॒ति॒ । तपः॑ । न॒व॒द॒श इति॑ नव - द॒शः । इति॑ । उ॒त्त॒र॒त इत्यु॑त्- त॒र॒तः । तस्मा᳚त् । स॒व्यः ।  \newline




\markright{ TS 5.3.3.4  \hfill https://www.vedavms.in \hfill}
\addcontentsline{toc}{section}{ TS 5.3.3.4 }
\section*{ TS 5.3.3.4 }

\textbf{TS 5.3.3.4 } \newline
\textbf{Samhita Paata} \newline

हस्त॑योस्तप॒स्वित॑रो॒ योनि॑श्चतुर्विꣳ॒॒श इति॑ पु॒रस्ता॒दुप॑ दधाति॒ चतु॑र्विꣳशत्यक्षरा गाय॒त्री गा॑य॒त्री य॑ज्ञ्मु॒खं ॅय॑ज्ञ्मु॒खमे॒व पु॒रस्ता॒द्-विया॑तयति॒ गर्भाः᳚ पञ्चविꣳ॒॒श इति॑ दक्षिण॒तोऽन्नं॒ ॅवै गर्भा॒ अन्नं॑ पञ्चविꣳ॒॒शोन्न॑मे॒व द॑क्षिण॒तो ध॑त्ते॒ तस्मा॒द् दक्षि॑णे॒नान्न॑मद्यत॒ ओज॑स्त्रिण॒व इति॑ प॒श्चादि॒मे वै लो॒कास्त्रि॑ण॒व ए॒ष्वे॑व लो॒केषु॒ प्रति॑तिष्ठति स॒भंर॑णस्त्रयोविꣳ॒॒श इत्यु॑ - [  ] \newline

\textbf{Pada Paata} \newline

हस्त॑योः । त॒प॒स्वित॑र॒ इति॑ तप॒स्वि - त॒रः॒ । योनिः॑ । च॒तु॒र्विꣳ॒॒श इति॑ चतुः - विꣳ॒॒शः । इति॑ । पु॒रस्ता᳚त् । उपेति॑ । द॒धा॒ति॒ । चतु॑र्विꣳशत्यक्ष॒रेति॒ चतु॑र्विꣳशति - अ॒क्ष॒रा॒ । गा॒य॒त्री । गा॒य॒त्री । य॒ज्ञ्॒मु॒खमिति॑ यज्ञ् - मु॒खम् । य॒ज्ञ्॒मु॒खमिति॑ यज्ञ् - मु॒खम् । ए॒व । पु॒रस्ता᳚त् । वीति॑ । या॒त॒य॒ति॒ । गर्भाः᳚ । प॒ञ्च॒विꣳ॒॒श इति॑ पञ्च - विꣳ॒॒शः । इति॑ । द॒क्षि॒ण॒तः । अन्न᳚म् । वै । गर्भाः᳚ । अन्न᳚म् । प॒ञ्च॒विꣳ॒॒श इति॑ पञ्च - विꣳ॒॒शः । अन्न᳚म् । ए॒व । द॒क्षि॒ण॒तः । ध॒त्ते॒ । तस्मा᳚त् । दक्षि॑णेन । अन्न᳚म् । अ॒द्य॒ते॒ । ओजः॑ । त्रि॒ण॒व इति॑ त्रि - न॒वः । इति॑ । प॒श्चात् । इ॒मे । वै । लो॒काः । त्रि॒ण॒व इति॑ त्रि - न॒वः । ए॒षु । ए॒व । लो॒केषु॑ । प्रतीति॑ । ति॒ष्ठ॒ति॒ । स॒भंर॑ण॒ इति॑ सं - भर॑णः । त्र॒यो॒विꣳ॒॒श इति॑ त्रयः - विꣳ॒॒शः । इति॑ ।  \newline




\markright{ TS 5.3.3.5  \hfill https://www.vedavms.in \hfill}
\addcontentsline{toc}{section}{ TS 5.3.3.5 }
\section*{ TS 5.3.3.5 }

\textbf{TS 5.3.3.5 } \newline
\textbf{Samhita Paata} \newline

-त्तर॒तस्तस्मा᳚थ् स॒व्यो हस्त॑योः सम्भा॒र्य॑तरः॒ क्रतु॑रेकत्रिꣳ॒॒श इति॑ पु॒रस्ता॒दुप॑ दधाति॒ वाग्वै क्रतु॑र्यज्ञ्मु॒खं ॅवाग्य॑ज्ञ्मु॒खमे॒व पु॒रस्ता॒द्वि या॑तयति ब्र॒द्ध्नस्य॑ वि॒ष्टपं॑ चतुस्त्रिꣳ॒॒श इति॑ दक्षिण॒तो॑ऽसौ वा आ॑दि॒त्यो ब्र॒द्ध्नस्य॑ वि॒ष्टपं॑ ब्रह्मवर्च॒समे॒व द॑क्षिण॒तो ध॑त्ते॒ तस्मा॒द् दक्षि॒णोऽर्द्धो᳚ ब्रह्मवर्च॒सित॑रः प्रति॒ष्ठा त्र॑यस्त्रिꣳ॒॒श इति॑ प॒श्चात् प्रति॑ष्ठित्यै॒ नाकः॑ षट्त्रिꣳ॒॒श इत्यु॑त्तर॒तः सु॑व॒र्गो वै ( ) लो॒को नाकः॑ सुव॒र्गस्य॑ लो॒कस्य॒ सम॑ष्ट्यै ॥श्पेचिअल् खोर्वै fओर् अनुवाकम्आ॒शु - र्व्यो॑म - ध॒रुणो॑ - भा॒न्तः - प्रतू᳚र्तिर -भिव॒र्तो - वर्च॒ - स्तपो॒ - योनि॒ - र्गर्भा॒ - ओजः॑ - स॒भंर॑णः॒ - क्रतु॑ - र्ब्र॒द्ध्रस्य॑ - प्रति॒ष्ठा - नाकः॒ - षोड॑श) \newline

\textbf{Pada Paata} \newline

उ॒त्त॒र॒त इत्यु॑त् - त॒र॒तः । तस्मा᳚त् । स॒व्यः । हस्त॑योः । स॒भां॒र्य॑तर॒ इति॑ संभा॒र्य॑ - त॒रः॒ । क्रतुः॑ । ए॒क॒त्रिꣳ॒॒श इत्ये॑क - त्रिꣳ॒॒शः । इति॑ । पु॒रस्ता᳚त् । उपेति॑ । द॒धा॒ति॒ । वाक् । वै । क्रतुः॑ । य॒ज्ञ्॒मु॒खमिति॑ यज्ञ् - मु॒खम् । वाक् । य॒ज्ञ्॒मु॒खमिति॑ यज्ञ् - मु॒खम् । ए॒व । पु॒रस्ता᳚त् । वीति॑ । या॒त॒य॒ति॒ । ब्र॒द्ध्नस्य॑ । वि॒ष्टप᳚म् । च॒तु॒स्त्रिꣳ॒॒श इति॑ चतुः - त्रिꣳ॒॒शः । इति॑ । द॒क्षि॒ण॒तः । अ॒सौ । वै । आ॒दि॒त्यः । ब्र॒द्ध्नस्य॑ । वि॒ष्टप᳚म् । ब्र॒ह्म॒व॒र्च॒समिति॑ ब्रह्म - व॒र्च॒सम् । ए॒व । द॒क्षि॒ण॒तः । ध॒त्ते॒ । तस्मा᳚त् । दक्षि॑णः । अद्‌र्धः॑ । ब्र॒ह्म॒व॒र्च॒सित॑र॒ इति॑ ब्रह्मवर्च॒सि - त॒रः॒ । प्र॒ति॒ष्ठेति॑ प्रति - स्था । त्र॒य॒स्त्रिꣳ॒॒श इति॑ त्रयः-त्रिꣳ॒॒शः । इति॑ । प॒श्चात् । प्रति॑ष्ठित्या॒ इति॒ प्रति॑-स्थि॒त्यै॒ । नाकः॑ । ष॒ट्त्रिꣳ॒॒श इति॑ षट्-त्रिꣳ॒॒शः । इति॑ । उ॒त्त॒र॒त इत्यु॑त्-त॒र॒तः । सु॒व॒र्ग इति॑ सुवः - गः । वै ( ) । लो॒कः । नाकः॑ । सु॒व॒र्गस्येति॑ सुवः - गस्य॑ । लो॒कस्य॑ । सम॑ष्ट्या॒ इति॒ सं - अ॒ष्ट्यै॒ ॥आ॒शु - र्व्यो॑म - ध॒रुणो॑ - भा॒न्तः - प्रतू᳚र्तिर -भिव॒र्तो - वर्च॒ - स्तपो॒ - योनि॒ - र्गर्भा॒ - ओजः॑ - स॒भंर॑णः॒ - क्रतु॑ - र्ब्र॒द्ध्रस्य॑ - प्रति॒ष्ठा - नाकः॒ - षोड॑श)  \newline




\markright{ TS 5.3.4.1  \hfill https://www.vedavms.in \hfill}
\addcontentsline{toc}{section}{ TS 5.3.4.1 }
\section*{ TS 5.3.4.1 }

\textbf{TS 5.3.4.1 } \newline
\textbf{Samhita Paata} \newline

अ॒ग्नेर्भा॒गो॑ऽसीति॑ पु॒रस्ता॒दुप॑ दधाति यज्ञ्मु॒खं ॅवा अ॒ग्निर्य॑ज्ञ्मु॒खं दी॒क्षा य॑ज्ञ्मु॒खं ब्रह्म॑ यज्ञ्मु॒खं त्रि॒वृद्-य॑ज्ञ्मु॒खमे॒व पु॒रस्ता॒द्वि या॑तयति नृ॒चक्ष॑सां भा॒गो॑ऽसीति॑ दक्षिण॒तः शु॑श्रु॒वाꣳसो॒ वै नृ॒चक्ष॒सोऽन्नं॑ धा॒ता जा॒तायै॒वास्मा॒ अन्न॒मपि॑ दधाति॒ तस्मा᳚ज्जा॒तोऽन्न॑मत्ति ज॒नित्रꣳ॑ स्पृ॒तꣳ स॑प्तद॒शः स्तोम॒ इत्या॒हाऽन्नं॒ ॅवै ज॒नित्र॒ - [  ] \newline

\textbf{Pada Paata} \newline

अ॒ग्नेः । भा॒गः । अ॒सि॒ । इति॑ । पु॒रस्ता᳚त् । उपेति॑ । द॒धा॒ति॒ । य॒ज्ञ्॒मु॒खमिति॑ यज्ञ् - मु॒खम् । वै । अ॒ग्निः । य॒ज्ञ्॒मु॒खमिति॑ यज्ञ्-मु॒खम् । दी॒क्षा । य॒ज्ञ्॒मु॒खमिति॑ यज्ञ् - मु॒खम् । ब्रह्म॑ । य॒ज्ञ्॒मु॒खमिति॑ यज्ञ् - मु॒खम् । त्रि॒वृदिति॑ त्रि - वृत् । य॒ज्ञ्॒मु॒खमिति॑ यज्ञ्-मु॒खम् । ए॒व । पु॒रस्ता᳚त् । वीति॑ । या॒त॒य॒ति॒ । नृ॒चक्ष॑सा॒मिति॑ नृ - चक्ष॑साम् । भा॒गः । अ॒सि॒ । इति॑ । द॒क्षि॒ण॒तः । शु॒श्रु॒वाꣳसः॑ । वै । नृ॒चक्ष॑स॒ इति॑ नृ - चक्ष॑सः । अन्न᳚म् । धा॒ता । जा॒ताय॑ । ए॒व । अ॒स्मै॒ । अन्न᳚म् । अपीति॑ । द॒धा॒ति॒ । तस्मा᳚त् । जा॒तः । अन्न᳚म् । अ॒त्ति॒ । ज॒नित्र᳚म् । स्पृ॒तम् । स॒प्त॒द॒श इति॑ सप्त - द॒शः । स्तोमः॑ । इति॑ । आ॒ह॒ । अन्न᳚म् । वै । ज॒नित्र᳚म् ।  \newline




\markright{ TS 5.3.4.2  \hfill https://www.vedavms.in \hfill}
\addcontentsline{toc}{section}{ TS 5.3.4.2 }
\section*{ TS 5.3.4.2 }

\textbf{TS 5.3.4.2 } \newline
\textbf{Samhita Paata} \newline

-मन्नꣳ॑ सप्तद॒शो-ऽन्न॑मे॒व द॑क्षिण॒तो ध॑त्ते॒ तस्मा॒द् दक्षि॑णे॒ना-न्न॑मद्यते मि॒त्रस्य॑ भा॒गो॑ऽसीति॑ प॒श्चात् प्रा॒णो वै मि॒त्रो॑ऽपा॒नो वरु॑णः प्राणापा॒नावे॒वास्मि॑न् दधाति दि॒वो वृ॒ष्टिर्वाताः᳚ स्पृ॒ता ए॑कविꣳ॒॒शः स्तोम॒ इत्या॑ह प्रति॒ष्ठा वा ए॑कविꣳ॒॒शः प्रति॑ष्ठित्या॒ इन्द्र॑स्य भा॒गो॑ऽसीत्यु॑त्तर॒त ओजो॒ वा इन्द्र॒ ओजो॒ विष्णु॒रोजः॑ क्ष॒त्रमोजः॑ पञ्चद॒श - [  ] \newline

\textbf{Pada Paata} \newline

अन्न᳚म् । स॒प्त॒द॒श इति॑ सप्त - द॒शः । अन्न᳚म् । ए॒व । द॒क्षि॒ण॒तः । ध॒त्ते॒ । तस्मा᳚त् । दक्षि॑णेन । अन्न᳚म् । अ॒द्य॒ते॒ । मि॒त्रस्य॑ । भा॒गः । अ॒सि॒ । इति॑ । प॒श्चात् । प्रा॒ण इति॑ प्र - अ॒नः । वै । मि॒त्रः । अ॒पा॒न इत्य॑प - अ॒नः । वरु॑णः । प्रा॒णा॒पा॒नाविति॑ प्राण - अ॒पा॒नौ । ए॒व । अ॒स्मि॒न्न् । द॒धा॒ति॒ । दि॒वः । वृ॒ष्टिः । वाताः᳚ । स्पृ॒ताः । ए॒क॒विꣳ॒॒श इत्ये॑क - विꣳ॒॒शः । स्तोमः॑ । इति॑ । आ॒ह॒ । प्र॒ति॒ष्ठेति॑ प्रति - स्था । वै । ए॒क॒विꣳ॒॒श इत्ये॑क - विꣳ॒॒शः । प्रति॑ष्ठित्या॒ इति॒ प्रति॑-स्थि॒त्यै॒ । इन्द्र॑स्य । भा॒गः । अ॒सि॒ । इति॑ । उ॒त्त॒र॒त इत्यु॑त् - त॒र॒तः । ओजः॑ । वै । इन्द्रः॑ । ओजः॑ । विष्णुः॑ । ओजः॑ । क्ष॒त्रम् । ओजः॑ । प॒ञ्च॒द॒श इति॑ पञ्च - द॒शः ।  \newline




\markright{ TS 5.3.4.3  \hfill https://www.vedavms.in \hfill}
\addcontentsline{toc}{section}{ TS 5.3.4.3 }
\section*{ TS 5.3.4.3 }

\textbf{TS 5.3.4.3 } \newline
\textbf{Samhita Paata} \newline

ओज॑ ए॒वोत्त॑र॒तो ध॑त्ते॒ तस्मा॑दुत्तरतो-ऽभिप्रया॒यी ज॑यति॒ वसू॑नां भा॒गो॑ऽसीति॑ पु॒रस्ता॒दुप॑ दधाति यज्ञ्मु॒खं ॅवै वस॑वो यज्ञ्मु॒खꣳ रु॒द्रा य॑ज्ञ्मु॒खं च॑तुर्विꣳ॒॒शो य॑ज्ञ्मु॒खमे॒व पु॒रस्ता॒द्वि या॑तयत्यादि॒त्यानां᳚ भा॒गो॑ऽसीति॑ दक्षिण॒तोऽन्नं॒ ॅवा आ॑दि॒त्या अन्नं॑ म॒रुतोऽन्नं॒ गर्भा॒ अन्नं॑ पञ्चविꣳ॒॒शोऽन्न॑मे॒व द॑क्षिण॒तो ध॑त्ते॒ तस्मा॒द् दक्षि॑णे॒नाऽन्न॑मद्य॒ते ऽदि॑त्यै भा॒गो॑ - [  ] \newline

\textbf{Pada Paata} \newline

ओजः॑ । ए॒व । उ॒त्त॒र॒त इत्यु॑त् - त॒र॒तः । ध॒त्ते॒ । तस्मा᳚त् । उ॒त्त॒र॒तो॒ऽभि॒प्र॒या॒यीत्यु॑त्तरतः-अ॒भि॒प्र॒या॒यी । ज॒य॒ति॒ । वसू॑नाम् । भा॒गः । अ॒सि॒ । इति॑ । पु॒रस्ता᳚त् । उपेति॑ । द॒धा॒ति॒ । य॒ज्ञ्॒मु॒खमिति॑ यज्ञ्-मु॒खम् । वै । वस॑वः । य॒ज्ञ्॒मु॒खमिति॑ यज्ञ् - मु॒खम् । रु॒द्राः । य॒ज्ञ्॒मु॒खमिति॑ यज्ञ् - मु॒खम् । च॒तु॒र्विꣳ॒॒श इति॑ चतुः - विꣳ॒॒शः । य॒ज्ञ्॒मु॒खमिति॑ यज्ञ् - मु॒खम् । ए॒व । पु॒रस्ता᳚त् । वीति॑ । या॒त॒य॒ति॒ । आ॒दि॒त्याना᳚म् । भा॒गः । अ॒सि॒ । इति॑ । द॒क्षि॒ण॒तः । अन्न᳚म् । वै । आ॒दि॒त्याः । अन्न᳚म् । म॒रुतः॑ । अन्न᳚म् । गर्भाः᳚ । अन्न᳚म् । प॒ञ्च॒विꣳ॒॒श इति॑ पञ्च - विꣳ॒॒शः । अन्न᳚म् । ए॒व । द॒क्षि॒ण॒तः । ध॒त्ते॒ । तस्मा᳚त् । दक्षि॑णेन । अन्न᳚म् । अ॒द्य॒ते॒ । अदि॑त्यै । भा॒गः ।  \newline




\markright{ TS 5.3.4.4  \hfill https://www.vedavms.in \hfill}
\addcontentsline{toc}{section}{ TS 5.3.4.4 }
\section*{ TS 5.3.4.4 }

\textbf{TS 5.3.4.4 } \newline
\textbf{Samhita Paata} \newline

ऽसीति॑ प॒श्चात् प्र॑ति॒ष्ठा वा अदि॑तिः प्रति॒ष्ठा पू॒षा प्र॑ति॒ष्ठा त्रि॑ण॒वः प्रति॑ष्ठित्यै दे॒वस्य॑ सवि॒तुर्भा॒गो॑ऽ सीत्यु॑त्तर॒तो ब्रह्म॒ वै दे॒वः स॑वि॒ता ब्रह्म॒ बृह॒स्पति॒र्ब्रह्म॑ चतुष्टो॒मो ब्र॑ह्मवर्च॒समे॒वोत्त॑र॒तो ध॑त्ते॒ तस्मा॒दुत्त॒रोऽर्द्धो᳚ ब्रह्मवर्च॒सित॑रः सावि॒त्रव॑ती भवति॒ प्रसू᳚त्यै॒ तस्मा᳚द् ब्राह्म॒णाना॒मुदी॑ची स॒निः प्रसू॑ता ध॒र्त्रश्च॑तुष्टो॒म इति॑ पु॒रस्ता॒दुप॑ दधाति यज्ञ्मु॒खं ॅवै ध॒र्त्रो - [  ] \newline

\textbf{Pada Paata} \newline

अ॒सि॒ । इति॑ । प॒श्चात् । प्र॒ति॒ष्ठेति॑ प्रति - स्था । वै । अदि॑तिः । प्र॒ति॒ष्ठेति॑ प्रति - स्था । पू॒षा । प्र॒ति॒ष्ठेति॑ प्रति - स्था । त्रि॒ण॒व इति॑ त्रि - न॒वः । प्रति॑ष्ठित्या॒ इति॒ प्रति॑-स्थि॒त्यै॒ । दे॒वस्य॑ । स॒वि॒तुः । भा॒गः । अ॒सि॒ । इति॑ । उ॒त्त॒र॒त इत्यु॑त् - त॒र॒तः । ब्रह्म॑ । वै । दे॒वः । स॒वि॒ता । ब्रह्म॑ । बृह॒स्पतिः॑ । ब्रह्म॑ । च॒तु॒ष्टो॒म इति॑ चतुः - स्तो॒मः । ब्र॒ह्म॒व॒र्च॒समिति॑ ब्रह्म - व॒र्च॒सम् । ए॒व । उ॒त्त॒र॒त इत्यु॑त् - त॒र॒तः । ध॒त्ते॒ । तस्मा᳚त् । उत्त॑र॒ इत्युत् - त॒रः॒ । अद्‌र्धः॑ । ब्र॒ह्म॒व॒र्च॒सित॑र॒ इति॑ ब्रह्मवर्च॒सि - त॒रः॒ । सा॒वि॒त्रव॒तीति॑ सावि॒त्र - व॒ती॒ । भ॒व॒ति॒ । प्रसू᳚त्या॒ इति॒ प्र - सू॒त्यै॒ । तस्मा᳚त् । ब्रा॒ह्म॒णाना᳚म् । उदी॑ची । स॒निः । प्रसू॒तेति॒ प्र - सू॒ता॒ । ध॒र्त्रः । च॒तु॒ष्टो॒म इति॑ चतुः - स्तो॒मः । इति॑ । पु॒रस्ता᳚त् । उपेति॑ । द॒धा॒ति॒ । य॒ज्ञ्॒मु॒खमिति॑ यज्ञ् - मु॒खम् । वै । ध॒र्त्रः ।  \newline




\markright{ TS 5.3.4.5  \hfill https://www.vedavms.in \hfill}
\addcontentsline{toc}{section}{ TS 5.3.4.5 }
\section*{ TS 5.3.4.5 }

\textbf{TS 5.3.4.5 } \newline
\textbf{Samhita Paata} \newline

य॑ज्ञ्मु॒खं च॑तुष्टो॒मो य॑ज्ञ्मु॒खमे॒व पु॒रस्ता॒द्वि या॑तयति॒ यावा॑नां भा॒गो॑ऽसीति॑ दक्षिण॒तो मासा॒ वै यावा॑ अर्द्धमा॒सा अया॑वा॒-स्तस्मा᳚द्-दक्षि॒णावृ॑तो॒ मासा॒ अन्नं॒ ॅवै यावा॒ अन्नं॑ प्र॒जा अन्न॑मे॒व द॑क्षिण॒तो ध॑त्ते॒ तस्मा॒द् दक्षि॑णे॒ना-न्न॑मद्यत ऋभू॒णां भा॒गो॑ऽसीति॑ प॒श्चात् प्रति॑ष्ठित्यै विव॒र्तो᳚ ऽष्टाचत्वारिꣳ॒॒श इत्यु॑त्तर॒तो॑ऽनयो᳚र्लो॒कयोः᳚ सवीर्य॒त्वाय॒ तस्मा॑दि॒मौ लो॒कौ स॒माव॑द्-वीर्यौ॒ - [  ] \newline

\textbf{Pada Paata} \newline

य॒ज्ञ्॒मु॒खमिति॑ यज्ञ् - मु॒खम् । च॒तु॒ष्टो॒म इति॑ चतुः - स्तो॒मः । य॒ज्ञ्॒मु॒खमिति॑ यज्ञ् - मु॒खम् । ए॒व । पु॒रस्ता᳚त् । वीति॑ । या॒त॒य॒ति॒ । यावा॑नाम् । भा॒गः । अ॒सि॒ । इति॑ । द॒क्षि॒ण॒तः । मासाः᳚ । वै । यावाः᳚ । अ॒द्‌र्ध॒मा॒सा इत्य॑द्‌र्ध - मा॒साः । अया॑वाः । तस्मा᳚त् । द॒क्षि॒णावृ॑त॒ इति॑ दक्षि॒णा - आ॒वृ॒तः॒ । मासाः᳚ । अन्न᳚म् । वै । यावाः᳚ । अन्न᳚म् । प्र॒जा इति॑ प्र - जाः । अन्न᳚म् । ए॒व । द॒क्षि॒ण॒तः । ध॒त्ते॒ । तस्मा᳚त् । दक्षि॑णेन । अन्न᳚म् । अ॒द्य॒ते॒ । ऋ॒भू॒णाम् । भा॒गः । अ॒सि॒ । इति॑ । प॒श्चात् । प्रति॑ष्ठित्या॒ इति॒ प्रति॑ - स्थि॒त्यै॒ । वि॒व॒र्त इति॑ वि - व॒र्तः । अ॒ष्टा॒च॒त्वा॒रिꣳ॒॒श इत्य॑ष्टा-च॒त्वा॒रिꣳ॒॒शः । इति॑ । उ॒त्त॒र॒त इत्यु॑त्-त॒र॒तः । अ॒नयोः᳚ । लो॒कयोः᳚ । स॒वी॒र्य॒त्वायेति॑ सवीर्य - त्वाय॑ । तस्मा᳚त् । इ॒मौ । लो॒कौ । स॒माव॑द्वीर्या॒विति॑ स॒माव॑त् - वी॒र्यौ॒ ।  \newline




\markright{ TS 5.3.4.6  \hfill https://www.vedavms.in \hfill}
\addcontentsline{toc}{section}{ TS 5.3.4.6 }
\section*{ TS 5.3.4.6 }

\textbf{TS 5.3.4.6 } \newline
\textbf{Samhita Paata} \newline

यस्य॒ मुख्य॑वतीः पु॒रस्ता॑दुपधी॒यन्ते॒ मुख्य॑ ए॒व भ॑व॒त्याऽस्य॒ मुख्यो॑ जायते॒ यस्या-न्न॑वती - र्दक्षिण॒तो-ऽत्त्यन्न॒माऽस्या᳚न्ना॒दो जा॑यते॒ यस्य॑ प्रति॒ष्ठाव॑तीः प॒श्चात् प्रत्ये॒व ति॑ष्ठति॒ यस्यौज॑स्वतीरुत्तर॒त ओ॑ज॒स्व्ये॑व भ॑व॒त्याऽस्यौ॑ज॒स्वी जा॑यते॒ ऽर्को वा ए॒ष यद॒ग्निस्तस्यै॒तदे॒व स्तो॒त्रमे॒तच्छ॒स्त्रं ॅयदे॒षा वि॒धा - [  ] \newline

\textbf{Pada Paata} \newline

यस्य॑ । मुख्य॑वती॒रिति॒ मुख्य॑ - व॒तीः॒ । पु॒रस्ता᳚त् । उ॒प॒धी॒यन्त॒ इत्यु॑प - धी॒यन्ते᳚ । मुख्यः॑ । ए॒व । भ॒व॒ति॒ । एति॑ । अ॒स्य॒ । मुख्यः॑ । जा॒य॒ते॒ । यस्य॑ । अन्न॑वती॒रित्यन्न॑ - व॒तीः॒ । द॒क्षि॒ण॒तः । अत्ति॑ । अन्न᳚म् । एति॑ । अ॒स्य॒ । अ॒न्ना॒द इत्य॑न्न - अ॒दः । जा॒य॒ते॒ । यस्य॑ । प्र॒ति॒ष्ठाव॑ती॒रिति॑ प्रति॒ष्ठा - व॒तीः॒ । प॒श्चात् । प्रतीति॑ । ए॒व । ति॒ष्ठ॒ति॒ । यस्य॑ । ओज॑स्वतीः । उ॒त्त॒र॒त इत्यु॑त् - त॒र॒तः । ओ॒ज॒स्वी । ए॒व । भ॒व॒ति॒ । एति॑ । अ॒स्य॒ । ओ॒ज॒स्वी । जा॒य॒ते॒ । अ॒र्कः । वै । ए॒षः । यत् । अ॒ग्निः । तस्य॑ । ए॒तत् । ए॒व । स्तो॒त्रम् । ए॒तत् । श॒स्त्रम् । यत् । ए॒षा । वि॒धेति॑ वि - धा ।  \newline




\markright{ TS 5.3.4.7  \hfill https://www.vedavms.in \hfill}
\addcontentsline{toc}{section}{ TS 5.3.4.7 }
\section*{ TS 5.3.4.7 }

\textbf{TS 5.3.4.7 } \newline
\textbf{Samhita Paata} \newline

वि॑धी॒यते॒ऽर्क ए॒व तद॒र्क्य॑मनु॒ वि धी॑य॒ते ऽत्त्यन्न॒माऽस्या᳚न्ना॒दो जा॑यते॒ यस्यै॒षा वि॒धा वि॑धी॒यते॒ य उ॑ चैनामे॒वं ॅवेद॒ सृष्टी॒रुप॑ दधाति यथासृ॒ष्टमे॒वाव॑ रुन्धे॒ न वा इ॒दं दिवा॒ न नक्त॑मासी॒दव्या॑वृत्तं॒ ते दे॒वा ए॒ता व्यु॑ष्टीरपश्य॒न् ता उपा॑दधत॒ ततो॒ वा इ॒दं ( ) ॅव्यौ᳚च्छ॒द्-यस्यै॒ता उ॑पधी॒यन्ते॒ व्ये॑वास्मा॑ उच्छ॒त्यथो॒ तम॑ ए॒वाप॑हते ॥श्पेचिअल् खोर्वै fओर् अनुवाकम्(अ॒ग्ने - र्नृ॒चक्ष॑सां - ज॒नित्रं॑ - मि॒त्र - स्येन्द्र॑स्य॒ -वसू॑ना - मादि॒त्याना॒ - मदि॑त्यै - दे॒वस्य॑ सवि॒तुः - सा॑वि॒त्रव॑ती - ध॒र्त्रो - यावा॑ना-मृभू॒णां - ॅवि॑व॒र्त - श्चतु॑र्दश) \newline

\textbf{Pada Paata} \newline

वि॒धी॒यत॒ इति॑ वि - धी॒यते᳚ । अ॒र्के । ए॒व । तत् । अ॒र्क्य᳚म् । अनु॑ । वीति॑ । धी॒य॒ते॒ । अत्ति॑ । अन्न᳚म् । एति॑ । अ॒स्य॒ । अ॒न्ना॒द इत्य॑न्न - अ॒दः । जा॒य॒ते॒ । यस्य॑ । ए॒षा । वि॒धेति॑ वि - धा । वि॒धी॒यत॒ इति॑ वि - धी॒यते᳚ । यः । उ॒ । च॒ । ए॒ना॒म् । ए॒वम् । वेद॑ । सृष्टीः᳚ । उपेति॑ । द॒धा॒ति॒ । य॒था॒सृ॒ष्टमिति॑ यथा - सृ॒ष्टम् । ए॒व । अवेति॑ । रु॒न्धे॒ । न । वै । इ॒दम् । दिवा᳚ । न । नक्त᳚म् । आ॒सी॒त् । अव्या॑वृत्त॒मित्यवि॑ - आ॒वृ॒त्त॒म् । ते । दे॒वाः । ए॒ताः । व्यु॑ष्टी॒रिति॒ वि - उ॒ष्टीः॒ । अ॒प॒श्य॒न्न् । ताः । उपेति॑ । अ॒द॒ध॒त॒ । ततः॑ । वै । इ॒दम् ( ) । वीति॑ । औ॒च्छ॒त् । यस्य॑ । ए॒ताः । उ॒प॒धी॒यन्त॒ इत्यु॑प - धी॒यन्ते᳚ । वीति॑ । ए॒व । अ॒स्मै॒ । उ॒च्छ॒ति॒ । अथो॒ इति॑ । तमः॑ । ए॒व । अपेति॑ । ह॒ते॒ ॥(अ॒ग्ने - र्नृ॒चक्ष॑सां - ज॒नित्रं॑ - मि॒त्र - स्येन्द्र॑स्य॒ -वसू॑ना - मादि॒त्याना॒ - मदि॑त्यै - दे॒वस्य॑ सवि॒तुः - सा॑वि॒त्रव॑ती - ध॒र्त्रो - यावा॑ना-मृभू॒णां - ॅवि॑व॒र्त - श्चतु॑र्दश)  \newline




\markright{ TS 5.3.5.1  \hfill https://www.vedavms.in \hfill}
\addcontentsline{toc}{section}{ TS 5.3.5.1 }
\section*{ TS 5.3.5.1 }

\textbf{TS 5.3.5.1 } \newline
\textbf{Samhita Paata} \newline

अग्ने॑ जा॒तान् प्रणु॑दा नः स॒पत्ना॒निति॑ पु॒रस्ता॒दुप॑ दधाति जा॒ताने॒व भ्रातृ॑व्या॒न् प्रणु॑दते॒ सह॑सा जा॒तानिति॑ प॒श्चाज्ज॑नि॒ष्यमा॑णाने॒व प्रति॑ नुदते चतुश्चत्वारिꣳ॒॒शः स्तोम॒ इति॑ दक्षिण॒तो ब्र॑ह्मवर्च॒सं ॅवै च॑तुश्चत्वारिꣳ॒॒शो ब्र॑ह्मवर्च॒समे॒व द॑क्षिण॒तो ध॑त्ते॒ तस्मा॒द् दक्षि॒णोऽर्द्धो᳚ ब्रह्मवर्च॒सित॑रः षोड॒शः स्तोम॒ इत्यु॑त्तर॒त ओजो॒ वै षो॑ड॒श ओज॑ ए॒वोत्त॑र॒तो ध॑त्ते॒ तस्मा॑ - [  ] \newline

\textbf{Pada Paata} \newline

अग्ने᳚ । जा॒तान् । प्रेति॑ । नु॒द॒ । नः॒ । स॒पत्नान्॑ । इति॑ । पु॒रस्ता᳚त् । उपेति॑ । द॒धा॒ति॒ । जा॒तान् । ए॒व । भ्रातृ॑व्यान् । प्रेति॑ । नु॒द॒ते॒ । सह॑सा । जा॒तान् । इति॑ । प॒श्चात् । ज॒नि॒ष्यमा॑णान् । ए॒व । प्रतीति॑ । नु॒द॒ते॒ । च॒तु॒श्च॒त्वा॒रिꣳ॒॒श इति॑ चतुः - च॒त्वा॒रिꣳ॒॒शः । स्तोमः॑ । इति॑ । द॒क्षि॒ण॒तः । ब्र॒ह्म॒व॒र्च॒समिति॑ ब्रह्म - व॒र्च॒सम् । वै । च॒तु॒श्च॒त्वा॒रिꣳ॒॒श इति॑ चतुः - च॒त्वा॒रिꣳ॒॒शः । ब्र॒ह्म॒व॒र्च॒समिति॑ ब्रह्म - व॒र्च॒सम् । ए॒व । द॒क्षि॒ण॒तः । ध॒त्ते॒ । तस्मा᳚त् । दक्षि॑णः । अद्‌र्धः॑ । ब्र॒ह्म॒व॒र्च॒सित॑र॒ इति॑ ब्रह्मवर्च॒सि - त॒रः॒ । षो॒ड॒शः । स्तोमः॑ । इति॑ । उ॒त्त॒र॒त इत्यु॑त् - त॒र॒तः । ओजः॑ । वै । षो॒ड॒शः । ओजः॑ । ए॒व । उ॒त्त॒र॒त इत्यु॑त् - त॒र॒तः । ध॒त्ते॒ । तस्मा᳚त् ।  \newline




\markright{ TS 5.3.5.2  \hfill https://www.vedavms.in \hfill}
\addcontentsline{toc}{section}{ TS 5.3.5.2 }
\section*{ TS 5.3.5.2 }

\textbf{TS 5.3.5.2 } \newline
\textbf{Samhita Paata} \newline

दुत्तरतोऽभिप्रया॒यी ज॑यति॒ वज्रो॒ वै च॑तुश्चत्वारिꣳ॒॒शो वज्रः॑ षोड॒शो यदे॒ते इष्ट॑के उप॒दधा॑ति जा॒ताꣳश्चै॒व ज॑नि॒ष्यमा॑णाꣳश्च॒ भ्रातृ॑व्यान् प्र॒णुद्य॒ वज्र॒मनु॒ प्रह॑रति॒ स्तृत्यै॒ पुरी॑षवतीं॒ मद्ध्य॒ उप॑दधाति॒ पुरी॑षं॒ ॅवै मद्ध्य॑मा॒त्मनः॒ सात्मा॑नमे॒वाग्निं चि॑नुते॒ सात्मा॒ऽमुष्मि॑न् ॅलो॒के भ॑वति॒ य ए॒वं ॅवेदै॒ता वा अ॑सप॒त्ना नामेष्ट॑का॒ यस्यै॒ता उ॑पधी॒यन्ते॒ - [  ] \newline

\textbf{Pada Paata} \newline

उ॒त्त॒र॒तो॒ऽभि॒प्र॒या॒यीत्यु॑त्तरतः - अ॒भि॒प्र॒या॒यी । ज॒य॒ति॒ । वज्रः॑ । वै । च॒तु॒श्च॒त्वा॒रिꣳ॒॒श इति॑ चतुः-च॒त्वा॒रिꣳ॒॒शः । वज्रः॑ । षो॒ड॒शः । यत् । ए॒ते इति॑ । इष्ट॑के॒ इति॑ । उ॒प॒दधा॒तीत्यु॑प - दधा॑ति । जा॒तान् । च॒ । ए॒व । ज॒नि॒ष्यमा॑णान् । च॒ । भ्रातृ॑व्यान् । प्र॒णुद्येति॑ प्र-नुद्य॑ । वज्र᳚म् । अनु॑ । प्रेति॑ । ह॒र॒ति॒ । स्तृत्यै᳚ । पुरी॑षवती॒मिति॒ पुरी॑ष-व॒ती॒म् । मद्ध्ये᳚ । उपेति॑ । द॒धा॒ति॒ । पुरी॑षम् । वै । मद्ध्य᳚म् । आ॒त्मनः॑ । सात्मा॑न॒मिति॒ स - आ॒त्मा॒न॒म् । ए॒व । अ॒ग्निम् । चि॒नु॒ते॒ । सात्मेति॒ स॒ - आ॒त्मा॒ । अ॒मुष्मिन्न्॑ । लो॒के । भ॒व॒ति॒ । यः । ए॒वम् । वेद॑ । ए॒ताः । वै । अ॒स॒प॒त्नाः । नाम॑ । इष्ट॑काः । यस्य॑ । ए॒ताः । उ॒प॒धी॒यन्त॒ इत्यु॑प - धी॒यन्ते᳚ ।  \newline




\markright{ TS 5.3.5.3  \hfill https://www.vedavms.in \hfill}
\addcontentsline{toc}{section}{ TS 5.3.5.3 }
\section*{ TS 5.3.5.3 }

\textbf{TS 5.3.5.3 } \newline
\textbf{Samhita Paata} \newline

नास्य॑ स॒पत्नो॑ भवति प॒शुर्वा ए॒ष यद॒ग्निर्वि॒राज॑ उत्त॒मायां॒ चित्या॒मुप॑ दधाति वि॒राज॑मे॒वोत्त॒मां प॒शुषु॑ दधाति॒ तस्मा᳚त् पशु॒मानु॑त्त॒मां ॅवाचं॑ ॅवदति॒ दश॑द॒शोप॑ दधाति सवीर्य॒त्वाया᳚ऽक्ष्ण॒योप॑ दधाति॒ तस्मा॑दक्ष्ण॒या प॒शवोऽङ्गा॑नि॒ प्रह॑रन्ति॒ प्रति॑ष्ठित्यै॒ यानि॒ वै छन्दाꣳ॑सि सुव॒र्ग्या᳚ण्यास॒न् तैर्दे॒वाः सु॑व॒र्गं ॅलो॒कमा॑य॒न् तेनर्.ष॑यो - [  ] \newline

\textbf{Pada Paata} \newline

न । अ॒स्य॒ । स॒पत्नः॑ । भ॒व॒ति॒ । प॒शुः । वै । ए॒षः । यत् । अ॒ग्निः । वि॒राज॒ इति॑ वि - राजः॑ । उ॒त्त॒माया॒मित्यु॑त् - त॒माया᳚म् । चित्या᳚म् । उपेति॑ । द॒धा॒ति॒ । वि॒राज॒मिति॑ वि - राज᳚म् । ए॒व । उ॒त्त॒मामित्यु॑त् - त॒माम् । प॒शुषु॑ । द॒धा॒ति॒ । तस्मा᳚त् । प॒शु॒मानिति॑ पशु - मान् । उ॒त्त॒मामित्यु॑त् - त॒माम् । वाच᳚म् । व॒द॒ति॒ । दश॑द॒शेति॒ दश॑ - द॒श॒ । उपेति॑ । द॒धा॒ति॒ । स॒वी॒र्य॒त्वायेति॑ सवीर्य-त्वाय॑ । अ॒क्ष्ण॒या । उपेति॑ । द॒धा॒ति॒ । तस्मा᳚त् । अ॒क्ष्ण॒या । प॒शवः॑ । अङ्गा॑नि । प्रेति॑ । ह॒र॒न्ति॒ । प्रति॑ष्ठित्या॒ इति॒ प्रति॑ - स्थि॒त्यै॒ । यानि॑ । वै । छन्दाꣳ॑सि । सु॒व॒र्ग्या॑णीति॑ सुवः - ग्या॑नि । आसन्न्॑ । तैः । दे॒वाः । सु॒व॒र्गमिति॑ सुवः - गम् । लो॒कम् । आ॒य॒न्न् । तेन॑ । ऋष॑यः ।  \newline




\markright{ TS 5.3.5.4  \hfill https://www.vedavms.in \hfill}
\addcontentsline{toc}{section}{ TS 5.3.5.4 }
\section*{ TS 5.3.5.4 }

\textbf{TS 5.3.5.4 } \newline
\textbf{Samhita Paata} \newline

ऽश्राम्य॒न् ते तपो॑ऽतप्यन्त॒ तानि॒ तप॑साऽपश्य॒न् तेभ्य॑ ए॒ता इष्ट॑का॒ निर॑मिम॒तेव॒श्छन्दो॒ वरि॑व॒श्छन्द॒ इति॒ ता उपा॑दधत॒ ताभि॒र्वै ते सु॑व॒र्गं ॅलो॒कमा॑य॒न॒. यदे॒ता इष्ट॑का उप॒दधा॑ति॒ यान्ये॒व छन्दाꣳ॑सि सुव॒र्ग्या॑णि॒ तैरे॒व यज॑मानः सुव॒र्गं ॅलो॒कमे॑ति य॒ज्ञेन॒ वै प्र॒जाप॑तिः प्र॒जा अ॑सृजत॒ ताः स्तोम॑ भागैरे॒वाऽसृ॑जत॒ यथ् - [  ] \newline

\textbf{Pada Paata} \newline

अ॒श्रा॒म्य॒न्न् । ते । तपः॑ । अ॒त॒प्य॒न्त॒ । तानि॑ । तप॑सा । अ॒प॒श्य॒न्न् । तेभ्यः॑ । ए॒ताः । इष्ट॑काः । निरिति॑ । अ॒मि॒म॒त॒ । एवः॑ । छन्दः॑ । वरि॑वः । छन्दः॑ । इति॑ । ताः । उपेति॑ । अ॒द॒ध॒त॒ । ताभिः॑ । वै । ते । सु॒व॒र्गमिति॑ सुवः-गम् । लो॒कम् । आ॒य॒न्न् । यत् । ए॒ताः । इष्ट॑काः । उ॒प॒दधा॒तीत्यु॑प - दधा॑ति । यानि॑ । ए॒व । छन्दाꣳ॑सि । सु॒व॒र्ग्या॑णीति॑ सुवः-ग्या॑नि । तैः । ए॒व । यज॑मानः । सु॒व॒र्गमिति॑ सुवः - गम् । लो॒कम् । ए॒ति॒ । य॒ज्ञेन॑ । वै । प्र॒जाप॑ति॒रिति॑ प्र॒जा - प॒तिः॒ । प्र॒जा इति॑ प्र - जाः । अ॒सृ॒ज॒त॒ । ताः । स्तोम॑भागै॒रिति॒ स्तोम॑ - भा॒गैः॒ । ए॒व । अ॒सृ॒ज॒त॒ । यत् ।  \newline




\markright{ TS 5.3.5.5  \hfill https://www.vedavms.in \hfill}
\addcontentsline{toc}{section}{ TS 5.3.5.5 }
\section*{ TS 5.3.5.5 }

\textbf{TS 5.3.5.5 } \newline
\textbf{Samhita Paata} \newline

स्तोम॑ भागा उप॒दधा॑ति प्र॒जा ए॒व तद्-यज॑मानः सृजते॒ बृह॒स्पति॒र्वा ए॒तद्-य॒ज्ञ्स्य॒ तेजः॒ सम॑भर॒द्यथ् स्तोम॑भागा॒ यथ् स्तोम॑भागा उप॒दधा॑ति॒ सते॑जसमे॒वाग्निं चि॑नुते॒ बृह॒स्पति॒र्वा ए॒तां ॅय॒ज्ञ्स्य॑ प्रति॒ष्ठाम॑पश्य॒द्यथ् स्तोम॑भागा॒ यथ् स्तोम॑भागा उप॒दधा॑ति य॒ज्ञ्स्य॒ प्रति॑ष्ठित्यै स॒प्तस॒प्तोप॑ दधाति सवीर्य॒त्वाय॑ ति॒स्रो मद्ध्ये॒ प्रति॑ष्ठित्यै ॥ \newline

\textbf{Pada Paata} \newline

स्तोम॑भागा॒ इति॒ स्तोम॑ - भा॒गाः॒ । उ॒प॒दधा॒तीत्यु॑प - दधा॑ति । प्र॒जा इति॑ प्र - जाः । ए॒व । तत् । यज॑मानः । सृ॒ज॒ते॒ । बृह॒स्पतिः॑ । वै । ए॒तत् । य॒ज्ञ्स्य॑ । तेजः॑ । समिति॑ । अ॒भ॒र॒त् । यत् । स्तोम॑भागा॒ इति॒ स्तोम॑ - भा॒गाः॒ । यत् । स्तोम॑भागा॒ इति॒ स्तोम॑ - भा॒गाः॒ । उ॒प॒दधा॒तीत्यु॑प - दधा॑ति । सते॑जस॒मिति॒ स - ते॒ज॒स॒म् । ए॒व । अ॒ग्निम् । चि॒नु॒ते॒ । बृह॒स्पतिः॑ । वै । ए॒ताम् । य॒ज्ञ्स्य॑ । प्र॒ति॒ष्ठामिति॑ प्रति - स्थाम् । अ॒प॒श्य॒त् । यत् । स्तोम॑भागा॒ इति॒ स्तोम॑ - भा॒गाः॒ । यत् । स्तोम॑भागा॒ इति॒ स्तोम॑ - भा॒गाः॒ । उ॒प॒दधा॒तीत्यु॑प - दधा॑ति । य॒ज्ञ्स्य॑ । प्रति॑ष्ठित्या॒ इति॒ प्रति॑-स्थि॒त्यै॒ । स॒प्तस॒प्तेति॑ स॒प्त - स॒प्त॒ । उपेति॑ । द॒धा॒ति॒ । स॒वी॒र्य॒त्वायेति॑ सवीर्य - त्वाय॑ । ति॒स्रः । मद्ध्ये᳚ । प्रति॑ष्ठित्या॒ इति॒ प्रति॑ - स्थि॒त्यै॒ ॥  \newline




\markright{ TS 5.3.6.1  \hfill https://www.vedavms.in \hfill}
\addcontentsline{toc}{section}{ TS 5.3.6.1 }
\section*{ TS 5.3.6.1 }

\textbf{TS 5.3.6.1 } \newline
\textbf{Samhita Paata} \newline

र॒श्मिरित्ये॒वा ऽऽदि॒त्यम॑सृजत॒ प्रेति॒रिति॒ धर्म॒मन्वि॑ति॒रिति॒ दिवꣳ॑ स॒धिंरित्य॒न्तरि॑क्षं प्रति॒धिरिति॑ पृथि॒वीं ॅवि॑ष्ट॒म्भ इति॒ वृष्टिं॑ प्र॒वेत्यह॑रनु॒वेति॒ रात्रि॑मु॒शिगिति॒ वसू᳚न् प्रके॒त इति॑ रु॒द्रान्थ् सु॑दी॒तिरित्या॑दि॒त्यानोज॒ इति॑ पि॒तॄꣳस्तन्तु॒रिति॑ प्र॒जाः पृ॑तना॒षाडिति॑ प॒शून् रे॒वदित्यो-ष॑धीरभि॒जिद॑सि यु॒क्तग्रा॒वे - [  ] \newline

\textbf{Pada Paata} \newline

र॒श्मिः । इति॑ । ए॒व । आ॒दि॒त्यम् । अ॒सृ॒ज॒त॒ । प्रेति॒रिति॒ प्र - इ॒तिः॒ । इति॑ । धर्म᳚म् । अन्वि॑ति॒रित्यनु॑ - इ॒तिः॒ । इति॑ । दिव᳚म् । स॒न्धिरिति॑ सं - धिः । इति॑ । अ॒न्तरि॑क्षम् । प्र॒ति॒धिरिति॑ प्रति - धिः । इति॑ । पृ॒थि॒वीम् । वि॒ष्ट॒भं इति॑ वि - स्त॒भंः । इति॑ । वृष्टि᳚म् । प्र॒वेति॑ प्र - वा । इति॑ । अहः॑ । अ॒नु॒वेत्य॑नु-वा । इति॑ । रात्रि᳚म् । उ॒शिक् । इति॑ । वसून्॑ । प्र॒के॒त इति॑ प्र - के॒तः । इति॑ । रु॒द्रान् । सु॒दी॒तिरिति॑ सु-दी॒तिः । इति॑ । आ॒दि॒त्यान् । ओजः॑ । इति॑ । पि॒तॄन् । तन्तुः॑ । इति॑ । प्र॒जा इति॑ प्र - जाः । पृ॒त॒ना॒षाट् । इति॑ । प॒शून् । रे॒वत् । इति॑ । ओष॑धीः । अ॒भि॒जिदित्य॑भि - जित् । अ॒सि॒ । यु॒क्तग्रा॒वेति॑ यु॒क्त - ग्रा॒वा॒ ।  \newline




\markright{ TS 5.3.6.2  \hfill https://www.vedavms.in \hfill}
\addcontentsline{toc}{section}{ TS 5.3.6.2 }
\section*{ TS 5.3.6.2 }

\textbf{TS 5.3.6.2 } \newline
\textbf{Samhita Paata} \newline

-न्द्रा॑य॒ त्वेन्द्रं॑ जि॒न्वेत्ये॒व द॑क्षिण॒तो वज्रं॒ पर्यौ॑हद॒भिजि॑त्यै॒ ताः प्र॒जा अप॑प्राणा असृजत॒ तास्वधि॑पतिर॒सीत्ये॒व प्रा॒णम॑दधा-द्य॒न्तेत्य॑पा॒नꣳ सꣳ॒॒सर्प॒ इति॒ चक्षु॑र्वयो॒धा इति॒ श्रोत्रं॒ ताः प्र॒जाः प्रा॑ण॒तीर॑पान॒तीः पश्य॑न्तीः शृण्व॒तीर्न मि॑थु॒नी अ॑भव॒न् तासु॑ त्रि॒वृद॒सीत्ये॒व मि॑थु॒नम॑दधा॒त् ताः प्र॒जा मि॑थु॒नी - [  ] \newline

\textbf{Pada Paata} \newline

इन्द्रा॑य । त्वा॒ । इन्द्र᳚म् । जि॒न्व॒ । इति॑ । ए॒व । द॒क्षि॒ण॒तः । वज्र᳚म् । परीति॑ । औ॒ह॒त् । अ॒भिजि॑त्या॒ इत्य॒भि - जि॒त्यै॒ । ताः । प्र॒जा इति॑ प्र - जाः । अप॑प्राणा॒ इत्यप॑ - प्रा॒णाः॒ । अ॒सृ॒ज॒त॒ । तासु॑ । अधि॑पति॒रित्यधि॑-प॒तिः॒ । अ॒सि॒ । इति॑ । ए॒व । प्रा॒णमिति॑ प्र-अ॒नम् । अ॒द॒धा॒त् । य॒न्ता । इति॑ । अ॒पा॒नमित्य॑प - अ॒नम् । सꣳ॒॒सर्प॒ इति॑ सं - सर्पः॑ । इति॑ । चक्षुः॑ । व॒यो॒धा इति॑ वयः - धाः । इति॑ । श्रोत्र᳚म् । ताः । प्र॒जा इति॑ प्र - जाः । प्रा॒ण॒तीरिति॑ प्र - अ॒न॒तीः । अ॒पा॒न॒तीरित्य॑प - अ॒न॒तीः । पश्य॑न्तीः । शृ॒ण्व॒तीः । न । मि॒थु॒नी । अ॒भ॒व॒न्न् । तासु॑ । त्रि॒वृदिति॑ त्रि - वृत् । अ॒सि॒ । इति॑ । ए॒व । मि॒थु॒नम् । अ॒द॒धा॒त् । ताः । प्र॒जा इति॑ प्र-जाः । मि॒थु॒नी ।  \newline




\markright{ TS 5.3.6.3  \hfill https://www.vedavms.in \hfill}
\addcontentsline{toc}{section}{ TS 5.3.6.3 }
\section*{ TS 5.3.6.3 }

\textbf{TS 5.3.6.3 } \newline
\textbf{Samhita Paata} \newline

भव॑न्ती॒र्न प्राजा॑यन्त॒ ताः सꣳ॑रो॒हो॑ऽसि नीरो॒हो॑ऽसीत्ये॒व प्राऽज॑नय॒त् ताः प्र॒जाः प्रजा॑ता॒ न प्रत्य॑तिष्ठ॒न् ता व॑सु॒को॑ऽसि॒ वेष॑श्रिरसि॒ वस्य॑ष्टिर॒सीत्ये॒वैषु लो॒केषु॒ प्रत्य॑स्थापय॒द्यदाह॑ वसु॒को॑ऽसि॒ वेष॑श्रिरसि॒ वस्य॑ष्टिर॒सीति॑ प्र॒जा ए॒व प्रजा॑ता ए॒षु लो॒केषु॒ प्रति॑ष्ठापयति॒ सात्मा॒ऽन्तरि॑क्षꣳ ( ) रोहति॒ सप्रा॑णो॒ऽमुष्मि॑न् ॅलो॒के प्रति॑ तिष्ठ॒त्यव्य॑र्द्धुकः प्राणापा॒नाभ्यां᳚ भवति॒ य ए॒वं ॅवेद॑ ॥ \newline

\textbf{Pada Paata} \newline

भव॑न्तीः । न । प्रेति॑ । अ॒जा॒य॒न्त॒ । ताः । सꣳ॒॒रो॒ह इति॑ सं - रो॒हः । अ॒सि॒ । नी॒रो॒ह इति॑ निः - रो॒हः । अ॒सि॒ । इति॑ । ए॒व । प्रेति॑ । अ॒ज॒न॒य॒त् । ताः । प्र॒जा इति॑ प्र - जाः । प्रजा॑ता॒ इति॒ प्र-जा॒ताः॒ । न । प्रतीति॑ । अ॒ति॒ष्ठ॒न्न् । ताः । व॒सु॒कः । अ॒सि॒ । वेष॑श्रि॒रिति॒ वेष॑-श्रिः॒ । अ॒सि॒ । वस्य॑ष्टिः । अ॒सि॒ । इति॑ । ए॒व । ए॒षु । लो॒केषु॑ । प्रतीति॑ । अ॒स्था॒प॒य॒त् । यत् । आह॑ । व॒सु॒कः । अ॒सि॒ । वेष॑श्रि॒रिति॒ वेष॑ - श्रिः॒ । अ॒सि॒ । वस्य॑ष्टिः । अ॒सि॒ । इति॑ । प्र॒जा इति॑ प्र - जाः । ए॒व । प्रजा॑ता॒ इति॒ प्र - जा॒ताः॒ । ए॒षु । लो॒केषु॑ । प्रतीति॑ । स्था॒प॒य॒ति॒ । सात्मेति॒ स - आ॒त्मा॒ । अ॒न्तरि॑क्षम् ( ) । रो॒ह॒ति॒ । सप्रा॑ण॒ इति॒ स - प्रा॒णः॒ । अ॒मुष्मिन्न्॑ । लो॒के । प्रतीति॑ । ति॒ष्ठ॒ति॒ । अव्य॑द्‌र्धुक॒ इत्यवि॑ - अ॒द्‌र्धु॒कः॒ । प्रा॒णा॒पा॒नाभ्या॒मिति॑ प्राण - अ॒पा॒नाभ्या᳚म् । भ॒व॒ति॒ । यः । ए॒वम् । वेद॑ ॥  \newline




\markright{ TS 5.3.7.1  \hfill https://www.vedavms.in \hfill}
\addcontentsline{toc}{section}{ TS 5.3.7.1 }
\section*{ TS 5.3.7.1 }

\textbf{TS 5.3.7.1 } \newline
\textbf{Samhita Paata} \newline

ना॒क॒सद्भि॒र्वै दे॒वाः सु॑व॒र्गं ॅलो॒कमा॑य॒न् तन्ना॑क॒सदां᳚ नाकस॒त्त्वं ॅयन्ना॑क॒सद॑ उप॒दधा॑ति नाक॒सद्भि॑रे॒व तद्-यज॑मानः सुव॒र्गं ॅलो॒कमे॑ति सुव॒र्गो वै लो॒को नाको॒ यस्यै॒ता उ॑पधी॒यन्ते॒ नास्मा॒ अकं॑ भवति यजमानायत॒नं ॅवै ना॑क॒सदो॒ यन्ना॑क॒सद॑ उप॒दधा᳚त्या॒यत॑नमे॒व तद्-यज॑मानः कुरुते पृ॒ष्ठानां॒ ॅवा ए॒तत् तेजः॒ संभृ॑तं॒ ॅयन्ना॑क॒सदो॒ यन्ना॑क॒सद॑ - [  ] \newline

\textbf{Pada Paata} \newline

ना॒क॒सद्भि॒रिति॑ नाक॒सत् - भिः॒ । वै । दे॒वाः । सु॒व॒र्गमिति॑ सुवः - गम् । लो॒कम् । आ॒य॒न्न् । तत् । ना॒क॒सदा॒मिति॑ नाक - सदा᳚म् । ना॒क॒स॒त्त्वमिति॑ नाकसत् - त्वम् । यत् । ना॒क॒सद॒ इति॑ नाक - सदः॑ । उ॒प॒दधा॒तीत्यु॑प - दधा॑ति । ना॒क॒सद्भि॒रिति॑ नाक॒सत् - भिः॒ । ए॒व । तत् । यज॑मानः । सु॒व॒र्गमिति॑ सुवः-गम् । लो॒कम् । ए॒ति॒ । सु॒व॒र्ग इति॑ सुवः-गः । वै । लो॒कः । नाकः॑ । यस्य॑ । ए॒ताः । उ॒प॒धी॒यन्त॒ इत्यु॑प - धी॒यन्ते᳚ । न । अ॒स्मै॒ । अक᳚म् । भ॒व॒ति॒ । य॒ज॒मा॒ना॒य॒त॒नमिति॑ यजमान - आ॒य॒त॒नम् । वै । ना॒क॒सद॒ इति॑ नाक - सदः॑ । यत् । ना॒क॒सद॒ इति॑ नाक - सदः॑ । उ॒प॒दधा॒तीत्यु॑प - दधा॑ति । आ॒यत॑न॒मित्या᳚ - यत॑नम् । ए॒व । तत् । यज॑मानः । कु॒रु॒ते॒ । पृ॒ष्ठाना᳚म् । वै । ए॒तत् । तेजः॑ । संभृ॑त॒मिति॒ सं - भृ॒त॒म् । यत् । ना॒क॒सद॒ इति॑ नाक - सदः॑ । यत् । ना॒क॒सद॒ इति॑ नाक - सदः॑ ।  \newline




\markright{ TS 5.3.7.2  \hfill https://www.vedavms.in \hfill}
\addcontentsline{toc}{section}{ TS 5.3.7.2 }
\section*{ TS 5.3.7.2 }

\textbf{TS 5.3.7.2 } \newline
\textbf{Samhita Paata} \newline

उप॒दधा॑ति पृ॒ष्ठाना॑मे॒व तेजोऽव॑ रुन्धे पञ्च॒चोडा॒ उप॑ दधात्यफ्स॒रस॑ ए॒वैन॑मे॒ता भू॒ता अ॒मुष्मि॑न् ॅलो॒क उप॑ शे॒रेऽथो॑ तनू॒पानी॑रे॒वैता यज॑मानस्य॒ यं द्वि॒ष्यात् तमु॑प॒दध॑द्ध्यायेदे॒ताभ्य॑ ए॒वैनं॑ दे॒वता᳚भ्य॒ आ वृ॑श्चति ता॒जगार्ति॒मार्च्छ॒त्युत्त॑रा नाक॒सद्भ्य॒ उप॑दधाति॒ यथा॑ जा॒यामा॒नीय॑ गृ॒हेषु॑ निषा॒दय॑ति ता॒दृगे॒व तत् - [  ] \newline

\textbf{Pada Paata} \newline

उ॒प॒दधा॒तीत्यु॑प - दधा॑ति । पृ॒ष्ठाना᳚म् । ए॒व । तेजः॑ । अवेति॑ । रु॒न्धे॒ । प॒ञ्च॒चोडा॒ इति॑ पञ्च - चोडाः᳚ । उपेति॑ । द॒धा॒ति॒ । अ॒फ्स॒रसः॑ । ए॒व । ए॒न॒म् । ए॒ताः । भू॒ताः । अ॒मुष्मिन्न्॑ । लो॒के । उपेति॑ । शे॒रे॒ । अथो॒ इति॑ । त॒नू॒पानी॒रिति॑ तनू - पानीः᳚ । ए॒व । ए॒ताः । यज॑मानस्य । यम् । द्वि॒ष्यात् । तम् । उ॒प॒दध॒दित्यु॑प - दध॑त् । ध्या॒ये॒त् । ए॒ताभ्यः॑ । ए॒व । ए॒न॒म् । दे॒वता᳚भ्यः । एति॑ । वृ॒श्च॒ति॒ । ता॒जक् । आर्ति᳚म् । एति॑ । ऋ॒च्छ॒ति॒ । उत्त॑रा॒ इत्युत् - त॒राः । ना॒क॒सद्भ्य॒ इति॑ नाक॒सत्-भ्यः॒ । उपेति॑ । द॒धा॒ति॒ । यथा᳚ । जा॒याम् । आ॒नीयेत्या᳚ - नीय॑ । गृ॒हेषु॑ । नि॒षा॒दय॒तीति॑ नि - सा॒दय॑ति । ता॒दृक् । ए॒व । तत् ।  \newline




\markright{ TS 5.3.7.3  \hfill https://www.vedavms.in \hfill}
\addcontentsline{toc}{section}{ TS 5.3.7.3 }
\section*{ TS 5.3.7.3 }

\textbf{TS 5.3.7.3 } \newline
\textbf{Samhita Paata} \newline

प॒श्चात् प्राची॑मुत्त॒मामुप॑ दधाति॒ तस्मा᳚त् प॒श्चात् प्राची॒ पत्न्यन्वा᳚स्ते स्वयमातृ॒ण्णां च॑ विक॒र्णीं चो᳚त्त॒मे उप॑ दधाति प्रा॒णो वै स्व॑यमातृ॒ण्णाऽऽयु॑र्विक॒र्णी प्रा॒णं चै॒वाऽऽ*यु॑श्च प्रा॒णाना॑मुत्त॒मौ ध॑त्ते॒ तस्मा᳚त् प्रा॒णश्चाऽऽ*यु॑श्च प्रा॒णाना॑मुत्त॒मौ नान्यामुत्त॑रा॒मिष्ट॑का॒मुप॑ दद्ध्या॒द्-यद॒न्यामुत्त॑रा॒-मिष्ट॑का-मुपद॒द्ध्यात् प॑शू॒नां - [  ] \newline

\textbf{Pada Paata} \newline

प॒श्चात् । प्राची᳚म् । उ॒त्त॒मामित्यु॑त् - त॒माम् । उपेति॑ । द॒धा॒ति॒ । तस्मा᳚त् । प॒श्चात् । प्राची᳚ । पत्नी᳚ । अन्विति॑ । आ॒स्ते॒ । स्व॒य॒मा॒तृ॒ण्णामिति॑ स्वयं - आ॒तृ॒ण्णाम् । च॒ । वि॒क॒र्णीमिति॑ वि - क॒र्णीम् । च॒ । उ॒त्त॒मे इत्यु॑त् - त॒मे । उपेति॑ । द॒धा॒ति॒ । प्रा॒ण इति॑ प्र - अ॒नः । वै । स्व॒य॒मा॒तृ॒ण्णेति॑ स्वयं - आ॒तृ॒ण्णा । आयुः॑ । वि॒क॒र्णीति॑ वि - क॒र्णी । प्रा॒णमिति॑ प्र - अ॒नम् । च॒ । ए॒व । आयुः॑ । च॒ । प्रा॒णाना॒मिति॑ प्र - अ॒नाना᳚म् । उ॒त्त॒मावित्यु॑त् - त॒मौ । ध॒त्ते॒ । तस्मा᳚त् । प्रा॒ण इति॑ प्र - अ॒नः । च॒ । आयुः॑ । च॒ । प्रा॒णाना॒मिति॑ प्र - अ॒नाना᳚म् । उ॒त्त॒मावित्यु॑त् - त॒मौ । न । अ॒न्याम् । उत्त॑रा॒मित्युत् - त॒रा॒म् । इष्ट॑काम् । उपेति॑ । द॒द्ध्या॒त् । यत् । अ॒न्याम् । उत्त॑रा॒मित्युत् - त॒रा॒म् । इष्ट॑काम् । उ॒प॒द॒द्ध्यादित्यु॑प - द॒द्ध्यात् । प॒शू॒नाम् ।  \newline




\markright{ TS 5.3.7.4  \hfill https://www.vedavms.in \hfill}
\addcontentsline{toc}{section}{ TS 5.3.7.4 }
\section*{ TS 5.3.7.4 }

\textbf{TS 5.3.7.4 } \newline
\textbf{Samhita Paata} \newline

च॒ यज॑मानस्य च प्रा॒णं चाऽऽ*यु॒श्चापि॑ दद्ध्या॒त् तस्मा॒न्ना-न्योत्त॒रेष्ट॑कोप॒धेया᳚ स्वयमातृ॒ण्णामुप॑ दधात्य॒सौ वै स्व॑यमातृ॒ण्णा- ऽमूमे॒वोप॑ ध॒त्ते ऽश्व॒मुप॑ घ्रापयति प्रा॒णमे॒वास्यां᳚ दधा॒त्यथो᳚ प्राजाप॒त्यो वा अश्वः॑ प्र॒जाप॑तिनै॒वाग्निं चि॑नुते स्वयमातृ॒ण्णा भ॑वति प्रा॒णाना॒मुथ्सृ॑ष्ट्या॒ अथो॑ सुव॒र्गस्य॑ लो॒कस्याऽनु॑ख्यात्या ए॒षा वै ( ) दे॒वानां॒ ॅविक्रा᳚न्ति॒र्यद्-वि॑क॒र्णी यद्-वि॑क॒र्णीमु॑प॒दधा॑ति दे॒वाना॑मे॒व विक्रा᳚न्ति॒मनु॒ विक्र॑मत उत्तर॒त उप॑दधाति॒ तस्मा॑दुत्तर॒त उ॑पचारो॒ऽग्नि र्वा॑यु॒मती॑ भवति॒ समि॑द्ध्यै ॥ \newline

\textbf{Pada Paata} \newline

च॒ । यज॑मानस्य । च॒ । प्रा॒णमिति॑ प्र - अ॒नम् । च॒ । आयुः॑ । च॒ । अपीति॑ । द॒द्ध्या॒त् । तस्मा᳚त् । न । अ॒न्या । उत्त॒रेत्युत्-त॒रा॒ । इष्ट॑का । उ॒प॒धेयेत्यु॑प - धेया᳚ । स्व॒य॒मा॒तृ॒ण्णामिति॑ स्वयं - आ॒तृ॒ण्णाम् । उपेति॑ । द॒धा॒ति॒ । अ॒सौ । वै । स्व॒य॒मा॒तृ॒ण्णेति॑ स्वयं - आ॒तृ॒ण्णा । अ॒मूम् । ए॒व । उपेति॑ । ध॒त्ते॒ । अश्व᳚म् । उपेति॑ । घ्रा॒प॒य॒ति॒ । प्रा॒णमिति॑ प्र - अ॒नम् । ए॒व । अ॒स्या॒म् । द॒धा॒ति॒ । अथो॒ इति॑ । प्रा॒जा॒प॒त्य इति॑ प्राजा - प॒त्यः । वै । अश्वः॑ । प्र॒जाप॑ति॒नेति॑ प्र॒जा - प॒ति॒ना॒ । ए॒व । अ॒ग्निम् । चि॒नु॒ते॒ । स्व॒य॒मा॒तृ॒ण्णेति॑ स्वयं - आ॒तृ॒ण्णा । भ॒व॒ति॒ । प्रा॒णाना॒मिति॑ प्र - अ॒नाना᳚म् । उथ्सृ॑ष्ट्या॒ इत्युत् - सृ॒ष्ट्यै॒ । अथो॒ इति॑ । सु॒व॒र्गस्येति॑ सुवः - गस्य॑ । लो॒कस्य॑ । अनु॑ख्यात्या॒ इत्यनु॑ - ख्या॒त्यै॒ । ए॒षा । वै ( ) । दे॒वाना᳚म् । विक्रा᳚न्ति॒रिति॒ वि - क्रा॒न्तिः॒ । यत् । वि॒क॒र्णीति॑ वि - क॒र्णी । यत् । वि॒क॒र्णीमिति॑ वि - क॒र्णीम् । उ॒प॒दधा॒तीत्यु॑प - दधा॑ति । दे॒वाना᳚म् । ए॒व । विक्रा᳚न्ति॒मिति॒ वि- क्रा॒न्ति॒म् । अनु॑ । वीति॑ । क्र॒म॒ते॒ । उ॒त्त॒र॒त इत्यु॑त् - त॒र॒तः । उपेति॑ । द॒धा॒ति॒ । तस्मा᳚त् । उ॒त्त॒र॒त उ॑पचार॒ इत्यु॑त्तर॒तः - उ॒प॒चा॒रः॒ । अ॒ग्निः । वा॒यु॒मतीति॑ वायु - मती᳚ । भ॒व॒ति॒ । समि॑द्ध्या॒ इति॒ सं - इ॒द्ध्यै॒ ॥  \newline




\markright{ TS 5.3.8.1  \hfill https://www.vedavms.in \hfill}
\addcontentsline{toc}{section}{ TS 5.3.8.1 }
\section*{ TS 5.3.8.1 }

\textbf{TS 5.3.8.1 } \newline
\textbf{Samhita Paata} \newline

छन्दाꣳ॒॒स्युप॑ दधाति प॒शवो॒ वै छन्दाꣳ॑सि प॒शूने॒वाव॑ रुन्धे॒ छन्दाꣳ॑सि॒ वै दे॒वानां᳚ ॅवा॒मं प॒शवो॑ वा॒ममे॒व प॒शूनव॑ रुन्ध ए॒ताꣳ ह॒ वै य॒ज्ञ्से॑न-श्चैत्रियाय॒ण-श्चितिं॑ ॅवि॒दां च॑कार॒ तया॒ वै स प॒शूनवा॑रुन्ध॒ यदे॒तामु॑प॒दधा॑ति प॒शूने॒वाव॑ रुन्धे गाय॒त्रीः पु॒रस्ता॒दुप॑ दधाति॒ तेजो॒ वै गा॑य॒त्री तेज॑ ए॒व - [  ] \newline

\textbf{Pada Paata} \newline

छन्दाꣳ॑सि । उपेति॑ । द॒धा॒ति॒ । प॒शवः॑ । वै । छन्दाꣳ॑सि । प॒शून् । ए॒व । अवेति॑ । रु॒न्धे॒ । छन्दाꣳ॑सि । वै । दे॒वाना᳚म् । वा॒मम् । प॒शवः॑ । वा॒मम् । ए॒व । प॒शून् । अवेति॑ । रु॒न्धे॒ । ए॒ताम् । ह॒ । वै । य॒ज्ञ्से॑न॒ इति॑ य॒ज्ञ् - से॒नः॒ । चै॒त्रि॒या॒य॒णः । चिति᳚म् । वि॒दाम् । च॒का॒र॒ । तया᳚ । वै । सः । प॒शून् । अवेति॑ । अ॒रु॒न्ध॒ । यत् । ए॒ताम् । उ॒प॒दधा॒तीत्यु॑प - द॒धा॑ति । प॒शून् । ए॒व । अवेति॑ । रु॒न्धे॒ । गा॒य॒त्रीः । पु॒रस्ता᳚त् । उपेति॑ । द॒धा॒ति॒ । तेजः॑ । वै । गा॒य॒त्री । तेजः॑ । ए॒व ।  \newline




\markright{ TS 5.3.8.2  \hfill https://www.vedavms.in \hfill}
\addcontentsline{toc}{section}{ TS 5.3.8.2 }
\section*{ TS 5.3.8.2 }

\textbf{TS 5.3.8.2 } \newline
\textbf{Samhita Paata} \newline

मु॑ख॒तो ध॑त्ते मूर्द्ध॒न्वती᳚र्भवन्ति मू॒र्द्धान॑मे॒वैनꣳ॑ समा॒नानां᳚ करोति त्रि॒ष्टुभ॒ उप॑ दधातीन्द्रि॒यं ॅवै त्रि॒ष्टुगि॑न्द्रि॒यमे॒व म॑द्ध्य॒तो ध॑त्ते॒ जग॑ती॒रुप॑ दधाति॒ जाग॑ता॒ वै प॒शवः॑ प॒शूने॒वाव॑ रुन्धे ऽनु॒ष्टुभ॒ उप॑ दधाति प्रा॒णा वा अ॑नु॒ष्टुप् प्रा॒णाना॒मुथ्सृ॑ष्ट्यै बृह॒तीरु॒ष्णिहाः᳚ प॒ङ्क्तीर॒क्षर॑पङ्क्ती॒रिति॒ विषु॑रूपाणि॒ छन्दाꣳ॒॒स्युप॑ दधाति॒ विषु॑रूपा॒ वै प॒शवः॑ प॒शवः॒ - [  ] \newline

\textbf{Pada Paata} \newline

मु॒ख॒तः । ध॒त्ते॒ । मू॒द्‌र्ध॒न्वती॒रिति॑ मूर्धन्न् - वतीः᳚ । भ॒व॒न्ति॒ । मू॒द्‌र्धान᳚म् । ए॒व । ए॒न॒म् । स॒मा॒नाना᳚म् । क॒रो॒ति॒ । त्रि॒ष्टुभः॑ । उपेति॑ । द॒धा॒ति॒ । इ॒न्द्रि॒यम् । वै । त्रि॒ष्टुक् । इ॒न्द्रि॒यम् । ए॒व । म॒द्ध्य॒तः । ध॒त्ते॒ । जग॑तीः । उपेति॑ । द॒धा॒ति॒ । जाग॑ताः । वै । प॒शवः॑ । प॒शून् । ए॒व । अवेति॑ । रु॒न्धे॒ । अ॒नु॒ष्टुभ॒ इत्य॑नु - स्तुभः॑ । उपेति॑ । द॒धा॒ति॒ । प्रा॒णा इति॑ प्र - अ॒नाः । वै । अ॒नु॒ष्टुबित्य॑नु - स्तुप् । प्रा॒णाना॒मिति॑ प्र -  अ॒नाना᳚म् । उथ्सृ॑ष्ट्या॒ इत्युत् - सृ॒ष्ट्यै॒ । बृ॒ह॒तीः । उ॒ष्णिहाः᳚ । प॒ङ्क्तीः । अ॒क्षर॑पङ्क्ती॒रित्य॒क्षर॑ - प॒ङ्क्तीः॒ । इति॑ । विषु॑रूपा॒णीति॒ विषु॑ - रू॒पा॒णि॒ । छन्दाꣳ॑सि । उपेति॑ । द॒धा॒ति॒ । विषु॑रूपा॒ इति॒ विषु॑ - रू॒पाः॒ । वै । प॒शवः॑ । प॒शवः॑ ।  \newline




\markright{ TS 5.3.8.3  \hfill https://www.vedavms.in \hfill}
\addcontentsline{toc}{section}{ TS 5.3.8.3 }
\section*{ TS 5.3.8.3 }

\textbf{TS 5.3.8.3 } \newline
\textbf{Samhita Paata} \newline

-छन्दाꣳ॑सि॒ विषु॑रूपाने॒व प॒शूनव॑ रुन्धे॒ विषु॑रूपमस्य गृ॒हे दृ॑श्यते॒ यस्यै॒ता उ॑पधी॒यन्ते॒ य उ॑ चैना ए॒वं ॅवेदाऽ*ति॑च्छन्दस॒मुप॑ दधा॒त्यति॑च्छन्दा॒ वै सर्वा॑णि॒ छन्दाꣳ॑सि॒ सर्वे॑भिरे॒वैनं॒ छन्दो॑भिश्चिनुते॒ वर्ष्म॒ वा ए॒षा छन्द॑सां॒ ॅयदति॑च्छन्दा॒ यदति॑च्छन्दस-मुप॒दधा॑ति॒ वर्ष्मै॒वैनꣳ॑ समा॒नानां᳚ करोति द्वि॒पदा॒ उप॑ दधाति द्वि॒पाद्-यज॑मानः॒ ( ) प्रति॑ष्ठित्यै ॥ \newline

\textbf{Pada Paata} \newline

छन्दाꣳ॑सि । विषु॑रूपा॒निति॒ विषु॑ - रू॒पा॒न् । ए॒व । प॒शून् । अवेति॑ । रु॒न्धे॒ । विषु॑रूप॒मिति॒ विषु॑-रू॒प॒म् । अ॒स्य॒ । गृ॒हे । दृ॒श्य॒ते॒ । यस्य॑ । ए॒ताः । उ॒प॒धी॒यन्त॒ इत्यु॑प - धी॒यन्ते᳚ । यः । उ॒ । च॒ । ए॒नाः॒ । ए॒वम् । वेद॑ । अति॑च्छन्दस॒मित्यति॑ - छ॒न्द॒स॒म् । उपेति॑ । द॒धा॒ति॒ । अति॑च्छन्दा॒ इत्यति॑ - छ॒न्दाः॒ । वै । सर्वा॑णि । छन्दाꣳ॑सि । सर्वे॑भिः । ए॒व । ए॒न॒म् । छन्दो॑भि॒रिति॒ छन्दः॑-भिः॒ । चि॒नु॒ते॒ । वर्ष्म॑ । वै । ए॒षा । छन्द॑साम् । यत् । अति॑च्छन्दा॒ इत्यति॑ - छ॒न्दाः॒ । यत् । अति॑च्छन्दस॒मित्यति॑ - छ॒न्द॒स॒म् । उ॒प॒दधा॒तीत्यु॑प - दधा॑ति । वर्ष्म॑ । ए॒व । ए॒न॒म् । स॒मा॒नाना᳚म् । क॒रो॒ति॒ । द्वि॒पदा॒ इति॑ द्वि - पदाः᳚ । उपेति॑ । द॒धा॒ति॒ । द्वि॒पादिति॑ द्वि - पात् । यज॑मानः ( ) । प्रति॑ष्ठित्या॒ इति॒ प्रति॑ - स्थि॒त्यै॒ ॥  \newline




\markright{ TS 5.3.9.1  \hfill https://www.vedavms.in \hfill}
\addcontentsline{toc}{section}{ TS 5.3.9.1 }
\section*{ TS 5.3.9.1 }

\textbf{TS 5.3.9.1 } \newline
\textbf{Samhita Paata} \newline

सर्वा᳚भ्यो॒ वै दे॒वता᳚भ्यो॒ऽग्निश्ची॑यते॒ यथ् स॒युजो॒ नोप॑द॒द्ध्याद् दे॒वता॑ अस्या॒ग्निं ॅवृ॑ञ्जीर॒न्॒. यथ् स॒युज॑ उप॒दधा᳚त्या॒त्मनै॒वैनꣳ॑ स॒युजं॑ चिनुते॒ नाग्निना॒ व्यृ॑द्ध्य॒तेऽथो॒ यथा॒ पुरु॑षः॒ स्नाव॑भिः॒ संत॑त ए॒वमे॒वैताभि॑र॒ग्निः संत॑तो॒ ऽग्निना॒ वै दे॒वाः सु॑व॒र्गं ॅलो॒कमा॑य॒न् ता अ॒मूः कृत्ति॑का अभव॒न्॒ यस्यै॒ता उ॑प धी॒यन्ते॑ सुव॒र्गमे॒व - [  ] \newline

\textbf{Pada Paata} \newline

सर्वा᳚भ्यः । वै । दे॒वता᳚भ्यः । अ॒ग्निः । ची॒य॒ते॒ । यत् । स॒युज॒ इति॑ स - युजः॑ । न । उ॒प॒द॒द्ध्यादित्यु॑प - द॒ध्यात् । दे॒वताः᳚ । अ॒स्य॒ । अ॒ग्निम् । वृ॒ञ्जी॒र॒न्न् । यत् । स॒युज॒ इति॑ स - युजः॑ । उ॒प॒दधा॒तीत्यु॑प - दधा॑ति । आ॒त्मना᳚ । ए॒व । ए॒न॒म् । स॒युज॒मिति॑ स-युज᳚म् । चि॒नु॒ते॒ । न । अ॒ग्निना᳚ । वीति॑ । ऋ॒द्ध्य॒ते॒ । अथो॒ इति॑ । यथा᳚ । पुरु॑षः । स्नाव॑भि॒रिति॒ स्नाव॑ - भिः॒ । संत॑त॒ इति॒ सं-त॒तः॒ । ए॒वम् । ए॒व । ए॒ताभिः॑ । अ॒ग्निः । संत॑त॒ इति॒ सं-त॒तः॒ । अ॒ग्निना᳚ । वै । दे॒वाः । सु॒व॒र्गमिति॑ सुवः - गम् । लो॒कम् । आ॒य॒न्न् । ताः । अ॒मूः । कृत्ति॑काः । अ॒भ॒व॒न्न् । यस्य॑ । ए॒ताः । उ॒प॒धी॒यन्त॒ इत्यु॑प - धी॒यन्ते᳚ । सु॒व॒र्गमिति॑ सुवः - गम् । ए॒व ।  \newline




\markright{ TS 5.3.9.2  \hfill https://www.vedavms.in \hfill}
\addcontentsline{toc}{section}{ TS 5.3.9.2 }
\section*{ TS 5.3.9.2 }

\textbf{TS 5.3.9.2 } \newline
\textbf{Samhita Paata} \newline

लो॒कमे॑ति॒ गच्छ॑ति प्रका॒शं चि॒त्रमे॒व भ॑वति मण्डलेष्ट॒का उप॑ दधाती॒मे वै लो॒का म॑ण्डलेष्ट॒का इ॒मे खलु॒ वै लो॒का दे॑वपु॒रा दे॑वपु॒रा ए॒व प्रवि॑शति॒ नाऽऽ*र्ति॒मार्च्छ॑त्य॒ग्निं चि॑क्या॒नो वि॒श्वज्यो॑तिष॒ उप॑ दधाती॒माने॒वैताभि-॑र्लो॒कान् ज्योति॑ष्मतः कुरु॒तेऽथो᳚ प्रा॒णाने॒वैता यज॑मानस्य दाद्ध्रत्ये॒ता वै दे॒वताः᳚ सुव॒र्ग्या᳚स्ता ए॒वा- ( ) -न्वा॒रभ्य॑ सुव॒र्गं ॅलो॒कमे॑ति ॥ \newline

\textbf{Pada Paata} \newline

लो॒कम् । ए॒ति॒ । गच्छ॑ति । प्र॒का॒शमिति॑ प्र-का॒शम् । चि॒त्रम् । ए॒व । भ॒व॒ति॒ । म॒ण्ड॒ले॒ष्ट॒का इति॑ मण्डल - इ॒ष्ट॒काः । उपेति॑ । द॒धा॒ति॒ । इ॒मे । वै । लो॒काः । म॒ण्ड॒ले॒ष्ट॒का इति॑ मण्डल - इ॒ष्ट॒काः । इ॒मे । खलु॑ । वै । लो॒काः । दे॒व॒पु॒रा इति॑ देव-पु॒राः । दे॒व॒पु॒रा इति॑-पु॒राः । ए॒व । प्रेति॑ । वि॒श॒ति॒ । न । आर्ति᳚म् । एति॑ । ऋ॒च्छ॒ति॒ । अ॒ग्निम् । चि॒क्या॒नः । वि॒श्वज्यो॑तिष॒ इति॑ वि॒श्व - ज्यो॒ति॒षः॒ । उपेति॑ । द॒धा॒ति॒ । इ॒मान् । ए॒व । एताभिः॑ । लो॒कान् । ज्योति॑ष्मतः । कु॒रु॒ते॒ । अथो॒ इति॑ । प्रा॒णानिति॑ प्र - अ॒नान् । ए॒व । ए॒ताः । यज॑मानस्य । दा॒द्ध्र॒ति॒ । ए॒ताः । वै । दे॒वताः᳚ । सु॒व॒र्ग्या॑ इति॑ सुवः - ग्याः᳚ । ताः । ए॒व ( ) । अ॒न्वा॒रभ्येत्य॑नु - आ॒रभ्य॑ । सु॒व॒र्गमिति॑ सुवः - गम् । लो॒कम् । ए॒ति॒ ॥  \newline




\markright{ TS 5.3.10.1  \hfill https://www.vedavms.in \hfill}
\addcontentsline{toc}{section}{ TS 5.3.10.1 }
\section*{ TS 5.3.10.1 }

\textbf{TS 5.3.10.1 } \newline
\textbf{Samhita Paata} \newline

वृ॒ष्टि॒सनी॒रुप॑ दधाति॒ वृष्टि॑मे॒वाव॑ रुन्धे॒ यदे॑क॒धोप॑द॒द्ध्यादेक॑मृ॒तुं व॑र्.षेदनुपरि॒हारꣳ॑ सादयति॒ तस्मा॒थ् सर्वा॑नृ॒तून्. व॑र्.षति पुरोवात॒सनि॑-र॒सीत्या॑है॒तद्वै वृष्ट्यै॑ रू॒पꣳ रू॒पेणै॒व वृष्टि॒मव॑ रुन्धे सं॒ॅयानी॑भि॒र्वै दे॒वा इ॒मान् ॅलो॒कान्थ् सम॑यु॒स्तथ् सं॒ॅयानी॑नाꣳ संॅयानि॒त्वं ॅयथ् सं॒ॅयानी॑रुप॒दधा॑ति॒ यथा॒ऽफ्सु ना॒वा सं॒ॅयात्ये॒व - [  ] \newline

\textbf{Pada Paata} \newline

वृ॒ष्टि॒सनी॒रिति॑ वृष्टि - सनीः᳚ । उपेति॑ । द॒धा॒ति॒ । वृष्टि᳚म् । ए॒व । अवेति॑ । रु॒न्धे॒ । यत् । ए॒क॒धेत्ये॑क - धा । उ॒प॒द॒द्ध्यादित्यु॑प - द॒द्ध्यात् । एक᳚म् । ऋ॒तुम् । व॒र्.॒षे॒त् । अ॒नु॒प॒रि॒हार॒मित्य॑नु - प॒रि॒हार᳚म् । सा॒द॒य॒ति॒ । तस्मा᳚त् । सर्वान्॑ । ऋ॒तून् । व॒र्.॒ष॒ति॒ । पु॒रो॒वा॒त॒सनि॒रिति॑ पुरोवात - सनिः॑ । अ॒सि॒ । इति॑ । आ॒ह॒ । ए॒तत् । वै । वृष्ट्यै᳚ । रू॒पम् । रू॒पेण॑ । ए॒व । वृष्टि᳚म् । अवेति॑ । रु॒न्धे॒ । सं॒ॅयानी॑भि॒रिति॑ सं - यानी॑भिः । वै । दे॒वाः । इ॒मान् । लो॒कान् । समिति॑ । अ॒युः॒ । तत् । सं॒ॅयानी॑ना॒मिति॑ सं - यानी॑नाम् । सं॒ॅया॒नि॒त्वमिति॑ संॅयानि - त्वम् । यत् । सं॒ॅयानी॒रिति॑ सं - यानीः᳚ । उ॒प॒दधा॒तीत्यु॑प - दधा॑ति । यथा᳚ । अ॒फ्स्वित्य॑प् - सु । ना॒वा । सं॒ॅयातीति॑ सं - याति॑ । ए॒वम् ।  \newline




\markright{ TS 5.3.10.2  \hfill https://www.vedavms.in \hfill}
\addcontentsline{toc}{section}{ TS 5.3.10.2 }
\section*{ TS 5.3.10.2 }

\textbf{TS 5.3.10.2 } \newline
\textbf{Samhita Paata} \newline

-मे॒वैताभि॒ र्यज॑मान इ॒मान् ॅलो॒कान्थ् सं ॅया॑ति प्ल॒वो वा ए॒षो᳚ऽग्नेर्यथ् सं॒ॅयानी॒र्यथ् सं॒ॅयानी॑रुप॒दधा॑ति प्ल॒वमे॒वैतम॒ग्नय॒ उप॑दधात्यु॒त यस्यै॒तासूप॑हिता॒स्वापो॒ऽग्निꣳ हर॒न्त्यहृ॑त ए॒वास्या॒-ग्निरा॑दित्येष्ट॒का उप॑ दधात्यादि॒त्या वा ए॒तं भूत्यै॒ प्रति॑नुदन्ते॒ योऽलं॒ भूत्यै॒ सन् भूतिं॒ न प्रा॒प्नोत्या॑दि॒त्या - [  ] \newline

\textbf{Pada Paata} \newline

ए॒व । ए॒ताभिः॑ । यज॑मानः । इ॒मान् । लो॒कान् । समिति॑ । या॒ति॒ । प्ल॒वः । वै । ए॒षः । अ॒ग्नेः । यत् । सं॒ॅयानी॒रिति॑ सं - यानीः᳚ । यत् । सं॒ॅयानी॒रिति॑ सं - यानीः᳚ । उ॒प॒दधा॒तीत्यु॑प-दधा॑ति । प्ल॒वम् । ए॒व । ए॒तम् । अ॒ग्नये᳚ । उपेति॑ । द॒धा॒ति॒ । उ॒त । यस्य॑ । ए॒तासु॑ । उप॑हिता॒स्वित्युप॑ - हि॒ता॒सु॒ । आपः॑ । अ॒ग्निम् । हर॑न्ति । अहृ॑तः । ए॒व । अ॒स्य॒ । अ॒ग्निः । आ॒दि॒त्ये॒ष्ट॒का इत्या॑दित्य - इ॒ष्ट॒काः । उपेति॑ । द॒धा॒ति॒ । आ॒दि॒त्याः । वै । ए॒तम् । भूत्यै᳚ । प्रतीति॑ । नु॒द॒न्ते॒ । यः । अल᳚म् । भूत्यै᳚ । सन्न् । भूति᳚म् । न । प्रा॒प्नोतीति॑ प्र - आ॒प्नोति॑ । आ॒दि॒त्याः ।  \newline




\markright{ TS 5.3.10.3  \hfill https://www.vedavms.in \hfill}
\addcontentsline{toc}{section}{ TS 5.3.10.3 }
\section*{ TS 5.3.10.3 }

\textbf{TS 5.3.10.3 } \newline
\textbf{Samhita Paata} \newline

ए॒वैनं॒ भूतिं॑ गमयन्त्य॒सौ वा ए॒तस्या॑ऽऽ*दि॒त्यो रुच॒मा द॑त्ते॒ यो᳚ऽग्निं चि॒त्वा न रोच॑ते॒ यदा॑दित्येष्ट॒का उ॑प॒दधा᳚त्य॒सावे॒-वास्मि॑न्नादि॒त्यो रुचं॑ दधाति॒ यथा॒ऽसौ दे॒वानाꣳ॒॒ रोच॑त ए॒वमे॒वैष म॑नु॒ष्या॑णाꣳ रोचते घृतेष्ट॒का उप॑ दधात्ये॒तद्वा अ॒ग्नेः प्रि॒यं धाम॒ यद्-घृ॒तं प्रि॒येणै॒वैनं॒ धाम्ना॒ सम॑र्द्धय॒त्य - [  ] \newline

\textbf{Pada Paata} \newline

ए॒व । ए॒न॒म् । भूति᳚म् । ग॒म॒य॒न्ति॒ । अ॒सौ । वै । ए॒तस्य॑ । आ॒दि॒त्यः । रुच᳚म् । एति॑ । द॒त्ते॒ । यः । अ॒ग्निम् । चि॒त्वा । न । रोच॑ते । यत् । आ॒दि॒त्ये॒ष्ट॒का इत्या॑दित्य - इ॒ष्ट॒काः । उ॒प॒दधा॒तीत्यु॑प - दधा॑ति । अ॒सौ । ए॒व । अ॒स्मि॒न्न् । आ॒दि॒त्यः । रुच᳚म् । द॒धा॒ति॒ । यथा᳚ । अ॒सौ । दे॒वाना᳚म् । रोच॑ते । ए॒वम् । ए॒व । ए॒षः । म॒नु॒ष्या॑णाम् । रो॒च॒ते॒ । घृ॒ते॒ष्ट॒का इति॑ घृत - इ॒ष्ट॒काः । उपेति॑ । द॒धा॒ति॒ । ए॒तत् । वै । अ॒ग्नेः । प्रि॒यम् । धाम॑ । यत् । घृ॒तम् । प्रि॒येण॑ । ए॒व । ए॒न॒म् । धाम्ना᳚ । समिति॑ । अ॒द्‌र्ध॒य॒ति॒ ।  \newline




\markright{ TS 5.3.10.4  \hfill https://www.vedavms.in \hfill}
\addcontentsline{toc}{section}{ TS 5.3.10.4 }
\section*{ TS 5.3.10.4 }

\textbf{TS 5.3.10.4 } \newline
\textbf{Samhita Paata} \newline

-थो॒ तेज॑सा ऽनुपरि॒हारꣳ॑ सादय॒-त्यप॑रिवर्ग-मे॒वास्मि॒न् तेजो॑ दधाति प्र॒जाप॑तिर॒ग्निम॑चिनुत॒ स यश॑सा॒ व्या᳚र्द्ध्यत॒ स ए॒ता य॑शो॒दा अ॑पश्य॒त् ता उपा॑धत्त॒ ताभि॒र्वै स यश॑ आ॒त्मन्न॑धत्त॒ यद्य॑शो॒दा उ॑प॒दधा॑ति॒ यश॑ ए॒व ताभि॒र्यज॑मान आ॒त्मन् ध॑त्ते॒ पञ्चोप॑ दधाति॒ पाङ्क्तः॒ पुरु॑षो॒ यावा॑ने॒व पुरु॑ष॒स्तस्मि॒न्॒ यशो॑ दधाति ॥ \newline

\textbf{Pada Paata} \newline

अथो॒ इति॑ । तेज॑सा । अ॒नु॒प॒रि॒हार॒मित्य॑नु - प॒रि॒हार᳚म् । सा॒द॒य॒ति॒ । अप॑रिवर्ग॒मित्यप॑रि - व॒र्ग॒म् । ए॒व । अ॒स्मि॒न्न् । तेजः॑ । द॒धा॒ति॒ । प्र॒जाप॑ति॒रिति॑ प्र॒जा - प॒तिः॒ । अ॒ग्निम् । अ॒चि॒नु॒त॒ । सः । यश॑सा । वीति॑ । आ॒द्‌र्ध्य॒त॒ । सः । ए॒ताः । य॒शो॒दा इति॑ यशः - दाः । अ॒प॒श्य॒त् । ताः । उपेति॑ । अ॒ध॒त्त॒ । ताभिः॑ । वै । सः । यशः॑ । आ॒त्मन्न् । अ॒ध॒त्त॒ । यत् । य॒शो॒दा इति॑ यशः - दाः । उ॒प॒दधा॒तीत्यु॑प - दधा॑ति । यशः॑ । ए॒व । ताभिः॑ । यज॑मानः । आ॒त्मन्न् । ध॒त्ते॒ । पञ्च॑ । उपेति॑ । द॒धा॒ति॒ । पाङ्क्तः॑ । पुरु॑षः । यावान्॑ । ए॒व । पुरु॑षः । तस्मिन्न्॑ । यशः॑ । द॒धा॒ति॒ ॥  \newline




\markright{ TS 5.3.11.1  \hfill https://www.vedavms.in \hfill}
\addcontentsline{toc}{section}{ TS 5.3.11.1 }
\section*{ TS 5.3.11.1 }

\textbf{TS 5.3.11.1 } \newline
\textbf{Samhita Paata} \newline

दे॒वा॒सु॒राः संॅय॑त्ता आस॒न् कनी॑याꣳसो दे॒वा आस॒न् भूयाꣳ॒॒सो-ऽसु॑रा॒स्ते दे॒वा ए॒ता इष्ट॑का अपश्य॒न् ता उपा॑दधत भूय॒स्कृद॒सीत्ये॒व भूयाꣳ॑सोऽभव॒न् वन॒स्पति॑भि॒-रोष॑धीभि-र्वरिव॒स्कृद॒सीती॒-माम॑जय॒न् प्राच्य॒सीति॒ प्राचीं॒ दिश॑मजयन्नू॒र्द्ध्वा ऽसीत्य॒मूम॑जयन्-नन्तरिक्ष॒सद॑स्य॒न्तरि॑क्षे सी॒देत्य॒-न्तरि॑क्षमजय॒न् ततो॑ दे॒वा अभ॑व॒न् - [  ] \newline

\textbf{Pada Paata} \newline

दे॒वा॒सु॒रा इति॑ देव-अ॒सु॒राः । संॅय॑त्ता॒ इति॒ सं - य॒त्ताः॒ । आ॒स॒न्न् । कनी॑याꣳसः । दे॒वाः । आसन्न्॑ । भूयाꣳ॑सः । असु॑राः । ते । दे॒वाः । ए॒ताः । इष्ट॑काः । अ॒प॒श्य॒न्न् । ताः । उपेति॑ । अ॒द॒ध॒त॒ । भू॒य॒स्कृदिति॑ भूयः - कृत् । अ॒सि॒ । इति॑ । ए॒व । भूयाꣳ॑सः । अ॒भ॒व॒न्न् । वन॒स्पति॑भि॒रिति॒ वन॒स्पति॑ - भिः॒ । ओष॑धीभि॒रित्योष॑धि - भिः॒ । व॒रि॒व॒स्कृदिति॑ वरिवः - कृत् । अ॒सि॒ । इति॑ । इ॒माम् । अ॒ज॒य॒न्न् । प्राची᳚ । अ॒सि॒ । इति॑ । प्राची᳚म् । दिश᳚म् । अ॒ज॒य॒न्न् । ऊ॒द्‌र्ध्वा । अ॒सि॒ । इति॑ । अ॒मूम् । अ॒ज॒य॒न्न् । अ॒न्त॒रि॒क्ष॒सदित्य॑न्तरिक्ष - सत् । अ॒सि॒ । अ॒न्तरि॑क्षे । सी॒द॒ । इति॑ । अ॒न्तरि॑क्षम् । अ॒ज॒य॒न्न् । ततः॑ । दे॒वाः । अभ॑वन्न् ।  \newline




\markright{ TS 5.3.11.2  \hfill https://www.vedavms.in \hfill}
\addcontentsline{toc}{section}{ TS 5.3.11.2 }
\section*{ TS 5.3.11.2 }

\textbf{TS 5.3.11.2 } \newline
\textbf{Samhita Paata} \newline

पराऽसु॑रा॒ यस्यै॒ता उ॑पधी॒यन्ते॒ भूया॑ने॒व भ॑वत्य॒भीमान् ॅलो॒कान् ज॑यति॒ भव॑त्या॒त्मना॒ परा᳚ऽस्य॒ भ्रातृ॑व्यो भवत्यफ्सु॒षद॑सि श्येन॒सद॒सीत्या॑है॒तद्वा अ॒ग्ने रू॒पꣳ रू॒पेणै॒वाग्निमव॑ रुन्धे पृथि॒व्यास्त्वा॒ द्रवि॑णे सादया॒मी-त्या॑हे॒माने॒वैताभि॑-र्लो॒कान् द्रवि॑णावतः कुरुत आयु॒ष्या॑ उप॑ दधा॒त्यायु॑रे॒वा - [  ] \newline

\textbf{Pada Paata} \newline

परेति॑ । असु॑राः । यस्य॑ । ए॒ताः । उ॒प॒धी॒यन्त॒ इत्यु॑प - धी॒यन्ते᳚ । भूयान्॑ । ए॒व । भ॒व॒ति॒ । अ॒भीति॑ । इ॒मान् । लो॒कान् । ज॒य॒ति॒ । भव॑ति । आ॒त्मना᳚ । परेति॑ । अ॒स्य॒ । भ्रातृ॑व्यः । भ॒व॒ति॒ । अ॒फ्सु॒षदित्य॑फ्सु - सत् । अ॒सि॒ । श्ये॒न॒सदिति॑ श्येन - सत् । अ॒सि॒ । इति॑ । आ॒ह॒ । ए॒तत् । वै । अ॒ग्नेः । रू॒पम् । रू॒पेण॑ । ए॒व । अ॒ग्निम् । अवेति॑ । रु॒न्धे॒ । पृ॒थि॒व्याः । त्वा॒ । द्रवि॑णे । सा॒द॒या॒मि॒ । इति॑ । आ॒ह॒ । इ॒मान् । ए॒व । ए॒ताभिः॑ । लो॒कान् । द्रवि॑णावत॒ इति॒ द्रवि॑ण - व॒तः॒ । कु॒रु॒ते॒ । आ॒यु॒ष्याः᳚ । उपेति॑ । द॒धा॒ति॒ । आयुः॑ । ए॒व ।  \newline




\markright{ TS 5.3.11.3  \hfill https://www.vedavms.in \hfill}
\addcontentsline{toc}{section}{ TS 5.3.11.3 }
\section*{ TS 5.3.11.3 }

\textbf{TS 5.3.11.3 } \newline
\textbf{Samhita Paata} \newline

-स्मि॑न् दधा॒त्यग्ने॒ यत्ते॒ परꣳ॒॒ हृन्नामेत्या॑है॒तद्वा अ॒ग्नेः प्रि॒यं धाम॑ प्रि॒यमे॒वास्य॒ धामोपा᳚ऽऽ*प्नोति॒ तावेहि॒ सꣳ र॑भावहा॒ इत्या॑ह॒ व्ये॑वैने॑न॒ परि॑ धत्ते॒ पाञ्च॑जन्ये॒ष्वप्ये᳚द्ध्यग्न॒ इत्या॑है॒ष वा अ॒ग्निः पाञ्च॑जन्यो॒ यः पञ्च॑चितीक॒-स्तस्मा॑दे॒वमा॑हर्त॒व्या॑ उप॑ ( ) दधात्ये॒तद्वा ऋ॑तू॒नां प्रि॒यं धाम॒ यदृ॑त॒व्या॑ ऋतू॒नामे॒व प्रि॒यं धामाव॑ रुन्धे सु॒मेक॒ इत्या॑ह संॅवथ्स॒रो वै सु॒मेकः॑ संॅवथ्स॒रस्यै॒व प्रि॒यं धामोपा᳚ऽऽ*प्नोति ॥ \newline

\textbf{Pada Paata} \newline

अ॒स्मि॒न्न् । द॒धा॒ति॒ । अग्ने᳚ । यत् । ते॒ । पर᳚म् । हृत् । नाम॑ । इति॑ । आ॒ह॒ । ए॒तत् । वै । अ॒ग्नेः । प्रि॒यम् । धाम॑ । प्रि॒यम् । ए॒व । अ॒स्य॒ । धाम॑ । उपेति॑ । आ॒प्नो॒ति॒ । तौ । एति॑ । इ॒हि॒ । समिति॑ । र॒भा॒व॒है॒ । इति॑ । आ॒ह॒ । वीति॑ । ए॒व । ए॒ने॒न॒ । परीति॑ । ध॒त्ते॒ । पाञ्च॑जन्ये॒ष्विति॒ पाञ्च॑ - ज॒न्ये॒षु॒ । अपीति॑ । ए॒धि॒ । अ॒ग्ने॒ । इति॑ । आ॒ह॒ । ए॒षः । वै । अ॒ग्निः । पाञ्च॑जन्य॒ इति॒ पाञ्च॑ - ज॒न्यः॒ । यः । पञ्च॑चितीक॒ इति॒ पञ्च॑-चि॒ती॒कः॒ । तस्मा᳚त् । ए॒वम् । आ॒ह॒ । ऋ॒त॒व्याः᳚ । उपेति॑ ( ) । द॒धा॒ति॒ । ए॒तत् । वै । ऋ॒तू॒नाम् । प्रि॒यम् । धाम॑ । यत् । ऋ॒त॒व्याः᳚ । ऋ॒तू॒नाम् । ए॒व । प्रि॒यम् । धाम॑ । अवेति॑ । रु॒न्धे॒ । सु॒मेक॒ इति॑ सु - मेकः॑ । इति॑ । आ॒ह॒ । सं॒ॅव॒थ्स॒र इति॑ सं - व॒थ्स॒रः । वै । सु॒मेक॒ इति॑ सु - मेकः॑ । सं॒ॅव॒थ्स॒रस्येति॑ सं - व॒थ्स॒रस्य॑ । ए॒व । प्रि॒यम् । धाम॑ । उपेति॑ । आ॒प्नो॒ति॒ ॥  \newline




\markright{ TS 5.3.12.1  \hfill https://www.vedavms.in \hfill}
\addcontentsline{toc}{section}{ TS 5.3.12.1 }
\section*{ TS 5.3.12.1 }

\textbf{TS 5.3.12.1 } \newline
\textbf{Samhita Paata} \newline

प्र॒जाप॑ते॒रक्ष्य॑श्वय॒त् तत् परा॑ऽपत॒त् तदश्वो॑ऽभव॒द्-यदश्व॑य॒त् तदश्व॑स्याश्व॒त्वं तद्दे॒वा अ॑श्वमे॒धेनै॒व प्रत्य॑दधुरे॒ष वै प्र॒जाप॑तिꣳ॒॒ सर्वं॑ करोति॒ यो᳚ऽश्वमे॒धेन॒ यज॑ते॒ सर्व॑ ए॒व भ॑वति॒ सर्व॑स्य॒ वा ए॒षा प्राय॑श्चित्तिः॒ सर्व॑स्य भेष॒जꣳ सर्वं॒ ॅवा ए॒तेन॑ पा॒प्मानं॑ दे॒वा अ॑तर॒न्नपि॒ वा ए॒तेन॑ ब्रह्मह॒त्याम॑तर॒न्थ् सर्वं॑ पा॒प्मानं॑ - [  ] \newline

\textbf{Pada Paata} \newline

प्र॒जाप॑ते॒रिति॑ प्र॒जा - प॒तेः॒ । अक्षि॑ । अ॒श्व॒य॒त् । तत् । परेति॑ । अ॒प॒त॒त् । तत् । अश्वः॑ । अ॒भ॒व॒त् । यत् । अश्व॑यत् । तत् । अश्व॑स्य । अ॒श्व॒त्वमित्य॑श्व- त्वम् । तत् । दे॒वाः । अ॒श्व॒मे॒धेनेत्य॑श्व - मे॒धेन॑ । ए॒व । प्रतीति॑ । अ॒द॒धुः॒ । ए॒षः । वै । प्र॒जाप॑ति॒मिति॑ प्र॒जा - प॒ति॒म् । सर्व᳚म् । क॒रो॒ति॒ । यः । अ॒श्व॒मे॒धेनेत्य॑श्व - मे॒धेन॑ । यज॑ते । सर्वः॑ । ए॒व । भ॒व॒ति॒ । सर्व॑स्य । वै । ए॒षा । प्राय॑श्चित्तिः । सर्व॑स्य । भे॒ष॒जम् । सर्व᳚म् । वै । ए॒तेन॑ । पा॒प्मान᳚म् । दे॒वाः । अ॒त॒र॒न्न् । अपीति॑ । वै । ए॒तेन॑ । ब्र॒ह्म॒ह॒त्यामिति॑ ब्रह्म - ह॒त्याम् । अ॒त॒र॒न्न् । सर्व᳚म् । पा॒प्मान᳚म् ।  \newline




\markright{ TS 5.3.12.2  \hfill https://www.vedavms.in \hfill}
\addcontentsline{toc}{section}{ TS 5.3.12.2 }
\section*{ TS 5.3.12.2 }

\textbf{TS 5.3.12.2 } \newline
\textbf{Samhita Paata} \newline

तरति॒ तर॑ति ब्रह्मह॒त्यां ॅयो᳚ऽश्वमे॒धेन॒ यज॑ते॒ य उ॑ चैनमे॒वं ॅवेदोत्त॑रं॒ ॅवै तत् प्र॒जाप॑ते॒रक्ष्य॑श्वय॒त् तस्मा॒दश्व॑स्योत्तर॒तोऽव॑ द्यन्ति दक्षिण॒तो᳚ऽन्येषां᳚ पशू॒नां ॅवै॑त॒सः कटो॑ भवत्य॒फ्सुयो॑नि॒र्वा अश्वो᳚ऽफ्सु॒जो वे॑त॒सः स्व ए॒वैनं॒ ॅयोनौ॒ प्रति॑ष्ठापयति चतुष्टो॒मः स्तोमो॑ भवति स॒रड्ढ॒ वा अश्व॑स्य॒ सक्थ्याऽवृ॑ह॒त् ( ) तद्-दे॒वाश्च॑तुष्टो॒मेनै॒व प्रत्य॑दधु॒र्यच्च॑तुष्टो॒मः स्तोमो॒ भव॒त्यश्व॑स्य सर्व॒त्वाय॑ ॥ \newline

\textbf{Pada Paata} \newline

त॒र॒ति॒ । तर॑ति । ब्र॒ह्म॒ह॒त्यामिति॑ ब्रह्म - ह॒त्याम् । यः । अ॒श्व॒मे॒धेनेत्य॑श्व - मे॒धेन॑ । यज॑ते । यः । उ॒ । च॒ । ए॒न॒म् । ए॒वम् । वेद॑ । उत्त॑र॒मित्युत् - त॒र॒म् । वै । तत् । प्र॒जाप॑ते॒रिति॑ प्र॒जा - प॒तेः॒ । अक्षि॑ । अ॒श्व॒य॒त् । तस्मा᳚त् । अश्व॑स्य । उ॒त्त॒र॒त इत्यु॑त् - त॒र॒तः । अवेति॑ । द्य॒न्ति॒ । द॒क्षि॒ण॒तः । अ॒न्येषा᳚म् । प॒शू॒नाम् । वै॒त॒सः । कटः॑ । भ॒व॒ति॒ । अ॒फ्सुयो॑नि॒रित्य॒फ्सु - यो॒निः॒ । वै । अश्वः॑ । अ॒फ्सु॒ज इत्य॑फ्सु - जः । वे॒त॒सः । स्वे । ए॒व । ए॒न॒म् । योनौ᳚ । प्रतीति॑ । स्था॒प॒य॒ति॒ । च॒तु॒ष्टो॒म इति॑ चतुः - स्तो॒मः । स्तोमः॑ । भ॒व॒ति॒ । स॒रट् । ह॒ । वै । अश्व॑स्य । सक्थि॑ । एति॑ । अ॒वृ॒ह॒त् ( ) । तत् । दे॒वाः । च॒तु॒ष्टो॒मेनेति॑ चतुः-स्तो॒मेन॑ । ए॒व । प्रतीति॑ । अ॒द॒धुः॒ । यत् । च॒तु॒ष्टो॒म इति॑ चतुः - स्तो॒मः । स्तोमः॑ । भव॑ति । अश्व॑स्य । स॒र्व॒त्वायेति॑ सर्व - त्वाय॑ ॥  \newline






\end{document}