\documentclass[17pt]{extarticle}
\usepackage{babel}
\usepackage{fontspec}
\usepackage{polyglossia}
\usepackage{extsizes}

\usepackage{color}   %May be necessary if you want to color links
\usepackage{hyperref}
\hypersetup{
    colorlinks=true, %set true if you want colored links
    linktoc=all,     %set to all if you want both sections and subsections linked
    linkcolor=black,  %choose some color if you want links to stand out
}

\setmainlanguage{sanskrit}
\setotherlanguages{english} %% or other languages
\setlength{\parindent}{0pt}
\pagestyle{myheadings}
\newfontfamily\devanagarifont[Script=Devanagari]{AdishilaVedic}
\renewcommand{\theHsection}{\thepart.section.\thesection}

\newcommand{\VAR}[1]{}
\newcommand{\BLOCK}[1]{}




\begin{document}
\begin{titlepage}
    \begin{center}
 
\begin{sanskrit}
    { \Large
    कृष्ण यजुर्वेदीय तैत्तिरीय संहिता,पद,जटा,घन पाठः 
    }
    \\
    \vspace{2.5cm}
    \mbox{ \Large
    3.3     तृतीयकाण्डे तृतीयः प्रश्नः - वैकृतविधीनामभिधानं   }
\end{sanskrit}
\end{center}

\end{titlepage}
\tableofcontents
\phantomsection
\pagebreak

\markright{ TS 3.3.1.1  \hfill https://www.vedavms.in \hfill}

\section{ TS 3.3.1.1 }

\textbf{TS 3.3.1.1 } \newline
\textbf{Samhita Paata} \newline

अग्ने॑ तेजस्विन् तेज॒स्वी त्वं दे॒वेषु॑ भूया॒स्तेज॑स्वन्तं॒ मामायु॑ष्मन्तं॒ ॅवर्च॑स्वन्तं मनु॒ष्ये॑षु कुरु दी॒क्षायै॑ च त्वा॒ तप॑सश्च॒ तेज॑से जुहोमि तेजो॒विद॑सि॒ तेजो॑ मा॒ मा हा॑सी॒न्माऽहं तेजो॑ हासिषं॒ मा मां तेजो॑ हासी॒दिन्द्रौ॑जस्विन्नोज॒स्वी त्वं दे॒वेषु॑ भूया॒ ओज॑स्वन्तं॒ मामायु॑ष्मन्तं॒ ॅवर्च॑स्वन्तं मनु॒ष्ये॑षु कुरु॒ ब्रह्म॑णश्च त्वा क्ष॒त्रस्य॒ चौ - [  ] \newline

\textbf{Pada Paata} \newline

अग्ने᳚ । ते॒ज॒स्वि॒न्न् । ते॒ज॒स्वी । त्वम् । दे॒वेषु॑ । भू॒याः॒ । तेज॑स्वन्तम् । माम् । आयु॑ष्मन्तम् । वर्च॑स्वन्तम् । म॒नु॒ष्ये॑षु । कु॒रु॒ । दी॒क्षायै᳚ । च॒ । त्वा॒ । तप॑सः । च॒ । तेज॑से । जु॒हो॒मि॒ । ते॒जो॒विदिति॑ तेजः - वित् । अ॒सि॒ । तेजः॑ । मा॒ । मा । हा॒सी॒त् । मा । अ॒हम् । तेजः॑ । हा॒सि॒ष॒म् । मा । माम् । तेजः॑ । हा॒सी॒त् । इन्द्र॑ । ओ॒ज॒स्वि॒न्न् । ओ॒ज॒स्वी । त्वम् । दे॒वेषु॑ । भू॒याः॒ । ओज॑स्वन्तम् । माम् । आयु॑ष्मन्तम् । वर्च॑स्वन्तम् । म॒नु॒ष्ये॑षु । कु॒रु॒ । ब्रह्म॑णः । च॒ । त्वा॒ । क्ष॒त्रस्य॑ । च॒ ।  \newline


\textbf{Krama Paata} \newline

अग्ने॑ तेजस्विन्न् । ते॒ज॒स्वि॒न् ते॒ज॒स्वी । ते॒ज॒स्वी त्वम् । त्वम् दे॒वेषु॑ । दे॒वेषु॑ भूयाः । भू॒या॒स्तेज॑स्वन्तम् । तेज॑स्वन्त॒म् माम् । मामायु॑ष्मन्तम् । आयु॑ष्मन्तं॒ ॅवर्च॑स्वन्तम् । वर्च॑स्वन्तम् मनु॒ष्ये॑षु । म॒नु॒ष्ये॑षु कुरु । कु॒रु॒ दी॒क्षायै᳚ । दी॒क्षायै॑ च । च॒ त्वा॒ । त्वा॒ तप॑सः । तप॑सश्च । च॒ तेज॑से । तेज॑से जुहोमि । जु॒हो॒मि॒ ते॒जो॒वित् । ते॒जो॒विद॑सि । ते॒जो॒विदिति॑ तेजः - वित् । अ॒सि॒ तेजः॑ । तेजो॑ मा । मा॒ मा । मा हा॑सीत् । हा॒सी॒न् मा । मा ऽहम् । अ॒हम् तेजः॑ । तेजो॑ हासिषम् । हा॒सि॒ष॒म् मा । मा माम् । माम् तेजः॑ । तेजो॑ हासीत् । हा॒सी॒दिन्द्र॑ । इन्द्रौ॑जस्विन्न् । ओ॒ज॒स्वि॒न्नो॒ज॒स्वी । ओ॒ज॒स्वी त्वम् । त्वम् दे॒वेषु॑ । दे॒वेषु॑ भूयाः । भू॒या॒ ओज॑स्वन्तम् । ओज॑स्वन्त॒म् माम् । मामायु॑ष्मन्तम् । आयु॑ष्मन्तं॒ ॅवर्च॑स्वन्तम् । वर्च॑स्वन्तम् मनु॒ष्ये॑षु । म॒नु॒ष्ये॑षु कुरु । कु॒रु॒ ब्रह्म॑णः । ब्रह्म॑णश्च । च॒ त्वा॒ । त्वा॒ क्ष॒त्रस्य॑ । क्ष॒त्रस्य॑ च । चौज॑से \newline

\textbf{Jatai Paata} \newline

1. अग्ने॑ तेजस्विन् तेजस्वि॒न् नग्ने ऽग्ने॑ तेजस्विन्न् । \newline
2. ते॒ज॒स्वि॒न् ते॒ज॒स्वी ते॑ज॒स्वी ते॑जस्विन् तेजस्विन् तेज॒स्वी । \newline
3. ते॒ज॒स्वी त्वम् त्वम् ते॑ज॒स्वी ते॑ज॒स्वी त्वम् । \newline
4. त्वम् दे॒वेषु॑ दे॒वेषु॒ त्वम् त्वम् दे॒वेषु॑ । \newline
5. दे॒वेषु॑ भूया भूया दे॒वेषु॑ दे॒वेषु॑ भूयाः । \newline
6. भू॒या॒ स्तेज॑स्वन्त॒म् तेज॑स्वन्तम् भूया भूया॒ स्तेज॑स्वन्तम् । \newline
7. तेज॑स्वन्त॒म् माम् माम् तेज॑स्वन्त॒म् तेज॑स्वन्त॒म् माम् । \newline
8. मा मायु॑ष्मन्त॒ मायु॑ष्मन्त॒म् माम् मा मायु॑ष्मन्तम् । \newline
9. आयु॑ष्मन्त॒म् ॅवर्च॑स्वन्त॒म् ॅवर्च॑स्वन्त॒ मायु॑ष्मन्त॒ मायु॑ष्मन्त॒म् ॅवर्च॑स्वन्तम् । \newline
10. वर्च॑स्वन्तम् मनु॒ष्ये॑षु मनु॒ष्ये॑षु॒ वर्च॑स्वन्त॒म् ॅवर्च॑स्वन्तम् मनु॒ष्ये॑षु । \newline
11. म॒नु॒ष्ये॑षु कुरु कुरु मनु॒ष्ये॑षु मनु॒ष्ये॑षु कुरु । \newline
12. कु॒रु॒ दी॒क्षायै॑ दी॒क्षायै॑ कुरु कुरु दी॒क्षायै᳚ । \newline
13. दी॒क्षायै॑ च च दी॒क्षायै॑ दी॒क्षायै॑ च । \newline
14. च॒ त्वा॒ त्वा॒ च॒ च॒ त्वा॒ । \newline
15. त्वा॒ तप॑स॒ स्तप॑स स्त्वा त्वा॒ तप॑सः । \newline
16. तप॑सश्च च॒ तप॑स॒ स्तप॑सश्च । \newline
17. च॒ तेज॑से॒ तेज॑से च च॒ तेज॑से । \newline
18. तेज॑से जुहोमि जुहोमि॒ तेज॑से॒ तेज॑से जुहोमि । \newline
19. जु॒हो॒मि॒ ते॒जो॒वित् ते॑जो॒विज् जु॑होमि जुहोमि तेजो॒वित् । \newline
20. ते॒जो॒वि द॑स्यसि तेजो॒वित् ते॑जो॒वि द॑सि । \newline
21. ते॒जो॒विदिति॑ तेजः - वित् । \newline
22. अ॒सि॒ तेज॒ स्तेजो᳚ ऽस्यसि॒ तेजः॑ । \newline
23. तेजो॑ मा मा॒ तेज॒ स्तेजो॑ मा । \newline
24. मा॒ मा मा मा॑ मा॒ मा । \newline
25. मा हा॑सी द्धासी॒न् मा मा हा॑सीत् । \newline
26. हा॒सी॒न् मा मा हा॑सी द्धासी॒न् मा । \newline
27. मा ऽह म॒हम् मा मा ऽहम् । \newline
28. अ॒हम् तेज॒ स्तेजो॒ ऽह म॒हम् तेजः॑ । \newline
29. तेजो॑ हासिषꣳ हासिष॒म् तेज॒ स्तेजो॑ हासिषम् । \newline
30. हा॒सि॒ष॒म् मा मा हा॑सिषꣳ हासिष॒म् मा । \newline
31. मा माम् माम् मा मा माम् । \newline
32. माम् तेज॒ स्तेजो॒ माम् माम् तेजः॑ । \newline
33. तेजो॑ हासी द्धासी॒त् तेज॒ स्तेजो॑ हासीत् । \newline
34. हा॒सी॒ दिन्द्रे न्द्र॑ हासी द्धासी॒ दिन्द्र॑ । \newline
35. इन्द्रौ॑जस्विन् नोजस्वि॒न् निन्द्रे न्द्रौ॑जस्विन्न् । \newline
36. ओ॒ज॒स्वि॒न् नो॒ज॒ स्व्यो॑ज॒ स्व्यो॑जस्विन् नोजस्विन् नोज॒स्वी । \newline
37. ओ॒ज॒स्वी त्वम् त्व मो॑ज॒ स्व्यो॑ज॒स्वी त्वम् । \newline
38. त्वम् दे॒वेषु॑ दे॒वेषु॒ त्वम् त्वम् दे॒वेषु॑ । \newline
39. दे॒वेषु॑ भूया भूया दे॒वेषु॑ दे॒वेषु॑ भूयाः । \newline
40. भू॒या॒ ओज॑स्वन्त॒ मोज॑स्वन्तम् भूया भूया॒ ओज॑स्वन्तम् । \newline
41. ओज॑स्वन्त॒म् माम् मा मोज॑स्वन्त॒ मोज॑स्वन्त॒म् माम् । \newline
42. मा मायु॑ष्मन्त॒ मायु॑ष्मन्त॒म् माम् मा मायु॑ष्मन्तम् । \newline
43. आयु॑ष्मन्त॒म् ॅवर्च॑स्वन्त॒म् ॅवर्च॑स्वन्त॒ मायु॑ष्मन्त॒ मायु॑ष्मन्त॒म् ॅवर्च॑स्वन्तम् । \newline
44. वर्च॑स्वन्तम् मनु॒ष्ये॑षु मनु॒ष्ये॑षु॒ वर्च॑स्वन्त॒म् ॅवर्च॑स्वन्तम् मनु॒ष्ये॑षु । \newline
45. म॒नु॒ष्ये॑षु कुरु कुरु मनु॒ष्ये॑षु मनु॒ष्ये॑षु कुरु । \newline
46. कु॒रु॒ ब्रह्म॑णो॒ ब्रह्म॑णः कुरु कुरु॒ ब्रह्म॑णः । \newline
47. ब्रह्म॑णश्च च॒ ब्रह्म॑णो॒ ब्रह्म॑णश्च । \newline
48. च॒ त्वा॒ त्वा॒ च॒ च॒ त्वा॒ । \newline
49. त्वा॒ क्ष॒त्रस्य॑ क्ष॒त्रस्य॑ त्वा त्वा क्ष॒त्रस्य॑ । \newline
50. क्ष॒त्रस्य॑ च च क्ष॒त्रस्य॑ क्ष॒त्रस्य॑ च । \newline
51. चौज॑स॒ ओज॑से च॒ चौज॑से । \newline

\textbf{Ghana Paata } \newline

1. अग्ने॑ तेजस्विन् तेजस्वि॒न्, नग्ने ऽग्ने॑ तेजस्विन् तेज॒स्वी ते॑ज॒स्वी ते॑जस्वि॒न्, नग्ने ऽग्ने॑ तेजस्विन् तेज॒स्वी । \newline
2. ते॒ज॒स्वि॒न् ते॒ज॒स्वी ते॑ज॒स्वी ते॑जस्विन् तेजस्विन् तेज॒स्वी त्वम् त्वम् ते॑ज॒स्वी ते॑जस्विन् तेजस्विन् तेज॒स्वी त्वम् । \newline
3. ते॒ज॒स्वी त्वम् त्वम् ते॑ज॒स्वी ते॑ज॒स्वी त्वम् दे॒वेषु॑ दे॒वेषु॒ त्वम् ते॑ज॒स्वी ते॑ज॒स्वी त्वम् दे॒वेषु॑ । \newline
4. त्वम् दे॒वेषु॑ दे॒वेषु॒ त्वम् त्वम् दे॒वेषु॑ भूया भूया दे॒वेषु॒ त्वम् त्वम् दे॒वेषु॑ भूयाः । \newline
5. दे॒वेषु॑ भूया भूया दे॒वेषु॑ दे॒वेषु॑ भूया॒ स्तेज॑स्वन्त॒म् तेज॑स्वन्तम् भूया दे॒वेषु॑ दे॒वेषु॑ भूया॒ स्तेज॑स्वन्तम् । \newline
6. भू॒या॒ स्तेज॑स्वन्त॒म् तेज॑स्वन्तम् भूया भूया॒ स्तेज॑स्वन्त॒म् माम् माम् तेज॑स्वन्तम् भूया भूया॒ स्तेज॑स्वन्त॒म् माम् । \newline
7. तेज॑स्वन्त॒म् माम् माम् तेज॑स्वन्त॒म् तेज॑स्वन्त॒म् मा मायु॑ष्मन्त॒ मायु॑ष्मन्त॒म् माम् तेज॑स्वन्त॒म् तेज॑स्वन्त॒म् मा मायु॑ष्मन्तम् । \newline
8. मा मायु॑ष्मन्त॒ मायु॑ष्मन्त॒म् माम् मा मायु॑ष्मन्त॒म् ॅवर्च॑स्वन्त॒म् ॅवर्च॑स्वन्त॒ मायु॑ष्मन्त॒म् माम् मा मायु॑ष्मन्त॒म् ॅवर्च॑स्वन्तम् । \newline
9. आयु॑ष्मन्त॒म् ॅवर्च॑स्वन्त॒म् ॅवर्च॑स्वन्त॒ मायु॑ष्मन्त॒ मायु॑ष्मन्त॒म् ॅवर्च॑स्वन्तम् मनु॒ष्ये॑षु मनु॒ष्ये॑षु॒ वर्च॑स्वन्त॒ मायु॑ष्मन्त॒ मायु॑ष्मन्त॒म् ॅवर्च॑स्वन्तम् मनु॒ष्ये॑षु । \newline
10. वर्च॑स्वन्तम् मनु॒ष्ये॑षु मनु॒ष्ये॑षु॒ वर्च॑स्वन्त॒म् ॅवर्च॑स्वन्तम् मनु॒ष्ये॑षु कुरु कुरु मनु॒ष्ये॑षु॒ वर्च॑स्वन्त॒म् ॅवर्च॑स्वन्तम् मनु॒ष्ये॑षु कुरु । \newline
11. म॒नु॒ष्ये॑षु कुरु कुरु मनु॒ष्ये॑षु मनु॒ष्ये॑षु कुरु दी॒क्षायै॑ दी॒क्षायै॑ कुरु मनु॒ष्ये॑षु मनु॒ष्ये॑षु कुरु दी॒क्षायै᳚ । \newline
12. कु॒रु॒ दी॒क्षायै॑ दी॒क्षायै॑ कुरु कुरु दी॒क्षायै॑ च च दी॒क्षायै॑ कुरु कुरु दी॒क्षायै॑ च । \newline
13. दी॒क्षायै॑ च च दी॒क्षायै॑ दी॒क्षायै॑ च त्वा त्वा च दी॒क्षायै॑ दी॒क्षायै॑ च त्वा । \newline
14. च॒ त्वा॒ त्वा॒ च॒ च॒ त्वा॒ तप॑स॒ स्तप॑स स्त्वा च च त्वा॒ तप॑सः । \newline
15. त्वा॒ तप॑स॒ स्तप॑स स्त्वा त्वा॒ तप॑सश्च च॒ तप॑स स्त्वा त्वा॒ तप॑सश्च । \newline
16. तप॑सश्च च॒ तप॑स॒ स्तप॑सश्च॒ तेज॑से॒ तेज॑से च॒ तप॑स॒ स्तप॑सश्च॒ तेज॑से । \newline
17. च॒ तेज॑से॒ तेज॑से च च॒ तेज॑से जुहोमि जुहोमि॒ तेज॑से च च॒ तेज॑से जुहोमि । \newline
18. तेज॑से जुहोमि जुहोमि॒ तेज॑से॒ तेज॑से जुहोमि तेजो॒वित् ते॑जो॒विज् जु॑होमि॒ तेज॑से॒ तेज॑से जुहोमि तेजो॒वित् । \newline
19. जु॒हो॒मि॒ ते॒जो॒वित् ते॑जो॒विज् जु॑होमि जुहोमि तेजो॒वि द॑स्यसि तेजो॒विज् जु॑होमि जुहोमि तेजो॒वि द॑सि । \newline
20. ते॒जो॒वि द॑स्यसि तेजो॒वित् ते॑जो॒वि द॑सि॒ तेज॒ स्तेजो॑ ऽसि तेजो॒वित् ते॑जो॒वि द॑सि॒ तेजः॑ । \newline
21. ते॒जो॒विदिति॑ तेजः - वित् । \newline
22. अ॒सि॒ तेज॒ स्तेजो᳚ ऽस्यसि॒ तेजो॑ मा मा॒ तेजो᳚ ऽस्यसि॒ तेजो॑ मा । \newline
23. तेजो॑ मा मा॒ तेज॒ स्तेजो॑ मा॒ मा मा मा॒ तेज॒ स्तेजो॑ मा॒ मा । \newline
24. मा॒ मा मा मा॑ मा॒ मा हा॑सी द्धासी॒न् मा मा॑ मा॒ मा हा॑सीत् । \newline
25. मा हा॑सी द्धासी॒न् मा मा हा॑सी॒न् मा मा हा॑सी॒न् मा मा हा॑सी॒न् मा । \newline
26. हा॒सी॒न् मा मा हा॑सी द्धासी॒न् मा ऽह म॒हम् मा हा॑सी द्धासी॒न् मा ऽहम् । \newline
27. मा ऽह म॒हम् मा मा ऽहम् तेज॒ स्तेजो॒ ऽहम् मा मा ऽहम् तेजः॑ । \newline
28. अ॒हम् तेज॒ स्तेजो॒ ऽह म॒हम् तेजो॑ हासिषꣳ हासिष॒म् तेजो॒ ऽह म॒हम् तेजो॑ हासिषम् । \newline
29. तेजो॑ हासिषꣳ हासिष॒म् तेज॒ स्तेजो॑ हासिष॒म् मा मा हा॑सिष॒म् तेज॒ स्तेजो॑ हासिष॒म् मा । \newline
30. हा॒सि॒ष॒म् मा मा हा॑सिषꣳ हासिष॒म् मा माम् माम् मा हा॑सिषꣳ हासिष॒म् मा माम् । \newline
31. मा माम् माम् मा मा माम् तेज॒ स्तेजो॒ माम् मा मा माम् तेजः॑ । \newline
32. माम् तेज॒ स्तेजो॒ माम् माम् तेजो॑ हासी द्धासी॒त् तेजो॒ माम् माम् तेजो॑ हासीत् । \newline
33. तेजो॑ हासी द्धासी॒त् तेज॒ स्तेजो॑ हासी॒ दिन्द्रे न्द्र॑ हासी॒त् तेज॒ स्तेजो॑ हासी॒दिन्द्र॑ । \newline
34. हा॒सी॒ दिन्द्रे न्द्र॑ हासी द्धासी॒ दिन्द्रौ॑जस्विन्, नोजस्वि॒न्, निन्द्र॑ हासी द्धासी॒ दिन्द्रौ॑जस्विन्न् । \newline
35. इन्द्रौ॑जस्विन्, नोजस्वि॒न्, निन्द्रे न्द्रौ॑जस्विन्, नोज॒ स्व्यो॑ज॒ स्व्यो॑जस्वि॒न्, निन्द्रे न्द्रौ॑जस्विन्, नोज॒स्वी । \newline
36. ओ॒ज॒स्वि॒न्, नो॒ज॒ स्व्यो॑ज॒ स्व्यो॑जस्विन्, नोजस्विन्, नोज॒स्वी त्वम् त्व मो॑ज॒ स्व्यो॑जस्विन्, नोजस्विन्, नोज॒स्वी त्वम् । \newline
37. ओ॒ज॒स्वी त्वम् त्व मो॑ज॒ स्व्यो॑ज॒स्वी त्वम् दे॒वेषु॑ दे॒वेषु॒ त्व मो॑ज॒ स्व्यो॑ज॒स्वी त्वम् दे॒वेषु॑ । \newline
38. त्वम् दे॒वेषु॑ दे॒वेषु॒ त्वम् त्वम् दे॒वेषु॑ भूया भूया दे॒वेषु॒ त्वम् त्वम् दे॒वेषु॑ भूयाः । \newline
39. दे॒वेषु॑ भूया भूया दे॒वेषु॑ दे॒वेषु॑ भूया॒ ओज॑स्वन्त॒ मोज॑स्वन्तम् भूया दे॒वेषु॑ दे॒वेषु॑ भूया॒ ओज॑स्वन्तम् । \newline
40. भू॒या॒ ओज॑स्वन्त॒ मोज॑स्वन्तम् भूया भूया॒ ओज॑स्वन्त॒म् माम् मा मोज॑स्वन्तम् भूया भूया॒ ओज॑स्वन्त॒म् माम् । \newline
41. ओज॑स्वन्त॒म् माम् मा मोज॑स्वन्त॒ मोज॑स्वन्त॒म् मा मायु॑ष्मन्त॒ मायु॑ष्मन्त॒म् मा मोज॑स्वन्त॒ मोज॑स्वन्त॒म् मा मायु॑ष्मन्तम् । \newline
42. मा मायु॑ष्मन्त॒ मायु॑ष्मन्त॒म् माम् मा मायु॑ष्मन्त॒म् ॅवर्च॑स्वन्त॒म् ॅवर्च॑स्वन्त॒ मायु॑ष्मन्त॒म् माम् मा मायु॑ष्मन्त॒म् ॅवर्च॑स्वन्तम् । \newline
43. आयु॑ष्मन्त॒म् ॅवर्च॑स्वन्त॒म् ॅवर्च॑स्वन्त॒ मायु॑ष्मन्त॒ मायु॑ष्मन्त॒म् ॅवर्च॑स्वन्तम् मनु॒ष्ये॑षु मनु॒ष्ये॑षु॒ वर्च॑स्वन्त॒ मायु॑ष्मन्त॒ मायु॑ष्मन्त॒म् ॅवर्च॑स्वन्तम् मनु॒ष्ये॑षु । \newline
44. वर्च॑स्वन्तम् मनु॒ष्ये॑षु मनु॒ष्ये॑षु॒ वर्च॑स्वन्त॒म् ॅवर्च॑स्वन्तम् मनु॒ष्ये॑षु कुरु कुरु मनु॒ष्ये॑षु॒ वर्च॑स्वन्त॒म् ॅवर्च॑स्वन्तम् मनु॒ष्ये॑षु कुरु । \newline
45. म॒नु॒ष्ये॑षु कुरु कुरु मनु॒ष्ये॑षु मनु॒ष्ये॑षु कुरु॒ ब्रह्म॑णो॒ ब्रह्म॑णः कुरु मनु॒ष्ये॑षु मनु॒ष्ये॑षु कुरु॒ ब्रह्म॑णः । \newline
46. कु॒रु॒ ब्रह्म॑णो॒ ब्रह्म॑णः कुरु कुरु॒ ब्रह्म॑णश्च च॒ ब्रह्म॑णः कुरु कुरु॒ ब्रह्म॑णश्च । \newline
47. ब्रह्म॑णश्च च॒ ब्रह्म॑णो॒ ब्रह्म॑णश्च त्वा त्वा च॒ ब्रह्म॑णो॒ ब्रह्म॑णश्च त्वा । \newline
48. च॒ त्वा॒ त्वा॒ च॒ च॒ त्वा॒ क्ष॒त्रस्य॑ क्ष॒त्रस्य॑ त्वा च च त्वा क्ष॒त्रस्य॑ । \newline
49. त्वा॒ क्ष॒त्रस्य॑ क्ष॒त्रस्य॑ त्वा त्वा क्ष॒त्रस्य॑ च च क्ष॒त्रस्य॑ त्वा त्वा क्ष॒त्रस्य॑ च । \newline
50. क्ष॒त्रस्य॑ च च क्ष॒त्रस्य॑ क्ष॒त्रस्य॒ चौज॑स॒ ओज॑से च क्ष॒त्रस्य॑ क्ष॒त्रस्य॒ चौज॑से । \newline
51. चौज॑स॒ ओज॑से च॒ चौज॑से जुहोमि जुहो॒ म्योज॑से च॒ चौज॑से जुहोमि । \newline
\pagebreak
\markright{ TS 3.3.1.2  \hfill https://www.vedavms.in \hfill}

\section{ TS 3.3.1.2 }

\textbf{TS 3.3.1.2 } \newline
\textbf{Samhita Paata} \newline

- ज॑से जुहोम्योजो॒विद॒स्योजो॑ मा॒ मा हा॑सी॒न्माऽहमोजो॑ हासिषं॒ मा मामोजो॑ हासी॒थ् सूर्य॑ भ्राजस्विन् भ्राज॒स्वी त्वं दे॒वेषु॑ भूया॒ भ्राज॑स्वन्तं॒ मामायु॑ष्मन्तं॒ ॅवर्च॑स्वन्तं मनु॒ष्ये॑षु कुरु वा॒योश्च॑ त्वा॒ऽपाञ्च॒ भ्राज॑से जुहोमिसुव॒र्विद॑सि॒ सुव॑र्मा॒ मा हा॑सी॒न्माऽहꣳ सुव॑र्.हासिषं॒ मा माꣳ सुव॑र्.हासी॒न् मयि॑ ( ) मे॒धां मयि॑ प्र॒जां मय्य॒ग्निस्तेजो॑ दधातु॒ मयि॑ मे॒धां मयि॑ प्र॒जां मयीन्द्र॑ इन्द्रि॒यं द॑धातु॒ मयि॑ मे॒धां मयि॑ प्र॒जां मयि॒ सूर्यो॒ भ्राजो॑ दधातु ॥ \newline

\textbf{Pada Paata} \newline

ओज॑से । जु॒हो॒मि॒ । ओ॒जो॒विदित्यो॑जः - वित् । अ॒सि॒ । ओजः॑ । मा॒ । मा । हा॒सी॒त् । मा । अ॒हम् । ओजः॑ । हा॒सि॒ष॒म् । मा । माम् । ओजः॑ । हा॒सी॒त् । सूर्य॑ । भ्रा॒ज॒स्वि॒न्न् । भ्रा॒ज॒स्वी । त्वम् । दे॒वेषु॑ । भू॒याः॒ । भ्राज॑स्वन्तम् । माम् । आयु॑ष्मन्तम् । वर्च॑स्वन्तम् । म॒नु॒ष्ये॑षु । कु॒रु॒ । वा॒योः । च॒ । त्वा॒ । अ॒पाम् । च॒ । भ्राज॑से । जु॒हो॒मि॒ । सु॒व॒र्विदिति॑ सुवः - वित् । अ॒सि॒ । सुवः॑ । मा॒ । मा । हा॒सी॒त् । मा । अ॒हम् । सुवः॑ । हा॒सि॒ष॒म् । मा । माम् । सुवः॑ । हा॒सी॒त् । मयि॑ ( ) । मे॒धाम् । मयि॑ । प्र॒जामिति॑ प्र - जाम् । मयि॑ । अ॒ग्निः । तेजः॑ । द॒धा॒तु॒ । मयि॑ । मे॒धाम् । मयि॑ । प्र॒जामिति॑ प्र - जाम् । मयि॑ । इन्द्रः॑ । इ॒न्द्रि॒यम् । द॒धा॒तु॒ । मयि॑ । मे॒धाम् । मयि॑ । प्र॒जामिति॑ प्र - जाम् । मयि॑ । सूर्यः॑ । भ्राजः॑ । द॒धा॒तु॒ ॥  \newline


\textbf{Krama Paata} \newline

ओज॑से जुहोमि । जु॒हो॒म्यो॒जो॒वित् । ओ॒जो॒विद॑सि । ओ॒जो॒विदित्यो॑जः - वित् । अ॒स्योजः॑ । ओजो॑ मा । मा॒ मा । मा हा॑सीत् । हा॒सी॒न् मा । मा ऽहम् । अ॒हमोजः॑ । ओजो॑ हासिषम् । हा॒सि॒ष॒म् मा । मा माम् । मामोजः॑ । ओजो॑ हासीत् । हा॒सी॒थ् सूर्य॑ । सूर्य॑ भ्राजस्विन्न् । भ्रा॒ज॒स्वि॒न् भ्रा॒ज॒स्वी । भ्रा॒ज॒स्वी त्वम् । त्वम् दे॒वेषु॑ । दे॒वेषु॑ भूयाः । भू॒या॒ भ्राज॑स्वन्तम् । भ्राज॑स्वन्त॒म् माम् । मामायु॑ष्मन्तम् । आयु॑ष्मन्तं॒ ॅवर्च॑स्वन्तम् । वर्च॑स्वन्तम् मनु॒ष्ये॑षु । म॒नु॒ष्ये॑षु कुरु । कु॒रु॒ वा॒योः । वा॒योश्च॑ । च॒ त्वा॒ । त्वा॒ ऽपाम् । अ॒पाम् च॑ । च॒ भ्राज॑से । भ्राज॑से जुहोमि । जु॒हो॒मि॒ सु॒व॒र्वित् । सु॒व॒र्विद॑सि । सु॒व॒र्विदिति॑ सुवः - वित् । अ॒सि॒ सुवः॑ । सुव॑र् मा । मा॒ मा । मा हा॑सीत् । हा॒सी॒न् मा । मा ऽहम् । अ॒हꣳ सुवः॑ । सुव॑र्. हासिषम् । हा॒सि॒ष॒म् मा । मा माम् । माꣳ सुवः॑ । सुव॑र् हासीत् । हा॒सी॒न् मयि॑ ( ) । मयि॑ मे॒धाम् । मे॒धाम् मयि॑ । मयि॑ प्र॒जाम् । प्र॒जाम् मयि॑ । प्र॒जामिति॑ प्र - जाम् । मय्य॒ग्निः । अ॒ग्निस्तेजः॑ । तेजो॑ दधातु । द॒धा॒तु॒ मयि॑ । मयि॑ मे॒धाम् । मे॒धाम् मयि॑ । मयि॑ प्र॒जाम् । प्र॒जाम् मयि॑ । प्र॒जामिति॑ प्र - जाम् । मयीन्द्रः॑ । इन्द्र॑ इन्द्रि॒यम् । इ॒न्द्रि॒यम् द॑धातु । द॒धा॒तु॒ मयि॑ । मयि॑ मे॒धाम् । मे॒धाम् मयि॑ । मयि॑ प्र॒जाम् । प्र॒जाम् मयि॑ । प्र॒जामिति॑ प्र - जाम् । मयि॒ सूर्यः॑ । सूर्यो॒ भ्राजः॑ । भ्राजो॑ दधातु । द॒धा॒त्विति॑ दधातु । \newline

\textbf{Jatai Paata} \newline

1. ओज॑से जुहोमि जुहो॒ म्योज॑स॒ ओज॑से जुहोमि । \newline
2. जु॒हो॒ म्यो॒जो॒वि दो॑जो॒विज् जु॑होमि जुहो म्योजो॒वित् । \newline
3. ओ॒जो॒वि द॑स्य स्योजो॒वि दो॑जो॒वि द॑सि । \newline
4. ओ॒जो॒विदित्यो॑जः - वित् । \newline
5. अ॒स्योज॒ ओजो᳚ ऽस्य॒ स्योजः॑ । \newline
6. ओजो॑ मा॒ मौज॒ ओजो॑ मा । \newline
7. मा॒ मा मा मा॑ मा॒ मा । \newline
8. मा हा॑सी द्धासी॒न् मा मा हा॑सीत् । \newline
9. हा॒सी॒न् मा मा हा॑सी द्धासी॒न् मा । \newline
10. मा ऽह म॒हम् मा मा ऽहम् । \newline
11. अ॒ह मोज॒ ओजो॒ ऽह म॒ह मोजः॑ । \newline
12. ओजो॑ हासिषꣳ हासिष॒ मोज॒ ओजो॑ हासिषम् । \newline
13. हा॒सि॒ष॒म् मा मा हा॑सिषꣳ हासिष॒म् मा । \newline
14. मा माम् माम् मा मा माम् । \newline
15. मा मोज॒ ओजो॒ माम् मा मोजः॑ । \newline
16. ओजो॑ हासी द्धासी॒ दोज॒ ओजो॑ हासीत् । \newline
17. हा॒सी॒थ् सूर्य॒ सूर्य॑ हासी द्धासी॒थ् सूर्य॑ । \newline
18. सूर्य॑ भ्राजस्विन् भ्राजस्वि॒न् थ्सूर्य॒ सूर्य॑ भ्राजस्विन्न् । \newline
19. भ्रा॒ज॒स्वि॒न् भ्रा॒ज॒स्वी भ्रा॑ज॒स्वी भ्रा॑जस्विन् भ्राजस्विन् भ्राज॒स्वी । \newline
20. भ्रा॒ज॒स्वी त्वम् त्वम् भ्रा॑ज॒स्वी भ्रा॑ज॒स्वी त्वम् । \newline
21. त्वम् दे॒वेषु॑ दे॒वेषु॒ त्वम् त्वम् दे॒वेषु॑ । \newline
22. दे॒वेषु॑ भूया भूया दे॒वेषु॑ दे॒वेषु॑ भूयाः । \newline
23. भू॒या॒ भ्राज॑स्वन्त॒म् भ्राज॑स्वन्तम् भूया भूया॒ भ्राज॑स्वन्तम् । \newline
24. भ्राज॑स्वन्त॒म् माम् माम् भ्राज॑स्वन्त॒म् भ्राज॑स्वन्त॒म् माम् । \newline
25. मा मायु॑ष्मन्त॒ मायु॑ष्मन्त॒म् माम् मा मायु॑ष्मन्तम् । \newline
26. आयु॑ष्मन्त॒म् ॅवर्च॑स्वन्त॒म् ॅवर्च॑स्वन्त॒ मायु॑ष्मन्त॒ मायु॑ष्मन्त॒म् ॅवर्च॑स्वन्तम् । \newline
27. वर्च॑स्वन्तम् मनु॒ष्ये॑षु मनु॒ष्ये॑षु॒ वर्च॑स्वन्त॒म् ॅवर्च॑स्वन्तम् मनु॒ष्ये॑षु । \newline
28. म॒नु॒ष्ये॑षु कुरु कुरु मनु॒ष्ये॑षु मनु॒ष्ये॑षु कुरु । \newline
29. कु॒रु॒ वा॒योर् वा॒योः कु॑रु कुरु वा॒योः । \newline
30. वा॒योश्च॑ च वा॒योर् वा॒योश्च॑ । \newline
31. च॒ त्वा॒ त्वा॒ च॒ च॒ त्वा॒ । \newline
32. त्वा॒ ऽपा म॒पाम् त्वा᳚ त्वा॒ ऽपाम् । \newline
33. अ॒पाम् च॑ चा॒पा म॒पाम् च॑ । \newline
34. च॒ भ्राज॑से॒ भ्राज॑से च च॒ भ्राज॑से । \newline
35. भ्राज॑से जुहोमि जुहोमि॒ भ्राज॑से॒ भ्राज॑से जुहोमि । \newline
36. जु॒हो॒मि॒ सु॒व॒र्विथ् सु॑व॒र्विज् जु॑होमि जुहोमि सुव॒र्वित् । \newline
37. सु॒व॒र्विद॑स्यसि सुव॒र्विथ् सु॑व॒र्विद॑सि । \newline
38. सु॒व॒र्विदिति॑ सुवः - वित् । \newline
39. अ॒सि॒ सुवः॒ सुव॑ रस्यसि॒ सुवः॑ । \newline
40. सुव॑र् मा मा॒ सुवः॒ सुव॑र् मा । \newline
41. मा॒ मा मा मा॑ मा॒ मा । \newline
42. मा हा॑सी द्धासी॒न् मा मा हा॑सीत् । \newline
43. हा॒सी॒न् मा मा हा॑सी द्धासी॒न् मा । \newline
44. मा ऽह म॒हम् मा मा ऽहम् । \newline
45. अ॒हꣳ सुवः॒ सुव॑ र॒ह म॒हꣳ सुवः॑ । \newline
46. सुव॑र्. हासिषꣳ हासिषꣳ॒॒ सुवः॒ सुव॑र्. हासिषम् । \newline
47. हा॒सि॒ष॒म् मा मा हा॑सिषꣳ हासिष॒म् मा । \newline
48. मा माम् माम् मा मा माम् । \newline
49. माꣳ सुवः॒ सुव॒र् माम् माꣳ सुवः॑ । \newline
50. सुव॑र्. हासी द्धासी॒थ् सुवः॒ सुव॑र्. हासीत् । \newline
51. हा॒सी॒न् मयि॒ मयि॑ हासी द्धासी॒न् मयि॑ । \newline
52. मयि॑ मे॒धाम् मे॒धाम् मयि॒ मयि॑ मे॒धाम् । \newline
53. मे॒धाम् मयि॒ मयि॑ मे॒धाम् मे॒धाम् मयि॑ । \newline
54. मयि॑ प्र॒जाम् प्र॒जाम् मयि॒ मयि॑ प्र॒जाम् । \newline
55. प्र॒जाम् मयि॒ मयि॑ प्र॒जाम् प्र॒जाम् मयि॑ । \newline
56. प्र॒जामिति॑ प्र - जाम् । \newline
57. मय्य॒ग्नि र॒ग्निर् मयि॒ मय्य॒ग्निः । \newline
58. अ॒ग्नि स्तेज॒ स्तेजो॒ ऽग्नि र॒ग्नि स्तेजः॑ । \newline
59. तेजो॑ दधातु दधातु॒ तेज॒ स्तेजो॑ दधातु । \newline
60. द॒धा॒तु॒ मयि॒ मयि॑ दधातु दधातु॒ मयि॑ । \newline
61. मयि॑ मे॒धाम् मे॒धाम् मयि॒ मयि॑ मे॒धाम् । \newline
62. मे॒धाम् मयि॒ मयि॑ मे॒धाम् मे॒धाम् मयि॑ । \newline
63. मयि॑ प्र॒जाम् प्र॒जाम् मयि॒ मयि॑ प्र॒जाम् । \newline
64. प्र॒जाम् मयि॒ मयि॑ प्र॒जाम् प्र॒जाम् मयि॑ । \newline
65. प्र॒जामिति॑ प्र - जाम् । \newline
66. मयीन्द्र॒ इन्द्रो॒ मयि॒ मयीन्द्रः॑ । \newline
67. इन्द्र॑ इन्द्रि॒य मि॑न्द्रि॒य मिन्द्र॒ इन्द्र॑ इन्द्रि॒यम् । \newline
68. इ॒न्द्रि॒यम् द॑धातु दधा त्विन्द्रि॒य मि॑न्द्रि॒यम् द॑धातु । \newline
69. द॒धा॒तु॒ मयि॒ मयि॑ दधातु दधातु॒ मयि॑ । \newline
70. मयि॑ मे॒धाम् मे॒धाम् मयि॒ मयि॑ मे॒धाम् । \newline
71. मे॒धाम् मयि॒ मयि॑ मे॒धाम् मे॒धाम् मयि॑ । \newline
72. मयि॑ प्र॒जाम् प्र॒जाम् मयि॒ मयि॑ प्र॒जाम् । \newline
73. प्र॒जाम् मयि॒ मयि॑ प्र॒जाम् प्र॒जाम् मयि॑ । \newline
74. प्र॒जामिति॑ प्र - जाम् । \newline
75. मयि॒ सूर्यः॒ सूर्यो॒ मयि॒ मयि॒ सूर्यः॑ । \newline
76. सूर्यो॒ भ्राजो॒ भ्राजः॒ सूर्यः॒ सूर्यो॒ भ्राजः॑ । \newline
77. भ्राजो॑ दधातु दधातु॒ भ्राजो॒ भ्राजो॑ दधातु । \newline
78. द॒धा॒त्विति॑ दधातु । \newline

\textbf{Ghana Paata } \newline

1. ओज॑से जुहोमि जुहो॒ म्योज॑स॒ ओज॑से जुहो म्योजो॒वि दो॑जो॒विज् जु॑हो॒ म्योज॑स॒ ओज॑से जुहो म्योजो॒वित् । \newline
2. जु॒हो॒ म्यो॒जो॒वि दो॑जो॒विज् जु॑होमि जुहो म्योजो॒वि द॑स्य स्योजो॒विज् जु॑होमि जुहो म्योजो॒वि द॑सि । \newline
3. ओ॒जो॒वि द॑स्य स्योजो॒वि दो॑जो॒वि द॒स्योज॒ ओजो᳚ ऽस्योजो॒वि दो॑जो॒वि द॒स्योजः॑ । \newline
4. ओ॒जो॒विदित्यो॑जः - वित् । \newline
5. अ॒स्योज॒ ओजो᳚ ऽस्य॒ स्योजो॑ मा॒ मौजो᳚ ऽस्य॒ स्योजो॑ मा । \newline
6. ओजो॑ मा॒ मौज॒ ओजो॑ मा॒ मा मा मौज॒ ओजो॑ मा॒ मा । \newline
7. मा॒ मा मा मा॑ मा॒ मा हा॑सी द्धासी॒न् मा मा॑ मा॒ मा हा॑सीत् । \newline
8. मा हा॑सी द्धासी॒न् मा मा हा॑सी॒न् मा मा हा॑सी॒न् मा मा हा॑सी॒न् मा । \newline
9. हा॒सी॒न् मा मा हा॑सी द्धासी॒न् मा ऽह म॒हम् मा हा॑सी द्धासी॒न् मा ऽहम् । \newline
10. मा ऽह म॒हम् मा मा ऽह मोज॒ ओजो॒ ऽहम् मा मा ऽह मोजः॑ । \newline
11. अ॒ह मोज॒ ओजो॒ ऽह म॒ह मोजो॑ हासिषꣳ हासिष॒ मोजो॒ ऽह म॒ह मोजो॑ हासिषम् । \newline
12. ओजो॑ हासिषꣳ हासिष॒ मोज॒ ओजो॑ हासिष॒म् मा मा हा॑सिष॒ मोज॒ ओजो॑ हासिष॒म् मा । \newline
13. हा॒सि॒ष॒म् मा मा हा॑सिषꣳ हासिष॒म् मा माम् माम् मा हा॑सिषꣳ हासिष॒म् मा माम् । \newline
14. मा माम् माम् मा मा मा मोज॒ ओजो॒ माम् मा मा मा मोजः॑ । \newline
15. मा मोज॒ ओजो॒ माम् मा मोजो॑ हासी द्धासी॒ दोजो॒ माम् मा मोजो॑ हासीत् । \newline
16. ओजो॑ हासी द्धासी॒ दोज॒ ओजो॑ हासी॒थ् सूर्य॒ सूर्य॑ हासी॒ दोज॒ ओजो॑ हासी॒थ् सूर्य॑ । \newline
17. हा॒सी॒थ् सूर्य॒ सूर्य॑ हासी द्धासी॒थ् सूर्य॑ भ्राजस्विन् भ्राजस्वि॒न् थ्सूर्य॑ हासी द्धासी॒थ् सूर्य॑ भ्राजस्विन्न् । \newline
18. सूर्य॑ भ्राजस्विन् भ्राजस्वि॒न् थ्सूर्य॒ सूर्य॑ भ्राजस्विन् भ्राज॒स्वी भ्रा॑ज॒स्वी भ्रा॑जस्वि॒न् थ्सूर्य॒ सूर्य॑ भ्राजस्विन् भ्राज॒स्वी । \newline
19. भ्रा॒ज॒स्वि॒न् भ्रा॒ज॒स्वी भ्रा॑ज॒स्वी भ्रा॑जस्विन् भ्राजस्विन् भ्राज॒स्वी त्वम् त्वम् भ्रा॑ज॒स्वी भ्रा॑जस्विन् भ्राजस्विन् भ्राज॒स्वी त्वम् । \newline
20. भ्रा॒ज॒स्वी त्वम् त्वम् भ्रा॑ज॒स्वी भ्रा॑ज॒स्वी त्वम् दे॒वेषु॑ दे॒वेषु॒ त्वम् भ्रा॑ज॒स्वी भ्रा॑ज॒स्वी त्वम् दे॒वेषु॑ । \newline
21. त्वम् दे॒वेषु॑ दे॒वेषु॒ त्वम् त्वम् दे॒वेषु॑ भूया भूया दे॒वेषु॒ त्वम् त्वम् दे॒वेषु॑ भूयाः । \newline
22. दे॒वेषु॑ भूया भूया दे॒वेषु॑ दे॒वेषु॑ भूया॒ भ्राज॑स्वन्त॒म् भ्राज॑स्वन्तम् भूया दे॒वेषु॑ दे॒वेषु॑ भूया॒ भ्राज॑स्वन्तम् । \newline
23. भू॒या॒ भ्राज॑स्वन्त॒म् भ्राज॑स्वन्तम् भूया भूया॒ भ्राज॑स्वन्त॒म् माम् माम् भ्राज॑स्वन्तम् भूया भूया॒ भ्राज॑स्वन्त॒म् माम् । \newline
24. भ्राज॑स्वन्त॒म् माम् माम् भ्राज॑स्वन्त॒म् भ्राज॑स्वन्त॒म् मा मायु॑ष्मन्त॒ मायु॑ष्मन्त॒म् माम् भ्राज॑स्वन्त॒म् भ्राज॑स्वन्त॒म् मा मायु॑ष्मन्तम् । \newline
25. मा मायु॑ष्मन्त॒ मायु॑ष्मन्त॒म् माम् मा मायु॑ष्मन्त॒म् ॅवर्च॑स्वन्त॒म् ॅवर्च॑स्वन्त॒ मायु॑ष्मन्त॒म् माम् मा मायु॑ष्मन्त॒म् ॅवर्च॑स्वन्तम् । \newline
26. आयु॑ष्मन्त॒म् ॅवर्च॑स्वन्त॒म् ॅवर्च॑स्वन्त॒ मायु॑ष्मन्त॒ मायु॑ष्मन्त॒म् ॅवर्च॑स्वन्तम् मनु॒ष्ये॑षु मनु॒ष्ये॑षु॒ वर्च॑स्वन्त॒ मायु॑ष्मन्त॒ मायु॑ष्मन्त॒म् ॅवर्च॑स्वन्तम् मनु॒ष्ये॑षु । \newline
27. वर्च॑स्वन्तम् मनु॒ष्ये॑षु मनु॒ष्ये॑षु॒ वर्च॑स्वन्त॒म् ॅवर्च॑स्वन्तम् मनु॒ष्ये॑षु कुरु कुरु मनु॒ष्ये॑षु॒ वर्च॑स्वन्त॒म् ॅवर्च॑स्वन्तम् मनु॒ष्ये॑षु कुरु । \newline
28. म॒नु॒ष्ये॑षु कुरु कुरु मनु॒ष्ये॑षु मनु॒ष्ये॑षु कुरु वा॒योर् वा॒योः कु॑रु मनु॒ष्ये॑षु मनु॒ष्ये॑षु कुरु वा॒योः । \newline
29. कु॒रु॒ वा॒योर् वा॒योः कु॑रु कुरु वा॒योश्च॑ च वा॒योः कु॑रु कुरु वा॒योश्च॑ । \newline
30. वा॒योश्च॑ च वा॒योर् वा॒योश्च॑ त्वा त्वा च वा॒योर् वा॒योश्च॑ त्वा । \newline
31. च॒ त्वा॒ त्वा॒ च॒ च॒ त्वा॒ ऽपा म॒पाम् त्वा॑ च च त्वा॒ ऽपाम् । \newline
32. त्वा॒ ऽपा म॒पाम् त्वा᳚ त्वा॒ ऽपाम् च॑ चा॒पाम् त्वा᳚ त्वा॒ ऽपाम् च॑ । \newline
33. अ॒पाम् च॑ चा॒पा म॒पाम् च॒ भ्राज॑से॒ भ्राज॑से चा॒पा म॒पाम् च॒ भ्राज॑से । \newline
34. च॒ भ्राज॑से॒ भ्राज॑से च च॒ भ्राज॑से जुहोमि जुहोमि॒ भ्राज॑से च च॒ भ्राज॑से जुहोमि । \newline
35. भ्राज॑से जुहोमि जुहोमि॒ भ्राज॑से॒ भ्राज॑से जुहोमि सुव॒र्विथ् सु॑व॒र्विज् जु॑होमि॒ भ्राज॑से॒ भ्राज॑से जुहोमि सुव॒र्वित् । \newline
36. जु॒हो॒मि॒ सु॒व॒र्विथ् सु॑व॒र्विज् जु॑होमि जुहोमि सुव॒र्वि द॑स्यसि सुव॒र्विज् जु॑होमि जुहोमि सुव॒र्वि द॑सि । \newline
37. सु॒व॒र्वि द॑स्यसि सुव॒र्विथ् सु॑व॒र्वि द॑सि॒ सुवः॒ सुव॑ रसि सुव॒र्विथ् सु॑व॒र्वि द॑सि॒ सुवः॑ । \newline
38. सु॒व॒र्विदिति॑ सुवः - वित् । \newline
39. अ॒सि॒ सुवः॒ सुव॑ रस्यसि॒ सुव॑र् मा मा॒ सुव॑ रस्यसि॒ सुव॑र् मा । \newline
40. सुव॑र् मा मा॒ सुवः॒ सुव॑र् मा॒ मा मा मा॒ सुवः॒ सुव॑र् मा॒ मा । \newline
41. मा॒ मा मा मा॑ मा॒ मा हा॑सी द्धासी॒न् मा मा॑ मा॒ मा हा॑सीत् । \newline
42. मा हा॑सी द्धासी॒न् मा मा हा॑सी॒न् मा मा हा॑सी॒न् मा मा हा॑सी॒न् मा । \newline
43. हा॒सी॒न् मा मा हा॑सी द्धासी॒न् मा ऽह म॒हम् मा हा॑सी द्धासी॒न् मा ऽहम् । \newline
44. मा ऽह म॒हम् मा मा ऽहꣳ सुवः॒ सुव॑ र॒हम् मा मा ऽहꣳ सुवः॑ । \newline
45. अ॒हꣳ सुवः॒ सुव॑ र॒ह म॒हꣳ सुव॑र्. हासिषꣳ हासिषꣳ॒॒ सुव॑ र॒ह म॒हꣳ सुव॑र्. हासिषम् । \newline
46. सुव॑र्. हासिषꣳ हासिषꣳ॒॒ सुवः॒ सुव॑र्. हासिष॒म् मा मा हा॑सिषꣳ॒॒ सुवः॒ सुव॑र्. हासिष॒म् मा । \newline
47. हा॒सि॒ष॒म् मा मा हा॑सिषꣳ हासिष॒म् मा माम् माम् मा हा॑सिषꣳ हासिष॒म् मा माम् । \newline
48. मा माम् माम् मा मा माꣳ सुवः॒ सुव॒र् माम् मा मा माꣳ सुवः॑ । \newline
49. माꣳ सुवः॒ सुव॒र् माम् माꣳ सुव॑र्. हासी द्धासी॒थ् सुव॒र् माम् माꣳ सुव॑र्. हासीत् । \newline
50. सुव॑र्. हासी द्धासी॒थ् सुवः॒ सुव॑र्. हासी॒न् मयि॒ मयि॑ हासी॒थ् सुवः॒ सुव॑र्. हासी॒न् मयि॑ । \newline
51. हा॒सी॒न् मयि॒ मयि॑ हासी द्धासी॒न् मयि॑ मे॒धाम् मे॒धाम् मयि॑ हासी द्धासी॒न् मयि॑ मे॒धाम् । \newline
52. मयि॑ मे॒धाम् मे॒धाम् मयि॒ मयि॑ मे॒धाम् मयि॒ मयि॑ मे॒धाम् मयि॒ मयि॑ मे॒धाम् मयि॑ । \newline
53. मे॒धाम् मयि॒ मयि॑ मे॒धाम् मे॒धाम् मयि॑ प्र॒जाम् प्र॒जाम् मयि॑ मे॒धाम् मे॒धाम् मयि॑ प्र॒जाम् । \newline
54. मयि॑ प्र॒जाम् प्र॒जाम् मयि॒ मयि॑ प्र॒जाम् मयि॒ मयि॑ प्र॒जाम् मयि॒ मयि॑ प्र॒जाम् मयि॑ । \newline
55. प्र॒जाम् मयि॒ मयि॑ प्र॒जाम् प्र॒जाम् मय्य॒ग्नि र॒ग्निर् मयि॑ प्र॒जाम् प्र॒जाम् मय्य॒ग्निः । \newline
56. प्र॒जामिति॑ प्र - जाम् । \newline
57. मय्य॒ग्नि र॒ग्निर् मयि॒ मय्य॒ग्नि स्तेज॒ स्तेजो॒ ऽग्निर् मयि॒ मय्य॒ग्नि स्तेजः॑ । \newline
58. अ॒ग्नि स्तेज॒ स्तेजो॒ ऽग्नि र॒ग्नि स्तेजो॑ दधातु दधातु॒ तेजो॒ ऽग्नि र॒ग्नि स्तेजो॑ दधातु । \newline
59. तेजो॑ दधातु दधातु॒ तेज॒ स्तेजो॑ दधातु॒ मयि॒ मयि॑ दधातु॒ तेज॒ स्तेजो॑ दधातु॒ मयि॑ । \newline
60. द॒धा॒तु॒ मयि॒ मयि॑ दधातु दधातु॒ मयि॑ मे॒धाम् मे॒धाम् मयि॑ दधातु दधातु॒ मयि॑ मे॒धाम् । \newline
61. मयि॑ मे॒धाम् मे॒धाम् मयि॒ मयि॑ मे॒धाम् मयि॒ मयि॑ मे॒धाम् मयि॒ मयि॑ मे॒धाम् मयि॑ । \newline
62. मे॒धाम् मयि॒ मयि॑ मे॒धाम् मे॒धाम् मयि॑ प्र॒जाम् प्र॒जाम् मयि॑ मे॒धाम् मे॒धाम् मयि॑ प्र॒जाम् । \newline
63. मयि॑ प्र॒जाम् प्र॒जाम् मयि॒ मयि॑ प्र॒जाम् मयि॒ मयि॑ प्र॒जाम् मयि॒ मयि॑ प्र॒जाम् मयि॑ । \newline
64. प्र॒जाम् मयि॒ मयि॑ प्र॒जाम् प्र॒जाम् मयीन्द्र॒ इन्द्रो॒ मयि॑ प्र॒जाम् प्र॒जाम् मयीन्द्रः॑ । \newline
65. प्र॒जामिति॑ प्र - जाम् । \newline
66. मयीन्द्र॒ इन्द्रो॒ मयि॒ मयीन्द्र॑ इन्द्रि॒य मि॑न्द्रि॒य मिन्द्रो॒ मयि॒ मयीन्द्र॑ इन्द्रि॒यम् । \newline
67. इन्द्र॑ इन्द्रि॒य मि॑न्द्रि॒य मिन्द्र॒ इन्द्र॑ इन्द्रि॒यम् द॑धातु दधा त्विन्द्रि॒य मिन्द्र॒ इन्द्र॑ इन्द्रि॒यम् द॑धातु । \newline
68. इ॒न्द्रि॒यम् द॑धातु दधा त्विन्द्रि॒य मि॑न्द्रि॒यम् द॑धातु॒ मयि॒ मयि॑ दधा त्विन्द्रि॒य मि॑न्द्रि॒यम् द॑धातु॒ मयि॑ । \newline
69. द॒धा॒तु॒ मयि॒ मयि॑ दधातु दधातु॒ मयि॑ मे॒धाम् मे॒धाम् मयि॑ दधातु दधातु॒ मयि॑ मे॒धाम् । \newline
70. मयि॑ मे॒धाम् मे॒धाम् मयि॒ मयि॑ मे॒धाम् मयि॒ मयि॑ मे॒धाम् मयि॒ मयि॑ मे॒धाम् मयि॑ । \newline
71. मे॒धाम् मयि॒ मयि॑ मे॒धाम् मे॒धाम् मयि॑ प्र॒जाम् प्र॒जाम् मयि॑ मे॒धाम् मे॒धाम् मयि॑ प्र॒जाम् । \newline
72. मयि॑ प्र॒जाम् प्र॒जाम् मयि॒ मयि॑ प्र॒जाम् मयि॒ मयि॑ प्र॒जाम् मयि॒ मयि॑ प्र॒जाम् मयि॑ । \newline
73. प्र॒जाम् मयि॒ मयि॑ प्र॒जाम् प्र॒जाम् मयि॒ सूर्यः॒ सूर्यो॒ मयि॑ प्र॒जाम् प्र॒जाम् मयि॒ सूर्यः॑ । \newline
74. प्र॒जामिति॑ प्र - जाम् । \newline
75. मयि॒ सूर्यः॒ सूर्यो॒ मयि॒ मयि॒ सूर्यो॒ भ्राजो॒ भ्राजः॒ सूर्यो॒ मयि॒ मयि॒ सूर्यो॒ भ्राजः॑ । \newline
76. सूर्यो॒ भ्राजो॒ भ्राजः॒ सूर्यः॒ सूर्यो॒ भ्राजो॑ दधातु दधातु॒ भ्राजः॒ सूर्यः॒ सूर्यो॒ भ्राजो॑ दधातु । \newline
77. भ्राजो॑ दधातु दधातु॒ भ्राजो॒ भ्राजो॑ दधातु । \newline
78. द॒धा॒त्विति॑ दधातु । \newline
\pagebreak
\markright{ TS 3.3.2.1  \hfill https://www.vedavms.in \hfill}

\section{ TS 3.3.2.1 }

\textbf{TS 3.3.2.1 } \newline
\textbf{Samhita Paata} \newline

वा॒युर्.हि॑कं॒र्ताऽग्निः प्र॑स्तो॒ता प्र॒जाप॑तिः॒ साम॒ बृह॒स्पति॑रुद्गा॒ता विश्वे॑ दे॒वा उ॑पगा॒तारो॑ म॒रुतः॑ प्रतिह॒र्तार॒ इन्द्रो॑ नि॒धन॒न्ते दे॒वाः प्रा॑ण॒भृतः॑ प्रा॒णं मयि॑ दधत्वे॒तद्वै सर्व॑मद्ध्व॒र्यु-रु॑पाकु॒र्वन्नु॑द्गा॒तृभ्य॑ उ॒पाक॑रोति॒ ते दे॒वाः प्रा॑ण॒भृतः॑ प्रा॒णं मयि॑ दध॒त्वित्या॑है॒तदे॒व सव॑र्मा॒त्मन् ध॑त्त॒ इडा॑ देव॒हू र्मनु॑-र्यज्ञ्॒नी-र्बृह॒स्पति॑रुक्थाम॒दानि॑ शꣳसिष॒द् विश्वे॑ दे॒वाः - [  ] \newline

\textbf{Pada Paata} \newline

वा॒युः । हि॒कं॒र्तेति॑ हिं - क॒र्ता । अ॒ग्निः । प्र॒स्तो॒तेति॑ प्र - स्तो॒ता । प्र॒जाप॑ति॒रिति॑ प्र॒जा - प॒तिः॒ । साम॑ । बृह॒स्पतिः॑ । उ॒द्गा॒तेत्यु॑त् - गा॒ता । विश्वे᳚ । दे॒वाः । उ॒प॒गा॒तार॒ इत्यु॑प - गा॒तारः॑ । म॒रुतः॑ । प्र॒ति॒ह॒र्तार॒ इति॑ प्रति - ह॒र्तारः॑ । इन्द्रः॑ । नि॒धन॒मिति॑ नि - धन᳚म् । ते । दे॒वाः । प्रा॒ण॒भृत॒ इति॑ प्राण - भृतः॑ । प्रा॒णमिति॑ प्र - अ॒नम् । मयि॑ । द॒ध॒तु॒ । ए॒तत् । वै । सर्व᳚म् । अ॒द्ध्व॒र्युः । उ॒पा॒कु॒र्वन्नित्यु॑प - आ॒कु॒र्वन्न् । उ॒द्गा॒तृभ्य॒ इत्यु॑द्गा॒तृ - भ्यः॒ । उ॒पाक॑रो॒तीत्यु॑प - आक॑रोति । ते । दे॒वाः । प्रा॒ण॒भृत॒ इति॑ प्राण - भृतः॑ । प्रा॒णमिति॑ प्र-अ॒नम् । मयि॑ । द॒ध॒तु॒ । इति॑ । आ॒ह॒ । ए॒तत् । ए॒व । सर्व᳚म् । आ॒त्मन्न् । ध॒त्ते॒ । इडा᳚ । दे॒व॒हूरिति॑ देव - हूः । मनुः॑ । य॒ज्ञ्॒नीरिति॑ यज्ञ् - नीः । बृह॒स्पतिः॑ । उ॒क्था॒म॒दानीत्यु॑क्थ - म॒दानि॑ । शꣳ॒॒सि॒ष॒त् । विश्वे᳚ । दे॒वाः ।  \newline


\textbf{Krama Paata} \newline

वा॒युर्. हि॑ङ्क॒र्ता । हि॒ङ्क॒र्ता ऽग्निः । हि॒ङ्क॒र्तेति॑ हिं - क॒र्ता । अ॒ग्निः प्र॑स्तो॒ता । प्र॒स्तो॒ता प्र॒जाप॑तिः । प्र॒स्तो॒तेति॑ प्र - स्तो॒ता । प्र॒जाप॑तिः॒ साम॑ । प्र॒जाप॑ति॒रिति॑ प्र॒जा - प॒तिः॒ । साम॒ बृह॒स्पतिः॑ । बृह॒स्पति॑रुद्गा॒ता । उ॒द्गा॒ता विश्वे᳚ । उ॒द्गा॒तेत्यु॑त् - गा॒ता । विश्वे॑ दे॒वाः । दे॒वा उ॑पगा॒तारः॑ । उ॒प॒गा॒तारो॑ म॒रुतः॑ । उ॒प॒गा॒तार॒ इत्यु॑प - गा॒तारः॑ । म॒रुतः॑ प्रतिह॒र्तारः॑ । प्र॒ति॒ह॒र्तार॒ इन्द्रः॑ । प्र॒ति॒ह॒र्तार॒ इति॑ प्रति - ह॒र्तारः॑ । इन्द्रो॑ नि॒धन᳚म् । नि॒धन॒म् ते । नि॒धन॒मिति॑ नि - धन᳚म् । ते दे॒वाः । दे॒वाः प्रा॑ण॒भृतः॑ । प्रा॒ण॒भृतः॑ प्रा॒णम् । प्रा॒ण॒भृत॒ इति॑ प्राण - भृतः॑ । प्रा॒णम् मयि॑ । प्रा॒णमिति॑ प्र - अ॒नम् । मयि॑ दधतु । द॒ध॒त्वे॒तत् । ए॒तद् वै । वै सर्व᳚म् । सर्व॑मद्ध्व॒र्युः । अ॒द्ध्व॒र्युरु॑पाकु॒र्वन्न् । उ॒पा॒कु॒र्वन्नु॑द्गा॒तृभ्यः॑ । उ॒पा॒कु॒र्वन्नित्यु॑प - आ॒कु॒र्वन्न् । उ॒द्गा॒तृभ्य॑ उ॒पाक॑रोति । उ॒द्गा॒तृभ्य॒ इत्यु॑द्गा॒तृ - भ्यः॒ । उ॒पाक॑रोति॒ ते । उ॒पाक॑रो॒तीत्यु॑प - आक॑रोति । ते दे॒वाः । दे॒वाः प्रा॑ण॒भृतः॑ । प्रा॒ण॒भृतः॑ प्रा॒णम् । प्रा॒ण॒भृत॒ इति॑ प्राण - भृतः॑ । प्रा॒णम् मयि॑ । प्रा॒णमिति॑ प्र - अ॒नम् । मयि॑ दधतु । द॒ध॒त्विति॑ । इत्या॑ह । आ॒है॒तत् । ए॒तदे॒व । ए॒व सर्व᳚म् । सर्व॑मा॒त्मन्न् । आ॒त्मन् ध॑त्ते । ध॒त्त॒ इडा᳚ । इडा॑ देव॒हूः । दे॒व॒हूर् मनुः॑ । दे॒व॒हूरिति॑ देव - हूः । मनु॑र् यज्ञ्॒नीः । य॒ज्ञ्॒नीर् बृह॒स्पतिः॑ । य॒ज्ञ्॒नीरिति॑ यज्ञ् - नीः । बृह॒स्पति॑रुक्थाम॒दानि॑ । उ॒क्था॒म॒दानि॑ शꣳसिषत् । उ॒क्था॒म॒दानीत्यु॑क्थ - म॒दानि॑ । शꣳ॒॒सि॒ष॒द् विश्वे᳚ । विश्वे॑ दे॒वाः ( ) । दे॒वाः सू᳚क्त॒वाचः॑ \newline

\textbf{Jatai Paata} \newline

1. वा॒युर्. हि॑ङ्क॒र्ता हि॑ङ्क॒र्ता वा॒युर् वा॒युर्. हि॑ङ्क॒र्ता । \newline
2. हि॒ङ्क॒र्ता ऽग्नि र॒ग्निर्. हि॑ङ्क॒र्ता हि॑ङ्क॒र्ता ऽग्निः । \newline
3. हि॒ङ्क॒र्तेति॑ हिम् - क॒र्ता । \newline
4. अ॒ग्निः प्र॑स्तो॒ता प्र॑स्तो॒ता ऽग्नि र॒ग्निः प्र॑स्तो॒ता । \newline
5. प्र॒स्तो॒ता प्र॒जाप॑तिः प्र॒जाप॑तिः प्रस्तो॒ता प्र॑स्तो॒ता प्र॒जाप॑तिः । \newline
6. प्र॒स्तो॒तेति॑ प्र - स्तो॒ता । \newline
7. प्र॒जाप॑तिः॒ साम॒ साम॑ प्र॒जाप॑तिः प्र॒जाप॑तिः॒ साम॑ । \newline
8. प्र॒जाप॑ति॒रिति॑ प्र॒जा - प॒तिः॒ । \newline
9. साम॒ बृह॒स्पति॒र् बृह॒स्पतिः॒ साम॒ साम॒ बृह॒स्पतिः॑ । \newline
10. बृह॒स्पति॑ रुद्‍गा॒ तोद्‍गा॒ता बृह॒स्पति॒र् बृह॒स्पति॑ रुद्‍गा॒ता । \newline
11. उ॒द्‍गा॒ता विश्वे॒ विश्व॑ उद्‍गा॒ तोद्‍गा॒ता विश्वे᳚ । \newline
12. उ॒द्‍गा॒तेत्यु॑त् - गा॒ता । \newline
13. विश्वे॑ दे॒वा दे॒वा विश्वे॒ विश्वे॑ दे॒वाः । \newline
14. दे॒वा उ॑पगा॒तार॑ उपगा॒तारो॑ दे॒वा दे॒वा उ॑पगा॒तारः॑ । \newline
15. उ॒प॒गा॒तारो॑ म॒रुतो॑ म॒रुत॑ उपगा॒तार॑ उपगा॒तारो॑ म॒रुतः॑ । \newline
16. उ॒प॒गा॒तार॒ इत्यु॑प - गा॒तारः॑ । \newline
17. म॒रुतः॑ प्रतिह॒र्तारः॑ प्रतिह॒र्तारो॑ म॒रुतो॑ म॒रुतः॑ प्रतिह॒र्तारः॑ । \newline
18. प्र॒ति॒ह॒र्तार॒ इन्द्र॒ इन्द्रः॑ प्रतिह॒र्तारः॑ प्रतिह॒र्तार॒ इन्द्रः॑ । \newline
19. प्र॒ति॒ह॒र्तार॒ इति॑ प्रति - ह॒र्तारः॑ । \newline
20. इन्द्रो॑ नि॒धन॑म् नि॒धन॒ मिन्द्र॒ इन्द्रो॑ नि॒धन᳚म् । \newline
21. नि॒धन॒म् ते ते नि॒धन॑म् नि॒धन॒म् ते । \newline
22. नि॒धन॒मिति॑ नि - धन᳚म् । \newline
23. ते दे॒वा दे॒वा स्ते ते दे॒वाः । \newline
24. दे॒वाः प्रा॑ण॒भृतः॑ प्राण॒भृतो॑ दे॒वा दे॒वाः प्रा॑ण॒भृतः॑ । \newline
25. प्रा॒ण॒भृतः॑ प्रा॒णम् प्रा॒णम् प्रा॑ण॒भृतः॑ प्राण॒भृतः॑ प्रा॒णम् । \newline
26. प्रा॒ण॒भृत॒ इति॑ प्राण - भृतः॑ । \newline
27. प्रा॒णम् मयि॒ मयि॑ प्रा॒णम् प्रा॒णम् मयि॑ । \newline
28. प्रा॒णमिति॑ प्र - अ॒नम् । \newline
29. मयि॑ दधतु दधतु॒ मयि॒ मयि॑ दधतु । \newline
30. द॒ध॒ त्वे॒त दे॒तद् द॑धतु दध त्वे॒तत् । \newline
31. ए॒तद् वै वा ए॒त दे॒तद् वै । \newline
32. वै सर्वꣳ॒॒ सर्व॒म् ॅवै वै सर्व᳚म् । \newline
33. सर्व॑ मद्ध्व॒र्यु र॑द्ध्व॒र्युः सर्वꣳ॒॒ सर्व॑ मद्ध्व॒र्युः । \newline
34. अ॒द्ध्व॒र्यु रु॑पाकु॒र्वन् नु॑पाकु॒र्वन् न॑द्ध्व॒र्यु र॑द्ध्व॒र्यु रु॑पाकु॒र्वन्न् । \newline
35. उ॒पा॒कु॒र्वन् नु॑द्‍गा॒तृभ्य॑ उद्‍गा॒तृभ्य॑ उपाकु॒र्वन् नु॑पाकु॒र्वन् नु॑द्‍गा॒तृभ्यः॑ । \newline
36. उ॒पा॒कु॒र्वन्नित्यु॑प - आ॒कु॒र्वन्न् । \newline
37. उ॒द्‍गा॒तृभ्य॑ उ॒पाक॑रो त्यु॒पाक॑रो त्युद्‍गा॒तृभ्य॑ उद्‍गा॒तृभ्य॑ उ॒पाक॑रोति । \newline
38. उ॒द्‍गा॒तृभ्य॒ इत्यु॑द्‍गा॒तृ - भ्यः॒ । \newline
39. उ॒पाक॑रोति॒ ते त उ॒पाक॑रो त्यु॒पाक॑रोति॒ ते । \newline
40. उ॒पाक॑रो॒तीत्यु॑प - आक॑रोति । \newline
41. ते दे॒वा दे॒वा स्ते ते दे॒वाः । \newline
42. दे॒वाः प्रा॑ण॒भृतः॑ प्राण॒भृतो॑ दे॒वा दे॒वाः प्रा॑ण॒भृतः॑ । \newline
43. प्रा॒ण॒भृतः॑ प्रा॒णम् प्रा॒णम् प्रा॑ण॒भृतः॑ प्राण॒भृतः॑ प्रा॒णम् । \newline
44. प्रा॒ण॒भृत॒ इति॑ प्राण - भृतः॑ । \newline
45. प्रा॒णम् मयि॒ मयि॑ प्रा॒णम् प्रा॒णम् मयि॑ । \newline
46. प्रा॒णमिति॑ प्र - अ॒नम् । \newline
47. मयि॑ दधतु दधतु॒ मयि॒ मयि॑ दधतु । \newline
48. द॒ध॒ त्वितीति॑ दधतु दध॒ त्विति॑ । \newline
49. इत्या॑हा॒हे तीत्या॑ह । \newline
50. आ॒है॒त दे॒त दा॑हाहै॒तत् । \newline
51. ए॒त दे॒वैवैत दे॒तदे॒व । \newline
52. ए॒व सर्वꣳ॒॒ सर्व॑ मे॒वैव सर्व᳚म् । \newline
53. सर्व॑ मा॒त्मन् ना॒त्मन् थ्सर्वꣳ॒॒ सर्व॑ मा॒त्मन्न् । \newline
54. आ॒त्मन् ध॑त्ते धत्त आ॒त्मन् ना॒त्मन् ध॑त्ते । \newline
55. ध॒त्त॒ इडेडा॑ धत्ते धत्त॒ इडा᳚ । \newline
56. इडा॑ देव॒हूर् दे॑व॒हू रिडेडा॑ देव॒हूः । \newline
57. दे॒व॒हूर् मनु॒र् मनु॑र् देव॒हूर् दे॑व॒हूर् मनुः॑ । \newline
58. दे॒व॒हूरिति॑ देव - हूः । \newline
59. मनु॑र् यज्ञ्॒नीर् य॑ज्ञ्॒नीर् मनु॒र् मनु॑र् यज्ञ्॒नीः । \newline
60. य॒ज्ञ्॒नीर् बृह॒स्पति॒र् बृह॒स्पति॑र् यज्ञ्॒नीर् य॑ज्ञ्॒नीर् बृह॒स्पतिः॑ । \newline
61. य॒ज्ञ्॒नीरिति॑ यज्ञ् - नीः । \newline
62. बृह॒स्पति॑ रुक्थाम॒दा न्यु॑क्थाम॒दानि॒ बृह॒स्पति॒र् बृह॒स्पति॑ रुक्थाम॒दानि॑ । \newline
63. उ॒क्था॒म॒दानि॑ शꣳसिष च्छꣳसिष दुक्थाम॒दा न्यु॑क्थाम॒दानि॑ शꣳसिषत् । \newline
64. उ॒क्था॒म॒दानीत्यु॑क्थ - म॒दानि॑ । \newline
65. शꣳ॒॒सि॒ष॒द् विश्वे॒ विश्वे॑ शꣳसिष च्छꣳसिष॒द् विश्वे᳚ । \newline
66. विश्वे॑ दे॒वा दे॒वा विश्वे॒ विश्वे॑ दे॒वाः । \newline
67. दे॒वाः सू᳚क्त॒वाचः॑ सूक्त॒वाचो॑ दे॒वा दे॒वाः सू᳚क्त॒वाचः॑ । \newline

\textbf{Ghana Paata } \newline

1. वा॒युर्. हि॑ङ्क॒र्ता हि॑ङ्क॒र्ता वा॒युर् वा॒युर्. हि॑ङ्क॒र्ता ऽग्नि र॒ग्निर्. हि॑ङ्क॒र्ता वा॒युर् वा॒युर्. हि॑ङ्क॒र्ता ऽग्निः । \newline
2. हि॒ङ्क॒र्ता ऽग्नि र॒ग्निर्. हि॑ङ्क॒र्ता हि॑ङ्क॒र्ता ऽग्निः प्र॑स्तो॒ता प्र॑स्तो॒ता ऽग्निर्. हि॑ङ्क॒र्ता हि॑ङ्क॒र्ता ऽग्निः प्र॑स्तो॒ता । \newline
3. हि॒ङ्क॒र्तेति॑ हिम् - क॒र्ता । \newline
4. अ॒ग्निः प्र॑स्तो॒ता प्र॑स्तो॒ता ऽग्नि र॒ग्निः प्र॑स्तो॒ता प्र॒जाप॑तिः प्र॒जाप॑तिः प्रस्तो॒ता ऽग्नि र॒ग्निः प्र॑स्तो॒ता प्र॒जाप॑तिः । \newline
5. प्र॒स्तो॒ता प्र॒जाप॑तिः प्र॒जाप॑तिः प्रस्तो॒ता प्र॑स्तो॒ता प्र॒जाप॑तिः॒ साम॒ साम॑ प्र॒जाप॑तिः प्रस्तो॒ता प्र॑स्तो॒ता प्र॒जाप॑तिः॒ साम॑ । \newline
6. प्र॒स्तो॒तेति॑ प्र - स्तो॒ता । \newline
7. प्र॒जाप॑तिः॒ साम॒ साम॑ प्र॒जाप॑तिः प्र॒जाप॑तिः॒ साम॒ बृह॒स्पति॒र् बृह॒स्पतिः॒ साम॑ प्र॒जाप॑तिः प्र॒जाप॑तिः॒ साम॒ बृह॒स्पतिः॑ । \newline
8. प्र॒जाप॑ति॒रिति॑ प्र॒जा - प॒तिः॒ । \newline
9. साम॒ बृह॒स्पति॒र् बृह॒स्पतिः॒ साम॒ साम॒ बृह॒स्पति॑ रुद्‍गा॒ तोद्‍गा॒ता बृह॒स्पतिः॒ साम॒ साम॒ बृह॒स्पति॑ रुद्‍गा॒ता । \newline
10. बृह॒स्पति॑ रुद्‍गा॒ तोद्‍गा॒ता बृह॒स्पति॒र् बृह॒स्पति॑ रुद्‍गा॒ता विश्वे॒ विश्व॑ उद्‍गा॒ता बृह॒स्पति॒र् बृह॒स्पति॑ रुद्‍गा॒ता विश्वे᳚ । \newline
11. उ॒द्‍गा॒ता विश्वे॒ विश्व॑ उद्‍गा॒ तोद्‍गा॒ता विश्वे॑ दे॒वा दे॒वा विश्व॑ उद्‍गा॒ तोद्‍गा॒ता विश्वे॑ दे॒वाः । \newline
12. उ॒द्‍गा॒तेत्यु॑त् - गा॒ता । \newline
13. विश्वे॑ दे॒वा दे॒वा विश्वे॒ विश्वे॑ दे॒वा उ॑पगा॒तार॑ उपगा॒तारो॑ दे॒वा विश्वे॒ विश्वे॑ दे॒वा उ॑पगा॒तारः॑ । \newline
14. दे॒वा उ॑पगा॒तार॑ उपगा॒तारो॑ दे॒वा दे॒वा उ॑पगा॒तारो॑ म॒रुतो॑ म॒रुत॑ उपगा॒तारो॑ दे॒वा दे॒वा उ॑पगा॒तारो॑ म॒रुतः॑ । \newline
15. उ॒प॒गा॒तारो॑ म॒रुतो॑ म॒रुत॑ उपगा॒तार॑ उपगा॒तारो॑ म॒रुतः॑ प्रतिह॒र्तारः॑ प्रतिह॒र्तारो॑ म॒रुत॑ उपगा॒तार॑ उपगा॒तारो॑ म॒रुतः॑ प्रतिह॒र्तारः॑ । \newline
16. उ॒प॒गा॒तार॒ इत्यु॑प - गा॒तारः॑ । \newline
17. म॒रुतः॑ प्रतिह॒र्तारः॑ प्रतिह॒र्तारो॑ म॒रुतो॑ म॒रुतः॑ प्रतिह॒र्तार॒ इन्द्र॒ इन्द्रः॑ प्रतिह॒र्तारो॑ म॒रुतो॑ म॒रुतः॑ प्रतिह॒र्तार॒ इन्द्रः॑ । \newline
18. प्र॒ति॒ह॒र्तार॒ इन्द्र॒ इन्द्रः॑ प्रतिह॒र्तारः॑ प्रतिह॒र्तार॒ इन्द्रो॑ नि॒धन॑म् नि॒धन॒ मिन्द्रः॑ प्रतिह॒र्तारः॑ प्रतिह॒र्तार॒ इन्द्रो॑ नि॒धन᳚म् । \newline
19. प्र॒ति॒ह॒र्तार॒ इति॑ प्रति - ह॒र्तारः॑ । \newline
20. इन्द्रो॑ नि॒धन॑म् नि॒धन॒ मिन्द्र॒ इन्द्रो॑ नि॒धन॒म् ते ते नि॒धन॒ मिन्द्र॒ इन्द्रो॑ नि॒धन॒म् ते । \newline
21. नि॒धन॒म् ते ते नि॒धन॑म् नि॒धन॒म् ते दे॒वा दे॒वा स्ते नि॒धन॑म् नि॒धन॒म् ते दे॒वाः । \newline
22. नि॒धन॒मिति॑ नि - धन᳚म् । \newline
23. ते दे॒वा दे॒वा स्ते ते दे॒वाः प्रा॑ण॒भृतः॑ प्राण॒भृतो॑ दे॒वा स्ते ते दे॒वाः प्रा॑ण॒भृतः॑ । \newline
24. दे॒वाः प्रा॑ण॒भृतः॑ प्राण॒भृतो॑ दे॒वा दे॒वाः प्रा॑ण॒भृतः॑ प्रा॒णम् प्रा॒णम् प्रा॑ण॒भृतो॑ दे॒वा दे॒वाः प्रा॑ण॒भृतः॑ प्रा॒णम् । \newline
25. प्रा॒ण॒भृतः॑ प्रा॒णम् प्रा॒णम् प्रा॑ण॒भृतः॑ प्राण॒भृतः॑ प्रा॒णम् मयि॒ मयि॑ प्रा॒णम् प्रा॑ण॒भृतः॑ प्राण॒भृतः॑ प्रा॒णम् मयि॑ । \newline
26. प्रा॒ण॒भृत॒ इति॑ प्राण - भृतः॑ । \newline
27. प्रा॒णम् मयि॒ मयि॑ प्रा॒णम् प्रा॒णम् मयि॑ दधतु दधतु॒ मयि॑ प्रा॒णम् प्रा॒णम् मयि॑ दधतु । \newline
28. प्रा॒णमिति॑ प्र - अ॒नम् । \newline
29. मयि॑ दधतु दधतु॒ मयि॒ मयि॑ दध त्वे॒त दे॒तद् द॑धतु॒ मयि॒ मयि॑ दध त्वे॒तत् । \newline
30. द॒ध॒ त्वे॒त दे॒तद् द॑धतु दध त्वे॒तद् वै वा ए॒तद् द॑धतु दध त्वे॒तद् वै । \newline
31. ए॒तद् वै वा ए॒त दे॒तद् वै सर्वꣳ॒॒ सर्व॒म् ॅवा ए॒त दे॒तद् वै सर्व᳚म् । \newline
32. वै सर्वꣳ॒॒ सर्व॒म् ॅवै वै सर्व॑ मद्ध्व॒र्यु र॑द्ध्व॒र्युः सर्व॒म् ॅवै वै सर्व॑ मद्ध्व॒र्युः । \newline
33. सर्व॑ मद्ध्व॒र्यु र॑द्ध्व॒र्युः सर्वꣳ॒॒ सर्व॑ मद्ध्व॒र्यु रु॑पाकु॒र्वन्, नु॑पाकु॒र्वन्, न॑द्ध्व॒र्युः सर्वꣳ॒॒ सर्व॑ मद्ध्व॒र्यु रु॑पाकु॒र्वन्न् । \newline
34. अ॒द्ध्व॒र्यु रु॑पाकु॒र्वन्, नु॑पाकु॒र्वन्, न॑द्ध्व॒र्यु र॑द्ध्व॒र्यु रु॑पाकु॒र्वन्, नु॑द्‍गा॒तृभ्य॑ उद्‍गा॒तृभ्य॑ उपाकु॒र्वन्, न॑द्ध्व॒र्यु र॑द्ध्व॒र्यु रु॑पाकु॒र्वन्, नु॑द्‍गा॒तृभ्यः॑ । \newline
35. उ॒पा॒कु॒र्वन्, नु॑द्‍गा॒तृभ्य॑ उद्‍गा॒तृभ्य॑ उपाकु॒र्वन्, नु॑पाकु॒र्वन्, नु॑द्‍गा॒तृभ्य॑ उ॒पाक॑रो त्यु॒पाक॑रो त्युद्‍गा॒तृभ्य॑ उपाकु॒र्वन्, नु॑पाकु॒र्वन्, नु॑द्‍गा॒तृभ्य॑ उ॒पाक॑रोति । \newline
36. उ॒पा॒कु॒र्वन्नित्यु॑प - आ॒कु॒र्वन्न् । \newline
37. उ॒द्‍गा॒तृभ्य॑ उ॒पाक॑रो त्यु॒पाक॑रो त्युद्‍गा॒तृभ्य॑ उद्‍गा॒तृभ्य॑ उ॒पाक॑रोति॒ ते त उ॒पाक॑रो त्युद्‍गा॒तृभ्य॑ उद्‍गा॒तृभ्य॑ उ॒पाक॑रोति॒ ते । \newline
38. उ॒द्‍गा॒तृभ्य॒ इत्यु॑द्‍गा॒तृ - भ्यः॒ । \newline
39. उ॒पाक॑रोति॒ ते त उ॒पाक॑रो त्यु॒पाक॑रोति॒ ते दे॒वा दे॒वा स्त उ॒पाक॑रो त्यु॒पाक॑रोति॒ ते दे॒वाः । \newline
40. उ॒पाक॑रो॒तीत्यु॑प - आक॑रोति । \newline
41. ते दे॒वा दे॒वा स्ते ते दे॒वाः प्रा॑ण॒भृतः॑ प्राण॒भृतो॑ दे॒वा स्ते ते दे॒वाः प्रा॑ण॒भृतः॑ । \newline
42. दे॒वाः प्रा॑ण॒भृतः॑ प्राण॒भृतो॑ दे॒वा दे॒वाः प्रा॑ण॒भृतः॑ प्रा॒णम् प्रा॒णम् प्रा॑ण॒भृतो॑ दे॒वा दे॒वाः प्रा॑ण॒भृतः॑ प्रा॒णम् । \newline
43. प्रा॒ण॒भृतः॑ प्रा॒णम् प्रा॒णम् प्रा॑ण॒भृतः॑ प्राण॒भृतः॑ प्रा॒णम् मयि॒ मयि॑ प्रा॒णम् प्रा॑ण॒भृतः॑ प्राण॒भृतः॑ प्रा॒णम् मयि॑ । \newline
44. प्रा॒ण॒भृत॒ इति॑ प्राण - भृतः॑ । \newline
45. प्रा॒णम् मयि॒ मयि॑ प्रा॒णम् प्रा॒णम् मयि॑ दधतु दधतु॒ मयि॑ प्रा॒णम् प्रा॒णम् मयि॑ दधतु । \newline
46. प्रा॒णमिति॑ प्र - अ॒नम् । \newline
47. मयि॑ दधतु दधतु॒ मयि॒ मयि॑ दध॒ त्वितीति॑ दधतु॒ मयि॒ मयि॑ दध॒ त्विति॑ । \newline
48. द॒ध॒ त्वितीति॑ दधतु दध॒ त्वित्या॑हा॒हे ति॑ दधतु दध॒ त्वित्या॑ह । \newline
49. इत्या॑हा॒हे तीत्या॑ है॒त दे॒त दा॒हे तीत्या॑है॒तत् । \newline
50. आ॒है॒त दे॒त दा॑हा है॒त दे॒वैवैत दा॑हाहै॒त दे॒व । \newline
51. ए॒त दे॒वैवै तदे॒त दे॒व सर्वꣳ॒॒ सर्व॑ मे॒वै तदे॒तदे॒व सर्व᳚म् । \newline
52. ए॒व सर्वꣳ॒॒ सर्व॑ मे॒वैव सर्व॑ मा॒त्मन्, ना॒त्मन् थ्सर्व॑ मे॒वैव सर्व॑ मा॒त्मन्न् । \newline
53. सर्व॑ मा॒त्मन्, ना॒त्मन् थ्सर्वꣳ॒॒ सर्व॑ मा॒त्मन् ध॑त्ते धत्त आ॒त्मन् थ्सर्वꣳ॒॒ सर्व॑ मा॒त्मन् ध॑त्ते । \newline
54. आ॒त्मन् ध॑त्ते धत्त आ॒त्मन्, ना॒त्मन् ध॑त्त॒ इडेडा॑ धत्त आ॒त्मन्, ना॒त्मन् ध॑त्त॒ इडा᳚ । \newline
55. ध॒त्त॒ इडेडा॑ धत्ते धत्त॒ इडा॑ देव॒हूर् दे॑व॒हू रिडा॑ धत्ते धत्त॒ इडा॑ देव॒हूः । \newline
56. इडा॑ देव॒हूर् दे॑व॒हू रिडेडा॑ देव॒हूर् मनु॒र् मनु॑र् देव॒हू रिडेडा॑ देव॒हूर् मनुः॑ । \newline
57. दे॒व॒हूर् मनु॒र् मनु॑र् देव॒हूर् दे॑व॒हूर् मनु॑र् यज्ञ्॒नीर् य॑ज्ञ्॒नीर् मनु॑र् देव॒हूर् दे॑व॒हूर् मनु॑र् यज्ञ्॒नीः । \newline
58. दे॒व॒हूरिति॑ देव - हूः । \newline
59. मनु॑र् यज्ञ्॒नीर् य॑ज्ञ्॒नीर् मनु॒र् मनु॑र् यज्ञ्॒नीर् बृह॒स्पति॒र् बृह॒स्पति॑र् यज्ञ्॒नीर् मनु॒र् मनु॑र् यज्ञ्॒नीर् बृह॒स्पतिः॑ । \newline
60. य॒ज्ञ्॒नीर् बृह॒स्पति॒र् बृह॒स्पति॑र् यज्ञ्॒नीर् य॑ज्ञ्॒नीर् बृह॒स्पति॑ रुक्थाम॒दा न्यु॑क्थाम॒दानि॒ बृह॒स्पति॑र् यज्ञ्॒नीर् य॑ज्ञ्॒नीर् बृह॒स्पति॑ रुक्थाम॒दानि॑ । \newline
61. य॒ज्ञ्॒नीरिति॑ यज्ञ् - नीः । \newline
62. बृह॒स्पति॑ रुक्थाम॒दा न्यु॑क्थाम॒दानि॒ बृह॒स्पति॒र् बृह॒स्पति॑ रुक्थाम॒दानि॑ 
शꣳसिषच् छꣳसिष दुक्थाम॒दानि॒ बृह॒स्पति॒र् बृह॒स्पति॑ रुक्थाम॒दानि॑ शꣳसिषत् । \newline
63. उ॒क्था॒म॒दानि॑ शꣳसिषच् छꣳसिष दुक्थाम॒दा न्यु॑क्थाम॒दानि॑ शꣳसिष॒द् विश्वे॒ विश्वे॑ 
शꣳसिष दुक्थाम॒दा न्यु॑क्थाम॒दानि॑ शꣳसिष॒द् विश्वे᳚ । \newline
64. उ॒क्था॒म॒दानीत्यु॑क्थ - म॒दानि॑ । \newline
65. शꣳ॒॒सि॒ष॒द् विश्वे॒ विश्वे॑ शꣳसिषच् छꣳसिष॒द् विश्वे॑ दे॒वा दे॒वा विश्वे॑ शꣳसिषच् छꣳसिष॒द् विश्वे॑ दे॒वाः । \newline
66. विश्वे॑ दे॒वा दे॒वा विश्वे॒ विश्वे॑ दे॒वाः सू᳚क्त॒वाचः॑ सूक्त॒वाचो॑ दे॒वा विश्वे॒ विश्वे॑ दे॒वाः सू᳚क्त॒वाचः॑ । \newline
67. दे॒वाः सू᳚क्त॒वाचः॑ सूक्त॒वाचो॑ दे॒वा दे॒वाः सू᳚क्त॒वाचः॒ पृथि॑वि॒ पृथि॑वि सूक्त॒वाचो॑ दे॒वा दे॒वाः सू᳚क्त॒वाचः॒ पृथि॑वि । \newline
\pagebreak
\markright{ TS 3.3.2.2  \hfill https://www.vedavms.in \hfill}

\section{ TS 3.3.2.2 }

\textbf{TS 3.3.2.2 } \newline
\textbf{Samhita Paata} \newline

सू᳚क्त॒वाचः॒ पृथि॑वि मात॒र्मा मा॑हिꣳसी॒ र्मधु॑ मनिष्ये॒ मधु॑ जनिष्ये॒ मधु॑वक्ष्यामि॒ मधु॑वदिष्यामि॒ मधु॑मतीं दे॒वेभ्यो॒ वाच॑मुद्यासꣳ शुश्रू॒षेण्यां᳚ मनु॒ष्ये᳚भ्य॒स्तं मा॑ दे॒वा अ॑वन्तु शो॒भायै॑ पि॒तरोऽनु॑ मदन्तु ॥ \newline

\textbf{Pada Paata} \newline

सू॒क्त॒वाच॒ इति॑ सूक्त - वाचः॑ । पृथि॑वि । मा॒तः॒ । मा । मा॒ । हिꣳ॒॒सीः॒ । मधु॑ । म॒नि॒ष्ये॒ । मधु॑ । ज॒नि॒ष्ये॒ । मधु॑ । व॒क्ष्या॒मि॒ । मधु॑ । व॒दि॒ष्या॒मि॒ । मधु॑मती॒मिति॒ मधु॑ - म॒ती॒म् । दे॒वेभ्यः॑ । वाच᳚म् । उ॒द्या॒स॒म् । शु॒श्रू॒षेण्या᳚म् । म॒नु॒ष्ये᳚भ्यः । तम् । मा॒ । दे॒वाः । अ॒व॒न्तु॒ । शो॒भायै᳚ । पि॒तरः॑ । अन्विति॑ । म॒द॒न्तु॒ ॥  \newline


\textbf{Krama Paata} \newline

सू॒क्त॒वाचः॒ पृथि॑वि । सू॒क्त॒वाच॒ इति॑ सूक्त - वाचः॑ । पृथि॑वि मातः । मा॒त॒र् मा । मा मा᳚ । मा॒ हिꣳ॒॒सीः॒ । हिꣳ॒॒सी॒र् मधु॑ । मधु॑ मनिष्ये । म॒नि॒ष्ये॒ मधु॑ । मधु॑ जनिष्ये । ज॒नि॒ष्ये॒ मधु॑ । मधु॑ वक्ष्यामि । व॒क्ष्या॒मि॒ मधु॑ । मधु॑ वदिष्यामि । व॒दि॒ष्या॒मि॒ मधु॑मतीम् । मधु॑मतीम् दे॒वेभ्यः॑ । मधु॑मती॒मिति॒ मधु॑ - म॒ती॒म् । दे॒वेभ्यो॒ वाच᳚म् । वाच॑मुद्यासम् । उ॒द्या॒सꣳ॒॒ शु॒श्रू॒षेण्या᳚म् । शु॒श्रू॒षेण्या᳚म् मनु॒ष्ये᳚भ्यः । म॒नु॒ष्ये᳚भ्य॒स्तम् । तम् मा᳚ । मा॒ दे॒वाः । दे॒वा अ॑वन्तु । अ॒व॒न्तु॒ शो॒भायै᳚ । शो॒भायै॑ पि॒तरः॑ । पि॒तरो ऽनु॑ । अनु॑मदन्तु । 
म॒द॒न्त्विति॑ मदन्तु । \newline

\textbf{Jatai Paata} \newline

1. सू॒क्त॒वाचः॒ पृथि॑वि॒ पृथि॑वि सूक्त॒वाचः॑ सूक्त॒वाचः॒ पृथि॑वि । \newline
2. सू॒क्त॒वाच॒ इति॑ सूक्त - वाचः॑ । \newline
3. पृथि॑वि मातर् मातः॒ पृथि॑वि॒ पृथि॑वि मातः । \newline
4. मा॒त॒र् मा मा मा॑तर् मात॒र् मा । \newline
5. मा मा॑ मा॒ मा मा मा᳚ । \newline
6. मा॒ हिꣳ॒॒सी॒र्॒. हिꣳ॒॒सी॒र् मा॒ मा॒ हिꣳ॒॒सीः॒ । \newline
7. हिꣳ॒॒सी॒र् मधु॒ मधु॑ हिꣳसीर्. हिꣳसी॒र् मधु॑ । \newline
8. मधु॑ मनिष्ये मनिष्ये॒ मधु॒ मधु॑ मनिष्ये । \newline
9. म॒नि॒ष्ये॒ मधु॒ मधु॑ मनिष्ये मनिष्ये॒ मधु॑ । \newline
10. मधु॑ जनिष्ये जनिष्ये॒ मधु॒ मधु॑ जनिष्ये । \newline
11. ज॒नि॒ष्ये॒ मधु॒ मधु॑ जनिष्ये जनिष्ये॒ मधु॑ । \newline
12. मधु॑ वक्ष्यामि वक्ष्यामि॒ मधु॒ मधु॑ वक्ष्यामि । \newline
13. व॒क्ष्या॒मि॒ मधु॒ मधु॑ वक्ष्यामि वक्ष्यामि॒ मधु॑ । \newline
14. मधु॑ वदिष्यामि वदिष्यामि॒ मधु॒ मधु॑ वदिष्यामि । \newline
15. व॒दि॒ष्या॒मि॒ मधु॑मती॒म् मधु॑मतीम् ॅवदिष्यामि वदिष्यामि॒ मधु॑मतीम् । \newline
16. मधु॑मतीम् दे॒वेभ्यो॑ दे॒वेभ्यो॒ मधु॑मती॒म् मधु॑मतीम् दे॒वेभ्यः॑ । \newline
17. मधु॑मती॒मिति॒ मधु॑ - म॒ती॒म् । \newline
18. दे॒वेभ्यो॒ वाच॒म् ॅवाच॑म् दे॒वेभ्यो॑ दे॒वेभ्यो॒ वाच᳚म् । \newline
19. वाच॑ मुद्यास मुद्यास॒म् ॅवाच॒म् ॅवाच॑ मुद्यासम् । \newline
20. उ॒द्या॒सꣳ॒॒ शु॒श्रू॒षेण्याꣳ॑ शुश्रू॒षेण्या॑ मुद्यास मुद्यासꣳ शुश्रू॒षेण्या᳚म् । \newline
21. शु॒श्रू॒षेण्या᳚म् मनु॒ष्ये᳚भ्यो मनु॒ष्ये᳚भ्यः शुश्रू॒षेण्याꣳ॑ शुश्रू॒षेण्या᳚म् मनु॒ष्ये᳚भ्यः । \newline
22. म॒नु॒ष्ये᳚भ्य॒ स्तम् तम् म॑नु॒ष्ये᳚भ्यो मनु॒ष्ये᳚भ्य॒ स्तम् । \newline
23. तम् मा॑ मा॒ तम् तम् मा᳚ । \newline
24. मा॒ दे॒वा दे॒वा मा॑ मा दे॒वाः । \newline
25. दे॒वा अ॑व न्त्ववन्तु दे॒वा दे॒वा अ॑वन्तु । \newline
26. अ॒व॒न्तु॒ शो॒भायै॑ शो॒भाया॑ अव न्त्ववन्तु शो॒भायै᳚ । \newline
27. शो॒भायै॑ पि॒तरः॑ पि॒तरः॑ शो॒भायै॑ शो॒भायै॑ पि॒तरः॑ । \newline
28. पि॒तरो ऽन्वनु॑ पि॒तरः॑ पि॒तरो ऽनु॑ । \newline
29. अनु॑ मदन्तु मद॒ न्त्वन्वनु॑ मदन्तु । \newline
30. म॒द॒न्त्विति॑ मदन्तु । \newline

\textbf{Ghana Paata } \newline

1. सू॒क्त॒वाचः॒ पृथि॑वि॒ पृथि॑वि सूक्त॒वाचः॑ सूक्त॒वाचः॒ पृथि॑वि मातर् मातः॒ पृथि॑वि सूक्त॒वाचः॑ सूक्त॒वाचः॒ पृथि॑वि मातः । \newline
2. सू॒क्त॒वाच॒ इति॑ सूक्त - वाचः॑ । \newline
3. पृथि॑वि मातर् मातः॒ पृथि॑वि॒ पृथि॑वि मात॒र् मा मा मा॑तः॒ पृथि॑वि॒ पृथि॑वि मात॒र् मा । \newline
4. मा॒त॒र् मा मा मा॑तर् मात॒र् मा मा॑ मा॒ मा मा॑तर् मात॒र् मा मा᳚ । \newline
5. मा मा॑ मा॒ मा मा मा॑ हिꣳसीर्. हिꣳसीर् मा॒ मा मा मा॑ हिꣳसीः । \newline
6. मा॒ हिꣳ॒॒सी॒र्॒. हिꣳ॒॒सी॒र् मा॒ मा॒ हिꣳ॒॒सी॒र् मधु॒ मधु॑ हिꣳसीर् मा मा हिꣳसी॒र् मधु॑ । \newline
7. हिꣳ॒॒सी॒र् मधु॒ मधु॑ हिꣳसीर्. हिꣳसी॒र् मधु॑ मनिष्ये मनिष्ये॒ मधु॑ हिꣳसीर्. हिꣳसी॒र् मधु॑ मनिष्ये । \newline
8. मधु॑ मनिष्ये मनिष्ये॒ मधु॒ मधु॑ मनिष्ये॒ मधु॒ मधु॑ मनिष्ये॒ मधु॒ मधु॑ मनिष्ये॒ मधु॑ । \newline
9. म॒नि॒ष्ये॒ मधु॒ मधु॑ मनिष्ये मनिष्ये॒ मधु॑ जनिष्ये जनिष्ये॒ मधु॑ मनिष्ये मनिष्ये॒ मधु॑ जनिष्ये । \newline
10. मधु॑ जनिष्ये जनिष्ये॒ मधु॒ मधु॑ जनिष्ये॒ मधु॒ मधु॑ जनिष्ये॒ मधु॒ मधु॑ जनिष्ये॒ मधु॑ । \newline
11. ज॒नि॒ष्ये॒ मधु॒ मधु॑ जनिष्ये जनिष्ये॒ मधु॑ वक्ष्यामि वक्ष्यामि॒ मधु॑ जनिष्ये जनिष्ये॒ मधु॑ वक्ष्यामि । \newline
12. मधु॑ वक्ष्यामि वक्ष्यामि॒ मधु॒ मधु॑ वक्ष्यामि॒ मधु॒ मधु॑ वक्ष्यामि॒ मधु॒ मधु॑ वक्ष्यामि॒ मधु॑ । \newline
13. व॒क्ष्या॒मि॒ मधु॒ मधु॑ वक्ष्यामि वक्ष्यामि॒ मधु॑ वदिष्यामि वदिष्यामि॒ मधु॑ वक्ष्यामि वक्ष्यामि॒ मधु॑ वदिष्यामि । \newline
14. मधु॑ वदिष्यामि वदिष्यामि॒ मधु॒ मधु॑ वदिष्यामि॒ मधु॑मती॒म् मधु॑मतीम् ॅवदिष्यामि॒ मधु॒ मधु॑ वदिष्यामि॒ मधु॑मतीम् । \newline
15. व॒दि॒ष्या॒मि॒ मधु॑मती॒म् मधु॑मतीम् ॅवदिष्यामि वदिष्यामि॒ मधु॑मतीम् दे॒वेभ्यो॑ दे॒वेभ्यो॒ मधु॑मतीम् ॅवदिष्यामि वदिष्यामि॒ मधु॑मतीम् दे॒वेभ्यः॑ । \newline
16. मधु॑मतीम् दे॒वेभ्यो॑ दे॒वेभ्यो॒ मधु॑मती॒म् मधु॑मतीम् दे॒वेभ्यो॒ वाच॒म् ॅवाच॑म् दे॒वेभ्यो॒ मधु॑मती॒म् मधु॑मतीम् दे॒वेभ्यो॒ वाच᳚म् । \newline
17. मधु॑मती॒मिति॒ मधु॑ - म॒ती॒म् । \newline
18. दे॒वेभ्यो॒ वाच॒म् ॅवाच॑म् दे॒वेभ्यो॑ दे॒वेभ्यो॒ वाच॑ मुद्यास मुद्यास॒म् ॅवाच॑म् दे॒वेभ्यो॑ दे॒वेभ्यो॒ वाच॑ मुद्यासम् । \newline
19. वाच॑ मुद्यास मुद्यास॒म् ॅवाच॒म् ॅवाच॑ मुद्यासꣳ शुश्रू॒षेण्याꣳ॑ शुश्रू॒षेण्या॑ मुद्यास॒म् ॅवाच॒म् ॅवाच॑ मुद्यासꣳ शुश्रू॒षेण्या᳚म् । \newline
20. उ॒द्या॒सꣳ॒॒ शु॒श्रू॒षेण्याꣳ॑ शुश्रू॒षेण्या॑ मुद्यास मुद्यासꣳ शुश्रू॒षेण्या᳚म् मनु॒ष्ये᳚भ्यो मनु॒ष्ये᳚भ्यः शुश्रू॒षेण्या॑ मुद्यास मुद्यासꣳ शुश्रू॒षेण्या᳚म् मनु॒ष्ये᳚भ्यः । \newline
21. शु॒श्रू॒षेण्या᳚म् मनु॒ष्ये᳚भ्यो मनु॒ष्ये᳚भ्यः शुश्रू॒षेण्याꣳ॑ शुश्रू॒षेण्या᳚म् मनु॒ष्ये᳚भ्य॒ स्तम् तम् म॑नु॒ष्ये᳚भ्यः शुश्रू॒षेण्याꣳ॑ शुश्रू॒षेण्या᳚म् मनु॒ष्ये᳚भ्य॒ स्तम् । \newline
22. म॒नु॒ष्ये᳚भ्य॒ स्तम् तम् म॑नु॒ष्ये᳚भ्यो मनु॒ष्ये᳚भ्य॒ स्तम् मा॑ मा॒ तम् म॑नु॒ष्ये᳚भ्यो मनु॒ष्ये᳚भ्य॒ स्तम् मा᳚ । \newline
23. तम् मा॑ मा॒ तम् तम् मा॑ दे॒वा दे॒वा मा॒ तम् तम् मा॑ दे॒वाः । \newline
24. मा॒ दे॒वा दे॒वा मा॑ मा दे॒वा अ॑व न्त्ववन्तु दे॒वा मा॑ मा दे॒वा अ॑वन्तु । \newline
25. दे॒वा अ॑व न्त्ववन्तु दे॒वा दे॒वा अ॑वन्तु शो॒भायै॑ शो॒भाया॑ अवन्तु दे॒वा दे॒वा अ॑वन्तु शो॒भायै᳚ । \newline
26. अ॒व॒न्तु॒ शो॒भायै॑ शो॒भाया॑ अव न्त्ववन्तु शो॒भायै॑ पि॒तरः॑ पि॒तरः॑ शो॒भाया॑ अव न्त्ववन्तु शो॒भायै॑ पि॒तरः॑ । \newline
27. शो॒भायै॑ पि॒तरः॑ पि॒तरः॑ शो॒भायै॑ शो॒भायै॑ पि॒तरो ऽन्वनु॑ पि॒तरः॑ शो॒भायै॑ शो॒भायै॑ पि॒तरो ऽनु॑ । \newline
28. पि॒तरो ऽन्वनु॑ पि॒तरः॑ पि॒तरो ऽनु॑ मदन्तु मद॒ न्त्वनु॑ पि॒तरः॑ पि॒तरो ऽनु॑ मदन्तु । \newline
29. अनु॑ मदन्तु मद॒ न्त्वन्वनु॑ मदन्तु । \newline
30. म॒द॒न्त्विति॑ मदन्तु । \newline
\pagebreak
\markright{ TS 3.3.3.1  \hfill https://www.vedavms.in \hfill}

\section{ TS 3.3.3.1 }

\textbf{TS 3.3.3.1 } \newline
\textbf{Samhita Paata} \newline

वस॑वस्त्वा॒ प्रव॑हन्तु गाय॒त्रेण॒ छन्द॑सा॒ऽग्नेः प्रि॒यं पाथ॒ उपे॑हि रु॒द्रास्त्वा॒ प्रवृ॑हन्तु॒ त्रैष्टु॑भेन॒ छन्द॒सेन्द्र॑स्य प्रि॒यं पाथ॒ उपे᳚ह्यादि॒त्यास्त्वा॒ प्रवृ॑हन्तु॒ जाग॑तेन॒ छन्द॑सा॒ विश्वे॑षां दे॒वानां᳚ प्रि॒यं पाथ॒ उपे॑हि॒ मान्दा॑सु ते शुक्र शु॒क्रमा धू॑नोमि भ॒न्दना॑सु॒ कोत॑नासु॒ नूत॑नासु॒ रेशी॑षु॒ मेषी॑षु॒ वाशी॑षु विश्व॒भृथ्सु॒ माद्ध्वी॑षु ककु॒हासु॒ शक्व॑रीषु - [  ] \newline

\textbf{Pada Paata} \newline

वस॑वः । त्वा॒ । प्रेति॑ । वृ॒ह॒न्तु॒ । गा॒य॒त्रेण॑ । छन्द॑सा । अ॒ग्नेः । प्रि॒यम् । पाथः॑ । उपेति॑ । इ॒हि॒ । रु॒द्राः । त्वा॒ । प्रेति॑ । वृ॒ह॒न्तु॒ । त्रैष्टु॑भेन । छन्द॑सा । इन्द्र॑स्य । प्रि॒यम् । पाथः॑ । उपेति॑ । इ॒हि॒ । आ॒दि॒त्याः । त्वा॒ । प्रेति॑ । वृ॒ह॒न्तु॒ । जाग॑तेन । छन्द॑सा । विश्वे॑षाम् । दे॒वाना᳚म् । प्रि॒यम् । पाथः॑ । उपेति॑ । इ॒हि॒ । मान्दा॑सु । ते॒ । शु॒क्र॒ । शु॒क्रम् । एति॑ । धू॒नो॒मि॒ । भ॒न्दना॑सु । कोत॑नासु । नूत॑नासु । रेशी॑षु । मेषी॑षु । वाशी॑षु । वि॒श्व॒भृथ्स्विति॑ विश्व॒भृत् - सु॒ । माद्ध्वी॑षु । क॒कु॒हासु॑ । शक्व॑रीषु ।  \newline


\textbf{Krama Paata} \newline

वस॑वस्त्वा । त्वा॒ प्र । प्र॑ वृहन्तु । वृ॒ह॒न्तु॒ गा॒य॒त्रेण॑ । गा॒य॒त्रेण॒ छन्द॑सा । छन्द॑सा॒ ऽग्नेः । अ॒ग्नेः प्रि॒यम् । प्रि॒यम् पाथः॑ । पाथ॒ उप॑ । उपे॑हि । इ॒हि॒ रु॒द्राः । रु॒द्रास्त्वा᳚ । त्वा॒ प्र । प्र वृ॑हन्तु । वृ॒ह॒न्तु॒ त्रैष्टु॑भेन । त्रैष्टु॑भेन॒ छन्द॑सा । छन्द॒सेन्द्र॑स्य । इन्द्र॑स्य प्रि॒यम् । प्रि॒यम् पाथः॑ । पाथ॒ उप॑ । उपे॑हि । इ॒ह्या॒दि॒त्याः । आ॒दि॒त्यास्त्वा᳚ । त्वा॒ प्र । प्र वृ॑हन्तु । वृ॒ह॒न्तु॒ जाग॑तेन । जाग॑तेन॒ छन्द॑सा । छन्द॑सा॒ विश्वे॑षाम् । विश्वे॑षाम् दे॒वाना᳚म् । दे॒वाना᳚म् प्रि॒यम् । प्रि॒यम् पाथः॑ । पाथ॒ उप॑ । उपे॑हि । इ॒हि॒ मान्दा॑सु । मान्दा॑सु ते । ते॒ शु॒क्र॒ । शु॒क्र॒ शु॒क्रम् । शु॒क्रमा । आ धू॑नोमि । धू॒नो॒मि॒ भ॒न्दना॑सु । भ॒न्दना॑सु॒ कोत॑नासु । कोत॑नासु॒ नूत॑नासु । नूत॑नासु॒ रेशी॑षु । रेशी॑षु॒ मेषी॑षु । मेषी॑षु॒ वाशी॑षु । वाशी॑षु विश्व॒भृथ्सु॑ । वि॒श्व॒भृथ्सु॒ माद्ध्वी॑षु । वि॒श्व॒भृथ्स्विति॑ विश्व॒भृत् - सु॒ । माद्ध्वी॑षु ककु॒हासु॑ । क॒कु॒हासु॒ शक्व॑रीषु । शक्व॑रीषु शु॒क्रासु॑ \newline

\textbf{Jatai Paata} \newline

1. वस॑व स्त्वा त्वा॒ वस॑वो॒ वस॑व स्त्वा । \newline
2. त्वा॒ प्र प्र त्वा᳚ त्वा॒ प्र । \newline
3. प्र वृ॑हन्तु वृहन्तु॒ प्र प्र वृ॑हन्तु । \newline
4. वृ॒ह॒न्तु॒ गा॒य॒त्रेण॑ गाय॒त्रेण॑ वृहन्तु वृहन्तु गाय॒त्रेण॑ । \newline
5. गा॒य॒त्रेण॒ छन्द॑सा॒ छन्द॑सा गाय॒त्रेण॑ गाय॒त्रेण॒ छन्द॑सा । \newline
6. छन्द॑सा॒ ऽग्ने र॒ग्ने श्छन्द॑सा॒ छन्द॑सा॒ ऽग्नेः । \newline
7. अ॒ग्नेः प्रि॒यम् प्रि॒य म॒ग्ने र॒ग्नेः प्रि॒यम् । \newline
8. प्रि॒यम् पाथः॒ पाथः॑ प्रि॒यम् प्रि॒यम् पाथः॑ । \newline
9. पाथ॒ उपोप॒ पाथः॒ पाथ॒ उप॑ । \newline
10. उपे॑ ही॒ह्युपोपे॑ हि । \newline
11. इ॒हि॒ रु॒द्रा रु॒द्रा इ॑हीहि रु॒द्राः । \newline
12. रु॒द्रा स्त्वा᳚ त्वा रु॒द्रा रु॒द्रा स्त्वा᳚ । \newline
13. त्वा॒ प्र प्र त्वा᳚ त्वा॒ प्र । \newline
14. प्र वृ॑हन्तु वृहन्तु॒ प्र प्र वृ॑हन्तु । \newline
15. वृ॒ह॒न्तु॒ त्रैष्टु॑भेन॒ त्रैष्टु॑भेन वृहन्तु वृहन्तु॒ त्रैष्टु॑भेन । \newline
16. त्रैष्टु॑भेन॒ छन्द॑सा॒ छन्द॑सा॒ त्रैष्टु॑भेन॒ त्रैष्टु॑भेन॒ छन्द॑सा । \newline
17. छन्द॒ सेन्द्र॒स्ये न्द्र॑स्य॒ छन्द॑सा॒ छन्द॒ सेन्द्र॑स्य । \newline
18. इन्द्र॑स्य प्रि॒यम् प्रि॒य मिन्द्र॒स्ये न्द्र॑स्य प्रि॒यम् । \newline
19. प्रि॒यम् पाथः॒ पाथः॑ प्रि॒यम् प्रि॒यम् पाथः॑ । \newline
20. पाथ॒ उपोप॒ पाथः॒ पाथ॒ उप॑ । \newline
21. उपे॑ ही॒ह्युपोपे॑ हि । \newline
22. इ॒ह्या॒दि॒त्या आ॑दि॒त्या इ॑ही ह्यादि॒त्याः । \newline
23. आ॒दि॒त्या स्त्वा᳚ त्वा ऽऽदि॒त्या आ॑दि॒त्या स्त्वा᳚ । \newline
24. त्वा॒ प्र प्र त्वा᳚ त्वा॒ प्र । \newline
25. प्र वृ॑हन्तु वृहन्तु॒ प्र प्र वृ॑हन्तु । \newline
26. वृ॒ह॒न्तु॒ जाग॑तेन॒ जाग॑तेन वृहन्तु वृहन्तु॒ जाग॑तेन । \newline
27. जाग॑तेन॒ छन्द॑सा॒ छन्द॑सा॒ जाग॑तेन॒ जाग॑तेन॒ छन्द॑सा । \newline
28. छन्द॑सा॒ विश्वे॑षा॒म् ॅविश्वे॑षा॒म् छन्द॑सा॒ छन्द॑सा॒ विश्वे॑षाम् । \newline
29. विश्वे॑षाम् दे॒वाना᳚म् दे॒वाना॒म् ॅविश्वे॑षा॒म् ॅविश्वे॑षाम् दे॒वाना᳚म् । \newline
30. दे॒वाना᳚म् प्रि॒यम् प्रि॒यम् दे॒वाना᳚म् दे॒वाना᳚म् प्रि॒यम् । \newline
31. प्रि॒यम् पाथः॒ पाथः॑ प्रि॒यम् प्रि॒यम् पाथः॑ । \newline
32. पाथ॒ उपोप॒ पाथः॒ पाथ॒ उप॑ । \newline
33. उपे॑ ही॒ह्युपोपे॑ हि । \newline
34. इ॒हि॒ मान्दा॑सु॒ मान्दा᳚ स्विहीहि॒ मान्दा॑सु । \newline
35. मान्दा॑सु ते ते॒ मान्दा॑सु॒ मान्दा॑सु ते । \newline
36. ते॒ शु॒क्र॒ शु॒क्र॒ ते॒ ते॒ शु॒क्र॒ । \newline
37. शु॒क्र॒ शु॒क्रꣳ शु॒क्रꣳ शु॑क्र शुक्र शु॒क्रम् । \newline
38. शु॒क्र मा शु॒क्रꣳ शु॒क्र मा । \newline
39. आ धू॑नोमि धूनो॒ म्या धू॑नोमि । \newline
40. धू॒नो॒मि॒ भ॒न्दना॑सु भ॒न्दना॑सु धूनोमि धूनोमि भ॒न्दना॑सु । \newline
41. भ॒न्दना॑सु॒ कोत॑नासु॒ कोत॑नासु भ॒न्दना॑सु भ॒न्दना॑सु॒ कोत॑नासु । \newline
42. कोत॑नासु॒ नूत॑नासु॒ नूत॑नासु॒ कोत॑नासु॒ कोत॑नासु॒ नूत॑नासु । \newline
43. नूत॑नासु॒ रेशी॑षु॒ रेशी॑षु॒ नूत॑नासु॒ नूत॑नासु॒ रेशी॑षु । \newline
44. रेशी॑षु॒ मेषी॑षु॒ मेषी॑षु॒ रेशी॑षु॒ रेशी॑षु॒ मेषी॑षु । \newline
45. मेषी॑षु॒ वाशी॑षु॒ वाशी॑षु॒ मेषी॑षु॒ मेषी॑षु॒ वाशी॑षु । \newline
46. वाशी॑षु विश्व॒भृथ्सु॑ विश्व॒भृथ्सु॒ वाशी॑षु॒ वाशी॑षु विश्व॒भृथ्सु॑ । \newline
47. वि॒श्व॒भृथ्सु॒ माद्ध्वी॑षु॒ माद्ध्वी॑षु विश्व॒भृथ्सु॑ विश्व॒भृथ्सु॒ माद्ध्वी॑षु । \newline
48. वि॒श्व॒भृथ्स्विति॑ विश्व॒भृत् - सु॒ । \newline
49. माद्ध्वी॑षु ककु॒हासु॑ ककु॒हासु॒ माद्ध्वी॑षु॒ माद्ध्वी॑षु ककु॒हासु॑ । \newline
50. क॒कु॒हासु॒ शक्व॑रीषु॒ शक्व॑रीषु ककु॒हासु॑ ककु॒हासु॒ शक्व॑रीषु । \newline
51. शक्व॑रीषु शु॒क्रासु॑ शु॒क्रासु॒ शक्व॑रीषु॒ शक्व॑रीषु शु॒क्रासु॑ । \newline

\textbf{Ghana Paata } \newline

1. वस॑व स्त्वा त्वा॒ वस॑वो॒ वस॑व स्त्वा॒ प्र प्र त्वा॒ वस॑वो॒ वस॑व स्त्वा॒ प्र । \newline
2. त्वा॒ प्र प्र त्वा᳚ त्वा॒ प्र वृ॑हन्तु वृहन्तु॒ प्र त्वा᳚ त्वा॒ प्र वृ॑हन्तु । \newline
3. प्र वृ॑हन्तु वृहन्तु॒ प्र प्र वृ॑हन्तु गाय॒त्रेण॑ गाय॒त्रेण॑ वृहन्तु॒ प्र प्र वृ॑हन्तु गाय॒त्रेण॑ । \newline
4. वृ॒ह॒न्तु॒ गा॒य॒त्रेण॑ गाय॒त्रेण॑ वृहन्तु वृहन्तु गाय॒त्रेण॒ छन्द॑सा॒ छन्द॑सा गाय॒त्रेण॑ वृहन्तु वृहन्तु गाय॒त्रेण॒ छन्द॑सा । \newline
5. गा॒य॒त्रेण॒ छन्द॑सा॒ छन्द॑सा गाय॒त्रेण॑ गाय॒त्रेण॒ छन्द॑सा॒ ऽग्ने र॒ग्ने श्छन्द॑सा गाय॒त्रेण॑ गाय॒त्रेण॒ छन्द॑सा॒ ऽग्नेः । \newline
6. छन्द॑सा॒ ऽग्ने र॒ग्ने श्छन्द॑सा॒ छन्द॑सा॒ ऽग्नेः प्रि॒यम् प्रि॒य म॒ग्ने श्छन्द॑सा॒ छन्द॑सा॒ ऽग्नेः प्रि॒यम् । \newline
7. अ॒ग्नेः प्रि॒यम् प्रि॒य म॒ग्ने र॒ग्नेः प्रि॒यम् पाथः॒ पाथः॑ प्रि॒य म॒ग्ने र॒ग्नेः प्रि॒यम् पाथः॑ । \newline
8. प्रि॒यम् पाथः॒ पाथः॑ प्रि॒यम् प्रि॒यम् पाथ॒ उपोप॒ पाथः॑ प्रि॒यम् प्रि॒यम् पाथ॒ उप॑ । \newline
9. पाथ॒ उपोप॒ पाथः॒ पाथ॒ उपे॑ ही॒ह्युप॒ पाथः॒ पाथ॒ उपे॑ हि । \newline
10. उपे॑ ही॒ह्युपोपे॑ हि रु॒द्रा रु॒द्रा इ॒ह्युपोपे॑ हि रु॒द्राः । \newline
11. इ॒हि॒ रु॒द्रा रु॒द्रा इ॑हीहि रु॒द्रा स्त्वा᳚ त्वा रु॒द्रा इ॑हीहि रु॒द्रा स्त्वा᳚ । \newline
12. रु॒द्रा स्त्वा᳚ त्वा रु॒द्रा रु॒द्रा स्त्वा॒ प्र प्र त्वा॑ रु॒द्रा रु॒द्रा स्त्वा॒ प्र । \newline
13. त्वा॒ प्र प्र त्वा᳚ त्वा॒ प्र वृ॑हन्तु वृहन्तु॒ प्र त्वा᳚ त्वा॒ प्र वृ॑हन्तु । \newline
14. प्र वृ॑हन्तु वृहन्तु॒ प्र प्र वृ॑हन्तु॒ त्रैष्टु॑भेन॒ त्रैष्टु॑भेन वृहन्तु॒ प्र प्र वृ॑हन्तु॒ त्रैष्टु॑भेन । \newline
15. वृ॒ह॒न्तु॒ त्रैष्टु॑भेन॒ त्रैष्टु॑भेन वृहन्तु वृहन्तु॒ त्रैष्टु॑भेन॒ छन्द॑सा॒ छन्द॑सा॒ त्रैष्टु॑भेन वृहन्तु वृहन्तु॒ त्रैष्टु॑भेन॒ छन्द॑सा । \newline
16. त्रैष्टु॑भेन॒ छन्द॑सा॒ छन्द॑सा॒ त्रैष्टु॑भेन॒ त्रैष्टु॑भेन॒ छन्द॒ सेन्द्र॒स्ये न्द्र॑स्य॒ छन्द॑सा॒ त्रैष्टु॑भेन॒ त्रैष्टु॑भेन॒ छन्द॒ सेन्द्र॑स्य । \newline
17. छन्द॒ सेन्द्र॒स्ये न्द्र॑स्य॒ छन्द॑सा॒ छन्द॒ सेन्द्र॑स्य प्रि॒यम् प्रि॒य मिन्द्र॑स्य॒ छन्द॑सा॒ छन्द॒ सेन्द्र॑स्य प्रि॒यम् । \newline
18. इन्द्र॑स्य प्रि॒यम् प्रि॒य मिन्द्र॒ स्येन्द्र॑स्य प्रि॒यम् पाथः॒ पाथः॑ प्रि॒य मिन्द्र॒ स्येन्द्र॑स्य प्रि॒यम् पाथः॑ । \newline
19. प्रि॒यम् पाथः॒ पाथः॑ प्रि॒यम् प्रि॒यम् पाथ॒ उपोप॒ पाथः॑ प्रि॒यम् प्रि॒यम् पाथ॒ उप॑ । \newline
20. पाथ॒ उपोप॒ पाथः॒ पाथ॒ उपे॑ ही॒ह्युप॒ पाथः॒ पाथ॒ उपे॑ हि । \newline
21. उपे॑ ही॒ह्युपोपे᳚ ह्यादि॒त्या आ॑दि॒त्या इ॒ह्युपोपे᳚ ह्यादि॒त्याः । \newline
22. इ॒ह्या॒दि॒त्या आ॑दि॒त्या इ॑हीह्या दि॒त्या स्त्वा᳚ त्वा ऽऽदि॒त्या इ॑हीह्या दि॒त्या स्त्वा᳚ । \newline
23. आ॒दि॒त्या स्त्वा᳚ त्वा ऽऽदि॒त्या आ॑दि॒त्या स्त्वा॒ प्र प्र त्वा॑ ऽऽदि॒त्या आ॑दि॒त्या स्त्वा॒ प्र । \newline
24. त्वा॒ प्र प्र त्वा᳚ त्वा॒ प्र वृ॑हन्तु वृहन्तु॒ प्र त्वा᳚ त्वा॒ प्र वृ॑हन्तु । \newline
25. प्र वृ॑हन्तु वृहन्तु॒ प्र प्र वृ॑हन्तु॒ जाग॑तेन॒ जाग॑तेन वृहन्तु॒ प्र प्र वृ॑हन्तु॒ जाग॑तेन । \newline
26. वृ॒ह॒न्तु॒ जाग॑तेन॒ जाग॑तेन वृहन्तु वृहन्तु॒ जाग॑तेन॒ छन्द॑सा॒ छन्द॑सा॒ जाग॑तेन वृहन्तु वृहन्तु॒ जाग॑तेन॒ छन्द॑सा । \newline
27. जाग॑तेन॒ छन्द॑सा॒ छन्द॑सा॒ जाग॑तेन॒ जाग॑तेन॒ छन्द॑सा॒ विश्वे॑षा॒म् ॅविश्वे॑षा॒म् छन्द॑सा॒ जाग॑तेन॒ जाग॑तेन॒ छन्द॑सा॒ विश्वे॑षाम् । \newline
28. छन्द॑सा॒ विश्वे॑षा॒म् ॅविश्वे॑षा॒म् छन्द॑सा॒ छन्द॑सा॒ विश्वे॑षाम् दे॒वाना᳚म् दे॒वाना॒म् ॅविश्वे॑षा॒म् छन्द॑सा॒ छन्द॑सा॒ विश्वे॑षाम् दे॒वाना᳚म् । \newline
29. विश्वे॑षाम् दे॒वाना᳚म् दे॒वाना॒म् ॅविश्वे॑षा॒म् ॅविश्वे॑षाम् दे॒वाना᳚म् प्रि॒यम् प्रि॒यम् दे॒वाना॒म् ॅविश्वे॑षा॒म् ॅविश्वे॑षाम् दे॒वाना᳚म् प्रि॒यम् । \newline
30. दे॒वाना᳚म् प्रि॒यम् प्रि॒यम् दे॒वाना᳚म् दे॒वाना᳚म् प्रि॒यम् पाथः॒ पाथः॑ प्रि॒यम् दे॒वाना᳚म् दे॒वाना᳚म् प्रि॒यम् पाथः॑ । \newline
31. प्रि॒यम् पाथः॒ पाथः॑ प्रि॒यम् प्रि॒यम् पाथ॒ उपोप॒ पाथः॑ प्रि॒यम् प्रि॒यम् पाथ॒ उप॑ । \newline
32. पाथ॒ उपोप॒ पाथः॒ पाथ॒ उपे॑ ही॒ह्युप॒ पाथः॒ पाथ॒ उपे॑ हि । \newline
33. उपे॑ ही॒ह्युपोपे॑ हि॒ मान्दा॑सु॒ मान्दा᳚ स्वि॒ह्युपोपे॑ हि॒ मान्दा॑सु । \newline
34. इ॒हि॒ मान्दा॑सु॒ मान्दा᳚ स्विहीहि॒ मान्दा॑सु ते ते॒ मान्दा᳚ स्विहीहि॒ मान्दा॑सु ते । \newline
35. मान्दा॑सु ते ते॒ मान्दा॑सु॒ मान्दा॑सु ते शुक्र शुक्र ते॒ मान्दा॑सु॒ मान्दा॑सु ते शुक्र । \newline
36. ते॒ शु॒क्र॒ शु॒क्र॒ ते॒ ते॒ शु॒क्र॒ शु॒क्रꣳ शु॒क्रꣳ शु॑क्र ते ते शुक्र शु॒क्रम् । \newline
37. शु॒क्र॒ शु॒क्रꣳ शु॒क्रꣳ शु॑क्र शुक्र शु॒क्र मा शु॒क्रꣳ शु॑क्र शुक्र शु॒क्र मा । \newline
38. शु॒क्र मा शु॒क्रꣳ शु॒क्र मा धू॑नोमि धूनो॒म्या शु॒क्रꣳ शु॒क्र मा धू॑नोमि । \newline
39. आ धू॑नोमि धूनो॒म्या धू॑नोमि भ॒न्दना॑सु भ॒न्दना॑सु धूनो॒म्या धू॑नोमि भ॒न्दना॑सु । \newline
40. धू॒नो॒मि॒ भ॒न्दना॑सु भ॒न्दना॑सु धूनोमि धूनोमि भ॒न्दना॑सु॒ कोत॑नासु॒ कोत॑नासु भ॒न्दना॑सु धूनोमि धूनोमि भ॒न्दना॑सु॒ कोत॑नासु । \newline
41. भ॒न्दना॑सु॒ कोत॑नासु॒ कोत॑नासु भ॒न्दना॑सु भ॒न्दना॑सु॒ कोत॑नासु॒ नूत॑नासु॒ नूत॑नासु॒ कोत॑नासु भ॒न्दना॑सु भ॒न्दना॑सु॒ कोत॑नासु॒ नूत॑नासु । \newline
42. कोत॑नासु॒ नूत॑नासु॒ नूत॑नासु॒ कोत॑नासु॒ कोत॑नासु॒ नूत॑नासु॒ रेशी॑षु॒ रेशी॑षु॒ नूत॑नासु॒ कोत॑नासु॒ कोत॑नासु॒ नूत॑नासु॒ रेशी॑षु । \newline
43. नूत॑नासु॒ रेशी॑षु॒ रेशी॑षु॒ नूत॑नासु॒ नूत॑नासु॒ रेशी॑षु॒ मेषी॑षु॒ मेषी॑षु॒ रेशी॑षु॒ नूत॑नासु॒ नूत॑नासु॒ रेशी॑षु॒ मेषी॑षु । \newline
44. रेशी॑षु॒ मेषी॑षु॒ मेषी॑षु॒ रेशी॑षु॒ रेशी॑षु॒ मेषी॑षु॒ वाशी॑षु॒ वाशी॑षु॒ मेषी॑षु॒ रेशी॑षु॒ रेशी॑षु॒ मेषी॑षु॒ वाशी॑षु । \newline
45. मेषी॑षु॒ वाशी॑षु॒ वाशी॑षु॒ मेषी॑षु॒ मेषी॑षु॒ वाशी॑षु विश्व॒भृथ्सु॑ विश्व॒भृथ्सु॒ वाशी॑षु॒ मेषी॑षु॒ मेषी॑षु॒ वाशी॑षु विश्व॒भृथ्सु॑ । \newline
46. वाशी॑षु विश्व॒भृथ्सु॑ विश्व॒भृथ्सु॒ वाशी॑षु॒ वाशी॑षु विश्व॒भृथ्सु॒ माद्ध्वी॑षु॒ माद्ध्वी॑षु विश्व॒भृथ्सु॒ वाशी॑षु॒ वाशी॑षु विश्व॒भृथ्सु॒ माद्ध्वी॑षु । \newline
47. वि॒श्व॒भृथ्सु॒ माद्ध्वी॑षु॒ माद्ध्वी॑षु विश्व॒भृथ्सु॑ विश्व॒भृथ्सु॒ माद्ध्वी॑षु ककु॒हासु॑ ककु॒हासु॒ माद्ध्वी॑षु विश्व॒भृथ्सु॑ विश्व॒भृथ्सु॒ माद्ध्वी॑षु ककु॒हासु॑ । \newline
48. वि॒श्व॒भृथ्स्विति॑ विश्व॒भृत् - सु॒ । \newline
49. माद्ध्वी॑षु ककु॒हासु॑ ककु॒हासु॒ माद्ध्वी॑षु॒ माद्ध्वी॑षु ककु॒हासु॒ शक्व॑रीषु॒ शक्व॑रीषु ककु॒हासु॒ माद्ध्वी॑षु॒ माद्ध्वी॑षु ककु॒हासु॒ शक्व॑रीषु । \newline
50. क॒कु॒हासु॒ शक्व॑रीषु॒ शक्व॑रीषु ककु॒हासु॑ ककु॒हासु॒ शक्व॑रीषु शु॒क्रासु॑ शु॒क्रासु॒ शक्व॑रीषु ककु॒हासु॑ ककु॒हासु॒ शक्व॑रीषु शु॒क्रासु॑ । \newline
51. शक्व॑रीषु शु॒क्रासु॑ शु॒क्रासु॒ शक्व॑रीषु॒ शक्व॑रीषु शु॒क्रासु॑ ते ते शु॒क्रासु॒ शक्व॑रीषु॒ शक्व॑रीषु शु॒क्रासु॑ ते । \newline
\pagebreak
\markright{ TS 3.3.3.2  \hfill https://www.vedavms.in \hfill}

\section{ TS 3.3.3.2 }

\textbf{TS 3.3.3.2 } \newline
\textbf{Samhita Paata} \newline

शु॒क्रासु॑ ते शुक्र शु॒क्रमा धू॑नोमि शु॒क्रं ते॑ शु॒क्रेण॑ गृह्णा॒म्यह्नो॑ रू॒पेण॒ सूर्य॑स्य र॒श्मिभिः॑ ॥ आऽस्मि॑न्नु॒ग्रा अ॑चुच्यवुर्दि॒वो धारा॑ असश्चत ॥ क॒कु॒हꣳ रू॒पं ॅवृ॑ष॒भस्य॑ रोचते बृ॒हथ् सोमः॒ सोम॑स्य पुरो॒गाः शु॒क्रः शु॒क्रस्य॑ पुरो॒गाः ॥ यत् ते॑ सो॒मादा᳚भ्यं॒ नाम॒ जागृ॑वि॒ तस्मै॑ ते सोम॒ सोमा॑य॒ स्वाहो॒शिक् त्वं दे॑व सोम गाय॒त्रेण॒ छन्द॑सा॒ऽग्नेः - [  ] \newline

\textbf{Pada Paata} \newline

शु॒क्रासु॑ । ते॒ । शु॒क्र॒ । शु॒क्रम् । एति॑ । धू॒नो॒मि॒ । शु॒क्रम् । ते॒ । शु॒क्रेण॑ । गृ॒ह्णा॒मि॒ । अह्नः॑ । रू॒पेण॑ । सूर्य॑स्य । र॒श्मिभि॒रिति॑ र॒श्मि-भिः॒ ॥ एति॑ । अ॒स्मि॒न्न् । उ॒ग्राः । अ॒चु॒च्य॒वुः॒ । दि॒वः । धाराः᳚ । अ॒स॒श्च॒त॒ ॥ क॒कु॒हम् । रू॒पम् । वृ॒ष॒भस्य॑ । रो॒च॒ते॒ । बृ॒हत् । सोमः॑ । सोम॑स्य । पु॒रो॒गा इति॑ पुरः - गाः । शु॒क्रः । शु॒क्रस्य॑ । पु॒रो॒गा इति॑ पुरः - गाः ॥ यत् । ते॒ । सो॒म॒ । अदा᳚भ्यम् । नाम॑ । जागृ॑वि । तस्मै᳚ । ते॒ । सो॒म॒ । सोमा॑य । स्वाहा᳚ । उ॒शिक् । त्वम् । दे॒व॒ । सो॒म॒ । गा॒य॒त्रेण॑ । छन्द॑सा । अ॒ग्नेः ।  \newline


\textbf{Krama Paata} \newline

शु॒क्रासु॑ ते । ते॒ शु॒क्र॒ । शु॒क्र॒ शु॒क्रम् । शु॒क्रमा । आ धू॑नोमि । धू॒नो॒मि॒ शु॒क्रम् । शु॒क्रम् ते᳚ । ते॒ शु॒क्रेण॑ । शु॒क्रेण॑ गृह्णामि । गृ॒ह्णा॒म्यह्नः॑ । अह्नो॑ रू॒पेण॑ । रू॒पेण॒ सूर्य॑स्य । सूर्य॑स्य र॒श्मिभिः॑ । र॒श्मिभि॒रिति॑ र॒श्मि - भिः॒ ॥ आ ऽस्मिन्न्॑ । अ॒स्मि॒न्नु॒ग्राः । उ॒ग्रा अ॑चुच्यवुः । अ॒चु॒च्य॒वु॒र् दि॒वः । दि॒वो धाराः᳚ । धारा॑ असश्चत । अ॒स॒श्च॒तेत्य॑सश्चत ॥ क॒कु॒हꣳ रू॒पम् । रू॒पं ॅवृ॑ष॒भस्य॑ । वृ॒ष॒भस्य॑ रोचते । रो॒च॒ते॒ बृ॒हत् । बृ॒हथ् सोमः॑ । सोमः॒ सोम॑स्य । सोम॑स्य पुरो॒गाः । पु॒रो॒गाः शु॒क्रः । पु॒रो॒गा इति॑ पुरः - गाः । शु॒क्रः शु॒क्रस्य॑ । शु॒क्रस्य॑ पुरो॒गाः । पु॒रो॒गा इति॑ पुरः - गाः ॥ यत् ते᳚ । ते॒ सो॒म॒ । सो॒मादा᳚भ्यम् । अदा᳚भ्य॒म् नाम॑ । नाम॒ जागृ॑वि । जागृ॑वि॒ तस्मै᳚ । तस्मै॑ ते । ते॒ सो॒म॒ । सो॒म॒ सोमा॑य । सोमा॑य॒ स्वाहा᳚ । स्वाहो॒शिक् । उ॒शिक् त्वम् । त्वम् दे॑व । दे॒व॒ सो॒म॒ । सो॒म॒ गा॒य॒त्रेण॑ । गा॒य॒त्रेण॒ छन्द॑सा । छन्द॑सा॒ ऽग्नेः । अ॒ग्नेः प्रि॒यम् \newline

\textbf{Jatai Paata} \newline

1. शु॒क्रासु॑ ते ते शु॒क्रासु॑ शु॒क्रासु॑ ते । \newline
2. ते॒ शु॒क्र॒ शु॒क्र॒ ते॒ ते॒ शु॒क्र॒ । \newline
3. शु॒क्र॒ शु॒क्रꣳ शु॒क्रꣳ शु॑क्र शुक्र शु॒क्रम् । \newline
4. शु॒क्र मा शु॒क्रꣳ शु॒क्र मा । \newline
5. आ धू॑नोमि धूनो॒ म्या धू॑नोमि । \newline
6. धू॒नो॒मि॒ शु॒क्रꣳ शु॒क्रम् धू॑नोमि धूनोमि शु॒क्रम् । \newline
7. शु॒क्रम् ते॑ ते शु॒क्रꣳ शु॒क्रम् ते᳚ । \newline
8. ते॒ शु॒क्रेण॑ शु॒क्रेण॑ ते ते शु॒क्रेण॑ । \newline
9. शु॒क्रेण॑ गृह्णामि गृह्णामि शु॒क्रेण॑ शु॒क्रेण॑ गृह्णामि । \newline
10. गृ॒ह्णा॒ म्यह्नो ऽह्नो॑ गृह्णामि गृह्णा॒ म्यह्नः॑ । \newline
11. अह्नो॑ रू॒पेण॑ रू॒पेणा ह्नो ऽह्नो॑ रू॒पेण॑ । \newline
12. रू॒पेण॒ सूर्य॑स्य॒ सूर्य॑स्य रू॒पेण॑ रू॒पेण॒ सूर्य॑स्य । \newline
13. सूर्य॑स्य र॒श्मिभी॑ र॒श्मिभिः॒ सूर्य॑स्य॒ सूर्य॑स्य र॒श्मिभिः॑ । \newline
14. र॒श्मिभि॒रिति॑ र॒श्मि - भिः॒ । \newline
15. आ ऽस्मि॑न् नस्मि॒न् ना ऽस्मिन्न्॑ । \newline
16. अ॒स्मि॒न् नु॒ग्रा उ॒ग्रा अ॑स्मिन् नस्मिन् नु॒ग्राः । \newline
17. उ॒ग्रा अ॑चुच्यवु रचुच्यवु रु॒ग्रा उ॒ग्रा अ॑चुच्यवुः । \newline
18. अ॒चु॒च्य॒वु॒र् दि॒वो दि॒वो॑ ऽचुच्यवु रचुच्यवुर् दि॒वः । \newline
19. दि॒वो धारा॒ धारा॑ दि॒वो दि॒वो धाराः᳚ । \newline
20. धारा॑ असश्चता सश्चत॒ धारा॒ धारा॑ असश्चत । \newline
21. अ॒स॒श्च॒तेत्य॑सश्चत । \newline
22. क॒कु॒हꣳ रू॒पꣳ रू॒पम् क॑कु॒हम् क॑कु॒हꣳ रू॒पम् । \newline
23. रू॒पम् ॅवृ॑ष॒भस्य॑ वृष॒भस्य॑ रू॒पꣳ रू॒पम् ॅवृ॑ष॒भस्य॑ । \newline
24. वृ॒ष॒भस्य॑ रोचते रोचते वृष॒भस्य॑ वृष॒भस्य॑ रोचते । \newline
25. रो॒च॒ते॒ बृ॒हद् बृ॒हद् रो॑चते रोचते बृ॒हत् । \newline
26. बृ॒हथ् सोमः॒ सोमो॑ बृ॒हद् बृ॒हथ् सोमः॑ । \newline
27. सोमः॒ सोम॑स्य॒ सोम॑स्य॒ सोमः॒ सोमः॒ सोम॑स्य । \newline
28. सोम॑स्य पुरो॒गाः पु॑रो॒गाः सोम॑स्य॒ सोम॑स्य पुरो॒गाः । \newline
29. पु॒रो॒गाः शु॒क्रः शु॒क्रः पु॑रो॒गाः पु॑रो॒गाः शु॒क्रः । \newline
30. पु॒रो॒गा इति॑ पुरः - गाः । \newline
31. शु॒क्रः शु॒क्रस्य॑ शु॒क्रस्य॑ शु॒क्रः शु॒क्रः शु॒क्रस्य॑ । \newline
32. शु॒क्रस्य॑ पुरो॒गाः पु॑रो॒गाः शु॒क्रस्य॑ शु॒क्रस्य॑ पुरो॒गाः । \newline
33. पु॒रो॒गा इति॑ पुरः - गाः । \newline
34. यत् ते॑ ते॒ यद् यत् ते᳚ । \newline
35. ते॒ सो॒म॒ सो॒म॒ ते॒ ते॒ सो॒म॒ । \newline
36. सो॒मा दा᳚भ्य॒ मदा᳚भ्यꣳ सोम सो॒मा दा᳚भ्यम् । \newline
37. अदा᳚भ्य॒म् नाम॒ नामादा᳚भ्य॒ मदा᳚भ्य॒म् नाम॑ । \newline
38. नाम॒ जागृ॑वि॒ जागृ॑वि॒ नाम॒ नाम॒ जागृ॑वि । \newline
39. जागृ॑वि॒ तस्मै॒ तस्मै॒ जागृ॑वि॒ जागृ॑वि॒ तस्मै᳚ । \newline
40. तस्मै॑ ते ते॒ तस्मै॒ तस्मै॑ ते । \newline
41. ते॒ सो॒म॒ सो॒म॒ ते॒ ते॒ सो॒म॒ । \newline
42. सो॒म॒ सोमा॑य॒ सोमा॑य सोम सोम॒ सोमा॑य । \newline
43. सोमा॑य॒ स्वाहा॒ स्वाहा॒ सोमा॑य॒ सोमा॑य॒ स्वाहा᳚ । \newline
44. स्वा हो॒शि गु॒शिख् स्वाहा॒ स्वा हो॒शिक् । \newline
45. उ॒शिक् त्वम् त्व मु॒शि गु॒शिक् त्वम् । \newline
46. त्वम् दे॑व देव॒ त्वम् त्वम् दे॑व । \newline
47. दे॒व॒ सो॒म॒ सो॒म॒ दे॒व॒ दे॒व॒ सो॒म॒ । \newline
48. सो॒म॒ गा॒य॒त्रेण॑ गाय॒त्रेण॑ सोम सोम गाय॒त्रेण॑ । \newline
49. गा॒य॒त्रेण॒ छन्द॑सा॒ छन्द॑सा गाय॒त्रेण॑ गाय॒त्रेण॒ छन्द॑सा । \newline
50. छन्द॑सा॒ ऽग्ने र॒ग्ने श्छन्द॑सा॒ छन्द॑सा॒ ऽग्नेः । \newline
51. अ॒ग्नेः प्रि॒यम् प्रि॒य म॒ग्ने र॒ग्नेः प्रि॒यम् । \newline

\textbf{Ghana Paata } \newline

1. शु॒क्रासु॑ ते ते शु॒क्रासु॑ शु॒क्रासु॑ ते शुक्र शुक्र ते शु॒क्रासु॑ शु॒क्रासु॑ ते शुक्र । \newline
2. ते॒ शु॒क्र॒ शु॒क्र॒ ते॒ ते॒ शु॒क्र॒ शु॒क्रꣳ शु॒क्रꣳ शु॑क्र ते ते शुक्र शु॒क्रम् । \newline
3. शु॒क्र॒ शु॒क्रꣳ शु॒क्रꣳ शु॑क्र शुक्र शु॒क्र मा शु॒क्रꣳ शु॑क्र शुक्र शु॒क्र मा । \newline
4. शु॒क्र मा शु॒क्रꣳ शु॒क्र मा धू॑नोमि धूनो॒म्या शु॒क्रꣳ शु॒क्र मा धू॑नोमि । \newline
5. आ धू॑नोमि धूनो॒म्या धू॑नोमि शु॒क्रꣳ शु॒क्रम् धू॑नो॒म्या धू॑नोमि शु॒क्रम् । \newline
6. धू॒नो॒मि॒ शु॒क्रꣳ शु॒क्रम् धू॑नोमि धूनोमि शु॒क्रम् ते॑ ते शु॒क्रम् धू॑नोमि धूनोमि शु॒क्रम् ते᳚ । \newline
7. शु॒क्रम् ते॑ ते शु॒क्रꣳ शु॒क्रम् ते॑ शु॒क्रेण॑ शु॒क्रेण॑ ते शु॒क्रꣳ शु॒क्रम् ते॑ शु॒क्रेण॑ । \newline
8. ते॒ शु॒क्रेण॑ शु॒क्रेण॑ ते ते शु॒क्रेण॑ गृह्णामि गृह्णामि शु॒क्रेण॑ ते ते शु॒क्रेण॑ गृह्णामि । \newline
9. शु॒क्रेण॑ गृह्णामि गृह्णामि शु॒क्रेण॑ शु॒क्रेण॑ गृह्णा॒ म्यह्नो ऽह्नो॑ गृह्णामि शु॒क्रेण॑ शु॒क्रेण॑ गृह्णा॒ म्यह्नः॑ । \newline
10. गृ॒ह्णा॒ म्यह्नो ऽह्नो॑ गृह्णामि गृह्णा॒ म्यह्नो॑ रू॒पेण॑ रू॒पेणाह्नो॑ गृह्णामि गृह्णा॒ म्यह्नो॑ रू॒पेण॑ । \newline
11. अह्नो॑ रू॒पेण॑ रू॒पेणाह्नो ऽह्नो॑ रू॒पेण॒ सूर्य॑स्य॒ सूर्य॑स्य रू॒पेणाह्नो ऽह्नो॑ रू॒पेण॒ सूर्य॑स्य । \newline
12. रू॒पेण॒ सूर्य॑स्य॒ सूर्य॑स्य रू॒पेण॑ रू॒पेण॒ सूर्य॑स्य र॒श्मिभी॑ र॒श्मिभिः॒ सूर्य॑स्य रू॒पेण॑ रू॒पेण॒ सूर्य॑स्य र॒श्मिभिः॑ । \newline
13. सूर्य॑स्य र॒श्मिभी॑ र॒श्मिभिः॒ सूर्य॑स्य॒ सूर्य॑स्य र॒श्मिभिः॑ । \newline
14. र॒श्मिभि॒रिति॑ र॒श्मि - भिः॒ । \newline
15. आ ऽस्मि॑न्, नस्मि॒न्, ना ऽस्मि॑न्, नु॒ग्रा उ॒ग्रा अ॑स्मि॒न्, ना ऽस्मि॑न्, नु॒ग्राः । \newline
16. अ॒स्मि॒न्, नु॒ग्रा उ॒ग्रा अ॑स्मिन्, नस्मिन्, नु॒ग्रा अ॑चुच्यवु रचुच्यवु रु॒ग्रा अ॑स्मिन्, नस्मिन्, नु॒ग्रा अ॑चुच्यवुः । \newline
17. उ॒ग्रा अ॑चुच्यवु रचुच्यवु रु॒ग्रा उ॒ग्रा अ॑चुच्यवुर् दि॒वो दि॒वो॑ ऽचुच्यवु रु॒ग्रा उ॒ग्रा अ॑चुच्यवुर् दि॒वः । \newline
18. अ॒चु॒च्य॒वु॒र् दि॒वो दि॒वो॑ ऽचुच्यवु रचुच्यवुर् दि॒वो धारा॒ धारा॑ दि॒वो॑ ऽचुच्यवु रचुच्यवुर् दि॒वो धाराः᳚ । \newline
19. दि॒वो धारा॒ धारा॑ दि॒वो दि॒वो धारा॑ असश्चता सश्चत॒ धारा॑ दि॒वो दि॒वो धारा॑ असश्चत । \newline
20. धारा॑ असश्चता सश्चत॒ धारा॒ धारा॑ असश्चत । \newline
21. अ॒स॒श्च॒तेत्य॑सश्चत । \newline
22. क॒कु॒हꣳ रू॒पꣳ रू॒पम् क॑कु॒हम् क॑कु॒हꣳ रू॒पम् ॅवृ॑ष॒भस्य॑ वृष॒भस्य॑ रू॒पम् क॑कु॒हम् क॑कु॒हꣳ रू॒पम् ॅवृ॑ष॒भस्य॑ । \newline
23. रू॒पम् ॅवृ॑ष॒भस्य॑ वृष॒भस्य॑ रू॒पꣳ रू॒पम् ॅवृ॑ष॒भस्य॑ रोचते रोचते वृष॒भस्य॑ रू॒पꣳ रू॒पम् ॅवृ॑ष॒भस्य॑ रोचते । \newline
24. वृ॒ष॒भस्य॑ रोचते रोचते वृष॒भस्य॑ वृष॒भस्य॑ रोचते बृ॒हद् बृ॒हद् रो॑चते वृष॒भस्य॑ वृष॒भस्य॑ रोचते बृ॒हत् । \newline
25. रो॒च॒ते॒ बृ॒हद् बृ॒हद् रो॑चते रोचते बृ॒हथ् सोमः॒ सोमो॑ बृ॒हद् रो॑चते रोचते बृ॒हथ् सोमः॑ । \newline
26. बृ॒हथ् सोमः॒ सोमो॑ बृ॒हद् बृ॒हथ् सोमः॒ सोम॑स्य॒ सोम॑स्य॒ सोमो॑ बृ॒हद् बृ॒हथ् सोमः॒ सोम॑स्य । \newline
27. सोमः॒ सोम॑स्य॒ सोम॑स्य॒ सोमः॒ सोमः॒ सोम॑स्य पुरो॒गाः पु॑रो॒गाः सोम॑स्य॒ सोमः॒ सोमः॒ सोम॑स्य पुरो॒गाः । \newline
28. सोम॑स्य पुरो॒गाः पु॑रो॒गाः सोम॑स्य॒ सोम॑स्य पुरो॒गाः शु॒क्रः शु॒क्रः पु॑रो॒गाः सोम॑स्य॒ सोम॑स्य पुरो॒गाः शु॒क्रः । \newline
29. पु॒रो॒गाः शु॒क्रः शु॒क्रः पु॑रो॒गाः पु॑रो॒गाः शु॒क्रः शु॒क्रस्य॑ शु॒क्रस्य॑ शु॒क्रः पु॑रो॒गाः पु॑रो॒गाः शु॒क्रः शु॒क्रस्य॑ । \newline
30. पु॒रो॒गा इति॑ पुरः - गाः । \newline
31. शु॒क्रः शु॒क्रस्य॑ शु॒क्रस्य॑ शु॒क्रः शु॒क्रः शु॒क्रस्य॑ पुरो॒गाः पु॑रो॒गाः शु॒क्रस्य॑ शु॒क्रः शु॒क्रः शु॒क्रस्य॑ पुरो॒गाः । \newline
32. शु॒क्रस्य॑ पुरो॒गाः पु॑रो॒गाः शु॒क्रस्य॑ शु॒क्रस्य॑ पुरो॒गाः । \newline
33. पु॒रो॒गा इति॑ पुरः - गाः । \newline
34. यत् ते॑ ते॒ यद् यत् ते॑ सोम सोम ते॒ यद् यत् ते॑ सोम । \newline
35. ते॒ सो॒म॒ सो॒म॒ ते॒ ते॒ सो॒मादा᳚भ्य॒ मदा᳚भ्यꣳ सोम ते ते सो॒मादा᳚भ्यम् । \newline
36. सो॒मादा᳚भ्य॒ मदा᳚भ्यꣳ सोम सो॒मादा᳚भ्य॒म् नाम॒ नामादा᳚भ्यꣳ सोम सो॒मादा᳚भ्य॒म् नाम॑ । \newline
37. अदा᳚भ्य॒म् नाम॒ नामादा᳚भ्य॒ मदा᳚भ्य॒म् नाम॒ जागृ॑वि॒ जागृ॑वि॒ नामादा᳚भ्य॒ मदा᳚भ्य॒म् नाम॒ जागृ॑वि । \newline
38. नाम॒ जागृ॑वि॒ जागृ॑वि॒ नाम॒ नाम॒ जागृ॑वि॒ तस्मै॒ तस्मै॒ जागृ॑वि॒ नाम॒ नाम॒ जागृ॑वि॒ तस्मै᳚ । \newline
39. जागृ॑वि॒ तस्मै॒ तस्मै॒ जागृ॑वि॒ जागृ॑वि॒ तस्मै॑ ते ते॒ तस्मै॒ जागृ॑वि॒ जागृ॑वि॒ तस्मै॑ ते । \newline
40. तस्मै॑ ते ते॒ तस्मै॒ तस्मै॑ ते सोम सोम ते॒ तस्मै॒ तस्मै॑ ते सोम । \newline
41. ते॒ सो॒म॒ सो॒म॒ ते॒ ते॒ सो॒म॒ सोमा॑य॒ सोमा॑य सोम ते ते सोम॒ सोमा॑य । \newline
42. सो॒म॒ सोमा॑य॒ सोमा॑य सोम सोम॒ सोमा॑य॒ स्वाहा॒ स्वाहा॒ सोमा॑य सोम सोम॒ सोमा॑य॒ स्वाहा᳚ । \newline
43. सोमा॑य॒ स्वाहा॒ स्वाहा॒ सोमा॑य॒ सोमा॑य॒ स्वाहो॒शि गु॒शिख् स्वाहा॒ सोमा॑य॒ सोमा॑य॒ स्वाहो॒शिक् । \newline
44. स्वाहो॒ शिगु॒शिख् स्वाहा॒ स्वाहो॒शिक् त्वम् त्व मु॒शिख् स्वाहा॒ स्वाहो॒शिक् त्वम् । \newline
45. उ॒शिक् त्वम् त्व मु॒शि गु॒शिक् त्वम् दे॑व देव॒ त्व मु॒शि गु॒शिक् त्वम् दे॑व । \newline
46. त्वम् दे॑व देव॒ त्वम् त्वम् दे॑व सोम सोम देव॒ त्वम् त्वम् दे॑व सोम । \newline
47. दे॒व॒ सो॒म॒ सो॒म॒ दे॒व॒ दे॒व॒ सो॒म॒ गा॒य॒त्रेण॑ गाय॒त्रेण॑ सोम देव देव सोम गाय॒त्रेण॑ । \newline
48. सो॒म॒ गा॒य॒त्रेण॑ गाय॒त्रेण॑ सोम सोम गाय॒त्रेण॒ छन्द॑सा॒ छन्द॑सा गाय॒त्रेण॑ सोम सोम गाय॒त्रेण॒ छन्द॑सा । \newline
49. गा॒य॒त्रेण॒ छन्द॑सा॒ छन्द॑सा गाय॒त्रेण॑ गाय॒त्रेण॒ छन्द॑सा॒ ऽग्ने र॒ग्ने श्छन्द॑सा गाय॒त्रेण॑ गाय॒त्रेण॒ छन्द॑सा॒ ऽग्नेः । \newline
50. छन्द॑सा॒ ऽग्ने र॒ग्ने श्छन्द॑सा॒ छन्द॑सा॒ ऽग्नेः प्रि॒यम् प्रि॒य म॒ग्ने श्छन्द॑सा॒ छन्द॑सा॒ ऽग्नेः प्रि॒यम् । \newline
51. अ॒ग्नेः प्रि॒यम् प्रि॒य म॒ग्ने र॒ग्नेः प्रि॒यम् पाथः॒ पाथः॑ प्रि॒य म॒ग्ने र॒ग्नेः प्रि॒यम् पाथः॑ । \newline
\pagebreak
\markright{ TS 3.3.3.3  \hfill https://www.vedavms.in \hfill}

\section{ TS 3.3.3.3 }

\textbf{TS 3.3.3.3 } \newline
\textbf{Samhita Paata} \newline

प्रि॒यं पाथो॒ अपी॑हि व॒शी त्वं दे॑व सोम॒ त्रैष्टु॑भेन॒ छन्द॒सेन्द्र॑स्य प्रि॒यं पाथो॒ अपी᳚ह्य॒स्मथ्स॑खा॒ त्वं दे॑व सोम॒ जाग॑तेन॒ छन्द॑सा॒ विश्वे॑षां दे॒वानां᳚ प्रि॒यं पाथो॒ अपी॒ह्या नः॑ प्रा॒ण ए॑तु परा॒वत॒ आऽन्तरि॑क्षाद्दि॒वस्परि॑ । आयुः॑ पृथि॒व्या अद्ध्य॒मृत॑मसि प्रा॒णाय॑ त्वा ॥ इ॒न्द्रा॒ग्नी मे॒ वर्चः॑ कृणुतां॒ ॅवर्चः॒ सोमो॒ बृह॒स्पतिः॑ ( ) । वर्चो॑ मे॒ विश्वे॑दे॒वा वर्चो॑ मे धत्तमश्विना ॥ द॒ध॒न्वे वा॒ यदी॒मनु॒ वोच॒द्ब्रह्मा॑णि॒ वेरु॒ तत् । परि॒ विश्वा॑नि॒ काव्या॑ ने॒मिश्च॒क्रमि॑वा भवत् ॥ \newline

\textbf{Pada Paata} \newline

प्रि॒यम् । पाथः॑ । अपीति॑ । इ॒हि॒ । व॒शी । त्वम् । दे॒व॒ । सो॒म॒ । त्रैष्टु॑भेन । छन्द॑सा । इन्द्र॑स्य । प्रि॒यम् । पाथः॑ । अपीति॑ । इ॒हि॒ । अ॒स्मथ्स॒खेत्य॒स्मत् - स॒खा॒ । त्वम् । दे॒व॒ । सो॒म॒ । जाग॑तेन । छन्द॑सा । विश्वे॑षाम् । दे॒वाना᳚म् । प्रि॒यम् । पाथः॑ । अपीति॑ । इ॒हि॒ । एति॑ । नः॒ । प्रा॒ण इति॑ प्र - अ॒नः । ए॒तु॒ । प॒रा॒वत॒ इति॑ परा-वतः॑ । एति॑ । अ॒न्तरि॑क्षात् । दि॒वः । परि॑ ॥ आयुः॑ । पृ॒थि॒व्याः । अधीति॑ । अ॒मृत᳚म् । अ॒सि॒ । प्रा॒णायेति॑ प्र - अ॒नाय॑ । त्वा॒ ॥ इ॒न्द्रा॒ग्नी इती᳚न्द्र-अ॒ग्नी । मे॒ । वर्चः॑ । कृ॒णु॒ता॒म् । वर्चः॑ । सोमः॑ । बृह॒स्पतिः॑ ( ) ॥ वर्चः॑ । मे॒ । विश्वे᳚ । दे॒वाः । वर्चः॑ । मे॒ । ध॒त्त॒म् । अ॒श्वि॒ना॒ ॥ द॒ध॒न्वे । वा॒ । यत् । ई॒म् । अन्विति॑ । वोच॑त् । ब्रह्मा॑णि । वेः । उ॒ । तत् ॥ परीति॑ । विश्वा॑नि । काव्या᳚ । ने॒मिः । च॒क्रम् । इ॒व॒ । अ॒भ॒व॒त् ॥  \newline


\textbf{Krama Paata} \newline

प्रि॒यम् पाथः॑ । पाथो॒ अपि॑ । अपी॑हि । इ॒हि॒ व॒शी । व॒शी त्वम् । त्वम् दे॑व । दे॒व॒ सो॒म॒ । सो॒म॒ त्रैष्टु॑भेन । त्रैष्ट॑भेन॒ छन्द॑सा । छन्द॒सेन्द्र॑स्य । इन्द्र॑स्य प्रि॒यम् । प्रि॒यम् पाथः॑ । पाथो॒ अपि॑ । अपी॑हि । इ॒ह्य॒स्मथ्स॑खा । अ॒स्मथ्स॑खा॒ त्वम् । अ॒स्मथ्स॒खेत्य॒स्मत् - स॒खा॒ । त्वम् दे॑व । दे॒व॒ सो॒म॒ । सो॒म॒ जाग॑तेन । जाग॑तेन॒ छन्द॑सा । छन्द॑सा॒ विश्वे॑षाम् । विश्वे॑षाम् दे॒वाना᳚म् । दे॒वाना᳚म् प्रि॒यम् । प्रि॒यम् पाथः॑ । पाथो॒ अपि॑ । अपी॑हि । इ॒ह्या । आ नः॑ । नः॒ प्रा॒णः । प्रा॒ण ए॑तु । प्रा॒ण इति॑ प्र - अ॒नः । ए॒तु॒ प॒रा॒वतः॑ । प॒रा॒वत॒ आ । प॒रा॒वत॒ इति॑ परा - वतः॑ । आ ऽन्तरि॑क्षात् । अ॒न्तरि॑क्षाद् दि॒वः । दि॒वस्परि॑ । परीति॒ परि॑ ॥ आयुः॑ पृथि॒व्याः । पृ॒थि॒व्या अधि॑ । अद्ध्य॒मृत᳚म् । अ॒मृत॑मसि । अ॒सि॒ प्रा॒णाय॑ । प्रा॒णाय॑ त्वा । प्रा॒णायेति॑ प्र - अ॒नाय॑ । त्वेति॑ त्वा ॥ इ॒न्द्रा॒ग्नी मे᳚ । इ॒न्द्रा॒ग्नी इती᳚न्द्र - अ॒ग्नी । मे॒ वर्चः॑ । वर्चः॑ कृणुताम् । कृ॒णु॒तां॒ ॅवर्चः॑ । वर्चः॒ सोमः॑ । सोमो॒ बृह॒स्पतिः॑ । बृह॒स्पति॒रिति॒ बृह॒स्पतिः॑ ( ) ॥ वर्चो॑ मे । मे॒ विश्वे᳚ । विश्वे॑ दे॒वाः । दे॒वा वर्चः॑ । वर्चो॑ मे । मे॒ ध॒त्त॒म् । ध॒त्त॒म॒श्वि॒ना॒ । अ॒श्वि॒नेत्य॑श्विना ॥ द॒ध॒न्वे वा᳚ । वा॒ यत् । यदी᳚म् । ई॒मनु॑ । अनु॒ वोच॑त् । वोच॒द् ब्रह्मा॑णि । ब्रह्मा॑णि॒ वेः । वेरु॑ । उ॒ तत् । तदिति॒ तत् ॥ परि॒ विश्वा॑नि । विश्वा॑नि॒ काव्या᳚ । काव्या॑ ने॒मिः । ने॒मिश्च॒क्रम् । च॒क्रमि॑व । इ॒वा॒भ॒व॒त्॒ । अ॒भ॒व॒दित्य॑भवत् । \newline

\textbf{Jatai Paata} \newline

1. प्रि॒यम् पाथः॒ पाथः॑ प्रि॒यम् प्रि॒यम् पाथः॑ । \newline
2. पाथो॒ अप्यपि॒ पाथः॒ पाथो॒ अपि॑ । \newline
3. अपी॑ही॒ ह्यप्यपी॑हि । \newline
4. इ॒हि॒ व॒शी व॒शीही॑हि व॒शी । \newline
5. व॒शी त्वम् त्वम् ॅव॒शी व॒शी त्वम् । \newline
6. त्वम् दे॑व देव॒ त्वम् त्वम् दे॑व । \newline
7. दे॒व॒ सो॒म॒ सो॒म॒ दे॒व॒ दे॒व॒ सो॒म॒ । \newline
8. सो॒म॒ त्रैष्टु॑भेन॒ त्रैष्टु॑भेन सोम सोम॒ त्रैष्टु॑भेन । \newline
9. त्रैष्टु॑भेन॒ छन्द॑सा॒ छन्द॑सा॒ त्रैष्टु॑भेन॒ त्रैष्टु॑भेन॒ छन्द॑सा । \newline
10. छन्द॒ सेन्द्र॒स्ये न्द्र॑स्य॒ छन्द॑सा॒ छन्द॒ सेन्द्र॑स्य । \newline
11. इन्द्र॑स्य प्रि॒यम् प्रि॒य मिन्द्र॒स्ये न्द्र॑स्य प्रि॒यम् । \newline
12. प्रि॒यम् पाथः॒ पाथः॑ प्रि॒यम् प्रि॒यम् पाथः॑ । \newline
13. पाथो॒ अप्यपि॒ पाथः॒ पाथो॒ अपि॑ । \newline
14. अपी॑ही॒ ह्यप्यपी॑हि । \newline
15. इ॒ह्य॒स्मथ्स॑खा॒ ऽस्मथ्स॑खे ही ह्य॒स्मथ्स॑खा । \newline
16. अ॒स्मथ्स॑खा॒ त्वम् त्व म॒स्मथ्स॑खा॒ ऽस्मथ्स॑खा॒ त्वम् । \newline
17. अ॒स्मथ्स॒खेत्य॒स्मत् - स॒खा॒ । \newline
18. त्वम् दे॑व देव॒ त्वम् त्वम् दे॑व । \newline
19. दे॒व॒ सो॒म॒ सो॒म॒ दे॒व॒ दे॒व॒ सो॒म॒ । \newline
20. सो॒म॒ जाग॑तेन॒ जाग॑तेन सोम सोम॒ जाग॑तेन । \newline
21. जाग॑तेन॒ छन्द॑सा॒ छन्द॑सा॒ जाग॑तेन॒ जाग॑तेन॒ छन्द॑सा । \newline
22. छन्द॑सा॒ विश्वे॑षा॒म् ॅविश्वे॑षा॒म् छन्द॑सा॒ छन्द॑सा॒ विश्वे॑षाम् । \newline
23. विश्वे॑षाम् दे॒वाना᳚म् दे॒वाना॒म् ॅविश्वे॑षा॒म् ॅविश्वे॑षाम् दे॒वाना᳚म् । \newline
24. दे॒वाना᳚म् प्रि॒यम् प्रि॒यम् दे॒वाना᳚म् दे॒वाना᳚म् प्रि॒यम् । \newline
25. प्रि॒यम् पाथः॒ पाथः॑ प्रि॒यम् प्रि॒यम् पाथः॑ । \newline
26. पाथो॒ अप्यपि॒ पाथः॒ पाथो॒ अपि॑ । \newline
27. अपी॑ही॒ ह्यप्यपी॑हि । \newline
28. इ॒ह्येही॒ह्या । \newline
29. आ नो॑ न॒ आ नः॑ । \newline
30. नः॒ प्रा॒णः प्रा॒णो नो॑ नः प्रा॒णः । \newline
31. प्रा॒ण ए᳚त्वेतु प्रा॒णः प्रा॒ण ए॑तु । \newline
32. प्रा॒ण इति॑ प्र - अ॒नः । \newline
33. ए॒तु॒ प॒रा॒वतः॑ परा॒वत॑ एत्वेतु परा॒वतः॑ । \newline
34. प॒रा॒वत॒ आ प॑रा॒वतः॑ परा॒वत॒ आ । \newline
35. प॒रा॒वत॒ इति॑ परा - वतः॑ । \newline
36. आ ऽन्तरि॑क्षा द॒न्तरि॑क्षा॒ दा ऽन्तरि॑क्षात् । \newline
37. अ॒न्तरि॑क्षाद् दि॒वो दि॒वो᳚ ऽन्तरि॑क्षा द॒न्तरि॑क्षाद् दि॒वः । \newline
38. दि॒व स्परि॒ परि॑ दि॒वो दि॒व स्परि॑ । \newline
39. परीति॒ परि॑ । \newline
40. आयुः॑ पृथि॒व्याः पृ॑थि॒व्या आयु॒रायुः॑ पृथि॒व्याः । \newline
41. पृ॒थि॒व्या अध्यधि॑ पृथि॒व्याः पृ॑थि॒व्या अधि॑ । \newline
42. अध्य॒मृत॑ म॒मृत॒ मध्य ध्य॒मृत᳚म् । \newline
43. अ॒मृत॑ मस्य स्य॒मृत॑ म॒मृत॑ मसि । \newline
44. अ॒सि॒ प्रा॒णाय॑ प्रा॒णाया᳚ स्यसि प्रा॒णाय॑ । \newline
45. प्रा॒णाय॑ त्वा त्वा प्रा॒णाय॑ प्रा॒णाय॑ त्वा । \newline
46. प्रा॒णायेति॑ प्र - अ॒नाय॑ । \newline
47. त्वेति॑ त्वा । \newline
48. इ॒न्द्रा॒ग्नी मे॑ म इन्द्रा॒ग्नी इ॑न्द्रा॒ग्नी मे᳚ । \newline
49. इ॒न्द्रा॒ग्नी इती᳚न्द्र - अ॒ग्नी । \newline
50. मे॒ वर्चो॒ वर्चो॑ मे मे॒ वर्चः॑ । \newline
51. वर्चः॑ कृणुताम् कृणुता॒म् ॅवर्चो॒ वर्चः॑ कृणुताम् । \newline
52. कृ॒णु॒ता॒म् ॅवर्चो॒ वर्चः॑ कृणुताम् कृणुता॒म् ॅवर्चः॑ । \newline
53. वर्चः॒ सोमः॒ सोमो॒ वर्चो॒ वर्चः॒ सोमः॑ । \newline
54. सोमो॒ बृह॒स्पति॒र् बृह॒स्पतिः॒ सोमः॒ सोमो॒ बृह॒स्पतिः॑ । \newline
55. बृह॒स्पति॒रिति॒ बृह॒स्पतिः॑ । \newline
56. वर्चो॑ मे मे॒ वर्चो॒ वर्चो॑ मे । \newline
57. मे॒ विश्वे॒ विश्वे॑ मे मे॒ विश्वे᳚ । \newline
58. विश्वे॑ दे॒वा दे॒वा विश्वे॒ विश्वे॑ दे॒वाः । \newline
59. दे॒वा वर्चो॒ वर्चो॑ दे॒वा दे॒वा वर्चः॑ । \newline
60. वर्चो॑ मे मे॒ वर्चो॒ वर्चो॑ मे । \newline
61. मे॒ ध॒त्त॒म् ध॒त्त॒म् मे॒ मे॒ ध॒त्त॒म् । \newline
62. ध॒त्त॒ म॒श्वि॒ना॒ ऽश्वि॒ना॒ ध॒त्त॒म् ध॒त्त॒ म॒श्वि॒ना॒ । \newline
63. अ॒श्वि॒नेत्य॑श्विना । \newline
64. द॒ध॒न्वे वा॑ वा दध॒न्वे द॑ध॒न्वे वा᳚ । \newline
65. वा॒ यद् यद् वा॑ वा॒ यत् । \newline
66. यदी॑ मी॒म् ॅयद् यदी᳚म् । \newline
67. ई॒ मन्वन्वी॑ मी॒ मनु॑ । \newline
68. अनु॒ वोच॒द् वोच॒ दन्वनु॒ वोच॑त् । \newline
69. वोच॒द् ब्रह्मा॑णि॒ ब्रह्मा॑णि॒ वोच॒द् वोच॒द् ब्रह्मा॑णि । \newline
70. ब्रह्मा॑णि॒ वेर् वेर् ब्रह्मा॑णि॒ ब्रह्मा॑णि॒ वेः । \newline
71. वेरु॑ वु॒ वेर् वेरु॑ । \newline
72. उ॒ तत् तदू॒ तत् । \newline
73. तदिति॒ तत् । \newline
74. परि॒ विश्वा॑नि॒ विश्वा॑नि॒ परि॒ परि॒ विश्वा॑नि । \newline
75. विश्वा॑नि॒ काव्या॒ काव्या॒ विश्वा॑नि॒ विश्वा॑नि॒ काव्या᳚ । \newline
76. काव्या॑ ने॒मिर् ने॒मिः काव्या॒ काव्या॑ ने॒मिः । \newline
77. ने॒मि श्च॒क्रम् च॒क्रम् ने॒मिर् ने॒मि श्च॒क्रम् । \newline
78. च॒क्र मि॑वे व च॒क्रम् च॒क्र मि॑व । \newline
79. इ॒वा॒ भ॒व॒ द॒भ॒व॒ दि॒वे॒ वा॒भ॒व॒त् । \newline
80. अ॒भ॒व॒दित्य॑भवत् । \newline

\textbf{Ghana Paata } \newline

1. प्रि॒यम् पाथः॒ पाथः॑ प्रि॒यम् प्रि॒यम् पाथो॒ अप्यपि॒ पाथः॑ प्रि॒यम् प्रि॒यम् पाथो॒ अपि॑ । \newline
2. पाथो॒ अप्यपि॒ पाथः॒ पाथो॒ अपी॑ ही॒ह्यपि॒ पाथः॒ पाथो॒ अपी॑हि । \newline
3. अपी॑ही॒ ह्यप्यपी॑हि व॒शी व॒शी ह्यप्यपी॑हि व॒शी । \newline
4. इ॒हि॒ व॒शी व॒शीही॑हि व॒शी त्वम् त्वम् ॅव॒शीही॑हि व॒शी त्वम् । \newline
5. व॒शी त्वम् त्वम् ॅव॒शी व॒शी त्वम् दे॑व देव॒ त्वम् ॅव॒शी व॒शी त्वम् दे॑व । \newline
6. त्वम् दे॑व देव॒ त्वम् त्वम् दे॑व सोम सोम देव॒ त्वम् त्वम् दे॑व सोम । \newline
7. दे॒व॒ सो॒म॒ सो॒म॒ दे॒व॒ दे॒व॒ सो॒म॒ त्रैष्टु॑भेन॒ त्रैष्टु॑भेन सोम देव देव सोम॒ त्रैष्टु॑भेन । \newline
8. सो॒म॒ त्रैष्टु॑भेन॒ त्रैष्टु॑भेन सोम सोम॒ त्रैष्टु॑भेन॒ छन्द॑सा॒ छन्द॑सा॒ त्रैष्टु॑भेन सोम सोम॒ त्रैष्टु॑भेन॒ छन्द॑सा । \newline
9. त्रैष्टु॑भेन॒ छन्द॑सा॒ छन्द॑सा॒ त्रैष्टु॑भेन॒ त्रैष्टु॑भेन॒ छन्द॒ सेन्द्र॒स्ये न्द्र॑स्य॒ छन्द॑सा॒ त्रैष्टु॑भेन॒ त्रैष्टु॑भेन॒ छन्द॒ सेन्द्र॑स्य । \newline
10. छन्द॒ सेन्द्र॒स्ये न्द्र॑स्य॒ छन्द॑सा॒ छन्द॒ सेन्द्र॑स्य प्रि॒यम् प्रि॒य मिन्द्र॑स्य॒ छन्द॑सा॒ छन्द॒ सेन्द्र॑स्य प्रि॒यम् । \newline
11. इन्द्र॑स्य प्रि॒यम् प्रि॒य मिन्द्र॒ स्येन्द्र॑स्य प्रि॒यम् पाथः॒ पाथः॑ प्रि॒य मिन्द्र॒ स्येन्द्र॑स्य प्रि॒यम् पाथः॑ । \newline
12. प्रि॒यम् पाथः॒ पाथः॑ प्रि॒यम् प्रि॒यम् पाथो॒ अप्यपि॒ पाथः॑ प्रि॒यम् प्रि॒यम् पाथो॒ अपि॑ । \newline
13. पाथो॒ अप्यपि॒ पाथः॒ पाथो॒ अपी॑ही॒ ह्यपि॒ पाथः॒ पाथो॒ अपी॑हि । \newline
14. अपी॑ही॒ ह्यप्यपी᳚ ह्य॒स्मथ्स॑खा॒ ऽस्मथ्स॑खे॒ ह्यप्यपी᳚ ह्य॒स्मथ्स॑खा । \newline
15. इ॒ह्य॒स्मथ्स॑खा॒ ऽस्मथ्स॑खेही ह्य॒स्मथ्स॑खा॒ त्वम् त्व म॒स्मथ्स॑खेही ह्य॒स्मथ्स॑खा॒ त्वम् । \newline
16. अ॒स्मथ्स॑खा॒ त्वम् त्व म॒स्मथ्स॑खा॒ ऽस्मथ्स॑खा॒ त्वम् दे॑व देव॒ त्व म॒स्मथ्स॑खा॒ ऽस्मथ्स॑खा॒ त्वम् दे॑व । \newline
17. अ॒स्मथ्स॒खेत्य॒स्मत् - स॒खा॒ । \newline
18. त्वम् दे॑व देव॒ त्वम् त्वम् दे॑व सोम सोम देव॒ त्वम् त्वम् दे॑व सोम । \newline
19. दे॒व॒ सो॒म॒ सो॒म॒ दे॒व॒ दे॒व॒ सो॒म॒ जाग॑तेन॒ जाग॑तेन सोम देव देव सोम॒ जाग॑तेन । \newline
20. सो॒म॒ जाग॑तेन॒ जाग॑तेन सोम सोम॒ जाग॑तेन॒ छन्द॑सा॒ छन्द॑सा॒ जाग॑तेन सोम सोम॒ जाग॑तेन॒ छन्द॑सा । \newline
21. जाग॑तेन॒ छन्द॑सा॒ छन्द॑सा॒ जाग॑तेन॒ जाग॑तेन॒ छन्द॑सा॒ विश्वे॑षा॒म् ॅविश्वे॑षा॒म् छन्द॑सा॒ जाग॑तेन॒ जाग॑तेन॒ छन्द॑सा॒ विश्वे॑षाम् । \newline
22. छन्द॑सा॒ विश्वे॑षा॒म् ॅविश्वे॑षा॒म् छन्द॑सा॒ छन्द॑सा॒ विश्वे॑षाम् दे॒वाना᳚म् दे॒वाना॒म् ॅविश्वे॑षा॒म् छन्द॑सा॒ छन्द॑सा॒ विश्वे॑षाम् दे॒वाना᳚म् । \newline
23. विश्वे॑षाम् दे॒वाना᳚म् दे॒वाना॒म् ॅविश्वे॑षा॒म् ॅविश्वे॑षाम् दे॒वाना᳚म् प्रि॒यम् प्रि॒यम् दे॒वाना॒म् ॅविश्वे॑षा॒म् ॅविश्वे॑षाम् दे॒वाना᳚म् प्रि॒यम् । \newline
24. दे॒वाना᳚म् प्रि॒यम् प्रि॒यम् दे॒वाना᳚म् दे॒वाना᳚म् प्रि॒यम् पाथः॒ पाथः॑ प्रि॒यम् दे॒वाना᳚म् दे॒वाना᳚म् प्रि॒यम् पाथः॑ । \newline
25. प्रि॒यम् पाथः॒ पाथः॑ प्रि॒यम् प्रि॒यम् पाथो॒ अप्यपि॒ पाथः॑ प्रि॒यम् प्रि॒यम् पाथो॒ अपि॑ । \newline
26. पाथो॒ अप्यपि॒ पाथः॒ पाथो॒ अपी॑ही॒ ह्यपि॒ पाथः॒ पाथो॒ अपी॑हि । \newline
27. अपी॑ही॒ ह्यप्यपी॒ ह्येह्यप्यपी॒ह्या । \newline
28. इ॒ह्येही॒ह्या नो॑ न॒ एही॒ह्या नः॑ । \newline
29. आ नो॑ न॒ आ नः॑ प्रा॒णः प्रा॒णो न॒ आ नः॑ प्रा॒णः । \newline
30. नः॒ प्रा॒णः प्रा॒णो नो॑ नः प्रा॒ण ए᳚त्वेतु प्रा॒णो नो॑ नः प्रा॒ण ए॑तु । \newline
31. प्रा॒ण ए᳚त्वेतु प्रा॒णः प्रा॒ण ए॑तु परा॒वतः॑ परा॒वत॑ एतु प्रा॒णः प्रा॒ण ए॑तु परा॒वतः॑ । \newline
32. प्रा॒ण इति॑ प्र - अ॒नः । \newline
33. ए॒तु॒ प॒रा॒वतः॑ परा॒वत॑ एत्वेतु परा॒वत॒ आ प॑रा॒वत॑ एत्वेतु परा॒वत॒ आ । \newline
34. प॒रा॒वत॒ आ प॑रा॒वतः॑ परा॒वत॒ आ ऽन्तरि॑क्षा द॒न्तरि॑क्षा॒दा प॑रा॒वतः॑ परा॒वत॒ आ ऽन्तरि॑क्षात् । \newline
35. प॒रा॒वत॒ इति॑ परा - वतः॑ । \newline
36. आ ऽन्तरि॑क्षा द॒न्तरि॑क्षा॒दा ऽन्तरि॑क्षाद् दि॒वो दि॒वो᳚ ऽन्तरि॑क्षा॒दा ऽन्तरि॑क्षाद् दि॒वः । \newline
37. अ॒न्तरि॑क्षाद् दि॒वो दि॒वो᳚ ऽन्तरि॑क्षा द॒न्तरि॑क्षाद् दि॒व स्परि॒ परि॑ दि॒वो᳚ ऽन्तरि॑क्षा द॒न्तरि॑क्षाद् दि॒व स्परि॑ । \newline
38. दि॒व स्परि॒ परि॑ दि॒वो दि॒व स्परि॑ । \newline
39. परीति॒ परि॑ । \newline
40. आयुः॑ पृथि॒व्याः पृ॑थि॒व्या आयु॒ रायुः॑ पृथि॒व्या अध्यधि॑ पृथि॒व्या आयु॒ रायुः॑ पृथि॒व्या अधि॑ । \newline
41. पृ॒थि॒व्या अध्यधि॑ पृथि॒व्याः पृ॑थि॒व्या अध्य॒मृत॑ म॒मृत॒ मधि॑ पृथि॒व्याः पृ॑थि॒व्या अध्य॒मृत᳚म् । \newline
42. अध्य॒मृत॑ म॒मृत॒ मध्य ध्य॒मृत॑ मस्य स्य॒मृत॒ मध्य ध्य॒मृत॑ मसि । \newline
43. अ॒मृत॑ मस्य स्य॒मृत॑ म॒मृत॑ मसि प्रा॒णाय॑ प्रा॒णाया᳚ स्य॒मृत॑ म॒मृत॑ मसि प्रा॒णाय॑ । \newline
44. अ॒सि॒ प्रा॒णाय॑ प्रा॒णाया᳚ स्यसि प्रा॒णाय॑ त्वा त्वा प्रा॒णाया᳚ स्यसि प्रा॒णाय॑ त्वा । \newline
45. प्रा॒णाय॑ त्वा त्वा प्रा॒णाय॑ प्रा॒णाय॑ त्वा । \newline
46. प्रा॒णायेति॑ प्र - अ॒नाय॑ । \newline
47. त्वेति॑ त्वा । \newline
48. इ॒न्द्रा॒ग्नी मे॑ म इन्द्रा॒ग्नी इ॑न्द्रा॒ग्नी मे॒ वर्चो॒ वर्चो॑ म इन्द्रा॒ग्नी इ॑न्द्रा॒ग्नी मे॒ वर्चः॑ । \newline
49. इ॒न्द्रा॒ग्नी इती᳚न्द्र - अ॒ग्नी । \newline
50. मे॒ वर्चो॒ वर्चो॑ मे मे॒ वर्चः॑ कृणुताम् कृणुता॒म् ॅवर्चो॑ मे मे॒ वर्चः॑ कृणुताम् । \newline
51. वर्चः॑ कृणुताम् कृणुता॒म् ॅवर्चो॒ वर्चः॑ कृणुता॒म् ॅवर्चो॒ वर्चः॑ कृणुता॒म् ॅवर्चो॒ वर्चः॑ कृणुता॒म् ॅवर्चः॑ । \newline
52. कृ॒णु॒ता॒म् ॅवर्चो॒ वर्चः॑ कृणुताम् कृणुता॒म् ॅवर्चः॒ सोमः॒ सोमो॒ वर्चः॑ कृणुताम् कृणुता॒म् ॅवर्चः॒ सोमः॑ । \newline
53. वर्चः॒ सोमः॒ सोमो॒ वर्चो॒ वर्चः॒ सोमो॒ बृह॒स्पति॒र् बृह॒स्पतिः॒ सोमो॒ वर्चो॒ वर्चः॒ सोमो॒ बृह॒स्पतिः॑ । \newline
54. सोमो॒ बृह॒स्पति॒र् बृह॒स्पतिः॒ सोमः॒ सोमो॒ बृह॒स्पतिः॑ । \newline
55. बृह॒स्पति॒रिति॒ बृह॒स्पतिः॑ । \newline
56. वर्चो॑ मे मे॒ वर्चो॒ वर्चो॑ मे॒ विश्वे॒ विश्वे॑ मे॒ वर्चो॒ वर्चो॑ मे॒ विश्वे᳚ । \newline
57. मे॒ विश्वे॒ विश्वे॑ मे मे॒ विश्वे॑ दे॒वा दे॒वा विश्वे॑ मे मे॒ विश्वे॑ दे॒वाः । \newline
58. विश्वे॑ दे॒वा दे॒वा विश्वे॒ विश्वे॑ दे॒वा वर्चो॒ वर्चो॑ दे॒वा विश्वे॒ विश्वे॑ दे॒वा वर्चः॑ । \newline
59. दे॒वा वर्चो॒ वर्चो॑ दे॒वा दे॒वा वर्चो॑ मे मे॒ वर्चो॑ दे॒वा दे॒वा वर्चो॑ मे । \newline
60. वर्चो॑ मे मे॒ वर्चो॒ वर्चो॑ मे धत्तम् धत्तम् मे॒ वर्चो॒ वर्चो॑ मे धत्तम् । \newline
61. मे॒ ध॒त्त॒म् ध॒त्त॒म् मे॒ मे॒ ध॒त्त॒ म॒श्वि॒ना॒ ऽश्वि॒ना॒ ध॒त्त॒म् मे॒ मे॒ ध॒त्त॒ म॒श्वि॒ना॒ । \newline
62. ध॒त्त॒ म॒श्वि॒ना॒ ऽश्वि॒ना॒ ध॒त्त॒म् ध॒त्त॒ म॒श्वि॒ना॒ । \newline
63. अ॒श्वि॒नेत्य॑श्विना । \newline
64. द॒ध॒न्वे वा॑ वा दध॒न्वे द॑ध॒न्वे वा॒ यद् यद् वा॑ दध॒न्वे द॑ध॒न्वे वा॒ यत् । \newline
65. वा॒ यद् यद् वा॑ वा॒ यदी॑ मी॒म् ॅयद् वा॑ वा॒ यदी᳚म् । \newline
66. यदी॑ मी॒म् ॅयद् यदी॒ मन्वन्वी॒म् ॅयद् यदी॒ मनु॑ । \newline
67. ई॒ मन्वन्वी॑ मी॒ मनु॒ वोच॒द् वोच॒ दन्वी॑ मी॒ मनु॒ वोच॑त् । \newline
68. अनु॒ वोच॒द् वोच॒ दन्वनु॒ वोच॒द् ब्रह्मा॑णि॒ ब्रह्मा॑णि॒ वोच॒ दन्वनु॒ वोच॒द् ब्रह्मा॑णि । \newline
69. वोच॒द् ब्रह्मा॑णि॒ ब्रह्मा॑णि॒ वोच॒द् वोच॒द् ब्रह्मा॑णि॒ वेर् वेर् ब्रह्मा॑णि॒ वोच॒द् वोच॒द् ब्रह्मा॑णि॒ वेः । \newline
70. ब्रह्मा॑णि॒ वेर् वेर् ब्रह्मा॑णि॒ ब्रह्मा॑णि॒ वेरु॑ वु॒ वेर् ब्रह्मा॑णि॒ ब्रह्मा॑णि॒ वेरु॑ । \newline
71. वेरु॑ वु॒ वेर् वेरु॒ तत् तदु॒ वेर् वेरु॒ तत् । \newline
72. उ॒ तत् तदू॒ तत् । \newline
73. तदिति॒ तत् । \newline
74. परि॒ विश्वा॑नि॒ विश्वा॑नि॒ परि॒ परि॒ विश्वा॑नि॒ काव्या॒ काव्या॒ विश्वा॑नि॒ परि॒ परि॒ विश्वा॑नि॒ काव्या᳚ । \newline
75. विश्वा॑नि॒ काव्या॒ काव्या॒ विश्वा॑नि॒ विश्वा॑नि॒ काव्या॑ ने॒मिर् ने॒मिः काव्या॒ विश्वा॑नि॒ विश्वा॑नि॒ काव्या॑ ने॒मिः । \newline
76. काव्या॑ ने॒मिर् ने॒मिः काव्या॒ काव्या॑ ने॒मि श्च॒क्रम् च॒क्रम् ने॒मिः काव्या॒ काव्या॑ ने॒मि श्च॒क्रम् । \newline
77. ने॒मि श्च॒क्रम् च॒क्रम् ने॒मिर् ने॒मि श्च॒क्र मि॑वे व च॒क्रम् ने॒मिर् ने॒मि श्च॒क्र मि॑व । \newline
78. च॒क्र मि॑वे व च॒क्रम् च॒क्र मि॑वा भव दभव दिव च॒क्रम् च॒क्र मि॑वा भवत् । \newline
79. इ॒वा॒ भ॒व॒ द॒भ॒व॒ दि॒वे॒ वा॒भ॒व॒त् । \newline
80. अ॒भ॒व॒दित्य॑भवत् । \newline
\pagebreak
\markright{ TS 3.3.4.1  \hfill https://www.vedavms.in \hfill}

\section{ TS 3.3.4.1 }

\textbf{TS 3.3.4.1 } \newline
\textbf{Samhita Paata} \newline

ए॒तद्वा अ॒पां ना॑म॒धेयं॒ गुह्यं॒ ॅयदा॑धा॒वा मान्दा॑सु ते शुक्र शु॒क्रमा धू॑नो॒मीत्या॑हा॒पामे॒व ना॑म॒धेये॑न॒ गुह्ये॑न दि॒वो वृष्टि॒मव॑ रुन्धे शु॒क्रं ते॑ शु॒क्रेण॑ गृह्णा॒मीत्या॑है॒तद्वा अह्नो॑ रू॒पं ॅयद्रात्रिः॒ सूर्य॑स्य र॒श्मयो॒ वृष्ट्या॑ ईश॒तेऽह्न॑ ए॒व रू॒पेण॒ सूर्य॑स्य र॒श्मिभि॑र्दि॒वो वृष्टिं॑ च्यावय॒त्याऽस्मि॑न्नु॒ग्रा - [  ] \newline

\textbf{Pada Paata} \newline

ए॒तत् । वै । अ॒पाम् । ना॒म॒धेय॒मिति॑ नाम - धेय᳚म् । गुह्य᳚म् । यत् । आ॒धा॒वा इत्या᳚ - धा॒वाः । मान्दा॑सु । ते॒ । शु॒क्र॒ । शु॒क्रम् । एति॑ । धू॒नो॒मि॒ । इति॑ । आ॒ह॒ । अ॒पाम् । ए॒व । ना॒म॒धेये॒नेति॑ नाम - धेये॑न । गुह्ये॑न । दि॒वः । वृष्टि᳚म् । अवेति॑ । रु॒न्धे॒ । शु॒क्रम् । ते॒ । शु॒क्रेण॑ । गृ॒ह्णा॒मि॒ । इति॑ । आ॒ह॒ । ए॒तत् । वै । अह्नः॑ । रू॒पम् । यत् । रात्रिः॑ । सूर्य॑स्य । र॒श्मयः॑ । वृष्ट्याः᳚ । ई॒श॒ते॒ । अह्नः॑ । ए॒व । रू॒पेण॑ । सूर्य॑स्य । र॒श्मिभि॒रिति॑ र॒श्मि - भिः॒ । दि॒वः । वृष्टि᳚म् । च्या॒व॒य॒ति॒ । एति॑ । अ॒स्मि॒न्न् । उ॒ग्राः ।  \newline


\textbf{Krama Paata} \newline

ए॒तद् वै । वा अ॒पाम् । अ॒पाम् ना॑म॒धेय᳚म् । ना॒म॒धेय॒म् गुह्य᳚म् । ना॒म॒धेय॒मिति॑ नाम - धेय᳚म् । गुह्यं॒ ॅयत् । यदा॑धा॒वाः । आ॒धा॒वा मान्दा॑सु । आ॒धा॒वा इत्या᳚ - धा॒वाः । मान्दा॑सु ते । ते॒ शु॒क्र॒ । शु॒क्र॒ शु॒क्रम् । शु॒क्रमा । आ धू॑नोमि । धू॒नो॒मीति॑ । इत्या॑ह । आ॒हा॒पाम् । अ॒पामे॒व । ए॒व ना॑म॒धेये॑न । ना॒म॒धेये॑न॒ गुह्ये॑न । ना॒म॒धेये॒नेति॑ नाम - धेये॑न । गुह्ये॑न दि॒वः । दि॒वो वृष्टि᳚म् । वृष्टि॒मव॑ । अव॑ रुन्धे । रु॒न्धे॒ शु॒क्रम् । शु॒क्रम् ते᳚ । ते॒ शु॒क्रेण॑ । शु॒क्रेण॑ गृह्णामि । गृ॒ह्णा॒मीति॑ । इत्या॑ह । आ॒है॒तत् । ए॒तद् वै । वा अह्नः॑ । अह्नो॑ रू॒पम् । रू॒पं ॅयत् । यद् रात्रिः॑ । रात्रिः॒ सूर्य॑स्य । सूर्य॑स्य र॒श्मयः॑ । र॒श्मयो॒ वृष्ट्याः᳚ । वृष्ट्या॑ ईशते । ई॒श॒ते ऽह्नः॑ । अह्न॑ ए॒व । ए॒व रू॒पेण॑ । रू॒पेण॒ सूर्य॑स्य । सूर्य॑स्य र॒श्मिभिः॑ । र॒श्मिभि॑र् दि॒वः । र॒श्मिभि॒रिति॑ र॒श्मि - भिः॒ । दि॒वो वृष्टि᳚म् । वृष्टि॑म् च्यावयति । च्या॒व॒य॒त्या । आ ऽस्मिन्न्॑ । अ॒स्मि॒न्नु॒ग्राः । उ॒ग्रा अ॑चुच्यवुः \newline

\textbf{Jatai Paata} \newline

1. ए॒तद् वै वा ए॒त दे॒तद् वै । \newline
2. वा अ॒पा म॒पाम् ॅवै वा अ॒पाम् । \newline
3. अ॒पाम् ना॑म॒धेय॑म् नाम॒धेय॑ म॒पा म॒पाम् ना॑म॒धेय᳚म् । \newline
4. ना॒म॒धेय॒म् गुह्य॒म् गुह्य॑म् नाम॒धेय॑म् नाम॒धेय॒म् गुह्य᳚म् । \newline
5. ना॒म॒धेय॒मिति॑ नाम - धेय᳚म् । \newline
6. गुह्य॒म् ॅयद् यद् गुह्य॒म् गुह्य॒म् ॅयत् । \newline
7. यदा॑धा॒वा आ॑धा॒वा यद् यदा॑धा॒वाः । \newline
8. आ॒धा॒वा मान्दा॑सु॒ मान्दा᳚ स्वाधा॒वा आ॑धा॒वा मान्दा॑सु । \newline
9. आ॒धा॒वा इत्या᳚ - धा॒वाः । \newline
10. मान्दा॑सु ते ते॒ मान्दा॑सु॒ मान्दा॑सु ते । \newline
11. ते॒ शु॒क्र॒ शु॒क्र॒ ते॒ ते॒ शु॒क्र॒ । \newline
12. शु॒क्र॒ शु॒क्रꣳ शु॒क्रꣳ शु॑क्र शुक्र शु॒क्रम् । \newline
13. शु॒क्र मा शु॒क्रꣳ शु॒क्र मा । \newline
14. आ धू॑नोमि धूनो॒ म्या धू॑नोमि । \newline
15. धू॒नो॒ मीतीति॑ धूनोमि धूनो॒मीति॑ । \newline
16. इत्या॑हा॒हे तीत्या॑ह । \newline
17. आ॒हा॒पा म॒पा मा॑हाहा॒ पाम् । \newline
18. अ॒पा मे॒वैवापा म॒पा मे॒व । \newline
19. ए॒व ना॑म॒धेये॑न नाम॒धेये॑ नै॒वैव ना॑म॒धेये॑न । \newline
20. ना॒म॒धेये॑न॒ गुह्ये॑न॒ गुह्ये॑न नाम॒धेये॑न नाम॒धेये॑न॒ गुह्ये॑न । \newline
21. ना॒म॒धेये॒नेति॑ नाम - धेये॑न । \newline
22. गुह्ये॑न दि॒वो दि॒वो गुह्ये॑न॒ गुह्ये॑न दि॒वः । \newline
23. दि॒वो वृष्टि॒म् ॅवृष्टि॑म् दि॒वो दि॒वो वृष्टि᳚म् । \newline
24. वृष्टि॒ मवाव॒ वृष्टि॒म् ॅवृष्टि॒ मव॑ । \newline
25. अव॑ रुन्धे रु॒न्धे ऽवाव॑ रुन्धे । \newline
26. रु॒न्धे॒ शु॒क्रꣳ शु॒क्रꣳ रु॑न्धे रुन्धे शु॒क्रम् । \newline
27. शु॒क्रम् ते॑ ते शु॒क्रꣳ शु॒क्रम् ते᳚ । \newline
28. ते॒ शु॒क्रेण॑ शु॒क्रेण॑ ते ते शु॒क्रेण॑ । \newline
29. शु॒क्रेण॑ गृह्णामि गृह्णामि शु॒क्रेण॑ शु॒क्रेण॑ गृह्णामि । \newline
30. गृ॒ह्णा॒ मीतीति॑ गृह्णामि गृह्णा॒ मीति॑ । \newline
31. इत्या॑हा॒हे तीत्या॑ह । \newline
32. आ॒है॒त दे॒त दा॑हा है॒तत् । \newline
33. ए॒तद् वै वा ए॒त दे॒तद् वै । \newline
34. वा अह्नो ऽह्नो॒ वै वा अह्नः॑ । \newline
35. अह्नो॑ रू॒पꣳ रू॒प मह्नो ऽह्नो॑ रू॒पम् । \newline
36. रू॒पम् ॅयद् यद् रू॒पꣳ रू॒पम् ॅयत् । \newline
37. यद् रात्री॒ रात्रि॒र् यद् यद् रात्रिः॑ । \newline
38. रात्रिः॒ सूर्य॑स्य॒ सूर्य॑स्य॒ रात्री॒ रात्रिः॒ सूर्य॑स्य । \newline
39. सूर्य॑स्य र॒श्मयो॑ र॒श्मयः॒ सूर्य॑स्य॒ सूर्य॑स्य र॒श्मयः॑ । \newline
40. र॒श्मयो॒ वृष्ट्या॒ वृष्ट्या॑ र॒श्मयो॑ र॒श्मयो॒ वृष्ट्याः᳚ । \newline
41. वृष्ट्या॑ ईशत ईशते॒ वृष्ट्या॒ वृष्ट्या॑ ईशते । \newline
42. ई॒श॒ते ऽह्नो ऽह्न॑ ईशत ईश॒ते ऽह्नः॑ । \newline
43. अह्न॑ ए॒वैवाह्नो ऽह्न॑ ए॒व । \newline
44. ए॒व रू॒पेण॑ रू॒पे णै॒वैव रू॒पेण॑ । \newline
45. रू॒पेण॒ सूर्य॑स्य॒ सूर्य॑स्य रू॒पेण॑ रू॒पेण॒ सूर्य॑स्य । \newline
46. सूर्य॑स्य र॒श्मिभी॑ र॒श्मिभिः॒ सूर्य॑स्य॒ सूर्य॑स्य र॒श्मिभिः॑ । \newline
47. र॒श्मिभि॑र् दि॒वो दि॒वो र॒श्मिभी॑ र॒श्मिभि॑र् दि॒वः । \newline
48. र॒श्मिभि॒रिति॑ र॒श्मि - भिः॒ । \newline
49. दि॒वो वृष्टि॒म् ॅवृष्टि॑म् दि॒वो दि॒वो वृष्टि᳚म् । \newline
50. वृष्टि॑म् च्यावयति च्यावयति॒ वृष्टि॒म् ॅवृष्टि॑म् च्यावयति । \newline
51. च्या॒व॒य॒त्या च्या॑वयति च्यावय॒त्या । \newline
52. आ ऽस्मि॑न् नस्मि॒न् ना ऽस्मिन्न्॑ । \newline
53. अ॒स्मि॒न् नु॒ग्रा उ॒ग्रा अ॑स्मिन् नस्मिन् नु॒ग्राः । \newline
54. उ॒ग्रा अ॑चुच्यवु रचुच्यवु रु॒ग्रा उ॒ग्रा अ॑चुच्यवुः । \newline

\textbf{Ghana Paata } \newline

1. ए॒तद् वै वा ए॒त दे॒तद् वा अ॒पा म॒पाम् ॅवा ए॒त दे॒तद् वा अ॒पाम् । \newline
2. वा अ॒पा म॒पाम् ॅवै वा अ॒पाम् ना॑म॒धेय॑म् नाम॒धेय॑ म॒पाम् ॅवै वा अ॒पाम् ना॑म॒धेय᳚म् । \newline
3. अ॒पाम् ना॑म॒धेय॑म् नाम॒धेय॑ म॒पा म॒पाम् ना॑म॒धेय॒म् गुह्य॒म् गुह्य॑म् नाम॒धेय॑ म॒पा म॒पाम् ना॑म॒धेय॒म् गुह्य᳚म् । \newline
4. ना॒म॒धेय॒म् गुह्य॒म् गुह्य॑म् नाम॒धेय॑म् नाम॒धेय॒म् गुह्य॒म् ॅयद् यद् गुह्य॑म् नाम॒धेय॑म् नाम॒धेय॒म् गुह्य॒म् ॅयत् । \newline
5. ना॒म॒धेय॒मिति॑ नाम - धेय᳚म् । \newline
6. गुह्य॒म् ॅयद् यद् गुह्य॒म् गुह्य॒म् ॅयदा॑धा॒वा आ॑धा॒वा यद् गुह्य॒म् गुह्य॒म् ॅयदा॑धा॒वाः । \newline
7. यदा॑धा॒वा आ॑धा॒वा यद् यदा॑धा॒वा मान्दा॑सु॒ मान्दा᳚ स्वाधा॒वा यद् यदा॑धा॒वा मान्दा॑सु । \newline
8. आ॒धा॒वा मान्दा॑सु॒ मान्दा᳚ स्वाधा॒वा आ॑धा॒वा मान्दा॑सु ते ते॒ मान्दा᳚ स्वाधा॒वा आ॑धा॒वा मान्दा॑सु ते । \newline
9. आ॒धा॒वा इत्या᳚ - धा॒वाः । \newline
10. मान्दा॑सु ते ते॒ मान्दा॑सु॒ मान्दा॑सु ते शुक्र शुक्र ते॒ मान्दा॑सु॒ मान्दा॑सु ते शुक्र । \newline
11. ते॒ शु॒क्र॒ शु॒क्र॒ ते॒ ते॒ शु॒क्र॒ शु॒क्रꣳ शु॒क्रꣳ शु॑क्र ते ते शुक्र शु॒क्रम् । \newline
12. शु॒क्र॒ शु॒क्रꣳ शु॒क्रꣳ शु॑क्र शुक्र शु॒क्र मा शु॒क्रꣳ शु॑क्र शुक्र शु॒क्र मा । \newline
13. शु॒क्र मा शु॒क्रꣳ शु॒क्र मा धू॑नोमि धूनो॒म्या शु॒क्रꣳ शु॒क्र मा धू॑नोमि । \newline
14. आ धू॑नोमि धूनो॒म्या धू॑नो॒मीतीति॑ धूनो॒म्या धू॑नो॒मीति॑ । \newline
15. धू॒नो॒मीतीति॑ धूनोमि धूनो॒मी त्या॑हा॒हे ति॑ धूनोमि धूनो॒मी त्या॑ह । \newline
16. इत्या॑हा॒हे तीत्या॑हा॒पा म॒पा मा॒हे तीत्या॑हा॒पाम् । \newline
17. आ॒हा॒पा म॒पा मा॑हाहा॒पा मे॒वैवापा मा॑हाहा॒पा मे॒व । \newline
18. अ॒पा मे॒वैवापा म॒पा मे॒व ना॑म॒धेये॑न नाम॒धेये॑ नै॒वापा म॒पा मे॒व ना॑म॒धेये॑न । \newline
19. ए॒व ना॑म॒धेये॑न नाम॒धेये॑ नै॒वैव ना॑म॒धेये॑न॒ गुह्ये॑न॒ गुह्ये॑न नाम॒धेये॑ नै॒वैव ना॑म॒धेये॑न॒ गुह्ये॑न । \newline
20. ना॒म॒धेये॑न॒ गुह्ये॑न॒ गुह्ये॑न नाम॒धेये॑न नाम॒धेये॑न॒ गुह्ये॑न दि॒वो दि॒वो गुह्ये॑न नाम॒धेये॑न नाम॒धेये॑न॒ गुह्ये॑न दि॒वः । \newline
21. ना॒म॒धेये॒नेति॑ नाम - धेये॑न । \newline
22. गुह्ये॑न दि॒वो दि॒वो गुह्ये॑न॒ गुह्ये॑न दि॒वो वृष्टि॒म् ॅवृष्टि॑म् दि॒वो गुह्ये॑न॒ गुह्ये॑न दि॒वो वृष्टि᳚म् । \newline
23. दि॒वो वृष्टि॒म् ॅवृष्टि॑म् दि॒वो दि॒वो वृष्टि॒ मवाव॒ वृष्टि॑म् दि॒वो दि॒वो वृष्टि॒ मव॑ । \newline
24. वृष्टि॒ मवाव॒ वृष्टि॒म् ॅवृष्टि॒ मव॑ रुन्धे रु॒न्धे ऽव॒ वृष्टि॒म् ॅवृष्टि॒ मव॑ रुन्धे । \newline
25. अव॑ रुन्धे रु॒न्धे ऽवाव॑ रुन्धे शु॒क्रꣳ शु॒क्रꣳ रु॒न्धे ऽवाव॑ रुन्धे शु॒क्रम् । \newline
26. रु॒न्धे॒ शु॒क्रꣳ शु॒क्रꣳ रु॑न्धे रुन्धे शु॒क्रम् ते॑ ते शु॒क्रꣳ रु॑न्धे रुन्धे शु॒क्रम् ते᳚ । \newline
27. शु॒क्रम् ते॑ ते शु॒क्रꣳ शु॒क्रम् ते॑ शु॒क्रेण॑ शु॒क्रेण॑ ते शु॒क्रꣳ शु॒क्रम् ते॑ शु॒क्रेण॑ । \newline
28. ते॒ शु॒क्रेण॑ शु॒क्रेण॑ ते ते शु॒क्रेण॑ गृह्णामि गृह्णामि शु॒क्रेण॑ ते ते शु॒क्रेण॑ गृह्णामि । \newline
29. शु॒क्रेण॑ गृह्णामि गृह्णामि शु॒क्रेण॑ शु॒क्रेण॑ गृह्णा॒मीतीति॑ गृह्णामि शु॒क्रेण॑ शु॒क्रेण॑ गृह्णा॒मीति॑ । \newline
30. गृ॒ह्णा॒मीतीति॑ गृह्णामि गृह्णा॒मी त्या॑हा॒हे ति॑ गृह्णामि गृह्णा॒मी त्या॑ह । \newline
31. इत्या॑हा॒हे तीत्या॑ है॒त दे॒तदा॒हे तीत्या॑है॒तत् । \newline
32. आ॒है॒त दे॒त दा॑हा है॒तद् वै वा ए॒त दा॑हा है॒तद् वै । \newline
33. ए॒तद् वै वा ए॒त दे॒तद् वा अह्नो ऽह्नो॒ वा ए॒त दे॒तद् वा अह्नः॑ । \newline
34. वा अह्नो ऽह्नो॒ वै वा अह्नो॑ रू॒पꣳ रू॒प मह्नो॒ वै वा अह्नो॑ रू॒पम् । \newline
35. अह्नो॑ रू॒पꣳ रू॒प मह्नो ऽह्नो॑ रू॒पम् ॅयद् यद् रू॒प मह्नो ऽह्नो॑ रू॒पम् ॅयत् । \newline
36. रू॒पम् ॅयद् यद् रू॒पꣳ रू॒पम् ॅयद् रात्री॒ रात्रि॒र् यद् रू॒पꣳ रू॒पम् ॅयद् रात्रिः॑ । \newline
37. यद् रात्री॒ रात्रि॒र् यद् यद् रात्रिः॒ सूर्य॑स्य॒ सूर्य॑स्य॒ रात्रि॒र् यद् यद् रात्रिः॒ सूर्य॑स्य । \newline
38. रात्रिः॒ सूर्य॑स्य॒ सूर्य॑स्य॒ रात्री॒ रात्रिः॒ सूर्य॑स्य र॒श्मयो॑ र॒श्मयः॒ सूर्य॑स्य॒ रात्री॒ रात्रिः॒ सूर्य॑स्य र॒श्मयः॑ । \newline
39. सूर्य॑स्य र॒श्मयो॑ र॒श्मयः॒ सूर्य॑स्य॒ सूर्य॑स्य र॒श्मयो॒ वृष्ट्या॒ वृष्ट्या॑ र॒श्मयः॒ सूर्य॑स्य॒ सूर्य॑स्य र॒श्मयो॒ वृष्ट्याः᳚ । \newline
40. र॒श्मयो॒ वृष्ट्या॒ वृष्ट्या॑ र॒श्मयो॑ र॒श्मयो॒ वृष्ट्या॑ ईशत ईशते॒ वृष्ट्या॑ र॒श्मयो॑ र॒श्मयो॒ वृष्ट्या॑ ईशते । \newline
41. वृष्ट्या॑ ईशत ईशते॒ वृष्ट्या॒ वृष्ट्या॑ ईश॒ते ऽह्नो ऽह्न॑ ईशते॒ वृष्ट्या॒ वृष्ट्या॑ ईश॒ते ऽह्नः॑ । \newline
42. ई॒श॒ते ऽह्नो ऽह्न॑ ईशत ईश॒ते ऽह्न॑ ए॒वैवाह्न॑ ईशत ईश॒ते ऽह्न॑ ए॒व । \newline
43. अह्न॑ ए॒वैवाह्नो ऽह्न॑ ए॒व रू॒पेण॑ रू॒पे णै॒वाह्नो ऽह्न॑ ए॒व रू॒पेण॑ । \newline
44. ए॒व रू॒पेण॑ रू॒पे णै॒वैव रू॒पेण॒ सूर्य॑स्य॒ सूर्य॑स्य रू॒पे णै॒वैव रू॒पेण॒ सूर्य॑स्य । \newline
45. रू॒पेण॒ सूर्य॑स्य॒ सूर्य॑स्य रू॒पेण॑ रू॒पेण॒ सूर्य॑स्य र॒श्मिभी॑ र॒श्मिभिः॒ सूर्य॑स्य रू॒पेण॑ रू॒पेण॒ सूर्य॑स्य र॒श्मिभिः॑ । \newline
46. सूर्य॑स्य र॒श्मिभी॑ र॒श्मिभिः॒ सूर्य॑स्य॒ सूर्य॑स्य र॒श्मिभि॑र् दि॒वो दि॒वो र॒श्मिभिः॒ सूर्य॑स्य॒ सूर्य॑स्य र॒श्मिभि॑र् दि॒वः । \newline
47. र॒श्मिभि॑र् दि॒वो दि॒वो र॒श्मिभी॑ र॒श्मिभि॑र् दि॒वो वृष्टि॒म् ॅवृष्टि॑म् दि॒वो र॒श्मिभी॑ र॒श्मिभि॑र् दि॒वो वृष्टि᳚म् । \newline
48. र॒श्मिभि॒रिति॑ र॒श्मि - भिः॒ । \newline
49. दि॒वो वृष्टि॒म् ॅवृष्टि॑म् दि॒वो दि॒वो वृष्टि॑म् च्यावयति च्यावयति॒ वृष्टि॑म् दि॒वो दि॒वो वृष्टि॑म् च्यावयति । \newline
50. वृष्टि॑म् च्यावयति च्यावयति॒ वृष्टि॒म् ॅवृष्टि॑म् च्यावय॒त्या च्या॑वयति॒ वृष्टि॒म् ॅवृष्टि॑म् च्यावय॒त्या । \newline
51. च्या॒व॒य॒त्या च्या॑वयति च्यावय॒त्या ऽस्मि॑न्, नस्मि॒न्, ना च्या॑वयति च्यावय॒त्या ऽस्मिन्न्॑ । \newline
52. आ ऽस्मि॑न्, नस्मि॒न्, ना ऽस्मि॑न्, नु॒ग्रा उ॒ग्रा अ॑स्मि॒न्, ना ऽस्मि॑न्, नु॒ग्राः । \newline
53. अ॒स्मि॒न्, नु॒ग्रा उ॒ग्रा अ॑स्मिन्, नस्मिन्, नु॒ग्रा अ॑चुच्यवु रचुच्यवु रु॒ग्रा अ॑स्मिन्, नस्मिन्, नु॒ग्रा अ॑चुच्यवुः । \newline
54. उ॒ग्रा अ॑चुच्यवु रचुच्यवु रु॒ग्रा उ॒ग्रा अ॑चुच्यवु॒ रिती त्य॑चुच्यवु रु॒ग्रा उ॒ग्रा अ॑चुच्यवु॒ रिति॑ । \newline
\pagebreak
\markright{ TS 3.3.4.2  \hfill https://www.vedavms.in \hfill}

\section{ TS 3.3.4.2 }

\textbf{TS 3.3.4.2 } \newline
\textbf{Samhita Paata} \newline

अ॑चुच्यवु॒रित्या॑ह यथाय॒जुरे॒वैतत् क॑कु॒हꣳ रू॒पं ॅवृ॑ष॒भस्य॑ रोचते बृ॒हदित्या॑है॒तद्वा अ॑स्य ककु॒हꣳ रू॒पं ॅयद्-वृष्टी॑ रू॒पेणै॒व वृष्टि॒मव॑ रुन्धे॒ यत्ते॑ सो॒मादा᳚भ्यं॒ नाम॒ जागृ॒वीत्या॑है॒ष ह॒ वै ह॒विषा॑ ह॒विर्य॑जति॒ योऽदा᳚भ्यं गृही॒त्वा सोमा॑य जु॒होति॒परा॒ वा ए॒तस्याऽऽ*युः॑ प्रा॒ण ए॑ति॒ - [  ] \newline

\textbf{Pada Paata} \newline

अ॒चु॒च्य॒वुः॒ । इति॑ । आ॒ह॒ । य॒था॒य॒जुरिति॑ यथा-य॒जुः । ए॒व । ए॒तत् । क॒कु॒हम् । रू॒पम् । वृ॒ष॒भस्य॑ । रो॒च॒ते॒ । बृ॒हत् । इति॑ । आ॒ह॒ । ए॒तत् । वै । अ॒स्य॒ । क॒कु॒हम् । रू॒पम् । यत् । वृष्टिः॑ । रू॒पेण॑ । ए॒व । वृष्टि᳚म् । अवेति॑ । रु॒न्धे॒ । यत् । ते॒ । सो॒म॒ । अदा᳚भ्यम् । नाम॑ । जागृ॑वि । इति॑ । आ॒ह॒ । ए॒षः । ह॒ । वै । ह॒विषा᳚ । ह॒विः । य॒ज॒ति॒ । यः । अदा᳚भ्यम् । गृ॒ही॒त्वा । सोमा॑य । जु॒होति॑ । परेति॑ । वै । ए॒तस्य॑ । आयुः॑ । प्रा॒ण इति॑ प्र - अ॒नः । ए॒ति॒ ।  \newline


\textbf{Krama Paata} \newline

अ॒चु॒च्य॒वु॒रिति॑ । इत्या॑ह । आ॒ह॒ य॒था॒य॒जुः । य॒था॒य॒जुरे॒व । य॒था॒य॒जुरिति॑ यथा - य॒जुः । ए॒वैतत् । ए॒तत् क॑कु॒हम् । क॒कु॒हꣳ रू॒पम् । रू॒पं ॅवृ॑ष॒भस्य॑ । वृ॒ष॒भस्य॑ रोचते । रो॒च॒ते॒ बृ॒हत् । बृ॒हदिति॑ । इत्या॑ह । आ॒है॒तत् । ए॒तद् वै । वा अ॑स्य । अ॒स्य॒ क॒कु॒हम् । क॒कु॒हꣳ रू॒पम् । रू॒पं ॅयत् । यद् वृष्टिः॑ । वृष्टी॑ रू॒पेण॑ । रू॒पेणै॒व । ए॒व वृष्टि᳚म् । वृष्टि॒मव॑ । अव॑ रुन्धे । रु॒न्धे॒ यत् । यत् ते᳚ । ते॒ सो॒म॒ । सो॒मादा᳚भ्यम् । अदा᳚भ्य॒म् नाम॑ । नाम॒ जागृ॑वि । जागृ॒वीति॑ । इत्या॑ह । आ॒है॒षः । ए॒ष ह॑ । ह॒ वै । वै ह॒विषा᳚ । ह॒विषा॑ ह॒विः । ह॒विर् य॑जति । य॒ज॒ति॒ यः । यो ऽदा᳚भ्यम् । अदा᳚भ्यम् गृही॒त्वा । गृ॒ही॒त्वा सोमा॑य । सोमा॑य जु॒होति॑ । जु॒होति॒ परा᳚ । परा॒ वै । वा ए॒तस्य॑ । ए॒तस्यायुः॑ । आयुः॑ प्रा॒णः । प्रा॒ण ए॑ति । प्रा॒ण इति॑ प्र - अ॒नः । ए॒ति॒ यः \newline

\textbf{Jatai Paata} \newline

1. अ॒चु॒च्य॒वु॒ रिती त्य॑चुच्यवु रचुच्यवु॒ रिति॑ । \newline
2. इत्या॑हा॒हे तीत्या॑ह । \newline
3. आ॒ह॒ य॒था॒य॒जुर् य॑थाय॒जु रा॑हाह यथाय॒जुः । \newline
4. य॒था॒य॒जु रे॒वैव य॑थाय॒जुर् य॑थाय॒जु रे॒व । \newline
5. य॒था॒य॒जुरिति॑ यथा - य॒जुः । \newline
6. ए॒वैत दे॒त दे॒वैवैतत् । \newline
7. ए॒तत् क॑कु॒हम् क॑कु॒ह मे॒त दे॒तत् क॑कु॒हम् । \newline
8. क॒कु॒हꣳ रू॒पꣳ रू॒पम् क॑कु॒हम् क॑कु॒हꣳ रू॒पम् । \newline
9. रू॒पम् ॅवृ॑ष॒भस्य॑ वृष॒भस्य॑ रू॒पꣳ रू॒पम् ॅवृ॑ष॒भस्य॑ । \newline
10. वृ॒ष॒भस्य॑ रोचते रोचते वृष॒भस्य॑ वृष॒भस्य॑ रोचते । \newline
11. रो॒च॒ते॒ बृ॒हद् बृ॒हद् रो॑चते रोचते बृ॒हत् । \newline
12. बृ॒ह दितीति॑ बृ॒हद् बृ॒ह दिति॑ । \newline
13. इत्या॑हा॒हे तीत्या॑ह । \newline
14. आ॒है॒त दे॒त दा॑हा है॒तत् । \newline
15. ए॒तद् वै वा ए॒त दे॒तद् वै । \newline
16. वा अ॑स्या स्य॒ वै वा अ॑स्य । \newline
17. अ॒स्य॒ क॒कु॒हम् क॑कु॒ह म॑स्या स्य ककु॒हम् । \newline
18. क॒कु॒हꣳ रू॒पꣳ रू॒पम् क॑कु॒हम् क॑कु॒हꣳ रू॒पम् । \newline
19. रू॒पम् ॅयद् यद् रू॒पꣳ रू॒पम् ॅयत् । \newline
20. यद् वृष्टि॒र् वृष्टि॒र् यद् यद् वृष्टिः॑ । \newline
21. वृष्टी॑ रू॒पेण॑ रू॒पेण॒ वृष्टि॒र् वृष्टी॑ रू॒पेण॑ । \newline
22. रू॒पे णै॒वैव रू॒पेण॑ रू॒पेणै॒व । \newline
23. ए॒व वृष्टि॒म् ॅवृष्टि॑ मे॒वैव वृष्टि᳚म् । \newline
24. वृष्टि॒ मवाव॒ वृष्टि॒म् ॅवृष्टि॒ मव॑ । \newline
25. अव॑ रुन्धे रु॒न्धे ऽवाव॑ रुन्धे । \newline
26. रु॒न्धे॒ यद् यद् रु॑न्धे रुन्धे॒ यत् । \newline
27. यत् ते॑ ते॒ यद् यत् ते᳚ । \newline
28. ते॒ सो॒म॒ सो॒म॒ ते॒ ते॒ सो॒म॒ । \newline
29. सो॒मा दा᳚भ्य॒ मदा᳚भ्यꣳ सोम सो॒मा दा᳚भ्यम् । \newline
30. अदा᳚भ्य॒म् नाम॒ नामा दा᳚भ्य॒ मदा᳚भ्य॒म् नाम॑ । \newline
31. नाम॒ जागृ॑वि॒ जागृ॑वि॒ नाम॒ नाम॒ जागृ॑वि । \newline
32. जागृ॒वी तीति॒ जागृ॑वि॒ जागृ॒वीति॑ । \newline
33. इत्या॑हा॒हे तीत्या॑ह । \newline
34. आ॒है॒ष ए॒ष आ॑हा है॒षः । \newline
35. ए॒ष ह॑ है॒ष ए॒ष ह॑ । \newline
36. ह॒ वै वै ह॑ ह॒ वै । \newline
37. वै ह॒विषा॑ ह॒विषा॒ वै वै ह॒विषा᳚ । \newline
38. ह॒विषा॑ ह॒विर्. ह॒विर्. ह॒विषा॑ ह॒विषा॑ ह॒विः । \newline
39. ह॒विर् य॑जति यजति ह॒विर्. ह॒विर् य॑जति । \newline
40. य॒ज॒ति॒ यो यो य॑जति यजति॒ यः । \newline
41. यो ऽदा᳚भ्य॒ मदा᳚भ्य॒म् ॅयो यो ऽदा᳚भ्यम् । \newline
42. अदा᳚भ्यम् गृही॒त्वा गृ॑ही॒त्वा ऽदा᳚भ्य॒ मदा᳚भ्यम् गृही॒त्वा । \newline
43. गृ॒ही॒त्वा सोमा॑य॒ सोमा॑य गृही॒त्वा गृ॑ही॒त्वा सोमा॑य । \newline
44. सोमा॑य जु॒होति॑ जु॒होति॒ सोमा॑य॒ सोमा॑य जु॒होति॑ । \newline
45. जु॒होति॒ परा॒ परा॑ जु॒होति॑ जु॒होति॒ परा᳚ । \newline
46. परा॒ वै वै परा॒ परा॒ वै । \newline
47. वा ए॒त स्यै॒तस्य॒ वै वा ए॒तस्य॑ । \newline
48. ए॒त स्यायु॒ रायु॑ रे॒त स्यै॒त स्यायुः॑ । \newline
49. आयुः॑ प्रा॒णः प्रा॒ण आयु॒ रायुः॑ प्रा॒णः । \newline
50. प्रा॒ण ए᳚त्येति प्रा॒णः प्रा॒ण ए॑ति । \newline
51. प्रा॒ण इति॑ प्र - अ॒नः । \newline
52. ए॒ति॒ यो य ए᳚त्येति॒ यः । \newline

\textbf{Ghana Paata } \newline

1. अ॒चु॒च्य॒वु॒ रिती त्य॑चुच्यवु रचुच्यवु॒ रित्या॑हा॒हे त्य॑चुच्यवु रचुच्यवु॒ रित्या॑ह । \newline
2. इत्या॑हा॒हे तीत्या॑ह यथाय॒जुर् य॑थाय॒जु रा॒हे तीत्या॑ह यथाय॒जुः । \newline
3. आ॒ह॒ य॒था॒य॒जुर् य॑थाय॒जु रा॑हाह यथाय॒जु रे॒वैव य॑थाय॒जु रा॑हाह यथाय॒जु रे॒व । \newline
4. य॒था॒य॒जु रे॒वैव य॑थाय॒जुर् य॑थाय॒जु रे॒वैत दे॒त दे॒व य॑थाय॒जुर् य॑थाय॒जु रे॒वैतत् । \newline
5. य॒था॒य॒जुरिति॑ यथा - य॒जुः । \newline
6. ए॒वैत दे॒त दे॒वैवैतत् क॑कु॒हम् क॑कु॒ह मे॒त दे॒वैवैतत् क॑कु॒हम् । \newline
7. ए॒तत् क॑कु॒हम् क॑कु॒ह मे॒त दे॒तत् क॑कु॒हꣳ रू॒पꣳ रू॒पम् क॑कु॒ह मे॒त दे॒तत् क॑कु॒हꣳ रू॒पम् । \newline
8. क॒कु॒हꣳ रू॒पꣳ रू॒पम् क॑कु॒हम् क॑कु॒हꣳ रू॒पम् ॅवृ॑ष॒भस्य॑ वृष॒भस्य॑ रू॒पम् क॑कु॒हम् क॑कु॒हꣳ रू॒पम् ॅवृ॑ष॒भस्य॑ । \newline
9. रू॒पम् ॅवृ॑ष॒भस्य॑ वृष॒भस्य॑ रू॒पꣳ रू॒पम् ॅवृ॑ष॒भस्य॑ रोचते रोचते वृष॒भस्य॑ रू॒पꣳ रू॒पम् ॅवृ॑ष॒भस्य॑ रोचते । \newline
10. वृ॒ष॒भस्य॑ रोचते रोचते वृष॒भस्य॑ वृष॒भस्य॑ रोचते बृ॒हद् बृ॒हद् रो॑चते वृष॒भस्य॑ वृष॒भस्य॑ रोचते बृ॒हत् । \newline
11. रो॒च॒ते॒ बृ॒हद् बृ॒हद् रो॑चते रोचते बृ॒ह दितीति॑ बृ॒हद् रो॑चते रोचते बृ॒हदिति॑ । \newline
12. बृ॒ह दितीति॑ बृ॒हद् बृ॒ह दित्या॑हा॒हे ति॑ बृ॒हद् बृ॒ह दित्या॑ह । \newline
13. इत्या॑हा॒हे तीत्या॑ है॒त दे॒त दा॒हे तीत्या॑ है॒तत् । \newline
14. आ॒है॒त दे॒त दा॑हा है॒तद् वै वा ए॒त दा॑हा है॒तद् वै । \newline
15. ए॒तद् वै वा ए॒त दे॒तद् वा अ॑स्या स्य॒ वा ए॒त दे॒तद् वा अ॑स्य । \newline
16. वा अ॑स्या स्य॒ वै वा अ॑स्य ककु॒हम् क॑कु॒ह म॑स्य॒ वै वा अ॑स्य ककु॒हम् । \newline
17. अ॒स्य॒ क॒कु॒हम् क॑कु॒ह म॑स्यास्य ककु॒हꣳ रू॒पꣳ रू॒पम् क॑कु॒ह म॑स्यास्य ककु॒हꣳ रू॒पम् । \newline
18. क॒कु॒हꣳ रू॒पꣳ रू॒पम् क॑कु॒हम् क॑कु॒हꣳ रू॒पम् ॅयद् यद् रू॒पम् क॑कु॒हम् क॑कु॒हꣳ रू॒पम् ॅयत् । \newline
19. रू॒पम् ॅयद् यद् रू॒पꣳ रू॒पम् ॅयद् वृष्टि॒र् वृष्टि॒र् यद् रू॒पꣳ रू॒पम् ॅयद् वृष्टिः॑ । \newline
20. यद् वृष्टि॒र् वृष्टि॒र् यद् यद् वृष्टी॑ रू॒पेण॑ रू॒पेण॒ वृष्टि॒र् यद् यद् वृष्टी॑ रू॒पेण॑ । \newline
21. वृष्टी॑ रू॒पेण॑ रू॒पेण॒ वृष्टि॒र् वृष्टी॑ रू॒पे णै॒वैव रू॒पेण॒ वृष्टि॒र् वृष्टी॑ रू॒पेणै॒व । \newline
22. रू॒पे णै॒वैव रू॒पेण॑ रू॒पेणै॒व वृष्टि॒म् ॅवृष्टि॑ मे॒व रू॒पेण॑ रू॒पेणै॒व वृष्टि᳚म् । \newline
23. ए॒व वृष्टि॒म् ॅवृष्टि॑ मे॒वैव वृष्टि॒ मवाव॒ वृष्टि॑ मे॒वैव वृष्टि॒ मव॑ । \newline
24. वृष्टि॒ मवाव॒ वृष्टि॒म् ॅवृष्टि॒ मव॑ रुन्धे रु॒न्धे ऽव॒ वृष्टि॒म् ॅवृष्टि॒ मव॑ रुन्धे । \newline
25. अव॑ रुन्धे रु॒न्धे ऽवाव॑ रुन्धे॒ यद् यद् रु॒न्धे ऽवाव॑ रुन्धे॒ यत् । \newline
26. रु॒न्धे॒ यद् यद् रु॑न्धे रुन्धे॒ यत् ते॑ ते॒ यद् रु॑न्धे रुन्धे॒ यत् ते᳚ । \newline
27. यत् ते॑ ते॒ यद् यत् ते॑ सोम सोम ते॒ यद् यत् ते॑ सोम । \newline
28. ते॒ सो॒म॒ सो॒म॒ ते॒ ते॒ सो॒मा दा᳚भ्य॒ मदा᳚भ्यꣳ सोम ते ते सो॒मा दा᳚भ्यम् । \newline
29. सो॒मादा᳚भ्य॒ मदा᳚भ्यꣳ सोम सो॒मा दा᳚भ्य॒म् नाम॒ नामा दा᳚भ्यꣳ सोम सो॒मादा᳚भ्य॒म् नाम॑ । \newline
30. अदा᳚भ्य॒म् नाम॒ नामा दा᳚भ्य॒ मदा᳚भ्य॒म् नाम॒ जागृ॑वि॒ जागृ॑वि॒ नामा दा᳚भ्य॒ मदा᳚भ्य॒म् नाम॒ जागृ॑वि । \newline
31. नाम॒ जागृ॑वि॒ जागृ॑वि॒ नाम॒ नाम॒ जागृ॒वीतीति॒ जागृ॑वि॒ नाम॒ नाम॒ जागृ॒वीति॑ । \newline
32. जागृ॒वीतीति॒ जागृ॑वि॒ जागृ॒वी त्या॑हा॒हे ति॒ जागृ॑वि॒ जागृ॒वी त्या॑ह । \newline
33. इत्या॑हा॒हे तीत्या॑ है॒ष ए॒ष आ॒हे तीत्या॑है॒षः । \newline
34. आ॒है॒ष ए॒ष आ॑हा है॒ष ह॑ है॒ष आ॑हा है॒ष ह॑ । \newline
35. ए॒ष ह॑ है॒ष ए॒ष ह॒ वै वै है॒ष ए॒ष ह॒ वै । \newline
36. ह॒ वै वै ह॑ ह॒ वै ह॒विषा॑ ह॒विषा॒ वै ह॑ ह॒ वै ह॒विषा᳚ । \newline
37. वै ह॒विषा॑ ह॒विषा॒ वै वै ह॒विषा॑ ह॒विर्. ह॒विर्. ह॒विषा॒ वै वै ह॒विषा॑ ह॒विः । \newline
38. ह॒विषा॑ ह॒विर्. ह॒विर्. ह॒विषा॑ ह॒विषा॑ ह॒विर् य॑जति यजति ह॒विर्. ह॒विषा॑ ह॒विषा॑ ह॒विर् य॑जति । \newline
39. ह॒विर् य॑जति यजति ह॒विर्. ह॒विर् य॑जति॒ यो यो य॑जति ह॒विर्. ह॒विर् य॑जति॒ यः । \newline
40. य॒ज॒ति॒ यो यो य॑जति यजति॒ यो ऽदा᳚भ्य॒ मदा᳚भ्य॒म् ॅयो य॑जति यजति॒ यो ऽदा᳚भ्यम् । \newline
41. यो ऽदा᳚भ्य॒ मदा᳚भ्य॒म् ॅयो यो ऽदा᳚भ्यम् गृही॒त्वा गृ॑ही॒त्वा ऽदा᳚भ्य॒म् ॅयो यो ऽदा᳚भ्यम् गृही॒त्वा । \newline
42. अदा᳚भ्यम् गृही॒त्वा गृ॑ही॒त्वा ऽदा᳚भ्य॒ मदा᳚भ्यम् गृही॒त्वा सोमा॑य॒ सोमा॑य गृही॒त्वा ऽदा᳚भ्य॒ मदा᳚भ्यम् गृही॒त्वा सोमा॑य । \newline
43. गृ॒ही॒त्वा सोमा॑य॒ सोमा॑य गृही॒त्वा गृ॑ही॒त्वा सोमा॑य जु॒होति॑ जु॒होति॒ सोमा॑य गृही॒त्वा गृ॑ही॒त्वा सोमा॑य जु॒होति॑ । \newline
44. सोमा॑य जु॒होति॑ जु॒होति॒ सोमा॑य॒ सोमा॑य जु॒होति॒ परा॒ परा॑ जु॒होति॒ सोमा॑य॒ सोमा॑य जु॒होति॒ परा᳚ । \newline
45. जु॒होति॒ परा॒ परा॑ जु॒होति॑ जु॒होति॒ परा॒ वै वै परा॑ जु॒होति॑ जु॒होति॒ परा॒ वै । \newline
46. परा॒ वै वै परा॒ परा॒ वा ए॒त स्यै॒तस्य॒ वै परा॒ परा॒ वा ए॒तस्य॑ । \newline
47. वा ए॒त स्यै॒तस्य॒ वै वा ए॒त स्यायु॒ रायु॑ रे॒तस्य॒ वै वा ए॒त स्यायुः॑ । \newline
48. ए॒त स्यायु॒ रायु॑ रे॒त स्यै॒त स्यायुः॑ प्रा॒णः प्रा॒ण आयु॑ रे॒त स्यै॒त स्यायुः॑ प्रा॒णः । \newline
49. आयुः॑ प्रा॒णः प्रा॒ण आयु॒ रायुः॑ प्रा॒ण ए᳚त्येति प्रा॒ण आयु॒ रायुः॑ प्रा॒ण ए॑ति । \newline
50. प्रा॒ण ए᳚त्येति प्रा॒णः प्रा॒ण ए॑ति॒ यो य ए॑ति प्रा॒णः प्रा॒ण ए॑ति॒ यः । \newline
51. प्रा॒ण इति॑ प्र - अ॒नः । \newline
52. ए॒ति॒ यो य ए᳚त्येति॒ यो ऽꣳ॑शु मꣳ॒॒शुम् ॅय ए᳚त्येति॒ यो ऽꣳ॑शुम् । \newline
\pagebreak
\markright{ TS 3.3.4.3  \hfill https://www.vedavms.in \hfill}

\section{ TS 3.3.4.3 }

\textbf{TS 3.3.4.3 } \newline
\textbf{Samhita Paata} \newline

योऽꣳ॑शुं गृ॒ह्णात्या नः॑ प्रा॒ण ए॑तु परा॒वत॒ इत्या॒हाऽऽ*यु॑रे॒व प्रा॒णमा॒त्मन् ध॑त्ते॒ ऽमृत॑मसि प्रा॒णाय॒ त्वेति॒ हिर॑ण्यम॒भि व्य॑नित्य॒मृतं॒ ॅवै हिर॑ण्य॒मायुः॑ प्रा॒णो॑ऽमृते॑नै॒वाऽऽ*यु॑रा॒त्मन् ध॑त्ते श॒तमा॑नं भवति श॒तायुः॒ पुरु॑षः श॒तेन्द्रि॑य॒ आयु॑ष्ये॒वेन्द्रि॒ये प्रति॑तिष्ठत्य॒प उप॑ स्पृशति भेष॒जं ॅवा आपो॑ ( ) भेष॒जमे॒व कु॑रुते ॥ \newline

\textbf{Pada Paata} \newline

यः । अꣳ॒॒शुम् । गृ॒ह्णाति॑ । एति॑ । नः॒ । प्रा॒ण इति॑ प्र - अ॒नः । ए॒तु॒ । प॒रा॒वत॒ इति॑ परा - वतः॑ । इति॑ । आ॒ह॒ । आयुः॑ । ए॒व । प्रा॒णमिति॑ प्र - अ॒नम् । आ॒त्मन्न् । ध॒त्ते॒ । अ॒मृत᳚म् । अ॒सि॒ । प्रा॒णायेति॑ प्र - अ॒नाय॑ । त्वा॒ । इति॑ । हिर॑ण्यम् । अ॒भि । वीति॑ । अ॒नि॒ति॒ । अ॒मृत᳚म् । वै । हिर॑ण्यम् । आयुः॑ । प्रा॒ण इति॑ प्र - अ॒नः । अ॒मृते॑न । ए॒व । आयुः॑ । आ॒त्मन्न् । ध॒त्ते॒ । श॒तमा॑न॒मिति॑ श॒त - मा॒न॒म् । भ॒व॒ति॒ । श॒तायु॒रिति॑ श॒त - आ॒युः॒ । पुरु॑षः । श॒तेन्द्रि॑य॒ इति॑ श॒त - इ॒न्द्रि॒यः॒ । आयु॑षि । ए॒व । इ॒न्द्रि॒ये । प्रतीति॑ । ति॒ष्ठ॒ति॒ । अ॒पः । उपेति॑ । स्पृ॒श॒ति॒ । भे॒ष॒जम् । वै । आपः॑ ( ) । भे॒ष॒जम् । ए॒व । कु॒रु॒ते॒ ॥  \newline


\textbf{Krama Paata} \newline

यो ऽꣳ॑शुम् । अꣳ॒॒शुम् गृ॒ह्णाति॑ । गृ॒ह्णात्या । आ नः॑ । नः॒ प्रा॒णः । प्रा॒ण ए॑तु । प्रा॒ण इति॑ प्र - अ॒नः । ए॒तु॒ प॒रा॒वतः॑ । प॒रा॒वत॒ इति॑ । प॒रा॒वत॒ इति॑ परा - वतः॑ । इत्या॑ह । आ॒हायुः॑ । आयु॑रे॒व । ए॒व प्रा॒णम् । प्रा॒णमा॒त्मन्न् । प्रा॒णमिति॑ प्र - अ॒नम् । आ॒त्मन् ध॑त्ते । ध॒त्ते॒ ऽमृत᳚म् । अ॒मृत॑मसि । अ॒सि॒ प्रा॒णाय॑ । प्रा॒णाय॑ त्वा । प्रा॒णायेति॑ प्र - अ॒नाय॑ । त्वेति॑ । इति॒ हिर॑ण्यम् । हिर॑ण्यम॒भि । अ॒भि वि । व्य॑निति । अ॒नि॒त्य॒मृत᳚म् । अ॒मृतं॒ ॅवै । वै हिर॑ण्यम् । हिर॑ण्य॒मायुः॑ । आयुः॑ प्रा॒णः । प्रा॒णो॑ ऽमृते॑न । प्रा॒ण इति॑ प्र - अ॒नः । अ॒मृते॑नै॒व । ए॒वायुः॑ । आयु॑र् आ॒त्मन्न् । आ॒त्मन् ध॑त्ते । ध॒त्ते॒ श॒तमा॑नम् । श॒तमा॑नम् भवति । श॒तमा॑न॒मिति॑ श॒त - मा॒न॒म् । भ॒व॒ति॒ श॒तायुः॑ । श॒तायुः॒ पुरु॑षः । श॒तायु॒रिति॑ श॒त - आ॒युः॒ । पुरु॑षः श॒तेन्द्रि॑यः । श॒तेन्द्रि॑य॒ आयु॑षि । श॒तेन्द्रि॑य॒ इति॑ श॒त - इ॒न्द्रि॒यः॒ । आयु॑ष्ये॒व । ए॒वेन्द्रि॒ये । इ॒न्द्रि॒ये प्रति॑ । प्रति॑ तिष्ठति । ति॒ष्ठ॒त्य॒पः । अ॒प उप॑ । उप॑ स्पृशति । स्पृ॒श॒ति॒ भे॒ष॒जम् । भे॒ष॒जं ॅवै । वा आपः॑ ( ) । आपो॑ भेष॒जम् । भे॒ष॒जमे॒व । ए॒व कु॑रुते । कु॒रु॒त॒ इति॑ कुरुते । \newline

\textbf{Jatai Paata} \newline

1. यो ऽꣳ॑शु मꣳ॒॒शुम् ॅयो यो ꣳ॑शुम् । \newline
2. अꣳ॒॒शुम् गृ॒ह्णाति॑ गृ॒ह्णा त्यꣳ॒॒शु मꣳ॒॒शुम् गृ॒ह्णाति॑ । \newline
3. गृ॒ह्णात्या गृ॒ह्णाति॑ गृ॒ह्णात्या । \newline
4. आ नो॑ न॒ आ नः॑ । \newline
5. नः॒ प्रा॒णः प्रा॒णो नो॑ नः प्रा॒णः । \newline
6. प्रा॒ण ए᳚त्वेतु प्रा॒णः प्रा॒ण ए॑तु । \newline
7. प्रा॒ण इति॑ प्र - अ॒नः । \newline
8. ए॒तु॒ प॒रा॒वतः॑ परा॒वत॑ एत्वेतु परा॒वतः॑ । \newline
9. प॒रा॒वत॒ इतीति॑ परा॒वतः॑ परा॒वत॒ इति॑ । \newline
10. प॒रा॒वत॒ इति॑ परा - वतः॑ । \newline
11. इत्या॑हा॒हे तीत्या॑ह । \newline
12. आ॒हायु॒ रायु॑ राहा॒हायुः॑ । \newline
13. आयु॑ रे॒वैवायु॒ रायु॑ रे॒व । \newline
14. ए॒व प्रा॒णम् प्रा॒ण मे॒वैव प्रा॒णम् । \newline
15. प्रा॒ण मा॒त्मन् ना॒त्मन् प्रा॒णम् प्रा॒ण मा॒त्मन्न् । \newline
16. प्रा॒णमिति॑ प्र - अ॒नम् । \newline
17. आ॒त्मन् ध॑त्ते धत्त आ॒त्मन् ना॒त्मन् ध॑त्ते । \newline
18. ध॒त्ते॒ ऽमृत॑ म॒मृत॑म् धत्ते धत्ते॒ ऽमृत᳚म् । \newline
19. अ॒मृत॑ मस्य स्य॒मृत॑ म॒मृत॑ मसि । \newline
20. अ॒सि॒ प्रा॒णाय॑ प्रा॒णाया᳚ स्यसि प्रा॒णाय॑ । \newline
21. प्रा॒णाय॑ त्वा त्वा प्रा॒णाय॑ प्रा॒णाय॑ त्वा । \newline
22. प्रा॒णायेति॑ प्र - अ॒नाय॑ । \newline
23. त्वेतीति॑ त्वा॒ त्वेति॑ । \newline
24. इति॒ हिर॑ण्यꣳ॒॒ हिर॑ण्य॒ मितीति॒ हिर॑ण्यम् । \newline
25. हिर॑ण्य म॒भ्य॑भि हिर॑ण्यꣳ॒॒ हिर॑ण्य म॒भि । \newline
26. अ॒भि वि व्या᳚(1॒)भ्य॑भि वि । \newline
27. व्य॑नि त्यनिति॒ वि व्य॑निति । \newline
28. अ॒नि॒ त्य॒मृत॑ म॒मृत॑ मनि त्यनि त्य॒मृत᳚म् । \newline
29. अ॒मृत॒म् ॅवै वा अ॒मृत॑ म॒मृत॒म् ॅवै । \newline
30. वै हिर॑ण्यꣳ॒॒ हिर॑ण्य॒म् ॅवै वै हिर॑ण्यम् । \newline
31. हिर॑ण्य॒ मायु॒ रायु॒र्॒. हिर॑ण्यꣳ॒॒ हिर॑ण्य॒ मायुः॑ । \newline
32. आयुः॑ प्रा॒णः प्रा॒ण आयु॒ रायुः॑ प्रा॒णः । \newline
33. प्रा॒णो॑ ऽमृते॑ना॒ मृते॑न प्रा॒णः प्रा॒णो॑ ऽमृते॑न । \newline
34. प्रा॒ण इति॑ प्र - अ॒नः । \newline
35. अ॒मृते॑ नै॒वैवा मृते॑ना॒ मृते॑ नै॒व । \newline
36. ए॒वायु॒ रायु॑ रे॒वै वायुः॑ । \newline
37. आयु॑ रा॒त्मन् ना॒त्मन् नायु॒ रायु॑ रा॒त्मन्न् । \newline
38. आ॒त्मन् ध॑त्ते धत्त आ॒त्मन् ना॒त्मन् ध॑त्ते । \newline
39. ध॒त्ते॒ श॒तमा॑नꣳ श॒तमा॑नम् धत्ते धत्ते श॒तमा॑नम् । \newline
40. श॒तमा॑नम् भवति भवति श॒तमा॑नꣳ श॒तमा॑नम् भवति । \newline
41. श॒तमा॑न॒मिति॑ श॒त - मा॒न॒म् । \newline
42. भ॒व॒ति॒ श॒तायुः॑ श॒तायु॑र् भवति भवति श॒तायुः॑ । \newline
43. श॒तायुः॒ पुरु॑षः॒ पुरु॑षः श॒तायुः॑ श॒तायुः॒ पुरु॑षः । \newline
44. श॒तायु॒रिति॑ श॒त - आ॒युः॒ । \newline
45. पुरु॑षः श॒तेन्द्रि॑यः श॒तेन्द्रि॑यः॒ पुरु॑षः॒ पुरु॑षः श॒तेन्द्रि॑यः । \newline
46. श॒तेन्द्रि॑य॒ आयु॒ ष्यायु॑षि श॒तेन्द्रि॑यः श॒तेन्द्रि॑य॒ आयु॑षि । \newline
47. श॒तेन्द्रि॑य॒ इति॑ श॒त - इ॒न्द्रि॒यः॒ । \newline
48. आयु॑ ष्ये॒वैवायु॒ ष्यायु॑ ष्ये॒व । \newline
49. ए॒वेन्द्रि॒य इ॑न्द्रि॒य ए॒वै वेन्द्रि॒ये । \newline
50. इ॒न्द्रि॒ये प्रति॒ प्रती᳚न्द्रि॒य इ॑न्द्रि॒ये प्रति॑ । \newline
51. प्रति॑ तिष्ठति तिष्ठति॒ प्रति॒ प्रति॑ तिष्ठति । \newline
52. ति॒ष्ठ॒ त्य॒पो॑ ऽपस्ति॑ष्ठति तिष्ठ त्य॒पः । \newline
53. अ॒प उपोपा॒पो॑ ऽप उप॑ । \newline
54. उप॑ स्पृशति स्पृश॒ त्युपोप॑ स्पृशति । \newline
55. स्पृ॒श॒ति॒ भे॒ष॒जम् भे॑ष॒जꣳ स्पृ॑शति स्पृशति भेष॒जम् । \newline
56. भे॒ष॒जम् ॅवै वै भे॑ष॒जम् भे॑ष॒जम् ॅवै । \newline
57. वा आप॒ आपो॒ वै वा आपः॑ । \newline
58. आपो॑ भेष॒जम् भे॑ष॒ज माप॒ आपो॑ भेष॒जम् । \newline
59. भे॒ष॒ज मे॒वैव भे॑ष॒जम् भे॑ष॒ज मे॒व । \newline
60. ए॒व कु॑रुते कुरुत ए॒वैव कु॑रुते । \newline
61. कु॒रु॒त॒ इति॑ कुरुते । \newline

\textbf{Ghana Paata } \newline

1. यो ꣳ॑शु मꣳ॒॒शुम् ॅयो यो ऽꣳ॑शुम् गृ॒ह्णाति॑ गृ॒ह्णा त्यꣳ॒॒शुम् ॅयो यो ऽꣳ॑शुम् गृ॒ह्णाति॑ । \newline
2. अꣳ॒॒शुम् गृ॒ह्णाति॑ गृ॒ह्णा त्यꣳ॒॒शु मꣳ॒॒शुम् गृ॒ह्णात्या गृ॒ह्णा त्यꣳ॒॒शु मꣳ॒॒शुम् गृ॒ह्णात्या । \newline
3. गृ॒ह्णात्या गृ॒ह्णाति॑ गृ॒ह्णात्या नो॑ न॒ आ गृ॒ह्णाति॑ गृ॒ह्णात्या नः॑ । \newline
4. आ नो॑ न॒ आ नः॑ प्रा॒णः प्रा॒णो न॒ आ नः॑ प्रा॒णः । \newline
5. नः॒ प्रा॒णः प्रा॒णो नो॑ नः प्रा॒ण ए᳚त्वेतु प्रा॒णो नो॑ नः प्रा॒ण ए॑तु । \newline
6. प्रा॒ण ए᳚त्वेतु प्रा॒णः प्रा॒ण ए॑तु परा॒वतः॑ परा॒वत॑ एतु प्रा॒णः प्रा॒ण ए॑तु परा॒वतः॑ । \newline
7. प्रा॒ण इति॑ प्र - अ॒नः । \newline
8. ए॒तु॒ प॒रा॒वतः॑ परा॒वत॑ एत्वेतु परा॒वत॒ इतीति॑ परा॒वत॑ एत्वेतु परा॒वत॒ इति॑ । \newline
9. प॒रा॒वत॒ इतीति॑ परा॒वतः॑ परा॒वत॒ इत्या॑हा॒हे ति॑ परा॒वतः॑ परा॒वत॒ इत्या॑ह । \newline
10. प॒रा॒वत॒ इति॑ परा - वतः॑ । \newline
11. इत्या॑हा॒हे तीत्या॒हायु॒ रायु॑ रा॒हे तीत्या॒हायुः॑ । \newline
12. आ॒हायु॒ रायु॑ राहा॒हायु॑ रे॒वैवायु॑ राहा॒हायु॑ रे॒व । \newline
13. आयु॑ रे॒वैवायु॒ रायु॑ रे॒व प्रा॒णम् प्रा॒ण मे॒वायु॒ रायु॑ रे॒व प्रा॒णम् । \newline
14. ए॒व प्रा॒णम् प्रा॒ण मे॒वैव प्रा॒ण मा॒त्मन्, ना॒त्मन् प्रा॒ण मे॒वैव प्रा॒ण मा॒त्मन्न् । \newline
15. प्रा॒ण मा॒त्मन्, ना॒त्मन् प्रा॒णम् प्रा॒ण मा॒त्मन् ध॑त्ते धत्त आ॒त्मन् प्रा॒णम् प्रा॒ण मा॒त्मन् ध॑त्ते । \newline
16. प्रा॒णमिति॑ प्र - अ॒नम् । \newline
17. आ॒त्मन् ध॑त्ते धत्त आ॒त्मन्, ना॒त्मन् ध॑त्ते॒ ऽमृत॑ म॒मृत॑म् धत्त आ॒त्मन्, ना॒त्मन् ध॑त्ते॒ ऽमृत᳚म् । \newline
18. ध॒त्ते॒ ऽमृत॑ म॒मृत॑म् धत्ते धत्ते॒ ऽमृत॑ मस्य स्य॒मृत॑म् धत्ते धत्ते॒ ऽमृत॑ मसि । \newline
19. अ॒मृत॑ मस्य स्य॒मृत॑ म॒मृत॑ मसि प्रा॒णाय॑ प्रा॒णाया᳚ स्य॒मृत॑ म॒मृत॑ मसि प्रा॒णाय॑ । \newline
20. अ॒सि॒ प्रा॒णाय॑ प्रा॒णाया᳚ स्यसि प्रा॒णाय॑ त्वा त्वा प्रा॒णाया᳚ स्यसि प्रा॒णाय॑ त्वा । \newline
21. प्रा॒णाय॑ त्वा त्वा प्रा॒णाय॑ प्रा॒णाय॒ त्वेतीति॑ त्वा प्रा॒णाय॑ प्रा॒णाय॒ त्वेति॑ । \newline
22. प्रा॒णायेति॑ प्र - अ॒नाय॑ । \newline
23. त्वेतीति॑ त्वा॒ त्वेति॒ हिर॑ण्यꣳ॒॒ हिर॑ण्य॒ मिति॑ त्वा॒ त्वेति॒ हिर॑ण्यम् । \newline
24. इति॒ हिर॑ण्यꣳ॒॒ हिर॑ण्य॒ मितीति॒ हिर॑ण्य म॒भ्य॑भि हिर॑ण्य॒ मितीति॒ हिर॑ण्य म॒भि । \newline
25. हिर॑ण्य म॒भ्य॑भि हिर॑ण्यꣳ॒॒ हिर॑ण्य म॒भि वि व्य॑भि हिर॑ण्यꣳ॒॒ हिर॑ण्य म॒भि वि । \newline
26. अ॒भि वि व्या᳚(1॒)भ्य॑भि व्य॑नि त्यनिति॒ व्या᳚(1॒)भ्य॑भि व्य॑निति । \newline
27. व्य॑नि त्यनिति॒ वि व्य॑नि त्य॒मृत॑ म॒मृत॑ मनिति॒ वि व्य॑नि त्य॒मृत᳚म् । \newline
28. अ॒नि॒ त्य॒मृत॑ म॒मृत॑ मनि त्यनि त्य॒मृत॒म् ॅवै वा अ॒मृत॑ मनि त्यनि त्य॒मृत॒म् ॅवै । \newline
29. अ॒मृत॒म् ॅवै वा अ॒मृत॑ म॒मृत॒म् ॅवै हिर॑ण्यꣳ॒॒ हिर॑ण्य॒म् ॅवा अ॒मृत॑ म॒मृत॒म् ॅवै हिर॑ण्यम् । \newline
30. वै हिर॑ण्यꣳ॒॒ हिर॑ण्य॒म् ॅवै वै हिर॑ण्य॒ मायु॒ रायु॒र्॒. हिर॑ण्य॒म् ॅवै वै हिर॑ण्य॒ मायुः॑ । \newline
31. हिर॑ण्य॒ मायु॒ रायु॒र्॒. हिर॑ण्यꣳ॒॒ हिर॑ण्य॒ मायुः॑ प्रा॒णः प्रा॒ण आयु॒र्॒. हिर॑ण्यꣳ॒॒ हिर॑ण्य॒ मायुः॑ प्रा॒णः । \newline
32. आयुः॑ प्रा॒णः प्रा॒ण आयु॒ रायुः॑ प्रा॒णो॑ ऽमृते॑ना॒ मृते॑न प्रा॒ण आयु॒ रायुः॑ प्रा॒णो॑ ऽमृते॑न । \newline
33. प्रा॒णो॑ ऽमृते॑ना॒ मृते॑न प्रा॒णः प्रा॒णो॑ ऽमृते॑ नै॒वैवा मृते॑न प्रा॒णः प्रा॒णो॑ ऽमृते॑नै॒व । \newline
34. प्रा॒ण इति॑ प्र - अ॒नः । \newline
35. अ॒मृते॑ नै॒वैवा मृते॑ना॒ मृते॑ नै॒वायु॒ रायु॑ रे॒वा मृते॑ना॒ मृते॑ नै॒वायुः॑ । \newline
36. ए॒वायु॒ रायु॑ रे॒वैवायु॑ रा॒त्मन्, ना॒त्मन्, नायु॑ रे॒वैवायु॑ रा॒त्मन्न् । \newline
37. आयु॑ रा॒त्मन्, ना॒त्मन्, नायु॒ रायु॑ रा॒त्मन् ध॑त्ते धत्त आ॒त्मन्, नायु॒ रायु॑ रा॒त्मन् ध॑त्ते । \newline
38. आ॒त्मन् ध॑त्ते धत्त आ॒त्मन्, ना॒त्मन् ध॑त्ते श॒तमा॑नꣳ श॒तमा॑नम् धत्त आ॒त्मन्, ना॒त्मन् ध॑त्ते श॒तमा॑नम् । \newline
39. ध॒त्ते॒ श॒तमा॑नꣳ श॒तमा॑नम् धत्ते धत्ते श॒तमा॑नम् भवति भवति श॒तमा॑नम् धत्ते धत्ते श॒तमा॑नम् भवति । \newline
40. श॒तमा॑नम् भवति भवति श॒तमा॑नꣳ श॒तमा॑नम् भवति श॒तायुः॑ श॒तायु॑र् भवति श॒तमा॑नꣳ श॒तमा॑नम् भवति श॒तायुः॑ । \newline
41. श॒तमा॑न॒मिति॑ श॒त - मा॒न॒म् । \newline
42. भ॒व॒ति॒ श॒तायुः॑ श॒तायु॑र् भवति भवति श॒तायुः॒ पुरु॑षः॒ पुरु॑षः श॒तायु॑र् भवति भवति श॒तायुः॒ पुरु॑षः । \newline
43. श॒तायुः॒ पुरु॑षः॒ पुरु॑षः श॒तायुः॑ श॒तायुः॒ पुरु॑षः श॒तेन्द्रि॑यः श॒तेन्द्रि॑यः॒ पुरु॑षः श॒तायुः॑ श॒तायुः॒ पुरु॑षः श॒तेन्द्रि॑यः । \newline
44. श॒तायु॒रिति॑ श॒त - आ॒युः॒ । \newline
45. पुरु॑षः श॒तेन्द्रि॑यः श॒तेन्द्रि॑यः॒ पुरु॑षः॒ पुरु॑षः श॒तेन्द्रि॑य॒ आयु॒ ष्यायु॑षि श॒तेन्द्रि॑यः॒ पुरु॑षः॒ पुरु॑षः श॒तेन्द्रि॑य॒ आयु॑षि । \newline
46. श॒तेन्द्रि॑य॒ आयु॒ ष्यायु॑षि श॒तेन्द्रि॑यः श॒तेन्द्रि॑य॒ आयु॑ ष्ये॒वै वायु॑षि श॒तेन्द्रि॑यः श॒तेन्द्रि॑य॒ आयु॑ ष्ये॒व । \newline
47. श॒तेन्द्रि॑य॒ इति॑ श॒त - इ॒न्द्रि॒यः॒ । \newline
48. आयु॑ ष्ये॒वै वायु॒ ष्यायु॑ ष्ये॒वेन्द्रि॒य इ॑न्द्रि॒य ए॒वायु॒ ष्यायु॑ ष्ये॒वेन्द्रि॒ये । \newline
49. ए॒वे न्द्रि॒य इ॑न्द्रि॒य ए॒वै वेन्द्रि॒ये प्रति॒ प्रती᳚न्द्रि॒य ए॒वै वेन्द्रि॒ये प्रति॑ । \newline
50. इ॒न्द्रि॒ये प्रति॒ प्रती᳚न्द्रि॒य इ॑न्द्रि॒ये प्रति॑ तिष्ठति तिष्ठति॒ प्रती᳚न्द्रि॒य इ॑न्द्रि॒ये प्रति॑ तिष्ठति । \newline
51. प्रति॑ तिष्ठति तिष्ठति॒ प्रति॒ प्रति॑ तिष्ठ त्य॒पो॑ ऽपस्ति॑ष्ठति॒ प्रति॒ प्रति॑ तिष्ठत्य॒पः । \newline
52. ति॒ष्ठ॒ त्य॒पो॑ ऽप स्ति॑ष्ठति तिष्ठ त्य॒प उपो पा॒प स्ति॑ष्ठति तिष्ठ त्य॒प उप॑ । \newline
53. अ॒प उपो पा॒पो॑ ऽप उप॑ स्पृशति स्पृश॒ त्युपा॒पो॑ ऽप उप॑ स्पृशति । \newline
54. उप॑ स्पृशति स्पृश॒ त्युपोप॑ स्पृशति भेष॒जम् भे॑ष॒जꣳ स्पृ॑श॒ त्युपोप॑ स्पृशति भेष॒जम् । \newline
55. स्पृ॒श॒ति॒ भे॒ष॒जम् भे॑ष॒जꣳ स्पृ॑शति स्पृशति भेष॒जम् ॅवै वै भे॑ष॒जꣳ स्पृ॑शति स्पृशति भेष॒जम् ॅवै । \newline
56. भे॒ष॒जम् ॅवै वै भे॑ष॒जम् भे॑ष॒जम् ॅवा आप॒ आपो॒ वै भे॑ष॒जम् भे॑ष॒जम् ॅवा आपः॑ । \newline
57. वा आप॒ आपो॒ वै वा आपो॑ भेष॒जम् भे॑ष॒ज मापो॒ वै वा आपो॑ भेष॒जम् । \newline
58. आपो॑ भेष॒जम् भे॑ष॒ज माप॒ आपो॑ भेष॒ज मे॒वैव भे॑ष॒ज माप॒ आपो॑ भेष॒ज मे॒व । \newline
59. भे॒ष॒ज मे॒वैव भे॑ष॒जम् भे॑ष॒ज मे॒व कु॑रुते कुरुत ए॒व भे॑ष॒जम् भे॑ष॒ज मे॒व कु॑रुते । \newline
60. ए॒व कु॑रुते कुरुत ए॒वैव कु॑रुते । \newline
61. कु॒रु॒त॒ इति॑ कुरुते । \newline
\pagebreak
\markright{ TS 3.3.5.1  \hfill https://www.vedavms.in \hfill}

\section{ TS 3.3.5.1 }

\textbf{TS 3.3.5.1 } \newline
\textbf{Samhita Paata} \newline

वा॒युर॑सि प्रा॒णो नाम॑ सवि॒तुराधि॑पत्येऽपा॒नं मे॑ दा॒श्चक्षु॑रसि॒ श्रोत्रं॒ नाम॑ धा॒तुराधि॑पत्य॒ आयु॑र्मे दा रू॒पम॑सि॒ वर्णो॒ नाम॒ बृह॒स्पते॒राधि॑पत्ये प्र॒जां मे॑ दा ऋ॒तम॑सि स॒त्यं नामेन्द्र॒स्याऽऽ*धि॑पत्ये क्ष॒त्रं मे॑ दा भू॒तम॑सि॒ भव्यं॒ नाम॑ पितृ॒णामाधि॑पत्ये॒ऽपा-मोष॑धीनां॒ गर्भं॑ धा ऋ॒तस्य॑ त्वा॒ व्यो॑मन ऋ॒तस्य॑ - [  ] \newline

\textbf{Pada Paata} \newline

वा॒युः । अ॒सि॒ । प्रा॒ण इति॑ प्र - अ॒नः । नाम॑ । स॒वि॒तुः । आधि॑पत्य॒ इत्याधि॑ - प॒त्ये॒ । अ॒पा॒नमित्य॑प - अ॒नम् । मे॒ । दाः॒ । चक्षुः॑ । अ॒सि॒ । श्रोत्र᳚म् । नाम॑ । धा॒तुः । आधि॑पत्य॒ इत्याधि॑-प॒त्ये॒ । आयुः॑ । मे॒ । दाः॒ । रू॒पम् । अ॒सि॒ । वर्णः॑ । नाम॑ । बृह॒स्पतेः᳚ । आधि॑पत्य॒ इत्याधि॑ - प॒त्ये॒ । प्र॒जामिति॑ प्र-जाम् । मे॒ । दाः॒ । ऋ॒तम् । अ॒सि॒ । स॒त्यम् । नाम॑ । इन्द्र॑स्य । आधि॑पत्य॒ इत्याधि॑ - प॒त्ये॒ । क्ष॒त्रम् । मे॒ । दाः॒ । भू॒तम् । अ॒सि॒ । भव्य᳚म् । नाम॑ । पि॒तृ॒णाम् । आधि॑पत्य॒ इत्याधि॑ - प॒त्ये॒ । अ॒पाम् । ओष॑धीनाम् । गर्भ᳚म् । धाः॒ । ऋ॒तस्य॑ । त्वा॒ । व्यो॑मन॒ इति॒ वि - ओ॒म॒ने॒ । ऋ॒तस्य॑ ।  \newline


\textbf{Krama Paata} \newline

वा॒युर॑सि । अ॒सि॒ प्रा॒णः । प्रा॒णो नाम॑ । प्रा॒ण इति॑ प्र - अ॒नः । नाम॑ सवि॒तुः । स॒वि॒तुराधि॑पत्ये । आधि॑पत्ये ऽपा॒नम् । आधि॑पत्य॒ इत्याधि॑ - प॒त्ये॒ । अ॒पा॒नम् मे᳚ । अ॒पा॒नमित्य॑प - अ॒नम् । मे॒ दाः॒ । दा॒ श्चक्षुः॑ । चक्षु॑रसि । अ॒सि॒ श्रोत्र᳚म् । श्रोत्र॒म् नाम॑ । नाम॑ धा॒तुः । धा॒तुराधि॑पत्ये । आधि॑पत्य॒ आयुः॑ । आधि॑पत्य॒ इत्याधि॑ - प॒त्ये॒ । आयु॑र् मे । मे॒ दाः॒ । दा॒ रू॒पम् । रू॒पम॑सि । अ॒सि॒ वर्णः॑ । वर्णो॒ नाम॑ । नाम॒ बृह॒स्पतेः᳚ । बृह॒स्पते॒राधि॑पत्ये । आधि॑पत्ये प्र॒जाम् । आधि॑पत्य॒ इत्याधि॑ - प॒त्ये॒ । प्र॒जाम् मे᳚ । प्र॒जामिति॑ प्र - जाम् । मे॒ दाः॒ । दा॒ ऋ॒तम् । ऋ॒तम॑सि । अ॒सि॒ स॒त्यम् । स॒त्यम् नाम॑ । नामेन्द्र॑स्य । इन्द्र॒स्याधि॑पत्ये । आधि॑पत्ये क्ष॒त्रम् । आधि॑पत्य॒ इत्याधि॑ - प॒त्ये॒ । क्ष॒त्रम् मे᳚ । मे॒ दाः॒ । दा॒ भू॒तम् । भू॒तम॑सि । अ॒सि॒ भव्य᳚म् । भव्य॒म् नाम॑ । नाम॑ पितृ॒णाम् । पि॒तृ॒णामाधि॑पत्ये । आधि॑पत्ये॒ ऽपाम् । आधि॑पत्य॒ इत्याधि॑ - प॒त्ये॒ । अ॒पामोष॑धीनाम् । ओष॑धीना॒म् गर्भ᳚म् । गर्भ॑म् धाः । धा॒ ऋ॒तस्य॑ । ऋ॒तस्य॑ त्वा । त्वा॒ व्यो॑मने । व्यो॑मन ऋ॒तस्य॑ । व्यो॑मन॒ इति॒ वि - ओ॒म॒ने॒ । ऋ॒तस्य॑ त्वा \newline

\textbf{Jatai Paata} \newline

1. वा॒यु र॑स्यसि वा॒युर् वा॒यु र॑सि । \newline
2. अ॒सि॒ प्रा॒णः प्रा॒णो᳚ ऽस्यसि प्रा॒णः । \newline
3. प्रा॒णो नाम॒ नाम॑ प्रा॒णः प्रा॒णो नाम॑ । \newline
4. प्रा॒ण इति॑ प्र - अ॒नः । \newline
5. नाम॑ सवि॒तुः स॑वि॒तुर् नाम॒ नाम॑ सवि॒तुः । \newline
6. स॒वि॒तु राधि॑पत्य॒ आधि॑पत्ये सवि॒तुः स॑वि॒तु राधि॑पत्ये । \newline
7. आधि॑पत्ये ऽपा॒न म॑पा॒न माधि॑पत्य॒ आधि॑पत्ये ऽपा॒नम् । \newline
8. आधि॑पत्य॒ इत्याधि॑ - प॒त्ये॒ । \newline
9. अ॒पा॒नम् मे॑ मे ऽपा॒न म॑पा॒नम् मे᳚ । \newline
10. अ॒पा॒नमित्य॑प - अ॒नम् । \newline
11. मे॒ दा॒ दा॒ मे॒ मे॒ दाः॒ । \newline
12. दा॒ श्चक्षु॒ श्चक्षु॑र् दा दा॒ श्चक्षुः॑ । \newline
13. चक्षु॑ रस्यसि॒ चक्षु॒ श्चक्षु॑ रसि । \newline
14. अ॒सि॒ श्रोत्रꣳ॒॒ श्रोत्र॑ मस्यसि॒ श्रोत्र᳚म् । \newline
15. श्रोत्र॒म् नाम॒ नाम॒ श्रोत्रꣳ॒॒ श्रोत्र॒म् नाम॑ । \newline
16. नाम॑ धा॒तुर् धा॒तुर् नाम॒ नाम॑ धा॒तुः । \newline
17. धा॒तु राधि॑पत्य॒ आधि॑पत्ये धा॒तुर् धा॒तु राधि॑पत्ये । \newline
18. आधि॑पत्य॒ आयु॒ रायु॒ राधि॑पत्य॒ आधि॑पत्य॒ आयुः॑ । \newline
19. आधि॑पत्य॒ इत्याधि॑ - प॒त्ये॒ । \newline
20. आयु॑र् मे म॒ आयु॒ रायु॑र् मे । \newline
21. मे॒ दा॒ दा॒ मे॒ मे॒ दाः॒ । \newline
22. दा॒ रू॒पꣳ रू॒पम् दा॑ दा रू॒पम् । \newline
23. रू॒प म॑स्यसि रू॒पꣳ रू॒प म॑सि । \newline
24. अ॒सि॒ वर्णो॒ वर्णो᳚ ऽस्यसि॒ वर्णः॑ । \newline
25. वर्णो॒ नाम॒ नाम॒ वर्णो॒ वर्णो॒ नाम॑ । \newline
26. नाम॒ बृह॒स्पते॒र् बृह॒स्पते॒र् नाम॒ नाम॒ बृह॒स्पतेः᳚ । \newline
27. बृह॒स्पते॒ राधि॑पत्य॒ आधि॑पत्ये॒ बृह॒स्पते॒र् बृह॒स्पते॒ राधि॑पत्ये । \newline
28. आधि॑पत्ये प्र॒जाम् प्र॒जा माधि॑पत्य॒ आधि॑पत्ये प्र॒जाम् । \newline
29. आधि॑पत्य॒ इत्याधि॑ - प॒त्ये॒ । \newline
30. प्र॒जाम् मे॑ मे प्र॒जाम् प्र॒जाम् मे᳚ । \newline
31. प्र॒जामिति॑ प्र - जाम् । \newline
32. मे॒ दा॒ दा॒ मे॒ मे॒ दाः॒ । \newline
33. दा॒ ऋ॒त मृ॒तम् दा॑ दा ऋ॒तम् । \newline
34. ऋ॒त म॑स्य स्यृ॒त मृ॒त म॑सि । \newline
35. अ॒सि॒ स॒त्यꣳ स॒त्य म॑स्यसि स॒त्यम् । \newline
36. स॒त्यम् नाम॒ नाम॑ स॒त्यꣳ स॒त्यम् नाम॑ । \newline
37. नामे न्द्र॒ स्येन्द्र॑स्य॒ नाम॒ नामेन्द्र॑स्य । \newline
38. इन्द्र॒ स्याधि॑पत्य॒ आधि॑पत्य॒ इन्द्र॒ स्येन्द्र॒ स्याधि॑पत्ये । \newline
39. आधि॑पत्ये क्ष॒त्रम् क्ष॒त्र माधि॑पत्य॒ आधि॑पत्ये क्ष॒त्रम् । \newline
40. आधि॑पत्य॒ इत्याधि॑ - प॒त्ये॒ । \newline
41. क्ष॒त्रम् मे॑ मे क्ष॒त्रम् क्ष॒त्रम् मे᳚ । \newline
42. मे॒ दा॒ दा॒ मे॒ मे॒ दाः॒ । \newline
43. दा॒ भू॒तम् भू॒तम् दा॑ दा भू॒तम् । \newline
44. भू॒त म॑स्यसि भू॒तम् भू॒त म॑सि । \newline
45. अ॒सि॒ भव्य॒म् भव्य॑ मस्यसि॒ भव्य᳚म् । \newline
46. भव्य॒म् नाम॒ नाम॒ भव्य॒म् भव्य॒म् नाम॑ । \newline
47. नाम॑ पितृ॒णाम् पि॑तृ॒णाम् नाम॒ नाम॑ पितृ॒णाम् । \newline
48. पि॒तृ॒णा माधि॑पत्य॒ आधि॑पत्ये पितृ॒णाम् पि॑तृ॒णा माधि॑पत्ये । \newline
49. आधि॑पत्ये॒ ऽपा म॒पा माधि॑पत्य॒ आधि॑पत्ये॒ ऽपाम् । \newline
50. आधि॑पत्य॒ इत्याधि॑ - प॒त्ये॒ । \newline
51. अ॒पा मोष॑धीना॒ मोष॑धीना म॒पा म॒पा मोष॑धीनाम् । \newline
52. ओष॑धीना॒म् गर्भ॒म् गर्भ॒ मोष॑धीना॒ मोष॑धीना॒म् गर्भ᳚म् । \newline
53. गर्भ॑म् धा धा॒ गर्भ॒म् गर्भ॑म् धाः । \newline
54. धा॒ ऋ॒तस्य॒ र्तस्य॑ धा धा ऋ॒तस्य॑ । \newline
55. ऋ॒तस्य॑ त्वा त्व॒ र्तस्य॒ र्तस्य॑ त्वा । \newline
56. त्वा॒ व्यो॑मने॒ व्यो॑मने त्वा त्वा॒ व्यो॑मने । \newline
57. व्यो॑मन ऋ॒तस्य॒ र्तस्य॒ व्यो॑मने॒ व्यो॑मन ऋ॒तस्य॑ । \newline
58. व्यो॑मन॒ इति॒ वि - ओ॒म॒ने॒ । \newline
59. ऋ॒तस्य॑ त्वा त्व॒ र्तस्य॒ र्तस्य॑ त्वा । \newline

\textbf{Ghana Paata } \newline

1. वा॒यु र॑स्यसि वा॒युर् वा॒यु र॑सि प्रा॒णः प्रा॒णो॑ ऽसि वा॒युर् वा॒यु र॑सि प्रा॒णः । \newline
2. अ॒सि॒ प्रा॒णः प्रा॒णो᳚ ऽस्यसि प्रा॒णो नाम॒ नाम॑ प्रा॒णो᳚ ऽस्यसि प्रा॒णो नाम॑ । \newline
3. प्रा॒णो नाम॒ नाम॑ प्रा॒णः प्रा॒णो नाम॑ सवि॒तुः स॑वि॒तुर् नाम॑ प्रा॒णः प्रा॒णो नाम॑ सवि॒तुः । \newline
4. प्रा॒ण इति॑ प्र - अ॒नः । \newline
5. नाम॑ सवि॒तुः स॑वि॒तुर् नाम॒ नाम॑ सवि॒तु राधि॑पत्य॒ आधि॑पत्ये सवि॒तुर् नाम॒ नाम॑ सवि॒तु राधि॑पत्ये । \newline
6. स॒वि॒तु राधि॑पत्य॒ आधि॑पत्ये सवि॒तुः स॑वि॒तु राधि॑पत्ये ऽपा॒न म॑पा॒न माधि॑पत्ये सवि॒तुः स॑वि॒तु राधि॑पत्ये ऽपा॒नम् । \newline
7. आधि॑पत्ये ऽपा॒न म॑पा॒न माधि॑पत्य॒ आधि॑पत्ये ऽपा॒नम् मे॑ मे ऽपा॒न माधि॑पत्य॒ आधि॑पत्ये ऽपा॒नम् मे᳚ । \newline
8. आधि॑पत्य॒ इत्याधि॑ - प॒त्ये॒ । \newline
9. अ॒पा॒नम् मे॑ मे ऽपा॒न म॑पा॒नम् मे॑ दा दा मे ऽपा॒न म॑पा॒नम् मे॑ दाः । \newline
10. अ॒पा॒नमित्य॑प - अ॒नम् । \newline
11. मे॒ दा॒ दा॒ मे॒ मे॒ दा॒ श्चक्षु॒ श्चक्षु॑र् दा मे मे दा॒ श्चक्षुः॑ । \newline
12. दा॒ श्चक्षु॒ श्चक्षु॑र् दा दा॒ श्चक्षु॑ रस्यसि॒ चक्षु॑र् दा दा॒ श्चक्षु॑ रसि । \newline
13. चक्षु॑ रस्यसि॒ चक्षु॒ श्चक्षु॑ रसि॒ श्रोत्रꣳ॒॒ श्रोत्र॑ मसि॒ चक्षु॒ श्चक्षु॑ रसि॒ श्रोत्र᳚म् । \newline
14. अ॒सि॒ श्रोत्रꣳ॒॒ श्रोत्र॑ मस्यसि॒ श्रोत्र॒म् नाम॒ नाम॒ श्रोत्र॑ मस्यसि॒ श्रोत्र॒म् नाम॑ । \newline
15. श्रोत्र॒म् नाम॒ नाम॒ श्रोत्रꣳ॒॒ श्रोत्र॒म् नाम॑ धा॒तुर् धा॒तुर् नाम॒ श्रोत्रꣳ॒॒ श्रोत्र॒म् नाम॑ धा॒तुः । \newline
16. नाम॑ धा॒तुर् धा॒तुर् नाम॒ नाम॑ धा॒तु राधि॑पत्य॒ आधि॑पत्ये धा॒तुर् नाम॒ नाम॑ धा॒तु राधि॑पत्ये । \newline
17. धा॒तु राधि॑पत्य॒ आधि॑पत्ये धा॒तुर् धा॒तु राधि॑पत्य॒ आयु॒ रायु॒ राधि॑पत्ये धा॒तुर् धा॒तु राधि॑पत्य॒ आयुः॑ । \newline
18. आधि॑पत्य॒ आयु॒ रायु॒ राधि॑पत्य॒ आधि॑पत्य॒ आयु॑र् मे म॒ आयु॒ राधि॑पत्य॒ आधि॑पत्य॒ आयु॑र् मे । \newline
19. आधि॑पत्य॒ इत्याधि॑ - प॒त्ये॒ । \newline
20. आयु॑र् मे म॒ आयु॒ रायु॑र् मे दा दा म॒ आयु॒ रायु॑र् मे दाः । \newline
21. मे॒ दा॒ दा॒ मे॒ मे॒ दा॒ रू॒पꣳ रू॒पम् दा॑ मे मे दा रू॒पम् । \newline
22. दा॒ रू॒पꣳ रू॒पम् दा॑ दा रू॒प म॑स्यसि रू॒पम् दा॑ दा रू॒प म॑सि । \newline
23. रू॒प म॑स्यसि रू॒पꣳ रू॒प म॑सि॒ वर्णो॒ वर्णो॑ ऽसि रू॒पꣳ रू॒प म॑सि॒ वर्णः॑ । \newline
24. अ॒सि॒ वर्णो॒ वर्णो᳚ ऽस्यसि॒ वर्णो॒ नाम॒ नाम॒ वर्णो᳚ ऽस्यसि॒ वर्णो॒ नाम॑ । \newline
25. वर्णो॒ नाम॒ नाम॒ वर्णो॒ वर्णो॒ नाम॒ बृह॒स्पते॒र् बृह॒स्पते॒र् नाम॒ वर्णो॒ वर्णो॒ नाम॒ बृह॒स्पतेः᳚ । \newline
26. नाम॒ बृह॒स्पते॒र् बृह॒स्पते॒र् नाम॒ नाम॒ बृह॒स्पते॒ राधि॑पत्य॒ आधि॑पत्ये॒ बृह॒स्पते॒र् नाम॒ नाम॒ बृह॒स्पते॒ राधि॑पत्ये । \newline
27. बृह॒स्पते॒ राधि॑पत्य॒ आधि॑पत्ये॒ बृह॒स्पते॒र् बृह॒स्पते॒ राधि॑पत्ये प्र॒जाम् प्र॒जा माधि॑पत्ये॒ बृह॒स्पते॒र् बृह॒स्पते॒ राधि॑पत्ये प्र॒जाम् । \newline
28. आधि॑पत्ये प्र॒जाम् प्र॒जा माधि॑पत्य॒ आधि॑पत्ये प्र॒जाम् मे॑ मे प्र॒जा माधि॑पत्य॒ आधि॑पत्ये प्र॒जाम् मे᳚ । \newline
29. आधि॑पत्य॒ इत्याधि॑ - प॒त्ये॒ । \newline
30. प्र॒जाम् मे॑ मे प्र॒जाम् प्र॒जाम् मे॑ दा दा मे प्र॒जाम् प्र॒जाम् मे॑ दाः । \newline
31. प्र॒जामिति॑ प्र - जाम् । \newline
32. मे॒ दा॒ दा॒ मे॒ मे॒ दा॒ ऋ॒त मृ॒तम् दा॑ मे मे दा ऋ॒तम् । \newline
33. दा॒ ऋ॒त मृ॒तम् दा॑ दा ऋ॒त म॑स्य स्यृ॒तम् दा॑ दा ऋ॒त म॑सि । \newline
34. ऋ॒त म॑स्य स्यृ॒त मृ॒त म॑सि स॒त्यꣳ स॒त्य म॑स्यृ॒त मृ॒त म॑सि स॒त्यम् । \newline
35. अ॒सि॒ स॒त्यꣳ स॒त्य म॑स्यसि स॒त्यम् नाम॒ नाम॑ स॒त्य म॑स्यसि स॒त्यम् नाम॑ । \newline
36. स॒त्यम् नाम॒ नाम॑ स॒त्यꣳ स॒त्यम् नामेन्द्र॒ स्येन्द्र॑स्य॒ नाम॑ स॒त्यꣳ स॒त्यम् नामेन्द्र॑स्य । \newline
37. नामेन्द्र॒ स्येन्द्र॑स्य॒ नाम॒ नामेन्द्र॒ स्याधि॑पत्य॒ आधि॑पत्य॒ इन्द्र॑स्य॒ नाम॒ नामेन्द्र॒ स्याधि॑पत्ये । \newline
38. इन्द्र॒स्याधि॑पत्य॒ आधि॑पत्य॒ इन्द्र॒ स्येन्द्र॒ स्याधि॑पत्ये क्ष॒त्रम् क्ष॒त्र माधि॑पत्य॒ 
इन्द्र॒ स्येन्द्र॒ स्याधि॑पत्ये क्ष॒त्रम् । \newline
39. आधि॑पत्ये क्ष॒त्रम् क्ष॒त्र माधि॑पत्य॒ आधि॑पत्ये क्ष॒त्रम् मे॑ मे क्ष॒त्र माधि॑पत्य॒ आधि॑पत्ये क्ष॒त्रम् मे᳚ । \newline
40. आधि॑पत्य॒ इत्याधि॑ - प॒त्ये॒ । \newline
41. क्ष॒त्रम् मे॑ मे क्ष॒त्रम् क्ष॒त्रम् मे॑ दा दा मे क्ष॒त्रम् क्ष॒त्रम् मे॑ दाः । \newline
42. मे॒ दा॒ दा॒ मे॒ मे॒ दा॒ भू॒तम् भू॒तम् दा॑ मे मे दा भू॒तम् । \newline
43. दा॒ भू॒तम् भू॒तम् दा॑ दा भू॒त म॑स्यसि भू॒तम् दा॑ दा भू॒त म॑सि । \newline
44. भू॒त म॑स्यसि भू॒तम् भू॒त म॑सि॒ भव्य॒म् भव्य॑ मसि भू॒तम् भू॒त म॑सि॒ भव्य᳚म् । \newline
45. अ॒सि॒ भव्य॒म् भव्य॑ मस्यसि॒ भव्य॒म् नाम॒ नाम॒ भव्य॑ मस्यसि॒ भव्य॒म् नाम॑ । \newline
46. भव्य॒म् नाम॒ नाम॒ भव्य॒म् भव्य॒म् नाम॑ पितृ॒णाम् पि॑तृ॒णाम् नाम॒ भव्य॒म् भव्य॒म् नाम॑ पितृ॒णाम् । \newline
47. नाम॑ पितृ॒णाम् पि॑तृ॒णाम् नाम॒ नाम॑ पितृ॒णा माधि॑पत्य॒ आधि॑पत्ये पितृ॒णाम् नाम॒ नाम॑ पितृ॒णा माधि॑पत्ये । \newline
48. पि॒तृ॒णा माधि॑पत्य॒ आधि॑पत्ये पितृ॒णाम् पि॑तृ॒णा माधि॑पत्ये॒ ऽपा म॒पा माधि॑पत्ये पितृ॒णाम् पि॑तृ॒णा माधि॑पत्ये॒ ऽपाम् । \newline
49. आधि॑पत्ये॒ ऽपा म॒पा माधि॑पत्य॒ आधि॑पत्ये॒ ऽपा मोष॑धीना॒ मोष॑धीना म॒पा माधि॑पत्य॒ आधि॑पत्ये॒ ऽपा मोष॑धीनाम् । \newline
50. आधि॑पत्य॒ इत्याधि॑ - प॒त्ये॒ । \newline
51. अ॒पा मोष॑धीना॒ मोष॑धीना म॒पा म॒पा मोष॑धीना॒म् गर्भ॒म् गर्भ॒ मोष॑धीना म॒पा म॒पा मोष॑धीना॒म् गर्भ᳚म् । \newline
52. ओष॑धीना॒म् गर्भ॒म् गर्भ॒ मोष॑धीना॒ मोष॑धीना॒म् गर्भ॑म् धा धा॒ गर्भ॒ मोष॑धीना॒ मोष॑धीना॒म् गर्भ॑म् धाः । \newline
53. गर्भ॑म् धा धा॒ गर्भ॒म् गर्भ॑म् धा ऋ॒तस्य॒ र्तस्य॑ धा॒ गर्भ॒म् गर्भ॑म् धा ऋ॒तस्य॑ । \newline
54. धा॒ ऋ॒तस्य॒ र्तस्य॑ धा धा ऋ॒तस्य॑ त्वा त्व॒र्तस्य॑ धा धा ऋ॒तस्य॑ त्वा । \newline
55. ऋ॒तस्य॑ त्वा त्व॒र्तस्य॒ र्तस्य॑ त्वा॒ व्यो॑मने॒ व्यो॑मने त्व॒र्तस्य॒ र्तस्य॑ त्वा॒ व्यो॑मने । \newline
56. त्वा॒ व्यो॑मने॒ व्यो॑मने त्वा त्वा॒ व्यो॑मन ऋ॒तस्य॒ र्तस्य॒ व्यो॑मने त्वा त्वा॒ व्यो॑मन ऋ॒तस्य॑ । \newline
57. व्यो॑मन ऋ॒तस्य॒ र्तस्य॒ व्यो॑मने॒ व्यो॑मन ऋ॒तस्य॑ त्वा त्व॒र्तस्य॒ व्यो॑मने॒ व्यो॑मन ऋ॒तस्य॑ त्वा । \newline
58. व्यो॑मन॒ इति॒ वि - ओ॒म॒ने॒ । \newline
59. ऋ॒तस्य॑ त्वा त्व॒र्तस्य॒ र्तस्य॑ त्वा॒ विभू॑मने॒ विभू॑मने त्व॒र्तस्य॒ र्तस्य॑ त्वा॒ विभू॑मने । \newline
\pagebreak
\markright{ TS 3.3.5.2  \hfill https://www.vedavms.in \hfill}

\section{ TS 3.3.5.2 }

\textbf{TS 3.3.5.2 } \newline
\textbf{Samhita Paata} \newline

त्वा॒ विभू॑मन ऋ॒तस्य॑ त्वा॒ विध॑र्मण ऋ॒तस्य॑ त्वा स॒त्याय॒र्तस्य॑ त्वा॒ ज्योति॑षे प्र॒जाप॑ति र्वि॒राज॑मपश्य॒त् तया॑ भू॒तं च॒ भव्यं॑ चा सृजत॒ तामृषि॑भ्यस्ति॒रो॑ऽदधा॒त् तां ज॒मद॑ग्नि॒स्तप॑साऽ पश्य॒त् तया॒ वै स पृश्ञी॒न् कामा॑नसृजत॒ तत् पृश्ञी॑नां पृश्ञि॒त्वं ॅयत् पृश्ञ॑यो गृ॒ह्यन्ते॒ पृश्ञी॑ने॒व तैः कामा॒न्॒. यज॑मा॒नोऽव॑ रुन्धे वा॒युर॑सि प्रा॒णो - [  ] \newline

\textbf{Pada Paata} \newline

त्वा॒ । विभू॑मन॒ इति॒ वि - भू॒म॒ने॒ । ऋ॒तस्य॑ । त्वा॒ । विध॑र्मण॒ इति॒ वि - ध॒र्म॒णे॒ । ऋ॒तस्य॑ । त्वा॒ । स॒त्याय॑ । ऋ॒तस्य॑ । त्वा॒ । ज्योति॑षे । प्र॒जाप॑ति॒रिति॑ प्र॒जा - प॒तिः॒ । वि॒राज॒मिति॑ वि - राज᳚म् । अ॒प॒श्य॒त् । तया᳚ । भू॒तम् । च॒ । भव्य᳚म् । च॒ । अ॒सृ॒ज॒त॒ । ताम् । ऋषि॑भ्य॒ इत्यृषि॑ - भ्यः॒ । ति॒रः । अ॒द॒धा॒त् । ताम् । ज॒मद॑ग्निः । तप॑सा । अ॒प॒श्य॒त् । तया᳚ । वै । सः । पृश्नीन्॑ । कामान्॑ । अ॒सृ॒ज॒त॒ । तत् । पृश्नी॑नाम् । पृ॒श्नि॒त्वमिति॑ पृश्नि - त्वम् । यत् । पृश्न॑यः । गृ॒ह्यन्ते᳚ । पृश्नीन्॑ । ए॒व । तैः । कामान्॑ । यज॑मानः । अवेति॑ । रु॒न्धे॒ । वा॒युः । अ॒सि॒ । प्रा॒ण इति॑ प्र - अ॒नः ।  \newline


\textbf{Krama Paata} \newline

त्वा॒ विभू॑मने । विभू॑मन ऋ॒तस्य॑ । विभू॑मन॒ इति॒ वि - भू॒म॒ने॒ । ऋ॒तस्य॑ त्वा । त्वा॒ विध॑र्मणे । विध॑र्मण ऋ॒तस्य॑ । विध॑र्मण॒ इति॒ वि - ध॒र्म॒णे॒ । ऋ॒तस्य॑ त्वा । त्वा॒ स॒त्याय॑ । स॒त्याय॒र्तस्य॑ । ऋ॒तस्य॑ त्वा । त्वा॒ ज्योति॑षे । ज्योति॑षे प्र॒जाप॑तिः । प्र॒जाप॑तिर् वि॒राज᳚म् । प्र॒जाप॑ति॒रिति॑ प्र॒जा - प॒तिः॒ । वि॒राज॑मपश्यत् । वि॒राज॒मिति॑ वि - राज᳚म् । अ॒प॒श्य॒त् तया᳚ । तया॑ भू॒तम् । भू॒तम् च॑ । च॒ भव्य᳚म् । भव्य॑म् च । चा॒सृ॒ज॒त॒ । अ॒सृ॒ज॒त॒ ताम् । तामृषि॑भ्यः । ऋषि॑भ्यस्ति॒रः । ऋषि॑भ्य॒ इत्यृषि॑ - भ्यः॒ । ति॒रो॑ ऽदधात् । अ॒द॒धा॒त् ताम् । ताम् ज॒मद॑ग्निः । ज॒मद॑ग्नि॒स्तप॑सा । तप॑सा ऽपश्यत् । अ॒प॒श्य॒त् तया᳚ । तया॒ वै । वै सः । स पृश्ञीन्॑ । पृश्ञी॒न् कामान्॑ । कामा॑नसृजत । अ॒सृ॒ज॒त॒ तत् । तत् पृश्ञी॑नाम् । पृश्ञी॑नाम् पृश्ञि॒त्वम् । पृ॒श्ञि॒त्वं ॅयत् । पृ॒श्ञि॒त्वमिति॑ पृश्ञि - त्वम् । यत् पृश्ञ॑यः । पृश्ञ॑यो गृ॒ह्यन्ते᳚ । गृ॒ह्यन्ते॒ पृश्ञीन्॑ । पृश्ञी॑ने॒व । ए॒व तैः । तैः कामान्॑ । कामा॒न्॒. यज॑मानः । यज॑मा॒नो ऽव॑ । अव॑ रुन्धे । रु॒न्धे॒ वा॒युः । वा॒युर॑सि । अ॒सि॒ प्रा॒णः । प्रा॒णो नाम॑ । प्रा॒ण इति॑ प्र - अ॒नः \newline

\textbf{Jatai Paata} \newline

1. त्वा॒ विभू॑मने॒ विभू॑मने त्वा त्वा॒ विभू॑मने । \newline
2. विभू॑मन ऋ॒तस्य॒ र्तस्य॒ विभू॑मने॒ विभू॑मन ऋ॒तस्य॑ । \newline
3. विभू॑मन॒ इति॒ वि - भू॒म॒ने॒ । \newline
4. ऋ॒तस्य॑ त्वा त्व॒ र्तस्य॒ र्तस्य॑ त्वा । \newline
5. त्वा॒ विध॑र्मणे॒ विध॑र्मणे त्वा त्वा॒ विध॑र्मणे । \newline
6. विध॑र्मण ऋ॒तस्य॒ र्तस्य॒ विध॑र्मणे॒ विध॑र्मण ऋ॒तस्य॑ । \newline
7. विध॑र्मण॒ इति॒ वि - ध॒र्म॒णे॒ । \newline
8. ऋ॒तस्य॑ त्वा त्व॒ र्तस्य॒ र्तस्य॑ त्वा । \newline
9. त्वा॒ स॒त्याय॑ स॒त्याय॑ त्वा त्वा स॒त्याय॑ । \newline
10. स॒त्याय॒ र्तस्य॒ र्तस्य॑ स॒त्याय॑ स॒त्याय॒ र्तस्य॑ । \newline
11. ऋ॒तस्य॑ त्वा त्व॒ र्तस्य॒ र्तस्य॑ त्वा । \newline
12. त्वा॒ ज्योति॑षे॒ ज्योति॑षे त्वा त्वा॒ ज्योति॑षे । \newline
13. ज्योति॑षे प्र॒जाप॑तिः प्र॒जाप॑ति॒र् ज्योति॑षे॒ ज्योति॑षे प्र॒जाप॑तिः । \newline
14. प्र॒जाप॑तिर् वि॒राज॑म् ॅवि॒राज॑म् प्र॒जाप॑तिः प्र॒जाप॑तिर् वि॒राज᳚म् । \newline
15. प्र॒जाप॑ति॒रिति॑ प्र॒जा - प॒तिः॒ । \newline
16. वि॒राज॑ मपश्य दपश्यद् वि॒राज॑म् ॅवि॒राज॑ मपश्यत् । \newline
17. वि॒राज॒मिति॑ वि - राज᳚म् । \newline
18. अ॒प॒श्य॒त् तया॒ तया॑ ऽपश्य दपश्य॒त् तया᳚ । \newline
19. तया॑ भू॒तम् भू॒तम् तया॒ तया॑ भू॒तम् । \newline
20. भू॒तम् च॑ च भू॒तम् भू॒तम् च॑ । \newline
21. च॒ भव्य॒म् भव्य॑म् च च॒ भव्य᳚म् । \newline
22. भव्य॑म् च च॒ भव्य॒म् भव्य॑म् च । \newline
23. चा॒सृ॒ज॒ता॒ सृ॒ज॒त॒ च॒ चा॒सृ॒ज॒त॒ । \newline
24. अ॒सृ॒ज॒त॒ ताम् ता म॑सृजता सृजत॒ ताम् । \newline
25. ता मृषि॑भ्य॒ ऋषि॑भ्य॒ स्ताम् ता मृषि॑भ्यः । \newline
26. ऋषि॑भ्य स्ति॒र स्ति॒र ऋषि॑भ्य॒ ऋषि॑भ्य स्ति॒रः । \newline
27. ऋषि॑भ्य॒ इत्यृषि॑ - भ्यः॒ । \newline
28. ति॒रो॑ ऽदधा ददधात् ति॒र स्ति॒रो॑ ऽदधात् । \newline
29. अ॒द॒धा॒त् ताम् ता म॑दधा ददधा॒त् ताम् । \newline
30. ताम् ज॒मद॑ग्निर् ज॒मद॑ग्नि॒ स्ताम् ताम् ज॒मद॑ग्निः । \newline
31. ज॒मद॑ग्नि॒ स्तप॑सा॒ तप॑सा ज॒मद॑ग्निर् ज॒मद॑ग्नि॒ स्तप॑सा । \newline
32. तप॑सा ऽपश्य दपश्य॒त् तप॑सा॒ तप॑सा ऽपश्यत् । \newline
33. अ॒प॒श्य॒त् तया॒ तया॑ ऽपश्य दपश्य॒त् तया᳚ । \newline
34. तया॒ वै वै तया॒ तया॒ वै । \newline
35. वै स स वै वै सः । \newline
36. स पृश्ञी॒न् पृश्ञी॒न् थ्स स पृश्ञीन्॑ । \newline
37. पृश्ञी॒न् कामा॒न् कामा॒न् पृश्ञी॒न् पृश्ञी॒न् कामान्॑ । \newline
38. कामा॑ नसृजता सृजत॒ कामा॒न् कामा॑ नसृजत । \newline
39. अ॒सृ॒ज॒त॒ तत् तद॑सृजता सृजत॒ तत् । \newline
40. तत् पृश्ञी॑ना॒म् पृश्ञी॑ना॒म् तत् तत् पृश्ञी॑नाम् । \newline
41. पृश्ञी॑नाम् पृश्ञि॒त्वम् पृ॑श्ञि॒त्वम् पृश्ञी॑ना॒म् पृश्ञी॑नाम् पृश्ञि॒त्वम् । \newline
42. पृ॒श्ञि॒त्वम् ॅयद् यत् पृ॑श्ञि॒त्वम् पृ॑श्ञि॒त्वम् ॅयत् । \newline
43. पृ॒श्ञि॒त्वमिति॑ पृश्ञि - त्वम् । \newline
44. यत् पृश्ञ॑यः॒ पृश्ञ॑यो॒ यद् यत् पृश्ञ॑यः । \newline
45. पृश्ञ॑यो गृ॒ह्यन्ते॑ गृ॒ह्यन्ते॒ पृश्ञ॑यः॒ पृश्ञ॑यो गृ॒ह्यन्ते᳚ । \newline
46. गृ॒ह्यन्ते॒ पृश्ञी॒न् पृश्ञी᳚न् गृ॒ह्यन्ते॑ गृ॒ह्यन्ते॒ पृश्ञीन्॑ । \newline
47. पृश्ञी॑ ने॒वैव पृश्ञी॒न् पृश्ञी॑ ने॒व । \newline
48. ए॒व तै स्तै रे॒वैव तैः । \newline
49. तैः कामा॒न् कामा॒न् तै स्तैः कामान्॑ । \newline
50. कामा॒न्॒. यज॑मानो॒ यज॑मानः॒ कामा॒न् कामा॒न्॒. यज॑मानः । \newline
51. यज॑मा॒नो ऽवाव॒ यज॑मानो॒ यज॑मा॒नो ऽव॑ । \newline
52. अव॑ रुन्धे रु॒न्धे ऽवाव॑ रुन्धे । \newline
53. रु॒न्धे॒ वा॒युर् वा॒यू रु॑न्धे रुन्धे वा॒युः । \newline
54. वा॒यु र॑स्यसि वा॒युर् वा॒यु र॑सि । \newline
55. अ॒सि॒ प्रा॒णः प्रा॒णो᳚ ऽस्यसि प्रा॒णः । \newline
56. प्रा॒णो नाम॒ नाम॑ प्रा॒णः प्रा॒णो नाम॑ । \newline
57. प्रा॒ण इति॑ प्र - अ॒नः । \newline

\textbf{Ghana Paata } \newline

1. त्वा॒ विभू॑मने॒ विभू॑मने त्वा त्वा॒ विभू॑मन ऋ॒तस्य॒ र्तस्य॒ विभू॑मने त्वा त्वा॒ विभू॑मन ऋ॒तस्य॑ । \newline
2. विभू॑मन ऋ॒तस्य॒ र्तस्य॒ विभू॑मने॒ विभू॑मन ऋ॒तस्य॑ त्वा त्व॒र्तस्य॒ विभू॑मने॒ विभू॑मन ऋ॒तस्य॑ त्वा । \newline
3. विभू॑मन॒ इति॒ वि - भू॒म॒ने॒ । \newline
4. ऋ॒तस्य॑ त्वा त्व॒र्तस्य॒ र्तस्य॑ त्वा॒ विध॑र्मणे॒ विध॑र्मणे त्व॒र्तस्य॒ र्तस्य॑ त्वा॒ विध॑र्मणे । \newline
5. त्वा॒ विध॑र्मणे॒ विध॑र्मणे त्वा त्वा॒ विध॑र्मण ऋ॒तस्य॒ र्तस्य॒ विध॑र्मणे त्वा त्वा॒ विध॑र्मण ऋ॒तस्य॑ । \newline
6. विध॑र्मण ऋ॒तस्य॒ र्‌तस्य॒ विध॑र्मणे॒ विध॑र्मण ऋ॒तस्य॑ त्वा त्व॒र्तस्य॒ विध॑र्मणे॒ विध॑र्मण ऋ॒तस्य॑ त्वा । \newline
7. विध॑र्मण॒ इति॒ वि - ध॒र्म॒णे॒ । \newline
8. ऋ॒तस्य॑ त्वा त्व॒र्तस्य॒ र्‌तस्य॑ त्वा स॒त्याय॑ स॒त्याय॑ त्व॒र्तस्य॒ र्‌तस्य॑ त्वा स॒त्याय॑ । \newline
9. त्वा॒ स॒त्याय॑ स॒त्याय॑ त्वा त्वा स॒त्याय॒ र्‌तस्य॒ र्‌तस्य॑ स॒त्याय॑ त्वा त्वा स॒त्याय॒ र्‌तस्य॑ । \newline
10. स॒त्याय॒ र्तस्य॒ र्तस्य॑ स॒त्याय॑ स॒त्याय॒ र्‌तस्य॑ त्वा त्व॒र्तस्य॑ स॒त्याय॑ स॒त्याय॒ र्‌तस्य॑ त्वा । \newline
11. ऋ॒तस्य॑ त्वा त्व॒र्तस्य॒ र्‌तस्य॑ त्वा॒ ज्योति॑षे॒ ज्योति॑षे त्व॒र्तस्य॒ र्‌तस्य॑ त्वा॒ ज्योति॑षे । \newline
12. त्वा॒ ज्योति॑षे॒ ज्योति॑षे त्वा त्वा॒ ज्योति॑षे प्र॒जाप॑तिः प्र॒जाप॑ति॒र् ज्योति॑षे त्वा त्वा॒ ज्योति॑षे प्र॒जाप॑तिः । \newline
13. ज्योति॑षे प्र॒जाप॑तिः प्र॒जाप॑ति॒र् ज्योति॑षे॒ ज्योति॑षे प्र॒जाप॑तिर् वि॒राज॑म् ॅवि॒राज॑म् प्र॒जाप॑ति॒र् ज्योति॑षे॒ ज्योति॑षे प्र॒जाप॑तिर् वि॒राज᳚म् । \newline
14. प्र॒जाप॑तिर् वि॒राज॑म् ॅवि॒राज॑म् प्र॒जाप॑तिः प्र॒जाप॑तिर् वि॒राज॑ मपश्य दपश्यद् वि॒राज॑म् प्र॒जाप॑तिः प्र॒जाप॑तिर् वि॒राज॑ मपश्यत् । \newline
15. प्र॒जाप॑ति॒रिति॑ प्र॒जा - प॒तिः॒ । \newline
16. वि॒राज॑ मपश्य दपश्यद् वि॒राज॑म् ॅवि॒राज॑ मपश्य॒त् तया॒ तया॑ ऽपश्यद् वि॒राज॑म् ॅवि॒राज॑ मपश्य॒त् तया᳚ । \newline
17. वि॒राज॒मिति॑ वि - राज᳚म् । \newline
18. अ॒प॒श्य॒त् तया॒ तया॑ ऽपश्य दपश्य॒त् तया॑ भू॒तम् भू॒तम् तया॑ ऽपश्य दपश्य॒त् तया॑ भू॒तम् । \newline
19. तया॑ भू॒तम् भू॒तम् तया॒ तया॑ भू॒तम् च॑ च भू॒तम् तया॒ तया॑ भू॒तम् च॑ । \newline
20. भू॒तम् च॑ च भू॒तम् भू॒तम् च॒ भव्य॒म् भव्य॑म् च भू॒तम् भू॒तम् च॒ भव्य᳚म् । \newline
21. च॒ भव्य॒म् भव्य॑म् च च॒ भव्य॑म् च च॒ भव्य॑म् च च॒ भव्य॑म् च । \newline
22. भव्य॑म् च च॒ भव्य॒म् भव्य॑म् चा सृजता सृजत च॒ भव्य॒म् भव्य॑म् चासृजत । \newline
23. चा॒सृ॒ज॒ता॒ सृ॒ज॒त॒ च॒ चा॒सृ॒ज॒त॒ ताम् ता म॑सृजत च चासृजत॒ ताम् । \newline
24. अ॒सृ॒ज॒त॒ ताम् ता म॑सृजता सृजत॒ ता मृषि॑भ्य॒ ऋषि॑भ्य॒ स्ता म॑सृजता सृजत॒ ता मृषि॑भ्यः । \newline
25. ता मृषि॑भ्य॒ ऋषि॑भ्य॒ स्ताम् ता मृषि॑भ्य स्ति॒र स्ति॒र ऋषि॑भ्य॒ स्ताम् ता मृषि॑भ्य स्ति॒रः । \newline
26. ऋषि॑भ्य स्ति॒र स्ति॒र ऋषि॑भ्य॒ ऋषि॑भ्य स्ति॒रो॑ ऽदधा ददधात् ति॒र ऋषि॑भ्य॒ ऋषि॑भ्य स्ति॒रो॑ ऽदधात् । \newline
27. ऋषि॑भ्य॒ इत्यृषि॑ - भ्यः॒ । \newline
28. ति॒रो॑ ऽदधा ददधात् ति॒र स्ति॒रो॑ ऽदधा॒त् ताम् ता म॑दधात् ति॒र स्ति॒रो॑ ऽदधा॒त् ताम् । \newline
29. अ॒द॒धा॒त् ताम् ता म॑दधा ददधा॒त् ताम् ज॒मद॑ग्निर् ज॒मद॑ग्नि॒ स्ता म॑दधा ददधा॒त् ताम् ज॒मद॑ग्निः । \newline
30. ताम् ज॒मद॑ग्निर् ज॒मद॑ग्नि॒ स्ताम् ताम् ज॒मद॑ग्नि॒ स्तप॑सा॒ तप॑सा ज॒मद॑ग्नि॒ स्ताम् ताम् ज॒मद॑ग्नि॒ स्तप॑सा । \newline
31. ज॒मद॑ग्नि॒ स्तप॑सा॒ तप॑सा ज॒मद॑ग्निर् ज॒मद॑ग्नि॒ स्तप॑सा ऽपश्य दपश्य॒त् तप॑सा ज॒मद॑ग्निर् ज॒मद॑ग्नि॒ स्तप॑सा ऽपश्यत् । \newline
32. तप॑सा ऽपश्य दपश्य॒त् तप॑सा॒ तप॑सा ऽपश्य॒त् तया॒ तया॑ ऽपश्य॒त् तप॑सा॒ तप॑सा ऽपश्य॒त् तया᳚ । \newline
33. अ॒प॒श्य॒त् तया॒ तया॑ ऽपश्य दपश्य॒त् तया॒ वै वै तया॑ ऽपश्य दपश्य॒त् तया॒ वै । \newline
34. तया॒ वै वै तया॒ तया॒ वै स स वै तया॒ तया॒ वै सः । \newline
35. वै स स वै वै स पृश्ञी॒न् पृश्ञी॒न् थ्स वै वै स पृश्ञीन्॑ । \newline
36. स पृश्ञी॒न् पृश्ञी॒न् थ्स स पृश्ञी॒न् कामा॒न् कामा॒न् पृश्ञी॒न् थ्स स पृश्ञी॒न् कामान्॑ । \newline
37. पृश्ञी॒न् कामा॒न् कामा॒न् पृश्ञी॒न् पृश्ञी॒न् कामा॑ नसृजता सृजत॒ कामा॒न् पृश्ञी॒न् पृश्ञी॒न् कामा॑ नसृजत । \newline
38. कामा॑ नसृजता सृजत॒ कामा॒न् कामा॑ नसृजत॒ तत् तद॑सृजत॒ कामा॒न् कामा॑ नसृजत॒ तत् । \newline
39. अ॒सृ॒ज॒त॒ तत् तद॑सृजता सृजत॒ तत् पृश्ञी॑ना॒म् पृश्ञी॑ना॒म् तद॑सृजता सृजत॒ तत् पृश्ञी॑नाम् । \newline
40. तत् पृश्ञी॑ना॒म् पृश्ञी॑ना॒म् तत् तत् पृश्ञी॑नाम् पृश्ञि॒त्वम् पृ॑श्ञि॒त्वम् पृश्ञी॑ना॒म् तत् तत् पृश्ञी॑नाम् पृश्ञि॒त्वम् । \newline
41. पृश्ञी॑नाम् पृश्ञि॒त्वम् पृ॑श्ञि॒त्वम् पृश्ञी॑ना॒म् पृश्ञी॑नाम् पृश्ञि॒त्वम् ॅयद् यत् पृ॑श्ञि॒त्वम् पृश्ञी॑ना॒म् पृश्ञी॑नाम् पृश्ञि॒त्वम् ॅयत् । \newline
42. पृ॒श्ञि॒त्वम् ॅयद् यत् पृ॑श्ञि॒त्वम् पृ॑श्ञि॒त्वम् ॅयत् पृश्ञ॑यः॒ पृश्ञ॑यो॒ यत् पृ॑श्ञि॒त्वम् पृ॑श्ञि॒त्वम् ॅयत् पृश्ञ॑यः । \newline
43. पृ॒श्ञि॒त्वमिति॑ पृश्ञि - त्वम् । \newline
44. यत् पृश्ञ॑यः॒ पृश्ञ॑यो॒ यद् यत् पृश्ञ॑यो गृ॒ह्यन्ते॑ गृ॒ह्यन्ते॒ पृश्ञ॑यो॒ यद् यत् पृश्ञ॑यो गृ॒ह्यन्ते᳚ । \newline
45. पृश्ञ॑यो गृ॒ह्यन्ते॑ गृ॒ह्यन्ते॒ पृश्ञ॑यः॒ पृश्ञ॑यो गृ॒ह्यन्ते॒ पृश्ञी॒न् पृश्ञी᳚न् गृ॒ह्यन्ते॒ पृश्ञ॑यः॒ पृश्ञ॑यो गृ॒ह्यन्ते॒ पृश्ञीन्॑ । \newline
46. गृ॒ह्यन्ते॒ पृश्ञी॒न् पृश्ञी᳚न् गृ॒ह्यन्ते॑ गृ॒ह्यन्ते॒ पृश्ञी॑ ने॒वैव पृश्ञी᳚न् गृ॒ह्यन्ते॑ गृ॒ह्यन्ते॒ पृश्ञी॑ ने॒व । \newline
47. पृश्ञी॑ ने॒वैव पृश्ञी॒न् पृश्ञी॑ ने॒व तै स्तै रे॒व पृश्ञी॒न् पृश्ञी॑ ने॒व तैः । \newline
48. ए॒व तै स्तै रे॒वैव तैः कामा॒न् कामा॒न् तै रे॒वैव तैः कामान्॑ । \newline
49. तैः कामा॒न् कामा॒न् तै स्तैः कामा॒न्॒. यज॑मानो॒ यज॑मानः॒ कामा॒न् तै स्तैः कामा॒न्॒. यज॑मानः । \newline
50. कामा॒न्॒. यज॑मानो॒ यज॑मानः॒ कामा॒न् कामा॒न्॒. यज॑मा॒नो ऽवाव॒ यज॑मानः॒ कामा॒न् कामा॒न्॒. यज॑मा॒नो ऽव॑ । \newline
51. यज॑मा॒नो ऽवाव॒ यज॑मानो॒ यज॑मा॒नो ऽव॑ रुन्धे रु॒न्धे ऽव॒ यज॑मानो॒ यज॑मा॒नो ऽव॑ रुन्धे । \newline
52. अव॑ रुन्धे रु॒न्धे ऽवाव॑ रुन्धे वा॒युर् वा॒यू रु॒न्धे ऽवाव॑ रुन्धे वा॒युः । \newline
53. रु॒न्धे॒ वा॒युर् वा॒यू रु॑न्धे रुन्धे वा॒यु र॑स्यसि वा॒यू रु॑न्धे रुन्धे वा॒यु र॑सि । \newline
54. वा॒यु र॑स्यसि वा॒युर् वा॒यु र॑सि प्रा॒णः प्रा॒णो॑ ऽसि वा॒युर् वा॒यु र॑सि प्रा॒णः । \newline
55. अ॒सि॒ प्रा॒णः प्रा॒णो᳚ ऽस्यसि प्रा॒णो नाम॒ नाम॑ प्रा॒णो᳚ ऽस्यसि प्रा॒णो नाम॑ । \newline
56. प्रा॒णो नाम॒ नाम॑ प्रा॒णः प्रा॒णो नामे तीति॒ नाम॑ प्रा॒णः प्रा॒णो नामे ति॑ । \newline
57. प्रा॒ण इति॑ प्र - अ॒नः । \newline
\pagebreak
\markright{ TS 3.3.5.3  \hfill https://www.vedavms.in \hfill}

\section{ TS 3.3.5.3 }

\textbf{TS 3.3.5.3 } \newline
\textbf{Samhita Paata} \newline

नामेत्या॑ह प्राणापा॒नावे॒वाव॑ रुन्धे॒ चक्षु॑रसि॒ श्रोत्रं॒ नामेत्या॒हाऽऽयु॑रे॒वाव॑ रुन्धे रू॒पम॑सि॒ वर्णो॒ नामेत्या॑ह प्र॒जामे॒वाव॑ रुन्धऋ॒तम॑सि स॒त्यं नामेत्या॑ह क्ष॒त्रमे॒वाव॑ रुन्धे भू॒तम॑सि॒ भव्यं॒ नामेत्या॑ह प॒शवो॒ वा अ॒पामोष॑धीनां॒ गर्भः॑ प॒शूने॒वा - [  ] \newline

\textbf{Pada Paata} \newline

नाम॑ । इति॑ । आ॒ह॒ । प्रा॒णा॒पा॒नाविति॑ प्राण-अ॒पा॒नौ । ए॒व । अवेति॑ । रु॒न्धे॒ । चक्षुः॑ । अ॒सि॒ । श्रोत्र᳚म् । नाम॑ । इति॑ । आ॒ह॒ । आयुः॑ । ए॒व । अवेति॑ । रु॒न्धे॒ । रू॒पम् । अ॒सि॒ । वर्णः॑ । नाम॑ । इति॑ । आ॒ह॒ । प्र॒जामिति॑ प्र - जाम् । ए॒व । अवेति॑ । रु॒न्धे॒ । ऋ॒तम् । अ॒सि॒ । स॒त्यम् । नाम॑ । इति॑ । आ॒ह॒ । क्ष॒त्रम् । ए॒व । अवेति॑ । रु॒न्धे॒ । भू॒तम् । अ॒सि॒ । भव्य᳚म् । नाम॑ । इति॑ । आ॒ह॒ । प॒शवः॑ । वै । अ॒पाम् । ओष॑धीनाम् । गर्भः॑ । प॒शून् । ए॒व ।  \newline


\textbf{Krama Paata} \newline

नामेति॑ । इत्या॑ह । आ॒ह॒ प्रा॒णा॒पा॒नौ । प्रा॒णा॒पा॒नावे॒व । प्रा॒णा॒पा॒नाविति॑ प्राण - अ॒पा॒नौ । ए॒वाव॑ । अव॑ रुन्धे । रु॒न्धे॒ चक्षुः॑ । चक्षु॑रसि । अ॒सि॒ श्रोत्र᳚म् । श्रोत्र॒म् नाम॑ । नामेति॑ । इत्या॑ह । आ॒हायुः॑ । आयु॑रे॒व । ए॒वाव॑ । अव॑ रुन्धे । रु॒न्धे॒ रू॒पम् । रू॒पम॑सि । अ॒सि॒ वर्णः॑ । वर्णो॒ नाम॑ । नामेति॑ । इत्या॑ह । आ॒ह॒ प्र॒जाम् । प्र॒जामे॒व । प्र॒जामिति॑ प्र - जाम् । ए॒वाव॑ । अव॑ रुन्धे । रु॒न्ध॒ ऋ॒तम् । ऋ॒तम॑सि । अ॒सि॒ स॒त्यम् । स॒त्यम् नाम॑ । नामेति॑ । इत्या॑ह । आ॒ह॒ क्ष॒त्रम् । क्ष॒त्रमे॒व । ए॒वाव॑ । अव॑ रुन्धे । रु॒न्धे॒ भू॒तम् । भू॒तम॑सि । अ॒सि॒ भव्य᳚म् । भव्य॒म् नाम॑ । नामेति॑ । इत्या॑ह । आ॒ह॒ प॒शवः॑ । प॒शवो॒ वै । वा अ॒पाम् । अ॒पामोष॑धीनाम् । ओष॑धीना॒म् गर्भः॑ । गर्भः॑ प॒शून् । प॒शूने॒व । ए॒वाव॑ \newline

\textbf{Jatai Paata} \newline

1. नामे तीति॒ नाम॒ नामे ति॑ । \newline
2. इत्या॑हा॒हे तीत्या॑ह । \newline
3. आ॒ह॒ प्रा॒णा॒पा॒नौ प्रा॑णापा॒ना वा॑हाह प्राणापा॒नौ । \newline
4. प्रा॒णा॒पा॒ना वे॒वैव प्रा॑णापा॒नौ प्रा॑णापा॒ना वे॒व । \newline
5. प्रा॒णा॒पा॒नाविति॑ प्राण - अ॒पा॒नौ । \newline
6. ए॒वावा वै॒वै वाव॑ । \newline
7. अव॑ रुन्धे रु॒न्धे ऽवाव॑ रुन्धे । \newline
8. रु॒न्धे॒ चक्षु॒ श्चक्षू॑ रुन्धे रुन्धे॒ चक्षुः॑ । \newline
9. चक्षु॑ रस्यसि॒ चक्षु॒ श्चक्षु॑ रसि । \newline
10. अ॒सि॒ श्रोत्रꣳ॒॒ श्रोत्र॑ मस्यसि॒ श्रोत्र᳚म् । \newline
11. श्रोत्र॒म् नाम॒ नाम॒ श्रोत्रꣳ॒॒ श्रोत्र॒म् नाम॑ । \newline
12. नामे तीति॒ नाम॒ नामे ति॑ । \newline
13. इत्या॑हा॒हे तीत्या॑ह । \newline
14. आ॒हायु॒ रायु॑ राहा॒हायुः॑ । \newline
15. आयु॑ रे॒वैवायु॒ रायु॑ रे॒व । \newline
16. ए॒वावा वै॒वै वाव॑ । \newline
17. अव॑ रुन्धे रु॒न्धे ऽवाव॑ रुन्धे । \newline
18. रु॒न्धे॒ रू॒पꣳ रू॒पꣳ रु॑न्धे रुन्धे रू॒पम् । \newline
19. रू॒प म॑स्यसि रू॒पꣳ रू॒प म॑सि । \newline
20. अ॒सि॒ वर्णो॒ वर्णो᳚ ऽस्यसि॒ वर्णः॑ । \newline
21. वर्णो॒ नाम॒ नाम॒ वर्णो॒ वर्णो॒ नाम॑ । \newline
22. नामे तीति॒ नाम॒ नामे ति॑ । \newline
23. इत्या॑हा॒हे तीत्या॑ह । \newline
24. आ॒ह॒ प्र॒जाम् प्र॒जा मा॑हाह प्र॒जाम् । \newline
25. प्र॒जा मे॒वैव प्र॒जाम् प्र॒जा मे॒व । \newline
26. प्र॒जामिति॑ प्र - जाम् । \newline
27. ए॒वावा वै॒वै वाव॑ । \newline
28. अव॑ रुन्धे रु॒न्धे ऽवाव॑ रुन्धे । \newline
29. रु॒न्ध॒ ऋ॒त मृ॒तꣳ रु॑न्धे रुन्ध ऋ॒तम् । \newline
30. ऋ॒त म॑स्य स्यृ॒त मृ॒त म॑सि । \newline
31. अ॒सि॒ स॒त्यꣳ स॒त्य म॑स्यसि स॒त्यम् । \newline
32. स॒त्यम् नाम॒ नाम॑ स॒त्यꣳ स॒त्यम् नाम॑ । \newline
33. नामे तीति॒ नाम॒ नामे ति॑ । \newline
34. इत्या॑हा॒हे तीत्या॑ह । \newline
35. आ॒ह॒ क्ष॒त्रम् क्ष॒त्र मा॑हाह क्ष॒त्रम् । \newline
36. क्ष॒त्र मे॒वैव क्ष॒त्रम् क्ष॒त्र मे॒व । \newline
37. ए॒वावा वै॒वै वाव॑ । \newline
38. अव॑ रुन्धे रु॒न्धे ऽवाव॑ रुन्धे । \newline
39. रु॒न्धे॒ भू॒तम् भू॒तꣳ रु॑न्धे रुन्धे भू॒तम् । \newline
40. भू॒त म॑स्यसि भू॒तम् भू॒त म॑सि । \newline
41. अ॒सि॒ भव्य॒म् भव्य॑ मस्यसि॒ भव्य᳚म् । \newline
42. भव्य॒म् नाम॒ नाम॒ भव्य॒म् भव्य॒म् नाम॑ । \newline
43. नामे तीति॒ नाम॒ नामे ति॑ । \newline
44. इत्या॑हा॒हे तीत्या॑ह । \newline
45. आ॒ह॒ प॒शवः॑ प॒शव॑ आहाह प॒शवः॑ । \newline
46. प॒शवो॒ वै वै प॒शवः॑ प॒शवो॒ वै । \newline
47. वा अ॒पा म॒पाम् ॅवै वा अ॒पाम् । \newline
48. अ॒पा मोष॑धीना॒ मोष॑धीना म॒पा म॒पा मोष॑धीनाम् । \newline
49. ओष॑धीना॒म् गर्भो॒ गर्भ॒ ओष॑धीना॒ मोष॑धीना॒म् गर्भः॑ । \newline
50. गर्भः॑ प॒शून् प॒शून् गर्भो॒ गर्भः॑ प॒शून् । \newline
51. प॒शू ने॒वैव प॒शून् प॒शू ने॒व । \newline
52. ए॒वावा वै॒वै वाव॑ । \newline

\textbf{Ghana Paata } \newline

1. नामे तीति॒ नाम॒ नामे त्या॑हा॒हे ति॒ नाम॒ नामे त्या॑ह । \newline
2. इत्या॑हा॒हे तीत्या॑ह प्राणापा॒नौ प्रा॑णापा॒ना वा॒हे तीत्या॑ह प्राणापा॒नौ । \newline
3. आ॒ह॒ प्रा॒णा॒पा॒नौ प्रा॑णापा॒ना वा॑हाह प्राणापा॒ना वे॒वैव प्रा॑णापा॒ना वा॑हाह प्राणापा॒ना वे॒व । \newline
4. प्रा॒णा॒पा॒ना वे॒वैव प्रा॑णापा॒नौ प्रा॑णापा॒ना वे॒वावा वै॒व प्रा॑णापा॒नौ प्रा॑णापा॒ना वे॒वाव॑ । \newline
5. प्रा॒णा॒पा॒नाविति॑ प्राण - अ॒पा॒नौ । \newline
6. ए॒वावा वै॒वै वाव॑ रुन्धे रु॒न्धे ऽवै॒वै वाव॑ रुन्धे । \newline
7. अव॑ रुन्धे रु॒न्धे ऽवाव॑ रुन्धे॒ चक्षु॒ श्चक्षू॑ रु॒न्धे ऽवाव॑ रुन्धे॒ चक्षुः॑ । \newline
8. रु॒न्धे॒ चक्षु॒ श्चक्षू॑ रुन्धे रुन्धे॒ चक्षु॑ रस्यसि॒ चक्षू॑ रुन्धे रुन्धे॒ चक्षु॑ रसि । \newline
9. चक्षु॑ रस्यसि॒ चक्षु॒ श्चक्षु॑ रसि॒ श्रोत्रꣳ॒॒ श्रोत्र॑ मसि॒ चक्षु॒ श्चक्षु॑ रसि॒ श्रोत्र᳚म् । \newline
10. अ॒सि॒ श्रोत्रꣳ॒॒ श्रोत्र॑ मस्यसि॒ श्रोत्र॒म् नाम॒ नाम॒ श्रोत्र॑ मस्यसि॒ श्रोत्र॒म् नाम॑ । \newline
11. श्रोत्र॒म् नाम॒ नाम॒ श्रोत्रꣳ॒॒ श्रोत्र॒म् नामे तीति॒ नाम॒ श्रोत्रꣳ॒॒ श्रोत्र॒म् नामे ति॑ । \newline
12. नामे तीति॒ नाम॒ नामे त्या॑हा॒हे ति॒ नाम॒ नामे त्या॑ह । \newline
13. इत्या॑ हा॒हे तीत्या॒ हायु॒ रायु॑रा॒हे तीत्या॒ हायुः॑ । \newline
14. आ॒हायु॒ रायु॑ राहा॒हायु॑ रे॒वैवायु॑ राहा॒हायु॑ रे॒व । \newline
15. आयु॑ रे॒वैवायु॒ रायु॑ रे॒वावावै॒ वायु॒ रायु॑ रे॒वाव॑ । \newline
16. ए॒वावा वै॒वैवाव॑ रुन्धे रु॒न्धे ऽवै॒वैवाव॑ रुन्धे । \newline
17. अव॑ रुन्धे रु॒न्धे ऽवाव॑ रुन्धे रू॒पꣳ रू॒पꣳ रु॒न्धे ऽवाव॑ रुन्धे रू॒पम् । \newline
18. रु॒न्धे॒ रू॒पꣳ रू॒पꣳ रु॑न्धे रुन्धे रू॒प म॑स्यसि रू॒पꣳ रु॑न्धे रुन्धे रू॒प म॑सि । \newline
19. रू॒प म॑स्यसि रू॒पꣳ रू॒प म॑सि॒ वर्णो॒ वर्णो॑ ऽसि रू॒पꣳ रू॒प म॑सि॒ वर्णः॑ । \newline
20. अ॒सि॒ वर्णो॒ वर्णो᳚ ऽस्यसि॒ वर्णो॒ नाम॒ नाम॒ वर्णो᳚ ऽस्यसि॒ वर्णो॒ नाम॑ । \newline
21. वर्णो॒ नाम॒ नाम॒ वर्णो॒ वर्णो॒ नामे तीति॒ नाम॒ वर्णो॒ वर्णो॒ नामे ति॑ । \newline
22. नामे तीति॒ नाम॒ नामे त्या॑हा॒हे ति॒ नाम॒ नामे त्या॑ह । \newline
23. इत्या॑हा॒हे तीत्या॑ह प्र॒जाम् प्र॒जा मा॒हे तीत्या॑ह प्र॒जाम् । \newline
24. आ॒ह॒ प्र॒जाम् प्र॒जा मा॑हाह प्र॒जा मे॒वैव प्र॒जा मा॑हाह प्र॒जा मे॒व । \newline
25. प्र॒जा मे॒वैव प्र॒जाम् प्र॒जा मे॒वावा वै॒व प्र॒जाम् प्र॒जा मे॒वाव॑ । \newline
26. प्र॒जामिति॑ प्र - जाम् । \newline
27. ए॒वावा वै॒वैवाव॑ रुन्धे रु॒न्धे ऽवै॒वैवाव॑ रुन्धे । \newline
28. अव॑ रुन्धे रु॒न्धे ऽवाव॑ रुन्ध ऋ॒त मृ॒तꣳ रु॒न्धे ऽवाव॑ रुन्ध ऋ॒तम् । \newline
29. रु॒न्ध॒ ऋ॒त मृ॒तꣳ रु॑न्धे रुन्ध ऋ॒त म॑स्य स्यृ॒तꣳ रु॑न्धे रुन्ध ऋ॒त म॑सि । \newline
30. ऋ॒त म॑स्यस्यृ॒त मृ॒त म॑सि स॒त्यꣳ स॒त्य म॑स्यृ॒त मृ॒त म॑सि स॒त्यम् । \newline
31. अ॒सि॒ स॒त्यꣳ स॒त्य म॑स्यसि स॒त्यम् नाम॒ नाम॑ स॒त्य म॑स्यसि स॒त्यम् नाम॑ । \newline
32. स॒त्यम् नाम॒ नाम॑ स॒त्यꣳ स॒त्यम् नामे तीति॒ नाम॑ स॒त्यꣳ स॒त्यम् नामे ति॑ । \newline
33. नामे तीति॒ नाम॒ नामे त्या॑हा॒हे ति॒ नाम॒ नामे त्या॑ह । \newline
34. इत्या॑हा॒हे तीत्या॑ह क्ष॒त्रम् क्ष॒त्र मा॒हे तीत्या॑ह क्ष॒त्रम् । \newline
35. आ॒ह॒ क्ष॒त्रम् क्ष॒त्र मा॑हाह क्ष॒त्र मे॒वैव क्ष॒त्र मा॑हाह क्ष॒त्र मे॒व । \newline
36. क्ष॒त्र मे॒वैव क्ष॒त्रम् क्ष॒त्र मे॒वावा वै॒व क्ष॒त्रम् क्ष॒त्र मे॒वाव॑ । \newline
37. ए॒वावावै॒ वैवाव॑ रुन्धे रु॒न्धे ऽवै॒वैवाव॑ रुन्धे । \newline
38. अव॑ रुन्धे रु॒न्धे ऽवाव॑ रुन्धे भू॒तम् भू॒तꣳ रु॒न्धे ऽवाव॑ रुन्धे भू॒तम् । \newline
39. रु॒न्धे॒ भू॒तम् भू॒तꣳ रु॑न्धे रुन्धे भू॒त म॑स्यसि भू॒तꣳ रु॑न्धे रुन्धे भू॒त म॑सि । \newline
40. भू॒त म॑स्यसि भू॒तम् भू॒त म॑सि॒ भव्य॒म् भव्य॑ मसि भू॒तम् भू॒त म॑सि॒ भव्य᳚म् । \newline
41. अ॒सि॒ भव्य॒म् भव्य॑ मस्यसि॒ भव्य॒म् नाम॒ नाम॒ भव्य॑ मस्यसि॒ भव्य॒म् नाम॑ । \newline
42. भव्य॒म् नाम॒ नाम॒ भव्य॒म् भव्य॒म् नामे तीति॒ नाम॒ भव्य॒म् भव्य॒म् नामे ति॑ । \newline
43. नामे तीति॒ नाम॒ नामे त्या॑हा॒हे ति॒ नाम॒ नामे त्या॑ह । \newline
44. इत्या॑हा॒हे तीत्या॑ह प॒शवः॑ प॒शव॑ आ॒हे तीत्या॑ह प॒शवः॑ । \newline
45. आ॒ह॒ प॒शवः॑ प॒शव॑ आहाह प॒शवो॒ वै वै प॒शव॑ आहाह प॒शवो॒ वै । \newline
46. प॒शवो॒ वै वै प॒शवः॑ प॒शवो॒ वा अ॒पा म॒पाम् ॅवै प॒शवः॑ प॒शवो॒ वा अ॒पाम् । \newline
47. वा अ॒पा म॒पाम् ॅवै वा अ॒पा मोष॑धीना॒ मोष॑धीना म॒पाम् ॅवै वा अ॒पा मोष॑धीनाम् । \newline
48. अ॒पा मोष॑धीना॒ मोष॑धीना म॒पा म॒पा मोष॑धीना॒म् गर्भो॒ गर्भ॒ ओष॑धीना म॒पा म॒पा मोष॑धीना॒म् गर्भः॑ । \newline
49. ओष॑धीना॒म् गर्भो॒ गर्भ॒ ओष॑धीना॒ मोष॑धीना॒म् गर्भः॑ प॒शून् प॒शून् गर्भ॒ ओष॑धीना॒ मोष॑धीना॒म् गर्भः॑ प॒शून् । \newline
50. गर्भः॑ प॒शून् प॒शून् गर्भो॒ गर्भः॑ प॒शू ने॒वैव प॒शून् गर्भो॒ गर्भः॑ प॒शू ने॒व । \newline
51. प॒शू ने॒वैव प॒शून् प॒शू ने॒वावा वै॒व प॒शून् प॒शू ने॒वाव॑ । \newline
52. ए॒वावा वै॒वैवाव॑ रुन्धे रु॒न्धे ऽवै॒वैवाव॑ रुन्धे । \newline
\pagebreak
\markright{ TS 3.3.5.4  \hfill https://www.vedavms.in \hfill}

\section{ TS 3.3.5.4 }

\textbf{TS 3.3.5.4 } \newline
\textbf{Samhita Paata} \newline

-व॑ रुन्ध ए॒ताव॒द्वै पुरु॑षं प॒रित॒स्तदे॒वाव॑ रुन्ध ऋ॒तस्य॑ त्वा॒ व्यो॑मन॒ इत्या॑हे॒यं ॅवा ऋ॒तस्य॒ व्यो॑मे॒मामे॒वाभि ज॑यत्यृ॒तस्य॑ त्वा॒ विभू॑मन॒ इत्या॑हा॒ऽन्तरि॑क्षं॒ ॅवा ऋ॒तस्य॒ विभू॑मा॒न्तरि॑क्षमे॒वाभि ज॑यत्यृ॒तस्य॑ त्वा॒ विध॑र्मण॒ इत्या॑ह॒ द्यौर्वा ऋ॒तस्य॒ विध॑र्म॒ दिव॑मे॒वाभि ज॑यत्यृ॒तस्य॑- [  ] \newline

\textbf{Pada Paata} \newline

अवेति॑ । रु॒न्धे॒ । ए॒ताव॑त् । वै । पुरु॑षम् । प॒रितः॑ । तत् । ए॒व । अवेति॑ । रु॒न्धे॒ । ऋ॒तस्य॑ । त्वा॒ । व्यो॑मन॒ इति॒ वि - ओ॒म॒ने॒ । इति॑ । आ॒ह॒ । इ॒यम् । वै । ऋ॒तस्य॑ । व्यो॑मेति॒ वि - ओ॒म॒ । इ॒माम् । ए॒व । अ॒भीति॑ । ज॒य॒ति॒ । ऋ॒तस्य॑ । त्वा॒ । विभू॑मन॒ इति॒ वि-भू॒म॒ने॒ । इति॑ । आ॒ह॒ । अ॒न्तरि॑क्षम् । वै । ऋ॒तस्य॑ । विभू॒मेति॒ वि - भू॒म॒ । अ॒न्तरि॑क्षम् । ए॒व । अ॒भीति॑ । ज॒य॒ति॒ । ऋ॒तस्य॑ । त्वा॒ । विध॑र्मण॒ इति॒ वि - ध॒र्म॒णे॒ । इति॑ । आ॒ह॒ । द्यौः । वै । ऋ॒तस्य॑ । विध॒र्मेति॒ वि -ध॒र्म॒ । दिव᳚म् । ए॒व । अ॒भीति॑ । ज॒य॒ति॒ । ऋ॒तस्य॑ ।  \newline


\textbf{Krama Paata} \newline

अव॑ रुन्धे । रु॒न्ध॒ ए॒ताव॑त् । ए॒ताव॒द् वै । वै पुरु॑षम् । पुरु॑षम् प॒रितः॑ । प॒रित॒स्तत् । तदे॒व । ए॒वाव॑ । अव॑ रुन्धे । रु॒न्ध॒ ऋ॒तस्य॑ । ऋ॒तस्य॑ त्वा । त्वा॒ व्यो॑मने । व्यो॑मन॒ इति॑ । व्यो॑मन॒ इति॒ वि - ओ॒म॒ने॒ । इत्या॑ह । आ॒हे॒यम् । इ॒यं ॅवै । वा ऋ॒तस्य॑ । ऋ॒तस्य॒ व्यो॑म । व्यो॑मे॒माम् । व्यो॑मेति॒ वि - ओ॒म॒ । इ॒मामे॒व । ए॒वाभि । अ॒भि ज॑यति । ज॒य॒त्यृ॒तस्य॑ । ऋ॒तस्य॑ त्वा । त्वा॒ विभू॑मने । विभू॑मन॒ इति॑ । विभू॑मन॒ इति॒ वि - भू॒म॒ने॒ । इत्या॑ह । आ॒हा॒न्तरि॑क्षम् । अ॒न्तरि॑क्षं॒ ॅवै । वा ऋ॒तस्य॑ । ऋ॒तस्य॒ विभू॑म । विभू॑मा॒न्तरि॑क्षम् । विभू॒मेति॒ वि - भू॒म॒ । अ॒न्तरि॑क्षमे॒व । ए॒वाभि । अ॒भि ज॑यति । ज॒य॒त्यृ॒तस्य॑ । ऋ॒तस्य॑ त्वा । त्वा॒ विध॑र्मणे । विध॑र्मण॒ इति॑ । विध॑र्मण॒ इति॒ वि - ध॒र्म॒णे॒ । इत्या॑ह । आ॒ह॒ द्यौः । द्यौर् वै । वा ऋ॒तस्य॑ । ऋ॒तस्य॒ विध॑र्म । विध॑र्म॒ दिव᳚म् । विध॒र्मेति॒ वि - ध॒र्म॒ । दिव॑मे॒व । ए॒वाभि । अ॒भि ज॑यति । ज॒य॒त्यृ॒तस्य॑ ( ) । ऋ॒तस्य॑ त्वा \newline

\textbf{Jatai Paata} \newline

1. अव॑ रुन्धे रु॒न्धे ऽवाव॑ रुन्धे । \newline
2. रु॒न्ध॒ ए॒ताव॑ दे॒ताव॑द् रुन्धे रुन्ध ए॒ताव॑त् । \newline
3. ए॒ताव॒द् वै वा ए॒ताव॑ दे॒ताव॒द् वै । \newline
4. वै पुरु॑ष॒म् पुरु॑ष॒म् ॅवै वै पुरु॑षम् । \newline
5. पुरु॑षम् प॒रितः॑ प॒रितः॒ पुरु॑ष॒म् पुरु॑षम् प॒रितः॑ । \newline
6. प॒रित॒ स्तत् तत् प॒रितः॑ प॒रित॒ स्तत् । \newline
7. तदे॒वैव तत् तदे॒व । \newline
8. ए॒वावा वै॒वै वाव॑ । \newline
9. अव॑ रुन्धे रु॒न्धे ऽवाव॑ रुन्धे । \newline
10. रु॒न्ध॒ ऋ॒तस्य॒ र्तस्य॑ रुन्धे रुन्ध ऋ॒तस्य॑ । \newline
11. ऋ॒तस्य॑ त्वा त्व॒ र्तस्य॒ र्तस्य॑ त्वा । \newline
12. त्वा॒ व्यो॑मने॒ व्यो॑मने त्वा त्वा॒ व्यो॑मने । \newline
13. व्यो॑मन॒ इतीति॒ व्यो॑मने॒ व्यो॑मन॒ इति॑ । \newline
14. व्यो॑मन॒ इति॒ वि - ओ॒म॒ने॒ । \newline
15. इत्या॑हा॒हे तीत्या॑ह । \newline
16. आ॒हे॒ य मि॒य मा॑हाहे॒ यम् । \newline
17. इ॒यम् ॅवै वा इ॒य मि॒यम् ॅवै । \newline
18. वा ऋ॒तस्य॒ र्तस्य॒ वै वा ऋ॒तस्य॑ । \newline
19. ऋ॒तस्य॒ व्यो॑म॒ व्यो॑म॒ र्तस्य॒ र्तस्य॒ व्यो॑म । \newline
20. व्यो॑मे॒ मा मि॒माम् ॅव्यो॑म॒ व्यो॑मे॒ माम् । \newline
21. व्यो॑मेति॒ वि - ओ॒म॒ । \newline
22. इ॒मा मे॒वैवे मा मि॒मा मे॒व । \newline
23. ए॒वाभ्या᳚(1॒)भ्ये॑वैवाभि । \newline
24. अ॒भि ज॑यति जय त्य॒भ्य॑भि ज॑यति । \newline
25. ज॒य॒ त्यृ॒तस्य॒ र्तस्य॑ जयति जय त्यृ॒तस्य॑ । \newline
26. ऋ॒तस्य॑ त्वा त्व॒ र्तस्य॒ र्तस्य॑ त्वा । \newline
27. त्वा॒ विभू॑मने॒ विभू॑मने त्वा त्वा॒ विभू॑मने । \newline
28. विभू॑मन॒ इतीति॒ विभू॑मने॒ विभू॑मन॒ इति॑ । \newline
29. विभू॑मन॒ इति॒ वि - भू॒म॒ने॒ । \newline
30. इत्या॑हा॒हे तीत्या॑ह । \newline
31. आ॒हा॒ न्तरि॑क्ष म॒न्तरि॑क्ष माहाहा॒ न्तरि॑क्षम् । \newline
32. अ॒न्तरि॑क्ष॒म् ॅवै वा अ॒न्तरि॑क्ष म॒न्तरि॑क्ष॒म् ॅवै । \newline
33. वा ऋ॒तस्य॒ र्तस्य॒ वै वा ऋ॒तस्य॑ । \newline
34. ऋ॒तस्य॒ विभू॑म॒ विभू॑म॒ र्तस्य॒ र्तस्य॒ विभू॑म । \newline
35. विभू॑मा॒ न्तरि॑क्ष म॒न्तरि॑क्ष॒म् ॅविभू॑म॒ विभू॑मा॒ न्तरि॑क्षम् । \newline
36. विभू॒मेति॒ वि - भू॒म॒ । \newline
37. अ॒न्तरि॑क्ष मे॒वैवा न्तरि॑क्ष म॒न्तरि॑क्ष मे॒व । \newline
38. ए॒वाभ्या᳚(1॒)भ्ये॑वैवाभि । \newline
39. अ॒भि ज॑यति जय त्य॒भ्य॑भि ज॑यति । \newline
40. ज॒य॒ त्यृ॒तस्य॒ र्तस्य॑ जयति जय त्यृ॒तस्य॑ । \newline
41. ऋ॒तस्य॑ त्वा त्व॒ र्तस्य॒ र्तस्य॑ त्वा । \newline
42. त्वा॒ विध॑र्मणे॒ विध॑र्मणे त्वा त्वा॒ विध॑र्मणे । \newline
43. विध॑र्मण॒ इतीति॒ विध॑र्मणे॒ विध॑र्मण॒ इति॑ । \newline
44. विध॑र्मण॒ इति॒ वि - ध॒र्म॒णे॒ । \newline
45. इत्या॑हा॒हे तीत्या॑ह । \newline
46. आ॒ह॒ द्यौर् द्यौ रा॑हाह॒ द्यौः । \newline
47. द्यौर् वै वै द्यौर् द्यौर् वै । \newline
48. वा ऋ॒तस्य॒ र्तस्य॒ वै वा ऋ॒तस्य॑ । \newline
49. ऋ॒तस्य॒ विध॑र्म॒ विध॑र्म॒ र्तस्य॒ र्तस्य॒ विध॑र्म । \newline
50. विध॑र्म॒ दिव॒म् दिव॒म् ॅविध॑र्म॒ विध॑र्म॒ दिव᳚म् । \newline
51. विध॒र्मेति॒ वि - ध॒र्म॒ । \newline
52. दिव॑ मे॒वैव दिव॒म् दिव॑ मे॒व । \newline
53. ए॒वाभ्या᳚(1॒)भ्ये॑वैवाभि । \newline
54. अ॒भि ज॑यति जय त्य॒भ्य॑भि ज॑यति । \newline
55. ज॒य॒ त्यृ॒तस्य॒ र्तस्य॑ जयति जय त्यृ॒तस्य॑ । \newline
56. ऋ॒तस्य॑ त्वा त्व॒ र्तस्य॒ र्तस्य॑ त्वा । \newline

\textbf{Ghana Paata } \newline

1. अव॑ रुन्धे रु॒न्धे ऽवाव॑ रुन्ध ए॒ताव॑ दे॒ताव॑द् रु॒न्धे ऽवाव॑ रुन्ध ए॒ताव॑त् । \newline
2. रु॒न्ध॒ ए॒ताव॑ दे॒ताव॑द् रुन्धे रुन्ध ए॒ताव॒द् वै वा ए॒ताव॑द् रुन्धे रुन्ध ए॒ताव॒द् वै । \newline
3. ए॒ताव॒द् वै वा ए॒ताव॑ दे॒ताव॒द् वै पुरु॑ष॒म् पुरु॑ष॒म् ॅवा ए॒ताव॑ दे॒ताव॒द् वै पुरु॑षम् । \newline
4. वै पुरु॑ष॒म् पुरु॑ष॒म् ॅवै वै पुरु॑षम् प॒रितः॑ प॒रितः॒ पुरु॑ष॒म् ॅवै वै पुरु॑षम् प॒रितः॑ । \newline
5. पुरु॑षम् प॒रितः॑ प॒रितः॒ पुरु॑ष॒म् पुरु॑षम् प॒रित॒ स्तत् तत् प॒रितः॒ पुरु॑ष॒म् पुरु॑षम् प॒रित॒ स्तत् । \newline
6. प॒रित॒ स्तत् तत् प॒रितः॑ प॒रित॒ स्तदे॒वैव तत् प॒रितः॑ प॒रित॒ स्तदे॒व । \newline
7. तदे॒वैव तत् तदे॒वावा वै॒व तत् तदे॒वाव॑ । \newline
8. ए॒वावा वै॒वैवाव॑ रुन्धे रु॒न्धे ऽवै॒वैवाव॑ रुन्धे । \newline
9. अव॑ रुन्धे रु॒न्धे ऽवाव॑ रुन्ध ऋ॒तस्य॒ र्तस्य॑ रु॒न्धे ऽवाव॑ रुन्ध ऋ॒तस्य॑ । \newline
10. रु॒न्ध॒ ऋ॒तस्य॒ र्तस्य॑ रुन्धे रुन्ध ऋ॒तस्य॑ त्वा त्व॒र्तस्य॑ रुन्धे रुन्ध ऋ॒तस्य॑ त्वा । \newline
11. ऋ॒तस्य॑ त्वा त्व॒र्तस्य॒ र्तस्य॑ त्वा॒ व्यो॑मने॒ व्यो॑मने त्व॒र्तस्य॒ र्तस्य॑ त्वा॒ व्यो॑मने । \newline
12. त्वा॒ व्यो॑मने॒ व्यो॑मने त्वा त्वा॒ व्यो॑मन॒ इतीति॒ व्यो॑मने त्वा त्वा॒ व्यो॑मन॒ इति॑ । \newline
13. व्यो॑मन॒ इतीति॒ व्यो॑मने॒ व्यो॑मन॒ इत्या॑हा॒हे ति॒ व्यो॑मने॒ व्यो॑मन॒ इत्या॑ह । \newline
14. व्यो॑मन॒ इति॒ वि - ओ॒म॒ने॒ । \newline
15. इत्या॑हा॒हे तीत्या॑हे॒ य मि॒य मा॒हे तीत्या॑हे॒ यम् । \newline
16. आ॒हे॒ य मि॒य मा॑हाहे॒ यम् ॅवै वा इ॒य मा॑हाहे॒ यम् ॅवै । \newline
17. इ॒यम् ॅवै वा इ॒य मि॒यम् ॅवा ऋ॒तस्य॒ र्‌तस्य॒ वा इ॒य मि॒यम् ॅवा ऋ॒तस्य॑ । \newline
18. वा ऋ॒तस्य॒ र्‌तस्य॒ वै वा ऋ॒तस्य॒ व्यो॑म॒ व्यो॑म॒ र्‌तस्य॒ वै वा ऋ॒तस्य॒ व्यो॑म । \newline
19. ऋ॒तस्य॒ व्यो॑म॒ व्यो॑म॒ र्तस्य॒ र्तस्य॒ व्यो॑मे॒ मा मि॒माम् ॅव्यो॑म॒ र्तस्य॒ र्तस्य॒ व्यो॑मे॒ माम् । \newline
20. व्यो॑मे॒ मा मि॒माम् ॅव्यो॑म॒ व्यो॑मे॒ मा मे॒वैवे माम् ॅव्यो॑म॒ व्यो॑मे॒ मा मे॒व । \newline
21. व्यो॑मेति॒ वि - ओ॒म॒ । \newline
22. इ॒मा मे॒वैवे मा मि॒मा मे॒वाभ्या᳚(1॒)भ्ये॑वे मा मि॒मा मे॒वाभि । \newline
23. ए॒वाभ्या᳚(1॒)भ्ये॑ वैवाभि ज॑यति जय त्य॒भ्ये॑ वैवाभि ज॑यति । \newline
24. अ॒भि ज॑यति जय त्य॒भ्य॑भि ज॑य त्यृ॒तस्य॒ र्तस्य॑ जय त्य॒भ्य॑भि ज॑य त्यृ॒तस्य॑ । \newline
25. ज॒य॒ त्यृ॒तस्य॒ र्तस्य॑ जयति जय त्यृ॒तस्य॑ त्वा त्व॒र्तस्य॑ जयति जय त्यृ॒तस्य॑ त्वा । \newline
26. ऋ॒तस्य॑ त्वा त्व॒र्तस्य॒ र्‌तस्य॑ त्वा॒ विभू॑मने॒ विभू॑मने त्व॒र्तस्य॒ र्‌तस्य॑ त्वा॒ विभू॑मने । \newline
27. त्वा॒ विभू॑मने॒ विभू॑मने त्वा त्वा॒ विभू॑मन॒ इतीति॒ विभू॑मने त्वा त्वा॒ विभू॑मन॒ इति॑ । \newline
28. विभू॑मन॒ इतीति॒ विभू॑मने॒ विभू॑मन॒ इत्या॑हा॒हे ति॒ विभू॑मने॒ विभू॑मन॒ इत्या॑ह । \newline
29. विभू॑मन॒ इति॒ वि - भू॒म॒ने॒ । \newline
30. इत्या॑हा॒हे तीत्या॑हा॒ न्तरि॑क्ष म॒न्तरि॑क्ष मा॒हे तीत्या॑हा॒ न्तरि॑क्षम् । \newline
31. आ॒हा॒न्तरि॑क्ष म॒न्तरि॑क्ष माहाहा॒ न्तरि॑क्ष॒म् ॅवै वा अ॒न्तरि॑क्ष माहाहा॒ न्तरि॑क्ष॒म् ॅवै । \newline
32. अ॒न्तरि॑क्ष॒म् ॅवै वा अ॒न्तरि॑क्ष म॒न्तरि॑क्ष॒म् ॅवा ऋ॒तस्य॒ र्तस्य॒ वा अ॒न्तरि॑क्ष म॒न्तरि॑क्ष॒म् ॅवा ऋ॒तस्य॑ । \newline
33. वा ऋ॒तस्य॒ र्‌तस्य॒ वै वा ऋ॒तस्य॒ विभू॑म॒ विभू॑म॒ र्‌तस्य॒ वै वा ऋ॒तस्य॒ विभू॑म । \newline
34. ऋ॒तस्य॒ विभू॑म॒ विभू॑म॒ र्तस्य॒ र्तस्य॒ विभू॑मा॒ न्तरि॑क्ष म॒न्तरि॑क्ष॒म् ॅविभू॑म॒ र्‌तस्य॒ र्‌तस्य॒ विभू॑मा॒ न्तरि॑क्षम् । \newline
35. विभू॑मा॒ न्तरि॑क्ष म॒न्तरि॑क्ष॒म् ॅविभू॑म॒ विभू॑मा॒ न्तरि॑क्ष मे॒वैवा न्तरि॑क्ष॒म् ॅविभू॑म॒ विभू॑मा॒ न्तरि॑क्ष मे॒व । \newline
36. विभू॒मेति॒ वि - भू॒म॒ । \newline
37. अ॒न्तरि॑क्ष मे॒वैवा न्तरि॑क्ष म॒न्तरि॑क्ष मे॒वाभ्या᳚(1॒)भ्ये॑वा न्तरि॑क्ष म॒न्तरि॑क्ष मे॒वाभि । \newline
38. ए॒वाभ्या᳚(1॒)भ्ये॑ वैवाभि ज॑यति जय त्य॒भ्ये॑ वैवाभि ज॑यति । \newline
39. अ॒भि ज॑यति जय त्य॒भ्य॑भि ज॑य त्यृ॒तस्य॒ र्तस्य॑ जय त्य॒भ्य॑भि ज॑य त्यृ॒तस्य॑ । \newline
40. ज॒य॒ त्यृ॒तस्य॒ र्तस्य॑ जयति जय त्यृ॒तस्य॑ त्वा त्व॒र्तस्य॑ जयति जय त्यृ॒तस्य॑ त्वा । \newline
41. ऋ॒तस्य॑ त्वा त्व॒र्तस्य॒ र्तस्य॑ त्वा॒ विध॑र्मणे॒ विध॑र्मणे त्व॒र्तस्य॒ र्तस्य॑ त्वा॒ विध॑र्मणे । \newline
42. त्वा॒ विध॑र्मणे॒ विध॑र्मणे त्वा त्वा॒ विध॑र्मण॒ इतीति॒ विध॑र्मणे त्वा त्वा॒ विध॑र्मण॒ इति॑ । \newline
43. विध॑र्मण॒ इतीति॒ विध॑र्मणे॒ विध॑र्मण॒ इत्या॑हा॒हे ति॒ विध॑र्मणे॒ विध॑र्मण॒ इत्या॑ह । \newline
44. विध॑र्मण॒ इति॒ वि - ध॒र्म॒णे॒ । \newline
45. इत्या॑हा॒हे तीत्या॑ह॒ द्यौर् द्यौरा॒हे तीत्या॑ह॒ द्यौः । \newline
46. आ॒ह॒ द्यौर् द्यौ रा॑हाह॒ द्यौर् वै वै द्यौ रा॑हाह॒ द्यौर् वै । \newline
47. द्यौर् वै वै द्यौर् द्यौर् वा ऋ॒तस्य॒ र्तस्य॒ वै द्यौर् द्यौर् वा ऋ॒तस्य॑ । \newline
48. वा ऋ॒तस्य॒ र्‌तस्य॒ वै वा ऋ॒तस्य॒ विध॑र्म॒ विध॑र्म॒ र्‌तस्य॒ वै वा ऋ॒तस्य॒ विध॑र्म । \newline
49. ऋ॒तस्य॒ विध॑र्म॒ विध॑र्म॒ र्‌तस्य॒ र्तस्य॒ विध॑र्म॒ दिव॒म् दिव॒म् ॅविध॑र्म॒ र्‌तस्य॒ र्‌तस्य॒ विध॑र्म॒ दिव᳚म् । \newline
50. विध॑र्म॒ दिव॒म् दिव॒म् ॅविध॑र्म॒ विध॑र्म॒ दिव॑ मे॒वैव दिव॒म् ॅविध॑र्म॒ विध॑र्म॒ दिव॑ मे॒व । \newline
51. विध॒र्मेति॒ वि - ध॒र्म॒ । \newline
52. दिव॑ मे॒वैव दिव॒म् दिव॑ मे॒वाभ्या᳚(1॒)भ्ये॑व दिव॒म् दिव॑ मे॒वाभि । \newline
53. ए॒वाभ्या᳚(1॒)भ्ये॑ वैवाभि ज॑यति जय त्य॒भ्ये॑ वैवाभि ज॑यति । \newline
54. अ॒भि ज॑यति जय त्य॒भ्य॑भि ज॑य त्यृ॒तस्य॒ र्तस्य॑ जय त्य॒भ्य॑भि ज॑य त्यृ॒तस्य॑ । \newline
55. ज॒य॒ त्यृ॒तस्य॒ र्तस्य॑ जयति जय त्यृ॒तस्य॑ त्वा त्व॒र्तस्य॑ जयति जय त्यृ॒तस्य॑ त्वा । \newline
56. ऋ॒तस्य॑ त्वा त्व॒र्तस्य॒ र्तस्य॑ त्वा स॒त्याय॑ स॒त्याय॑ त्व॒र्तस्य॒ र्तस्य॑ त्वा स॒त्याय॑ । \newline
\pagebreak
\markright{ TS 3.3.5.5  \hfill https://www.vedavms.in \hfill}

\section{ TS 3.3.5.5 }

\textbf{TS 3.3.5.5 } \newline
\textbf{Samhita Paata} \newline

त्वा स॒त्यायेत्या॑ह॒ दिशो॒ वा ऋ॒तस्य॑ स॒त्यं दिश॑ ए॒वाभि ज॑यत्यृ॒तस्य॑ त्वा॒ ज्योति॑ष॒ इत्या॑ह सुव॒र्गो वै लो॒क ऋ॒तस्य॒ ज्योतिः॑ सुव॒र्गमे॒व लो॒कम॒भि ज॑यत्ये॒ताव॑न्तो॒ वै दे॑वलो॒कास्ताने॒वाभि ज॑यति॒ दश॒ संप॑द्यन्ते॒ दशा᳚क्षरा वि॒राडन्नं॑ ॅवि॒राड् वि॒राज्ये॒वान्नाद्ये॒ प्रति॑तिष्ठति ॥ \newline

\textbf{Pada Paata} \newline

त्वा॒ । स॒त्याय॑ । इति॑ । आ॒ह॒ । दिशः॑ । वै । ऋ॒तस्य॑ । स॒त्यम् । दिशः॑ । ए॒व । अ॒भीति॑ । ज॒य॒ति॒ । ऋ॒तस्य॑ । त्वा॒ । ज्योति॑षे । इति॑ । आ॒ह॒ । सु॒व॒र्ग इति॑ सुवः - गः । वै । लो॒कः । ऋ॒तस्य॑ । ज्योतिः॑ । सु॒व॒र्गमिति॑ सुवः - गम् । ए॒व । लो॒कम् । अ॒भीति॑ । ज॒य॒ति॒ । ए॒ताव॑न्तः । वै । दे॒व॒लो॒का इति॑ देव - लो॒काः । तान् । ए॒व । अ॒भीति॑ । ज॒य॒ति॒ । दश॑ । समिति॑ । प॒द्य॒न्ते॒ । दशा᳚क्ष॒रेति॒ दश॑ - अ॒क्ष॒रा॒ । वि॒राडिति॑ वि - राट् । अन्न᳚म् । वि॒राडिति॑ वि - राट् । वि॒राजीति॑ वि - राजि॑ । ए॒व । अ॒न्नाद्य॒ इत्य॑न्न - अद्ये᳚ । प्रतीति॑ । ति॒ष्ठ॒ति॒ ॥  \newline


\textbf{Krama Paata} \newline

त्वा॒ स॒त्याय॑ । स॒त्यायेति॑ । इत्या॑ह । आ॒ह॒ दिशः॑ । दिशो॒ वै । वा ऋ॒तस्य॑ । ऋ॒तस्य॑ स॒त्यम् । स॒त्यम् दिशः॑ । दिश॑ ए॒व । ए॒वाभि । अ॒भि ज॑यति । ज॒य॒त्यृ॒तस्य॑ । ऋ॒तस्य॑ त्वा । त्वा॒ ज्योति॑षे । ज्योति॑ष॒ इति॑ । इत्या॑ह । आ॒ह॒ सु॒व॒र्गः । सु॒व॒र्गो वै । सु॒व॒र्ग इति॑ सुवः - गः । वै लो॒कः । लो॒क ऋ॒तस्य॑ । ऋ॒तस्य॒ ज्योतिः॑ । ज्योतिः॑ सुव॒र्गम् । सु॒व॒र्गमे॒व । सु॒व॒र्गमिति॑ सुवः - गम् । ए॒व लो॒कम् । लो॒कम॒भि । अ॒भि ज॑यति । ज॒य॒त्ये॒ताव॑न्तः । ए॒ताव॑न्तो॒ वै । वै दे॑वलो॒काः । दे॒व॒लो॒कास्तान् । दे॒व॒लो॒का इति॑ देव - लो॒काः । ताने॒व । ए॒वाभि । अ॒भि ज॑यति । ज॒य॒ति॒ दश॑ । दश॒ सम् । सम् प॑द्यन्ते । प॒द्य॒न्ते॒ दशा᳚क्षरा । दशा᳚क्षरा वि॒राट् । दशा᳚क्ष॒रेति॒ दश॑ - अ॒क्ष॒रा॒ । वि॒राडन्न᳚म् । वि॒राडिति॑ वि - राट् । अन्नं॑ ॅवि॒राट् । वि॒राड् वि॒राजि॑ । वि॒राडिति॑ वि - राट् । वि॒राज्ये॒व । वि॒राजीति॑ वि - राजि॑ । ए॒वान्नाद्ये᳚ । अ॒न्नाद्ये॒ प्रति॑ । अ॒न्नाद्य॒ इत्य॑न्न - अद्ये᳚ । प्रति॑ तिष्ठति । ति॒ष्ठ॒तीति॑ तिष्ठति । \newline

\textbf{Jatai Paata} \newline

1. त्वा॒ स॒त्याय॑ स॒त्याय॑ त्वा त्वा स॒त्याय॑ । \newline
2. स॒त्याये तीति॑ स॒त्याय॑ स॒त्याये ति॑ । \newline
3. इत्या॑हा॒हे तीत्या॑ह । \newline
4. आ॒ह॒ दिशो॒ दिश॑ आहाह॒ दिशः॑ । \newline
5. दिशो॒ वै वै दिशो॒ दिशो॒ वै । \newline
6. वा ऋ॒तस्य॒ र्तस्य॒ वै वा ऋ॒तस्य॑ । \newline
7. ऋ॒तस्य॑ स॒त्यꣳ स॒त्य मृ॒तस्य॒ र्तस्य॑ स॒त्यम् । \newline
8. स॒त्यम् दिशो॒ दिशः॑ स॒त्यꣳ स॒त्यम् दिशः॑ । \newline
9. दिश॑ ए॒वैव दिशो॒ दिश॑ ए॒व । \newline
10. ए॒वाभ्या᳚(1॒)भ्ये॑ वैवाभि । \newline
11. अ॒भि ज॑यति जय त्य॒भ्य॑भि ज॑यति । \newline
12. ज॒य॒ त्यृ॒तस्य॒ र्तस्य॑ जयति जय त्यृ॒तस्य॑ । \newline
13. ऋ॒तस्य॑ त्वा त्व॒ र्तस्य॒ र्तस्य॑ त्वा । \newline
14. त्वा॒ ज्योति॑षे॒ ज्योति॑षे त्वा त्वा॒ ज्योति॑षे । \newline
15. ज्योति॑ष॒ इतीति॒ ज्योति॑षे॒ ज्योति॑ष॒ इति॑ । \newline
16. इत्या॑हा॒हे तीत्या॑ह । \newline
17. आ॒ह॒ सु॒व॒र्गः सु॑व॒र्ग आ॑हाह सुव॒र्गः । \newline
18. सु॒व॒र्गो वै वै सु॑व॒र्गः सु॑व॒र्गो वै । \newline
19. सु॒व॒र्ग इति॑ सुवः - गः । \newline
20. वै लो॒को लो॒को वै वै लो॒कः । \newline
21. लो॒क ऋ॒तस्य॒ र्तस्य॑ लो॒को लो॒क ऋ॒तस्य॑ । \newline
22. ऋ॒तस्य॒ ज्योति॒र् ज्योति॑र्. ऋ॒तस्य॒ र्तस्य॒ ज्योतिः॑ । \newline
23. ज्योतिः॑ सुव॒र्गꣳ सु॑व॒र्गम् ज्योति॒र् ज्योतिः॑ सुव॒र्गम् । \newline
24. सु॒व॒र्ग मे॒वैव सु॑व॒र्गꣳ सु॑व॒र्ग मे॒व । \newline
25. सु॒व॒र्गमिति॑ सुवः - गम् । \newline
26. ए॒व लो॒कम् ॅलो॒क मे॒वैव लो॒कम् । \newline
27. लो॒क म॒भ्य॑भि लो॒कम् ॅलो॒क म॒भि । \newline
28. अ॒भि ज॑यति जय त्य॒भ्य॑भि ज॑यति । \newline
29. ज॒य॒ त्ये॒ताव॑न्त ए॒ताव॑न्तो जयति जय त्ये॒ताव॑न्तः । \newline
30. ए॒ताव॑न्तो॒ वै वा ए॒ताव॑न्त ए॒ताव॑न्तो॒ वै । \newline
31. वै दे॑वलो॒का दे॑वलो॒का वै वै दे॑वलो॒काः । \newline
32. दे॒व॒लो॒का स्ताꣳ स्तान् दे॑वलो॒का दे॑वलो॒का स्तान् । \newline
33. दे॒व॒लो॒का इति॑ देव - लो॒काः । \newline
34. ता ने॒वैव ताꣳ स्ता ने॒व । \newline
35. ए॒वाभ्या᳚(1॒)भ्ये॑ वैवाभि । \newline
36. अ॒भि ज॑यति जय त्य॒भ्य॑भि ज॑यति । \newline
37. ज॒य॒ति॒ दश॒ दश॑ जयति जयति॒ दश॑ । \newline
38. दश॒ सꣳ सम् दश॒ दश॒ सम् । \newline
39. सम् प॑द्यन्ते पद्यन्ते॒ सꣳ सम् प॑द्यन्ते । \newline
40. प॒द्य॒न्ते॒ दशा᳚क्षरा॒ दशा᳚क्षरा पद्यन्ते पद्यन्ते॒ दशा᳚क्षरा । \newline
41. दशा᳚क्षरा वि॒राड् वि॒राड् दशा᳚क्षरा॒ दशा᳚क्षरा वि॒राट् । \newline
42. दशा᳚क्ष॒रेति॒ दश॑ - अ॒क्ष॒रा॒ । \newline
43. वि॒रा डन्न॒ मन्न॑म् ॅवि॒राड् वि॒रा डन्न᳚म् । \newline
44. वि॒राडिति॑ वि - राट् । \newline
45. अन्न॑म् ॅवि॒राड् वि॒रा डन्न॒ मन्न॑म् ॅवि॒राट् । \newline
46. वि॒राड् वि॒राजि॑ वि॒राजि॑ वि॒राड् वि॒राड् वि॒राजि॑ । \newline
47. वि॒राडिति॑ वि - राट् । \newline
48. वि॒रा ज्ये॒वैव वि॒राजि॑ वि॒रा ज्ये॒व । \newline
49. वि॒राजीति॑ वि - राजि॑ । \newline
50. ए॒वा न्नाद्ये॒ ऽन्नाद्य॑ ए॒वैवा न्नाद्ये᳚ । \newline
51. अ॒न्नाद्ये॒ प्रति॒ प्रत्य॒न्नाद्ये॒ ऽन्नाद्ये॒ प्रति॑ । \newline
52. अ॒न्नाद्य॒ इत्य॑न्न - अद्ये᳚ । \newline
53. प्रति॑ तिष्ठति तिष्ठति॒ प्रति॒ प्रति॑ तिष्ठति । \newline
54. ति॒ष्ठ॒तीति॑ तिष्ठति । \newline

\textbf{Ghana Paata } \newline

1. त्वा॒ स॒त्याय॑ स॒त्याय॑ त्वा त्वा स॒त्याये तीति॑ स॒त्याय॑ त्वा त्वा स॒त्याये ति॑ । \newline
2. स॒त्याये तीति॑ स॒त्याय॑ स॒त्याये त्या॑हा॒हे ति॑ स॒त्याय॑ स॒त्याये त्या॑ह । \newline
3. इत्या॑हा॒हे तीत्या॑ह॒ दिशो॒ दिश॑ आ॒हे तीत्या॑ह॒ दिशः॑ । \newline
4. आ॒ह॒ दिशो॒ दिश॑ आहाह॒ दिशो॒ वै वै दिश॑ आहाह॒ दिशो॒ वै । \newline
5. दिशो॒ वै वै दिशो॒ दिशो॒ वा ऋ॒तस्य॒ र्‌तस्य॒ वै दिशो॒ दिशो॒ वा ऋ॒तस्य॑ । \newline
6. वा ऋ॒तस्य॒ र्तस्य॒ वै वा ऋ॒तस्य॑ स॒त्यꣳ स॒त्य मृ॒तस्य॒ वै वा ऋ॒तस्य॑ स॒त्यम् । \newline
7. ऋ॒तस्य॑ स॒त्यꣳ स॒त्य मृ॒तस्य॒ र्‌तस्य॑ स॒त्यम् दिशो॒ दिशः॑ स॒त्य मृ॒तस्य॒ र्‌तस्य॑ स॒त्यम् दिशः॑ । \newline
8. स॒त्यम् दिशो॒ दिशः॑ स॒त्यꣳ स॒त्यम् दिश॑ ए॒वैव दिशः॑ स॒त्यꣳ स॒त्यम् दिश॑ ए॒व । \newline
9. दिश॑ ए॒वैव दिशो॒ दिश॑ ए॒वाभ्या᳚(1॒)भ्ये॑व दिशो॒ दिश॑ ए॒वाभि । \newline
10. ए॒वाभ्या᳚(1॒)भ्ये॑ वैवाभि ज॑यति जय त्य॒भ्ये॑ वैवाभि ज॑यति । \newline
11. अ॒भि ज॑यति जय त्य॒भ्य॑भि ज॑य त्यृ॒तस्य॒ र्तस्य॑ जय त्य॒भ्य॑भि ज॑य त्यृ॒तस्य॑ । \newline
12. ज॒य॒ त्यृ॒तस्य॒ र्तस्य॑ जयति जय त्यृ॒तस्य॑ त्वा त्व॒र्तस्य॑ जयति जय त्यृ॒तस्य॑ त्वा । \newline
13. ऋ॒तस्य॑ त्वा त्व॒र्तस्य॒ र्‌तस्य॑ त्वा॒ ज्योति॑षे॒ ज्योति॑षे त्व॒र्तस्य॒ र्‌तस्य॑ त्वा॒ ज्योति॑षे । \newline
14. त्वा॒ ज्योति॑षे॒ ज्योति॑षे त्वा त्वा॒ ज्योति॑ष॒ इतीति॒ ज्योति॑षे त्वा त्वा॒ ज्योति॑ष॒ इति॑ । \newline
15. ज्योति॑ष॒ इतीति॒ ज्योति॑षे॒ ज्योति॑ष॒ इत्या॑हा॒हे ति॒ ज्योति॑षे॒ ज्योति॑ष॒ इत्या॑ह । \newline
16. इत्या॑हा॒हे तीत्या॑ह सुव॒र्गः सु॑व॒र्ग आ॒हे तीत्या॑ह सुव॒र्गः । \newline
17. आ॒ह॒ सु॒व॒र्गः सु॑व॒र्ग आ॑हाह सुव॒र्गो वै वै सु॑व॒र्ग आ॑हाह सुव॒र्गो वै । \newline
18. सु॒व॒र्गो वै वै सु॑व॒र्गः सु॑व॒र्गो वै लो॒को लो॒को वै सु॑व॒र्गः सु॑व॒र्गो वै लो॒कः । \newline
19. सु॒व॒र्ग इति॑ सुवः - गः । \newline
20. वै लो॒को लो॒को वै वै लो॒क ऋ॒तस्य॒ र्तस्य॑ लो॒को वै वै लो॒क ऋ॒तस्य॑ । \newline
21. लो॒क ऋ॒तस्य॒ र्तस्य॑ लो॒को लो॒क ऋ॒तस्य॒ ज्योति॒र् ज्योति॑र्. ऋ॒तस्य॑ लो॒को लो॒क ऋ॒तस्य॒ ज्योतिः॑ । \newline
22. ऋ॒तस्य॒ ज्योति॒र् ज्योति॑र्. ऋ॒तस्य॒ र्तस्य॒ ज्योतिः॑ सुव॒र्गꣳ सु॑व॒र्गम् ज्योति॑र्. ऋ॒तस्य॒ र्तस्य॒ ज्योतिः॑ सुव॒र्गम् । \newline
23. ज्योतिः॑ सुव॒र्गꣳ सु॑व॒र्गम् ज्योति॒र् ज्योतिः॑ सुव॒र्ग मे॒वैव सु॑व॒र्गम् ज्योति॒र् ज्योतिः॑ सुव॒र्ग मे॒व । \newline
24. सु॒व॒र्ग मे॒वैव सु॑व॒र्गꣳ सु॑व॒र्ग मे॒व लो॒कम् ॅलो॒क मे॒व सु॑व॒र्गꣳ सु॑व॒र्ग मे॒व लो॒कम् । \newline
25. सु॒व॒र्गमिति॑ सुवः - गम् । \newline
26. ए॒व लो॒कम् ॅलो॒क मे॒वैव लो॒क म॒भ्य॑भि लो॒क मे॒वैव लो॒क म॒भि । \newline
27. लो॒क म॒भ्य॑भि लो॒कम् ॅलो॒क म॒भि ज॑यति जयत्य॒भि लो॒कम् ॅलो॒क म॒भि ज॑यति । \newline
28. अ॒भि ज॑यति जय त्य॒भ्य॑भि ज॑य त्ये॒ताव॑न्त ए॒ताव॑न्तो जय त्य॒भ्य॑भि ज॑य त्ये॒ताव॑न्तः । \newline
29. ज॒य॒ त्ये॒ताव॑न्त ए॒ताव॑न्तो जयति जय त्ये॒ताव॑न्तो॒ वै वा ए॒ताव॑न्तो जयति जय त्ये॒ताव॑न्तो॒ वै । \newline
30. ए॒ताव॑न्तो॒ वै वा ए॒ताव॑न्त ए॒ताव॑न्तो॒ वै दे॑वलो॒का दे॑वलो॒का वा ए॒ताव॑न्त ए॒ताव॑न्तो॒ वै दे॑वलो॒काः । \newline
31. वै दे॑वलो॒का दे॑वलो॒का वै वै दे॑वलो॒का स्ताꣳ स्तान् दे॑वलो॒का वै वै दे॑वलो॒का स्तान् । \newline
32. दे॒व॒लो॒का स्ताꣳ स्तान् दे॑वलो॒का दे॑वलो॒का स्ता ने॒वैव तान् दे॑वलो॒का दे॑वलो॒का स्ता ने॒व । \newline
33. दे॒व॒लो॒का इति॑ देव - लो॒काः । \newline
34. ता ने॒वैव ताꣳ स्ता ने॒वाभ्या᳚(1॒)भ्ये॑व ताꣳ स्ता ने॒वाभि । \newline
35. ए॒वाभ्या᳚(1॒)भ्ये॑ वैवाभि ज॑यति जय त्य॒भ्ये॑ वैवाभि ज॑यति । \newline
36. अ॒भि ज॑यति जय त्य॒भ्य॑भि ज॑यति॒ दश॒ दश॑ जय त्य॒भ्य॑भि ज॑यति॒ दश॑ । \newline
37. ज॒य॒ति॒ दश॒ दश॑ जयति जयति॒ दश॒ सꣳ सम् दश॑ जयति जयति॒ दश॒ सम् । \newline
38. दश॒ सꣳ सम् दश॒ दश॒ सम् प॑द्यन्ते पद्यन्ते॒ सम् दश॒ दश॒ सम् प॑द्यन्ते । \newline
39. सम् प॑द्यन्ते पद्यन्ते॒ सꣳ सम् प॑द्यन्ते॒ दशा᳚क्षरा॒ दशा᳚क्षरा पद्यन्ते॒ सꣳ सम् प॑द्यन्ते॒ दशा᳚क्षरा । \newline
40. प॒द्य॒न्ते॒ दशा᳚क्षरा॒ दशा᳚क्षरा पद्यन्ते पद्यन्ते॒ दशा᳚क्षरा वि॒राड् वि॒राड् दशा᳚क्षरा पद्यन्ते पद्यन्ते॒ दशा᳚क्षरा वि॒राट् । \newline
41. दशा᳚क्षरा वि॒राड् वि॒राड् दशा᳚क्षरा॒ दशा᳚क्षरा वि॒राडन्न॒ मन्न॑म् ॅवि॒राड् दशा᳚क्षरा॒ दशा᳚क्षरा वि॒राडन्न᳚म् । \newline
42. दशा᳚क्ष॒रेति॒ दश॑ - अ॒क्ष॒रा॒ । \newline
43. वि॒राडन्न॒ मन्न॑म् ॅवि॒राड् वि॒राडन्न॑म् ॅवि॒राड् वि॒राडन्न॑म् ॅवि॒राड् वि॒राडन्न॑म् ॅवि॒राट् । \newline
44. वि॒राडिति॑ वि - राट् । \newline
45. अन्न॑म् ॅवि॒राड् वि॒राडन्न॒ मन्न॑म् ॅवि॒राड् वि॒राजि॑ वि॒राजि॑ वि॒राडन्न॒ मन्न॑म् ॅवि॒राड् वि॒राजि॑ । \newline
46. वि॒राड् वि॒राजि॑ वि॒राजि॑ वि॒राड् वि॒राड् वि॒रा ज्ये॒वैव वि॒राजि॑ वि॒राड् वि॒राड् वि॒रा ज्ये॒व । \newline
47. वि॒राडिति॑ वि - राट् । \newline
48. वि॒रा ज्ये॒वैव वि॒राजि॑ वि॒रा ज्ये॒वा न्नाद्ये॒ ऽन्नाद्य॑ ए॒व वि॒राजि॑ वि॒रा ज्ये॒वा न्नाद्ये᳚ । \newline
49. वि॒राजीति॑ वि - राजि॑ । \newline
50. ए॒वान्नाद्ये॒ ऽन्नाद्य॑ ए॒वैवान्नाद्ये॒ प्रति॒ प्रत्य॒न्नाद्य॑ ए॒वैवा न्नाद्ये॒ प्रति॑ । \newline
51. अ॒न्नाद्ये॒ प्रति॒ प्रत्य॒न्नाद्ये॒ ऽन्नाद्ये॒ प्रति॑ तिष्ठति तिष्ठति॒ प्रत्य॒न्नाद्ये॒ ऽन्नाद्ये॒ प्रति॑ तिष्ठति । \newline
52. अ॒न्नाद्य॒ इत्य॑न्न - अद्ये᳚ । \newline
53. प्रति॑ तिष्ठति तिष्ठति॒ प्रति॒ प्रति॑ तिष्ठति । \newline
54. ति॒ष्ठ॒तीति॑ तिष्ठति । \newline
\pagebreak
\markright{ TS 3.3.6.1  \hfill https://www.vedavms.in \hfill}

\section{ TS 3.3.6.1 }

\textbf{TS 3.3.6.1 } \newline
\textbf{Samhita Paata} \newline

दे॒वा वै यद्-य॒ज्ञेन॒ नावारु॑न्धत॒ तत् परै॒रवा॑रुन्धत॒ तत् परा॑णां पर॒त्वं ॅयत् परे॑ गृ॒ह्यन्ते॒ यदे॒व य॒ज्ञेन॒नाव॑रु॒न्धे तस्याव॑रुद्ध्यै॒ यं प्र॑थ॒मं गृ॒ह्णाती॒ममे॒व तेन॑ लो॒कम॒भि ज॑यति॒यं द्वि॒तीय॑म॒न्तरि॑क्षं॒ तेन॒ यं तृ॒तीय॑म॒मुमे॒व तेन॑ लो॒कम॒भि ज॑यति॒ यदे॒ते गृ॒ह्यन्त॑ ए॒षां ॅलो॒काना॑म॒भिजि॑त्या॒ - [  ] \newline

\textbf{Pada Paata} \newline

दे॒वाः । वै । यत् । य॒ज्ञेन॑ । न । अ॒वारु॑न्ध॒तेत्य॑व - अरु॑न्धत । तत् । परैः᳚ । अवेति॑ । अ॒रु॒न्ध॒त॒ । तत् । परा॑णाम् । प॒र॒त्वमिति॑ पर - त्वम् । यत् । परे᳚ । गृ॒ह्यन्ते᳚ । यत् । ए॒व । य॒ज्ञेन॑ । न । अ॒व॒रु॒न्ध इत्यव॑-रु॒न्धे । तस्य॑ । अव॑रुद्ध्या॒ इत्यव॑ -रु॒द्ध्यै॒ । यम् । प्र॒थ॒मम् । गृ॒ह्णाति॑ । इ॒मम् । ए॒व । तेन॑ । लो॒कम् । अ॒भीति॑ । ज॒य॒ति॒ । यम् । द्वि॒तीय᳚म् । अ॒न्तरि॑क्षम् । तेन॑ । यम् । तृ॒तीय᳚म् । अ॒मुम् । ए॒व । तेन॑ । लो॒कम् । अ॒भीति॑ । ज॒य॒ति॒ । यत् । ए॒ते । गृ॒ह्यन्ते᳚ । ए॒षाम् । लो॒काना᳚म् । अ॒भिजि॑त्या॒ इत्य॒भि - जि॒त्यै॒ ।  \newline


\textbf{Krama Paata} \newline

दे॒वा वै । वै यत् । यद् य॒ज्ञेन॑ । य॒ज्ञेन॒ न । नावारु॑न्धत । अ॒वारु॑न्धत॒ तत् । अ॒वारु॑न्ध॒तेत्य॑व - अरु॑न्धत । तत् परैः᳚ । परै॒रव॑ । अवा॑रुन्धत । अ॒रु॒न्ध॒त॒ तत् । तत् परा॑णाम् । परा॑णाम् पर॒त्वम् । प॒र॒त्वं ॅयत् । प॒र॒त्वमिति॑ पर - त्वम् । यत् परे᳚ । परे॑ गृ॒ह्यन्ते᳚ । गृ॒ह्यन्ते॒ यत् । यदे॒व । ए॒व य॒ज्ञेन॑ । य॒ज्ञेन॒ न । नाव॑रु॒न्धे । अ॒व॒रु॒न्धे तस्य॑ । अ॒व॒रु॒न्ध इत्य॑व - रु॒न्धे । तस्याव॑रुद्ध्यै । अव॑रुद्ध्यै॒ यम् । अव॑रुद्ध्या॒ इत्यव॑ - रु॒द्ध्यै॒ । यम् प्र॑थ॒मम् । प्र॒थ॒मम् गृ॒ह्णाति॑ । गृ॒ह्णाती॒मम् । इ॒ममे॒व । ए॒व तेन॑ । तेन॑ लो॒कम् । लो॒कम॒भि । अ॒भि ज॑यति । ज॒य॒ति॒ यम् । यम् द्वि॒तीय᳚म् । द्वि॒तीय॑म॒न्तरि॑क्षम् । अ॒न्तरि॑क्ष॒म् तेन॑ । तेन॒ यम् । यन् तृ॒तीय᳚म् । तृ॒तीय॑म॒मुम् । अ॒मुमे॒व । ए॒व तेन॑ । तेन॑ लो॒कम् । लो॒कम॒भि । अ॒भि ज॑यति । ज॒य॒ति॒ यत् । यदे॒ते । ए॒ते गृ॒ह्यन्ते᳚ । गृ॒ह्यन्त॑ ए॒षाम् । ए॒षां ॅलो॒काना᳚म् । लो॒काना॑म॒भिजि॑त्यै । अ॒भिजि॑त्या॒ उत्त॑रेषु । अ॒भिजि॑त्या॒ इत्य॒भि - जि॒त्यै॒ \newline

\textbf{Jatai Paata} \newline

1. दे॒वा वै वै दे॒वा दे॒वा वै । \newline
2. वै यद् यद् वै वै यत् । \newline
3. यद् य॒ज्ञेन॑ य॒ज्ञेन॒ यद् यद् य॒ज्ञेन॑ । \newline
4. य॒ज्ञेन॒ न न य॒ज्ञेन॑ य॒ज्ञेन॒ न । \newline
5. ना वारु॑न्धता॒ वारु॑न्धत॒ न नावारु॑न्धत । \newline
6. अ॒वारु॑न्धत॒ तत् तद॒वारु॑न्धता॒ वारु॑न्धत॒ तत् । \newline
7. अ॒वारु॑न्ध॒तेत्य॑व - अरु॑न्धत । \newline
8. तत् परैः॒ परै॒ स्तत् तत् परैः᳚ । \newline
9. परै॒ रवाव॒ परैः॒ परै॒ रव॑ । \newline
10. अवा॑रुन्धता रुन्ध॒ता वावा॑रुन्धत । \newline
11. अ॒रु॒न्ध॒त॒ तत् तद॑रुन्धता रुन्धत॒ तत् । \newline
12. तत् परा॑णा॒म् परा॑णा॒म् तत् तत् परा॑णाम् । \newline
13. परा॑णाम् पर॒त्वम् प॑र॒त्वम् परा॑णा॒म् परा॑णाम् पर॒त्वम् । \newline
14. प॒र॒त्वम् ॅयद् यत् प॑र॒त्वम् प॑र॒त्वम् ॅयत् । \newline
15. प॒र॒त्वमिति॑ पर - त्वम् । \newline
16. यत् परे॒ परे॒ यद् यत् परे᳚ । \newline
17. परे॑ गृ॒ह्यन्ते॑ गृ॒ह्यन्ते॒ परे॒ परे॑ गृ॒ह्यन्ते᳚ । \newline
18. गृ॒ह्यन्ते॒ यद् यद् गृ॒ह्यन्ते॑ गृ॒ह्यन्ते॒ यत् । \newline
19. यदे॒वैव यद् यदे॒व । \newline
20. ए॒व य॒ज्ञेन॑ य॒ज्ञे नै॒वैव य॒ज्ञेन॑ । \newline
21. य॒ज्ञेन॒ न न य॒ज्ञेन॑ य॒ज्ञेन॒ न । \newline
22. नाव॑रु॒न्धे॑ ऽवरु॒न्धे न नाव॑रु॒न्धे । \newline
23. अ॒व॒रु॒न्धे तस्य॒ तस्या॑ वरु॒न्धे॑ ऽवरु॒न्धे तस्य॑ । \newline
24. अ॒व॒रु॒न्ध इत्यव॑ - रु॒न्धे । \newline
25. तस्या व॑रुद्ध्या॒ अव॑रुद्ध्यै॒ तस्य॒ तस्या व॑रुद्ध्यै । \newline
26. अव॑रुद्ध्यै॒ यम् ॅय मव॑रुद्ध्या॒ अव॑रुद्ध्यै॒ यम् । \newline
27. अव॑रुद्ध्या॒ इत्यव॑ - रु॒द्ध्यै॒ । \newline
28. यम् प्र॑थ॒मम् प्र॑थ॒मम् ॅयम् ॅयम् प्र॑थ॒मम् । \newline
29. प्र॒थ॒मम् गृ॒ह्णाति॑ गृ॒ह्णाति॑ प्रथ॒मम् प्र॑थ॒मम् गृ॒ह्णाति॑ । \newline
30. गृ॒ह्णाती॒म मि॒मम् गृ॒ह्णाति॑ गृ॒ह्णाती॒मम् । \newline
31. इ॒म मे॒वैवे म मि॒म मे॒व । \newline
32. ए॒व तेन॒ तेनै॒वैव तेन॑ । \newline
33. तेन॑ लो॒कम् ॅलो॒कम् तेन॒ तेन॑ लो॒कम् । \newline
34. लो॒क म॒भ्य॑भि लो॒कम् ॅलो॒क म॒भि । \newline
35. अ॒भि ज॑यति जय त्य॒भ्य॑भि ज॑यति । \newline
36. ज॒य॒ति॒ यम् ॅयम् ज॑यति जयति॒ यम् । \newline
37. यम् द्वि॒तीय॑म् द्वि॒तीय॒म् ॅयम् ॅयम् द्वि॒तीय᳚म् । \newline
38. द्वि॒तीय॑ म॒न्तरि॑क्ष म॒न्तरि॑क्षम् द्वि॒तीय॑म् द्वि॒तीय॑ म॒न्तरि॑क्षम् । \newline
39. अ॒न्तरि॑क्ष॒म् तेन॒ तेना॒न्तरि॑क्ष म॒न्तरि॑क्ष॒म् तेन॑ । \newline
40. तेन॒ यम् ॅयम् तेन॒ तेन॒ यम् । \newline
41. यम् तृ॒तीय॑म् तृ॒तीय॒म् ॅयम् ॅयम् तृ॒तीय᳚म् । \newline
42. तृ॒तीय॑ म॒मु म॒मुम् तृ॒तीय॑म् तृ॒तीय॑ म॒मुम् । \newline
43. अ॒मु मे॒वैवामु म॒मु मे॒व । \newline
44. ए॒व तेन॒ तेनै॒वैव तेन॑ । \newline
45. तेन॑ लो॒कम् ॅलो॒कम् तेन॒ तेन॑ लो॒कम् । \newline
46. लो॒क म॒भ्य॑भि लो॒कम् ॅलो॒क म॒भि । \newline
47. अ॒भि ज॑यति जय त्य॒भ्य॑भि ज॑यति । \newline
48. ज॒य॒ति॒ यद् यज् ज॑यति जयति॒ यत् । \newline
49. यदे॒त ए॒ते यद् यदे॒ते । \newline
50. ए॒ते गृ॒ह्यन्ते॑ गृ॒ह्यन्त॑ ए॒त ए॒ते गृ॒ह्यन्ते᳚ । \newline
51. गृ॒ह्यन्त॑ ए॒षा मे॒षाम् गृ॒ह्यन्ते॑ गृ॒ह्यन्त॑ ए॒षाम् । \newline
52. ए॒षाम् ॅलो॒काना᳚म् ॅलो॒काना॑ मे॒षा मे॒षाम् ॅलो॒काना᳚म् । \newline
53. लो॒काना॑ म॒भिजि॑त्या अ॒भिजि॑त्यै लो॒काना᳚म् ॅलो॒काना॑ म॒भिजि॑त्यै । \newline
54. अ॒भिजि॑त्या॒ उत्त॑रे॒षू त्त॑रे ष्व॒भिजि॑त्या अ॒भिजि॑त्या॒ उत्त॑रेषु । \newline
55. अ॒भिजि॑त्या॒ इत्य॒भि - जि॒त्यै॒ । \newline

\textbf{Ghana Paata } \newline

1. दे॒वा वै वै दे॒वा दे॒वा वै यद् यद् वै दे॒वा दे॒वा वै यत् । \newline
2. वै यद् यद् वै वै यद् य॒ज्ञेन॑ य॒ज्ञेन॒ यद् वै वै यद् य॒ज्ञेन॑ । \newline
3. यद् य॒ज्ञेन॑ य॒ज्ञेन॒ यद् यद् य॒ज्ञेन॒ न न य॒ज्ञेन॒ यद् यद् य॒ज्ञेन॒ न । \newline
4. य॒ज्ञेन॒ न न य॒ज्ञेन॑ य॒ज्ञेन॒ नावारु॑न्धता॒ वारु॑न्धत॒ न य॒ज्ञेन॑ य॒ज्ञेन॒ नावारु॑न्धत । \newline
5. नावारु॑न्धता॒ वारु॑न्धत॒ न नावारु॑न्धत॒ तत् तद॒वारु॑न्धत॒ न नावारु॑न्धत॒ तत् । \newline
6. अ॒वारु॑न्धत॒ तत् तद॒वारु॑न्धता॒ वारु॑न्धत॒ तत् परैः॒ परै॒ स्तद॒वारु॑न्धता॒ वारु॑न्धत॒ तत् परैः᳚ । \newline
7. अ॒वारु॑न्ध॒तेत्य॑व - अरु॑न्धत । \newline
8. तत् परैः॒ परै॒ स्तत् तत् परै॒ रवाव॒ परै॒ स्तत् तत् परै॒रव॑ । \newline
9. परै॒ रवाव॒ परैः॒ परै॒ रवा॑रुन्धता रुन्ध॒ताव॒ परैः॒ परै॒ रवा॑रुन्धत । \newline
10. अवा॑रुन्धता रुन्ध॒ता वावा॑ रुन्धत॒ तत् तद॑रुन्ध॒ता वावा॑ रुन्धत॒ तत् । \newline
11. अ॒रु॒न्ध॒त॒ तत् तद॑रुन्धता रुन्धत॒ तत् परा॑णा॒म् परा॑णा॒म् तद॑रुन्धता रुन्धत॒ तत् परा॑णाम् । \newline
12. तत् परा॑णा॒म् परा॑णा॒म् तत् तत् परा॑णाम् पर॒त्वम् प॑र॒त्वम् परा॑णा॒म् तत् तत् परा॑णाम् पर॒त्वम् । \newline
13. परा॑णाम् पर॒त्वम् प॑र॒त्वम् परा॑णा॒म् परा॑णाम् पर॒त्वम् ॅयद् यत् प॑र॒त्वम् परा॑णा॒म् परा॑णाम् पर॒त्वम् ॅयत् । \newline
14. प॒र॒त्वम् ॅयद् यत् प॑र॒त्वम् प॑र॒त्वम् ॅयत् परे॒ परे॒ यत् प॑र॒त्वम् प॑र॒त्वम् ॅयत् परे᳚ । \newline
15. प॒र॒त्वमिति॑ पर - त्वम् । \newline
16. यत् परे॒ परे॒ यद् यत् परे॑ गृ॒ह्यन्ते॑ गृ॒ह्यन्ते॒ परे॒ यद् यत् परे॑ गृ॒ह्यन्ते᳚ । \newline
17. परे॑ गृ॒ह्यन्ते॑ गृ॒ह्यन्ते॒ परे॒ परे॑ गृ॒ह्यन्ते॒ यद् यद् गृ॒ह्यन्ते॒ परे॒ परे॑ गृ॒ह्यन्ते॒ यत् । \newline
18. गृ॒ह्यन्ते॒ यद् यद् गृ॒ह्यन्ते॑ गृ॒ह्यन्ते॒ यदे॒वैव यद् गृ॒ह्यन्ते॑ गृ॒ह्यन्ते॒ यदे॒व । \newline
19. यदे॒वैव यद् यदे॒व य॒ज्ञेन॑ य॒ज्ञेनै॒व यद् यदे॒व य॒ज्ञेन॑ । \newline
20. ए॒व य॒ज्ञेन॑ य॒ज्ञे नै॒वैव य॒ज्ञेन॒ न न य॒ज्ञे नै॒वैव य॒ज्ञेन॒ न । \newline
21. य॒ज्ञेन॒ न न य॒ज्ञेन॑ य॒ज्ञेन॒ नाव॑रु॒न्धे॑ ऽवरु॒न्धे न य॒ज्ञेन॑ य॒ज्ञेन॒ नाव॑रु॒न्धे । \newline
22. नाव॑रु॒न्धे॑ ऽवरु॒न्धे न नाव॑रु॒न्धे तस्य॒ तस्या॑ वरु॒न्धे न नाव॑रु॒न्धे तस्य॑ । \newline
23. अ॒व॒रु॒न्धे तस्य॒ तस्या॑ वरु॒न्धे॑ ऽवरु॒न्धे तस्या व॑रुद्ध्या॒ अव॑रुद्ध्यै॒ तस्या॑ वरु॒न्धे॑ ऽवरु॒न्धे तस्या व॑रुद्ध्यै । \newline
24. अ॒व॒रु॒न्ध इत्यव॑ - रु॒न्धे । \newline
25. तस्या व॑रुद्ध्या॒ अव॑रुद्ध्यै॒ तस्य॒ तस्या व॑रुद्ध्यै॒ यम् ॅय मव॑रुद्ध्यै॒ तस्य॒ तस्या व॑रुद्ध्यै॒ यम् । \newline
26. अव॑रुद्ध्यै॒ यम् ॅय मव॑रुद्ध्या॒ अव॑रुद्ध्यै॒ यम् प्र॑थ॒मम् प्र॑थ॒मम् ॅय मव॑रुद्ध्या॒ अव॑रुद्ध्यै॒ यम् प्र॑थ॒मम् । \newline
27. अव॑रुद्ध्या॒ इत्यव॑ - रु॒द्ध्यै॒ । \newline
28. यम् प्र॑थ॒मम् प्र॑थ॒मम् ॅयम् ॅयम् प्र॑थ॒मम् गृ॒ह्णाति॑ गृ॒ह्णाति॑ प्रथ॒मम् ॅयम् ॅयम् प्र॑थ॒मम् गृ॒ह्णाति॑ । \newline
29. प्र॒थ॒मम् गृ॒ह्णाति॑ गृ॒ह्णाति॑ प्रथ॒मम् प्र॑थ॒मम् गृ॒ह्णाती॒म मि॒मम् गृ॒ह्णाति॑ प्रथ॒मम् प्र॑थ॒मम् गृ॒ह्णाती॒मम् । \newline
30. गृ॒ह्णाती॒म मि॒मम् गृ॒ह्णाति॑ गृ॒ह्णाती॒म मे॒वैवे मम् गृ॒ह्णाति॑ गृ॒ह्णाती॒म मे॒व । \newline
31. इ॒म मे॒वैवे म मि॒म मे॒व तेन॒ तेनै॒वे म मि॒म मे॒व तेन॑ । \newline
32. ए॒व तेन॒ तेनै॒वैव तेन॑ लो॒कम् ॅलो॒कम् तेनै॒वैव तेन॑ लो॒कम् । \newline
33. तेन॑ लो॒कम् ॅलो॒कम् तेन॒ तेन॑ लो॒क म॒भ्य॑भि लो॒कम् तेन॒ तेन॑ लो॒क म॒भि । \newline
34. लो॒क म॒भ्य॑भि लो॒कम् ॅलो॒क म॒भि ज॑यति जय त्य॒भि लो॒कम् ॅलो॒क म॒भि ज॑यति । \newline
35. अ॒भि ज॑यति जय त्य॒भ्य॑भि ज॑यति॒ यम् ॅयम् ज॑य त्य॒भ्य॑भि ज॑यति॒ यम् । \newline
36. ज॒य॒ति॒ यम् ॅयम् ज॑यति जयति॒ यम् द्वि॒तीय॑म् द्वि॒तीय॒म् ॅयम् ज॑यति जयति॒ यम् द्वि॒तीय᳚म् । \newline
37. यम् द्वि॒तीय॑म् द्वि॒तीय॒म् ॅयम् ॅयम् द्वि॒तीय॑ म॒न्तरि॑क्ष म॒न्तरि॑क्षम् द्वि॒तीय॒म् ॅयम् ॅयम् द्वि॒तीय॑ म॒न्तरि॑क्षम् । \newline
38. द्वि॒तीय॑ म॒न्तरि॑क्ष म॒न्तरि॑क्षम् द्वि॒तीय॑म् द्वि॒तीय॑ म॒न्तरि॑क्ष॒म् तेन॒ तेना॒न्तरि॑क्षम् द्वि॒तीय॑म् द्वि॒तीय॑ म॒न्तरि॑क्ष॒म् तेन॑ । \newline
39. अ॒न्तरि॑क्ष॒म् तेन॒ तेना॒न्तरि॑क्ष म॒न्तरि॑क्ष॒म् तेन॒ यम् ॅयम् तेना॒न्तरि॑क्ष म॒न्तरि॑क्ष॒म् तेन॒ यम् । \newline
40. तेन॒ यम् ॅयम् तेन॒ तेन॒ यम् तृ॒तीय॑म् तृ॒तीय॒म् ॅयम् तेन॒ तेन॒ यम् तृ॒तीय᳚म् । \newline
41. यम् तृ॒तीय॑म् तृ॒तीय॒म् ॅयम् ॅयम् तृ॒तीय॑ म॒मु म॒मुम् तृ॒तीय॒म् ॅयम् ॅयम् तृ॒तीय॑ म॒मुम् । \newline
42. तृ॒तीय॑ म॒मु म॒मुम् तृ॒तीय॑म् तृ॒तीय॑ म॒मु मे॒वैवामुम् तृ॒तीय॑म् तृ॒तीय॑ म॒मु मे॒व । \newline
43. अ॒मु मे॒वैवामु म॒मु मे॒व तेन॒ तेनै॒वामु म॒मु मे॒व तेन॑ । \newline
44. ए॒व तेन॒ तेनै॒वैव तेन॑ लो॒कम् ॅलो॒कम् तेनै॒वैव तेन॑ लो॒कम् । \newline
45. तेन॑ लो॒कम् ॅलो॒कम् तेन॒ तेन॑ लो॒क म॒भ्य॑भि लो॒कम् तेन॒ तेन॑ लो॒क म॒भि । \newline
46. लो॒क म॒भ्य॑भि लो॒कम् ॅलो॒क म॒भि ज॑यति जय त्य॒भि लो॒कम् ॅलो॒क म॒भि ज॑यति । \newline
47. अ॒भि ज॑यति जय त्य॒भ्य॑भि ज॑यति॒ यद् यज् ज॑य त्य॒भ्य॑भि ज॑यति॒ यत् । \newline
48. ज॒य॒ति॒ यद् यज् ज॑यति जयति॒ यदे॒त ए॒ते यज् ज॑यति जयति॒ यदे॒ते । \newline
49. यदे॒त ए॒ते यद् यदे॒ते गृ॒ह्यन्ते॑ गृ॒ह्यन्त॑ ए॒ते यद् यदे॒ते गृ॒ह्यन्ते᳚ । \newline
50. ए॒ते गृ॒ह्यन्ते॑ गृ॒ह्यन्त॑ ए॒त ए॒ते गृ॒ह्यन्त॑ ए॒षा मे॒षाम् गृ॒ह्यन्त॑ ए॒त ए॒ते गृ॒ह्यन्त॑ ए॒षाम् । \newline
51. गृ॒ह्यन्त॑ ए॒षा मे॒षाम् गृ॒ह्यन्ते॑ गृ॒ह्यन्त॑ ए॒षाम् ॅलो॒काना᳚म् ॅलो॒काना॑ मे॒षाम् गृ॒ह्यन्ते॑ गृ॒ह्यन्त॑ ए॒षाम् ॅलो॒काना᳚म् । \newline
52. ए॒षाम् ॅलो॒काना᳚म् ॅलो॒काना॑ मे॒षा मे॒षाम् ॅलो॒काना॑ म॒भिजि॑त्या अ॒भिजि॑त्यै लो॒काना॑ मे॒षा मे॒षाम् ॅलो॒काना॑ म॒भिजि॑त्यै । \newline
53. लो॒काना॑ म॒भिजि॑त्या अ॒भिजि॑त्यै लो॒काना᳚म् ॅलो॒काना॑ म॒भिजि॑त्या॒ उत्त॑रे॒ षूत्त॑रे॒ ष्व॑भिजि॑त्यै लो॒काना᳚म् ॅलो॒काना॑ म॒भिजि॑त्या॒ उत्त॑रेषु । \newline
54. अ॒भिजि॑त्या॒ उत्त॑रे॒ षूत्त॑रे ष्व॒भिजि॑त्या अ॒भिजि॑त्या॒ उत्त॑रे॒ ष्वह॒ स्स्वह॒ स्सूत्त॑रे ष्व॒भिजि॑त्या अ॒भिजि॑त्या॒ उत्त॑रे॒ ष्वह॑स्सु । \newline
55. अ॒भिजि॑त्या॒ इत्य॒भि - जि॒त्यै॒ । \newline
\pagebreak
\markright{ TS 3.3.6.2  \hfill https://www.vedavms.in \hfill}

\section{ TS 3.3.6.2 }

\textbf{TS 3.3.6.2 } \newline
\textbf{Samhita Paata} \newline

उत्त॑रे॒ष्वहः॑ स्व॒मुतो॒ऽर्वाञ्चो॑ गृह्यन्ते ऽभि॒जित्यै॒वेमां ॅलो॒कान् पुन॑रि॒मं ॅलो॒कं प्र॒त्यव॑रोहन्ति॒ यत् पूर्वे॒ष्वहः॑ स्वि॒तः परा᳚ञ्चो गृ॒ह्यन्ते॒ तस्मा॑दि॒तः परा᳚ञ्च इ॒मे लो॒का यदुत्त॑रे॒ष्वहः॑ स्व॒मुतो॒ऽर्वाञ्चो॑ गृ॒ह्यन्ते॒ तस्मा॑द॒मुतो॒ ऽर्वाञ्च॑ इ॒मे लो॒कास्तस्मा॒दया॑तयाम्नो लो॒कान् म॑नु॒ष्या॑ उप॑ जीवन्ति ब्रह्मवा॒दिनो॑ वदन्ति॒ कस्मा᳚थ् स॒त्याद॒द्भ्य ओष॑धयः॒ सं भ॑व॒न्त्योष॑धयो - [  ] \newline

\textbf{Pada Paata} \newline

उत्त॑रे॒ष्वित्युत् - त॒रे॒षु॒ । अहः॒ स्वित्यहः॑-सु॒ । अ॒मुतः॑ । अ॒र्वाञ्चः॑ । गृ॒ह्य॒न्ते॒ । अ॒भि॒जित्येत्य॑भि-जित्य॑ । ए॒व । इ॒मान् । लो॒कान् । पुनः॑ । इ॒मम् । लो॒कम् । प्र॒त्यव॑रोह॒न्तीति॑ प्रति - अव॑रोहन्ति । यत् । पूर्वे॑षु । अहः॒ स्वित्यहः॑ - सु॒ । इ॒तः । परा᳚ञ्चः । गृ॒ह्यन्ते᳚ । तस्मा᳚त् । इ॒तः । परा᳚ञ्चः । इ॒मे । लो॒काः । यत् । उत्त॑रे॒ष्वित्युत् - त॒रे॒षु॒ । अहः॒ स्वित्यहः॑ - सु॒ । अ॒मुतः॑ । अ॒र्वाञ्चः॑ । गृ॒ह्यन्ते᳚ । तस्मा᳚त् । अ॒मुतः॑ । अ॒र्वाञ्चः॑ । इ॒मे । लो॒काः । तस्मा᳚त् । अया॑तयाम्न॒ इत्यया॑त-या॒म्नः॒ । लो॒कान् । म॒नु॒ष्याः᳚ । उपेति॑ । जी॒व॒न्ति॒ । ब्र॒ह्म॒वा॒दिन॒ इति॑ ब्रह्म - वा॒दिनः॑ । व॒द॒न्ति॒ । कस्मा᳚त् । स॒त्यात् । अ॒द्भ्य इत्य॑त् - भ्यः । ओष॑धयः । समिति॑ । भ॒व॒न्ति॒ । ओष॑धयः ।  \newline


\textbf{Krama Paata} \newline

उत्त॑रे॒ष्वह॑स्सु । उत्त॑रे॒ष्वित्युत् - त॒रे॒षु॒ । अह॑स्स्व॒मुतः॑ । अह॒स्स्वित्यहः॑ - सु॒ । अ॒मुतो॒ ऽर्वाञ्चः॑ । अ॒र्वाञ्चो॑ गृह्यन्ते । गृ॒ह्य॒न्ते॒ ऽभि॒जित्य॑ । अ॒भि॒जित्यै॒व । अ॒भि॒जित्येत्य॑भि - जित्य॑ । ए॒वेमान् । इ॒मान् ॅलो॒कान् । लो॒कान् पुनः॑ । पुन॑रि॒मम् । इ॒मं ॅलो॒कम् । लो॒कम् प्र॒त्यव॑रोहन्ति । प्र॒त्यव॑रोहन्ति॒ यत् । प्र॒त्यव॑रोह॒न्तीति॑ प्रति - अव॑रोहन्ति । यत् पूर्वे॑षु । पूर्वे॒ष्वह॑स्सु । अह॑स्स्वि॒तः । अह॒स्स्वित्यहः॑ - सु॒ । इ॒तः परा᳚ञ्चः । परा᳚ञ्चो गृ॒ह्यन्ते᳚ । गृ॒ह्यन्ते॒ तस्मा᳚त् । तस्मा॑दि॒तः । इ॒तः परा᳚ञ्चः । परा᳚ञ्च इ॒मे । इ॒मे लो॒काः । लो॒का यत् । यदुत्त॑रेषु । उत्त॑रे॒ष्वह॑स्सु । उत्त॑रे॒ष्वित्युत् - त॒रे॒षु॒ । अह॑स्स्व॒मुतः॑ । अह॒स्स्वित्यहः॑ - सु॒ । अ॒मुतो॒ ऽर्वाञ्चः॑ । अ॒र्वाञ्चो॑ गृ॒ह्यन्ते᳚ । गृ॒ह्यन्ते॒ तस्मा᳚त् । तस्मा॑द॒मुतः॑ । अ॒मुतो॒ऽर्वाञ्चः॑ । अ॒र्वाञ्च॑ इ॒मे । इ॒मे लो॒काः । लो॒कास्तस्मा᳚त् । तस्मा॒दया॑तयाम्नः । अया॑तयाम्नो लो॒कान् । अया॑तयाम्न॒ इत्यया॑त - या॒म्नः॒ । लो॒कान् म॑नु॒ष्याः᳚ । म॒नु॒ष्या॑ उप॑ । उप॑ जीवन्ति । जी॒व॒न्ति॒ ब्र॒ह्म॒वा॒दिनः॑ । ब्र॒ह्म॒वा॒दिनो॑ वदन्ति । ब्र॒ह्म॒वा॒दिन॒ इति॑ ब्रह्म - वा॒दिनः॑ । व॒द॒न्ति॒ कस्मा᳚त् । कस्मा᳚थ् स॒त्यात् । स॒त्याद॒द्भ्यः । अ॒द्भ्य ओष॑धयः । अ॒द्भ्य इत्य॑त् - भ्यः । ओष॑धयः॒ सम् । सम् भ॑वन्ति । भ॒व॒न्त्योष॑धयः ( ) । ओष॑धयो मनु॒ष्या॑णाम् \newline

\textbf{Jatai Paata} \newline

1. उत्त॑रे॒ ष्वह॒ स्स्वह॒ स्सूत्त॑रे॒ षूत्त॑रे॒ ष्वह॑स्सु । \newline
2. उत्त॑रे॒ष्वित्युत् - त॒रे॒षु॒ । \newline
3. अह॑ स्स्व॒मुतो॒ ऽमुतो ऽह॒ स्स्वह॑ स्स्व॒मुतः॑ । \newline
4. अह॒स्स्वित्यहः॑ - सु॒ । \newline
5. अ॒मुतो॒ ऽर्वाञ्चो॒ ऽर्वाञ्चो॒ ऽमुतो॒ ऽमुतो॒ ऽर्वाञ्चः॑ । \newline
6. अ॒र्वाञ्चो॑ गृह्यन्ते गृह्यन्ते॒ ऽर्वाञ्चो॒ ऽर्वाञ्चो॑ गृह्यन्ते । \newline
7. गृ॒ह्य॒न्ते॒ ऽभि॒जि त्या॑भि॒जित्य॑ गृह्यन्ते गृह्यन्ते ऽभि॒जित्य॑ । \newline
8. अ॒भि॒जि त्यै॒वैवा भि॒जित्या॑ भि॒जि त्यै॒व । \newline
9. अ॒भि॒जित्येत्य॑भि - जित्य॑ । \newline
10. ए॒वे मा नि॒मा ने॒वैवे मान् । \newline
11. इ॒मान् ॅलो॒कान् ॅलो॒का नि॒मा नि॒मान् ॅलो॒कान् । \newline
12. लो॒कान् पुनः॒ पुन॑र् लो॒कान् ॅलो॒कान् पुनः॑ । \newline
13. पुन॑ रि॒म मि॒मम् पुनः॒ पुन॑ रि॒मम् । \newline
14. इ॒मम् ॅलो॒कम् ॅलो॒क मि॒म मि॒मम् ॅलो॒कम् । \newline
15. लो॒कम् प्र॒त्यव॑रोहन्ति प्र॒त्यव॑रोहन्ति लो॒कम् ॅलो॒कम् प्र॒त्यव॑रोहन्ति । \newline
16. प्र॒त्यव॑रोहन्ति॒ यद् यत् प्र॒त्यव॑रोहन्ति प्र॒त्यव॑रोहन्ति॒ यत् । \newline
17. प्र॒त्यव॑रोह॒न्तीति॑ प्रति - अव॑रोहन्ति । \newline
18. यत् पूर्वे॑षु॒ पूर्वे॑षु॒ यद् यत् पूर्वे॑षु । \newline
19. पूर्वे॒ ष्वह॒ स्स्वह॑स्सु॒ पूर्वे॑षु॒ पूर्वे॒ ष्वह॑स्सु । \newline
20. अह॑ स्स्वि॒त इ॒तो ऽह॒ स्स्वह॑ स्स्वि॒तः । \newline
21. अह॒स्स्वित्यहः॑ - सु॒ । \newline
22. इ॒तः परा᳚ञ्चः॒ परा᳚ञ्च इ॒त इ॒तः परा᳚ञ्चः । \newline
23. परा᳚ञ्चो गृ॒ह्यन्ते॑ गृ॒ह्यन्ते॒ परा᳚ञ्चः॒ परा᳚ञ्चो गृ॒ह्यन्ते᳚ । \newline
24. गृ॒ह्यन्ते॒ तस्मा॒त् तस्मा᳚द् गृ॒ह्यन्ते॑ गृ॒ह्यन्ते॒ तस्मा᳚त् । \newline
25. तस्मा॑ दि॒त इ॒त स्तस्मा॒त् तस्मा॑ दि॒तः । \newline
26. इ॒तः परा᳚ञ्चः॒ परा᳚ञ्च इ॒त इ॒तः परा᳚ञ्चः । \newline
27. परा᳚ञ्च इ॒म इ॒मे परा᳚ञ्चः॒ परा᳚ञ्च इ॒मे । \newline
28. इ॒मे लो॒का लो॒का इ॒म इ॒मे लो॒काः । \newline
29. लो॒का यद् यल् लो॒का लो॒का यत् । \newline
30. यदुत्त॑रे॒ षूत्त॑रेषु॒ यद् यदुत्त॑रेषु । \newline
31. उत्त॑रे॒ ष्वह॒ स्स्वह॒ स्सूत्त॑रे॒ षूत्त॑रे॒ ष्वह॑स्सु । \newline
32. उत्त॑रे॒ष्वित्युत् - त॒रे॒षु॒ । \newline
33. अह॑ स्स्व॒मुतो॒ ऽमुतो ऽह॒ स्स्वह॑ स्स्व॒मुतः॑ । \newline
34. अह॒स्स्वित्यहः॑ - सु॒ । \newline
35. अ॒मुतो॒ ऽर्वाञ्चो॒ ऽर्वाञ्चो॒ ऽमुतो॒ ऽमुतो॒ ऽर्वाञ्चः॑ । \newline
36. अ॒र्वाञ्चो॑ गृ॒ह्यन्ते॑ गृ॒ह्यन्ते॒ ऽर्वाञ्चो॒ ऽर्वाञ्चो॑ गृ॒ह्यन्ते᳚ । \newline
37. गृ॒ह्यन्ते॒ तस्मा॒त् तस्मा᳚द् गृ॒ह्यन्ते॑ गृ॒ह्यन्ते॒ तस्मा᳚त् । \newline
38. तस्मा॑ द॒मुतो॒ ऽमुत॒ स्तस्मा॒त् तस्मा॑ द॒मुतः॑ । \newline
39. अ॒मुतो॒ ऽर्वाञ्चो॒ ऽर्वाञ्चो॒ ऽमुतो॒ ऽमुतो॒ ऽर्वाञ्चः॑ । \newline
40. अ॒र्वाञ्च॑ इ॒म इ॒मे᳚ ऽर्वाञ्चो॒ ऽर्वाञ्च॑ इ॒मे । \newline
41. इ॒मे लो॒का लो॒का इ॒म इ॒मे लो॒काः । \newline
42. लो॒का स्तस्मा॒त् तस्मा᳚ल् लो॒का लो॒का स्तस्मा᳚त् । \newline
43. तस्मा॒ दया॑तया॒म्नो ऽया॑तयाम्न॒ स्तस्मा॒त् तस्मा॒ दया॑तयाम्नः । \newline
44. अया॑तयाम्नो लो॒कान् ॅलो॒का नया॑तया॒म्नो ऽया॑तयाम्नो लो॒कान् । \newline
45. अया॑तयाम्न॒ इत्यया॑त - या॒म्नः॒ । \newline
46. लो॒कान् म॑नु॒ष्या॑ मनु॒ष्या॑ लो॒कान् ॅलो॒कान् म॑नु॒ष्याः᳚ । \newline
47. म॒नु॒ष्या॑ उपोप॑ मनु॒ष्या॑ मनु॒ष्या॑ उप॑ । \newline
48. उप॑ जीवन्ति जीव॒ न्त्युपोप॑ जीवन्ति । \newline
49. जी॒व॒न्ति॒ ब्र॒ह्म॒वा॒दिनो᳚ ब्रह्मवा॒दिनो॑ जीवन्ति जीवन्ति ब्रह्मवा॒दिनः॑ । \newline
50. ब्र॒ह्म॒वा॒दिनो॑ वदन्ति वदन्ति ब्रह्मवा॒दिनो᳚ ब्रह्मवा॒दिनो॑ वदन्ति । \newline
51. ब्र॒ह्म॒वा॒दिन॒ इति॑ ब्रह्म - वा॒दिनः॑ । \newline
52. व॒द॒न्ति॒ कस्मा॒त् कस्मा᳚द् वदन्ति वदन्ति॒ कस्मा᳚त् । \newline
53. कस्मा᳚थ् स॒त्याथ् स॒त्यात् कस्मा॒त् कस्मा᳚थ् स॒त्यात् । \newline
54. स॒त्या द॒द्भ्यो᳚ ऽद्भ्यः स॒त्याथ् स॒त्या द॒द्भ्यः । \newline
55. अ॒द्भ्य ओष॑धय॒ ओष॑धयो॒ ऽद्भ्यो᳚ ऽद्भ्य ओष॑धयः । \newline
56. अ॒द्भ्य इत्य॑त् - भ्यः । \newline
57. ओष॑धयः॒ सꣳ स मोष॑धय॒ ओष॑धयः॒ सम् । \newline
58. सम् भ॑वन्ति भवन्ति॒ सꣳ सम् भ॑वन्ति । \newline
59. भ॒व॒ न्त्योष॑धय॒ ओष॑धयो भवन्ति भव॒ न्त्योष॑धयः । \newline
60. ओष॑धयो मनु॒ष्या॑णाम् मनु॒ष्या॑णा॒ मोष॑धय॒ ओष॑धयो मनु॒ष्या॑णाम् । \newline

\textbf{Ghana Paata } \newline

1. उत्त॑रे॒ ष्वह॒ स्स्वह॒ स्सूत्त॑रे॒ षूत्त॑रे॒ ष्वह॑ स्स्व॒मुतो॒ ऽमुतो ऽह॒स्सूत्त॑रे॒ षूत्त॑रे॒ ष्वह॑ स्स्व॒मुतः॑ । \newline
2. उत्त॑रे॒ष्वित्युत् - त॒रे॒षु॒ । \newline
3. अह॑स्स्व॒मुतो॒ ऽमुतो ऽह॒ स्स्वह॑ स्स्व॒मुतो॒ ऽर्वाञ्चो॒ ऽर्वाञ्चो॒ ऽमुतो ऽह॒ स्स्वह॑ स्स्व॒मुतो॒ ऽर्वाञ्चः॑ । \newline
4. अह॒स्स्वित्यहः॑ - सु॒ । \newline
5. अ॒मुतो॒ ऽर्वाञ्चो॒ ऽर्वाञ्चो॒ ऽमुतो॒ ऽमुतो॒ ऽर्वाञ्चो॑ गृह्यन्ते गृह्यन्ते॒ ऽर्वाञ्चो॒ ऽमुतो॒ ऽमुतो॒ ऽर्वाञ्चो॑ गृह्यन्ते । \newline
6. अ॒र्वाञ्चो॑ गृह्यन्ते गृह्यन्ते॒ ऽर्वाञ्चो॒ ऽर्वाञ्चो॑ गृह्यन्ते ऽभि॒जित्या॑ भि॒जित्य॑ गृह्यन्ते॒ ऽर्वाञ्चो॒ ऽर्वाञ्चो॑ गृह्यन्ते ऽभि॒जित्य॑ । \newline
7. गृ॒ह्य॒न्ते॒ ऽभि॒जित्या॑ भि॒जित्य॑ गृह्यन्ते गृह्यन्ते ऽभि॒जि त्यै॒वैवा भि॒जित्य॑ गृह्यन्ते गृह्यन्ते ऽभि॒जित्यै॒व । \newline
8. अ॒भि॒जि त्यै॒वैवा भि॒जित्या॑ भि॒जि त्यै॒वे मा नि॒मा ने॒वा भि॒जित्या॑ भि॒जित्यै॒वे मान् । \newline
9. अ॒भि॒जित्येत्य॑भि - जित्य॑ । \newline
10. ए॒वे मा नि॒मा ने॒वैवे मान् ॅलो॒कान् ॅलो॒का नि॒मा ने॒वैवे मान् ॅलो॒कान् । \newline
11. इ॒मान् ॅलो॒कान् ॅलो॒का नि॒मा नि॒मान् ॅलो॒कान् पुनः॒ पुन॑र् लो॒का नि॒मा नि॒मान् ॅलो॒कान् पुनः॑ । \newline
12. लो॒कान् पुनः॒ पुन॑र् लो॒कान् ॅलो॒कान् पुन॑ रि॒म मि॒मम् पुन॑र् लो॒कान् ॅलो॒कान् पुन॑ रि॒मम् । \newline
13. पुन॑ रि॒म मि॒मम् पुनः॒ पुन॑ रि॒मम् ॅलो॒कम् ॅलो॒क मि॒मम् पुनः॒ पुन॑ रि॒मम् ॅलो॒कम् । \newline
14. इ॒मम् ॅलो॒कम् ॅलो॒क मि॒म मि॒मम् ॅलो॒कम् प्र॒त्यव॑रोहन्ति प्र॒त्यव॑रोहन्ति लो॒क मि॒म मि॒मम् ॅलो॒कम् प्र॒त्यव॑रोहन्ति । \newline
15. लो॒कम् प्र॒त्यव॑रोहन्ति प्र॒त्यव॑रोहन्ति लो॒कम् ॅलो॒कम् प्र॒त्यव॑रोहन्ति॒ यद् यत् प्र॒त्यव॑रोहन्ति लो॒कम् ॅलो॒कम् प्र॒त्यव॑रोहन्ति॒ यत् । \newline
16. प्र॒त्यव॑रोहन्ति॒ यद् यत् प्र॒त्यव॑रोहन्ति प्र॒त्यव॑रोहन्ति॒ यत् पूर्वे॑षु॒ पूर्वे॑षु॒ यत् प्र॒त्यव॑रोहन्ति प्र॒त्यव॑रोहन्ति॒ यत् पूर्वे॑षु । \newline
17. प्र॒त्यव॑रोह॒न्तीति॑ प्रति - अव॑रोहन्ति । \newline
18. यत् पूर्वे॑षु॒ पूर्वे॑षु॒ यद् यत् पूर्वे॒ष्वह॒ स्स्वह॑स्सु॒ पूर्वे॑षु॒ यद् यत् पूर्वे॒ष्वह॑स्सु । \newline
19. पूर्वे॒ष्वह॒ स्स्वह॑स्सु॒ पूर्वे॑षु॒ पूर्वे॒ष्वह॑ स्स्वि॒त इ॒तो ऽह॑स्सु॒ पूर्वे॑षु॒ पूर्वे॒ष्वह॑ स्स्वि॒तः । \newline
20. अह॑स्स्वि॒त इ॒तो ऽह॒स्स्वह॑ स्स्वि॒तः परा᳚ञ्चः॒ परा᳚ञ्च इ॒तो ऽह॒स्स्वह॑ स्स्वि॒तः परा᳚ञ्चः । \newline
21. अह॒स्स्वित्यहः॑ - सु॒ । \newline
22. इ॒तः परा᳚ञ्चः॒ परा᳚ञ्च इ॒त इ॒तः परा᳚ञ्चो गृ॒ह्यन्ते॑ गृ॒ह्यन्ते॒ परा᳚ञ्च इ॒त इ॒तः परा᳚ञ्चो गृ॒ह्यन्ते᳚ । \newline
23. परा᳚ञ्चो गृ॒ह्यन्ते॑ गृ॒ह्यन्ते॒ परा᳚ञ्चः॒ परा᳚ञ्चो गृ॒ह्यन्ते॒ तस्मा॒त् तस्मा᳚द् गृ॒ह्यन्ते॒ परा᳚ञ्चः॒ परा᳚ञ्चो गृ॒ह्यन्ते॒ तस्मा᳚त् । \newline
24. गृ॒ह्यन्ते॒ तस्मा॒त् तस्मा᳚द् गृ॒ह्यन्ते॑ गृ॒ह्यन्ते॒ तस्मा॑ दि॒त इ॒त स्तस्मा᳚द् गृ॒ह्यन्ते॑ गृ॒ह्यन्ते॒ तस्मा॑ दि॒तः । \newline
25. तस्मा॑ दि॒त इ॒त स्तस्मा॒त् तस्मा॑ दि॒तः परा᳚ञ्चः॒ परा᳚ञ्च इ॒त स्तस्मा॒त् तस्मा॑ दि॒तः परा᳚ञ्चः । \newline
26. इ॒तः परा᳚ञ्चः॒ परा᳚ञ्च इ॒त इ॒तः परा᳚ञ्च इ॒म इ॒मे परा᳚ञ्च इ॒त इ॒तः परा᳚ञ्च इ॒मे । \newline
27. परा᳚ञ्च इ॒म इ॒मे परा᳚ञ्चः॒ परा᳚ञ्च इ॒मे लो॒का लो॒का इ॒मे परा᳚ञ्चः॒ परा᳚ञ्च इ॒मे लो॒काः । \newline
28. इ॒मे लो॒का लो॒का इ॒म इ॒मे लो॒का यद् यल्लो॒का इ॒म इ॒मे लो॒का यत् । \newline
29. लो॒का यद् यल्लो॒का लो॒का यदुत्त॑रे॒ षूत्त॑रेषु॒ यल्लो॒का लो॒का यदुत्त॑रेषु । \newline
30. यदुत्त॑रे॒ षूत्त॑रेषु॒ यद् यदुत्त॑रे॒ ष्वह॒ स्स्वह॒ स्सूत्त॑रेषु॒ यद् यदुत्त॑रे॒ ष्वह॑स्सु । \newline
31. उत्त॑रे॒ ष्वह॒ स्स्वह॒ स्सूत्त॑रे॒षू त्त॑रे॒ष्वह॑ स्स्व॒मुतो॒ ऽमुतो ऽह॒स्सूत्त॑रे॒षू त्त॑रे॒ ष्वह॑ स्स्व॒मुतः॑ । \newline
32. उत्त॑रे॒ष्वित्युत् - त॒रे॒षु॒ । \newline
33. अह॑स्स्व॒मुतो॒ ऽमुतो ऽह॒स्स्वह॑ स्स्व॒मुतो॒ ऽर्वाञ्चो॒ ऽर्वाञ्चो॒ ऽमुतो ऽह॒स्स्वह॑ स्स्व॒मुतो॒ ऽर्वाञ्चः॑ । \newline
34. अह॒स्स्वित्यहः॑ - सु॒ । \newline
35. अ॒मुतो॒ ऽर्वाञ्चो॒ ऽर्वाञ्चो॒ ऽमुतो॒ ऽमुतो॒ ऽर्वाञ्चो॑ गृ॒ह्यन्ते॑ गृ॒ह्यन्ते॒ ऽर्वाञ्चो॒ ऽमुतो॒ ऽमुतो॒ ऽर्वाञ्चो॑ गृ॒ह्यन्ते᳚ । \newline
36. अ॒र्वाञ्चो॑ गृ॒ह्यन्ते॑ गृ॒ह्यन्ते॒ ऽर्वाञ्चो॒ ऽर्वाञ्चो॑ गृ॒ह्यन्ते॒ तस्मा॒त् तस्मा᳚द् गृ॒ह्यन्ते॒ ऽर्वाञ्चो॒ ऽर्वाञ्चो॑ गृ॒ह्यन्ते॒ तस्मा᳚त् । \newline
37. गृ॒ह्यन्ते॒ तस्मा॒त् तस्मा᳚द् गृ॒ह्यन्ते॑ गृ॒ह्यन्ते॒ तस्मा॑ द॒मुतो॒ ऽमुत॒ स्तस्मा᳚द् गृ॒ह्यन्ते॑ गृ॒ह्यन्ते॒ तस्मा॑ द॒मुतः॑ । \newline
38. तस्मा॑ द॒मुतो॒ ऽमुत॒ स्तस्मा॒त् तस्मा॑ द॒मुतो॒ ऽर्वाञ्चो॒ ऽर्वाञ्चो॒ ऽमुत॒ स्तस्मा॒त् तस्मा॑ द॒मुतो॒ ऽर्वाञ्चः॑ । \newline
39. अ॒मुतो॒ ऽर्वाञ्चो॒ ऽर्वाञ्चो॒ ऽमुतो॒ ऽमुतो॒ ऽर्वाञ्च॑ इ॒म इ॒मे᳚ ऽर्वाञ्चो॒ ऽमुतो॒ ऽमुतो॒ ऽर्वाञ्च॑ इ॒मे । \newline
40. अ॒र्वाञ्च॑ इ॒म इ॒मे᳚ ऽर्वाञ्चो॒ ऽर्वाञ्च॑ इ॒मे लो॒का लो॒का इ॒मे᳚ ऽर्वाञ्चो॒ ऽर्वाञ्च॑ इ॒मे लो॒काः । \newline
41. इ॒मे लो॒का लो॒का इ॒म इ॒मे लो॒का स्तस्मा॒त् तस्मा᳚ ल्लो॒का इ॒म इ॒मे लो॒का स्तस्मा᳚त् । \newline
42. लो॒का स्तस्मा॒त् तस्मा᳚ ल्लो॒का लो॒का स्तस्मा॒ दया॑तया॒म्नो ऽया॑तयाम्न॒ स्तस्मा᳚ ल्लो॒का लो॒का स्तस्मा॒ दया॑तयाम्नः । \newline
43. तस्मा॒ दया॑तया॒म्नो ऽया॑तयाम्न॒ स्तस्मा॒त् तस्मा॒ दया॑तयाम्नो लो॒कान् ॅलो॒का नया॑तयाम्न॒ स्तस्मा॒त् तस्मा॒ दया॑तयाम्नो लो॒कान् । \newline
44. अया॑तयाम्नो लो॒कान् ॅलो॒का नया॑तया॒म्नो ऽया॑तयाम्नो लो॒कान् म॑नु॒ष्या॑ मनु॒ष्या॑ लो॒का नया॑तया॒म्नो ऽया॑तयाम्नो लो॒कान् म॑नु॒ष्याः᳚ । \newline
45. अया॑तयाम्न॒ इत्यया॑त - या॒म्नः॒ । \newline
46. लो॒कान् म॑नु॒ष्या॑ मनु॒ष्या॑ लो॒कान् ॅलो॒कान् म॑नु॒ष्या॑ उपोप॑ मनु॒ष्या॑ लो॒कान् ॅलो॒कान् म॑नु॒ष्या॑ उप॑ । \newline
47. म॒नु॒ष्या॑ उपोप॑ मनु॒ष्या॑ मनु॒ष्या॑ उप॑ जीवन्ति जीव॒ न्त्युप॑ मनु॒ष्या॑ मनु॒ष्या॑ उप॑ जीवन्ति । \newline
48. उप॑ जीवन्ति जीव॒ न्त्युपोप॑ जीवन्ति ब्रह्मवा॒दिनो᳚ ब्रह्मवा॒दिनो॑ जीव॒ न्त्युपोप॑ जीवन्ति ब्रह्मवा॒दिनः॑ । \newline
49. जी॒व॒न्ति॒ ब्र॒ह्म॒वा॒दिनो᳚ ब्रह्मवा॒दिनो॑ जीवन्ति जीवन्ति ब्रह्मवा॒दिनो॑ वदन्ति वदन्ति ब्रह्मवा॒दिनो॑ जीवन्ति जीवन्ति ब्रह्मवा॒दिनो॑ वदन्ति । \newline
50. ब्र॒ह्म॒वा॒दिनो॑ वदन्ति वदन्ति ब्रह्मवा॒दिनो᳚ ब्रह्मवा॒दिनो॑ वदन्ति॒ कस्मा॒त् कस्मा᳚द् वदन्ति ब्रह्मवा॒दिनो᳚ ब्रह्मवा॒दिनो॑ वदन्ति॒ कस्मा᳚त् । \newline
51. ब्र॒ह्म॒वा॒दिन॒ इति॑ ब्रह्म - वा॒दिनः॑ । \newline
52. व॒द॒न्ति॒ कस्मा॒त् कस्मा᳚द् वदन्ति वदन्ति॒ कस्मा᳚थ् स॒त्याथ् स॒त्यात् कस्मा᳚द् वदन्ति वदन्ति॒ कस्मा᳚थ् स॒त्यात् । \newline
53. कस्मा᳚थ् स॒त्याथ् स॒त्यात् कस्मा॒त् कस्मा᳚थ् स॒त्या द॒द्भ्यो᳚ ऽद्भ्यः स॒त्यात् कस्मा॒त् कस्मा᳚थ् स॒त्या द॒द्भ्यः । \newline
54. स॒त्या द॒द्भ्यो᳚ ऽद्भ्यः स॒त्याथ् स॒त्या द॒द्भ्य ओष॑धय॒ ओष॑धयो॒ ऽद्भ्यः स॒त्याथ् स॒त्या द॒द्भ्य ओष॑धयः । \newline
55. अ॒द्भ्य ओष॑धय॒ ओष॑धयो॒ ऽद्भ्यो᳚ ऽद्भ्य ओष॑धयः॒ सꣳ स मोष॑धयो॒ ऽद्भ्यो᳚ ऽद्भ्य ओष॑धयः॒ सम् । \newline
56. अ॒द्भ्य इत्य॑त् - भ्यः । \newline
57. ओष॑धयः॒ सꣳ स मोष॑धय॒ ओष॑धयः॒ सम् भ॑वन्ति भवन्ति॒ स मोष॑धय॒ ओष॑धयः॒ सम् भ॑वन्ति । \newline
58. सम् भ॑वन्ति भवन्ति॒ सꣳ सम् भ॑व॒ न्त्योष॑धय॒ ओष॑धयो भवन्ति॒ सꣳ सम् भ॑व॒ न्त्योष॑धयः । \newline
59. भ॒व॒ न्त्योष॑धय॒ ओष॑धयो भवन्ति भव॒ न्त्योष॑धयो मनु॒ष्या॑णाम् मनु॒ष्या॑णा॒ मोष॑धयो भवन्ति भव॒ न्त्योष॑धयो मनु॒ष्या॑णाम् । \newline
60. ओष॑धयो मनु॒ष्या॑णाम् मनु॒ष्या॑णा॒ मोष॑धय॒ ओष॑धयो मनु॒ष्या॑णा॒ मन्न॒ मन्न॑म् मनु॒ष्या॑णा॒ मोष॑धय॒ ओष॑धयो मनु॒ष्या॑णा॒ मन्न᳚म् । \newline
\pagebreak
\markright{ TS 3.3.6.3  \hfill https://www.vedavms.in \hfill}

\section{ TS 3.3.6.3 }

\textbf{TS 3.3.6.3 } \newline
\textbf{Samhita Paata} \newline

मनु॒ष्या॑णा॒मन्नं॑ प्र॒जाप॑तिं प्र॒जा अनु॒ प्रजा॑यन्त॒ इति॒ परा॒नन्विति॑ ब्रूया॒द्-यद्-गृ॒ह्णात्य॒द्भ्यस्त्वौष॑धीभ्यो गृह्णा॒मीति॒ तस्मा॑द॒द्भ्य ओष॑धयः॒ संभ॑वन्ति॒ यद्-गृ॒ह्णात्योष॑धीभ्यस्त्वा प्र॒जाभ्यो॑ गृह्णा॒मीति॒ तस्मा॒दोष॑धयो मनु॒ष्या॑णा॒मन्नं॒ ॅयद्-गृ॒ह्णाति॑ प्र॒जाभ्य॑स्त्वा प्र॒जाप॑तये गृह्णा॒मीति॒ तस्मा᳚त् प्र॒जाप॑तिं प्र॒जा अनु॒ प्रजा॑यन्ते ॥ \newline

\textbf{Pada Paata} \newline

म॒नु॒ष्या॑णाम् । अन्न᳚म् । प्र॒जाप॑ति॒मिति॑ प्र॒जा - प॒ति॒म् । प्र॒जा इति॑ प्र - जाः । अनु॑ । प्रेति॑ । जा॒य॒न्ते॒ । इति॑ । परान्॑ । अन्विति॑ । इति॑ । ब्रू॒या॒त् । यत् । गृ॒ह्णाति॑ । अ॒द्भ्य इत्य॑त् - भ्यः । त्वा॒ । ओष॑धीभ्य॒ इत्योष॑धि - भ्यः॒ । गृ॒ह्णा॒मि॒ । इति॑ । तस्मा᳚त् । अ॒द्भ्य इत्य॑त्-भ्यः । ओष॑धयः । समिति॑ । भ॒व॒न्ति॒ । यत् । गृ॒ह्णाति॑ । ओष॑धीभ्य॒ इत्योष॑धि - भ्यः॒ । त्वा॒ । प्र॒जाभ्य॒ इति॑ प्र - जाभ्यः॑ । गृ॒ह्णा॒मि॒ । इति॑ । तस्मा᳚त् । ओष॑धयः । म॒नु॒ष्या॑णाम् । अन्न᳚म् । यत् । गृ॒ह्णाति॑ । प्र॒जाभ्य॒ इति॑ प्र - जाभ्यः॑ । त्वा॒ । प्र॒जाप॑तय॒ इति॑ प्र॒जा - प॒त॒ये॒ । गृ॒ह्णा॒मि॒ । इति॑ । तस्मा᳚त् । प्र॒जाप॑ति॒मिति॑ प्र॒जा - प॒ति॒म् । प्र॒जा इति॑ प्र - जाः । अनु॑ । प्रेति॑ । जा॒य॒न्ते॒ ॥  \newline


\textbf{Krama Paata} \newline

म॒नु॒ष्या॑णा॒मन्न᳚म् । अन्न॑म् प्र॒जाप॑तिम् । प्र॒जाप॑तिम् प्र॒जाः । प्र॒जाप॑ति॒मिति॑ प्र॒जा - प॒ति॒म् । प्र॒जा अनु॑ । प्र॒जा इति॑ प्र - जाः । अनु॒ प्र । प्र जा॑यन्ते । जा॒य॒न्त॒ इति॑ । इति॒ परान्॑ । परा॒ननु॑ । अन्विति॑ । इति॑ ब्रूयात् । ब्रू॒या॒द् यत् । यद् गृ॒ह्णाति॑ । गृ॒ह्णात्य॒द्भ्यः । अ॒द्भ्यस्त्वा᳚ । अ॒द्भ्य इत्य॑त् - भ्यः । त्वौष॑धीभ्यः । ओष॑धीभ्यो गृह्णामि । ओष॑धीभ्य॒ इत्योष॑धि - भ्यः॒ । गृ॒ह्णा॒मीति॑ । इति॒ तस्मा᳚त् । तस्मा॑द॒द्भ्यः । अ॒द्भ्य ओष॑धयः । अ॒द्भ्य इत्य॑त् - भ्यः । ओष॑धयः॒ सम् । सम् भ॑वन्ति । भ॒व॒न्ति॒ यत् । यद् गृ॒ह्णाति॑ । गृ॒ह्णात्योष॑धीभ्यः । ओष॑धीभ्यस्त्वा । ओष॑धीभ्य॒ इत्योष॑धि - भ्यः॒ । त्वा॒ प्र॒जाभ्यः॑ । प्र॒जाभ्यो॑ गृह्णामि । प्र॒जाभ्य॒ इति॑ प्र - जाभ्यः॑ । गृ॒ह्णा॒मीति॑ । इति॒ तस्मा᳚त् । तस्मा॒दोष॑धयः । ओष॑धयो मनु॒ष्या॑णाम् । म॒नु॒ष्या॑णा॒मन्न᳚म् । अन्नं॒ ॅयत् । यद् गृ॒ह्णाति॑ । गृ॒ह्णाति॑ प्र॒जाभ्यः॑ । प्र॒जाभ्य॑स्त्वा । प्र॒जाभ्य॒ इति॑ प्र - जाभ्यः॑ । त्वा॒ प्र॒जाप॑तये । प्र॒जाप॑तये गृह्णामि । प्र॒जाप॑तय॒ इति॑ प्र॒जा - प॒त॒ये॒ । गृ॒ह्णा॒मीति॑ । इति॒ तस्मा᳚त् । तस्मा᳚त् प्र॒जाप॑तिम् । प्र॒जाप॑तिम् प्र॒जाः । प्र॒जाप॑ति॒मिति॑ प्र॒जा - प॒ति॒म् । प्र॒जा अनु॑ । प्र॒जा इति॑ प्र - जाः । अनु॒ प्र । प्र जा॑यन्ते । जा॒य॒न्त॒ इति॑ जायन्ते । \newline

\textbf{Jatai Paata} \newline

1. म॒नु॒ष्या॑णा॒ मन्न॒ मन्न॑म् मनु॒ष्या॑णाम् मनु॒ष्या॑णा॒ मन्न᳚म् । \newline
2. अन्न॑म् प्र॒जाप॑तिम् प्र॒जाप॑ति॒ मन्न॒ मन्न॑म् प्र॒जाप॑तिम् । \newline
3. प्र॒जाप॑तिम् प्र॒जाः प्र॒जाः प्र॒जाप॑तिम् प्र॒जाप॑तिम् प्र॒जाः । \newline
4. प्र॒जाप॑ति॒मिति॑ प्र॒जा - प॒ति॒म् । \newline
5. प्र॒जा अन्वनु॑ प्र॒जाः प्र॒जा अनु॑ । \newline
6. प्र॒जा इति॑ प्र - जाः । \newline
7. अनु॒ प्र प्राण्वनु॒ प्र । \newline
8. प्र जा॑यन्ते जायन्ते॒ प्र प्र जा॑यन्ते । \newline
9. जा॒य॒न्त॒ इतीति॑ जायन्ते जायन्त॒ इति॑ । \newline
10. इति॒ परा॒न् परा॒ नितीति॒ परान्॑ । \newline
11. परा॒ नन्वनु॒ परा॒न् परा॒ ननु॑ । \newline
12. अन्विती त्यन्वन्विति॑ । \newline
13. इति॑ ब्रूयाद् ब्रूया॒ दितीति॑ ब्रूयात् । \newline
14. ब्रू॒या॒द् यद् यद् ब्रू॑याद् ब्रूया॒द् यत् । \newline
15. यद् गृ॒ह्णाति॑ गृ॒ह्णाति॒ यद् यद् गृ॒ह्णाति॑ । \newline
16. गृ॒ह्णा त्य॒द्भ्यो᳚ ऽद्भ्यो गृ॒ह्णाति॑ गृ॒ह्णा त्य॒द्भ्यः । \newline
17. अ॒द्भ्य स्त्वा᳚ त्वा॒ ऽद्भ्यो᳚ ऽद्भ्य स्त्वा᳚ । \newline
18. अ॒द्भ्य इत्य॑त् - भ्यः । \newline
19. त्वौष॑धीभ्य॒ ओष॑धीभ्य स्त्वा॒ त्वौष॑धीभ्यः । \newline
20. ओष॑धीभ्यो गृह्णामि गृह्णा॒ म्योष॑धीभ्य॒ ओष॑धीभ्यो गृह्णामि । \newline
21. ओष॑धीभ्य॒ इत्योष॑धि - भ्यः॒ । \newline
22. गृ॒ह्णा॒मीतीति॑ गृह्णामि गृह्णा॒मीति॑ । \newline
23. इति॒ तस्मा॒त् तस्मा॒ दितीति॒ तस्मा᳚त् । \newline
24. तस्मा॑ द॒द्भ्यो᳚ ऽद्भ्य स्तस्मा॒त् तस्मा॑ द॒द्भ्यः । \newline
25. अ॒द्भ्य ओष॑धय॒ ओष॑धयो॒ ऽद्भ्यो᳚ ऽद्भ्य ओष॑धयः । \newline
26. अ॒द्भ्य इत्य॑त् - भ्यः । \newline
27. ओष॑धयः॒ सꣳ स मोष॑धय॒ ओष॑धयः॒ सम् । \newline
28. सम् भ॑वन्ति भवन्ति॒ सꣳ सम् भ॑वन्ति । \newline
29. भ॒व॒न्ति॒ यद् यद् भ॑वन्ति भवन्ति॒ यत् । \newline
30. यद् गृ॒ह्णाति॑ गृ॒ह्णाति॒ यद् यद् गृ॒ह्णाति॑ । \newline
31. गृ॒ह्णा त्योष॑धीभ्य॒ ओष॑धीभ्यो गृ॒ह्णाति॑ गृ॒ह्णा त्योष॑धीभ्यः । \newline
32. ओष॑धीभ्य स्त्वा॒ त्वौष॑धीभ्य॒ ओष॑धीभ्य स्त्वा । \newline
33. ओष॑धीभ्य॒ इत्योष॑धि - भ्यः॒ । \newline
34. त्वा॒ प्र॒जाभ्यः॑ प्र॒जाभ्य॑ स्त्वा त्वा प्र॒जाभ्यः॑ । \newline
35. प्र॒जाभ्यो॑ गृह्णामि गृह्णामि प्र॒जाभ्यः॑ प्र॒जाभ्यो॑ गृह्णामि । \newline
36. प्र॒जाभ्य॒ इति॑ प्र - जाभ्यः॑ । \newline
37. गृ॒ह्णा॒मीतीति॑ गृह्णामि गृह्णा॒मीति॑ । \newline
38. इति॒ तस्मा॒त् तस्मा॒ दितीति॒ तस्मा᳚त् । \newline
39. तस्मा॒ दोष॑धय॒ ओष॑धय॒ स्तस्मा॒त् तस्मा॒ दोष॑धयः । \newline
40. ओष॑धयो मनु॒ष्या॑णाम् मनु॒ष्या॑णा॒ मोष॑धय॒ ओष॑धयो मनु॒ष्या॑णाम् । \newline
41. म॒नु॒ष्या॑णा॒ मन्न॒ मन्न॑म् मनु॒ष्या॑णाम् मनु॒ष्या॑णा॒ मन्न᳚म् । \newline
42. अन्न॒म् ॅयद् यदन्न॒ मन्न॒म् ॅयत् । \newline
43. यद् गृ॒ह्णाति॑ गृ॒ह्णाति॒ यद् यद् गृ॒ह्णाति॑ । \newline
44. गृ॒ह्णाति॑ प्र॒जाभ्यः॑ प्र॒जाभ्यो॑ गृ॒ह्णाति॑ गृ॒ह्णाति॑ प्र॒जाभ्यः॑ । \newline
45. प्र॒जाभ्य॑ स्त्वा त्वा प्र॒जाभ्यः॑ प्र॒जाभ्य॑ स्त्वा । \newline
46. प्र॒जाभ्य॒ इति॑ प्र - जाभ्यः॑ । \newline
47. त्वा॒ प्र॒जाप॑तये प्र॒जाप॑तये त्वा त्वा प्र॒जाप॑तये । \newline
48. प्र॒जाप॑तये गृह्णामि गृह्णामि प्र॒जाप॑तये प्र॒जाप॑तये गृह्णामि । \newline
49. प्र॒जाप॑तय॒ इति॑ प्र॒जा - प॒त॒ये॒ । \newline
50. गृ॒ह्णा॒मीतीति॑ गृह्णामि गृह्णा॒मीति॑ । \newline
51. इति॒ तस्मा॒त् तस्मा॒ दितीति॒ तस्मा᳚त् । \newline
52. तस्मा᳚त् प्र॒जाप॑तिम् प्र॒जाप॑ति॒म् तस्मा॒त् तस्मा᳚त् प्र॒जाप॑तिम् । \newline
53. प्र॒जाप॑तिम् प्र॒जाः प्र॒जाः प्र॒जाप॑तिम् प्र॒जाप॑तिम् प्र॒जाः । \newline
54. प्र॒जाप॑ति॒मिति॑ प्र॒जा - प॒ति॒म् । \newline
55. प्र॒जा अन्वनु॑ प्र॒जाः प्र॒जा अनु॑ । \newline
56. प्र॒जा इति॑ प्र - जाः । \newline
57. अनु॒ प्र प्राण्वनु॒ प्र । \newline
58. प्र जा॑यन्ते जायन्ते॒ प्र प्र जा॑यन्ते । \newline
59. जा॒य॒न्त॒ इति॑ जायन्ते । \newline

\textbf{Ghana Paata } \newline

1. म॒नु॒ष्या॑णा॒ मन्न॒ मन्न॑म् मनु॒ष्या॑णाम् मनु॒ष्या॑णा॒ मन्न॑म् प्र॒जाप॑तिम् प्र॒जाप॑ति॒ मन्न॑म् मनु॒ष्या॑णाम् मनु॒ष्या॑णा॒ मन्न॑म् प्र॒जाप॑तिम् । \newline
2. अन्न॑म् प्र॒जाप॑तिम् प्र॒जाप॑ति॒ मन्न॒ मन्न॑म् प्र॒जाप॑तिम् प्र॒जाः प्र॒जाः प्र॒जाप॑ति॒ मन्न॒ मन्न॑म् प्र॒जाप॑तिम् प्र॒जाः । \newline
3. प्र॒जाप॑तिम् प्र॒जाः प्र॒जाः प्र॒जाप॑तिम् प्र॒जाप॑तिम् प्र॒जा अन्वनु॑ प्र॒जाः प्र॒जाप॑तिम् प्र॒जाप॑तिम् प्र॒जा अनु॑ । \newline
4. प्र॒जाप॑ति॒मिति॑ प्र॒जा - प॒ति॒म् । \newline
5. प्र॒जा अन्वनु॑ प्र॒जाः प्र॒जा अनु॒ प्र प्राणु॑ प्र॒जाः प्र॒जा अनु॒ प्र । \newline
6. प्र॒जा इति॑ प्र - जाः । \newline
7. अनु॒ प्र प्राण्वनु॒ प्र जा॑यन्ते जायन्ते॒ प्राण्वनु॒ प्र जा॑यन्ते । \newline
8. प्र जा॑यन्ते जायन्ते॒ प्र प्र जा॑यन्त॒ इतीति॑ जायन्ते॒ प्र प्र जा॑यन्त॒ इति॑ । \newline
9. जा॒य॒न्त॒ इतीति॑ जायन्ते जायन्त॒ इति॒ परा॒न् परा॒ निति॑ जायन्ते जायन्त॒ इति॒ परान्॑ । \newline
10. इति॒ परा॒न् परा॒ नितीति॒ परा॒ नन्वनु॒ परा॒ नितीति॒ परा॒ ननु॑ । \newline
11. परा॒ नन्वनु॒ परा॒न् परा॒ नन्विती त्यनु॒ परा॒न् परा॒ नन्विति॑ । \newline
12. अन्विती त्यन्वन् विति॑ ब्रूयाद् ब्रूया॒दि त्यन्वन्विति॑ ब्रूयात् । \newline
13. इति॑ ब्रूयाद् ब्रूया॒ दितीति॑ ब्रूया॒द् यद् यद् ब्रू॑या॒ दितीति॑ ब्रूया॒द् यत् । \newline
14. ब्रू॒या॒द् यद् यद् ब्रू॑याद् ब्रूया॒द् यद् गृ॒ह्णाति॑ गृ॒ह्णाति॒ यद् ब्रू॑याद् ब्रूया॒द् यद् गृ॒ह्णाति॑ । \newline
15. यद् गृ॒ह्णाति॑ गृ॒ह्णाति॒ यद् यद् गृ॒ह्णा त्य॒द्भ्यो᳚ ऽद्भ्यो गृ॒ह्णाति॒ यद् यद् गृ॒ह्णा त्य॒द्भ्यः । \newline
16. गृ॒ह्णा त्य॒द्भ्यो᳚ ऽद्भ्यो गृ॒ह्णाति॑ गृ॒ह्णा त्य॒द्भ्य स्त्वा᳚ त्वा॒ ऽद्भ्यो गृ॒ह्णाति॑ गृ॒ह्णा त्य॒द्भ्य स्त्वा᳚ । \newline
17. अ॒द्भ्य स्त्वा᳚ त्वा॒ ऽद्भ्यो᳚ ऽद्भ्य स्त्वौष॑धीभ्य॒ ओष॑धीभ्य स्त्वा॒ ऽद्भ्यो᳚ ऽद्भ्य स्त्वौष॑धीभ्यः । \newline
18. अ॒द्भ्य इत्य॑त् - भ्यः । \newline
19. त्वौष॑धीभ्य॒ ओष॑धीभ्य स्त्वा॒ त्वौष॑धीभ्यो गृह्णामि गृह्णा॒ म्योष॑धीभ्य स्त्वा॒ त्वौष॑धीभ्यो गृह्णामि । \newline
20. ओष॑धीभ्यो गृह्णामि गृह्णा॒ म्योष॑धीभ्य॒ ओष॑धीभ्यो गृह्णा॒मीतीति॑ गृह्णा॒ म्योष॑धीभ्य॒ ओष॑धीभ्यो गृह्णा॒मीति॑ । \newline
21. ओष॑धीभ्य॒ इत्योष॑धि - भ्यः॒ । \newline
22. गृ॒ह्णा॒मीतीति॑ गृह्णामि गृह्णा॒मीति॒ तस्मा॒त् तस्मा॒ दिति॑ गृह्णामि गृह्णा॒मीति॒ तस्मा᳚त् । \newline
23. इति॒ तस्मा॒त् तस्मा॒ दितीति॒ तस्मा॑द॒ द्भ्यो᳚ ऽद्भ्य स्तस्मा॒ दितीति॒ तस्मा॑ द॒द्भ्यः । \newline
24. तस्मा॑ द॒द्भ्यो᳚ ऽद्भ्य स्तस्मा॒त् तस्मा॑ द॒द्भ्य ओष॑धय॒ ओष॑धयो॒ ऽद्भ्य स्तस्मा॒त् तस्मा॑ द॒द्भ्य ओष॑धयः । \newline
25. अ॒द्भ्य ओष॑धय॒ ओष॑धयो॒ ऽद्भ्यो᳚ ऽद्भ्य ओष॑धयः॒ सꣳ स मोष॑धयो॒ ऽद्भ्यो᳚ ऽद्भ्य ओष॑धयः॒ सम् । \newline
26. अ॒द्भ्य इत्य॑त् - भ्यः । \newline
27. ओष॑धयः॒ सꣳ स मोष॑धय॒ ओष॑धयः॒ सम् भ॑वन्ति भवन्ति॒ स मोष॑धय॒ ओष॑धयः॒ सम् भ॑वन्ति । \newline
28. सम् भ॑वन्ति भवन्ति॒ सꣳ सम् भ॑वन्ति॒ यद् यद् भ॑वन्ति॒ सꣳ सम् भ॑वन्ति॒ यत् । \newline
29. भ॒व॒न्ति॒ यद् यद् भ॑वन्ति भवन्ति॒ यद् गृ॒ह्णाति॑ गृ॒ह्णाति॒ यद् भ॑वन्ति भवन्ति॒ यद् गृ॒ह्णाति॑ । \newline
30. यद् गृ॒ह्णाति॑ गृ॒ह्णाति॒ यद् यद् गृ॒ह्णा त्योष॑धीभ्य॒ ओष॑धीभ्यो गृ॒ह्णाति॒ यद् यद् गृ॒ह्णा त्योष॑धीभ्यः । \newline
31. गृ॒ह्णा त्योष॑धीभ्य॒ ओष॑धीभ्यो गृ॒ह्णाति॑ गृ॒ह्णा त्योष॑धीभ्य स्त्वा॒ त्वौष॑धीभ्यो गृ॒ह्णाति॑ गृ॒ह्णा त्योष॑धीभ्य स्त्वा । \newline
32. ओष॑धीभ्य स्त्वा॒ त्वौष॑धीभ्य॒ ओष॑धीभ्य स्त्वा प्र॒जाभ्यः॑ प्र॒जाभ्य॒ स्त्वौष॑धीभ्य॒ ओष॑धीभ्य स्त्वा प्र॒जाभ्यः॑ । \newline
33. ओष॑धीभ्य॒ इत्योष॑धि - भ्यः॒ । \newline
34. त्वा॒ प्र॒जाभ्यः॑ प्र॒जाभ्य॑ स्त्वा त्वा प्र॒जाभ्यो॑ गृह्णामि गृह्णामि प्र॒जाभ्य॑ स्त्वा त्वा प्र॒जाभ्यो॑ गृह्णामि । \newline
35. प्र॒जाभ्यो॑ गृह्णामि गृह्णामि प्र॒जाभ्यः॑ प्र॒जाभ्यो॑ गृह्णा॒मीतीति॑ गृह्णामि प्र॒जाभ्यः॑ प्र॒जाभ्यो॑ गृह्णा॒मीति॑ । \newline
36. प्र॒जाभ्य॒ इति॑ प्र - जाभ्यः॑ । \newline
37. गृ॒ह्णा॒मीतीति॑ गृह्णामि गृह्णा॒मीति॒ तस्मा॒त् तस्मा॒ दिति॑ गृह्णामि गृह्णा॒मीति॒ तस्मा᳚त् । \newline
38. इति॒ तस्मा॒त् तस्मा॒ दितीति॒ तस्मा॒ दोष॑धय॒ ओष॑धय॒ स्तस्मा॒ दितीति॒ तस्मा॒ दोष॑धयः । \newline
39. तस्मा॒ दोष॑धय॒ ओष॑धय॒ स्तस्मा॒त् तस्मा॒ दोष॑धयो मनु॒ष्या॑णाम् मनु॒ष्या॑णा॒ मोष॑धय॒ स्तस्मा॒त् तस्मा॒ दोष॑धयो मनु॒ष्या॑णाम् । \newline
40. ओष॑धयो मनु॒ष्या॑णाम् मनु॒ष्या॑णा॒ मोष॑धय॒ ओष॑धयो मनु॒ष्या॑णा॒ मन्न॒ मन्न॑म् मनु॒ष्या॑णा॒ मोष॑धय॒ ओष॑धयो मनु॒ष्या॑णा॒ मन्न᳚म् । \newline
41. म॒नु॒ष्या॑णा॒ मन्न॒ मन्न॑म् मनु॒ष्या॑णाम् मनु॒ष्या॑णा॒ मन्न॒म् ॅयद् यदन्न॑म् मनु॒ष्या॑णाम् मनु॒ष्या॑णा॒ मन्न॒म् ॅयत् । \newline
42. अन्न॒म् ॅयद् यदन्न॒ मन्न॒म् ॅयद् गृ॒ह्णाति॑ गृ॒ह्णाति॒ यदन्न॒ मन्न॒म् ॅयद् गृ॒ह्णाति॑ । \newline
43. यद् गृ॒ह्णाति॑ गृ॒ह्णाति॒ यद् यद् गृ॒ह्णाति॑ प्र॒जाभ्यः॑ प्र॒जाभ्यो॑ गृ॒ह्णाति॒ यद् यद् गृ॒ह्णाति॑ प्र॒जाभ्यः॑ । \newline
44. गृ॒ह्णाति॑ प्र॒जाभ्यः॑ प्र॒जाभ्यो॑ गृ॒ह्णाति॑ गृ॒ह्णाति॑ प्र॒जाभ्य॑ स्त्वा त्वा प्र॒जाभ्यो॑ गृ॒ह्णाति॑ गृ॒ह्णाति॑ प्र॒जाभ्य॑ स्त्वा । \newline
45. प्र॒जाभ्य॑ स्त्वा त्वा प्र॒जाभ्यः॑ प्र॒जाभ्य॑ स्त्वा प्र॒जाप॑तये प्र॒जाप॑तये त्वा प्र॒जाभ्यः॑ प्र॒जाभ्य॑ स्त्वा प्र॒जाप॑तये । \newline
46. प्र॒जाभ्य॒ इति॑ प्र - जाभ्यः॑ । \newline
47. त्वा॒ प्र॒जाप॑तये प्र॒जाप॑तये त्वा त्वा प्र॒जाप॑तये गृह्णामि गृह्णामि प्र॒जाप॑तये त्वा त्वा प्र॒जाप॑तये गृह्णामि । \newline
48. प्र॒जाप॑तये गृह्णामि गृह्णामि प्र॒जाप॑तये प्र॒जाप॑तये गृह्णा॒मीतीति॑ गृह्णामि प्र॒जाप॑तये प्र॒जाप॑तये गृह्णा॒मीति॑ । \newline
49. प्र॒जाप॑तय॒ इति॑ प्र॒जा - प॒त॒ये॒ । \newline
50. गृ॒ह्णा॒मीतीति॑ गृह्णामि गृह्णा॒मीति॒ तस्मा॒त् तस्मा॒ दिति॑ गृह्णामि गृह्णा॒मीति॒ तस्मा᳚त् । \newline
51. इति॒ तस्मा॒त् तस्मा॒ दितीति॒ तस्मा᳚त् प्र॒जाप॑तिम् प्र॒जाप॑ति॒म् तस्मा॒ दितीति॒ तस्मा᳚त् प्र॒जाप॑तिम् । \newline
52. तस्मा᳚त् प्र॒जाप॑तिम् प्र॒जाप॑ति॒म् तस्मा॒त् तस्मा᳚त् प्र॒जाप॑तिम् प्र॒जाः प्र॒जाः प्र॒जाप॑ति॒म् तस्मा॒त् तस्मा᳚त् प्र॒जाप॑तिम् प्र॒जाः । \newline
53. प्र॒जाप॑तिम् प्र॒जाः प्र॒जाः प्र॒जाप॑तिम् प्र॒जाप॑तिम् प्र॒जा अन्वनु॑ प्र॒जाः प्र॒जाप॑तिम् प्र॒जाप॑तिम् प्र॒जा अनु॑ । \newline
54. प्र॒जाप॑ति॒मिति॑ प्र॒जा - प॒ति॒म् । \newline
55. प्र॒जा अन्वनु॑ प्र॒जाः प्र॒जा अनु॒ प्र प्राणु॑ प्र॒जाः प्र॒जा अनु॒ प्र । \newline
56. प्र॒जा इति॑ प्र - जाः । \newline
57. अनु॒ प्र प्राण्वनु॒ प्र जा॑यन्ते जायन्ते॒ प्राण्वनु॒ प्र जा॑यन्ते । \newline
58. प्र जा॑यन्ते जायन्ते॒ प्र प्र जा॑यन्ते । \newline
59. जा॒य॒न्त॒ इति॑ जायन्ते । \newline
\pagebreak
\markright{ TS 3.3.7.1  \hfill https://www.vedavms.in \hfill}

\section{ TS 3.3.7.1 }

\textbf{TS 3.3.7.1 } \newline
\textbf{Samhita Paata} \newline

प्र॒जाप॑तिर्देवासु॒रान॑ सृजत॒ तदनु॑ य॒ज्ञो॑ऽसृज्यत य॒ज्ञ्ं छन्दाꣳ॑सि॒ ते विष्व॑ञ्चो॒ व्य॑क्राम॒न्थ् सोऽसु॑रा॒ननु॑ य॒ज्ञोऽपा᳚क्रामद्-य॒ज्ञ्ं छन्दाꣳ॑सि॒ ते दे॒वा अ॑मन्यन्ता॒मी वा इ॒दम॑भूव॒न॒. यद्-व॒यꣳ स्म इति॒ ते प्र॒जाप॑ति॒मुपा॑ऽधाव॒न्थ् सो᳚ऽब्रवीत्-प्र॒जाप॑ति॒श्छन्द॑सां ॅवी॒र्य॑मा॒दाय॒ तद्वः॒ प्र दा᳚स्या॒मीति॒ स छन्द॑सां ॅवी॒र्य॑ - [  ] \newline

\textbf{Pada Paata} \newline

प्र॒जाप॑ति॒रिति॑ प्र॒जा - प॒तिः॒ । दे॒वा॒सु॒रानिति॑ देव - अ॒सु॒रान् । अ॒सृ॒ज॒त॒ । तत् । अन्विति॑ । य॒ज्ञ्ः । अ॒सृ॒ज्य॒त॒ । य॒ज्ञ्म् । छन्दाꣳ॑सि । ते । विष्व॑ञ्चः । वीति॑ । अ॒क्रा॒म॒न्न् । सः । असु॑रान् । अन्विति॑ । य॒ज्ञ्ः । अपेति॑ । अ॒क्रा॒म॒त् । य॒ज्ञ्म् । छन्दाꣳ॑सि । ते । दे॒वाः । अ॒म॒न्य॒न्त॒ । अ॒मी इति॑ । वै । इ॒दम् । अ॒भू॒व॒न्न् । यत् । व॒यम् । स्मः । इति॑ । ते । प्र॒जाप॑ति॒मिति॑ प्र॒जा-प॒ति॒म् । उपेति॑ । अ॒धा॒व॒न्न् । सः । अ॒ब्र॒वी॒त् । प्र॒जाप॑ति॒रिति॑ प्र॒जा - प॒तिः॒ । छन्द॑साम् । वी॒र्य᳚म् । आ॒दायेत्या᳚ - दाय॑ । तत् । वः॒ । प्रेति॑ । दा॒स्या॒मि॒ । इति॑ । सः । छन्द॑साम् । वी॒र्य᳚म् ।  \newline


\textbf{Krama Paata} \newline

प्र॒जाप॑तिर् देवासु॒रान् । प्र॒जाप॑ति॒रिति॑ प्र॒जा - प॒तिः॒ । दे॒वा॒सु॒रान॑सृजत । दे॒वा॒सु॒रानिति॑ देव - अ॒सु॒रान् । अ॒सृ॒ज॒त॒ तत् । तदनु॑ । अनु॑ य॒ज्ञ्ः । य॒ज्ञो॑ ऽसृज्यत । अ॒सृ॒ज्य॒त॒ य॒ज्ञ्म् । य॒ज्ञ्म् छन्दाꣳ॑सि । छन्दाꣳ॑सि॒ ते । ते विष्व॑ञ्चः । विष्व॑ञ्चो॒ वि । व्य॑क्रामन्न् । अ॒क्रा॒म॒न्थ् सः । सो ऽसु॑रान् । असु॑रा॒ननु॑ । अनु॑ य॒ज्ञ्ः । य॒ज्ञो ऽप॑ । अपा᳚क्रामत् । अ॒क्रा॒म॒द् य॒ज्ञ्म् । य॒ज्ञ्म् छन्दाꣳ॑सि । छन्दाꣳ॑सि॒ ते । ते दे॒वाः । दे॒वा अ॑मन्यन्त । अ॒म॒न्य॒न्ता॒मी । अ॒मी वै । अ॒मी इत्य॒मी । वा इ॒दम् । इ॒दम॑भूवन्न् । अ॒भू॒व॒न्॒. यत् । यद् व॒यम् । व॒यꣳ स्मः । स्म इति॑ । इति॒ ते । ते प्र॒जाप॑तिम् । प्र॒जाप॑ति॒मुप॑ । प्र॒जाप॑ति॒मिति॑ प्र॒जा - प॒ति॒म् । उपा॑धावन्न् । अ॒धा॒व॒न्थ् सः । सो᳚ ऽब्रवीत् । अ॒ब्र॒वी॒त् प्र॒जाप॑तिः । प्र॒जाप॑ति॒ श्छन्द॑साम् । प्र॒जाप॑ति॒रिति॑ प्र॒जा - प॒तिः॒ । छन्द॑सां ॅवी॒र्य᳚म् । वी॒र्य॑मा॒दाय॑ । आ॒दाय॒ तत् । आ॒दायेत्या᳚ - दाय॑ । तद् वः॑ । वः॒ प्र । प्र दा᳚स्यामि । दा॒स्या॒मीति॑ । इति॒ सः । स छन्द॑साम् । छन्द॑सां ॅवी॒र्य᳚म् । वी॒र्य॑मा॒दाय॑ \newline

\textbf{Jatai Paata} \newline

1. प्र॒जाप॑तिर् देवासु॒रान् दे॑वासु॒रान् प्र॒जाप॑तिः प्र॒जाप॑तिर् देवासु॒रान् । \newline
2. प्र॒जाप॑ति॒रिति॑ प्र॒जा - प॒तिः॒ । \newline
3. दे॒वा॒सु॒रा न॑सृजता सृजत देवासु॒रान् दे॑वासु॒रा न॑सृजत । \newline
4. दे॒वा॒सु॒रानिति॑ देव - अ॒सु॒रान् । \newline
5. अ॒सृ॒ज॒त॒ तत् तद॑सृजता सृजत॒ तत् । \newline
6. तदन्वनु॒ तत् तदनु॑ । \newline
7. अनु॑ य॒ज्ञो य॒ज्ञो ऽन्वनु॑ य॒ज्ञ्ः । \newline
8. य॒ज्ञो॑ ऽसृज्यता सृज्यत य॒ज्ञो य॒ज्ञो॑ ऽसृज्यत । \newline
9. अ॒सृ॒ज्य॒त॒ य॒ज्ञ्म् ॅय॒ज्ञ् म॑सृज्यता सृज्यत य॒ज्ञ्म् । \newline
10. य॒ज्ञ्म् छन्दाꣳ॑सि॒ छन्दाꣳ॑सि य॒ज्ञ्म् ॅय॒ज्ञ्म् छन्दाꣳ॑सि । \newline
11. छन्दाꣳ॑सि॒ ते ते छन्दाꣳ॑सि॒ छन्दाꣳ॑सि॒ ते । \newline
12. ते विष्व॑ञ्चो॒ विष्व॑ञ्च॒ स्ते ते विष्व॑ञ्चः । \newline
13. विष्व॑ञ्चो॒ वि वि विष्व॑ञ्चो॒ विष्व॑ञ्चो॒ वि । \newline
14. व्य॑क्रामन् नक्राम॒न्॒. वि व्य॑क्रामन्न् । \newline
15. अ॒क्रा॒म॒न् थ्स सो᳚ ऽक्रामन् नक्राम॒न् थ्सः । \newline
16. सो ऽसु॑रा॒ नसु॑रा॒न् थ्स सो ऽसु॑रान् । \newline
17. असु॑रा॒ नन्वन्व सु॑रा॒ नसु॑रा॒ ननु॑ । \newline
18. अनु॑ य॒ज्ञो य॒ज्ञो ऽन्वनु॑ य॒ज्ञ्ः । \newline
19. य॒ज्ञो ऽपाप॑ य॒ज्ञो य॒ज्ञो ऽप॑ । \newline
20. अपा᳚ क्राम दक्राम॒ दपापा᳚ क्रामत् । \newline
21. अ॒क्रा॒म॒द् य॒ज्ञ्म् ॅय॒ज्ञ् म॑क्राम दक्रामद् य॒ज्ञ्म् । \newline
22. य॒ज्ञ्म् छन्दाꣳ॑सि॒ छन्दाꣳ॑सि य॒ज्ञ्म् ॅय॒ज्ञ्म् छन्दाꣳ॑सि । \newline
23. छन्दाꣳ॑सि॒ ते ते छन्दाꣳ॑सि॒ छन्दाꣳ॑सि॒ ते । \newline
24. ते दे॒वा दे॒वा स्ते ते दे॒वाः । \newline
25. दे॒वा अ॑मन्यन्ता मन्यन्त दे॒वा दे॒वा अ॑मन्यन्त । \newline
26. अ॒म॒न्य॒न्ता॒मी अ॒मी अ॑मन्यन्ता मन्यन्ता॒मी । \newline
27. अ॒मी वै वा अ॒मी अ॒मी वै । \newline
28. अ॒मी इत्य॒मी । \newline
29. वा इ॒द मि॒दम् ॅवै वा इ॒दम् । \newline
30. इ॒द म॑भूवन् नभूवन् नि॒द मि॒द म॑भूवन्न् । \newline
31. अ॒भू॒व॒न्॒. यद् यद॑भूवन् नभूव॒न्॒. यत् । \newline
32. यद् व॒यम् ॅव॒यम् ॅयद् यद् व॒यम् । \newline
33. व॒यꣳ स्मः स्मो व॒यम् ॅव॒यꣳ स्मः । \newline
34. स्म इतीति॒ स्मः स्म इति॑ । \newline
35. इति॒ ते त इतीति॒ ते । \newline
36. ते प्र॒जाप॑तिम् प्र॒जाप॑ति॒म् ते ते प्र॒जाप॑तिम् । \newline
37. प्र॒जाप॑ति॒ मुपोप॑ प्र॒जाप॑तिम् प्र॒जाप॑ति॒ मुप॑ । \newline
38. प्र॒जाप॑ति॒मिति॑ प्र॒जा - प॒ति॒म् । \newline
39. उपा॑धावन् नधाव॒न् नुपोपा॑ धावन्न् । \newline
40. अ॒धा॒व॒न् थ्स सो॑ ऽधावन् नधाव॒न् थ्सः । \newline
41. सो᳚ ऽब्रवी दब्रवी॒थ् स सो᳚ ऽब्रवीत् । \newline
42. अ॒ब्र॒वी॒त् प्र॒जाप॑तिः प्र॒जाप॑ति रब्रवी दब्रवीत् प्र॒जाप॑तिः । \newline
43. प्र॒जाप॑ति॒ श्छन्द॑सा॒म् छन्द॑साम् प्र॒जाप॑तिः प्र॒जाप॑ति॒ श्छन्द॑साम् । \newline
44. प्र॒जाप॑ति॒रिति॑ प्र॒जा - प॒तिः॒ । \newline
45. छन्द॑साम् ॅवी॒र्य॑म् ॅवी॒र्य॑म् छन्द॑सा॒म् छन्द॑साम् ॅवी॒र्य᳚म् । \newline
46. वी॒र्य॑ मा॒दाया॒ दाय॑ वी॒र्य॑म् ॅवी॒र्य॑ मा॒दाय॑ । \newline
47. आ॒दाय॒ तत् तदा॒ दाया॒ दाय॒ तत् । \newline
48. आ॒दायेत्या᳚ - दाय॑ । \newline
49. तद् वो॑ व॒ स्तत् तद् वः॑ । \newline
50. वः॒ प्र प्र वो॑ वः॒ प्र । \newline
51. प्र दा᳚स्यामि दास्यामि॒ प्र प्र दा᳚स्यामि । \newline
52. दा॒स्या॒मीतीति॑ दास्यामि दास्या॒मीति॑ । \newline
53. इति॒ स स इतीति॒ सः । \newline
54. स छन्द॑सा॒म् छन्द॑साꣳ॒॒ स स छन्द॑साम् । \newline
55. छन्द॑साम् ॅवी॒र्य॑म् ॅवी॒र्य॑म् छन्द॑सा॒म् छन्द॑साम् ॅवी॒र्य᳚म् । \newline
56. वी॒र्य॑ मा॒दाया॒ दाय॑ वी॒र्य॑म् ॅवी॒र्य॑ मा॒दाय॑ । \newline

\textbf{Ghana Paata } \newline

1. प्र॒जाप॑तिर् देवासु॒रान् दे॑वासु॒रान् प्र॒जाप॑तिः प्र॒जाप॑तिर् देवासु॒रा न॑सृजता सृजत देवासु॒रान् प्र॒जाप॑तिः प्र॒जाप॑तिर् देवासु॒रा न॑सृजत । \newline
2. प्र॒जाप॑ति॒रिति॑ प्र॒जा - प॒तिः॒ । \newline
3. दे॒वा॒सु॒रा न॑सृजता सृजत देवासु॒रान् दे॑वासु॒रा न॑सृजत॒ तत् तद॑सृजत देवासु॒रान् दे॑वासु॒रा न॑सृजत॒ तत् । \newline
4. दे॒वा॒सु॒रानिति॑ देव - अ॒सु॒रान् । \newline
5. अ॒सृ॒ज॒त॒ तत् तद॑सृजता सृजत॒ तदन्वनु॒ तद॑सृजता सृजत॒ तदनु॑ । \newline
6. तदन्वनु॒ तत् तदनु॑ य॒ज्ञो य॒ज्ञो ऽनु॒ तत् तदनु॑ य॒ज्ञ्ः । \newline
7. अनु॑ य॒ज्ञो य॒ज्ञो ऽन्वनु॑ य॒ज्ञो॑ ऽसृज्यता सृज्यत य॒ज्ञो ऽन्वनु॑ य॒ज्ञो॑ ऽसृज्यत । \newline
8. य॒ज्ञो॑ ऽसृज्यता सृज्यत य॒ज्ञो य॒ज्ञो॑ ऽसृज्यत य॒ज्ञ्म् ॅय॒ज्ञ् म॑सृज्यत य॒ज्ञो य॒ज्ञो॑ ऽसृज्यत य॒ज्ञ्म् । \newline
9. अ॒सृ॒ज्य॒त॒ य॒ज्ञ्म् ॅय॒ज्ञ् म॑सृज्यता सृज्यत य॒ज्ञ्म् छन्दाꣳ॑सि॒ छन्दाꣳ॑सि य॒ज्ञ् म॑सृज्यता सृज्यत य॒ज्ञ्म् छन्दाꣳ॑सि । \newline
10. य॒ज्ञ्म् छन्दाꣳ॑सि॒ छन्दाꣳ॑सि य॒ज्ञ्म् ॅय॒ज्ञ्म् छन्दाꣳ॑सि॒ ते ते छन्दाꣳ॑सि य॒ज्ञ्म् ॅय॒ज्ञ्म् छन्दाꣳ॑सि॒ ते । \newline
11. छन्दाꣳ॑सि॒ ते ते छन्दाꣳ॑सि॒ छन्दाꣳ॑सि॒ ते विष्व॑ञ्चो॒ विष्व॑ञ्च॒ स्ते छन्दाꣳ॑सि॒ छन्दाꣳ॑सि॒ ते विष्व॑ञ्चः । \newline
12. ते विष्व॑ञ्चो॒ विष्व॑ञ्च॒ स्ते ते विष्व॑ञ्चो॒ वि वि विष्व॑ञ्च॒ स्ते ते विष्व॑ञ्चो॒ वि । \newline
13. विष्व॑ञ्चो॒ वि वि विष्व॑ञ्चो॒ विष्व॑ञ्चो॒ व्य॑क्रामन्, नक्राम॒न्॒. वि विष्व॑ञ्चो॒ विष्व॑ञ्चो॒ व्य॑क्रामन्न् । \newline
14. व्य॑क्रामन्, नक्राम॒न्॒. वि व्य॑क्राम॒न् थ्स सो᳚ ऽक्राम॒न्॒. वि व्य॑क्राम॒न् थ्सः । \newline
15. अ॒क्रा॒म॒न् थ्स सो᳚ ऽक्रामन्, नक्राम॒न् थ्सो ऽसु॑रा॒ नसु॑रा॒न् थ्सो᳚ ऽक्रामन्, नक्राम॒न् थ्सो ऽसु॑रान् । \newline
16. सो ऽसु॑रा॒ नसु॑रा॒न् थ्स सो ऽसु॑रा॒ नन्वन् वसु॑रा॒न् थ्स सो ऽसु॑रा॒ ननु॑ । \newline
17. असु॑रा॒ नन्वन् वसु॑रा॒ नसु॑रा॒ ननु॑ य॒ज्ञो य॒ज्ञो ऽन्वसु॑रा॒ नसु॑रा॒ ननु॑ य॒ज्ञ्ः । \newline
18. अनु॑ य॒ज्ञो य॒ज्ञो ऽन्वनु॑ य॒ज्ञो ऽपाप॑ य॒ज्ञो ऽन्वनु॑ य॒ज्ञो ऽप॑ । \newline
19. य॒ज्ञो ऽपाप॑ य॒ज्ञो य॒ज्ञो ऽपा᳚क्राम दक्राम॒ दप॑ य॒ज्ञो य॒ज्ञो ऽपा᳚क्रामत् । \newline
20. अपा᳚क्राम दक्राम॒ दपापा᳚क्रामद् य॒ज्ञ्म् ॅय॒ज्ञ् म॑क्राम॒ दपापा᳚क्रामद् य॒ज्ञ्म् । \newline
21. अ॒क्रा॒म॒द् य॒ज्ञ्म् ॅय॒ज्ञ् म॑क्राम दक्रामद् य॒ज्ञ्म् छन्दाꣳ॑सि॒ छन्दाꣳ॑सि य॒ज्ञ् म॑क्राम दक्रामद् य॒ज्ञ्म् छन्दाꣳ॑सि । \newline
22. य॒ज्ञ्म् छन्दाꣳ॑सि॒ छन्दाꣳ॑सि य॒ज्ञ्म् ॅय॒ज्ञ्म् छन्दाꣳ॑सि॒ ते ते छन्दाꣳ॑सि य॒ज्ञ्म् ॅय॒ज्ञ्म् छन्दाꣳ॑सि॒ ते । \newline
23. छन्दाꣳ॑सि॒ ते ते छन्दाꣳ॑सि॒ छन्दाꣳ॑सि॒ ते दे॒वा दे॒वा स्ते छन्दाꣳ॑सि॒ छन्दाꣳ॑सि॒ ते दे॒वाः । \newline
24. ते दे॒वा दे॒वा स्ते ते दे॒वा अ॑मन्यन्ता मन्यन्त दे॒वा स्ते ते दे॒वा अ॑मन्यन्त । \newline
25. दे॒वा अ॑मन्यन्ता मन्यन्त दे॒वा दे॒वा अ॑मन्यन्ता॒मी अ॒मी अ॑मन्यन्त दे॒वा दे॒वा अ॑मन्यन्ता॒मी । \newline
26. अ॒म॒न्य॒न्ता॒मी अ॒मी अ॑मन्यन्ता मन्यन्ता॒मी वै वा अ॒मी अ॑मन्यन्ता मन्यन्ता॒मी वै । \newline
27. अ॒मी वै वा अ॒मी अ॒मी वा इ॒द मि॒दम् ॅवा अ॒मी अ॒मी वा इ॒दम् । \newline
28. अ॒मी इत्य॒मी । \newline
29. वा इ॒द मि॒दम् ॅवै वा इ॒द म॑भूवन्, नभूवन्, नि॒दम् ॅवै वा इ॒द म॑भूवन्न् । \newline
30. इ॒द म॑भूवन्, नभूवन्, नि॒द मि॒द म॑भूव॒न्॒. यद् यद॑भूवन्, नि॒द मि॒द म॑भूव॒न्॒. यत् । \newline
31. अ॒भू॒व॒न्॒. यद् यद॑भूवन्, नभूव॒न्॒. यद् व॒यम् ॅव॒यम् ॅयद॑भूवन्, नभूव॒न्॒. यद् व॒यम् । \newline
32. यद् व॒यम् ॅव॒यम् ॅयद् यद् व॒यꣳ स्मः स्मो व॒यम् ॅयद् यद् व॒यꣳ स्मः । \newline
33. व॒यꣳ स्मः स्मो व॒यम् ॅव॒यꣳ स्म इतीति॒ स्मो व॒यम् ॅव॒यꣳ स्म इति॑ । \newline
34. स्म इतीति॒ स्मः स्म इति॒ ते त इति॒ स्मः स्म इति॒ ते । \newline
35. इति॒ ते त इतीति॒ ते प्र॒जाप॑तिम् प्र॒जाप॑ति॒म् त इतीति॒ ते प्र॒जाप॑तिम् । \newline
36. ते प्र॒जाप॑तिम् प्र॒जाप॑ति॒म् ते ते प्र॒जाप॑ति॒ मुपोप॑ प्र॒जाप॑ति॒म् ते ते प्र॒जाप॑ति॒ मुप॑ । \newline
37. प्र॒जाप॑ति॒ मुपोप॑ प्र॒जाप॑तिम् प्र॒जाप॑ति॒ मुपा॑धावन्, नधाव॒न्, नुप॑ प्र॒जाप॑तिम् प्र॒जाप॑ति॒ मुपा॑धावन्न् । \newline
38. प्र॒जाप॑ति॒मिति॑ प्र॒जा - प॒ति॒म् । \newline
39. उपा॑धावन्, नधाव॒न्, नुपोपा॑धाव॒न् थ्स सो॑ ऽधाव॒न्, नुपोपा॑धाव॒न् थ्सः । \newline
40. अ॒धा॒व॒न् थ्स सो॑ ऽधावन्, नधाव॒न् थ्सो᳚ ऽब्रवी दब्रवी॒थ् सो॑ ऽधावन्, नधाव॒न् थ्सो᳚ ऽब्रवीत् । \newline
41. सो᳚ ऽब्रवी दब्रवी॒थ् स सो᳚ ऽब्रवीत् प्र॒जाप॑तिः प्र॒जाप॑ति रब्रवी॒थ् स सो᳚ ऽब्रवीत् प्र॒जाप॑तिः । \newline
42. अ॒ब्र॒वी॒त् प्र॒जाप॑तिः प्र॒जाप॑ति रब्रवी दब्रवीत् प्र॒जाप॑ति॒ श्छन्द॑सा॒म् छन्द॑साम् प्र॒जाप॑ति रब्रवी दब्रवीत् प्र॒जाप॑ति॒ श्छन्द॑साम् । \newline
43. प्र॒जाप॑ति॒ श्छन्द॑सा॒म् छन्द॑साम् प्र॒जाप॑तिः प्र॒जाप॑ति॒ श्छन्द॑साम् ॅवी॒र्य॑म् ॅवी॒र्य॑म् छन्द॑साम् प्र॒जाप॑तिः प्र॒जाप॑ति॒ श्छन्द॑साम् ॅवी॒र्य᳚म् । \newline
44. प्र॒जाप॑ति॒रिति॑ प्र॒जा - प॒तिः॒ । \newline
45. छन्द॑साम् ॅवी॒र्य॑म् ॅवी॒र्य॑म् छन्द॑सा॒म् छन्द॑साम् ॅवी॒र्य॑ मा॒दाया॒ दाय॑ वी॒र्य॑म् छन्द॑सा॒म् छन्द॑साम् ॅवी॒र्य॑ मा॒दाय॑ । \newline
46. वी॒र्य॑ मा॒दाया॒ दाय॑ वी॒र्य॑म् ॅवी॒र्य॑ मा॒दाय॒ तत् तदा॒दाय॑ वी॒र्य॑म् ॅवी॒र्य॑ मा॒दाय॒ तत् । \newline
47. आ॒दाय॒ तत् तदा॒दाया॒ दाय॒ तद् वो॑ व॒ स्तदा॒दाया॒ दाय॒ तद् वः॑ । \newline
48. आ॒दायेत्या᳚ - दाय॑ । \newline
49. तद् वो॑ व॒ स्तत् तद् वः॒ प्र प्र व॒ स्तत् तद् वः॒ प्र । \newline
50. वः॒ प्र प्र वो॑ वः॒ प्र दा᳚स्यामि दास्यामि॒ प्र वो॑ वः॒ प्र दा᳚स्यामि । \newline
51. प्र दा᳚स्यामि दास्यामि॒ प्र प्र दा᳚स्या॒ मीतीति॑ दास्यामि॒ प्र प्र दा᳚स्या॒ मीति॑ । \newline
52. दा॒स्या॒ मीतीति॑ दास्यामि दास्या॒ मीति॒ स स इति॑ दास्यामि दास्या॒ मीति॒ सः । \newline
53. इति॒ स स इतीति॒ स छन्द॑सा॒म् छन्द॑साꣳ॒॒ स इतीति॒ स छन्द॑साम् । \newline
54. स छन्द॑सा॒म् छन्द॑साꣳ॒॒ स स छन्द॑साम् ॅवी॒र्य॑म् ॅवी॒र्य॑म् छन्द॑साꣳ॒॒ स स छन्द॑साम् ॅवी॒र्य᳚म् । \newline
55. छन्द॑साम् ॅवी॒र्य॑म् ॅवी॒र्य॑म् छन्द॑सा॒म् छन्द॑साम् ॅवी॒र्य॑ मा॒दाया॒ दाय॑ वी॒र्य॑म् छन्द॑सा॒म् छन्द॑साम् ॅवी॒र्य॑ मा॒दाय॑ । \newline
56. वी॒र्य॑ मा॒दाया॒ दाय॑ वी॒र्य॑म् ॅवी॒र्य॑ मा॒दाय॒ तत् तदा॒दाय॑ वी॒र्य॑म् ॅवी॒र्य॑ मा॒दाय॒ तत् । \newline
\pagebreak
\markright{ TS 3.3.7.2  \hfill https://www.vedavms.in \hfill}

\section{ TS 3.3.7.2 }

\textbf{TS 3.3.7.2 } \newline
\textbf{Samhita Paata} \newline

मा॒दाय॒ तदे᳚भ्यः॒ प्राय॑च्छ॒त् तदनु॒ छन्दाꣳ॒॒स्यपा᳚ऽक्राम॒न् छन्दाꣳ॑सि य॒ज्ञ्स्ततो॑ दे॒वा अभ॑व॒न् पराऽसु॑रा॒ य ए॒वं छन्द॑सां ॅवी॒र्यं॑ ॅवेदाऽऽ श्रा॑व॒याऽस्तु॒ श्रौष॒ड् यज॒ ये यजा॑महे वषट्का॒रो भव॑त्या॒त्मना॒ परा᳚ऽस्य॒ भ्रातृ॑व्यो भवति ब्रह्मवा॒दिनो॑ वदन्ति॒ कस्मै॒ कम॑द्ध्व॒र्युरा श्रा॑वय॒तीति॒ छन्द॑सां ॅवी॒र्या॑येति॑ ब्रूयादे॒ तद् वै - [  ] \newline

\textbf{Pada Paata} \newline

आ॒दायेत्या᳚-दाय॑ । तत् । ए॒भ्यः॒ । प्रेति॑ । अ॒य॒च्छ॒त् । तत् । अन्विति॑ । छन्दाꣳ॑सि । अपेति॑ । अ॒क्रा॒म॒न्न् । छन्दाꣳ॑सि । य॒ज्ञ्ः । ततः॑ । दे॒वाः । अभ॑वन्न् । परेति॑ । असु॑राः । यः । ए॒वम् । छन्द॑साम् । वी॒र्य᳚म् । वेद॑ । एति॑ । श्रा॒व॒य॒ । अस्तु॑ । श्रौष॑ट् । यज॑ । ये । यजा॑महे । व॒ष॒ट्का॒र इति॑ वषट् - का॒रः । भव॑ति । आ॒त्मना᳚ । परेति॑ । अ॒स्य॒ । भ्रातृ॑व्यः । भ॒व॒ति॒ । ब्र॒ह्म॒वा॒दिन॒ इति॑ ब्रह्म - वा॒दिनः॑ । व॒द॒न्ति॒ । कस्मै᳚ । कम् । अ॒द्ध्व॒र्युः । एति॑ । श्रा॒व॒य॒ति॒ । इति॑ । छन्द॑साम् । वी॒र्या॑य । इति॑ । ब्रू॒या॒त् । ए॒तत् । वै ।  \newline


\textbf{Krama Paata} \newline

आ॒दाय॒ तत् । आ॒दायेत्या᳚ - दाय॑ । तदे᳚भ्यः । ए॒भ्यः॒ प्र । प्राय॑च्छत् । अ॒य॒च्छ॒त् तत् । तदनु॑ । अनु॒ छन्दाꣳ॑सि । छन्दाꣳ॒॒स्यप॑ । अपा᳚क्रामन्न् । अ॒क्रा॒म॒न् छन्दाꣳ॑सि । छन्दाꣳ॑सि य॒ज्ञ्ः । य॒ज्ञ्स्ततः॑ । ततो॑ दे॒वाः । दे॒वा अभ॑वन्न् । अभ॑व॒न् परा᳚ । परा ऽसु॑राः । असु॑रा॒ यः । य ए॒वम् । ए॒वम् छन्द॑साम् । छन्द॑सां ॅवी॒र्य᳚म् । वी॒र्यं॑ ॅवेद॑ । वेदा । आ श्रा॑वय । श्रा॒व॒यास्तु॑ । अस्तु॒ श्रौष॑ट् । श्रौष॒ड् यज॑ । यज॒ ये । ये यजा॑महे । यजा॑महे वषट्का॒रः । व॒ष॒ट्का॒रो भव॑ति । व॒ष॒ट्का॒र इति॑ वषट् - का॒रः । भव॑त्या॒त्मना᳚ । आ॒त्मना॒ परा᳚ । परा᳚ ऽस्य । अ॒स्य॒ भ्रातृ॑व्यः । भ्रातृ॑व्यो भवति । भ॒व॒ति॒ ब्र॒ह्म॒वा॒दिनः॑ । ब्र॒ह्म॒वा॒दिनो॑ वदन्ति । ब्र॒ह्म॒वा॒दिन॒ इति॑ ब्रह्म - वा॒दिनः॑ । व॒द॒न्ति॒ कस्मै᳚ । कस्मै॒ कम् । कम॑द्ध्व॒र्युः । अ॒द्ध्व॒र्युरा । आ श्रा॑वयति । श्रा॒व॒य॒तीति॑ । इति॒ छन्द॑साम् । छन्द॑सां ॅवी॒र्या॑य । वी॒र्या॑येति॑ । इति॑ ब्रूयात् । ब्रू॒या॒दे॒तत् । ए॒तद् वै । वै छन्द॑साम् \newline

\textbf{Jatai Paata} \newline

1. आ॒दाय॒ तत् तदा॒दाया॒ दाय॒ तत् । \newline
2. आ॒दायेत्या᳚ - दाय॑ । \newline
3. तदे᳚भ्य एभ्य॒ स्तत् तदे᳚भ्यः । \newline
4. ए॒भ्यः॒ प्र प्रैभ्य॑ एभ्यः॒ प्र । \newline
5. प्राय॑च्छ दयच्छ॒त् प्र प्राय॑च्छत् । \newline
6. अ॒य॒च्छ॒त् तत् तद॑यच्छ दयच्छ॒त् तत् । \newline
7. तदन्वनु॒ तत् तदनु॑ । \newline
8. अनु॒ छन्दाꣳ॑सि॒ छन्दाꣳ॒॒ स्यन्वनु॒ छन्दाꣳ॑सि । \newline
9. छन्दाꣳ॒॒ स्यपाप॒च् छन्दाꣳ॑सि॒ छन्दाꣳ॒॒ स्यप॑ । \newline
10. अपा᳚क्रामन् नक्राम॒न् नपापा᳚ क्रामन्न् । \newline
11. अ॒क्रा॒म॒न् छन्दाꣳ॑सि॒ छन्दाꣳ॑ स्यक्रामन् नक्राम॒न् छन्दाꣳ॑सि । \newline
12. छन्दाꣳ॑सि य॒ज्ञो य॒ज्ञ् श्छन्दाꣳ॑सि॒ छन्दाꣳ॑सि य॒ज्ञ्ः । \newline
13. य॒ज्ञ् स्तत॒ स्ततो॑ य॒ज्ञो य॒ज्ञ् स्ततः॑ । \newline
14. ततो॑ दे॒वा दे॒वा स्तत॒ स्ततो॑ दे॒वाः । \newline
15. दे॒वा अभ॑व॒न् नभ॑वन् दे॒वा दे॒वा अभ॑वन्न् । \newline
16. अभ॑व॒न् परा॒ परा ऽभ॑व॒न् नभ॑व॒न् परा᳚ । \newline
17. परा ऽसु॑रा॒ असु॑राः॒ परा॒ परा ऽसु॑राः । \newline
18. असु॑रा॒ यो यो ऽसु॑रा॒ असु॑रा॒ यः । \newline
19. य ए॒व मे॒वम् ॅयो य ए॒वम् । \newline
20. ए॒वम् छन्द॑सा॒म् छन्द॑सा मे॒व मे॒वम् छन्द॑साम् । \newline
21. छन्द॑साम् ॅवी॒र्य॑म् ॅवी॒र्य॑म् छन्द॑सा॒म् छन्द॑साम् ॅवी॒र्य᳚म् । \newline
22. वी॒र्य॑म् ॅवेद॒ वेद॑ वी॒र्य॑म् ॅवी॒र्य॑म् ॅवेद॑ । \newline
23. वेदा वेद॒ वेदा । \newline
24. आ श्रा॑वय श्राव॒या श्रा॑वय । \newline
25. श्रा॒व॒या स्त्वस्तु॑ श्रावय श्राव॒यास्तु॑ । \newline
26. अस्तु॒ श्रौष॒ट् छ्रौष॒ डस्त्वस्तु॒ श्रौष॑ट् । \newline
27. श्रौष॒ड् यज॒ यज॒ श्रौष॒ट् छ्रौष॒ड् यज॑ । \newline
28. यज॒ ये ये यज॒ यज॒ ये । \newline
29. ये यजा॑महे॒ यजा॑महे॒ ये ये यजा॑महे । \newline
30. यजा॑महे वषट्का॒रो व॑षट्का॒रो यजा॑महे॒ यजा॑महे वषट्का॒रः । \newline
31. व॒ष॒ट्का॒रो भव॑ति॒ भव॑ति वषट्का॒रो व॑षट्का॒रो भव॑ति । \newline
32. व॒ष॒ट्का॒र इति॑ वषट् - का॒रः । \newline
33. भव॑ त्या॒त्मना॒ ऽऽत्मना॒ भव॑ति॒ भव॑ त्या॒त्मना᳚ । \newline
34. आ॒त्मना॒ परा॒ परा॒ ऽऽत्मना॒ ऽऽत्मना॒ परा᳚ । \newline
35. परा᳚ ऽस्यास्य॒ परा॒ परा᳚ ऽस्य । \newline
36. अ॒स्य॒ भ्रातृ॑व्यो॒ भ्रातृ॑व्यो ऽस्यास्य॒ भ्रातृ॑व्यः । \newline
37. भ्रातृ॑व्यो भवति भवति॒ भ्रातृ॑व्यो॒ भ्रातृ॑व्यो भवति । \newline
38. भ॒व॒ति॒ ब्र॒ह्म॒वा॒दिनो᳚ ब्रह्मवा॒दिनो॑ भवति भवति ब्रह्मवा॒दिनः॑ । \newline
39. ब्र॒ह्म॒वा॒दिनो॑ वदन्ति वदन्ति ब्रह्मवा॒दिनो᳚ ब्रह्मवा॒दिनो॑ वदन्ति । \newline
40. ब्र॒ह्म॒वा॒दिन॒ इति॑ ब्रह्म - वा॒दिनः॑ । \newline
41. व॒द॒न्ति॒ कस्मै॒ कस्मै॑ वदन्ति वदन्ति॒ कस्मै᳚ । \newline
42. कस्मै॒ कम् कम् कस्मै॒ कस्मै॒ कम् । \newline
43. क म॑द्ध्व॒र्यु र॑द्ध्व॒र्युः कम् क म॑द्ध्व॒र्युः । \newline
44. अ॒द्ध्व॒र्युरा ऽद्ध्व॒र्यु र॑द्ध्व॒र्युरा । \newline
45. आ श्रा॑वयति श्रावय॒त्या श्रा॑वयति । \newline
46. श्रा॒व॒य॒ तीतीति॑ श्रावयति श्रावय॒तीति॑ । \newline
47. इति॒ छन्द॑सा॒म् छन्द॑सा॒ मितीति॒ छन्द॑साम् । \newline
48. छन्द॑साम् ॅवी॒र्या॑य वी॒र्या॑य॒ छन्द॑सा॒म् छन्द॑साम् ॅवी॒र्या॑य । \newline
49. वी॒र्या॑ये तीति॑ वी॒र्या॑य वी॒र्या॑ये ति॑ । \newline
50. इति॑ ब्रूयाद् ब्रूया॒ दितीति॑ ब्रूयात् । \newline
51. ब्रू॒या॒ दे॒त दे॒तद् ब्रू॑याद् ब्रूया दे॒तत् । \newline
52. ए॒तद् वै वा ए॒त दे॒तद् वै । \newline
53. वै छन्द॑सा॒म् छन्द॑सा॒म् ॅवै वै छन्द॑साम् । \newline

\textbf{Ghana Paata } \newline

1. आ॒दाय॒ तत् तदा॒दाया॒ दाय॒ तदे᳚भ्य एभ्य॒ स्तदा॒दाया॒ दाय॒ तदे᳚भ्यः । \newline
2. आ॒दायेत्या᳚ - दाय॑ । \newline
3. तदे᳚भ्य एभ्य॒ स्तत् तदे᳚भ्यः॒ प्र प्रैभ्य॒ स्तत् तदे᳚भ्यः॒ प्र । \newline
4. ए॒भ्यः॒ प्र प्रैभ्य॑ एभ्यः॒ प्राय॑च्छ दयच्छ॒त् प्रैभ्य॑ एभ्यः॒ प्राय॑च्छत् । \newline
5. प्राय॑च्छ दयच्छ॒त् प्र प्राय॑च्छ॒त् तत् तद॑यच्छ॒त् प्र प्राय॑च्छ॒त् तत् । \newline
6. अ॒य॒च्छ॒त् तत् तद॑यच्छ दयच्छ॒त् तदन्वनु॒ तद॑यच्छ दयच्छ॒त् तदनु॑ । \newline
7. तदन्वनु॒ तत् तदनु॒ छन्दाꣳ॑सि॒ छन्दाꣳ॒॒ स्यनु॒ तत् तदनु॒ छन्दाꣳ॑सि । \newline
8. अनु॒ छन्दाꣳ॑सि॒ छन्दाꣳ॒॒ स्यन्वनु॒ छन्दाꣳ॒॒ स्यपाप॒ च्छन्दाꣳ॒॒ स्यन्वनु॒ छन्दाꣳ॒॒ स्यप॑ । \newline
9. छन्दाꣳ॒॒ स्यपाप॒ च्छन्दाꣳ॑सि॒ छन्दाꣳ॒॒ स्यपा᳚क्रामन्, नक्राम॒न्, नप॒ च्छन्दाꣳ॑सि॒ छन्दाꣳ॒॒ स्यपा᳚क्रामन्न् । \newline
10. अपा᳚क्रामन्, नक्राम॒न्, नपापा᳚ क्राम॒न् छन्दाꣳ॑सि॒ छन्दाꣳ॑ स्यक्राम॒न्, नपापा᳚क्राम॒न् छन्दाꣳ॑सि । \newline
11. अ॒क्रा॒म॒न् छन्दाꣳ॑सि॒ छन्दाꣳ॑ स्यक्रामन्, नक्राम॒न् छन्दाꣳ॑सि य॒ज्ञो य॒ज्ञ् श्छन्दाꣳ॑ स्यक्रामन्, नक्राम॒न् छन्दाꣳ॑सि य॒ज्ञ्ः । \newline
12. छन्दाꣳ॑सि य॒ज्ञो य॒ज्ञ् श्छन्दाꣳ॑सि॒ छन्दाꣳ॑सि य॒ज्ञ् स्तत॒ स्ततो॑ य॒ज्ञ् श्छन्दाꣳ॑सि॒ छन्दाꣳ॑सि य॒ज्ञ् स्ततः॑ । \newline
13. य॒ज्ञ् स्तत॒ स्ततो॑ य॒ज्ञो य॒ज्ञ् स्ततो॑ दे॒वा दे॒वा स्ततो॑ य॒ज्ञो य॒ज्ञ् स्ततो॑ दे॒वाः । \newline
14. ततो॑ दे॒वा दे॒वा स्तत॒ स्ततो॑ दे॒वा अभ॑व॒न्, नभ॑वन् दे॒वा स्तत॒ स्ततो॑ दे॒वा अभ॑वन्न् । \newline
15. दे॒वा अभ॑व॒न्, नभ॑वन् दे॒वा दे॒वा अभ॑व॒न् परा॒ परा ऽभ॑वन् दे॒वा दे॒वा अभ॑व॒न् परा᳚ । \newline
16. अभ॑व॒न् परा॒ परा ऽभ॑व॒न्, नभ॑व॒न् परा ऽसु॑रा॒ असु॑राः॒ परा ऽभ॑व॒न्, नभ॑व॒न् परा ऽसु॑राः । \newline
17. परा ऽसु॑रा॒ असु॑राः॒ परा॒ परा ऽसु॑रा॒ यो यो ऽसु॑राः॒ परा॒ परा ऽसु॑रा॒ यः । \newline
18. असु॑रा॒ यो यो ऽसु॑रा॒ असु॑रा॒ य ए॒व मे॒वम् ॅयो ऽसु॑रा॒ असु॑रा॒ य ए॒वम् । \newline
19. य ए॒व मे॒वम् ॅयो य ए॒वम् छन्द॑सा॒म् छन्द॑सा मे॒वम् ॅयो य ए॒वम् छन्द॑साम् । \newline
20. ए॒वम् छन्द॑सा॒म् छन्द॑सा मे॒व मे॒वम् छन्द॑साम् ॅवी॒र्य॑म् ॅवी॒र्य॑म् छन्द॑सा मे॒व मे॒वम् छन्द॑साम् ॅवी॒र्य᳚म् । \newline
21. छन्द॑साम् ॅवी॒र्य॑म् ॅवी॒र्य॑म् छन्द॑सा॒म् छन्द॑साम् ॅवी॒र्य॑म् ॅवेद॒ वेद॑ वी॒र्य॑म् छन्द॑सा॒म् छन्द॑साम् ॅवी॒र्य॑म् ॅवेद॑ । \newline
22. वी॒र्य॑म् ॅवेद॒ वेद॑ वी॒र्य॑म् ॅवी॒र्य॑म् ॅवेदा वेद॑ वी॒र्य॑म् ॅवी॒र्य॑म् ॅवेदा । \newline
23. वेदा वेद॒ वेदा श्रा॑वय श्राव॒या वेद॒ वेदा श्रा॑वय । \newline
24. आ श्रा॑वय श्राव॒या श्रा॑व॒या स्त्वस्तु॑ श्राव॒या श्रा॑व॒यास्तु॑ । \newline
25. श्रा॒व॒या स्त्वस्तु॑ श्रावय श्राव॒यास्तु॒ श्रौष॒ट् छ्रौष॒ डस्तु॑ श्रावय श्राव॒यास्तु॒ श्रौष॑ट् । \newline
26. अस्तु॒ श्रौष॒ट् छ्रौष॒ डस्त्वस्तु॒ श्रौष॒ड् यज॒ यज॒ श्रौष॒ डस्त्वस्तु॒ श्रौष॒ड् यज॑ । \newline
27. श्रौष॒ड् यज॒ यज॒ श्रौष॒ट् छ्रौष॒ड् यज॒ ये ये यज॒ श्रौष॒ट् छ्रौष॒ड् यज॒ ये । \newline
28. यज॒ ये ये यज॒ यज॒ ये यजा॑महे॒ यजा॑महे॒ ये यज॒ यज॒ ये यजा॑महे । \newline
29. ये यजा॑महे॒ यजा॑महे॒ ये ये यजा॑महे वषट्का॒रो व॑षट्का॒रो यजा॑महे॒ ये ये यजा॑महे वषट्का॒रः । \newline
30. यजा॑महे वषट्का॒रो व॑षट्का॒रो यजा॑महे॒ यजा॑महे वषट्का॒रो भव॑ति॒ भव॑ति वषट्का॒रो यजा॑महे॒ यजा॑महे वषट्का॒रो भव॑ति । \newline
31. व॒ष॒ट्का॒रो भव॑ति॒ भव॑ति वषट्का॒रो व॑षट्का॒रो भव॑ त्या॒त्मना॒ ऽऽत्मना॒ भव॑ति वषट्का॒रो व॑षट्का॒रो भव॑ त्या॒त्मना᳚ । \newline
32. व॒ष॒ट्का॒र इति॑ वषट् - का॒रः । \newline
33. भव॑ त्या॒त्मना॒ ऽऽत्मना॒ भव॑ति॒ भव॑ त्या॒त्मना॒ परा॒ परा॒ ऽऽत्मना॒ भव॑ति॒ भव॑ त्या॒त्मना॒ परा᳚ । \newline
34. आ॒त्मना॒ परा॒ परा॒ ऽऽत्मना॒ ऽऽत्मना॒ परा᳚ ऽस्यास्य॒ परा॒ ऽऽत्मना॒ ऽऽत्मना॒ परा᳚ ऽस्य । \newline
35. परा᳚ ऽस्यास्य॒ परा॒ परा᳚ ऽस्य॒ भ्रातृ॑व्यो॒ भ्रातृ॑व्यो ऽस्य॒ परा॒ परा᳚ ऽस्य॒ भ्रातृ॑व्यः । \newline
36. अ॒स्य॒ भ्रातृ॑व्यो॒ भ्रातृ॑व्यो ऽस्यास्य॒ भ्रातृ॑व्यो भवति भवति॒ भ्रातृ॑व्यो ऽस्यास्य॒ भ्रातृ॑व्यो भवति । \newline
37. भ्रातृ॑व्यो भवति भवति॒ भ्रातृ॑व्यो॒ भ्रातृ॑व्यो भवति ब्रह्मवा॒दिनो᳚ ब्रह्मवा॒दिनो॑ भवति॒ भ्रातृ॑व्यो॒ भ्रातृ॑व्यो भवति ब्रह्मवा॒दिनः॑ । \newline
38. भ॒व॒ति॒ ब्र॒ह्म॒वा॒दिनो᳚ ब्रह्मवा॒दिनो॑ भवति भवति ब्रह्मवा॒दिनो॑ वदन्ति वदन्ति ब्रह्मवा॒दिनो॑ भवति भवति ब्रह्मवा॒दिनो॑ वदन्ति । \newline
39. ब्र॒ह्म॒वा॒दिनो॑ वदन्ति वदन्ति ब्रह्मवा॒दिनो᳚ ब्रह्मवा॒दिनो॑ वदन्ति॒ कस्मै॒ कस्मै॑ वदन्ति ब्रह्मवा॒दिनो᳚ ब्रह्मवा॒दिनो॑ वदन्ति॒ कस्मै᳚ । \newline
40. ब्र॒ह्म॒वा॒दिन॒ इति॑ ब्रह्म - वा॒दिनः॑ । \newline
41. व॒द॒न्ति॒ कस्मै॒ कस्मै॑ वदन्ति वदन्ति॒ कस्मै॒ कम् कम् कस्मै॑ वदन्ति वदन्ति॒ कस्मै॒ कम् । \newline
42. कस्मै॒ कम् कम् कस्मै॒ कस्मै॒ क म॑द्ध्व॒र्यु र॑द्ध्व॒र्युः कम् कस्मै॒ कस्मै॒ क म॑द्ध्व॒र्युः । \newline
43. क म॑द्ध्व॒र्यु र॑द्ध्व॒र्युः कम् क म॑द्ध्व॒र्युरा ऽद्ध्व॒र्युः कम् क म॑द्ध्व॒र्युरा । \newline
44. अ॒द्ध्व॒र्युरा ऽद्ध्व॒र्यु र॑द्ध्व॒र्युरा श्रा॑वयति श्रावय॒त्या ऽद्ध्व॒र्यु र॑द्ध्व॒र्युरा श्रा॑वयति । \newline
45. आ श्रा॑वयति श्रावय॒त्या श्रा॑वय॒तीतीति॑ श्रावय॒त्या श्रा॑वय॒तीति॑ । \newline
46. श्रा॒व॒य॒तीतीति॑ श्रावयति श्रावय॒तीति॒ छन्द॑सा॒म् छन्द॑सा॒ मिति॑ श्रावयति श्रावय॒तीति॒ छन्द॑साम् । \newline
47. इति॒ छन्द॑सा॒म् छन्द॑सा॒ मितीति॒ छन्द॑साम् ॅवी॒र्या॑य वी॒र्या॑य॒ छन्द॑सा॒ मितीति॒ छन्द॑साम् ॅवी॒र्या॑य । \newline
48. छन्द॑साम् ॅवी॒र्या॑य वी॒र्या॑य॒ छन्द॑सा॒म् छन्द॑साम् ॅवी॒र्या॑ येतीति॑ वी॒र्या॑य॒ छन्द॑सा॒म् छन्द॑साम् ॅवी॒र्या॑येति॑ । \newline
49. वी॒र्या॑ येतीति॑ वी॒र्या॑य वी॒र्या॑येति॑ ब्रूयाद् ब्रूया॒ दिति॑ वी॒र्या॑य वी॒र्या॑येति॑ ब्रूयात् । \newline
50. इति॑ ब्रूयाद् ब्रूया॒ दितीति॑ ब्रूया दे॒त दे॒तद् ब्रू॑या॒ दितीति॑ ब्रूया दे॒तत् । \newline
51. ब्रू॒या॒ दे॒त दे॒तद् ब्रू॑याद् ब्रूया दे॒तद् वै वा ए॒तद् ब्रू॑याद् ब्रूया दे॒तद् वै । \newline
52. ए॒तद् वै वा ए॒त दे॒तद् वै छन्द॑सा॒म् छन्द॑सा॒म् ॅवा ए॒त दे॒तद् वै छन्द॑साम् । \newline
53. वै छन्द॑सा॒म् छन्द॑सा॒म् ॅवै वै छन्द॑साम् ॅवी॒र्य॑म् ॅवी॒र्य॑म् छन्द॑सा॒म् ॅवै वै छन्द॑साम् ॅवी॒र्य᳚म् । \newline
\pagebreak
\markright{ TS 3.3.7.3  \hfill https://www.vedavms.in \hfill}

\section{ TS 3.3.7.3 }

\textbf{TS 3.3.7.3 } \newline
\textbf{Samhita Paata} \newline

छन्द॑सां ॅवी॒र्य॑मा श्रा॑व॒याऽस्तु॒ श्रौष॒ड् यज॒ ये यजा॑महे वषट्का॒रो य ए॒वं ॅवेद॒ सवी᳚र्यैरे॒व छन्दो॑भिरर्चति॒ यत् किं चार्च॑ति॒ यदिन्द्रो॑ वृ॒त्रमह॑न्न-मे॒द्ध्यं तद्-यद्-यती॑न॒पाव॑पद-मे॒द्ध्यं तदथ॒ कस्मा॑दै॒न्द्रो य॒ज्ञ् आ सꣳस्था॑तो॒रित्या॑हु॒रिन्द्र॑स्य॒ वा ए॒षा य॒ज्ञिया॑ त॒नूर्यद्-य॒ज्ञ्स्तामे॒व त ( ) द्य॑जन्ति॒ य ए॒वं ॅवेदोपै॑नं ॅय॒ज्ञो न॑मति ॥ \newline

\textbf{Pada Paata} \newline

छन्द॑साम् । वी॒र्य᳚म् । एति॑ । श्रा॒व॒य॒ । अस्तु॑ । श्रौष॑ट् । यज॑ । ये । यजा॑महे । व॒ष॒ट्का॒र इति॑ वषट् - का॒रः । यः । ए॒वम् । वेद॑ । सवी᳚र्यै॒रिति॒ स - वी॒र्यैः॒ । ए॒व । छन्दो॑भि॒रिति॒ छन्दः॑ - भिः॒ । अ॒र्च॒ति॒ । यत् । किम् । च॒ । अर्च॑ति । यत् । इन्द्रः॑ । वृ॒त्रम् । अहन्न्॑ । अ॒मे॒द्ध्यम् । तत् । यत् । यतीन्॑ । अ॒पाव॑प॒दित्य॑प - अव॑पत् । अ॒मे॒द्ध्यम् । तत् । अथ॑ । कस्मा᳚त् । ऐ॒न्द्रः । य॒ज्ञ्ः । एति॑ । सꣳस्था॑तो॒रिति॒ सं - स्था॒तोः॒ । इति॑ । आ॒हुः॒ । इन्द्र॑स्य । वै । ए॒षा । य॒ज्ञिया᳚ । त॒नूः । यत् । य॒ज्ञ्ः । ताम् । ए॒व । तत् ( ) । य॒ज॒न्ति॒ । यः । ए॒वम् । वेद॑ । उपेति॑ । ए॒न॒म् । य॒ज्ञ्ः । न॒म॒ति॒ ॥  \newline


\textbf{Krama Paata} \newline

छन्द॑सां ॅवी॒र्य᳚म् । वी॒र्य॑मा । आ श्रा॑वय । श्रा॒व॒यास्तु॑ । अस्तु॒ श्रौष॑ट् । श्रौष॒ड् यज॑ । यज॒ ये । ये यजा॑महे । यजा॑महे वषट्का॒रः । व॒ष॒ट्का॒रो यः । व॒ष॒ट्का॒र इति॑ वषट् - का॒रः । य ए॒वम् । ए॒वं ॅवेद॑ । वेद॒ सवी᳚र्यैः । सवी᳚र्यैरे॒व । सवी᳚र्यै॒रिति॒ स - वी॒र्यैः॒ । ए॒व छन्दो॑भिः । छन्दो॑भिरर्चति । छन्दो॑भि॒रिति॒ छन्दः॑ - भिः॒ । अ॒र्च॒ति॒ यत् । यत् किम् । किम् च॑ । चार्च॑ति । अर्च॑ति॒ यत् । यदिन्द्रः॑ । इन्द्रो॑ वृ॒त्रम् । वृ॒त्रमहन्न्॑ । अह॑न्नमे॒द्ध्यम् । अ॒मे॒द्ध्यम् तत् । तद् यत् । यद् यतीन्॑ । यती॑न॒पाव॑पत् । अ॒पाव॑पदमे॒द्ध्यम् । अ॒पाव॑प॒दित्य॑प - अव॑पत् । अ॒मे॒द्ध्यम् तत् । तदथ॑ । अथ॒ कस्मा᳚त् । कस्मा॑दै॒न्द्रः । ऐ॒न्द्रो य॒ज्ञ्ः । य॒ज्ञ् आ । आ सꣳस्था॑तोः । सꣳस्था॑तो॒रिति॑ । सꣳस्था॑तो॒रिति॒ सम् - स्था॒तोः॒ । इत्या॑हुः । आ॒हु॒रिन्द्र॑स्य । इन्द्र॑स्य॒ वै । वा ए॒षा । ए॒षा य॒ज्ञिया᳚ । य॒ज्ञिया॑ त॒नूः । त॒नूर् यत् । यद् य॒ज्ञ्ः । य॒ज्ञ्स्ताम् । तामे॒व । ए॒व तत् ( ) । तद् य॑जन्ति । य॒ज॒न्ति॒ यः । य ए॒वम् । ए॒वं ॅवेद॑ । वेदोप॑ । उपै॑नम् । ए॒नं॒ ॅय॒ज्ञ्ः । य॒ज्ञो न॑मति । न॒म॒तीति॑ नमति । \newline

\textbf{Jatai Paata} \newline

1. छन्द॑साम् ॅवी॒र्य॑म् ॅवी॒र्य॑म् छन्द॑सा॒म् छन्द॑साम् ॅवी॒र्य᳚म् । \newline
2. वी॒र्य॑ मा वी॒र्य॑म् ॅवी॒र्य॑ मा । \newline
3. आ श्रा॑वय श्राव॒या श्रा॑वय । \newline
4. श्रा॒व॒या स्त्वस्तु॑ श्रावय श्राव॒यास्तु॑ । \newline
5. अस्तु॒ श्रौष॒ट् छ्रौष॒ड स्त्वस्तु॒ श्रौष॑ट् । \newline
6. श्रौष॒ड् यज॒ यज॒ श्रौष॒ट् छ्रौष॒ड् यज॑ । \newline
7. यज॒ ये ये यज॒ यज॒ ये । \newline
8. ये यजा॑महे॒ यजा॑महे॒ ये ये यजा॑महे । \newline
9. यजा॑महे वषट्का॒रो व॑षट्का॒रो यजा॑महे॒ यजा॑महे वषट्का॒रः । \newline
10. व॒ष॒ट्का॒रो यो यो व॑षट्का॒रो व॑षट्का॒रो यः । \newline
11. व॒ष॒ट्का॒र इति॑ वषट् - का॒रः । \newline
12. य ए॒व मे॒वम् ॅयो य ए॒वम् । \newline
13. ए॒वम् ॅवेद॒ वेदै॒व मे॒वम् ॅवेद॑ । \newline
14. वेद॒ सवी᳚र्यैः॒ सवी᳚र्यै॒र् वेद॒ वेद॒ सवी᳚र्यैः । \newline
15. सवी᳚र्यै रे॒वैव सवी᳚र्यैः॒ सवी᳚र्यै रे॒व । \newline
16. सवी᳚र्यै॒रिति॒ स - वी॒र्यैः॒ । \newline
17. ए॒व छन्दो॑भि॒ श्छन्दो॑भि रे॒वैव छन्दो॑भिः । \newline
18. छन्दो॑भि रर्च त्यर्चति॒ छन्दो॑भि॒ श्छन्दो॑भि रर्चति । \newline
19. छन्दो॑भि॒रिति॒ छन्दः॑ - भिः॒ । \newline
20. अ॒र्च॒ति॒ यद् यद॑र्च त्यर्चति॒ यत् । \newline
21. यत् किम् किम् ॅयद् यत् किम् । \newline
22. किम् च॑ च॒ किम् किम् च॑ । \newline
23. चार्च॒ त्यर्च॑ति च॒ चार्च॑ति । \newline
24. अर्च॑ति॒ यद् यदर्च॒ त्यर्च॑ति॒ यत् । \newline
25. यदिन्द्र॒ इन्द्रो॒ यद् यदिन्द्रः॑ । \newline
26. इन्द्रो॑ वृ॒त्रम् ॅवृ॒त्र मिन्द्र॒ इन्द्रो॑ वृ॒त्रम् । \newline
27. वृ॒त्र मह॒न् नह॑न् वृ॒त्रम् ॅवृ॒त्र महन्न्॑ । \newline
28. अह॑न् नमे॒द्ध्य म॑मे॒द्ध्य मह॒न् नह॑न् नमे॒द्ध्यम् । \newline
29. अ॒मे॒द्ध्यम् तत् तद॑मे॒द्ध्य म॑मे॒द्ध्यम् तत् । \newline
30. तद् यद् यत् तत् तद् यत् । \newline
31. यद् यती॒न्॒. यती॒न्॒. यद् यद् यतीन्॑ । \newline
32. यती॑ न॒पाव॑प द॒पाव॑प॒द् यती॒न्॒. यती॑ न॒पाव॑पत् । \newline
33. अ॒पाव॑प दमे॒द्ध्य म॑मे॒द्ध्य म॒पाव॑प द॒पाव॑प दमे॒द्ध्यम् । \newline
34. अ॒पाव॑प॒दित्य॑प - अव॑पत् । \newline
35. अ॒मे॒द्ध्यम् तत् तद॑मे॒द्ध्य म॑मे॒द्ध्यम् तत् । \newline
36. तदथाथ॒ तत् तदथ॑ । \newline
37. अथ॒ कस्मा॒त् कस्मा॒ दथाथ॒ कस्मा᳚त् । \newline
38. कस्मा॑ दै॒न्द्र ऐ॒न्द्रः कस्मा॒त् कस्मा॑ दै॒न्द्रः । \newline
39. ऐ॒न्द्रो य॒ज्ञो य॒ज्ञ् ऐ॒न्द्र ऐ॒न्द्रो य॒ज्ञ्ः । \newline
40. य॒ज्ञ् आ य॒ज्ञो य॒ज्ञ् आ । \newline
41. आ सꣳस्था॑तोः॒ सꣳस्था॑तो॒ रा सꣳस्था॑तोः । \newline
42. सꣳस्था॑तो॒ रितीति॒ सꣳस्था॑तोः॒ सꣳस्था॑तो॒ रिति॑ । \newline
43. सꣳस्था॑तो॒रिति॒ सम् - स्था॒तोः॒ । \newline
44. इत्या॑हु राहु॒ रिती त्या॑हुः । \newline
45. आ॒हु॒ रिन्द्र॒ स्येन्द्र॑ स्याहु राहु॒ रिन्द्र॑स्य । \newline
46. इन्द्र॑स्य॒ वै वा इन्द्र॒ स्येन्द्र॑स्य॒ वै । \newline
47. वा ए॒षैषा वै वा ए॒षा । \newline
48. ए॒षा य॒ज्ञिया॑ य॒ज्ञि यै॒षैषा य॒ज्ञिया᳚ । \newline
49. य॒ज्ञिया॑ त॒नू स्त॒नूर् य॒ज्ञिया॑ य॒ज्ञिया॑ त॒नूः । \newline
50. त॒नूर् यद् यत् त॒नू स्त॒नूर् यत् । \newline
51. यद् य॒ज्ञो य॒ज्ञो यद् यद् य॒ज्ञ्ः । \newline
52. य॒ज्ञ् स्ताम् ताम् ॅय॒ज्ञो य॒ज्ञ् स्ताम् । \newline
53. ता मे॒वैव ताम् ता मे॒व । \newline
54. ए॒व तत् तदे॒वैव तत् । \newline
55. तद् य॑जन्ति यजन्ति॒ तत् तद् य॑जन्ति । \newline
56. य॒ज॒न्ति॒ यो यो य॑जन्ति यजन्ति॒ यः । \newline
57. य ए॒व मे॒वम् ॅयो य ए॒वम् । \newline
58. ए॒वम् ॅवेद॒ वेदै॒व मे॒वम् ॅवेद॑ । \newline
59. वेदोपोप॒ वेद॒ वेदोप॑ । \newline
60. उपै॑न मेन॒ मुपोपै॑नम् । \newline
61. ए॒न॒म् ॅय॒ज्ञो य॒ज्ञ् ए॑न मेनम् ॅय॒ज्ञ्ः । \newline
62. य॒ज्ञो न॑मति नमति य॒ज्ञो य॒ज्ञो न॑मति । \newline
63. न॒म॒तीति॑ नमति । \newline

\textbf{Ghana Paata } \newline

1. छन्द॑साम् ॅवी॒र्य॑म् ॅवी॒र्य॑म् छन्द॑सा॒म् छन्द॑साम् ॅवी॒र्य॑ मा वी॒र्य॑म् छन्द॑सा॒म् छन्द॑साम् ॅवी॒र्य॑ मा । \newline
2. वी॒र्य॑ मा वी॒र्य॑म् ॅवी॒र्य॑ मा श्रा॑वय श्राव॒या वी॒र्य॑म् ॅवी॒र्य॑ मा श्रा॑वय । \newline
3. आ श्रा॑वय श्राव॒या श्रा॑व॒या स्त्वस्तु॑ श्राव॒या श्रा॑व॒यास्तु॑ । \newline
4. श्रा॒व॒या स्त्वस्तु॑ श्रावय श्राव॒यास्तु॒ श्रौष॒ट् छ्रौष॒ डस्तु॑ श्रावय श्राव॒यास्तु॒ श्रौष॑ट् । \newline
5. अस्तु॒ श्रौष॒ट् छ्रौष॒ड स्त्वस्तु॒ श्रौष॒ड् यज॒ यज॒ श्रौष॒ डस्त्वस्तु॒ श्रौष॒ड् यज॑ । \newline
6. श्रौष॒ड् यज॒ यज॒ श्रौष॒ट् छ्रौष॒ड् यज॒ ये ये यज॒ श्रौष॒ट् छ्रौष॒ड् यज॒ ये । \newline
7. यज॒ ये ये यज॒ यज॒ ये यजा॑महे॒ यजा॑महे॒ ये यज॒ यज॒ ये यजा॑महे । \newline
8. ये यजा॑महे॒ यजा॑महे॒ ये ये यजा॑महे वषट्का॒रो व॑षट्का॒रो यजा॑महे॒ ये ये यजा॑महे वषट्का॒रः । \newline
9. यजा॑महे वषट्का॒रो व॑षट्का॒रो यजा॑महे॒ यजा॑महे वषट्का॒रो यो यो व॑षट्का॒रो यजा॑महे॒ यजा॑महे वषट्का॒रो यः । \newline
10. व॒ष॒ट्का॒रो यो यो व॑षट्का॒रो व॑षट्का॒रो य ए॒व मे॒वम् ॅयो व॑षट्का॒रो व॑षट्का॒रो य ए॒वम् । \newline
11. व॒ष॒ट्का॒र इति॑ वषट् - का॒रः । \newline
12. य ए॒व मे॒वम् ॅयो य ए॒वम् ॅवेद॒ वेदै॒वम् ॅयो य ए॒वम् ॅवेद॑ । \newline
13. ए॒वम् ॅवेद॒ वेदै॒व मे॒वम् ॅवेद॒ सवी᳚र्यैः॒ सवी᳚र्यै॒र् वेदै॒व मे॒वम् ॅवेद॒ सवी᳚र्यैः । \newline
14. वेद॒ सवी᳚र्यैः॒ सवी᳚र्यै॒र् वेद॒ वेद॒ सवी᳚र्यै रे॒वैव सवी᳚र्यै॒र् वेद॒ वेद॒ सवी᳚र्यै रे॒व । \newline
15. सवी᳚र्यै रे॒वैव सवी᳚र्यैः॒ सवी᳚र्यै रे॒व छन्दो॑भि॒ श्छन्दो॑भि रे॒व सवी᳚र्यैः॒ सवी᳚र्यै रे॒व छन्दो॑भिः । \newline
16. सवी᳚र्यै॒रिति॒ स - वी॒र्यैः॒ । \newline
17. ए॒व छन्दो॑भि॒ श्छन्दो॑भि रे॒वैव छन्दो॑भि रर्च त्यर्चति॒ छन्दो॑भि रे॒वैव छन्दो॑भि रर्चति । \newline
18. छन्दो॑भि रर्च त्यर्चति॒ छन्दो॑भि॒ श्छन्दो॑भि रर्चति॒ यद् यद॑र्चति॒ छन्दो॑भि॒ श्छन्दो॑भि रर्चति॒ यत् । \newline
19. छन्दो॑भि॒रिति॒ छन्दः॑ - भिः॒ । \newline
20. अ॒र्च॒ति॒ यद् यद॑र्च त्यर्चति॒ यत् किम् किम् ॅयद॑र्च त्यर्चति॒ यत् किम् । \newline
21. यत् किम् किम् ॅयद् यत् किम् च॑ च॒ किम् ॅयद् यत् किम् च॑ । \newline
22. किम् च॑ च॒ किम् किम् चार्च॒ त्यर्च॑ति च॒ किम् किम् चार्च॑ति । \newline
23. चार्च॒ त्यर्च॑ति च॒ चार्च॑ति॒ यद् यदर्च॑ति च॒ चार्च॑ति॒ यत् । \newline
24. अर्च॑ति॒ यद् यदर्च॒ त्यर्च॑ति॒ यदिन्द्र॒ इन्द्रो॒ यदर्च॒ त्यर्च॑ति॒ यदिन्द्रः॑ । \newline
25. यदिन्द्र॒ इन्द्रो॒ यद् यदिन्द्रो॑ वृ॒त्रम् ॅवृ॒त्र मिन्द्रो॒ यद् यदिन्द्रो॑ वृ॒त्रम् । \newline
26. इन्द्रो॑ वृ॒त्रम् ॅवृ॒त्र मिन्द्र॒ इन्द्रो॑ वृ॒त्र मह॒न्, नह॑न् वृ॒त्र मिन्द्र॒ इन्द्रो॑ वृ॒त्र महन्न्॑ । \newline
27. वृ॒त्र मह॒न्, नह॑न् वृ॒त्रम् ॅवृ॒त्र मह॑न्, नमे॒द्ध्य म॑मे॒द्ध्य मह॑न् वृ॒त्रम् ॅवृ॒त्र मह॑न्, नमे॒द्ध्यम् । \newline
28. अह॑न्, नमे॒द्ध्य म॑मे॒द्ध्य मह॒न्, नह॑न्, नमे॒द्ध्यम् तत् तद॑मे॒द्ध्य मह॒न्, नह॑न्, नमे॒द्ध्यम् तत् । \newline
29. अ॒मे॒द्ध्यम् तत् तद॑मे॒द्ध्य म॑मे॒द्ध्यम् तद् यद् यत् तद॑मे॒द्ध्य म॑मे॒द्ध्यम् तद् यत् । \newline
30. तद् यद् यत् तत् तद् यद् यती॒न्॒. यती॒न्॒. यत् तत् तद् यद् यतीन्॑ । \newline
31. यद् यती॒न्॒. यती॒न्॒. यद् यद् यती॑ न॒पाव॑प द॒पाव॑प॒द् यती॒न्॒. यद् यद् यती॑ न॒पाव॑पत् । \newline
32. यती॑ न॒पाव॑प द॒पाव॑प॒द् यती॒न्॒. यती॑ न॒पाव॑प दमे॒द्ध्य म॑मे॒द्ध्य म॒पाव॑प॒द् यती॒न्॒. यती॑ न॒पाव॑प दमे॒द्ध्यम् । \newline
33. अ॒पाव॑प दमे॒द्ध्य म॑मे॒द्ध्य म॒पाव॑प द॒पाव॑प दमे॒द्ध्यम् तत् तद॑मे॒द्ध्य म॒पाव॑प द॒पाव॑प दमे॒द्ध्यम् तत् । \newline
34. अ॒पाव॑प॒दित्य॑प - अव॑पत् । \newline
35. अ॒मे॒द्ध्यम् तत् तद॑मे॒द्ध्य म॑मे॒द्ध्यम् तदथाथ॒ तद॑मे॒द्ध्य म॑मे॒द्ध्यम् तदथ॑ । \newline
36. तदथाथ॒ तत् तदथ॒ कस्मा॒त् कस्मा॒ दथ॒ तत् तदथ॒ कस्मा᳚त् । \newline
37. अथ॒ कस्मा॒त् कस्मा॒ दथाथ॒ कस्मा॑ दै॒न्द्र ऐ॒न्द्रः कस्मा॒ दथाथ॒ कस्मा॑ दै॒न्द्रः । \newline
38. कस्मा॑ दै॒न्द्र ऐ॒न्द्रः कस्मा॒त् कस्मा॑ दै॒न्द्रो य॒ज्ञो य॒ज्ञ् ऐ॒न्द्रः कस्मा॒त् कस्मा॑ दै॒न्द्रो य॒ज्ञ्ः । \newline
39. ऐ॒न्द्रो य॒ज्ञो य॒ज्ञ् ऐ॒न्द्र ऐ॒न्द्रो य॒ज्ञ् आ य॒ज्ञ् ऐ॒न्द्र ऐ॒न्द्रो य॒ज्ञ् आ । \newline
40. य॒ज्ञ् आ य॒ज्ञो य॒ज्ञ् आ सꣳस्था॑तोः॒ सꣳस्था॑तो॒रा य॒ज्ञो य॒ज्ञ् आ सꣳस्था॑तोः । \newline
41. आ सꣳस्था॑तोः॒ सꣳस्था॑तो॒रा सꣳस्था॑तो॒ रितीति॒ सꣳस्था॑तो॒रा सꣳस्था॑तो॒ रिति॑ । \newline
42. सꣳस्था॑तो॒ रितीति॒ सꣳस्था॑तोः॒ सꣳस्था॑तो॒ रित्या॑हु राहु॒रिति॒ सꣳस्था॑तोः॒ सꣳस्था॑तो॒ रित्या॑हुः । \newline
43. सꣳस्था॑तो॒रिति॒ सम् - स्था॒तोः॒ । \newline
44. इत्या॑हु राहु॒ रितीत्या॑हु॒ रिन्द्र॒ स्येन्द्र॑ स्याहु॒ रिती त्या॑हु॒ रिन्द्र॑स्य । \newline
45. आ॒हु॒ रिन्द्र॒ स्येन्द्र॑ स्याहु राहु॒ रिन्द्र॑स्य॒ वै वा इन्द्र॑स्याहु राहु॒ रिन्द्र॑स्य॒ वै । \newline
46. इन्द्र॑स्य॒ वै वा इन्द्र॒ स्येन्द्र॑स्य॒ वा ए॒षैषा वा इन्द्र॒ स्येन्द्र॑स्य॒ वा ए॒षा । \newline
47. वा ए॒षैषा वै वा ए॒षा य॒ज्ञिया॑ य॒ज्ञियै॒षा वै वा ए॒षा य॒ज्ञिया᳚ । \newline
48. ए॒षा य॒ज्ञिया॑ य॒ज्ञि यै॒षैषा य॒ज्ञिया॑ त॒नू स्त॒नूर् य॒ज्ञि यै॒षैषा य॒ज्ञिया॑ त॒नूः । \newline
49. य॒ज्ञिया॑ त॒नू स्त॒नूर् य॒ज्ञिया॑ य॒ज्ञिया॑ त॒नूर् यद् यत् त॒नूर् य॒ज्ञिया॑ य॒ज्ञिया॑ त॒नूर् यत् । \newline
50. त॒नूर् यद् यत् त॒नू स्त॒नूर् यद् य॒ज्ञो य॒ज्ञो यत् त॒नू स्त॒नूर् यद् य॒ज्ञ्ः । \newline
51. यद् य॒ज्ञो य॒ज्ञो यद् यद् य॒ज्ञ् स्ताम् ताम् ॅय॒ज्ञो यद् यद् य॒ज्ञ् स्ताम् । \newline
52. य॒ज्ञ् स्ताम् ताम् ॅय॒ज्ञो य॒ज्ञ् स्ता मे॒वैव ताम् ॅय॒ज्ञो य॒ज्ञ् स्ता मे॒व । \newline
53. ता मे॒वैव ताम् ता मे॒व तत् तदे॒व ताम् ता मे॒व तत् । \newline
54. ए॒व तत् तदे॒वैव तद् य॑जन्ति यजन्ति॒ तदे॒वैव तद् य॑जन्ति । \newline
55. तद् य॑जन्ति यजन्ति॒ तत् तद् य॑जन्ति॒ यो यो य॑जन्ति॒ तत् तद् य॑जन्ति॒ यः । \newline
56. य॒ज॒न्ति॒ यो यो य॑जन्ति यजन्ति॒ य ए॒व मे॒वम् ॅयो य॑जन्ति यजन्ति॒ य ए॒वम् । \newline
57. य ए॒व मे॒वम् ॅयो य ए॒वम् ॅवेद॒ वेदै॒वम् ॅयो य ए॒वम् ॅवेद॑ । \newline
58. ए॒वम् ॅवेद॒ वेदै॒व मे॒वम् ॅवेदोपोप॒ वेदै॒व मे॒वम् ॅवेदोप॑ । \newline
59. वेदोपोप॒ वेद॒ वेदोपै॑न मेन॒ मुप॒ वेद॒ वेदोपै॑नम् । \newline
60. उपै॑न मेन॒ मुपोपै॑नम् ॅय॒ज्ञो य॒ज्ञ् ए॑न॒ मुपोपै॑नम् ॅय॒ज्ञ्ः । \newline
61. ए॒न॒म् ॅय॒ज्ञो य॒ज्ञ् ए॑न मेनम् ॅय॒ज्ञो न॑मति नमति य॒ज्ञ् ए॑न मेनम् ॅय॒ज्ञो न॑मति । \newline
62. य॒ज्ञो न॑मति नमति य॒ज्ञो य॒ज्ञो न॑मति । \newline
63. न॒म॒तीति॑ नमति । \newline
\pagebreak
\markright{ TS 3.3.8.1  \hfill https://www.vedavms.in \hfill}

\section{ TS 3.3.8.1 }

\textbf{TS 3.3.8.1 } \newline
\textbf{Samhita Paata} \newline

आ॒युर्दा अ॑ग्ने ह॒विषो॑ जुषा॒णो घृ॒तप्र॑तीको घृ॒तयो॑निरेधि । घृ॒तं पी॒त्वा मधु॒चारु॒ गव्यं॑ पि॒तेव॑पु॒त्रम॒भि र॑क्षतादि॒मं ॥आ वृ॑श्च्यते॒ वा ए॒तद्-यज॑मानो॒ऽग्निभ्यां॒ ॅयदे॑नयोः शृतं॒ कृत्याथा॒ऽन्यत्रा॑-वभृ॒थम॒वैत्या॑यु॒र्दा अ॑ग्ने ह॒विषो॑ जुषा॒ण इत्य॑वभृ॒थम॑वै॒ष्यन् जु॑हुया॒दाहु॑त्यै॒वैनौ॑ शमयति॒ नाऽऽ*र्ति॒मार्च्छ॑ति॒ यज॑मानो॒ यत् कुसी॑द॒ - [  ] \newline

\textbf{Pada Paata} \newline

आ॒यु॒र्दा इत्या॑युः - दाः । अ॒ग्ने॒ । ह॒विषः॑ । जु॒षा॒णः । घृ॒तप्र॑तीक॒ इति॑ घृ॒त - प्र॒ती॒कः॒ । घृ॒तयो॑नि॒रिति॑ घृ॒त-यो॒निः॒ । ए॒धि॒ ॥ घृ॒तम् । पी॒त्वा । मधु॑ । चारु॑ । गव्य᳚म् । पि॒ता । इ॒व॒ । पु॒त्रम् । अ॒भीति॑ । र॒क्ष॒ता॒त् । इ॒मम् ॥ एति॑ । वृ॒श्च्य॒ते॒ । वै । ए॒तत् । यज॑मानः । अ॒ग्निभ्या॒मित्य॒ग्नि - भ्या॒म् । यत् । ए॒न॒योः॒ । शृ॒त॒कृंत्येति॑ शृतं - कृत्य॑ । अथ॑ । अ॒न्यत्र॑ । अ॒व॒भृ॒थमित्य॑व - भृ॒थम् । अ॒वैतीत्य॑व - एति॑ । आ॒यु॒र्दा इत्या॑युः - दाः । अ॒ग्ने॒ । ह॒विषः॑ । जु॒षा॒णः । इति॑ । अ॒व॒भृ॒थमित्य॑व - भृ॒थम् । अ॒वै॒ष्यन्नित्य॑व-ए॒ष्यन्न् । जु॒हु॒या॒त् । आहु॒त्येत्या - हु॒त्या॒ । ए॒व । ए॒नौ॒ । श॒म॒य॒ति॒ । न । आर्ति᳚म् । एति॑ । ऋ॒च्छ॒ति॒ । यज॑मानः । यत् । कुसी॑दम् ।  \newline


\textbf{Krama Paata} \newline

आ॒यु॒र्दा अ॑ग्ने । आ॒यु॒र्दा इत्या॑युः - दाः । अ॒ग्ने॒ ह॒विषः॑ । ह॒विषो॑ जुषा॒णः । जु॒षा॒णो घृ॒तप्र॑तीकः । घृ॒तप्र॑तीको घृ॒तयो॑निः । घृ॒तप्र॑तीक॒ इति॑ घृ॒त - प्र॒ती॒कः॒ । घृ॒तयो॑निरेधि । घृ॒तयो॑नि॒रिति॑ घृ॒त - यो॒निः॒ । ए॒धीत्ये॑धि ॥ घृ॒तम् पी॒त्वा । पी॒त्वा मधु॑ । मधु॒ चारु॑ । चारु॒ गव्य᳚म् । गव्य॑म् पि॒ता । पि॒तेव॑ । इ॒व॒ पु॒त्रम् । पु॒त्रम॒भि । अ॒भि र॑क्षतात् । र॒क्ष॒ता॒दि॒मम् । इ॒ममिती॒मम् ॥ आ वृ॑श्च्यते । वृ॒श्च्य॒ते॒ वै । वा ए॒तत् । ए॒तद् यज॑मानः । यज॑मानो॒ ऽग्निभ्या᳚म् । अ॒ग्निभ्यां॒ ॅयत् । अ॒ग्निभ्या॒मित्य॒ग्नि - भ्या॒म् । यदे॑नयोः । ए॒न॒योः॒ शृ॒त॒ङ्कृत्य॑ । शृ॒त॒ङ्कृत्याथ॑ । शृ॒त॒ङ्कृत्येति॑ शृतम् - कृत्य॑ । अथा॒न्यत्र॑ । अ॒न्यत्रा॑वभृ॒थम् । अ॒व॒भृ॒थम॒वैति॑ । अ॒व॒भृ॒थमित्य॑व - भृ॒थम् । अ॒वैत्या॑यु॒र्दाः । अ॒वैतीत्य॑व - एति॑ । आ॒यु॒र्दा अ॑ग्ने । आ॒यु॒र्दा इत्या॑युः - दाः । अ॒ग्ने॒ ह॒विषः॑ । ह॒विषो॑ जुषा॒णः । जु॒षा॒ण इति॑ । इत्य॑वभृ॒थम् । अ॒व॒भृ॒थम॑वै॒ष्यन्न् । अ॒व॒भृ॒थमित्य॑व - भृ॒थम् । अ॒वै॒ष्यन् जु॑हुयात् । अ॒वै॒ष्यन्नित्य॑व - ए॒ष्यन्न् । जु॒हु॒या॒दाहु॑त्या । आहु॑त्यै॒व । आहु॒त्येत्या - हु॒त्या॒ । ए॒वैनौ᳚ । ए॒नौ॒ श॒म॒य॒ति॒ । श॒म॒य॒ति॒ न । नार्ति᳚म् । आर्ति॒मा । आर्च्छ॑ति । ऋ॒च्छ॒ति॒ यज॑मानः । यज॑मानो॒ यत् । यत् कुसी॑दम् । कुसी॑द॒मप्र॑तीत्तम् \newline

\textbf{Jatai Paata} \newline

1. आ॒यु॒र्दा अ॑ग्ने अग्न आयु॒र्दा आ॑यु॒र्दा अ॑ग्ने । \newline
2. आ॒यु॒र्दा इत्या॑युः - दाः । \newline
3. अ॒ग्ने॒ ह॒विषो॑ ह॒विषो॑ अग्ने अग्ने ह॒विषः॑ । \newline
4. ह॒विषो॑ जुषा॒णो जु॑षा॒णो ह॒विषो॑ ह॒विषो॑ जुषा॒णः । \newline
5. जु॒षा॒णो घृ॒तप्र॑तीको घृ॒तप्र॑तीको जुषा॒णो जु॑षा॒णो घृ॒तप्र॑तीकः । \newline
6. घृ॒तप्र॑तीको घृ॒तयो॑निर् घृ॒तयो॑निर् घृ॒तप्र॑तीको घृ॒तप्र॑तीको घृ॒तयो॑निः । \newline
7. घृ॒तप्र॑तीक॒ इति॑ घृ॒त - प्र॒ती॒कः॒ । \newline
8. घृ॒तयो॑नि रेध्येधि घृ॒तयो॑निर् घृ॒तयो॑नि रेधि । \newline
9. घृ॒तयो॑नि॒रिति॑ घृ॒त - यो॒निः॒ । \newline
10. ए॒धीत्ये॑धि । \newline
11. घृ॒तम् पी॒त्वा पी॒त्वा घृ॒तम् घृ॒तम् पी॒त्वा । \newline
12. पी॒त्वा मधु॒ मधु॑ पी॒त्वा पी॒त्वा मधु॑ । \newline
13. मधु॒ चारु॒ चारु॒ मधु॒ मधु॒ चारु॑ । \newline
14. चारु॒ गव्य॒म् गव्य॒म् चारु॒ चारु॒ गव्य᳚म् । \newline
15. गव्य॑म् पि॒ता पि॒ता गव्य॒म् गव्य॑म् पि॒ता । \newline
16. पि॒तेवे॑ व पि॒ता पि॒तेव॑ । \newline
17. इ॒व॒ पु॒त्रम् पु॒त्र मि॑वे व पु॒त्रम् । \newline
18. पु॒त्र म॒भ्य॑भि पु॒त्रम् पु॒त्र म॒भि । \newline
19. अ॒भि र॑क्षताद् रक्षता द॒भ्य॑भि र॑क्षतात् । \newline
20. र॒क्ष॒ता॒ दि॒म मि॒मꣳ र॑क्षताद् रक्षता दि॒मम् । \newline
21. इ॒ममिती॒मम् । \newline
22. आ वृ॑श्च्यते वृश्च्यत॒ आ वृ॑श्च्यते । \newline
23. वृ॒श्च्य॒ते॒ वै वै वृ॑श्च्यते वृश्च्यते॒ वै । \newline
24. वा ए॒त दे॒तद् वै वा ए॒तत् । \newline
25. ए॒तद् यज॑मानो॒ यज॑मान ए॒त दे॒तद् यज॑मानः । \newline
26. यज॑मानो॒ ऽग्निभ्या॑ म॒ग्निभ्या॒म् ॅयज॑मानो॒ यज॑मानो॒ ऽग्निभ्या᳚म् । \newline
27. अ॒ग्निभ्या॒म् ॅयद् यद॒ग्निभ्या॑ म॒ग्निभ्या॒म् ॅयत् । \newline
28. अ॒ग्निभ्या॒मित्य॒ग्नि - भ्या॒म् । \newline
29. यदे॑नयो रेनयो॒र् यद् यदे॑नयोः । \newline
30. ए॒न॒योः॒ शृ॒त॒ङ्कृत्य॑ शृत॒ङ्कृ त्यै॑नयो रेनयोः शृत॒ङ्कृत्य॑ । \newline
31. शृ॒त॒ङ्कृ त्याथाथ॑ शृत॒ङ्कृत्य॑ शृत॒ङ्कृ त्याथ॑ । \newline
32. शृ॒त॒ङ्कृत्येति॑ शृतम् - कृत्य॑ । \newline
33. अथा॒ न्यत्रा॒ न्यत्रा थाथा॒ न्यत्र॑ । \newline
34. अ॒न्यत्रा॑ वभृ॒थ म॑वभृ॒थ म॒न्यत्रा॒ न्यत्रा॑ वभृ॒थम् । \newline
35. अ॒व॒भृ॒थ म॒वै त्य॒वै त्य॑वभृ॒थ म॑वभृ॒थ म॒वैति॑ । \newline
36. अ॒व॒भृ॒थमित्य॑व - भृ॒थम् । \newline
37. अ॒वै त्या॑यु॒र्दा आ॑यु॒र्दा अ॒वै त्य॒वै त्या॑यु॒र्दाः । \newline
38. अ॒वैतीत्य॑व - एति॑ । \newline
39. आ॒यु॒र्दा अ॑ग्ने अग्न आयु॒र्दा आ॑यु॒र्दा अ॑ग्ने । \newline
40. आ॒यु॒र्दा इत्या॑युः - दाः । \newline
41. अ॒ग्ने॒ ह॒विषो॑ ह॒विषो॑ अग्ने अग्ने ह॒विषः॑ । \newline
42. ह॒विषो॑ जुषा॒णो जु॑षा॒णो ह॒विषो॑ ह॒विषो॑ जुषा॒णः । \newline
43. जु॒षा॒ण इतीति॑ जुषा॒णो जु॑षा॒ण इति॑ । \newline
44. इत्य॑वभृ॒थ म॑वभृ॒थ मिती त्य॑वभृ॒थम् । \newline
45. अ॒व॒भृ॒थ म॑वै॒ष्यन् न॑वै॒ष्यन् न॑वभृ॒थ म॑वभृ॒थ म॑वै॒ष्यन्न् । \newline
46. अ॒व॒भृ॒थमित्य॑व - भृ॒थम् । \newline
47. अ॒वै॒ष्यन् जु॑हुयाज् जुहुया दवै॒ष्यन् न॑वै॒ष्यन् जु॑हुयात् । \newline
48. अ॒वै॒ष्यन्नित्य॑व - ए॒ष्यन्न् । \newline
49. जु॒हु॒या॒ दाहु॒त्या ऽऽहु॑त्या जुहुयाज् जुहुया॒ दाहु॑त्या । \newline
50. आहु॑ त्यै॒वै वाहु॒त्या ऽऽहु॑त्यै॒व । \newline
51. आहु॒त्येत्या - हु॒त्या॒ । \newline
52. ए॒वैना॑ वेना वे॒वैवैनौ᳚ । \newline
53. ए॒नौ॒ श॒म॒य॒ति॒ श॒म॒य॒ त्ये॒ना॒ वे॒नौ॒ श॒म॒य॒ति॒ । \newline
54. श॒म॒य॒ति॒ न न श॑मयति शमयति॒ न । \newline
55. नार्ति॒ मार्ति॒म् न नार्ति᳚म् । \newline
56. आर्ति॒ मा ऽऽर्ति॒ मार्ति॒ मा । \newline
57. आर्च्छ॑ त्यृच्छ॒ त्यार्च्छ॑ति । \newline
58. ऋ॒च्छ॒ति॒ यज॑मानो॒ यज॑मान ऋच्छ त्यृच्छति॒ यज॑मानः । \newline
59. यज॑मानो॒ यद् यद् यज॑मानो॒ यज॑मानो॒ यत् । \newline
60. यत् कुसी॑द॒म् कुसी॑द॒म् ॅयद् यत् कुसी॑दम् । \newline
61. कुसी॑द॒ मप्र॑तीत्त॒ मप्र॑तीत्त॒म् कुसी॑द॒म् कुसी॑द॒ मप्र॑तीत्तम् । \newline

\textbf{Ghana Paata } \newline

1. आ॒यु॒र्दा अ॑ग्ने अग्न आयु॒र्दा आ॑यु॒र्दा अ॑ग्ने ह॒विषो॑ ह॒विषो॑ अग्न आयु॒र्दा आ॑यु॒र्दा अ॑ग्ने ह॒विषः॑ । \newline
2. आ॒यु॒र्दा इत्या॑युः - दाः । \newline
3. अ॒ग्ने॒ ह॒विषो॑ ह॒विषो॑ अग्ने अग्ने ह॒विषो॑ जुषा॒णो जु॑षा॒णो ह॒विषो॑ अग्ने अग्ने ह॒विषो॑ जुषा॒णः । \newline
4. ह॒विषो॑ जुषा॒णो जु॑षा॒णो ह॒विषो॑ ह॒विषो॑ जुषा॒णो घृ॒तप्र॑तीको घृ॒तप्र॑तीको जुषा॒णो ह॒विषो॑ ह॒विषो॑ जुषा॒णो घृ॒तप्र॑तीकः । \newline
5. जु॒षा॒णो घृ॒तप्र॑तीको घृ॒तप्र॑तीको जुषा॒णो जु॑षा॒णो घृ॒तप्र॑तीको घृ॒तयो॑निर् घृ॒तयो॑निर् घृ॒तप्र॑तीको जुषा॒णो जु॑षा॒णो घृ॒तप्र॑तीको घृ॒तयो॑निः । \newline
6. घृ॒तप्र॑तीको घृ॒तयो॑निर् घृ॒तयो॑निर् घृ॒तप्र॑तीको घृ॒तप्र॑तीको घृ॒तयो॑नि रेध्येधि घृ॒तयो॑निर् घृ॒तप्र॑तीको घृ॒तप्र॑तीको घृ॒तयो॑नि रेधि । \newline
7. घृ॒तप्र॑तीक॒ इति॑ घृ॒त - प्र॒ती॒कः॒ । \newline
8. घृ॒तयो॑नि रेध्येधि घृ॒तयो॑निर् घृ॒तयो॑नि रेधि । \newline
9. घृ॒तयो॑नि॒रिति॑ घृ॒त - यो॒निः॒ । \newline
10. ए॒धीत्ये॑धि । \newline
11. घृ॒तम् पी॒त्वा पी॒त्वा घृ॒तम् घृ॒तम् पी॒त्वा मधु॒ मधु॑ पी॒त्वा घृ॒तम् घृ॒तम् पी॒त्वा मधु॑ । \newline
12. पी॒त्वा मधु॒ मधु॑ पी॒त्वा पी॒त्वा मधु॒ चारु॒ चारु॒ मधु॑ पी॒त्वा पी॒त्वा मधु॒ चारु॑ । \newline
13. मधु॒ चारु॒ चारु॒ मधु॒ मधु॒ चारु॒ गव्य॒म् गव्य॒म् चारु॒ मधु॒ मधु॒ चारु॒ गव्य᳚म् । \newline
14. चारु॒ गव्य॒म् गव्य॒म् चारु॒ चारु॒ गव्य॑म् पि॒ता पि॒ता गव्य॒म् चारु॒ चारु॒ गव्य॑म् पि॒ता । \newline
15. गव्य॑म् पि॒ता पि॒ता गव्य॒म् गव्य॑म् पि॒तेवे॑ व पि॒ता गव्य॒म् गव्य॑म् पि॒तेव॑ । \newline
16. पि॒तेवे॑ व पि॒ता पि॒तेव॑ पु॒त्रम् पु॒त्र मि॑व पि॒ता पि॒तेव॑ पु॒त्रम् । \newline
17. इ॒व॒ पु॒त्रम् पु॒त्र मि॑वे व पु॒त्र म॒भ्य॑भि पु॒त्र मि॑वे व पु॒त्र म॒भि । \newline
18. पु॒त्र म॒भ्य॑भि पु॒त्रम् पु॒त्र म॒भि र॑क्षताद् रक्षता द॒भि पु॒त्रम् पु॒त्र म॒भि र॑क्षतात् । \newline
19. अ॒भि र॑क्षताद् रक्षता द॒भ्य॑भि र॑क्षता दि॒म मि॒मꣳ र॑क्षता द॒भ्य॑भि र॑क्षता दि॒मम् । \newline
20. र॒क्ष॒ता॒ दि॒म मि॒मꣳ र॑क्षताद् रक्षता दि॒मम् । \newline
21. इ॒ममिती॒मम् । \newline
22. आ वृ॑श्च्यते वृश्च्यत॒ आ वृ॑श्च्यते॒ वै वै वृ॑श्च्यत॒ आ वृ॑श्च्यते॒ वै । \newline
23. वृ॒श्च्य॒ते॒ वै वै वृ॑श्च्यते वृश्च्यते॒ वा ए॒त दे॒तद् वै वृ॑श्च्यते वृश्च्यते॒ वा ए॒तत् । \newline
24. वा ए॒त दे॒तद् वै वा ए॒तद् यज॑मानो॒ यज॑मान ए॒तद् वै वा ए॒तद् यज॑मानः । \newline
25. ए॒तद् यज॑मानो॒ यज॑मान ए॒त दे॒तद् यज॑मानो॒ ऽग्निभ्या॑ म॒ग्निभ्या॒म् ॅयज॑मान ए॒त दे॒तद् यज॑मानो॒ ऽग्निभ्या᳚म् । \newline
26. यज॑मानो॒ ऽग्निभ्या॑ म॒ग्निभ्या॒म् ॅयज॑मानो॒ यज॑मानो॒ ऽग्निभ्या॒म् ॅयद् यद॒ग्निभ्या॒म् ॅयज॑मानो॒ यज॑मानो॒ ऽग्निभ्या॒म् ॅयत् । \newline
27. अ॒ग्निभ्या॒म् ॅयद् यद॒ग्निभ्या॑ म॒ग्निभ्या॒म् ॅयदे॑नयो रेनयो॒र् यद॒ग्निभ्या॑ म॒ग्निभ्या॒म् ॅयदे॑नयोः । \newline
28. अ॒ग्निभ्या॒मित्य॒ग्नि - भ्या॒म् । \newline
29. यदे॑नयो रेनयो॒र् यद् यदे॑नयोः शृत॒ङ्कृत्य॑ शृत॒ङ्कृ त्यै॑नयो॒र् यद् यदे॑नयोः शृत॒ङ्कृत्य॑ । \newline
30. ए॒न॒योः॒ शृ॒त॒ङ्कृत्य॑ शृत॒ङ्कृ त्यै॑नयो रेनयोः शृत॒ङ्कृ त्याथाथ॑ शृत॒ङ्कृ त्यै॑नयो रेनयोः शृत॒ङ्कृत्याथ॑ । \newline
31. शृ॒त॒ङ्कृ त्याथाथ॑ शृत॒ङ्कृत्य॑ शृत॒ङ्कृ त्याथा॒ न्यत्रा॒ न्यत्राथ॑ शृत॒ङ्कृत्य॑ शृत॒ङ्कृ त्याथा॒ न्यत्र॑ । \newline
32. शृ॒त॒ङ्कृत्येति॑ शृतम् - कृत्य॑ । \newline
33. अथा॒ न्यत्रा॒ न्यत्रा थाथा॒ न्यत्रा॑ वभृ॒थ म॑वभृ॒थ म॒न्यत्रा थाथा॒ न्यत्रा॑ वभृ॒थम् । \newline
34. अ॒न्यत्रा॑ वभृ॒थ म॑वभृ॒थ म॒न्यत्रा॒ न्यत्रा॑ वभृ॒थ म॒वै त्य॒वै त्य॑वभृ॒थ म॒न्यत्रा॒ न्यत्रा॑ वभृ॒थ म॒वैति॑ । \newline
35. अ॒व॒भृ॒थ म॒वै त्य॒वै त्य॑वभृ॒थ म॑वभृ॒थ म॒वै त्या॑यु॒र्दा आ॑यु॒र्दा अ॒वै त्य॑वभृ॒थ म॑वभृ॒थ म॒वै त्या॑यु॒र्दाः । \newline
36. अ॒व॒भृ॒थमित्य॑व - भृ॒थम् । \newline
37. अ॒वै त्या॑यु॒र्दा आ॑यु॒र्दा अ॒वै त्य॒वै त्या॑यु॒र्दा अ॑ग्ने अग्न आयु॒र्दा अ॒वै त्य॒वै त्या॑यु॒र्दा अ॑ग्ने । \newline
38. अ॒वैतीत्य॑व - एति॑ । \newline
39. आ॒यु॒र्दा अ॑ग्ने अग्न आयु॒र्दा आ॑यु॒र्दा अ॑ग्ने ह॒विषो॑ ह॒विषो॑ अग्न आयु॒र्दा आ॑यु॒र्दा अ॑ग्ने ह॒विषः॑ । \newline
40. आ॒यु॒र्दा इत्या॑युः - दाः । \newline
41. अ॒ग्ने॒ ह॒विषो॑ ह॒विषो॑ अग्ने अग्ने ह॒विषो॑ जुषा॒णो जु॑षा॒णो ह॒विषो॑ अग्ने अग्ने ह॒विषो॑ जुषा॒णः । \newline
42. ह॒विषो॑ जुषा॒णो जु॑षा॒णो ह॒विषो॑ ह॒विषो॑ जुषा॒ण इतीति॑ जुषा॒णो ह॒विषो॑ ह॒विषो॑ जुषा॒ण इति॑ । \newline
43. जु॒षा॒ण इतीति॑ जुषा॒णो जु॑षा॒ण इत्य॑वभृ॒थ म॑वभृ॒थ मिति॑ जुषा॒णो जु॑षा॒ण इत्य॑ वभृ॒थम् । \newline
44. इत्य॑वभृ॒थ म॑वभृ॒थ मिती त्य॑वभृ॒थ म॑वै॒ष्यन्, न॑वै॒ष्यन्, न॑वभृ॒थ मिती त्य॑वभृ॒थ म॑वै॒ष्यन्न् । \newline
45. अ॒व॒भृ॒थ म॑वै॒ष्यन्, न॑वै॒ष्यन्, न॑वभृ॒थ म॑वभृ॒थ म॑वै॒ष्यन् जु॑हुयाज् जुहुया दवै॒ष्यन्, न॑वभृ॒थ म॑वभृ॒थ म॑वै॒ष्यन् जु॑हुयात् । \newline
46. अ॒व॒भृ॒थमित्य॑व - भृ॒थम् । \newline
47. अ॒वै॒ष्यन् जु॑हुयाज् जुहुया दवै॒ष्यन्, न॑वै॒ष्यन् जु॑हुया॒ दाहु॒त्या ऽऽहु॑त्या जुहुया दवै॒ष्यन्, न॑वै॒ष्यन् जु॑हुया॒ दाहु॑त्या । \newline
48. अ॒वै॒ष्यन्नित्य॑व - ए॒ष्यन्न् । \newline
49. जु॒हु॒या॒ दाहु॒त्या ऽऽहु॑त्या जुहुयाज् जुहुया॒ दाहु॑ त्यै॒वैवा हु॑त्या जुहुयाज् जुहुया॒ दाहु॑ त्यै॒व । \newline
50. आहु॑ त्यै॒वैवा हु॒त्या ऽऽहु॑त्यै॒वैना॑ वेना वे॒वा हु॒त्या ऽऽहु॑त्यै॒वैनौ᳚ । \newline
51. आहु॒त्येत्या - हु॒त्या॒ । \newline
52. ए॒वैना॑ वेना वे॒वैवैनौ॑ शमयति शमय त्येना वे॒वैवैनौ॑ शमयति । \newline
53. ए॒नौ॒ श॒म॒य॒ति॒ श॒म॒य॒ त्ये॒ना॒ वे॒नौ॒ श॒म॒य॒ति॒ न न श॑मय त्येना वेनौ शमयति॒ न । \newline
54. श॒म॒य॒ति॒ न न श॑मयति शमयति॒ नार्ति॒ मार्ति॒म् न श॑मयति शमयति॒ नार्ति᳚म् । \newline
55. नार्ति॒ मार्ति॒म् न नार्ति॒ मा ऽऽर्ति॒म् न नार्ति॒ मा । \newline
56. आर्ति॒ मा ऽऽर्ति॒ मार्ति॒ मार्च्छ॑ त्यृच्छ॒त्या ऽऽर्ति॒ मार्ति॒ मार्च्छ॑ति । \newline
57. आर्च्छ॑ त्यृच्छ॒ त्यार्च्छ॑ति॒ यज॑मानो॒ यज॑मान ऋच्छ॒ त्यार्च्छ॑ति॒ यज॑मानः । \newline
58. ऋ॒च्छ॒ति॒ यज॑मानो॒ यज॑मान ऋच्छ त्यृच्छति॒ यज॑मानो॒ यद् यद् यज॑मान ऋच्छ त्यृच्छति॒ यज॑मानो॒ यत् । \newline
59. यज॑मानो॒ यद् यद् यज॑मानो॒ यज॑मानो॒ यत् कुसी॑द॒म् कुसी॑द॒म् ॅयद् यज॑मानो॒ यज॑मानो॒ यत् कुसी॑दम् । \newline
60. यत् कुसी॑द॒म् कुसी॑द॒म् ॅयद् यत् कुसी॑द॒ मप्र॑तीत्त॒ मप्र॑तीत्त॒म् कुसी॑द॒म् ॅयद् यत् कुसी॑द॒ मप्र॑तीत्तम् । \newline
61. कुसी॑द॒ मप्र॑तीत्त॒ मप्र॑तीत्त॒म् कुसी॑द॒म् कुसी॑द॒ मप्र॑तीत्त॒म् मयि॒ मय्यप्र॑तीत्त॒म् कुसी॑द॒म् कुसी॑द॒ मप्र॑तीत्त॒म् मयि॑ । \newline
\pagebreak
\markright{ TS 3.3.8.2  \hfill https://www.vedavms.in \hfill}

\section{ TS 3.3.8.2 }

\textbf{TS 3.3.8.2 } \newline
\textbf{Samhita Paata} \newline

मप्र॑तीत्तं॒ मयि॒ येन॑ य॒मस्य॑ ब॒लिना॒ चरा॑मि । इ॒हैव सन् नि॒रव॑दये॒ तदे॒तत् तद॑ग्ने अनृ॒णो भ॑वामि । विश्व॑लोप विश्वदा॒वस्य॑ त्वा॒ ऽऽसञ्जु॑होम्य॒ग्धादेको॑ ऽहु॒तादेकः॑ समस॒नादेकः॑ । तेनः॑ कृण्वन्तु भेष॒जꣳ सदः॒ सहो॒ वरे᳚ण्यं ॥ अ॒यं नो॒ नभ॑सा पु॒रः सꣳ॒॒स्फानो॑ अ॒भि र॑क्षतु । गृ॒हाणा॒मस॑मर्त्यै ब॒हवो॑ नो गृ॒हा अ॑सन्न् ॥ स त्वन्नो॑ - [  ] \newline

\textbf{Pada Paata} \newline

अप्र॑तीत्त॒मित्यप्र॑ति-इ॒त्त॒म् । मयि॑ । येन॑ । य॒मस्य॑ । ब॒लिना᳚ । चरा॑मि ॥ इ॒ह । ए॒व । सन्न् । नि॒रव॑दय॒ इति॑ निः-अव॑दये । तत् । ए॒तत् । तत् । अ॒ग्ने॒ । अ॒नृ॒णः । भ॒वा॒मि॒ ॥ विश्व॑लो॒पेति॒ विश्व॑ - लो॒प॒ । वि॒श्व॒दा॒वस्येति॑ विश्व - दा॒वस्य॑ । त्वा॒ । आ॒सन्न् । जु॒हो॒मि॒ । अ॒ग्धादित्य॑ग्ध - अत् । एकः॑ । अ॒हु॒तादित्य॑हुत - अत् । एकः॑ । स॒म॒स॒नादिति॑ समसन - अत् । एकः॑ ॥ ते । नः॒ । कृ॒ण्व॒न्तु॒ । भे॒ष॒जम् । सदः॑ । सहः॑ । वरे᳚ण्यम् ॥ अ॒यम् । नः॒ । नभ॑सा । पु॒रः । सꣳ॒॒स्फान॒ इति॑ सं - स्फानः॑ । अ॒भीति॑ । र॒क्ष॒तु॒ ॥ गृ॒हाणा᳚म् । अस॑मर्त्या॒ इत्यसं᳚ - ऋ॒त्यै॒ । ब॒हवः॑ । नः॒ । गृ॒हाः । अ॒स॒न्न् ॥ सः । त्वम् । नः॒ ।  \newline


\textbf{Krama Paata} \newline

अप्र॑तीत्त॒म् मयि॑ । अप्र॑तीत्त॒मित्यप्र॑ति - इ॒त्त॒म् । मयि॒ येन॑ । येन॑ य॒मस्य॑ । य॒मस्य॑ ब॒लिना᳚ । ब॒लिना॒ चरा॑मि । चरा॒मीति॒ चरा॑मि ॥ इ॒हैव । ए॒व सन्न् । स॒न् नि॒रव॑दये । नि॒रव॑दये॒ तत् । नि॒रव॑दय॒ इति॑ निः - अव॑दये । तदे॒तत् । ए॒तत् तत् । तद॑ग्ने । अ॒ग्ने॒ अ॒नृ॒णः । अ॒नृ॒णो भ॑वामि । भ॒वा॒मीति॑ भवामि ॥ विश्व॑लोप विश्वदा॒वस्य॑ । विश्व॑लो॒पेति॒ विश्व॑ - लो॒प॒ । वि॒श्व॒दा॒वस्य॑ त्वा । वि॒श्व॒दा॒वस्येति॑ विश्व - दा॒वस्य॑ । त्वा॒ऽऽसन्न् । आ॒सन् जु॑होमि । जु॒हो॒म्य॒ग्धात् । अ॒ग्धादेकः॑ । अ॒ग्धादित्य॑ग्ध - अत् । एको॑ ऽहु॒तात् । अ॒हु॒तादेकः॑ । अ॒हु॒तादित्य॑हुत - अत् । एकः॑ समस॒नात् । स॒म॒स॒नादेकः॑ । स॒म॒स॒नादिति॑ समसन - अत् । एक॒ इत्येकः॑ ॥ ते नः॑ । नः॒ कृ॒ण्व॒न्तु॒ । कृ॒ण्व॒न्तु॒ भे॒ष॒जम् । भे॒ष॒जꣳ सदः॑ । सदः॒ सहः॑ । सहो॒ वरे᳚ण्यम् । वरे᳚ण्य॒मिति॒ वरे᳚ण्यम् ॥ अ॒यम् नः॑ । नो॒ नभ॑सा । नभ॑सा पु॒रः । पु॒रः सꣳ॒॒स्फानः॑ । सꣳ॒॒स्फानो॑ अ॒भि । सꣳ॒॒स्फान॒ इति॑ सम् - स्फानः॑ । अ॒भि र॑क्षतु । र॒क्ष॒त्विति॑ रक्षतु ॥ गृ॒हाणा॒मस॑मर्त्यै । अस॑मर्त्यै ब॒हवः॑ । अस॑मर्त्या॒ इत्यस᳚म् - ऋ॒त्यै॒ । ब॒हवो॑ नः । नो॒ गृ॒हाः । गृ॒हा अ॑सन्न् । अ॒स॒न्नित्य॑सन्न् ॥ स त्वम् । त्वम् नः॑ । नो॒ न॒भ॒सः॒ \newline

\textbf{Jatai Paata} \newline

1. अप्र॑तीत्त॒म् मयि॒ मय्यप्र॑तीत्त॒ मप्र॑तीत्त॒म् मयि॑ । \newline
2. अप्र॑तीत्त॒मित्यप्र॑ति - इ॒त्त॒म् । \newline
3. मयि॒ येन॒ येन॒ मयि॒ मयि॒ येन॑ । \newline
4. येन॑ य॒मस्य॑ य॒मस्य॒ येन॒ येन॑ य॒मस्य॑ । \newline
5. य॒मस्य॑ ब॒लिना॑ ब॒लिना॑ य॒मस्य॑ य॒मस्य॑ ब॒लिना᳚ । \newline
6. ब॒लिना॒ चरा॑मि॒ चरा॑मि ब॒लिना॑ ब॒लिना॒ चरा॑मि । \newline
7. चरा॒मीति॒ चरा॑मि । \newline
8. इ॒हैवैवे हे हैव । \newline
9. ए॒व सन् थ्सन् ने॒वैव सन्न् । \newline
10. सन् नि॒रव॑दये नि॒रव॑दये॒ सन् थ्सन् नि॒रव॑दये । \newline
11. नि॒रव॑दये॒ तत् तन् नि॒रव॑दये नि॒रव॑दये॒ तत् । \newline
12. नि॒रव॑दय॒ इति॑ निः - अव॑दये । \newline
13. तदे॒त दे॒तत् तत् तदे॒तत् । \newline
14. ए॒तत् तत् तदे॒त दे॒तत् तत् । \newline
15. तद॑ग्ने ऽग्ने॒ तत् तद॑ग्ने । \newline
16. अ॒ग्ने॒ अ॒नृ॒णो अ॑नृ॒णो᳚ ऽग्ने ऽग्ने अनृ॒णः । \newline
17. अ॒नृ॒णो भ॑वामि भवा म्यनृ॒णो अ॑नृ॒णो भ॑वामि । \newline
18. भ॒वा॒मीति॑ भवामि । \newline
19. विश्व॑लोप विश्वदा॒वस्य॑ विश्वदा॒वस्य॒ विश्व॑लोप॒ विश्व॑लोप विश्वदा॒वस्य॑ । \newline
20. विश्व॑लो॒पेति॒ विश्व॑ - लो॒प॒ । \newline
21. वि॒श्व॒दा॒वस्य॑ त्वा त्वा विश्वदा॒वस्य॑ विश्वदा॒वस्य॑ त्वा । \newline
22. वि॒श्व॒दा॒वस्येति॑ विश्व - दा॒वस्य॑ । \newline
23. त्वा॒ ऽऽसन् ना॒सन् त्वा᳚ त्वा॒ ऽऽसन्न् । \newline
24. आ॒सन् जु॑होमि जुहो म्या॒सन् ना॒सन् जु॑होमि । \newline
25. जु॒हो॒ म्य॒ग्धा द॒ग्धाज् जु॑होमि जुहो म्य॒ग्धात् । \newline
26. अ॒ग्धा देक॒ एको॒ ऽग्धा द॒ग्धा देकः॑ । \newline
27. अ॒ग्धादित्य॑ग्ध - अत् । \newline
28. एको॑ ऽहु॒ता द॑हु॒ता देक॒ एको॑ ऽहु॒तात् । \newline
29. अ॒हु॒ता देक॒ एको॑ ऽहु॒ता द॑हु॒ता देकः॑ । \newline
30. अ॒हु॒तादित्य॑हुत - अत् । \newline
31. एकः॑ समस॒नाथ् स॑मस॒ना देक॒ एकः॑ समस॒नात् । \newline
32. स॒म॒स॒ना देक॒ एकः॑ समस॒नाथ् स॑मस॒ना देकः॑ । \newline
33. स॒म॒स॒नादिति॑ समसन - अत् । \newline
34. एक॒ इत्येकः॑ । \newline
35. ते नो॑ न॒ स्ते ते नः॑ । \newline
36. नः॒ कृ॒ण्व॒न्तु॒ कृ॒ण्व॒न्तु॒ नो॒ नः॒ कृ॒ण्व॒न्तु॒ । \newline
37. कृ॒ण्व॒न्तु॒ भे॒ष॒जम् भे॑ष॒जम् कृ॑ण्वन्तु कृण्वन्तु भेष॒जम् । \newline
38. भे॒ष॒जꣳ सदः॒ सदो॑ भेष॒जम् भे॑ष॒जꣳ सदः॑ । \newline
39. सदः॒ सहः॒ सहः॒ सदः॒ सदः॒ सहः॑ । \newline
40. सहो॒ वरे᳚ण्य॒म् ॅवरे᳚ण्यꣳ॒॒ सहः॒ सहो॒ वरे᳚ण्यम् । \newline
41. वरे᳚ण्य॒मिति॒ वरे᳚ण्यम् । \newline
42. अ॒यम् नो॑ नो॒ ऽय म॒यम् नः॑ । \newline
43. नो॒ नभ॑सा॒ नभ॑सा नो नो॒ नभ॑सा । \newline
44. नभ॑सा पु॒रः पु॒रो नभ॑सा॒ नभ॑सा पु॒रः । \newline
45. पु॒रः सꣳ॒॒स्फानः॑ सꣳ॒॒स्फानः॑ पु॒रः पु॒रः सꣳ॒॒स्फानः॑ । \newline
46. सꣳ॒॒स्फानो॑ अ॒भ्य॑भि सꣳ॒॒स्फानः॑ सꣳ॒॒स्फानो॑ अ॒भि । \newline
47. सꣳ॒॒स्फान॒ इति॑ सम् - स्फानः॑ । \newline
48. अ॒भि र॑क्षतु रक्ष त्व॒भ्य॑भि र॑क्षतु । \newline
49. र॒क्ष॒त्विति॑ रक्षतु । \newline
50. गृ॒हाणा॒ मस॑मर्त्या॒ अस॑मर्त्यै गृ॒हाणा᳚म् गृ॒हाणा॒ मस॑मर्त्यै । \newline
51. अस॑मर्त्यै ब॒हवो॑ ब॒हवो ऽस॑मर्त्या॒ अस॑मर्त्यै ब॒हवः॑ । \newline
52. अस॑मर्त्या॒ इत्यस᳚म् - ऋ॒त्यै॒ । \newline
53. ब॒हवो॑ नो नो ब॒हवो॑ ब॒हवो॑ नः । \newline
54. नो॒ गृ॒हा गृ॒हा नो॑ नो गृ॒हाः । \newline
55. गृ॒हा अ॑सन् नसन् गृ॒हा गृ॒हा अ॑सन्न् । \newline
56. अ॒स॒न्नित्य॑सन्न् । \newline
57. स त्वम् त्वꣳ स स त्वम् । \newline
58. त्वम् नो॑ न॒स्त्वम् त्वम् नः॑ । \newline
59. नो॒ न॒भ॒सो॒ न॒भ॒सो॒ नो॒ नो॒ न॒भ॒सः॒ । \newline

\textbf{Ghana Paata } \newline

1. अप्र॑तीत्त॒म् मयि॒ मय्यप्र॑तीत्त॒ मप्र॑तीत्त॒म् मयि॒ येन॒ येन॒ मय्यप्र॑तीत्त॒ मप्र॑तीत्त॒म् मयि॒ येन॑ । \newline
2. अप्र॑तीत्त॒मित्यप्र॑ति - इ॒त्त॒म् । \newline
3. मयि॒ येन॒ येन॒ मयि॒ मयि॒ येन॑ य॒मस्य॑ य॒मस्य॒ येन॒ मयि॒ मयि॒ येन॑ य॒मस्य॑ । \newline
4. येन॑ य॒मस्य॑ य॒मस्य॒ येन॒ येन॑ य॒मस्य॑ ब॒लिना॑ ब॒लिना॑ य॒मस्य॒ येन॒ येन॑ य॒मस्य॑ ब॒लिना᳚ । \newline
5. य॒मस्य॑ ब॒लिना॑ ब॒लिना॑ य॒मस्य॑ य॒मस्य॑ ब॒लिना॒ चरा॑मि॒ चरा॑मि ब॒लिना॑ य॒मस्य॑ य॒मस्य॑ ब॒लिना॒ चरा॑मि । \newline
6. ब॒लिना॒ चरा॑मि॒ चरा॑मि ब॒लिना॑ ब॒लिना॒ चरा॑मि । \newline
7. चरा॒मीति॒ चरा॑मि । \newline
8. इ॒हैवैवे हे हैव सन् थ्सन्, ने॒वे हे हैव सन्न् । \newline
9. ए॒व सन् थ्सन्, ने॒वैव सन् नि॒रव॑दये नि॒रव॑दये॒ सन्, ने॒वैव सन् नि॒रव॑दये । \newline
10. सन् नि॒रव॑दये नि॒रव॑दये॒ सन् थ्सन् नि॒रव॑दये॒ तत् तन् नि॒रव॑दये॒ सन् थ्सन् नि॒रव॑दये॒ तत् । \newline
11. नि॒रव॑दये॒ तत् तन् नि॒रव॑दये नि॒रव॑दये॒ तदे॒त दे॒तत् तन् नि॒रव॑दये नि॒रव॑दये॒ तदे॒तत् । \newline
12. नि॒रव॑दय॒ इति॑ निः - अव॑दये । \newline
13. तदे॒त दे॒तत् तत् तदे॒तत् तत् तदे॒तत् तत् तदे॒तत् तत् । \newline
14. ए॒तत् तत् तदे॒त दे॒तत् तद॑ग्ने ऽग्ने॒ तदे॒त दे॒तत् तद॑ग्ने । \newline
15. तद॑ग्ने ऽग्ने॒ तत् तद॑ग्ने अनृ॒णो अ॑नृ॒णो᳚ ऽग्ने॒ तत् तद॑ग्ने अनृ॒णः । \newline
16. अ॒ग्ने॒ अ॒नृ॒णो अ॑नृ॒णो᳚ ऽग्ने ऽग्ने अनृ॒णो भ॑वामि भवा म्यनृ॒णो᳚ ऽग्ने ऽग्ने अनृ॒णो भ॑वामि । \newline
17. अ॒नृ॒णो भ॑वामि भवा म्यनृ॒णो अ॑नृ॒णो भ॑वामि । \newline
18. भ॒वा॒मीति॑ भवामि । \newline
19. विश्व॑लोप विश्वदा॒वस्य॑ विश्वदा॒वस्य॒ विश्व॑लोप॒ विश्व॑लोप विश्वदा॒वस्य॑ त्वा त्वा विश्वदा॒वस्य॒ विश्व॑लोप॒ विश्व॑लोप विश्वदा॒वस्य॑ त्वा । \newline
20. विश्व॑लो॒पेति॒ विश्व॑ - लो॒प॒ । \newline
21. वि॒श्व॒दा॒वस्य॑ त्वा त्वा विश्वदा॒वस्य॑ विश्वदा॒वस्य॑ त्वा॒ ऽऽसन्, ना॒सन् त्वा॑ विश्वदा॒वस्य॑ विश्वदा॒वस्य॑ त्वा॒ ऽऽसन्न् । \newline
22. वि॒श्व॒दा॒वस्येति॑ विश्व - दा॒वस्य॑ । \newline
23. त्वा॒ ऽऽसन्, ना॒सन् त्वा᳚ त्वा॒ ऽऽसन् जु॑होमि जुहो म्या॒सन् त्वा᳚ त्वा॒ ऽऽसन् जु॑होमि । \newline
24. आ॒सन् जु॑होमि जुहो म्या॒सन्, ना॒सन् जु॑हो म्य॒ग्धा द॒ग्धाज् जु॑हो म्या॒सन्, ना॒सन् जु॑हो म्य॒ग्धात् । \newline
25. जु॒हो॒ म्य॒ग्धा द॒ग्धाज् जु॑होमि जुहो म्य॒ग्धा देक॒ एको॒ ऽग्धाज् जु॑होमि जुहो म्य॒ग्धा देकः॑ । \newline
26. अ॒ग्धा देक॒ एको॒ ऽग्धा द॒ग्धा देको॑ ऽहु॒ता द॑हु॒ता देको॒ ऽग्धा द॒ग्धा देको॑ ऽहु॒तात् । \newline
27. अ॒ग्धादित्य॑ग्ध - अत् । \newline
28. एको॑ ऽहु॒ता द॑हु॒ता देक॒ एको॑ ऽहु॒ता देक॒ एको॑ ऽहु॒ता देक॒ एको॑ ऽहु॒ता देकः॑ । \newline
29. अ॒हु॒ता देक॒ एको॑ ऽहु॒ता द॑हु॒ता देकः॑ समस॒नाथ् स॑मस॒ना देको॑ ऽहु॒ता द॑हु॒ता देकः॑ समस॒नात् । \newline
30. अ॒हु॒तादित्य॑हुत - अत् । \newline
31. एकः॑ समस॒नाथ् स॑मस॒ना देक॒ एकः॑ समस॒ना देक॒ एकः॑ समस॒ना देक॒ एकः॑ समस॒ना देकः॑ । \newline
32. स॒म॒स॒ना देक॒ एकः॑ समस॒नाथ् स॑मस॒ना देकः॑ । \newline
33. स॒म॒स॒नादिति॑ समसन - अत् । \newline
34. एक॒ इत्येकः॑ । \newline
35. ते नो॑ न॒स्ते ते नः॑ कृण्वन्तु कृण्वन्तु न॒स्ते ते नः॑ कृण्वन्तु । \newline
36. नः॒ कृ॒ण्व॒न्तु॒ कृ॒ण्व॒न्तु॒ नो॒ नः॒ कृ॒ण्व॒न्तु॒ भे॒ष॒जम् भे॑ष॒जम् कृ॑ण्वन्तु नो नः कृण्वन्तु भेष॒जम् । \newline
37. कृ॒ण्व॒न्तु॒ भे॒ष॒जम् भे॑ष॒जम् कृ॑ण्वन्तु कृण्वन्तु भेष॒जꣳ सदः॒ सदो॑ भेष॒जम् कृ॑ण्वन्तु कृण्वन्तु भेष॒जꣳ सदः॑ । \newline
38. भे॒ष॒जꣳ सदः॒ सदो॑ भेष॒जम् भे॑ष॒जꣳ सदः॒ सहः॒ सहः॒ सदो॑ भेष॒जम् भे॑ष॒जꣳ सदः॒ सहः॑ । \newline
39. सदः॒ सहः॒ सहः॒ सदः॒ सदः॒ सहो॒ वरे᳚ण्य॒म् ॅवरे᳚ण्यꣳ॒॒ सहः॒ सदः॒ सदः॒ सहो॒ वरे᳚ण्यम् । \newline
40. सहो॒ वरे᳚ण्य॒म् ॅवरे᳚ण्यꣳ॒॒ सहः॒ सहो॒ वरे᳚ण्यम् । \newline
41. वरे᳚ण्य॒मिति॒ वरे᳚ण्यम् । \newline
42. अ॒यम् नो॑ नो॒ ऽय म॒यम् नो॒ नभ॑सा॒ नभ॑सा नो॒ ऽय म॒यम् नो॒ नभ॑सा । \newline
43. नो॒ नभ॑सा॒ नभ॑सा नो नो॒ नभ॑सा पु॒रः पु॒रो नभ॑सा नो नो॒ नभ॑सा पु॒रः । \newline
44. नभ॑सा पु॒रः पु॒रो नभ॑सा॒ नभ॑सा पु॒रः सꣳ॒॒स्फानः॑ सꣳ॒॒स्फानः॑ पु॒रो नभ॑सा॒ नभ॑सा पु॒रः सꣳ॒॒स्फानः॑ । \newline
45. पु॒रः सꣳ॒॒स्फानः॑ सꣳ॒॒स्फानः॑ पु॒रः पु॒रः सꣳ॒॒स्फानो॑ अ॒भ्य॑भि सꣳ॒॒स्फानः॑ पु॒रः पु॒रः सꣳ॒॒स्फानो॑ अ॒भि । \newline
46. सꣳ॒॒स्फानो॑ अ॒भ्य॑भि सꣳ॒॒स्फानः॑ सꣳ॒॒स्फानो॑ अ॒भि र॑क्षतु रक्षत्व॒भि सꣳ॒॒स्फानः॑ सꣳ॒॒स्फानो॑ अ॒भि र॑क्षतु । \newline
47. सꣳ॒॒स्फान॒ इति॑ सम् - स्फानः॑ । \newline
48. अ॒भि र॑क्षतु रक्ष त्व॒भ्य॑भि र॑क्षतु । \newline
49. र॒क्ष॒त्विति॑ रक्षतु । \newline
50. गृ॒हाणा॒ मस॑मर्त्या॒ अस॑मर्त्यै गृ॒हाणा᳚म् गृ॒हाणा॒ मस॑मर्त्यै ब॒हवो॑ ब॒हवो ऽस॑मर्त्यै गृ॒हाणा᳚म् गृ॒हाणा॒ मस॑मर्त्यै ब॒हवः॑ । \newline
51. अस॑मर्त्यै ब॒हवो॑ ब॒हवो ऽस॑मर्त्या॒ अस॑मर्त्यै ब॒हवो॑ नो नो ब॒हवो ऽस॑मर्त्या॒ अस॑मर्त्यै ब॒हवो॑ नः । \newline
52. अस॑मर्त्या॒ इत्यस᳚म् - ऋ॒त्यै॒ । \newline
53. ब॒हवो॑ नो नो ब॒हवो॑ ब॒हवो॑ नो गृ॒हा गृ॒हा नो॑ ब॒हवो॑ ब॒हवो॑ नो गृ॒हाः । \newline
54. नो॒ गृ॒हा गृ॒हा नो॑ नो गृ॒हा अ॑सन्, नसन् गृ॒हा नो॑ नो गृ॒हा अ॑सन्न् । \newline
55. गृ॒हा अ॑सन्, नसन् गृ॒हा गृ॒हा अ॑सन्न् । \newline
56. अ॒स॒न्नित्य॑सन्न् । \newline
57. स त्वम् त्वꣳ स स त्वम् नो॑ न॒ स्त्वꣳ स स त्वम् नः॑ । \newline
58. त्वन्नो॑ न॒ स्त्वम् त्वम् नो॑ नभसो नभसो न॒ स्त्वम् त्वम् नो॑ नभसः । \newline
59. नो॒ न॒भ॒सो॒ न॒भ॒सो॒ नो॒ नो॒ न॒भ॒स॒ स्प॒ते॒ प॒ते॒ न॒भ॒सो॒ नो॒ नो॒ न॒भ॒स॒ स्प॒ते॒ । \newline
\pagebreak
\markright{ TS 3.3.8.3  \hfill https://www.vedavms.in \hfill}

\section{ TS 3.3.8.3 }

\textbf{TS 3.3.8.3 } \newline
\textbf{Samhita Paata} \newline

नभसस्पत॒ ऊर्जं॑ नो धेहि भ॒द्रया᳚ । पुन॑र्नो न॒ष्टमा कृ॑धि॒ पुन॑र्नो र॒यिमा कृ॑धि ॥ देव॑ सꣳस्फान सहस्रपो॒षस्ये॑शिषे॒ स नो॑ रा॒स्वाऽज्या॑निꣳ रा॒यस्पोषꣳ॑ सु॒वीर्यꣳ॑ संॅवथ्स॒रीणाꣳ॑ स्व॒स्तिं ॥ अ॒ग्निर्वाव य॒म इ॒यं ॅय॒मी कुसी॑दं॒ ॅवा ए॒तद्-य॒मस्य॒ यज॑मान॒ आ द॑त्ते॒ यदोष॑धीभि॒र्वेदिꣳ॑ स्तृ॒णाति॒ यदनु॑पौष्य प्रया॒याद्ग्री॑वब॒द्धमे॑न - [  ] \newline

\textbf{Pada Paata} \newline

न॒भ॒सः॒ । प॒ते॒ । ऊर्ज᳚म् । नः॒ । धे॒हि॒ । भ॒द्रया᳚ ॥ पुनः॑ । नः॒ । न॒ष्टम् । एति॑ । कृ॒धि॒ । पुनः॑ । नः॒ । र॒यिम् । एति॑ । कृ॒धि॒ ॥ देव॑ । सꣳ॒॒स्फा॒नेति॑ सं - स्फा॒न॒ । स॒ह॒स्र॒पो॒षस्येति॑ सहस्र - पो॒षस्य॑ । ई॒शि॒षे॒ । सः । नः॒ । रा॒स्व॒ । अज्या॑निम् । रा॒यः । पोष᳚म् । सु॒वीर्य॒मिति॑ सु - वीर्य᳚म् । सं॒ॅव॒थ्स॒रीणा॒मिति॑ सं - व॒थ्स॒रीणा᳚म् । स्व॒स्तिम् ॥ अ॒ग्निः । वाव । य॒मः । इ॒यम् । य॒मी । कुसी॑दम् । वै । ए॒तत् । य॒मस्य॑ । यज॑मानः । एति॑ । द॒त्ते॒ । यत् । ओष॑धीभि॒रित्योष॑धि - भिः॒ । वेदि᳚म् । स्तृ॒णाति॑ । यत् । अनु॑पौ॒ष्येत्यनु॑प - ओ॒ष्य॒ । प्र॒या॒यादिति॑ प्र - या॒यात् । ग्री॒व॒ब॒द्धमिति॑ ग्रीव - ब॒द्धम् । ए॒न॒म् ।  \newline


\textbf{Krama Paata} \newline

न॒भ॒स॒स्प॒ते॒ । प॒त॒ ऊर्ज᳚म् । ऊर्ज॑म् नः । नो॒ धे॒हि॒ । धे॒हि॒ भ॒द्रया᳚ । भ॒द्रयेति॑ भ॒द्रया᳚ ॥ पुन॑र् नः । नो॒ न॒ष्टम् । न॒ष्टमा । आ कृ॑धि । कृ॒धि॒ पुनः॑ । पुन॑र् नः । नो॒ र॒यिम् । र॒यिमा । आ कृ॑धि । कृ॒धीति॑ कृधि ॥ देव॑ सꣳस्फान । सꣳ॒॒स्फा॒न॒ स॒ह॒स्र॒पो॒षस्य॑ । सꣳ॒॒स्फा॒नेति॑ सम् - स्फा॒न॒ । स॒ह॒स्र॒पो॒षस्ये॑शिषे । स॒ह॒स्र॒पो॒षस्येति॑ सहस्र - पो॒षस्य॑ । ई॒शि॒षे॒ सः । स नः॑ । नो॒ रा॒स्व॒ । रा॒स्वाज्या॑निम् । अज्या॑निꣳ रा॒यः । रा॒यस्पोष᳚म् । पोषꣳ॑ सु॒वीर्य᳚म् । सु॒वीर्यꣳ॑ सम्ॅवथ्स॒रीणा᳚म् । सु॒वीर्य॒मिति॑ सु - वीर्य᳚म् । स॒म्ॅव॒थ्स॒रीणाꣳ॑ स्व॒स्तिम् । स॒म्ॅव॒थ्स॒रीणा॒मिति॑ सम् - व॒थ्स॒रीणा᳚म् । स्व॒स्तिमिति॑ स्व॒स्तिम् ॥ अ॒ग्निर् वाव । वाव य॒मः । य॒म इ॒यम् । इ॒यं ॅय॒मी । य॒मी कुसी॑दम् । कुसी॑दं॒ ॅवै । वा ए॒तत् । ए॒तद् य॒मस्य॑ । य॒मस्य॒ यज॑मानः । यज॑मान॒ आ । आ द॑त्ते । द॒त्ते॒ यत् । यदोष॑धीभिः । ओष॑धीभि॒र् वेदि᳚म् । ओष॑धीभि॒रित्योष॑धि - भिः॒ । वेदिꣳ॑ स्तृ॒णाति॑ । स्तृ॒णाति॒ यत् । यदनु॑पौष्य । अनु॑पौष्य प्रया॒यात् । अनु॑पौ॒ष्येत्यनु॑प - ओ॒ष्य॒ । प्र॒या॒याद् ग्री॑वब॒द्धम् । प्र॒या॒यादिति॑ प्र - या॒यात् । ग्री॒व॒ब॒द्धमे॑नम् । ग्री॒व॒ब॒द्धमिति॑ ग्रीव - ब॒द्धम् । ए॒न॒म॒मुष्मिन्न्॑ \newline

\textbf{Jatai Paata} \newline

1. न॒भ॒स॒ स्प॒ते॒ प॒ते॒ न॒भ॒सो॒ न॒भ॒स॒ स्प॒ते॒ । \newline
2. प॒त॒ ऊर्ज॒ मूर्ज॑म् पते पत॒ ऊर्ज᳚म् । \newline
3. ऊर्ज॑म् नो न॒ ऊर्ज॒ मूर्ज॑म् नः । \newline
4. नो॒ धे॒हि॒ धे॒हि॒ नो॒ नो॒ धे॒हि॒ । \newline
5. धे॒हि॒ भ॒द्रया॑ भ॒द्रया॑ धेहि धेहि भ॒द्रया᳚ । \newline
6. भ॒द्रयेति॑ भ॒द्रया᳚ । \newline
7. पुन॑र् नो नः॒ पुनः॒ पुन॑र् नः । \newline
8. नो॒ न॒ष्टम् न॒ष्टम् नो॑ नो न॒ष्टम् । \newline
9. न॒ष्ट मा न॒ष्टम् न॒ष्ट मा । \newline
10. आ कृ॑धि कृ॒ध्या कृ॑धि । \newline
11. कृ॒धि॒ पुनः॒ पुन॑ स्कृधि कृधि॒ पुनः॑ । \newline
12. पुन॑र् नो नः॒ पुनः॒ पुन॑र् नः । \newline
13. नो॒ र॒यिꣳ र॒यिम् नो॑ नो र॒यिम् । \newline
14. र॒यि मा र॒यिꣳ र॒यि मा । \newline
15. आ कृ॑धि कृ॒ध्या कृ॑धि । \newline
16. कृ॒धीति॑ कृधि । \newline
17. देव॑ सꣳस्फान सꣳस्फान॒ देव॒ देव॑ सꣳस्फान । \newline
18. सꣳ॒॒स्फा॒न॒ स॒ह॒स्र॒पो॒षस्य॑ सहस्रपो॒षस्य॑ सꣳस्फान सꣳस्फान सहस्रपो॒षस्य॑ । \newline
19. सꣳ॒॒स्फा॒नेति॑ सम् - स्फा॒न॒ । \newline
20. स॒ह॒स्र॒पो॒ष स्ये॑शिष ईशिषे सहस्रपो॒षस्य॑ सहस्रपो॒ष स्ये॑शिषे । \newline
21. स॒ह॒स्र॒पो॒षस्येति॑ सहस्र - पो॒षस्य॑ । \newline
22. ई॒शि॒षे॒ स स ई॑शिष ईशिषे॒ सः । \newline
23. स नो॑ नः॒ स स नः॑ । \newline
24. नो॒ रा॒स्व॒ रा॒स्व॒ नो॒ नो॒ रा॒स्व॒ । \newline
25. रा॒स्वाज्या॑नि॒ मज्या॑निꣳ रास्व रा॒स्वाज्या॑निम् । \newline
26. अज्या॑निꣳ रा॒यो रा॒यो अज्या॑नि॒ मज्या॑निꣳ रा॒यः । \newline
27. रा॒य स्पोष॒म् पोषꣳ॑ रा॒यो रा॒य स्पोष᳚म् । \newline
28. पोषꣳ॑ सु॒वीर्यꣳ॑ सु॒वीर्य॒म् पोष॒म् पोषꣳ॑ सु॒वीर्य᳚म् । \newline
29. सु॒वीर्यꣳ॑ सम्ॅवथ्स॒रीणाꣳ॑ सम्ॅवथ्स॒रीणाꣳ॑ सु॒वीर्यꣳ॑ सु॒वीर्यꣳ॑ सम्ॅवथ्स॒रीणा᳚म् । \newline
30. सु॒वीर्य॒मिति॑ सु - वीर्य᳚म् । \newline
31. स॒म्ॅव॒थ्स॒रीणाꣳ॑ स्व॒स्तिꣳ स्व॒स्तिꣳ स॑म्ॅवथ्स॒रीणाꣳ॑ सम्ॅवथ्स॒रीणाꣳ॑ स्व॒स्तिम् । \newline
32. स॒म्ॅव॒थ्स॒रीणा॒मिति॑ सम् - व॒थ्स॒रीणा᳚म् । \newline
33. स्व॒स्तिमिति॑ स्व॒स्तिम् । \newline
34. अ॒ग्निर् वाव वावाग्नि र॒ग्निर् वाव । \newline
35. वाव य॒मो य॒मो वाव वाव य॒मः । \newline
36. य॒म इ॒य मि॒यम् ॅय॒मो य॒म इ॒यम् । \newline
37. इ॒यम् ॅय॒मी य॒मीय मि॒यम् ॅय॒मी । \newline
38. य॒मी कुसी॑द॒म् कुसी॑दम् ॅय॒मी य॒मी कुसी॑दम् । \newline
39. कुसी॑द॒म् ॅवै वै कुसी॑द॒म् कुसी॑द॒म् ॅवै । \newline
40. वा ए॒त दे॒तद् वै वा ए॒तत् । \newline
41. ए॒तद् य॒मस्य॑ य॒म स्यै॒त दे॒तद् य॒मस्य॑ । \newline
42. य॒मस्य॒ यज॑मानो॒ यज॑मानो य॒मस्य॑ य॒मस्य॒ यज॑मानः । \newline
43. यज॑मान॒ आ यज॑मानो॒ यज॑मान॒ आ । \newline
44. आ द॑त्ते दत्त॒ आ द॑त्ते । \newline
45. द॒त्ते॒ यद् यद् द॑त्ते दत्ते॒ यत् । \newline
46. यदोष॑धीभि॒ रोष॑धीभि॒र् यद् यदोष॑धीभिः । \newline
47. ओष॑धीभि॒र् वेदि॒म् ॅवेदि॒ मोष॑धीभि॒ रोष॑धीभि॒र् वेदि᳚म् । \newline
48. ओष॑धीभि॒रित्योष॑धि - भिः॒ । \newline
49. वेदिꣳ॑ स्तृ॒णाति॑ स्तृ॒णाति॒ वेदि॒म् ॅवेदिꣳ॑ स्तृ॒णाति॑ । \newline
50. स्तृ॒णाति॒ यद् यथ् स्तृ॒णाति॑ स्तृ॒णाति॒ यत् । \newline
51. यदनु॑पौ॒ ष्यानु॑पौ॒ष्य यद् यदनु॑पौ॒ष्य । \newline
52. अनु॑पौ॒ष्य प्र॑या॒यात् प्र॑या॒या दनु॑पौ॒ ष्यानु॑पौ॒ष्य प्र॑या॒यात् । \newline
53. अनु॑पौ॒ष्येत्यनु॑प - ओ॒ष्य॒ । \newline
54. प्र॒या॒याद् ग्री॑वब॒द्धम् ग्री॑वब॒द्धम् प्र॑या॒यात् प्र॑या॒याद् ग्री॑वब॒द्धम् । \newline
55. प्र॒या॒यादिति॑ प्र - या॒यात् । \newline
56. ग्री॒व॒ब॒द्ध मे॑न मेनम् ग्रीवब॒द्धम् ग्री॑वब॒द्ध मे॑नम् । \newline
57. ग्री॒व॒ब॒द्धमिति॑ ग्रीव - ब॒द्धम् । \newline
58. ए॒न॒ म॒मुष्मि॑न् न॒मुष्मि॑न् नेन मेन म॒मुष्मिन्न्॑ । \newline

\textbf{Ghana Paata } \newline

1. न॒भ॒स॒ स्प॒ते॒ प॒ते॒ न॒भ॒सो॒ न॒भ॒स॒ स्प॒त॒ ऊर्ज॒ मूर्ज॑म् पते नभसो नभस स्पत॒ ऊर्ज᳚म् । \newline
2. प॒त॒ ऊर्ज॒ मूर्ज॑म् पते पत॒ ऊर्ज॑म् नो न॒ ऊर्ज॑म् पते पत॒ ऊर्ज॑म् नः । \newline
3. ऊर्ज॑म् नो न॒ ऊर्ज॒ मूर्ज॑म् नो धेहि धेहि न॒ ऊर्ज॒ मूर्ज॑म् नो धेहि । \newline
4. नो॒ धे॒हि॒ धे॒हि॒ नो॒ नो॒ धे॒हि॒ भ॒द्रया॑ भ॒द्रया॑ धेहि नो नो धेहि भ॒द्रया᳚ । \newline
5. धे॒हि॒ भ॒द्रया॑ भ॒द्रया॑ धेहि धेहि भ॒द्रया᳚ । \newline
6. भ॒द्रयेति॑ भ॒द्रया᳚ । \newline
7. पुन॑र् नो नः॒ पुनः॒ पुन॑र् नो न॒ष्टम् न॒ष्टम् नः॒ पुनः॒ पुन॑र् नो न॒ष्टम् । \newline
8. नो॒ न॒ष्टम् न॒ष्टम् नो॑ नो न॒ष्ट मा न॒ष्टम् नो॑ नो न॒ष्ट मा । \newline
9. न॒ष्ट मा न॒ष्टम् न॒ष्ट मा कृ॑धि कृ॒ध्या न॒ष्टम् न॒ष्ट मा कृ॑धि । \newline
10. आ कृ॑धि कृ॒ध्या कृ॑धि॒ पुनः॒ पुन॑ स्कृ॒ध्या कृ॑धि॒ पुनः॑ । \newline
11. कृ॒धि॒ पुनः॒ पुन॑ स्कृधि कृधि॒ पुन॑र् नो नः॒ पुन॑ स्कृधि कृधि॒ पुन॑र् नः । \newline
12. पुन॑र् नो नः॒ पुनः॒ पुन॑र् नो र॒यिꣳ र॒यिम् नः॒ पुनः॒ पुन॑र् नो र॒यिम् । \newline
13. नो॒ र॒यिꣳ र॒यिम् नो॑ नो र॒यि मा र॒यिम् नो॑ नो र॒यि मा । \newline
14. र॒यि मा र॒यिꣳ र॒यि मा कृ॑धि कृ॒ध्या र॒यिꣳ र॒यि मा कृ॑धि । \newline
15. आ कृ॑धि कृ॒ध्या कृ॑धि । \newline
16. कृ॒धीति॑ कृधि । \newline
17. देव॑ सꣳस्फान सꣳस्फान॒ देव॒ देव॑ सꣳस्फान सहस्रपो॒षस्य॑ सहस्रपो॒षस्य॑ सꣳस्फान॒ देव॒ देव॑ सꣳस्फान सहस्रपो॒षस्य॑ । \newline
18. सꣳ॒॒स्फा॒न॒ स॒ह॒स्र॒पो॒षस्य॑ सहस्रपो॒षस्य॑ सꣳस्फान सꣳस्फान सहस्रपो॒ष स्ये॑शिष ईशिषे सहस्रपो॒षस्य॑ सꣳस्फान सꣳस्फान सहस्रपो॒ष स्ये॑शिषे । \newline
19. सꣳ॒॒स्फा॒नेति॑ सम् - स्फा॒न॒ । \newline
20. स॒ह॒स्र॒पो॒ष स्ये॑शिष ईशिषे सहस्रपो॒षस्य॑ सहस्रपो॒ष स्ये॑शिषे॒ स स ई॑शिषे सहस्रपो॒षस्य॑ सहस्रपो॒ष स्ये॑शिषे॒ सः । \newline
21. स॒ह॒स्र॒पो॒षस्येति॑ सहस्र - पो॒षस्य॑ । \newline
22. ई॒शि॒षे॒ स स ई॑शिष ईशिषे॒ स नो॑ नः॒ स ई॑शिष ईशिषे॒ स नः॑ । \newline
23. स नो॑ नः॒ स स नो॑ रास्व रास्व नः॒ स स नो॑ रास्व । \newline
24. नो॒ रा॒स्व॒ रा॒स्व॒ नो॒ नो॒ रा॒स्वाज्या॑नि॒ मज्या॑निꣳ रास्व नो नो रा॒स्वाज्या॑निम् । \newline
25. रा॒स्वाज्या॑नि॒ मज्या॑निꣳ रास्व रा॒स्वाज्या॑निꣳ रा॒यो रा॒यो अज्या॑निꣳ रास्व रा॒स्वाज्या॑निꣳ रा॒यः । \newline
26. अज्या॑निꣳ रा॒यो रा॒यो अज्या॑नि॒ मज्या॑निꣳ रा॒य स्पोष॒म् पोषꣳ॑ रा॒यो अज्या॑नि॒ मज्या॑निꣳ रा॒य स्पोष᳚म् । \newline
27. रा॒य स्पोष॒म् पोषꣳ॑ रा॒यो रा॒य स्पोषꣳ॑ सु॒वीर्यꣳ॑ सु॒वीर्य॒म् पोषꣳ॑ रा॒यो रा॒य स्पोषꣳ॑ सु॒वीर्य᳚म् । \newline
28. पोषꣳ॑ सु॒वीर्यꣳ॑ सु॒वीर्य॒म् पोष॒म् पोषꣳ॑ सु॒वीर्यꣳ॑ सम्ॅवथ्स॒रीणाꣳ॑ सम्ॅवथ्स॒रीणाꣳ॑ सु॒वीर्य॒म् पोष॒म् पोषꣳ॑ सु॒वीर्यꣳ॑ सम्ॅवथ्स॒रीणा᳚म् । \newline
29. सु॒वीर्यꣳ॑ सम्ॅवथ्स॒रीणाꣳ॑ सम्ॅवथ्स॒रीणाꣳ॑ सु॒वीर्यꣳ॑ सु॒वीर्यꣳ॑ सम्ॅवथ्स॒रीणाꣳ॑ स्व॒स्तिꣳ स्व॒स्तिꣳ स॑म्ॅवथ्स॒रीणाꣳ॑ सु॒वीर्यꣳ॑ सु॒वीर्यꣳ॑ सम्ॅवथ्स॒रीणाꣳ॑ स्व॒स्तिम् । \newline
30. सु॒वीर्य॒मिति॑ सु - वीर्य᳚म् । \newline
31. स॒म्ॅव॒थ्स॒रीणाꣳ॑ स्व॒स्तिꣳ स्व॒स्तिꣳ स॑म्ॅवथ्स॒रीणाꣳ॑ सम्ॅवथ्स॒रीणाꣳ॑ स्व॒स्तिम् । \newline
32. स॒म्ॅव॒थ्स॒रीणा॒मिति॑ सम् - व॒थ्स॒रीणा᳚म् । \newline
33. स्व॒स्तिमिति॑ स्व॒स्तिम् । \newline
34. अ॒ग्निर् वाव वावा ग्नि र॒ग्निर् वाव य॒मो य॒मो वावा ग्नि र॒ग्निर् वाव य॒मः । \newline
35. वाव य॒मो य॒मो वाव वाव य॒म इ॒य मि॒यम् ॅय॒मो वाव वाव य॒म इ॒यम् । \newline
36. य॒म इ॒य मि॒यम् ॅय॒मो य॒म इ॒यम् ॅय॒मी य॒मीयम् ॅय॒मो य॒म इ॒यम् ॅय॒मी । \newline
37. इ॒यम् ॅय॒मी य॒मीय मि॒यम् ॅय॒मी कुसी॑द॒म् कुसी॑दम् ॅय॒मीय मि॒यम् ॅय॒मी कुसी॑दम् । \newline
38. य॒मी कुसी॑द॒म् कुसी॑दम् ॅय॒मी य॒मी कुसी॑द॒म् ॅवै वै कुसी॑दम् ॅय॒मी य॒मी कुसी॑द॒म् ॅवै । \newline
39. कुसी॑द॒म् ॅवै वै कुसी॑द॒म् कुसी॑द॒म् ॅवा ए॒त दे॒तद् वै कुसी॑द॒म् कुसी॑द॒म् ॅवा ए॒तत् । \newline
40. वा ए॒त दे॒तद् वै वा ए॒तद् य॒मस्य॑ य॒म स्यै॒तद् वै वा ए॒तद् य॒मस्य॑ । \newline
41. ए॒तद् य॒मस्य॑ य॒म स्यै॒त दे॒तद् य॒मस्य॒ यज॑मानो॒ यज॑मानो य॒म स्यै॒त दे॒तद् य॒मस्य॒ यज॑मानः । \newline
42. य॒मस्य॒ यज॑मानो॒ यज॑मानो य॒मस्य॑ य॒मस्य॒ यज॑मान॒ आ यज॑मानो य॒मस्य॑ य॒मस्य॒ यज॑मान॒ आ । \newline
43. यज॑मान॒ आ यज॑मानो॒ यज॑मान॒ आ द॑त्ते दत्त॒ आ यज॑मानो॒ यज॑मान॒ आ द॑त्ते । \newline
44. आ द॑त्ते दत्त॒ आ द॑त्ते॒ यद् यद् द॑त्त॒ आ द॑त्ते॒ यत् । \newline
45. द॒त्ते॒ यद् यद् द॑त्ते दत्ते॒ यदोष॑धीभि॒ रोष॑धीभि॒र् यद् द॑त्ते दत्ते॒ यदोष॑धीभिः । \newline
46. यदोष॑धीभि॒ रोष॑धीभि॒र् यद् यदोष॑धीभि॒र् वेदि॒म् ॅवेदि॒ मोष॑धीभि॒र् यद् यदोष॑धीभि॒र् वेदि᳚म् । \newline
47. ओष॑धीभि॒र् वेदि॒म् ॅवेदि॒ मोष॑धीभि॒ रोष॑धीभि॒र् वेदिꣳ॑ स्तृ॒णाति॑ स्तृ॒णाति॒ वेदि॒ मोष॑धीभि॒ रोष॑धीभि॒र् वेदिꣳ॑ स्तृ॒णाति॑ । \newline
48. ओष॑धीभि॒रित्योष॑धि - भिः॒ । \newline
49. वेदिꣳ॑ स्तृ॒णाति॑ स्तृ॒णाति॒ वेदि॒म् ॅवेदिꣳ॑ स्तृ॒णाति॒ यद् यथ् स्तृ॒णाति॒ वेदि॒म् ॅवेदिꣳ॑ स्तृ॒णाति॒ यत् । \newline
50. स्तृ॒णाति॒ यद् यथ् स्तृ॒णाति॑ स्तृ॒णाति॒ यदनु॑पौ॒ष्या नु॑पौ॒ष्य यथ् स्तृ॒णाति॑ स्तृ॒णाति॒ यदनु॑पौ॒ष्य । \newline
51. यदनु॑पौ॒ष्या नु॑पौ॒ष्य यद् यदनु॑पौ॒ष्य प्र॑या॒यात् प्र॑या॒या दनु॑पौ॒ष्य यद् यदनु॑पौ॒ष्य प्र॑या॒यात् । \newline
52. अनु॑पौ॒ष्य प्र॑या॒यात् प्र॑या॒या दनु॑पौ॒ष्या नु॑पौ॒ष्य प्र॑या॒याद् ग्री॑वब॒द्धम् ग्री॑वब॒द्धम् प्र॑या॒या दनु॑पौ॒ष्या नु॑पौ॒ष्य प्र॑या॒याद् ग्री॑वब॒द्धम् । \newline
53. अनु॑पौ॒ष्येत्यनु॑प - ओ॒ष्य॒ । \newline
54. प्र॒या॒याद् ग्री॑वब॒द्धम् ग्री॑वब॒द्धम् प्र॑या॒यात् प्र॑या॒याद् ग्री॑वब॒द्ध मे॑न मेनम् ग्रीवब॒द्धम् प्र॑या॒यात् प्र॑या॒याद् ग्री॑वब॒द्ध मे॑नम् । \newline
55. प्र॒या॒यादिति॑ प्र - या॒यात् । \newline
56. ग्री॒व॒ब॒द्ध मे॑न मेनम् ग्रीवब॒द्धम् ग्री॑वब॒द्ध मे॑न म॒मुष्मि॑न्, न॒मुष्मि॑न्, नेनम् ग्रीवब॒द्धम् ग्री॑वब॒द्ध मे॑न म॒मुष्मिन्न्॑ । \newline
57. ग्री॒व॒ब॒द्धमिति॑ ग्रीव - ब॒द्धम् । \newline
58. ए॒न॒ म॒मुष्मि॑न्, न॒मुष्मि॑न्, नेन मेन म॒मुष्मि॑न् ॅलो॒के लो॒के॑ ऽमुष्मि॑न्, नेन मेन म॒मुष्मि॑न् ॅलो॒के । \newline
\pagebreak
\markright{ TS 3.3.8.4  \hfill https://www.vedavms.in \hfill}

\section{ TS 3.3.8.4 }

\textbf{TS 3.3.8.4 } \newline
\textbf{Samhita Paata} \newline

म॒मुष्मि॑न् ॅलो॒के ने॑नीयेर॒न्॒. यत् कुसी॑द॒मप्र॑तीत्तं॒ मयीत्युपौ॑षती॒हैव सन्. य॒मं कुसी॑दं निरव॒दाया॑नृ॒णः सु॑व॒र्गं ॅलो॒कमे॑ति॒यदि॑ मि॒श्रमि॑व॒ चरे॑दञ्ज॒लिना॒ सक्तू᳚न् प्रदा॒व्ये॑ जुहुयादे॒ष वा अ॒ग्निर्वै᳚श्वान॒रो यत् प्र॑दा॒व्यः॑ स ए॒वैनꣳ॑स्वदय॒त्यह्नां᳚ ॅवि॒धान्या॑-मेकाष्ट॒काया॑मपू॒पं चतुः॑ शरावं प॒क्त्वा प्रा॒तरे॒तेन॒ कक्ष॒मुपौ॑षे॒द्यदि॒ - [  ] \newline

\textbf{Pada Paata} \newline

अ॒मुष्मिन्न्॑ । लो॒के । ने॒नी॒ये॒र॒न्न् । यत् । कुसी॑दम् । अप्र॑तीत्त॒मित्यप्र॑ति - इ॒त्त॒म् । मयि॑ । इति॑ । उपेति॑ । ओ॒ष॒ति॒ । इ॒ह । ए॒व । सन्न् । य॒मम् । कुसी॑दम् । नि॒र॒व॒दायेति॑ निः - अ॒व॒दाय॑ । अ॒नृ॒णः । सु॒व॒र्गमिति॑ सुवः-गम् । लो॒कम् । ए॒ति॒ । यदि॑ । मि॒श्रम् । इ॒व॒ । चरे᳚त् । अ॒ञ्ज॒लिना᳚ । सक्तून्॑ । प्र॒दा॒व्य॑ इति॑ प्र - दा॒व्ये᳚ । जु॒हु॒या॒त् । ए॒षः । वै । अ॒ग्निः । वै॒श्वा॒न॒रः । यत् । प्र॒दा॒व्य॑ इति॑ प्र - दा॒व्यः॑ । सः । ए॒व । ए॒न॒म् । स्व॒द॒य॒ति॒ । अह्ना᳚म् । वि॒धान्या॒मिति॑ वि - धान्या᳚म् । ऐ॒का॒ष्ट॒काया॒मित्ये॑क - अ॒ष्ट॒काया᳚म् । अ॒पू॒पम् । चतुः॑ शराव॒मिति॒ चतुः॑ - श॒रा॒व॒म् । प॒क्त्वा । प्रा॒तः । ए॒तेन॑ । कक्ष᳚म् । उपेति॑ । ओ॒षे॒त् । यदि॑ ।  \newline


\textbf{Krama Paata} \newline

अ॒मुष्मि॑न् ॅलो॒के । लो॒के ने॑नीयेरन्न् । ने॒नी॒ये॒र॒न्॒. यत् । यत् कुसी॑दम् । कुसी॑द॒मप्र॑तीत्तम् । अप्र॑तीत्त॒म् मयि॑ । अप्र॑तीत्त॒मित्यप्र॑ति - इ॒त्त॒म् । मयीति॑ । इत्युप॑ । उपौ॑षति । ओ॒ष॒ती॒ह । इ॒हैव । ए॒व सन्न् । सन्. य॒मम् । य॒मम् कुसी॑दम् । कुसी॑दम् निरव॒दाय॑ । नि॒र॒व॒दाया॑नृ॒णः । नि॒र॒व॒दायेति॑ निः - अ॒व॒दाय॑ । अ॒नृ॒णः सु॑व॒र्गम् । सु॒व॒र्गं ॅलो॒कम् । सु॒व॒र्गमिति॑ सुवः - गम् । लो॒कमे॑ति । ए॒ति॒ यदि॑ । यदि॑ मि॒श्रम् । मि॒श्रमि॑व । इ॒व॒ चरे᳚त् । चरे॑दञ्ज॒लिना᳚ । अ॒ञ्ज॒लिना॒ सक्तून्॑ । सक्तू᳚न् प्रदा॒व्ये᳚ । प्र॒दा॒व्ये॑ जुहुयात् । प्र॒दा॒व्य॑ इति॑ प्र - दा॒व्ये᳚ । जु॒हु॒या॒दे॒षः । ए॒ष वै । वा अ॒ग्निः । अ॒ग्निर् वै᳚श्वान॒रः । वै॒श्वा॒न॒रो यत् । यत् प्र॑दा॒व्यः॑ । प्र॒दा॒व्यः॑ सः । प्र॒दा॒व्य॑ इति॑ प्र - दा॒व्यः॑ । स ए॒व । ए॒वैन᳚म् । ए॒नꣳ॒॒ स्व॒द॒य॒ति॒ । स्व॒द॒य॒त्यह्ना᳚म् । अह्नां᳚ ॅवि॒धान्या᳚म् । वि॒धान्या॑मेकाष्ट॒काया᳚म् । वि॒धान्या॒मिति॑ वि - धान्या᳚म् । ए॒का॒ष्ट॒काया॑मपू॒पम् । ए॒का॒ष्ट॒काया॒मित्ये॑क - अ॒ष्ट॒काया᳚म् । अ॒पू॒पम् चतु॑श्शरावम् । चतु॑श्शरावम् प॒क्त्वा । चतु॑श्शराव॒मिति॒ चतुः॑ - श॒रा॒व॒म् । प॒क्त्वा प्रा॒तः । प्रा॒तरे॒तेन॑ । ए॒तेन॒ कक्ष᳚म् । कक्ष॒मुप॑ । उपौ॑षेत् । ओ॒षे॒द् यदि॑ । यदि॒ दह॑ति \newline

\textbf{Jatai Paata} \newline

1. अ॒मुष्मि॑न् ॅलो॒के लो॒के॑ ऽमुष्मि॑न् न॒मुष्मि॑न् ॅलो॒के । \newline
2. लो॒के ने॑नीयेरन् नेनीयेरन् ॅलो॒के लो॒के ने॑नीयेरन्न् । \newline
3. ने॒नी॒ये॒र॒न्॒. यद् यन् ने॑नीयेरन् नेनीयेर॒न्॒. यत् । \newline
4. यत् कुसी॑द॒म् कुसी॑द॒म् ॅयद् यत् कुसी॑दम् । \newline
5. कुसी॑द॒ मप्र॑तीत्त॒ मप्र॑तीत्त॒म् कुसी॑द॒म् कुसी॑द॒ मप्र॑तीत्तम् । \newline
6. अप्र॑तीत्त॒म् मयि॒ मय्यप्र॑तीत्त॒ मप्र॑तीत्त॒म् मयि॑ । \newline
7. अप्र॑तीत्त॒मित्यप्र॑ति - इ॒त्त॒म् । \newline
8. मयीतीति॒ मयि॒ मयीति॑ । \newline
9. इत्युपोपे तीत्युप॑ । \newline
10. उपौ॑ष त्योष॒ त्युपो पौ॑षति । \newline
11. ओ॒ष॒ती॒हे हौष॑ त्योषती॒ह । \newline
12. इ॒हैवैवे हे हैव । \newline
13. ए॒व सन् थ्सन् ने॒वैव सन्न् । \newline
14. सन्. य॒मम् ॅय॒मꣳ सन् थ्सन्. य॒मम् । \newline
15. य॒मम् कुसी॑द॒म् कुसी॑दम् ॅय॒मम् ॅय॒मम् कुसी॑दम् । \newline
16. कुसी॑दम् निरव॒दाय॑ निरव॒दाय॒ कुसी॑द॒म् कुसी॑दम् निरव॒दाय॑ । \newline
17. नि॒र॒व॒दाया॑ नृ॒णो अ॑नृ॒णो नि॑रव॒दाय॑ निरव॒दाया॑ नृ॒णः । \newline
18. नि॒र॒व॒दायेति॑ निः - अ॒व॒दाय॑ । \newline
19. अ॒नृ॒णः सु॑व॒र्गꣳ सु॑व॒र्ग म॑नृ॒णो अ॑नृ॒णः सु॑व॒र्गम् । \newline
20. सु॒व॒र्गम् ॅलो॒कम् ॅलो॒कꣳ सु॑व॒र्गꣳ सु॑व॒र्गम् ॅलो॒कम् । \newline
21. सु॒व॒र्गमिति॑ सुवः - गम् । \newline
22. लो॒क मे᳚त्येति लो॒कम् ॅलो॒क मे॑ति । \newline
23. ए॒ति॒ यदि॒ यद्ये᳚ त्येति॒ यदि॑ । \newline
24. यदि॑ मि॒श्रम् मि॒श्रम् ॅयदि॒ यदि॑ मि॒श्रम् । \newline
25. मि॒श्र मि॑वे व मि॒श्रम् मि॒श्र मि॑व । \newline
26. इ॒व॒ चरे॒च् चरे॑ दिवे व॒ चरे᳚त् । \newline
27. चरे॑ दञ्ज॒लिना᳚ ऽञ्ज॒लिना॒ चरे॒च् चरे॑ दञ्ज॒लिना᳚ । \newline
28. अ॒ञ्ज॒लिना॒ सक्तू॒न् थ्सक्तू॑ नञ्ज॒लिना᳚ ऽञ्ज॒लिना॒ सक्तून्॑ । \newline
29. सक्तू᳚न् प्रदा॒व्ये᳚ प्रदा॒व्ये॑ सक्तू॒न् थ्सक्तू᳚न् प्रदा॒व्ये᳚ । \newline
30. प्र॒दा॒व्ये॑ जुहुयाज् जुहुयात् प्रदा॒व्ये᳚ प्रदा॒व्ये॑ जुहुयात् । \newline
31. प्र॒दा॒व्य॑ इति॑ प्र - दा॒व्ये᳚ । \newline
32. जु॒हु॒या॒ दे॒ष ए॒ष जु॑हुयाज् जुहुया दे॒षः । \newline
33. ए॒ष वै वा ए॒ष ए॒ष वै । \newline
34. वा अ॒ग्नि र॒ग्निर् वै वा अ॒ग्निः । \newline
35. अ॒ग्निर् वै᳚श्वान॒रो वै᳚श्वान॒रो᳚ ऽग्नि र॒ग्निर् वै᳚श्वान॒रः । \newline
36. वै॒श्वा॒न॒रो यद् यद् वै᳚श्वान॒रो वै᳚श्वान॒रो यत् । \newline
37. यत् प्र॑दा॒व्यः॑ प्रदा॒व्यो॑ यद् यत् प्र॑दा॒व्यः॑ । \newline
38. प्र॒दा॒व्यः॑ स स प्र॑दा॒व्यः॑ प्रदा॒व्यः॑ सः । \newline
39. प्र॒दा॒व्य॑ इति॑ प्र - दा॒व्यः॑ । \newline
40. स ए॒वैव स स ए॒व । \newline
41. ए॒वैन॑ मेन मे॒वैवैन᳚म् । \newline
42. ए॒नꣳ॒॒ स्व॒द॒य॒ति॒ स्व॒द॒य॒ त्ये॒न॒ मे॒नꣳ॒॒ स्व॒द॒य॒ति॒ । \newline
43. स्व॒द॒य॒ त्यह्ना॒ मह्नाꣳ॑ स्वदयति स्वदय॒ त्यह्ना᳚म् । \newline
44. अह्ना᳚म् ॅवि॒धान्या᳚म् ॅवि॒धान्या॒ मह्ना॒ मह्ना᳚म् ॅवि॒धान्या᳚म् । \newline
45. वि॒धान्या॑ मेकाष्ट॒काया॑ मेकाष्ट॒काया᳚म् ॅवि॒धान्या᳚म् ॅवि॒धान्या॑ मेकाष्ट॒काया᳚म् । \newline
46. वि॒धान्या॒मिति॑ वि - धान्या᳚म् । \newline
47. ए॒का॒ष्ट॒काया॑ मपू॒प म॑पू॒प मे॑काष्ट॒काया॑ मेकाष्ट॒काया॑ मपू॒पम् । \newline
48. ए॒का॒ष्ट॒काया॒मित्ये॑क - अ॒ष्ट॒काया᳚म् । \newline
49. अ॒पू॒पम् चतु॑श्शराव॒म् चतु॑श्शराव मपू॒प म॑पू॒पम् चतु॑श्शरावम् । \newline
50. चतु॑श्शरावम् प॒क्त्वा प॒क्त्वा चतु॑श्शराव॒म् चतु॑श्शरावम् प॒क्त्वा । \newline
51. चतु॑श्शराव॒मिति॒ चतुः॑ - श॒रा॒व॒म् । \newline
52. प॒क्त्वा प्रा॒तः प्रा॒तः प॒क्त्वा प॒क्त्वा प्रा॒तः । \newline
53. प्रा॒त रे॒ते नै॒तेन॑ प्रा॒तः प्रा॒त रे॒तेन॑ । \newline
54. ए॒तेन॒ कक्ष॒म् कक्ष॑ मे॒ते नै॒तेन॒ कक्ष᳚म् । \newline
55. कक्ष॒ मुपोप॒ कक्ष॒म् कक्ष॒ मुप॑ । \newline
56. उपौ॑षे दोषे॒ दुपोपौ॑षेत् । \newline
57. ओ॒षे॒द् यदि॒ यद्यो॑षे दोषे॒द् यदि॑ । \newline
58. यदि॒ दह॑ति॒ दह॑ति॒ यदि॒ यदि॒ दह॑ति । \newline

\textbf{Ghana Paata } \newline

1. अ॒मुष्मि॑न् ॅलो॒के लो॒के॑ ऽमुष्मि॑न्, न॒मुष्मि॑न् ॅलो॒के ने॑नीयेरन् नेनीयेरन् ॅलो॒के॑ ऽमुष्मि॑न्, न॒मुष्मि॑न् ॅलो॒के ने॑नीयेरन्न् । \newline
2. लो॒के ने॑नीयेरन् नेनीयेरन् ॅलो॒के लो॒के ने॑नीयेर॒न्॒. यद् यन् ने॑नीयेरन् ॅलो॒के लो॒के ने॑नीयेर॒न्॒. यत् । \newline
3. ने॒नी॒ये॒र॒न्॒. यद् यन् ने॑नीयेरन् नेनीयेर॒न्॒. यत् कुसी॑द॒म् कुसी॑द॒म् ॅयन् ने॑नीयेरन् नेनीयेर॒न्॒. यत् कुसी॑दम् । \newline
4. यत् कुसी॑द॒म् कुसी॑द॒म् ॅयद् यत् कुसी॑द॒ मप्र॑तीत्त॒ मप्र॑तीत्त॒म् कुसी॑द॒म् ॅयद् यत् कुसी॑द॒ मप्र॑तीत्तम् । \newline
5. कुसी॑द॒ मप्र॑तीत्त॒ मप्र॑तीत्त॒म् कुसी॑द॒म् कुसी॑द॒ मप्र॑तीत्त॒म् मयि॒ मय्यप्र॑तीत्त॒म् कुसी॑द॒म् कुसी॑द॒ मप्र॑तीत्त॒म् मयि॑ । \newline
6. अप्र॑तीत्त॒म् मयि॒ मय्यप्र॑तीत्त॒ मप्र॑तीत्त॒म् मयीतीति॒ मय्यप्र॑तीत्त॒ मप्र॑तीत्त॒म् मयीति॑ । \newline
7. अप्र॑तीत्त॒मित्यप्र॑ति - इ॒त्त॒म् । \newline
8. मयीतीति॒ मयि॒ मयी त्युपोपे ति॒ मयि॒ मयी त्युप॑ । \newline
9. इत्युपोपे तीत्युपौ॑ष त्योष॒ त्युपे ती त्युपौ॑षति । \newline
10. उपौ॑ष त्योष॒ त्युपोपौ॑ष ती॒हे हौष॒ त्युपोपौ॑ष ती॒ह । \newline
11. ओ॒ष॒ती॒हे हौष॑ त्योषती॒ हैवैवे हौष॑ त्योषती॒ हैव । \newline
12. इ॒हैवैवे हे हैव सन् थ्सन्, ने॒वे हे हैव सन्न् । \newline
13. ए॒व सन् थ्सन्, ने॒वैव सन्. य॒मम् ॅय॒मꣳ सन्, ने॒वैव सन्. य॒मम् । \newline
14. सन्. य॒मम् ॅय॒मꣳ सन् थ्सन्. य॒मम् कुसी॑द॒म् कुसी॑दम् ॅय॒मꣳ सन् थ्सन्. य॒मम् कुसी॑दम् । \newline
15. य॒मम् कुसी॑द॒म् कुसी॑दम् ॅय॒मम् ॅय॒मम् कुसी॑दम् निरव॒दाय॑ निरव॒दाय॒ कुसी॑दम् ॅय॒मम् ॅय॒मम् कुसी॑दम् निरव॒दाय॑ । \newline
16. कुसी॑दम् निरव॒दाय॑ निरव॒दाय॒ कुसी॑द॒म् कुसी॑दम् निरव॒दाया॑ नृ॒णो अ॑नृ॒णो नि॑रव॒दाय॒ कुसी॑द॒म् कुसी॑दम् निरव॒दाया॑ नृ॒णः । \newline
17. नि॒र॒व॒दाया॑ नृ॒णो अ॑नृ॒णो नि॑रव॒दाय॑ निरव॒दाया॑ नृ॒णः सु॑व॒र्गꣳ सु॑व॒र्ग म॑नृ॒णो नि॑रव॒दाय॑ निरव॒दाया॑ नृ॒णः सु॑व॒र्गम् । \newline
18. नि॒र॒व॒दायेति॑ निः - अ॒व॒दाय॑ । \newline
19. अ॒नृ॒णः सु॑व॒र्गꣳ सु॑व॒र्ग म॑नृ॒णो अ॑नृ॒णः सु॑व॒र्गम् ॅलो॒कम् ॅलो॒कꣳ सु॑व॒र्ग म॑नृ॒णो अ॑नृ॒णः सु॑व॒र्गम् ॅलो॒कम् । \newline
20. सु॒व॒र्गम् ॅलो॒कम् ॅलो॒कꣳ सु॑व॒र्गꣳ सु॑व॒र्गम् ॅलो॒क मे᳚त्येति लो॒कꣳ सु॑व॒र्गꣳ सु॑व॒र्गम् ॅलो॒क मे॑ति । \newline
21. सु॒व॒र्गमिति॑ सुवः - गम् । \newline
22. लो॒क मे᳚त्येति लो॒कम् ॅलो॒क मे॑ति॒ यदि॒ यद्ये॑ति लो॒कम् ॅलो॒क मे॑ति॒ यदि॑ । \newline
23. ए॒ति॒ यदि॒ यद्ये᳚त्येति॒ यदि॑ मि॒श्रम् मि॒श्रम् ॅयद्ये᳚ त्येति॒ यदि॑ मि॒श्रम् । \newline
24. यदि॑ मि॒श्रम् मि॒श्रम् ॅयदि॒ यदि॑ मि॒श्र मि॑वे व मि॒श्रम् ॅयदि॒ यदि॑ मि॒श्र मि॑व । \newline
25. मि॒श्र मि॑वे व मि॒श्रम् मि॒श्र मि॑व॒ चरे॒च् चरे॑ दिव मि॒श्रम् मि॒श्र मि॑व॒ चरे᳚त् । \newline
26. इ॒व॒ चरे॒च् चरे॑दिवे व॒ चरे॑ दञ्ज॒लिना᳚ ऽञ्ज॒लिना॒ चरे॑दिवे व॒ चरे॑ दञ्ज॒लिना᳚ । \newline
27. चरे॑ दञ्ज॒लिना᳚ ऽञ्ज॒लिना॒ चरे॒च् चरे॑ दञ्ज॒लिना॒ सक्तू॒न् थ्सक्तू॑ नञ्ज॒लिना॒ चरे॒च् चरे॑ दञ्ज॒लिना॒ सक्तून्॑ । \newline
28. अ॒ञ्ज॒लिना॒ सक्तू॒न् थ्सक्तू॑ नञ्ज॒लिना᳚ ऽञ्ज॒लिना॒ सक्तू᳚न् प्रदा॒व्ये᳚ प्रदा॒व्ये॑ सक्तू॑ नञ्ज॒लिना᳚ ऽञ्ज॒लिना॒ 
सक्तू᳚न् प्रदा॒व्ये᳚ । \newline
29. सक्तू᳚न् प्रदा॒व्ये᳚ प्रदा॒व्ये॑ सक्तू॒न् थ्सक्तू᳚न् प्रदा॒व्ये॑ जुहुयाज् जुहुयात् प्रदा॒व्ये॑ सक्तू॒न् थ्सक्तू᳚न् प्रदा॒व्ये॑ जुहुयात् । \newline
30. प्र॒दा॒व्ये॑ जुहुयाज् जुहुयात् प्रदा॒व्ये᳚ प्रदा॒व्ये॑ जुहुयादे॒ष ए॒ष जु॑हुयात् प्रदा॒व्ये᳚ प्रदा॒व्ये॑ जुहुयादे॒षः । \newline
31. प्र॒दा॒व्य॑ इति॑ प्र - दा॒व्ये᳚ । \newline
32. जु॒हु॒या॒ दे॒ष ए॒ष जु॑हुयाज् जुहुया दे॒ष वै वा ए॒ष जु॑हुयाज् जुहुया दे॒ष वै । \newline
33. ए॒ष वै वा ए॒ष ए॒ष वा अ॒ग्नि र॒ग्निर् वा ए॒ष ए॒ष वा अ॒ग्निः । \newline
34. वा अ॒ग्नि र॒ग्निर् वै वा अ॒ग्निर् वै᳚श्वान॒रो वै᳚श्वान॒रो᳚ ऽग्निर् वै वा अ॒ग्निर् वै᳚श्वान॒रः । \newline
35. अ॒ग्निर् वै᳚श्वान॒रो वै᳚श्वान॒रो᳚ ऽग्नि र॒ग्निर् वै᳚श्वान॒रो यद् यद् वै᳚श्वान॒रो᳚ ऽग्नि र॒ग्निर् वै᳚श्वान॒रो यत् । \newline
36. वै॒श्वा॒न॒रो यद् यद् वै᳚श्वान॒रो वै᳚श्वान॒रो यत् प्र॑दा॒व्यः॑ प्रदा॒व्यो॑ यद् वै᳚श्वान॒रो वै᳚श्वान॒रो यत् प्र॑दा॒व्यः॑ । \newline
37. यत् प्र॑दा॒व्यः॑ प्रदा॒व्यो॑ यद् यत् प्र॑दा॒व्यः॑ स स प्र॑दा॒व्यो॑ यद् यत् प्र॑दा॒व्यः॑ सः । \newline
38. प्र॒दा॒व्यः॑ स स प्र॑दा॒व्यः॑ प्रदा॒व्यः॑ स ए॒वैव स प्र॑दा॒व्यः॑ प्रदा॒व्यः॑ स ए॒व । \newline
39. प्र॒दा॒व्य॑ इति॑ प्र - दा॒व्यः॑ । \newline
40. स ए॒वैव स स ए॒वैन॑ मेन मे॒व स स ए॒वैन᳚म् । \newline
41. ए॒वैन॑ मेन मे॒वैवैनꣳ॑ स्वदयति स्वदय त्येन मे॒वैवैनꣳ॑ स्वदयति । \newline
42. ए॒नꣳ॒॒ स्व॒द॒य॒ति॒ स्व॒द॒य॒ त्ये॒न॒ मे॒नꣳ॒॒ स्व॒द॒य॒ त्यह्ना॒ मह्नाꣳ॑ स्वदय त्येन मेनꣳ स्वदय॒ त्यह्ना᳚म् । \newline
43. स्व॒द॒य॒ त्यह्ना॒ मह्नाꣳ॑ स्वदयति स्वदय॒ त्यह्ना᳚म् ॅवि॒धान्या᳚म् ॅवि॒धान्या॒ मह्नाꣳ॑ स्वदयति स्वदय॒ त्यह्ना᳚म् ॅवि॒धान्या᳚म् । \newline
44. अह्ना᳚म् ॅवि॒धान्या᳚म् ॅवि॒धान्या॒ मह्ना॒ मह्ना᳚म् ॅवि॒धान्या॑ मेकाष्ट॒काया॑ मेकाष्ट॒काया᳚म् ॅवि॒धान्या॒ मह्ना॒ मह्ना᳚म् ॅवि॒धान्या॑ मेकाष्ट॒काया᳚म् । \newline
45. वि॒धान्या॑ मेकाष्ट॒काया॑ मेकाष्ट॒काया᳚म् ॅवि॒धान्या᳚म् ॅवि॒धान्या॑ मेकाष्ट॒काया॑ मपू॒प म॑पू॒प मे॑काष्ट॒काया᳚म् ॅवि॒धान्या᳚म् ॅवि॒धान्या॑ मेकाष्ट॒काया॑ मपू॒पम् । \newline
46. वि॒धान्या॒मिति॑ वि - धान्या᳚म् । \newline
47. ए॒का॒ष्ट॒काया॑ मपू॒प म॑पू॒प मे॑काष्ट॒काया॑ मेकाष्ट॒काया॑ मपू॒पम् चतु॑श्शराव॒म् चतु॑श्शराव मपू॒प मे॑काष्ट॒काया॑ मेकाष्ट॒काया॑ मपू॒पम् चतु॑श्शरावम् । \newline
48. ए॒का॒ष्ट॒काया॒मित्ये॑क - अ॒ष्ट॒काया᳚म् । \newline
49. अ॒पू॒पम् चतु॑श्शराव॒म् चतु॑श्शराव मपू॒प म॑पू॒पम् चतु॑श्शरावम् प॒क्त्वा प॒क्त्वा चतु॑श्शराव मपू॒प म॑पू॒पम् चतु॑श्शरावम् प॒क्त्वा । \newline
50. चतु॑श्शरावम् प॒क्त्वा प॒क्त्वा चतु॑श्शराव॒म् चतु॑श्शरावम् प॒क्त्वा प्रा॒तः प्रा॒तः प॒क्त्वा चतु॑श्शराव॒म् चतु॑श्शरावम् प॒क्त्वा प्रा॒तः । \newline
51. चतु॑श्शराव॒मिति॒ चतुः॑ - श॒रा॒व॒म् । \newline
52. प॒क्त्वा प्रा॒तः प्रा॒तः प॒क्त्वा प॒क्त्वा प्रा॒त रे॒ते नै॒तेन॑ प्रा॒तः प॒क्त्वा प॒क्त्वा प्रा॒त रे॒तेन॑ । \newline
53. प्रा॒त रे॒ते नै॒तेन॑ प्रा॒तः प्रा॒त रे॒तेन॒ कक्ष॒म् कक्ष॑ मे॒तेन॑ प्रा॒तः प्रा॒त रे॒तेन॒ कक्ष᳚म् । \newline
54. ए॒तेन॒ कक्ष॒म् कक्ष॑ मे॒ते नै॒तेन॒ कक्ष॒ मुपोप॒ कक्ष॑ मे॒ते नै॒तेन॒ कक्ष॒ मुप॑ । \newline
55. कक्ष॒ मुपोप॒ कक्ष॒म् कक्ष॒ मुपौ॑षे दोषे॒दुप॒ कक्ष॒म् कक्ष॒ मुपौ॑षेत् । \newline
56. उपौ॑षे दोषे॒ दुपोपौ॑षे॒द् यदि॒ यद्यो॑षे॒ दुपोपौ॑षे॒द् यदि॑ । \newline
57. ओ॒षे॒द् यदि॒ यद्यो॑षे दोषे॒द् यदि॒ दह॑ति॒ दह॑ति॒ यद्यो॑षे दोषे॒द् यदि॒ दह॑ति । \newline
58. यदि॒ दह॑ति॒ दह॑ति॒ यदि॒ यदि॒ दह॑ति पुण्य॒सम॑म् पुण्य॒सम॒म् दह॑ति॒ यदि॒ यदि॒ दह॑ति पुण्य॒सम᳚म् । \newline
\pagebreak
\markright{ TS 3.3.8.5  \hfill https://www.vedavms.in \hfill}

\section{ TS 3.3.8.5 }

\textbf{TS 3.3.8.5 } \newline
\textbf{Samhita Paata} \newline

दह॑ति पुण्य॒समं॑ भवति॒ यदि॒ न दह॑ति पाप॒सम॑मे॒तेन॑ हस्म॒ वा ऋष॑यः पु॒रा वि॒ज्ञाने॑न दीर्घस॒त्रमुप॑ यन्ति॒ यो वा उ॑पद्र॒ष्टार॑मुप-श्रो॒तार॑मनुख्या॒तारं॑ ॅवि॒द्वान्. यज॑ते॒ सम॒मुष्मि॑न् ॅलो॒क इ॑ष्टापू॒र्तेन॑ गच्छते॒ऽग्निर्वा उ॑पद्र॒ष्टा वा॒युरु॑पश्रो॒ता ऽऽदि॒त्यो॑ऽनुख्या॒ता तान्. य ए॒वं ॅवि॒द्वान्. यज॑ते॒ सम॒मुष्मि॑न् ॅलो॒क इ॑ष्टापू॒र्तेन॑ गच्छते॒ ऽयं नो॒ नभ॑सा पु॒र - [  ] \newline

\textbf{Pada Paata} \newline

दह॑ति । पु॒ण्य॒सम॒मिति॑ पुण्य - सम᳚म् । भ॒व॒ति॒ । यदि॑ । न । दह॑ति । पा॒प॒सम॒मिति॑ पाप - सम᳚म् । ए॒तेन॑ । ह॒ । स्म॒ । वै । ऋष॑यः । पु॒रा । वि॒ज्ञाने॒नेति॑ वि - ज्ञाने॑न । दी॒र्घ॒स॒त्रमिति॑ दीर्घ - स॒त्रम् । उपेति॑ । य॒न्ति॒ । यः । वै । उ॒प॒द्र॒ष्टार॒मित्यु॑प - द्र॒ष्टार᳚म् । उ॒प॒श्रो॒तार॒मित्यु॑प - श्रो॒तार᳚म् । अ॒नु॒ख्या॒तार॒मित्य॑नु - ख्या॒तार᳚म् । वि॒द्वान् । यज॑ते । समिति॑ । अ॒मुष्मिन्न्॑ । लो॒के । इ॒ष्टा॒पू॒र्तेनेती᳚ष्ट - पू॒र्तेन॑ । ग॒च्छ॒ते॒ । अ॒ग्निः । वै । उ॒प॒द्र॒ष्टेत्यु॑प-द्र॒ष्टा । वा॒युः । उ॒प॒श्रो॒तेत्यु॑प - श्रो॒ता । आ॒दि॒त्यः । अ॒नु॒ख्या॒तेत्य॑नु-ख्या॒ता । तान् । यः । ए॒वम् । वि॒द्वान् । यज॑ते । समिति॑ । अ॒मुष्मिन्न्॑ । लो॒के । इ॒ष्टा॒पू॒र्तेनेती᳚ष्ट - पू॒र्तेन॑ । ग॒च्छ॒ते॒ । अ॒यम् । नः॒ । नभ॑सा । पु॒रः ।  \newline


\textbf{Krama Paata} \newline

दह॑ति पुण्य॒सम᳚म् । पु॒ण्य॒सम॑म् भवति । पु॒ण्य॒सम॒मिति॑ पुण्य - सम᳚म् । भ॒व॒ति॒ यदि॑ । यदि॒ न । न दह॑ति । दह॑ति पाप॒सम᳚म् । पा॒प॒सम॑मे॒तेन॑ । पा॒प॒सम॒मिति॑ पाप - सम᳚म् । ए॒तेन॑ ह । ह॒ स्म॒ । स्म॒ वै । वा ऋष॑यः । ऋष॑यः पु॒रा । पु॒रा वि॒ज्ञाने॑न । वि॒ज्ञाने॑न दीर्घस॒त्रम् । वि॒ज्ञाने॒नेति॑ वि - ज्ञाने॑न । दी॒र्घ॒स॒त्रमुप॑ । दी॒र्घ॒स॒त्रमिति॑ दीर्घ - स॒त्रम् । उप॑ यन्ति । य॒न्ति॒ यः । यो वै । वा उ॑पद्र॒ष्टार᳚म् । उ॒प॒द्र॒ष्टार॑मुपश्रो॒तार॑म् । उ॒प॒द्र॒ष्टार॒मित्यु॑प - द्र॒ष्टार᳚म् । उ॒प॒श्रो॒तार॑मनुख्या॒तार᳚म् । उ॒प॒श्रो॒तार॒मित्यु॑प - श्रो॒तार᳚म् । अ॒नु॒ख्या॒तारं॑ ॅवि॒द्वान् । अ॒नु॒ख्या॒तार॒मित्य॑नु - ख्या॒तार᳚म् । वि॒द्वान्. यज॑ते । यज॑ते॒ सम् । सम॒मुष्मिन्न्॑ । अ॒मुष्मि॑न् ॅलो॒के । लो॒क इ॑ष्टापू॒र्तेन॑ । इ॒ष्टा॒पू॒र्तेन॑ गच्छते । इ॒ष्टा॒पू॒र्तेनेती᳚ष्ट - पू॒र्तेन॑ । ग॒च्छ॒ते॒ ऽग्निः । अ॒ग्निर् वै । वा उ॑पद्र॒ष्टा । उ॒प॒द्र॒ष्टा वा॒युः । उ॒प॒द्र॒ष्टेत्यु॑प - द्र॒ष्टा । वा॒युरु॑पश्रो॒ता । उ॒प॒श्रो॒ता ऽऽदि॒त्यः । उ॒प॒श्रो॒तेत्यु॑प - श्रो॒ता । आ॒दि॒त्यो॑ ऽनुख्या॒ता । अ॒नु॒ख्या॒ता तान् । अ॒नु॒ख्या॒तेत्य॑नु - ख्या॒ता । तान्. यः । य ए॒वम् । ए॒वं ॅवि॒द्वान् । वि॒द्वान्. यज॑ते । यज॑ते॒ सम् । सम॒मुष्मिन्न्॑ । अ॒मुष्मि॑न् ॅलो॒के । लो॒क इ॑ष्टापू॒र्तेन॑ । इ॒ष्टा॒पू॒र्तेन॑ गच्छते । इ॒ष्टा॒पू॒र्तेनेती᳚ष्ट - पू॒र्तेन॑ । ग॒च्छ॒ते॒ ऽयम् । अ॒यम् नः॑ । नो॒ नभ॑सा । नभ॑सा पु॒रः । पु॒र इति॑ \newline

\textbf{Jatai Paata} \newline

1. दह॑ति पुण्य॒सम॑म् पुण्य॒सम॒म् दह॑ति॒ दह॑ति पुण्य॒सम᳚म् । \newline
2. पु॒ण्य॒सम॑म् भवति भवति पुण्य॒सम॑म् पुण्य॒सम॑म् भवति । \newline
3. पु॒ण्य॒सम॒मिति॑ पुण्य - सम᳚म् । \newline
4. भ॒व॒ति॒ यदि॒ यदि॑ भवति भवति॒ यदि॑ । \newline
5. यदि॒ न न यदि॒ यदि॒ न । \newline
6. न दह॑ति॒ दह॑ति॒ न न दह॑ति । \newline
7. दह॑ति पाप॒सम॑म् पाप॒सम॒म् दह॑ति॒ दह॑ति पाप॒सम᳚म् । \newline
8. पा॒प॒सम॑ मे॒तेनै॒तेन॑ पाप॒सम॑म् पाप॒सम॑ मे॒तेन॑ । \newline
9. पा॒प॒सम॒मिति॑ पाप - सम᳚म् । \newline
10. ए॒तेन॑ ह है॒ते नै॒तेन॑ ह । \newline
11. ह॒ स्म॒ स्म॒ ह॒ ह॒ स्म॒ । \newline
12. स्म॒ वै वै स्म॑ स्म॒ वै । \newline
13. वा ऋष॑य॒ ऋष॑यो॒ वै वा ऋष॑यः । \newline
14. ऋष॑यः पु॒रा पु॒र र्.ष॑य॒ ऋष॑यः पु॒रा । \newline
15. पु॒रा वि॒ज्ञाने॑न वि॒ज्ञाने॑न पु॒रा पु॒रा वि॒ज्ञाने॑न । \newline
16. वि॒ज्ञाने॑न दीर्घस॒त्रम् दी᳚र्घस॒त्रम् ॅवि॒ज्ञाने॑न वि॒ज्ञाने॑न दीर्घस॒त्रम् । \newline
17. वि॒ज्ञाने॒नेति॑ वि - ज्ञाने॑न । \newline
18. दी॒र्घ॒स॒त्र मुपोप॑ दीर्घस॒त्रम् दी᳚र्घस॒त्र मुप॑ । \newline
19. दी॒र्घ॒स॒त्रमिति॑ दीर्घ - स॒त्रम् । \newline
20. उप॑ यन्ति य॒ न्त्युपोप॑ यन्ति । \newline
21. य॒न्ति॒ यो यो य॑न्ति यन्ति॒ यः । \newline
22. यो वै वै यो यो वै । \newline
23. वा उ॑पद्र॒ष्टार॑ मुपद्र॒ष्टार॒म् ॅवै वा उ॑पद्र॒ष्टार᳚म् । \newline
24. उ॒प॒द्र॒ष्टार॑ मुपश्रो॒तार॑ मुपश्रो॒तार॑ मुपद्र॒ष्टार॑ मुपद्र॒ष्टार॑ मुपश्रो॒तार᳚म् । \newline
25. उ॒प॒द्र॒ष्टार॒मित्यु॑प - द्र॒ष्टार᳚म् । \newline
26. उ॒प॒श्रो॒तार॑ मनुख्या॒तार॑ मनुख्या॒तार॑ मुपश्रो॒तार॑ मुपश्रो॒तार॑ मनुख्या॒तार᳚म् । \newline
27. उ॒प॒श्रो॒तार॒मित्यु॑प - श्रो॒तार᳚म् । \newline
28. अ॒नु॒ख्या॒तार॑म् ॅवि॒द्वान्. वि॒द्वा न॑नुख्या॒तार॑ मनुख्या॒तार॑म् ॅवि॒द्वान् । \newline
29. अ॒नु॒ख्या॒तार॒मित्य॑नु - ख्या॒तार᳚म् । \newline
30. वि॒द्वान्. यज॑ते॒ यज॑ते वि॒द्वान्. वि॒द्वान्. यज॑ते । \newline
31. यज॑ते॒ सꣳ सम् ॅयज॑ते॒ यज॑ते॒ सम् । \newline
32. स म॒मुष्मि॑न् न॒मुष्मि॒न् थ्सꣳ स म॒मुष्मिन्न्॑ । \newline
33. अ॒मुष्मि॑न् ॅलो॒के लो॒के॑ ऽमुष्मि॑न् न॒मुष्मि॑न् ॅलो॒के । \newline
34. लो॒क इ॑ष्टापू॒र्तेने᳚ ष्टापू॒र्तेन॑ लो॒के लो॒क इ॑ष्टापू॒र्तेन॑ । \newline
35. इ॒ष्टा॒पू॒र्तेन॑ गच्छते गच्छत इष्टापू॒र्तेने᳚ ष्टापू॒र्तेन॑ गच्छते । \newline
36. इ॒ष्टा॒पू॒र्तेनेती᳚ष्ट - पू॒र्तेन॑ । \newline
37. ग॒च्छ॒ते॒ ऽग्नि र॒ग्निर् ग॑च्छते गच्छते॒ ऽग्निः । \newline
38. अ॒ग्निर् वै वा अ॒ग्नि र॒ग्निर् वै । \newline
39. वा उ॑पद्र॒ ष्टोप॑द्र॒ष्टा वै वा उ॑पद्र॒ष्टा । \newline
40. उ॒प॒द्र॒ष्टा वा॒युर् वा॒यु रु॑पद्र॒ ष्टोप॑द्र॒ष्टा वा॒युः । \newline
41. उ॒प॒द्र॒ष्टेत्यु॑प - द्र॒ष्टा । \newline
42. वा॒यु रु॑पश्रो॒ तोप॑श्रो॒ता वा॒युर् वा॒यु रु॑पश्रो॒ता । \newline
43. उ॒प॒श्रो॒ता ऽऽदि॒त्य आ॑दि॒त्य उ॑पश्रो॒ तोप॑श्रो॒ता ऽऽदि॒त्यः । \newline
44. उ॒प॒श्रो॒तेत्यु॑प - श्रो॒ता । \newline
45. आ॒दि॒त्यो॑ ऽनुख्या॒ता ऽनु॑ख्या॒ता ऽऽदि॒त्य आ॑दि॒त्यो॑ ऽनुख्या॒ता । \newline
46. अ॒नु॒ख्या॒ता ताꣳ स्ता न॑नुख्या॒ता ऽनु॑ख्या॒ता तान् । \newline
47. अ॒नु॒ख्या॒तेत्य॑नु - ख्या॒ता । \newline
48. तान्. यो य स्ताꣳ स्तान्. यः । \newline
49. य ए॒व मे॒वम् ॅयो य ए॒वम् । \newline
50. ए॒वम् ॅवि॒द्वान्. वि॒द्वा ने॒व मे॒वम् ॅवि॒द्वान् । \newline
51. वि॒द्वान्. यज॑ते॒ यज॑ते वि॒द्वान्. वि॒द्वान्. यज॑ते । \newline
52. यज॑ते॒ सꣳ सम् ॅयज॑ते॒ यज॑ते॒ सम् । \newline
53. स म॒मुष्मि॑न् न॒मुष्मि॒न् थ्सꣳ स म॒मुष्मिन्न्॑ । \newline
54. अ॒मुष्मि॑न् ॅलो॒के लो॒के॑ ऽमुष्मि॑न् न॒मुष्मि॑न् ॅलो॒के । \newline
55. लो॒क इ॑ष्टापू॒र्तेने᳚ ष्टापू॒र्तेन॑ लो॒के लो॒क इ॑ष्टापू॒र्तेन॑ । \newline
56. इ॒ष्टा॒पू॒र्तेन॑ गच्छते गच्छत इष्टापू॒र्तेने᳚ ष्टापू॒र्तेन॑ गच्छते । \newline
57. इ॒ष्टा॒पू॒र्तेनेती᳚ष्ट - पू॒र्तेन॑ । \newline
58. ग॒च्छ॒ते॒ ऽय म॒यम् ग॑च्छते गच्छते॒ ऽयम् । \newline
59. अ॒यम् नो॑ नो॒ ऽय म॒यम् नः॑ । \newline
60. नो॒ नभ॑सा॒ नभ॑सा नो नो॒ नभ॑सा । \newline
61. नभ॑सा पु॒रः पु॒रो नभ॑सा॒ नभ॑सा पु॒रः । \newline
62. पु॒र इतीति॑ पु॒रः पु॒र इति॑ । \newline

\textbf{Ghana Paata } \newline

1. दह॑ति पुण्य॒सम॑म् पुण्य॒सम॒म् दह॑ति॒ दह॑ति पुण्य॒सम॑म् भवति भवति पुण्य॒सम॒म् दह॑ति॒ दह॑ति पुण्य॒सम॑म् भवति । \newline
2. पु॒ण्य॒सम॑म् भवति भवति पुण्य॒सम॑म् पुण्य॒सम॑म् भवति॒ यदि॒ यदि॑ भवति पुण्य॒सम॑म् पुण्य॒सम॑म् भवति॒ यदि॑ । \newline
3. पु॒ण्य॒सम॒मिति॑ पुण्य - सम᳚म् । \newline
4. भ॒व॒ति॒ यदि॒ यदि॑ भवति भवति॒ यदि॒ न न यदि॑ भवति भवति॒ यदि॒ न । \newline
5. यदि॒ न न यदि॒ यदि॒ न दह॑ति॒ दह॑ति॒ न यदि॒ यदि॒ न दह॑ति । \newline
6. न दह॑ति॒ दह॑ति॒ न न दह॑ति पाप॒सम॑म् पाप॒सम॒म् दह॑ति॒ न न दह॑ति पाप॒सम᳚म् । \newline
7. दह॑ति पाप॒सम॑म् पाप॒सम॒म् दह॑ति॒ दह॑ति पाप॒सम॑ मे॒ते नै॒तेन॑ पाप॒सम॒म् दह॑ति॒ दह॑ति पाप॒सम॑ मे॒तेन॑ । \newline
8. पा॒प॒सम॑ मे॒ते नै॒तेन॑ पाप॒सम॑म् पाप॒सम॑ मे॒तेन॑ ह है॒तेन॑ पाप॒सम॑म् पाप॒सम॑ मे॒तेन॑ ह । \newline
9. पा॒प॒सम॒मिति॑ पाप - सम᳚म् । \newline
10. ए॒तेन॑ ह है॒ते नै॒तेन॑ ह स्म स्म है॒ते नै॒तेन॑ ह स्म । \newline
11. ह॒ स्म॒ स्म॒ ह॒ ह॒ स्म॒ वै वै स्म॑ ह ह स्म॒ वै । \newline
12. स्म॒ वै वै स्म॑ स्म॒ वा ऋष॑य॒ ऋष॑यो॒ वै स्म॑ स्म॒ वा ऋष॑यः । \newline
13. वा ऋष॑य॒ ऋष॑यो॒ वै वा ऋष॑यः पु॒रा पु॒र र्.ष॑यो॒ वै वा ऋष॑यः पु॒रा । \newline
14. ऋष॑यः पु॒रा पु॒र र्.ष॑य॒ ऋष॑यः पु॒रा वि॒ज्ञाने॑न वि॒ज्ञाने॑न पु॒र र्.ष॑य॒ ऋष॑यः पु॒रा वि॒ज्ञाने॑न । \newline
15. पु॒रा वि॒ज्ञाने॑न वि॒ज्ञाने॑न पु॒रा पु॒रा वि॒ज्ञाने॑न दीर्घस॒त्रम् दी᳚र्घस॒त्रम् ॅवि॒ज्ञाने॑न पु॒रा पु॒रा वि॒ज्ञाने॑न दीर्घस॒त्रम् । \newline
16. वि॒ज्ञाने॑न दीर्घस॒त्रम् दी᳚र्घस॒त्रम् ॅवि॒ज्ञाने॑न वि॒ज्ञाने॑न दीर्घस॒त्र मुपोप॑ दीर्घस॒त्रम् ॅवि॒ज्ञाने॑न वि॒ज्ञाने॑न दीर्घस॒त्र मुप॑ । \newline
17. वि॒ज्ञाने॒नेति॑ वि - ज्ञाने॑न । \newline
18. दी॒र्घ॒स॒त्र मुपोप॑ दीर्घस॒त्रम् दी᳚र्घस॒त्र मुप॑ यन्ति य॒न्त्युप॑ दीर्घस॒त्रम् दी᳚र्घस॒त्र मुप॑ यन्ति । \newline
19. दी॒र्घ॒स॒त्रमिति॑ दीर्घ - स॒त्रम् । \newline
20. उप॑ यन्ति य॒न्त्युपोप॑ यन्ति॒ यो यो य॒न्त्युपोप॑ यन्ति॒ यः । \newline
21. य॒न्ति॒ यो यो य॑न्ति यन्ति॒ यो वै वै यो य॑न्ति यन्ति॒ यो वै । \newline
22. यो वै वै यो यो वा उ॑पद्र॒ष्टार॑ मुपद्र॒ष्टार॒म् ॅवै यो यो वा उ॑पद्र॒ष्टार᳚म् । \newline
23. वा उ॑पद्र॒ष्टार॑ मुपद्र॒ष्टार॒म् ॅवै वा उ॑पद्र॒ष्टार॑ मुपश्रो॒तार॑ मुपश्रो॒तार॑ मुपद्र॒ष्टार॒म् ॅवै वा उ॑पद्र॒ष्टार॑ मुपश्रो॒तार᳚म् । \newline
24. उ॒प॒द्र॒ष्टार॑ मुपश्रो॒तार॑ मुपश्रो॒तार॑ मुपद्र॒ष्टार॑ मुपद्र॒ष्टार॑ मुपश्रो॒तार॑ मनुख्या॒तार॑ मनुख्या॒तार॑ मुपश्रो॒तार॑ मुपद्र॒ष्टार॑ मुपद्र॒ष्टार॑ मुपश्रो॒तार॑ मनुख्या॒तार᳚म् । \newline
25. उ॒प॒द्र॒ष्टार॒मित्यु॑प - द्र॒ष्टार᳚म् । \newline
26. उ॒प॒श्रो॒तार॑ मनुख्या॒तार॑ मनुख्या॒तार॑ मुपश्रो॒तार॑ मुपश्रो॒तार॑ मनुख्या॒तार॑म् ॅवि॒द्वान्. वि॒द्वा न॑नुख्या॒तार॑ मुपश्रो॒तार॑ मुपश्रो॒तार॑ मनुख्या॒तार॑म् ॅवि॒द्वान् । \newline
27. उ॒प॒श्रो॒तार॒मित्यु॑प - श्रो॒तार᳚म् । \newline
28. अ॒नु॒ख्या॒तार॑म् ॅवि॒द्वान्. वि॒द्वा न॑नुख्या॒तार॑ मनुख्या॒तार॑म् ॅवि॒द्वान्. यज॑ते॒ यज॑ते वि॒द्वा न॑नुख्या॒तार॑ मनुख्या॒तार॑म् ॅवि॒द्वान्. यज॑ते । \newline
29. अ॒नु॒ख्या॒तार॒मित्य॑नु - ख्या॒तार᳚म् । \newline
30. वि॒द्वान्. यज॑ते॒ यज॑ते वि॒द्वान्. वि॒द्वान्. यज॑ते॒ सꣳ सम् ॅयज॑ते वि॒द्वान्. वि॒द्वान्. यज॑ते॒ सम् । \newline
31. यज॑ते॒ सꣳ सम् ॅयज॑ते॒ यज॑ते॒ स म॒मुष्मि॑न्, न॒मुष्मि॒न् थ्सम् ॅयज॑ते॒ यज॑ते॒ स म॒मुष्मिन्न्॑ । \newline
32. स म॒मुष्मि॑न्, न॒मुष्मि॒न् थ्सꣳ स म॒मुष्मि॑न् ॅलो॒के लो॒के॑ ऽमुष्मि॒न् थ्सꣳ स म॒मुष्मि॑न् ॅलो॒के । \newline
33. अ॒मुष्मि॑न् ॅलो॒के लो॒के॑ ऽमुष्मि॑न्, न॒मुष्मि॑न् ॅलो॒क इ॑ष्टापू॒र्ते ने᳚ष्टापू॒र्तेन॑ लो॒के॑ ऽमुष्मि॑न्, न॒मुष्मि॑न् ॅलो॒क इ॑ष्टापू॒र्तेन॑ । \newline
34. लो॒क इ॑ष्टापू॒र्ते ने᳚ष्टापू॒र्तेन॑ लो॒के लो॒क इ॑ष्टापू॒र्तेन॑ गच्छते गच्छत इष्टापू॒र्तेन॑ लो॒के लो॒क इ॑ष्टापू॒र्तेन॑ गच्छते । \newline
35. इ॒ष्टा॒पू॒र्तेन॑ गच्छते गच्छत इष्टापू॒र्ते ने᳚ष्टापू॒र्तेन॑ गच्छते॒ ऽग्नि र॒ग्निर् ग॑च्छत 
इष्टापू॒र्ते ने᳚ष्टापू॒र्तेन॑ गच्छते॒ ऽग्निः । \newline
36. इ॒ष्टा॒पू॒र्तेनेती᳚ष्ट - पू॒र्तेन॑ । \newline
37. ग॒च्छ॒ते॒ ऽग्नि र॒ग्निर् ग॑च्छते गच्छते॒ ऽग्निर् वै वा अ॒ग्निर् ग॑च्छते गच्छते॒ ऽग्निर् वै । \newline
38. अ॒ग्निर् वै वा अ॒ग्नि र॒ग्निर् वा उ॑पद्र॒ ष्टोप॑द्र॒ष्टा वा अ॒ग्नि र॒ग्निर् वा उ॑पद्र॒ष्टा । \newline
39. वा उ॑पद्र॒ ष्टोप॑द्र॒ष्टा वै वा उ॑पद्र॒ष्टा वा॒युर् वा॒यु रु॑पद्र॒ष्टा वै वा उ॑पद्र॒ष्टा वा॒युः । \newline
40. उ॒प॒द्र॒ष्टा वा॒युर् वा॒यु रु॑पद्र॒ ष्टोप॑द्र॒ष्टा वा॒यु रु॑पश्रो॒तो प॑श्रो॒ता वा॒यु रु॑पद्र॒ 
ष्टोप॑द्र॒ष्टा वा॒यु रु॑पश्रो॒ता । \newline
41. उ॒प॒द्र॒ष्टेत्यु॑प - द्र॒ष्टा । \newline
42. वा॒यु रु॑पश्रो॒तो प॑श्रो॒ता वा॒युर् वा॒यु रु॑पश्रो॒ता ऽऽदि॒त्य आ॑दि॒त्य उ॑पश्रो॒ता वा॒युर् वा॒यु रु॑पश्रो॒ता ऽऽदि॒त्यः । \newline
43. उ॒प॒श्रो॒ता ऽऽदि॒त्य आ॑दि॒त्य उ॑पश्रो॒तो प॑श्रो॒ता ऽऽदि॒त्यो॑ ऽनुख्या॒ता ऽनु॑ख्या॒ता ऽऽदि॒त्य उ॑पश्रो॒तो प॑श्रो॒ता ऽऽदि॒त्यो॑ ऽनुख्या॒ता । \newline
44. उ॒प॒श्रो॒तेत्यु॑प - श्रो॒ता । \newline
45. आ॒दि॒त्यो॑ ऽनुख्या॒ता ऽनु॑ख्या॒ता ऽऽदि॒त्य आ॑दि॒त्यो॑ ऽनुख्या॒ता ताꣳ स्ता न॑नुख्या॒ता ऽऽदि॒त्य आ॑दि॒त्यो॑ ऽनुख्या॒ता तान् । \newline
46. अ॒नु॒ख्या॒ता ताꣳ स्ता न॑नुख्या॒ता ऽनु॑ख्या॒ता तान्. यो य स्ता न॑नुख्या॒ता ऽनु॑ख्या॒ता तान्. यः । \newline
47. अ॒नु॒ख्या॒तेत्य॑नु - ख्या॒ता । \newline
48. तान्. यो य स्ताꣳ स्तान्. य ए॒व मे॒वम् ॅय स्ताꣳ स्तान्. य ए॒वम् । \newline
49. य ए॒व मे॒वम् ॅयो य ए॒वम् ॅवि॒द्वान्. वि॒द्वा ने॒वम् ॅयो य ए॒वम् ॅवि॒द्वान् । \newline
50. ए॒वम् ॅवि॒द्वान्. वि॒द्वा ने॒व मे॒वम् ॅवि॒द्वान्. यज॑ते॒ यज॑ते वि॒द्वा ने॒व मे॒वम् ॅवि॒द्वान्. यज॑ते । \newline
51. वि॒द्वान्. यज॑ते॒ यज॑ते वि॒द्वान्. वि॒द्वान्. यज॑ते॒ सꣳ सम् ॅयज॑ते वि॒द्वान्. वि॒द्वान्. यज॑ते॒ सम् । \newline
52. यज॑ते॒ सꣳ सम् ॅयज॑ते॒ यज॑ते॒ स म॒मुष्मि॑न्, न॒मुष्मि॒न् थ्सम् ॅयज॑ते॒ यज॑ते॒ स म॒मुष्मिन्न्॑ । \newline
53. स म॒मुष्मि॑न्, न॒मुष्मि॒न् थ्सꣳ स म॒मुष्मि॑न् ॅलो॒के लो॒के॑ ऽमुष्मि॒न् थ्सꣳ स म॒मुष्मि॑न् ॅलो॒के । \newline
54. अ॒मुष्मि॑न् ॅलो॒के लो॒के॑ ऽमुष्मि॑न्, न॒मुष्मि॑न् ॅलो॒क इ॑ष्टापू॒र्ते ने᳚ष्टापू॒र्तेन॑ लो॒के॑ ऽमुष्मि॑न्, न॒मुष्मि॑न् ॅलो॒क इ॑ष्टापू॒र्तेन॑ । \newline
55. लो॒क इ॑ष्टापू॒र्ते ने᳚ष्टापू॒र्तेन॑ लो॒के लो॒क इ॑ष्टापू॒र्तेन॑ गच्छते गच्छत इष्टापू॒र्तेन॑ लो॒के लो॒क इ॑ष्टापू॒र्तेन॑ गच्छते । \newline
56. इ॒ष्टा॒पू॒र्तेन॑ गच्छते गच्छत इष्टापू॒र्ते ने᳚ष्टापू॒र्तेन॑ गच्छते॒ ऽय म॒यम् ग॑च्छत 
इष्टापू॒र्ते ने᳚ष्टापू॒र्तेन॑ गच्छते॒ ऽयम् । \newline
57. इ॒ष्टा॒पू॒र्तेनेती᳚ष्ट - पू॒र्तेन॑ । \newline
58. ग॒च्छ॒ते॒ ऽय म॒यम् ग॑च्छते गच्छते॒ ऽयम् नो॑ नो॒ ऽयम् ग॑च्छते गच्छते॒ ऽयम् नः॑ । \newline
59. अ॒यम् नो॑ नो॒ ऽय म॒यम् नो॒ नभ॑सा॒ नभ॑सा नो॒ ऽय म॒यम् नो॒ नभ॑सा । \newline
60. नो॒ नभ॑सा॒ नभ॑सा नो नो॒ नभ॑सा पु॒रः पु॒रो नभ॑सा नो नो॒ नभ॑सा पु॒रः । \newline
61. नभ॑सा पु॒रः पु॒रो नभ॑सा॒ नभ॑सा पु॒र इतीति॑ पु॒रो नभ॑सा॒ नभ॑सा पु॒र इति॑ । \newline
62. पु॒र इतीति॑ पु॒रः पु॒र इत्या॑हा॒हे ति॑ पु॒रः पु॒र इत्या॑ह । \newline
\pagebreak
\markright{ TS 3.3.8.6  \hfill https://www.vedavms.in \hfill}

\section{ TS 3.3.8.6 }

\textbf{TS 3.3.8.6 } \newline
\textbf{Samhita Paata} \newline

इत्या॑हा॒ग्निर्वै नभ॑सा पु॒रो᳚ऽग्निमे॒व तदा॑है॒तन्मे॑ गोपा॒येति॒ स त्वं नो॑ नभसस्पत॒ इत्या॑ह वा॒युर्वै नभ॑स॒स्पति॑र्वा॒युमे॒व तदा॑है॒तन्मे॑ गोपा॒येति॒ देव॑ सꣳस्फा॒नेत्या॑हा॒ऽसौ वा आ॑दि॒त्यो दे॒वः सꣳ॒॒स्फान॑ आदि॒त्यमे॒व तदा॑है॒तन्मे॑ गोपा॒येति॑ ॥ \newline

\textbf{Pada Paata} \newline

इति॑ । आ॒ह॒ । अ॒ग्निः । वै । नभ॑सा । पु॒रः । अ॒ग्निम् । ए॒व । तत् । आ॒ह॒ । ए॒तत् । मे॒ । गो॒पा॒य॒ । इति॑ । सः । त्वम् । नः॒ । न॒भ॒सः॒ । प॒ते॒ । इति॑ । आ॒ह॒ । वा॒युः । वै । नभ॑सः । पतिः॑ । वा॒युम् । ए॒व । तत् । आ॒ह॒ । ए॒तत् । मे॒ । गो॒पा॒य॒ । इति॑ । देव॑ । सꣳ॒॒स्फा॒नेति॑ सं - स्फा॒न॒ । इति॑ । आ॒ह॒ । अ॒सौ । वै । आ॒दि॒त्यः । दे॒वः । सꣳ॒॒स्फान॒ इति॑ सं - स्फानः॑ । आ॒दि॒त्यम् । ए॒व । तत् । आ॒ह॒ । ए॒तत् । मे॒ । गो॒पा॒य॒ । इति॑ ( ) ॥  \newline


\textbf{Krama Paata} \newline

इत्या॑ह । आ॒हा॒ग्निः । अ॒ग्निर् वै । वै नभ॑सा । नभ॑सा पु॒रः । पुरो᳚ ऽग्निम् । अ॒ग्निमे॒व । ए॒व तत् । तदा॑ह । आ॒है॒तत् । ए॒तन् मे᳚ । मे॒ गो॒पा॒य॒ । गो॒पा॒येति॑ । इति॒ सः । स त्वम् । त्वम् नः॑ । नो॒ न॒भ॒सः॒ । न॒भ॒स॒स्प॒ते॒ । प॒त॒ इति॑ । इत्या॑ह । आ॒ह॒ वा॒युः । वा॒युर् वै । वै नभ॑सः । नभ॑स॒स्पतिः॑ । पति॑र् वा॒युम् । वा॒युमे॒व । ए॒व तत् । तदा॑ह । आ॒है॒तत् । ए॒तन् मे᳚ । मे॒ गो॒पा॒य॒ । गो॒पा॒येति॑ । इति॒ देव॑ । देव॑ सꣳस्फान । सꣳ॒॒स्फा॒नेति॑ । सꣳ॒॒स्फा॒नेति॑ सम् - स्फा॒न॒ । इत्या॑ह । आ॒हा॒सौ । अ॒सौ वै । वा आ॑दि॒त्यः । आ॒दि॒त्यो दे॒वः । दे॒वः सꣳ॒॒स्फानः॑ । सꣳ॒॒स्फान॑ आदि॒त्यम् । सꣳ॒॒स्फान॒ इति॑ सं - स्फानः॑ । आ॒दि॒त्यमे॒व । ए॒व तत् । तदा॑ह । आ॒है॒तत् । ए॒तन् मे᳚ । मे॒ गो॒पा॒य॒ । गो॒पा॒येति॑ ( ) । इतीतीति॑ । \newline

\textbf{Jatai Paata} \newline

1. इत्या॑हा॒हे तीत्या॑ह । \newline
2. आ॒हा॒ ग्नि र॒ग्नि रा॑हाहा॒ग्निः । \newline
3. अ॒ग्निर् वै वा अ॒ग्नि र॒ग्निर् वै । \newline
4. वै नभ॑सा॒ नभ॑सा॒ वै वै नभ॑सा । \newline
5. नभ॑सा पु॒रः पु॒रो नभ॑सा॒ नभ॑सा पु॒रः । \newline
6. पु॒रो᳚ ऽग्नि म॒ग्निम् पु॒रः पु॒रो᳚ ऽग्निम् । \newline
7. अ॒ग्नि मे॒वैवाग्नि म॒ग्नि मे॒व । \newline
8. ए॒व तत् तदे॒वैव तत् । \newline
9. तदा॑हाह॒ तत् तदा॑ह । \newline
10. आ॒है॒त दे॒त दा॑हाहै॒तत् । \newline
11. ए॒तन् मे॑ म ए॒त दे॒तन् मे᳚ । \newline
12. मे॒ गो॒पा॒य॒ गो॒पा॒य॒ मे॒ मे॒ गो॒पा॒य॒ । \newline
13. गो॒पा॒ये तीति॑ गोपाय गोपा॒ये ति॑ । \newline
14. इति॒ स स इतीति॒ सः । \newline
15. स त्वम् त्वꣳ स स त्वम् । \newline
16. त्वम् नो॑ न॒ स्त्वम् त्वम् नः॑ । \newline
17. नो॒ न॒भ॒सो॒ न॒भ॒सो॒ नो॒ नो॒ न॒भ॒सः॒ । \newline
18. न॒भ॒स॒ स्प॒ते॒ प॒ते॒ न॒भ॒सो॒ न॒भ॒स॒ स्प॒ते॒ । \newline
19. प॒त॒ इतीति॑ पते पत॒ इति॑ । \newline
20. इत्या॑हा॒हे तीत्या॑ह । \newline
21. आ॒ह॒ वा॒युर् वा॒यु रा॑हाह वा॒युः । \newline
22. वा॒युर् वै वै वा॒युर् वा॒युर् वै । \newline
23. वै नभ॑सो॒ नभ॑सो॒ वै वै नभ॑सः । \newline
24. नभ॑स॒ स्पति॒ष् पति॒र् नभ॑सो॒ नभ॑स॒ स्पतिः॑ । \newline
25. पति॑र् वा॒युम् ॅवा॒युम् पति॒ष् पति॑र् वा॒युम् । \newline
26. वा॒यु मे॒वैव वा॒युम् ॅवा॒यु मे॒व । \newline
27. ए॒व तत् तदे॒वैव तत् । \newline
28. तदा॑हाह॒ तत् तदा॑ह । \newline
29. आ॒है॒त दे॒त दा॑हाहै॒तत् । \newline
30. ए॒तन् मे॑ म ए॒त दे॒तन् मे᳚ । \newline
31. मे॒ गो॒पा॒य॒ गो॒पा॒य॒ मे॒ मे॒ गो॒पा॒य॒ । \newline
32. गो॒पा॒ये तीति॑ गोपाय गोपा॒ये ति॑ । \newline
33. इति॒ देव॒ देवे तीति॒ देव॑ । \newline
34. देव॑ सꣳस्फान सꣳस्फान॒ देव॒ देव॑ सꣳस्फान । \newline
35. सꣳ॒॒स्फा॒ने तीति॑ सꣳस्फान सꣳस्फा॒ने ति॑ । \newline
36. सꣳ॒॒स्फा॒नेति॑ सम् - स्फा॒न॒ । \newline
37. इत्या॑हा॒हे तीत्या॑ह । \newline
38. आ॒हा॒सा व॒सा वा॑हा हा॒सौ । \newline
39. अ॒सौ वै वा अ॒सा व॒सौ वै । \newline
40. वा आ॑दि॒त्य आ॑दि॒त्यो वै वा आ॑दि॒त्यः । \newline
41. आ॒दि॒त्यो दे॒वो दे॒व आ॑दि॒त्य आ॑दि॒त्यो दे॒वः । \newline
42. दे॒वः सꣳ॒॒स्फानः॑ सꣳ॒॒स्फानो॑ दे॒वो दे॒वः सꣳ॒॒स्फानः॑ । \newline
43. सꣳ॒॒स्फान॑ आदि॒त्य मा॑दि॒त्यꣳ सꣳ॒॒स्फानः॑ सꣳ॒॒स्फान॑ आदि॒त्यम् । \newline
44. सꣳ॒॒स्फान॒ इति॑ सम् - स्फानः॑ । \newline
45. आ॒दि॒त्य मे॒वै वादि॒त्य मा॑दि॒त्य मे॒व । \newline
46. ए॒व तत् तदे॒वैव तत् । \newline
47. तदा॑हाह॒ तत् तदा॑ह । \newline
48. आ॒है॒त दे॒त दा॑हाहै॒तत् । \newline
49. ए॒तन् मे॑ म ए॒त दे॒तन् मे᳚ । \newline
50. मे॒ गो॒पा॒य॒ गो॒पा॒य॒ मे॒ मे॒ गो॒पा॒य॒ । \newline
51. गो॒पा॒ये तीति॑ गोपाय गोपा॒ये ति॑ । \newline
52. इतीतीति॑ । \newline

\textbf{Ghana Paata } \newline

1. इत्या॑हा॒हे तीत्या॑हा॒ ग्नि र॒ग्नि रा॒हे तीत्या॑हा॒ग्निः । \newline
2. आ॒हा॒ग्नि र॒ग्नि रा॑हाहा॒ग्निर् वै वा अ॒ग्नि रा॑हाहा॒ग्निर् वै । \newline
3. अ॒ग्निर् वै वा अ॒ग्नि र॒ग्निर् वै नभ॑सा॒ नभ॑सा॒ वा अ॒ग्नि र॒ग्निर् वै नभ॑सा । \newline
4. वै नभ॑सा॒ नभ॑सा॒ वै वै नभ॑सा पु॒रः पु॒रो नभ॑सा॒ वै वै नभ॑सा पु॒रः । \newline
5. नभ॑सा पु॒रः पु॒रो नभ॑सा॒ नभ॑सा पु॒रो᳚ ऽग्नि म॒ग्निम् पु॒रो नभ॑सा॒ नभ॑सा पु॒रो᳚ ऽग्निम् । \newline
6. पु॒रो᳚ ऽग्नि म॒ग्निम् पु॒रः पु॒रो᳚ ऽग्नि मे॒वैवाग्निम् पु॒रः पु॒रो᳚ ऽग्नि मे॒व । \newline
7. अ॒ग्नि मे॒वैवाग्नि म॒ग्नि मे॒व तत् तदे॒वाग्नि म॒ग्नि मे॒व तत् । \newline
8. ए॒व तत् तदे॒वैव तदा॑हाह॒ तदे॒वैव तदा॑ह । \newline
9. तदा॑हाह॒ तत् तदा॑ है॒त दे॒त दा॑ह॒ तत् तदा॑ है॒तत् । \newline
10. आ॒है॒त दे॒त दा॑हा है॒तन् मे॑ म ए॒त दा॑हा है॒तन् मे᳚ । \newline
11. ए॒तन् मे॑ म ए॒त दे॒तन् मे॑ गोपाय गोपाय म ए॒त दे॒तन् मे॑ गोपाय । \newline
12. मे॒ गो॒पा॒य॒ गो॒पा॒य॒ मे॒ मे॒ गो॒पा॒ये तीति॑ गोपाय मे मे गोपा॒ये ति॑ । \newline
13. गो॒पा॒ये तीति॑ गोपाय गोपा॒ये ति॒ स स इति॑ गोपाय गोपा॒ये ति॒ सः । \newline
14. इति॒ स स इतीति॒ स त्वम् त्वꣳ स इतीति॒ स त्वम् । \newline
15. स त्वम् त्वꣳ स स त्वम् नो॑ न॒ स्त्वꣳ स स त्वम् नः॑ । \newline
16. त्वन्नो॑ न॒ स्त्वम् त्वम् नो॑ नभसो नभसो न॒ स्त्वम् त्वम् नो॑ नभसः । \newline
17. नो॒ न॒भ॒सो॒ न॒भ॒सो॒ नो॒ नो॒ न॒भ॒स॒ स्प॒ते॒ प॒ते॒ न॒भ॒सो॒ नो॒ नो॒ न॒भ॒स॒ स्प॒ते॒ । \newline
18. न॒भ॒स॒ स्प॒ते॒ प॒ते॒ न॒भ॒सो॒ न॒भ॒स॒ स्प॒त॒ इतीति॑ पते नभसो नभस स्पत॒ इति॑ । \newline
19. प॒त॒ इतीति॑ पते पत॒ इत्या॑हा॒हे ति॑ पते पत॒ इत्या॑ह । \newline
20. इत्या॑हा॒हे तीत्या॑ह वा॒युर् वा॒युरा॒हे तीत्या॑ह वा॒युः । \newline
21. आ॒ह॒ वा॒युर् वा॒यु रा॑हाह वा॒युर् वै वै वा॒यु रा॑हाह वा॒युर् वै । \newline
22. वा॒युर् वै वै वा॒युर् वा॒युर् वै नभ॑सो॒ नभ॑सो॒ वै वा॒युर् वा॒युर् वै नभ॑सः । \newline
23. वै नभ॑सो॒ नभ॑सो॒ वै वै नभ॑स॒ स्पति॒ष् पति॒र् नभ॑सो॒ वै वै नभ॑स॒ स्पतिः॑ । \newline
24. नभ॑स॒ स्पति॒ष् पति॒र् नभ॑सो॒ नभ॑स॒ स्पति॑र् वा॒युम् ॅवा॒युम् पति॒र् नभ॑सो॒ नभ॑स॒ स्पति॑र् वा॒युम् । \newline
25. पति॑र् वा॒युम् ॅवा॒युम् पति॒ष् पति॑र् वा॒यु मे॒वैव वा॒युम् पति॒ष् पति॑र् वा॒यु मे॒व । \newline
26. वा॒यु मे॒वैव वा॒युम् ॅवा॒यु मे॒व तत् तदे॒व वा॒युम् ॅवा॒यु मे॒व तत् । \newline
27. ए॒व तत् तदे॒वैव तदा॑हाह॒ तदे॒वैव तदा॑ह । \newline
28. तदा॑हाह॒ तत् तदा॑ है॒त दे॒त दा॑ह॒ तत् तदा॑ है॒तत् । \newline
29. आ॒है॒त दे॒त दा॑हा है॒तन् मे॑ म ए॒त दा॑हा है॒तन् मे᳚ । \newline
30. ए॒तन् मे॑ म ए॒त दे॒तन् मे॑ गोपाय गोपाय म ए॒त दे॒तन् मे॑ गोपाय । \newline
31. मे॒ गो॒पा॒य॒ गो॒पा॒य॒ मे॒ मे॒ गो॒पा॒ये तीति॑ गोपाय मे मे गोपा॒ये ति॑ । \newline
32. गो॒पा॒ये तीति॑ गोपाय गोपा॒ये ति॒ देव॒ देवे ति॑ गोपाय गोपा॒ये ति॒ देव॑ । \newline
33. इति॒ देव॒ देवे तीति॒ देव॑ सꣳस्फान सꣳस्फान॒ देवे तीति॒ देव॑ सꣳस्फान । \newline
34. देव॑ सꣳस्फान सꣳस्फान॒ देव॒ देव॑ सꣳस्फा॒ने तीति॑ सꣳस्फान॒ देव॒ देव॑ सꣳस्फा॒ने ति॑ । \newline
35. सꣳ॒॒स्फा॒ने तीति॑ सꣳस्फान सꣳस्फा॒ने त्या॑हा॒हे ति॑ सꣳस्फान सꣳस्फा॒ने त्या॑ह । \newline
36. सꣳ॒॒स्फा॒नेति॑ सम् - स्फा॒न॒ । \newline
37. इत्या॑हा॒हे तीत्या॑हा॒सा व॒सा वा॒हे तीत्या॑हा॒सौ । \newline
38. आ॒हा॒सा व॒सा वा॑हाहा॒सौ वै वा अ॒सा वा॑हाहा॒सौ वै । \newline
39. अ॒सौ वै वा अ॒सा व॒सौ वा आ॑दि॒त्य आ॑दि॒त्यो वा अ॒सा व॒सौ वा आ॑दि॒त्यः । \newline
40. वा आ॑दि॒त्य आ॑दि॒त्यो वै वा आ॑दि॒त्यो दे॒वो दे॒व आ॑दि॒त्यो वै वा आ॑दि॒त्यो दे॒वः । \newline
41. आ॒दि॒त्यो दे॒वो दे॒व आ॑दि॒त्य आ॑दि॒त्यो दे॒वः सꣳ॒॒स्फानः॑ सꣳ॒॒स्फानो॑ दे॒व आ॑दि॒त्य आ॑दि॒त्यो दे॒वः सꣳ॒॒स्फानः॑ । \newline
42. दे॒वः सꣳ॒॒स्फानः॑ सꣳ॒॒स्फानो॑ दे॒वो दे॒वः सꣳ॒॒स्फान॑ आदि॒त्य मा॑दि॒त्यꣳ सꣳ॒॒स्फानो॑ दे॒वो दे॒वः सꣳ॒॒स्फान॑ आदि॒त्यम् । \newline
43. सꣳ॒॒स्फान॑ आदि॒त्य मा॑दि॒त्यꣳ सꣳ॒॒स्फानः॑ सꣳ॒॒स्फान॑ आदि॒त्य मे॒वैवादि॒त्यꣳ सꣳ॒॒स्फानः॑ सꣳ॒॒स्फान॑ आदि॒त्य मे॒व । \newline
44. सꣳ॒॒स्फान॒ इति॑ सम् - स्फानः॑ । \newline
45. आ॒दि॒त्य मे॒वै वादि॒त्य मा॑दि॒त्य मे॒व तत् तदे॒वादि॒त्य मा॑दि॒त्य मे॒व तत् । \newline
46. ए॒व तत् तदे॒वैव तदा॑हाह॒ तदे॒वैव तदा॑ह । \newline
47. तदा॑हाह॒ तत् तदा॑ है॒त दे॒त दा॑ह॒ तत् तदा॑ है॒तत् । \newline
48. आ॒है॒त दे॒त दा॑हा है॒तन् मे॑ म ए॒त दा॑हा है॒तन् मे᳚ । \newline
49. ए॒तन् मे॑ म ए॒त दे॒तन् मे॑ गोपाय गोपाय म ए॒त दे॒तन् मे॑ गोपाय । \newline
50. मे॒ गो॒पा॒य॒ गो॒पा॒य॒ मे॒ मे॒ गो॒पा॒ये तीति॑ गोपाय मे मे गोपा॒ये ति॑ । \newline
51. गो॒पा॒ये तीति॑ गोपाय गोपा॒ये ति॑ । \newline
52. इतीतीति॑ । \newline
\pagebreak
\markright{ TS 3.3.9.1  \hfill https://www.vedavms.in \hfill}

\section{ TS 3.3.9.1 }

\textbf{TS 3.3.9.1 } \newline
\textbf{Samhita Paata} \newline

ए॒तं ॅयुवा॑नं॒ परि॑ वो ददामि॒ तेन॒ क्रीड॑न्तीश्चरत प्रि॒येण॑ ।मा नः॑ शाप्त ज॒नुषा॑ सुभागा रा॒यस्पोषे॑ण॒ समि॒षा म॑देम ॥ नमो॑ महि॒म्न उ॒त चक्षु॑षे ते॒ मरु॑तां पित॒स्तद॒हं गृ॑णामि । अनु॑ मन्यस्व सु॒यजा॑ यजाम॒ जुष्टं॑ दे॒वाना॑मि॒दम॑स्तु ह॒व्यं ॥ दे॒वाना॑मे॒ष उ॑पना॒ह आ॑सीद॒पां गर्भ॒ ओष॑धीषु॒ न्य॑क्तः । सोम॑स्य द्र॒फ्सम॑वृणीत पू॒षा - [  ] \newline

\textbf{Pada Paata} \newline

ए॒तम् । युवा॑नम् । परीति॑ । वः॒ । द॒दा॒मि॒ । तेन॑ । क्रीड॑न्तीः । च॒र॒त॒ । प्रि॒येण॑ ॥ मा । नः॒ । शा॒प्त॒ । ज॒नुषा᳚ । सु॒भा॒गा॒ इति॑ सु - भा॒गाः॒ । रा॒यः । पोषे॑ण । समिति॑ । इ॒षा । म॒दे॒म॒ ॥ नमः॑ । म॒हि॒म्ने । उ॒त । चक्षु॑षे । ते॒ । मरु॑ताम् । पि॒तः॒ । तत् । अ॒हम् । गृ॒णा॒मि॒ ॥ अन्विति॑ । म॒न्य॒स्व॒ । सु॒यजेति॑ सु-यजा᳚ । य॒जा॒म॒ । जुष्ट᳚म् । दे॒वाना᳚म् । इ॒दम् । अ॒स्तु॒ । ह॒व्यम् ॥ दे॒वाना᳚म् । ए॒षः । उ॒प॒ना॒ह इत्यु॑प - ना॒हः । आ॒सी॒त् । अ॒पाम् । गर्भः॑ । ओष॑धीषु । न्य॑क्त॒ इति॒ नि - अ॒क्तः॒ ॥ सोम॑स्य । द्र॒फ्सम् । अ॒वृ॒णी॒त॒ । पू॒षा ।  \newline


\textbf{Krama Paata} \newline

ए॒तं ॅयुवा॑नम् । युवा॑न॒म् परि॑ । परि॑ वः । वो॒ द॒दा॒मि॒ । द॒दा॒मि॒ तेन॑ । तेन॒ क्रीड॑न्तीः । क्रीड॑न्ती श्चरत । च॒र॒त॒ प्रि॒येण॑ । प्रि॒येणेति॑ प्रि॒येण॑ ॥ मा नः॑ । नः॒ शा॒प्त॒ । शा॒प्त॒ ज॒नुषा᳚ । ज॒नुषा॑ सुभागाः । सु॒भा॒गा॒ रा॒यः । सु॒भा॒गा॒ इति॑ सु - भा॒गाः॒ । रा॒यस्पोषे॑ण । पोषे॑ण॒ सम् । समि॒षा । इ॒षा म॑देम । म॒दे॒मेति॑ मदेम ॥ नमो॑ महि॒म्ने । म॒हि॒म्न उ॒त । उ॒त चक्षु॑षे । चक्षु॑षे ते । ते॒ मरु॑ताम् । मरु॑ताम् पितः । पि॒त॒स्तत् । तद॒हम् । अ॒हम् गृ॑णामि । गृ॒णा॒मीति॑ गृणामि ॥ अनु॑ मन्यस्व । म॒न्य॒स्व॒ सु॒यजा᳚ । सु॒यजा॑ यजाम । सु॒यजेति॑ सु - यजा᳚ । य॒जा॒म॒ जुष्ट᳚म् । जुष्ट॑म् दे॒वाना᳚म् । दे॒वाना॑मि॒दम् । इ॒दम॑स्तु । अ॒स्तु॒ ह॒व्यम् । ह॒व्यमिति॑ ह॒व्यम् ॥ दे॒वाना॑मे॒षः । ए॒ष उ॑पना॒हः । उ॒प॒ना॒ह आ॑सीत् । उ॒प॒ना॒ह इत्यु॑प - ना॒हः । आ॒सी॒द॒पाम् । अ॒पाम् गर्भः॑ । गर्भ॒ ओष॑धीषु । ओष॑धीषु॒ न्य॑क्तः । न्य॑क्त॒ इति॒ नि - अ॒क्तः॒ ॥ सोम॑स्य द्र॒फ्सम् । द्र॒फ्सम॑वृणीत । अ॒वृ॒णी॒त॒ पू॒षा । पू॒षा बृ॒हन्न् \newline

\textbf{Jatai Paata} \newline

1. ए॒तम् ॅयुवा॑न॒म् ॅयुवा॑न मे॒त मे॒तम् ॅयुवा॑नम् । \newline
2. युवा॑न॒म् परि॒ परि॒ युवा॑न॒म् ॅयुवा॑न॒म् परि॑ । \newline
3. परि॑ वो वः॒ परि॒ परि॑ वः । \newline
4. वो॒ द॒दा॒मि॒ द॒दा॒मि॒ वो॒ वो॒ द॒दा॒मि॒ । \newline
5. द॒दा॒मि॒ तेन॒ तेन॑ ददामि ददामि॒ तेन॑ । \newline
6. तेन॒ क्रीड॑न्तीः॒ क्रीड॑न्ती॒ स्तेन॒ तेन॒ क्रीड॑न्तीः । \newline
7. क्रीड॑न्ती श्चरत चरत॒ क्रीड॑न्तीः॒ क्रीड॑न्ती श्चरत । \newline
8. च॒र॒त॒ प्रि॒येण॑ प्रि॒येण॑ चरत चरत प्रि॒येण॑ । \newline
9. प्रि॒येणेति॑ प्रि॒येण॑ । \newline
10. मा नो॑ नो॒ मा मा नः॑ । \newline
11. नः॒ शा॒प्त॒ शा॒प्त॒ नो॒ नः॒ शा॒प्त॒ । \newline
12. शा॒प्त॒ ज॒नुषा॑ ज॒नुषा॑ शाप्त शाप्त ज॒नुषा᳚ । \newline
13. ज॒नुषा॑ सुभागाः सुभागा ज॒नुषा॑ ज॒नुषा॑ सुभागाः । \newline
14. सु॒भा॒गा॒ रा॒यो रा॒यः सु॑भागाः सुभागा रा॒यः । \newline
15. सु॒भा॒गा॒ इति॑ सु - भा॒गाः॒ । \newline
16. रा॒य स्पोषे॑ण॒ पोषे॑ण रा॒यो रा॒य स्पोषे॑ण । \newline
17. पोषे॑ण॒ सꣳ सम् पोषे॑ण॒ पोषे॑ण॒ सम् । \newline
18. स मि॒षेषा सꣳ स मि॒षा । \newline
19. इ॒षा म॑देम मदेमे॒ षेषा म॑देम । \newline
20. म॒दे॒मेति॑ मदेम । \newline
21. नमो॑ महि॒म्ने म॑हि॒म्ने नमो॒ नमो॑ महि॒म्ने । \newline
22. म॒हि॒म्न उ॒तोत म॑हि॒म्ने म॑हि॒म्न उ॒त । \newline
23. उ॒त चक्षु॑षे॒ चक्षु॑ष उ॒तोत चक्षु॑षे । \newline
24. चक्षु॑षे ते ते॒ चक्षु॑षे॒ चक्षु॑षे ते । \newline
25. ते॒ मरु॑ता॒म् मरु॑ताम् ते ते॒ मरु॑ताम् । \newline
26. मरु॑ताम् पितः पित॒र् मरु॑ता॒म् मरु॑ताम् पितः । \newline
27. पि॒त॒ स्तत् तत् पि॑तः पित॒ स्तत् । \newline
28. तद॒ह म॒हम् तत् तद॒हम् । \newline
29. अ॒हम् गृ॑णामि गृणा म्य॒ह म॒हम् गृ॑णामि । \newline
30. गृ॒णा॒मीति॑ गृणामि । \newline
31. अनु॑ मन्यस्व मन्य॒स्वा न्वनु॑ मन्यस्व । \newline
32. म॒न्य॒स्व॒ सु॒यजा॑ सु॒यजा॑ मन्यस्व मन्यस्व सु॒यजा᳚ । \newline
33. सु॒यजा॑ यजाम यजाम सु॒यजा॑ सु॒यजा॑ यजाम । \newline
34. सु॒यजेति॑ सु - यजा᳚ । \newline
35. य॒जा॒म॒ जुष्ट॒म् जुष्ट॑म् ॅयजाम यजाम॒ जुष्ट᳚म् । \newline
36. जुष्ट॑म् दे॒वाना᳚म् दे॒वाना॒म् जुष्ट॒म् जुष्ट॑म् दे॒वाना᳚म् । \newline
37. दे॒वाना॑ मि॒द मि॒दम् दे॒वाना᳚म् दे॒वाना॑ मि॒दम् । \newline
38. इ॒द म॑स्त्व स्त्वि॒द मि॒द म॑स्तु । \newline
39. अ॒स्तु॒ ह॒व्यꣳ ह॒व्य म॑स्त्वस्तु ह॒व्यम् । \newline
40. ह॒व्यमिति॑ ह॒व्यम् । \newline
41. दे॒वाना॑ मे॒ष ए॒ष दे॒वाना᳚म् दे॒वाना॑ मे॒षः । \newline
42. ए॒ष उ॑पना॒ह उ॑पना॒ह ए॒ष ए॒ष उ॑पना॒हः । \newline
43. उ॒प॒ना॒ह आ॑सी दासी दुपना॒ह उ॑पना॒ह आ॑सीत् । \newline
44. उ॒प॒ना॒ह इत्यु॑प - ना॒हः । \newline
45. आ॒सी॒द॒पा म॒पा मा॑सी दासी द॒पाम् । \newline
46. अ॒पाम् गर्भो॒ गर्भो॒ ऽपा म॒पाम् गर्भः॑ । \newline
47. गर्भ॒ ओष॑धी॒ ष्वोष॑धीषु॒ गर्भो॒ गर्भ॒ ओष॑धीषु । \newline
48. ओष॑धीषु॒ न्य॑क्तो॒ न्य॑क्त॒ ओष॑धी॒ ष्वोष॑धीषु॒ न्य॑क्तः । \newline
49. न्य॑क्त॒ इति॒ नि - अ॒क्तः॒ । \newline
50. सोम॑स्य द्र॒फ्सम् द्र॒फ्सꣳ सोम॑स्य॒ सोम॑स्य द्र॒फ्सम् । \newline
51. द्र॒फ्स म॑वृणीता वृणीत द्र॒फ्सम् द्र॒फ्स म॑वृणीत । \newline
52. अ॒वृ॒णी॒त॒ पू॒षा पू॒षा ऽवृ॑णीता वृणीत पू॒षा । \newline
53. पू॒षा बृ॒हन् बृ॒हन् पू॒षा पू॒षा बृ॒हन्न् । \newline

\textbf{Ghana Paata } \newline

1. ए॒तम् ॅयुवा॑न॒म् ॅयुवा॑न मे॒त मे॒तम् ॅयुवा॑न॒म् परि॒ परि॒ युवा॑न मे॒त मे॒तम् ॅयुवा॑न॒म् परि॑ । \newline
2. युवा॑न॒म् परि॒ परि॒ युवा॑न॒म् ॅयुवा॑न॒म् परि॑ वो वः॒ परि॒ युवा॑न॒म् ॅयुवा॑न॒म् परि॑ वः । \newline
3. परि॑ वो वः॒ परि॒ परि॑ वो ददामि ददामि वः॒ परि॒ परि॑ वो ददामि । \newline
4. वो॒ द॒दा॒मि॒ द॒दा॒मि॒ वो॒ वो॒ द॒दा॒मि॒ तेन॒ तेन॑ ददामि वो वो ददामि॒ तेन॑ । \newline
5. द॒दा॒मि॒ तेन॒ तेन॑ ददामि ददामि॒ तेन॒ क्रीड॑न्तीः॒ क्रीड॑न्ती॒ स्तेन॑ ददामि ददामि॒ तेन॒ क्रीड॑न्तीः । \newline
6. तेन॒ क्रीड॑न्तीः॒ क्रीड॑न्ती॒ स्तेन॒ तेन॒ क्रीड॑न्ती श्चरत चरत॒ क्रीड॑न्ती॒ स्तेन॒ तेन॒ क्रीड॑न्ती श्चरत । \newline
7. क्रीड॑न्ती श्चरत चरत॒ क्रीड॑न्तीः॒ क्रीड॑न्ती श्चरत प्रि॒येण॑ प्रि॒येण॑ चरत॒ क्रीड॑न्तीः॒ क्रीड॑न्ती श्चरत प्रि॒येण॑ । \newline
8. च॒र॒त॒ प्रि॒येण॑ प्रि॒येण॑ चरत चरत प्रि॒येण॑ । \newline
9. प्रि॒येणेति॑ प्रि॒येण॑ । \newline
10. मा नो॑ नो॒ मा मा नः॑ शाप्त शाप्त नो॒ मा मा नः॑ शाप्त । \newline
11. नः॒ शा॒प्त॒ शा॒प्त॒ नो॒ नः॒ शा॒प्त॒ ज॒नुषा॑ ज॒नुषा॑ शाप्त नो नः शाप्त ज॒नुषा᳚ । \newline
12. शा॒प्त॒ ज॒नुषा॑ ज॒नुषा॑ शाप्त शाप्त ज॒नुषा॑ सुभागाः सुभागा ज॒नुषा॑ शाप्त शाप्त ज॒नुषा॑ सुभागाः । \newline
13. ज॒नुषा॑ सुभागाः सुभागा ज॒नुषा॑ ज॒नुषा॑ सुभागा रा॒यो रा॒यः सु॑भागा ज॒नुषा॑ ज॒नुषा॑ सुभागा रा॒यः । \newline
14. सु॒भा॒गा॒ रा॒यो रा॒यः सु॑भागाः सुभागा रा॒य स्पोषे॑ण॒ पोषे॑ण रा॒यः सु॑भागाः सुभागा रा॒य स्पोषे॑ण । \newline
15. सु॒भा॒गा॒ इति॑ सु - भा॒गाः॒ । \newline
16. रा॒य स्पोषे॑ण॒ पोषे॑ण रा॒यो रा॒य स्पोषे॑ण॒ सꣳ सम् पोषे॑ण रा॒यो रा॒य स्पोषे॑ण॒ सम् । \newline
17. पोषे॑ण॒ सꣳ सम् पोषे॑ण॒ पोषे॑ण॒ स मि॒षेषा सम् पोषे॑ण॒ पोषे॑ण॒ स मि॒षा । \newline
18. स मि॒षेषा सꣳ स मि॒षा म॑देम मदेमे॒ षा सꣳ स मि॒षा म॑देम । \newline
19. इ॒षा म॑देम मदेमे॒ षेषा म॑देम । \newline
20. म॒दे॒मेति॑ मदेम । \newline
21. नमो॑ महि॒म्ने म॑हि॒म्ने नमो॒ नमो॑ महि॒म्न उ॒तोत म॑हि॒म्ने नमो॒ नमो॑ महि॒म्न उ॒त । \newline
22. म॒हि॒म्न उ॒तोत म॑हि॒म्ने म॑हि॒म्न उ॒त चक्षु॑षे॒ चक्षु॑ष उ॒त म॑हि॒म्ने म॑हि॒म्न उ॒त चक्षु॑षे । \newline
23. उ॒त चक्षु॑षे॒ चक्षु॑ष उ॒तोत चक्षु॑षे ते ते॒ चक्षु॑ष उ॒तोत चक्षु॑षे ते । \newline
24. चक्षु॑षे ते ते॒ चक्षु॑षे॒ चक्षु॑षे ते॒ मरु॑ता॒म् मरु॑ताम् ते॒ चक्षु॑षे॒ चक्षु॑षे ते॒ मरु॑ताम् । \newline
25. ते॒ मरु॑ता॒म् मरु॑ताम् ते ते॒ मरु॑ताम् पितः पित॒र् मरु॑ताम् ते ते॒ मरु॑ताम् पितः । \newline
26. मरु॑ताम् पितः पित॒र् मरु॑ता॒म् मरु॑ताम् पित॒ स्तत् तत् पि॑त॒र् मरु॑ता॒म् मरु॑ताम् पित॒ स्तत् । \newline
27. पि॒त॒ स्तत् तत् पि॑तः पित॒ स्तद॒ह म॒हम् तत् पि॑तः पित॒ स्तद॒हम् । \newline
28. तद॒ह म॒हम् तत् तद॒हम् गृ॑णामि गृणा म्य॒हम् तत् तद॒हम् गृ॑णामि । \newline
29. अ॒हम् गृ॑णामि गृणा म्य॒ह म॒हम् गृ॑णामि । \newline
30. गृ॒णा॒मीति॑ गृणामि । \newline
31. अनु॑ मन्यस्व मन्य॒स्वा न्वनु॑ मन्यस्व सु॒यजा॑ सु॒यजा॑ मन्य॒स्वा न्वनु॑ मन्यस्व सु॒यजा᳚ । \newline
32. म॒न्य॒स्व॒ सु॒यजा॑ सु॒यजा॑ मन्यस्व मन्यस्व सु॒यजा॑ यजाम यजाम सु॒यजा॑ मन्यस्व मन्यस्व सु॒यजा॑ यजाम । \newline
33. सु॒यजा॑ यजाम यजाम सु॒यजा॑ सु॒यजा॑ यजाम॒ जुष्ट॒म् जुष्ट॑म् ॅयजाम सु॒यजा॑ सु॒यजा॑ यजाम॒ जुष्ट᳚म् । \newline
34. सु॒यजेति॑ सु - यजा᳚ । \newline
35. य॒जा॒म॒ जुष्ट॒म् जुष्ट॑म् ॅयजाम यजाम॒ जुष्ट॑म् दे॒वाना᳚म् दे॒वाना॒म् जुष्ट॑म् ॅयजाम यजाम॒ जुष्ट॑म् दे॒वाना᳚म् । \newline
36. जुष्ट॑म् दे॒वाना᳚म् दे॒वाना॒म् जुष्ट॒म् जुष्ट॑म् दे॒वाना॑ मि॒द मि॒दम् दे॒वाना॒म् जुष्ट॒म् जुष्ट॑म् दे॒वाना॑ मि॒दम् । \newline
37. दे॒वाना॑ मि॒द मि॒दम् दे॒वाना᳚म् दे॒वाना॑ मि॒द म॑स्त्व स्त्वि॒दम् दे॒वाना᳚म् दे॒वाना॑ मि॒द म॑स्तु । \newline
38. इ॒द म॑स्त्वस्त्वि॒द मि॒द म॑स्तु ह॒व्यꣳ ह॒व्य म॑स्त्वि॒द मि॒द म॑स्तु ह॒व्यम् । \newline
39. अ॒स्तु॒ ह॒व्यꣳ ह॒व्य म॑स्त्वस्तु ह॒व्यम् । \newline
40. ह॒व्यमिति॑ ह॒व्यम् । \newline
41. दे॒वाना॑ मे॒ष ए॒ष दे॒वाना᳚म् दे॒वाना॑ मे॒ष उ॑पना॒ह उ॑पना॒ह ए॒ष दे॒वाना᳚म् दे॒वाना॑ मे॒ष उ॑पना॒हः । \newline
42. ए॒ष उ॑पना॒ह उ॑पना॒ह ए॒ष ए॒ष उ॑पना॒ह आ॑सी दासी दुपना॒ह ए॒ष ए॒ष उ॑पना॒ह आ॑सीत् । \newline
43. उ॒प॒ना॒ह आ॑सी दासी दुपना॒ह उ॑पना॒ह आ॑सी द॒पा म॒पा मा॑सी दुपना॒ह उ॑पना॒ह आ॑सी द॒पाम् । \newline
44. उ॒प॒ना॒ह इत्यु॑प - ना॒हः । \newline
45. आ॒सी॒ द॒पा म॒पा मा॑सी दासी द॒पाम् गर्भो॒ गर्भो॒ ऽपा मा॑सी दासी द॒पाम् गर्भः॑ । \newline
46. अ॒पाम् गर्भो॒ गर्भो॒ ऽपा म॒पाम् गर्भ॒ ओष॑धी॒ ष्वोष॑धीषु॒ गर्भो॒ ऽपा म॒पाम् गर्भ॒ ओष॑धीषु । \newline
47. गर्भ॒ ओष॑धी॒ ष्वोष॑धीषु॒ गर्भो॒ गर्भ॒ ओष॑धीषु॒ न्य॑क्तो॒ न्य॑क्त॒ ओष॑धीषु॒ गर्भो॒ गर्भ॒ ओष॑धीषु॒ न्य॑क्तः । \newline
48. ओष॑धीषु॒ न्य॑क्तो॒ न्य॑क्त॒ ओष॑धी॒ ष्वोष॑धीषु॒ न्य॑क्तः । \newline
49. न्य॑क्त॒ इति॒ नि - अ॒क्तः॒ । \newline
50. सोम॑स्य द्र॒फ्सम् द्र॒फ्सꣳ सोम॑स्य॒ सोम॑स्य द्र॒फ्स म॑वृणीता वृणीत द्र॒फ्सꣳ सोम॑स्य॒ सोम॑स्य द्र॒फ्स म॑वृणीत । \newline
51. द्र॒फ्स म॑वृणीता वृणीत द्र॒फ्सम् द्र॒फ्स म॑वृणीत पू॒षा पू॒षा ऽवृ॑णीत द्र॒फ्सम् द्र॒फ्स म॑वृणीत पू॒षा । \newline
52. अ॒वृ॒णी॒त॒ पू॒षा पू॒षा ऽवृ॑णीता वृणीत पू॒षा बृ॒हन् बृ॒हन् पू॒षा ऽवृ॑णीता वृणीत पू॒षा बृ॒हन्न् । \newline
53. पू॒षा बृ॒हन् बृ॒हन् पू॒षा पू॒षा बृ॒हन्, नद्रि॒ रद्रि॑र् बृ॒हन् पू॒षा पू॒षा बृ॒हन्, नद्रिः॑ । \newline
\pagebreak
\markright{ TS 3.3.9.2  \hfill https://www.vedavms.in \hfill}

\section{ TS 3.3.9.2 }

\textbf{TS 3.3.9.2 } \newline
\textbf{Samhita Paata} \newline

बृ॒हन्नद्रि॑रभव॒त् तदे॑षां ॥ पि॒ता व॒थ्सानां॒ पति॑रघ्नि॒याना॒मथो॑ पि॒ता म॑ह॒तां गर्ग॑राणां । व॒थ्सो ज॒रायु॑ प्रति॒धुक् पी॒यूष॑ आ॒मिक्षा॒ मस्तु॑ घृ॒तम॑स्य॒ रेतः॑ ॥ त्वां गावो॑ऽवृणत रा॒ज्याय॒ त्वाꣳ ह॑वन्त म॒रुतः॑ स्व॒र्काः । वर्ष्म॑न् क्ष॒त्रस्य॑ क॒कुभि॑ शिश्रिया॒णस्ततो॑ न उ॒ग्रो वि भ॑जा॒ वसू॑नि ॥ व्यृ॑द्धेन॒ वा ए॒ष प॒शुना॑ यजते॒ यस्यै॒तानि॒ न क्रि॒यन्त॑ ए॒ष ( ) ह॒ त्वै समृ॑द्धेन यजते॒ यस्यै॒तानि॑ क्रि॒यन्ते᳚ ॥ \newline

\textbf{Pada Paata} \newline

बृ॒हन्न् । अद्रिः॑ । अ॒भ॒व॒त् । तत् । ए॒षा॒म् ॥ पि॒ता । व॒थ्साना᳚म् । पतिः॑ । अ॒घ्नि॒याना᳚म् । अथो॒ इति॑ । पि॒ता । म॒ह॒ताम् । गर्ग॑राणाम् ॥ व॒थ्सः । ज॒रायु॑ । प्र॒ति॒धुगिति॑ प्रति - धुक् । पी॒यूषः॑ । आ॒मिक्षा᳚ । मस्तु॑ । घृ॒तम् । अ॒स्य॒ । रेतः॑ ॥ त्वाम् । गावः॑ । अ॒वृ॒ण॒त॒ । रा॒ज्याय॑ । त्वाम् । ह॒व॒न्त॒ । म॒रुतः॑ । स्व॒र्का इति॑ सु - अ॒र्काः ॥ वर्ष्मन्न्॑ । क्ष॒त्रस्य॑ । क॒कुभि॑ । शि॒श्रि॒या॒णः । ततः॑ । नः॒ । उ॒ग्रः । वीति॑ । भ॒ज॒ । वसू॑नि ॥ व्यृ॑द्धे॒नेति॒ वि - ऋ॒द्धे॒न॒ । वै । ए॒षः । प॒शुना᳚ । य॒ज॒ते॒ । यस्य॑ । ए॒तानि॑ । न । क्रि॒यन्ते᳚ । ए॒षः ( ) । ह॒ । तु । वै । समृ॑द्धे॒नेति॒ सं - ऋ॒द्धे॒न॒ । य॒ज॒ते॒ । यस्य॑ । ए॒तानि॑ । क्रि॒यन्ते᳚ ॥  \newline


\textbf{Krama Paata} \newline

बृ॒हन्नद्रिः॑ । अद्रि॑रभवत् । अ॒भ॒व॒त् तत् । तदे॑षाम् । ए॒षा॒मित्ये॑षाम् ॥ पि॒ता व॒थ्साना᳚म् । व॒थ्साना॒म् पतिः॑ । पति॑रघ्नि॒याना᳚म् । अ॒घ्नि॒याना॒मथो᳚ । अथो॑ पि॒ता । अथो॒ इत्यथो᳚ । पि॒ता म॑ह॒ताम् । म॒ह॒ताम् गर्ग॑राणाम् । गर्ग॑राणा॒मिति॒ गर्ग॑राणाम् ॥ व॒थ्सो ज॒रायु॑ । ज॒रायु॑ प्रति॒धुक् । प्र॒ति॒धुक् पी॒यूषः॑ । प्र॒ति॒धुगिति॑ प्रति - धुक् । पी॒यूष॑ आ॒मिक्षा᳚ । आ॒मिक्षा॒ मस्तु॑ । मस्तु॑ घृ॒तम् । घृ॒तम॑स्य । अ॒स्य॒ रेतः॑ । रेत॒ इति॒ रेतः॑ ॥ त्वाम् गावः॑ । गावो॑ ऽवृणत । अ॒वृ॒ण॒त॒ रा॒ज्याय॑ । रा॒ज्याय॒ त्वाम् । त्वाꣳ ह॑वन्त । ह॒व॒न्त॒ म॒रुतः॑ । म॒रुतः॑ स्व॒र्काः । स्व॒र्का इति॑ सु - अ॒र्काः ॥ वर्ष्म॑न् क्ष॒त्रस्य॑ । क्ष॒त्रस्य॑ क॒कुभि॑ । क॒कुभि॑ शिश्रिया॒णः । शि॒श्रि॒या॒ण स्ततः॑ । ततो॑ नः । न उ॒ग्रः । उ॒ग्रो वि । वि भ॑ज । भ॒जा॒ वसू॑नि । वसू॒नीति॒ वसू॑नि ॥ व्यृ॑द्धेन॒ वै । व्यृ॑द्धे॒नेति॒ वि - ऋ॒द्धे॒न॒ । वा ए॒षः । ए॒ष प॒शुना᳚ । प॒शुना॑ यजते । य॒ज॒ते॒ यस्य॑ । यस्यै॒तानि॑ । ए॒तानि॒ न । न क्रि॒यन्ते᳚ । क्रि॒यन्त॑ ए॒षः ( ) । ए॒ष ह॑ । ह॒ तु । त्वै । वै समृ॑द्धेन । समृ॑द्धेन यजते । समृ॑द्धे॒नेति॒ सम् - ऋ॒द्धे॒न॒ । य॒ज॒ते॒ यस्य॑ । यस्यै॒तानि॑ । ए॒तानि॑ क्रि॒यन्ते᳚ । क्रि॒यन्त॒ इति॑ क्रि॒यन्ते᳚ । \newline

\textbf{Jatai Paata} \newline

1. बृ॒हन् नद्रि॒ रद्रि॑र् बृ॒हन् बृ॒हन् नद्रिः॑ । \newline
2. अद्रि॑ रभव दभव॒ दद्रि॒ रद्रि॑ रभवत् । \newline
3. अ॒भ॒व॒त् तत् तद॑भव दभव॒त् तत् । \newline
4. तदे॑षा मेषा॒म् तत् तदे॑षाम् । \newline
5. ए॒षा॒मित्ये॑षाम् । \newline
6. पि॒ता व॒थ्साना᳚म् ॅव॒थ्साना᳚म् पि॒ता पि॒ता व॒थ्साना᳚म् । \newline
7. व॒थ्साना॒म् पति॒ष् पति॑र् व॒थ्साना᳚म् ॅव॒थ्साना॒म् पतिः॑ । \newline
8. पति॑ रघ्नि॒याना॑ मघ्नि॒याना॒म् पति॒ष् पति॑ रघ्नि॒याना᳚म् । \newline
9. अ॒घ्नि॒याना॒ मथो॒ अथो॑ अघ्नि॒याना॑ मघ्नि॒याना॒ मथो᳚ । \newline
10. अथो॑ पि॒ता पि॒ता ऽथो॒ अथो॑ पि॒ता । \newline
11. अथो॒ इत्यथो᳚ । \newline
12. पि॒ता म॑ह॒ताम् म॑ह॒ताम् पि॒ता पि॒ता म॑ह॒ताम् । \newline
13. म॒ह॒ताम् गर्ग॑राणा॒म् गर्ग॑राणाम् मह॒ताम् म॑ह॒ताम् गर्ग॑राणाम् । \newline
14. गर्ग॑राणा॒मिति॒ गर्ग॑राणाम् । \newline
15. व॒थ्सो ज॒रायु॑ ज॒रायु॑ व॒थ्सो व॒थ्सो ज॒रायु॑ । \newline
16. ज॒रायु॑ प्रति॒धुक् प्र॑ति॒धुग् ज॒रायु॑ ज॒रायु॑ प्रति॒धुक् । \newline
17. प्र॒ति॒धुक् पी॒यूषः॑ पी॒यूषः॑ प्रति॒धुक् प्र॑ति॒धुक् पी॒यूषः॑ । \newline
18. प्र॒ति॒धुगिति॑ प्रति - धुक् । \newline
19. पी॒यूष॑ आ॒मिक्षा॒ ऽऽमिक्षा॑ पी॒यूषः॑ पी॒यूष॑ आ॒मिक्षा᳚ । \newline
20. आ॒मिक्षा॒ मस्तु॒ मस्त्वा॒ मिक्षा॒ ऽऽमिक्षा॒ मस्तु॑ । \newline
21. मस्तु॑ घृ॒तम् घृ॒तम् मस्तु॒ मस्तु॑ घृ॒तम् । \newline
22. घृ॒त म॑स्या स्य घृ॒तम् घृ॒त म॑स्य । \newline
23. अ॒स्य॒ रेतो॒ रेतो᳚ ऽस्यास्य॒ रेतः॑ । \newline
24. रेत॒ इति॒ रेतः॑ । \newline
25. त्वाम् गावो॒ गाव॒ स्त्वाम् त्वाम् गावः॑ । \newline
26. गावो॑ ऽवृणता वृणत॒ गावो॒ गावो॑ ऽवृणत । \newline
27. अ॒वृ॒ण॒त॒ रा॒ज्याय॑ रा॒ज्याया॑ वृणता वृणत रा॒ज्याय॑ । \newline
28. रा॒ज्याय॒ त्वाम् त्वाꣳ रा॒ज्याय॑ रा॒ज्याय॒ त्वाम् । \newline
29. त्वाꣳ ह॑वन्त हवन्त॒ त्वाम् त्वाꣳ ह॑वन्त । \newline
30. ह॒व॒न्त॒ म॒रुतो॑ म॒रुतो॑ हवन्त हवन्त म॒रुतः॑ । \newline
31. म॒रुतः॑ स्व॒र्काः स्व॒र्का म॒रुतो॑ म॒रुतः॑ स्व॒र्काः । \newline
32. स्व॒र्का इति॑ सु - अ॒र्काः । \newline
33. वर्ष्म॑न् क्ष॒त्रस्य॑ क्ष॒त्रस्य॒ वर्ष्म॒न्॒. वर्ष्म॑न् क्ष॒त्रस्य॑ । \newline
34. क्ष॒त्रस्य॑ क॒कुभि॑ क॒कुभि॑ क्ष॒त्रस्य॑ क्ष॒त्रस्य॑ क॒कुभि॑ । \newline
35. क॒कुभि॑ शिश्रिया॒णः शि॑श्रिया॒णः क॒कुभि॑ क॒कुभि॑ शिश्रिया॒णः । \newline
36. शि॒श्रि॒या॒ण स्तत॒ स्ततः॑ शिश्रिया॒णः शि॑श्रिया॒ण स्ततः॑ । \newline
37. ततो॑ नो न॒ स्तत॒ स्ततो॑ नः । \newline
38. न॒ उ॒ग्र उ॒ग्रो नो॑ न उ॒ग्रः । \newline
39. उ॒ग्रो वि व्यु॑ग्र उ॒ग्रो वि । \newline
40. वि भ॑ज भज॒ वि वि भ॑ज । \newline
41. भ॒जा॒ वसू॑नि॒ वसू॑नि भज भजा॒ वसू॑नि । \newline
42. वसू॒नीति॒ वसू॑नि । \newline
43. व्यृ॑द्धेन॒ वै वै व्यृ॑द्धेन॒ व्यृ॑द्धेन॒ वै । \newline
44. व्यृ॑द्धे॒नेति॒ वि - ऋ॒द्धे॒न॒ । \newline
45. वा ए॒ष ए॒ष वै वा ए॒षः । \newline
46. ए॒ष प॒शुना॑ प॒शुनै॒ष ए॒ष प॒शुना᳚ । \newline
47. प॒शुना॑ यजते यजते प॒शुना॑ प॒शुना॑ यजते । \newline
48. य॒ज॒ते॒ यस्य॒ यस्य॑ यजते यजते॒ यस्य॑ । \newline
49. यस्यै॒ता न्ये॒तानि॒ यस्य॒ यस्यै॒तानि॑ । \newline
50. ए॒तानि॒ न नैता न्ये॒तानि॒ न । \newline
51. न क्रि॒यन्ते᳚ क्रि॒यन्ते॒ न न क्रि॒यन्ते᳚ । \newline
52. क्रि॒यन्त॑ ए॒ष ए॒ष क्रि॒यन्ते᳚ क्रि॒यन्त॑ ए॒षः । \newline
53. ए॒ष ह॑ है॒ष ए॒ष ह॑ । \newline
54. ह॒ तु तु ह॑ ह॒ तु । \newline
55. त्वै वै तु त्वै । \newline
56. वै समृ॑द्धेन॒ समृ॑द्धेन॒ वै वै समृ॑द्धेन । \newline
57. समृ॑द्धेन यजते यजते॒ समृ॑द्धेन॒ समृ॑द्धेन यजते । \newline
58. समृ॑द्धे॒नेति॒ सम् - ऋ॒द्धे॒न॒ । \newline
59. य॒ज॒ते॒ यस्य॒ यस्य॑ यजते यजते॒ यस्य॑ । \newline
60. यस्यै॒ता न्ये॒तानि॒ यस्य॒ यस्यै॒तानि॑ । \newline
61. ए॒तानि॑ क्रि॒यन्ते᳚ क्रि॒यन्त॑ ए॒ता न्ये॒तानि॑ क्रि॒यन्ते᳚ । \newline
62. क्रि॒यन्त॒ इति॑ क्रि॒यन्ते᳚ । \newline

\textbf{Ghana Paata } \newline

1. बृ॒हन्, नद्रि॒ रद्रि॑र् बृ॒हन् बृ॒हन्, नद्रि॑ रभव दभव॒ दद्रि॑र् बृ॒हन् बृ॒हन्, नद्रि॑ रभवत् । \newline
2. अद्रि॑ रभव दभव॒ दद्रि॒ रद्रि॑ रभव॒त् तत् तद॑भव॒ दद्रि॒ रद्रि॑ रभव॒त् तत् । \newline
3. अ॒भ॒व॒त् तत् तद॑भव दभव॒त् तदे॑षा मेषा॒म् तद॑भव दभव॒त् तदे॑षाम् । \newline
4. तदे॑षा मेषा॒म् तत् तदे॑षाम् । \newline
5. ए॒षा॒मित्ये॑षाम् । \newline
6. पि॒ता व॒थ्साना᳚म् ॅव॒थ्साना᳚म् पि॒ता पि॒ता व॒थ्साना॒म् पति॒ष् पति॑र् व॒थ्साना᳚म् पि॒ता पि॒ता व॒थ्साना॒म् पतिः॑ । \newline
7. व॒थ्साना॒म् पति॒ष् पति॑र् व॒थ्साना᳚म् ॅव॒थ्साना॒म् पति॑ रघ्नि॒याना॑ मघ्नि॒याना॒म् पति॑र् व॒थ्साना᳚म् ॅव॒थ्साना॒म् पति॑ रघ्नि॒याना᳚म् । \newline
8. पति॑ रघ्नि॒याना॑ मघ्नि॒याना॒म् पति॒ष् पति॑ रघ्नि॒याना॒ मथो॒ अथो॑ अघ्नि॒याना॒म् पति॒ष् पति॑ रघ्नि॒याना॒ मथो᳚ । \newline
9. अ॒घ्नि॒याना॒ मथो॒ अथो॑ अघ्नि॒याना॑ मघ्नि॒याना॒ मथो॑ पि॒ता पि॒ता ऽथो॑ अघ्नि॒याना॑ मघ्नि॒याना॒ मथो॑ पि॒ता । \newline
10. अथो॑ पि॒ता पि॒ता ऽथो॒ अथो॑ पि॒ता म॑ह॒ताम् म॑ह॒ताम् पि॒ता ऽथो॒ अथो॑ पि॒ता म॑ह॒ताम् । \newline
11. अथो॒ इत्यथो᳚ । \newline
12. पि॒ता म॑ह॒ताम् म॑ह॒ताम् पि॒ता पि॒ता म॑ह॒ताम् गर्ग॑राणा॒म् गर्ग॑राणाम् मह॒ताम् पि॒ता पि॒ता म॑ह॒ताम् गर्ग॑राणाम् । \newline
13. म॒ह॒ताम् गर्ग॑राणा॒म् गर्ग॑राणाम् मह॒ताम् म॑ह॒ताम् गर्ग॑राणाम् । \newline
14. गर्ग॑राणा॒मिति॒ गर्ग॑राणाम् । \newline
15. व॒थ्सो ज॒रायु॑ ज॒रायु॑ व॒थ्सो व॒थ्सो ज॒रायु॑ प्रति॒धुक् प्र॑ति॒धुग् ज॒रायु॑ व॒थ्सो व॒थ्सो ज॒रायु॑ प्रति॒धुक् । \newline
16. ज॒रायु॑ प्रति॒धुक् प्र॑ति॒धुग् ज॒रायु॑ ज॒रायु॑ प्रति॒धुक् पी॒यूषः॑ पी॒यूषः॑ प्रति॒धुग् ज॒रायु॑ ज॒रायु॑ प्रति॒धुक् पी॒यूषः॑ । \newline
17. प्र॒ति॒धुक् पी॒यूषः॑ पी॒यूषः॑ प्रति॒धुक् प्र॑ति॒धुक् पी॒यूष॑ आ॒मिक्षा॒ ऽऽमिक्षा॑ पी॒यूषः॑ प्रति॒धुक् प्र॑ति॒धुक् पी॒यूष॑ आ॒मिक्षा᳚ । \newline
18. प्र॒ति॒धुगिति॑ प्रति - धुक् । \newline
19. पी॒यूष॑ आ॒मिक्षा॒ ऽऽमिक्षा॑ पी॒यूषः॑ पी॒यूष॑ आ॒मिक्षा॒ मस्तु॒ मस्त्वा॒ मिक्षा॑ पी॒यूषः॑ पी॒यूष॑ आ॒मिक्षा॒ मस्तु॑ । \newline
20. आ॒मिक्षा॒ मस्तु॒ मस्त्वा॒ मिक्षा॒ ऽऽमिक्षा॒ मस्तु॑ घृ॒तम् घृ॒तम् मस्त्वा॒ मिक्षा॒ ऽऽमिक्षा॒ मस्तु॑ घृ॒तम् । \newline
21. मस्तु॑ घृ॒तम् घृ॒तम् मस्तु॒ मस्तु॑ घृ॒त म॑स्यास्य घृ॒तम् मस्तु॒ मस्तु॑ घृ॒त म॑स्य । \newline
22. घृ॒त म॑स्यास्य घृ॒तम् घृ॒त म॑स्य॒ रेतो॒ रेतो᳚ ऽस्य घृ॒तम् घृ॒त म॑स्य॒ रेतः॑ । \newline
23. अ॒स्य॒ रेतो॒ रेतो᳚ ऽस्यास्य॒ रेतः॑ । \newline
24. रेत॒ इति॒ रेतः॑ । \newline
25. त्वाम् गावो॒ गाव॒ स्त्वाम् त्वाम् गावो॑ ऽवृणता वृणत॒ गाव॒ स्त्वाम् त्वाम् गावो॑ ऽवृणत । \newline
26. गावो॑ ऽवृणता वृणत॒ गावो॒ गावो॑ ऽवृणत रा॒ज्याय॑ रा॒ज्याया॑ वृणत॒ गावो॒ गावो॑ ऽवृणत रा॒ज्याय॑ । \newline
27. अ॒वृ॒ण॒त॒ रा॒ज्याय॑ रा॒ज्याया॑ वृणता वृणत रा॒ज्याय॒ त्वाम् त्वाꣳ रा॒ज्याया॑ वृणता वृणत रा॒ज्याय॒ त्वाम् । \newline
28. रा॒ज्याय॒ त्वाम् त्वाꣳ रा॒ज्याय॑ रा॒ज्याय॒ त्वाꣳ ह॑वन्त हवन्त॒ त्वाꣳ रा॒ज्याय॑ रा॒ज्याय॒ त्वाꣳ ह॑वन्त । \newline
29. त्वाꣳ ह॑वन्त हवन्त॒ त्वाम् त्वाꣳ ह॑वन्त म॒रुतो॑ म॒रुतो॑ हवन्त॒ त्वाम् त्वाꣳ ह॑वन्त म॒रुतः॑ । \newline
30. ह॒व॒न्त॒ म॒रुतो॑ म॒रुतो॑ हवन्त हवन्त म॒रुतः॑ स्व॒र्काः स्व॒र्का म॒रुतो॑ हवन्त हवन्त म॒रुतः॑ स्व॒र्काः । \newline
31. म॒रुतः॑ स्व॒र्काः स्व॒र्का म॒रुतो॑ म॒रुतः॑ स्व॒र्काः । \newline
32. स्व॒र्का इति॑ सु - अ॒र्काः । \newline
33. वर्ष्म॑न् क्ष॒त्रस्य॑ क्ष॒त्रस्य॒ वर्ष्म॒न्॒. वर्ष्म॑न् क्ष॒त्रस्य॑ क॒कुभि॑ क॒कुभि॑ क्ष॒त्रस्य॒ वर्ष्म॒न्॒. वर्ष्म॑न् क्ष॒त्रस्य॑ क॒कुभि॑ । \newline
34. क्ष॒त्रस्य॑ क॒कुभि॑ क॒कुभि॑ क्ष॒त्रस्य॑ क्ष॒त्रस्य॑ क॒कुभि॑ शिश्रिया॒णः शि॑श्रिया॒णः क॒कुभि॑ क्ष॒त्रस्य॑ क्ष॒त्रस्य॑ क॒कुभि॑ शिश्रिया॒णः । \newline
35. क॒कुभि॑ शिश्रिया॒णः शि॑श्रिया॒णः क॒कुभि॑ क॒कुभि॑ शिश्रिया॒ण स्तत॒ स्ततः॑ शिश्रिया॒णः क॒कुभि॑ क॒कुभि॑ शिश्रिया॒ण स्ततः॑ । \newline
36. शि॒श्रि॒या॒ण स्तत॒ स्ततः॑ शिश्रिया॒णः शि॑श्रिया॒ण स्ततो॑ नो न॒ स्ततः॑ शिश्रिया॒णः शि॑श्रिया॒ण स्ततो॑ नः । \newline
37. ततो॑ नो न॒ स्तत॒ स्ततो॑ न उ॒ग्र उ॒ग्रो न॒ स्तत॒ स्ततो॑ न उ॒ग्रः । \newline
38. न॒ उ॒ग्र उ॒ग्रो नो॑ न उ॒ग्रो वि व्यु॑ग्रो नो॑ न उ॒ग्रो वि । \newline
39. उ॒ग्रो वि व्यु॑ग्र उ॒ग्रो वि भ॑ज भज॒ व्यु॑ग्र उ॒ग्रो वि भ॑ज । \newline
40. वि भ॑ज भज॒ वि वि भ॑जा॒ वसू॑नि॒ वसू॑नि भज॒ वि वि भ॑जा॒ वसू॑नि । \newline
41. भ॒जा॒ वसू॑नि॒ वसू॑नि भज भजा॒ वसू॑नि । \newline
42. वसू॒नीति॒ वसू॑नि । \newline
43. व्यृ॑द्धेन॒ वै वै व्यृ॑द्धेन॒ व्यृ॑द्धेन॒ वा ए॒ष ए॒ष वै व्यृ॑द्धेन॒ व्यृ॑द्धेन॒ वा ए॒षः । \newline
44. व्यृ॑द्धे॒नेति॒ वि - ऋ॒द्धे॒न॒ । \newline
45. वा ए॒ष ए॒ष वै वा ए॒ष प॒शुना॑ प॒शुनै॒ष वै वा ए॒ष प॒शुना᳚ । \newline
46. ए॒ष प॒शुना॑ प॒शुनै॒ष ए॒ष प॒शुना॑ यजते यजते प॒शुनै॒ष ए॒ष प॒शुना॑ यजते । \newline
47. प॒शुना॑ यजते यजते प॒शुना॑ प॒शुना॑ यजते॒ यस्य॒ यस्य॑ यजते प॒शुना॑ प॒शुना॑ यजते॒ यस्य॑ । \newline
48. य॒ज॒ते॒ यस्य॒ यस्य॑ यजते यजते॒ यस्यै॒ता न्ये॒तानि॒ यस्य॑ यजते यजते॒ यस्यै॒तानि॑ । \newline
49. यस्यै॒ता न्ये॒तानि॒ यस्य॒ यस्यै॒तानि॒ न नैतानि॒ यस्य॒ यस्यै॒तानि॒ न । \newline
50. ए॒तानि॒ न नैता न्ये॒तानि॒ न क्रि॒यन्ते᳚ क्रि॒यन्ते॒ नैता न्ये॒तानि॒ न क्रि॒यन्ते᳚ । \newline
51. न क्रि॒यन्ते᳚ क्रि॒यन्ते॒ न न क्रि॒यन्त॑ ए॒ष ए॒ष क्रि॒यन्ते॒ न न क्रि॒यन्त॑ ए॒षः । \newline
52. क्रि॒यन्त॑ ए॒ष ए॒ष क्रि॒यन्ते᳚ क्रि॒यन्त॑ ए॒ष ह॑ है॒ष क्रि॒यन्ते᳚ क्रि॒यन्त॑ ए॒ष ह॑ । \newline
53. ए॒ष ह॑ है॒ष ए॒ष ह॒ तु तु है॒ष ए॒ष ह॒ तु । \newline
54. ह॒ तु तु ह॑ ह॒ त्वै वै तु ह॑ ह॒ त्वै । \newline
55. त्वै वै तु त्वै समृ॑द्धेन॒ समृ॑द्धेन॒ वै तु त्वै समृ॑द्धेन । \newline
56. वै समृ॑द्धेन॒ समृ॑द्धेन॒ वै वै समृ॑द्धेन यजते यजते॒ समृ॑द्धेन॒ वै वै समृ॑द्धेन यजते । \newline
57. समृ॑द्धेन यजते यजते॒ समृ॑द्धेन॒ समृ॑द्धेन यजते॒ यस्य॒ यस्य॑ यजते॒ समृ॑द्धेन॒ समृ॑द्धेन यजते॒ यस्य॑ । \newline
58. समृ॑द्धे॒नेति॒ सम् - ऋ॒द्धे॒न॒ । \newline
59. य॒ज॒ते॒ यस्य॒ यस्य॑ यजते यजते॒ यस्यै॒ता न्ये॒तानि॒ यस्य॑ यजते यजते॒ यस्यै॒तानि॑ । \newline
60. यस्यै॒ता न्ये॒तानि॒ यस्य॒ यस्यै॒तानि॑ क्रि॒यन्ते᳚ क्रि॒यन्त॑ ए॒तानि॒ यस्य॒ यस्यै॒तानि॑ क्रि॒यन्ते᳚ । \newline
61. ए॒तानि॑ क्रि॒यन्ते᳚ क्रि॒यन्त॑ ए॒ता न्ये॒तानि॑ क्रि॒यन्ते᳚ । \newline
62. क्रि॒यन्त॒ इति॑ क्रि॒यन्ते᳚ । \newline
\pagebreak
\markright{ TS 3.3.10.1  \hfill https://www.vedavms.in \hfill}

\section{ TS 3.3.10.1 }

\textbf{TS 3.3.10.1 } \newline
\textbf{Samhita Paata} \newline

सूर्यो॑ दे॒वो दि॑वि॒षद्भ्यो॑ धा॒ता क्ष॒त्राय॑ वा॒युः प्र॒जाभ्यः॑ । बृह॒स्पति॑स्त्वा प्र॒जाप॑तये॒ ज्योति॑ष्मतीं जुहोतु ॥ यस्या᳚स्ते॒ हरि॑तो॒ गर्भोऽथो॒ योनि॑र्.हिर॒ण्ययी᳚ । अङ्गा॒न्यह्रु॑ता॒ यस्यै॒ तां दे॒वैः सम॑जीगमं ॥आ व॑र्तन वर्तय॒ नि नि॑वर्तन वर्त॒येन्द्र॑ नर्दबुद ।भूम्या॒श्चत॑स्रः प्र॒दिश॒स्ताभि॒रा व॑र्तया॒ पुनः॑ ॥वि ते॑ भिनद्मि तक॒रीं ॅवियोनिं॒ ॅवि ग॑वी॒न्यौ᳚ । वि - [  ] \newline

\textbf{Pada Paata} \newline

सूर्यः॑ । दे॒वः । दि॒वि॒षद्भ्य॒ इति॑ दिवि॒षत् - भ्यः॒ । धा॒ता । क्ष॒त्राय॑ । वा॒युः । प्र॒जाभ्य॒ इति॑ प्र - जाभ्यः॑ ॥ बृह॒स्पतिः॑ । त्वा॒ । प्र॒जाप॑तय॒ इति॑ प्र॒जा-प॒त॒ये॒ । ज्योति॑ष्मतीम् । जु॒हो॒तु॒ ॥ यस्याः᳚ । ते॒ । हरि॑तः । गर्भः॑ । अथो॒ इति॑ । योनिः॑ । हि॒र॒ण्ययी᳚ ॥ अङ्गा॑नि । अह्रु॑ता । यस्यै᳚ । ताम् । दे॒वैः । समिति॑ । अ॒जी॒ग॒म॒म् ॥ एति॑ । व॒र्त॒न॒ । व॒र्त॒य॒ । नीति॑ । नि॒व॒र्त॒नेति॑ नि - व॒र्त॒न॒ । व॒र्त॒य॒ । इन्द्र॑ । न॒र्द॒बु॒द॒ ॥ भूम्याः᳚ । चत॑स्रः । प्र॒दिश॒ इति॑ प्र - दिशः॑ । ताभिः॑ । एति॑ । व॒र्त॒य॒ । पुनः॑ ॥ वीति॑ । ते॒ । भि॒न॒द्मि॒ । त॒क॒रीम् । वीति॑ । योनि᳚म् । वीति॑ । ग॒वी॒न्यौ᳚ ॥ वीति॑ ।  \newline


\textbf{Krama Paata} \newline

सूर्यो॑ दे॒वः । दे॒वो दि॑वि॒षद्भ्यः॑ । दि॒वि॒षद्भ्यो॑ धा॒ता । दि॒वि॒षद्भ्य॒ इति॑ दिवि॒षत् - भ्यः॒ । धा॒ता क्ष॒त्राय॑ । क्ष॒त्राय॑ वा॒युः । वा॒युः प्र॒जाभ्यः॑ । प्र॒जाभ्य॒ इति॑ प्र - जाभ्यः॑ ॥ बृह॒स्पति॑स्त्वा । त्वा॒ प्र॒जाप॑तये । प्र॒जाप॑तये॒ ज्योति॑ष्मतीम् । प्र॒जाप॑तय॒ इति॑ प्र॒जा - प॒त॒ये॒ । ज्योति॑ष्मतीम् जुहोतु । जु॒हो॒त्विति॑ जुहोतु ॥ यस्या᳚स्ते । ते॒ हरि॑तः । हरि॑तो॒ गर्भः॑ । गर्भो ऽथो᳚ । अथो॒ योनिः॑ । अथो॒ इत्यथो᳚ । योनि॑र् हिर॒ण्ययी᳚ । हि॒र॒ण्ययीति॑ हिर॒ण्ययी᳚ ॥ अङ्गा॒न्यह्रु॑ता । अह्रु॑ता॒ यस्यै᳚ । यस्यै॒ ताम् । ताम् दे॒वैः । दे॒वैः सम् । सम॑जीगमम् । अ॒जी॒ग॒म॒मित्य॑जीगमम् ॥ आ व॑र्तन । व॒र्त॒न॒ व॒र्त॒य॒ । व॒र्त॒य॒ नि । नि नि॑वर्तन । नि॒व॒र्त॒न॒ व॒र्त॒य॒ । नि॒व॒र्त॒नेति॑ नि - व॒र्त॒न॒ । व॒र्त॒येन्द्र॑ । इन्द्र॑ नर्दबुद । न॒र्द॒बु॒देति॑ नर्दबुद ॥ भूम्या॒ श्चत॑स्रः । चत॑स्रः प्र॒दिशः॑ । प्र॒दिश॒स्ताभिः॑ । प्र॒दिश॒ इति॑ प्र - दिशः॑ । ताभि॒रा । आ व॑र्तय । व॒र्त॒या॒ पुनः॑ । पुन॒रिति॒ पुनः॑ ॥ वि ते᳚ । ते॒ भि॒न॒द्मि॒ । भि॒न॒द्मि॒ त॒क॒रीम् । त॒क॒रीं ॅवि । वि योनि᳚म् । योनिं॒ ॅवि । वि ग॑वी॒न्यौ᳚ । ग॒वी॒न्या॑विति॑ गवी॒न्यौ᳚ ॥ वि मा॒तरं᳚ \newline

\textbf{Jatai Paata} \newline

1. सूर्यो॑ दे॒वो दे॒वः सूर्यः॒ सूर्यो॑ दे॒वः । \newline
2. दे॒वो दि॑वि॒षद्भ्यो॑ दिवि॒षद्भ्यो॑ दे॒वो दे॒वो दि॑वि॒षद्भ्यः॑ । \newline
3. दि॒वि॒षद्भ्यो॑ धा॒ता धा॒ता दि॑वि॒षद्भ्यो॑ दिवि॒षद्भ्यो॑ धा॒ता । \newline
4. दि॒वि॒षद्भ्य॒ इति॑ दिवि॒षत् - भ्यः॒ । \newline
5. धा॒ता क्ष॒त्राय॑ क्ष॒त्राय॑ धा॒ता धा॒ता क्ष॒त्राय॑ । \newline
6. क्ष॒त्राय॑ वा॒युर् वा॒युः क्ष॒त्राय॑ क्ष॒त्राय॑ वा॒युः । \newline
7. वा॒युः प्र॒जाभ्यः॑ प्र॒जाभ्यो॑ वा॒युर् वा॒युः प्र॒जाभ्यः॑ । \newline
8. प्र॒जाभ्य॒ इति॑ प्र - जाभ्यः॑ । \newline
9. बृह॒स्पति॑ स्त्वा त्वा॒ बृह॒स्पति॒र् बृह॒स्पति॑ स्त्वा । \newline
10. त्वा॒ प्र॒जाप॑तये प्र॒जाप॑तये त्वा त्वा प्र॒जाप॑तये । \newline
11. प्र॒जाप॑तये॒ ज्योति॑ष्मती॒म् ज्योति॑ष्मतीम् प्र॒जाप॑तये प्र॒जाप॑तये॒ ज्योति॑ष्मतीम् । \newline
12. प्र॒जाप॑तय॒ इति॑ प्र॒जा - प॒त॒ये॒ । \newline
13. ज्योति॑ष्मतीम् जुहोतु जुहोतु॒ ज्योति॑ष्मती॒म् ज्योति॑ष्मतीम् जुहोतु । \newline
14. जु॒हो॒त्विति॑ जुहोतु । \newline
15. यस्या᳚ स्ते ते॒ यस्या॒ यस्या᳚ स्ते । \newline
16. ते॒ हरि॑तो॒ हरि॑त स्ते ते॒ हरि॑तः । \newline
17. हरि॑तो॒ गर्भो॒ गर्भो॒ हरि॑तो॒ हरि॑तो॒ गर्भः॑ । \newline
18. गर्भो ऽथो॒ अथो॒ गर्भो॒ गर्भो ऽथो᳚ । \newline
19. अथो॒ योनि॒र् योनि॒ रथो॒ अथो॒ योनिः॑ । \newline
20. अथो॒ इत्यथो᳚ । \newline
21. योनि॑र्. हिर॒ण्ययी॑ हिर॒ण्ययी॒ योनि॒र् योनि॑र्. हिर॒ण्ययी᳚ । \newline
22. हि॒र॒ण्ययीति॑ हिर॒ण्ययी᳚ । \newline
23. अङ्गा॒ न्यह्रु॒ता ऽह्रु॒ता ऽङ्गा॒ न्यङ्गा॒ न्यह्रु॑ता । \newline
24. अह्रु॑ता॒ यस्यै॒ यस्या॒ अह्रु॒ता ऽह्रु॑ता॒ यस्यै᳚ । \newline
25. यस्यै॒ ताम् ताम् ॅयस्यै॒ यस्यै॒ ताम् । \newline
26. ताम् दे॒वैर् दे॒वै स्ताम् ताम् दे॒वैः । \newline
27. दे॒वैः सꣳ सम् दे॒वैर् दे॒वैः सम् । \newline
28. स म॑जीगम मजीगमꣳ॒॒ सꣳ स म॑जीगमम् । \newline
29. अ॒जी॒ग॒म॒मित्य॑जीगमम् । \newline
30. आ व॑र्तन वर्त॒ना व॑र्तन । \newline
31. व॒र्त॒न॒ व॒र्त॒य॒ व॒र्त॒य॒ व॒र्त॒न॒ व॒र्त॒न॒ व॒र्त॒य॒ । \newline
32. व॒र्त॒य॒ नि नि व॑र्तय वर्तय॒ नि । \newline
33. नि नि॑वर्तन निवर्तन॒ नि नि नि॑वर्तन । \newline
34. नि॒व॒र्त॒न॒ व॒र्त॒य॒ व॒र्त॒य॒ नि॒व॒र्त॒न॒ नि॒व॒र्त॒न॒ व॒र्त॒य॒ । \newline
35. नि॒व॒र्त॒नेति॑ नि - व॒र्त॒न॒ । \newline
36. व॒र्त॒ येन्द्रे न्द्र॑ वर्तय वर्त॒ येन्द्र॑ । \newline
37. इन्द्र॑ नर्दबुद नर्दबु॒दे न्द्रे न्द्र॑ नर्दबुद । \newline
38. न॒र्द॒बु॒देति॑ नर्दबुद । \newline
39. भूम्या॒ श्चत॑स्र॒ श्चत॑स्रो॒ भूम्या॒ भूम्या॒ श्चत॑स्रः । \newline
40. चत॑स्रः प्र॒दिशः॑ प्र॒दिश॒ श्चत॑स्र॒ श्चत॑स्रः प्र॒दिशः॑ । \newline
41. प्र॒दिश॒ स्ताभि॒ स्ताभिः॑ प्र॒दिशः॑ प्र॒दिश॒ स्ताभिः॑ । \newline
42. प्र॒दिश॒ इति॑ प्र - दिशः॑ । \newline
43. ताभि॒रा ताभि॒ स्ताभि॒रा । \newline
44. आ व॑र्तय वर्त॒या व॑र्तय । \newline
45. व॒र्त॒या॒ पुनः॒ पुन॑र् वर्तय वर्तया॒ पुनः॑ । \newline
46. पुन॒रिति॒ पुनः॑ । \newline
47. वि ते॑ ते॒ वि वि ते᳚ । \newline
48. ते॒ भि॒न॒द्मि॒ भि॒न॒द्मि॒ ते॒ ते॒ भि॒न॒द्मि॒ । \newline
49. भि॒न॒द्मि॒ त॒क॒रीम् त॑क॒रीम् भि॑नद्मि भिनद्मि तक॒रीम् । \newline
50. त॒क॒रीम् ॅवि वि त॑क॒रीम् त॑क॒रीम् ॅवि । \newline
51. वि योनि॒म् ॅयोनि॒म् ॅवि वि योनि᳚म् । \newline
52. योनि॒म् ॅवि वि योनि॒म् ॅयोनि॒म् ॅवि । \newline
53. वि ग॑वी॒न्यौ॑ गवी॒न्यौ॑ वि वि ग॑वी॒न्यौ᳚ । \newline
54. ग॒वी॒न्या॑विति॑ गवी॒न्यौ᳚ । \newline
55. वि मा॒तर॑म् मा॒तर॒म् ॅवि वि मा॒तर᳚म् । \newline

\textbf{Ghana Paata } \newline

1. सूर्यो॑ दे॒वो दे॒वः सूर्यः॒ सूर्यो॑ दे॒वो दि॑वि॒षद्भ्यो॑ दिवि॒षद्भ्यो॑ दे॒वः सूर्यः॒ सूर्यो॑ दे॒वो दि॑वि॒षद्भ्यः॑ । \newline
2. दे॒वो दि॑वि॒षद्भ्यो॑ दिवि॒षद्भ्यो॑ दे॒वो दे॒वो दि॑वि॒षद्भ्यो॑ धा॒ता धा॒ता दि॑वि॒षद्भ्यो॑ दे॒वो दे॒वो दि॑वि॒षद्भ्यो॑ धा॒ता । \newline
3. दि॒वि॒षद्भ्यो॑ धा॒ता धा॒ता दि॑वि॒षद्भ्यो॑ दिवि॒षद्भ्यो॑ धा॒ता क्ष॒त्राय॑ क्ष॒त्राय॑ धा॒ता दि॑वि॒षद्भ्यो॑ दिवि॒षद्भ्यो॑ धा॒ता क्ष॒त्राय॑ । \newline
4. दि॒वि॒षद्भ्य॒ इति॑ दिवि॒षत् - भ्यः॒ । \newline
5. धा॒ता क्ष॒त्राय॑ क्ष॒त्राय॑ धा॒ता धा॒ता क्ष॒त्राय॑ वा॒युर् वा॒युः क्ष॒त्राय॑ धा॒ता धा॒ता क्ष॒त्राय॑ वा॒युः । \newline
6. क्ष॒त्राय॑ वा॒युर् वा॒युः क्ष॒त्राय॑ क्ष॒त्राय॑ वा॒युः प्र॒जाभ्यः॑ प्र॒जाभ्यो॑ वा॒युः क्ष॒त्राय॑ क्ष॒त्राय॑ वा॒युः प्र॒जाभ्यः॑ । \newline
7. वा॒युः प्र॒जाभ्यः॑ प्र॒जाभ्यो॑ वा॒युर् वा॒युः प्र॒जाभ्यः॑ । \newline
8. प्र॒जाभ्य॒ इति॑ प्र - जाभ्यः॑ । \newline
9. बृह॒स्पति॑ स्त्वा त्वा॒ बृह॒स्पति॒र् बृह॒स्पति॑ स्त्वा प्र॒जाप॑तये प्र॒जाप॑तये त्वा॒ बृह॒स्पति॒र् बृह॒स्पति॑ स्त्वा प्र॒जाप॑तये । \newline
10. त्वा॒ प्र॒जाप॑तये प्र॒जाप॑तये त्वा त्वा प्र॒जाप॑तये॒ ज्योति॑ष्मती॒म् ज्योति॑ष्मतीम् प्र॒जाप॑तये त्वा त्वा प्र॒जाप॑तये॒ ज्योति॑ष्मतीम् । \newline
11. प्र॒जाप॑तये॒ ज्योति॑ष्मती॒म् ज्योति॑ष्मतीम् प्र॒जाप॑तये प्र॒जाप॑तये॒ ज्योति॑ष्मतीम् जुहोतु जुहोतु॒ ज्योति॑ष्मतीम् प्र॒जाप॑तये प्र॒जाप॑तये॒ ज्योति॑ष्मतीम् जुहोतु । \newline
12. प्र॒जाप॑तय॒ इति॑ प्र॒जा - प॒त॒ये॒ । \newline
13. ज्योति॑ष्मतीम् जुहोतु जुहोतु॒ ज्योति॑ष्मती॒म् ज्योति॑ष्मतीम् जुहोतु । \newline
14. जु॒हो॒त्विति॑ जुहोतु । \newline
15. यस्या᳚ स्ते ते॒ यस्या॒ यस्या᳚ स्ते॒ हरि॑तो॒ हरि॑त स्ते॒ यस्या॒ यस्या᳚ स्ते॒ हरि॑तः । \newline
16. ते॒ हरि॑तो॒ हरि॑त स्ते ते॒ हरि॑तो॒ गर्भो॒ गर्भो॒ हरि॑त स्ते ते॒ हरि॑तो॒ गर्भः॑ । \newline
17. हरि॑तो॒ गर्भो॒ गर्भो॒ हरि॑तो॒ हरि॑तो॒ गर्भो ऽथो॒ अथो॒ गर्भो॒ हरि॑तो॒ हरि॑तो॒ गर्भो ऽथो᳚ । \newline
18. गर्भो ऽथो॒ अथो॒ गर्भो॒ गर्भो ऽथो॒ योनि॒र् योनि॒ रथो॒ गर्भो॒ गर्भो ऽथो॒ योनिः॑ । \newline
19. अथो॒ योनि॒र् योनि॒ रथो॒ अथो॒ योनि॑र्. हिर॒ण्ययी॑ हिर॒ण्ययी॒ योनि॒ रथो॒ अथो॒ योनि॑र्. हिर॒ण्ययी᳚ । \newline
20. अथो॒ इत्यथो᳚ । \newline
21. योनि॑र्. हिर॒ण्ययी॑ हिर॒ण्ययी॒ योनि॒र् योनि॑र्. हिर॒ण्ययी᳚ । \newline
22. हि॒र॒ण्ययीति॑ हिर॒ण्ययी᳚ । \newline
23. अङ्गा॒ न्यह्रु॒ता ऽह्रु॒ता ऽङ्गा॒ न्यङ्गा॒ न्यह्रु॑ता॒ यस्यै॒ यस्या॒ अह्रु॒ता ऽङ्गा॒ न्यङ्गा॒ न्यह्रु॑ता॒ यस्यै᳚ । \newline
24. अह्रु॑ता॒ यस्यै॒ यस्या॒ अह्रु॒ता ऽह्रु॑ता॒ यस्यै॒ ताम् ताम् ॅयस्या॒ अह्रु॒ता ऽह्रु॑ता॒ यस्यै॒ ताम् । \newline
25. यस्यै॒ ताम् ताम् ॅयस्यै॒ यस्यै॒ ताम् दे॒वैर् दे॒वै स्ताम् ॅयस्यै॒ यस्यै॒ ताम् दे॒वैः । \newline
26. ताम् दे॒वैर् दे॒वै स्ताम् ताम् दे॒वैः सꣳ सम् दे॒वै स्ताम् ताम् दे॒वैः सम् । \newline
27. दे॒वैः सꣳ सम् दे॒वैर् दे॒वैः स म॑जीगम मजीगमꣳ॒॒ सम् दे॒वैर् दे॒वैः स म॑जीगमम् । \newline
28. स म॑जीगम मजीगमꣳ॒॒ सꣳ स म॑जीगमम् । \newline
29. अ॒जी॒ग॒म॒मित्य॑जीगमम् । \newline
30. आ व॑र्तन वर्त॒ना व॑र्तन वर्तय वर्तय वर्त॒ना व॑र्तन वर्तय । \newline
31. व॒र्त॒न॒ व॒र्त॒य॒ व॒र्त॒य॒ व॒र्त॒न॒ व॒र्त॒न॒ व॒र्त॒य॒ नि नि व॑र्तय वर्तन वर्तन वर्तय॒ नि । \newline
32. व॒र्त॒य॒ नि नि व॑र्तय वर्तय॒ नि नि॑वर्तन निवर्तन॒ नि व॑र्तय वर्तय॒ नि नि॑वर्तन । \newline
33. नि नि॑वर्तन निवर्तन॒ नि नि नि॑वर्तन वर्तय वर्तय निवर्तन॒ नि नि नि॑वर्तन वर्तय । \newline
34. नि॒व॒र्त॒न॒ व॒र्त॒य॒ व॒र्त॒य॒ नि॒व॒र्त॒न॒ नि॒व॒र्त॒न॒ व॒र्त॒ये न्द्रे न्द्र॑ वर्तय निवर्तन निवर्तन वर्त॒ये न्द्र॑ । \newline
35. नि॒व॒र्त॒नेति॑ नि - व॒र्त॒न॒ । \newline
36. व॒र्त॒ये न्द्रे न्द्र॑ वर्तय वर्त॒ये न्द्र॑ नर्दबुद नर्दबु॒दे न्द्र॑ वर्तय वर्त॒ये न्द्र॑ नर्दबुद । \newline
37. इन्द्र॑ नर्दबुद नर्दबु॒दे न्द्रे न्द्र॑ नर्दबुद । \newline
38. न॒र्द॒बु॒देति॑ नर्दबुद । \newline
39. भूम्या॒ श्चत॑स्र॒ श्चत॑स्रो॒ भूम्या॒ भूम्या॒ श्चत॑स्रः प्र॒दिशः॑ प्र॒दिश॒ श्चत॑स्रो॒ भूम्या॒ भूम्या॒ श्चत॑स्रः प्र॒दिशः॑ । \newline
40. चत॑स्रः प्र॒दिशः॑ प्र॒दिश॒ श्चत॑स्र॒ श्चत॑स्रः प्र॒दिश॒ स्ताभि॒ स्ताभिः॑ प्र॒दिश॒ श्चत॑स्र॒ श्चत॑स्रः प्र॒दिश॒ स्ताभिः॑ । \newline
41. प्र॒दिश॒ स्ताभि॒ स्ताभिः॑ प्र॒दिशः॑ प्र॒दिश॒ स्ताभि॒रा ताभिः॑ प्र॒दिशः॑ प्र॒दिश॒ स्ताभि॒रा । \newline
42. प्र॒दिश॒ इति॑ प्र - दिशः॑ । \newline
43. ताभि॒रा ताभि॒ स्ताभि॒रा व॑र्तय वर्त॒या ताभि॒ स्ताभि॒रा व॑र्तय । \newline
44. आ व॑र्तय वर्त॒या व॑र्तया॒ पुनः॒ पुन॑र् वर्त॒या व॑र्तया॒ पुनः॑ । \newline
45. व॒र्त॒या॒ पुनः॒ पुन॑र् वर्तय वर्तया॒ पुनः॑ । \newline
46. पुन॒रिति॒ पुनः॑ । \newline
47. वि ते॑ ते॒ वि वि ते॑ भिनद्मि भिनद्मि ते॒ वि वि ते॑ भिनद्मि । \newline
48. ते॒ भि॒न॒द्मि॒ भि॒न॒द्मि॒ ते॒ ते॒ भि॒न॒द्मि॒ त॒क॒रीम् त॑क॒रीम् भि॑नद्मि ते ते भिनद्मि तक॒रीम् । \newline
49. भि॒न॒द्मि॒ त॒क॒रीम् त॑क॒रीम् भि॑नद्मि भिनद्मि तक॒रीम् ॅवि वि त॑क॒रीम् भि॑नद्मि भिनद्मि तक॒रीम् ॅवि । \newline
50. त॒क॒रीम् ॅवि वि त॑क॒रीम् त॑क॒रीम् ॅवि योनि॒म् ॅयोनि॒म् ॅवि त॑क॒रीम् त॑क॒रीम् ॅवि योनि᳚म् । \newline
51. वि योनि॒म् ॅयोनि॒म् ॅवि वि योनि॒म् ॅवि वि योनि॒म् ॅवि वि योनि॒म् ॅवि । \newline
52. योनि॒म् ॅवि वि योनि॒म् ॅयोनि॒म् ॅवि ग॑वी॒न्यौ॑ गवी॒न्यौ॑ वि योनि॒म् ॅयोनि॒म् ॅवि ग॑वी॒न्यौ᳚ । \newline
53. वि ग॑वी॒न्यौ॑ गवी॒न्यौ॑ वि वि ग॑वी॒न्यौ᳚ । \newline
54. ग॒वी॒न्या॑विति॑ गवी॒न्यौ᳚ । \newline
55. वि मा॒तर॑म् मा॒तर॒म् ॅवि वि मा॒तर॑म् च च मा॒तर॒म् ॅवि वि मा॒तर॑म् च । \newline
\pagebreak
\markright{ TS 3.3.10.2  \hfill https://www.vedavms.in \hfill}

\section{ TS 3.3.10.2 }

\textbf{TS 3.3.10.2 } \newline
\textbf{Samhita Paata} \newline

मा॒तर॑ञ्च पु॒त्रं च॒ वि गर्भं॑ च ज॒रायु॑ च ॥ ब॒हिस्ते॑ अस्तु॒ बालिति॑ ॥ उ॒रु॒द्र॒फ्सो वि॒श्वरू॑प॒ इन्दुः॒ पव॑मानो॒ धीर॑ आनञ्ज॒ गर्भं᳚ ॥एक॑पदी द्वि॒पदी᳚ त्रि॒पदी॒ चतु॑ष्पदी॒ पञ्च॑पदी॒ षट्प॑दी स॒प्तप॑द्य॒ष्टाप॑दी॒ भुव॒नाऽनु॑ प्रथताꣳ॒॒ स्वाहा᳚ ॥ म॒ही द्यौः पृ॑थि॒वी च॑ न इ॒मं ॅय॒ज्ञ्ं मि॑मिक्षतां । पि॒पृ॒तान्नो॒ भरी॑मभिः ॥ \newline

\textbf{Pada Paata} \newline

मा॒तर᳚म् । च॒ । पु॒त्रम् । च॒ । वीति॑ । गर्भ᳚म् । च॒ । ज॒रायु॑ । च॒ ॥ ब॒हिः । ते॒ । अ॒स्तु॒ । बाल् । इति॑ ॥ उ॒रु॒द्र॒फ्स इत्यु॑रु - द्र॒फ्सः । वि॒श्वरू॑प॒ इति॑ वि॒श्व - रू॒पः॒ । इन्दुः॑ । पव॑मानः । धीरः॑ । आ॒न॒ञ्ज॒ । गर्भ᳚म् ॥ एक॑प॒दीत्येक॑ - प॒दी॒ । द्वि॒पदीति॑ द्वि- पदी᳚ । त्रि॒पदीति॑ त्रि - पदी᳚ । चतु॑ष्प॒दीति॒ चतुः॑ - प॒दी॒ । पञ्च॑प॒दीति॒ पञ्च॑-प॒दी॒ । षट्प॒दीति॒ षट् - प॒दी॒ । स॒प्तप॒दीति॑ स॒प्त - प॒दी॒ । अ॒ष्टाप॒दीत्य॒ष्टा-प॒दी॒ । भुव॑ना । अन्विति॑ । प्र॒थ॒ता॒म् । स्वाहा᳚ ॥ म॒ही । द्यौः । पृ॒थि॒वी । च॒ । नः॒ । इ॒मम् । य॒ज्ञ्म् । मि॒मि॒क्ष॒ता॒म् ॥ पि॒पृ॒ताम् । नः॒ । भरी॑मभि॒रिति॒ भरी॑म - भिः॒ ॥  \newline


\textbf{Krama Paata} \newline

मा॒तर॑म् च । च॒ पु॒त्रम् । पु॒त्रम् च॑ । च॒ वि । वि गर्भ᳚म् । गर्भ॑म् च । च॒ ज॒रायु॑ । ज॒रायु॑ च । चेति॑ च ॥ ब॒हिस्ते᳚ । ते॒ अ॒स्तु॒ । अ॒स्तु॒ बाल् । बालिति॑ । इतीतीति॑ ॥ उ॒रु॒द्र॒फ्सो वि॒श्वरू॑पः । उ॒रु॒द्र॒फ्स इत्यु॑रु - द्र॒फ्सः । वि॒श्वरू॑प॒ इन्दुः॑ । वि॒श्वरू॑प॒ इति॑ वि॒श्व - रू॒पः॒ । इन्दुः॒ पव॑मानः । पव॑मानो॒ धीरः॑ । धीर॑ आनञ्ज । आ॒न॒ञ्ज॒ गर्भ᳚म् । गर्भ॒मिति॒ गर्भ᳚म् ॥ एक॑पदी द्वि॒पदी᳚ । एक॑प॒दीत्येक॑ - प॒दी॒ । द्वि॒पदी᳚ त्रि॒पदी᳚ । द्वि॒पदीति॑ द्वि - पदी᳚ । त्रि॒पदी॒ चतु॑ष्पदी । त्रि॒पदीति॑ त्रि - पदी᳚ । चतु॑ष्पदी॒ पञ्च॑पदी । चतु॑ष्प॒दीति॒ चतुः॑ - प॒दी॒ । पञ्च॑पदी॒ षट्प॑दी । पञ्च॑प॒दीति॒ पञ्च॑ - प॒दी॒ । षट्प॑दी स॒प्तप॑दी । षट्प॒दीति॒ षट् - प॒दी॒ । स॒प्तप॑द्य॒ष्टाप॑दी । स॒प्तप॒दीति॑ स॒प्त - प॒दी॒ । अ॒ष्टाप॑दी॒ भुव॑ना । अ॒ष्टाप॒दीत्य॒ष्टा - प॒दी॒ । भुव॒ना ऽनु॑ । अनु॑ प्रथताम् । प्र॒थ॒ताꣳ॒॒ स्वाहा᳚ । स्वाहेति॒ स्वाहा᳚ ॥ म॒ही द्यौः । द्यौः पृ॑थि॒वी । पृ॒थि॒वी च॑ । च॒ नः॒ । न॒ इ॒मम् । इ॒मं ॅय॒ज्ञ्म् । य॒ज्ञ्म् मि॑मिक्षताम् । मि॒मि॒क्ष॒ता॒मिति॑ मिमिक्षताम् ॥ पि॒पृ॒ताम् नः॑ । नो॒ भरी॑मभिः । भरी॑मभि॒रिति॒ भरी॑म - भिः॒ । \newline

\textbf{Jatai Paata} \newline

1. मा॒तर॑म् च च मा॒तर॑म् मा॒तर॑म् च । \newline
2. च॒ पु॒त्रम् पु॒त्रम् च॑ च पु॒त्रम् । \newline
3. पु॒त्रम् च॑ च पु॒त्रम् पु॒त्रम् च॑ । \newline
4. च॒ वि वि च॑ च॒ वि । \newline
5. वि गर्भ॒म् गर्भ॒म् ॅवि वि गर्भ᳚म् । \newline
6. गर्भ॑म् च च॒ गर्भ॒म् गर्भ॑म् च । \newline
7. च॒ ज॒रायु॑ ज॒रायु॑ च च ज॒रायु॑ । \newline
8. ज॒रायु॑ च च ज॒रायु॑ ज॒रायु॑ च । \newline
9. चेति॑ च । \newline
10. ब॒हि स्ते॑ ते ब॒हिर् ब॒हि स्ते᳚ । \newline
11. ते॒ अ॒स्त्व॒स्तु॒ ते॒ ते॒ अ॒स्तु॒ । \newline
12. अ॒स्तु॒ बाल् बाल॑स्त्वस्तु॒ बाल् । \newline
13. बालितीति॒ बाल् बालिति॑ । \newline
14. इतीतीति॑ । \newline
15. उ॒रु॒द्र॒फ्सो वि॒श्वरू॑पो वि॒श्वरू॑प उरुद्र॒फ्स उ॑रुद्र॒फ्सो वि॒श्वरू॑पः । \newline
16. उ॒रु॒द्र॒फ्स इत्यु॑रु - द्र॒फ्सः । \newline
17. वि॒श्वरू॑प॒ इन्दु॒ रिन्दु॑र् वि॒श्वरू॑पो वि॒श्वरू॑प॒ इन्दुः॑ । \newline
18. वि॒श्वरू॑प॒ इति॑ वि॒श्व - रू॒पः॒ । \newline
19. इन्दुः॒ पव॑मानः॒ पव॑मान॒ इन्दु॒ रिन्दुः॒ पव॑मानः । \newline
20. पव॑मानो॒ धीरो॒ धीरः॒ पव॑मानः॒ पव॑मानो॒ धीरः॑ । \newline
21. धीर॑ आनञ्जा नञ्ज॒ धीरो॒ धीर॑ आनञ्ज । \newline
22. आ॒न॒ञ्ज॒ गर्भ॒म् गर्भ॑ मानञ्जा नञ्ज॒ गर्भ᳚म् । \newline
23. गर्भ॒मिति॒ गर्भ᳚म् । \newline
24. एक॑पदी द्वि॒पदी᳚ द्वि॒प द्येक॑प॒ द्येक॑पदी द्वि॒पदी᳚ । \newline
25. एक॑प॒दीत्येक॑ - प॒दी॒ । \newline
26. द्वि॒पदी᳚ त्रि॒पदी᳚ त्रि॒पदी᳚ द्वि॒पदी᳚ द्वि॒पदी᳚ त्रि॒पदी᳚ । \newline
27. द्वि॒पदीति॑ द्वि - पदी᳚ । \newline
28. त्रि॒पदी॒ चतु॑ष्पदी॒ चतु॑ष्पदी त्रि॒पदी᳚ त्रि॒पदी॒ चतु॑ष्पदी । \newline
29. त्रि॒पदीति॑ त्रि - पदी᳚ । \newline
30. चतु॑ष्पदी॒ पञ्च॑पदी॒ पञ्च॑पदी॒ चतु॑ष्पदी॒ चतु॑ष्पदी॒ पञ्च॑पदी । \newline
31. चतु॑ष्प॒दीति॒ चतुः॑ - प॒दी॒ । \newline
32. पञ्च॑पदी॒ षट्प॑दी॒ षट्प॑दी॒ पञ्च॑पदी॒ पञ्च॑पदी॒ षट्प॑दी । \newline
33. पञ्च॑प॒दीति॒ पञ्च॑ - प॒दी॒ । \newline
34. षट्प॑दी स॒प्तप॑दी स॒प्तप॑दी॒ षट्प॑दी॒ षट्प॑दी स॒प्तप॑दी । \newline
35. षट्प॒दीति॒ षट् - प॒दी॒ । \newline
36. स॒प्तप॑ द्य॒ष्टाप॑ द्य॒ष्टाप॑दी स॒प्तप॑दी स॒प्तप॑ द्य॒ष्टाप॑दी । \newline
37. स॒प्तप॒दीति॑ स॒प्त - प॒दी॒ । \newline
38. अ॒ष्टाप॑दी॒ भुव॑ना॒ भुव॑ना॒ ऽष्टाप॑ द्य॒ष्टाप॑दी॒ भुव॑ना । \newline
39. अ॒ष्टाप॒दीत्य॒ष्टा - प॒दी॒ । \newline
40. भुव॒ना ऽन्वनु॒ भुव॑ना॒ भुव॒ना ऽनु॑ । \newline
41. अनु॑ प्रथताम् प्रथता॒ मन्वनु॑ प्रथताम् । \newline
42. प्र॒थ॒ताꣳ॒॒ स्वाहा॒ स्वाहा᳚ प्रथताम् प्रथताꣳ॒॒ स्वाहा᳚ । \newline
43. स्वाहेति॒ स्वाहा᳚ । \newline
44. म॒ही द्यौर् द्यौर् म॒ही म॒ही द्यौः । \newline
45. द्यौः पृ॑थि॒वी पृ॑थि॒वी द्यौर् द्यौः पृ॑थि॒वी । \newline
46. पृ॒थि॒वी च॑ च पृथि॒वी पृ॑थि॒वी च॑ । \newline
47. च॒ नो॒ न॒श्च॒ च॒ नः॒ । \newline
48. न॒ इ॒म मि॒मम् नो॑ न इ॒मम् । \newline
49. इ॒मम् ॅय॒ज्ञ्म् ॅय॒ज्ञ् मि॒म मि॒मम् ॅय॒ज्ञ्म् । \newline
50. य॒ज्ञ्म् मि॑मिक्षताम् मिमिक्षताम् ॅय॒ज्ञ्म् ॅय॒ज्ञ्म् मि॑मिक्षताम् । \newline
51. मि॒मि॒क्ष॒ता॒मिति॑ मिमिक्षताम् । \newline
52. पि॒पृ॒ताम् नो॑ नः पिपृ॒ताम् पि॑पृ॒ताम् नः॑ । \newline
53. नो॒ भरी॑मभि॒र् भरी॑मभिर् नो नो॒ भरी॑मभिः । \newline
54. भरी॑मभि॒रिति॒ भरी॑म - भिः॒ । \newline

\textbf{Ghana Paata } \newline

1. मा॒तर॑म् च च मा॒तर॑म् मा॒तर॑म् च पु॒त्रम् पु॒त्रम् च॑ मा॒तर॑म् मा॒तर॑म् च पु॒त्रम् । \newline
2. च॒ पु॒त्रम् पु॒त्रम् च॑ च पु॒त्रम् च॑ च पु॒त्रम् च॑ च पु॒त्रम् च॑ । \newline
3. पु॒त्रम् च॑ च पु॒त्रम् पु॒त्रम् च॒ वि वि च॑ पु॒त्रम् पु॒त्रम् च॒ वि । \newline
4. च॒ वि वि च॑ च॒ वि गर्भ॒म् गर्भ॒म् ॅवि च॑ च॒ वि गर्भ᳚म् । \newline
5. वि गर्भ॒म् गर्भ॒म् ॅवि वि गर्भ॑म् च च॒ गर्भ॒म् ॅवि वि गर्भ॑म् च । \newline
6. गर्भ॑म् च च॒ गर्भ॒म् गर्भ॑म् च ज॒रायु॑ ज॒रायु॑ च॒ गर्भ॒म् गर्भ॑म् च ज॒रायु॑ । \newline
7. च॒ ज॒रायु॑ ज॒रायु॑ च च ज॒रायु॑ च च ज॒रायु॑ च च ज॒रायु॑ च । \newline
8. ज॒रायु॑ च च ज॒रायु॑ ज॒रायु॑ च । \newline
9. चेति॑ च । \newline
10. ब॒हिस्ते॑ ते ब॒हिर् ब॒हिस्ते॑ अस्त्वस्तु ते ब॒हिर् ब॒हिस्ते॑ अस्तु । \newline
11. ते॒ अ॒स्त्व॒स्तु॒ ते॒ ते॒ अ॒स्तु॒ बाल् बाल॑स्तु ते ते अस्तु॒ बाल् । \newline
12. अ॒स्तु॒ बाल् बाल॑स्त्वस्तु॒ बालितीति॒ बाल॑स्त्वस्तु॒ बालिति॑ । \newline
13. बालितीति॒ बाल् बालिति॑ । \newline
14. इतीतीति॑ । \newline
15. उ॒रु॒द्र॒फ्सो वि॒श्वरू॑पो वि॒श्वरू॑प उरुद्र॒फ्स उ॑रुद्र॒फ्सो वि॒श्वरू॑प॒ इन्दु॒ रिन्दु॑र् वि॒श्वरू॑प उरुद्र॒फ्स उ॑रुद्र॒फ्सो वि॒श्वरू॑प॒ इन्दुः॑ । \newline
16. उ॒रु॒द्र॒फ्स इत्यु॑रु - द्र॒फ्सः । \newline
17. वि॒श्वरू॑प॒ इन्दु॒ रिन्दु॑र् वि॒श्वरू॑पो वि॒श्वरू॑प॒ इन्दुः॒ पव॑मानः॒ पव॑मान॒ इन्दु॑र् वि॒श्वरू॑पो वि॒श्वरू॑प॒ इन्दुः॒ पव॑मानः । \newline
18. वि॒श्वरू॑प॒ इति॑ वि॒श्व - रू॒पः॒ । \newline
19. इन्दुः॒ पव॑मानः॒ पव॑मान॒ इन्दु॒ रिन्दुः॒ पव॑मानो॒ धीरो॒ धीरः॒ पव॑मान॒ इन्दु॒ रिन्दुः॒ पव॑मानो॒ धीरः॑ । \newline
20. पव॑मानो॒ धीरो॒ धीरः॒ पव॑मानः॒ पव॑मानो॒ धीर॑ आनञ्जा नञ्ज॒ धीरः॒ पव॑मानः॒ पव॑मानो॒ धीर॑ आनञ्ज । \newline
21. धीर॑ आनञ्जा नञ्ज॒ धीरो॒ धीर॑ आनञ्ज॒ गर्भ॒म् गर्भ॑ मानञ्ज॒ धीरो॒ धीर॑ आनञ्ज॒ गर्भ᳚म् । \newline
22. आ॒न॒ञ्ज॒ गर्भ॒म् गर्भ॑ मानञ्जा नञ्ज॒ गर्भ᳚म् । \newline
23. गर्भ॒मिति॒ गर्भ᳚म् । \newline
24. एक॑पदी द्वि॒पदी᳚ द्वि॒प द्येक॑प॒ द्येक॑पदी द्वि॒पदी᳚ त्रि॒पदी᳚ त्रि॒पदी᳚ द्वि॒प द्येक॑प॒ द्येक॑पदी द्वि॒पदी᳚ त्रि॒पदी᳚ । \newline
25. एक॑प॒दीत्येक॑ - प॒दी॒ । \newline
26. द्वि॒पदी᳚ त्रि॒पदी᳚ त्रि॒पदी᳚ द्वि॒पदी᳚ द्वि॒पदी᳚ त्रि॒पदी॒ चतु॑ष्पदी॒ चतु॑ष्पदी त्रि॒पदी᳚ द्वि॒पदी᳚ द्वि॒पदी᳚ त्रि॒पदी॒ चतु॑ष्पदी । \newline
27. द्वि॒पदीति॑ द्वि - पदी᳚ । \newline
28. त्रि॒पदी॒ चतु॑ष्पदी॒ चतु॑ष्पदी त्रि॒पदी᳚ त्रि॒पदी॒ चतु॑ष्पदी॒ पञ्च॑पदी॒ पञ्च॑पदी॒ चतु॑ष्पदी त्रि॒पदी᳚ त्रि॒पदी॒ चतु॑ष्पदी॒ पञ्च॑पदी । \newline
29. त्रि॒पदीति॑ त्रि - पदी᳚ । \newline
30. चतु॑ष्पदी॒ पञ्च॑पदी॒ पञ्च॑पदी॒ चतु॑ष्पदी॒ चतु॑ष्पदी॒ पञ्च॑पदी॒ षट्प॑दी॒ षट्प॑दी॒ पञ्च॑पदी॒ चतु॑ष्पदी॒ चतु॑ष्पदी॒ पञ्च॑पदी॒ षट्प॑दी । \newline
31. चतु॑ष्प॒दीति॒ चतुः॑ - प॒दी॒ । \newline
32. पञ्च॑पदी॒ षट्प॑दी॒ षट्प॑दी॒ पञ्च॑पदी॒ पञ्च॑पदी॒ षट्प॑दी स॒प्तप॑दी स॒प्तप॑दी॒ षट्प॑दी॒ पञ्च॑पदी॒ पञ्च॑पदी॒ षट्प॑दी स॒प्तप॑दी । \newline
33. पञ्च॑प॒दीति॒ पञ्च॑ - प॒दी॒ । \newline
34. षट्प॑दी स॒प्तप॑दी स॒प्तप॑दी॒ षट्प॑दी॒ षट्प॑दी स॒प्तप॑ द्य॒ष्टाप॑ द्य॒ष्टाप॑दी स॒प्तप॑दी॒ षट्प॑दी॒ षट्प॑दी स॒प्तप॑ द्य॒ष्टाप॑दी । \newline
35. षट्प॒दीति॒ षट् - प॒दी॒ । \newline
36. स॒प्तप॑ द्य॒ष्टाप॑ द्य॒ष्टाप॑दी स॒प्तप॑दी स॒प्तप॑ द्य॒ष्टाप॑दी॒ भुव॑ना॒ भुव॑ना॒ ऽष्टाप॑दी स॒प्तप॑दी स॒प्तप॑ द्य॒ष्टाप॑दी॒ भुव॑ना । \newline
37. स॒प्तप॒दीति॑ स॒प्त - प॒दी॒ । \newline
38. अ॒ष्टाप॑दी॒ भुव॑ना॒ भुव॑ना॒ ऽष्टाप॑ द्य॒ष्टाप॑दी॒ भुव॒ना ऽन्वनु॒ भुव॑ना॒ ऽष्टाप॑ द्य॒ष्टाप॑दी॒ भुव॒ना ऽनु॑ । \newline
39. अ॒ष्टाप॒दीत्य॒ष्टा - प॒दी॒ । \newline
40. भुव॒ना ऽन्वनु॒ भुव॑ना॒ भुव॒ना ऽनु॑ प्रथताम् प्रथता॒ मनु॒ भुव॑ना॒ भुव॒ना ऽनु॑ प्रथताम् । \newline
41. अनु॑ प्रथताम् प्रथता॒ मन्वनु॑ प्रथताꣳ॒॒ स्वाहा॒ स्वाहा᳚ प्रथता॒ मन्वनु॑ प्रथताꣳ॒॒ स्वाहा᳚ । \newline
42. प्र॒थ॒ताꣳ॒॒ स्वाहा॒ स्वाहा᳚ प्रथताम् प्रथताꣳ॒॒ स्वाहा᳚ । \newline
43. स्वाहेति॒ स्वाहा᳚ । \newline
44. म॒ही द्यौर् द्यौर् म॒ही म॒ही द्यौः पृ॑थि॒वी पृ॑थि॒वी द्यौर् म॒ही म॒ही द्यौः पृ॑थि॒वी । \newline
45. द्यौः पृ॑थि॒वी पृ॑थि॒वी द्यौर् द्यौः पृ॑थि॒वी च॑ च पृथि॒वी द्यौर् द्यौः पृ॑थि॒वी च॑ । \newline
46. पृ॒थि॒वी च॑ च पृथि॒वी पृ॑थि॒वी च॑ नो नश्च पृथि॒वी पृ॑थि॒वी च॑ नः । \newline
47. च॒ नो॒ न॒श्च॒ च॒ न॒ इ॒म मि॒मम् न॑श्च च न इ॒मम् । \newline
48. न॒ इ॒म मि॒मम् नो॑ न इ॒मम् ॅय॒ज्ञ्म् ॅय॒ज्ञ् मि॒मम् नो॑ न इ॒मम् ॅय॒ज्ञ्म् । \newline
49. इ॒मम् ॅय॒ज्ञ्म् ॅय॒ज्ञ् मि॒म मि॒मम् ॅय॒ज्ञ्म् मि॑मिक्षताम् मिमिक्षताम् ॅय॒ज्ञ् मि॒म मि॒मम् ॅय॒ज्ञ्म् मि॑मिक्षताम् । \newline
50. य॒ज्ञ्म् मि॑मिक्षताम् मिमिक्षताम् ॅय॒ज्ञ्म् ॅय॒ज्ञ्म् मि॑मिक्षताम् । \newline
51. मि॒मि॒क्ष॒ता॒मिति॑ मिमिक्षताम् । \newline
52. पि॒पृ॒तान्नो॑ नः पिपृ॒ताम् पि॑पृ॒ताम् नो॒ भरी॑मभि॒र् भरी॑मभिर् नः पिपृ॒ताम् पि॑पृ॒ताम् नो॒ भरी॑मभिः । \newline
53. नो॒ भरी॑मभि॒र् भरी॑मभिर् नो नो॒ भरी॑मभिः । \newline
54. भरी॑मभि॒रिति॒ भरी॑म - भिः॒ । \newline
\pagebreak
\markright{ TS 3.3.11.1  \hfill https://www.vedavms.in \hfill}

\section{ TS 3.3.11.1 }

\textbf{TS 3.3.11.1 } \newline
\textbf{Samhita Paata} \newline

इ॒दं ॅवा॑मा॒स्ये॑ ह॒विः प्रि॒यमि॑न्द्राबृहस्पती । उ॒क्थं मद॑श्च शस्यते ॥ अ॒यं ॅवां॒ परि॑ षिच्यते॒ सोम॑इन्द्राबृहस्पती । चारु॒र्मदा॑य पी॒तये᳚ ॥ अ॒स्मे इ॑न्द्राबृहस्पती र॒यिं ध॑त्तꣳ शत॒ग्विनं᳚ । अश्वा॑वन्तꣳ सह॒स्रिणं᳚ ॥बृह॒स्पति॑र्नः॒ परि॑पातु प॒श्चादु॒तोत्त॑रस्मा॒दध॑रादघा॒योः । इन्द्रः॑ पु॒रस्ता॑दु॒त म॑द्ध्य॒तो नः॒ सखा॒ सखि॑भ्यो॒ वरि॑वः कृणोतु ॥ वि ते॒ विष्व॒ग्वात॑जूतासो अग्ने॒ भामा॑सः - [  ] \newline

\textbf{Pada Paata} \newline

इ॒दम् । वा॒म् । आ॒स्ये᳚ । ह॒विः । प्रि॒यम् । इ॒न्द्रा॒बृ॒ह॒स्प॒ती॒ इती᳚न्द्रा-बृ॒ह॒स्प॒ती॒ ॥ उ॒क्थम् । मदः॑ । च॒ । श॒स्य॒ते॒ ॥ अ॒यम् । वा॒म् । परीति॑ । सि॒च्य॒ते॒ । सोमः॑ । इ॒न्द्रा॒बृ॒ह॒स्प॒ती॒ इती᳚न्द्रा-बृ॒ह॒स्प॒ती॒ ॥ चारुः॑ । मदा॑य । पी॒तये᳚ ॥ अ॒स्मे इति॑ । इ॒न्द्रा॒बृ॒ह॒स्प॒ती॒ इती᳚न्द्रा - बृ॒ह॒स्प॒ती॒ । र॒यिम् । ध॒त्त॒म् । श॒त॒ग्विन॒मिति॑ शत - ग्विन᳚म् ॥ अश्वा॑वन्त॒मित्यश्व॑ - व॒न्त॒म् । स॒ह॒स्रिण᳚म् ॥ बृह॒स्पतिः॑ । नः॒ । परीति॑ । पा॒तु॒ । प॒श्चात् । उ॒त । उत्त॑रस्मा॒दित्युत् - त॒र॒स्मा॒त् । अध॑रात् । अ॒घा॒योरित्य॑घ - योः ॥ इन्द्रः॑ । पु॒रस्ता᳚त् । उ॒त । म॒द्ध्य॒तः । नः॒ । सखा᳚ । सखि॑भ्य॒ इति॒ सखि॑ - भ्यः॒ । वरि॑वः । कृ॒णो॒तु॒ ॥ वीति॑ । ते॒ । विष्व॑क् । वात॑जूतास॒ इति॒ वात॑ - जू॒ता॒सः॒ । अ॒ग्ने॒ । भामा॑सः ।  \newline


\textbf{Krama Paata} \newline

इ॒दं ॅवा᳚म् । वा॒मा॒स्ये᳚ । आ॒स्ये॑ ह॒विः । ह॒विः प्रि॒यम् । प्रि॒यमि॑न्द्राबृहस्पती । इ॒न्द्रा॒बृ॒ह॒स्प॒ती॒ इती᳚न्द्रा - बृ॒ह॒स्प॒ती॒ । उ॒क्थम् मदः॑ । मद॑श्च । च॒ श॒स्य॒ते॒ । श॒स्य॒त॒ इति॑ शस्यते ॥ अ॒यं ॅवा᳚म् । वा॒म् परि॑ । 
परि॑ षिच्यते । सि॒च्य॒ते॒ सोमः॑ । सोम॑ इन्द्राबृहस्पती । इ॒न्द्रा॒बृ॒ह॒स्प॒ती॒ इती᳚न्द्रा - बृ॒ह॒स्प॒ती॒ ॥ चारु॒र् मदा॑य । मदा॑य पी॒तये᳚ । पी॒तय॒ इति॑ पी॒तये᳚ ॥ अ॒स्मे इ॑न्द्राबृहस्पती । अ॒स्मे इत्य॒स्मे । इ॒न्द्रा॒बृ॒ह॒स्प॒ती॒ र॒यिम् । इ॒न्द्रा॒बृ॒ह॒स्प॒ती॒ इती᳚न्द्रा - बृ॒ह॒स्प॒ती॒ । र॒यिम् ध॑त्तम् । ध॒त्तꣳ॒॒ श॒त॒ग्विन᳚म् । श॒त॒ग्विन॒मिति॑ शत - ग्विन᳚म् ॥ अश्वा॑वन्तꣳ सह॒स्रिण᳚म् । अश्वा॑वन्त॒मित्यश्व॑ - व॒न्त॒म् । स॒ह॒स्रिण॒मिति॑ सह॒स्रिण᳚म् ॥ बृह॒स्पति॑र् नः । नः॒ परि॑ । परि॑ पातु । पा॒तु॒ प॒श्चात् । प॒श्चादु॒त । उ॒तोत्त॑रस्मात् । उत्त॑रस्मा॒दध॑रात् । उत्त॑रस्मा॒दित्युत् - त॒र॒स्मा॒त्॒ । अध॑रादघा॒योः । अ॒घा॒योरित्य॑घ - योः ॥ इन्द्रः॑ पु॒रस्ता᳚त् । पु॒रस्ता॑दु॒त । उ॒त म॑द्ध्य॒तः । म॒द्ध्य॒तो नः॑ । नः॒ सखा᳚ । सखा॒ सखि॑भ्यः । सखि॑भ्यो॒ वरि॑वः । सखि॑भ्य॒ इति॒ सखि॑ - भ्यः॒ । वरि॑वः कृणोतु । कृ॒णो॒त्विति॑ कृणोतु ॥ वि ते᳚ । ते॒ विष्व॑क् । विष्व॒ग् वात॑जूतासः । वात॑जूतासो अग्ने । वात॑जूतास॒ इति॒ वात॑ - जू॒ता॒सः॒ । अ॒ग्ने॒ भामा॑सः । भामा॑सः शुचे \newline

\textbf{Jatai Paata} \newline

1. इ॒दम् ॅवा᳚म् ॅवा मि॒द मि॒दम् ॅवा᳚म् । \newline
2. वा॒ मा॒स्य॑ आ॒स्ये॑ वाम् ॅवा मा॒स्ये᳚ । \newline
3. आ॒स्ये॑ ह॒विर्. ह॒वि रा॒स्य॑ आ॒स्ये॑ ह॒विः । \newline
4. ह॒विः प्रि॒यम् प्रि॒यꣳ ह॒विर्. ह॒विः प्रि॒यम् । \newline
5. प्रि॒य मि॑न्द्राबृहस्पती इन्द्राबृहस्पती प्रि॒यम् प्रि॒य मि॑न्द्राबृहस्पती । \newline
6. इ॒न्द्रा॒बृ॒ह॒स्प॒ती॒ इती᳚न्द्रा - बृ॒ह॒स्प॒ती॒ । \newline
7. उ॒क्थम् मदो॒ मद॑ उ॒क्थ मु॒क्थम् मदः॑ । \newline
8. मद॑श्च च॒ मदो॒ मद॑श्च । \newline
9. च॒ श॒स्य॒ते॒ श॒स्य॒ते॒ च॒ च॒ श॒स्य॒ते॒ । \newline
10. श॒स्य॒त॒ इति॑ शस्यते । \newline
11. अ॒यम् ॅवा᳚म् ॅवा म॒य म॒यम् ॅवा᳚म् । \newline
12. वा॒म् परि॒ परि॑ वाम् ॅवा॒म् परि॑ । \newline
13. परि॑ षिच्यते सिच्यते॒ परि॒ परि॑ षिच्यते । \newline
14. सि॒च्य॒ते॒ सोमः॒ सोमः॑ सिच्यते सिच्यते॒ सोमः॑ । \newline
15. सोम॑ इन्द्राबृहस्पती इन्द्राबृहस्पती॒ सोमः॒ सोम॑ इन्द्राबृहस्पती । \newline
16. इ॒न्द्रा॒बृ॒ह॒स्प॒ती॒ इती᳚न्द्रा - बृ॒ह॒स्प॒ती॒ । \newline
17. चारु॒र् मदा॑य॒ मदा॑य॒ चारु॒ श्चारु॒र् मदा॑य । \newline
18. मदा॑य पी॒तये॑ पी॒तये॒ मदा॑य॒ मदा॑य पी॒तये᳚ । \newline
19. पी॒तय॒ इति॑ पी॒तये᳚ । \newline
20. अ॒स्मे इ॑न्द्राबृहस्पती इन्द्राबृहस्पती अ॒स्मे अ॒स्मे इ॑न्द्राबृहस्पती । \newline
21. अ॒स्मे इत्य॒स्मे । \newline
22. इ॒न्द्रा॒बृ॒ह॒स्प॒ती॒ र॒यिꣳ र॒यि मि॑न्द्राबृहस्पती इन्द्राबृहस्पती र॒यिम् । \newline
23. इ॒न्द्रा॒बृ॒ह॒स्प॒ती॒ इती᳚न्द्रा - बृ॒ह॒स्प॒ती॒ । \newline
24. र॒यिम् ध॑त्तम् धत्तꣳ र॒यिꣳ र॒यिम् ध॑त्तम् । \newline
25. ध॒त्तꣳ॒॒ श॒त॒ग्विनꣳ॑ शत॒ग्विन॑म् धत्तम् धत्तꣳ शत॒ग्विन᳚म् । \newline
26. श॒त॒ग्विन॒मिति॑ शत - ग्विन᳚म् । \newline
27. अश्वा॑वन्तꣳ सह॒स्रिणꣳ॑ सह॒स्रिण॒ मश्वा॑वन्त॒ मश्वा॑वन्तꣳ सह॒स्रिण᳚म् । \newline
28. अश्वा॑वन्त॒मित्यश्व॑ - व॒न्त॒म् । \newline
29. स॒ह॒स्रिण॒मिति॑ सह॒स्रिण᳚म् । \newline
30. बृह॒स्पति॑र् नो नो॒ बृह॒स्पति॒र् बृह॒स्पति॑र् नः । \newline
31. नः॒ परि॒ परि॑ णो नः॒ परि॑ । \newline
32. परि॑ पातु पातु॒ परि॒ परि॑ पातु । \newline
33. पा॒तु॒ प॒श्चात् प॒श्चात् पा॑तु पातु प॒श्चात् । \newline
34. प॒श्चा दु॒तोत प॒श्चात् प॒श्चा दु॒त । \newline
35. उ॒तोत्त॑रस्मा॒ दुत्त॑रस्मा दु॒तोतोत्त॑रस्मात् । \newline
36. उत्त॑रस्मा॒ दध॑रा॒ दध॑रा॒ दुत्त॑रस्मा॒ दुत्त॑रस्मा॒ दध॑रात् । \newline
37. उत्त॑रस्मा॒दित्युत् - त॒र॒स्मा॒त् । \newline
38. अध॑रा दघा॒यो र॑घा॒यो रध॑रा॒ दध॑रा दघा॒योः । \newline
39. अ॒घा॒योरित्य॑घ - योः । \newline
40. इन्द्रः॑ पु॒रस्ता᳚त् पु॒रस्ता॒ दिन्द्र॒ इन्द्रः॑ पु॒रस्ता᳚त् । \newline
41. पु॒रस्ता॑ दु॒तोत पु॒रस्ता᳚त् पु॒रस्ता॑ दु॒त । \newline
42. उ॒त म॑द्ध्य॒तो म॑द्ध्य॒त उ॒तोत म॑द्ध्य॒तः । \newline
43. म॒द्ध्य॒तो नो॑ नो मद्ध्य॒तो म॑द्ध्य॒तो नः॑ । \newline
44. नः॒ सखा॒ सखा॑ नो नः॒ सखा᳚ । \newline
45. सखा॒ सखि॑भ्यः॒ सखि॑भ्यः॒ सखा॒ सखा॒ सखि॑भ्यः । \newline
46. सखि॑भ्यो॒ वरि॑वो॒ वरि॑वः॒ सखि॑भ्यः॒ सखि॑भ्यो॒ वरि॑वः । \newline
47. सखि॑भ्य॒ इति॒ सखि॑ - भ्यः॒ । \newline
48. वरि॑वः कृणोतु कृणोतु॒ वरि॑वो॒ वरि॑वः कृणोतु । \newline
49. कृ॒णो॒त्विति॑ कृणोतु । \newline
50. वि ते॑ ते॒ वि वि ते᳚ । \newline
51. ते॒ विष्व॒ग् विष्व॑क् ते ते॒ विष्व॑क् । \newline
52. विष्व॒ग् वात॑जूतासो॒ वात॑जूतासो॒ विष्व॒ग् विष्व॒ग् वात॑जूतासः । \newline
53. वात॑जूतासो अग्ने अग्ने॒ वात॑जूतासो॒ वात॑जूतासो अग्ने । \newline
54. वात॑जूतास॒ इति॒ वात॑ - जू॒ता॒सः॒ । \newline
55. अ॒ग्ने॒ भामा॑सो॒ भामा॑सो अग्ने अग्ने॒ भामा॑सः । \newline
56. भामा॑सः शुचे शुचे॒ भामा॑सो॒ भामा॑सः शुचे । \newline

\textbf{Ghana Paata } \newline

1. इ॒दम् ॅवा᳚म् ॅवा मि॒द मि॒दम् ॅवा॑ मा॒स्य॑ आ॒स्ये॑ वा मि॒द मि॒दम् ॅवा॑ मा॒स्ये᳚ । \newline
2. वा॒ मा॒स्य॑ आ॒स्ये॑ वाम् ॅवा मा॒स्ये॑ ह॒विर्. ह॒वि रा॒स्ये॑ वाम् ॅवा मा॒स्ये॑ ह॒विः । \newline
3. आ॒स्ये॑ ह॒विर्. ह॒वि रा॒स्य॑ आ॒स्ये॑ ह॒विः प्रि॒यम् प्रि॒यꣳ ह॒वि रा॒स्य॑ आ॒स्ये॑ ह॒विः प्रि॒यम् । \newline
4. ह॒विः प्रि॒यम् प्रि॒यꣳ ह॒विर्. ह॒विः प्रि॒य मि॑न्द्राबृहस्पती इन्द्राबृहस्पती प्रि॒यꣳ ह॒विर्. ह॒विः प्रि॒य मि॑न्द्राबृहस्पती । \newline
5. प्रि॒य मि॑न्द्राबृहस्पती इन्द्राबृहस्पती प्रि॒यम् प्रि॒य मि॑न्द्राबृहस्पती । \newline
6. इ॒न्द्रा॒बृ॒ह॒स्प॒ती॒ इती᳚न्द्रा - बृ॒ह॒स्प॒ती॒ । \newline
7. उ॒क्थम् मदो॒ मद॑ उ॒क्थ मु॒क्थम् मद॑श्च च॒ मद॑ उ॒क्थ मु॒क्थम् मद॑श्च । \newline
8. मद॑श्च च॒ मदो॒ मद॑श्च शस्यते शस्यते च॒ मदो॒ मद॑श्च शस्यते । \newline
9. च॒ श॒स्य॒ते॒ श॒स्य॒ते॒ च॒ च॒ श॒स्य॒ते॒ । \newline
10. श॒स्य॒त॒ इति॑ शस्यते । \newline
11. अ॒यम् ॅवा᳚म् ॅवा म॒य म॒यम् ॅवा॒म् परि॒ परि॑ वा म॒य म॒यम् ॅवा॒म् परि॑ । \newline
12. वा॒म् परि॒ परि॑ वाम् ॅवा॒म् परि॑ षिच्यते सिच्यते॒ परि॑ वाम् ॅवा॒म् परि॑ षिच्यते । \newline
13. परि॑ षिच्यते सिच्यते॒ परि॒ परि॑ षिच्यते॒ सोमः॒ सोमः॑ सिच्यते॒ परि॒ परि॑ षिच्यते॒ सोमः॑ । \newline
14. सि॒च्य॒ते॒ सोमः॒ सोमः॑ सिच्यते सिच्यते॒ सोम॑ इन्द्राबृहस्पती इन्द्राबृहस्पती॒ सोमः॑ सिच्यते सिच्यते॒ सोम॑ इन्द्राबृहस्पती । \newline
15. सोम॑ इन्द्राबृहस्पती इन्द्राबृहस्पती॒ सोमः॒ सोम॑ इन्द्राबृहस्पती । \newline
16. इ॒न्द्रा॒बृ॒ह॒स्प॒ती॒ इती᳚न्द्रा - बृ॒ह॒स्प॒ती॒ । \newline
17. चारु॒र् मदा॑य॒ मदा॑य॒ चारु॒ श्चारु॒र् मदा॑य पी॒तये॑ पी॒तये॒ मदा॑य॒ चारु॒ श्चारु॒र् मदा॑य पी॒तये᳚ । \newline
18. मदा॑य पी॒तये॑ पी॒तये॒ मदा॑य॒ मदा॑य पी॒तये᳚ । \newline
19. पी॒तय॒ इति॑ पी॒तये᳚ । \newline
20. अ॒स्मे इ॑न्द्राबृहस्पती इन्द्राबृहस्पती अ॒स्मे अ॒स्मे इ॑न्द्राबृहस्पती र॒यिꣳ र॒यि मि॑न्द्राबृहस्पती अ॒स्मे अ॒स्मे इ॑न्द्राबृहस्पती र॒यिम् । \newline
21. अ॒स्मे इत्य॒स्मे । \newline
22. इ॒न्द्रा॒बृ॒ह॒स्प॒ती॒ र॒यिꣳ र॒यि मि॑न्द्राबृहस्पती इन्द्राबृहस्पती र॒यिम् ध॑त्तम् धत्तꣳ र॒यि मि॑न्द्राबृहस्पती इन्द्राबृहस्पती र॒यिम् ध॑त्तम् । \newline
23. इ॒न्द्रा॒बृ॒ह॒स्प॒ती॒ इती᳚न्द्रा - बृ॒ह॒स्प॒ती॒ । \newline
24. र॒यिम् ध॑त्तम् धत्तꣳ र॒यिꣳ र॒यिम् ध॑त्तꣳ शत॒ग्विनꣳ॑ शत॒ग्विन॑म् धत्तꣳ र॒यिꣳ र॒यिम् ध॑त्तꣳ शत॒ग्विन᳚म् । \newline
25. ध॒त्तꣳ॒॒ श॒त॒ग्विनꣳ॑ शत॒ग्विन॑म् धत्तम् धत्तꣳ शत॒ग्विन᳚म् । \newline
26. श॒त॒ग्विन॒मिति॑ शत - ग्विन᳚म् । \newline
27. अश्वा॑वन्तꣳ सह॒स्रिणꣳ॑ सह॒स्रिण॒ मश्वा॑वन्त॒ मश्वा॑वन्तꣳ सह॒स्रिण᳚म् । \newline
28. अश्वा॑वन्त॒मित्यश्व॑ - व॒न्त॒म् । \newline
29. स॒ह॒स्रिण॒मिति॑ सह॒स्रिण᳚म् । \newline
30. बृह॒स्पति॑र् नो नो॒ बृह॒स्पति॒र् बृह॒स्पति॑र् नः॒ परि॒ परि॑ णो॒ बृह॒स्पति॒र् बृह॒स्पति॑र् नः॒ परि॑ । \newline
31. नः॒ परि॒ परि॑ णो नः॒ परि॑ पातु पातु॒ परि॑ णो नः॒ परि॑ पातु । \newline
32. परि॑ पातु पातु॒ परि॒ परि॑ पातु प॒श्चात् प॒श्चात् पा॑तु॒ परि॒ परि॑ पातु प॒श्चात् । \newline
33. पा॒तु॒ प॒श्चात् प॒श्चात् पा॑तु पातु प॒श्चा दु॒तोत प॒श्चात् पा॑तु पातु प॒श्चा दु॒त । \newline
34. प॒श्चादु॒तोत प॒श्चात् प॒श्चा दु॒तोत्त॑रस्मा॒ दुत्त॑रस्मा दु॒त प॒श्चात् प॒श्चा दु॒तोत्त॑रस्मात् । \newline
35. उ॒तोत्त॑रस्मा॒ दुत्त॑रस्मा दु॒तोतोत्त॑रस्मा॒ दध॑रा॒ दध॑रा॒ दुत्त॑रस्मा दु॒तोतोत्त॑रस्मा॒ दध॑रात् । \newline
36. उत्त॑रस्मा॒ दध॑रा॒ दध॑रा॒ दुत्त॑रस्मा॒ दुत्त॑रस्मा॒ दध॑रा दघा॒यो र॑घा॒यो रध॑रा॒ दुत्त॑रस्मा॒ दुत्त॑रस्मा॒ दध॑रा दघा॒योः । \newline
37. उत्त॑रस्मा॒दित्युत् - त॒र॒स्मा॒त् । \newline
38. अध॑रा दघा॒यो र॑घा॒यो रध॑रा॒ दध॑रा दघा॒योः । \newline
39. अ॒घा॒योरित्य॑घ - योः । \newline
40. इन्द्रः॑ पु॒रस्ता᳚त् पु॒रस्ता॒ दिन्द्र॒ इन्द्रः॑ पु॒रस्ता॑ दु॒तोत पु॒रस्ता॒ दिन्द्र॒ इन्द्रः॑ पु॒रस्ता॑ दु॒त । \newline
41. पु॒रस्ता॑ दु॒तोत पु॒रस्ता᳚त् पु॒रस्ता॑ दु॒त म॑द्ध्य॒तो म॑द्ध्य॒त उ॒त पु॒रस्ता᳚त् पु॒रस्ता॑ दु॒त म॑द्ध्य॒तः । \newline
42. उ॒त म॑द्ध्य॒तो म॑द्ध्य॒त उ॒तोत म॑द्ध्य॒तो नो॑ नो मद्ध्य॒त उ॒तोत म॑द्ध्य॒तो नः॑ । \newline
43. म॒द्ध्य॒तो नो॑ नो मद्ध्य॒तो म॑द्ध्य॒तो नः॒ सखा॒ सखा॑ नो मद्ध्य॒तो म॑द्ध्य॒तो नः॒ सखा᳚ । \newline
44. नः॒ सखा॒ सखा॑ नो नः॒ सखा॒ सखि॑भ्यः॒ सखि॑भ्यः॒ सखा॑ नो नः॒ सखा॒ सखि॑भ्यः । \newline
45. सखा॒ सखि॑भ्यः॒ सखि॑भ्यः॒ सखा॒ सखा॒ सखि॑भ्यो॒ वरि॑वो॒ वरि॑वः॒ सखि॑भ्यः॒ सखा॒ सखा॒ सखि॑भ्यो॒ वरि॑वः । \newline
46. सखि॑भ्यो॒ वरि॑वो॒ वरि॑वः॒ सखि॑भ्यः॒ सखि॑भ्यो॒ वरि॑वः कृणोतु कृणोतु॒ वरि॑वः॒ सखि॑भ्यः॒ सखि॑भ्यो॒ वरि॑वः कृणोतु । \newline
47. सखि॑भ्य॒ इति॒ सखि॑ - भ्यः॒ । \newline
48. वरि॑वः कृणोतु कृणोतु॒ वरि॑वो॒ वरि॑वः कृणोतु । \newline
49. कृ॒णो॒त्विति॑ कृणोतु । \newline
50. वि ते॑ ते॒ वि वि ते॒ विष्व॒ग् विष्व॑क् ते॒ वि वि ते॒ विष्व॑क् । \newline
51. ते॒ विष्व॒ग् विष्व॑क् ते ते॒ विष्व॒ग् वात॑जूतासो॒ वात॑जूतासो॒ विष्व॑क् ते ते॒ विष्व॒ग् वात॑जूतासः । \newline
52. विष्व॒ग् वात॑जूतासो॒ वात॑जूतासो॒ विष्व॒ग् विष्व॒ग् वात॑जूतासो अग्ने अग्ने॒ वात॑जूतासो॒ विष्व॒ग् विष्व॒ग् वात॑जूतासो अग्ने । \newline
53. वात॑जूतासो अग्ने अग्ने॒ वात॑जूतासो॒ वात॑जूतासो अग्ने॒ भामा॑सो॒ भामा॑सो अग्ने॒ वात॑जूतासो॒ वात॑जूतासो अग्ने॒ भामा॑सः । \newline
54. वात॑जूतास॒ इति॒ वात॑ - जू॒ता॒सः॒ । \newline
55. अ॒ग्ने॒ भामा॑सो॒ भामा॑सो अग्ने अग्ने॒ भामा॑सः शुचे शुचे॒ भामा॑सो अग्ने अग्ने॒ भामा॑सः शुचे । \newline
56. भामा॑सः शुचे शुचे॒ भामा॑सो॒ भामा॑सः शुचे॒ शुच॑यः॒ शुच॑यः शुचे॒ भामा॑सो॒ भामा॑सः शुचे॒ शुच॑यः । \newline
\pagebreak
\markright{ TS 3.3.11.2  \hfill https://www.vedavms.in \hfill}

\section{ TS 3.3.11.2 }

\textbf{TS 3.3.11.2 } \newline
\textbf{Samhita Paata} \newline

शुचे॒ शुच॑यश्चरन्ति । तु॒वि॒म्र॒क्षासो॑ दि॒व्या नव॑ग्वा॒ वना॑ वनन्ति धृष॒ता रु॒जन्तः॑ ॥त्वाम॑ग्ने॒ मानु॑षीरीडते॒ विशो॑ होत्रा॒विदं॒ ॅविवि॑चिꣳ रत्न॒धात॑मं । गुहा॒ सन्तꣳ॑ सुभग वि॒श्वद॑र्.शतं तु विष्म॒णसꣳ॑ सु॒यजं॑ घृत॒श्रियं᳚ ॥धा॒ता द॑दातु नो र॒यिमीशा॑नो॒ जग॑त॒स्पतिः॑ । स नः॑ पू॒र्णेन॑ वावनत् ॥ धा॒ता प्र॒जाया॑ उ॒त रा॒य ई॑शे धा॒तेदं ॅविश्वं॒ भुव॑नं जजान । धा॒ता पु॒त्रं ॅयज॑मानाय॒ दाता॒ - [  ] \newline

\textbf{Pada Paata} \newline

शु॒चे॒ । शुच॑यः । च॒र॒न्ति॒ ॥ तु॒वि॒म्र॒क्षास॒ इति॑ तुवि - म्र॒क्षासः॑ । दि॒व्याः । नव॑ग्वाः । वना᳚ । व॒न॒न्ति॒ । धृ॒ष॒ता । रु॒जन्तः॑ ॥ त्वाम् । अ॒ग्ने॒ । मानु॑षीः । ई॒ड॒ते॒ । विशः॑ । हो॒त्रा॒विद॒मिति॑ होत्रा - विद᳚म् । विवि॑चि॒मिति॒ वि-वि॒चि॒म् । र॒त्न॒धात॑म॒मिति॑ रत्न - धात॑मम् ॥ गुहा᳚ । सन्त᳚म् । सु॒भ॒गेति॑ सु - भ॒ग॒ । वि॒श्वद॑र्.शत॒मिति॑ वि॒श्व-द॒र॒.श॒त॒म् । तु॒वि॒ष्म॒णस᳚म् । सु॒यज॒मिति॑ सु - यज᳚म् । घृ॒त॒श्रिय॒मिति॑ घृत - श्रिय᳚म् ॥ धा॒ता । द॒दा॒तु॒ । नः॒ । र॒यिम् । ईशा॑नः । जग॑तः । पतिः॑ ॥ सः । नः॒ । पू॒र्णेन॑ । वा॒व॒न॒त् ॥ धा॒ता । प्र॒जाया॒ इति॑ प्र - जयाः᳚ । उ॒त । रा॒यः । ई॒शे॒ । धा॒ता । इ॒दम् । विश्व᳚म् । भुव॑नम् । ज॒जा॒न॒ ॥ धा॒ता । पु॒त्रम् । यज॑मानाय । दाता᳚ ।  \newline


\textbf{Krama Paata} \newline

शु॒चे॒ शुच॑यः । शुच॑य श्चरन्ति । च॒र॒न्तीति॑ चरन्ति ॥ तु॒वि॒म्र॒क्षासो॑ दि॒व्याः । तु॒वि॒म्र॒क्षास॒ इति॑ तुवि - म्र॒क्षासः॑ । दि॒व्या नव॑ग्वाः । नव॑ग्वा॒ वना᳚ । वना॑ वनन्ति । व॒न॒न्ति॒ धृ॒ष॒ता । धृ॒ष॒ता रु॒जन्तः॑ । रु॒जन्त॒ इति॑ रु॒जन्तः॑ ॥ त्वाम॑ग्ने । अ॒ग्ने॒ मानु॑षीः । मानु॑षीरीडते । ई॒ड॒ते॒ विशः॑ । विशो॑ होत्रा॒विद᳚म् । हो॒त्रा॒विदं॒ ॅविवि॑चिम् । हो॒त्रा॒विद॒मिति॑ होत्रा - विद᳚म् । विवि॑चिꣳ रत्न॒धात॑मम् । विवि॑चि॒मिति॒ वि - वि॒चि॒म् । र॒त्न॒धात॑म॒मिति॑ रत्न - धात॑मम् ॥ गुहा॒ सन्त᳚म् । सन्तꣳ॑ सुभग । सु॒भ॒ग॒ वि॒श्वद॑र्.शतम् । सु॒भ॒गेति॑ सु - भ॒ग॒ । वि॒श्वद॑र्.शतम् तुविष्म॒णस᳚म् । वि॒श्वद॑र्.शत॒मिति॑ वि॒श्व - द॒र्॒.श॒त॒म् । तु॒वि॒ष्म॒णसꣳ॑ सु॒यज᳚म् । सु॒यज॑म् घृत॒श्रिय᳚म् । सु॒यज॒मिति॑ सु - यज᳚म् । घृ॒त॒श्रिय॒मिति॑ घृत - श्रिय᳚म् ॥ धा॒ता द॑दातु । द॒दा॒तु॒ नः॒ । नो॒ र॒यिम् । र॒यिमीशा॑नः । ईशा॑नो॒ जग॑तः । जग॑त॒स्पतिः॑ । पति॒रिति॒ पतिः॑ ॥ स नः॑ । नः॒ पू॒र्णेन॑ । पू॒र्णेन॑ वावनत् । वा॒व॒न॒दिति॑ वावनत् ॥ धा॒ता प्र॒जायाः᳚ । प्र॒जाया॑ उ॒त । प्र॒जाया॒ इति॑ प्र - जायाः᳚ । उ॒त रा॒यः । रा॒य ई॑शे । ई॒शे॒ धा॒ता । धा॒तेदम् । इ॒दं ॅविश्व᳚म् । विश्व॒म् भुव॑नम् । भुव॑नम् जजान । ज॒जा॒नेति॑ जजान ॥ धा॒ता पु॒त्रम् । पु॒त्रं ॅयज॑मानाय । यज॑मानाय॒ दाता᳚ । दाता॒ तस्मै᳚ \newline

\textbf{Jatai Paata} \newline

1. शु॒चे॒ शुच॑यः॒ शुच॑यः शुचे शुचे॒ शुच॑यः । \newline
2. शुच॑य श्चरन्ति चरन्ति॒ शुच॑यः॒ शुच॑य श्चरन्ति । \newline
3. च॒र॒न्तीति॑ चरन्ति । \newline
4. तु॒वि॒म्र॒क्षासो॑ दि॒व्या दि॒व्या स्तु॑विम्र॒क्षास॑ स्तुविम्र॒क्षासो॑ दि॒व्याः । \newline
5. तु॒वि॒म्र॒क्षास॒ इति॑ तुवि - म्र॒क्षासः॑ । \newline
6. दि॒व्या नव॑ग्वा॒ नव॑ग्वा दि॒व्या दि॒व्या नव॑ग्वाः । \newline
7. नव॑ग्वा॒ वना॒ वना॒ नव॑ग्वा॒ नव॑ग्वा॒ वना᳚ । \newline
8. वना॑ वनन्ति वनन्ति॒ वना॒ वना॑ वनन्ति । \newline
9. व॒न॒न्ति॒ धृ॒ष॒ता धृ॑ष॒ता व॑नन्ति वनन्ति धृष॒ता । \newline
10. धृ॒ष॒ता रु॒जन्तो॑ रु॒जन्तो॑ धृष॒ता धृ॑ष॒ता रु॒जन्तः॑ । \newline
11. रु॒जन्त॒ इति॑ रु॒जन्तः॑ । \newline
12. त्वा म॑ग्ने अग्ने॒ त्वाम् त्वा म॑ग्ने । \newline
13. अ॒ग्ने॒ मानु॑षी॒र् मानु॑षी रग्ने अग्ने॒ मानु॑षीः । \newline
14. मानु॑षी रीडत ईडते॒ मानु॑षी॒र् मानु॑षी रीडते । \newline
15. ई॒ड॒ते॒ विशो॒ विश॑ ईडत ईडते॒ विशः॑ । \newline
16. विशो॑ होत्रा॒विदꣳ॑ होत्रा॒विद॒म् ॅविशो॒ विशो॑ होत्रा॒विद᳚म् । \newline
17. हो॒त्रा॒विद॒म् ॅविवि॑चि॒म् ॅविवि॑चिꣳ होत्रा॒विदꣳ॑ होत्रा॒विद॒म् ॅविवि॑चिम् । \newline
18. हो॒त्रा॒विद॒मिति॑ होत्रा - विद᳚म् । \newline
19. विवि॑चिꣳ रत्न॒धात॑मꣳ रत्न॒धात॑म॒म् ॅविवि॑चि॒म् ॅविवि॑चिꣳ रत्न॒धात॑मम् । \newline
20. विवि॑चि॒मिति॒ वि - वि॒चि॒म् । \newline
21. र॒त्न॒धात॑म॒मिति॑ रत्न - धात॑मम् । \newline
22. गुहा॒ सन्तꣳ॒॒ सन्त॒म् गुहा॒ गुहा॒ सन्त᳚म् । \newline
23. सन्तꣳ॑ सुभग सुभग सन्तꣳ॒॒ सन्तꣳ॑ सुभग । \newline
24. सु॒भ॒ग॒ वि॒श्वद॑र्.शतम् ॅवि॒श्वद॑र्.शतꣳ सुभग सुभग वि॒श्वद॑र्.शतम् । \newline
25. सु॒भ॒गेति॑ सु - भ॒ग॒ । \newline
26. वि॒श्वद॑र्.शतम् तुविष्म॒णस॑म् तुविष्म॒णस॑म् ॅवि॒श्वद॑र्.शतम् ॅवि॒श्वद॑र्.शतम् तुविष्म॒णस᳚म् । \newline
27. वि॒श्वद॑र्.शत॒मिति॑ वि॒श्व - द॒र्॒.श॒त॒म् । \newline
28. तु॒वि॒ष्म॒णसꣳ॑ सु॒यजꣳ॑ सु॒यज॑म् तुविष्म॒णस॑म् तुविष्म॒णसꣳ॑ सु॒यज᳚म् । \newline
29. सु॒यज॑म् घृत॒श्रिय॑म् घृत॒श्रियꣳ॑ सु॒यजꣳ॑ सु॒यज॑म् घृत॒श्रिय᳚म् । \newline
30. सु॒यज॒मिति॑ सु - यज᳚म् । \newline
31. घृ॒त॒श्रिय॒मिति॑ घृत - श्रिय᳚म् । \newline
32. धा॒ता द॑दातु ददातु धा॒ता धा॒ता द॑दातु । \newline
33. द॒दा॒तु॒ नो॒ नो॒ द॒दा॒तु॒ द॒दा॒तु॒ नः॒ । \newline
34. नो॒ र॒यिꣳ र॒यिम् नो॑ नो र॒यिम् । \newline
35. र॒यि मीशा॑न॒ ईशा॑नो र॒यिꣳ र॒यि मीशा॑नः । \newline
36. ईशा॑नो॒ जग॑तो॒ जग॑त॒ ईशा॑न॒ ईशा॑नो॒ जग॑तः । \newline
37. जग॑त॒ स्पति॒ष् पति॒र् जग॑तो॒ जग॑त॒ स्पतिः॑ । \newline
38. पति॒रिति॒ पतिः॑ । \newline
39. स नो॑ नः॒ स स नः॑ । \newline
40. नः॒ पू॒र्णेन॑ पू॒र्णेन॑ नो नः पू॒र्णेन॑ । \newline
41. पू॒र्णेन॑ वावनद् वावनत् पू॒र्णेन॑ पू॒र्णेन॑ वावनत् । \newline
42. वा॒व॒न॒दिति॑ वावनत् । \newline
43. धा॒ता प्र॒जायाः᳚ प्र॒जाया॑ धा॒ता धा॒ता प्र॒जायाः᳚ । \newline
44. प्र॒जाया॑ उ॒तोत प्र॒जायाः᳚ प्र॒जाया॑ उ॒त । \newline
45. प्र॒जाया॒ इति॑ प्र - जायाः᳚ । \newline
46. उ॒त रा॒यो रा॒य उ॒तोत रा॒यः । \newline
47. रा॒य ई॑श ईशे रा॒यो रा॒य ई॑शे । \newline
48. ई॒शे॒ धा॒ता धा॒तेश॑ ईशे धा॒ता । \newline
49. धा॒तेद मि॒दम् धा॒ता धा॒तेदम् । \newline
50. इ॒दम् ॅविश्व॒म् ॅविश्व॑ मि॒द मि॒दम् ॅविश्व᳚म् । \newline
51. विश्व॒म् भुव॑न॒म् भुव॑न॒म् ॅविश्व॒म् ॅविश्व॒म् भुव॑नम् । \newline
52. भुव॑नम् जजान जजान॒ भुव॑न॒म् भुव॑नम् जजान । \newline
53. ज॒जा॒नेति॑ जजान । \newline
54. धा॒ता पु॒त्रम् पु॒त्रम् धा॒ता धा॒ता पु॒त्रम् । \newline
55. पु॒त्रम् ॅयज॑मानाय॒ यज॑मानाय पु॒त्रम् पु॒त्रम् ॅयज॑मानाय । \newline
56. यज॑मानाय॒ दाता॒ दाता॒ यज॑मानाय॒ यज॑मानाय॒ दाता᳚ । \newline
57. दाता॒ तस्मै॒ तस्मै॒ दाता॒ दाता॒ तस्मै᳚ । \newline

\textbf{Ghana Paata } \newline

1. शु॒चे॒ शुच॑यः॒ शुच॑यः शुचे शुचे॒ शुच॑य श्चरन्ति चरन्ति॒ शुच॑यः शुचे शुचे॒ शुच॑य श्चरन्ति । \newline
2. शुच॑य श्चरन्ति चरन्ति॒ शुच॑यः॒ शुच॑य श्चरन्ति । \newline
3. च॒र॒न्तीति॑ चरन्ति । \newline
4. तु॒वि॒म्र॒क्षासो॑ दि॒व्या दि॒व्या स्तु॑विम्र॒क्षास॑ स्तुविम्र॒क्षासो॑ दि॒व्या नव॑ग्वा॒ नव॑ग्वा दि॒व्या स्तु॑विम्र॒क्षास॑ स्तुविम्र॒क्षासो॑ दि॒व्या नव॑ग्वाः । \newline
5. तु॒वि॒म्र॒क्षास॒ इति॑ तुवि - म्र॒क्षासः॑ । \newline
6. दि॒व्या नव॑ग्वा॒ नव॑ग्वा दि॒व्या दि॒व्या नव॑ग्वा॒ वना॒ वना॒ नव॑ग्वा दि॒व्या दि॒व्या नव॑ग्वा॒ वना᳚ । \newline
7. नव॑ग्वा॒ वना॒ वना॒ नव॑ग्वा॒ नव॑ग्वा॒ वना॑ वनन्ति वनन्ति॒ वना॒ नव॑ग्वा॒ नव॑ग्वा॒ वना॑ वनन्ति । \newline
8. वना॑ वनन्ति वनन्ति॒ वना॒ वना॑ वनन्ति धृष॒ता धृ॑ष॒ता व॑नन्ति॒ वना॒ वना॑ वनन्ति धृष॒ता । \newline
9. व॒न॒न्ति॒ धृ॒ष॒ता धृ॑ष॒ता व॑नन्ति वनन्ति धृष॒ता रु॒जन्तो॑ रु॒जन्तो॑ धृष॒ता व॑नन्ति वनन्ति धृष॒ता रु॒जन्तः॑ । \newline
10. धृ॒ष॒ता रु॒जन्तो॑ रु॒जन्तो॑ धृष॒ता धृ॑ष॒ता रु॒जन्तः॑ । \newline
11. रु॒जन्त॒ इति॑ रु॒जन्तः॑ । \newline
12. त्वा म॑ग्ने अग्ने॒ त्वाम् त्वा म॑ग्ने॒ मानु॑षी॒र् मानु॑षी रग्ने॒ त्वाम् त्वा म॑ग्ने॒ मानु॑षीः । \newline
13. अ॒ग्ने॒ मानु॑षी॒र् मानु॑षी रग्ने अग्ने॒ मानु॑षी रीडत ईडते॒ मानु॑षी रग्ने अग्ने॒ मानु॑षी रीडते । \newline
14. मानु॑षी रीडत ईडते॒ मानु॑षी॒र् मानु॑षी रीडते॒ विशो॒ विश॑ ईडते॒ मानु॑षी॒र् मानु॑षी रीडते॒ विशः॑ । \newline
15. ई॒ड॒ते॒ विशो॒ विश॑ ईडत ईडते॒ विशो॑ होत्रा॒विदꣳ॑ होत्रा॒विद॒म् ॅविश॑ ईडत ईडते॒ विशो॑ होत्रा॒विद᳚म् । \newline
16. विशो॑ होत्रा॒विदꣳ॑ होत्रा॒विद॒म् ॅविशो॒ विशो॑ होत्रा॒विद॒म् ॅविवि॑चि॒म् ॅविवि॑चिꣳ होत्रा॒विद॒म् ॅविशो॒ विशो॑ होत्रा॒विद॒म् ॅविवि॑चिम् । \newline
17. हो॒त्रा॒विद॒म् ॅविवि॑चि॒म् ॅविवि॑चिꣳ होत्रा॒विदꣳ॑ होत्रा॒विद॒म् ॅविवि॑चिꣳ रत्न॒धात॑मꣳ रत्न॒धात॑म॒म् ॅविवि॑चिꣳ होत्रा॒विदꣳ॑ होत्रा॒विद॒म् ॅविवि॑चिꣳ रत्न॒धात॑मम् । \newline
18. हो॒त्रा॒विद॒मिति॑ होत्रा - विद᳚म् । \newline
19. विवि॑चिꣳ रत्न॒धात॑मꣳ रत्न॒धात॑म॒म् ॅविवि॑चि॒म् ॅविवि॑चिꣳ रत्न॒धात॑मम् । \newline
20. विवि॑चि॒मिति॒ वि - वि॒चि॒म् । \newline
21. र॒त्न॒धात॑म॒मिति॑ रत्न - धात॑मम् । \newline
22. गुहा॒ सन्तꣳ॒॒ सन्त॒म् गुहा॒ गुहा॒ सन्तꣳ॑ सुभग सुभग॒ सन्त॒म् गुहा॒ गुहा॒ सन्तꣳ॑ सुभग । \newline
23. सन्तꣳ॑ सुभग सुभग॒ सन्तꣳ॒॒ सन्तꣳ॑ सुभग वि॒श्वद॑र्.शतम् ॅवि॒श्वद॑र्.शतꣳ सुभग॒ सन्तꣳ॒॒ सन्तꣳ॑ सुभग वि॒श्वद॑र्.शतम् । \newline
24. सु॒भ॒ग॒ वि॒श्वद॑र्.शतम् ॅवि॒श्वद॑र्.शतꣳ सुभग सुभग वि॒श्वद॑र्.शतम् तुविष्म॒णस॑म् तुविष्म॒णस॑म् ॅवि॒श्वद॑र्.शतꣳ सुभग सुभग वि॒श्वद॑र्.शतम् तुविष्म॒णस᳚म् । \newline
25. सु॒भ॒गेति॑ सु - भ॒ग॒ । \newline
26. वि॒श्वद॑र्.शतम् तुविष्म॒णस॑म् तुविष्म॒णस॑म् ॅवि॒श्वद॑र्.शतम् ॅवि॒श्वद॑र्.शतम् तुविष्म॒णसꣳ॑ सु॒यजꣳ॑ सु॒यज॑म् तुविष्म॒णस॑म् ॅवि॒श्वद॑र्.शतम् ॅवि॒श्वद॑र्.शतम् तुविष्म॒णसꣳ॑ सु॒यज᳚म् । \newline
27. वि॒श्वद॑र्.शत॒मिति॑ वि॒श्व - द॒र्॒.श॒त॒म् । \newline
28. तु॒वि॒ष्म॒णसꣳ॑ सु॒यजꣳ॑ सु॒यज॑म् तुविष्म॒णस॑म् तुविष्म॒णसꣳ॑ सु॒यज॑म् घृत॒श्रिय॑म् घृत॒श्रियꣳ॑ सु॒यज॑म् तुविष्म॒णस॑म् तुविष्म॒णसꣳ॑ सु॒यज॑म् घृत॒श्रिय᳚म् । \newline
29. सु॒यज॑म् घृत॒श्रिय॑म् घृत॒श्रियꣳ॑ सु॒यजꣳ॑ सु॒यज॑म् घृत॒श्रिय᳚म् । \newline
30. सु॒यज॒मिति॑ सु - यज᳚म् । \newline
31. घृ॒त॒श्रिय॒मिति॑ घृत - श्रिय᳚म् । \newline
32. धा॒ता द॑दातु ददातु धा॒ता धा॒ता द॑दातु नो नो ददातु धा॒ता धा॒ता द॑दातु नः । \newline
33. द॒दा॒तु॒ नो॒ नो॒ द॒दा॒तु॒ द॒दा॒तु॒ नो॒ र॒यिꣳ र॒यिम् नो॑ ददातु ददातु नो र॒यिम् । \newline
34. नो॒ र॒यिꣳ र॒यिम् नो॑ नो र॒यि मीशा॑न॒ ईशा॑नो र॒यिम् नो॑ नो र॒यि मीशा॑नः । \newline
35. र॒यि मीशा॑न॒ ईशा॑नो र॒यिꣳ र॒यि मीशा॑नो॒ जग॑तो॒ जग॑त॒ ईशा॑नो र॒यिꣳ र॒यि मीशा॑नो॒ जग॑तः । \newline
36. ईशा॑नो॒ जग॑तो॒ जग॑त॒ ईशा॑न॒ ईशा॑नो॒ जग॑त॒ स्पति॒ष् पति॒र् जग॑त॒ ईशा॑न॒ ईशा॑नो॒ जग॑त॒ स्पतिः॑ । \newline
37. जग॑त॒ स्पति॒ष् पति॒र् जग॑तो॒ जग॑त॒ स्पतिः॑ । \newline
38. पति॒रिति॒ पतिः॑ । \newline
39. स नो॑ नः॒ स स नः॑ पू॒र्णेन॑ पू॒र्णेन॑ नः॒ स स नः॑ पू॒र्णेन॑ । \newline
40. नः॒ पू॒र्णेन॑ पू॒र्णेन॑ नो नः पू॒र्णेन॑ वावनद् वावनत् पू॒र्णेन॑ नो नः पू॒र्णेन॑ वावनत् । \newline
41. पू॒र्णेन॑ वावनद् वावनत् पू॒र्णेन॑ पू॒र्णेन॑ वावनत् । \newline
42. वा॒व॒न॒दिति॑ वावनत् । \newline
43. धा॒ता प्र॒जायाः᳚ प्र॒जाया॑ धा॒ता धा॒ता प्र॒जाया॑ उ॒तोत प्र॒जाया॑ धा॒ता धा॒ता प्र॒जाया॑ उ॒त । \newline
44. प्र॒जाया॑ उ॒तोत प्र॒जायाः᳚ प्र॒जाया॑ उ॒त रा॒यो रा॒य उ॒त प्र॒जायाः᳚ प्र॒जाया॑ उ॒त रा॒यः । \newline
45. प्र॒जाया॒ इति॑ प्र - जायाः᳚ । \newline
46. उ॒त रा॒यो रा॒य उ॒तोत रा॒य ई॑श ईशे रा॒य उ॒तोत रा॒य ई॑शे । \newline
47. रा॒य ई॑श ईशे रा॒यो रा॒य ई॑शे धा॒ता धा॒तेशे॑ रा॒यो रा॒य ई॑शे धा॒ता । \newline
48. ई॒शे॒ धा॒ता धा॒तेश॑ ईशे धा॒तेद मि॒दम् धा॒तेश॑ ईशे धा॒तेदम् । \newline
49. धा॒तेद मि॒दम् धा॒ता धा॒तेदम् ॅविश्व॒म् ॅविश्व॑ मि॒दम् धा॒ता धा॒तेदम् ॅविश्व᳚म् । \newline
50. इ॒दम् ॅविश्व॒म् ॅविश्व॑ मि॒द मि॒दम् ॅविश्व॒म् भुव॑न॒म् भुव॑न॒म् ॅविश्व॑ मि॒द मि॒दम् ॅविश्व॒म् भुव॑नम् । \newline
51. विश्व॒म् भुव॑न॒म् भुव॑न॒म् ॅविश्व॒म् ॅविश्व॒म् भुव॑नम् जजान जजान॒ भुव॑न॒म् ॅविश्व॒म् ॅविश्व॒म् भुव॑नम् जजान । \newline
52. भुव॑नम् जजान जजान॒ भुव॑न॒म् भुव॑नम् जजान । \newline
53. ज॒जा॒नेति॑ जजान । \newline
54. धा॒ता पु॒त्रम् पु॒त्रम् धा॒ता धा॒ता पु॒त्रम् ॅयज॑मानाय॒ यज॑मानाय पु॒त्रम् धा॒ता धा॒ता पु॒त्रम् ॅयज॑मानाय । \newline
55. पु॒त्रम् ॅयज॑मानाय॒ यज॑मानाय पु॒त्रम् पु॒त्रम् ॅयज॑मानाय॒ दाता॒ दाता॒ यज॑मानाय पु॒त्रम् पु॒त्रम् ॅयज॑मानाय॒ दाता᳚ । \newline
56. यज॑मानाय॒ दाता॒ दाता॒ यज॑मानाय॒ यज॑मानाय॒ दाता॒ तस्मै॒ तस्मै॒ दाता॒ यज॑मानाय॒ यज॑मानाय॒ दाता॒ तस्मै᳚ । \newline
57. दाता॒ तस्मै॒ तस्मै॒ दाता॒ दाता॒ तस्मा॑ उ वु॒ तस्मै॒ दाता॒ दाता॒ तस्मा॑ उ । \newline
\pagebreak
\markright{ TS 3.3.11.3  \hfill https://www.vedavms.in \hfill}

\section{ TS 3.3.11.3 }

\textbf{TS 3.3.11.3 } \newline
\textbf{Samhita Paata} \newline

तस्मा॑ उ ह॒व्यं घृ॒तव॑द्विधेम ॥ धा॒ता द॑दातु नो र॒यिं प्राचीं᳚ जी॒वातु॒मक्षि॑तां । व॒यं दे॒वस्य॑ धीमहि सुम॒तिꣳ स॒त्यरा॑धसः ॥धा॒ता द॑दातु दा॒शुषे॒ वसू॑नि प्र॒जाका॑माय मी॒ढुषे॑ दुरो॒णे । तस्मै॑ दे॒वा अ॒मृताः॒ संॅव्य॑यन्तां॒ ॅविश्वे॑ दे॒वासो॒ अदि॑तिः स॒जोषाः᳚ ॥ अनु॑ नो॒ऽद्याऽनु॑मतिर्य॒ज्ञ्ं दे॒वेषु॑ मन्यतां । अ॒ग्निश्च॑ हव्य॒वाह॑नो॒ भव॑तां दा॒शुषे॒ मयः॑ ॥ अन्विद॑नुमते॒ त्वं - [  ] \newline

\textbf{Pada Paata} \newline

तस्मै᳚ । उ॒ । ह॒व्यम् । घृ॒तव॒दिति॑ घृ॒त-व॒त् । वि॒धे॒म॒ ॥ धा॒ता । द॒दा॒तु॒ । नः॒ । र॒यिम् । प्राची᳚म् । जी॒वातु᳚म् । अक्षि॑ताम् ॥ व॒यम् । दे॒वस्य॑ । धी॒म॒हि॒ । सु॒म॒तिमिति॑ सु - म॒तिम् । स॒त्यरा॑धस॒ इति॑ स॒त्य - रा॒ध॒सः॒ ॥ धा॒ता । द॒दा॒तु॒ । दा॒शुषे᳚ । वसू॑नि । प्र॒जाका॑मा॒येति॑ प्र॒जा - का॒मा॒य॒ । मी॒ढुषे᳚ । दु॒रो॒ण इति॑ दुः-ओ॒ने ॥ तस्मै᳚ । दे॒वाः । अ॒मृताः᳚ । समिति॑ । व्य॒य॒न्ता॒म् । विश्वे᳚ । दे॒वासः॑ । अदि॑तिः । स॒जोषा॒ इति॑ स - जोषाः᳚ ॥ अन्विति॑ । नः॒ । अ॒द्य । अनु॑मति॒रित्यनु॑ - म॒तिः॒ । य॒ज्ञ्म् । दे॒वेषु॑ । म॒न्य॒ता॒म् ॥ अ॒ग्निः । च॒ । ह॒व्य॒वाह॑न॒ इति॑ हव्य - वाह॑नः । भव॑ताम् । दा॒शुषे᳚ । मयः॑ ॥ अन्विति॑ । इत् । अ॒नु॒म॒त॒ इत्य॑नु - म॒ते॒ । त्वम् ।  \newline


\textbf{Krama Paata} \newline

तस्मा॑ उ । उ॒ ह॒व्यम् । ह॒व्यम् घृ॒तव॑त् । घृ॒तव॑द् विधेम । घृ॒तव॒दिति॑ घृत - व॒त्॒ । वि॒धे॒मेति॑ विधेम ॥ धा॒ता द॑दातु । द॒दा॒तु॒ नः॒ । नो॒ र॒यिम् । र॒यिम् प्राची᳚म् । प्राची᳚म् जी॒वातु᳚म् । जी॒वातु॒मक्षि॑ताम् । अक्षि॑ता॒मित्यक्षि॑ताम् ॥ व॒यम् दे॒वस्य॑ । दे॒वस्य॑ धीमहि । धी॒म॒हि॒ सु॒म॒तिम् । सु॒म॒तिꣳ स॒त्यरा॑धसः । सु॒म॒तिमिति॑ सु - म॒तिम् । स॒त्यरा॑धस॒ इति॑ स॒त्य - रा॒ध॒सः॒ ॥ धा॒ता द॑दातु । द॒दा॒तु॒ दा॒शुषे᳚ । दा॒शुषे॒ वसू॑नि । वसू॑नि प्र॒जाका॑माय । प्र॒जाका॑माय मी॒ढुषे᳚ । प्र॒जाका॑मा॒येति॑ प्र॒जा - का॒मा॒य॒ । मी॒ढुषे॑ दुरो॒णे । दु॒रो॒ण इति॑ दुः - ओ॒ने ॥ तस्मै॑ दे॒वाः । दे॒वा अ॒मृताः᳚ । अ॒मृताः॒ सम् । सं ॅव्य॑यन्ताम् । व्य॒य॒न्तां॒ ॅविश्वे᳚ । विश्वे॑ दे॒वासः॑ । दे॒वासो॒ अदि॑तिः । अदि॑तिः स॒जोषाः᳚ । स॒जोषा॒ इति॑ स - जोषाः᳚ ॥ अनु॑ नः । नो॒ ऽद्य । अ॒द्यानु॑मतिः । अनु॑मतिर् य॒ज्ञ्म् । अनु॑मति॒रित्यनु॑ - म॒तिः॒ । य॒ज्ञ्म् दे॒वेषु॑ । दे॒वेषु॑ मन्यताम् । म॒न्य॒ता॒मिति॑ मन्यताम् ॥ अ॒ग्निश्च॑ । च॒ ह॒व्य॒वाह॑नः । ह॒व्य॒वाह॑नो॒ भव॑ताम् । ह॒व्य॒वाह॑न॒ इति॑ हव्य - वाह॑नः । भव॑ताम् दा॒शुषे᳚ । दा॒शुषे॒ मयः॑ । मय॒ इति॒ मयः॑ ॥ अन्वित् । इद॑नुमते । अ॒नु॒म॒ते॒ त्वम् । अ॒नु॒म॒त॒ इत्य॑नु - म॒ते॒ । त्वम् मन्या॑सै \newline

\textbf{Jatai Paata} \newline

1. तस्मा॑ उ वु॒ तस्मै॒ तस्मा॑ उ । \newline
2. उ॒ ह॒व्यꣳ ह॒व्य मु॑ वु ह॒व्यम् । \newline
3. ह॒व्यम् घृ॒तव॑द् घृ॒तव॑ द्ध॒व्यꣳ ह॒व्यम् घृ॒तव॑त् । \newline
4. घृ॒तव॑द् विधेम विधेम घृ॒तव॑द् घृ॒तव॑द् विधेम । \newline
5. घृ॒तव॒दिति॑ घृ॒त - व॒त् । \newline
6. वि॒धे॒मेति॑ विधेम । \newline
7. धा॒ता द॑दातु ददातु धा॒ता धा॒ता द॑दातु । \newline
8. द॒दा॒तु॒ नो॒ नो॒ द॒दा॒तु॒ द॒दा॒तु॒ नः॒ । \newline
9. नो॒ र॒यिꣳ र॒यिम् नो॑ नो र॒यिम् । \newline
10. र॒यिम् प्राची॒म् प्राचीꣳ॑ र॒यिꣳ र॒यिम् प्राची᳚म् । \newline
11. प्राची᳚म् जी॒वातु॑म् जी॒वातु॒म् प्राची॒म् प्राची᳚म् जी॒वातु᳚म् । \newline
12. जी॒वातु॒ मक्षि॑ता॒ मक्षि॑ताम् जी॒वातु॑म् जी॒वातु॒ मक्षि॑ताम् । \newline
13. अक्षि॑ता॒मित्यक्षि॑ताम् । \newline
14. व॒यम् दे॒वस्य॑ दे॒वस्य॑ व॒यम् ॅव॒यम् दे॒वस्य॑ । \newline
15. दे॒वस्य॑ धीमहि धीमहि दे॒वस्य॑ दे॒वस्य॑ धीमहि । \newline
16. धी॒म॒हि॒ सु॒म॒तिꣳ सु॑म॒तिम् धी॑महि धीमहि सुम॒तिम् । \newline
17. सु॒म॒तिꣳ स॒त्यरा॑धसः स॒त्यरा॑धसः सुम॒तिꣳ सु॑म॒तिꣳ स॒त्यरा॑धसः । \newline
18. सु॒म॒तिमिति॑ सु - म॒तिम् । \newline
19. स॒त्यरा॑धस॒ इति॑ स॒त्य - रा॒ध॒सः॒ । \newline
20. धा॒ता द॑दातु ददातु धा॒ता धा॒ता द॑दातु । \newline
21. द॒दा॒तु॒ दा॒शुषे॑ दा॒शुषे॑ ददातु ददातु दा॒शुषे᳚ । \newline
22. दा॒शुषे॒ वसू॑नि॒ वसू॑नि दा॒शुषे॑ दा॒शुषे॒ वसू॑नि । \newline
23. वसू॑नि प्र॒जाका॑माय प्र॒जाका॑माय॒ वसू॑नि॒ वसू॑नि प्र॒जाका॑माय । \newline
24. प्र॒जाका॑माय मी॒ढुषे॑ मी॒ढुषे᳚ प्र॒जाका॑माय प्र॒जाका॑माय मी॒ढुषे᳚ । \newline
25. प्र॒जाका॑मा॒येति॑ प्र॒जा - का॒मा॒य॒ । \newline
26. मी॒ढुषे॑ दुरो॒णे दु॑रो॒णे मी॒ढुषे॑ मी॒ढुषे॑ दुरो॒णे । \newline
27. दु॒रो॒ण इति॑ दुः - ओ॒ने । \newline
28. तस्मै॑ दे॒वा दे॒वा स्तस्मै॒ तस्मै॑ दे॒वाः । \newline
29. दे॒वा अ॒मृता॑ अ॒मृता॑ दे॒वा दे॒वा अ॒मृताः᳚ । \newline
30. अ॒मृताः॒ सꣳ स म॒मृता॑ अ॒मृताः॒ सम् । \newline
31. सम् ॅव्य॑यन्ताम् ॅव्ययन्ताꣳ॒॒ सꣳ सम् ॅव्य॑यन्ताम् । \newline
32. व्य॒य॒न्ता॒म् ॅविश्वे॒ विश्वे᳚ व्ययन्ताम् ॅव्ययन्ता॒म् ॅविश्वे᳚ । \newline
33. विश्वे॑ दे॒वासो॑ दे॒वासो॒ विश्वे॒ विश्वे॑ दे॒वासः॑ । \newline
34. दे॒वासो॒ अदि॑ति॒ रदि॑तिर् दे॒वासो॑ दे॒वासो॒ अदि॑तिः । \newline
35. अदि॑तिः स॒जोषाः᳚ स॒जोषा॒ अदि॑ति॒ रदि॑तिः स॒जोषाः᳚ । \newline
36. स॒जोषा॒ इति॑ स - जोषाः᳚ । \newline
37. अनु॑ नो नो॒ अन्वनु॑ नः । \newline
38. नो॒ ऽद्याद्य नो॑ नो॒ ऽद्य । \newline
39. अ॒द्या नु॑मति॒ रनु॑मति र॒द्याद्या नु॑मतिः । \newline
40. अनु॑मतिर् य॒ज्ञ्म् ॅय॒ज्ञ् मनु॑मति॒ रनु॑मतिर् य॒ज्ञ्म् । \newline
41. अनु॑मति॒रित्यनु॑ - म॒तिः॒ । \newline
42. य॒ज्ञ्म् दे॒वेषु॑ दे॒वेषु॑ य॒ज्ञ्म् ॅय॒ज्ञ्म् दे॒वेषु॑ । \newline
43. दे॒वेषु॑ मन्यताम् मन्यताम् दे॒वेषु॑ दे॒वेषु॑ मन्यताम् । \newline
44. म॒न्य॒ता॒मिति॑ मन्यताम् । \newline
45. अ॒ग्निश्च॑ चा॒ग्नि र॒ग्निश्च॑ । \newline
46. च॒ ह॒व्य॒वाह॑नो हव्य॒वाह॑नश्च च हव्य॒वाह॑नः । \newline
47. ह॒व्य॒वाह॑नो॒ भव॑ता॒म् भव॑ताꣳ हव्य॒वाह॑नो हव्य॒वाह॑नो॒ भव॑ताम् । \newline
48. ह॒व्य॒वाह॑न॒ इति॑ हव्य - वाह॑नः । \newline
49. भव॑ताम् दा॒शुषे॑ दा॒शुषे॒ भव॑ता॒म् भव॑ताम् दा॒शुषे᳚ । \newline
50. दा॒शुषे॒ मयो॒ मयो॑ दा॒शुषे॑ दा॒शुषे॒ मयः॑ । \newline
51. मय॒ इति॒ मयः॑ । \newline
52. अन्वि दिद न्वन्वित् । \newline
53. इद॑नुमते ऽनुमत॒ इदिद॑नुमते । \newline
54. अ॒नु॒म॒ते॒ त्वम् त्व म॑नुमते ऽनुमते॒ त्वम् । \newline
55. अ॒नु॒म॒त॒ इत्य॑नु - म॒ते॒ । \newline
56. त्वम् मन्या॑सै॒ मन्या॑सै॒ त्वम् त्वम् मन्या॑सै । \newline

\textbf{Ghana Paata } \newline

1. तस्मा॑ उ वु॒ तस्मै॒ तस्मा॑ उ ह॒व्यꣳ ह॒व्य मु॒ तस्मै॒ तस्मा॑ उ ह॒व्यम् । \newline
2. उ॒ ह॒व्यꣳ ह॒व्य मु॑ वु ह॒व्यम् घृ॒तव॑द् घृ॒तव॑ द्ध॒व्य मु॑ वु ह॒व्यम् घृ॒तव॑त् । \newline
3. ह॒व्यम् घृ॒तव॑द् घृ॒तव॑ द्ध॒व्यꣳ ह॒व्यम् घृ॒तव॑द् विधेम विधेम घृ॒तव॑ द्ध॒व्यꣳ ह॒व्यम् घृ॒तव॑द् विधेम । \newline
4. घृ॒तव॑द् विधेम विधेम घृ॒तव॑द् घृ॒तव॑द् विधेम । \newline
5. घृ॒तव॒दिति॑ घृ॒त - व॒त् । \newline
6. वि॒धे॒मेति॑ विधेम । \newline
7. धा॒ता द॑दातु ददातु धा॒ता धा॒ता द॑दातु नो नो ददातु धा॒ता धा॒ता द॑दातु नः । \newline
8. द॒दा॒तु॒ नो॒ नो॒ द॒दा॒तु॒ द॒दा॒तु॒ नो॒ र॒यिꣳ र॒यिम् नो॑ ददातु ददातु नो र॒यिम् । \newline
9. नो॒ र॒यिꣳ र॒यिम् नो॑ नो र॒यिम् प्राची॒म् प्राचीꣳ॑ र॒यिम् नो॑ नो र॒यिम् प्राची᳚म् । \newline
10. र॒यिम् प्राची॒म् प्राचीꣳ॑ र॒यिꣳ र॒यिम् प्राची᳚म् जी॒वातु॑म् जी॒वातु॒म् प्राचीꣳ॑ र॒यिꣳ र॒यिम् प्राची᳚म् जी॒वातु᳚म् । \newline
11. प्राची᳚म् जी॒वातु॑म् जी॒वातु॒म् प्राची॒म् प्राची᳚म् जी॒वातु॒ मक्षि॑ता॒ मक्षि॑ताम् जी॒वातु॒म् प्राची॒म् प्राची᳚म् जी॒वातु॒ मक्षि॑ताम् । \newline
12. जी॒वातु॒ मक्षि॑ता॒ मक्षि॑ताम् जी॒वातु॑म् जी॒वातु॒ मक्षि॑ताम् । \newline
13. अक्षि॑ता॒मित्यक्षि॑ताम् । \newline
14. व॒यम् दे॒वस्य॑ दे॒वस्य॑ व॒यम् ॅव॒यम् दे॒वस्य॑ धीमहि धीमहि दे॒वस्य॑ व॒यम् ॅव॒यम् दे॒वस्य॑ धीमहि । \newline
15. दे॒वस्य॑ धीमहि धीमहि दे॒वस्य॑ दे॒वस्य॑ धीमहि सुम॒तिꣳ सु॑म॒तिम् धी॑महि दे॒वस्य॑ दे॒वस्य॑ धीमहि सुम॒तिम् । \newline
16. धी॒म॒हि॒ सु॒म॒तिꣳ सु॑म॒तिम् धी॑महि धीमहि सुम॒तिꣳ स॒त्यरा॑धसः स॒त्यरा॑धसः सुम॒तिम् धी॑महि धीमहि सुम॒तिꣳ स॒त्यरा॑धसः । \newline
17. सु॒म॒तिꣳ स॒त्यरा॑धसः स॒त्यरा॑धसः सुम॒तिꣳ सु॑म॒तिꣳ स॒त्यरा॑धसः । \newline
18. सु॒म॒तिमिति॑ सु - म॒तिम् । \newline
19. स॒त्यरा॑धस॒ इति॑ स॒त्य - रा॒ध॒सः॒ । \newline
20. धा॒ता द॑दातु ददातु धा॒ता धा॒ता द॑दातु दा॒शुषे॑ दा॒शुषे॑ ददातु धा॒ता धा॒ता द॑दातु दा॒शुषे᳚ । \newline
21. द॒दा॒तु॒ दा॒शुषे॑ दा॒शुषे॑ ददातु ददातु दा॒शुषे॒ वसू॑नि॒ वसू॑नि दा॒शुषे॑ ददातु ददातु दा॒शुषे॒ वसू॑नि । \newline
22. दा॒शुषे॒ वसू॑नि॒ वसू॑नि दा॒शुषे॑ दा॒शुषे॒ वसू॑नि प्र॒जाका॑माय प्र॒जाका॑माय॒ वसू॑नि दा॒शुषे॑ दा॒शुषे॒ वसू॑नि प्र॒जाका॑माय । \newline
23. वसू॑नि प्र॒जाका॑माय प्र॒जाका॑माय॒ वसू॑नि॒ वसू॑नि प्र॒जाका॑माय मी॒ढुषे॑ मी॒ढुषे᳚ प्र॒जाका॑माय॒ वसू॑नि॒ वसू॑नि प्र॒जाका॑माय मी॒ढुषे᳚ । \newline
24. प्र॒जाका॑माय मी॒ढुषे॑ मी॒ढुषे᳚ प्र॒जाका॑माय प्र॒जाका॑माय मी॒ढुषे॑ दुरो॒णे दु॑रो॒णे मी॒ढुषे᳚ प्र॒जाका॑माय प्र॒जाका॑माय मी॒ढुषे॑ दुरो॒णे । \newline
25. प्र॒जाका॑मा॒येति॑ प्र॒जा - का॒मा॒य॒ । \newline
26. मी॒ढुषे॑ दुरो॒णे दु॑रो॒णे मी॒ढुषे॑ मी॒ढुषे॑ दुरो॒णे । \newline
27. दु॒रो॒ण इति॑ दुः - ओ॒ने । \newline
28. तस्मै॑ दे॒वा दे॒वा स्तस्मै॒ तस्मै॑ दे॒वा अ॒मृता॑ अ॒मृता॑ दे॒वा स्तस्मै॒ तस्मै॑ दे॒वा अ॒मृताः᳚ । \newline
29. दे॒वा अ॒मृता॑ अ॒मृता॑ दे॒वा दे॒वा अ॒मृताः॒ सꣳ स म॒मृता॑ दे॒वा दे॒वा अ॒मृताः॒ सम् । \newline
30. अ॒मृताः॒ सꣳ स म॒मृता॑ अ॒मृताः॒ सम् ॅव्य॑यन्ताम् ॅव्ययन्ताꣳ॒॒ स म॒मृता॑ अ॒मृताः॒ सम् ॅव्य॑यन्ताम् । \newline
31. सम् ॅव्य॑यन्ताम् ॅव्ययन्ताꣳ॒॒ सꣳ सम् ॅव्य॑यन्ता॒म् ॅविश्वे॒ विश्वे᳚ व्ययन्ताꣳ॒॒ सꣳ सम् ॅव्य॑यन्ता॒म् ॅविश्वे᳚ । \newline
32. व्य॒य॒न्ता॒म् ॅविश्वे॒ विश्वे᳚ व्ययन्ताम् ॅव्ययन्ता॒म् ॅविश्वे॑ दे॒वासो॑ दे॒वासो॒ विश्वे᳚ व्ययन्ताम् ॅव्ययन्ता॒म् ॅविश्वे॑ दे॒वासः॑ । \newline
33. विश्वे॑ दे॒वासो॑ दे॒वासो॒ विश्वे॒ विश्वे॑ दे॒वासो॒ अदि॑ति॒ रदि॑तिर् दे॒वासो॒ विश्वे॒ विश्वे॑ दे॒वासो॒ अदि॑तिः । \newline
34. दे॒वासो॒ अदि॑ति॒ रदि॑तिर् दे॒वासो॑ दे॒वासो॒ अदि॑तिः स॒जोषाः᳚ स॒जोषा॒ अदि॑तिर् दे॒वासो॑ दे॒वासो॒ अदि॑तिः स॒जोषाः᳚ । \newline
35. अदि॑तिः स॒जोषाः᳚ स॒जोषा॒ अदि॑ति॒ रदि॑तिः स॒जोषाः᳚ । \newline
36. स॒जोषा॒ इति॑ स - जोषाः᳚ । \newline
37. अनु॑ नो नो॒ अन्वनु॑ नो॒ ऽद्याद्य नो॒ अन्वनु॑ नो॒ ऽद्य । \newline
38. नो॒ ऽद्याद्य नो॑ नो॒ ऽद्यानु॑मति॒ रनु॑मति र॒द्य नो॑ नो॒ ऽद्यानु॑मतिः । \newline
39. अ॒द्यानु॑मति॒ रनु॑मति र॒द्याद्या नु॑मतिर् य॒ज्ञ्म् ॅय॒ज्ञ् मनु॑मति र॒द्याद्या नु॑मतिर् य॒ज्ञ्म् । \newline
40. अनु॑मतिर् य॒ज्ञ्म् ॅय॒ज्ञ् मनु॑मति॒ रनु॑मतिर् य॒ज्ञ्म् दे॒वेषु॑ दे॒वेषु॑ य॒ज्ञ् मनु॑मति॒ रनु॑मतिर् य॒ज्ञ्म् दे॒वेषु॑ । \newline
41. अनु॑मति॒रित्यनु॑ - म॒तिः॒ । \newline
42. य॒ज्ञ्म् दे॒वेषु॑ दे॒वेषु॑ य॒ज्ञ्म् ॅय॒ज्ञ्म् दे॒वेषु॑ मन्यताम् मन्यताम् दे॒वेषु॑ य॒ज्ञ्म् ॅय॒ज्ञ्म् दे॒वेषु॑ मन्यताम् । \newline
43. दे॒वेषु॑ मन्यताम् मन्यताम् दे॒वेषु॑ दे॒वेषु॑ मन्यताम् । \newline
44. म॒न्य॒ता॒मिति॑ मन्यताम् । \newline
45. अ॒ग्निश्च॑ चा॒ग्नि र॒ग्निश्च॑ हव्य॒वाह॑नो हव्य॒वाह॑न श्चा॒ग्नि र॒ग्निश्च॑ हव्य॒वाह॑नः । \newline
46. च॒ ह॒व्य॒वाह॑नो हव्य॒वाह॑नश्च च हव्य॒वाह॑नो॒ भव॑ता॒म् भव॑ताꣳ हव्य॒वाह॑नश्च च हव्य॒वाह॑नो॒ भव॑ताम् । \newline
47. ह॒व्य॒वाह॑नो॒ भव॑ता॒म् भव॑ताꣳ हव्य॒वाह॑नो हव्य॒वाह॑नो॒ भव॑ताम् दा॒शुषे॑ दा॒शुषे॒ भव॑ताꣳ हव्य॒वाह॑नो हव्य॒वाह॑नो॒ भव॑ताम् दा॒शुषे᳚ । \newline
48. ह॒व्य॒वाह॑न॒ इति॑ हव्य - वाह॑नः । \newline
49. भव॑ताम् दा॒शुषे॑ दा॒शुषे॒ भव॑ता॒म् भव॑ताम् दा॒शुषे॒ मयो॒ मयो॑ दा॒शुषे॒ भव॑ता॒म् भव॑ताम् दा॒शुषे॒ मयः॑ । \newline
50. दा॒शुषे॒ मयो॒ मयो॑ दा॒शुषे॑ दा॒शुषे॒ मयः॑ । \newline
51. मय॒ इति॒ मयः॑ । \newline
52. अन्वि दिदन्वन् वि द॑नुमते ऽनुमत॒ इद न्वन्वि द॑नुमते । \newline
53. इद॑नुमते ऽनुमत॒ इदि द॑नुमते॒ त्वम् त्व म॑नुमत॒ इदि द॑नुमते॒ त्वम् । \newline
54. अ॒नु॒म॒ते॒ त्वम् त्व म॑नुमते ऽनुमते॒ त्वम् मन्या॑सै॒ मन्या॑सै॒ त्व म॑नुमते ऽनुमते॒ त्वम् मन्या॑सै । \newline
55. अ॒नु॒म॒त॒ इत्य॑नु - म॒ते॒ । \newline
56. त्वम् मन्या॑सै॒ मन्या॑सै॒ त्वम् त्वम् मन्या॑सै॒ शꣳ शम् मन्या॑सै॒ त्वम् त्वम् मन्या॑सै॒ शम् । \newline
\pagebreak
\markright{ TS 3.3.11.4  \hfill https://www.vedavms.in \hfill}

\section{ TS 3.3.11.4 }

\textbf{TS 3.3.11.4 } \newline
\textbf{Samhita Paata} \newline

मन्या॑सै॒ शञ्च॑नः कृधि । क्रत्वे॒ दक्षा॑य नो हिनु॒ प्रण॒ आयूꣳ॑षि तारिषः ॥ अनु॑ मन्यता-मनु॒मन्य॑माना प्र॒जाव॑न्तꣳ र॒यिमक्षी॑यमाणं । तस्यै॑ व॒यꣳ हेड॑सि॒ माऽपि॑ भूम॒ सा नो॑ दे॒वी सु॒हवा॒ शर्म॑ यच्छतु ॥ यस्या॑मि॒दं प्र॒दिशि॒ यद्वि॒रोच॒तेऽनु॑मतिं॒ प्रति॑ भूषन्त्या॒यवः॑ । यस्या॑ उ॒पस्थ॑ उ॒र्व॑न्तरि॑क्षꣳ॒॒ सा नो॑ दे॒वी सु॒हवा॒ शर्म॑ यच्छतु ॥ \newline

\textbf{Pada Paata} \newline

मन्या॑सै । शम् । च॒ । नः॒ । कृ॒धि॒ ॥ क्रत्वे᳚ । दक्षा॑य । नः॒ । हि॒नु॒ । प्रेति॑ । नः॒ । आयूꣳ॑षि । ता॒रि॒षः॒ ॥ अन्विति॑ । म॒न्य॒ता॒म् । अ॒नु॒मन्य॑मा॒नेत्य॑नु - मन्य॑माना । प्र॒जाव॑न्त॒मिति॑ प्र॒जा - व॒न्त॒म् । र॒यिम् । अक्षी॑यमाणम् ॥ तस्यै᳚ । व॒यम् । हेड॑सि । मा । अपीति॑ । भू॒म॒ । सा । नः॒ । दे॒वी । सु॒हवेति॑ सु - हवा᳚ । शर्म॑ । य॒च्छ॒तु॒ ॥ यस्या᳚म् । इ॒दम् । प्र॒दिशीति॑ प्र - दिशि॑ । यत् । वि॒रोच॑त॒ इति॑ वि - रोच॑ते । अनु॑मति॒मित्यनु॑ - म॒ति॒म् । प्रतीति॑ । भू॒ष॒न्ति॒ । आ॒यवः॑ ॥ यस्याः᳚ । उ॒पस्थ॒ इत्यु॒प - स्थः॒ । उ॒रु । अ॒न्तरि॑क्षम् । सा । नः॒ । दे॒वी । सु॒हवेति॑ सु - हवा᳚ । शर्म॑ । य॒च्छ॒तु॒ ॥  \newline


\textbf{Krama Paata} \newline

मन्या॑सै॒ शम् । शम् च॑ । च॒ नः॒ । नः॒ कृ॒धि॒ । कृ॒धीति॑ कृधि ॥ क्रत्वे॒ दक्षा॑य । दक्षा॑य नः । नो॒ हि॒नु॒ । हि॒नु॒ प्र । प्र णः॑ । न॒ आयूꣳ॑षि । आयूꣳ॑षि तारिषः । ता॒रि॒ष॒ इति॑ तारिषः ॥ अनु॑ मन्यताम् । म॒न्य॒ता॒म॒नु॒मन्य॑माना । अ॒नु॒मन्य॑माना प्र॒जाव॑न्तम् । अ॒नु॒मन्य॑मा॒नेत्य॑नु - मन्य॑माना । प्र॒जाव॑न्तꣳ र॒यिम् । प्र॒जाव॑न्त॒मिति॑ प्र॒जा - व॒न्त॒म् । र॒यिमक्षी॑यमाणम् । अक्षी॑यमाण॒मित्यक्षी॑यमाणम् ॥ तस्यै॑ व॒यम् । व॒यꣳ हेड॑सि । हेड॑सि॒ मा । मा ऽपि॑ । अपि॑ भूम । भू॒म॒ सा । सा नः॑ । नो॒ दे॒वी । दे॒वी सु॒हवा᳚ । सु॒हवा॒ शर्म॑ । सु॒हवेति॑ सु - हवा᳚ । शर्म॑ यच्छतु । य॒च्छ॒त्विति॑ यच्छतु ॥ यस्या॑मि॒दम् । इ॒दम् प्र॒दिशि॑ । प्र॒दिशि॒ यत् । प्र॒दिशीति॑ प्र - दिशि॑ । यद् वि॒रोच॑ते । वि॒रोच॒ते ऽनु॑मतिम् । वि॒रोच॑त॒ इति॑ वि - रोच॑ते । अनु॑मति॒म् प्रति॑ । अनु॑मति॒मित्यनु॑ - म॒ति॒म् । प्रति॑ भूषन्ति । भू॒ष॒न्त्या॒यवः॑ । आ॒यव॒ इत्या॒यवः॑ ॥ यस्या॑ उ॒पस्थः॑ । उ॒पस्थ॑ उ॒रु । उ॒पस्थ॒ इत्यु॒प - स्थः॒ । उ॒र्व॑न्तरि॑क्षम् । अ॒न्तरि॑क्षꣳ॒॒ सा । सा नः॑ । नो॒ दे॒वी॒ । दे॒वी सु॒हवा᳚ । सु॒हवा॒ शर्म॑ । सु॒हवेति॑ सु - हवा᳚ । शर्म॑ यच्छतु । य॒च्छ॒त्विति॑ यच्छतु । \newline

\textbf{Jatai Paata} \newline

1. मन्या॑सै॒ शꣳ शम् मन्या॑सै॒ मन्या॑सै॒ शम् । \newline
2. शम् च॑ च॒ शꣳ शम् च॑ । \newline
3. च॒ नो॒ न॒श्च॒ च॒ नः॒ । \newline
4. नः॒ कृ॒धि॒ कृ॒धि॒ नो॒ नः॒ कृ॒धि॒ । \newline
5. कृ॒धीति॑ कृधि । \newline
6. क्रत्वे॒ दक्षा॑य॒ दक्षा॑य॒ क्रत्वे॒ क्रत्वे॒ दक्षा॑य । \newline
7. दक्षा॑य नो नो॒ दक्षा॑य॒ दक्षा॑य नः । \newline
8. नो॒ हि॒नु॒ हि॒नु॒ नो॒ नो॒ हि॒नु॒ । \newline
9. हि॒नु॒ प्र प्र हि॑नु हिनु॒ प्र । \newline
10. प्र णो॑ नः॒ प्र प्र णः॑ । \newline
11. न॒ आयूꣳ॒॒ ष्यायूꣳ॑षि नो न॒ आयूꣳ॑षि । \newline
12. आयूꣳ॑षि तारिष स्तारिष॒ आयूꣳ॒॒ ष्यायूꣳ॑षि तारिषः । \newline
13. ता॒रि॒ष॒ इति॑ तारिषः । \newline
14. अनु॑ मन्यताम् मन्यता॒ मन्वनु॑ मन्यताम् । \newline
15. म॒न्य॒ता॒ म॒नु॒मन्य॑माना ऽनु॒मन्य॑माना मन्यताम् मन्यता मनु॒मन्य॑माना । \newline
16. अ॒नु॒मन्य॑माना प्र॒जाव॑न्तम् प्र॒जाव॑न्त मनु॒मन्य॑माना ऽनु॒मन्य॑माना प्र॒जाव॑न्तम् । \newline
17. अ॒नु॒मन्य॑मा॒नेत्य॑नु - मन्य॑माना । \newline
18. प्र॒जाव॑न्तꣳ र॒यिꣳ र॒यिम् प्र॒जाव॑न्तम् प्र॒जाव॑न्तꣳ र॒यिम् । \newline
19. प्र॒जाव॑न्त॒मिति॑ प्र॒जा - व॒न्त॒म् । \newline
20. र॒यि मक्षी॑यमाण॒ मक्षी॑यमाणꣳ र॒यिꣳ र॒यि मक्षी॑यमाणम् । \newline
21. अक्षी॑यमाण॒ मित्यक्षी॑यमाणम् । \newline
22. तस्यै॑ व॒यम् ॅव॒यम् तस्यै॒ तस्यै॑ व॒यम् । \newline
23. व॒यꣳ हेड॑सि॒ हेड॑सि व॒यम् ॅव॒यꣳ हेड॑सि । \newline
24. हेड॑सि॒ मा मा हेड॑सि॒ हेड॑सि॒ मा । \newline
25. मा ऽप्यपि॒ मा मा ऽपि॑ । \newline
26. अपि॑ भूम भू॒मा प्यपि॑ भूम । \newline
27. भू॒म॒ सा सा भू॑म भूम॒ सा । \newline
28. सा नो॑ नः॒ सा सा नः॑ । \newline
29. नो॒ दे॒वी दे॒वी नो॑ नो दे॒वी । \newline
30. दे॒वी सु॒हवा॑ सु॒हवा॑ दे॒वी दे॒वी सु॒हवा᳚ । \newline
31. सु॒हवा॒ शर्म॒ शर्म॑ सु॒हवा॑ सु॒हवा॒ शर्म॑ । \newline
32. सु॒हवेति॑ सु - हवा᳚ । \newline
33. शर्म॑ यच्छतु यच्छतु॒ शर्म॒ शर्म॑ यच्छतु । \newline
34. य॒च्छ॒त्विति॑ यच्छतु । \newline
35. यस्या॑ मि॒द मि॒दम् ॅयस्या॒म् ॅयस्या॑ मि॒दम् । \newline
36. इ॒दम् प्र॒दिशि॑ प्र॒दिशी॒द मि॒दम् प्र॒दिशि॑ । \newline
37. प्र॒दिशि॒ यद् यत् प्र॒दिशि॑ प्र॒दिशि॒ यत् । \newline
38. प्र॒दिशीति॑ प्र - दिशि॑ । \newline
39. यद् वि॒रोच॑ते वि॒रोच॑ते॒ यद् यद् वि॒रोच॑ते । \newline
40. वि॒रोच॒ते ऽनु॑मति॒ मनु॑मतिम् ॅवि॒रोच॑ते वि॒रोच॒ते ऽनु॑मतिम् । \newline
41. वि॒रोच॑त॒ इति॑ वि - रोच॑ते । \newline
42. अनु॑मति॒म् प्रति॒ प्रत्यनु॑मति॒ मनु॑मति॒म् प्रति॑ । \newline
43. अनु॑मति॒मित्यनु॑ - म॒ति॒म् । \newline
44. प्रति॑ भूषन्ति भूषन्ति॒ प्रति॒ प्रति॑ भूषन्ति । \newline
45. भू॒ष॒ न्त्या॒यव॑ आ॒यवो॑ भूषन्ति भूष न्त्या॒यवः॑ । \newline
46. आ॒यव॒ इत्या॒यवः॑ । \newline
47. यस्या॑ उ॒पस्थ॑ उ॒पस्थो॒ यस्या॒ यस्या॑ उ॒पस्थः॑ । \newline
48. उ॒पस्थ॑ उ॒रू᳚(1॒)रू॑पस्थ॑ उ॒पस्थ॑ उ॒रु । \newline
49. उ॒पस्थ॒ इत्यु॒प - स्थः॒ । \newline
50. उ॒र्व॑न्तरि॑क्ष म॒न्तरि॑क्ष मु॒रू᳚(1॒)र्व॑न्तरि॑क्षम् । \newline
51. अ॒न्तरि॑क्षꣳ॒॒ सा सा ऽन्तरि॑क्ष म॒न्तरि॑क्षꣳ॒॒ सा । \newline
52. सा नो॑ नः॒ सा सा नः॑ । \newline
53. नो॒ दे॒वी दे॒वी नो॑ नो दे॒वी । \newline
54. दे॒वी सु॒हवा॑ सु॒हवा॑ दे॒वी दे॒वी सु॒हवा᳚ । \newline
55. सु॒हवा॒ शर्म॒ शर्म॑ सु॒हवा॑ सु॒हवा॒ शर्म॑ । \newline
56. सु॒हवेति॑ सु - हवा᳚ । \newline
57. शर्म॑ यच्छतु यच्छतु॒ शर्म॒ शर्म॑ यच्छतु । \newline
58. य॒च्छ॒त्विति॑ यच्छतु । \newline

\textbf{Ghana Paata } \newline

1. मन्या॑सै॒ शꣳ शम् मन्या॑सै॒ मन्या॑सै॒ शम् च॑ च॒ शम् मन्या॑सै॒ मन्या॑सै॒ शम् च॑ । \newline
2. शम् च॑ च॒ शꣳ शम् च॑ नो नश्च॒ शꣳ शम् च॑ नः । \newline
3. च॒ नो॒ न॒श्च॒ च॒ नः॒ कृ॒धि॒ कृ॒धि॒ न॒श्च॒ च॒ नः॒ कृ॒धि॒ । \newline
4. नः॒ कृ॒धि॒ कृ॒धि॒ नो॒ नः॒ कृ॒धि॒ । \newline
5. कृ॒धीति॑ कृधि । \newline
6. क्रत्वे॒ दक्षा॑य॒ दक्षा॑य॒ क्रत्वे॒ क्रत्वे॒ दक्षा॑य नो नो॒ दक्षा॑य॒ क्रत्वे॒ क्रत्वे॒ दक्षा॑य नः । \newline
7. दक्षा॑य नो नो॒ दक्षा॑य॒ दक्षा॑य नो हिनु हिनु नो॒ दक्षा॑य॒ दक्षा॑य नो हिनु । \newline
8. नो॒ हि॒नु॒ हि॒नु॒ नो॒ नो॒ हि॒नु॒ प्र प्र हि॑नु नो नो हिनु॒ प्र । \newline
9. हि॒नु॒ प्र प्र हि॑नु हिनु॒ प्र णो॑ नः॒ प्र हि॑नु हिनु॒ प्र णः॑ । \newline
10. प्र णो॑ नः॒ प्र प्र ण॒ आयूꣳ॒॒ ष्यायूꣳ॑षि नः॒ प्र प्र ण॒ आयूꣳ॑षि । \newline
11. न॒ आयूꣳ॒॒ ष्यायूꣳ॑षि नो न॒ आयूꣳ॑षि तारिष स्तारिष॒ आयूꣳ॑षि नो न॒ आयूꣳ॑षि तारिषः । \newline
12. आयूꣳ॑षि तारिष स्तारिष॒ आयूꣳ॒॒ ष्यायूꣳ॑षि तारिषः । \newline
13. ता॒रि॒ष॒ इति॑ तारिषः । \newline
14. अनु॑ मन्यताम् मन्यता॒ मन्वनु॑ मन्यता मनु॒मन्य॑माना ऽनु॒मन्य॑माना मन्यता॒ मन्वनु॑ मन्यता मनु॒मन्य॑माना । \newline
15. म॒न्य॒ता॒ म॒नु॒मन्य॑माना ऽनु॒मन्य॑माना मन्यताम् मन्यता मनु॒मन्य॑माना प्र॒जाव॑न्तम् प्र॒जाव॑न्त मनु॒मन्य॑माना मन्यताम् मन्यता मनु॒मन्य॑माना प्र॒जाव॑न्तम् । \newline
16. अ॒नु॒मन्य॑माना प्र॒जाव॑न्तम् प्र॒जाव॑न्त मनु॒मन्य॑माना ऽनु॒मन्य॑माना प्र॒जाव॑न्तꣳ र॒यिꣳ र॒यिम् प्र॒जाव॑न्त मनु॒मन्य॑माना ऽनु॒मन्य॑माना प्र॒जाव॑न्तꣳ र॒यिम् । \newline
17. अ॒नु॒मन्य॑मा॒नेत्य॑नु - मन्य॑माना । \newline
18. प्र॒जाव॑न्तꣳ र॒यिꣳ र॒यिम् प्र॒जाव॑न्तम् प्र॒जाव॑न्तꣳ र॒यि मक्षी॑यमाण॒ मक्षी॑यमाणꣳ र॒यिम् प्र॒जाव॑न्तम् प्र॒जाव॑न्तꣳ र॒यि मक्षी॑यमाणम् । \newline
19. प्र॒जाव॑न्त॒मिति॑ प्र॒जा - व॒न्त॒म् । \newline
20. र॒यि मक्षी॑यमाण॒ मक्षी॑यमाणꣳ र॒यिꣳ र॒यि मक्षी॑यमाणम् । \newline
21. अक्षी॑यमाण॒ मित्यक्षी॑यमाणम् । \newline
22. तस्यै॑ व॒यम् ॅव॒यम् तस्यै॒ तस्यै॑ व॒यꣳ हेड॑सि॒ हेड॑सि व॒यम् तस्यै॒ तस्यै॑ व॒यꣳ हेड॑सि । \newline
23. व॒यꣳ हेड॑सि॒ हेड॑सि व॒यम् ॅव॒यꣳ हेड॑सि॒ मा मा हेड॑सि व॒यम् ॅव॒यꣳ हेड॑सि॒ मा । \newline
24. हेड॑सि॒ मा मा हेड॑सि॒ हेड॑सि॒ मा ऽप्यपि॒ मा हेड॑सि॒ हेड॑सि॒ मा ऽपि॑ । \newline
25. मा ऽप्यपि॒ मा मा ऽपि॑ भूम भू॒मापि॒ मा मा ऽपि॑ भूम । \newline
26. अपि॑ भूम भू॒माप्यपि॑ भूम॒ सा सा भू॒मा प्यपि॑ भूम॒ सा । \newline
27. भू॒म॒ सा सा भू॑म भूम॒ सा नो॑ नः॒ सा भू॑म भूम॒ सा नः॑ । \newline
28. सा नो॑ नः॒ सा सा नो॑ दे॒वी दे॒वी नः॒ सा सा नो॑ दे॒वी । \newline
29. नो॒ दे॒वी दे॒वी नो॑ नो दे॒वी सु॒हवा॑ सु॒हवा॑ दे॒वी नो॑ नो दे॒वी सु॒हवा᳚ । \newline
30. दे॒वी सु॒हवा॑ सु॒हवा॑ दे॒वी दे॒वी सु॒हवा॒ शर्म॒ शर्म॑ सु॒हवा॑ दे॒वी दे॒वी सु॒हवा॒ शर्म॑ । \newline
31. सु॒हवा॒ शर्म॒ शर्म॑ सु॒हवा॑ सु॒हवा॒ शर्म॑ यच्छतु यच्छतु॒ शर्म॑ सु॒हवा॑ सु॒हवा॒ शर्म॑ यच्छतु । \newline
32. सु॒हवेति॑ सु - हवा᳚ । \newline
33. शर्म॑ यच्छतु यच्छतु॒ शर्म॒ शर्म॑ यच्छतु । \newline
34. य॒च्छ॒त्विति॑ यच्छतु । \newline
35. यस्या॑ मि॒द मि॒दम् ॅयस्या॒म् ॅयस्या॑ मि॒दम् प्र॒दिशि॑ प्र॒दिशी॒दम् ॅयस्या॒म् ॅयस्या॑ मि॒दम् प्र॒दिशि॑ । \newline
36. इ॒दम् प्र॒दिशि॑ प्र॒दिशी॒द मि॒दम् प्र॒दिशि॒ यद् यत् प्र॒दिशी॒द मि॒दम् प्र॒दिशि॒ यत् । \newline
37. प्र॒दिशि॒ यद् यत् प्र॒दिशि॑ प्र॒दिशि॒ यद् वि॒रोच॑ते वि॒रोच॑ते॒ यत् प्र॒दिशि॑ प्र॒दिशि॒ यद् वि॒रोच॑ते । \newline
38. प्र॒दिशीति॑ प्र - दिशि॑ । \newline
39. यद् वि॒रोच॑ते वि॒रोच॑ते॒ यद् यद् वि॒रोच॒ते ऽनु॑मति॒ मनु॑मतिम् ॅवि॒रोच॑ते॒ यद् यद् वि॒रोच॒ते ऽनु॑मतिम् । \newline
40. वि॒रोच॒ते ऽनु॑मति॒ मनु॑मतिम् ॅवि॒रोच॑ते वि॒रोच॒ते ऽनु॑मति॒म् प्रति॒ प्रत्यनु॑मतिम् ॅवि॒रोच॑ते वि॒रोच॒ते ऽनु॑मति॒म् प्रति॑ । \newline
41. वि॒रोच॑त॒ इति॑ वि - रोच॑ते । \newline
42. अनु॑मति॒म् प्रति॒ प्रत्यनु॑मति॒ मनु॑मति॒म् प्रति॑ भूषन्ति भूषन्ति॒ प्रत्यनु॑मति॒ मनु॑मति॒म् प्रति॑ भूषन्ति । \newline
43. अनु॑मति॒मित्यनु॑ - म॒ति॒म् । \newline
44. प्रति॑ भूषन्ति भूषन्ति॒ प्रति॒ प्रति॑ भूषन्त्या॒यव॑ आ॒यवो॑ भूषन्ति॒ प्रति॒ प्रति॑ भूषन्त्या॒यवः॑ । \newline
45. भू॒ष॒न्त्या॒यव॑ आ॒यवो॑ भूषन्ति भूषन्त्या॒यवः॑ । \newline
46. आ॒यव॒ इत्या॒यवः॑ । \newline
47. यस्या॑ उ॒पस्थ॑ उ॒पस्थो॒ यस्या॒ यस्या॑ उ॒पस्थ॑ उ॒रू᳚(1॒)रू॑पस्थो॒ यस्या॒ यस्या॑ उ॒पस्थ॑ उ॒रु । \newline
48. उ॒पस्थ॑ उ॒रू᳚(1॒)रू॑पस्थ॑ उ॒पस्थ॑ उ॒र्व॑न्तरि॑क्ष म॒न्तरि॑क्ष मु॒रू॑पस्थ॑ उ॒पस्थ॑ उ॒र्व॑न्तरि॑क्षम् । \newline
49. उ॒पस्थ॒ इत्यु॒प - स्थः॒ । \newline
50. उ॒र्व॑न्तरि॑क्ष म॒न्तरि॑क्ष मु॒रू᳚(1॒)र्व॑न्तरि॑क्षꣳ॒॒ सा सा ऽन्तरि॑क्ष मु॒रू᳚(1॒)र्व॑न्तरि॑क्षꣳ॒॒ सा । \newline
51. अ॒न्तरि॑क्षꣳ॒॒ सा सा ऽन्तरि॑क्ष म॒न्तरि॑क्षꣳ॒॒ सा नो॑ नः॒ सा ऽन्तरि॑क्ष म॒न्तरि॑क्षꣳ॒॒ सा नः॑ । \newline
52. सा नो॑ नः॒ सा सा नो॑ दे॒वी दे॒वी नः॒ सा सा नो॑ दे॒वी । \newline
53. नो॒ दे॒वी दे॒वी नो॑ नो दे॒वी सु॒हवा॑ सु॒हवा॑ दे॒वी नो॑ नो दे॒वी सु॒हवा᳚ । \newline
54. दे॒वी सु॒हवा॑ सु॒हवा॑ दे॒वी दे॒वी सु॒हवा॒ शर्म॒ शर्म॑ सु॒हवा॑ दे॒वी दे॒वी सु॒हवा॒ शर्म॑ । \newline
55. सु॒हवा॒ शर्म॒ शर्म॑ सु॒हवा॑ सु॒हवा॒ शर्म॑ यच्छतु यच्छतु॒ शर्म॑ सु॒हवा॑ सु॒हवा॒ शर्म॑ यच्छतु । \newline
56. सु॒हवेति॑ सु - हवा᳚ । \newline
57. शर्म॑ यच्छतु यच्छतु॒ शर्म॒ शर्म॑ यच्छतु । \newline
58. य॒च्छ॒त्विति॑ यच्छतु । \newline
\pagebreak
\markright{ TS 3.3.11.5  \hfill https://www.vedavms.in \hfill}

\section{ TS 3.3.11.5 }

\textbf{TS 3.3.11.5 } \newline
\textbf{Samhita Paata} \newline

रा॒काम॒हꣳ सु॒हवाꣳ॑ सुष्टु॒ती हु॑वे शृ॒णोतु॑ नः सु॒भगा॒ बोध॑तु॒ त्मना᳚ । सीव्य॒त्वपः॑ सू॒च्याऽच्छि॑द्यमानया॒ ददा॑तु वी॒रꣳ श॒तदा॑यमु॒क्थ्यं᳚ ॥ यास्ते॑ राके सुम॒तयः॑ सु॒पेश॑सो॒ याभि॒र्ददा॑सि दा॒शुषे॒ वसू॑नि । ताभि॑र्नो अ॒द्य सु॒मना॑ उ॒पाग॑हि सहस्रपो॒षꣳ सु॑भगे॒ ररा॑णा ॥ सिनी॑वालि॒ >1, या सु॑पा॒णिः >2 ॥ कु॒हूम॒हꣳ सु॒भगां᳚ ॅविद्म॒नाप॑सम॒स्मिन्. य॒ज्ञे सु॒हवां᳚ जोहवीमि ।सा नो॑ ददातु॒ श्रव॑णं ( ) पितृ॒णां तस्या᳚स्ते देवि ह॒विषा॑ विधेम ॥कु॒हू-र्दे॒वाना॑म॒मृत॑स्य॒ पत्नी॒ हव्या॑ नो अ॒स्य ह॒विष॑श्चिकेतु । सं दा॒शुषे॑ कि॒रतु॒ भूरि॑ वा॒मꣳ रा॒यस्पोषं॑ चिकि॒तुषे॑ दधातु ॥ \newline

\textbf{Pada Paata} \newline

रा॒काम् । अ॒हम् । सु॒हवा॒मिति॑ सु - हवा᳚म् । सु॒ष्टु॒तीति॑ सु-स्तु॒ती । हु॒वे॒ । शृ॒णोतु॑ । नः॒ । सु॒भगेति॑ सु - भगा᳚ । बोध॑तु । त्मना᳚ ॥ सीव्य॑तु । अपः॑ । सू॒च्या । अच्छि॑द्यमानया । ददा॑तु । वी॒रम् । श॒तदा॑य॒मिति॑ श॒त - दा॒य॒म् । उ॒क्थ्य᳚म् ॥ याः । ते॒ । रा॒के॒ । सु॒म॒तय॒ इति॑ सु - म॒तयः॑ । सु॒पेश॑स॒ इति॑ सु - पेश॑सः । याभिः॑ । ददा॑सि । दा॒शुषे᳚ । वसू॑नि ॥ ताभिः॑ । नः॒ । अ॒द्य । सु॒मना॒ इति॑ सु - मनाः᳚ । उ॒पाग॒हीत्यु॑प - आग॑हि । स॒ह॒स्र॒पो॒षमिति॑ सहस्र - पो॒षम् । सु॒भ॒ग॒ इति॑ सु - भ॒गे॒ । ररा॑णा ॥ सिनी॑वालि । या । सु॒पा॒णिरिति॑ सु - पा॒णिः ॥ कु॒हूम् । अ॒हम् । सु॒भगा॒मिति॑ सु - भगा᳚म् । वि॒द्म॒नाप॑स॒मिति॑ विद्म॒न - अ॒प॒स॒म् । अ॒स्मिन्न् । य॒ज्ञे । सु॒हवा॒मिति॑ सु - हवा᳚म् । जो॒ह॒वी॒मि॒ ॥ सा । नः॒ । द॒दा॒तु॒ । श्रव॑णम् ( ) । पि॒तृ॒णाम् । तस्याः᳚ । ते॒ । दे॒वि॒ । ह॒विषा᳚ । वि॒धे॒म॒ ॥ कु॒हूः । दे॒वाना᳚म् । अ॒मृत॑स्य । पत्नी᳚ । हव्या᳚ । नः॒ । अ॒स्य । ह॒विषः॑ । चि॒के॒तु॒ ॥ समिति॑ । दा॒शुषे᳚ । कि॒रतु॑ । भूरि॑ । वा॒मम् । रा॒यः । पोष᳚म् । चि॒कि॒तुषे᳚ । द॒धा॒तु॒ ॥  \newline


\textbf{Krama Paata} \newline

रा॒काम॒हम् । अ॒हꣳ सु॒हवा᳚म् । सु॒हवाꣳ॑ सुष्टु॒ती । सु॒हवा॒मिति॑ सु - हवा᳚म् । सु॒ष्टु॒ती हु॑वे । सु॒ष्टु॒तीति॑ सु - स्तु॒ती । हु॒वे॒ शृ॒णोतु॑ । शृ॒णोतु॑ नः । नः॒ सु॒भगा᳚ । सु॒भगा॒ बोध॑तु । सु॒भगेति॑ सु - भगा᳚ । बोध॑तु॒ त्मना᳚ । त्मनेति॒ त्मना᳚ ॥ सीव्य॒त्वपः॑ । अपः॑ सू॒च्या । सू॒च्या ऽच्छि॑द्यमानया । अच्छि॑द्यमानया॒ ददा॑तु । ददा॑तु वी॒रम् । वी॒रꣳ श॒तदा॑यम् । श॒तदा॑यमु॒क्थ्य᳚म् । श॒तदा॑य॒मिति॑ श॒त - दा॒य॒म् । उ॒क्थ्य॑मित्यु॒क्थ्य᳚म् ॥ यास्ते᳚ । ते॒ रा॒के॒ । रा॒के॒ सु॒म॒तयः॑ । सु॒म॒तयः॑ सु॒पेश॑सः । सु॒म॒तय॒ इति॑ सु - म॒तयः॑ । सु॒पेश॑सो॒ याभिः॑ । सु॒पेश॑स॒ इति॑ सु - पेश॑सः । याभि॒र् ददा॑सि । ददा॑सि दा॒शुषे᳚ । दा॒शुषे॒ वसू॑नि । वसू॒नीति॒ वसू॑नि ॥ ताभि॑र् नः । नो॒ अ॒द्य । अ॒द्य सु॒मनाः᳚ । सु॒मना॑ उ॒पाग॑हि । सु॒मना॒ इति॑ सु - मनाः᳚ । उ॒पाग॑हि सहस्रपो॒षम् । उ॒पाग॒हीत्यु॑प - आग॑हि । स॒ह॒स्र॒पो॒षꣳ सु॑भगे । स॒ह॒स्र॒पो॒षमिति॑ सहस्र - पो॒षम् । सु॒भ॒गे॒ ररा॑णा । सु॒भ॒ग॒ इति॑ सु - भ॒गे॒ । ररा॒णेति॒ ररा॑णा ॥ सिनी॑वालि॒ या । या सु॑पा॒णिः । सु॒पा॒णिरिति॑ सु - पा॒णिः ॥ कु॒हूम॒हम् । अ॒हꣳ सु॒भगा᳚म् । सु॒भगा᳚म् ॅविद्म॒नाप॑सम् । सु॒भगा॒मिति॑ सु - भगा᳚म् । वि॒द्म॒नाप॑सम॒स्मिन्न् । वि॒द्म॒नाप॑स॒मिति॑ विद्म॒न - अ॒प॒स॒म् । अ॒स्मिन्. य॒ज्ञे । य॒ज्ञे सु॒हवा᳚म् । सु॒हवा᳚म् जोहवीमि । सु॒हवा॒मिति॑ सु - हवा᳚म् । जो॒ह॒वी॒मीति॑ जोहवीमि ॥ सा नः॑ । नो॒ द॒दा॒तु॒ । द॒दा॒तु॒ श्रव॑णम् ( ) । श्रव॑णम् पितृ॒णाम् । पि॒तृ॒णाम् तस्याः᳚ । तस्या᳚स्ते । ते॒ दे॒वि॒ । दे॒वि॒ ह॒विषा᳚ । ह॒विषा॑ विधेम । वि॒धे॒मेति॑ विधेम ॥ 
कु॒हूर् दे॒वाना᳚म् । दे॒वाना॑म॒मृत॑स्य । अ॒मृत॑स्य॒ पत्नी᳚ । पत्नी॒ हव्या᳚ । हव्या॑ नः । नो॒ अ॒स्य । अ॒स्य ह॒विषः॑ । ह॒विष॑श्चिकेतु । चि॒के॒त्विति॑ चिकेतु ॥ सम् दा॒शुषे᳚ । दा॒शुषे॑ कि॒रतु॑ । कि॒रतु॒ भूरि॑ । भूरि॑ वा॒मम् । वा॒मꣳ रा॒यः । रा॒यस्पोष᳚म् । पोष॑म् चिकि॒तुषे᳚ । चि॒कि॒तुषे॑ दधातु । द॒धा॒त्विति॑ दधातु । \newline

\textbf{Jatai Paata} \newline

1. रा॒का म॒ह म॒हꣳ रा॒काꣳ रा॒का म॒हम् । \newline
2. अ॒हꣳ सु॒हवाꣳ॑ सु॒हवा॑ म॒ह म॒हꣳ सु॒हवा᳚म् । \newline
3. सु॒हवाꣳ॑ सुष्टु॒ती सु॑ष्टु॒ती सु॒हवाꣳ॑ सु॒हवाꣳ॑ सुष्टु॒ती । \newline
4. सु॒हवा॒मिति॑ सु - हवा᳚म् । \newline
5. सु॒ष्टु॒ती हु॑वे हुवे सुष्टु॒ती सु॑ष्टु॒ती हु॑वे । \newline
6. सु॒ष्टु॒तीति॑ सु - स्तु॒ती । \newline
7. हु॒वे॒ शृ॒णोतु॑ शृ॒णोतु॑ हुवे हुवे शृ॒णोतु॑ । \newline
8. शृ॒णोतु॑ नो नः शृ॒णोतु॑ शृ॒णोतु॑ नः । \newline
9. नः॒ सु॒भगा॑ सु॒भगा॑ नो नः सु॒भगा᳚ । \newline
10. सु॒भगा॒ बोध॑तु॒ बोध॑तु सु॒भगा॑ सु॒भगा॒ बोध॑तु । \newline
11. सु॒भगेति॑ सु - भगा᳚ । \newline
12. बोध॑तु॒ त्मना॒ त्मना॒ बोध॑तु॒ बोध॑तु॒ त्मना᳚ । \newline
13. त्मनेति॒ त्मना᳚ । \newline
14. सीव्य॒ त्वपो॒ अपः॒ सीव्य॑तु॒ सीव्य॒ त्वपः॑ । \newline
15. अपः॑ सू॒च्या सू॒च्या ऽपो॒ अपः॑ सू॒च्या । \newline
16. सू॒च्या ऽच्छि॑द्यमान॒या ऽच्छि॑द्यमानया सू॒च्या सू॒च्या ऽच्छि॑द्यमानया । \newline
17. अच्छि॑द्यमानया॒ ददा॑तु॒ ददा॒ त्वच्छि॑द्यमान॒या ऽच्छि॑द्यमानया॒ ददा॑तु । \newline
18. ददा॑तु वी॒रम् ॅवी॒रम् ददा॑तु॒ ददा॑तु वी॒रम् । \newline
19. वी॒रꣳ श॒तदा॑यꣳ श॒तदा॑यम् ॅवी॒रम् ॅवी॒रꣳ श॒तदा॑यम् । \newline
20. श॒तदा॑य मु॒क्थ्य॑ मु॒क्थ्यꣳ॑ श॒तदा॑यꣳ श॒तदा॑य मु॒क्थ्य᳚म् । \newline
21. श॒तदा॑य॒मिति॑ श॒त - दा॒य॒म् । \newline
22. उ॒क्थ्य॑मित्यु॒क्थ्य᳚म् । \newline
23. या स्ते॑ ते॒ या या स्ते᳚ । \newline
24. ते॒ रा॒के॒ रा॒के॒ ते॒ ते॒ रा॒के॒ । \newline
25. रा॒के॒ सु॒म॒तयः॑ सुम॒तयो॑ राके राके सुम॒तयः॑ । \newline
26. सु॒म॒तयः॑ सु॒पेश॑सः सु॒पेश॑सः सुम॒तयः॑ सुम॒तयः॑ सु॒पेश॑सः । \newline
27. सु॒म॒तय॒ इति॑ सु - म॒तयः॑ । \newline
28. सु॒पेश॑सो॒ याभि॒र् याभिः॑ सु॒पेश॑सः सु॒पेश॑सो॒ याभिः॑ । \newline
29. सु॒पेश॑स॒ इति॑ सु - पेश॑सः । \newline
30. याभि॒र् ददा॑सि॒ ददा॑सि॒ याभि॒र् याभि॒र् ददा॑सि । \newline
31. ददा॑सि दा॒शुषे॑ दा॒शुषे॒ ददा॑सि॒ ददा॑सि दा॒शुषे᳚ । \newline
32. दा॒शुषे॒ वसू॑नि॒ वसू॑नि दा॒शुषे॑ दा॒शुषे॒ वसू॑नि । \newline
33. वसू॒नीति॒ वसू॑नि । \newline
34. ताभि॑र् नो न॒ स्ताभि॒ स्ताभि॑र् नः । \newline
35. नो॒ अ॒द्याद्य नो॑ नो अ॒द्य । \newline
36. अ॒द्य सु॒मनाः᳚ सु॒मना॑ अ॒द्याद्य सु॒मनाः᳚ । \newline
37. सु॒मना॑ उ॒पाग॑ ह्यु॒पाग॑हि सु॒मनाः᳚ सु॒मना॑ उ॒पाग॑हि । \newline
38. सु॒मना॒ इति॑ सु - मनाः᳚ । \newline
39. उ॒पाग॑हि सहस्रपो॒षꣳ स॑हस्रपो॒ष मु॒पाग॑ ह्यु॒पाग॑हि सहस्रपो॒षम् । \newline
40. उ॒पाग॒हीत्यु॑प - आग॑हि । \newline
41. स॒ह॒स्र॒पो॒षꣳ सु॑भगे सुभगे सहस्रपो॒षꣳ स॑हस्रपो॒षꣳ सु॑भगे । \newline
42. स॒ह॒स्र॒पो॒षमिति॑ सहस्र - पो॒षम् । \newline
43. सु॒भ॒गे॒ ररा॑णा॒ ररा॑णा सुभगे सुभगे॒ ररा॑णा । \newline
44. सु॒भ॒ग॒ इति॑ सु - भ॒गे॒ । \newline
45. ररा॒णेति॒ ररा॑णा । \newline
46. सिनी॑वालि॒ या या सिनी॑वालि॒ सिनी॑वालि॒ या । \newline
47. या सु॑पा॒णिः सु॑पा॒णिर् या या सु॑पा॒णिः । \newline
48. सु॒पा॒णिरिति॑ सु - पा॒णिः । \newline
49. कु॒हू म॒ह म॒हम् कु॒हूम् कु॒हू म॒हम् । \newline
50. अ॒हꣳ सु॒भगाꣳ॑ सु॒भगा॑ म॒ह म॒हꣳ सु॒भगा᳚म् । \newline
51. सु॒भगा᳚म् ॅविद्म॒नाप॑सम् ॅविद्म॒नाप॑सꣳ सु॒भगाꣳ॑ सु॒भगा᳚म् ॅविद्म॒नाप॑सम् । \newline
52. सु॒भगा॒मिति॑ सु - भगा᳚म् । \newline
53. वि॒द्म॒नाप॑स म॒स्मिन् न॒स्मिन्. वि॑द्म॒नाप॑सम् ॅविद्म॒नाप॑स म॒स्मिन्न् । \newline
54. वि॒द्म॒नाप॑स॒मिति॑ विद्म॒न - अ॒प॒स॒म् । \newline
55. अ॒स्मिन्. य॒ज्ञे य॒ज्ञे अ॒स्मिन् न॒स्मिन्. य॒ज्ञे । \newline
56. य॒ज्ञे सु॒हवाꣳ॑ सु॒हवा᳚म् ॅय॒ज्ञे य॒ज्ञे सु॒हवा᳚म् । \newline
57. सु॒हवा᳚म् जोहवीमि जोहवीमि सु॒हवाꣳ॑ सु॒हवा᳚म् जोहवीमि । \newline
58. सु॒हवा॒मिति॑ सु - हवा᳚म् । \newline
59. जो॒ह॒वी॒मीति॑ जोहवीमि । \newline
60. सा नो॑ नः॒ सा सा नः॑ । \newline
61. नो॒ द॒दा॒तु॒ द॒दा॒तु॒ नो॒ नो॒ द॒दा॒तु॒ । \newline
62. द॒दा॒तु॒ श्रव॑णꣳ॒॒ श्रव॑णम् ददातु ददातु॒ श्रव॑णम् । \newline
63. श्रव॑णम् पितृ॒णाम् पि॑तृ॒णाꣳ श्रव॑णꣳ॒॒ श्रव॑णम् पितृ॒णाम् । \newline
64. पि॒तृ॒णाम् तस्या॒ स्तस्याः᳚ पितृ॒णाम् पि॑तृ॒णाम् तस्याः᳚ । \newline
65. तस्या᳚ स्ते ते॒ तस्या॒ स्तस्या᳚ स्ते । \newline
66. ते॒ दे॒वि॒ दे॒वि॒ ते॒ ते॒ दे॒वि॒ । \newline
67. दे॒वि॒ ह॒विषा॑ ह॒विषा॑ देवि देवि ह॒विषा᳚ । \newline
68. ह॒विषा॑ विधेम विधेम ह॒विषा॑ ह॒विषा॑ विधेम । \newline
69. वि॒धे॒मेति॑ विधेम । \newline
70. कु॒हूर् दे॒वाना᳚म् दे॒वाना᳚म् कु॒हूः कु॒हूर् दे॒वाना᳚म् । \newline
71. दे॒वाना॑ म॒मृत॑स्या॒ मृत॑स्य दे॒वाना᳚म् दे॒वाना॑ म॒मृत॑स्य । \newline
72. अ॒मृत॑स्य॒ पत्नी॒ पत्न्य॒मृत॑स्या॒ मृत॑स्य॒ पत्नी᳚ । \newline
73. पत्नी॒ हव्या॒ हव्या॒ पत्नी॒ पत्नी॒ हव्या᳚ । \newline
74. हव्या॑ नो नो॒ हव्या॒ हव्या॑ नः । \newline
75. नो॒ अ॒स्यास्य नो॑ नो अ॒स्य । \newline
76. अ॒स्य ह॒विषो॑ ह॒विषो॑ अ॒स्यास्य ह॒विषः॑ । \newline
77. ह॒विष॑ श्चिकेतु चिकेतु ह॒विषो॑ ह॒विष॑ श्चिकेतु । \newline
78. चि॒के॒त्विति॑ चिकेतु । \newline
79. सम् दा॒शुषे॑ दा॒शुषे॒ सꣳ सम् दा॒शुषे᳚ । \newline
80. दा॒शुषे॑ कि॒रतु॑ कि॒रतु॑ दा॒शुषे॑ दा॒शुषे॑ कि॒रतु॑ । \newline
81. कि॒रतु॒ भूरि॒ भूरि॑ कि॒रतु॑ कि॒रतु॒ भूरि॑ । \newline
82. भूरि॑ वा॒मम् ॅवा॒मम् भूरि॒ भूरि॑ वा॒मम् । \newline
83. वा॒मꣳ रा॒यो रा॒यो वा॒मम् ॅवा॒मꣳ रा॒यः । \newline
84. रा॒य स्पोष॒म् पोषꣳ॑ रा॒यो रा॒य स्पोष᳚म् । \newline
85. पोष॑म् चिकि॒तुषे॑ चिकि॒तुषे॒ पोष॒म् पोष॑म् चिकि॒तुषे᳚ । \newline
86. चि॒कि॒तुषे॑ दधातु दधातु चिकि॒तुषे॑ चिकि॒तुषे॑ दधातु । \newline
87. द॒धा॒त्विति॑ दधातु । \newline

\textbf{Ghana Paata } \newline

1. रा॒का म॒ह म॒हꣳ रा॒काꣳ रा॒का म॒हꣳ सु॒हवाꣳ॑ सु॒हवा॑ म॒हꣳ रा॒काꣳ रा॒का म॒हꣳ सु॒हवा᳚म् । \newline
2. अ॒हꣳ सु॒हवाꣳ॑ सु॒हवा॑ म॒ह म॒हꣳ सु॒हवाꣳ॑ सुष्टु॒ती सु॑ष्टु॒ती सु॒हवा॑ म॒ह म॒हꣳ सु॒हवाꣳ॑ सुष्टु॒ती । \newline
3. सु॒हवाꣳ॑ सुष्टु॒ती सु॑ष्टु॒ती सु॒हवाꣳ॑ सु॒हवाꣳ॑ सुष्टु॒ती हु॑वे हुवे सुष्टु॒ती सु॒हवाꣳ॑ सु॒हवाꣳ॑ सुष्टु॒ती हु॑वे । \newline
4. सु॒हवा॒मिति॑ सु - हवा᳚म् । \newline
5. सु॒ष्टु॒ती हु॑वे हुवे सुष्टु॒ती सु॑ष्टु॒ती हु॑वे शृ॒णोतु॑ शृ॒णोतु॑ हुवे सुष्टु॒ती सु॑ष्टु॒ती हु॑वे शृ॒णोतु॑ । \newline
6. सु॒ष्टु॒तीति॑ सु - स्तु॒ती । \newline
7. हु॒वे॒ शृ॒णोतु॑ शृ॒णोतु॑ हुवे हुवे शृ॒णोतु॑ नो नः शृ॒णोतु॑ हुवे हुवे शृ॒णोतु॑ नः । \newline
8. शृ॒णोतु॑ नो नः शृ॒णोतु॑ शृ॒णोतु॑ नः सु॒भगा॑ सु॒भगा॑ नः शृ॒णोतु॑ शृ॒णोतु॑ नः सु॒भगा᳚ । \newline
9. नः॒ सु॒भगा॑ सु॒भगा॑ नो नः सु॒भगा॒ बोध॑तु॒ बोध॑तु सु॒भगा॑ नो नः सु॒भगा॒ बोध॑तु । \newline
10. सु॒भगा॒ बोध॑तु॒ बोध॑तु सु॒भगा॑ सु॒भगा॒ बोध॑तु॒ त्मना॒ त्मना॒ बोध॑तु सु॒भगा॑ सु॒भगा॒ बोध॑तु॒ त्मना᳚ । \newline
11. सु॒भगेति॑ सु - भगा᳚ । \newline
12. बोध॑तु॒ त्मना॒ त्मना॒ बोध॑तु॒ बोध॑तु॒ त्मना᳚ । \newline
13. त्मनेति॒ त्मना᳚ । \newline
14. सीव्य॒ त्वपो॒ अपः॒ सीव्य॑तु॒ सीव्य॒ त्वपः॑ सू॒च्या सू॒च्या ऽपः॒ सीव्य॑तु॒ सीव्य॒ त्वपः॑ सू॒च्या । \newline
15. अपः॑ सू॒च्या सू॒च्या ऽपो॒ अपः॑ सू॒च्या ऽच्छि॑द्यमान॒या ऽच्छि॑द्यमानया सू॒च्या ऽपो॒ अपः॑ सू॒च्या ऽच्छि॑द्यमानया । \newline
16. सू॒च्या ऽच्छि॑द्यमान॒या ऽच्छि॑द्यमानया सू॒च्या सू॒च्या ऽच्छि॑द्यमानया॒ ददा॑तु॒ ददा॒ त्वच्छि॑द्यमानया सू॒च्या सू॒च्या ऽच्छि॑द्यमानया॒ ददा॑तु । \newline
17. अच्छि॑द्यमानया॒ ददा॑तु॒ ददा॒ त्वच्छि॑द्यमान॒या ऽच्छि॑द्यमानया॒ ददा॑तु वी॒रम् ॅवी॒रम् ददा॒ त्वच्छि॑द्यमान॒या ऽच्छि॑द्यमानया॒ ददा॑तु वी॒रम् । \newline
18. ददा॑तु वी॒रम् ॅवी॒रम् ददा॑तु॒ ददा॑तु वी॒रꣳ श॒तदा॑यꣳ श॒तदा॑यम् ॅवी॒रम् ददा॑तु॒ ददा॑तु वी॒रꣳ श॒तदा॑यम् । \newline
19. वी॒रꣳ श॒तदा॑यꣳ श॒तदा॑यम् ॅवी॒रम् ॅवी॒रꣳ श॒तदा॑य मु॒क्थ्य॑ मु॒क्थ्यꣳ॑ श॒तदा॑यम् ॅवी॒रम् ॅवी॒रꣳ श॒तदा॑य मु॒क्थ्य᳚म् । \newline
20. श॒तदा॑य मु॒क्थ्य॑ मु॒क्थ्यꣳ॑ श॒तदा॑यꣳ श॒तदा॑य मु॒क्थ्य᳚म् । \newline
21. श॒तदा॑य॒मिति॑ श॒त - दा॒य॒म् । \newline
22. उ॒क्थ्य॑मित्यु॒क्थ्य᳚म् । \newline
23. या स्ते॑ ते॒ या या स्ते॑ राके राके ते॒ या या स्ते॑ राके । \newline
24. ते॒ रा॒के॒ रा॒के॒ ते॒ ते॒ रा॒के॒ सु॒म॒तयः॑ सुम॒तयो॑ राके ते ते राके सुम॒तयः॑ । \newline
25. रा॒के॒ सु॒म॒तयः॑ सुम॒तयो॑ राके राके सुम॒तयः॑ सु॒पेश॑सः सु॒पेश॑सः सुम॒तयो॑ राके राके सुम॒तयः॑ सु॒पेश॑सः । \newline
26. सु॒म॒तयः॑ सु॒पेश॑सः सु॒पेश॑सः सुम॒तयः॑ सुम॒तयः॑ सु॒पेश॑सो॒ याभि॒र् याभिः॑ सु॒पेश॑सः सुम॒तयः॑ सुम॒तयः॑ सु॒पेश॑सो॒ याभिः॑ । \newline
27. सु॒म॒तय॒ इति॑ सु - म॒तयः॑ । \newline
28. सु॒पेश॑सो॒ याभि॒र् याभिः॑ सु॒पेश॑सः सु॒पेश॑सो॒ याभि॒र् ददा॑सि॒ ददा॑सि॒ याभिः॑ सु॒पेश॑सः सु॒पेश॑सो॒ याभि॒र् ददा॑सि । \newline
29. सु॒पेश॑स॒ इति॑ सु - पेश॑सः । \newline
30. याभि॒र् ददा॑सि॒ ददा॑सि॒ याभि॒र् याभि॒र् ददा॑सि दा॒शुषे॑ दा॒शुषे॒ ददा॑सि॒ याभि॒र् याभि॒र् ददा॑सि दा॒शुषे᳚ । \newline
31. ददा॑सि दा॒शुषे॑ दा॒शुषे॒ ददा॑सि॒ ददा॑सि दा॒शुषे॒ वसू॑नि॒ वसू॑नि दा॒शुषे॒ ददा॑सि॒ ददा॑सि दा॒शुषे॒ वसू॑नि । \newline
32. दा॒शुषे॒ वसू॑नि॒ वसू॑नि दा॒शुषे॑ दा॒शुषे॒ वसू॑नि । \newline
33. वसू॒नीति॒ वसू॑नि । \newline
34. ताभि॑र् नो न॒ स्ताभि॒ स्ताभि॑र् नो अ॒द्याद्य न॒ स्ताभि॒ स्ताभि॑र् नो अ॒द्य । \newline
35. नो॒ अ॒द्याद्य नो॑ नो अ॒द्य सु॒मनाः᳚ सु॒मना॑ अ॒द्य नो॑ नो अ॒द्य सु॒मनाः᳚ । \newline
36. अ॒द्य सु॒मनाः᳚ सु॒मना॑ अ॒द्याद्य सु॒मना॑ उ॒पाग॑ ह्यु॒पाग॑हि सु॒मना॑ अ॒द्याद्य सु॒मना॑ उ॒पाग॑हि । \newline
37. सु॒मना॑ उ॒पाग॑ ह्यु॒पाग॑हि सु॒मनाः᳚ सु॒मना॑ उ॒पाग॑हि सहस्रपो॒षꣳ स॑हस्रपो॒ष मु॒पाग॑हि सु॒मनाः᳚ सु॒मना॑ उ॒पाग॑हि सहस्रपो॒षम् । \newline
38. सु॒मना॒ इति॑ सु - मनाः᳚ । \newline
39. उ॒पाग॑हि सहस्रपो॒षꣳ स॑हस्रपो॒ष मु॒पाग॑ ह्यु॒पाग॑हि सहस्रपो॒षꣳ सु॑भगे सुभगे सहस्रपो॒ष मु॒पाग॑ ह्यु॒पाग॑हि सहस्रपो॒षꣳ सु॑भगे । \newline
40. उ॒पाग॒हीत्यु॑प - आग॑हि । \newline
41. स॒ह॒स्र॒पो॒षꣳ सु॑भगे सुभगे सहस्रपो॒षꣳ स॑हस्रपो॒षꣳ सु॑भगे॒ ररा॑णा॒ ररा॑णा सुभगे सहस्रपो॒षꣳ स॑हस्रपो॒षꣳ सु॑भगे॒ ररा॑णा । \newline
42. स॒ह॒स्र॒पो॒षमिति॑ सहस्र - पो॒षम् । \newline
43. सु॒भ॒गे॒ ररा॑णा॒ ररा॑णा सुभगे सुभगे॒ ररा॑णा । \newline
44. सु॒भ॒ग॒ इति॑ सु - भ॒गे॒ । \newline
45. ररा॒णेति॒ ररा॑णा । \newline
46. सिनी॑वालि॒ या या सिनी॑वालि॒ सिनी॑वालि॒ या सु॑पा॒णिः सु॑पा॒णिर् या सिनी॑वालि॒ सिनी॑वालि॒ या सु॑पा॒णिः । \newline
47. या सु॑पा॒णिः सु॑पा॒णिर् या या सु॑पाणिः । \newline
48. सु॒पा॒णिरिति॑ सु - पा॒णिः । \newline
49. कु॒हू म॒ह म॒हम् कु॒हूम् कु॒हू म॒हꣳ सु॒भगाꣳ॑ सु॒भगा॑ म॒हम् कु॒हूम् कु॒हू म॒हꣳ सु॒भगा᳚म् । \newline
50. अ॒हꣳ सु॒भगाꣳ॑ सु॒भगा॑ म॒ह म॒हꣳ सु॒भगा᳚म् ॅविद्म॒नाप॑सम् ॅविद्म॒नाप॑सꣳ सु॒भगा॑ म॒ह म॒हꣳ सु॒भगा᳚म् ॅविद्म॒नाप॑सम् । \newline
51. सु॒भगा᳚म् ॅविद्म॒नाप॑सम् ॅविद्म॒नाप॑सꣳ सु॒भगाꣳ॑ सु॒भगा᳚म् ॅविद्म॒नाप॑स म॒स्मिन्, न॒स्मिन्. वि॑द्म॒नाप॑सꣳ सु॒भगाꣳ॑ सु॒भगा᳚म् ॅविद्म॒नाप॑स म॒स्मिन्न् । \newline
52. सु॒भगा॒मिति॑ सु - भगा᳚म् । \newline
53. वि॒द्म॒नाप॑स म॒स्मिन्, न॒स्मिन्. वि॑द्म॒नाप॑सम् ॅविद्म॒नाप॑स म॒स्मिन्. य॒ज्ञे य॒ज्ञे अ॒स्मिन्. वि॑द्म॒नाप॑सम् ॅविद्म॒नाप॑स म॒स्मिन्. य॒ज्ञे । \newline
54. वि॒द्म॒नाप॑स॒मिति॑ विद्म॒न - अ॒प॒स॒म् । \newline
55. अ॒स्मिन्. य॒ज्ञे य॒ज्ञे अ॒स्मिन्, न॒स्मिन्. य॒ज्ञे सु॒हवाꣳ॑ सु॒हवा᳚म् ॅय॒ज्ञे अ॒स्मिन्, न॒स्मिन्. य॒ज्ञे सु॒हवा᳚म् । \newline
56. य॒ज्ञे सु॒हवाꣳ॑ सु॒हवा᳚म् ॅय॒ज्ञे य॒ज्ञे सु॒हवा᳚म् जोहवीमि जोहवीमि सु॒हवा᳚म् ॅय॒ज्ञे य॒ज्ञे सु॒हवा᳚म् जोहवीमि । \newline
57. सु॒हवा᳚म् जोहवीमि जोहवीमि सु॒हवाꣳ॑ सु॒हवा᳚म् जोहवीमि । \newline
58. सु॒हवा॒मिति॑ सु - हवा᳚म् । \newline
59. जो॒ह॒वी॒मीति॑ जोहवीमि । \newline
60. सा नो॑ नः॒ सा सा नो॑ ददातु ददातु नः॒ सा सा नो॑ ददातु । \newline
61. नो॒ द॒दा॒तु॒ द॒दा॒तु॒ नो॒ नो॒ द॒दा॒तु॒ श्रव॑णꣳ॒॒ श्रव॑णम् ददातु नो नो ददातु॒ श्रव॑णम् । \newline
62. द॒दा॒तु॒ श्रव॑णꣳ॒॒ श्रव॑णम् ददातु ददातु॒ श्रव॑णम् पितृ॒णाम् पि॑तृ॒णाꣳ श्रव॑णम् ददातु ददातु॒ श्रव॑णम् पितृ॒णाम् । \newline
63. श्रव॑णम् पितृ॒णाम् पि॑तृ॒णाꣳ श्रव॑णꣳ॒॒ श्रव॑णम् पितृ॒णाम् तस्या॒ स्तस्याः᳚ पितृ॒णाꣳ श्रव॑णꣳ॒॒ श्रव॑णम् पितृ॒णाम् तस्याः᳚ । \newline
64. पि॒तृ॒णाम् तस्या॒ स्तस्याः᳚ पितृ॒णाम् पि॑तृ॒णाम् तस्या᳚ स्ते ते॒ तस्याः᳚ पितृ॒णाम् पि॑तृ॒णाम् तस्या᳚ स्ते । \newline
65. तस्या᳚ स्ते ते॒ तस्या॒ स्तस्या᳚ स्ते देवि देवि ते॒ तस्या॒ स्तस्या᳚ स्ते देवि । \newline
66. ते॒ दे॒वि॒ दे॒वि॒ ते॒ ते॒ दे॒वि॒ ह॒विषा॑ ह॒विषा॑ देवि ते ते देवि ह॒विषा᳚ । \newline
67. दे॒वि॒ ह॒विषा॑ ह॒विषा॑ देवि देवि ह॒विषा॑ विधेम विधेम ह॒विषा॑ देवि देवि ह॒विषा॑ विधेम । \newline
68. ह॒विषा॑ विधेम विधेम ह॒विषा॑ ह॒विषा॑ विधेम । \newline
69. वि॒धे॒मेति॑ विधेम । \newline
70. कु॒हूर् दे॒वाना᳚म् दे॒वाना᳚म् कु॒हूः कु॒हूर् दे॒वाना॑ म॒मृत॑स्या॒ मृत॑स्य दे॒वाना᳚म् कु॒हूः कु॒हूर् दे॒वाना॑ म॒मृत॑स्य । \newline
71. दे॒वाना॑ म॒मृत॑स्या॒ मृत॑स्य दे॒वाना᳚म् दे॒वाना॑ म॒मृत॑स्य॒ पत्नी॒ पत्न्य॒मृत॑स्य दे॒वाना᳚म् दे॒वाना॑ म॒मृत॑स्य॒ पत्नी᳚ । \newline
72. अ॒मृत॑स्य॒ पत्नी॒ पत्न्य॒मृत॑स्या॒ मृत॑स्य॒ पत्नी॒ हव्या॒ हव्या॒ पत्न्य॒मृत॑स्या॒ मृत॑स्य॒ पत्नी॒ हव्या᳚ । \newline
73. पत्नी॒ हव्या॒ हव्या॒ पत्नी॒ पत्नी॒ हव्या॑ नो नो॒ हव्या॒ पत्नी॒ पत्नी॒ हव्या॑ नः । \newline
74. हव्या॑ नो नो॒ हव्या॒ हव्या॑ नो अ॒स्यास्य नो॒ हव्या॒ हव्या॑ नो अ॒स्य । \newline
75. नो॒ अ॒स्यास्य नो॑ नो अ॒स्य ह॒विषो॑ ह॒विषो॑ अ॒स्य नो॑ नो अ॒स्य ह॒विषः॑ । \newline
76. अ॒स्य ह॒विषो॑ ह॒विषो॑ अ॒स्यास्य ह॒विष॑ श्चिकेतु चिकेतु ह॒विषो॑ अ॒स्यास्य ह॒विष॑ श्चिकेतु । \newline
77. ह॒विष॑ श्चिकेतु चिकेतु ह॒विषो॑ ह॒विष॑ श्चिकेतु । \newline
78. चि॒के॒त्विति॑ चिकेतु । \newline
79. सम् दा॒शुषे॑ दा॒शुषे॒ सꣳ सम् दा॒शुषे॑ कि॒रतु॑ कि॒रतु॑ दा॒शुषे॒ सꣳ सम् दा॒शुषे॑ कि॒रतु॑ । \newline
80. दा॒शुषे॑ कि॒रतु॑ कि॒रतु॑ दा॒शुषे॑ दा॒शुषे॑ कि॒रतु॒ भूरि॒ भूरि॑ कि॒रतु॑ दा॒शुषे॑ दा॒शुषे॑ कि॒रतु॒ भूरि॑ । \newline
81. कि॒रतु॒ भूरि॒ भूरि॑ कि॒रतु॑ कि॒रतु॒ भूरि॑ वा॒मम् ॅवा॒मम् भूरि॑ कि॒रतु॑ कि॒रतु॒ भूरि॑ वा॒मम् । \newline
82. भूरि॑ वा॒मम् ॅवा॒मम् भूरि॒ भूरि॑ वा॒मꣳ रा॒यो रा॒यो वा॒मम् भूरि॒ भूरि॑ वा॒मꣳ रा॒यः । \newline
83. वा॒मꣳ रा॒यो रा॒यो वा॒मम् ॅवा॒मꣳ रा॒य स्पोष॒म् पोषꣳ॑ रा॒यो वा॒मम् ॅवा॒मꣳ रा॒य स्पोष᳚म् । \newline
84. रा॒य स्पोष॒म् पोषꣳ॑ रा॒यो रा॒य स्पोष॑म् चिकि॒तुषे॑ चिकि॒तुषे॒ पोषꣳ॑ रा॒यो रा॒य स्पोष॑म् चिकि॒तुषे᳚ । \newline
85. पोष॑म् चिकि॒तुषे॑ चिकि॒तुषे॒ पोष॒म् पोष॑म् चिकि॒तुषे॑ दधातु दधातु चिकि॒तुषे॒ पोष॒म् पोष॑म् चिकि॒तुषे॑ दधातु । \newline
86. चि॒कि॒तुषे॑ दधातु दधातु चिकि॒तुषे॑ चिकि॒तुषे॑ दधातु । \newline
87. द॒धा॒त्विति॑ दधातु । \newline
\pagebreak


\end{document}