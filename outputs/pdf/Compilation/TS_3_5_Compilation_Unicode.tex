\documentclass[17pt]{extarticle}
\usepackage{babel}
\usepackage{fontspec}
\usepackage{polyglossia}
\usepackage{extsizes}

\usepackage{color}   %May be necessary if you want to color links
\usepackage{hyperref}
\hypersetup{
    colorlinks=true, %set true if you want colored links
    linktoc=all,     %set to all if you want both sections and subsections linked
    linkcolor=black,  %choose some color if you want links to stand out
}

\setmainlanguage{sanskrit}
\setotherlanguages{english} %% or other languages
\setlength{\parindent}{0pt}
\pagestyle{myheadings}
\newfontfamily\devanagarifont[Script=Devanagari]{AdishilaVedic}
\renewcommand{\theHsection}{\thepart.section.\thesection}

\newcommand{\VAR}[1]{}
\newcommand{\BLOCK}[1]{}




\begin{document}
\begin{titlepage}
    \begin{center}
 
\begin{sanskrit}
    { \Large
    कृष्ण यजुर्वेदीय तैत्तिरीय संहिता,पद,जटा,घन पाठः 
    }
    \\
    \vspace{2.5cm}
    \mbox{ \Large
    3.5      तृतीयकाण्डे पञ्चमः प्रश्नः - इष्टिशेषाभिधानं   }
\end{sanskrit}
\end{center}

\end{titlepage}
\tableofcontents
\phantomsection
\pagebreak

\markright{ TS 3.5.1.1  \hfill https://www.vedavms.in \hfill}

\section{ TS 3.5.1.1 }

\textbf{TS 3.5.1.1 } \newline
\textbf{Samhita Paata} \newline

पू॒र्णा प॒श्चादु॒त पू॒र्णा पु॒रस्ता॒दुन्-म॑द्ध्य॒तः पौ᳚र्णमा॒सी जि॑गाय । तस्या᳚न् दे॒वा अधि॑ सं॒ॅवस॑न्त उत्त॒मे नाक॑ इ॒ह मा॑दयन्तां ॥ यत्ते॑ दे॒वा अद॑धु र्भाग॒धेय॒ममा॑वास्ये सं॒ॅवस॑न्तो महि॒त्वा । सानो॑ य॒ज्ञ्ं पि॑पृहि विश्ववारे र॒यिं नो॑ धेहि सुभगे सु॒वीरं᳚ ॥नि॒वेश॑नी स॒ङ्गम॑नी॒ वसू॑नां॒ ॅविश्वा॑ रू॒पाणि॒ वसू᳚न्यावे॒शय॑न्ती । स॒ह॒स्र॒पो॒षꣳ सु॒भगा॒ ररा॑णा॒ सा न॒ आग॒न्. वर्च॑सा - [  ] \newline

\textbf{Pada Paata} \newline

पू॒र्णा । प॒श्चात् । उ॒त । पू॒र्णा । पु॒रस्ता᳚त् । उदिति॑ । म॒द्ध्य॒तः । पौ॒र्ण॒मा॒सीति॑ पौर्ण - मा॒सी । जि॒गा॒य॒ ॥ तस्या᳚म् । दे॒वाः । अधीति॑ । सं॒ॅवस॑न्त॒ इति॑ सं - वस॑न्तः । उ॒त्त॒म इत्यु॑त् - त॒मे । नाके᳚ । इ॒ह । मा॒द॒य॒न्ता॒म् ॥ यत् । ते॒ । दे॒वाः । अद॑धुः । भा॒ग॒धेय॒मिति॑ भाग - धेय᳚म् । अमा॑वास्य॒ इत्यमा᳚ - वा॒स्ये॒ । सं॒ॅवस॑न्त॒ इति॑ सं - वस॑न्तः । म॒हि॒त्वेति॑ महि-त्वा ॥ सा । नः॒ । य॒ज्ञ्म् । पि॒पृ॒हि॒ । वि॒श्व॒वा॒र॒ इति॑ विश्व - वा॒रे॒ । र॒यिम् । नः॒ । धे॒हि॒ । सु॒भ॒ग॒ इति॑ सु - भ॒गे॒ । सु॒वीर॒मिति॑ सु - वीर᳚म् ॥ नि॒वेश॒नीति॑ नि - वेश॑नी । स॒गंम॒नीति॑ सं - गम॑नी । वसू॑नाम् । विश्वा᳚ । रू॒पाणि॑ । वसू॑नि । आ॒वे॒शय॒न्तीत्या᳚ - वे॒शय॑न्ती ॥ स॒ह॒स्र॒पो॒षमिति॑ सहस्र - पो॒षम् । सु॒भगेति॑ सु-भगा᳚ । ररा॑णा । सा । नः॒ । एति॑ । ग॒न्न् । वर्च॑सा ।  \newline


\textbf{Krama Paata} \newline

पू॒र्णा प॒श्चात् । प॒श्चादु॒त । उ॒त पू॒र्णा । पू॒र्णा पु॒रस्ता᳚त् । पु॒रस्ता॒दुत् । उन् म॑द्ध्य॒तः । म॒द्ध्य॒तः पौ᳚र्णमा॒सी । पौ॒र्ण॒मा॒सी जि॑गाय । पौ॒र्ण॒मा॒सीति॑ पौर्ण - मा॒सी । जि॒गा॒येति॑ जिगाय ॥ तस्या᳚न् दे॒वाः । दे॒वा अधि॑ । अधि॑ स॒म्ॅवस॑न्तः । स॒म्ॅवस॑न्त उत्त॒मे । स॒म्ॅवस॑न्त॒ इति॑ सम् - वस॑न्तः । उ॒त्त॒मे नाके᳚ । उ॒त्त॒म इत्यु॑त् - त॒मे । नाक॑ इ॒ह । इ॒ह मा॑दयन्ताम् । मा॒द॒य॒न्ता॒मिति॑ मादयन्ताम् ॥ यत् ते᳚ । ते॒ दे॒वाः । दे॒वा अद॑धुः । अद॑धुर् भाग॒धेय᳚म् । भा॒ग॒धेय॒ममा॑वास्ये । भा॒ग॒धेय॒मिति॑ भाग - धेय᳚म् । अमा॑वास्ये स॒म्ॅवस॑न्तः । अमा॑वास्य॒ इत्यमा᳚ - वा॒स्ये॒ । स॒म्ॅवस॑न्तो महि॒त्वा । स॒म्ॅवस॑न्त॒ इति॑ सम् - वस॑न्तः । म॒हि॒त्वेति॑ महि - त्वा ॥ सा नः॑ । नो॒ य॒ज्ञ्म् । य॒ज्ञ्म् पि॑पृहि । पि॒पृ॒हि॒ वि॒श्व॒वा॒रे॒ । वि॒श्व॒वा॒रे॒ र॒यिम् । वि॒श्व॒वा॒र॒ इति॑ विश्व - वा॒रे॒ । र॒यिम् नः॑ । नो॒ धे॒हि॒ । धे॒हि॒ सु॒भ॒गे॒ । सु॒भ॒गे॒ सु॒वीर᳚म् । सु॒भ॒ग॒ इति॑ सु - भ॒गे॒ । सु॒वीर॒मिति॑ सु - वीर᳚म् ॥ नि॒वेश॑नी स॒ङ्गम॑नी । नि॒वेश॒नीति॑ नि - वेश॑नी । स॒ङ्गम॑नी॒ वसू॑नाम् । स॒ङ्गम॒नीति॑ सम् - गम॑नी । वसू॑नां॒ ॅविश्वा᳚ । विश्वा॑ रू॒पाणि॑ । रू॒पाणि॒ वसू॑नि । वसू᳚न्यावे॒शय॑न्ती । आ॒वे॒शय॒न्तीत्या᳚ - वे॒शय॑न्ती ॥ स॒ह॒स्र॒पो॒षꣳ सु॒भगा᳚ । स॒ह॒स्र॒पो॒षमिति॑ सहस्र - पो॒षम् । सु॒भगा॒ ररा॑णा । सु॒भगेति॑ सु - भगा᳚ । ररा॑णा॒ सा । सा नः॑ । न॒ आ । आ गन्न्॑ । ग॒न् वर्च॑सा । वर्च॑सा सम्ॅविदा॒ना \newline

\textbf{Jatai Paata} \newline

1. पू॒र्णा प॒श्चात् प॒श्चात् पू॒र्णा पू॒र्णा प॒श्चात् । \newline
2. प॒श्चा दु॒तोत प॒श्चात् प॒श्चा दु॒त । \newline
3. उ॒त पू॒र्णा पू॒र्णोतोत पू॒र्णा । \newline
4. पू॒र्णा पु॒रस्ता᳚त् पु॒रस्ता᳚त् पू॒र्णा पू॒र्णा पु॒रस्ता᳚त् । \newline
5. पु॒रस्ता॒ दुदुत् पु॒रस्ता᳚त् पु॒रस्ता॒ दुत् । \newline
6. उन् म॑द्ध्य॒तो म॑द्ध्य॒त उदुन् म॑द्ध्य॒तः । \newline
7. म॒द्ध्य॒तः पौ᳚र्णमा॒सी पौ᳚र्णमा॒सी म॑द्ध्य॒तो म॑द्ध्य॒तः पौ᳚र्णमा॒सी । \newline
8. पौ॒र्ण॒मा॒सी जि॑गाय जिगाय पौर्णमा॒सी पौ᳚र्णमा॒सी जि॑गाय । \newline
9. पौ॒र्ण॒मा॒सीति॑ पौर्ण - मा॒सी । \newline
10. जि॒गा॒येति॑ जिगाय । \newline
11. तस्या᳚म् दे॒वा दे॒वा स्तस्या॒म् तस्या᳚म् दे॒वाः । \newline
12. दे॒वा अध्यधि॑ दे॒वा दे॒वा अधि॑ । \newline
13. अधि॑ सं॒ॅवस॑न्तः सं॒ॅवस॒न्तो ऽध्यधि॑ सं॒ॅवस॑न्तः । \newline
14. सं॒ॅवस॑न्त उत्त॒म उ॑त्त॒मे सं॒ॅवस॑न्तः सं॒ॅवस॑न्त उत्त॒मे । \newline
15. सं॒ॅवस॑न्त॒ इति॑ सं - वस॑न्तः । \newline
16. उ॒त्त॒मे नाके॒ नाक॑ उत्त॒म उ॑त्त॒मे नाके᳚ । \newline
17. उ॒त्त॒म इत्यु॑त् - त॒मे । \newline
18. नाक॑ इ॒हेह नाके॒ नाक॑ इ॒ह । \newline
19. इ॒ह मा॑दयन्ताम् मादयन्ता मि॒हेह मा॑दयन्ताम् । \newline
20. मा॒द॒य॒न्ता॒मिति॑ मादयन्ताम् । \newline
21. यत् ते॑ ते॒ यद् यत् ते᳚ । \newline
22. ते॒ दे॒वा दे॒वा स्ते॑ ते दे॒वाः । \newline
23. दे॒वा अद॑धु॒ रद॑धुर् दे॒वा दे॒वा अद॑धुः । \newline
24. अद॑धुर् भाग॒धेय॑म् भाग॒धेय॒ मद॑धु॒ रद॑धुर् भाग॒धेय᳚म् । \newline
25. भा॒ग॒धेय॒ ममा॑वा॒स्ये ऽमा॑वास्ये भाग॒धेय॑म् भाग॒धेय॒ ममा॑वास्ये । \newline
26. भा॒ग॒धेय॒मिति॑ भाग - धेय᳚म् । \newline
27. अमा॑वास्ये सं॒ॅवस॑न्तः सं॒ॅवस॒न्तो ऽमा॑वा॒स्ये ऽमा॑वास्ये सं॒ॅवस॑न्तः । \newline
28. अमा॑वास्य॒ इत्यमा᳚ - वा॒स्ये॒ । \newline
29. सं॒ॅवस॑न्तो महि॒त्वा म॑हि॒त्वा सं॒ॅवस॑न्तः सं॒ॅवस॑न्तो महि॒त्वा । \newline
30. सं॒ॅवस॑न्त॒ इति॑ सं - वस॑न्तः । \newline
31. म॒हि॒त्वेति॑ महि - त्वा । \newline
32. सा नो॑ नः॒ सा सा नः॑ । \newline
33. नो॒ य॒ज्ञ्ं ॅय॒ज्ञ्म् नो॑ नो य॒ज्ञ्म् । \newline
34. य॒ज्ञ्म् पि॑पृहि पिपृहि य॒ज्ञ्ं ॅय॒ज्ञ्म् पि॑पृहि । \newline
35. पि॒पृ॒हि॒ वि॒श्व॒वा॒रे॒ वि॒श्व॒वा॒रे॒ पि॒पृ॒हि॒ पि॒पृ॒हि॒ वि॒श्व॒वा॒रे॒ । \newline
36. वि॒श्व॒वा॒रे॒ र॒यिꣳ र॒यिं ॅवि॑श्ववारे विश्ववारे र॒यिम् । \newline
37. वि॒श्व॒वा॒र॒ इति॑ विश्व - वा॒रे॒ । \newline
38. र॒यिम् नो॑ नो र॒यिꣳ र॒यिम् नः॑ । \newline
39. नो॒ धे॒हि॒ धे॒हि॒ नो॒ नो॒ धे॒हि॒ । \newline
40. धे॒हि॒ सु॒भ॒गे॒ सु॒भ॒गे॒ धे॒हि॒ धे॒हि॒ सु॒भ॒गे॒ । \newline
41. सु॒भ॒गे॒ सु॒वीरꣳ॑ सु॒वीरꣳ॑ सुभगे सुभगे सु॒वीर᳚म् । \newline
42. सु॒भ॒ग॒ इति॑ सु - भ॒गे॒ । \newline
43. सु॒वीर॒मिति॑ सु - वीर᳚म् । \newline
44. नि॒वेश॑नी स॒ङ्गम॑नी स॒ङ्गम॑नी नि॒वेश॑नी नि॒वेश॑नी स॒ङ्गम॑नी । \newline
45. नि॒वेश॒नीति॑ नि - वेश॑नी । \newline
46. स॒ङ्गम॑नी॒ वसू॑नां॒ ॅवसू॑नाꣳ स॒ङ्गम॑नी स॒ङ्गम॑नी॒ वसू॑नाम् । \newline
47. स॒ङ्गम॒नीति॑ सं - गम॑नी । \newline
48. वसू॑नां॒ ॅविश्वा॒ विश्वा॒ वसू॑नां॒ ॅवसू॑नां॒ ॅविश्वा᳚ । \newline
49. विश्वा॑ रू॒पाणि॑ रू॒पाणि॒ विश्वा॒ विश्वा॑ रू॒पाणि॑ । \newline
50. रू॒पाणि॒ वसू॑नि॒ वसू॑नि रू॒पाणि॑ रू॒पाणि॒ वसू॑नि । \newline
51. वसू᳚ न्यावे॒शय॑ न्त्यावे॒शय॑न्ती॒ वसू॑नि॒ वसू᳚ न्यावे॒शय॑न्ती । \newline
52. आ॒वे॒शय॒न्तीत्या᳚ - वे॒शय॑न्ती । \newline
53. स॒ह॒स्र॒पो॒षꣳ सु॒भगा॑ सु॒भगा॑ सहस्रपो॒षꣳ स॑हस्रपो॒षꣳ सु॒भगा᳚ । \newline
54. स॒ह॒स्र॒पो॒षमिति॑ सहस्र - पो॒षम् । \newline
55. सु॒भगा॒ ररा॑णा॒ ररा॑णा सु॒भगा॑ सु॒भगा॒ ररा॑णा । \newline
56. सु॒भगेति॑ सु - भगा᳚ । \newline
57. ररा॑णा॒ सा सा ररा॑णा॒ ररा॑णा॒ सा । \newline
58. सा नो॑ नः॒ सा सा नः॑ । \newline
59. न॒ आ नो॑ न॒ आ । \newline
60. आ ग॑न् ग॒न् ना गन्न्॑ । \newline
61. ग॒न् वर्च॑सा॒ वर्च॑सा गन् ग॒न् वर्च॑सा । \newline
62. वर्च॑सा संॅविदा॒ना सं॑ॅविदा॒ना वर्च॑सा॒ वर्च॑सा संॅविदा॒ना । \newline

\textbf{Ghana Paata } \newline

1. पू॒र्णा प॒श्चात् प॒श्चात् पू॒र्णा पू॒र्णा प॒श्चा दु॒तोत प॒श्चात् पू॒र्णा पू॒र्णा प॒श्चा दु॒त । \newline
2. प॒श्चा दु॒तोत प॒श्चात् प॒श्चा दु॒त पू॒र्णा पू॒र्णोत प॒श्चात् प॒श्चा दु॒त पू॒र्णा । \newline
3. उ॒त पू॒र्णा पू॒र्णोतोत पू॒र्णा पु॒रस्ता᳚त् पु॒रस्ता᳚त् पू॒र्णोतोत पू॒र्णा पु॒रस्ता᳚त् । \newline
4. पू॒र्णा पु॒रस्ता᳚त् पु॒रस्ता᳚त् पू॒र्णा पू॒र्णा पु॒रस्ता॒ दुदुत् पु॒रस्ता᳚त् पू॒र्णा पू॒र्णा पु॒रस्ता॒ दुत् । \newline
5. पु॒रस्ता॒ दुदुत् पु॒रस्ता᳚त् पु॒रस्ता॒ दुन् म॑द्ध्य॒तो म॑द्ध्य॒त उत् पु॒रस्ता᳚त् पु॒रस्ता॒ दुन् म॑द्ध्य॒तः । \newline
6. उन् म॑द्ध्य॒तो म॑द्ध्य॒त उदुन् म॑द्ध्य॒तः पौ᳚र्णमा॒सी पौ᳚र्णमा॒सी म॑द्ध्य॒त उदुन् म॑द्ध्य॒तः पौ᳚र्णमा॒सी । \newline
7. म॒द्ध्य॒तः पौ᳚र्णमा॒सी पौ᳚र्णमा॒सी म॑द्ध्य॒तो म॑द्ध्य॒तः पौ᳚र्णमा॒सी जि॑गाय जिगाय पौर्णमा॒सी म॑द्ध्य॒तो म॑द्ध्य॒तः पौ᳚र्णमा॒सी जि॑गाय । \newline
8. पौ॒र्ण॒मा॒सी जि॑गाय जिगाय पौर्णमा॒सी पौ᳚र्णमा॒सी जि॑गाय । \newline
9. पौ॒र्ण॒मा॒सीति॑ पौर्ण - मा॒सी । \newline
10. जि॒गा॒येति॑ जिगाय । \newline
11. तस्या᳚म् दे॒वा दे॒वा स्तस्या॒म् तस्या᳚म् दे॒वा अध्यधि॑ दे॒वा स्तस्या॒म् तस्या᳚म् दे॒वा अधि॑ । \newline
12. दे॒वा अध्यधि॑ दे॒वा दे॒वा अधि॑ सं॒ॅवस॑न्तः सं॒ॅवस॒न्तो ऽधि॑ दे॒वा दे॒वा अधि॑ सं॒ॅवस॑न्तः । \newline
13. अधि॑ सं॒ॅवस॑न्तः सं॒ॅवस॒न्तो ऽध्यधि॑ सं॒ॅवस॑न्त उत्त॒म उ॑त्त॒मे सं॒ॅवस॒न्तो ऽध्यधि॑ सं॒ॅवस॑न्त उत्त॒मे । \newline
14. सं॒ॅवस॑न्त उत्त॒म उ॑त्त॒मे सं॒ॅवस॑न्तः सं॒ॅवस॑न्त उत्त॒मे नाके॒ नाक॑ उत्त॒मे सं॒ॅवस॑न्तः सं॒ॅवस॑न्त उत्त॒मे नाके᳚ । \newline
15. सं॒ॅवस॑न्त॒ इति॑ सं - वस॑न्तः । \newline
16. उ॒त्त॒मे नाके॒ नाक॑ उत्त॒म उ॑त्त॒मे नाक॑ इ॒हेह नाक॑ उत्त॒म उ॑त्त॒मे नाक॑ इ॒ह । \newline
17. उ॒त्त॒म इत्यु॑त् - त॒मे । \newline
18. नाक॑ इ॒हेह नाके॒ नाक॑ इ॒ह मा॑दयन्ताम् मादयन्ता मि॒ह नाके॒ नाक॑ इ॒ह मा॑दयन्ताम् । \newline
19. इ॒ह मा॑दयन्ताम् मादयन्ता मि॒हेह मा॑दयन्ताम् । \newline
20. मा॒द॒य॒न्ता॒मिति॑ मादयन्ताम् । \newline
21. यत् ते॑ ते॒ यद् यत् ते॑ दे॒वा दे॒वा स्ते॒ यद् यत् ते॑ दे॒वाः । \newline
22. ते॒ दे॒वा दे॒वा स्ते॑ ते दे॒वा अद॑धु॒ रद॑धुर् दे॒वा स्ते॑ ते दे॒वा अद॑धुः । \newline
23. दे॒वा अद॑धु॒ रद॑धुर् दे॒वा दे॒वा अद॑धुर् भाग॒धेय॑म् भाग॒धेय॒ मद॑धुर् दे॒वा दे॒वा अद॑धुर् भाग॒धेय᳚म् । \newline
24. अद॑धुर् भाग॒धेय॑म् भाग॒धेय॒ मद॑धु॒ रद॑धुर् भाग॒धेय॒ ममा॑वा॒स्ये ऽमा॑वास्ये भाग॒धेय॒ मद॑धु॒ रद॑धुर् भाग॒धेय॒ ममा॑वास्ये । \newline
25. भा॒ग॒धेय॒ ममा॑वा॒स्ये ऽमा॑वास्ये भाग॒धेय॑म् भाग॒धेय॒ ममा॑वास्ये सं॒ॅवस॑न्तः सं॒ॅवस॒न्तो ऽमा॑वास्ये भाग॒धेय॑म् भाग॒धेय॒ ममा॑वास्ये सं॒ॅवस॑न्तः । \newline
26. भा॒ग॒धेय॒मिति॑ भाग - धेय᳚म् । \newline
27. अमा॑वास्ये सं॒ॅवस॑न्तः सं॒ॅवस॒न्तो ऽमा॑वा॒स्ये ऽमा॑वास्ये सं॒ॅवस॑न्तो महि॒त्वा म॑हि॒त्वा सं॒ॅवस॒न्तो ऽमा॑वा॒स्ये ऽमा॑वास्ये सं॒ॅवस॑न्तो महि॒त्वा । \newline
28. अमा॑वास्य॒ इत्यमा᳚ - वा॒स्ये॒ । \newline
29. सं॒ॅवस॑न्तो महि॒त्वा म॑हि॒त्वा सं॒ॅवस॑न्तः सं॒ॅवस॑न्तो महि॒त्वा । \newline
30. सं॒ॅवस॑न्त॒ इति॑ सं - वस॑न्तः । \newline
31. म॒हि॒त्वेति॑ महि - त्वा । \newline
32. सा नो॑ नः॒ सा सा नो॑ य॒ज्ञ्ं ॅय॒ज्ञ्म् नः॒ सा सा नो॑ य॒ज्ञ्म् । \newline
33. नो॒ य॒ज्ञ्ं ॅय॒ज्ञ्म् नो॑ नो य॒ज्ञ्म् पि॑पृहि पिपृहि य॒ज्ञ्म् नो॑ नो य॒ज्ञ्म् पि॑पृहि । \newline
34. य॒ज्ञ्म् पि॑पृहि पिपृहि य॒ज्ञ्ं ॅय॒ज्ञ्म् पि॑पृहि विश्ववारे विश्ववारे पिपृहि य॒ज्ञ्ं ॅय॒ज्ञ्म् पि॑पृहि विश्ववारे । \newline
35. पि॒पृ॒हि॒ वि॒श्व॒वा॒रे॒ वि॒श्व॒वा॒रे॒ पि॒पृ॒हि॒ पि॒पृ॒हि॒ वि॒श्व॒वा॒रे॒ र॒यिꣳ र॒यिं ॅवि॑श्ववारे पिपृहि पिपृहि विश्ववारे र॒यिम् । \newline
36. वि॒श्व॒वा॒रे॒ र॒यिꣳ र॒यिं ॅवि॑श्ववारे विश्ववारे र॒यिम् नो॑ नो र॒यिं ॅवि॑श्ववारे विश्ववारे र॒यिम् नः॑ । \newline
37. वि॒श्व॒वा॒र॒ इति॑ विश्व - वा॒रे॒ । \newline
38. र॒यिम् नो॑ नो र॒यिꣳ र॒यिम् नो॑ धेहि धेहि नो र॒यिꣳ र॒यिम् नो॑ धेहि । \newline
39. नो॒ धे॒हि॒ धे॒हि॒ नो॒ नो॒ धे॒हि॒ सु॒भ॒गे॒ सु॒भ॒गे॒ धे॒हि॒ नो॒ नो॒ धे॒हि॒ सु॒भ॒गे॒ । \newline
40. धे॒हि॒ सु॒भ॒गे॒ सु॒भ॒गे॒ धे॒हि॒ धे॒हि॒ सु॒भ॒गे॒ सु॒वीरꣳ॑ सु॒वीरꣳ॑ सुभगे धेहि धेहि सुभगे सु॒वीर᳚म् । \newline
41. सु॒भ॒गे॒ सु॒वीरꣳ॑ सु॒वीरꣳ॑ सुभगे सुभगे सु॒वीर᳚म् । \newline
42. सु॒भ॒ग॒ इति॑ सु - भ॒गे॒ । \newline
43. सु॒वीर॒मिति॑ सु - वीर᳚म् । \newline
44. नि॒वेश॑नी स॒ङ्गम॑नी स॒ङ्गम॑नी नि॒वेश॑नी नि॒वेश॑नी स॒ङ्गम॑नी॒ वसू॑नां॒ ॅवसू॑नाꣳ स॒ङ्गम॑नी नि॒वेश॑नी नि॒वेश॑नी स॒ङ्गम॑नी॒ वसू॑नाम् । \newline
45. नि॒वेश॒नीति॑ नि - वेश॑नी । \newline
46. स॒ङ्गम॑नी॒ वसू॑नां॒ ॅवसू॑नाꣳ स॒ङ्गम॑नी स॒ङ्गम॑नी॒ वसू॑नां॒ ॅविश्वा॒ विश्वा॒ वसू॑नाꣳ स॒ङ्गम॑नी स॒ङ्गम॑नी॒ वसू॑नां॒ ॅविश्वा᳚ । \newline
47. स॒ङ्गम॒नीति॑ सं - गम॑नी । \newline
48. वसू॑नां॒ ॅविश्वा॒ विश्वा॒ वसू॑नां॒ ॅवसू॑नां॒ ॅविश्वा॑ रू॒पाणि॑ रू॒पाणि॒ विश्वा॒ वसू॑नां॒ ॅवसू॑नां॒ ॅविश्वा॑ रू॒पाणि॑ । \newline
49. विश्वा॑ रू॒पाणि॑ रू॒पाणि॒ विश्वा॒ विश्वा॑ रू॒पाणि॒ वसू॑नि॒ वसू॑नि रू॒पाणि॒ विश्वा॒ विश्वा॑ रू॒पाणि॒ वसू॑नि । \newline
50. रू॒पाणि॒ वसू॑नि॒ वसू॑नि रू॒पाणि॑ रू॒पाणि॒ वसू᳚ न्यावे॒शय॑ न्त्यावे॒शय॑न्ती॒ वसू॑नि रू॒पाणि॑ रू॒पाणि॒ 
वसू᳚ न्यावे॒शय॑न्ती । \newline
51. वसू᳚ न्यावे॒शय॑ न्त्यावे॒शय॑न्ती॒ वसू॑नि॒ वसू᳚ न्यावे॒शय॑न्ती । \newline
52. आ॒वे॒शय॒न्तीत्या᳚ - वे॒शय॑न्ती । \newline
53. स॒ह॒स्र॒पो॒षꣳ सु॒भगा॑ सु॒भगा॑ सहस्रपो॒षꣳ स॑हस्रपो॒षꣳ सु॒भगा॒ ररा॑णा॒ ररा॑णा सु॒भगा॑ सहस्रपो॒षꣳ स॑हस्रपो॒षꣳ सु॒भगा॒ ररा॑णा । \newline
54. स॒ह॒स्र॒पो॒षमिति॑ सहस्र - पो॒षम् । \newline
55. सु॒भगा॒ ररा॑णा॒ ररा॑णा सु॒भगा॑ सु॒भगा॒ ररा॑णा॒ सा सा ररा॑णा सु॒भगा॑ सु॒भगा॒ ररा॑णा॒ सा । \newline
56. सु॒भगेति॑ सु - भगा᳚ । \newline
57. ररा॑णा॒ सा सा ररा॑णा॒ ररा॑णा॒ सा नो॑ नः॒ सा ररा॑णा॒ ररा॑णा॒ सा नः॑ । \newline
58. सा नो॑ नः॒ सा सा न॒ आ नः॒ सा सा न॒ आ । \newline
59. न॒ आ नो॑ न॒ आ ग॑न् ग॒न् ना नो॑ न॒ आ गन्न्॑ । \newline
60. आ ग॑न् ग॒न् ना ग॒न् वर्च॑सा॒ वर्च॑सा ग॒न् ना ग॒न् वर्च॑सा । \newline
61. ग॒न् वर्च॑सा॒ वर्च॑सा गन् ग॒न् वर्च॑सा संॅविदा॒ना सं॑ॅविदा॒ना वर्च॑सा गन् ग॒न् वर्च॑सा संॅविदा॒ना । \newline
62. वर्च॑सा संॅविदा॒ना सं॑ॅविदा॒ना वर्च॑सा॒ वर्च॑सा संॅविदा॒ना । \newline
\pagebreak
\markright{ TS 3.5.1.2  \hfill https://www.vedavms.in \hfill}

\section{ TS 3.5.1.2 }

\textbf{TS 3.5.1.2 } \newline
\textbf{Samhita Paata} \newline

संॅविदा॒ना ॥ अग्नी॑षोमौ प्रथ॒मौ वी॒र्ये॑ण॒ वसू᳚न् रु॒द्राना॑दि॒त्यानि॒ह जि॑न्वतं ।मा॒द्ध्यꣳ हि पौ᳚र्णमा॒सं जु॒षेथां॒ ब्रह्म॑णा वृ॒द्धौ सु॑कृ॒तेन॑ सा॒तावथा॒ऽस्मभ्यꣳ॑ स॒हवी॑राꣳ र॒यिं नि य॑च्छतं ॥ आ॒दि॒त्याश्चाऽङ्गि॑रसश्चा॒ग्नीनाऽद॑धत॒ ते द॑र्.शपूर्णमा॒सौ प्रैफ्स॒न् तेषा॒मङ्गि॑रसां॒ निरु॑प्तꣳ ह॒विरासी॒दथा॑ऽऽ*दि॒त्या ए॒तौ होमा॑वपश्य॒न् ताव॑जुहवु॒स्ततो॒ वै ते द॑र्.शपूर्णमा॒सौ - [  ] \newline

\textbf{Pada Paata} \newline

सं॒ॅवि॒दा॒नेति॑ सं - वि॒दा॒ना ॥ अग्नी॑षोमा॒वित्यग्नी᳚ - सो॒मौ॒ । प्र॒थ॒मौ । वी॒र्ये॑ण । वसून्॑ । रु॒द्रान् । आ॒दि॒त्यान् । इ॒ह । जि॒न्व॒त॒म् ॥ मा॒द्ध्यम् । हि । पौ॒र्ण॒मा॒समिति॑ पौर्ण - मा॒सम् । जु॒षेथा᳚म् । ब्रह्म॑णा । वृ॒द्धौ । सु॒कृ॒तेनेति॑ सु - कृ॒तेन॑ । सा॒तौ । अथ॑ । अ॒स्मभ्य॒मित्य॒स्म-भ्य॒म् । स॒हवी॑रा॒मिति॑ स॒ह-वी॒रा॒म् । र॒यिम् । नीति॑ । य॒च्छ॒त॒म् ॥ आ॒दि॒त्याः । च॒ । अङ्गि॑रसः । च॒ । अ॒ग्नीन् । एति॑ । अ॒द॒ध॒त॒ । ते । द॒र्॒.श॒पू॒र्ण॒मा॒साविति॑ दर्.श - पू॒र्ण॒मा॒सौ । प्रेति॑ । ऐ॒फ्स॒न्न् । तेषा᳚म् । अङ्गि॑रसाम् । निरु॑प्त॒मिति॒ निः - उ॒प्त॒म् । ह॒विः । आसी᳚त् । अथ॑ । आ॒दि॒त्याः । ए॒तौ । होमौ᳚ । अ॒प॒श्य॒न्न् । तौ । अ॒जु॒ह॒वुः॒ । ततः॑ । वै । ते । द॒र्॒.श॒पू॒र्ण॒मा॒साविति॑ दर्.श - पू॒र्ण॒मा॒सौ ।  \newline


\textbf{Krama Paata} \newline

स॒म्ॅवि॒दा॒नेति॑ सम् - वि॒दा॒ना ॥ अग्नी॑षोमौ प्रथ॒मौ । अग्नी॑षोमा॒वित्यग्नी᳚ - सो॒मौ॒ । प्र॒थ॒मौ वी॒र्ये॑ण । वी॒र्ये॑ण॒ वसून्॑ । वसू᳚न् रु॒द्रान् । रु॒द्राना॑दि॒त्यान् । आ॒दि॒त्यानि॒ह । इ॒ह जि॑न्वतम् । जि॒न्व॒त॒मिति॑ जिन्वतम् ॥ मा॒द्ध्यꣳ हि । हि पौ᳚र्णमा॒सम् । पौ॒र्ण॒मा॒सम् जु॒षेथा᳚म् । पौ॒र्ण॒मा॒समिति॑ पौर्ण - मा॒सम् । जु॒षेथा॒म् ब्रह्म॑णा । ब्रह्म॑णा वृ॒द्धौ । वृ॒द्धौ सु॑कृ॒तेन॑ । सु॒कृ॒तेन॑ सा॒तौ । सु॒कृ॒तेनेति॑ सु - कृ॒तेन॑ । सा॒तावथ॑ । अथा॒स्मभ्य᳚म् । अ॒स्मभ्यꣳ॑ स॒हवी॑राम् । अ॒स्मभ्य॒मित्य॒स्म - भ्य॒म् । स॒हवी॑राꣳ र॒यिम् । स॒हवी॑रा॒मिति॑ स॒ह - वी॒रा॒म् । र॒यिम् नि । नि य॑च्छतम् । य॒च्छ॒त॒मिति॑ यच्छतम् ॥ आ॒दि॒त्याश्च॑ । चाङ्गि॑रसः । अङ्गि॑रसश्च । चा॒ग्नीन् । अ॒ग्नीना । आ ऽद॑धत । अ॒द॒ध॒त॒ ते । ते द॑र्.शपूर्णमा॒सौ । द॒र्॒श॒पू॒र्ण॒मा॒सौ प्र । द॒र्॒.श॒पू॒र्ण॒मा॒साविति॑ दर्.श - पू॒र्ण॒मा॒सौ । प्रैफ्सन्न्॑ । ऐ॒फ्स॒न् तेषा᳚म् । तेषा॒मङ्गि॑रसाम् । अङ्गि॑रसा॒म् निरु॑प्तम् । निरु॑प्तꣳ ह॒विः । निरु॑प्त॒मिति॒ निः - उ॒प्त॒म् । ह॒विरासी᳚त् । आसी॒दथ॑ । अथा॑दि॒त्याः । आ॒दि॒त्या ए॒तौ । ए॒तौ होमौ᳚ । होमा॑वपश्यन्न् । अ॒प॒श्य॒न् तौ । ताव॑जुहवुः । अ॒जु॒ह॒वु॒स्ततः॑ । ततो॒ वै । वै ते । ते द॑र्.शपूर्णमा॒सौ । द॒र्॒.श॒पू॒र्ण॒मा॒सौ पूर्वे᳚ । द॒र्॒.श॒पू॒र्ण॒मा॒साविति॑ दर्.श - पू॒र्ण॒मा॒सौ \newline

\textbf{Jatai Paata} \newline

1. सं॒ॅवि॒दा॒नेति॑ सं - वि॒दा॒ना । \newline
2. अग्नी॑षोमौ प्रथ॒मौ प्र॑थ॒मा वग्नी॑षोमा॒ वग्नी॑षोमौ प्रथ॒मौ । \newline
3. अग्नी॑षोमा॒वित्यग्नी᳚ - सो॒मौ॒ । \newline
4. प्र॒थ॒मौ वी॒र्ये॑ण वी॒र्ये॑ण प्रथ॒मौ प्र॑थ॒मौ वी॒र्ये॑ण । \newline
5. वी॒र्ये॑ण॒ वसू॒न्॒. वसून्॑. वी॒र्ये॑ण वी॒र्ये॑ण॒ वसून्॑ । \newline
6. वसू᳚न् रु॒द्रान् रु॒द्रान्. वसू॒न्॒. वसू᳚न् रु॒द्रान् । \newline
7. रु॒द्रा ना॑दि॒त्या ना॑दि॒त्यान् रु॒द्रान् रु॒द्रा ना॑दि॒त्यान् । \newline
8. आ॒दि॒त्या नि॒हे हादि॒त्या ना॑दि॒त्या नि॒ह । \newline
9. इ॒ह जि॑न्वतम् जिन्वत मि॒हे ह जि॑न्वतम् । \newline
10. जि॒न्व॒त॒मिति॑ जिन्वतम् । \newline
11. मा॒द्ध्यꣳ हि हि मा॒द्ध्यम् मा॒द्ध्यꣳ हि । \newline
12. हि पौ᳚र्णमा॒सम् पौ᳚र्णमा॒सꣳ हि हि पौ᳚र्णमा॒सम् । \newline
13. पौ॒र्ण॒मा॒सम् जु॒षेथा᳚म् जु॒षेथा᳚म् पौर्णमा॒सम् पौ᳚र्णमा॒सम् जु॒षेथा᳚म् । \newline
14. पौ॒र्ण॒मा॒समिति॑ पौर्ण - मा॒सम् । \newline
15. जु॒षेथा॒म् ब्रह्म॑णा॒ ब्रह्म॑णा जु॒षेथा᳚म् जु॒षेथा॒म् ब्रह्म॑णा । \newline
16. ब्रह्म॑णा वृ॒द्धौ वृ॒द्धौ ब्रह्म॑णा॒ ब्रह्म॑णा वृ॒द्धौ । \newline
17. वृ॒द्धौ सु॑कृ॒तेन॑ सुकृ॒तेन॑ वृ॒द्धौ वृ॒द्धौ सु॑कृ॒तेन॑ । \newline
18. सु॒कृ॒तेन॑ सा॒तौ सा॒तौ सु॑कृ॒तेन॑ सुकृ॒तेन॑ सा॒तौ । \newline
19. सु॒कृ॒तेनेति॑ सु - कृ॒तेन॑ । \newline
20. सा॒ता वथाथ॑ सा॒तौ सा॒ता वथ॑ । \newline
21. अथा॒स्मभ्य॑ म॒स्मभ्य॒ मथाथा॒ स्मभ्य᳚म् । \newline
22. अ॒स्मभ्यꣳ॑ स॒हवी॑राꣳ स॒हवी॑रा म॒स्मभ्य॑ म॒स्मभ्यꣳ॑ स॒हवी॑राम् । \newline
23. अ॒स्मभ्य॒मित्य॒स्म - भ्य॒म् । \newline
24. स॒हवी॑राꣳ र॒यिꣳ र॒यिꣳ स॒हवी॑राꣳ स॒हवी॑राꣳ र॒यिम् । \newline
25. स॒हवी॑रा॒मिति॑ स॒ह - वी॒रा॒म् । \newline
26. र॒यिम् नि नि र॒यिꣳ र॒यिम् नि । \newline
27. नि य॑च्छतं ॅयच्छत॒म् नि नि य॑च्छतम् । \newline
28. य॒च्छ॒त॒मिति॑ यच्छतम् । \newline
29. आ॒दि॒ त्याश्च॑ चादि॒त्या आ॑दि॒ त्याश्च॑ । \newline
30. चाङ्गि॑र॒सो ऽङ्गि॑रसश्च॒ चाङ्गि॑रसः । \newline
31. अङ्गि॑रसश्च॒ चाङ्गि॑र॒सो ऽङ्गि॑रसश्च । \newline
32. चा॒ग्नी न॒ग्नीꣳ श्च॑ चा॒ग्नीन् । \newline
33. अ॒ग्नी ना ऽग्नी न॒ग्नी ना । \newline
34. आ ऽद॑धता दध॒ता ऽद॑धत । \newline
35. अ॒द॒ध॒त॒ ते ते॑ ऽदधता दधत॒ ते । \newline
36. ते द॑र्.शपूर्णमा॒सौ द॑र्.शपूर्णमा॒सौ ते ते द॑र्.शपूर्णमा॒सौ । \newline
37. द॒र्॒.श॒पू॒र्ण॒मा॒सौ प्र प्र द॑र्.शपूर्णमा॒सौ द॑र्.शपूर्णमा॒सौ प्र । \newline
38. द॒र्॒.श॒पू॒र्ण॒मा॒साविति॑ दर्.श - पू॒र्ण॒मा॒सौ । \newline
39. प्रैफ्स॑न् नैफ्स॒न् प्र प्रैफ्सन्न्॑ । \newline
40. ऐ॒फ्स॒न् तेषा॒म् तेषा॑ मैफ्सन् नैफ्स॒न् तेषा᳚म् । \newline
41. तेषा॒ मङ्गि॑रसा॒ मङ्गि॑रसा॒म् तेषा॒म् तेषा॒ मङ्गि॑रसाम् । \newline
42. अङ्गि॑रसा॒म् निरु॑प्त॒म् निरु॑प्त॒ मङ्गि॑रसा॒ मङ्गि॑रसा॒म् निरु॑प्तम् । \newline
43. निरु॑प्तꣳ ह॒विर्. ह॒विर् निरु॑प्त॒म् निरु॑प्तꣳ ह॒विः । \newline
44. निरु॑प्त॒मिति॒ निः - उ॒प्त॒म् । \newline
45. ह॒वि रासी॒ दासी᳚ द्ध॒विर्. ह॒वि रासी᳚त् । \newline
46. आसी॒ दथाथा सी॒ दासी॒ दथ॑ । \newline
47. अथा॑दि॒त्या आ॑दि॒त्या अथाथा॑ दि॒त्याः । \newline
48. आ॒दि॒त्या ए॒ता वे॒ता वा॑दि॒त्या आ॑दि॒त्या ए॒तौ । \newline
49. ए॒तौ होमौ॒ होमा॑ वे॒ता वे॒तौ होमौ᳚ । \newline
50. होमा॑ वपश्यन् नपश्य॒न्॒. होमौ॒ होमा॑ वपश्यन्न् । \newline
51. अ॒प॒श्य॒न् तौ ता व॑पश्यन् नपश्य॒न् तौ । \newline
52. ता व॑जुहवु रजुहवु॒ स्तौ ता व॑जुहवुः । \newline
53. अ॒जु॒ह॒वु॒ स्तत॒ स्ततो॑ ऽजुहवु रजुहवु॒ स्ततः॑ । \newline
54. ततो॒ वै वै तत॒ स्ततो॒ वै । \newline
55. वै ते ते वै वै ते । \newline
56. ते द॑र्.शपूर्णमा॒सौ द॑र्.शपूर्णमा॒सौ ते ते द॑र्.शपूर्णमा॒सौ । \newline
57. द॒र्॒.श॒पू॒र्ण॒मा॒सौ पूर्वे॒ पूर्वे॑ दर्.शपूर्णमा॒सौ द॑र्.शपूर्णमा॒सौ पूर्वे᳚ । \newline
58. द॒र्॒.श॒पू॒र्ण॒मा॒साविति॑ दर्.श - पू॒र्ण॒मा॒सौ । \newline

\textbf{Ghana Paata } \newline

1. सं॒ॅवि॒दा॒नेति॑ सं - वि॒दा॒ना । \newline
2. अग्नी॑षोमौ प्रथ॒मौ प्र॑थ॒मा वग्नी॑षोमा॒ वग्नी॑षोमौ प्रथ॒मौ वी॒र्ये॑ण वी॒र्ये॑ण प्रथ॒मा वग्नी॑षोमा॒ वग्नी॑षोमौ प्रथ॒मौ वी॒र्ये॑ण । \newline
3. अग्नी॑षोमा॒वित्यग्नी᳚ - सो॒मौ॒ । \newline
4. प्र॒थ॒मौ वी॒र्ये॑ण वी॒र्ये॑ण प्रथ॒मौ प्र॑थ॒मौ वी॒र्ये॑ण॒ वसू॒न्॒. वसून्॑. वी॒र्ये॑ण प्रथ॒मौ प्र॑थ॒मौ वी॒र्ये॑ण॒ वसून्॑ । \newline
5. वी॒र्ये॑ण॒ वसू॒न्॒. वसून्॑. वी॒र्ये॑ण वी॒र्ये॑ण॒ वसू᳚न् रु॒द्रान् रु॒द्रान्. वसून्॑. वी॒र्ये॑ण वी॒र्ये॑ण॒ वसू᳚न् रु॒द्रान् । \newline
6. वसू᳚न् रु॒द्रान् रु॒द्रान्. वसू॒न्॒. वसू᳚न् रु॒द्रा ना॑दि॒त्या ना॑दि॒त्यान् रु॒द्रान्. वसू॒न्॒. वसू᳚न् रु॒द्रा ना॑दि॒त्यान् । \newline
7. रु॒द्रा ना॑दि॒त्या ना॑दि॒त्यान् रु॒द्रान् रु॒द्रा ना॑दि॒त्या नि॒हे हादि॒त्यान् रु॒द्रान् रु॒द्रा ना॑दि॒त्या नि॒ह । \newline
8. आ॒दि॒त्या नि॒हेहा दि॒त्या ना॑दि॒त्या नि॒ह जि॑न्वतम् जिन्वत मि॒हा दि॒त्या ना॑दि॒त्या नि॒ह जि॑न्वतम् । \newline
9. इ॒ह जि॑न्वतम् जिन्वत मि॒हेह जि॑न्वतम् । \newline
10. जि॒न्व॒त॒मिति॑ जिन्वतम् । \newline
11. मा॒द्ध्यꣳ हि हि मा॒द्ध्यम् मा॒द्ध्यꣳ हि पौ᳚र्णमा॒सम् पौ᳚र्णमा॒सꣳ हि मा॒द्ध्यम् मा॒द्ध्यꣳ हि 
पौ᳚र्णमा॒सम् । \newline
12. हि पौ᳚र्णमा॒सम् पौ᳚र्णमा॒सꣳ हि हि पौ᳚र्णमा॒सम् जु॒षेथा᳚म् जु॒षेथा᳚म् पौर्णमा॒सꣳ हि हि 
पौ᳚र्णमा॒सम् जु॒षेथा᳚म् । \newline
13. पौ॒र्ण॒मा॒सम् जु॒षेथा᳚म् जु॒षेथा᳚म् पौर्णमा॒सम् पौ᳚र्णमा॒सम् जु॒षेथा॒म् ब्रह्म॑णा॒ ब्रह्म॑णा जु॒षेथा᳚म् पौर्णमा॒सम् पौ᳚र्णमा॒सम् जु॒षेथा॒म् ब्रह्म॑णा । \newline
14. पौ॒र्ण॒मा॒समिति॑ पौर्ण - मा॒सम् । \newline
15. जु॒षेथा॒म् ब्रह्म॑णा॒ ब्रह्म॑णा जु॒षेथा᳚म् जु॒षेथा॒म् ब्रह्म॑णा वृ॒द्धौ वृ॒द्धौ ब्रह्म॑णा जु॒षेथा᳚म् जु॒षेथा॒म् ब्रह्म॑णा वृ॒द्धौ । \newline
16. ब्रह्म॑णा वृ॒द्धौ वृ॒द्धौ ब्रह्म॑णा॒ ब्रह्म॑णा वृ॒द्धौ सु॑कृ॒तेन॑ सुकृ॒तेन॑ वृ॒द्धौ ब्रह्म॑णा॒ ब्रह्म॑णा वृ॒द्धौ सु॑कृ॒तेन॑ । \newline
17. वृ॒द्धौ सु॑कृ॒तेन॑ सुकृ॒तेन॑ वृ॒द्धौ वृ॒द्धौ सु॑कृ॒तेन॑ सा॒तौ सा॒तौ सु॑कृ॒तेन॑ वृ॒द्धौ वृ॒द्धौ सु॑कृ॒तेन॑ सा॒तौ । \newline
18. सु॒कृ॒तेन॑ सा॒तौ सा॒तौ सु॑कृ॒तेन॑ सुकृ॒तेन॑ सा॒ता वथाथ॑ सा॒तौ सु॑कृ॒तेन॑ सुकृ॒तेन॑ सा॒ता वथ॑ । \newline
19. सु॒कृ॒तेनेति॑ सु - कृ॒तेन॑ । \newline
20. सा॒ता वथाथ॑ सा॒तौ सा॒ता वथा॒स्मभ्य॑ म॒स्मभ्य॒ मथ॑ सा॒तौ सा॒ता वथा॒स्मभ्य᳚म् । \newline
21. अथा॒स्मभ्य॑ म॒स्मभ्य॒ मथाथा॒ स्मभ्यꣳ॑ स॒हवी॑राꣳ स॒हवी॑रा म॒स्मभ्य॒ मथाथा॒ स्मभ्यꣳ॑ स॒हवी॑राम् । \newline
22. अ॒स्मभ्यꣳ॑ स॒हवी॑राꣳ स॒हवी॑रा म॒स्मभ्य॑ म॒स्मभ्यꣳ॑ स॒हवी॑राꣳ र॒यिꣳ र॒यिꣳ स॒हवी॑रा म॒स्मभ्य॑ म॒स्मभ्यꣳ॑ स॒हवी॑राꣳ र॒यिम् । \newline
23. अ॒स्मभ्य॒मित्य॒स्म - भ्य॒म् । \newline
24. स॒हवी॑राꣳ र॒यिꣳ र॒यिꣳ स॒हवी॑राꣳ स॒हवी॑राꣳ र॒यिम् नि नि र॒यिꣳ स॒हवी॑राꣳ स॒हवी॑राꣳ र॒यिम् नि । \newline
25. स॒हवी॑रा॒मिति॑ स॒ह - वी॒रा॒म् । \newline
26. र॒यिम् नि नि र॒यिꣳ र॒यिम् नि य॑च्छतं ॅयच्छत॒म् नि र॒यिꣳ र॒यिम् नि य॑च्छतम् । \newline
27. नि य॑च्छतं ॅयच्छत॒म् नि नि य॑च्छतम् । \newline
28. य॒च्छ॒त॒मिति॑ यच्छतम् । \newline
29. आ॒दि॒त्याश्च॑ चादि॒त्या आ॑दि॒त्या श्चाङ्गि॑र॒सो ऽङ्गि॑रस श्चादि॒त्या आ॑दि॒त्या श्चाङ्गि॑रसः । \newline
30. चाङ्गि॑र॒सो ऽङ्गि॑रसश्च॒ चाङ्गि॑रसश्च॒ चाङ्गि॑रसश्च॒ चाङ्गि॑रसश्च । \newline
31. अङ्गि॑रसश्च॒ चाङ्गि॑र॒सो ऽङ्गि॑रस श्चा॒ग्नी न॒ग्नीꣳ श्चाङ्गि॑र॒सो ऽङ्गि॑रस श्चा॒ग्नीन् । \newline
32. चा॒ग्नी न॒ग्नीꣳश्च॑ चा॒ग्नी ना ऽग्नीꣳश्च॑ चा॒ग्नी ना । \newline
33. अ॒ग्नीना ऽग्नी न॒ग्नीना ऽद॑धता दध॒ता ऽग्नी न॒ग्नीना ऽद॑धत । \newline
34. आ ऽद॑धता दध॒ता ऽद॑धत॒ ते ते॑ ऽदध॒ता ऽद॑धत॒ ते । \newline
35. अ॒द॒ध॒त॒ ते ते॑ ऽदधता दधत॒ ते द॑र्.शपूर्णमा॒सौ द॑र्.शपूर्णमा॒सौ ते॑ ऽदधता दधत॒ ते द॑र्.शपूर्णमा॒सौ । \newline
36. ते द॑र्.शपूर्णमा॒सौ द॑र्.शपूर्णमा॒सौ ते ते द॑र्.शपूर्णमा॒सौ प्र प्र द॑र्.शपूर्णमा॒सौ ते ते द॑र्.शपूर्णमा॒सौ प्र । \newline
37. द॒र्॒.श॒पू॒र्ण॒मा॒सौ प्र प्र द॑र्.शपूर्णमा॒सौ द॑र्.शपूर्णमा॒सौ प्रैफ्स॑न् नैफ्स॒न् प्र द॑र्.शपूर्णमा॒सौ द॑र्.शपूर्णमा॒सौ प्रैफ्सन्न्॑ । \newline
38. द॒र्॒.श॒पू॒र्ण॒मा॒साविति॑ दर्.श - पू॒र्ण॒मा॒सौ । \newline
39. प्रैफ्स॑न् नैफ्स॒न् प्र प्रैफ्स॒न् तेषा॒म् तेषा॑ मैफ्स॒न् प्र प्रैफ्स॒न् तेषा᳚म् । \newline
40. ऐ॒फ्स॒न् तेषा॒म् तेषा॑ मैफ्सन् नैफ्स॒न् तेषा॒ मङ्गि॑रसा॒ मङ्गि॑रसा॒म् तेषा॑ मैफ्सन् नैफ्स॒न् तेषा॒ मङ्गि॑रसाम् । \newline
41. तेषा॒ मङ्गि॑रसा॒ मङ्गि॑रसा॒म् तेषा॒म् तेषा॒ मङ्गि॑रसा॒म् निरु॑प्त॒म् निरु॑प्त॒ मङ्गि॑रसा॒म् तेषा॒म् तेषा॒ मङ्गि॑रसा॒म् निरु॑प्तम् । \newline
42. अङ्गि॑रसा॒म् निरु॑प्त॒म् निरु॑प्त॒ मङ्गि॑रसा॒ मङ्गि॑रसा॒म् निरु॑प्तꣳ ह॒विर्. ह॒विर् निरु॑प्त॒ मङ्गि॑रसा॒ मङ्गि॑रसा॒म् निरु॑प्तꣳ ह॒विः । \newline
43. निरु॑प्तꣳ ह॒विर्. ह॒विर् निरु॑प्त॒म् निरु॑प्तꣳ ह॒विरासी॒ दासी᳚ द्ध॒विर् निरु॑प्त॒म् निरु॑प्तꣳ ह॒वि रासी᳚त् । \newline
44. निरु॑प्त॒मिति॒ निः - उ॒प्त॒म् । \newline
45. ह॒विरासी॒ दासी᳚ द्ध॒विर्. ह॒वि रासी॒ दथाथासी᳚ द्ध॒विर्. ह॒वि रासी॒ दथ॑ । \newline
46. आसी॒ दथाथासी॒ दासी॒ दथा॑ दि॒त्या आ॑दि॒त्या अथासी॒ दासी॒ दथा॑ दि॒त्याः । \newline
47. अथा॑दि॒त्या आ॑दि॒त्या अथाथा॑ दि॒त्या ए॒ता वे॒ता वा॑दि॒त्या अथाथा॑ दि॒त्या ए॒तौ । \newline
48. आ॒दि॒त्या ए॒ता वे॒ता वा॑दि॒त्या आ॑दि॒त्या ए॒तौ होमौ॒ होमा॑ वे॒ता वा॑दि॒त्या आ॑दि॒त्या ए॒तौ होमौ᳚ । \newline
49. ए॒तौ होमौ॒ होमा॑ वे॒ता वे॒तौ होमा॑ वपश्यन् नपश्य॒न्॒. होमा॑ वे॒ता वे॒तौ होमा॑ वपश्यन्न् । \newline
50. होमा॑ वपश्यन् नपश्य॒न्॒. होमौ॒ होमा॑ वपश्य॒न् तौ ता व॑पश्य॒न्॒. होमौ॒ होमा॑ वपश्य॒न् तौ । \newline
51. अ॒प॒श्य॒न् तौ ता व॑पश्यन् नपश्य॒न् ता व॑जुहवु रजुहवु॒ स्ता व॑पश्यन् नपश्य॒न् ता व॑जुहवुः । \newline
52. ता व॑जुहवु रजुहवु॒ स्तौ ता व॑जुहवु॒ स्तत॒ स्ततो॑ ऽजुहवु॒स्तौ ता व॑जुहवु॒ स्ततः॑ । \newline
53. अ॒जु॒ह॒वु॒ स्तत॒ स्ततो॑ ऽजुहवु रजुहवु॒ स्ततो॒ वै वै ततो॑ ऽजुहवु रजुहवु॒ स्ततो॒ वै । \newline
54. ततो॒ वै वै तत॒ स्ततो॒ वै ते ते वै तत॒ स्ततो॒ वै ते । \newline
55. वै ते ते वै वै ते द॑र्.शपूर्णमा॒सौ द॑र्.शपूर्णमा॒सौ ते वै वै ते द॑र्.शपूर्णमा॒सौ । \newline
56. ते द॑र्.शपूर्णमा॒सौ द॑र्.शपूर्णमा॒सौ ते ते द॑र्.शपूर्णमा॒सौ पूर्वे॒ पूर्वे॑ दर्.शपूर्णमा॒सौ ते ते द॑र्.शपूर्णमा॒सौ पूर्वे᳚ । \newline
57. द॒र्॒.श॒पू॒र्ण॒मा॒सौ पूर्वे॒ पूर्वे॑ दर्.शपूर्णमा॒सौ द॑र्.शपूर्णमा॒सौ पूर्व॒ आ पूर्वे॑ दर्.शपूर्णमा॒सौ द॑र्.शपूर्णमा॒सौ पूर्व॒ आ । \newline
58. द॒र्॒.श॒पू॒र्ण॒मा॒साविति॑ दर्.श - पू॒र्ण॒मा॒सौ । \newline
\pagebreak
\markright{ TS 3.5.1.3  \hfill https://www.vedavms.in \hfill}

\section{ TS 3.5.1.3 }

\textbf{TS 3.5.1.3 } \newline
\textbf{Samhita Paata} \newline

पूर्व॒ आ ऽल॑भन्त दर्.शपूर्णमा॒सा-वा॒लभ॑मान ए॒तौ होमौ॑ पु॒रस्ता᳚ज्जुहुयाथ् सा॒क्षादे॒व द॑र्.शपूर्णमा॒सावा ल॑भते ब्रह्मवा॒दिनो॑ वदन्ति॒ स त्वै द॑र्.शपूर्णमा॒सावा ल॑भेत॒ य ए॑नयोरनु-लो॒मञ्च॑ प्रतिलो॒मञ्च॑ वि॒द्यादित्य॑मावा॒स्या॑या ऊ॒र्द्ध्वं तद॑नुलो॒मं पौ᳚र्णमा॒स्यै प्र॑ती॒चीनं॒ तत् प्र॑तिलो॒मं ॅयत् पौ᳚र्णमा॒सीं पूर्वा॑मा॒लभे॑त प्रतिलो॒ममे॑ना॒वा ल॑भेता॒-मुम॑प॒क्षीय॑माण॒-मन्वप॑ - [  ] \newline

\textbf{Pada Paata} \newline

पूर्वे᳚ । एति॑ । अ॒ल॒भ॒न्त॒ । द॒र्॒.श॒पू॒र्ण॒मा॒साविति॑ दर्.श - पू॒र्ण॒मा॒सौ । आ॒लभ॑मान॒ इत्या᳚-लभ॑मानः । ए॒तौ । होमौ᳚ । पु॒रस्ता᳚त् । जु॒हु॒या॒त् । सा॒क्षादिति॑ स - अ॒क्षात् । ए॒व । द॒र्॒.श॒पू॒र्ण॒मा॒साविति॑ दर्.श - पू॒र्ण॒मा॒सौ । एति॑ । ल॒भे॒ते॒ । ब्र॒ह्म॒वा॒दिन॒ इति॑ ब्रह्म - वा॒दिनः॑ । व॒द॒न्ति॒ । सः । तु । वै । द॒र्॒.श॒पू॒र्ण॒मा॒साविति॑ दर्.श - पू॒र्ण॒मा॒सौ । एति॑ । ल॒भे॒त॒ । यः । ए॒न॒योः॒ । अ॒नु॒लो॒ममित्य॑नु - लो॒मम् । च॒ । प्र॒ति॒लो॒ममिति॑ प्रति - लो॒मम् । च॒ । वि॒द्यात् । इति॑ । अ॒मा॒वा॒स्या॑या॒ इत्य॑मा - वा॒स्या॑याः । ऊ॒र्द्ध्वम् । तत् । अ॒नु॒लो॒ममित्य॑नु - लो॒मम् । पौ॒र्ण॒मा॒स्या इति॑ पौर्ण - मा॒स्यै । प्र॒ती॒चीन᳚म् । तत् । प्र॒ति॒लो॒ममिति॑ प्रति - लो॒मम् । यत् । पौ॒र्ण॒मा॒सीमिति॑ पौर्ण - मा॒सीम् । पूर्वा᳚म् । आ॒लभे॒तेत्या᳚ - लभे॑त । प्र॒ति॒लो॒ममिति॑ प्रति-लो॒मम् । ए॒नौ॒ । एति॑ । ल॒भे॒त॒ । अ॒मुम् । अ॒प॒क्षीय॑माण॒मित्य॑प - क्षीय॑माणम् । अनु॑ । अपेति॑ ।  \newline


\textbf{Krama Paata} \newline

पूर्व॒ आ । आ ऽल॑भन्त । अ॒ल॒भ॒न्त॒ द॒र्॒.श॒पू॒र्ण॒मा॒सौ । द॒र्॒.श॒पू॒र्ण॒मा॒सावा॒,लभ॑मानः । द॒र्॒.श॒पू॒र्ण॒मा॒साविति॑ दर्.श - पू॒र्ण॒मा॒सौ । आ॒लभ॑मान ए॒तौ । आ॒लभ॑मान॒ इत्या᳚ - लभ॑मानः । ए॒तौ होमौ᳚ । होमौ॑ पु॒रस्ता᳚त् । पु॒रस्ता᳚ज् जुहुयात् । जु॒हु॒या॒थ् सा॒क्षात् । सा॒क्षादे॒व । सा॒क्षादिति॑ स - अ॒क्षात् । ए॒व द॑र्.शपूर्णमा॒सौ । द॒र्॒.श॒पू॒र्ण॒मा॒सावा । द॒र्॒.श॒पू॒र्ण॒मा॒साविति॑ दर्.श - पू॒र्ण॒मा॒सौ । आ ल॑भते । ल॒भ॒ते॒ ब्र॒ह्म॒वा॒दिनः॑ । ब्र॒ह्म॒वा॒दिनो॑ वदन्ति । ब्र॒ह्म॒वा॒दिन॒ इति॑ ब्रह्म - वा॒दिनः॑ । व॒द॒न्ति॒ सः । स तु । त्वै । वै द॑र्.शपूर्णमा॒सौ । द॒र्॒.श॒पू॒र्ण॒मा॒सावा । द॒र्॒.श॒पू॒र्ण॒मा॒साविति॑ दर्.श - पू॒र्ण॒मा॒सौ । 
आ ल॑भेत । ल॒भे॒त॒ यः । य ए॑नयोः । ए॒न॒यो॒र॒नु॒लो॒मम् । अ॒नु॒लो॒मम् च॑ । अ॒नु॒लो॒ममित्य॑नु - लो॒मम् । च॒ प्र॒ति॒लो॒मम् । प्र॒ति॒लो॒मम् च॑ । प्र॒ति॒लो॒ममिति॑ प्रति - लो॒मम् । च॒ वि॒द्यात् । वि॒द्यादिति॑ । इत्य॑मावा॒स्या॑याः । अ॒मा॒वा॒स्या॑या ऊ॒र्द्ध्वम् । अ॒मा॒वा॒स्या॑या॒ इत्य॑मा - वा॒स्या॑याः । ऊ॒र्द्ध्वम् तत् । तद॑नुलो॒मम् । अ॒नु॒लो॒मम् पौ᳚र्णमा॒स्यै । अ॒नु॒लो॒ममित्य॑नु - लो॒मम् । पू॒र्ण॒मा॒स्यै प्र॑ती॒चीन᳚म् । पौ॒र्ण॒मा॒स्या इति॑ पौर्ण - मा॒स्यै । प्र॒ती॒चीन॒म् तत् । तत् प्र॑तिलो॒मम् । प्र॒ति॒लो॒मं ॅयत् । प्र॒ति॒लो॒ममिति॑ प्रति - लो॒मम् । यत् पौ᳚र्णमा॒सीम् । पौ॒र्ण॒मा॒सीम् पूर्वा᳚म् । पौ॒र्ण॒मा॒सीमिति॑ पौर्ण - मा॒सीम् । पूर्वा॑मा॒लभे॑त । आ॒लभे॑त प्रतिलो॒मम् । आ॒लभे॒तेत्या᳚ - लभे॑त । प्र॒ति॒लो॒ममे॑नौ । प्र॒ति॒लो॒ममिति॑ प्रति - लो॒मम् । ए॒ना॒वा । आ ल॑भेत । ल॒भे॒ता॒मुम् । अ॒मुम॑प॒क्षीय॑माणम् । अ॒प॒क्षीय॑माण॒मनु॑ । अ॒प॒क्षीय॑माण॒मित्य॑प - क्षीय॑माणम् । अन्वप॑ । अप॑ क्षीयेत \newline

\textbf{Jatai Paata} \newline

1. पूर्व॒ आ पूर्वे॒ पूर्व॒ आ । \newline
2. आ ऽल॑भन्ता लभ॒न्ता ऽल॑भन्त । \newline
3. अ॒ल॒भ॒न्त॒ द॒र्॒.श॒पू॒र्ण॒मा॒सौ द॑र्.शपूर्णमा॒सा व॑लभन्ता लभन्त दर्.शपूर्णमा॒सौ । \newline
4. द॒र्॒.श॒पू॒र्ण॒मा॒सा वा॒लभ॑मान आ॒लभ॑मानो दर्.शपूर्णमा॒सौ द॑र्.शपूर्णमा॒सा वा॒लभ॑मानः । \newline
5. द॒र्॒.श॒पू॒र्ण॒मा॒साविति॑ दर्.श - पू॒र्ण॒मा॒सौ । \newline
6. आ॒लभ॑मान ए॒ता वे॒ता वा॒लभ॑मान आ॒लभ॑मान ए॒तौ । \newline
7. आ॒लभ॑मान॒ इत्या᳚ - लभ॑मानः । \newline
8. ए॒तौ होमौ॒ होमा॑ वे॒ता वे॒तौ होमौ᳚ । \newline
9. होमौ॑ पु॒रस्ता᳚त् पु॒रस्ता॒ द्धोमौ॒ होमौ॑ पु॒रस्ता᳚त् । \newline
10. पु॒रस्ता᳚ज् जुहुयाज् जुहुयात् पु॒रस्ता᳚त् पु॒रस्ता᳚ज् जुहुयात् । \newline
11. जु॒हु॒या॒थ् सा॒क्षाथ् सा॒क्षाज् जु॑हुयाज् जुहुयाथ् सा॒क्षात् । \newline
12. सा॒क्षा दे॒वैव सा॒क्षाथ् सा॒क्षा दे॒व । \newline
13. सा॒क्षादिति॑ स - अ॒क्षात् । \newline
14. ए॒व द॑र्.शपूर्णमा॒सौ द॑र्.शपूर्णमा॒सा वे॒वैव द॑र्.शपूर्णमा॒सौ । \newline
15. द॒र्॒.श॒पू॒र्ण॒मा॒सा वा द॑र्.शपूर्णमा॒सौ द॑र्.शपूर्णमा॒सा वा । \newline
16. द॒र्॒.श॒पू॒र्ण॒मा॒साविति॑ दर्.श - पू॒र्ण॒मा॒सौ । \newline
17. आ ल॑भते लभत॒ आ ल॑भते । \newline
18. ल॒भ॒ते॒ ब्र॒ह्म॒वा॒दिनो᳚ ब्रह्मवा॒दिनो॑ लभते लभते ब्रह्मवा॒दिनः॑ । \newline
19. ब्र॒ह्म॒वा॒दिनो॑ वदन्ति वदन्ति ब्रह्मवा॒दिनो᳚ ब्रह्मवा॒दिनो॑ वदन्ति । \newline
20. ब्र॒ह्म॒वा॒दिन॒ इति॑ ब्रह्म - वा॒दिनः॑ । \newline
21. व॒द॒न्ति॒ स स व॑दन्ति वदन्ति॒ सः । \newline
22. स तु तु स स तु । \newline
23. त्वै वै तु त्वै । \newline
24. वै द॑र्.शपूर्णमा॒सौ द॑र्.शपूर्णमा॒सौ वै वै द॑र्.शपूर्णमा॒सौ । \newline
25. द॒र्॒.श॒पू॒र्ण॒मा॒सा वा द॑र्.शपूर्णमा॒सौ द॑र्.शपूर्णमा॒सा वा । \newline
26. द॒र्॒.श॒पू॒र्ण॒मा॒साविति॑ दर्.श - पू॒र्ण॒मा॒सौ । \newline
27. आ ल॑भेत लभे॒ता ल॑भेत । \newline
28. ल॒भे॒त॒ यो यो ल॑भेत लभेत॒ यः । \newline
29. य ए॑नयो रेनयो॒र् यो य ए॑नयोः । \newline
30. ए॒न॒यो॒ र॒नु॒लो॒म म॑नुलो॒म मे॑नयो रेनयो रनुलो॒मम् । \newline
31. अ॒नु॒लो॒मम् च॑ चानुलो॒म म॑नुलो॒मम् च॑ । \newline
32. अ॒नु॒लो॒ममित्य॑नु - लो॒मम् । \newline
33. च॒ प्र॒ति॒लो॒मम् प्र॑तिलो॒मम् च॑ च प्रतिलो॒मम् । \newline
34. प्र॒ति॒लो॒मम् च॑ च प्रतिलो॒मम् प्र॑तिलो॒मम् च॑ । \newline
35. प्र॒ति॒लो॒ममिति॑ प्रति - लो॒मम् । \newline
36. च॒ वि॒द्याद् वि॒द्याच् च॑ च वि॒द्यात् । \newline
37. वि॒द्या दितीति॑ वि॒द्याद् वि॒द्या दिति॑ । \newline
38. इत्य॑मावा॒स्या॑या अमावा॒स्या॑या॒ इती त्य॑मावा॒स्या॑याः । \newline
39. अ॒मा॒वा॒स्या॑या ऊ॒र्द्ध्व मू॒र्द्ध्व म॑मावा॒स्या॑या अमावा॒स्या॑या ऊ॒र्द्ध्वम् । \newline
40. अ॒मा॒वा॒स्या॑या॒ इत्य॑मा - वा॒स्या॑याः । \newline
41. ऊ॒र्द्ध्वम् तत् तदू॒र्द्ध्व मू॒र्द्ध्वम् तत् । \newline
42. तद॑नुलो॒म म॑नुलो॒मम् तत् तद॑नुलो॒मम् । \newline
43. अ॒नु॒लो॒मम् पौ᳚र्णमा॒स्यै पौ᳚र्णमा॒स्या अ॑नुलो॒म म॑नुलो॒मम् पौ᳚र्णमा॒स्यै । \newline
44. अ॒नु॒लो॒ममित्य॑नु - लो॒मम् । \newline
45. पौ॒र्ण॒मा॒स्यै प्र॑ती॒चीन॑म् प्रती॒चीन॑म् पौर्णमा॒स्यै पौ᳚र्णमा॒स्यै प्र॑ती॒चीन᳚म् । \newline
46. पौ॒र्ण॒मा॒स्या इति॑ पौर्ण - मा॒स्यै । \newline
47. प्र॒ती॒चीन॒म् तत् तत् प्र॑ती॒चीन॑म् प्रती॒चीन॒म् तत् । \newline
48. तत् प्र॑तिलो॒मम् प्र॑तिलो॒मम् तत् तत् प्र॑तिलो॒मम् । \newline
49. प्र॒ति॒लो॒मं ॅयद् यत् प्र॑तिलो॒मम् प्र॑तिलो॒मं ॅयत् । \newline
50. प्र॒ति॒लो॒ममिति॑ प्रति - लो॒मम् । \newline
51. यत् पौ᳚र्णमा॒सीम् पौ᳚र्णमा॒सीं ॅयद् यत् पौ᳚र्णमा॒सीम् । \newline
52. पौ॒र्ण॒मा॒सीम् पूर्वा॒म् पूर्वा᳚म् पौर्णमा॒सीम् पौ᳚र्णमा॒सीम् पूर्वा᳚म् । \newline
53. पौ॒र्ण॒मा॒सीमिति॑ पौर्ण - मा॒सीम् । \newline
54. पूर्वा॑ मा॒लभे॑ता॒ लभे॑त॒ पूर्वा॒म् पूर्वा॑ मा॒लभे॑त । \newline
55. आ॒लभे॑त प्रतिलो॒मम् प्र॑तिलो॒म मा॒लभे॑ता॒ लभे॑त प्रतिलो॒मम् । \newline
56. आ॒लभे॒तेत्या᳚ - लभे॑त । \newline
57. प्र॒ति॒लो॒म मे॑ना वेनौ प्रतिलो॒मम् प्र॑तिलो॒म मे॑नौ । \newline
58. प्र॒ति॒लो॒ममिति॑ प्रति - लो॒मम् । \newline
59. ए॒ना॒ वैना॑ वेना॒ वा । \newline
60. आ ल॑भेत लभे॒ता ल॑भेत । \newline
61. ल॒भे॒ता॒मु म॒मुम् ॅल॑भेत लभेता॒मुम् । \newline
62. अ॒मु म॑प॒क्षीय॑माण मप॒क्षीय॑माण म॒मु म॒मु म॑प॒क्षीय॑माणम् । \newline
63. अ॒प॒क्षीय॑माण॒ मन्वन् व॑प॒क्षीय॑माण मप॒क्षीय॑माण॒ मनु॑ । \newline
64. अ॒प॒क्षीय॑माण॒मित्य॑प - क्षीय॑माणम् । \newline
65. अन्व पापान् वन्वप॑ । \newline
66. अप॑ क्षीयेत क्षीये॒ता पाप॑ क्षीयेत । \newline

\textbf{Ghana Paata } \newline

1. पूर्व॒ आ पूर्वे॒ पूर्व॒ आ ऽल॑भन्ता लभ॒न्ता पूर्वे॒ पूर्व॒ आ ऽल॑भन्त । \newline
2. आ ऽल॑भन्ता लभ॒न्ता ऽल॑भन्त दर्.शपूर्णमा॒सौ द॑र्.शपूर्णमा॒सा व॑लभ॒न्ता ऽल॑भन्त दर्.शपूर्णमा॒सौ । \newline
3. अ॒ल॒भ॒न्त॒ द॒र्॒.श॒पू॒र्ण॒मा॒सौ द॑र्.शपूर्णमा॒सा व॑लभन्ता लभन्त दर्.शपूर्णमा॒सा वा॒लभ॑मान आ॒लभ॑मानो दर्.शपूर्णमा॒सा व॑लभन्ता लभन्त दर्.शपूर्णमा॒सा वा॒लभ॑मानः । \newline
4. द॒र्॒.श॒पू॒र्ण॒मा॒सा वा॒लभ॑मान आ॒लभ॑मानो दर्.शपूर्णमा॒सौ द॑र्.शपूर्णमा॒सा वा॒लभ॑मान ए॒ता वे॒ता वा॒लभ॑मानो दर्.शपूर्णमा॒सौ द॑र्.शपूर्णमा॒सा वा॒लभ॑मान ए॒तौ । \newline
5. द॒र्॒.श॒पू॒र्ण॒मा॒साविति॑ दर्.श - पू॒र्ण॒मा॒सौ । \newline
6. आ॒लभ॑मान ए॒ता वे॒ता वा॒लभ॑मान आ॒लभ॑मान ए॒तौ होमौ॒ होमा॑ वे॒ता वा॒लभ॑मान आ॒लभ॑मान ए॒तौ होमौ᳚ । \newline
7. आ॒लभ॑मान॒ इत्या᳚ - लभ॑मानः । \newline
8. ए॒तौ होमौ॒ होमा॑ वे॒ता वे॒तौ होमौ॑ पु॒रस्ता᳚त् पु॒रस्ता॒ द्धोमा॑ वे॒ता वे॒तौ होमौ॑ पु॒रस्ता᳚त् । \newline
9. होमौ॑ पु॒रस्ता᳚त् पु॒रस्ता॒ द्धोमौ॒ होमौ॑ पु॒रस्ता᳚ज् जुहुयाज् जुहुयात् पु॒रस्ता॒ द्धोमौ॒ होमौ॑ पु॒रस्ता᳚ज् जुहुयात् । \newline
10. पु॒रस्ता᳚ज् जुहुयाज् जुहुयात् पु॒रस्ता᳚त् पु॒रस्ता᳚ज् जुहुयाथ् सा॒क्षाथ् सा॒क्षाज् जु॑हुयात् पु॒रस्ता᳚त् पु॒रस्ता᳚ज् जुहुयाथ् सा॒क्षात् । \newline
11. जु॒हु॒या॒थ् सा॒क्षाथ् सा॒क्षाज् जु॑हुयाज् जुहुयाथ् सा॒क्षा दे॒वैव सा॒क्षाज् जु॑हुयाज् जुहुयाथ् सा॒क्षा दे॒व । \newline
12. सा॒क्षा दे॒वैव सा॒क्षाथ् सा॒क्षा दे॒व द॑र्.शपूर्णमा॒सौ द॑र्.शपूर्णमा॒सावे॒व सा॒क्षाथ् सा॒क्षा दे॒व द॑र्.शपूर्णमा॒सौ । \newline
13. सा॒क्षादिति॑ स - अ॒क्षात् । \newline
14. ए॒व द॑र्.शपूर्णमा॒सौ द॑र्.शपूर्णमा॒सा वे॒वैव द॑र्.शपूर्णमा॒सा वा द॑र्.शपूर्णमा॒सा वे॒वैव द॑र्.शपूर्णमा॒सा वा । \newline
15. द॒र्॒.श॒पू॒र्ण॒मा॒सा वा द॑र्.शपूर्णमा॒सौ द॑र्.शपूर्णमा॒सा वा ल॑भते लभत॒ आ द॑र्.शपूर्णमा॒सौ द॑र्.शपूर्णमा॒सा वा ल॑भते । \newline
16. द॒र्॒.श॒पू॒र्ण॒मा॒साविति॑ दर्.श - पू॒र्ण॒मा॒सौ । \newline
17. आ ल॑भते लभत॒ आ ल॑भते ब्रह्मवा॒दिनो᳚ ब्रह्मवा॒दिनो॑ लभत॒ आ ल॑भते ब्रह्मवा॒दिनः॑ । \newline
18. ल॒भ॒ते॒ ब्र॒ह्म॒वा॒दिनो᳚ ब्रह्मवा॒दिनो॑ लभते लभते ब्रह्मवा॒दिनो॑ वदन्ति वदन्ति ब्रह्मवा॒दिनो॑ लभते लभते ब्रह्मवा॒दिनो॑ वदन्ति । \newline
19. ब्र॒ह्म॒वा॒दिनो॑ वदन्ति वदन्ति ब्रह्मवा॒दिनो᳚ ब्रह्मवा॒दिनो॑ वदन्ति॒ स स व॑दन्ति ब्रह्मवा॒दिनो᳚ ब्रह्मवा॒दिनो॑ वदन्ति॒ सः । \newline
20. ब्र॒ह्म॒वा॒दिन॒ इति॑ ब्रह्म - वा॒दिनः॑ । \newline
21. व॒द॒न्ति॒ स स व॑दन्ति वदन्ति॒ स तु तु स व॑दन्ति वदन्ति॒ स तु । \newline
22. स तु तु स स त्वै वै तु स स त्वै । \newline
23. त्वै वै तु त्वै द॑र्.शपूर्णमा॒सौ द॑र्.शपूर्णमा॒सौ वै तु त्वै द॑र्.शपूर्णमा॒सौ । \newline
24. वै द॑र्.शपूर्णमा॒सौ द॑र्.शपूर्णमा॒सौ वै वै द॑र्.शपूर्णमा॒सा वा द॑र्.शपूर्णमा॒सौ वै वै द॑र्.शपूर्णमा॒सा वा । \newline
25. द॒र्॒.श॒पू॒र्ण॒मा॒सा वा द॑र्.शपूर्णमा॒सौ द॑र्.शपूर्णमा॒सा वा ल॑भेत लभे॒ता द॑र्.शपूर्णमा॒सौ द॑र्.शपूर्णमा॒सा वा ल॑भेत । \newline
26. द॒र्॒.श॒पू॒र्ण॒मा॒साविति॑ दर्.श - पू॒र्ण॒मा॒सौ । \newline
27. आ ल॑भते लभे॒ता ल॑भेत॒ यो यो ल॑भे॒ता ल॑भेत॒ यः । \newline
28. ल॒भे॒त॒ यो यो ल॑भेत लभेत॒ य ए॑नयो रेनयो॒र् यो ल॑भेत लभेत॒ य ए॑नयोः । \newline
29. य ए॑नयो रेनयो॒र् यो य ए॑नयो रनुलो॒म म॑नुलो॒म मे॑नयो॒र् यो य ए॑नयो रनुलो॒मम् । \newline
30. ए॒न॒यो॒ र॒नु॒लो॒म म॑नुलो॒म मे॑नयो रेनयो रनुलो॒मम् च॑ चानुलो॒म मे॑नयो रेनयो रनुलो॒मम् च॑ । \newline
31. अ॒नु॒लो॒मम् च॑ चानुलो॒म म॑नुलो॒मम् च॑ प्रतिलो॒मम् प्र॑तिलो॒मम् चा॑नुलो॒म म॑नुलो॒मम् च॑ प्रतिलो॒मम् । \newline
32. अ॒नु॒लो॒ममित्य॑नु - लो॒मम् । \newline
33. च॒ प्र॒ति॒लो॒मम् प्र॑तिलो॒मम् च॑ च प्रतिलो॒मम् च॑ च प्रतिलो॒मम् च॑ च प्रतिलो॒मम् च॑ । \newline
34. प्र॒ति॒लो॒मम् च॑ च प्रतिलो॒मम् प्र॑तिलो॒मम् च॑ वि॒द्याद् वि॒द्याच् च॑ प्रतिलो॒मम् प्र॑तिलो॒मम् च॑ वि॒द्यात् । \newline
35. प्र॒ति॒लो॒ममिति॑ प्रति - लो॒मम् । \newline
36. च॒ वि॒द्याद् वि॒द्याच् च॑ च वि॒द्या दितीति॑ वि॒द्याच् च॑ च वि॒द्या दिति॑ । \newline
37. वि॒द्या दितीति॑ वि॒द्याद् वि॒द्यादि त्य॑मावा॒स्या॑या अमावा॒स्या॑या॒ इति॑ वि॒द्याद् वि॒द्यादि त्य॑मावा॒स्या॑याः । \newline
38. इत्य॑मावा॒स्या॑या अमावा॒स्या॑या॒ इती त्य॑मावा॒स्या॑या ऊ॒र्द्ध्व मू॒र्द्ध्व म॑मावा॒स्या॑या॒ इती त्य॑मावा॒स्या॑या ऊ॒र्द्ध्वम् । \newline
39. अ॒मा॒वा॒स्या॑या ऊ॒र्द्ध्व मू॒र्द्ध्व म॑मावा॒स्या॑या अमावा॒स्या॑या ऊ॒र्द्ध्वम् तत् तदू॒र्द्ध्व म॑मावा॒स्या॑या अमावा॒स्या॑या ऊ॒र्द्ध्वम् तत् । \newline
40. अ॒मा॒वा॒स्या॑या॒ इत्य॑मा - वा॒स्या॑याः । \newline
41. ऊ॒र्द्ध्वम् तत् तदू॒र्द्ध्व मू॒र्द्ध्वम् तद॑नुलो॒म म॑नुलो॒मम् तदू॒र्द्ध्व मू॒र्द्ध्वम् तद॑नुलो॒मम् । \newline
42. तद॑नुलो॒म म॑नुलो॒मम् तत् तद॑नुलो॒मम् पौ᳚र्णमा॒स्यै पौ᳚र्णमा॒स्या अ॑नुलो॒मम् तत् तद॑नुलो॒मम् 
पौ᳚र्णमा॒स्यै । \newline
43. अ॒नु॒लो॒मम् पौ᳚र्णमा॒स्यै पौ᳚र्णमा॒स्या अ॑नुलो॒म म॑नुलो॒मम् पौ᳚र्णमा॒स्यै प्र॑ती॒चीन॑म् प्रती॒चीन॑म् पौर्णमा॒स्या अ॑नुलो॒म म॑नुलो॒मम् पौ᳚र्णमा॒स्यै प्र॑ती॒चीन᳚म् । \newline
44. अ॒नु॒लो॒ममित्य॑नु - लो॒मम् । \newline
45. पौ॒र्ण॒मा॒स्यै प्र॑ती॒चीन॑म् प्रती॒चीन॑म् पौर्णमा॒स्यै पौ᳚र्णमा॒स्यै प्र॑ती॒चीन॒म् तत् तत् प्र॑ती॒चीन॑म् पौर्णमा॒स्यै पौ᳚र्णमा॒स्यै प्र॑ती॒चीन॒म् तत् । \newline
46. पौ॒र्ण॒मा॒स्या इति॑ पौर्ण - मा॒स्यै । \newline
47. प्र॒ती॒चीन॒म् तत् तत् प्र॑ती॒चीन॑म् प्रती॒चीन॒म् तत् प्र॑तिलो॒मम् प्र॑तिलो॒मम् तत् प्र॑ती॒चीन॑म् प्रती॒चीन॒म् तत् प्र॑तिलो॒मम् । \newline
48. तत् प्र॑तिलो॒मम् प्र॑तिलो॒मम् तत् तत् प्र॑तिलो॒मं ॅयद् यत् प्र॑तिलो॒मम् तत् तत् प्र॑तिलो॒मं ॅयत् । \newline
49. प्र॒ति॒लो॒मं ॅयद् यत् प्र॑तिलो॒मम् प्र॑तिलो॒मं ॅयत् पौ᳚र्णमा॒सीम् पौ᳚र्णमा॒सीं ॅयत् प्र॑तिलो॒मम् प्र॑तिलो॒मं ॅयत् पौ᳚र्णमा॒सीम् । \newline
50. प्र॒ति॒लो॒ममिति॑ प्रति - लो॒मम् । \newline
51. यत् पौ᳚र्णमा॒सीम् पौ᳚र्णमा॒सीं ॅयद् यत् पौ᳚र्णमा॒सीम् पूर्वा॒म् पूर्वा᳚म् पौर्णमा॒सीं ॅयद् यत् पौ᳚र्णमा॒सीम् पूर्वा᳚म् । \newline
52. पौ॒र्ण॒मा॒सीम् पूर्वा॒म् पूर्वा᳚म् पौर्णमा॒सीम् पौ᳚र्णमा॒सीम् पूर्वा॑ मा॒लभे॑ता॒ लभे॑त॒ पूर्वा᳚म् पौर्णमा॒सीम् पौ᳚र्णमा॒सीम् पूर्वा॑ मा॒लभे॑त । \newline
53. पौ॒र्ण॒मा॒सीमिति॑ पौर्ण - मा॒सीम् । \newline
54. पूर्वा॑ मा॒लभे॑ता॒ लभे॑त॒ पूर्वा॒म् पूर्वा॑ मा॒लभे॑त प्रतिलो॒मम् प्र॑तिलो॒म मा॒लभे॑त॒ पूर्वा॒म् पूर्वा॑ मा॒लभे॑त प्रतिलो॒मम् । \newline
55. आ॒लभे॑त प्रतिलो॒मम् प्र॑तिलो॒म मा॒लभे॑ता॒ लभे॑त प्रतिलो॒म मे॑ना वेनौ प्रतिलो॒म मा॒लभे॑ता॒ लभे॑त प्रतिलो॒म मे॑नौ । \newline
56. आ॒लभे॒तेत्या᳚ - लभे॑त । \newline
57. प्र॒ति॒लो॒म मे॑ना वेनौ प्रतिलो॒मम् प्र॑तिलो॒म मे॑ना॒ वैनौ᳚ प्रतिलो॒मम् प्र॑तिलो॒म मे॑ना॒वा । \newline
58. प्र॒ति॒लो॒ममिति॑ प्रति - लो॒मम् । \newline
59. ए॒ना॒ वैना॑ वेना॒ वा ल॑भेत लभे॒तैना॑ वेना॒ वा ल॑भेत । \newline
60. आ ल॑भेत लभे॒ता ल॑भेता॒मु म॒मुम् ॅल॑भे॒ता ल॑भेता॒मुम् । \newline
61. ल॒भे॒ता॒मु म॒मुम् ॅल॑भेत लभेता॒मु म॑प॒क्षीय॑माण मप॒क्षीय॑माण म॒मुम् ॅल॑भेत 
लभेता॒मु म॑प॒क्षीय॑माणम् । \newline
62. अ॒मु म॑प॒क्षीय॑माण मप॒क्षीय॑माण म॒मु म॒मु म॑प॒क्षीय॑माण॒ मन्वन्व॑प॒क्षीय॑माण म॒मु म॒मु म॑प॒क्षीय॑माण॒ मनु॑ । \newline
63. अ॒प॒क्षीय॑माण॒ मन्वन्व॑प॒क्षीय॑माण मप॒क्षीय॑माण॒ मन्व पापा न्व॑प॒क्षीय॑माण मप॒क्षीय॑माण॒ मन्वप॑ । \newline
64. अ॒प॒क्षीय॑माण॒मित्य॑प - क्षीय॑माणम् । \newline
65. अन्व पापा न्वन्वप॑ क्षीयेत क्षीये॒ता पान्वन्वप॑ क्षीयेत । \newline
66. अप॑ क्षीयेत क्षीये॒ता पाप॑ क्षीयेत सारस्व॒तौ सा॑रस्व॒तौ क्षी॑ये॒ता पाप॑ क्षीयेत सारस्व॒तौ । \newline
\pagebreak
\markright{ TS 3.5.1.4  \hfill https://www.vedavms.in \hfill}

\section{ TS 3.5.1.4 }

\textbf{TS 3.5.1.4 } \newline
\textbf{Samhita Paata} \newline

क्षीयेत सारस्व॒तौ होमौ॑ पु॒रस्ता᳚ज्जुहुयादमावा॒स्या॑ वै सर॑स्वत्यनुलो॒म-मे॒वैना॒वा ल॑भते॒ ऽमुमा॒प्याय॑मान॒मन्वा प्या॑यत आग्नावैष्ण॒व-मेका॑दशकपालं पु॒रस्ता॒न्निव॑र्पे॒थ् सर॑स्वत्यै च॒रुꣳ सर॑स्वते॒ द्वाद॑शकपालं॒ ॅयदा᳚ग्ने॒यो भव॑त्य॒ग्निर्वै य॑ज्ञ्मु॒खं ॅय॑ज्ञ्मु॒खमे॒वर्द्धिं॑ पु॒रस्ता᳚द्-धत्ते॒ यद्-वै᳚ष्ण॒वो भव॑ति य॒ज्ञो वै विष्णु॑र्य॒ज्ञ्मे॒वाऽऽ*रभ्य॒ प्रत॑नुते॒ सर॑स्वत्यै ( ) च॒रुर्भ॑वति॒ सर॑स्वते॒ द्वाद॑शकपालोऽमावा॒स्या॑ वै सर॑स्वती पू॒र्णमा॑सः॒ सर॑स्वा॒न् तावे॒व सा॒क्षादा र॑भत ऋ॒द्ध्नोत्या᳚भ्यां॒ द्वाद॑शकपालः॒ सर॑स्वते भवति मिथुन॒त्वाय॒ प्रजा᳚त्यै मिथु॒नौ गावौ॒ दक्षि॑णा॒ समृ॑द्ध्यै ॥ \newline

\textbf{Pada Paata} \newline

क्षी॒ये॒ते॒ । सा॒र॒स्व॒तौ । होमौ᳚ । पु॒रस्ता᳚त् । जु॒हु॒या॒त् । अ॒मा॒वा॒स्येत्य॑मा-वा॒स्या᳚ । वै । सर॑स्वती । अ॒नु॒लो॒ममित्य॑नु-लो॒मम् । ए॒व । ए॒नौ॒ । एति॑ । ल॒भे॒ते॒ । अ॒मुम् । आ॒प्याय॑मान॒मित्या᳚ - प्याय॑मानम् । अनु॑ । एति॑ । प्या॒य॒ते॒ । आ॒ग्ना॒वै॒ष्ण॒वमित्या᳚ग्ना - वै॒ष्ण॒वम् । एका॑दशकपाल॒मित्येका॑दश-क॒पा॒ल॒म् । पु॒रस्ता᳚त् । निरिति॑ । व॒पे॒त् । सर॑स्वत्यै । च॒रुम् । सर॑स्वते । द्वाद॑शकपाल॒मिति॒ द्वाद॑श-क॒पा॒ल॒म् । यत् । आ॒ग्ने॒यः । भव॑ति । अ॒ग्निः । वै । य॒ज्ञ्॒मु॒खमिति॑ यज्ञ्-मु॒खम् । य॒ज्ञ्॒मु॒खमिति॑ यज्ञ्-मु॒खम् । ए॒व । ऋद्धि᳚म् । पु॒रस्ता᳚त् । ध॒त्ते॒ । यत् । वै॒ष्ण॒वः । भव॑ति । य॒ज्ञ्ः । वै । विष्णुः॑ । य॒ज्ञ्म् । ए॒व । आ॒रभ्येत्या᳚ - रभ्य॑ । प्रेति॑ । त॒नु॒ते॒ । सर॑स्वत्यै ( ) । च॒रुः । भ॒व॒ति॒ । सर॑स्वते । द्वाद॑शकपाल॒ इति॒ द्वाद॑श-क॒पा॒लः॒ । अ॒मा॒वा॒स्येत्य॑मा - वा॒स्या᳚ । वै । सर॑स्वती । पू॒र्णमा॑स॒ इति॑ पू॒र्ण - मा॒सः॒ । सर॑स्वान् । तौ । ए॒व । सा॒क्षादिति॑ स - अ॒क्षात् । एति॑ । र॒भ॒ते॒ । ऋ॒द्ध्नोति॑ । आ॒भ्या॒म् । द्वाद॑शकपाल॒ इति॒ द्वाद॑श - क॒पा॒लः॒ । सर॑स्वते । भ॒व॒ति॒ । मि॒थु॒न॒त्वायेति॑ मिथुन - त्वाय॑ । प्रजा᳚त्या॒ इति॒ प्र - जा॒त्यै॒ । मि॒थु॒नौ । गावौ᳚ । दक्षि॑णा । समृ॑द्ध्या॒ इति॒ सं - ऋ॒द्ध्यै॒ ॥  \newline


\textbf{Krama Paata} \newline

क्षी॒ये॒त॒ सा॒र॒स्व॒तौ । सा॒र॒स्व॒तौ होमौ᳚ । होमौ॑ पु॒रस्ता᳚त् । पु॒रस्ता᳚ज् जुहुयात् । जु॒हु॒या॒द॒मा॒वा॒स्या᳚ । अ॒मा॒वा॒स्या॑ वै । अ॒मा॒वा॒स्येत्य॑मा - वा॒स्या᳚ । वै सर॑स्वती । सर॑स्वत्यनुलो॒मम् । अ॒नु॒लो॒ममे॒व । अ॒नु॒लो॒ममित्य॑नु - लो॒मम् । ए॒वैनौ᳚ । ए॒ना॒वा । आ ल॑भते । ल॒भ॒ते॒ ऽमुम् । अ॒मुमा॒प्याय॑मानम् । आ॒प्याय॑मान॒मनु॑ । आ॒प्याय॑मान॒मित्या᳚ - प्याय॑मानम् । अन्वा । आ प्या॑यते । प्या॒य॒त॒ आ॒ग्ना॒वै॒ष्ण॒वम् । आ॒ग्ना॒वै॒ष्ण॒वमेका॑दशकपालम् । आ॒ग्ना॒वै॒ष्ण॒वमित्या᳚ग्ना - वै॒ष्ण॒वम् । एका॑दशकपालम् पु॒रस्ता᳚त् । एका॑दशकपा॒लमित्येका॑दश - क॒पा॒ल॒म् । पु॒रस्ता॒न् निः । निर् व॑पेत् । व॒पे॒थ् सर॑स्वत्यै । सर॑स्वत्यै च॒रुम् । च॒रुꣳ सर॑स्वते । सर॑स्वते॒ द्वाद॑शकपालम् । द्वाद॑शकपालं॒ ॅयत् । द्वाद॑शकपाल॒मिति॒ द्वाद॑श - क॒पा॒ल॒म् । यदा᳚ग्ने॒यः । आ॒ग्ने॒यो भव॑ति । भव॑त्य॒ग्निः । अ॒ग्निर् वै । वै य॑ज्ञ्मु॒खम् । य॒ज्ञ्॒मु॒खं ॅय॑ज्ञ्मु॒खम् । य॒ज्ञ्॒मु॒खमिति॑ यज्ञ् - मु॒खम् । य॒ज्ञ्॒मु॒खमे॒व । य॒ज्ञ्॒मु॒खमिति॑ यज्ञ् - मु॒खम् । ए॒वर्द्धि᳚म् । ऋद्धि॑म् पु॒रस्ता᳚त् । पु॒रस्ता᳚द् धत्ते । ध॒त्ते॒ यत् । यद् वै᳚ष्ण॒वः । वै॒ष्ण॒वो भव॑ति । भव॑ति य॒ज्ञ्ः । य॒ज्ञो वै । वै विष्णुः॑ । विष्णु॑र् य॒ज्ञ्म् । य॒ज्ञ्मे॒व । ए॒वारभ्य॑ । आ॒रभ्य॒ प्र । आ॒रभ्येत्या᳚ - रभ्य॑ । प्र त॑नुते । त॒नु॒ते॒ सर॑स्वत्यै ( ) । सर॑स्वत्यै च॒रुः । च॒रुर् भ॑वति । भ॒व॒ति॒ सर॑स्वते । सर॑स्वते॒ द्वाद॑शकपालः । द्वाद॑शकपालो ऽमावा॒स्या᳚ । द्वाद॑शकपाल॒ इति॒ द्वाद॑श - क॒पा॒लः॒ । अ॒मा॒वा॒स्या॑ वै । अ॒मा॒वा॒स्येत्य॑मा - वा॒स्या᳚ । वै सर॑स्वती । सर॑स्वती पू॒र्णमा॑सः । पू॒र्णमा॑सः॒ सर॑स्वान् । पू॒र्णमा॑स॒ इति॑ पू॒र्ण - मा॒सः॒ । सर॑स्वा॒न् तौ । तावे॒व । ए॒व सा॒क्षात् । सा॒क्षादा । सा॒क्षादिति॑ स - अ॒क्षात् । आ र॑भते । र॒भ॒त॒ ऋ॒ध्नोति॑ । ऋ॒ध्नोत्या᳚भ्याम् । आ॒भ्यां॒ द्वाद॑शकपालः । द्वाद॑शकपालः॒ सर॑स्वते । द्वाद॑शकपाल॒ इति॒ द्वाद॑श - क॒पा॒लः॒ । सर॑स्वते भवति । भ॒व॒ति॒ मि॒थु॒न॒त्वाय॑ । मि॒थु॒न॒त्वाय॒ प्रजा᳚त्यै । मि॒थु॒न॒त्वायेति॑ मिथुन - त्वाय॑ । प्रजा᳚त्यै मिथु॒नौ । प्रजा᳚त्या॒ इति॒ प्र - जा॒त्यै॒ । मि॒थु॒नौ गावौ᳚ । गावौ॒ दक्षि॑णा । दक्षि॑णा॒ समृ॑द्ध्यै । समृ॑द्ध्या॒ इति॒ सं - ऋ॒द्ध्यै॒ । \newline

\textbf{Jatai Paata} \newline

1. क्षी॒ये॒त॒ सा॒र॒स्व॒तौ सा॑रस्व॒तौ क्षी॑येत क्षीयेत सारस्व॒तौ । \newline
2. सा॒र॒स्व॒तौ होमौ॒ होमौ॑ सारस्व॒तौ सा॑रस्व॒तौ होमौ᳚ । \newline
3. होमौ॑ पु॒रस्ता᳚त् पु॒रस्ता॒ द्धोमौ॒ होमौ॑ पु॒रस्ता᳚त् । \newline
4. पु॒रस्ता᳚ज् जुहुयाज् जुहुयात् पु॒रस्ता᳚त् पु॒रस्ता᳚ज् जुहुयात् । \newline
5. जु॒हु॒या॒ द॒मा॒वा॒स्या॑ ऽमावा॒स्या॑ जुहुयाज् जुहुया दमावा॒स्या᳚ । \newline
6. अ॒मा॒वा॒स्या॑ वै वा अ॑मावा॒स्या॑ ऽमावा॒स्या॑ वै । \newline
7. अ॒मा॒वा॒स्येत्य॑मा - वा॒स्या᳚ । \newline
8. वै सर॑स्वती॒ सर॑स्वती॒ वै वै सर॑स्वती । \newline
9. सर॑स्व त्यनुलो॒म म॑नुलो॒मꣳ सर॑स्वती॒ सर॑स्व त्यनुलो॒मम् । \newline
10. अ॒नु॒लो॒म मे॒वैवानु॑लो॒म म॑नुलो॒म मे॒व । \newline
11. अ॒नु॒लो॒ममित्य॑नु - लो॒मम् । \newline
12. ए॒वैना॑ वेना वे॒वै वैनौ᳚ । \newline
13. ए॒ना॒ वैना॑ वेना॒ वा । \newline
14. आ ल॑भते लभत॒ आ ल॑भते । \newline
15. ल॒भ॒ते॒ ऽमु म॒मुम् ॅल॑भते लभते॒ ऽमुम् । \newline
16. अ॒मु मा॒प्याय॑मान मा॒प्याय॑मान म॒मु म॒मु मा॒प्याय॑मानम् । \newline
17. आ॒प्याय॑मान॒ मन्वन् वा॒प्याय॑मान मा॒प्याय॑मान॒ मनु॑ । \newline
18. आ॒प्याय॑मान॒मित्या᳚ - प्याय॑मानम् । \newline
19. अन्वा ऽन्वन्वा । \newline
20. आ प्या॑यते प्यायत॒ आ प्या॑यते । \newline
21. प्या॒य॒त॒ आ॒ग्ना॒वै॒ष्ण॒व मा᳚ग्नावैष्ण॒वम् प्या॑यते प्यायत आग्नावैष्ण॒वम् । \newline
22. आ॒ग्ना॒वै॒ष्ण॒व मेका॑दशकपाल॒ मेका॑दशकपाल माग्नावैष्ण॒व मा᳚ग्नावैष्ण॒व मेका॑दशकपालम् । \newline
23. आ॒ग्ना॒वै॒ष्ण॒वमित्या᳚ग्ना - वै॒ष्ण॒वम् । \newline
24. एका॑दशकपालम् पु॒रस्ता᳚त् पु॒रस्ता॒ देका॑दशकपाल॒ मेका॑दशकपालम् पु॒रस्ता᳚त् । \newline
25. एका॑दशकपाल॒मित्येका॑दश - क॒पा॒ल॒म् । \newline
26. पु॒रस्ता॒न् निर् णिष् पु॒रस्ता᳚त् पु॒रस्ता॒न् निः । \newline
27. निर् व॑पेद् वपे॒न् निर् णिर् व॑पेत् । \newline
28. व॒पे॒थ् सर॑स्वत्यै॒ सर॑स्वत्यै वपेद् वपे॒थ् सर॑स्वत्यै । \newline
29. सर॑स्वत्यै च॒रुम् च॒रुꣳ सर॑स्वत्यै॒ सर॑स्वत्यै च॒रुम् । \newline
30. च॒रुꣳ सर॑स्वते॒ सर॑स्वते च॒रुम् च॒रुꣳ सर॑स्वते । \newline
31. सर॑स्वते॒ द्वाद॑शकपाल॒म् द्वाद॑शकपालꣳ॒॒ सर॑स्वते॒ सर॑स्वते॒ द्वाद॑शकपालम् । \newline
32. द्वाद॑शकपालं॒ ॅयद् यद् द्वाद॑शकपाल॒म् द्वाद॑शकपालं॒ ॅयत् । \newline
33. द्वाद॑शकपाल॒मिति॒ द्वाद॑श - क॒पा॒ल॒म् । \newline
34. यदा᳚ग्ने॒य आ᳚ग्ने॒यो यद् यदा᳚ग्ने॒यः । \newline
35. आ॒ग्ने॒यो भव॑ति॒ भव॑ त्याग्ने॒य आ᳚ग्ने॒यो भव॑ति । \newline
36. भव॑ त्य॒ग्नि र॒ग्निर् भव॑ति॒ भव॑ त्य॒ग्निः । \newline
37. अ॒ग्निर् वै वा अ॒ग्नि र॒ग्निर् वै । \newline
38. वै य॑ज्ञ्मु॒खं ॅय॑ज्ञ्मु॒खं ॅवै वै य॑ज्ञ्मु॒खम् । \newline
39. य॒ज्ञ्॒मु॒खं ॅय॑ज्ञ्मु॒खम् । \newline
40. य॒ज्ञ्॒मु॒खमिति॑ यज्ञ् - मु॒खम् । \newline
41. य॒ज्ञ्॒मु॒ख मे॒वैव य॑ज्ञ्मु॒खं ॅय॑ज्ञ्मु॒ख मे॒व । \newline
42. य॒ज्ञ्॒मु॒खमिति॑ यज्ञ् - मु॒खम् । \newline
43. ए॒व र्‌द्धि॒ मृद्धि॑ मे॒वैव र्‌द्धि᳚म् । \newline
44. ऋद्धि॑म् पु॒रस्ता᳚त् पु॒रस्ता॒ दृद्धि॒ मृद्धि॑म् पु॒रस्ता᳚त् । \newline
45. पु॒रस्ता᳚द् धत्ते धत्ते पु॒रस्ता᳚त् पु॒रस्ता᳚द् धत्ते । \newline
46. ध॒त्ते॒ यद् यद् ध॑त्ते धत्ते॒ यत् । \newline
47. यद् वै᳚ष्ण॒वो वै᳚ष्ण॒वो यद् यद् वै᳚ष्ण॒वः । \newline
48. वै॒ष्ण॒वो भव॑ति॒ भव॑ति वैष्ण॒वो वै᳚ष्ण॒वो भव॑ति । \newline
49. भव॑ति य॒ज्ञो य॒ज्ञो भव॑ति॒ भव॑ति य॒ज्ञ्ः । \newline
50. य॒ज्ञो वै वै य॒ज्ञो य॒ज्ञो वै । \newline
51. वै विष्णु॒र् विष्णु॒र् वै वै विष्णुः॑ । \newline
52. विष्णु॑र् य॒ज्ञ्ं ॅय॒ज्ञ्ं ॅविष्णु॒र् विष्णु॑र् य॒ज्ञ्म् । \newline
53. य॒ज्ञ् मे॒वैव य॒ज्ञ्ं ॅय॒ज्ञ् मे॒व । \newline
54. ए॒वारभ्या॒ रभ्यै॒ वैवारभ्य॑ । \newline
55. आ॒रभ्य॒ प्र प्रारभ्या॒ रभ्य॒ प्र । \newline
56. आ॒रभ्येत्या᳚ - रभ्य॑ । \newline
57. प्र त॑नुते तनुते॒ प्र प्र त॑नुते । \newline
58. त॒नु॒ते॒ सर॑स्वत्यै॒ सर॑स्वत्यै तनुते तनुते॒ सर॑स्वत्यै । \newline
59. सर॑स्वत्यै च॒रु श्च॒रुः सर॑स्वत्यै॒ सर॑स्वत्यै च॒रुः । \newline
60. च॒रुर् भ॑वति भवति च॒रु श्च॒रुर् भ॑वति । \newline
61. भ॒व॒ति॒ सर॑स्वते॒ सर॑स्वते भवति भवति॒ सर॑स्वते । \newline
62. सर॑स्वते॒ द्वाद॑शकपालो॒ द्वाद॑शकपालः॒ सर॑स्वते॒ सर॑स्वते॒ द्वाद॑शकपालः । \newline
63. द्वाद॑शकपालो ऽमावा॒स्या॑ ऽमावा॒स्या᳚ द्वाद॑शकपालो॒ द्वाद॑शकपालो ऽमावा॒स्या᳚ । \newline
64. द्वाद॑शकपाल॒ इति॒ द्वाद॑श - क॒पा॒लः॒ । \newline
65. अ॒मा॒वा॒स्या॑ वै वा अ॑मावा॒स्या॑ ऽमावा॒स्या॑ वै । \newline
66. अ॒मा॒वा॒स्येत्य॑मा - वा॒स्या᳚ । \newline
67. वै सर॑स्वती॒ सर॑स्वती॒ वै वै सर॑स्वती । \newline
68. सर॑स्वती पू॒र्णमा॑सः पू॒र्णमा॑सः॒ सर॑स्वती॒ सर॑स्वती पू॒र्णमा॑सः । \newline
69. पू॒र्णमा॑सः॒ सर॑स्वा॒न् थ्सर॑स्वान् पू॒र्णमा॑सः पू॒र्णमा॑सः॒ सर॑स्वान् । \newline
70. पू॒र्णमा॑स॒ इति॑ पू॒र्ण - मा॒सः॒ । \newline
71. सर॑स्वा॒न् तौ तौ सर॑स्वा॒न् थ्सर॑स्वा॒न् तौ । \newline
72. ता वे॒वैव तौ ता वे॒व । \newline
73. ए॒व सा॒क्षाथ् सा॒क्षा दे॒वैव सा॒क्षात् । \newline
74. सा॒क्षादा सा॒क्षाथ् सा॒क्षादा । \newline
75. सा॒क्षादिति॑ स - अ॒क्षात् । \newline
76. आ र॑भते रभत॒ आ र॑भते । \newline
77. र॒भ॒त॒ ऋ॒द्ध्नो त्यृ॒द्ध्नोति॑ रभते रभत ऋ॒द्ध्नोति॑ । \newline
78. ऋ॒द्ध्नो त्या᳚भ्या माभ्या मृ॒द्ध्नो त्यृ॒द्ध्नो त्या᳚भ्याम् । \newline
79. आ॒भ्या॒म् द्वाद॑शकपालो॒ द्वाद॑शकपाल आभ्या माभ्या॒म् द्वाद॑शकपालः । \newline
80. द्वाद॑शकपालः॒ सर॑स्वते॒ सर॑स्वते॒ द्वाद॑शकपालो॒ द्वाद॑शकपालः॒ सर॑स्वते । \newline
81. द्वाद॑शकपाल॒ इति॒ द्वाद॑श - क॒पा॒लः॒ । \newline
82. सर॑स्वते भवति भवति॒ सर॑स्वते॒ सर॑स्वते भवति । \newline
83. भ॒व॒ति॒ मि॒थु॒न॒त्वाय॑ मिथुन॒त्वाय॑ भवति भवति मिथुन॒त्वाय॑ । \newline
84. मि॒थु॒न॒त्वाय॒ प्रजा᳚त्यै॒ प्रजा᳚त्यै मिथुन॒त्वाय॑ मिथुन॒त्वाय॒ प्रजा᳚त्यै । \newline
85. मि॒थु॒न॒त्वायेति॑ मिथुन - त्वाय॑ । \newline
86. प्रजा᳚त्यै मिथु॒नौ मि॑थु॒नौ प्रजा᳚त्यै॒ प्रजा᳚त्यै मिथु॒नौ । \newline
87. प्रजा᳚त्या॒ इति॒ प्र - जा॒त्यै॒ । \newline
88. मि॒थु॒नौ गावौ॒ गावौ॑ मिथु॒नौ मि॑थु॒नौ गावौ᳚ । \newline
89. गावौ॒ दक्षि॑णा॒ दक्षि॑णा॒ गावौ॒ गावौ॒ दक्षि॑णा । \newline
90. दक्षि॑णा॒ समृ॑द्ध्यै॒ समृ॑द्ध्यै॒ दक्षि॑णा॒ दक्षि॑णा॒ समृ॑द्ध्यै । \newline
91. समृ॑द्ध्या॒ इति॒ सं - ऋ॒द्ध्यै॒ । \newline

\textbf{Ghana Paata } \newline

1. क्षी॒ये॒त॒ सा॒र॒स्व॒तौ सा॑रस्व॒तौ क्षी॑येत क्षीयेत सारस्व॒तौ होमौ॒ होमौ॑ सारस्व॒तौ क्षी॑येत क्षीयेत सारस्व॒तौ होमौ᳚ । \newline
2. सा॒र॒स्व॒तौ होमौ॒ होमौ॑ सारस्व॒तौ सा॑रस्व॒तौ होमौ॑ पु॒रस्ता᳚त् पु॒रस्ता॒ द्धोमौ॑ सारस्व॒तौ सा॑रस्व॒तौ होमौ॑ पु॒रस्ता᳚त् । \newline
3. होमौ॑ पु॒रस्ता᳚त् पु॒रस्ता॒ द्धोमौ॒ होमौ॑ पु॒रस्ता᳚ज् जुहुयाज् जुहुयात् पु॒रस्ता॒ द्धोमौ॒ होमौ॑ पु॒रस्ता᳚ज् जुहुयात् । \newline
4. पु॒रस्ता᳚ज् जुहुयाज् जुहुयात् पु॒रस्ता᳚त् पु॒रस्ता᳚ज् जुहुया दमावा॒स्या॑ ऽमावा॒स्या॑ जुहुयात् पु॒रस्ता᳚त् पु॒रस्ता᳚ज् जुहुया दमावा॒स्या᳚ । \newline
5. जु॒हु॒या॒ द॒मा॒वा॒स्या॑ ऽमावा॒स्या॑ जुहुयाज् जुहुया दमावा॒स्या॑ वै वा अ॑मावा॒स्या॑ जुहुयाज् जुहुयादमावा॒स्या॑ वै । \newline
6. अ॒मा॒वा॒स्या॑ वै वा अ॑मावा॒स्या॑ ऽमावा॒स्या॑ वै सर॑स्वती॒ सर॑स्वती॒ वा अ॑मावा॒स्या॑ ऽमावा॒स्या॑ वै सर॑स्वती । \newline
7. अ॒मा॒वा॒स्येत्य॑मा - वा॒स्या᳚ । \newline
8. वै सर॑स्वती॒ सर॑स्वती॒ वै वै सर॑स्व त्यनुलो॒म म॑नुलो॒मꣳ सर॑स्वती॒ वै वै सर॑स्व त्यनुलो॒मम् । \newline
9. सर॑स्व त्यनुलो॒म म॑नुलो॒मꣳ सर॑स्वती॒ सर॑स्व त्यनुलो॒म मे॒वैवा नु॑लो॒मꣳ सर॑स्वती॒ सर॑स्व त्यनुलो॒म मे॒व । \newline
10. अ॒नु॒लो॒म मे॒वैवा नु॑लो॒म म॑नुलो॒म मे॒वैना॑ वेना वे॒वा नु॑लो॒म म॑नुलो॒म मे॒वैनौ᳚ । \newline
11. अ॒नु॒लो॒ममित्य॑नु - लो॒मम् । \newline
12. ए॒वैना॑ वेना वे॒वैवैना॒ वैना॑ वे॒वैवैना॒ वा । \newline
13. ए॒ना॒ वैना॑ वेना॒ वा ल॑भते लभत॒ ऐना॑ वेना॒ वा ल॑भते । \newline
14. आ ल॑भते लभत॒ आ ल॑भते॒ ऽमु म॒मुम् ॅल॑भत॒ आ ल॑भते॒ ऽमुम् । \newline
15. ल॒भ॒ते॒ ऽमु म॒मुम् ॅल॑भते लभते॒ऽमु मा॒प्याय॑मान मा॒प्याय॑मान म॒मुम् ॅल॑भते लभते॒ऽमु मा॒प्याय॑मानम् । \newline
16. अ॒मु मा॒प्याय॑मान मा॒प्याय॑मान म॒मु म॒मु मा॒प्याय॑मान॒ मन्वन्वा॒ प्याय॑मान म॒मु म॒मु मा॒प्याय॑मान॒ मनु॑ । \newline
17. आ॒प्याय॑मान॒ मन्वन्वा॒ प्याय॑मान मा॒प्याय॑मान॒ मन्वा ऽन्वा॒प्याय॑मान मा॒प्याय॑मान॒ मन्वा । \newline
18. आ॒प्याय॑मान॒मित्या᳚ - प्याय॑मानम् । \newline
19. अन्वा ऽन्वन्वा प्या॑यते प्यायत॒ आ ऽन्वन्वा प्या॑यते । \newline
20. आ प्या॑यते प्यायत॒ आ प्या॑यत आग्नावैष्ण॒व मा᳚ग्नावैष्ण॒वम् प्या॑यत॒ आ प्या॑यत आग्नावैष्ण॒वम् । \newline
21. प्या॒य॒त॒ आ॒ग्ना॒वै॒ष्ण॒व मा᳚ग्नावैष्ण॒वम् प्या॑यते प्यायत आग्नावैष्ण॒व मेका॑दशकपाल॒ मेका॑दशकपाल माग्नावैष्ण॒वम् प्या॑यते प्यायत आग्नावैष्ण॒व मेका॑दशकपालम् । \newline
22. आ॒ग्ना॒वै॒ष्ण॒व मेका॑दशकपाल॒ मेका॑दशकपाल माग्नावैष्ण॒व मा᳚ग्नावैष्ण॒व मेका॑दशकपालम् पु॒रस्ता᳚त् पु॒रस्ता॒ देका॑दशकपाल माग्नावैष्ण॒व मा᳚ग्नावैष्ण॒व मेका॑दशकपालम् पु॒रस्ता᳚त् । \newline
23. आ॒ग्ना॒वै॒ष्ण॒वमित्या᳚ग्ना - वै॒ष्ण॒वम् । \newline
24. एका॑दशकपालम् पु॒रस्ता᳚त् पु॒रस्ता॒ देका॑दशकपाल॒ मेका॑दशकपालम् पु॒रस्ता॒न् निर् णिष् पु॒रस्ता॒ देका॑दशकपाल॒ मेका॑दशकपालम् पु॒रस्ता॒न् निः । \newline
25. एका॑दशकपाल॒मित्येका॑दश - क॒पा॒ल॒म् । \newline
26. पु॒रस्ता॒न् निर् णिष् पु॒रस्ता᳚त् पु॒रस्ता॒न् निर् व॑पेद् वपे॒न् निष् पु॒रस्ता᳚त् पु॒रस्ता॒न् निर् व॑पेत् । \newline
27. निर् व॑पेद् वपे॒न् निर् णिर् व॑पे॒थ् सर॑स्वत्यै॒ सर॑स्वत्यै वपे॒न् निर् णिर् व॑पे॒थ् सर॑स्वत्यै । \newline
28. व॒पे॒थ् सर॑स्वत्यै॒ सर॑स्वत्यै वपेद् वपे॒थ् सर॑स्वत्यै च॒रुम् च॒रुꣳ सर॑स्वत्यै वपेद् वपे॒थ् सर॑स्वत्यै च॒रुम् । \newline
29. सर॑स्वत्यै च॒रुम् च॒रुꣳ सर॑स्वत्यै॒ सर॑स्वत्यै च॒रुꣳ सर॑स्वते॒ सर॑स्वते च॒रुꣳ सर॑स्वत्यै॒ सर॑स्वत्यै च॒रुꣳ सर॑स्वते । \newline
30. च॒रुꣳ सर॑स्वते॒ सर॑स्वते च॒रुम् च॒रुꣳ सर॑स्वते॒ द्वाद॑शकपाल॒म् द्वाद॑शकपालꣳ॒॒ सर॑स्वते च॒रुम् च॒रुꣳ सर॑स्वते॒ द्वाद॑शकपालम् । \newline
31. सर॑स्वते॒ द्वाद॑शकपाल॒म् द्वाद॑शकपालꣳ॒॒ सर॑स्वते॒ सर॑स्वते॒ द्वाद॑शकपालं॒ ॅयद् यद् द्वाद॑शकपालꣳ॒॒ सर॑स्वते॒ सर॑स्वते॒ द्वाद॑शकपालं॒ ॅयत् । \newline
32. द्वाद॑शकपालं॒ ॅयद् यद् द्वाद॑शकपाल॒म् द्वाद॑शकपालं॒ ॅयदा᳚ग्ने॒य आ᳚ग्ने॒यो यद् द्वाद॑शकपाल॒म् द्वाद॑शकपालं॒ ॅयदा᳚ग्ने॒यः । \newline
33. द्वाद॑शकपाल॒मिति॒ द्वाद॑श - क॒पा॒ल॒म् । \newline
34. यदा᳚ग्ने॒य आ᳚ग्ने॒यो यद् यदा᳚ग्ने॒यो भव॑ति॒ भव॑ त्याग्ने॒यो यद् यदा᳚ग्ने॒यो भव॑ति । \newline
35. आ॒ग्ने॒यो भव॑ति॒ भव॑ त्याग्ने॒य आ᳚ग्ने॒यो भव॑ त्य॒ग्नि र॒ग्निर् भव॑ त्याग्ने॒य आ᳚ग्ने॒यो भव॑ त्य॒ग्निः । \newline
36. भव॑ त्य॒ग्नि र॒ग्निर् भव॑ति॒ भव॑ त्य॒ग्निर् वै वा अ॒ग्निर् भव॑ति॒ भव॑ त्य॒ग्निर् वै । \newline
37. अ॒ग्निर् वै वा अ॒ग्नि र॒ग्निर् वै य॑ज्ञ्मु॒खं ॅय॑ज्ञ्मु॒खं ॅवा अ॒ग्नि र॒ग्निर् वै य॑ज्ञ्मु॒खम् । \newline
38. वै य॑ज्ञ्मु॒खं ॅय॑ज्ञ्मु॒खं ॅवै वै य॑ज्ञ्मु॒खम् । \newline
39. य॒ज्ञ्॒मु॒खं ॅय॑ज्ञ्मु॒खम् । \newline
40. य॒ज्ञ्॒मु॒खमिति॑ यज्ञ् - मु॒खम् । \newline
41. य॒ज्ञ्॒मु॒ख मे॒वैव य॑ज्ञ्मु॒खं ॅय॑ज्ञ्मु॒ख मे॒व र्‌द्धि॒ मृद्धि॑ मे॒व य॑ज्ञ्मु॒खं ॅय॑ज्ञ्मु॒ख मे॒व र्‌द्धि᳚म् । \newline
42. य॒ज्ञ्॒मु॒खमिति॑ यज्ञ् - मु॒खम् । \newline
43. ए॒व र्‌द्धि॒ मृद्धि॑ मे॒वैव र्‌द्धि॑म् पु॒रस्ता᳚त् पु॒रस्ता॒ दृद्धि॑ मे॒वैव र्‌द्धि॑म् पु॒रस्ता᳚त् । \newline
44. ऋद्धि॑म् पु॒रस्ता᳚त् पु॒रस्ता॒ दृद्धि॒ मृद्धि॑म् पु॒रस्ता᳚द् धत्ते धत्ते पु॒रस्ता॒ दृद्धि॒ मृद्धि॑म् पु॒रस्ता᳚द् धत्ते । \newline
45. पु॒रस्ता᳚द् धत्ते धत्ते पु॒रस्ता᳚त् पु॒रस्ता᳚द् धत्ते॒ यद् यद् ध॑त्ते पु॒रस्ता᳚त् पु॒रस्ता᳚द् धत्ते॒ यत् । \newline
46. ध॒त्ते॒ यद् यद् ध॑त्ते धत्ते॒ यद् वै᳚ष्ण॒वो वै᳚ष्ण॒वो यद् ध॑त्ते धत्ते॒ यद् वै᳚ष्ण॒वः । \newline
47. यद् वै᳚ष्ण॒वो वै᳚ष्ण॒वो यद् यद् वै᳚ष्ण॒वो भव॑ति॒ भव॑ति वैष्ण॒वो यद् यद् वै᳚ष्ण॒वो भव॑ति । \newline
48. वै॒ष्ण॒वो भव॑ति॒ भव॑ति वैष्ण॒वो वै᳚ष्ण॒वो भव॑ति य॒ज्ञो य॒ज्ञो भव॑ति वैष्ण॒वो वै᳚ष्ण॒वो भव॑ति य॒ज्ञ्ः । \newline
49. भव॑ति य॒ज्ञो य॒ज्ञो भव॑ति॒ भव॑ति य॒ज्ञो वै वै य॒ज्ञो भव॑ति॒ भव॑ति य॒ज्ञो वै । \newline
50. य॒ज्ञो वै वै य॒ज्ञो य॒ज्ञो वै विष्णु॒र् विष्णु॒र् वै य॒ज्ञो य॒ज्ञो वै विष्णुः॑ । \newline
51. वै विष्णु॒र् विष्णु॒र् वै वै विष्णु॑र् य॒ज्ञ्ं ॅय॒ज्ञ्ं ॅविष्णु॒र् वै वै विष्णु॑र् य॒ज्ञ्म् । \newline
52. विष्णु॑र् य॒ज्ञ्ं ॅय॒ज्ञ्ं ॅविष्णु॒र् विष्णु॑र् य॒ज्ञ् मे॒वैव य॒ज्ञ्ं ॅविष्णु॒र् विष्णु॑र् य॒ज्ञ् मे॒व । \newline
53. य॒ज्ञ् मे॒वैव य॒ज्ञ्ं ॅय॒ज्ञ् मे॒वा रभ्या॒ रभ्यै॒व य॒ज्ञ्ं ॅय॒ज्ञ् मे॒वारभ्य॑ । \newline
54. ए॒वारभ्या॒ रभ्यै॒वैवा रभ्य॒ प्र प्रारभ्यै॒ वैवारभ्य॒ प्र । \newline
55. आ॒रभ्य॒ प्र प्रारभ्या॒ रभ्य॒ प्र त॑नुते तनुते॒ प्रारभ्या॒ रभ्य॒ प्र त॑नुते । \newline
56. आ॒रभ्येत्या᳚ - रभ्य॑ । \newline
57. प्र त॑नुते तनुते॒ प्र प्र त॑नुते॒ सर॑स्वत्यै॒ सर॑स्वत्यै तनुते॒ प्र प्र त॑नुते॒ सर॑स्वत्यै । \newline
58. त॒नु॒ते॒ सर॑स्वत्यै॒ सर॑स्वत्यै तनुते तनुते॒ सर॑स्वत्यै च॒रु श्च॒रुः सर॑स्वत्यै तनुते तनुते॒ सर॑स्वत्यै च॒रुः । \newline
59. सर॑स्वत्यै च॒रु श्च॒रुः सर॑स्वत्यै॒ सर॑स्वत्यै च॒रुर् भ॑वति भवति च॒रुः सर॑स्वत्यै॒ सर॑स्वत्यै च॒रुर् भ॑वति । \newline
60. च॒रुर् भ॑वति भवति च॒रु श्च॒रुर् भ॑वति॒ सर॑स्वते॒ सर॑स्वते भवति च॒रु श्च॒रुर् भ॑वति॒ सर॑स्वते । \newline
61. भ॒व॒ति॒ सर॑स्वते॒ सर॑स्वते भवति भवति॒ सर॑स्वते॒ द्वाद॑शकपालो॒ द्वाद॑शकपालः॒ सर॑स्वते भवति भवति॒ सर॑स्वते॒ द्वाद॑शकपालः । \newline
62. सर॑स्वते॒ द्वाद॑शकपालो॒ द्वाद॑शकपालः॒ सर॑स्वते॒ सर॑स्वते॒ द्वाद॑शकपालो ऽमावा॒स्या॑ ऽमावा॒स्या᳚ द्वाद॑शकपालः॒ सर॑स्वते॒ सर॑स्वते॒ द्वाद॑शकपालो ऽमावा॒स्या᳚ । \newline
63. द्वाद॑शकपालो ऽमावा॒स्या॑ ऽमावा॒स्या᳚ द्वाद॑शकपालो॒ द्वाद॑शकपालो ऽमावा॒स्या॑ वै वा अ॑मावा॒स्या᳚ द्वाद॑शकपालो॒ द्वाद॑शकपालो ऽमावा॒स्या॑ वै । \newline
64. द्वाद॑शकपाल॒ इति॒ द्वाद॑श - क॒पा॒लः॒ । \newline
65. अ॒मा॒वा॒स्या॑ वै वा अ॑मावा॒स्या॑ ऽमावा॒स्या॑ वै सर॑स्वती॒ सर॑स्वती॒ वा अ॑मावा॒स्या॑ ऽमावा॒स्या॑ वै सर॑स्वती । \newline
66. अ॒मा॒वा॒स्येत्य॑मा - वा॒स्या᳚ । \newline
67. वै सर॑स्वती॒ सर॑स्वती॒ वै वै सर॑स्वती पू॒र्णमा॑सः पू॒र्णमा॑सः॒ सर॑स्वती॒ वै वै सर॑स्वती पू॒र्णमा॑सः । \newline
68. सर॑स्वती पू॒र्णमा॑सः पू॒र्णमा॑सः॒ सर॑स्वती॒ सर॑स्वती पू॒र्णमा॑सः॒ सर॑स्वा॒न् थ्सर॑स्वान् पू॒र्णमा॑सः॒ सर॑स्वती॒ सर॑स्वती पू॒र्णमा॑सः॒ सर॑स्वान् । \newline
69. पू॒र्णमा॑सः॒ सर॑स्वा॒न् थ्सर॑स्वान् पू॒र्णमा॑सः पू॒र्णमा॑सः॒ सर॑स्वा॒न् तौ तौ सर॑स्वान् पू॒र्णमा॑सः पू॒र्णमा॑सः॒ सर॑स्वा॒न् तौ । \newline
70. पू॒र्णमा॑स॒ इति॑ पू॒र्ण - मा॒सः॒ । \newline
71. सर॑स्वा॒न् तौ तौ सर॑स्वा॒न् थ्सर॑स्वा॒न् ता वे॒वैव तौ सर॑स्वा॒न् थ्सर॑स्वा॒न् ता वे॒व । \newline
72. ता वे॒वैव तौ ता वे॒व सा॒क्षाथ् सा॒क्षा दे॒व तौ ता वे॒व सा॒क्षात् । \newline
73. ए॒व सा॒क्षाथ् सा॒क्षा दे॒वैव सा॒क्षादा सा॒क्षा दे॒वैव सा॒क्षादा । \newline
74. सा॒क्षादा सा॒क्षाथ् सा॒क्षादा र॑भते रभत॒ आ सा॒क्षाथ् सा॒क्षादा र॑भते । \newline
75. सा॒क्षादिति॑ स - अ॒क्षात् । \newline
76. आ र॑भते रभत॒ आ र॑भत ऋ॒द्ध्नो त्यृ॒द्ध्नोति॑ रभत॒ आ र॑भत ऋ॒द्ध्नोति॑ । \newline
77. र॒भ॒त॒ ऋ॒द्ध्नो त्यृ॒द्ध्नोति॑ रभते रभत ऋ॒द्ध्नो त्या᳚भ्या माभ्या मृ॒द्ध्नोति॑ रभते रभत 
ऋ॒द्ध्नो त्या᳚भ्याम् । \newline
78. ऋ॒द्ध्नो त्या᳚भ्या माभ्या मृ॒द्ध्नो त्यृ॒द्ध्नो त्या᳚भ्या॒म् द्वाद॑शकपालो॒ द्वाद॑शकपाल 
आभ्या मृ॒द्ध्नो त्यृ॒द्ध्नो त्या᳚भ्या॒म् द्वाद॑शकपालः । \newline
79. आ॒भ्या॒म् द्वाद॑शकपालो॒ द्वाद॑शकपाल आभ्या माभ्या॒म् द्वाद॑शकपालः॒ सर॑स्वते॒ सर॑स्वते॒ द्वाद॑शकपाल आभ्या माभ्या॒म् द्वाद॑शकपालः॒ सर॑स्वते । \newline
80. द्वाद॑शकपालः॒ सर॑स्वते॒ सर॑स्वते॒ द्वाद॑शकपालो॒ द्वाद॑शकपालः॒ सर॑स्वते भवति भवति॒ सर॑स्वते॒ द्वाद॑शकपालो॒ द्वाद॑शकपालः॒ सर॑स्वते भवति । \newline
81. द्वाद॑शकपाल॒ इति॒ द्वाद॑श - क॒पा॒लः॒ । \newline
82. सर॑स्वते भवति भवति॒ सर॑स्वते॒ सर॑स्वते भवति मिथुन॒त्वाय॑ मिथुन॒त्वाय॑ भवति॒ सर॑स्वते॒ सर॑स्वते भवति मिथुन॒त्वाय॑ । \newline
83. भ॒व॒ति॒ मि॒थु॒न॒त्वाय॑ मिथुन॒त्वाय॑ भवति भवति मिथुन॒त्वाय॒ प्रजा᳚त्यै॒ प्रजा᳚त्यै मिथुन॒त्वाय॑ भवति भवति मिथुन॒त्वाय॒ प्रजा᳚त्यै । \newline
84. मि॒थु॒न॒त्वाय॒ प्रजा᳚त्यै॒ प्रजा᳚त्यै मिथुन॒त्वाय॑ मिथुन॒त्वाय॒ प्रजा᳚त्यै मिथु॒नौ मि॑थु॒नौ प्रजा᳚त्यै मिथुन॒त्वाय॑ मिथुन॒त्वाय॒ प्रजा᳚त्यै मिथु॒नौ । \newline
85. मि॒थु॒न॒त्वायेति॑ मिथुन - त्वाय॑ । \newline
86. प्रजा᳚त्यै मिथु॒नौ मि॑थु॒नौ प्रजा᳚त्यै॒ प्रजा᳚त्यै मिथु॒नौ गावौ॒ गावौ॑ मिथु॒नौ प्रजा᳚त्यै॒ प्रजा᳚त्यै मिथु॒नौ गावौ᳚ । \newline
87. प्रजा᳚त्या॒ इति॒ प्र - जा॒त्यै॒ । \newline
88. मि॒थु॒नौ गावौ॒ गावौ॑ मिथु॒नौ मि॑थु॒नौ गावौ॒ दक्षि॑णा॒ दक्षि॑णा॒ गावौ॑ मिथु॒नौ मि॑थु॒नौ गावौ॒ दक्षि॑णा । \newline
89. गावौ॒ दक्षि॑णा॒ दक्षि॑णा॒ गावौ॒ गावौ॒ दक्षि॑णा॒ समृ॑द्ध्यै॒ समृ॑द्ध्यै॒ दक्षि॑णा॒ गावौ॒ गावौ॒ दक्षि॑णा॒ समृ॑द्ध्यै । \newline
90. दक्षि॑णा॒ समृ॑द्ध्यै॒ समृ॑द्ध्यै॒ दक्षि॑णा॒ दक्षि॑णा॒ समृ॑द्ध्यै । \newline
91. समृ॑द्ध्या॒ इति॒ सं - ऋ॒द्ध्यै॒ । \newline
\pagebreak
\markright{ TS 3.5.2.1  \hfill https://www.vedavms.in \hfill}

\section{ TS 3.5.2.1 }

\textbf{TS 3.5.2.1 } \newline
\textbf{Samhita Paata} \newline

ऋष॑यो॒ वा इन्द्रं॑ प्र॒त्यक्षं॒ नाप॑श्य॒न् तं ॅवसि॑ष्ठः प्र॒त्यक्ष॑मपश्य॒थ् सो᳚ऽब्रवी॒द्-ब्राह्म॑णं ते वक्ष्यामि॒ यथा॒ त्वत्पु॑रोहिताः प्र॒जाः प्र॑जनि॒ष्यन्तेऽथ॒ मेत॑रेभ्य॒ ऋषि॑भ्यो॒ मा प्रवो॑च॒ इति॒ तस्मा॑ ए॒तान्थ्स्तोम॑-भागानब्रवी॒त् ततो॒ वसि॑ष्ठपुरोहिताः प्र॒जाः प्राजा॑यन्त॒ तस्मा᳚द्-वासि॒ष्ठो ब्र॒ह्मा का॒र्यः॑ प्रैव जा॑यते र॒श्मिर॑सि॒ क्षया॑य त्वा॒ क्षयं॑ जि॒न्वे - [  ] \newline

\textbf{Pada Paata} \newline

ऋष॑यः । वै । इन्द्र᳚म् । प्र॒त्यक्ष॒मिति॑ प्रति - अक्ष᳚म् । न । अ॒प॒श्य॒न्न् । तम् । वसि॑ष्ठः । प्र॒त्यक्ष॒मिति॑ प्रति - अक्ष᳚म् । अ॒प॒श्य॒त् । सः । अ॒ब्र॒वी॒त् । ब्राह्म॑णम् । ते॒ । व॒क्ष्या॒मि॒ । यथा᳚ । त्वत्पु॑रोहिता॒ इति॒ त्वत् - पु॒रो॒हि॒ताः॒ । प्र॒जा इति॑ प्र - जाः । प्र॒ज॒नि॒ष्यन्त॒ इति॑ प्र - ज॒नि॒ष्यन्ते᳚ । अथ॑ । मा॒ । इत॑रेभ्यः । ऋषि॑भ्य॒ इत्यृषि॑ - भ्यः॒ । मा । प्रेति॑ । वो॒चः॒ । इति॑ । तस्मै᳚ । ए॒तान् । स्तोम॑भागा॒निति॒ स्तोम॑ - भा॒गा॒न् । अ॒ब्र॒वी॒त् । ततः॑ । वसि॑ष्ठपुरोहिता॒ इति॒ वसि॑ष्ठ - पु॒रो॒हि॒ताः॒ । प्र॒जा इति॑ प्र - जाः । प्रेति॑ । अ॒जा॒य॒न्त॒ । तस्मा᳚त् । वा॒सि॒ष्ठः । ब्र॒ह्मा । का॒र्यः॑ । प्रेति॑ । ए॒व । जा॒य॒ते॒ । र॒श्मिः । अ॒सि॒ । क्षया॑य । त्वा॒ । क्षय᳚म् । जि॒न्व॒ । इति॑ ।  \newline


\textbf{Krama Paata} \newline

ऋष॑यो॒ वै । वा इन्द्र᳚म् । इन्द्र॑म् प्र॒त्यक्ष᳚म् । प्र॒त्यक्ष॒म् न । प्र॒त्यक्ष॒मिति॑ प्रति - अक्ष᳚म् । नाप॑श्यन्न् । अ॒प॒श्य॒न् तम् । तं ॅवसि॑ष्ठः । वसि॑ष्ठः प्र॒त्यक्ष᳚म् । प्र॒त्यक्ष॑मपश्यत् । प्र॒त्यक्ष॒मिति॑ प्रति - अक्ष᳚म् । अ॒प॒श्य॒थ् सः । सो᳚ ऽब्रवीत् । अ॒ब्र॒वी॒द् ब्राह्म॑णम् । ब्राह्म॑णम् ते । ते॒ व॒क्ष्या॒मि॒ । व॒क्ष्या॒मि॒ यथा᳚ । यथा॒ त्वत्पु॑रोहिताः । त्वत्पु॑रोहिताः प्र॒जाः । त्वत्पु॑रोहिता॒ इति॒ त्वत् - पु॒रो॒हि॒ताः॒ । प्र॒जाः प्र॑जनि॒ष्यन्ते᳚ । प्र॒जा इति॑ प्र - जाः । प्र॒ज॒नि॒ष्यन्ते ऽथ॑ । प्र॒ज॒नि॒ष्यन्त॒ इति॑ प्र - ज॒नि॒ष्यन्ते᳚ । अथ॑ मा । मेत॑रेभ्यः । इत॑रेभ्य॒ ऋषि॑भ्यः । ऋषि॑भ्यो॒ मा । ऋषि॑भ्य॒ इत्यृषि॑ - भ्यः॒ । मा प्र । प्र वो॑चः । वो॒च॒ इति॑ । इति॒ तस्मै᳚ । तस्मा॑ ए॒तान् । ए॒तान्थ् स्तोम॑भागान् । स्तोम॑भागानब्रवीत् । स्तोम॑भागा॒निति॒ स्तोम॑ - भा॒गा॒न्॒ । अ॒ब्र॒वी॒त् ततः॑ । ततो॒ वसि॑ष्ठपुरोहिताः । वसि॑ष्ठपुरोहिताः प्र॒जाः । वसि॑ष्ठपुरोहिता॒ इति॒ वसि॑ष्ठ - पु॒रो॒हि॒ताः॒ । प्र॒जाः प्र । प्र॒जा इति॑ प्र - जाः । प्राजा॑यन्त । अ॒जा॒य॒न्त॒ तस्मा᳚त् । तस्मा᳚द् वासि॒ष्ठः । वा॒सि॒ष्ठो ब्र॒ह्मा । ब्र॒ह्मा का॒र्यः॑ । का॒र्यः॑ प्र । प्रैव । ए॒व जा॑यते । जा॒य॒ते॒ र॒श्मिः । र॒श्मिर॑सि । अ॒सि॒ क्षया॑य । क्षया॑य त्वा । त्वा॒ क्षय᳚म् । क्षय॑म् जिन्व । जि॒न्वेति॑ । इत्या॑ह \newline

\textbf{Jatai Paata} \newline

1. ऋष॑यो॒ वै वा ऋष॑य॒ ऋष॑यो॒ वै । \newline
2. वा इन्द्र॒ मिन्द्रं॒ ॅवै वा इन्द्र᳚म् । \newline
3. इन्द्र॑म् प्र॒त्यक्ष॑म् प्र॒त्यक्ष॒ मिन्द्र॒ मिन्द्र॑म् प्र॒त्यक्ष᳚म् । \newline
4. प्र॒त्यक्ष॒म् न न प्र॒त्यक्ष॑म् प्र॒त्यक्ष॒म् न । \newline
5. प्र॒त्यक्ष॒मिति॑ प्रति - अक्ष᳚म् । \newline
6. नाप॑श्यन् नपश्य॒न् न नाप॑श्यन्न् । \newline
7. अ॒प॒श्य॒न् तम् त म॑पश्यन् नपश्य॒न् तम् । \newline
8. तं ॅवसि॑ष्ठो॒ वसि॑ष्ठ॒ स्तम् तं ॅवसि॑ष्ठः । \newline
9. वसि॑ष्ठः प्र॒त्यक्ष॑म् प्र॒त्यक्षं॒ ॅवसि॑ष्ठो॒ वसि॑ष्ठः प्र॒त्यक्ष᳚म् । \newline
10. प्र॒त्यक्ष॑ मपश्य दपश्यत् प्र॒त्यक्ष॑म् प्र॒त्यक्ष॑ मपश्यत् । \newline
11. प्र॒त्यक्ष॒मिति॑ प्रति - अक्ष᳚म् । \newline
12. अ॒प॒श्य॒थ् स सो॑ ऽपश्य दपश्य॒थ् सः । \newline
13. सो᳚ ऽब्रवी दब्रवी॒थ् स सो᳚ ऽब्रवीत् । \newline
14. अ॒ब्र॒वी॒द् ब्राह्म॑ण॒म् ब्राह्म॑ण मब्रवी दब्रवी॒द् ब्राह्म॑णम् । \newline
15. ब्राह्म॑णम् ते ते॒ ब्राह्म॑ण॒म् ब्राह्म॑णम् ते । \newline
16. ते॒ व॒क्ष्या॒मि॒ व॒क्ष्या॒मि॒ ते॒ ते॒ व॒क्ष्या॒मि॒ । \newline
17. व॒क्ष्या॒मि॒ यथा॒ यथा॑ वक्ष्यामि वक्ष्यामि॒ यथा᳚ । \newline
18. यथा॒ त्वत्पु॑रोहिता॒ स्त्वत्पु॑रोहिता॒ यथा॒ यथा॒ त्वत्पु॑रोहिताः । \newline
19. त्वत्पु॑रोहिताः प्र॒जाः प्र॒जा स्त्वत्पु॑रोहिता॒ स्त्वत्पु॑रोहिताः प्र॒जाः । \newline
20. त्वत्पु॑रोहिता॒ इति॒ त्वत् - पु॒रो॒हि॒ताः॒ । \newline
21. प्र॒जाः प्र॑जनि॒ष्यन्ते᳚ प्रजनि॒ष्यन्ते᳚ प्र॒जाः प्र॒जाः प्र॑जनि॒ष्यन्ते᳚ । \newline
22. प्र॒जा इति॑ प्र - जाः । \newline
23. प्र॒ज॒नि॒ष्यन्ते ऽथाथ॑ प्रजनि॒ष्यन्ते᳚ प्रजनि॒ष्यन्ते ऽथ॑ । \newline
24. प्र॒ज॒नि॒ष्यन्त॒ इति॑ प्र - ज॒नि॒ष्यन्ते᳚ । \newline
25. अथ॑ मा॒ मा ऽथाथ॑ मा । \newline
26. मेत॑रेभ्य॒ इत॑ रेभ्यो मा॒ मेत॑ रेभ्यः । \newline
27. इत॑रेभ्य॒ ऋषि॑भ्य॒ ऋषि॑भ्य॒ इत॑रेभ्य॒ इत॑रेभ्य॒ ऋषि॑भ्यः । \newline
28. ऋषि॑भ्यो॒ मा मर्.षि॑भ्य॒ ऋषि॑भ्यो॒ मा । \newline
29. ऋषि॑भ्य॒ इत्यृषि॑ - भ्यः॒ । \newline
30. मा प्र प्र मा मा प्र । \newline
31. प्र वो॑चो वोचः॒ प्र प्र वो॑चः । \newline
32. वो॒च॒ इतीति॑ वोचो वोच॒ इति॑ । \newline
33. इति॒ तस्मै॒ तस्मा॒ इतीति॒ तस्मै᳚ । \newline
34. तस्मा॑ ए॒ता ने॒तान् तस्मै॒ तस्मा॑ ए॒तान् । \newline
35. ए॒तान् थ्स्तोम॑भागा॒न् थ्स्तोम॑भागा ने॒ता ने॒तान् थ्स्तोम॑भागान् । \newline
36. स्तोम॑भागा नब्रवी दब्रवी॒थ् स्तोम॑भागा॒न् थ्स्तोम॑भागा नब्रवीत् । \newline
37. स्तोम॑भागा॒निति॒ स्तोम॑ - भा॒गा॒न् । \newline
38. अ॒ब्र॒वी॒त् तत॒स्ततो᳚ ऽब्रवी दब्रवी॒त् ततः॑ । \newline
39. ततो॒ वसि॑ष्ठपुरोहिता॒ वसि॑ष्ठपुरोहिता॒ स्तत॒ स्ततो॒ वसि॑ष्ठपुरोहिताः । \newline
40. वसि॑ष्ठपुरोहिताः प्र॒जाः प्र॒जा वसि॑ष्ठपुरोहिता॒ वसि॑ष्ठपुरोहिताः प्र॒जाः । \newline
41. वसि॑ष्ठपुरोहिता॒ इति॒ वसि॑ष्ठ - पु॒रो॒हि॒ताः॒ । \newline
42. प्र॒जाः प्र प्र प्र॒जाः प्र॒जाः प्र । \newline
43. प्र॒जा इति॑ प्र - जाः । \newline
44. प्राजा॑यन्ता जायन्त॒ प्र प्राजा॑यन्त । \newline
45. अ॒जा॒य॒न्त॒ तस्मा॒त् तस्मा॑ दजायन्ता जायन्त॒ तस्मा᳚त् । \newline
46. तस्मा᳚द् वासि॒ष्ठो वा॑सि॒ष्ठ स्तस्मा॒त् तस्मा᳚द् वासि॒ष्ठः । \newline
47. वा॒सि॒ष्ठो ब्र॒ह्मा ब्र॒ह्मा वा॑सि॒ष्ठो वा॑सि॒ष्ठो ब्र॒ह्मा । \newline
48. ब्र॒ह्मा का॒र्यः॑ का॒र्यो᳚ ब्र॒ह्मा ब्र॒ह्मा का॒र्यः॑ । \newline
49. का॒र्यः॑ प्र प्र का॒र्यः॑ का॒र्यः॑ प्र । \newline
50. प्रैवैव प्र प्रैव । \newline
51. ए॒व जा॑यते जायत ए॒वैव जा॑यते । \newline
52. जा॒य॒ते॒ र॒श्मी र॒श्मिर् जा॑यते जायते र॒श्मिः । \newline
53. र॒श्मि र॑स्यसि र॒श्मी र॒श्मि र॑सि । \newline
54. अ॒सि॒ क्षया॑य॒ क्षया॑या स्यसि॒ क्षया॑य । \newline
55. क्षया॑य त्वा त्वा॒ क्षया॑य॒ क्षया॑य त्वा । \newline
56. त्वा॒ क्षय॒म् क्षय॑म् त्वा त्वा॒ क्षय᳚म् । \newline
57. क्षय॑म् जिन्व जिन्व॒ क्षय॒म् क्षय॑म् जिन्व । \newline
58. जि॒न्वे तीति॑ जिन्व जि॒न्वेति॑ । \newline
59. इत्या॑हा॒हे तीत्या॑ह । \newline

\textbf{Ghana Paata } \newline

1. ऋष॑यो॒ वै वा ऋष॑य॒ ऋष॑यो॒ वा इन्द्र॒ मिन्द्रं॒ ॅवा ऋष॑य॒ ऋष॑यो॒ वा इन्द्र᳚म् । \newline
2. वा इन्द्र॒ मिन्द्रं॒ ॅवै वा इन्द्र॑म् प्र॒त्यक्ष॑म् प्र॒त्यक्ष॒ मिन्द्रं॒ ॅवै वा इन्द्र॑म् प्र॒त्यक्ष᳚म् । \newline
3. इन्द्र॑म् प्र॒त्यक्ष॑म् प्र॒त्यक्ष॒ मिन्द्र॒ मिन्द्र॑म् प्र॒त्यक्ष॒न्न न प्र॒त्यक्ष॒ मिन्द्र॒ मिन्द्र॑म् प्र॒त्यक्ष॒न्न । \newline
4. प्र॒त्यक्ष॒न्न न प्र॒त्यक्ष॑म् प्र॒त्यक्ष॒म् नाप॑श्यन् नपश्य॒न् न प्र॒त्यक्ष॑म् प्र॒त्यक्ष॒म् नाप॑श्यन्न् । \newline
5. प्र॒त्यक्ष॒मिति॑ प्रति - अक्ष᳚म् । \newline
6. नाप॑श्यन् नपश्य॒न् न नाप॑श्य॒न् तम् त म॑पश्य॒न् न नाप॑श्य॒न् तम् । \newline
7. अ॒प॒श्य॒न् तम् त म॑पश्यन् नपश्य॒न् तं ॅवसि॑ष्ठो॒ वसि॑ष्ठ॒ स्त म॑पश्यन् नपश्य॒न् तं ॅवसि॑ष्ठः । \newline
8. तं ॅवसि॑ष्ठो॒ वसि॑ष्ठ॒ स्तम् तं ॅवसि॑ष्ठः प्र॒त्यक्ष॑म् प्र॒त्यक्षं॒ ॅवसि॑ष्ठ॒ स्तम् तं ॅवसि॑ष्ठः प्र॒त्यक्ष᳚म् । \newline
9. वसि॑ष्ठः प्र॒त्यक्ष॑म् प्र॒त्यक्षं॒ ॅवसि॑ष्ठो॒ वसि॑ष्ठः प्र॒त्यक्ष॑ मपश्य दपश्यत् प्र॒त्यक्षं॒ ॅवसि॑ष्ठो॒ वसि॑ष्ठः प्र॒त्यक्ष॑ मपश्यत् । \newline
10. प्र॒त्यक्ष॑ मपश्य दपश्यत् प्र॒त्यक्ष॑म् प्र॒त्यक्ष॑ मपश्य॒थ् स सो॑ ऽपश्यत् प्र॒त्यक्ष॑म् प्र॒त्यक्ष॑ मपश्य॒थ् सः । \newline
11. प्र॒त्यक्ष॒मिति॑ प्रति - अक्ष᳚म् । \newline
12. अ॒प॒श्य॒थ् स सो॑ ऽपश्य दपश्य॒थ् सो᳚ ऽब्रवी दब्रवी॒थ् सो॑ ऽपश्य दपश्य॒थ् सो᳚ ऽब्रवीत् । \newline
13. सो᳚ ऽब्रवी दब्रवी॒थ् स सो᳚ ऽब्रवी॒द् ब्राह्म॑ण॒म् ब्राह्म॑ण मब्रवी॒थ् स सो᳚ ऽब्रवी॒द् ब्राह्म॑णम् । \newline
14. अ॒ब्र॒वी॒द् ब्राह्म॑ण॒म् ब्राह्म॑ण मब्रवी दब्रवी॒द् ब्राह्म॑णम् ते ते॒ ब्राह्म॑ण मब्रवी दब्रवी॒द् ब्राह्म॑णम् ते । \newline
15. ब्राह्म॑णम् ते ते॒ ब्राह्म॑ण॒म् ब्राह्म॑णम् ते वक्ष्यामि वक्ष्यामि ते॒ ब्राह्म॑ण॒म् ब्राह्म॑णम् ते वक्ष्यामि । \newline
16. ते॒ व॒क्ष्या॒मि॒ व॒क्ष्या॒मि॒ ते॒ ते॒ व॒क्ष्या॒मि॒ यथा॒ यथा॑ वक्ष्यामि ते ते वक्ष्यामि॒ यथा᳚ । \newline
17. व॒क्ष्या॒मि॒ यथा॒ यथा॑ वक्ष्यामि वक्ष्यामि॒ यथा॒ त्वत्पु॑रोहिता॒ स्त्वत्पु॑रोहिता॒ यथा॑ वक्ष्यामि वक्ष्यामि॒ यथा॒ त्वत्पु॑रोहिताः । \newline
18. यथा॒ त्वत्पु॑रोहिता॒ स्त्वत्पु॑रोहिता॒ यथा॒ यथा॒ त्वत्पु॑रोहिताः प्र॒जाः प्र॒जा स्त्वत्पु॑रोहिता॒ यथा॒ यथा॒ त्वत्पु॑रोहिताः प्र॒जाः । \newline
19. त्वत्पु॑रोहिताः प्र॒जाः प्र॒जा स्त्वत्पु॑रोहिता॒ स्त्वत्पु॑रोहिताः प्र॒जाः प्र॑जनि॒ष्यन्ते᳚ प्रजनि॒ष्यन्ते᳚ प्र॒जा स्त्वत्पु॑रोहिता॒ स्त्वत्पु॑रोहिताः प्र॒जाः प्र॑जनि॒ष्यन्ते᳚ । \newline
20. त्वत्पु॑रोहिता॒ इति॒ त्वत् - पु॒रो॒हि॒ताः॒ । \newline
21. प्र॒जाः प्र॑जनि॒ष्यन्ते᳚ प्रजनि॒ष्यन्ते᳚ प्र॒जाः प्र॒जाः प्र॑जनि॒ष्यन्ते ऽथाथ॑ प्रजनि॒ष्यन्ते᳚ प्र॒जाः प्र॒जाः प्र॑जनि॒ष्यन्ते ऽथ॑ । \newline
22. प्र॒जा इति॑ प्र - जाः । \newline
23. प्र॒ज॒नि॒ष्यन्ते ऽथाथ॑ प्रजनि॒ष्यन्ते᳚ प्रजनि॒ष्यन्ते ऽथ॑ मा॒ मा ऽथ॑ प्रजनि॒ष्यन्ते᳚ प्रजनि॒ष्यन्ते ऽथ॑ मा । \newline
24. प्र॒ज॒नि॒ष्यन्त॒ इति॑ प्र - ज॒नि॒ष्यन्ते᳚ । \newline
25. अथ॑ मा॒ मा ऽथाथ॒ मेत॑रेभ्य॒ इत॑रेभ्यो॒ मा ऽथाथ॒ मेत॑रेभ्यः । \newline
26. मेत॑रेभ्य॒ इत॑रेभ्यो मा॒ मेत॑रेभ्य॒ ऋषि॑भ्य॒ ऋषि॑भ्य॒ इत॑रेभ्यो मा॒ मेत॑रेभ्य॒ ऋषि॑भ्यः । \newline
27. इत॑रेभ्य॒ ऋषि॑भ्य॒ ऋषि॑भ्य॒ इत॑रेभ्य॒ इत॑रेभ्य॒ ऋषि॑भ्यो॒ मा मर्.षि॑भ्य॒ इत॑रेभ्य॒ इत॑रेभ्य॒ ऋषि॑भ्यो॒ मा । \newline
28. ऋषि॑भ्यो॒ मा मर्.षि॑भ्य॒ ऋषि॑भ्यो॒ मा प्र प्र मर्.षि॑भ्य॒ ऋषि॑भ्यो॒ मा प्र । \newline
29. ऋषि॑भ्य॒ इत्यृषि॑ - भ्यः॒ । \newline
30. मा प्र प्र मा मा प्र वो॑चो वोचः॒ प्र मा मा प्र वो॑चः । \newline
31. प्र वो॑चो वोचः॒ प्र प्र वो॑च॒ इतीति॑ वोचः॒ प्र प्र वो॑च॒ इति॑ । \newline
32. वो॒च॒ इतीति॑ वोचो वोच॒ इति॒ तस्मै॒ तस्मा॒ इति॑ वोचो वोच॒ इति॒ तस्मै᳚ । \newline
33. इति॒ तस्मै॒ तस्मा॒ इतीति॒ तस्मा॑ ए॒ता ने॒तान् तस्मा॒ इतीति॒ तस्मा॑ ए॒तान् । \newline
34. तस्मा॑ ए॒ता ने॒तान् तस्मै॒ तस्मा॑ ए॒तान् थ्स्तोम॑भागा॒न् थ्स्तोम॑भागा ने॒तान् तस्मै॒ तस्मा॑ ए॒तान् थ्स्तोम॑भागान् । \newline
35. ए॒तान् थ्स्तोम॑भागा॒न् थ्स्तोम॑भागा ने॒ता ने॒तान् थ्स्तोम॑भागा नब्रवी दब्रवी॒थ् स्तोम॑भागा ने॒ता ने॒तान् थ्स्तोम॑भागा नब्रवीत् । \newline
36. स्तोम॑भागा नब्रवी दब्रवी॒थ् स्तोम॑भागा॒न् थ्स्तोम॑भागा नब्रवी॒त् तत॒ स्ततो᳚ ऽब्रवी॒थ् स्तोम॑भागा॒न् थ्स्तोम॑भागा नब्रवी॒त् ततः॑ । \newline
37. स्तोम॑भागा॒निति॒ स्तोम॑ - भा॒गा॒न् । \newline
38. अ॒ब्र॒वी॒त् तत॒ स्ततो᳚ ऽब्रवी दब्रवी॒त् ततो॒ वसि॑ष्ठपुरोहिता॒ वसि॑ष्ठपुरोहिता॒ स्ततो᳚ ऽब्रवी दब्रवी॒त् ततो॒ वसि॑ष्ठपुरोहिताः । \newline
39. ततो॒ वसि॑ष्ठपुरोहिता॒ वसि॑ष्ठपुरोहिता॒ स्तत॒ स्ततो॒ वसि॑ष्ठपुरोहिताः प्र॒जाः प्र॒जा वसि॑ष्ठपुरोहिता॒ स्तत॒ स्ततो॒ वसि॑ष्ठपुरोहिताः प्र॒जाः । \newline
40. वसि॑ष्ठपुरोहिताः प्र॒जाः प्र॒जा वसि॑ष्ठपुरोहिता॒ वसि॑ष्ठपुरोहिताः प्र॒जाः प्र प्र प्र॒जा वसि॑ष्ठपुरोहिता॒ वसि॑ष्ठपुरोहिताः प्र॒जाः प्र । \newline
41. वसि॑ष्ठपुरोहिता॒ इति॒ वसि॑ष्ठ - पु॒रो॒हि॒ताः॒ । \newline
42. प्र॒जाः प्र प्र प्र॒जाः प्र॒जाः प्राजा॑यन्ता जायन्त॒ प्र प्र॒जाः प्र॒जाः प्राजा॑यन्त । \newline
43. प्र॒जा इति॑ प्र - जाः । \newline
44. प्राजा॑यन्ता जायन्त॒ प्र प्राजा॑यन्त॒ तस्मा॒त् तस्मा॑द जायन्त॒ प्र प्राजा॑यन्त॒ तस्मा᳚त् । \newline
45. अ॒जा॒य॒न्त॒ तस्मा॒त् तस्मा॑ दजायन्ता जायन्त॒ तस्मा᳚द् वासि॒ष्ठो वा॑सि॒ष्ठ स्तस्मा॑ दजायन्ता जायन्त॒ तस्मा᳚द् वासि॒ष्ठः । \newline
46. तस्मा᳚द् वासि॒ष्ठो वा॑सि॒ष्ठ स्तस्मा॒त् तस्मा᳚द् वासि॒ष्ठो ब्र॒ह्मा ब्र॒ह्मा वा॑सि॒ष्ठ स्तस्मा॒त् तस्मा᳚द् वासि॒ष्ठो ब्र॒ह्मा । \newline
47. वा॒सि॒ष्ठो ब्र॒ह्मा ब्र॒ह्मा वा॑सि॒ष्ठो वा॑सि॒ष्ठो ब्र॒ह्मा का॒र्यः॑ का॒र्यो᳚ ब्र॒ह्मा वा॑सि॒ष्ठो वा॑सि॒ष्ठो ब्र॒ह्मा का॒र्यः॑ । \newline
48. ब्र॒ह्मा का॒र्यः॑ का॒र्यो᳚ ब्र॒ह्मा ब्र॒ह्मा का॒र्यः॑ प्र प्र का॒र्यो᳚ ब्र॒ह्मा ब्र॒ह्मा का॒र्यः॑ प्र । \newline
49. का॒र्यः॑ प्र प्र का॒र्यः॑ का॒र्यः॑ प्रैवैव प्र का॒र्यः॑ का॒र्यः॑ प्रैव । \newline
50. प्रैवैव प्र प्रैव जा॑यते जायत ए॒व प्र प्रैव जा॑यते । \newline
51. ए॒व जा॑यते जायत ए॒वैव जा॑यते र॒श्मी र॒श्मिर् जा॑यत ए॒वैव जा॑यते र॒श्मिः । \newline
52. जा॒य॒ते॒ र॒श्मी र॒श्मिर् जा॑यते जायते र॒श्मि र॑स्यसि र॒श्मिर् जा॑यते जायते र॒श्मि र॑सि । \newline
53. र॒श्मि र॑स्यसि र॒श्मी र॒श्मि र॑सि॒ क्षया॑य॒ क्षया॑यासि र॒श्मी र॒श्मि र॑सि॒ क्षया॑य । \newline
54. अ॒सि॒ क्षया॑य॒ क्षया॑या स्यसि॒ क्षया॑य त्वा त्वा॒ क्षया॑या स्यसि॒ क्षया॑य त्वा । \newline
55. क्षया॑य त्वा त्वा॒ क्षया॑य॒ क्षया॑य त्वा॒ क्षय॒म् क्षय॑म् त्वा॒ क्षया॑य॒ क्षया॑य त्वा॒ क्षय᳚म् । \newline
56. त्वा॒ क्षय॒म् क्षय॑म् त्वा त्वा॒ क्षय॑म् जिन्व जिन्व॒ क्षय॑म् त्वा त्वा॒ क्षय॑म् जिन्व । \newline
57. क्षय॑म् जिन्व जिन्व॒ क्षय॒म् क्षय॑म् जि॒न्वे तीति॑ जिन्व॒ क्षय॒म् क्षय॑म् जि॒न्वेति॑ । \newline
58. जि॒न्वे तीति॑ जिन्व जि॒न्वे त्या॑हा॒हेति॑ जिन्व जि॒न्वे त्या॑ह । \newline
59. इत्या॑हा॒हे तीत्या॑ह दे॒वा दे॒वा आ॒हे तीत्या॑ह दे॒वाः । \newline
\pagebreak
\markright{ TS 3.5.2.2  \hfill https://www.vedavms.in \hfill}

\section{ TS 3.5.2.2 }

\textbf{TS 3.5.2.2 } \newline
\textbf{Samhita Paata} \newline

-त्या॑ह दे॒वा वै क्षयो॑ दे॒वेभ्य॑ ए॒व य॒ज्ञ्ं प्राऽऽ*ह॒ प्रेति॑रसि॒ धर्मा॑य त्वा॒ धर्मं॑ जि॒न्वेत्या॑ह मनु॒ष्या॑ वै धर्मो॑ मनु॒ष्ये᳚भ्य ए॒व य॒ज्ञ्ं प्राऽऽ*हान्वि॑तिरसि दि॒वे त्वा॒ दिवं॑ जि॒न्वेत्या॑है॒भ्य ए॒व लो॒केभ्यो॑ य॒ज्ञ्ं प्राऽऽ*ह॑विष्ट॒म्भो॑ऽसि॒ वृष्ट्यै᳚ त्वा॒ वृष्टिं॑ जि॒न्वेत्या॑ह॒ वृष्टि॑मे॒वाव॑ - [  ] \newline

\textbf{Pada Paata} \newline

आ॒ह॒ । दे॒वाः । वै । क्षयः॑ । दे॒वेभ्यः॑ । ए॒व । य॒ज्ञ्म् । प्रेति॑ । आ॒ह॒ । प्रेति॒रिति॒ प्र - इ॒तिः॒ । अ॒सि॒ । धर्मा॑य । त्वा॒ । धर्म᳚म् । जि॒न्व॒ । इति॑ । आ॒ह॒ । म॒नु॒ष्याः᳚ । वै । धर्मः॑ । म॒नु॒ष्ये᳚भ्यः । ए॒व । य॒ज्ञ्म् । प्रेति॑ । आ॒ह॒ । अन्वि॑ति॒रित्यनु॑-इ॒तिः॒ । अ॒सि॒ । दि॒वे । त्वा॒ । दिव᳚म् । जि॒न्व॒ । इति॑ । आ॒ह॒ । ए॒भ्यः । ए॒व । लो॒केभ्यः॑ । य॒ज्ञ्म् । प्रेति॑ । आ॒ह॒ । वि॒ष्ट॒भं इति॑ वि - स्त॒भंः । अ॒सि॒ । वृष्ट्यै᳚ । त्वा॒ । वृष्टि᳚म् । जि॒न्व॒ । इति॑ । आ॒ह॒ । वृष्टि᳚म् । ए॒व । अवेति॑ ।  \newline


\textbf{Krama Paata} \newline

आ॒ह॒ दे॒वाः । दे॒वा वै । वै क्षयः॑ । क्षयो॑ दे॒वेभ्यः॑ । दे॒वेभ्य॑ ए॒व । ए॒व य॒ज्ञ्म् । य॒ज्ञ्म् प्र । प्राह॑ । आ॒ह॒ प्रेतिः॑ । प्रेति॑रसि । प्रेति॒रिति॒ प्र - इ॒तिः॒ । अ॒सि॒ धर्मा॑य । धर्मा॑य त्वा । त्वा॒ धर्म᳚म् । धर्मं॑ जिन्व । जि॒न्वेति॑ । इत्या॑ह । आ॒ह॒ म॒नु॒ष्याः᳚ । म॒नु॒ष्या॑ वै । वै धर्मः॑ । धर्मो॑ मनु॒ष्ये᳚भ्यः । म॒नु॒ष्ये᳚भ्य ए॒व । ए॒व य॒ज्ञ्म् । य॒ज्ञ्म् प्र । प्राह॑ । आ॒हान्वि॑तिः । अन्वि॑तिरसि । अन्वि॑ति॒रित्यनु॑ - इ॒तिः॒ । अ॒सि॒ दि॒वे । दि॒वे त्वा᳚ । त्वा॒ दिव᳚म् । दिव॑म् जिन्व । जि॒न्वेति॑ । इत्या॑ह । आ॒है॒भ्यः । ए॒भ्य ए॒व । ए॒व लो॒केभ्यः॑ । लो॒केभ्यो॑ य॒ज्ञ्म् । य॒ज्ञ्म् प्र । प्राह॑ । आ॒ह॒ वि॒ष्ट॒म्भः । वि॒ष्ट॒म्भो॑ ऽसि । वि॒ष्ट॒म्भ इति॑ वि - स्त॒म्भः । अ॒सि॒ वृष्ट्यै᳚ । वृष्ट्यै᳚ त्वा । त्वा॒ वृष्टि᳚म् । वृष्टि॑म् जिन्व । जि॒न्वेति॑ । इत्या॑ह । आ॒ह॒ वृष्टि᳚म् । वृष्टि॑मे॒व । ए॒वाव॑ । अव॑ रुन्धे \newline

\textbf{Jatai Paata} \newline

1. आ॒ह॒ दे॒वा दे॒वा आ॑हाह दे॒वाः । \newline
2. दे॒वा वै वै दे॒वा दे॒वा वै । \newline
3. वै क्षयः॒ क्षयो॒ वै वै क्षयः॑ । \newline
4. क्षयो॑ दे॒वेभ्यो॑ दे॒वेभ्यः॒ क्षयः॒ क्षयो॑ दे॒वेभ्यः॑ । \newline
5. दे॒वेभ्य॑ ए॒वैव दे॒वेभ्यो॑ दे॒वेभ्य॑ ए॒व । \newline
6. ए॒व य॒ज्ञ्ं ॅय॒ज्ञ् मे॒वैव य॒ज्ञ्म् । \newline
7. य॒ज्ञ्म् प्र प्र य॒ज्ञ्ं ॅय॒ज्ञ्म् प्र । \newline
8. प्राहा॑ह॒ प्र प्राह॑ । \newline
9. आ॒ह॒ प्रेतिः॒ प्रेति॑ राहाह॒ प्रेतिः॑ । \newline
10. प्रेति॑ रस्यसि॒ प्रेतिः॒ प्रेति॑ रसि । \newline
11. प्रेति॒रिति॒ प्र - इ॒तिः॒ । \newline
12. अ॒सि॒ धर्मा॑य॒ धर्मा॑या स्यसि॒ धर्मा॑य । \newline
13. धर्मा॑य त्वा त्वा॒ धर्मा॑य॒ धर्मा॑य त्वा । \newline
14. त्वा॒ धर्म॒म् धर्म॑म् त्वा त्वा॒ धर्म᳚म् । \newline
15. धर्म॑म् जिन्व जिन्व॒ धर्म॒म् धर्म॑म् जिन्व । \newline
16. जि॒न्वे तीति॑ जिन्व जि॒न्वेति॑ । \newline
17. इत्या॑ हा॒हे तीत्या॑ह । \newline
18. आ॒ह॒ म॒नु॒ष्या॑ मनु॒ष्या॑ आहाह मनु॒ष्याः᳚ । \newline
19. म॒नु॒ष्या॑ वै वै म॑नु॒ष्या॑ मनु॒ष्या॑ वै । \newline
20. वै धर्मो॒ धर्मो॒ वै वै धर्मः॑ । \newline
21. धर्मो॑ मनु॒ष्ये᳚भ्यो मनु॒ष्ये᳚भ्यो॒ धर्मो॒ धर्मो॑ मनु॒ष्ये᳚भ्यः । \newline
22. म॒नु॒ष्ये᳚भ्य ए॒वैव म॑नु॒ष्ये᳚भ्यो मनु॒ष्ये᳚भ्य ए॒व । \newline
23. ए॒व य॒ज्ञ्ं ॅय॒ज्ञ् मे॒वैव य॒ज्ञ्म् । \newline
24. य॒ज्ञ्म् प्र प्र य॒ज्ञ्ं ॅय॒ज्ञ्म् प्र । \newline
25. प्राहा॑ह॒ प्र प्राह॑ । \newline
26. आ॒हा न्वि॑ति॒ रन्वि॑ति राहा॒हा न्वि॑तिः । \newline
27. अन्वि॑ति रस्य॒स्य न्वि॑ति॒ रन्वि॑ति रसि । \newline
28. अन्वि॑ति॒रित्यनु॑ - इ॒तिः॒ । \newline
29. अ॒सि॒ दि॒वे दि॒वे᳚ ऽस्यसि दि॒वे । \newline
30. दि॒वे त्वा᳚ त्वा दि॒वे दि॒वे त्वा᳚ । \newline
31. त्वा॒ दिव॒म् दिव॑म् त्वा त्वा॒ दिव᳚म् । \newline
32. दिव॑म् जिन्व जिन्व॒ दिव॒म् दिव॑म् जिन्व । \newline
33. जि॒न्वे तीति॑ जिन्व जि॒न्वेति॑ । \newline
34. इत्या॑हा॒हे तीत्या॑ह । \newline
35. आ॒है॒भ्य ए॒भ्य आ॑हा है॒भ्यः । \newline
36. ए॒भ्य ए॒वैवैभ्य ए॒भ्य ए॒व । \newline
37. ए॒व लो॒केभ्यो॑ लो॒केभ्य॑ ए॒वैव लो॒केभ्यः॑ । \newline
38. लो॒केभ्यो॑ य॒ज्ञ्ं ॅय॒ज्ञ्म् ॅलो॒केभ्यो॑ लो॒केभ्यो॑ य॒ज्ञ्म् । \newline
39. य॒ज्ञ्म् प्र प्र य॒ज्ञ्ं ॅय॒ज्ञ्म् प्र । \newline
40. प्राहा॑ह॒ प्र प्राह॑ । \newline
41. आ॒ह॒ वि॒ष्ट॒म्भो वि॑ष्ट॒म्भ आ॑हाह विष्ट॒म्भः । \newline
42. वि॒ष्ट॒म्भो᳚ ऽस्यसि विष्ट॒म्भो वि॑ष्ट॒म्भो॑ ऽसि । \newline
43. वि॒ष्ट॒म्भ इति॑ वि - स्त॒म्भः । \newline
44. अ॒सि॒ वृष्ट्यै॒ वृष्ट्या॑ अस्यसि॒ वृष्ट्यै᳚ । \newline
45. वृष्ट्यै᳚ त्वा त्वा॒ वृष्ट्यै॒ वृष्ट्यै᳚ त्वा । \newline
46. त्वा॒ वृष्टिं॒ ॅवृष्टि॑म् त्वा त्वा॒ वृष्टि᳚म् । \newline
47. वृष्टि॑म् जिन्व जिन्व॒ वृष्टिं॒ ॅवृष्टि॑म् जिन्व । \newline
48. जि॒न्वे तीति॑ जिन्व जि॒न्वेति॑ । \newline
49. इत्या॑हा॒हे तीत्या॑ह । \newline
50. आ॒ह॒ वृष्टिं॒ ॅवृष्टि॑ माहाह॒ वृष्टि᳚म् । \newline
51. वृष्टि॑ मे॒वैव वृष्टिं॒ ॅवृष्टि॑ मे॒व । \newline
52. ए॒वावा वै॒वै वाव॑ । \newline
53. अव॑ रुन्धे रु॒न्धे ऽवाव॑ रुन्धे । \newline

\textbf{Ghana Paata } \newline

1. आ॒ह॒ दे॒वा दे॒वा आ॑हाह दे॒वा वै वै दे॒वा आ॑हाह दे॒वा वै । \newline
2. दे॒वा वै वै दे॒वा दे॒वा वै क्षयः॒ क्षयो॒ वै दे॒वा दे॒वा वै क्षयः॑ । \newline
3. वै क्षयः॒ क्षयो॒ वै वै क्षयो॑ दे॒वेभ्यो॑ दे॒वेभ्यः॒ क्षयो॒ वै वै क्षयो॑ दे॒वेभ्यः॑ । \newline
4. क्षयो॑ दे॒वेभ्यो॑ दे॒वेभ्यः॒ क्षयः॒ क्षयो॑ दे॒वेभ्य॑ ए॒वैव दे॒वेभ्यः॒ क्षयः॒ क्षयो॑ दे॒वेभ्य॑ ए॒व । \newline
5. दे॒वेभ्य॑ ए॒वैव दे॒वेभ्यो॑ दे॒वेभ्य॑ ए॒व य॒ज्ञ्ं ॅय॒ज्ञ् मे॒व दे॒वेभ्यो॑ दे॒वेभ्य॑ ए॒व य॒ज्ञ्म् । \newline
6. ए॒व य॒ज्ञ्ं ॅय॒ज्ञ् मे॒वैव य॒ज्ञ्म् प्र प्र य॒ज्ञ् मे॒वैव य॒ज्ञ्म् प्र । \newline
7. य॒ज्ञ्म् प्र प्र य॒ज्ञ्ं ॅय॒ज्ञ्म् प्राहा॑ह॒ प्र य॒ज्ञ्ं ॅय॒ज्ञ्म् प्राह॑ । \newline
8. प्राहा॑ह॒ प्र प्राह॒ प्रेतिः॒ प्रेति॑ राह॒ प्र प्राह॒ प्रेतिः॑ । \newline
9. आ॒ह॒ प्रेतिः॒ प्रेति॑ राहाह॒ प्रेति॑ रस्यसि॒ प्रेति॑ राहाह॒ प्रेति॑ रसि । \newline
10. प्रेति॑ रस्यसि॒ प्रेतिः॒ प्रेति॑ रसि॒ धर्मा॑य॒ धर्मा॑यासि॒ प्रेतिः॒ प्रेति॑ रसि॒ धर्मा॑य । \newline
11. प्रेति॒रिति॒ प्र - इ॒तिः॒ । \newline
12. अ॒सि॒ धर्मा॑य॒ धर्मा॑या स्यसि॒ धर्मा॑य त्वा त्वा॒ धर्मा॑या स्यसि॒ धर्मा॑य त्वा । \newline
13. धर्मा॑य त्वा त्वा॒ धर्मा॑य॒ धर्मा॑य त्वा॒ धर्म॒म् धर्म॑म् त्वा॒ धर्मा॑य॒ धर्मा॑य त्वा॒ धर्म᳚म् । \newline
14. त्वा॒ धर्म॒म् धर्म॑म् त्वा त्वा॒ धर्म॑म् जिन्व जिन्व॒ धर्म॑म् त्वा त्वा॒ धर्म॑म् जिन्व । \newline
15. धर्म॑म् जिन्व जिन्व॒ धर्म॒म् धर्म॑म् जि॒न्वे तीति॑ जिन्व॒ धर्म॒म् धर्म॑म् जि॒न्वेति॑ । \newline
16. जि॒न्वे तीति॑ जिन्व जि॒न्वे त्या॑हा॒हेति॑ जिन्व जि॒न्वे त्या॑ह । \newline
17. इत्या॑हा॒हे तीत्या॑ह मनु॒ष्या॑ मनु॒ष्या॑ आ॒हे तीत्या॑ह मनु॒ष्याः᳚ । \newline
18. आ॒ह॒ म॒नु॒ष्या॑ मनु॒ष्या॑ आहाह मनु॒ष्या॑ वै वै म॑नु॒ष्या॑ आहाह मनु॒ष्या॑ वै । \newline
19. म॒नु॒ष्या॑ वै वै म॑नु॒ष्या॑ मनु॒ष्या॑ वै धर्मो॒ धर्मो॒ वै म॑नु॒ष्या॑ मनु॒ष्या॑ वै धर्मः॑ । \newline
20. वै धर्मो॒ धर्मो॒ वै वै धर्मो॑ मनु॒ष्ये᳚भ्यो मनु॒ष्ये᳚भ्यो॒ धर्मो॒ वै वै धर्मो॑ मनु॒ष्ये᳚भ्यः । \newline
21. धर्मो॑ मनु॒ष्ये᳚भ्यो मनु॒ष्ये᳚भ्यो॒ धर्मो॒ धर्मो॑ मनु॒ष्ये᳚भ्य ए॒वैव म॑नु॒ष्ये᳚भ्यो॒ धर्मो॒ धर्मो॑ मनु॒ष्ये᳚भ्य ए॒व । \newline
22. म॒नु॒ष्ये᳚भ्य ए॒वैव म॑नु॒ष्ये᳚भ्यो मनु॒ष्ये᳚भ्य ए॒व य॒ज्ञ्ं ॅय॒ज्ञ् मे॒व म॑नु॒ष्ये᳚भ्यो 
मनु॒ष्ये᳚भ्य ए॒व य॒ज्ञ्म् । \newline
23. ए॒व य॒ज्ञ्ं ॅय॒ज्ञ् मे॒वैव य॒ज्ञ्म् प्र प्र य॒ज्ञ् मे॒वैव य॒ज्ञ्म् प्र । \newline
24. य॒ज्ञ्म् प्र प्र य॒ज्ञ्ं ॅय॒ज्ञ्म् प्रा हा॑ह॒ प्र य॒ज्ञ्ं ॅय॒ज्ञ्म् प्राह॑ । \newline
25. प्रा हा॑ह॒ प्र प्राहान्वि॑ति॒ रन्वि॑ति राह॒ प्र प्राहान्वि॑तिः । \newline
26. आ॒हा न्वि॑ति॒ रन्वि॑ति राहा॒हा न्वि॑ति रस्य॒स्य न्वि॑ति राहा॒हा न्वि॑ति रसि । \newline
27. अन्वि॑ति रस्य॒स्यन्वि॑ति॒ रन्वि॑ति रसि दि॒वे दि॒वे᳚ ऽस्यन्वि॑ति॒ रन्वि॑ति रसि दि॒वे । \newline
28. अन्वि॑ति॒रित्यनु॑ - इ॒तिः॒ । \newline
29. अ॒सि॒ दि॒वे दि॒वे᳚ ऽस्यसि दि॒वे त्वा᳚ त्वा दि॒वे᳚ ऽस्यसि दि॒वे त्वा᳚ । \newline
30. दि॒वे त्वा᳚ त्वा दि॒वे दि॒वे त्वा॒ दिव॒म् दिव॑म् त्वा दि॒वे दि॒वे त्वा॒ दिव᳚म् । \newline
31. त्वा॒ दिव॒म् दिव॑म् त्वा त्वा॒ दिव॑म् जिन्व जिन्व॒ दिव॑म् त्वा त्वा॒ दिव॑म् जिन्व । \newline
32. दिव॑म् जिन्व जिन्व॒ दिव॒म् दिव॑म् जि॒न्वे तीति॑ जिन्व॒ दिव॒म् दिव॑म् जि॒न्वेति॑ । \newline
33. जि॒न्वे तीति॑ जिन्व जि॒न्वे त्या॑हा॒हेति॑ जिन्व जि॒न्वे त्या॑ह । \newline
34. इत्या॑हा॒हे तीत्या॑ है॒भ्य ए॒भ्य आ॒हे तीत्या॑ है॒भ्यः । \newline
35. आ॒है॒भ्य ए॒भ्य आ॑हाहै॒भ्य ए॒वैवैभ्य आ॑हाहै॒भ्य ए॒व । \newline
36. ए॒भ्य ए॒वैवैभ्य ए॒भ्य ए॒व लो॒केभ्यो॑ लो॒केभ्य॑ ए॒वैभ्य ए॒भ्य ए॒व लो॒केभ्यः॑ । \newline
37. ए॒व लो॒केभ्यो॑ लो॒केभ्य॑ ए॒वैव लो॒केभ्यो॑ य॒ज्ञ्ं ॅय॒ज्ञ्म् ॅलो॒केभ्य॑ ए॒वैव लो॒केभ्यो॑ य॒ज्ञ्म् । \newline
38. लो॒केभ्यो॑ य॒ज्ञ्ं ॅय॒ज्ञ्म् ॅलो॒केभ्यो॑ लो॒केभ्यो॑ य॒ज्ञ्म् प्र प्र य॒ज्ञ्म् ॅलो॒केभ्यो॑ लो॒केभ्यो॑ य॒ज्ञ्म् प्र । \newline
39. य॒ज्ञ्म् प्र प्र य॒ज्ञ्ं ॅय॒ज्ञ्म् प्राहा॑ह॒ प्र य॒ज्ञ्ं ॅय॒ज्ञ्म् प्राह॑ । \newline
40. प्राहा॑ह॒ प्र प्राह॑ विष्ट॒म्भो वि॑ष्ट॒म्भ आ॑ह॒ प्र प्राह॑ विष्ट॒म्भः । \newline
41. आ॒ह॒ वि॒ष्ट॒म्भो वि॑ष्ट॒म्भ आ॑हाह विष्ट॒म्भो᳚ ऽस्यसि विष्ट॒म्भ आ॑हाह विष्ट॒म्भो॑ ऽसि । \newline
42. वि॒ष्ट॒म्भो᳚ ऽस्यसि विष्ट॒म्भो वि॑ष्ट॒म्भो॑ ऽसि॒ वृष्ट्यै॒ वृष्ट्या॑ असि विष्ट॒म्भो वि॑ष्ट॒म्भो॑ ऽसि॒ वृष्ट्यै᳚ । \newline
43. वि॒ष्ट॒म्भ इति॑ वि - स्त॒म्भः । \newline
44. अ॒सि॒ वृष्ट्यै॒ वृष्ट्या॑ अस्यसि॒ वृष्ट्यै᳚ त्वा त्वा॒ वृष्ट्या॑ अस्यसि॒ वृष्ट्यै᳚ त्वा । \newline
45. वृष्ट्यै᳚ त्वा त्वा॒ वृष्ट्यै॒ वृष्ट्यै᳚ त्वा॒ वृष्टिं॒ ॅवृष्टि॑म् त्वा॒ वृष्ट्यै॒ वृष्ट्यै᳚ त्वा॒ वृष्टि᳚म् । \newline
46. त्वा॒ वृष्टिं॒ ॅवृष्टि॑म् त्वा त्वा॒ वृष्टि॑म् जिन्व जिन्व॒ वृष्टि॑म् त्वा त्वा॒ वृष्टि॑म् जिन्व । \newline
47. वृष्टि॑म् जिन्व जिन्व॒ वृष्टिं॒ ॅवृष्टि॑म् जि॒न्वे तीति॑ जिन्व॒ वृष्टिं॒ ॅवृष्टि॑म् जि॒न्वेति॑ । \newline
48. जि॒न्वे तीति॑ जिन्व जि॒न्वे त्या॑हा॒हेति॑ जिन्व जि॒न्वे त्या॑ह । \newline
49. इत्या॑हा॒हेती त्या॑ह॒ वृष्टिं॒ ॅवृष्टि॑ मा॒हेती त्या॑ह॒ वृष्टि᳚म् । \newline
50. आ॒ह॒ वृष्टिं॒ ॅवृष्टि॑ माहाह॒ वृष्टि॑ मे॒वैव वृष्टि॑ माहाह॒ वृष्टि॑ मे॒व । \newline
51. वृष्टि॑ मे॒वैव वृष्टिं॒ ॅवृष्टि॑ मे॒वा वावै॒व वृष्टिं॒ ॅवृष्टि॑ मे॒वाव॑ । \newline
52. ए॒वावा वै॒वैवाव॑ रुन्धे रु॒न्धे ऽवै॒वैवाव॑ रुन्धे । \newline
53. अव॑ रुन्धे रु॒न्धे ऽवाव॑ रुन्धे प्र॒वा प्र॒वा रु॒न्धे ऽवाव॑ रुन्धे प्र॒वा । \newline
\pagebreak
\markright{ TS 3.5.2.3  \hfill https://www.vedavms.in \hfill}

\section{ TS 3.5.2.3 }

\textbf{TS 3.5.2.3 } \newline
\textbf{Samhita Paata} \newline

रुन्धे प्र॒वाऽस्य॑नु॒वाऽसीत्या॑ह मिथुन॒त्वायो॒शिग॑सि॒ वसु॑भ्यस्त्वा॒ वसू᳚ञ्जि॒न्वेत्या॑हा॒ष्टौ वस॑व॒ एका॑दश रु॒द्रा द्वाद॑शाऽऽ*दि॒त्या ए॒ताव॑न्तो॒ वै दे॒वास्तेभ्य॑ ए॒व य॒ज्ञ्ं प्राऽऽ*हौजो॑ऽसि पि॒तृभ्य॑स्त्वा पि॒तॄन् जि॒न्वेत्या॑ह दे॒वाने॒व पि॒तॄननु॒ सन्त॑नोति॒ तन्तु॑रसि प्र॒जाभ्य॑स्त्वा प्र॒जा जि॒न्वे - [  ] \newline

\textbf{Pada Paata} \newline

रु॒न्धे॒ । प्र॒वेति॑ प्र - वा । अ॒सि॒ । अ॒नु॒वेत्य॑नु - वा । अ॒सि॒ । इति॑ । आ॒ह॒ । मि॒थु॒न॒त्वायेति॑ मिथुन - त्वाय॑ । उ॒शिक् । अ॒सि॒ । वसु॑भ्य॒ इति॒ वसु॑ - भ्यः॒ । त्वा॒ । वसून्॑ । जि॒न्व॒ । इति॑ । आ॒ह॒ । अ॒ष्टौ । वस॑वः । एका॑दश । रु॒द्राः । द्वाद॑श । आ॒दि॒त्याः । ए॒ताव॑न्तः । वै । दे॒वाः । तेभ्यः॑ । ए॒व । य॒ज्ञ्म् । प्रेति॑ । आ॒ह॒ । ओजः॑ । अ॒सि॒ । पि॒तृभ्य॒ इति॑ पि॒तृ - भ्यः॒ । त्वा॒ । पि॒तॄन् । जि॒न्व॒ । इति॑ । आ॒ह॒ । दे॒वान् । ए॒व । पि॒तॄन् । अनु॑ । समिति॑ । त॒नो॒ति॒ । तन्तुः॑ । अ॒सि॒ । प्र॒जाभ्य॒ इति॑ प्र - जाभ्यः॑ । त्वा॒ । प्र॒जा इति॑ प्र - जाः । जि॒न्व॒ ।  \newline


\textbf{Krama Paata} \newline

रु॒न्धे॒ प्र॒वा । प्र॒वा ऽसि॑ । प्र॒वेति॑ प्र - वा । अ॒स्य॒नु॒वा । अ॒नु॒वा ऽसि॑ । अ॒नु॒वेत्य॑नु - वा । अ॒सीति॑ । इत्या॑ह । आ॒ह॒ मि॒थु॒न॒त्वाय॑ । मि॒थु॒न॒त्वायो॒शिक् । मि॒थु॒न॒त्वायेति॑ मिथुन - त्वाय॑ । उ॒शिग॑सि । अ॒सि॒ वसु॑भ्यः । वसु॑भ्यस्त्वा । वसु॑भ्य॒ इति॒ वसु॑ - भ्यः॒ । त्वा॒ वसून्॑ । वसू᳚न् जिन्व । जि॒न्वेति॑ । इत्या॑ह । आ॒हा॒ष्टौ । अ॒ष्टौ वस॑वः । वस॑व॒ एका॑दश । एका॑दश रु॒द्राः । रु॒द्रा द्वाद॑श । द्वाद॑शादि॒त्याः । आ॒दि॒त्या ए॒ताव॑न्तः । ए॒ताव॑न्तो॒ वै । वै दे॒वाः । दे॒वास्तेभ्यः॑ । तेभ्य॑ ए॒व । ए॒व य॒ज्ञ्म् । य॒ज्ञ्म् प्र । प्राह॑ । आ॒हौजः॑ । ओजो॑ ऽसि । अ॒सि॒ पि॒तृभ्यः॑ । पि॒तृभ्य॑स्त्वा । पि॒तृभ्य॒ इति॑ पि॒तृ - भ्यः॒ । त्वा॒ पि॒तॄन् । पि॒तॄन् जि॑न्व । जि॒न्वेति॑ । इत्या॑ह । आ॒ह॒ दे॒वान् । दे॒वाने॒व । ए॒व पि॒तॄन् । पि॒तॄननु॑ । अनु॒ सम् । सम् त॑नोति । त॒नो॒ति॒ तन्तुः॑ । तन्तु॑रसि । अ॒सि॒ प्र॒जाभ्यः॑ । प्र॒जाभ्य॑स्त्वा । प्र॒जाभ्य॒ इति॑ प्र - जाभ्यः॑ । त्वा॒ प्र॒जाः । प्र॒जा जि॑न्व । प्र॒जा इति॑ प्र - जाः । जि॒न्वेति॑ । इत्या॑ह \newline

\textbf{Jatai Paata} \newline

1. रु॒न्धे॒ प्र॒वा प्र॒वा रु॑न्धे रुन्धे प्र॒वा । \newline
2. प्र॒वा ऽस्य॑सि प्र॒वा प्र॒वा ऽसि॑ । \newline
3. प्र॒वेति॑ प्र - वा । \newline
4. अ॒स्य॒नु॒वा ऽनु॒वा ऽस्य॑ स्यनु॒वा । \newline
5. अ॒नु॒वा ऽस्य॑ स्यनु॒वा ऽनु॒वा ऽसि॑ । \newline
6. अ॒नु॒वेत्य॑नु - वा । \newline
7. अ॒सीती त्य॑स्य॒सीति॑ । \newline
8. इत्या॑हा॒हे तीत्या॑ह । \newline
9. आ॒ह॒ मि॒थु॒न॒त्वाय॑ मिथुन॒त्वाया॑ हाह मिथुन॒त्वाय॑ । \newline
10. मि॒थु॒न॒त्वा यो॒शि गु॒शिङ् मि॑थुन॒त्वाय॑ मिथुन॒त्वा यो॒शिक् । \newline
11. मि॒थु॒न॒त्वायेति॑ मिथुन - त्वाय॑ । \newline
12. उ॒शि ग॑स्य स्यु॒शि गु॒शि ग॑सि । \newline
13. अ॒सि॒ वसु॑भ्यो॒ वसु॑भ्यो ऽस्यसि॒ वसु॑भ्यः । \newline
14. वसु॑भ्य स्त्वा त्वा॒ वसु॑भ्यो॒ वसु॑भ्य स्त्वा । \newline
15. वसु॑भ्य॒ इति॒ वसु॑ - भ्यः॒ । \newline
16. त्वा॒ वसू॒न्॒. वसू᳚न् त्वा त्वा॒ वसून्॑ । \newline
17. वसू᳚न् जिन्व जिन्व॒ वसू॒न्॒. वसू᳚न् जिन्व । \newline
18. जि॒न्वे तीति॑ जिन्व जि॒न्वे ति॑ । \newline
19. इत्या॑हा॒हे तीत्या॑ह । \newline
20. आ॒हा॒ष्टा व॒ष्टा वा॑हा हा॒ष्टौ । \newline
21. अ॒ष्टौ वस॑वो॒ वस॑वो॒ ऽष्टा व॒ष्टौ वस॑वः । \newline
22. वस॑व॒ एका॑द॒ शैका॑दश॒ वस॑वो॒ वस॑व॒ एका॑दश । \newline
23. एका॑दश रु॒द्रा रु॒द्रा एका॑द॒ शैका॑दश रु॒द्राः । \newline
24. रु॒द्रा द्वाद॑श॒ द्वाद॑श रु॒द्रा रु॒द्रा द्वाद॑श । \newline
25. द्वाद॑शा दि॒त्या आ॑दि॒त्या द्वाद॑श॒ द्वाद॑शा दि॒त्याः । \newline
26. आ॒दि॒त्या ए॒ताव॑न्त ए॒ताव॑न्त आदि॒त्या आ॑दि॒त्या ए॒ताव॑न्तः । \newline
27. ए॒ताव॑न्तो॒ वै वा ए॒ताव॑न्त ए॒ताव॑न्तो॒ वै । \newline
28. वै दे॒वा दे॒वा वै वै दे॒वाः । \newline
29. दे॒वा स्तेभ्य॒ स्तेभ्यो॑ दे॒वा दे॒वा स्तेभ्यः॑ । \newline
30. तेभ्य॑ ए॒वैव तेभ्य॒ स्तेभ्य॑ ए॒व । \newline
31. ए॒व य॒ज्ञ्ं ॅय॒ज्ञ् मे॒वैव य॒ज्ञ्म् । \newline
32. य॒ज्ञ्म् प्र प्र य॒ज्ञ्ं ॅय॒ज्ञ्म् प्र । \newline
33. प्राहा॑ह॒ प्र प्राह॑ । \newline
34. आ॒हौज॒ ओज॑ आहा॒ हौजः॑ । \newline
35. ओजो᳚ ऽस्य॒स्योज॒ ओजो॑ ऽसि । \newline
36. अ॒सि॒ पि॒तृभ्यः॑ पि॒तृभ्यो᳚ ऽस्यसि पि॒तृभ्यः॑ । \newline
37. पि॒तृभ्य॑ स्त्वा त्वा पि॒तृभ्यः॑ पि॒तृभ्य॑ स्त्वा । \newline
38. पि॒तृभ्य॒ इति॑ पि॒तृ - भ्यः॒ । \newline
39. त्वा॒ पि॒तॄन् पि॒तॄꣳ स्त्वा᳚ त्वा पि॒तॄन् । \newline
40. पि॒तॄन् जि॑न्व जिन्व पि॒तॄन् पि॒तॄन् जि॑न्व । \newline
41. जि॒न्वे तीति॑ जिन्व जि॒न्वे ति॑ । \newline
42. इत्या॑ हा॒हे तीत्या॑ह । \newline
43. आ॒ह॒ दे॒वान् दे॒वा ना॑हाह दे॒वान् । \newline
44. दे॒वा ने॒वैव दे॒वान् दे॒वा ने॒व । \newline
45. ए॒व पि॒तॄन् पि॒तॄ ने॒वैव पि॒तॄन् । \newline
46. पि॒तॄ नन्वनु॑ पि॒तॄन् पि॒तॄ ननु॑ । \newline
47. अनु॒ सꣳ स मन्वनु॒ सम् । \newline
48. सम् त॑नोति तनोति॒ सꣳ सम् त॑नोति । \newline
49. त॒नो॒ति॒ तन्तु॒ स्तन्तु॑ स्तनोति तनोति॒ तन्तुः॑ । \newline
50. तन्तु॑ रस्यसि॒ तन्तु॒ स्तन्तु॑ रसि । \newline
51. अ॒सि॒ प्र॒जाभ्यः॑ प्र॒जाभ्यो᳚ ऽस्यसि प्र॒जाभ्यः॑ । \newline
52. प्र॒जाभ्य॑ स्त्वा त्वा प्र॒जाभ्यः॑ प्र॒जाभ्य॑ स्त्वा । \newline
53. प्र॒जाभ्य॒ इति॑ प्र - जाभ्यः॑ । \newline
54. त्वा॒ प्र॒जाः प्र॒जा स्त्वा᳚ त्वा प्र॒जाः । \newline
55. प्र॒जा जि॑न्व जिन्व प्र॒जाः प्र॒जा जि॑न्व । \newline
56. प्र॒जा इति॑ प्र - जाः । \newline
57. जि॒न्वे तीति॑ जिन्व जि॒न्वेति॑ । \newline

\textbf{Ghana Paata } \newline

1. रु॒न्धे॒ प्र॒वा प्र॒वा रु॑न्धे रुन्धे प्र॒वा ऽस्य॑सि प्र॒वा रु॑न्धे रुन्धे प्र॒वा ऽसि॑ । \newline
2. प्र॒वा ऽस्य॑सि प्र॒वा प्र॒वा ऽस्य॑नु॒वा ऽनु॒वा ऽसि॑ प्र॒वा प्र॒वा ऽस्य॑नु॒वा । \newline
3. प्र॒वेति॑ प्र - वा । \newline
4. अ॒स्य॒नु॒वा ऽनु॒वा ऽस्य॑स्यनु॒वा ऽस्य॑स्यनु॒वा ऽस्य॑स्यनु॒वा ऽसि॑ । \newline
5. अ॒नु॒वा ऽस्य॑स्यनु॒वा ऽनु॒वा ऽसीती त्य॑स्यनु॒वा ऽनु॒वा ऽसीति॑ । \newline
6. अ॒नु॒वेत्य॑नु - वा । \newline
7. अ॒सीती त्य॑स्य॒सी त्या॑हा॒हे त्य॑स्य॒सी त्या॑ह । \newline
8. इत्या॑हा॒हेती त्या॑ह मिथुन॒त्वाय॑ मिथुन॒त्वा या॒हेती त्या॑ह मिथुन॒त्वाय॑ । \newline
9. आ॒ह॒ मि॒थु॒न॒त्वाय॑ मिथुन॒त्वाया॑ हाह मिथुन॒त्वा यो॒शिगु॒शिङ् मि॑थुन॒त्वाया॑ हाह मिथुन॒त्वा यो॒शिक् । \newline
10. मि॒थु॒न॒त्वा यो॒शि गु॒शिङ् मि॑थुन॒त्वाय॑ मिथुन॒त्वा यो॒शिग॑स्य स्यु॒शिङ् मि॑थुन॒त्वाय॑ मिथुन॒त्वा यो॒शिग॑सि । \newline
11. मि॒थु॒न॒त्वायेति॑ मिथुन - त्वाय॑ । \newline
12. उ॒शिग॑स्य स्यु॒शि गु॒शिग॑सि॒ वसु॑भ्यो॒ वसु॑भ्यो ऽस्यु॒शि गु॒शिग॑सि॒ वसु॑भ्यः । \newline
13. अ॒सि॒ वसु॑भ्यो॒ वसु॑भ्यो ऽस्यसि॒ वसु॑भ्य स्त्वा त्वा॒ वसु॑भ्यो ऽस्यसि॒ वसु॑भ्य स्त्वा । \newline
14. वसु॑भ्य स्त्वा त्वा॒ वसु॑भ्यो॒ वसु॑भ्य स्त्वा॒ वसू॒न्॒. वसू᳚न् त्वा॒ वसु॑भ्यो॒ वसु॑भ्य स्त्वा॒ वसून्॑ । \newline
15. वसु॑भ्य॒ इति॒ वसु॑ - भ्यः॒ । \newline
16. त्वा॒ वसू॒न्॒. वसू᳚न् त्वा त्वा॒ वसू᳚न् जिन्व जिन्व॒ वसू᳚न् त्वा त्वा॒ वसू᳚न् जिन्व । \newline
17. वसू᳚न् जिन्व जिन्व॒ वसू॒न्॒. वसू᳚न् जि॒न्वे तीति॑ जिन्व॒ वसू॒न्॒. वसू᳚न् जि॒न्वेति॑ । \newline
18. जि॒न्वे तीति॑ जिन्व जि॒न्वे त्या॑हा॒हेति॑ जिन्व जि॒न्वे त्या॑ह । \newline
19. इत्या॑ हा॒हेती त्या॑हा॒ष्टा व॒ष्टा वा॒हेती त्या॑हा॒ष्टौ । \newline
20. आ॒हा॒ष्टा व॒ष्टा वा॑हा हा॒ष्टौ वस॑वो॒ वस॑वो॒ ऽष्टा वा॑हा हा॒ष्टौ वस॑वः । \newline
21. अ॒ष्टौ वस॑वो॒ वस॑वो॒ ऽष्टा व॒ष्टौ वस॑व॒ एका॑द॒ शैका॑दश॒ वस॑वो॒ ऽष्टा व॒ष्टौ वस॑व॒ एका॑दश । \newline
22. वस॑व॒ एका॑द॒ शैका॑दश॒ वस॑वो॒ वस॑व॒ एका॑दश रु॒द्रा रु॒द्रा एका॑दश॒ वस॑वो॒ वस॑व॒ एका॑दश रु॒द्राः । \newline
23. एका॑दश रु॒द्रा रु॒द्रा एका॑द॒ शैका॑दश रु॒द्रा द्वाद॑श॒ द्वाद॑श रु॒द्रा एका॑द॒ शैका॑दश रु॒द्रा द्वाद॑श । \newline
24. रु॒द्रा द्वाद॑श॒ द्वाद॑श रु॒द्रा रु॒द्रा द्वाद॑शादि॒त्या आ॑दि॒त्या द्वाद॑श रु॒द्रा रु॒द्रा द्वाद॑शादि॒त्याः । \newline
25. द्वाद॑शादि॒त्या आ॑दि॒त्या द्वाद॑श॒ द्वाद॑शादि॒त्या ए॒ताव॑न्त ए॒ताव॑न्त आदि॒त्या द्वाद॑श॒ द्वाद॑शादि॒त्या ए॒ताव॑न्तः । \newline
26. आ॒दि॒त्या ए॒ताव॑न्त ए॒ताव॑न्त आदि॒त्या आ॑दि॒त्या ए॒ताव॑न्तो॒ वै वा ए॒ताव॑न्त आदि॒त्या आ॑दि॒त्या ए॒ताव॑न्तो॒ वै । \newline
27. ए॒ताव॑न्तो॒ वै वा ए॒ताव॑न्त ए॒ताव॑न्तो॒ वै दे॒वा दे॒वा वा ए॒ताव॑न्त ए॒ताव॑न्तो॒ वै दे॒वाः । \newline
28. वै दे॒वा दे॒वा वै वै दे॒वा स्तेभ्य॒ स्तेभ्यो॑ दे॒वा वै वै दे॒वा स्तेभ्यः॑ । \newline
29. दे॒वा स्तेभ्य॒ स्तेभ्यो॑ दे॒वा दे॒वा स्तेभ्य॑ ए॒वैव तेभ्यो॑ दे॒वा दे॒वा स्तेभ्य॑ ए॒व । \newline
30. तेभ्य॑ ए॒वैव तेभ्य॒ स्तेभ्य॑ ए॒व य॒ज्ञ्ं ॅय॒ज्ञ् मे॒व तेभ्य॒ स्तेभ्य॑ ए॒व य॒ज्ञ्म् । \newline
31. ए॒व य॒ज्ञ्ं ॅय॒ज्ञ् मे॒वैव य॒ज्ञ्म् प्र प्र य॒ज्ञ् मे॒वैव य॒ज्ञ्म् प्र । \newline
32. य॒ज्ञ्म् प्र प्र य॒ज्ञ्ं ॅय॒ज्ञ्म् प्राहा॑ह॒ प्र य॒ज्ञ्ं ॅय॒ज्ञ्म् प्राह॑ । \newline
33. प्राहा॑ह॒ प्र प्राहौज॒ ओज॑ आह॒ प्र प्राहौजः॑ । \newline
34. आ॒हौज॒ ओज॑ आहा॒ हौजो᳚ ऽस्य॒स्योज॑ आहा॒ हौजो॑ ऽसि । \newline
35. ओजो᳚ ऽस्य॒स्योज॒ ओजो॑ ऽसि पि॒तृभ्यः॑ पि॒तृभ्यो॒ ऽस्योज॒ ओजो॑ ऽसि पि॒तृभ्यः॑ । \newline
36. अ॒सि॒ पि॒तृभ्यः॑ पि॒तृभ्यो᳚ ऽस्यसि पि॒तृभ्य॑ स्त्वा त्वा पि॒तृभ्यो᳚ ऽस्यसि पि॒तृभ्य॑ स्त्वा । \newline
37. पि॒तृभ्य॑ स्त्वा त्वा पि॒तृभ्यः॑ पि॒तृभ्य॑ स्त्वा पि॒तॄन् पि॒तॄꣳ स्त्वा॑ पि॒तृभ्यः॑ पि॒तृभ्य॑ स्त्वा पि॒तॄन् । \newline
38. पि॒तृभ्य॒ इति॑ पि॒तृ - भ्यः॒ । \newline
39. त्वा॒ पि॒तॄन् पि॒तॄꣳ स्त्वा᳚ त्वा पि॒तॄन् जि॑न्व जिन्व पि॒तॄꣳ स्त्वा᳚ त्वा पि॒तॄन् जि॑न्व । \newline
40. पि॒तॄन् जि॑न्व जिन्व पि॒तॄन् पि॒तॄन् जि॒न्वे तीति॑ जिन्व पि॒तॄन् पि॒तॄन् जि॒न्वेति॑ । \newline
41. जि॒न्वे तीति॑ जिन्व जि॒न्वे त्या॑हा॒हेति॑ जिन्व जि॒न्वे त्या॑ह । \newline
42. इत्या॑हा॒हे तीत्या॑ह दे॒वान् दे॒वा ना॒हे तीत्या॑ह दे॒वान् । \newline
43. आ॒ह॒ दे॒वान् दे॒वा ना॑हाह दे॒वा ने॒वैव दे॒वा ना॑हाह दे॒वा ने॒व । \newline
44. दे॒वा ने॒वैव दे॒वान् दे॒वा ने॒व पि॒तॄन् पि॒तॄ ने॒व दे॒वान् दे॒वा ने॒व पि॒तॄन् । \newline
45. ए॒व पि॒तॄन् पि॒तॄ ने॒वैव पि॒तॄ नन्वनु॑ पि॒तॄ ने॒वैव पि॒तॄ ननु॑ । \newline
46. पि॒तॄ नन्वनु॑ पि॒तॄन् पि॒तॄ ननु॒ सꣳ स मनु॑ पि॒तॄन् पि॒तॄ ननु॒ सम् । \newline
47. अनु॒ सꣳ स मन्वनु॒ सम् त॑नोति तनोति॒ स मन्वनु॒ सम् त॑नोति । \newline
48. सम् त॑नोति तनोति॒ सꣳ सम् त॑नोति॒ तन्तु॒ स्तन्तु॑ स्तनोति॒ सꣳ सम् त॑नोति॒ तन्तुः॑ । \newline
49. त॒नो॒ति॒ तन्तु॒ स्तन्तु॑ स्तनोति तनोति॒ तन्तु॑ रस्यसि॒ तन्तु॑ स्तनोति तनोति॒ तन्तु॑ रसि । \newline
50. तन्तु॑ रस्यसि॒ तन्तु॒ स्तन्तु॑ रसि प्र॒जाभ्यः॑ प्र॒जाभ्यो॑ ऽसि॒ तन्तु॒ स्तन्तु॑ रसि प्र॒जाभ्यः॑ । \newline
51. अ॒सि॒ प्र॒जाभ्यः॑ प्र॒जाभ्यो᳚ ऽस्यसि प्र॒जाभ्य॑ स्त्वा त्वा प्र॒जाभ्यो᳚ ऽस्यसि प्र॒जाभ्य॑ स्त्वा । \newline
52. प्र॒जाभ्य॑ स्त्वा त्वा प्र॒जाभ्यः॑ प्र॒जाभ्य॑ स्त्वा प्र॒जाः प्र॒जा स्त्वा᳚ प्र॒जाभ्यः॑ प्र॒जाभ्य॑ स्त्वा प्र॒जाः । \newline
53. प्र॒जाभ्य॒ इति॑ प्र - जाभ्यः॑ । \newline
54. त्वा॒ प्र॒जाः प्र॒जा स्त्वा᳚ त्वा प्र॒जा जि॑न्व जिन्व प्र॒जा स्त्वा᳚ त्वा प्र॒जा जि॑न्व । \newline
55. प्र॒जा जि॑न्व जिन्व प्र॒जाः प्र॒जा जि॒न्वे तीति॑ जिन्व प्र॒जाः प्र॒जा जि॒न्वेति॑ । \newline
56. प्र॒जा इति॑ प्र - जाः । \newline
57. जि॒न्वे तीति॑ जिन्व जि॒न्वे त्या॑हा॒ हेति॑ जिन्व जि॒न्वे त्या॑ह । \newline
\pagebreak
\markright{ TS 3.5.2.4  \hfill https://www.vedavms.in \hfill}

\section{ TS 3.5.2.4 }

\textbf{TS 3.5.2.4 } \newline
\textbf{Samhita Paata} \newline

-त्या॑ह पि॒तॄने॒व प्र॒जा अनु॒ सन्त॑नोति पृतना॒षाड॑सि प॒शुभ्य॑स्त्वा प॒शूञ्जि॒न्वेत्या॑ह प्र॒जा ए॒व प॒शूननु॒ सन्त॑नोतिरे॒वद॒स्यो-ष॑धीभ्य॒ स्त्वौष॑धी-र्जि॒न्वेत्या॒हौष॑धीष्वे॒व प॒शून् प्रति॑ष्ठापयत्यभि॒जिद॑सि यु॒क्तग्रा॒वेन्द्रा॑य॒ त्वेन्द्रं॑ जि॒न्वेत्या॑हा॒भिजि॑त्या॒ अधि॑पतिरसि प्रा॒णाय॑ त्वा प्रा॒णं - [  ] \newline

\textbf{Pada Paata} \newline

इति॑ । आ॒ह॒ । पि॒तॄन् । ए॒व । प्र॒जा इति॑ प्र - जाः । अनु॑ । समिति॑ । त॒नो॒ति॒ । पृ॒त॒ना॒षाट् । अ॒सि॒ । प॒शुभ्य॒ इति॑ प॒शु - भ्यः॒ । त्वा॒ । प॒शून् । जि॒न्व॒ । इति॑ । आ॒ह॒ । प्र॒जा इति॑ प्र - जाः । ए॒व । प॒शून् । अनु॑ । समिति॑ । त॒नो॒ति॒ । रे॒वत् । अ॒सि॒ । ओष॑धीभ्य॒ इत्योष॑धि - भ्यः॒ । त्वा॒ । ओष॑धीः । जि॒न्व॒ । इति॑ । आ॒ह॒ । ओष॑धीषु । ए॒व । प॒शून् । प्रतीति॑ । स्था॒प॒य॒ति॒ । अ॒भि॒जिदित्य॑भि - जित् । अ॒सि॒ । यु॒क्तग्रा॒वेति॑ यु॒क्त - ग्रा॒वा॒ । इन्द्रा॑य । त्वा॒ । इन्द्र᳚म् । जि॒न्व॒ । इति॑ । आ॒ह॒ । अ॒भिजि॑त्या॒ इत्य॒भि - जि॒त्यै॒ । अधि॑पति॒रित्यधि॑ - प॒तिः॒ । अ॒सि॒ । प्रा॒णायेति॑ प्र - अ॒नाय॑ । त्वा॒ । प्रा॒णमिति॑ प्र-अ॒नम् ।  \newline


\textbf{Krama Paata} \newline

आ॒ह॒ पि॒तॄन् । पि॒तॄने॒व । ए॒व प्र॒जाः । प्र॒जा अनु॑ । प्र॒जा इति॑ प्र - जाः । अनु॒ सम् । सम् त॑नोति । त॒नो॒ति॒ पृ॒त॒ना॒षाट् । पृ॒त॒ना॒षाड॑सि । अ॒सि॒ प॒शुभ्यः॑ । प॒शुभ्य॑स्त्वा । प॒शुभ्य॒ इति॑ प॒शु - भ्यः॒ । त्वा॒ प॒शून् । प॒शून् जि॑न्व । जि॒न्वेति॑ । इत्या॑ह । आ॒ह॒ प्र॒जाः । प्र॒जा ए॒व । प्र॒जा इति॑ प्र - जाः । ए॒व प॒शून् । प॒शूननु॑ । अनु॒ सम् । सम् त॑नोति । त॒नो॒ति॒ रे॒वत् । रे॒वद॑सि । अ॒स्योष॑धीभ्यः । ओष॑धीभ्यस्त्वा । ओष॑धीभ्य॒ इत्योष॑धि - भ्यः॒ । त्वौष॑धीः । ओष॑धीर् जिन्व । जि॒न्वेति॑ । इत्या॑ह । आ॒हौष॑धीषु । ओष॑धीष्वे॒व । ए॒व प॒शून् । प॒शून् प्रति॑ । प्रति॑ ष्ठापयति । स्था॒प॒य॒त्य॒भि॒जित् । अ॒भि॒जिद॑सि । अ॒भि॒जिदित्य॑भि - जित् । अ॒सि॒ यु॒क्तग्रा॑वा । यु॒क्तग्रा॒वेन्द्रा॑य । यु॒क्तग्रा॒वेति॑ यु॒क्त - ग्रा॒वा॒ । इन्द्रा॑य त्वा । त्वेन्द्र᳚म् । इन्द्र॑म् जिन्व । जि॒न्वेति॑ । इत्या॑ह । आ॒हा॒भिजि॑त्यै । अ॒भिजि॑त्या॒ अधि॑पतिः । अ॒भिजि॑त्या॒ इत्य॒भि - जि॒त्यै॒ । अधि॑पतिरसि । अधि॑पति॒रित्यधि॑ - प॒तिः॒ । अ॒सि॒ प्रा॒णाय॑ । प्रा॒णाय॑ त्वा । प्रा॒णायेति॑ प्र - अ॒नाय॑ । त्वा॒ प्रा॒णम् ( ) । प्रा॒णम् जि॑न्व । प्रा॒णमिति॑ प्र - अ॒नम् \newline

\textbf{Jatai Paata} \newline

1. इत्या॑हा॒हे तीत्या॑ह । \newline
2. आ॒ह॒ पि॒तॄन् पि॒तॄ ना॑हाह पि॒तॄन् । \newline
3. पि॒तॄ ने॒वैव पि॒तॄन् पि॒तॄ ने॒व । \newline
4. ए॒व प्र॒जाः प्र॒जा ए॒वैव प्र॒जाः । \newline
5. प्र॒जा अन्वनु॑ प्र॒जाः प्र॒जा अनु॑ । \newline
6. प्र॒जा इति॑ प्र - जाः । \newline
7. अनु॒ सꣳ स मन्वनु॒ सम् । \newline
8. सम् त॑नोति तनोति॒ सꣳ सम् त॑नोति । \newline
9. त॒नो॒ति॒ पृ॒त॒ना॒षाट् पृ॑तना॒षाट् त॑नोति तनोति पृतना॒षाट् । \newline
10. पृ॒त॒ना॒षा ड॑स्यसि पृतना॒षाट् पृ॑तना॒षा ड॑सि । \newline
11. अ॒सि॒ प॒शुभ्यः॑ प॒शुभ्यो᳚ ऽस्यसि प॒शुभ्यः॑ । \newline
12. प॒शुभ्य॑ स्त्वा त्वा प॒शुभ्यः॑ प॒शुभ्य॑ स्त्वा । \newline
13. प॒शुभ्य॒ इति॑ प॒शु - भ्यः॒ । \newline
14. त्वा॒ प॒शून् प॒शून् त्वा᳚ त्वा प॒शून् । \newline
15. प॒शून् जि॑न्व जिन्व प॒शून् प॒शून् जि॑न्व । \newline
16. जि॒न्वे तीति॑ जिन्व जि॒न्वेति॑ । \newline
17. इत्या॑ हा॒हे तीत्या॑ह । \newline
18. आ॒ह॒ प्र॒जाः प्र॒जा आ॑हाह प्र॒जाः । \newline
19. प्र॒जा ए॒वैव प्र॒जाः प्र॒जा ए॒व । \newline
20. प्र॒जा इति॑ प्र - जाः । \newline
21. ए॒व प॒शून् प॒शू ने॒वैव प॒शून् । \newline
22. प॒शू नन्वनु॑ प॒शून् प॒शू ननु॑ । \newline
23. अनु॒ सꣳ स मन्वनु॒ सम् । \newline
24. सम् त॑नोति तनोति॒ सꣳ सम् त॑नोति । \newline
25. त॒नो॒ति॒ रे॒वद् रे॒वत् त॑नोति तनोति रे॒वत् । \newline
26. रे॒व द॑स्यसि रे॒वद् रे॒व द॑सि । \newline
27. अ॒स्योष॑धीभ्य॒ ओष॑धीभ्यो ऽस्य॒ स्योष॑धीभ्यः । \newline
28. ओष॑धीभ्य स्त्वा॒ त्वौष॑धीभ्य॒ ओष॑धीभ्य स्त्वा । \newline
29. ओष॑धीभ्य॒ इत्योष॑धि - भ्यः॒ । \newline
30. त्वौष॑धी॒ रोष॑धी स्त्वा॒ त्वौष॑धीः । \newline
31. ओष॑धीर् जिन्व जि॒न्वौष॑धी॒ रोष॑धीर् जिन्व । \newline
32. जि॒न्वे तीति॑ जिन्व जि॒न्वे ति॑ । \newline
33. इत्या॑ हा॒हे तीत्या॑ह । \newline
34. आ॒हौष॑धी॒ ष्वोष॑धी ष्वाहा॒ हौष॑धीषु । \newline
35. ओष॑धी ष्वे॒वै वौष॑धी॒ ष्वोष॑धी ष्वे॒व । \newline
36. ए॒व प॒शून् प॒शू ने॒वैव प॒शून् । \newline
37. प॒शून् प्रति॒ प्रति॑ प॒शून् प॒शून् प्रति॑ । \newline
38. प्रति॑ ष्ठापयति स्थापयति॒ प्रति॒ प्रति॑ ष्ठापयति । \newline
39. स्था॒प॒य॒ त्य॒भि॒जि द॑भि॒जिथ् स्था॑पयति स्थापय त्यभि॒जित् । \newline
40. अ॒भि॒जि द॑स्य स्यभि॒जि द॑भि॒जि द॑सि । \newline
41. अ॒भि॒जिदित्य॑भि - जित् । \newline
42. अ॒सि॒ यु॒क्तग्रा॑वा यु॒क्तग्रा॑वा ऽस्यसि यु॒क्तग्रा॑वा । \newline
43. यु॒क्तग्रा॒वेन्द्रा॒ येन्द्रा॑य यु॒क्तग्रा॑वा यु॒क्तग्रा॒वेन्द्रा॑य । \newline
44. यु॒क्तग्रा॒वेति॑ यु॒क्त - ग्रा॒वा॒ । \newline
45. इन्द्रा॑य त्वा॒ त्वेन्द्रा॒ये न्द्रा॑य त्वा । \newline
46. त्वेन्द्र॒ मिन्द्र॑म् त्वा॒ त्वेन्द्र᳚म् । \newline
47. इन्द्र॑म् जिन्व जि॒न्वेन्द्र॒ मिन्द्र॑म् जिन्व । \newline
48. जि॒न्वे तीति॑ जिन्व जि॒न्वेति॑ । \newline
49. इत्या॑ हा॒हे तीत्या॑ह । \newline
50. आ॒हा॒भिजि॑त्या अ॒भिजि॑त्या आहाहा॒ भिजि॑त्यै । \newline
51. अ॒भिजि॑त्या॒ अधि॑पति॒ रधि॑पति र॒भिजि॑त्या अ॒भिजि॑त्या॒ अधि॑पतिः । \newline
52. अ॒भिजि॑त्या॒ इत्य॒भि - जि॒त्यै॒ । \newline
53. अधि॑पति रस्य॒ स्यधि॑पति॒ रधि॑पति रसि । \newline
54. अधि॑पति॒रित्यधि॑ - प॒तिः॒ । \newline
55. अ॒सि॒ प्रा॒णाय॑ प्रा॒णाया᳚ स्यसि प्रा॒णाय॑ । \newline
56. प्रा॒णाय॑ त्वा त्वा प्रा॒णाय॑ प्रा॒णाय॑ त्वा । \newline
57. प्रा॒णायेति॑ प्र - अ॒नाय॑ । \newline
58. त्वा॒ प्रा॒णम् प्रा॒णम् त्वा᳚ त्वा प्रा॒णम् । \newline
59. प्रा॒णम् जि॑न्व जिन्व प्रा॒णम् प्रा॒णम् जि॑न्व । \newline
60. प्रा॒णमिति॑ प्र - अ॒नम् । \newline

\textbf{Ghana Paata } \newline

1. इत्या॑हा॒हेती त्या॑ह पि॒तॄन् पि॒तॄ ना॒हेती त्या॑ह पि॒तॄन् । \newline
2. आ॒ह॒ पि॒तॄन् पि॒तॄ ना॑हाह पि॒तॄ ने॒वैव पि॒तॄ ना॑हाह पि॒तॄ ने॒व । \newline
3. पि॒तॄ ने॒वैव पि॒तॄन् पि॒तॄ ने॒व प्र॒जाः प्र॒जा ए॒व पि॒तॄन् पि॒तॄ ने॒व प्र॒जाः । \newline
4. ए॒व प्र॒जाः प्र॒जा ए॒वैव प्र॒जा अन्वनु॑ प्र॒जा ए॒वैव प्र॒जा अनु॑ । \newline
5. प्र॒जा अन्वनु॑ प्र॒जाः प्र॒जा अनु॒ सꣳ स मनु॑ प्र॒जाः प्र॒जा अनु॒ सम् । \newline
6. प्र॒जा इति॑ प्र - जाः । \newline
7. अनु॒ सꣳ स मन्वनु॒ सम् त॑नोति तनोति॒ स मन्वनु॒ सम् त॑नोति । \newline
8. सम् त॑नोति तनोति॒ सꣳ सम् त॑नोति पृतना॒षाट् पृ॑तना॒षाट् त॑नोति॒ सꣳ सम् त॑नोति पृतना॒षाट् । \newline
9. त॒नो॒ति॒ पृ॒त॒ना॒षाट् पृ॑तना॒षाट् त॑नोति तनोति पृतना॒षा ड॑स्यसि पृतना॒षाट् त॑नोति तनोति पृतना॒षा ड॑सि । \newline
10. पृ॒त॒ना॒षा ड॑स्यसि पृतना॒षाट् पृ॑तना॒षा ड॑सि प॒शुभ्यः॑ प॒शुभ्यो॑ ऽसि पृतना॒षाट् पृ॑तना॒षा ड॑सि प॒शुभ्यः॑ । \newline
11. अ॒सि॒ प॒शुभ्यः॑ प॒शुभ्यो᳚ ऽस्यसि प॒शुभ्य॑ स्त्वा त्वा प॒शुभ्यो᳚ ऽस्यसि प॒शुभ्य॑ स्त्वा । \newline
12. प॒शुभ्य॑ स्त्वा त्वा प॒शुभ्यः॑ प॒शुभ्य॑ स्त्वा प॒शून् प॒शून् त्वा॑ प॒शुभ्यः॑ प॒शुभ्य॑ स्त्वा प॒शून् । \newline
13. प॒शुभ्य॒ इति॑ प॒शु - भ्यः॒ । \newline
14. त्वा॒ प॒शून् प॒शून् त्वा᳚ त्वा प॒शून् जि॑न्व जिन्व प॒शून् त्वा᳚ त्वा प॒शून् जि॑न्व । \newline
15. प॒शून् जि॑न्व जिन्व प॒शून् प॒शून् जि॒न्वे तीति॑ जिन्व प॒शून् प॒शून् जि॒न्वेति॑ । \newline
16. जि॒न्वे तीति॑ जिन्व जि॒न्वे त्या॑हा॒हेति॑ जिन्व जि॒न्वे त्या॑ह । \newline
17. इत्या॑हा॒हे तीत्या॑ह प्र॒जाः प्र॒जा आ॒हे तीत्या॑ह प्र॒जाः । \newline
18. आ॒ह॒ प्र॒जाः प्र॒जा आ॑हाह प्र॒जा ए॒वैव प्र॒जा आ॑हाह प्र॒जा ए॒व । \newline
19. प्र॒जा ए॒वैव प्र॒जाः प्र॒जा ए॒व प॒शून् प॒शू ने॒व प्र॒जाः प्र॒जा ए॒व प॒शून् । \newline
20. प्र॒जा इति॑ प्र - जाः । \newline
21. ए॒व प॒शून् प॒शू ने॒वैव प॒शू नन्वनु॑ प॒शू ने॒वैव प॒शू ननु॑ । \newline
22. प॒शू नन्वनु॑ प॒शून् प॒शू ननु॒ सꣳ स मनु॑ प॒शून् प॒शू ननु॒ सम् । \newline
23. अनु॒ सꣳ स मन्वनु॒ सम् त॑नोति तनोति॒ स मन्वनु॒ सम् त॑नोति । \newline
24. सम् त॑नोति तनोति॒ सꣳ सम् त॑नोति रे॒वद् रे॒वत् त॑नोति॒ सꣳ सम् त॑नोति रे॒वत् । \newline
25. त॒नो॒ति॒ रे॒वद् रे॒वत् त॑नोति तनोति रे॒व द॑स्यसि रे॒वत् त॑नोति तनोति रे॒व द॑सि । \newline
26. रे॒व द॑स्यसि रे॒वद् रे॒व द॒स्योष॑धीभ्य॒ ओष॑धीभ्यो ऽसि रे॒वद् रे॒व द॒स्योष॑धीभ्यः । \newline
27. अ॒स्योष॑धीभ्य॒ ओष॑धीभ्यो ऽस्य॒स्योष॑धीभ्य स्त्वा॒ त्वौष॑धीभ्यो ऽस्य॒स्योष॑धीभ्य स्त्वा । \newline
28. ओष॑धीभ्य स्त्वा॒ त्वौष॑धीभ्य॒ ओष॑धीभ्य॒ स्त्वौष॑धी॒ रोष॑धी॒ स्त्वौष॑धीभ्य॒ ओष॑धीभ्य॒ स्त्वौष॑धीः । \newline
29. ओष॑धीभ्य॒ इत्योष॑धि - भ्यः॒ । \newline
30. त्वौष॑धी॒ रोष॑धी स्त्वा॒ त्वौष॑धीर् जिन्व जि॒न्वौष॑धी स्त्वा॒ त्वौष॑धीर् जिन्व । \newline
31. ओष॑धीर् जिन्व जि॒न्वौष॑धी॒ रोष॑धीर् जि॒न्वे तीति॑ जि॒न्वौष॑धी॒ रोष॑धीर् जि॒न्वेति॑ । \newline
32. जि॒न्वे तीति॑ जिन्व जि॒न्वे त्या॑हा॒हेति॑ जिन्व जि॒न्वे त्या॑ह । \newline
33. इत्या॑हा॒हेती त्या॒हौष॑धी॒ ष्वोष॑धी ष्वा॒हेती त्या॒हौष॑धीषु । \newline
34. आ॒हौष॑धी॒ ष्वोष॑धी ष्वाहा॒ हौष॑धी ष्वे॒वैवौष॑धी ष्वाहा॒ हौष॑धी ष्वे॒व । \newline
35. ओष॑धी ष्वे॒वैवौष॑धी॒ ष्वोष॑धी ष्वे॒व प॒शून् प॒शू ने॒वौष॑धी॒ ष्वोष॑धी ष्वे॒व प॒शून् । \newline
36. ए॒व प॒शून् प॒शू ने॒वैव प॒शून् प्रति॒ प्रति॑ प॒शू ने॒वैव प॒शून् प्रति॑ । \newline
37. प॒शून् प्रति॒ प्रति॑ प॒शून् प॒शून् प्रति॑ ष्ठापयति स्थापयति॒ प्रति॑ प॒शून् प॒शून् प्रति॑ ष्ठापयति । \newline
38. प्रति॑ ष्ठापयति स्थापयति॒ प्रति॒ प्रति॑ ष्ठापय त्यभि॒जि द॑भि॒जिथ् स्था॑पयति॒ प्रति॒ प्रति॑ ष्ठापय त्यभि॒जित् । \newline
39. स्था॒प॒य॒ त्य॒भि॒जि द॑भि॒जिथ् स्था॑पयति स्थापय त्यभि॒जि द॑स्यस्यभि॒जिथ् स्था॑पयति स्थापय त्यभि॒जि द॑सि । \newline
40. अ॒भि॒जि द॑स्यस्यभि॒जि द॑भि॒जि द॑सि यु॒क्तग्रा॑वा यु॒क्तग्रा॑वा ऽस्यभि॒जि द॑भि॒जि द॑सि यु॒क्तग्रा॑वा । \newline
41. अ॒भि॒जिदित्य॑भि - जित् । \newline
42. अ॒सि॒ यु॒क्तग्रा॑वा यु॒क्तग्रा॑वा ऽस्यसि यु॒क्तग्रा॒ वेन्द्रा॒ येन्द्रा॑य यु॒क्तग्रा॑वा ऽस्यसि यु॒क्तग्रा॒ वेन्द्रा॑य । \newline
43. यु॒क्तग्रा॒ वेन्द्रा॒ये न्द्रा॑य यु॒क्तग्रा॑वा यु॒क्तग्रा॒ वेन्द्रा॑य त्वा॒ त्वेन्द्रा॑य यु॒क्तग्रा॑वा यु॒क्तग्रा॒ वेन्द्रा॑य त्वा । \newline
44. यु॒क्तग्रा॒वेति॑ यु॒क्त - ग्रा॒वा॒ । \newline
45. इन्द्रा॑य त्वा॒ त्वेन्द्रा॒ येन्द्रा॑य॒ त्वेन्द्र॒ मिन्द्र॒म् त्वेन्द्रा॒ येन्द्रा॑य॒ त्वेन्द्र᳚म् । \newline
46. त्वेन्द्र॒ मिन्द्र॑म् त्वा॒ त्वेन्द्र॑म् जिन्व जि॒न्वेन्द्र॑म् त्वा॒ त्वेन्द्र॑म् जिन्व । \newline
47. इन्द्र॑म् जिन्व जि॒न्वेन्द्र॒ मिन्द्र॑म् जि॒न्वे तीति॑ जि॒न्वेन्द्र॒ मिन्द्र॑म् जि॒न्वेति॑ । \newline
48. जि॒न्वे तीति॑ जिन्व जि॒न्वे त्या॑हा॒हेति॑ जिन्व जि॒न्वे त्या॑ह । \newline
49. इत्या॑हा॒हेती त्या॑हा॒ भिजि॑त्या अ॒भिजि॑त्या आ॒हेती त्या॑हा॒ भिजि॑त्यै । \newline
50. आ॒हा॒ भिजि॑त्या अ॒भिजि॑त्या आहाहा॒ भिजि॑त्या॒ अधि॑पति॒ रधि॑पति र॒भिजि॑त्या आहाहा॒ भिजि॑त्या॒ अधि॑पतिः । \newline
51. अ॒भिजि॑त्या॒ अधि॑पति॒ रधि॑पति र॒भिजि॑त्या अ॒भिजि॑त्या॒ अधि॑पति रस्य॒स्यधि॑पति र॒भिजि॑त्या अ॒भिजि॑त्या॒ अधि॑पति रसि । \newline
52. अ॒भिजि॑त्या॒ इत्य॒भि - जि॒त्यै॒ । \newline
53. अधि॑पति रस्य॒स्यधि॑पति॒ रधि॑पति रसि प्रा॒णाय॑ प्रा॒णाया॒ स्यधि॑पति॒ रधि॑पति रसि प्रा॒णाय॑ । \newline
54. अधि॑पति॒रित्यधि॑ - प॒तिः॒ । \newline
55. अ॒सि॒ प्रा॒णाय॑ प्रा॒णाया᳚स्यसि प्रा॒णाय॑ त्वा त्वा प्रा॒णाया᳚स्यसि प्रा॒णाय॑ त्वा । \newline
56. प्रा॒णाय॑ त्वा त्वा प्रा॒णाय॑ प्रा॒णाय॑ त्वा प्रा॒णम् प्रा॒णम् त्वा᳚ प्रा॒णाय॑ प्रा॒णाय॑ त्वा प्रा॒णम् । \newline
57. प्रा॒णायेति॑ प्र - अ॒नाय॑ । \newline
58. त्वा॒ प्रा॒णम् प्रा॒णम् त्वा᳚ त्वा प्रा॒णम् जि॑न्व जिन्व प्रा॒णम् त्वा᳚ त्वा प्रा॒णम् जि॑न्व । \newline
59. प्रा॒णम् जि॑न्व जिन्व प्रा॒णम् प्रा॒णम् जि॒न्वे तीति॑ जिन्व प्रा॒णम् प्रा॒णम् जि॒न्वेति॑ । \newline
60. प्रा॒णमिति॑ प्र - अ॒नम् । \newline
\pagebreak
\markright{ TS 3.5.2.5  \hfill https://www.vedavms.in \hfill}

\section{ TS 3.5.2.5 }

\textbf{TS 3.5.2.5 } \newline
\textbf{Samhita Paata} \newline

जि॒न्वेत्या॑ह प्र॒जास्वे॒व प्रा॒णान् द॑धाति त्रि॒वृद॑सि प्र॒वृद॒सीत्या॑ह मिथुन॒त्वाय॑ सꣳरो॒हो॑ऽसि नीरो॒हो॑ऽसीत्या॑ह॒ प्रजा᳚त्यै वसु॒को॑ऽसि॒ वेष॑श्रिरसि॒ वस्य॑ष्टिर॒सीत्या॑ह॒ प्रति॑ष्ठित्यै ॥ \newline

\textbf{Pada Paata} \newline

जि॒न्व॒ । इति॑ । आ॒ह॒ । प्र॒जास्विति॑ प्र - जासु॑ । ए॒व । प्रा॒णानिति॑ प्र - अ॒नान् । द॒धा॒ति॒ । त्रि॒वृदिति॑ त्रि - वृत् । अ॒सि॒ । प्र॒वृदिति॑ प्र - वृत् । अ॒सि॒ । इति॑ । आ॒ह॒ । मि॒थु॒न॒त्वायेति॑ मिथुन - त्वाय॑ । सꣳ॒॒रो॒ह इति॑ सं - रो॒हः । अ॒सि॒ । नी॒रो॒ह इति॑ निः - रो॒हः । अ॒सि॒ । इति॑ । आ॒ह॒ । प्रजा᳚त्या॒ इति॒ प्र - जा॒त्यै॒ । व॒सु॒कः । अ॒सि॒ । वेष॑श्रि॒रिति॒ वेष॑- श्रिः॒ । अ॒सि॒ । वस्य॑ष्टिः । अ॒सि॒ । इति॑ । आ॒ह॒ । प्रति॑ष्ठित्या॒ इति॒ प्रति॑ - स्थि॒त्यै॒ ॥  \newline


\textbf{Krama Paata} \newline

जि॒न्वेति॑ । इत्या॑ह । आ॒ह॒ प्र॒जासु॑ । प्र॒जास्वे॒व । प्र॒जास्विति॑ प्र - जासु॑ । ए॒व प्रा॒णान् । प्रा॒णान् द॑धाति । प्रा॒णानिति॑ प्र - अ॒नान् । द॒धा॒ति॒ त्रि॒वृत् । त्रि॒वृद॑सि । त्रि॒वृदिति॑ त्रि - वृत् । अ॒सि॒ प्र॒वृत् । प्र॒वृद॑सि । प्र॒वृदिति॑ प्र - वृत् । अ॒सीति॑ । इत्या॑ह । आ॒ह॒ मि॒थु॒न॒त्वाय॑ । मि॒थु॒न॒त्वाय॑ सꣳरो॒हः । मि॒थु॒न॒त्वायेति॑ मिथुन - त्वाय॑ । सꣳ॒॒रो॒हो॑ ऽसि । सꣳ॒॒रो॒ह इति॑ सम् - रो॒हः । अ॒सि॒ नी॒रो॒हः । नी॒रो॒हो॑ ऽसि । नी॒रो॒ह इति॑ निः - रो॒हः । अ॒सीति॑ । इत्या॑ह । आ॒ह॒ प्रजा᳚त्यै । प्रजा᳚त्यै वसु॒कः । प्रजा᳚त्या॒ इति॒ प्र - जा॒त्यै॒ । व॒सु॒को॑ ऽसि । अ॒सि॒ वेष॑श्रिः । वेष॑श्रिरसि । वेष॑श्रि॒रिति॒ वेष॑ - श्रिः॒ । अ॒सि॒ वस्य॑ष्टिः । वस्य॑ष्टिरसि । अ॒सीति॑ । इत्या॑ह । आ॒ह॒ प्रति॑ष्ठित्यै । प्रति॑ष्ठित्या॒ इति॒ प्रति॑ - स्थि॒त्यै॒ । \newline

\textbf{Jatai Paata} \newline

1. जि॒न्वे तीति॑ जिन्व जि॒न्वेति॑ । \newline
2. इत्या॑ हा॒हे तीत्या॑ह । \newline
3. आ॒ह॒ प्र॒जासु॑ प्र॒जा स्वा॑हाह प्र॒जासु॑ । \newline
4. प्र॒जा स्वे॒वैव प्र॒जासु॑ प्र॒जा स्वे॒व । \newline
5. प्र॒जास्विति॑ प्र - जासु॑ । \newline
6. ए॒व प्रा॒णान् प्रा॒णा ने॒वैव प्रा॒णान् । \newline
7. प्रा॒णान् द॑धाति दधाति प्रा॒णान् प्रा॒णान् द॑धाति । \newline
8. प्रा॒णानिति॑ प्र - अ॒नान् । \newline
9. द॒धा॒ति॒ त्रि॒वृत् त्रि॒वृद् द॑धाति दधाति त्रि॒वृत् । \newline
10. त्रि॒वृ द॑स्यसि त्रि॒वृत् त्रि॒वृ द॑सि । \newline
11. त्रि॒वृदिति॑ त्रि - वृत् । \newline
12. अ॒सि॒ प्र॒वृत् प्र॒वृ द॑स्यसि प्र॒वृत् । \newline
13. प्र॒वृ द॑स्यसि प्र॒वृत् प्र॒वृ द॑सि । \newline
14. प्र॒वृदिति॑ प्र - वृत् । \newline
15. अ॒सीती त्य॑स्य॒सीति॑ । \newline
16. इत्या॑ हा॒हे तीत्या॑ह । \newline
17. आ॒ह॒ मि॒थु॒न॒त्वाय॑ मिथुन॒त्वाया॑ हाह मिथुन॒त्वाय॑ । \newline
18. मि॒थु॒न॒त्वाय॑ सꣳरो॒हः सꣳ॑रो॒हो मि॑थुन॒त्वाय॑ मिथुन॒त्वाय॑ सꣳरो॒हः । \newline
19. मि॒थु॒न॒त्वायेति॑ मिथुन - त्वाय॑ । \newline
20. सꣳ॒॒रो॒हो᳚ ऽस्यसि सꣳरो॒हः सꣳ॑रो॒हो॑ ऽसि । \newline
21. सꣳ॒॒रो॒ह इति॑ सं - रो॒हः । \newline
22. अ॒सि॒ नी॒रो॒हो नी॑रो॒हो᳚ ऽस्यसि नीरो॒हः । \newline
23. नी॒रो॒हो᳚ ऽस्यसि नीरो॒हो नी॑रो॒हो॑ ऽसि । \newline
24. नी॒रो॒ह इति॑ निः - रो॒हः । \newline
25. अ॒सीती त्य॑स्य॒सीति॑ । \newline
26. इत्या॑ हा॒हे तीत्या॑ह । \newline
27. आ॒ह॒ प्रजा᳚त्यै॒ प्रजा᳚त्या आहाह॒ प्रजा᳚त्यै । \newline
28. प्रजा᳚त्यै वसु॒को व॑सु॒कः प्रजा᳚त्यै॒ प्रजा᳚त्यै वसु॒कः । \newline
29. प्रजा᳚त्या॒ इति॒ प्र - जा॒त्यै॒ । \newline
30. व॒सु॒को᳚ ऽस्यसि वसु॒को व॑सु॒को॑ ऽसि । \newline
31. अ॒सि॒ वेष॑श्रि॒र् वेष॑श्रि रस्यसि॒ वेष॑श्रिः । \newline
32. वेष॑श्रि रस्यसि॒ वेष॑श्रि॒र् वेष॑श्रि रसि । \newline
33. वेष॑श्रि॒रिति॒ वेष॑ - श्रिः॒ । \newline
34. अ॒सि॒ वस्य॑ष्टि॒र् वस्य॑ष्टि रस्यसि॒ वस्य॑ष्टिः । \newline
35. वस्य॑ष्टि रस्यसि॒ वस्य॑ष्टि॒र् वस्य॑ष्टि रसि । \newline
36. अ॒सीती त्य॑स्य॒सीति॑ । \newline
37. इत्या॑ हा॒हे तीत्या॑ह । \newline
38. आ॒ह॒ प्रति॑ष्ठित्यै॒ प्रति॑ष्ठित्या आहाह॒ प्रति॑ष्ठित्यै । \newline
39. प्रति॑ष्ठित्या॒ इति॒ प्रति॑ - स्थि॒त्यै॒ । \newline

\textbf{Ghana Paata } \newline

1. जि॒न्वे तीति॑ जिन्व जि॒न्वे त्या॑हा॒हेति॑ जिन्व जि॒न्वे त्या॑ह । \newline
2. इत्या॑हा॒हेती त्या॑ह प्र॒जासु॑ प्र॒जा स्वा॒हेती त्या॑ह प्र॒जासु॑ । \newline
3. आ॒ह॒ प्र॒जासु॑ प्र॒जा स्वा॑हाह प्र॒जा स्वे॒वैव प्र॒जा स्वा॑हाह प्र॒जा स्वे॒व । \newline
4. प्र॒जा स्वे॒वैव प्र॒जासु॑ प्र॒जा स्वे॒व प्रा॒णान् प्रा॒णा ने॒व प्र॒जासु॑ प्र॒जा स्वे॒व प्रा॒णान् । \newline
5. प्र॒जास्विति॑ प्र - जासु॑ । \newline
6. ए॒व प्रा॒णान् प्रा॒णा ने॒वैव प्रा॒णान् द॑धाति दधाति प्रा॒णा ने॒वैव प्रा॒णान् द॑धाति । \newline
7. प्रा॒णान् द॑धाति दधाति प्रा॒णान् प्रा॒णान् द॑धाति त्रि॒वृत् त्रि॒वृद् द॑धाति प्रा॒णान् प्रा॒णान् द॑धाति त्रि॒वृत् । \newline
8. प्रा॒णानिति॑ प्र - अ॒नान् । \newline
9. द॒धा॒ति॒ त्रि॒वृत् त्रि॒वृद् द॑धाति दधाति त्रि॒वृ द॑स्यसि त्रि॒वृद् द॑धाति दधाति त्रि॒वृ द॑सि । \newline
10. त्रि॒वृ द॑स्यसि त्रि॒वृत् त्रि॒वृ द॑सि प्र॒वृत् प्र॒वृ द॑सि त्रि॒वृत् त्रि॒वृ द॑सि प्र॒वृत् । \newline
11. त्रि॒वृदिति॑ त्रि - वृत् । \newline
12. अ॒सि॒ प्र॒वृत् प्र॒वृ द॑स्यसि प्र॒वृ द॑स्यसि प्र॒वृ द॑स्यसि प्र॒वृ द॑सि । \newline
13. प्र॒वृ द॑स्यसि प्र॒वृत् प्र॒वृ द॒सीती त्य॑सि प्र॒वृत् प्र॒वृ द॒सीति॑ । \newline
14. प्र॒वृदिति॑ प्र - वृत् । \newline
15. अ॒सीती त्य॑स्य॒सी त्या॑हा॒हे त्य॑स्य॒सी त्या॑ह । \newline
16. इत्या॑हा॒हेती त्या॑ह मिथुन॒त्वाय॑ मिथुन॒त्वाया॒हेती त्या॑ह मिथुन॒त्वाय॑ । \newline
17. आ॒ह॒ मि॒थु॒न॒त्वाय॑ मिथुन॒त्वाया॑ हाह मिथुन॒त्वाय॑ सꣳरो॒हः सꣳ॑रो॒हो मि॑थुन॒त्वाया॑ हाह मिथुन॒त्वाय॑ सꣳरो॒हः । \newline
18. मि॒थु॒न॒त्वाय॑ सꣳरो॒हः सꣳ॑रो॒हो मि॑थुन॒त्वाय॑ मिथुन॒त्वाय॑ सꣳरो॒हो᳚ ऽस्यसि सꣳरो॒हो मि॑थुन॒त्वाय॑ मिथुन॒त्वाय॑ सꣳरो॒हो॑ ऽसि । \newline
19. मि॒थु॒न॒त्वायेति॑ मिथुन - त्वाय॑ । \newline
20. सꣳ॒॒रो॒हो᳚ ऽस्यसि सꣳरो॒हः सꣳ॑रो॒हो॑ ऽसि नीरो॒हो नी॑रो॒हो॑ ऽसि सꣳरो॒हः सꣳ॑रो॒हो॑ ऽसि नीरो॒हः । \newline
21. सꣳ॒॒रो॒ह इति॑ सं - रो॒हः । \newline
22. अ॒सि॒ नी॒रो॒हो नी॑रो॒हो᳚ ऽस्यसि नीरो॒हो᳚ ऽस्यसि नीरो॒हो᳚ ऽस्यसि नीरो॒हो॑ ऽसि । \newline
23. नी॒रो॒हो᳚ ऽस्यसि नीरो॒हो नी॑रो॒हो॑ ऽसीती त्य॑सि नीरो॒हो नी॑रो॒हो॑ ऽसीति॑ । \newline
24. नी॒रो॒ह इति॑ निः - रो॒हः । \newline
25. अ॒सीती त्य॑स्य॒सी त्या॑हा॒हे त्य॑स्य॒सी त्या॑ह । \newline
26. इत्या॑हा॒हे तीत्या॑ह॒ प्रजा᳚त्यै॒ प्रजा᳚त्या आ॒हे तीत्या॑ह॒ प्रजा᳚त्यै । \newline
27. आ॒ह॒ प्रजा᳚त्यै॒ प्रजा᳚त्या आहाह॒ प्रजा᳚त्यै वसु॒को व॑सु॒कः प्रजा᳚त्या आहाह॒ प्रजा᳚त्यै वसु॒कः । \newline
28. प्रजा᳚त्यै वसु॒को व॑सु॒कः प्रजा᳚त्यै॒ प्रजा᳚त्यै वसु॒को᳚ ऽस्यसि वसु॒कः प्रजा᳚त्यै॒ प्रजा᳚त्यै वसु॒को॑ ऽसि । \newline
29. प्रजा᳚त्या॒ इति॒ प्र - जा॒त्यै॒ । \newline
30. व॒सु॒को᳚ ऽस्यसि वसु॒को व॑सु॒को॑ ऽसि॒ वेष॑श्रि॒र् वेष॑श्रि रसि वसु॒को व॑सु॒को॑ ऽसि॒ वेष॑श्रिः । \newline
31. अ॒सि॒ वेष॑श्रि॒र् वेष॑श्रि रस्यसि॒ वेष॑श्रि रस्यसि॒ वेष॑श्रि रस्यसि॒ वेष॑श्रि रसि । \newline
32. वेष॑श्रि रस्यसि॒ वेष॑श्रि॒र् वेष॑श्रि रसि॒ वस्य॑ष्टि॒र् वस्य॑ष्टि रसि॒ वेष॑श्रि॒र् वेष॑श्रि रसि॒ वस्य॑ष्टिः । \newline
33. वेष॑श्रि॒रिति॒ वेष॑ - श्रिः॒ । \newline
34. अ॒सि॒ वस्य॑ष्टि॒र् वस्य॑ष्टि रस्यसि॒ वस्य॑ष्टि रस्यसि॒ वस्य॑ष्टि रस्यसि॒ वस्य॑ष्टि रसि । \newline
35. वस्य॑ष्टि रस्यसि॒ वस्य॑ष्टि॒र् वस्य॑ष्टि र॒सीती त्य॑सि॒ वस्य॑ष्टि॒र् वस्य॑ष्टि र॒सीति॑ । \newline
36. अ॒सीती त्य॑स्य॒सी त्या॑हा॒हे त्य॑स्य॒सी त्या॑ह । \newline
37. इत्या॑हा॒हे तीत्या॑ह॒ प्रति॑ष्ठित्यै॒ प्रति॑ष्ठित्या आ॒हे तीत्या॑ह॒ प्रति॑ष्ठित्यै । \newline
38. आ॒ह॒ प्रति॑ष्ठित्यै॒ प्रति॑ष्ठित्या आहाह॒ प्रति॑ष्ठित्यै । \newline
39. प्रति॑ष्ठित्या॒ इति॒ प्रति॑ - स्थि॒त्यै॒ । \newline
\pagebreak
\markright{ TS 3.5.3.1  \hfill https://www.vedavms.in \hfill}

\section{ TS 3.5.3.1 }

\textbf{TS 3.5.3.1 } \newline
\textbf{Samhita Paata} \newline

अ॒ग्निना॑ दे॒वेन॒ पृत॑ना जयामि गाय॒त्रेण॒ छन्द॑सा त्रि॒वृता॒ स्तोमे॑न रथन्त॒रेण॒ साम्ना॑ वषट्का॒रेण॒ वज्रे॑ण पूर्व॒जान् भ्रातृ॑व्या॒नध॑रान् पादया॒म्यवै॑नान् बाधे॒ प्रत्ये॑नान्नुदे॒ऽस्मिन् क्षये॒ऽस्मिन् भू॑मिलो॒के यो᳚ऽस्मान् द्वेष्टि॒ यञ्च॑ व॒यं द्वि॒ष्मो विष्णोः॒ क्रमे॒णाऽत्ये॑नान् क्रामा॒मीन्द्रे॑ण दे॒वेन॒ पृत॑ना जयामि॒ त्रैष्टु॑भेन॒ छन्द॑सा पञ्चद॒शेन॒ स्तोमे॑न बृह॒ता साम्ना॑ वषट्का॒रेण॒ वज्रे॑ण - [  ] \newline

\textbf{Pada Paata} \newline

अ॒ग्निना᳚ । दे॒वेन॑ । पृत॑नाः । ज॒या॒मि॒ । गा॒य॒त्रेण॑ । छन्द॑सा । त्रि॒वृतेति॑ त्रि - वृता᳚ । स्तोमे॑न । र॒थ॒न्त॒रेणेति॑ रथं - त॒रेण॑ । साम्ना᳚ । व॒ष॒ट्का॒रेणेति॑ वषट् - का॒रेण॑ । वज्रे॑ण । पू॒र्व॒जानिति॑ पूर्व - जान् । भ्रातृ॑व्यान् । अध॑रान् । पा॒द॒या॒मि॒ । अवेति॑ । ए॒ना॒न् । बा॒धे॒ । प्रतीति॑ । ए॒ना॒न् । नु॒दे॒ । अ॒स्मिन्न् । क्षये᳚ । अ॒स्मिन्न् । भू॒मि॒लो॒क इति॑ भूमि - लो॒के । यः । अ॒स्मान् । द्वेष्टि॑ । यम् । च॒ । व॒यम् । द्वि॒ष्मः । विष्णोः᳚ । क्रमे॑ण । अतीति॑ । ए॒ना॒न् । क्रा॒मा॒मि॒ । इन्द्रे॑ण । दे॒वेन॑ । पृत॑नाः । ज॒या॒मि॒ । त्रैष्टु॑भेन । छन्द॑सा । प॒ञ्च॒द॒शेनेति॑ पञ्च - द॒शेन॑ । स्तोमे॑न । बृ॒ह॒ता । साम्ना᳚ । व॒ष॒ट्का॒रेणेति॑ वषट् - का॒रेण॑ । वज्रे॑ण ।  \newline


\textbf{Krama Paata} \newline

अ॒ग्निना॑ दे॒वेन॑ । दे॒वेन॒ पृत॑नाः । पृत॑ना जयामि । ज॒या॒मि॒ गा॒य॒त्रेण॑ । गा॒य॒त्रेण॒ छन्द॑सा । छन्द॑सा त्रि॒वृता᳚ । त्रि॒वृता॒ स्तोमे॑न । त्रि॒वृतेति॑ त्रि - वृता᳚ । स्तोमे॑न रथन्त॒रेण॑ । र॒थ॒न्त॒रेण॒ साम्ना᳚ । र॒थ॒न्त॒रेणेति॑ रथम् - त॒रेण॑ । साम्ना॑ वषट्का॒रेण॑ । व॒ष॒ट्का॒रेण॒ वज्रे॑ण । व॒ष॒ट्का॒रेणेति॑ वषट् - का॒रेण॑ । वज्रे॑ण पूर्व॒जान् । पू॒र्व॒जान् भ्रातृ॑व्यान् । पू॒र्व॒जानिति॑ पूर्व - जान् । भ्रातृ॑व्या॒नध॑रान् । अध॑रान् पादयामि । पा॒द॒या॒म्यव॑ । अवै॑नान् । ए॒ना॒न् बा॒धे॒ । बा॒धे॒ प्रति॑ । प्रत्ये॑नान् । ए॒ना॒न् नु॒दे॒ । नु॒दे॒ ऽस्मिन्न् । अ॒स्मिन् क्षये᳚ । क्षये॒ ऽस्मिन्न् । अ॒स्मिन् भू॑मिलो॒के । भू॒मि॒लो॒के यः । भू॒मि॒लो॒क इति॑ भूमि - लो॒के । यो᳚ ऽस्मान् । अ॒स्मान् द्वेष्टि॑ । द्वेष्टि॒ यम् । यम् च॑ । च॒ व॒यम् । व॒यम् द्वि॒ष्मः । द्वि॒ष्मो विष्णोः᳚ । विष्णोः॒ क्रमे॑ण । क्रमे॒णाति॑ । अत्ये॑नान् । ए॒ना॒न् का॒मा॒मि॒ । क्रा॒मा॒मीन्द्रे॑ण । इन्द्रे॑ण दे॒वेन॑ । दे॒वेन॒ पृत॑नाः । पृत॑ना जयामि । ज॒या॒मि॒ त्रैष्टु॑भेन । त्रैष्टु॑भेन॒ छन्द॑सा । छन्द॑सा पञ्चद॒शेन॑ । प॒ञ्च॒द॒शेन॒ स्तोमे॑न । प॒ञ्च॒द॒शेनेति॑ पञ्च - द॒शेन॑ । स्तोमे॑न बृह॒ता । बृ॒ह॒ता साम्ना᳚ । साम्ना॑ वषट्का॒रेण॑ । व॒ष॒ट्का॒रेण॒ वज्रे॑ण ( ) । व॒ष॒ट्का॒रेणेति॑ वषट् - का॒रेण॑ । वज्रे॑ण सह॒जान् \newline

\textbf{Jatai Paata} \newline

1. अ॒ग्निना॑ दे॒वेन॑ दे॒वेना॒ग्निना॒ ऽग्निना॑ दे॒वेन॑ । \newline
2. दे॒वेन॒ पृत॑नाः॒ पृत॑ना दे॒वेन॑ दे॒वेन॒ पृत॑नाः । \newline
3. पृत॑ना जयामि जयामि॒ पृत॑नाः॒ पृत॑ना जयामि । \newline
4. ज॒या॒मि॒ गा॒य॒त्रेण॑ गाय॒त्रेण॑ जयामि जयामि गाय॒त्रेण॑ । \newline
5. गा॒य॒त्रेण॒ छन्द॑सा॒ छन्द॑सा गाय॒त्रेण॑ गाय॒त्रेण॒ छन्द॑सा । \newline
6. छन्द॑सा त्रि॒वृता᳚ त्रि॒वृता॒ छन्द॑सा॒ छन्द॑सा त्रि॒वृता᳚ । \newline
7. त्रि॒वृता॒ स्तोमे॑न॒ स्तोमे॑न त्रि॒वृता᳚ त्रि॒वृता॒ स्तोमे॑न । \newline
8. त्रि॒वृतेति॑ त्रि - वृता᳚ । \newline
9. स्तोमे॑न रथन्त॒रेण॑ रथन्त॒रेण॒ स्तोमे॑न॒ स्तोमे॑न रथन्त॒रेण॑ । \newline
10. र॒थ॒न्त॒रेण॒ साम्ना॒ साम्ना॑ रथन्त॒रेण॑ रथन्त॒रेण॒ साम्ना᳚ । \newline
11. र॒थ॒न्त॒रेणेति॑ रथं - त॒रेण॑ । \newline
12. साम्ना॑ वषट्का॒रेण॑ वषट्का॒रेण॒ साम्ना॒ साम्ना॑ वषट्का॒रेण॑ । \newline
13. व॒ष॒ट्का॒रेण॒ वज्रे॑ण॒ वज्रे॑ण वषट्का॒रेण॑ वषट्का॒रेण॒ वज्रे॑ण । \newline
14. व॒ष॒ट्का॒रेणेति॑ वषट् - का॒रेण॑ । \newline
15. वज्रे॑ण पूर्व॒जान् पू᳚र्व॒जान्. वज्रे॑ण॒ वज्रे॑ण पूर्व॒जान् । \newline
16. पू॒र्व॒जान् भ्रातृ॑व्या॒न् भ्रातृ॑व्यान् पूर्व॒जान् पू᳚र्व॒जान् भ्रातृ॑व्यान् । \newline
17. पू॒र्व॒जानिति॑ पूर्व - जान् । \newline
18. भ्रातृ॑व्या॒ नध॑रा॒ नध॑रा॒न् भ्रातृ॑व्या॒न् भ्रातृ॑व्या॒ नध॑रान् । \newline
19. अध॑रान् पादयामि पादया॒ म्यध॑रा॒ नध॑रान् पादयामि । \newline
20. पा॒द॒या॒ म्यवाव॑ पादयामि पादया॒ म्यव॑ । \newline
21. अवै॑ना नेना॒ नवावै॑नान् । \newline
22. ए॒ना॒न् बा॒धे॒ बा॒ध॒ ए॒ना॒ ने॒ना॒न् बा॒धे॒ । \newline
23. बा॒धे॒ प्रति॒ प्रति॑ बाधे बाधे॒ प्रति॑ । \newline
24. प्रत्ये॑ना नेना॒न् प्रति॒ प्रत्ये॑नान् । \newline
25. ए॒ना॒न् नु॒दे॒ नु॒द॒ ए॒ना॒ ने॒ना॒न् नु॒दे॒ । \newline
26. नु॒दे॒ ऽस्मिन् न॒स्मिन् नु॑दे नुदे॒ ऽस्मिन्न् । \newline
27. अ॒स्मिन् क्षये॒ क्षये॒ ऽस्मिन् न॒स्मिन् क्षये᳚ । \newline
28. क्षये॒ ऽस्मिन् न॒स्मिन् क्षये॒ क्षये॒ ऽस्मिन्न् । \newline
29. अ॒स्मिन् भू॑मिलो॒के भू॑मिलो॒के᳚ ऽस्मिन् न॒स्मिन् भू॑मिलो॒के । \newline
30. भू॒मि॒लो॒के यो यो भू॑मिलो॒के भू॑मिलो॒के यः । \newline
31. भू॒मि॒लो॒क इति॑ भूमि - लो॒के । \newline
32. यो᳚ ऽस्मा न॒स्मान्. यो यो᳚ ऽस्मान् । \newline
33. अ॒स्मान् द्वेष्टि॒ द्वेष्ट्य॒स्मा न॒स्मान् द्वेष्टि॑ । \newline
34. द्वेष्टि॒ यं ॅयम् द्वेष्टि॒ द्वेष्टि॒ यम् । \newline
35. यम् च॑ च॒ यं ॅयम् च॑ । \newline
36. च॒ व॒यं ॅव॒यम् च॑ च व॒यम् । \newline
37. व॒यम् द्वि॒ष्मो द्वि॒ष्मो व॒यं ॅव॒यम् द्वि॒ष्मः । \newline
38. द्वि॒ष्मो विष्णो॒र् विष्णो᳚र् द्वि॒ष्मो द्वि॒ष्मो विष्णोः᳚ । \newline
39. विष्णोः॒ क्रमे॑ण॒ क्रमे॑ण॒ विष्णो॒र् विष्णोः॒ क्रमे॑ण । \newline
40. क्रमे॒णा त्यति॒ क्रमे॑ण॒ क्रमे॒णाति॑ । \newline
41. अत्ये॑ना नेना॒ नत्य त्ये॑नान् । \newline
42. ए॒ना॒न् क्रा॒मा॒मि॒ क्रा॒मा॒ म्ये॒ना॒ ने॒ना॒न् क्रा॒मा॒मि॒ । \newline
43. क्रा॒मा॒ मीन्द्रे॒ णेन्द्रे॑ण क्रामामि क्रामा॒ मीन्द्रे॑ण । \newline
44. इन्द्रे॑ण दे॒वेन॑ दे॒वेनेन्द्रे॒ णेन्द्रे॑ण दे॒वेन॑ । \newline
45. दे॒वेन॒ पृत॑नाः॒ पृत॑ना दे॒वेन॑ दे॒वेन॒ पृत॑नाः । \newline
46. पृत॑ना जयामि जयामि॒ पृत॑नाः॒ पृत॑ना जयामि । \newline
47. ज॒या॒मि॒ त्रैष्टु॑भेन॒ त्रैष्टु॑भेन जयामि जयामि॒ त्रैष्टु॑भेन । \newline
48. त्रैष्टु॑भेन॒ छन्द॑सा॒ छन्द॑सा॒ त्रैष्टु॑भेन॒ त्रैष्टु॑भेन॒ छन्द॑सा । \newline
49. छन्द॑सा पञ्चद॒शेन॑ पञ्चद॒शेन॒ छन्द॑सा॒ छन्द॑सा पञ्चद॒शेन॑ । \newline
50. प॒ञ्च॒द॒शेन॒ स्तोमे॑न॒ स्तोमे॑न पञ्चद॒शेन॑ पञ्चद॒शेन॒ स्तोमे॑न । \newline
51. प॒ञ्च॒द॒शेनेति॑ पञ्च - द॒शेन॑ । \newline
52. स्तोमे॑न बृह॒ता बृ॑ह॒ता स्तोमे॑न॒ स्तोमे॑न बृह॒ता । \newline
53. बृ॒ह॒ता साम्ना॒ साम्ना॑ बृह॒ता बृ॑ह॒ता साम्ना᳚ । \newline
54. साम्ना॑ वषट्का॒रेण॑ वषट्का॒रेण॒ साम्ना॒ साम्ना॑ वषट्का॒रेण॑ । \newline
55. व॒ष॒ट्का॒रेण॒ वज्रे॑ण॒ वज्रे॑ण वषट्का॒रेण॑ वषट्का॒रेण॒ वज्रे॑ण । \newline
56. व॒ष॒ट्का॒रेणेति॑ वषट् - का॒रेण॑ । \newline
57. वज्रे॑ण सह॒जान् थ्स॑ह॒जान्. वज्रे॑ण॒ वज्रे॑ण सह॒जान् । \newline

\textbf{Ghana Paata } \newline

1. अ॒ग्निना॑ दे॒वेन॑ दे॒वेना॒ग्निना॒ ऽग्निना॑ दे॒वेन॒ पृत॑नाः॒ पृत॑ना दे॒वेना॒ग्निना॒ ऽग्निना॑ दे॒वेन॒ पृत॑नाः । \newline
2. दे॒वेन॒ पृत॑नाः॒ पृत॑ना दे॒वेन॑ दे॒वेन॒ पृत॑ना जयामि जयामि॒ पृत॑ना दे॒वेन॑ दे॒वेन॒ पृत॑ना जयामि । \newline
3. पृत॑ना जयामि जयामि॒ पृत॑नाः॒ पृत॑ना जयामि गाय॒त्रेण॑ गाय॒त्रेण॑ जयामि॒ पृत॑नाः॒ पृत॑ना जयामि गाय॒त्रेण॑ । \newline
4. ज॒या॒मि॒ गा॒य॒त्रेण॑ गाय॒त्रेण॑ जयामि जयामि गाय॒त्रेण॒ छन्द॑सा॒ छन्द॑सा गाय॒त्रेण॑ जयामि जयामि गाय॒त्रेण॒ छन्द॑सा । \newline
5. गा॒य॒त्रेण॒ छन्द॑सा॒ छन्द॑सा गाय॒त्रेण॑ गाय॒त्रेण॒ छन्द॑सा त्रि॒वृता᳚ त्रि॒वृता॒ छन्द॑सा गाय॒त्रेण॑ गाय॒त्रेण॒ छन्द॑सा त्रि॒वृता᳚ । \newline
6. छन्द॑सा त्रि॒वृता᳚ त्रि॒वृता॒ छन्द॑सा॒ छन्द॑सा त्रि॒वृता॒ स्तोमे॑न॒ स्तोमे॑न त्रि॒वृता॒ छन्द॑सा॒ छन्द॑सा त्रि॒वृता॒ स्तोमे॑न । \newline
7. त्रि॒वृता॒ स्तोमे॑न॒ स्तोमे॑न त्रि॒वृता᳚ त्रि॒वृता॒ स्तोमे॑न रथन्त॒रेण॑ रथन्त॒रेण॒ स्तोमे॑न त्रि॒वृता᳚ त्रि॒वृता॒ स्तोमे॑न रथन्त॒रेण॑ । \newline
8. त्रि॒वृतेति॑ त्रि - वृता᳚ । \newline
9. स्तोमे॑न रथन्त॒रेण॑ रथन्त॒रेण॒ स्तोमे॑न॒ स्तोमे॑न रथन्त॒रेण॒ साम्ना॒ साम्ना॑ रथन्त॒रेण॒ स्तोमे॑न॒ स्तोमे॑न रथन्त॒रेण॒ साम्ना᳚ । \newline
10. र॒थ॒न्त॒रेण॒ साम्ना॒ साम्ना॑ रथन्त॒रेण॑ रथन्त॒रेण॒ साम्ना॑ वषट्का॒रेण॑ वषट्का॒रेण॒ साम्ना॑ रथन्त॒रेण॑ रथन्त॒रेण॒ साम्ना॑ वषट्का॒रेण॑ । \newline
11. र॒थ॒न्त॒रेणेति॑ रथं - त॒रेण॑ । \newline
12. साम्ना॑ वषट्का॒रेण॑ वषट्का॒रेण॒ साम्ना॒ साम्ना॑ वषट्का॒रेण॒ वज्रे॑ण॒ वज्रे॑ण वषट्का॒रेण॒ साम्ना॒ साम्ना॑ वषट्का॒रेण॒ वज्रे॑ण । \newline
13. व॒ष॒ट्का॒रेण॒ वज्रे॑ण॒ वज्रे॑ण वषट्का॒रेण॑ वषट्का॒रेण॒ वज्रे॑ण पूर्व॒जान् पू᳚र्व॒जान्. वज्रे॑ण वषट्का॒रेण॑ वषट्का॒रेण॒ वज्रे॑ण पूर्व॒जान् । \newline
14. व॒ष॒ट्का॒रेणेति॑ वषट् - का॒रेण॑ । \newline
15. वज्रे॑ण पूर्व॒जान् पू᳚र्व॒जान्. वज्रे॑ण॒ वज्रे॑ण पूर्व॒जान् भ्रातृ॑व्या॒न् भ्रातृ॑व्यान् पूर्व॒जान्. वज्रे॑ण॒ वज्रे॑ण पूर्व॒जान् भ्रातृ॑व्यान् । \newline
16. पू॒र्व॒जान् भ्रातृ॑व्या॒न् भ्रातृ॑व्यान् पूर्व॒जान् पू᳚र्व॒जान् भ्रातृ॑व्या॒ नध॑रा॒ नध॑रा॒न् भ्रातृ॑व्यान् पूर्व॒जान् पू᳚र्व॒जान् भ्रातृ॑व्या॒ नध॑रान् । \newline
17. पू॒र्व॒जानिति॑ पूर्व - जान् । \newline
18. भ्रातृ॑व्या॒ नध॑रा॒ नध॑रा॒न् भ्रातृ॑व्या॒न् भ्रातृ॑व्या॒ नध॑रान् पादयामि पादया॒ म्यध॑रा॒न् भ्रातृ॑व्या॒न् भ्रातृ॑व्या॒ नध॑रान् पादयामि । \newline
19. अध॑रान् पादयामि पादया॒ म्यध॑रा॒ नध॑रान् पादया॒ म्यवाव॑ पादया॒ म्यध॑रा॒ नध॑रान् पादया॒ म्यव॑ । \newline
20. पा॒द॒या॒ म्यवाव॑ पादयामि पादया॒ म्यवै॑ना नेना॒ नव॑ पादयामि पादया॒ म्यवै॑नान् । \newline
21. अवै॑ना नेना॒ नवावै॑नान् बाधे बाध एना॒ नवावै॑नान् बाधे । \newline
22. ए॒ना॒न् बा॒धे॒ बा॒ध॒ ए॒ना॒ ने॒ना॒न् बा॒धे॒ प्रति॒ प्रति॑ बाध एना नेनान् बाधे॒ प्रति॑ । \newline
23. बा॒धे॒ प्रति॒ प्रति॑ बाधे बाधे॒ प्रत्ये॑ना नेना॒न् प्रति॑ बाधे बाधे॒ प्रत्ये॑नान् । \newline
24. प्रत्ये॑ना नेना॒न् प्रति॒ प्रत्ये॑नान् नुदे नुद एना॒न् प्रति॒ प्रत्ये॑नान् नुदे । \newline
25. ए॒ना॒न् नु॒दे॒ नु॒द॒ ए॒ना॒ ने॒ना॒न् नु॒दे॒ ऽस्मिन् न॒स्मिन् नु॑द एना नेनान् नुदे॒ ऽस्मिन्न् । \newline
26. नु॒दे॒ ऽस्मिन् न॒स्मिन् नु॑दे नुदे॒ ऽस्मिन् क्षये॒ क्षये॒ ऽस्मिन् नु॑दे नुदे॒ ऽस्मिन् क्षये᳚ । \newline
27. अ॒स्मिन् क्षये॒ क्षये॒ ऽस्मिन् न॒स्मिन् क्षये॒ ऽस्मिन् न॒स्मिन् क्षये॒ ऽस्मिन् न॒स्मिन् क्षये॒ ऽस्मिन्न् । \newline
28. क्षये॒ ऽस्मिन् न॒स्मिन् क्षये॒ क्षये॒ ऽस्मिन् भू॑मिलो॒के भू॑मिलो॒के᳚ ऽस्मिन् क्षये॒ क्षये॒ ऽस्मिन् भू॑मिलो॒के । \newline
29. अ॒स्मिन् भू॑मिलो॒के भू॑मिलो॒के᳚ ऽस्मिन् न॒स्मिन् भू॑मिलो॒के यो यो भू॑मिलो॒के᳚ ऽस्मिन् न॒स्मिन् भू॑मिलो॒के यः । \newline
30. भू॒मि॒लो॒के यो यो भू॑मिलो॒के भू॑मिलो॒के यो᳚ ऽस्मा न॒स्मान्. यो भू॑मिलो॒के भू॑मिलो॒के यो᳚ ऽस्मान् । \newline
31. भू॒मि॒लो॒क इति॑ भूमि - लो॒के । \newline
32. यो᳚ ऽस्मा न॒स्मान्. यो यो᳚ ऽस्मान् द्वेष्टि॒ द्वेष्ट्य॒स्मान्. यो यो᳚ ऽस्मान् द्वेष्टि॑ । \newline
33. अ॒स्मान् द्वेष्टि॒ द्वेष्ट्य॒स्मा न॒स्मान् द्वेष्टि॒ यं ॅयम् द्वेष्ट्य॒स्मा न॒स्मान् द्वेष्टि॒ यम् । \newline
34. द्वेष्टि॒ यं ॅयम् द्वेष्टि॒ द्वेष्टि॒ यम् च॑ च॒ यम् द्वेष्टि॒ द्वेष्टि॒ यम् च॑ । \newline
35. यम् च॑ च॒ यं ॅयम् च॑ व॒यं ॅव॒यम् च॒ यं ॅयम् च॑ व॒यम् । \newline
36. च॒ व॒यं ॅव॒यम् च॑ च व॒यम् द्वि॒ष्मो द्वि॒ष्मो व॒यम् च॑ च व॒यम् द्वि॒ष्मः । \newline
37. व॒यम् द्वि॒ष्मो द्वि॒ष्मो व॒यं ॅव॒यम् द्वि॒ष्मो विष्णो॒र् विष्णो᳚र् द्वि॒ष्मो व॒यं ॅव॒यम् द्वि॒ष्मो विष्णोः᳚ । \newline
38. द्वि॒ष्मो विष्णो॒र् विष्णो᳚र् द्वि॒ष्मो द्वि॒ष्मो विष्णोः॒ क्रमे॑ण॒ क्रमे॑ण॒ विष्णो᳚र् द्वि॒ष्मो द्वि॒ष्मो विष्णोः॒ क्रमे॑ण । \newline
39. विष्णोः॒ क्रमे॑ण॒ क्रमे॑ण॒ विष्णो॒र् विष्णोः॒ क्रमे॒णा त्यति॒ क्रमे॑ण॒ विष्णो॒र् विष्णोः॒ क्रमे॒णाति॑ । \newline
40. क्रमे॒णा त्यति॒ क्रमे॑ण॒ क्रमे॒णा त्ये॑ना नेना॒ नति॒ क्रमे॑ण॒ क्रमे॒णा त्ये॑नान् । \newline
41. अत्ये॑ना नेना॒ नत्य त्ये॑नान् क्रामामि क्रामा म्येना॒ नत्य त्ये॑नान् क्रामामि । \newline
42. ए॒ना॒न् क्रा॒मा॒मि॒ क्रा॒मा॒ म्ये॒ना॒ ने॒ना॒न् क्रा॒मा॒मीन्द्रे॒ णेन्द्रे॑ण क्रामा म्येना नेनान् क्रामा॒मीन्द्रे॑ण । \newline
43. क्रा॒मा॒मीन्द्रे॒ णेन्द्रे॑ण क्रामामि क्रामा॒मीन्द्रे॑ण दे॒वेन॑ दे॒वेनेन्द्रे॑ण क्रामामि क्रामा॒मीन्द्रे॑ण दे॒वेन॑ । \newline
44. इन्द्रे॑ण दे॒वेन॑ दे॒वेनेन्द्रे॒णे न्द्रे॑ण दे॒वेन॒ पृत॑नाः॒ पृत॑ना दे॒वेनेन्द्रे॒ णेन्द्रे॑ण दे॒वेन॒ पृत॑नाः । \newline
45. दे॒वेन॒ पृत॑नाः॒ पृत॑ना दे॒वेन॑ दे॒वेन॒ पृत॑ना जयामि जयामि॒ पृत॑ना दे॒वेन॑ दे॒वेन॒ पृत॑ना जयामि । \newline
46. पृत॑ना जयामि जयामि॒ पृत॑नाः॒ पृत॑ना जयामि॒ त्रैष्टु॑भेन॒ त्रैष्टु॑भेन जयामि॒ पृत॑नाः॒ पृत॑ना जयामि॒ त्रैष्टु॑भेन । \newline
47. ज॒या॒मि॒ त्रैष्टु॑भेन॒ त्रैष्टु॑भेन जयामि जयामि॒ त्रैष्टु॑भेन॒ छन्द॑सा॒ छन्द॑सा॒ त्रैष्टु॑भेन जयामि जयामि॒ त्रैष्टु॑भेन॒ छन्द॑सा । \newline
48. त्रैष्टु॑भेन॒ छन्द॑सा॒ छन्द॑सा॒ त्रैष्टु॑भेन॒ त्रैष्टु॑भेन॒ छन्द॑सा पञ्चद॒शेन॑ पञ्चद॒शेन॒ छन्द॑सा॒ त्रैष्टु॑भेन॒ त्रैष्टु॑भेन॒ छन्द॑सा पञ्चद॒शेन॑ । \newline
49. छन्द॑सा पञ्चद॒शेन॑ पञ्चद॒शेन॒ छन्द॑सा॒ छन्द॑सा पञ्चद॒शेन॒ स्तोमे॑न॒ स्तोमे॑न पञ्चद॒शेन॒ छन्द॑सा॒ छन्द॑सा पञ्चद॒शेन॒ स्तोमे॑न । \newline
50. प॒ञ्च॒द॒शेन॒ स्तोमे॑न॒ स्तोमे॑न पञ्चद॒शेन॑ पञ्चद॒शेन॒ स्तोमे॑न बृह॒ता बृ॑ह॒ता स्तोमे॑न पञ्चद॒शेन॑ पञ्चद॒शेन॒ स्तोमे॑न बृह॒ता । \newline
51. प॒ञ्च॒द॒शेनेति॑ पञ्च - द॒शेन॑ । \newline
52. स्तोमे॑न बृह॒ता बृ॑ह॒ता स्तोमे॑न॒ स्तोमे॑न बृह॒ता साम्ना॒ साम्ना॑ बृह॒ता स्तोमे॑न॒ स्तोमे॑न बृह॒ता साम्ना᳚ । \newline
53. बृ॒ह॒ता साम्ना॒ साम्ना॑ बृह॒ता बृ॑ह॒ता साम्ना॑ वषट्का॒रेण॑ वषट्का॒रेण॒ साम्ना॑ बृह॒ता बृ॑ह॒ता साम्ना॑ वषट्का॒रेण॑ । \newline
54. साम्ना॑ वषट्का॒रेण॑ वषट्का॒रेण॒ साम्ना॒ साम्ना॑ वषट्का॒रेण॒ वज्रे॑ण॒ वज्रे॑ण वषट्का॒रेण॒ साम्ना॒ साम्ना॑ वषट्का॒रेण॒ वज्रे॑ण । \newline
55. व॒ष॒ट्का॒रेण॒ वज्रे॑ण॒ वज्रे॑ण वषट्का॒रेण॑ वषट्का॒रेण॒ वज्रे॑ण सह॒जान् थ्स॑ह॒जान्. वज्रे॑ण वषट्का॒रेण॑ वषट्का॒रेण॒ वज्रे॑ण सह॒जान् । \newline
56. व॒ष॒ट्का॒रेणेति॑ वषट् - का॒रेण॑ । \newline
57. वज्रे॑ण सह॒जान् थ्स॑ह॒जान्. वज्रे॑ण॒ वज्रे॑ण सह॒जान्. विश्वे॑भि॒र् विश्वे॑भिः सह॒जान्. वज्रे॑ण॒ वज्रे॑ण सह॒जान्. विश्वे॑भिः । \newline
\pagebreak
\markright{ TS 3.5.3.2  \hfill https://www.vedavms.in \hfill}

\section{ TS 3.5.3.2 }

\textbf{TS 3.5.3.2 } \newline
\textbf{Samhita Paata} \newline

सह॒जान्. विश्वे॑भिर्दे॒वेभिः॒ पृत॑ना जयामि॒ जाग॑तेन॒ छन्द॑सा सप्तद॒शेन॒ स्तोमे॑न वामदे॒व्येन॒ साम्ना॑ वषट्का॒रेण॒ वज्रे॑णा पर॒जानिन्द्रे॑ण स॒युजो॑ व॒यꣳ सा॑स॒ह्याम॑ पृतन्य॒तः । घ्नन्तो॑ वृ॒त्राण्य॑प्र॒ति । यत्ते॑ अग्ने॒ तेज॒स्तेना॒हं ते॑ज॒स्वी भू॑यासं॒ ॅयत्ते॑ अग्ने॒ वर्च॒स्तेना॒हं ॅव॑र्च॒स्वी भू॑यासं॒ ॅयत्ते॑ अग्ने॒ हर॒स्तेना॒हꣳ ह॑र॒स्वी भू॑यासं ॥ \newline

\textbf{Pada Paata} \newline

स॒ह॒जानिति॑ सह - जान् । विश्वे॑भिः । दे॒वेभिः॑ । पृत॑नाः । ज॒या॒मि॒ । जाग॑तेन । छन्द॑सा । स॒प्त॒द॒शेनेति॑ सप्त - द॒शेन॑ । स्तोमे॑न । वा॒म॒दे॒व्येनेति॑ वाम-दे॒व्येन॑ । साम्ना᳚ । व॒ष॒ट्का॒रेणेति॑ वषट् - का॒रेण॑ । वज्रे॑ण । अ॒प॒र॒जानित्य॑पर - जान् । इन्द्रे॑ण । स॒युज॒ इति॑ स-युजः॑ । व॒यम् । सा॒स॒ह्याम॑ । पृ॒त॒न्य॒तः ॥ घ्नन्तः॑ । वृ॒त्राणि॑ । अ॒प्र॒ति ॥ यत् । ते॒ । अ॒ग्ने॒ । तेजः॑ । तेन॑ । अ॒हम् । ते॒ज॒स्वी । भू॒या॒स॒म् । यत् । ते॒ । अ॒ग्ने॒ । वर्चः॑ । तेन॑ । अ॒हम् । व॒र्च॒स्वी । भू॒या॒स॒म् । यत् । ते॒ । अ॒ग्ने॒ । हरः॑ । तेन॑ । अ॒हम् । ह॒र॒स्वी । भू॒या॒स॒म् ॥  \newline


\textbf{Krama Paata} \newline

स॒ह॒जान्. विश्वे॑भिः । स॒ह॒जानिति॑ सह - जान् । विश्वे॑भिर् दे॒वेभिः॑ । दे॒वेभिः॒ पृत॑नाः । पृत॑ना जयामि । ज॒या॒मि॒ जाग॑तेन । जाग॑तेन॒ छन्द॑सा । छन्द॑सा सप्तद॒शेन॑ । स॒प्त॒द॒शेन॒ स्तोमे॑न । स॒प्त॒द॒शेनेति॑ सप्त - द॒शेन॑ । स्तोमे॑न वामदे॒व्येन॑ । वा॒म॒दे॒व्येन॒ साम्ना᳚ । वा॒म॒दे॒व्येनेति॑ वाम - दे॒व्येन॑ । साम्ना॑ वषट्का॒रेण॑ । व॒ष॒ट्का॒रेण॒ वज्रे॑ण । व॒ष॒ट्का॒रेणेति॑ वषट् - का॒रेण॑ । वज्रे॑णापर॒जान् । अ॒प॒र॒जानिन्द्रे॑ण । अ॒प॒र॒जानित्य॑पर - जान् । इन्द्रे॑ण स॒युजः॑ । स॒युजो॑ व॒यम् । स॒युज॒ इति॑ स - युजः॑ । व॒यꣳ सा॑स॒ह्याम॑ । सा॒स॒ह्याम॑ पृतन्य॒तः । पृ॒त॒न्य॒त इति॑ पृतन्य॒तः ॥ घ्नन्तो॑ वृ॒त्राणि॑ । वृ॒त्राण्य॑प्र॒ति । अ॒प्र॒तीत्य॑प्र॒ति ॥ यत् ते᳚ । ते॒ अ॒ग्ने॒ । अ॒ग्ने॒ तेजः॑ । तेज॒स्तेन॑ । तेना॒हम् । अ॒हम् ते॑ज॒स्वी । ते॒ज॒स्वी भू॑यासम् । भू॒या॒सं॒ ॅयत् । यत् ते᳚ । ते॒ अ॒ग्ने॒ । अ॒ग्ने॒ वर्चः॑ । वर्च॒स्तेन॑ । तेना॒हम् । अ॒हं ॅव॑र्च॒स्वी । व॒र्च॒स्वी भू॑यासम् । भू॒या॒सं॒ ॅयत् । यत् ते᳚ । ते॒ अ॒ग्ने॒ । अ॒ग्ने॒ हरः॑ । हर॒स्तेन॑ । तेना॒हम् । अ॒हꣳ ह॑र॒स्वी । ह॒र॒स्वी भू॑यासम् । भू॒या॒स॒मिति॑ भूयासम् । \newline

\textbf{Jatai Paata} \newline

1. स॒ह॒जान्. विश्वे॑भि॒र् विश्वे॑भिः सह॒जान् थ्स॑ह॒जान्. विश्वे॑भिः । \newline
2. स॒ह॒जानिति॑ सह - जान् । \newline
3. विश्वे॑भिर् दे॒वेभि॑र् दे॒वेभि॒र् विश्वे॑भि॒र् विश्वे॑भिर् दे॒वेभिः॑ । \newline
4. दे॒वेभिः॒ पृत॑नाः॒ पृत॑ना दे॒वेभि॑र् दे॒वेभिः॒ पृत॑नाः । \newline
5. पृत॑ना जयामि जयामि॒ पृत॑नाः॒ पृत॑ना जयामि । \newline
6. ज॒या॒मि॒ जाग॑तेन॒ जाग॑तेन जयामि जयामि॒ जाग॑तेन । \newline
7. जाग॑तेन॒ छन्द॑सा॒ छन्द॑सा॒ जाग॑तेन॒ जाग॑तेन॒ छन्द॑सा । \newline
8. छन्द॑सा सप्तद॒शेन॑ सप्तद॒शेन॒ छन्द॑सा॒ छन्द॑सा सप्तद॒शेन॑ । \newline
9. स॒प्त॒द॒शेन॒ स्तोमे॑न॒ स्तोमे॑न सप्तद॒शेन॑ सप्तद॒शेन॒ स्तोमे॑न । \newline
10. स॒प्त॒द॒शेनेति॑ सप्त - द॒शेन॑ । \newline
11. स्तोमे॑न वामदे॒व्येन॑ वामदे॒व्येन॒ स्तोमे॑न॒ स्तोमे॑न वामदे॒व्येन॑ । \newline
12. वा॒म॒दे॒व्येन॒ साम्ना॒ साम्ना॑ वामदे॒व्येन॑ वामदे॒व्येन॒ साम्ना᳚ । \newline
13. वा॒म॒दे॒व्येनेति॑ वाम - दे॒व्येन॑ । \newline
14. साम्ना॑ वषट्का॒रेण॑ वषट्का॒रेण॒ साम्ना॒ साम्ना॑ वषट्का॒रेण॑ । \newline
15. व॒ष॒ट्का॒रेण॒ वज्रे॑ण॒ वज्रे॑ण वषट्का॒रेण॑ वषट्का॒रेण॒ वज्रे॑ण । \newline
16. व॒ष॒ट्का॒रेणेति॑ वषट् - का॒रेण॑ । \newline
17. वज्रे॑णा पर॒जा न॑पर॒जान्. वज्रे॑ण॒ वज्रे॑णा पर॒जान् । \newline
18. अ॒प॒र॒जा निन्द्रे॒ णेन्द्रे॑णा पर॒जा न॑पर॒जा निन्द्रे॑ण । \newline
19. अ॒प॒र॒जानित्य॑पर - जान् । \newline
20. इन्द्रे॑ण स॒युजः॑ स॒युज॒ इन्द्रे॒ णेन्द्रे॑ण स॒युजः॑ । \newline
21. स॒युजो॑ व॒यं ॅव॒यꣳ स॒युजः॑ स॒युजो॑ व॒यम् । \newline
22. स॒युज॒ इति॑ स - युजः॑ । \newline
23. व॒यꣳ सा॑स॒ह्याम॑ सास॒ह्याम॑ व॒यं ॅव॒यꣳ सा॑स॒ह्याम॑ । \newline
24. सा॒स॒ह्याम॑ पृतन्य॒तः पृ॑तन्य॒तः सा॑स॒ह्याम॑ सास॒ह्याम॑ पृतन्य॒तः । \newline
25. पृ॒त॒न्य॒त इति॑ पृतन्य॒तः । \newline
26. घ्नन्तो॑ वृ॒त्राणि॑ वृ॒त्राणि॒ घ्नन्तो॒ घ्नन्तो॑ वृ॒त्राणि॑ । \newline
27. वृ॒त्रा ण्य॑प्र॒ त्य॑प्र॒ति वृ॒त्राणि॑ वृ॒त्रा ण्य॑प्र॒ति । \newline
28. अ॒प्र॒तीत्य॑प्र॒ति । \newline
29. यत् ते॑ ते॒ यद् यत् ते᳚ । \newline
30. ते॒ अ॒ग्ने॒ ऽग्ने॒ ते॒ ते॒ अ॒ग्ने॒ । \newline
31. अ॒ग्ने॒ तेज॒ स्तेजो᳚ ऽग्ने ऽग्ने॒ तेजः॑ । \newline
32. तेज॒ स्तेन॒ तेन॒ तेज॒ स्तेज॒ स्तेन॑ । \newline
33. तेना॒ह म॒हम् तेन॒ तेना॒हम् । \newline
34. अ॒हम् ते॑ज॒स्वी ते॑ज॒स्व्य॑ह म॒हम् ते॑ज॒स्वी । \newline
35. ते॒ज॒स्वी भू॑यासम् भूयासम् तेज॒स्वी ते॑ज॒स्वी भू॑यासम् । \newline
36. भू॒या॒सं॒ ॅयद् यद् भू॑यासम् भूयासं॒ ॅयत् । \newline
37. यत् ते॑ ते॒ यद् यत् ते᳚ । \newline
38. ते॒ अ॒ग्ने॒ ऽग्ने॒ ते॒ ते॒ अ॒ग्ने॒ । \newline
39. अ॒ग्ने॒ वर्चो॒ वर्चो᳚ ऽग्ने ऽग्ने॒ वर्चः॑ । \newline
40. वर्च॒ स्तेन॒ तेन॒ वर्चो॒ वर्च॒ स्तेन॑ । \newline
41. तेना॒ह म॒हम् तेन॒ तेना॒हम् । \newline
42. अ॒हं ॅव॑र्च॒स्वी व॑र्च॒स्व्य॑ह म॒हं ॅव॑र्च॒स्वी । \newline
43. व॒र्च॒स्वी भू॑यासम् भूयासं ॅवर्च॒स्वी व॑र्च॒स्वी भू॑यासम् । \newline
44. भू॒या॒सं॒ ॅयद् यद् भू॑यासम् भूयासं॒ ॅयत् । \newline
45. यत् ते॑ ते॒ यद् यत् ते᳚ । \newline
46. ते॒ अ॒ग्ने॒ ऽग्ने॒ ते॒ ते॒ अ॒ग्ने॒ । \newline
47. अ॒ग्ने॒ हरो॒ हरो᳚ ऽग्ने ऽग्ने॒ हरः॑ । \newline
48. हर॒ स्तेन॒ तेन॒ हरो॒ हर॒ स्तेन॑ । \newline
49. तेना॒ह म॒हम् तेन॒ तेना॒हम् । \newline
50. अ॒हꣳ ह॑र॒स्वी ह॑र॒स्व्य॑ह म॒हꣳ ह॑र॒स्वी । \newline
51. ह॒र॒स्वी भू॑यासम् भूयासꣳ हर॒स्वी ह॑र॒स्वी भू॑यासम् । \newline
52. भू॒या॒स॒मिति॑ भूयासम् । \newline

\textbf{Ghana Paata } \newline

1. स॒ह॒जान्. विश्वे॑भि॒र् विश्वे॑भिः सह॒जान् थ्स॑ह॒जान्. विश्वे॑भिर् दे॒वेभि॑र् दे॒वेभि॒र् विश्वे॑भिः सह॒जान् थ्स॑ह॒जान्. विश्वे॑भिर् दे॒वेभिः॑ । \newline
2. स॒ह॒जानिति॑ सह - जान् । \newline
3. विश्वे॑भिर् दे॒वेभि॑र् दे॒वेभि॒र् विश्वे॑भि॒र् विश्वे॑भिर् दे॒वेभिः॒ पृत॑नाः॒ पृत॑ना दे॒वेभि॒र् विश्वे॑भि॒र् विश्वे॑भिर् दे॒वेभिः॒ पृत॑नाः । \newline
4. दे॒वेभिः॒ पृत॑नाः॒ पृत॑ना दे॒वेभि॑र् दे॒वेभिः॒ पृत॑ना जयामि जयामि॒ पृत॑ना दे॒वेभि॑र् दे॒वेभिः॒ पृत॑ना जयामि । \newline
5. पृत॑ना जयामि जयामि॒ पृत॑नाः॒ पृत॑ना जयामि॒ जाग॑तेन॒ जाग॑तेन जयामि॒ पृत॑नाः॒ पृत॑ना जयामि॒ जाग॑तेन । \newline
6. ज॒या॒मि॒ जाग॑तेन॒ जाग॑तेन जयामि जयामि॒ जाग॑तेन॒ छन्द॑सा॒ छन्द॑सा॒ जाग॑तेन जयामि जयामि॒ जाग॑तेन॒ छन्द॑सा । \newline
7. जाग॑तेन॒ छन्द॑सा॒ छन्द॑सा॒ जाग॑तेन॒ जाग॑तेन॒ छन्द॑सा सप्तद॒शेन॑ सप्तद॒शेन॒ छन्द॑सा॒ जाग॑तेन॒ जाग॑तेन॒ छन्द॑सा सप्तद॒शेन॑ । \newline
8. छन्द॑सा सप्तद॒शेन॑ सप्तद॒शेन॒ छन्द॑सा॒ छन्द॑सा सप्तद॒शेन॒ स्तोमे॑न॒ स्तोमे॑न सप्तद॒शेन॒ छन्द॑सा॒ छन्द॑सा सप्तद॒शेन॒ स्तोमे॑न । \newline
9. स॒प्त॒द॒शेन॒ स्तोमे॑न॒ स्तोमे॑न सप्तद॒शेन॑ सप्तद॒शेन॒ स्तोमे॑न वामदे॒व्येन॑ वामदे॒व्येन॒ स्तोमे॑न सप्तद॒शेन॑ सप्तद॒शेन॒ स्तोमे॑न वामदे॒व्येन॑ । \newline
10. स॒प्त॒द॒शेनेति॑ सप्त - द॒शेन॑ । \newline
11. स्तोमे॑न वामदे॒व्येन॑ वामदे॒व्येन॒ स्तोमे॑न॒ स्तोमे॑न वामदे॒व्येन॒ साम्ना॒ साम्ना॑ वामदे॒व्येन॒ स्तोमे॑न॒ स्तोमे॑न वामदे॒व्येन॒ साम्ना᳚ । \newline
12. वा॒म॒दे॒व्येन॒ साम्ना॒ साम्ना॑ वामदे॒व्येन॑ वामदे॒व्येन॒ साम्ना॑ वषट्का॒रेण॑ वषट्का॒रेण॒ साम्ना॑ वामदे॒व्येन॑ वामदे॒व्येन॒ साम्ना॑ वषट्का॒रेण॑ । \newline
13. वा॒म॒दे॒व्येनेति॑ वाम - दे॒व्येन॑ । \newline
14. साम्ना॑ वषट्का॒रेण॑ वषट्का॒रेण॒ साम्ना॒ साम्ना॑ वषट्का॒रेण॒ वज्रे॑ण॒ वज्रे॑ण वषट्का॒रेण॒ साम्ना॒ साम्ना॑ वषट्का॒रेण॒ वज्रे॑ण । \newline
15. व॒ष॒ट्का॒रेण॒ वज्रे॑ण॒ वज्रे॑ण वषट्का॒रेण॑ वषट्का॒रेण॒ वज्रे॑णा पर॒जा न॑पर॒जान्. वज्रे॑ण वषट्का॒रेण॑ वषट्का॒रेण॒ वज्रे॑णा पर॒जान् । \newline
16. व॒ष॒ट्का॒रेणेति॑ वषट् - का॒रेण॑ । \newline
17. वज्रे॑णा पर॒जा न॑पर॒जान्. वज्रे॑ण॒ वज्रे॑णा पर॒जा निन्द्रे॒ णेन्द्रे॑णा पर॒जान्. वज्रे॑ण॒ वज्रे॑णा पर॒जा निन्द्रे॑ण । \newline
18. अ॒प॒र॒जा निन्द्रे॒ णेन्द्रे॑णा पर॒जा न॑पर॒जा निन्द्रे॑ण स॒युजः॑ स॒युज॒ इन्द्रे॑णा पर॒जा न॑पर॒जा निन्द्रे॑ण स॒युजः॑ । \newline
19. अ॒प॒र॒जानित्य॑पर - जान् । \newline
20. इन्द्रे॑ण स॒युजः॑ स॒युज॒ इन्द्रे॒णेन्द्रे॑ण स॒युजो॑ व॒यं ॅव॒यꣳ स॒युज॒ इन्द्रे॒णेन्द्रे॑ण स॒युजो॑ व॒यम् । \newline
21. स॒युजो॑ व॒यं ॅव॒यꣳ स॒युजः॑ स॒युजो॑ व॒यꣳ सा॑स॒ह्याम॑ सास॒ह्याम॑ व॒यꣳ स॒युजः॑ स॒युजो॑ व॒यꣳ सा॑स॒ह्याम॑ । \newline
22. स॒युज॒ इति॑ स - युजः॑ । \newline
23. व॒यꣳ सा॑स॒ह्याम॑ सास॒ह्याम॑ व॒यं ॅव॒यꣳ सा॑स॒ह्याम॑ पृतन्य॒तः पृ॑तन्य॒तः सा॑स॒ह्याम॑ व॒यं ॅव॒यꣳ सा॑स॒ह्याम॑ पृतन्य॒तः । \newline
24. सा॒स॒ह्याम॑ पृतन्य॒तः पृ॑तन्य॒तः सा॑स॒ह्याम॑ सास॒ह्याम॑ पृतन्य॒तः । \newline
25. पृ॒त॒न्य॒त इति॑ पृतन्य॒तः । \newline
26. घ्नन्तो॑ वृ॒त्राणि॑ वृ॒त्राणि॒ घ्नन्तो॒ घ्नन्तो॑ वृ॒त्रा ण्य॑प्र॒ त्य॑प्र॒ति वृ॒त्राणि॒ घ्नन्तो॒ घ्नन्तो॑ वृ॒त्रा ण्य॑प्र॒ति । \newline
27. वृ॒त्रा ण्य॑प्र॒ त्य॑प्र॒ति वृ॒त्राणि॑ वृ॒त्रा ण्य॑प्र॒ति । \newline
28. अ॒प्र॒तीत्य॑प्र॒ति । \newline
29. यत् ते॑ ते॒ यद् यत् ते॑ अग्ने ऽग्ने ते॒ यद् यत् ते॑ अग्ने । \newline
30. ते॒ अ॒ग्ने॒ ऽग्ने॒ ते॒ ते॒ अ॒ग्ने॒ तेज॒ स्तेजो᳚ ऽग्ने ते ते अग्ने॒ तेजः॑ । \newline
31. अ॒ग्ने॒ तेज॒ स्तेजो᳚ ऽग्ने ऽग्ने॒ तेज॒ स्तेन॒ तेन॒ तेजो᳚ ऽग्ने ऽग्ने॒ तेज॒ स्तेन॑ । \newline
32. तेज॒ स्तेन॒ तेन॒ तेज॒ स्तेज॒ स्तेना॒ह म॒हम् तेन॒ तेज॒ स्तेज॒ स्तेना॒हम् । \newline
33. तेना॒ह म॒हम् तेन॒ तेना॒हम् ते॑ज॒स्वी ते॑ज॒स्व्य॑हम् तेन॒ तेना॒हम् ते॑ज॒स्वी । \newline
34. अ॒हम् ते॑ज॒स्वी ते॑ज॒स्व्य॑ह म॒हम् ते॑ज॒स्वी भू॑यासम् भूयासम् तेज॒स्व्य॑ह म॒हम् ते॑ज॒स्वी भू॑यासम् । \newline
35. ते॒ज॒स्वी भू॑यासम् भूयासम् तेज॒स्वी ते॑ज॒स्वी भू॑यासं॒ ॅयद् यद् भू॑यासम् तेज॒स्वी ते॑ज॒स्वी भू॑यासं॒ ॅयत् । \newline
36. भू॒या॒सं॒ ॅयद् यद् भू॑यासम् भूयासं॒ ॅयत् ते॑ ते॒ यद् भू॑यासम् भूयासं॒ ॅयत् ते᳚ । \newline
37. यत् ते॑ ते॒ यद् यत् ते॑ अग्ने ऽग्ने ते॒ यद् यत् ते॑ अग्ने । \newline
38. ते॒ अ॒ग्ने॒ ऽग्ने॒ ते॒ ते॒ अ॒ग्ने॒ वर्चो॒ वर्चो᳚ ऽग्ने ते ते अग्ने॒ वर्चः॑ । \newline
39. अ॒ग्ने॒ वर्चो॒ वर्चो᳚ ऽग्ने ऽग्ने॒ वर्च॒ स्तेन॒ तेन॒ वर्चो᳚ ऽग्ने ऽग्ने॒ वर्च॒ स्तेन॑ । \newline
40. वर्च॒ स्तेन॒ तेन॒ वर्चो॒ वर्च॒ स्तेना॒ह म॒हम् तेन॒ वर्चो॒ वर्च॒ स्तेना॒हम् । \newline
41. तेना॒ह म॒हम् तेन॒ तेना॒हं ॅव॑र्च॒स्वी व॑र्च॒ स्व्य॑हम् तेन॒ तेना॒हं ॅव॑र्च॒स्वी । \newline
42. अ॒हं ॅव॑र्च॒स्वी व॑र्च॒ स्व्य॑ह म॒हं ॅव॑र्च॒स्वी भू॑यासम् भूयासं ॅवर्च॒ स्व्य॑ह म॒हं ॅव॑र्च॒स्वी भू॑यासम् । \newline
43. व॒र्च॒स्वी भू॑यासम् भूयासं ॅवर्च॒स्वी व॑र्च॒स्वी भू॑यासं॒ ॅयद् यद् भू॑यासं ॅवर्च॒स्वी व॑र्च॒स्वी भू॑यासं॒ ॅयत् । \newline
44. भू॒या॒सं॒ ॅयद् यद् भू॑यासम् भूयासं॒ ॅयत् ते॑ ते॒ यद् भू॑यासम् भूयासं॒ ॅयत् ते᳚ । \newline
45. यत् ते॑ ते॒ यद् यत् ते॑ अग्ने ऽग्ने ते॒ यद् यत् ते॑ अग्ने । \newline
46. ते॒ अ॒ग्ने॒ ऽग्ने॒ ते॒ ते॒ अ॒ग्ने॒ हरो॒ हरो᳚ ऽग्ने ते ते अग्ने॒ हरः॑ । \newline
47. अ॒ग्ने॒ हरो॒ हरो᳚ ऽग्ने ऽग्ने॒ हर॒ स्तेन॒ तेन॒ हरो᳚ ऽग्ने ऽग्ने॒ हर॒ स्तेन॑ । \newline
48. हर॒ स्तेन॒ तेन॒ हरो॒ हर॒ स्तेना॒ह म॒हम् तेन॒ हरो॒ हर॒ स्तेना॒हम् । \newline
49. तेना॒ह म॒हम् तेन॒ तेना॒हꣳ ह॑र॒स्वी ह॑र॒ स्व्य॑हम् तेन॒ तेना॒हꣳ ह॑र॒स्वी । \newline
50. अ॒हꣳ ह॑र॒स्वी ह॑र॒ स्व्य॑ह म॒हꣳ ह॑र॒स्वी भू॑यासम् भूयासꣳ हर॒ स्व्य॑ह म॒हꣳ ह॑र॒स्वी भू॑यासम् । \newline
51. ह॒र॒स्वी भू॑यासम् भूयासꣳ हर॒स्वी ह॑र॒स्वी भू॑यासम् । \newline
52. भू॒या॒स॒मिति॑ भूयासम् । \newline
\pagebreak
\markright{ TS 3.5.4.1  \hfill https://www.vedavms.in \hfill}

\section{ TS 3.5.4.1 }

\textbf{TS 3.5.4.1 } \newline
\textbf{Samhita Paata} \newline

ये दे॒वा य॑ज्ञ्॒हनो॑ यज्ञ्॒मुषः॑ पृथि॒व्यामद्ध्यास॑ते ।अ॒ग्निर्मा॒ तेभ्यो॑ रक्षतु॒ गच्छे॑म सु॒कृतो॑ व॒यं ॥ आऽग॑न्म मित्रावरुणा वरेण्या॒ रात्री॑णां भा॒गो यु॒वयो॒र्यो अस्ति॑ । नाकं॑ गृह्णा॒नाः सु॑कृ॒तस्य॑ लो॒के तृ॒तीये॑ पृ॒ष्ठे अधि॑ रोच॒ने दि॒वः ॥ये दे॒वा य॑ज्ञ्॒हनो॑ यज्ञ्॒मुषो॒ऽन्तरि॒क्षेऽद्ध्यास॑ते । वा॒युर्मा॒ तेभ्यो॑ रक्षतु॒ गच्छे॑म सु॒कृतो॑ व॒यं ॥ यास्ते॒ रात्रीः᳚ सवित - [  ] \newline

\textbf{Pada Paata} \newline

ये । दे॒वाः । य॒ज्ञ्॒हन॒ इति॑ यज्ञ् - हनः॑ । य॒ज्ञ्॒मुष॒ इति॑ यज्ञ् - मुषः॑ । पृ॒थि॒व्याम् । अधीति॑ । आस॑ते ॥ अ॒ग्निः । मा॒ । तेभ्यः॑ । र॒क्ष॒तु॒ । गच्छे॑म । सु॒कृत॒ इति॑ सु - कृतः॑ । व॒यम् ॥ एति॑ । अ॒ग॒न्म॒ । मि॒त्रा॒व॒रु॒णेति॑ मित्रा-व॒रु॒णा॒ । व॒रे॒ण्या॒ । रात्री॑णाम् । भा॒गः । यु॒वयोः᳚ । यः । अस्ति॑ ॥ नाक᳚म् । गृ॒ह्णा॒नाः । सु॒कृ॒तस्येति॑ सु - कृ॒तस्य॑ । लो॒के । तृ॒तीये᳚ । पृ॒ष्ठे । अधीति॑ । रो॒च॒ने । दि॒वः ॥ ये । दे॒वाः । य॒ज्ञ्॒हन॒ इति॑ यज्ञ् - हनः॑ । य॒ज्ञ्॒मुष॒ इति॑ यज्ञ् - मुषः॑ । अ॒न्तरि॑क्षे । अधीति॑ । आस॑ते ॥ वा॒युः । मा॒ । तेभ्यः॑ । र॒क्ष॒तु॒ । गच्छे॑म । सु॒कृत॒ इति॑ सु - कृतः॑ । व॒यम् ॥ याः । ते॒ । रात्रीः᳚ । स॒वि॒तः॒ ।  \newline


\textbf{Krama Paata} \newline

ये दे॒वाः । दे॒वा य॑ज्ञ्॒हनः॑ । य॒ज्ञ्॒हनो॑ यज्ञ्॒मुषः॑ । य॒ज्ञ्॒हन॒ इति॑ यज्ञ् - हनः॑ । य॒ज्ञ्॒मुषः॑ पृथि॒व्याम् । य॒ज्ञ्॒मुष॒ इति॑ यज्ञ् - मुषः॑ । पृ॒थि॒व्यामधि॑ । अध्यास॑ते । आस॑त॒ इत्यास॑ते ॥ अ॒ग्निर् मा᳚ । मा॒ तेभ्यः॑ । तेभ्यो॑ रक्षतु । र॒क्ष॒तु॒ गच्छे॑म । गच्छे॑म सु॒कृतः॑ । सु॒कृतो॑ व॒यम् । सु॒कृत॒ इति॑ सु - कृतः॑ । व॒यमिति॑ व॒यम् ॥ आ ऽग॑न्म । अ॒ग॒न्म॒ मि॒त्रा॒व॒रु॒णा॒ । मि॒त्रा॒व॒रु॒णा॒ व॒रे॒ण्या॒ । मि॒त्रा॒व॒रु॒णेति॑ मित्रा - व॒रु॒णा॒ । व॒रे॒ण्या॒ रात्री॑णाम् । रात्री॑णाम् भा॒गः । भा॒गो यु॒वयोः᳚ । यु॒वयो॒र् यः । यो अस्ति॑ । अस्ती॒त्यस्ति॑ ॥ नाक॑म् गृह्णा॒नाः । गृ॒ह्णा॒नाः सु॑कृ॒तस्य॑ । सु॒कृ॒तस्य॑ लो॒के । सु॒कृ॒तस्येति॑ सु - कृ॒तस्य॑ । लो॒के तृ॒तीये᳚ । तृ॒तीये॑ पृ॒ष्ठे । पृ॒ष्ठे अधि॑ । अधि॑ रोच॒ने । रो॒च॒ने दि॒वः । दि॒व इति॑ दि॒वः ॥ ये दे॒वाः । दे॒वा य॑ज्ञ्॒हनः॑ । य॒ज्ञ्॒हनो॑ यज्ञ्॒मुषः॑ । य॒ज्ञ्॒हन॒ इति॑ यज्ञ् - हनः॑ । य॒ज्ञ्॒मुषो॒ ऽन्तरि॑क्षे । य॒ज्ञ्॒मुष॒ इति॑ यज्ञ् - मुषः॑ । अ॒न्तरि॒क्षे ऽधि॑ । अध्यास॑ते । आस॑त॒ इत्यास॑ते ॥ वा॒युर् मा᳚ । मा॒ तेभ्यः॑ । तेभ्यो॑ रक्षतु । र॒क्ष॒तु॒ गच्छे॑म । गच्छे॑म सु॒कृतः॑ । सु॒कृतो॑ व॒यम् । सु॒कृत॒ इति॑ सु - कृतः॑ । व॒यमिति॑ व॒यम् ॥ यास्ते᳚ । ते॒ रात्रीः᳚ । रात्रीः᳚ सवितः । स॒वि॒त॒र् दे॒व॒यानीः᳚ \newline

\textbf{Jatai Paata} \newline

1. ये दे॒वा दे॒वा ये ये दे॒वाः । \newline
2. दे॒वा य॑ज्ञ्॒हनो॑ यज्ञ्॒हनो॑ दे॒वा दे॒वा य॑ज्ञ्॒हनः॑ । \newline
3. य॒ज्ञ्॒हनो॑ यज्ञ्॒मुषो॑ यज्ञ्॒मुषो॑ यज्ञ्॒हनो॑ यज्ञ्॒हनो॑ यज्ञ्॒मुषः॑ । \newline
4. य॒ज्ञ्॒हन॒ इति॑ यज्ञ् - हनः॑ । \newline
5. य॒ज्ञ्॒मुषः॑ पृथि॒व्याम् पृ॑थि॒व्यां ॅय॑ज्ञ्॒मुषो॑ यज्ञ्॒मुषः॑ पृथि॒व्याम् । \newline
6. य॒ज्ञ्॒मुष॒ इति॑ यज्ञ् - मुषः॑ । \newline
7. पृ॒थि॒व्या मध्यधि॑ पृथि॒व्याम् पृ॑थि॒व्या मधि॑ । \newline
8. अध्यास॑त॒ आस॒ते ऽध्यध्यास॑ते । \newline
9. आस॑त॒ इत्यास॑ते । \newline
10. अ॒ग्निर् मा॑ मा॒ ऽग्नि र॒ग्निर् मा᳚ । \newline
11. मा॒ तेभ्य॒ स्तेभ्यो॑ मा मा॒ तेभ्यः॑ । \newline
12. तेभ्यो॑ रक्षतु रक्षतु॒ तेभ्य॒ स्तेभ्यो॑ रक्षतु । \newline
13. र॒क्ष॒तु॒ गच्छे॑म॒ गच्छे॑म रक्षतु रक्षतु॒ गच्छे॑म । \newline
14. गच्छे॑म सु॒कृतः॑ सु॒कृतो॒ गच्छे॑म॒ गच्छे॑म सु॒कृतः॑ । \newline
15. सु॒कृतो॑ व॒यं ॅव॒यꣳ सु॒कृतः॑ सु॒कृतो॑ व॒यम् । \newline
16. सु॒कृत॒ इति॑ सु - कृतः॑ । \newline
17. व॒यमिति॑ व॒यम् । \newline
18. आ ऽग॑न्मा ग॒न्मा ऽग॑न्म । \newline
19. अ॒ग॒न्म॒ मि॒त्रा॒व॒रु॒णा॒ मि॒त्रा॒व॒रु॒णा॒ ऽग॒न्मा॒ ग॒न्म॒ मि॒त्रा॒व॒रु॒णा॒ । \newline
20. मि॒त्रा॒व॒रु॒णा॒ व॒रे॒ण्या॒ व॒रे॒ण्या॒ मि॒त्रा॒व॒रु॒णा॒ मि॒त्रा॒व॒रु॒णा॒ व॒रे॒ण्या॒ । \newline
21. मि॒त्रा॒व॒रु॒णेति॑ मित्रा - व॒रु॒णा॒ । \newline
22. व॒रे॒ण्या॒ रात्री॑णाꣳ॒॒ रात्री॑णां ॅवरेण्या वरेण्या॒ रात्री॑णाम् । \newline
23. रात्री॑णाम् भा॒गो भा॒गो रात्री॑णाꣳ॒॒ रात्री॑णाम् भा॒गः । \newline
24. भा॒गो यु॒वयो᳚र् यु॒वयो᳚र् भा॒गो भा॒गो यु॒वयोः᳚ । \newline
25. यु॒वयो॒र् यो यो यु॒वयो᳚र् यु॒वयो॒र् यः । \newline
26. यो अस्त्यस्ति॒ यो यो अस्ति॑ । \newline
27. अस्तीत्यस्ति॑ । \newline
28. नाक॑म् गृह्णा॒ना गृ॑ह्णा॒ना नाक॒म् नाक॑म् गृह्णा॒नाः । \newline
29. गृ॒ह्णा॒नाः सु॑कृ॒तस्य॑ सुकृ॒तस्य॑ गृह्णा॒ना गृ॑ह्णा॒नाः सु॑कृ॒तस्य॑ । \newline
30. सु॒कृ॒तस्य॑ लो॒के लो॒के सु॑कृ॒तस्य॑ सुकृ॒तस्य॑ लो॒के । \newline
31. सु॒कृ॒तस्येति॑ सु - कृ॒तस्य॑ । \newline
32. लो॒के तृ॒तीये॑ तृ॒तीये॑ लो॒के लो॒के तृ॒तीये᳚ । \newline
33. तृ॒तीये॑ पृ॒ष्ठे पृ॒ष्ठे तृ॒तीये॑ तृ॒तीये॑ पृ॒ष्ठे । \newline
34. पृ॒ष्ठे अध्यधि॑ पृ॒ष्ठे पृ॒ष्ठे अधि॑ । \newline
35. अधि॑ रोच॒ने रो॑च॒ने ऽध्यधि॑ रोच॒ने । \newline
36. रो॒च॒ने दि॒वो दि॒वो रो॑च॒ने रो॑च॒ने दि॒वः । \newline
37. दि॒व इति॑ दि॒वः । \newline
38. ये दे॒वा दे॒वा ये ये दे॒वाः । \newline
39. दे॒वा य॑ज्ञ्॒हनो॑ यज्ञ्॒हनो॑ दे॒वा दे॒वा य॑ज्ञ्॒हनः॑ । \newline
40. य॒ज्ञ्॒हनो॑ यज्ञ्॒मुषो॑ यज्ञ्॒मुषो॑ यज्ञ्॒हनो॑ यज्ञ्॒हनो॑ यज्ञ्॒मुषः॑ । \newline
41. य॒ज्ञ्॒हन॒ इति॑ यज्ञ् - हनः॑ । \newline
42. य॒ज्ञ्॒मुषो॒ ऽन्तरि॑क्षे॒ ऽन्तरि॑क्षे यज्ञ्॒मुषो॑ यज्ञ्॒मुषो॒ ऽन्तरि॑क्षे । \newline
43. य॒ज्ञ्॒मुष॒ इति॑ यज्ञ् - मुषः॑ । \newline
44. अ॒न्तरि॒क्षे ऽध्य ध्य॒न्तरि॑क्षे॒ ऽन्तरि॒क्षे ऽधि॑ । \newline
45. अध्यास॑त॒ आस॒ते ऽध्य ध्यास॑ते । \newline
46. आस॑त॒ इत्यास॑ते । \newline
47. वा॒युर् मा॑ मा वा॒युर् वा॒युर् मा᳚ । \newline
48. मा॒ तेभ्य॒ स्तेभ्यो॑ मा मा॒ तेभ्यः॑ । \newline
49. तेभ्यो॑ रक्षतु रक्षतु॒ तेभ्य॒ स्तेभ्यो॑ रक्षतु । \newline
50. र॒क्ष॒तु॒ गच्छे॑म॒ गच्छे॑म रक्षतु रक्षतु॒ गच्छे॑म । \newline
51. गच्छे॑म सु॒कृतः॑ सु॒कृतो॒ गच्छे॑म॒ गच्छे॑म सु॒कृतः॑ । \newline
52. सु॒कृतो॑ व॒यं ॅव॒यꣳ सु॒कृतः॑ सु॒कृतो॑ व॒यम् । \newline
53. सु॒कृत॒ इति॑ सु - कृतः॑ । \newline
54. व॒यमिति॑ व॒यम् । \newline
55. या स्ते॑ ते॒ या या स्ते᳚ । \newline
56. ते॒ रात्री॒ रात्री᳚ स्ते ते॒ रात्रीः᳚ । \newline
57. रात्रीः᳚ सवितः सविता॒ रात्री॒ रात्रीः᳚ सवितः । \newline
58. स॒वि॒त॒र् दे॒व॒यानी᳚र् देव॒यानीः᳚ सवितः सवितर् देव॒यानीः᳚ । \newline

\textbf{Ghana Paata } \newline

1. ये दे॒वा दे॒वा ये ये दे॒वा य॑ज्ञ्॒हनो॑ यज्ञ्॒हनो॑ दे॒वा ये ये दे॒वा य॑ज्ञ्॒हनः॑ । \newline
2. दे॒वा य॑ज्ञ्॒हनो॑ यज्ञ्॒हनो॑ दे॒वा दे॒वा य॑ज्ञ्॒हनो॑ यज्ञ्॒मुषो॑ यज्ञ्॒मुषो॑ यज्ञ्॒हनो॑ दे॒वा दे॒वा य॑ज्ञ्॒हनो॑ यज्ञ्॒मुषः॑ । \newline
3. य॒ज्ञ्॒हनो॑ यज्ञ्॒मुषो॑ यज्ञ्॒मुषो॑ यज्ञ्॒हनो॑ यज्ञ्॒हनो॑ यज्ञ्॒मुषः॑ पृथि॒व्याम् पृ॑थि॒व्यां ॅय॑ज्ञ्॒मुषो॑ यज्ञ्॒हनो॑ यज्ञ्॒हनो॑ यज्ञ्॒मुषः॑ पृथि॒व्याम् । \newline
4. य॒ज्ञ्॒हन॒ इति॑ यज्ञ् - हनः॑ । \newline
5. य॒ज्ञ्॒मुषः॑ पृथि॒व्याम् पृ॑थि॒व्यां ॅय॑ज्ञ्॒मुषो॑ यज्ञ्॒मुषः॑ पृथि॒व्या मध्यधि॑ पृथि॒व्यां ॅय॑ज्ञ्॒मुषो॑ यज्ञ्॒मुषः॑ पृथि॒व्या मधि॑ । \newline
6. य॒ज्ञ्॒मुष॒ इति॑ यज्ञ् - मुषः॑ । \newline
7. पृ॒थि॒व्या मध्यधि॑ पृथि॒व्याम् पृ॑थि॒व्या मध्यास॑त॒ आस॒ते ऽधि॑ पृथि॒व्याम् पृ॑थि॒व्या मध्यास॑ते । \newline
8. अध्यास॑त॒ आस॒ते ऽध्यध्यास॑ते । \newline
9. आस॑त॒ इत्यास॑ते । \newline
10. अ॒ग्निर् मा॑ मा॒ ऽग्नि र॒ग्निर् मा॒ तेभ्य॒ स्तेभ्यो॑ मा॒ ऽग्नि र॒ग्निर् मा॒ तेभ्यः॑ । \newline
11. मा॒ तेभ्य॒ स्तेभ्यो॑ मा मा॒ तेभ्यो॑ रक्षतु रक्षतु॒ तेभ्यो॑ मा मा॒ तेभ्यो॑ रक्षतु । \newline
12. तेभ्यो॑ रक्षतु रक्षतु॒ तेभ्य॒ स्तेभ्यो॑ रक्षतु॒ गच्छे॑म॒ गच्छे॑म रक्षतु॒ तेभ्य॒ स्तेभ्यो॑ रक्षतु॒ गच्छे॑म । \newline
13. र॒क्ष॒तु॒ गच्छे॑म॒ गच्छे॑म रक्षतु रक्षतु॒ गच्छे॑म सु॒कृतः॑ सु॒कृतो॒ गच्छे॑म रक्षतु रक्षतु॒ गच्छे॑म सु॒कृतः॑ । \newline
14. गच्छे॑म सु॒कृतः॑ सु॒कृतो॒ गच्छे॑म॒ गच्छे॑म सु॒कृतो॑ व॒यं ॅव॒यꣳ सु॒कृतो॒ गच्छे॑म॒ गच्छे॑म सु॒कृतो॑ व॒यम् । \newline
15. सु॒कृतो॑ व॒यं ॅव॒यꣳ सु॒कृतः॑ सु॒कृतो॑ व॒यम् । \newline
16. सु॒कृत॒ इति॑ सु - कृतः॑ । \newline
17. व॒यमिति॑ व॒यम् । \newline
18. आ ऽग॑न्मा ग॒न्मा ऽग॑न्म मित्रावरुणा मित्रावरुणा ऽग॒न्मा ऽग॑न्म मित्रावरुणा । \newline
19. अ॒ग॒न्म॒ मि॒त्रा॒व॒रु॒णा॒ मि॒त्रा॒व॒रु॒णा॒ ऽग॒न्मा॒ ग॒न्म॒ मि॒त्रा॒व॒रु॒णा॒ व॒रे॒ण्या॒ व॒रे॒ण्या॒ मि॒त्रा॒व॒रु॒णा॒ ऽग॒न्मा॒ ग॒न्म॒ मि॒त्रा॒व॒रु॒णा॒ व॒रे॒ण्या॒ । \newline
20. मि॒त्रा॒व॒रु॒णा॒ व॒रे॒ण्या॒ व॒रे॒ण्या॒ मि॒त्रा॒व॒रु॒णा॒ मि॒त्रा॒व॒रु॒णा॒ व॒रे॒ण्या॒ रात्री॑णाꣳ॒॒ रात्री॑णां ॅवरेण्या मित्रावरुणा मित्रावरुणा वरेण्या॒ रात्री॑णाम् । \newline
21. मि॒त्रा॒व॒रु॒णेति॑ मित्रा - व॒रु॒णा॒ । \newline
22. व॒रे॒ण्या॒ रात्री॑णाꣳ॒॒ रात्री॑णां ॅवरेण्या वरेण्या॒ रात्री॑णाम् भा॒गो भा॒गो रात्री॑णां ॅवरेण्या वरेण्या॒ रात्री॑णाम् भा॒गः । \newline
23. रात्री॑णाम् भा॒गो भा॒गो रात्री॑णाꣳ॒॒ रात्री॑णाम् भा॒गो यु॒वयो᳚र् यु॒वयो᳚र् भा॒गो रात्री॑णाꣳ॒॒ रात्री॑णाम् भा॒गो यु॒वयोः᳚ । \newline
24. भा॒गो यु॒वयो᳚र् यु॒वयो᳚र् भा॒गो भा॒गो यु॒वयो॒र् यो यो यु॒वयो᳚र् भा॒गो भा॒गो यु॒वयो॒र् यः । \newline
25. यु॒वयो॒र् यो यो यु॒वयो᳚र् यु॒वयो॒र् यो अस्त्यस्ति॒ यो यु॒वयो᳚र् यु॒वयो॒र् यो अस्ति॑ । \newline
26. यो अस्त्यस्ति॒ यो यो अस्ति॑ । \newline
27. अस्तीत्यस्ति॑ । \newline
28. नाक॑म् गृह्णा॒ना गृ॑ह्णा॒ना नाक॒न्नाक॑म् गृह्णा॒नाः सु॑कृ॒तस्य॑ सुकृ॒तस्य॑ गृह्णा॒ना नाक॒न्नाक॑म् गृह्णा॒नाः सु॑कृ॒तस्य॑ । \newline
29. गृ॒ह्णा॒नाः सु॑कृ॒तस्य॑ सुकृ॒तस्य॑ गृह्णा॒ना गृ॑ह्णा॒नाः सु॑कृ॒तस्य॑ लो॒के लो॒के सु॑कृ॒तस्य॑ गृह्णा॒ना गृ॑ह्णा॒नाः सु॑कृ॒तस्य॑ लो॒के । \newline
30. सु॒कृ॒तस्य॑ लो॒के लो॒के सु॑कृ॒तस्य॑ सुकृ॒तस्य॑ लो॒के तृ॒तीये॑ तृ॒तीये॑ लो॒के सु॑कृ॒तस्य॑ सुकृ॒तस्य॑ लो॒के तृ॒तीये᳚ । \newline
31. सु॒कृ॒तस्येति॑ सु - कृ॒तस्य॑ । \newline
32. लो॒के तृ॒तीये॑ तृ॒तीये॑ लो॒के लो॒के तृ॒तीये॑ पृ॒ष्ठे पृ॒ष्ठे तृ॒तीये॑ लो॒के लो॒के तृ॒तीये॑ पृ॒ष्ठे । \newline
33. तृ॒तीये॑ पृ॒ष्ठे पृ॒ष्ठे तृ॒तीये॑ तृ॒तीये॑ पृ॒ष्ठे अध्यधि॑ पृ॒ष्ठे तृ॒तीये॑ तृ॒तीये॑ पृ॒ष्ठे अधि॑ । \newline
34. पृ॒ष्ठे अध्यधि॑ पृ॒ष्ठे पृ॒ष्ठे अधि॑ रोच॒ने रो॑च॒ने ऽधि॑ पृ॒ष्ठे पृ॒ष्ठे अधि॑ रोच॒ने । \newline
35. अधि॑ रोच॒ने रो॑च॒ने ऽध्यधि॑ रोच॒ने दि॒वो दि॒वो रो॑च॒ने ऽध्यधि॑ रोच॒ने दि॒वः । \newline
36. रो॒च॒ने दि॒वो दि॒वो रो॑च॒ने रो॑च॒ने दि॒वः । \newline
37. दि॒व इति॑ दि॒वः । \newline
38. ये दे॒वा दे॒वा ये ये दे॒वा य॑ज्ञ्॒हनो॑ यज्ञ्॒हनो॑ दे॒वा ये ये दे॒वा य॑ज्ञ्॒हनः॑ । \newline
39. दे॒वा य॑ज्ञ्॒हनो॑ यज्ञ्॒हनो॑ दे॒वा दे॒वा य॑ज्ञ्॒हनो॑ यज्ञ्॒मुषो॑ यज्ञ्॒मुषो॑ यज्ञ्॒हनो॑ दे॒वा दे॒वा य॑ज्ञ्॒हनो॑ यज्ञ्॒मुषः॑ । \newline
40. य॒ज्ञ्॒हनो॑ यज्ञ्॒मुषो॑ यज्ञ्॒मुषो॑ यज्ञ्॒हनो॑ यज्ञ्॒हनो॑ यज्ञ्॒मुषो॒ ऽन्तरि॑क्षे॒ ऽन्तरि॑क्षे यज्ञ्॒मुषो॑ यज्ञ्॒हनो॑ यज्ञ्॒हनो॑ यज्ञ्॒मुषो॒ ऽन्तरि॑क्षे । \newline
41. य॒ज्ञ्॒हन॒ इति॑ यज्ञ् - हनः॑ । \newline
42. य॒ज्ञ्॒मुषो॒ ऽन्तरि॑क्षे॒ ऽन्तरि॑क्षे यज्ञ्॒मुषो॑ यज्ञ्॒मुषो॒ ऽन्तरि॒क्षे ऽध्यध्य॒न्तरि॑क्षे यज्ञ्॒मुषो॑ यज्ञ्॒मुषो॒ ऽन्तरि॒क्षे ऽधि॑ । \newline
43. य॒ज्ञ्॒मुष॒ इति॑ यज्ञ् - मुषः॑ । \newline
44. अ॒न्तरि॒क्षे ऽध्यध्य॒न्तरि॑क्षे॒ ऽन्तरि॒क्षे ऽध्यास॑त॒ आस॒ते ऽध्य॒न्तरि॑क्षे॒ ऽन्तरि॒क्षे ऽध्यास॑ते । \newline
45. अध्यास॑त॒ आस॒ते ऽध्यध्यास॑ते । \newline
46. आस॑त॒ इत्यास॑ते । \newline
47. वा॒युर् मा॑ मा वा॒युर् वा॒युर् मा॒ तेभ्य॒ स्तेभ्यो॑ मा वा॒युर् वा॒युर् मा॒ तेभ्यः॑ । \newline
48. मा॒ तेभ्य॒ स्तेभ्यो॑ मा मा॒ तेभ्यो॑ रक्षतु रक्षतु॒ तेभ्यो॑ मा मा॒ तेभ्यो॑ रक्षतु । \newline
49. तेभ्यो॑ रक्षतु रक्षतु॒ तेभ्य॒ स्तेभ्यो॑ रक्षतु॒ गच्छे॑म॒ गच्छे॑म रक्षतु॒ तेभ्य॒ स्तेभ्यो॑ रक्षतु॒ गच्छे॑म । \newline
50. र॒क्ष॒तु॒ गच्छे॑म॒ गच्छे॑म रक्षतु रक्षतु॒ गच्छे॑म सु॒कृतः॑ सु॒कृतो॒ गच्छे॑म रक्षतु रक्षतु॒ गच्छे॑म सु॒कृतः॑ । \newline
51. गच्छे॑म सु॒कृतः॑ सु॒कृतो॒ गच्छे॑म॒ गच्छे॑म सु॒कृतो॑ व॒यं ॅव॒यꣳ सु॒कृतो॒ गच्छे॑म॒ गच्छे॑म सु॒कृतो॑ व॒यम् । \newline
52. सु॒कृतो॑ व॒यं ॅव॒यꣳ सु॒कृतः॑ सु॒कृतो॑ व॒यम् । \newline
53. सु॒कृत॒ इति॑ सु - कृतः॑ । \newline
54. व॒यमिति॑ व॒यम् । \newline
55. या स्ते॑ ते॒ या या स्ते॒ रात्री॒ रात्री᳚ स्ते॒ या या स्ते॒ रात्रीः᳚ । \newline
56. ते॒ रात्री॒ रात्री᳚ स्ते ते॒ रात्रीः᳚ सवितः सविता॒ रात्री᳚ स्ते ते॒ रात्रीः᳚ सवितः । \newline
57. रात्रीः᳚ सवितः सविता॒ रात्री॒ रात्रीः᳚ सवितर् देव॒यानी᳚र् देव॒यानीः᳚ सविता॒ रात्री॒ रात्रीः᳚ सवितर् देव॒यानीः᳚ । \newline
58. स॒वि॒त॒र् दे॒व॒यानी᳚र् देव॒यानीः᳚ सवितः सवितर् देव॒यानी॑ रन्त॒रा ऽन्त॒रा दे॑व॒यानीः᳚ सवितः सवितर् देव॒यानी॑ रन्त॒रा । \newline
\pagebreak
\markright{ TS 3.5.4.2  \hfill https://www.vedavms.in \hfill}

\section{ TS 3.5.4.2 }

\textbf{TS 3.5.4.2 } \newline
\textbf{Samhita Paata} \newline

-र्देव॒यानी॑रन्त॒रा द्यावा॑पृथि॒वी वि॒यन्ति॑ । गृ॒हैश्च॒ सर्वैः᳚ प्र॒जया॒ न्वग्रे॒ सुवो॒ रुहा॑णास्तरता॒ रजाꣳ॑सि ॥ ये दे॒वा य॑ज्ञ्॒हनो॑ यज्ञ्॒मुषो॑ दि॒व्यद्ध्यास॑ते । सूर्यो॑ मा॒ तेभ्यो॑ रक्षतु॒ गच्छे॑म सु॒कृतो॑ व॒यं ॥ येनेन्द्रा॑य स॒मभ॑रः॒ पयाꣳ॑स्युत्त॒मेन॑ ह॒विषा॑ जातवेदः । तेना᳚ऽग्ने॒ त्वमु॒त व॑र्द्धये॒मꣳ स॑जा॒तानाꣳ॒॒ श्रैष्ठ्य॒ आ धे᳚ह्येनं ॥ य॒ज्ञ्॒हनो॒ वै दे॒वा य॑ज्ञ्॒मुषः॑ - [  ] \newline

\textbf{Pada Paata} \newline

दे॒व॒यानी॒रिति॑ देव - यानीः᳚ । अ॒न्त॒रा । द्यावा॑पृथि॒वी इति॒ द्यावा᳚ - पृ॒थि॒वी । वि॒यन्तीति॑ वि - यन्ति॑ ॥ गृ॒हैः । च॒ । सर्वैः᳚ । प्र॒जयेति॑ प्र - जया᳚ । नु । अग्रे᳚ । सुवः॑ । रुहा॑णाः । त॒र॒त॒ । रजाꣳ॑सि ॥ ये । दे॒वाः । य॒ज्ञ्॒हन॒ इति॑ यज्ञ् - हनः॑ । य॒ज्ञ्॒मुष॒ इति॑ यज्ञ् - मुषः॑ । दि॒वि । अधीति॑ । आस॑ते ॥ सूर्यः॑ । मा॒ । तेभ्यः॑ । र॒क्ष॒तु॒ । गच्छे॑म । सु॒कृत॒ इति॑ सु - कृतः॑ । व॒यम् ॥ येन॑ । इन्द्रा॑य । स॒मभ॑र॒ इति॑ सं - अभ॑रः । पयाꣳ॑सि । उ॒त्त॒मेनेत्यु॑त् - त॒मेन॑ । ह॒विषा᳚ । जा॒त॒वे॒द॒ इति॑ जात - वे॒दः॒ ॥ तेन॑ । अ॒ग्ने॒ । त्वम् । उ॒त । व॒र्द्ध॒य॒ । इ॒मम् । स॒जा॒ताना॒मिति॑ स - जा॒ताना᳚म् । श्रैष्ठ्ये᳚ । एति॑ । धे॒हि॒ । ए॒न॒म् ॥ य॒ज्ञ्॒हन॒ इति॑ यज्ञ् - हनः॑ । वै । दे॒वाः । य॒ज्ञ्॒मुष॒ इति॑ यज्ञ् - मुषः॑ ।  \newline


\textbf{Krama Paata} \newline

दे॒व॒यानी॑रन्त॒रा । दे॒व॒यानी॒रिति॑ देव - यानीः᳚ । अ॒न्त॒रा द्यावा॑पृथि॒वी । द्यावा॑पृथि॒वी वि॒यन्ति॑ । द्यावा॑पृथि॒वी इति॒ द्यावा᳚ - पृ॒थि॒वी । वि॒यन्तीति॑ वि - यन्ति॑ ॥ गृ॒हैश्च॑ । च॒ सर्वैः᳚ । सर्वैः᳚ प्र॒जया᳚ । प्र॒जया॒ नु । प्र॒जयेति॑ प्र - जया᳚ । न्वग्रे᳚ । अग्रे॒ सुवः॑ । सुवो॒ रुहा॑णाः । रुहा॑णास्तरत । त॒र॒ता॒ रजाꣳ॑सि । रजाꣳ॒॒सीति॒ रजाꣳ॑सि ॥ ये दे॒वाः । दे॒वा य॑ज्ञ्॒हनः॑ । य॒ज्ञ्॒हनो॑ यज्ञ्॒मुषः॑ । य॒ज्ञ्॒हन॒ इति॑ यज्ञ् - हनः॑ । य॒ज्ञ्॒मुषो॑ दि॒वि । य॒ज्ञ्॒मुष॒ इति॑ यज्ञ् - मुषः॑ । दि॒व्यधि॑ । अध्यास॑ते । आस॑त॒ इत्यास॑ते ॥ सूर्यो॑ मा । मा॒ तेभ्यः॑ । तेभ्यो॑ रक्षतु । र॒क्ष॒तु॒ गच्छे॑म । गच्छे॑म सु॒कृतः॑ । सु॒कृतो॑ व॒यम् । सु॒कृत॒ इति॑ सु - कृतः॑ । व॒यमिति॑ व॒यम् ॥ येनेन्द्रा॑य । इन्द्रा॑य स॒मभ॑रः । स॒मभ॑रः॒ पयाꣳ॑सि । स॒मभ॑र॒ इति॑ सम् - अभ॑रः । पयाꣳ॑स्युत्त॒मेन॑ । उ॒त्त॒मेन॑ ह॒विषा᳚ । उ॒त्त॒मेनेत्यु॑त् - त॒मेन॑ । ह॒विषा॑ जातवेदः । जा॒त॒वे॒द॒ इति॑ जात - वे॒दः॒ ॥ तेना᳚ग्ने । अ॒ग्ने॒ त्वम् । त्वमु॒त । उ॒त व॑र्द्धय । व॒र्द्ध॒ये॒मम् । इ॒मꣳ स॑जा॒ताना᳚म् । स॒जा॒तानाꣳ॒॒ श्रैष्ठ्ये᳚ । स॒जा॒ताना॒मिति॑ स - जा॒ताना᳚म् । श्रेष्ठ्य॒ आ । आ धे॑हि । धे॒ह्ये॒न॒म् । ए॒न॒मित्ये॑नम् ॥ य॒ज्ञ्॒हनो॒ वै । य॒ज्ञ्॒हन॒ इति॑ यज्ञ् - हनः॑ । वै दे॒वाः । दे॒वा य॑ज्ञ्॒मुषः॑ । य॒ज्ञ्॒मुषः॑ सन्ति । य॒ज्ञ्॒मुष॒ इति॑ यज्ञ् - मुषः॑ \newline

\textbf{Jatai Paata} \newline

1. दे॒व॒यानी॑ रन्त॒रा ऽन्त॒रा दे॑व॒यानी᳚र् देव॒यानी॑ रन्त॒रा । \newline
2. दे॒व॒यानी॒रिति॑ देव - यानीः᳚ । \newline
3. अ॒न्त॒रा द्यावा॑पृथि॒वी द्यावा॑पृथि॒वी अ॑न्त॒रा ऽन्त॒रा द्यावा॑पृथि॒वी । \newline
4. द्यावा॑पृथि॒वी वि॒यन्ति॑ वि॒यन्ति॒ द्यावा॑पृथि॒वी द्यावा॑पृथि॒वी वि॒यन्ति॑ । \newline
5. द्यावा॑पृथि॒वी इति॒ द्यावा᳚ - पृ॒थि॒वी । \newline
6. वि॒यन्तीति॑ वि - यन्ति॑ । \newline
7. गृ॒हैश्च॑ च गृ॒हैर् गृ॒हैश्च॑ । \newline
8. च॒ सर्वैः॒ सर्वै᳚श्च च॒ सर्वैः᳚ । \newline
9. सर्वैः᳚ प्र॒जया᳚ प्र॒जया॒ सर्वैः॒ सर्वैः᳚ प्र॒जया᳚ । \newline
10. प्र॒जया॒ नु नु प्र॒जया᳚ प्र॒जया॒ नु । \newline
11. प्र॒जयेति॑ प्र - जया᳚ । \newline
12. न्वग्रे ऽग्रे॒ नु न्वग्रे᳚ । \newline
13. अग्रे॒ सुवः॒ सुव॒ रग्रे ऽग्रे॒ सुवः॑ । \newline
14. सुवो॒ रुहा॑णा॒ रुहा॑णाः॒ सुवः॒ सुवो॒ रुहा॑णाः । \newline
15. रुहा॑णा स्तरत तरत॒ रुहा॑णा॒ रुहा॑णा स्तरत । \newline
16. त॒र॒ता॒ रजाꣳ॑सि॒ रजाꣳ॑सि तरत तरता॒ रजाꣳ॑सि । \newline
17. रजाꣳ॒॒सीति॒ रजाꣳ॑सि । \newline
18. ये दे॒वा दे॒वा ये ये दे॒वाः । \newline
19. दे॒वा य॑ज्ञ्॒हनो॑ यज्ञ्॒हनो॑ दे॒वा दे॒वा य॑ज्ञ्॒हनः॑ । \newline
20. य॒ज्ञ्॒हनो॑ यज्ञ्॒मुषो॑ यज्ञ्॒मुषो॑ यज्ञ्॒हनो॑ यज्ञ्॒हनो॑ यज्ञ्॒मुषः॑ । \newline
21. य॒ज्ञ्॒हन॒ इति॑ यज्ञ् - हनः॑ । \newline
22. य॒ज्ञ्॒मुषो॑ दि॒वि दि॒वि य॑ज्ञ्॒मुषो॑ यज्ञ्॒मुषो॑ दि॒वि । \newline
23. य॒ज्ञ्॒मुष॒ इति॑ यज्ञ् - मुषः॑ । \newline
24. दि॒व्यध्यधि॑ दि॒वि दि॒व्यधि॑ । \newline
25. अध्यास॑त॒ आस॒ते ऽध्य ध्यास॑ते । \newline
26. आस॑त॒ इत्यास॑ते । \newline
27. सूर्यो॑ मा मा॒ सूर्यः॒ सूर्यो॑ मा । \newline
28. मा॒ तेभ्य॒ स्तेभ्यो॑ मा मा॒ तेभ्यः॑ । \newline
29. तेभ्यो॑ रक्षतु रक्षतु॒ तेभ्य॒ स्तेभ्यो॑ रक्षतु । \newline
30. र॒क्ष॒तु॒ गच्छे॑म॒ गच्छे॑म रक्षतु रक्षतु॒ गच्छे॑म । \newline
31. गच्छे॑म सु॒कृतः॑ सु॒कृतो॒ गच्छे॑म॒ गच्छे॑म सु॒कृतः॑ । \newline
32. सु॒कृतो॑ व॒यं ॅव॒यꣳ सु॒कृतः॑ सु॒कृतो॑ व॒यम् । \newline
33. सु॒कृत॒ इति॑ सु - कृतः॑ । \newline
34. व॒यमिति॑ व॒यम् । \newline
35. येनेन्द्रा॒ येन्द्रा॑य॒ येन॒ येनेन्द्रा॑य । \newline
36. इन्द्रा॑य स॒मभ॑रः स॒मभ॑र॒ इन्द्रा॒ येन्द्रा॑य स॒मभ॑रः । \newline
37. स॒मभ॑रः॒ पयाꣳ॑सि॒ पयाꣳ॑सि स॒मभ॑रः स॒मभ॑रः॒ पयाꣳ॑सि । \newline
38. स॒मभ॑र॒ इति॑ सं - अभ॑रः । \newline
39. पयाꣳ॑ स्युत्त॒मे नो᳚त्त॒मेन॒ पयाꣳ॑सि॒ पयाꣳ॑ स्युत्त॒मेन॑ । \newline
40. उ॒त्त॒मेन॑ ह॒विषा॑ ह॒विषो᳚ त्त॒मेनो᳚त्त॒मेन॑ ह॒विषा᳚ । \newline
41. उ॒त्त॒मेनेत्यु॑त् - त॒मेन॑ । \newline
42. ह॒विषा॑ जातवेदो जातवेदो ह॒विषा॑ ह॒विषा॑ जातवेदः । \newline
43. जा॒त॒वे॒द॒ इति॑ जात - वे॒दः॒ । \newline
44. तेना᳚ग्ने ऽग्ने॒ तेन॒ तेना᳚ग्ने । \newline
45. अ॒ग्ने॒ त्वम् त्व म॑ग्ने ऽग्ने॒ त्वम् । \newline
46. त्व मु॒तोत त्वम् त्व मु॒त । \newline
47. उ॒त व॑र्द्धय वर्द्धयो॒तोत व॑र्द्धय । \newline
48. व॒र्द्ध॒ये॒म मि॒मं ॅव॑र्द्धय वर्द्धये॒मम् । \newline
49. इ॒मꣳ स॑जा॒तानाꣳ॑ सजा॒ताना॑ मि॒म मि॒मꣳ स॑जा॒ताना᳚म् । \newline
50. स॒जा॒तानाꣳ॒॒ श्रैष्ठ्ये॒ श्रैष्ठ्ये॑ सजा॒तानाꣳ॑ सजा॒तानाꣳ॒॒ श्रैष्ठ्ये᳚ । \newline
51. स॒जा॒ताना॒मिति॑ स - जा॒ताना᳚म् । \newline
52. श्रैष्ठ्य॒ आ श्रैष्ठ्ये॒ श्रैष्ठ्य॒ आ । \newline
53. आ धे॑हि धे॒ह्या धे॑हि । \newline
54. धे॒ह्ये॒न॒ मे॒न॒म् धे॒हि॒ धे॒ह्ये॒न॒म् । \newline
55. ए॒न॒मित्ये॑नम् । \newline
56. य॒ज्ञ्॒हनो॒ वै वै य॑ज्ञ्॒हनो॑ यज्ञ्॒हनो॒ वै । \newline
57. य॒ज्ञ्॒हन॒ इति॑ यज्ञ् - हनः॑ । \newline
58. वै दे॒वा दे॒वा वै वै दे॒वाः । \newline
59. दे॒वा य॑ज्ञ्॒मुषो॑ यज्ञ्॒मुषो॑ दे॒वा दे॒वा य॑ज्ञ्॒मुषः॑ । \newline
60. य॒ज्ञ्॒मुषः॑ सन्ति सन्ति यज्ञ्॒मुषो॑ यज्ञ्॒मुषः॑ सन्ति । \newline
61. य॒ज्ञ्॒मुष॒ इति॑ यज्ञ् - मुषः॑ । \newline

\textbf{Ghana Paata } \newline

1. दे॒व॒यानी॑ रन्त॒रा ऽन्त॒रा दे॑व॒यानी᳚र् देव॒यानी॑ रन्त॒रा द्यावा॑पृथि॒वी द्यावा॑पृथि॒वी अ॑न्त॒रा दे॑व॒यानी᳚र् देव॒यानी॑ रन्त॒रा द्यावा॑पृथि॒वी । \newline
2. दे॒व॒यानी॒रिति॑ देव - यानीः᳚ । \newline
3. अ॒न्त॒रा द्यावा॑पृथि॒वी द्यावा॑पृथि॒वी अ॑न्त॒रा ऽन्त॒रा द्यावा॑पृथि॒वी वि॒यन्ति॑ वि॒यन्ति॒ द्यावा॑पृथि॒वी अ॑न्त॒रा ऽन्त॒रा द्यावा॑पृथि॒वी वि॒यन्ति॑ । \newline
4. द्यावा॑पृथि॒वी वि॒यन्ति॑ वि॒यन्ति॒ द्यावा॑पृथि॒वी द्यावा॑पृथि॒वी वि॒यन्ति॑ । \newline
5. द्यावा॑पृथि॒वी इति॒ द्यावा᳚ - पृ॒थि॒वी । \newline
6. वि॒यन्तीति॑ वि - यन्ति॑ । \newline
7. गृ॒हैश्च॑ च गृ॒हैर् गृ॒हैश्च॒ सर्वैः॒ सर्वै᳚श्च गृ॒हैर् गृ॒हैश्च॒ सर्वैः᳚ । \newline
8. च॒ सर्वैः॒ सर्वै᳚श्च च॒ सर्वैः᳚ प्र॒जया᳚ प्र॒जया॒ सर्वै᳚श्च च॒ सर्वैः᳚ प्र॒जया᳚ । \newline
9. सर्वैः᳚ प्र॒जया᳚ प्र॒जया॒ सर्वैः॒ सर्वैः᳚ प्र॒जया॒ नु नु प्र॒जया॒ सर्वैः॒ सर्वैः᳚ प्र॒जया॒ नु । \newline
10. प्र॒जया॒ नु नु प्र॒जया᳚ प्र॒जया॒ न्वग्रे ऽग्रे॒ नु प्र॒जया᳚ प्र॒जया॒ न्वग्रे᳚ । \newline
11. प्र॒जयेति॑ प्र - जया᳚ । \newline
12. न्वग्रे ऽग्रे॒ नु न्वग्रे॒ सुवः॒ सुव॒ रग्रे॒ नु न्वग्रे॒ सुवः॑ । \newline
13. अग्रे॒ सुवः॒ सुव॒ रग्रे ऽग्रे॒ सुवो॒ रुहा॑णा॒ रुहा॑णाः॒ सुव॒ रग्रे ऽग्रे॒ सुवो॒ रुहा॑णाः । \newline
14. सुवो॒ रुहा॑णा॒ रुहा॑णाः॒ सुवः॒ सुवो॒ रुहा॑णा स्तरत तरत॒ रुहा॑णाः॒ सुवः॒ सुवो॒ रुहा॑णा स्तरत । \newline
15. रुहा॑णा स्तरत तरत॒ रुहा॑णा॒ रुहा॑णा स्तरता॒ रजाꣳ॑सि॒ रजाꣳ॑सि तरत॒ रुहा॑णा॒ रुहा॑णा स्तरता॒ रजाꣳ॑सि । \newline
16. त॒र॒ता॒ रजाꣳ॑सि॒ रजाꣳ॑सि तरत तरता॒ रजाꣳ॑सि । \newline
17. रजाꣳ॒॒सीति॒ रजाꣳ॑सि । \newline
18. ये दे॒वा दे॒वा ये ये दे॒वा य॑ज्ञ्॒हनो॑ यज्ञ्॒हनो॑ दे॒वा ये ये दे॒वा य॑ज्ञ्॒हनः॑ । \newline
19. दे॒वा य॑ज्ञ्॒हनो॑ यज्ञ्॒हनो॑ दे॒वा दे॒वा य॑ज्ञ्॒हनो॑ यज्ञ्॒मुषो॑ यज्ञ्॒मुषो॑ यज्ञ्॒हनो॑ दे॒वा दे॒वा य॑ज्ञ्॒हनो॑ यज्ञ्॒मुषः॑ । \newline
20. य॒ज्ञ्॒हनो॑ यज्ञ्॒मुषो॑ यज्ञ्॒मुषो॑ यज्ञ्॒हनो॑ यज्ञ्॒हनो॑ यज्ञ्॒मुषो॑ दि॒वि दि॒वि य॑ज्ञ्॒मुषो॑ यज्ञ्॒हनो॑ यज्ञ्॒हनो॑ यज्ञ्॒मुषो॑ दि॒वि । \newline
21. य॒ज्ञ्॒हन॒ इति॑ यज्ञ् - हनः॑ । \newline
22. य॒ज्ञ्॒मुषो॑ दि॒वि दि॒वि य॑ज्ञ्॒मुषो॑ यज्ञ्॒मुषो॑ दि॒व्यध्यधि॑ दि॒वि य॑ज्ञ्॒मुषो॑ यज्ञ्॒मुषो॑ दि॒व्यधि॑ । \newline
23. य॒ज्ञ्॒मुष॒ इति॑ यज्ञ् - मुषः॑ । \newline
24. दि॒व्यध्यधि॑ दि॒वि दि॒व्यध्यास॑त॒ आस॒ते ऽधि॑ दि॒वि दि॒व्यध्यास॑ते । \newline
25. अध्यास॑त॒ आस॒ते ऽध्य ध्यास॑ते । \newline
26. आस॑त॒ इत्यास॑ते । \newline
27. सूर्यो॑ मा मा॒ सूर्यः॒ सूर्यो॑ मा॒ तेभ्य॒ स्तेभ्यो॑ मा॒ सूर्यः॒ सूर्यो॑ मा॒ तेभ्यः॑ । \newline
28. मा॒ तेभ्य॒ स्तेभ्यो॑ मा मा॒ तेभ्यो॑ रक्षतु रक्षतु॒ तेभ्यो॑ मा मा॒ तेभ्यो॑ रक्षतु । \newline
29. तेभ्यो॑ रक्षतु रक्षतु॒ तेभ्य॒ स्तेभ्यो॑ रक्षतु॒ गच्छे॑म॒ गच्छे॑म रक्षतु॒ तेभ्य॒ स्तेभ्यो॑ रक्षतु॒ गच्छे॑म । \newline
30. र॒क्ष॒तु॒ गच्छे॑म॒ गच्छे॑म रक्षतु रक्षतु॒ गच्छे॑म सु॒कृतः॑ सु॒कृतो॒ गच्छे॑म रक्षतु रक्षतु॒ गच्छे॑म सु॒कृतः॑ । \newline
31. गच्छे॑म सु॒कृतः॑ सु॒कृतो॒ गच्छे॑म॒ गच्छे॑म सु॒कृतो॑ व॒यं ॅव॒यꣳ सु॒कृतो॒ गच्छे॑म॒ गच्छे॑म सु॒कृतो॑ व॒यम् । \newline
32. सु॒कृतो॑ व॒यं ॅव॒यꣳ सु॒कृतः॑ सु॒कृतो॑ व॒यम् । \newline
33. सु॒कृत॒ इति॑ सु - कृतः॑ । \newline
34. व॒यमिति॑ व॒यम् । \newline
35. येनेन्द्रा॒ येन्द्रा॑य॒ येन॒ येनेन्द्रा॑य स॒मभ॑रः स॒मभ॑र॒ इन्द्रा॑य॒ येन॒ येनेन्द्रा॑य स॒मभ॑रः । \newline
36. इन्द्रा॑य स॒मभ॑रः स॒मभ॑र॒ इन्द्रा॒ येन्द्रा॑य स॒मभ॑रः॒ पयाꣳ॑सि॒ पयाꣳ॑सि स॒मभ॑र॒ 
इन्द्रा॒ येन्द्रा॑य स॒मभ॑रः॒ पयाꣳ॑सि । \newline
37. स॒मभ॑रः॒ पयाꣳ॑सि॒ पयाꣳ॑सि स॒मभ॑रः स॒मभ॑रः॒ पयाꣳ॑ स्युत्त॒मेनो᳚ त्त॒मेन॒ पयाꣳ॑सि स॒मभ॑रः स॒मभ॑रः॒ पयाꣳ॑ स्युत्त॒मेन॑ । \newline
38. स॒मभ॑र॒ इति॑ सं - अभ॑रः । \newline
39. पयाꣳ॑ स्युत्त॒मे नो᳚त्त॒मेन॒ पयाꣳ॑सि॒ पयाꣳ॑ स्युत्त॒मेन॑ ह॒विषा॑ ह॒वि षो᳚त्त॒मेन॒ पयाꣳ॑सि॒ पयाꣳ॑ स्युत्त॒मेन॑ ह॒विषा᳚ । \newline
40. उ॒त्त॒मेन॑ ह॒विषा॑ ह॒वि षो᳚त्त॒मे नो᳚त्त॒मेन॑ ह॒विषा॑ जातवेदो जातवेदो ह॒वि षो᳚त्त॒मे नो᳚त्त॒मेन॑ ह॒विषा॑ जातवेदः । \newline
41. उ॒त्त॒मेनेत्यु॑त् - त॒मेन॑ । \newline
42. ह॒विषा॑ जातवेदो जातवेदो ह॒विषा॑ ह॒विषा॑ जातवेदः । \newline
43. जा॒त॒वे॒द॒ इति॑ जात - वे॒दः॒ । \newline
44. तेना᳚ग्ने ऽग्ने॒ तेन॒ तेना᳚ग्ने॒ त्वम् त्व म॑ग्ने॒ तेन॒ तेना᳚ग्ने॒ त्वम् । \newline
45. अ॒ग्ने॒ त्वम् त्व म॑ग्ने ऽग्ने॒ त्व मु॒तोत त्व म॑ग्ने ऽग्ने॒ त्व मु॒त । \newline
46. त्व मु॒तोत त्वम् त्व मु॒त व॑र्द्धय वर्द्धयो॒त त्वम् त्व मु॒त व॑र्द्धय । \newline
47. उ॒त व॑र्द्धय वर्द्धयो॒तोत व॑र्द्धये॒म मि॒मं ॅव॑र्द्धयो॒तोत व॑र्द्धये॒मम् । \newline
48. व॒र्द्ध॒ये॒म मि॒मं ॅव॑र्द्धय वर्द्धये॒मꣳ स॑जा॒तानाꣳ॑ सजा॒ताना॑ मि॒मं ॅव॑र्द्धय 
वर्द्धये॒मꣳ स॑जा॒ताना᳚म् । \newline
49. इ॒मꣳ स॑जा॒तानाꣳ॑ सजा॒ताना॑ मि॒म मि॒मꣳ स॑जा॒तानाꣳ॒॒ श्रैष्ठ्‌ये॒ श्रैष्ठ्‌ये॑ सजा॒ताना॑ मि॒म मि॒मꣳ स॑जा॒तानाꣳ॒॒ श्रैष्ठ्‌ये᳚ । \newline
50. स॒जा॒तानाꣳ॒॒ श्रैष्ठ्‌ये॒ श्रैष्ठ्‌ये॑ सजा॒तानाꣳ॑ सजा॒तानाꣳ॒॒ श्रैष्ठ्‌य॒ आ श्रैष्ठ्‌ये॑ सजा॒तानाꣳ॑ सजा॒तानाꣳ॒॒ श्रैष्ठ्‌य॒ आ । \newline
51. स॒जा॒ताना॒मिति॑ स - जा॒ताना᳚म् । \newline
52. श्रैष्ठ्‌य॒ आ श्रैष्ठ्‌ये॒ श्रैष्ठ्‌य॒ आ धे॑हि धे॒ह्या श्रैष्ठ्‌ये॒ श्रैष्ठ्‌य॒ आ धे॑हि । \newline
53. आ धे॑हि धे॒ह्या धे᳚ह्येन मेनम् धे॒ह्या धे᳚ह्येनम् । \newline
54. धे॒ह्ये॒न॒ मे॒न॒म् धे॒हि॒ धे॒ह्ये॒न॒म् । \newline
55. ए॒न॒मित्ये॑नम् । \newline
56. य॒ज्ञ्॒हनो॒ वै वै य॑ज्ञ्॒हनो॑ यज्ञ्॒हनो॒ वै दे॒वा दे॒वा वै य॑ज्ञ्॒हनो॑ यज्ञ्॒हनो॒ वै दे॒वाः । \newline
57. य॒ज्ञ्॒हन॒ इति॑ यज्ञ् - हनः॑ । \newline
58. वै दे॒वा दे॒वा वै वै दे॒वा य॑ज्ञ्॒मुषो॑ यज्ञ्॒मुषो॑ दे॒वा वै वै दे॒वा य॑ज्ञ्॒मुषः॑ । \newline
59. दे॒वा य॑ज्ञ्॒मुषो॑ यज्ञ्॒मुषो॑ दे॒वा दे॒वा य॑ज्ञ्॒मुषः॑ सन्ति सन्ति यज्ञ्॒मुषो॑ दे॒वा दे॒वा य॑ज्ञ्॒मुषः॑ सन्ति । \newline
60. य॒ज्ञ्॒मुषः॑ सन्ति सन्ति यज्ञ्॒मुषो॑ यज्ञ्॒मुषः॑ सन्ति॒ ते ते स॑न्ति यज्ञ्॒मुषो॑ यज्ञ्॒मुषः॑ सन्ति॒ ते । \newline
61. य॒ज्ञ्॒मुष॒ इति॑ यज्ञ् - मुषः॑ । \newline
\pagebreak
\markright{ TS 3.5.4.3  \hfill https://www.vedavms.in \hfill}

\section{ TS 3.5.4.3 }

\textbf{TS 3.5.4.3 } \newline
\textbf{Samhita Paata} \newline

सन्ति॒ त ए॒षु लो॒केष्वा॑सत आ॒ददा॑ना विमथ्ना॒ना यो ददा॑ति॒ यो यज॑ते॒ तस्य॑ । ये दे॒वा य॑ज्ञ्॒हनः॑ पृथि॒व्यामद्ध्यास॑ते॒ ये अ॒न्तरि॑क्षे॒ ये दि॒वीत्या॑हे॒माने॒व लो॒काꣳस्ती॒र्त्वा सगृ॑हः॒ सप॑शुः सुव॒र्गं ॅलो॒कमे॒त्यप॒ वै सोमे॑नेजा॒नाद्दे॒वता᳚श्च य॒ज्ञ्श्च॑ क्रामन्त्याग्ने॒यं पञ्च॑कपालमुदवसा॒नीयं॒ निर्व॑पेद॒ग्निः सर्वा॑ दे॒वताः॒ - [  ] \newline

\textbf{Pada Paata} \newline

स॒न्ति॒ । ते । ए॒षु । लो॒केषु॑ । आ॒स॒ते॒ । आ॒ददा॑ना॒ इत्या᳚ - ददा॑नाः । वि॒म॒थ्ना॒ना इति॑ वि-म॒थ्ना॒नाः । यः । ददा॑ति । यः । यज॑ते । तस्य॑ ॥ ये । दे॒वाः । य॒ज्ञ्॒हन॒ इति॑ यज्ञ् - हनः॑ । पृ॒थि॒व्याम् । अधीति॑ । आस॑ते । ये । अ॒न्तरि॑क्षे । ये । दि॒वि । इति॑ । आ॒ह॒ । इ॒मान् । ए॒व । लो॒कान् । ती॒र्त्वा । सगृ॑ह॒ इति॒ स - गृ॒हः॒ । सप॑शु॒रिति॒ स - प॒शुः॒ । सु॒व॒र्गमिति॑ सुवः - गम् । लो॒कम् । ए॒ति॒ । अपेति॑ । वै । सोमे॑न । ई॒जा॒नात् । दे॒वताः᳚ । च॒ । य॒ज्ञ्ः । च॒ । क्रा॒म॒न्ति॒ । आ॒ग्ने॒यम् । पञ्च॑कपाल॒मिति॒ पञ्च॑ - क॒पा॒ल॒म् । उ॒द॒व॒सा॒नीय॒मित्यु॑त् - अ॒व॒सा॒नीय᳚म् । निरिति॑ । व॒पे॒त् । अ॒ग्निः । सर्वाः᳚ । दे॒वताः᳚ ।  \newline


\textbf{Krama Paata} \newline

स॒न्ति॒ ते । त ए॒षु । ए॒षु लो॒केषु॑ । लो॒केष्वा॑सते । आ॒स॒त॒ आ॒ददा॑नाः । आ॒ददा॑ना विमथ्ना॒नाः । आ॒ददा॑ना॒ इत्या᳚ - ददा॑नाः । वि॒म॒थ्ना॒ना यः । वि॒म॒थ्ना॒ना इति॑ वि - म॒थ्ना॒नाः । यो ददा॑ति । ददा॑ति॒ यः । यो यज॑ते । यज॑ते॒ तस्य॑ । तस्येति॒ तस्य॑ ॥ ये दे॒वाः । दे॒वा य॑ज्ञ्॒हनः॑ । य॒ज्ञ्॒हनः॑ पृथि॒व्याम् । य॒ज्ञ्॒हन॒ इति॑ यज्ञ् - हनः॑ । पृ॒थि॒व्यामधि॑ । अध्यास॑ते । आस॑ते॒ ये । ये अ॒न्तरि॑क्षे । अ॒न्तरि॑क्षे॒ ये । ये दि॒वि । दि॒वीति॑ । इत्या॑ह । आ॒हे॒मान् । इ॒माने॒व । ए॒व लो॒कान् । लो॒काꣳ स्ती॒र्त्वा । ती॒र्त्वा सगृ॑हः । सगृ॑हः॒ सप॑शुः । सगृ॑ह॒ इति॒ स - गृ॒हः॒ । सप॑शुः सुव॒र्गम् । सप॑शु॒रिति॒ स - प॒शुः॒ । सु॒व॒र्गं ॅलो॒कम् । सु॒व॒र्गमिति॑ सुवः - गम् । लो॒कमे॑ति । ए॒त्यप॑ । अप॒ वै । वै सोमे॑न । सोमे॑नेजा॒नात् । ई॒जा॒नाद् दे॒वताः᳚ । दे॒वता᳚श्च । च॒ य॒ज्ञ्ः । य॒ज्ञ्श्च॑ । च॒ क्रा॒म॒न्ति॒ । क्रा॒म॒न्त्या॒ग्ने॒यम् । आ॒ग्ने॒यम् पञ्च॑कपालम् । पञ्च॑कपालमुदवसा॒नीय᳚म् । पञ्च॑कपाल॒मिति॒ पञ्च॑ - क॒पा॒ल॒म् । उ॒द॒व॒सा॒नीय॒म् निः । उ॒द॒व॒सा॒नीय॒मित्यु॑त् - अ॒व॒सा॒नीय᳚म् । निर् व॑पेत् । व॒पे॒द॒ग्निः । अ॒ग्निः सर्वाः᳚ । सर्वा॑ दे॒वताः᳚ ( ) । दे॒वताः॒ पाङ्क्तः॑ \newline

\textbf{Jatai Paata} \newline

1. स॒न्ति॒ ते ते स॑न्ति सन्ति॒ ते । \newline
2. त ए॒ष्वे॑षु ते त ए॒षु । \newline
3. ए॒षु लो॒केषु॑ लो॒के ष्वे॒ष्वे॑षु लो॒केषु॑ । \newline
4. लो॒के ष्वा॑सत आसते लो॒केषु॑ लो॒के ष्वा॑सते । \newline
5. आ॒स॒त॒ आ॒ददा॑ना आ॒ददा॑ना आसत आसत आ॒ददा॑नाः । \newline
6. आ॒ददा॑ना विमथ्ना॒ना वि॑मथ्ना॒ना आ॒ददा॑ना आ॒ददा॑ना विमथ्ना॒नाः । \newline
7. आ॒ददा॑ना॒ इत्या᳚ - ददा॑नाः । \newline
8. वि॒म॒थ्ना॒ना यो यो वि॑मथ्ना॒ना वि॑मथ्ना॒ना यः । \newline
9. वि॒म॒थ्ना॒ना इति॑ वि - म॒थ्ना॒नाः । \newline
10. यो ददा॑ति॒ ददा॑ति॒ यो यो ददा॑ति । \newline
11. ददा॑ति॒ यो यो ददा॑ति॒ ददा॑ति॒ यः । \newline
12. यो यज॑ते॒ यज॑ते॒ यो यो यज॑ते । \newline
13. यज॑ते॒ तस्य॒ तस्य॒ यज॑ते॒ यज॑ते॒ तस्य॑ । \newline
14. तस्येति॒ तस्य॑ । \newline
15. ये दे॒वा दे॒वा ये ये दे॒वाः । \newline
16. दे॒वा य॑ज्ञ्॒हनो॑ यज्ञ्॒हनो॑ दे॒वा दे॒वा य॑ज्ञ्॒हनः॑ । \newline
17. य॒ज्ञ्॒हनः॑ पृथि॒व्याम् पृ॑थि॒व्यां ॅय॑ज्ञ्॒हनो॑ यज्ञ्॒हनः॑ पृथि॒व्याम् । \newline
18. य॒ज्ञ्॒हन॒ इति॑ यज्ञ् - हनः॑ । \newline
19. पृ॒थि॒व्या मध्यधि॑ पृथि॒व्याम् पृ॑थि॒व्या मधि॑ । \newline
20. अध्यास॑त॒ आस॒ते ऽध्यध्यास॑ते । \newline
21. आस॑ते॒ ये य आस॑त॒ आस॑ते॒ ये । \newline
22. ये अ॒न्तरि॑क्षे॒ ऽन्तरि॑क्षे॒ ये ये अ॒न्तरि॑क्षे । \newline
23. अ॒न्तरि॑क्षे॒ ये ये᳚ ऽन्तरि॑क्षे॒ ऽन्तरि॑क्षे॒ ये । \newline
24. ये दि॒वि दि॒वि ये ये दि॒वि । \newline
25. दि॒वीतीति॑ दि॒वि दि॒वीति॑ । \newline
26. इत्या॑ हा॒हे तीत्या॑ह । \newline
27. आ॒हे॒मा नि॒मा ना॑हाहे॒मान् । \newline
28. इ॒मा ने॒वैवेमा नि॒मा ने॒व । \newline
29. ए॒व लो॒कान् ॅलो॒का ने॒वैव लो॒कान् । \newline
30. लो॒काꣳ स्ती॒र्त्वा ती॒र्त्वा लो॒कान् ॅलो॒काꣳ स्ती॒र्त्वा । \newline
31. ती॒र्त्वा सगृ॑हः॒ सगृ॑ह स्ती॒र्त्वा ती॒र्त्वा सगृ॑हः । \newline
32. सगृ॑हः॒ सप॑शुः॒ सप॑शुः॒ सगृ॑हः॒ सगृ॑हः॒ सप॑शुः । \newline
33. सगृ॑ह॒ इति॒ स - गृ॒हः॒ । \newline
34. सप॑शुः सुव॒र्गꣳ सु॑व॒र्गꣳ सप॑शुः॒ सप॑शुः सुव॒र्गम् । \newline
35. सप॑शु॒रिति॒ स - प॒शुः॒ । \newline
36. सु॒व॒र्गम् ॅलो॒कम् ॅलो॒कꣳ सु॑व॒र्गꣳ सु॑व॒र्गम् ॅलो॒कम् । \newline
37. सु॒व॒र्गमिति॑ सुवः - गम् । \newline
38. लो॒क मे᳚त्येति लो॒कम् ॅलो॒क मे॑ति । \newline
39. ए॒त्यपा पै᳚त्ये॒त्यप॑ । \newline
40. अप॒ वै वा अपाप॒ वै । \newline
41. वै सोमे॑न॒ सोमे॑न॒ वै वै सोमे॑न । \newline
42. सोमे॑ नेजा॒ना दी॑जा॒नाथ् सोमे॑न॒ सोमे॑ नेजा॒नात् । \newline
43. ई॒जा॒नाद् दे॒वता॑ दे॒वता॑ ईजा॒ना दी॑जा॒नाद् दे॒वताः᳚ । \newline
44. दे॒वता᳚श्च च दे॒वता॑ दे॒वता᳚श्च । \newline
45. च॒ य॒ज्ञो य॒ज्ञ्श्च॑ च य॒ज्ञ्ः । \newline
46. य॒ज्ञ्श्च॑ च य॒ज्ञो य॒ज्ञ्श्च॑ । \newline
47. च॒ क्रा॒म॒न्ति॒ क्रा॒म॒न्ति॒ च॒ च॒ क्रा॒म॒न्ति॒ । \newline
48. क्रा॒म॒ न्त्या॒ग्ने॒य मा᳚ग्ने॒यम् क्रा॑मन्ति क्राम न्त्याग्ने॒यम् । \newline
49. आ॒ग्ने॒यम् पञ्च॑कपाल॒म् पञ्च॑कपाल माग्ने॒य मा᳚ग्ने॒यम् पञ्च॑कपालम् । \newline
50. पञ्च॑कपाल मुदवसा॒नीय॑ मुदवसा॒नीय॒म् पञ्च॑कपाल॒म् पञ्च॑कपाल मुदवसा॒नीय᳚म् । \newline
51. पञ्च॑कपाल॒मिति॒ पञ्च॑ - क॒पा॒ल॒म् । \newline
52. उ॒द॒व॒सा॒नीय॒म् निर् णि रु॑दवसा॒नीय॑ मुदवसा॒नीय॒म् निः । \newline
53. उ॒द॒व॒सा॒नीय॒मित्यु॑त् - अ॒व॒सा॒नीय᳚म् । \newline
54. निर् व॑पेद् वपे॒न् निर् णिर् व॑पेत् । \newline
55. व॒पे॒ द॒ग्नि र॒ग्निर् व॑पेद् वपे द॒ग्निः । \newline
56. अ॒ग्निः सर्वाः॒ सर्वा॑ अ॒ग्नि र॒ग्निः सर्वाः᳚ । \newline
57. सर्वा॑ दे॒वता॑ दे॒वताः॒ सर्वाः॒ सर्वा॑ दे॒वताः᳚ । \newline
58. दे॒वताः॒ पाङ्क्तः॒ पाङ्क्तो॑ दे॒वता॑ दे॒वताः॒ पाङ्क्तः॑ । \newline

\textbf{Ghana Paata } \newline

1. स॒न्ति॒ ते ते स॑न्ति सन्ति॒ त ए॒ष्वे॑षु ते स॑न्ति सन्ति॒ त ए॒षु । \newline
2. त ए॒ष्वे॑षु ते त ए॒षु लो॒केषु॑ लो॒के ष्वे॒षु ते त ए॒षु लो॒केषु॑ । \newline
3. ए॒षु लो॒केषु॑ लो॒के ष्वे॒ष्वे॑षु लो॒के ष्वा॑सत आसते लो॒के ष्वे॒ष्वे॑षु लो॒के ष्वा॑सते । \newline
4. लो॒के ष्वा॑सत आसते लो॒केषु॑ लो॒के ष्वा॑सत आ॒ददा॑ना आ॒ददा॑ना आसते लो॒केषु॑ लो॒के ष्वा॑सत आ॒ददा॑नाः । \newline
5. आ॒स॒त॒ आ॒ददा॑ना आ॒ददा॑ना आसत आसत आ॒ददा॑ना विमथ्ना॒ना वि॑मथ्ना॒ना आ॒ददा॑ना आसत आसत आ॒ददा॑ना विमथ्ना॒नाः । \newline
6. आ॒ददा॑ना विमथ्ना॒ना वि॑मथ्ना॒ना आ॒ददा॑ना आ॒ददा॑ना विमथ्ना॒ना यो यो वि॑मथ्ना॒ना आ॒ददा॑ना आ॒ददा॑ना विमथ्ना॒ना यः । \newline
7. आ॒ददा॑ना॒ इत्या᳚ - ददा॑नाः । \newline
8. वि॒म॒थ्ना॒ना यो यो वि॑मथ्ना॒ना वि॑मथ्ना॒ना यो ददा॑ति॒ ददा॑ति॒ यो वि॑मथ्ना॒ना वि॑मथ्ना॒ना यो ददा॑ति । \newline
9. वि॒म॒थ्ना॒ना इति॑ वि - म॒थ्ना॒नाः । \newline
10. यो ददा॑ति॒ ददा॑ति॒ यो यो ददा॑ति॒ यो यो ददा॑ति॒ यो यो ददा॑ति॒ यः । \newline
11. ददा॑ति॒ यो यो ददा॑ति॒ ददा॑ति॒ यो यज॑ते॒ यज॑ते॒ यो ददा॑ति॒ ददा॑ति॒ यो यज॑ते । \newline
12. यो यज॑ते॒ यज॑ते॒ यो यो यज॑ते॒ तस्य॒ तस्य॒ यज॑ते॒ यो यो यज॑ते॒ तस्य॑ । \newline
13. यज॑ते॒ तस्य॒ तस्य॒ यज॑ते॒ यज॑ते॒ तस्य॑ । \newline
14. तस्येति॒ तस्य॑ । \newline
15. ये दे॒वा दे॒वा ये ये दे॒वा य॑ज्ञ्॒हनो॑ यज्ञ्॒हनो॑ दे॒वा ये ये दे॒वा य॑ज्ञ्॒हनः॑ । \newline
16. दे॒वा य॑ज्ञ्॒हनो॑ यज्ञ्॒हनो॑ दे॒वा दे॒वा य॑ज्ञ्॒हनः॑ पृथि॒व्याम् पृ॑थि॒व्यां ॅय॑ज्ञ्॒हनो॑ दे॒वा दे॒वा य॑ज्ञ्॒हनः॑ पृथि॒व्याम् । \newline
17. य॒ज्ञ्॒हनः॑ पृथि॒व्याम् पृ॑थि॒व्यां ॅय॑ज्ञ्॒हनो॑ यज्ञ्॒हनः॑ पृथि॒व्या मध्यधि॑ पृथि॒व्यां ॅय॑ज्ञ्॒हनो॑ यज्ञ्॒हनः॑ पृथि॒व्या मधि॑ । \newline
18. य॒ज्ञ्॒हन॒ इति॑ यज्ञ् - हनः॑ । \newline
19. पृ॒थि॒व्या मध्यधि॑ पृथि॒व्याम् पृ॑थि॒व्या मध्यास॑त॒ आस॒ते ऽधि॑ पृथि॒व्याम् पृ॑थि॒व्या मध्यास॑ते । \newline
20. अध्यास॑त॒ आस॒ते ऽध्यध्यास॑ते॒ ये य आस॒ते ऽध्यध्यास॑ते॒ ये । \newline
21. आस॑ते॒ ये य आस॑त॒ आस॑ते॒ ये अ॒न्तरि॑क्षे॒ ऽन्तरि॑क्षे॒ य आस॑त॒ आस॑ते॒ ये अ॒न्तरि॑क्षे । \newline
22. ये अ॒न्तरि॑क्षे॒ ऽन्तरि॑क्षे॒ ये ये अ॒न्तरि॑क्षे॒ ये ये᳚ ऽन्तरि॑क्षे॒ ये ये अ॒न्तरि॑क्षे॒ ये । \newline
23. अ॒न्तरि॑क्षे॒ ये ये᳚ ऽन्तरि॑क्षे॒ ऽन्तरि॑क्षे॒ ये दि॒वि दि॒वि ये᳚ ऽन्तरि॑क्षे॒ ऽन्तरि॑क्षे॒ ये दि॒वि । \newline
24. ये दि॒वि दि॒वि ये ये दि॒वीतीति॑ दि॒वि ये ये दि॒वीति॑ । \newline
25. दि॒वीतीति॑ दि॒वि दि॒वी त्या॑हा॒हेति॑ दि॒वि दि॒वी त्या॑ह । \newline
26. इत्या॑हा॒हेती त्या॑हे॒ मा नि॒मा ना॒हेती त्या॑हे॒ मान् । \newline
27. आ॒हे॒ मा नि॒मा ना॑हाहे॒मा ने॒वैवे मा ना॑हाहे॒मा ने॒व । \newline
28. इ॒मा ने॒वैवेमा नि॒मा ने॒व लो॒कान् ॅलो॒का ने॒वेमा नि॒मा ने॒व लो॒कान् । \newline
29. ए॒व लो॒कान् ॅलो॒का ने॒वैव लो॒काꣳ स्ती॒र्त्वा ती॒र्त्वा लो॒का ने॒वैव लो॒काꣳ स्ती॒र्त्वा । \newline
30. लो॒काꣳ स्ती॒र्त्वा ती॒र्त्वा लो॒कान् ॅलो॒काꣳ स्ती॒र्त्वा सगृ॑हः॒ सगृ॑ह स्ती॒र्त्वा लो॒कान् ॅलो॒काꣳ स्ती॒र्त्वा सगृ॑हः । \newline
31. ती॒र्त्वा सगृ॑हः॒ सगृ॑ह स्ती॒र्त्वा ती॒र्त्वा सगृ॑हः॒ सप॑शुः॒ सप॑शुः॒ सगृ॑ह स्ती॒र्त्वा ती॒र्त्वा सगृ॑हः॒ सप॑शुः । \newline
32. सगृ॑हः॒ सप॑शुः॒ सप॑शुः॒ सगृ॑हः॒ सगृ॑हः॒ सप॑शुः सुव॒र्गꣳ सु॑व॒र्गꣳ सप॑शुः॒ सगृ॑हः॒ सगृ॑हः॒ सप॑शुः सुव॒र्गम् । \newline
33. सगृ॑ह॒ इति॒ स - गृ॒हः॒ । \newline
34. सप॑शुः सुव॒र्गꣳ सु॑व॒र्गꣳ सप॑शुः॒ सप॑शुः सुव॒र्गम् ॅलो॒कम् ॅलो॒कꣳ सु॑व॒र्गꣳ सप॑शुः॒ सप॑शुः सुव॒र्गम् ॅलो॒कम् । \newline
35. सप॑शु॒रिति॒ स - प॒शुः॒ । \newline
36. सु॒व॒र्गम् ॅलो॒कम् ॅलो॒कꣳ सु॑व॒र्गꣳ सु॑व॒र्गम् ॅलो॒क मे᳚त्येति लो॒कꣳ सु॑व॒र्गꣳ सु॑व॒र्गम् ॅलो॒क मे॑ति । \newline
37. सु॒व॒र्गमिति॑ सुवः - गम् । \newline
38. लो॒क मे᳚त्येति लो॒कम् ॅलो॒क मे॒त्यपा पै॑ति लो॒कम् ॅलो॒क मे॒त्यप॑ । \newline
39. ए॒त्य पापै᳚ त्ये॒त्यप॒ वै वा अपै᳚ त्ये॒त्यप॒ वै । \newline
40. अप॒ वै वा अपाप॒ वै सोमे॑न॒ सोमे॑न॒ वा अपाप॒ वै सोमे॑न । \newline
41. वै सोमे॑न॒ सोमे॑न॒ वै वै सोमे॑ने जा॒ना दी॑जा॒नाथ् सोमे॑न॒ वै वै सोमे॑ने जा॒नात् । \newline
42. सोमे॑ने जा॒ना दी॑जा॒नाथ् सोमे॑न॒ सोमे॑ नेजा॒नाद् दे॒वता॑ दे॒वता॑ ईजा॒नाथ् सोमे॑न॒ सोमे॑ने जा॒नाद् दे॒वताः᳚ । \newline
43. ई॒जा॒नाद् दे॒वता॑ दे॒वता॑ ईजा॒ना दी॑जा॒नाद् दे॒वता᳚श्च च दे॒वता॑ ईजा॒ना दी॑जा॒नाद् दे॒वता᳚श्च । \newline
44. दे॒वता᳚श्च च दे॒वता॑ दे॒वता᳚श्च य॒ज्ञो य॒ज्ञ्श्च॑ दे॒वता॑ दे॒वता᳚श्च य॒ज्ञ्ः । \newline
45. च॒ य॒ज्ञो य॒ज्ञ्श्च॑ च य॒ज्ञ्श्च॑ च य॒ज्ञ्श्च॑ च य॒ज्ञ्श्च॑ । \newline
46. य॒ज्ञ्श्च॑ च य॒ज्ञो य॒ज्ञ्श्च॑ क्रामन्ति क्रामन्ति च य॒ज्ञो य॒ज्ञ्श्च॑ क्रामन्ति । \newline
47. च॒ क्रा॒म॒न्ति॒ क्रा॒म॒न्ति॒ च॒ च॒ क्रा॒म॒ न्त्या॒ग्ने॒य मा᳚ग्ने॒यम् क्रा॑मन्ति च च क्राम न्त्याग्ने॒यम् । \newline
48. क्रा॒म॒ न्त्या॒ग्ने॒य मा᳚ग्ने॒यम् क्रा॑मन्ति क्राम न्त्याग्ने॒यम् पञ्च॑कपाल॒म् पञ्च॑कपाल माग्ने॒यम् क्रा॑मन्ति क्राम न्त्याग्ने॒यम् पञ्च॑कपालम् । \newline
49. आ॒ग्ने॒यम् पञ्च॑कपाल॒म् पञ्च॑कपाल माग्ने॒य मा᳚ग्ने॒यम् पञ्च॑कपाल मुदवसा॒नीय॑ मुदवसा॒नीय॒म् पञ्च॑कपाल माग्ने॒य मा᳚ग्ने॒यम् पञ्च॑कपाल मुदवसा॒नीय᳚म् । \newline
50. पञ्च॑कपाल मुदवसा॒नीय॑ मुदवसा॒नीय॒म् पञ्च॑कपाल॒म् पञ्च॑कपाल मुदवसा॒नीय॒न् निर् णिरु॑दवसा॒नीय॒म् पञ्च॑कपाल॒म् पञ्च॑कपाल मुदवसा॒नीय॒न् निः । \newline
51. पञ्च॑कपाल॒मिति॒ पञ्च॑ - क॒पा॒ल॒म् । \newline
52. उ॒द॒व॒सा॒नीय॒न् निर् णिरु॑दवसा॒नीय॑ मुदवसा॒नीय॒न् निर् व॑पेद् वपे॒न् निरु॑दवसा॒नीय॑ मुदवसा॒नीय॒न् निर् व॑पेत् । \newline
53. उ॒द॒व॒सा॒नीय॒मित्यु॑त् - अ॒व॒सा॒नीय᳚म् । \newline
54. निर् व॑पेद् वपे॒न् निर् णिर् व॑पे द॒ग्नि र॒ग्निर् व॑पे॒न् निर् णिर् व॑पे द॒ग्निः । \newline
55. व॒पे॒ द॒ग्नि र॒ग्निर् व॑पेद् वपे द॒ग्निः सर्वाः॒ सर्वा॑ अ॒ग्निर् व॑पेद् वपे द॒ग्निः सर्वाः᳚ । \newline
56. अ॒ग्निः सर्वाः॒ सर्वा॑ अ॒ग्नि र॒ग्निः सर्वा॑ दे॒वता॑ दे॒वताः॒ सर्वा॑ अ॒ग्नि र॒ग्निः सर्वा॑ दे॒वताः᳚ । \newline
57. सर्वा॑ दे॒वता॑ दे॒वताः॒ सर्वाः॒ सर्वा॑ दे॒वताः॒ पाङ्क्तः॒ पाङ्क्तो॑ दे॒वताः॒ सर्वाः॒ सर्वा॑ दे॒वताः॒ पाङ्क्तः॑ । \newline
58. दे॒वताः॒ पाङ्क्तः॒ पाङ्क्तो॑ दे॒वता॑ दे॒वताः॒ पाङ्क्तो॑ य॒ज्ञो य॒ज्ञ्ः पाङ्क्तो॑ दे॒वता॑ दे॒वताः॒ पाङ्क्तो॑ य॒ज्ञ्ः । \newline
\pagebreak
\markright{ TS 3.5.4.4  \hfill https://www.vedavms.in \hfill}

\section{ TS 3.5.4.4 }

\textbf{TS 3.5.4.4 } \newline
\textbf{Samhita Paata} \newline

पाङ्क्तो॑ य॒ज्ञो दे॒वता᳚श्चै॒व य॒ज्ञ्ञ्चाव॑ रुन्धेगाय॒त्रो वा अ॒ग्निर्गा॑य॒त्र छ॑न्दा॒स्तं छन्द॑सा॒ व्य॑र्द्धयति॒ यत् पञ्च॑कपालं क॒रोत्य॒ष्टाक॑पालः का॒र्यो᳚ऽष्टाक्ष॑रा गाय॒त्री गा॑य॒त्रो᳚ऽग्नि-र्गा॑य॒त्र छ॑न्दाः॒ स्वेनै॒वैनं॒ छन्द॑सा॒ सम॑र्द्धयति प॒ङ्क्त्यौ॑ याज्यानुवा॒क्ये॑ भवतः॒ पाङ्क्तो॑ य॒ज्ञ्स्तेनै॒व य॒ज्ञान्नैति॑ ॥ \newline

\textbf{Pada Paata} \newline

पाङ्क्तः॑ । य॒ज्ञ्ः । दे॒वताः᳚ । च॒ । ए॒व । य॒ज्ञ्म् । च॒ । अवेति॑ । रु॒न्धे॒ । गा॒य॒त्रः । वै । अ॒ग्निः । गा॒य॒त्रछ॑न्दा॒ इति॑ गाय॒त्र - छ॒न्दाः॒ । तम् । छन्द॑सा । वीति॑ । अ॒र्द्ध॒य॒ति॒ । यत् । पञ्च॑कपाल॒मिति॒ पञ्च॑ - क॒पा॒ल॒म् । क॒रोति॑ । अ॒ष्टाक॑पाल॒ इत्य॒ष्टा - क॒पा॒लः॒ । का॒र्यः॑ । अ॒ष्टाक्ष॒रेत्य॒ष्टा - अ॒क्ष॒रा॒ । गा॒य॒त्री । गा॒य॒त्रः । अ॒ग्निः । गा॒य॒त्रछ॑न्दा॒ इति॑ गाय॒त्र - छ॒न्दाः॒ । स्वेन॑ । ए॒व । ए॒न॒म् । छन्द॑सा । समिति॑ । अ॒र्द्ध॒य॒ति॒ । प॒ङ्क्त्यौ᳚ । या॒ज्या॒नु॒वा॒क्ये॑ इति॑ याज्या - अ॒नु॒वा॒क्ये᳚ । भ॒व॒तः॒ । पाङ्क्तः॑ । य॒ज्ञ्ः । तेन॑ । ए॒व । य॒ज्ञात् । न । ए॒ति॒ ॥  \newline


\textbf{Krama Paata} \newline

पाङ्क्तो॑ य॒ज्ञ्ः । य॒ज्ञो दे॒वताः᳚ । दे॒वता᳚श्च । चै॒व । ए॒व य॒ज्ञ्म् । य॒ज्ञ्म् च॑ । चाव॑ । अव॑ रुन्धे । रु॒न्धे॒ गा॒य॒त्रः । गा॒य॒त्रो वै । वा अ॒ग्निः । अ॒ग्निर् गा॑य॒त्रछ॑न्दाः । गा॒य॒त्रछ॑न्दा॒स्तम् । गा॒य॒त्रछ॑न्दा॒ इति॑ गाय॒त्र - छ॒न्दाः॒ । तम् छन्द॑सा । छन्द॑सा॒ वि । व्य॑र्द्धयति । अ॒र्द्ध॒य॒ति॒ यत् । यत् पञ्च॑कपालम् । पञ्च॑कपालम् क॒रोति॑ । पञ्च॑कपाल॒मिति॒ पञ्च॑ - क॒पा॒ल॒म् । क॒रोत्य॒ष्टाक॑पालः । अ॒ष्टाक॑पालः का॒र्यः॑ । अ॒ष्टाक॑पाल॒ इत्य॒ष्टा - क॒पा॒लः॒ । का॒र्यो᳚ ऽष्टाक्ष॑रा । अ॒ष्टाक्ष॑रा गाय॒त्री । अ॒ष्टाक्ष॒रेत्य॒ष्टा - अ॒क्ष॒रा॒ । गा॒य॒त्री गा॑य॒त्रः । गा॒य॒त्रो᳚ ऽग्निः । अ॒ग्निर् गा॑य॒त्रछ॑न्दाः । गा॒य॒त्रछ॑न्दाः॒ स्वेन॑ । गा॒य॒त्रछ॑न्दा॒ इति॑ गाय॒त्र - छ॒न्दाः॒ । स्वेनै॒व । ए॒वैन᳚म् । ए॒न॒म् छन्द॑सा । छन्द॑सा॒ सम् । सम॑र्द्धयति । अ॒र्द्ध॒य॒ति॒ प॒ङ्क्त्यौ᳚ । प॒ङ्क्त्यौ॑ याज्यानुवा॒क्ये᳚ । या॒ज्या॒नु॒वा॒क्ये॑ भवतः । या॒ज्या॒नु॒वा॒क्ये॑ इति॑ याज्या - अ॒नु॒वाक्ये᳚ । भ॒व॒तः॒ पाङ्क्तः॑ । पाङ्क्तो॑ य॒ज्ञ्ः । य॒ज्ञ्स्तेन॑ । तेनै॒व । ए॒व य॒ज्ञात् । य॒ज्ञान् न । नैति॑ । ए॒तीत्ये॑ति । \newline

\textbf{Jatai Paata} \newline

1. पाङ्क्तो॑ य॒ज्ञो य॒ज्ञ्ः पाङ्क्तः॒ पाङ्क्तो॑ य॒ज्ञ्ः । \newline
2. य॒ज्ञो दे॒वता॑ दे॒वता॑ य॒ज्ञो य॒ज्ञो दे॒वताः᳚ । \newline
3. दे॒वता᳚श्च च दे॒वता॑ दे॒वता᳚श्च । \newline
4. चै॒वैव च॑ चै॒व । \newline
5. ए॒व य॒ज्ञ्ं ॅय॒ज्ञ् मे॒वैव य॒ज्ञ्म् । \newline
6. य॒ज्ञ्म् च॑ च य॒ज्ञ्ं ॅय॒ज्ञ्म् च॑ । \newline
7. चावाव॑ च॒ चाव॑ । \newline
8. अव॑ रुन्धे रु॒न्धे ऽवाव॑ रुन्धे । \newline
9. रु॒न्धे॒ गा॒य॒त्रो गा॑य॒त्रो रु॑न्धे रुन्धे गाय॒त्रः । \newline
10. गा॒य॒त्रो वै वै गा॑य॒त्रो गा॑य॒त्रो वै । \newline
11. वा अ॒ग्नि र॒ग्निर् वै वा अ॒ग्निः । \newline
12. अ॒ग्निर् गा॑य॒त्रछ॑न्दा गाय॒त्रछ॑न्दा अ॒ग्नि र॒ग्निर् गा॑य॒त्रछ॑न्दाः । \newline
13. गा॒य॒त्रछ॑न्दा॒ स्तम् तम् गा॑य॒त्रछ॑न्दा गाय॒त्रछ॑न्दा॒ स्तम् । \newline
14. गा॒य॒त्रछ॑न्दा॒ इति॑ गाय॒त्र - छ॒न्दाः॒ । \newline
15. तम् छन्द॑सा॒ छन्द॑सा॒ तम् तम् छन्द॑सा । \newline
16. छन्द॑सा॒ वि विच् छन्द॑सा॒ छन्द॑सा॒ वि । \newline
17. व्य॑र्द्धय त्यर्द्धयति॒ वि व्य॑र्द्धयति । \newline
18. अ॒र्द्ध॒य॒ति॒ यद् यद॑र्द्धय त्यर्द्धयति॒ यत् । \newline
19. यत् पञ्च॑कपाल॒म् पञ्च॑कपालं॒ ॅयद् यत् पञ्च॑कपालम् । \newline
20. पञ्च॑कपालम् क॒रोति॑ क॒रोति॒ पञ्च॑कपाल॒म् पञ्च॑कपालम् क॒रोति॑ । \newline
21. पञ्च॑कपाल॒मिति॒ पञ्च॑ - क॒पा॒ल॒म् । \newline
22. क॒रो त्य॒ष्टाक॑पालो॒ ऽष्टाक॑पालः क॒रोति॑ क॒रो त्य॒ष्टाक॑पालः । \newline
23. अ॒ष्टाक॑पालः का॒र्यः॑ का॒र्यो᳚ ऽष्टाक॑पालो॒ ऽष्टाक॑पालः का॒र्यः॑ । \newline
24. अ॒ष्टाक॑पाल॒ इत्य॒ष्टा - क॒पा॒लः॒ । \newline
25. का॒र्यो᳚ ऽष्टाक्ष॑रा॒ ऽष्टाक्ष॑रा का॒र्यः॑ का॒र्यो᳚ ऽष्टाक्ष॑रा । \newline
26. अ॒ष्टाक्ष॑रा गाय॒त्री गा॑य॒ त्र्य॑ष्टाक्ष॑रा॒ ऽष्टाक्ष॑रा गाय॒त्री । \newline
27. अ॒ष्टाक्ष॒रेत्य॒ष्टा - अ॒क्ष॒रा॒ । \newline
28. गा॒य॒त्री गा॑य॒त्रो गा॑य॒त्रो गा॑य॒त्री गा॑य॒त्री गा॑य॒त्रः । \newline
29. गा॒य॒त्रो᳚ ऽग्नि र॒ग्निर् गा॑य॒त्रो गा॑य॒त्रो᳚ ऽग्निः । \newline
30. अ॒ग्निर् गा॑य॒त्रछ॑न्दा गाय॒त्रछ॑न्दा अ॒ग्नि र॒ग्निर् गा॑य॒त्रछ॑न्दाः । \newline
31. गा॒य॒त्रछ॑न्दाः॒ स्वेन॒ स्वेन॑ गाय॒त्रछ॑न्दा गाय॒त्रछ॑न्दाः॒ स्वेन॑ । \newline
32. गा॒य॒त्रछ॑न्दा॒ इति॑ गाय॒त्र - छ॒न्दाः॒ । \newline
33. स्वे नै॒वैव स्वेन॒ स्वेनै॒व । \newline
34. ए॒वैन॑ मेन मे॒वैवैन᳚म् । \newline
35. ए॒न॒म् छन्द॑सा॒ छन्द॑सैन मेन॒म् छन्द॑सा । \newline
36. छन्द॑सा॒ सꣳ सम् छन्द॑सा॒ छन्द॑सा॒ सम् । \newline
37. स म॑र्द्धय त्यर्द्धयति॒ सꣳ स म॑र्द्धयति । \newline
38. अ॒र्द्ध॒य॒ति॒ प॒ङ्क्त्यौ॑ प॒ङ्क्त्या॑ वर्द्धय त्यर्द्धयति प॒ङ्क्त्यौ᳚ । \newline
39. प॒ङ्क्त्यौ॑ याज्यानुवा॒क्ये॑ याज्यानुवा॒क्ये॑ प॒ङ्क्त्यौ॑ प॒ङ्क्त्यौ॑ याज्यानुवा॒क्ये᳚ । \newline
40. या॒ज्या॒नु॒वा॒क्ये॑ भवतो भवतो याज्यानुवा॒क्ये॑ याज्यानुवा॒क्ये॑ भवतः । \newline
41. या॒ज्या॒नु॒वा॒क्ये॑ इति॑ याज्या - अ॒नु॒वा॒क्ये᳚ । \newline
42. भ॒व॒तः॒ पाङ्क्तः॒ पाङ्क्तो॑ भवतो भवतः॒ पाङ्क्तः॑ । \newline
43. पाङ्क्तो॑ य॒ज्ञो य॒ज्ञ्ः पाङ्क्तः॒ पाङ्क्तो॑ य॒ज्ञ्ः । \newline
44. य॒ज्ञ् स्तेन॒ तेन॑ य॒ज्ञो य॒ज्ञ् स्तेन॑ । \newline
45. तेनै॒वैव तेन॒ तेनै॒व । \newline
46. ए॒व य॒ज्ञाद् य॒ज्ञा दे॒वैव य॒ज्ञात् । \newline
47. य॒ज्ञान् न न य॒ज्ञाद् य॒ज्ञान् न । \newline
48. नैत्ये॑ति॒ न नैति॑ । \newline
49. ए॒तीत्ये॑ति । \newline

\textbf{Ghana Paata } \newline

1. पाङ्क्तो॑ य॒ज्ञो य॒ज्ञ्ः पाङ्क्तः॒ पाङ्क्तो॑ य॒ज्ञो दे॒वता॑ दे॒वता॑ य॒ज्ञ्ः पाङ्क्तः॒ पाङ्क्तो॑ य॒ज्ञो दे॒वताः᳚ । \newline
2. य॒ज्ञो दे॒वता॑ दे॒वता॑ य॒ज्ञो य॒ज्ञो दे॒वता᳚श्च च दे॒वता॑ य॒ज्ञो य॒ज्ञो दे॒वता᳚श्च । \newline
3. दे॒वता᳚श्च च दे॒वता॑ दे॒वता᳚ श्चै॒वैव च॑ दे॒वता॑ दे॒वता᳚ श्चै॒व । \newline
4. चै॒वैव च॑ चै॒व य॒ज्ञ्ं ॅय॒ज्ञ् मे॒व च॑ चै॒व य॒ज्ञ्म् । \newline
5. ए॒व य॒ज्ञ्ं ॅय॒ज्ञ् मे॒वैव य॒ज्ञ्म् च॑ च य॒ज्ञ् मे॒वैव य॒ज्ञ्म् च॑ । \newline
6. य॒ज्ञ्म् च॑ च य॒ज्ञ्ं ॅय॒ज्ञ्म् चावाव॑ च य॒ज्ञ्ं ॅय॒ज्ञ्म् चाव॑ । \newline
7. चावाव॑ च॒ चाव॑ रुन्धे रु॒न्धे ऽव॑ च॒ चाव॑ रुन्धे । \newline
8. अव॑ रुन्धे रु॒न्धे ऽवाव॑ रुन्धे गाय॒त्रो गा॑य॒त्रो रु॒न्धे ऽवाव॑ रुन्धे गाय॒त्रः । \newline
9. रु॒न्धे॒ गा॒य॒त्रो गा॑य॒त्रो रु॑न्धे रुन्धे गाय॒त्रो वै वै गा॑य॒त्रो रु॑न्धे रुन्धे गाय॒त्रो वै । \newline
10. गा॒य॒त्रो वै वै गा॑य॒त्रो गा॑य॒त्रो वा अ॒ग्नि र॒ग्निर् वै गा॑य॒त्रो गा॑य॒त्रो वा अ॒ग्निः । \newline
11. वा अ॒ग्नि र॒ग्निर् वै वा अ॒ग्निर् गा॑य॒त्रछ॑न्दा गाय॒त्रछ॑न्दा अ॒ग्निर् वै वा अ॒ग्निर् गा॑य॒त्रछ॑न्दाः । \newline
12. अ॒ग्निर् गा॑य॒त्रछ॑न्दा गाय॒त्रछ॑न्दा अ॒ग्नि र॒ग्निर् गा॑य॒त्रछ॑न्दा॒ स्तम् तम् गा॑य॒त्रछ॑न्दा अ॒ग्नि र॒ग्निर् गा॑य॒त्रछ॑न्दा॒ स्तम् । \newline
13. गा॒य॒त्रछ॑न्दा॒ स्तम् तम् गा॑य॒त्रछ॑न्दा गाय॒त्रछ॑न्दा॒ स्तम् छन्द॑सा॒ छन्द॑सा॒ तम् गा॑य॒त्रछ॑न्दा गाय॒त्रछ॑न्दा॒ स्तम् छन्द॑सा । \newline
14. गा॒य॒त्रछ॑न्दा॒ इति॑ गाय॒त्र - छ॒न्दाः॒ । \newline
15. तम् छन्द॑सा॒ छन्द॑सा॒ तम् तम् छन्द॑सा॒ वि विच् छन्द॑सा॒ तम् तम् छन्द॑सा॒ वि । \newline
16. छन्द॑सा॒ वि विच् छन्द॑सा॒ छन्द॑सा॒ व्य॑र्द्धय त्यर्द्धयति॒ विच् छन्द॑सा॒ छन्द॑सा॒ व्य॑र्द्धयति । \newline
17. व्य॑र्द्धय त्यर्द्धयति॒ वि व्य॑र्द्धयति॒ यद् यद॑र्द्धयति॒ वि व्य॑र्द्धयति॒ यत् । \newline
18. अ॒र्द्ध॒य॒ति॒ यद् यद॑र्द्धय त्यर्द्धयति॒ यत् पञ्च॑कपाल॒म् पञ्च॑कपालं॒ ॅयद॑र्द्धय त्यर्द्धयति॒ यत् पञ्च॑कपालम् । \newline
19. यत् पञ्च॑कपाल॒म् पञ्च॑कपालं॒ ॅयद् यत् पञ्च॑कपालम् क॒रोति॑ क॒रोति॒ पञ्च॑कपालं॒ ॅयद् यत् पञ्च॑कपालम् क॒रोति॑ । \newline
20. पञ्च॑कपालम् क॒रोति॑ क॒रोति॒ पञ्च॑कपाल॒म् पञ्च॑कपालम् क॒रो त्य॒ष्टाक॑पालो॒ ऽष्टाक॑पालः क॒रोति॒ पञ्च॑कपाल॒म् पञ्च॑कपालम् क॒रो त्य॒ष्टाक॑पालः । \newline
21. पञ्च॑कपाल॒मिति॒ पञ्च॑ - क॒पा॒ल॒म् । \newline
22. क॒रो त्य॒ष्टाक॑पालो॒ ऽष्टाक॑पालः क॒रोति॑ क॒रो त्य॒ष्टाक॑पालः का॒र्यः॑ का॒र्यो᳚ ऽष्टाक॑पालः क॒रोति॑ क॒रो त्य॒ष्टाक॑पालः का॒र्यः॑ । \newline
23. अ॒ष्टाक॑पालः का॒र्यः॑ का॒र्यो᳚ ऽष्टाक॑पालो॒ ऽष्टाक॑पालः का॒र्यो᳚ ऽष्टाक्ष॑रा॒ ऽष्टाक्ष॑रा का॒र्यो᳚ ऽष्टाक॑पालो॒ ऽष्टाक॑पालः का॒र्यो᳚ ऽष्टाक्ष॑रा । \newline
24. अ॒ष्टाक॑पाल॒ इत्य॒ष्टा - क॒पा॒लः॒ । \newline
25. का॒र्यो᳚ ऽष्टाक्ष॑रा॒ ऽष्टाक्ष॑रा का॒र्यः॑ का॒र्यो᳚ ऽष्टाक्ष॑रा गाय॒त्री गा॑य॒ त्र्य॑ष्टाक्ष॑रा का॒र्यः॑ का॒र्यो᳚ ऽष्टाक्ष॑रा गाय॒त्री । \newline
26. अ॒ष्टाक्ष॑रा गाय॒त्री गा॑य॒ त्र्य॑ष्टाक्ष॑रा॒ ऽष्टाक्ष॑रा गाय॒त्री गा॑य॒त्रो गा॑य॒त्रो गा॑य॒ त्र्य॑ष्टाक्ष॑रा॒ ऽष्टाक्ष॑रा गाय॒त्री गा॑य॒त्रः । \newline
27. अ॒ष्टाक्ष॒रेत्य॒ष्टा - अ॒क्ष॒रा॒ । \newline
28. गा॒य॒त्री गा॑य॒त्रो गा॑य॒त्रो गा॑य॒त्री गा॑य॒त्री गा॑य॒त्रो᳚ ऽग्नि र॒ग्निर् गा॑य॒त्रो गा॑य॒त्री गा॑य॒त्री गा॑य॒त्रो᳚ ऽग्निः । \newline
29. गा॒य॒त्रो᳚ ऽग्नि र॒ग्निर् गा॑य॒त्रो गा॑य॒त्रो᳚ ऽग्निर् गा॑य॒त्रछ॑न्दा गाय॒त्रछ॑न्दा अ॒ग्निर् गा॑य॒त्रो गा॑य॒त्रो᳚ ऽग्निर् गा॑य॒त्रछ॑न्दाः । \newline
30. अ॒ग्निर् गा॑य॒त्रछ॑न्दा गाय॒त्रछ॑न्दा अ॒ग्नि र॒ग्निर् गा॑य॒त्रछ॑न्दाः॒ स्वेन॒ स्वेन॑ गाय॒त्रछ॑न्दा अ॒ग्नि र॒ग्निर् गा॑य॒त्रछ॑न्दाः॒ स्वेन॑ । \newline
31. गा॒य॒त्रछ॑न्दाः॒ स्वेन॒ स्वेन॑ गाय॒त्रछ॑न्दा गाय॒त्रछ॑न्दाः॒ स्वेनै॒वैव स्वेन॑ गाय॒त्रछ॑न्दा गाय॒त्रछ॑न्दाः॒ स्वेनै॒व । \newline
32. गा॒य॒त्रछ॑न्दा॒ इति॑ गाय॒त्र - छ॒न्दाः॒ । \newline
33. स्वेनै॒वैव स्वेन॒ स्वेनै॒वैन॑ मेन मे॒व स्वेन॒ स्वेनै॒वैन᳚म् । \newline
34. ए॒वैन॑ मेन मे॒वैवैन॒म् छन्द॑सा॒ छन्द॑सैन मे॒वैवैन॒म् छन्द॑सा । \newline
35. ए॒न॒म् छन्द॑सा॒ छन्द॑सैन मेन॒म् छन्द॑सा॒ सꣳ सम् छन्द॑सैन मेन॒म् छन्द॑सा॒ सम् । \newline
36. छन्द॑सा॒ सꣳ सम् छन्द॑सा॒ छन्द॑सा॒ स म॑र्द्धय त्यर्द्धयति॒ सम् छन्द॑सा॒ छन्द॑सा॒ स म॑र्द्धयति । \newline
37. स म॑र्द्धय त्यर्द्धयति॒ सꣳ स म॑र्द्धयति प॒ङ्क्त्यौ॑ प॒ङ्क्त्या॑ वर्द्धयति॒ सꣳ स म॑र्द्धयति प॒ङ्क्त्यौ᳚ । \newline
38. अ॒र्द्ध॒य॒ति॒ प॒ङ्क्त्यौ॑ प॒ङ्क्त्या॑ वर्द्धय त्यर्द्धयति प॒ङ्क्त्यौ॑ याज्यानुवा॒क्ये॑ याज्यानुवा॒क्ये॑ प॒ङ्क्त्या॑ वर्द्धय त्यर्द्धयति प॒ङ्क्त्यौ॑ याज्यानुवा॒क्ये᳚ । \newline
39. प॒ङ्क्त्यौ॑ याज्यानुवा॒क्ये॑ याज्यानुवा॒क्ये॑ प॒ङ्क्त्यौ॑ प॒ङ्क्त्यौ॑ याज्यानुवा॒क्ये॑ भवतो भवतो याज्यानुवा॒क्ये॑ प॒ङ्क्त्यौ॑ प॒ङ्क्त्यौ॑ याज्यानुवा॒क्ये॑ भवतः । \newline
40. या॒ज्या॒नु॒वा॒क्ये॑ भवतो भवतो याज्यानुवा॒क्ये॑ याज्यानुवा॒क्ये॑ भवतः॒ पाङ्क्तः॒ पाङ्क्तो॑ भवतो याज्यानुवा॒क्ये॑ याज्यानुवा॒क्ये॑ भवतः॒ पाङ्क्तः॑ । \newline
41. या॒ज्या॒नु॒वा॒क्ये॑ इति॑ याज्या - अ॒नु॒वा॒क्ये᳚ । \newline
42. भ॒व॒तः॒ पाङ्क्तः॒ पाङ्क्तो॑ भवतो भवतः॒ पाङ्क्तो॑ य॒ज्ञो य॒ज्ञ्ः पाङ्क्तो॑ भवतो भवतः॒ पाङ्क्तो॑ य॒ज्ञ्ः । \newline
43. पाङ्क्तो॑ य॒ज्ञो य॒ज्ञ्ः पाङ्क्तः॒ पाङ्क्तो॑ य॒ज्ञ् स्तेन॒ तेन॑ य॒ज्ञ्ः पाङ्क्तः॒ पाङ्क्तो॑ य॒ज्ञ् स्तेन॑ । \newline
44. य॒ज्ञ् स्तेन॒ तेन॑ य॒ज्ञो य॒ज्ञ् स्तेनै॒वैव तेन॑ य॒ज्ञो य॒ज्ञ् स्तेनै॒व । \newline
45. तेनै॒वैव तेन॒ तेनै॒व य॒ज्ञाद् य॒ज्ञा दे॒व तेन॒ तेनै॒व य॒ज्ञात् । \newline
46. ए॒व य॒ज्ञाद् य॒ज्ञा दे॒वैव य॒ज्ञान् न न य॒ज्ञा दे॒वैव य॒ज्ञान् न । \newline
47. य॒ज्ञान् न न य॒ज्ञाद् य॒ज्ञान् नैत्ये॑ति॒ न य॒ज्ञाद् य॒ज्ञान् नैति॑ । \newline
48. नैत्ये॑ति॒ न नैति॑ । \newline
49. ए॒तीत्ये॑ति । \newline
\pagebreak
\markright{ TS 3.5.5.1  \hfill https://www.vedavms.in \hfill}

\section{ TS 3.5.5.1 }

\textbf{TS 3.5.5.1 } \newline
\textbf{Samhita Paata} \newline

सूर्यो॑ मा दे॒वो दे॒वेभ्यः॑ पातु वा॒युर॒न्तरि॑क्षा॒द्-यज॑मानो॒ऽग्निर्मा॑ पातु॒ चक्षु॑षः । सक्ष॒ शूष॒ सवि॑त॒र्विश्व॑चर्.षण ए॒तेभिः॑ सोम॒ नाम॑भिर्विधेम ते॒ तेभिः॑ सोम॒ नाम॑भिर्विधेम ते ॥ अ॒हं प॒रस्ता॑द॒-हम॒वस्ता॑द॒हं ज्योति॑षा॒ वि तमो॑ ववार । यद॒न्तरि॑क्षं॒ तदु॑ मे पि॒ताऽभू॑द॒हꣳ सूर्य॑मुभ॒यतो॑ ददर्.शा॒हं भू॑या समुत्त॒मः स॑मा॒नाना॒ - [  ] \newline

\textbf{Pada Paata} \newline

सूर्यः॑ । मा॒ । दे॒वः । दे॒वेभ्यः॑ । पा॒तु॒ । वा॒युः । अ॒न्तरि॑क्षात् । यज॑मानः । अ॒ग्निः । मा॒ । पा॒तु॒ । चक्षु॑षः ॥ सक्ष॑ । शूष॑ । सवि॑तः । विश्व॑चर्.षण॒ इति॒ विश्व॑ - च॒र्॒.ष॒णे॒ । ए॒तेभिः॑ । सो॒म॒ । नाम॑भि॒रिति॒ नाम॑ - भिः॒ । वि॒धे॒म॒ । ते॒ । तेभिः॑ । सो॒म॒ । नाम॑भि॒रिति॒ नाम॑-भिः॒ । वि॒धे॒म॒ । ते॒ ॥ अ॒हम् । प॒रस्ता᳚त् । अ॒हम् । अ॒वस्ता᳚त् । अ॒हम् । ज्योति॑षा । वीति॑ । तमः॑ । व॒वा॒र॒ ॥ यत् । अ॒न्तरि॑क्षम् । तत् । उ॒ । मे॒ । पि॒ता । अ॒भू॒त् । अ॒हम् । सूर्य᳚म् । उ॒भ॒यतः॑ । द॒द॒र्॒.श॒ । अ॒हम् । भू॒या॒स॒म् । उ॒त्त॒म इत्यु॑त् - त॒मः । स॒मा॒नाना᳚म् ।  \newline


\textbf{Krama Paata} \newline

सूर्यो॑ मा । मा॒ दे॒वः । दे॒वो दे॒वेभ्यः॑ । दे॒वेभ्यः॑ पातु । पा॒तु॒ वा॒युः । वा॒युर॒न्तरि॑क्षात् । अ॒न्तरि॑क्षा॒द् यज॑मानः । यज॑मानो॒ ऽग्निः । अ॒ग्निर् मा᳚ । मा॒ पा॒तु॒ । पा॒तु॒ चक्षु॑षः । चक्षु॑ष॒ इति॒ चक्षु॑षः ॥ सक्ष॒ शूष॑ । शूष॒ सवि॑तः । सवि॑त॒र् विश्व॑चर्.षणे । विश्व॑चर्.षण ए॒तेभिः॑ । विश्व॑चर्.षण॒ इति॒ विश्व॑ - च॒र्॒ष॒णे॒ । ए॒तेभिः॑ सोम । सो॒म॒ नाम॑भिः । नाम॑भिर् विधेम । नाम॑भि॒रिति॒ नाम॑ - भिः॒ । वि॒धे॒म॒ ते॒ । ते॒ तेभिः॑ । तेभिः॑ सोम । सो॒म॒ नाम॑भिः । नाम॑भिर् विधेम । नाम॑भि॒रिति॒ नाम॑ - भिः॒ । वि॒धे॒म॒ ते॒ । त॒ इति॑ ते ॥ अ॒हम् प॒रस्ता᳚त् । प॒रस्ता॑द॒हम् । अ॒हम॒वस्ता᳚त् । अ॒वस्ता॑द॒हम् । अ॒हम् ज्योति॑षा । ज्योति॑षा॒ वि । वि तमः॑ । तमो॑ ववार । व॒वा॒रेति॑ ववार ॥ यद॒न्तरि॑क्षम् । अ॒न्तरि॑क्ष॒म् तत् । तदु॑ । उ॒ मे॒ । मे॒ पि॒ता । पि॒ता ऽभू᳚त् । अ॒भू॒द॒हम् । अ॒हꣳ सूर्य᳚म् । सूर्य॑मुभ॒यतः॑ । उ॒भ॒यतो॑ ददर्.श । द॒द॒र्॒.शा॒हम् । अ॒हम् भू॑यासम् । भू॒या॒स॒मु॒त्त॒मः । उ॒त्त॒मः स॑मा॒नाना᳚म् । उ॒त्त॒म इत्यु॑त् - त॒मः । स॒मा॒नाना॒मा \newline

\textbf{Jatai Paata} \newline

1. सूर्यो॑ मा मा॒ सूर्यः॒ सूर्यो॑ मा । \newline
2. मा॒ दे॒वो दे॒वो मा॑ मा दे॒वः । \newline
3. दे॒वो दे॒वेभ्यो॑ दे॒वेभ्यो॑ दे॒वो दे॒वो दे॒वेभ्यः॑ । \newline
4. दे॒वेभ्यः॑ पातु पातु दे॒वेभ्यो॑ दे॒वेभ्यः॑ पातु । \newline
5. पा॒तु॒ वा॒युर् वा॒युः पा॑तु पातु वा॒युः । \newline
6. वा॒यु र॒न्तरि॑क्षा द॒न्तरि॑क्षाद् वा॒युर् वा॒यु र॒न्तरि॑क्षात् । \newline
7. अ॒न्तरि॑क्षा॒द् यज॑मानो॒ यज॑मानो॒ ऽन्तरि॑क्षा द॒न्तरि॑क्षा॒द् यज॑मानः । \newline
8. यज॑मानो॒ ऽग्नि र॒ग्निर् यज॑मानो॒ यज॑मानो॒ ऽग्निः । \newline
9. अ॒ग्निर् मा॑ मा॒ ऽग्नि र॒ग्निर् मा᳚ । \newline
10. मा॒ पा॒तु॒ पा॒तु॒ मा॒ मा॒ पा॒तु॒ । \newline
11. पा॒तु॒ चक्षु॑ष॒ श्चक्षु॑षः पातु पातु॒ चक्षु॑षः । \newline
12. चक्षु॑ष॒ इति॒ चक्षु॑षः । \newline
13. सक्ष॒ शूष॒ शूष॒ सक्ष॒ सक्ष॒ शूष॑ । \newline
14. शूष॒ सवि॑तः॒ सवि॑तः॒ शूष॒ शूष॒ सवि॑तः । \newline
15. सवि॑त॒र् विश्व॑चर्.षणे॒ विश्व॑चर्.षणे॒ सवि॑तः॒ सवि॑त॒र् विश्व॑चर्.षणे । \newline
16. विश्व॑चर्.षण ए॒तेभि॑ रे॒तेभि॒र् विश्व॑चर्.षणे॒ विश्व॑चर्.षण ए॒तेभिः॑ । \newline
17. विश्व॑चर्.षण॒ इति॒ विश्व॑ - च॒र्॒.ष॒णे॒ । \newline
18. ए॒तेभिः॑ सोम सोमै॒तेभि॑ रे॒तेभिः॑ सोम । \newline
19. सो॒म॒ नाम॑भि॒र् नाम॑भिः सोम सोम॒ नाम॑भिः । \newline
20. नाम॑भिर् विधेम विधेम॒ नाम॑भि॒र् नाम॑भिर् विधेम । \newline
21. नाम॑भि॒रिति॒ नाम॑ - भिः॒ । \newline
22. वि॒धे॒म॒ ते॒ ते॒ वि॒धे॒म॒ वि॒धे॒म॒ ते॒ । \newline
23. ते॒ तेभि॒ स्तेभि॑ स्ते ते॒ तेभिः॑ । \newline
24. तेभिः॑ सोम सोम॒ तेभि॒ स्तेभिः॑ सोम । \newline
25. सो॒म॒ नाम॑भि॒र् नाम॑भिः सोम सोम॒ नाम॑भिः । \newline
26. नाम॑भिर् विधेम विधेम॒ नाम॑भि॒र् नाम॑भिर् विधेम । \newline
27. नाम॑भि॒रिति॒ नाम॑ - भिः॒ । \newline
28. वि॒धे॒म॒ ते॒ ते॒ वि॒धे॒म॒ वि॒धे॒म॒ ते॒ । \newline
29. त॒ इति॑ ते । \newline
30. अ॒हम् प॒रस्ता᳚त् प॒रस्ता॑ द॒ह म॒हम् प॒रस्ता᳚त् । \newline
31. प॒रस्ता॑ द॒ह म॒हम् प॒रस्ता᳚त् प॒रस्ता॑ द॒हम् । \newline
32. अ॒ह म॒वस्ता॑ द॒वस्ता॑ द॒ह म॒ह म॒वस्ता᳚त् । \newline
33. अ॒वस्ता॑ द॒ह म॒ह म॒वस्ता॑ द॒वस्ता॑ द॒हम् । \newline
34. अ॒हम् ज्योति॑षा॒ ज्योति॑षा॒ ऽह म॒हम् ज्योति॑षा । \newline
35. ज्योति॑षा॒ वि वि ज्योति॑षा॒ ज्योति॑षा॒ वि । \newline
36. वि तम॒ स्तमो॒ वि वि तमः॑ । \newline
37. तमो॑ ववार ववार॒ तम॒ स्तमो॑ ववार । \newline
38. व॒वा॒रेति॑ ववार । \newline
39. यद॒न्तरि॑क्ष म॒न्तरि॑क्षं॒ ॅयद् यद॒न्तरि॑क्षम् । \newline
40. अ॒न्तरि॑क्ष॒म् तत् तद॒न्तरि॑क्ष म॒न्तरि॑क्ष॒म् तत् । \newline
41. तदू॒ तत् तदु॑ । \newline
42. उ॒ मे॒ म॒ उ॒ वु॒ मे॒ । \newline
43. मे॒ पि॒ता पि॒ता मे॑ मे पि॒ता । \newline
44. पि॒ता ऽभू॑दभूत् पि॒ता पि॒ता ऽभू᳚त् । \newline
45. अ॒भू॒ द॒ह म॒ह म॑भू दभू द॒हम् । \newline
46. अ॒हꣳ सूर्यꣳ॒॒ सूर्य॑ म॒ह म॒हꣳ सूर्य᳚म् । \newline
47. सूर्य॑ मुभ॒यत॑ उभ॒यतः॒ सूर्यꣳ॒॒ सूर्य॑ मुभ॒यतः॑ । \newline
48. उ॒भ॒यतो॑ ददर्.श ददर्.शो भ॒यत॑ उभ॒यतो॑ ददर्.श । \newline
49. द॒द॒र्॒.शा॒ह म॒हम् द॑दर्.श ददर्.शा॒हम् । \newline
50. अ॒हम् भू॑यासम् भूयास म॒ह म॒हम् भू॑यासम् । \newline
51. भू॒या॒स॒ मु॒त्त॒म उ॑त्त॒मो भू॑यासम् भूयास मुत्त॒मः । \newline
52. उ॒त्त॒मः स॑मा॒नानाꣳ॑ समा॒नाना॑ मुत्त॒म उ॑त्त॒मः स॑मा॒नाना᳚म् । \newline
53. उ॒त्त॒म इत्यु॑त् - त॒मः । \newline
54. स॒मा॒नाना॒ मा स॑मा॒नानाꣳ॑ समा॒नाना॒ मा । \newline

\textbf{Ghana Paata } \newline

1. सूर्यो॑ मा मा॒ सूर्यः॒ सूर्यो॑ मा दे॒वो दे॒वो मा॒ सूर्यः॒ सूर्यो॑ मा दे॒वः । \newline
2. मा॒ दे॒वो दे॒वो मा॑ मा दे॒वो दे॒वेभ्यो॑ दे॒वेभ्यो॑ दे॒वो मा॑ मा दे॒वो दे॒वेभ्यः॑ । \newline
3. दे॒वो दे॒वेभ्यो॑ दे॒वेभ्यो॑ दे॒वो दे॒वो दे॒वेभ्यः॑ पातु पातु दे॒वेभ्यो॑ दे॒वो दे॒वो दे॒वेभ्यः॑ पातु । \newline
4. दे॒वेभ्यः॑ पातु पातु दे॒वेभ्यो॑ दे॒वेभ्यः॑ पातु वा॒युर् वा॒युः पा॑तु दे॒वेभ्यो॑ दे॒वेभ्यः॑ पातु वा॒युः । \newline
5. पा॒तु॒ वा॒युर् वा॒युः पा॑तु पातु वा॒यु र॒न्तरि॑क्षा द॒न्तरि॑क्षाद् वा॒युः पा॑तु पातु वा॒यु र॒न्तरि॑क्षात् । \newline
6. वा॒यु र॒न्तरि॑क्षा द॒न्तरि॑क्षाद् वा॒युर् वा॒यु र॒न्तरि॑क्षा॒द् यज॑मानो॒ यज॑मानो॒ ऽन्तरि॑क्षाद् वा॒युर् वा॒यु र॒न्तरि॑क्षा॒द् यज॑मानः । \newline
7. अ॒न्तरि॑क्षा॒द् यज॑मानो॒ यज॑मानो॒ ऽन्तरि॑क्षा द॒न्तरि॑क्षा॒द् यज॑मानो॒ ऽग्नि र॒ग्निर् यज॑मानो॒ ऽन्तरि॑क्षा द॒न्तरि॑क्षा॒द् यज॑मानो॒ ऽग्निः । \newline
8. यज॑मानो॒ ऽग्नि र॒ग्निर् यज॑मानो॒ यज॑मानो॒ ऽग्निर् मा॑ मा॒ ऽग्निर् यज॑मानो॒ यज॑मानो॒ ऽग्निर् मा᳚ । \newline
9. अ॒ग्निर् मा॑ मा॒ ऽग्नि र॒ग्निर् मा॑ पातु पातु मा॒ ऽग्नि र॒ग्निर् मा॑ पातु । \newline
10. मा॒ पा॒तु॒ पा॒तु॒ मा॒ मा॒ पा॒तु॒ चक्षु॑ष॒ श्चक्षु॑षः पातु मा मा पातु॒ चक्षु॑षः । \newline
11. पा॒तु॒ चक्षु॑ष॒ श्चक्षु॑षः पातु पातु॒ चक्षु॑षः । \newline
12. चक्षु॑ष॒ इति॒ चक्षु॑षः । \newline
13. सक्ष॒ शूष॒ शूष॒ सक्ष॒ सक्ष॒ शूष॒ सवि॑तः॒ सवि॑तः॒ शूष॒ सक्ष॒ सक्ष॒ शूष॒ सवि॑तः । \newline
14. शूष॒ सवि॑तः॒ सवि॑तः॒ शूष॒ शूष॒ सवि॑त॒र् विश्व॑चर्.षणे॒ विश्व॑चर्.षणे॒ सवि॑तः॒ शूष॒ शूष॒ सवि॑त॒र् विश्व॑चर्.षणे । \newline
15. सवि॑त॒र् विश्व॑चर्.षणे॒ विश्व॑चर्.षणे॒ सवि॑तः॒ सवि॑त॒र् विश्व॑चर्.षण ए॒तेभि॑ रे॒तेभि॒र् विश्व॑चर्.षणे॒ सवि॑तः॒ सवि॑त॒र् विश्व॑चर्.षण ए॒तेभिः॑ । \newline
16. विश्व॑चर्.षण ए॒तेभि॑ रे॒तेभि॒र् विश्व॑चर्.षणे॒ विश्व॑चर्.षण ए॒तेभिः॑ सोम सोमै॒ तेभि॒र् विश्व॑चर्.षणे॒ विश्व॑चर्.षण ए॒तेभिः॑ सोम । \newline
17. विश्व॑चर्.षण॒ इति॒ विश्व॑ - च॒र्॒.ष॒णे॒ । \newline
18. ए॒तेभिः॑ सोम सोमै॒ तेभि॑ रे॒तेभिः॑ सोम॒ नाम॑भि॒र् नाम॑भिः सोमै॒ तेभि॑ रे॒तेभिः॑ सोम॒ नाम॑भिः । \newline
19. सो॒म॒ नाम॑भि॒र् नाम॑भिः सोम सोम॒ नाम॑भिर् विधेम विधेम॒ नाम॑भिः सोम सोम॒ नाम॑भिर् विधेम । \newline
20. नाम॑भिर् विधेम विधेम॒ नाम॑भि॒र् नाम॑भिर् विधेम ते ते विधेम॒ नाम॑भि॒र् नाम॑भिर् विधेम ते । \newline
21. नाम॑भि॒रिति॒ नाम॑ - भिः॒ । \newline
22. वि॒धे॒म॒ ते॒ ते॒ वि॒धे॒म॒ वि॒धे॒म॒ ते॒ तेभि॒ स्तेभि॑ स्ते विधेम विधेम ते॒ तेभिः॑ । \newline
23. ते॒ तेभि॒ स्तेभि॑ स्ते ते॒ तेभिः॑ सोम सोम॒ तेभि॑ स्ते ते॒ तेभिः॑ सोम । \newline
24. तेभिः॑ सोम सोम॒ तेभि॒ स्तेभिः॑ सोम॒ नाम॑भि॒र् नाम॑भिः सोम॒ तेभि॒ स्तेभिः॑ सोम॒ नाम॑भिः । \newline
25. सो॒म॒ नाम॑भि॒र् नाम॑भिः सोम सोम॒ नाम॑भिर् विधेम विधेम॒ नाम॑भिः सोम सोम॒ नाम॑भिर् विधेम । \newline
26. नाम॑भिर् विधेम विधेम॒ नाम॑भि॒र् नाम॑भिर् विधेम ते ते विधेम॒ नाम॑भि॒र् नाम॑भिर् विधेम ते । \newline
27. नाम॑भि॒रिति॒ नाम॑ - भिः॒ । \newline
28. वि॒धे॒म॒ ते॒ ते॒ वि॒धे॒म॒ वि॒धे॒म॒ ते॒ । \newline
29. त॒ इति॑ ते । \newline
30. अ॒हम् प॒रस्ता᳚त् प॒रस्ता॑ द॒ह म॒हम् प॒रस्ता॑ द॒ह म॒हम् प॒रस्ता॑ द॒ह म॒हम् प॒रस्ता॑ द॒हम् । \newline
31. प॒रस्ता॑ द॒ह म॒हम् प॒रस्ता᳚त् प॒रस्ता॑ द॒ह म॒वस्ता॑ द॒वस्ता॑ द॒हम् प॒रस्ता᳚त् प॒रस्ता॑ द॒ह म॒वस्ता᳚त् । \newline
32. अ॒ह म॒वस्ता॑ द॒वस्ता॑ द॒ह म॒ह म॒वस्ता॑ द॒ह म॒ह म॒वस्ता॑ द॒ह म॒ह म॒वस्ता॑ द॒हम् । \newline
33. अ॒वस्ता॑ द॒ह म॒ह म॒वस्ता॑ द॒वस्ता॑ द॒हम् ज्योति॑षा॒ ज्योति॑षा॒ ऽह म॒वस्ता॑ द॒वस्ता॑ द॒हम् ज्योति॑षा । \newline
34. अ॒हम् ज्योति॑षा॒ ज्योति॑षा॒ ऽह म॒हम् ज्योति॑षा॒ वि वि ज्योति॑षा॒ ऽह म॒हम् ज्योति॑षा॒ वि । \newline
35. ज्योति॑षा॒ वि वि ज्योति॑षा॒ ज्योति॑षा॒ वि तम॒ स्तमो॒ वि ज्योति॑षा॒ ज्योति॑षा॒ वि तमः॑ । \newline
36. वि तम॒ स्तमो॒ वि वि तमो॑ ववार ववार॒ तमो॒ वि वि तमो॑ ववार । \newline
37. तमो॑ ववार ववार॒ तम॒ स्तमो॑ ववार । \newline
38. व॒वा॒रेति॑ ववार । \newline
39. यद॒न्तरि॑क्ष म॒न्तरि॑क्षं॒ ॅयद् यद॒न्तरि॑क्ष॒म् तत् तद॒न्तरि॑क्षं॒ ॅयद् यद॒न्तरि॑क्ष॒म् तत् । \newline
40. अ॒न्तरि॑क्ष॒म् तत् तद॒न्तरि॑क्ष म॒न्तरि॑क्ष॒म् तदू॒ तद॒न्तरि॑क्ष म॒न्तरि॑क्ष॒म् तदु॑ । \newline
41. तदू॒ तत् तदु॑ मे म उ॒ तत् तदु॑ मे । \newline
42. उ॒ मे॒ म॒ उ॒ वु॒ मे॒ पि॒ता पि॒ता म॑ उ वु मे पि॒ता । \newline
43. मे॒ पि॒ता पि॒ता मे॑ मे पि॒ता ऽभू॑ दभूत् पि॒ता मे॑ मे पि॒ता ऽभू᳚त् । \newline
44. पि॒ता ऽभू॑ दभूत् पि॒ता पि॒ता ऽभू॑ द॒ह म॒ह म॑भूत् पि॒ता पि॒ता ऽभू॑ द॒हम् । \newline
45. अ॒भू॒ द॒ह म॒ह म॑भू दभू द॒हꣳ सूर्यꣳ॒॒ सूर्य॑ म॒ह म॑भू दभू द॒हꣳ सूर्य᳚म् । \newline
46. अ॒हꣳ सूर्यꣳ॒॒ सूर्य॑ म॒ह म॒हꣳ सूर्य॑ मुभ॒यत॑ उभ॒यतः॒ सूर्य॑ म॒ह म॒हꣳ सूर्य॑ मुभ॒यतः॑ । \newline
47. सूर्य॑ मुभ॒यत॑ उभ॒यतः॒ सूर्यꣳ॒॒ सूर्य॑ मुभ॒यतो॑ ददर्.श ददर्.शोभ॒यतः॒ सूर्यꣳ॒॒ सूर्य॑ मुभ॒यतो॑ ददर्.श । \newline
48. उ॒भ॒यतो॑ ददर्.श ददर्.शोभ॒यत॑ उभ॒यतो॑ ददर्.शा॒ह म॒हम् द॑दर्.शोभ॒यत॑ उभ॒यतो॑ ददर्.शा॒हम् । \newline
49. द॒द॒र्॒.शा॒ह म॒हम् द॑दर्.श ददर्.शा॒हम् भू॑यासम् भूयास म॒हम् द॑दर्.श ददर्.शा॒हम् भू॑यासम् । \newline
50. अ॒हम् भू॑यासम् भूयास म॒ह म॒हम् भू॑यास मुत्त॒म उ॑त्त॒मो भू॑यास म॒ह म॒हम् भू॑यास मुत्त॒मः । \newline
51. भू॒या॒स॒ मु॒त्त॒म उ॑त्त॒मो भू॑यासम् भूयास मुत्त॒मः स॑मा॒नानाꣳ॑ समा॒नाना॑ मुत्त॒मो भू॑यासम् भूयास मुत्त॒मः स॑मा॒नाना᳚म् । \newline
52. उ॒त्त॒मः स॑मा॒नानाꣳ॑ समा॒नाना॑ मुत्त॒म उ॑त्त॒मः स॑मा॒नाना॒ मा स॑मा॒नाना॑ मुत्त॒म उ॑त्त॒मः स॑मा॒नाना॒ मा । \newline
53. उ॒त्त॒म इत्यु॑त् - त॒मः । \newline
54. स॒मा॒नाना॒ मा स॑मा॒नानाꣳ॑ समा॒नाना॒ मा स॑मु॒द्राथ् स॑मु॒द्रादा स॑मा॒नानाꣳ॑ समा॒नाना॒ मा स॑मु॒द्रात् । \newline
\pagebreak
\markright{ TS 3.5.5.2  \hfill https://www.vedavms.in \hfill}

\section{ TS 3.5.5.2 }

\textbf{TS 3.5.5.2 } \newline
\textbf{Samhita Paata} \newline

मा स॑मु॒द्रा-दाऽन्तरि॑क्षात्-प्र॒जाप॑तिरुद॒धिं च्या॑वया॒तीन्द्रः॒ प्रस्नौ॑तु म॒रुतो॑ वर्.षय॒न्तून्न॑म्भय पृथि॒वीं भि॒न्धीदं दि॒व्यं नभः॑ । उ॒द्रो दि॒व्यस्य॑ नो दे॒हीशा॑नो॒ विसृ॑जा॒ दृतिं᳚ ॥ प॒शवो॒ वा ए॒ते यदा॑दि॒त्य ए॒ष रु॒द्रो यद॒ग्निरोष॑धीः॒ प्रास्या॒ग्नावा॑दि॒त्यं जु॑होति रु॒द्रादे॒व प॒शून॒न्तर्द॑धा॒त्यथो॒ ओष॑धीष्वे॒व प॒शून् - [  ] \newline

\textbf{Pada Paata} \newline

एति॑ । स॒मु॒द्रात् । एति॑ । अ॒न्तरि॑क्षात् । प्र॒जाप॑ति॒रिति॑ प्र॒जा - प॒तिः॒ । उ॒द॒धिमित्यु॑द - धिम् । च्या॒व॒या॒ति॒ । इन्द्रः॑ । प्रेति॑ । स्नौ॒तु॒ । म॒रुतः॑ । व॒र्॒.ष॒य॒न्तु॒ । उदिति॑ । न॒भं॒य॒ । पृ॒थि॒वीम् । भि॒न्धि । इ॒दम् । दि॒व्यम् । नभः॑ ॥ उ॒द्रः । दि॒व्यस्य॑ । नः॒ । दे॒हि॒ । ईशा॑नः । वीति॑ । सृ॒ज॒ । दृति᳚म् ॥ प॒शवः॑ । वै । ए॒ते । यत् । आ॒दि॒त्यः । ए॒षः । रु॒द्रः । यत् । अ॒ग्निः । ओष॑धीः । प्रास्येति॑ प्र - अस्य॑ । अ॒ग्नौ । आ॒दि॒त्यम् । जु॒हो॒ति॒ । रु॒द्रात् । ए॒व । प॒शून् । अ॒न्तः । द॒धा॒ति॒ । अथो॒ इति॑ । ओष॑धीषु । ए॒व । प॒शून् ।  \newline


\textbf{Krama Paata} \newline

आ स॑मु॒द्रात् । स॒मु॒द्रादा । आ ऽन्तरि॑क्षात् । अ॒न्तरि॑क्षात् प्र॒जाप॑तिः । प्र॒जाप॑तिरुद॒धिम् । प्र॒जाप॑ति॒रिति॑ प्र॒जा - प॒तिः॒ । उ॒द॒धिम् च्या॑वयाति । उ॒द॒धिमित्यु॑द - धिम् । च्या॒व॒या॒तीन्द्रः॑ । इन्द्रः॒ प्र । प्र स्नौ॑तु । स्नौ॒तु॒ म॒रुतः॑ । म॒रुतो॑ वर्.षयन्तु । व॒र्॒.य॒न्तूत् । उन् न॑म्भय । न॒म्भ॒य॒ पृ॒थि॒वीम् । पृ॒थि॒वीम् भि॒न्धि । भि॒न्धीदम् । इ॒दम् दि॒व्यम् । दि॒व्यम् नभः॑ । नभ॒ इति॒ नभः॑ ॥ उ॒द्नो दि॒व्यस्य॑ । दि॒व्यस्य॑ नः । नो॒ दे॒हि॒ । दे॒हीशा॑नः । ईशा॑नो॒ वि । वि सृ॑ज । सृ॒जा॒ दृति᳚म् । दृति॒मिति॒ दृति᳚म् ॥ प॒शवो॒ वै । वा ए॒ते । ए॒ते यत् । यदा॑दि॒त्यः । आ॒दि॒त्य ए॒षः । ए॒ष रु॒द्रः । रु॒द्रो यत् । यद॒ग्निः । अ॒ग्निरोष॑धीः । ओष॑धीः॒ प्रास्य॑ । प्रास्या॒ग्नौ । प्रास्येति॑ प्र - अस्य॑ । अ॒ग्नावा॑दि॒त्यम् । आ॒दि॒त्यम् जु॑होति । जु॒हो॒ति॒ रु॒द्रात् । रु॒द्रादे॒व । ए॒व प॒शून् । प॒शून॒न्तः । अ॒न्तर् द॑धाति । द॒धा॒त्यथो᳚ । अथो॒ ओष॑धीषु । अथो॒ इत्यथो᳚ । ओष॑धीष्वे॒व । ए॒व प॒शून् । प॒शून् प्रति॑ \newline

\textbf{Jatai Paata} \newline

1. आ स॑मु॒द्राथ् स॑मु॒द्रादा स॑मु॒द्रात् । \newline
2. स॒मु॒द्रादा स॑मु॒द्राथ् स॑मु॒द्रादा । \newline
3. आ ऽन्तरि॑क्षा द॒न्तरि॑क्षा॒दा ऽन्तरि॑क्षात् । \newline
4. अ॒न्तरि॑क्षात् प्र॒जाप॑तिः प्र॒जाप॑ति र॒न्तरि॑क्षा द॒न्तरि॑क्षात् प्र॒जाप॑तिः । \newline
5. प्र॒जाप॑ति रुद॒धि मु॑द॒धिम् प्र॒जाप॑तिः प्र॒जाप॑ति रुद॒धिम् । \newline
6. प्र॒जाप॑ति॒रिति॑ प्र॒जा - प॒तिः॒ । \newline
7. उ॒द॒धिम् च्या॑वयाति च्यावया त्युद॒धि मु॑द॒धिम् च्या॑वयाति । \newline
8. उ॒द॒धिमित्यु॑द - धिम् । \newline
9. च्या॒व॒या॒तीन्द्र॒ इन्द्र॑ श्च्यावयाति च्यावया॒तीन्द्रः॑ । \newline
10. इन्द्रः॒ प्र प्रेन्द्र॒ इन्द्रः॒ प्र । \newline
11. प्र स्नौ॑तु स्नौतु॒ प्र प्र स्नौ॑तु । \newline
12. स्नौ॒तु॒ म॒रुतो॑ म॒रुतः॑ स्नौतु स्नौतु म॒रुतः॑ । \newline
13. म॒रुतो॑ वर्.षयन्तु वर्.षयन्तु म॒रुतो॑ म॒रुतो॑ वर्.षयन्तु । \newline
14. व॒र्॒.ष॒य॒ न्तूदुद् व॑र्.षयन्तु वर्.षय॒न्तूत् । \newline
15. उन् न॑म्भय नम्भ॒यो दुन् न॑म्भय । \newline
16. न॒म्भ॒य॒ पृ॒थि॒वीम् पृ॑थि॒वीम् न॑म्भय नम्भय पृथि॒वीम् । \newline
17. पृ॒थि॒वीम् भि॒न्धि भि॒न्धि पृ॑थि॒वीम् पृ॑थि॒वीम् भि॒न्धि । \newline
18. भि॒न्धीद मि॒दम् भि॒न्धि भि॒न्धीदम् । \newline
19. इ॒दम् दि॒व्यम् दि॒व्य मि॒द मि॒दम् दि॒व्यम् । \newline
20. दि॒व्यम् नभो॒ नभो॑ दि॒व्यम् दि॒व्यम् नभः॑ । \newline
21. नभ॒ इति॒ नभः॑ । \newline
22. उ॒द्नो दि॒व्यस्य॑ दि॒व्यस्यो॒द्न उ॒द्नो दि॒व्यस्य॑ । \newline
23. दि॒व्यस्य॑ नो नो दि॒व्यस्य॑ दि॒व्यस्य॑ नः । \newline
24. नो॒ दे॒हि॒ दे॒हि॒ नो॒ नो॒ दे॒हि॒ । \newline
25. दे॒हीशा॑न॒ ईशा॑नो देहि दे॒हीशा॑नः । \newline
26. ईशा॑नो॒ वि वीशा॑न॒ ईशा॑नो॒ वि । \newline
27. वि सृ॑ज सृज॒ वि वि सृ॑ज । \newline
28. सृ॒जा॒ दृति॒म् दृतिꣳ॑ सृज सृजा॒ दृति᳚म् । \newline
29. दृति॒मिति॒ दृति᳚म् । \newline
30. प॒शवो॒ वै वै प॒शवः॑ प॒शवो॒ वै । \newline
31. वा ए॒त ए॒ते वै वा ए॒ते । \newline
32. ए॒ते यद् यदे॒त ए॒ते यत् । \newline
33. यदा॑दि॒त्य आ॑दि॒त्यो यद् यदा॑दि॒त्यः । \newline
34. आ॒दि॒त्य ए॒ष ए॒ष आ॑दि॒त्य आ॑दि॒त्य ए॒षः । \newline
35. ए॒ष रु॒द्रो रु॒द्र ए॒ष ए॒ष रु॒द्रः । \newline
36. रु॒द्रो यद् यद् रु॒द्रो रु॒द्रो यत् । \newline
37. यद॒ग्नि र॒ग्निर् यद् यद॒ग्निः । \newline
38. अ॒ग्नि रोष॑धी॒ रोष॑धी र॒ग्नि र॒ग्नि रोष॑धीः । \newline
39. ओष॑धीः॒ प्रास्य॒ प्रास्यौष॑धी॒ रोष॑धीः॒ प्रास्य॑ । \newline
40. प्रास्या॒ग्ना व॒ग्नौ प्रास्य॒ प्रास्या॒ग्नौ । \newline
41. प्रास्येति॑ प्र - अस्य॑ । \newline
42. अ॒ग्ना वा॑दि॒त्य मा॑दि॒त्य म॒ग्ना व॒ग्ना वा॑दि॒त्यम् । \newline
43. आ॒दि॒त्यम् जु॑होति जुहो त्यादि॒त्य मा॑दि॒त्यम् जु॑होति । \newline
44. जु॒हो॒ति॒ रु॒द्राद् रु॒द्राज् जु॑होति जुहोति रु॒द्रात् । \newline
45. रु॒द्रा दे॒वैव रु॒द्राद् रु॒द्रा दे॒व । \newline
46. ए॒व प॒शून् प॒शू ने॒वैव प॒शून् । \newline
47. प॒शू न॒न्त र॒न्तः प॒शून् प॒शू न॒न्तः । \newline
48. अ॒न्तर् द॑धाति दधा त्य॒न्त र॒न्तर् द॑धाति । \newline
49. द॒धा॒ त्यथो॒ अथो॑ दधाति दधा॒ त्यथो᳚ । \newline
50. अथो॒ ओष॑धी॒ ष्वोष॑धी॒ ष्वथो॒ अथो॒ ओष॑धीषु । \newline
51. अथो॒ इत्यथो᳚ । \newline
52. ओष॑धी ष्वे॒वै वौष॑धी॒ ष्वोष॑धी ष्वे॒व । \newline
53. ए॒व प॒शून् प॒शू ने॒वैव प॒शून् । \newline
54. प॒शून् प्रति॒ प्रति॑ प॒शून् प॒शून् प्रति॑ । \newline

\textbf{Ghana Paata } \newline

1. आ स॑मु॒द्राथ् स॑मु॒द्रादा स॑मु॒द्रादा स॑मु॒द्रादा स॑मु॒द्रादा । \newline
2. स॒मु॒द्रादा स॑मु॒द्राथ् स॑मु॒द्रादा ऽन्तरि॑क्षा द॒न्तरि॑क्षा॒दा स॑मु॒द्राथ् स॑मु॒द्रादा ऽन्तरि॑क्षात् । \newline
3. आ ऽन्तरि॑क्षा द॒न्तरि॑क्षा॒दा ऽन्तरि॑क्षात् प्र॒जाप॑तिः प्र॒जाप॑ति र॒न्तरि॑क्षा॒दा ऽन्तरि॑क्षात् प्र॒जाप॑तिः । \newline
4. अ॒न्तरि॑क्षात् प्र॒जाप॑तिः प्र॒जाप॑ति र॒न्तरि॑क्षा द॒न्तरि॑क्षात् प्र॒जाप॑ति रुद॒धि मु॑द॒धिम् प्र॒जाप॑ति र॒न्तरि॑क्षा द॒न्तरि॑क्षात् प्र॒जाप॑ति रुद॒धिम् । \newline
5. प्र॒जाप॑ति रुद॒धि मु॑द॒धिम् प्र॒जाप॑तिः प्र॒जाप॑ति रुद॒धिम् च्या॑वयाति च्यावया त्युद॒धिम् प्र॒जाप॑तिः प्र॒जाप॑ति रुद॒धिम् च्या॑वयाति । \newline
6. प्र॒जाप॑ति॒रिति॑ प्र॒जा - प॒तिः॒ । \newline
7. उ॒द॒धिम् च्या॑वयाति च्यावया त्युद॒धि मु॑द॒धिम् च्या॑वया॒तीन्द्र॒ इन्द्र॑ श्च्यावया त्युद॒धि मु॑द॒धिम् च्या॑वया॒तीन्द्रः॑ । \newline
8. उ॒द॒धिमित्यु॑द - धिम् । \newline
9. च्या॒व॒या॒तीन्द्र॒ इन्द्र॑ श्च्यावयाति च्यावया॒तीन्द्रः॒ प्र प्रेन्द्र॑ श्च्यावयाति च्यावया॒तीन्द्रः॒ प्र । \newline
10. इन्द्रः॒ प्र प्रेन्द्र॒ इन्द्रः॒ प्र स्नौ॑तु स्नौतु॒ प्रेन्द्र॒ इन्द्रः॒ प्र स्नौ॑तु । \newline
11. प्र स्नौ॑तु स्नौतु॒ प्र प्र स्नौ॑तु म॒रुतो॑ म॒रुतः॑ स्नौतु॒ प्र प्र स्नौ॑तु म॒रुतः॑ । \newline
12. स्नौ॒तु॒ म॒रुतो॑ म॒रुतः॑ स्नौतु स्नौतु म॒रुतो॑ वर्.षयन्तु वर्.षयन्तु म॒रुतः॑ स्नौतु स्नौतु म॒रुतो॑ वर्.षयन्तु । \newline
13. म॒रुतो॑ वर्.षयन्तु वर्.षयन्तु म॒रुतो॑ म॒रुतो॑ वर्.षय॒न्तूदुद् व॑र्.षयन्तु म॒रुतो॑ म॒रुतो॑ वर्.षय॒न्तूत् । \newline
14. व॒र्॒.ष॒य॒न्तूदुद् व॑र्.षयन्तु वर्.षय॒न्तून् न॑म्भय नम्भ॒योद् व॑र्.षयन्तु वर्.षय॒न्तून् न॑म्भय । \newline
15. उन् न॑म्भय नम्भ॒योदुन् न॑म्भय पृथि॒वीम् पृ॑थि॒वीम् न॑म्भ॒योदुन् न॑म्भय पृथि॒वीम् । \newline
16. न॒म्भ॒य॒ पृ॒थि॒वीम् पृ॑थि॒वीम् न॑म्भय नम्भय पृथि॒वीम् भि॒न्धि भि॒न्धि पृ॑थि॒वीम् न॑म्भय नम्भय पृथि॒वीम् भि॒न्धि । \newline
17. पृ॒थि॒वीम् भि॒न्धि भि॒न्धि पृ॑थि॒वीम् पृ॑थि॒वीम् भि॒न्धीद मि॒दम् भि॒न्धि पृ॑थि॒वीम् पृ॑थि॒वीम् भि॒न्धीदम् । \newline
18. भि॒न्धीद मि॒दम् भि॒न्धि भि॒न्धीदम् दि॒व्यम् दि॒व्य मि॒दम् भि॒न्धि भि॒न्धीदम् दि॒व्यम् । \newline
19. इ॒दम् दि॒व्यम् दि॒व्य मि॒द मि॒दम् दि॒व्यन् नभो॒ नभो॑ दि॒व्य मि॒द मि॒दम् दि॒व्यन् नभः॑ । \newline
20. दि॒व्यन् नभो॒ नभो॑ दि॒व्यम् दि॒व्यन् नभः॑ । \newline
21. नभ॒ इति॒ नभः॑ । \newline
22. उ॒द्नो दि॒व्यस्य॑ दि॒व्यस्यो॒द्न उ॒द्नो दि॒व्यस्य॑ नो नो दि॒व्यस्यो॒द्न उ॒द्नो दि॒व्यस्य॑ नः । \newline
23. दि॒व्यस्य॑ नो नो दि॒व्यस्य॑ दि॒व्यस्य॑ नो देहि देहि नो दि॒व्यस्य॑ दि॒व्यस्य॑ नो देहि । \newline
24. नो॒ दे॒हि॒ दे॒हि॒ नो॒ नो॒ दे॒हीशा॑न॒ ईशा॑नो देहि नो नो दे॒हीशा॑नः । \newline
25. दे॒हीशा॑न॒ ईशा॑नो देहि दे॒हीशा॑नो॒ वि वीशा॑नो देहि दे॒हीशा॑नो॒ वि । \newline
26. ईशा॑नो॒ वि वीशा॑न॒ ईशा॑नो॒ वि सृ॑ज सृज॒ वीशा॑न॒ ईशा॑नो॒ वि सृ॑ज । \newline
27. वि सृ॑ज सृज॒ वि वि सृ॑जा॒ दृति॒म् दृतिꣳ॑ सृज॒ वि वि सृ॑जा॒ दृति᳚म् । \newline
28. सृ॒जा॒ दृति॒म् दृतिꣳ॑ सृज सृजा॒ दृति᳚म् । \newline
29. दृति॒मिति॒ दृति᳚म् । \newline
30. प॒शवो॒ वै वै प॒शवः॑ प॒शवो॒ वा ए॒त ए॒ते वै प॒शवः॑ प॒शवो॒ वा ए॒ते । \newline
31. वा ए॒त ए॒ते वै वा ए॒ते यद् यदे॒ते वै वा ए॒ते यत् । \newline
32. ए॒ते यद् यदे॒त ए॒ते यदा॑दि॒त्य आ॑दि॒त्यो यदे॒त ए॒ते यदा॑दि॒त्यः । \newline
33. यदा॑दि॒त्य आ॑दि॒त्यो यद् यदा॑दि॒त्य ए॒ष ए॒ष आ॑दि॒त्यो यद् यदा॑दि॒त्य ए॒षः । \newline
34. आ॒दि॒त्य ए॒ष ए॒ष आ॑दि॒त्य आ॑दि॒त्य ए॒ष रु॒द्रो रु॒द्र ए॒ष आ॑दि॒त्य आ॑दि॒त्य ए॒ष रु॒द्रः । \newline
35. ए॒ष रु॒द्रो रु॒द्र ए॒ष ए॒ष रु॒द्रो यद् यद् रु॒द्र ए॒ष ए॒ष रु॒द्रो यत् । \newline
36. रु॒द्रो यद् यद् रु॒द्रो रु॒द्रो यद॒ग्नि र॒ग्निर् यद् रु॒द्रो रु॒द्रो यद॒ग्निः । \newline
37. यद॒ग्नि र॒ग्निर् यद् यद॒ग्नि रोष॑धी॒ रोष॑धी र॒ग्निर् यद् यद॒ग्नि रोष॑धीः । \newline
38. अ॒ग्नि रोष॑धी॒ रोष॑धी र॒ग्नि र॒ग्नि रोष॑धीः॒ प्रास्य॒ प्रास्यौष॑धी र॒ग्नि र॒ग्नि रोष॑धीः॒ प्रास्य॑ । \newline
39. ओष॑धीः॒ प्रास्य॒ प्रास्यौष॑धी॒ रोष॑धीः॒ प्रास्या॒ग्ना व॒ग्नौ प्रास्यौष॑धी॒ रोष॑धीः॒ प्रास्या॒ग्नौ । \newline
40. प्रास्या॒ग्ना व॒ग्नौ प्रास्य॒ प्रास्या॒ग्ना वा॑दि॒त्य मा॑दि॒त्य म॒ग्नौ प्रास्य॒ प्रास्या॒ग्ना वा॑दि॒त्यम् । \newline
41. प्रास्येति॑ प्र - अस्य॑ । \newline
42. अ॒ग्ना वा॑दि॒त्य मा॑दि॒त्य म॒ग्ना व॒ग्ना वा॑दि॒त्यम् जु॑होति जुहो त्यादि॒त्य म॒ग्ना व॒ग्ना वा॑दि॒त्यम् जु॑होति । \newline
43. आ॒दि॒त्यम् जु॑होति जुहो त्यादि॒त्य मा॑दि॒त्यम् जु॑होति रु॒द्राद् रु॒द्राज् जु॑हो त्यादि॒त्य मा॑दि॒त्यम् जु॑होति रु॒द्रात् । \newline
44. जु॒हो॒ति॒ रु॒द्राद् रु॒द्राज् जु॑होति जुहोति रु॒द्रा दे॒वैव रु॒द्राज् जु॑होति जुहोति रु॒द्रा दे॒व । \newline
45. रु॒द्रा दे॒वैव रु॒द्राद् रु॒द्रा दे॒व प॒शून् प॒शू ने॒व रु॒द्राद् रु॒द्रा दे॒व प॒शून् । \newline
46. ए॒व प॒शून् प॒शू ने॒वैव प॒शू न॒न्त र॒न्तः प॒शू ने॒वैव प॒शू न॒न्तः । \newline
47. प॒शू न॒न्त र॒न्तः प॒शून् प॒शू न॒न्तर् द॑धाति दधा त्य॒न्तः प॒शून् प॒शू न॒न्तर् द॑धाति । \newline
48. अ॒न्तर् द॑धाति दधा त्य॒न्त र॒न्तर् द॑धा॒ त्यथो॒ अथो॑ दधा त्य॒न्त र॒न्तर् द॑धा॒ त्यथो᳚ । \newline
49. द॒धा॒ त्यथो॒ अथो॑ दधाति दधा॒ त्यथो॒ ओष॑धी॒ ष्वोष॑धी॒ ष्वथो॑ दधाति दधा॒ त्यथो॒ ओष॑धीषु । \newline
50. अथो॒ ओष॑धी॒ ष्वोष॑धी॒ ष्वथो॒ अथो॒ ओष॑धी ष्वे॒वैवौष॑धी॒ ष्वथो॒ अथो॒ ओष॑धी ष्वे॒व । \newline
51. अथो॒ इत्यथो᳚ । \newline
52. ओष॑धी ष्वे॒वैवौष॑धी॒ ष्वोष॑धी ष्वे॒व प॒शून् प॒शू ने॒वौष॑धी॒ ष्वोष॑धी ष्वे॒व प॒शून् । \newline
53. ए॒व प॒शून् प॒शू ने॒वैव प॒शून् प्रति॒ प्रति॑ प॒शू ने॒वैव प॒शून् प्रति॑ । \newline
54. प॒शून् प्रति॒ प्रति॑ प॒शून् प॒शून् प्रति॑ ष्ठापयति स्थापयति॒ प्रति॑ प॒शून् प॒शून् प्रति॑ ष्ठापयति । \newline
\pagebreak
\markright{ TS 3.5.5.3  \hfill https://www.vedavms.in \hfill}

\section{ TS 3.5.5.3 }

\textbf{TS 3.5.5.3 } \newline
\textbf{Samhita Paata} \newline

प्रति॑ष्ठापयति क॒विर्य॒ज्ञ्स्य॒ वित॑नोति॒ पन्थां॒ नाक॑स्य पृ॒ष्ठे अधि॑ रोच॒ने दि॒वः । येन॑ ह॒व्यं ॅवह॑सि॒ यासि॑ दू॒त इ॒तः प्रचे॑ता अ॒मुतः॒ सनी॑यान् ॥ यास्ते॒ विश्वाः᳚ स॒मिधः॒ सन्त्य॑ग्ने॒याः पृ॑थि॒व्यां ब॒र्॒.हिषि॒ सूर्ये॒ याः । तास्ते॑ गच्छ॒न्त्वाहु॑तिं घृ॒तस्य॑ देवाय॒ते यज॑मानाय॒ शर्म॑ ॥आ॒शासा॑नः सु॒वीर्यꣳ॑ रा॒यस्पोषꣳ॒॒ स्वश्वि॑यं । बृह॒स्पति॑ना रा॒या स्व॒गाकृ॑तो॒ मह्यं॒ ॅयज॑मानाय ( ) तिष्ठ ॥ \newline

\textbf{Pada Paata} \newline

प्रतीति॑ । स्था॒प॒य॒ति॒ । क॒विः । य॒ज्ञ्स्य॑ । वीति॑ । त॒नो॒ति॒ । पन्था᳚म् । नाक॑स्य । पृ॒ष्ठे । अधीति॑ । रो॒च॒ने । दि॒वः ॥ येन॑ । ह॒व्यम् । वह॑सि । यासि॑ । दू॒तः । इ॒तः । प्रचे॑ता॒ इति॒ प्र-चे॒ताः॒ । अ॒मुतः॑ । सनी॑यान् ॥ याः । ते॒ । विश्वाः᳚ । स॒मिध॒ इति॑ सं - इधः॑ । सन्ति॑ । अ॒ग्ने॒ । याः । पृ॒थि॒व्याम् । ब॒र्॒.हिषि॑ । सूर्ये᳚ । याः ॥ ताः । ते॒ । ग॒च्छ॒न्तु॒ । आहु॑ति॒मित्या - हु॒ति॒म् । घृ॒तस्य॑ । दे॒वा॒य॒त इति॑ देव - य॒ते । यज॑मानाय । शर्म॑ ॥ आ॒शासा॑न॒ इत्या᳚ - शासा॑नः । सु॒वीर्य॒मिति॑ सु - वीर्य᳚म् । रा॒यः । पोष᳚म् । स्वश्वि॑य॒मिति॑ सु - अश्वि॑यम् ॥ बृह॒स्पति॑ना । रा॒या । स्व॒गाकृ॑त॒ इति॑ स्व॒गा - कृ॒तः॒ । मह्य᳚म् । यज॑मानाय ( ) । ति॒ष्ठ॒ ॥  \newline


\textbf{Krama Paata} \newline

प्रति॑ ष्ठापयति । स्था॒प॒य॒ति॒ क॒विः । क॒विर् य॒ज्ञ्स्य॑ । 
य॒ज्ञ्स्य॒ वि । वि त॑नोति । त॒नो॒ति॒ पन्था᳚म् । पन्था॒म् नाक॑स्य । नाक॑स्य पृ॒ष्ठे । पृ॒ष्ठे अधि॑ । अधि॑ रोच॒ने । रो॒च॒ने दि॒वः । दि॒व इति॑ दि॒वः ॥ येन॑ ह॒व्यम् । ह॒व्यं ॅवह॑सि । वह॑सि॒ यासि॑ । यासि॑ दू॒तः । दू॒त इ॒तः । इ॒तः प्रचे॑ताः । प्रचे॑ता अ॒मुतः॑ । प्रचे॑ता॒ इति॒ प्र - चे॒ताः॒ । अ॒मुतः॒ सनी॑यान् । सनी॑या॒निति॒ सनी॑यान् ॥ यास्ते᳚ । ते॒ विश्वाः᳚ । विश्वाः᳚ स॒मिधः॑ । स॒मिधः॒ सन्ति॑ । स॒मिध॒ इति॑ सम् - इधः॑ । सन्त्य॑ग्ने । अ॒ग्ने॒ याः । याः पृ॑थि॒व्याम् । पृ॒थि॒व्याम् ब॒र्॒.हिषि॑ । ब॒र्॒.हिषि॒ सूर्ये᳚ । सूर्ये॒ याः । या इति॒ याः ॥ तास्ते᳚ । ते॒ ग॒च्छ॒न्तु॒ । ग॒च्छ॒न्त्वाहु॑तिम् । आहु॑तिम् घृ॒तस्य॑ । आहु॑ति॒मित्या - हु॒ति॒म् । घृ॒तस्य॑ देवाय॒ते । दे॒वा॒य॒ते यज॑मानाय । दे॒वा॒य॒त इति॑ देव - य॒ते । यज॑मानाय॒ शर्म॑ । शर्मेति॒ शर्म॑ ॥ आ॒शासा॑नः सु॒वीर्य᳚म् । आ॒शासा॑न॒ इत्या᳚ - शासा॑नः । सु॒वीर्यꣳ॑ रा॒यः । सु॒वीर्य॒मिति॑ सु - वीर्य᳚म् । रा॒यस्पोष᳚म् । पोषꣳ॒॒ स्वश्वि॑यम् । स्वश्वि॑य॒मिति॑ सु - अश्वि॑यम् ॥ बृह॒स्पति॑ना रा॒या । रा॒या स्व॒गाकृ॑तः । स्व॒गाकृ॑तो॒ मह्य᳚म् । स्व॒गाकृ॑त॒ इति॑ स्व॒गा - कृ॒तः॒ । मह्यं॒ ॅयज॑मानाय । यज॑मानाय तिष्ठ । ति॒ष्ठेति॑ तिष्ठ । \newline

\textbf{Jatai Paata} \newline

1. प्रति॑ ष्ठापयति स्थापयति॒ प्रति॒ प्रति॑ ष्ठापयति । \newline
2. स्था॒प॒य॒ति॒ क॒विः क॒विः स्था॑पयति स्थापयति क॒विः । \newline
3. क॒विर् य॒ज्ञ्स्य॑ य॒ज्ञ्स्य॑ क॒विः क॒विर् य॒ज्ञ्स्य॑ । \newline
4. य॒ज्ञ्स्य॒ वि वि य॒ज्ञ्स्य॑ य॒ज्ञ्स्य॒ वि । \newline
5. वि त॑नोति तनोति॒ वि वि त॑नोति । \newline
6. त॒नो॒ति॒ पन्था॒म् पन्था᳚म् तनोति तनोति॒ पन्था᳚म् । \newline
7. पन्था॒म् नाक॑स्य॒ नाक॑स्य॒ पन्था॒म् पन्था॒म् नाक॑स्य । \newline
8. नाक॑स्य पृ॒ष्ठे पृ॒ष्ठे नाक॑स्य॒ नाक॑स्य पृ॒ष्ठे । \newline
9. पृ॒ष्ठे अध्यधि॑ पृ॒ष्ठे पृ॒ष्ठे अधि॑ । \newline
10. अधि॑ रोच॒ने रो॑च॒ने ऽध्यधि॑ रोच॒ने । \newline
11. रो॒च॒ने दि॒वो दि॒वो रो॑च॒ने रो॑च॒ने दि॒वः । \newline
12. दि॒व इति॑ दि॒वः । \newline
13. येन॑ ह॒व्यꣳ ह॒व्यं ॅयेन॒ येन॑ ह॒व्यम् । \newline
14. ह॒व्यं ॅवह॑सि॒ वह॑सि ह॒व्यꣳ ह॒व्यं ॅवह॑सि । \newline
15. वह॑सि॒ यासि॒ यासि॒ वह॑सि॒ वह॑सि॒ यासि॑ । \newline
16. यासि॑ दू॒तो दू॒तो यासि॒ यासि॑ दू॒तः । \newline
17. दू॒त इ॒त इ॒तो दू॒तो दू॒त इ॒तः । \newline
18. इ॒तः प्रचे॑ताः॒ प्रचे॑ता इ॒त इ॒तः प्रचे॑ताः । \newline
19. प्रचे॑ता अ॒मुतो॒ ऽमुतः॒ प्रचे॑ताः॒ प्रचे॑ता अ॒मुतः॑ । \newline
20. प्रचे॑ता॒ इति॒ प्र - चे॒ताः॒ । \newline
21. अ॒मुतः॒ सनी॑या॒न् थ्सनी॑या न॒मुतो॒ ऽमुतः॒ सनी॑यान् । \newline
22. सनी॑या॒निति॒ सनी॑यान् । \newline
23. या स्ते॑ ते॒ या या स्ते᳚ । \newline
24. ते॒ विश्वा॒ विश्वा᳚ स्ते ते॒ विश्वाः᳚ । \newline
25. विश्वाः᳚ स॒मिधः॑ स॒मिधो॒ विश्वा॒ विश्वाः᳚ स॒मिधः॑ । \newline
26. स॒मिधः॒ सन्ति॒ सन्ति॑ स॒मिधः॑ स॒मिधः॒ सन्ति॑ । \newline
27. स॒मिध॒ इति॑ सं - इधः॑ । \newline
28. सन्त्य॑ग्ने ऽग्ने॒ सन्ति॒ सन्त्य॑ग्ने । \newline
29. अ॒ग्ने॒ या या अ॑ग्ने ऽग्ने॒ याः । \newline
30. याः पृ॑थि॒व्याम् पृ॑थि॒व्यां ॅया याः पृ॑थि॒व्याम् । \newline
31. पृ॒थि॒व्याम् ब॒र्॒.हिषि॑ ब॒र्॒.हिषि॑ पृथि॒व्याम् पृ॑थि॒व्याम् ब॒र्॒.हिषि॑ । \newline
32. ब॒र्॒.हिषि॒ सूर्ये॒ सूर्ये॑ ब॒र्॒.हिषि॑ ब॒र्॒.हिषि॒ सूर्ये᳚ । \newline
33. सूर्ये॒ या याः सूर्ये॒ सूर्ये॒ याः । \newline
34. या इति॒ याः । \newline
35. ता स्ते॑ ते॒ ता स्ता स्ते᳚ । \newline
36. ते॒ ग॒च्छ॒न्तु॒ ग॒च्छ॒न्तु॒ ते॒ ते॒ ग॒च्छ॒न्तु॒ । \newline
37. ग॒च्छ॒ न्त्वाहु॑ति॒ माहु॑तिम् गच्छन्तु गच्छ॒ न्त्वाहु॑तिम् । \newline
38. आहु॑तिम् घृ॒तस्य॑ घृ॒तस्या हु॑ति॒ माहु॑तिम् घृ॒तस्य॑ । \newline
39. आहु॑ति॒मित्या - हु॒ति॒म् । \newline
40. घृ॒तस्य॑ देवाय॒ते दे॑वाय॒ते घृ॒तस्य॑ घृ॒तस्य॑ देवाय॒ते । \newline
41. दे॒वा॒य॒ते यज॑मानाय॒ यज॑मानाय देवाय॒ते दे॑वाय॒ते यज॑मानाय । \newline
42. दे॒वा॒य॒त इति॑ देव - य॒ते । \newline
43. यज॑मानाय॒ शर्म॒ शर्म॒ यज॑मानाय॒ यज॑मानाय॒ शर्म॑ । \newline
44. शर्मेति॒ शर्म॑ । \newline
45. आ॒शासा॑नः सु॒वीर्यꣳ॑ सु॒वीर्य॑ मा॒शासा॑न आ॒शासा॑नः सु॒वीर्य᳚म् । \newline
46. आ॒शासा॑न॒ इत्या᳚ - शासा॑नः । \newline
47. सु॒वीर्यꣳ॑ रा॒यो रा॒यः सु॒वीर्यꣳ॑ सु॒वीर्यꣳ॑ रा॒यः । \newline
48. सु॒वीर्य॒मिति॑ सु - वीर्य᳚म् । \newline
49. रा॒य स्पोष॒म् पोषꣳ॑ रा॒यो रा॒य स्पोष᳚म् । \newline
50. पोषꣳ॒॒ स्वश्वि॑यꣳ॒॒ स्वश्वि॑य॒म् पोष॒म् पोषꣳ॒॒ स्वश्वि॑यम् । \newline
51. स्वश्वि॑य॒मिति॑ सु - अश्वि॑यम् । \newline
52. बृह॒स्पति॑ना रा॒या रा॒या बृह॒स्पति॑ना॒ बृह॒स्पति॑ना रा॒या । \newline
53. रा॒या स्व॒गाकृ॑तः स्व॒गाकृ॑तो रा॒या रा॒या स्व॒गाकृ॑तः । \newline
54. स्व॒गाकृ॑तो॒ मह्य॒म् मह्यꣳ॑ स्व॒गाकृ॑तः स्व॒गाकृ॑तो॒ मह्य᳚म् । \newline
55. स्व॒गाकृ॑त॒ इति॑ स्व॒गा - कृ॒तः॒ । \newline
56. मह्यं॒ ॅयज॑मानाय॒ यज॑मानाय॒ मह्य॒म् मह्यं॒ ॅयज॑मानाय । \newline
57. यज॑मानाय तिष्ठ तिष्ठ॒ यज॑मानाय॒ यज॑मानाय तिष्ठ । \newline
58. ति॒ष्ठेति॑ तिष्ठ । \newline

\textbf{Ghana Paata } \newline

1. प्रति॑ ष्ठापयति स्थापयति॒ प्रति॒ प्रति॑ ष्ठापयति क॒विः क॒विः स्था॑पयति॒ प्रति॒ प्रति॑ ष्ठापयति क॒विः । \newline
2. स्था॒प॒य॒ति॒ क॒विः क॒विः स्था॑पयति स्थापयति क॒विर् य॒ज्ञ्स्य॑ य॒ज्ञ्स्य॑ क॒विः स्था॑पयति स्थापयति क॒विर् य॒ज्ञ्स्य॑ । \newline
3. क॒विर् य॒ज्ञ्स्य॑ य॒ज्ञ्स्य॑ क॒विः क॒विर् य॒ज्ञ्स्य॒ वि वि य॒ज्ञ्स्य॑ क॒विः क॒विर् य॒ज्ञ्स्य॒ वि । \newline
4. य॒ज्ञ्स्य॒ वि वि य॒ज्ञ्स्य॑ य॒ज्ञ्स्य॒ वि त॑नोति तनोति॒ वि य॒ज्ञ्स्य॑ य॒ज्ञ्स्य॒ वि त॑नोति । \newline
5. वि त॑नोति तनोति॒ वि वि त॑नोति॒ पन्था॒म् पन्था᳚म् तनोति॒ वि वि त॑नोति॒ पन्था᳚म् । \newline
6. त॒नो॒ति॒ पन्था॒म् पन्था᳚म् तनोति तनोति॒ पन्था॒म् नाक॑स्य॒ नाक॑स्य॒ पन्था᳚म् तनोति तनोति॒ पन्था॒म् नाक॑स्य । \newline
7. पन्था॒म् नाक॑स्य॒ नाक॑स्य॒ पन्था॒म् पन्था॒म् नाक॑स्य पृ॒ष्ठे पृ॒ष्ठे नाक॑स्य॒ पन्था॒म् पन्था॒म् नाक॑स्य पृ॒ष्ठे । \newline
8. नाक॑स्य पृ॒ष्ठे पृ॒ष्ठे नाक॑स्य॒ नाक॑स्य पृ॒ष्ठे अध्यधि॑ पृ॒ष्ठे नाक॑स्य॒ नाक॑स्य पृ॒ष्ठे अधि॑ । \newline
9. पृ॒ष्ठे अध्यधि॑ पृ॒ष्ठे पृ॒ष्ठे अधि॑ रोच॒ने रो॑च॒ने ऽधि॑ पृ॒ष्ठे पृ॒ष्ठे अधि॑ रोच॒ने । \newline
10. अधि॑ रोच॒ने रो॑च॒ने ऽध्यधि॑ रोच॒ने दि॒वो दि॒वो रो॑च॒ने ऽध्यधि॑ रोच॒ने दि॒वः । \newline
11. रो॒च॒ने दि॒वो दि॒वो रो॑च॒ने रो॑च॒ने दि॒वः । \newline
12. दि॒व इति॑ दि॒वः । \newline
13. येन॑ ह॒व्यꣳ ह॒व्यं ॅयेन॒ येन॑ ह॒व्यं ॅवह॑सि॒ वह॑सि ह॒व्यं ॅयेन॒ येन॑ ह॒व्यं ॅवह॑सि । \newline
14. ह॒व्यं ॅवह॑सि॒ वह॑सि ह॒व्यꣳ ह॒व्यं ॅवह॑सि॒ यासि॒ यासि॒ वह॑सि ह॒व्यꣳ ह॒व्यं ॅवह॑सि॒ यासि॑ । \newline
15. वह॑सि॒ यासि॒ यासि॒ वह॑सि॒ वह॑सि॒ यासि॑ दू॒तो दू॒तो यासि॒ वह॑सि॒ वह॑सि॒ यासि॑ दू॒तः । \newline
16. यासि॑ दू॒तो दू॒तो यासि॒ यासि॑ दू॒त इ॒त इ॒तो दू॒तो यासि॒ यासि॑ दू॒त इ॒तः । \newline
17. दू॒त इ॒त इ॒तो दू॒तो दू॒त इ॒तः प्रचे॑ताः॒ प्रचे॑ता इ॒तो दू॒तो दू॒त इ॒तः प्रचे॑ताः । \newline
18. इ॒तः प्रचे॑ताः॒ प्रचे॑ता इ॒त इ॒तः प्रचे॑ता अ॒मुतो॒ ऽमुतः॒ प्रचे॑ता इ॒त इ॒तः प्रचे॑ता अ॒मुतः॑ । \newline
19. प्रचे॑ता अ॒मुतो॒ ऽमुतः॒ प्रचे॑ताः॒ प्रचे॑ता अ॒मुतः॒ सनी॑या॒न् थ्सनी॑या न॒मुतः॒ प्रचे॑ताः॒ प्रचे॑ता अ॒मुतः॒ सनी॑यान् । \newline
20. प्रचे॑ता॒ इति॒ प्र - चे॒ताः॒ । \newline
21. अ॒मुतः॒ सनी॑या॒न् थ्सनी॑या न॒मुतो॒ ऽमुतः॒ सनी॑यान् । \newline
22. सनी॑या॒निति॒ सनी॑यान् । \newline
23. या स्ते॑ ते॒ या या स्ते॒ विश्वा॒ विश्वा᳚ स्ते॒ या या स्ते॒ विश्वाः᳚ । \newline
24. ते॒ विश्वा॒ विश्वा᳚ स्ते ते॒ विश्वाः᳚ स॒मिधः॑ स॒मिधो॒ विश्वा᳚ स्ते ते॒ विश्वाः᳚ स॒मिधः॑ । \newline
25. विश्वाः᳚ स॒मिधः॑ स॒मिधो॒ विश्वा॒ विश्वाः᳚ स॒मिधः॒ सन्ति॒ सन्ति॑ स॒मिधो॒ विश्वा॒ विश्वाः᳚ स॒मिधः॒ सन्ति॑ । \newline
26. स॒मिधः॒ सन्ति॒ सन्ति॑ स॒मिधः॑ स॒मिधः॒ सन्त्य॑ग्ने ऽग्ने॒ सन्ति॑ स॒मिधः॑ स॒मिधः॒ सन्त्य॑ग्ने । \newline
27. स॒मिध॒ इति॑ सं - इधः॑ । \newline
28. सन्त्य॑ग्ने ऽग्ने॒ सन्ति॒ सन्त्य॑ग्ने॒ या या अ॑ग्ने॒ सन्ति॒ सन्त्य॑ग्ने॒ याः । \newline
29. अ॒ग्ने॒ या या अ॑ग्ने ऽग्ने॒ याः पृ॑थि॒व्याम् पृ॑थि॒व्यां ॅया अ॑ग्ने ऽग्ने॒ याः पृ॑थि॒व्याम् । \newline
30. याः पृ॑थि॒व्याम् पृ॑थि॒व्यां ॅया याः पृ॑थि॒व्याम् ब॒र्॒.हिषि॑ ब॒र्॒.हिषि॑ पृथि॒व्यां ॅया याः पृ॑थि॒व्याम् ब॒र्॒.हिषि॑ । \newline
31. पृ॒थि॒व्याम् ब॒र्॒.हिषि॑ ब॒र्॒.हिषि॑ पृथि॒व्याम् पृ॑थि॒व्याम् ब॒र्॒.हिषि॒ सूर्ये॒ सूर्ये॑ ब॒र्॒.हिषि॑ पृथि॒व्याम् पृ॑थि॒व्याम् ब॒र्॒.हिषि॒ सूर्ये᳚ । \newline
32. ब॒र्॒.हिषि॒ सूर्ये॒ सूर्ये॑ ब॒र्॒.हिषि॑ ब॒र्॒.हिषि॒ सूर्ये॒ या याः सूर्ये॑ ब॒र्॒.हिषि॑ ब॒र्॒.हिषि॒ सूर्ये॒ याः । \newline
33. सूर्ये॒ या याः सूर्ये॒ सूर्ये॒ याः । \newline
34. या इति॒ याः । \newline
35. ता स्ते॑ ते॒ ता स्ता स्ते॑ गच्छन्तु गच्छन्तु ते॒ ता स्ता स्ते॑ गच्छन्तु । \newline
36. ते॒ ग॒च्छ॒न्तु॒ ग॒च्छ॒न्तु॒ ते॒ ते॒ ग॒च्छ॒ न्त्वाहु॑ति॒ माहु॑तिम् गच्छन्तु ते ते गच्छ॒ न्त्वाहु॑तिम् । \newline
37. ग॒च्छ॒ न्त्वाहु॑ति॒ माहु॑तिम् गच्छन्तु गच्छ॒ न्त्वाहु॑तिम् घृ॒तस्य॑ घृ॒तस्याहु॑तिम् गच्छन्तु गच्छ॒ न्त्वाहु॑तिम् घृ॒तस्य॑ । \newline
38. आहु॑तिम् घृ॒तस्य॑ घृ॒तस्याहु॑ति॒ माहु॑तिम् घृ॒तस्य॑ देवाय॒ते दे॑वाय॒ते घृ॒तस्याहु॑ति॒ माहु॑तिम् घृ॒तस्य॑ देवाय॒ते । \newline
39. आहु॑ति॒मित्या - हु॒ति॒म् । \newline
40. घृ॒तस्य॑ देवाय॒ते दे॑वाय॒ते घृ॒तस्य॑ घृ॒तस्य॑ देवाय॒ते यज॑मानाय॒ यज॑मानाय देवाय॒ते घृ॒तस्य॑ घृ॒तस्य॑ देवाय॒ते यज॑मानाय । \newline
41. दे॒वा॒य॒ते यज॑मानाय॒ यज॑मानाय देवाय॒ते दे॑वाय॒ते यज॑मानाय॒ शर्म॒ शर्म॒ यज॑मानाय देवाय॒ते दे॑वाय॒ते यज॑मानाय॒ शर्म॑ । \newline
42. दे॒वा॒य॒त इति॑ देव - य॒ते । \newline
43. यज॑मानाय॒ शर्म॒ शर्म॒ यज॑मानाय॒ यज॑मानाय॒ शर्म॑ । \newline
44. शर्मेति॒ शर्म॑ । \newline
45. आ॒शासा॑नः सु॒वीर्यꣳ॑ सु॒वीर्य॑ मा॒शासा॑न आ॒शासा॑नः सु॒वीर्यꣳ॑ रा॒यो रा॒यः सु॒वीर्य॑ मा॒शासा॑न आ॒शासा॑नः सु॒वीर्यꣳ॑ रा॒यः । \newline
46. आ॒शासा॑न॒ इत्या᳚ - शासा॑नः । \newline
47. सु॒वीर्यꣳ॑ रा॒यो रा॒यः सु॒वीर्यꣳ॑ सु॒वीर्यꣳ॑ रा॒य स्पोष॒म् पोषꣳ॑ रा॒यः सु॒वीर्यꣳ॑ सु॒वीर्यꣳ॑ रा॒य स्पोष᳚म् । \newline
48. सु॒वीर्य॒मिति॑ सु - वीर्य᳚म् । \newline
49. रा॒य स्पोष॒म् पोषꣳ॑ रा॒यो रा॒य स्पोषꣳ॒॒ स्वश्वि॑यꣳ॒॒ स्वश्वि॑य॒म् पोषꣳ॑ रा॒यो रा॒य स्पोषꣳ॒॒ स्वश्वि॑यम् । \newline
50. पोषꣳ॒॒ स्वश्वि॑यꣳ॒॒ स्वश्वि॑य॒म् पोष॒म् पोषꣳ॒॒ स्वश्वि॑यम् । \newline
51. स्वश्वि॑य॒मिति॑ सु - अश्वि॑यम् । \newline
52. बृह॒स्पति॑ना रा॒या रा॒या बृह॒स्पति॑ना॒ बृह॒स्पति॑ना रा॒या स्व॒गाकृ॑तः स्व॒गाकृ॑तो रा॒या बृह॒स्पति॑ना॒ बृह॒स्पति॑ना रा॒या स्व॒गाकृ॑तः । \newline
53. रा॒या स्व॒गाकृ॑तः स्व॒गाकृ॑तो रा॒या रा॒या स्व॒गाकृ॑तो॒ मह्य॒म् मह्यꣳ॑ स्व॒गाकृ॑तो रा॒या रा॒या स्व॒गाकृ॑तो॒ मह्य᳚म् । \newline
54. स्व॒गाकृ॑तो॒ मह्य॒म् मह्यꣳ॑ स्व॒गाकृ॑तः स्व॒गाकृ॑तो॒ मह्यं॒ ॅयज॑मानाय॒ यज॑मानाय॒ मह्यꣳ॑ स्व॒गाकृ॑तः स्व॒गाकृ॑तो॒ मह्यं॒ ॅयज॑मानाय । \newline
55. स्व॒गाकृ॑त॒ इति॑ स्व॒गा - कृ॒तः॒ । \newline
56. मह्यं॒ ॅयज॑मानाय॒ यज॑मानाय॒ मह्य॒म् मह्यं॒ ॅयज॑मानाय तिष्ठ तिष्ठ॒ यज॑मानाय॒ मह्य॒म् मह्यं॒ ॅयज॑मानाय तिष्ठ । \newline
57. यज॑मानाय तिष्ठ तिष्ठ॒ यज॑मानाय॒ यज॑मानाय तिष्ठ । \newline
58. ति॒ष्ठेति॑ तिष्ठ । \newline
\pagebreak
\markright{ TS 3.5.6.1  \hfill https://www.vedavms.in \hfill}

\section{ TS 3.5.6.1 }

\textbf{TS 3.5.6.1 } \newline
\textbf{Samhita Paata} \newline

सं त्वा॑ नह्यामि॒ पय॑सा घृ॒तेन॒ सं त्वा॑ नह्याम्य॒प ओष॑धीभिः । सं त्वा॑ नह्यामि प्र॒जया॒ऽहम॒द्य सा दी᳚क्षि॒ता स॑नवो॒ वाज॑म॒स्मे ॥ प्रैतु॒ ब्रह्म॑ण॒स्पत्नी॒ वेदिं॒ ॅवर्णे॑न सीदतु । अथा॒हम॑नुका॒मिनी॒ स्वे लो॒के वि॒शा इ॒ह ॥ सु॒प्र॒जस॑स्त्वा व॒यꣳ सु॒पत्नी॒रुप॑ सेदिम । अग्ने॑ सपत्न॒दम्भ॑न॒मद॑ब्धासो॒ अदा᳚भ्यं ॥ इ॒मं ॅविष्या॑मि॒ वरु॑णस्य॒ पाशं॒ - [  ] \newline

\textbf{Pada Paata} \newline

समिति॑ । त्वा॒ । न॒ह्या॒मि॒ । पय॑सा । घृ॒तेन॑ । समिति॑ । त्वा॒ । न॒ह्या॒मि॒ । अ॒पः । ओष॑धीभि॒रित्योष॑धि - भिः॒ ॥ समिति॑ । त्वा॒ । न॒ह्या॒मि॒ । प्र॒जयेति॑ प्र - जया᳚ । अ॒हम् । अ॒द्य । सा । दी॒क्षि॒ता । स॒न॒वः॒ । वाज᳚म् । अ॒स्मे इति॑ ॥ प्रेति॑ । ए॒तु॒ । ब्रह्म॑णः । पत्नी᳚ । वेदि᳚म् । वर्णे॑न । सी॒द॒तु॒ ॥ अथ॑ । अ॒हम् । अ॒नु॒का॒मिनीत्य॑नु-का॒मिनी᳚ । स्वे । लो॒के । वि॒शै । इ॒ह ॥ सु॒प्र॒जस॒ इति॑ सु - प्र॒जसः॑ । त्वा॒ । व॒यम् । सु॒पत्नी॒रिति॑ सु - पत्नीः᳚ । उपेति॑ । से॒दि॒म॒ ॥ अग्ने᳚ । स॒प॒त्न॒दंभ॑न॒मिति॑ सपत्न - दंभ॑नम् । अद॑ब्धासः । अदा᳚भ्यम् ॥ इ॒मम् । वीति॑ । स्या॒मि॒ । वरु॑णस्य । पाश᳚म् ।  \newline


\textbf{Krama Paata} \newline

सम् त्वा᳚ । त्वा॒ न॒ह्या॒मि॒ । न॒ह्या॒मि॒ पय॑सा । पय॑सा घृ॒तेन॑ । घृ॒तेन॒ सम् । सम् त्वा᳚ । त्वा॒ न॒ह्या॒मि॒ । न॒ह्या॒म्य॒पः । अ॒प ओष॑धीभिः । ओष॑धीभि॒रित्योष॑धि - भिः॒ ॥ सम् त्वा᳚ । 
त्वा॒ न॒ह्या॒मि॒ । न॒ह्या॒मि॒ प्र॒जया᳚ । प्र॒जया॒ ऽहम् । प्र॒जयेति॑ प्र - जया᳚ । अ॒हम॒द्य । अ॒द्य सा । सा दी᳚क्षि॒ता । दी॒क्षि॒ता स॑नवः । स॒न॒वो॒ वाज᳚म् । वाज॑म॒स्मे । अ॒स्मे इत्य॒स्मे ॥ प्रैतु॑ । ए॒तु॒ ब्रह्म॑णः । ब्रह्म॑ण॒स्पत्नी᳚ । पत्नी॒ वेदि᳚म् । वेदिं॒ ॅवर्णे॑न । वर्णे॑न सीदतु । सी॒द॒त्विति॑ सीदतु ॥ अथा॒हम् । अ॒हम॑नुका॒मिनी᳚ । अ॒नु॒का॒मिनी॒ स्वे । अ॒नु॒का॒मिनीत्य॑नु - का॒मिनी᳚ । स्वे लो॒के । लो॒के वि॒शै । वि॒शा इ॒ह । इ॒हेती॒ह ॥ सु॒प्र॒जस॑ स्त्वा । सु॒प्र॒जस॒ इति॑ सु - प्र॒जसः॑ । त्वा॒ व॒यम् । व॒यꣳ सु॒पत्नीः᳚ । सु॒पत्नी॒रुप॑ । सु॒पत्नी॒रिति॑ सु - पत्नीः᳚ । उप॑ सेदिम । से॒दि॒मेति॑ सेदिम ॥ अग्ने॑ सपत्न॒दम्भ॑नम् । स॒प॒त्न॒दम्भ॑न॒मद॑ब्धासः । स॒प॒त्न॒दम्भ॑न॒मिति॑ सपत्न - दम्भ॑नम् । अद॑ब्धासो॒ अदा᳚भ्यम् । अदा᳚भ्य॒मित्यदा᳚भ्यम् ॥ इ॒मं ॅवि । वि ष्या॑मि । स्या॒मि॒ वरु॑णस्य । वरु॑णस्य॒ पाश᳚म् । पाशं॒ ॅयम् \newline

\textbf{Jatai Paata} \newline

1. सम् त्वा᳚ त्वा॒ सꣳ सम् त्वा᳚ । \newline
2. त्वा॒ न॒ह्या॒मि॒ न॒ह्या॒मि॒ त्वा॒ त्वा॒ न॒ह्या॒मि॒ । \newline
3. न॒ह्या॒मि॒ पय॑सा॒ पय॑सा नह्यामि नह्यामि॒ पय॑सा । \newline
4. पय॑सा घृ॒तेन॑ घृ॒तेन॒ पय॑सा॒ पय॑सा घृ॒तेन॑ । \newline
5. घृ॒तेन॒ सꣳ सम् घृ॒तेन॑ घृ॒तेन॒ सम् । \newline
6. सम् त्वा᳚ त्वा॒ सꣳ सम् त्वा᳚ । \newline
7. त्वा॒ न॒ह्या॒मि॒ न॒ह्या॒मि॒ त्वा॒ त्वा॒ न॒ह्या॒मि॒ । \newline
8. न॒ह्या॒ म्य॒पो॑ ऽपो न॑ह्यामि नह्या म्य॒पः । \newline
9. अ॒प ओष॑धीभि॒ रोष॑धीभि र॒पो॑ ऽप ओष॑धीभिः । \newline
10. ओष॑धीभि॒रित्योष॑धि - भिः॒ । \newline
11. सम् त्वा᳚ त्वा॒ सꣳ सम् त्वा᳚ । \newline
12. त्वा॒ न॒ह्या॒मि॒ न॒ह्या॒मि॒ त्वा॒ त्वा॒ न॒ह्या॒मि॒ । \newline
13. न॒ह्या॒मि॒ प्र॒जया᳚ प्र॒जया॑ नह्यामि नह्यामि प्र॒जया᳚ । \newline
14. प्र॒जया॒ ऽह म॒हम् प्र॒जया᳚ प्र॒जया॒ ऽहम् । \newline
15. प्र॒जयेति॑ प्र - जया᳚ । \newline
16. अ॒ह म॒द्याद्या ह म॒ह म॒द्य । \newline
17. अ॒द्य सा सा ऽद्याद्य सा । \newline
18. सा दी᳚क्षि॒ता दी᳚क्षि॒ता सा सा दी᳚क्षि॒ता । \newline
19. दी॒क्षि॒ता स॑नवः सनवो दीक्षि॒ता दी᳚क्षि॒ता स॑नवः । \newline
20. स॒न॒वो॒ वाजं॒ ॅवाजꣳ॑ सनवः सनवो॒ वाज᳚म् । \newline
21. वाज॑ म॒स्मे अ॒स्मे वाजं॒ ॅवाज॑ म॒स्मे । \newline
22. अ॒स्मे इत्य॒स्मे । \newline
23. प्रैत्वे॑तु॒ प्र प्रैतु॑ । \newline
24. ए॒तु॒ ब्रह्म॑णो॒ ब्रह्म॑ण एत्वेतु॒ ब्रह्म॑णः । \newline
25. ब्रह्म॑ण॒ स्पत्नी॒ पत्नी॒ ब्रह्म॑णो॒ ब्रह्म॑ण॒ स्पत्नी᳚ । \newline
26. पत्नी॒ वेदिं॒ ॅवेदि॒म् पत्नी॒ पत्नी॒ वेदि᳚म् । \newline
27. वेदिं॒ ॅवर्णे॑न॒ वर्णे॑न॒ वेदिं॒ ॅवेदिं॒ ॅवर्णे॑न । \newline
28. वर्णे॑न सीदतु सीदतु॒ वर्णे॑न॒ वर्णे॑न सीदतु । \newline
29. सी॒द॒त्विति॑ सीदतु । \newline
30. अथा॒ह म॒ह मथा था॒हम् । \newline
31. अ॒ह म॑नुका॒मि न्य॑नुका॒मि न्य॒ह म॒ह म॑नुका॒मिनी᳚ । \newline
32. अ॒नु॒का॒मिनी॒ स्वे स्वे॑ ऽनुका॒मि न्य॑नुका॒मिनी॒ स्वे । \newline
33. अ॒नु॒का॒मिनीत्य॑नु - का॒मिनी᳚ । \newline
34. स्वे लो॒के लो॒के स्वे स्वे लो॒के । \newline
35. लो॒के वि॒शै वि॒शै लो॒के लो॒के वि॒शै । \newline
36. वि॒शा इ॒हेह वि॒शै वि॒शा इ॒ह । \newline
37. इ॒हेती॒ह । \newline
38. सु॒प्र॒जस॑ स्त्वा त्वा सुप्र॒जसः॑ सुप्र॒जस॑ स्त्वा । \newline
39. सु॒प्र॒जस॒ इति॑ सु - प्र॒जसः॑ । \newline
40. त्वा॒ व॒यं ॅव॒यम् त्वा᳚ त्वा व॒यम् । \newline
41. व॒यꣳ सु॒पत्नीः᳚ सु॒पत्नी᳚र् व॒यं ॅव॒यꣳ सु॒पत्नीः᳚ । \newline
42. सु॒पत्नी॒ रुपोप॑ सु॒पत्नीः᳚ सु॒पत्नी॒ रुप॑ । \newline
43. सु॒पत्नी॒रिति॑ सु - पत्नीः᳚ । \newline
44. उप॑ सेदिम सेदि॒ मोपोप॑ सेदिम । \newline
45. से॒दि॒मेति॑ सेदिम । \newline
46. अग्ने॑ सपत्न॒दम्भ॑नꣳ सपत्न॒दम्भ॑न॒ मग्ने ऽग्ने॑ सपत्न॒दम्भ॑नम् । \newline
47. स॒प॒त्न॒दम्भ॑न॒ मद॑ब्धासो॒ अद॑ब्धासः सपत्न॒दम्भ॑नꣳ सपत्न॒दम्भ॑न॒ मद॑ब्धासः । \newline
48. स॒प॒त्न॒दम्भ॑न॒मिति॑ सपत्न - दम्भ॑नम् । \newline
49. अद॑ब्धासो॒ अदा᳚भ्य॒ मदा᳚भ्य॒ मद॑ब्धासो॒ अद॑ब्धासो॒ अदा᳚भ्यम् । \newline
50. अदा᳚भ्य॒मित्यदा᳚भ्यम् । \newline
51. इ॒मं ॅवि वीम मि॒मं ॅवि । \newline
52. वि ष्या॑मि स्यामि॒ वि वि ष्या॑मि । \newline
53. स्या॒मि॒ वरु॑णस्य॒ वरु॑णस्य स्यामि स्यामि॒ वरु॑णस्य । \newline
54. वरु॑णस्य॒ पाश॒म् पाशं॒ ॅवरु॑णस्य॒ वरु॑णस्य॒ पाश᳚म् । \newline
55. पाशं॒ ॅयं ॅयम् पाश॒म् पाशं॒ ॅयम् । \newline

\textbf{Ghana Paata } \newline

1. सम् त्वा᳚ त्वा॒ सꣳ सम् त्वा॑ नह्यामि नह्यामि त्वा॒ सꣳ सम् त्वा॑ नह्यामि । \newline
2. त्वा॒ न॒ह्या॒मि॒ न॒ह्या॒मि॒ त्वा॒ त्वा॒ न॒ह्या॒मि॒ पय॑सा॒ पय॑सा नह्यामि त्वा त्वा नह्यामि॒ पय॑सा । \newline
3. न॒ह्या॒मि॒ पय॑सा॒ पय॑सा नह्यामि नह्यामि॒ पय॑सा घृ॒तेन॑ घृ॒तेन॒ पय॑सा नह्यामि नह्यामि॒ पय॑सा घृ॒तेन॑ । \newline
4. पय॑सा घृ॒तेन॑ घृ॒तेन॒ पय॑सा॒ पय॑सा घृ॒तेन॒ सꣳ सम् घृ॒तेन॒ पय॑सा॒ पय॑सा घृ॒तेन॒ सम् । \newline
5. घृ॒तेन॒ सꣳ सम् घृ॒तेन॑ घृ॒तेन॒ सम् त्वा᳚ त्वा॒ सम् घृ॒तेन॑ घृ॒तेन॒ सम् त्वा᳚ । \newline
6. सम् त्वा᳚ त्वा॒ सꣳ सम् त्वा॑ नह्यामि नह्यामि त्वा॒ सꣳ सम् त्वा॑ नह्यामि । \newline
7. त्वा॒ न॒ह्या॒मि॒ न॒ह्या॒मि॒ त्वा॒ त्वा॒ न॒ह्या॒ म्य॒पो॑ ऽपो न॑ह्यामि त्वा त्वा नह्या म्य॒पः । \newline
8. न॒ह्या॒ म्य॒पो॑ ऽपो न॑ह्यामि नह्या म्य॒प ओष॑धीभि॒ रोष॑धीभि र॒पो न॑ह्यामि नह्या म्य॒प ओष॑धीभिः । \newline
9. अ॒प ओष॑धीभि॒ रोष॑धीभि र॒पो॑ ऽप ओष॑धीभिः । \newline
10. ओष॑धीभि॒रित्योष॑धि - भिः॒ । \newline
11. सम् त्वा᳚ त्वा॒ सꣳ सम् त्वा॑ नह्यामि नह्यामि त्वा॒ सꣳ सम् त्वा॑ नह्यामि । \newline
12. त्वा॒ न॒ह्या॒मि॒ न॒ह्या॒मि॒ त्वा॒ त्वा॒ न॒ह्या॒मि॒ प्र॒जया᳚ प्र॒जया॑ नह्यामि त्वा त्वा नह्यामि प्र॒जया᳚ । \newline
13. न॒ह्या॒मि॒ प्र॒जया᳚ प्र॒जया॑ नह्यामि नह्यामि प्र॒जया॒ ऽह म॒हम् प्र॒जया॑ नह्यामि नह्यामि प्र॒जया॒ ऽहम् । \newline
14. प्र॒जया॒ ऽह म॒हम् प्र॒जया᳚ प्र॒जया॒ ऽह म॒द्याद्याहम् प्र॒जया᳚ प्र॒जया॒ ऽह म॒द्य । \newline
15. प्र॒जयेति॑ प्र - जया᳚ । \newline
16. अ॒ह म॒द्याद्याह म॒ह म॒द्य सा सा ऽद्याह म॒ह म॒द्य सा । \newline
17. अ॒द्य सा सा ऽद्याद्य सा दी᳚क्षि॒ता दी᳚क्षि॒ता सा ऽद्याद्य सा दी᳚क्षि॒ता । \newline
18. सा दी᳚क्षि॒ता दी᳚क्षि॒ता सा सा दी᳚क्षि॒ता स॑नवः सनवो दीक्षि॒ता सा सा दी᳚क्षि॒ता स॑नवः । \newline
19. दी॒क्षि॒ता स॑नवः सनवो दीक्षि॒ता दी᳚क्षि॒ता स॑नवो॒ वाजं॒ ॅवाजꣳ॑ सनवो दीक्षि॒ता दी᳚क्षि॒ता स॑नवो॒ वाज᳚म् । \newline
20. स॒न॒वो॒ वाजं॒ ॅवाजꣳ॑ सनवः सनवो॒ वाज॑ म॒स्मे अ॒स्मे वाजꣳ॑ सनवः सनवो॒ वाज॑ म॒स्मे । \newline
21. वाज॑ म॒स्मे अ॒स्मे वाजं॒ ॅवाज॑ म॒स्मे । \newline
22. अ॒स्मे इत्य॒स्मे । \newline
23. प्रैत्वे॑तु॒ प्र प्रैतु॒ ब्रह्म॑णो॒ ब्रह्म॑ण एतु॒ प्र प्रैतु॒ ब्रह्म॑णः । \newline
24. ए॒तु॒ ब्रह्म॑णो॒ ब्रह्म॑ण एत्वेतु॒ ब्रह्म॑ण॒ स्पत्नी॒ पत्नी॒ ब्रह्म॑ण एत्वेतु॒ ब्रह्म॑ण॒ स्पत्नी᳚ । \newline
25. ब्रह्म॑ण॒ स्पत्नी॒ पत्नी॒ ब्रह्म॑णो॒ ब्रह्म॑ण॒ स्पत्नी॒ वेदिं॒ ॅवेदि॒म् पत्नी॒ ब्रह्म॑णो॒ ब्रह्म॑ण॒ स्पत्नी॒ वेदि᳚म् । \newline
26. पत्नी॒ वेदिं॒ ॅवेदि॒म् पत्नी॒ पत्नी॒ वेदिं॒ ॅवर्णे॑न॒ वर्णे॑न॒ वेदि॒म् पत्नी॒ पत्नी॒ वेदिं॒ ॅवर्णे॑न । \newline
27. वेदिं॒ ॅवर्णे॑न॒ वर्णे॑न॒ वेदिं॒ ॅवेदिं॒ ॅवर्णे॑न सीदतु सीदतु॒ वर्णे॑न॒ वेदिं॒ ॅवेदिं॒ ॅवर्णे॑न सीदतु । \newline
28. वर्णे॑न सीदतु सीदतु॒ वर्णे॑न॒ वर्णे॑न सीदतु । \newline
29. सी॒द॒त्विति॑ सीदतु । \newline
30. अथा॒ह म॒ह मथाथा॒ह म॑नुका॒मि न्य॑नुका॒मि न्य॒ह मथाथा॒ह म॑नुका॒मिनी᳚ । \newline
31. अ॒ह म॑नुका॒मि न्य॑नुका॒मि न्य॒ह म॒ह म॑नुका॒मिनी॒ स्वे स्वे॑ ऽनुका॒मि न्य॒ह म॒ह म॑नुका॒मिनी॒ स्वे । \newline
32. अ॒नु॒का॒मिनी॒ स्वे स्वे॑ ऽनुका॒मि न्य॑नुका॒मिनी॒ स्वे लो॒के लो॒के स्वे॑ ऽनुका॒मि न्य॑नुका॒मिनी॒ स्वे लो॒के । \newline
33. अ॒नु॒का॒मिनीत्य॑नु - का॒मिनी᳚ । \newline
34. स्वे लो॒के लो॒के स्वे स्वे लो॒के वि॒शै वि॒शै लो॒के स्वे स्वे लो॒के वि॒शै । \newline
35. लो॒के वि॒शै वि॒शै लो॒के लो॒के वि॒शा इ॒हेह वि॒शै लो॒के लो॒के वि॒शा इ॒ह । \newline
36. वि॒शा इ॒हेह वि॒शै वि॒शा इ॒ह । \newline
37. इ॒हेती॒ह । \newline
38. सु॒प्र॒जस॑ स्त्वा त्वा सुप्र॒जसः॑ सुप्र॒जस॑ स्त्वा व॒यं ॅव॒यम् त्वा॑ सुप्र॒जसः॑ सुप्र॒जस॑ स्त्वा व॒यम् । \newline
39. सु॒प्र॒जस॒ इति॑ सु - प्र॒जसः॑ । \newline
40. त्वा॒ व॒यं ॅव॒यम् त्वा᳚ त्वा व॒यꣳ सु॒पत्नीः᳚ सु॒पत्नी᳚र् व॒यम् त्वा᳚ त्वा व॒यꣳ सु॒पत्नीः᳚ । \newline
41. व॒यꣳ सु॒पत्नीः᳚ सु॒पत्नी᳚र् व॒यं ॅव॒यꣳ सु॒पत्नी॒ रुपोप॑ सु॒पत्नी᳚र् व॒यं ॅव॒यꣳ सु॒पत्नी॒ रुप॑ । \newline
42. सु॒पत्नी॒ रुपोप॑ सु॒पत्नीः᳚ सु॒पत्नी॒ रुप॑ सेदिम सेदि॒मोप॑ सु॒पत्नीः᳚ सु॒पत्नी॒ रुप॑ सेदिम । \newline
43. सु॒पत्नी॒रिति॑ सु - पत्नीः᳚ । \newline
44. उप॑ सेदिम सेदि॒ मोपोप॑ सेदिम । \newline
45. से॒दि॒मेति॑ सेदिम । \newline
46. अग्ने॑ सपत्न॒दम्भ॑नꣳ सपत्न॒दम्भ॑न॒ मग्ने ऽग्ने॑ सपत्न॒दम्भ॑न॒ मद॑ब्धासो॒ अद॑ब्धासः सपत्न॒दम्भ॑न॒ मग्ने ऽग्ने॑ सपत्न॒दम्भ॑न॒ मद॑ब्धासः । \newline
47. स॒प॒त्न॒दम्भ॑न॒ मद॑ब्धासो॒ अद॑ब्धासः सपत्न॒दम्भ॑नꣳ सपत्न॒दम्भ॑न॒ मद॑ब्धासो॒ अदा᳚भ्य॒ मदा᳚भ्य॒ मद॑ब्धासः सपत्न॒दम्भ॑नꣳ सपत्न॒दम्भ॑न॒ मद॑ब्धासो॒ अदा᳚भ्यम् । \newline
48. स॒प॒त्न॒दम्भ॑न॒मिति॑ सपत्न - दम्भ॑नम् । \newline
49. अद॑ब्धासो॒ अदा᳚भ्य॒ मदा᳚भ्य॒ मद॑ब्धासो॒ अद॑ब्धासो॒ अदा᳚भ्यम् । \newline
50. अदा᳚भ्य॒मित्यदा᳚भ्यम् । \newline
51. इ॒मं ॅवि वीम मि॒मं ॅवि ष्या॑मि स्यामि॒ वीम मि॒मं ॅवि ष्या॑मि । \newline
52. वि ष्या॑मि स्यामि॒ वि वि ष्या॑मि॒ वरु॑णस्य॒ वरु॑णस्य स्यामि॒ वि वि ष्या॑मि॒ वरु॑णस्य । \newline
53. स्या॒मि॒ वरु॑णस्य॒ वरु॑णस्य स्यामि स्यामि॒ वरु॑णस्य॒ पाश॒म् पाशं॒ ॅवरु॑णस्य स्यामि स्यामि॒ वरु॑णस्य॒ पाश᳚म् । \newline
54. वरु॑णस्य॒ पाश॒म् पाशं॒ ॅवरु॑णस्य॒ वरु॑णस्य॒ पाशं॒ ॅयं ॅयम् पाशं॒ ॅवरु॑णस्य॒ वरु॑णस्य॒ पाशं॒ ॅयम् । \newline
55. पाशं॒ ॅयं ॅयम् पाश॒म् पाशं॒ ॅय मब॑द्ध्नी॒ता ब॑द्ध्नीत॒ यम् पाश॒म् पाशं॒ ॅय मब॑द्ध्नीत । \newline
\pagebreak
\markright{ TS 3.5.6.2  \hfill https://www.vedavms.in \hfill}

\section{ TS 3.5.6.2 }

\textbf{TS 3.5.6.2 } \newline
\textbf{Samhita Paata} \newline

ॅयमब॑द्ध्नीत सवि॒ता सु॒केतः॑ । धा॒तुश्च॒ योनौ॑ सुकृ॒तस्य॑ लो॒के स्यो॒नं मे॑ स॒ह पत्या॑ करोमि ॥प्रेह्यु॒देह्यृ॒तस्य॑ वा॒मीरन्व॒ग्निस्तेऽग्रं॑ नय॒त्वदि॑ति॒र्मद्ध्यं॑ ददताꣳ रु॒द्राव॑सृष्टाऽसि यु॒वा नाम॒ मा मा॑ हिꣳसी॒र्वसु॑भ्यो रु॒द्रेभ्य॑ आदि॒त्येभ्यो॒ विश्वे᳚भ्यो वो दे॒वेभ्यः॑ प॒न्नेज॑नीर्गृह्णामि य॒ज्ञाय॑ वः प॒न्नेज॑नीः सादयामि॒ विश्व॑स्य ते॒ विश्वा॑वतो॒ वृष्णि॑यावत॒ - [  ] \newline

\textbf{Pada Paata} \newline

यम् । अब॑द्ध्नीत । स॒वि॒ता । सु॒केत॒ इति॑ सु-केतः॑ ॥ धा॒तुः । च॒ । योनौ᳚ । सु॒कृ॒तस्येति॑ सु - कृ॒तस्य॑ । लो॒के । स्यो॒नम् । मे॒ । स॒ह । पत्या᳚ । क॒रो॒मि॒ ॥ प्रेति॑ । इ॒हि॒ । उ॒देहीत्यु॑त्-एहि॑ । ऋ॒तस्य॑ । वा॒मीः । अन्विति॑ । अ॒ग्निः । ते॒ । अग्र᳚म् । न॒य॒तु॒ । अदि॑तिः । मद्ध्य᳚म् । द॒द॒ता॒म् । रु॒द्राव॑सृ॒ष्टेति॑ रु॒द्र-अ॒व॒सृ॒ष्टा॒ । अ॒सि॒ । यु॒वा । नाम॑ । मा । मा॒ । हिꣳ॒॒सीः॒ । वसु॑भ्य॒ इति॒ वसु॑-भ्यः॒ । रु॒द्रेभ्यः॑ । आ॒दि॒त्येभ्यः॑ । विश्वे᳚भ्यः । वः॒ । दे॒वेभ्यः॑ । प॒न्नेज॑नी॒रिति॑ पत् - नेज॑नीः । गृ॒ह्णा॒मि॒ । य॒ज्ञाय॑ । वः॒ । प॒न्नेज॑नी॒रिति॑ पत् - नेज॑नीः । सा॒द॒या॒मि॒ । विश्व॑स्य । ते॒ । विश्वा॑वत॒ इति॒ विश्व॑-व॒तः॒ । वृष्णि॑यावत॒ इति॒ वृष्णि॑य-व॒तः॒ ।  \newline


\textbf{Krama Paata} \newline

यमब॑द्ध्नीत । अब॑द्ध्नीत सवि॒ता । स॒वि॒ता सु॒केतः॑ । सु॒केत॒ इति॑ सु - केतः॑ । धा॒तुश्च॑ । च॒ योनौ᳚ । योनौ॑ सुकृ॒तस्य॑ । सु॒कृ॒तस्य॑ लो॒के । सु॒कृ॒तस्येति॑ सु - कृ॒तस्य॑ । लो॒के स्यो॒नम् । स्यो॒नम् मे᳚ । मे॒ स॒ह । स॒ह पत्या᳚ । पत्या॑ करोमि । क॒रो॒मीति॑ करोमि ॥ प्रेहि॑ । इ॒ह्यु॒देहि॑ । उ॒देह्यृ॒तस्य॑ । उ॒देहीत्यु॑त् - एहि॑ । ऋ॒तस्य॑ वा॒मीः । वा॒मीरनु॑ । अन्व॒ग्निः । अ॒ग्निस्ते᳚ । ते ऽग्र᳚म् । अग्रं॑ नयतु । न॒य॒त्वदि॑तिः । अदि॑ति॒र् मद्ध्य᳚म् । मद्ध्य॑म् ददताम् । द॒द॒ताꣳ॒॒ रु॒द्राव॑सृष्टा । रु॒द्राव॑सृष्टा ऽसि । रु॒द्राव॑सृ॒ष्टेति॑ रु॒द्र - अ॒व॒सृ॒ष्टा॒ । अ॒सि॒ यु॒वा । यु॒वा नाम॑ । नाम॒ मा । मा मा᳚ । मा॒ हिꣳ॒॒सीः॒ । हिꣳ॒॒सी॒र् वसु॑भ्यः । वसु॑भ्यो रु॒द्रेभ्यः॑ । वसु॑भ्य॒ इति॒ वसु॑ - भ्यः॒ । रु॒द्रेभ्य॑ आदि॒त्येभ्यः॑ । आ॒दि॒त्येभ्यो॒ विश्वे᳚भ्यः । विश्वे᳚भ्यो वः । वो॒ दे॒वेभ्यः॑ । दे॒वेभ्यः॑ प॒न्नेज॑नीः । प॒न्नेज॑नीर् गृह्णामि । प॒न्नेज॑नी॒रिति॑ पत् - नेज॑नीः । गृ॒ह्णा॒मि॒ य॒ज्ञाय॑ । य॒ज्ञाय॑ वः । वः॒ प॒न्नेज॑नीः । प॒न्नेज॑नीः सादयामि । प॒न्नेज॑नी॒रिति॑ पत् - नेज॑नीः । सा॒द॒या॒मि॒ विश्व॑स्य । विश्व॑स्य ते । ते॒ विश्वा॑वतः । विश्वा॑वतो॒ वृष्णि॑यावतः ( ) । विश्वा॑वत॒ इति॒ विश्व॑ - व॒तः॒ । वृष्णि॑यावत॒स्तव॑ । वृष्णि॑यावत॒ इति॒ वृष्णि॑य - व॒तः॒ \newline

\textbf{Jatai Paata} \newline

1. य मब॑द्ध्नी॒ता ब॑द्ध्नीत॒ यं ॅय मब॑द्ध्नीत । \newline
2. अब॑द्ध्नीत सवि॒ता स॑वि॒ता ऽब॑द्ध्नी॒ता ब॑द्ध्नीत सवि॒ता । \newline
3. स॒वि॒ता सु॒केतः॑ सु॒केतः॑ सवि॒ता स॑वि॒ता सु॒केतः॑ । \newline
4. सु॒केत॒ इति॑ सु - केतः॑ । \newline
5. धा॒तुश्च॑ च धा॒तुर् धा॒तुश्च॑ । \newline
6. च॒ योनौ॒ योनौ॑ च च॒ योनौ᳚ । \newline
7. योनौ॑ सुकृ॒तस्य॑ सुकृ॒तस्य॒ योनौ॒ योनौ॑ सुकृ॒तस्य॑ । \newline
8. सु॒कृ॒तस्य॑ लो॒के लो॒के सु॑कृ॒तस्य॑ सुकृ॒तस्य॑ लो॒के । \newline
9. सु॒कृ॒तस्येति॑ सु - कृ॒तस्य॑ । \newline
10. लो॒के स्यो॒नꣳ स्यो॒नम् ॅलो॒के लो॒के स्यो॒नम् । \newline
11. स्यो॒नम् मे॑ मे स्यो॒नꣳ स्यो॒नम् मे᳚ । \newline
12. मे॒ स॒ह स॒ह मे॑ मे स॒ह । \newline
13. स॒ह पत्या॒ पत्या॑ स॒ह स॒ह पत्या᳚ । \newline
14. पत्या॑ करोमि करोमि॒ पत्या॒ पत्या॑ करोमि । \newline
15. क॒रो॒मीति॑ करोमि । \newline
16. प्रे ही॑हि॒ प्र प्रे हि॑ । \newline
17. इ॒ह्यु॒ देह्यु॒ देही॑ही ह्यु॒देहि॑ । \newline
18. उ॒देह्यृ॒तस्य॒ र्‌तस्यो॒ देह्यु॒ देह्यृ॒तस्य॑ । \newline
19. उ॒देहीत्यु॑त् - एहि॑ । \newline
20. ऋ॒तस्य॑ वा॒मीर् वा॒मीर्. ऋ॒तस्य॒ र्तस्य॑ वा॒मीः । \newline
21. वा॒मी रन्वनु॑ वा॒मीर् वा॒मी रनु॑ । \newline
22. अन्व॒ग्नि र॒ग्नि रन्वन् व॒ग्निः । \newline
23. अ॒ग्नि स्ते॑ ते॒ ऽग्नि र॒ग्नि स्ते᳚ । \newline
24. ते ऽग्र॒ मग्र॑म् ते॒ ते ऽग्र᳚म् । \newline
25. अग्र॑म् नयतु नय॒ त्वग्र॒ मग्र॑म् नयतु । \newline
26. न॒य॒ त्वदि॑ति॒ रदि॑तिर् नयतु नय॒ त्वदि॑तिः । \newline
27. अदि॑ति॒र् मद्ध्य॒म् मद्ध्य॒ मदि॑ति॒ रदि॑ति॒र् मद्ध्य᳚म् । \newline
28. मद्ध्य॑म् ददताम् ददता॒म् मद्ध्य॒म् मद्ध्य॑म् ददताम् । \newline
29. द॒द॒ताꣳ॒॒ रु॒द्राव॑सृष्टा रु॒द्राव॑सृष्टा ददताम् ददताꣳ रु॒द्राव॑सृष्टा । \newline
30. रु॒द्राव॑सृष्टा ऽस्यसि रु॒द्राव॑सृष्टा रु॒द्राव॑सृष्टा ऽसि । \newline
31. रु॒द्राव॑सृ॒ष्टेति॑ रु॒द्र - अ॒व॒सृ॒ष्टा॒ । \newline
32. अ॒सि॒ यु॒वा यु॒वा ऽस्य॑सि यु॒वा । \newline
33. यु॒वा नाम॒ नाम॑ यु॒वा यु॒वा नाम॑ । \newline
34. नाम॒ मा मा नाम॒ नाम॒ मा । \newline
35. मा मा॑ मा॒ मा मा मा᳚ । \newline
36. मा॒ हिꣳ॒॒सी॒र्॒. हिꣳ॒॒सी॒र् मा॒ मा॒ हिꣳ॒॒सीः॒ । \newline
37. हिꣳ॒॒सी॒र् वसु॑भ्यो॒ वसु॑भ्यो हिꣳसीर्. हिꣳसी॒र् वसु॑भ्यः । \newline
38. वसु॑भ्यो रु॒द्रेभ्यो॑ रु॒द्रेभ्यो॒ वसु॑भ्यो॒ वसु॑भ्यो रु॒द्रेभ्यः॑ । \newline
39. वसु॑भ्य॒ इति॒ वसु॑ - भ्यः॒ । \newline
40. रु॒द्रेभ्य॑ आदि॒त्येभ्य॑ आदि॒त्येभ्यो॑ रु॒द्रेभ्यो॑ रु॒द्रेभ्य॑ आदि॒त्येभ्यः॑ । \newline
41. आ॒दि॒त्येभ्यो॒ विश्वे᳚भ्यो॒ विश्वे᳚भ्य आदि॒त्येभ्य॑ आदि॒त्येभ्यो॒ विश्वे᳚भ्यः । \newline
42. विश्वे᳚भ्यो वो वो॒ विश्वे᳚भ्यो॒ विश्वे᳚भ्यो वः । \newline
43. वो॒ दे॒वेभ्यो॑ दे॒वेभ्यो॑ वो वो दे॒वेभ्यः॑ । \newline
44. दे॒वेभ्यः॑ प॒न्नेज॑नीः प॒न्नेज॑नीर् दे॒वेभ्यो॑ दे॒वेभ्यः॑ प॒न्नेज॑नीः । \newline
45. प॒न्नेज॑नीर् गृह्णामि गृह्णामि प॒न्नेज॑नीः प॒न्नेज॑नीर् गृह्णामि । \newline
46. प॒न्नेज॑नी॒रिति॑ पत् - नेज॑नीः । \newline
47. गृ॒ह्णा॒मि॒ य॒ज्ञाय॑ य॒ज्ञाय॑ गृह्णामि गृह्णामि य॒ज्ञाय॑ । \newline
48. य॒ज्ञाय॑ वो वो य॒ज्ञाय॑ य॒ज्ञाय॑ वः । \newline
49. वः॒ प॒न्नेज॑नीः प॒न्नेज॑नीर् वो वः प॒न्नेज॑नीः । \newline
50. प॒न्नेज॑नीः सादयामि सादयामि प॒न्नेज॑नीः प॒न्नेज॑नीः सादयामि । \newline
51. प॒न्नेज॑नी॒रिति॑ पत् - नेज॑नीः । \newline
52. सा॒द॒या॒मि॒ विश्व॑स्य॒ विश्व॑स्य सादयामि सादयामि॒ विश्व॑स्य । \newline
53. विश्व॑स्य ते ते॒ विश्व॑स्य॒ विश्व॑स्य ते । \newline
54. ते॒ विश्वा॑वतो॒ विश्वा॑वत स्ते ते॒ विश्वा॑वतः । \newline
55. विश्वा॑वतो॒ वृष्णि॑यावतो॒ वृष्णि॑यावतो॒ विश्वा॑वतो॒ विश्वा॑वतो॒ वृष्णि॑यावतः । \newline
56. विश्वा॑वत॒ इति॒ विश्व॑ - व॒तः॒ । \newline
57. वृष्णि॑यावत॒ स्तव॒ तव॒ वृष्णि॑यावतो॒ वृष्णि॑यावत॒ स्तव॑ । \newline
58. वृष्णि॑यावत॒ इति॒ वृष्णि॑य - व॒तः॒ । \newline

\textbf{Ghana Paata } \newline

1. य मब॑द्ध्नी॒ता ब॑द्ध्नीत॒ यं ॅय मब॑द्ध्नीत सवि॒ता स॑वि॒ता ऽब॑द्ध्नीत॒ यं ॅय मब॑द्ध्नीत सवि॒ता । \newline
2. अब॑द्ध्नीत सवि॒ता स॑वि॒ता ऽब॑द्ध्नी॒ता ब॑द्ध्नीत सवि॒ता सु॒केतः॑ सु॒केतः॑ सवि॒ता ऽब॑द्ध्नी॒ता ब॑द्ध्नीत सवि॒ता सु॒केतः॑ । \newline
3. स॒वि॒ता सु॒केतः॑ सु॒केतः॑ सवि॒ता स॑वि॒ता सु॒केतः॑ । \newline
4. सु॒केत॒ इति॑ सु - केतः॑ । \newline
5. धा॒तुश्च॑ च धा॒तुर् धा॒तुश्च॒ योनौ॒ योनौ॑ च धा॒तुर् धा॒तुश्च॒ योनौ᳚ । \newline
6. च॒ योनौ॒ योनौ॑ च च॒ योनौ॑ सुकृ॒तस्य॑ सुकृ॒तस्य॒ योनौ॑ च च॒ योनौ॑ सुकृ॒तस्य॑ । \newline
7. योनौ॑ सुकृ॒तस्य॑ सुकृ॒तस्य॒ योनौ॒ योनौ॑ सुकृ॒तस्य॑ लो॒के लो॒के सु॑कृ॒तस्य॒ योनौ॒ योनौ॑ सुकृ॒तस्य॑ लो॒के । \newline
8. सु॒कृ॒तस्य॑ लो॒के लो॒के सु॑कृ॒तस्य॑ सुकृ॒तस्य॑ लो॒के स्यो॒नꣳ स्यो॒नम् ॅलो॒के सु॑कृ॒तस्य॑ सुकृ॒तस्य॑ लो॒के स्यो॒नम् । \newline
9. सु॒कृ॒तस्येति॑ सु - कृ॒तस्य॑ । \newline
10. लो॒के स्यो॒नꣳ स्यो॒नम् ॅलो॒के लो॒के स्यो॒नम् मे॑ मे स्यो॒नम् ॅलो॒के लो॒के स्यो॒नम् मे᳚ । \newline
11. स्यो॒नम् मे॑ मे स्यो॒नꣳ स्यो॒नम् मे॑ स॒ह स॒ह मे᳚ स्यो॒नꣳ स्यो॒नम् मे॑ स॒ह । \newline
12. मे॒ स॒ह स॒ह मे॑ मे स॒ह पत्या॒ पत्या॑ स॒ह मे॑ मे स॒ह पत्या᳚ । \newline
13. स॒ह पत्या॒ पत्या॑ स॒ह स॒ह पत्या॑ करोमि करोमि॒ पत्या॑ स॒ह स॒ह पत्या॑ करोमि । \newline
14. पत्या॑ करोमि करोमि॒ पत्या॒ पत्या॑ करोमि । \newline
15. क॒रो॒मीति॑ करोमि । \newline
16. प्रे ही॑हि॒ प्र प्रे ह्यु॒दे ह्यु॒देही॑हि॒ प्र प्रे ह्यु॒देहि॑ । \newline
17. इ॒ह्यु॒दे ह्यु॒देही॑ही ह्यु॒दे ह्यृ॒तस्य॒ र्‌तस्यो॒ देही॑ही ह्यु॒दे ह्यृ॒तस्य॑ । \newline
18. उ॒दे ह्यृ॒तस्य॒ र्‌तस्यो॒दे ह्यु॒दे ह्यृ॒तस्य॑ वा॒मीर् वा॒मीर्. ऋ॒तस्यो॒दे ह्यु॒दे ह्यृ॒तस्य॑ वा॒मीः । \newline
19. उ॒देहीत्यु॑त् - एहि॑ । \newline
20. ऋ॒तस्य॑ वा॒मीर् वा॒मीर्. ऋ॒तस्य॒ र्‌तस्य॑ वा॒मी रन्वनु॑ वा॒मीर्. ऋ॒तस्य॒ र्‌तस्य॑ वा॒मी रनु॑ । \newline
21. वा॒मी रन्वनु॑ वा॒मीर् वा॒मी रन्व॒ग्नि र॒ग्नि रनु॑ वा॒मीर् वा॒मी रन्व॒ग्निः । \newline
22. अन्व॒ग्नि र॒ग्नि रन्वन्व॒ग्नि स्ते॑ ते॒ ऽग्नि रन्वन्व॒ग्नि स्ते᳚ । \newline
23. अ॒ग्नि स्ते॑ ते॒ ऽग्नि र॒ग्नि स्ते ऽग्र॒ मग्र॑म् ते॒ ऽग्नि र॒ग्नि स्ते ऽग्र᳚म् । \newline
24. ते ऽग्र॒ मग्र॑म् ते॒ ते ऽग्र॑न् नयतु नय॒ त्वग्र॑म् ते॒ ते ऽग्र॑न् नयतु । \newline
25. अग्र॑न् नयतु नय॒ त्वग्र॒ मग्र॑न् नय॒ त्वदि॑ति॒ रदि॑तिर् नय॒ त्वग्र॒ मग्र॑न् नय॒ त्वदि॑तिः । \newline
26. न॒य॒ त्वदि॑ति॒ रदि॑तिर् नयतु नय॒ त्वदि॑ति॒र् मद्ध्य॒म् मद्ध्य॒ मदि॑तिर् नयतु नय॒ त्वदि॑ति॒र् मद्ध्य᳚म् । \newline
27. अदि॑ति॒र् मद्ध्य॒म् मद्ध्य॒ मदि॑ति॒ रदि॑ति॒र् मद्ध्य॑म् ददताम् ददता॒म् मद्ध्य॒ मदि॑ति॒ रदि॑ति॒र् मद्ध्य॑म् ददताम् । \newline
28. मद्ध्य॑म् ददताम् ददता॒म् मद्ध्य॒म् मद्ध्य॑म् ददताꣳ रु॒द्राव॑सृष्टा रु॒द्राव॑सृष्टा ददता॒म् मद्ध्य॒म् मद्ध्य॑म् ददताꣳ रु॒द्राव॑सृष्टा । \newline
29. द॒द॒ताꣳ॒॒ रु॒द्राव॑सृष्टा रु॒द्राव॑सृष्टा ददताम् ददताꣳ रु॒द्राव॑सृष्टा ऽस्यसि रु॒द्राव॑सृष्टा ददताम् ददताꣳ रु॒द्राव॑सृष्टा ऽसि । \newline
30. रु॒द्राव॑सृष्टा ऽस्यसि रु॒द्राव॑सृष्टा रु॒द्राव॑सृष्टा ऽसि यु॒वा यु॒वा ऽसि॑ रु॒द्राव॑सृष्टा रु॒द्राव॑सृष्टा ऽसि यु॒वा । \newline
31. रु॒द्राव॑सृ॒ष्टेति॑ रु॒द्र - अ॒व॒सृ॒ष्टा॒ । \newline
32. अ॒सि॒ यु॒वा यु॒वा ऽस्य॑सि यु॒वा नाम॒ नाम॑ यु॒वा ऽस्य॑सि यु॒वा नाम॑ । \newline
33. यु॒वा नाम॒ नाम॑ यु॒वा यु॒वा नाम॒ मा मा नाम॑ यु॒वा यु॒वा नाम॒ मा । \newline
34. नाम॒ मा मा नाम॒ नाम॒ मा मा॑ मा॒ मा नाम॒ नाम॒ मा मा᳚ । \newline
35. मा मा॑ मा॒ मा मा मा॑ हिꣳसीर्. हिꣳसीर् मा॒ मा मा मा॑ हिꣳसीः । \newline
36. मा॒ हिꣳ॒॒सी॒र्॒. हिꣳ॒॒सी॒र् मा॒ मा॒ हिꣳ॒॒सी॒र् वसु॑भ्यो॒ वसु॑भ्यो हिꣳसीर् मा मा हिꣳसी॒र् वसु॑भ्यः । \newline
37. हिꣳ॒॒सी॒र् वसु॑भ्यो॒ वसु॑भ्यो हिꣳसीर्. हिꣳसी॒र् वसु॑भ्यो रु॒द्रेभ्यो॑ रु॒द्रेभ्यो॒ वसु॑भ्यो हिꣳसीर्. हिꣳसी॒र् वसु॑भ्यो रु॒द्रेभ्यः॑ । \newline
38. वसु॑भ्यो रु॒द्रेभ्यो॑ रु॒द्रेभ्यो॒ वसु॑भ्यो॒ वसु॑भ्यो रु॒द्रेभ्य॑ आदि॒त्येभ्य॑ आदि॒त्येभ्यो॑ रु॒द्रेभ्यो॒ वसु॑भ्यो॒ वसु॑भ्यो रु॒द्रेभ्य॑ आदि॒त्येभ्यः॑ । \newline
39. वसु॑भ्य॒ इति॒ वसु॑ - भ्यः॒ । \newline
40. रु॒द्रेभ्य॑ आदि॒त्येभ्य॑ आदि॒त्येभ्यो॑ रु॒द्रेभ्यो॑ रु॒द्रेभ्य॑ आदि॒त्येभ्यो॒ विश्वे᳚भ्यो॒ विश्वे᳚भ्य आदि॒त्येभ्यो॑ रु॒द्रेभ्यो॑ रु॒द्रेभ्य॑ आदि॒त्येभ्यो॒ विश्वे᳚भ्यः । \newline
41. आ॒दि॒त्येभ्यो॒ विश्वे᳚भ्यो॒ विश्वे᳚भ्य आदि॒त्येभ्य॑ आदि॒त्येभ्यो॒ विश्वे᳚भ्यो वो वो॒ विश्वे᳚भ्य आदि॒त्येभ्य॑ आदि॒त्येभ्यो॒ विश्वे᳚भ्यो वः । \newline
42. विश्वे᳚भ्यो वो वो॒ विश्वे᳚भ्यो॒ विश्वे᳚भ्यो वो दे॒वेभ्यो॑ दे॒वेभ्यो॑ वो॒ विश्वे᳚भ्यो॒ विश्वे᳚भ्यो वो दे॒वेभ्यः॑ । \newline
43. वो॒ दे॒वेभ्यो॑ दे॒वेभ्यो॑ वो वो दे॒वेभ्यः॑ प॒न्नेज॑नीः प॒न्नेज॑नीर् दे॒वेभ्यो॑ वो वो दे॒वेभ्यः॑ प॒न्नेज॑नीः । \newline
44. दे॒वेभ्यः॑ प॒न्नेज॑नीः प॒न्नेज॑नीर् दे॒वेभ्यो॑ दे॒वेभ्यः॑ प॒न्नेज॑नीर् गृह्णामि गृह्णामि प॒न्नेज॑नीर् दे॒वेभ्यो॑ दे॒वेभ्यः॑ प॒न्नेज॑नीर् गृह्णामि । \newline
45. प॒न्नेज॑नीर् गृह्णामि गृह्णामि प॒न्नेज॑नीः प॒न्नेज॑नीर् गृह्णामि य॒ज्ञाय॑ य॒ज्ञाय॑ गृह्णामि प॒न्नेज॑नीः प॒न्नेज॑नीर् गृह्णामि य॒ज्ञाय॑ । \newline
46. प॒न्नेज॑नी॒रिति॑ पत् - नेज॑नीः । \newline
47. गृ॒ह्णा॒मि॒ य॒ज्ञाय॑ य॒ज्ञाय॑ गृह्णामि गृह्णामि य॒ज्ञाय॑ वो वो य॒ज्ञाय॑ गृह्णामि गृह्णामि य॒ज्ञाय॑ वः । \newline
48. य॒ज्ञाय॑ वो वो य॒ज्ञाय॑ य॒ज्ञाय॑ वः प॒न्नेज॑नीः प॒न्नेज॑नीर् वो य॒ज्ञाय॑ य॒ज्ञाय॑ वः प॒न्नेज॑नीः । \newline
49. वः॒ प॒न्नेज॑नीः प॒न्नेज॑नीर् वो वः प॒न्नेज॑नीः सादयामि सादयामि प॒न्नेज॑नीर् वो वः प॒न्नेज॑नीः सादयामि । \newline
50. प॒न्नेज॑नीः सादयामि सादयामि प॒न्नेज॑नीः प॒न्नेज॑नीः सादयामि॒ विश्व॑स्य॒ विश्व॑स्य सादयामि प॒न्नेज॑नीः प॒न्नेज॑नीः सादयामि॒ विश्व॑स्य । \newline
51. प॒न्नेज॑नी॒रिति॑ पत् - नेज॑नीः । \newline
52. सा॒द॒या॒मि॒ विश्व॑स्य॒ विश्व॑स्य सादयामि सादयामि॒ विश्व॑स्य ते ते॒ विश्व॑स्य सादयामि सादयामि॒ विश्व॑स्य ते । \newline
53. विश्व॑स्य ते ते॒ विश्व॑स्य॒ विश्व॑स्य ते॒ विश्वा॑वतो॒ विश्वा॑वत स्ते॒ विश्व॑स्य॒ विश्व॑स्य ते॒ विश्वा॑वतः । \newline
54. ते॒ विश्वा॑वतो॒ विश्वा॑वत स्ते ते॒ विश्वा॑वतो॒ वृष्णि॑यावतो॒ वृष्णि॑यावतो॒ विश्वा॑वत स्ते ते॒ विश्वा॑वतो॒ वृष्णि॑यावतः । \newline
55. विश्वा॑वतो॒ वृष्णि॑यावतो॒ वृष्णि॑यावतो॒ विश्वा॑वतो॒ विश्वा॑वतो॒ वृष्णि॑यावत॒ स्तव॒ तव॒ वृष्णि॑यावतो॒ विश्वा॑वतो॒ विश्वा॑वतो॒ वृष्णि॑यावत॒ स्तव॑ । \newline
56. विश्वा॑वत॒ इति॒ विश्व॑ - व॒तः॒ । \newline
57. वृष्णि॑यावत॒ स्तव॒ तव॒ वृष्णि॑यावतो॒ वृष्णि॑यावत॒ स्तवा᳚ग्ने ऽग्ने॒ तव॒ वृष्णि॑यावतो॒ वृष्णि॑यावत॒ स्तवा᳚ग्ने । \newline
58. वृष्णि॑यावत॒ इति॒ वृष्णि॑य - व॒तः॒ । \newline
\pagebreak
\markright{ TS 3.5.6.3  \hfill https://www.vedavms.in \hfill}

\section{ TS 3.5.6.3 }

\textbf{TS 3.5.6.3 } \newline
\textbf{Samhita Paata} \newline

स्तवा᳚ग्ने वा॒मीरनु॑ स॒दृंशि॒ विश्वा॒ रेताꣳ॑सि धिषी॒याऽग॑न् दे॒वान्. य॒ज्ञो नि दे॒वीर्दे॒वेभ्यो॑ य॒ज्ञ्म॑शिषन्न॒स्मिन्थ् सु॑न्व॒ति यज॑मान आ॒शिषः॒ स्वाहा॑कृताः समुद्रे॒ष्ठा ग॑न्ध॒र्वमाति॑ष्ठ॒तानु॑ । वात॑स्य॒ पत्म॑न्नि॒ड ई॑डि॒ताः ॥ \newline

\textbf{Pada Paata} \newline

तव॑ । अ॒ग्ने॒ । वा॒मीः । अन्विति॑ । स॒दृंशीति॑ सं - दृशि॑ । विश्वा᳚ । रेताꣳ॑सि । धि॒षी॒य॒ । अगन्न्॑ । दे॒वान् । य॒ज्ञ्ः । नीति॑ । दे॒वीः । दे॒वेभ्यः॑ । य॒ज्ञ्म् । अ॒शि॒ष॒न्न् । अ॒स्मिन्न् । सु॒न्व॒ति । यज॑माने । आ॒शिष॒ इत्या᳚ - शिषः॑ । स्वाहा॑कृता॒ इति॒ स्वाहा᳚ - कृ॒ताः॒ । स॒मु॒द्रे॒ष्ठा इति॑ समुद्रे - स्थाः । ग॒न्ध॒र्वम् । एति॑ । ति॒ष्ठ॒त॒ । अनु॑ ॥ वात॑स्य । पत्मन्न्॑ । इ॒डः । ई॒डि॒ताः ॥  \newline


\textbf{Krama Paata} \newline

तवा᳚ग्ने । अ॒ग्ने॒ वा॒मीः । वा॒मीरनु॑ । अनु॑ स॒न्दृशि॑ । स॒न्दृशि॒ विश्वा᳚ । स॒न्दृशीति॑ सम् - दृशि॑ । विश्वा॒ रेताꣳ॑सि । रेताꣳ॑सि धिषीय । धि॒षी॒यागन्न्॑ । अग॑न् दे॒वान् । दे॒वान्. य॒ज्ञ्ः । य॒ज्ञो नि । नि दे॒वीः । दे॒वीर् दे॒वेभ्यः॑ । दे॒वेभ्यो॑ य॒ज्ञ्म् । य॒ज्ञ्म॑शिषन्न् । अ॒शि॒ष॒न्न॒स्मिन्न् । अ॒स्मिन्थ् सु॑न्व॒ति । सु॒न्व॒ति यज॑माने । यज॑मान आ॒शिषः॑ । आ॒शिषः॒ स्वाहा॑कृताः । आ॒शिष॒ इत्या᳚ - शिषः॑ । स्वाहा॑कृताः समुद्रे॒ष्ठाः । स्वाहा॑कृता॒ इति॒ स्वाहा᳚ - कृ॒ताः॒ । स॒मु॒द्रे॒ष्ठा ग॑न्ध॒र्वम् । स॒मु॒द्रे॒ष्ठा इति॑ समुद्रे - स्थाः । ग॒न्ध॒र्वमा । आ ति॑ष्ठत । ति॒ष्ठ॒तानु॑ । अन्वित्यनु॑ ॥ वात॑स्य॒ पत्मन्न्॑ । पत्म॑न्नि॒डः । इ॒ड ई॑डि॒ताः । ई॒डि॒ता इती॑डि॒ताः । \newline

\textbf{Jatai Paata} \newline

1. तवा᳚ग्ने ऽग्ने॒ तव॒ तवा᳚ग्ने । \newline
2. अ॒ग्ने॒ वा॒मीर् वा॒मी र॑ग्ने ऽग्ने वा॒मीः । \newline
3. वा॒मी रन्वनु॑ वा॒मीर् वा॒मी रनु॑ । \newline
4. अनु॑ स॒न्दृशि॑ स॒न्दृश्यन्वनु॑ स॒न्दृशि॑ । \newline
5. स॒न्दृशि॒ विश्वा॒ विश्वा॑ स॒न्दृशि॑ स॒न्दृशि॒ विश्वा᳚ । \newline
6. स॒न्दृशीति॑ सं - दृशि॑ । \newline
7. विश्वा॒ रेताꣳ॑सि॒ रेताꣳ॑सि॒ विश्वा॒ विश्वा॒ रेताꣳ॑सि । \newline
8. रेताꣳ॑सि धिषीय धिषीय॒ रेताꣳ॑सि॒ रेताꣳ॑सि धिषीय । \newline
9. धि॒षी॒याग॒न् नग॑न् धिषीय धिषी॒यागन्न्॑ । \newline
10. अग॑न् दे॒वान् दे॒वा नग॒न् नग॑न् दे॒वान् । \newline
11. दे॒वान्. य॒ज्ञो य॒ज्ञो दे॒वान् दे॒वान्. य॒ज्ञ्ः । \newline
12. य॒ज्ञो नि नि य॒ज्ञो य॒ज्ञो नि । \newline
13. नि दे॒वीर् दे॒वीर् नि नि दे॒वीः । \newline
14. दे॒वीर् दे॒वेभ्यो॑ दे॒वेभ्यो॑ दे॒वीर् दे॒वीर् दे॒वेभ्यः॑ । \newline
15. दे॒वेभ्यो॑ य॒ज्ञ्ं ॅय॒ज्ञ्म् दे॒वेभ्यो॑ दे॒वेभ्यो॑ य॒ज्ञ्म् । \newline
16. य॒ज्ञ् म॑शिषन् नशिषन्. य॒ज्ञ्ं ॅय॒ज्ञ् म॑शिषन्न् । \newline
17. अ॒शि॒ष॒न् न॒स्मिन् न॒स्मिन् न॑शिषन् नशिषन् न॒स्मिन्न् । \newline
18. अ॒स्मिन् थ्सु॑न्व॒ति सु॑न्व॒ त्य॑स्मिन् न॒स्मिन् थ्सु॑न्व॒ति । \newline
19. सु॒न्व॒ति यज॑माने॒ यज॑माने सुन्व॒ति सु॑न्व॒ति यज॑माने । \newline
20. यज॑मान आ॒शिष॑ आ॒शिषो॒ यज॑माने॒ यज॑मान आ॒शिषः॑ । \newline
21. आ॒शिषः॒ स्वाहा॑कृताः॒ स्वाहा॑कृता आ॒शिष॑ आ॒शिषः॒ स्वाहा॑कृताः । \newline
22. आ॒शिष॒ इत्या᳚ - शिषः॑ । \newline
23. स्वाहा॑कृताः समुद्रे॒ष्ठाः स॑मुद्रे॒ष्ठाः स्वाहा॑कृताः॒ स्वाहा॑कृताः समुद्रे॒ष्ठाः । \newline
24. स्वाहा॑कृता॒ इति॒ स्वाहा᳚ - कृ॒ताः॒ । \newline
25. स॒मु॒द्रे॒ष्ठा ग॑न्ध॒र्वम् ग॑न्ध॒र्वꣳ स॑मुद्रे॒ष्ठाः स॑मुद्रे॒ष्ठा ग॑न्ध॒र्वम् । \newline
26. स॒मु॒द्रे॒ष्ठा इति॑ समुद्रे - स्थाः । \newline
27. ग॒न्ध॒र्व मा ग॑न्ध॒र्वम् ग॑न्ध॒र्व मा । \newline
28. आ ति॑ष्ठत तिष्ठ॒ता ति॑ष्ठत । \newline
29. ति॒ष्ठ॒तान्वनु॑ तिष्ठत तिष्ठ॒तानु॑ । \newline
30. अन्वित्यनु॑ । \newline
31. वात॑स्य॒ पत्म॒न् पत्म॒न्॒. वात॑स्य॒ वात॑स्य॒ पत्मन्न्॑ । \newline
32. पत्म॑न् नि॒ड इ॒ड स्पत्म॒न् पत्म॑न् नि॒डः । \newline
33. इ॒ड ई॑डि॒ता ई॑डि॒ता इ॒ड इ॒ड ई॑डि॒ताः । \newline
34. ई॒डि॒ता इती॑डि॒ताः । \newline

\textbf{Ghana Paata } \newline

1. तवा᳚ग्ने ऽग्ने॒ तव॒ तवा᳚ग्ने वा॒मीर् वा॒मी र॑ग्ने॒ तव॒ तवा᳚ग्ने वा॒मीः । \newline
2. अ॒ग्ने॒ वा॒मीर् वा॒मी र॑ग्ने ऽग्ने वा॒मी रन्वनु॑ वा॒मी र॑ग्ने ऽग्ने वा॒मी रनु॑ । \newline
3. वा॒मी रन्वनु॑ वा॒मीर् वा॒मी रनु॑ स॒न्दृशि॑ स॒न्दृश्यनु॑ वा॒मीर् वा॒मी रनु॑ स॒न्दृशि॑ । \newline
4. अनु॑ स॒न्दृशि॑ स॒न्दृ श्यन्वनु॑ स॒न्दृशि॒ विश्वा॒ विश्वा॑ स॒न्दृ श्यन्वनु॑ स॒न्दृशि॒ विश्वा᳚ । \newline
5. स॒न्दृशि॒ विश्वा॒ विश्वा॑ स॒न्दृशि॑ स॒न्दृशि॒ विश्वा॒ रेताꣳ॑सि॒ रेताꣳ॑सि॒ विश्वा॑ स॒न्दृशि॑ स॒न्दृशि॒ विश्वा॒ रेताꣳ॑सि । \newline
6. स॒न्दृशीति॑ सं - दृशि॑ । \newline
7. विश्वा॒ रेताꣳ॑सि॒ रेताꣳ॑सि॒ विश्वा॒ विश्वा॒ रेताꣳ॑सि धिषीय धिषीय॒ रेताꣳ॑सि॒ विश्वा॒ विश्वा॒ रेताꣳ॑सि धिषीय । \newline
8. रेताꣳ॑सि धिषीय धिषीय॒ रेताꣳ॑सि॒ रेताꣳ॑सि धिषी॒याग॒न् नग॑न् धिषीय॒ रेताꣳ॑सि॒ रेताꣳ॑सि धिषी॒यागन्न्॑ । \newline
9. धि॒षी॒याग॒न् नग॑न् धिषीय धिषी॒याग॑न् दे॒वान् दे॒वा नग॑न् धिषीय धिषी॒याग॑न् दे॒वान् । \newline
10. अग॑न् दे॒वान् दे॒वा नग॒न् नग॑न् दे॒वान्. य॒ज्ञो य॒ज्ञो दे॒वा नग॒न् नग॑न् दे॒वान्. य॒ज्ञ्ः । \newline
11. दे॒वान्. य॒ज्ञो य॒ज्ञो दे॒वान् दे॒वान्. य॒ज्ञो नि नि य॒ज्ञो दे॒वान् दे॒वान्. य॒ज्ञो नि । \newline
12. य॒ज्ञो नि नि य॒ज्ञो य॒ज्ञो नि दे॒वीर् दे॒वीर् नि य॒ज्ञो य॒ज्ञो नि दे॒वीः । \newline
13. नि दे॒वीर् दे॒वीर् नि नि दे॒वीर् दे॒वेभ्यो॑ दे॒वेभ्यो॑ दे॒वीर् नि नि दे॒वीर् दे॒वेभ्यः॑ । \newline
14. दे॒वीर् दे॒वेभ्यो॑ दे॒वेभ्यो॑ दे॒वीर् दे॒वीर् दे॒वेभ्यो॑ य॒ज्ञ्ं ॅय॒ज्ञ्म् दे॒वेभ्यो॑ दे॒वीर् दे॒वीर् दे॒वेभ्यो॑ य॒ज्ञ्म् । \newline
15. दे॒वेभ्यो॑ य॒ज्ञ्ं ॅय॒ज्ञ्म् दे॒वेभ्यो॑ दे॒वेभ्यो॑ य॒ज्ञ् म॑शिषन् नशिषन्. य॒ज्ञ्म् दे॒वेभ्यो॑ दे॒वेभ्यो॑ य॒ज्ञ् म॑शिषन्न् । \newline
16. य॒ज्ञ् म॑शिषन् नशिषन्. य॒ज्ञ्ं ॅय॒ज्ञ् म॑शिषन् न॒स्मिन् न॒स्मिन् न॑शिषन्. य॒ज्ञ्ं ॅय॒ज्ञ् म॑शिषन् न॒स्मिन्न् । \newline
17. अ॒शि॒ष॒न् न॒स्मिन् न॒स्मिन् न॑शिषन् नशिषन् न॒स्मिन् थ्सु॑न्व॒ति सु॑न्व॒ त्य॑स्मिन् न॑शिषन् नशिषन् न॒स्मिन् थ्सु॑न्व॒ति । \newline
18. अ॒स्मिन् थ्सु॑न्व॒ति सु॑न्व॒ त्य॑स्मिन् न॒स्मिन् थ्सु॑न्व॒ति यज॑माने॒ यज॑माने सुन्व॒ त्य॑स्मिन् न॒स्मिन् थ्सु॑न्व॒ति यज॑माने । \newline
19. सु॒न्व॒ति यज॑माने॒ यज॑माने सुन्व॒ति सु॑न्व॒ति यज॑मान आ॒शिष॑ आ॒शिषो॒ यज॑माने सुन्व॒ति सु॑न्व॒ति यज॑मान आ॒शिषः॑ । \newline
20. यज॑मान आ॒शिष॑ आ॒शिषो॒ यज॑माने॒ यज॑मान आ॒शिषः॒ स्वाहा॑कृताः॒ स्वाहा॑कृता आ॒शिषो॒ यज॑माने॒ यज॑मान आ॒शिषः॒ स्वाहा॑कृताः । \newline
21. आ॒शिषः॒ स्वाहा॑कृताः॒ स्वाहा॑कृता आ॒शिष॑ आ॒शिषः॒ स्वाहा॑कृताः समुद्रे॒ष्ठाः स॑मुद्रे॒ष्ठाः स्वाहा॑कृता आ॒शिष॑ आ॒शिषः॒ स्वाहा॑कृताः समुद्रे॒ष्ठाः । \newline
22. आ॒शिष॒ इत्या᳚ - शिषः॑ । \newline
23. स्वाहा॑कृताः समुद्रे॒ष्ठाः स॑मुद्रे॒ष्ठाः स्वाहा॑कृताः॒ स्वाहा॑कृताः समुद्रे॒ष्ठा ग॑न्ध॒र्वम् ग॑न्ध॒र्वꣳ स॑मुद्रे॒ष्ठाः स्वाहा॑कृताः॒ स्वाहा॑कृताः समुद्रे॒ष्ठा ग॑न्ध॒र्वम् । \newline
24. स्वाहा॑कृता॒ इति॒ स्वाहा᳚ - कृ॒ताः॒ । \newline
25. स॒मु॒द्रे॒ष्ठा ग॑न्ध॒र्वम् ग॑न्ध॒र्वꣳ स॑मुद्रे॒ष्ठाः स॑मुद्रे॒ष्ठा ग॑न्ध॒र्व मा ग॑न्ध॒र्वꣳ स॑मुद्रे॒ष्ठाः स॑मुद्रे॒ष्ठा ग॑न्ध॒र्व मा । \newline
26. स॒मु॒द्रे॒ष्ठा इति॑ समुद्रे - स्थाः । \newline
27. ग॒न्ध॒र्व मा ग॑न्ध॒र्वम् ग॑न्ध॒र्व मा ति॑ष्ठत तिष्ठ॒ता ग॑न्ध॒र्वम् ग॑न्ध॒र्व मा ति॑ष्ठत । \newline
28. आ ति॑ष्ठत तिष्ठ॒ता ति॑ष्ठ॒ता न्वनु॑ तिष्ठ॒ता ति॑ष्ठ॒तानु॑ । \newline
29. ति॒ष्ठ॒ता न्वनु॑ तिष्ठत तिष्ठ॒तानु॑ । \newline
30. अन्वित्यनु॑ । \newline
31. वात॑स्य॒ पत्म॒न् पत्म॒न्॒. वात॑स्य॒ वात॑स्य॒ पत्म॑न् नि॒ड इ॒ड स्पत्म॒न्॒. वात॑स्य॒ वात॑स्य॒ पत्म॑न् नि॒डः । \newline
32. पत्म॑न् नि॒ड इ॒ड स्पत्म॒न् पत्म॑न् नि॒ड ई॑डि॒ता ई॑डि॒ता इ॒ड स्पत्म॒न् पत्म॑न् नि॒ड ई॑डि॒ताः । \newline
33. इ॒ड ई॑डि॒ता ई॑डि॒ता इ॒ड इ॒ड ई॑डि॒ताः । \newline
34. ई॒डि॒ता इती॑डि॒ताः । \newline
\pagebreak
\markright{ TS 3.5.7.1  \hfill https://www.vedavms.in \hfill}

\section{ TS 3.5.7.1 }

\textbf{TS 3.5.7.1 } \newline
\textbf{Samhita Paata} \newline

व॒ष॒ट्का॒रो वै गा॑यत्रि॒यै शिरो᳚ऽछिन॒त् तस्यै॒ रसः॒ परा॑ऽपत॒थ् स पृ॑थि॒वीं प्रावि॑श॒थ्स ख॑दि॒रो॑ऽभव॒द्यस्य॑ खादि॒रः स्रु॒वो भव॑ति॒ छन्द॑सामे॒व रसे॒नाव॑ द्यति॒ सर॑सा अ॒स्याऽऽ*हु॑तयो भवन्ति तृ॒तीय॑स्यामि॒तो दि॒वि सोम॑ आसी॒त् तं गाय॒त्र्याऽ ह॑र॒त् तस्य॑ प॒र्णम॑च्छिद्यत॒ तत् प॒र्णो॑ऽभव॒त् तत् प॒र्णस्य॑ पर्ण॒त्वं ॅयस्य॑ पर्ण॒मयी॑ जु॒हू - [  ] \newline

\textbf{Pada Paata} \newline

व॒ष॒ट्का॒र इति॑ वषट् - का॒रः । वै । गा॒य॒त्रि॒यै । शिरः॑ । अ॒च्छि॒न॒त् । तस्यै᳚ । रसः॑ । परेति॑ । अ॒प॒त॒त् । सः । पृ॒थि॒वीम् । प्रेति॑ । अ॒वि॒श॒त् । सः । ख॒दि॒रः । अ॒भ॒व॒त् । यस्य॑ । खा॒दि॒रः । स्रु॒वः । भव॑ति । छन्द॑साम् । ए॒व । रसे॑न । अवेति॑ । द्य॒ति॒ । सर॑सा॒ इति॒ स - र॒साः॒ । अ॒स्य॒ । आहु॑तय॒ इत्या - हु॒त॒यः॒ । भ॒व॒न्ति॒ । तृ॒तीय॑स्याम् । इ॒तः । दि॒वि । सोमः॑ । आ॒सी॒त् । तम् । गा॒य॒त्री । एति॑ । अ॒ह॒र॒त् । तस्य॑ । प॒र्णम् । अ॒च्छि॒द्य॒त॒ । तत् । प॒र्णः । अ॒भ॒व॒त् । तत् । प॒र्णस्य॑ । प॒र्ण॒त्वमिति॑ पर्ण - त्वम् । यस्य॑ । प॒र्ण॒मयीति॑ पर्ण - मयी᳚ । जु॒हूः ।  \newline


\textbf{Krama Paata} \newline

व॒ष॒ट्का॒रो वै । व॒ष॒ट्का॒र इति॑ वषट् - का॒रः । वै गा॑यत्रि॒यै । गा॒य॒त्रि॒यै शिरः॑ । शिरो᳚ ऽच्छिनत् । अ॒च्छि॒न॒त् तस्यै᳚ । तस्यै॒ रसः॑ । रसः॒ परा᳚ । परा॑ ऽपतत् । अ॒प॒त॒थ् सः । स पृ॑थि॒वीम् । पृ॒थि॒वीम् प्र । प्रावि॑शत् । अ॒वि॒श॒थ् सः । स ख॑दि॒रः । ख॒दि॒रो॑ ऽभवत् । अ॒भ॒व॒द् यस्य॑ । यस्य॑ खादि॒रः । खा॒दि॒रः स्रु॒वः । स्रु॒वो भव॑ति । भव॑ति॒ छन्द॑साम् । छन्द॑सामे॒व । ए॒व रसे॑न । रसे॒नाव॑ । अव॑ द्यति । द्य॒ति॒ सर॑साः । सर॑सा अस्य । सर॑सा॒ इति॒ स - र॒साः॒ । अ॒स्याहु॑तयः । आहु॑तयो भवन्ति । आहु॑तय॒ इत्या - हु॒त॒यः॒ । भ॒व॒न्ति॒ तृ॒तीय॑स्याम् । तृ॒तीय॑स्यामि॒तः । इ॒तो दि॒वि । दि॒वि सोमः॑ । सोम॑ आसीत् । आ॒सी॒त् तम् । तम् गा॑य॒त्री । गा॒य॒त्र्या । आ ऽह॑रत् । अ॒ह॒र॒त् तस्य॑ । तस्य॑ प॒र्णम् । प॒र्णम॑च्छिद्यत । अ॒च्छि॒द्य॒त॒ तत् । तत् प॒र्णः । प॒र्णो॑ ऽभवत् । अ॒भ॒व॒त् तत् । तत् प॒र्णस्य॑ । प॒र्णस्य॑ पर्ण॒त्वम् । प॒र्ण॒त्वं ॅयस्य॑ । प॒र्ण॒त्वमिति॑ पर्ण - त्वम् । यस्य॑ पर्ण॒मयी᳚ । प॒र्ण॒मयी॑ जु॒हूः । प॒र्ण॒मयीति॑ पर्ण - मयी᳚ । जु॒हूर् भव॑ति \newline

\textbf{Jatai Paata} \newline

1. व॒ष॒ट्का॒रो वै वै व॑षट्का॒रो व॑षट्का॒रो वै । \newline
2. व॒ष॒ट्का॒र इति॑ वषट् - का॒रः । \newline
3. वै गा॑यत्रि॒यै गा॑यत्रि॒यै वै वै गा॑यत्रि॒यै । \newline
4. गा॒य॒त्रि॒यै शिरः॒ शिरो॑ गायत्रि॒यै गा॑यत्रि॒यै शिरः॑ । \newline
5. शिरो᳚ ऽच्छिन दच्छिन॒ च्छिरः॒ शिरो᳚ ऽच्छिनत् । \newline
6. अ॒च्छि॒न॒त् तस्यै॒ तस्या॑ अच्छिन दच्छिन॒त् तस्यै᳚ । \newline
7. तस्यै॒ रसो॒ रस॒ स्तस्यै॒ तस्यै॒ रसः॑ । \newline
8. रसः॒ परा॒ परा॒ रसो॒ रसः॒ परा᳚ । \newline
9. परा॑ ऽपत दपत॒त् परा॒ परा॑ ऽपतत् । \newline
10. अ॒प॒त॒थ् स सो॑ ऽपत दपत॒थ् सः । \newline
11. स पृ॑थि॒वीम् पृ॑थि॒वीꣳ स स पृ॑थि॒वीम् । \newline
12. पृ॒थि॒वीम् प्र प्र पृ॑थि॒वीम् पृ॑थि॒वीम् प्र । \newline
13. प्रावि॑श दविश॒त् प्र प्रावि॑शत् । \newline
14. अ॒वि॒श॒थ् स सो॑ ऽविश दविश॒थ् सः । \newline
15. स ख॑दि॒रः ख॑दि॒रः स स ख॑दि॒रः । \newline
16. ख॒दि॒रो॑ ऽभव दभवत् खदि॒रः ख॑दि॒रो॑ ऽभवत् । \newline
17. अ॒भ॒व॒द् यस्य॒ यस्या॑भव दभव॒द् यस्य॑ । \newline
18. यस्य॑ खादि॒रः खा॑दि॒रो यस्य॒ यस्य॑ खादि॒रः । \newline
19. खा॒दि॒रः स्रु॒वः स्रु॒वः खा॑दि॒रः खा॑दि॒रः स्रु॒वः । \newline
20. स्रु॒वो भव॑ति॒ भव॑ति स्रु॒वः स्रु॒वो भव॑ति । \newline
21. भव॑ति॒ छन्द॑सा॒म् छन्द॑सा॒म् भव॑ति॒ भव॑ति॒ छन्द॑साम् । \newline
22. छन्द॑सा मे॒वैव छन्द॑सा॒म् छन्द॑सा मे॒व । \newline
23. ए॒व रसे॑न॒ रसे॑नै॒वैव रसे॑न । \newline
24. रसे॒नावाव॒ रसे॑न॒ रसे॒नाव॑ । \newline
25. अव॑ द्यति द्य॒त्यवाव॑ द्यति । \newline
26. द्य॒ति॒ सर॑साः॒ सर॑सा द्यति द्यति॒ सर॑साः । \newline
27. सर॑सा अस्यास्य॒ सर॑साः॒ सर॑सा अस्य । \newline
28. सर॑सा॒ इति॒ स - र॒साः॒ । \newline
29. अ॒स्या हु॑तय॒ आहु॑तयो ऽस्या॒ स्याहु॑तयः । \newline
30. आहु॑तयो भवन्ति भव॒ न्त्याहु॑तय॒ आहु॑तयो भवन्ति । \newline
31. आहु॑तय॒ इत्या - हु॒त॒यः॒ । \newline
32. भ॒व॒न्ति॒ तृ॒तीय॑स्याम् तृ॒तीय॑स्याम् भवन्ति भवन्ति तृ॒तीय॑स्याम् । \newline
33. तृ॒तीय॑स्या मि॒त इ॒तस्तृ॒तीय॑स्याम् तृ॒तीय॑स्या मि॒तः । \newline
34. इ॒तो दि॒वि दि॒वीत इ॒तो दि॒वि । \newline
35. दि॒वि सोमः॒ सोमो॑ दि॒वि दि॒वि सोमः॑ । \newline
36. सोम॑ आसी दासी॒थ् सोमः॒ सोम॑ आसीत् । \newline
37. आ॒सी॒त् तम् त मा॑सी दासी॒त् तम् । \newline
38. तम् गा॑य॒त्री गा॑य॒त्री तम् तम् गा॑य॒त्री । \newline
39. गा॒य॒त्र्या गा॑य॒त्री गा॑य॒त्र्या । \newline
40. आ ऽह॑र दहर॒दा ऽह॑रत् । \newline
41. अ॒ह॒र॒त् तस्य॒ तस्या॑ हर दहर॒त् तस्य॑ । \newline
42. तस्य॑ प॒र्णम् प॒र्णम् तस्य॒ तस्य॑ प॒र्णम् । \newline
43. प॒र्ण म॑च्छिद्यता च्छिद्यत प॒र्णम् प॒र्ण म॑च्छिद्यत । \newline
44. अ॒च्छि॒द्य॒त॒ तत् तद॑च्छिद्यता च्छिद्यत॒ तत् । \newline
45. तत् प॒र्णः प॒र्ण स्तत् तत् प॒र्णः । \newline
46. प॒र्णो॑ ऽभवद भवत् प॒र्णः प॒र्णो॑ ऽभवत् । \newline
47. अ॒भ॒व॒त् तत् तद॑भव दभव॒त् तत् । \newline
48. तत् प॒र्णस्य॑ प॒र्णस्य॒ तत् तत् प॒र्णस्य॑ । \newline
49. प॒र्णस्य॑ पर्ण॒त्वम् प॑र्ण॒त्वम् प॒र्णस्य॑ प॒र्णस्य॑ पर्ण॒त्वम् । \newline
50. प॒र्ण॒त्वं ॅयस्य॒ यस्य॑ पर्ण॒त्वम् प॑र्ण॒त्वं ॅयस्य॑ । \newline
51. प॒र्ण॒त्वमिति॑ पर्ण - त्वम् । \newline
52. यस्य॑ पर्ण॒मयी॑ पर्ण॒मयी॒ यस्य॒ यस्य॑ पर्ण॒मयी᳚ । \newline
53. प॒र्ण॒मयी॑ जु॒हूर् जु॒हूः प॑र्ण॒मयी॑ पर्ण॒मयी॑ जु॒हूः । \newline
54. प॒र्ण॒मयीति॑ पर्ण - मयी᳚ । \newline
55. जु॒हूर् भव॑ति॒ भव॑ति जु॒हूर् जु॒हूर् भव॑ति । \newline

\textbf{Ghana Paata } \newline

1. व॒ष॒ट्का॒रो वै वै व॑षट्का॒रो व॑षट्का॒रो वै गा॑यत्रि॒यै गा॑यत्रि॒यै वै व॑षट्का॒रो व॑षट्का॒रो वै गा॑यत्रि॒यै । \newline
2. व॒ष॒ट्का॒र इति॑ वषट् - का॒रः । \newline
3. वै गा॑यत्रि॒यै गा॑यत्रि॒यै वै वै गा॑यत्रि॒यै शिरः॒ शिरो॑ गायत्रि॒यै वै वै गा॑यत्रि॒यै शिरः॑ । \newline
4. गा॒य॒त्रि॒यै शिरः॒ शिरो॑ गायत्रि॒यै गा॑यत्रि॒यै शिरो᳚ ऽच्छिन दच्छिन॒ च्छिरो॑ गायत्रि॒यै गा॑यत्रि॒यै शिरो᳚ ऽच्छिनत् । \newline
5. शिरो᳚ ऽच्छिन दच्छिन॒ च्छिरः॒ शिरो᳚ ऽच्छिन॒त् तस्यै॒ तस्या॑ अच्छिन॒ च्छिरः॒ शिरो᳚ ऽच्छिन॒त् तस्यै᳚ । \newline
6. अ॒च्छि॒न॒त् तस्यै॒ तस्या॑ अच्छिन दच्छिन॒त् तस्यै॒ रसो॒ रस॒ स्तस्या॑ अच्छिन दच्छिन॒त् तस्यै॒ रसः॑ । \newline
7. तस्यै॒ रसो॒ रस॒ स्तस्यै॒ तस्यै॒ रसः॒ परा॒ परा॒ रस॒ स्तस्यै॒ तस्यै॒ रसः॒ परा᳚ । \newline
8. रसः॒ परा॒ परा॒ रसो॒ रसः॒ परा॑ ऽपत दपत॒त् परा॒ रसो॒ रसः॒ परा॑ ऽपतत् । \newline
9. परा॑ ऽपत दपत॒त् परा॒ परा॑ ऽपत॒थ् स सो॑ ऽपत॒त् परा॒ परा॑ ऽपत॒थ् सः । \newline
10. अ॒प॒त॒थ् स सो॑ ऽपत दपत॒थ् स पृ॑थि॒वीम् पृ॑थि॒वीꣳ सो॑ ऽपत दपत॒थ् स पृ॑थि॒वीम् । \newline
11. स पृ॑थि॒वीम् पृ॑थि॒वीꣳ स स पृ॑थि॒वीम् प्र प्र पृ॑थि॒वीꣳ स स पृ॑थि॒वीम् प्र । \newline
12. पृ॒थि॒वीम् प्र प्र पृ॑थि॒वीम् पृ॑थि॒वीम् प्रावि॑श दविश॒त् प्र पृ॑थि॒वीम् पृ॑थि॒वीम् प्रावि॑शत् । \newline
13. प्रावि॑श दविश॒त् प्र प्रावि॑श॒थ् स सो॑ ऽविश॒त् प्र प्रावि॑श॒थ् सः । \newline
14. अ॒वि॒श॒थ् स सो॑ ऽविश दविश॒थ् स ख॑दि॒रः ख॑दि॒रः सो॑ ऽविश दविश॒थ् स ख॑दि॒रः । \newline
15. स ख॑दि॒रः ख॑दि॒रः स स ख॑दि॒रो॑ ऽभव दभवत् खदि॒रः स स ख॑दि॒रो॑ ऽभवत् । \newline
16. ख॒दि॒रो॑ ऽभव दभवत् खदि॒रः ख॑दि॒रो॑ ऽभव॒द् यस्य॒ यस्या॑ भवत् खदि॒रः ख॑दि॒रो॑ ऽभव॒द् यस्य॑ । \newline
17. अ॒भ॒व॒द् यस्य॒ यस्या॑ भव दभव॒द् यस्य॑ खादि॒रः खा॑दि॒रो यस्या॑ भव दभव॒द् यस्य॑ खादि॒रः । \newline
18. यस्य॑ खादि॒रः खा॑दि॒रो यस्य॒ यस्य॑ खादि॒रः स्रु॒वः स्रु॒वः खा॑दि॒रो यस्य॒ यस्य॑ खादि॒रः स्रु॒वः । \newline
19. खा॒दि॒रः स्रु॒वः स्रु॒वः खा॑दि॒रः खा॑दि॒रः स्रु॒वो भव॑ति॒ भव॑ति स्रु॒वः खा॑दि॒रः खा॑दि॒रः स्रु॒वो भव॑ति । \newline
20. स्रु॒वो भव॑ति॒ भव॑ति स्रु॒वः स्रु॒वो भव॑ति॒ छन्द॑सा॒म् छन्द॑सा॒म् भव॑ति स्रु॒वः स्रु॒वो भव॑ति॒ छन्द॑साम् । \newline
21. भव॑ति॒ छन्द॑सा॒म् छन्द॑सा॒म् भव॑ति॒ भव॑ति॒ छन्द॑सा मे॒वैव छन्द॑सा॒म् भव॑ति॒ भव॑ति॒ छन्द॑सा मे॒व । \newline
22. छन्द॑सा मे॒वैव छन्द॑सा॒म् छन्द॑सा मे॒व रसे॑न॒ रसे॑नै॒व छन्द॑सा॒म् छन्द॑सा मे॒व रसे॑न । \newline
23. ए॒व रसे॑न॒ रसे॑नै॒वैव रसे॒नावाव॒ रसे॑नै॒वैव रसे॒नाव॑ । \newline
24. रसे॒नावाव॒ रसे॑न॒ रसे॒नाव॑ द्यति द्य॒त्यव॒ रसे॑न॒ रसे॒नाव॑ द्यति । \newline
25. अव॑ द्यति द्य॒त्यवाव॑ द्यति॒ सर॑साः॒ सर॑सा द्य॒त्यवाव॑ द्यति॒ सर॑साः । \newline
26. द्य॒ति॒ सर॑साः॒ सर॑सा द्यति द्यति॒ सर॑सा अस्यास्य॒ सर॑सा द्यति द्यति॒ सर॑सा अस्य । \newline
27. सर॑सा अस्यास्य॒ सर॑साः॒ सर॑सा अ॒स्या हु॑तय॒ आहु॑तयो ऽस्य॒ सर॑साः॒ सर॑सा अ॒स्या हु॑तयः । \newline
28. सर॑सा॒ इति॒ स - र॒साः॒ । \newline
29. अ॒स्या हु॑तय॒ आहु॑तयो ऽस्या॒स्या हु॑तयो भवन्ति भव॒ न्त्याहु॑तयो ऽस्या॒स्या हु॑तयो भवन्ति । \newline
30. आहु॑तयो भवन्ति भव॒ न्त्याहु॑तय॒ आहु॑तयो भवन्ति तृ॒तीय॑स्याम् तृ॒तीय॑स्याम् भव॒ न्त्याहु॑तय॒ आहु॑तयो भवन्ति तृ॒तीय॑स्याम् । \newline
31. आहु॑तय॒ इत्या - हु॒त॒यः॒ । \newline
32. भ॒व॒न्ति॒ तृ॒तीय॑स्याम् तृ॒तीय॑स्याम् भवन्ति भवन्ति तृ॒तीय॑स्या मि॒त इ॒त स्तृ॒तीय॑स्याम् भवन्ति भवन्ति तृ॒तीय॑स्या मि॒तः । \newline
33. तृ॒तीय॑स्या मि॒त इ॒त स्तृ॒तीय॑स्याम् तृ॒तीय॑स्या मि॒तो दि॒वि दि॒वीत स्तृ॒तीय॑स्याम् तृ॒तीय॑स्या मि॒तो दि॒वि । \newline
34. इ॒तो दि॒वि दि॒वीत इ॒तो दि॒वि सोमः॒ सोमो॑ दि॒वीत इ॒तो दि॒वि सोमः॑ । \newline
35. दि॒वि सोमः॒ सोमो॑ दि॒वि दि॒वि सोम॑ आसी दासी॒थ् सोमो॑ दि॒वि दि॒वि सोम॑ आसीत् । \newline
36. सोम॑ आसी दासी॒थ् सोमः॒ सोम॑ आसी॒त् तम् त मा॑सी॒थ् सोमः॒ सोम॑ आसी॒त् तम् । \newline
37. आ॒सी॒त् तम् त मा॑सी दासी॒त् तम् गा॑य॒त्री गा॑य॒त्री त मा॑सी दासी॒त् तम् गा॑य॒त्री । \newline
38. तम् गा॑य॒त्री गा॑य॒त्री तम् तम् गा॑य॒त्र्या गा॑य॒त्री तम् तम् गा॑य॒त्र्या । \newline
39. गा॒य॒त्र्या गा॑य॒त्री गा॑य॒त्र्या ऽह॑र दहर॒दा गा॑य॒त्री गा॑य॒त्र्या ऽह॑रत् । \newline
40. आ ऽह॑र दहर॒ दा ऽह॑र॒त् तस्य॒ तस्या॑ हर॒ दा ऽह॑र॒त् तस्य॑ । \newline
41. अ॒ह॒र॒त् तस्य॒ तस्या॑ हर दहर॒त् तस्य॑ प॒र्णम् प॒र्णम् तस्या॑ हर दहर॒त् तस्य॑ प॒र्णम् । \newline
42. तस्य॑ प॒र्णम् प॒र्णम् तस्य॒ तस्य॑ प॒र्ण म॑च्छिद्यता च्छिद्यत प॒र्णम् तस्य॒ तस्य॑ प॒र्ण म॑च्छिद्यत । \newline
43. प॒र्ण म॑च्छिद्यता च्छिद्यत प॒र्णम् प॒र्ण म॑च्छिद्यत॒ तत् तद॑च्छिद्यत प॒र्णम् प॒र्ण म॑च्छिद्यत॒ तत् । \newline
44. अ॒च्छि॒द्य॒त॒ तत् तद॑च्छिद्यता च्छिद्यत॒ तत् प॒र्णः प॒र्ण स्त द॑च्छिद्यताच् छिद्यत॒ तत् प॒र्णः । \newline
45. तत् प॒र्णः प॒र्ण स्तत् तत् प॒र्णो॑ ऽभव दभवत् प॒र्ण स्तत् तत् प॒र्णो॑ ऽभवत् । \newline
46. प॒र्णो॑ ऽभव दभवत् प॒र्णः प॒र्णो॑ ऽभव॒त् तत् तद॑भवत् प॒र्णः प॒र्णो॑ ऽभव॒त् तत् । \newline
47. अ॒भ॒व॒त् तत् तद॑भव दभव॒त् तत् प॒र्णस्य॑ प॒र्णस्य॒ तद॑भव दभव॒त् तत् प॒र्णस्य॑ । \newline
48. तत् प॒र्णस्य॑ प॒र्णस्य॒ तत् तत् प॒र्णस्य॑ पर्ण॒त्वम् प॑र्ण॒त्वम् प॒र्णस्य॒ तत् तत् प॒र्णस्य॑ पर्ण॒त्वम् । \newline
49. प॒र्णस्य॑ पर्ण॒त्वम् प॑र्ण॒त्वम् प॒र्णस्य॑ प॒र्णस्य॑ पर्ण॒त्वं ॅयस्य॒ यस्य॑ पर्ण॒त्वम् प॒र्णस्य॑ प॒र्णस्य॑ पर्ण॒त्वं ॅयस्य॑ । \newline
50. प॒र्ण॒त्वं ॅयस्य॒ यस्य॑ पर्ण॒त्वम् प॑र्ण॒त्वं ॅयस्य॑ पर्ण॒मयी॑ पर्ण॒मयी॒ यस्य॑ पर्ण॒त्वम् प॑र्ण॒त्वं ॅयस्य॑ पर्ण॒मयी᳚ । \newline
51. प॒र्ण॒त्वमिति॑ पर्ण - त्वम् । \newline
52. यस्य॑ पर्ण॒मयी॑ पर्ण॒मयी॒ यस्य॒ यस्य॑ पर्ण॒मयी॑ जु॒हूर् जु॒हूः प॑र्ण॒मयी॒ यस्य॒ यस्य॑ पर्ण॒मयी॑ जु॒हूः । \newline
53. प॒र्ण॒मयी॑ जु॒हूर् जु॒हूः प॑र्ण॒मयी॑ पर्ण॒मयी॑ जु॒हूर् भव॑ति॒ भव॑ति जु॒हूः प॑र्ण॒मयी॑ पर्ण॒मयी॑ जु॒हूर् भव॑ति । \newline
54. प॒र्ण॒मयीति॑ पर्ण - मयी᳚ । \newline
55. जु॒हूर् भव॑ति॒ भव॑ति जु॒हूर् जु॒हूर् भव॑ति सौ॒म्याः सौ॒म्या भव॑ति जु॒हूर् जु॒हूर् भव॑ति सौ॒म्याः । \newline
\pagebreak
\markright{ TS 3.5.7.2  \hfill https://www.vedavms.in \hfill}

\section{ TS 3.5.7.2 }

\textbf{TS 3.5.7.2 } \newline
\textbf{Samhita Paata} \newline

-र्भव॑ति सौ॒म्या अ॒स्याऽऽ*हु॑तयो भवन्ति जु॒षन्ते᳚ऽस्य दे॒वा आहु॑तीर्दे॒वा वै ब्रह्म॑न्नवदन्त॒ तत् प॒र्ण उपा॑ऽ*शृणोथ् सु॒श्रवा॒ वै नाम॒ यस्य॑ पर्ण॒मयी॑ जु॒हूर्भव॑ति॒ न पा॒पꣳ श्लोकꣳ॑ शृणोति॒ ब्रह्म॒ वै प॒र्णो विण्म॒रुतोऽन्नं॒ ॅविण्मा॑रु॒तो᳚ऽश्व॒त्थो यस्य॑ पर्ण॒मयी॑ जु॒हूर्भव॒त्या-श्व॑त्-थ्युप॒भृद्- ब्रह्म॑णै॒वान्न॒मव॑ रु॒न्धेऽथो॒ ब्रह्मै॒ - [  ] \newline

\textbf{Pada Paata} \newline

भव॑ति । सौ॒म्याः । अ॒स्य॒ । आहु॑तय॒ इत्या - हु॒त॒यः॒ । भ॒व॒न्ति॒ । जु॒षन्ते᳚ । अ॒स्य॒ । दे॒वाः । आहु॑ती॒रित्या - हु॒तीः॒ । दे॒वाः । वै । ब्रह्मन्न्॑ । अ॒व॒द॒न्त॒ । तत् । प॒र्णः । उपेति॑ । अ॒शृ॒णो॒त् । सु॒श्रवा॒ इति॑ सु - श्रवाः᳚ । वै । नाम॑ । यस्य॑ । प॒र्ण॒मयीति॑ पर्ण - मयी᳚ । जु॒हूः । भव॑ति । न । पा॒पम् । श्लोक᳚म् । शृ॒णो॒ति॒ । ब्रह्म॑ । वै । प॒र्णः । विट् । म॒रुतः॑ । अन्न᳚म् । विट् । मा॒रु॒तः । अ॒श्व॒त्थः । यस्य॑ । प॒र्ण॒मयीति॑ पर्ण-मयी᳚ । जु॒हूः । भव॑ति । आश्व॑त्थी । उ॒प॒भृतित्यु॑प - भृत् । ब्रह्म॑णा । ए॒व । अन्न᳚म् । अवेति॑ । रु॒न्धे॒ । अथो॒ इति॑ । ब्रह्म॑ ।  \newline


\textbf{Krama Paata} \newline

भव॑ति सौ॒म्याः । सौ॒म्या अ॑स्य । अ॒स्याहु॑तयः । आहु॑तयो भवन्ति । आहु॑तय॒ इत्या - हु॒त॒यः॒ । भ॒व॒न्ति॒ जु॒षन्ते᳚ । जु॒षन्ते᳚ ऽस्य । अ॒स्य॒ दे॒वाः । दे॒वा आहु॑तीः । आहु॑तीर् दे॒वाः । आहु॑ती॒रित्या - हु॒तीः॒ । दे॒वा वै । वै ब्रह्मन्न्॑ । ब्रह्म॑न्नवदन्त । अ॒व॒द॒न्त॒ तत् । तत् प॒र्णः । प॒र्ण उप॑ । उपा॑शृणोत् । अ॒शृ॒णो॒थ् सु॒श्रवाः᳚ । सु॒श्रवा॒ वै । सु॒श्रवा॒ इति॑ सु - श्रवाः᳚ । वै नाम॑ । नाम॒ यस्य॑ । यस्य॑ पर्ण॒मयी᳚ । प॒र्ण॒मयी॑ जु॒हूः । प॒र्ण॒मयीति॑ पर्ण - मयी᳚ । जु॒हूर् भव॑ति । भव॑ति॒ न । न पा॒पम् । पा॒पꣳ श्लोक᳚म् । श्लोकꣳ॑ शृणोति । शृ॒णो॒ति॒ ब्रह्म॑ । ब्रह्म॒ वै । वै प॒र्णः । प॒र्णो विट् । विण्म॒रुतः॑ । म॒रुतो ऽन्न᳚म् । अन्नं॒ ॅविट् । विण् मा॑रु॒तः । मा॒रु॒तो᳚ ऽश्व॒त्थः । अ॒श्व॒त्थो यस्य॑ । यस्य॑ पर्ण॒मयी᳚ । प॒र्ण॒मयी॑ जु॒हूः । प॒र्ण॒मयीति॑ पर्ण - मयी᳚ । जु॒हूर् भव॑ति । भव॒त्याश्व॑त्थी । आश्व॑त्थ्युप॒भृत् । उ॒प॒भृद् ब्रह्म॑णा । उ॒प॒भृदित्यु॑प - भृत् । ब्रह्म॑णै॒व । ए॒वान्न᳚म् । अन्न॒मव॑ । अव॑ रुन्धे । रु॒न्धे ऽथो᳚ । अथो॒ ब्रह्म॑ । अथो॒ इत्यथो᳚ । ब्रह्मै॒व \newline

\textbf{Jatai Paata} \newline

1. भव॑ति सौ॒म्याः सौ॒म्या भव॑ति॒ भव॑ति सौ॒म्याः । \newline
2. सौ॒म्या अ॑स्यास्य सौ॒म्याः सौ॒म्या अ॑स्य । \newline
3. अ॒स्याहु॑तय॒ आहु॑तयो ऽस्या॒ स्याहु॑तयः । \newline
4. आहु॑तयो भवन्ति भव॒ न्त्याहु॑तय॒ आहु॑तयो भवन्ति । \newline
5. आहु॑तय॒ इत्या - हु॒त॒यः॒ । \newline
6. भ॒व॒न्ति॒ जु॒षन्ते॑ जु॒षन्ते॑ भवन्ति भवन्ति जु॒षन्ते᳚ । \newline
7. जु॒षन्ते᳚ ऽस्यास्य जु॒षन्ते॑ जु॒षन्ते᳚ ऽस्य । \newline
8. अ॒स्य॒ दे॒वा दे॒वा अ॑स्यास्य दे॒वाः । \newline
9. दे॒वा आहु॑ती॒ राहु॑तीर् दे॒वा दे॒वा आहु॑तीः । \newline
10. आहु॑तीर् दे॒वा दे॒वा आहु॑ती॒ राहु॑तीर् दे॒वाः । \newline
11. आहु॑ती॒रित्या - हु॒तीः॒ । \newline
12. दे॒वा वै वै दे॒वा दे॒वा वै । \newline
13. वै ब्रह्म॒न् ब्रह्म॒न्॒. वै वै ब्रह्मन्न्॑ । \newline
14. ब्रह्म॑न् नवदन्ता वदन्त॒ ब्रह्म॒न् ब्रह्म॑न् नवदन्त । \newline
15. अ॒व॒द॒न्त॒ तत् तद॑वदन्ता वदन्त॒ तत् । \newline
16. तत् प॒र्णः प॒र्ण स्तत् तत् प॒र्णः । \newline
17. प॒र्ण उपोप॑ प॒र्णः प॒र्ण उप॑ । \newline
18. उपा॑शृणो दशृणो॒ दुपोपा॑शृणोत् । \newline
19. अ॒शृ॒णो॒थ् सु॒श्रवाः᳚ सु॒श्रवा॑ अशृणो दशृणोथ् सु॒श्रवाः᳚ । \newline
20. सु॒श्रवा॒ वै वै सु॒श्रवाः᳚ सु॒श्रवा॒ वै । \newline
21. सु॒श्रवा॒ इति॑ सु - श्रवाः᳚ । \newline
22. वै नाम॒ नाम॒ वै वै नाम॑ । \newline
23. नाम॒ यस्य॒ यस्य॒ नाम॒ नाम॒ यस्य॑ । \newline
24. यस्य॑ पर्ण॒मयी॑ पर्ण॒मयी॒ यस्य॒ यस्य॑ पर्ण॒मयी᳚ । \newline
25. प॒र्ण॒मयी॑ जु॒हूर् जु॒हूः प॑र्ण॒मयी॑ पर्ण॒मयी॑ जु॒हूः । \newline
26. प॒र्ण॒मयीति॑ पर्ण - मयी᳚ । \newline
27. जु॒हूर् भव॑ति॒ भव॑ति जु॒हूर् जु॒हूर् भव॑ति । \newline
28. भव॑ति॒ न न भव॑ति॒ भव॑ति॒ न । \newline
29. न पा॒पम् पा॒पम् न न पा॒पम् । \newline
30. पा॒पꣳ श्लोकꣳ॒॒ श्लोक॑म् पा॒पम् पा॒पꣳ श्लोक᳚म् । \newline
31. श्लोकꣳ॑ शृणोति शृणोति॒ श्लोकꣳ॒॒ श्लोकꣳ॑ शृणोति । \newline
32. शृ॒णो॒ति॒ ब्रह्म॒ ब्रह्म॑ शृणोति शृणोति॒ ब्रह्म॑ । \newline
33. ब्रह्म॒ वै वै ब्रह्म॒ ब्रह्म॒ वै । \newline
34. वै प॒र्णः प॒र्णो वै वै प॒र्णः । \newline
35. प॒र्णो विड् विट् प॒र्णः प॒र्णो विट् । \newline
36. विण् म॒रुतो॑ म॒रुतो॒ विड् विण् म॒रुतः॑ । \newline
37. म॒रुतो ऽन्न॒ मन्न॑म् म॒रुतो॑ म॒रुतो ऽन्न᳚म् । \newline
38. अन्नं॒ ॅविड् वि डन्न॒ मन्नं॒ ॅविट् । \newline
39. विण् मा॑रु॒तो मा॑रु॒तो विड् विण् मा॑रु॒तः । \newline
40. मा॒रु॒तो᳚ ऽश्व॒त्थो᳚ ऽश्व॒त्थो मा॑रु॒तो मा॑रु॒तो᳚ ऽश्व॒त्थः । \newline
41. अ॒श्व॒त्थो यस्य॒ यस्या᳚ श्व॒त्थो᳚ ऽश्व॒त्थो यस्य॑ । \newline
42. यस्य॑ पर्ण॒मयी॑ पर्ण॒मयी॒ यस्य॒ यस्य॑ पर्ण॒मयी᳚ । \newline
43. प॒र्ण॒मयी॑ जु॒हूर् जु॒हूः प॑र्ण॒मयी॑ पर्ण॒मयी॑ जु॒हूः । \newline
44. प॒र्ण॒मयीति॑ पर्ण - मयी᳚ । \newline
45. जु॒हूर् भव॑ति॒ भव॑ति जु॒हूर् जु॒हूर् भव॑ति । \newline
46. भव॒ त्याश्व॒ त्थ्याश्व॑त्थी॒ भव॑ति॒ भव॒ त्याश्व॑त्थी । \newline
47. आश्व॑ त्थ्युप॒भृ दु॑प॒भृ दाश्व॒ त्थ्याश्व॑ त्थ्युप॒भृत् । \newline
48. उ॒प॒भृद् ब्रह्म॑णा॒ ब्रह्म॑ णोप॒भृ दु॑प॒भृद् ब्रह्म॑णा । \newline
49. उ॒प॒भृतित्यु॑प - भृत् । \newline
50. ब्रह्म॑ णै॒वैव ब्रह्म॑णा॒ ब्रह्म॑ णै॒व । \newline
51. ए॒वान्न॒ मन्न॑ मे॒वैवान्न᳚म् । \newline
52. अन्न॒ मवावान्न॒ मन्न॒ मव॑ । \newline
53. अव॑ रुन्धे रु॒न्धे ऽवाव॑ रुन्धे । \newline
54. रु॒न्धे ऽथो॒ अथो॑ रुन्धे रु॒न्धे ऽथो᳚ । \newline
55. अथो॒ ब्रह्म॒ ब्रह्माथो॒ अथो॒ ब्रह्म॑ । \newline
56. अथो॒ इत्यथो᳚ । \newline
57. ब्रह्मै॒वैव ब्रह्म॒ ब्रह्मै॒व । \newline

\textbf{Ghana Paata } \newline

1. भव॑ति सौ॒म्याः सौ॒म्या भव॑ति॒ भव॑ति सौ॒म्या अ॑स्यास्य सौ॒म्या भव॑ति॒ भव॑ति सौ॒म्या अ॑स्य । \newline
2. सौ॒म्या अ॑स्यास्य सौ॒म्याः सौ॒म्या अ॒स्या हु॑तय॒ आहु॑तयो ऽस्य सौ॒म्याः सौ॒म्या अ॒स्या हु॑तयः । \newline
3. अ॒स्या हु॑तय॒ आहु॑तयो ऽस्या॒स्या हु॑तयो भवन्ति भव॒ न्त्याहु॑तयो ऽस्या॒स्या हु॑तयो भवन्ति । \newline
4. आहु॑तयो भवन्ति भव॒ न्त्याहु॑तय॒ आहु॑तयो भवन्ति जु॒षन्ते॑ जु॒षन्ते॑ भव॒ न्त्याहु॑तय॒ आहु॑तयो भवन्ति जु॒षन्ते᳚ । \newline
5. आहु॑तय॒ इत्या - हु॒त॒यः॒ । \newline
6. भ॒व॒न्ति॒ जु॒षन्ते॑ जु॒षन्ते॑ भवन्ति भवन्ति जु॒षन्ते᳚ ऽस्यास्य जु॒षन्ते॑ भवन्ति भवन्ति जु॒षन्ते᳚ ऽस्य । \newline
7. जु॒षन्ते᳚ ऽस्यास्य जु॒षन्ते॑ जु॒षन्ते᳚ ऽस्य दे॒वा दे॒वा अ॑स्य जु॒षन्ते॑ जु॒षन्ते᳚ ऽस्य दे॒वाः । \newline
8. अ॒स्य॒ दे॒वा दे॒वा अ॑स्यास्य दे॒वा आहु॑ती॒ राहु॑तीर् दे॒वा अ॑स्यास्य दे॒वा आहु॑तीः । \newline
9. दे॒वा आहु॑ती॒ राहु॑तीर् दे॒वा दे॒वा आहु॑तीर् दे॒वा दे॒वा आहु॑तीर् दे॒वा दे॒वा आहु॑तीर् दे॒वाः । \newline
10. आहु॑तीर् दे॒वा दे॒वा आहु॑ती॒ राहु॑तीर् दे॒वा वै वै दे॒वा आहु॑ती॒ राहु॑तीर् दे॒वा वै । \newline
11. आहु॑ती॒रित्या - हु॒तीः॒ । \newline
12. दे॒वा वै वै दे॒वा दे॒वा वै ब्रह्म॒न् ब्रह्म॒न्॒. वै दे॒वा दे॒वा वै ब्रह्मन्न्॑ । \newline
13. वै ब्रह्म॒न् ब्रह्म॒न्॒. वै वै ब्रह्म॑न् नवदन्ता वदन्त॒ ब्रह्म॒न्॒. वै वै ब्रह्म॑न् नवदन्त । \newline
14. ब्रह्म॑न् नवदन्ता वदन्त॒ ब्रह्म॒न् ब्रह्म॑न् नवदन्त॒ तत् तद॑वदन्त॒ ब्रह्म॒न् ब्रह्म॑न् नवदन्त॒ तत् । \newline
15. अ॒व॒द॒न्त॒ तत् तद॑वदन्ता वदन्त॒ तत् प॒र्णः प॒र्ण स्तद॑वदन्ता वदन्त॒ तत् प॒र्णः । \newline
16. तत् प॒र्णः प॒र्ण स्तत् तत् प॒र्ण उपोप॑ प॒र्ण स्तत् तत् प॒र्ण उप॑ । \newline
17. प॒र्ण उपोप॑ प॒र्णः प॒र्ण उपा॑शृणो दशृणो॒ दुप॑ प॒र्णः प॒र्ण उपा॑शृणोत् । \newline
18. उपा॑शृणो दशृणो॒ दुपोपा॑ शृणोथ् सु॒श्रवाः᳚ सु॒श्रवा॑ अशृणो॒ दुपोपा॑ शृणोथ् सु॒श्रवाः᳚ । \newline
19. अ॒शृ॒णो॒थ् सु॒श्रवाः᳚ सु॒श्रवा॑ अशृणो दशृणोथ् सु॒श्रवा॒ वै वै सु॒श्रवा॑ अशृणो दशृणोथ् सु॒श्रवा॒ वै । \newline
20. सु॒श्रवा॒ वै वै सु॒श्रवाः᳚ सु॒श्रवा॒ वै नाम॒ नाम॒ वै सु॒श्रवाः᳚ सु॒श्रवा॒ वै नाम॑ । \newline
21. सु॒श्रवा॒ इति॑ सु - श्रवाः᳚ । \newline
22. वै नाम॒ नाम॒ वै वै नाम॒ यस्य॒ यस्य॒ नाम॒ वै वै नाम॒ यस्य॑ । \newline
23. नाम॒ यस्य॒ यस्य॒ नाम॒ नाम॒ यस्य॑ पर्ण॒मयी॑ पर्ण॒मयी॒ यस्य॒ नाम॒ नाम॒ यस्य॑ पर्ण॒मयी᳚ । \newline
24. यस्य॑ पर्ण॒मयी॑ पर्ण॒मयी॒ यस्य॒ यस्य॑ पर्ण॒मयी॑ जु॒हूर् जु॒हूः प॑र्ण॒मयी॒ यस्य॒ यस्य॑ पर्ण॒मयी॑ जु॒हूः । \newline
25. प॒र्ण॒मयी॑ जु॒हूर् जु॒हूः प॑र्ण॒मयी॑ पर्ण॒मयी॑ जु॒हूर् भव॑ति॒ भव॑ति जु॒हूः प॑र्ण॒मयी॑ पर्ण॒मयी॑ जु॒हूर् भव॑ति । \newline
26. प॒र्ण॒मयीति॑ पर्ण - मयी᳚ । \newline
27. जु॒हूर् भव॑ति॒ भव॑ति जु॒हूर् जु॒हूर् भव॑ति॒ न न भव॑ति जु॒हूर् जु॒हूर् भव॑ति॒ न । \newline
28. भव॑ति॒ न न भव॑ति॒ भव॑ति॒ न पा॒पम् पा॒पम् न भव॑ति॒ भव॑ति॒ न पा॒पम् । \newline
29. न पा॒पम् पा॒पम् न न पा॒पꣳ श्लोकꣳ॒॒ श्लोक॑म् पा॒पम् न न पा॒पꣳ श्लोक᳚म् । \newline
30. पा॒पꣳ श्लोकꣳ॒॒ श्लोक॑म् पा॒पम् पा॒पꣳ श्लोकꣳ॑ शृणोति शृणोति॒ श्लोक॑म् पा॒पम् पा॒पꣳ श्लोकꣳ॑ शृणोति । \newline
31. श्लोकꣳ॑ शृणोति शृणोति॒ श्लोकꣳ॒॒ श्लोकꣳ॑ शृणोति॒ ब्रह्म॒ ब्रह्म॑ शृणोति॒ श्लोकꣳ॒॒ श्लोकꣳ॑ शृणोति॒ ब्रह्म॑ । \newline
32. शृ॒णो॒ति॒ ब्रह्म॒ ब्रह्म॑ शृणोति शृणोति॒ ब्रह्म॒ वै वै ब्रह्म॑ शृणोति शृणोति॒ ब्रह्म॒ वै । \newline
33. ब्रह्म॒ वै वै ब्रह्म॒ ब्रह्म॒ वै प॒र्णः प॒र्णो वै ब्रह्म॒ ब्रह्म॒ वै प॒र्णः । \newline
34. वै प॒र्णः प॒र्णो वै वै प॒र्णो विड् विट् प॒र्णो वै वै प॒र्णो विट् । \newline
35. प॒र्णो विड् विट् प॒र्णः प॒र्णो विण् म॒रुतो॑ म॒रुतो॒ विट् प॒र्णः प॒र्णो विण् म॒रुतः॑ । \newline
36. विण् म॒रुतो॑ म॒रुतो॒ विड् विण् म॒रुतो ऽन्न॒ मन्न॑म् म॒रुतो॒ विड् विण् म॒रुतो ऽन्न᳚म् । \newline
37. म॒रुतो ऽन्न॒ मन्न॑म् म॒रुतो॑ म॒रुतो ऽन्नं॒ ॅविड् विडन्न॑म् म॒रुतो॑ म॒रुतो ऽन्नं॒ ॅविट् । \newline
38. अन्नं॒ ॅविड् विडन्न॒ मन्नं॒ ॅविण् मा॑रु॒तो मा॑रु॒तो विडन्न॒ मन्नं॒ ॅविण् मा॑रु॒तः । \newline
39. विण् मा॑रु॒तो मा॑रु॒तो विड् विण् मा॑रु॒तो᳚ ऽश्व॒त्थो᳚ ऽश्व॒त्थो मा॑रु॒तो विड् विण् मा॑रु॒तो᳚ ऽश्व॒त्थः । \newline
40. मा॒रु॒तो᳚ ऽश्व॒त्थो᳚ ऽश्व॒त्थो मा॑रु॒तो मा॑रु॒तो᳚ ऽश्व॒त्थो यस्य॒ यस्या᳚ श्व॒त्थो मा॑रु॒तो मा॑रु॒तो᳚ ऽश्व॒त्थो यस्य॑ । \newline
41. अ॒श्व॒त्थो यस्य॒ यस्या᳚ श्व॒त्थो᳚ ऽश्व॒त्थो यस्य॑ पर्ण॒मयी॑ पर्ण॒मयी॒ यस्या᳚ श्व॒त्थो᳚ ऽश्व॒त्थो यस्य॑ पर्ण॒मयी᳚ । \newline
42. यस्य॑ पर्ण॒मयी॑ पर्ण॒मयी॒ यस्य॒ यस्य॑ पर्ण॒मयी॑ जु॒हूर् जु॒हूः प॑र्ण॒मयी॒ यस्य॒ यस्य॑ पर्ण॒मयी॑ जु॒हूः । \newline
43. प॒र्ण॒मयी॑ जु॒हूर् जु॒हूः प॑र्ण॒मयी॑ पर्ण॒मयी॑ जु॒हूर् भव॑ति॒ भव॑ति जु॒हूः प॑र्ण॒मयी॑ पर्ण॒मयी॑ जु॒हूर् भव॑ति । \newline
44. प॒र्ण॒मयीति॑ पर्ण - मयी᳚ । \newline
45. जु॒हूर् भव॑ति॒ भव॑ति जु॒हूर् जु॒हूर् भव॒ त्याश्व॒ त्थ्याश्व॑त्थी॒ भव॑ति जु॒हूर् जु॒हूर् भव॒ त्याश्व॑त्थी । \newline
46. भव॒ त्याश्व॒ त्थ्याश्व॑त्थी॒ भव॑ति॒ भव॒ त्याश्व॑ त्थ्युप॒भृ दु॑प॒भृदा श्व॑त्थी॒ भव॑ति॒ भव॒ त्याश्व॑ त्थ्युप॒भृत् । \newline
47. आश्व॑ त्थ्युप॒भृ दु॑प॒भृ दाश्व॒ त्थ्याश्व॑ त्थ्युप॒भृद् ब्रह्म॑णा॒ ब्रह्म॑ णोप॒भृ दाश्व॒ त्थ्याश्व॑ त्थ्युप॒भृद् ब्रह्म॑णा । \newline
48. उ॒प॒भृद् ब्रह्म॑णा॒ ब्रह्म॑ णोप॒भृ दु॑प॒भृद् ब्रह्म॑णै॒वैव ब्रह्म॑ णोप॒भृ दु॑प॒भृद् ब्रह्म॑णै॒व । \newline
49. उ॒प॒भृतित्यु॑प - भृत् । \newline
50. ब्रह्म॑णै॒वैव ब्रह्म॑णा॒ ब्रह्म॑ णै॒वान्न॒ मन्न॑ मे॒व ब्रह्म॑णा॒ ब्रह्म॑ णै॒वान्न᳚म् । \newline
51. ए॒वान्न॒ मन्न॑ मे॒वैवान्न॒ मवा वान्न॑ मे॒वैवान्न॒ मव॑ । \newline
52. अन्न॒ मवावान्न॒ मन्न॒ मव॑ रुन्धे रु॒न्धे ऽवान्न॒ मन्न॒ मव॑ रुन्धे । \newline
53. अव॑ रुन्धे रु॒न्धे ऽवाव॑ रु॒न्धे ऽथो॒ अथो॑ रु॒न्धे ऽवाव॑ रु॒न्धे ऽथो᳚ । \newline
54. रु॒न्धे ऽथो॒ अथो॑ रुन्धे रु॒न्धे ऽथो॒ ब्रह्म॒ ब्रह्माथो॑ रुन्धे रु॒न्धे ऽथो॒ ब्रह्म॑ । \newline
55. अथो॒ ब्रह्म॒ ब्रह्माथो॒ अथो॒ ब्रह्मै॒वैव ब्रह्माथो॒ अथो॒ ब्रह्मै॒व । \newline
56. अथो॒ इत्यथो᳚ । \newline
57. ब्रह्मै॒वैव ब्रह्म॒ ब्रह्मै॒व वि॒शि वि॒श्ये॑व ब्रह्म॒ ब्रह्मै॒व वि॒शि । \newline
\pagebreak
\markright{ TS 3.5.7.3  \hfill https://www.vedavms.in \hfill}

\section{ TS 3.5.7.3 }

\textbf{TS 3.5.7.3 } \newline
\textbf{Samhita Paata} \newline

-व वि॒श्यद्ध्यू॑हति रा॒ष्ट्रं ॅवै प॒र्णो विड॑श्व॒त्थो यत् प॑र्ण॒मयी॑ जु॒हूर्भव॒त्या-श्व॑त्थ्युप॒भृद्-रा॒ष्ट्रमे॒व वि॒श्यद्ध्यू॑हति प्र॒जाप॑ति॒र्वा अ॑जुहो॒थ् सा यत्राऽऽ*हु॑तिः प्र॒त्यति॑ष्ठ॒त् ततो॒ विक॑ङ्कत॒ उद॑तिष्ठ॒त् ततः॑ प्र॒जा अ॑सृजत॒ यस्य॒ वैक॑ङ्कती ध्रु॒वा भव॑ति॒ प्रत्य॒वास्या ऽऽ*हु॑तयस्तिष्ठ॒न्त्यथो॒ प्रैव जा॑यत ए॒तद्वै स्रु॒चाꣳ ( ) रू॒पं ॅयस्यै॒वꣳ रू॑पाः॒ स्रुचो॒ भव॑न्ति॒ सर्वा᳚ण्ये॒वैनꣳ॑ रू॒पाणि॑ पशू॒नामुप॑तिष्ठन्ते॒ नास्याप॑-रूपमा॒त्मञ्जा॑यते ॥ \newline

\textbf{Pada Paata} \newline

ए॒व । वि॒शि । अधीति॑ । ऊ॒ह॒ति॒ । रा॒ष्ट्रम् । वै । प॒र्णः । विट् । अ॒श्व॒त्थः । यत् । प॒र्ण॒मयीति॑ पर्ण - मयी᳚ । जु॒हूः । भव॑ति । आश्व॑त्थी । उ॒प॒भृतित्यु॑प-भृत् । रा॒ष्ट्रम् । ए॒व । वि॒शि । अधीति॑ । ऊ॒ह॒ति॒ । प्र॒जाप॑ति॒रिति॑ प्र॒जा - प॒तिः॒ । वै । अ॒ज॒हो॒त् । सा । यत्र॑ । आहु॑ति॒रित्या - हु॒तः॒ । प्र॒त्यति॑ष्ठ॒दिति॑ प्रति - अति॑ष्ठत् । ततः॑ । विक॑ङ्कत॒ इति॒ वि - क॒ङ्क॒तः॒ । उदिति॑ । अ॒ति॒ष्ठ॒त् । ततः॑ । प्र॒जा इति॑ प्र - जाः । अ॒सृ॒ज॒त॒ । यस्य॑ । वैक॑ङ्कती । ध्रु॒वा । भव॑ति । प्रतीति॑ । ए॒व । अ॒स्य॒ । आहु॑तय॒ इत्या-हु॒त॒यः॒ । ति॒ष्ठ॒न्ति॒ । अथो॒ इति॑ । प्रेति॑ । ए॒व । जा॒य॒ते॒ । ए॒तत् । वै । स्रु॒चाम् ( ) । रू॒पम् । यस्य॑ । ए॒वꣳरू॑पा॒ इत्ये॒वं - रू॒पाः॒ । स्रुचः॑ । भव॑न्ति । सर्वा॑णि । ए॒व । ए॒न॒म् । रू॒पाणि॑ । प॒शू॒नाम् । उपेति॑ । ति॒ष्ठ॒न्ते॒ । न । अ॒स्य॒ । अप॑रूप॒मित्यप॑ - रू॒प॒म् । आ॒त्मन्न् । जा॒य॒ते॒ ॥  \newline


\textbf{Krama Paata} \newline

ए॒व वि॒शि । वि॒श्यधि॑ । अध्यू॑हति । ऊ॒ह॒ति॒ रा॒ष्ट्रम् । रा॒ष्ट्रं ॅवै । वै प॒र्णः । प॒र्णो विट् । विड॑श्व॒त्थः । अ॒श्व॒त्थो यत् । यत् प॑र्ण॒मयी᳚ । प॒र्ण॒मयी॑ जु॒हूः । प॒र्ण॒मयीति॑ पर्ण - मयी᳚ । जु॒हूर् भव॑ति । भव॒त्याश्व॑त्थी । आश्व॑त्थ्युप॒भृत् । उ॒प॒भृद् रा॒ष्ट्रम् । उ॒प॒भृदित्यु॑प - भृत् । रा॒ष्ट्रमे॒व । ए॒व वि॒शि । वि॒श्यधि॑ । अध्यू॑हति । ऊ॒ह॒ति॒ प्र॒जाप॑तिः । प्र॒जाप॑ति॒र् वै । प्र॒जाप॑ति॒रिति॑ प्र॒जा - प॒तिः॒ । वा अ॑जुहोत् । अ॒जु॒हो॒थ् सा । सा यत्र॑ । यत्राहु॑तिः । आहु॑तिः प्र॒त्यति॑ष्ठत् । आहु॑ति॒रित्या - हु॒तिः॒ । प्र॒त्यति॑ष्ठ॒त् ततः॑ । प्र॒त्यति॑ष्ठ॒दिति॑ प्रति - अति॑ष्ठत् । ततो॒ विक॑ङ्कतः । विक॑ङ्कत॒ उत् । विक॑ङ्कत॒ इति॒ वि - क॒ङ्क॒तः॒ । उद॑तिष्ठत् । अ॒ति॒ष्ठ॒त् ततः॑ । ततः॑ प्र॒जाः । प्र॒जा अ॑सृजत । प्र॒जा इति॑ प्र - जाः । अ॒सृ॒ज॒त॒ यस्य॑ । यस्य॒ वैक॑ङ्कती । वैक॑ङ्कती ध्रु॒वा । धु॒वा भव॑ति । भव॑ति॒ प्रति॑ । प्रत्ये॒व । ए॒वास्य॑ । अ॒स्याहु॑तयः । आहु॑तयस्तिष्ठन्ति । आहु॑तय॒ इत्या - हु॒त॒यः॒ । ति॒ष्ठ॒न्त्यथो᳚ । अथो॒ प्र । अथो॒ इत्यथो᳚ । प्रैव । ए॒व जा॑यते । जा॒य॒त॒ ए॒तत् । ए॒तद् वै । वै स्रु॒चाम् ( ) । स्रु॒चाꣳ रू॒पम् । रू॒पं ॅयस्य॑ । यस्यै॒वꣳरू॑पाः । ए॒वꣳरू॑पाः॒ स्रुचः॑ । ए॒वꣳरू॑पा॒ इत्ये॒वम् - रू॒पाः॒ । स्रुचो॒ भव॑न्ति । भव॑न्ति॒ सर्वा॑णि । सर्वा᳚ण्ये॒व । ए॒वैन᳚म् । ए॒नꣳ॒॒ रू॒पाणि॑ । रू॒पाणि॑ पशू॒नाम् । प॒शू॒नामुप॑ । उप॑ तिष्ठन्ते । ति॒ष्ठ॒न्ते॒ न । नास्य॑ । अ॒स्याप॑रूपम् । अप॑रूपमा॒त्मन्न् । अप॑रूप॒मित्यप॑ - रू॒प॒म् । आ॒त्मन् जा॑यते । जा॒य॒त॒ इति॑ जायते । \newline

\textbf{Jatai Paata} \newline

1. ए॒व वि॒शि वि॒श्ये॑वैव वि॒शि । \newline
2. वि॒श्य ध्यधि॑ वि॒शि वि॒श्यधि॑ । \newline
3. अध्यू॑ह त्यूह॒ त्यध्य ध्यू॑हति । \newline
4. ऊ॒ह॒ति॒ रा॒ष्ट्रꣳ रा॒ष्ट्र मू॑ह त्यूहति रा॒ष्ट्रम् । \newline
5. रा॒ष्ट्रं ॅवै वै रा॒ष्ट्रꣳ रा॒ष्ट्रं ॅवै । \newline
6. वै प॒र्णः प॒र्णो वै वै प॒र्णः । \newline
7. प॒र्णो विड् विट् प॒र्णः प॒र्णो विट् । \newline
8. विड॑श्व॒त्थो᳚ ऽश्व॒त्थो विड् विड॑श्व॒त्थः । \newline
9. अ॒श्व॒त्थो यद् यद॑श्व॒त्थो᳚ ऽश्व॒त्थो यत् । \newline
10. यत् प॑र्ण॒मयी॑ पर्ण॒मयी॒ यद् यत् प॑र्ण॒मयी᳚ । \newline
11. प॒र्ण॒मयी॑ जु॒हूर् जु॒हूः प॑र्ण॒मयी॑ पर्ण॒मयी॑ जु॒हूः । \newline
12. प॒र्ण॒मयीति॑ पर्ण - मयी᳚ । \newline
13. जु॒हूर् भव॑ति॒ भव॑ति जु॒हूर् जु॒हूर् भव॑ति । \newline
14. भव॒ त्याश्व॒ त्थ्याश्व॑त्थी॒ भव॑ति॒ भव॒ त्याश्व॑त्थी । \newline
15. आश्व॑ त्थ्युप॒भृ दु॑प॒भृ दाश्व॒ त्थ्याश्व॑ त्थ्युप॒भृत् । \newline
16. उ॒प॒भृद् रा॒ष्ट्रꣳ रा॒ष्ट्र मु॑प॒भृ दु॑प॒भृद् रा॒ष्ट्रम् । \newline
17. उ॒प॒भृतित्यु॑प - भृत् । \newline
18. रा॒ष्ट्र मे॒वैव रा॒ष्ट्रꣳ रा॒ष्ट्र मे॒व । \newline
19. ए॒व वि॒शि वि॒श्ये॑वैव वि॒शि । \newline
20. वि॒श्यध्यधि॑ वि॒शि वि॒श्यधि॑ । \newline
21. अध्यू॑ह त्यूह॒ त्यध्य ध्यू॑हति । \newline
22. ऊ॒ह॒ति॒ प्र॒जाप॑तिः प्र॒जाप॑ति रूह त्यूहति प्र॒जाप॑तिः । \newline
23. प्र॒जाप॑ति॒र् वै वै प्र॒जाप॑तिः प्र॒जाप॑ति॒र् वै । \newline
24. प्र॒जाप॑ति॒रिति॑ प्र॒जा - प॒तिः॒ । \newline
25. वा अ॑जुहो दजुहो॒द् वै वा अ॑जुहोत् । \newline
26. अ॒जु॒हो॒थ् सा सा ऽजु॑हो दजुहो॒थ् सा । \newline
27. सा यत्र॒ यत्र॒ सा सा यत्र॑ । \newline
28. यत्राहु॑ति॒ राहु॑ति॒र् यत्र॒ यत्राहु॑तिः । \newline
29. आहु॑तिः प्र॒त्यति॑ष्ठत् प्र॒त्यति॑ष्ठ॒ दाहु॑ति॒ राहु॑तिः प्र॒त्यति॑ष्ठत् । \newline
30. आहु॑ति॒रित्या - हु॒तिः॒ । \newline
31. प्र॒त्यति॑ष्ठ॒त् तत॒ स्ततः॑ प्र॒त्यति॑ष्ठत् प्र॒त्यति॑ष्ठ॒त् ततः॑ । \newline
32. प्र॒त्यति॑ष्ठ॒दिति॑ प्रति - अति॑ष्ठत् । \newline
33. ततो॒ विक॑ङ्कतो॒ विक॑ङ्कत॒ स्तत॒ स्ततो॒ विक॑ङ्कतः । \newline
34. विक॑ङ्कत॒ उदुद् विक॑ङ्कतो॒ विक॑ङ्कत॒ उत् । \newline
35. विक॑ङ्कत॒ इति॒ वि - क॒ङ्क॒तः॒ । \newline
36. उद॑तिष्ठ दतिष्ठ॒ दुदु द॑तिष्ठत् । \newline
37. अ॒ति॒ष्ठ॒त् तत॒ स्ततो॑ ऽतिष्ठ दतिष्ठ॒त् ततः॑ । \newline
38. ततः॑ प्र॒जाः प्र॒जा स्तत॒ स्ततः॑ प्र॒जाः । \newline
39. प्र॒जा अ॑सृजता सृजत प्र॒जाः प्र॒जा अ॑सृजत । \newline
40. प्र॒जा इति॑ प्र - जाः । \newline
41. अ॒सृ॒ज॒त॒ यस्य॒ यस्या॑ सृजता सृजत॒ यस्य॑ । \newline
42. यस्य॒ वैक॑ङ्कती॒ वैक॑ङ्कती॒ यस्य॒ यस्य॒ वैक॑ङ्कती । \newline
43. वैक॑ङ्कती ध्रु॒वा ध्रु॒वा वैक॑ङ्कती॒ वैक॑ङ्कती ध्रु॒वा । \newline
44. ध्रु॒वा भव॑ति॒ भव॑ति ध्रु॒वा ध्रु॒वा भव॑ति । \newline
45. भव॑ति॒ प्रति॒ प्रति॒ भव॑ति॒ भव॑ति॒ प्रति॑ । \newline
46. प्रत्ये॒वैव प्रति॒ प्रत्ये॒व । \newline
47. ए॒वास्या᳚ स्यै॒वैवास्य॑ । \newline
48. अ॒स्या हु॑तय॒ आहु॑तयो ऽस्या॒ स्याहु॑तयः । \newline
49. आहु॑तय स्तिष्ठन्ति तिष्ठ॒ न्त्याहु॑तय॒ आहु॑तय स्तिष्ठन्ति । \newline
50. आहु॑तय॒ इत्या - हु॒त॒यः॒ । \newline
51. ति॒ष्ठ॒ न्त्यथो॒ अथो॑ तिष्ठन्ति तिष्ठ॒ न्त्यथो᳚ । \newline
52. अथो॒ प्र प्राथो॒ अथो॒ प्र । \newline
53. अथो॒ इत्यथो᳚ । \newline
54. प्रैवैव प्र प्रैव । \newline
55. ए॒व जा॑यते जायत ए॒वैव जा॑यते । \newline
56. जा॒य॒त॒ ए॒त दे॒तज् जा॑यते जायत ए॒तत् । \newline
57. ए॒तद् वै वा ए॒त दे॒तद् वै । \newline
58. वै स्रु॒चाꣳ स्रु॒चां ॅवै वै स्रु॒चाम् । \newline
59. स्रु॒चाꣳ रू॒पꣳ रू॒पꣳ स्रु॒चाꣳ स्रु॒चाꣳ रू॒पम् । \newline
60. रू॒पं ॅयस्य॒ यस्य॑ रू॒पꣳ रू॒पं ॅयस्य॑ । \newline
61. यस्यै॒वꣳरू॑पा ए॒वꣳरू॑पा॒ यस्य॒ यस्यै॒वꣳरू॑पाः । \newline
62. ए॒वꣳरू॑पाः॒ स्रुचः॒ स्रुच॑ ए॒वꣳरू॑पा ए॒वꣳरू॑पाः॒ स्रुचः॑ । \newline
63. ए॒वꣳरू॑पा॒ इत्ये॒वं - रू॒पाः॒ । \newline
64. स्रुचो॒ भव॑न्ति॒ भव॑न्ति॒ स्रुचः॒ स्रुचो॒ भव॑न्ति । \newline
65. भव॑न्ति॒ सर्वा॑णि॒ सर्वा॑णि॒ भव॑न्ति॒ भव॑न्ति॒ सर्वा॑णि । \newline
66. सर्वा᳚ ण्ये॒वैव सर्वा॑णि॒ सर्वा᳚ ण्ये॒व । \newline
67. ए॒वैन॑ मेन मे॒वैवैन᳚म् । \newline
68. ए॒नꣳ॒॒ रू॒पाणि॑ रू॒पाण्ये॑न मेनꣳ रू॒पाणि॑ । \newline
69. रू॒पाणि॑ पशू॒नाम् प॑शू॒नाꣳ रू॒पाणि॑ रू॒पाणि॑ पशू॒नाम् । \newline
70. प॒शू॒ना मुपोप॑ पशू॒नाम् प॑शू॒ना मुप॑ । \newline
71. उप॑ तिष्ठन्ते तिष्ठन्त॒ उपोप॑ तिष्ठन्ते । \newline
72. ति॒ष्ठ॒न्ते॒ न न ति॑ष्ठन्ते तिष्ठन्ते॒ न । \newline
73. नास्या᳚स्य॒ न नास्य॑ । \newline
74. अ॒स्याप॑रूप॒ मप॑रूप मस्या॒ स्याप॑रूपम् । \newline
75. अप॑रूप मा॒त्मन् ना॒त्मन् नप॑रूप॒ मप॑रूप मा॒त्मन्न् । \newline
76. अप॑रूप॒मित्यप॑ - रू॒प॒म् । \newline
77. आ॒त्मन् जा॑यते जायत आ॒त्मन् ना॒त्मन् जा॑यते । \newline
78. जा॒य॒त॒ इति॑ जायते । \newline

\textbf{Ghana Paata } \newline

1. ए॒व वि॒शि वि॒श्ये॑वैव वि॒श्य ध्यधि॑ वि॒श्ये॑वैव वि॒श्यधि॑ । \newline
2. वि॒श्य ध्यधि॑ वि॒शि वि॒श्य ध्यू॑ह त्यूह॒ त्यधि॑ वि॒शि वि॒श्य ध्यू॑हति । \newline
3. अध्यू॑ह त्यूह॒ त्यध्य ध्यू॑हति रा॒ष्ट्रꣳ रा॒ष्ट्र मू॑ह॒ त्यध्य ध्यू॑हति रा॒ष्ट्रम् । \newline
4. ऊ॒ह॒ति॒ रा॒ष्ट्रꣳ रा॒ष्ट्र मू॑ह त्यूहति रा॒ष्ट्रं ॅवै वै रा॒ष्ट्र मू॑ह त्यूहति रा॒ष्ट्रं ॅवै । \newline
5. रा॒ष्ट्रं ॅवै वै रा॒ष्ट्रꣳ रा॒ष्ट्रं ॅवै प॒र्णः प॒र्णो वै रा॒ष्ट्रꣳ रा॒ष्ट्रं ॅवै प॒र्णः । \newline
6. वै प॒र्णः प॒र्णो वै वै प॒र्णो विड् विट् प॒र्णो वै वै प॒र्णो विट् । \newline
7. प॒र्णो विड् विट् प॒र्णः प॒र्णो विड॑श्व॒त्थो᳚ ऽश्व॒त्थो विट् प॒र्णः प॒र्णो विड॑श्व॒त्थः । \newline
8. विड॑श्व॒त्थो᳚ ऽश्व॒त्थो विड् विड॑श्व॒त्थो यद् यद॑श्व॒त्थो विड् विड॑श्व॒त्थो यत् । \newline
9. अ॒श्व॒त्थो यद् यद॑श्व॒त्थो᳚ ऽश्व॒त्थो यत् प॑र्ण॒मयी॑ पर्ण॒मयी॒ यद॑श्व॒त्थो᳚ ऽश्व॒त्थो यत् प॑र्ण॒मयी᳚ । \newline
10. यत् प॑र्ण॒मयी॑ पर्ण॒मयी॒ यद् यत् प॑र्ण॒मयी॑ जु॒हूर् जु॒हूः प॑र्ण॒मयी॒ यद् यत् प॑र्ण॒मयी॑ जु॒हूः । \newline
11. प॒र्ण॒मयी॑ जु॒हूर् जु॒हूः प॑र्ण॒मयी॑ पर्ण॒मयी॑ जु॒हूर् भव॑ति॒ भव॑ति जु॒हूः प॑र्ण॒मयी॑ पर्ण॒मयी॑ जु॒हूर् भव॑ति । \newline
12. प॒र्ण॒मयीति॑ पर्ण - मयी᳚ । \newline
13. जु॒हूर् भव॑ति॒ भव॑ति जु॒हूर् जु॒हूर् भव॒ त्याश्व॒ त्थ्याश्व॑त्थी॒ भव॑ति जु॒हूर् जु॒हूर् भव॒ त्याश्व॑त्थी । \newline
14. भव॒ त्याश्व॒ त्थ्याश्व॑त्थी॒ भव॑ति॒ भव॒ त्याश्व॑ त्थ्युप॒भृ दु॑प॒भृ दाश्व॑त्थी॒ भव॑ति॒ भव॒ त्याश्व॑ त्थ्युप॒भृत् । \newline
15. आश्व॑ त्थ्युप॒भृ दु॑प॒भृ दाश्व॒ त्थ्याश्व॑ त्थ्युप॒भृद् रा॒ष्ट्रꣳ रा॒ष्ट्र मु॑प॒भृ दाश्व॒ त्थ्याश्व॑ त्थ्युप॒भृद् रा॒ष्ट्रम् । \newline
16. उ॒प॒भृद् रा॒ष्ट्रꣳ रा॒ष्ट्र मु॑प॒भृ दु॑प॒भृद् रा॒ष्ट्र मे॒वैव रा॒ष्ट्र मु॑प॒भृ दु॑प॒भृद् रा॒ष्ट्र मे॒व । \newline
17. उ॒प॒भृतित्यु॑प - भृत् । \newline
18. रा॒ष्ट्र मे॒वैव रा॒ष्ट्रꣳ रा॒ष्ट्र मे॒व वि॒शि वि॒श्ये॑व रा॒ष्ट्रꣳ रा॒ष्ट्र मे॒व वि॒शि । \newline
19. ए॒व वि॒शि वि॒श्ये॑वैव वि॒श्य ध्यधि॑ वि॒श्ये॑वैव वि॒श्यधि॑ । \newline
20. वि॒श्य ध्यधि॑ वि॒शि वि॒श्य ध्यू॑ह त्यूह॒ त्यधि॑ वि॒शि वि॒श्य ध्यू॑हति । \newline
21. अध्यू॑ह त्यूह॒ त्यध्य ध्यू॑हति प्र॒जाप॑तिः प्र॒जाप॑ति रूह॒ त्यध्य ध्यू॑हति प्र॒जाप॑तिः । \newline
22. ऊ॒ह॒ति॒ प्र॒जाप॑तिः प्र॒जाप॑ति रूह त्यूहति प्र॒जाप॑ति॒र् वै वै प्र॒जाप॑ति रूह त्यूहति प्र॒जाप॑ति॒र् वै । \newline
23. प्र॒जाप॑ति॒र् वै वै प्र॒जाप॑तिः प्र॒जाप॑ति॒र् वा अ॑जुहो दजुहो॒द् वै प्र॒जाप॑तिः प्र॒जाप॑ति॒र् वा अ॑जुहोत् । \newline
24. प्र॒जाप॑ति॒रिति॑ प्र॒जा - प॒तिः॒ । \newline
25. वा अ॑जुहो दजुहो॒द् वै वा अ॑जुहो॒थ् सा सा ऽजु॑हो॒द् वै वा अ॑जुहो॒थ् सा । \newline
26. अ॒जु॒हो॒थ् सा सा ऽजु॑हो दजुहो॒थ् सा यत्र॒ यत्र॒ सा ऽजु॑हो दजुहो॒थ् सा यत्र॑ । \newline
27. सा यत्र॒ यत्र॒ सा सा यत्राहु॑ति॒ राहु॑ति॒र् यत्र॒ सा सा यत्राहु॑तिः । \newline
28. यत्राहु॑ति॒ राहु॑ति॒र् यत्र॒ यत्राहु॑तिः प्र॒त्यति॑ष्ठत् प्र॒त्यति॑ष्ठ॒ दाहु॑ति॒र् यत्र॒ यत्राहु॑तिः प्र॒त्यति॑ष्ठत् । \newline
29. आहु॑तिः प्र॒त्यति॑ष्ठत् प्र॒त्यति॑ष्ठ॒ दाहु॑ति॒ राहु॑तिः प्र॒त्यति॑ष्ठ॒त् तत॒ स्ततः॑ प्र॒त्यति॑ष्ठ॒ दाहु॑ति॒ राहु॑तिः प्र॒त्यति॑ष्ठ॒त् ततः॑ । \newline
30. आहु॑ति॒रित्या - हु॒तिः॒ । \newline
31. प्र॒त्यति॑ष्ठ॒त् तत॒ स्ततः॑ प्र॒त्यति॑ष्ठत् प्र॒त्यति॑ष्ठ॒त् ततो॒ विक॑ङ्कतो॒ विक॑ङ्कत॒ स्ततः॑ प्र॒त्यति॑ष्ठत् प्र॒त्यति॑ष्ठ॒त् ततो॒ विक॑ङ्कतः । \newline
32. प्र॒त्यति॑ष्ठ॒दिति॑ प्रति - अति॑ष्ठत् । \newline
33. ततो॒ विक॑ङ्कतो॒ विक॑ङ्कत॒ स्तत॒ स्ततो॒ विक॑ङ्कत॒ उदुद् विक॑ङ्कत॒ स्तत॒ स्ततो॒ विक॑ङ्कत॒ उत् । \newline
34. विक॑ङ्कत॒ उदुद् विक॑ङ्कतो॒ विक॑ङ्कत॒ उद॑तिष्ठ दतिष्ठ॒ दुद् विक॑ङ्कतो॒ विक॑ङ्कत॒ उद॑तिष्ठत् । \newline
35. विक॑ङ्कत॒ इति॒ वि - क॒ङ्क॒तः॒ । \newline
36. उद॑तिष्ठ दतिष्ठ॒ दुदु द॑तिष्ठ॒त् तत॒ स्ततो॑ ऽतिष्ठ॒ दुदु द॑तिष्ठ॒त् ततः॑ । \newline
37. अ॒ति॒ष्ठ॒त् तत॒ स्ततो॑ ऽतिष्ठ दतिष्ठ॒त् ततः॑ प्र॒जाः प्र॒जा स्ततो॑ ऽतिष्ठ दतिष्ठ॒त् ततः॑ प्र॒जाः । \newline
38. ततः॑ प्र॒जाः प्र॒जा स्तत॒ स्ततः॑ प्र॒जा अ॑सृजता सृजत प्र॒जा स्तत॒ स्ततः॑ प्र॒जा अ॑सृजत । \newline
39. प्र॒जा अ॑सृजता सृजत प्र॒जाः प्र॒जा अ॑सृजत॒ यस्य॒ यस्या॑ सृजत प्र॒जाः प्र॒जा अ॑सृजत॒ यस्य॑ । \newline
40. प्र॒जा इति॑ प्र - जाः । \newline
41. अ॒सृ॒ज॒त॒ यस्य॒ यस्या॑ सृजता सृजत॒ यस्य॒ वैक॑ङ्कती॒ वैक॑ङ्कती॒ यस्या॑ सृजता सृजत॒ यस्य॒ वैक॑ङ्कती । \newline
42. यस्य॒ वैक॑ङ्कती॒ वैक॑ङ्कती॒ यस्य॒ यस्य॒ वैक॑ङ्कती ध्रु॒वा ध्रु॒वा वैक॑ङ्कती॒ यस्य॒ यस्य॒ वैक॑ङ्कती ध्रु॒वा । \newline
43. वैक॑ङ्कती ध्रु॒वा ध्रु॒वा वैक॑ङ्कती॒ वैक॑ङ्कती ध्रु॒वा भव॑ति॒ भव॑ति ध्रु॒वा वैक॑ङ्कती॒ वैक॑ङ्कती ध्रु॒वा भव॑ति । \newline
44. ध्रु॒वा भव॑ति॒ भव॑ति ध्रु॒वा ध्रु॒वा भव॑ति॒ प्रति॒ प्रति॒ भव॑ति ध्रु॒वा ध्रु॒वा भव॑ति॒ प्रति॑ । \newline
45. भव॑ति॒ प्रति॒ प्रति॒ भव॑ति॒ भव॑ति॒ प्रत्ये॒वैव प्रति॒ भव॑ति॒ भव॑ति॒ प्रत्ये॒व । \newline
46. प्रत्ये॒वैव प्रति॒ प्रत्ये॒वास्या᳚ स्यै॒व प्रति॒ प्रत्ये॒वास्य॑ । \newline
47. ए॒वास्या᳚ स्यै॒वैवा स्याहु॑तय॒ आहु॑तयो ऽस्यै॒वैवा स्याहु॑तयः । \newline
48. अ॒स्या हु॑तय॒ आहु॑तयो ऽस्या॒स्या हु॑तय स्तिष्ठन्ति तिष्ठ॒ न्त्याहु॑तयो ऽस्या॒स्या हु॑तय स्तिष्ठन्ति । \newline
49. आहु॑तय स्तिष्ठन्ति तिष्ठ॒ न्त्याहु॑तय॒ आहु॑तय स्तिष्ठ॒ न्त्यथो॒ अथो॑ तिष्ठ॒ न्त्याहु॑तय॒ आहु॑तय स्तिष्ठ॒ न्त्यथो᳚ । \newline
50. आहु॑तय॒ इत्या - हु॒त॒यः॒ । \newline
51. ति॒ष्ठ॒ न्त्यथो॒ अथो॑ तिष्ठन्ति तिष्ठ॒ न्त्यथो॒ प्र प्राथो॑ तिष्ठन्ति तिष्ठ॒ न्त्यथो॒ प्र । \newline
52. अथो॒ प्र प्राथो॒ अथो॒ प्रैवैव प्राथो॒ अथो॒ प्रैव । \newline
53. अथो॒ इत्यथो᳚ । \newline
54. प्रैवैव प्र प्रैव जा॑यते जायत ए॒व प्र प्रैव जा॑यते । \newline
55. ए॒व जा॑यते जायत ए॒वैव जा॑यत ए॒त दे॒तज् जा॑यत ए॒वैव जा॑यत ए॒तत् । \newline
56. जा॒य॒त॒ ए॒त दे॒तज् जा॑यते जायत ए॒तद् वै वा ए॒तज् जा॑यते जायत ए॒तद् वै । \newline
57. ए॒तद् वै वा ए॒त दे॒तद् वै स्रु॒चाꣳ स्रु॒चां ॅवा ए॒त दे॒तद् वै स्रु॒चाम् । \newline
58. वै स्रु॒चाꣳ स्रु॒चां ॅवै वै स्रु॒चाꣳ रू॒पꣳ रू॒पꣳ स्रु॒चां ॅवै वै स्रु॒चाꣳ रू॒पम् । \newline
59. स्रु॒चाꣳ रू॒पꣳ रू॒पꣳ स्रु॒चाꣳ स्रु॒चाꣳ रू॒पं ॅयस्य॒ यस्य॑ रू॒पꣳ स्रु॒चाꣳ स्रु॒चाꣳ रू॒पं ॅयस्य॑ । \newline
60. रू॒पं ॅयस्य॒ यस्य॑ रू॒पꣳ रू॒पं ॅयस्यै॒वꣳरू॑पा ए॒वꣳरू॑पा॒ यस्य॑ रू॒पꣳ रू॒पं ॅयस्यै॒वꣳरू॑पाः । \newline
61. यस्यै॒वꣳरू॑पा ए॒वꣳरू॑पा॒ यस्य॒ यस्यै॒वꣳरू॑पाः॒ स्रुचः॒ स्रुच॑ ए॒वꣳरू॑पा॒ यस्य॒ यस्यै॒वꣳरू॑पाः॒ स्रुचः॑ । \newline
62. ए॒वꣳरू॑पाः॒ स्रुचः॒ स्रुच॑ ए॒वꣳरू॑पा ए॒वꣳरू॑पाः॒ स्रुचो॒ भव॑न्ति॒ भव॑न्ति॒ स्रुच॑ ए॒वꣳरू॑पा ए॒वꣳरू॑पाः॒ स्रुचो॒ भव॑न्ति । \newline
63. ए॒वꣳरू॑पा॒ इत्ये॒वं - रू॒पाः॒ । \newline
64. स्रुचो॒ भव॑न्ति॒ भव॑न्ति॒ स्रुचः॒ स्रुचो॒ भव॑न्ति॒ सर्वा॑णि॒ सर्वा॑णि॒ भव॑न्ति॒ स्रुचः॒ स्रुचो॒ भव॑न्ति॒ सर्वा॑णि । \newline
65. भव॑न्ति॒ सर्वा॑णि॒ सर्वा॑णि॒ भव॑न्ति॒ भव॑न्ति॒ सर्वा᳚ ण्ये॒वैव सर्वा॑णि॒ भव॑न्ति॒ भव॑न्ति॒ सर्वा᳚ ण्ये॒व । \newline
66. सर्वा᳚ ण्ये॒वैव सर्वा॑णि॒ सर्वा᳚ ण्ये॒वैन॑ मेन मे॒व सर्वा॑णि॒ सर्वा᳚ ण्ये॒वैन᳚म् । \newline
67. ए॒वैन॑ मेन मे॒वैवैनꣳ॑ रू॒पाणि॑ रू॒पा ण्ये॑न मे॒वैवैनꣳ॑ रू॒पाणि॑ । \newline
68. ए॒नꣳ॒॒ रू॒पाणि॑ रू॒पा ण्ये॑न मेनꣳ रू॒पाणि॑ पशू॒नाम् प॑शू॒नाꣳ रू॒पा ण्ये॑न मेनꣳ रू॒पाणि॑ पशू॒नाम् । \newline
69. रू॒पाणि॑ पशू॒नाम् प॑शू॒नाꣳ रू॒पाणि॑ रू॒पाणि॑ पशू॒ना मुपोप॑ पशू॒नाꣳ रू॒पाणि॑ रू॒पाणि॑ पशू॒ना मुप॑ । \newline
70. प॒शू॒ना मुपोप॑ पशू॒नाम् प॑शू॒ना मुप॑ तिष्ठन्ते तिष्ठन्त॒ उप॑ पशू॒नाम् प॑शू॒ना मुप॑ तिष्ठन्ते । \newline
71. उप॑ तिष्ठन्ते तिष्ठन्त॒ उपोप॑ तिष्ठन्ते॒ न न ति॑ष्ठन्त॒ उपोप॑ तिष्ठन्ते॒ न । \newline
72. ति॒ष्ठ॒न्ते॒ न न ति॑ष्ठन्ते तिष्ठन्ते॒ नास्या᳚स्य॒ न ति॑ष्ठन्ते तिष्ठन्ते॒ नास्य॑ । \newline
73. नास्या᳚स्य॒ न नास्या प॑रूप॒ मप॑रूप मस्य॒ न नास्या प॑रूपम् । \newline
74. अ॒स्या प॑रूप॒ मप॑रूप मस्या॒स्या प॑रूप मा॒त्मन् ना॒त्मन् नप॑रूप मस्या॒स्या प॑रूप मा॒त्मन्न् । \newline
75. अप॑रूप मा॒त्मन् ना॒त्मन् नप॑रूप॒ मप॑रूप मा॒त्मन् जा॑यते जायत आ॒त्मन् नप॑रूप॒ मप॑रूप मा॒त्मन् जा॑यते । \newline
76. अप॑रूप॒मित्यप॑ - रू॒प॒म् । \newline
77. आ॒त्मन् जा॑यते जायत आ॒त्मन् ना॒त्मन् जा॑यते । \newline
78. जा॒य॒त॒ इति॑ जायते । \newline
\pagebreak
\markright{ TS 3.5.8.1  \hfill https://www.vedavms.in \hfill}

\section{ TS 3.5.8.1 }

\textbf{TS 3.5.8.1 } \newline
\textbf{Samhita Paata} \newline

उ॒प॒या॒मगृ॑हीतोऽसि प्र॒जाप॑तये त्वा॒ ज्योति॑ष्मते॒ ज्योति॑ष्मन्तं गृह्णामि॒ दक्षा॑य दक्ष॒वृधे॑ रा॒तं दे॒वेभ्यो᳚ऽग्नि जि॒ह्वेभ्य॑स्त्वर्ता॒युभ्य॒ इन्द्र॑ज्येष्ठेभ्यो॒ वरु॑णराजभ्यो॒ वाता॑पिभ्यः प॒र्जन्या᳚त्मभ्यो दि॒वे त्वा॒ऽन्तरि॑क्षाय त्वा पृथि॒व्यै त्वाऽपे᳚न्द्र द्विष॒तो मनोऽप॒ जिज्या॑सतो ज॒ह्यप॒ यो नो॑ऽराती॒यति॒ तं ज॑हि प्रा॒णाय॑ त्वाऽपा॒नाय॑ त्वा व्या॒नाय॑ त्वा स॒ते त्वाऽस॑ते त्वा॒ऽद्भ्यस्त्वौष॑धीभ्यो॒ ( ) विश्वे᳚भ्यस्त्वा भू॒तेभ्यो॒ यतः॑ प्र॒जा अक्खि॑द्रा॒ अजा॑यन्त॒ तस्मै᳚ त्वा प्र॒जाप॑तये विभू॒दाव्.न्ने॒ ज्योति॑ष्मते॒ ज्योति॑ष्मन्तं जुहोमि ॥ \newline

\textbf{Pada Paata} \newline

उ॒प॒या॒मगृ॑हीत॒ इत्यु॑पया॒म - गृ॒ही॒तः॒ । अ॒सि॒ । प्र॒जाप॑तय॒ इति॑ प्र॒जा-प॒त॒ये॒ । त्वा॒ । ज्योति॑ष्मते । ज्योति॑ष्मन्तम् । गृ॒ह्णा॒मि॒ । दक्षा॑य । द॒क्ष॒वृध॒ इति॑ दक्ष - वृधे᳚ । रा॒तम् । दे॒वेभ्यः॑ । अ॒ग्नि॒जि॒ह्वेभ्य॒ इत्य॑ग्नि - जि॒ह्वेभ्यः॑ । त्वा॒ । ऋ॒ता॒युभ्य॒ इत्यृ॑ता॒यु - भ्यः॒ । इन्द्र॑ज्येष्ठेभ्य॒ इतीन्द्र॑-ज्ये॒ष्ठे॒भ्यः॒ । वरु॑णराजभ्य॒ इति॒ वरु॑णराज-भ्यः॒ । वाता॑पिभ्य॒ इति॒ वाता॑पि-भ्यः॒ । प॒र्जन्या᳚त्मभ्य॒ इति॑ प॒र्जन्या᳚त्म-भ्यः॒ । दि॒वे । त्वा॒ । अ॒न्तरि॑क्षाय । त्वा॒ । पृ॒थि॒व्यै । त्वा॒ । अपेति॑ । इ॒न्द्र॒ । द्वि॒ष॒तः । मनः॑ । अपेति॑ । जिज्या॑सतः । ज॒हि॒ । अपेति॑ । यः । नः॒ । अ॒रा॒ती॒यति॑ । तम् । ज॒हि॒ । प्रा॒णायेति॑ प्रा - अ॒नाय॑ । त्वा॒ । अ॒पा॒नायेत्य॑प-अ॒नाय॑ । त्वा॒ । व्या॒नायेति॑ वि - अ॒नाय॑ । त्वा॒ । स॒ते । त्वा॒ । अस॑ते । त्वा॒ । अ॒द्भ्य इत्य॑त् - भ्यः । त्वा॒ । ओष॑धीभ्य॒ इत्योष॑धि-भ्यः॒ ( ) । विश्वे᳚भ्यः । त्वा॒ । भू॒तेभ्यः॑ । यतः॑ । प्र॒जा इति॑ प्र - जाः । अक्खि॑द्राः । अजा॑यन्त । तस्मै᳚ । त्वा॒ । प्र॒जाप॑तय॒ इति॑ प्र॒जा - प॒त॒ये॒ । वि॒भू॒दाव्.न्न॒ इति॑ विभु - दाव्.न्ने᳚ । ज्योति॑ष्मते । ज्योति॑ष्मन्तम् । जु॒हो॒मि॒ ॥  \newline


\textbf{Krama Paata} \newline

उ॒प॒या॒मगृ॑हीतो ऽसि । उ॒प॒या॒मगृ॑हीत॒ इत्यु॑पया॒म - गृ॒ही॒तः॒ । अ॒सि॒ प्र॒जाप॑तये । प्र॒जाप॑तये त्वा । प्र॒जाप॑तय॒ इति॑ प्र॒जा - प॒त॒ये॒ । त्वा॒ ज्योति॑ष्मते । ज्योति॑ष्मते॒ ज्योति॑ष्मन्तम् । ज्योति॑ष्मन्तम् गृह्णामि । गृ॒ह्णा॒मि॒ दक्षा॑य । दक्षा॑य दक्ष॒वृधे᳚ । द॒क्ष॒वृधे॑ रा॒तम् । द॒क्ष॒वृध॒ इति॑ दक्ष - वृधे᳚ । रा॒तम् दे॒वेभ्यः॑ । दे॒वेभ्यो᳚ ऽग्निजि॒ह्वेभ्यः॑ । अ॒ग्नि॒जि॒ह्वेभ्य॑स्त्वा । अ॒ग्नि॒जि॒ह्वेभ्य॒ इत्य॑ग्नि - जि॒ह्वेभ्यः॑ । त्व॒र्ता॒युभ्यः॑ । ऋ॒ता॒युभ्य॒ इन्द्र॑ज्येष्ठेभ्यः । ऋ॒ता॒युभ्य॒ इत्यृ॑ता॒यु - भ्यः॒ । इन्द्र॑ज्येष्ठेभ्यो॒ वरु॑णराजभ्यः । इन्द्र॑ज्येष्ठेभ्य॒ इतीन्द्र॑ - ज्ये॒ष्ठे॒भ्यः॒ । वरु॑णराजभ्यो॒ वाता॑पिभ्यः । वरु॑णराजभ्य॒ इति॒ वरु॑णराज - भ्यः॒ । वाता॑पिभ्यः प॒र्जन्या᳚त्मभ्यः । वाता॑पिभ्य॒ इति॒ वाता॑पि - भ्यः॒ । प॒र्जन्या᳚त्मभ्यो दि॒वे । प॒र्जन्या᳚त्मभ्य॒ इति॑ प॒र्जन्या᳚त्म - भ्यः॒ । दि॒वे त्वा᳚ । त्वा॒ ऽन्तरि॑क्षाय । अ॒न्तरि॑क्षाय त्वा । त्वा॒ पृ॒थि॒व्यै । पृ॒थि॒व्यै त्वा᳚ । त्वा ऽप॑ । अपे᳚न्द्र । इ॒न्द्र॒ द्वि॒ष॒तः । द्वि॒ष॒तो मनः॑ । मनो ऽप॑ । अप॒ जिज्या॑सतः । जिज्या॑सतो जहि । ज॒ह्यप॑ । अप॒ यः । यो नः॑ । नो॒ ऽरा॒ती॒यति॑ । अ॒रा॒ती॒यति॒ तम् । तम् ज॑हि । ज॒हि॒ प्रा॒णाय॑ । प्रा॒णाय॑ त्वा । प्रा॒णायेति॑ प्र - अ॒नाय॑ । त्वा॒ऽपा॒नाय॑ । अ॒पा॒नाय॑ त्वा । अ॒पा॒नायेत्य॑प - अ॒नाय॑ । त्वा॒ व्या॒नाय॑ । व्या॒नाय॑ त्वा । व्या॒नायेति॑ वि - अ॒नाय॑ । त्वा॒ स॒ते । स॒ते त्वा᳚ । त्वा ऽस॑ते । अस॑ते त्वा । त्वा॒ ऽद्भ्यः । अ॒द्भ्यस्त्वा᳚ । अ॒द्भ्य इत्य॑त् - भ्यः । त्वौष॑धीभ्यः ( ) । ओष॑धीभ्यो॒ विश्वे᳚भ्यः । ओष॑धीभ्य॒ इत्योष॑धि - भ्यः॒ । विश्वे᳚भ्यस्त्वा । त्वा॒ भू॒तेभ्यः॑ । भू॒तेभ्यो॒ यतः॑ । यतः॑ प्र॒जाः । प्र॒जा अक्खि॑द्राः । प्र॒जा इति॑ प्र - जाः । अक्खि॑द्रा॒ अजा॑यन्त । अजा॑यन्त॒ तस्मै᳚ । तस्मै᳚ त्वा । त्वा॒ प्र॒जाप॑तये । प्र॒जाप॑तये विभू॒दाव्.न्ने᳚ । प्र॒जाप॑तय॒ इति॑ प्र॒जा - प॒त॒ये॒ । वि॒भू॒दाव्.न्ने॒ ज्योति॑ष्मते । वि॒भू॒दाव्.न्न॒ इति॑ विभु - दाव्.न्ने᳚ । ज्योति॑ष्मते॒ ज्योति॑ष्मन्तम् । ज्योति॑ष्मन्तम् जुहोमि । जु॒हो॒मीति॑ जुहोमि । \newline

\textbf{Jatai Paata} \newline

1. उ॒प॒या॒मगृ॑हीतो ऽस्य स्युपया॒मगृ॑हीत उपया॒मगृ॑हीतो ऽसि । \newline
2. उ॒प॒या॒मगृ॑हीत॒ इत्यु॑पया॒म - गृ॒ही॒तः॒ । \newline
3. अ॒सि॒ प्र॒जाप॑तये प्र॒जाप॑तये ऽस्यसि प्र॒जाप॑तये । \newline
4. प्र॒जाप॑तये त्वा त्वा प्र॒जाप॑तये प्र॒जाप॑तये त्वा । \newline
5. प्र॒जाप॑तय॒ इति॑ प्र॒जा - प॒त॒ये॒ । \newline
6. त्वा॒ ज्योति॑ष्मते॒ ज्योति॑ष्मते त्वा त्वा॒ ज्योति॑ष्मते । \newline
7. ज्योति॑ष्मते॒ ज्योति॑ष्मन्त॒म् ज्योति॑ष्मन्त॒म् ज्योति॑ष्मते॒ ज्योति॑ष्मते॒ ज्योति॑ष्मन्तम् । \newline
8. ज्योति॑ष्मन्तम् गृह्णामि गृह्णामि॒ ज्योति॑ष्मन्त॒म् ज्योति॑ष्मन्तम् गृह्णामि । \newline
9. गृ॒ह्णा॒मि॒ दक्षा॑य॒ दक्षा॑य गृह्णामि गृह्णामि॒ दक्षा॑य । \newline
10. दक्षा॑य दक्ष॒वृधे॑ दक्ष॒वृधे॒ दक्षा॑य॒ दक्षा॑य दक्ष॒वृधे᳚ । \newline
11. द॒क्ष॒वृधे॑ रा॒तꣳ रा॒तम् द॑क्ष॒वृधे॑ दक्ष॒वृधे॑ रा॒तम् । \newline
12. द॒क्ष॒वृध॒ इति॑ दक्ष - वृधे᳚ । \newline
13. रा॒तम् दे॒वेभ्यो॑ दे॒वेभ्यो॑ रा॒तꣳ रा॒तम् दे॒वेभ्यः॑ । \newline
14. दे॒वेभ्यो᳚ ऽग्निजि॒ह्वेभ्यो᳚ ऽग्निजि॒ह्वेभ्यो॑ दे॒वेभ्यो॑ दे॒वेभ्यो᳚ ऽग्निजि॒ह्वेभ्यः॑ । \newline
15. अ॒ग्नि॒जि॒ह्वेभ्य॑ स्त्वा त्वा ऽग्निजि॒ह्वेभ्यो᳚ ऽग्निजि॒ह्वेभ्य॑ स्त्वा । \newline
16. अ॒ग्नि॒जि॒ह्वेभ्य॒ इत्य॑ग्नि - जि॒ह्वेभ्यः॑ । \newline
17. त्व॒ र्‌ता॒युभ्य॑ ऋता॒युभ्य॑ स्त्वा त्व र्‌ता॒युभ्यः॑ । \newline
18. ऋ॒ता॒युभ्य॒ इन्द्र॑ज्येष्ठेभ्य॒ इन्द्र॑ज्येष्ठेभ्य ऋता॒युभ्य॑ ऋता॒युभ्य॒ इन्द्र॑ज्येष्ठेभ्यः । \newline
19. ऋ॒ता॒युभ्य॒ इत्यृ॑ता॒यु - भ्यः॒ । \newline
20. इन्द्र॑ज्येष्ठेभ्यो॒ वरु॑णराजभ्यो॒ वरु॑णराजभ्य॒ इन्द्र॑ज्येष्ठेभ्य॒ इन्द्र॑ज्येष्ठेभ्यो॒ वरु॑णराजभ्यः । \newline
21. इन्द्र॑ज्येष्ठेभ्य॒ इतीन्द्र॑ - ज्ये॒ष्ठे॒भ्यः॒ । \newline
22. वरु॑णराजभ्यो॒ वाता॑पिभ्यो॒ वाता॑पिभ्यो॒ वरु॑णराजभ्यो॒ वरु॑णराजभ्यो॒ वाता॑पिभ्यः । \newline
23. वरु॑णराजभ्य॒ इति॒ वरु॑णराज - भ्यः॒ । \newline
24. वाता॑पिभ्यः प॒र्जन्या᳚त्मभ्यः प॒र्जन्या᳚त्मभ्यो॒ वाता॑पिभ्यो॒ वाता॑पिभ्यः प॒र्जन्या᳚त्मभ्यः । \newline
25. वाता॑पिभ्य॒ इति॒ वाता॑पि - भ्यः॒ । \newline
26. प॒र्जन्या᳚त्मभ्यो दि॒वे दि॒वे प॒र्जन्या᳚त्मभ्यः प॒र्जन्या᳚त्मभ्यो दि॒वे । \newline
27. प॒र्जन्या᳚त्मभ्य॒ इति॑ प॒र्जन्या᳚त्म - भ्यः॒ । \newline
28. दि॒वे त्वा᳚ त्वा दि॒वे दि॒वे त्वा᳚ । \newline
29. त्वा॒ ऽन्तरि॑क्षाया॒ न्तरि॑क्षाय त्वा त्वा॒ ऽन्तरि॑क्षाय । \newline
30. अ॒न्तरि॑क्षाय त्वा त्वा॒ ऽन्तरि॑क्षाया॒ न्तरि॑क्षाय त्वा । \newline
31. त्वा॒ पृ॒थि॒व्यै पृ॑थि॒व्यै त्वा᳚ त्वा पृथि॒व्यै । \newline
32. पृ॒थि॒व्यै त्वा᳚ त्वा पृथि॒व्यै पृ॑थि॒व्यै त्वा᳚ । \newline
33. त्वा ऽपाप॑ त्वा॒ त्वा ऽप॑ । \newline
34. अपे᳚न्द्रे॒ न्द्रापा पे᳚न्द्र । \newline
35. इ॒न्द्र॒ द्वि॒ष॒तो द्वि॑ष॒त इ॑न्द्रे न्द्र द्विष॒तः । \newline
36. द्वि॒ष॒तो मनो॒ मनो᳚ द्विष॒तो द्वि॑ष॒तो मनः॑ । \newline
37. मनो ऽपाप॒ मनो॒ मनो ऽप॑ । \newline
38. अप॒ जिज्या॑सतो॒ जिज्या॑स॒तो ऽपाप॒ जिज्या॑सतः । \newline
39. जिज्या॑सतो जहि जहि॒ जिज्या॑सतो॒ जिज्या॑सतो जहि । \newline
40. ज॒ह्य पाप॑ जहि ज॒ह्यप॑ । \newline
41. अप॒ यो यो ऽपाप॒ यः । \newline
42. यो नो॑ नो॒ यो यो नः॑ । \newline
43. नो॒ ऽरा॒ती॒य त्य॑राती॒यति॑ नो नो ऽराती॒यति॑ । \newline
44. अ॒रा॒ती॒यति॒ तम् त म॑राती॒य त्य॑राती॒यति॒ तम् । \newline
45. तम् ज॑हि जहि॒ तम् तम् ज॑हि । \newline
46. ज॒हि॒ प्रा॒णाय॑ प्रा॒णाय॑ जहि जहि प्रा॒णाय॑ । \newline
47. प्रा॒णाय॑ त्वा त्वा प्रा॒णाय॑ प्रा॒णाय॑ त्वा । \newline
48. प्रा॒णायेति॑ प्र - अ॒नाय॑ । \newline
49. त्वा॒ ऽपा॒नाया॑ पा॒नाय॑ त्वा त्वा ऽपा॒नाय॑ । \newline
50. अ॒पा॒नाय॑ त्वा त्वा ऽपा॒नाया॑ पा॒नाय॑ त्वा । \newline
51. अ॒पा॒नायेत्य॑प - अ॒नाय॑ । \newline
52. त्वा॒ व्या॒नाय॑ व्या॒नाय॑ त्वा त्वा व्या॒नाय॑ । \newline
53. व्या॒नाय॑ त्वा त्वा व्या॒नाय॑ व्या॒नाय॑ त्वा । \newline
54. व्या॒नायेति॑ वि - अ॒नाय॑ । \newline
55. त्वा॒ स॒ते स॒ते त्वा᳚ त्वा स॒ते । \newline
56. स॒ते त्वा᳚ त्वा स॒ते स॒ते त्वा᳚ । \newline
57. त्वा ऽस॒ते ऽस॑ते त्वा॒ त्वा ऽस॑ते । \newline
58. अस॑ते त्वा॒ त्वा ऽस॒ते ऽस॑ते त्वा । \newline
59. त्वा॒ ऽद्भ्यो᳚ ऽद्भ्य स्त्वा᳚ त्वा॒ ऽद्भ्यः । \newline
60. अ॒द्भ्य स्त्वा᳚ त्वा॒ ऽद्भ्यो᳚ ऽद्भ्य स्त्वा᳚ । \newline
61. अ॒द्भ्य इत्य॑त् - भ्यः । \newline
62. त्वौष॑धीभ्य॒ ओष॑धीभ्य स्त्वा॒ त्वौष॑धीभ्यः । \newline
63. ओष॑धीभ्यो॒ विश्वे᳚भ्यो॒ विश्वे᳚भ्य॒ ओष॑धीभ्य॒ ओष॑धीभ्यो॒ विश्वे᳚भ्यः । \newline
64. ओष॑धीभ्य॒ इत्योष॑धि - भ्यः॒ । \newline
65. विश्वे᳚भ्य स्त्वा त्वा॒ विश्वे᳚भ्यो॒ विश्वे᳚भ्य स्त्वा । \newline
66. त्वा॒ भू॒तेभ्यो॑ भू॒तेभ्य॑ स्त्वा त्वा भू॒तेभ्यः॑ । \newline
67. भू॒तेभ्यो॒ यतो॒ यतो॑ भू॒तेभ्यो॑ भू॒तेभ्यो॒ यतः॑ । \newline
68. यतः॑ प्र॒जाः प्र॒जा यतो॒ यतः॑ प्र॒जाः । \newline
69. प्र॒जा अक्खि॑द्रा॒ अक्खि॑द्राः प्र॒जाः प्र॒जा अक्खि॑द्राः । \newline
70. प्र॒जा इति॑ प्र - जाः । \newline
71. अक्खि॑द्रा॒ अजा॑य॒न्ता जा॑य॒न्ता क्खि॑द्रा॒ अक्खि॑द्रा॒ अजा॑यन्त । \newline
72. अजा॑यन्त॒ तस्मै॒ तस्मा॒ अजा॑य॒न्ता जा॑यन्त॒ तस्मै᳚ । \newline
73. तस्मै᳚ त्वा त्वा॒ तस्मै॒ तस्मै᳚ त्वा । \newline
74. त्वा॒ प्र॒जाप॑तये प्र॒जाप॑तये त्वा त्वा प्र॒जाप॑तये । \newline
75. प्र॒जाप॑तये विभू॒दाव्.न्ने॑ विभू॒दाव्.न्ने᳚ प्र॒जाप॑तये प्र॒जाप॑तये विभू॒दाव्.न्ने᳚ । \newline
76. प्र॒जाप॑तय॒ इति॑ प्र॒जा - प॒त॒ये॒ । \newline
77. वि॒भू॒दाव्.न्ने॒ ज्योति॑ष्मते॒ ज्योति॑ष्मते विभू॒दाव्.न्ने॑ विभू॒दाव्.न्ने॒ ज्योति॑ष्मते । \newline
78. वि॒भू॒दाव्.न्न॒ इति॑ विभु - दाव्.न्ने᳚ । \newline
79. ज्योति॑ष्मते॒ ज्योति॑ष्मन्त॒म् ज्योति॑ष्मन्त॒म् ज्योति॑ष्मते॒ ज्योति॑ष्मते॒ ज्योति॑ष्मन्तम् । \newline
80. ज्योति॑ष्मन्तम् जुहोमि जुहोमि॒ ज्योति॑ष्मन्त॒म् ज्योति॑ष्मन्तम् जुहोमि । \newline
81. जु॒हो॒मीति॑ जुहोमि । \newline

\textbf{Ghana Paata } \newline

1. उ॒प॒या॒मगृ॑हीतो ऽस्य स्युपया॒मगृ॑हीत उपया॒मगृ॑हीतो ऽसि प्र॒जाप॑तये प्र॒जाप॑तये ऽस्युपया॒मगृ॑हीत उपया॒मगृ॑हीतो ऽसि प्र॒जाप॑तये । \newline
2. उ॒प॒या॒मगृ॑हीत॒ इत्यु॑पया॒म - गृ॒ही॒तः॒ । \newline
3. अ॒सि॒ प्र॒जाप॑तये प्र॒जाप॑तये ऽस्यसि प्र॒जाप॑तये त्वा त्वा प्र॒जाप॑तये ऽस्यसि प्र॒जाप॑तये त्वा । \newline
4. प्र॒जाप॑तये त्वा त्वा प्र॒जाप॑तये प्र॒जाप॑तये त्वा॒ ज्योति॑ष्मते॒ ज्योति॑ष्मते त्वा प्र॒जाप॑तये प्र॒जाप॑तये त्वा॒ ज्योति॑ष्मते । \newline
5. प्र॒जाप॑तय॒ इति॑ प्र॒जा - प॒त॒ये॒ । \newline
6. त्वा॒ ज्योति॑ष्मते॒ ज्योति॑ष्मते त्वा त्वा॒ ज्योति॑ष्मते॒ ज्योति॑ष्मन्त॒म् ज्योति॑ष्मन्त॒म् ज्योति॑ष्मते त्वा त्वा॒ ज्योति॑ष्मते॒ ज्योति॑ष्मन्तम् । \newline
7. ज्योति॑ष्मते॒ ज्योति॑ष्मन्त॒म् ज्योति॑ष्मन्त॒म् ज्योति॑ष्मते॒ ज्योति॑ष्मते॒ ज्योति॑ष्मन्तम् गृह्णामि गृह्णामि॒ ज्योति॑ष्मन्त॒म् ज्योति॑ष्मते॒ ज्योति॑ष्मते॒ ज्योति॑ष्मन्तम् गृह्णामि । \newline
8. ज्योति॑ष्मन्तम् गृह्णामि गृह्णामि॒ ज्योति॑ष्मन्त॒म् ज्योति॑ष्मन्तम् गृह्णामि॒ दक्षा॑य॒ दक्षा॑य गृह्णामि॒ ज्योति॑ष्मन्त॒म् ज्योति॑ष्मन्तम् गृह्णामि॒ दक्षा॑य । \newline
9. गृ॒ह्णा॒मि॒ दक्षा॑य॒ दक्षा॑य गृह्णामि गृह्णामि॒ दक्षा॑य दक्ष॒वृधे॑ दक्ष॒वृधे॒ दक्षा॑य गृह्णामि गृह्णामि॒ दक्षा॑य दक्ष॒वृधे᳚ । \newline
10. दक्षा॑य दक्ष॒वृधे॑ दक्ष॒वृधे॒ दक्षा॑य॒ दक्षा॑य दक्ष॒वृधे॑ रा॒तꣳ रा॒तम् द॑क्ष॒वृधे॒ दक्षा॑य॒ दक्षा॑य दक्ष॒वृधे॑ रा॒तम् । \newline
11. द॒क्ष॒वृधे॑ रा॒तꣳ रा॒तम् द॑क्ष॒वृधे॑ दक्ष॒वृधे॑ रा॒तम् दे॒वेभ्यो॑ दे॒वेभ्यो॑ रा॒तम् द॑क्ष॒वृधे॑ दक्ष॒वृधे॑ रा॒तम् दे॒वेभ्यः॑ । \newline
12. द॒क्ष॒वृध॒ इति॑ दक्ष - वृधे᳚ । \newline
13. रा॒तम् दे॒वेभ्यो॑ दे॒वेभ्यो॑ रा॒तꣳ रा॒तम् दे॒वेभ्यो᳚ ऽग्निजि॒ह्वेभ्यो᳚ ऽग्निजि॒ह्वेभ्यो॑ दे॒वेभ्यो॑ रा॒तꣳ रा॒तम् दे॒वेभ्यो᳚ ऽग्निजि॒ह्वेभ्यः॑ । \newline
14. दे॒वेभ्यो᳚ ऽग्निजि॒ह्वेभ्यो᳚ ऽग्निजि॒ह्वेभ्यो॑ दे॒वेभ्यो॑ दे॒वेभ्यो᳚ ऽग्निजि॒ह्वेभ्य॑ स्त्वा त्वा ऽग्निजि॒ह्वेभ्यो॑ दे॒वेभ्यो॑ दे॒वेभ्यो᳚ ऽग्निजि॒ह्वेभ्य॑ स्त्वा । \newline
15. अ॒ग्नि॒जि॒ह्वेभ्य॑ स्त्वा त्वा ऽग्निजि॒ह्वेभ्यो᳚ ऽग्निजि॒ह्वेभ्य॑ स्त्व र्‌ता॒युभ्य॑ ऋता॒युभ्य॑ स्त्वा ऽग्निजि॒ह्वेभ्यो᳚ ऽग्निजि॒ह्वेभ्य॑ स्त्व र्‌ता॒युभ्यः॑ । \newline
16. अ॒ग्नि॒जि॒ह्वेभ्य॒ इत्य॑ग्नि - जि॒ह्वेभ्यः॑ । \newline
17. त्व॒ र्‌ता॒युभ्य॑ ऋता॒युभ्य॑ स्त्वा त्व र्‌ता॒युभ्य॒ इन्द्र॑ज्येष्ठेभ्य॒ इन्द्र॑ज्येष्ठेभ्य ऋता॒युभ्य॑ स्त्वा त्व र्‌ता॒युभ्य॒ इन्द्र॑ज्येष्ठेभ्यः । \newline
18. ऋ॒ता॒युभ्य॒ इन्द्र॑ज्येष्ठेभ्य॒ इन्द्र॑ज्येष्ठेभ्य ऋता॒युभ्य॑ ऋता॒युभ्य॒ इन्द्र॑ज्येष्ठेभ्यो॒ वरु॑णराजभ्यो॒ वरु॑णराजभ्य॒ इन्द्र॑ज्येष्ठेभ्य ऋता॒युभ्य॑ ऋता॒युभ्य॒ इन्द्र॑ज्येष्ठेभ्यो॒ वरु॑णराजभ्यः । \newline
19. ऋ॒ता॒युभ्य॒ इत्यृ॑ता॒यु - भ्यः॒ । \newline
20. इन्द्र॑ज्येष्ठेभ्यो॒ वरु॑णराजभ्यो॒ वरु॑णराजभ्य॒ इन्द्र॑ज्येष्ठेभ्य॒ इन्द्र॑ज्येष्ठेभ्यो॒ वरु॑णराजभ्यो॒ वाता॑पिभ्यो॒ वाता॑पिभ्यो॒ वरु॑णराजभ्य॒ इन्द्र॑ज्येष्ठेभ्य॒ इन्द्र॑ज्येष्ठेभ्यो॒ वरु॑णराजभ्यो॒ वाता॑पिभ्यः । \newline
21. इन्द्र॑ज्येष्ठेभ्य॒ इतीन्द्र॑ - ज्ये॒ष्ठे॒भ्यः॒ । \newline
22. वरु॑णराजभ्यो॒ वाता॑पिभ्यो॒ वाता॑पिभ्यो॒ वरु॑णराजभ्यो॒ वरु॑णराजभ्यो॒ वाता॑पिभ्यः प॒र्जन्या᳚त्मभ्यः 
प॒र्जन्या᳚त्मभ्यो॒ वाता॑पिभ्यो॒ वरु॑णराजभ्यो॒ वरु॑णराजभ्यो॒ वाता॑पिभ्यः प॒र्जन्या᳚त्मभ्यः । \newline
23. वरु॑णराजभ्य॒ इति॒ वरु॑णराज - भ्यः॒ । \newline
24. वाता॑पिभ्यः प॒र्जन्या᳚त्मभ्यः प॒र्जन्या᳚त्मभ्यो॒ वाता॑पिभ्यो॒ वाता॑पिभ्यः प॒र्जन्या᳚त्मभ्यो दि॒वे दि॒वे प॒र्जन्या᳚त्मभ्यो॒ वाता॑पिभ्यो॒ वाता॑पिभ्यः प॒र्जन्या᳚त्मभ्यो दि॒वे । \newline
25. वाता॑पिभ्य॒ इति॒ वाता॑पि - भ्यः॒ । \newline
26. प॒र्जन्या᳚त्मभ्यो दि॒वे दि॒वे प॒र्जन्या᳚त्मभ्यः प॒र्जन्या᳚त्मभ्यो दि॒वे त्वा᳚ त्वा दि॒वे प॒र्जन्या᳚त्मभ्यः प॒र्जन्या᳚त्मभ्यो दि॒वे त्वा᳚ । \newline
27. प॒र्जन्या᳚त्मभ्य॒ इति॑ प॒र्जन्या᳚त्म - भ्यः॒ । \newline
28. दि॒वे त्वा᳚ त्वा दि॒वे दि॒वे त्वा॒ ऽन्तरि॑क्षाया॒ न्तरि॑क्षाय त्वा दि॒वे दि॒वे त्वा॒ ऽन्तरि॑क्षाय । \newline
29. त्वा॒ ऽन्तरि॑क्षाया॒ न्तरि॑क्षाय त्वा त्वा॒ ऽन्तरि॑क्षाय त्वा त्वा॒ ऽन्तरि॑क्षाय त्वा त्वा॒ ऽन्तरि॑क्षाय त्वा । \newline
30. अ॒न्तरि॑क्षाय त्वा त्वा॒ ऽन्तरि॑क्षाया॒ न्तरि॑क्षाय त्वा पृथि॒व्यै पृ॑थि॒व्यै त्वा॒ ऽन्तरि॑क्षाया॒ न्तरि॑क्षाय त्वा पृथि॒व्यै । \newline
31. त्वा॒ पृ॒थि॒व्यै पृ॑थि॒व्यै त्वा᳚ त्वा पृथि॒व्यै त्वा᳚ त्वा पृथि॒व्यै त्वा᳚ त्वा पृथि॒व्यै त्वा᳚ । \newline
32. पृ॒थि॒व्यै त्वा᳚ त्वा पृथि॒व्यै पृ॑थि॒व्यै त्वा ऽपाप॑ त्वा पृथि॒व्यै पृ॑थि॒व्यै त्वा ऽप॑ । \newline
33. त्वा ऽपाप॑ त्वा॒ त्वा ऽपे᳚न्द्रे॒ न्द्राप॑ त्वा॒ त्वा ऽपे᳚न्द्र । \newline
34. अपे᳚न्द्रे॒ न्द्रापापे᳚न्द्र द्विष॒तो द्वि॑ष॒त इ॒न्द्रा पापे᳚न्द्र द्विष॒तः । \newline
35. इ॒न्द्र॒ द्वि॒ष॒तो द्वि॑ष॒त इ॑न्द्रेन्द्र द्विष॒तो मनो॒ मनो᳚ द्विष॒त इ॑न्द्रेन्द्र द्विष॒तो मनः॑ । \newline
36. द्वि॒ष॒तो मनो॒ मनो᳚ द्विष॒तो द्वि॑ष॒तो मनो ऽपाप॒ मनो᳚ द्विष॒तो द्वि॑ष॒तो मनो ऽप॑ । \newline
37. मनो ऽपाप॒ मनो॒ मनो ऽप॒ जिज्या॑सतो॒ जिज्या॑स॒तो ऽप॒ मनो॒ मनो ऽप॒ जिज्या॑सतः । \newline
38. अप॒ जिज्या॑सतो॒ जिज्या॑स॒तो ऽपाप॒ जिज्या॑सतो जहि जहि॒ जिज्या॑स॒तो ऽपाप॒ जिज्या॑सतो जहि । \newline
39. जिज्या॑सतो जहि जहि॒ जिज्या॑सतो॒ जिज्या॑सतो ज॒ह्यपाप॑ जहि॒ जिज्या॑सतो॒ जिज्या॑सतो ज॒ह्यप॑ । \newline
40. ज॒ह्यपाप॑ जहि ज॒ह्यप॒ यो यो ऽप॑ जहि ज॒ह्यप॒ यः । \newline
41. अप॒ यो यो ऽपाप॒ यो नो॑ नो॒ यो ऽपाप॒ यो नः॑ । \newline
42. यो नो॑ नो॒ यो यो नो॑ ऽराती॒य त्य॑राती॒यति॑ नो॒ यो यो नो॑ ऽराती॒यति॑ । \newline
43. नो॒ ऽरा॒ती॒य त्य॑राती॒यति॑ नो नो ऽराती॒यति॒ तम् त म॑राती॒यति॑ नो नो ऽराती॒यति॒ तम् । \newline
44. अ॒रा॒ती॒यति॒ तम् त म॑राती॒य त्य॑राती॒यति॒ तम् ज॑हि जहि॒ त म॑राती॒य त्य॑राती॒यति॒ तम् ज॑हि । \newline
45. तम् ज॑हि जहि॒ तम् तम् ज॑हि प्रा॒णाय॑ प्रा॒णाय॑ जहि॒ तम् तम् ज॑हि प्रा॒णाय॑ । \newline
46. ज॒हि॒ प्रा॒णाय॑ प्रा॒णाय॑ जहि जहि प्रा॒णाय॑ त्वा त्वा प्रा॒णाय॑ जहि जहि प्रा॒णाय॑ त्वा । \newline
47. प्रा॒णाय॑ त्वा त्वा प्रा॒णाय॑ प्रा॒णाय॑ त्वा ऽपा॒नाया॑ पा॒नाय॑ त्वा प्रा॒णाय॑ प्रा॒णाय॑ त्वा ऽपा॒नाय॑ । \newline
48. प्रा॒णायेति॑ प्र - अ॒नाय॑ । \newline
49. त्वा॒ ऽपा॒नाया॑ पा॒नाय॑ त्वा त्वा ऽपा॒नाय॑ त्वा त्वा ऽपा॒नाय॑ त्वा त्वा ऽपा॒नाय॑ त्वा । \newline
50. अ॒पा॒नाय॑ त्वा त्वा ऽपा॒नाया॑ पा॒नाय॑ त्वा व्या॒नाय॑ व्या॒नाय॑ त्वा ऽपा॒नाया॑ पा॒नाय॑ त्वा व्या॒नाय॑ । \newline
51. अ॒पा॒नायेत्य॑प - अ॒नाय॑ । \newline
52. त्वा॒ व्या॒नाय॑ व्या॒नाय॑ त्वा त्वा व्या॒नाय॑ त्वा त्वा व्या॒नाय॑ त्वा त्वा व्या॒नाय॑ त्वा । \newline
53. व्या॒नाय॑ त्वा त्वा व्या॒नाय॑ व्या॒नाय॑ त्वा स॒ते स॒ते त्वा᳚ व्या॒नाय॑ व्या॒नाय॑ त्वा स॒ते । \newline
54. व्या॒नायेति॑ वि - अ॒नाय॑ । \newline
55. त्वा॒ स॒ते स॒ते त्वा᳚ त्वा स॒ते त्वा᳚ त्वा स॒ते त्वा᳚ त्वा स॒ते त्वा᳚ । \newline
56. स॒ते त्वा᳚ त्वा स॒ते स॒ते त्वा ऽस॒ते ऽस॑ते त्वा स॒ते स॒ते त्वा ऽस॑ते । \newline
57. त्वा ऽस॒ते ऽस॑ते त्वा॒ त्वा ऽस॑ते त्वा॒ त्वा ऽस॑ते त्वा॒ त्वा ऽस॑ते त्वा । \newline
58. अस॑ते त्वा॒ त्वा ऽस॒ते ऽस॑ते त्वा॒ ऽद्भ्यो᳚ ऽद्भ्य स्त्वा ऽस॒ते ऽस॑ते त्वा॒ ऽद्भ्यः । \newline
59. त्वा॒ ऽद्भ्यो᳚ ऽद्भ्य स्त्वा᳚ त्वा॒ ऽद्भ्य स्त्वा᳚ त्वा॒ ऽद्भ्य स्त्वा᳚ त्वा॒ ऽद्भ्य स्त्वा᳚ । \newline
60. अ॒द्भ्य स्त्वा᳚ त्वा॒ ऽद्भ्यो᳚ ऽद्भ्य स्त्वौष॑धीभ्य॒ ओष॑धीभ्य स्त्वा॒ ऽद्भ्यो᳚ ऽद्भ्य स्त्वौष॑धीभ्यः । \newline
61. अ॒द्भ्य इत्य॑त् - भ्यः । \newline
62. त्वौष॑धीभ्य॒ ओष॑धीभ्य स्त्वा॒ त्वौष॑धीभ्यो॒ विश्वे᳚भ्यो॒ विश्वे᳚भ्य॒ ओष॑धीभ्य स्त्वा॒ त्वौष॑धीभ्यो॒ विश्वे᳚भ्यः । \newline
63. ओष॑धीभ्यो॒ विश्वे᳚भ्यो॒ विश्वे᳚भ्य॒ ओष॑धीभ्य॒ ओष॑धीभ्यो॒ विश्वे᳚भ्य स्त्वा त्वा॒ विश्वे᳚भ्य॒ ओष॑धीभ्य॒ ओष॑धीभ्यो॒ विश्वे᳚भ्य स्त्वा । \newline
64. ओष॑धीभ्य॒ इत्योष॑धि - भ्यः॒ । \newline
65. विश्वे᳚भ्य स्त्वा त्वा॒ विश्वे᳚भ्यो॒ विश्वे᳚भ्य स्त्वा भू॒तेभ्यो॑ भू॒तेभ्य॑ स्त्वा॒ विश्वे᳚भ्यो॒ विश्वे᳚भ्य स्त्वा भू॒तेभ्यः॑ । \newline
66. त्वा॒ भू॒तेभ्यो॑ भू॒तेभ्य॑ स्त्वा त्वा भू॒तेभ्यो॒ यतो॒ यतो॑ भू॒तेभ्य॑ स्त्वा त्वा भू॒तेभ्यो॒ यतः॑ । \newline
67. भू॒तेभ्यो॒ यतो॒ यतो॑ भू॒तेभ्यो॑ भू॒तेभ्यो॒ यतः॑ प्र॒जाः प्र॒जा यतो॑ भू॒तेभ्यो॑ भू॒तेभ्यो॒ यतः॑ प्र॒जाः । \newline
68. यतः॑ प्र॒जाः प्र॒जा यतो॒ यतः॑ प्र॒जा अक्खि॑द्रा॒ अक्खि॑द्राः प्र॒जा यतो॒ यतः॑ प्र॒जा अक्खि॑द्राः । \newline
69. प्र॒जा अक्खि॑द्रा॒ अक्खि॑द्राः प्र॒जाः प्र॒जा अक्खि॑द्रा॒ अजा॑य॒न्ता जा॑य॒न्ता क्खि॑द्राः प्र॒जाः प्र॒जा अक्खि॑द्रा॒ अजा॑यन्त । \newline
70. प्र॒जा इति॑ प्र - जाः । \newline
71. अक्खि॑द्रा॒ अजा॑य॒न्ता जा॑य॒न्ता क्खि॑द्रा॒ अक्खि॑द्रा॒ अजा॑यन्त॒ तस्मै॒ तस्मा॒ अजा॑य॒न्ता क्खि॑द्रा॒ अक्खि॑द्रा॒ अजा॑यन्त॒ तस्मै᳚ । \newline
72. अजा॑यन्त॒ तस्मै॒ तस्मा॒ अजा॑य॒न्ता जा॑यन्त॒ तस्मै᳚ त्वा त्वा॒ तस्मा॒ अजा॑य॒न्ता जा॑यन्त॒ तस्मै᳚ त्वा । \newline
73. तस्मै᳚ त्वा त्वा॒ तस्मै॒ तस्मै᳚ त्वा प्र॒जाप॑तये प्र॒जाप॑तये त्वा॒ तस्मै॒ तस्मै᳚ त्वा प्र॒जाप॑तये । \newline
74. त्वा॒ प्र॒जाप॑तये प्र॒जाप॑तये त्वा त्वा प्र॒जाप॑तये विभू॒दाव्.न्ने॑ विभू॒दाव्.न्ने᳚ प्र॒जाप॑तये त्वा त्वा प्र॒जाप॑तये विभू॒दाव्.न्ने᳚ । \newline
75. प्र॒जाप॑तये विभू॒दाव्.न्ने॑ विभू॒दाव्.न्ने᳚ प्र॒जाप॑तये प्र॒जाप॑तये विभू॒दाव्.न्ने॒ ज्योति॑ष्मते॒ ज्योति॑ष्मते विभू॒दाव्.न्ने᳚ प्र॒जाप॑तये प्र॒जाप॑तये विभू॒दाव्.न्ने॒ ज्योति॑ष्मते । \newline
76. प्र॒जाप॑तय॒ इति॑ प्र॒जा - प॒त॒ये॒ । \newline
77. वि॒भू॒दाव्.न्ने॒ ज्योति॑ष्मते॒ ज्योति॑ष्मते विभू॒दाव्.न्ने॑ विभू॒दाव्.न्ने॒ ज्योति॑ष्मते॒ ज्योति॑ष्मन्त॒म् ज्योति॑ष्मन्त॒म् ज्योति॑ष्मते विभू॒दाव्.न्ने॑ विभू॒दाव्.न्ने॒ ज्योति॑ष्मते॒ ज्योति॑ष्मन्तम् । \newline
78. वि॒भू॒दाव्.न्न॒ इति॑ विभु - दाव्.न्ने᳚ । \newline
79. ज्योति॑ष्मते॒ ज्योति॑ष्मन्त॒म् ज्योति॑ष्मन्त॒म् ज्योति॑ष्मते॒ ज्योति॑ष्मते॒ ज्योति॑ष्मन्तम् जुहोमि जुहोमि॒ ज्योति॑ष्मन्त॒म् ज्योति॑ष्मते॒ ज्योति॑ष्मते॒ ज्योति॑ष्मन्तम् जुहोमि । \newline
80. ज्योति॑ष्मन्तम् जुहोमि जुहोमि॒ ज्योति॑ष्मन्त॒म् ज्योति॑ष्मन्तम् जुहोमि । \newline
81. जु॒हो॒मीति॑ जुहोमि । \newline
\pagebreak
\markright{ TS 3.5.9.1  \hfill https://www.vedavms.in \hfill}

\section{ TS 3.5.9.1 }

\textbf{TS 3.5.9.1 } \newline
\textbf{Samhita Paata} \newline

यां ॅवा अ॑द्ध्व॒र्युश्च॒ यज॑मानश्च दे॒वता॑मन्तरि॒तस्तस्या॒ आ वृ॑श्च्येते प्राजाप॒त्यं द॑धिग्र॒हं गृ॑ह्णीयात् प्र॒जाप॑तिः॒ सर्वा॑ दे॒वता॑ दे॒वता᳚भ्य ए॒व निह्नु॑वाते ज्ये॒ष्ठो वा ए॒ष ग्रहा॑णां॒ ॅयस्यै॒ष गृ॒ह्यते॒ ज्यैष्ठ्य॑मे॒व ग॑च्छति॒ सर्वा॑सां॒ ॅवा ए॒तद्दे॒वता॑नाꣳ रू॒पं ॅयदे॒ष ग्रहो॒ यस्यै॒ष गृ॒ह्यते॒ सर्वा᳚ण्ये॒वैनꣳ॑ रू॒पाणि॑ पशू॒नामुप॑तिष्ठन्त उपया॒मगृ॑हीतो - [  ] \newline

\textbf{Pada Paata} \newline

याम् । वै । अ॒द्ध्व॒र्युः । च॒ । यज॑मानः । च॒ । दे॒वता᳚म् । अ॒न्त॒रि॒त इत्य॑न्तः - इ॒तः । तस्यै᳚ । एति॑ । वृ॒श्च्ये॒ते॒ इति॑ । प्रा॒जा॒प॒त्यमिति॑ प्राजा - प॒त्यम् । द॒धि॒ग्र॒हमिति॑ दधि - ग्र॒हम् । गृ॒ह्णी॒या॒त् । प्र॒जाप॑ति॒रिति॑ प्र॒जा - प॒तिः॒ । सर्वाः᳚ । दे॒वताः᳚ । दे॒वता᳚भ्यः । ए॒व । नीति॑ । ह्नु॒वा॒ते॒ इति॑ । ज्ये॒ष्ठः । वै । ए॒षः । ग्रहा॑णाम् । यस्य॑ । ए॒षः । गृ॒ह्यते᳚ । ज्यैष्ट्य᳚म् । ए॒व । ग॒च्छ॒ति॒ । सर्वा॑साम् । वै । ए॒तत् । दे॒वता॑नाम् । रू॒पम् । यत् । ए॒षः । ग्रहः॑ । यस्य॑ । ए॒षः । गृ॒ह्यते᳚ । सर्वा॑णि । ए॒व । ए॒न॒म् । रू॒पाणि॑ । प॒शू॒नाम् । उपेति॑ । ति॒ष्ठ॒न्ते॒ । उ॒प॒या॒मगृ॑हीत॒ इत्यु॑पया॒म - गृ॒ही॒तः॒ ।  \newline


\textbf{Krama Paata} \newline

यां ॅवै । वा अ॑द्ध्व॒र्युः । अ॒द्ध्व॒र्युश्च॑ । च॒ यज॑मानः । यज॑मानश्च । च॒ दे॒वता᳚म् । दे॒वता॑मन्तरि॒तः । अ॒न्त॒रि॒तस्तस्यै᳚ । अ॒न्त॒रि॒त इत्य॑न्तः - इ॒तः । तस्या॒ आ । आ वृ॑श्च्येते । वृ॒श्च्ये॒ते॒ प्रा॒जा॒प॒त्यम् । वृ॒श्च्ये॒ते॒ इति॑ वृश्च्येते । प्रा॒जा॒प॒त्यम् द॑धिग्र॒हम् । प्रा॒जा॒प॒त्यमिति॑ प्राजा - प॒त्यम् । द॒धि॒ग्र॒हम् गृ॑ह्णीयात् । द॒धि॒ग्र॒हमिति॑ दधि - ग्र॒हम् । गृ॒ह्णी॒या॒त् 
प्र॒जाप॑तिः । प्र॒जाप॑तिः॒ सर्वाः᳚ । प्र॒जाप॑ति॒रिति॑ प्र॒जा - प॒तिः॒ । सर्वा॑ दे॒वताः᳚ । दे॒वता॑ दे॒वता᳚भ्यः । दे॒वता᳚भ्य ए॒व । ए॒व नि । नि ह्नु॑वाते । ह्नु॒वा॒ते॒ ज्ये॒ष्ठः । ह्नु॒वा॒ते॒ इति॑ ह्नुवाते । ज्ये॒ष्ठो 
वै । वा ए॒षः । ए॒ष ग्रहा॑णाम् । ग्रहा॑णां॒ ॅयस्य॑ । यस्यै॒षः । ए॒ष गृ॒ह्यते᳚ । गृ॒ह्यते॒ ज्यैष्ठ्य᳚म् । ज्यैष्ठ्य॑मे॒व । ए॒व ग॑च्छति । ग॒च्छ॒ति॒ सर्वा॑साम् । सर्वा॑सां॒ ॅवै । वा ए॒तत् । ए॒तद् दे॒वता॑नाम् । दे॒वता॑नाꣳ रू॒पम् । रू॒पं ॅयत् । यदे॒षः । ए॒ष ग्रहः॑ । ग्रहो॒ यस्य॑ । यस्यै॒षः । ए॒ष गृ॒ह्यते᳚ । गृ॒ह्यते॒ सर्वा॑णि । सर्वा᳚ण्ये॒व । ए॒वैन᳚म् । ए॒नꣳ॒॒ रू॒पाणि॑ । रू॒पाणि॑ पशू॒नाम् । प॒शू॒नामुप॑ । उप॑ तिष्ठन्ते । ति॒ष्ठ॒न्त॒ उ॒प॒या॒मगृ॑हीतः । उ॒प॒या॒मगृ॑हीतो ऽसि । उ॒प॒या॒मगृ॑हीत॒ इत्यु॑पया॒म - गृ॒ही॒तः॒ \newline

\textbf{Jatai Paata} \newline

1. यां ॅवै वै यां ॅयां ॅवै । \newline
2. वा अ॑द्ध्व॒र्यु र॑द्ध्व॒र्युर् वै वा अ॑द्ध्व॒र्युः । \newline
3. अ॒द्ध्व॒र्युश्च॑ चाद्ध्व॒र्यु र॑द्ध्व॒र्युश्च॑ । \newline
4. च॒ यज॑मानो॒ यज॑मानश्च च॒ यज॑मानः । \newline
5. यज॑मानश्च च॒ यज॑मानो॒ यज॑मानश्च । \newline
6. च॒ दे॒वता᳚म् दे॒वता᳚म् च च दे॒वता᳚म् । \newline
7. दे॒वता॑ मन्तरि॒तो᳚ ऽन्तरि॒तो दे॒वता᳚म् दे॒वता॑ मन्तरि॒तः । \newline
8. अ॒न्त॒रि॒त स्तस्यै॒ तस्या॑ अन्तरि॒तो᳚ ऽन्तरि॒त स्तस्यै᳚ । \newline
9. अ॒न्त॒रि॒त इत्य॑न्तः - इ॒तः । \newline
10. तस्या॒ आ तस्यै॒ तस्या॒ आ । \newline
11. आ वृ॑श्च्येते वृश्च्येते॒ आ वृ॑श्च्येते । \newline
12. वृ॒श्च्ये॒ते॒ प्रा॒जा॒प॒त्यम् प्रा॑जाप॒त्यं ॅवृ॑श्च्येते वृश्च्येते प्राजाप॒त्यम् । \newline
13. वृ॒श्च्ये॒ते॒ इति॑ वृश्च्येते । \newline
14. प्रा॒जा॒प॒त्यम् द॑धिग्र॒हम् द॑धिग्र॒हम् प्रा॑जाप॒त्यम् प्रा॑जाप॒त्यम् द॑धिग्र॒हम् । \newline
15. प्रा॒जा॒प॒त्यमिति॑ प्राजा - प॒त्यम् । \newline
16. द॒धि॒ग्र॒हम् गृ॑ह्णीयाद् गृह्णीयाद् दधिग्र॒हम् द॑धिग्र॒हम् गृ॑ह्णीयात् । \newline
17. द॒धि॒ग्र॒हमिति॑ दधि - ग्र॒हम् । \newline
18. गृ॒ह्णी॒या॒त् प्र॒जाप॑तिः प्र॒जाप॑तिर् गृह्णीयाद् गृह्णीयात् प्र॒जाप॑तिः । \newline
19. प्र॒जाप॑तिः॒ सर्वाः॒ सर्वाः᳚ प्र॒जाप॑तिः प्र॒जाप॑तिः॒ सर्वाः᳚ । \newline
20. प्र॒जाप॑ति॒रिति॑ प्र॒जा - प॒तिः॒ । \newline
21. सर्वा॑ दे॒वता॑ दे॒वताः॒ सर्वाः॒ सर्वा॑ दे॒वताः᳚ । \newline
22. दे॒वता॑ दे॒वता᳚भ्यो दे॒वता᳚भ्यो दे॒वता॑ दे॒वता॑ दे॒वता᳚भ्यः । \newline
23. दे॒वता᳚भ्य ए॒वैव दे॒वता᳚भ्यो दे॒वता᳚भ्य ए॒व । \newline
24. ए॒व नि न्ये॑वैव नि । \newline
25. नि ह्नु॑वाते ह्नुवाते॒ नि नि ह्नु॑वाते । \newline
26. ह्नु॒वा॒ते॒ ज्ये॒ष्ठो ज्ये॒ष्ठो ह्नु॑वाते ह्नुवाते ज्ये॒ष्ठः । \newline
27. ह्नु॒वा॒ते॒ इति॑ ह्नुवाते । \newline
28. ज्ये॒ष्ठो वै वै ज्ये॒ष्ठो ज्ये॒ष्ठो वै । \newline
29. वा ए॒ष ए॒ष वै वा ए॒षः । \newline
30. ए॒ष ग्रहा॑णा॒म् ग्रहा॑णा मे॒ष ए॒ष ग्रहा॑णाम् । \newline
31. ग्रहा॑णां॒ ॅयस्य॒ यस्य॒ ग्रहा॑णा॒म् ग्रहा॑णां॒ ॅयस्य॑ । \newline
32. यस्यै॒ष ए॒ष यस्य॒ यस्यै॒षः । \newline
33. ए॒ष गृ॒ह्यते॑ गृ॒ह्यत॑ ए॒ष ए॒ष गृ॒ह्यते᳚ । \newline
34. गृ॒ह्यते॒ ज्यैष्ट्य॒म् ज्यैष्ट्य॑म् गृ॒ह्यते॑ गृ॒ह्यते॒ ज्यैष्ट्य᳚म् । \newline
35. ज्यैष्ट्य॑ मे॒वैव ज्यैष्ट्य॒म् ज्यैष्ट्य॑ मे॒व । \newline
36. ए॒व ग॑च्छति गच्छ त्ये॒वैव ग॑च्छति । \newline
37. ग॒च्छ॒ति॒ सर्वा॑साꣳ॒॒ सर्वा॑साम् गच्छति गच्छति॒ सर्वा॑साम् । \newline
38. सर्वा॑सां॒ ॅवै वै सर्वा॑साꣳ॒॒ सर्वा॑सां॒ ॅवै । \newline
39. वा ए॒त दे॒तद् वै वा ए॒तत् । \newline
40. ए॒तद् दे॒वता॑नाम् दे॒वता॑ना मे॒त दे॒तद् दे॒वता॑नाम् । \newline
41. दे॒वता॑नाꣳ रू॒पꣳ रू॒पम् दे॒वता॑नाम् दे॒वता॑नाꣳ रू॒पम् । \newline
42. रू॒पं ॅयद् यद् रू॒पꣳ रू॒पं ॅयत् । \newline
43. यदे॒ष ए॒ष यद् यदे॒षः । \newline
44. ए॒ष ग्रहो॒ ग्रह॑ ए॒ष ए॒ष ग्रहः॑ । \newline
45. ग्रहो॒ यस्य॒ यस्य॒ ग्रहो॒ ग्रहो॒ यस्य॑ । \newline
46. यस्यै॒ष ए॒ष यस्य॒ यस्यै॒षः । \newline
47. ए॒ष गृ॒ह्यते॑ गृ॒ह्यत॑ ए॒ष ए॒ष गृ॒ह्यते᳚ । \newline
48. गृ॒ह्यते॒ सर्वा॑णि॒ सर्वा॑णि गृ॒ह्यते॑ गृ॒ह्यते॒ सर्वा॑णि । \newline
49. सर्वा᳚ ण्ये॒वैव सर्वा॑णि॒ सर्वा᳚ ण्ये॒व । \newline
50. ए॒वैन॑ मेन मे॒वैवैन᳚म् । \newline
51. ए॒नꣳ॒॒ रू॒पाणि॑ रू॒पाण्ये॑न मेनꣳ रू॒पाणि॑ । \newline
52. रू॒पाणि॑ पशू॒नाम् प॑शू॒नाꣳ रू॒पाणि॑ रू॒पाणि॑ पशू॒नाम् । \newline
53. प॒शू॒ना मुपोप॑ पशू॒नाम् प॑शू॒ना मुप॑ । \newline
54. उप॑ तिष्ठन्ते तिष्ठन्त॒ उपोप॑ तिष्ठन्ते । \newline
55. ति॒ष्ठ॒न्त॒ उ॒प॒या॒मगृ॑हीत उपया॒मगृ॑हीत स्तिष्ठन्ते तिष्ठन्त उपया॒मगृ॑हीतः । \newline
56. उ॒प॒या॒मगृ॑हीतो ऽस्य स्युपया॒मगृ॑हीत उपया॒मगृ॑हीतो ऽसि । \newline
57. उ॒प॒या॒मगृ॑हीत॒ इत्यु॑पया॒म - गृ॒ही॒तः॒ । \newline

\textbf{Ghana Paata } \newline

1. यां ॅवै वै यां ॅयां ॅवा अ॑द्ध्व॒र्यु र॑द्ध्व॒र्युर् वै यां ॅयां ॅवा अ॑द्ध्व॒र्युः । \newline
2. वा अ॑द्ध्व॒र्यु र॑द्ध्व॒र्युर् वै वा अ॑द्ध्व॒र्युश्च॑ चाद्ध्व॒र्युर् वै वा अ॑द्ध्व॒र्युश्च॑ । \newline
3. अ॒द्ध्व॒र्युश्च॑ चाद्ध्व॒र्यु र॑द्ध्व॒र्युश्च॒ यज॑मानो॒ यज॑मान श्चाद्ध्व॒र्यु र॑द्ध्व॒र्युश्च॒ यज॑मानः । \newline
4. च॒ यज॑मानो॒ यज॑मानश्च च॒ यज॑मानश्च च॒ यज॑मानश्च च॒ यज॑मानश्च । \newline
5. यज॑मानश्च च॒ यज॑मानो॒ यज॑मानश्च दे॒वता᳚म् दे॒वता᳚म् च॒ यज॑मानो॒ यज॑मानश्च दे॒वता᳚म् । \newline
6. च॒ दे॒वता᳚म् दे॒वता᳚म् च च दे॒वता॑ मन्तरि॒तो᳚ ऽन्तरि॒तो दे॒वता᳚म् च च दे॒वता॑ मन्तरि॒तः । \newline
7. दे॒वता॑ मन्तरि॒तो᳚ ऽन्तरि॒तो दे॒वता᳚म् दे॒वता॑ मन्तरि॒त स्तस्यै॒ तस्या॑ अन्तरि॒तो दे॒वता᳚म् दे॒वता॑ मन्तरि॒त स्तस्यै᳚ । \newline
8. अ॒न्त॒रि॒त स्तस्यै॒ तस्या॑ अन्तरि॒तो᳚ ऽन्तरि॒त स्तस्या॒ आ तस्या॑ अन्तरि॒तो᳚ ऽन्तरि॒त स्तस्या॒ आ । \newline
9. अ॒न्त॒रि॒त इत्य॑न्तः - इ॒तः । \newline
10. तस्या॒ आ तस्यै॒ तस्या॒ आ वृ॑श्च्येते वृश्च्येते॒ आ तस्यै॒ तस्या॒ आ वृ॑श्च्येते । \newline
11. आ वृ॑श्च्येते वृश्च्येते॒ आ वृ॑श्च्येते प्राजाप॒त्यम् प्रा॑जाप॒त्यं ॅवृ॑श्च्येते॒ आ वृ॑श्च्येते प्राजाप॒त्यम् । \newline
12. वृ॒श्च्ये॒ते॒ प्रा॒जा॒प॒त्यम् प्रा॑जाप॒त्यं ॅवृ॑श्च्येते वृश्च्येते प्राजाप॒त्यम् द॑धिग्र॒हम् द॑धिग्र॒हम् प्रा॑जाप॒त्यं ॅवृ॑श्च्येते वृश्च्येते प्राजाप॒त्यम् द॑धिग्र॒हम् । \newline
13. वृ॒श्च्ये॒ते॒ इति॑ वृश्च्येते । \newline
14. प्रा॒जा॒प॒त्यम् द॑धिग्र॒हम् द॑धिग्र॒हम् प्रा॑जाप॒त्यम् प्रा॑जाप॒त्यम् द॑धिग्र॒हम् गृ॑ह्णीयाद् गृह्णीयाद् दधिग्र॒हम् प्रा॑जाप॒त्यम् प्रा॑जाप॒त्यम् द॑धिग्र॒हम् गृ॑ह्णीयात् । \newline
15. प्रा॒जा॒प॒त्यमिति॑ प्राजा - प॒त्यम् । \newline
16. द॒धि॒ग्र॒हम् गृ॑ह्णीयाद् गृह्णीयाद् दधिग्र॒हम् द॑धिग्र॒हम् गृ॑ह्णीयात् प्र॒जाप॑तिः प्र॒जाप॑तिर् गृह्णीयाद् दधिग्र॒हम् द॑धिग्र॒हम् गृ॑ह्णीयात् प्र॒जाप॑तिः । \newline
17. द॒धि॒ग्र॒हमिति॑ दधि - ग्र॒हम् । \newline
18. गृ॒ह्णी॒या॒त् प्र॒जाप॑तिः प्र॒जाप॑तिर् गृह्णीयाद् गृह्णीयात् प्र॒जाप॑तिः॒ सर्वाः॒ सर्वाः᳚ प्र॒जाप॑तिर् गृह्णीयाद् गृह्णीयात् प्र॒जाप॑तिः॒ सर्वाः᳚ । \newline
19. प्र॒जाप॑तिः॒ सर्वाः॒ सर्वाः᳚ प्र॒जाप॑तिः प्र॒जाप॑तिः॒ सर्वा॑ दे॒वता॑ दे॒वताः॒ सर्वाः᳚ प्र॒जाप॑तिः प्र॒जाप॑तिः॒ सर्वा॑ दे॒वताः᳚ । \newline
20. प्र॒जाप॑ति॒रिति॑ प्र॒जा - प॒तिः॒ । \newline
21. सर्वा॑ दे॒वता॑ दे॒वताः॒ सर्वाः॒ सर्वा॑ दे॒वता॑ दे॒वता᳚भ्यो दे॒वता᳚भ्यो दे॒वताः॒ सर्वाः॒ सर्वा॑ दे॒वता॑ दे॒वता᳚भ्यः । \newline
22. दे॒वता॑ दे॒वता᳚भ्यो दे॒वता᳚भ्यो दे॒वता॑ दे॒वता॑ दे॒वता᳚भ्य ए॒वैव दे॒वता᳚भ्यो दे॒वता॑ दे॒वता॑ दे॒वता᳚भ्य ए॒व । \newline
23. दे॒वता᳚भ्य ए॒वैव दे॒वता᳚भ्यो दे॒वता᳚भ्य ए॒व नि न्ये॑व दे॒वता᳚भ्यो दे॒वता᳚भ्य ए॒व नि । \newline
24. ए॒व नि न्ये॑वैव नि ह्नु॑वाते ह्नुवाते॒ न्ये॑वैव नि ह्नु॑वाते । \newline
25. नि ह्नु॑वाते ह्नुवाते॒ नि नि ह्नु॑वाते ज्ये॒ष्ठो ज्ये॒ष्ठो ह्नु॑वाते॒ नि नि ह्नु॑वाते ज्ये॒ष्ठः । \newline
26. ह्नु॒वा॒ते॒ ज्ये॒ष्ठो ज्ये॒ष्ठो ह्नु॑वाते ह्नुवाते ज्ये॒ष्ठो वै वै ज्ये॒ष्ठो ह्नु॑वाते ह्नुवाते ज्ये॒ष्ठो वै । \newline
27. ह्नु॒वा॒ते॒ इति॑ ह्नुवाते । \newline
28. ज्ये॒ष्ठो वै वै ज्ये॒ष्ठो ज्ये॒ष्ठो वा ए॒ष ए॒ष वै ज्ये॒ष्ठो ज्ये॒ष्ठो वा ए॒षः । \newline
29. वा ए॒ष ए॒ष वै वा ए॒ष ग्रहा॑णा॒म् ग्रहा॑णा मे॒ष वै वा ए॒ष ग्रहा॑णाम् । \newline
30. ए॒ष ग्रहा॑णा॒म् ग्रहा॑णा मे॒ष ए॒ष ग्रहा॑णां॒ ॅयस्य॒ यस्य॒ ग्रहा॑णा मे॒ष ए॒ष ग्रहा॑णां॒ ॅयस्य॑ । \newline
31. ग्रहा॑णां॒ ॅयस्य॒ यस्य॒ ग्रहा॑णा॒म् ग्रहा॑णां॒ ॅयस्यै॒ष ए॒ष यस्य॒ ग्रहा॑णा॒म् ग्रहा॑णां॒ ॅयस्यै॒षः । \newline
32. यस्यै॒ष ए॒ष यस्य॒ यस्यै॒ष गृ॒ह्यते॑ गृ॒ह्यत॑ ए॒ष यस्य॒ यस्यै॒ष गृ॒ह्यते᳚ । \newline
33. ए॒ष गृ॒ह्यते॑ गृ॒ह्यत॑ ए॒ष ए॒ष गृ॒ह्यते॒ ज्यैष्ट्य॒म् ज्यैष्ट्य॑म् गृ॒ह्यत॑ ए॒ष ए॒ष गृ॒ह्यते॒ ज्यैष्ट्य᳚म् । \newline
34. गृ॒ह्यते॒ ज्यैष्ट्य॒म् ज्यैष्ट्य॑म् गृ॒ह्यते॑ गृ॒ह्यते॒ ज्यैष्ट्य॑ मे॒वैव ज्यैष्ट्य॑म् गृ॒ह्यते॑ गृ॒ह्यते॒ ज्यैष्ट्य॑ मे॒व । \newline
35. ज्यैष्ट्य॑ मे॒वैव ज्यैष्ट्य॒म् ज्यैष्ट्य॑ मे॒व ग॑च्छति गच्छत्ये॒व ज्यैष्ट्य॒म् ज्यैष्ट्य॑ मे॒व ग॑च्छति । \newline
36. ए॒व ग॑च्छति गच्छ त्ये॒वैव ग॑च्छति॒ सर्वा॑साꣳ॒॒ सर्वा॑साम् गच्छ त्ये॒वैव ग॑च्छति॒ सर्वा॑साम् । \newline
37. ग॒च्छ॒ति॒ सर्वा॑साꣳ॒॒ सर्वा॑साम् गच्छति गच्छति॒ सर्वा॑सां॒ ॅवै वै सर्वा॑साम् गच्छति गच्छति॒ सर्वा॑सां॒ ॅवै । \newline
38. सर्वा॑सां॒ ॅवै वै सर्वा॑साꣳ॒॒ सर्वा॑सां॒ ॅवा ए॒त दे॒तद् वै सर्वा॑साꣳ॒॒ सर्वा॑सां॒ ॅवा ए॒तत् । \newline
39. वा ए॒त दे॒तद् वै वा ए॒तद् दे॒वता॑नाम् दे॒वता॑ना मे॒तद् वै वा ए॒तद् दे॒वता॑नाम् । \newline
40. ए॒तद् दे॒वता॑नाम् दे॒वता॑ना मे॒त दे॒तद् दे॒वता॑नाꣳ रू॒पꣳ रू॒पम् दे॒वता॑ना मे॒त दे॒तद् दे॒वता॑नाꣳ रू॒पम् । \newline
41. दे॒वता॑नाꣳ रू॒पꣳ रू॒पम् दे॒वता॑नाम् दे॒वता॑नाꣳ रू॒पं ॅयद् यद् रू॒पम् दे॒वता॑नाम् दे॒वता॑नाꣳ रू॒पं ॅयत् । \newline
42. रू॒पं ॅयद् यद् रू॒पꣳ रू॒पं ॅयदे॒ष ए॒ष यद् रू॒पꣳ रू॒पं ॅयदे॒षः । \newline
43. यदे॒ष ए॒ष यद् यदे॒ष ग्रहो॒ ग्रह॑ ए॒ष यद् यदे॒ष ग्रहः॑ । \newline
44. ए॒ष ग्रहो॒ ग्रह॑ ए॒ष ए॒ष ग्रहो॒ यस्य॒ यस्य॒ ग्रह॑ ए॒ष ए॒ष ग्रहो॒ यस्य॑ । \newline
45. ग्रहो॒ यस्य॒ यस्य॒ ग्रहो॒ ग्रहो॒ यस्यै॒ष ए॒ष यस्य॒ ग्रहो॒ ग्रहो॒ यस्यै॒षः । \newline
46. यस्यै॒ष ए॒ष यस्य॒ यस्यै॒ष गृ॒ह्यते॑ गृ॒ह्यत॑ ए॒ष यस्य॒ यस्यै॒ष गृ॒ह्यते᳚ । \newline
47. ए॒ष गृ॒ह्यते॑ गृ॒ह्यत॑ ए॒ष ए॒ष गृ॒ह्यते॒ सर्वा॑णि॒ सर्वा॑णि गृ॒ह्यत॑ ए॒ष ए॒ष गृ॒ह्यते॒ सर्वा॑णि । \newline
48. गृ॒ह्यते॒ सर्वा॑णि॒ सर्वा॑णि गृ॒ह्यते॑ गृ॒ह्यते॒ सर्वा᳚ ण्ये॒वैव सर्वा॑णि गृ॒ह्यते॑ गृ॒ह्यते॒ सर्वा᳚ ण्ये॒व । \newline
49. सर्वा᳚ ण्ये॒वैव सर्वा॑णि॒ सर्वा᳚ ण्ये॒वैन॑ मेन मे॒व सर्वा॑णि॒ सर्वा᳚ ण्ये॒वैन᳚म् । \newline
50. ए॒वैन॑ मेन मे॒वैवैनꣳ॑ रू॒पाणि॑ रू॒पाण्ये॑न मे॒वैवैनꣳ॑ रू॒पाणि॑ । \newline
51. ए॒नꣳ॒॒ रू॒पाणि॑ रू॒पाण्ये॑न मेनꣳ रू॒पाणि॑ पशू॒नाम् प॑शू॒नाꣳ रू॒पाण्ये॑न मेनꣳ रू॒पाणि॑ पशू॒नाम् । \newline
52. रू॒पाणि॑ पशू॒नाम् प॑शू॒नाꣳ रू॒पाणि॑ रू॒पाणि॑ पशू॒ना मुपोप॑ पशू॒नाꣳ रू॒पाणि॑ रू॒पाणि॑ पशू॒ना मुप॑ । \newline
53. प॒शू॒ना मुपोप॑ पशू॒नाम् प॑शू॒ना मुप॑ तिष्ठन्ते तिष्ठन्त॒ उप॑ पशू॒नाम् प॑शू॒ना मुप॑ तिष्ठन्ते । \newline
54. उप॑ तिष्ठन्ते तिष्ठन्त॒ उपोप॑ तिष्ठन्त उपया॒मगृ॑हीत उपया॒मगृ॑हीत स्तिष्ठन्त॒ उपोप॑ तिष्ठन्त उपया॒मगृ॑हीतः । \newline
55. ति॒ष्ठ॒न्त॒ उ॒प॒या॒मगृ॑हीत उपया॒मगृ॑हीत स्तिष्ठन्ते तिष्ठन्त उपया॒मगृ॑हीतो ऽस्य स्युपया॒मगृ॑हीत स्तिष्ठन्ते तिष्ठन्त उपया॒मगृ॑हीतो ऽसि । \newline
56. उ॒प॒या॒मगृ॑हीतो ऽस्य स्युपया॒मगृ॑हीत उपया॒मगृ॑हीतो ऽसि प्र॒जाप॑तये प्र॒जाप॑तये ऽस्युपया॒मगृ॑हीत उपया॒मगृ॑हीतो ऽसि प्र॒जाप॑तये । \newline
57. उ॒प॒या॒मगृ॑हीत॒ इत्यु॑पया॒म - गृ॒ही॒तः॒ । \newline
\pagebreak
\markright{ TS 3.5.9.2  \hfill https://www.vedavms.in \hfill}

\section{ TS 3.5.9.2 }

\textbf{TS 3.5.9.2 } \newline
\textbf{Samhita Paata} \newline

-ऽसि प्र॒जाप॑तये त्वा॒ ज्योति॑ष्मते॒ ज्योति॑ष्मन्तं गृह्णा॒मीत्या॑ह॒ ज्योति॑रे॒वैनꣳ॑ समा॒नानां᳚ करोत्यग्नि-जि॒ह्वेभ्य॑स्त्वर्ता॒युभ्य॒ इत्या॑है॒ताव॑ती॒र्वै दे॒वता॒स्ताभ्य॑ ए॒वैनꣳ॒॒ सर्वा᳚भ्यो गृह्णा॒त्यपे᳚न्द्र द्विष॒तो मन॒ इत्या॑ह॒ भ्रातृ॑व्यापनुत्त्यै प्रा॒णाय॑ त्वाऽपा॒नाय॒ त्वेत्या॑ह प्रा॒णाने॒व यज॑माने दधाति॒ तस्मै᳚ त्वा प्र॒जाप॑तये विभू॒दाव्.न्ने॒ ज्योति॑ष्मते॒ ज्योति॑ष्मन्तं जुहो॒मी - [  ] \newline

\textbf{Pada Paata} \newline

अ॒सि॒ । प्र॒जाप॑तय॒ इति॑ प्र॒जा - प॒त॒ये॒ । त्वा॒ । ज्योति॑ष्मते । ज्योति॑ष्मन्तम् । गृ॒ह्णा॒मि॒ । इति॑ । आ॒ह॒ । ज्योतिः॑ । ए॒व । ए॒न॒म् । स॒मा॒नाना᳚म् । क॒रो॒ति॒ । अ॒ग्नि॒जि॒ह्वेभ्य॒ इत्य॑ग्नि - जि॒ह्वेभ्यः॑ । त्वा॒ । ऋ॒ता॒युभ्य॒ इत्यृ॑ता॒यु - भ्यः॒ । इति॑ । आ॒ह॒ । ए॒ताव॑तीः । वै । दे॒वताः᳚ । ताभ्यः॑ । ए॒व । ए॒न॒म् । सर्वा᳚भ्यः । गृ॒ह्णा॒ति॒ । अपेति॑ । इ॒न्द्र॒ । द्वि॒ष॒तः । मनः॑ । इति॑ । आ॒ह॒ । भ्रातृ॑व्यापनुत्त्या॒ इति॒ भ्रातृ॑व्य - अ॒प॒नु॒त्त्यै॒ । प्रा॒णायेति॑ प्र - अ॒नाय॑ । त्वा॒ । अ॒पा॒नायेत्य॑प - अ॒नाय॑ । त्वा॒ । इति॑ । आ॒ह॒ । प्रा॒णानिति॑ प्र - अ॒नान् । ए॒व । यज॑माने । द॒धा॒ति॒ । तस्मै᳚ । त्वा॒ । प्र॒जाप॑तय॒ इति॑ प्र॒जा - प॒त॒ये॒ । वि॒भू॒दाव्.न्न॒ इति॑ विभु - दाव्.न्ने᳚ । ज्योति॑ष्मते । ज्योति॑ष्मन्तम् । जु॒हो॒मि॒ ।  \newline


\textbf{Krama Paata} \newline

अ॒सि॒ प्र॒जाप॑तये । प्र॒जाप॑तये त्वा । प्र॒जाप॑तय॒ इति॑ प्र॒जा - प॒त॒ये॒ । त्वा॒ ज्योति॑ष्मते । ज्योति॑ष्मते॒ ज्योति॑ष्मन्तम् । ज्योति॑ष्मन्तम् गृह्णामि । गृ॒ह्णा॒मीति॑ । इत्या॑ह । आ॒ह॒ ज्योतिः॑ । ज्योति॑रे॒व । ए॒वैन᳚म् । ए॒नꣳ॒॒ स॒मा॒नाना᳚म् । स॒मा॒नाना᳚म् करोति । क॒रो॒त्य॒ग्नि॒जि॒ह्वेभ्यः॑ । अ॒ग्नि॒जि॒ह्वेभ्य॑स्त्वा । अ॒ग्नि॒जि॒ह्वेभ्य॒ इत्य॑ग्नि - जि॒ह्वेभ्यः॑ । त्व॒र्ता॒युभ्यः॑ । ऋ॒ता॒युभ्य॒ इति॑ । ऋ॒ता॒युभ्य॒ इत्यृ॑ता॒यु - भ्यः॒ । इत्या॑ह । आ॒है॒ताव॑तीः । ए॒ताव॑ती॒र् वै । वै दे॒वताः᳚ । दे॒वता॒ स्ताभ्यः॑ । ताभ्य॑ ए॒व । ए॒वैन᳚म् । ए॒नꣳ॒॒ सर्वा᳚भ्यः । सर्वा᳚भ्यो गृह्णाति । गृ॒ह्णा॒त्यप॑ । अपे᳚न्द्र । इ॒न्द्र॒ द्वि॒ष॒तः । द्वि॒ष॒तो मनः॑ । मन॒ इति॑ । इत्या॑ह । आ॒ह॒ भ्रातृ॑व्यापनुत्यै । भ्रातृ॑व्यापनुत्यै प्रा॒णाय॑ । भ्रातृ॑व्यापनुत्या॒ इति॒ भ्रातृ॑व्य - अ॒प॒नु॒त्यै॒ । प्रा॒णाय॑ त्वा । प्रा॒णायेति॑ प्र - अ॒नाय॑ । त्वा॒ ऽपा॒नाय॑ । अ॒पा॒नाय॑ त्वा । अ॒पा॒नायेत्य॑प - अ॒नाय॑ । त्वेति॑ । इत्या॑ह । आ॒ह॒ प्रा॒णान् । प्रा॒णाने॒व । प्रा॒णानिति॑ प्र - अ॒नान् । ए॒व यज॑माने । यज॑माने दधाति । द॒धा॒ति॒ तस्मै᳚ । तस्मै᳚ त्वा । त्वा॒ प्र॒जाप॑तये । प्र॒जाप॑तये विभू॒दाव्.न्ने᳚ । प्र॒जाप॑तय॒ इति॑ प्र॒जा - प॒त॒ये॒ । वि॒भू॒दाव्.न्ने॒ ज्योति॑ष्मते । वि॒भू॒दाव्.न्न॒ इति॑ विभु - दाव्.न्ने᳚ । ज्योति॑ष्मते॒ ज्योति॑ष्मन्तम् । ज्योति॑ष्मन्तम् जुहोमि ( ) । जु॒हो॒मीति॑ \newline

\textbf{Jatai Paata} \newline

1. अ॒सि॒ प्र॒जाप॑तये प्र॒जाप॑तये ऽस्यसि प्र॒जाप॑तये । \newline
2. प्र॒जाप॑तये त्वा त्वा प्र॒जाप॑तये प्र॒जाप॑तये त्वा । \newline
3. प्र॒जाप॑तय॒ इति॑ प्र॒जा - प॒त॒ये॒ । \newline
4. त्वा॒ ज्योति॑ष्मते॒ ज्योति॑ष्मते त्वा त्वा॒ ज्योति॑ष्मते । \newline
5. ज्योति॑ष्मते॒ ज्योति॑ष्मन्त॒म् ज्योति॑ष्मन्त॒म् ज्योति॑ष्मते॒ ज्योति॑ष्मते॒ ज्योति॑ष्मन्तम् । \newline
6. ज्योति॑ष्मन्तम् गृह्णामि गृह्णामि॒ ज्योति॑ष्मन्त॒म् ज्योति॑ष्मन्तम् गृह्णामि । \newline
7. गृ॒ह्णा॒मीतीति॑ गृह्णामि गृह्णा॒मीति॑ । \newline
8. इत्या॑हा॒हे तीत्या॑ह । \newline
9. आ॒ह॒ ज्योति॒र् ज्योति॑राहाह॒ ज्योतिः॑ । \newline
10. ज्योति॑ रे॒वैव ज्योति॒र् ज्योति॑ रे॒व । \newline
11. ए॒वैन॑ मेन मे॒वैवैन᳚म् । \newline
12. ए॒नꣳ॒॒ स॒मा॒नानाꣳ॑ समा॒नाना॑ मेन मेनꣳ समा॒नाना᳚म् । \newline
13. स॒मा॒नाना᳚म् करोति करोति समा॒नानाꣳ॑ समा॒नाना᳚म् करोति । \newline
14. क॒रो॒ त्य॒ग्नि॒जि॒ह्वेभ्यो᳚ ऽग्निजि॒ह्वेभ्यः॑ करोति करो त्यग्निजि॒ह्वेभ्यः॑ । \newline
15. अ॒ग्नि॒जि॒ह्वेभ्य॑ स्त्वा त्वा ऽग्निजि॒ह्वेभ्यो᳚ ऽग्निजि॒ह्वेभ्य॑ स्त्वा । \newline
16. अ॒ग्नि॒जि॒ह्वेभ्य॒ इत्य॑ग्नि - जि॒ह्वेभ्यः॑ । \newline
17. त्व॒ र्‌ता॒युभ्य॑ ऋता॒युभ्य॑ स्त्वा त्व र्‌ता॒युभ्यः॑ । \newline
18. ऋ॒ता॒युभ्य॒ इती त्यृ॑ता॒युभ्य॑ ऋता॒युभ्य॒ इति॑ । \newline
19. ऋ॒ता॒युभ्य॒ इत्यृ॑ता॒यु - भ्यः॒ । \newline
20. इत्या॑ हा॒हे तीत्या॑ह । \newline
21. आ॒है॒ताव॑ती रे॒ताव॑ती राहा है॒ताव॑तीः । \newline
22. ए॒ताव॑ती॒र् वै वा ए॒ताव॑ती रे॒ताव॑ती॒र् वै । \newline
23. वै दे॒वता॑ दे॒वता॒ वै वै दे॒वताः᳚ । \newline
24. दे॒वता॒ स्ताभ्य॒ स्ताभ्यो॑ दे॒वता॑ दे॒वता॒ स्ताभ्यः॑ । \newline
25. ताभ्य॑ ए॒वैव ताभ्य॒ स्ताभ्य॑ ए॒व । \newline
26. ए॒वैन॑ मेन मे॒वैवैन᳚म् । \newline
27. ए॒नꣳ॒॒ सर्वा᳚भ्यः॒ सर्वा᳚भ्य एन मेनꣳ॒॒ सर्वा᳚भ्यः । \newline
28. सर्वा᳚भ्यो गृह्णाति गृह्णाति॒ सर्वा᳚भ्यः॒ सर्वा᳚भ्यो गृह्णाति । \newline
29. गृ॒ह्णा॒ त्यपाप॑ गृह्णाति गृह्णा॒ त्यप॑ । \newline
30. अपे᳚न्द्रे॒ न्द्रापा पे᳚न्द्र । \newline
31. इ॒न्द्र॒ द्वि॒ष॒तो द्वि॑ष॒त इ॑न्द्रेन्द्र द्विष॒तः । \newline
32. द्वि॒ष॒तो मनो॒ मनो᳚ द्विष॒तो द्वि॑ष॒तो मनः॑ । \newline
33. मन॒ इतीति॒ मनो॒ मन॒ इति॑ । \newline
34. इत्या॑ हा॒हे तीत्या॑ह । \newline
35. आ॒ह॒ भ्रातृ॑व्यापनुत्त्यै॒ भ्रातृ॑व्यापनुत्त्या आहाह॒ भ्रातृ॑व्यापनुत्त्यै । \newline
36. भ्रातृ॑व्यापनुत्त्यै प्रा॒णाय॑ प्रा॒णाय॒ भ्रातृ॑व्यापनुत्त्यै॒ भ्रातृ॑व्यापनुत्त्यै प्रा॒णाय॑ । \newline
37. भ्रातृ॑व्यापनुत्त्या॒ इति॒ भ्रातृ॑व्य - अ॒प॒नु॒त्त्यै॒ । \newline
38. प्रा॒णाय॑ त्वा त्वा प्रा॒णाय॑ प्रा॒णाय॑ त्वा । \newline
39. प्रा॒णायेति॑ प्र - अ॒नाय॑ । \newline
40. त्वा॒ ऽपा॒नाया॑ पा॒नाय॑ त्वा त्वा ऽपा॒नाय॑ । \newline
41. अ॒पा॒नाय॑ त्वा त्वा ऽपा॒नाया॑ पा॒नाय॑ त्वा । \newline
42. अ॒पा॒नायेत्य॑प - अ॒नाय॑ । \newline
43. त्वेतीति॑ त्वा॒ त्वेति॑ । \newline
44. इत्या॑ हा॒हे तीत्या॑ह । \newline
45. आ॒ह॒ प्रा॒णान् प्रा॒णा ना॑हाह प्रा॒णान् । \newline
46. प्रा॒णा ने॒वैव प्रा॒णान् प्रा॒णा ने॒व । \newline
47. प्रा॒णानिति॑ प्र - अ॒नान् । \newline
48. ए॒व यज॑माने॒ यज॑मान ए॒वैव यज॑माने । \newline
49. यज॑माने दधाति दधाति॒ यज॑माने॒ यज॑माने दधाति । \newline
50. द॒धा॒ति॒ तस्मै॒ तस्मै॑ दधाति दधाति॒ तस्मै᳚ । \newline
51. तस्मै᳚ त्वा त्वा॒ तस्मै॒ तस्मै᳚ त्वा । \newline
52. त्वा॒ प्र॒जाप॑तये प्र॒जाप॑तये त्वा त्वा प्र॒जाप॑तये । \newline
53. प्र॒जाप॑तये विभू॒दाव्.न्ने॑ विभू॒दाव्.न्ने᳚ प्र॒जाप॑तये प्र॒जाप॑तये विभू॒दाव्.न्ने᳚ । \newline
54. प्र॒जाप॑तय॒ इति॑ प्र॒जा - प॒त॒ये॒ । \newline
55. वि॒भू॒दाव्.न्ने॒ ज्योति॑ष्मते॒ ज्योति॑ष्मते विभू॒दाव्.न्ने॑ विभू॒दाव्.न्ने॒ ज्योति॑ष्मते । \newline
56. वि॒भू॒दाव्.न्न॒ इति॑ विभु - दाव्.न्ने᳚ । \newline
57. ज्योति॑ष्मते॒ ज्योति॑ष्मन्त॒म् ज्योति॑ष्मन्त॒म् ज्योति॑ष्मते॒ ज्योति॑ष्मते॒ ज्योति॑ष्मन्तम् । \newline
58. ज्योति॑ष्मन्तम् जुहोमि जुहोमि॒ ज्योति॑ष्मन्त॒म् ज्योति॑ष्मन्तम् जुहोमि । \newline
59. जु॒हो॒मीतीति॑ जुहोमि जुहो॒मीति॑ । \newline

\textbf{Ghana Paata } \newline

1. अ॒सि॒ प्र॒जाप॑तये प्र॒जाप॑तये ऽस्यसि प्र॒जाप॑तये त्वा त्वा प्र॒जाप॑तये ऽस्यसि प्र॒जाप॑तये त्वा । \newline
2. प्र॒जाप॑तये त्वा त्वा प्र॒जाप॑तये प्र॒जाप॑तये त्वा॒ ज्योति॑ष्मते॒ ज्योति॑ष्मते त्वा प्र॒जाप॑तये प्र॒जाप॑तये त्वा॒ ज्योति॑ष्मते । \newline
3. प्र॒जाप॑तय॒ इति॑ प्र॒जा - प॒त॒ये॒ । \newline
4. त्वा॒ ज्योति॑ष्मते॒ ज्योति॑ष्मते त्वा त्वा॒ ज्योति॑ष्मते॒ ज्योति॑ष्मन्त॒म् ज्योति॑ष्मन्त॒म् ज्योति॑ष्मते त्वा त्वा॒ ज्योति॑ष्मते॒ ज्योति॑ष्मन्तम् । \newline
5. ज्योति॑ष्मते॒ ज्योति॑ष्मन्त॒म् ज्योति॑ष्मन्त॒म् ज्योति॑ष्मते॒ ज्योति॑ष्मते॒ ज्योति॑ष्मन्तम् गृह्णामि गृह्णामि॒ ज्योति॑ष्मन्त॒म् ज्योति॑ष्मते॒ ज्योति॑ष्मते॒ ज्योति॑ष्मन्तम् गृह्णामि । \newline
6. ज्योति॑ष्मन्तम् गृह्णामि गृह्णामि॒ ज्योति॑ष्मन्त॒म् ज्योति॑ष्मन्तम् गृह्णा॒मीतीति॑ गृह्णामि॒ ज्योति॑ष्मन्त॒म् ज्योति॑ष्मन्तम् गृह्णा॒मीति॑ । \newline
7. गृ॒ह्णा॒मीतीति॑ गृह्णामि गृह्णा॒मी त्या॑हा॒हेति॑ गृह्णामि गृह्णा॒मी त्या॑ह । \newline
8. इत्या॑हा॒हे तीत्या॑ह॒ ज्योति॒र् ज्योति॑रा॒हे तीत्या॑ह॒ ज्योतिः॑ । \newline
9. आ॒ह॒ ज्योति॒र् ज्योति॑ राहाह॒ ज्योति॑ रे॒वैव ज्योति॑ राहाह॒ ज्योति॑ रे॒व । \newline
10. ज्योति॑ रे॒वैव ज्योति॒र् ज्योति॑ रे॒वैन॑ मेन मे॒व ज्योति॒र् ज्योति॑ रे॒वैन᳚म् । \newline
11. ए॒वैन॑ मेन मे॒वैवैनꣳ॑ समा॒नानाꣳ॑ समा॒नाना॑ मेन मे॒वैवैनꣳ॑ समा॒नाना᳚म् । \newline
12. ए॒नꣳ॒॒ स॒मा॒नानाꣳ॑ समा॒नाना॑ मेन मेनꣳ समा॒नाना᳚म् करोति करोति समा॒नाना॑ मेन मेनꣳ समा॒नाना᳚म् करोति । \newline
13. स॒मा॒नाना᳚म् करोति करोति समा॒नानाꣳ॑ समा॒नाना᳚म् करो त्यग्निजि॒ह्वेभ्यो᳚ ऽग्निजि॒ह्वेभ्यः॑ करोति समा॒नानाꣳ॑ समा॒नाना᳚म् करो त्यग्निजि॒ह्वेभ्यः॑ । \newline
14. क॒रो॒ त्य॒ग्नि॒जि॒ह्वेभ्यो᳚ ऽग्निजि॒ह्वेभ्यः॑ करोति करो त्यग्निजि॒ह्वेभ्य॑ स्त्वा त्वा ऽग्निजि॒ह्वेभ्यः॑ करोति करो त्यग्निजि॒ह्वेभ्य॑ स्त्वा । \newline
15. अ॒ग्नि॒जि॒ह्वेभ्य॑ स्त्वा त्वा ऽग्निजि॒ह्वेभ्यो᳚ ऽग्निजि॒ह्वेभ्य॑ स्त्व र्‌ता॒युभ्य॑ ऋता॒युभ्य॑ स्त्वा ऽग्निजि॒ह्वेभ्यो᳚ ऽग्निजि॒ह्वेभ्य॑ स्त्व र्‌ता॒युभ्यः॑ । \newline
16. अ॒ग्नि॒जि॒ह्वेभ्य॒ इत्य॑ग्नि - जि॒ह्वेभ्यः॑ । \newline
17. त्व॒ र्‌ता॒युभ्य॑ ऋता॒युभ्य॑ स्त्वा त्व र्‌ता॒युभ्य॒ इती त्यृ॑ता॒युभ्य॑ स्त्वा त्व र्‌ता॒युभ्य॒ इति॑ । \newline
18. ऋ॒ता॒युभ्य॒ इती त्यृ॑ता॒युभ्य॑ ऋता॒युभ्य॒ इत्या॑हा॒हे त्यृ॑ता॒युभ्य॑ ऋता॒युभ्य॒ इत्या॑ह । \newline
19. ऋ॒ता॒युभ्य॒ इत्यृ॑ता॒यु - भ्यः॒ । \newline
20. इत्या॑हा॒हे तीत्या॑ है॒ताव॑ती रे॒ताव॑ती रा॒हेती त्या॑है॒ताव॑तीः । \newline
21. आ॒है॒ताव॑ती रे॒ताव॑ती राहा है॒ताव॑ती॒र् वै वा ए॒ताव॑ती राहा है॒ताव॑ती॒र् वै । \newline
22. ए॒ताव॑ती॒र् वै वा ए॒ताव॑ती रे॒ताव॑ती॒र् वै दे॒वता॑ दे॒वता॒ वा ए॒ताव॑ती रे॒ताव॑ती॒र् वै दे॒वताः᳚ । \newline
23. वै दे॒वता॑ दे॒वता॒ वै वै दे॒वता॒ स्ताभ्य॒ स्ताभ्यो॑ दे॒वता॒ वै वै दे॒वता॒ स्ताभ्यः॑ । \newline
24. दे॒वता॒ स्ताभ्य॒ स्ताभ्यो॑ दे॒वता॑ दे॒वता॒ स्ताभ्य॑ ए॒वैव ताभ्यो॑ दे॒वता॑ दे॒वता॒ स्ताभ्य॑ ए॒व । \newline
25. ताभ्य॑ ए॒वैव ताभ्य॒ स्ताभ्य॑ ए॒वैन॑ मेन मे॒व ताभ्य॒ स्ताभ्य॑ ए॒वैन᳚म् । \newline
26. ए॒वैन॑ मेन मे॒वैवैनꣳ॒॒ सर्वा᳚भ्यः॒ सर्वा᳚भ्य एन मे॒वैवैनꣳ॒॒ सर्वा᳚भ्यः । \newline
27. ए॒नꣳ॒॒ सर्वा᳚भ्यः॒ सर्वा᳚भ्य एन मेनꣳ॒॒ सर्वा᳚भ्यो गृह्णाति गृह्णाति॒ सर्वा᳚भ्य एन मेनꣳ॒॒ सर्वा᳚भ्यो गृह्णाति । \newline
28. सर्वा᳚भ्यो गृह्णाति गृह्णाति॒ सर्वा᳚भ्यः॒ सर्वा᳚भ्यो गृह्णा॒ त्यपाप॑ गृह्णाति॒ सर्वा᳚भ्यः॒ सर्वा᳚भ्यो गृह्णा॒ त्यप॑ । \newline
29. गृ॒ह्णा॒ त्यपाप॑ गृह्णाति गृह्णा॒ त्यपे᳚न्द्रे॒ न्द्राप॑ गृह्णाति गृह्णा॒ त्यपे᳚न्द्र । \newline
30. अपे᳚न्द्रे॒ न्द्रा पापे᳚न्द्र द्विष॒तो द्वि॑ष॒त इ॒न्द्रा पापे᳚न्द्र द्विष॒तः । \newline
31. इ॒न्द्र॒ द्वि॒ष॒तो द्वि॑ष॒त इ॑न्द्रेन्द्र द्विष॒तो मनो॒ मनो᳚ द्विष॒त इ॑न्द्रेन्द्र द्विष॒तो मनः॑ । \newline
32. द्वि॒ष॒तो मनो॒ मनो᳚ द्विष॒तो द्वि॑ष॒तो मन॒ इतीति॒ मनो᳚ द्विष॒तो द्वि॑ष॒तो मन॒ इति॑ । \newline
33. मन॒ इतीति॒ मनो॒ मन॒ इत्या॑हा॒हेति॒ मनो॒ मन॒ इत्या॑ह । \newline
34. इत्या॑हा॒हे तीत्या॑ह॒ भ्रातृ॑व्यापनुत्त्यै॒ भ्रातृ॑व्यापनुत्त्या आ॒हे तीत्या॑ह॒ भ्रातृ॑व्यापनुत्त्यै । \newline
35. आ॒ह॒ भ्रातृ॑व्यापनुत्त्यै॒ भ्रातृ॑व्यापनुत्त्या आहाह॒ भ्रातृ॑व्यापनुत्त्यै प्रा॒णाय॑ प्रा॒णाय॒ भ्रातृ॑व्यापनुत्त्या आहाह॒ भ्रातृ॑व्यापनुत्त्यै प्रा॒णाय॑ । \newline
36. भ्रातृ॑व्यापनुत्त्यै प्रा॒णाय॑ प्रा॒णाय॒ भ्रातृ॑व्यापनुत्त्यै॒ भ्रातृ॑व्यापनुत्त्यै प्रा॒णाय॑ त्वा त्वा प्रा॒णाय॒ भ्रातृ॑व्यापनुत्त्यै॒ भ्रातृ॑व्यापनुत्त्यै प्रा॒णाय॑ त्वा । \newline
37. भ्रातृ॑व्यापनुत्त्या॒ इति॒ भ्रातृ॑व्य - अ॒प॒नु॒त्त्यै॒ । \newline
38. प्रा॒णाय॑ त्वा त्वा प्रा॒णाय॑ प्रा॒णाय॑ त्वा ऽपा॒नाया॑ पा॒नाय॑ त्वा प्रा॒णाय॑ प्रा॒णाय॑ त्वा ऽपा॒नाय॑ । \newline
39. प्रा॒णायेति॑ प्र - अ॒नाय॑ । \newline
40. त्वा॒ ऽपा॒नाया॑ पा॒नाय॑ त्वा त्वा ऽपा॒नाय॑ त्वा त्वा ऽपा॒नाय॑ त्वा त्वा ऽपा॒नाय॑ त्वा । \newline
41. अ॒पा॒नाय॑ त्वा त्वा ऽपा॒नाया॑ पा॒नाय॒ त्वेतीति॑ त्वा ऽपा॒नाया॑ पा॒नाय॒ त्वेति॑ । \newline
42. अ॒पा॒नायेत्य॑प - अ॒नाय॑ । \newline
43. त्वेतीति॑ त्वा॒ त्वेत्या॑ हा॒हेति॑ त्वा॒ त्वेत्या॑ह । \newline
44. इत्या॑हा॒हे तीत्या॑ह प्रा॒णान् प्रा॒णा ना॒हे तीत्या॑ह प्रा॒णान् । \newline
45. आ॒ह॒ प्रा॒णान् प्रा॒णा ना॑हाह प्रा॒णा ने॒वैव प्रा॒णा ना॑हाह प्रा॒णा ने॒व । \newline
46. प्रा॒णा ने॒वैव प्रा॒णान् प्रा॒णा ने॒व यज॑माने॒ यज॑मान ए॒व प्रा॒णान् प्रा॒णा ने॒व यज॑माने । \newline
47. प्रा॒णानिति॑ प्र - अ॒नान् । \newline
48. ए॒व यज॑माने॒ यज॑मान ए॒वैव यज॑माने दधाति दधाति॒ यज॑मान ए॒वैव यज॑माने दधाति । \newline
49. यज॑माने दधाति दधाति॒ यज॑माने॒ यज॑माने दधाति॒ तस्मै॒ तस्मै॑ दधाति॒ यज॑माने॒ यज॑माने दधाति॒ तस्मै᳚ । \newline
50. द॒धा॒ति॒ तस्मै॒ तस्मै॑ दधाति दधाति॒ तस्मै᳚ त्वा त्वा॒ तस्मै॑ दधाति दधाति॒ तस्मै᳚ त्वा । \newline
51. तस्मै᳚ त्वा त्वा॒ तस्मै॒ तस्मै᳚ त्वा प्र॒जाप॑तये प्र॒जाप॑तये त्वा॒ तस्मै॒ तस्मै᳚ त्वा प्र॒जाप॑तये । \newline
52. त्वा॒ प्र॒जाप॑तये प्र॒जाप॑तये त्वा त्वा प्र॒जाप॑तये विभू॒दाव्.न्ने॑ विभू॒दाव्.न्ने᳚ प्र॒जाप॑तये त्वा त्वा प्र॒जाप॑तये विभू॒दाव्.न्ने᳚ । \newline
53. प्र॒जाप॑तये विभू॒दाव्.न्ने॑ विभू॒दाव्.न्ने᳚ प्र॒जाप॑तये प्र॒जाप॑तये विभू॒दाव्.न्ने॒ ज्योति॑ष्मते॒ ज्योति॑ष्मते विभू॒दाव्.न्ने᳚ प्र॒जाप॑तये प्र॒जाप॑तये विभू॒दाव्.न्ने॒ ज्योति॑ष्मते । \newline
54. प्र॒जाप॑तय॒ इति॑ प्र॒जा - प॒त॒ये॒ । \newline
55. वि॒भू॒दाव्.न्ने॒ ज्योति॑ष्मते॒ ज्योति॑ष्मते विभू॒दाव्.न्ने॑ विभू॒दाव्.न्ने॒ ज्योति॑ष्मते॒ ज्योति॑ष्मन्त॒म् ज्योति॑ष्मन्त॒म् ज्योति॑ष्मते विभू॒दाव्.न्ने॑ विभू॒दाव्.न्ने॒ ज्योति॑ष्मते॒ ज्योति॑ष्मन्तम् । \newline
56. वि॒भू॒दाव्.न्न॒ इति॑ विभु - दाव्.न्ने᳚ । \newline
57. ज्योति॑ष्मते॒ ज्योति॑ष्मन्त॒म् ज्योति॑ष्मन्त॒म् ज्योति॑ष्मते॒ ज्योति॑ष्मते॒ ज्योति॑ष्मन्तम् जुहोमि जुहोमि॒ ज्योति॑ष्मन्त॒म् ज्योति॑ष्मते॒ ज्योति॑ष्मते॒ ज्योति॑ष्मन्तम् जुहोमि । \newline
58. ज्योति॑ष्मन्तम् जुहोमि जुहोमि॒ ज्योति॑ष्मन्त॒म् ज्योति॑ष्मन्तम् जुहो॒मीतीति॑ जुहोमि॒ ज्योति॑ष्मन्त॒म् ज्योति॑ष्मन्तम् जुहो॒मीति॑ । \newline
59. जु॒हो॒मीतीति॑ जुहोमि जुहो॒मी त्या॑हा॒हेति॑ जुहोमि जुहो॒मी त्या॑ह । \newline
\pagebreak
\markright{ TS 3.5.9.3  \hfill https://www.vedavms.in \hfill}

\section{ TS 3.5.9.3 }

\textbf{TS 3.5.9.3 } \newline
\textbf{Samhita Paata} \newline

-त्या॑ह प्र॒जाप॑तिः॒ सर्वा॑ दे॒वताः॒ सर्वा᳚भ्य ए॒वैनं॑ दे॒वता᳚भ्यो जुहोत्याज्यग्र॒हं गृ॑ह्णीया॒त् तेज॑स्कामस्य॒ तेजो॒ वा आज्यं॑ तेज॒स्व्ये॑व भ॑वति सोमग्र॒हं गृ॑ह्णीयाद्-ब्रह्मवर्च॒सका॑मस्य ब्रह्मवर्च॒सं ॅवै सोमो᳚ ब्रह्मवर्च॒स्ये॑व भ॑वति दधिग्र॒हं गृ॑ह्णीयात् प॒शुका॑म॒स्योर्ग्वै दद्ध्यूर्क् प॒शव॑ ऊ॒र्जैवास्मा॒ ऊर्जं॑ प॒शूनव॑ रुन्धे ॥ \newline

\textbf{Pada Paata} \newline

इति॑ । आ॒ह॒ । प्र॒जाप॑ति॒रिति॑ प्र॒जा - प॒तिः॒ । सर्वाः᳚ । दे॒वताः᳚ । सर्वा᳚भ्यः । ए॒व । ए॒न॒म् । दे॒वता᳚भ्यः । जु॒हो॒ति॒ । आ॒ज्य॒ग्र॒हमित्या᳚ज्य - ग्र॒हम् । गृ॒ह्णी॒या॒त् । तेज॑स्काम॒स्येति॒ तेजः॑ - का॒म॒स्य॒ । तेजः॑ । वै । आज्य᳚म् । ते॒ज॒स्वी । ए॒व । भ॒व॒ति॒ । सो॒म॒ग्र॒हमिति॑ सोम - ग्र॒हम् । गृ॒ह्णी॒या॒त् । ब्र॒ह्म॒व॒र्च॒सका॑म॒स्येति॑ ब्रह्मवर्च॒स - का॒म॒स्य॒ । ब्र॒ह्म॒व॒र्च॒समिति॑ ब्रह्म - व॒र्च॒सम् । वै । सोमः॑ । ब्र॒ह्म॒व॒र्च॒सीति॑ ब्रह्म - व॒र्च॒सी । ए॒व । भ॒व॒ति॒ । द॒धि॒ग्र॒हमिति॑ दधि - ग्र॒हम् । गृ॒ह्णी॒या॒त् । प॒शुका॑म॒स्येति॑ प॒शु - का॒म॒स्य॒ । ऊर्क् । वै । दधि॑ । ऊर्क् । प॒शवः॑ । ऊ॒र्जा । ए॒व । अ॒स्मै॒ । ऊर्ज᳚म् । प॒शून् । अवेति॑ । रु॒न्धे॒ ॥  \newline


\textbf{Krama Paata} \newline

इत्या॑ह । आ॒ह॒ प्र॒जाप॑तिः । प्र॒जाप॑तिः॒ सर्वाः᳚ । प्र॒जाप॑ति॒रिति॑ प्र॒जा - प॒तिः॒ । सर्वा॑ दे॒वताः᳚ । दे॒वताः॒ सर्वा᳚भ्यः । सर्वा᳚भ्य ए॒व । ए॒वैन᳚म् । ए॒न॒म् दे॒वता᳚भ्यः । दे॒वता᳚भ्यो जुहोति । जु॒हो॒त्या॒ज्य॒ग्र॒हम् । आ॒ज्य॒ग्र॒हम् गृ॑ह्णीयात् । आ॒ज्य॒ग्र॒हमित्या᳚ज्य - ग्र॒हम् । गृ॒ह्णी॒या॒त् तेज॑स्कामस्य । तेज॑स्कामस्य॒ तेजः॑ । तेज॑स्काम॒स्येति॒ तेजः॑ - का॒म॒स्य॒ । तेजो॒ वै । वा आज्य᳚म् । आज्य॑म् तेज॒स्वी । ते॒ज॒स्व्ये॑व । ए॒व भ॑वति । भ॒व॒ति॒ सो॒म॒ग्र॒हम् । सो॒म॒ग्र॒हम् गृ॑ह्णीयात् । सो॒म॒ग्र॒हमिति॑ सोम - ग्र॒हम् । गृ॒ह्णी॒या॒द् ब्र॒ह्म॒व॒र्च॒सका॑मस्य । 
ब्र॒ह्म॒व॒र्च॒सका॑मस्य ब्रह्मवर्च॒सम् । ब्र॒ह्म॒व॒र्च॒सका॑म॒स्येति॑ ब्रह्मवर्च॒स - का॒म॒स्य॒ । ब्र॒ह्म॒व॒र्च॒सं ॅवै । ब्र॒ह्म॒व॒र्च॒समिति॑ ब्रह्म - व॒र्च॒सम् । वै सोमः॑ । सोमो᳚ ब्रह्मवर्च॒सी । ब्र॒ह्म॒व॒र्च॒स्ये॑व । ब्र॒ह्म॒व॒र्च॒सीति॑ ब्रह्म - व॒र्च॒सी । ए॒व भ॑वति । भ॒व॒ति॒ द॒धि॒ग्र॒हम् । द॒धि॒ग्र॒हम् गृ॑ह्णीयात् । द॒धि॒ग्र॒हमिति॑ दधि - ग्र॒हम् । गृ॒ह्णी॒या॒त् प॒शुका॑मस्य । प॒शुका॑म॒स्योर्क् । प॒शुका॑म॒स्येति॑ प॒शु - का॒म॒स्य॒ । ऊर्ग् वै । वै दधि॑ । दध्यूर्क् । ऊर्क् प॒शवः॑ । प॒शव॑ ऊ॒र्जा । ऊ॒र्जैव । ए॒वास्मै᳚ । अ॒स्मा॒ ऊर्ज᳚म् । ऊर्ज॑म् प॒शून् । प॒शूनव॑ । अव॑ रुन्धे । रु॒न्ध॒ इति॑ रुन्धे । \newline

\textbf{Jatai Paata} \newline

1. इत्या॑ हा॒हे तीत्या॑ह । \newline
2. आ॒ह॒ प्र॒जाप॑तिः प्र॒जाप॑ति राहाह प्र॒जाप॑तिः । \newline
3. प्र॒जाप॑तिः॒ सर्वाः॒ सर्वाः᳚ प्र॒जाप॑तिः प्र॒जाप॑तिः॒ सर्वाः᳚ । \newline
4. प्र॒जाप॑ति॒रिति॑ प्र॒जा - प॒तिः॒ । \newline
5. सर्वा॑ दे॒वता॑ दे॒वताः॒ सर्वाः॒ सर्वा॑ दे॒वताः᳚ । \newline
6. दे॒वताः॒ सर्वा᳚भ्यः॒ सर्वा᳚भ्यो दे॒वता॑ दे॒वताः॒ सर्वा᳚भ्यः । \newline
7. सर्वा᳚भ्य ए॒वैव सर्वा᳚भ्यः॒ सर्वा᳚भ्य ए॒व । \newline
8. ए॒वैन॑ मेन मे॒वैवैन᳚म् । \newline
9. ए॒न॒म् दे॒वता᳚भ्यो दे॒वता᳚भ्य एन मेनम् दे॒वता᳚भ्यः । \newline
10. दे॒वता᳚भ्यो जुहोति जुहोति दे॒वता᳚भ्यो दे॒वता᳚भ्यो जुहोति । \newline
11. जु॒हो॒ त्या॒ज्य॒ग्र॒ह मा᳚ज्यग्र॒हम् जु॑होति जुहो त्याज्यग्र॒हम् । \newline
12. आ॒ज्य॒ग्र॒हम् गृ॑ह्णीयाद् गृह्णीया दाज्यग्र॒ह मा᳚ज्यग्र॒हम् गृ॑ह्णीयात् । \newline
13. आ॒ज्य॒ग्र॒हमित्या᳚ज्य - ग्र॒हम् । \newline
14. गृ॒ह्णी॒या॒त् तेज॑स्कामस्य॒ तेज॑स्कामस्य गृह्णीयाद् गृह्णीया॒त् तेज॑स्कामस्य । \newline
15. तेज॑स्कामस्य॒ तेज॒ स्तेज॒ स्तेज॑स्कामस्य॒ तेज॑स्कामस्य॒ तेजः॑ । \newline
16. तेज॑स्काम॒स्येति॒ तेजः॑ - का॒म॒स्य॒ । \newline
17. तेजो॒ वै वै तेज॒ स्तेजो॒ वै । \newline
18. वा आज्य॒ माज्यं॒ ॅवै वा आज्य᳚म् । \newline
19. आज्य॑म् तेज॒स्वी ते॑ज॒ स्व्याज्य॒ माज्य॑म् तेज॒स्वी । \newline
20. ते॒ज॒ स्व्ये॑वैव ते॑ज॒स्वी ते॑ज॒ स्व्ये॑व । \newline
21. ए॒व भ॑वति भव त्ये॒वैव भ॑वति । \newline
22. भ॒व॒ति॒ सो॒म॒ग्र॒हꣳ सो॑मग्र॒हम् भ॑वति भवति सोमग्र॒हम् । \newline
23. सो॒म॒ग्र॒हम् गृ॑ह्णीयाद् गृह्णीयाथ् सोमग्र॒हꣳ सो॑मग्र॒हम् गृ॑ह्णीयात् । \newline
24. सो॒म॒ग्र॒हमिति॑ सोम - ग्र॒हम् । \newline
25. गृ॒ह्णी॒या॒द् ब्र॒ह्म॒व॒र्च॒सका॑मस्य ब्रह्मवर्च॒सका॑मस्य गृह्णीयाद् गृह्णीयाद् ब्रह्मवर्च॒सका॑मस्य । \newline
26. ब्र॒ह्म॒व॒र्च॒सका॑मस्य ब्रह्मवर्च॒सम् ब्र॑ह्मवर्च॒सम् ब्र॑ह्मवर्च॒सका॑मस्य ब्रह्मवर्च॒सका॑मस्य ब्रह्मवर्च॒सम् । \newline
27. ब्र॒ह्म॒व॒र्च॒सका॑म॒स्येति॑ ब्रह्मवर्च॒स - का॒म॒स्य॒ । \newline
28. ब्र॒ह्म॒व॒र्च॒सं ॅवै वै ब्र॑ह्मवर्च॒सम् ब्र॑ह्मवर्च॒सं ॅवै । \newline
29. ब्र॒ह्म॒व॒र्च॒समिति॑ ब्रह्म - व॒र्च॒सम् । \newline
30. वै सोमः॒ सोमो॒ वै वै सोमः॑ । \newline
31. सोमो᳚ ब्रह्मवर्च॒सी ब्र॑ह्मवर्च॒सी सोमः॒ सोमो᳚ ब्रह्मवर्च॒सी । \newline
32. ब्र॒ह्म॒व॒र्च॒ स्ये॑वैव ब्र॑ह्मवर्च॒सी ब्र॑ह्मवर्च॒ स्ये॑व । \newline
33. ब्र॒ह्म॒व॒र्च॒सीति॑ ब्रह्म - व॒र्च॒सी । \newline
34. ए॒व भ॑वति भव त्ये॒वैव भ॑वति । \newline
35. भ॒व॒ति॒ द॒धि॒ग्र॒हम् द॑धिग्र॒हम् भ॑वति भवति दधिग्र॒हम् । \newline
36. द॒धि॒ग्र॒हम् गृ॑ह्णीयाद् गृह्णीयाद् दधिग्र॒हम् द॑धिग्र॒हम् गृ॑ह्णीयात् । \newline
37. द॒धि॒ग्र॒हमिति॑ दधि - ग्र॒हम् । \newline
38. गृ॒ह्णी॒या॒त् प॒शुका॑मस्य प॒शुका॑मस्य गृह्णीयाद् गृह्णीयात् प॒शुका॑मस्य । \newline
39. प॒शुका॑म॒ स्योर्गूर्क् प॒शुका॑मस्य प॒शुका॑म॒ स्योर्क् । \newline
40. प॒शुका॑म॒स्येति॑ प॒शु - का॒म॒स्य॒ । \newline
41. ऊर्ग् वै वा ऊर् गूर्ग् वै । \newline
42. वै दधि॒ दधि॒ वै वै दधि॑ । \newline
43. दध्यूर् गूर्ग् दधि॒ दध्यूर्क् । \newline
44. ऊर्क् प॒शवः॑ प॒शव॒ ऊर् गूर्क् प॒शवः॑ । \newline
45. प॒शव॑ ऊ॒र्जोर्जा प॒शवः॑ प॒शव॑ ऊ॒र्जा । \newline
46. ऊ॒र्जै वैवोर् जोर्जैव । \newline
47. ए॒वास्मा॑ अस्मा ए॒वैवास्मै᳚ । \newline
48. अ॒स्मा॒ ऊर्ज॒ मूर्ज॑ मस्मा अस्मा॒ ऊर्ज᳚म् । \newline
49. ऊर्ज॑म् प॒शून् प॒शू नूर्ज॒ मूर्ज॑म् प॒शून् । \newline
50. प॒शू नवाव॑ प॒शून् प॒शू नव॑ । \newline
51. अव॑ रुन्धे रु॒न्धे ऽवाव॑ रुन्धे । \newline
52. रु॒न्ध॒ इति॑ रुन्धे । \newline

\textbf{Ghana Paata } \newline

1. इत्या॑हा॒हे तीत्या॑ह प्र॒जाप॑तिः प्र॒जाप॑ति रा॒हे तीत्या॑ह प्र॒जाप॑तिः । \newline
2. आ॒ह॒ प्र॒जाप॑तिः प्र॒जाप॑ति राहाह प्र॒जाप॑तिः॒ सर्वाः॒ सर्वाः᳚ प्र॒जाप॑ति राहाह प्र॒जाप॑तिः॒ सर्वाः᳚ । \newline
3. प्र॒जाप॑तिः॒ सर्वाः॒ सर्वाः᳚ प्र॒जाप॑तिः प्र॒जाप॑तिः॒ सर्वा॑ दे॒वता॑ दे॒वताः॒ सर्वाः᳚ प्र॒जाप॑तिः प्र॒जाप॑तिः॒ सर्वा॑ दे॒वताः᳚ । \newline
4. प्र॒जाप॑ति॒रिति॑ प्र॒जा - प॒तिः॒ । \newline
5. सर्वा॑ दे॒वता॑ दे॒वताः॒ सर्वाः॒ सर्वा॑ दे॒वताः॒ सर्वा᳚भ्यः॒ सर्वा᳚भ्यो दे॒वताः॒ सर्वाः॒ सर्वा॑ दे॒वताः॒ सर्वा᳚भ्यः । \newline
6. दे॒वताः॒ सर्वा᳚भ्यः॒ सर्वा᳚भ्यो दे॒वता॑ दे॒वताः॒ सर्वा᳚भ्य ए॒वैव सर्वा᳚भ्यो दे॒वता॑ दे॒वताः॒ सर्वा᳚भ्य ए॒व । \newline
7. सर्वा᳚भ्य ए॒वैव सर्वा᳚भ्यः॒ सर्वा᳚भ्य ए॒वैन॑ मेन मे॒व सर्वा᳚भ्यः॒ सर्वा᳚भ्य ए॒वैन᳚म् । \newline
8. ए॒वैन॑ मेन मे॒वैवैन॑म् दे॒वता᳚भ्यो दे॒वता᳚भ्य एन मे॒वैवैन॑म् दे॒वता᳚भ्यः । \newline
9. ए॒न॒म् दे॒वता᳚भ्यो दे॒वता᳚भ्य एन मेनम् दे॒वता᳚भ्यो जुहोति जुहोति दे॒वता᳚भ्य एन मेनम् दे॒वता᳚भ्यो जुहोति । \newline
10. दे॒वता᳚भ्यो जुहोति जुहोति दे॒वता᳚भ्यो दे॒वता᳚भ्यो जुहो त्याज्यग्र॒ह मा᳚ज्यग्र॒हम् जु॑होति दे॒वता᳚भ्यो दे॒वता᳚भ्यो जुहो त्याज्यग्र॒हम् । \newline
11. जु॒हो॒ त्या॒ज्य॒ग्र॒ह मा᳚ज्यग्र॒हम् जु॑होति जुहो त्याज्यग्र॒हम् गृ॑ह्णीयाद् गृह्णीया दाज्यग्र॒हम् जु॑होति जुहो त्याज्यग्र॒हम् गृ॑ह्णीयात् । \newline
12. आ॒ज्य॒ग्र॒हम् गृ॑ह्णीयाद् गृह्णीया दाज्यग्र॒ह मा᳚ज्यग्र॒हम् गृ॑ह्णीया॒त् तेज॑स्कामस्य॒ तेज॑स्कामस्य गृह्णीया दाज्यग्र॒ह मा᳚ज्यग्र॒हम् गृ॑ह्णीया॒त् तेज॑स्कामस्य । \newline
13. आ॒ज्य॒ग्र॒हमित्या᳚ज्य - ग्र॒हम् । \newline
14. गृ॒ह्णी॒या॒त् तेज॑स्कामस्य॒ तेज॑स्कामस्य गृह्णीयाद् गृह्णीया॒त् तेज॑स्कामस्य॒ तेज॒ स्तेज॒ स्तेज॑स्कामस्य गृह्णीयाद् गृह्णीया॒त् तेज॑स्कामस्य॒ तेजः॑ । \newline
15. तेज॑स्कामस्य॒ तेज॒ स्तेज॒ स्तेज॑स्कामस्य॒ तेज॑स्कामस्य॒ तेजो॒ वै वै तेज॒ स्तेज॑स्कामस्य॒ तेज॑स्कामस्य॒ तेजो॒ वै । \newline
16. तेज॑स्काम॒स्येति॒ तेजः॑ - का॒म॒स्य॒ । \newline
17. तेजो॒ वै वै तेज॒ स्तेजो॒ वा आज्य॒ माज्यं॒ ॅवै तेज॒ स्तेजो॒ वा आज्य᳚म् । \newline
18. वा आज्य॒ माज्यं॒ ॅवै वा आज्य॑म् तेज॒स्वी ते॑ज॒ स्व्याज्यं॒ ॅवै वा आज्य॑म् तेज॒स्वी । \newline
19. आज्य॑म् तेज॒स्वी ते॑ज॒ स्व्याज्य॒ माज्य॑म् तेज॒स्व्ये॑वैव ते॑ज॒ स्व्याज्य॒ माज्य॑म् तेज॒ स्व्ये॑व । \newline
20. ते॒ज॒ स्व्ये॑वैव ते॑ज॒स्वी ते॑ज॒ स्व्ये॑व भ॑वति भव त्ये॒व ते॑ज॒स्वी ते॑ज॒ स्व्ये॑व भ॑वति । \newline
21. ए॒व भ॑वति भव त्ये॒वैव भ॑वति सोमग्र॒हꣳ सो॑मग्र॒हम् भ॑व त्ये॒वैव भ॑वति सोमग्र॒हम् । \newline
22. भ॒व॒ति॒ सो॒म॒ग्र॒हꣳ सो॑मग्र॒हम् भ॑वति भवति सोमग्र॒हम् गृ॑ह्णीयाद् गृह्णीयाथ् सोमग्र॒हम् भ॑वति भवति सोमग्र॒हम् गृ॑ह्णीयात् । \newline
23. सो॒म॒ग्र॒हम् गृ॑ह्णीयाद् गृह्णीयाथ् सोमग्र॒हꣳ सो॑मग्र॒हम् गृ॑ह्णीयाद् ब्रह्मवर्च॒सका॑मस्य ब्रह्मवर्च॒सका॑मस्य गृह्णीयाथ् सोमग्र॒हꣳ सो॑मग्र॒हम् गृ॑ह्णीयाद् ब्रह्मवर्च॒सका॑मस्य । \newline
24. सो॒म॒ग्र॒हमिति॑ सोम - ग्र॒हम् । \newline
25. गृ॒ह्णी॒या॒द् ब्र॒ह्म॒व॒र्च॒सका॑मस्य ब्रह्मवर्च॒सका॑मस्य गृह्णीयाद् गृह्णीयाद् ब्रह्मवर्च॒सका॑मस्य ब्रह्मवर्च॒सम् ब्र॑ह्मवर्च॒सम् ब्र॑ह्मवर्च॒सका॑मस्य गृह्णीयाद् गृह्णीयाद् ब्रह्मवर्च॒सका॑मस्य ब्रह्मवर्च॒सम् । \newline
26. ब्र॒ह्म॒व॒र्च॒सका॑मस्य ब्रह्मवर्च॒सम् ब्र॑ह्मवर्च॒सम् ब्र॑ह्मवर्च॒सका॑मस्य ब्रह्मवर्च॒सका॑मस्य ब्रह्मवर्च॒सं ॅवै वै ब्र॑ह्मवर्च॒सम् ब्र॑ह्मवर्च॒सका॑मस्य ब्रह्मवर्च॒सका॑मस्य ब्रह्मवर्च॒सं ॅवै । \newline
27. ब्र॒ह्म॒व॒र्च॒सका॑म॒स्येति॑ ब्रह्मवर्च॒स - का॒म॒स्य॒ । \newline
28. ब्र॒ह्म॒व॒र्च॒सं ॅवै वै ब्र॑ह्मवर्च॒सम् ब्र॑ह्मवर्च॒सं ॅवै सोमः॒ सोमो॒ वै ब्र॑ह्मवर्च॒सम् ब्र॑ह्मवर्च॒सं ॅवै सोमः॑ । \newline
29. ब्र॒ह्म॒व॒र्च॒समिति॑ ब्रह्म - व॒र्च॒सम् । \newline
30. वै सोमः॒ सोमो॒ वै वै सोमो᳚ ब्रह्मवर्च॒सी ब्र॑ह्मवर्च॒सी सोमो॒ वै वै सोमो᳚ ब्रह्मवर्च॒सी । \newline
31. सोमो᳚ ब्रह्मवर्च॒सी ब्र॑ह्मवर्च॒सी सोमः॒ सोमो᳚ ब्रह्मवर्च॒ स्ये॑वैव ब्र॑ह्मवर्च॒सी सोमः॒ सोमो᳚ ब्रह्मवर्च॒ स्ये॑व । \newline
32. ब्र॒ह्म॒व॒र्च॒ स्ये॑वैव ब्र॑ह्मवर्च॒सी ब्र॑ह्मवर्च॒ स्ये॑व भ॑वति भव त्ये॒व ब्र॑ह्मवर्च॒सी ब्र॑ह्मवर्च॒ स्ये॑व भ॑वति । \newline
33. ब्र॒ह्म॒व॒र्च॒सीति॑ ब्रह्म - व॒र्च॒सी । \newline
34. ए॒व भ॑वति भव त्ये॒वैव भ॑वति दधिग्र॒हम् द॑धिग्र॒हम् भ॑व त्ये॒वैव भ॑वति दधिग्र॒हम् । \newline
35. भ॒व॒ति॒ द॒धि॒ग्र॒हम् द॑धिग्र॒हम् भ॑वति भवति दधिग्र॒हम् गृ॑ह्णीयाद् गृह्णीयाद् दधिग्र॒हम् भ॑वति भवति दधिग्र॒हम् गृ॑ह्णीयात् । \newline
36. द॒धि॒ग्र॒हम् गृ॑ह्णीयाद् गृह्णीयाद् दधिग्र॒हम् द॑धिग्र॒हम् गृ॑ह्णीयात् प॒शुका॑मस्य प॒शुका॑मस्य गृह्णीयाद् दधिग्र॒हम् द॑धिग्र॒हम् गृ॑ह्णीयात् प॒शुका॑मस्य । \newline
37. द॒धि॒ग्र॒हमिति॑ दधि - ग्र॒हम् । \newline
38. गृ॒ह्णी॒या॒त् प॒शुका॑मस्य प॒शुका॑मस्य गृह्णीयाद् गृह्णीयात् प॒शुका॑म॒ स्योर्गूर्क् प॒शुका॑मस्य गृह्णीयाद् गृह्णीयात् प॒शुका॑म॒ स्योर्क् । \newline
39. प॒शुका॑म॒ स्योर्गूर्क् प॒शुका॑मस्य प॒शुका॑म॒ स्योर्ग् वै वा ऊर्क् प॒शुका॑मस्य प॒शुका॑म॒ स्योर्ग् वै । \newline
40. प॒शुका॑म॒स्येति॑ प॒शु - का॒म॒स्य॒ । \newline
41. ऊर्ग् वै वा ऊर्गूर्ग् वै दधि॒ दधि॒ वा ऊर्गूर्ग् वै दधि॑ । \newline
42. वै दधि॒ दधि॒ वै वै दध्यूर्गूर्ग् दधि॒ वै वै दध्यूर्क् । \newline
43. दध्यूर्गूर्ग् दधि॒ दध्यूर्क् प॒शवः॑ प॒शव॒ ऊर्ग् दधि॒ दध्यूर्क् प॒शवः॑ । \newline
44. ऊर्क् प॒शवः॑ प॒शव॒ ऊर्गूर्क् प॒शव॑ ऊ॒र्जोर्जा प॒शव॒ ऊर्गूर्क् प॒शव॑ ऊ॒र्जा । \newline
45. प॒शव॑ ऊ॒र्जोर्जा प॒शवः॑ प॒शव॑ ऊ॒र्जै वैवोर्जा प॒शवः॑ प॒शव॑ ऊ॒र्जैव । \newline
46. ऊ॒र्जै वैवोर्जो र्जैवास्मा॑ अस्मा ए॒वोर्जो र्जैवास्मै᳚ । \newline
47. ए॒वास्मा॑ अस्मा ए॒वैवास्मा॒ ऊर्ज॒ मूर्ज॑ मस्मा ए॒वैवास्मा॒ ऊर्ज᳚म् । \newline
48. अ॒स्मा॒ ऊर्ज॒ मूर्ज॑ मस्मा अस्मा॒ ऊर्ज॑म् प॒शून् प॒शू नूर्ज॑ मस्मा अस्मा॒ ऊर्ज॑म् प॒शून् । \newline
49. ऊर्ज॑म् प॒शून् प॒शू नूर्ज॒ मूर्ज॑म् प॒शू नवाव॑ प॒शू नूर्ज॒ मूर्ज॑म् प॒शू नव॑ । \newline
50. प॒शू नवाव॑ प॒शून् प॒शू नव॑ रुन्धे रु॒न्धे ऽव॑ प॒शून् प॒शू नव॑ रुन्धे । \newline
51. अव॑ रुन्धे रु॒न्धे ऽवाव॑ रुन्धे । \newline
52. रु॒न्ध॒ इति॑ रुन्धे । \newline
\pagebreak
\markright{ TS 3.5.10.1  \hfill https://www.vedavms.in \hfill}

\section{ TS 3.5.10.1 }

\textbf{TS 3.5.10.1 } \newline
\textbf{Samhita Paata} \newline

त्वे क्रतु॒मपि॑ वृञ्जन्ति॒ विश्वे॒ द्विर्यदे॒ते त्रि-र्भव॒न्त्यूमाः᳚ । स्वा॒दोः स्वादी॑यः स्वा॒दुना॑ सृजा॒ समत॑ ऊ॒ षु मधु॒ मधु॑ना॒ऽभि यो॑धि ।उ॒प॒या॒मगृ॑हीतोऽसि प्र॒जाप॑तये त्वा॒ जुष्टं॑ गृह्णाम्ये॒ष ते॒ योनिः॑ प्र॒जाप॑तये त्वा ॥प्रा॒ण॒ग्र॒हान् गृ॑ह्णात्ये॒ताव॒द्वा अ॑स्ति॒ याव॑दे॒ते ग्रहाः॒ स्तोमा॒श्छन्दाꣳ॑सि पृ॒ष्ठानि॒ दिशो॒ याव॑दे॒वास्ति॒ त - [  ] \newline

\textbf{Pada Paata} \newline

त्वे इति॑ । क्रतु᳚म् । अपीति॑ । वृ॒ञ्ज॒न्ति॒ । विश्वे᳚ । द्विः । यत् । ए॒ते । त्रिः । भव॑न्ति । ऊमाः᳚ ॥ स्वा॒दोः । स्वादी॑यः । स्वा॒दुना᳚ । सृ॒ज॒ । समिति॑ । अतः॑ । उ॒ । स्विति॑ । मधु॑ । मधु॑ना । अ॒भीति॑ । यो॒धि॒ ॥ उ॒प॒या॒मगृ॑हीत॒ इत्यु॑पया॒म - गृ॒ही॒तः॒ । अ॒सि॒ । प्र॒जाप॑तय॒ इति॑ प्र॒जा - प॒त॒ये॒ । त्वा॒ । जुष्ट᳚म् । गृ॒ह्णा॒मि॒ । ए॒षः । ते॒ । योनिः॑ । प्र॒जाप॑तय॒ इति॑ प्र॒जा - प॒त॒ये॒ । त्वा॒ ॥ प्रा॒ण॒ग्र॒हानिति॑ प्राण-ग्र॒हान् । गृ॒ह्णा॒ति॒ । ए॒ताव॑त् । वै । अ॒स्ति॒ । याव॑त् । ए॒ते । ग्रहाः᳚ । स्तोमाः᳚ । छन्दाꣳ॑सि । पृ॒ष्ठानि॑ । दिशः॑ । याव॑त् । ए॒व । अस्ति॑ । तत् ।  \newline


\textbf{Krama Paata} \newline

त्वे क्रतु᳚म् । त्वे इति॒ त्वे । क्रतु॒मपि॑ । अपि॑ वृञ्जन्ति । वृ॒ञ्ज॒न्ति॒ विश्वे᳚ । विश्वे॒ द्विः । द्विर् यत् । यदे॒ते । ए॒ते त्रिः । त्रिर् भव॑न्ति । भव॒न्त्यूमाः᳚ । ऊमा॒ इत्यूमाः᳚ ॥ स्वा॒दोः स्वादी॑यः । स्वादी॑यः स्वा॒दुना᳚ । स्वा॒दुना॑ सृज । सृ॒जा॒ सम् । स मतः॑ । 
अत॑ उ । ऊ॒ षु । सु मधु॑ । मधु॒ मधु॑ना । मधु॑ना॒ ऽभि । अ॒भि यो॑धि । यो॒धीति॑ योधि ॥ उ॒प॒या॒मगृ॑हीतो ऽसि । उ॒प॒या॒मगृ॑हीत॒ इत्यु॑पया॒म - गृ॒ही॒तः॒ । अ॒सि॒ प्र॒जाप॑तये । प्र॒जाप॑तये त्वा । प्र॒जाप॑तय॒ इति॑ प्र॒जा - प॒त॒ये॒ । त्वा॒ जुष्ट᳚म् । जुष्ट॑म् गृह्णामि । गृ॒ह्णा॒म्ये॒षः । ए॒ष ते᳚ । ते॒ योनिः॑ । योनिः॑ प्र॒जाप॑तये । प्र॒जाप॑तये त्वा । प्र॒जाप॑तय॒ इति॑ प्र॒जा - प॒त॒ये॒ । त्वेति॑ त्वा ॥ प्रा॒ण॒ग्र॒हान् गृ॑ह्णाति । प्रा॒ण॒ग्र॒हानिति॑ प्राण - ग्र॒हान् । गृ॒ह्णा॒त्ये॒ताव॑त् । ए॒ताव॒द् वै । वा अ॑स्ति । अ॒स्ति॒ याव॑त् । याव॑दे॒ते । ए॒ते ग्रहाः᳚ । ग्रहाः॒ स्तोमाः᳚ । स्तोमा॒ श्छन्दाꣳ॑सि । छन्दाꣳ॑सि पृ॒ष्ठानि॑ । पृ॒ष्ठानि॒ दिशः॑ । दिशो॒ याव॑त् । याव॑दे॒व । ए॒वास्ति॑ । अस्ति॒ तत् । तदव॑ \newline

\textbf{Jatai Paata} \newline

1. त्वे क्रतु॒म् क्रतु॒म् त्वे त्वे क्रतु᳚म् । \newline
2. त्वे इति॒ त्वे । \newline
3. क्रतु॒ मप्यपि॒ क्रतु॒म् क्रतु॒ मपि॑ । \newline
4. अपि॑ वृञ्जन्ति वृञ्ज॒ न्त्यप्यपि॑ वृञ्जन्ति । \newline
5. वृ॒ञ्ज॒न्ति॒ विश्वे॒ विश्वे॑ वृञ्जन्ति वृञ्जन्ति॒ विश्वे᳚ । \newline
6. विश्वे॒ द्विर् द्विर् विश्वे॒ विश्वे॒ द्विः । \newline
7. द्विर् यद् यद् द्विर् द्विर् यत् । \newline
8. यदे॒त ए॒ते यद् यदे॒ते । \newline
9. ए॒ते त्रि स्त्रि रे॒त ए॒ते त्रिः । \newline
10. त्रिर् भव॑न्ति॒ भव॑न्ति॒ त्रि स्त्रिर् भव॑न्ति । \newline
11. भव॒ न्त्यूमा॒ ऊमा॒ भव॑न्ति॒ भव॒ न्त्यूमाः᳚ । \newline
12. ऊमा॒ इत्यूमाः᳚ । \newline
13. स्वा॒दोः स्वादी॑यः॒ स्वादी॑यः स्वा॒दोः स्वा॒दोः स्वादी॑यः । \newline
14. स्वादी॑यः स्वा॒दुना᳚ स्वा॒दुना॒ स्वादी॑यः॒ स्वादी॑यः स्वा॒दुना᳚ । \newline
15. स्वा॒दुना॑ सृज सृज स्वा॒दुना᳚ स्वा॒दुना॑ सृज । \newline
16. सृ॒जा॒ सꣳ सꣳ सृ॑ज सृजा॒ सम् । \newline
17. स मतो ऽतः॒ सꣳ स मतः॑ । \newline
18. अत॑ उ वु॒ वतो ऽत॑ उ । \newline
19. ऊ॒ षु सू॑ षु । \newline
20. सु मधु॒ मधु॒ सु सु मधु॑ । \newline
21. मधु॒ मधु॑ना॒ मधु॑ना॒ मधु॒ मधु॒ मधु॑ना । \newline
22. मधु॑ना॒ ऽभ्य॑भि मधु॑ना॒ मधु॑ना॒ ऽभि । \newline
23. अ॒भि यो॑धि योध्य॒भ्य॑भि यो॑धि । \newline
24. यो॒धीति॑ योधि । \newline
25. उ॒प॒या॒मगृ॑हीतो ऽस्य स्युपया॒मगृ॑हीत उपया॒मगृ॑हीतो ऽसि । \newline
26. उ॒प॒या॒मगृ॑हीत॒ इत्यु॑पया॒म - गृ॒ही॒तः॒ । \newline
27. अ॒सि॒ प्र॒जाप॑तये प्र॒जाप॑तये ऽस्यसि प्र॒जाप॑तये । \newline
28. प्र॒जाप॑तये त्वा त्वा प्र॒जाप॑तये प्र॒जाप॑तये त्वा । \newline
29. प्र॒जाप॑तय॒ इति॑ प्र॒जा - प॒त॒ये॒ । \newline
30. त्वा॒ जुष्ट॒म् जुष्ट॑म् त्वा त्वा॒ जुष्ट᳚म् । \newline
31. जुष्ट॑म् गृह्णामि गृह्णामि॒ जुष्ट॒म् जुष्ट॑म् गृह्णामि । \newline
32. गृ॒ह्णा॒ म्ये॒ष ए॒ष गृ॑ह्णामि गृह्णा म्ये॒षः । \newline
33. ए॒ष ते॑ त ए॒ष ए॒ष ते᳚ । \newline
34. ते॒ योनि॒र् योनि॑ स्ते ते॒ योनिः॑ । \newline
35. योनिः॑ प्र॒जाप॑तये प्र॒जाप॑तये॒ योनि॒र् योनिः॑ प्र॒जाप॑तये । \newline
36. प्र॒जाप॑तये त्वा त्वा प्र॒जाप॑तये प्र॒जाप॑तये त्वा । \newline
37. प्र॒जाप॑तय॒ इति॑ प्र॒जा - प॒त॒ये॒ । \newline
38. त्वेति॑ त्वा । \newline
39. प्रा॒ण॒ग्र॒हान् गृ॑ह्णाति गृह्णाति प्राणग्र॒हान् प्रा॑णग्र॒हान् गृ॑ह्णाति । \newline
40. प्रा॒ण॒ग्र॒हानिति॑ प्राण - ग्र॒हान् । \newline
41. गृ॒ह्णा॒ त्ये॒ताव॑ दे॒ताव॑द् गृह्णाति गृह्णा त्ये॒ताव॑त् । \newline
42. ए॒ताव॒द् वै वा ए॒ताव॑ दे॒ताव॒द् वै । \newline
43. वा अ॑स्त्यस्ति॒ वै वा अ॑स्ति । \newline
44. अ॒स्ति॒ याव॒द् याव॑ दस्त्यस्ति॒ याव॑त् । \newline
45. याव॑ दे॒त ए॒ते याव॒द् याव॑ दे॒ते । \newline
46. ए॒ते ग्रहा॒ ग्रहा॑ ए॒त ए॒ते ग्रहाः᳚ । \newline
47. ग्रहाः॒ स्तोमाः॒ स्तोमा॒ ग्रहा॒ ग्रहाः॒ स्तोमाः᳚ । \newline
48. स्तोमा॒ श्छन्दाꣳ॑सि॒ छन्दाꣳ॑सि॒ स्तोमाः॒ स्तोमा॒ श्छन्दाꣳ॑सि । \newline
49. छन्दाꣳ॑सि पृ॒ष्ठानि॑ पृ॒ष्ठानि॒ छन्दाꣳ॑सि॒ छन्दाꣳ॑सि पृ॒ष्ठानि॑ । \newline
50. पृ॒ष्ठानि॒ दिशो॒ दिशः॑ पृ॒ष्ठानि॑ पृ॒ष्ठानि॒ दिशः॑ । \newline
51. दिशो॒ याव॒द् याव॒द् दिशो॒ दिशो॒ याव॑त् । \newline
52. याव॑ दे॒वैव याव॒द् याव॑ दे॒व । \newline
53. ए॒वा स्त्य स्त्ये॒वै वास्ति॑ । \newline
54. अस्ति॒ तत् तद स्त्यस्ति॒ तत् । \newline
55. तदवाव॒ तत् तदव॑ । \newline

\textbf{Ghana Paata } \newline

1. त्वे क्रतु॒म् क्रतु॒म् त्वे त्वे क्रतु॒ मप्यपि॒ क्रतु॒म् त्वे त्वे क्रतु॒ मपि॑ । \newline
2. त्वे इति॒ त्वे । \newline
3. क्रतु॒ मप्यपि॒ क्रतु॒म् क्रतु॒ मपि॑ वृञ्जन्ति वृञ्ज॒ न्त्यपि॒ क्रतु॒म् क्रतु॒ मपि॑ वृञ्जन्ति । \newline
4. अपि॑ वृञ्जन्ति वृञ्ज॒ न्त्यप्यपि॑ वृञ्जन्ति॒ विश्वे॒ विश्वे॑ वृञ्ज॒ न्त्यप्यपि॑ वृञ्जन्ति॒ विश्वे᳚ । \newline
5. वृ॒ञ्ज॒न्ति॒ विश्वे॒ विश्वे॑ वृञ्जन्ति वृञ्जन्ति॒ विश्वे॒ द्विर् द्विर् विश्वे॑ वृञ्जन्ति वृञ्जन्ति॒ विश्वे॒ द्विः । \newline
6. विश्वे॒ द्विर् द्विर् विश्वे॒ विश्वे॒ द्विर् यद् यद् द्विर् विश्वे॒ विश्वे॒ द्विर् यत् । \newline
7. द्विर् यद् यद् द्विर् द्विर् यदे॒त ए॒ते यद् द्विर् द्विर् यदे॒ते । \newline
8. यदे॒त ए॒ते यद् यदे॒ते त्रि स्त्रि रे॒ते यद् यदे॒ते त्रिः । \newline
9. ए॒ते त्रि स्त्रि रे॒त ए॒ते त्रिर् भव॑न्ति॒ भव॑न्ति॒ त्रिरे॒त ए॒ते त्रिर् भव॑न्ति । \newline
10. त्रिर् भव॑न्ति॒ भव॑न्ति॒ त्रिस्त्रिर् भव॒ न्त्यूमा॒ ऊमा॒ भव॑न्ति॒ त्रि स्त्रिर् भव॒ न्त्यूमाः᳚ । \newline
11. भव॒ न्त्यूमा॒ ऊमा॒ भव॑न्ति॒ भव॒ न्त्यूमाः᳚ । \newline
12. ऊमा॒ इत्यूमाः᳚ । \newline
13. स्वा॒दोः स्वादी॑यः॒ स्वादी॑यः स्वा॒दोः स्वा॒दोः स्वादी॑यः स्वा॒दुना᳚ स्वा॒दुना॒ स्वादी॑यः स्वा॒दोः स्वा॒दोः स्वादी॑यः स्वा॒दुना᳚ । \newline
14. स्वादी॑यः स्वा॒दुना᳚ स्वा॒दुना॒ स्वादी॑यः॒ स्वादी॑यः स्वा॒दुना॑ सृज सृज स्वा॒दुना॒ स्वादी॑यः॒ स्वादी॑यः स्वा॒दुना॑ सृज । \newline
15. स्वा॒दुना॑ सृज सृज स्वा॒दुना᳚ स्वा॒दुना॑ सृजा॒ सꣳ सꣳ सृ॑ज स्वा॒दुना᳚ स्वा॒दुना॑ सृजा॒ सम् । \newline
16. सृ॒जा॒ सꣳ सꣳ सृ॑ज सृजा॒ स मतो ऽतः॒ सꣳ सृ॑ज सृजा॒ स मतः॑ । \newline
17. स मतो ऽतः॒ सꣳ स मत॑ उ वु॒ वतः॒ सꣳ स मत॑ उ । \newline
18. अत॑ उ वु॒ वतो ऽत॑ ऊ॒ षु स्व तोऽत॑ ऊ॒ षु । \newline
19. ऊ॒ षु सू॑ षु मधु॒ मधु॒ सू॑ षु मधु॑ । \newline
20. सु मधु॒ मधु॒ सु सु मधु॒ मधु॑ना॒ मधु॑ना॒ मधु॒ सु सु मधु॒ मधु॑ना । \newline
21. मधु॒ मधु॑ना॒ मधु॑ना॒ मधु॒ मधु॒ मधु॑ना॒ ऽभ्य॑भि मधु॑ना॒ मधु॒ मधु॒ मधु॑ना॒ ऽभि । \newline
22. मधु॑ना॒ ऽभ्य॑भि मधु॑ना॒ मधु॑ना॒ ऽभि यो॑धि योध्य॒भि मधु॑ना॒ मधु॑ना॒ ऽभि यो॑धि । \newline
23. अ॒भि यो॑धि योध्य॒ भ्य॑भि यो॑धि । \newline
24. यो॒धीति॑ योधि । \newline
25. उ॒प॒या॒मगृ॑हीतो ऽस्य स्युपया॒मगृ॑हीत उपया॒मगृ॑हीतो ऽसि प्र॒जाप॑तये प्र॒जाप॑तये ऽस्युपया॒मगृ॑हीत उपया॒मगृ॑हीतो ऽसि प्र॒जाप॑तये । \newline
26. उ॒प॒या॒मगृ॑हीत॒ इत्यु॑पया॒म - गृ॒ही॒तः॒ । \newline
27. अ॒सि॒ प्र॒जाप॑तये प्र॒जाप॑तये ऽस्यसि प्र॒जाप॑तये त्वा त्वा प्र॒जाप॑तये ऽस्यसि प्र॒जाप॑तये त्वा । \newline
28. प्र॒जाप॑तये त्वा त्वा प्र॒जाप॑तये प्र॒जाप॑तये त्वा॒ जुष्ट॒म् जुष्ट॑म् त्वा प्र॒जाप॑तये प्र॒जाप॑तये त्वा॒ जुष्ट᳚म् । \newline
29. प्र॒जाप॑तय॒ इति॑ प्र॒जा - प॒त॒ये॒ । \newline
30. त्वा॒ जुष्ट॒म् जुष्ट॑म् त्वा त्वा॒ जुष्ट॑म् गृह्णामि गृह्णामि॒ जुष्ट॑म् त्वा त्वा॒ जुष्ट॑म् गृह्णामि । \newline
31. जुष्ट॑म् गृह्णामि गृह्णामि॒ जुष्ट॒म् जुष्ट॑म् गृह्णा म्ये॒ष ए॒ष गृ॑ह्णामि॒ जुष्ट॒म् जुष्ट॑म् गृह्णा म्ये॒षः । \newline
32. गृ॒ह्णा॒ म्ये॒ष ए॒ष गृ॑ह्णामि गृह्णा म्ये॒ष ते॑ त ए॒ष गृ॑ह्णामि गृह्णा म्ये॒ष ते᳚ । \newline
33. ए॒ष ते॑ त ए॒ष ए॒ष ते॒ योनि॒र् योनि॑ स्त ए॒ष ए॒ष ते॒ योनिः॑ । \newline
34. ते॒ योनि॒र् योनि॑ स्ते ते॒ योनिः॑ प्र॒जाप॑तये प्र॒जाप॑तये॒ योनि॑ स्ते ते॒ योनिः॑ प्र॒जाप॑तये । \newline
35. योनिः॑ प्र॒जाप॑तये प्र॒जाप॑तये॒ योनि॒र् योनिः॑ प्र॒जाप॑तये त्वा त्वा प्र॒जाप॑तये॒ योनि॒र् योनिः॑ प्र॒जाप॑तये त्वा । \newline
36. प्र॒जाप॑तये त्वा त्वा प्र॒जाप॑तये प्र॒जाप॑तये त्वा । \newline
37. प्र॒जाप॑तय॒ इति॑ प्र॒जा - प॒त॒ये॒ । \newline
38. त्वेति॑ त्वा । \newline
39. प्रा॒ण॒ग्र॒हान् गृ॑ह्णाति गृह्णाति प्राणग्र॒हान् प्रा॑णग्र॒हान् गृ॑ह्णा त्ये॒ताव॑ दे॒ताव॑द् गृह्णाति प्राणग्र॒हान् प्रा॑णग्र॒हान् गृ॑ह्णा त्ये॒ताव॑त् । \newline
40. प्रा॒ण॒ग्र॒हानिति॑ प्राण - ग्र॒हान् । \newline
41. गृ॒ह्णा॒ त्ये॒ताव॑ दे॒ताव॑द् गृह्णाति गृह्णा त्ये॒ताव॒द् वै वा ए॒ताव॑द् गृह्णाति गृह्णा त्ये॒ताव॒द् वै । \newline
42. ए॒ताव॒द् वै वा ए॒ताव॑ दे॒ताव॒द् वा अ॑स्त्यस्ति॒ वा ए॒ताव॑ दे॒ताव॒द् वा अ॑स्ति । \newline
43. वा अ॑स्त्यस्ति॒ वै वा अ॑स्ति॒ याव॒द् याव॑ दस्ति॒ वै वा अ॑स्ति॒ याव॑त् । \newline
44. अ॒स्ति॒ याव॒द् याव॑ दस्त्यस्ति॒ याव॑दे॒त ए॒ते याव॑ दस्त्यस्ति॒ याव॑ दे॒ते । \newline
45. याव॑ दे॒त ए॒ते याव॒द् याव॑ दे॒ते ग्रहा॒ ग्रहा॑ ए॒ते याव॒द् याव॑ दे॒ते ग्रहाः᳚ । \newline
46. ए॒ते ग्रहा॒ ग्रहा॑ ए॒त ए॒ते ग्रहाः॒ स्तोमाः॒ स्तोमा॒ ग्रहा॑ ए॒त ए॒ते ग्रहाः॒ स्तोमाः᳚ । \newline
47. ग्रहाः॒ स्तोमाः॒ स्तोमा॒ ग्रहा॒ ग्रहाः॒ स्तोमा॒ श्छन्दाꣳ॑सि॒ छन्दाꣳ॑सि॒ स्तोमा॒ ग्रहा॒ ग्रहाः॒ स्तोमा॒ श्छन्दाꣳ॑सि । \newline
48. स्तोमा॒ श्छन्दाꣳ॑सि॒ छन्दाꣳ॑सि॒ स्तोमाः॒ स्तोमा॒ श्छन्दाꣳ॑सि पृ॒ष्ठानि॑ पृ॒ष्ठानि॒ छन्दाꣳ॑सि॒ स्तोमाः॒ स्तोमा॒ श्छन्दाꣳ॑सि पृ॒ष्ठानि॑ । \newline
49. छन्दाꣳ॑सि पृ॒ष्ठानि॑ पृ॒ष्ठानि॒ छन्दाꣳ॑सि॒ छन्दाꣳ॑सि पृ॒ष्ठानि॒ दिशो॒ दिशः॑ पृ॒ष्ठानि॒ छन्दाꣳ॑सि॒ छन्दाꣳ॑सि पृ॒ष्ठानि॒ दिशः॑ । \newline
50. पृ॒ष्ठानि॒ दिशो॒ दिशः॑ पृ॒ष्ठानि॑ पृ॒ष्ठानि॒ दिशो॒ याव॒द् याव॒द् दिशः॑ पृ॒ष्ठानि॑ पृ॒ष्ठानि॒ दिशो॒ याव॑त् । \newline
51. दिशो॒ याव॒द् याव॒द् दिशो॒ दिशो॒ याव॑ दे॒वैव याव॒द् दिशो॒ दिशो॒ याव॑ दे॒व । \newline
52. याव॑ दे॒वैव याव॒द् याव॑ दे॒वा स्त्य स्त्ये॒व याव॒द् याव॑ दे॒वास्ति॑ । \newline
53. ए॒वा स्त्य स्त्ये॒ वैवास्ति॒ तत् तद स्त्ये॒ वैवास्ति॒ तत् । \newline
54. अस्ति॒ तत् तद स्त्यस्ति॒ तदवाव॒ तद स्त्यस्ति॒ तदव॑ । \newline
55. तदवाव॒ तत् तदव॑ रुन्धे रु॒न्धे ऽव॒ तत् तदव॑ रुन्धे । \newline
\pagebreak
\markright{ TS 3.5.10.2  \hfill https://www.vedavms.in \hfill}

\section{ TS 3.5.10.2 }

\textbf{TS 3.5.10.2 } \newline
\textbf{Samhita Paata} \newline

-दव॑ रुन्धे ज्ये॒ष्ठा वा ए॒तान् ब्रा᳚ह्म॒णाः पु॒रा विदाम॑क्र॒न् तस्मा॒त् तेषाꣳ॒॒ सर्वा॒ दिशो॒ऽभिजि॑ता अभूव॒न्॒. यस्यै॒ ते गृ॒ह्यन्ते॒ ज्यैष्ठ्य॑मे॒व ग॑च्छत्य॒भि दिशो॑ जयति॒ पञ्च॑ गृह्यन्ते॒ पञ्च॒ दिशः॒ सर्वा᳚स्वे॒व दि॒क्ष्-वृ॑द्ध्नुवन्ति॒ नव॑नव गृह्यन्ते॒ नव॒ वै पुरु॑षे प्रा॒णाः प्रा॒णाने॒व यज॑मानेषु दधति प्राय॒णीये॑ चोदय॒नीये॑ च गृह्यन्ते प्रा॒णा वै प्रा॑णग्र॒हाः - [  ] \newline

\textbf{Pada Paata} \newline

अवेति॑ । रु॒न्धे॒ । ज्ये॒ष्ठाः । वै । ए॒तान् । ब्रा॒ह्म॒णाः । पु॒रा । वि॒दाम् । अ॒क्र॒न्न् । तस्मा᳚त् । तेषा᳚म् । सर्वाः᳚ । दिशः॑ । अ॒भिजि॑ता॒ इत्य॒भि - जि॒ताः॒ । अ॒भू॒व॒न्न् । यस्य॑ । ए॒ते । गृ॒ह्यन्ते᳚ । ज्यैष्ठ्य᳚म् । ए॒व । ग॒च्छ॒ति॒ । अ॒भीति॑ । दिशः॑ । ज॒य॒ति॒ । पञ्च॑ । गृ॒ह्य॒न्ते॒ । पञ्च॑ । दिशः॑ । सर्वा॑सु । ए॒व । दि॒क्षु । ऋ॒द्ध्नु॒व॒न्ति॒ । नव॑न॒वेति॒ नव॑ - न॒व॒ । गृ॒ह्य॒न्ते॒ । नव॑ । वै । पुरु॑षे । प्रा॒णा इति॑ प्र - अ॒नाः । प्रा॒णानिति॑ प्र - अ॒नान् । ए॒व । यज॑मानेषु । द॒ध॒ति॒ । प्रा॒य॒णीय॒ इति॑ प्र - अ॒य॒नीये᳚ । च॒ । उ॒द॒य॒नीय॒ इत्यु॑त् - अ॒य॒नीये᳚ । च॒ । गृ॒ह्य॒न्ते॒ । प्रा॒णा इति॑ प्र - अ॒नाः । वै । प्रा॒ण॒ग्र॒हा इति॑ प्राण - ग्र॒हाः ।  \newline


\textbf{Krama Paata} \newline

अव॑ रुन्धे । रु॒न्धे॒ ज्ये॒ष्ठाः । ज्ये॒ष्ठा वै । वा ए॒तान् । ए॒तान् ब्रा᳚ह्म॒णाः । ब्रा॒ह्मा॒णाः पु॒रा । पु॒रा वि॒दाम् । वि॒दाम॑क्रन्न् । अ॒क्र॒न् तस्मा᳚त् । तस्मा॒त् तेषा᳚म् । तेषाꣳ॒॒ सर्वाः᳚ । सर्वा॒ दिशः॑ । दिशो॒ ऽभिजि॑ताः । अ॒भिजि॑ता अभूवन्न् । अ॒भिजि॑ता॒ इत्य॒भि - जि॒ताः॒ । अ॒भू॒व॒न्॒. यस्य॑ । यस्यै॒ते । ए॒ते गृ॒ह्यन्ते᳚ । गृ॒ह्यन्ते॒ ज्यैष्ठ्य᳚म् । ज्यैष्ठ्य॑मे॒व । ए॒व ग॑च्छति । ग॒च्छ॒त्य॒भि । अ॒भि दिशः॑ । दिशो॑ जयति । ज॒य॒ति॒ पञ्च॑ । पञ्च॑ गृह्यन्ते । गृ॒ह्य॒न्ते॒ पञ्च॑ । पञ्च॒ दिशः॑ । दिशः॒ सर्वा॑सु । सर्वा᳚स्वे॒व । ए॒व दि॒क्षु । दि॒क्ष्वृ॑ध्नुवन्ति । ऋ॒ध्नु॒व॒न्ति॒ नव॑नव । नव॑नव गृह्यन्ते । नव॑न॒वेति॒ नव॑ - न॒व॒ । गृ॒ह्य॒न्ते॒ नव॑ । नव॒ वै । वै पुरु॑षे । पुरु॑षे प्रा॒णाः । प्रा॒णाः प्रा॒णान् । प्रा॒णा इति॑ प्र - अ॒नाः । प्रा॒णाने॒व । प्रा॒णानिति॑ प्र - अ॒नान् । ए॒व यज॑मानेषु । यज॑मानेषु दधति । द॒ध॒ति॒ प्रा॒य॒णीये᳚ । प्रा॒य॒णीये॑ च । प्रा॒य॒णीय॒ इति॑ प्र - अ॒य॒नीये᳚ । चो॒द॒य॒नीये᳚ । उ॒द॒य॒नीये॑ च । उ॒द॒य॒नीय॒ इत्यु॑त् - अ॒य॒नीये᳚ । च॒ गृ॒ह्य॒न्ते॒ । गृ॒ह्य॒न्ते॒ प्रा॒णाः । प्रा॒णा वै । प्रा॒णा इति॑ प्र - अ॒नाः । वै प्रा॑णग्र॒हाः ( ) । प्रा॒ण॒ग्र॒हाः प्रा॒णैः । प्रा॒ण॒ग्र॒हा इति॑ प्राण - ग्र॒हाः \newline

\textbf{Jatai Paata} \newline

1. अव॑ रुन्धे रु॒न्धे ऽवाव॑ रुन्धे । \newline
2. रु॒न्धे॒ ज्ये॒ष्ठा ज्ये॒ष्ठा रु॑न्धे रुन्धे ज्ये॒ष्ठाः । \newline
3. ज्ये॒ष्ठा वै वै ज्ये॒ष्ठा ज्ये॒ष्ठा वै । \newline
4. वा ए॒ता ने॒तान्. वै वा ए॒तान् । \newline
5. ए॒तान् ब्रा᳚ह्म॒णा ब्रा᳚ह्म॒णा ए॒ता ने॒तान् ब्रा᳚ह्म॒णाः । \newline
6. ब्रा॒ह्म॒णाः पु॒रा पु॒रा ब्रा᳚ह्म॒णा ब्रा᳚ह्म॒णाः पु॒रा । \newline
7. पु॒रा वि॒दां ॅवि॒दाम् पु॒रा पु॒रा वि॒दाम् । \newline
8. वि॒दा म॑क्रन् नक्रन्. वि॒दां ॅवि॒दा म॑क्रन्न् । \newline
9. अ॒क्र॒न् तस्मा॒त् तस्मा॑ दक्रन् नक्र॒न् तस्मा᳚त् । \newline
10. तस्मा॒त् तेषा॒म् तेषा॒म् तस्मा॒त् तस्मा॒त् तेषा᳚म् । \newline
11. तेषाꣳ॒॒ सर्वाः॒ सर्वा॒ स्तेषा॒म् तेषाꣳ॒॒ सर्वाः᳚ । \newline
12. सर्वा॒ दिशो॒ दिशः॒ सर्वाः॒ सर्वा॒ दिशः॑ । \newline
13. दिशो॒ ऽभिजि॑ता अ॒भिजि॑ता॒ दिशो॒ दिशो॒ ऽभिजि॑ताः । \newline
14. अ॒भिजि॑ता अभूवन् नभूवन् न॒भिजि॑ता अ॒भिजि॑ता अभूवन्न् । \newline
15. अ॒भिजि॑ता॒ इत्य॒भि - जि॒ताः॒ । \newline
16. अ॒भू॒व॒न्॒. यस्य॒ यस्या॑ भूवन् नभूव॒न्॒. यस्य॑ । \newline
17. यस्यै॒त ए॒ते यस्य॒ यस्यै॒ते । \newline
18. ए॒ते गृ॒ह्यन्ते॑ गृ॒ह्यन्त॑ ए॒त ए॒ते गृ॒ह्यन्ते᳚ । \newline
19. गृ॒ह्यन्ते॒ ज्यैष्ठ्य॒म् ज्यैष्ठ्य॑म् गृ॒ह्यन्ते॑ गृ॒ह्यन्ते॒ ज्यैष्ठ्य᳚म् । \newline
20. ज्यैष्ठ्य॑ मे॒वैव ज्यैष्ठ्य॒म् ज्यैष्ठ्य॑ मे॒व । \newline
21. ए॒व ग॑च्छति गच्छ त्ये॒वैव ग॑च्छति । \newline
22. ग॒च्छ॒ त्य॒भ्य॑भि ग॑च्छति गच्छ त्य॒भि । \newline
23. अ॒भि दिशो॒ दिशो॒ ऽभ्य॑भि दिशः॑ । \newline
24. दिशो॑ जयति जयति॒ दिशो॒ दिशो॑ जयति । \newline
25. ज॒य॒ति॒ पञ्च॒ पञ्च॑ जयति जयति॒ पञ्च॑ । \newline
26. पञ्च॑ गृह्यन्ते गृह्यन्ते॒ पञ्च॒ पञ्च॑ गृह्यन्ते । \newline
27. गृ॒ह्य॒न्ते॒ पञ्च॒ पञ्च॑ गृह्यन्ते गृह्यन्ते॒ पञ्च॑ । \newline
28. पञ्च॒ दिशो॒ दिशः॒ पञ्च॒ पञ्च॒ दिशः॑ । \newline
29. दिशः॒ सर्वा॑सु॒ सर्वा॑सु॒ दिशो॒ दिशः॒ सर्वा॑सु । \newline
30. सर्वा᳚ स्वे॒वैव सर्वा॑सु॒ सर्वा᳚ स्वे॒व । \newline
31. ए॒व दि॒क्षु दि॒क्ष्वे॑वैव दि॒क्षु । \newline
32. दि॒क्ष्वृ॑द्ध्नुव न्त्यृद्ध्नुवन्ति दि॒क्षु दि॒क्ष्वृ॑द्ध्नुवन्ति । \newline
33. ऋ॒द्ध्नु॒व॒न्ति॒ नव॑नव॒ नव॑नव र्‌द्ध्नुव न्त्यृद्ध्नुवन्ति॒ नव॑नव । \newline
34. नव॑नव गृह्यन्ते गृह्यन्ते॒ नव॑नव॒ नव॑नव गृह्यन्ते । \newline
35. नव॑न॒वेति॒ नव॑ - न॒व॒ । \newline
36. गृ॒ह्य॒न्ते॒ नव॒ नव॑ गृह्यन्ते गृह्यन्ते॒ नव॑ । \newline
37. नव॒ वै वै नव॒ नव॒ वै । \newline
38. वै पुरु॑षे॒ पुरु॑षे॒ वै वै पुरु॑षे । \newline
39. पुरु॑षे प्रा॒णाः प्रा॒णाः पुरु॑षे॒ पुरु॑षे प्रा॒णाः । \newline
40. प्रा॒णाः प्रा॒णान् प्रा॒णान् प्रा॒णाः प्रा॒णाः प्रा॒णान् । \newline
41. प्रा॒णा इति॑ प्र - अ॒नाः । \newline
42. प्रा॒णा ने॒वैव प्रा॒णान् प्रा॒णा ने॒व । \newline
43. प्रा॒णानिति॑ प्र - अ॒नान् । \newline
44. ए॒व यज॑मानेषु॒ यज॑माने ष्वे॒वैव यज॑मानेषु । \newline
45. यज॑मानेषु दधति दधति॒ यज॑मानेषु॒ यज॑मानेषु दधति । \newline
46. द॒ध॒ति॒ प्रा॒य॒णीये᳚ प्राय॒णीये॑ दधति दधति प्राय॒णीये᳚ । \newline
47. प्रा॒य॒णीये॑ च च प्राय॒णीये᳚ प्राय॒णीये॑ च । \newline
48. प्रा॒य॒णीय॒ इति॑ प्र - अ॒य॒नीये᳚ । \newline
49. चो॒द॒य॒नीय॑ उदय॒नीये॑ च चोदय॒नीये᳚ । \newline
50. उ॒द॒य॒नीये॑ च चोदय॒नीय॑ उदय॒नीये॑ च । \newline
51. उ॒द॒य॒नीय॒ इत्यु॑त् - अ॒य॒नीये᳚ । \newline
52. च॒ गृ॒ह्य॒न्ते॒ गृ॒ह्य॒न्ते॒ च॒ च॒ गृ॒ह्य॒न्ते॒ । \newline
53. गृ॒ह्य॒न्ते॒ प्रा॒णाः प्रा॒णा गृ॑ह्यन्ते गृह्यन्ते प्रा॒णाः । \newline
54. प्रा॒णा वै वै प्रा॒णाः प्रा॒णा वै । \newline
55. प्रा॒णा इति॑ प्र - अ॒नाः । \newline
56. वै प्रा॑णग्र॒हाः प्रा॑णग्र॒हा वै वै प्रा॑णग्र॒हाः । \newline
57. प्रा॒ण॒ग्र॒हाः प्रा॒णैः प्रा॒णैः प्रा॑णग्र॒हाः प्रा॑णग्र॒हाः प्रा॒णैः । \newline
58. प्रा॒ण॒ग्र॒हा इति॑ प्राण - ग्र॒हाः । \newline

\textbf{Ghana Paata } \newline

1. अव॑ रुन्धे रु॒न्धे ऽवाव॑ रुन्धे ज्ये॒ष्ठा ज्ये॒ष्ठा रु॒न्धे ऽवाव॑ रुन्धे ज्ये॒ष्ठाः । \newline
2. रु॒न्धे॒ ज्ये॒ष्ठा ज्ये॒ष्ठा रु॑न्धे रुन्धे ज्ये॒ष्ठा वै वै ज्ये॒ष्ठा रु॑न्धे रुन्धे ज्ये॒ष्ठा वै । \newline
3. ज्ये॒ष्ठा वै वै ज्ये॒ष्ठा ज्ये॒ष्ठा वा ए॒ता ने॒तान्. वै ज्ये॒ष्ठा ज्ये॒ष्ठा वा ए॒तान् । \newline
4. वा ए॒ता ने॒तान्. वै वा ए॒तान् ब्रा᳚ह्म॒णा ब्रा᳚ह्म॒णा ए॒तान्. वै वा ए॒तान् ब्रा᳚ह्म॒णाः । \newline
5. ए॒तान् ब्रा᳚ह्म॒णा ब्रा᳚ह्म॒णा ए॒ता ने॒तान् ब्रा᳚ह्म॒णाः पु॒रा पु॒रा ब्रा᳚ह्म॒णा ए॒ता ने॒तान् ब्रा᳚ह्म॒णाः पु॒रा । \newline
6. ब्रा॒ह्म॒णाः पु॒रा पु॒रा ब्रा᳚ह्म॒णा ब्रा᳚ह्म॒णाः पु॒रा वि॒दां ॅवि॒दाम् पु॒रा ब्रा᳚ह्म॒णा ब्रा᳚ह्म॒णाः पु॒रा वि॒दाम् । \newline
7. पु॒रा वि॒दां ॅवि॒दाम् पु॒रा पु॒रा वि॒दा म॑क्रन् नक्रन्. वि॒दाम् पु॒रा पु॒रा वि॒दा म॑क्रन्न् । \newline
8. वि॒दा म॑क्रन् नक्रन्. वि॒दां ॅवि॒दा म॑क्र॒न् तस्मा॒त् तस्मा॑ दक्रन्. वि॒दां ॅवि॒दा म॑क्र॒न् तस्मा᳚त् । \newline
9. अ॒क्र॒न् तस्मा॒त् तस्मा॑ दक्रन् नक्र॒न् तस्मा॒त् तेषा॒म् तेषा॒म् तस्मा॑ दक्रन् नक्र॒न् तस्मा॒त् तेषा᳚म् । \newline
10. तस्मा॒त् तेषा॒म् तेषा॒म् तस्मा॒त् तस्मा॒त् तेषाꣳ॒॒ सर्वाः॒ सर्वा॒ स्तेषा॒म् तस्मा॒त् तस्मा॒त् तेषाꣳ॒॒ सर्वाः᳚ । \newline
11. तेषाꣳ॒॒ सर्वाः॒ सर्वा॒ स्तेषा॒म् तेषाꣳ॒॒ सर्वा॒ दिशो॒ दिशः॒ सर्वा॒ स्तेषा॒म् तेषाꣳ॒॒ सर्वा॒ दिशः॑ । \newline
12. सर्वा॒ दिशो॒ दिशः॒ सर्वाः॒ सर्वा॒ दिशो॒ ऽभिजि॑ता अ॒भिजि॑ता॒ दिशः॒ सर्वाः॒ सर्वा॒ दिशो॒ ऽभिजि॑ताः । \newline
13. दिशो॒ ऽभिजि॑ता अ॒भिजि॑ता॒ दिशो॒ दिशो॒ ऽभिजि॑ता अभूवन् नभूवन् न॒भिजि॑ता॒ दिशो॒ दिशो॒ ऽभिजि॑ता अभूवन्न् । \newline
14. अ॒भिजि॑ता अभूवन् नभूवन् न॒भिजि॑ता अ॒भिजि॑ता अभूव॒न्॒. यस्य॒ यस्या॑ भूवन् न॒भिजि॑ता अ॒भिजि॑ता अभूव॒न्॒. यस्य॑ । \newline
15. अ॒भिजि॑ता॒ इत्य॒भि - जि॒ताः॒ । \newline
16. अ॒भू॒व॒न्॒. यस्य॒ यस्या॑ भूवन् नभूव॒न्॒. यस्यै॒त ए॒ते यस्या॑ भूवन् नभूव॒न्॒. यस्यै॒ते । \newline
17. यस्यै॒त ए॒ते यस्य॒ यस्यै॒ते गृ॒ह्यन्ते॑ गृ॒ह्यन्त॑ ए॒ते यस्य॒ यस्यै॒ते गृ॒ह्यन्ते᳚ । \newline
18. ए॒ते गृ॒ह्यन्ते॑ गृ॒ह्यन्त॑ ए॒त ए॒ते गृ॒ह्यन्ते॒ ज्यैष्ठ्य॒म् ज्यैष्ठ्य॑म् गृ॒ह्यन्त॑ ए॒त ए॒ते गृ॒ह्यन्ते॒ ज्यैष्ठ्य᳚म् । \newline
19. गृ॒ह्यन्ते॒ ज्यैष्ठ्य॒म् ज्यैष्ठ्य॑म् गृ॒ह्यन्ते॑ गृ॒ह्यन्ते॒ ज्यैष्ठ्य॑ मे॒वैव ज्यैष्ठ्य॑म् गृ॒ह्यन्ते॑ गृ॒ह्यन्ते॒ ज्यैष्ठ्य॑ मे॒व । \newline
20. ज्यैष्ठ्य॑ मे॒वैव ज्यैष्ठ्य॒म् ज्यैष्ठ्य॑ मे॒व ग॑च्छति गच्छ त्ये॒व ज्यैष्ठ्य॒म् ज्यैष्ठ्य॑ मे॒व ग॑च्छति । \newline
21. ए॒व ग॑च्छति गच्छ त्ये॒वैव ग॑च्छ त्य॒भ्य॑भि ग॑च्छ त्ये॒वैव ग॑च्छ त्य॒भि । \newline
22. ग॒च्छ॒ त्य॒भ्य॑भि ग॑च्छति गच्छ त्य॒भि दिशो॒ दिशो॒ ऽभि ग॑च्छति गच्छ त्य॒भि दिशः॑ । \newline
23. अ॒भि दिशो॒ दिशो॒ ऽभ्य॑भि दिशो॑ जयति जयति॒ दिशो॒ ऽभ्य॑भि दिशो॑ जयति । \newline
24. दिशो॑ जयति जयति॒ दिशो॒ दिशो॑ जयति॒ पञ्च॒ पञ्च॑ जयति॒ दिशो॒ दिशो॑ जयति॒ पञ्च॑ । \newline
25. ज॒य॒ति॒ पञ्च॒ पञ्च॑ जयति जयति॒ पञ्च॑ गृह्यन्ते गृह्यन्ते॒ पञ्च॑ जयति जयति॒ पञ्च॑ गृह्यन्ते । \newline
26. पञ्च॑ गृह्यन्ते गृह्यन्ते॒ पञ्च॒ पञ्च॑ गृह्यन्ते॒ पञ्च॒ पञ्च॑ गृह्यन्ते॒ पञ्च॒ पञ्च॑ गृह्यन्ते॒ पञ्च॑ । \newline
27. गृ॒ह्य॒न्ते॒ पञ्च॒ पञ्च॑ गृह्यन्ते गृह्यन्ते॒ पञ्च॒ दिशो॒ दिशः॒ पञ्च॑ गृह्यन्ते गृह्यन्ते॒ पञ्च॒ दिशः॑ । \newline
28. पञ्च॒ दिशो॒ दिशः॒ पञ्च॒ पञ्च॒ दिशः॒ सर्वा॑सु॒ सर्वा॑सु॒ दिशः॒ पञ्च॒ पञ्च॒ दिशः॒ सर्वा॑सु । \newline
29. दिशः॒ सर्वा॑सु॒ सर्वा॑सु॒ दिशो॒ दिशः॒ सर्वा᳚ स्वे॒वैव सर्वा॑सु॒ दिशो॒ दिशः॒ सर्वा᳚ स्वे॒व । \newline
30. सर्वा᳚ स्वे॒वैव सर्वा॑सु॒ सर्वा᳚ स्वे॒व दि॒क्षु दि॒क्ष्वे॑व सर्वा॑सु॒ सर्वा᳚ स्वे॒व दि॒क्षु । \newline
31. ए॒व दि॒क्षु दि॒क्ष्वे॑वैव दि॒क्ष्वृ॑द्ध्नुव न्त्यृद्ध्नुवन्ति दि॒क्ष्वे॑वैव दि॒क्ष्वृ॑द्ध्नुवन्ति । \newline
32. दि॒क्ष्वृ॑द्ध्नुव न्त्यृद्ध्नुवन्ति दि॒क्षु दि॒क्ष्वृ॑द्ध्नुवन्ति॒ नव॑नव॒ नव॑नव र्‌द्ध्नुवन्ति दि॒क्षु दि॒क्ष्वृ॑द्ध्नुवन्ति॒ नव॑नव । \newline
33. ऋ॒द्ध्नु॒व॒न्ति॒ नव॑नव॒ नव॑नव र्‌द्ध्नुव न्त्यृद्ध्नुवन्ति॒ नव॑नव गृह्यन्ते गृह्यन्ते॒ नव॑नव र्‌द्ध्नुव न्त्यृद्ध्नुवन्ति॒ नव॑नव गृह्यन्ते । \newline
34. नव॑नव गृह्यन्ते गृह्यन्ते॒ नव॑नव॒ नव॑नव गृह्यन्ते॒ नव॒ नव॑ गृह्यन्ते॒ नव॑नव॒ नव॑नव गृह्यन्ते॒ नव॑ । \newline
35. नव॑न॒वेति॒ नव॑ - न॒व॒ । \newline
36. गृ॒ह्य॒न्ते॒ नव॒ नव॑ गृह्यन्ते गृह्यन्ते॒ नव॒ वै वै नव॑ गृह्यन्ते गृह्यन्ते॒ नव॒ वै । \newline
37. नव॒ वै वै नव॒ नव॒ वै पुरु॑षे॒ पुरु॑षे॒ वै नव॒ नव॒ वै पुरु॑षे । \newline
38. वै पुरु॑षे॒ पुरु॑षे॒ वै वै पुरु॑षे प्रा॒णाः प्रा॒णाः पुरु॑षे॒ वै वै पुरु॑षे प्रा॒णाः । \newline
39. पुरु॑षे प्रा॒णाः प्रा॒णाः पुरु॑षे॒ पुरु॑षे प्रा॒णाः प्रा॒णान् प्रा॒णान् प्रा॒णाः पुरु॑षे॒ पुरु॑षे प्रा॒णाः प्रा॒णान् । \newline
40. प्रा॒णाः प्रा॒णान् प्रा॒णान् प्रा॒णाः प्रा॒णाः प्रा॒णा ने॒वैव प्रा॒णान् प्रा॒णाः प्रा॒णाः प्रा॒णा ने॒व । \newline
41. प्रा॒णा इति॑ प्र - अ॒नाः । \newline
42. प्रा॒णा ने॒वैव प्रा॒णान् प्रा॒णा ने॒व यज॑मानेषु॒ यज॑माने ष्वे॒व प्रा॒णान् प्रा॒णा ने॒व यज॑मानेषु । \newline
43. प्रा॒णानिति॑ प्र - अ॒नान् । \newline
44. ए॒व यज॑मानेषु॒ यज॑माने ष्वे॒वैव यज॑मानेषु दधति दधति॒ यज॑माने ष्वे॒वैव यज॑मानेषु दधति । \newline
45. यज॑मानेषु दधति दधति॒ यज॑मानेषु॒ यज॑मानेषु दधति प्राय॒णीये᳚ प्राय॒णीये॑ दधति॒ यज॑मानेषु॒ यज॑मानेषु दधति प्राय॒णीये᳚ । \newline
46. द॒ध॒ति॒ प्रा॒य॒णीये᳚ प्राय॒णीये॑ दधति दधति प्राय॒णीये॑ च च प्राय॒णीये॑ दधति दधति प्राय॒णीये॑ च । \newline
47. प्रा॒य॒णीये॑ च च प्राय॒णीये᳚ प्राय॒णीये॑ चोदय॒नीय॑ उदय॒नीये॑ च प्राय॒णीये᳚ प्राय॒णीये॑ चोदय॒नीये᳚ । \newline
48. प्रा॒य॒णीय॒ इति॑ प्र - अ॒य॒नीये᳚ । \newline
49. चो॒द॒य॒नीय॑ उदय॒नीये॑ च चोदय॒नीये॑ च चोदय॒नीये॑ च चोदय॒नीये॑ च । \newline
50. उ॒द॒य॒नीये॑ च चोदय॒नीय॑ उदय॒नीये॑ च गृह्यन्ते गृह्यन्ते चोदय॒नीय॑ उदय॒नीये॑ च गृह्यन्ते । \newline
51. उ॒द॒य॒नीय॒ इत्यु॑त् - अ॒य॒नीये᳚ । \newline
52. च॒ गृ॒ह्य॒न्ते॒ गृ॒ह्य॒न्ते॒ च॒ च॒ गृ॒ह्य॒न्ते॒ प्रा॒णाः प्रा॒णा गृ॑ह्यन्ते च च गृह्यन्ते प्रा॒णाः । \newline
53. गृ॒ह्य॒न्ते॒ प्रा॒णाः प्रा॒णा गृ॑ह्यन्ते गृह्यन्ते प्रा॒णा वै वै प्रा॒णा गृ॑ह्यन्ते गृह्यन्ते प्रा॒णा वै । \newline
54. प्रा॒णा वै वै प्रा॒णाः प्रा॒णा वै प्रा॑णग्र॒हाः प्रा॑णग्र॒हा वै प्रा॒णाः प्रा॒णा वै प्रा॑णग्र॒हाः । \newline
55. प्रा॒णा इति॑ प्र - अ॒नाः । \newline
56. वै प्रा॑णग्र॒हाः प्रा॑णग्र॒हा वै वै प्रा॑णग्र॒हाः प्रा॒णैः प्रा॒णैः प्रा॑णग्र॒हा वै वै प्रा॑णग्र॒हाः प्रा॒णैः । \newline
57. प्रा॒ण॒ग्र॒हाः प्रा॒णैः प्रा॒णैः प्रा॑णग्र॒हाः प्रा॑णग्र॒हाः प्रा॒णै रे॒वैव प्रा॒णैः प्रा॑णग्र॒हाः प्रा॑णग्र॒हाः प्रा॒णै रे॒व । \newline
58. प्रा॒ण॒ग्र॒हा इति॑ प्राण - ग्र॒हाः । \newline
\pagebreak
\markright{ TS 3.5.10.3  \hfill https://www.vedavms.in \hfill}

\section{ TS 3.5.10.3 }

\textbf{TS 3.5.10.3 } \newline
\textbf{Samhita Paata} \newline

प्रा॒णैरे॒व प्र॒यन्ति॑ प्रा॒णैरुद्य॑न्ति दश॒मेऽह॑न् गृह्यन्ते प्रा॒णा वै प्रा॑णग्र॒हाः प्रा॒णेभ्यः॒ खलु॒ वा ए॒तत् प्र॒जा य॑न्ति॒ यद्वा॑मदे॒व्यं ॅयोने॒श्च्यव॑ते दश॒मेऽह॑न्. वामदे॒व्यं ॅयोने᳚श्च्यवते॒ यद्-द॑श॒मेऽह॑न् गृ॒ह्यन्ते᳚ प्रा॒णेभ्य॑ ए॒व तत् प्र॒जा नय॑न्ति ॥ \newline

\textbf{Pada Paata} \newline

प्रा॒णैरिति॑ प्र - अ॒नैः । ए॒व । प्र॒यन्तीति॑ प्र - यन्ति॑ । प्रा॒णैरिति॑ प्र - अ॒नैः । उदिति॑ । य॒न्ति॒ । द॒श॒मे । अहन्न्॑ । गृ॒ह्य॒न्ते॒ । प्रा॒णा इति॑ प्र - अ॒नाः । वै । प्रा॒ण॒ग्र॒हा इति॑ प्राण - ग्र॒हाः । प्रा॒णेभ्य॒ इति॑ प्र - अ॒नेभ्यः॑ । खलु॑ । वै । ए॒तत् । प्र॒जा इति॑ प्र - जाः । य॒न्ति॒ । यत् । वा॒म॒दे॒व्यमिति॑ वाम - दे॒व्यम् । योनेः᳚ । च्यव॑ते । द॒श॒मे । अहन्न्॑ । वा॒म॒दे॒व्यमिति॑ वाम - दे॒व्यम् । योनेः᳚ । च्य॒व॒ते॒ । यत् । द॒श॒मे । अहन्न्॑ । गृ॒ह्यन्ते᳚ । प्रा॒णेभ्य॒ इति॑ प्र - अ॒नेभ्यः॑ । ए॒व । तत् । प्र॒जा इति॑ प्र - जाः । न । य॒न्ति॒ ॥  \newline


\textbf{Krama Paata} \newline

प्रा॒णैरे॒व । प्रा॒णैरिति॑ प्र - अ॒नैः । ए॒व प्र॒यन्ति॑ । प्र॒यन्ति॑ प्रा॒णैः । प्र॒यन्तीति॑ प्र - यन्ति॑ । प्रा॒णैरुत् । प्रा॒णैरिति॑ प्र - अ॒नैः । उद् य॑न्ति । य॒न्ति॒ द॒श॒मे । द॒श॒मे ऽहन्न्॑ । अह॑न् गृह्यन्ते । गृ॒ह्य॒न्ते॒ प्रा॒णाः । प्रा॒णा वै । प्रा॒णा इति॑ प्र - अ॒नाः । वै प्रा॑णग्र॒हाः । प्रा॒ण॒ग्र॒हाः प्रा॒णेभ्यः॑ । प्रा॒ण॒ग्र॒हा इति॑ प्राण - ग्र॒हाः । प्रा॒णेभ्यः॒ खलु॑ । प्रा॒णेभ्य॒ इति॑ प्र - अ॒नेभ्यः॑ । खलु॒ वै । वा ए॒तत् । ए॒तत् प्र॒जाः । प्र॒जा य॑न्ति । प्र॒जा इति॑ प्र - जाः । य॒न्ति॒ यत् । यद् वा॑मदे॒व्यम् । वा॒म॒दे॒व्यं ॅयोनेः᳚ । वा॒म॒दे॒व्यमिति॑ वाम - दे॒व्यम् । योने॒श्च्यव॑ते । च्यव॑ते दश॒मे । द॒श॒मे ऽहन्न्॑ । अह॑न् वामदे॒व्यम् । वा॒म॒दे॒व्यं ॅयोनेः᳚ । वा॒म॒दे॒व्यमिति॑ वाम - दे॒व्यम् । योने᳚श्च्यवते । च्य॒व॒ते॒ यत् । यद् द॑श॒मे । द॒श॒मे ऽहन्न्॑ । अह॑न् गृ॒ह्यन्ते᳚ । गृ॒ह्यन्ते᳚ प्रा॒णेभ्यः॑ । प्रा॒णेभ्य॑ ए॒व । प्रा॒णेभ्य॒ इति॑ प्र - अ॒नेभ्यः॑ । ए॒व तत् । तत् प्र॒जाः । प्र॒जा न । प्र॒जा इति॑ प्र - जाः । न य॑न्ति । य॒न्तीति॑ यन्ति । \newline

\textbf{Jatai Paata} \newline

1. प्रा॒णै रे॒वैव प्रा॒णैः प्रा॒णै रे॒व । \newline
2. प्रा॒णैरिति॑ प्र - अ॒नैः । \newline
3. ए॒व प्र॒यन्ति॑ प्र॒य न्त्ये॒वैव प्र॒यन्ति॑ । \newline
4. प्र॒यन्ति॑ प्रा॒णैः प्रा॒णैः प्र॒यन्ति॑ प्र॒यन्ति॑ प्रा॒णैः । \newline
5. प्र॒यन्तीति॑ प्र - यन्ति॑ । \newline
6. प्रा॒णै रुदुत् प्रा॒णैः प्रा॒णै रुत् । \newline
7. प्रा॒णैरिति॑ प्र - अ॒नैः । \newline
8. उद् य॑न्ति य॒ न्त्युदुद् य॑न्ति । \newline
9. य॒न्ति॒ द॒श॒मे द॑श॒मे य॑न्ति यन्ति दश॒मे । \newline
10. द॒श॒मे ऽह॒न् नह॑न् दश॒मे द॑श॒मे ऽहन्न्॑ । \newline
11. अह॑न् गृह्यन्ते गृह्य॒न्ते ऽह॒न् नह॑न् गृह्यन्ते । \newline
12. गृ॒ह्य॒न्ते॒ प्रा॒णाः प्रा॒णा गृ॑ह्यन्ते गृह्यन्ते प्रा॒णाः । \newline
13. प्रा॒णा वै वै प्रा॒णाः प्रा॒णा वै । \newline
14. प्रा॒णा इति॑ प्र - अ॒नाः । \newline
15. वै प्रा॑णग्र॒हाः प्रा॑णग्र॒हा वै वै प्रा॑णग्र॒हाः । \newline
16. प्रा॒ण॒ग्र॒हाः प्रा॒णेभ्यः॑ प्रा॒णेभ्यः॑ प्राणग्र॒हाः प्रा॑णग्र॒हाः प्रा॒णेभ्यः॑ । \newline
17. प्रा॒ण॒ग्र॒हा इति॑ प्राण - ग्र॒हाः । \newline
18. प्रा॒णेभ्यः॒ खलु॒ खलु॑ प्रा॒णेभ्यः॑ प्रा॒णेभ्यः॒ खलु॑ । \newline
19. प्रा॒णेभ्य॒ इति॑ प्र - अ॒नेभ्यः॑ । \newline
20. खलु॒ वै वै खलु॒ खलु॒ वै । \newline
21. वा ए॒त दे॒तद् वै वा ए॒तत् । \newline
22. ए॒तत् प्र॒जाः प्र॒जा ए॒त दे॒तत् प्र॒जाः । \newline
23. प्र॒जा य॑न्ति यन्ति प्र॒जाः प्र॒जा य॑न्ति । \newline
24. प्र॒जा इति॑ प्र - जाः । \newline
25. य॒न्ति॒ यद् यद् य॑न्ति यन्ति॒ यत् । \newline
26. यद् वा॑मदे॒व्यं ॅवा॑मदे॒व्यं ॅयद् यद् वा॑मदे॒व्यम् । \newline
27. वा॒म॒दे॒व्यं ॅयोने॒र् योने᳚र् वामदे॒व्यं ॅवा॑मदे॒व्यं ॅयोनेः᳚ । \newline
28. वा॒म॒दे॒व्यमिति॑ वाम - दे॒व्यम् । \newline
29. योने॒ श्च्यव॑ते॒ च्यव॑ते॒ योने॒र् योने॒ श्च्यव॑ते । \newline
30. च्यव॑ते दश॒मे द॑श॒मे च्यव॑ते॒ च्यव॑ते दश॒मे । \newline
31. द॒श॒मे ऽह॒न् नह॑न् दश॒मे द॑श॒मे ऽहन्न्॑ । \newline
32. अह॑न्. वामदे॒व्यं ॅवा॑मदे॒व्य मह॒न् नह॑न्. वामदे॒व्यम् । \newline
33. वा॒म॒दे॒व्यं ॅयोने॒र् योने᳚र् वामदे॒व्यं ॅवा॑मदे॒व्यं ॅयोनेः᳚ । \newline
34. वा॒म॒दे॒व्यमिति॑ वाम - दे॒व्यम् । \newline
35. योने᳚ श्च्यवते च्यवते॒ योने॒र् योने᳚ श्च्यवते । \newline
36. च्य॒व॒ते॒ यद् यच् च्य॑वते च्यवते॒ यत् । \newline
37. यद् द॑श॒मे द॑श॒मे यद् यद् द॑श॒मे । \newline
38. द॒श॒मे ऽह॒न् नह॑न् दश॒मे द॑श॒मे ऽहन्न्॑ । \newline
39. अह॑न् गृ॒ह्यन्ते॑ गृ॒ह्यन्ते ऽह॒न् नह॑न् गृ॒ह्यन्ते᳚ । \newline
40. गृ॒ह्यन्ते᳚ प्रा॒णेभ्यः॑ प्रा॒णेभ्यो॑ गृ॒ह्यन्ते॑ गृ॒ह्यन्ते᳚ प्रा॒णेभ्यः॑ । \newline
41. प्रा॒णेभ्य॑ ए॒वैव प्रा॒णेभ्यः॑ प्रा॒णेभ्य॑ ए॒व । \newline
42. प्रा॒णेभ्य॒ इति॑ प्र - अ॒नेभ्यः॑ । \newline
43. ए॒व तत् तदे॒वैव तत् । \newline
44. तत् प्र॒जाः प्र॒जा स्तत् तत् प्र॒जाः । \newline
45. प्र॒जा न न प्र॒जाः प्र॒जा न । \newline
46. प्र॒जा इति॑ प्र - जाः । \newline
47. न य॑न्ति यन्ति॒ न न य॑न्ति । \newline
48. य॒न्तीति॑ यन्ति । \newline

\textbf{Ghana Paata } \newline

1. प्रा॒णै रे॒वैव प्रा॒णैः प्रा॒णै रे॒व प्र॒यन्ति॑ प्र॒य न्त्ये॒व प्रा॒णैः प्रा॒णै रे॒व प्र॒यन्ति॑ । \newline
2. प्रा॒णैरिति॑ प्र - अ॒नैः । \newline
3. ए॒व प्र॒यन्ति॑ प्र॒य न्त्ये॒वैव प्र॒यन्ति॑ प्रा॒णैः प्रा॒णैः प्र॒य न्त्ये॒वैव प्र॒यन्ति॑ प्रा॒णैः । \newline
4. प्र॒यन्ति॑ प्रा॒णैः प्रा॒णैः प्र॒यन्ति॑ प्र॒यन्ति॑ प्रा॒णै रुदुत् प्रा॒णैः प्र॒यन्ति॑ प्र॒यन्ति॑ प्रा॒णै रुत् । \newline
5. प्र॒यन्तीति॑ प्र - यन्ति॑ । \newline
6. प्रा॒णै रुदुत् प्रा॒णैः प्रा॒णै रुद् य॑न्ति य॒न्त्युत् प्रा॒णैः प्रा॒णै रुद् य॑न्ति । \newline
7. प्रा॒णैरिति॑ प्र - अ॒नैः । \newline
8. उद् य॑न्ति य॒न्त्युदुद् य॑न्ति दश॒मे द॑श॒मे य॒न्त्युदुद् य॑न्ति दश॒मे । \newline
9. य॒न्ति॒ द॒श॒मे द॑श॒मे य॑न्ति यन्ति दश॒मे ऽह॒न् नह॑न् दश॒मे य॑न्ति यन्ति दश॒मे ऽहन्न्॑ । \newline
10. द॒श॒मे ऽह॒न् नह॑न् दश॒मे द॑श॒मे ऽह॑न् गृह्यन्ते गृह्य॒न्ते ऽह॑न् दश॒मे द॑श॒मे ऽह॑न् गृह्यन्ते । \newline
11. अह॑न् गृह्यन्ते गृह्य॒न्ते ऽह॒न् नह॑न् गृह्यन्ते प्रा॒णाः प्रा॒णा गृ॑ह्य॒न्ते ऽह॒न् नह॑न् गृह्यन्ते प्रा॒णाः । \newline
12. गृ॒ह्य॒न्ते॒ प्रा॒णाः प्रा॒णा गृ॑ह्यन्ते गृह्यन्ते प्रा॒णा वै वै प्रा॒णा गृ॑ह्यन्ते गृह्यन्ते प्रा॒णा वै । \newline
13. प्रा॒णा वै वै प्रा॒णाः प्रा॒णा वै प्रा॑णग्र॒हाः प्रा॑णग्र॒हा वै प्रा॒णाः प्रा॒णा वै प्रा॑णग्र॒हाः । \newline
14. प्रा॒णा इति॑ प्र - अ॒नाः । \newline
15. वै प्रा॑णग्र॒हाः प्रा॑णग्र॒हा वै वै प्रा॑णग्र॒हाः प्रा॒णेभ्यः॑ प्रा॒णेभ्यः॑ प्राणग्र॒हा वै वै प्रा॑णग्र॒हाः प्रा॒णेभ्यः॑ । \newline
16. प्रा॒ण॒ग्र॒हाः प्रा॒णेभ्यः॑ प्रा॒णेभ्यः॑ प्राणग्र॒हाः प्रा॑णग्र॒हाः प्रा॒णेभ्यः॒ खलु॒ खलु॑ प्रा॒णेभ्यः॑ प्राणग्र॒हाः प्रा॑णग्र॒हाः प्रा॒णेभ्यः॒ खलु॑ । \newline
17. प्रा॒ण॒ग्र॒हा इति॑ प्राण - ग्र॒हाः । \newline
18. प्रा॒णेभ्यः॒ खलु॒ खलु॑ प्रा॒णेभ्यः॑ प्रा॒णेभ्यः॒ खलु॒ वै वै खलु॑ प्रा॒णेभ्यः॑ प्रा॒णेभ्यः॒ खलु॒ वै । \newline
19. प्रा॒णेभ्य॒ इति॑ प्र - अ॒नेभ्यः॑ । \newline
20. खलु॒ वै वै खलु॒ खलु॒ वा ए॒त दे॒तद् वै खलु॒ खलु॒ वा ए॒तत् । \newline
21. वा ए॒त दे॒तद् वै वा ए॒तत् प्र॒जाः प्र॒जा ए॒तद् वै वा ए॒तत् प्र॒जाः । \newline
22. ए॒तत् प्र॒जाः प्र॒जा ए॒त दे॒तत् प्र॒जा य॑न्ति यन्ति प्र॒जा ए॒त दे॒तत् प्र॒जा य॑न्ति । \newline
23. प्र॒जा य॑न्ति यन्ति प्र॒जाः प्र॒जा य॑न्ति॒ यद् यद् य॑न्ति प्र॒जाः प्र॒जा य॑न्ति॒ यत् । \newline
24. प्र॒जा इति॑ प्र - जाः । \newline
25. य॒न्ति॒ यद् यद् य॑न्ति यन्ति॒ यद् वा॑मदे॒व्यं ॅवा॑मदे॒व्यं ॅयद् य॑न्ति यन्ति॒ यद् वा॑मदे॒व्यम् । \newline
26. यद् वा॑मदे॒व्यं ॅवा॑मदे॒व्यं ॅयद् यद् वा॑मदे॒व्यं ॅयोने॒र् योने᳚र् वामदे॒व्यं ॅयद् यद् वा॑मदे॒व्यं ॅयोनेः᳚ । \newline
27. वा॒म॒दे॒व्यं ॅयोने॒र् योने᳚र् वामदे॒व्यं ॅवा॑मदे॒व्यं ॅयोने॒ श्च्यव॑ते॒ च्यव॑ते॒ योने᳚र् वामदे॒व्यं ॅवा॑मदे॒व्यं ॅयोने॒ श्च्यव॑ते । \newline
28. वा॒म॒दे॒व्यमिति॑ वाम - दे॒व्यम् । \newline
29. योने॒ श्च्यव॑ते॒ च्यव॑ते॒ योने॒र् योने॒ श्च्यव॑ते दश॒मे द॑श॒मे च्यव॑ते॒ योने॒र् योने॒ श्च्यव॑ते दश॒मे । \newline
30. च्यव॑ते दश॒मे द॑श॒मे च्यव॑ते॒ च्यव॑ते दश॒मे ऽह॒न् नह॑न् दश॒मे च्यव॑ते॒ च्यव॑ते दश॒मे ऽहन्न्॑ । \newline
31. द॒श॒मे ऽह॒न् नह॑न् दश॒मे द॑श॒मे ऽह॑न्. वामदे॒व्यं ॅवा॑मदे॒व्य मह॑न् दश॒मे द॑श॒मे ऽह॑न्. वामदे॒व्यम् । \newline
32. अह॑न्. वामदे॒व्यं ॅवा॑मदे॒व्य मह॒न् नह॑न्. वामदे॒व्यं ॅयोने॒र् योने᳚र् वामदे॒व्य मह॒न् नह॑न्. वामदे॒व्यं ॅयोनेः᳚ । \newline
33. वा॒म॒दे॒व्यं ॅयोने॒र् योने᳚र् वामदे॒व्यं ॅवा॑मदे॒व्यं ॅयोने᳚ श्च्यवते च्यवते॒ योने᳚र् वामदे॒व्यं ॅवा॑मदे॒व्यं ॅयोने᳚ श्च्यवते । \newline
34. वा॒म॒दे॒व्यमिति॑ वाम - दे॒व्यम् । \newline
35. योने᳚ श्च्यवते च्यवते॒ योने॒र् योने᳚ श्च्यवते॒ यद् यच् च्य॑वते॒ योने॒र् योने᳚ श्च्यवते॒ यत् । \newline
36. च्य॒व॒ते॒ यद् यच् च्य॑वते च्यवते॒ यद् द॑श॒मे द॑श॒मे यच् च्य॑वते च्यवते॒ यद् द॑श॒मे । \newline
37. यद् द॑श॒मे द॑श॒मे यद् यद् द॑श॒मे ऽह॒न् नह॑न् दश॒मे यद् यद् द॑श॒मे ऽहन्न्॑ । \newline
38. द॒श॒मे ऽह॒न् नह॑न् दश॒मे द॑श॒मे ऽह॑न् गृ॒ह्यन्ते॑ गृ॒ह्यन्ते ऽह॑न् दश॒मे द॑श॒मे ऽह॑न् गृ॒ह्यन्ते᳚ । \newline
39. अह॑न् गृ॒ह्यन्ते॑ गृ॒ह्यन्ते ऽह॒न् नह॑न् गृ॒ह्यन्ते᳚ प्रा॒णेभ्यः॑ प्रा॒णेभ्यो॑ गृ॒ह्यन्ते ऽह॒न् नह॑न् गृ॒ह्यन्ते᳚ प्रा॒णेभ्यः॑ । \newline
40. गृ॒ह्यन्ते᳚ प्रा॒णेभ्यः॑ प्रा॒णेभ्यो॑ गृ॒ह्यन्ते॑ गृ॒ह्यन्ते᳚ प्रा॒णेभ्य॑ ए॒वैव प्रा॒णेभ्यो॑ गृ॒ह्यन्ते॑ गृ॒ह्यन्ते᳚ प्रा॒णेभ्य॑ ए॒व । \newline
41. प्रा॒णेभ्य॑ ए॒वैव प्रा॒णेभ्यः॑ प्रा॒णेभ्य॑ ए॒व तत् तदे॒व प्रा॒णेभ्यः॑ प्रा॒णेभ्य॑ ए॒व तत् । \newline
42. प्रा॒णेभ्य॒ इति॑ प्र - अ॒नेभ्यः॑ । \newline
43. ए॒व तत् तदे॒वैव तत् प्र॒जाः प्र॒जा स्तदे॒वैव तत् प्र॒जाः । \newline
44. तत् प्र॒जाः प्र॒जा स्तत् तत् प्र॒जा न न प्र॒जा स्तत् तत् प्र॒जा न । \newline
45. प्र॒जा न न प्र॒जाः प्र॒जा न य॑न्ति यन्ति॒ न प्र॒जाः प्र॒जा न य॑न्ति । \newline
46. प्र॒जा इति॑ प्र - जाः । \newline
47. न य॑न्ति यन्ति॒ न न य॑न्ति । \newline
48. य॒न्तीति॑ यन्ति । \newline
\pagebreak
\markright{ TS 3.5.11.1  \hfill https://www.vedavms.in \hfill}

\section{ TS 3.5.11.1 }

\textbf{TS 3.5.11.1 } \newline
\textbf{Samhita Paata} \newline

प्र दे॒वन्दे॒व्या धि॒या भर॑ता जा॒तवे॑दसं । ह॒व्या नो॑ वक्षदानु॒षक् ॥ अ॒यमु॒ ष्य प्रदे॑व॒युर्.होता॑ य॒ज्ञाय॑ नीयते । रथो॒ न योर॒भीवृ॑तो॒ घृणी॑वान् चेतति॒ त्मना᳚ ॥ अ॒यम॒ग्निरु॑रुष्यत्य॒मृता॑दिव॒ जन्म॑नः । सह॑सश्चि॒थ् सही॑यान् दे॒वो जी॒वात॑वे कृ॒तः ॥ इडा॑यास्त्वा प॒दे व॒यं नाभा॑ पृथि॒व्या अधि॑ । जात॑वेदो॒ नि धी॑म॒ह्यग्ने॑ ह॒व्याय॒ वोढ॑वे ॥ \newline

\textbf{Pada Paata} \newline

प्रेति॑ । दे॒वम् । दे॒व्या । धि॒या । भर॑त । जा॒तवे॑दस॒मिति॑ जा॒त - वे॒द॒स॒म् ॥ ह॒व्या । नः॒ । व॒क्ष॒त् । आ॒नु॒षक् ॥ अ॒यम् । उ॒ । स्यः । प्रेति॑ । दे॒व॒युरिति॑ देव-युः । होता᳚ । य॒ज्ञाय॑ । नी॒य॒ते॒ ॥ रथः॑ । न । योः । अ॒भीवृ॑त॒ इत्य॒भि - वृ॒तः॒ । घृणी॑वान् । चे॒त॒ति॒ । त्मना᳚ ॥ अ॒यम् । अ॒ग्निः । उ॒रु॒ष्य॒ति॒ । अ॒मृता᳚त् । इ॒व॒ । जन्म॑नः ॥ सह॑सः । चि॒त् । सही॑यान् । दे॒वः । जी॒वात॑वे । कृ॒तः ॥ इडा॑याः । त्वा॒ । प॒दे । व॒यम् । नाभा᳚ । पृ॒थि॒व्याः । अधि॑ ॥ जात॑वेद॒ इति॒ जात॑ - वे॒दः॒ । नीति॑ । धी॒म॒हि॒ । अग्ने᳚ । ह॒व्याय॑ । वोढ॑वे ॥  \newline


\textbf{Krama Paata} \newline

प्र दे॒वम् । दे॒वम् दे॒व्या । दे॒व्या धि॒या । धि॒या भर॑त । भर॑ता जा॒तवे॑दसम् । जा॒तवे॑दस॒मिति॑ जा॒त - वे॒द॒स॒म् ॥ ह॒व्या नः॑ । नो॒ व॒क्ष॒त्॒ । व॒क्ष॒दा॒नु॒षक् । आ॒नु॒षगित्या॑नु॒षक् ॥ अ॒यमु॑ । उ॒ ष्यः । स्य प्र । प्र दे॑व॒युः । दे॒व॒युर्. होता᳚ । दे॒व॒युरिति॑ देव - युः । होता॑ य॒ज्ञाय॑ । य॒ज्ञाय॑ नीयते । नी॒य॒त॒ इति॑ नीयते ॥ रथो॒ न । न योः । योर॒भीवृ॑तः । अ॒भीवृ॑तो॒ घृणी॑वान् । अ॒भीवृ॑त॒ इत्य॒भि - वृ॒तः॒ । घृणी॑वान् चेतति । चे॒त॒ति॒ त्मना᳚ । त्मनेति॒ त्मना᳚ ॥ अ॒यम॒ग्निः । अ॒ग्निरु॑रुष्यति । उ॒रु॒ष्य॒त्य॒मृता᳚त् । अ॒मृता॑दिव । इ॒व॒ जन्म॑नः । जन्म॑न॒ इति॒ जन्म॑नः ॥ सह॑सश्चित् । चि॒थ् सही॑यान् । सही॑यान् दे॒वः । दे॒वो जी॒वात॑वे । जी॒वात॑वे कृ॒तः । कृ॒त इति॑ कृ॒तः ॥ इडा॑यास्त्वा । त्वा॒ प॒दे । प॒दे व॒यम् । व॒यम् नाभा᳚ । नाभा॑ पृथि॒व्याः । पृ॒थि॒व्या अधि॑ । अधीत्यधि॑ ॥ जात॑वेदो॒ नि । जात॑वेद॒ इति॒ जात॑ - वे॒दः॒ । नि धी॑महि । धी॒म॒ह्यग्ने᳚ । अग्ने॑ ह॒व्याय॑ । ह॒व्याय॒ वोढ॑वे । वोढ॑व॒ इति॒ वोढ॑वे । \newline

\textbf{Jatai Paata} \newline

1. प्र दे॒वम् दे॒वम् प्र प्र दे॒वम् । \newline
2. दे॒वम् दे॒व्या दे॒व्या दे॒वम् दे॒वम् दे॒व्या । \newline
3. दे॒व्या धि॒या धि॒या दे॒व्या दे॒व्या धि॒या । \newline
4. धि॒या भर॑त॒ भर॑त धि॒या धि॒या भर॑त । \newline
5. भर॑ता जा॒तवे॑दसम् जा॒तवे॑दस॒म् भर॑त॒ भर॑ता जा॒तवे॑दसम् । \newline
6. जा॒तवे॑दस॒मिति॑ जा॒त - वे॒द॒स॒म् । \newline
7. ह॒व्या नो॑ नो ह॒व्या ह॒व्या नः॑ । \newline
8. नो॒ व॒क्ष॒द् व॒क्ष॒न् नो॒ नो॒ व॒क्ष॒त् । \newline
9. व॒क्ष॒ दा॒नु॒ष गा॑नु॒षग् व॑क्षद् वक्ष दानु॒षक् । \newline
10. आ॒नु॒षगित्या॑नु॒षक् । \newline
11. अ॒य मु॑ वु व॒य म॒य मु॑ । \newline
12. उ॒ ष्य स्य उ॑ वु॒ ष्यः । \newline
13. स्य प्र प्र स्य स्य प्र । \newline
14. प्र दे॑व॒युर् दे॑व॒युः प्र प्र दे॑व॒युः । \newline
15. दे॒व॒युर्. होता॒ होता॑ देव॒युर् दे॑व॒युर्. होता᳚ । \newline
16. दे॒व॒युरिति॑ देव - युः । \newline
17. होता॑ य॒ज्ञाय॑ य॒ज्ञाय॒ होता॒ होता॑ य॒ज्ञाय॑ । \newline
18. य॒ज्ञाय॑ नीयते नीयते य॒ज्ञाय॑ य॒ज्ञाय॑ नीयते । \newline
19. नी॒य॒त॒ इति॑ नीयते । \newline
20. रथो॒ न न रथो॒ रथो॒ न । \newline
21. न योर् योर् न न योः । \newline
22. यो र॒भीवृ॑तो अ॒भीवृ॑तो॒ योर् यो र॒भीवृ॑तः । \newline
23. अ॒भीवृ॑तो॒ घृणी॑वा॒न् घृणी॑वा न॒भीवृ॑तो अ॒भीवृ॑तो॒ घृणी॑वान् । \newline
24. अ॒भीवृ॑त॒ इत्य॒भि - वृ॒तः॒ । \newline
25. घृणी॑वान् चेतति चेतति॒ घृणी॑वा॒न् घृणी॑वान् चेतति । \newline
26. चे॒त॒ति॒ त्मना॒ त्मना॑ चेतति चेतति॒ त्मना᳚ । \newline
27. त्मनेति॒ त्मना᳚ । \newline
28. अ॒य म॒ग्नि र॒ग्नि र॒य म॒य म॒ग्निः । \newline
29. अ॒ग्नि रु॑रुष्य त्युरुष्य त्य॒ग्नि र॒ग्नि रु॑रुष्यति । \newline
30. उ॒रु॒ष्य॒ त्य॒मृता॑ द॒मृता॑ दुरुष्य त्युरुष्य त्य॒मृता᳚त् । \newline
31. अ॒मृता॑ दिवे वा॒मृता॑ द॒मृता॑ दिव । \newline
32. इ॒व॒ जन्म॑नो॒ जन्म॑न इवे व॒ जन्म॑नः । \newline
33. जन्म॑न॒ इति॒ जन्म॑नः । \newline
34. सह॑स श्चिच् चि॒थ् सह॑सः॒ सह॑स श्चित् । \newline
35. चि॒थ् सही॑या॒न् थ्सही॑याꣳ श्चिच् चि॒थ् सही॑यान् । \newline
36. सही॑यान् दे॒वो दे॒वः सही॑या॒न् थ्सही॑यान् दे॒वः । \newline
37. दे॒वो जी॒वात॑वे जी॒वात॑वे दे॒वो दे॒वो जी॒वात॑वे । \newline
38. जी॒वात॑वे कृ॒तः कृ॒तो जी॒वात॑वे जी॒वात॑वे कृ॒तः । \newline
39. कृ॒त इति॑ कृ॒तः । \newline
40. इडा॑या स्त्वा॒ त्वेडा॑या॒ इडा॑या स्त्वा । \newline
41. त्वा॒ प॒दे प॒दे त्वा᳚ त्वा प॒दे । \newline
42. प॒दे व॒यं ॅव॒यम् प॒दे प॒दे व॒यम् । \newline
43. व॒यम् नाभा॒ नाभा॑ व॒यं ॅव॒यम् नाभा᳚ । \newline
44. नाभा॑ पृथि॒व्याः पृ॑थि॒व्या नाभा॒ नाभा॑ पृथि॒व्याः । \newline
45. पृ॒थि॒व्या अध्यधि॑ पृथि॒व्याः पृ॑थि॒व्या अधि॑ । \newline
46. अधीत्यधि॑ । \newline
47. जात॑वेदो॒ नि नि जात॑वेदो॒ जात॑वेदो॒ नि । \newline
48. जात॑वेद॒ इति॒ जात॑ - वे॒दः॒ । \newline
49. नि धी॑महि धीमहि॒ नि नि धी॑महि । \newline
50. धी॒म॒ह्यग्ने ऽग्ने॑ धीमहि धीम॒ह्यग्ने᳚ । \newline
51. अग्ने॑ ह॒व्याय॑ ह॒व्यायाग्ने ऽग्ने॑ ह॒व्याय॑ । \newline
52. ह॒व्याय॒ वोढ॑वे॒ वोढ॑वे ह॒व्याय॑ ह॒व्याय॒ वोढ॑वे । \newline
53. वोढ॑व॒ इति॒ वोढ॑वे । \newline

\textbf{Ghana Paata } \newline

1. प्र दे॒वम् दे॒वम् प्र प्र दे॒वम् दे॒व्या दे॒व्या दे॒वम् प्र प्र दे॒वम् दे॒व्या । \newline
2. दे॒वम् दे॒व्या दे॒व्या दे॒वम् दे॒वम् दे॒व्या धि॒या धि॒या दे॒व्या दे॒वम् दे॒वम् दे॒व्या धि॒या । \newline
3. दे॒व्या धि॒या धि॒या दे॒व्या दे॒व्या धि॒या भर॑त॒ भर॑त धि॒या दे॒व्या दे॒व्या धि॒या भर॑त । \newline
4. धि॒या भर॑त॒ भर॑त धि॒या धि॒या भर॑ता जा॒तवे॑दसम् जा॒तवे॑दस॒म् भर॑त धि॒या धि॒या भर॑ता जा॒तवे॑दसम् । \newline
5. भर॑ता जा॒तवे॑दसम् जा॒तवे॑दस॒म् भर॑त॒ भर॑ता जा॒तवे॑दसम् । \newline
6. जा॒तवे॑दस॒मिति॑ जा॒त - वे॒द॒स॒म् । \newline
7. ह॒व्या नो॑ नो ह॒व्या ह॒व्या नो॑ वक्षद् वक्षन् नो ह॒व्या ह॒व्या नो॑ वक्षत् । \newline
8. नो॒ व॒क्ष॒द् व॒क्ष॒न् नो॒ नो॒ व॒क्ष॒ दा॒नु॒ष गा॑नु॒षग् व॑क्षन् नो नो वक्ष दानु॒षक् । \newline
9. व॒क्ष॒ दा॒नु॒ष गा॑नु॒षग् व॑क्षद् वक्ष दानु॒षक् । \newline
10. आ॒नु॒षगित्या॑नु॒षक् । \newline
11. अ॒य मु॑ वु व॒य म॒य मु॒ ष्य स्य उ॑ व॒य म॒य मु॒ ष्यः । \newline
12. उ॒ ष्य स्य उ॑ वु॒ ष्य प्र प्र स्य उ॑ वु॒ ष्य प्र । \newline
13. स्य प्र प्र स्य स्य प्र दे॑व॒युर् दे॑व॒युः प्र स्य स्य प्र दे॑व॒युः । \newline
14. प्र दे॑व॒युर् दे॑व॒युः प्र प्र दे॑व॒युर्. होता॒ होता॑ देव॒युः प्र प्र दे॑व॒युर्. होता᳚ । \newline
15. दे॒व॒युर्. होता॒ होता॑ देव॒युर् दे॑व॒युर्. होता॑ य॒ज्ञाय॑ य॒ज्ञाय॒ होता॑ देव॒युर् दे॑व॒युर्. होता॑ य॒ज्ञाय॑ । \newline
16. दे॒व॒युरिति॑ देव - युः । \newline
17. होता॑ य॒ज्ञाय॑ य॒ज्ञाय॒ होता॒ होता॑ य॒ज्ञाय॑ नीयते नीयते य॒ज्ञाय॒ होता॒ होता॑ य॒ज्ञाय॑ नीयते । \newline
18. य॒ज्ञाय॑ नीयते नीयते य॒ज्ञाय॑ य॒ज्ञाय॑ नीयते । \newline
19. नी॒य॒त॒ इति॑ नीयते । \newline
20. रथो॒ न न रथो॒ रथो॒ न योर् योर् न रथो॒ रथो॒ न योः । \newline
21. न योर् योर् न न यो र॒भीवृ॑तो अ॒भीवृ॑तो॒ योर् न न यो र॒भीवृ॑तः । \newline
22. यो र॒भीवृ॑तो अ॒भीवृ॑तो॒ योर् यो र॒भीवृ॑तो॒ घृणी॑वा॒न् घृणी॑वा न॒भीवृ॑तो॒ योर् यो र॒भीवृ॑तो॒ घृणी॑वान् । \newline
23. अ॒भीवृ॑तो॒ घृणी॑वा॒न् घृणी॑वा न॒भीवृ॑तो अ॒भीवृ॑तो॒ घृणी॑वान् चेतति चेतति॒ घृणी॑वा न॒भीवृ॑तो अ॒भीवृ॑तो॒ घृणी॑वान् चेतति । \newline
24. अ॒भीवृ॑त॒ इत्य॒भि - वृ॒तः॒ । \newline
25. घृणी॑वान् चेतति चेतति॒ घृणी॑वा॒न् घृणी॑वान् चेतति॒ त्मना॒ त्मना॑ चेतति॒ घृणी॑वा॒न् घृणी॑वान् चेतति॒ त्मना᳚ । \newline
26. चे॒त॒ति॒ त्मना॒ त्मना॑ चेतति चेतति॒ त्मना᳚ । \newline
27. त्मनेति॒ त्मना᳚ । \newline
28. अ॒य म॒ग्नि र॒ग्नि र॒य म॒य म॒ग्नि रु॑रुष्य त्युरुष्य त्य॒ग्नि र॒य म॒य म॒ग्नि रु॑रुष्यति । \newline
29. अ॒ग्निरु॑ रुष्य त्युरुष्य त्य॒ग्नि र॒ग्नि रु॑रुष्य त्य॒मृता॑ द॒मृता॑ दुरुष्य त्य॒ग्नि र॒ग्नि रु॑रुष्य त्य॒मृता᳚त् । \newline
30. उ॒रु॒ष्य॒ त्य॒मृता॑ द॒मृता॑ दुरुष्य त्युरुष्य त्य॒मृता॑ दिवे वा॒मृता॑ दुरुष्य त्युरुष्य त्य॒मृता॑ दिव । \newline
31. अ॒मृता॑ दिवे वा॒मृता॑ द॒मृता॑ दिव॒ जन्म॑नो॒ जन्म॑न इवा॒मृता॑ द॒मृता॑ दिव॒ जन्म॑नः । \newline
32. इ॒व॒ जन्म॑नो॒ जन्म॑न इवेव॒ जन्म॑नः । \newline
33. जन्म॑न॒ इति॒ जन्म॑नः । \newline
34. सह॑स श्चिच् चि॒थ् सह॑सः॒ सह॑स श्चि॒थ् सही॑या॒न् थ्सही॑याꣳ श्चि॒थ् सह॑सः॒ सह॑स श्चि॒थ् सही॑यान् । \newline
35. चि॒थ् सही॑या॒न् थ्सही॑याꣳ श्चिच् चि॒थ् सही॑यान् दे॒वो दे॒वः सही॑याꣳ श्चिच् चि॒थ् सही॑यान् दे॒वः । \newline
36. सही॑यान् दे॒वो दे॒वः सही॑या॒न् थ्सही॑यान् दे॒वो जी॒वात॑वे जी॒वात॑वे दे॒वः सही॑या॒न् थ्सही॑यान् दे॒वो जी॒वात॑वे । \newline
37. दे॒वो जी॒वात॑वे जी॒वात॑वे दे॒वो दे॒वो जी॒वात॑वे कृ॒तः कृ॒तो जी॒वात॑वे दे॒वो दे॒वो जी॒वात॑वे कृ॒तः । \newline
38. जी॒वात॑वे कृ॒तः कृ॒तो जी॒वात॑वे जी॒वात॑वे कृ॒तः । \newline
39. कृ॒त इति॑ कृ॒तः । \newline
40. इडा॑या स्त्वा॒ त्वेडा॑या॒ इडा॑या स्त्वा प॒दे प॒दे त्वेडा॑या॒ इडा॑या स्त्वा प॒दे । \newline
41. त्वा॒ प॒दे प॒दे त्वा᳚ त्वा प॒दे व॒यं ॅव॒यम् प॒दे त्वा᳚ त्वा प॒दे व॒यम् । \newline
42. प॒दे व॒यं ॅव॒यम् प॒दे प॒दे व॒यम् नाभा॒ नाभा॑ व॒यम् प॒दे प॒दे व॒यम् नाभा᳚ । \newline
43. व॒यम् नाभा॒ नाभा॑ व॒यं ॅव॒यम् नाभा॑ पृथि॒व्याः पृ॑थि॒व्या नाभा॑ व॒यं ॅव॒यम् नाभा॑ पृथि॒व्याः । \newline
44. नाभा॑ पृथि॒व्याः पृ॑थि॒व्या नाभा॒ नाभा॑ पृथि॒व्या अध्यधि॑ पृथि॒व्या नाभा॒ नाभा॑ पृथि॒व्या अधि॑ । \newline
45. पृ॒थि॒व्या अध्यधि॑ पृथि॒व्याः पृ॑थि॒व्या अधि॑ । \newline
46. अधीत्यधि॑ । \newline
47. जात॑वेदो॒ नि नि जात॑वेदो॒ जात॑वेदो॒ नि धी॑महि धीमहि॒ नि जात॑वेदो॒ जात॑वेदो॒ नि धी॑महि । \newline
48. जात॑वेद॒ इति॒ जात॑ - वे॒दः॒ । \newline
49. नि धी॑महि धीमहि॒ नि नि धी॑म॒ ह्यग्ने ऽग्ने॑ धीमहि॒ नि नि धी॑म॒ ह्यग्ने᳚ । \newline
50. धी॒म॒ ह्यग्ने ऽग्ने॑ धीमहि धीम॒ ह्यग्ने॑ ह॒व्याय॑ ह॒व्यायाग्ने॑ धीमहि धीम॒ ह्यग्ने॑ ह॒व्याय॑ । \newline
51. अग्ने॑ ह॒व्याय॑ ह॒व्यायाग्ने ऽग्ने॑ ह॒व्याय॒ वोढ॑वे॒ वोढ॑वे ह॒व्यायाग्ने ऽग्ने॑ ह॒व्याय॒ वोढ॑वे । \newline
52. ह॒व्याय॒ वोढ॑वे॒ वोढ॑वे ह॒व्याय॑ ह॒व्याय॒ वोढ॑वे । \newline
53. वोढ॑व॒ इति॒ वोढ॑वे । \newline
\pagebreak
\markright{ TS 3.5.11.2  \hfill https://www.vedavms.in \hfill}

\section{ TS 3.5.11.2 }

\textbf{TS 3.5.11.2 } \newline
\textbf{Samhita Paata} \newline

अग्ने॒ विश्वे॑भिः स्वनीक दे॒वैरूर्णा॑वन्तं प्रथ॒मः सी॑द॒ योनिं᳚ । कु॒ला॒यिनं॑ घृ॒तव॑न्तꣳ सवि॒त्रे य॒ज्ञ्ं न॑य॒ यज॑मानाय सा॒धु ॥ सीद॑ होतः॒ स्व उ॑ लो॒के चि॑कि॒त्वान्थ्सा॒दया॑ य॒ज्ञ्ꣳ सु॑कृ॒तस्य॒ योनौ᳚ । दे॒वा॒वीर्दे॒वान्. ह॒विषा॑ यजा॒स्यग्ने॑ बृ॒हद्-यज॑माने॒ वयो॑ धाः ॥ नि होता॑ होतृ॒षद॑ने॒ विदा॑नस्त्वे॒षो दी॑दि॒वाꣳ अ॑सदथ् सु॒दक्षः॑ । अद॑ब्धव्रत-प्रमति॒र्वसि॑ष्ठः सहस्रं भ॒रः शुचि॑जिह्वो अ॒ग्निः ॥ त्वं दू॒तस्त्व - [  ] \newline

\textbf{Pada Paata} \newline

अग्ने᳚ । विश्वे॑भिः । स्व॒नी॒केति॑ सु - अ॒नी॒क॒ । दे॒वैः । ऊर्णा॑वन्त॒मित्यूर्णा᳚-व॒न्त॒म् । प्र॒थ॒मः । सी॒द॒ । योनि᳚म् ॥ कु॒ला॒यिन᳚म् । घृ॒तव॑न्त॒मिति॑ घृ॒त - व॒न्त॒म् । स॒वि॒त्रे । य॒ज्ञ्म् । न॒य॒ । यज॑मानाय । सा॒धु ॥ सीद॑ । हो॒तः॒ । स्वे । उ॒ । लो॒के । चि॒कि॒त्वान् । सा॒दय॑ । य॒ज्ञ्म् । सु॒कृ॒तस्येति॑ सु - कृ॒तस्य॑ । योनौ᳚ ॥ दे॒वा॒वीरिति॑ देव - अ॒वीः । दे॒वान् । ह॒विषा᳚ । य॒जा॒सि॒ । अग्ने᳚ । बृ॒हत् । यज॑माने । वयः॑ । धाः॒ ॥ नीति॑ । होता᳚ । हो॒तृ॒षद॑न॒ इति॑ होतृ-सद॑ने । विदा॑नः । त्वे॒षः । दी॒दि॒वान् । अ॒स॒द॒त् । सु॒दक्ष॒ इति॑ सु - दक्षः॑ ॥ अद॑ब्धव्रतप्रमति॒रित्यद॑ब्धव्रत - प्र॒म॒तिः॒ । वसि॑ष्ठः । स॒ह॒स्र॒भं॒र इति॑ सहस्रं - भ॒रः । शुचि॑जिह्व॒ इति॒ शुचि॑ - जि॒ह्वः॒ । अ॒ग्निः ॥ त्वम् । दू॒तः । त्वम् ।  \newline


\textbf{Krama Paata} \newline

अग्ने॒ विश्वे॑भिः । विश्वे॑भिः स्वनीक । स्व॒नी॒क॒ दे॒वैः । स्व॒नी॒केति॑ सु - अ॒नी॒क॒ । दे॒वैरूर्णा॑वन्तम् । ऊर्णा॑वन्तम् प्रथ॒मः । ऊर्णा॑वन्त॒मित्यूर्णा᳚ - व॒न्त॒म् । प्र॒थ॒मः सी॑द । सी॒द॒ योनि᳚म् । योनि॒मिति॒योनि᳚म् ॥ कु॒ला॒यिन॑म् घृ॒तव॑न्तम् । घृ॒तव॑न्तꣳ सवि॒त्रे । घृ॒तव॑न्त॒मिति॑ घृ॒त - व॒न्त॒म् । स॒वि॒त्रे य॒ज्ञ्म् । य॒ज्ञ्म् न॑य । न॒य॒ यज॑मानाय । यज॑मनाय सा॒धु । सा॒द्ध्विति॑ सा॒धु ॥ सीद॑ होतः । हो॒तः॒ स्वे । स्व उ॑ । उ॒ लो॒के । लो॒के चि॑कि॒त्वान् । चि॒कि॒त्वान्थ् सा॒दय॑ । सा॒दया॑ य॒ज्ञ्म् । य॒ज्ञ्ꣳ सु॑कृ॒तस्य॑ । सु॒कृ॒तस्य॒ योनौ᳚ । सु॒कृ॒तस्येति॑ सु - कृ॒तस्य॑ । योना॒विति॒ योनौ᳚ ॥ दे॒वा॒वीर् दे॒वान् । दे॒वा॒वीरिति॑ देव - अ॒वीः । दे॒वान्. ह॒विषा᳚ । ह॒विषा॑ यजासि । य॒जा॒स्यग्ने᳚ । अग्ने॑ बृ॒हत् । बृ॒हद् यज॑माने । यज॑माने॒ वयः॑ । वयो॑ धाः । धा॒ इति॑ धाः ॥ नि होता᳚ । होता॑ होतृ॒षद॑ने । हो॒तृ॒षद॑ने॒ विदा॑नः । हो॒तृ॒षद॑न॒ इति॑ होतृ - सद॑ने । विदा॑नस्त्वे॒षः । त्वे॒षो दी॑दि॒वान् । दि॒दि॒वाꣳ अ॑सदत् । अ॒स॒द॒थ् सु॒दक्षः॑ । सु॒दक्ष॒ इति॑ सु - दक्षः॑ ॥ अद॑ब्धव्रतप्रमति॒र् वसि॑ष्ठः । अद॑ब्धव्रतप्रमति॒रित्यद॑ब्धव्रत - प्र॒म॒तिः॒ । वसि॑ष्ठः सहस्रम्भ॒रः । स॒ह॒स्र॒म्भ॒रः शुचि॑जिह्वः । स॒ह॒स्र॒म्भ॒र इति॑ सहस्रम् - भ॒रः । शुचि॑जिह्वो अ॒ग्निः । शुचि॑जिह्व॒ इति॒ शुचि॑ - जि॒ह्वः॒ । अ॒ग्निरित्य॒ग्निः ॥ त्वम् दू॒तः । दू॒तस्त्वम् । त्वमु॑ \newline

\textbf{Jatai Paata} \newline

1. अग्ने॒ विश्वे॑भि॒र् विश्वे॑भि॒ रग्ने ऽग्ने॒ विश्वे॑भिः । \newline
2. विश्वे॑भिः स्वनीक स्वनीक॒ विश्वे॑भि॒र् विश्वे॑भिः स्वनीक । \newline
3. स्व॒नी॒क॒ दे॒वैर् दे॒वैः स्व॑नीक स्वनीक दे॒वैः । \newline
4. स्व॒नी॒केति॑ सु - अ॒नी॒क॒ । \newline
5. दे॒वै रूर्णा॑वन्त॒ मूर्णा॑वन्तम् दे॒वैर् दे॒वै रूर्णा॑वन्तम् । \newline
6. ऊर्णा॑वन्तम् प्रथ॒मः प्र॑थ॒म ऊर्णा॑वन्त॒ मूर्णा॑वन्तम् प्रथ॒मः । \newline
7. ऊर्णा॑वन्त॒मित्यूर्णा᳚ - व॒न्त॒म् । \newline
8. प्र॒थ॒मः सी॑द सीद प्रथ॒मः प्र॑थ॒मः सी॑द । \newline
9. सी॒द॒ योनिं॒ ॅयोनिꣳ॑ सीद सीद॒ योनि᳚म् । \newline
10. योनि॒मिति॒योनि᳚म् । \newline
11. कु॒ला॒यिन॑म् घृ॒तव॑न्तम् घृ॒तव॑न्तम् कुला॒यिन॑म् कुला॒यिन॑म् घृ॒तव॑न्तम् । \newline
12. घृ॒तव॑न्तꣳ सवि॒त्रे स॑वि॒त्रे घृ॒तव॑न्तम् घृ॒तव॑न्तꣳ सवि॒त्रे । \newline
13. घृ॒तव॑न्त॒मिति॑ घृ॒त - व॒न्त॒म् । \newline
14. स॒वि॒त्रे य॒ज्ञ्ं ॅय॒ज्ञ्ꣳ स॑वि॒त्रे स॑वि॒त्रे य॒ज्ञ्म् । \newline
15. य॒ज्ञ्म् न॑य नय य॒ज्ञ्ं ॅय॒ज्ञ्म् न॑य । \newline
16. न॒य॒ यज॑मानाय॒ यज॑मानाय नय नय॒ यज॑मानाय । \newline
17. यज॑मानाय सा॒धु सा॒धु यज॑मानाय॒ यज॑मानाय सा॒धु । \newline
18. सा॒द्ध्विति॑ सा॒धु । \newline
19. सीद॑ होतर्. होतः॒ सीद॒ सीद॑ होतः । \newline
20. हो॒तः॒ स्वे स्वे हो॑तर्. होतः॒ स्वे । \newline
21. स्व उ॑ वु॒ स्वे स्व उ॑ । \newline
22. उ॒ लो॒के लो॒क उ॑ वु लो॒के । \newline
23. लो॒के चि॑कि॒त्वाꣳ श्चि॑कि॒त्वान् ॅलो॒के लो॒के चि॑कि॒त्वान् । \newline
24. चि॒कि॒त्वान् थ्सा॒दय॑ सा॒दय॑ चिकि॒त्वाꣳ श्चि॑कि॒त्वान् थ्सा॒दय॑ । \newline
25. सा॒दया॑ य॒ज्ञ्ं ॅय॒ज्ञ्ꣳ सा॒दय॑ सा॒दया॑ य॒ज्ञ्म् । \newline
26. य॒ज्ञ्ꣳ सु॑कृ॒तस्य॑ सुकृ॒तस्य॑ य॒ज्ञ्ं ॅय॒ज्ञ्ꣳ सु॑कृ॒तस्य॑ । \newline
27. सु॒कृ॒तस्य॒ योनौ॒ योनौ॑ सुकृ॒तस्य॑ सुकृ॒तस्य॒ योनौ᳚ । \newline
28. सु॒कृ॒तस्येति॑ सु - कृ॒तस्य॑ । \newline
29. योना॒विति॒ योनौ᳚ । \newline
30. दे॒वा॒वीर् दे॒वान् दे॒वान् दे॑वा॒वीर् दे॑वा॒वीर् दे॒वान् । \newline
31. दे॒वा॒वीरिति॑ देव - अ॒वीः । \newline
32. दे॒वान्. ह॒विषा॑ ह॒विषा॑ दे॒वान् दे॒वान्. ह॒विषा᳚ । \newline
33. ह॒विषा॑ यजासि यजासि ह॒विषा॑ ह॒विषा॑ यजासि । \newline
34. य॒जा॒स्यग्ने ऽग्ने॑ यजासि यजा॒स्यग्ने᳚ । \newline
35. अग्ने॑ बृ॒हद् बृ॒ह दग्ने ऽग्ने॑ बृ॒हत् । \newline
36. बृ॒हद् यज॑माने॒ यज॑माने बृ॒हद् बृ॒हद् यज॑माने । \newline
37. यज॑माने॒ वयो॒ वयो॒ यज॑माने॒ यज॑माने॒ वयः॑ । \newline
38. वयो॑ धा धा॒ वयो॒ वयो॑ धाः । \newline
39. धा॒ इति॑ धाः । \newline
40. नि होता॒ होता॒ नि नि होता᳚ । \newline
41. होता॑ होतृ॒षद॑ने होतृ॒षद॑ने॒ होता॒ होता॑ होतृ॒षद॑ने । \newline
42. हो॒तृ॒षद॑ने॒ विदा॑नो॒ विदा॑नो होतृ॒षद॑ने होतृ॒षद॑ने॒ विदा॑नः । \newline
43. हो॒तृ॒षद॑न॒ इति॑ होतृ - सद॑ने । \newline
44. विदा॑न स्त्वे॒ष स्त्वे॒षो विदा॑नो॒ विदा॑न स्त्वे॒षः । \newline
45. त्वे॒षो दी॑दि॒वान् दी॑दि॒वान् त्वे॒ष स्त्वे॒षो दी॑दि॒वान् । \newline
46. दी॒दि॒वाꣳ अ॑सद दसदद् दीदि॒वान् दी॑दि॒वाꣳ अ॑सदत् । \newline
47. अ॒स॒द॒थ् सु॒दक्षः॑ सु॒दक्षो॑ असद दसदथ् सु॒दक्षः॑ । \newline
48. सु॒दक्ष॒ इति॑ सु - दक्षः॑ । \newline
49. अद॑ब्धव्रतप्रमति॒र् वसि॑ष्ठो॒ वसि॑ष्ठो॒ अद॑ब्धव्रतप्रमति॒ रद॑ब्धव्रतप्रमति॒र् वसि॑ष्ठः । \newline
50. अद॑ब्धव्रतप्रमति॒रित्यद॑ब्धव्रत - प्र॒म॒तिः॒ । \newline
51. वसि॑ष्ठः सहस्रम्भ॒रः स॑हस्रम्भ॒रो वसि॑ष्ठो॒ वसि॑ष्ठः सहस्रम्भ॒रः । \newline
52. स॒ह॒स्र॒म्भ॒रः शुचि॑जिह्वः॒ शुचि॑जिह्वः सहस्रम्भ॒रः स॑हस्रम्भ॒रः शुचि॑जिह्वः । \newline
53. स॒ह॒स्र॒म्भ॒र इति॑ सहस्रं - भ॒रः । \newline
54. शुचि॑जिह्वो अ॒ग्नि र॒ग्निः शुचि॑जिह्वः॒ शुचि॑जिह्वो अ॒ग्निः । \newline
55. शुचि॑जिह्व॒ इति॒ शुचि॑ - जि॒ह्वः॒ । \newline
56. अ॒ग्निरित्य॒ग्निः । \newline
57. त्वम् दू॒तो दू॒त स्त्वम् त्वम् दू॒तः । \newline
58. दू॒त स्त्वम् त्वम् दू॒तो दू॒त स्त्वम् । \newline
59. त्व मु॑ वु॒ त्वम् त्व मु॑ । \newline

\textbf{Ghana Paata } \newline

1. अग्ने॒ विश्वे॑भि॒र् विश्वे॑भि॒ रग्ने ऽग्ने॒ विश्वे॑भिः स्वनीक स्वनीक॒ विश्वे॑भि॒ रग्ने ऽग्ने॒ विश्वे॑भिः स्वनीक । \newline
2. विश्वे॑भिः स्वनीक स्वनीक॒ विश्वे॑भि॒र् विश्वे॑भिः स्वनीक दे॒वैर् दे॒वैः स्व॑नीक॒ विश्वे॑भि॒र् विश्वे॑भिः स्वनीक दे॒वैः । \newline
3. स्व॒नी॒क॒ दे॒वैर् दे॒वैः स्व॑नीक स्वनीक दे॒वै रूर्णा॑वन्त॒ मूर्णा॑वन्तम् दे॒वैः स्व॑नीक स्वनीक दे॒वै रूर्णा॑वन्तम् । \newline
4. स्व॒नी॒केति॑ सु - अ॒नी॒क॒ । \newline
5. दे॒वै रूर्णा॑वन्त॒ मूर्णा॑वन्तम् दे॒वैर् दे॒वै रूर्णा॑वन्तम् प्रथ॒मः प्र॑थ॒म ऊर्णा॑वन्तम् दे॒वैर् दे॒वै रूर्णा॑वन्तम् प्रथ॒मः । \newline
6. ऊर्णा॑वन्तम् प्रथ॒मः प्र॑थ॒म ऊर्णा॑वन्त॒ मूर्णा॑वन्तम् प्रथ॒मः सी॑द सीद प्रथ॒म ऊर्णा॑वन्त॒ मूर्णा॑वन्तम् प्रथ॒मः सी॑द । \newline
7. ऊर्णा॑वन्त॒मित्यूर्णा᳚ - व॒न्त॒म् । \newline
8. प्र॒थ॒मः सी॑द सीद प्रथ॒मः प्र॑थ॒मः सी॑द॒ योनिं॒ ॅयोनिꣳ॑ सीद प्रथ॒मः प्र॑थ॒मः सी॑द॒ योनि᳚म् । \newline
9. सी॒द॒ योनिं॒ ॅयोनिꣳ॑ सीद सीद॒ योनि᳚म् । \newline
10. योनि॒मिति॒योनि᳚म् । \newline
11. कु॒ला॒यिन॑म् घृ॒तव॑न्तम् घृ॒तव॑न्तम् कुला॒यिन॑म् कुला॒यिन॑म् घृ॒तव॑न्तꣳ सवि॒त्रे स॑वि॒त्रे घृ॒तव॑न्तम् कुला॒यिन॑म् कुला॒यिन॑म् घृ॒तव॑न्तꣳ सवि॒त्रे । \newline
12. घृ॒तव॑न्तꣳ सवि॒त्रे स॑वि॒त्रे घृ॒तव॑न्तम् घृ॒तव॑न्तꣳ सवि॒त्रे य॒ज्ञ्ं ॅय॒ज्ञ्ꣳ स॑वि॒त्रे घृ॒तव॑न्तम् घृ॒तव॑न्तꣳ सवि॒त्रे य॒ज्ञ्म् । \newline
13. घृ॒तव॑न्त॒मिति॑ घृ॒त - व॒न्त॒म् । \newline
14. स॒वि॒त्रे य॒ज्ञ्ं ॅय॒ज्ञ्ꣳ स॑वि॒त्रे स॑वि॒त्रे य॒ज्ञ्म् न॑य नय य॒ज्ञ्ꣳ स॑वि॒त्रे स॑वि॒त्रे य॒ज्ञ्म् न॑य । \newline
15. य॒ज्ञ्म् न॑य नय य॒ज्ञ्ं ॅय॒ज्ञ्म् न॑य॒ यज॑मानाय॒ यज॑मानाय नय य॒ज्ञ्ं ॅय॒ज्ञ्म् न॑य॒ यज॑मानाय । \newline
16. न॒य॒ यज॑मानाय॒ यज॑मानाय नय नय॒ यज॑मानाय सा॒धु सा॒धु यज॑मानाय नय नय॒ यज॑मानाय सा॒धु । \newline
17. यज॑मानाय सा॒धु सा॒धु यज॑मानाय॒ यज॑मानाय सा॒धु । \newline
18. सा॒द्ध्विति॑ सा॒धु । \newline
19. सीद॑ होतर्. होतः॒ सीद॒ सीद॑ होतः॒ स्वे स्वे हो॑तः॒ सीद॒ सीद॑ होतः॒ स्वे । \newline
20. हो॒तः॒ स्वे स्वे हो॑तर्. होतः॒ स्व उ॑ वु॒ स्वे हो॑तर्. होतः॒ स्व उ॑ । \newline
21. स्व उ॑ वु॒ स्वे स्व उ॑ लो॒के लो॒क उ॒ स्वे स्व उ॑ लो॒के । \newline
22. उ॒ लो॒के लो॒क उ॑ वु लो॒के चि॑कि॒त्वाꣳ श्चि॑कि॒त्वान् ॅलो॒क उ॑ वु लो॒के चि॑कि॒त्वान् । \newline
23. लो॒के चि॑कि॒त्वाꣳ श्चि॑कि॒त्वान् ॅलो॒के लो॒के चि॑कि॒त्वान् थ्सा॒दय॑ सा॒दय॑ चिकि॒त्वान् ॅलो॒के लो॒के चि॑कि॒त्वान् थ्सा॒दय॑ । \newline
24. चि॒कि॒त्वान् थ्सा॒दय॑ सा॒दय॑ चिकि॒त्वाꣳ श्चि॑कि॒त्वान् थ्सा॒दया॑ य॒ज्ञ्ं ॅय॒ज्ञ्ꣳ सा॒दय॑ चिकि॒त्वाꣳ
श्चि॑कि॒त्वान् थ्सा॒दया॑ य॒ज्ञ्म् । \newline
25. सा॒दया॑ य॒ज्ञ्ं ॅय॒ज्ञ्ꣳ सा॒दय॑ सा॒दया॑ य॒ज्ञ्ꣳ सु॑कृ॒तस्य॑ सुकृ॒तस्य॑ य॒ज्ञ्ꣳ सा॒दय॑ सा॒दया॑ य॒ज्ञ्ꣳ सु॑कृ॒तस्य॑ । \newline
26. य॒ज्ञ्ꣳ सु॑कृ॒तस्य॑ सुकृ॒तस्य॑ य॒ज्ञ्ं ॅय॒ज्ञ्ꣳ सु॑कृ॒तस्य॒ योनौ॒ योनौ॑ सुकृ॒तस्य॑ य॒ज्ञ्ं ॅय॒ज्ञ्ꣳ सु॑कृ॒तस्य॒ योनौ᳚ । \newline
27. सु॒कृ॒तस्य॒ योनौ॒ योनौ॑ सुकृ॒तस्य॑ सुकृ॒तस्य॒ योनौ᳚ । \newline
28. सु॒कृ॒तस्येति॑ सु - कृ॒तस्य॑ । \newline
29. योना॒विति॒ योनौ᳚ । \newline
30. दे॒वा॒वीर् दे॒वान् दे॒वान् दे॑वा॒वीर् दे॑वा॒वीर् दे॒वान्. ह॒विषा॑ ह॒विषा॑ दे॒वान् दे॑वा॒वीर् दे॑वा॒वीर् दे॒वान्. ह॒विषा᳚ । \newline
31. दे॒वा॒वीरिति॑ देव - अ॒वीः । \newline
32. दे॒वान्. ह॒विषा॑ ह॒विषा॑ दे॒वान् दे॒वान्. ह॒विषा॑ यजासि यजासि ह॒विषा॑ दे॒वान् दे॒वान्. ह॒विषा॑ यजासि । \newline
33. ह॒विषा॑ यजासि यजासि ह॒विषा॑ ह॒विषा॑ यजा॒स्यग्ने ऽग्ने॑ यजासि ह॒विषा॑ ह॒विषा॑ यजा॒स्यग्ने᳚ । \newline
34. य॒जा॒स्यग्ने ऽग्ने॑ यजासि यजा॒स्यग्ने॑ बृ॒हद् बृ॒हदग्ने॑ यजासि यजा॒स्यग्ने॑ बृ॒हत् । \newline
35. अग्ने॑ बृ॒हद् बृ॒हदग्ने ऽग्ने॑ बृ॒हद् यज॑माने॒ यज॑माने बृ॒हदग्ने ऽग्ने॑ बृ॒हद् यज॑माने । \newline
36. बृ॒हद् यज॑माने॒ यज॑माने बृ॒हद् बृ॒हद् यज॑माने॒ वयो॒ वयो॒ यज॑माने बृ॒हद् बृ॒हद् यज॑माने॒ वयः॑ । \newline
37. यज॑माने॒ वयो॒ वयो॒ यज॑माने॒ यज॑माने॒ वयो॑ धा धा॒ वयो॒ यज॑माने॒ यज॑माने॒ वयो॑ धाः । \newline
38. वयो॑ धा धा॒ वयो॒ वयो॑ धाः । \newline
39. धा॒ इति॑ धाः । \newline
40. नि होता॒ होता॒ नि नि होता॑ होतृ॒षद॑ने होतृ॒षद॑ने॒ होता॒ नि नि होता॑ होतृ॒षद॑ने । \newline
41. होता॑ होतृ॒षद॑ने होतृ॒षद॑ने॒ होता॒ होता॑ होतृ॒षद॑ने॒ विदा॑नो॒ विदा॑नो होतृ॒षद॑ने॒ होता॒ होता॑ होतृ॒षद॑ने॒ विदा॑नः । \newline
42. हो॒तृ॒षद॑ने॒ विदा॑नो॒ विदा॑नो होतृ॒षद॑ने होतृ॒षद॑ने॒ विदा॑न स्त्वे॒ष स्त्वे॒षो विदा॑नो होतृ॒षद॑ने होतृ॒षद॑ने॒ विदा॑न स्त्वे॒षः । \newline
43. हो॒तृ॒षद॑न॒ इति॑ होतृ - सद॑ने । \newline
44. विदा॑न स्त्वे॒ष स्त्वे॒षो विदा॑नो॒ विदा॑न स्त्वे॒षो दी॑दि॒वान् दी॑दि॒वान् त्वे॒षो विदा॑नो॒ विदा॑न स्त्वे॒षो दी॑दि॒वान् । \newline
45. त्वे॒षो दी॑दि॒वान् दी॑दि॒वान् त्वे॒ष स्त्वे॒षो दी॑दि॒वाꣳ अ॑सद दसदद् दीदि॒वान् त्वे॒ष स्त्वे॒षो दी॑दि॒वाꣳ अ॑सदत् । \newline
46. दी॒दि॒वाꣳ अ॑सद दसदद् दीदि॒वान् दी॑दि॒वाꣳ अ॑सदथ् सु॒दक्षः॑ सु॒दक्षो॑ असदद् दीदि॒वान् दी॑दि॒वाꣳ अ॑सदथ् सु॒दक्षः॑ । \newline
47. अ॒स॒द॒थ् सु॒दक्षः॑ सु॒दक्षो॑ असद दसदथ् सु॒दक्षः॑ । \newline
48. सु॒दक्ष॒ इति॑ सु - दक्षः॑ । \newline
49. अद॑ब्धव्रतप्रमति॒र् वसि॑ष्ठो॒ वसि॑ष्ठो॒ अद॑ब्धव्रतप्रमति॒ रद॑ब्धव्रतप्रमति॒र् वसि॑ष्ठः सहस्रम्भ॒रः स॑हस्रम्भ॒रो वसि॑ष्ठो॒ अद॑ब्धव्रतप्रमति॒ रद॑ब्धव्रतप्रमति॒र् वसि॑ष्ठः सहस्रम्भ॒रः । \newline
50. अद॑ब्धव्रतप्रमति॒रित्यद॑ब्धव्रत - प्र॒म॒तिः॒ । \newline
51. वसि॑ष्ठः सहस्रम्भ॒रः स॑हस्रम्भ॒रो वसि॑ष्ठो॒ वसि॑ष्ठः सहस्रम्भ॒रः शुचि॑जिह्वः॒ शुचि॑जिह्वः सहस्रम्भ॒रो वसि॑ष्ठो॒ वसि॑ष्ठः सहस्रम्भ॒रः शुचि॑जिह्वः । \newline
52. स॒ह॒स्र॒म्भ॒रः शुचि॑जिह्वः॒ शुचि॑जिह्वः सहस्रम्भ॒रः स॑हस्रम्भ॒रः शुचि॑जिह्वो अ॒ग्नि र॒ग्निः शुचि॑जिह्वः सहस्रम्भ॒रः स॑हस्रम्भ॒रः शुचि॑जिह्वो अ॒ग्निः । \newline
53. स॒ह॒स्र॒म्भ॒र इति॑ सहस्रं - भ॒रः । \newline
54. शुचि॑जिह्वो अ॒ग्नि र॒ग्निः शुचि॑जिह्वः॒ शुचि॑जिह्वो अ॒ग्निः । \newline
55. शुचि॑जिह्व॒ इति॒ शुचि॑ - जि॒ह्वः॒ । \newline
56. अ॒ग्निरित्य॒ग्निः । \newline
57. त्वम् दू॒तो दू॒त स्त्वम् त्वम् दू॒त स्त्वम् त्वम् दू॒त स्त्वम् त्वम् दू॒त स्त्वम् । \newline
58. दू॒त स्त्वम् त्वम् दू॒तो दू॒त स्त्व मु॑ वु॒ त्वम् दू॒तो दू॒त स्त्व मु॑ । \newline
59. त्व मु॑ वु॒ त्वम् त्व मु॑ नो न उ॒ त्वम् त्व मु॑ नः । \newline
\pagebreak
\markright{ TS 3.5.11.3  \hfill https://www.vedavms.in \hfill}

\section{ TS 3.5.11.3 }

\textbf{TS 3.5.11.3 } \newline
\textbf{Samhita Paata} \newline

मु॑ नः पर॒स्पास्त्वं ॅवस्य॒ आ वृ॑षभ प्रणे॒ता । अग्ने॑ तो॒कस्य॑ न॒स्तने॑ त॒नूना॒मप्र॑युच्छ॒न् दीद्य॑द्बोधि गो॒पाः ॥ अ॒भि त्वा॑ देव सवित॒रीशा॑नं॒ ॅवार्या॑णां । सदा॑ऽवन् भा॒गमी॑महे ॥ म॒ही द्यौः पृ॑थि॒वी च॑न इ॒मं ॅय॒ज्ञ्ं मि॑मिक्षतां । पि॒पृ॒तां नो॒ भरी॑मभिः ॥ त्वाम॑ग्ने॒ पुष्क॑रा॒दद्ध्यथ॑र्वा॒ निर॑मन्थत । मू॒र्द्ध्नो विश्व॑स्य वा॒घतः॑ ॥ तमु॑ - [  ] \newline

\textbf{Pada Paata} \newline

उ॒ । नः॒ । प॒र॒स्पा इति॑ परः - पाः । त्वम् । वस्यः॑ । एति॑ । वृ॒ष॒भ॒ । प्र॒णे॒तेति॑ प्र - ने॒ता ॥ अग्ने᳚ । तो॒कस्य॑ । नः॒ । तने᳚ । त॒नूना᳚म् । अप्र॑युच्छ॒नित्यप्र॑ - यु॒च्छ॒न्न् । दीद्य॑त् । बो॒धि॒ । गो॒पा इति॑ गो-पाः ॥ अ॒भीति॑ । त्वा॒ । दे॒व॒ । स॒वि॒तः॒ । ईशा॑नम् । वार्या॑णाम् ॥ सदा᳚ । अ॒व॒न्न् । भा॒गम् । ई॒म॒हे॒ ॥ म॒ही । द्यौः । पृ॒थि॒वी । च॒ । नः॒ । इ॒मम् । य॒ज्ञ्म् । मि॒मि॒क्ष॒ता॒म् ॥ पि॒पृ॒ताम् । नः॒ । भरी॑मभि॒रिति॒ भरी॑म-भिः॒ ॥ त्वाम् । अ॒ग्ने॒ । पुष्क॑रात् । अधीति॑ । अथ॑र्वा । निरिति॑ । अ॒म॒न्थ॒त॒ ॥ मू॒र्द्ध्नः । विश्व॑स्य । वा॒घतः॑ ॥ तम् । उ॒ ।  \newline


\textbf{Krama Paata} \newline

उ॒ नः॒ । नः॒ प॒र॒स्पाः । प॒र॒स्पास्त्वम् । प॒र॒स्पा इति॑ परः - पाः । त्वं ॅवस्यः॑ । वस्य॒ आ । आ वृ॑षभ । वृ॒ष॒भ॒ प्र॒णे॒ता । प्र॒णे॒तेति॑ प्र - ने॒ता ॥ अग्ने॑ तो॒कस्य॑ । तो॒कस्य॑ नः । न॒ स्तने᳚ । तने॑ त॒नूना᳚म् । त॒नूना॒मप्र॑युच्छन्न् । अप्र॑युच्छ॒न् दीद्य॑त् । अप्र॑युच्छ॒न्नित्यप्र॑ - यु॒च्छ॒न्न्॒ । दीद्य॑द् बोधि । बो॒धि॒ गो॒पाः । गो॒पा इति॑ गो - पाः ॥ अ॒भि त्वा᳚ । त्वा॒ दे॒व॒ । दे॒व॒ स॒वि॒तः॒ । स॒वि॒त॒रीशा॑नम् । ईशा॑नं॒ ॅवार्या॑णाम् । वार्या॑णा॒मिति॒ वार्या॑णाम् ॥ सदा॑ऽवन्न् । अ॒व॒न् भा॒गम् । भा॒गमी॑महे । ई॒म॒ह॒ इती॑महे ॥ म॒ही द्यौः । द्यौः पृ॑थि॒वी । पृ॒थि॒वी च॑ । च॒ नः॒ । न॒ इ॒मम् । इ॒मं ॅय॒ज्ञ्म् । य॒ज्ञ्म् मि॑मिक्षताम् । मि॒मि॒क्ष॒ता॒मिति॑ मिमिक्षताम् ॥ पि॒पृ॒ताम् नः॑ । नो॒ भरी॑मभिः । भरी॑मभि॒रिति॒ भरी॑म - भिः॒ ॥ त्वाम॑ग्ने । अ॒ग्ने॒ पुष्क॑रात् । पुष्क॑रा॒दधि॑ । अ॒ध्यथ॑र्वा । अथ॑र्वा॒ निः । निर॑मन्थत । अ॒म॒न्थ॒तेत्य॑मन्थत ॥ मू॒र्द्ध्नो विश्व॑स्य । विश्व॑स्य वा॒घतः॑ । वा॒घत॒ इति॑ वा॒घतः॑ ॥ तमु॑ । उ॒ त्वा॒ \newline

\textbf{Jatai Paata} \newline

1. उ॒ नो॒ न॒ उ॒ वु॒ नः॒ । \newline
2. नः॒ प॒र॒स्पाः प॑र॒स्पा नो॑ नः पर॒स्पाः । \newline
3. प॒र॒स्पा स्त्वम् त्वम् प॑र॒स्पाः प॑र॒स्पा स्त्वम् । \newline
4. प॒र॒स्पा इति॑ परः - पाः । \newline
5. त्वं ॅवस्यो॒ वस्य॒ स्त्वम् त्वं ॅवस्यः॑ । \newline
6. वस्य॒ आ वस्यो॒ वस्य॒ आ । \newline
7. आ वृ॑षभ वृष॒भा वृ॑षभ । \newline
8. वृ॒ष॒भ॒ प्र॒णे॒ता प्र॑णे॒ता वृ॑षभ वृषभ प्रणे॒ता । \newline
9. प्र॒णे॒तेति॑ प्र - ने॒ता । \newline
10. अग्ने॑ तो॒कस्य॑ तो॒कस्याग्ने ऽग्ने॑ तो॒कस्य॑ । \newline
11. तो॒कस्य॑ नो न स्तो॒कस्य॑ तो॒कस्य॑ नः । \newline
12. न॒ स्तने॒ तने॑ नो न॒ स्तने᳚ । \newline
13. तने॑ त॒नूना᳚म् त॒नूना॒म् तने॒ तने॑ त॒नूना᳚म् । \newline
14. त॒नूना॒ मप्र॑युच्छ॒न् नप्र॑युच्छन् त॒नूना᳚म् त॒नूना॒ मप्र॑युच्छन्न् । \newline
15. अप्र॑युच्छ॒न् दीद्य॒द् दीद्य॒ दप्र॑युच्छ॒न् नप्र॑युच्छ॒न् दीद्य॑त् । \newline
16. अप्र॑युच्छ॒नित्यप्र॑ - यु॒च्छ॒न्न् । \newline
17. दीद्य॑द् बोधि बोधि॒ दीद्य॒द् दीद्य॑द् बोधि । \newline
18. बो॒धि॒ गो॒पा गो॒पा बो॑धि बोधि गो॒पाः । \newline
19. गो॒पा इति॑ गो - पाः । \newline
20. अ॒भि त्वा᳚ त्वा॒ ऽभ्य॑भि त्वा᳚ । \newline
21. त्वा॒ दे॒व॒ दे॒व॒ त्वा॒ त्वा॒ दे॒व॒ । \newline
22. दे॒व॒ स॒वि॒तः॒ स॒वि॒त॒र् दे॒व॒ दे॒व॒ स॒वि॒तः॒ । \newline
23. स॒वि॒त॒ रीशा॑न॒ मीशा॑नꣳ सवितः सवित॒ रीशा॑नम् । \newline
24. ईशा॑नं॒ ॅवार्या॑णां॒ ॅवार्या॑णा॒ मीशा॑न॒ मीशा॑नं॒ ॅवार्या॑णाम् । \newline
25. वार्या॑णा॒मिति॒ वार्या॑णाम् । \newline
26. सदा॑ ऽवन् नव॒न् थ्सदा॒ सदा॑ ऽवन्न् । \newline
27. अ॒व॒न् भा॒गम् भा॒ग म॑वन् नवन् भा॒गम् । \newline
28. भा॒ग मी॑मह ईमहे भा॒गम् भा॒ग मी॑महे । \newline
29. ई॒म॒ह॒ इती॑महे । \newline
30. म॒ही द्यौर् द्यौर् म॒ही म॒ही द्यौः । \newline
31. द्यौः पृ॑थि॒वी पृ॑थि॒वी द्यौर् द्यौः पृ॑थि॒वी । \newline
32. पृ॒थि॒वी च॑ च पृथि॒वी पृ॑थि॒वी च॑ । \newline
33. च॒ नो॒ न॒श्च॒ च॒ नः॒ । \newline
34. न॒ इ॒म मि॒मम् नो॑ न इ॒मम् । \newline
35. इ॒मं ॅय॒ज्ञ्ं ॅय॒ज्ञ् मि॒म मि॒मं ॅय॒ज्ञ्म् । \newline
36. य॒ज्ञ्म् मि॑मिक्षताम् मिमिक्षतां ॅय॒ज्ञ्ं ॅय॒ज्ञ्म् मि॑मिक्षताम् । \newline
37. मि॒मि॒क्ष॒ता॒मिति॑ मिमिक्षताम् । \newline
38. पि॒पृ॒ताम् नो॑ नः पिपृ॒ताम् पि॑पृ॒ताम् नः॑ । \newline
39. नो॒ भरी॑मभि॒र् भरी॑मभिर् नो नो॒ भरी॑मभिः । \newline
40. भरी॑मभि॒रिति॒ भरी॑म - भिः॒ । \newline
41. त्वा म॑ग्ने अग्ने॒ त्वाम् त्वा म॑ग्ने । \newline
42. अ॒ग्ने॒ पुष्क॑रा॒त् पुष्क॑रा दग्ने अग्ने॒ पुष्क॑रात् । \newline
43. पुष्क॑रा॒ दध्यधि॒ पुष्क॑रा॒त् पुष्क॑रा॒ दधि॑ । \newline
44. अध्यथ॒र्वा ऽथ॒र्वा ऽध्य ध्यथ॑र्वा । \newline
45. अथ॑र्वा॒ निर् णि रथ॒र्वा ऽथ॑र्वा॒ निः । \newline
46. निर॑मन्थता मन्थत॒ निर् णि र॑मन्थत । \newline
47. अ॒म॒न्थ॒तेत्य॑मन्थत । \newline
48. मू॒र्द्ध्नो विश्व॑स्य॒ विश्व॑स्य मू॒र्द्ध्नो मू॒र्द्ध्नो विश्व॑स्य । \newline
49. विश्व॑स्य वा॒घतो॑ वा॒घतो॒ विश्व॑स्य॒ विश्व॑स्य वा॒घतः॑ । \newline
50. वा॒घत॒ इति॑ वा॒घतः॑ । \newline
51. त मु॑ वु॒ तम् त मु॑ । \newline
52. उ॒ त्वा॒ त्व॒ वु॒ त्वा॒ । \newline

\textbf{Ghana Paata } \newline

1. उ॒ नो॒ न॒ उ॒ वु॒ नः॒ प॒र॒स्पाः प॑र॒स्पा न॑ उ वु नः पर॒स्पाः । \newline
2. नः॒ प॒र॒स्पाः प॑र॒स्पा नो॑ नः पर॒स्पा स्त्वम् त्वम् प॑र॒स्पा नो॑ नः पर॒स्पा स्त्वम् । \newline
3. प॒र॒स्पा स्त्वम् त्वम् प॑र॒स्पाः प॑र॒स्पा स्त्वं ॅवस्यो॒ वस्य॒ स्त्वम् प॑र॒स्पाः प॑र॒स्पा स्त्वं ॅवस्यः॑ । \newline
4. प॒र॒स्पा इति॑ परः - पाः । \newline
5. त्वं ॅवस्यो॒ वस्य॒ स्त्वम् त्वं ॅवस्य॒ आ वस्य॒ स्त्वम् त्वं ॅवस्य॒ आ । \newline
6. वस्य॒ आ वस्यो॒ वस्य॒ आ वृ॑षभ वृष॒भा वस्यो॒ वस्य॒ आ वृ॑षभ । \newline
7. आ वृ॑षभ वृष॒भा वृ॑षभ प्रणे॒ता प्र॑णे॒ता वृ॑ष॒भा वृ॑षभ प्रणे॒ता । \newline
8. वृ॒ष॒भ॒ प्र॒णे॒ता प्र॑णे॒ता वृ॑षभ वृषभ प्रणे॒ता । \newline
9. प्र॒णे॒तेति॑ प्र - ने॒ता । \newline
10. अग्ने॑ तो॒कस्य॑ तो॒कस्याग्ने ऽग्ने॑ तो॒कस्य॑ नो न स्तो॒कस्याग्ने ऽग्ने॑ तो॒कस्य॑ नः । \newline
11. तो॒कस्य॑ नो न स्तो॒कस्य॑ तो॒कस्य॑ न॒ स्तने॒ तने॑ न स्तो॒कस्य॑ तो॒कस्य॑ न ॒स्तने᳚ । \newline
12. न॒ स्तने॒ तने॑ नो न॒ स्तने॑ त॒नूना᳚म् त॒नूना॒म् तने॑ नो न॒ स्तने॑ त॒नूना᳚म् । \newline
13. तने॑ त॒नूना᳚म् त॒नूना॒म् तने॒ तने॑ त॒नूना॒ मप्र॑युच्छ॒न् नप्र॑युच्छन् त॒नूना॒म् तने॒ तने॑ त॒नूना॒ मप्र॑युच्छन्न् । \newline
14. त॒नूना॒ मप्र॑युच्छ॒न् नप्र॑युच्छन् त॒नूना᳚म् त॒नूना॒ मप्र॑युच्छ॒न् दीद्य॒द् दीद्य॒ दप्र॑युच्छन् त॒नूना᳚म् त॒नूना॒ मप्र॑युच्छ॒न् दीद्य॑त् । \newline
15. अप्र॑युच्छ॒न् दीद्य॒द् दीद्य॒ दप्र॑युच्छ॒न् नप्र॑युच्छ॒न् दीद्य॑द् बोधि बोधि॒ दीद्य॒ दप्र॑युच्छ॒न् नप्र॑युच्छ॒न् दीद्य॑द् बोधि । \newline
16. अप्र॑युच्छ॒नित्यप्र॑ - यु॒च्छ॒न्न् । \newline
17. दीद्य॑द् बोधि बोधि॒ दीद्य॒द् दीद्य॑द् बोधि गो॒पा गो॒पा बो॑धि॒ दीद्य॒द् दीद्य॑द् बोधि गो॒पाः । \newline
18. बो॒धि॒ गो॒पा गो॒पा बो॑धि बोधि गो॒पाः । \newline
19. गो॒पा इति॑ गो - पाः । \newline
20. अ॒भि त्वा᳚ त्वा॒ ऽभ्य॑भि त्वा॑ देव देव त्वा॒ ऽभ्य॑भि त्वा॑ देव । \newline
21. त्वा॒ दे॒व॒ दे॒व॒ त्वा॒ त्वा॒ दे॒व॒ स॒वि॒तः॒ स॒वि॒त॒र् दे॒व॒ त्वा॒ त्वा॒ दे॒व॒ स॒वि॒तः॒ । \newline
22. दे॒व॒ स॒वि॒तः॒ स॒वि॒त॒र् दे॒व॒ दे॒व॒ स॒वि॒त॒ रीशा॑न॒ मीशा॑नꣳ सवितर् देव देव सवित॒ रीशा॑नम् । \newline
23. स॒वि॒त॒ रीशा॑न॒ मीशा॑नꣳ सवितः सवित॒ रीशा॑नं॒ ॅवार्या॑णां॒ ॅवार्या॑णा॒ मीशा॑नꣳ सवितः सवित॒ रीशा॑नं॒ ॅवार्या॑णाम् । \newline
24. ईशा॑नं॒ ॅवार्या॑णां॒ ॅवार्या॑णा॒ मीशा॑न॒ मीशा॑नं॒ ॅवार्या॑णाम् । \newline
25. वार्या॑णा॒मिति॒ वार्या॑णाम् । \newline
26. सदा॑ ऽवन् नव॒न् थ्सदा॒ सदा॑ ऽवन् भा॒गम् भा॒ग म॑व॒न् थ्सदा॒ सदा॑ ऽवन् भा॒गम् । \newline
27. अ॒व॒न् भा॒गम् भा॒ग म॑वन् नवन् भा॒ग मी॑मह ईमहे भा॒ग म॑वन् नवन् भा॒ग मी॑महे । \newline
28. भा॒ग मी॑मह ईमहे भा॒गम् भा॒ग मी॑महे । \newline
29. ई॒म॒ह॒ इती॑महे । \newline
30. म॒ही द्यौर् द्यौर् म॒ही म॒ही द्यौः पृ॑थि॒वी पृ॑थि॒वी द्यौर् म॒ही म॒ही द्यौः पृ॑थि॒वी । \newline
31. द्यौः पृ॑थि॒वी पृ॑थि॒वी द्यौर् द्यौः पृ॑थि॒वी च॑ च पृथि॒वी द्यौर् द्यौः पृ॑थि॒वी च॑ । \newline
32. पृ॒थि॒वी च॑ च पृथि॒वी पृ॑थि॒वी च॑ नो नश्च पृथि॒वी पृ॑थि॒वी च॑ नः । \newline
33. च॒ नो॒ न॒श्च॒ च॒ न॒ इ॒म मि॒मम् न॑श्च च न इ॒मम् । \newline
34. न॒ इ॒म मि॒मम् नो॑ न इ॒मं ॅय॒ज्ञ्ं ॅय॒ज्ञ् मि॒मम् नो॑ न इ॒मं ॅय॒ज्ञ्म् । \newline
35. इ॒मं ॅय॒ज्ञ्ं ॅय॒ज्ञ् मि॒म मि॒मं ॅय॒ज्ञ्म् मि॑मिक्षताम् मिमिक्षतां ॅय॒ज्ञ् मि॒म मि॒मं ॅय॒ज्ञ्म् 
मि॑मिक्षताम् । \newline
36. य॒ज्ञ्म् मि॑मिक्षताम् मिमिक्षतां ॅय॒ज्ञ्ं ॅय॒ज्ञ्म् मि॑मिक्षताम् । \newline
37. मि॒मि॒क्ष॒ता॒मिति॑ मिमिक्षताम् । \newline
38. पि॒पृ॒ताम् नो॑ नः पिपृ॒ताम् पि॑पृ॒ताम् नो॒ भरी॑मभि॒र् भरी॑मभिर् नः पिपृ॒ताम् पि॑पृ॒ताम् नो॒ भरी॑मभिः । \newline
39. नो॒ भरी॑मभि॒र् भरी॑मभिर् नो नो॒ भरी॑मभिः । \newline
40. भरी॑मभि॒रिति॒ भरी॑म - भिः॒ । \newline
41. त्वा म॑ग्ने अग्ने॒ त्वाम् त्वा म॑ग्ने॒ पुष्क॑रा॒त् पुष्क॑रा दग्ने॒ त्वाम् त्वा म॑ग्ने॒ पुष्क॑रात् । \newline
42. अ॒ग्ने॒ पुष्क॑रा॒त् पुष्क॑रा दग्ने अग्ने॒ पुष्क॑रा॒ दध्यधि॒ पुष्क॑रा दग्ने अग्ने॒ पुष्क॑रा॒ दधि॑ । \newline
43. पुष्क॑रा॒ दध्यधि॒ पुष्क॑रा॒त् पुष्क॑रा॒ दध्यथ॒र्वा ऽथ॒र्वा ऽधि॒ पुष्क॑रा॒त् पुष्क॑रा॒ दध्यथ॑र्वा । \newline
44. अध्यथ॒र्वा ऽथ॒र्वा ऽध्य ध्यथ॑र्वा॒ निर् णि रथ॒र्वा ऽध्य ध्यथ॑र्वा॒ निः । \newline
45. अथ॑र्वा॒ निर् णि रथ॒र्वा ऽथ॑र्वा॒ निर॑मन्थता मन्थत॒ निरथ॒र्वा ऽथ॑र्वा॒ निर॑मन्थत । \newline
46. निर॑मन्थता मन्थत॒ निर् णि र॑मन्थत । \newline
47. अ॒म॒न्थ॒तेत्य॑मन्थत । \newline
48. मू॒र्द्ध्नो विश्व॑स्य॒ विश्व॑स्य मू॒र्द्ध्नो मू॒र्द्ध्नो विश्व॑स्य वा॒घतो॑ वा॒घतो॒ विश्व॑स्य मू॒र्द्ध्नो मू॒र्द्ध्नो विश्व॑स्य वा॒घतः॑ । \newline
49. विश्व॑स्य वा॒घतो॑ वा॒घतो॒ विश्व॑स्य॒ विश्व॑स्य वा॒घतः॑ । \newline
50. वा॒घत॒ इति॑ वा॒घतः॑ । \newline
51. त मु॑ वु॒ तम् त मु॑ त्वा त्वो॒ तम् त मु॑ त्वा । \newline
52. उ॒ त्वा॒ त्व॒ वु॒त्वा॒ द॒द्ध्यङ् द॒द्ध्यङ् त्व॑ वुत्वा द॒द्ध्यङ् । \newline
\pagebreak
\markright{ TS 3.5.11.4  \hfill https://www.vedavms.in \hfill}

\section{ TS 3.5.11.4 }

\textbf{TS 3.5.11.4 } \newline
\textbf{Samhita Paata} \newline

त्वा द॒द्ध्यङ्ङृषिः॑ पु॒त्र ई॑धे॒ अथ॑र्वणः । वृ॒त्र॒हणं॑ पुरन्द॒रं ॥ तमु॑ त्वा पा॒थ्यो वृषा॒ समी॑धे दस्यु॒हन्त॑मं । ध॒नं॒ ज॒यꣳ रणे॑रणे ॥ उ॒त ब्रु॑वन्तु ज॒न्तव॒ उद॒ग्निर्वृ॑त्र॒हाऽज॑नि । ध॒नं॒ ज॒यो रणे॑रणे ॥ आ यꣳ हस्ते॒ न खा॒दिनꣳ॒॒ शिशुं॑ जा॒तं न बिभ्र॑ति । वि॒शाम॒ग्निꣳ स्व॑द्ध्व॒रं ॥ प्रदे॒वं दे॒ववी॑तये॒ भर॑ता वसु॒वित्त॑मं । आस्वे योनौ॒ नि षी॑दतु ॥ आ - [  ] \newline

\textbf{Pada Paata} \newline

त्वा॒ । द॒द्ध्यङ् । ऋषिः॑ । पु॒त्रः । ई॒धे॒ । अथ॑र्वणः ॥ वृ॒त्र॒हण॒मिति॑ वृत्र - हन᳚म् । पु॒र॒न्द॒रमिति॑ पुरं - द॒रम् ॥ तम् । उ॒ । त्वा॒ । पा॒थ्यः । वृषा᳚ । समिति॑ । ई॒धे॒ । द॒स्यु॒हन्त॑म॒मिति॑ दस्यु - हन्त॑मम् ॥ ध॒न॒ञ्ज॒यमिति॑ धनं - ज॒यम् । रणे॑रण॒ इति॒ रणे᳚ - र॒णे॒ ॥ उ॒त । ब्रु॒व॒न्तु॒ । ज॒न्तवः॑ । उदिति॑ । अ॒ग्निः । वृ॒त्र॒हेति॑ वृत्र - हा । अ॒ज॒नि॒ ॥ ध॒न॒ञ्ज॒य इति॑ धनं - ज॒यः । रणे॑रण॒ इति॒ रणे᳚-र॒णे॒ ॥ एति॑ । यम् । हस्ते᳚ । न । खा॒दिन᳚म् । शिशु᳚म् । जा॒तम् । न । बिभ्र॑ति ॥ वि॒शाम् । अ॒ग्निम् । स्व॒द्ध्व॒रमिति॑ सु - अ॒ध्व॒रम् ॥ प्रेति॑ । दे॒वम् । दे॒ववी॑तय॒ इति॑ दे॒व - वी॒त॒ये॒ । भर॑त । व॒सु॒वित्त॑म॒मिति॑ वसु॒वित्-त॒म॒म् ॥ एति॑ । स्वे । योनौ᳚ । नीति॑ । सी॒द॒तु॒ ॥ एति॑ ।  \newline


\textbf{Krama Paata} \newline

त्वा॒ द॒द्ध्यङ्ङ् । द॒द्ध्यङ् ऋषिः॑ । ऋषिः॑ पु॒त्रः । पु॒त्र ई॑धे । ई॒धे॒ अथ॑र्वणः । अथ॑र्वण॒ इत्यथ॑र्वणः ॥ वृ॒त्र॒हण॑म् पुरन्द॒रम् । वृ॒त्र॒हण॒मिति॑ वृत्र - हन᳚म् । पु॒र॒न्द॒रमिति॑ पुरम् - द॒रम् ॥ तमु॑ । उ॒ त्वा॒ । त्वा॒ पा॒थ्यः । पा॒थ्यो वृषा᳚ । वृषा॒ सम् । समी॑धे । ई॒धे॒ द॒स्यु॒हन्त॑मम् । द॒स्यु॒हन्त॑म॒मिति॑ दस्यु - हन्त॑मम् ॥ ध॒न॒ञ्ज॒यꣳ रणे॑रणे । ध॒न॒ञ्ज॒यमिति॑ धनम् - ज॒यम् । रणे॑रण॒ इति॒ रणे᳚ - र॒णे॒ ॥ उ॒त ब्रु॑वन्तु । ब्रु॒व॒न्तु॒ ज॒न्तवः॑ । ज॒न्तव॒ उत् । उद॒ग्निः । अ॒ग्निर् वृ॑त्र॒हा । वृ॒त्र॒हा ऽज॑नि । वृ॒त्र॒हेति॑ वृत्र - हा । अ॒ज॒नीत्य॑जनि ॥ ध॒न॒ञ्ज॒यो रणे॑रणे । ध॒न॒ञ्ज॒य इति॑ धनम् - ज॒यः । रणे॑रण॒ इति॒ रणे᳚ - र॒णे॒ ॥ आ यम् । यꣳ हस्ते᳚ । हस्ते॒ न । न खा॒दिन᳚म् । खा॒दिनꣳ॒॒ शिशु᳚म् । शिशु॑म् जा॒तम् । जा॒तम् न । न बिभ्र॑ति । बिभ्र॒तीति॒ बिभ्र॑ति ॥ वि॒शाम॒ग्निम् । अ॒ग्निꣳ स्व॑द्ध्व॒रम् । स्व॒द्ध्व॒रमिति॑ सु - अ॒द्ध्व॒रम् ॥ प्र दे॒वम् । दे॒वम् दे॒ववी॑तये । दे॒ववी॑तये॒ भर॑त । दे॒ववी॑तय॒ इति॑ दे॒व - वी॒त॒ये॒ । भर॑ता वसु॒वित्त॑मम् । व॒सु॒वित्त॑म॒मिति॑ वसु॒वित् - त॒म॒म् ॥ आ स्वे । स्वे योनौ᳚ । योनौ॒ नि । 
नि षी॑दतु । सी॒द॒त्विति॑ सीदतु ॥ आ जा॒तम् \newline

\textbf{Jatai Paata} \newline

1. त्वा॒ द॒द्ध्यङ् द॒द्ध्यङ् त्वा᳚ त्वा द॒द्ध्यङ् । \newline
2. द॒द्ध्यङ् ङृषि॒र्॒. ऋषि॑र् द॒द्ध्यङ् द॒द्ध्यङ् ङृषिः॑ । \newline
3. ऋषिः॑ पु॒त्रः पु॒त्र ऋषि॒र्॒. ऋषिः॑ पु॒त्रः । \newline
4. पु॒त्र ई॑ध ईधे पु॒त्रः पु॒त्र ई॑धे । \newline
5. ई॒धे॒ अथ॑र्वणो॒ अथ॑र्वण ईध ईधे॒ अथ॑र्वणः । \newline
6. अथ॑र्वण॒ इत्यथ॑र्वणः । \newline
7. वृ॒त्र॒हण॑म् पुरन्द॒रम् पु॑रन्द॒रं ॅवृ॑त्र॒हणं॑ ॅवृत्र॒हण॑म् पुरन्द॒रम् । \newline
8. वृ॒त्र॒हण॒मिति॑ वृत्र - हन᳚म् । \newline
9. पु॒र॒न्द॒रमिति॑ पुरं - द॒रम् । \newline
10. त मु॑ वु॒ तम् त मु॑ । \newline
11. उ॒ त्वा॒ त्व॒ वु॒ त्वा॒ । \newline
12. त्वा॒ पा॒थ्यः पा॒थ्य स्त्वा᳚ त्वा पा॒थ्यः । \newline
13. पा॒थ्यो वृषा॒ वृषा॑ पा॒थ्यः पा॒थ्यो वृषा᳚ । \newline
14. वृषा॒ सꣳ सं ॅवृषा॒ वृषा॒ सम् । \newline
15. स मी॑ध ईधे॒ सꣳ स मी॑धे । \newline
16. ई॒धे॒ द॒स्यु॒हन्त॑मम् दस्यु॒हन्त॑म मीध ईधे दस्यु॒हन्त॑मम् । \newline
17. द॒स्यु॒हन्त॑म॒मिति॑ दस्यु - हन्त॑मम् । \newline
18. ध॒न॒ञ्ज॒यꣳ रणे॑रणे॒ रणे॑रणे धनञ्ज॒यम् ध॑नञ्ज॒यꣳ रणे॑रणे । \newline
19. ध॒न॒ञ्ज॒यमिति॑ धनं - ज॒यम् । \newline
20. रणे॑रण॒ इति॒ रणे᳚ - र॒णे॒ । \newline
21. उ॒त ब्रु॑वन्तु ब्रुवन्तू॒तोत ब्रु॑वन्तु । \newline
22. ब्रु॒व॒न्तु॒ ज॒न्तवो॑ ज॒न्तवो᳚ ब्रुवन्तु ब्रुवन्तु ज॒न्तवः॑ । \newline
23. ज॒न्तव॒ उदुज् ज॒न्तवो॑ ज॒न्तव॒ उत् । \newline
24. उद॒ग्नि र॒ग्नि रुदुद॒ग्निः । \newline
25. अ॒ग्निर् वृ॑त्र॒हा वृ॑त्र॒हा ऽग्नि र॒ग्निर् वृ॑त्र॒हा । \newline
26. वृ॒त्र॒हा ऽज॑ न्यजनि वृत्र॒हा वृ॑त्र॒हा ऽज॑नि । \newline
27. वृ॒त्र॒हेति॑ वृत्र - हा । \newline
28. अ॒ज॒नीत्य॑जनि । \newline
29. ध॒न॒ञ्ज॒यो रणे॑रणे॒ रणे॑रणे धनञ्ज॒यो ध॑नञ्ज॒यो रणे॑रणे । \newline
30. ध॒न॒ञ्ज॒य इति॑ धनं - ज॒यः । \newline
31. रणे॑रण॒ इति॒ रणे᳚ - र॒णे॒ । \newline
32. आ यं ॅय मा यम् । \newline
33. यꣳ हस्ते॒ हस्ते॒ यं ॅयꣳ हस्ते᳚ । \newline
34. हस्ते॒ न न हस्ते॒ हस्ते॒ न । \newline
35. न खा॒दिन॑म् खा॒दिन॒म् न न खा॒दिन᳚म् । \newline
36. खा॒दिनꣳ॒॒ शिशुꣳ॒॒ शिशु॑म् खा॒दिन॑म् खा॒दिनꣳ॒॒ शिशु᳚म् । \newline
37. शिशु॑म् जा॒तम् जा॒तꣳ शिशुꣳ॒॒ शिशु॑म् जा॒तम् । \newline
38. जा॒तम् न न जा॒तम् जा॒तम् न । \newline
39. न बिभ्र॑ति॒ बिभ्र॑ति॒ न न बिभ्र॑ति । \newline
40. बिभ्र॒तीति॒ बिभ्र॑ति । \newline
41. वि॒शा म॒ग्नि म॒ग्निं ॅवि॒शां ॅवि॒शा म॒ग्निम् । \newline
42. अ॒ग्निꣳ स्व॑द्ध्व॒रꣳ स्व॑द्ध्व॒र म॒ग्नि म॒ग्निꣳ स्व॑द्ध्व॒रम् । \newline
43. स्व॒द्ध्व॒रमिति॑ सु - अ॒ध्व॒रम् । \newline
44. प्र दे॒वम् दे॒वम् प्र प्र दे॒वम् । \newline
45. दे॒वम् दे॒ववी॑तये दे॒ववी॑तये दे॒वम् दे॒वम् दे॒ववी॑तये । \newline
46. दे॒ववी॑तये॒ भर॑त॒ भर॑त दे॒ववी॑तये दे॒ववी॑तये॒ भर॑त । \newline
47. दे॒ववी॑तय॒ इति॑ दे॒व - वी॒त॒ये॒ । \newline
48. भर॑ता वसु॒वित्त॑मं ॅवसु॒वित्त॑म॒म् भर॑त॒ भर॑ता वसु॒वित्त॑मम् । \newline
49. व॒सु॒वित्त॑म॒मिति॑ वसु॒वित् - त॒म॒म् । \newline
50. आ स्वे स्व आ स्वे । \newline
51. स्वे योनौ॒ योनौ॒ स्वे स्वे योनौ᳚ । \newline
52. योनौ॒ नि नि योनौ॒ योनौ॒ नि । \newline
53. नि षी॑दतु सीदतु॒ नि नि षी॑दतु । \newline
54. सी॒द॒त्विति॑ सीदतु । \newline
55. आ जा॒तम् जा॒त मा जा॒तम् । \newline

\textbf{Ghana Paata } \newline

1. त्वा॒ द॒द्ध्यङ् द॒द्ध्यङ् त्वा᳚ त्वा द॒द्ध्यङ् ङृषि॒र्॒. ऋषि॑र् द॒द्ध्यङ् त्वा᳚ त्वा द॒द्ध्यङ् ङृषिः॑ । \newline
2. द॒द्ध्यङ् ङृषि॒र्॒. ऋषि॑र् द॒द्ध्यङ् द॒द्ध्यङ् ङृषिः॑ पु॒त्रः पु॒त्र ऋषि॑र् द॒द्ध्यङ् द॒द्ध्यङ्
ङृषिः॑ पु॒त्रः । \newline
3. ऋषिः॑ पु॒त्रः पु॒त्र ऋषि॒र्॒. ऋषिः॑ पु॒त्र ई॑ध ईधे पु॒त्र ऋषि॒र्॒. ऋषिः॑ पु॒त्र ई॑धे । \newline
4. पु॒त्र ई॑ध ईधे पु॒त्रः पु॒त्र ई॑धे॒ अथ॑र्वणो॒ अथ॑र्वण ईधे पु॒त्रः पु॒त्र ई॑धे॒ अथ॑र्वणः । \newline
5. ई॒धे॒ अथ॑र्वणो॒ अथ॑र्वण ईध ईधे॒ अथ॑र्वणः । \newline
6. अथ॑र्वण॒ इत्यथ॑र्वणः । \newline
7. वृ॒त्र॒हण॑म् पुरन्द॒रम् पु॑रन्द॒रं ॅवृ॑त्र॒हणं॑ ॅवृत्र॒हण॑म् पुरन्द॒रम् । \newline
8. वृ॒त्र॒हण॒मिति॑ वृत्र - हन᳚म् । \newline
9. पु॒र॒न्द॒रमिति॑ पुरं - द॒रम् । \newline
10. त मु॑ वु॒ तम् त मु॑ त्वा त्वो॒ तम् त मु॑ त्वा । \newline
11. उ॒ त्वा॒ त्व॒ वु॒त्वा॒ पा॒थ्यः पा॒थ्य स्त्व॑ वुत्वा पा॒थ्यः । \newline
12. त्वा॒ पा॒थ्यः पा॒थ्य स्त्वा᳚ त्वा पा॒थ्यो वृषा॒ वृषा॑ पा॒थ्य स्त्वा᳚ त्वा पा॒थ्यो वृषा᳚ । \newline
13. पा॒थ्यो वृषा॒ वृषा॑ पा॒थ्यः पा॒थ्यो वृषा॒ सꣳ सं ॅवृषा॑ पा॒थ्यः पा॒थ्यो वृषा॒ सम् । \newline
14. वृषा॒ सꣳ सं ॅवृषा॒ वृषा॒ स मी॑ध ईधे॒ सं ॅवृषा॒ वृषा॒ स मी॑धे । \newline
15. स मी॑ध ईधे॒ सꣳ स मी॑धे दस्यु॒हन्त॑मम् दस्यु॒हन्त॑म मीधे॒ सꣳ स मी॑धे दस्यु॒हन्त॑मम् । \newline
16. ई॒धे॒ द॒स्यु॒हन्त॑मम् दस्यु॒हन्त॑म मीध ईधे दस्यु॒हन्त॑मम् । \newline
17. द॒स्यु॒हन्त॑म॒मिति॑ दस्यु - हन्त॑मम् । \newline
18. ध॒न॒ञ्ज॒यꣳ रणे॑रणे॒ रणे॑रणे धनञ्ज॒यम् ध॑नञ्ज॒यꣳ रणे॑रणे । \newline
19. ध॒न॒ञ्ज॒यमिति॑ धनं - ज॒यम् । \newline
20. रणे॑रण॒ इति॒ रणे᳚ - र॒णे॒ । \newline
21. उ॒त ब्रु॑वन्तु ब्रुवन्तू॒तोत ब्रु॑वन्तु ज॒न्तवो॑ ज॒न्तवो᳚ ब्रुवन्तू॒तोत ब्रु॑वन्तु ज॒न्तवः॑ । \newline
22. ब्रु॒व॒न्तु॒ ज॒न्तवो॑ ज॒न्तवो᳚ ब्रुवन्तु ब्रुवन्तु ज॒न्तव॒ उदुज् ज॒न्तवो᳚ ब्रुवन्तु ब्रुवन्तु ज॒न्तव॒ उत् । \newline
23. ज॒न्तव॒ उदुज् ज॒न्तवो॑ ज॒न्तव॒ उद॒ग्नि र॒ग्नि रुज् ज॒न्तवो॑ ज॒न्तव॒ उद॒ग्निः । \newline
24. उद॒ग्नि र॒ग्नि रुदु द॒ग्निर् वृ॑त्र॒हा वृ॑त्र॒हा ऽग्नि रुदु द॒ग्निर् वृ॑त्र॒हा । \newline
25. अ॒ग्निर् वृ॑त्र॒हा वृ॑त्र॒हा ऽग्नि र॒ग्निर् वृ॑त्र॒हा ऽज॑न्यजनि वृत्र॒हा ऽग्नि र॒ग्निर् वृ॑त्र॒हा ऽज॑नि । \newline
26. वृ॒त्र॒हा ऽज॑ न्यजनि वृत्र॒हा वृ॑त्र॒हा ऽज॑नि । \newline
27. वृ॒त्र॒हेति॑ वृत्र - हा । \newline
28. अ॒ज॒नीत्य॑जनि । \newline
29. ध॒न॒ञ्ज॒यो रणे॑रणे॒ रणे॑रणे धनञ्ज॒यो ध॑नञ्ज॒यो रणे॑रणे । \newline
30. ध॒न॒ञ्ज॒य इति॑ धनं - ज॒यः । \newline
31. रणे॑रण॒ इति॒ रणे᳚ - र॒णे॒ । \newline
32. आ यं ॅय मा यꣳ हस्ते॒ हस्ते॒ य मा यꣳ हस्ते᳚ । \newline
33. यꣳ हस्ते॒ हस्ते॒ यं ॅयꣳ हस्ते॒ न न हस्ते॒ यं ॅयꣳ हस्ते॒ न । \newline
34. हस्ते॒ न न हस्ते॒ हस्ते॒ न खा॒दिन॑म् खा॒दिन॒म् न हस्ते॒ हस्ते॒ न खा॒दिन᳚म् । \newline
35. न खा॒दिन॑म् खा॒दिन॒म् न न खा॒दिनꣳ॒॒ शिशुꣳ॒॒ शिशु॑म् खा॒दिन॒म् न न खा॒दिनꣳ॒॒ शिशु᳚म् । \newline
36. खा॒दिनꣳ॒॒ शिशुꣳ॒॒ शिशु॑म् खा॒दिन॑म् खा॒दिनꣳ॒॒ शिशु॑म् जा॒तम् जा॒तꣳ शिशु॑म् खा॒दिन॑म् खा॒दिनꣳ॒॒ शिशु॑म् जा॒तम् । \newline
37. शिशु॑म् जा॒तम् जा॒तꣳ शिशुꣳ॒॒ शिशु॑म् जा॒तम् न न जा॒तꣳ शिशुꣳ॒॒ शिशु॑म् जा॒तम् न । \newline
38. जा॒तम् न न जा॒तम् जा॒तन्न बिभ्र॑ति॒ बिभ्र॑ति॒ न जा॒तम् जा॒तम् न बिभ्र॑ति । \newline
39. न बिभ्र॑ति॒ बिभ्र॑ति॒ न न बिभ्र॑ति । \newline
40. बिभ्र॒तीति॒ बिभ्र॑ति । \newline
41. वि॒शा म॒ग्नि म॒ग्निं ॅवि॒शां ॅवि॒शा म॒ग्निꣳ स्व॑द्ध्व॒रꣳ स्व॑द्ध्व॒र म॒ग्निं ॅवि॒शां ॅवि॒शा म॒ग्निꣳ स्व॑द्ध्व॒रम् । \newline
42. अ॒ग्निꣳ स्व॑द्ध्व॒रꣳ स्व॑द्ध्व॒र म॒ग्नि म॒ग्निꣳ स्व॑द्ध्व॒रम् । \newline
43. स्व॒द्ध्व॒रमिति॑ सु - अ॒ध्व॒रम् । \newline
44. प्र दे॒वम् दे॒वम् प्र प्र दे॒वम् दे॒ववी॑तये दे॒ववी॑तये दे॒वम् प्र प्र दे॒वम् दे॒ववी॑तये । \newline
45. दे॒वम् दे॒ववी॑तये दे॒ववी॑तये दे॒वम् दे॒वम् दे॒ववी॑तये॒ भर॑त॒ भर॑त दे॒ववी॑तये दे॒वम् दे॒वम् दे॒ववी॑तये॒ भर॑त । \newline
46. दे॒ववी॑तये॒ भर॑त॒ भर॑त दे॒ववी॑तये दे॒ववी॑तये॒ भर॑ता वसु॒वित्त॑मं ॅवसु॒वित्त॑म॒म् भर॑त दे॒ववी॑तये दे॒ववी॑तये॒ भर॑ता वसु॒वित्त॑मम् । \newline
47. दे॒ववी॑तय॒ इति॑ दे॒व - वी॒त॒ये॒ । \newline
48. भर॑ता वसु॒वित्त॑मं ॅवसु॒वित्त॑म॒म् भर॑त॒ भर॑ता वसु॒वित्त॑मम् । \newline
49. व॒सु॒वित्त॑म॒मिति॑ वसु॒वित् - त॒म॒म् । \newline
50. आ स्वे स्व आ स्वे योनौ॒ योनौ॒ स्व आ स्वे योनौ᳚ । \newline
51. स्वे योनौ॒ योनौ॒ स्वे स्वे योनौ॒ नि नि योनौ॒ स्वे स्वे योनौ॒ नि । \newline
52. योनौ॒ नि नि योनौ॒ योनौ॒ नि षी॑दतु सीदतु॒ नि योनौ॒ योनौ॒ नि षी॑दतु । \newline
53. नि षी॑दतु सीदतु॒ नि नि षी॑दतु । \newline
54. सी॒द॒त्विति॑ सीदतु । \newline
55. आ जा॒तम् जा॒त मा जा॒तम् जा॒तवे॑दसि जा॒तवे॑दसि जा॒त मा जा॒तम् जा॒तवे॑दसि । \newline
\pagebreak
\markright{ TS 3.5.11.5  \hfill https://www.vedavms.in \hfill}

\section{ TS 3.5.11.5 }

\textbf{TS 3.5.11.5 } \newline
\textbf{Samhita Paata} \newline

जा॒तं जा॒तवे॑दसि प्रि॒यꣳ शि॑शी॒ताऽति॑थिं । स्यो॒न आ गृ॒हप॑तिं ॥ अ॒ग्निना॒ऽग्निः समि॑द्ध्यते क॒विर्गृ॒हप॑ति॒र्युवा᳚ । ह॒व्य॒वाड् जु॒ह्वा᳚स्यः ॥त्वꣳ ह्य॑ग्ने अ॒ग्निना॒ विप्रो॒ विप्रे॑ण॒ सन्थ्स॒ता । सखा॒ सख्या॑ समि॒द्ध्यसे᳚ ॥ तं म॑र्जयन्त सु॒क्रतुं॑ पुरो॒यावा॑नमा॒जिषु॑ । स्वेषु॒ क्षये॑षु वा॒जिनं᳚ ॥ य॒ज्ञेन॑ य॒ज्ञ्म॑यजन्त दे॒वास्तानि॒ धर्मा॑णि प्रथ॒मान्या॑सन्न् । ते ह॒ नाकं॑ महि॒मानः॑ सचन्ते॒ यत्र॒ ( ) पूर्वे॑ सा॒द्ध्याः सन्ति॑ दे॒वाः ॥ \newline

\textbf{Pada Paata} \newline

जा॒तम् । जा॒तवे॑द॒सीति॑ जा॒त - वे॒द॒सि॒ । प्रि॒यम् । शि॒शी॒त॒ । अति॑थिम् ॥ स्यो॒ने । एति॑ । गृ॒हप॑ति॒मिति॑ गृ॒ह - प॒ति॒म् ॥ अ॒ग्निना᳚ । अ॒ग्निः । समिति॑ । इ॒द्ध्य॒ते॒ । क॒विः । गृ॒हप॑ति॒रिति॑ गृ॒ह - प॒तिः॒ । युवा᳚ ॥ ह॒व्य॒वाडिति॑ हव्य - वाट् । जु॒ह्वा᳚स्य॒ इति॑ जु॒हु - आ॒स्यः॒ ॥ त्वम् । हि । अ॒ग्ने॒ । अ॒ग्निना᳚ । विप्रः॑ । विप्रे॑ण । सन्न् । स॒ता ॥ सखा᳚ । सख्या᳚ । स॒मि॒द्ध्यस॒ इति॑ सं - इ॒ध्यसे᳚ ॥ तम् । म॒र्ज॒य॒न्त॒ । सु॒क्रतु॒मिति॑ सु - क्रतु᳚म् । पु॒रो॒यावा॑न॒मिति॑ पुरः - यावा॑नम् । आ॒जिषु॑ ॥ स्वेषु॑ । क्षये॑षु । वा॒जिन᳚म् ॥ य॒ज्ञेन॑ । य॒ज्ञ्म् । अ॒य॒ज॒न्त॒ । दे॒वाः । तानि॑ । धर्मा॑णि । प्र॒थ॒मानि॑ । आ॒स॒न्न् ॥ ते । ह॒ । नाक᳚म् । म॒हि॒मानः॑ । स॒च॒न्ते॒ । यत्र॑ ( ) । पूर्वे᳚ । सा॒द्ध्याः । सन्ति॑ । दे॒वाः ॥  \newline


\textbf{Krama Paata} \newline

जा॒तम् जा॒तवे॑दसि । जा॒तवे॑दसि प्रि॒यम् । जा॒तवे॑द॒सीति॑ जा॒त - वे॒द॒सि॒ । प्रि॒यꣳ शि॑शीत । शि॒शी॒ताति॑थिम् । अति॑थि॒मित्यति॑थिम् ॥ स्यो॒न आ । आ गृ॒हप॑तिम् । गृ॒हप॑ति॒मिति॑ गृ॒ह - प॒ति॒म् ॥ अ॒ग्निना॒ ऽग्निः । अ॒ग्निः सम् । समि॑द्ध्यते । इ॒द्ध्य॒ते॒ क॒विः । क॒विर् गृ॒हप॑तिः । गृ॒हप॑ति॒र् युवा᳚ । गृ॒हप॑ति॒रिति॑ गृ॒ह - प॒तिः॒ । युवेति॒ युवा᳚ ॥ ह॒व्य॒वाड् जु॒ह्वा᳚स्यः । ह॒व्य॒वाडिति॑ हव्य - वाट् । जु॒ह्वा᳚स्य॒ इति॑ जु॒हु - आ॒स्यः॒ ॥ त्वꣳ हि । ह्य॑ग्ने । अ॒ग्ने॒ अ॒ग्निना᳚ । अ॒ग्निना॒ विप्रः॑ । विप्रो॒ विप्रे॑ण । विप्रे॑ण॒ सन्न् । सन्थ् स॒ता । स॒तेति॑ स॒ता ॥ सखा॒ सख्या᳚ । सख्या॑ समि॒द्ध्यसे᳚ । स॒मि॒द्ध्यस॒ इति॑ सम् - इ॒द्ध्यसे᳚ ॥ तम् म॑र्जयन्त । म॒र्ज॒य॒न्त॒ सु॒क्रतु᳚म् । सु॒क्रतु॑म् पुरो॒यावा॑नम् । सु॒क्रतु॒मिति॑ सु - क्रतु᳚म् । पु॒रो॒यावा॑नमा॒जिषु॑ । पु॒रो॒यावा॑न॒मिति॑ पुरः - यावा॑नम् । आ॒जिष्वित्या॒जिषु॑ ॥ स्वेषु॒ क्षये॑षु । क्षये॑षु वा॒जिन᳚म् । वा॒जिन॒मिति॑ वा॒जिन᳚म् ॥ य॒ज्ञेन॑ य॒ज्ञ्म् । य॒ज्ञ्म॑यजन्त । अ॒य॒ज॒न्त॒ दे॒वाः । दे॒वास्तानि॑ । तानि॒ धर्मा॑णि । धर्मा॑णि प्रथ॒मानि॑ । प्र॒थ॒मान्या॑सन्न् । आ॒स॒न्नित्या॑सन्न् ॥ ते ह॑ । ह॒ नाक᳚म् । नाक॑म् महि॒मानः॑ । म॒हि॒मानः॑ सचन्ते । स॒च॒न्ते॒ यत्र॑ ( ) । यत्र॒ पूर्वे᳚ । पूर्वे॑ सा॒द्ध्याः । सा॒द्ध्याः सन्ति॑ । सन्ति॑ दे॒वाः । दे॒वा इति॑ दे॒वाः । \newline

\textbf{Jatai Paata} \newline

1. जा॒तम् जा॒तवे॑दसि जा॒तवे॑दसि जा॒तम् जा॒तम् जा॒तवे॑दसि । \newline
2. जा॒तवे॑दसि प्रि॒यम् प्रि॒यम् जा॒तवे॑दसि जा॒तवे॑दसि प्रि॒यम् । \newline
3. जा॒तवे॑द॒सीति॑ जा॒त - वे॒द॒सि॒ । \newline
4. प्रि॒यꣳ शि॑शीत शिशीत प्रि॒यम् प्रि॒यꣳ शि॑शीत । \newline
5. शि॒शी॒ता ति॑थि॒ मति॑थिꣳ शिशीत शिशी॒ता ति॑थिम् । \newline
6. अति॑थि॒मित्यति॑थिम् । \newline
7. स्यो॒न आ स्यो॒ने स्यो॒न आ । \newline
8. आ गृ॒हप॑तिम् गृ॒हप॑ति॒ मा गृ॒हप॑तिम् । \newline
9. गृ॒हप॑ति॒मिति॑ गृ॒ह - प॒ति॒म् । \newline
10. अ॒ग्निना॒ ऽग्नि र॒ग्नि र॒ग्निना॒ ऽग्निना॒ ऽग्निः । \newline
11. अ॒ग्निः सꣳ स म॒ग्नि र॒ग्निः सम् । \newline
12. स मि॑द्ध्यत इद्ध्यते॒ सꣳ स मि॑द्ध्यते । \newline
13. इ॒द्ध्य॒ते॒ क॒विः क॒वि रि॑द्ध्यत इद्ध्यते क॒विः । \newline
14. क॒विर् गृ॒हप॑तिर् गृ॒हप॑तिः क॒विः क॒विर् गृ॒हप॑तिः । \newline
15. गृ॒हप॑ति॒र् युवा॒ युवा॑ गृ॒हप॑तिर् गृ॒हप॑ति॒र् युवा᳚ । \newline
16. गृ॒हप॑ति॒रिति॑ गृ॒ह - प॒तिः॒ । \newline
17. युवेति॒ युवा᳚ । \newline
18. ह॒व्य॒वाड् जु॒ह्वा᳚स्यो जु॒ह्वा᳚स्यो हव्य॒वा ड्ढ॑व्य॒वाड् जु॒ह्वा᳚स्यः । \newline
19. ह॒व्य॒वाडिति॑ हव्य - वाट् । \newline
20. जु॒ह्वा᳚स्य॒ इति॑ जु॒हु - आ॒स्यः॒ । \newline
21. त्वꣳ हि हि त्वम् त्वꣳ हि । \newline
22. ह्य॑ग्ने अग्ने॒ हि ह्य॑ग्ने । \newline
23. अ॒ग्ने॒ अ॒ग्निना॒ ऽग्निना᳚ ऽग्ने अग्ने अ॒ग्निना᳚ । \newline
24. अ॒ग्निना॒ विप्रो॒ विप्रो॑ अ॒ग्निना॒ ऽग्निना॒ विप्रः॑ । \newline
25. विप्रो॒ विप्रे॑ण॒ विप्रे॑ण॒ विप्रो॒ विप्रो॒ विप्रे॑ण । \newline
26. विप्रे॑ण॒ सन् थ्सन्. विप्रे॑ण॒ विप्रे॑ण॒ सन्न् । \newline
27. सन् थ्स॒ता स॒ता सन् थ्सन् थ्स॒ता । \newline
28. स॒तेति॑ स॒ता । \newline
29. सखा॒ सख्या॒ सख्या॒ सखा॒ सखा॒ सख्या᳚ । \newline
30. सख्या॑ समि॒द्ध्यसे॑ समि॒द्ध्यसे॒ सख्या॒ सख्या॑ समि॒द्ध्यसे᳚ । \newline
31. स॒मि॒द्ध्यस॒ इति॑ सं - इ॒ध्यसे᳚ । \newline
32. तम् म॑र्जयन्त मर्जयन्त॒ तम् तम् म॑र्जयन्त । \newline
33. म॒र्ज॒य॒न्त॒ सु॒क्रतुꣳ॑ सु॒क्रतु॑म् मर्जयन्त मर्जयन्त सु॒क्रतु᳚म् । \newline
34. सु॒क्रतु॑म् पुरो॒यावा॑नम् पुरो॒यावा॑नꣳ सु॒क्रतुꣳ॑ सु॒क्रतु॑म् पुरो॒यावा॑नम् । \newline
35. सु॒क्रतु॒मिति॑ सु - क्रतु᳚म् । \newline
36. पु॒रो॒यावा॑न मा॒जि ष्वा॒जिषु॑ पुरो॒यावा॑नम् पुरो॒यावा॑न मा॒जिषु॑ । \newline
37. पु॒रो॒यावा॑न॒मिति॑ पुरः - यावा॑नम् । \newline
38. आ॒जिष्वित्या॒जिषु॑ । \newline
39. स्वेषु॒ क्षये॑षु॒ क्षये॑षु॒ स्वेषु॒ स्वेषु॒ क्षये॑षु । \newline
40. क्षये॑षु वा॒जिनं॑ ॅवा॒जिन॒म् क्षये॑षु॒ क्षये॑षु वा॒जिन᳚म् । \newline
41. वा॒जिन॒मिति॑ वा॒जिन᳚म् । \newline
42. य॒ज्ञेन॑ य॒ज्ञ्ं ॅय॒ज्ञ्ं ॅय॒ज्ञेन॑ य॒ज्ञेन॑ य॒ज्ञ्म् । \newline
43. य॒ज्ञ् म॑यजन्ता यजन्त य॒ज्ञ्ं ॅय॒ज्ञ् म॑यजन्त । \newline
44. अ॒य॒ज॒न्त॒ दे॒वा दे॒वा अ॑यजन्ता यजन्त दे॒वाः । \newline
45. दे॒वा स्तानि॒ तानि॑ दे॒वा दे॒वा स्तानि॑ । \newline
46. तानि॒ धर्मा॑णि॒ धर्मा॑णि॒ तानि॒ तानि॒ धर्मा॑णि । \newline
47. धर्मा॑णि प्रथ॒मानि॑ प्रथ॒मानि॒ धर्मा॑णि॒ धर्मा॑णि प्रथ॒मानि॑ । \newline
48. प्र॒थ॒मा न्या॑सन् नासन् प्रथ॒मानि॑ प्रथ॒मा न्या॑सन्न् । \newline
49. आ॒स॒न्नित्या॑सन्न् । \newline
50. ते ह॑ ह॒ ते ते ह॑ । \newline
51. ह॒ नाक॒म् नाकꣳ॑ ह ह॒ नाक᳚म् । \newline
52. नाक॑म् महि॒मानो॑ महि॒मानो॒ नाक॒म् नाक॑म् महि॒मानः॑ । \newline
53. म॒हि॒मानः॑ सचन्ते सचन्ते महि॒मानो॑ महि॒मानः॑ सचन्ते । \newline
54. स॒च॒न्ते॒ यत्र॒ यत्र॑ सचन्ते सचन्ते॒ यत्र॑ । \newline
55. यत्र॒ पूर्वे॒ पूर्वे॒ यत्र॒ यत्र॒ पूर्वे᳚ । \newline
56. पूर्वे॑ सा॒द्ध्याः सा॒द्ध्याः पूर्वे॒ पूर्वे॑ सा॒द्ध्याः । \newline
57. सा॒द्ध्याः सन्ति॒ सन्ति॑ सा॒द्ध्याः सा॒द्ध्याः सन्ति॑ । \newline
58. सन्ति॑ दे॒वा दे॒वाः सन्ति॒ सन्ति॑ दे॒वाः । \newline
59. दे॒वा इति॑ दे॒वाः । \newline

\textbf{Ghana Paata } \newline

1. जा॒तम् जा॒तवे॑दसि जा॒तवे॑दसि जा॒तम् जा॒तम् जा॒तवे॑दसि प्रि॒यम् प्रि॒यम् जा॒तवे॑दसि जा॒तम् जा॒तम् जा॒तवे॑दसि प्रि॒यम् । \newline
2. जा॒तवे॑दसि प्रि॒यम् प्रि॒यम् जा॒तवे॑दसि जा॒तवे॑दसि प्रि॒यꣳ शि॑शीत शिशीत प्रि॒यम् जा॒तवे॑दसि जा॒तवे॑दसि प्रि॒यꣳ शि॑शीत । \newline
3. जा॒तवे॑द॒सीति॑ जा॒त - वे॒द॒सि॒ । \newline
4. प्रि॒यꣳ शि॑शीत शिशीत प्रि॒यम् प्रि॒यꣳ शि॑शी॒ता ति॑थि॒ मति॑थिꣳ शिशीत प्रि॒यम् प्रि॒यꣳ शि॑शी॒ता ति॑थिम् । \newline
5. शि॒शी॒ता ति॑थि॒ मति॑थिꣳ शिशीत शिशी॒ता ति॑थिम् । \newline
6. अति॑थि॒मित्यति॑थिम् । \newline
7. स्यो॒न आ स्यो॒ने स्यो॒न आ गृ॒हप॑तिम् गृ॒हप॑ति॒ मा स्यो॒ने स्यो॒न आ गृ॒हप॑तिम् । \newline
8. आ गृ॒हप॑तिम् गृ॒हप॑ति॒ मा गृ॒हप॑तिम् । \newline
9. गृ॒हप॑ति॒मिति॑ गृ॒ह - प॒ति॒म् । \newline
10. अ॒ग्निना॒ ऽग्नि र॒ग्नि र॒ग्निना॒ ऽग्निना॒ ऽग्निः सꣳ स म॒ग्नि र॒ग्निना॒ ऽग्निना॒ ऽग्निः सम् । \newline
11. अ॒ग्निः सꣳ स म॒ग्नि र॒ग्निः स मि॑द्ध्यत इद्ध्यते॒ स म॒ग्नि र॒ग्निः स मि॑द्ध्यते । \newline
12. स मि॑द्ध्यत इद्ध्यते॒ सꣳ स मि॑द्ध्यते क॒विः क॒वि रि॑द्ध्यते॒ सꣳ स मि॑द्ध्यते क॒विः । \newline
13. इ॒द्ध्य॒ते॒ क॒विः क॒वि रि॑द्ध्यत इद्ध्यते क॒विर् गृ॒हप॑तिर् गृ॒हप॑तिः क॒वि रि॑द्ध्यत इद्ध्यते क॒विर् गृ॒हप॑तिः । \newline
14. क॒विर् गृ॒हप॑तिर् गृ॒हप॑तिः क॒विः क॒विर् गृ॒हप॑ति॒र् युवा॒ युवा॑ गृ॒हप॑तिः क॒विः क॒विर् गृ॒हप॑ति॒र् युवा᳚ । \newline
15. गृ॒हप॑ति॒र् युवा॒ युवा॑ गृ॒हप॑तिर् गृ॒हप॑ति॒र् युवा᳚ । \newline
16. गृ॒हप॑ति॒रिति॑ गृ॒ह - प॒तिः॒ । \newline
17. युवेति॒ युवा᳚ । \newline
18. ह॒व्य॒वाड् जु॒ह्वा᳚स्यो जु॒ह्वा᳚स्यो हव्य॒वा ड्ढ॑व्य॒वाड् जु॒ह्वा᳚स्यः । \newline
19. ह॒व्य॒वाडिति॑ हव्य - वाट् । \newline
20. जु॒ह्वा᳚स्य॒ इति॑ जु॒हु - आ॒स्यः॒ । \newline
21. त्वꣳ हि हि त्वम् त्वꣳ ह्य॑ग्ने अग्ने॒ हि त्वम् त्वꣳ ह्य॑ग्ने । \newline
22. ह्य॑ग्ने अग्ने॒ हि ह्य॑ग्ने अ॒ग्निना॒ ऽग्निना᳚ ऽग्ने॒ हि ह्य॑ग्ने अ॒ग्निना᳚ । \newline
23. अ॒ग्ने॒ अ॒ग्निना॒ ऽग्निना᳚ ऽग्ने अग्ने अ॒ग्निना॒ विप्रो॒ विप्रो॑ अ॒ग्निना᳚ ऽग्ने अग्ने अ॒ग्निना॒ विप्रः॑ । \newline
24. अ॒ग्निना॒ विप्रो॒ विप्रो॑ अ॒ग्निना॒ ऽग्निना॒ विप्रो॒ विप्रे॑ण॒ विप्रे॑ण॒ विप्रो॑ अ॒ग्निना॒ ऽग्निना॒ विप्रो॒ विप्रे॑ण । \newline
25. विप्रो॒ विप्रे॑ण॒ विप्रे॑ण॒ विप्रो॒ विप्रो॒ विप्रे॑ण॒ सन् थ्सन्. विप्रे॑ण॒ विप्रो॒ विप्रो॒ विप्रे॑ण॒ सन्न् । \newline
26. विप्रे॑ण॒ सन् थ्सन्. विप्रे॑ण॒ विप्रे॑ण॒ सन् थ्स॒ता स॒ता सन्. विप्रे॑ण॒ विप्रे॑ण॒ सन् थ्स॒ता । \newline
27. सन् थ्स॒ता स॒ता सन् थ्सन् थ्स॒ता । \newline
28. स॒तेति॑ स॒ता । \newline
29. सखा॒ सख्या॒ सख्या॒ सखा॒ सखा॒ सख्या॑ समि॒द्ध्यसे॑ समि॒द्ध्यसे॒ सख्या॒ सखा॒ सखा॒ सख्या॑ समि॒द्ध्यसे᳚ । \newline
30. सख्या॑ समि॒द्ध्यसे॑ समि॒द्ध्यसे॒ सख्या॒ सख्या॑ समि॒द्ध्यसे᳚ । \newline
31. स॒मि॒द्ध्यस॒ इति॑ सं - इ॒ध्यसे᳚ । \newline
32. तम् म॑र्जयन्त मर्जयन्त॒ तम् तम् म॑र्जयन्त सु॒क्रतुꣳ॑ सु॒क्रतु॑म् मर्जयन्त॒ तम् तम् म॑र्जयन्त सु॒क्रतु᳚म् । \newline
33. म॒र्ज॒य॒न्त॒ सु॒क्रतुꣳ॑ सु॒क्रतु॑म् मर्जयन्त मर्जयन्त सु॒क्रतु॑म् पुरो॒यावा॑नम् पुरो॒यावा॑नꣳ सु॒क्रतु॑म् मर्जयन्त मर्जयन्त सु॒क्रतु॑म् पुरो॒यावा॑नम् । \newline
34. सु॒क्रतु॑म् पुरो॒यावा॑नम् पुरो॒यावा॑नꣳ सु॒क्रतुꣳ॑ सु॒क्रतु॑म् पुरो॒यावा॑न मा॒जि ष्वा॒जिषु॑ पुरो॒यावा॑नꣳ सु॒क्रतुꣳ॑ सु॒क्रतु॑म् पुरो॒यावा॑न मा॒जिषु॑ । \newline
35. सु॒क्रतु॒मिति॑ सु - क्रतु᳚म् । \newline
36. पु॒रो॒यावा॑न मा॒जि ष्वा॒जिषु॑ पुरो॒यावा॑नम् पुरो॒यावा॑न मा॒जिषु॑ । \newline
37. पु॒रो॒यावा॑न॒मिति॑ पुरः - यावा॑नम् । \newline
38. आ॒जिष्वित्या॒जिषु॑ । \newline
39. स्वेषु॒ क्षये॑षु॒ क्षये॑षु॒ स्वेषु॒ स्वेषु॒ क्षये॑षु वा॒जिनं॑ ॅवा॒जिन॒म् क्षये॑षु॒ स्वेषु॒ स्वेषु॒ क्षये॑षु वा॒जिन᳚म् । \newline
40. क्षये॑षु वा॒जिनं॑ ॅवा॒जिन॒म् क्षये॑षु॒ क्षये॑षु वा॒जिन᳚म् । \newline
41. वा॒जिन॒मिति॑ वा॒जिन᳚म् । \newline
42. य॒ज्ञेन॑ य॒ज्ञ्ं ॅय॒ज्ञ्ं ॅय॒ज्ञेन॑ य॒ज्ञेन॑ य॒ज्ञ् म॑यजन्ता यजन्त य॒ज्ञ्ं ॅय॒ज्ञेन॑ य॒ज्ञेन॑ य॒ज्ञ् म॑यजन्त । \newline
43. य॒ज्ञ् म॑यजन्ता यजन्त य॒ज्ञ्ं ॅय॒ज्ञ् म॑यजन्त दे॒वा दे॒वा अ॑यजन्त य॒ज्ञ्ं ॅय॒ज्ञ् म॑यजन्त दे॒वाः । \newline
44. अ॒य॒ज॒न्त॒ दे॒वा दे॒वा अ॑यजन्ता यजन्त दे॒वा स्तानि॒ तानि॑ दे॒वा अ॑यजन्ता यजन्त दे॒वा स्तानि॑ । \newline
45. दे॒वा स्तानि॒ तानि॑ दे॒वा दे॒वा स्तानि॒ धर्मा॑णि॒ धर्मा॑णि॒ तानि॑ दे॒वा दे॒वा स्तानि॒ धर्मा॑णि । \newline
46. तानि॒ धर्मा॑णि॒ धर्मा॑णि॒ तानि॒ तानि॒ धर्मा॑णि प्रथ॒मानि॑ प्रथ॒मानि॒ धर्मा॑णि॒ तानि॒ तानि॒ धर्मा॑णि प्रथ॒मानि॑ । \newline
47. धर्मा॑णि प्रथ॒मानि॑ प्रथ॒मानि॒ धर्मा॑णि॒ धर्मा॑णि प्रथ॒मा न्या॑सन् नासन् प्रथ॒मानि॒ धर्मा॑णि॒ धर्मा॑णि प्रथ॒मा न्या॑सन्न् । \newline
48. प्र॒थ॒मा न्या॑सन् नासन् प्रथ॒मानि॑ प्रथ॒मा न्या॑सन्न् । \newline
49. आ॒स॒न्नित्या॑सन्न् । \newline
50. ते ह॑ ह॒ ते ते ह॒ नाक॒म् नाकꣳ॑ ह॒ ते ते ह॒ नाक᳚म् । \newline
51. ह॒ नाक॒म् नाकꣳ॑ ह ह॒ नाक॑म् महि॒मानो॑ महि॒मानो॒ नाकꣳ॑ ह ह॒ नाक॑म् महि॒मानः॑ । \newline
52. नाक॑म् महि॒मानो॑ महि॒मानो॒ नाक॒म् नाक॑म् महि॒मानः॑ सचन्ते सचन्ते महि॒मानो॒ नाक॒म् नाक॑म् महि॒मानः॑ सचन्ते । \newline
53. म॒हि॒मानः॑ सचन्ते सचन्ते महि॒मानो॑ महि॒मानः॑ सचन्ते॒ यत्र॒ यत्र॑ सचन्ते महि॒मानो॑ महि॒मानः॑ सचन्ते॒ यत्र॑ । \newline
54. स॒च॒न्ते॒ यत्र॒ यत्र॑ सचन्ते सचन्ते॒ यत्र॒ पूर्वे॒ पूर्वे॒ यत्र॑ सचन्ते सचन्ते॒ यत्र॒ पूर्वे᳚ । \newline
55. यत्र॒ पूर्वे॒ पूर्वे॒ यत्र॒ यत्र॒ पूर्वे॑ सा॒द्ध्याः सा॒द्ध्याः पूर्वे॒ यत्र॒ यत्र॒ पूर्वे॑ सा॒द्ध्याः । \newline
56. पूर्वे॑ सा॒द्ध्याः सा॒द्ध्याः पूर्वे॒ पूर्वे॑ सा॒द्ध्याः सन्ति॒ सन्ति॑ सा॒द्ध्याः पूर्वे॒ पूर्वे॑ सा॒द्ध्याः सन्ति॑ । \newline
57. सा॒द्ध्याः सन्ति॒ सन्ति॑ सा॒द्ध्याः सा॒द्ध्याः सन्ति॑ दे॒वा दे॒वाः सन्ति॑ सा॒द्ध्याः सा॒द्ध्याः सन्ति॑ दे॒वाः । \newline
58. सन्ति॑ दे॒वा दे॒वाः सन्ति॒ सन्ति॑ दे॒वाः । \newline
59. दे॒वा इति॑ दे॒वाः । \newline
\pagebreak


\end{document}