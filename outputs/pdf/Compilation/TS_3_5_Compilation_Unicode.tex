\documentclass[17pt]{extarticle}
\usepackage{babel}
\usepackage{fontspec}
\usepackage{polyglossia}
\usepackage{extsizes}



\setmainlanguage{sanskrit}
\setotherlanguages{english} %% or other languages
\setlength{\parindent}{0pt}
\pagestyle{myheadings}
\newfontfamily\devanagarifont[Script=Devanagari]{AdishilaVedic}


\newcommand{\VAR}[1]{}
\newcommand{\BLOCK}[1]{}




\begin{document}
\begin{titlepage}
    \begin{center}
 
\begin{sanskrit}
    { \Huge
    कृष्ण यजुर्वेदीय तैत्तिरीय संहिता,पद,जटा,घन पाठः 
    }
    \\
    \vspace{2.5cm}
    \mbox{ \Huge
    3.5      तृतीयकाण्डे पञ्चमः प्रश्नः - इष्टिशेषाभिधानं   }
\end{sanskrit}
\end{center}

\end{titlepage}
\tableofcontents
\pagebreak

\markright{ TS 3.5.1.1  \hfill https://www.vedavms.in \hfill}
\addcontentsline{toc}{section}{ TS 3.5.1.1 }
\section*{ TS 3.5.1.1 }

\textbf{TS 3.5.1.1 } \newline
\textbf{Samhita Paata} \newline

पू॒र्णा प॒श्चादु॒त पू॒र्णा पु॒रस्ता॒दुन्-म॑द्ध्य॒तः पौ᳚र्णमा॒सी जि॑गाय । तस्या᳚न् दे॒वा अधि॑ सं॒ॅवस॑न्त उत्त॒मे नाक॑ इ॒ह मा॑दयन्तां ॥ यत्ते॑ दे॒वा अद॑धु र्भाग॒धेय॒ममा॑वास्ये सं॒ॅवस॑न्तो महि॒त्वा । सानो॑ य॒ज्ञ्ं पि॑पृहि विश्ववारे र॒यिं नो॑ धेहि सुभगे सु॒वीरं᳚ ॥नि॒वेश॑नी स॒ङ्गम॑नी॒ वसू॑नां॒ ॅविश्वा॑ रू॒पाणि॒ वसू᳚न्यावे॒शय॑न्ती । स॒ह॒स्र॒पो॒षꣳ सु॒भगा॒ ररा॑णा॒ सा न॒ आग॒न्. वर्च॑सा - [  ] \newline

\textbf{Pada Paata} \newline

पू॒र्णा । प॒श्चात् । उ॒त । पू॒र्णा । पु॒रस्ता᳚त् । उदिति॑ । म॒द्ध्य॒तः । पौ॒र्ण॒मा॒सीति॑ पौर्ण - मा॒सी । जि॒गा॒य॒ ॥ तस्या᳚म् । दे॒वाः । अधीति॑ । सं॒ॅवस॑न्त॒ इति॑ सं - वस॑न्तः । उ॒त्त॒म इत्यु॑त् - त॒मे । नाके᳚ । इ॒ह । मा॒द॒य॒न्ता॒म् ॥ यत् । ते॒ । दे॒वाः । अद॑धुः । भा॒ग॒धेय॒मिति॑ भाग - धेय᳚म् । अमा॑वास्य॒ इत्यमा᳚ - वा॒स्ये॒ । सं॒ॅवस॑न्त॒ इति॑ सं - वस॑न्तः । म॒हि॒त्वेति॑ महि-त्वा ॥ सा । नः॒ । य॒ज्ञ्म् । पि॒पृ॒हि॒ । वि॒श्व॒वा॒र॒ इति॑ विश्व - वा॒रे॒ । र॒यिम् । नः॒ । धे॒हि॒ । सु॒भ॒ग॒ इति॑ सु - भ॒गे॒ । सु॒वीर॒मिति॑ सु - वीर᳚म् ॥ नि॒वेश॒नीति॑ नि - वेश॑नी । स॒गंम॒नीति॑ सं - गम॑नी । वसू॑नाम् । विश्वा᳚ । रू॒पाणि॑ । वसू॑नि । आ॒वे॒शय॒न्तीत्या᳚ - वे॒शय॑न्ती ॥ स॒ह॒स्र॒पो॒षमिति॑ सहस्र - पो॒षम् । सु॒भगेति॑ सु-भगा᳚ । ररा॑णा । सा । नः॒ । एति॑ । ग॒न्न् । वर्च॑सा ।  \newline




\markright{ TS 3.5.1.2  \hfill https://www.vedavms.in \hfill}
\addcontentsline{toc}{section}{ TS 3.5.1.2 }
\section*{ TS 3.5.1.2 }

\textbf{TS 3.5.1.2 } \newline
\textbf{Samhita Paata} \newline

संॅविदा॒ना ॥ अग्नी॑षोमौ प्रथ॒मौ वी॒र्ये॑ण॒ वसू᳚न् रु॒द्राना॑दि॒त्यानि॒ह जि॑न्वतं ।मा॒द्ध्यꣳ हि पौ᳚र्णमा॒सं जु॒षेथां॒ ब्रह्म॑णा वृ॒द्धौ सु॑कृ॒तेन॑ सा॒तावथा॒ऽस्मभ्यꣳ॑ स॒हवी॑राꣳ र॒यिं नि य॑च्छतं ॥ आ॒दि॒त्याश्चाऽङ्गि॑रसश्चा॒ग्नीनाऽद॑धत॒ ते द॑र्.शपूर्णमा॒सौ प्रैफ्स॒न् तेषा॒मङ्गि॑रसां॒ निरु॑प्तꣳ ह॒विरासी॒दथा॑ऽऽ*दि॒त्या ए॒तौ होमा॑वपश्य॒न् ताव॑जुहवु॒स्ततो॒ वै ते द॑र्.शपूर्णमा॒सौ - [  ] \newline

\textbf{Pada Paata} \newline

सं॒ॅवि॒दा॒नेति॑ सं - वि॒दा॒ना ॥ अग्नी॑षोमा॒वित्यग्नी᳚ - सो॒मौ॒ । प्र॒थ॒मौ । वी॒र्ये॑ण । वसून्॑ । रु॒द्रान् । आ॒दि॒त्यान् । इ॒ह । जि॒न्व॒त॒म् ॥ मा॒द्ध्यम् । हि । पौ॒र्ण॒मा॒समिति॑ पौर्ण - मा॒सम् । जु॒षेथा᳚म् । ब्रह्म॑णा । वृ॒द्धौ । सु॒कृ॒तेनेति॑ सु - कृ॒तेन॑ । सा॒तौ । अथ॑ । अ॒स्मभ्य॒मित्य॒स्म-भ्य॒म् । स॒हवी॑रा॒मिति॑ स॒ह-वी॒रा॒म् । र॒यिम् । नीति॑ । य॒च्छ॒त॒म् ॥ आ॒दि॒त्याः । च॒ । अङ्गि॑रसः । च॒ । अ॒ग्नीन् । एति॑ । अ॒द॒ध॒त॒ । ते । द॒र्॒.श॒पू॒र्ण॒मा॒साविति॑ दर्.श - पू॒र्ण॒मा॒सौ । प्रेति॑ । ऐ॒फ्स॒न्न् । तेषा᳚म् । अङ्गि॑रसाम् । निरु॑प्त॒मिति॒ निः - उ॒प्त॒म् । ह॒विः । आसी᳚त् । अथ॑ । आ॒दि॒त्याः । ए॒तौ । होमौ᳚ । अ॒प॒श्य॒न्न् । तौ । अ॒जु॒ह॒वुः॒ । ततः॑ । वै । ते । द॒र्॒.श॒पू॒र्ण॒मा॒साविति॑ दर्.श - पू॒र्ण॒मा॒सौ ।  \newline




\markright{ TS 3.5.1.3  \hfill https://www.vedavms.in \hfill}
\addcontentsline{toc}{section}{ TS 3.5.1.3 }
\section*{ TS 3.5.1.3 }

\textbf{TS 3.5.1.3 } \newline
\textbf{Samhita Paata} \newline

पूर्व॒ आ ऽल॑भन्त दर्.शपूर्णमा॒सा-वा॒लभ॑मान ए॒तौ होमौ॑ पु॒रस्ता᳚ज्जुहुयाथ् सा॒क्षादे॒व द॑र्.शपूर्णमा॒सावा ल॑भते ब्रह्मवा॒दिनो॑ वदन्ति॒ स त्वै द॑र्.शपूर्णमा॒सावा ल॑भेत॒ य ए॑नयोरनु-लो॒मञ्च॑ प्रतिलो॒मञ्च॑ वि॒द्यादित्य॑मावा॒स्या॑या ऊ॒र्द्ध्वं तद॑नुलो॒मं पौ᳚र्णमा॒स्यै प्र॑ती॒चीनं॒ तत् प्र॑तिलो॒मं ॅयत् पौ᳚र्णमा॒सीं पूर्वा॑मा॒लभे॑त प्रतिलो॒ममे॑ना॒वा ल॑भेता॒-मुम॑प॒क्षीय॑माण॒-मन्वप॑ - [  ] \newline

\textbf{Pada Paata} \newline

पूर्वे᳚ । एति॑ । अ॒ल॒भ॒न्त॒ । द॒र्॒.श॒पू॒र्ण॒मा॒साविति॑ दर्.श - पू॒र्ण॒मा॒सौ । आ॒लभ॑मान॒ इत्या᳚-लभ॑मानः । ए॒तौ । होमौ᳚ । पु॒रस्ता᳚त् । जु॒हु॒या॒त् । सा॒क्षादिति॑ स - अ॒क्षात् । ए॒व । द॒र्॒.श॒पू॒र्ण॒मा॒साविति॑ दर्.श - पू॒र्ण॒मा॒सौ । एति॑ । ल॒भे॒ते॒ । ब्र॒ह्म॒वा॒दिन॒ इति॑ ब्रह्म - वा॒दिनः॑ । व॒द॒न्ति॒ । सः । तु । वै । द॒र्॒.श॒पू॒र्ण॒मा॒साविति॑ दर्.श - पू॒र्ण॒मा॒सौ । एति॑ । ल॒भे॒त॒ । यः । ए॒न॒योः॒ । अ॒नु॒लो॒ममित्य॑नु - लो॒मम् । च॒ । प्र॒ति॒लो॒ममिति॑ प्रति - लो॒मम् । च॒ । वि॒द्यात् । इति॑ । अ॒मा॒वा॒स्या॑या॒ इत्य॑मा - वा॒स्या॑याः । ऊ॒र्द्ध्वम् । तत् । अ॒नु॒लो॒ममित्य॑नु - लो॒मम् । पौ॒र्ण॒मा॒स्या इति॑ पौर्ण - मा॒स्यै । प्र॒ती॒चीन᳚म् । तत् । प्र॒ति॒लो॒ममिति॑ प्रति - लो॒मम् । यत् । पौ॒र्ण॒मा॒सीमिति॑ पौर्ण - मा॒सीम् । पूर्वा᳚म् । आ॒लभे॒तेत्या᳚ - लभे॑त । प्र॒ति॒लो॒ममिति॑ प्रति-लो॒मम् । ए॒नौ॒ । एति॑ । ल॒भे॒त॒ । अ॒मुम् । अ॒प॒क्षीय॑माण॒मित्य॑प - क्षीय॑माणम् । अनु॑ । अपेति॑ ।  \newline




\markright{ TS 3.5.1.4  \hfill https://www.vedavms.in \hfill}
\addcontentsline{toc}{section}{ TS 3.5.1.4 }
\section*{ TS 3.5.1.4 }

\textbf{TS 3.5.1.4 } \newline
\textbf{Samhita Paata} \newline

क्षीयेत सारस्व॒तौ होमौ॑ पु॒रस्ता᳚ज्जुहुयादमावा॒स्या॑ वै सर॑स्वत्यनुलो॒म-मे॒वैना॒वा ल॑भते॒ ऽमुमा॒प्याय॑मान॒मन्वा प्या॑यत आग्नावैष्ण॒व-मेका॑दशकपालं पु॒रस्ता॒न्निव॑र्पे॒थ् सर॑स्वत्यै च॒रुꣳ सर॑स्वते॒ द्वाद॑शकपालं॒ ॅयदा᳚ग्ने॒यो भव॑त्य॒ग्निर्वै य॑ज्ञ्मु॒खं ॅय॑ज्ञ्मु॒खमे॒वर्द्धिं॑ पु॒रस्ता᳚द्-धत्ते॒ यद्-वै᳚ष्ण॒वो भव॑ति य॒ज्ञो वै विष्णु॑र्य॒ज्ञ्मे॒वाऽऽ*रभ्य॒ प्रत॑नुते॒ सर॑स्वत्यै ( ) च॒रुर्भ॑वति॒ सर॑स्वते॒ द्वाद॑शकपालोऽमावा॒स्या॑ वै सर॑स्वती पू॒र्णमा॑सः॒ सर॑स्वा॒न् तावे॒व सा॒क्षादा र॑भत ऋ॒द्ध्नोत्या᳚भ्यां॒ द्वाद॑शकपालः॒ सर॑स्वते भवति मिथुन॒त्वाय॒ प्रजा᳚त्यै मिथु॒नौ गावौ॒ दक्षि॑णा॒ समृ॑द्ध्यै ॥ \newline

\textbf{Pada Paata} \newline

क्षी॒ये॒ते॒ । सा॒र॒स्व॒तौ । होमौ᳚ । पु॒रस्ता᳚त् । जु॒हु॒या॒त् । अ॒मा॒वा॒स्येत्य॑मा-वा॒स्या᳚ । वै । सर॑स्वती । अ॒नु॒लो॒ममित्य॑नु-लो॒मम् । ए॒व । ए॒नौ॒ । एति॑ । ल॒भे॒ते॒ । अ॒मुम् । आ॒प्याय॑मान॒मित्या᳚ - प्याय॑मानम् । अनु॑ । एति॑ । प्या॒य॒ते॒ । आ॒ग्ना॒वै॒ष्ण॒वमित्या᳚ग्ना - वै॒ष्ण॒वम् । एका॑दशकपाल॒मित्येका॑दश-क॒पा॒ल॒म् । पु॒रस्ता᳚त् । निरिति॑ । व॒पे॒त् । सर॑स्वत्यै । च॒रुम् । सर॑स्वते । द्वाद॑शकपाल॒मिति॒ द्वाद॑श-क॒पा॒ल॒म् । यत् । आ॒ग्ने॒यः । भव॑ति । अ॒ग्निः । वै । य॒ज्ञ्॒मु॒खमिति॑ यज्ञ्-मु॒खम् । य॒ज्ञ्॒मु॒खमिति॑ यज्ञ्-मु॒खम् । ए॒व । ऋद्धि᳚म् । पु॒रस्ता᳚त् । ध॒त्ते॒ । यत् । वै॒ष्ण॒वः । भव॑ति । य॒ज्ञ्ः । वै । विष्णुः॑ । य॒ज्ञ्म् । ए॒व । आ॒रभ्येत्या᳚ - रभ्य॑ । प्रेति॑ । त॒नु॒ते॒ । सर॑स्वत्यै ( ) । च॒रुः । भ॒व॒ति॒ । सर॑स्वते । द्वाद॑शकपाल॒ इति॒ द्वाद॑श-क॒पा॒लः॒ । अ॒मा॒वा॒स्येत्य॑मा - वा॒स्या᳚ । वै । सर॑स्वती । पू॒र्णमा॑स॒ इति॑ पू॒र्ण - मा॒सः॒ । सर॑स्वान् । तौ । ए॒व । सा॒क्षादिति॑ स - अ॒क्षात् । एति॑ । र॒भ॒ते॒ । ऋ॒द्ध्नोति॑ । आ॒भ्या॒म् । द्वाद॑शकपाल॒ इति॒ द्वाद॑श - क॒पा॒लः॒ । सर॑स्वते । भ॒व॒ति॒ । मि॒थु॒न॒त्वायेति॑ मिथुन - त्वाय॑ । प्रजा᳚त्या॒ इति॒ प्र - जा॒त्यै॒ । मि॒थु॒नौ । गावौ᳚ । दक्षि॑णा । समृ॑द्ध्या॒ इति॒ सं - ऋ॒द्ध्यै॒ ॥  \newline




\markright{ TS 3.5.2.1  \hfill https://www.vedavms.in \hfill}
\addcontentsline{toc}{section}{ TS 3.5.2.1 }
\section*{ TS 3.5.2.1 }

\textbf{TS 3.5.2.1 } \newline
\textbf{Samhita Paata} \newline

ऋष॑यो॒ वा इन्द्रं॑ प्र॒त्यक्षं॒ नाप॑श्य॒न् तं ॅवसि॑ष्ठः प्र॒त्यक्ष॑मपश्य॒थ् सो᳚ऽब्रवी॒द्-ब्राह्म॑णं ते वक्ष्यामि॒ यथा॒ त्वत्पु॑रोहिताः प्र॒जाः प्र॑जनि॒ष्यन्तेऽथ॒ मेत॑रेभ्य॒ ऋषि॑भ्यो॒ मा प्रवो॑च॒ इति॒ तस्मा॑ ए॒तान्थ्स्तोम॑-भागानब्रवी॒त् ततो॒ वसि॑ष्ठपुरोहिताः प्र॒जाः प्राजा॑यन्त॒ तस्मा᳚द्-वासि॒ष्ठो ब्र॒ह्मा का॒र्यः॑ प्रैव जा॑यते र॒श्मिर॑सि॒ क्षया॑य त्वा॒ क्षयं॑ जि॒न्वे - [  ] \newline

\textbf{Pada Paata} \newline

ऋष॑यः । वै । इन्द्र᳚म् । प्र॒त्यक्ष॒मिति॑ प्रति - अक्ष᳚म् । न । अ॒प॒श्य॒न्न् । तम् । वसि॑ष्ठः । प्र॒त्यक्ष॒मिति॑ प्रति - अक्ष᳚म् । अ॒प॒श्य॒त् । सः । अ॒ब्र॒वी॒त् । ब्राह्म॑णम् । ते॒ । व॒क्ष्या॒मि॒ । यथा᳚ । त्वत्पु॑रोहिता॒ इति॒ त्वत् - पु॒रो॒हि॒ताः॒ । प्र॒जा इति॑ प्र - जाः । प्र॒ज॒नि॒ष्यन्त॒ इति॑ प्र - ज॒नि॒ष्यन्ते᳚ । अथ॑ । मा॒ । इत॑रेभ्यः । ऋषि॑भ्य॒ इत्यृषि॑ - भ्यः॒ । मा । प्रेति॑ । वो॒चः॒ । इति॑ । तस्मै᳚ । ए॒तान् । स्तोम॑भागा॒निति॒ स्तोम॑ - भा॒गा॒न् । अ॒ब्र॒वी॒त् । ततः॑ । वसि॑ष्ठपुरोहिता॒ इति॒ वसि॑ष्ठ - पु॒रो॒हि॒ताः॒ । प्र॒जा इति॑ प्र - जाः । प्रेति॑ । अ॒जा॒य॒न्त॒ । तस्मा᳚त् । वा॒सि॒ष्ठः । ब्र॒ह्मा । का॒र्यः॑ । प्रेति॑ । ए॒व । जा॒य॒ते॒ । र॒श्मिः । अ॒सि॒ । क्षया॑य । त्वा॒ । क्षय᳚म् । जि॒न्व॒ । इति॑ ।  \newline




\markright{ TS 3.5.2.2  \hfill https://www.vedavms.in \hfill}
\addcontentsline{toc}{section}{ TS 3.5.2.2 }
\section*{ TS 3.5.2.2 }

\textbf{TS 3.5.2.2 } \newline
\textbf{Samhita Paata} \newline

-त्या॑ह दे॒वा वै क्षयो॑ दे॒वेभ्य॑ ए॒व य॒ज्ञ्ं प्राऽऽ*ह॒ प्रेति॑रसि॒ धर्मा॑य त्वा॒ धर्मं॑ जि॒न्वेत्या॑ह मनु॒ष्या॑ वै धर्मो॑ मनु॒ष्ये᳚भ्य ए॒व य॒ज्ञ्ं प्राऽऽ*हान्वि॑तिरसि दि॒वे त्वा॒ दिवं॑ जि॒न्वेत्या॑है॒भ्य ए॒व लो॒केभ्यो॑ य॒ज्ञ्ं प्राऽऽ*ह॑विष्ट॒म्भो॑ऽसि॒ वृष्ट्यै᳚ त्वा॒ वृष्टिं॑ जि॒न्वेत्या॑ह॒ वृष्टि॑मे॒वाव॑ - [  ] \newline

\textbf{Pada Paata} \newline

आ॒ह॒ । दे॒वाः । वै । क्षयः॑ । दे॒वेभ्यः॑ । ए॒व । य॒ज्ञ्म् । प्रेति॑ । आ॒ह॒ । प्रेति॒रिति॒ प्र - इ॒तिः॒ । अ॒सि॒ । धर्मा॑य । त्वा॒ । धर्म᳚म् । जि॒न्व॒ । इति॑ । आ॒ह॒ । म॒नु॒ष्याः᳚ । वै । धर्मः॑ । म॒नु॒ष्ये᳚भ्यः । ए॒व । य॒ज्ञ्म् । प्रेति॑ । आ॒ह॒ । अन्वि॑ति॒रित्यनु॑-इ॒तिः॒ । अ॒सि॒ । दि॒वे । त्वा॒ । दिव᳚म् । जि॒न्व॒ । इति॑ । आ॒ह॒ । ए॒भ्यः । ए॒व । लो॒केभ्यः॑ । य॒ज्ञ्म् । प्रेति॑ । आ॒ह॒ । वि॒ष्ट॒भं इति॑ वि - स्त॒भंः । अ॒सि॒ । वृष्ट्यै᳚ । त्वा॒ । वृष्टि᳚म् । जि॒न्व॒ । इति॑ । आ॒ह॒ । वृष्टि᳚म् । ए॒व । अवेति॑ ।  \newline




\markright{ TS 3.5.2.3  \hfill https://www.vedavms.in \hfill}
\addcontentsline{toc}{section}{ TS 3.5.2.3 }
\section*{ TS 3.5.2.3 }

\textbf{TS 3.5.2.3 } \newline
\textbf{Samhita Paata} \newline

रुन्धे प्र॒वाऽस्य॑नु॒वाऽसीत्या॑ह मिथुन॒त्वायो॒शिग॑सि॒ वसु॑भ्यस्त्वा॒ वसू᳚ञ्जि॒न्वेत्या॑हा॒ष्टौ वस॑व॒ एका॑दश रु॒द्रा द्वाद॑शाऽऽ*दि॒त्या ए॒ताव॑न्तो॒ वै दे॒वास्तेभ्य॑ ए॒व य॒ज्ञ्ं प्राऽऽ*हौजो॑ऽसि पि॒तृभ्य॑स्त्वा पि॒तॄन् जि॒न्वेत्या॑ह दे॒वाने॒व पि॒तॄननु॒ सन्त॑नोति॒ तन्तु॑रसि प्र॒जाभ्य॑स्त्वा प्र॒जा जि॒न्वे - [  ] \newline

\textbf{Pada Paata} \newline

रु॒न्धे॒ । प्र॒वेति॑ प्र - वा । अ॒सि॒ । अ॒नु॒वेत्य॑नु - वा । अ॒सि॒ । इति॑ । आ॒ह॒ । मि॒थु॒न॒त्वायेति॑ मिथुन - त्वाय॑ । उ॒शिक् । अ॒सि॒ । वसु॑भ्य॒ इति॒ वसु॑ - भ्यः॒ । त्वा॒ । वसून्॑ । जि॒न्व॒ । इति॑ । आ॒ह॒ । अ॒ष्टौ । वस॑वः । एका॑दश । रु॒द्राः । द्वाद॑श । आ॒दि॒त्याः । ए॒ताव॑न्तः । वै । दे॒वाः । तेभ्यः॑ । ए॒व । य॒ज्ञ्म् । प्रेति॑ । आ॒ह॒ । ओजः॑ । अ॒सि॒ । पि॒तृभ्य॒ इति॑ पि॒तृ - भ्यः॒ । त्वा॒ । पि॒तॄन् । जि॒न्व॒ । इति॑ । आ॒ह॒ । दे॒वान् । ए॒व । पि॒तॄन् । अनु॑ । समिति॑ । त॒नो॒ति॒ । तन्तुः॑ । अ॒सि॒ । प्र॒जाभ्य॒ इति॑ प्र - जाभ्यः॑ । त्वा॒ । प्र॒जा इति॑ प्र - जाः । जि॒न्व॒ ।  \newline




\markright{ TS 3.5.2.4  \hfill https://www.vedavms.in \hfill}
\addcontentsline{toc}{section}{ TS 3.5.2.4 }
\section*{ TS 3.5.2.4 }

\textbf{TS 3.5.2.4 } \newline
\textbf{Samhita Paata} \newline

-त्या॑ह पि॒तॄने॒व प्र॒जा अनु॒ सन्त॑नोति पृतना॒षाड॑सि प॒शुभ्य॑स्त्वा प॒शूञ्जि॒न्वेत्या॑ह प्र॒जा ए॒व प॒शूननु॒ सन्त॑नोतिरे॒वद॒स्यो-ष॑धीभ्य॒ स्त्वौष॑धी-र्जि॒न्वेत्या॒हौष॑धीष्वे॒व प॒शून् प्रति॑ष्ठापयत्यभि॒जिद॑सि यु॒क्तग्रा॒वेन्द्रा॑य॒ त्वेन्द्रं॑ जि॒न्वेत्या॑हा॒भिजि॑त्या॒ अधि॑पतिरसि प्रा॒णाय॑ त्वा प्रा॒णं - [  ] \newline

\textbf{Pada Paata} \newline

इति॑ । आ॒ह॒ । पि॒तॄन् । ए॒व । प्र॒जा इति॑ प्र - जाः । अनु॑ । समिति॑ । त॒नो॒ति॒ । पृ॒त॒ना॒षाट् । अ॒सि॒ । प॒शुभ्य॒ इति॑ प॒शु - भ्यः॒ । त्वा॒ । प॒शून् । जि॒न्व॒ । इति॑ । आ॒ह॒ । प्र॒जा इति॑ प्र - जाः । ए॒व । प॒शून् । अनु॑ । समिति॑ । त॒नो॒ति॒ । रे॒वत् । अ॒सि॒ । ओष॑धीभ्य॒ इत्योष॑धि - भ्यः॒ । त्वा॒ । ओष॑धीः । जि॒न्व॒ । इति॑ । आ॒ह॒ । ओष॑धीषु । ए॒व । प॒शून् । प्रतीति॑ । स्था॒प॒य॒ति॒ । अ॒भि॒जिदित्य॑भि - जित् । अ॒सि॒ । यु॒क्तग्रा॒वेति॑ यु॒क्त - ग्रा॒वा॒ । इन्द्रा॑य । त्वा॒ । इन्द्र᳚म् । जि॒न्व॒ । इति॑ । आ॒ह॒ । अ॒भिजि॑त्या॒ इत्य॒भि - जि॒त्यै॒ । अधि॑पति॒रित्यधि॑ - प॒तिः॒ । अ॒सि॒ । प्रा॒णायेति॑ प्र - अ॒नाय॑ । त्वा॒ । प्रा॒णमिति॑ प्र-अ॒नम् ।  \newline




\markright{ TS 3.5.2.5  \hfill https://www.vedavms.in \hfill}
\addcontentsline{toc}{section}{ TS 3.5.2.5 }
\section*{ TS 3.5.2.5 }

\textbf{TS 3.5.2.5 } \newline
\textbf{Samhita Paata} \newline

जि॒न्वेत्या॑ह प्र॒जास्वे॒व प्रा॒णान् द॑धाति त्रि॒वृद॑सि प्र॒वृद॒सीत्या॑ह मिथुन॒त्वाय॑ सꣳरो॒हो॑ऽसि नीरो॒हो॑ऽसीत्या॑ह॒ प्रजा᳚त्यै वसु॒को॑ऽसि॒ वेष॑श्रिरसि॒ वस्य॑ष्टिर॒सीत्या॑ह॒ प्रति॑ष्ठित्यै ॥ \newline

\textbf{Pada Paata} \newline

जि॒न्व॒ । इति॑ । आ॒ह॒ । प्र॒जास्विति॑ प्र - जासु॑ । ए॒व । प्रा॒णानिति॑ प्र - अ॒नान् । द॒धा॒ति॒ । त्रि॒वृदिति॑ त्रि - वृत् । अ॒सि॒ । प्र॒वृदिति॑ प्र - वृत् । अ॒सि॒ । इति॑ । आ॒ह॒ । मि॒थु॒न॒त्वायेति॑ मिथुन - त्वाय॑ । सꣳ॒॒रो॒ह इति॑ सं - रो॒हः । अ॒सि॒ । नी॒रो॒ह इति॑ निः - रो॒हः । अ॒सि॒ । इति॑ । आ॒ह॒ । प्रजा᳚त्या॒ इति॒ प्र - जा॒त्यै॒ । व॒सु॒कः । अ॒सि॒ । वेष॑श्रि॒रिति॒ वेष॑- श्रिः॒ । अ॒सि॒ । वस्य॑ष्टिः । अ॒सि॒ । इति॑ । आ॒ह॒ । प्रति॑ष्ठित्या॒ इति॒ प्रति॑ - स्थि॒त्यै॒ ॥  \newline




\markright{ TS 3.5.3.1  \hfill https://www.vedavms.in \hfill}
\addcontentsline{toc}{section}{ TS 3.5.3.1 }
\section*{ TS 3.5.3.1 }

\textbf{TS 3.5.3.1 } \newline
\textbf{Samhita Paata} \newline

अ॒ग्निना॑ दे॒वेन॒ पृत॑ना जयामि गाय॒त्रेण॒ छन्द॑सा त्रि॒वृता॒ स्तोमे॑न रथन्त॒रेण॒ साम्ना॑ वषट्का॒रेण॒ वज्रे॑ण पूर्व॒जान् भ्रातृ॑व्या॒नध॑रान् पादया॒म्यवै॑नान् बाधे॒ प्रत्ये॑नान्नुदे॒ऽस्मिन् क्षये॒ऽस्मिन् भू॑मिलो॒के यो᳚ऽस्मान् द्वेष्टि॒ यञ्च॑ व॒यं द्वि॒ष्मो विष्णोः॒ क्रमे॒णाऽत्ये॑नान् क्रामा॒मीन्द्रे॑ण दे॒वेन॒ पृत॑ना जयामि॒ त्रैष्टु॑भेन॒ छन्द॑सा पञ्चद॒शेन॒ स्तोमे॑न बृह॒ता साम्ना॑ वषट्का॒रेण॒ वज्रे॑ण - [  ] \newline

\textbf{Pada Paata} \newline

अ॒ग्निना᳚ । दे॒वेन॑ । पृत॑नाः । ज॒या॒मि॒ । गा॒य॒त्रेण॑ । छन्द॑सा । त्रि॒वृतेति॑ त्रि - वृता᳚ । स्तोमे॑न । र॒थ॒न्त॒रेणेति॑ रथं - त॒रेण॑ । साम्ना᳚ । व॒ष॒ट्का॒रेणेति॑ वषट् - का॒रेण॑ । वज्रे॑ण । पू॒र्व॒जानिति॑ पूर्व - जान् । भ्रातृ॑व्यान् । अध॑रान् । पा॒द॒या॒मि॒ । अवेति॑ । ए॒ना॒न् । बा॒धे॒ । प्रतीति॑ । ए॒ना॒न् । नु॒दे॒ । अ॒स्मिन्न् । क्षये᳚ । अ॒स्मिन्न् । भू॒मि॒लो॒क इति॑ भूमि - लो॒के । यः । अ॒स्मान् । द्वेष्टि॑ । यम् । च॒ । व॒यम् । द्वि॒ष्मः । विष्णोः᳚ । क्रमे॑ण । अतीति॑ । ए॒ना॒न् । क्रा॒मा॒मि॒ । इन्द्रे॑ण । दे॒वेन॑ । पृत॑नाः । ज॒या॒मि॒ । त्रैष्टु॑भेन । छन्द॑सा । प॒ञ्च॒द॒शेनेति॑ पञ्च - द॒शेन॑ । स्तोमे॑न । बृ॒ह॒ता । साम्ना᳚ । व॒ष॒ट्का॒रेणेति॑ वषट् - का॒रेण॑ । वज्रे॑ण ।  \newline




\markright{ TS 3.5.3.2  \hfill https://www.vedavms.in \hfill}
\addcontentsline{toc}{section}{ TS 3.5.3.2 }
\section*{ TS 3.5.3.2 }

\textbf{TS 3.5.3.2 } \newline
\textbf{Samhita Paata} \newline

सह॒जान्. विश्वे॑भिर्दे॒वेभिः॒ पृत॑ना जयामि॒ जाग॑तेन॒ छन्द॑सा सप्तद॒शेन॒ स्तोमे॑न वामदे॒व्येन॒ साम्ना॑ वषट्का॒रेण॒ वज्रे॑णा पर॒जानिन्द्रे॑ण स॒युजो॑ व॒यꣳ सा॑स॒ह्याम॑ पृतन्य॒तः । घ्नन्तो॑ वृ॒त्राण्य॑प्र॒ति । यत्ते॑ अग्ने॒ तेज॒स्तेना॒हं ते॑ज॒स्वी भू॑यासं॒ ॅयत्ते॑ अग्ने॒ वर्च॒स्तेना॒हं ॅव॑र्च॒स्वी भू॑यासं॒ ॅयत्ते॑ अग्ने॒ हर॒स्तेना॒हꣳ ह॑र॒स्वी भू॑यासं ॥ \newline

\textbf{Pada Paata} \newline

स॒ह॒जानिति॑ सह - जान् । विश्वे॑भिः । दे॒वेभिः॑ । पृत॑नाः । ज॒या॒मि॒ । जाग॑तेन । छन्द॑सा । स॒प्त॒द॒शेनेति॑ सप्त - द॒शेन॑ । स्तोमे॑न । वा॒म॒दे॒व्येनेति॑ वाम-दे॒व्येन॑ । साम्ना᳚ । व॒ष॒ट्का॒रेणेति॑ वषट् - का॒रेण॑ । वज्रे॑ण । अ॒प॒र॒जानित्य॑पर - जान् । इन्द्रे॑ण । स॒युज॒ इति॑ स-युजः॑ । व॒यम् । सा॒स॒ह्याम॑ । पृ॒त॒न्य॒तः ॥ घ्नन्तः॑ । वृ॒त्राणि॑ । अ॒प्र॒ति ॥ यत् । ते॒ । अ॒ग्ने॒ । तेजः॑ । तेन॑ । अ॒हम् । ते॒ज॒स्वी । भू॒या॒स॒म् । यत् । ते॒ । अ॒ग्ने॒ । वर्चः॑ । तेन॑ । अ॒हम् । व॒र्च॒स्वी । भू॒या॒स॒म् । यत् । ते॒ । अ॒ग्ने॒ । हरः॑ । तेन॑ । अ॒हम् । ह॒र॒स्वी । भू॒या॒स॒म् ॥  \newline




\markright{ TS 3.5.4.1  \hfill https://www.vedavms.in \hfill}
\addcontentsline{toc}{section}{ TS 3.5.4.1 }
\section*{ TS 3.5.4.1 }

\textbf{TS 3.5.4.1 } \newline
\textbf{Samhita Paata} \newline

ये दे॒वा य॑ज्ञ्॒हनो॑ यज्ञ्॒मुषः॑ पृथि॒व्यामद्ध्यास॑ते ।अ॒ग्निर्मा॒ तेभ्यो॑ रक्षतु॒ गच्छे॑म सु॒कृतो॑ व॒यं ॥ आऽग॑न्म मित्रावरुणा वरेण्या॒ रात्री॑णां भा॒गो यु॒वयो॒र्यो अस्ति॑ । नाकं॑ गृह्णा॒नाः सु॑कृ॒तस्य॑ लो॒के तृ॒तीये॑ पृ॒ष्ठे अधि॑ रोच॒ने दि॒वः ॥ये दे॒वा य॑ज्ञ्॒हनो॑ यज्ञ्॒मुषो॒ऽन्तरि॒क्षेऽद्ध्यास॑ते । वा॒युर्मा॒ तेभ्यो॑ रक्षतु॒ गच्छे॑म सु॒कृतो॑ व॒यं ॥ यास्ते॒ रात्रीः᳚ सवित - [  ] \newline

\textbf{Pada Paata} \newline

ये । दे॒वाः । य॒ज्ञ्॒हन॒ इति॑ यज्ञ् - हनः॑ । य॒ज्ञ्॒मुष॒ इति॑ यज्ञ् - मुषः॑ । पृ॒थि॒व्याम् । अधीति॑ । आस॑ते ॥ अ॒ग्निः । मा॒ । तेभ्यः॑ । र॒क्ष॒तु॒ । गच्छे॑म । सु॒कृत॒ इति॑ सु - कृतः॑ । व॒यम् ॥ एति॑ । अ॒ग॒न्म॒ । मि॒त्रा॒व॒रु॒णेति॑ मित्रा-व॒रु॒णा॒ । व॒रे॒ण्या॒ । रात्री॑णाम् । भा॒गः । यु॒वयोः᳚ । यः । अस्ति॑ ॥ नाक᳚म् । गृ॒ह्णा॒नाः । सु॒कृ॒तस्येति॑ सु - कृ॒तस्य॑ । लो॒के । तृ॒तीये᳚ । पृ॒ष्ठे । अधीति॑ । रो॒च॒ने । दि॒वः ॥ ये । दे॒वाः । य॒ज्ञ्॒हन॒ इति॑ यज्ञ् - हनः॑ । य॒ज्ञ्॒मुष॒ इति॑ यज्ञ् - मुषः॑ । अ॒न्तरि॑क्षे । अधीति॑ । आस॑ते ॥ वा॒युः । मा॒ । तेभ्यः॑ । र॒क्ष॒तु॒ । गच्छे॑म । सु॒कृत॒ इति॑ सु - कृतः॑ । व॒यम् ॥ याः । ते॒ । रात्रीः᳚ । स॒वि॒तः॒ ।  \newline




\markright{ TS 3.5.4.2  \hfill https://www.vedavms.in \hfill}
\addcontentsline{toc}{section}{ TS 3.5.4.2 }
\section*{ TS 3.5.4.2 }

\textbf{TS 3.5.4.2 } \newline
\textbf{Samhita Paata} \newline

-र्देव॒यानी॑रन्त॒रा द्यावा॑पृथि॒वी वि॒यन्ति॑ । गृ॒हैश्च॒ सर्वैः᳚ प्र॒जया॒ न्वग्रे॒ सुवो॒ रुहा॑णास्तरता॒ रजाꣳ॑सि ॥ ये दे॒वा य॑ज्ञ्॒हनो॑ यज्ञ्॒मुषो॑ दि॒व्यद्ध्यास॑ते । सूर्यो॑ मा॒ तेभ्यो॑ रक्षतु॒ गच्छे॑म सु॒कृतो॑ व॒यं ॥ येनेन्द्रा॑य स॒मभ॑रः॒ पयाꣳ॑स्युत्त॒मेन॑ ह॒विषा॑ जातवेदः । तेना᳚ऽग्ने॒ त्वमु॒त व॑र्द्धये॒मꣳ स॑जा॒तानाꣳ॒॒ श्रैष्ठ्य॒ आ धे᳚ह्येनं ॥ य॒ज्ञ्॒हनो॒ वै दे॒वा य॑ज्ञ्॒मुषः॑ - [  ] \newline

\textbf{Pada Paata} \newline

दे॒व॒यानी॒रिति॑ देव - यानीः᳚ । अ॒न्त॒रा । द्यावा॑पृथि॒वी इति॒ द्यावा᳚ - पृ॒थि॒वी । वि॒यन्तीति॑ वि - यन्ति॑ ॥ गृ॒हैः । च॒ । सर्वैः᳚ । प्र॒जयेति॑ प्र - जया᳚ । नु । अग्रे᳚ । सुवः॑ । रुहा॑णाः । त॒र॒त॒ । रजाꣳ॑सि ॥ ये । दे॒वाः । य॒ज्ञ्॒हन॒ इति॑ यज्ञ् - हनः॑ । य॒ज्ञ्॒मुष॒ इति॑ यज्ञ् - मुषः॑ । दि॒वि । अधीति॑ । आस॑ते ॥ सूर्यः॑ । मा॒ । तेभ्यः॑ । र॒क्ष॒तु॒ । गच्छे॑म । सु॒कृत॒ इति॑ सु - कृतः॑ । व॒यम् ॥ येन॑ । इन्द्रा॑य । स॒मभ॑र॒ इति॑ सं - अभ॑रः । पयाꣳ॑सि । उ॒त्त॒मेनेत्यु॑त् - त॒मेन॑ । ह॒विषा᳚ । जा॒त॒वे॒द॒ इति॑ जात - वे॒दः॒ ॥ तेन॑ । अ॒ग्ने॒ । त्वम् । उ॒त । व॒र्द्ध॒य॒ । इ॒मम् । स॒जा॒ताना॒मिति॑ स - जा॒ताना᳚म् । श्रैष्ठ्ये᳚ । एति॑ । धे॒हि॒ । ए॒न॒म् ॥ य॒ज्ञ्॒हन॒ इति॑ यज्ञ् - हनः॑ । वै । दे॒वाः । य॒ज्ञ्॒मुष॒ इति॑ यज्ञ् - मुषः॑ ।  \newline




\markright{ TS 3.5.4.3  \hfill https://www.vedavms.in \hfill}
\addcontentsline{toc}{section}{ TS 3.5.4.3 }
\section*{ TS 3.5.4.3 }

\textbf{TS 3.5.4.3 } \newline
\textbf{Samhita Paata} \newline

सन्ति॒ त ए॒षु लो॒केष्वा॑सत आ॒ददा॑ना विमथ्ना॒ना यो ददा॑ति॒ यो यज॑ते॒ तस्य॑ । ये दे॒वा य॑ज्ञ्॒हनः॑ पृथि॒व्यामद्ध्यास॑ते॒ ये अ॒न्तरि॑क्षे॒ ये दि॒वीत्या॑हे॒माने॒व लो॒काꣳस्ती॒र्त्वा सगृ॑हः॒ सप॑शुः सुव॒र्गं ॅलो॒कमे॒त्यप॒ वै सोमे॑नेजा॒नाद्दे॒वता᳚श्च य॒ज्ञ्श्च॑ क्रामन्त्याग्ने॒यं पञ्च॑कपालमुदवसा॒नीयं॒ निर्व॑पेद॒ग्निः सर्वा॑ दे॒वताः॒ - [  ] \newline

\textbf{Pada Paata} \newline

स॒न्ति॒ । ते । ए॒षु । लो॒केषु॑ । आ॒स॒ते॒ । आ॒ददा॑ना॒ इत्या᳚ - ददा॑नाः । वि॒म॒थ्ना॒ना इति॑ वि-म॒थ्ना॒नाः । यः । ददा॑ति । यः । यज॑ते । तस्य॑ ॥ ये । दे॒वाः । य॒ज्ञ्॒हन॒ इति॑ यज्ञ् - हनः॑ । पृ॒थि॒व्याम् । अधीति॑ । आस॑ते । ये । अ॒न्तरि॑क्षे । ये । दि॒वि । इति॑ । आ॒ह॒ । इ॒मान् । ए॒व । लो॒कान् । ती॒र्त्वा । सगृ॑ह॒ इति॒ स - गृ॒हः॒ । सप॑शु॒रिति॒ स - प॒शुः॒ । सु॒व॒र्गमिति॑ सुवः - गम् । लो॒कम् । ए॒ति॒ । अपेति॑ । वै । सोमे॑न । ई॒जा॒नात् । दे॒वताः᳚ । च॒ । य॒ज्ञ्ः । च॒ । क्रा॒म॒न्ति॒ । आ॒ग्ने॒यम् । पञ्च॑कपाल॒मिति॒ पञ्च॑ - क॒पा॒ल॒म् । उ॒द॒व॒सा॒नीय॒मित्यु॑त् - अ॒व॒सा॒नीय᳚म् । निरिति॑ । व॒पे॒त् । अ॒ग्निः । सर्वाः᳚ । दे॒वताः᳚ ।  \newline




\markright{ TS 3.5.4.4  \hfill https://www.vedavms.in \hfill}
\addcontentsline{toc}{section}{ TS 3.5.4.4 }
\section*{ TS 3.5.4.4 }

\textbf{TS 3.5.4.4 } \newline
\textbf{Samhita Paata} \newline

पाङ्क्तो॑ य॒ज्ञो दे॒वता᳚श्चै॒व य॒ज्ञ्ञ्चाव॑ रुन्धेगाय॒त्रो वा अ॒ग्निर्गा॑य॒त्र छ॑न्दा॒स्तं छन्द॑सा॒ व्य॑र्द्धयति॒ यत् पञ्च॑कपालं क॒रोत्य॒ष्टाक॑पालः का॒र्यो᳚ऽष्टाक्ष॑रा गाय॒त्री गा॑य॒त्रो᳚ऽग्नि-र्गा॑य॒त्र छ॑न्दाः॒ स्वेनै॒वैनं॒ छन्द॑सा॒ सम॑र्द्धयति प॒ङ्क्त्यौ॑ याज्यानुवा॒क्ये॑ भवतः॒ पाङ्क्तो॑ य॒ज्ञ्स्तेनै॒व य॒ज्ञान्नैति॑ ॥ \newline

\textbf{Pada Paata} \newline

पाङ्क्तः॑ । य॒ज्ञ्ः । दे॒वताः᳚ । च॒ । ए॒व । य॒ज्ञ्म् । च॒ । अवेति॑ । रु॒न्धे॒ । गा॒य॒त्रः । वै । अ॒ग्निः । गा॒य॒त्रछ॑न्दा॒ इति॑ गाय॒त्र - छ॒न्दाः॒ । तम् । छन्द॑सा । वीति॑ । अ॒र्द्ध॒य॒ति॒ । यत् । पञ्च॑कपाल॒मिति॒ पञ्च॑ - क॒पा॒ल॒म् । क॒रोति॑ । अ॒ष्टाक॑पाल॒ इत्य॒ष्टा - क॒पा॒लः॒ । का॒र्यः॑ । अ॒ष्टाक्ष॒रेत्य॒ष्टा - अ॒क्ष॒रा॒ । गा॒य॒त्री । गा॒य॒त्रः । अ॒ग्निः । गा॒य॒त्रछ॑न्दा॒ इति॑ गाय॒त्र - छ॒न्दाः॒ । स्वेन॑ । ए॒व । ए॒न॒म् । छन्द॑सा । समिति॑ । अ॒र्द्ध॒य॒ति॒ । प॒ङ्क्त्यौ᳚ । या॒ज्या॒नु॒वा॒क्ये॑ इति॑ याज्या - अ॒नु॒वा॒क्ये᳚ । भ॒व॒तः॒ । पाङ्क्तः॑ । य॒ज्ञ्ः । तेन॑ । ए॒व । य॒ज्ञात् । न । ए॒ति॒ ॥  \newline




\markright{ TS 3.5.5.1  \hfill https://www.vedavms.in \hfill}
\addcontentsline{toc}{section}{ TS 3.5.5.1 }
\section*{ TS 3.5.5.1 }

\textbf{TS 3.5.5.1 } \newline
\textbf{Samhita Paata} \newline

सूर्यो॑ मा दे॒वो दे॒वेभ्यः॑ पातु वा॒युर॒न्तरि॑क्षा॒द्-यज॑मानो॒ऽग्निर्मा॑ पातु॒ चक्षु॑षः । सक्ष॒ शूष॒ सवि॑त॒र्विश्व॑चर्.षण ए॒तेभिः॑ सोम॒ नाम॑भिर्विधेम ते॒ तेभिः॑ सोम॒ नाम॑भिर्विधेम ते ॥ अ॒हं प॒रस्ता॑द॒-हम॒वस्ता॑द॒हं ज्योति॑षा॒ वि तमो॑ ववार । यद॒न्तरि॑क्षं॒ तदु॑ मे पि॒ताऽभू॑द॒हꣳ सूर्य॑मुभ॒यतो॑ ददर्.शा॒हं भू॑या समुत्त॒मः स॑मा॒नाना॒ - [  ] \newline

\textbf{Pada Paata} \newline

सूर्यः॑ । मा॒ । दे॒वः । दे॒वेभ्यः॑ । पा॒तु॒ । वा॒युः । अ॒न्तरि॑क्षात् । यज॑मानः । अ॒ग्निः । मा॒ । पा॒तु॒ । चक्षु॑षः ॥ सक्ष॑ । शूष॑ । सवि॑तः । विश्व॑चर्.षण॒ इति॒ विश्व॑ - च॒र्॒.ष॒णे॒ । ए॒तेभिः॑ । सो॒म॒ । नाम॑भि॒रिति॒ नाम॑ - भिः॒ । वि॒धे॒म॒ । ते॒ । तेभिः॑ । सो॒म॒ । नाम॑भि॒रिति॒ नाम॑-भिः॒ । वि॒धे॒म॒ । ते॒ ॥ अ॒हम् । प॒रस्ता᳚त् । अ॒हम् । अ॒वस्ता᳚त् । अ॒हम् । ज्योति॑षा । वीति॑ । तमः॑ । व॒वा॒र॒ ॥ यत् । अ॒न्तरि॑क्षम् । तत् । उ॒ । मे॒ । पि॒ता । अ॒भू॒त् । अ॒हम् । सूर्य᳚म् । उ॒भ॒यतः॑ । द॒द॒र्॒.श॒ । अ॒हम् । भू॒या॒स॒म् । उ॒त्त॒म इत्यु॑त् - त॒मः । स॒मा॒नाना᳚म् ।  \newline




\markright{ TS 3.5.5.2  \hfill https://www.vedavms.in \hfill}
\addcontentsline{toc}{section}{ TS 3.5.5.2 }
\section*{ TS 3.5.5.2 }

\textbf{TS 3.5.5.2 } \newline
\textbf{Samhita Paata} \newline

मा स॑मु॒द्रा-दाऽन्तरि॑क्षात्-प्र॒जाप॑तिरुद॒धिं च्या॑वया॒तीन्द्रः॒ प्रस्नौ॑तु म॒रुतो॑ वर्.षय॒न्तून्न॑म्भय पृथि॒वीं भि॒न्धीदं दि॒व्यं नभः॑ । उ॒द्रो दि॒व्यस्य॑ नो दे॒हीशा॑नो॒ विसृ॑जा॒ दृतिं᳚ ॥ प॒शवो॒ वा ए॒ते यदा॑दि॒त्य ए॒ष रु॒द्रो यद॒ग्निरोष॑धीः॒ प्रास्या॒ग्नावा॑दि॒त्यं जु॑होति रु॒द्रादे॒व प॒शून॒न्तर्द॑धा॒त्यथो॒ ओष॑धीष्वे॒व प॒शून् - [  ] \newline

\textbf{Pada Paata} \newline

एति॑ । स॒मु॒द्रात् । एति॑ । अ॒न्तरि॑क्षात् । प्र॒जाप॑ति॒रिति॑ प्र॒जा - प॒तिः॒ । उ॒द॒धिमित्यु॑द - धिम् । च्या॒व॒या॒ति॒ । इन्द्रः॑ । प्रेति॑ । स्नौ॒तु॒ । म॒रुतः॑ । व॒र्॒.ष॒य॒न्तु॒ । उदिति॑ । न॒भं॒य॒ । पृ॒थि॒वीम् । भि॒न्धि । इ॒दम् । दि॒व्यम् । नभः॑ ॥ उ॒द्रः । दि॒व्यस्य॑ । नः॒ । दे॒हि॒ । ईशा॑नः । वीति॑ । सृ॒ज॒ । दृति᳚म् ॥ प॒शवः॑ । वै । ए॒ते । यत् । आ॒दि॒त्यः । ए॒षः । रु॒द्रः । यत् । अ॒ग्निः । ओष॑धीः । प्रास्येति॑ प्र - अस्य॑ । अ॒ग्नौ । आ॒दि॒त्यम् । जु॒हो॒ति॒ । रु॒द्रात् । ए॒व । प॒शून् । अ॒न्तः । द॒धा॒ति॒ । अथो॒ इति॑ । ओष॑धीषु । ए॒व । प॒शून् ।  \newline




\markright{ TS 3.5.5.3  \hfill https://www.vedavms.in \hfill}
\addcontentsline{toc}{section}{ TS 3.5.5.3 }
\section*{ TS 3.5.5.3 }

\textbf{TS 3.5.5.3 } \newline
\textbf{Samhita Paata} \newline

प्रति॑ष्ठापयति क॒विर्य॒ज्ञ्स्य॒ वित॑नोति॒ पन्थां॒ नाक॑स्य पृ॒ष्ठे अधि॑ रोच॒ने दि॒वः । येन॑ ह॒व्यं ॅवह॑सि॒ यासि॑ दू॒त इ॒तः प्रचे॑ता अ॒मुतः॒ सनी॑यान् ॥ यास्ते॒ विश्वाः᳚ स॒मिधः॒ सन्त्य॑ग्ने॒याः पृ॑थि॒व्यां ब॒र्॒.हिषि॒ सूर्ये॒ याः । तास्ते॑ गच्छ॒न्त्वाहु॑तिं घृ॒तस्य॑ देवाय॒ते यज॑मानाय॒ शर्म॑ ॥आ॒शासा॑नः सु॒वीर्यꣳ॑ रा॒यस्पोषꣳ॒॒ स्वश्वि॑यं । बृह॒स्पति॑ना रा॒या स्व॒गाकृ॑तो॒ मह्यं॒ ॅयज॑मानाय ( ) तिष्ठ ॥ \newline

\textbf{Pada Paata} \newline

प्रतीति॑ । स्था॒प॒य॒ति॒ । क॒विः । य॒ज्ञ्स्य॑ । वीति॑ । त॒नो॒ति॒ । पन्था᳚म् । नाक॑स्य । पृ॒ष्ठे । अधीति॑ । रो॒च॒ने । दि॒वः ॥ येन॑ । ह॒व्यम् । वह॑सि । यासि॑ । दू॒तः । इ॒तः । प्रचे॑ता॒ इति॒ प्र-चे॒ताः॒ । अ॒मुतः॑ । सनी॑यान् ॥ याः । ते॒ । विश्वाः᳚ । स॒मिध॒ इति॑ सं - इधः॑ । सन्ति॑ । अ॒ग्ने॒ । याः । पृ॒थि॒व्याम् । ब॒र्॒.हिषि॑ । सूर्ये᳚ । याः ॥ ताः । ते॒ । ग॒च्छ॒न्तु॒ । आहु॑ति॒मित्या - हु॒ति॒म् । घृ॒तस्य॑ । दे॒वा॒य॒त इति॑ देव - य॒ते । यज॑मानाय । शर्म॑ ॥ आ॒शासा॑न॒ इत्या᳚ - शासा॑नः । सु॒वीर्य॒मिति॑ सु - वीर्य᳚म् । रा॒यः । पोष᳚म् । स्वश्वि॑य॒मिति॑ सु - अश्वि॑यम् ॥ बृह॒स्पति॑ना । रा॒या । स्व॒गाकृ॑त॒ इति॑ स्व॒गा - कृ॒तः॒ । मह्य᳚म् । यज॑मानाय ( ) । ति॒ष्ठ॒ ॥  \newline




\markright{ TS 3.5.6.1  \hfill https://www.vedavms.in \hfill}
\addcontentsline{toc}{section}{ TS 3.5.6.1 }
\section*{ TS 3.5.6.1 }

\textbf{TS 3.5.6.1 } \newline
\textbf{Samhita Paata} \newline

सं त्वा॑ नह्यामि॒ पय॑सा घृ॒तेन॒ सं त्वा॑ नह्याम्य॒प ओष॑धीभिः । सं त्वा॑ नह्यामि प्र॒जया॒ऽहम॒द्य सा दी᳚क्षि॒ता स॑नवो॒ वाज॑म॒स्मे ॥ प्रैतु॒ ब्रह्म॑ण॒स्पत्नी॒ वेदिं॒ ॅवर्णे॑न सीदतु । अथा॒हम॑नुका॒मिनी॒ स्वे लो॒के वि॒शा इ॒ह ॥ सु॒प्र॒जस॑स्त्वा व॒यꣳ सु॒पत्नी॒रुप॑ सेदिम । अग्ने॑ सपत्न॒दम्भ॑न॒मद॑ब्धासो॒ अदा᳚भ्यं ॥ इ॒मं ॅविष्या॑मि॒ वरु॑णस्य॒ पाशं॒ - [  ] \newline

\textbf{Pada Paata} \newline

समिति॑ । त्वा॒ । न॒ह्या॒मि॒ । पय॑सा । घृ॒तेन॑ । समिति॑ । त्वा॒ । न॒ह्या॒मि॒ । अ॒पः । ओष॑धीभि॒रित्योष॑धि - भिः॒ ॥ समिति॑ । त्वा॒ । न॒ह्या॒मि॒ । प्र॒जयेति॑ प्र - जया᳚ । अ॒हम् । अ॒द्य । सा । दी॒क्षि॒ता । स॒न॒वः॒ । वाज᳚म् । अ॒स्मे इति॑ ॥ प्रेति॑ । ए॒तु॒ । ब्रह्म॑णः । पत्नी᳚ । वेदि᳚म् । वर्णे॑न । सी॒द॒तु॒ ॥ अथ॑ । अ॒हम् । अ॒नु॒का॒मिनीत्य॑नु-का॒मिनी᳚ । स्वे । लो॒के । वि॒शै । इ॒ह ॥ सु॒प्र॒जस॒ इति॑ सु - प्र॒जसः॑ । त्वा॒ । व॒यम् । सु॒पत्नी॒रिति॑ सु - पत्नीः᳚ । उपेति॑ । से॒दि॒म॒ ॥ अग्ने᳚ । स॒प॒त्न॒दंभ॑न॒मिति॑ सपत्न - दंभ॑नम् । अद॑ब्धासः । अदा᳚भ्यम् ॥ इ॒मम् । वीति॑ । स्या॒मि॒ । वरु॑णस्य । पाश᳚म् ।  \newline




\markright{ TS 3.5.6.2  \hfill https://www.vedavms.in \hfill}
\addcontentsline{toc}{section}{ TS 3.5.6.2 }
\section*{ TS 3.5.6.2 }

\textbf{TS 3.5.6.2 } \newline
\textbf{Samhita Paata} \newline

ॅयमब॑द्ध्नीत सवि॒ता सु॒केतः॑ । धा॒तुश्च॒ योनौ॑ सुकृ॒तस्य॑ लो॒के स्यो॒नं मे॑ स॒ह पत्या॑ करोमि ॥प्रेह्यु॒देह्यृ॒तस्य॑ वा॒मीरन्व॒ग्निस्तेऽग्रं॑ नय॒त्वदि॑ति॒र्मद्ध्यं॑ ददताꣳ रु॒द्राव॑सृष्टाऽसि यु॒वा नाम॒ मा मा॑ हिꣳसी॒र्वसु॑भ्यो रु॒द्रेभ्य॑ आदि॒त्येभ्यो॒ विश्वे᳚भ्यो वो दे॒वेभ्यः॑ प॒न्नेज॑नीर्गृह्णामि य॒ज्ञाय॑ वः प॒न्नेज॑नीः सादयामि॒ विश्व॑स्य ते॒ विश्वा॑वतो॒ वृष्णि॑यावत॒ - [  ] \newline

\textbf{Pada Paata} \newline

यम् । अब॑द्ध्नीत । स॒वि॒ता । सु॒केत॒ इति॑ सु-केतः॑ ॥ धा॒तुः । च॒ । योनौ᳚ । सु॒कृ॒तस्येति॑ सु - कृ॒तस्य॑ । लो॒के । स्यो॒नम् । मे॒ । स॒ह । पत्या᳚ । क॒रो॒मि॒ ॥ प्रेति॑ । इ॒हि॒ । उ॒देहीत्यु॑त्-एहि॑ । ऋ॒तस्य॑ । वा॒मीः । अन्विति॑ । अ॒ग्निः । ते॒ । अग्र᳚म् । न॒य॒तु॒ । अदि॑तिः । मद्ध्य᳚म् । द॒द॒ता॒म् । रु॒द्राव॑सृ॒ष्टेति॑ रु॒द्र-अ॒व॒सृ॒ष्टा॒ । अ॒सि॒ । यु॒वा । नाम॑ । मा । मा॒ । हिꣳ॒॒सीः॒ । वसु॑भ्य॒ इति॒ वसु॑-भ्यः॒ । रु॒द्रेभ्यः॑ । आ॒दि॒त्येभ्यः॑ । विश्वे᳚भ्यः । वः॒ । दे॒वेभ्यः॑ । प॒न्नेज॑नी॒रिति॑ पत् - नेज॑नीः । गृ॒ह्णा॒मि॒ । य॒ज्ञाय॑ । वः॒ । प॒न्नेज॑नी॒रिति॑ पत् - नेज॑नीः । सा॒द॒या॒मि॒ । विश्व॑स्य । ते॒ । विश्वा॑वत॒ इति॒ विश्व॑-व॒तः॒ । वृष्णि॑यावत॒ इति॒ वृष्णि॑य-व॒तः॒ ।  \newline




\markright{ TS 3.5.6.3  \hfill https://www.vedavms.in \hfill}
\addcontentsline{toc}{section}{ TS 3.5.6.3 }
\section*{ TS 3.5.6.3 }

\textbf{TS 3.5.6.3 } \newline
\textbf{Samhita Paata} \newline

स्तवा᳚ग्ने वा॒मीरनु॑ स॒दृंशि॒ विश्वा॒ रेताꣳ॑सि धिषी॒याऽग॑न् दे॒वान्. य॒ज्ञो नि दे॒वीर्दे॒वेभ्यो॑ य॒ज्ञ्म॑शिषन्न॒स्मिन्थ् सु॑न्व॒ति यज॑मान आ॒शिषः॒ स्वाहा॑कृताः समुद्रे॒ष्ठा ग॑न्ध॒र्वमाति॑ष्ठ॒तानु॑ । वात॑स्य॒ पत्म॑न्नि॒ड ई॑डि॒ताः ॥ \newline

\textbf{Pada Paata} \newline

तव॑ । अ॒ग्ने॒ । वा॒मीः । अन्विति॑ । स॒दृंशीति॑ सं - दृशि॑ । विश्वा᳚ । रेताꣳ॑सि । धि॒षी॒य॒ । अगन्न्॑ । दे॒वान् । य॒ज्ञ्ः । नीति॑ । दे॒वीः । दे॒वेभ्यः॑ । य॒ज्ञ्म् । अ॒शि॒ष॒न्न् । अ॒स्मिन्न् । सु॒न्व॒ति । यज॑माने । आ॒शिष॒ इत्या᳚ - शिषः॑ । स्वाहा॑कृता॒ इति॒ स्वाहा᳚ - कृ॒ताः॒ । स॒मु॒द्रे॒ष्ठा इति॑ समुद्रे - स्थाः । ग॒न्ध॒र्वम् । एति॑ । ति॒ष्ठ॒त॒ । अनु॑ ॥ वात॑स्य । पत्मन्न्॑ । इ॒डः । ई॒डि॒ताः ॥  \newline




\markright{ TS 3.5.7.1  \hfill https://www.vedavms.in \hfill}
\addcontentsline{toc}{section}{ TS 3.5.7.1 }
\section*{ TS 3.5.7.1 }

\textbf{TS 3.5.7.1 } \newline
\textbf{Samhita Paata} \newline

व॒ष॒ट्का॒रो वै गा॑यत्रि॒यै शिरो᳚ऽछिन॒त् तस्यै॒ रसः॒ परा॑ऽपत॒थ् स पृ॑थि॒वीं प्रावि॑श॒थ्स ख॑दि॒रो॑ऽभव॒द्यस्य॑ खादि॒रः स्रु॒वो भव॑ति॒ छन्द॑सामे॒व रसे॒नाव॑ द्यति॒ सर॑सा अ॒स्याऽऽ*हु॑तयो भवन्ति तृ॒तीय॑स्यामि॒तो दि॒वि सोम॑ आसी॒त् तं गाय॒त्र्याऽ ह॑र॒त् तस्य॑ प॒र्णम॑च्छिद्यत॒ तत् प॒र्णो॑ऽभव॒त् तत् प॒र्णस्य॑ पर्ण॒त्वं ॅयस्य॑ पर्ण॒मयी॑ जु॒हू - [  ] \newline

\textbf{Pada Paata} \newline

व॒ष॒ट्का॒र इति॑ वषट् - का॒रः । वै । गा॒य॒त्रि॒यै । शिरः॑ । अ॒च्छि॒न॒त् । तस्यै᳚ । रसः॑ । परेति॑ । अ॒प॒त॒त् । सः । पृ॒थि॒वीम् । प्रेति॑ । अ॒वि॒श॒त् । सः । ख॒दि॒रः । अ॒भ॒व॒त् । यस्य॑ । खा॒दि॒रः । स्रु॒वः । भव॑ति । छन्द॑साम् । ए॒व । रसे॑न । अवेति॑ । द्य॒ति॒ । सर॑सा॒ इति॒ स - र॒साः॒ । अ॒स्य॒ । आहु॑तय॒ इत्या - हु॒त॒यः॒ । भ॒व॒न्ति॒ । तृ॒तीय॑स्याम् । इ॒तः । दि॒वि । सोमः॑ । आ॒सी॒त् । तम् । गा॒य॒त्री । एति॑ । अ॒ह॒र॒त् । तस्य॑ । प॒र्णम् । अ॒च्छि॒द्य॒त॒ । तत् । प॒र्णः । अ॒भ॒व॒त् । तत् । प॒र्णस्य॑ । प॒र्ण॒त्वमिति॑ पर्ण - त्वम् । यस्य॑ । प॒र्ण॒मयीति॑ पर्ण - मयी᳚ । जु॒हूः ।  \newline




\markright{ TS 3.5.7.2  \hfill https://www.vedavms.in \hfill}
\addcontentsline{toc}{section}{ TS 3.5.7.2 }
\section*{ TS 3.5.7.2 }

\textbf{TS 3.5.7.2 } \newline
\textbf{Samhita Paata} \newline

-र्भव॑ति सौ॒म्या अ॒स्याऽऽ*हु॑तयो भवन्ति जु॒षन्ते᳚ऽस्य दे॒वा आहु॑तीर्दे॒वा वै ब्रह्म॑न्नवदन्त॒ तत् प॒र्ण उपा॑ऽ*शृणोथ् सु॒श्रवा॒ वै नाम॒ यस्य॑ पर्ण॒मयी॑ जु॒हूर्भव॑ति॒ न पा॒पꣳ श्लोकꣳ॑ शृणोति॒ ब्रह्म॒ वै प॒र्णो विण्म॒रुतोऽन्नं॒ ॅविण्मा॑रु॒तो᳚ऽश्व॒त्थो यस्य॑ पर्ण॒मयी॑ जु॒हूर्भव॒त्या-श्व॑त्-थ्युप॒भृद्- ब्रह्म॑णै॒वान्न॒मव॑ रु॒न्धेऽथो॒ ब्रह्मै॒ - [  ] \newline

\textbf{Pada Paata} \newline

भव॑ति । सौ॒म्याः । अ॒स्य॒ । आहु॑तय॒ इत्या - हु॒त॒यः॒ । भ॒व॒न्ति॒ । जु॒षन्ते᳚ । अ॒स्य॒ । दे॒वाः । आहु॑ती॒रित्या - हु॒तीः॒ । दे॒वाः । वै । ब्रह्मन्न्॑ । अ॒व॒द॒न्त॒ । तत् । प॒र्णः । उपेति॑ । अ॒शृ॒णो॒त् । सु॒श्रवा॒ इति॑ सु - श्रवाः᳚ । वै । नाम॑ । यस्य॑ । प॒र्ण॒मयीति॑ पर्ण - मयी᳚ । जु॒हूः । भव॑ति । न । पा॒पम् । श्लोक᳚म् । शृ॒णो॒ति॒ । ब्रह्म॑ । वै । प॒र्णः । विट् । म॒रुतः॑ । अन्न᳚म् । विट् । मा॒रु॒तः । अ॒श्व॒त्थः । यस्य॑ । प॒र्ण॒मयीति॑ पर्ण-मयी᳚ । जु॒हूः । भव॑ति । आश्व॑त्थी । उ॒प॒भृतित्यु॑प - भृत् । ब्रह्म॑णा । ए॒व । अन्न᳚म् । अवेति॑ । रु॒न्धे॒ । अथो॒ इति॑ । ब्रह्म॑ ।  \newline




\markright{ TS 3.5.7.3  \hfill https://www.vedavms.in \hfill}
\addcontentsline{toc}{section}{ TS 3.5.7.3 }
\section*{ TS 3.5.7.3 }

\textbf{TS 3.5.7.3 } \newline
\textbf{Samhita Paata} \newline

-व वि॒श्यद्ध्यू॑हति रा॒ष्ट्रं ॅवै प॒र्णो विड॑श्व॒त्थो यत् प॑र्ण॒मयी॑ जु॒हूर्भव॒त्या-श्व॑त्थ्युप॒भृद्-रा॒ष्ट्रमे॒व वि॒श्यद्ध्यू॑हति प्र॒जाप॑ति॒र्वा अ॑जुहो॒थ् सा यत्राऽऽ*हु॑तिः प्र॒त्यति॑ष्ठ॒त् ततो॒ विक॑ङ्कत॒ उद॑तिष्ठ॒त् ततः॑ प्र॒जा अ॑सृजत॒ यस्य॒ वैक॑ङ्कती ध्रु॒वा भव॑ति॒ प्रत्य॒वास्या ऽऽ*हु॑तयस्तिष्ठ॒न्त्यथो॒ प्रैव जा॑यत ए॒तद्वै स्रु॒चाꣳ ( ) रू॒पं ॅयस्यै॒वꣳ रू॑पाः॒ स्रुचो॒ भव॑न्ति॒ सर्वा᳚ण्ये॒वैनꣳ॑ रू॒पाणि॑ पशू॒नामुप॑तिष्ठन्ते॒ नास्याप॑-रूपमा॒त्मञ्जा॑यते ॥ \newline

\textbf{Pada Paata} \newline

ए॒व । वि॒शि । अधीति॑ । ऊ॒ह॒ति॒ । रा॒ष्ट्रम् । वै । प॒र्णः । विट् । अ॒श्व॒त्थः । यत् । प॒र्ण॒मयीति॑ पर्ण - मयी᳚ । जु॒हूः । भव॑ति । आश्व॑त्थी । उ॒प॒भृतित्यु॑प-भृत् । रा॒ष्ट्रम् । ए॒व । वि॒शि । अधीति॑ । ऊ॒ह॒ति॒ । प्र॒जाप॑ति॒रिति॑ प्र॒जा - प॒तिः॒ । वै । अ॒ज॒हो॒त् । सा । यत्र॑ । आहु॑ति॒रित्या - हु॒तः॒ । प्र॒त्यति॑ष्ठ॒दिति॑ प्रति - अति॑ष्ठत् । ततः॑ । विक॑ङ्कत॒ इति॒ वि - क॒ङ्क॒तः॒ । उदिति॑ । अ॒ति॒ष्ठ॒त् । ततः॑ । प्र॒जा इति॑ प्र - जाः । अ॒सृ॒ज॒त॒ । यस्य॑ । वैक॑ङ्कती । ध्रु॒वा । भव॑ति । प्रतीति॑ । ए॒व । अ॒स्य॒ । आहु॑तय॒ इत्या-हु॒त॒यः॒ । ति॒ष्ठ॒न्ति॒ । अथो॒ इति॑ । प्रेति॑ । ए॒व । जा॒य॒ते॒ । ए॒तत् । वै । स्रु॒चाम् ( ) । रू॒पम् । यस्य॑ । ए॒वꣳरू॑पा॒ इत्ये॒वं - रू॒पाः॒ । स्रुचः॑ । भव॑न्ति । सर्वा॑णि । ए॒व । ए॒न॒म् । रू॒पाणि॑ । प॒शू॒नाम् । उपेति॑ । ति॒ष्ठ॒न्ते॒ । न । अ॒स्य॒ । अप॑रूप॒मित्यप॑ - रू॒प॒म् । आ॒त्मन्न् । जा॒य॒ते॒ ॥  \newline




\markright{ TS 3.5.8.1  \hfill https://www.vedavms.in \hfill}
\addcontentsline{toc}{section}{ TS 3.5.8.1 }
\section*{ TS 3.5.8.1 }

\textbf{TS 3.5.8.1 } \newline
\textbf{Samhita Paata} \newline

उ॒प॒या॒मगृ॑हीतोऽसि प्र॒जाप॑तये त्वा॒ ज्योति॑ष्मते॒ ज्योति॑ष्मन्तं गृह्णामि॒ दक्षा॑य दक्ष॒वृधे॑ रा॒तं दे॒वेभ्यो᳚ऽग्नि जि॒ह्वेभ्य॑स्त्वर्ता॒युभ्य॒ इन्द्र॑ज्येष्ठेभ्यो॒ वरु॑णराजभ्यो॒ वाता॑पिभ्यः प॒र्जन्या᳚त्मभ्यो दि॒वे त्वा॒ऽन्तरि॑क्षाय त्वा पृथि॒व्यै त्वाऽपे᳚न्द्र द्विष॒तो मनोऽप॒ जिज्या॑सतो ज॒ह्यप॒ यो नो॑ऽराती॒यति॒ तं ज॑हि प्रा॒णाय॑ त्वाऽपा॒नाय॑ त्वा व्या॒नाय॑ त्वा स॒ते त्वाऽस॑ते त्वा॒ऽद्भ्यस्त्वौष॑धीभ्यो॒ ( ) विश्वे᳚भ्यस्त्वा भू॒तेभ्यो॒ यतः॑ प्र॒जा अक्खि॑द्रा॒ अजा॑यन्त॒ तस्मै᳚ त्वा प्र॒जाप॑तये विभू॒दाव्.न्ने॒ ज्योति॑ष्मते॒ ज्योति॑ष्मन्तं जुहोमि ॥ \newline

\textbf{Pada Paata} \newline

उ॒प॒या॒मगृ॑हीत॒ इत्यु॑पया॒म - गृ॒ही॒तः॒ । अ॒सि॒ । प्र॒जाप॑तय॒ इति॑ प्र॒जा-प॒त॒ये॒ । त्वा॒ । ज्योति॑ष्मते । ज्योति॑ष्मन्तम् । गृ॒ह्णा॒मि॒ । दक्षा॑य । द॒क्ष॒वृध॒ इति॑ दक्ष - वृधे᳚ । रा॒तम् । दे॒वेभ्यः॑ । अ॒ग्नि॒जि॒ह्वेभ्य॒ इत्य॑ग्नि - जि॒ह्वेभ्यः॑ । त्वा॒ । ऋ॒ता॒युभ्य॒ इत्यृ॑ता॒यु - भ्यः॒ । इन्द्र॑ज्येष्ठेभ्य॒ इतीन्द्र॑-ज्ये॒ष्ठे॒भ्यः॒ । वरु॑णराजभ्य॒ इति॒ वरु॑णराज-भ्यः॒ । वाता॑पिभ्य॒ इति॒ वाता॑पि-भ्यः॒ । प॒र्जन्या᳚त्मभ्य॒ इति॑ प॒र्जन्या᳚त्म-भ्यः॒ । दि॒वे । त्वा॒ । अ॒न्तरि॑क्षाय । त्वा॒ । पृ॒थि॒व्यै । त्वा॒ । अपेति॑ । इ॒न्द्र॒ । द्वि॒ष॒तः । मनः॑ । अपेति॑ । जिज्या॑सतः । ज॒हि॒ । अपेति॑ । यः । नः॒ । अ॒रा॒ती॒यति॑ । तम् । ज॒हि॒ । प्रा॒णायेति॑ प्रा - अ॒नाय॑ । त्वा॒ । अ॒पा॒नायेत्य॑प-अ॒नाय॑ । त्वा॒ । व्या॒नायेति॑ वि - अ॒नाय॑ । त्वा॒ । स॒ते । त्वा॒ । अस॑ते । त्वा॒ । अ॒द्भ्य इत्य॑त् - भ्यः । त्वा॒ । ओष॑धीभ्य॒ इत्योष॑धि-भ्यः॒ ( ) । विश्वे᳚भ्यः । त्वा॒ । भू॒तेभ्यः॑ । यतः॑ । प्र॒जा इति॑ प्र - जाः । अक्खि॑द्राः । अजा॑यन्त । तस्मै᳚ । त्वा॒ । प्र॒जाप॑तय॒ इति॑ प्र॒जा - प॒त॒ये॒ । वि॒भू॒दाव्.न्न॒ इति॑ विभु - दाव्.न्ने᳚ । ज्योति॑ष्मते । ज्योति॑ष्मन्तम् । जु॒हो॒मि॒ ॥  \newline




\markright{ TS 3.5.9.1  \hfill https://www.vedavms.in \hfill}
\addcontentsline{toc}{section}{ TS 3.5.9.1 }
\section*{ TS 3.5.9.1 }

\textbf{TS 3.5.9.1 } \newline
\textbf{Samhita Paata} \newline

यां ॅवा अ॑द्ध्व॒र्युश्च॒ यज॑मानश्च दे॒वता॑मन्तरि॒तस्तस्या॒ आ वृ॑श्च्येते प्राजाप॒त्यं द॑धिग्र॒हं गृ॑ह्णीयात् प्र॒जाप॑तिः॒ सर्वा॑ दे॒वता॑ दे॒वता᳚भ्य ए॒व निह्नु॑वाते ज्ये॒ष्ठो वा ए॒ष ग्रहा॑णां॒ ॅयस्यै॒ष गृ॒ह्यते॒ ज्यैष्ठ्य॑मे॒व ग॑च्छति॒ सर्वा॑सां॒ ॅवा ए॒तद्दे॒वता॑नाꣳ रू॒पं ॅयदे॒ष ग्रहो॒ यस्यै॒ष गृ॒ह्यते॒ सर्वा᳚ण्ये॒वैनꣳ॑ रू॒पाणि॑ पशू॒नामुप॑तिष्ठन्त उपया॒मगृ॑हीतो - [  ] \newline

\textbf{Pada Paata} \newline

याम् । वै । अ॒द्ध्व॒र्युः । च॒ । यज॑मानः । च॒ । दे॒वता᳚म् । अ॒न्त॒रि॒त इत्य॑न्तः - इ॒तः । तस्यै᳚ । एति॑ । वृ॒श्च्ये॒ते॒ इति॑ । प्रा॒जा॒प॒त्यमिति॑ प्राजा - प॒त्यम् । द॒धि॒ग्र॒हमिति॑ दधि - ग्र॒हम् । गृ॒ह्णी॒या॒त् । प्र॒जाप॑ति॒रिति॑ प्र॒जा - प॒तिः॒ । सर्वाः᳚ । दे॒वताः᳚ । दे॒वता᳚भ्यः । ए॒व । नीति॑ । ह्नु॒वा॒ते॒ इति॑ । ज्ये॒ष्ठः । वै । ए॒षः । ग्रहा॑णाम् । यस्य॑ । ए॒षः । गृ॒ह्यते᳚ । ज्यैष्ट्य᳚म् । ए॒व । ग॒च्छ॒ति॒ । सर्वा॑साम् । वै । ए॒तत् । दे॒वता॑नाम् । रू॒पम् । यत् । ए॒षः । ग्रहः॑ । यस्य॑ । ए॒षः । गृ॒ह्यते᳚ । सर्वा॑णि । ए॒व । ए॒न॒म् । रू॒पाणि॑ । प॒शू॒नाम् । उपेति॑ । ति॒ष्ठ॒न्ते॒ । उ॒प॒या॒मगृ॑हीत॒ इत्यु॑पया॒म - गृ॒ही॒तः॒ ।  \newline




\markright{ TS 3.5.9.2  \hfill https://www.vedavms.in \hfill}
\addcontentsline{toc}{section}{ TS 3.5.9.2 }
\section*{ TS 3.5.9.2 }

\textbf{TS 3.5.9.2 } \newline
\textbf{Samhita Paata} \newline

-ऽसि प्र॒जाप॑तये त्वा॒ ज्योति॑ष्मते॒ ज्योति॑ष्मन्तं गृह्णा॒मीत्या॑ह॒ ज्योति॑रे॒वैनꣳ॑ समा॒नानां᳚ करोत्यग्नि-जि॒ह्वेभ्य॑स्त्वर्ता॒युभ्य॒ इत्या॑है॒ताव॑ती॒र्वै दे॒वता॒स्ताभ्य॑ ए॒वैनꣳ॒॒ सर्वा᳚भ्यो गृह्णा॒त्यपे᳚न्द्र द्विष॒तो मन॒ इत्या॑ह॒ भ्रातृ॑व्यापनुत्त्यै प्रा॒णाय॑ त्वाऽपा॒नाय॒ त्वेत्या॑ह प्रा॒णाने॒व यज॑माने दधाति॒ तस्मै᳚ त्वा प्र॒जाप॑तये विभू॒दाव्.न्ने॒ ज्योति॑ष्मते॒ ज्योति॑ष्मन्तं जुहो॒मी - [  ] \newline

\textbf{Pada Paata} \newline

अ॒सि॒ । प्र॒जाप॑तय॒ इति॑ प्र॒जा - प॒त॒ये॒ । त्वा॒ । ज्योति॑ष्मते । ज्योति॑ष्मन्तम् । गृ॒ह्णा॒मि॒ । इति॑ । आ॒ह॒ । ज्योतिः॑ । ए॒व । ए॒न॒म् । स॒मा॒नाना᳚म् । क॒रो॒ति॒ । अ॒ग्नि॒जि॒ह्वेभ्य॒ इत्य॑ग्नि - जि॒ह्वेभ्यः॑ । त्वा॒ । ऋ॒ता॒युभ्य॒ इत्यृ॑ता॒यु - भ्यः॒ । इति॑ । आ॒ह॒ । ए॒ताव॑तीः । वै । दे॒वताः᳚ । ताभ्यः॑ । ए॒व । ए॒न॒म् । सर्वा᳚भ्यः । गृ॒ह्णा॒ति॒ । अपेति॑ । इ॒न्द्र॒ । द्वि॒ष॒तः । मनः॑ । इति॑ । आ॒ह॒ । भ्रातृ॑व्यापनुत्त्या॒ इति॒ भ्रातृ॑व्य - अ॒प॒नु॒त्त्यै॒ । प्रा॒णायेति॑ प्र - अ॒नाय॑ । त्वा॒ । अ॒पा॒नायेत्य॑प - अ॒नाय॑ । त्वा॒ । इति॑ । आ॒ह॒ । प्रा॒णानिति॑ प्र - अ॒नान् । ए॒व । यज॑माने । द॒धा॒ति॒ । तस्मै᳚ । त्वा॒ । प्र॒जाप॑तय॒ इति॑ प्र॒जा - प॒त॒ये॒ । वि॒भू॒दाव्.न्न॒ इति॑ विभु - दाव्.न्ने᳚ । ज्योति॑ष्मते । ज्योति॑ष्मन्तम् । जु॒हो॒मि॒ ।  \newline




\markright{ TS 3.5.9.3  \hfill https://www.vedavms.in \hfill}
\addcontentsline{toc}{section}{ TS 3.5.9.3 }
\section*{ TS 3.5.9.3 }

\textbf{TS 3.5.9.3 } \newline
\textbf{Samhita Paata} \newline

-त्या॑ह प्र॒जाप॑तिः॒ सर्वा॑ दे॒वताः॒ सर्वा᳚भ्य ए॒वैनं॑ दे॒वता᳚भ्यो जुहोत्याज्यग्र॒हं गृ॑ह्णीया॒त् तेज॑स्कामस्य॒ तेजो॒ वा आज्यं॑ तेज॒स्व्ये॑व भ॑वति सोमग्र॒हं गृ॑ह्णीयाद्-ब्रह्मवर्च॒सका॑मस्य ब्रह्मवर्च॒सं ॅवै सोमो᳚ ब्रह्मवर्च॒स्ये॑व भ॑वति दधिग्र॒हं गृ॑ह्णीयात् प॒शुका॑म॒स्योर्ग्वै दद्ध्यूर्क् प॒शव॑ ऊ॒र्जैवास्मा॒ ऊर्जं॑ प॒शूनव॑ रुन्धे ॥ \newline

\textbf{Pada Paata} \newline

इति॑ । आ॒ह॒ । प्र॒जाप॑ति॒रिति॑ प्र॒जा - प॒तिः॒ । सर्वाः᳚ । दे॒वताः᳚ । सर्वा᳚भ्यः । ए॒व । ए॒न॒म् । दे॒वता᳚भ्यः । जु॒हो॒ति॒ । आ॒ज्य॒ग्र॒हमित्या᳚ज्य - ग्र॒हम् । गृ॒ह्णी॒या॒त् । तेज॑स्काम॒स्येति॒ तेजः॑ - का॒म॒स्य॒ । तेजः॑ । वै । आज्य᳚म् । ते॒ज॒स्वी । ए॒व । भ॒व॒ति॒ । सो॒म॒ग्र॒हमिति॑ सोम - ग्र॒हम् । गृ॒ह्णी॒या॒त् । ब्र॒ह्म॒व॒र्च॒सका॑म॒स्येति॑ ब्रह्मवर्च॒स - का॒म॒स्य॒ । ब्र॒ह्म॒व॒र्च॒समिति॑ ब्रह्म - व॒र्च॒सम् । वै । सोमः॑ । ब्र॒ह्म॒व॒र्च॒सीति॑ ब्रह्म - व॒र्च॒सी । ए॒व । भ॒व॒ति॒ । द॒धि॒ग्र॒हमिति॑ दधि - ग्र॒हम् । गृ॒ह्णी॒या॒त् । प॒शुका॑म॒स्येति॑ प॒शु - का॒म॒स्य॒ । ऊर्क् । वै । दधि॑ । ऊर्क् । प॒शवः॑ । ऊ॒र्जा । ए॒व । अ॒स्मै॒ । ऊर्ज᳚म् । प॒शून् । अवेति॑ । रु॒न्धे॒ ॥  \newline




\markright{ TS 3.5.10.1  \hfill https://www.vedavms.in \hfill}
\addcontentsline{toc}{section}{ TS 3.5.10.1 }
\section*{ TS 3.5.10.1 }

\textbf{TS 3.5.10.1 } \newline
\textbf{Samhita Paata} \newline

त्वे क्रतु॒मपि॑ वृञ्जन्ति॒ विश्वे॒ द्विर्यदे॒ते त्रि-र्भव॒न्त्यूमाः᳚ । स्वा॒दोः स्वादी॑यः स्वा॒दुना॑ सृजा॒ समत॑ ऊ॒ षु मधु॒ मधु॑ना॒ऽभि यो॑धि ।उ॒प॒या॒मगृ॑हीतोऽसि प्र॒जाप॑तये त्वा॒ जुष्टं॑ गृह्णाम्ये॒ष ते॒ योनिः॑ प्र॒जाप॑तये त्वा ॥प्रा॒ण॒ग्र॒हान् गृ॑ह्णात्ये॒ताव॒द्वा अ॑स्ति॒ याव॑दे॒ते ग्रहाः॒ स्तोमा॒श्छन्दाꣳ॑सि पृ॒ष्ठानि॒ दिशो॒ याव॑दे॒वास्ति॒ त - [  ] \newline

\textbf{Pada Paata} \newline

त्वे इति॑ । क्रतु᳚म् । अपीति॑ । वृ॒ञ्ज॒न्ति॒ । विश्वे᳚ । द्विः । यत् । ए॒ते । त्रिः । भव॑न्ति । ऊमाः᳚ ॥ स्वा॒दोः । स्वादी॑यः । स्वा॒दुना᳚ । सृ॒ज॒ । समिति॑ । अतः॑ । उ॒ । स्विति॑ । मधु॑ । मधु॑ना । अ॒भीति॑ । यो॒धि॒ ॥ उ॒प॒या॒मगृ॑हीत॒ इत्यु॑पया॒म - गृ॒ही॒तः॒ । अ॒सि॒ । प्र॒जाप॑तय॒ इति॑ प्र॒जा - प॒त॒ये॒ । त्वा॒ । जुष्ट᳚म् । गृ॒ह्णा॒मि॒ । ए॒षः । ते॒ । योनिः॑ । प्र॒जाप॑तय॒ इति॑ प्र॒जा - प॒त॒ये॒ । त्वा॒ ॥ प्रा॒ण॒ग्र॒हानिति॑ प्राण-ग्र॒हान् । गृ॒ह्णा॒ति॒ । ए॒ताव॑त् । वै । अ॒स्ति॒ । याव॑त् । ए॒ते । ग्रहाः᳚ । स्तोमाः᳚ । छन्दाꣳ॑सि । पृ॒ष्ठानि॑ । दिशः॑ । याव॑त् । ए॒व । अस्ति॑ । तत् ।  \newline




\markright{ TS 3.5.10.2  \hfill https://www.vedavms.in \hfill}
\addcontentsline{toc}{section}{ TS 3.5.10.2 }
\section*{ TS 3.5.10.2 }

\textbf{TS 3.5.10.2 } \newline
\textbf{Samhita Paata} \newline

-दव॑ रुन्धे ज्ये॒ष्ठा वा ए॒तान् ब्रा᳚ह्म॒णाः पु॒रा विदाम॑क्र॒न् तस्मा॒त् तेषाꣳ॒॒ सर्वा॒ दिशो॒ऽभिजि॑ता अभूव॒न्॒. यस्यै॒ ते गृ॒ह्यन्ते॒ ज्यैष्ठ्य॑मे॒व ग॑च्छत्य॒भि दिशो॑ जयति॒ पञ्च॑ गृह्यन्ते॒ पञ्च॒ दिशः॒ सर्वा᳚स्वे॒व दि॒क्ष्-वृ॑द्ध्नुवन्ति॒ नव॑नव गृह्यन्ते॒ नव॒ वै पुरु॑षे प्रा॒णाः प्रा॒णाने॒व यज॑मानेषु दधति प्राय॒णीये॑ चोदय॒नीये॑ च गृह्यन्ते प्रा॒णा वै प्रा॑णग्र॒हाः - [  ] \newline

\textbf{Pada Paata} \newline

अवेति॑ । रु॒न्धे॒ । ज्ये॒ष्ठाः । वै । ए॒तान् । ब्रा॒ह्म॒णाः । पु॒रा । वि॒दाम् । अ॒क्र॒न्न् । तस्मा᳚त् । तेषा᳚म् । सर्वाः᳚ । दिशः॑ । अ॒भिजि॑ता॒ इत्य॒भि - जि॒ताः॒ । अ॒भू॒व॒न्न् । यस्य॑ । ए॒ते । गृ॒ह्यन्ते᳚ । ज्यैष्ठ्य᳚म् । ए॒व । ग॒च्छ॒ति॒ । अ॒भीति॑ । दिशः॑ । ज॒य॒ति॒ । पञ्च॑ । गृ॒ह्य॒न्ते॒ । पञ्च॑ । दिशः॑ । सर्वा॑सु । ए॒व । दि॒क्षु । ऋ॒द्ध्नु॒व॒न्ति॒ । नव॑न॒वेति॒ नव॑ - न॒व॒ । गृ॒ह्य॒न्ते॒ । नव॑ । वै । पुरु॑षे । प्रा॒णा इति॑ प्र - अ॒नाः । प्रा॒णानिति॑ प्र - अ॒नान् । ए॒व । यज॑मानेषु । द॒ध॒ति॒ । प्रा॒य॒णीय॒ इति॑ प्र - अ॒य॒नीये᳚ । च॒ । उ॒द॒य॒नीय॒ इत्यु॑त् - अ॒य॒नीये᳚ । च॒ । गृ॒ह्य॒न्ते॒ । प्रा॒णा इति॑ प्र - अ॒नाः । वै । प्रा॒ण॒ग्र॒हा इति॑ प्राण - ग्र॒हाः ।  \newline




\markright{ TS 3.5.10.3  \hfill https://www.vedavms.in \hfill}
\addcontentsline{toc}{section}{ TS 3.5.10.3 }
\section*{ TS 3.5.10.3 }

\textbf{TS 3.5.10.3 } \newline
\textbf{Samhita Paata} \newline

प्रा॒णैरे॒व प्र॒यन्ति॑ प्रा॒णैरुद्य॑न्ति दश॒मेऽह॑न् गृह्यन्ते प्रा॒णा वै प्रा॑णग्र॒हाः प्रा॒णेभ्यः॒ खलु॒ वा ए॒तत् प्र॒जा य॑न्ति॒ यद्वा॑मदे॒व्यं ॅयोने॒श्च्यव॑ते दश॒मेऽह॑न्. वामदे॒व्यं ॅयोने᳚श्च्यवते॒ यद्-द॑श॒मेऽह॑न् गृ॒ह्यन्ते᳚ प्रा॒णेभ्य॑ ए॒व तत् प्र॒जा नय॑न्ति ॥ \newline

\textbf{Pada Paata} \newline

प्रा॒णैरिति॑ प्र - अ॒नैः । ए॒व । प्र॒यन्तीति॑ प्र - यन्ति॑ । प्रा॒णैरिति॑ प्र - अ॒नैः । उदिति॑ । य॒न्ति॒ । द॒श॒मे । अहन्न्॑ । गृ॒ह्य॒न्ते॒ । प्रा॒णा इति॑ प्र - अ॒नाः । वै । प्रा॒ण॒ग्र॒हा इति॑ प्राण - ग्र॒हाः । प्रा॒णेभ्य॒ इति॑ प्र - अ॒नेभ्यः॑ । खलु॑ । वै । ए॒तत् । प्र॒जा इति॑ प्र - जाः । य॒न्ति॒ । यत् । वा॒म॒दे॒व्यमिति॑ वाम - दे॒व्यम् । योनेः᳚ । च्यव॑ते । द॒श॒मे । अहन्न्॑ । वा॒म॒दे॒व्यमिति॑ वाम - दे॒व्यम् । योनेः᳚ । च्य॒व॒ते॒ । यत् । द॒श॒मे । अहन्न्॑ । गृ॒ह्यन्ते᳚ । प्रा॒णेभ्य॒ इति॑ प्र - अ॒नेभ्यः॑ । ए॒व । तत् । प्र॒जा इति॑ प्र - जाः । न । य॒न्ति॒ ॥  \newline




\markright{ TS 3.5.11.1  \hfill https://www.vedavms.in \hfill}
\addcontentsline{toc}{section}{ TS 3.5.11.1 }
\section*{ TS 3.5.11.1 }

\textbf{TS 3.5.11.1 } \newline
\textbf{Samhita Paata} \newline

प्र दे॒वन्दे॒व्या धि॒या भर॑ता जा॒तवे॑दसं । ह॒व्या नो॑ वक्षदानु॒षक् ॥ अ॒यमु॒ ष्य प्रदे॑व॒युर्.होता॑ य॒ज्ञाय॑ नीयते । रथो॒ न योर॒भीवृ॑तो॒ घृणी॑वान् चेतति॒ त्मना᳚ ॥ अ॒यम॒ग्निरु॑रुष्यत्य॒मृता॑दिव॒ जन्म॑नः । सह॑सश्चि॒थ् सही॑यान् दे॒वो जी॒वात॑वे कृ॒तः ॥ इडा॑यास्त्वा प॒दे व॒यं नाभा॑ पृथि॒व्या अधि॑ । जात॑वेदो॒ नि धी॑म॒ह्यग्ने॑ ह॒व्याय॒ वोढ॑वे ॥ \newline

\textbf{Pada Paata} \newline

प्रेति॑ । दे॒वम् । दे॒व्या । धि॒या । भर॑त । जा॒तवे॑दस॒मिति॑ जा॒त - वे॒द॒स॒म् ॥ ह॒व्या । नः॒ । व॒क्ष॒त् । आ॒नु॒षक् ॥ अ॒यम् । उ॒ । स्यः । प्रेति॑ । दे॒व॒युरिति॑ देव-युः । होता᳚ । य॒ज्ञाय॑ । नी॒य॒ते॒ ॥ रथः॑ । न । योः । अ॒भीवृ॑त॒ इत्य॒भि - वृ॒तः॒ । घृणी॑वान् । चे॒त॒ति॒ । त्मना᳚ ॥ अ॒यम् । अ॒ग्निः । उ॒रु॒ष्य॒ति॒ । अ॒मृता᳚त् । इ॒व॒ । जन्म॑नः ॥ सह॑सः । चि॒त् । सही॑यान् । दे॒वः । जी॒वात॑वे । कृ॒तः ॥ इडा॑याः । त्वा॒ । प॒दे । व॒यम् । नाभा᳚ । पृ॒थि॒व्याः । अधि॑ ॥ जात॑वेद॒ इति॒ जात॑ - वे॒दः॒ । नीति॑ । धी॒म॒हि॒ । अग्ने᳚ । ह॒व्याय॑ । वोढ॑वे ॥  \newline




\markright{ TS 3.5.11.2  \hfill https://www.vedavms.in \hfill}
\addcontentsline{toc}{section}{ TS 3.5.11.2 }
\section*{ TS 3.5.11.2 }

\textbf{TS 3.5.11.2 } \newline
\textbf{Samhita Paata} \newline

अग्ने॒ विश्वे॑भिः स्वनीक दे॒वैरूर्णा॑वन्तं प्रथ॒मः सी॑द॒ योनिं᳚ । कु॒ला॒यिनं॑ घृ॒तव॑न्तꣳ सवि॒त्रे य॒ज्ञ्ं न॑य॒ यज॑मानाय सा॒धु ॥ सीद॑ होतः॒ स्व उ॑ लो॒के चि॑कि॒त्वान्थ्सा॒दया॑ य॒ज्ञ्ꣳ सु॑कृ॒तस्य॒ योनौ᳚ । दे॒वा॒वीर्दे॒वान्. ह॒विषा॑ यजा॒स्यग्ने॑ बृ॒हद्-यज॑माने॒ वयो॑ धाः ॥ नि होता॑ होतृ॒षद॑ने॒ विदा॑नस्त्वे॒षो दी॑दि॒वाꣳ अ॑सदथ् सु॒दक्षः॑ । अद॑ब्धव्रत-प्रमति॒र्वसि॑ष्ठः सहस्रं भ॒रः शुचि॑जिह्वो अ॒ग्निः ॥ त्वं दू॒तस्त्व - [  ] \newline

\textbf{Pada Paata} \newline

अग्ने᳚ । विश्वे॑भिः । स्व॒नी॒केति॑ सु - अ॒नी॒क॒ । दे॒वैः । ऊर्णा॑वन्त॒मित्यूर्णा᳚-व॒न्त॒म् । प्र॒थ॒मः । सी॒द॒ । योनि᳚म् ॥ कु॒ला॒यिन᳚म् । घृ॒तव॑न्त॒मिति॑ घृ॒त - व॒न्त॒म् । स॒वि॒त्रे । य॒ज्ञ्म् । न॒य॒ । यज॑मानाय । सा॒धु ॥ सीद॑ । हो॒तः॒ । स्वे । उ॒ । लो॒के । चि॒कि॒त्वान् । सा॒दय॑ । य॒ज्ञ्म् । सु॒कृ॒तस्येति॑ सु - कृ॒तस्य॑ । योनौ᳚ ॥ दे॒वा॒वीरिति॑ देव - अ॒वीः । दे॒वान् । ह॒विषा᳚ । य॒जा॒सि॒ । अग्ने᳚ । बृ॒हत् । यज॑माने । वयः॑ । धाः॒ ॥ नीति॑ । होता᳚ । हो॒तृ॒षद॑न॒ इति॑ होतृ-सद॑ने । विदा॑नः । त्वे॒षः । दी॒दि॒वान् । अ॒स॒द॒त् । सु॒दक्ष॒ इति॑ सु - दक्षः॑ ॥ अद॑ब्धव्रतप्रमति॒रित्यद॑ब्धव्रत - प्र॒म॒तिः॒ । वसि॑ष्ठः । स॒ह॒स्र॒भं॒र इति॑ सहस्रं - भ॒रः । शुचि॑जिह्व॒ इति॒ शुचि॑ - जि॒ह्वः॒ । अ॒ग्निः ॥ त्वम् । दू॒तः । त्वम् ।  \newline




\markright{ TS 3.5.11.3  \hfill https://www.vedavms.in \hfill}
\addcontentsline{toc}{section}{ TS 3.5.11.3 }
\section*{ TS 3.5.11.3 }

\textbf{TS 3.5.11.3 } \newline
\textbf{Samhita Paata} \newline

मु॑ नः पर॒स्पास्त्वं ॅवस्य॒ आ वृ॑षभ प्रणे॒ता । अग्ने॑ तो॒कस्य॑ न॒स्तने॑ त॒नूना॒मप्र॑युच्छ॒न् दीद्य॑द्बोधि गो॒पाः ॥ अ॒भि त्वा॑ देव सवित॒रीशा॑नं॒ ॅवार्या॑णां । सदा॑ऽवन् भा॒गमी॑महे ॥ म॒ही द्यौः पृ॑थि॒वी च॑न इ॒मं ॅय॒ज्ञ्ं मि॑मिक्षतां । पि॒पृ॒तां नो॒ भरी॑मभिः ॥ त्वाम॑ग्ने॒ पुष्क॑रा॒दद्ध्यथ॑र्वा॒ निर॑मन्थत । मू॒र्द्ध्नो विश्व॑स्य वा॒घतः॑ ॥ तमु॑ - [  ] \newline

\textbf{Pada Paata} \newline

उ॒ । नः॒ । प॒र॒स्पा इति॑ परः - पाः । त्वम् । वस्यः॑ । एति॑ । वृ॒ष॒भ॒ । प्र॒णे॒तेति॑ प्र - ने॒ता ॥ अग्ने᳚ । तो॒कस्य॑ । नः॒ । तने᳚ । त॒नूना᳚म् । अप्र॑युच्छ॒नित्यप्र॑ - यु॒च्छ॒न्न् । दीद्य॑त् । बो॒धि॒ । गो॒पा इति॑ गो-पाः ॥ अ॒भीति॑ । त्वा॒ । दे॒व॒ । स॒वि॒तः॒ । ईशा॑नम् । वार्या॑णाम् ॥ सदा᳚ । अ॒व॒न्न् । भा॒गम् । ई॒म॒हे॒ ॥ म॒ही । द्यौः । पृ॒थि॒वी । च॒ । नः॒ । इ॒मम् । य॒ज्ञ्म् । मि॒मि॒क्ष॒ता॒म् ॥ पि॒पृ॒ताम् । नः॒ । भरी॑मभि॒रिति॒ भरी॑म-भिः॒ ॥ त्वाम् । अ॒ग्ने॒ । पुष्क॑रात् । अधीति॑ । अथ॑र्वा । निरिति॑ । अ॒म॒न्थ॒त॒ ॥ मू॒र्द्ध्नः । विश्व॑स्य । वा॒घतः॑ ॥ तम् । उ॒ ।  \newline




\markright{ TS 3.5.11.4  \hfill https://www.vedavms.in \hfill}
\addcontentsline{toc}{section}{ TS 3.5.11.4 }
\section*{ TS 3.5.11.4 }

\textbf{TS 3.5.11.4 } \newline
\textbf{Samhita Paata} \newline

त्वा द॒द्ध्यङ्ङृषिः॑ पु॒त्र ई॑धे॒ अथ॑र्वणः । वृ॒त्र॒हणं॑ पुरन्द॒रं ॥ तमु॑ त्वा पा॒थ्यो वृषा॒ समी॑धे दस्यु॒हन्त॑मं । ध॒नं॒ ज॒यꣳ रणे॑रणे ॥ उ॒त ब्रु॑वन्तु ज॒न्तव॒ उद॒ग्निर्वृ॑त्र॒हाऽज॑नि । ध॒नं॒ ज॒यो रणे॑रणे ॥ आ यꣳ हस्ते॒ न खा॒दिनꣳ॒॒ शिशुं॑ जा॒तं न बिभ्र॑ति । वि॒शाम॒ग्निꣳ स्व॑द्ध्व॒रं ॥ प्रदे॒वं दे॒ववी॑तये॒ भर॑ता वसु॒वित्त॑मं । आस्वे योनौ॒ नि षी॑दतु ॥ आ - [  ] \newline

\textbf{Pada Paata} \newline

त्वा॒ । द॒द्ध्यङ् । ऋषिः॑ । पु॒त्रः । ई॒धे॒ । अथ॑र्वणः ॥ वृ॒त्र॒हण॒मिति॑ वृत्र - हन᳚म् । पु॒र॒न्द॒रमिति॑ पुरं - द॒रम् ॥ तम् । उ॒ । त्वा॒ । पा॒थ्यः । वृषा᳚ । समिति॑ । ई॒धे॒ । द॒स्यु॒हन्त॑म॒मिति॑ दस्यु - हन्त॑मम् ॥ ध॒न॒ञ्ज॒यमिति॑ धनं - ज॒यम् । रणे॑रण॒ इति॒ रणे᳚ - र॒णे॒ ॥ उ॒त । ब्रु॒व॒न्तु॒ । ज॒न्तवः॑ । उदिति॑ । अ॒ग्निः । वृ॒त्र॒हेति॑ वृत्र - हा । अ॒ज॒नि॒ ॥ ध॒न॒ञ्ज॒य इति॑ धनं - ज॒यः । रणे॑रण॒ इति॒ रणे᳚-र॒णे॒ ॥ एति॑ । यम् । हस्ते᳚ । न । खा॒दिन᳚म् । शिशु᳚म् । जा॒तम् । न । बिभ्र॑ति ॥ वि॒शाम् । अ॒ग्निम् । स्व॒द्ध्व॒रमिति॑ सु - अ॒ध्व॒रम् ॥ प्रेति॑ । दे॒वम् । दे॒ववी॑तय॒ इति॑ दे॒व - वी॒त॒ये॒ । भर॑त । व॒सु॒वित्त॑म॒मिति॑ वसु॒वित्-त॒म॒म् ॥ एति॑ । स्वे । योनौ᳚ । नीति॑ । सी॒द॒तु॒ ॥ एति॑ ।  \newline




\markright{ TS 3.5.11.5  \hfill https://www.vedavms.in \hfill}
\addcontentsline{toc}{section}{ TS 3.5.11.5 }
\section*{ TS 3.5.11.5 }

\textbf{TS 3.5.11.5 } \newline
\textbf{Samhita Paata} \newline

जा॒तं जा॒तवे॑दसि प्रि॒यꣳ शि॑शी॒ताऽति॑थिं । स्यो॒न आ गृ॒हप॑तिं ॥ अ॒ग्निना॒ऽग्निः समि॑द्ध्यते क॒विर्गृ॒हप॑ति॒र्युवा᳚ । ह॒व्य॒वाड् जु॒ह्वा᳚स्यः ॥त्वꣳ ह्य॑ग्ने अ॒ग्निना॒ विप्रो॒ विप्रे॑ण॒ सन्थ्स॒ता । सखा॒ सख्या॑ समि॒द्ध्यसे᳚ ॥ तं म॑र्जयन्त सु॒क्रतुं॑ पुरो॒यावा॑नमा॒जिषु॑ । स्वेषु॒ क्षये॑षु वा॒जिनं᳚ ॥ य॒ज्ञेन॑ य॒ज्ञ्म॑यजन्त दे॒वास्तानि॒ धर्मा॑णि प्रथ॒मान्या॑सन्न् । ते ह॒ नाकं॑ महि॒मानः॑ सचन्ते॒ यत्र॒ ( ) पूर्वे॑ सा॒द्ध्याः सन्ति॑ दे॒वाः ॥ \newline

\textbf{Pada Paata} \newline

जा॒तम् । जा॒तवे॑द॒सीति॑ जा॒त - वे॒द॒सि॒ । प्रि॒यम् । शि॒शी॒त॒ । अति॑थिम् ॥ स्यो॒ने । एति॑ । गृ॒हप॑ति॒मिति॑ गृ॒ह - प॒ति॒म् ॥ अ॒ग्निना᳚ । अ॒ग्निः । समिति॑ । इ॒द्ध्य॒ते॒ । क॒विः । गृ॒हप॑ति॒रिति॑ गृ॒ह - प॒तिः॒ । युवा᳚ ॥ ह॒व्य॒वाडिति॑ हव्य - वाट् । जु॒ह्वा᳚स्य॒ इति॑ जु॒हु - आ॒स्यः॒ ॥ त्वम् । हि । अ॒ग्ने॒ । अ॒ग्निना᳚ । विप्रः॑ । विप्रे॑ण । सन्न् । स॒ता ॥ सखा᳚ । सख्या᳚ । स॒मि॒द्ध्यस॒ इति॑ सं - इ॒ध्यसे᳚ ॥ तम् । म॒र्ज॒य॒न्त॒ । सु॒क्रतु॒मिति॑ सु - क्रतु᳚म् । पु॒रो॒यावा॑न॒मिति॑ पुरः - यावा॑नम् । आ॒जिषु॑ ॥ स्वेषु॑ । क्षये॑षु । वा॒जिन᳚म् ॥ य॒ज्ञेन॑ । य॒ज्ञ्म् । अ॒य॒ज॒न्त॒ । दे॒वाः । तानि॑ । धर्मा॑णि । प्र॒थ॒मानि॑ । आ॒स॒न्न् ॥ ते । ह॒ । नाक᳚म् । म॒हि॒मानः॑ । स॒च॒न्ते॒ । यत्र॑ ( ) । पूर्वे᳚ । सा॒द्ध्याः । सन्ति॑ । दे॒वाः ॥  \newline






\end{document}