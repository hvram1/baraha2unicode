\documentclass[17pt]{extarticle}
\usepackage{babel}
\usepackage{fontspec}
\usepackage{polyglossia}
\usepackage{extsizes}

\usepackage{color}   %May be necessary if you want to color links
\usepackage{hyperref}
\hypersetup{
    colorlinks=true, %set true if you want colored links
    linktoc=all,     %set to all if you want both sections and subsections linked
    linkcolor=black,  %choose some color if you want links to stand out
}

\setmainlanguage{sanskrit}
\setotherlanguages{english} %% or other languages
\setlength{\parindent}{0pt}
\pagestyle{myheadings}
\newfontfamily\devanagarifont[Script=Devanagari]{AdishilaVedic}
\renewcommand{\theHsection}{\thepart.section.\thesection}

\newcommand{\VAR}[1]{}
\newcommand{\BLOCK}[1]{}




\begin{document}
\begin{titlepage}
    \begin{center}
 
\begin{sanskrit}
    { \Large
    कृष्ण यजुर्वेदीय तैत्तिरीय संहिता,पद,जटा,घन पाठः 
    }
    \\
    \vspace{2.5cm}
    \mbox{ \Large
    6.4       षष्ठकाण्डे चतुर्थः प्रश्नः - सोममन्त्रब्राह्मणनिरूपणं   }
\end{sanskrit}
\end{center}

\end{titlepage}
\tableofcontents
\phantomsection
\pagebreak

\markright{ TS 6.4.1.1  \hfill https://www.vedavms.in \hfill}

\section{ TS 6.4.1.1 }

\textbf{TS 6.4.1.1 } \newline
\textbf{Samhita Paata} \newline

य॒ज्ञेन॒ वै प्र॒जाप॑तिः प्र॒जा अ॑सृजत॒ ता उ॑प॒यड्भि॑-रे॒वासृ॑जत॒ यदु॑प॒यज॑ उप॒यज॑ति प्र॒जा ए॒व तद् यज॑मानः सृजते जघना॒र्द्धादव॑ द्यति जघना॒र्द्धाद्धि प्र॒जाः प्र॒जाय॑न्ते स्थविम॒तोऽव॑ द्यति स्थविम॒तो हि प्र॒जाः प्र॒जाय॒न्ते ऽस॑भिंन्द॒न्नव॑ द्यति प्रा॒णाना॒-मस॑भेंदाय॒ न प॒र्याव॑र्तयति॒ यत् प॑र्याव॒र्तये॑दुदाव॒र्तः प्र॒जा ग्राहु॑कः स्याथ् समु॒द्रं ग॑च्छ॒ स्वाहेत्या॑ह॒ रेत॑- [  ] \newline

\textbf{Pada Paata} \newline

य॒ज्ञेन॑ । वै । प्र॒जाप॑ति॒रिति॑ प्र॒जा - प॒तिः॒ । प्र॒जा इति॑ प्र - जाः । अ॒सृ॒ज॒त॒ । ताः । उ॒प॒यड्भि॒रित्यु॑प॒यट्-भिः॒ । ए॒व । अ॒सृ॒ज॒त॒ । यत् । उ॒प॒यज॒ इत्यु॑प - यजः॑ । उ॒प॒यज॒तीत्यु॑प - यज॑ति । प्र॒जा इति॑ प्र - जाः । ए॒व । तत् । यज॑मानः । सृ॒ज॒ते॒ । ज॒घ॒ना॒द्‌र्धादिति॑ जघन - अ॒द्‌र्धात् । अवेति॑ । द्य॒ति॒ । ज॒घ॒ना॒द्‌र्धादिति॑ जघन-अ॒द्‌र्धात् । हि । प्र॒जा इति॑ प्र - जाः । प्र॒जाय॑न्त॒ इति॑ प्र - जाय॑न्ते । स्थ॒वि॒म॒तः । अवेति॑ । द्य॒ति॒ । स्थ॒वि॒म॒तः । हि । प्र॒जा इति॑ प्र-जाः । प्र॒जाय॑न्त॒ इति॑ प्र-जाय॑न्ते । अस॑भिंन्द॒न्नित्यसं᳚ - भि॒न्द॒न्न् । अवेति॑ । द्य॒ति॒ । प्रा॒णाना॒मिति॑ प्र - अ॒नाना᳚म् । अस॑भेंदा॒येत्यसं᳚ - भे॒दा॒य॒ । न । प॒र्याव॑र्तय॒तीति॑ परि - आव॑र्तयति । यत् । प॒र्या॒व॒र्तये॒दिति॑ परि - आ॒व॒र्तये᳚त् । उ॒दा॒व॒र्तः । प्र॒जा इति॑ प्र - जाः । ग्राहु॑कः । स्या॒त् । स॒मु॒द्रम् । ग॒च्छ॒ । स्वाहा᳚ । इति॑ । आ॒ह॒ । रेतः॑ ।  \newline


\textbf{Krama Paata} \newline

य॒ज्ञेन॒ वै । वै प्र॒जाप॑तिः । प्र॒जाप॑तिः प्र॒जाः । प्र॒जाप॑ति॒रिति॑ प्र॒जा - प॒तिः॒ । प्र॒जा अ॑सृजत । प्र॒जा इति॑ प्र - जाः । अ॒सृ॒ज॒त॒ ताः । ता उ॑प॒यड्भिः॑ । उ॒प॒यड्भि॑रे॒व । उ॒प॒यड्भि॒रित्यु॑प॒यट् - भिः॒ । ए॒वासृ॑जत । अ॒सृ॒ज॒त॒ यत् । यदु॑प॒यजः॑ । उ॒प॒यज॑ उप॒यज॑ति । उ॒प॒यज॒ इत्यु॑प - यजः॑ । उ॒प॒यज॑ति प्र॒जाः । उ॒प॒यज॒तीत्यु॑प - यज॑ति । प्र॒जा ए॒व । प्र॒जा इति॑ प्र - जाः । ए॒व तत् । तद् यज॑मानः । यज॑मानः सृजते । सृ॒ज॒ते॒ ज॒घ॒ना॒र्द्धात् । ज॒घ॒ना॒र्द्धादव॑ । ज॒घ॒ना॒र्द्धादि॑ति जघन - अ॒र्द्धात् । अव॑ द्यति । द्य॒ति॒ ज॒घ॒ना॒र्द्धात् । ज॒घ॒ना॒र्द्धाद्‌धि । ज॒घ॒ना॒र्द्धादिति॑ जघन - अ॒र्द्धात् । हि प्र॒जाः । प्र॒जाः प्र॒जाय॑न्ते । प्र॒जा इति॑ प्र - जाः । प्र॒जाय॑न्ते स्थविम॒तः । प्र॒जाय॑न्त॒ इति॑ प्र - जाय॑न्ते । स्थ॒वि॒म॒तोऽव॑ । अव॑ द्यति । द्य॒ति॒ स्थ॒वि॒म॒तः । स्थ॒वि॒म॒तो हि । हि प्र॒जाः । प्र॒जाः प्र॒जाय॑न्ते । प्र॒जा इति॑ प्र - जाः । प्र॒जाय॒न्तेऽस॑म्भिन्दन्न् । प्र॒जाय॑न्त॒ इति॑ प्र - जाय॑न्ते । अस॑म्भिन्द॒न्नव॑ । अस॑म्भिन्द॒न्नित्यस᳚म् - भि॒न्द॒न्न्॒ । अव॑ द्यति । द्य॒ति॒ प्रा॒णाना᳚म् । प्रा॒णाना॒मस॑म्भेदाय । प्रा॒णाना॒मिति॑ प्र - अ॒नाना᳚म् । अस॑म्भेदाय॒ न । अस॑म्भेदा॒येत्यस᳚म् - भे॒दा॒य॒ । न प॒र्याव॑र्तयति । प॒र्याव॑र्तयति॒ यत् । प॒र्याव॑र्तय॒तीति॑ परि - आव॑र्तयति । यत् प॑र्याव॒र्तये᳚त् । प॒र्या॒व॒र्तये॑दुदाव॒र्तः । प॒र्या॒व॒र्तये॒दिति॑ परि - आ॒व॒र्तये᳚त् । उ॒दा॒व॒र्तः प्र॒जाः । प्र॒जाः ग्राहु॑कः । प्र॒जा इति॑ प्र - जाः । ग्राहु॑कः स्यात् । स्या॒थ् स॒मु॒द्रम् । स॒मु॒द्रम् ग॑च्छ । ग॒च्छ॒ स्वाहा᳚ । स्वाहेति॑ । इत्या॑ह । आ॒ह॒ रेतः॑ । रेत॑ ए॒व \newline

\textbf{Jatai Paata} \newline

1. य॒ज्ञेन॒ वै वै य॒ज्ञेन॑ य॒ज्ञेन॒ वै । \newline
2. वै प्र॒जाप॑तिः प्र॒जाप॑ति॒र् वै वै प्र॒जाप॑तिः । \newline
3. प्र॒जाप॑तिः प्र॒जाः प्र॒जाः प्र॒जाप॑तिः प्र॒जाप॑तिः प्र॒जाः । \newline
4. प्र॒जाप॑ति॒रिति॑ प्र॒जा - प॒तिः॒ । \newline
5. प्र॒जा अ॑सृजता सृजत प्र॒जाः प्र॒जा अ॑सृजत । \newline
6. प्र॒जा इति॑ प्र - जाः । \newline
7. अ॒सृ॒ज॒त॒ ता स्ता अ॑सृजता सृजत॒ ताः । \newline
8. ता उ॑प॒यड्भि॑ रुप॒यड्भि॒ स्ता स्ता उ॑प॒यड्भिः॑ । \newline
9. उ॒प॒यड्भि॑ रे॒वै वोप॒यड्भि॑ रुप॒यड्भि॑ रे॒व । \newline
10. उ॒प॒यड्भि॒रित्यु॑प॒यट् - भिः॒ । \newline
11. ए॒वा सृ॑जता सृज तै॒वैवा सृ॑जत । \newline
12. अ॒सृ॒ज॒त॒ यद् यद॑सृजता सृजत॒ यत् । \newline
13. यदु॑प॒यज॑ उप॒यजो॒ यद् यदु॑प॒यजः॑ । \newline
14. उ॒प॒यज॑ उप॒यज॑ त्युप॒यज॑ त्युप॒यज॑ उप॒यज॑ उप॒यज॑ति । \newline
15. उ॒प॒यज॒ इत्यु॑प - यजः॑ । \newline
16. उ॒प॒यज॑ति प्र॒जाः प्र॒जा उ॑प॒यज॑ त्युप॒यज॑ति प्र॒जाः । \newline
17. उ॒प॒यज॒तीत्यु॑प - यज॑ति । \newline
18. प्र॒जा ए॒वैव प्र॒जाः प्र॒जा ए॒व । \newline
19. प्र॒जा इति॑ प्र - जाः । \newline
20. ए॒व तत् तदे॒वैव तत् । \newline
21. तद् यज॑मानो॒ यज॑मान॒ स्तत् तद् यज॑मानः । \newline
22. यज॑मानः सृजते सृजते॒ यज॑मानो॒ यज॑मानः सृजते । \newline
23. सृ॒ज॒ते॒ ज॒घ॒ना॒र्द्धाज् ज॑घना॒र्द्धाथ् सृ॑जते सृजते जघना॒र्द्धात् । \newline
24. ज॒घ॒ना॒र्द्धा दवाव॑ जघना॒र्द्धाज् ज॑घना॒र्द्धा दव॑ । \newline
25. ज॒घ॒ना॒र्द्धादिति॑ जघन - अ॒र्द्धात् । \newline
26. अव॑ द्यति द्य॒त्यवाव॑ द्यति । \newline
27. द्य॒ति॒ ज॒घ॒ना॒र्द्धाज् ज॑घना॒र्द्धाद् द्य॑ति द्यति जघना॒र्द्धात् । \newline
28. ज॒घ॒ना॒र्द्धा द्धि हि ज॑घना॒र्द्धाज् ज॑घना॒र्द्धा द्धि । \newline
29. ज॒घ॒ना॒र्द्धादिति॑ जघन - अ॒र्द्धात् । \newline
30. हि प्र॒जाः प्र॒जा हि हि प्र॒जाः । \newline
31. प्र॒जाः प्र॒जाय॑न्ते प्र॒जाय॑न्ते प्र॒जाः प्र॒जाः प्र॒जाय॑न्ते । \newline
32. प्र॒जा इति॑ प्र - जाः । \newline
33. प्र॒जाय॑न्ते स्थविम॒तः स्थ॑विम॒तः प्र॒जाय॑न्ते प्र॒जाय॑न्ते स्थविम॒तः । \newline
34. प्र॒जाय॑न्त॒ इति॑ प्र - जाय॑न्ते । \newline
35. स्थ॒वि॒म॒तो ऽवाव॑ स्थविम॒तः स्थ॑विम॒तो ऽव॑ । \newline
36. अव॑ द्यति द्य॒त्यवाव॑ द्यति । \newline
37. द्य॒ति॒ स्थ॒वि॒म॒तः स्थ॑विम॒तो द्य॑ति द्यति स्थविम॒तः । \newline
38. स्थ॒वि॒म॒तो हि हि स्थ॑विम॒तः स्थ॑विम॒तो हि । \newline
39. हि प्र॒जाः प्र॒जा हि हि प्र॒जाः । \newline
40. प्र॒जाः प्र॒जाय॑न्ते प्र॒जाय॑न्ते प्र॒जाः प्र॒जाः प्र॒जाय॑न्ते । \newline
41. प्र॒जा इति॑ प्र - जाः । \newline
42. प्र॒जाय॒न्ते ऽसं॑भिन्द॒न् नसं॑भिन्दन् प्र॒जाय॑न्ते प्र॒जाय॒न्ते ऽसं॑भिन्दन्न् । \newline
43. प्र॒जाय॑न्त॒ इति॑ प्र - जाय॑न्ते । \newline
44. असं॑भिन्द॒न् नवा वासं॑भिन्द॒न् नसं॑भिन्द॒न् नव॑ । \newline
45. असं॑भिन्द॒न्नित्यसं᳚ - भि॒न्द॒न्न् । \newline
46. अव॑ द्यति द्य॒त्यवाव॑ द्यति । \newline
47. द्य॒ति॒ प्रा॒णाना᳚म् प्रा॒णाना᳚म् द्यति द्यति प्रा॒णाना᳚म् । \newline
48. प्रा॒णाना॒ मसं॑भेदा॒या सं॑भेदाय प्रा॒णाना᳚म् प्रा॒णाना॒ मसं॑भेदाय । \newline
49. प्रा॒णाना॒मिति॑ प्र - अ॒नाना᳚म् । \newline
50. असं॑भेदाय॒ न नासं॑भेदा॒या सं॑भेदाय॒ न । \newline
51. असं॑भेदा॒येत्यसं᳚ - भे॒दा॒य॒ । \newline
52. न प॒र्याव॑र्तयति प॒र्याव॑र्तयति॒ न न प॒र्याव॑र्तयति । \newline
53. प॒र्याव॑र्तयति॒ यद् यत् प॒र्याव॑र्तयति प॒र्याव॑र्तयति॒ यत् । \newline
54. प॒र्याव॑र्तय॒तीति॑ परि - आव॑र्तयति । \newline
55. यत् प॑र्याव॒र्तये᳚त् पर्याव॒र्तये॒द् यद् यत् प॑र्याव॒र्तये᳚त् । \newline
56. प॒र्या॒व॒र्तये॑ दुदाव॒र्त उ॑दाव॒र्तः प॑र्याव॒र्तये᳚त् पर्याव॒र्तये॑ दुदाव॒र्तः । \newline
57. प॒र्या॒व॒र्तये॒दिति॑ परि - आ॒व॒र्तये᳚त् । \newline
58. उ॒दा॒व॒र्तः प्र॒जाः प्र॒जा उ॑दाव॒र्त उ॑दाव॒र्तः प्र॒जाः । \newline
59. प्र॒जा ग्राहु॑को॒ ग्राहु॑कः प्र॒जाः प्र॒जा ग्राहु॑कः । \newline
60. प्र॒जा इति॑ प्र - जाः । \newline
61. ग्राहु॑कः स्याथ् स्या॒द् ग्राहु॑को॒ ग्राहु॑कः स्यात् । \newline
62. स्या॒थ् स॒मु॒द्रꣳ स॑मु॒द्रꣳ स्या᳚थ् स्याथ् समु॒द्रम् । \newline
63. स॒मु॒द्रम् ग॑च्छ गच्छ समु॒द्रꣳ स॑मु॒द्रम् ग॑च्छ । \newline
64. ग॒च्छ॒ स्वाहा॒ स्वाहा॑ गच्छ गच्छ॒ स्वाहा᳚ । \newline
65. स्वाहे तीति॒ स्वाहा॒ स्वाहेति॑ । \newline
66. इत्या॑हा॒हे तीत्या॑ह । \newline
67. आ॒ह॒ रेतो॒ रेत॑ आहाह॒ रेतः॑ । \newline
68. रेत॑ ए॒वैव रेतो॒ रेत॑ ए॒व । \newline

\textbf{Ghana Paata } \newline

1. य॒ज्ञेन॒ वै वै य॒ज्ञेन॑ य॒ज्ञेन॒ वै प्र॒जाप॑तिः प्र॒जाप॑ति॒र् वै य॒ज्ञेन॑ य॒ज्ञेन॒ वै प्र॒जाप॑तिः । \newline
2. वै प्र॒जाप॑तिः प्र॒जाप॑ति॒र् वै वै प्र॒जाप॑तिः प्र॒जाः प्र॒जाः प्र॒जाप॑ति॒र् वै वै प्र॒जाप॑तिः प्र॒जाः । \newline
3. प्र॒जाप॑तिः प्र॒जाः प्र॒जाः प्र॒जाप॑तिः प्र॒जाप॑तिः प्र॒जा अ॑सृजता सृजत प्र॒जाः प्र॒जाप॑तिः प्र॒जाप॑तिः प्र॒जा अ॑सृजत । \newline
4. प्र॒जाप॑ति॒रिति॑ प्र॒जा - प॒तिः॒ । \newline
5. प्र॒जा अ॑सृजता सृजत प्र॒जाः प्र॒जा अ॑सृजत॒ ता स्ता अ॑सृजत प्र॒जाः प्र॒जा अ॑सृजत॒ ताः । \newline
6. प्र॒जा इति॑ प्र - जाः । \newline
7. अ॒सृ॒ज॒त॒ ता स्ता अ॑सृजता सृजत॒ ता उ॑प॒यड्‌भि॑ रुप॒यड्‌भि॒ स्ता अ॑सृजता सृजत॒ ता उ॑प॒यड्‌भिः॑ । \newline
8. ता उ॑प॒यड्‌भि॑ रुप॒यड्‌भि॒ स्ता स्ता उ॑प॒यड्‌भि॑ रे॒वैवोप॒यड्‌भि॒ स्ता स्ता उ॑प॒यड्‌भि॑रे॒व । \newline
9. उ॒प॒यड्‌भि॑ रे॒वैवोप॒यड्‌भि॑ रुप॒यड्‌भि॑ रे॒वा सृ॑जता सृज तै॒वोप॒यड्‌भि॑ रुप॒यड्‌भि॑ रे॒वा सृ॑जत । \newline
10. उ॒प॒यड्‌भि॒रित्यु॑प॒यट् - भिः॒ । \newline
11. ए॒वासृ॑जता सृजतै॒वैवा सृ॑जत॒ यद् यद॑सृज तै॒वैवा सृ॑जत॒ यत् । \newline
12. अ॒सृ॒ज॒त॒ यद् यद॑सृजता सृजत॒ यदु॑प॒यज॑ उप॒यजो॒ यद॑सृजता सृजत॒ यदु॑प॒यजः॑ । \newline
13. यदु॑प॒यज॑ उप॒यजो॒ यद् यदु॑प॒यज॑ उप॒यज॑ त्युप॒यज॑ त्युप॒यजो॒ यद् यदु॑प॒यज॑ उप॒यज॑ति । \newline
14. उ॒प॒यज॑ उप॒यज॑ त्युप॒यज॑ त्युप॒यज॑ उप॒यज॑ उप॒यज॑ति प्र॒जाः प्र॒जा उ॑प॒यज॑ त्युप॒यज॑ उप॒यज॑ उप॒यज॑ति प्र॒जाः । \newline
15. उ॒प॒यज॒ इत्यु॑प - यजः॑ । \newline
16. उ॒प॒यज॑ति प्र॒जाः प्र॒जा उ॑प॒यज॑ त्युप॒यज॑ति प्र॒जा ए॒वैव प्र॒जा उ॑प॒यज॑ त्युप॒यज॑ति प्र॒जा ए॒व । \newline
17. उ॒प॒यज॒तीत्यु॑प - यज॑ति । \newline
18. प्र॒जा ए॒वैव प्र॒जाः प्र॒जा ए॒व तत् तदे॒व प्र॒जाः प्र॒जा ए॒व तत् । \newline
19. प्र॒जा इति॑ प्र - जाः । \newline
20. ए॒व तत् तदे॒वैव तद् यज॑मानो॒ यज॑मान॒ स्तदे॒वैव तद् यज॑मानः । \newline
21. तद् यज॑मानो॒ यज॑मान॒ स्तत् तद् यज॑मानः सृजते सृजते॒ यज॑मान॒ स्तत् तद् यज॑मानः सृजते । \newline
22. यज॑मानः सृजते सृजते॒ यज॑मानो॒ यज॑मानः सृजते जघना॒र्द्धाज् ज॑घना॒र्द्धाथ् सृ॑जते॒ यज॑मानो॒ यज॑मानः सृजते जघना॒र्द्धात् । \newline
23. सृ॒ज॒ते॒ ज॒घ॒ना॒र्द्धाज् ज॑घना॒र्द्धाथ् सृ॑जते सृजते जघना॒र्द्धा दवाव॑ जघना॒र्द्धाथ् सृ॑जते सृजते जघना॒र्द्धा दव॑ । \newline
24. ज॒घ॒ना॒र्द्धा दवाव॑ जघना॒र्द्धाज् ज॑घना॒र्द्धा दव॑ द्यति द्य॒त्यव॑ जघना॒र्द्धाज् ज॑घना॒र्द्धा दव॑ द्यति । \newline
25. ज॒घ॒ना॒र्द्धादिति॑ जघन - अ॒र्द्धात् । \newline
26. अव॑ द्यति द्य॒त्य वाव॑ द्यति जघना॒र्द्धाज् ज॑घना॒र्द्धाद् द्य॒त्य वाव॑ द्यति जघना॒र्द्धात् । \newline
27. द्य॒ति॒ ज॒घ॒ना॒र्द्धाज् ज॑घना॒र्द्धाद् द्य॑ति द्यति जघना॒र्द्धाद्धि हि ज॑घना॒र्द्धाद् द्य॑ति द्यति जघना॒र्द्धाद्धि । \newline
28. ज॒घ॒ना॒र्द्धाद्धि हि ज॑घना॒र्द्धाज् ज॑घना॒र्द्धाद्धि प्र॒जाः प्र॒जा हि ज॑घना॒र्द्धाज् ज॑घना॒र्द्धाद्धि प्र॒जाः । \newline
29. ज॒घ॒ना॒र्द्धादिति॑ जघन - अ॒र्द्धात् । \newline
30. हि प्र॒जाः प्र॒जा हि हि प्र॒जाः प्र॒जाय॑न्ते प्र॒जाय॑न्ते प्र॒जा हि हि प्र॒जाः प्र॒जाय॑न्ते । \newline
31. प्र॒जाः प्र॒जाय॑न्ते प्र॒जाय॑न्ते प्र॒जाः प्र॒जाः प्र॒जाय॑न्ते स्थविम॒तः स्थ॑विम॒तः प्र॒जाय॑न्ते प्र॒जाः प्र॒जाः प्र॒जाय॑न्ते स्थविम॒तः । \newline
32. प्र॒जा इति॑ प्र - जाः । \newline
33. प्र॒जाय॑न्ते स्थविम॒तः स्थ॑विम॒तः प्र॒जाय॑न्ते प्र॒जाय॑न्ते स्थविम॒तो ऽवाव॑ स्थविम॒तः प्र॒जाय॑न्ते प्र॒जाय॑न्ते स्थविम॒तो ऽव॑ । \newline
34. प्र॒जाय॑न्त॒ इति॑ प्र - जाय॑न्ते । \newline
35. स्थ॒वि॒म॒तो ऽवाव॑ स्थविम॒तः स्थ॑विम॒तो ऽव॑ द्यति द्य॒त्यव॑ स्थविम॒तः स्थ॑विम॒तो ऽव॑ द्यति । \newline
36. अव॑ द्यति द्य॒त्यवाव॑ द्यति स्थविम॒तः स्थ॑विम॒तो द्य॒त्यवाव॑ द्यति स्थविम॒तः । \newline
37. द्य॒ति॒ स्थ॒वि॒म॒तः स्थ॑विम॒तो द्य॑ति द्यति स्थविम॒तो हि हि स्थ॑विम॒तो द्य॑ति द्यति स्थविम॒तो हि । \newline
38. स्थ॒वि॒म॒तो हि हि स्थ॑विम॒तः स्थ॑विम॒तो हि प्र॒जाः प्र॒जा हि स्थ॑विम॒तः स्थ॑विम॒तो हि प्र॒जाः । \newline
39. हि प्र॒जाः प्र॒जा हि हि प्र॒जाः प्र॒जाय॑न्ते प्र॒जाय॑न्ते प्र॒जा हि हि प्र॒जाः प्र॒जाय॑न्ते । \newline
40. प्र॒जाः प्र॒जाय॑न्ते प्र॒जाय॑न्ते प्र॒जाः प्र॒जाः प्र॒जाय॒न्ते ऽस॑म्भिन्द॒न् नस॑म्भिन्दन् प्र॒जाय॑न्ते प्र॒जाः प्र॒जाः प्र॒जाय॒न्ते ऽस॑म्भिन्दन्न् । \newline
41. प्र॒जा इति॑ प्र - जाः । \newline
42. प्र॒जाय॒न्ते ऽस॑म्भिन्द॒न् नस॑म्भिन्दन् प्र॒जाय॑न्ते प्र॒जाय॒न्ते ऽस॑म्भिन्द॒न् नवा वास॑म्भिन्दन् प्र॒जाय॑न्ते प्र॒जाय॒न्ते ऽस॑म्भिन्द॒न् नव॑ । \newline
43. प्र॒जाय॑न्त॒ इति॑ प्र - जाय॑न्ते । \newline
44. अस॑म्भिन्द॒न् नवा वास॑म्भिन्द॒न् नस॑म्भिन्द॒न् नव॑ द्यति द्य॒त्यवास॑म्भिन्द॒न् नस॑म्भिन्द॒न् नव॑ द्यति । \newline
45. अस॑म्भिन्द॒न्नित्यस᳚म् - भि॒न्द॒न्न् । \newline
46. अव॑ द्यति द्य॒त्यवाव॑ द्यति प्रा॒णाना᳚म् प्रा॒णाना᳚म् द्य॒त्यवाव॑ द्यति प्रा॒णाना᳚म् । \newline
47. द्य॒ति॒ प्रा॒णाना᳚म् प्रा॒णाना᳚म् द्यति द्यति प्रा॒णाना॒ मस॑म्भेदा॒या स॑म्भेदाय प्रा॒णाना᳚म् द्यति द्यति प्रा॒णाना॒ मस॑म्भेदाय । \newline
48. प्रा॒णाना॒ मस॑म्भेदा॒या स॑म्भेदाय प्रा॒णाना᳚म् प्रा॒णाना॒ मस॑म्भेदाय॒ न नास॑म्भेदाय प्रा॒णाना᳚म् प्रा॒णाना॒ मस॑म्भेदाय॒ न । \newline
49. प्रा॒णाना॒मिति॑ प्र - अ॒नाना᳚म् । \newline
50. अस॑म्भेदाय॒ न नास॑म्भेदा॒या स॑म्भेदाय॒ न प॒र्याव॑र्तयति प॒र्याव॑र्तयति॒ नास॑म्भेदा॒या स॑म्भेदाय॒ न प॒र्याव॑र्तयति । \newline
51. अस॑म्भेदा॒येत्यस᳚म् - भे॒दा॒य॒ । \newline
52. न प॒र्याव॑र्तयति प॒र्याव॑र्तयति॒ न न प॒र्याव॑र्तयति॒ यद् यत् प॒र्याव॑र्तयति॒ न न प॒र्याव॑र्तयति॒ यत् । \newline
53. प॒र्याव॑र्तयति॒ यद् यत् प॒र्याव॑र्तयति प॒र्याव॑र्तयति॒ यत् प॑र्याव॒र्तये᳚त् पर्याव॒र्तये॒द् यत् प॒र्याव॑र्तयति प॒र्याव॑र्तयति॒ यत् प॑र्याव॒र्तये᳚त् । \newline
54. प॒र्याव॑र्तय॒तीति॑ परि - आव॑र्तयति । \newline
55. यत् प॑र्याव॒र्तये᳚त् पर्याव॒र्तये॒द् यद् यत् प॑र्याव॒र्तये॑ दुदाव॒र्त उ॑दाव॒र्तः प॑र्याव॒र्तये॒द् यद् यत् प॑र्याव॒र्तये॑ दुदाव॒र्तः । \newline
56. प॒र्या॒व॒र्तये॑ दुदाव॒र्त उ॑दाव॒र्तः प॑र्याव॒र्तये᳚त् पर्याव॒र्तये॑ दुदाव॒र्तः प्र॒जाः प्र॒जा उ॑दाव॒र्तः प॑र्याव॒र्तये᳚त् पर्याव॒र्तये॑ दुदाव॒र्तः प्र॒जाः । \newline
57. प॒र्या॒व॒र्तये॒दिति॑ परि - आ॒व॒र्तये᳚त् । \newline
58. उ॒दा॒व॒र्तः प्र॒जाः प्र॒जा उ॑दाव॒र्त उ॑दाव॒र्तः प्र॒जा ग्राहु॑को॒ ग्राहु॑कः प्र॒जा उ॑दाव॒र्त उ॑दाव॒र्तः प्र॒जा ग्राहु॑कः । \newline
59. प्र॒जा ग्राहु॑को॒ ग्राहु॑कः प्र॒जाः प्र॒जा ग्राहु॑कः स्याथ् स्या॒द् ग्राहु॑कः प्र॒जाः प्र॒जा ग्राहु॑कः स्यात् । \newline
60. प्र॒जा इति॑ प्र - जाः । \newline
61. ग्राहु॑कः स्याथ् स्या॒द् ग्राहु॑को॒ ग्राहु॑कः स्याथ् समु॒द्रꣳ स॑मु॒द्रꣳ स्या॒द् ग्राहु॑को॒ ग्राहु॑कः स्याथ् समु॒द्रम् । \newline
62. स्या॒थ् स॒मु॒द्रꣳ स॑मु॒द्रꣳ स्या᳚थ् स्याथ् समु॒द्रम् ग॑च्छ गच्छ समु॒द्रꣳ स्या᳚थ् स्याथ् समु॒द्रम् ग॑च्छ । \newline
63. स॒मु॒द्रम् ग॑च्छ गच्छ समु॒द्रꣳ स॑मु॒द्रम् ग॑च्छ॒ स्वाहा॒ स्वाहा॑ गच्छ समु॒द्रꣳ स॑मु॒द्रम् ग॑च्छ॒ स्वाहा᳚ । \newline
64. ग॒च्छ॒ स्वाहा॒ स्वाहा॑ गच्छ गच्छ॒ स्वाहेतीति॒ स्वाहा॑ गच्छ गच्छ॒ स्वाहेति॑ । \newline
65. स्वाहे तीति॒ स्वाहा॒ स्वाहे त्या॑हा॒ हेति॒ स्वाहा॒ स्वाहे त्या॑ह । \newline
66. इत्या॑हा॒हे तीत्या॑ह॒ रेतो॒ रेत॑ आ॒हे तीत्या॑ह॒ रेतः॑ । \newline
67. आ॒ह॒ रेतो॒ रेत॑ आहाह॒ रेत॑ ए॒वैव रेत॑ आहाह॒ रेत॑ ए॒व । \newline
68. रेत॑ ए॒वैव रेतो॒ रेत॑ ए॒व तत् तदे॒व रेतो॒ रेत॑ ए॒व तत् । \newline
\pagebreak
\markright{ TS 6.4.1.2  \hfill https://www.vedavms.in \hfill}

\section{ TS 6.4.1.2 }

\textbf{TS 6.4.1.2 } \newline
\textbf{Samhita Paata} \newline

ए॒व तद् द॑धात्य॒न्तरि॑क्षं गच्छ॒ स्वाहेत्या॑हा॒ऽन्तरि॑क्षेणै॒वास्मै᳚ प्र॒जाः प्र ज॑नयत्य॒न्तरि॑क्षꣳ॒॒ ह्यनु॑ प्र॒जाः प्र॒जाय॑न्ते दे॒वꣳ स॑वि॒तारं॑ गच्छ॒ स्वाहेत्या॑ह सवि॒तृप्र॑सूत ए॒वास्मै᳚ प्र॒जाः प्र ज॑नयत्य-होरा॒त्रे ग॑च्छ॒ स्वाहेत्या॑हा-होरा॒त्राभ्या॑-मे॒वास्मै᳚ प्र॒जाः प्र ज॑नयत्य-होरा॒त्रे ह्यनु॑ प्र॒जाः प्र॒जाय॑न्ते मि॒त्रावरु॑णौ गच्छ॒ स्वाहे- [  ] \newline

\textbf{Pada Paata} \newline

ए॒व । तत् । द॒धा॒ति॒ । अ॒न्तरि॑क्षम् । ग॒च्छ॒ । स्वाहा᳚ । इति॑ । आ॒ह॒ । अ॒न्तरि॑क्षेण । ए॒व । अ॒स्मै॒ । प्र॒जा इति॑ प्र - जाः । प्रेति॑ । ज॒न॒य॒ति॒ । अ॒न्तरि॑क्षम् । हि । अन्विति॑ । प्र॒जा इति॑ प्र - जाः । प्र॒जाय॑न्त॒ इति॑ प्र - जाय॑न्ते । दे॒वम् । स॒वि॒तार᳚म् । ग॒च्छ॒ । स्वाहा᳚ । इति॑ । आ॒ह॒ । स॒वि॒तृप्र॑सूत॒ इति॑ सवि॒तृ - प्र॒सू॒तः॒ । ए॒व । अ॒स्मै॒ । प्र॒जा इति॑ प्र - जाः । प्रेति॑ । ज॒न॒य॒ति॒ । अ॒हो॒रा॒त्रे इत्य॑हः - रा॒त्रे । ग॒च्छ॒ । स्वाहा᳚ । इति॑ । आ॒ह॒ । अ॒हो॒रा॒त्राभ्या॒मित्य॑हः - रा॒त्राभ्या᳚म् । ए॒व । अ॒स्मै॒ । प्र॒जा इति॑ प्र-जाः । प्रेति॑ । ज॒न॒य॒ति॒ । अ॒हो॒रा॒त्रे इत्य॑हः-रा॒त्रे । हि । अन्विति॑ । प्र॒जा इति॑ प्र - जाः । प्र॒जाय॑न्त॒ इति॑ प्र-जाय॑न्ते । मि॒त्रावरु॑णा॒विति॑ मि॒त्रा-वरु॑णौ । ग॒च्छ॒ । स्वाहा᳚ ।  \newline


\textbf{Krama Paata} \newline

ए॒व तत् । तद् द॑धाति । द॒धा॒त्य॒न्तरि॑क्षम् । अ॒न्तरि॑क्षम् गच्छ । ग॒च्छ॒ स्वाहा᳚ । स्वाहेति॑ । इत्या॑ह । आ॒हा॒न्तरि॑क्षेण । अ॒न्तरि॑क्षेणै॒व । ए॒वास्मै᳚ । अ॒स्मै॒ प्र॒जाः । प्र॒जाः प्र । प्र॒जा इति॑ प्र - जाः । प्र ज॑नयति । ज॒न॒य॒त्य॒न्तरि॑क्षम् । अ॒न्तरि॑क्षꣳ॒॒ हि । ह्यनु॑ । अनु॑ प्र॒जाः । प्र॒जाः प्र॒जाय॑न्ते । प्र॒जा इति॑ प्र - जाः । प्र॒जाय॑न्ते दे॒वम् । प्र॒जाय॑न्त॒ इति॑ प्र - जाय॑न्ते । दे॒वꣳ स॑वि॒तार᳚म् । स॒वि॒तार॑म् गच्छ । ग॒च्छ॒ स्वाहा᳚ । स्वाहेति॑ । इत्या॑ह । आ॒ह॒ स॒वि॒तृप्र॑सूतः । स॒वि॒तृप्र॑सूत ए॒व । स॒वि॒तृप्र॑सूत॒ इति॑ सवि॒तृ - प्र॒सू॒तः॒ । ए॒वास्मै᳚ । अ॒स्मै॒ प्र॒जाः । प्र॒जाः प्र । प्र॒जा इति॑ प्र - जाः । प्र ज॑नयति । ज॒न॒य॒त्य॒हो॒रा॒त्रे । अ॒हो॒रा॒त्रे ग॑च्छ । अ॒हो॒रा॒त्रे इत्य॑हः - रा॒त्रे । ग॒च्छ॒ स्वाहा᳚ । स्वाहेति॑ । इत्या॑ह । आ॒हा॒हो॒रा॒त्राभ्या᳚म् । अ॒हो॒रा॒त्राभ्या॑मे॒व । अ॒हो॒रा॒त्राभ्या॒मित्य॑हः - रा॒त्राभ्या᳚म् । ए॒वास्मै᳚ । अ॒स्मै॒ प्र॒जाः । प्र॒जाः प्र । प्र॒जा इति॑ प्र - जाः । प्र ज॑नयति । ज॒न॒य॒त्य॒हो॒रा॒त्रे । अ॒हो॒रा॒त्रे हि । अ॒हो॒रा॒त्रे इत्य॑हः - रा॒त्रे । ह्यनु॑ । अनु॑ प्र॒जाः । प्र॒जाः प्र॒जाय॑न्ते । प्र॒जा इति॑ प्र - जाः । प्र॒जाय॑न्ते मि॒त्रावरु॑णौ । प्र॒जाय॑न्त॒ इति॑ प्र - जाय॑न्ते । मि॒त्रावरु॑णौ गच्छ । मि॒त्रावरु॑णा॒विति॑ मि॒त्रा - वरु॑णौ । ग॒च्छ॒ स्वाहा᳚ । स्वाहेति॑ \newline

\textbf{Jatai Paata} \newline

1. ए॒व तत् तदे॒ वैव तत् । \newline
2. तद् द॑धाति दधाति॒ तत् तद् द॑धाति । \newline
3. द॒धा॒ त्य॒न्तरि॑क्ष म॒न्तरि॑क्षम् दधाति दधा त्य॒न्तरि॑क्षम् । \newline
4. अ॒न्तरि॑क्षम् गच्छ गच्छा॒ न्तरि॑क्ष म॒न्तरि॑क्षम् गच्छ । \newline
5. ग॒च्छ॒ स्वाहा॒ स्वाहा॑ गच्छ गच्छ॒ स्वाहा᳚ । \newline
6. स्वाहे तीति॒ स्वाहा॒ स्वाहेति॑ । \newline
7. इत्या॑हा॒हे तीत्या॑ह । \newline
8. आ॒हा॒ न्तरि॑क्षेणा॒ न्तरि॑क्षेणा हाहा॒ न्तरि॑क्षेण । \newline
9. अ॒न्तरि॑क्षे णै॒वैवा न्तरि॑क्षेणा॒ न्तरि॑क्षेणै॒व । \newline
10. ए॒वास्मा॑ अस्मा ए॒वै वास्मै᳚ । \newline
11. अ॒स्मै॒ प्र॒जाः प्र॒जा अ॑स्मा अस्मै प्र॒जाः । \newline
12. प्र॒जाः प्र प्र प्र॒जाः प्र॒जाः प्र । \newline
13. प्र॒जा इति॑ प्र - जाः । \newline
14. प्र ज॑नयति जनयति॒ प्र प्र ज॑नयति । \newline
15. ज॒न॒य॒ त्य॒न्तरि॑क्ष म॒न्तरि॑क्षम् जनयति जनय त्य॒न्तरि॑क्षम् । \newline
16. अ॒न्तरि॑क्षꣳ॒॒ हि ह्य॑न्तरि॑क्ष म॒न्तरि॑क्षꣳ॒॒ हि । \newline
17. ह्यन् वनु॒ हि ह्यनु॑ । \newline
18. अनु॑ प्र॒जाः प्र॒जा अन् वनु॑ प्र॒जाः । \newline
19. प्र॒जाः प्र॒जाय॑न्ते प्र॒जाय॑न्ते प्र॒जाः प्र॒जाः प्र॒जाय॑न्ते । \newline
20. प्र॒जा इति॑ प्र - जाः । \newline
21. प्र॒जाय॑न्ते दे॒वम् दे॒वम् प्र॒जाय॑न्ते प्र॒जाय॑न्ते दे॒वम् । \newline
22. प्र॒जाय॑न्त॒ इति॑ प्र - जाय॑न्ते । \newline
23. दे॒वꣳ स॑वि॒तारꣳ॑ सवि॒तार॑म् दे॒वम् दे॒वꣳ स॑वि॒तार᳚म् । \newline
24. स॒वि॒तार॑म् गच्छ गच्छ सवि॒तारꣳ॑ सवि॒तार॑म् गच्छ । \newline
25. ग॒च्छ॒ स्वाहा॒ स्वाहा॑ गच्छ गच्छ॒ स्वाहा᳚ । \newline
26. स्वाहे तीति॒ स्वाहा॒ स्वाहेति॑ । \newline
27. इत्या॑हा॒हे तीत्या॑ह । \newline
28. आ॒ह॒ स॒वि॒तृप्र॑सूतः सवि॒तृप्र॑सूत आहाह सवि॒तृप्र॑सूतः । \newline
29. स॒वि॒तृप्र॑सूत ए॒वैव स॑वि॒तृप्र॑सूतः सवि॒तृप्र॑सूत ए॒व । \newline
30. स॒वि॒तृप्र॑सूत॒ इति॑ सवि॒तृ - प्र॒सू॒तः॒ । \newline
31. ए॒वास्मा॑ अस्मा ए॒वैवास्मै᳚ । \newline
32. अ॒स्मै॒ प्र॒जाः प्र॒जा अ॑स्मा अस्मै प्र॒जाः । \newline
33. प्र॒जाः प्र प्र प्र॒जाः प्र॒जाः प्र । \newline
34. प्र॒जा इति॑ प्र - जाः । \newline
35. प्र ज॑नयति जनयति॒ प्र प्र ज॑नयति । \newline
36. ज॒न॒य॒ त्य॒हो॒रा॒त्रे अ॑होरा॒त्रे ज॑नयति जनय त्यहोरा॒त्रे । \newline
37. अ॒हो॒रा॒त्रे ग॑च्छ गच्छा होरा॒त्रे अ॑होरा॒त्रे ग॑च्छ । \newline
38. अ॒हो॒रा॒त्रे इत्य॑हः - रा॒त्रे । \newline
39. ग॒च्छ॒ स्वाहा॒ स्वाहा॑ गच्छ गच्छ॒ स्वाहा᳚ । \newline
40. स्वाहेतीति॒ स्वाहा॒ स्वाहेति॑ । \newline
41. इत्या॑हा॒हे तीत्या॑ह । \newline
42. आ॒हा॒ हो॒रा॒त्राभ्या॑ महोरा॒त्राभ्या॑ माहाहा होरा॒त्राभ्या᳚म् । \newline
43. अ॒हो॒रा॒त्राभ्या॑ मे॒वैवा हो॑रा॒त्राभ्या॑ महोरा॒त्राभ्या॑ मे॒व । \newline
44. अ॒हो॒रा॒त्राभ्या॒मित्य॑हः - रा॒त्राभ्या᳚म् । \newline
45. ए॒वास्मा॑ अस्मा ए॒वै वास्मै᳚ । \newline
46. अ॒स्मै॒ प्र॒जाः प्र॒जा अ॑स्मा अस्मै प्र॒जाः । \newline
47. प्र॒जाः प्र प्र प्र॒जाः प्र॒जाः प्र । \newline
48. प्र॒जा इति॑ प्र - जाः । \newline
49. प्र ज॑नयति जनयति॒ प्र प्र ज॑नयति । \newline
50. ज॒न॒य॒ त्य॒हो॒रा॒त्रे अ॑होरा॒त्रे ज॑नयति जनय त्यहोरा॒त्रे । \newline
51. अ॒हो॒रा॒त्रे हि ह्य॑होरा॒त्रे अ॑होरा॒त्रे हि । \newline
52. अ॒हो॒रा॒त्रे इत्य॑हः - रा॒त्रे । \newline
53. ह्यन् वनु॒ हि ह्यनु॑ । \newline
54. अनु॑ प्र॒जाः प्र॒जा अन् वनु॑ प्र॒जाः । \newline
55. प्र॒जाः प्र॒जाय॑न्ते प्र॒जाय॑न्ते प्र॒जाः प्र॒जाः प्र॒जाय॑न्ते । \newline
56. प्र॒जा इति॑ प्र - जाः । \newline
57. प्र॒जाय॑न्ते मि॒त्रावरु॑णौ मि॒त्रावरु॑णौ प्र॒जाय॑न्ते प्र॒जाय॑न्ते मि॒त्रावरु॑णौ । \newline
58. प्र॒जाय॑न्त॒ इति॑ प्र - जाय॑न्ते । \newline
59. मि॒त्रावरु॑णौ गच्छ गच्छ मि॒त्रावरु॑णौ मि॒त्रावरु॑णौ गच्छ । \newline
60. मि॒त्रावरु॑णा॒विति॑ मि॒त्रा - वरु॑णौ । \newline
61. ग॒च्छ॒ स्वाहा॒ स्वाहा॑ गच्छ गच्छ॒ स्वाहा᳚ । \newline
62. स्वाहेतीति॒ स्वाहा॒ स्वाहेति॑ । \newline

\textbf{Ghana Paata } \newline

1. ए॒व तत् तदे॒ वैव तद् द॑धाति दधाति॒ तदे॒ वैव तद् द॑धाति । \newline
2. तद् द॑धाति दधाति॒ तत् तद् द॑धा त्य॒न्तरि॑क्ष म॒न्तरि॑क्षम् दधाति॒ तत् तद् द॑धा त्य॒न्तरि॑क्षम् । \newline
3. द॒धा॒ त्य॒न्तरि॑क्ष म॒न्तरि॑क्षम् दधाति दधा त्य॒न्तरि॑क्षम् गच्छ गच्छा॒न्तरि॑क्षम् दधाति दधा त्य॒न्तरि॑क्षम् गच्छ । \newline
4. अ॒न्तरि॑क्षम् गच्छ गच्छा॒ न्तरि॑क्ष म॒न्तरि॑क्षम् गच्छ॒ स्वाहा॒ स्वाहा॑ गच्छा॒ न्तरि॑क्ष म॒न्तरि॑क्षम् गच्छ॒ स्वाहा᳚ । \newline
5. ग॒च्छ॒ स्वाहा॒ स्वाहा॑ गच्छ गच्छ॒ स्वाहे तीति॒ स्वाहा॑ गच्छ गच्छ॒ स्वाहेति॑ । \newline
6. स्वाहेतीति॒ स्वाहा॒ स्वाहे त्या॑हा॒हेति॒ स्वाहा॒ स्वाहे त्या॑ह । \newline
7. इत्या॑हा॒हे तीत्या॑हा॒ न्तरि॑क्षेणा॒ न्तरि॑क्षेणा॒ हेतीत्या॑हा॒ न्तरि॑क्षेण । \newline
8. आ॒हा॒ न्तरि॑क्षेणा॒ न्तरि॑क्षेणा हाहा॒ न्तरि॑क्षे णै॒वैवा न्तरि॑क्षेणा हाहा॒ न्तरि॑क्षेणै॒व । \newline
9. अ॒न्तरि॑क्षे णै॒वैवा न्तरि॑क्षेणा॒ न्तरि॑क्षेणै॒ वास्मा॑ अस्मा ए॒वा न्तरि॑क्षेणा॒ न्तरि॑क्षे णै॒वास्मै᳚ । \newline
10. ए॒वास्मा॑ अस्मा ए॒वै वास्मै᳚ प्र॒जाः प्र॒जा अ॑स्मा ए॒वै वास्मै᳚ प्र॒जाः । \newline
11. अ॒स्मै॒ प्र॒जाः प्र॒जा अ॑स्मा अस्मै प्र॒जाः प्र प्र प्र॒जा अ॑स्मा अस्मै प्र॒जाः प्र । \newline
12. प्र॒जाः प्र प्र प्र॒जाः प्र॒जाः प्र ज॑नयति जनयति॒ प्र प्र॒जाः प्र॒जाः प्र ज॑नयति । \newline
13. प्र॒जा इति॑ प्र - जाः । \newline
14. प्र ज॑नयति जनयति॒ प्र प्र ज॑नय त्य॒न्तरि॑क्ष म॒न्तरि॑क्षम् जनयति॒ प्र प्र ज॑नय त्य॒न्तरि॑क्षम् । \newline
15. ज॒न॒य॒ त्य॒न्तरि॑क्ष म॒न्तरि॑क्षम् जनयति जनय त्य॒न्तरि॑क्षꣳ॒॒ हि ह्य॑न्तरि॑क्षम् जनयति जनय त्य॒न्तरि॑क्षꣳ॒॒ हि । \newline
16. अ॒न्तरि॑क्षꣳ॒॒ हि ह्य॑न्तरि॑क्ष म॒न्तरि॑क्षꣳ॒॒ ह्यन् वनु॒ ह्य॑न्तरि॑क्ष म॒न्तरि॑क्षꣳ॒॒ ह्यनु॑ । \newline
17. ह्यन् वनु॒ हि ह्यनु॑ प्र॒जाः प्र॒जा अनु॒ हि ह्यनु॑ प्र॒जाः । \newline
18. अनु॑ प्र॒जाः प्र॒जा अन् वनु॑ प्र॒जाः प्र॒जाय॑न्ते प्र॒जाय॑न्ते प्र॒जा अन् वनु॑ प्र॒जाः प्र॒जाय॑न्ते । \newline
19. प्र॒जाः प्र॒जाय॑न्ते प्र॒जाय॑न्ते प्र॒जाः प्र॒जाः प्र॒जाय॑न्ते दे॒वम् दे॒वम् प्र॒जाय॑न्ते प्र॒जाः प्र॒जाः प्र॒जाय॑न्ते दे॒वम् । \newline
20. प्र॒जा इति॑ प्र - जाः । \newline
21. प्र॒जाय॑न्ते दे॒वम् दे॒वम् प्र॒जाय॑न्ते प्र॒जाय॑न्ते दे॒वꣳ स॑वि॒तारꣳ॑ सवि॒तार॑म् दे॒वम् प्र॒जाय॑न्ते प्र॒जाय॑न्ते दे॒वꣳ स॑वि॒तार᳚म् । \newline
22. प्र॒जाय॑न्त॒ इति॑ प्र - जाय॑न्ते । \newline
23. दे॒वꣳ स॑वि॒तारꣳ॑ सवि॒तार॑म् दे॒वम् दे॒वꣳ स॑वि॒तार॑म् गच्छ गच्छ सवि॒तार॑म् दे॒वम् दे॒वꣳ स॑वि॒तार॑म् गच्छ । \newline
24. स॒वि॒तार॑म् गच्छ गच्छ सवि॒तारꣳ॑ सवि॒तार॑म् गच्छ॒ स्वाहा॒ स्वाहा॑ गच्छ सवि॒तारꣳ॑ सवि॒तार॑म् गच्छ॒ स्वाहा᳚ । \newline
25. ग॒च्छ॒ स्वाहा॒ स्वाहा॑ गच्छ गच्छ॒ स्वाहे तीति॒ स्वाहा॑ गच्छ गच्छ॒ स्वाहेति॑ । \newline
26. स्वाहेतीति॒ स्वाहा॒ स्वाहे त्या॑हा॒ हेति॒ स्वाहा॒ स्वाहे त्या॑ह । \newline
27. इत्या॑हा॒हे तीत्या॑ह सवि॒तृप्र॑सूतः सवि॒तृप्र॑सूत आ॒हे तीत्या॑ह सवि॒तृप्र॑सूतः । \newline
28. आ॒ह॒ स॒वि॒तृप्र॑सूतः सवि॒तृप्र॑सूत आहाह सवि॒तृप्र॑सूत ए॒वैव स॑वि॒तृप्र॑सूत आहाह सवि॒तृप्र॑सूत ए॒व । \newline
29. स॒वि॒तृप्र॑सूत ए॒वैव स॑वि॒तृप्र॑सूतः सवि॒तृप्र॑सूत ए॒वास्मा॑ अस्मा ए॒व स॑वि॒तृप्र॑सूतः सवि॒तृप्र॑सूत ए॒वास्मै᳚ । \newline
30. स॒वि॒तृप्र॑सूत॒ इति॑ सवि॒तृ - प्र॒सू॒तः॒ । \newline
31. ए॒वास्मा॑ अस्मा ए॒वै वास्मै᳚ प्र॒जाः प्र॒जा अ॑स्मा ए॒वै वास्मै᳚ प्र॒जाः । \newline
32. अ॒स्मै॒ प्र॒जाः प्र॒जा अ॑स्मा अस्मै प्र॒जाः प्र प्र प्र॒जा अ॑स्मा अस्मै प्र॒जाः प्र । \newline
33. प्र॒जाः प्र प्र प्र॒जाः प्र॒जाः प्र ज॑नयति जनयति॒ प्र प्र॒जाः प्र॒जाः प्र ज॑नयति । \newline
34. प्र॒जा इति॑ प्र - जाः । \newline
35. प्र ज॑नयति जनयति॒ प्र प्र ज॑नय त्यहोरा॒त्रे अ॑होरा॒त्रे ज॑नयति॒ प्र प्र ज॑नय त्यहोरा॒त्रे । \newline
36. ज॒न॒य॒ त्य॒हो॒रा॒त्रे अ॑होरा॒त्रे ज॑नयति जनय त्यहोरा॒त्रे ग॑च्छ गच्छाहोरा॒त्रे ज॑नयति जनय त्यहोरा॒त्रे ग॑च्छ । \newline
37. अ॒हो॒रा॒त्रे ग॑च्छ गच्छा होरा॒त्रे अ॑होरा॒त्रे ग॑च्छ॒ स्वाहा॒ स्वाहा॑ गच्छा होरा॒त्रे अ॑होरा॒त्रे ग॑च्छ॒ स्वाहा᳚ । \newline
38. अ॒हो॒रा॒त्रे इत्य॑हः - रा॒त्रे । \newline
39. ग॒च्छ॒ स्वाहा॒ स्वाहा॑ गच्छ गच्छ॒ स्वाहेतीति॒ स्वाहा॑ गच्छ गच्छ॒ स्वाहेति॑ । \newline
40. स्वाहेतीति॒ स्वाहा॒ स्वाहेत्या॑हा॒ हेति॒ स्वाहा॒ स्वाहेत्या॑ह । \newline
41. इत्या॑हा॒हे तीत्या॑हा होरा॒त्राभ्या॑ महोरा॒त्राभ्या॑ मा॒हे तीत्या॑हा होरा॒त्राभ्या᳚म् । \newline
42. आ॒हा॒ हो॒रा॒त्राभ्या॑ महोरा॒त्राभ्या॑ माहाहा होरा॒त्राभ्या॑ मे॒वैवा हो॑रा॒त्राभ्या॑ माहाहा होरा॒त्राभ्या॑ मे॒व । \newline
43. अ॒हो॒रा॒त्राभ्या॑ मे॒वैवा हो॑रा॒त्राभ्या॑ महोरा॒त्राभ्या॑ मे॒वास्मा॑ अस्मा ए॒वा हो॑रा॒त्राभ्या॑ महोरा॒त्राभ्या॑ मे॒वास्मै᳚ । \newline
44. अ॒हो॒रा॒त्राभ्या॒मित्य॑हः - रा॒त्राभ्या᳚म् । \newline
45. ए॒वास्मा॑ अस्मा ए॒वै वास्मै᳚ प्र॒जाः प्र॒जा अ॑स्मा ए॒वै वास्मै᳚ प्र॒जाः । \newline
46. अ॒स्मै॒ प्र॒जाः प्र॒जा अ॑स्मा अस्मै प्र॒जाः प्र प्र प्र॒जा अ॑स्मा अस्मै प्र॒जाः प्र । \newline
47. प्र॒जाः प्र प्र प्र॒जाः प्र॒जाः प्र ज॑नयति जनयति॒ प्र प्र॒जाः प्र॒जाः प्र ज॑नयति । \newline
48. प्र॒जा इति॑ प्र - जाः । \newline
49. प्र ज॑नयति जनयति॒ प्र प्र ज॑नय त्यहोरा॒त्रे अ॑होरा॒त्रे ज॑नयति॒ प्र प्र ज॑नय त्यहोरा॒त्रे । \newline
50. ज॒न॒य॒ त्य॒हो॒रा॒त्रे अ॑होरा॒त्रे ज॑नयति जनय त्यहोरा॒त्रे हि ह्य॑होरा॒त्रे ज॑नयति जनय त्यहोरा॒त्रे हि । \newline
51. अ॒हो॒रा॒त्रे हि ह्य॑होरा॒त्रे अ॑होरा॒त्रे ह्यन् वनु॒ ह्य॑होरा॒त्रे अ॑होरा॒त्रे ह्यनु॑ । \newline
52. अ॒हो॒रा॒त्रे इत्य॑हः - रा॒त्रे । \newline
53. ह्यन् वनु॒ हि ह्यनु॑ प्र॒जाः प्र॒जा अनु॒ हि ह्यनु॑ प्र॒जाः । \newline
54. अनु॑ प्र॒जाः प्र॒जा अन् वनु॑ प्र॒जाः प्र॒जाय॑न्ते प्र॒जाय॑न्ते प्र॒जा अन् वनु॑ प्र॒जाः प्र॒जाय॑न्ते । \newline
55. प्र॒जाः प्र॒जाय॑न्ते प्र॒जाय॑न्ते प्र॒जाः प्र॒जाः प्र॒जाय॑न्ते मि॒त्रावरु॑णौ मि॒त्रावरु॑णौ प्र॒जाय॑न्ते प्र॒जाः प्र॒जाः प्र॒जाय॑न्ते मि॒त्रावरु॑णौ । \newline
56. प्र॒जा इति॑ प्र - जाः । \newline
57. प्र॒जाय॑न्ते मि॒त्रावरु॑णौ मि॒त्रावरु॑णौ प्र॒जाय॑न्ते प्र॒जाय॑न्ते मि॒त्रावरु॑णौ गच्छ गच्छ मि॒त्रावरु॑णौ प्र॒जाय॑न्ते प्र॒जाय॑न्ते मि॒त्रावरु॑णौ गच्छ । \newline
58. प्र॒जाय॑न्त॒ इति॑ प्र - जाय॑न्ते । \newline
59. मि॒त्रावरु॑णौ गच्छ गच्छ मि॒त्रावरु॑णौ मि॒त्रावरु॑णौ गच्छ॒ स्वाहा॒ स्वाहा॑ गच्छ मि॒त्रावरु॑णौ मि॒त्रावरु॑णौ गच्छ॒ स्वाहा᳚ । \newline
60. मि॒त्रावरु॑णा॒विति॑ मि॒त्रा - वरु॑णौ । \newline
61. ग॒च्छ॒ स्वाहा॒ स्वाहा॑ गच्छ गच्छ॒ स्वाहेतीति॒ स्वाहा॑ गच्छ गच्छ॒ स्वाहेति॑ । \newline
62. स्वाहेतीति॒ स्वाहा॒ स्वाहे त्या॑हा॒हेति॒ स्वाहा॒ स्वाहे त्या॑ह । \newline
\pagebreak
\markright{ TS 6.4.1.3  \hfill https://www.vedavms.in \hfill}

\section{ TS 6.4.1.3 }

\textbf{TS 6.4.1.3 } \newline
\textbf{Samhita Paata} \newline

-त्या॑ह प्र॒जास्वे॒व प्रजा॑तासु प्राणापा॒नौ द॑धाति॒ सोमं॑ गच्छ॒ स्वाहेत्या॑ह सौ॒म्या हि दे॒वत॑या प्र॒जा य॒ज्ञ्ं ग॑च्छ॒ स्वाहेत्या॑ह प्र॒जा ए॒व य॒ज्ञियाः᳚ करोति॒ छन्दाꣳ॑सि गच्छ॒ स्वाहेत्या॑ह प॒शवो॒ वै छन्दाꣳ॑सि प॒शूने॒वाव॑ रुन्धे॒ द्यावा॑पृथि॒वी ग॑च्छ॒ स्वाहेत्या॑ह प्र॒जा ए॒व प्रजा॑ता॒ द्यावा॑पृथि॒वीभ्या॑मुभ॒यतः॒ परि॑ गृह्णाति॒ नभो॑- [  ] \newline

\textbf{Pada Paata} \newline

इति॑ । आ॒ह॒ । प्र॒जास्विति॑ प्र - जासु॑ । ए॒व । प्रजा॑ता॒स्विति॒ प्र - जा॒ता॒सु॒ । प्रा॒णा॒पा॒नाविति॑ प्राण - अ॒पा॒नौ । द॒धा॒ति॒ । सोम᳚म् । ग॒च्छ॒ । स्वाहा᳚ । इति॑ । आ॒ह॒ । सौ॒म्याः । हि । दे॒वत॑या । प्र॒जा इति॑ प्र - जाः । य॒ज्ञ्म् । ग॒च्छ॒ । स्वाहा᳚ । इति॑ । आ॒ह॒ । प्र॒जा इति॑ प्र-जाः । ए॒व । य॒ज्ञियाः᳚ । क॒रो॒ति॒ । छन्दाꣳ॑सि । ग॒च्छ॒ । स्वाहा᳚ । इति॑ । आ॒ह॒ । प॒शवः॑ । वै । छन्दाꣳ॑सि । प॒शून् । ए॒व । अवेति॑ । रु॒न्धे॒ । द्यावा॑पृथि॒वी इति॒ द्यावा᳚-पृ॒थि॒वी । ग॒च्छ॒ । स्वाहा᳚ । इति॑ । आ॒ह॒ । प्र॒जा इति॑ प्र - जाः । ए॒व । प्रजा॑ता॒ इति॒ प्र-जा॒ताः॒ । द्यावा॑पृथि॒वीभ्या॒मिति॒ द्यावा᳚ - पृ॒थि॒वीभ्या᳚म् । उ॒भ॒यतः॑ । परीति॑ । गृ॒ह्णा॒ति॒ । नभः॑ ।  \newline


\textbf{Krama Paata} \newline

इत्या॑ह । आ॒ह॒ प्र॒जासु॑ । प्र॒जास्वे॒व । प्र॒जास्विति॑ प्र - जासु॑ । ए॒व प्रजा॑तासु । प्रजा॑तासु प्राणापा॒नौ । प्रजा॑ता॒स्विति॒ प्र - जा॒ता॒सु॒ । प्रा॒णा॒पा॒नौ द॑धाति । प्रा॒णा॒पा॒नाविति॑ प्राण - अ॒पा॒नौ । द॒धा॒ति॒ सोम᳚म् । सोम॑म् गच्छ । ग॒च्छ॒ स्वाहा᳚ । स्वाहेति॑ । इत्या॑ह । आ॒ह॒ सौ॒म्याः । सौ॒म्या हि । हि दे॒वत॑या । दे॒वत॑या प्र॒जाः । प्र॒जा य॒ज्ञ्म् । प्र॒जा इति॑ प्र - जाः । य॒ज्ञ्म् ग॑च्छ । ग॒च्छ॒ स्वाहा᳚ । स्वाहेति॑ । इत्या॑ह । आ॒ह॒ प्र॒जाः । प्र॒जा ए॒व । प्र॒जा इति॑ प्र - जाः । ए॒व य॒ज्ञियाः᳚ । य॒ज्ञियाः᳚ करोति । क॒रो॒ति॒ छन्दाꣳ॑सि । छन्दाꣳ॑सि गच्छ । ग॒च्छ॒ स्वाहा᳚ । स्वाहेति॑ । इत्या॑ह । आ॒ह॒ प॒शवः॑ । प॒शवो॒ वै । वै छन्दाꣳ॑सि । छन्दाꣳ॑सि प॒शून् । प॒शूने॒व । ए॒वाव॑ । अव॑ रुन्धे । रु॒न्धे॒ द्यावा॑पृथि॒वी । द्यावा॑पृथि॒वी ग॑च्छ । द्यावा॑पृथि॒वी इति॒ द्यावा᳚ - पृ॒थि॒वी । ग॒च्छ॒ स्वाहा᳚ । स्वाहेति॑ । इत्या॑ह । आ॒ह॒ प्र॒जाः । प्र॒जा ए॒व । प्र॒जा इति॑ प्र - जाः । ए॒व प्रजा॑ताः । प्रजा॑ता॒ द्यावा॑पृथि॒वीभ्या᳚म् । प्रजा॑ता॒ इति॒ प्र - जा॒ताः॒ । द्यावा॑पृथि॒वीभ्या॑मुभ॒यतः॑ । द्यावा॑पृथि॒वीभ्या॒मिति॒ द्यावा᳚ - पृ॒थि॒वीभ्या᳚म् । उ॒भ॒यतः॒ परि॑ । परि॑ गृह्णाति । गृ॒ह्णा॒ति॒ नभः॑ । नभो॑ दि॒व्यम् \newline

\textbf{Jatai Paata} \newline

1. इत्या॑हा॒हे तीत्या॑ह । \newline
2. आ॒ह॒ प्र॒जासु॑ प्र॒जा स्वा॑हाह प्र॒जासु॑ । \newline
3. प्र॒जा स्वे॒वैव प्र॒जासु॑ प्र॒जा स्वे॒व । \newline
4. प्र॒जास्विति॑ प्र - जासु॑ । \newline
5. ए॒व प्रजा॑तासु॒ प्रजा॑ता स्वे॒वैव प्रजा॑तासु । \newline
6. प्रजा॑तासु प्राणापा॒नौ प्रा॑णापा॒नौ प्रजा॑तासु॒ प्रजा॑तासु प्राणापा॒नौ । \newline
7. प्रजा॑ता॒स्विति॒ प्र - जा॒ता॒सु॒ । \newline
8. प्रा॒णा॒पा॒नौ द॑धाति दधाति प्राणापा॒नौ प्रा॑णापा॒नौ द॑धाति । \newline
9. प्रा॒णा॒पा॒नाविति॑ प्राण - अ॒पा॒नौ । \newline
10. द॒धा॒ति॒ सोमꣳ॒॒ सोम॑म् दधाति दधाति॒ सोम᳚म् । \newline
11. सोम॑म् गच्छ गच्छ॒ सोमꣳ॒॒ सोम॑म् गच्छ । \newline
12. ग॒च्छ॒ स्वाहा॒ स्वाहा॑ गच्छ गच्छ॒ स्वाहा᳚ । \newline
13. स्वाहेतीति॒ स्वाहा॒ स्वाहेति॑ । \newline
14. इत्या॑हा॒हे तीत्या॑ह । \newline
15. आ॒ह॒ सौ॒म्याः सौ॒म्या आ॑हाह सौ॒म्याः । \newline
16. सौ॒म्या हि हि सौ॒म्याः सौ॒म्या हि । \newline
17. हि दे॒वत॑या दे॒वत॑या॒ हि हि दे॒वत॑या । \newline
18. दे॒वत॑या प्र॒जाः प्र॒जा दे॒वत॑या दे॒वत॑या प्र॒जाः । \newline
19. प्र॒जा य॒ज्ञ्ं ॅय॒ज्ञ्म् प्र॒जाः प्र॒जा य॒ज्ञ्म् । \newline
20. प्र॒जा इति॑ प्र - जाः । \newline
21. य॒ज्ञ्म् ग॑च्छ गच्छ य॒ज्ञ्ं ॅय॒ज्ञ्म् ग॑च्छ । \newline
22. ग॒च्छ॒ स्वाहा॒ स्वाहा॑ गच्छ गच्छ॒ स्वाहा᳚ । \newline
23. स्वाहेतीति॒ स्वाहा॒ स्वाहेति॑ । \newline
24. इत्या॑हा॒हे तीत्या॑ह । \newline
25. आ॒ह॒ प्र॒जाः प्र॒जा आ॑हाह प्र॒जाः । \newline
26. प्र॒जा ए॒वैव प्र॒जाः प्र॒जा ए॒व । \newline
27. प्र॒जा इति॑ प्र - जाः । \newline
28. ए॒व य॒ज्ञिया॑ य॒ज्ञिया॑ ए॒वैव य॒ज्ञियाः᳚ । \newline
29. य॒ज्ञियाः᳚ करोति करोति य॒ज्ञिया॑ य॒ज्ञियाः᳚ करोति । \newline
30. क॒रो॒ति॒ छन्दाꣳ॑सि॒ छन्दाꣳ॑सि करोति करोति॒ छन्दाꣳ॑सि । \newline
31. छन्दाꣳ॑सि गच्छ गच्छ॒ छन्दाꣳ॑सि॒ छन्दाꣳ॑सि गच्छ । \newline
32. ग॒च्छ॒ स्वाहा॒ स्वाहा॑ गच्छ गच्छ॒ स्वाहा᳚ । \newline
33. स्वाहेतीति॒ स्वाहा॒ स्वाहेति॑ । \newline
34. इत्या॑हा॒हे तीत्या॑ह । \newline
35. आ॒ह॒ प॒शवः॑ प॒शव॑ आहाह प॒शवः॑ । \newline
36. प॒शवो॒ वै वै प॒शवः॑ प॒शवो॒ वै । \newline
37. वै छन्दाꣳ॑सि॒ छन्दाꣳ॑सि॒ वै वै छन्दाꣳ॑सि । \newline
38. छन्दाꣳ॑सि प॒शून् प॒शून् छन्दाꣳ॑सि॒ छन्दाꣳ॑सि प॒शून् । \newline
39. प॒शू ने॒वैव प॒शून् प॒शूने॒व । \newline
40. ए॒वावा वै॒वै वाव॑ । \newline
41. अव॑ रुन्धे रु॒न्धे ऽवाव॑ रुन्धे । \newline
42. रु॒न्धे॒ द्यावा॑पृथि॒वी द्यावा॑पृथि॒वी रु॑न्धे रुन्धे॒ द्यावा॑पृथि॒वी । \newline
43. द्यावा॑पृथि॒वी ग॑च्छ गच्छ॒ द्यावा॑पृथि॒वी द्यावा॑पृथि॒वी ग॑च्छ । \newline
44. द्यावा॑पृथि॒वी इति॒ द्यावा᳚ - पृ॒थि॒वी । \newline
45. ग॒च्छ॒ स्वाहा॒ स्वाहा॑ गच्छ गच्छ॒ स्वाहा᳚ । \newline
46. स्वाहेतीति॒ स्वाहा॒ स्वाहेति॑ । \newline
47. इत्या॑हा॒हे तीत्या॑ह । \newline
48. आ॒ह॒ प्र॒जाः प्र॒जा आ॑हाह प्र॒जाः । \newline
49. प्र॒जा ए॒वैव प्र॒जाः प्र॒जा ए॒व । \newline
50. प्र॒जा इति॑ प्र - जाः । \newline
51. ए॒व प्रजा॑ताः॒ प्रजा॑ता ए॒वैव प्रजा॑ताः । \newline
52. प्रजा॑ता॒ द्यावा॑पृथि॒वीभ्या॒म् द्यावा॑पृथि॒वीभ्या॒म् प्रजा॑ताः॒ प्रजा॑ता॒ द्यावा॑पृथि॒वीभ्या᳚म् । \newline
53. प्रजा॑ता॒ इति॒ प्र - जा॒ताः॒ । \newline
54. द्यावा॑पृथि॒वीभ्या॑ मुभ॒यत॑ उभ॒यतो॒ द्यावा॑पृथि॒वीभ्या॒म् द्यावा॑पृथि॒वीभ्या॑ मुभ॒यतः॑ । \newline
55. द्यावा॑पृथि॒वीभ्या॒मिति॒ द्यावा᳚ - पृ॒थि॒वीभ्या᳚म् । \newline
56. उ॒भ॒यतः॒ परि॒ पर् यु॑भ॒यत॑ उभ॒यतः॒ परि॑ । \newline
57. परि॑ गृह्णाति गृह्णाति॒ परि॒ परि॑ गृह्णाति । \newline
58. गृ॒ह्णा॒ति॒ नभो॒ नभो॑ गृह्णाति गृह्णाति॒ नभः॑ । \newline
59. नभो॑ दि॒व्यम् दि॒व्यम् नभो॒ नभो॑ दि॒व्यम् । \newline

\textbf{Ghana Paata } \newline

1. इत्या॑हा॒हे तीत्या॑ह प्र॒जासु॑ प्र॒जा स्वा॒हेतीत्या॑ह प्र॒जासु॑ । \newline
2. आ॒ह॒ प्र॒जासु॑ प्र॒जास्वा॑हाह प्र॒जास्वे॒ वैव प्र॒जास्वा॑हाह प्र॒जा स्वे॒व । \newline
3. प्र॒जास्वे॒वैव प्र॒जासु॑ प्र॒जा स्वे॒व प्रजा॑तासु॒ प्रजा॑ता स्वे॒व प्र॒जासु॑ प्र॒जा स्वे॒व प्रजा॑तासु । \newline
4. प्र॒जास्विति॑ प्र - जासु॑ । \newline
5. ए॒व प्रजा॑तासु॒ प्रजा॑ता स्वे॒वैव प्रजा॑तासु प्राणापा॒नौ प्रा॑णापा॒नौ प्रजा॑ता स्वे॒वैव प्रजा॑तासु प्राणापा॒नौ । \newline
6. प्रजा॑तासु प्राणापा॒नौ प्रा॑णापा॒नौ प्रजा॑तासु॒ प्रजा॑तासु प्राणापा॒नौ द॑धाति दधाति प्राणापा॒नौ प्रजा॑तासु॒ प्रजा॑तासु प्राणापा॒नौ द॑धाति । \newline
7. प्रजा॑ता॒स्विति॒ प्र - जा॒ता॒सु॒ । \newline
8. प्रा॒णा॒पा॒नौ द॑धाति दधाति प्राणापा॒नौ प्रा॑णापा॒नौ द॑धाति॒ सोमꣳ॒॒ सोम॑म् दधाति प्राणापा॒नौ प्रा॑णापा॒नौ द॑धाति॒ सोम᳚म् । \newline
9. प्रा॒णा॒पा॒नाविति॑ प्राण - अ॒पा॒नौ । \newline
10. द॒धा॒ति॒ सोमꣳ॒॒ सोम॑म् दधाति दधाति॒ सोम॑म् गच्छ गच्छ॒ सोम॑म् दधाति दधाति॒ सोम॑म् गच्छ । \newline
11. सोम॑म् गच्छ गच्छ॒ सोमꣳ॒॒ सोम॑म् गच्छ॒ स्वाहा॒ स्वाहा॑ गच्छ॒ सोमꣳ॒॒ सोम॑म् गच्छ॒ स्वाहा᳚ । \newline
12. ग॒च्छ॒ स्वाहा॒ स्वाहा॑ गच्छ गच्छ॒ स्वाहेतीति॒ स्वाहा॑ गच्छ गच्छ॒ स्वाहेति॑ । \newline
13. स्वाहे तीति॒ स्वाहा॒ स्वाहे त्या॑हा॒ हेति॒ स्वाहा॒ स्वाहे त्या॑ह । \newline
14. इत्या॑हा॒हे तीत्या॑ह सौ॒म्याः सौ॒म्या आ॒हे तीत्या॑ह सौ॒म्याः । \newline
15. आ॒ह॒ सौ॒म्याः सौ॒म्या आ॑हाह सौ॒म्या हि हि सौ॒म्या आ॑हाह सौ॒म्या हि । \newline
16. सौ॒म्या हि हि सौ॒म्याः सौ॒म्या हि दे॒वत॑या दे॒वत॑या॒ हि सौ॒म्याः सौ॒म्या हि दे॒वत॑या । \newline
17. हि दे॒वत॑या दे॒वत॑या॒ हि हि दे॒वत॑या प्र॒जाः प्र॒जा दे॒वत॑या॒ हि हि दे॒वत॑या प्र॒जाः । \newline
18. दे॒वत॑या प्र॒जाः प्र॒जा दे॒वत॑या दे॒वत॑या प्र॒जा य॒ज्ञ्म् ॅय॒ज्ञ्म् प्र॒जा दे॒वत॑या दे॒वत॑या प्र॒जा य॒ज्ञ्म् । \newline
19. प्र॒जा य॒ज्ञ्ं ॅय॒ज्ञ्म् प्र॒जाः प्र॒जा य॒ज्ञ्म् ग॑च्छ गच्छ य॒ज्ञ्म् प्र॒जाः प्र॒जा य॒ज्ञ्म् ग॑च्छ । \newline
20. प्र॒जा इति॑ प्र - जाः । \newline
21. य॒ज्ञ्म् ग॑च्छ गच्छ य॒ज्ञ्ं ॅय॒ज्ञ्म् ग॑च्छ॒ स्वाहा॒ स्वाहा॑ गच्छ य॒ज्ञ्ं ॅय॒ज्ञ्म् ग॑च्छ॒ स्वाहा᳚ । \newline
22. ग॒च्छ॒ स्वाहा॒ स्वाहा॑ गच्छ गच्छ॒ स्वाहेतीति॒ स्वाहा॑ गच्छ गच्छ॒ स्वाहेति॑ । \newline
23. स्वाहे तीति॒ स्वाहा॒ स्वाहे त्या॑हा॒हेति॒ स्वाहा॒ स्वाहे त्या॑ह । \newline
24. इत्या॑हा॒हे तीत्या॑ह प्र॒जाः प्र॒जा आ॒हे तीत्या॑ह प्र॒जाः । \newline
25. आ॒ह॒ प्र॒जाः प्र॒जा आ॑हाह प्र॒जा ए॒वैव प्र॒जा आ॑हाह प्र॒जा ए॒व । \newline
26. प्र॒जा ए॒वैव प्र॒जाः प्र॒जा ए॒व य॒ज्ञिया॑ य॒ज्ञिया॑ ए॒व प्र॒जाः प्र॒जा ए॒व य॒ज्ञियाः᳚ । \newline
27. प्र॒जा इति॑ प्र - जाः । \newline
28. ए॒व य॒ज्ञिया॑ य॒ज्ञिया॑ ए॒वैव य॒ज्ञियाः᳚ करोति करोति य॒ज्ञिया॑ ए॒वैव य॒ज्ञियाः᳚ करोति । \newline
29. य॒ज्ञियाः᳚ करोति करोति य॒ज्ञिया॑ य॒ज्ञियाः᳚ करोति॒ छन्दाꣳ॑सि॒ छन्दाꣳ॑सि करोति य॒ज्ञिया॑ य॒ज्ञियाः᳚ करोति॒ छन्दाꣳ॑सि । \newline
30. क॒रो॒ति॒ छन्दाꣳ॑सि॒ छन्दाꣳ॑सि करोति करोति॒ छन्दाꣳ॑सि गच्छ गच्छ॒ छन्दाꣳ॑सि करोति करोति॒ छन्दाꣳ॑सि गच्छ । \newline
31. छन्दाꣳ॑सि गच्छ गच्छ॒ छन्दाꣳ॑सि॒ छन्दाꣳ॑सि गच्छ॒ स्वाहा॒ स्वाहा॑ गच्छ॒ छन्दाꣳ॑सि॒ छन्दाꣳ॑सि गच्छ॒ स्वाहा᳚ । \newline
32. ग॒च्छ॒ स्वाहा॒ स्वाहा॑ गच्छ गच्छ॒ स्वाहे तीति॒ स्वाहा॑ गच्छ गच्छ॒ स्वाहेति॑ । \newline
33. स्वाहे तीति॒ स्वाहा॒ स्वाहे त्या॑हा॒ हेति॒ स्वाहा॒ स्वाहे त्या॑ह । \newline
34. इत्या॑हा॒हे तीत्या॑ह प॒शवः॑ प॒शव॑ आ॒हे तीत्या॑ह प॒शवः॑ । \newline
35. आ॒ह॒ प॒शवः॑ प॒शव॑ आहाह प॒शवो॒ वै वै प॒शव॑ आहाह प॒शवो॒ वै । \newline
36. प॒शवो॒ वै वै प॒शवः॑ प॒शवो॒ वै छन्दाꣳ॑सि॒ छन्दाꣳ॑सि॒ वै प॒शवः॑ प॒शवो॒ वै छन्दाꣳ॑सि । \newline
37. वै छन्दाꣳ॑सि॒ छन्दाꣳ॑सि॒ वै वै छन्दाꣳ॑सि प॒शून् प॒शून् छन्दाꣳ॑सि॒ वै वै छन्दाꣳ॑सि प॒शून् । \newline
38. छन्दाꣳ॑सि प॒शून् प॒शून् छन्दाꣳ॑सि॒ छन्दाꣳ॑सि प॒शूने॒वैव प॒शून् छन्दाꣳ॑सि॒ छन्दाꣳ॑सि प॒शूने॒व । \newline
39. प॒शूने॒वैव प॒शून् प॒शूने॒ वावा वै॒व प॒शून् प॒शूने॒वाव॑ । \newline
40. ए॒वावा वै॒वै वाव॑ रुन्धे रु॒न्धे ऽवै॒वै वाव॑ रुन्धे । \newline
41. अव॑ रुन्धे रु॒न्धे ऽवाव॑ रुन्धे॒ द्यावा॑पृथि॒वी द्यावा॑पृथि॒वी रु॒न्धे ऽवाव॑ रुन्धे॒ द्यावा॑पृथि॒वी । \newline
42. रु॒न्धे॒ द्यावा॑पृथि॒वी द्यावा॑पृथि॒वी रु॑न्धे रुन्धे॒ द्यावा॑पृथि॒वी ग॑च्छ गच्छ॒ द्यावा॑पृथि॒वी रु॑न्धे रुन्धे॒ द्यावा॑पृथि॒वी ग॑च्छ । \newline
43. द्यावा॑पृथि॒वी ग॑च्छ गच्छ॒ द्यावा॑पृथि॒वी द्यावा॑पृथि॒वी ग॑च्छ॒ स्वाहा॒ स्वाहा॑ गच्छ॒ द्यावा॑पृथि॒वी द्यावा॑पृथि॒वी ग॑च्छ॒ स्वाहा᳚ । \newline
44. द्यावा॑पृथि॒वी इति॒ द्यावा᳚ - पृ॒थि॒वी । \newline
45. ग॒च्छ॒ स्वाहा॒ स्वाहा॑ गच्छ गच्छ॒ स्वाहे तीति॒ स्वाहा॑ गच्छ गच्छ॒ स्वाहेति॑ । \newline
46. स्वाहे तीति॒ स्वाहा॒ स्वाहे त्या॑हा॒ हेति॒ स्वाहा॒ स्वाहे त्या॑ह । \newline
47. इत्या॑हा॒हे तीत्या॑ह प्र॒जाः प्र॒जा आ॒हे तीत्या॑ह प्र॒जाः । \newline
48. आ॒ह॒ प्र॒जाः प्र॒जा आ॑हाह प्र॒जा ए॒वैव प्र॒जा आ॑हाह प्र॒जा ए॒व । \newline
49. प्र॒जा ए॒वैव प्र॒जाः प्र॒जा ए॒व प्रजा॑ताः॒ प्रजा॑ता ए॒व प्र॒जाः प्र॒जा ए॒व प्रजा॑ताः । \newline
50. प्र॒जा इति॑ प्र - जाः । \newline
51. ए॒व प्रजा॑ताः॒ प्रजा॑ता ए॒वैव प्रजा॑ता॒ द्यावा॑पृथि॒वीभ्या॒म् द्यावा॑पृथि॒वीभ्या॒म् प्रजा॑ता ए॒वैव प्रजा॑ता॒ द्यावा॑पृथि॒वीभ्या᳚म् । \newline
52. प्रजा॑ता॒ द्यावा॑पृथि॒वीभ्या॒म् द्यावा॑पृथि॒वीभ्या॒म् प्रजा॑ताः॒ प्रजा॑ता॒ द्यावा॑पृथि॒वीभ्या॑ मुभ॒यत॑ उभ॒यतो॒ द्यावा॑पृथि॒वीभ्या॒म् प्रजा॑ताः॒ प्रजा॑ता॒ द्यावा॑पृथि॒वीभ्या॑ मुभ॒यतः॑ । \newline
53. प्रजा॑ता॒ इति॒ प्र - जा॒ताः॒ । \newline
54. द्यावा॑पृथि॒वीभ्या॑ मुभ॒यत॑ उभ॒यतो॒ द्यावा॑पृथि॒वीभ्या॒म् द्यावा॑पृथि॒वीभ्या॑ मुभ॒यतः॒ परि॒ पर्यु॑ भ॒यतो॒ द्यावा॑पृथि॒वीभ्या॒म् द्यावा॑पृथि॒वीभ्या॑ मुभ॒यतः॒ परि॑ । \newline
55. द्यावा॑पृथि॒वीभ्या॒मिति॒ द्यावा᳚ - पृ॒थि॒वीभ्या᳚म् । \newline
56. उ॒भ॒यतः॒ परि॒ पर्यु॑भ॒यत॑ उभ॒यतः॒ परि॑ गृह्णाति गृह्णाति॒ पर्यु॑भ॒यत॑ उभ॒यतः॒ परि॑ गृह्णाति । \newline
57. परि॑ गृह्णाति गृह्णाति॒ परि॒ परि॑ गृह्णाति॒ नभो॒ नभो॑ गृह्णाति॒ परि॒ परि॑ गृह्णाति॒ नभः॑ । \newline
58. गृ॒ह्णा॒ति॒ नभो॒ नभो॑ गृह्णाति गृह्णाति॒ नभो॑ दि॒व्यम् दि॒व्यम् नभो॑ गृह्णाति गृह्णाति॒ नभो॑ दि॒व्यम् । \newline
59. नभो॑ दि॒व्यम् दि॒व्यम् नभो॒ नभो॑ दि॒व्यम् ग॑च्छ गच्छ दि॒व्यम् नभो॒ नभो॑ दि॒व्यम् ग॑च्छ । \newline
\pagebreak
\markright{ TS 6.4.1.4  \hfill https://www.vedavms.in \hfill}

\section{ TS 6.4.1.4 }

\textbf{TS 6.4.1.4 } \newline
\textbf{Samhita Paata} \newline

दि॒व्यं ग॑च्छ॒ स्वाहेत्या॑ह प्र॒जाभ्य॑ ए॒व प्रजा॑ताभ्यो॒ वृष्टिं॒ निय॑च्छत्य॒ग्निं ॅवै᳚श्वान॒रं ग॑च्छ॒ स्वाहेत्या॑ह प्र॒जा ए॒व प्रजा॑ता अ॒स्यां प्रति॑ ष्ठापयति प्रा॒णानां॒ ॅवा ए॒षोऽव॑ द्यति॒ यो॑ऽव॒द्यति॑ गु॒दस्य॒ मनो॑ मे॒ हार्दि॑ य॒च्छेत्या॑ह प्रा॒णाने॒व य॑थास्था॒नमुप॑ ह्वयते प॒शोर्वा आल॑ब्धस्य॒ हृद॑यꣳ॒॒ शुगृ॑च्छति॒ सा हृ॑दयशू॒ल- [  ] \newline

\textbf{Pada Paata} \newline

दि॒व्यम् । ग॒च्छ॒ । स्वाहा᳚ । इति॑ । आ॒ह॒ । प्र॒जाभ्य॒ इति॑ प्र-जाभ्यः॑ । ए॒व । प्रजा॑ताभ्य॒ इति॒ प्र - जा॒ता॒भ्यः॒ । वृष्टि᳚म् । नीति॑ । य॒च्छ॒ति॒ । अ॒ग्निम् । वै॒श्वा॒न॒रम् । ग॒च्छ॒ । स्वाहा᳚ । इति॑ । आ॒ह॒ । प्र॒जा इति॑ प्र - जाः । ए॒व । प्रजा॑ता॒ इति॒ प्र - जा॒ताः॒ । अ॒स्याम् । प्रतीति॑ । स्था॒प॒य॒ति॒ । प्रा॒णाना॒मिति॑ प्र-अ॒नाना᳚म् । वै । ए॒षः । अवेति॑ । द्य॒ति॒ । यः । अ॒व॒द्यतीय॑व - द्यति॑ । गु॒दस्य॑ । मनः॑ । मे॒ । हार्दि॑ । य॒च्छ॒ । इति॑ । आ॒ह॒ । प्रा॒णानिति॑ प्र - अ॒नान् । ए॒व । य॒था॒स्था॒नमिति॑ यथा - स्था॒नम् । उपेति॑ । ह्व॒य॒ते॒ । प॒शोः । वै । आल॑ब्ध॒स्येत्या- ल॒ब्ध॒स्य॒ । हृद॑यम् । शुक् । ऋ॒च्छ॒ति॒ । सा । हृ॒द॒य॒शू॒लमिति॑ हृदय - शू॒लम् ।  \newline


\textbf{Krama Paata} \newline

दि॒व्यम् ग॑च्छ । ग॒च्छ॒ स्वाहा᳚ । स्वाहेति॑ । इत्या॑ह । आ॒ह॒ प्र॒जाभ्यः॑ । प्र॒जाभ्य॑ ए॒व । प्र॒जाभ्य॒ इति॑ प्र - जाभ्यः॑ । ए॒व प्रजा॑ताभ्यः । प्रजा॑ताभ्यो॒ वृष्टि᳚म् । प्रजा॑ताभ्य॒ इति॒ प्र - जा॒ता॒भ्यः॒ । वृष्टि॒म् नि । नि य॑च्छति । य॒च्छ॒त्य॒ग्निम् । अ॒ग्निम् ॅवै᳚श्वान॒रम् । वै॒श्वा॒न॒रम् ग॑च्छ । ग॒च्छ॒ स्वाहा᳚ । स्वाहेति॑ । इत्या॑ह । आ॒ह॒ प्र॒जाः । प्र॒जा ए॒व । प्र॒जा इति॑ प्र - जाः । ए॒व प्रजा॑ताः । प्रजा॑ता अ॒स्याम् । प्रजा॑ता॒ इति॒ प्र - जा॒ताः॒ । अ॒स्याम् प्रति॑ । प्रति॑ष्ठापयति । स्था॒प॒य॒ति॒ प्रा॒णाना᳚म् । प्रा॒णाना॒म् ॅवै । प्रा॒णाना॒मिति॑ प्र - अ॒नाना᳚म् । वा ए॒षः । ए॒षोऽव॑ । अव॑ द्यति । द्य॒ति॒ यः । यो॑ऽव॒द्यति॑ । अ॒व॒द्यति॑ गु॒दस्य॑ । अ॒व॒द्यतीत्य॑व - द्यति॑ । गु॒दस्य॒ मनः॑ । मनो॑ मे । मे॒ हार्दि॑ । हार्दि॑ यच्छ । य॒च्छेति॑ । इत्या॑ह । आ॒ह॒ प्रा॒णान् । प्रा॒णाने॒व । प्रा॒णानिति॑ प्र - अ॒नान् । ए॒व य॑थास्था॒नम् । य॒था॒स्था॒नमुप॑ । य॒था॒स्था॒नमिति॑ यथा - स्था॒नम् । उप॑ ह्वयते । ह्व॒य॒ते॒ प॒शोः । प॒शोर् वै । वा आल॑ब्धस्य । आल॑ब्धस्य॒ हृद॑यम् । आल॑ब्ध॒स्येत्या - ल॒ब्ध॒स्य॒ । हृद॑यꣳ॒॒ शुक् । शुगृ॑च्छति । ऋ॒च्छ॒ति॒ सा । सा हृ॑दयशू॒लम् ( ) । हृ॒द॒य॒शू॒लम॒भि । हृ॒द॒य॒शू॒लमिति॑ हृदय - शू॒लम् \newline

\textbf{Jatai Paata} \newline

1. दि॒व्यम् ग॑च्छ गच्छ दि॒व्यम् दि॒व्यम् ग॑च्छ । \newline
2. ग॒च्छ॒ स्वाहा॒ स्वाहा॑ गच्छ गच्छ॒ स्वाहा᳚ । \newline
3. स्वाहेतीति॒ स्वाहा॒ स्वाहेति॑ । \newline
4. इत्या॑हा॒हे तीत्या॑ह । \newline
5. आ॒ह॒ प्र॒जाभ्यः॑ प्र॒जाभ्य॑ आहाह प्र॒जाभ्यः॑ । \newline
6. प्र॒जाभ्य॑ ए॒वैव प्र॒जाभ्यः॑ प्र॒जाभ्य॑ ए॒व । \newline
7. प्र॒जाभ्य॒ इति॑ प्र - जाभ्यः॑ । \newline
8. ए॒व प्रजा॑ताभ्यः॒ प्रजा॑ताभ्य ए॒वैव प्रजा॑ताभ्यः । \newline
9. प्रजा॑ताभ्यो॒ वृष्टिं॒ ॅवृष्टि॒म् प्रजा॑ताभ्यः॒ प्रजा॑ताभ्यो॒ वृष्टि᳚म् । \newline
10. प्रजा॑ताभ्य॒ इति॒ प्र - जा॒ता॒भ्यः॒ । \newline
11. वृष्टि॒न् नि नि वृष्टिं॒ ॅवृष्टि॒न् नि । \newline
12. नि य॑च्छति यच्छति॒ नि नि य॑च्छति । \newline
13. य॒च्छ॒ त्य॒ग्नि म॒ग्निं ॅय॑च्छति यच्छ त्य॒ग्निम् । \newline
14. अ॒ग्निं ॅवै᳚श्वान॒रं ॅवै᳚श्वान॒र म॒ग्नि म॒ग्निं ॅवै᳚श्वान॒रम् । \newline
15. वै॒श्वा॒न॒रम् ग॑च्छ गच्छ वैश्वान॒रं ॅवै᳚श्वान॒रम् ग॑च्छ । \newline
16. ग॒च्छ॒ स्वाहा॒ स्वाहा॑ गच्छ गच्छ॒ स्वाहा᳚ । \newline
17. स्वाहेतीति॒ स्वाहा॒ स्वाहेति॑ । \newline
18. इत्या॑हा॒हे तीत्या॑ह । \newline
19. आ॒ह॒ प्र॒जाः प्र॒जा आ॑हाह प्र॒जाः । \newline
20. प्र॒जा ए॒वैव प्र॒जाः प्र॒जा ए॒व । \newline
21. प्र॒जा इति॑ प्र - जाः । \newline
22. ए॒व प्रजा॑ताः॒ प्रजा॑ता ए॒वैव प्रजा॑ताः । \newline
23. प्रजा॑ता अ॒स्या म॒स्याम् प्रजा॑ताः॒ प्रजा॑ता अ॒स्याम् । \newline
24. प्रजा॑ता॒ इति॒ प्र - जा॒ताः॒ । \newline
25. अ॒स्याम् प्रति॒ प्रत्य॒स्या म॒स्याम् प्रति॑ । \newline
26. प्रति॑ ष्ठापयति स्थापयति॒ प्रति॒ प्रति॑ ष्ठापयति । \newline
27. स्था॒प॒य॒ति॒ प्रा॒णाना᳚म् प्रा॒णानाꣳ॑ स्थापयति स्थापयति प्रा॒णाना᳚म् । \newline
28. प्रा॒णानां॒ ॅवै वै प्रा॒णाना᳚म् प्रा॒णानां॒ ॅवै । \newline
29. प्रा॒णाना॒मिति॑ प्र - अ॒नाना᳚म् । \newline
30. वा ए॒ष ए॒ष वै वा ए॒षः । \newline
31. ए॒षो ऽवा वै॒ष ए॒षो ऽव॑ । \newline
32. अव॑ द्यति द्य॒ त्यवाव॑ द्यति । \newline
33. द्य॒ति॒ यो यो द्य॑ति द्यति॒ यः । \newline
34. यो॑ ऽव॒द्य त्य॑व॒द्यति॒ यो यो॑ ऽव॒द्यति॑ । \newline
35. अ॒व॒द्यति॑ गु॒दस्य॑ गु॒दस्या॑ व॒द्य त्य॑व॒द्यति॑ गु॒दस्य॑ । \newline
36. अ॒व॒द्यतीत्य॑व - द्यति॑ । \newline
37. गु॒दस्य॒ मनो॒ मनो॑ गु॒दस्य॑ गु॒दस्य॒ मनः॑ । \newline
38. मनो॑ मे मे॒ मनो॒ मनो॑ मे । \newline
39. मे॒ हार्दि॒ हार्दि॑ मे मे॒ हार्दि॑ । \newline
40. हार्दि॑ यच्छ यच्छ॒ हार्दि॒ हार्दि॑ यच्छ । \newline
41. य॒च्छेतीति॑ यच्छ य॒च्छेति॑ । \newline
42. इत्या॑हा॒हे तीत्या॑ह । \newline
43. आ॒ह॒ प्रा॒णान् प्रा॒णा ना॑हाह प्रा॒णान् । \newline
44. प्रा॒णा ने॒वैव प्रा॒णान् प्रा॒णाने॒व । \newline
45. प्रा॒णानिति॑ प्र - अ॒नान् । \newline
46. ए॒व य॑थास्था॒नं ॅय॑थास्था॒न मे॒वैव य॑थास्था॒नम् । \newline
47. य॒था॒स्था॒न मुपोप॑ यथास्था॒नं ॅय॑थास्था॒न मुप॑ । \newline
48. य॒था॒स्था॒नमिति॑ यथा - स्था॒नम् । \newline
49. उप॑ ह्वयते ह्वयत॒ उपोप॑ ह्वयते । \newline
50. ह्व॒य॒ते॒ प॒शोः प॒शोर् ह्व॑यते ह्वयते प॒शोः । \newline
51. प॒शोर् वै वै प॒शोः प॒शोर् वै । \newline
52. वा आल॑ब्ध॒स्या ल॑ब्धस्य॒ वै वा आल॑ब्धस्य । \newline
53. आल॑ब्धस्य॒ हृद॑यꣳ॒॒ हृद॑य॒ माल॑ब्ध॒स्या ल॑ब्धस्य॒ हृद॑यम् । \newline
54. आल॑ब्ध॒स्येत्या - ल॒ब्ध॒स्य॒ । \newline
55. हृद॑यꣳ॒॒ शुक् छुग् घृद॑यꣳ॒॒ हृद॑यꣳ॒॒ शुक् । \newline
56. शुगृ॑च्छ त्यृच्छति॒ शुक् छुगृ॑च्छति । \newline
57. ऋ॒च्छ॒ति॒ सा सर्च्छ॑ त्यृच्छति॒ सा । \newline
58. सा हृ॑दयशू॒लꣳ हृ॑दयशू॒लꣳ सा सा हृ॑दयशू॒लम् । \newline
59. हृ॒द॒य॒शू॒ल म॒भ्य॑भि हृ॑दयशू॒लꣳ हृ॑दयशू॒ल म॒भि । \newline
60. हृ॒द॒य॒शू॒लमिति॑ हृदय - शू॒लम् । \newline

\textbf{Ghana Paata } \newline

1. दि॒व्यम् ग॑च्छ गच्छ दि॒व्यम् दि॒व्यम् ग॑च्छ॒ स्वाहा॒ स्वाहा॑ गच्छ दि॒व्यम् दि॒व्यम् ग॑च्छ॒ स्वाहा᳚ । \newline
2. ग॒च्छ॒ स्वाहा॒ स्वाहा॑ गच्छ गच्छ॒ स्वाहे तीति॒ स्वाहा॑ गच्छ गच्छ॒ स्वाहेति॑ । \newline
3. स्वाहे तीति॒ स्वाहा॒ स्वाहे त्या॑हा॒ हेति॒ स्वाहा॒ स्वाहे त्या॑ह । \newline
4. इत्या॑हा॒हे तीत्या॑ह प्र॒जाभ्यः॑ प्र॒जाभ्य॑ आ॒हे तीत्या॑ह प्र॒जाभ्यः॑ । \newline
5. आ॒ह॒ प्र॒जाभ्यः॑ प्र॒जाभ्य॑ आहाह प्र॒जाभ्य॑ ए॒वैव प्र॒जाभ्य॑ आहाह प्र॒जाभ्य॑ ए॒व । \newline
6. प्र॒जाभ्य॑ ए॒वैव प्र॒जाभ्यः॑ प्र॒जाभ्य॑ ए॒व प्रजा॑ताभ्यः॒ प्रजा॑ताभ्य ए॒व प्र॒जाभ्यः॑ प्र॒जाभ्य॑ ए॒व प्रजा॑ताभ्यः । \newline
7. प्र॒जाभ्य॒ इति॑ प्र - जाभ्यः॑ । \newline
8. ए॒व प्रजा॑ताभ्यः॒ प्रजा॑ताभ्य ए॒वैव प्रजा॑ताभ्यो॒ वृष्टिं॒ ॅवृष्टि॒म् प्रजा॑ताभ्य ए॒वैव प्रजा॑ताभ्यो॒ वृष्टि᳚म् । \newline
9. प्रजा॑ताभ्यो॒ वृष्टिं॒ ॅवृष्टि॒म् प्रजा॑ताभ्यः॒ प्रजा॑ताभ्यो॒ वृष्टि॒न् नि नि वृष्टि॒म् प्रजा॑ताभ्यः॒ प्रजा॑ताभ्यो॒ वृष्टि॒न् नि । \newline
10. प्रजा॑ताभ्य॒ इति॒ प्र - जा॒ता॒भ्यः॒ । \newline
11. वृष्टि॒न् नि नि वृष्टिं॒ ॅवृष्टि॒न् नि य॑च्छति यच्छति॒ नि वृष्टिं॒ ॅवृष्टि॒न् नि य॑च्छति । \newline
12. नि य॑च्छति यच्छति॒ नि नि य॑च्छ त्य॒ग्नि म॒ग्निं ॅय॑च्छति॒ नि नि य॑च्छ त्य॒ग्निम् । \newline
13. य॒च्छ॒ त्य॒ग्नि म॒ग्निं ॅय॑च्छति यच्छ त्य॒ग्निं ॅवै᳚श्वान॒रं ॅवै᳚श्वान॒र म॒ग्निं ॅय॑च्छति यच्छ त्य॒ग्निं ॅवै᳚श्वान॒रम् । \newline
14. अ॒ग्निं ॅवै᳚श्वान॒रं ॅवै᳚श्वान॒र म॒ग्नि म॒ग्निं ॅवै᳚श्वान॒रम् ग॑च्छ गच्छ वैश्वान॒र म॒ग्नि म॒ग्निं ॅवै᳚श्वान॒रम् ग॑च्छ । \newline
15. वै॒श्वा॒न॒रम् ग॑च्छ गच्छ वैश्वान॒रं ॅवै᳚श्वान॒रम् ग॑च्छ॒ स्वाहा॒ स्वाहा॑ गच्छ वैश्वान॒रं ॅवै᳚श्वान॒रम् ग॑च्छ॒ स्वाहा᳚ । \newline
16. ग॒च्छ॒ स्वाहा॒ स्वाहा॑ गच्छ गच्छ॒ स्वाहे तीति॒ स्वाहा॑ गच्छ गच्छ॒ स्वाहेति॑ । \newline
17. स्वाहे तीति॒ स्वाहा॒ स्वाहे त्या॑हा॒ हेति॒ स्वाहा॒ स्वाहे त्या॑ह । \newline
18. इत्या॑हा॒हे तीत्या॑ह प्र॒जाः प्र॒जा आ॒हे तीत्या॑ह प्र॒जाः । \newline
19. आ॒ह॒ प्र॒जाः प्र॒जा आ॑हाह प्र॒जा ए॒वैव प्र॒जा आ॑हाह प्र॒जा ए॒व । \newline
20. प्र॒जा ए॒वैव प्र॒जाः प्र॒जा ए॒व प्रजा॑ताः॒ प्रजा॑ता ए॒व प्र॒जाः प्र॒जा ए॒व प्रजा॑ताः । \newline
21. प्र॒जा इति॑ प्र - जाः । \newline
22. ए॒व प्रजा॑ताः॒ प्रजा॑ता ए॒वैव प्रजा॑ता अ॒स्या म॒स्याम् प्रजा॑ता ए॒वैव प्रजा॑ता अ॒स्याम् । \newline
23. प्रजा॑ता अ॒स्या म॒स्याम् प्रजा॑ताः॒ प्रजा॑ता अ॒स्याम् प्रति॒ प्रत्य॒स्याम् प्रजा॑ताः॒ प्रजा॑ता अ॒स्याम् प्रति॑ । \newline
24. प्रजा॑ता॒ इति॒ प्र - जा॒ताः॒ । \newline
25. अ॒स्याम् प्रति॒ प्रत्य॒स्या म॒स्याम् प्रति॑ ष्ठापयति स्थापयति॒ प्रत्य॒स्या म॒स्याम् प्रति॑ ष्ठापयति । \newline
26. प्रति॑ ष्ठापयति स्थापयति॒ प्रति॒ प्रति॑ ष्ठापयति प्रा॒णाना᳚म् प्रा॒णानाꣳ॑ स्थापयति॒ प्रति॒ प्रति॑ ष्ठापयति प्रा॒णाना᳚म् । \newline
27. स्था॒प॒य॒ति॒ प्रा॒णाना᳚म् प्रा॒णानाꣳ॑ स्थापयति स्थापयति प्रा॒णानां॒ ॅवै वै प्रा॒णानाꣳ॑ स्थापयति स्थापयति प्रा॒णानां॒ ॅवै । \newline
28. प्रा॒णानां॒ ॅवै वै प्रा॒णाना᳚म् प्रा॒णानां॒ ॅवा ए॒ष ए॒ष वै प्रा॒णाना᳚म् प्रा॒णानां॒ ॅवा ए॒षः । \newline
29. प्रा॒णाना॒मिति॑ प्र - अ॒नाना᳚म् । \newline
30. वा ए॒ष ए॒ष वै वा ए॒षो ऽवावै॒ष वै वा ए॒षो ऽव॑ । \newline
31. ए॒षो ऽवावै॒ष ए॒षो ऽव॑ द्यति द्य॒त्यवै॒ष ए॒षो ऽव॑ द्यति । \newline
32. अव॑ द्यति द्य॒त्यवाव॑ द्यति॒ यो यो द्य॒त्य वाव॑ द्यति॒ यः । \newline
33. द्य॒ति॒ यो यो द्य॑ति द्यति॒ यो॑ ऽव॒द्य त्य॑व॒द्यति॒ यो द्य॑ति द्यति॒ यो॑ ऽव॒द्यति॑ । \newline
34. यो॑ ऽव॒द्य त्य॑व॒द्यति॒ यो यो॑ ऽव॒द्यति॑ गु॒दस्य॑ गु॒दस्या॑ व॒द्यति॒ यो यो॑ ऽव॒द्यति॑ गु॒दस्य॑ । \newline
35. अ॒व॒द्यति॑ गु॒दस्य॑ गु॒दस्या॑ व॒द्य त्य॑व॒द्यति॑ गु॒दस्य॒ मनो॒ मनो॑ गु॒दस्या॑ व॒द्य त्य॑व॒द्यति॑ गु॒दस्य॒ मनः॑ । \newline
36. अ॒व॒द्यतीत्य॑व - द्यति॑ । \newline
37. गु॒दस्य॒ मनो॒ मनो॑ गु॒दस्य॑ गु॒दस्य॒ मनो॑ मे मे॒ मनो॑ गु॒दस्य॑ गु॒दस्य॒ मनो॑ मे । \newline
38. मनो॑ मे मे॒ मनो॒ मनो॑ मे॒ हार्दि॒ हार्दि॑ मे॒ मनो॒ मनो॑ मे॒ हार्दि॑ । \newline
39. मे॒ हार्दि॒ हार्दि॑ मे मे॒ हार्दि॑ यच्छ यच्छ॒ हार्दि॑ मे मे॒ हार्दि॑ यच्छ । \newline
40. हार्दि॑ यच्छ यच्छ॒ हार्दि॒ हार्दि॑ य॒च्छेतीति॑ यच्छ॒ हार्दि॒ हार्दि॑ य॒च्छेति॑ । \newline
41. य॒च्छेतीति॑ यच्छ य॒च्छे त्या॑हा॒ हेति॑ यच्छ य॒च्छे त्या॑ह । \newline
42. इत्या॑हा॒हे तीत्या॑ह प्रा॒णान् प्रा॒णा ना॒हे तीत्या॑ह प्रा॒णान् । \newline
43. आ॒ह॒ प्रा॒णान् प्रा॒णा ना॑हाह प्रा॒णा ने॒वैव प्रा॒णाना॑ हाह प्रा॒णाने॒व । \newline
44. प्रा॒णा ने॒वैव प्रा॒णान् प्रा॒णाने॒व य॑थास्था॒नं ॅय॑थास्था॒न मे॒व प्रा॒णान् प्रा॒णा ने॒व य॑थास्था॒नम् । \newline
45. प्रा॒णानिति॑ प्र - अ॒नान् । \newline
46. ए॒व य॑थास्था॒नं ॅय॑थास्था॒न मे॒वैव य॑थास्था॒न मुपोप॑ यथास्था॒न मे॒वैव य॑थास्था॒न मुप॑ । \newline
47. य॒था॒स्था॒न मुपोप॑ यथास्था॒नं ॅय॑थास्था॒न मुप॑ ह्वयते ह्वयत॒ उप॑ यथास्था॒नं ॅय॑थास्था॒न मुप॑ ह्वयते । \newline
48. य॒था॒स्था॒नमिति॑ यथा - स्था॒नम् । \newline
49. उप॑ ह्वयते ह्वयत॒ उपोप॑ ह्वयते प॒शोः प॒शोर् ह्व॑यत॒ उपोप॑ ह्वयते प॒शोः । \newline
50. ह्व॒य॒ते॒ प॒शोः प॒शोर् ह्व॑यते ह्वयते प॒शोर् वै वै प॒शोर् ह्व॑यते ह्वयते प॒शोर् वै । \newline
51. प॒शोर् वै वै प॒शोः प॒शोर् वा आल॑ब्ध॒स्या ल॑ब्धस्य॒ वै प॒शोः प॒शोर् वा आल॑ब्धस्य । \newline
52. वा आल॑ब्ध॒स्या ल॑ब्धस्य॒ वै वा आल॑ब्धस्य॒ हृद॑यꣳ॒॒ हृद॑य॒ माल॑ब्धस्य॒ वै वा आल॑ब्धस्य॒ हृद॑यम् । \newline
53. आल॑ब्धस्य॒ हृद॑यꣳ॒॒ हृद॑य॒ माल॑ब्ध॒स्या ल॑ब्धस्य॒ हृद॑यꣳ॒॒ शुक् छुग्घृद॑य॒ माल॑ब्ध॒स्या ल॑ब्धस्य॒ हृद॑यꣳ॒॒ शुक् । \newline
54. आल॑ब्ध॒स्येत्या - ल॒ब्ध॒स्य॒ । \newline
55. हृद॑यꣳ॒॒ शुक् छुग्घृद॑यꣳ॒॒ हृद॑यꣳ॒॒ शु गृ॑च्छ त्यृच्छति॒ शुग्घृद॑यꣳ॒॒ हृद॑यꣳ॒॒ शुगृ॑च्छति । \newline
56. शुगृ॑च्छ त्यृच्छति॒ शुक् छुगृ॑च्छति॒ सा सर्च्छ॑ति॒ शुक् छुगृ॑च्छति॒ सा । \newline
57. ऋ॒च्छ॒ति॒ सा स र्‌च्छ॑ त्यृच्छति॒ सा हृ॑दयशू॒लꣳ हृ॑दयशू॒लꣳ स र्‌च्छ॑ त्यृच्छति॒ सा हृ॑दयशू॒लम् । \newline
58. सा हृ॑दयशू॒लꣳ हृ॑दयशू॒लꣳ सा सा हृ॑दयशू॒ल म॒भ्य॑भि हृ॑दयशू॒लꣳ सा सा हृ॑दयशू॒ल म॒भि । \newline
59. हृ॒द॒य॒शू॒ल म॒भ्य॑भि हृ॑दयशू॒लꣳ हृ॑दयशू॒ल म॒भि सꣳ स म॒भि हृ॑दयशू॒लꣳ हृ॑दयशू॒ल म॒भि सम् । \newline
60. हृ॒द॒य॒शू॒लमिति॑ हृदय - शू॒लम् । \newline
\pagebreak
\markright{ TS 6.4.1.5  \hfill https://www.vedavms.in \hfill}

\section{ TS 6.4.1.5 }

\textbf{TS 6.4.1.5 } \newline
\textbf{Samhita Paata} \newline

-म॒भि समे॑ति॒ यत् पृ॑थि॒व्याꣳ हृ॑दयशू॒ल-मु॑द्वा॒सये᳚त् पृथि॒वीꣳ शु॒चाऽर्प॑ये॒द् यद॒फ्स्व॑पः शु॒चाऽर्प॑ये॒च्छुष्क॑स्य चा॒ऽऽ*र्द्रस्य॑ च स॒न्धावुद्वा॑सयत्यु॒भय॑स्य॒ शान्त्यै॒ यं द्वि॒ष्यात् तं ध्या॑ये-च्छु॒चैवैन॑-मर्पयति ॥ \newline

\textbf{Pada Paata} \newline

अ॒भि । समिति॑ । ए॒ति॒ । यत् । पृ॒थि॒व्याम् । हृ॒द॒य॒शू॒लमिति॑ हृदय - शू॒लम् । उ॒द्वा॒सये॒दित्यु॑त् - वा॒सये᳚त् । पृ॒थि॒वीम् । शु॒चा । अ॒र्प॒ये॒त् । यत् । अ॒फ्स्वित्य॑प्- सु । अ॒पः । शु॒चा । अ॒र्प॒ये॒त् । शुष्क॑स्य । च॒ । आ॒र्द्रस्य॑ । च॒ । स॒न्धाविति॑ सं - धौ । उदिति॑ । वा॒स॒य॒ति॒ । उ॒भय॑स्य । शान्त्यै᳚ । यम् । द्वि॒ष्यात् । तम् । ध्या॒ये॒त् । शु॒चा । ए॒व । ए॒न॒म् । अ॒र्प॒य॒ति॒ ॥  \newline


\textbf{Krama Paata} \newline

अ॒भि सम् । समे॑ति । ए॒ति॒ यत् । यत् पृ॑थि॒व्याम् । पृ॒थि॒व्याꣳ हृ॑दयशू॒लम् । हृ॒द॒य॒शू॒लमु॑द्‌वा॒सये᳚त् । हृ॒द॒य॒शू॒लमिति॑ हृदय - शू॒लम् । उ॒द्‌वा॒सये᳚त् पृथि॒वीम् । उ॒द्‌वा॒सये॒दित्यु॑त् - वा॒सये᳚त् । पृ॒थि॒वीꣳ शु॒चा । शु॒चाऽर्प॑येत् । अ॒र्प॒ये॒द् यत् । यद॒फ्सु । अ॒फ्स्व॑पः । अ॒फ्स्वित्य॑प् - सु । अ॒पः शु॒चा । शु॒चाऽर्प॑येत् । अ॒र्प॒ये॒च्छुष्क॑स्य । शुष्क॑स्य च । चा॒र्द्रस्य॑ । आ॒र्द्रस्य॑ च । च॒ स॒न्धौ । स॒न्धावुत् । स॒न्धाविति॑ सम् - धौ । उद् वा॑सयति । वा॒स॒य॒त्यु॒भय॑स्य । उ॒भय॑स्य॒ शान्त्यै᳚ । शान्त्यै॒ यम् । यम् द्वि॒ष्यात् । द्वि॒ष्यात् तम् । तम् ध्या॑येत् । ध्या॒ये॒च्छु॒चा । शु॒चैव । ए॒वैन᳚म् । ए॒न॒म॒र्प॒य॒ति॒ । अ॒र्प॒य॒तीत्य॑र्पयति । \newline

\textbf{Jatai Paata} \newline

1. अ॒भि सꣳ स म॒भ्य॑भि सम् । \newline
2. स मे᳚त्येति॒ सꣳ स मे॑ति । \newline
3. ए॒ति॒ यद् यदे᳚त्येति॒ यत् । \newline
4. यत् पृ॑थि॒व्याम् पृ॑थि॒व्यां ॅयद् यत् पृ॑थि॒व्याम् । \newline
5. पृ॒थि॒व्याꣳ हृ॑दयशू॒लꣳ हृ॑दयशू॒लम् पृ॑थि॒व्याम् पृ॑थि॒व्याꣳ हृ॑दयशू॒लम् । \newline
6. हृ॒द॒य॒शू॒ल मु॑द्वा॒सये॑ दुद्वा॒सये᳚ द्धृदयशू॒लꣳ हृ॑दयशू॒ल मु॑द्वा॒सये᳚त् । \newline
7. हृ॒द॒य॒शू॒लमिति॑ हृदय - शू॒लम् । \newline
8. उ॒द्वा॒सये᳚त् पृथि॒वीम् पृ॑थि॒वी मु॑द्वा॒सये॑ दुद्वा॒सये᳚त् पृथि॒वीम् । \newline
9. उ॒द्वा॒सये॒दित्यु॑त् - वा॒सये᳚त् । \newline
10. पृ॒थि॒वीꣳ शु॒चा शु॒चा पृ॑थि॒वीम् पृ॑थि॒वीꣳ शु॒चा । \newline
11. शु॒चा ऽर्प॑ये दर्पयेच् छु॒चा शु॒चा ऽर्प॑येत् । \newline
12. अ॒र्प॒ये॒द् यद् यद॑र्पये दर्पये॒द् यत् । \newline
13. यद॒ फ्स्व॑फ्सु यद् यद॒फ्सु । \newline
14. अ॒फ्स्वा᳚(1॒)पो᳚(1॒) ऽपो᳚(1॒) ऽफ्स्वा᳚(1॒)फ्स्व॑पः । \newline
15. अ॒फ्स्वित्य॑प् - सु । \newline
16. अ॒पः शु॒चा शु॒चा ऽपो॑ ऽपः शु॒चा । \newline
17. शु॒चा ऽर्प॑ये दर्पयेच् छु॒चा शु॒चा ऽर्प॑येत् । \newline
18. अ॒र्प॒ये॒च् छुष्क॑स्य॒ शुष्क॑ स्यार्पये दर्पये॒च् छुष्क॑स्य । \newline
19. शुष्क॑स्य च च॒ शुष्क॑स्य॒ शुष्क॑स्य च । \newline
20. चा॒र्द्र स्या॒र्द्रस्य॑ च चा॒र्द्रस्य॑ । \newline
21. आ॒र्द्रस्य॑ च चा॒र्द्र स्या॒र्द्रस्य॑ च । \newline
22. च॒ स॒न्धौ स॒न्धौ च॑ च स॒न्धौ । \newline
23. स॒न्धा वुदुथ् स॒न्धौ स॒न्धा वुत् । \newline
24. स॒न्धाविति॑ सं - धौ । \newline
25. उद् वा॑सयति वासय॒ त्युदुद् वा॑सयति । \newline
26. वा॒स॒य॒ त्यु॒भय॑ स्यो॒भय॑स्य वासयति वासय त्यु॒भय॑स्य । \newline
27. उ॒भय॑स्य॒ शान्त्यै॒ शान्त्या॑ उ॒भय॑ स्यो॒भय॑स्य॒ शान्त्यै᳚ । \newline
28. शान्त्यै॒ यं ॅयꣳ शान्त्यै॒ शान्त्यै॒ यम् । \newline
29. यम् द्वि॒ष्याद् द्वि॒ष्याद् यं ॅयम् द्वि॒ष्यात् । \newline
30. द्वि॒ष्यात् तम् तम् द्वि॒ष्याद् द्वि॒ष्यात् तम् । \newline
31. तम् ध्या॑येद् ध्याये॒त् तम् तम् ध्या॑येत् । \newline
32. ध्या॒ये॒च् छु॒चा शु॒चा ध्या॑येद् ध्यायेच् छु॒चा । \newline
33. शु॒चैवैव शु॒चा शु॒चैव । \newline
34. ए॒वैन॑ मेन मे॒वै वैन᳚म् । \newline
35. ए॒न॒ म॒र्प॒य॒ त्य॒र्प॒य॒ त्ये॒न॒ मे॒न॒ म॒र्प॒य॒ति॒ । \newline
36. अ॒र्प॒य॒तीत्य॑र्पयति । \newline

\textbf{Ghana Paata } \newline

1. अ॒भि सꣳ स म॒भ्य॑भि स मे᳚त्येति॒ स म॒भ्य॑भि स मे॑ति । \newline
2. स मे᳚त्येति॒ सꣳ स मे॑ति॒ यद् यदे॑ति॒ सꣳ स मे॑ति॒ यत् । \newline
3. ए॒ति॒ यद् यदे᳚ त्येति॒ यत् पृ॑थि॒व्याम् पृ॑थि॒व्यां ॅयदे᳚ त्येति॒ यत् पृ॑थि॒व्याम् । \newline
4. यत् पृ॑थि॒व्याम् पृ॑थि॒व्यां ॅयद् यत् पृ॑थि॒व्याꣳ हृ॑दयशू॒लꣳ हृ॑दयशू॒लम् पृ॑थि॒व्यां ॅयद् यत् पृ॑थि॒व्याꣳ हृ॑दयशू॒लम् । \newline
5. पृ॒थि॒व्याꣳ हृ॑दयशू॒लꣳ हृ॑दयशू॒लम् पृ॑थि॒व्याम् पृ॑थि॒व्याꣳ हृ॑दयशू॒ल मु॑द्वा॒सये॑ दुद्वा॒सये᳚ द्धृदयशू॒लम् पृ॑थि॒व्याम् पृ॑थि॒व्याꣳ हृ॑दयशू॒ल मु॑द्वा॒सये᳚त् । \newline
6. हृ॒द॒य॒शू॒ल मु॑द्वा॒सये॑ दुद्वा॒सये᳚ द्धृदयशू॒लꣳ हृ॑दयशू॒ल मु॑द्वा॒सये᳚त् पृथि॒वीम् पृ॑थि॒वी मु॑द्वा॒सये᳚ द्धृदयशू॒लꣳ हृ॑दयशू॒ल मु॑द्वा॒सये᳚त् पृथि॒वीम् । \newline
7. हृ॒द॒य॒शू॒लमिति॑ हृदय - शू॒लम् । \newline
8. उ॒द्वा॒सये᳚त् पृथि॒वीम् पृ॑थि॒वी मु॑द्वा॒सये॑ दुद्वा॒सये᳚त् पृथि॒वीꣳ शु॒चा शु॒चा पृ॑थि॒वी मु॑द्वा॒सये॑ दुद्वा॒सये᳚त् पृथि॒वीꣳ शु॒चा । \newline
9. उ॒द्वा॒सये॒दित्यु॑त् - वा॒सये᳚त् । \newline
10. पृ॒थि॒वीꣳ शु॒चा शु॒चा पृ॑थि॒वीम् पृ॑थि॒वीꣳ शु॒चा ऽर्प॑ये दर्पयेच् छु॒चा पृ॑थि॒वीम् पृ॑थि॒वीꣳ शु॒चा ऽर्प॑येत् । \newline
11. शु॒चा ऽर्प॑ये दर्पयेच् छु॒चा शु॒चा ऽर्प॑ये॒द् यद् यद॑र्पयेच् छु॒चा शु॒चा ऽर्प॑ये॒द् यत् । \newline
12. अ॒र्प॒ये॒द् यद् यद॑र्पये दर्पये॒द् यद॒ फ्स्व॑फ्सु यद॑र्पये दर्पये॒द् यद॒फ्सु । \newline
13. यद॒ फ्स्व॑फ्सु यद् यद॒फ्स्वा᳚(1॒)पो᳚(1॒) ऽपो᳚ ऽफ्सु यद् यद॒ फ्स्व॑पः । \newline
14. अ॒फ्स्वा᳚(1॒)पो᳚(1॒) ऽपो᳚(1॒) ऽफ्स्वा᳚(1॒)फ्स्व॑पः शु॒चा शु॒चा ऽपो᳚(1॒) ऽफ्स्वा᳚(1॒)फ्स्व॑पः शु॒चा । \newline
15. अ॒फ्स्वित्य॑प् - सु । \newline
16. अ॒पः शु॒चा शु॒चा ऽपो॑ ऽपः शु॒चा ऽर्प॑ये दर्पयेच् छु॒चा ऽपो॑ ऽपः शु॒चा ऽर्प॑येत् । \newline
17. शु॒चा ऽर्प॑ये दर्पयेच् छु॒चा शु॒चा ऽर्प॑ये॒च् छुष्क॑स्य॒ शुष्क॑स्या र्पयेच् छु॒चा शु॒चा ऽर्प॑ये॒च् छुष्क॑स्य । \newline
18. अ॒र्प॒ये॒च् छुष्क॑स्य॒ शुष्क॑स्या र्पये दर्पये॒च् छुष्क॑स्य च च॒ शुष्क॑स्या र्पये दर्पये॒च् छुष्क॑स्य च । \newline
19. शुष्क॑स्य च च॒ शुष्क॑स्य॒ शुष्क॑स्य चा॒र्द्रस्या॒ र्द्रस्य॑ च॒ शुष्क॑स्य॒ शुष्क॑स्य चा॒र्द्रस्य॑ । \newline
20. चा॒र्द्रस्या॒ र्द्रस्य॑ च चा॒र्द्रस्य॑ च चा॒र्द्रस्य॑ च चा॒र्द्रस्य॑ च । \newline
21. आ॒र्द्रस्य॑ च चा॒र्द्रस्या॒ र्द्रस्य॑ च स॒न्धौ स॒न्धौ चा॒र्द्रस्या॒ र्द्रस्य॑ च स॒न्धौ । \newline
22. च॒ स॒न्धौ स॒न्धौ च॑ च स॒न्धा वुदुथ् स॒न्धौ च॑ च स॒न्धा वुत् । \newline
23. स॒न्धा वुदुथ् स॒न्धौ स॒न्धा वुद् वा॑सयति वासय॒ त्युथ् स॒न्धौ स॒न्धा वुद् वा॑सयति । \newline
24. स॒न्धाविति॑ सं - धौ । \newline
25. उद् वा॑सयति वासय॒ त्युदुद् वा॑सय त्यु॒भय॑स्यो॒ भय॑स्य वासय॒ त्युदुद् वा॑सय त्यु॒भय॑स्य । \newline
26. वा॒स॒य॒ त्यु॒भय॑ स्यो॒भय॑स्य वासयति वासय त्यु॒भय॑स्य॒ शान्त्यै॒ शान्त्या॑ उ॒भय॑स्य वासयति वासय
त्यु॒भय॑स्य॒ शान्त्यै᳚ । \newline
27. उ॒भय॑स्य॒ शान्त्यै॒ शान्त्या॑ उ॒भय॑ स्यो॒भय॑स्य॒ शान्त्यै॒ यं ॅयꣳ शान्त्या॑ उ॒भय॑ स्यो॒भय॑स्य॒ शान्त्यै॒ यम् । \newline
28. शान्त्यै॒ यं ॅयꣳ शान्त्यै॒ शान्त्यै॒ यम् द्वि॒ष्याद् द्वि॒ष्याद् यꣳ शान्त्यै॒ शान्त्यै॒ यम् द्वि॒ष्यात् । \newline
29. यम् द्वि॒ष्याद् द्वि॒ष्याद् यं ॅयम् द्वि॒ष्यात् तम् तम् द्वि॒ष्याद् यं ॅयम् द्वि॒ष्यात् तम् । \newline
30. द्वि॒ष्यात् तम् तम् द्वि॒ष्याद् द्वि॒ष्यात् तम् ध्या॑येद् ध्याये॒त् तम् द्वि॒ष्याद् द्वि॒ष्यात् तम् ध्या॑येत् । \newline
31. तम् ध्या॑येद् ध्याये॒त् तम् तम् ध्या॑येच् छु॒चा शु॒चा ध्या॑ये॒त् तम् तम् ध्या॑येच् छु॒चा । \newline
32. ध्या॒ये॒च् छु॒चा शु॒चा ध्या॑येद् ध्यायेच् छु॒चैवैव शु॒चा ध्या॑येद् ध्यायेच् छु॒चैव । \newline
33. शु॒चैवैव शु॒चा शु॒चैवैन॑ मेन मे॒व शु॒चा शु॒चैवैन᳚म् । \newline
34. ए॒वैन॑ मेन मे॒वै वैन॑ मर्पय त्यर्पय त्येन मे॒वै वैन॑ मर्पयति । \newline
35. ए॒न॒ म॒र्प॒य॒ त्य॒र्प॒य॒ त्ये॒न॒ मे॒न॒ म॒र्प॒य॒ति॒ । \newline
36. अ॒र्प॒य॒तीत्य॑र्पयति । \newline
\pagebreak
\markright{ TS 6.4.2.1  \hfill https://www.vedavms.in \hfill}

\section{ TS 6.4.2.1 }

\textbf{TS 6.4.2.1 } \newline
\textbf{Samhita Paata} \newline

दे॒वा वै य॒ज्ञ्माग्नी᳚द्ध्रे॒ व्य॑भजन्त॒ ततो॒ यद॒त्यशि॑ष्यत॒ तद॑ब्रुव॒न् वस॑तु॒ नु न॑ इ॒दमिति॒ तद् व॑सती॒वरी॑णां ॅवसती वरि॒त्वं तस्मि॑न् प्रा॒तर्न सम॑शक्नुव॒न् तद॒फ्सु प्रावे॑शय॒न् ता व॑सती॒ वरी॑रभवन् वसती॒वरी᳚र्गृह्णाति य॒ज्ञो वै व॑सती॒ वरी᳚र्य॒ज्ञ्मे॒वाऽऽ*रभ्य॑ गृही॒त्वोप॑ वसति॒ यस्यागृ॑हीता अ॒भि नि॒म्रोचे॒-दना॑रब्धोऽस्य य॒ज्ञ्ः स्या᳚- [  ] \newline

\textbf{Pada Paata} \newline

दे॒वाः । वै । य॒ज्ञ्म् । आग्नी᳚द्ध्र॒ इत्याग्नि॑ - इ॒द्ध्रे॒ । वीति॑ । अ॒भ॒ज॒न्त॒ । ततः॑ । यत् । अ॒त्यशि॑ष्य॒तेत्य॑ति-अशि॑ष्यत । तत् । अ॒ब्रु॒व॒न्न् । वस॑तु । नु । नः॒ । इ॒दम् । इति॑ । तत् । व॒स॒ती॒वरी॑णाम् । व॒स॒ती॒व॒रि॒त्वमिति॑ वसतीवरि - त्वम् । तस्मिन्न्॑ । प्रा॒तः । न । समिति॑ । अ॒श॒क्नु॒व॒न्न् । तत् । अ॒फ्स्वित्य॑प् - सु । प्रेति॑ । अ॒वे॒श॒य॒न्न् । ताः । व॒स॒ती॒वरीः᳚ । अ॒भ॒व॒न्न् । व॒स॒ती॒वरीः᳚ । गृ॒ह्णा॒ति॒ । य॒ज्ञ्ः । वै । व॒स॒ती॒वरीः᳚ । य॒ज्ञ्म् । ए॒व । आ॒रभ्येत्या᳚ - रभ्य॑ । गृ॒ही॒त्वा । उपेति॑ । व॒स॒ति॒ । यस्य॑ । अगृ॑हीताः । अ॒भीति॑ । नि॒म्रोचे॒दिति॑ नि - म्रोचे᳚त् । अना॑रब्ध॒ इत्यना᳚ - र॒ब्धः॒ । अ॒स्य॒ । य॒ज्ञ्ः । स्या॒त् ।  \newline


\textbf{Krama Paata} \newline

दे॒वा वै । वै य॒ज्ञ्म् । य॒ज्ञ्माग्नी᳚द्ध्रे । आग्नी᳚द्ध्रे॒ वि । आग्नी᳚द्ध्र॒ इत्याग्नि॑ - इ॒द्ध्रे॒ । व्य॑भजन्त । अ॒भ॒ज॒न्त॒ ततः॑ । ततो॒ यत् । यद॒त्यशि॑ष्यत । अ॒त्यशि॑ष्यत॒ तत् । अ॒त्यशि॑ष्य॒तेत्य॑ति - अशि॑ष्यत । तद॑ब्रुवन्न् । अ॒ब्रु॒व॒न् वस॑तु । वस॑तु॒ नु । नु नः॑ । न॒ इ॒दम् । इ॒दमिति॑ । इति॒ तत् । तद् व॑सती॒वरी॑णाम् । व॒स॒ती॒वरी॑णाम् ॅवसतीवरि॒त्वम् । व॒स॒ती॒व॒रि॒त्वम् तस्मिन्न्॑ । व॒स॒ती॒व॒रि॒त्वमिति॑ वसतीवरि - त्वम् । तस्मि॑न् प्रा॒तः । प्रा॒तर् न । न सम् । सम॑शक्नुवन्न् । अ॒श॒क्नु॒व॒न् तत् । तद॒फ्सु । अ॒फ्सु प्र । अ॒फ्सित्य॑प् - सु । प्रावे॑शयन्न् । अ॒वे॒श॒य॒न् ताः । ता व॑सती॒वरीः᳚ । व॒स॒ती॒वरी॑रभवन्न् । अ॒भ॒व॒न् व॒स॒ती॒वरीः᳚ । व॒स॒ती॒वरी᳚र् गृह्णाति । गृ॒ह्णा॒ति॒ य॒ज्ञ्ः । य॒ज्ञो वै । वै व॑सती॒वरीः᳚ । व॒स॒ती॒वरी᳚र् य॒ज्ञ्म् । य॒ज्ञ्मे॒व । ए॒वारभ्य॑ । आ॒रभ्य॑ गृही॒त्वा । आ॒रभ्येत्या᳚ - रभ्य॑ । गृ॒ही॒त्वोप॑ । उप॑ वसति । व॒स॒ति॒ यस्य॑ । यस्यागृ॑हीताः । अगृ॑हीता अ॒भि । अ॒भि नि॒म्रोचे᳚त् । नि॒म्रोचे॒दना॑रब्धः । नि॒म्रोचे॒दिति॑ नि - म्रोचे᳚त् । अना॑रब्धोऽस्य । अना॑रब्ध॒ इत्यना᳚ - र॒ब्धः॒ । अ॒स्य॒ य॒ज्ञ्ः । य॒ज्ञ्ः स्या᳚त् । स्या॒द् य॒ज्ञ्म् \newline

\textbf{Jatai Paata} \newline

1. दे॒वा वै वै दे॒वा दे॒वा वै । \newline
2. वै य॒ज्ञ्ं ॅय॒ज्ञ्ं ॅवै वै य॒ज्ञ्म् । \newline
3. य॒ज्ञ् माग्नी᳚द्ध्र॒ आग्नी᳚द्ध्रे य॒ज्ञ्ं ॅय॒ज्ञ् माग्नी᳚द्ध्रे । \newline
4. आग्नी᳚द्ध्रे॒ वि व्याग्नी᳚द्ध्र॒ आग्नी᳚द्ध्रे॒ वि । \newline
5. आग्नी᳚द्ध्र॒ इत्याग्नि॑ - इ॒द्ध्रे॒ । \newline
6. व्य॑भजन्ता भजन्त॒ वि व्य॑भजन्त । \newline
7. अ॒भ॒ज॒न्त॒ तत॒ स्ततो॑ ऽभजन्ता भजन्त॒ ततः॑ । \newline
8. ततो॒ यद् यत् तत॒ स्ततो॒ यत् । \newline
9. यद॒त्यशि॑ष्यता॒ त्यशि॑ष्यत॒ यद् यद॒त्यशि॑ष्यत । \newline
10. अ॒त्यशि॑ष्यत॒ तत् तद॒त्यशि॑ष्यता॒ त्यशि॑ष्यत॒ तत् । \newline
11. अ॒त्यशि॑ष्य॒तेत्य॑ति - अशि॑ष्यत । \newline
12. तद॑ब्रुवन् नब्रुव॒न् तत् तद॑ब्रुवन्न् । \newline
13. अ॒ब्रु॒व॒न्॒. वस॑तु॒ वस॑ त्वब्रुवन् नब्रुव॒न्॒. वस॑तु । \newline
14. वस॑तु॒ नु नु वस॑तु॒ वस॑तु॒ नु । \newline
15. नु नो॑ नो॒ नु नु नः॑ । \newline
16. न॒ इ॒द मि॒दन् नो॑ न इ॒दम् । \newline
17. इ॒द मिती ती॒द मि॒द मिति॑ । \newline
18. इति॒ तत् तदि तीति॒ तत् । \newline
19. तद् व॑सती॒वरी॑णां ॅवसती॒वरी॑णा॒म् तत् तद् व॑सती॒वरी॑णाम् । \newline
20. व॒स॒ती॒वरी॑णां ॅवसतीवरि॒त्वं ॅव॑सतीवरि॒त्वं ॅव॑सती॒वरी॑णां ॅवसती॒वरी॑णां ॅवसतीवरि॒त्वम् । \newline
21. व॒स॒ती॒व॒रि॒त्वम् तस्मिꣳ॒॒ स्तस्मि॑न्. वसतीवरि॒त्वं ॅव॑सतीवरि॒त्वम् तस्मिन्न्॑ । \newline
22. व॒स॒ती॒व॒रि॒त्वमिति॑ वसतीवरि - त्वम् । \newline
23. तस्मि॑न् प्रा॒तः प्रा॒त स्तस्मिꣳ॒॒ स्तस्मि॑न् प्रा॒तः । \newline
24. प्रा॒तर् न न प्रा॒तः प्रा॒तर् न । \newline
25. न सꣳ सन् न न सम् । \newline
26. स म॑शक्नुवन् नशक्नुव॒न् थ्सꣳ स म॑शक्नुवन्न् । \newline
27. अ॒श॒क्नु॒व॒न् तत् तद॑शक्नुवन् नशक्नुव॒न् तत् । \newline
28. तद॒ फ्स्व॑फ्सु तत् तद॒फ्सु । \newline
29. अ॒फ्सु प्र प्रा फ्स्व॑फ्सु प्र । \newline
30. अ॒फ्स्वित्य॑प् - सु । \newline
31. प्रा वे॑शयन् नवेशय॒न् प्र प्रा वे॑शयन्न् । \newline
32. अ॒वे॒श॒य॒न् ता स्ता अ॑वेशयन् नवेशय॒न् ताः । \newline
33. ता व॑सती॒वरी᳚र् वसती॒वरी॒ स्ता स्ता व॑सती॒वरीः᳚ । \newline
34. व॒स॒ती॒वरी॑ रभवन् नभवन्. वसती॒वरी᳚र् वसती॒वरी॑ रभवन्न् । \newline
35. अ॒भ॒व॒न्॒. व॒स॒ती॒वरी᳚र् वसती॒वरी॑ रभवन् नभवन्. वसती॒वरीः᳚ । \newline
36. व॒स॒ती॒वरी᳚र् गृह्णाति गृह्णाति वसती॒वरी᳚र् वसती॒वरी᳚र् गृह्णाति । \newline
37. गृ॒ह्णा॒ति॒ य॒ज्ञो य॒ज्ञो गृ॑ह्णाति गृह्णाति य॒ज्ञ्ः । \newline
38. य॒ज्ञो वै वै य॒ज्ञो य॒ज्ञो वै । \newline
39. वै व॑सती॒वरी᳚र् वसती॒वरी॒र् वै वै व॑सती॒वरीः᳚ । \newline
40. व॒स॒ती॒वरी᳚र् य॒ज्ञ्ं ॅय॒ज्ञ्ं ॅव॑सती॒वरी᳚र् वसती॒वरी᳚र् य॒ज्ञ्म् । \newline
41. य॒ज्ञ् मे॒वैव य॒ज्ञ्ं ॅय॒ज्ञ् मे॒व । \newline
42. ए॒वा रभ्या॒ रभ्यै॒ वैवा रभ्य॑ । \newline
43. आ॒रभ्य॑ गृही॒त्वा गृ॑ही॒त्वा ऽऽरभ्या॒ रभ्य॑ गृही॒त्वा । \newline
44. आ॒रभ्येत्या᳚ - रभ्य॑ । \newline
45. गृ॒ही॒त्वो पोप॑ गृही॒त्वा गृ॑ही॒ त्वोप॑ । \newline
46. उप॑ वसति वस॒ त्युपोप॑ वसति । \newline
47. व॒स॒ति॒ यस्य॒ यस्य॑ वसति वसति॒ यस्य॑ । \newline
48. यस्या गृ॑हीता॒ अगृ॑हीता॒ यस्य॒ यस्या गृ॑हीताः । \newline
49. अगृ॑हीता अ॒भ्य॑भ्य गृ॑हीता॒ अगृ॑हीता अ॒भि । \newline
50. अ॒भि नि॒म्रोचे᳚न् नि॒म्रोचे॑ द॒भ्य॑भि नि॒म्रोचे᳚त् । \newline
51. नि॒म्रोचे॒ दना॑र॒ब्धो ऽना॑रब्धो नि॒म्रोचे᳚न् नि॒म्रोचे॒ दना॑रब्धः । \newline
52. नि॒म्रोचे॒दिति॑ नि - म्रोचे᳚त् । \newline
53. अना॑रब्धो ऽस्या॒स्या ना॑र॒ब्धो ऽना॑रब्धो ऽस्य । \newline
54. अना॑रब्ध॒ इत्यना᳚ - र॒ब्धः॒ । \newline
55. अ॒स्य॒ य॒ज्ञो य॒ज्ञो᳚ ऽस्यास्य य॒ज्ञ्ः । \newline
56. य॒ज्ञ्ः स्या᳚थ् स्याद् य॒ज्ञो य॒ज्ञ्ः स्या᳚त् । \newline
57. स्या॒द् य॒ज्ञ्ं ॅय॒ज्ञ्ꣳ स्या᳚थ् स्याद् य॒ज्ञ्म् । \newline

\textbf{Ghana Paata } \newline

1. दे॒वा वै वै दे॒वा दे॒वा वै य॒ज्ञ्ं ॅय॒ज्ञ्ं ॅवै दे॒वा दे॒वा वै य॒ज्ञ्म् । \newline
2. वै य॒ज्ञ्ं ॅय॒ज्ञ्ं ॅवै वै य॒ज्ञ् माग्नी᳚द्ध्र॒ आग्नी᳚द्ध्रे य॒ज्ञ्ं ॅवै वै य॒ज्ञ् माग्नी᳚द्ध्रे । \newline
3. य॒ज्ञ् माग्नी᳚द्ध्र॒ आग्नी᳚द्ध्रे य॒ज्ञ्ं ॅय॒ज्ञ् माग्नी᳚द्ध्रे॒ वि व्याग्नी᳚द्ध्रे य॒ज्ञ्ं ॅय॒ज्ञ् माग्नी᳚द्ध्रे॒ वि । \newline
4. आग्नी᳚द्ध्रे॒ वि व्याग्नी᳚द्ध्र॒ आग्नी᳚द्ध्रे॒ व्य॑भजन्ता भजन्त॒ व्याग्नी᳚द्ध्र॒ आग्नी᳚द्ध्रे॒ व्य॑भजन्त । \newline
5. आग्नी᳚द्ध्र॒ इत्याग्नि॑ - इ॒द्ध्रे॒ । \newline
6. व्य॑भजन्ता भजन्त॒ वि व्य॑भजन्त॒ तत॒ स्ततो॑ ऽभजन्त॒ वि व्य॑भजन्त॒ ततः॑ । \newline
7. अ॒भ॒ज॒न्त॒ तत॒ स्ततो॑ ऽभजन्ता भजन्त॒ ततो॒ यद् यत् ततो॑ ऽभजन्ता भजन्त॒ ततो॒ यत् । \newline
8. ततो॒ यद् यत् तत॒ स्ततो॒ यद॒त्यशि॑ष्यता॒ त्यशि॑ष्यत॒ यत् तत॒ स्ततो॒ यद॒त्यशि॑ष्यत । \newline
9. यद॒त्यशि॑ष्यता॒ त्यशि॑ष्यत॒ यद् यद॒त्यशि॑ष्यत॒ तत् तद॒त्यशि॑ष्यत॒ यद् यद॒त्यशि॑ष्यत॒ तत् । \newline
10. अ॒त्यशि॑ष्यत॒ तत् तद॒त्यशि॑ष्यता॒ त्यशि॑ष्यत॒ तद॑ब्रुवन् नब्रुव॒न् तद॒त्यशि॑ष्यता॒ त्यशि॑ष्यत॒ तद॑ब्रुवन्न् । \newline
11. अ॒त्यशि॑ष्य॒तेत्य॑ति - अशि॑ष्यत । \newline
12. तद॑ब्रुवन् नब्रुव॒न् तत् तद॑ब्रुव॒न्॒. वस॑तु॒ वस॑ त्वब्रुव॒न् तत् तद॑ब्रुव॒न्॒. वस॑तु । \newline
13. अ॒ब्रु॒व॒न्॒. वस॑तु॒ वस॑ त्वब्रुवन् नब्रुव॒न्॒. वस॑तु॒ नु नु वस॑ त्वब्रुवन् नब्रुव॒न्॒. वस॑तु॒ नु । \newline
14. वस॑तु॒ नु नु वस॑तु॒ वस॑तु॒ नु नो॑ नो॒ नु वस॑तु॒ वस॑तु॒ नु नः॑ । \newline
15. नु नो॑ नो॒ नु नु न॑ इ॒द मि॒दन् नो॒ नु नु न॑ इ॒दम् । \newline
16. न॒ इ॒द मि॒दन् नो॑ न इ॒द मिती ती॒दन् नो॑ न इ॒द मिति॑ । \newline
17. इ॒द मिती ती॒द मि॒द मिति॒ तत् तदि ती॒द मि॒द मिति॒ तत् । \newline
18. इति॒ तत् तदि तीति॒ तद् व॑सती॒वरी॑णां ॅवसती॒वरी॑णा॒म् तदितीति॒ तद् व॑सती॒वरी॑णाम् । \newline
19. तद् व॑सती॒वरी॑णां ॅवसती॒वरी॑णा॒म् तत् तद् व॑सती॒वरी॑णां ॅवसतीवरि॒त्वं ॅव॑सतीवरि॒त्वं ॅव॑सती॒वरी॑णा॒म् तत् तद् व॑सती॒वरी॑णां ॅवसतीवरि॒त्वम् । \newline
20. व॒स॒ती॒वरी॑णां ॅवसतीवरि॒त्वं ॅव॑सतीवरि॒त्वं ॅव॑सती॒वरी॑णां ॅवसती॒वरी॑णां ॅवसतीवरि॒त्वम् तस्मिꣳ॒॒ स्तस्मि॑न्. वसतीवरि॒त्वं ॅव॑सती॒वरी॑णां ॅवसती॒वरी॑णां ॅवसतीवरि॒त्वम् तस्मिन्न्॑ । \newline
21. व॒स॒ती॒व॒रि॒त्वम् तस्मिꣳ॒॒ स्तस्मि॑न्. वसतीवरि॒त्वं ॅव॑सतीवरि॒त्वम् तस्मि॑न् प्रा॒तः प्रा॒त स्तस्मि॑न्. वसतीवरि॒त्वं ॅव॑सतीवरि॒त्वम् तस्मि॑न् प्रा॒तः । \newline
22. व॒स॒ती॒व॒रि॒त्वमिति॑ वसतीवरि - त्वम् । \newline
23. तस्मि॑न् प्रा॒तः प्रा॒त स्तस्मिꣳ॒॒ स्तस्मि॑न् प्रा॒तर् न न प्रा॒त स्तस्मिꣳ॒॒ स्तस्मि॑न् प्रा॒तर् न । \newline
24. प्रा॒तर् न न प्रा॒तः प्रा॒तर् न सꣳ सन् न प्रा॒तः प्रा॒तर् न सम् । \newline
25. न सꣳ सन् न न स म॑शक्नुवन् नशक्नुव॒न् थ्सन् न न स म॑शक्नुवन्न् । \newline
26. स म॑शक्नुवन् नशक्नुव॒न् थ्सꣳ स म॑शक्नुव॒न् तत् तद॑शक्नुव॒न् थ्सꣳ स म॑शक्नुव॒न् तत् । \newline
27. अ॒श॒क्नु॒व॒न् तत् तद॑शक्नुवन् नशक्नुव॒न् तद॒ फ्स्व॑फ्सु तद॑शक्नुवन् नशक्नुव॒न् तद॒फ्सु । \newline
28. तद॒ फ्स्व॑फ्सु तत् तद॒फ्सु प्र प्राफ्सु तत् तद॒फ्सु प्र । \newline
29. अ॒फ्सु प्र प्राफ्स्व॑फ्सु प्रावे॑शयन् नवेशय॒न् प्राफ्स्व॑फ्सु प्रावे॑शयन्न् । \newline
30. अ॒फ्स्वित्य॑प् - सु । \newline
31. प्रावे॑शयन् नवेशय॒न् प्र प्रावे॑शय॒न् ता स्ता अ॑वेशय॒न् प्र प्रावे॑शय॒न् ताः । \newline
32. अ॒वे॒श॒य॒न् ता स्ता अ॑वेशयन् नवेशय॒न् ता व॑सती॒वरी᳚र् वसती॒वरी॒ स्ता अ॑वेशयन् नवेशय॒न् ता व॑सती॒वरीः᳚ । \newline
33. ता व॑सती॒वरी᳚र् वसती॒वरी॒ स्ता स्ता व॑सती॒वरी॑ रभवन् नभवन्. वसती॒वरी॒ स्ता स्ता व॑सती॒वरी॑ रभवन्न् । \newline
34. व॒स॒ती॒वरी॑ रभवन् नभवन्. वसती॒वरी᳚र् वसती॒वरी॑ रभवन्. वसती॒वरी᳚र् वसती॒वरी॑ रभवन्. वसती॒वरी᳚र् वसती॒वरी॑ रभवन्. वसती॒वरीः᳚ । \newline
35. अ॒भ॒व॒न्॒. व॒स॒ती॒वरी᳚र् वसती॒वरी॑ रभवन् नभवन्. वसती॒वरी᳚र् गृह्णाति गृह्णाति वसती॒वरी॑ रभवन् नभवन्. वसती॒वरी᳚र् गृह्णाति । \newline
36. व॒स॒ती॒वरी᳚र् गृह्णाति गृह्णाति वसती॒वरी᳚र् वसती॒वरी᳚र् गृह्णाति य॒ज्ञो य॒ज्ञो गृ॑ह्णाति वसती॒वरी᳚र् वसती॒वरी᳚र् गृह्णाति य॒ज्ञ्ः । \newline
37. गृ॒ह्णा॒ति॒ य॒ज्ञो य॒ज्ञो गृ॑ह्णाति गृह्णाति य॒ज्ञो वै वै य॒ज्ञो गृ॑ह्णाति गृह्णाति य॒ज्ञो वै । \newline
38. य॒ज्ञो वै वै य॒ज्ञो य॒ज्ञो वै व॑सती॒वरी᳚र् वसती॒वरी॒र् वै य॒ज्ञो य॒ज्ञो वै व॑सती॒वरीः᳚ । \newline
39. वै व॑सती॒वरी᳚र् वसती॒वरी॒र् वै वै व॑सती॒वरी᳚र् य॒ज्ञ्ं ॅय॒ज्ञ्ं ॅव॑सती॒वरी॒र् वै वै व॑सती॒वरी᳚र् य॒ज्ञ्म् । \newline
40. व॒स॒ती॒वरी᳚र् य॒ज्ञ्ं ॅय॒ज्ञ्ं ॅव॑सती॒वरी᳚र् वसती॒वरी᳚र् य॒ज्ञ् मे॒वैव य॒ज्ञ्ं ॅव॑सती॒वरी᳚र् वसती॒वरी᳚र् य॒ज्ञ् मे॒व । \newline
41. य॒ज्ञ् मे॒वैव य॒ज्ञ्ं ॅय॒ज्ञ् मे॒वा रभ्या॒ रभ्यै॒व य॒ज्ञ्ं ॅय॒ज्ञ् मे॒वारभ्य॑ । \newline
42. ए॒वारभ्या॒ रभ्यै॒वै वारभ्य॑ गृही॒त्वा गृ॑ही॒त्वा ऽऽरभ्यै॒वै वारभ्य॑ गृही॒त्वा । \newline
43. आ॒रभ्य॑ गृही॒त्वा गृ॑ही॒त्वा ऽऽरभ्या॒ रभ्य॑ गृही॒त्वोपोप॑ गृही॒त्वा ऽऽरभ्या॒ रभ्य॑ गृही॒त्वोप॑ । \newline
44. आ॒रभ्येत्या᳚ - रभ्य॑ । \newline
45. गृ॒ही॒त्वोपोप॑ गृही॒त्वा गृ॑ही॒त्वोप॑ वसति वस॒ त्युप॑ गृही॒त्वा गृ॑ही॒त्वोप॑ वसति । \newline
46. उप॑ वसति वस॒ त्युपोप॑ वसति॒ यस्य॒ यस्य॑ वस॒ त्युपोप॑ वसति॒ यस्य॑ । \newline
47. व॒स॒ति॒ यस्य॒ यस्य॑ वसति वसति॒ यस्या गृ॑हीता॒ अगृ॑हीता॒ यस्य॑ वसति वसति॒ यस्या गृ॑हीताः । \newline
48. यस्या गृ॑हीता॒ अगृ॑हीता॒ यस्य॒ यस्या गृ॑हीता अ॒भ्य॑भ्य गृ॑हीता॒ यस्य॒ यस्या गृ॑हीता अ॒भि । \newline
49. अगृ॑हीता अ॒भ्य॑भ्य गृ॑हीता॒ अगृ॑हीता अ॒भि नि॒म्रोचे᳚न् नि॒म्रोचे॑ द॒भ्यगृ॑हीता॒ अगृ॑हीता अ॒भि नि॒म्रोचे᳚त् । \newline
50. अ॒भि नि॒म्रोचे᳚न् नि॒म्रोचे॑ द॒भ्य॑भि नि॒म्रोचे॒ दना॑र॒ब्धो ऽना॑रब्धो नि॒म्रोचे॑ द॒भ्य॑भि नि॒म्रोचे॒ दना॑रब्धः । \newline
51. नि॒म्रोचे॒ दना॑र॒ब्धो ऽना॑रब्धो नि॒म्रोचे᳚न् नि॒म्रोचे॒ दना॑रब्धो ऽस्या॒स्या ना॑रब्धो नि॒म्रोचे᳚न् नि॒म्रोचे॒ दना॑रब्धो ऽस्य । \newline
52. नि॒म्रोचे॒दिति॑ नि - म्रोचे᳚त् । \newline
53. अना॑रब्धो ऽस्या॒स्या ना॑र॒ब्धो ऽना॑रब्धो ऽस्य य॒ज्ञो य॒ज्ञो᳚ ऽस्या ना॑र॒ब्धो ऽना॑रब्धो ऽस्य य॒ज्ञ्ः । \newline
54. अना॑रब्ध॒ इत्यना᳚ - र॒ब्धः॒ । \newline
55. अ॒स्य॒ य॒ज्ञो य॒ज्ञो᳚ ऽस्यास्य य॒ज्ञ्ः स्या᳚थ् स्याद् य॒ज्ञो᳚ ऽस्यास्य य॒ज्ञ्ः स्या᳚त् । \newline
56. य॒ज्ञ्ः स्या᳚थ् स्याद् य॒ज्ञो य॒ज्ञ्ः स्या᳚द् य॒ज्ञ्ं ॅय॒ज्ञ्ꣳ स्या᳚द् य॒ज्ञो य॒ज्ञ्ः स्या᳚द् य॒ज्ञ्म् । \newline
57. स्या॒द् य॒ज्ञ्ं ॅय॒ज्ञ्ꣳ स्या᳚थ् स्याद् य॒ज्ञ्ं ॅवि वि य॒ज्ञ्ꣳ स्या᳚थ् स्याद् य॒ज्ञ्ं ॅवि । \newline
\pagebreak
\markright{ TS 6.4.2.2  \hfill https://www.vedavms.in \hfill}

\section{ TS 6.4.2.2 }

\textbf{TS 6.4.2.2 } \newline
\textbf{Samhita Paata} \newline

द्य॒ज्ञ्ं ॅवि च्छि॑न्द्या-ज्ज्योति॒ष्या॑ वा गृह्णी॒याद्धिर॑ण्यं ॅवा ऽव॒धाय॒ सशु॑क्राणामे॒व गृ॑ह्णाति॒ यो वा᳚ ब्राह्म॒णो ब॑हुया॒जी तस्य॒ कुंभ्या॑नां गृह्णीया॒थ् स हि गृ॑ही॒त व॑सतीवरीको वसती॒वरी᳚र्गृह्णाति प॒शवो॒ वै व॑सती॒वरीः᳚ प॒शूने॒वाऽऽ*रभ्य॑ गृही॒त्वोप॑ वसति॒ यद॑न्वी॒पं तिष्ठ॑न् गृह्णी॒यान्नि॒र्मार्गु॑का अस्मात् प॒शवः॑ स्युः प्रती॒पं तिष्ठ॑न् गृह्णाति प्रति॒रुद्ध्यै॒वास्मै॑ प॒शून् गृ॑ह्णा॒तीन्द्रो॑- [  ] \newline

\textbf{Pada Paata} \newline

य॒ज्ञ्म् । वीति॑ । छि॒न्द्या॒त् । ज्यो॒ति॒ष्या᳚ । वा॒ । गृ॒ह्णी॒यात् । हिर॑ण्यम् । वा॒ । अ॒व॒धायेत्य॑व - धाय॑ । सशु॑क्राणा॒मिति॒ स-शु॒क्रा॒णा॒म् । ए॒व । गृ॒ह्णा॒ति॒ । यः । वा॒ । ब्रा॒ह्म॒णः । ब॒हु॒या॒जीति॑ बहु - या॒जी । तस्य॑ । कुंभ्या॑नाम् । गृ॒ह्णी॒या॒त् । सः । हि । गृ॒ही॒तव॑सतीवरीक॒ इति॑ गृही॒त - व॒स॒ती॒व॒री॒कः॒ । व॒स॒ती॒वरीः᳚ । गृ॒ह्णा॒ति॒ । प॒शवः॑ । वै । व॒स॒ती॒वरीः᳚ । प॒शून् । ए॒व । आ॒रभ्येत्या᳚- रभ्य॑ । गृ॒ही॒त्वा । उपेति॑ । व॒स॒ति॒ । यत् । अ॒न्वी॒पम् । तिष्ठन्न्॑ । गृ॒ह्णी॒यात् । नि॒र्मार्गु॑का॒ इति॑ निः - मार्गु॑काः । अ॒स्मा॒त् । प॒शवः॑ । स्युः॒ । प्र॒ती॒पम् । तिष्ठन्न्॑ । गृ॒ह्णा॒ति॒ । प्र॒ति॒रुद्ध्येति॑ प्रति - रुद्ध्य॑ । ए॒व । अ॒स्मै॒ । प॒शून् । गृ॒ह्णा॒ति॒ । इन्द्रः॑ ।  \newline


\textbf{Krama Paata} \newline

य॒ज्ञ्म् ॅवि । विच्छि॑न्द्यात् । छि॒न्द्या॒ज् ज्यो॒ति॒ष्या᳚ । ज्यो॒ति॒ष्या॑ वा । वा॒ गृ॒ह्णी॒यात् । गृ॒ह्णि॒याद्‌धिर॑ण्यम् । हिर॑ण्यम् ॅवा । वा॒ऽव॒धाय॑ । अ॒व॒धाय॒ सशु॑क्राणाम् । अ॒व॒धायेत्य॑व - धाय॑ । सशु॑क्राणमे॒व । सशु॑क्राणा॒मिति॒ स - शु॒क्रा॒णा॒म् । ए॒व गृ॑ह्णाति । गृ॒ह्णा॒ति॒ यः । यो वा᳚ । वा॒ ब्रा॒ह्म॒णः । ब्रा॒ह्म॒णो ब॑हुया॒जी । ब॒हु॒या॒जी तस्य॑ । ब॒हु॒या॒जीति॑ बहु - या॒जी । तस्य॒ कुम्भ्या॑नाम् । कुम्भ्या॑नाम् गृह्णीयात् । गृ॒ह्णी॒या॒थ् सः । स हि । हि गृ॑ही॒तव॑सतीवरीकः । गृ॒ही॒तव॑सतीवरीको वसती॒वरीः᳚ । गृ॒ही॒तव॑सतीवरीक॒ इति॑ गृही॒त - व॒स॒ती॒व॒री॒कः॒ । व॒स॒ती॒वरी᳚र् गृह्णाति । गृ॒ह्णा॒ति॒ प॒शवः॑ । प॒शवो॒ वै । वै व॑सती॒वरीः᳚ । व॒स॒ती॒वरीः᳚ प॒शून् । प॒शूने॒व । ए॒वारभ्य॑ । आ॒रभ्य॑ गृही॒त्वा । आ॒रभ्येत्या᳚ - रभ्य॑ । गृ॒ही॒त्वोप॑ । उप॑ वसति । व॒स॒ति॒ यत् । यद॑न्वी॒पम् । अ॒न्वी॒पम् तिष्ठन्न्॑ । तिष्ठ॑न् गृह्णी॒यात् । गृ॒ह्णी॒यान् नि॒र्मार्गु॑काः । नि॒र्मार्गु॑का अस्मात् । नि॒र्मार्गु॑का॒ इति॑ निः - मार्गु॑काः । अ॒स्मा॒त् प॒शवः॑ । प॒शवः॑ स्युः । स्युः॒ प्र॒ती॒पम् । प्र॒ती॒पम् तिष्ठन्न्॑ । तिष्ठ॑न् गृह्णाति । गृ॒ह्णा॒ति॒ प्र॒ति॒रुद्ध्य॑ । प्र॒ति॒रुद्ध्यै॒व । प्र॒ति॒रुद्ध्येति॑ प्रति - रुद्ध्य॑ । ए॒वास्मै᳚ । अ॒स्मै॒ प॒शून् । प॒शून् गृ॑ह्णाति । गृ॒ह्णा॒तीन्द्रः॑ । इन्द्रो॑ वृ॒त्रम् \newline

\textbf{Jatai Paata} \newline

1. य॒ज्ञ्ं ॅवि वि य॒ज्ञ्ं ॅय॒ज्ञ्ं ॅवि । \newline
2. वि च्छि॑न्द्याच् छिन्द्या॒द् वि वि च्छि॑न्द्यात् । \newline
3. छि॒न्द्या॒ज् ज्यो॒ति॒ष्या᳚ ज्योति॒ष्या॑ छिन्द्याच् छिन्द्याज् ज्योति॒ष्या᳚ । \newline
4. ज्यो॒ति॒ष्या॑ वा वा ज्योति॒ष्या᳚ ज्योति॒ष्या॑ वा । \newline
5. वा॒ गृ॒ह्णी॒याद् गृ॑ह्णी॒याद् वा॑ वा गृह्णी॒यात् । \newline
6. गृ॒ह्णी॒या द्धिर॑ण्यꣳ॒॒ हिर॑ण्यम् गृह्णी॒याद् गृ॑ह्णी॒या द्धिर॑ण्यम् । \newline
7. हिर॑ण्यं ॅवा वा॒ हिर॑ण्यꣳ॒॒ हिर॑ण्यं ॅवा । \newline
8. वा॒ ऽव॒धाया॑ व॒धाय॑ वा वा ऽव॒धाय॑ । \newline
9. अ॒व॒धाय॒ सशु॑क्राणाꣳ॒॒ सशु॑क्राणा मव॒धाया॑ व॒धाय॒ सशु॑क्राणाम् । \newline
10. अ॒व॒धायेत्य॑व - धाय॑ । \newline
11. सशु॑क्राणा मे॒वैव सशु॑क्राणाꣳ॒॒ सशु॑क्राणा मे॒व । \newline
12. सशु॑क्राणा॒मिति॒ स - शु॒क्रा॒णा॒म् । \newline
13. ए॒व गृ॑ह्णाति गृह्णा त्ये॒वैव गृ॑ह्णाति । \newline
14. गृ॒ह्णा॒ति॒ यो यो गृ॑ह्णाति गृह्णाति॒ यः । \newline
15. यो वा॑ वा॒ यो यो वा᳚ । \newline
16. वा॒ ब्रा॒ह्म॒णो ब्रा᳚ह्म॒णो वा॑ वा ब्राह्म॒णः । \newline
17. ब्रा॒ह्म॒णो ब॑हुया॒जी ब॑हुया॒जी ब्रा᳚ह्म॒णो ब्रा᳚ह्म॒णो ब॑हुया॒जी । \newline
18. ब॒हु॒या॒जी तस्य॒ तस्य॑ बहुया॒जी ब॑हुया॒जी तस्य॑ । \newline
19. ब॒हु॒या॒जीति॑ बहु - या॒जी । \newline
20. तस्य॒ कुंभ्या॑ना॒म् कुंभ्या॑ना॒म् तस्य॒ तस्य॒ कुंभ्या॑नाम् । \newline
21. कुंभ्या॑नाम् गृह्णीयाद् गृह्णीया॒त् कुंभ्या॑ना॒म् कुंभ्या॑नाम् गृह्णीयात् । \newline
22. गृ॒ह्णी॒या॒थ् स स गृ॑ह्णीयाद् गृह्णीया॒थ् सः । \newline
23. स हि हि स स हि । \newline
24. हि गृ॑ही॒तव॑सतीवरीको गृही॒तव॑सतीवरीको॒ हि हि गृ॑ही॒तव॑सतीवरीकः । \newline
25. गृ॒ही॒तव॑सतीवरीको वसती॒वरी᳚र् वसती॒वरी᳚र् गृही॒तव॑सतीवरीको गृही॒तव॑सतीवरीको वसती॒वरीः᳚ । \newline
26. गृ॒ही॒तव॑सतीवरीक॒ इति॑ गृही॒त - व॒स॒ती॒व॒री॒कः॒ । \newline
27. व॒स॒ती॒वरी᳚र् गृह्णाति गृह्णाति वसती॒वरी᳚र् वसती॒वरी᳚र् गृह्णाति । \newline
28. गृ॒ह्णा॒ति॒ प॒शवः॑ प॒शवो॑ गृह्णाति गृह्णाति प॒शवः॑ । \newline
29. प॒शवो॒ वै वै प॒शवः॑ प॒शवो॒ वै । \newline
30. वै व॑सती॒वरी᳚र् वसती॒वरी॒र् वै वै व॑सती॒वरीः᳚ । \newline
31. व॒स॒ती॒वरीः᳚ प॒शून् प॒शून्. व॑सती॒वरी᳚र् वसती॒वरीः᳚ प॒शून् । \newline
32. प॒शू ने॒वैव प॒शून् प॒शूने॒व । \newline
33. ए॒वा रभ्या॒ रभ्यै॒ वैवा रभ्य॑ । \newline
34. आ॒रभ्य॑ गृही॒त्वा गृ॑ही॒त्वा ऽऽरभ्या॒ रभ्य॑ गृही॒त्वा । \newline
35. आ॒रभ्येत्या᳚ - रभ्य॑ । \newline
36. गृ॒ही॒ त्वोपोप॑ गृही॒त्वा गृ॑ही॒ त्वोप॑ । \newline
37. उप॑ वसति वस॒ त्युपोप॑ वसति । \newline
38. व॒स॒ति॒ यद् यद् व॑सति वसति॒ यत् । \newline
39. यद॑न्वी॒प म॑न्वी॒पं ॅयद् यद॑न्वी॒पम् । \newline
40. अ॒न्वी॒पम् तिष्ठꣳ॒॒ स्तिष्ठ॑न् नन्वी॒प म॑न्वी॒पम् तिष्ठन्न्॑ । \newline
41. तिष्ठ॑न् गृह्णी॒याद् गृ॑ह्णी॒यात् तिष्ठꣳ॒॒ स्तिष्ठ॑न् गृह्णी॒यात् । \newline
42. गृ॒ह्णी॒यान् नि॒र्मार्गु॑का नि॒र्मार्गु॑का गृह्णी॒याद् गृ॑ह्णी॒यान् नि॒र्मार्गु॑काः । \newline
43. नि॒र्मार्गु॑का अस्मा दस्मान् नि॒र्मार्गु॑का नि॒र्मार्गु॑का अस्मात् । \newline
44. नि॒र्मार्गु॑का॒ इति॑ निः - मार्गु॑काः । \newline
45. अ॒स्मा॒त् प॒शवः॑ प॒शवो᳚ ऽस्मा दस्मात् प॒शवः॑ । \newline
46. प॒शवः॑ स्युः स्युः प॒शवः॑ प॒शवः॑ स्युः । \newline
47. स्युः॒ प्र॒ती॒पम् प्र॑ती॒पꣳ स्युः॑ स्युः प्रती॒पम् । \newline
48. प्र॒ती॒पम् तिष्ठꣳ॒॒ स्तिष्ठ॑न् प्रती॒पम् प्र॑ती॒पम् तिष्ठन्न्॑ । \newline
49. तिष्ठ॑न् गृह्णाति गृह्णाति॒ तिष्ठꣳ॒॒ स्तिष्ठ॑न् गृह्णाति । \newline
50. गृ॒ह्णा॒ति॒ प्र॒ति॒रुद्ध्य॑ प्रति॒रुद्ध्य॑ गृह्णाति गृह्णाति प्रति॒रुद्ध्य॑ । \newline
51. प्र॒ति॒रुद्ध्यै॒वैव प्र॑ति॒रुद्ध्य॑ प्रति॒रुद्ध्यै॒व । \newline
52. प्र॒ति॒रुद्ध्येति॑ प्रति - रुद्ध्य॑ । \newline
53. ए॒वास्मा॑ अस्मा ए॒वै वास्मै᳚ । \newline
54. अ॒स्मै॒ प॒शून् प॒शून॑ स्मा अस्मै प॒शून् । \newline
55. प॒शून् गृ॑ह्णाति गृह्णाति प॒शून् प॒शून् गृ॑ह्णाति । \newline
56. गृ॒ह्णा॒ तीन्द्र॒ इन्द्रो॑ गृह्णाति गृह्णा॒ तीन्द्रः॑ । \newline
57. इन्द्रो॑ वृ॒त्रं ॅवृ॒त्र मिन्द्र॒ इन्द्रो॑ वृ॒त्रम् । \newline

\textbf{Ghana Paata } \newline

1. य॒ज्ञ्ं ॅवि वि य॒ज्ञ्ं ॅय॒ज्ञ्ं ॅवि च्छि॑न्द्याच् छिन्द्या॒द् वि य॒ज्ञ्ं ॅय॒ज्ञ्ं ॅवि च्छि॑न्द्यात् । \newline
2. वि च्छि॑न्द्याच् छिन्द्या॒द् वि वि च्छि॑न्द्याज् ज्योति॒ष्या᳚ ज्योति॒ष्या॑ छिन्द्या॒द् वि वि च्छि॑न्द्याज् ज्योति॒ष्या᳚ । \newline
3. छि॒न्द्या॒ज् ज्यो॒ति॒ष्या᳚ ज्योति॒ष्या॑ छिन्द्याच् छिन्द्याज् ज्योति॒ष्या॑ वा वा ज्योति॒ष्या॑ छिन्द्याच् छिन्द्याज् ज्योति॒ष्या॑ वा । \newline
4. ज्यो॒ति॒ष्या॑ वा वा ज्योति॒ष्या᳚ ज्योति॒ष्या॑ वा गृह्णी॒याद् गृ॑ह्णी॒याद् वा᳚ ज्योति॒ष्या᳚ ज्योति॒ष्या॑ वा गृह्णी॒यात् । \newline
5. वा॒ गृ॒ह्णी॒याद् गृ॑ह्णी॒याद् वा॑ वा गृह्णी॒या द्धिर॑ण्यꣳ॒॒ हिर॑ण्यम् गृह्णी॒याद् वा॑ वा गृह्णी॒या द्धिर॑ण्यम् । \newline
6. गृ॒ह्णी॒या द्धिर॑ण्यꣳ॒॒ हिर॑ण्यम् गृह्णी॒याद् गृ॑ह्णी॒या द्धिर॑ण्यं ॅवा वा॒ हिर॑ण्यम् गृह्णी॒याद् गृ॑ह्णी॒या द्धिर॑ण्यं ॅवा । \newline
7. हिर॑ण्यं ॅवा वा॒ हिर॑ण्यꣳ॒॒ हिर॑ण्यं ॅवा ऽव॒धाया॑ व॒धाय॑ वा॒ हिर॑ण्यꣳ॒॒ हिर॑ण्यं ॅवा ऽव॒धाय॑ । \newline
8. वा॒ ऽव॒धाया॑ व॒धाय॑ वा वा ऽव॒धाय॒ सशु॑क्राणाꣳ॒॒ सशु॑क्राणा मव॒धाय॑ वा वा ऽव॒धाय॒ सशु॑क्राणाम् । \newline
9. अ॒व॒धाय॒ सशु॑क्राणाꣳ॒॒ सशु॑क्राणा मव॒धाया॑ व॒धाय॒ सशु॑क्राणा मे॒वैव सशु॑क्राणा मव॒धाया॑ व॒धाय॒ सशु॑क्राणा मे॒व । \newline
10. अ॒व॒धायेत्य॑व - धाय॑ । \newline
11. सशु॑क्राणा मे॒वैव सशु॑क्राणाꣳ॒॒ सशु॑क्राणा मे॒व गृ॑ह्णाति गृह्णा त्ये॒व सशु॑क्राणाꣳ॒॒ सशु॑क्राणा मे॒व गृ॑ह्णाति । \newline
12. सशु॑क्राणा॒मिति॒ स - शु॒क्रा॒णा॒म् । \newline
13. ए॒व गृ॑ह्णाति गृह्णा त्ये॒वैव गृ॑ह्णाति॒ यो यो गृ॑ह्णा त्ये॒वैव गृ॑ह्णाति॒ यः । \newline
14. गृ॒ह्णा॒ति॒ यो यो गृ॑ह्णाति गृह्णाति॒ यो वा॑ वा॒ यो गृ॑ह्णाति गृह्णाति॒ यो वा᳚ । \newline
15. यो वा॑ वा॒ यो यो वा᳚ ब्राह्म॒णो ब्रा᳚ह्म॒णो वा॒ यो यो वा᳚ ब्राह्म॒णः । \newline
16. वा॒ ब्रा॒ह्म॒णो ब्रा᳚ह्म॒णो वा॑ वा ब्राह्म॒णो ब॑हुया॒जी ब॑हुया॒जी ब्रा᳚ह्म॒णो वा॑ वा ब्राह्म॒णो ब॑हुया॒जी । \newline
17. ब्रा॒ह्म॒णो ब॑हुया॒जी ब॑हुया॒जी ब्रा᳚ह्म॒णो ब्रा᳚ह्म॒णो ब॑हुया॒जी तस्य॒ तस्य॑ बहुया॒जी ब्रा᳚ह्म॒णो ब्रा᳚ह्म॒णो ब॑हुया॒जी तस्य॑ । \newline
18. ब॒हु॒या॒जी तस्य॒ तस्य॑ बहुया॒जी ब॑हुया॒जी तस्य॒ कुंभ्या॑ना॒म् कुंभ्या॑ना॒म् तस्य॑ बहुया॒जी ब॑हुया॒जी तस्य॒ कुंभ्या॑नाम् । \newline
19. ब॒हु॒या॒जीति॑ बहु - या॒जी । \newline
20. तस्य॒ कुंभ्या॑ना॒म् कुंभ्या॑ना॒म् तस्य॒ तस्य॒ कुंभ्या॑नाम् गृह्णीयाद् गृह्णीया॒त् कुंभ्या॑ना॒म् तस्य॒ तस्य॒ कुंभ्या॑नाम् गृह्णीयात् । \newline
21. कुंभ्या॑नाम् गृह्णीयाद् गृह्णीया॒त् कुंभ्या॑ना॒म् कुंभ्या॑नाम् गृह्णीया॒थ् स स गृ॑ह्णीया॒त् कुंभ्या॑ना॒म् कुंभ्या॑नाम् गृह्णीया॒थ् सः । \newline
22. गृ॒ह्णी॒या॒थ् स स गृ॑ह्णीयाद् गृह्णीया॒थ् स हि हि स गृ॑ह्णीयाद् गृह्णीया॒थ् स हि । \newline
23. स हि हि स स हि गृ॑ही॒तव॑सतीवरीको गृही॒तव॑सतीवरीको॒ हि स स हि गृ॑ही॒तव॑सतीवरीकः । \newline
24. हि गृ॑ही॒तव॑सतीवरीको गृही॒तव॑सतीवरीको॒ हि हि गृ॑ही॒तव॑सतीवरीको वसती॒वरी᳚र् वसती॒वरी᳚र् गृही॒तव॑सतीवरीको॒ हि हि गृ॑ही॒तव॑सतीवरीको वसती॒वरीः᳚ । \newline
25. गृ॒ही॒तव॑सतीवरीको वसती॒वरी᳚र् वसती॒वरी᳚र् गृही॒तव॑सतीवरीको गृही॒तव॑सतीवरीको वसती॒वरी᳚र् गृह्णाति गृह्णाति वसती॒वरी᳚र् गृही॒तव॑सतीवरीको गृही॒तव॑सतीवरीको वसती॒वरी᳚र् गृह्णाति । \newline
26. गृ॒ही॒तव॑सतीवरीक॒ इति॑ गृही॒त - व॒स॒ती॒व॒री॒कः॒ । \newline
27. व॒स॒ती॒वरी᳚र् गृह्णाति गृह्णाति वसती॒वरी᳚र् वसती॒वरी᳚र् गृह्णाति प॒शवः॑ प॒शवो॑ गृह्णाति वसती॒वरी᳚र् वसती॒वरी᳚र् गृह्णाति प॒शवः॑ । \newline
28. गृ॒ह्णा॒ति॒ प॒शवः॑ प॒शवो॑ गृह्णाति गृह्णाति प॒शवो॒ वै वै प॒शवो॑ गृह्णाति गृह्णाति प॒शवो॒ वै । \newline
29. प॒शवो॒ वै वै प॒शवः॑ प॒शवो॒ वै व॑सती॒वरी᳚र् वसती॒वरी॒र् वै प॒शवः॑ प॒शवो॒ वै व॑सती॒वरीः᳚ । \newline
30. वै व॑सती॒वरी᳚र् वसती॒वरी॒र् वै वै व॑सती॒वरीः᳚ प॒शून् प॒शून्. व॑सती॒वरी॒र् वै वै व॑सती॒वरीः᳚ प॒शून् । \newline
31. व॒स॒ती॒वरीः᳚ प॒शून् प॒शून्. व॑सती॒वरी᳚र् वसती॒वरीः᳚ प॒शूने॒ वैव प॒शून्. व॑सती॒वरी᳚र् वसती॒वरीः᳚ प॒शू ने॒व । \newline
32. प॒शूने॒ वैव प॒शून् प॒शू ने॒वारभ्या॒ रभ्यै॒व प॒शून् प॒शू ने॒वारभ्य॑ । \newline
33. ए॒वारभ्या॒ रभ्यै॒ वैवा रभ्य॑ गृही॒त्वा गृ॑ही॒त्वा ऽऽरभ्यै॒ वैवा रभ्य॑ गृही॒त्वा । \newline
34. आ॒रभ्य॑ गृही॒त्वा गृ॑ही॒त्वा ऽऽरभ्या॒ रभ्य॑ गृही॒त्वोपोप॑ गृही॒त्वा ऽऽरभ्या॒ रभ्य॑ गृही॒त्वोप॑ । \newline
35. आ॒रभ्येत्या᳚ - रभ्य॑ । \newline
36. गृ॒ही॒त्वो पोप॑ गृही॒त्वा गृ॑ही॒ त्वोप॑ वसति वस॒ त्युप॑ गृही॒त्वा गृ॑ही॒ त्वोप॑ वसति । \newline
37. उप॑ वसति वस॒ त्युपोप॑ वसति॒ यद् यद् व॑स॒ त्युपोप॑ वसति॒ यत् । \newline
38. व॒स॒ति॒ यद् यद् व॑सति वसति॒ यद॑न्वी॒प म॑न्वी॒पं ॅयद् व॑सति वसति॒ यद॑न्वी॒पम् । \newline
39. यद॑न्वी॒प म॑न्वी॒पं ॅयद् यद॑न्वी॒पम् तिष्ठꣳ॒॒ स्तिष्ठ॑न् नन्वी॒पं ॅयद् यद॑न्वी॒पम् तिष्ठन्न्॑ । \newline
40. अ॒न्वी॒पम् तिष्ठꣳ॒॒ स्तिष्ठ॑न् नन्वी॒प म॑न्वी॒पम् तिष्ठ॑न् गृह्णी॒याद् गृ॑ह्णी॒यात् तिष्ठ॑न् नन्वी॒प म॑न्वी॒पम् तिष्ठ॑न् गृह्णी॒यात् । \newline
41. तिष्ठ॑न् गृह्णी॒याद् गृ॑ह्णी॒यात् तिष्ठꣳ॒॒ स्तिष्ठ॑न् गृह्णी॒यान् नि॒र्मार्गु॑का नि॒र्मार्गु॑का गृह्णी॒यात् तिष्ठꣳ॒॒ स्तिष्ठ॑न् गृह्णी॒यान् नि॒र्मार्गु॑काः । \newline
42. गृ॒ह्णी॒यान् नि॒र्मार्गु॑का नि॒र्मार्गु॑का गृह्णी॒याद् गृ॑ह्णी॒यान् नि॒र्मार्गु॑का अस्मा दस्मान् नि॒र्मार्गु॑का गृह्णी॒याद् गृ॑ह्णी॒यान् नि॒र्मार्गु॑का अस्मात् । \newline
43. नि॒र्मार्गु॑का अस्मा दस्मान् नि॒र्मार्गु॑का नि॒र्मार्गु॑का अस्मात् प॒शवः॑ प॒शवो᳚ ऽस्मान् नि॒र्मार्गु॑का नि॒र्मार्गु॑का अस्मात् प॒शवः॑ । \newline
44. नि॒र्मार्गु॑का॒ इति॑ निः - मार्गु॑काः । \newline
45. अ॒स्मा॒त् प॒शवः॑ प॒शवो᳚ ऽस्मा दस्मात् प॒शवः॑ स्युः स्युः प॒शवो᳚ ऽस्मा दस्मात् प॒शवः॑ स्युः । \newline
46. प॒शवः॑ स्युः स्युः प॒शवः॑ प॒शवः॑ स्युः प्रती॒पम् प्र॑ती॒पꣳ स्युः॑ प॒शवः॑ प॒शवः॑ स्युः प्रती॒पम् । \newline
47. स्युः॒ प्र॒ती॒पम् प्र॑ती॒पꣳ स्युः॑ स्युः प्रती॒पम् तिष्ठꣳ॒॒ स्तिष्ठ॑न् प्रती॒पꣳ स्युः॑ स्युः प्रती॒पम् तिष्ठन्न्॑ । \newline
48. प्र॒ती॒पम् तिष्ठꣳ॒॒ स्तिष्ठ॑न् प्रती॒पम् प्र॑ती॒पम् तिष्ठ॑न् गृह्णाति गृह्णाति॒ तिष्ठ॑न् प्रती॒पम् प्र॑ती॒पम् तिष्ठ॑न् गृह्णाति । \newline
49. तिष्ठ॑न् गृह्णाति गृह्णाति॒ तिष्ठꣳ॒॒ स्तिष्ठ॑न् गृह्णाति प्रति॒रुद्ध्य॑ प्रति॒रुद्ध्य॑ गृह्णाति॒ तिष्ठꣳ॒॒ स्तिष्ठ॑न् गृह्णाति प्रति॒रुद्ध्य॑ । \newline
50. गृ॒ह्णा॒ति॒ प्र॒ति॒रुद्ध्य॑ प्रति॒रुद्ध्य॑ गृह्णाति गृह्णाति प्रति॒रुद्ध्यै॒ वैव प्र॑ति॒रुद्ध्य॑ गृह्णाति गृह्णाति प्रति॒रुद्ध्यै॒व । \newline
51. प्र॒ति॒रुद्ध्यै॒ वैव प्र॑ति॒रुद्ध्य॑ प्रति॒रुद्ध्यै॒ वास्मा॑ अस्मा ए॒व प्र॑ति॒रुद्ध्य॑ प्रति॒रुद्ध्यै॒ वास्मै᳚ । \newline
52. प्र॒ति॒रुद्ध्येति॑ प्रति - रुद्ध्य॑ । \newline
53. ए॒वास्मा॑ अस्मा ए॒वै वास्मै॑ प॒शून् प॒शून॑ स्मा ए॒वै वास्मै॑ प॒शून् । \newline
54. अ॒स्मै॒ प॒शून् प॒शून॑स्मा अस्मै प॒शून् गृ॑ह्णाति गृह्णाति प॒शून॑स्मा अस्मै प॒शून् गृ॑ह्णाति । \newline
55. प॒शून् गृ॑ह्णाति गृह्णाति प॒शून् प॒शून् गृ॑ह्णा॒ तीन्द्र॒ इन्द्रो॑ गृह्णाति प॒शून् प॒शून् गृ॑ह्णा॒ तीन्द्रः॑ । \newline
56. गृ॒ह्णा॒ तीन्द्र॒ इन्द्रो॑ गृह्णाति गृह्णा॒ तीन्द्रो॑ वृ॒त्रं ॅवृ॒त्र मिन्द्रो॑ गृह्णाति गृह्णा॒ तीन्द्रो॑ वृ॒त्रम् । \newline
57. इन्द्रो॑ वृ॒त्रं ॅवृ॒त्र मिन्द्र॒ इन्द्रो॑ वृ॒त्र म॑हन् नहन् वृ॒त्र मिन्द्र॒ इन्द्रो॑ वृ॒त्र म॑हन्न् । \newline
\pagebreak
\markright{ TS 6.4.2.3  \hfill https://www.vedavms.in \hfill}

\section{ TS 6.4.2.3 }

\textbf{TS 6.4.2.3 } \newline
\textbf{Samhita Paata} \newline

वृ॒त्रम॑ह॒न्थ् सो᳚ऽ(1॒)पो᳚ऽ(1॒)भ्य॑म्रियत॒ तासां॒ ॅयन्मेद्ध्यं॑ ॅय॒ज्ञियꣳ॒॒ सदे॑व॒मासी॒त् तदत्य॑मुच्यत॒ ता वह॑न्तीरभव॒न् वह॑न्तीनां गृह्णाति॒ या ए॒व मेद्ध्या॑ य॒ज्ञियाः॒ सदे॑वा॒ आप॒स्ता सा॑मे॒व गृ॑ह्णाति॒ नान्त॒मा वह॑न्ती॒रती॑या॒द्- यद॑न्त॒मा वह॑न्तीरती॒याद् य॒ज्ञ्मति॑ मन्येत॒ न स्था॑व॒राणां᳚ गृह्णीया॒द् वरु॑णगृहीता॒ वै स्था॑व॒रा यथ् स्था॑व॒राणां᳚ं गृह्णी॒याद्- [  ] \newline

\textbf{Pada Paata} \newline

वृ॒त्रम् । अ॒ह॒न्न् । सः । अ॒पः । अ॒भीति॑ । अ॒म्रि॒य॒त॒ । तासा᳚म् । यत् । मेद्ध्य᳚म् । य॒ज्ञिय᳚म् । सदे॑व॒मिति॒ स - दे॒व॒म् । आसी᳚त् । तत् । अतीति॑ । अ॒मु॒च्य॒त॒ । ताः । वह॑न्तीः । अ॒भ॒व॒न्न् । वह॑न्तीनाम् । गृ॒ह्णा॒ति॒ । याः । ए॒व । मेद्ध्याः᳚ । य॒ज्ञियाः᳚ । सदे॑वा॒ इति॒ स - दे॒वाः॒ । आपः॑ । तासा᳚म् । ए॒व । गृ॒ह्णा॒ति॒ । न । अ॒न्त॒माः । वह॑न्तीः । अतीति॑ । इ॒या॒त् । यत् । अ॒न्त॒माः । वह॑न्तीः । अ॒ती॒यादित्य॑ति-इ॒यात् । य॒ज्ञ्म् । अतीति॑ । म॒न्ये॒त॒ । न । स्था॒व॒राणा᳚म् । गृ॒ह्णी॒या॒त् । वरु॑णगृहीता॒ इति॒ वरु॑ण - गृ॒ही॒ताः॒ । वै । स्था॒व॒राः । यत् । स्था॒व॒राणा᳚म् । गृ॒ह्णी॒यात् ।  \newline


\textbf{Krama Paata} \newline

वृ॒त्रम॑हन्न् । अ॒ह॒न्थ् सः । सो॑ऽपः । अ॒पो॑ऽभि । अ॒भ्य॑म्रियत । अ॒म्रि॒य॒त॒ तासा᳚म् । तासा॒म् ॅयत् । यन् मेद्ध्य᳚म् । मेद्ध्य॑म् ॅय॒ज्ञिय᳚म् । य॒ज्ञियꣳ॒॒ सदे॑वम् । सदे॑व॒मासी᳚त् । सदे॑व॒मिति॒ स - दे॒व॒म् । आसी॒त् तत् । तदति॑ । अत्य॑मुच्यत । अ॒मु॒च्य॒त॒ ताः । ता वह॑न्तीः । वह॑न्तिरभवन्न् । अ॒भ॒व॒न् वह॑न्तीनाम् । वह॑न्तीनाम् गृह्णाति । गृ॒ह्णा॒ति॒ याः । या ए॒व । ए॒व मेद्ध्याः᳚ । मेद्ध्या॑ य॒ज्ञियाः᳚ । य॒ज्ञियाः॒ सदे॑वाः । सदे॑वा॒ आपः॑ । सदे॑वा॒ इति॒ स - दे॒वाः॒ । आप॒स्तासा᳚म् । तासा॑मे॒व । ए॒व गृ॑ह्णाति । गृ॒ह्णा॒ति॒ न । नान्त॒माः । अ॒न्त॒मा वह॑न्तीः । वह॑न्ती॒रति॑ । अती॑यात् । इ॒या॒द् यत् । यद॑न्त॒माः । अ॒न्त॒मा वह॑न्तीः । वह॑न्तीरती॒यात् । अ॒ती॒याद् य॒ज्ञ्म् । अ॒ती॒यादित्य॑ति - इ॒यात् । य॒ज्ञ्मति॑ । अति॑ मन्येत । म॒न्ये॒त॒ न । न स्था॑व॒राणा᳚म् । स्था॒व॒राणा᳚म् गृह्णीयात् । गृ॒ह्णी॒या॒द् वरु॑णगृहीताः । वरु॑णगृहीता॒ वै । वरु॑णगृहीता॒ इति॒ वरु॑ण - गृ॒ही॒ताः॒ । वै स्था॑व॒राः । स्था॒व॒रा यत् । यथ् स्था॑व॒राणा᳚म् । स्था॒व॒राणा᳚म् गृह्णी॒यात् । गृ॒ह्णी॒याद् वरु॑णेन \newline

\textbf{Jatai Paata} \newline

1. वृ॒त्र म॑हन् नहन् वृ॒त्रं ॅवृ॒त्र म॑हन्न् । \newline
2. अ॒ह॒न् थ्स सो॑ ऽहन् नह॒न् थ्सः । \newline
3. सो᳚(1॒) ऽपो॑ ऽपः स सो॑ ऽपः । \newline
4. अ॒पो᳚(1॒) ऽभ्या᳚(1॒)भ्या᳚(1॒)पो᳚(1॒) ऽपो॑ ऽभि । \newline
5. अ॒भ्य॑ म्रियता म्रियता॒ भ्या᳚(1॒)भ्य॑ म्रियत । \newline
6. अ॒म्रि॒य॒त॒ तासा॒म् तासा॑ मम्रियता म्रियत॒ तासा᳚म् । \newline
7. तासां॒ ॅयद् यत् तासा॒म् तासां॒ ॅयत् । \newline
8. यन् मेद्ध्य॒म् मेद्ध्यं॒ ॅयद् यन् मेद्ध्य᳚म् । \newline
9. मेद्ध्यं॑ ॅय॒ज्ञियं॑ ॅय॒ज्ञिय॒म् मेद्ध्य॒म् मेद्ध्यं॑ ॅय॒ज्ञिय᳚म् । \newline
10. य॒ज्ञियꣳ॒॒ सदे॑वꣳ॒॒ सदे॑वं ॅय॒ज्ञियं॑ ॅय॒ज्ञियꣳ॒॒ सदे॑वम् । \newline
11. सदे॑व॒ मासी॒ दासी॒थ् सदे॑वꣳ॒॒ सदे॑व॒ मासी᳚त् । \newline
12. सदे॑व॒मिति॒ स - दे॒व॒म् । \newline
13. आसी॒त् तत् तदासी॒ दासी॒त् तत् । \newline
14. तद त्यति॒ तत् तदति॑ । \newline
15. अत्य॑मुच्यता मुच्य॒ता त्य त्य॑मुच्यत । \newline
16. अ॒मु॒च्य॒त॒ ता स्ता अ॑मुच्यता मुच्यत॒ ताः । \newline
17. ता वह॑न्ती॒र् वह॑न्ती॒ स्ता स्ता वह॑न्तीः । \newline
18. वह॑न्ती रभवन् नभव॒न्॒. वह॑न्ती॒र् वह॑न्ती रभवन्न् । \newline
19. अ॒भ॒व॒न्॒. वह॑न्तीनां॒ ॅवह॑न्तीना मभवन् नभव॒न्॒. वह॑न्तीनाम् । \newline
20. वह॑न्तीनाम् गृह्णाति गृह्णाति॒ वह॑न्तीनां॒ ॅवह॑न्तीनाम् गृह्णाति । \newline
21. गृ॒ह्णा॒ति॒ या या गृ॑ह्णाति गृह्णाति॒ याः । \newline
22. या ए॒वैव या या ए॒व । \newline
23. ए॒व मेद्ध्या॒ मेद्ध्या॑ ए॒वैव मेद्ध्याः᳚ । \newline
24. मेद्ध्या॑ य॒ज्ञिया॑ य॒ज्ञिया॒ मेद्ध्या॒ मेद्ध्या॑ य॒ज्ञियाः᳚ । \newline
25. य॒ज्ञियाः॒ सदे॑वाः॒ सदे॑वा य॒ज्ञिया॑ य॒ज्ञियाः॒ सदे॑वाः । \newline
26. सदे॑वा॒ आप॒ आपः॒ सदे॑वाः॒ सदे॑वा॒ आपः॑ । \newline
27. सदे॑वा॒ इति॒ स - दे॒वाः॒ । \newline
28. आप॒ स्तासा॒म् तासा॒ माप॒ आप॒ स्तासा᳚म् । \newline
29. तासा॑ मे॒वैव तासा॒म् तासा॑ मे॒व । \newline
30. ए॒व गृ॑ह्णाति गृह्णा त्ये॒वैव गृ॑ह्णाति । \newline
31. गृ॒ह्णा॒ति॒ न न गृ॑ह्णाति गृह्णाति॒ न । \newline
32. नान्त॒मा अ॑न्त॒मा न नान्त॒माः । \newline
33. अ॒न्त॒मा वह॑न्ती॒र् वह॑न्ती रन्त॒मा अ॑न्त॒मा वह॑न्तीः । \newline
34. वह॑न्ती॒ रत्यति॒ वह॑न्ती॒र् वह॑न्ती॒ रति॑ । \newline
35. अती॑या दिया॒ दत्य ती॑यात् । \newline
36. इ॒या॒द् यद् यदि॑या दिया॒द् यत् । \newline
37. यद॑न्त॒मा अ॑न्त॒मा यद् यद॑न्त॒माः । \newline
38. अ॒न्त॒मा वह॑न्ती॒र् वह॑न्ती रन्त॒मा अ॑न्त॒मा वह॑न्तीः । \newline
39. वह॑न्ती रती॒या द॑ती॒याद् वह॑न्ती॒र् वह॑न्ती रती॒यात् । \newline
40. अ॒ती॒याद् य॒ज्ञ्ं ॅय॒ज्ञ् म॑ती॒या द॑ती॒याद् य॒ज्ञ्म् । \newline
41. अ॒ती॒यादित्य॑ति - इ॒यात् । \newline
42. य॒ज्ञ् मत्यति॑ य॒ज्ञ्ं ॅय॒ज्ञ् मति॑ । \newline
43. अति॑ मन्येत मन्ये॒ता त्यति॑ मन्येत । \newline
44. म॒न्ये॒त॒ न न म॑न्येत मन्येत॒ न । \newline
45. न स्था॑व॒राणाꣳ॑ स्थाव॒राणा॒न् न न स्था॑व॒राणा᳚म् । \newline
46. स्था॒व॒राणा᳚म् गृह्णीयाद् गृह्णीयाथ् स्थाव॒राणाꣳ॑ स्थाव॒राणा᳚म् गृह्णीयात् । \newline
47. गृ॒ह्णी॒या॒द् वरु॑णगृहीता॒ वरु॑णगृहीता गृह्णीयाद् गृह्णीया॒द् वरु॑णगृहीताः । \newline
48. वरु॑णगृहीता॒ वै वै वरु॑णगृहीता॒ वरु॑णगृहीता॒ वै । \newline
49. वरु॑णगृहीता॒ इति॒ वरु॑ण - गृ॒ही॒ताः॒ । \newline
50. वै स्था॑व॒राः स्था॑व॒रा वै वै स्था॑व॒राः । \newline
51. स्था॒व॒रा यद् यथ् स्था॑व॒राः स्था॑व॒रा यत् । \newline
52. यथ् स्था॑व॒राणाꣳ॑ स्थाव॒राणां॒ ॅयद् यथ् स्था॑व॒राणा᳚म् । \newline
53. स्था॒व॒राणा᳚म् गृह्णी॒याद् गृ॑ह्णी॒याथ् स्था॑व॒राणाꣳ॑ स्थाव॒राणा᳚म् गृह्णी॒यात् । \newline
54. गृ॒ह्णी॒याद् वरु॑णेन॒ वरु॑णेन गृह्णी॒याद् गृ॑ह्णी॒याद् वरु॑णेन । \newline

\textbf{Ghana Paata } \newline

1. वृ॒त्र म॑हन् नहन् वृ॒त्रं ॅवृ॒त्र म॑ह॒न् थ्स सो॑ ऽहन् वृ॒त्रं ॅवृ॒त्र म॑ह॒न् थ्सः । \newline
2. अ॒ह॒न् थ्स सो॑ ऽहन् नह॒न् थ्सो᳚(1॒) ऽपो॑ ऽपः सो॑ ऽहन् नह॒न् थ्सो॑ ऽपः । \newline
3. सो᳚(1॒) ऽपो॑ ऽपः स सो᳚(1॒) ऽपो᳚(1॒) ऽभ्या᳚(1॒)भ्य॑पः स सो᳚(1॒) ऽपो॑ ऽभि । \newline
4. अ॒पो᳚(1॒) ऽभ्या᳚(1॒)भ्या᳚(1॒)पो᳚(1॒) ऽपो᳚(1॒) ऽभ्य॑ म्रियता म्रियता॒भ्या᳚(1॒)पो᳚(1॒) ऽपो᳚(1॒) ऽभ्य॑म्रियत । \newline
5. अ॒भ्य॑म्रियता म्रियता॒भ्या᳚(1॒)भ्य॑म्रियत॒ तासा॒म् तासा॑ मम्रियता॒भ्या᳚(1॒)भ्य॑म्रियत॒ तासा᳚म् । \newline
6. अ॒म्रि॒य॒त॒ तासा॒म् तासा॑ मम्रियता म्रियत॒ तासां॒ ॅयद् यत् तासा॑ मम्रियता म्रियत॒ तासां॒ ॅयत् । \newline
7. तासां॒ ॅयद् यत् तासा॒म् तासां॒ ॅयन् मेद्ध्य॒म् मेद्ध्यं॒ ॅयत् तासा॒म् तासां॒ ॅयन् मेद्ध्य᳚म् । \newline
8. यन् मेद्ध्य॒म् मेद्ध्यं॒ ॅयद् यन् मेद्ध्यं॑ ॅय॒ज्ञियं॑ ॅय॒ज्ञिय॒म् मेद्ध्यं॒ ॅयद् यन् मेद्ध्यं॑ ॅय॒ज्ञिय᳚म् । \newline
9. मेद्ध्यं॑ ॅय॒ज्ञियं॑ ॅय॒ज्ञिय॒म् मेद्ध्य॒म् मेद्ध्यं॑ ॅय॒ज्ञियꣳ॒॒ सदे॑वꣳ॒॒ सदे॑वं ॅय॒ज्ञिय॒म् मेद्ध्य॒म् मेद्ध्यं॑ ॅय॒ज्ञियꣳ॒॒ सदे॑वम् । \newline
10. य॒ज्ञियꣳ॒॒ सदे॑वꣳ॒॒ सदे॑वं ॅय॒ज्ञियं॑ ॅय॒ज्ञियꣳ॒॒ सदे॑व॒ मासी॒ दासी॒थ् सदे॑वं ॅय॒ज्ञियं॑ ॅय॒ज्ञियꣳ॒॒ सदे॑व॒ मासी᳚त् । \newline
11. सदे॑व॒ मासी॒ दासी॒थ् सदे॑वꣳ॒॒ सदे॑व॒ मासी॒त् तत् तदासी॒थ् सदे॑वꣳ॒॒ सदे॑व॒ मासी॒त् तत् । \newline
12. सदे॑व॒मिति॒ स - दे॒व॒म् । \newline
13. आसी॒त् तत् तदासी॒ दासी॒त् तद त्यति॒ तदासी॒ दासी॒त् तदति॑ । \newline
14. तद त्यति॒ तत् तद त्य॑मुच्यता मुच्य॒ ताति॒ तत् तदत्य॑ मुच्यत । \newline
15. अत्य॑मुच्यता मुच्य॒ता त्यत्य॑ मुच्यत॒ ता स्ता अ॑मुच्य॒ता त्यत्य॑ मुच्यत॒ ताः । \newline
16. अ॒मु॒च्य॒त॒ ता स्ता अ॑मुच्यता मुच्यत॒ ता वह॑न्ती॒र् वह॑न्ती॒ स्ता अ॑मुच्यता मुच्यत॒ ता वह॑न्तीः । \newline
17. ता वह॑न्ती॒र् वह॑न्ती॒ स्ता स्ता वह॑न्ती रभवन् नभव॒न्॒. वह॑न्ती॒ स्ता स्ता वह॑न्ती रभवन्न् । \newline
18. वह॑न्ती रभवन् नभव॒न्॒. वह॑न्ती॒र् वह॑न्ती रभव॒न्॒. वह॑न्तीनां॒ ॅवह॑न्तीना मभव॒न्॒. वह॑न्ती॒र् वह॑न्ती रभव॒न्॒. वह॑न्तीनाम् । \newline
19. अ॒भ॒व॒न्॒. वह॑न्तीनां॒ ॅवह॑न्तीना मभवन् नभव॒न्॒. वह॑न्तीनाम् गृह्णाति गृह्णाति॒ वह॑न्तीना मभवन् नभव॒न्॒. वह॑न्तीनाम् गृह्णाति । \newline
20. वह॑न्तीनाम् गृह्णाति गृह्णाति॒ वह॑न्तीनां॒ ॅवह॑न्तीनाम् गृह्णाति॒ या या गृ॑ह्णाति॒ वह॑न्तीनां॒ ॅवह॑न्तीनाम् गृह्णाति॒ याः । \newline
21. गृ॒ह्णा॒ति॒ या या गृ॑ह्णाति गृह्णाति॒ या ए॒वैव या गृ॑ह्णाति गृह्णाति॒ या ए॒व । \newline
22. या ए॒वैव या या ए॒व मेद्ध्या॒ मेद्ध्या॑ ए॒व या या ए॒व मेद्ध्याः᳚ । \newline
23. ए॒व मेद्ध्या॒ मेद्ध्या॑ ए॒वैव मेद्ध्या॑ य॒ज्ञिया॑ य॒ज्ञिया॒ मेद्ध्या॑ ए॒वैव मेद्ध्या॑ य॒ज्ञियाः᳚ । \newline
24. मेद्ध्या॑ य॒ज्ञिया॑ य॒ज्ञिया॒ मेद्ध्या॒ मेद्ध्या॑ य॒ज्ञियाः॒ सदे॑वाः॒ सदे॑वा य॒ज्ञिया॒ मेद्ध्या॒ मेद्ध्या॑ य॒ज्ञियाः॒ सदे॑वाः । \newline
25. य॒ज्ञियाः॒ सदे॑वाः॒ सदे॑वा य॒ज्ञिया॑ य॒ज्ञियाः॒ सदे॑वा॒ आप॒ आपः॒ सदे॑वा य॒ज्ञिया॑ य॒ज्ञियाः॒ सदे॑वा॒ आपः॑ । \newline
26. सदे॑वा॒ आप॒ आपः॒ सदे॑वाः॒ सदे॑वा॒ आप॒ स्तासा॒म् तासा॒ मापः॒ सदे॑वाः॒ सदे॑वा॒ आप॒ स्तासा᳚म् । \newline
27. सदे॑वा॒ इति॒ स - दे॒वाः॒ । \newline
28. आप॒ स्तासा॒म् तासा॒ माप॒ आप॒ स्तासा॑ मे॒वैव तासा॒ माप॒ आप॒ स्तासा॑ मे॒व । \newline
29. तासा॑ मे॒वैव तासा॒म् तासा॑ मे॒व गृ॑ह्णाति गृह्णा त्ये॒व तासा॒म् तासा॑ मे॒व गृ॑ह्णाति । \newline
30. ए॒व गृ॑ह्णाति गृह्णा त्ये॒वैव गृ॑ह्णाति॒ न न गृ॑ह्णा त्ये॒वैव गृ॑ह्णाति॒ न । \newline
31. गृ॒ह्णा॒ति॒ न न गृ॑ह्णाति गृह्णाति॒ नान्त॒मा अ॑न्त॒मा न गृ॑ह्णाति गृह्णाति॒ नान्त॒माः । \newline
32. नान्त॒मा अ॑न्त॒मा न नान्त॒मा वह॑न्ती॒र् वह॑न्ती रन्त॒मा न नान्त॒मा वह॑न्तीः । \newline
33. अ॒न्त॒मा वह॑न्ती॒र् वह॑न्ती रन्त॒मा अ॑न्त॒मा वह॑न्ती॒ रत्यति॒ वह॑न्ती रन्त॒मा अ॑न्त॒मा वह॑न्ती॒ रति॑ । \newline
34. वह॑न्ती॒ रत्यति॒ वह॑न्ती॒र् वह॑न्ती॒ रती॑या दिया॒ दति॒ वह॑न्ती॒र् वह॑न्ती॒ रती॑यात् । \newline
35. अती॑या दिया॒ दत्य ती॑या॒द् यद् यदि॑या॒ दत्य ती॑या॒द् यत् । \newline
36. इ॒या॒द् यद् यदि॑या दिया॒द् यद॑न्त॒मा अ॑न्त॒मा यदि॑या दिया॒द् यद॑न्त॒माः । \newline
37. यद॑न्त॒मा अ॑न्त॒मा यद् यद॑न्त॒मा वह॑न्ती॒र् वह॑न्ती रन्त॒मा यद् यद॑न्त॒मा वह॑न्तीः । \newline
38. अ॒न्त॒मा वह॑न्ती॒र् वह॑न्ती रन्त॒मा अ॑न्त॒मा वह॑न्ती रती॒या द॑ती॒याद् वह॑न्ती रन्त॒मा अ॑न्त॒मा वह॑न्ती रती॒यात् । \newline
39. वह॑न्ती रती॒या द॑ती॒याद् वह॑न्ती॒र् वह॑न्ती रती॒याद् य॒ज्ञ्ं ॅय॒ज्ञ् म॑ती॒याद् वह॑न्ती॒र् वह॑न्ती रती॒याद् य॒ज्ञ्म् । \newline
40. अ॒ती॒याद् य॒ज्ञ्ं ॅय॒ज्ञ् म॑ती॒या द॑ती॒याद् य॒ज्ञ् मत्यति॑ य॒ज्ञ् म॑ती॒या द॑ती॒याद् य॒ज्ञ् मति॑ । \newline
41. अ॒ती॒यादित्य॑ति - इ॒यात् । \newline
42. य॒ज्ञ् मत्यति॑ य॒ज्ञ्ं ॅय॒ज्ञ् मति॑ मन्येत मन्ये॒ताति॑ य॒ज्ञ्ं ॅय॒ज्ञ् मति॑ मन्येत । \newline
43. अति॑ मन्येत मन्ये॒ तात्यति॑ मन्येत॒ न न म॑न्ये॒ तात्यति॑ मन्येत॒ न । \newline
44. म॒न्ये॒त॒ न न म॑न्येत मन्येत॒ न स्था॑व॒राणाꣳ॑ स्थाव॒राणा॒न् न म॑न्येत मन्येत॒ न स्था॑व॒राणा᳚म् । \newline
45. न स्था॑व॒राणाꣳ॑ स्थाव॒राणा॒न् न न स्था॑व॒राणा᳚म् गृह्णीयाद् गृह्णीयाथ् स्थाव॒राणा॒न् न न स्था॑व॒राणा᳚म् गृह्णीयात् । \newline
46. स्था॒व॒राणा᳚म् गृह्णीयाद् गृह्णीयाथ् स्थाव॒राणाꣳ॑ स्थाव॒राणा᳚म् गृह्णीया॒द् वरु॑णगृहीता॒ वरु॑णगृहीता गृह्णीयाथ् स्थाव॒राणाꣳ॑ स्थाव॒राणा᳚म् गृह्णीया॒द् वरु॑णगृहीताः । \newline
47. गृ॒ह्णी॒या॒द् वरु॑णगृहीता॒ वरु॑णगृहीता गृह्णीयाद् गृह्णीया॒द् वरु॑णगृहीता॒ वै वै वरु॑णगृहीता गृह्णीयाद् गृह्णीया॒द् वरु॑णगृहीता॒ वै । \newline
48. वरु॑णगृहीता॒ वै वै वरु॑णगृहीता॒ वरु॑णगृहीता॒ वै स्था॑व॒राः स्था॑व॒रा वै वरु॑णगृहीता॒ वरु॑णगृहीता॒ वै स्था॑व॒राः । \newline
49. वरु॑णगृहीता॒ इति॒ वरु॑ण - गृ॒ही॒ताः॒ । \newline
50. वै स्था॑व॒राः स्था॑व॒रा वै वै स्था॑व॒रा यद् यथ् स्था॑व॒रा वै वै स्था॑व॒रा यत् । \newline
51. स्था॒व॒रा यद् यथ् स्था॑व॒राः स्था॑व॒रा यथ् स्था॑व॒राणाꣳ॑ स्थाव॒राणां॒ ॅयथ् स्था॑व॒राः स्था॑व॒रा यथ् स्था॑व॒राणा᳚म् । \newline
52. यथ् स्था॑व॒राणाꣳ॑ स्थाव॒राणां॒ ॅयद् यथ् स्था॑व॒राणा᳚म् गृह्णी॒याद् गृ॑ह्णी॒याथ् स्था॑व॒राणां॒ ॅयद् यथ् स्था॑व॒राणा᳚म् गृह्णी॒यात् । \newline
53. स्था॒व॒राणा᳚म् गृह्णी॒याद् गृ॑ह्णी॒याथ् स्था॑व॒राणाꣳ॑ स्थाव॒राणा᳚म् गृह्णी॒याद् वरु॑णेन॒ वरु॑णेन गृह्णी॒याथ् स्था॑व॒राणाꣳ॑ स्थाव॒राणा᳚म् गृह्णी॒याद् वरु॑णेन । \newline
54. गृ॒ह्णी॒याद् वरु॑णेन॒ वरु॑णेन गृह्णी॒याद् गृ॑ह्णी॒याद् वरु॑णेना स्यास्य॒ वरु॑णेन गृह्णी॒याद् गृ॑ह्णी॒याद् वरु॑णे नास्य । \newline
\pagebreak
\markright{ TS 6.4.2.4  \hfill https://www.vedavms.in \hfill}

\section{ TS 6.4.2.4 }

\textbf{TS 6.4.2.4 } \newline
\textbf{Samhita Paata} \newline

-वरु॑णेनास्य य॒ज्ञ्ं ग्रा॑हये॒द् यद्वै दिवा॒ भव॑त्य॒पो रात्रिः॒ प्र वि॑शति॒ तस्मा᳚त् ता॒म्रा आपो॒ दिवा॑ ददृश्रे॒ यन्नक्तं॒ भव॑त्य॒पोऽहः॒ प्र वि॑शति॒ तस्मा᳚च्च॒न्द्रा आपो॒ नक्तं॑ ददृश्रे छा॒यायै॑ चा॒ऽऽ*तप॑तश्च स॒धौं गृ॑ह्णात्य-होरा॒त्रयो॑रे॒वास्मै॒ वर्णं॑ गृह्णाति ह॒विष्म॑तीरि॒मा आप॒ इत्या॑ह ह॒विष्कृ॑तानामे॒व गृ॑ह्णाति ह॒विष्माꣳ॑ अस्तु॒- [  ] \newline

\textbf{Pada Paata} \newline

वरु॑णेन । अ॒स्य॒ । य॒ज्ञ्म् । ग्रा॒ह॒ये॒त् । यत् । वै । दिवा᳚ । भव॑ति । अ॒पः । रात्रिः॑ । प्रेति॑ । वि॒श॒ति॒ । तस्मा᳚त् । ता॒म्राः । आपः॑ । दिवा᳚ । द॒दृ॒श्रे॒ । यत् । नक्त᳚म् । भव॑ति । अ॒पः । अहः॑ । प्रेति॑ । वि॒श॒ति॒ । तस्मा᳚त् । च॒न्द्राः । आपः॑ । नक्त᳚म् । द॒दृ॒श्रे॒ । छा॒यायै᳚ । च॒ । आ॒तप॑त॒ इत्या᳚ - तप॑तः । च॒ । स॒न्धाविति॑ सं - धौ । गृ॒ह्णा॒ति॒ । अ॒हो॒रा॒त्रयो॒रित्य॑हः - रा॒त्रयोः᳚ । ए॒व । अ॒स्मै॒ । वर्ण᳚म् । गृ॒ह्णा॒ति॒ । ह॒विष्म॑तीः । इ॒माः । आपः॑ । इति॑ । आ॒ह॒ । ह॒विष्कृ॑ताना॒मिति॑ ह॒विः - कृ॒ता॒ना॒म् । ए॒व । गृ॒ह्णा॒ति॒ । ह॒विष्मान्॑ । अ॒स्तु॒ ।  \newline


\textbf{Krama Paata} \newline

वरु॑णेनास्य । अ॒स्य॒ य॒ज्ञ्म् । य॒ज्ञ्म् ग्रा॑हयेत् । ग्रा॒ह॒ये॒द् यत् । यद् वै । वै दिवा᳚ । दिवा॒ भव॑ति । भव॑त्य॒पः । अ॒पो रात्रिः॑ । रात्रिः॒ प्र । प्र वि॑शति । वि॒श॒ति॒ तस्मा᳚त् । तस्मा᳚त् ता॒म्राः । ता॒म्रा आपः॑ । आपो॒ दिवा᳚ । दिवा॑ ददृश्रे । द॒दृ॒श्रे॒ यत् । यन् नक्त᳚म् । नक्त॒म् भव॑ति । भव॑त्य॒पः । अ॒पोऽहः॑ । अहः॒ प्र । प्र वि॑शति । वि॒श॒ति॒ तस्मा᳚त् । तस्मा᳚च् च॒न्द्राः । च॒न्द्रा आपः॑ । आपो॒ नक्त᳚म् । नक्त॑म् ददृश्रे । द॒दृ॒श्रे॒ छा॒यायै᳚ । छा॒यायै॑ च । चा॒तप॑तः । आ॒तप॑तश्च । आ॒तप॑त॒ इत्या᳚ - तप॑तः । च॒ स॒न्धौ । स॒न्धौ गृ॑ह्णाति । स॒न्धाविति॑ सम् - धौ । गृ॒ह्णा॒त्य॒हो॒रा॒त्रयोः᳚ । अ॒हो॒रा॒त्रयो॑रे॒व । अ॒हो॒रा॒त्रयो॒रित्य॑हः - रा॒त्रयोः᳚ । ए॒वास्मै᳚ । अ॒स्मै॒ वर्ण᳚म् । वर्ण॑म् गृह्णाति । गृ॒ह्णा॒ति॒ ह॒वीष्म॑तीः । ह॒विष्म॑तीरि॒माः । इ॒मा आपः॑ । आप॒ इति॑ । इत्या॑ह । आ॒ह॒ ह॒विष्कृ॑तानाम् । ह॒विष्कृ॑तानामे॒व । ह॒विष्कृ॑ताना॒मिति॑ ह॒विः - कृ॒ता॒ना॒म् । ए॒व गृ॑ह्णाति । गृ॒ह्णा॒ति॒ ह॒विष्मान्॑ । ह॒विष्माꣳ॑ अस्तु । अ॒स्तु॒ सूर्यः॑ \newline

\textbf{Jatai Paata} \newline

1. वरु॑णेना स्यास्य॒ वरु॑णेन॒ वरु॑णेनास्य । \newline
2. अ॒स्य॒ य॒ज्ञ्ं ॅय॒ज्ञ् म॑स्यास्य य॒ज्ञ्म् । \newline
3. य॒ज्ञ्म् ग्रा॑हयेद् ग्राहयेद् य॒ज्ञ्ं ॅय॒ज्ञ्म् ग्रा॑हयेत् । \newline
4. ग्रा॒ह॒ये॒द् यद् यद् ग्रा॑हयेद् ग्राहये॒द् यत् । \newline
5. यद् वै वै यद् यद् वै । \newline
6. वै दिवा॒ दिवा॒ वै वै दिवा᳚ । \newline
7. दिवा॒ भव॑ति॒ भव॑ति॒ दिवा॒ दिवा॒ भव॑ति । \newline
8. भव॑ त्य॒पो॑ ऽपो भव॑ति॒ भव॑ त्य॒पः । \newline
9. अ॒पो रात्री॒ रात्रि॑ र॒पो॑ ऽपो रात्रिः॑ । \newline
10. रात्रिः॒ प्र प्र रात्री॒ रात्रिः॒ प्र । \newline
11. प्र वि॑शति विशति॒ प्र प्र वि॑शति । \newline
12. वि॒श॒ति॒ तस्मा॒त् तस्मा᳚द् विशति विशति॒ तस्मा᳚त् । \newline
13. तस्मा᳚त् ता॒म्रा स्ता॒म्रा स्तस्मा॒त् तस्मा᳚त् ता॒म्राः । \newline
14. ता॒म्रा आप॒ आप॑ स्ता॒म्रा स्ता॒म्रा आपः॑ । \newline
15. आपो॒ दिवा॒ दिवा ऽऽप॒ आपो॒ दिवा᳚ । \newline
16. दिवा॑ ददृश्रे ददृश्रे॒ दिवा॒ दिवा॑ ददृश्रे । \newline
17. द॒दृ॒श्रे॒ यद् यद् द॑दृश्रे ददृश्रे॒ यत् । \newline
18. यन् नक्त॒म् नक्तं॒ ॅयद् यन् नक्त᳚म् । \newline
19. नक्त॒म् भव॑ति॒ भव॑ति॒ नक्त॒म् नक्त॒म् भव॑ति । \newline
20. भव॑ त्य॒पो॑ ऽपो भव॑ति॒ भव॑ त्य॒पः । \newline
21. अ॒पो ऽह॒ रह॑ र॒पो॑ ऽपो ऽहः॑ । \newline
22. अहः॒ प्र प्राह॒ रहः॒ प्र । \newline
23. प्र वि॑शति विशति॒ प्र प्र वि॑शति । \newline
24. वि॒श॒ति॒ तस्मा॒त् तस्मा᳚द् विशति विशति॒ तस्मा᳚त् । \newline
25. तस्मा᳚च् च॒न्द्रा श्च॒न्द्रा स्तस्मा॒त् तस्मा᳚च् च॒न्द्राः । \newline
26. च॒न्द्रा आप॒ आप॑ श्च॒न्द्रा श्च॒न्द्रा आपः॑ । \newline
27. आपो॒ नक्त॒म् नक्त॒ माप॒ आपो॒ नक्त᳚म् । \newline
28. नक्त॑म् ददृश्रे ददृश्रे॒ नक्त॒म् नक्त॑म् ददृश्रे । \newline
29. द॒दृ॒श्रे॒ छा॒यायै॑ छा॒यायै॑ ददृश्रे ददृश्रे छा॒यायै᳚ । \newline
30. छा॒यायै॑ च च छा॒यायै॑ छा॒यायै॑ च । \newline
31. चा॒तप॑त आ॒तप॑त श्च चा॒तप॑तः । \newline
32. आ॒तप॑त श्च चा॒तप॑त आ॒तप॑त श्च । \newline
33. आ॒तप॑त॒ इत्या᳚ - तप॑तः । \newline
34. च॒ स॒न्धौ स॒न्धौ च॑ च स॒न्धौ । \newline
35. स॒न्धौ गृ॑ह्णाति गृह्णाति स॒न्धौ स॒न्धौ गृ॑ह्णाति । \newline
36. स॒न्धाविति॑ सं - धौ । \newline
37. गृ॒ह्णा॒ त्य॒हो॒रा॒त्रयो॑ रहोरा॒त्रयो᳚र् गृह्णाति गृह्णा त्यहोरा॒त्रयोः᳚ । \newline
38. अ॒हो॒रा॒त्रयो॑ रे॒वैवा हो॑रा॒त्रयो॑ रहोरा॒त्रयो॑ रे॒व । \newline
39. अ॒हो॒रा॒त्रयो॒रित्य॑हः - रा॒त्रयोः᳚ । \newline
40. ए॒वास्मा॑ अस्मा ए॒वै वास्मै᳚ । \newline
41. अ॒स्मै॒ वर्णं॒ ॅवर्ण॑ मस्मा अस्मै॒ वर्ण᳚म् । \newline
42. वर्ण॑म् गृह्णाति गृह्णाति॒ वर्णं॒ ॅवर्ण॑म् गृह्णाति । \newline
43. गृ॒ह्णा॒ति॒ ह॒विष्म॑तीर्. ह॒विष्म॑तीर् गृह्णाति गृह्णाति ह॒विष्म॑तीः । \newline
44. ह॒विष्म॑ती रि॒मा इ॒मा ह॒विष्म॑तीर्. ह॒विष्म॑ती रि॒माः । \newline
45. इ॒मा आप॒ आप॑ इ॒मा इ॒मा आपः॑ । \newline
46. आप॒ इती त्याप॒ आप॒ इति॑ । \newline
47. इत्या॑हा॒हे तीत्या॑ह । \newline
48. आ॒ह॒ ह॒विष्कृ॑तानाꣳ ह॒विष्कृ॑ताना माहाह ह॒विष्कृ॑तानाम् । \newline
49. ह॒विष्कृ॑ताना मे॒वैव ह॒विष्कृ॑तानाꣳ ह॒विष्कृ॑ताना मे॒व । \newline
50. ह॒विष्कृ॑ताना॒मिति॑ ह॒विः - कृ॒ता॒ना॒म् । \newline
51. ए॒व गृ॑ह्णाति गृह्णा त्ये॒वैव गृ॑ह्णाति । \newline
52. गृ॒ह्णा॒ति॒ ह॒विष्मान्॑. ह॒विष्मा᳚न् गृह्णाति गृह्णाति ह॒विष्मान्॑ । \newline
53. ह॒विष्माꣳ॑ अस्त्वस्तु ह॒विष्मान्॑. ह॒विष्माꣳ॑ अस्तु । \newline
54. अ॒स्तु॒ सूर्यः॒ सूर्यो॑ अस्त्वस्तु॒ सूर्यः॑ । \newline

\textbf{Ghana Paata } \newline

1. वरु॑णेना स्यास्य॒ वरु॑णेन॒ वरु॑णे नास्य य॒ज्ञ्ं ॅय॒ज्ञ् म॑स्य॒ वरु॑णेन॒ वरु॑णे नास्य य॒ज्ञ्म् । \newline
2. अ॒स्य॒ य॒ज्ञ्ं ॅय॒ज्ञ् म॑स्यास्य य॒ज्ञ्म् ग्रा॑हयेद् ग्राहयेद् य॒ज्ञ् म॑स्यास्य य॒ज्ञ्म् ग्रा॑हयेत् । \newline
3. य॒ज्ञ्म् ग्रा॑हयेद् ग्राहयेद् य॒ज्ञ्ं ॅय॒ज्ञ्म् ग्रा॑हये॒द् यद् यद् ग्रा॑हयेद् य॒ज्ञ्ं ॅय॒ज्ञ्म् ग्रा॑हये॒द् यत् । \newline
4. ग्रा॒ह॒ये॒द् यद् यद् ग्रा॑हयेद् ग्राहये॒द् यद् वै वै यद् ग्रा॑हयेद् ग्राहये॒द् यद् वै । \newline
5. यद् वै वै यद् यद् वै दिवा॒ दिवा॒ वै यद् यद् वै दिवा᳚ । \newline
6. वै दिवा॒ दिवा॒ वै वै दिवा॒ भव॑ति॒ भव॑ति॒ दिवा॒ वै वै दिवा॒ भव॑ति । \newline
7. दिवा॒ भव॑ति॒ भव॑ति॒ दिवा॒ दिवा॒ भव॑ त्य॒पो॑ ऽपो भव॑ति॒ दिवा॒ दिवा॒ भव॑ त्य॒पः । \newline
8. भव॑ त्य॒पो॑ ऽपो भव॑ति॒ भव॑ त्य॒पो रात्री॒ रात्रि॑ र॒पो भव॑ति॒ भव॑ त्य॒पो रात्रिः॑ । \newline
9. अ॒पो रात्री॒ रात्रि॑ र॒पो॑ ऽपो रात्रिः॒ प्र प्र रात्रि॑ र॒पो॑ ऽपो रात्रिः॒ प्र । \newline
10. रात्रिः॒ प्र प्र रात्री॒ रात्रिः॒ प्र वि॑शति विशति॒ प्र रात्री॒ रात्रिः॒ प्र वि॑शति । \newline
11. प्र वि॑शति विशति॒ प्र प्र वि॑शति॒ तस्मा॒त् तस्मा᳚द् विशति॒ प्र प्र वि॑शति॒ तस्मा᳚त् । \newline
12. वि॒श॒ति॒ तस्मा॒त् तस्मा᳚द् विशति विशति॒ तस्मा᳚त् ता॒म्रा स्ता॒म्रा स्तस्मा᳚द् विशति विशति॒ तस्मा᳚त् ता॒म्राः । \newline
13. तस्मा᳚त् ता॒म्रा स्ता॒म्रा स्तस्मा॒त् तस्मा᳚त् ता॒म्रा आप॒ आप॑ स्ता॒म्रा स्तस्मा॒त् तस्मा᳚त् ता॒म्रा आपः॑ । \newline
14. ता॒म्रा आप॒ आप॑ स्ता॒म्रा स्ता॒म्रा आपो॒ दिवा॒ दिवा ऽऽप॑ स्ता॒म्रा स्ता॒म्रा आपो॒ दिवा᳚ । \newline
15. आपो॒ दिवा॒ दिवा ऽऽप॒ आपो॒ दिवा॑ ददृश्रे ददृश्रे॒ दिवा ऽऽप॒ आपो॒ दिवा॑ ददृश्रे । \newline
16. दिवा॑ ददृश्रे ददृश्रे॒ दिवा॒ दिवा॑ ददृश्रे॒ यद् यद् द॑दृश्रे॒ दिवा॒ दिवा॑ ददृश्रे॒ यत् । \newline
17. द॒दृ॒श्रे॒ यद् यद् द॑दृश्रे ददृश्रे॒ यन् नक्त॒म् नक्तं॒ ॅयद् द॑दृश्रे ददृश्रे॒ यन् नक्त᳚म् । \newline
18. यन् नक्त॒म् नक्तं॒ ॅयद् यन् नक्त॒म् भव॑ति॒ भव॑ति॒ नक्तं॒ ॅयद् यन् नक्त॒म् भव॑ति । \newline
19. नक्त॒म् भव॑ति॒ भव॑ति॒ नक्त॒न् नक्त॒म् भव॑ त्य॒पो॑ ऽपो भव॑ति॒ नक्त॒म् नक्त॒म् भव॑ त्य॒पः । \newline
20. भव॑ त्य॒पो॑ ऽपो भव॑ति॒ भव॑ त्य॒पो ऽह॒ रह॑ र॒पो भव॑ति॒ भव॑ त्य॒पो ऽहः॑ । \newline
21. अ॒पो ऽह॒ रह॑ र॒पो॑ ऽपो ऽहः॒ प्र प्राह॑ र॒पो॑ ऽपो ऽहः॒ प्र । \newline
22. अहः॒ प्र प्राह॒ रहः॒ प्र वि॑शति विशति॒ प्राह॒ रहः॒ प्र वि॑शति । \newline
23. प्र वि॑शति विशति॒ प्र प्र वि॑शति॒ तस्मा॒त् तस्मा᳚द् विशति॒ प्र प्र वि॑शति॒ तस्मा᳚त् । \newline
24. वि॒श॒ति॒ तस्मा॒त् तस्मा᳚द् विशति विशति॒ तस्मा᳚च् च॒न्द्रा श्च॒न्द्रा स्तस्मा᳚द् विशति विशति॒ तस्मा᳚च् च॒न्द्राः । \newline
25. तस्मा᳚च् च॒न्द्रा श्च॒न्द्रा स्तस्मा॒त् तस्मा᳚च् च॒न्द्रा आप॒ आप॑ श्च॒न्द्रा स्तस्मा॒त् तस्मा᳚च् च॒न्द्रा आपः॑ । \newline
26. च॒न्द्रा आप॒ आप॑ श्च॒न्द्रा श्च॒न्द्रा आपो॒ नक्त॒म् नक्त॒ माप॑ श्च॒न्द्रा श्च॒न्द्रा आपो॒ नक्त᳚म् । \newline
27. आपो॒ नक्त॒म् नक्त॒ माप॒ आपो॒ नक्त॑म् ददृश्रे ददृश्रे॒ नक्त॒ माप॒ आपो॒ नक्त॑म् ददृश्रे । \newline
28. नक्त॑म् ददृश्रे ददृश्रे॒ नक्त॒न् नक्त॑म् ददृश्रे छा॒यायै॑ छा॒यायै॑ ददृश्रे॒ नक्त॒म् नक्त॑म् ददृश्रे छा॒यायै᳚ । \newline
29. द॒दृ॒श्रे॒ छा॒यायै॑ छा॒यायै॑ ददृश्रे ददृश्रे छा॒यायै॑ च च छा॒यायै॑ ददृश्रे ददृश्रे छा॒यायै॑ च । \newline
30. छा॒यायै॑ च च छा॒यायै॑ छा॒यायै॑ चा॒तप॑त आ॒तप॑तश्च छा॒यायै॑ छा॒यायै॑ चा॒तप॑तः । \newline
31. चा॒तप॑त आ॒तप॑तश्च चा॒तप॑तश्च चा॒तप॑तश्च चा॒तप॑तश्च । \newline
32. आ॒तप॑तश्च चा॒तप॑त आ॒तप॑तश्च स॒न्धौ स॒न्धौ चा॒तप॑त आ॒तप॑तश्च स॒न्धौ । \newline
33. आ॒तप॑त॒ इत्या᳚ - तप॑तः । \newline
34. च॒ स॒न्धौ स॒न्धौ च॑ च स॒न्धौ गृ॑ह्णाति गृह्णाति स॒न्धौ च॑ च स॒न्धौ गृ॑ह्णाति । \newline
35. स॒न्धौ गृ॑ह्णाति गृह्णाति स॒न्धौ स॒न्धौ गृ॑ह्णा त्यहोरा॒त्रयो॑ रहोरा॒त्रयो᳚र् गृह्णाति स॒न्धौ स॒न्धौ गृ॑ह्णा त्यहोरा॒त्रयोः᳚ । \newline
36. स॒न्धाविति॑ सं - धौ । \newline
37. गृ॒ह्णा॒ त्य॒हो॒रा॒त्रयो॑ रहोरा॒त्रयो᳚र् गृह्णाति गृह्णा त्यहोरा॒त्रयो॑ रे॒वै वाहो॑रा॒त्रयो᳚र् गृह्णाति गृह्णा त्यहोरा॒त्रयो॑ रे॒व । \newline
38. अ॒हो॒रा॒त्रयो॑ रे॒वै वाहो॑रा॒त्रयो॑ रहोरा॒त्रयो॑ रे॒वास्मा॑ अस्मा ए॒वाहो॑रा॒त्रयो॑ रहोरा॒त्रयो॑ रे॒वास्मै᳚ । \newline
39. अ॒हो॒रा॒त्रयो॒रित्य॑हः - रा॒त्रयोः᳚ । \newline
40. ए॒वास्मा॑ अस्मा ए॒वै वास्मै॒ वर्णं॒ ॅवर्ण॑ मस्मा ए॒वै वास्मै॒ वर्ण᳚म् । \newline
41. अ॒स्मै॒ वर्णं॒ ॅवर्ण॑ मस्मा अस्मै॒ वर्ण॑म् गृह्णाति गृह्णाति॒ वर्ण॑ मस्मा अस्मै॒ वर्ण॑म् गृह्णाति । \newline
42. वर्ण॑म् गृह्णाति गृह्णाति॒ वर्णं॒ ॅवर्ण॑म् गृह्णाति ह॒विष्म॑तीर्. ह॒विष्म॑तीर् गृह्णाति॒ वर्णं॒ ॅवर्ण॑म् गृह्णाति ह॒विष्म॑तीः । \newline
43. गृ॒ह्णा॒ति॒ ह॒विष्म॑तीर्. ह॒विष्म॑तीर् गृह्णाति गृह्णाति ह॒विष्म॑ती रि॒मा इ॒मा ह॒विष्म॑तीर् गृह्णाति गृह्णाति ह॒विष्म॑ती रि॒माः । \newline
44. ह॒विष्म॑ती रि॒मा इ॒मा ह॒विष्म॑तीर्. ह॒विष्म॑ती रि॒मा आप॒ आप॑ इ॒मा ह॒विष्म॑तीर्. ह॒विष्म॑ती रि॒मा आपः॑ । \newline
45. इ॒मा आप॒ आप॑ इ॒मा इ॒मा आप॒ इती त्याप॑ इ॒मा इ॒मा आप॒ इति॑ । \newline
46. आप॒ इती त्याप॒ आप॒ इत्या॑हा॒हे त्याप॒ आप॒ इत्या॑ह । \newline
47. इत्या॑हा॒हे तीत्या॑ह ह॒विष्कृ॑तानाꣳ ह॒विष्कृ॑ताना मा॒हे तीत्या॑ह ह॒विष्कृ॑तानाम् । \newline
48. आ॒ह॒ ह॒विष्कृ॑तानाꣳ ह॒विष्कृ॑ताना माहाह ह॒विष्कृ॑ताना मे॒वैव ह॒विष्कृ॑ताना माहाह ह॒विष्कृ॑ताना मे॒व । \newline
49. ह॒विष्कृ॑ताना मे॒वैव ह॒विष्कृ॑तानाꣳ ह॒विष्कृ॑ताना मे॒व गृ॑ह्णाति गृह्णा त्ये॒व ह॒विष्कृ॑तानाꣳ ह॒विष्कृ॑ताना मे॒व गृ॑ह्णाति । \newline
50. ह॒विष्कृ॑ताना॒मिति॑ ह॒विः - कृ॒ता॒ना॒म् । \newline
51. ए॒व गृ॑ह्णाति गृह्णा त्ये॒वैव गृ॑ह्णाति ह॒विष्मान्॑. ह॒विष्मा᳚न् गृह्णा त्ये॒वैव गृ॑ह्णाति ह॒विष्मान्॑ । \newline
52. गृ॒ह्णा॒ति॒ ह॒विष्मान्॑. ह॒विष्मा᳚न् गृह्णाति गृह्णाति ह॒विष्माꣳ॑ अस्त्वस्तु ह॒विष्मा᳚न् गृह्णाति गृह्णाति ह॒विष्माꣳ॑ अस्तु । \newline
53. ह॒विष्माꣳ॑ अस्त्वस्तु ह॒विष्मान्॑. ह॒विष्माꣳ॑ अस्तु॒ सूर्यः॒ सूर्यो॑ अस्तु ह॒विष्मान्॑. ह॒विष्माꣳ॑ अस्तु॒ सूर्यः॑ । \newline
54. अ॒स्तु॒ सूर्यः॒ सूर्यो॑ अस्त्वस्तु॒ सूर्य॒ इतीति॒ सूर्यो॑ अस्त्वस्तु॒ सूर्य॒ इति॑ । \newline
\pagebreak
\markright{ TS 6.4.2.5  \hfill https://www.vedavms.in \hfill}

\section{ TS 6.4.2.5 }

\textbf{TS 6.4.2.5 } \newline
\textbf{Samhita Paata} \newline

सूर्य॒ इत्या॑ह॒ सशु॑क्राणामे॒व गृ॑ह्णात्यनु॒ष्टुभा॑ गृह्णाति॒ वाग्वा अ॑नु॒ष्टुग् वा॒चैवैनाः॒ सर्व॑या गृह्णाति॒ चतु॑ष्पदय॒र्चा गृ॑ह्णाति॒ त्रिः सा॑दयति स॒प्त सं प॑द्यन्ते स॒प्तप॑दा॒ शक्व॑री प॒शवः॒ शक्व॑री प॒शूने॒वाव॑ रुन्धे॒ ऽस्मै वै लो॒काय॒ गार्.ह॑पत्य॒ आ धी॑यते॒ऽमुष्मा॑ आहव॒नीयो॒ यद् गार्.ह॑पत्य उपसा॒दये॑द॒स्मिन् ॅलो॒के प॑शु॒मान्थ् स्या॒द् यदा॑हव॒नीये॒ऽमुष्मि॑न्- [  ] \newline

\textbf{Pada Paata} \newline

सूर्यः॑ । इति॑ । आ॒ह॒ । सशु॑क्राणा॒मिति॒ स - शु॒क्रा॒णा॒म् । ए॒व । गृ॒ह्णा॒ति॒ । अ॒नु॒ष्टुभेत्य॑नु - स्तुभा᳚ । गृ॒ह्णा॒ति॒ । वाक् । वै । अ॒नु॒ष्टुगित्य॑नु - स्तुक् । वा॒चा । ए॒व । ए॒नाः॒ । सर्व॑या । गृ॒ह्णा॒ति॒ । चतु॑ष्पद॒येति॒ चतुः॑ - प॒द॒या॒ । ऋ॒चा । गृ॒ह्णा॒ति॒ । त्रिः । सा॒द॒य॒ति॒ । स॒प्त । समिति॑ । प॒द्य॒न्ते॒ । स॒प्तप॒देति॑ स॒प्त - प॒दा॒ । शक्व॑री । प॒शवः॑ । शक्व॑री । प॒शून् । ए॒व । अवेति॑ । रु॒न्धे॒ । अ॒स्मै । वै । लो॒काय॑ । गार्.ह॑पत्य॒ इति॒ गार्.ह॑ - प॒त्यः॒ । एति॑ । धी॒य॒ते॒ । अ॒मुष्मै᳚ । आ॒ह॒व॒नीय॒ इत्या᳚ - ह॒व॒नीयः॑ । यत् । गार्.ह॑पत्य॒ इति॒ गार्.ह॑ - प॒त्ये॒ । उ॒प॒सा॒दये॒दित्यु॑प - सा॒दये᳚त् । अ॒स्मिन्न् । लो॒के । प॒शु॒मानिति॑ पशु - मान् । स्या॒त् । यत् । आ॒ह॒व॒नीय॒ इत्या᳚ - ह॒व॒नीये᳚ । अ॒मुष्मिन्न्॑ ।  \newline


\textbf{Krama Paata} \newline

सूर्य॒ इति॑ । इत्या॑ह । आ॒ह॒ सशु॑क्राणाम् । सशु॑क्राणामे॒व । सशु॑क्राणा॒मिति॒ स - शु॒क्रा॒णा॒म् । ए॒व गृ॑ह्णाति । गृ॒ह्णा॒त्य॒नु॒ष्टुभा᳚ । अ॒नु॒ष्टुभा॑ गृह्णाति । अ॒नु॒ष्टुभेत्य॑नु - स्तुभा᳚ । गृ॒ह्णा॒ति॒ वाक् । वाग् वै । वा अ॑नु॒ष्टुक् । अ॒नु॒ष्टुग् वा॒चा । अ॒नु॒ष्टुगित्य॑नु - स्तुक् । वा॒चैव । ए॒वैनाः᳚ । ए॒नाः॒ सर्व॑या । सर्व॑या गृह्णाति । गृ॒ह्णा॒ति॒ चतु॑ष्पदया । चतु॑ष्पदय॒र्चा । चतु॑ष्पद॒येति॒ चतुः॑ - प॒द॒या॒ । ऋ॒चा गृ॑ह्णाति । गृ॒ह्णा॒ति॒ त्रिः । त्रिः सा॑दयति । सा॒द॒य॒ति॒ स॒प्त । स॒प्त सम् । सम् प॑द्यन्ते । प॒द्य॒न्ते॒ स॒प्तप॑दा । स॒प्तप॑दा॒ शक्व॑री । स॒प्तप॒देति॑ स॒प्त - प॒दा॒ । शक्व॑री प॒शवः॑ । प॒शवः॒ शक्व॑री । शक्व॑री प॒शून् । प॒शूने॒व । ए॒वाव॑ । अव॑ रुन्धे । रु॒न्धे॒ऽस्मै । अ॒स्मै वै । वै लो॒काय॑ । लो॒काय॒ गार्.ह॑पत्यः । गार्.ह॑पत्य॒ आ । गार्.ह॑पत्य॒ इति॒ गार्.ह॑ - प॒त्यः॒ । आ धी॑यते । धी॒य॒ते॒ऽमुष्मै᳚ । अ॒मुष्मा॑ आहव॒नीयः॑ । आ॒ह॒व॒नीयो॒ यत् । आ॒ह॒व॒नीय॒ इत्या᳚ - ह॒व॒नीयः॑ । यद् गार्.ह॑पत्ये । गार्.ह॑पत्य उपसा॒दये᳚त् । गार्.ह॑पत्य॒ इति॒ गार्.ह॑ - प॒त्ये॒ । उ॒प॒सा॒दये॑द॒स्मिन्न् । उ॒प॒सा॒दये॒दित्यु॑प - सा॒दये᳚त् । अ॒स्मिन् ॅलो॒के । लो॒के प॑शु॒मान् । प॒शु॒मान्थ् स्या᳚त् । प॒शु॒मानिति॑ पशु - मान् । स्या॒द् यत् । यदा॑हव॒नीये᳚ । आ॒ह॒व॒नीये॒ऽमुष्मिन्न्॑ । आ॒ह॒व॒नीय॒ इत्या᳚ - ह॒व॒नीये᳚ । अ॒मुष्मि॑न् ॅलो॒के \newline

\textbf{Jatai Paata} \newline

1. सूर्य॒ इतीति॒ सूर्यः॒ सूर्य॒ इति॑ । \newline
2. इत्या॑हा॒हे तीत्या॑ह । \newline
3. आ॒ह॒ सशु॑क्राणाꣳ॒॒ सशु॑क्राणा माहाह॒ सशु॑क्राणाम् । \newline
4. सशु॑क्राणा मे॒वैव सशु॑क्राणाꣳ॒॒ सशु॑क्राणा मे॒व । \newline
5. सशु॑क्राणा॒मिति॒ स - शु॒क्रा॒णा॒म् । \newline
6. ए॒व गृ॑ह्णाति गृह्णा त्ये॒वैव गृ॑ह्णाति । \newline
7. गृ॒ह्णा॒ त्य॒नु॒ष्टुभा॑ ऽनु॒ष्टुभा॑ गृह्णाति गृह्णा त्यनु॒ष्टुभा᳚ । \newline
8. अ॒नु॒ष्टुभा॑ गृह्णाति गृह्णा त्यनु॒ष्टुभा॑ ऽनु॒ष्टुभा॑ गृह्णाति । \newline
9. अ॒नु॒ष्टुभेत्य॑नु - स्तुभा᳚ । \newline
10. गृ॒ह्णा॒ति॒ वाग् वाग् गृ॑ह्णाति गृह्णाति॒ वाक् । \newline
11. वाग् वै वै वाग् वाग् वै । \newline
12. वा अ॑नु॒ष्टु ग॑नु॒ष्टुग् वै वा अ॑नु॒ष्टुक् । \newline
13. अ॒नु॒ष्टुग् वा॒चा वा॒चा ऽनु॒ष्टु ग॑नु॒ष्टुग् वा॒चा । \newline
14. अ॒नु॒ष्टुगित्य॑नु - स्तुक् । \newline
15. वा॒चैवैव वा॒चा वा॒चैव । \newline
16. ए॒वैना॑ एना ए॒वै वैनाः᳚ । \newline
17. ए॒नाः॒ सर्व॑या॒ सर्व॑यैना एनाः॒ सर्व॑या । \newline
18. सर्व॑या गृह्णाति गृह्णाति॒ सर्व॑या॒ सर्व॑या गृह्णाति । \newline
19. गृ॒ह्णा॒ति॒ चतु॑ष्पदया॒ चतु॑ष्पदया गृह्णाति गृह्णाति॒ चतु॑ष्पदया । \newline
20. चतु॑ष्पदय॒ र्‌च र्‌चा चतु॑ष्पदया॒ चतु॑ष्पदय॒ र्‌चा । \newline
21. चतु॑ष्पद॒येति॒ चतुः॑ - प॒द॒या॒ । \newline
22. ऋ॒चा गृ॑ह्णाति गृह्णा त्यृ॒च र्‌चा गृ॑ह्णाति । \newline
23. गृ॒ह्णा॒ति॒ त्रि स्त्रिर् गृ॑ह्णाति गृह्णाति॒ त्रिः । \newline
24. त्रिः सा॑दयति सादयति॒ त्रि स्त्रिः सा॑दयति । \newline
25. सा॒द॒य॒ति॒ स॒प्त स॒प्त सा॑दयति सादयति स॒प्त । \newline
26. स॒प्त सꣳ सꣳ स॒प्त स॒प्त सम् । \newline
27. सम् प॑द्यन्ते पद्यन्ते॒ सꣳ सम् प॑द्यन्ते । \newline
28. प॒द्य॒न्ते॒ स॒प्तप॑दा स॒प्तप॑दा पद्यन्ते पद्यन्ते स॒प्तप॑दा । \newline
29. स॒प्तप॑दा॒ शक्व॑री॒ शक्व॑री स॒प्तप॑दा स॒प्तप॑दा॒ शक्व॑री । \newline
30. स॒प्तप॒देति॑ स॒प्त - प॒दा॒ । \newline
31. शक्व॑री प॒शवः॑ प॒शवः॒ शक्व॑री॒ शक्व॑री प॒शवः॑ । \newline
32. प॒शवः॒ शक्व॑री॒ शक्व॑री प॒शवः॑ प॒शवः॒ शक्व॑री । \newline
33. शक्व॑री प॒शून् प॒शूञ् छक्व॑री॒ शक्व॑री प॒शून् । \newline
34. प॒शू ने॒वैव प॒शून् प॒शू ने॒व । \newline
35. ए॒वावा वै॒वै वाव॑ । \newline
36. अव॑ रुन्धे रु॒न्धे ऽवाव॑ रुन्धे । \newline
37. रु॒न्धे॒ ऽस्मा अ॒स्मै रु॑न्धे रुन्धे॒ ऽस्मै । \newline
38. अ॒स्मै वै वा अ॒स्मा अ॒स्मै वै । \newline
39. वै लो॒काय॑ लो॒काय॒ वै वै लो॒काय॑ । \newline
40. लो॒काय॒ गार्.ह॑पत्यो॒ गार्.ह॑पत्यो लो॒काय॑ लो॒काय॒ गार्.ह॑पत्यः । \newline
41. गार्.ह॑पत्य॒ आ गार्.ह॑पत्यो॒ गार्.ह॑पत्य॒ आ । \newline
42. गार्.ह॑पत्य॒ इति॒ गार्.ह॑ - प॒त्यः॒ । \newline
43. आ धी॑यते धीयत॒ आ धी॑यते । \newline
44. धी॒य॒ते॒ ऽमुष्मा॑ अ॒मुष्मै॑ धीयते धीयते॒ ऽमुष्मै᳚ । \newline
45. अ॒मुष्मा॑ आहव॒नीय॑ आहव॒नीयो॒ ऽमुष्मा॑ अ॒मुष्मा॑ आहव॒नीयः॑ । \newline
46. आ॒ह॒व॒नीयो॒ यद् यदा॑हव॒नीय॑ आहव॒नीयो॒ यत् । \newline
47. आ॒ह॒व॒नीय॒ इत्या᳚ - ह॒व॒नीयः॑ । \newline
48. यद् गार्.ह॑पत्ये॒ गार्.ह॑पत्ये॒ यद् यद् गार्.ह॑पत्ये । \newline
49. गार्.ह॑पत्य उपसा॒दये॑ दुपसा॒दये॒द् गार्.ह॑पत्ये॒ गार्.ह॑पत्य उपसा॒दये᳚त् । \newline
50. गार्.ह॑पत्य॒ इति॒ गार्.ह॑ - प॒त्ये॒ । \newline
51. उ॒प॒सा॒दये॑ द॒स्मिन् न॒स्मिन् नु॑पसा॒दये॑ दुपसा॒दये॑ द॒स्मिन्न् । \newline
52. उ॒प॒सा॒दये॒दित्यु॑प - सा॒दये᳚त् । \newline
53. अ॒स्मिन् ॅलो॒के लो॒के᳚ ऽस्मिन् न॒स्मिन् ॅलो॒के । \newline
54. लो॒के प॑शु॒मान् प॑शु॒मान् ॅलो॒के लो॒के प॑शु॒मान् । \newline
55. प॒शु॒मान् थ्स्या᳚थ् स्यात् पशु॒मान् प॑शु॒मान् थ्स्या᳚त् । \newline
56. प॒शु॒मानिति॑ पशु - मान् । \newline
57. स्या॒द् यद् यथ् स्या᳚थ् स्या॒द् यत् । \newline
58. यदा॑हव॒नीय॑ आहव॒नीये॒ यद् यदा॑हव॒नीये᳚ । \newline
59. आ॒ह॒व॒नीये॒ ऽमुष्मि॑न् न॒मुष्मि॑न् नाहव॒नीय॑ आहव॒नीये॒ ऽमुष्मिन्न्॑ । \newline
60. आ॒ह॒व॒नीय॒ इत्या᳚ - ह॒व॒नीये᳚ । \newline
61. अ॒मुष्मि॑न् ॅलो॒के लो॒के॑ ऽमुष्मि॑न् न॒मुष्मि॑न् ॅलो॒के । \newline

\textbf{Ghana Paata } \newline

1. सूर्य॒ इतीति॒ सूर्यः॒ सूर्य॒ इत्या॑हा॒ हेति॒ सूर्यः॒ सूर्य॒ इत्या॑ह । \newline
2. इत्या॑हा॒हे तीत्या॑ह॒ सशु॑क्राणाꣳ॒॒ सशु॑क्राणा मा॒हे तीत्या॑ह॒ सशु॑क्राणाम् । \newline
3. आ॒ह॒ सशु॑क्राणाꣳ॒॒ सशु॑क्राणा माहाह॒ सशु॑क्राणा मे॒वैव सशु॑क्राणा माहाह॒ सशु॑क्राणा मे॒व । \newline
4. सशु॑क्राणा मे॒वैव सशु॑क्राणाꣳ॒॒ सशु॑क्राणा मे॒व गृ॑ह्णाति गृह्णा त्ये॒व सशु॑क्राणाꣳ॒॒ सशु॑क्राणा मे॒व गृ॑ह्णाति । \newline
5. सशु॑क्राणा॒मिति॒ स - शु॒क्रा॒णा॒म् । \newline
6. ए॒व गृ॑ह्णाति गृह्णा त्ये॒वैव गृ॑ह्णा त्यनु॒ष्टुभा॑ ऽनु॒ष्टुभा॑ गृह्णा त्ये॒वैव गृ॑ह्णा त्यनु॒ष्टुभा᳚ । \newline
7. गृ॒ह्णा॒ त्य॒नु॒ष्टुभा॑ ऽनु॒ष्टुभा॑ गृह्णाति गृह्णा त्यनु॒ष्टुभा॑ गृह्णाति गृह्णा त्यनु॒ष्टुभा॑ गृह्णाति गृह्णा त्यनु॒ष्टुभा॑ गृह्णाति । \newline
8. अ॒नु॒ष्टुभा॑ गृह्णाति गृह्णा त्यनु॒ष्टुभा॑ ऽनु॒ष्टुभा॑ गृह्णाति॒ वाग् वाग् गृ॑ह्णा त्यनु॒ष्टुभा॑ ऽनु॒ष्टुभा॑ गृह्णाति॒ वाक् । \newline
9. अ॒नु॒ष्टुभेत्य॑नु - स्तुभा᳚ । \newline
10. गृ॒ह्णा॒ति॒ वाग् वाग् गृ॑ह्णाति गृह्णाति॒ वाग् वै वै वाग् गृ॑ह्णाति गृह्णाति॒ वाग् वै । \newline
11. वाग् वै वै वाग् वाग् वा अ॑नु॒ष्टु ग॑नु॒ष्टुग् वै वाग् वाग् वा अ॑नु॒ष्टुक् । \newline
12. वा अ॑नु॒ष्टु ग॑नु॒ष्टुग् वै वा अ॑नु॒ष्टुग् वा॒चा वा॒चा ऽनु॒ष्टुग् वै वा अ॑नु॒ष्टुग् वा॒चा । \newline
13. अ॒नु॒ष्टुग् वा॒चा वा॒चा ऽनु॒ष्टु ग॑नु॒ष्टुग् वा॒चैवैव वा॒चा ऽनु॒ष्टु ग॑नु॒ष्टुग् वा॒चैव । \newline
14. अ॒नु॒ष्टुगित्य॑नु - स्तुक् । \newline
15. वा॒चैवैव वा॒चा वा॒चैवैना॑ एना ए॒व वा॒चा वा॒चैवैनाः᳚ । \newline
16. ए॒वैना॑ एना ए॒वै वैनाः॒ सर्व॑या॒ सर्व॑यैना ए॒वै वैनाः॒ सर्व॑या । \newline
17. ए॒नाः॒ सर्व॑या॒ सर्व॑यैना एनाः॒ सर्व॑या गृह्णाति गृह्णाति॒ सर्व॑यैना एनाः॒ सर्व॑या गृह्णाति । \newline
18. सर्व॑या गृह्णाति गृह्णाति॒ सर्व॑या॒ सर्व॑या गृह्णाति॒ चतु॑ष्पदया॒ चतु॑ष्पदया गृह्णाति॒ सर्व॑या॒ सर्व॑या गृह्णाति॒ चतु॑ष्पदया । \newline
19. गृ॒ह्णा॒ति॒ चतु॑ष्पदया॒ चतु॑ष्पदया गृह्णाति गृह्णाति॒ चतु॑ष्पदय॒ र्‌च र्‌चा चतु॑ष्पदया गृह्णाति गृह्णाति॒ चतु॑ष्पदय॒ र्‌चा । \newline
20. चतु॑ष्पदय॒ र्‌च र्‌चा चतु॑ष्पदया॒ चतु॑ष्पदय॒ र्‌चा गृ॑ह्णाति गृह्णा त्यृ॒चा चतु॑ष्पदया॒ चतु॑ष्पदय॒ र्‌चा गृ॑ह्णाति । \newline
21. चतु॑ष्पद॒येति॒ चतुः॑ - प॒द॒या॒ । \newline
22. ऋ॒चा गृ॑ह्णाति गृह्णा त्यृ॒च र्‌चा गृ॑ह्णाति॒ त्रि स्त्रिर् गृ॑ह्णा त्यृ॒च र्‌चा गृ॑ह्णाति॒ त्रिः । \newline
23. गृ॒ह्णा॒ति॒ त्रि स्त्रिर् गृ॑ह्णाति गृह्णाति॒ त्रिः सा॑दयति सादयति॒ त्रिर् गृ॑ह्णाति गृह्णाति॒ त्रिः सा॑दयति । \newline
24. त्रिः सा॑दयति सादयति॒ त्रि स्त्रिः सा॑दयति स॒प्त स॒प्त सा॑दयति॒ त्रि स्त्रिः सा॑दयति स॒प्त । \newline
25. सा॒द॒य॒ति॒ स॒प्त स॒प्त सा॑दयति सादयति स॒प्त सꣳ सꣳ स॒प्त सा॑दयति सादयति स॒प्त सम् । \newline
26. स॒प्त सꣳ सꣳ स॒प्त स॒प्त सम् प॑द्यन्ते पद्यन्ते॒ सꣳ स॒प्त स॒प्त सम् प॑द्यन्ते । \newline
27. सम् प॑द्यन्ते पद्यन्ते॒ सꣳ सम् प॑द्यन्ते स॒प्तप॑दा स॒प्तप॑दा पद्यन्ते॒ सꣳ सम् प॑द्यन्ते स॒प्तप॑दा । \newline
28. प॒द्य॒न्ते॒ स॒प्तप॑दा स॒प्तप॑दा पद्यन्ते पद्यन्ते स॒प्तप॑दा॒ शक्व॑री॒ शक्व॑री स॒प्तप॑दा पद्यन्ते पद्यन्ते स॒प्तप॑दा॒ शक्व॑री । \newline
29. स॒प्तप॑दा॒ शक्व॑री॒ शक्व॑री स॒प्तप॑दा स॒प्तप॑दा॒ शक्व॑री प॒शवः॑ प॒शवः॒ शक्व॑री स॒प्तप॑दा स॒प्तप॑दा॒ शक्व॑री प॒शवः॑ । \newline
30. स॒प्तप॒देति॑ स॒प्त - प॒दा॒ । \newline
31. शक्व॑री प॒शवः॑ प॒शवः॒ शक्व॑री॒ शक्व॑री प॒शवः॒ शक्व॑री॒ शक्व॑री प॒शवः॒ शक्व॑री॒ शक्व॑री प॒शवः॒ शक्व॑री । \newline
32. प॒शवः॒ शक्व॑री॒ शक्व॑री प॒शवः॑ प॒शवः॒ शक्व॑री प॒शून् प॒शूञ् छक्व॑री प॒शवः॑ प॒शवः॒ शक्व॑री प॒शून् । \newline
33. शक्व॑री प॒शून् प॒शूञ् छक्व॑री॒ शक्व॑री प॒शूने॒वैव प॒शूञ् छक्व॑री॒ शक्व॑री प॒शूने॒व । \newline
34. प॒शूने॒वैव प॒शून् प॒शूने॒ वावा वै॒व प॒शून् प॒शूने॒वाव॑ । \newline
35. ए॒वावा वै॒वै वाव॑ रुन्धे रु॒न्धे ऽवै॒वै वाव॑ रुन्धे । \newline
36. अव॑ रुन्धे रु॒न्धे ऽवाव॑ रुन्धे॒ ऽस्मा अ॒स्मै रु॒न्धे ऽवाव॑ रुन्धे॒ ऽस्मै । \newline
37. रु॒न्धे॒ ऽस्मा अ॒स्मै रु॑न्धे रुन्धे॒ ऽस्मै वै वा अ॒स्मै रु॑न्धे रुन्धे॒ ऽस्मै वै । \newline
38. अ॒स्मै वै वा अ॒स्मा अ॒स्मै वै लो॒काय॑ लो॒काय॒ वा अ॒स्मा अ॒स्मै वै लो॒काय॑ । \newline
39. वै लो॒काय॑ लो॒काय॒ वै वै लो॒काय॒ गार्.ह॑पत्यो॒ गार्.ह॑पत्यो लो॒काय॒ वै वै लो॒काय॒ गार्.ह॑पत्यः । \newline
40. लो॒काय॒ गार्.ह॑पत्यो॒ गार्.ह॑पत्यो लो॒काय॑ लो॒काय॒ गार्.ह॑पत्य॒ आ गार्.ह॑पत्यो लो॒काय॑ लो॒काय॒ गार्.ह॑पत्य॒ आ । \newline
41. गार्.ह॑पत्य॒ आ गार्.ह॑पत्यो॒ गार्.ह॑पत्य॒ आ धी॑यते धीयत॒ आ गार्.ह॑पत्यो॒ गार्.ह॑पत्य॒ आ धी॑यते । \newline
42. गार्.ह॑पत्य॒ इति॒ गार्.ह॑ - प॒त्यः॒ । \newline
43. आ धी॑यते धीयत॒ आ धी॑यते॒ ऽमुष्मा॑ अ॒मुष्मै॑ धीयत॒ आ धी॑यते॒ ऽमुष्मै᳚ । \newline
44. धी॒य॒ते॒ ऽमुष्मा॑ अ॒मुष्मै॑ धीयते धीयते॒ ऽमुष्मा॑ आहव॒नीय॑ आहव॒नीयो॒ ऽमुष्मै॑ धीयते धीयते॒ ऽमुष्मा॑ आहव॒नीयः॑ । \newline
45. अ॒मुष्मा॑ आहव॒नीय॑ आहव॒नीयो॒ ऽमुष्मा॑ अ॒मुष्मा॑ आहव॒नीयो॒ यद् यदा॑हव॒नीयो॒ ऽमुष्मा॑ अ॒मुष्मा॑ आहव॒नीयो॒ यत् । \newline
46. आ॒ह॒व॒नीयो॒ यद् यदा॑हव॒नीय॑ आहव॒नीयो॒ यद् गार्.ह॑पत्ये॒ गार्.ह॑पत्ये॒ यदा॑हव॒नीय॑ आहव॒नीयो॒ यद् गार्.ह॑पत्ये । \newline
47. आ॒ह॒व॒नीय॒ इत्या᳚ - ह॒व॒नीयः॑ । \newline
48. यद् गार्.ह॑पत्ये॒ गार्.ह॑पत्ये॒ यद् यद् गार्.ह॑पत्य उपसा॒दये॑ दुपसा॒दये॒द् गार्.ह॑पत्ये॒ यद् यद् गार्.ह॑पत्य उपसा॒दये᳚त् । \newline
49. गार्.ह॑पत्य उपसा॒दये॑ दुपसा॒दये॒द् गार्.ह॑पत्ये॒ गार्.ह॑पत्य उपसा॒दये॑ द॒स्मिन् न॒स्मिन् नु॑पसा॒दये॒द् गार्.ह॑पत्ये॒ गार्.ह॑पत्य उपसा॒दये॑ द॒स्मिन्न् । \newline
50. गार्.ह॑पत्य॒ इति॒ गार्.ह॑ - प॒त्ये॒ । \newline
51. उ॒प॒सा॒दये॑ द॒स्मिन् न॒स्मिन् नु॑पसा॒दये॑ दुपसा॒दये॑ द॒स्मिन् ॅलो॒के लो॒के᳚ ऽस्मिन् नु॑पसा॒दये॑ दुपसा॒दये॑ द॒स्मिन् ॅलो॒के । \newline
52. उ॒प॒सा॒दये॒दित्यु॑प - सा॒दये᳚त् । \newline
53. अ॒स्मिन् ॅलो॒के लो॒के᳚ ऽस्मिन् न॒स्मिन् ॅलो॒के प॑शु॒मान् प॑शु॒मान् ॅलो॒के᳚ ऽस्मिन् न॒स्मिन् ॅलो॒के प॑शु॒मान् । \newline
54. लो॒के प॑शु॒मान् प॑शु॒मान् ॅलो॒के लो॒के प॑शु॒मान् थ्स्या᳚थ् स्यात् पशु॒मान् ॅलो॒के लो॒के प॑शु॒मान् थ्स्या᳚त् । \newline
55. प॒शु॒मान् थ्स्या᳚थ् स्यात् पशु॒मान् प॑शु॒मान् थ्स्या॒द् यद् यथ् स्या᳚त् पशु॒मान् प॑शु॒मान् थ्स्या॒द् यत् । \newline
56. प॒शु॒मानिति॑ पशु - मान् । \newline
57. स्या॒द् यद् यथ् स्या᳚थ् स्या॒द् यदा॑हव॒नीय॑ आहव॒नीये॒ यथ् स्या᳚थ् स्या॒द् यदा॑हव॒नीये᳚ । \newline
58. यदा॑हव॒नीय॑ आहव॒नीये॒ यद् यदा॑हव॒नीये॒ ऽमुष्मि॑न् न॒मुष्मि॑न् नाहव॒नीये॒ यद् यदा॑हव॒नीये॒ ऽमुष्मिन्न्॑ । \newline
59. आ॒ह॒व॒नीये॒ ऽमुष्मि॑न् न॒मुष्मि॑न् नाहव॒नीय॑ आहव॒नीये॒ ऽमुष्मि॑न् ॅलो॒के लो॒के॑ ऽमुष्मि॑न् नाहव॒नीय॑ आहव॒नीये॒ ऽमुष्मि॑न् ॅलो॒के । \newline
60. आ॒ह॒व॒नीय॒ इत्या᳚ - ह॒व॒नीये᳚ । \newline
61. अ॒मुष्मि॑न् ॅलो॒के लो॒के॑ ऽमुष्मि॑न् न॒मुष्मि॑न् ॅलो॒के प॑शु॒मान् प॑शु॒मान् ॅलो॒के॑ ऽमुष्मि॑न् न॒मुष्मि॑न् ॅलो॒के प॑शु॒मान् । \newline
\pagebreak
\markright{ TS 6.4.2.6  \hfill https://www.vedavms.in \hfill}

\section{ TS 6.4.2.6 }

\textbf{TS 6.4.2.6 } \newline
\textbf{Samhita Paata} \newline

ॅलो॒के प॑शु॒मान्थ् स्या॑दु॒भयो॒रुप॑ सादयत्यु॒भयो॑रे॒वैनं॑ ॅलो॒कयोः᳚ पशु॒मन्तं॑ करोति स॒र्वतः॒ परि॑ हरति॒ रक्ष॑सा॒मप॑हत्या इन्द्राग्नि॒योर्भा॑ग॒धेयीः॒ स्थेत्या॑ह यथाय॒जुरे॒वैतदाग्नी᳚द्ध्र॒ उप॑ वासयत्ये॒तद्वै य॒ज्ञ्स्याप॑राजितं॒ ॅयदाग्नी᳚द्ध्रं॒ ॅयदे॒व य॒ज्ञ्स्याप॑राजितं॒ तदे॒वैना॒ उप॑ वासयति॒ यतः॒ खलु॒ वै य॒ज्ञ्स्य॒ वित॑तस्य॒ न क्रि॒यते॒ ( ) तदनु॑ य॒ज्ञ्ꣳ रक्षाꣳ॒॒स्यव॑ चरन्ति॒ यद् वह॑न्तीनां गृ॒ह्णाति॑ क्रि॒यमा॑णमे॒व तद् य॒ज्ञ्स्य॑ शये॒ रक्ष॑सा॒-मन॑न्ववचाराय॒ न ह्ये॑ता ई॒लय॒न्त्या तृ॑तीयसव॒नात् परि॑ शेरे य॒ज्ञ्स्य॒ संत॑त्यै ॥ \newline

\textbf{Pada Paata} \newline

लो॒के । प॒शु॒मानिति॑ पशु - मान् । स्या॒त् । उ॒भयोः᳚ । उपेति॑ । सा॒द॒य॒ति॒ । उ॒भयोः᳚ । ए॒व । ए॒न॒म् । लो॒कयोः᳚ । प॒शु॒मन्त॒मिति॑ पशु - मन्त᳚म् । क॒रो॒ति॒ । स॒र्वतः॑ । परीति॑ । ह॒र॒ति॒ । रक्ष॑साम् । अप॑हत्या॒ इत्यप॑ - ह॒त्यै॒ । इ॒न्द्रा॒ग्नि॒योरिती᳚न्द्र - अ॒ग्नि॒योः । भा॒ग॒धेयी॒रिति॑ भाग - धेयीः᳚ । स्थ॒ । इति॑ । आ॒ह॒ । य॒था॒य॒जुरिति॑ यथा-य॒जुः । ए॒व । ए॒तत् । आग्नी᳚द्ध्र॒ इत्याग्नि॑ - इ॒द्ध्रे॒ । उपेति॑ । वा॒स॒य॒ति॒ । ए॒तत् । वै । य॒ज्ञ्स्य॑ । अप॑राजित॒मित्यप॑रा-जि॒त॒म् । यत् । आग्नी᳚द्ध्र॒मित्याग्नि॑ - इ॒द्ध्र॒म् । यत् । ए॒व । य॒ज्ञ्स्य॑ । अप॑राजित॒मित्यप॑रा - जि॒त॒म् । तत् । ए॒व । ए॒नाः॒ । उपेति॑ । वा॒स॒य॒ति॒ । यतः॑ । खलु॑ । वै । य॒ज्ञ्स्य॑ । वित॑त॒स्येति॒ वि - त॒त॒स्य॒ । न । क्रि॒यते᳚ ( ) । तत् । अन्विति॑ । य॒ज्ञ्म् । रक्षाꣳ॑सि । अवेति॑ । च॒र॒न्ति॒ । यत् । वह॑न्तीनाम् । गृ॒ह्णाति॑ । क्रि॒यमा॑णम् । ए॒व । तत् । य॒ज्ञ्स्य॑ । श॒ये॒ । रक्ष॑साम् । अन॑न्ववचारा॒येत्यन॑नु - अ॒व॒चा॒रा॒य॒ । न । हि । ए॒ताः । ई॒लय॑न्ति । एति॑ । तृ॒ती॒य॒स॒व॒नादिति॑ तृतीय - स॒व॒नात् । परीति॑ । शे॒रे॒ । य॒ज्ञ्स्य॑ । सन्त॑त्या॒ इति॒ सं - त॒त्यै॒ ॥  \newline


\textbf{Krama Paata} \newline

लो॒के प॑शु॒मान् । प॒शु॒मान्थ् स्या᳚त् । प॒शु॒मानिति॑ पशु - मान् । स्या॒दु॒भयोः᳚ । उ॒भयो॒रुप॑ । उप॑ सादयति । सा॒द॒य॒त्यु॒भयोः᳚ । उ॒भयो॑रे॒व । ए॒वैन᳚म् । ए॒न॒म् ॅलो॒कयोः᳚ । लो॒कयोः᳚ पशु॒मन्त᳚म् । प॒शु॒मन्त॑म् करोति । प॒शु॒मन्त॒मिति॑ पशु - मन्त᳚म् । क॒रो॒ति॒ स॒र्वतः॑ । स॒र्वतः॒ परि॑ । परि॑ हरति । ह॒र॒ति॒ रक्ष॑साम् । रक्ष॑सा॒मप॑हत्यै । अप॑हत्या इन्द्राग्नि॒योः । अप॑हत्या॒ इत्यप॑ - ह॒त्यै॒ । इ॒न्द्रा॒ग्नि॒योर् भा॑ग॒धेयीः᳚ । इ॒न्द्रा॒ग्नि॒योरिती᳚न्द्र - अ॒ग्नि॒योः । भा॒ग॒धेयीः᳚ स्थ । भा॒ग॒धेयी॒रिति॑ भाग - धेयीः᳚ । स्थेति॑ । इत्या॑ह । आ॒ह॒ य॒था॒य॒जुः । य॒था॒य॒जुरे॒व । य॒था॒य॒जुरिति॑ यथा - य॒जुः । ए॒वैतत् । ए॒तदाग्नी᳚द्ध्रे । आग्नी᳚द्ध्र॒ उप॑ । आग्नी᳚द्ध्र॒ इत्याग्नि॑ - इ॒द्ध्रे॒ । उप॑ वासयति । वा॒स॒य॒त्ये॒तत् । ए॒तद् वै । वै य॒ज्ञ्स्य॑ । य॒ज्ञ्स्याप॑राजितम् । अप॑राजित॒म् ॅयत् । अप॑राजित॒मित्यप॑रा - जि॒त॒म् । यदाग्नी᳚द्ध्रम् । आग्नी᳚द्ध्र॒म् ॅयत् । आग्नी᳚द्ध्र॒मित्याग्नि॑ - इ॒द्ध्र॒म् । यदे॒व । ए॒व य॒ज्ञ्स्य॑ । य॒ज्ञ्स्याप॑राजितम् । अप॑राजित॒म् तत् । अप॑राजित॒मित्यप॑रा - जि॒त॒म् । तदे॒व । ए॒वैनाः᳚ । ए॒ना॒ उप॑ । उप॑ वासयति । वा॒स॒य॒ति॒ यतः॑ । यतः॒ खलु॑ । खलु॒ वै । वै य॒ज्ञ्स्य॑ । य॒ज्ञ्स्य॒ वित॑तस्य । वित॑तस्य॒ न । वित॑त॒स्येति॒ वि - त॒त॒स्य॒ । न क्रि॒यते᳚ ( ) । क्रि॒यते॒ तत् । तदनु॑ । अनु॑ य॒ज्ञ्म् । य॒ज्ञ्ꣳ रक्षाꣳ॑सि । रक्षाꣳ॒॒स्यव॑ । अव॑ चरन्ति । च॒र॒न्ति॒ यत् । यद् वह॑न्तीनाम् । वह॑न्तीनाम् गृ॒ह्णाति॑ । गृ॒ह्णाति॑ क्रि॒यमा॑णम् । क्रि॒यमा॑णमे॒व । ए॒व तत् । तद् य॒ज्ञ्स्य॑ । य॒ज्ञ्स्य॑ शये । श॒ये॒ रक्ष॑साम् । रक्ष॑सा॒मन॑न्ववचाराय । अन॑न्ववचाराय॒ न । अन॑न्ववचारा॒येत्यन॑नु - अ॒व॒चा॒रा॒य॒ । न हि । ह्ये॑ताः । ए॒ता ई॒लय॑न्ति । ई॒लय॒न्त्या । आ तृ॑तीयसव॒नात् । तृ॒ती॒य॒स॒व॒नात् परि॑ । तृ॒ती॒य॒स॒व॒नादिति॑ तृतीय - स॒व॒नात् । परि॑ शेरे । शे॒रे॒ य॒ज्ञ्स्य॑ । य॒ज्ञ्स्य॒ सन्त॑त्यै । सन्त॑त्या॒ इति॒ सम् - त॒त्यै॒ । \newline

\textbf{Jatai Paata} \newline

1. लो॒के प॑शु॒मान् प॑शु॒मान् ॅलो॒के लो॒के प॑शु॒मान् । \newline
2. प॒शु॒मान् थ्स्या᳚थ् स्यात् पशु॒मान् प॑शु॒मान् थ्स्या᳚त् । \newline
3. प॒शु॒मानिति॑ पशु - मान् । \newline
4. स्या॒ दु॒भयो॑ रु॒भयोः᳚ स्याथ् स्या दु॒भयोः᳚ । \newline
5. उ॒भयो॒ रुपो पो॒भयो॑ रु॒भयो॒ रुप॑ । \newline
6. उप॑ सादयति सादय॒ त्युपोप॑ सादयति । \newline
7. सा॒द॒य॒ त्यु॒भयो॑ रु॒भयोः᳚ सादयति सादय त्यु॒भयोः᳚ । \newline
8. उ॒भयो॑ रे॒वै वोभयो॑ रु॒भयो॑ रे॒व । \newline
9. ए॒वैन॑ मेन मे॒वै वैन᳚म् । \newline
10. ए॒न॒म् ॅलो॒कयो᳚र् लो॒कयो॑ रेन मेनम् ॅलो॒कयोः᳚ । \newline
11. लो॒कयोः᳚ पशु॒मन्त॑म् पशु॒मन्त॑म् ॅलो॒कयो᳚र् लो॒कयोः᳚ पशु॒मन्त᳚म् । \newline
12. प॒शु॒मन्त॑म् करोति करोति पशु॒मन्त॑म् पशु॒मन्त॑म् करोति । \newline
13. प॒शु॒मन्त॒मिति॑ पशु - मन्त᳚म् । \newline
14. क॒रो॒ति॒ स॒र्वतः॑ स॒र्वतः॑ करोति करोति स॒र्वतः॑ । \newline
15. स॒र्वतः॒ परि॒ परि॑ स॒र्वतः॑ स॒र्वतः॒ परि॑ । \newline
16. परि॑ हरति हरति॒ परि॒ परि॑ हरति । \newline
17. ह॒र॒ति॒ रक्ष॑साꣳ॒॒ रक्ष॑साꣳ हरति हरति॒ रक्ष॑साम् । \newline
18. रक्ष॑सा॒ मप॑हत्या॒ अप॑हत्यै॒ रक्ष॑साꣳ॒॒ रक्ष॑सा॒ मप॑हत्यै । \newline
19. अप॑हत्या इन्द्राग्नि॒यो रि॑न्द्राग्नि॒यो रप॑हत्या॒ अप॑हत्या इन्द्राग्नि॒योः । \newline
20. अप॑हत्या॒ इत्यप॑ - ह॒त्यै॒ । \newline
21. इ॒न्द्रा॒ग्नि॒योर् भा॑ग॒धेयी᳚र् भाग॒धेयी॑ रिन्द्राग्नि॒यो रि॑न्द्राग्नि॒योर् भा॑ग॒धेयीः᳚ । \newline
22. इ॒न्द्रा॒ग्नि॒योरिती᳚न्द्र - अ॒ग्नि॒योः । \newline
23. भा॒ग॒धेयीः᳚ स्थ स्थ भाग॒धेयी᳚र् भाग॒धेयीः᳚ स्थ । \newline
24. भा॒ग॒धेयी॒रिति॑ भाग - धेयीः᳚ । \newline
25. स्थेतीति॑ स्थ॒ स्थेति॑ । \newline
26. इत्या॑हा॒हे तीत्या॑ह । \newline
27. आ॒ह॒ य॒था॒य॒जुर् य॑थाय॒जु रा॑हाह यथाय॒जुः । \newline
28. य॒था॒य॒जु रे॒वैव य॑थाय॒जुर् य॑थाय॒जु रे॒व । \newline
29. य॒था॒य॒जुरिति॑ यथा - य॒जुः । \newline
30. ए॒वैत दे॒त दे॒वै वैतत् । \newline
31. ए॒त दाग्नी᳚द्ध्र॒ आग्नी᳚द्ध्र ए॒त दे॒त दाग्नी᳚द्ध्रे । \newline
32. आग्नी᳚द्ध्र॒ उपोपाग्नी᳚द्ध्र॒ आग्नी᳚द्ध्र॒ उप॑ । \newline
33. आग्नी᳚द्ध्र॒ इत्याग्नि॑ - इ॒द्ध्रे॒ । \newline
34. उप॑ वासयति वासय॒ त्युपोप॑ वासयति । \newline
35. वा॒स॒य॒ त्ये॒त दे॒तद् वा॑सयति वासय त्ये॒तत् । \newline
36. ए॒तद् वै वा ए॒त दे॒तद् वै । \newline
37. वै य॒ज्ञ्स्य॑ य॒ज्ञ्स्य॒ वै वै य॒ज्ञ्स्य॑ । \newline
38. य॒ज्ञ्स्या प॑राजित॒ मप॑राजितं ॅय॒ज्ञ्स्य॑ य॒ज्ञ्स्या प॑राजितम् । \newline
39. अप॑राजितं॒ ॅयद् यदप॑राजित॒ मप॑राजितं॒ ॅयत् । \newline
40. अप॑राजित॒मित्यप॑रा - जि॒त॒म् । \newline
41. यदाग्नी᳚द्ध्र॒ माग्नी᳚द्ध्रं॒ ॅयद् यदाग्नी᳚द्ध्रम् । \newline
42. आग्नी᳚द्ध्रं॒ ॅयद् यदाग्नी᳚द्ध्र॒ माग्नी᳚द्ध्रं॒ ॅयत् । \newline
43. आग्नी᳚द्ध्र॒मित्याग्नि॑ - इ॒द्ध्र॒म् । \newline
44. यदे॒वैव यद् यदे॒व । \newline
45. ए॒व य॒ज्ञ्स्य॑ य॒ज्ञ्स्यै॒वैव य॒ज्ञ्स्य॑ । \newline
46. य॒ज्ञ्स्या प॑राजित॒ मप॑राजितं ॅय॒ज्ञ्स्य॑ य॒ज्ञ्स्या प॑राजितम् । \newline
47. अप॑राजित॒म् तत् तदप॑राजित॒ मप॑राजित॒म् तत् । \newline
48. अप॑राजित॒मित्यप॑रा - जि॒त॒म् । \newline
49. तदे॒वैव तत् तदे॒व । \newline
50. ए॒वैना॑ एना ए॒वै वैनाः᳚ । \newline
51. ए॒ना॒ उपोपै॑ना एना॒ उप॑ । \newline
52. उप॑ वासयति वासय॒ त्युपोप॑ वासयति । \newline
53. वा॒स॒य॒ति॒ यतो॒ यतो॑ वासयति वासयति॒ यतः॑ । \newline
54. यतः॒ खलु॒ खलु॒ यतो॒ यतः॒ खलु॑ । \newline
55. खलु॒ वै वै खलु॒ खलु॒ वै । \newline
56. वै य॒ज्ञ्स्य॑ य॒ज्ञ्स्य॒ वै वै य॒ज्ञ्स्य॑ । \newline
57. य॒ज्ञ्स्य॒ वित॑तस्य॒ वित॑तस्य य॒ज्ञ्स्य॑ य॒ज्ञ्स्य॒ वित॑तस्य । \newline
58. वित॑तस्य॒ न न वित॑तस्य॒ वित॑तस्य॒ न । \newline
59. वित॑त॒स्येति॒ वि - त॒त॒स्य॒ । \newline
60. न क्रि॒यते᳚ क्रि॒यते॒ न न क्रि॒यते᳚ । \newline
61. क्रि॒यते॒ तत् तत् क्रि॒यते᳚ क्रि॒यते॒ तत् । \newline
62. तदन् वनु॒ तत् तदनु॑ । \newline
63. अनु॑ य॒ज्ञ्ं ॅय॒ज्ञ् मन् वनु॑ य॒ज्ञ्म् । \newline
64. य॒ज्ञ्ꣳ रक्षाꣳ॑सि॒ रक्षाꣳ॑सि य॒ज्ञ्ं ॅय॒ज्ञ्ꣳ रक्षाꣳ॑सि । \newline
65. रक्षाꣳ॒॒ स्यवाव॒ रक्षाꣳ॑सि॒ रक्षाꣳ॒॒ स्यव॑ । \newline
66. अव॑ चरन्ति चर॒न् त्यवाव॑ चरन्ति । \newline
67. च॒र॒न्ति॒ यद् यच् च॑रन्ति चरन्ति॒ यत् । \newline
68. यद् वह॑न्तीनां॒ ॅवह॑न्तीनां॒ ॅयद् यद् वह॑न्तीनाम् । \newline
69. वह॑न्तीनाम् गृ॒ह्णाति॑ गृ॒ह्णाति॒ वह॑न्तीनां॒ ॅवह॑न्तीनाम् गृ॒ह्णाति॑ । \newline
70. गृ॒ह्णाति॑ क्रि॒यमा॑णम् क्रि॒यमा॑णम् गृ॒ह्णाति॑ गृ॒ह्णाति॑ क्रि॒यमा॑णम् । \newline
71. क्रि॒यमा॑ण मे॒वैव क्रि॒यमा॑णम् क्रि॒यमा॑ण मे॒व । \newline
72. ए॒व तत् तदे॒वैव तत् । \newline
73. तद् य॒ज्ञ्स्य॑ य॒ज्ञ्स्य॒ तत् तद् य॒ज्ञ्स्य॑ । \newline
74. य॒ज्ञ्स्य॑ शये शये य॒ज्ञ्स्य॑ य॒ज्ञ्स्य॑ शये । \newline
75. श॒ये॒ रक्ष॑साꣳ॒॒ रक्ष॑साꣳ शये शये॒ रक्ष॑साम् । \newline
76. रक्ष॑सा॒ मन॑न्ववचारा॒या न॑न्ववचाराय॒ रक्ष॑साꣳ॒॒ रक्ष॑सा॒ मन॑न्ववचाराय । \newline
77. अन॑न्ववचाराय॒ न नान॑न्ववचारा॒या न॑न्ववचाराय॒ न । \newline
78. अन॑न्ववचारा॒येत्यन॑नु - अ॒व॒चा॒रा॒य॒ । \newline
79. न हि हि न न हि । \newline
80. ह्ये॑ता ए॒ता हि ह्ये॑ताः । \newline
81. ए॒ता ई॒लय॑न्ती॒ लय॑ न्त्ये॒ता ए॒ता ई॒लय॑न्ति । \newline
82. ई॒लय॒ न्त्येलय॑न्ती॒ लय॒न्त्या । \newline
83. आ तृ॑तीयसव॒नात् तृ॑तीयसव॒नादा तृ॑तीयसव॒नात् । \newline
84. तृ॒ती॒य॒स॒व॒नात् परि॒ परि॑ तृतीयसव॒नात् तृ॑तीयसव॒नात् परि॑ । \newline
85. तृ॒ती॒य॒स॒व॒नादिति॑ तृतीय - स॒व॒नात् । \newline
86. परि॑ शेरे शेरे॒ परि॒ परि॑ शेरे । \newline
87. शे॒रे॒ य॒ज्ञ्स्य॑ य॒ज्ञ्स्य॑ शेरे शेरे य॒ज्ञ्स्य॑ । \newline
88. य॒ज्ञ्स्य॒ सन्त॑त्यै॒ सन्त॑त्यै य॒ज्ञ्स्य॑ य॒ज्ञ्स्य॒ सन्त॑त्यै । \newline
89. सन्त॑त्या॒ इति॒ सं - त॒त्यै॒ । \newline

\textbf{Ghana Paata } \newline

1. लो॒के प॑शु॒मान् प॑शु॒मान् ॅलो॒के लो॒के प॑शु॒मान् थ्स्या᳚थ् स्यात् पशु॒मान् ॅलो॒के लो॒के प॑शु॒मान् थ्स्या᳚त् । \newline
2. प॒शु॒मान् थ्स्या᳚थ् स्यात् पशु॒मान् प॑शु॒मान् थ्स्या॑ दु॒भयो॑ रु॒भयोः᳚ स्यात् पशु॒मान् प॑शु॒मान् थ्स्या॑ दु॒भयोः᳚ । \newline
3. प॒शु॒मानिति॑ पशु - मान् । \newline
4. स्या॒ दु॒भयो॑ रु॒भयोः᳚ स्याथ् स्या दु॒भयो॒ रुपो पो॒भयोः᳚ स्याथ् स्या दु॒भयो॒ रुप॑ । \newline
5. उ॒भयो॒ रुपो पो॒भयो॑ रु॒भयो॒ रुप॑ सादयति सादय॒ त्युपो॒भयो॑ रु॒भयो॒ रुप॑ सादयति । \newline
6. उप॑ सादयति सादय॒ त्युपोप॑ सादय त्यु॒भयो॑ रु॒भयोः᳚ सादय॒ त्युपोप॑ सादय त्यु॒भयोः᳚ । \newline
7. सा॒द॒य॒ त्यु॒भयो॑ रु॒भयोः᳚ सादयति सादय त्यु॒भयो॑ रे॒वै वोभयोः᳚ सादयति सादय त्यु॒भयो॑ रे॒व । \newline
8. उ॒भयो॑ रे॒वै वोभयो॑ रु॒भयो॑ रे॒वैन॑ मेन मे॒वोभयो॑ रु॒भयो॑ रे॒वैन᳚म् । \newline
9. ए॒वैन॑ मेन मे॒वै वैन॑म् ॅलो॒कयो᳚र् लो॒कयो॑ रेन मे॒वै वैन॑म् ॅलो॒कयोः᳚ । \newline
10. ए॒न॒म् ॅलो॒कयो᳚र् लो॒कयो॑ रेन मेनम् ॅलो॒कयोः᳚ पशु॒मन्त॑म् पशु॒मन्त॑म् ॅलो॒कयो॑ रेन मेनम् ॅलो॒कयोः᳚ पशु॒मन्त᳚म् । \newline
11. लो॒कयोः᳚ पशु॒मन्त॑म् पशु॒मन्त॑म् ॅलो॒कयो᳚र् लो॒कयोः᳚ पशु॒मन्त॑म् करोति करोति पशु॒मन्त॑म् ॅलो॒कयो᳚र् लो॒कयोः᳚ पशु॒मन्त॑म् करोति । \newline
12. प॒शु॒मन्त॑म् करोति करोति पशु॒मन्त॑म् पशु॒मन्त॑म् करोति स॒र्वतः॑ स॒र्वतः॑ करोति पशु॒मन्त॑म् पशु॒मन्त॑म् करोति स॒र्वतः॑ । \newline
13. प॒शु॒मन्त॒मिति॑ पशु - मन्त᳚म् । \newline
14. क॒रो॒ति॒ स॒र्वतः॑ स॒र्वतः॑ करोति करोति स॒र्वतः॒ परि॒ परि॑ स॒र्वतः॑ करोति करोति स॒र्वतः॒ परि॑ । \newline
15. स॒र्वतः॒ परि॒ परि॑ स॒र्वतः॑ स॒र्वतः॒ परि॑ हरति हरति॒ परि॑ स॒र्वतः॑ स॒र्वतः॒ परि॑ हरति । \newline
16. परि॑ हरति हरति॒ परि॒ परि॑ हरति॒ रक्ष॑साꣳ॒॒ रक्ष॑साꣳ हरति॒ परि॒ परि॑ हरति॒ रक्ष॑साम् । \newline
17. ह॒र॒ति॒ रक्ष॑साꣳ॒॒ रक्ष॑साꣳ हरति हरति॒ रक्ष॑सा॒ मप॑हत्या॒ अप॑हत्यै॒ रक्ष॑साꣳ हरति हरति॒ रक्ष॑सा॒ मप॑हत्यै । \newline
18. रक्ष॑सा॒ मप॑हत्या॒ अप॑हत्यै॒ रक्ष॑साꣳ॒॒ रक्ष॑सा॒ मप॑हत्या इन्द्राग्नि॒यो रि॑न्द्राग्नि॒यो रप॑हत्यै॒ रक्ष॑साꣳ॒॒ रक्ष॑सा॒ मप॑हत्या इन्द्राग्नि॒योः । \newline
19. अप॑हत्या इन्द्राग्नि॒यो रि॑न्द्राग्नि॒यो रप॑हत्या॒ अप॑हत्या इन्द्राग्नि॒योर् भा॑ग॒धेयी᳚र् भाग॒धेयी॑ रिन्द्राग्नि॒यो रप॑हत्या॒ अप॑हत्या इन्द्राग्नि॒योर् भा॑ग॒धेयीः᳚ । \newline
20. अप॑हत्या॒ इत्यप॑ - ह॒त्यै॒ । \newline
21. इ॒न्द्रा॒ग्नि॒योर् भा॑ग॒धेयी᳚र् भाग॒धेयी॑ रिन्द्राग्नि॒यो रि॑न्द्राग्नि॒योर् भा॑ग॒धेयीः᳚ स्थ स्थ भाग॒धेयी॑ रिन्द्राग्नि॒यो रि॑न्द्राग्नि॒योर् भा॑ग॒धेयीः᳚ स्थ । \newline
22. इ॒न्द्रा॒ग्नि॒योरिती᳚न्द्र - अ॒ग्नि॒योः । \newline
23. भा॒ग॒धेयीः᳚ स्थ स्थ भाग॒धेयी᳚र् भाग॒धेयीः॒ स्थेतीति॑ स्थ भाग॒धेयी᳚र् भाग॒धेयीः॒ स्थेति॑ । \newline
24. भा॒ग॒धेयी॒रिति॑ भाग - धेयीः᳚ । \newline
25. स्थेतीति॑ स्थ॒ स्थे त्या॑हा॒ हेति॑ स्थ॒ स्थे त्या॑ह । \newline
26. इत्या॑हा॒हे तीत्या॑ह यथाय॒जुर् य॑थाय॒जु रा॒हे तीत्या॑ह यथाय॒जुः । \newline
27. आ॒ह॒ य॒था॒य॒जुर् य॑थाय॒जु रा॑हाह यथाय॒जु रे॒वैव य॑थाय॒जु रा॑हाह यथाय॒जु रे॒व । \newline
28. य॒था॒य॒जु रे॒वैव य॑थाय॒जुर् य॑थाय॒जु रे॒वैत दे॒त दे॒व य॑थाय॒जुर् य॑थाय॒जु रे॒वैतत् । \newline
29. य॒था॒य॒जुरिति॑ यथा - य॒जुः । \newline
30. ए॒वैत दे॒त दे॒वैवैत दाग्नी᳚द्ध्र॒ आग्नी᳚द्ध्र ए॒त दे॒वैवैत दाग्नी᳚द्ध्रे । \newline
31. ए॒त दाग्नी᳚द्ध्र॒ आग्नी᳚द्ध्र ए॒त दे॒त दाग्नी᳚द्ध्र॒ उपोपाग्नी᳚द्ध्र ए॒त दे॒त दाग्नी᳚द्ध्र॒ उप॑ । \newline
32. आग्नी᳚द्ध्र॒ उपोपाग्नी᳚द्ध्र॒ आग्नी᳚द्ध्र॒ उप॑ वासयति वासय॒ त्युपाग्नी᳚द्ध्र॒ आग्नी᳚द्ध्र॒ उप॑ वासयति । \newline
33. आग्नी᳚द्ध्र॒ इत्याग्नि॑ - इ॒द्ध्रे॒ । \newline
34. उप॑ वासयति वासय॒ त्युपोप॑ वासय त्ये॒त दे॒तद् वा॑सय॒ त्युपोप॑ वासय त्ये॒तत् । \newline
35. वा॒स॒य॒ त्ये॒त दे॒तद् वा॑सयति वासय त्ये॒तद् वै वा ए॒तद् वा॑सयति वासय त्ये॒तद् वै । \newline
36. ए॒तद् वै वा ए॒त दे॒तद् वै य॒ज्ञ्स्य॑ य॒ज्ञ्स्य॒ वा ए॒त दे॒तद् वै य॒ज्ञ्स्य॑ । \newline
37. वै य॒ज्ञ्स्य॑ य॒ज्ञ्स्य॒ वै वै य॒ज्ञ्स्या प॑राजित॒ मप॑राजितं ॅय॒ज्ञ्स्य॒ वै वै य॒ज्ञ्स्या प॑राजितम् । \newline
38. य॒ज्ञ्स्या प॑राजित॒ मप॑राजितं ॅय॒ज्ञ्स्य॑ य॒ज्ञ्स्या प॑राजितं॒ ॅयद् यदप॑राजितं ॅय॒ज्ञ्स्य॑ य॒ज्ञ्स्या प॑राजितं॒ ॅयत् । \newline
39. अप॑राजितं॒ ॅयद् यदप॑राजित॒ मप॑राजितं॒ ॅयदाग्नी᳚द्ध्र॒ माग्नी᳚द्ध्रं॒ ॅयदप॑राजित॒ मप॑राजितं॒ ॅयदाग्नी᳚द्ध्रम् । \newline
40. अप॑राजित॒मित्यप॑रा - जि॒त॒म् । \newline
41. यदाग्नी᳚द्ध्र॒ माग्नी᳚द्ध्रं॒ ॅयद् यदाग्नी᳚द्ध्रं॒ ॅयद् यदाग्नी᳚द्ध्रं॒ ॅयद् यदाग्नी᳚द्ध्रं॒ ॅयत् । \newline
42. आग्नी᳚द्ध्रं॒ ॅयद् यदाग्नी᳚द्ध्र॒ माग्नी᳚द्ध्रं॒ ॅयदे॒वैव यदाग्नी᳚द्ध्र॒ माग्नी᳚द्ध्रं॒ ॅयदे॒व । \newline
43. आग्नी᳚द्ध्र॒मित्याग्नि॑ - इ॒द्ध्र॒म् । \newline
44. यदे॒वैव यद् यदे॒व य॒ज्ञ्स्य॑ य॒ज्ञ्स्यै॒व यद् यदे॒व य॒ज्ञ्स्य॑ । \newline
45. ए॒व य॒ज्ञ्स्य॑ य॒ज्ञ्स्यै॒वैव य॒ज्ञ्स्या प॑राजित॒ मप॑राजितं ॅय॒ज्ञ्स्यै॒वैव य॒ज्ञ्स्या प॑राजितम् । \newline
46. य॒ज्ञ्स्या प॑राजित॒ मप॑राजितं ॅय॒ज्ञ्स्य॑ य॒ज्ञ्स्या प॑राजित॒म् तत् तदप॑राजितं ॅय॒ज्ञ्स्य॑ य॒ज्ञ्स्या प॑राजित॒म् तत् । \newline
47. अप॑राजित॒म् तत् तदप॑राजित॒ मप॑राजित॒म् तदे॒वैव तदप॑राजित॒ मप॑राजित॒म् तदे॒व । \newline
48. अप॑राजित॒मित्यप॑रा - जि॒त॒म् । \newline
49. तदे॒ वैव तत् तदे॒ वैना॑ एना ए॒व तत् तदे॒ वैनाः᳚ । \newline
50. ए॒वैना॑ एना ए॒वै वैना॒ उपोपै॑ना ए॒वै वैना॒ उप॑ । \newline
51. ए॒ना॒ उपोपै॑ना एना॒ उप॑ वासयति वासय॒ त्युपै॑ना एना॒ उप॑ वासयति । \newline
52. उप॑ वासयति वासय॒ त्युपोप॑ वासयति॒ यतो॒ यतो॑ वासय॒ त्युपोप॑ वासयति॒ यतः॑ । \newline
53. वा॒स॒य॒ति॒ यतो॒ यतो॑ वासयति वासयति॒ यतः॒ खलु॒ खलु॒ यतो॑ वासयति वासयति॒ यतः॒ खलु॑ । \newline
54. यतः॒ खलु॒ खलु॒ यतो॒ यतः॒ खलु॒ वै वै खलु॒ यतो॒ यतः॒ खलु॒ वै । \newline
55. खलु॒ वै वै खलु॒ खलु॒ वै य॒ज्ञ्स्य॑ य॒ज्ञ्स्य॒ वै खलु॒ खलु॒ वै य॒ज्ञ्स्य॑ । \newline
56. वै य॒ज्ञ्स्य॑ य॒ज्ञ्स्य॒ वै वै य॒ज्ञ्स्य॒ वित॑तस्य॒ वित॑तस्य य॒ज्ञ्स्य॒ वै वै य॒ज्ञ्स्य॒ वित॑तस्य । \newline
57. य॒ज्ञ्स्य॒ वित॑तस्य॒ वित॑तस्य य॒ज्ञ्स्य॑ य॒ज्ञ्स्य॒ वित॑तस्य॒ न न वित॑तस्य य॒ज्ञ्स्य॑ य॒ज्ञ्स्य॒ वित॑तस्य॒ न । \newline
58. वित॑तस्य॒ न न वित॑तस्य॒ वित॑तस्य॒ न क्रि॒यते᳚ क्रि॒यते॒ न वित॑तस्य॒ वित॑तस्य॒ न क्रि॒यते᳚ । \newline
59. वित॑त॒स्येति॒ वि - त॒त॒स्य॒ । \newline
60. न क्रि॒यते᳚ क्रि॒यते॒ न न क्रि॒यते॒ तत् तत् क्रि॒यते॒ न न क्रि॒यते॒ तत् । \newline
61. क्रि॒यते॒ तत् तत् क्रि॒यते᳚ क्रि॒यते॒ तदन् वनु॒ तत् क्रि॒यते᳚ क्रि॒यते॒ तदनु॑ । \newline
62. तदन् वनु॒ तत् तदनु॑ य॒ज्ञ्ं ॅय॒ज्ञ् मनु॒ तत् तदनु॑ य॒ज्ञ्म् । \newline
63. अनु॑ य॒ज्ञ्ं ॅय॒ज्ञ् मन्वनु॑ य॒ज्ञ्ꣳ रक्षाꣳ॑सि॒ रक्षाꣳ॑सि य॒ज्ञ् मन्वनु॑ य॒ज्ञ्ꣳ रक्षाꣳ॑सि । \newline
64. य॒ज्ञ्ꣳ रक्षाꣳ॑सि॒ रक्षाꣳ॑सि य॒ज्ञ्ं ॅय॒ज्ञ्ꣳ रक्षाꣳ॒॒ स्यवाव॒ रक्षाꣳ॑सि य॒ज्ञ्ं ॅय॒ज्ञ्ꣳ रक्षाꣳ॒॒ स्यव॑ । \newline
65. रक्षाꣳ॒॒ स्यवाव॒ रक्षाꣳ॑सि॒ रक्षाꣳ॒॒ स्यव॑ चरन्ति चर॒न्त्यव॒ रक्षाꣳ॑सि॒ रक्षाꣳ॒॒ स्यव॑ चरन्ति । \newline
66. अव॑ चरन्ति चर॒ न्त्यवाव॑ चरन्ति॒ यद् यच् च॑र॒ न्त्यवाव॑ चरन्ति॒ यत् । \newline
67. च॒र॒न्ति॒ यद् यच् च॑रन्ति चरन्ति॒ यद् वह॑न्तीनां॒ ॅवह॑न्तीनां॒ ॅयच् च॑रन्ति चरन्ति॒ यद् वह॑न्तीनाम् । \newline
68. यद् वह॑न्तीनां॒ ॅवह॑न्तीनां॒ ॅयद् यद् वह॑न्तीनाम् गृ॒ह्णाति॑ गृ॒ह्णाति॒ वह॑न्तीनां॒ ॅयद् यद् वह॑न्तीनाम् गृ॒ह्णाति॑ । \newline
69. वह॑न्तीनाम् गृ॒ह्णाति॑ गृ॒ह्णाति॒ वह॑न्तीनां॒ ॅवह॑न्तीनाम् गृ॒ह्णाति॑ क्रि॒यमा॑णम् क्रि॒यमा॑णम् गृ॒ह्णाति॒ वह॑न्तीनां॒ ॅवह॑न्तीनाम् गृ॒ह्णाति॑ क्रि॒यमा॑णम् । \newline
70. गृ॒ह्णाति॑ क्रि॒यमा॑णम् क्रि॒यमा॑णम् गृ॒ह्णाति॑ गृ॒ह्णाति॑ क्रि॒यमा॑ण मे॒वैव क्रि॒यमा॑णम् गृ॒ह्णाति॑ गृ॒ह्णाति॑ क्रि॒यमा॑ण मे॒व । \newline
71. क्रि॒यमा॑ण मे॒वैव क्रि॒यमा॑णम् क्रि॒यमा॑ण मे॒व तत् तदे॒व क्रि॒यमा॑णम् क्रि॒यमा॑ण मे॒व तत् । \newline
72. ए॒व तत् तदे॒वैव तद् य॒ज्ञ्स्य॑ य॒ज्ञ्स्य॒ तदे॒वैव तद् य॒ज्ञ्स्य॑ । \newline
73. तद् य॒ज्ञ्स्य॑ य॒ज्ञ्स्य॒ तत् तद् य॒ज्ञ्स्य॑ शये शये य॒ज्ञ्स्य॒ तत् तद् य॒ज्ञ्स्य॑ शये । \newline
74. य॒ज्ञ्स्य॑ शये शये य॒ज्ञ्स्य॑ य॒ज्ञ्स्य॑ शये॒ रक्ष॑साꣳ॒॒ रक्ष॑साꣳ शये य॒ज्ञ्स्य॑ य॒ज्ञ्स्य॑ शये॒ रक्ष॑साम् । \newline
75. श॒ये॒ रक्ष॑साꣳ॒॒ रक्ष॑साꣳ शये शये॒ रक्ष॑सा॒ मन॑न्ववचारा॒या न॑न्ववचाराय॒ रक्ष॑साꣳ शये शये॒ रक्ष॑सा॒ मन॑न्ववचाराय । \newline
76. रक्ष॑सा॒ मन॑न्ववचारा॒या न॑न्ववचाराय॒ रक्ष॑साꣳ॒॒ रक्ष॑सा॒ मन॑न्ववचाराय॒ न नान॑न्ववचाराय॒ रक्ष॑साꣳ॒॒ रक्ष॑सा॒ मन॑न्ववचाराय॒ न । \newline
77. अन॑न्ववचाराय॒ न नान॑न्ववचारा॒या न॑न्ववचाराय॒ न हि हि नान॑न्ववचारा॒या न॑न्ववचाराय॒ न हि । \newline
78. अन॑न्ववचारा॒येत्यन॑नु - अ॒व॒चा॒रा॒य॒ । \newline
79. न हि हि न न ह्ये॑ता ए॒ता हि न न ह्ये॑ताः । \newline
80. ह्ये॑ता ए॒ता हि ह्ये॑ता ई॒लय॑ न्ती॒लय॑ न्त्ये॒ता हि ह्ये॑ता ई॒लय॑न्ति । \newline
81. ए॒ता ई॒लय॑ न्ती॒लय॑ न्त्ये॒ता ए॒ता ई॒लय॒ न्त्येलय॑ न्त्ये॒ता ए॒ता ई॒लय॒न्त्या । \newline
82. ई॒लय॒ न्त्येलय॑ न्ती॒लय॒न्त्या तृ॑तीयसव॒नात् तृ॑तीयसव॒ना देलय॑न्ती॒ लय॒न्त्या तृ॑तीयसव॒नात् । \newline
83. आ तृ॑तीयसव॒नात् तृ॑तीयसव॒नादा तृ॑तीयसव॒नात् परि॒ परि॑ तृतीयसव॒नादा तृ॑तीयसव॒नात् परि॑ । \newline
84. तृ॒ती॒य॒स॒व॒नात् परि॒ परि॑ तृतीयसव॒नात् तृ॑तीयसव॒नात् परि॑ शेरे शेरे॒ परि॑ तृतीयसव॒नात् तृ॑तीयसव॒नात् परि॑ शेरे । \newline
85. तृ॒ती॒य॒स॒व॒नादिति॑ तृतीय - स॒व॒नात् । \newline
86. परि॑ शेरे शेरे॒ परि॒ परि॑ शेरे य॒ज्ञ्स्य॑ य॒ज्ञ्स्य॑ शेरे॒ परि॒ परि॑ शेरे य॒ज्ञ्स्य॑ । \newline
87. शे॒रे॒ य॒ज्ञ्स्य॑ य॒ज्ञ्स्य॑ शेरे शेरे य॒ज्ञ्स्य॒ सन्त॑त्यै॒ सन्त॑त्यै य॒ज्ञ्स्य॑ शेरे शेरे य॒ज्ञ्स्य॒ सन्त॑त्यै । \newline
88. य॒ज्ञ्स्य॒ सन्त॑त्यै॒ सन्त॑त्यै य॒ज्ञ्स्य॑ य॒ज्ञ्स्य॒ सन्त॑त्यै । \newline
89. सन्त॑त्या॒ इति॒ सं - त॒त्यै॒ । \newline
\pagebreak
\markright{ TS 6.4.3.1  \hfill https://www.vedavms.in \hfill}

\section{ TS 6.4.3.1 }

\textbf{TS 6.4.3.1 } \newline
\textbf{Samhita Paata} \newline

ब्र॒ह्म॒वा॒दिनो॑ वदन्ति॒ स त्वा अ॑द्ध्व॒र्युः स्या॒द्यः सोम॑मुपाव॒हर॒न्थ् सर्वा᳚भ्यो दे॒वता᳚भ्य उपाव॒हरे॒दिति॑ हृ॒दे त्वेत्या॑ह मनु॒ष्ये᳚भ्य ए॒वैतेन॑ करोति॒ मन॑से॒ त्वेत्या॑ह पि॒तृभ्य॑ ए॒वैतेन॑ करोति दि॒वे त्वा॒ सूर्या॑य॒ त्वेत्या॑ह दे॒वेभ्य॑ ए॒वैतेन॑ करोत्ये॒ताव॑ती॒र्वै दे॒वता॒स्ताभ्य॑ ए॒वैनꣳ॒॒ सर्वा᳚भ्य उ॒पाव॑हरति पु॒रा वा॒चः- [  ] \newline

\textbf{Pada Paata} \newline

ब्र॒ह्म॒वा॒दिन॒ इति॑ ब्रह्म-वा॒दिनः॑ । व॒द॒न्ति॒ । सः । तु । वै । अ॒द्ध्व॒र्युः । स्या॒त् । यः । सोम᳚म् । उ॒पा॒व॒हर॒न्नित्यु॑प - अ॒व॒हरन्न्॑ । सर्वा᳚भ्यः । दे॒वता᳚भ्यः । उ॒पा॒व॒हरे॒दित्यु॑प - अ॒व॒हरे᳚त् । इति॑ । हृ॒दे । त्वा॒ । इति॑ । आ॒ह॒ । म॒नु॒ष्ये᳚भ्यः । ए॒व । ए॒तेन॑ । क॒रो॒ति॒ । मन॑से । त्वा॒ । इति॑ । आ॒ह॒ । पि॒तृभ्य॒ इति॑ पि॒तृ - भ्यः॒ । ए॒व । ए॒तेन॑ । क॒रो॒ति॒ । दि॒वे । त्वा॒ । सूर्या॑य । त्वा॒ । इति॑ । आ॒ह॒ । दे॒वेभ्यः॑ । ए॒व । ए॒तेन॑ । क॒रो॒ति॒ । ए॒ताव॑तीः । वै । दे॒वताः᳚ । ताभ्यः॑ । ए॒व । ए॒न॒म् । सर्वा᳚भ्यः । उ॒पाव॑हर॒तीत्यु॑प - अव॑हरति । पु॒रा । वा॒चः ।  \newline


\textbf{Krama Paata} \newline

ब्र॒ह्म॒वा॒दिनो॑ वदन्ति । ब्र॒ह्म॒वा॒दिन॒ इति॑ ब्रह्म - वा॒दिनः॑ । व॒द॒न्ति॒ सः । स तु । त्वै । वा अ॑द्ध्व॒र्युः । अ॒द्ध्व॒र्युः स्या᳚त् । स्या॒द् यः । यः सोम᳚म् । सोम॑मुपाव॒हरन्न्॑ । उ॒पा॒व॒हर॒न्थ् सर्वा᳚भ्यः । उ॒पा॒व॒हर॒नित्यु॑प - अ॒व॒हरन्न्॑ । सर्वा᳚भ्यो दे॒वता᳚भ्यः । दे॒वता᳚भ्य उपाव॒हरे᳚त् । उ॒पा॒व॒हरे॒दिति॑ । उ॒पा॒व॒हरे॒दित्यु॑प - अ॒व॒हरे᳚त् । इति॑ हृ॒दे । हृ॒दे त्वा᳚ । त्वेति॑ । इत्या॑ह । आ॒ह॒ म॒नु॒ष्ये᳚भ्यः । म॒नु॒ष्ये᳚भ्य ए॒व । ए॒वैतेन॑ । ए॒तेन॑ करोति । क॒रो॒ति॒ मन॑से । मन॑से त्वा । त्वेति॑ । इत्या॑ह । आ॒ह॒ पि॒तृभ्यः॑ । पि॒तृभ्य॑ ए॒व । पि॒तृभ्य॒ इति॑ पि॒तृ - भ्यः॒ । ए॒वैतेन॑ । ए॒तेन॑ करोति । क॒रो॒ति॒ दि॒वे । दि॒वे त्वा᳚ । त्वा॒ सूर्या॑य । सूर्या॑य त्वा । त्वेति॑ । इत्या॑ह । आ॒ह॒ दे॒वेभ्यः॑ । दे॒वेभ्य॑ ए॒व । ए॒वैतेन॑ । ए॒तेन॑ करोति । क॒रो॒त्ये॒ताव॑तीः । ए॒ताव॑ती॒र् वै । वै दे॒वताः᳚ । दे॒वता॒स्ताभ्यः॑ । ताभ्य॑ ए॒व । ए॒वैन᳚म् । ए॒नꣳ॒॒ सर्वा᳚भ्यः । सर्वा᳚भ्य उ॒पाव॑हरति । उ॒पाव॑हरति पु॒रा । उ॒पाव॑हर॒तीत्यु॑प - अव॑हरति । पु॒रा वा॒चः । वा॒चः प्रव॑दितोः \newline

\textbf{Jatai Paata} \newline

1. ब्र॒ह्म॒वा॒दिनो॑ वदन्ति वदन्ति ब्रह्मवा॒दिनो᳚ ब्रह्मवा॒दिनो॑ वदन्ति । \newline
2. ब्र॒ह्म॒वा॒दिन॒ इति॑ ब्रह्म - वा॒दिनः॑ । \newline
3. व॒द॒न्ति॒ स स व॑दन्ति वदन्ति॒ सः । \newline
4. स तु तु स स तु । \newline
5. त्वै वै तु त्वै । \newline
6. वा अ॑द्ध्व॒र्यु र॑द्ध्व॒र्युर् वै वा अ॑द्ध्व॒र्युः । \newline
7. अ॒द्ध्व॒र्युः स्या᳚थ् स्या दद्ध्व॒र्यु र॑द्ध्व॒र्युः स्या᳚त् । \newline
8. स्या॒द् यो यः स्या᳚थ् स्या॒द् यः । \newline
9. यः सोमꣳ॒॒ सोमं॒ ॅयो यः सोम᳚म् । \newline
10. सोम॑ मुपाव॒हर॑न् नुपाव॒हर॒न् थ्सोमꣳ॒॒ सोम॑ मुपाव॒हरन्न्॑ । \newline
11. उ॒पा॒व॒हर॒न् थ्सर्वा᳚भ्यः॒ सर्वा᳚भ्य उपाव॒हर॑न् नुपाव॒हर॒न् थ्सर्वा᳚भ्यः । \newline
12. उ॒पा॒व॒हर॒न्नित्यु॑प - अ॒व॒हरन्न्॑ । \newline
13. सर्वा᳚भ्यो दे॒वता᳚भ्यो दे॒वता᳚भ्यः॒ सर्वा᳚भ्यः॒ सर्वा᳚भ्यो दे॒वता᳚भ्यः । \newline
14. दे॒वता᳚भ्य उपाव॒हरे॑ दुपाव॒हरे᳚द् दे॒वता᳚भ्यो दे॒वता᳚भ्य उपाव॒हरे᳚त् । \newline
15. उ॒पा॒व॒हरे॒ दिती त्यु॑पाव॒हरे॑ दुपाव॒हरे॒ दिति॑ । \newline
16. उ॒पा॒व॒हरे॒दित्यु॑प - अ॒व॒हरे᳚त् । \newline
17. इति॑ हृ॒दे हृ॒द इतीति॑ हृ॒दे । \newline
18. हृ॒दे त्वा᳚ त्वा हृ॒दे हृ॒दे त्वा᳚ । \newline
19. त्वेतीति॑ त्वा॒ त्वेति॑ । \newline
20. इत्या॑हा॒हे तीत्या॑ह । \newline
21. आ॒ह॒ म॒नु॒ष्ये᳚भ्यो मनु॒ष्ये᳚भ्य आहाह मनु॒ष्ये᳚भ्यः । \newline
22. म॒नु॒ष्ये᳚भ्य ए॒वैव म॑नु॒ष्ये᳚भ्यो मनु॒ष्ये᳚भ्य ए॒व । \newline
23. ए॒वैते नै॒ते नै॒वै वैतेन॑ । \newline
24. ए॒तेन॑ करोति करो त्ये॒ते नै॒तेन॑ करोति । \newline
25. क॒रो॒ति॒ मन॑से॒ मन॑से करोति करोति॒ मन॑से । \newline
26. मन॑से त्वा त्वा॒ मन॑से॒ मन॑से त्वा । \newline
27. त्वेतीति॑ त्वा॒ त्वेति॑ । \newline
28. इत्या॑हा॒हे तीत्या॑ह । \newline
29. आ॒ह॒ पि॒तृभ्यः॑ पि॒तृभ्य॑ आहाह पि॒तृभ्यः॑ । \newline
30. पि॒तृभ्य॑ ए॒वैव पि॒तृभ्यः॑ पि॒तृभ्य॑ ए॒व । \newline
31. पि॒तृभ्य॒ इति॑ पि॒तृ - भ्यः॒ । \newline
32. ए॒वैते नै॒ते नै॒वै वैतेन॑ । \newline
33. ए॒तेन॑ करोति करो त्ये॒तेनै॒ तेन॑ करोति । \newline
34. क॒रो॒ति॒ दि॒वे दि॒वे क॑रोति करोति दि॒वे । \newline
35. दि॒वे त्वा᳚ त्वा दि॒वे दि॒वे त्वा᳚ । \newline
36. त्वा॒ सूर्या॑य॒ सूर्या॑य त्वा त्वा॒ सूर्या॑य । \newline
37. सूर्या॑य त्वा त्वा॒ सूर्या॑य॒ सूर्या॑य त्वा । \newline
38. त्वेतीति॑ त्वा॒ त्वेति॑ । \newline
39. इत्या॑हा॒हे तीत्या॑ह । \newline
40. आ॒ह॒ दे॒वेभ्यो॑ दे॒वेभ्य॑ आहाह दे॒वेभ्यः॑ । \newline
41. दे॒वेभ्य॑ ए॒वैव दे॒वेभ्यो॑ दे॒वेभ्य॑ ए॒व । \newline
42. ए॒वैते नै॒ते नै॒वै वैतेन॑ । \newline
43. ए॒तेन॑ करोति करो त्ये॒ते नै॒तेन॑ करोति । \newline
44. क॒रो॒ त्ये॒ताव॑ती रे॒ताव॑तीः करोति करो त्ये॒ताव॑तीः । \newline
45. ए॒ताव॑ती॒र् वै वा ए॒ताव॑ती रे॒ताव॑ती॒र् वै । \newline
46. वै दे॒वता॑ दे॒वता॒ वै वै दे॒वताः᳚ । \newline
47. दे॒वता॒ स्ताभ्य॒ स्ताभ्यो॑ दे॒वता॑ दे॒वता॒ स्ताभ्यः॑ । \newline
48. ताभ्य॑ ए॒वैव ताभ्य॒ स्ताभ्य॑ ए॒व । \newline
49. ए॒वैन॑ मेन मे॒वै वैन᳚म् । \newline
50. ए॒नꣳ॒॒ सर्वा᳚भ्यः॒ सर्वा᳚भ्य एन मेनꣳ॒॒ सर्वा᳚भ्यः । \newline
51. सर्वा᳚भ्य उ॒पाव॑हर त्यु॒पाव॑हरति॒ सर्वा᳚भ्यः॒ सर्वा᳚भ्य उ॒पाव॑हरति । \newline
52. उ॒पाव॑हरति पु॒रा पु॒रोपाव॑हर त्यु॒पाव॑हरति पु॒रा । \newline
53. उ॒पाव॑हर॒तीत्यु॑प - अव॑हरति । \newline
54. पु॒रा वा॒चो वा॒चः पु॒रा पु॒रा वा॒चः । \newline
55. वा॒चः प्रव॑दितोः॒ प्रव॑दितोर् वा॒चो वा॒चः प्रव॑दितोः । \newline

\textbf{Ghana Paata } \newline

1. ब्र॒ह्म॒वा॒दिनो॑ वदन्ति वदन्ति ब्रह्मवा॒दिनो᳚ ब्रह्मवा॒दिनो॑ वदन्ति॒ स स व॑दन्ति ब्रह्मवा॒दिनो᳚ ब्रह्मवा॒दिनो॑ वदन्ति॒ सः । \newline
2. ब्र॒ह्म॒वा॒दिन॒ इति॑ ब्रह्म - वा॒दिनः॑ । \newline
3. व॒द॒न्ति॒ स स व॑दन्ति वदन्ति॒ स तु तु स व॑दन्ति वदन्ति॒ स तु । \newline
4. स तु तु स सत्वै वै तु स सत्वै । \newline
5. त्वै वै तुत्वा अ॑द्ध्व॒र्यु र॑द्ध्व॒र्युर् वै तुत्वा अ॑द्ध्व॒र्युः । \newline
6. वा अ॑द्ध्व॒र्यु र॑द्ध्व॒र्युर् वै वा अ॑द्ध्व॒र्युः स्या᳚थ् स्या दद्ध्व॒र्युर् वै वा अ॑द्ध्व॒र्युः स्या᳚त् । \newline
7. अ॒द्ध्व॒र्युः स्या᳚थ् स्या दद्ध्व॒र्यु र॑द्ध्व॒र्युः स्या॒द् यो यः स्या॑ दद्ध्व॒र्यु र॑द्ध्व॒र्युः स्या॒द् यः । \newline
8. स्या॒द् यो यः स्या᳚थ् स्या॒द् यः सोमꣳ॒॒ सोमं॒ ॅयः स्या᳚थ् स्या॒द् यः सोम᳚म् । \newline
9. यः सोमꣳ॒॒ सोमं॒ ॅयो यः सोम॑ मुपाव॒हर॑न् नुपाव॒हर॒न् थ्सोमं॒ ॅयो यः सोम॑ मुपाव॒हरन्न्॑ । \newline
10. सोम॑ मुपाव॒हर॑न् नुपाव॒हर॒न् थ्सोमꣳ॒॒ सोम॑ मुपाव॒हर॒न् थ्सर्वा᳚भ्यः॒ सर्वा᳚भ्य उपाव॒हर॒न् थ्सोमꣳ॒॒ सोम॑ मुपाव॒हर॒न् थ्सर्वा᳚भ्यः । \newline
11. उ॒पा॒व॒हर॒न् थ्सर्वा᳚भ्यः॒ सर्वा᳚भ्य उपाव॒हर॑न् नुपाव॒हर॒न् थ्सर्वा᳚भ्यो दे॒वता᳚भ्यो दे॒वता᳚भ्यः॒ सर्वा᳚भ्य उपाव॒हर॑न् नुपाव॒हर॒न् थ्सर्वा᳚भ्यो दे॒वता᳚भ्यः । \newline
12. उ॒पा॒व॒हर॒न्नित्यु॑प - अ॒व॒हरन्न्॑ । \newline
13. सर्वा᳚भ्यो दे॒वता᳚भ्यो दे॒वता᳚भ्यः॒ सर्वा᳚भ्यः॒ सर्वा᳚भ्यो दे॒वता᳚भ्य उपाव॒हरे॑ दुपाव॒हरे᳚द् दे॒वता᳚भ्यः॒ सर्वा᳚भ्यः॒ सर्वा᳚भ्यो दे॒वता᳚भ्य उपाव॒हरे᳚त् । \newline
14. दे॒वता᳚भ्य उपाव॒हरे॑ दुपाव॒हरे᳚द् दे॒वता᳚भ्यो दे॒वता᳚भ्य उपाव॒हरे॒ दिती त्यु॑पाव॒हरे᳚द् दे॒वता᳚भ्यो दे॒वता᳚भ्य उपाव॒हरे॒ दिति॑ । \newline
15. उ॒पा॒व॒हरे॒ दिती त्यु॑पाव॒हरे॑ दुपाव॒हरे॒ दिति॑ हृ॒दे हृ॒द इत्यु॑पाव॒हरे॑ दुपाव॒हरे॒ दिति॑ हृ॒दे । \newline
16. उ॒पा॒व॒हरे॒दित्यु॑प - अ॒व॒हरे᳚त् । \newline
17. इति॑ हृ॒दे हृ॒द इतीति॑ हृ॒दे त्वा᳚ त्वा हृ॒द इतीति॑ हृ॒दे त्वा᳚ । \newline
18. हृ॒दे त्वा᳚ त्वा हृ॒दे हृ॒दे त्वेतीति॑ त्वा हृ॒दे हृ॒दे त्वेति॑ । \newline
19. त्वेतीति॑ त्वा॒ त्वेत्या॑हा॒ हेति॑ त्वा॒ त्वेत्या॑ह । \newline
20. इत्या॑हा॒हे तीत्या॑ह मनु॒ष्ये᳚भ्यो मनु॒ष्ये᳚भ्य आ॒हे तीत्या॑ह मनु॒ष्ये᳚भ्यः । \newline
21. आ॒ह॒ म॒नु॒ष्ये᳚भ्यो मनु॒ष्ये᳚भ्य आहाह मनु॒ष्ये᳚भ्य ए॒वैव म॑नु॒ष्ये᳚भ्य आहाह मनु॒ष्ये᳚भ्य ए॒व । \newline
22. म॒नु॒ष्ये᳚भ्य ए॒वैव म॑नु॒ष्ये᳚भ्यो मनु॒ष्ये᳚भ्य ए॒वैते नै॒तेनै॒व म॑नु॒ष्ये᳚भ्यो मनु॒ष्ये᳚भ्य ए॒वै तेन॑ । \newline
23. ए॒वैते नै॒ते नै॒वैवैतेन॑ करोति करो त्ये॒ते नै॒वैवैतेन॑ करोति । \newline
24. ए॒तेन॑ करोति करो त्ये॒तेनै॒तेन॑ करोति॒ मन॑से॒ मन॑से करो त्ये॒ते नै॒तेन॑ करोति॒ मन॑से । \newline
25. क॒रो॒ति॒ मन॑से॒ मन॑से करोति करोति॒ मन॑से त्वा त्वा॒ मन॑से करोति करोति॒ मन॑से त्वा । \newline
26. मन॑से त्वा त्वा॒ मन॑से॒ मन॑से॒ त्वेतीति॑ त्वा॒ मन॑से॒ मन॑से॒ त्वेति॑ । \newline
27. त्वेतीति॑ त्वा॒ त्वेत्या॑हा॒ हेति॑ त्वा॒ त्वेत्या॑ह । \newline
28. इत्या॑हा॒हे तीत्या॑ह पि॒तृभ्यः॑ पि॒तृभ्य॑ आ॒हे तीत्या॑ह पि॒तृभ्यः॑ । \newline
29. आ॒ह॒ पि॒तृभ्यः॑ पि॒तृभ्य॑ आहाह पि॒तृभ्य॑ ए॒वैव पि॒तृभ्य॑ आहाह पि॒तृभ्य॑ ए॒व । \newline
30. पि॒तृभ्य॑ ए॒वैव पि॒तृभ्यः॑ पि॒तृभ्य॑ ए॒वैते नै॒तेनै॒व पि॒तृभ्यः॑ पि॒तृभ्य॑ ए॒वै तेन॑ । \newline
31. पि॒तृभ्य॒ इति॑ पि॒तृ - भ्यः॒ । \newline
32. ए॒वैते नै॒ते नै॒वैवैतेन॑ करोति करो त्ये॒ते नै॒वैवैतेन॑ करोति । \newline
33. ए॒तेन॑ करोति करो त्ये॒तेनै॒तेन॑ करोति दि॒वे दि॒वे क॑रो त्ये॒तेनै॒तेन॑ करोति दि॒वे । \newline
34. क॒रो॒ति॒ दि॒वे दि॒वे क॑रोति करोति दि॒वे त्वा᳚ त्वा दि॒वे क॑रोति करोति दि॒वे त्वा᳚ । \newline
35. दि॒वे त्वा᳚ त्वा दि॒वे दि॒वे त्वा॒ सूर्या॑य॒ सूर्या॑य त्वा दि॒वे दि॒वे त्वा॒ सूर्या॑य । \newline
36. त्वा॒ सूर्या॑य॒ सूर्या॑य त्वा त्वा॒ सूर्या॑य त्वा त्वा॒ सूर्या॑य त्वा त्वा॒ सूर्या॑य त्वा । \newline
37. सूर्या॑य त्वा त्वा॒ सूर्या॑य॒ सूर्या॑य॒ त्वेतीति॑ त्वा॒ सूर्या॑य॒ सूर्या॑य॒ त्वेति॑ । \newline
38. त्वेतीति॑ त्वा॒ त्वेत्या॑हा॒ हेति॑ त्वा॒ त्वेत्या॑ह । \newline
39. इत्या॑हा॒हे तीत्या॑ह दे॒वेभ्यो॑ दे॒वेभ्य॑ आ॒हे तीत्या॑ह दे॒वेभ्यः॑ । \newline
40. आ॒ह॒ दे॒वेभ्यो॑ दे॒वेभ्य॑ आहाह दे॒वेभ्य॑ ए॒वैव दे॒वेभ्य॑ आहाह दे॒वेभ्य॑ ए॒व । \newline
41. दे॒वेभ्य॑ ए॒वैव दे॒वेभ्यो॑ दे॒वेभ्य॑ ए॒वैते नै॒तेनै॒व दे॒वेभ्यो॑ दे॒वेभ्य॑ ए॒वै तेन॑ । \newline
42. ए॒वैते नै॒ते नै॒वैवैतेन॑ करोति करो त्ये॒ते नै॒वैवैतेन॑ करोति । \newline
43. ए॒तेन॑ करोति करो त्ये॒तेनै॒तेन॑ करो त्ये॒ताव॑ती रे॒ताव॑तीः करो त्ये॒ते नै॒तेन॑ करो त्ये॒ताव॑तीः । \newline
44. क॒रो॒ त्ये॒ताव॑ती रे॒ताव॑तीः करोति करो त्ये॒ताव॑ती॒र् वै वा ए॒ताव॑तीः करोति करो त्ये॒ताव॑ती॒र् वै । \newline
45. ए॒ताव॑ती॒र् वै वा ए॒ताव॑ती रे॒ताव॑ती॒र् वै दे॒वता॑ दे॒वता॒ वा ए॒ताव॑ती रे॒ताव॑ती॒र् वै दे॒वताः᳚ । \newline
46. वै दे॒वता॑ दे॒वता॒ वै वै दे॒वता॒ स्ताभ्य॒ स्ताभ्यो॑ दे॒वता॒ वै वै दे॒वता॒ स्ताभ्यः॑ । \newline
47. दे॒वता॒ स्ताभ्य॒ स्ताभ्यो॑ दे॒वता॑ दे॒वता॒ स्ताभ्य॑ ए॒वैव ताभ्यो॑ दे॒वता॑ दे॒वता॒ स्ताभ्य॑ ए॒व । \newline
48. ताभ्य॑ ए॒वैव ताभ्य॒ स्ताभ्य॑ ए॒वैन॑ मेन मे॒व ताभ्य॒ स्ताभ्य॑ ए॒वैन᳚म् । \newline
49. ए॒वैन॑ मेन मे॒वै वैनꣳ॒॒ सर्वा᳚भ्यः॒ सर्वा᳚भ्य एन मे॒वै वैनꣳ॒॒ सर्वा᳚भ्यः । \newline
50. ए॒नꣳ॒॒ सर्वा᳚भ्यः॒ सर्वा᳚भ्य एन मेनꣳ॒॒ सर्वा᳚भ्य उ॒पाव॑हर त्यु॒पाव॑हरति॒ सर्वा᳚भ्य एन मेनꣳ॒॒ सर्वा᳚भ्य उ॒पाव॑हरति । \newline
51. सर्वा᳚भ्य उ॒पाव॑हर त्यु॒पाव॑हरति॒ सर्वा᳚भ्यः॒ सर्वा᳚भ्य उ॒पाव॑हरति पु॒रा पु॒रो पाव॑हरति॒ सर्वा᳚भ्यः॒ सर्वा᳚भ्य उ॒पाव॑हरति पु॒रा । \newline
52. उ॒पाव॑हरति पु॒रा पु॒रोपाव॑हर त्यु॒पाव॑हरति पु॒रा वा॒चो वा॒चः पु॒रोपाव॑हर त्यु॒पाव॑हरति पु॒रा वा॒चः । \newline
53. उ॒पाव॑हर॒तीत्यु॑प - अव॑हरति । \newline
54. पु॒रा वा॒चो वा॒चः पु॒रा पु॒रा वा॒चः प्रव॑दितोः॒ प्रव॑दितोर् वा॒चः पु॒रा पु॒रा वा॒चः प्रव॑दितोः । \newline
55. वा॒चः प्रव॑दितोः॒ प्रव॑दितोर् वा॒चो वा॒चः प्रव॑दितोः प्रातरनुवा॒कम् प्रा॑तरनुवा॒कम् प्रव॑दितोर् वा॒चो वा॒चः प्रव॑दितोः प्रातरनुवा॒कम् । \newline
\pagebreak
\markright{ TS 6.4.3.2  \hfill https://www.vedavms.in \hfill}

\section{ TS 6.4.3.2 }

\textbf{TS 6.4.3.2 } \newline
\textbf{Samhita Paata} \newline

प्रव॑दितोः प्रातरनुवा॒कमु॒पाक॑रोति॒ याव॑त्ये॒व वाक् तामव॑ रुन्धे॒ ऽपोऽग्रे॑ऽभि॒व्याह॑रति य॒ज्ञो वा आपो॑ य॒ज्ञ्मे॒वाभि वाचं॒ ॅविसृ॑जति॒ सर्वा॑णि॒ छन्दाꣳ॒॒स्यन्वा॑ह प॒शवो॒ वै छन्दाꣳ॑सि प॒शूने॒वाव॑ रुन्धे गायत्रि॒या तेज॑स्कामस्य॒ परि॑ दद्ध्यात् त्रि॒ष्टुभे᳚न्द्रि॒य का॑मस्य॒ जग॑त्या प॒शुका॑मस्यानु॒ष्टुभा᳚ प्रति॒ष्ठाका॑मस्य प॒ङ्क्त्या य॒ज्ञ्का॑मस्य वि॒राजाऽन्न॑कामस्य शृ॒णोत्व॒ग्निः स॒मिधा॒ हवं॑- [  ] \newline

\textbf{Pada Paata} \newline

प्रव॑दितो॒रिति॒ प्र - व॒दि॒तोः॒ । प्रा॒त॒र॒नु॒वा॒कमिति॑ प्रातः - अ॒नु॒वा॒कम् । उ॒पाक॑रो॒तीत्यु॑प-आक॑रोति । याव॑ती । ए॒व । वाक् । ताम् । अवेति॑ । रु॒न्धे॒ । अ॒पः । अग्रे᳚ । अ॒भि॒व्याह॑र॒तीत्य॑भि - व्याह॑रति । य॒ज्ञ्ः । वै । आपः॑ । य॒ज्ञ्म् । ए॒व । अ॒भीति॑ । वाच᳚म् । वीति॑ । सृ॒ज॒ति॒ । सर्वा॑णि । छन्दाꣳ॑सि । अन्विति॑ । आ॒ह॒ । प॒शवः॑ । वै । छन्दाꣳ॑सि । प॒शून् । ए॒व । अवेति॑ । रु॒न्धे॒ । गा॒य॒त्रि॒या । तेज॑स्काम॒स्येति॒ तेजः॑ - का॒म॒स्य॒ । परीति॑ । द॒द्ध्या॒त् । त्रि॒ष्टुभा᳚ । इ॒न्द्रि॒यका॑म॒स्येती᳚न्द्रि॒य - का॒म॒स्य॒ । जग॑त्या । प॒शुका॑म॒स्येति॑ प॒शु - का॒म॒स्य॒ । अ॒नु॒ष्टुभेत्य॑नु - स्तुभा᳚ । प्र॒ति॒ष्ठाका॑म॒स्येति॑ प्रति॒ष्ठा - का॒म॒स्य॒ । प॒ङ्क्त्या । य॒ज्ञ्का॑म॒स्येति॑ य॒ज्ञ् - का॒म॒स्य॒ । वि॒राजेति॑ वि - राजा᳚ । अन्न॑काम॒स्येत्यन्न॑ - का॒म॒स्य॒ । शृ॒णोतु॑ । अ॒ग्निः । स॒मिधेति॑ सं - इधा᳚ । हव᳚म् ।  \newline


\textbf{Krama Paata} \newline

प्रव॑दितोः प्रातरनुवा॒कम् । प्रव॑दितो॒रिति॒ प्र - व॒दि॒तोः॒ । प्रा॒त॒र॒नु॒वा॒कमु॒पाक॑रोति । प्रा॒त॒र॒नु॒वा॒कमिति॑ प्रातः - अ॒नु॒वा॒कम् । उ॒पाक॑रोति॒ याव॑ती । उ॒पाक॑रो॒तीत्यु॑प - आक॑रोति । याव॑त्ये॒व । ए॒व वाक् । वाक् ताम् । तामव॑ । अव॑ रुन्धे । रु॒न्धे॒ऽपः । अ॒पोऽग्रे᳚ । अग्रे॑ऽभि॒व्याह॑रति । अ॒भि॒व्याह॑रति य॒ज्ञ्ः । अ॒भि॒व्याह॑र॒तीत्य॑भि - व्याह॑रति । य॒ज्ञो वै । वा आपः॑ । आपो॑ य॒ज्ञ्म् । य॒ज्ञ्मे॒व । ए॒वाभि । अ॒भि वाच᳚म् । वाच॒म् ॅवि । वि सृ॑जति । सृ॒ज॒ति॒ सर्वा॑णि । सर्वा॑णि॒ छन्दाꣳ॑सि । छन्दाꣳ॒॒स्यनु॑ । अन्वा॑ह । आ॒ह॒ प॒शवः॑ । प॒शवो॒ वै । वै छन्दाꣳ॑सि । छन्दाꣳ॑सि प॒शून् । प॒शूने॒व । ए॒वाव॑ । अव॑ रुन्धे । रु॒न्धे॒ गा॒य॒त्रि॒या । गा॒य॒त्रि॒या तेज॑स्कामस्य । तेज॑स्कामस्य॒ परि॑ । तेज॑स्काम॒स्येति॒ तेजः॑ - का॒म॒स्य॒ । परि॑ दद्ध्यात् । द॒द्ध्या॒त् त्रि॒ष्टुभा᳚ । त्रि॒ष्टुभे᳚न्द्रि॒यका॑मस्य । इ॒न्द्रि॒यका॑मस्य॒ जग॑त्या । इ॒न्द्रि॒यका॑म॒स्येती᳚न्द्रि॒य - का॒म॒स्य॒ । जग॑त्या प॒शुका॑मस्य । प॒शुका॑मस्यानु॒ष्टुभा᳚ । प॒शुका॑म॒स्येति॑ प॒शु - का॒म॒स्य॒ । अ॒नु॒ष्टुभा᳚ प्रति॒ष्ठाका॑मस्य । अ॒नु॒ष्टुभेत्य॑नु - स्तुभा᳚ । प्र॒ति॒ष्ठाका॑मस्य प॒ङ्‍क्त्या । प्र॒ति॒ष्ठाका॑म॒स्येति॑ प्रति॒ष्ठा - का॒म॒स्य॒ । प॒ङ्‍क्त्या य॒ज्ञ्का॑मस्य । य॒ज्ञ्का॑मस्य वि॒राजा᳚ । य॒ज्ञ्का॑म॒स्येति॑ य॒ज्ञ् - का॒म॒स्य॒ । वि॒राजाऽन्न॑कामस्य । वि॒राजेति॑ वि - राजा᳚ । अन्न॑कामस्य शृ॒णोतु॑ । अन्न॑काम॒स्येत्यन्न॑ - का॒म॒स्य॒ । शृ॒णोत्व॒ग्निः । अ॒ग्निः स॒मिधा᳚ । स॒मिधा॒ हव᳚म् । स॒मिधेति॑ सम् - इधा᳚ । हव॑म् मे \newline

\textbf{Jatai Paata} \newline

1. प्रव॑दितोः प्रातरनुवा॒कम् प्रा॑तरनुवा॒कम् प्रव॑दितोः॒ प्रव॑दितोः प्रातरनुवा॒कम् । \newline
2. प्रव॑दितो॒रिति॒ प्र - व॒दि॒तोः॒ । \newline
3. प्रा॒त॒र॒नु॒वा॒क मु॒पाक॑रो त्यु॒पाक॑रोति प्रातरनुवा॒कम् प्रा॑तरनुवा॒क मु॒पाक॑रोति । \newline
4. प्रा॒त॒र॒नु॒वा॒कमिति॑ प्रातः - अ॒नु॒वा॒कम् । \newline
5. उ॒पाक॑रोति॒ याव॑ती॒ याव॑ त्यु॒पाक॑रो त्यु॒पाक॑रोति॒ याव॑ती । \newline
6. उ॒पाक॑रो॒तीत्यु॑प - आक॑रोति । \newline
7. याव॑ त्ये॒वैव याव॑ती॒ याव॑ त्ये॒व । \newline
8. ए॒व वाग् वागे॒वैव वाक् । \newline
9. वाक् ताम् तां ॅवाग् वाक् ताम् । \newline
10. ता मवाव॒ ताम् ता मव॑ । \newline
11. अव॑ रुन्धे रु॒न्धे ऽवाव॑ रुन्धे । \newline
12. रु॒न्धे॒ ऽपो॑ ऽपो रु॑न्धे रुन्धे॒ ऽपः । \newline
13. अ॒पो ऽग्रे ऽग्रे॒ ऽपो॑ ऽपो ऽग्रे᳚ । \newline
14. अग्रे॑ ऽभि॒व्याह॑र त्यभि॒व्याह॑र॒ त्यग्रे ऽग्रे॑ ऽभि॒व्याह॑रति । \newline
15. अ॒भि॒व्याह॑रति य॒ज्ञो य॒ज्ञो॑ ऽभि॒व्याह॑र त्यभि॒व्याह॑रति य॒ज्ञ्ः । \newline
16. अ॒भि॒व्याह॑र॒तीत्य॑भि - व्याह॑रति । \newline
17. य॒ज्ञो वै वै य॒ज्ञो य॒ज्ञो वै । \newline
18. वा आप॒ आपो॒ वै वा आपः॑ । \newline
19. आपो॑ य॒ज्ञ्ं ॅय॒ज्ञ् माप॒ आपो॑ य॒ज्ञ्म् । \newline
20. य॒ज्ञ् मे॒वैव य॒ज्ञ्ं ॅय॒ज्ञ् मे॒व । \newline
21. ए॒वाभ्या᳚(1॒)भ्ये॑ वैवाभि । \newline
22. अ॒भि वाचं॒ ॅवाच॑ म॒भ्य॑भि वाच᳚म् । \newline
23. वाचं॒ ॅवि वि वाचं॒ ॅवाचं॒ ॅवि । \newline
24. वि सृ॑जति सृजति॒ वि वि सृ॑जति । \newline
25. सृ॒ज॒ति॒ सर्वा॑णि॒ सर्वा॑णि सृजति सृजति॒ सर्वा॑णि । \newline
26. सर्वा॑णि॒ छन्दाꣳ॑सि॒ छन्दाꣳ॑सि॒ सर्वा॑णि॒ सर्वा॑णि॒ छन्दाꣳ॑सि । \newline
27. छन्दाꣳ॒॒ स्यन् वनु॒ च्छन्दाꣳ॑सि॒ छन्दाꣳ॒॒ स्यनु॑ । \newline
28. अन्वा॑हा॒ हान् वन् वा॑ह । \newline
29. आ॒ह॒ प॒शवः॑ प॒शव॑ आहाह प॒शवः॑ । \newline
30. प॒शवो॒ वै वै प॒शवः॑ प॒शवो॒ वै । \newline
31. वै छन्दाꣳ॑सि॒ छन्दाꣳ॑सि॒ वै वै छन्दाꣳ॑सि । \newline
32. छन्दाꣳ॑सि प॒शून् प॒शून् छन्दाꣳ॑सि॒ छन्दाꣳ॑सि प॒शून् । \newline
33. प॒शूने॒ वैव प॒शून् प॒शूने॒व । \newline
34. ए॒वावा वै॒वै वाव॑ । \newline
35. अव॑ रुन्धे रु॒न्धे ऽवाव॑ रुन्धे । \newline
36. रु॒न्धे॒ गा॒य॒त्रि॒या गा॑यत्रि॒या रु॑न्धे रुन्धे गायत्रि॒या । \newline
37. गा॒य॒त्रि॒या तेज॑स्कामस्य॒ तेज॑स्कामस्य गायत्रि॒या गा॑यत्रि॒या तेज॑स्कामस्य । \newline
38. तेज॑स्कामस्य॒ परि॒ परि॒ तेज॑स्कामस्य॒ तेज॑स्कामस्य॒ परि॑ । \newline
39. तेज॑स्काम॒स्येति॒ तेजः॑ - का॒म॒स्य॒ । \newline
40. परि॑ दद्ध्याद् दद्ध्या॒त् परि॒ परि॑ दद्ध्यात् । \newline
41. द॒द्ध्या॒त् त्रि॒ष्टुभा᳚ त्रि॒ष्टुभा॑ दद्ध्याद् दद्ध्यात् त्रि॒ष्टुभा᳚ । \newline
42. त्रि॒ष्टु भे᳚न्द्रि॒यका॑म स्येन्द्रि॒यका॑मस्य त्रि॒ष्टुभा᳚ त्रि॒ष्टु भे᳚न्द्रि॒यका॑मस्य । \newline
43. इ॒न्द्रि॒यका॑मस्य॒ जग॑त्या॒ जग॑ त्येन्द्रि॒यका॑म स्येन्द्रि॒यका॑मस्य॒ जग॑त्या । \newline
44. इ॒न्द्रि॒यका॑म॒स्येती᳚न्द्रि॒य - का॒म॒स्य॒ । \newline
45. जग॑त्या प॒शुका॑मस्य प॒शुका॑मस्य॒ जग॑त्या॒ जग॑त्या प॒शुका॑मस्य । \newline
46. प॒शुका॑मस्या नु॒ष्टुभा॑ ऽनु॒ष्टुभा॑ प॒शुका॑मस्य प॒शुका॑मस्या नु॒ष्टुभा᳚ । \newline
47. प॒शुका॑म॒स्येति॑ प॒शु - का॒म॒स्य॒ । \newline
48. अ॒नु॒ष्टुभा᳚ प्रति॒ष्ठाका॑मस्य प्रति॒ष्ठाका॑मस्या नु॒ष्टुभा॑ ऽनु॒ष्टुभा᳚ प्रति॒ष्ठाका॑मस्य । \newline
49. अ॒नु॒ष्टुभेत्य॑नु - स्तुभा᳚ । \newline
50. प्र॒ति॒ष्ठाका॑मस्य प॒ङ्क्त्या प॒ङ्क्त्या प्र॑ति॒ष्ठाका॑मस्य प्रति॒ष्ठाका॑मस्य प॒ङ्क्त्या । \newline
51. प्र॒ति॒ष्ठाका॑म॒स्येति॑ प्रति॒ष्ठा - का॒म॒स्य॒ । \newline
52. प॒ङ्क्त्या य॒ज्ञ्का॑मस्य य॒ज्ञ्का॑मस्य प॒ङ्क्त्या प॒ङ्क्त्या य॒ज्ञ्का॑मस्य । \newline
53. य॒ज्ञ्का॑मस्य वि॒राजा॑ वि॒राजा॑ य॒ज्ञ्का॑मस्य य॒ज्ञ्का॑मस्य वि॒राजा᳚ । \newline
54. य॒ज्ञ्का॑म॒स्येति॑ य॒ज्ञ् - का॒म॒स्य॒ । \newline
55. वि॒राजा ऽन्न॑काम॒स्या न्न॑कामस्य वि॒राजा॑ वि॒राजा ऽन्न॑कामस्य । \newline
56. वि॒राजेति॑ वि - राजा᳚ । \newline
57. अन्न॑कामस्य शृ॒णोतु॑ शृ॒णो त्वन्न॑काम॒स्या न्न॑कामस्य शृ॒णोतु॑ । \newline
58. अन्न॑काम॒स्येत्यन्न॑ - का॒म॒स्य॒ । \newline
59. शृ॒णो त्व॒ग्नि र॒ग्निः शृ॒णोतु॑ शृ॒णो त्व॒ग्निः । \newline
60. अ॒ग्निः स॒मिधा॑ स॒मिधा॒ ऽग्नि र॒ग्निः स॒मिधा᳚ । \newline
61. स॒मिधा॒ हवꣳ॒॒ हवꣳ॑ स॒मिधा॑ स॒मिधा॒ हव᳚म् । \newline
62. स॒मिधेति॑ सं - इधा᳚ । \newline
63. हव॑म् मे मे॒ हवꣳ॒॒ हव॑म् मे । \newline

\textbf{Ghana Paata } \newline

1. प्रव॑दितोः प्रातरनुवा॒कम् प्रा॑तरनुवा॒कम् प्रव॑दितोः॒ प्रव॑दितोः प्रातरनुवा॒क मु॒पाक॑रो त्यु॒पाक॑रोति प्रातरनुवा॒कम् प्रव॑दितोः॒ प्रव॑दितोः प्रातरनुवा॒क मु॒पाक॑रोति । \newline
2. प्रव॑दितो॒रिति॒ प्र - व॒दि॒तोः॒ । \newline
3. प्रा॒त॒र॒नु॒वा॒क मु॒पाक॑रो त्यु॒पाक॑रोति प्रातरनुवा॒कम् प्रा॑तरनुवा॒क मु॒पाक॑रोति॒ याव॑ती॒ याव॑
त्यु॒पाक॑रोति प्रातरनुवा॒कम् प्रा॑तरनुवा॒क मु॒पाक॑रोति॒ याव॑ती । \newline
4. प्रा॒त॒र॒नु॒वा॒कमिति॑ प्रातः - अ॒नु॒वा॒कम् । \newline
5. उ॒पाक॑रोति॒ याव॑ती॒ याव॑ त्यु॒पाक॑रो त्यु॒पाक॑रोति॒ याव॑ त्ये॒वैव याव॑ त्यु॒पाक॑रो त्यु॒पाक॑रोति॒ याव॑ त्ये॒व । \newline
6. उ॒पाक॑रो॒तीत्यु॑प - आक॑रोति । \newline
7. याव॑ त्ये॒वैव याव॑ती॒ याव॑ त्ये॒व वाग् वागे॒व याव॑ती॒ याव॑ त्ये॒व वाक् । \newline
8. ए॒व वाग् वागे॒ वैव वाक् ताम् तां ॅवागे॒ वैव वाक् ताम् । \newline
9. वाक् ताम् तां ॅवाग् वाक् ता मवाव॒ तां ॅवाग् वाक् ता मव॑ । \newline
10. ता मवाव॒ ताम् ता मव॑ रुन्धे रु॒न्धे ऽव॒ ताम् ता मव॑ रुन्धे । \newline
11. अव॑ रुन्धे रु॒न्धे ऽवाव॑ रुन्धे॒ ऽपो॑ ऽपो रु॒न्धे ऽवाव॑ रुन्धे॒ ऽपः । \newline
12. रु॒न्धे॒ ऽपो॑ ऽपो रु॑न्धे रुन्धे॒ ऽपो ऽग्रे ऽग्रे॒ ऽपो रु॑न्धे रुन्धे॒ ऽपो ऽग्रे᳚ । \newline
13. अ॒पो ऽग्रे ऽग्रे॒ ऽपो॑ ऽपो ऽग्रे॑ ऽभि॒व्याह॑र त्यभि॒व्याह॑र॒ त्यग्रे॒ ऽपो॑ ऽपो ऽग्रे॑ ऽभि॒व्याह॑रति । \newline
14. अग्रे॑ ऽभि॒व्याह॑र त्यभि॒व्याह॑र॒ त्यग्रे ऽग्रे॑ ऽभि॒व्याह॑रति य॒ज्ञो य॒ज्ञो॑ ऽभि॒व्याह॑र॒ त्यग्रे ऽग्रे॑ ऽभि॒व्याह॑रति य॒ज्ञ्ः । \newline
15. अ॒भि॒व्याह॑रति य॒ज्ञो य॒ज्ञो॑ ऽभि॒व्याह॑र त्यभि॒व्याह॑रति य॒ज्ञो वै वै य॒ज्ञो॑ ऽभि॒व्याह॑र त्यभि॒व्याह॑रति य॒ज्ञो वै । \newline
16. अ॒भि॒व्याह॑र॒तीत्य॑भि - व्याह॑रति । \newline
17. य॒ज्ञो वै वै य॒ज्ञो य॒ज्ञो वा आप॒ आपो॒ वै य॒ज्ञो य॒ज्ञो वा आपः॑ । \newline
18. वा आप॒ आपो॒ वै वा आपो॑ य॒ज्ञ्ं ॅय॒ज्ञ् मापो॒ वै वा आपो॑ य॒ज्ञ्म् । \newline
19. आपो॑ य॒ज्ञ्ं ॅय॒ज्ञ् माप॒ आपो॑ य॒ज्ञ् मे॒वैव य॒ज्ञ् माप॒ आपो॑ य॒ज्ञ् मे॒व । \newline
20. य॒ज्ञ् मे॒वैव य॒ज्ञ्ं ॅय॒ज्ञ् मे॒वाभ्या᳚(1॒)भ्ये॑व य॒ज्ञ्ं ॅय॒ज्ञ् मे॒वाभि । \newline
21. ए॒वाभ्या᳚(1॒) भ्ये॑वैवाभि वाचं॒ ॅवाच॑ म॒भ्ये॑ वैवाभि वाच᳚म् । \newline
22. अ॒भि वाचं॒ ॅवाच॑ म॒भ्य॑भि वाचं॒ ॅवि वि वाच॑ म॒भ्य॑भि वाचं॒ ॅवि । \newline
23. वाचं॒ ॅवि वि वाचं॒ ॅवाचं॒ ॅवि सृ॑जति सृजति॒ वि वाचं॒ ॅवाचं॒ ॅवि सृ॑जति । \newline
24. वि सृ॑जति सृजति॒ वि वि सृ॑जति॒ सर्वा॑णि॒ सर्वा॑णि सृजति॒ वि वि सृ॑जति॒ सर्वा॑णि । \newline
25. सृ॒ज॒ति॒ सर्वा॑णि॒ सर्वा॑णि सृजति सृजति॒ सर्वा॑णि॒ छन्दाꣳ॑सि॒ छन्दाꣳ॑सि॒ सर्वा॑णि सृजति सृजति॒ सर्वा॑णि॒ छन्दाꣳ॑सि । \newline
26. सर्वा॑णि॒ छन्दाꣳ॑सि॒ छन्दाꣳ॑सि॒ सर्वा॑णि॒ सर्वा॑णि॒ छन्दाꣳ॒॒ स्यन् वनु॒ च्छन्दाꣳ॑सि॒ सर्वा॑णि॒ सर्वा॑णि॒ छन्दाꣳ॒॒ स्यनु॑ । \newline
27. छन्दाꣳ॒॒ स्यन् वनु॒ च्छन्दाꣳ॑सि॒ छन्दाꣳ॒॒ स्यन् वा॑हा॒ हानु॒ च्छन्दाꣳ॑सि॒ छन्दाꣳ॒॒ स्यन्वा॑ह । \newline
28. अन्वा॑ हा॒हान् वन् वा॑ह प॒शवः॑ प॒शव॑ आ॒हान् वन् वा॑ह प॒शवः॑ । \newline
29. आ॒ह॒ प॒शवः॑ प॒शव॑ आहाह प॒शवो॒ वै वै प॒शव॑ आहाह प॒शवो॒ वै । \newline
30. प॒शवो॒ वै वै प॒शवः॑ प॒शवो॒ वै छन्दाꣳ॑सि॒ छन्दाꣳ॑सि॒ वै प॒शवः॑ प॒शवो॒ वै छन्दाꣳ॑सि । \newline
31. वै छन्दाꣳ॑सि॒ छन्दाꣳ॑सि॒ वै वै छन्दाꣳ॑सि प॒शून् प॒शून् छन्दाꣳ॑सि॒ वै वै छन्दाꣳ॑सि प॒शून् । \newline
32. छन्दाꣳ॑सि प॒शून् प॒शून् छन्दाꣳ॑सि॒ छन्दाꣳ॑सि प॒शूने॒ वैव प॒शून् छन्दाꣳ॑सि॒ छन्दाꣳ॑सि प॒शूने॒व । \newline
33. प॒शूने॒ वैव प॒शून् प॒शूने॒ वावा वै॒व प॒शून् प॒शूने॒ वाव॑ । \newline
34. ए॒वावा वै॒वै वाव॑ रुन्धे रु॒न्धे ऽवै॒वै वाव॑ रुन्धे । \newline
35. अव॑ रुन्धे रु॒न्धे ऽवाव॑ रुन्धे गायत्रि॒या गा॑यत्रि॒या रु॒न्धे ऽवाव॑ रुन्धे गायत्रि॒या । \newline
36. रु॒न्धे॒ गा॒य॒त्रि॒या गा॑यत्रि॒या रु॑न्धे रुन्धे गायत्रि॒या तेज॑स्कामस्य॒ तेज॑स्कामस्य गायत्रि॒या रु॑न्धे रुन्धे गायत्रि॒या तेज॑स्कामस्य । \newline
37. गा॒य॒त्रि॒या तेज॑स्कामस्य॒ तेज॑स्कामस्य गायत्रि॒या गा॑यत्रि॒या तेज॑स्कामस्य॒ परि॒ परि॒ तेज॑स्कामस्य गायत्रि॒या गा॑यत्रि॒या तेज॑स्कामस्य॒ परि॑ । \newline
38. तेज॑स्कामस्य॒ परि॒ परि॒ तेज॑स्कामस्य॒ तेज॑स्कामस्य॒ परि॑ दद्ध्याद् दद्ध्या॒त् परि॒ तेज॑स्कामस्य॒ तेज॑स्कामस्य॒ परि॑ दद्ध्यात् । \newline
39. तेज॑स्काम॒स्येति॒ तेजः॑ - का॒म॒स्य॒ । \newline
40. परि॑ दद्ध्याद् दद्ध्या॒त् परि॒ परि॑ दद्ध्यात् त्रि॒ष्टुभा᳚ त्रि॒ष्टुभा॑ दद्ध्या॒त् परि॒ परि॑ दद्ध्यात् त्रि॒ष्टुभा᳚ । \newline
41. द॒द्ध्या॒त् त्रि॒ष्टुभा᳚ त्रि॒ष्टुभा॑ दद्ध्याद् दद्ध्यात् त्रि॒ष्टु भे᳚न्द्रि॒यका॑म स्येन्द्रि॒यका॑मस्य त्रि॒ष्टुभा॑ दद्ध्याद् दद्ध्यात् त्रि॒ष्टु भे᳚न्द्रि॒यका॑मस्य । \newline
42. त्रि॒ष्टु भे᳚न्द्रि॒यका॑म स्येन्द्रि॒यका॑मस्य त्रि॒ष्टुभा᳚ त्रि॒ष्टु भे᳚न्द्रि॒यका॑मस्य॒ जग॑त्या॒ जग॑
त्येन्द्रि॒यका॑मस्य त्रि॒ष्टुभा᳚ त्रि॒ष्टु भे᳚न्द्रि॒यका॑मस्य॒ जग॑त्या । \newline
43. इ॒न्द्रि॒यका॑मस्य॒ जग॑त्या॒ जग॑ त्येन्द्रि॒यका॑म स्येन्द्रि॒यका॑मस्य॒ जग॑त्या प॒शुका॑मस्य प॒शुका॑मस्य॒ जग॑ त्येन्द्रि॒यका॑म स्येन्द्रि॒यका॑मस्य॒ जग॑त्या प॒शुका॑मस्य । \newline
44. इ॒न्द्रि॒यका॑म॒स्येती᳚न्द्रि॒य - का॒म॒स्य॒ । \newline
45. जग॑त्या प॒शुका॑मस्य प॒शुका॑मस्य॒ जग॑त्या॒ जग॑त्या प॒शुका॑म स्यानु॒ष्टुभा॑ ऽनु॒ष्टुभा॑ प॒शुका॑मस्य॒ जग॑त्या॒ जग॑त्या प॒शुका॑म स्यानु॒ष्टुभा᳚ । \newline
46. प॒शुका॑मस्या नु॒ष्टुभा॑ ऽनु॒ष्टुभा॑ प॒शुका॑मस्य प॒शुका॑मस्या नु॒ष्टुभा᳚ प्रति॒ष्ठाका॑मस्य प्रति॒ष्ठाका॑मस्या नु॒ष्टुभा॑ प॒शुका॑मस्य प॒शुका॑मस्या नु॒ष्टुभा᳚ प्रति॒ष्ठाका॑मस्य । \newline
47. प॒शुका॑म॒स्येति॑ प॒शु - का॒म॒स्य॒ । \newline
48. अ॒नु॒ष्टुभा᳚ प्रति॒ष्ठाका॑मस्य प्रति॒ष्ठाका॑मस्या नु॒ष्टुभा॑ ऽनु॒ष्टुभा᳚ प्रति॒ष्ठाका॑मस्य प॒ङ्क्त्या प॒ङ्क्त्या प्र॑ति॒ष्ठाका॑मस्या नु॒ष्टुभा॑ ऽनु॒ष्टुभा᳚ प्रति॒ष्ठाका॑मस्य प॒ङ्क्त्या । \newline
49. अ॒नु॒ष्टुभेत्य॑नु - स्तुभा᳚ । \newline
50. प्र॒ति॒ष्ठाका॑मस्य प॒ङ्क्त्या प॒ङ्क्त्या प्र॑ति॒ष्ठाका॑मस्य प्रति॒ष्ठाका॑मस्य प॒ङ्क्त्या य॒ज्ञ्का॑मस्य य॒ज्ञ्का॑मस्य प॒ङ्क्त्या प्र॑ति॒ष्ठाका॑मस्य प्रति॒ष्ठाका॑मस्य प॒ङ्क्त्या य॒ज्ञ्का॑मस्य । \newline
51. प्र॒ति॒ष्ठाका॑म॒स्येति॑ प्रति॒ष्ठा - का॒म॒स्य॒ । \newline
52. प॒ङ्क्त्या य॒ज्ञ्का॑मस्य य॒ज्ञ्का॑मस्य प॒ङ्क्त्या प॒ङ्क्त्या य॒ज्ञ्का॑मस्य वि॒राजा॑ वि॒राजा॑ य॒ज्ञ्का॑मस्य प॒ङ्क्त्या प॒ङ्क्त्या य॒ज्ञ्का॑मस्य वि॒राजा᳚ । \newline
53. य॒ज्ञ्का॑मस्य वि॒राजा॑ वि॒राजा॑ य॒ज्ञ्का॑मस्य य॒ज्ञ्का॑मस्य वि॒राजा ऽन्न॑काम॒स्या न्न॑कामस्य वि॒राजा॑ य॒ज्ञ्का॑मस्य य॒ज्ञ्का॑मस्य वि॒राजा ऽन्न॑कामस्य । \newline
54. य॒ज्ञ्का॑म॒स्येति॑ य॒ज्ञ् - का॒म॒स्य॒ । \newline
55. वि॒राजा ऽन्न॑काम॒स्या न्न॑कामस्य वि॒राजा॑ वि॒राजा ऽन्न॑कामस्य शृ॒णोतु॑ शृ॒णो त्वन्न॑कामस्य वि॒राजा॑ वि॒राजा ऽन्न॑कामस्य शृ॒णोतु॑ । \newline
56. वि॒राजेति॑ वि - राजा᳚ । \newline
57. अन्न॑कामस्य शृ॒णोतु॑ शृ॒णो त्वन्न॑काम॒स्या न्न॑कामस्य शृ॒णो त्व॒ग्नि र॒ग्निः शृ॒णो त्वन्न॑काम॒स्या न्न॑कामस्य शृ॒णो त्व॒ग्निः । \newline
58. अन्न॑काम॒स्येत्यन्न॑ - का॒म॒स्य॒ । \newline
59. शृ॒णो त्व॒ग्नि र॒ग्निः शृ॒णोतु॑ शृ॒णो त्व॒ग्निः स॒मिधा॑ स॒मिधा॒ ऽग्निः शृ॒णोतु॑ शृ॒णो त्व॒ग्निः स॒मिधा᳚ । \newline
60. अ॒ग्निः स॒मिधा॑ स॒मिधा॒ ऽग्नि र॒ग्निः स॒मिधा॒ हवꣳ॒॒ हवꣳ॑ स॒मिधा॒ ऽग्नि र॒ग्निः स॒मिधा॒ हव᳚म् । \newline
61. स॒मिधा॒ हवꣳ॒॒ हवꣳ॑ स॒मिधा॑ स॒मिधा॒ हव॑म् मे मे॒ हवꣳ॑ स॒मिधा॑ स॒मिधा॒ हव॑म् मे । \newline
62. स॒मिधेति॑ सं - इधा᳚ । \newline
63. हव॑म् मे मे॒ हवꣳ॒॒ हव॑म् म॒ इतीति॑ मे॒ हवꣳ॒॒ हव॑म् म॒ इति॑ । \newline
\pagebreak
\markright{ TS 6.4.3.3  \hfill https://www.vedavms.in \hfill}

\section{ TS 6.4.3.3 }

\textbf{TS 6.4.3.3 } \newline
\textbf{Samhita Paata} \newline

म॒ इत्या॑ह सवि॒तृप्र॑सूत ए॒व दे॒वता᳚भ्यो नि॒वेद्या॒पोऽच्छै᳚त्य॒प इ॑ष्य होत॒रित्या॑हेषि॒तꣳ हि कर्म॑ क्रि॒यते॒ मैत्रा॑वरुणस्य चमसाद्ध्वर्य॒वा द्र॒वेत्या॑ह मि॒त्रावरु॑णौ॒ वा अ॒पां ने॒तारौ॒ ताभ्या॑मे॒वैना॒ अच्छै॑ति॒ देवी॑रापो अपां नपा॒दित्या॒हाऽऽ*हु॑त्यै॒वैना॑ नि॒ष्क्रीय॑ गृह्णा॒त्यथो॑ ह॒विष्कृ॑ताना-मे॒वाभिघृ॑तानां गृह्णाति॒- [  ] \newline

\textbf{Pada Paata} \newline

मे॒ । इति॑ । आ॒ह॒ । स॒वि॒तृप्र॑सूत॒ इति॑ सवि॒तृ - प्र॒सू॒तः॒ । ए॒व । दे॒वता᳚भ्यः । नि॒वेद्येति॑ नि - वेद्य॑ । अ॒पः । अच्छ॑ । ए॒ति॒ । अ॒पः । इ॒ष्य॒ । हो॒तः॒ । इति॑ । आ॒ह॒ । इ॒षि॒तम् । हि । कर्म॑ । क्रि॒यते᳚ । मैत्रा॑वरुण॒स्येति॒ मैत्रा᳚-व॒रु॒ण॒स्य॒ । च॒म॒सा॒द्ध्व॒र्य॒विति॑ चमस-अ॒द्ध्व॒र्यो॒ । एति॑ । द्र॒व॒ । इति॑ । आ॒ह॒ । मि॒त्रावरु॑णा॒विति॑ मि॒त्रा - वरु॑णौ । वै । अ॒पाम् । ने॒तारौ᳚ । ताभ्या᳚म् । ए॒व । ए॒नाः॒ । अच्छ॑ । ए॒ति॒ । देवीः᳚ । आ॒पः॒ । अ॒पा॒म् । न॒पा॒त् । इति॑ । आ॒ह॒ । आह॒त्येत्या - हु॒त्या॒ । ए॒व । ए॒नाः॒ । नि॒ष्क्रीयेति॑ निः - क्रीय॑ । गृ॒ह्णा॒ति॒ । अथो॒ इति॑ । ह॒विष्कृ॑ताना॒मिति॑ ह॒विः - कृ॒ता॒ना॒म् । ए॒व । अ॒भिघृ॑ताना॒मित्य॒भि - घृ॒ता॒ना॒म् । गृ॒ह्णा॒ति॒ ।  \newline


\textbf{Krama Paata} \newline

म॒ इति॑ । इत्या॑ह । आ॒ह॒ स॒वि॒तृप्र॑सूतः । स॒वि॒तृप्रसू॑त ए॒व । स॒वि॒तृप्र॑सूत॒ इति॑ सवि॒तृ - प्र॒सू॒तः॒ । ए॒व दे॒वता᳚भ्यः । दे॒वता᳚भ्यो नि॒वेद्य॑ । नि॒वेद्या॒पः । नि॒वेद्येति॑ नि - वेद्य॑ । अ॒पोऽच्छ॑ । अच्छै॑ति । ए॒त्य॒पः । अ॒प इ॑ष्य । इ॒ष्य॒ हो॒तः॒ । हो॒त॒रिति॑ । इत्या॑ह । आ॒हे॒षि॒तम् । इ॒षि॒तꣳ हि । हि कर्म॑ । कर्म॑ क्रि॒यते᳚ । क्रि॒यते॒ मैत्रा॑वरुणस्य । मैत्रा॑वरुणस्य चमसाद्ध्वर्यो । मैत्रा॑वरुण॒स्येति॒ मैत्रा᳚ - व॒रु॒ण॒स्य॒ । च॒म॒सा॒द्ध्व॒र्य॒वा । च॒म॒सा॒द्ध्व॒र्य॒विति॑ चमस - अ॒द्ध्व॒र्यो॒ । आ द्र॑व । द्र॒वेति॑ । इत्या॑ह । आ॒ह॒ मि॒त्रावरु॑णौ । मि॒त्रावरु॑णौ॒ वै । मि॒त्रावरु॑णा॒विति॑ मि॒त्रा - वरु॑णौ । वा अ॒पाम् । अ॒पाम् ने॒तारौ᳚ । ने॒तारौ॒ ताभ्या᳚म् । ताभ्या॑मे॒व । ए॒वैनाः᳚ । ए॒ना॒ अच्छ॑ । अच्छै॑ति । ए॒ति॒ देवीः᳚ । देवी॑रापः । आ॒पो॒ अ॒पा॒म् । अ॒पा॒म् न॒पा॒त्॒ । न॒पा॒दिति॑ । इत्या॑ह । आ॒हाहु॑त्या । आहु॑त्यै॒व । आहु॒त्येत्या - हु॒त्या॒ । ए॒वैनाः᳚ । ए॒ना॒ नि॒ष्क्रीय॑ । नि॒ष्क्रीय॑ गृह्णाति । नि॒ष्क्रीयेति॑ निः - क्रीय॑ । गृ॒ह्णा॒त्यथो᳚ । अथो॑ ह॒विष्कृ॑तानाम् । अथो॒ इत्यथो᳚ । ह॒विष्कृ॑तानामे॒व । ह॒विष्कृ॑ताना॒मिति॑ ह॒विः - कृ॒ता॒ना॒म् । ए॒वाभिघृ॑तानाम् । अ॒भिघृ॑तानाम् गृह्णाति । अ॒भिघृ॑ताना॒मित्य॒भि - घृ॒ता॒ना॒म् । गृ॒ह्णा॒ति॒ कार्.षिः॑ \newline

\textbf{Jatai Paata} \newline

1. म॒ इतीति॑ मे म॒ इति॑ । \newline
2. इत्या॑हा॒हे तीत्या॑ह । \newline
3. आ॒ह॒ स॒वि॒तृप्र॑सूतः सवि॒तृप्र॑सूत आहाह सवि॒तृप्र॑सूतः । \newline
4. स॒वि॒तृप्र॑सूत ए॒वैव स॑वि॒तृप्र॑सूतः सवि॒तृप्र॑सूत ए॒व । \newline
5. स॒वि॒तृप्र॑सूत॒ इति॑ सवि॒तृ - प्र॒सू॒तः॒ । \newline
6. ए॒व दे॒वता᳚भ्यो दे॒वता᳚भ्य ए॒वैव दे॒वता᳚भ्यः । \newline
7. दे॒वता᳚भ्यो नि॒वेद्य॑ नि॒वेद्य॑ दे॒वता᳚भ्यो दे॒वता᳚भ्यो नि॒वेद्य॑ । \newline
8. नि॒वेद्या॒पो॑ ऽपो नि॒वेद्य॑ नि॒वेद्या॒पः । \newline
9. नि॒वेद्येति॑ नि - वेद्य॑ । \newline
10. अ॒पो ऽच्छा च्छा॒पो॑ ऽपो ऽच्छ॑ । \newline
11. अच्छै᳚ त्ये॒ त्यच्छा च्छै॑ति । \newline
12. ए॒त्य॒पो॑ ऽप ए᳚त्ये त्य॒पः । \newline
13. अ॒प इ॑ष्ये ष्या॒पो॑ ऽप इ॑ष्य । \newline
14. इ॒ष्य॒ हो॒त॒र्॒. हो॒त॒ रि॒ष्ये॒ ष्य॒ हो॒तः॒ । \newline
15. हो॒त॒ रितीति॑ होतर्. होत॒ रिति॑ । \newline
16. इत्या॑हा॒हे तीत्या॑ह । \newline
17. आ॒हे॒ षि॒त मि॑षि॒त मा॑हाहे षि॒तम् । \newline
18. इ॒षि॒तꣳ हि हीषि॒त मि॑षि॒तꣳ हि । \newline
19. हि कर्म॒ कर्म॒ हि हि कर्म॑ । \newline
20. कर्म॑ क्रि॒यते᳚ क्रि॒यते॒ कर्म॒ कर्म॑ क्रि॒यते᳚ । \newline
21. क्रि॒यते॒ मैत्रा॑वरुणस्य॒ मैत्रा॑वरुणस्य क्रि॒यते᳚ क्रि॒यते॒ मैत्रा॑वरुणस्य । \newline
22. मैत्रा॑वरुणस्य चमसाद्ध्वर्यो चमसाद्ध्वर्यो॒ मैत्रा॑वरुणस्य॒ मैत्रा॑वरुणस्य चमसाद्ध्वर्यो । \newline
23. मैत्रा॑वरुण॒स्येति॒ मैत्रा᳚ - व॒रु॒ण॒स्य॒ । \newline
24. च॒म॒सा॒द्ध्व॒र्य॒वा च॑मसाद्ध्वर्यो चमसाद्ध्वर्य॒वा । \newline
25. च॒म॒सा॒द्ध्व॒र्य॒विति॑ चमस - अ॒द्ध्व॒र्यो॒ । \newline
26. आ द्र॑व द्र॒वा द्र॑व । \newline
27. द्र॒वेतीति॑ द्रव द्र॒वेति॑ । \newline
28. इत्या॑हा॒हे तीत्या॑ह । \newline
29. आ॒ह॒ मि॒त्रावरु॑णौ मि॒त्रावरु॑णा वाहाह मि॒त्रावरु॑णौ । \newline
30. मि॒त्रावरु॑णौ॒ वै वै मि॒त्रावरु॑णौ मि॒त्रावरु॑णौ॒ वै । \newline
31. मि॒त्रावरु॑णा॒विति॑ मि॒त्रा - वरु॑णौ । \newline
32. वा अ॒पा म॒पां ॅवै वा अ॒पाम् । \newline
33. अ॒पाम् ने॒तारौ॑ ने॒तारा॑ व॒पा म॒पाम् ने॒तारौ᳚ । \newline
34. ने॒तारौ॒ ताभ्या॒म् ताभ्या᳚म् ने॒तारौ॑ ने॒तारौ॒ ताभ्या᳚म् । \newline
35. ताभ्या॑ मे॒वैव ताभ्या॒म् ताभ्या॑ मे॒व । \newline
36. ए॒वैना॑ एना ए॒वै वैनाः᳚ । \newline
37. ए॒ना॒ अच्छा च्छै॑ना एना॒ अच्छ॑ । \newline
38. अच्छै᳚ त्ये॒ त्यच्छा च्छै॑ति । \newline
39. ए॒ति॒ देवी॒र् देवी॑रे त्येति॒ देवीः᳚ । \newline
40. देवी॑ राप आपो॒ देवी॒र् देवी॑ रापः । \newline
41. आ॒पो॒ अ॒पा॒ म॒पा॒ मा॒प॒ आ॒पो॒ अ॒पा॒म् । \newline
42. अ॒पा॒म् न॒पा॒न् न॒पा॒ द॒पा॒ म॒पा॒म् न॒पा॒त् । \newline
43. न॒पा॒ दितीति॑ नपान् नपा॒ दिति॑ । \newline
44. इत्या॑हा॒हे तीत्या॑ह । \newline
45. आ॒हा हु॒त्या ऽऽहु॑त्या ऽऽहा॒हा हु॑त्या । \newline
46. आहु॑ त्यै॒वैवा हु॒त्या ऽऽहु॑त्यै॒व । \newline
47. आहु॒त्येत्या - हु॒त्या॒ । \newline
48. ए॒वैना॑ एना ए॒वै वैनाः᳚ । \newline
49. ए॒ना॒ नि॒ष्क्रीय॑ नि॒ष्क्रीयै॑ना एना नि॒ष्क्रीय॑ । \newline
50. नि॒ष्क्रीय॑ गृह्णाति गृह्णाति नि॒ष्क्रीय॑ नि॒ष्क्रीय॑ गृह्णाति । \newline
51. नि॒ष्क्रीयेति॑ निः - क्रीय॑ । \newline
52. गृ॒ह्णा॒ त्यथो॒ अथो॑ गृह्णाति गृह्णा॒ त्यथो᳚ । \newline
53. अथो॑ ह॒विष्कृ॑तानाꣳ ह॒विष्कृ॑ताना॒ मथो॒ अथो॑ ह॒विष्कृ॑तानाम् । \newline
54. अथो॒ इत्यथो᳚ । \newline
55. ह॒विष्कृ॑ताना मे॒वैव ह॒विष्कृ॑तानाꣳ ह॒विष्कृ॑ताना मे॒व । \newline
56. ह॒विष्कृ॑ताना॒मिति॑ ह॒विः - कृ॒ता॒ना॒म् । \newline
57. ए॒वा भिघृ॑ताना म॒भिघृ॑ताना मे॒वैवा भिघृ॑तानाम् । \newline
58. अ॒भिघृ॑तानाम् गृह्णाति गृह्णा त्य॒भिघृ॑ताना म॒भिघृ॑तानाम् गृह्णाति । \newline
59. अ॒भिघृ॑ताना॒मित्य॒भि - घृ॒ता॒ना॒म् । \newline
60. गृ॒ह्णा॒ति॒ कार्.षिः॒ कार्.षि॑र् गृह्णाति गृह्णाति॒ कार्.षिः॑ । \newline

\textbf{Ghana Paata } \newline

1. म॒ इतीति॑ मे म॒ इत्या॑हा॒ हेति॑ मे म॒ इत्या॑ह । \newline
2. इत्या॑हा॒हे तीत्या॑ह सवि॒तृप्र॑सूतः सवि॒तृप्र॑सूत आ॒हे तीत्या॑ह सवि॒तृप्र॑सूतः । \newline
3. आ॒ह॒ स॒वि॒तृप्र॑सूतः सवि॒तृप्र॑सूत आहाह सवि॒तृप्र॑सूत ए॒वैव स॑वि॒तृप्र॑सूत आहाह सवि॒तृप्र॑सूत ए॒व । \newline
4. स॒वि॒तृप्र॑सूत ए॒वैव स॑वि॒तृप्र॑सूतः सवि॒तृप्र॑सूत ए॒व दे॒वता᳚भ्यो दे॒वता᳚भ्य ए॒व स॑वि॒तृप्र॑सूतः सवि॒तृप्र॑सूत ए॒व दे॒वता᳚भ्यः । \newline
5. स॒वि॒तृप्र॑सूत॒ इति॑ सवि॒तृ - प्र॒सू॒तः॒ । \newline
6. ए॒व दे॒वता᳚भ्यो दे॒वता᳚भ्य ए॒वैव दे॒वता᳚भ्यो नि॒वेद्य॑ नि॒वेद्य॑ दे॒वता᳚भ्य ए॒वैव दे॒वता᳚भ्यो नि॒वेद्य॑ । \newline
7. दे॒वता᳚भ्यो नि॒वेद्य॑ नि॒वेद्य॑ दे॒वता᳚भ्यो दे॒वता᳚भ्यो नि॒वे द्या॒पो॑ ऽपो नि॒वेद्य॑ दे॒वता᳚भ्यो दे॒वता᳚भ्यो नि॒वे द्या॒पः । \newline
8. नि॒वे द्या॒पो॑ ऽपो नि॒वेद्य॑ नि॒वे द्या॒पो ऽच्छा च्छा॒पो नि॒वेद्य॑ नि॒वे द्या॒पो ऽच्छ॑ । \newline
9. नि॒वेद्येति॑ नि - वेद्य॑ । \newline
10. अ॒पो ऽच्छा च्छा॒पो॑ ऽपो ऽच्छै᳚ त्ये॒ त्यच्छा॒पो॑ ऽपो ऽच्छै॑ति । \newline
11. अच्छै᳚ त्ये॒ त्यच्छा च्छै᳚ त्य॒पो॑ ऽप ए॒त्यच्छाच् छै᳚त्य॒पः । \newline
12. ए॒त्य॒पो॑ ऽप ए᳚त्ये त्य॒प इ॑ष्ये ष्या॒प ए᳚त्ये त्य॒प इ॑ष्य । \newline
13. अ॒प इ॑ष्ये ष्या॒पो॑ ऽप इ॑ष्य होतर्. होत रिष्या॒पो॑ ऽप इ॑ष्य होतः । \newline
14. इ॒ष्य॒ हो॒त॒र्॒. हो॒त॒ रि॒ष्ये॒ ष्य॒ हो॒त॒ रितीति॑ होत रिष्ये ष्य होत॒ रिति॑ । \newline
15. हो॒त॒ रितीति॑ होतर्. होत॒ रित्या॑हा॒ हेति॑ होतर्. होत॒ रित्या॑ह । \newline
16. इत्या॑हा॒हे तीत्या॑हेषि॒त मि॑षि॒त मा॒हे तीत्या॑हेषि॒तम् । \newline
17. आ॒हे॒षि॒त मि॑षि॒त मा॑हाहे षि॒तꣳ हि हीषि॒त मा॑हा हेषि॒तꣳ हि । \newline
18. इ॒षि॒तꣳ हि हीषि॒त मि॑षि॒तꣳ हि कर्म॒ कर्म॒ हीषि॒त मि॑षि॒तꣳ हि कर्म॑ । \newline
19. हि कर्म॒ कर्म॒ हि हि कर्म॑ क्रि॒यते᳚ क्रि॒यते॒ कर्म॒ हि हि कर्म॑ क्रि॒यते᳚ । \newline
20. कर्म॑ क्रि॒यते᳚ क्रि॒यते॒ कर्म॒ कर्म॑ क्रि॒यते॒ मैत्रा॑वरुणस्य॒ मैत्रा॑वरुणस्य क्रि॒यते॒ कर्म॒ कर्म॑ क्रि॒यते॒ मैत्रा॑वरुणस्य । \newline
21. क्रि॒यते॒ मैत्रा॑वरुणस्य॒ मैत्रा॑वरुणस्य क्रि॒यते᳚ क्रि॒यते॒ मैत्रा॑वरुणस्य चमसाद्ध्वर्यो चमसाद्ध्वर्यो॒ मैत्रा॑वरुणस्य क्रि॒यते᳚ क्रि॒यते॒ मैत्रा॑वरुणस्य चमसाद्ध्वर्यो । \newline
22. मैत्रा॑वरुणस्य चमसाद्ध्वर्यो चमसाद्ध्वर्यो॒ मैत्रा॑वरुणस्य॒ मैत्रा॑वरुणस्य चमसाद्ध्वर्य॒वा च॑मसाद्ध्वर्यो॒ मैत्रा॑वरुणस्य॒ मैत्रा॑वरुणस्य चमसाद्ध्वर्य॒वा । \newline
23. मैत्रा॑वरुण॒स्येति॒ मैत्रा᳚ - व॒रु॒ण॒स्य॒ । \newline
24. च॒म॒सा॒द्ध्व॒र्य॒वा च॑मसाद्ध्वर्यो चमसाद्ध्वर्य॒वा द्र॑व द्र॒वा च॑मसाद्ध्वर्यो चमसाद्ध्वर्य॒वा द्र॑व । \newline
25. च॒म॒सा॒द्ध्व॒र्य॒विति॑ चमस - अ॒द्ध्व॒र्यो॒ । \newline
26. आ द्र॑व द्र॒वा द्र॒वे तीति॑ द्र॒वा द्र॒वेति॑ । \newline
27. द्र॒वे तीति॑ द्रव द्र॒वे त्या॑हा॒ हेति॑ द्रव द्र॒वे त्या॑ह । \newline
28. इत्या॑हा॒हे तीत्या॑ह मि॒त्रावरु॑णौ मि॒त्रावरु॑णा वा॒हे तीत्या॑ह मि॒त्रावरु॑णौ । \newline
29. आ॒ह॒ मि॒त्रावरु॑णौ मि॒त्रावरु॑णा वाहाह मि॒त्रावरु॑णौ॒ वै वै मि॒त्रावरु॑णा वाहाह मि॒त्रावरु॑णौ॒ वै । \newline
30. मि॒त्रावरु॑णौ॒ वै वै मि॒त्रावरु॑णौ मि॒त्रावरु॑णौ॒ वा अ॒पा म॒पां ॅवै मि॒त्रावरु॑णौ मि॒त्रावरु॑णौ॒ वा अ॒पाम् । \newline
31. मि॒त्रावरु॑णा॒विति॑ मि॒त्रा - वरु॑णौ । \newline
32. वा अ॒पा म॒पां ॅवै वा अ॒पाम् ने॒तारौ॑ ने॒तारा॑ व॒पां ॅवै वा अ॒पाम् ने॒तारौ᳚ । \newline
33. अ॒पाम् ने॒तारौ॑ ने॒तारा॑ व॒पा म॒पाम् ने॒तारौ॒ ताभ्या॒म् ताभ्या᳚म् ने॒तारा॑ व॒पा म॒पाम् ने॒तारौ॒ ताभ्या᳚म् । \newline
34. ने॒तारौ॒ ताभ्या॒म् ताभ्या᳚म् ने॒तारौ॑ ने॒तारौ॒ ताभ्या॑ मे॒वैव ताभ्या᳚म् ने॒तारौ॑ ने॒तारौ॒ ताभ्या॑ मे॒व । \newline
35. ताभ्या॑ मे॒वैव ताभ्या॒म् ताभ्या॑ मे॒वैना॑ एना ए॒व ताभ्या॒म् ताभ्या॑ मे॒वैनाः᳚ । \newline
36. ए॒वैना॑ एना ए॒वै वैना॒ अच्छा च्छै॑ना ए॒वै वैना॒ अच्छ॑ । \newline
37. ए॒ना॒ अच्छा च्छै॑ना एना॒ अच्छै᳚त्ये॒ त्यच्छै॑ना एना॒ अच्छै॑ति । \newline
38. अच्छै᳚त्ये॒ त्यच्छा च्छै॑ति॒ देवी॒र् देवी॑ रे॒त्य च्छा च्छै॑ति॒ देवीः᳚ । \newline
39. ए॒ति॒ देवी॒र् देवी॑ रेत्येति॒ देवी॑राप आपो॒ देवी॑ रेत्येति॒ देवी॑रापः । \newline
40. देवी॑राप आपो॒ देवी॒र् देवी॑रापो अपा मपा मापो॒ देवी॒र् देवी॑रापो अपाम् । \newline
41. आ॒पो॒ अ॒पा॒ म॒पा॒ मा॒प॒ आ॒पो॒ अ॒पा॒म् न॒पा॒न् न॒पा॒ द॒पा॒ मा॒प॒ आ॒पो॒ अ॒पा॒म् न॒पा॒त् । \newline
42. अ॒पा॒म् न॒पा॒न् न॒पा॒ द॒पा॒ म॒पा॒म् न॒पा॒दि तीति॑ नपा दपा मपाम् नपा॒ दिति॑ । \newline
43. न॒पा॒दि तीति॑ नपान् नपा॒ दित्या॑हा॒ हेति॑ नपान् नपा॒ दित्या॑ह । \newline
44. इत्या॑हा॒हे तीत्या॒हा हु॒त्या ऽऽहु॑त्या॒ ऽऽहे तीत्या॒हा हु॑त्या । \newline
45. आ॒हा हु॒त्या ऽऽहु॑त्या ऽऽहा॒हा हु॑त्यै॒ वैवा हु॑त्या ऽऽहा॒हा हु॑त्यै॒व । \newline
46. आहु॑त्यै॒ वैवा हु॒त्या ऽऽहु॑त्यै॒ वैना॑ एना ए॒वा हु॒त्या ऽऽहु॑त्यै॒ वैनाः᳚ । \newline
47. आहु॒त्येत्या - हु॒त्या॒ । \newline
48. ए॒वैना॑ एना ए॒वै वैना॑ नि॒ष्क्रीय॑ नि॒ष्क्री यै॑ना ए॒वै वैना॑ नि॒ष्क्रीय॑ । \newline
49. ए॒ना॒ नि॒ष्क्रीय॑ नि॒ष्क्री यै॑ना एना नि॒ष्क्रीय॑ गृह्णाति गृह्णाति नि॒ष्क्री यै॑ना एना नि॒ष्क्रीय॑ गृह्णाति । \newline
50. नि॒ष्क्रीय॑ गृह्णाति गृह्णाति नि॒ष्क्रीय॑ नि॒ष्क्रीय॑ गृह्णा॒ त्यथो॒ अथो॑ गृह्णाति नि॒ष्क्रीय॑ नि॒ष्क्रीय॑ गृह्णा॒ त्यथो᳚ । \newline
51. नि॒ष्क्रीयेति॑ निः - क्रीय॑ । \newline
52. गृ॒ह्णा॒ त्यथो॒ अथो॑ गृह्णाति गृह्णा॒ त्यथो॑ ह॒विष्कृ॑तानाꣳ ह॒विष्कृ॑ताना॒ मथो॑ गृह्णाति गृह्णा॒त्यथो॑ ह॒विष्कृ॑तानाम् । \newline
53. अथो॑ ह॒विष्कृ॑तानाꣳ ह॒विष्कृ॑ताना॒ मथो॒ अथो॑ ह॒विष्कृ॑ताना मे॒वैव ह॒विष्कृ॑ताना॒ मथो॒ अथो॑ ह॒विष्कृ॑ताना मे॒व । \newline
54. अथो॒ इत्यथो᳚ । \newline
55. ह॒विष्कृ॑ताना मे॒वैव ह॒विष्कृ॑तानाꣳ ह॒विष्कृ॑ताना मे॒वाभिघृ॑ताना म॒भिघृ॑ताना मे॒व ह॒विष्कृ॑तानाꣳ ह॒विष्कृ॑ताना मे॒वाभिघृ॑तानाम् । \newline
56. ह॒विष्कृ॑ताना॒मिति॑ ह॒विः - कृ॒ता॒ना॒म् । \newline
57. ए॒वाभिघृ॑ताना म॒भिघृ॑ताना मे॒वैवा भिघृ॑तानाम् गृह्णाति गृह्णा त्य॒भिघृ॑ताना मे॒वैवा भिघृ॑तानाम् गृह्णाति । \newline
58. अ॒भिघृ॑तानाम् गृह्णाति गृह्णा त्य॒भिघृ॑ताना म॒भिघृ॑तानाम् गृह्णाति॒ कार्.षिः॒ कार्.षि॑र् गृह्णा त्य॒भिघृ॑ताना म॒भिघृ॑तानाम् गृह्णाति॒ कार्.षिः॑ । \newline
59. अ॒भिघृ॑ताना॒मित्य॒भि - घृ॒ता॒ना॒म् । \newline
60. गृ॒ह्णा॒ति॒ कार्.षिः॒ कार्.षि॑र् गृह्णाति गृह्णाति॒ कार्.षि॑ रस्यसि॒ कार्.षि॑र् गृह्णाति गृह्णाति॒ कार्.षि॑ रसि । \newline
\pagebreak
\markright{ TS 6.4.3.4  \hfill https://www.vedavms.in \hfill}

\section{ TS 6.4.3.4 }

\textbf{TS 6.4.3.4 } \newline
\textbf{Samhita Paata} \newline

कार्.षि॑र॒सीत्या॑ह॒ शम॑लमे॒वाऽऽ*सा॒मप॑ प्लावयति समु॒द्रस्य॒ वोऽक्षि॑त्या॒ उन्न॑य॒ इत्या॑ह॒ तस्मा॑द॒द्यमा॑नाः पी॒यमा॑ना॒ आपो॒ न क्षी॑यन्ते॒ योनि॒र्वै य॒ज्ञ्स्य॒ चात्वा॑लं ॅय॒ज्ञो व॑सती॒वरीर्॑.होतृचम॒सं च॑ मैत्रावरुणचम॒सं च॑ सꣳ॒॒स्पर्श्य॑ वसती॒वरी॒र्व्यान॑यति य॒ज्ञ्स्य॑ सयोनि॒त्वायाथो॒ स्वादे॒वैना॒ योनेः॒ प्र ज॑नय॒त्यद्ध्व॒र्योऽवे॑र॒पा(3) इत्या॑हो॒ते ( ) -म॑नन्नमुरु॒तेमाः प॒श्येति॒ वावैतदा॑ह॒ यद्य॑ग्निष्टो॒मो जु॒होति॒ यद् यु॒क्थ्यः॑ परि॒धौ नि मा᳚र्ष्टि॒ यद् य॑तिरा॒त्रो यजु॒र्वद॒न् प्र प॑द्यते यज्ञ्क्रतू॒नां ॅव्यावृ॑त्त्यै ॥ \newline

\textbf{Pada Paata} \newline

कार्.षिः॑ । अ॒सि॒ । इति॑ । आ॒ह॒ । शम॑लम् । ए॒व । आ॒सा॒म् । अपेति॑ । प्ला॒व॒य॒ति॒ । स॒मु॒द्रस्य॑ । वः॒ । अक्षि॑त्यै । उदिति॑ । न॒ये॒ । इति॑ । आ॒ह॒ । तस्मा᳚त् । अ॒द्यमा॑नाः । पी॒यमा॑नाः । आपः॑ । न । क्षी॒य॒न्ते॒ । योनिः॑ । वै । य॒ज्ञ्स्य॑ । चात्वा॑लम् । य॒ज्ञ्ः । व॒स॒ती॒वरीः᳚ । हो॒तृ॒च॒म॒समिति॑ होतृ - च॒म॒सम् । च॒ । मै॒त्रा॒व॒रु॒ण॒च॒म॒समिति॑ मैत्रावरुण - च॒म॒सम् । च॒ । सꣳ॒॒स्पर्श्येति॑ सं - स्पर्श्य॑ । व॒स॒ती॒वरीः᳚ । व्यान॑य॒तीति॑ वि - आन॑यति । य॒ज्ञ्स्य॑ । स॒यो॒नि॒त्वायेति॑ सयोनि - त्वाय॑ । अथो॒ इति॑ । स्वात् । ए॒व । ए॒नाः॒ । योनेः᳚ । प्रेति॑ । ज॒न॒य॒ति॒ । अद्ध्व॑र्यो॒ इति॑ । अवेः᳚ । अ॒पा(3)ः । इति॑ । आ॒ह॒ । उ॒त ( ) । ई॒म् । अ॒न॒न्न॒मुः॒ । उ॒त । इ॒माः । प॒श्य॒ । इति॑ । वाव । ए॒तत् । आ॒ह॒ । यदि॑ । अ॒ग्नि॒ष्टो॒म इत्य॑ग्नि - स्तो॒मः । जु॒होति॑ । यदि॑ । उ॒क्थ्यः॑ । प॒रि॒धाविति॑ परि - धौ । नीति॑ । मा॒र्ष्टि॒ । यदि॑ । अ॒ति॒रा॒त्र इत्य॑ति - रा॒त्रः । यजुः॑ । वदन्न्॑ । प्रेति॑ । प॒द्य॒ते॒ । य॒ज्ञ्॒क्र॒तू॒नामिति॑ यज्ञ् - क्र॒तू॒नाम् । व्यावृ॑त्त्या॒ इति॑ वि - आवृ॑त्त्यै ॥  \newline


\textbf{Krama Paata} \newline

कार्.षि॑रसि । अ॒सीति॑ । इत्या॑ह । आ॒ह॒ शम॑लम् । शम॑लमे॒व । ए॒वासा᳚म् । आ॒सा॒मप॑ । अप॑ प्लावयति । प्ला॒व॒य॒ति॒ स॒मु॒द्रस्य॑ । स॒मु॒द्रस्य॑ वः । वोऽक्षि॑त्यै । अक्षि॑त्या॒ उत् । उन् न॑ये । न॒य॒ इति॑ । इत्या॑ह । आ॒ह॒ तस्मा᳚त् । तस्मा॑द॒द्यमा॑नाः । अ॒द्यमा॑नाः पी॒यमा॑नाः । पी॒यमा॑ना॒ आपः॑ । आपो॒ न । न क्षी॑यन्ते । क्षी॒य॒न्ते॒ योनिः॑ । योनि॒र् वै । वै य॒ज्ञ्स्य॑ । य॒ज्ञ्स्य॒ चात्वा॑लम् । चात्वा॑लम् ॅय॒ज्ञ्ः । य॒ज्ञो व॑सती॒वरीः᳚ । व॒स॒ती॒वरी॑र्. होतृचम॒सम् । हो॒तृ॒च॒म॒सम् च॑ । हो॒तृ॒च॒म॒समिति॑ होतृ - च॒म॒सम् । च॒ मै॒त्रा॒व॒रु॒ण॒च॒म॒सम् । मै॒त्रा॒व॒रु॒ण॒च॒म॒सम् च॑ । मै॒त्रा॒व॒रु॒ण॒च॒म॒समिति॑ मैत्रावरुण - च॒म॒सम् । च॒ सꣳ॒॒स्पर्श्य॑ । सꣳ॒॒स्पर्श्य॑ वसती॒वरीः᳚ । सꣳ॒॒स्पर्श्येति॑ सम् - स्पर्श्य॑ । व॒स॒ती॒वरी॒र् व्यान॑यति । व्यान॑यति य॒ज्ञ्स्य॑ । व्यान॑य॒तीति॑ वि - आन॑यति । य॒ज्ञ्स्य॑ सयोनि॒त्वाय॑ । स॒यो॒नि॒त्वायाथो᳚ । स॒यो॒नि॒त्वायेति॑ सयोनि - त्वाय॑ । अथो॒ स्वात् । अथो॒ इत्यथो᳚ । स्वादे॒व । ए॒वैनाः᳚ । ए॒ना॒ योनेः᳚ । योनेः॒ प्र । प्र ज॑नयति । ज॒न॒य॒त्यद्ध्व॑र्यो । अद्ध्व॒र्योऽवेः᳚ । अद्ध्व॑र्यो॒ इत्यद्ध्व॑र्यो । अवे॑र॒पा(3)ः । अ॒पा(3) इति॑ । इत्या॑ह । आ॒हो॒त ( ) । उ॒तेम् । ई॒म॒न॒न्न॒मुः॒ । अ॒न॒न्न॒मु॒रु॒त । उ॒तेमाः । इ॒माः प॑श्य । प॒श्येति॑ । इति॒ वाव । वावैतत् । ए॒तदा॑ह । आ॒ह॒ यदि॑ । यद्य॑ग्निष्टो॒मः । अ॒ग्नि॒ष्टो॒मो जु॒होति॑ । अ॒ग्नि॒ष्टो॒म इत्य॑ग्नि - स्तो॒मः । जु॒होति॒ यदि॑ । यद्यु॒क्थ्यः॑ । उ॒क्थ्यः॑ परि॒धौ । प॒रि॒धौ नि । प॒रि॒धाविति॑ परि - धौ । नि मा᳚र्ष्टि । मा॒र्ष्टि॒ यदि॑ । यद्य॑तिरा॒त्रः । अ॒ति॒रा॒त्रो यजुः॑ । अ॒ति॒रा॒त्र इत्य॑ति - रा॒त्रः । यजु॒र् वदन्न्॑ । वद॒न् प्र । प्र प॑द्यते । प॒द्य॒ते॒ य॒ज्ञ्॒क्र॒तू॒नाम् । य॒ज्ञ्॒क्र॒तू॒नाम् ॅव्यावृ॑त्त्यै । य॒ज्ञ्॒क्र॒तू॒नामिति॑ यज्ञ् - क्र॒तू॒नाम् । व्यावृ॑त्त्या॒ इति॑ वि - आवृ॑त्त्यै । \newline

\textbf{Jatai Paata} \newline

1. कार्.षि॑ रस्यसि॒ कार्.षिः॒ कार्.षि॑ रसि । \newline
2. अ॒सीती त्य॑स्य॒ सीति॑ । \newline
3. इत्या॑हा॒हे तीत्या॑ह । \newline
4. आ॒ह॒ शम॑लꣳ॒॒ शम॑ल माहाह॒ शम॑लम् । \newline
5. शम॑ल मे॒वैव शम॑लꣳ॒॒ शम॑ल मे॒व । \newline
6. ए॒वासा॑ मासा मे॒वै वासा᳚म् । \newline
7. आ॒सा॒ मपा पा॑सा मासा॒ मप॑ । \newline
8. अप॑ प्लावयति प्लावय॒ त्यपाप॑ प्लावयति । \newline
9. प्ला॒व॒य॒ति॒ स॒मु॒द्रस्य॑ समु॒द्रस्य॑ प्लावयति प्लावयति समु॒द्रस्य॑ । \newline
10. स॒मु॒द्रस्य॑ वो वः समु॒द्रस्य॑ समु॒द्रस्य॑ वः । \newline
11. वो ऽक्षि॑त्या॒ अक्षि॑त्यै वो॒ वो ऽक्षि॑त्यै । \newline
12. अक्षि॑त्या॒ उदु दक्षि॑त्या॒ अक्षि॑त्या॒ उत् । \newline
13. उन् न॑ये नय॒ उदुन् न॑ये । \newline
14. न॒य॒ इतीति॑ नये नय॒ इति॑ । \newline
15. इत्या॑हा॒हे तीत्या॑ह । \newline
16. आ॒ह॒ तस्मा॒त् तस्मा॑ दाहाह॒ तस्मा᳚त् । \newline
17. तस्मा॑ द॒द्यमा॑ना अ॒द्यमा॑ना॒ स्तस्मा॒त् तस्मा॑ द॒द्यमा॑नाः । \newline
18. अ॒द्यमा॑नाः पी॒यमा॑नाः पी॒यमा॑ना अ॒द्यमा॑ना अ॒द्यमा॑नाः पी॒यमा॑नाः । \newline
19. पी॒यमा॑ना॒ आप॒ आपः॑ पी॒यमा॑नाः पी॒यमा॑ना॒ आपः॑ । \newline
20. आपो॒ न नाप॒ आपो॒ न । \newline
21. न क्षी॑यन्ते क्षीयन्ते॒ न न क्षी॑यन्ते । \newline
22. क्षी॒य॒न्ते॒ योनि॒र् योनिः॑ क्षीयन्ते क्षीयन्ते॒ योनिः॑ । \newline
23. योनि॒र् वै वै योनि॒र् योनि॒र् वै । \newline
24. वै य॒ज्ञ्स्य॑ य॒ज्ञ्स्य॒ वै वै य॒ज्ञ्स्य॑ । \newline
25. य॒ज्ञ्स्य॒ चात्वा॑ल॒म् चात्वा॑लं ॅय॒ज्ञ्स्य॑ य॒ज्ञ्स्य॒ चात्वा॑लम् । \newline
26. चात्वा॑लं ॅय॒ज्ञो य॒ज्ञ् श्चात्वा॑ल॒म् चात्वा॑लं ॅय॒ज्ञ्ः । \newline
27. य॒ज्ञो व॑सती॒वरी᳚र् वसती॒वरी᳚र् य॒ज्ञो य॒ज्ञो व॑सती॒वरीः᳚ । \newline
28. व॒स॒ती॒वरीर्॑. होतृचम॒सꣳ हो॑तृचम॒सं ॅव॑सती॒वरी᳚र् वसती॒वरीर्॑. होतृचम॒सम् । \newline
29. हो॒तृ॒च॒म॒सम् च॑ च होतृचम॒सꣳ हो॑तृचम॒सम् च॑ । \newline
30. हो॒तृ॒च॒म॒समिति॑ होतृ - च॒म॒सम् । \newline
31. च॒ मै॒त्रा॒व॒रु॒ण॒च॒म॒सम् मै᳚त्रावरुणचम॒सम् च॑ च मैत्रावरुणचम॒सम् । \newline
32. मै॒त्रा॒व॒रु॒ण॒च॒म॒सम् च॑ च मैत्रावरुणचम॒सम् मै᳚त्रावरुणचम॒सम् च॑ । \newline
33. मै॒त्रा॒व॒रु॒ण॒च॒म॒समिति॑ मैत्रावरुण - च॒म॒सम् । \newline
34. च॒ सꣳ॒॒स्पर्श्य॑ सꣳ॒॒स्पर्श्य॑ च च सꣳ॒॒स्पर्श्य॑ । \newline
35. सꣳ॒॒स्पर्श्य॑ वसती॒वरी᳚र् वसती॒वरीः᳚ सꣳ॒॒स्पर्श्य॑ सꣳ॒॒स्पर्श्य॑ वसती॒वरीः᳚ । \newline
36. सꣳ॒॒स्पर्श्येति॑ सं - स्पर्श्य॑ । \newline
37. व॒स॒ती॒वरी॒र् व्यान॑यति॒ व्यान॑यति वसती॒वरी᳚र् वसती॒वरी॒र् व्यान॑यति । \newline
38. व्यान॑यति य॒ज्ञ्स्य॑ य॒ज्ञ्स्य॒ व्यान॑यति॒ व्यान॑यति य॒ज्ञ्स्य॑ । \newline
39. व्यान॑य॒तीति॑ वि - आन॑यति । \newline
40. य॒ज्ञ्स्य॑ सयोनि॒त्वाय॑ सयोनि॒त्वाय॑ य॒ज्ञ्स्य॑ य॒ज्ञ्स्य॑ सयोनि॒त्वाय॑ । \newline
41. स॒यो॒नि॒त्वा याथो॒ अथो॑ सयोनि॒त्वाय॑ सयोनि॒त्वा याथो᳚ । \newline
42. स॒यो॒नि॒त्वायेति॑ सयोनि - त्वाय॑ । \newline
43. अथो॒ स्वाथ् स्वा दथो॒ अथो॒ स्वात् । \newline
44. अथो॒ इत्यथो᳚ । \newline
45. स्वा दे॒वैव स्वाथ् स्वा दे॒व । \newline
46. ए॒वैना॑ एना ए॒वै वैनाः᳚ । \newline
47. ए॒ना॒ योने॒र् योने॑ रेना एना॒ योनेः᳚ । \newline
48. योनेः॒ प्र प्र योने॒र् योनेः॒ प्र । \newline
49. प्र ज॑नयति जनयति॒ प्र प्र ज॑नयति । \newline
50. ज॒न॒य॒ त्यद्ध्व॒र्यो ऽद्ध्व॑र्यो जनयति जनय॒ त्यद्ध्व॑र्यो । \newline
51. अद्ध्व॒र्यो ऽवे॒ रवे॒ रद्ध्व॒र्यो ऽद्ध्व॒र्यो ऽवेः᳚ । \newline
52. अद्ध्व॑र्यो॒ इत्यद्ध्व॑र्यो । \newline
53. अवे॑ र॒पा(3) अ॒पा(3) अवे॒ रवे॑ र॒पा(3)ः । \newline
54. अ॒पा(3) इतीत्य॒पा(3) अ॒पा(3) इति॑ । \newline
55. इत्या॑हा॒हे तीत्या॑ह । \newline
56. आ॒हो॒ तोताहा॑ हो॒त । \newline
57. उ॒तेमी॑ मु॒तोतेम् । \newline
58. ई॒ म॒न॒न्न॒मु॒ र॒न॒न्न॒मु॒री॒ मी॒ म॒न॒न्न॒मुः॒ । \newline
59. अ॒न॒न्न॒मु॒ रु॒तो तान॑न्नमु रनन्नमु रु॒त । \newline
60. उ॒ते मा इ॒मा उ॒तोते माः । \newline
61. इ॒माः प॑श्य पश्ये॒ मा इ॒माः प॑श्य । \newline
62. प॒श्येतीति॑ पश्य प॒श्येति॑ । \newline
63. इति॒ वाव वावेतीति॒ वाव । \newline
64. वावैत दे॒तद् वाव वावैतत् । \newline
65. ए॒त दा॑हा है॒त दे॒त दा॑ह । \newline
66. आ॒ह॒ यदि॒ यद्या॑ हाह॒ यदि॑ । \newline
67. यद्य॑ग्निष्टो॒मो᳚ ऽग्निष्टो॒मो यदि॒ यद्य॑ग्निष्टो॒मः । \newline
68. अ॒ग्नि॒ष्टो॒मो जु॒होति॑ जु॒हो त्य॑ग्निष्टो॒मो᳚ ऽग्निष्टो॒मो जु॒होति॑ । \newline
69. अ॒ग्नि॒ष्टो॒म इत्य॑ग्नि - स्तो॒मः । \newline
70. जु॒होति॒ यदि॒ यदि॑ जु॒होति॑ जु॒होति॒ यदि॑ । \newline
71. यद्यु॒क्थ्य॑ उ॒क्थ्यो॑ यदि॒ यद्यु॒क्थ्यः॑ । \newline
72. उ॒क्थ्यः॑ परि॒धौ प॑रि॒धा वु॒क्थ्य॑ उ॒क्थ्यः॑ परि॒धौ । \newline
73. प॒रि॒धौ नि नि प॑रि॒धौ प॑रि॒धौ नि । \newline
74. प॒रि॒धाविति॑ परि - धौ । \newline
75. नि मा᳚र्ष्टि मार्ष्टि॒ नि नि मा᳚र्ष्टि । \newline
76. मा॒र्ष्टि॒ यदि॒ यदि॑ मार्ष्टि मार्ष्टि॒ यदि॑ । \newline
77. यद्य॑तिरा॒त्रो॑ ऽतिरा॒त्रो यदि॒ यद्य॑तिरा॒त्रः । \newline
78. अ॒ति॒रा॒त्रो यजु॒र् यजु॑ रतिरा॒त्रो॑ ऽतिरा॒त्रो यजुः॑ । \newline
79. अ॒ति॒रा॒त्र इत्य॑ति - रा॒त्रः । \newline
80. यजु॒र् वद॒न्॒. वद॒न्॒. यजु॒र् यजु॒र् वदन्न्॑ । \newline
81. वद॒न् प्र प्र वद॒न्॒. वद॒न् प्र । \newline
82. प्र प॑द्यते पद्यते॒ प्र प्र प॑द्यते । \newline
83. प॒द्य॒ते॒ य॒ज्ञ्॒क्र॒तू॒नां ॅय॑ज्ञ्क्रतू॒नाम् प॑द्यते पद्यते यज्ञ्क्रतू॒नाम् । \newline
84. य॒ज्ञ्॒क्र॒तू॒नां ॅव्यावृ॑त्त्यै॒ व्यावृ॑त्त्यै यज्ञ्क्रतू॒नां ॅय॑ज्ञ्क्रतू॒नां ॅव्यावृ॑त्त्यै । \newline
85. य॒ज्ञ्॒क्र॒तू॒नामिति॑ यज्ञ् - क्र॒तू॒नाम् । \newline
86. व्यावृ॑त्त्या॒ इति॑ वि - आवृ॑त्त्यै । \newline

\textbf{Ghana Paata } \newline

1. कार्.षि॑ रस्यसि॒ कार्.षिः॒ कार्.षि॑ र॒सी तीत्य॑सि॒ कार्.षिः॒ कार्.षि॑ र॒सीति॑ । \newline
2. अ॒सीती त्य॑स्य॒ सीत्या॑हा॒हे त्य॑स्य॒ सीत्या॑ह । \newline
3. इत्या॑हा॒हे तीत्या॑ह॒ शम॑लꣳ॒॒ शम॑ल मा॒हे तीत्या॑ह॒ शम॑लम् । \newline
4. आ॒ह॒ शम॑लꣳ॒॒ शम॑ल माहाह॒ शम॑ल मे॒वैव शम॑ल माहाह॒ शम॑ल मे॒व । \newline
5. शम॑ल मे॒वैव शम॑लꣳ॒॒ शम॑ल मे॒वासा॑ मासा मे॒व शम॑लꣳ॒॒ शम॑ल मे॒वासा᳚म् । \newline
6. ए॒वासा॑ मासा मे॒वै वासा॒ मपा पा॑सा मे॒वै वासा॒ मप॑ । \newline
7. आ॒सा॒ मपापा॑सा मासा॒ मप॑ प्लावयति प्लावय॒ त्यपा॑सा मासा॒ मप॑ प्लावयति । \newline
8. अप॑ प्लावयति प्लावय॒ त्यपाप॑ प्लावयति समु॒द्रस्य॑ समु॒द्रस्य॑ प्लावय॒ त्यपाप॑ प्लावयति समु॒द्रस्य॑ । \newline
9. प्ला॒व॒य॒ति॒ स॒मु॒द्रस्य॑ समु॒द्रस्य॑ प्लावयति प्लावयति समु॒द्रस्य॑ वो वः समु॒द्रस्य॑ प्लावयति प्लावयति समु॒द्रस्य॑ वः । \newline
10. स॒मु॒द्रस्य॑ वो वः समु॒द्रस्य॑ समु॒द्रस्य॒ वो ऽक्षि॑त्या॒ अक्षि॑त्यै वः समु॒द्रस्य॑ समु॒द्रस्य॒ वो ऽक्षि॑त्यै । \newline
11. वो ऽक्षि॑त्या॒ अक्षि॑त्यै वो॒ वो ऽक्षि॑त्या॒ उदुदक्षि॑त्यै वो॒ वो ऽक्षि॑त्या॒ उत् । \newline
12. अक्षि॑त्या॒ उदुदक्षि॑त्या॒ अक्षि॑त्या॒ उन् न॑ये नय॒ उदक्षि॑त्या॒ अक्षि॑त्या॒ उन् न॑ये । \newline
13. उन् न॑ये नय॒ उदुन् न॑य॒ इतीति॑ नय॒ उदुन् न॑य॒ इति॑ । \newline
14. न॒य॒ इतीति॑ नये नय॒ इत्या॑हा॒ हेति॑ नये नय॒ इत्या॑ह । \newline
15. इत्या॑हा॒हे तीत्या॑ह॒ तस्मा॒त् तस्मा॑ दा॒हे तीत्या॑ह॒ तस्मा᳚त् । \newline
16. आ॒ह॒ तस्मा॒त् तस्मा॑दा हाह॒ तस्मा॑ द॒द्यमा॑ना अ॒द्यमा॑ना॒ स्तस्मा॑ दाहाह॒ तस्मा॑ द॒द्यमा॑नाः । \newline
17. तस्मा॑ द॒द्यमा॑ना अ॒द्यमा॑ना॒ स्तस्मा॒त् तस्मा॑ द॒द्यमा॑नाः पी॒यमा॑नाः पी॒यमा॑ना अ॒द्यमा॑ना॒ स्तस्मा॒त् तस्मा॑ द॒द्यमा॑नाः पी॒यमा॑नाः । \newline
18. अ॒द्यमा॑नाः पी॒यमा॑नाः पी॒यमा॑ना अ॒द्यमा॑ना अ॒द्यमा॑नाः पी॒यमा॑ना॒ आप॒ आपः॑ पी॒यमा॑ना अ॒द्यमा॑ना अ॒द्यमा॑नाः पी॒यमा॑ना॒ आपः॑ । \newline
19. पी॒यमा॑ना॒ आप॒ आपः॑ पी॒यमा॑नाः पी॒यमा॑ना॒ आपो॒ न नापः॑ पी॒यमा॑नाः पी॒यमा॑ना॒ आपो॒ न । \newline
20. आपो॒ न नाप॒ आपो॒ न क्षी॑यन्ते क्षीयन्ते॒ नाप॒ आपो॒ न क्षी॑यन्ते । \newline
21. न क्षी॑यन्ते क्षीयन्ते॒ न न क्षी॑यन्ते॒ योनि॒र् योनिः॑ क्षीयन्ते॒ न न क्षी॑यन्ते॒ योनिः॑ । \newline
22. क्षी॒य॒न्ते॒ योनि॒र् योनिः॑ क्षीयन्ते क्षीयन्ते॒ योनि॒र् वै वै योनिः॑ क्षीयन्ते क्षीयन्ते॒ योनि॒र् वै । \newline
23. योनि॒र् वै वै योनि॒र् योनि॒र् वै य॒ज्ञ्स्य॑ य॒ज्ञ्स्य॒ वै योनि॒र् योनि॒र् वै य॒ज्ञ्स्य॑ । \newline
24. वै य॒ज्ञ्स्य॑ य॒ज्ञ्स्य॒ वै वै य॒ज्ञ्स्य॒ चात्वा॑ल॒म् चात्वा॑लं ॅय॒ज्ञ्स्य॒ वै वै य॒ज्ञ्स्य॒ चात्वा॑लम् । \newline
25. य॒ज्ञ्स्य॒ चात्वा॑ल॒म् चात्वा॑लं ॅय॒ज्ञ्स्य॑ य॒ज्ञ्स्य॒ चात्वा॑लं ॅय॒ज्ञो य॒ज्ञ् श्चात्वा॑लं ॅय॒ज्ञ्स्य॑ य॒ज्ञ्स्य॒ चात्वा॑लं ॅय॒ज्ञ्ः । \newline
26. चात्वा॑लं ॅय॒ज्ञो य॒ज्ञ् श्चात्वा॑ल॒म् चात्वा॑लं ॅय॒ज्ञो व॑सती॒वरी᳚र् वसती॒वरी᳚र् य॒ज्ञ् श्चात्वा॑ल॒म् चात्वा॑लं ॅय॒ज्ञो व॑सती॒वरीः᳚ । \newline
27. य॒ज्ञो व॑सती॒वरी᳚र् वसती॒वरी᳚र् य॒ज्ञो य॒ज्ञो व॑सती॒वरीर्॑. होतृचम॒सꣳ हो॑तृचम॒सं ॅव॑सती॒वरी᳚र् य॒ज्ञो य॒ज्ञो व॑सती॒वरीर्॑. होतृचम॒सम् । \newline
28. व॒स॒ती॒वरीर्॑. होतृचम॒सꣳ हो॑तृचम॒सं ॅव॑सती॒वरी᳚र् वसती॒वरीर्॑. होतृचम॒सम् च॑ च होतृचम॒सं ॅव॑सती॒वरी᳚र् वसती॒वरीर्॑. होतृचम॒सम् च॑ । \newline
29. हो॒तृ॒च॒म॒सम् च॑ च होतृचम॒सꣳ हो॑तृचम॒सम् च॑ मैत्रावरुणचम॒सम् मै᳚त्रावरुणचम॒सम् च॑ होतृचम॒सꣳ हो॑तृचम॒सम् च॑ मैत्रावरुणचम॒सम् । \newline
30. हो॒तृ॒च॒म॒समिति॑ होतृ - च॒म॒सम् । \newline
31. च॒ मै॒त्रा॒व॒रु॒ण॒च॒म॒सम् मै᳚त्रावरुणचम॒सम् च॑ च मैत्रावरुणचम॒सम् च॑ च मैत्रावरुणचम॒सम् च॑ च मैत्रावरुणचम॒सम् च॑ । \newline
32. मै॒त्रा॒व॒रु॒ण॒च॒म॒सम् च॑ च मैत्रावरुणचम॒सम् मै᳚त्रावरुणचम॒सम् च॑ सꣳ॒॒स्पर्श्य॑ सꣳ॒॒स्पर्श्य॑ च मैत्रावरुणचम॒सम् मै᳚त्रावरुणचम॒सम् च॑ सꣳ॒॒स्पर्श्य॑ । \newline
33. मै॒त्रा॒व॒रु॒ण॒च॒म॒समिति॑ मैत्रावरुण - च॒म॒सम् । \newline
34. च॒ सꣳ॒॒स्पर्श्य॑ सꣳ॒॒स्पर्श्य॑ च च सꣳ॒॒स्पर्श्य॑ वसती॒वरी᳚र् वसती॒वरीः᳚ सꣳ॒॒स्पर्श्य॑ च च सꣳ॒॒स्पर्श्य॑ वसती॒वरीः᳚ । \newline
35. सꣳ॒॒स्पर्श्य॑ वसती॒वरी᳚र् वसती॒वरीः᳚ सꣳ॒॒स्पर्श्य॑ सꣳ॒॒स्पर्श्य॑ वसती॒वरी॒र् व्यान॑यति॒ व्यान॑यति वसती॒वरीः᳚ सꣳ॒॒स्पर्श्य॑ सꣳ॒॒स्पर्श्य॑ वसती॒वरी॒र् व्यान॑यति । \newline
36. सꣳ॒॒स्पर्श्येति॑ सं - स्पर्श्य॑ । \newline
37. व॒स॒ती॒वरी॒र् व्यान॑यति॒ व्यान॑यति वसती॒वरी᳚र् वसती॒वरी॒र् व्यान॑यति य॒ज्ञ्स्य॑ य॒ज्ञ्स्य॒ व्यान॑यति वसती॒वरी᳚र् वसती॒वरी॒र् व्यान॑यति य॒ज्ञ्स्य॑ । \newline
38. व्यान॑यति य॒ज्ञ्स्य॑ य॒ज्ञ्स्य॒ व्यान॑यति॒ व्यान॑यति य॒ज्ञ्स्य॑ सयोनि॒त्वाय॑ सयोनि॒त्वाय॑ य॒ज्ञ्स्य॒ व्यान॑यति॒ व्यान॑यति य॒ज्ञ्स्य॑ सयोनि॒त्वाय॑ । \newline
39. व्यान॑य॒तीति॑ वि - आन॑यति । \newline
40. य॒ज्ञ्स्य॑ सयोनि॒त्वाय॑ सयोनि॒त्वाय॑ य॒ज्ञ्स्य॑ य॒ज्ञ्स्य॑ सयोनि॒त्वायाथो॒ अथो॑ सयोनि॒त्वाय॑ य॒ज्ञ्स्य॑ य॒ज्ञ्स्य॑ सयोनि॒त्वायाथो᳚ । \newline
41. स॒यो॒नि॒त्वायाथो॒ अथो॑ सयोनि॒त्वाय॑ सयोनि॒त्वायाथो॒ स्वाथ् स्वा दथो॑ सयोनि॒त्वाय॑ सयोनि॒त्वायाथो॒ स्वात् । \newline
42. स॒यो॒नि॒त्वायेति॑ सयोनि - त्वाय॑ । \newline
43. अथो॒ स्वाथ् स्वा दथो॒ अथो॒ स्वा दे॒वैव स्वा दथो॒ अथो॒ स्वा दे॒व । \newline
44. अथो॒ इत्यथो᳚ । \newline
45. स्वा दे॒वैव स्वाथ् स्वा दे॒वैना॑ एना ए॒व स्वाथ् स्वा दे॒वैनाः᳚ । \newline
46. ए॒वैना॑ एना ए॒वै वैना॒ योने॒र् योने॑ रेना ए॒वै वैना॒ योनेः᳚ । \newline
47. ए॒ना॒ योने॒र् योने॑ रेना एना॒ योनेः॒ प्र प्र योने॑ रेना एना॒ योनेः॒ प्र । \newline
48. योनेः॒ प्र प्र योने॒र् योनेः॒ प्र ज॑नयति जनयति॒ प्र योने॒र् योनेः॒ प्र ज॑नयति । \newline
49. प्र ज॑नयति जनयति॒ प्र प्र ज॑नय॒ त्यद्ध्व॒र्यो ऽद्ध्व॑र्यो जनयति॒ प्र प्र ज॑नय॒ त्यद्ध्व॑र्यो । \newline
50. ज॒न॒य॒ त्यद्ध्व॒र्यो ऽद्ध्व॑र्यो जनयति जनय॒ त्यद्ध्व॒र्यो ऽवे॒ रवे॒ रद्ध्व॑र्यो जनयति जनय॒ 
त्यद्ध्व॒र्यो ऽवेः᳚ । \newline
51. अद्ध्व॒र्यो ऽवे॒ रवे॒ रद्ध्व॒र्यो ऽद्ध्व॒र्यो ऽवे॑ र॒पा(3) अ॒पा(3) अवे॒ रद्ध्व॒र्यो ऽद्ध्व॒र्यो ऽवे॑ र॒पा(3)ः । \newline
52. अद्ध्व॑र्यो॒ इत्यद्ध्व॑र्यो । \newline
53. अवे॑ र॒पा(3) अ॒पा(3) अवे॒ रवे॑ र॒पा(3) इतीत्य॒पा(3) अवे॒ रवे॑ र॒पा(3) इति॑ । \newline
54. अ॒पा(3) इतीत्य॒पा(3) अ॒पा(3) इत्या॑हा॒हे त्य॒पा(3) अ॒पा(3) इत्या॑ह । \newline
55. इत्या॑हा॒हे तीत्या॑हो॒तो ताहे तीत्या॑ हो॒त । \newline
56. आ॒हो॒ तोता हा॑हो॒तेमी॑ मु॒ताहा॑ हो॒तेम् । \newline
57. उ॒तेमी॑ मु॒तोते म॑नन्नमु रनन्नमुरी मु॒तोते म॑नन्नमुः । \newline
58. ई॒म॒न॒न्न॒मु॒ र॒न॒न्न॒मु॒री॒ मी॒ म॒न॒न्न॒मु॒ रु॒तोता न॑न्नमुरीमी मनन्नमुरु॒त । \newline
59. अ॒न॒न्न॒मु॒ रु॒तोता न॑न्नमुर नन्नमु रु॒तेमा इ॒मा उ॒तान॑न्नमु रनन्न मुरु॒तेमाः । \newline
60. उ॒तेमा इ॒मा उ॒तोते माः प॑श्य पश्ये॒मा उ॒तोतेमाः प॑श्य । \newline
61. इ॒माः प॑श्य पश्ये॒मा इ॒माः प॒श्येतीति॑ पश्ये॒मा इ॒माः प॒श्येति॑ । \newline
62. प॒श्ये तीति॑ पश्य प॒श्येति॒ वाव वावेति॑ पश्य प॒श्येति॒ वाव । \newline
63. इति॒ वाव वावे तीति॒ वावैत दे॒तद् वावे तीति॒ वावैतत् । \newline
64. वावैत दे॒तद् वाव वावै तदा॑हा है॒तद् वाव वावै तदा॑ह । \newline
65. ए॒तदा॑हा है॒त दे॒त दा॑ह॒ यदि॒ यद्या॑ है॒त दे॒त दा॑ह॒ यदि॑ । \newline
66. आ॒ह॒ यदि॒ यद्या॑ हाह॒ यद्य॑ग्निष्टो॒मो᳚ ऽग्निष्टो॒मो यद्या॑हाह॒ यद्य॑ग्निष्टो॒मः । \newline
67. यद्य॑ग्निष्टो॒मो᳚ ऽग्निष्टो॒मो यदि॒ यद्य॑ग्निष्टो॒मो जु॒होति॑ जु॒हो त्य॑ग्निष्टो॒मो यदि॒ यद्य॑ग्निष्टो॒मो जु॒होति॑ । \newline
68. अ॒ग्नि॒ष्टो॒मो जु॒होति॑ जु॒हो त्य॑ग्निष्टो॒मो᳚ ऽग्निष्टो॒मो जु॒होति॒ यदि॒ यदि॑ जु॒हो त्य॑ग्निष्टो॒मो᳚ ऽग्निष्टो॒मो जु॒होति॒ यदि॑ । \newline
69. अ॒ग्नि॒ष्टो॒म इत्य॑ग्नि - स्तो॒मः । \newline
70. जु॒होति॒ यदि॒ यदि॑ जु॒होति॑ जु॒होति॒ यद्यु॒क्थ्य॑ उ॒क्थ्यो॑ यदि॑ जु॒होति॑ जु॒होति॒ यद्यु॒क्थ्यः॑ । \newline
71. यद्यु॒क्थ्य॑ उ॒क्थ्यो॑ यदि॒ यद्यु॒क्थ्यः॑ परि॒धौ प॑रि॒धा वु॒क्थ्यो॑ यदि॒ यद्यु॒क्थ्यः॑ परि॒धौ । \newline
72. उ॒क्थ्यः॑ परि॒धौ प॑रि॒धा वु॒क्थ्य॑ उ॒क्थ्यः॑ परि॒धौ नि नि प॑रि॒धा वु॒क्थ्य॑ उ॒क्थ्यः॑ परि॒धौ नि । \newline
73. प॒रि॒धौ नि नि प॑रि॒धौ प॑रि॒धौ नि मा᳚र्ष्टि मार्ष्टि॒ नि प॑रि॒धौ प॑रि॒धौ नि मा᳚र्ष्टि । \newline
74. प॒रि॒धाविति॑ परि - धौ । \newline
75. नि मा᳚र्ष्टि मार्ष्टि॒ नि नि मा᳚र्ष्टि॒ यदि॒ यदि॑ मार्ष्टि॒ नि नि मा᳚र्ष्टि॒ यदि॑ । \newline
76. मा॒र्ष्टि॒ यदि॒ यदि॑ मार्ष्टि मार्ष्टि॒ यद्य॑तिरा॒त्रो॑ ऽतिरा॒त्रो यदि॑ मार्ष्टि मार्ष्टि॒ यद्य॑तिरा॒त्रः । \newline
77. यद्य॑तिरा॒त्रो॑ ऽतिरा॒त्रो यदि॒ यद्य॑तिरा॒त्रो यजु॒र् यजु॑ रतिरा॒त्रो यदि॒ यद्य॑तिरा॒त्रो यजुः॑ । \newline
78. अ॒ति॒रा॒त्रो यजु॒र् यजु॑ रतिरा॒त्रो॑ ऽतिरा॒त्रो यजु॒र् वद॒न्॒. वद॒न्॒. यजु॑ रतिरा॒त्रो॑ ऽतिरा॒त्रो यजु॒र् वदन्न्॑ । \newline
79. अ॒ति॒रा॒त्र इत्य॑ति - रा॒त्रः । \newline
80. यजु॒र् वद॒न्॒. वद॒न्॒. यजु॒र् यजु॒र् वद॒न् प्र प्र वद॒न्॒. यजु॒र् यजु॒र् वद॒न् प्र । \newline
81. वद॒न् प्र प्र वद॒न्॒. वद॒न् प्र प॑द्यते पद्यते॒ प्र वद॒न्॒. वद॒न् प्र प॑द्यते । \newline
82. प्र प॑द्यते पद्यते॒ प्र प्र प॑द्यते यज्ञ्क्रतू॒नां ॅय॑ज्ञ्क्रतू॒नाम् प॑द्यते॒ प्र प्र प॑द्यते यज्ञ्क्रतू॒नाम् । \newline
83. प॒द्य॒ते॒ य॒ज्ञ्॒क्र॒तू॒नां ॅय॑ज्ञ्क्रतू॒नाम् प॑द्यते पद्यते यज्ञ्क्रतू॒नां ॅव्यावृ॑त्त्यै॒ व्यावृ॑त्त्यै यज्ञ्क्रतू॒नाम् प॑द्यते पद्यते यज्ञ्क्रतू॒नां ॅव्यावृ॑त्त्यै । \newline
84. य॒ज्ञ्॒क्र॒तू॒नां ॅव्यावृ॑त्त्यै॒ व्यावृ॑त्त्यै यज्ञ्क्रतू॒नां ॅय॑ज्ञ्क्रतू॒नां ॅव्यावृ॑त्त्यै । \newline
85. य॒ज्ञ्॒क्र॒तू॒नामिति॑ यज्ञ् - क्र॒तू॒नाम् । \newline
86. व्यावृ॑त्त्या॒ इति॑ वि - आवृ॑त्त्यै । \newline
\pagebreak
\markright{ TS 6.4.4.1  \hfill https://www.vedavms.in \hfill}

\section{ TS 6.4.4.1 }

\textbf{TS 6.4.4.1 } \newline
\textbf{Samhita Paata} \newline

दे॒वस्य॑ त्वा सवि॒तुः प्रस॑व॒ इति॒ ग्रावा॑ण॒मा द॑त्ते॒ प्रसू᳚त्या अ॒श्विनो᳚-र्बा॒हुभ्या॒मित्या॑हा॒श्विनौ॒ हि दे॒वाना॑मद्ध्व॒र्यू आस्तां᳚ पू॒ष्णो हस्ता᳚भ्या॒मित्या॑ह॒ यत्यै॑ प॒शवो॒ वै सोमो᳚ व्या॒न उ॑पाꣳशु॒सव॑नो॒ यदु॑पाꣳशु॒सव॑न-म॒भि मिमी॑ते व्या॒नमे॒व प॒शुषु॑ दधा॒तीन्द्रा॑य॒ त्वेन्द्रा॑य॒ त्वेति॑ मिमीत॒ इन्द्रा॑य॒ हि सोम॑ आह्रि॒यते॒ पञ्च॒ कृत्वो॒ यजु॑षा मिमीते॒- [  ] \newline

\textbf{Pada Paata} \newline

दे॒वस्य॑ । त्वा॒ । स॒वि॒तुः । प्र॒स॒व इति॑ प्र - स॒वे । इति॑ । ग्रावा॑णम् । एति॑ । द॒त्ते॒ । प्रसू᳚त्या॒ इति॒ प्र - सू॒त्यै॒ । अ॒श्विनोः᳚ । बा॒हुभ्या॒मिति॑ बा॒हु-भ्या॒म् । इति॑ । आ॒ह॒ । अ॒श्विनौ᳚ । हि । दे॒वाना᳚म् । अ॒द्ध्व॒र्यू इति॑ । आस्ता᳚म् । पू॒ष्णः । हस्ता᳚भ्याम् । इति॑ । आ॒ह॒ । यत्यै᳚ । प॒शवः॑ । वै । सोमः॑ । व्या॒न इति॑ वि - अ॒नः । उ॒पाꣳ॒॒शु॒सव॑न॒ इत्यु॑पाꣳशु - सव॑नः । यत् । उ॒पाꣳ॒॒शु॒सव॑न॒मित्यु॑पाꣳशु - सव॑नम् । अ॒भीति॑ । मिमी॑ते । व्या॒नमिति॑ वि - अ॒नम् । ए॒व । प॒शुषु॑ । द॒धा॒ति॒ । इन्द्रा॑य । त्वा॒ । इन्द्रा॑य । त्वा॒ । इति॑ । मि॒मी॒ते॒ । इन्द्रा॑य । हि । सोमः॑ । आ॒ह्रि॒यत॒ इत्या᳚ - ह्रि॒यते᳚ । पञ्च॑ । कृत्वः॑ । यजु॑षा । मि॒मी॒ते॒ ।  \newline


\textbf{Krama Paata} \newline

दे॒वस्य॑ त्वा । त्वा॒ स॒वि॒तुः । स॒वि॒तुः प्र॑स॒वे । प्र॒स॒व इति॑ । प्र॒स॒व इति॑ प्र - स॒वे । इति॒ ग्रावा॑णम् । ग्रावा॑ण॒मा । आ द॑त्ते । द॒त्ते॒ प्रसू᳚त्यै । प्रसू᳚त्या अ॒श्विनोः᳚ । प्रसू᳚त्या॒ इति॒ प्र - सू॒त्यै॒ । अ॒श्विनो᳚र् बा॒हुभ्या᳚म् । बा॒हुभ्या॒मिति॑ । बा॒हुभ्या॒मिति॑ बा॒हु - भ्या॒म् । इत्या॑ह । आ॒हा॒श्विनौ᳚ । अ॒श्विनौ॒ हि । हि दे॒वाना᳚म् । दे॒वाना॑मद्ध्व॒र्यू । अ॒द्ध्व॒र्यू आस्ता᳚म् । अ॒द्ध्व॒र्यू इत्य॑द्ध्व॒र्यू । आस्ता᳚म् पू॒ष्णः । पू॒ष्णो हस्ता᳚भ्याम् । हस्ता᳚भ्या॒मिति॑ । इत्या॑ह । आ॒ह॒ यत्यै᳚ । यत्यै॑ प॒शवः॑ । प॒शवो॒ वै । वै सोमः॑ । सोमो᳚ व्या॒नः । व्या॒न उ॑पाꣳशु॒सव॑नः । व्या॒न इति॑ वि - अ॒नः । उ॒पाꣳ॒॒शु॒सव॑नो॒ यत् । उ॒पाꣳ॒॒शु॒सव॑न॒ इत्यु॑पाꣳशु - सव॑नः । यदु॑पाꣳशु॒सव॑नम् । उ॒पाꣳ॒॒शु॒सव॑नम॒भि । उ॒पाꣳ॒॒शु॒सव॑न॒मित्यु॑पाꣳशु - सव॑नम् । अ॒भि मिमी॑ते । मिमी॑ते व्या॒नम् । व्या॒नमे॒व । व्या॒नमिति॑ वि - अ॒नम् । ए॒व प॒शुषु॑ । प॒शुषु॑ दधाति । द॒धा॒तीन्द्रा॑य । इन्द्रा॑य त्वा । त्वेन्द्रा॑य । इन्द्रा॑य त्वा । त्वेति॑ । इति॑ मिमीते । मि॒मी॒त॒ इन्द्रा॑य । इन्द्रा॑य॒ हि । हि सोमः॑ । सोम॑ आह्रि॒यते᳚ । आ॒ह्रि॒यते॒ पञ्च॑ । आ॒ह्रि॒यत॒ इत्या᳚ - ह्रि॒यते᳚ । पञ्च॒ कृत्वः॑ । कृत्वो॒ यजु॑षा । यजु॑षा मिमीते । मि॒मी॒ते॒ पञ्चा᳚क्षरा \newline

\textbf{Jatai Paata} \newline

1. दे॒वस्य॑ त्वा त्वा दे॒वस्य॑ दे॒वस्य॑ त्वा । \newline
2. त्वा॒ स॒वि॒तुः स॑वि॒तु स्त्वा᳚ त्वा सवि॒तुः । \newline
3. स॒वि॒तुः प्र॑स॒वे प्र॑स॒वे स॑वि॒तुः स॑वि॒तुः प्र॑स॒वे । \newline
4. प्र॒स॒व इतीति॑ प्रस॒वे प्र॑स॒व इति॑ । \newline
5. प्र॒स॒व इति॑ प्र - स॒वे । \newline
6. इति॒ ग्रावा॑ण॒म् ग्रावा॑ण॒ मितीति॒ ग्रावा॑णम् । \newline
7. ग्रावा॑ण॒ मा ग्रावा॑ण॒म् ग्रावा॑ण॒ मा । \newline
8. आ द॑त्ते दत्त॒ आ द॑त्ते । \newline
9. द॒त्ते॒ प्रसू᳚त्यै॒ प्रसू᳚त्यै दत्ते दत्ते॒ प्रसू᳚त्यै । \newline
10. प्रसू᳚त्या अ॒श्विनो॑ र॒श्विनोः॒ प्रसू᳚त्यै॒ प्रसू᳚त्या अ॒श्विनोः᳚ । \newline
11. प्रसू᳚त्या॒ इति॒ प्र - सू॒त्यै॒ । \newline
12. अ॒श्विनो᳚र् बा॒हुभ्या᳚म् बा॒हुभ्या॑ म॒श्विनो॑ र॒श्विनो᳚र् बा॒हुभ्या᳚म् । \newline
13. बा॒हुभ्या॒ मितीति॑ बा॒हुभ्या᳚म् बा॒हुभ्या॒ मिति॑ । \newline
14. बा॒हुभ्या॒मिति॑ बा॒हु - भ्या॒म् । \newline
15. इत्या॑हा॒हे तीत्या॑ह । \newline
16. आ॒हा॒श्विना॑ व॒श्विना॑ वाहा हा॒श्विनौ᳚ । \newline
17. अ॒श्विनौ॒ हि ह्य॑श्विना॑ व॒श्विनौ॒ हि । \newline
18. हि दे॒वाना᳚म् दे॒वानाꣳ॒॒ हि हि दे॒वाना᳚म् । \newline
19. दे॒वाना॑ मद्ध्व॒र्यू अ॑द्ध्व॒र्यू दे॒वाना᳚म् दे॒वाना॑ मद्ध्व॒र्यू । \newline
20. अ॒द्ध्व॒र्यू आस्ता॒ मास्ता॑ मद्ध्व॒र्यू अ॑द्ध्व॒र्यू आस्ता᳚म् । \newline
21. अ॒द्ध्व॒र्यू इत्य॑द्ध्व॒र्यू । \newline
22. आस्ता᳚म् पू॒ष्णः पू॒ष्ण आस्ता॒ मास्ता᳚म् पू॒ष्णः । \newline
23. पू॒ष्णो हस्ता᳚भ्याꣳ॒॒ हस्ता᳚भ्याम् पू॒ष्णः पू॒ष्णो हस्ता᳚भ्याम् । \newline
24. हस्ता᳚भ्या॒ मितीति॒ हस्ता᳚भ्याꣳ॒॒ हस्ता᳚भ्या॒ मिति॑ । \newline
25. इत्या॑हा॒हे तीत्या॑ह । \newline
26. आ॒ह॒ यत्यै॒ यत्या॑ आहाह॒ यत्यै᳚ । \newline
27. यत्यै॑ प॒शवः॑ प॒शवो॒ यत्यै॒ यत्यै॑ प॒शवः॑ । \newline
28. प॒शवो॒ वै वै प॒शवः॑ प॒शवो॒ वै । \newline
29. वै सोमः॒ सोमो॒ वै वै सोमः॑ । \newline
30. सोमो᳚ व्या॒नो व्या॒नः सोमः॒ सोमो᳚ व्या॒नः । \newline
31. व्या॒न उ॑पाꣳशु॒सव॑न उपाꣳशु॒सव॑नो व्या॒नो व्या॒न उ॑पाꣳशु॒सव॑नः । \newline
32. व्या॒न इति॑ वि - अ॒नः । \newline
33. उ॒पाꣳ॒॒शु॒सव॑नो॒ यद् यदु॑पाꣳशु॒सव॑न उपाꣳशु॒सव॑नो॒ यत् । \newline
34. उ॒पाꣳ॒॒शु॒सव॑न॒ इत्यु॑पाꣳशु - सव॑नः । \newline
35. यदु॑पाꣳशु॒सव॑न मुपाꣳशु॒सव॑नं॒ ॅयद् यदु॑पाꣳशु॒सव॑नम् । \newline
36. उ॒पाꣳ॒॒शु॒सव॑न म॒भ्या᳚(1॒)भ्यु॑ पाꣳशु॒सव॑न मुपाꣳशु॒सव॑न म॒भि । \newline
37. उ॒पाꣳ॒॒शु॒सव॑न॒मित्यु॑पाꣳशु - सव॑नम् । \newline
38. अ॒भि मिमी॑ते॒ मिमी॑ते॒ ऽभ्य॑भि मिमी॑ते । \newline
39. मिमी॑ते व्या॒नं ॅव्या॒नम् मिमी॑ते॒ मिमी॑ते व्या॒नम् । \newline
40. व्या॒न मे॒वैव व्या॒नं ॅव्या॒न मे॒व । \newline
41. व्या॒नमिति॑ वि - अ॒नम् । \newline
42. ए॒व प॒शुषु॑ प॒शु ष्वे॒वैव प॒शुषु॑ । \newline
43. प॒शुषु॑ दधाति दधाति प॒शुषु॑ प॒शुषु॑ दधाति । \newline
44. द॒धा॒ तीन्द्रा॒ येन्द्रा॑य दधाति दधा॒ तीन्द्रा॑य । \newline
45. इन्द्रा॑य त्वा॒ त्वेन्द्रा॒ येन्द्रा॑य त्वा । \newline
46. त्वेन्द्रा॒ येन्द्रा॑य त्वा॒ त्वेन्द्रा॑य । \newline
47. इन्द्रा॑य त्वा॒ त्वेन्द्रा॒ येन्द्रा॑य त्वा । \newline
48. त्वेतीति॑ त्वा॒ त्वेति॑ । \newline
49. इति॑ मिमीते मिमीत॒ इतीति॑ मिमीते । \newline
50. मि॒मी॒त॒ इन्द्रा॒ येन्द्रा॑य मिमीते मिमीत॒ इन्द्रा॑य । \newline
51. इन्द्रा॑य॒ हि हीन्द्रा॒ येन्द्रा॑य॒ हि । \newline
52. हि सोमः॒ सोमो॒ हि हि सोमः॑ । \newline
53. सोम॑ आह्रि॒यत॑ आह्रि॒यते॒ सोमः॒ सोम॑ आह्रि॒यते᳚ । \newline
54. आ॒ह्रि॒यते॒ पञ्च॒ पञ्चा᳚ ह्रि॒यत॑ आह्रि॒यते॒ पञ्च॑ । \newline
55. आ॒ह्रि॒यत॒ इत्या᳚ - ह्रि॒यते᳚ । \newline
56. पञ्च॒ कृत्वः॒ कृत्वः॒ पञ्च॒ पञ्च॒ कृत्वः॑ । \newline
57. कृत्वो॒ यजु॑षा॒ यजु॑षा॒ कृत्वः॒ कृत्वो॒ यजु॑षा । \newline
58. यजु॑षा मिमीते मिमीते॒ यजु॑षा॒ यजु॑षा मिमीते । \newline
59. मि॒मी॒ते॒ पञ्चा᳚क्षरा॒ पञ्चा᳚क्षरा मिमीते मिमीते॒ पञ्चा᳚क्षरा । \newline

\textbf{Ghana Paata } \newline

1. दे॒वस्य॑ त्वा त्वा दे॒वस्य॑ दे॒वस्य॑ त्वा सवि॒तुः स॑वि॒तु स्त्वा॑ दे॒वस्य॑ दे॒वस्य॑ त्वा सवि॒तुः । \newline
2. त्वा॒ स॒वि॒तुः स॑वि॒तु स्त्वा᳚ त्वा सवि॒तुः प्र॑स॒वे प्र॑स॒वे स॑वि॒तु स्त्वा᳚ त्वा सवि॒तुः प्र॑स॒वे । \newline
3. स॒वि॒तुः प्र॑स॒वे प्र॑स॒वे स॑वि॒तुः स॑वि॒तुः प्र॑स॒व इतीति॑ प्रस॒वे स॑वि॒तुः स॑वि॒तुः प्र॑स॒व इति॑ । \newline
4. प्र॒स॒व इतीति॑ प्रस॒वे प्र॑स॒व इति॒ ग्रावा॑ण॒म् ग्रावा॑ण॒ मिति॑ प्रस॒वे प्र॑स॒व इति॒ ग्रावा॑णम् । \newline
5. प्र॒स॒व इति॑ प्र - स॒वे । \newline
6. इति॒ ग्रावा॑ण॒म् ग्रावा॑ण॒ मितीति॒ ग्रावा॑ण॒ मा ग्रावा॑ण॒ मितीति॒ ग्रावा॑ण॒ मा । \newline
7. ग्रावा॑ण॒ मा ग्रावा॑ण॒म् ग्रावा॑ण॒ मा द॑त्ते दत्त॒ आ ग्रावा॑ण॒म् ग्रावा॑ण॒ मा द॑त्ते । \newline
8. आ द॑त्ते दत्त॒ आ द॑त्ते॒ प्रसू᳚त्यै॒ प्रसू᳚त्यै दत्त॒ आ द॑त्ते॒ प्रसू᳚त्यै । \newline
9. द॒त्ते॒ प्रसू᳚त्यै॒ प्रसू᳚त्यै दत्ते दत्ते॒ प्रसू᳚त्या अ॒श्विनो॑ र॒श्विनोः॒ प्रसू᳚त्यै दत्ते दत्ते॒ प्रसू᳚त्या अ॒श्विनोः᳚ । \newline
10. प्रसू᳚त्या अ॒श्विनो॑ र॒श्विनोः॒ प्रसू᳚त्यै॒ प्रसू᳚त्या अ॒श्विनो᳚र् बा॒हुभ्या᳚म् बा॒हुभ्या॑ म॒श्विनोः॒ प्रसू᳚त्यै॒ प्रसू᳚त्या अ॒श्विनो᳚र् बा॒हुभ्या᳚म् । \newline
11. प्रसू᳚त्या॒ इति॒ प्र - सू॒त्यै॒ । \newline
12. अ॒श्विनो᳚र् बा॒हुभ्या᳚म् बा॒हुभ्या॑ म॒श्विनो॑ र॒श्विनो᳚र् बा॒हुभ्या॒ मितीति॑ बा॒हुभ्या॑ म॒श्विनो॑ र॒श्विनो᳚र् बा॒हुभ्या॒ मिति॑ । \newline
13. बा॒हुभ्या॒ मितीति॑ बा॒हुभ्या᳚म् बा॒हुभ्या॒ मित्या॑हा॒ हेति॑ बा॒हुभ्या᳚म् बा॒हुभ्या॒ मित्या॑ह । \newline
14. बा॒हुभ्या॒मिति॑ बा॒हु - भ्या॒म् । \newline
15. इत्या॑हा॒हे तीत्या॑ हा॒श्विना॑ व॒श्विना॑ वा॒हे तीत्या॑ हा॒श्विनौ᳚ । \newline
16. आ॒हा॒श्विना॑ व॒श्विना॑ वाहा हा॒श्विनौ॒ हि ह्य॑श्विना॑ वाहा हा॒श्विनौ॒ हि । \newline
17. अ॒श्विनौ॒ हि ह्य॑श्विना॑ व॒श्विनौ॒ हि दे॒वाना᳚म् दे॒वानाꣳ॒॒ ह्य॑श्विना॑ व॒श्विनौ॒ हि दे॒वाना᳚म् । \newline
18. हि दे॒वाना᳚म् दे॒वानाꣳ॒॒ हि हि दे॒वाना॑ मद्ध्व॒र्यू अ॑द्ध्व॒र्यू दे॒वानाꣳ॒॒ हि हि दे॒वाना॑ मद्ध्व॒र्यू । \newline
19. दे॒वाना॑ मद्ध्व॒र्यू अ॑द्ध्व॒र्यू दे॒वाना᳚म् दे॒वाना॑ मद्ध्व॒र्यू आस्ता॒ मास्ता॑ मद्ध्व॒र्यू दे॒वाना᳚म् दे॒वाना॑ मद्ध्व॒र्यू आस्ता᳚म् । \newline
20. अ॒द्ध्व॒र्यू आस्ता॒ मास्ता॑ मद्ध्व॒र्यू अ॑द्ध्व॒र्यू आस्ता᳚म् पू॒ष्णः पू॒ष्ण आस्ता॑ मद्ध्व॒र्यू अ॑द्ध्व॒र्यू आस्ता᳚म् पू॒ष्णः । \newline
21. अ॒द्ध्व॒र्यू इत्य॑द्ध्व॒र्यू । \newline
22. आस्ता᳚म् पू॒ष्णः पू॒ष्ण आस्ता॒ मास्ता᳚म् पू॒ष्णो हस्ता᳚भ्याꣳ॒॒ हस्ता᳚भ्याम् पू॒ष्ण आस्ता॒ मास्ता᳚म् पू॒ष्णो हस्ता᳚भ्याम् । \newline
23. पू॒ष्णो हस्ता᳚भ्याꣳ॒॒ हस्ता᳚भ्याम् पू॒ष्णः पू॒ष्णो हस्ता᳚भ्या॒ मितीति॒ हस्ता᳚भ्याम् पू॒ष्णः पू॒ष्णो हस्ता᳚भ्या॒ मिति॑ । \newline
24. हस्ता᳚भ्या॒ मितीति॒ हस्ता᳚भ्याꣳ॒॒ हस्ता᳚भ्या॒ मित्या॑हा॒ हेति॒ हस्ता᳚भ्याꣳ॒॒ हस्ता᳚भ्या॒ मित्या॑ह । \newline
25. इत्या॑हा॒हे तीत्या॑ह॒ यत्यै॒ यत्या॑ आ॒हे तीत्या॑ह॒ यत्यै᳚ । \newline
26. आ॒ह॒ यत्यै॒ यत्या॑ आहाह॒ यत्यै॑ प॒शवः॑ प॒शवो॒ यत्या॑ आहाह॒ यत्यै॑ प॒शवः॑ । \newline
27. यत्यै॑ प॒शवः॑ प॒शवो॒ यत्यै॒ यत्यै॑ प॒शवो॒ वै वै प॒शवो॒ यत्यै॒ यत्यै॑ प॒शवो॒ वै । \newline
28. प॒शवो॒ वै वै प॒शवः॑ प॒शवो॒ वै सोमः॒ सोमो॒ वै प॒शवः॑ प॒शवो॒ वै सोमः॑ । \newline
29. वै सोमः॒ सोमो॒ वै वै सोमो᳚ व्या॒नो व्या॒नः सोमो॒ वै वै सोमो᳚ व्या॒नः । \newline
30. सोमो᳚ व्या॒नो व्या॒नः सोमः॒ सोमो᳚ व्या॒न उ॑पाꣳशु॒सव॑न उपाꣳशु॒सव॑नो व्या॒नः सोमः॒ सोमो᳚ व्या॒न उ॑पाꣳशु॒सव॑नः । \newline
31. व्या॒न उ॑पाꣳशु॒सव॑न उपाꣳशु॒सव॑नो व्या॒नो व्या॒न उ॑पाꣳशु॒सव॑नो॒ यद् यदु॑पाꣳशु॒सव॑नो व्या॒नो व्या॒न उ॑पाꣳशु॒सव॑नो॒ यत् । \newline
32. व्या॒न इति॑ वि - अ॒नः । \newline
33. उ॒पाꣳ॒॒शु॒सव॑नो॒ यद् यदु॑पाꣳशु॒सव॑न उपाꣳशु॒सव॑नो॒ यदु॑पाꣳशु॒सव॑न मुपाꣳशु॒सव॑नं॒ ॅयदु॑पाꣳशु॒सव॑न उपाꣳशु॒सव॑नो॒ यदु॑पाꣳशु॒सव॑नम् । \newline
34. उ॒पाꣳ॒॒शु॒सव॑न॒ इत्यु॑पाꣳशु - सव॑नः । \newline
35. यदु॑पाꣳशु॒सव॑न मुपाꣳशु॒सव॑नं॒ ॅयद् यदु॑पाꣳशु॒सव॑न म॒भ्या᳚(1॒)भ्यु॑पाꣳशु॒सव॑नं॒ ॅयद् यदु॑पाꣳशु॒सव॑न म॒भि । \newline
36. उ॒पाꣳ॒॒शु॒सव॑न म॒भ्या᳚(1॒)भ्यु॑पाꣳशु॒सव॑न मुपाꣳशु॒सव॑न म॒भि मिमी॑ते॒ मिमी॑ते॒ ऽभ्यु॑पाꣳशु॒सव॑न मुपाꣳशु॒सव॑न म॒भि मिमी॑ते । \newline
37. उ॒पाꣳ॒॒शु॒सव॑न॒मित्यु॑पाꣳशु - सव॑नम् । \newline
38. अ॒भि मिमी॑ते॒ मिमी॑ते॒ ऽभ्य॑भि मिमी॑ते व्या॒नं ॅव्या॒नम् मिमी॑ते॒ ऽभ्य॑भि मिमी॑ते व्या॒नम् । \newline
39. मिमी॑ते व्या॒नं ॅव्या॒नम् मिमी॑ते॒ मिमी॑ते व्या॒न मे॒वैव व्या॒नम् मिमी॑ते॒ मिमी॑ते व्या॒न मे॒व । \newline
40. व्या॒न मे॒वैव व्या॒नं ॅव्या॒न मे॒व प॒शुषु॑ प॒शुष्वे॒व व्या॒नं ॅव्या॒न मे॒व प॒शुषु॑ । \newline
41. व्या॒नमिति॑ वि - अ॒नम् । \newline
42. ए॒व प॒शुषु॑ प॒शु ष्वे॒वैव प॒शुषु॑ दधाति दधाति प॒शु ष्वे॒वैव प॒शुषु॑ दधाति । \newline
43. प॒शुषु॑ दधाति दधाति प॒शुषु॑ प॒शुषु॑ दधा॒तीन्द्रा॒ येन्द्रा॑य दधाति प॒शुषु॑ प॒शुषु॑ दधा॒ तीन्द्रा॑य । \newline
44. द॒धा॒ तीन्द्रा॒ येन्द्रा॑य दधाति दधा॒ तीन्द्रा॑य त्वा॒ त्वेन्द्रा॑य दधाति दधा॒ तीन्द्रा॑य त्वा । \newline
45. इन्द्रा॑य त्वा॒ त्वेन्द्रा॒ येन्द्रा॑य॒ त्वेन्द्रा॒ येन्द्रा॑य॒ त्वेन्द्रा॒ येन्द्रा॑य॒ त्वेन्द्रा॑य । \newline
46. त्वेन्द्रा॒ येन्द्रा॑य त्वा॒ त्वेन्द्रा॑य त्वा॒ त्वेन्द्रा॑य त्वा॒ त्वेन्द्रा॑य त्वा । \newline
47. इन्द्रा॑य त्वा॒ त्वेन्द्रा॒ येन्द्रा॑य॒ त्वेतीति॒ त्वेन्द्रा॒ येन्द्रा॑य॒ त्वेति॑ । \newline
48. त्वेतीति॑ त्वा॒ त्वेति॑ मिमीते मिमीत॒ इति॑ त्वा॒ त्वेति॑ मिमीते । \newline
49. इति॑ मिमीते मिमीत॒ इतीति॑ मिमीत॒ इन्द्रा॒ येन्द्रा॑य मिमीत॒ इतीति॑ मिमीत॒ इन्द्रा॑य । \newline
50. मि॒मी॒त॒ इन्द्रा॒ येन्द्रा॑य मिमीते मिमीत॒ इन्द्रा॑य॒ हि हीन्द्रा॑य मिमीते मिमीत॒ इन्द्रा॑य॒ हि । \newline
51. इन्द्रा॑य॒ हि हीन्द्रा॒ येन्द्रा॑य॒ हि सोमः॒ सोमो॒ हीन्द्रा॒ येन्द्रा॑य॒ हि सोमः॑ । \newline
52. हि सोमः॒ सोमो॒ हि हि सोम॑ आह्रि॒यत॑ आह्रि॒यते॒ सोमो॒ हि हि सोम॑ आह्रि॒यते᳚ । \newline
53. सोम॑ आह्रि॒यत॑ आह्रि॒यते॒ सोमः॒ सोम॑ आह्रि॒यते॒ पञ्च॒ पञ्चा᳚ ह्रि॒यते॒ सोमः॒ सोम॑ आह्रि॒यते॒ पञ्च॑ । \newline
54. आ॒ह्रि॒यते॒ पञ्च॒ पञ्चा᳚ ह्रि॒यत॑ आह्रि॒यते॒ पञ्च॒ कृत्वः॒ कृत्वः॒ पञ्चा᳚ ह्रि॒यत॑ आह्रि॒यते॒ पञ्च॒ कृत्वः॑ । \newline
55. आ॒ह्रि॒यत॒ इत्या᳚ - ह्रि॒यते᳚ । \newline
56. पञ्च॒ कृत्वः॒ कृत्वः॒ पञ्च॒ पञ्च॒ कृत्वो॒ यजु॑षा॒ यजु॑षा॒ कृत्वः॒ पञ्च॒ पञ्च॒ कृत्वो॒ यजु॑षा । \newline
57. कृत्वो॒ यजु॑षा॒ यजु॑षा॒ कृत्वः॒ कृत्वो॒ यजु॑षा मिमीते मिमीते॒ यजु॑षा॒ कृत्वः॒ कृत्वो॒ यजु॑षा मिमीते । \newline
58. यजु॑षा मिमीते मिमीते॒ यजु॑षा॒ यजु॑षा मिमीते॒ पञ्चा᳚क्षरा॒ पञ्चा᳚क्षरा मिमीते॒ यजु॑षा॒ यजु॑षा मिमीते॒ पञ्चा᳚क्षरा । \newline
59. मि॒मी॒ते॒ पञ्चा᳚क्षरा॒ पञ्चा᳚क्षरा मिमीते मिमीते॒ पञ्चा᳚क्षरा प॒ङ्क्तिः प॒ङ्क्तिः पञ्चा᳚क्षरा मिमीते मिमीते॒ पञ्चा᳚क्षरा प॒ङ्क्तिः । \newline
\pagebreak
\markright{ TS 6.4.4.2  \hfill https://www.vedavms.in \hfill}

\section{ TS 6.4.4.2 }

\textbf{TS 6.4.4.2 } \newline
\textbf{Samhita Paata} \newline

पञ्चा᳚क्षरा प॒ङ्क्तिः पाङ्क्तो॑ य॒ज्ञो य॒ज्ञ्मे॒वाव॑ रुन्धे॒ पञ्च॒ कृत्व॑स्तू॒ष्णीं दश॒ सं प॑द्यन्ते॒ दशा᳚क्षरा वि॒राडन्नं॑ ॅवि॒राड् वि॒राजै॒वान्नाद्य॒मव॑ रुन्धे श्वा॒त्राः स्थ॑ वृत्र॒तुर॒ इत्या॑है॒ष वा अ॒पाꣳ सो॑मपी॒थो य ए॒वं ॅवेद॒ नाफ्स्वार्ति॒मार्च्छ॑ति॒ यत्ते॑ सोम दि॒वि ज्योति॒रित्या॑है॒भ्य ए॒वैनं॑- [  ] \newline

\textbf{Pada Paata} \newline

पञ्चा᳚क्ष॒रेति॒ पञ्च॑-अ॒क्ष॒रा॒ । प॒ङ्क्तिः । पाङ्क्तः॑ । य॒ज्ञ्ः । य॒ज्ञ्म् । ए॒व । अवेति॑ । रु॒न्धे॒ । पञ्च॑ । कृत्वः॑ । तू॒ष्णीम् । दश॑ । समिति॑ । प॒द्य॒न्ते॒ । दशा᳚क्ष॒रेति॒ दश॑ - अ॒क्ष॒रा॒ । वि॒राडिति॑ वि - राट् । अन्न᳚म् । वि॒राडिति॑ वि - राट् । वि॒राजेति॑ वि - राजा᳚ । ए॒व । अ॒न्नाद्य॒मित्य॑न्न -अद्य᳚म् । अवेति॑ । रु॒न्धे॒ । श्वा॒त्राः । स्थ॒ । वृ॒त्र॒तुर॒ इति॑ वृत्र - तुरः॑ । इति॑ । आ॒ह॒ । ए॒षः । वै । अ॒पाम् । सो॒म॒पी॒थ इति॑ सोम - पी॒थः । यः । ए॒वम् । वेद॑ । न । अ॒फ्स्वित्य॑प् - सु । आर्ति᳚म् । एति॑ । ऋ॒च्छ॒ति॒ । यत् । ते॒ । सो॒म॒ । दि॒वि । ज्योतिः॑ । इति॑ । आ॒ह॒ । ए॒भ्यः । ए॒व । ए॒न॒म् ।  \newline


\textbf{Krama Paata} \newline

पञ्चा᳚क्षरा प॒ङ्‍क्तिः । पञ्चा᳚क्ष॒रेति॒ पञ्च॑ - अ॒क्ष॒रा॒ । प॒ङ्‍क्तिः पाङ्‍क्तः॑ । पाङ्‍क्तो॑ य॒ज्ञ्ः । य॒ज्ञो य॒ज्ञ्म् । य॒ज्ञ्मे॒व । ए॒वाव॑ । अव॑ रुन्धे । रु॒न्धे॒ पञ्च॑ । पञ्च॒ कृत्वः॑ । कृत्व॑स्तू॒ष्णीम् । तू॒ष्णीम् दश॑ । दश॒ सम् । सम् प॑द्यन्ते । प॒द्य॒न्ते॒ दशा᳚क्षरा । दशा᳚क्षरा वि॒राट् । दशा᳚क्ष॒रेति॒ दश॑ - अ॒क्ष॒रा॒ । वि॒राडन्न᳚म् । वि॒राडिति॑ वि - राट् । अन्न॑म् ॅवि॒राट् । वि॒राड् वि॒राजा᳚ । वि॒राडिति॑ वि - राट् । वि॒राजै॒व । वि॒राजेति॑ वि - राजा᳚ । ए॒वान्नाद्य᳚म् । अ॒न्नाद्य॒मव॑ । अ॒न्नाद्य॒मित्य॑न्न - अद्य᳚म् । अव॑ रुन्धे । रु॒न्धे॒ श्वा॒त्राः । श्वा॒त्राः स्थ॑ । स्थ॒ वृ॒त्र॒तुरः॑ । वृ॒त्र॒तुर॒ इति॑ । वृ॒त्र॒तुर॒ इति॑ वृत्र - तुरः॑ । इत्या॑ह । आ॒है॒षः । ए॒ष वै । वा अ॒पाम् । अ॒पाꣳ सो॑मपी॒थः । सो॒म॒पी॒थो यः । सो॒म॒पी॒थ इति॑ सोम - पी॒थः । य ए॒वम् । ए॒वम् ॅवेद॑ । वेद॒ न । नाफ्सु । अ॒फ्स्वार्ति᳚म् । अ॒फ्स्वित्य॑प् - सु । आर्ति॒मा । आर्च्छ॑ति । ऋ॒च्छ॒ति॒ यत् । यत् ते᳚ । ते॒ सो॒म॒ । सो॒म॒ दि॒वि । दि॒वि ज्योतिः॑ । ज्योति॒रिति॑ । इत्या॑ह । आ॒है॒भ्यः । ए॒भ्य ए॒व । ए॒वैन᳚म् । ए॒न॒म् ॅलो॒केभ्यः॑ \newline

\textbf{Jatai Paata} \newline

1. पञ्चा᳚क्षरा प॒ङ्क्तिः प॒ङ्क्तिः पञ्चा᳚क्षरा॒ पञ्चा᳚क्षरा प॒ङ्क्तिः । \newline
2. पञ्चा᳚क्ष॒रेति॒ पञ्च॑ - अ॒क्ष॒रा॒ । \newline
3. प॒ङ्क्तिः पाङ्क्तः॒ पाङ्क्तः॑ प॒ङ्क्तिः प॒ङ्क्तिः पाङ्क्तः॑ । \newline
4. पाङ्क्तो॑ य॒ज्ञो य॒ज्ञ्ः पाङ्क्तः॒ पाङ्क्तो॑ य॒ज्ञ्ः । \newline
5. य॒ज्ञो य॒ज्ञ्ं ॅय॒ज्ञ्ं ॅय॒ज्ञो य॒ज्ञो य॒ज्ञ्म् । \newline
6. य॒ज्ञ् मे॒वैव य॒ज्ञ्ं ॅय॒ज्ञ् मे॒व । \newline
7. ए॒वावा वै॒वै वाव॑ । \newline
8. अव॑ रुन्धे रु॒न्धे ऽवाव॑ रुन्धे । \newline
9. रु॒न्धे॒ पञ्च॒ पञ्च॑ रुन्धे रुन्धे॒ पञ्च॑ । \newline
10. पञ्च॒ कृत्वः॒ कृत्वः॒ पञ्च॒ पञ्च॒ कृत्वः॑ । \newline
11. कृत्व॑ स्तू॒ष्णीम् तू॒ष्णीम् कृत्वः॒ कृत्व॑ स्तू॒ष्णीम् । \newline
12. तू॒ष्णीम् दश॒ दश॑ तू॒ष्णीम् तू॒ष्णीम् दश॑ । \newline
13. दश॒ सꣳ सम् दश॒ दश॒ सम् । \newline
14. सम् प॑द्यन्ते पद्यन्ते॒ सꣳ सम् प॑द्यन्ते । \newline
15. प॒द्य॒न्ते॒ दशा᳚क्षरा॒ दशा᳚क्षरा पद्यन्ते पद्यन्ते॒ दशा᳚क्षरा । \newline
16. दशा᳚क्षरा वि॒राड् वि॒राड् दशा᳚क्षरा॒ दशा᳚क्षरा वि॒राट् । \newline
17. दशा᳚क्ष॒रेति॒ दश॑ - अ॒क्ष॒रा॒ । \newline
18. वि॒रा डन्न॒ मन्नं॑ ॅवि॒राड् वि॒रा डन्न᳚म् । \newline
19. वि॒राडिति॑ वि - राट् । \newline
20. अन्नं॑ ॅवि॒राड् वि॒रा डन्न॒ मन्नं॑ ॅवि॒राट् । \newline
21. वि॒राड् वि॒राजा॑ वि॒राजा॑ वि॒राड् वि॒राड् वि॒राजा᳚ । \newline
22. वि॒राडिति॑ वि - राट् । \newline
23. वि॒रा जै॒वैव वि॒राजा॑ वि॒रा जै॒व । \newline
24. वि॒राजेति॑ वि - राजा᳚ । \newline
25. ए॒वान्नाद्य॑ म॒न्नाद्य॑ मे॒वैवान्नाद्य᳚म् । \newline
26. अ॒न्नाद्य॒ मवा वा॒न्नाद्य॑ म॒न्नाद्य॒ मव॑ । \newline
27. अ॒न्नाद्य॒मित्य॑न्न - अद्य᳚म् । \newline
28. अव॑ रुन्धे रु॒न्धे ऽवाव॑ रुन्धे । \newline
29. रु॒न्धे॒ श्वा॒त्राः श्वा॒त्रा रु॑न्धे रुन्धे श्वा॒त्राः । \newline
30. श्वा॒त्राः स्थ॑ स्थ श्वा॒त्राः श्वा॒त्राः स्थ॑ । \newline
31. स्थ॒ वृ॒त्र॒तुरो॑ वृत्र॒तुरः॑ स्थ स्थ वृत्र॒तुरः॑ । \newline
32. वृ॒त्र॒तुर॒ इतीति॑ वृत्र॒तुरो॑ वृत्र॒तुर॒ इति॑ । \newline
33. वृ॒त्र॒तुर॒ इति॑ वृत्र - तुरः॑ । \newline
34. इत्या॑हा॒हे तीत्या॑ह । \newline
35. आ॒है॒ष ए॒ष आ॑हा है॒षः । \newline
36. ए॒ष वै वा ए॒ष ए॒ष वै । \newline
37. वा अ॒पा म॒पां ॅवै वा अ॒पाम् । \newline
38. अ॒पाꣳ सो॑मपी॒थः सो॑मपी॒थो॑ ऽपा म॒पाꣳ सो॑मपी॒थः । \newline
39. सो॒म॒पी॒थो यो यः सो॑मपी॒थः सो॑मपी॒थो यः । \newline
40. सो॒म॒पी॒थ इति॑ सोम - पी॒थः । \newline
41. य ए॒व मे॒वं ॅयो य ए॒वम् । \newline
42. ए॒वं ॅवेद॒ वेदै॒व मे॒वं ॅवेद॑ । \newline
43. वेद॒ न न वेद॒ वेद॒ न । \newline
44. नाफ्स्व॑फ्सु न नाफ्सु । \newline
45. अ॒फ्स्वार्ति॒ मार्ति॑ म॒फ्स्व॑ फ्स्वार्ति᳚म् । \newline
46. अ॒फ्स्वित्य॑प् - सु । \newline
47. आर्ति॒ मा ऽऽर्ति॒ मार्ति॒ मा । \newline
48. आर्च्छ॑ त्यृच्छ॒ त्यार्च्छ॑ति । \newline
49. ऋ॒च्छ॒ति॒ यद् यदृ॑च्छ त्यृच्छति॒ यत् । \newline
50. यत् ते॑ ते॒ यद् यत् ते᳚ । \newline
51. ते॒ सो॒म॒ सो॒म॒ ते॒ ते॒ सो॒म॒ । \newline
52. सो॒म॒ दि॒वि दि॒वि सो॑म सोम दि॒वि । \newline
53. दि॒वि ज्योति॒र् ज्योति॑र् दि॒वि दि॒वि ज्योतिः॑ । \newline
54. ज्योति॒ रितीति॒ ज्योति॒र् ज्योति॒ रिति॑ । \newline
55. इत्या॑हा॒हे तीत्या॑ह । \newline
56. आ॒है॒भ्य ए॒भ्य आ॑हा है॒भ्यः । \newline
57. ए॒भ्य ए॒वै वैभ्य ए॒भ्य ए॒व । \newline
58. ए॒वैन॑ मेन मे॒वै वैन᳚म् । \newline
59. ए॒न॒म् ॅलो॒केभ्यो॑ लो॒केभ्य॑ एन मेनम् ॅलो॒केभ्यः॑ । \newline

\textbf{Ghana Paata } \newline

1. पञ्चा᳚क्षरा प॒ङ्क्तिः प॒ङ्क्तिः पञ्चा᳚क्षरा॒ पञ्चा᳚क्षरा प॒ङ्क्तिः पाङ्क्तः॒ पाङ्क्तः॑ प॒ङ्क्तिः पञ्चा᳚क्षरा॒ पञ्चा᳚क्षरा प॒ङ्क्तिः पाङ्क्तः॑ । \newline
2. पञ्चा᳚क्ष॒रेति॒ पञ्च॑ - अ॒क्ष॒रा॒ । \newline
3. प॒ङ्क्तिः पाङ्क्तः॒ पाङ्क्तः॑ प॒ङ्क्तिः प॒ङ्क्तिः पाङ्क्तो॑ य॒ज्ञो य॒ज्ञ्ः पाङ्क्तः॑ प॒ङ्क्तिः प॒ङ्क्तिः पाङ्क्तो॑ य॒ज्ञ्ः । \newline
4. पाङ्क्तो॑ य॒ज्ञो य॒ज्ञ्ः पाङ्क्तः॒ पाङ्क्तो॑ य॒ज्ञो य॒ज्ञ्ं ॅय॒ज्ञ्ं ॅय॒ज्ञ्ः पाङ्क्तः॒ पाङ्क्तो॑ य॒ज्ञो य॒ज्ञ्म् । \newline
5. य॒ज्ञो य॒ज्ञ्ं ॅय॒ज्ञ्ं ॅय॒ज्ञो य॒ज्ञो य॒ज्ञ् मे॒वैव य॒ज्ञ्ं ॅय॒ज्ञो य॒ज्ञो य॒ज्ञ् मे॒व । \newline
6. य॒ज्ञ् मे॒वैव य॒ज्ञ्ं ॅय॒ज्ञ् मे॒वावा वै॒व य॒ज्ञ्ं ॅय॒ज्ञ् मे॒वाव॑ । \newline
7. ए॒वावा वै॒वै वाव॑ रुन्धे रु॒न्धे ऽवै॒वै वाव॑ रुन्धे । \newline
8. अव॑ रुन्धे रु॒न्धे ऽवाव॑ रुन्धे॒ पञ्च॒ पञ्च॑ रु॒न्धे ऽवाव॑ रुन्धे॒ पञ्च॑ । \newline
9. रु॒न्धे॒ पञ्च॒ पञ्च॑ रुन्धे रुन्धे॒ पञ्च॒ कृत्वः॒ कृत्वः॒ पञ्च॑ रुन्धे रुन्धे॒ पञ्च॒ कृत्वः॑ । \newline
10. पञ्च॒ कृत्वः॒ कृत्वः॒ पञ्च॒ पञ्च॒ कृत्व॑ स्तू॒ष्णीम् तू॒ष्णीम् कृत्वः॒ पञ्च॒ पञ्च॒ कृत्व॑ स्तू॒ष्णीम् । \newline
11. कृत्व॑ स्तू॒ष्णीम् तू॒ष्णीम् कृत्वः॒ कृत्व॑ स्तू॒ष्णीम् दश॒ दश॑ तू॒ष्णीम् कृत्वः॒ कृत्व॑ स्तू॒ष्णीम् दश॑ । \newline
12. तू॒ष्णीम् दश॒ दश॑ तू॒ष्णीम् तू॒ष्णीम् दश॒ सꣳ सम् दश॑ तू॒ष्णीम् तू॒ष्णीम् दश॒ सम् । \newline
13. दश॒ सꣳ सम् दश॒ दश॒ सम् प॑द्यन्ते पद्यन्ते॒ सम् दश॒ दश॒ सम् प॑द्यन्ते । \newline
14. सम् प॑द्यन्ते पद्यन्ते॒ सꣳ सम् प॑द्यन्ते॒ दशा᳚क्षरा॒ दशा᳚क्षरा पद्यन्ते॒ सꣳ सम् प॑द्यन्ते॒ दशा᳚क्षरा । \newline
15. प॒द्य॒न्ते॒ दशा᳚क्षरा॒ दशा᳚क्षरा पद्यन्ते पद्यन्ते॒ दशा᳚क्षरा वि॒राड् वि॒राड् दशा᳚क्षरा पद्यन्ते पद्यन्ते॒ दशा᳚क्षरा वि॒राट् । \newline
16. दशा᳚क्षरा वि॒राड् वि॒राड् दशा᳚क्षरा॒ दशा᳚क्षरा वि॒रा डन्न॒ मन्नं॑ ॅवि॒राड् दशा᳚क्षरा॒ दशा᳚क्षरा वि॒रा डन्न᳚म् । \newline
17. दशा᳚क्ष॒रेति॒ दश॑ - अ॒क्ष॒रा॒ । \newline
18. वि॒रा डन्न॒ मन्नं॑ ॅवि॒राड् वि॒रा डन्नं॑ ॅवि॒राड् वि॒रा डन्नं॑ ॅवि॒राड् वि॒रा डन्नं॑ ॅवि॒राट् । \newline
19. वि॒राडिति॑ वि - राट् । \newline
20. अन्नं॑ ॅवि॒राड् वि॒रा डन्न॒ मन्नं॑ ॅवि॒राड् वि॒राजा॑ वि॒राजा॑ वि॒रा डन्न॒ मन्नं॑ ॅवि॒राड् वि॒राजा᳚ । \newline
21. वि॒राड् वि॒राजा॑ वि॒राजा॑ वि॒राड् वि॒राड् वि॒राजै॒ वैव वि॒राजा॑ वि॒राड् वि॒राड् वि॒रा जै॒व । \newline
22. वि॒राडिति॑ वि - राट् । \newline
23. वि॒राजै॒ वैव वि॒राजा॑ वि॒राजै॒ वान्नाद्य॑ म॒न्नाद्य॑ मे॒व वि॒राजा॑ वि॒राजै॒ वान्नाद्य᳚म् । \newline
24. वि॒राजेति॑ वि - राजा᳚ । \newline
25. ए॒वान्नाद्य॑ म॒न्नाद्य॑ मे॒वै वान्नाद्य॒ मवा वा॒न्नाद्य॑ मे॒वै वान्नाद्य॒ मव॑ । \newline
26. अ॒न्नाद्य॒ मवा वा॒न्नाद्य॑ म॒न्नाद्य॒ मव॑ रुन्धे रु॒न्धे ऽवा॒न्नाद्य॑ म॒न्नाद्य॒ मव॑ रुन्धे । \newline
27. अ॒न्नाद्य॒मित्य॑न्न - अद्य᳚म् । \newline
28. अव॑ रुन्धे रु॒न्धे ऽवाव॑ रुन्धे श्वा॒त्राः श्वा॒त्रा रु॒न्धे ऽवाव॑ रुन्धे श्वा॒त्राः । \newline
29. रु॒न्धे॒ श्वा॒त्राः श्वा॒त्रा रु॑न्धे रुन्धे श्वा॒त्राः स्थ॑ स्थ श्वा॒त्रा रु॑न्धे रुन्धे श्वा॒त्राः स्थ॑ । \newline
30. श्वा॒त्राः स्थ॑ स्थ श्वा॒त्राः श्वा॒त्राः स्थ॑ वृत्र॒तुरो॑ वृत्र॒तुरः॑ स्थ श्वा॒त्राः श्वा॒त्राः स्थ॑ वृत्र॒तुरः॑ । \newline
31. स्थ॒ वृ॒त्र॒तुरो॑ वृत्र॒तुरः॑ स्थ स्थ वृत्र॒तुर॒ इतीति॑ वृत्र॒तुरः॑ स्थ स्थ वृत्र॒तुर॒ इति॑ । \newline
32. वृ॒त्र॒तुर॒ इतीति॑ वृत्र॒तुरो॑ वृत्र॒तुर॒ इत्या॑हा॒ हेति॑ वृत्र॒तुरो॑ वृत्र॒तुर॒ इत्या॑ह । \newline
33. वृ॒त्र॒तुर॒ इति॑ वृत्र - तुरः॑ । \newline
34. इत्या॑हा॒हे तीत्या॑ है॒ष ए॒ष आ॒हे तीत्या॑ है॒षः । \newline
35. आ॒है॒ष ए॒ष आ॑हा है॒ष वै वा ए॒ष आ॑हा है॒ष वै । \newline
36. ए॒ष वै वा ए॒ष ए॒ष वा अ॒पा म॒पां ॅवा ए॒ष ए॒ष वा अ॒पाम् । \newline
37. वा अ॒पा म॒पां ॅवै वा अ॒पाꣳ सो॑मपी॒थः सो॑मपी॒थो॑ ऽपां ॅवै वा अ॒पाꣳ सो॑मपी॒थः । \newline
38. अ॒पाꣳ सो॑मपी॒थः सो॑मपी॒थो॑ ऽपा म॒पाꣳ सो॑मपी॒थो यो यः सो॑मपी॒थो॑ ऽपा म॒पाꣳ सो॑मपी॒थो यः । \newline
39. सो॒म॒पी॒थो यो यः सो॑मपी॒थः सो॑मपी॒थो य ए॒व मे॒वं ॅयः सो॑मपी॒थः सो॑मपी॒थो य ए॒वम् । \newline
40. सो॒म॒पी॒थ इति॑ सोम - पी॒थः । \newline
41. य ए॒व मे॒वं ॅयो य ए॒वं ॅवेद॒ वेदै॒वं ॅयो य ए॒वं ॅवेद॑ । \newline
42. ए॒वं ॅवेद॒ वेदै॒व मे॒वं ॅवेद॒ न न वेदै॒व मे॒वं ॅवेद॒ न । \newline
43. वेद॒ न न वेद॒ वेद॒ नाफ्स्व॑फ्सु न वेद॒ वेद॒ नाफ्सु । \newline
44. नाफ्स्व॑फ्सु न नाफ्स्वार्ति॒ मार्ति॑ म॒फ्सु न नाफ्स्वार्ति᳚म् । \newline
45. अ॒फ्स्वार्ति॒ मार्ति॑ म॒फ्स्व॑ फ्स्वार्ति॒ मा ऽऽर्ति॑ म॒फ्स्व॑ फ्स्वार्ति॒ मा । \newline
46. अ॒फ्स्वित्य॑प् - सु । \newline
47. आर्ति॒ मा ऽऽर्ति॒ मार्ति॒ मार्च्छ॑ त्यृच्छ॒त्या ऽऽर्ति॒ मार्ति॒ मार्च्छ॑ति । \newline
48. आर्च्छ॑त्यृच्छ॒ त्यार्च्छ॑ति॒ यद् यदृ॑च्छ॒ त्यार्च्छ॑ति॒ यत् । \newline
49. ऋ॒च्छ॒ति॒ यद् यदृ॑च्छ त्यृच्छति॒ यत् ते॑ ते॒ यदृ॑च्छ त्यृच्छति॒ यत् ते᳚ । \newline
50. यत् ते॑ ते॒ यद् यत् ते॑ सोम सोम ते॒ यद् यत् ते॑ सोम । \newline
51. ते॒ सो॒म॒ सो॒म॒ ते॒ ते॒ सो॒म॒ दि॒वि दि॒वि सो॑म ते ते सोम दि॒वि । \newline
52. सो॒म॒ दि॒वि दि॒वि सो॑म सोम दि॒वि ज्योति॒र् ज्योति॑र् दि॒वि सो॑म सोम दि॒वि ज्योतिः॑ । \newline
53. दि॒वि ज्योति॒र् ज्योति॑र् दि॒वि दि॒वि ज्योति॒ रितीति॒ ज्योति॑र् दि॒वि दि॒वि ज्योति॒ रिति॑ । \newline
54. ज्योति॒रि तीति॒ ज्योति॒र् ज्योति॒रि त्या॑हा॒ हेति॒ ज्योति॒र् ज्योति॒रि त्या॑ह । \newline
55. इत्या॑हा॒हे तीत्या॑ है॒भ्य ए॒भ्य आ॒हे तीत्या॑ है॒भ्यः । \newline
56. आ॒है॒भ्य ए॒भ्य आ॑हा है॒भ्य ए॒वै वैभ्य आ॑हा है॒भ्य ए॒व । \newline
57. ए॒भ्य ए॒वै वैभ्य ए॒भ्य ए॒वैन॑ मेन मे॒वैभ्य ए॒भ्य ए॒वैन᳚म् । \newline
58. ए॒वैन॑ मेन मे॒वै वैन॑म् ॅलो॒केभ्यो॑ लो॒केभ्य॑ एन मे॒वै वैन॑म् ॅलो॒केभ्यः॑ । \newline
59. ए॒न॒म् ॅलो॒केभ्यो॑ लो॒केभ्य॑ एन मेनम् ॅलो॒केभ्यः॒ सꣳ सम् ॅलो॒केभ्य॑ एन मेनम् ॅलो॒केभ्यः॒ सम् । \newline
\pagebreak
\markright{ TS 6.4.4.3  \hfill https://www.vedavms.in \hfill}

\section{ TS 6.4.4.3 }

\textbf{TS 6.4.4.3 } \newline
\textbf{Samhita Paata} \newline

ॅलो॒केभ्यः॒ सं भ॑रति॒ सोमो॒ वै राजा॒ दिशो॒ऽभ्य॑द्ध्याय॒थ् स दिशोऽनु॒ प्रावि॑श॒त् प्रागपा॒गुद॑गध॒रागित्या॑ह दि॒ग्भ्य ए॒वैनꣳ॒॒ सं भ॑र॒त्यथो॒ दिश॑ ए॒वास्मा॒ अव॑ रु॒न्धे ऽम्ब॒ नि ष्व॒रेत्या॑ह॒ कामु॑का एनꣳ॒॒ स्त्रियो॑ भवन्ति॒ य ए॒वं ॅवेद॒ यत् ते॑ सो॒मादा᳚भ्यं॒ नाम॒ जागृ॒वीत्या॑- [  ] \newline

\textbf{Pada Paata} \newline

लो॒केभ्यः॑ । समिति॑ । भ॒र॒ति॒ । सोमः॑ । वै । राजा᳚ । दिशः॑ । अ॒भीति॑ । अ॒द्ध्या॒य॒त् । सः । दिशः॑ । अनु॑ । प्रेति॑ । अ॒वि॒श॒त् । प्राक् । अपा᳚क् । उद॑क् । अ॒ध॒राक् । इति॑ । आ॒ह॒ । दि॒ग्भ्य इति॑ दिक् - भ्यः । ए॒व । ए॒न॒म् । समिति॑ । भ॒र॒ति॒ । अथो॒ इति॑ । दिशः॑ । ए॒व । अ॒स्मै॒ । अवेति॑ । रु॒न्धे॒ । अम्ब॑ । नीति॑ । स्व॒र॒ । इति॑ । आ॒ह॒ । कामु॑काः । ए॒न॒म् । स्त्रियः॑ । भ॒व॒न्ति॒ । यः । ए॒वम् । वेद॑ । यत् । ते॒ । सो॒म॒ । अदा᳚भ्यम् । नाम॑ । जागृ॑वि । इति॑ ।  \newline


\textbf{Krama Paata} \newline

लो॒केभ्यः॒ सम् । सम् भ॑रति । भ॒र॒ति॒ सोमः॑ । सोमो॒ वै । वै राजा᳚ । राजा॒ दिशः॑ । दिशो॒ऽभि । अ॒भ्य॑द्ध्यायत् । अ॒द्ध्या॒य॒थ् सः । स दिशः॑ । दिशोऽनु॑ । अनु॒ प्र । प्रावि॑शत् । अ॒वि॒श॒त् प्राक् । प्रागपा᳚क् । अपा॒गुद॑क् । उद॑गध॒राक् । अ॒ध॒रागिति॑ । इत्या॑ह । आ॒ह॒ दि॒ग्भ्यः । दि॒ग्भ्य ए॒व । दि॒ग्भ्य इति॑ दिक् - भ्यः । ए॒वैन᳚म् । ए॒नꣳ॒॒ सम् । सम् भ॑रति । भ॒र॒त्यथो᳚ । अथो॒ दिशः॑ । अथो॒ इत्यथो᳚ । दिश॑ ए॒व । ए॒वास्मै᳚ । अ॒स्मा॒ अव॑ । अव॑ रुन्धे । रु॒न्धेऽम्ब॑ । अम्ब॒ नि । नि ष्व॑र । स्व॒रेति॑ । इत्या॑ह । आ॒ह॒ कामु॑काः । कामु॑का एनम् । ए॒नꣳ॒॒ स्त्रियः॑ । स्त्रियो॑ भवन्ति । भ॒व॒न्ति॒ यः । य ए॒वम् । ए॒वम् ॅवेद॑ । वेद॒ यत् । यत् ते᳚ । ते॒ सो॒म॒ । सो॒मादा᳚भ्यम् । अदा᳚भ्य॒म् नाम॑ । नाम॒ जागृ॑वि । जागृ॒वीति॑ ( ) । इत्या॑ह \newline

\textbf{Jatai Paata} \newline

1. लो॒केभ्यः॒ सꣳ सम् ॅलो॒केभ्यो॑ लो॒केभ्यः॒ सम् । \newline
2. सम् भ॑रति भरति॒ सꣳ सम् भ॑रति । \newline
3. भ॒र॒ति॒ सोमः॒ सोमो॑ भरति भरति॒ सोमः॑ । \newline
4. सोमो॒ वै वै सोमः॒ सोमो॒ वै । \newline
5. वै राजा॒ राजा॒ वै वै राजा᳚ । \newline
6. राजा॒ दिशो॒ दिशो॒ राजा॒ राजा॒ दिशः॑ । \newline
7. दिशो॒ ऽभ्य॑भि दिशो॒ दिशो॒ ऽभि । \newline
8. अ॒भ्य॑ द्ध्याय दद्ध्याय द॒भ्या᳚(1॒)भ्य॑ द्ध्यायत् । \newline
9. अ॒द्ध्या॒य॒थ् स सो᳚ ऽद्ध्याय दद्ध्याय॒थ् सः । \newline
10. स दिशो॒ दिशः॒ स स दिशः॑ । \newline
11. दिशो ऽन्वनु॒ दिशो॒ दिशो ऽनु॑ । \newline
12. अनु॒ प्र प्राण्वनु॒ प्र । \newline
13. प्रावि॑श दविश॒त् प्र प्रावि॑शत् । \newline
14. अ॒वि॒श॒त् प्राक् प्राग॑विश दविश॒त् प्राक् । \newline
15. प्रा गपा॒ गपा॒क् प्राक् प्रा गपा᳚क् । \newline
16. अपा॒ गुद॒ गुद॒ गपा॒ गपा॒ गुद॑क् । \newline
17. उद॑ गध॒रा ग॑ध॒रा गुद॒ गुद॑ गध॒राक् । \newline
18. अ॒ध॒रा गिती त्य॑ध॒रा ग॑ध॒रा गिति॑ । \newline
19. इत्या॑हा॒हे तीत्या॑ह । \newline
20. आ॒ह॒ दि॒ग्भ्यो दि॒ग्भ्य आ॑हाह दि॒ग्भ्यः । \newline
21. दि॒ग्भ्य ए॒वैव दि॒ग्भ्यो दि॒ग्भ्य ए॒व । \newline
22. दि॒ग्भ्य इति॑ दिक् - भ्यः । \newline
23. ए॒वैन॑ मेन मे॒वै वैन᳚म् । \newline
24. ए॒नꣳ॒॒ सꣳ स मे॑न मेनꣳ॒॒ सम् । \newline
25. सम् भ॑रति भरति॒ सꣳ सम् भ॑रति । \newline
26. भ॒र॒ त्यथो॒ अथो॑ भरति भर॒ त्यथो᳚ । \newline
27. अथो॒ दिशो॒ दिशो ऽथो॒ अथो॒ दिशः॑ । \newline
28. अथो॒ इत्यथो᳚ । \newline
29. दिश॑ ए॒वैव दिशो॒ दिश॑ ए॒व । \newline
30. ए॒वास्मा॑ अस्मा ए॒वै वास्मै᳚ । \newline
31. अ॒स्मा॒ अवावा᳚स्मा अस्मा॒ अव॑ । \newline
32. अव॑ रुन्धे रु॒न्धे ऽवाव॑ रुन्धे । \newline
33. रु॒न्धे ऽम्बाम्ब॑ रुन्धे रु॒न्धे ऽम्ब॑ । \newline
34. अम्ब॒ नि न्यम्बाम्ब॒ नि । \newline
35. नि ष्व॑र स्वर॒ नि नि ष्व॑र । \newline
36. स्व॒रेतीति॑ स्वर स्व॒रेति॑ । \newline
37. इत्या॑हा॒हे तीत्या॑ह । \newline
38. आ॒ह॒ कामु॑काः॒ कामु॑का आहाह॒ कामु॑काः । \newline
39. कामु॑का एन मेन॒म् कामु॑काः॒ कामु॑का एनम् । \newline
40. ए॒नꣳ॒॒ स्त्रियः॒ स्त्रिय॑ एन मेनꣳ॒॒ स्त्रियः॑ । \newline
41. स्त्रियो॑ भवन्ति भवन्ति॒ स्त्रियः॒ स्त्रियो॑ भवन्ति । \newline
42. भ॒व॒न्ति॒ यो यो भ॑वन्ति भवन्ति॒ यः । \newline
43. य ए॒व मे॒वं ॅयो य ए॒वम् । \newline
44. ए॒वं ॅवेद॒ वेदै॒व मे॒वं ॅवेद॑ । \newline
45. वेद॒ यद् यद् वेद॒ वेद॒ यत् । \newline
46. यत् ते॑ ते॒ यद् यत् ते᳚ । \newline
47. ते॒ सो॒म॒ सो॒म॒ ते॒ ते॒ सो॒म॒ । \newline
48. सो॒मादा᳚भ्य॒ मदा᳚भ्यꣳ सोम सो॒मादा᳚भ्यम् । \newline
49. अदा᳚भ्य॒म् नाम॒ नामादा᳚भ्य॒ मदा᳚भ्य॒म् नाम॑ । \newline
50. नाम॒ जागृ॑वि॒ जागृ॑वि॒ नाम॒ नाम॒ जागृ॑वि । \newline
51. जागृ॒वीतीति॒ जागृ॑वि॒ जागृ॒वीति॑ । \newline
52. इत्या॑हा॒हे तीत्या॑ह । \newline

\textbf{Ghana Paata } \newline

1. लो॒केभ्यः॒ सꣳ सम् ॅलो॒केभ्यो॑ लो॒केभ्यः॒ सम् भ॑रति भरति॒ सम् ॅलो॒केभ्यो॑ लो॒केभ्यः॒ सम् भ॑रति । \newline
2. सम् भ॑रति भरति॒ सꣳ सम् भ॑रति॒ सोमः॒ सोमो॑ भरति॒ सꣳ सम् भ॑रति॒ सोमः॑ । \newline
3. भ॒र॒ति॒ सोमः॒ सोमो॑ भरति भरति॒ सोमो॒ वै वै सोमो॑ भरति भरति॒ सोमो॒ वै । \newline
4. सोमो॒ वै वै सोमः॒ सोमो॒ वै राजा॒ राजा॒ वै सोमः॒ सोमो॒ वै राजा᳚ । \newline
5. वै राजा॒ राजा॒ वै वै राजा॒ दिशो॒ दिशो॒ राजा॒ वै वै राजा॒ दिशः॑ । \newline
6. राजा॒ दिशो॒ दिशो॒ राजा॒ राजा॒ दिशो॒ ऽभ्य॑भि दिशो॒ राजा॒ राजा॒ दिशो॒ ऽभि । \newline
7. दिशो॒ ऽभ्य॑भि दिशो॒ दिशो॒ ऽभ्य॑द्ध्याय दद्ध्याय द॒भि दिशो॒ दिशो॒ ऽभ्य॑द्ध्यायत् । \newline
8. अ॒भ्य॑द्ध्याय दद्ध्याय द॒भ्या᳚(1॒)भ्य॑ द्ध्याय॒थ् स सो᳚ ऽद्ध्याय द॒भ्या᳚(1॒)भ्य॑द्ध्याय॒थ् सः । \newline
9. अ॒द्ध्या॒य॒थ् स सो᳚ ऽद्ध्याय दद्ध्याय॒थ् स दिशो॒ दिशः॒ सो᳚ ऽद्ध्याय दद्ध्याय॒थ् स दिशः॑ । \newline
10. स दिशो॒ दिशः॒ स स दिशो ऽन्वनु॒ दिशः॒ स स दिशो ऽनु॑ । \newline
11. दिशो ऽन्वनु॒ दिशो॒ दिशो ऽनु॒ प्र प्राणु॒ दिशो॒ दिशो ऽनु॒ प्र । \newline
12. अनु॒ प्र प्राण्वनु॒ प्रावि॑श दविश॒त् प्राण्वनु॒ प्रावि॑शत् । \newline
13. प्रावि॑श दविश॒त् प्र प्रावि॑श॒त् प्राक् प्रा ग॑विश॒त् प्र प्रावि॑श॒त् प्राक् । \newline
14. अ॒वि॒श॒त् प्राक् प्रा ग॑विश दविश॒त् प्रा गपा॒ गपा॒क् प्रा ग॑विश दविश॒त् प्रा गपा᳚क् । \newline
15. प्रा गपा॒ गपा॒क् प्राक् प्रागपा॒ गुद॒ गुद॒ गपा॒क् प्राक् प्रा गपा॒ गुद॑क् । \newline
16. अपा॒ गुद॒ गुद॒ गपा॒ गपा॒ गुद॑ गध॒रा ग॑ध॒रा गुद॒ गपा॒ गपा॒ गुद॑ गध॒राक् । \newline
17. उद॑ गध॒रा ग॑ध॒रा गुद॒ गुद॑ गध॒रा गिती त्य॑ध॒रा गुद॒ गुद॑ गध॒रा गिति॑ । \newline
18. अ॒ध॒रागि तीत्य॑ध॒रा ग॑ध॒रा गित्या॑हा॒हे त्य॑ध॒रा ग॑ध॒रा गित्या॑ह । \newline
19. इत्या॑हा॒हे तीत्या॑ह दि॒ग्भ्यो दि॒ग्भ्य आ॒हे तीत्या॑ह दि॒ग्भ्यः । \newline
20. आ॒ह॒ दि॒ग्भ्यो दि॒ग्भ्य आ॑हाह दि॒ग्भ्य ए॒वैव दि॒ग्भ्य आ॑हाह दि॒ग्भ्य ए॒व । \newline
21. दि॒ग्भ्य ए॒वैव दि॒ग्भ्यो दि॒ग्भ्य ए॒वैन॑ मेन मे॒व दि॒ग्भ्यो दि॒ग्भ्य ए॒वैन᳚म् । \newline
22. दि॒ग्भ्य इति॑ दिक् - भ्यः । \newline
23. ए॒वैन॑ मेन मे॒वै वैनꣳ॒॒ सꣳ स मे॑न मे॒वै वैनꣳ॒॒ सम् । \newline
24. ए॒नꣳ॒॒ सꣳ स मे॑न मेनꣳ॒॒ सम् भ॑रति भरति॒ स मे॑न मेनꣳ॒॒ सम् भ॑रति । \newline
25. सम् भ॑रति भरति॒ सꣳ सम् भ॑र॒ त्यथो॒ अथो॑ भरति॒ सꣳ सम् भ॑र॒ त्यथो᳚ । \newline
26. भ॒र॒ त्यथो॒ अथो॑ भरति भर॒ त्यथो॒ दिशो॒ दिशो ऽथो॑ भरति भर॒ त्यथो॒ दिशः॑ । \newline
27. अथो॒ दिशो॒ दिशो ऽथो॒ अथो॒ दिश॑ ए॒वैव दिशो ऽथो॒ अथो॒ दिश॑ ए॒व । \newline
28. अथो॒ इत्यथो᳚ । \newline
29. दिश॑ ए॒वैव दिशो॒ दिश॑ ए॒वास्मा॑ अस्मा ए॒व दिशो॒ दिश॑ ए॒वास्मै᳚ । \newline
30. ए॒वास्मा॑ अस्मा ए॒वै वास्मा॒ अवावा᳚स्मा ए॒वै वास्मा॒ अव॑ । \newline
31. अ॒स्मा॒ अवावा᳚स्मा अस्मा॒ अव॑ रुन्धे रु॒न्धे ऽवा᳚स्मा अस्मा॒ अव॑ रुन्धे । \newline
32. अव॑ रुन्धे रु॒न्धे ऽवाव॑ रु॒न्धे ऽम्बाम्ब॑ रु॒न्धे ऽवाव॑ रु॒न्धे ऽम्ब॑ । \newline
33. रु॒न्धे ऽम्बाम्ब॑ रुन्धे रु॒न्धे ऽम्ब॒ नि न्यम्ब॑ रुन्धे रु॒न्धे ऽम्ब॒ नि । \newline
34. अम्ब॒ नि न्यम्बा म्ब॒ नि ष्व॑र स्वर॒ न्यम्बा म्ब॒ नि ष्व॑र । \newline
35. नि ष्व॑र स्वर॒ नि नि ष्व॒रे तीति॑ स्वर॒ नि नि ष्व॒रेति॑ । \newline
36. स्व॒रे तीति॑ स्वर स्व॒रे त्या॑हा॒ हेति॑ स्वर स्व॒रे त्या॑ह । \newline
37. इत्या॑हा॒हे तीत्या॑ह॒ कामु॑काः॒ कामु॑का आ॒हे तीत्या॑ह॒ कामु॑काः । \newline
38. आ॒ह॒ कामु॑काः॒ कामु॑का आहाह॒ कामु॑का एन मेन॒म् कामु॑का आहाह॒ कामु॑का एनम् । \newline
39. कामु॑का एन मेन॒म् कामु॑काः॒ कामु॑का एनꣳ॒॒ स्त्रियः॒ स्त्रिय॑ एन॒म् कामु॑काः॒ कामु॑का एनꣳ॒॒ स्त्रियः॑ । \newline
40. ए॒नꣳ॒॒ स्त्रियः॒ स्त्रिय॑ एन मेनꣳ॒॒ स्त्रियो॑ भवन्ति भवन्ति॒ स्त्रिय॑ एन मेनꣳ॒॒ स्त्रियो॑ भवन्ति । \newline
41. स्त्रियो॑ भवन्ति भवन्ति॒ स्त्रियः॒ स्त्रियो॑ भवन्ति॒ यो यो भ॑वन्ति॒ स्त्रियः॒ स्त्रियो॑ भवन्ति॒ यः । \newline
42. भ॒व॒न्ति॒ यो यो भ॑वन्ति भवन्ति॒ य ए॒व मे॒वं ॅयो भ॑वन्ति भवन्ति॒ य ए॒वम् । \newline
43. य ए॒व मे॒वं ॅयो य ए॒वं ॅवेद॒ वेदै॒वं ॅयो य ए॒वं ॅवेद॑ । \newline
44. ए॒वं ॅवेद॒ वेदै॒व मे॒वं ॅवेद॒ यद् यद् वेदै॒व मे॒वं ॅवेद॒ यत् । \newline
45. वेद॒ यद् यद् वेद॒ वेद॒ यत् ते॑ ते॒ यद् वेद॒ वेद॒ यत् ते᳚ । \newline
46. यत् ते॑ ते॒ यद् यत् ते॑ सोम सोम ते॒ यद् यत् ते॑ सोम । \newline
47. ते॒ सो॒म॒ सो॒म॒ ते॒ ते॒ सो॒मादा᳚भ्य॒ मदा᳚भ्यꣳ सोम ते ते सो॒मादा᳚भ्यम् । \newline
48. सो॒मादा᳚भ्य॒ मदा᳚भ्यꣳ सोम सो॒मादा᳚भ्य॒म् नाम॒ नामादा᳚भ्यꣳ सोम सो॒मादा᳚भ्य॒म् नाम॑ । \newline
49. अदा᳚भ्य॒म् नाम॒ नामादा᳚भ्य॒ मदा᳚भ्य॒म् नाम॒ जागृ॑वि॒ जागृ॑वि॒ नामादा᳚भ्य॒ मदा᳚भ्य॒म् नाम॒ जागृ॑वि । \newline
50. नाम॒ जागृ॑वि॒ जागृ॑वि॒ नाम॒ नाम॒ जागृ॒वी तीति॒ जागृ॑वि॒ नाम॒ नाम॒ जागृ॒ वीति॑ । \newline
51. जागृ॒वी तीति॒ जागृ॑वि॒ जागृ॒वी त्या॑हा॒ हेति॒ जागृ॑वि॒ जागृ॒वी त्या॑ह । \newline
52. इत्या॑हा॒हे तीत्या॑ है॒ष ए॒ष आ॒हे तीत्या॑ है॒षः । \newline
\pagebreak
\markright{ TS 6.4.4.4  \hfill https://www.vedavms.in \hfill}

\section{ TS 6.4.4.4 }

\textbf{TS 6.4.4.4 } \newline
\textbf{Samhita Paata} \newline

-है॒ष वै सोम॑स्य सोमपी॒थो य ए॒वं ॅवेद॒ न सौ॒म्यामार्ति॒मार्च्छ॑ति॒ घ्नन्ति॒ वा ए॒तथ् सोमं॒ ॅयद॑भिषु॒ण्वन्त्यꣳ॒॒ शूनप॑ गृह्णाति॒ त्राय॑त ए॒वैनं॑ प्रा॒णा वा अꣳ॒॒शवः॑ प॒शवः॒ सोमो॒ ऽꣳ॒शून् पुन॒रपि॑ सृजति प्रा॒णाने॒व प॒शुषु॑ दधाति॒ द्वौद्वा॒वपि॑ सृजति॒ तस्मा॒द् द्वौद्वौ᳚ प्रा॒णाः ॥ \newline

\textbf{Pada Paata} \newline

आ॒ह॒ । ए॒षः । वै । सोम॑स्य । सो॒म॒पी॒थ इति॑ सोम - पी॒थः । यः । ए॒वम् । वेद॑ । न । सौ॒म्याम् । आर्ति᳚म् । एति॑ । ऋ॒च्छ॒ति॒ । घ्नन्ति॑ । वै । ए॒तत् । सोम᳚म् । यत् । अ॒भि॒षु॒ण्वन्तीत्य॑भि - सु॒न्वन्ति॑ । अꣳ॒॒शून् । अपेति॑ । गृ॒ह्णा॒ति॒ । त्राय॑ते । ए॒व । ए॒न॒म् । प्रा॒णा इति॑ प्र - अ॒नाः । वै । अꣳ॒॒शवः॑ । प॒शवः॑ । सोमः॑ । अꣳ॒॒शून् । पुनः॑ । अपीति॑ । सृ॒ज॒ति॒ । प्रा॒णानिति॑ प्र - अ॒नान् । ए॒व । प॒शुषु॑ । द॒धा॒ति॒ । द्वौद्वा॒विति॒ द्वौ - द्वौ॒ । अपीति॑ । सृ॒ज॒ति॒ । तस्मा᳚त् । द्वौद्वा॒विति॒ द्वौ - द्वौ॒ । प्रा॒णा इति॑ प्र - अ॒नाः ॥  \newline


\textbf{Krama Paata} \newline

आ॒है॒षः । ए॒ष वै । वै सोम॑स्य । सोम॑स्य सोमपी॒थः । सो॒म॒पी॒थो यः । सो॒म॒पी॒थ इति॑ सोम - पी॒थः । य ए॒वम् । ए॒वम् ॅवेद॑ । वेद॒ न । न सौ॒म्याम् । सौ॒म्यामार्ति᳚म् । आर्ति॒मा । आर्च्छ॑ति । ऋ॒च्छ॒ति॒ घ्नन्ति॑ । घ्नन्ति॒ वै । वा ए॒तत् । ए॒तथ् सोम᳚म् । सोम॒म् ॅयत् । यद॑भिषु॒ण्वन्ति॑ । अ॒भि॒षु॒ण्वन्त्यꣳ॒॒शून् । अ॒भि॒षु॒ण्वन्तीत्य॑भि - सु॒न्वन्ति॑ । अꣳ॒॒शूनप॑ । अप॑ गृह्णाति । गृ॒ह्णा॒ति॒ त्राय॑ते । त्राय॑त ए॒व । ए॒वैन᳚म् । ए॒न॒म् प्रा॒णाः । प्रा॒णा वै । प्रा॒णा इति॑ प्र - अ॒नाः । वा अꣳ॒॒शवः॑ । अꣳ॒॒शवः॑ प॒शवः॑ । प॒शवः॒ सोमः॑ । सोमो॒ऽꣳ॒शून् । अꣳ॒॒शून् पुनः॑ । पुन॒रपि॑ । अपि॑ सृजति । सृ॒ज॒ति॒ प्रा॒णान् । प्रा॒णाने॒व । प्रा॒णानिति॑ प्र - अ॒नान् । ए॒व प॒शुषु॑ । प॒शुषु॑ दधाति । द॒धा॒ति॒ द्वौद्वौ᳚ । द्वौद्वा॒वपि॑ । द्वौद्वा॒विति॒ द्वौ - द्वौ॒ । अपि॑ सृजति । सृ॒ज॒ति॒ तस्मा᳚त् । तस्मा॒द् द्वौद्वौ᳚ । द्वौद्वौ᳚ प्रा॒णाः । द्वौद्वा॒विति॒ द्वौ - द्वौ॒ । प्रा॒णा इति॑ प्र - अ॒नाः । \newline

\textbf{Jatai Paata} \newline

1. आ॒है॒ष ए॒ष आ॑हा है॒षः । \newline
2. ए॒ष वै वा ए॒ष ए॒ष वै । \newline
3. वै सोम॑स्य॒ सोम॑स्य॒ वै वै सोम॑स्य । \newline
4. सोम॑स्य सोमपी॒थः सो॑मपी॒थः सोम॑स्य॒ सोम॑स्य सोमपी॒थः । \newline
5. सो॒म॒पी॒थो यो यः सो॑मपी॒थः सो॑मपी॒थो यः । \newline
6. सो॒म॒पी॒थ इति॑ सोम - पी॒थः । \newline
7. य ए॒व मे॒वं ॅयो य ए॒वम् । \newline
8. ए॒वं ॅवेद॒ वेदै॒व मे॒वं ॅवेद॑ । \newline
9. वेद॒ न न वेद॒ वेद॒ न । \newline
10. न सौ॒म्याꣳ सौ॒म्यान् न न सौ॒म्याम् । \newline
11. सौ॒म्या मार्ति॒ मार्तिꣳ॑ सौ॒म्याꣳ सौ॒म्या मार्ति᳚म् । \newline
12. आर्ति॒ मा ऽऽर्ति॒ मार्ति॒ मा । \newline
13. आर्च्छ॑ त्यृच्छ॒ त्यार्च्छ॑ति । \newline
14. ऋ॒च्छ॒ति॒ घ्नन्ति॒ घ्नन् त्यृ॑च्छ त्यृच्छति॒ घ्नन्ति॑ । \newline
15. घ्नन्ति॒ वै वै घ्नन्ति॒ घ्नन्ति॒ वै । \newline
16. वा ए॒त दे॒तद् वै वा ए॒तत् । \newline
17. ए॒तथ् सोमꣳ॒॒ सोम॑ मे॒त दे॒तथ् सोम᳚म् । \newline
18. सोमं॒ ॅयद् यथ् सोमꣳ॒॒ सोमं॒ ॅयत् । \newline
19. यद॑भिषु॒ण्व न्त्य॑भिषु॒ण्वन्ति॒ यद् यद॑भिषु॒ण्वन्ति॑ । \newline
20. अ॒भि॒षु॒ण्व न्त्यꣳ॒॒शू नꣳ॒॒शू न॑भिषु॒ण्वन् त्य॑भिषु॒ण्व न्त्यꣳ॒॒शून् । \newline
21. अ॒भि॒षु॒ण्वन्तीत्य॑भि - सु॒न्वन्ति॑ । \newline
22. अꣳ॒॒शू नपापाꣳ॒॒शू नꣳ॒॒शू नप॑ । \newline
23. अप॑ गृह्णाति गृह्णा॒ त्यपाप॑ गृह्णाति । \newline
24. गृ॒ह्णा॒ति॒ त्राय॑ते॒ त्राय॑ते गृह्णाति गृह्णाति॒ त्राय॑ते । \newline
25. त्राय॑त ए॒वैव त्राय॑ते॒ त्राय॑त ए॒व । \newline
26. ए॒वैन॑ मेन मे॒वै वैन᳚म् । \newline
27. ए॒न॒म् प्रा॒णाः प्रा॒णा ए॑न मेनम् प्रा॒णाः । \newline
28. प्रा॒णा वै वै प्रा॒णाः प्रा॒णा वै । \newline
29. प्रा॒णा इति॑ प्र - अ॒नाः । \newline
30. वा अꣳ॒॒शवो॒ ऽꣳ॒शवो॒ वै वा अꣳ॒॒शवः॑ । \newline
31. अꣳ॒॒शवः॑ प॒शवः॑ प॒शवो॒ ऽꣳ॒शवो॒ ऽꣳ॒शवः॑ प॒शवः॑ । \newline
32. प॒शवः॒ सोमः॒ सोमः॑ प॒शवः॑ प॒शवः॒ सोमः॑ । \newline
33. सोमो॒ ऽꣳ॒शू नꣳ॒॒शून् थ्सोमः॒ सोमो॒ ऽꣳ॒शून् । \newline
34. अꣳ॒॒शून् पुनः॒ पुन॑ रꣳ॒॒शू नꣳ॒॒शून् पुनः॑ । \newline
35. पुन॒ रप्यपि॒ पुनः॒ पुन॒ रपि॑ । \newline
36. अपि॑ सृजति सृज॒ त्यप्यपि॑ सृजति । \newline
37. सृ॒ज॒ति॒ प्रा॒णान् प्रा॒णान् थ्सृ॑जति सृजति प्रा॒णान् । \newline
38. प्रा॒णा ने॒वैव प्रा॒णान् प्रा॒णाने॒व । \newline
39. प्रा॒णानिति॑ प्र - अ॒नान् । \newline
40. ए॒व प॒शुषु॑ प॒शु ष्वे॒वैव प॒शुषु॑ । \newline
41. प॒शुषु॑ दधाति दधाति प॒शुषु॑ प॒शुषु॑ दधाति । \newline
42. द॒धा॒ति॒ द्वौद्वौ॒ द्वौद्वौ॑ दधाति दधाति॒ द्वौद्वौ᳚ । \newline
43. द्वौद्वा॒ वप्यपि॒ द्वौद्वौ॒ द्वौद्वा॒ वपि॑ । \newline
44. द्वौद्वा॒विति॒ द्वौ - द्वौ॒ । \newline
45. अपि॑ सृजति सृज॒ त्यप्यपि॑ सृजति । \newline
46. सृ॒ज॒ति॒ तस्मा॒त् तस्मा᳚थ् सृजति सृजति॒ तस्मा᳚त् । \newline
47. तस्मा॒द् द्वौद्वौ॒ द्वौद्वौ॒ तस्मा॒त् तस्मा॒द् द्वौद्वौ᳚ । \newline
48. द्वौद्वौ᳚ प्रा॒णाः प्रा॒णा द्वौद्वौ॒ द्वौद्वौ᳚ प्रा॒णाः । \newline
49. द्वौद्वा॒विति॒ द्वौ - द्वौ॒ । \newline
50. प्रा॒णा इति॑ प्र - अ॒नाः । \newline

\textbf{Ghana Paata } \newline

1. आ॒है॒ष ए॒ष आ॑हा है॒ष वै वा ए॒ष आ॑हा है॒ष वै । \newline
2. ए॒ष वै वा ए॒ष ए॒ष वै सोम॑स्य॒ सोम॑स्य॒ वा ए॒ष ए॒ष वै सोम॑स्य । \newline
3. वै सोम॑स्य॒ सोम॑स्य॒ वै वै सोम॑स्य सोमपी॒थः सो॑मपी॒थः सोम॑स्य॒ वै वै सोम॑स्य सोमपी॒थः । \newline
4. सोम॑स्य सोमपी॒थः सो॑मपी॒थः सोम॑स्य॒ सोम॑स्य सोमपी॒थो यो यः सो॑मपी॒थः सोम॑स्य॒ सोम॑स्य सोमपी॒थो यः । \newline
5. सो॒म॒पी॒थो यो यः सो॑मपी॒थः सो॑मपी॒थो य ए॒व मे॒वं ॅयः सो॑मपी॒थः सो॑मपी॒थो य ए॒वम् । \newline
6. सो॒म॒पी॒थ इति॑ सोम - पी॒थः । \newline
7. य ए॒व मे॒वं ॅयो य ए॒वं ॅवेद॒ वेदै॒वं ॅयो य ए॒वं ॅवेद॑ । \newline
8. ए॒वं ॅवेद॒ वेदै॒व मे॒वं ॅवेद॒ न न वेदै॒व मे॒वं ॅवेद॒ न । \newline
9. वेद॒ न न वेद॒ वेद॒ न सौ॒म्याꣳ सौ॒म्यान्न वेद॒ वेद॒ न सौ॒म्याम् । \newline
10. न सौ॒म्याꣳ सौ॒म्यान्न न सौ॒म्या मार्ति॒ मार्तिꣳ॑ सौ॒म्यान्न न सौ॒म्या मार्ति᳚म् । \newline
11. सौ॒म्या मार्ति॒ मार्तिꣳ॑ सौ॒म्याꣳ सौ॒म्या मार्ति॒ मा ऽऽर्तिꣳ॑ सौ॒म्याꣳ सौ॒म्या मार्ति॒ मा । \newline
12. आर्ति॒ मा ऽऽर्ति॒ मार्ति॒ मार्च्छ॑ त्यृच्छ॒त्या ऽऽर्ति॒ मार्ति॒ मार्च्छ॑ति । \newline
13. आर्च्छ॑ त्यृच्छ॒ त्यार्च्छ॑ति॒ घ्नन्ति॒ घ्नन्त्यृ॑च्छ॒ त्यार्च्छ॑ति॒ घ्नन्ति॑ । \newline
14. ऋ॒च्छ॒ति॒ घ्नन्ति॒ घ्नन्त्यृ॑च्छ त्यृच्छति॒ घ्नन्ति॒ वै वै घ्नन्त्यृ॑च्छ त्यृच्छति॒ घ्नन्ति॒ वै । \newline
15. घ्नन्ति॒ वै वै घ्नन्ति॒ घ्नन्ति॒ वा ए॒त दे॒तद् वै घ्नन्ति॒ घ्नन्ति॒ वा ए॒तत् । \newline
16. वा ए॒त दे॒तद् वै वा ए॒तथ् सोमꣳ॒॒ सोम॑ मे॒तद् वै वा ए॒तथ् सोम᳚म् । \newline
17. ए॒तथ् सोमꣳ॒॒ सोम॑ मे॒त दे॒तथ् सोमं॒ ॅयद् यथ् सोम॑ मे॒त दे॒तथ् सोमं॒ ॅयत् । \newline
18. सोमं॒ ॅयद् यथ् सोमꣳ॒॒ सोमं॒ ॅयद॑भिषु॒ण्व न्त्य॑भिषु॒ण्वन्ति॒ यथ् सोमꣳ॒॒ सोमं॒ ॅयद॑भिषु॒ण्वन्ति॑ । \newline
19. यद॑भिषु॒ण्व न्त्य॑भिषु॒ण्वन्ति॒ यद् यद॑भिषु॒ण्व न्त्यꣳ॒॒शू नꣳ॒॒शू न॑भिषु॒ण्वन्ति॒ यद् यद॑भिषु॒ण्वन् त्यꣳ॒॒शून् । \newline
20. अ॒भि॒षु॒ण्व न्त्यꣳ॒॒शू नꣳ॒॒शू न॑भिषु॒ण्व न्त्य॑भिषु॒ण्व न्त्यꣳ॒॒शून पापाꣳ॒॒शू न॑भिषु॒ण्व
न्त्य॑भिषु॒ण्व न्त्यꣳ॒॒शूनप॑ । \newline
21. अ॒भि॒षु॒ण्वन्तीत्य॑भि - सु॒न्वन्ति॑ । \newline
22. अꣳ॒॒शू नपापाꣳ॒॒शू नꣳ॒॒शूनप॑ गृह्णाति गृह्णा॒ त्यपाꣳ॒॒शू नꣳ॒॒शू नप॑ गृह्णाति । \newline
23. अप॑ गृह्णाति गृह्णा॒ त्यपाप॑ गृह्णाति॒ त्राय॑ते॒ त्राय॑ते गृह्णा॒ त्यपाप॑ गृह्णाति॒ त्राय॑ते । \newline
24. गृ॒ह्णा॒ति॒ त्राय॑ते॒ त्राय॑ते गृह्णाति गृह्णाति॒ त्राय॑त ए॒वैव त्राय॑ते गृह्णाति गृह्णाति॒ त्राय॑त ए॒व । \newline
25. त्राय॑त ए॒वैव त्राय॑ते॒ त्राय॑त ए॒वैन॑ मेन मे॒व त्राय॑ते॒ त्राय॑त ए॒वैन᳚म् । \newline
26. ए॒वैन॑ मेन मे॒वै वैन॑म् प्रा॒णाः प्रा॒णा ए॑न मे॒वै वैन॑म् प्रा॒णाः । \newline
27. ए॒न॒म् प्रा॒णाः प्रा॒णा ए॑न मेनम् प्रा॒णा वै वै प्रा॒णा ए॑न मेनम् प्रा॒णा वै । \newline
28. प्रा॒णा वै वै प्रा॒णाः प्रा॒णा वा अꣳ॒॒शवो॒ ऽꣳ॒शवो॒ वै प्रा॒णाः प्रा॒णा वा अꣳ॒॒शवः॑ । \newline
29. प्रा॒णा इति॑ प्र - अ॒नाः । \newline
30. वा अꣳ॒॒शवो॒ ऽꣳ॒शवो॒ वै वा अꣳ॒॒शवः॑ प॒शवः॑ प॒शवो॒ ऽꣳ॒शवो॒ वै वा 
अꣳ॒॒शवः॑ प॒शवः॑ । \newline
31. अꣳ॒॒शवः॑ प॒शवः॑ प॒शवो॒ ऽꣳ॒शवो॒ ऽꣳ॒शवः॑ प॒शवः॒ सोमः॒ सोमः॑ 
प॒शवो॒ ऽꣳ॒शवो॒ ऽꣳ॒शवः॑ प॒शवः॒ सोमः॑ । \newline
32. प॒शवः॒ सोमः॒ सोमः॑ प॒शवः॑ प॒शवः॒ सोमो॒ ऽꣳ॒शू नꣳ॒॒शून् थ्सोमः॑ प॒शवः॑ प॒शवः॒ सोमो॒ ऽꣳ॒शून् । \newline
33. सोमो॒ ऽꣳ॒शू नꣳ॒॒शून् थ्सोमः॒ सोमो॒ ऽꣳ॒शून् पुनः॒ पुन॑ रꣳ॒॒शून् थ्सोमः॒ सोमो॒ ऽꣳ॒शून् पुनः॑ । \newline
34. अꣳ॒॒शून् पुनः॒ पुन॑ रꣳ॒॒शू नꣳ॒॒शून् पुन॒ रप्यपि॒ पुन॑ रꣳ॒॒शू नꣳ॒॒शून् पुन॒ रपि॑ । \newline
35. पुन॒ रप्यपि॒ पुनः॒ पुन॒ रपि॑ सृजति सृज॒ त्यपि॒ पुनः॒ पुन॒ रपि॑ सृजति । \newline
36. अपि॑ सृजति सृज॒ त्यप्यपि॑ सृजति प्रा॒णान् प्रा॒णान् थ्सृ॑ज॒ त्यप्यपि॑ सृजति प्रा॒णान् । \newline
37. सृ॒ज॒ति॒ प्रा॒णान् प्रा॒णान् थ्सृ॑जति सृजति प्रा॒णाने॒वैव प्रा॒णान् थ्सृ॑जति सृजति प्रा॒णा ने॒व । \newline
38. प्रा॒णाने॒वैव प्रा॒णान् प्रा॒णाने॒व प॒शुषु॑ प॒शुष्वे॒व प्रा॒णान् प्रा॒णाने॒व प॒शुषु॑ । \newline
39. प्रा॒णानिति॑ प्र - अ॒नान् । \newline
40. ए॒व प॒शुषु॑ प॒शु ष्वे॒वैव प॒शुषु॑ दधाति दधाति प॒शु ष्वे॒वैव प॒शुषु॑ दधाति । \newline
41. प॒शुषु॑ दधाति दधाति प॒शुषु॑ प॒शुषु॑ दधाति॒ द्वौद्वौ॒ द्वौद्वौ॑ दधाति प॒शुषु॑ प॒शुषु॑ दधाति॒ द्वौद्वौ᳚ । \newline
42. द॒धा॒ति॒ द्वौद्वौ॒ द्वौद्वौ॑ दधाति दधाति॒ द्वौद्वा॒ वप्यपि॒ द्वौद्वौ॑ दधाति दधाति॒ द्वौद्वा॒ वपि॑ । \newline
43. द्वौद्वा॒ वप्यपि॒ द्वौद्वौ॒ द्वौद्वा॒ वपि॑ सृजति सृज॒ त्यपि॒ द्वौद्वौ॒ द्वौद्वा॒ वपि॑ सृजति । \newline
44. द्वौद्वा॒विति॒ द्वौ - द्वौ॒ । \newline
45. अपि॑ सृजति सृज॒ त्यप्यपि॑ सृजति॒ तस्मा॒त् तस्मा᳚थ् सृज॒ त्यप्यपि॑ सृजति॒ तस्मा᳚त् । \newline
46. सृ॒ज॒ति॒ तस्मा॒त् तस्मा᳚थ् सृजति सृजति॒ तस्मा॒द् द्वौद्वौ॒ द्वौद्वौ॒ तस्मा᳚थ् सृजति सृजति॒ तस्मा॒द् द्वौद्वौ᳚ । \newline
47. तस्मा॒द् द्वौद्वौ॒ द्वौद्वौ॒ तस्मा॒त् तस्मा॒द् द्वौद्वौ᳚ प्रा॒णाः प्रा॒णा द्वौद्वौ॒ तस्मा॒त् तस्मा॒द् द्वौद्वौ᳚ प्रा॒णाः । \newline
48. द्वौद्वौ᳚ प्रा॒णाः प्रा॒णा द्वौद्वौ॒ द्वौद्वौ᳚ प्रा॒णाः । \newline
49. द्वौद्वा॒विति॒ द्वौ - द्वौ॒ । \newline
50. प्रा॒णा इति॑ प्र - अ॒नाः । \newline
\pagebreak
\markright{ TS 6.4.5.1  \hfill https://www.vedavms.in \hfill}

\section{ TS 6.4.5.1 }

\textbf{TS 6.4.5.1 } \newline
\textbf{Samhita Paata} \newline

प्रा॒णो वा ए॒ष यदु॑पाꣳ॒॒शु र्यदुपाꣳ॒॒श्व॑ग्रा॒ ग्रहा॑ गृ॒ह्यन्ते᳚ प्रा॒णमे॒वानु॒ प्र य॑न्त्यरु॒णो ह॑ स्मा॒ऽऽ*हौप॑वेशिः प्रातस्सव॒न ए॒वाहं ॅय॒ज्ञ्ꣳ सꣳ स्था॑पयामि॒ तेन॒ ततः॒ सꣳस्थि॑तेन चरा॒मीत्य॒ष्टौ कृत्वोऽग्रे॒-ऽभिषु॑णो-त्य॒ष्टाक्ष॑रा गाय॒त्री गा॑य॒त्रं प्रा॑तस्सव॒नं प्रा॑तस्सव॒नमे॒व तेना᳚ ऽऽ*प्नो॒त्येका॑दश॒ कृत्वो᳚ द्वि॒तीय॒-मेका॑दशाक्षरा त्रि॒ष्टुप् त्रैष्टु॑भं॒ माद्ध्य॑दिंनꣳ॒॒- [  ] \newline

\textbf{Pada Paata} \newline

प्रा॒ण इति॑ प्र - अ॒नः । वै । ए॒षः । यत् । उ॒पाꣳ॒॒शुरित्यु॑प-अ॒शुः । यत् । उ॒पाꣳ॒॒श्व॑ग्रा॒ इत्यु॑पाꣳ॒॒शु - अ॒ग्राः॒ । ग्रहाः᳚ । गृ॒ह्यन्ते᳚ । प्रा॒णमिति॑ प्र - अ॒नम् । ए॒व । अनु॑ । प्रेति॑ । य॒न्ति॒ । अ॒रु॒णः । ह॒ । स्म॒ । आ॒ह॒ । औप॑वेशि॒रित्यौप॑ - वे॒शिः॒ । प्रा॒त॒स्स॒व॒न इति॑ प्रातः-स॒व॒ने । ए॒व । अ॒हम् । य॒ज्ञ्म् । समिति॑ । स्था॒प॒या॒मि॒ । तेन॑ । ततः॑ । सꣳस्थि॑ते॒नेति॒ सं-स्थि॒ते॒न॒ । च॒रा॒मि॒ । इति॑ । अ॒ष्टौ । कृत्वः॑ । अग्रे᳚ । अ॒भीति॑ । सु॒नो॒ति॒ । अ॒ष्टाक्ष॒रेत्य॒ष्टा - अ॒क्ष॒रा॒ । गा॒य॒त्री । गा॒य॒त्रम् । प्रा॒त॒स्स॒व॒नमिति॑ प्रातः - स॒व॒नम् । प्रा॒त॒स्स॒व॒नमिति॑ प्रातः - स॒व॒नम् । ए॒व । तेन॑ । आ॒प्नो॒ति॒ । एका॑दश । कृत्वः॑ । द्वि॒तीय᳚म् । एका॑दशाक्ष॒रेत्येका॑दश-अ॒क्ष॒रा॒ । त्रि॒ष्टुप् । त्रैष्टु॑भम् । माद्ध्य॑न्दिनम् ।  \newline


\textbf{Krama Paata} \newline

प्रा॒णो वै । प्रा॒ण इति॑ प्र - अ॒नः । वा ए॒षः । ए॒ष यत् । यदु॑पाꣳ॒॒शुः । उ॒पाꣳ॒॒शुर् यत् । उ॒पाꣳ॒॒शुरित्यु॑प - अꣳ॒॒शुः । यदु॑पाꣳ॒॒श्व॑ग्राः । उ॒पाꣳ॒॒श्व॑ग्रा॒ ग्रहाः᳚ । उ॒पाꣳ॒॒श्व॑ग्रा॒ इत्यु॑पाꣳ॒॒शु - अ॒ग्राः॒ । ग्रहा॑ गृ॒ह्यन्ते᳚ । गृ॒ह्यन्ते᳚ प्रा॒णम् । प्रा॒णमे॒व । प्रा॒णमिति॑ प्र - अ॒नम् । ए॒वानु॑ । अनु॒ प्र । प्र य॑न्ति । य॒न्त्य॒रु॒णः । अ॒रु॒णो ह॑ । ह॒ स्म॒ । स्मा॒ह॒ । आ॒हौप॑वेशिः । औप॑वेशिः प्रातस्सव॒ने । औप॑वेशि॒रित्यौप॑ - वे॒शिः॒ । प्रा॒त॒स्स॒व॒न ए॒व । प्रा॒त॒स्स॒व॒न इति॑ प्रातः - स॒व॒ने । ए॒वाहम् । अ॒हम् ॅय॒ज्ञ्म् । य॒ज्ञ्ꣳ सम् । सꣳ स्था॑पयामि । स्था॒प॒या॒मि॒ तेन॑ । तेन॒ ततः॑ । ततः॒ सꣳस्थि॑तेन । सꣳस्थि॑तेन चरामि । सꣳस्थि॑ते॒नेति॒ सम् - स्थि॒ते॒न॒ । च॒रा॒मीति॑ । इत्य॒ष्टौ । अ॒ष्टौ कृत्वः॑ । कृत्वोऽग्रे᳚ । अग्रे॒ऽभि । अ॒भि षु॑णोति । सु॒नो॒त्य॒ष्टाक्ष॑रा । अ॒ष्टाक्ष॑रा गाय॒त्री । अ॒ष्टाक्ष॒रेत्य॒ष्टा - अ॒क्ष॒रा॒ । गा॒य॒त्री गा॑य॒त्रम् । गा॒य॒त्रम् प्रा॑तस्सव॒नम् । प्रा॒त॒स्स॒व॒नम् प्रा॑तस्सव॒नम् । प्रा॒त॒स्स॒व॒नमिति॑ प्रातः - स॒व॒नम् । प्रा॒त॒स्स॒व॒नमे॒व । प्रा॒त॒स्स॒व॒नमिति॑ प्रातः - स॒व॒नम् । ए॒व तेन॑ । तेना᳚प्नोति । आ॒प्नो॒त्येका॑दश । एका॑दश॒ कृत्वः॑ । कृत्वो᳚ द्वि॒तीय᳚म् । द्वि॒तीय॒मेका॑दशाक्षरा । एका॑दशाक्षरा त्रि॒ष्टुप् । एका॑दशाक्ष॒रेत्येका॑दश - अ॒क्ष॒रा॒ । त्रि॒ष्टुप् त्रैष्टु॑भम् । त्रैष्टु॑भ॒म् माद्ध्य॑न्दिनम् । माद्ध्य॑न्दिनꣳ॒॒ सव॑नम् \newline

\textbf{Jatai Paata} \newline

1. प्रा॒णो वै वै प्रा॒णः प्रा॒णो वै । \newline
2. प्रा॒ण इति॑ प्र - अ॒नः । \newline
3. वा ए॒ष ए॒ष वै वा ए॒षः । \newline
4. ए॒ष यद् यदे॒ष ए॒ष यत् । \newline
5. यदु॑पाꣳ॒॒शु रु॑पाꣳ॒॒शुर् यद् यदु॑पाꣳ॒॒शुः । \newline
6. उ॒पाꣳ॒॒शुर् यद् यदु॑पाꣳ॒॒शु रु॑पाꣳ॒॒शुर् यत् । \newline
7. उ॒पाꣳ॒॒शुरित्यु॑प - अꣳ॒॒शुः । \newline
8. यदु॑पाꣳ॒॒श्व॑ग्रा उपाꣳ॒॒श्व॑ग्रा॒ यद् यदु॑पाꣳ॒॒श्व॑ग्राः । \newline
9. उ॒पाꣳ॒॒श्व॑ग्रा॒ ग्रहा॒ ग्रहा॑ उपाꣳ॒॒श्व॑ग्रा उपाꣳ॒॒श्व॑ग्रा॒ ग्रहाः᳚ । \newline
10. उ॒पाꣳ॒॒श्व॑ग्रा॒ इत्यु॑पाꣳ॒॒शु - अ॒ग्राः॒ । \newline
11. ग्रहा॑ गृ॒ह्यन्ते॑ गृ॒ह्यन्ते॒ ग्रहा॒ ग्रहा॑ गृ॒ह्यन्ते᳚ । \newline
12. गृ॒ह्यन्ते᳚ प्रा॒णम् प्रा॒णम् गृ॒ह्यन्ते॑ गृ॒ह्यन्ते᳚ प्रा॒णम् । \newline
13. प्रा॒ण मे॒वैव प्रा॒णम् प्रा॒ण मे॒व । \newline
14. प्रा॒णमिति॑ प्र - अ॒नम् । \newline
15. ए॒वान् वन् वे॒वै वानु॑ । \newline
16. अनु॒ प्र प्राण्वनु॒ प्र । \newline
17. प्र य॑न्ति यन्ति॒ प्र प्र य॑न्ति । \newline
18. य॒न्त्य॒रु॒णो॑ ऽरु॒णो य॑न्ति यन्त्यरु॒णः । \newline
19. अ॒रु॒णो ह॑ हारु॒णो॑ ऽरु॒णो ह॑ । \newline
20. ह॒ स्म॒ स्म॒ ह॒ ह॒ स्म॒ । \newline
21. स्मा॒ हा॒ह॒ स्म॒ स्मा॒ह॒ । \newline
22. आ॒हौप॑वेशि॒ रौप॑वेशि राहा॒ हौप॑वेशिः । \newline
23. औप॑वेशिः प्रातस्सव॒ने प्रा॑तस्सव॒न औप॑वेशि॒ रौप॑वेशिः प्रातस्सव॒ने । \newline
24. औप॑वेशि॒रित्यौप॑ - वे॒शिः॒ । \newline
25. प्रा॒त॒स्स॒व॒न ए॒वैव प्रा॑तस्सव॒ने प्रा॑तस्सव॒न ए॒व । \newline
26. प्रा॒त॒स्स॒व॒न इति॑ प्रातः - स॒व॒ने । \newline
27. ए॒वाह म॒ह मे॒वै वाहम् । \newline
28. अ॒हं ॅय॒ज्ञ्ं ॅय॒ज्ञ् म॒ह म॒हं ॅय॒ज्ञ्म् । \newline
29. य॒ज्ञ्ꣳ सꣳ सं ॅय॒ज्ञ्ं ॅय॒ज्ञ्ꣳ सम् । \newline
30. सꣳ स्था॑पयामि स्थापयामि॒ सꣳ सꣳ स्था॑पयामि । \newline
31. स्था॒प॒या॒मि॒ तेन॒ तेन॑ स्थापयामि स्थापयामि॒ तेन॑ । \newline
32. तेन॒ तत॒ स्तत॒ स्तेन॒ तेन॒ ततः॑ । \newline
33. ततः॒ सꣳस्थि॑तेन॒ सꣳस्थि॑तेन॒ तत॒ स्ततः॒ सꣳस्थि॑तेन । \newline
34. सꣳस्थि॑तेन चरामि चरामि॒ सꣳस्थि॑तेन॒ सꣳस्थि॑तेन चरामि । \newline
35. सꣳस्थि॑ते॒नेति॒ सं - स्थि॒ते॒न॒ । \newline
36. च॒रा॒मीतीति॑ चरामि चरा॒मीति॑ । \newline
37. इत्य॒ष्टा व॒ष्टा विती त्य॒ष्टौ । \newline
38. अ॒ष्टौ कृत्वः॒ कृत्वो॒ ऽष्टा व॒ष्टौ कृत्वः॑ । \newline
39. कृत्वो ऽग्रे ऽग्रे॒ कृत्वः॒ कृत्वो ऽग्रे᳚ । \newline
40. अग्रे॒ ऽभ्य॑भ्यग्रे ऽग्रे॒ ऽभि । \newline
41. अ॒भि षु॑णोति सुनो त्य॒भ्य॑भि षु॑णोति । \newline
42. सु॒नो॒ त्य॒ष्टाक्ष॑रा॒ ऽष्टाक्ष॑रा सुनोति सुनो त्य॒ष्टाक्ष॑रा । \newline
43. अ॒ष्टाक्ष॑रा गाय॒त्री गा॑य॒ त्र्य॑ष्टाक्ष॑रा॒ ऽष्टाक्ष॑रा गाय॒त्री । \newline
44. अ॒ष्टाक्ष॒रेत्य॒ष्टा - अ॒क्ष॒रा॒ । \newline
45. गा॒य॒त्री गा॑य॒त्रम् गा॑य॒त्रम् गा॑य॒त्री गा॑य॒त्री गा॑य॒त्रम् । \newline
46. गा॒य॒त्रम् प्रा॑तस्सव॒नम् प्रा॑तस्सव॒नम् गा॑य॒त्रम् गा॑य॒त्रम् प्रा॑तस्सव॒नम् । \newline
47. प्रा॒त॒स्स॒व॒नम् प्रा॑तस्सव॒नम् । \newline
48. प्रा॒त॒स्स॒व॒नमिति॑ प्रातः - स॒व॒नम् । \newline
49. प्रा॒त॒स्स॒व॒न मे॒वैव प्रा॑तस्सव॒नम् प्रा॑तस्सव॒न मे॒व । \newline
50. प्रा॒त॒स्स॒व॒नमिति॑ प्रातः - स॒व॒नम् । \newline
51. ए॒व तेन॒ तेनै॒ वैव तेन॑ । \newline
52. तेना᳚प्नो त्याप्नोति॒ तेन॒ तेना᳚प्नोति । \newline
53. आ॒प्नो॒ त्येका॑द॒ शैका॑दशाप्नो त्याप्नो॒ त्येका॑दश । \newline
54. एका॑दश॒ कृत्वः॒ कृत्व॒ एका॑द॒ शैका॑दश॒ कृत्वः॑ । \newline
55. कृत्वो᳚ द्वि॒तीय॑म् द्वि॒तीय॒म् कृत्वः॒ कृत्वो᳚ द्वि॒तीय᳚म् । \newline
56. द्वि॒तीय॒ मेका॑दशाक्ष॒ रैका॑दशाक्षरा द्वि॒तीय॑म् द्वि॒तीय॒ मेका॑दशाक्षरा । \newline
57. एका॑दशाक्षरा त्रि॒ष्टुप् त्रि॒ष्टु बेका॑दशाक्ष॒ रैका॑दशाक्षरा त्रि॒ष्टुप् । \newline
58. एका॑दशाक्ष॒रेत्येका॑दश - अ॒क्ष॒रा॒ । \newline
59. त्रि॒ष्टुप् त्रैष्टु॑भ॒म् त्रैष्टु॑भम् त्रि॒ष्टुप् त्रि॒ष्टुप् त्रैष्टु॑भम् । \newline
60. त्रैष्टु॑भ॒म् माद्ध्य॑न्दिन॒म् माद्ध्य॑न्दिन॒म् त्रैष्टु॑भ॒म् त्रैष्टु॑भ॒म् माद्ध्य॑न्दिनम् । \newline
61. माद्ध्य॑न्दिनꣳ॒॒ सव॑नꣳ॒॒ सव॑न॒म् माद्ध्य॑न्दिन॒म् माद्ध्य॑न्दिनꣳ॒॒ सव॑नम् । \newline

\textbf{Ghana Paata } \newline

1. प्रा॒णो वै वै प्रा॒णः प्रा॒णो वा ए॒ष ए॒ष वै प्रा॒णः प्रा॒णो वा ए॒षः । \newline
2. प्रा॒ण इति॑ प्र - अ॒नः । \newline
3. वा ए॒ष ए॒ष वै वा ए॒ष यद् यदे॒ष वै वा ए॒ष यत् । \newline
4. ए॒ष यद् यदे॒ष ए॒ष यदु॑पाꣳ॒॒शु रु॑पाꣳ॒॒शुर् यदे॒ष ए॒ष यदु॑पाꣳ॒॒शुः । \newline
5. यदु॑पाꣳ॒॒शु रु॑पाꣳ॒॒शुर् यद् यदु॑पाꣳ॒॒शुर् यद् यदु॑पाꣳ॒॒शुर् यद् यदु॑पाꣳ॒॒शुर् यत् । \newline
6. उ॒पाꣳ॒॒शुर् यद् यदु॑पाꣳ॒॒शु रु॑पाꣳ॒॒शुर् यदु॑पाꣳ॒॒श्व॑ग्रा उपाꣳ॒॒श्व॑ग्रा॒ यदु॑पाꣳ॒॒शु रु॑पाꣳ॒॒शुर् यदु॑पाꣳ॒॒श्व॑ग्राः । \newline
7. उ॒पाꣳ॒॒शुरित्यु॑प - अꣳ॒॒शुः । \newline
8. यदु॑पाꣳ॒॒श्व॑ग्रा उपाꣳ॒॒श्व॑ग्रा॒ यद् यदु॑पाꣳ॒॒श्व॑ग्रा॒ ग्रहा॒ ग्रहा॑ उपाꣳ॒॒श्व॑ग्रा॒ यद् यदु॑पाꣳ॒॒श्व॑ग्रा॒ ग्रहाः᳚ । \newline
9. उ॒पाꣳ॒॒श्व॑ग्रा॒ ग्रहा॒ ग्रहा॑ उपाꣳ॒॒श्व॑ग्रा उपाꣳ॒॒श्व॑ग्रा॒ ग्रहा॑ गृ॒ह्यन्ते॑ गृ॒ह्यन्ते॒ ग्रहा॑ उपाꣳ॒॒श्व॑ग्रा उपाꣳ॒॒श्व॑ग्रा॒ ग्रहा॑ गृ॒ह्यन्ते᳚ । \newline
10. उ॒पाꣳ॒॒श्व॑ग्रा॒ इत्यु॑पाꣳ॒॒शु - अ॒ग्राः॒ । \newline
11. ग्रहा॑ गृ॒ह्यन्ते॑ गृ॒ह्यन्ते॒ ग्रहा॒ ग्रहा॑ गृ॒ह्यन्ते᳚ प्रा॒णम् प्रा॒णम् गृ॒ह्यन्ते॒ ग्रहा॒ ग्रहा॑ गृ॒ह्यन्ते᳚ प्रा॒णम् । \newline
12. गृ॒ह्यन्ते᳚ प्रा॒णम् प्रा॒णम् गृ॒ह्यन्ते॑ गृ॒ह्यन्ते᳚ प्रा॒ण मे॒वैव प्रा॒णम् गृ॒ह्यन्ते॑ गृ॒ह्यन्ते᳚ प्रा॒ण मे॒व । \newline
13. प्रा॒ण मे॒वैव प्रा॒णम् प्रा॒ण मे॒वान् वन् वे॒व प्रा॒णम् प्रा॒ण मे॒वानु॑ । \newline
14. प्रा॒णमिति॑ प्र - अ॒नम् । \newline
15. ए॒वान् वन् वे॒वै वानु॒ प्र प्राण्वे॒वै वानु॒ प्र । \newline
16. अनु॒ प्र प्राण्वनु॒ प्र य॑न्ति यन्ति॒ प्राण्वनु॒ प्र य॑न्ति । \newline
17. प्र य॑न्ति यन्ति॒ प्र प्र य॑न्त्यरु॒णो॑ ऽरु॒णो य॑न्ति॒ प्र प्र य॑न्त्यरु॒णः । \newline
18. य॒न्त्य॒रु॒णो॑ ऽरु॒णो य॑न्ति यन्त्यरु॒णो ह॑ हारु॒णो य॑न्ति यन्त्यरु॒णो ह॑ । \newline
19. अ॒रु॒णो ह॑ हारु॒णो॑ ऽरु॒णो ह॑ स्म स्म हारु॒णो॑ ऽरु॒णो ह॑ स्म । \newline
20. ह॒ स्म॒ स्म॒ ह॒ ह॒ स्मा॒ हा॒ह॒ स्म॒ ह॒ ह॒ स्मा॒ह॒ । \newline
21. स्मा॒ हा॒ह॒ स्म॒ स्मा॒ हौप॑वेशि॒ रौप॑वेशि राह स्म स्मा॒ हौप॑वेशिः । \newline
22. आ॒हौप॑वेशि॒ रौप॑वेशि राहा॒ हौप॑वेशिः प्रातस्सव॒ने प्रा॑तस्सव॒न औप॑वेशि राहा॒ हौप॑वेशिः प्रातस्सव॒ने । \newline
23. औप॑वेशिः प्रातस्सव॒ने प्रा॑तस्सव॒न औप॑वेशि॒ रौप॑वेशिः प्रातस्सव॒न ए॒वैव प्रा॑तस्सव॒न औप॑वेशि॒ रौप॑वेशिः प्रातस्सव॒न ए॒व । \newline
24. औप॑वेशि॒रित्यौप॑ - वे॒शिः॒ । \newline
25. प्रा॒त॒स्स॒व॒न ए॒वैव प्रा॑तस्सव॒ने प्रा॑तस्सव॒न ए॒वाह म॒ह मे॒व प्रा॑तस्सव॒ने प्रा॑तस्सव॒न ए॒वाहम् । \newline
26. प्रा॒त॒स्स॒व॒न इति॑ प्रातः - स॒व॒ने । \newline
27. ए॒वाह म॒ह मे॒वै वाहं ॅय॒ज्ञ्ं ॅय॒ज्ञ् म॒ह मे॒वै वाहं ॅय॒ज्ञ्म् । \newline
28. अ॒हं ॅय॒ज्ञ्ं ॅय॒ज्ञ् म॒ह म॒हं ॅय॒ज्ञ्ꣳ सꣳ सं ॅय॒ज्ञ् म॒ह म॒हं ॅय॒ज्ञ्ꣳ सम् । \newline
29. य॒ज्ञ्ꣳ सꣳ सं ॅय॒ज्ञ्ं ॅय॒ज्ञ्ꣳ सꣳ स्था॑पयामि स्थापयामि॒ सं ॅय॒ज्ञ्ं ॅय॒ज्ञ्ꣳ सꣳ स्था॑पयामि । \newline
30. सꣳ स्था॑पयामि स्थापयामि॒ सꣳ सꣳ स्था॑पयामि॒ तेन॒ तेन॑ स्थापयामि॒ सꣳ सꣳ स्था॑पयामि॒ तेन॑ । \newline
31. स्था॒प॒या॒मि॒ तेन॒ तेन॑ स्थापयामि स्थापयामि॒ तेन॒ तत॒ स्तत॒ स्तेन॑ स्थापयामि स्थापयामि॒ तेन॒ ततः॑ । \newline
32. तेन॒ तत॒ स्तत॒ स्तेन॒ तेन॒ ततः॒ सꣳस्थि॑तेन॒ सꣳस्थि॑तेन॒ तत॒ स्तेन॒ तेन॒ ततः॒ सꣳस्थि॑तेन । \newline
33. ततः॒ सꣳस्थि॑तेन॒ सꣳस्थि॑तेन॒ तत॒ स्ततः॒ सꣳस्थि॑तेन चरामि चरामि॒ सꣳस्थि॑तेन॒ तत॒ स्ततः॒ सꣳस्थि॑तेन चरामि । \newline
34. सꣳस्थि॑तेन चरामि चरामि॒ सꣳस्थि॑तेन॒ सꣳस्थि॑तेन चरा॒मी तीति॑ चरामि॒ सꣳस्थि॑तेन॒ सꣳस्थि॑तेन चरा॒मीति॑ । \newline
35. सꣳस्थि॑ते॒नेति॒ सं - स्थि॒ते॒न॒ । \newline
36. च॒रा॒मी तीति॑ चरामि चरा॒मी त्य॒ष्टा व॒ष्टा विति॑ चरामि चरा॒मी त्य॒ष्टौ । \newline
37. इत्य॒ष्टा व॒ष्टा विती त्य॒ष्टौ कृत्वः॒ कृत्वो॒ ऽष्टा विती त्य॒ष्टौ कृत्वः॑ । \newline
38. अ॒ष्टौ कृत्वः॒ कृत्वो॒ ऽष्टा व॒ष्टौ कृत्वो ऽग्रे ऽग्रे॒ कृत्वो॒ ऽष्टा व॒ष्टौ कृत्वो ऽग्रे᳚ । \newline
39. कृत्वो ऽग्रे ऽग्रे॒ कृत्वः॒ कृत्वो ऽग्रे॒ ऽभ्य॑ भ्यग्रे॒ कृत्वः॒ कृत्वो ऽग्रे॒ ऽभि । \newline
40. अग्रे॒ ऽभ्य॑ भ्यग्रे ऽग्रे॒ ऽभि षु॑णोति सुनो त्य॒भ्यग्रे ऽग्रे॒ ऽभि षु॑णोति । \newline
41. अ॒भि षु॑णोति सुनो त्य॒भ्य॑भि षु॑णो त्य॒ष्टाक्ष॑रा॒ ऽष्टाक्ष॑रा सुनो त्य॒भ्य॑भि षु॑णो त्य॒ष्टाक्ष॑रा । \newline
42. सु॒नो॒ त्य॒ष्टाक्ष॑रा॒ ऽष्टाक्ष॑रा सुनोति सुनो त्य॒ष्टाक्ष॑रा गाय॒त्री गा॑य॒ त्र्य॑ष्टाक्ष॑रा सुनोति सुनो त्य॒ष्टाक्ष॑रा गाय॒त्री । \newline
43. अ॒ष्टाक्ष॑रा गाय॒त्री गा॑य॒ त्र्य॑ष्टाक्ष॑रा॒ ऽष्टाक्ष॑रा गाय॒त्री गा॑य॒त्रम् गा॑य॒त्रम् गा॑य॒ त्र्य॑ष्टाक्ष॑रा॒ ऽष्टाक्ष॑रा गाय॒त्री गा॑य॒त्रम् । \newline
44. अ॒ष्टाक्ष॒रेत्य॒ष्टा - अ॒क्ष॒रा॒ । \newline
45. गा॒य॒त्री गा॑य॒त्रम् गा॑य॒त्रम् गा॑य॒त्री गा॑य॒त्री गा॑य॒त्रम् प्रा॑तस्सव॒नम् प्रा॑तस्सव॒नम् गा॑य॒त्रम् गा॑य॒त्री गा॑य॒त्री गा॑य॒त्रम् प्रा॑तस्सव॒नम् । \newline
46. गा॒य॒त्रम् प्रा॑तस्सव॒नम् प्रा॑तस्सव॒नम् गा॑य॒त्रम् गा॑य॒त्रम् प्रा॑तस्सव॒नम् । \newline
47. प्रा॒त॒स्स॒व॒नम् प्रा॑तस्सव॒नम् । \newline
48. प्रा॒त॒स्स॒व॒नमिति॑ प्रातः - स॒व॒नम् । \newline
49. प्रा॒त॒स्स॒व॒न मे॒वैव प्रा॑तस्सव॒नम् प्रा॑तस्सव॒न मे॒व तेन॒ तेनै॒व प्रा॑तस्सव॒नम् प्रा॑तस्सव॒न मे॒व तेन॑ । \newline
50. प्रा॒त॒स्स॒व॒नमिति॑ प्रातः - स॒व॒नम् । \newline
51. ए॒व तेन॒ तेनै॒ वैव तेना᳚प्नो त्याप्नोति॒ तेनै॒ वैव तेना᳚प्नोति । \newline
52. तेना᳚प्नो त्याप्नोति॒ तेन॒ तेना᳚प्नो॒ त्येका॑द॒ शैका॑दशाप्नोति॒ तेन॒ तेना᳚प्नो॒ त्येका॑दश । \newline
53. आ॒प्नो॒ त्येका॑द॒ शैका॑दशाप्नो त्याप्नो॒ त्येका॑दश॒ कृत्वः॒ कृत्व॒ एका॑दशाप्नो त्याप्नो॒ त्येका॑दश॒ कृत्वः॑ । \newline
54. एका॑दश॒ कृत्वः॒ कृत्व॒ एका॑द॒ शैका॑दश॒ कृत्वो᳚ द्वि॒तीय॑म् द्वि॒तीय॒म् कृत्व॒ एका॑द॒ शैका॑दश॒ कृत्वो᳚ द्वि॒तीय᳚म् । \newline
55. कृत्वो᳚ द्वि॒तीय॑म् द्वि॒तीय॒म् कृत्वः॒ कृत्वो᳚ द्वि॒तीय॒ मेका॑दशाक्ष॒ रैका॑दशाक्षरा द्वि॒तीय॒म् कृत्वः॒ कृत्वो᳚ द्वि॒तीय॒ मेका॑दशाक्षरा । \newline
56. द्वि॒तीय॒ मेका॑दशाक्ष॒ रैका॑दशाक्षरा द्वि॒तीय॑म् द्वि॒तीय॒ मेका॑दशाक्षरा त्रि॒ष्टुप् त्रि॒ष्टु
बेका॑दशाक्षरा द्वि॒तीय॑म् द्वि॒तीय॒ मेका॑दशाक्षरा त्रि॒ष्टुप् । \newline
57. एका॑दशाक्षरा त्रि॒ष्टुप् त्रि॒ष्टु बेका॑दशाक्ष॒ रैका॑दशाक्षरा त्रि॒ष्टुप् त्रैष्टु॑भ॒म् त्रैष्टु॑भम् त्रि॒ष्टु बेका॑दशाक्ष॒ रैका॑दशाक्षरा त्रि॒ष्टुप् त्रैष्टु॑भम् । \newline
58. एका॑दशाक्ष॒रेत्येका॑दश - अ॒क्ष॒रा॒ । \newline
59. त्रि॒ष्टुप् त्रैष्टु॑भ॒म् त्रैष्टु॑भम् त्रि॒ष्टुप् त्रि॒ष्टुप् त्रैष्टु॑भ॒म् माद्ध्य॑न्दिन॒म् माद्ध्य॑न्दिन॒म् त्रैष्टु॑भम् त्रि॒ष्टुप् त्रि॒ष्टुप् त्रैष्टु॑भ॒म् माद्ध्य॑न्दिनम् । \newline
60. त्रैष्टु॑भ॒म् माद्ध्य॑न्दिन॒म् माद्ध्य॑न्दिन॒म् त्रैष्टु॑भ॒म् त्रैष्टु॑भ॒म् माद्ध्य॑न्दिनꣳ॒॒ सव॑नꣳ॒॒ सव॑न॒म् माद्ध्य॑न्दिन॒म् त्रैष्टु॑भ॒म् त्रैष्टु॑भ॒म् माद्ध्य॑न्दिनꣳ॒॒ सव॑नम् । \newline
61. माद्ध्य॑न्दिनꣳ॒॒ सव॑नꣳ॒॒ सव॑न॒म् माद्ध्य॑न्दिन॒म् माद्ध्य॑न्दिनꣳ॒॒ सव॑न॒म् माद्ध्य॑न्दिन॒म् माद्ध्य॑न्दिनꣳ॒॒ सव॑न॒म् माद्ध्य॑न्दिन॒म् माद्ध्य॑न्दिनꣳ॒॒ सव॑न॒म् माद्ध्य॑न्दिनम् । \newline
\pagebreak
\markright{ TS 6.4.5.2  \hfill https://www.vedavms.in \hfill}

\section{ TS 6.4.5.2 }

\textbf{TS 6.4.5.2 } \newline
\textbf{Samhita Paata} \newline

सव॑नं॒ माद्ध्य॑न्दिनमे॒व सव॑नं॒ तेना᳚ऽऽ*प्नोति॒ द्वाद॑श॒ कृत्व॑स्तृ॒तीयं॒ द्वाद॑शाक्षरा॒ जग॑ती॒ जाग॑तं तृतीयसव॒नं तृ॑तीयसव॒नमे॒व तेना᳚ ऽऽ*प्नोत्ये॒ताꣳ ह॒ वाव स य॒ज्ञ्स्य॒ सꣳस्थि॑तिमुवा॒चा स्क॑न्दा॒यास्क॑न्नꣳ॒॒ हि तद्-यद्-य॒ज्ञ्स्य॒ सꣳस्थि॑तस्य॒ स्कन्द॒त्यथो॒ खल्वा॑हुर्गाय॒त्री वाव प्रा॑तस्सव॒ने नाति॒वाद॒ इत्यन॑तिवादुक एनं॒ भ्रातृ॑व्यो भवति॒ य ए॒वं ॅवेद॒ तस्मा॑द॒ष्टाव॑ष्टौ॒- [  ] \newline

\textbf{Pada Paata} \newline

सव॑नम् । माद्ध्य॑न्दिनम् । ए॒व । सव॑नम् । तेन॑ । आ॒प्नो॒ति॒ । द्वाद॑श । कृत्वः॑ । तृ॒तीय᳚म् । द्वाद॑शाक्ष॒रेति॒ द्वाद॑श - अ॒क्ष॒रा॒ । जग॑ती । जाग॑तम् । तृ॒ती॒य॒स॒व॒नमिति॑ तृतीय - स॒व॒नम् । तृ॒ती॒य॒स॒व॒नमिति॑ तृतीय-स॒व॒नम् । ए॒व । तेन॑ । आ॒प्नो॒ति॒ । ए॒ताम् । ह॒ । वाव । सः । य॒ज्ञ्स्य॑ । सꣳस्थि॑ति॒मिति॒ सं - स्थि॒ति॒म् । उ॒वा॒च॒ । अस्क॑न्दाय । अस्क॑न्नम् । हि । तत् । यत् । य॒ज्ञ्स्य॑ । सꣳस्थि॑त॒स्येति॒ सं - स्थि॒त॒स्य॒ । स्कन्द॑ति । अथो॒ इति॑ । खलु॑ । आ॒हुः॒ । गा॒य॒त्री । वाव । प्रा॒त॒स्स॒व॒न इति॑ प्रातः - स॒व॒ने । न । अ॒ति॒वाद॒ इत्य॑ति - वादे᳚ । इति॑ । अन॑तिवादुक॒ इत्यन॑ति - वा॒दु॒कः॒ । ए॒न॒म् । भ्रातृ॑व्यः । भ॒व॒ति॒ । यः । ए॒वम् । वेद॑ । तस्मा᳚त् । अ॒ष्टाव॑ष्टा॒वित्य॒ष्टौ - अ॒ष्टौ॒ ।  \newline


\textbf{Krama Paata} \newline

सव॑न॒म् माद्ध्य॑न्दिनम् । माद्ध्य॑न्दिनमे॒व । ए॒व सव॑नम् । सव॑न॒म् तेन॑ । तेना᳚प्नोति । आ॒प्नो॒ति॒ द्वाद॑श । द्वाद॑श॒ कृत्वः॑ । कृत्व॑स्तृ॒तीय᳚म् । तृ॒तीय॒म् द्वाद॑शाक्षरा । द्वाद॑शाक्षरा॒ जग॑ती । द्वाद॑शाक्ष॒रेति॒ द्वाद॑श - अ॒क्ष॒रा॒ । जग॑ती॒ जाग॑तम् । जाग॑तम् तृतीयसव॒नम् । तृ॒ती॒य॒स॒व॒नम् तृ॑तीयसव॒नम् । तृ॒ती॒य॒स॒व॒नमिति॑ तृतीय - स॒व॒नम् । तृ॒ती॒य॒स॒व॒नमे॒व । तृ॒ती॒य॒स॒व॒नमिति॑ तृतीय - स॒व॒नम् । ए॒व तेन॑ । तेना᳚प्नोति । आ॒प्नो॒त्ये॒ताम् । ए॒ताꣳ ह॑ । ह॒ वाव । वाव सः । स य॒ज्ञ्स्य॑ । य॒ज्ञ्स्य॒ सꣳस्थि॑तिम् । सꣳस्थि॑तिमुवाच । सꣳस्थि॑ति॒मिति॒ सम् - स्थि॒ति॒म् । उ॒वा॒चास्क॑न्दाय । अस्क॑न्दा॒यास्क॑न्नम् । अस्क॑न्नꣳ॒॒ हि । हि तत् । तद् यत् । यद् य॒ज्ञ्स्य॑ । य॒ज्ञ्स्य॒ सꣳस्थि॑तस्य । सꣳस्थि॑तस्य॒ स्कन्द॑ति । सꣳस्थि॑त॒स्येति॒ सम् - स्थि॒त॒स्य॒ । स्कन्द॒त्यथो᳚ । अथो॒ खलु॑ । अथो॒ इत्यथो᳚ । खल्वा॑हुः । आ॒हु॒र् गा॒य॒त्री । गा॒य॒त्री वाव । वाव प्रा॑तस्सव॒ने । प्रा॒त॒स्स॒व॒ने न । प्रा॒त॒स्स॒व॒न इति॑ प्रातः - स॒व॒ने । नाति॒वादे᳚ । अ॒ति॒वाद॒ इति॑ । अ॒ति॒वाद॒ इत्य॑ति - वादे᳚ । इत्यन॑तिवादुकः । अन॑तिवादुक एनम् । अन॑तिवादुक॒ इत्यन॑ति - वा॒दु॒कः॒ । ए॒न॒म् भ्रातृ॑व्यः । भ्रातृ॑व्यो भवति । भ॒व॒ति॒ यः । य ए॒वम् । ए॒वम् ॅवेद॑ । वेद॒ तस्मा᳚त् । तस्मा॑द॒ष्टाव॑ष्टौ । अ॒ष्टाव॑ष्टौ॒ कृत्वः॑ । अ॒ष्टाव॑ष्टा॒वित्य॒ष्टौ - अ॒ष्टौ॒ \newline

\textbf{Jatai Paata} \newline

1. सव॑न॒म् माद्ध्य॑न्दिन॒म् माद्ध्य॑न्दिनꣳ॒॒ सव॑नꣳ॒॒ सव॑न॒म् माद्ध्य॑न्दिनम् । \newline
2. माद्ध्य॑न्दिन मे॒वैव माद्ध्य॑न्दिन॒म् माद्ध्य॑न्दिन मे॒व । \newline
3. ए॒व सव॑नꣳ॒॒ सव॑न मे॒वैव सव॑नम् । \newline
4. सव॑न॒म् तेन॒ तेन॒ सव॑नꣳ॒॒ सव॑न॒म् तेन॑ । \newline
5. तेना᳚प्नो त्याप्नोति॒ तेन॒ तेना᳚प्नोति । \newline
6. आ॒प्नो॒ति॒ द्वाद॑श॒ द्वाद॑शाप्नो त्याप्नोति॒ द्वाद॑श । \newline
7. द्वाद॑श॒ कृत्वः॒ कृत्वो॒ द्वाद॑श॒ द्वाद॑श॒ कृत्वः॑ । \newline
8. कृत्व॑ स्तृ॒तीय॑म् तृ॒तीय॒म् कृत्वः॒ कृत्व॑ स्तृ॒तीय᳚म् । \newline
9. तृ॒तीय॒म् द्वाद॑शाक्षरा॒ द्वाद॑शाक्षरा तृ॒तीय॑म् तृ॒तीय॒म् द्वाद॑शाक्षरा । \newline
10. द्वाद॑शाक्षरा॒ जग॑ती॒ जग॑ती॒ द्वाद॑शाक्षरा॒ द्वाद॑शाक्षरा॒ जग॑ती । \newline
11. द्वाद॑शाक्ष॒रेति॒ द्वाद॑श - अ॒क्ष॒रा॒ । \newline
12. जग॑ती॒ जाग॑त॒म् जाग॑त॒म् जग॑ती॒ जग॑ती॒ जाग॑तम् । \newline
13. जाग॑तम् तृतीयसव॒नम् तृ॑तीयसव॒नम् जाग॑त॒म् जाग॑तम् तृतीयसव॒नम् । \newline
14. तृ॒ती॒य॒स॒व॒नम् तृ॑तीयसव॒नम् । \newline
15. तृ॒ती॒य॒स॒व॒नमिति॑ तृतीय - स॒व॒नम् । \newline
16. तृ॒ती॒य॒स॒व॒न मे॒वैव तृ॑तीयसव॒नम् तृ॑तीयसव॒न मे॒व । \newline
17. तृ॒ती॒य॒स॒व॒नमिति॑ तृतीय - स॒व॒नम् । \newline
18. ए॒व तेन॒ तेनै॒ वैव तेन॑ । \newline
19. तेना᳚प्नो त्याप्नोति॒ तेन॒ तेना᳚प्नोति । \newline
20. आ॒प्नो॒ त्ये॒ता मे॒ता मा᳚प्नो त्याप्नो त्ये॒ताम् । \newline
21. ए॒ताꣳ ह॑ है॒ता मे॒ताꣳ ह॑ । \newline
22. ह॒ वाव वाव ह॑ ह॒ वाव । \newline
23. वाव स स वाव वाव सः । \newline
24. स य॒ज्ञ्स्य॑ य॒ज्ञ्स्य॒ स स य॒ज्ञ्स्य॑ । \newline
25. य॒ज्ञ्स्य॒ सꣳस्थि॑तिꣳ॒॒ सꣳस्थि॑तिं ॅय॒ज्ञ्स्य॑ य॒ज्ञ्स्य॒ सꣳस्थि॑तिम् । \newline
26. सꣳस्थि॑ति मुवा चोवाच॒ सꣳस्थि॑तिꣳ॒॒ सꣳस्थि॑ति मुवाच । \newline
27. सꣳस्थि॑ति॒मिति॒ सं - स्थि॒ति॒म् । \newline
28. उ॒वा॒चा स्क॑न्दा॒या स्क॑न्दा योवाचो वा॒चा स्क॑न्दाय । \newline
29. अस्क॑न्दा॒या स्क॑न्न॒ मस्क॑न्न॒ मस्क॑न्दा॒या स्क॑न्दा॒या स्क॑न्नम् । \newline
30. अस्क॑न्नꣳ॒॒ हि ह्यस्क॑न्न॒ मस्क॑न्नꣳ॒॒ हि । \newline
31. हि तत् तद्धि हि तत् । \newline
32. तद् यद् यत् तत् तद् यत् । \newline
33. यद् य॒ज्ञ्स्य॑ य॒ज्ञ्स्य॒ यद् यद् य॒ज्ञ्स्य॑ । \newline
34. य॒ज्ञ्स्य॒ सꣳस्थि॑तस्य॒ सꣳस्थि॑तस्य य॒ज्ञ्स्य॑ य॒ज्ञ्स्य॒ सꣳस्थि॑तस्य । \newline
35. सꣳस्थि॑तस्य॒ स्कन्द॑ति॒ स्कन्द॑ति॒ सꣳस्थि॑तस्य॒ सꣳस्थि॑तस्य॒ स्कन्द॑ति । \newline
36. सꣳस्थि॑त॒स्येति॒ सं - स्थि॒त॒स्य॒ । \newline
37. स्कन्द॒ त्यथो॒ अथो॒ स्कन्द॑ति॒ स्कन्द॒ त्यथो᳚ । \newline
38. अथो॒ खलु॒ खल्वथो॒ अथो॒ खलु॑ । \newline
39. अथो॒ इत्यथो᳚ । \newline
40. खल्वा॑हु राहुः॒ खलु॒ खल्वा॑हुः । \newline
41. आ॒हु॒र् गा॒य॒त्री गा॑य॒ त्र्या॑हु राहुर् गाय॒त्री । \newline
42. गा॒य॒त्री वाव वाव गा॑य॒त्री गा॑य॒त्री वाव । \newline
43. वाव प्रा॑तस्सव॒ने प्रा॑तस्सव॒ने वाव वाव प्रा॑तस्सव॒ने । \newline
44. प्रा॒त॒स्स॒व॒ने न न प्रा॑तस्सव॒ने प्रा॑तस्सव॒ने न । \newline
45. प्रा॒त॒स्स॒व॒न इति॑ प्रातः - स॒व॒ने । \newline
46. नाति॒वादे॑ ऽति॒वादे॒ न नाति॒वादे᳚ । \newline
47. अ॒ति॒वाद॒ इती त्य॑ति॒वादे॑ ऽति॒वाद॒ इति॑ । \newline
48. अ॒ति॒वाद॒ इत्य॑ति - वादे᳚ । \newline
49. इत्यन॑तिवादु॒को ऽन॑तिवादुक॒ इती त्यन॑तिवादुकः । \newline
50. अन॑तिवादुक एन मेन॒ मन॑तिवादु॒को ऽन॑तिवादुक एनम् । \newline
51. अन॑तिवादुक॒ इत्यन॑ति - वा॒दु॒कः॒ । \newline
52. ए॒न॒म् भ्रातृ॑व्यो॒ भ्रातृ॑व्य एन मेन॒म् भ्रातृ॑व्यः । \newline
53. भ्रातृ॑व्यो भवति भवति॒ भ्रातृ॑व्यो॒ भ्रातृ॑व्यो भवति । \newline
54. भ॒व॒ति॒ यो यो भ॑वति भवति॒ यः । \newline
55. य ए॒व मे॒वं ॅयो य ए॒वम् । \newline
56. ए॒वं ॅवेद॒ वेदै॒व मे॒वं ॅवेद॑ । \newline
57. वेद॒ तस्मा॒त् तस्मा॒द् वेद॒ वेद॒ तस्मा᳚त् । \newline
58. तस्मा॑ द॒ष्टाव॑ष्टा व॒ष्टाव॑ष्टौ॒ तस्मा॒त् तस्मा॑ द॒ष्टाव॑ष्टौ । \newline
59. अ॒ष्टाव॑ष्टौ॒ कृत्वः॒ कृत्वो॒ ऽष्टाव॑ष्टा व॒ष्टाव॑ष्टौ॒ कृत्वः॑ । \newline
60. अ॒ष्टाव॑ष्टा॒वित्य॒ष्टौ - अ॒ष्टौ॒ । \newline

\textbf{Ghana Paata } \newline

1. सव॑न॒म् माद्ध्य॑न्दिन॒म् माद्ध्य॑न्दिनꣳ॒॒ सव॑नꣳ॒॒ सव॑न॒म् माद्ध्य॑न्दिन मे॒वैव माद्ध्य॑न्दिनꣳ॒॒ सव॑नꣳ॒॒ सव॑न॒म् माद्ध्य॑न्दिन मे॒व । \newline
2. माद्ध्य॑न्दिन मे॒वैव माद्ध्य॑न्दिन॒म् माद्ध्य॑न्दिन मे॒व सव॑नꣳ॒॒ सव॑न मे॒व माद्ध्य॑न्दिन॒म् माद्ध्य॑न्दिन मे॒व सव॑नम् । \newline
3. ए॒व सव॑नꣳ॒॒ सव॑न मे॒वैव सव॑न॒म् तेन॒ तेन॒ सव॑न मे॒वैव सव॑न॒म् तेन॑ । \newline
4. सव॑न॒म् तेन॒ तेन॒ सव॑नꣳ॒॒ सव॑न॒म् तेना᳚प्नो त्याप्नोति॒ तेन॒ सव॑नꣳ॒॒ सव॑न॒म् तेना᳚प्नोति । \newline
5. तेना᳚प्नो त्याप्नोति॒ तेन॒ तेना᳚प्नोति॒ द्वाद॑श॒ द्वाद॑शाप्नोति॒ तेन॒ तेना᳚प्नोति॒ द्वाद॑श । \newline
6. आ॒प्नो॒ति॒ द्वाद॑श॒ द्वाद॑शाप्नो त्याप्नोति॒ द्वाद॑श॒ कृत्वः॒ कृत्वो॒ द्वाद॑शाप्नो त्याप्नोति॒ द्वाद॑श॒ कृत्वः॑ । \newline
7. द्वाद॑श॒ कृत्वः॒ कृत्वो॒ द्वाद॑श॒ द्वाद॑श॒ कृत्व॑ स्तृ॒तीय॑म् तृ॒तीय॒म् कृत्वो॒ द्वाद॑श॒ द्वाद॑श॒ कृत्व॑ स्तृ॒तीय᳚म् । \newline
8. कृत्व॑ स्तृ॒तीय॑म् तृ॒तीय॒म् कृत्वः॒ कृत्व॑ स्तृ॒तीय॒म् द्वाद॑शाक्षरा॒ द्वाद॑शाक्षरा तृ॒तीय॒म् कृत्वः॒ कृत्व॑ स्तृ॒तीय॒म् द्वाद॑शाक्षरा । \newline
9. तृ॒तीय॒म् द्वाद॑शाक्षरा॒ द्वाद॑शाक्षरा तृ॒तीय॑म् तृ॒तीय॒म् द्वाद॑शाक्षरा॒ जग॑ती॒ जग॑ती॒ द्वाद॑शाक्षरा तृ॒तीय॑म् तृ॒तीय॒म् द्वाद॑शाक्षरा॒ जग॑ती । \newline
10. द्वाद॑शाक्षरा॒ जग॑ती॒ जग॑ती॒ द्वाद॑शाक्षरा॒ द्वाद॑शाक्षरा॒ जग॑ती॒ जाग॑त॒म् जाग॑त॒म् जग॑ती॒ द्वाद॑शाक्षरा॒ द्वाद॑शाक्षरा॒ जग॑ती॒ जाग॑तम् । \newline
11. द्वाद॑शाक्ष॒रेति॒ द्वाद॑श - अ॒क्ष॒रा॒ । \newline
12. जग॑ती॒ जाग॑त॒म् जाग॑त॒म् जग॑ती॒ जग॑ती॒ जाग॑तम् तृतीयसव॒नम् तृ॑तीयसव॒नम् जाग॑त॒म् जग॑ती॒ जग॑ती॒ जाग॑तम् तृतीयसव॒नम् । \newline
13. जाग॑तम् तृतीयसव॒नम् तृ॑तीयसव॒नम् जाग॑त॒म् जाग॑तम् तृतीयसव॒नम् । \newline
14. तृ॒ती॒य॒स॒व॒नम् तृ॑तीयसव॒नम् । \newline
15. तृ॒ती॒य॒स॒व॒नमिति॑ तृतीय - स॒व॒नम् । \newline
16. तृ॒ती॒य॒स॒व॒न मे॒वैव तृ॑तीयसव॒नम् तृ॑तीयसव॒न मे॒व तेन॒ तेनै॒व तृ॑तीयसव॒नम् तृ॑तीयसव॒न मे॒व तेन॑ । \newline
17. तृ॒ती॒य॒स॒व॒नमिति॑ तृतीय - स॒व॒नम् । \newline
18. ए॒व तेन॒ तेनै॒ वैव तेना᳚प्नो त्याप्नोति॒ तेनै॒ वैव तेना᳚प्नोति । \newline
19. तेना᳚प्नो त्याप्नोति॒ तेन॒ तेना᳚प्नो त्ये॒ता मे॒ता मा᳚प्नोति॒ तेन॒ तेना᳚प्नो त्ये॒ताम् । \newline
20. आ॒प्नो॒ त्ये॒ता मे॒ता मा᳚प्नो त्याप्नो त्ये॒ताꣳ ह॑ है॒ता मा᳚प्नो त्याप्नो त्ये॒ताꣳ ह॑ । \newline
21. ए॒ताꣳ ह॑ है॒ता मे॒ताꣳ ह॒ वाव वाव है॒ता मे॒ताꣳ ह॒ वाव । \newline
22. ह॒ वाव वाव ह॑ ह॒ वाव स स वाव ह॑ ह॒ वाव सः । \newline
23. वाव स स वाव वाव स य॒ज्ञ्स्य॑ य॒ज्ञ्स्य॒ स वाव वाव स य॒ज्ञ्स्य॑ । \newline
24. स य॒ज्ञ्स्य॑ य॒ज्ञ्स्य॒ स स य॒ज्ञ्स्य॒ सꣳस्थि॑तिꣳ॒॒ सꣳस्थि॑तिं ॅय॒ज्ञ्स्य॒ स स य॒ज्ञ्स्य॒ सꣳस्थि॑तिम् । \newline
25. य॒ज्ञ्स्य॒ सꣳस्थि॑तिꣳ॒॒ सꣳस्थि॑तिं ॅय॒ज्ञ्स्य॑ य॒ज्ञ्स्य॒ सꣳस्थि॑ति मुवा चोवाच॒ सꣳस्थि॑तिं ॅय॒ज्ञ्स्य॑ य॒ज्ञ्स्य॒ सꣳस्थि॑ति मुवाच । \newline
26. सꣳस्थि॑ति मुवा चोवाच॒ सꣳस्थि॑तिꣳ॒॒ सꣳस्थि॑ति मुवा॒चा स्क॑न्दा॒या स्क॑न्दायोवाच॒ सꣳस्थि॑तिꣳ॒॒ सꣳस्थि॑ति मुवा॒चा स्क॑न्दाय । \newline
27. सꣳस्थि॑ति॒मिति॒ सं - स्थि॒ति॒म् । \newline
28. उ॒वा॒चा स्क॑न्दा॒या स्क॑न्दायोवा चोवा॒चा स्क॑न्दा॒या स्क॑न्न॒ मस्क॑न्न॒ मस्क॑न्दा योवाचोवा॒चा स्क॑न्दा॒या स्क॑न्नम् । \newline
29. अस्क॑न्दा॒या स्क॑न्न॒ मस्क॑न्न॒ मस्क॑न्दा॒या स्क॑न्दा॒या स्क॑न्नꣳ॒॒ हि ह्यस्क॑न्न॒ मस्क॑न्दा॒या स्क॑न्दा॒या स्क॑न्नꣳ॒॒ हि । \newline
30. अस्क॑न्नꣳ॒॒ हि ह्यस्क॑न्न॒ मस्क॑न्नꣳ॒॒ हि तत् तद्ध्यस्क॑न्न॒ मस्क॑न्नꣳ॒॒ हि तत् । \newline
31. हि तत् तद्धि हि तद् यद् यत् तद्धि हि तद् यत् । \newline
32. तद् यद् यत् तत् तद् यद् य॒ज्ञ्स्य॑ य॒ज्ञ्स्य॒ यत् तत् तद् यद् य॒ज्ञ्स्य॑ । \newline
33. यद् य॒ज्ञ्स्य॑ य॒ज्ञ्स्य॒ यद् यद् य॒ज्ञ्स्य॒ सꣳस्थि॑तस्य॒ सꣳस्थि॑तस्य य॒ज्ञ्स्य॒ यद् यद् य॒ज्ञ्स्य॒ सꣳस्थि॑तस्य । \newline
34. य॒ज्ञ्स्य॒ सꣳस्थि॑तस्य॒ सꣳस्थि॑तस्य य॒ज्ञ्स्य॑ य॒ज्ञ्स्य॒ सꣳस्थि॑तस्य॒ स्कन्द॑ति॒ स्कन्द॑ति॒ सꣳस्थि॑तस्य य॒ज्ञ्स्य॑ य॒ज्ञ्स्य॒ सꣳस्थि॑तस्य॒ स्कन्द॑ति । \newline
35. सꣳस्थि॑तस्य॒ स्कन्द॑ति॒ स्कन्द॑ति॒ सꣳस्थि॑तस्य॒ सꣳस्थि॑तस्य॒ स्कन्द॒ त्यथो॒ अथो॒ स्कन्द॑ति॒ सꣳस्थि॑तस्य॒ सꣳस्थि॑तस्य॒ स्कन्द॒ त्यथो᳚ । \newline
36. सꣳस्थि॑त॒स्येति॒ सं - स्थि॒त॒स्य॒ । \newline
37. स्कन्द॒ त्यथो॒ अथो॒ स्कन्द॑ति॒ स्कन्द॒ त्यथो॒ खलु॒ खल्वथो॒ स्कन्द॑ति॒ स्कन्द॒ त्यथो॒ खलु॑ । \newline
38. अथो॒ खलु॒ खल्वथो॒ अथो॒ खल्वा॑हु राहुः॒ खल्वथो॒ अथो॒ खल्वा॑हुः । \newline
39. अथो॒ इत्यथो᳚ । \newline
40. खल्वा॑हु राहुः॒ खलु॒ खल्वा॑हुर् गाय॒त्री गा॑य॒ त्र्या॑हुः॒ खलु॒ खल्वा॑हुर् गाय॒त्री । \newline
41. आ॒हु॒र् गा॒य॒त्री गा॑य॒ त्र्या॑हु राहुर् गाय॒त्री वाव वाव गा॑य॒ त्र्या॑हु राहुर् गाय॒त्री वाव । \newline
42. गा॒य॒त्री वाव वाव गा॑य॒त्री गा॑य॒त्री वाव प्रा॑तस्सव॒ने प्रा॑तस्सव॒ने वाव गा॑य॒त्री गा॑य॒त्री वाव प्रा॑तस्सव॒ने । \newline
43. वाव प्रा॑तस्सव॒ने प्रा॑तस्सव॒ने वाव वाव प्रा॑तस्सव॒ने न न प्रा॑तस्सव॒ने वाव वाव प्रा॑तस्सव॒ने न । \newline
44. प्रा॒त॒स्स॒व॒ने न न प्रा॑तस्सव॒ने प्रा॑तस्सव॒ने नाति॒वादे॑ ऽति॒वादे॒ न प्रा॑तस्सव॒ने प्रा॑तस्सव॒ने नाति॒वादे᳚ । \newline
45. प्रा॒त॒स्स॒व॒न इति॑ प्रातः - स॒व॒ने । \newline
46. नाति॒वादे॑ ऽति॒वादे॒ न नाति॒वाद॒ इती त्य॑ति॒वादे॒ न नाति॒वाद॒ इति॑ । \newline
47. अ॒ति॒वाद॒ इती त्य॑ति॒वादे॑ ऽति॒वाद॒ इत्य न॑तिवादु॒को ऽन॑तिवादुक॒ इत्य॑ति॒वादे॑ ऽति॒वाद॒ इत्यन॑तिवादुकः । \newline
48. अ॒ति॒वाद॒ इत्य॑ति - वादे᳚ । \newline
49. इत्यन॑तिवादु॒को ऽन॑तिवादुक॒ इती त्यन॑तिवादुक एन मेन॒ मन॑तिवादुक॒ इती त्यन॑तिवादुक एनम् । \newline
50. अन॑तिवादुक एन मेन॒ मन॑तिवादु॒को ऽन॑तिवादुक एन॒म् भ्रातृ॑व्यो॒ भ्रातृ॑व्य एन॒ मन॑तिवादु॒को ऽन॑तिवादुक एन॒म् भ्रातृ॑व्यः । \newline
51. अन॑तिवादुक॒ इत्यन॑ति - वा॒दु॒कः॒ । \newline
52. ए॒न॒म् भ्रातृ॑व्यो॒ भ्रातृ॑व्य एन मेन॒म् भ्रातृ॑व्यो भवति भवति॒ भ्रातृ॑व्य एन मेन॒म् भ्रातृ॑व्यो भवति । \newline
53. भ्रातृ॑व्यो भवति भवति॒ भ्रातृ॑व्यो॒ भ्रातृ॑व्यो भवति॒ यो यो भ॑वति॒ भ्रातृ॑व्यो॒ भ्रातृ॑व्यो भवति॒ यः । \newline
54. भ॒व॒ति॒ यो यो भ॑वति भवति॒ य ए॒व मे॒वं ॅयो भ॑वति भवति॒ य ए॒वम् । \newline
55. य ए॒व मे॒वं ॅयो य ए॒वं ॅवेद॒ वेदै॒वं ॅयो य ए॒वं ॅवेद॑ । \newline
56. ए॒वं ॅवेद॒ वेदै॒व मे॒वं ॅवेद॒ तस्मा॒त् तस्मा॒द् वेदै॒व मे॒वं ॅवेद॒ तस्मा᳚त् । \newline
57. वेद॒ तस्मा॒त् तस्मा॒द् वेद॒ वेद॒ तस्मा॑ द॒ष्टाव॑ष्टा व॒ष्टाव॑ष्टौ॒ तस्मा॒द् वेद॒ वेद॒ तस्मा॑ द॒ष्टाव॑ष्टौ । \newline
58. तस्मा॑ द॒ष्टाव॑ष्टा व॒ष्टाव॑ष्टौ॒ तस्मा॒त् तस्मा॑ द॒ष्टाव॑ष्टौ॒ कृत्वः॒ कृत्वो॒ ऽष्टाव॑ष्टौ॒ तस्मा॒त् तस्मा॑ द॒ष्टाव॑ष्टौ॒ कृत्वः॑ । \newline
59. अ॒ष्टाव॑ष्टौ॒ कृत्वः॒ कृत्वो॒ ऽष्टाव॑ष्टा व॒ष्टाव॑ष्टौ॒ कृत्वो॑ ऽभि॒षु त्य॑मभि॒षुत्य॒म् कृत्वो॒ ऽष्टाव॑ष्टा व॒ष्टाव॑ष्टौ॒ कृत्वो॑ ऽभि॒षुत्य᳚म् । \newline
60. अ॒ष्टाव॑ष्टा॒वित्य॒ष्टौ - अ॒ष्टौ॒ । \newline
\pagebreak
\markright{ TS 6.4.5.3  \hfill https://www.vedavms.in \hfill}

\section{ TS 6.4.5.3 }

\textbf{TS 6.4.5.3 } \newline
\textbf{Samhita Paata} \newline

कृत्वो॑ऽभि॒षुत्यं॑ ब्रह्मवा॒दिनो॑ वदन्ति प॒वित्र॑वन्तो॒ऽन्ये ग्रहा॑ गृ॒ह्यन्ते॒ किंप॑वित्र उपाꣳ॒॒शुरिति॒ वाक्प॑वित्र॒ इति॑ ब्रूयाद्-वा॒चस्पत॑ये पवस्व वाजि॒न्नित्या॑ह वा॒चैवैनं॑ पवयति॒ वृष्णो॑ अꣳ॒॒शुभ्या॒मित्या॑ह॒ वृष्णो॒ ह्ये॑तावꣳ॒॒शू यौ सोम॑स्य॒ गभ॑स्तिपूत॒ इत्या॑ह॒ गभ॑स्तिना॒ ह्ये॑नं प॒वय॑ति दे॒वो दे॒वानां᳚ प॒वित्र॑म॒सीत्या॑ह दे॒वो ह्ये॑ष- [  ] \newline

\textbf{Pada Paata} \newline

कृत्वः॑ । अ॒भि॒षुत्य॒मित्य॑भि-सुत्य᳚म् । ब्र॒ह्म॒वा॒दिन॒ इति॑ ब्रह्म-वा॒दिनः॑ । व॒द॒न्ति॒ । प॒वित्र॑वन्त॒ इति॑ प॒वित्र॑ - व॒न्तः॒ । अ॒न्ये । ग्रहाः᳚ । गृ॒ह्यन्ते᳚ । किम्प॑वित्र॒ इति॒ किम् - प॒वि॒त्रः॒ । उ॒पाꣳ॒॒शुरित्यु॑प - अꣳ॒॒शुः । इति॑ । वाक्प॑वित्र॒ इति॒ वाक्-प॒वि॒त्र॒ ः । इति॑ । ब्रू॒या॒त् । वा॒चः । पत॑ये । प॒व॒स्व॒ । वा॒जि॒न्न् । इति॑ । आ॒ह॒ । वा॒चा । ए॒व । ए॒न॒म् । प॒व॒य॒ति॒ । वृषः॑ । अꣳ॒॒शुभ्या॒मित्यꣳ॒॒शु - भ्या॒म् । इति॑ । आ॒ह॒ । वृष्णः॑ । हि । ए॒तौ । अꣳ॒॒शू इति॑ । यौ । सोम॑स्य । गभ॑स्तिपूत॒ इति॒ गभ॑स्ति - पू॒तः॒ । इति॑ । आ॒ह॒ । गभ॑स्तिना । हि । ए॒न॒म् । प॒वय॑ति । दे॒वः । दे॒वाना᳚म् । प॒वित्र᳚म् । अ॒सि॒ । इति॑ । आ॒ह॒ । दे॒वः । हि । ए॒षः ।  \newline


\textbf{Krama Paata} \newline

कृत्वो॑ऽभि॒षुत्य᳚म् । अ॒भि॒षुत्य॑म् ब्रह्मवा॒दिनः॑ । अ॒भि॒षुत्य॒मित्य॑भि - सुत्य᳚म् । ब्र॒ह्म॒वा॒दिनो॑ वदन्ति । ब्र॒ह्म॒वा॒दिन॒ इति॑ ब्रह्म - वा॒दिनः॑ । व॒द॒न्ति॒ प॒वित्र॑वन्तः । प॒वित्र॑वन्तो॒ऽन्ये । प॒वित्र॑वन्त॒ इति॑ प॒वित्र॑ - व॒न्तः॒ । अ॒न्ये ग्रहाः᳚ । ग्रहा॑ गृ॒ह्यन्ते᳚ । गृ॒ह्यन्ते॒ किम्प॑वित्रः । किम्प॑वित्र उपाꣳ॒॒शुः । किम्प॑वित्र॒ इति॒ किम् - प॒वि॒त्रः॒ । उ॒पाꣳ॒॒शुरिति॑ । उ॒पाꣳ॒॒शुरित्यु॑प - अꣳ॒॒शुः । इति॒ वाक्प॑वित्रः । वाक्प॑वित्र॒ इति॑ । वाक्प॑वित्र॒ इति॒ वाक् - प॒वि॒त्रः॒ । इति॑ ब्रूयात् । ब्रू॒या॒द् वा॒चः । वा॒चस्पत॑ये । पत॑ये पवस्व । प॒व॒स्व॒ वा॒जि॒न्न्॒ । वा॒जि॒न्निति॑ । इत्या॑ह । आ॒ह॒ वा॒चा । वा॒चैव । ए॒वैन᳚म् । ए॒न॒म् प॒व॒य॒ति॒ । प॒व॒य॒ति॒ वृष्णः॑ । वृष्णो॑ अꣳ॒॒शुभ्या᳚म् । अꣳ॒॒शुभ्या॒मिति॑ । अꣳ॒॒शुभ्या॒मित्यꣳ॒॒शु - भ्या॒म् । इत्या॑ह । आ॒ह॒ वृष्णः॑ । वृष्णो॒ हि । ह्ये॑तौ । ए॒तावꣳ॒॒शू । अꣳ॒॒शू यौ । अꣳ॒॒शू इत्यꣳ॒॒शू । यौ सोम॑स्य । सोम॑स्य॒ गभ॑स्तिपूतः । गभ॑स्तिपूत॒ इति॑ । गभ॑स्तिपूत॒ इति॒ गभ॑स्ति - पू॒तः॒ । इत्या॑ह । आ॒ह॒ गभ॑स्तिना । गभ॑स्तिना॒ हि । ह्ये॑नम् । ए॒न॒म् प॒वय॑ति । प॒वय॑ति दे॒वः । दे॒वो दे॒वाना᳚म् । दे॒वाना᳚म् प॒वित्र᳚म् । प॒वित्र॑मसि । अ॒सीति॑ । इत्या॑ह । आ॒ह॒ दे॒वः । दे॒वो हि । ह्ये॑षः । ए॒ष सन्न् \newline

\textbf{Jatai Paata} \newline

1. कृत्वो॑ ऽभि॒षुत्य॑ मभि॒षुत्य॒म् कृत्वः॒ कृत्वो॑ ऽभि॒षुत्य᳚म् । \newline
2. अ॒भि॒षुत्य॑म् ब्रह्मवा॒दिनो᳚ ब्रह्मवा॒दिनो॑ ऽभि॒षुत्य॑ मभि॒षुत्य॑म् ब्रह्मवा॒दिनः॑ । \newline
3. अ॒भि॒षुत्य॒मित्य॑भि - सुत्य᳚म् । \newline
4. ब्र॒ह्म॒वा॒दिनो॑ वदन्ति वदन्ति ब्रह्मवा॒दिनो᳚ ब्रह्मवा॒दिनो॑ वदन्ति । \newline
5. ब्र॒ह्म॒वा॒दिन॒ इति॑ ब्रह्म - वा॒दिनः॑ । \newline
6. व॒द॒न्ति॒ प॒वित्र॑वन्तः प॒वित्र॑वन्तो वदन्ति वदन्ति प॒वित्र॑वन्तः । \newline
7. प॒वित्र॑वन्तो॒ ऽन्ये᳚ ऽन्ये प॒वित्र॑वन्तः प॒वित्र॑वन्तो॒ ऽन्ये । \newline
8. प॒वित्र॑वन्त॒ इति॑ प॒वित्र॑ - व॒न्तः॒ । \newline
9. अ॒न्ये ग्रहा॒ ग्रहा॑ अ॒न्ये᳚ ऽन्ये ग्रहाः᳚ । \newline
10. ग्रहा॑ गृ॒ह्यन्ते॑ गृ॒ह्यन्ते॒ ग्रहा॒ ग्रहा॑ गृ॒ह्यन्ते᳚ । \newline
11. गृ॒ह्यन्ते॒ किम्प॑वित्रः॒ किम्प॑वित्रो गृ॒ह्यन्ते॑ गृ॒ह्यन्ते॒ किम्प॑वित्रः । \newline
12. किम्प॑वित्र उपाꣳ॒॒शु रु॑पाꣳ॒॒शुः किम्प॑वित्रः॒ किम्प॑वित्र उपाꣳ॒॒शुः । \newline
13. किम्प॑वित्र॒ इति॒ किम् - प॒वि॒त्रः॒ । \newline
14. उ॒पाꣳ॒॒शु रिती त्यु॑पाꣳ॒॒शु रु॑पाꣳ॒॒शु रिति॑ । \newline
15. उ॒पाꣳ॒॒शुरित्यु॑प - अꣳ॒॒शुः । \newline
16. इति॒ वाक्प॑वित्रो॒ वाक्प॑वित्र॒ इतीति॒ वाक्प॑वित्रः । \newline
17. वाक्प॑वित्र॒ इतीति॒ वाक्प॑वित्रो॒ वाक्प॑वित्र॒ इति॑ । \newline
18. वाक्प॑वित्र॒ इति॒ वाक् - प॒वि॒त्रः॒ । \newline
19. इति॑ ब्रूयाद् ब्रूया॒दितीति॑ ब्रूयात् । \newline
20. ब्रू॒या॒द् वा॒चो वा॒चो ब्रू॑याद् ब्रूयाद् वा॒चः । \newline
21. वा॒चस् पत॑ये॒ पत॑ये वा॒चो वा॒चस् पत॑ये । \newline
22. पत॑ये पवस्व पवस्व॒ पत॑ये॒ पत॑ये पवस्व । \newline
23. प॒व॒स्व॒ वा॒जि॒न्॒. वा॒जि॒न् प॒व॒स्व॒ प॒व॒स्व॒ वा॒जि॒न्न् । \newline
24. वा॒जि॒न्नितीति॑ वाजिन्. वाजि॒न् निति॑ । \newline
25. इत्या॑हा॒हे तीत्या॑ह । \newline
26. आ॒ह॒ वा॒चा वा॒चा ऽऽहा॑ह वा॒चा । \newline
27. वा॒चै वैव वा॒चा वा॒चैव । \newline
28. ए॒वैन॑ मेन मे॒वै वैन᳚म् । \newline
29. ए॒न॒म् प॒व॒य॒ति॒ प॒व॒य॒ त्ये॒न॒ मे॒न॒म् प॒व॒य॒ति॒ । \newline
30. प॒व॒य॒ति॒ वृष्णो॒ वृष्णः॑ पवयति पवयति॒ वृष्णः॑ । \newline
31. वृष्णो॑ अꣳ॒॒शुभ्या॑ मꣳ॒॒शुभ्यां॒ ॅवृष्णो॒ वृष्णो॑ अꣳ॒॒शुभ्या᳚म् । \newline
32. अꣳ॒॒शुभ्या॒ मिती त्यꣳ॒॒शुभ्या॑ मꣳ॒॒शुभ्या॒ मिति॑ । \newline
33. अꣳ॒॒शुभ्या॒मित्यꣳ॒॒शु - भ्या॒म् । \newline
34. इत्या॑हा॒हे तीत्या॑ह । \newline
35. आ॒ह॒ वृष्णो॒ वृष्ण॑ आहाह॒ वृष्णः॑ । \newline
36. वृष्णो॒ हि हि वृष्णो॒ वृष्णो॒ हि । \newline
37. ह्ये॑ता वे॒तौ हि ह्ये॑तौ । \newline
38. ए॒ता वꣳ॒॒शू अꣳ॒॒शू ए॒ता वे॒ता वꣳ॒॒शू । \newline
39. अꣳ॒॒शू यौ या वꣳ॒॒शू अꣳ॒॒शू यौ । \newline
40. अꣳ॒॒शू इत्यꣳ॒॒शू । \newline
41. यौ सोम॑स्य॒ सोम॑स्य॒ यौ यौ सोम॑स्य । \newline
42. सोम॑स्य॒ गभ॑स्तिपूतो॒ गभ॑स्तिपूतः॒ सोम॑स्य॒ सोम॑स्य॒ गभ॑स्तिपूतः । \newline
43. गभ॑स्तिपूत॒ इतीति॒ गभ॑स्तिपूतो॒ गभ॑स्तिपूत॒ इति॑ । \newline
44. गभ॑स्तिपूत॒ इति॒ गभ॑स्ति - पू॒तः॒ । \newline
45. इत्या॑हा॒हे तीत्या॑ह । \newline
46. आ॒ह॒ गभ॑स्तिना॒ गभ॑स्तिना ऽऽहाह॒ गभ॑स्तिना । \newline
47. गभ॑स्तिना॒ हि हि गभ॑स्तिना॒ गभ॑स्तिना॒ हि । \newline
48. ह्ये॑न मेनꣳ॒॒ हि ह्ये॑नम् । \newline
49. ए॒न॒म् प॒वय॑ति प॒वय॑ त्येन मेनम् प॒वय॑ति । \newline
50. प॒वय॑ति दे॒वो दे॒वः प॒वय॑ति प॒वय॑ति दे॒वः । \newline
51. दे॒वो दे॒वाना᳚म् दे॒वाना᳚म् दे॒वो दे॒वो दे॒वाना᳚म् । \newline
52. दे॒वाना᳚म् प॒वित्र॑म् प॒वित्र॑म् दे॒वाना᳚म् दे॒वाना᳚म् प॒वित्र᳚म् । \newline
53. प॒वित्र॑ मस्यसि प॒वित्र॑म् प॒वित्र॑ मसि । \newline
54. अ॒सीती त्य॑स्य॒ सीति॑ । \newline
55. इत्या॑हा॒हे तीत्या॑ह । \newline
56. आ॒ह॒ दे॒वो दे॒व आ॑हाह दे॒वः । \newline
57. दे॒वो हि हि दे॒वो दे॒वो हि । \newline
58. ह्ये॑ष ए॒ष हि ह्ये॑षः । \newline
59. ए॒ष सन् थ्सन् ने॒ष ए॒ष सन्न् । \newline

\textbf{Ghana Paata } \newline

1. कृत्वो॑ ऽभि॒षु त्य॑मभि॒षुत्य॒म् कृत्वः॒ कृत्वो॑ ऽभि॒षुत्य॑म् ब्रह्मवा॒दिनो᳚ ब्रह्मवा॒दिनो॑ ऽभि॒षुत्य॒म् कृत्वः॒ कृत्वो॑ ऽभि॒षुत्य॑म् ब्रह्मवा॒दिनः॑ । \newline
2. अ॒भि॒षुत्य॑म् ब्रह्मवा॒दिनो᳚ ब्रह्मवा॒दिनो॑ ऽभि॒षुत्य॑ मभि॒षुत्य॑म् ब्रह्मवा॒दिनो॑ वदन्ति वदन्ति ब्रह्मवा॒दिनो॑ ऽभि॒षु त्य॑मभि॒षुत्य॑म् ब्रह्मवा॒दिनो॑ वदन्ति । \newline
3. अ॒भि॒षुत्य॒मित्य॑भि - सुत्य᳚म् । \newline
4. ब्र॒ह्म॒वा॒दिनो॑ वदन्ति वदन्ति ब्रह्मवा॒दिनो᳚ ब्रह्मवा॒दिनो॑ वदन्ति प॒वित्र॑वन्तः प॒वित्र॑वन्तो वदन्ति ब्रह्मवा॒दिनो᳚ ब्रह्मवा॒दिनो॑ वदन्ति प॒वित्र॑वन्तः । \newline
5. ब्र॒ह्म॒वा॒दिन॒ इति॑ ब्रह्म - वा॒दिनः॑ । \newline
6. व॒द॒न्ति॒ प॒वित्र॑वन्तः प॒वित्र॑वन्तो वदन्ति वदन्ति प॒वित्र॑वन्तो॒ ऽन्ये᳚ ऽन्ये प॒वित्र॑वन्तो वदन्ति वदन्ति प॒वित्र॑वन्तो॒ ऽन्ये । \newline
7. प॒वित्र॑वन्तो॒ ऽन्ये᳚ ऽन्ये प॒वित्र॑वन्तः प॒वित्र॑वन्तो॒ ऽन्ये ग्रहा॒ ग्रहा॑ अ॒न्ये प॒वित्र॑वन्तः प॒वित्र॑वन्तो॒ ऽन्ये ग्रहाः᳚ । \newline
8. प॒वित्र॑वन्त॒ इति॑ प॒वित्र॑ - व॒न्तः॒ । \newline
9. अ॒न्ये ग्रहा॒ ग्रहा॑ अ॒न्ये᳚ ऽन्ये ग्रहा॑ गृ॒ह्यन्ते॑ गृ॒ह्यन्ते॒ ग्रहा॑ अ॒न्ये᳚ ऽन्ये ग्रहा॑ गृ॒ह्यन्ते᳚ । \newline
10. ग्रहा॑ गृ॒ह्यन्ते॑ गृ॒ह्यन्ते॒ ग्रहा॒ ग्रहा॑ गृ॒ह्यन्ते॒ किम्प॑वित्रः॒ किम्प॑वित्रो गृ॒ह्यन्ते॒ ग्रहा॒ ग्रहा॑ गृ॒ह्यन्ते॒ किम्प॑वित्रः । \newline
11. गृ॒ह्यन्ते॒ किम्प॑वित्रः॒ किम्प॑वित्रो गृ॒ह्यन्ते॑ गृ॒ह्यन्ते॒ किम्प॑वित्र उपाꣳ॒॒शु रु॑पाꣳ॒॒शुः किम्प॑वित्रो गृ॒ह्यन्ते॑ गृ॒ह्यन्ते॒ किम्प॑वित्र उपाꣳ॒॒शुः । \newline
12. किम्प॑वित्र उपाꣳ॒॒शु रु॑पाꣳ॒॒शुः किम्प॑वित्रः॒ किम्प॑वित्र उपाꣳ॒॒शु रिती त्यु॑पाꣳ॒॒शुः किम्प॑वित्रः॒ किम्प॑वित्र उपाꣳ॒॒शुरिति॑ । \newline
13. किम्प॑वित्र॒ इति॒ किम् - प॒वि॒त्रः॒ । \newline
14. उ॒पाꣳ॒॒शु रिती त्यु॑पाꣳ॒॒शु रु॑पाꣳ॒॒शुरिति॒ वाक्प॑वित्रो॒ वाक्प॑वित्र॒ इत्यु॑पाꣳ॒॒शु रु॑पाꣳ॒॒शु रिति॒ वाक्प॑वित्रः । \newline
15. उ॒पाꣳ॒॒शुरित्यु॑प - अꣳ॒॒शुः । \newline
16. इति॒ वाक्प॑वित्रो॒ वाक्प॑वित्र॒ इतीति॒ वाक्प॑वित्र॒ इतीति॒ वाक्प॑वित्र॒ इतीति॒ वाक्प॑वित्र॒ इति॑ । \newline
17. वाक्प॑वित्र॒ इतीति॒ वाक्प॑वित्रो॒ वाक्प॑वित्र॒ इति॑ ब्रूयाद् ब्रूया॒ दिति॒ वाक्प॑वित्रो॒ वाक्प॑वित्र॒ इति॑ ब्रूयात् । \newline
18. वाक्प॑वित्र॒ इति॒ वाक् - प॒वि॒त्रः॒ । \newline
19. इति॑ ब्रूयाद् ब्रूया॒दितीति॑ ब्रूयाद् वा॒चो वा॒चो ब्रू॑या॒दितीति॑ ब्रूयाद् वा॒चः । \newline
20. ब्रू॒या॒द् वा॒चो वा॒चो ब्रू॑याद् ब्रूयाद् वा॒च स्पत॑ये॒ पत॑ये वा॒चो ब्रू॑याद् ब्रूयाद् वा॒च स्पत॑ये । \newline
21. वा॒च स्पत॑ये॒ पत॑ये वा॒चो वा॒च स्पत॑ये पवस्व पवस्व॒ पत॑ये वा॒चो वा॒च स्पत॑ये पवस्व । \newline
22. पत॑ये पवस्व पवस्व॒ पत॑ये॒ पत॑ये पवस्व वाजिन्. वाजिन् पवस्व॒ पत॑ये॒ पत॑ये पवस्व वाजिन्न् । \newline
23. प॒व॒स्व॒ वा॒जि॒न्॒. वा॒जि॒न् प॒व॒स्व॒ प॒व॒स्व॒ वा॒जि॒न् नितीति॑ वाजिन् पवस्व पवस्व वाजि॒न् निति॑ । \newline
24. वा॒जि॒न् नितीति॑ वाजिन्. वाजि॒न् नित्या॑हा॒हेति॑ वाजिन्. वाजि॒न् नित्या॑ह । \newline
25. इत्या॑हा॒हे तीत्या॑ह वा॒चा वा॒चा ऽऽहे तीत्या॑ह वा॒चा । \newline
26. आ॒ह॒ वा॒चा वा॒चा ऽऽहा॑ह वा॒चै वैव वा॒चा ऽऽहा॑ह वा॒चैव । \newline
27. वा॒चै वैव वा॒चा वा॒चै वैन॑ मेन मे॒व वा॒चा वा॒चै वैन᳚म् । \newline
28. ए॒वैन॑ मेन मे॒वै वैन॑म् पवयति पवय त्येन मे॒वै वैन॑म् पवयति । \newline
29. ए॒न॒म् प॒व॒य॒ति॒ प॒व॒य॒ त्ये॒न॒ मे॒न॒म् प॒व॒य॒ति॒ वृष्णो॒ वृष्णः॑ पवय त्येन मेनम् पवयति॒ वृष्णः॑ । \newline
30. प॒व॒य॒ति॒ वृष्णो॒ वृष्णः॑ पवयति पवयति॒ वृष्णो॑ अꣳ॒॒शुभ्या॑ मꣳ॒॒शुभ्यां॒ ॅवृष्णः॑ पवयति पवयति॒ वृष्णो॑ अꣳ॒॒शुभ्या᳚म् । \newline
31. वृष्णो॑ अꣳ॒॒शुभ्या॑ मꣳ॒॒शुभ्यां॒ ॅवृष्णो॒ वृष्णो॑ अꣳ॒॒शुभ्या॒ मिती त्यꣳ॒॒शुभ्यां॒ ॅवृष्णो॒ वृष्णो॑ अꣳ॒॒शुभ्या॒ मिति॑ । \newline
32. अꣳ॒॒शुभ्या॒ मिती त्यꣳ॒॒शुभ्या॑ मꣳ॒॒शुभ्या॒ मित्या॑हा॒हे त्यꣳ॒॒शुभ्या॑ मꣳ॒॒शुभ्या॒ मित्या॑ह । \newline
33. अꣳ॒॒शुभ्या॒मित्यꣳ॒॒शु - भ्या॒म् । \newline
34. इत्या॑हा॒हे तीत्या॑ह॒ वृष्णो॒ वृष्ण॑ आ॒हे तीत्या॑ह॒ वृष्णः॑ । \newline
35. आ॒ह॒ वृष्णो॒ वृष्ण॑ आहाह॒ वृष्णो॒ हि हि वृष्ण॑ आहाह॒ वृष्णो॒ हि । \newline
36. वृष्णो॒ हि हि वृष्णो॒ वृष्णो॒ ह्ये॑ता वे॒तौ हि वृष्णो॒ वृष्णो॒ ह्ये॑तौ । \newline
37. ह्ये॑ता वे॒तौ हि ह्ये॑ता वꣳ॒॒शू अꣳ॒॒शू ए॒तौ हि ह्ये॑ता वꣳ॒॒शू । \newline
38. ए॒ता वꣳ॒॒शू अꣳ॒॒शू ए॒ता वे॒ता वꣳ॒॒शू यौ या वꣳ॒॒शू ए॒ता वे॒ता वꣳ॒॒शू यौ । \newline
39. अꣳ॒॒शू यौ या वꣳ॒॒शू अꣳ॒॒शू यौ सोम॑स्य॒ सोम॑स्य॒ या वꣳ॒॒शू अꣳ॒॒शू यौ सोम॑स्य । \newline
40. अꣳ॒॒शू इत्यꣳ॒॒शू । \newline
41. यौ सोम॑स्य॒ सोम॑स्य॒ यौ यौ सोम॑स्य॒ गभ॑स्तिपूतो॒ गभ॑स्तिपूतः॒ सोम॑स्य॒ यौ यौ सोम॑स्य॒ गभ॑स्तिपूतः । \newline
42. सोम॑स्य॒ गभ॑स्तिपूतो॒ गभ॑स्तिपूतः॒ सोम॑स्य॒ सोम॑स्य॒ गभ॑स्तिपूत॒ इतीति॒ गभ॑स्तिपूतः॒ सोम॑स्य॒ सोम॑स्य॒ गभ॑स्तिपूत॒ इति॑ । \newline
43. गभ॑स्तिपूत॒ इतीति॒ गभ॑स्तिपूतो॒ गभ॑स्तिपूत॒ इत्या॑हा॒हेति॒ गभ॑स्तिपूतो॒ गभ॑स्तिपूत॒ इत्या॑ह । \newline
44. गभ॑स्तिपूत॒ इति॒ गभ॑स्ति - पू॒तः॒ । \newline
45. इत्या॑हा॒हे तीत्या॑ह॒ गभ॑स्तिना॒ गभ॑स्तिना॒ ऽऽहेतीत्या॑ह॒ गभ॑स्तिना । \newline
46. आ॒ह॒ गभ॑स्तिना॒ गभ॑स्तिना ऽऽहाह॒ गभ॑स्तिना॒ हि हि गभ॑स्तिना ऽऽहाह॒ गभ॑स्तिना॒ हि । \newline
47. गभ॑स्तिना॒ हि हि गभ॑स्तिना॒ गभ॑स्तिना॒ ह्ये॑न मेनꣳ॒॒ हि गभ॑स्तिना॒ गभ॑स्तिना॒ ह्ये॑नम् । \newline
48. ह्ये॑न मेनꣳ॒॒ हि ह्ये॑नम् प॒वय॑ति प॒वय॑ त्येनꣳ॒॒ हि ह्ये॑नम् प॒वय॑ति । \newline
49. ए॒न॒म् प॒वय॑ति प॒वय॑ त्येन मेनम् प॒वय॑ति दे॒वो दे॒वः प॒वय॑ त्येन मेनम् प॒वय॑ति दे॒वः । \newline
50. प॒वय॑ति दे॒वो दे॒वः प॒वय॑ति प॒वय॑ति दे॒वो दे॒वाना᳚म् दे॒वाना᳚म् दे॒वः प॒वय॑ति प॒वय॑ति दे॒वो दे॒वाना᳚म् । \newline
51. दे॒वो दे॒वाना᳚म् दे॒वाना᳚म् दे॒वो दे॒वो दे॒वाना᳚म् प॒वित्र॑म् प॒वित्र॑म् दे॒वाना᳚म् दे॒वो दे॒वो दे॒वाना᳚म् प॒वित्र᳚म् । \newline
52. दे॒वाना᳚म् प॒वित्र॑म् प॒वित्र॑म् दे॒वाना᳚म् दे॒वाना᳚म् प॒वित्र॑ मस्यसि प॒वित्र॑म् दे॒वाना᳚म् दे॒वाना᳚म् प॒वित्र॑ मसि । \newline
53. प॒वित्र॑ मस्यसि प॒वित्र॑म् प॒वित्र॑ म॒सीती त्य॑सि प॒वित्र॑म् प॒वित्र॑ म॒सीति॑ । \newline
54. अ॒सीती त्य॑स्य॒ सीत्या॑ हा॒हे त्य॑स्य॒ सीत्या॑ह । \newline
55. इत्या॑हा॒हे तीत्या॑ह दे॒वो दे॒व आ॒हे तीत्या॑ह दे॒वः । \newline
56. आ॒ह॒ दे॒वो दे॒व आ॑हाह दे॒वो हि हि दे॒व आ॑हाह दे॒वो हि । \newline
57. दे॒वो हि हि दे॒वो दे॒वो ह्ये॑ष ए॒ष हि दे॒वो दे॒वो ह्ये॑षः । \newline
58. ह्ये॑ष ए॒ष हि ह्ये॑ष सन् थ्सन् ने॒ष हि ह्ये॑ष सन्न् । \newline
59. ए॒ष सन् थ्सन् ने॒ष ए॒ष सन् दे॒वाना᳚म् दे॒वानाꣳ॒॒ सन् ने॒ष ए॒ष सन् दे॒वाना᳚म् । \newline
\pagebreak
\markright{ TS 6.4.5.4  \hfill https://www.vedavms.in \hfill}

\section{ TS 6.4.5.4 }

\textbf{TS 6.4.5.4 } \newline
\textbf{Samhita Paata} \newline

सन् दे॒वानां᳚ प॒वित्रं॒ ॅयेषां᳚ भा॒गोऽसि॒ तेभ्य॒स्त्वेत्या॑ह॒ येषाꣳ॒॒ ह्ये॑ष भा॒गस्तेभ्य॑ एनं गृ॒ह्णाति॒ स्वां कृ॑तो॒ऽसीत्या॑ह प्रा॒णमे॒व स्वम॑कृत॒ मधु॑मतीर्न॒ इष॑स्कृ॒धीत्या॑ह॒ सर्व॑मे॒वास्मा॑ इ॒दꣳ स्व॑दयति॒ विश्वे᳚भ्य-स्त्वेन्द्रि॒येभ्यो॑ दि॒व्येभ्यः॒ पार्थि॑वेभ्य॒ इत्या॑हो॒भये᳚ष्वे॒व दे॑वमनु॒ष्येषु॑ प्रा॒णान् द॑धाति॒ मन॑स्त्वा॒- [  ] \newline

\textbf{Pada Paata} \newline

सन्न् । दे॒वाना᳚म् । प॒वित्र᳚म् । येषा᳚म् । भा॒गः । असि॑ । तेभ्यः॑ । त्वा॒ । इति॑ । आ॒ह॒ । येषा᳚म् । हि । ए॒षः । भा॒गः । तेभ्यः॑ । ए॒न॒म् । गृ॒ह्णाति॑ । स्वाङ्कृ॑तः । अ॒सि॒ । इति॑ । आ॒ह॒ । प्रा॒णमिति॑ प्र - अ॒नम् । ए॒व । स्वम् । अ॒कृ॒त॒ । मधु॑मती॒रिति॒ मधु॑ - म॒तीः॒ । नः॒ । इषः॑ । कृ॒धि॒ । इति॑ । आ॒ह॒ । सर्व᳚म् । ए॒व । अ॒स्मै॒ । इ॒दम् । स्व॒द॒य॒ति॒ । विश्वे᳚भ्यः । त्वा॒ । इ॒न्द्रि॒येभ्यः॑ । दि॒व्येभ्यः॑ । पार्थि॑वेभ्यः । इति॑ । आ॒ह॒ । उ॒भये॑षु । ए॒व । दे॒व॒म॒नु॒ष्येष्विति॑ देव - म॒नु॒ष्येषु॑ । प्रा॒णानिति॑ प्र - अ॒नान् । द॒धा॒ति॒ । मनः॑ । त्वा॒ ।  \newline


\textbf{Krama Paata} \newline

सन् दे॒वाना᳚म् । दे॒वाना᳚म् प॒वित्र᳚म् । प॒वित्र॒म् ॅयेषा᳚म् । येषा᳚म् भा॒गः । भा॒गोऽसि॑ । असि॒ तेभ्यः॑ । तेभ्य॑स्त्वा । त्वेति॑ । इत्या॑ह । आ॒ह॒ येषा᳚म् । येषाꣳ॒॒ हि । ह्ये॑षः । ए॒ष भा॒गः । भा॒गस्तेभ्यः॑ । तेभ्य॑ एनम् । ए॒न॒म् गृ॒ह्णाति॑ । गृ॒ह्णाति॒ स्वाङ्‍कृ॑तः । स्वाङ्‍कृ॑तोऽसि । अ॒सीति॑ । इत्या॑ह । आ॒ह॒ प्रा॒णम् । प्रा॒णमे॒व । प्रा॒णमिति॑ प्र - अ॒नम् । ए॒व स्वम् । स्वम॑कृत । अ॒कृ॒त॒ मधु॑मतीः । मधु॑मतीर् नः । मधु॑मती॒रिति॒ मधु॑ - म॒तीः॒ । न॒ इषः॑ । इष॑स्कृधि । कृ॒धीति॑ । इत्या॑ह । आ॒ह॒ सर्व᳚म् । सर्व॑मे॒व । ए॒वास्मै᳚ । अ॒स्मा॒ इ॒दम् । इ॒दꣳ स्व॑दयति । स्व॒द॒य॒ति॒ विश्वे᳚भ्यः । विश्वे᳚भ्यस्त्वा । त्वे॒न्द्रि॒येभ्यः॑ । इ॒न्द्रि॒येभ्यो॑ दि॒व्येभ्यः॑ । दि॒व्येभ्यः॒ पार्थि॑वेभ्यः । पार्थि॑वेभ्य॒ इति॑ । इत्या॑ह । आ॒हो॒भये॑षु । उ॒भये᳚ष्वे॒व । ए॒व दे॑वमनु॒ष्येषु॑ । दे॒व॒म॒नु॒ष्येषु॑ प्रा॒णान् । दे॒व॒म॒नु॒ष्येष्विति॑ देव - म॒नु॒ष्येषु॑ । प्रा॒णान् द॑धाति । प्रा॒णानिति॑ प्र - अ॒नान् । द॒धा॒ति॒ मनः॑ । मन॑स्त्वा । त्वा॒ऽष्टु॒ \newline

\textbf{Jatai Paata} \newline

1. सन् दे॒वाना᳚म् दे॒वानाꣳ॒॒ सन् थ्सन् दे॒वाना᳚म् । \newline
2. दे॒वाना᳚म् प॒वित्र॑म् प॒वित्र॑म् दे॒वाना᳚म् दे॒वाना᳚म् प॒वित्र᳚म् । \newline
3. प॒वित्रं॒ ॅयेषां॒ ॅयेषा᳚म् प॒वित्र॑म् प॒वित्रं॒ ॅयेषा᳚म् । \newline
4. येषा᳚म् भा॒गो भा॒गो येषां॒ ॅयेषा᳚म् भा॒गः । \newline
5. भा॒गो ऽस्यसि॑ भा॒गो भा॒गो ऽसि॑ । \newline
6. असि॒ तेभ्य॒ स्तेभ्यो ऽस्यसि॒ तेभ्यः॑ । \newline
7. तेभ्य॑ स्त्वा त्वा॒ तेभ्य॒ स्तेभ्य॑ स्त्वा । \newline
8. त्वेतीति॑ त्वा॒ त्वेति॑ । \newline
9. इत्या॑हा॒हे तीत्या॑ह । \newline
10. आ॒ह॒ येषां॒ ॅयेषा॑ माहाह॒ येषा᳚म् । \newline
11. येषाꣳ॒॒ हि हि येषां॒ ॅयेषाꣳ॒॒ हि । \newline
12. ह्ये॑ष ए॒ष हि ह्ये॑षः । \newline
13. ए॒ष भा॒गो भा॒ग ए॒ष ए॒ष भा॒गः । \newline
14. भा॒ग स्तेभ्य॒ स्तेभ्यो॑ भा॒गो भा॒ग स्तेभ्यः॑ । \newline
15. तेभ्य॑ एन मेन॒म् तेभ्य॒ स्तेभ्य॑ एनम् । \newline
16. ए॒न॒म् गृ॒ह्णाति॑ गृ॒ह्णा त्ये॑न मेनम् गृ॒ह्णाति॑ । \newline
17. गृ॒ह्णाति॒ स्वाङ्कृ॑तः॒ स्वाङ्कृ॑तो गृ॒ह्णाति॑ गृ॒ह्णाति॒ स्वाङ्कृ॑तः । \newline
18. स्वाङ्कृ॑तो ऽस्यसि॒ स्वाङ्कृ॑तः॒ स्वाङ्कृ॑तो ऽसि । \newline
19. अ॒सीती त्य॑स्य॒ सीति॑ । \newline
20. इत्या॑हा॒हे तीत्या॑ह । \newline
21. आ॒ह॒ प्रा॒णम् प्रा॒ण मा॑हाह प्रा॒णम् । \newline
22. प्रा॒ण मे॒वैव प्रा॒णम् प्रा॒ण मे॒व । \newline
23. प्रा॒णमिति॑ प्र - अ॒नम् । \newline
24. ए॒व स्वꣳ स्व मे॒वैव स्वम् । \newline
25. स्व म॑कृता कृत॒ स्वꣳ स्व म॑कृत । \newline
26. अ॒कृ॒त॒ मधु॑मती॒र् मधु॑मती रकृता कृत॒ मधु॑मतीः । \newline
27. मधु॑मतीर् नो नो॒ मधु॑मती॒र् मधु॑मतीर् नः । \newline
28. मधु॑मती॒रिति॒ मधु॑ - म॒तीः॒ । \newline
29. न॒ इष॒ इषो॑ नो न॒ इषः॑ । \newline
30. इष॑ स्कृधि कृ॒धीष॒ इष॑ स्कृधि । \newline
31. कृ॒धीतीति॑ कृधि कृ॒धीति॑ । \newline
32. इत्या॑हा॒हे तीत्या॑ह । \newline
33. आ॒ह॒ सर्वꣳ॒॒ सर्व॑ माहाह॒ सर्व᳚म् । \newline
34. सर्व॑ मे॒वैव सर्वꣳ॒॒ सर्व॑ मे॒व । \newline
35. ए॒वास्मा॑ अस्मा ए॒वै वास्मै᳚ । \newline
36. अ॒स्मा॒ इ॒द मि॒द म॑स्मा अस्मा इ॒दम् । \newline
37. इ॒दꣳ स्व॑दयति स्वदयती॒द मि॒दꣳ स्व॑दयति । \newline
38. स्व॒द॒य॒ति॒ विश्वे᳚भ्यो॒ विश्वे᳚भ्यः स्वदयति स्वदयति॒ विश्वे᳚भ्यः । \newline
39. विश्वे᳚भ्य स्त्वा त्वा॒ विश्वे᳚भ्यो॒ विश्वे᳚भ्य स्त्वा । \newline
40. त्वे॒न्द्रि॒येभ्य॑ इन्द्रि॒येभ्य॑ स्त्वा त्वेन्द्रि॒येभ्यः॑ । \newline
41. इ॒न्द्रि॒येभ्यो॑ दि॒व्येभ्यो॑ दि॒व्येभ्य॑ इन्द्रि॒येभ्य॑ इन्द्रि॒येभ्यो॑ दि॒व्येभ्यः॑ । \newline
42. दि॒व्येभ्यः॒ पार्थि॑वेभ्यः॒ पार्थि॑वेभ्यो दि॒व्येभ्यो॑ दि॒व्येभ्यः॒ पार्थि॑वेभ्यः । \newline
43. पार्थि॑वेभ्य॒ इतीति॒ पार्थि॑वेभ्यः॒ पार्थि॑वेभ्य॒ इति॑ । \newline
44. इत्या॑हा॒हे तीत्या॑ह । \newline
45. आ॒हो॒ भये॑षू॒ भये᳚ ष्वाहाहो॒ भये॑षु । \newline
46. उ॒भये᳚ ष्वे॒वै वोभये॑षू॒ भये᳚ ष्वे॒व । \newline
47. ए॒व दे॑वमनु॒ष्येषु॑ देवमनु॒ष्ये ष्वे॒वैव दे॑वमनु॒ष्येषु॑ । \newline
48. दे॒व॒म॒नु॒ष्येषु॑ प्रा॒णान् प्रा॒णान् दे॑वमनु॒ष्येषु॑ देवमनु॒ष्येषु॑ प्रा॒णान् । \newline
49. दे॒व॒म॒नु॒ष्येष्विति॑ देव - म॒नु॒ष्येषु॑ । \newline
50. प्रा॒णान् द॑धाति दधाति प्रा॒णान् प्रा॒णान् द॑धाति । \newline
51. प्रा॒णानिति॑ प्र - अ॒नान् । \newline
52. द॒धा॒ति॒ मनो॒ मनो॑ दधाति दधाति॒ मनः॑ । \newline
53. मन॑ स्त्वा त्वा॒ मनो॒ मन॑ स्त्वा । \newline
54. त्वा॒ ऽष्ट् व॒ष्टु॒ त्वा॒ त्वा॒ ऽष्टु॒ । \newline

\textbf{Ghana Paata } \newline

1. सन् दे॒वाना᳚म् दे॒वानाꣳ॒॒ सन् थ्सन् दे॒वाना᳚म् प॒वित्र॑म् प॒वित्र॑म् दे॒वानाꣳ॒॒ सन् थ्सन् दे॒वाना᳚म् प॒वित्र᳚म् । \newline
2. दे॒वाना᳚म् प॒वित्र॑म् प॒वित्र॑म् दे॒वाना᳚म् दे॒वाना᳚म् प॒वित्रं॒ ॅयेषां॒ ॅयेषा᳚म् प॒वित्र॑म् दे॒वाना᳚म् दे॒वाना᳚म् प॒वित्रं॒ ॅयेषा᳚म् । \newline
3. प॒वित्रं॒ ॅयेषां॒ ॅयेषा᳚म् प॒वित्र॑म् प॒वित्रं॒ ॅयेषा᳚म् भा॒गो भा॒गो येषा᳚म् प॒वित्र॑म् प॒वित्रं॒ ॅयेषा᳚म् भा॒गः । \newline
4. येषा᳚म् भा॒गो भा॒गो येषां॒ ॅयेषा᳚म् भा॒गो ऽस्यसि॑ भा॒गो येषां॒ ॅयेषा᳚म् भा॒गो ऽसि॑ । \newline
5. भा॒गो ऽस्यसि॑ भा॒गो भा॒गो ऽसि॒ तेभ्य॒ स्तेभ्यो ऽसि॑ भा॒गो भा॒गो ऽसि॒ तेभ्यः॑ । \newline
6. असि॒ तेभ्य॒ स्तेभ्यो ऽस्यसि॒ तेभ्य॑ स्त्वा त्वा॒ तेभ्यो ऽस्यसि॒ तेभ्य॑ स्त्वा । \newline
7. तेभ्य॑ स्त्वा त्वा॒ तेभ्य॒ स्तेभ्य॒ स्त्वेतीति॑ त्वा॒ तेभ्य॒ स्तेभ्य॒ स्त्वेति॑ । \newline
8. त्वेतीति॑ त्वा॒ त्वेत्या॑हा॒ हेति॑ त्वा॒ त्वेत्या॑ह । \newline
9. इत्या॑हा॒हे तीत्या॑ह॒ येषां॒ ॅयेषा॑ मा॒हे तीत्या॑ह॒ येषा᳚म् । \newline
10. आ॒ह॒ येषां॒ ॅयेषा॑ माहाह॒ येषाꣳ॒॒ हि हि येषा॑ माहाह॒ येषाꣳ॒॒ हि । \newline
11. येषाꣳ॒॒ हि हि येषां॒ ॅयेषाꣳ॒॒ ह्ये॑ष ए॒ष हि येषां॒ ॅयेषाꣳ॒॒ ह्ये॑षः । \newline
12. ह्ये॑ष ए॒ष हि ह्ये॑ष भा॒गो भा॒ग ए॒ष हि ह्ये॑ष भा॒गः । \newline
13. ए॒ष भा॒गो भा॒ग ए॒ष ए॒ष भा॒ग स्तेभ्य॒ स्तेभ्यो॑ भा॒ग ए॒ष ए॒ष भा॒ग स्तेभ्यः॑ । \newline
14. भा॒ग स्तेभ्य॒ स्तेभ्यो॑ भा॒गो भा॒ग स्तेभ्य॑ एन मेन॒म् तेभ्यो॑ भा॒गो भा॒ग स्तेभ्य॑ एनम् । \newline
15. तेभ्य॑ एन मेन॒म् तेभ्य॒ स्तेभ्य॑ एनम् गृ॒ह्णाति॑ गृ॒ह्णा त्ये॑न॒म् तेभ्य॒ स्तेभ्य॑ एनम् गृ॒ह्णाति॑ । \newline
16. ए॒न॒म् गृ॒ह्णाति॑ गृ॒ह्णा त्ये॑न मेनम् गृ॒ह्णाति॒ स्वाङ्कृ॑तः॒ स्वाङ्कृ॑तो गृ॒ह्णा त्ये॑न मेनम् गृ॒ह्णाति॒ स्वाङ्कृ॑तः । \newline
17. गृ॒ह्णाति॒ स्वाङ्कृ॑तः॒ स्वाङ्कृ॑तो गृ॒ह्णाति॑ गृ॒ह्णाति॒ स्वाङ्कृ॑तो ऽस्यसि॒ स्वाङ्कृ॑तो गृ॒ह्णाति॑ गृ॒ह्णाति॒ स्वाङ्कृ॑तो ऽसि । \newline
18. स्वाङ्कृ॑तो ऽस्यसि॒ स्वाङ्कृ॑तः॒ स्वाङ्कृ॑तो॒ ऽसीती त्य॑सि॒ स्वाङ्कृ॑तः॒ स्वाङ्कृ॑तो॒ ऽसीति॑ । \newline
19. अ॒सीती त्य॑स्य॒ सीत्या॑हा॒हे त्य॑स्य॒ सीत्या॑ह । \newline
20. इत्या॑हा॒हे तीत्या॑ह प्रा॒णम् प्रा॒ण मा॒हे तीत्या॑ह प्रा॒णम् । \newline
21. आ॒ह॒ प्रा॒णम् प्रा॒ण मा॑हाह प्रा॒ण मे॒वैव प्रा॒ण मा॑हाह प्रा॒ण मे॒व । \newline
22. प्रा॒ण मे॒वैव प्रा॒णम् प्रा॒ण मे॒व स्वꣳ स्व मे॒व प्रा॒णम् प्रा॒ण मे॒व स्वम् । \newline
23. प्रा॒णमिति॑ प्र - अ॒नम् । \newline
24. ए॒व स्वꣳ स्व मे॒वैव स्व म॑कृता कृत॒ स्व मे॒वैव स्व म॑कृत । \newline
25. स्व म॑कृता कृत॒ स्वꣳ स्व म॑कृत॒ मधु॑मती॒र् मधु॑मती रकृत॒ स्वꣳ स्व म॑कृत॒ मधु॑मतीः । \newline
26. अ॒कृ॒त॒ मधु॑मती॒र् मधु॑मती रकृता कृत॒ मधु॑मतीर् नो नो॒ मधु॑मती रकृता कृत॒ मधु॑मतीर् नः । \newline
27. मधु॑मतीर् नो नो॒ मधु॑मती॒र् मधु॑मतीर् न॒ इष॒ इषो॑ नो॒ मधु॑मती॒र् मधु॑मतीर् न॒ इषः॑ । \newline
28. मधु॑मती॒रिति॒ मधु॑ - म॒तीः॒ । \newline
29. न॒ इष॒ इषो॑ नो न॒ इष॑ स्कृधि कृ॒धीषो॑ नो न॒ इष॑ स्कृधि । \newline
30. इष॑ स्कृधि कृ॒धीष॒ इष॑ स्कृ॒धीतीति॑ कृ॒धीष॒ इष॑ स्कृ॒धीति॑ । \newline
31. कृ॒धीतीति॑ कृधि कृ॒धी त्या॑हा॒ हेति॑ कृधि कृ॒धी त्या॑ह । \newline
32. इत्या॑हा॒हे तीत्या॑ह॒ सर्वꣳ॒॒ सर्व॑ मा॒हे तीत्या॑ह॒ सर्व᳚म् । \newline
33. आ॒ह॒ सर्वꣳ॒॒ सर्व॑ माहाह॒ सर्व॑ मे॒वैव सर्व॑ माहाह॒ सर्व॑ मे॒व । \newline
34. सर्व॑ मे॒वैव सर्वꣳ॒॒ सर्व॑ मे॒वास्मा॑ अस्मा ए॒व सर्वꣳ॒॒ सर्व॑ मे॒वास्मै᳚ । \newline
35. ए॒वास्मा॑ अस्मा ए॒वै वास्मा॑ इ॒द मि॒द म॑स्मा ए॒वै वास्मा॑ इ॒दम् । \newline
36. अ॒स्मा॒ इ॒द मि॒द म॑स्मा अस्मा इ॒दꣳ स्व॑दयति स्वदयती॒द म॑स्मा अस्मा इ॒दꣳ स्व॑दयति । \newline
37. इ॒दꣳ स्व॑दयति स्वदयती॒द मि॒दꣳ स्व॑दयति॒ विश्वे᳚भ्यो॒ विश्वे᳚भ्यः स्वदयती॒द मि॒दꣳ स्व॑दयति॒ विश्वे᳚भ्यः । \newline
38. स्व॒द॒य॒ति॒ विश्वे᳚भ्यो॒ विश्वे᳚भ्यः स्वदयति स्वदयति॒ विश्वे᳚भ्य स्त्वा त्वा॒ विश्वे᳚भ्यः स्वदयति स्वदयति॒ विश्वे᳚भ्य स्त्वा । \newline
39. विश्वे᳚भ्य स्त्वा त्वा॒ विश्वे᳚भ्यो॒ विश्वे᳚भ्य स्त्वेन्द्रि॒येभ्य॑ इन्द्रि॒येभ्य॑ स्त्वा॒ विश्वे᳚भ्यो॒ विश्वे᳚भ्य स्त्वेन्द्रि॒येभ्यः॑ । \newline
40. त्वे॒न्द्रि॒येभ्य॑ इन्द्रि॒येभ्य॑ स्त्वा त्वेन्द्रि॒येभ्यो॑ दि॒व्येभ्यो॑ दि॒व्येभ्य॑ इन्द्रि॒येभ्य॑ स्त्वा त्वेन्द्रि॒येभ्यो॑ दि॒व्येभ्यः॑ । \newline
41. इ॒न्द्रि॒येभ्यो॑ दि॒व्येभ्यो॑ दि॒व्येभ्य॑ इन्द्रि॒येभ्य॑ इन्द्रि॒येभ्यो॑ दि॒व्येभ्यः॒ पार्थि॑वेभ्यः॒ पार्थि॑वेभ्यो दि॒व्येभ्य॑ इन्द्रि॒येभ्य॑ इन्द्रि॒येभ्यो॑ दि॒व्येभ्यः॒ पार्थि॑वेभ्यः । \newline
42. दि॒व्येभ्यः॒ पार्थि॑वेभ्यः॒ पार्थि॑वेभ्यो दि॒व्येभ्यो॑ दि॒व्येभ्यः॒ पार्थि॑वेभ्य॒ इतीति॒ पार्थि॑वेभ्यो दि॒व्येभ्यो॑ दि॒व्येभ्यः॒ पार्थि॑वेभ्य॒ इति॑ । \newline
43. पार्थि॑वेभ्य॒ इतीति॒ पार्थि॑वेभ्यः॒ पार्थि॑वेभ्य॒ इत्या॑हा॒ हेति॒ पार्थि॑वेभ्यः॒ पार्थि॑वेभ्य॒ इत्या॑ह । \newline
44. इत्या॑हा॒हे तीत्या॑ हो॒भये॑ षू॒भये᳚ ष्वा॒हेतीत्या॑ हो॒भये॑षु । \newline
45. आ॒हो॒भये॑ षू॒भये᳚ ष्वाहा हो॒भये᳚ ष्वे॒वै वोभये᳚ ष्वाहा हो॒भये᳚ ष्वे॒व । \newline
46. उ॒भये᳚ ष्वे॒वै वोभये॑ षू॒भये᳚ ष्वे॒व दे॑वमनु॒ष्येषु॑ देवमनु॒ष्ये ष्वे॒वोभये॑ षू॒भये᳚ष्वे॒व दे॑वमनु॒ष्येषु॑ । \newline
47. ए॒व दे॑वमनु॒ष्येषु॑ देवमनु॒ष्ये ष्वे॒वैव दे॑वमनु॒ष्येषु॑ प्रा॒णान् प्रा॒णान् दे॑वमनु॒ष्ये
ष्वे॒वैव दे॑वमनु॒ष्येषु॑ प्रा॒णान् । \newline
48. दे॒व॒म॒नु॒ष्येषु॑ प्रा॒णान् प्रा॒णान् दे॑वमनु॒ष्येषु॑ देवमनु॒ष्येषु॑ प्रा॒णान् द॑धाति दधाति प्रा॒णान् दे॑वमनु॒ष्येषु॑ देवमनु॒ष्येषु॑ प्रा॒णान् द॑धाति । \newline
49. दे॒व॒म॒नु॒ष्येष्विति॑ देव - म॒नु॒ष्येषु॑ । \newline
50. प्रा॒णान् द॑धाति दधाति प्रा॒णान् प्रा॒णान् द॑धाति॒ मनो॒ मनो॑ दधाति प्रा॒णान् प्रा॒णान् द॑धाति॒ मनः॑ । \newline
51. प्रा॒णानिति॑ प्र - अ॒नान् । \newline
52. द॒धा॒ति॒ मनो॒ मनो॑ दधाति दधाति॒ मन॑ स्त्वा त्वा॒ मनो॑ दधाति दधाति॒ मन॑ स्त्वा । \newline
53. मन॑ स्त्वा त्वा॒ मनो॒ मन॑ स्त्वा ऽष्ट्वष्टु त्वा॒ मनो॒ मन॑ स्त्वा ऽष्टु । \newline
54. त्वा॒ ऽष्ट्व॒ष्टु॒ त्वा॒ त्वा॒ऽष्ट् विती त्य॑ष्टु त्वा त्वा॒ऽष्ट् विति॑ । \newline
\pagebreak
\markright{ TS 6.4.5.5  \hfill https://www.vedavms.in \hfill}

\section{ TS 6.4.5.5 }

\textbf{TS 6.4.5.5 } \newline
\textbf{Samhita Paata} \newline

ऽष्ट्वित्या॑ह॒ मन॑ ए॒वाश्नु॑त उ॒र्व॑न्तरि॑क्ष॒-मन्वि॒हीत्या॑हा-न्तरिक्षदेव॒त्यो॑ हि प्रा॒णः स्वाहा᳚ त्वा सुभवः॒ सूर्या॒येत्या॑ह प्रा॒णा वै स्वभ॑वसो दे॒वास्तेष्वे॒व प॒रोक्षं॑ जुहोति दे॒वेभ्य॑स्त्वा मरीचि॒पेभ्य॒ इत्या॑हा ऽऽ*दि॒त्यस्य॒ वै र॒श्मयो॑ दे॒वा म॑रीचि॒पास्तेषां॒ तद् भा॑ग॒धेयं॒ ताने॒व तेन॑ प्रीणाति॒ यदि॑ का॒मये॑त॒ वर्.षु॑कः प॒र्जन्यः॑- [  ] \newline

\textbf{Pada Paata} \newline

अ॒ष्टु॒ । इति॑ । आ॒ह॒ । मनः॑ । ए॒व । आ॒श्नु॒ते॒ । उ॒रु । अ॒न्तरि॑क्षम् । अन्विति॑ । इ॒हि॒ । इति॑ । आ॒ह॒ । अ॒न्त॒रि॒क्ष॒दे॒व॒त्य॑ इत्य॑न्तरिक्ष-दे॒व॒त्यः॑ । हि । प्रा॒ण इति॑ प्र - अ॒नः । स्वाहा᳚ । त्वा॒ । सु॒भ॒व॒ इति॑ सु - भ॒वः॒ । सूर्या॑य । इति॑ । आ॒ह॒ । प्रा॒णा इति॑ प्र - अ॒नाः । वै । स्वभ॑वस॒ इति॒ स्व - भ॒व॒सः॒ । दे॒वाः । तेषु॑ । ए॒व । प॒रोक्ष॒मिति॑ परः - अक्ष᳚म् । जु॒हो॒ति॒ । दे॒वेभ्यः॑ । त्वा॒ । म॒री॒चि॒पेभ्य॒ इति॑ मरीचि - पेभ्यः॑ । इति॑ । आ॒ह॒ । आ॒दि॒त्यस्य॑ । वै । र॒श्मयः॑ । दे॒वाः । म॒री॒चि॒पा इति॑ मरीचि-पाः । तेषा᳚म् । तत् । भा॒ग॒धेय॒मिति॑ भाग-धेय᳚म् । तान् । ए॒व । तेन॑ । प्री॒णा॒ति॒ । यदि॑ । का॒मये॑त । वर्.षु॑कः । प॒र्जन्यः॑ ।  \newline


\textbf{Krama Paata} \newline

अ॒ष्ट्विति॑ । इत्या॑ह । आ॒ह॒ मनः॑ । मन॑ ए॒व । ए॒वाश्ञु॑ते । अ॒श्ञु॒त॒ उ॒रु । उ॒र्व॑न्तरि॑क्षम् । अ॒न्तरि॑क्ष॒मनु॑ । अन्वि॑हि । इ॒हीति॑ । इत्या॑ह । आ॒हा॒न्त॒रि॒क्ष॒दे॒व॒त्यः॑ । अ॒न्त॒रि॒क्ष॒दे॒व॒त्यो॑ हि । अ॒न्त॒रि॒क्ष॒दे॒व॒त्य॑ इत्य॑न्तरिक्ष - दे॒व॒त्यः॑ । हि प्रा॒णः । प्रा॒णः स्वाहा᳚ । प्रा॒ण इति॑ प्र - अ॒नः । स्वाहा᳚ त्वा । त्वा॒ सु॒भ॒वः॒ । सु॒भ॒वः॒ सूर्या॑य । सु॒भ॒व॒ इति॑ सु - भ॒वः॒ । सूर्या॒येति॑ । इत्या॑ह । आ॒ह॒ प्रा॒णाः । प्रा॒णा वै । प्रा॒णा इति॑ प्र - अ॒नाः । वै स्वभ॑वसः । स्वभ॑वसो दे॒वाः । स्वभ॑वस॒ इति॒ स्व - भ॒व॒सः॒ । दे॒वास्तेषु॑ । तेष्वे॒व । ए॒व प॒रोक्ष᳚म् । प॒रोक्ष॑म् जुहोति । प॒रोक्ष॒मिति॑ परः - अक्ष᳚म् । जु॒हो॒ति॒ दे॒वेभ्यः॑ । दे॒वेभ्य॑स्त्वा । त्वा॒ म॒री॒चि॒पेभ्य॑ । म॒री॒चि॒पेभ्य॒ इति॑ । म॒री॒चि॒पेभ्य॒ इति॑ मरीचि - पेभ्यः॑ । इत्या॑ह । आ॒हा॒दि॒त्यस्य॑ । आ॒दि॒त्यस्य॒ वै । वै र॒श्मयः॑ । र॒श्मयो॑ दे॒वाः । दे॒वा म॑रीचि॒पाः । म॒री॒चि॒पास्तेषा᳚म् । म॒री॒चि॒पा इति॑ मरीचि - पाः । तेषा॒म् तत् । तद् भा॑ग॒धेय᳚म् । भा॒ग॒धेय॒म् तान् । भा॒ग॒धेय॒मिति॑ भाग - धेय᳚म् । ताने॒व । ए॒व तेन॑ । तेन॑ प्रीणाति । प्री॒णा॒ति॒ यदि॑ । यदि॑ का॒मये॑त । का॒मये॑त॒ वर्.षु॑कः । वर्.षु॑कः प॒र्जन्यः॑ । प॒र्जन्यः॑ स्यात् \newline

\textbf{Jatai Paata} \newline

1. अ॒ष्ट् विती त्य॑ष्ट् व॒ष्ट् विति॑ । \newline
2. इत्या॑हा॒हे तीत्या॑ह । \newline
3. आ॒ह॒ मनो॒ मन॑ आहाह॒ मनः॑ । \newline
4. मन॑ ए॒वैव मनो॒ मन॑ ए॒व । \newline
5. ए॒वाश्ञु॑ते ऽश्ञुत ए॒वैवा श्ञु॑ते । \newline
6. अ॒श्ञु॒त॒ उ॒रू᳚(1॒)र्व॑श्ञुते ऽश्ञुत उ॒रु । \newline
7. उ॒र्व॑न्तरि॑क्ष म॒न्तरि॑क्ष मु॒रू᳚(1॒)र्व॑न्तरि॑क्षम् । \newline
8. अ॒न्तरि॑क्ष॒ मन् वन् व॒न्तरि॑क्ष म॒न्तरि॑क्ष॒ मनु॑ । \newline
9. अन्वि॑ही॒ ह्यन्वन् वि॑हि । \newline
10. इ॒ही तीती॑ ही॒हीति॑ । \newline
11. इत्या॑हा॒हे तीत्या॑ह । \newline
12. आ॒हा॒ न्त॒रि॒क्ष॒दे॒व॒त्यो᳚ ऽन्तरिक्षदेव॒त्य॑ आहाहा न्तरिक्षदेव॒त्यः॑ । \newline
13. अ॒न्त॒रि॒क्ष॒दे॒व॒त्यो॑ हि ह्य॑न्तरिक्षदेव॒त्यो᳚ ऽन्तरिक्षदेव॒त्यो॑ हि । \newline
14. अ॒न्त॒रि॒क्ष॒दे॒व॒त्य॑ इत्य॑न्तरिक्ष - दे॒व॒त्यः॑ । \newline
15. हि प्रा॒णः प्रा॒णो हि हि प्रा॒णः । \newline
16. प्रा॒णः स्वाहा॒ स्वाहा᳚ प्रा॒णः प्रा॒णः स्वाहा᳚ । \newline
17. प्रा॒ण इति॑ प्र - अ॒नः । \newline
18. स्वाहा᳚ त्वा त्वा॒ स्वाहा॒ स्वाहा᳚ त्वा । \newline
19. त्वा॒ सु॒भ॒वः॒ सु॒भ॒व॒ स्त्वा॒ त्वा॒ सु॒भ॒वः॒ । \newline
20. सु॒भ॒वः॒ सूर्या॑य॒ सूर्या॑य सुभवः सुभवः॒ सूर्या॑य । \newline
21. सु॒भ॒व॒ इति॑ सु - भ॒वः॒ । \newline
22. सूर्या॒येतीति॒ सूर्या॑य॒ सूर्या॒येति॑ । \newline
23. इत्या॑हा॒हे तीत्या॑ह । \newline
24. आ॒ह॒ प्रा॒णाः प्रा॒णा आ॑हाह प्रा॒णाः । \newline
25. प्रा॒णा वै वै प्रा॒णाः प्रा॒णा वै । \newline
26. प्रा॒णा इति॑ प्र - अ॒नाः । \newline
27. वै स्वभ॑वसः॒ स्वभ॑वसो॒ वै वै स्वभ॑वसः । \newline
28. स्वभ॑वसो दे॒वा दे॒वाः स्वभ॑वसः॒ स्वभ॑वसो दे॒वाः । \newline
29. स्वभ॑वस॒ इति॒ स्व - भ॒व॒सः॒ । \newline
30. दे॒वा स्तेषु॒ तेषु॑ दे॒वा दे॒वा स्तेषु॑ । \newline
31. तेष्वे॒वैव तेषु॒ तेष्वे॒व । \newline
32. ए॒व प॒रोक्ष॑म् प॒रोक्ष॑ मे॒वैव प॒रोक्ष᳚म् । \newline
33. प॒रोक्ष॑म् जुहोति जुहोति प॒रोक्ष॑म् प॒रोक्ष॑म् जुहोति । \newline
34. प॒रोक्ष॒मिति॑ परः - अक्ष᳚म् । \newline
35. जु॒हो॒ति॒ दे॒वेभ्यो॑ दे॒वेभ्यो॑ जुहोति जुहोति दे॒वेभ्यः॑ । \newline
36. दे॒वेभ्य॑ स्त्वा त्वा दे॒वेभ्यो॑ दे॒वेभ्य॑ स्त्वा । \newline
37. त्वा॒ म॒री॒चि॒पेभ्यो॑ मरीचि॒पेभ्य॑ स्त्वा त्वा मरीचि॒पेभ्यः॑ । \newline
38. म॒री॒चि॒पेभ्य॒ इतीति॑ मरीचि॒पेभ्यो॑ मरीचि॒पेभ्य॒ इति॑ । \newline
39. म॒री॒चि॒पेभ्य॒ इति॑ मरीचि - पेभ्यः॑ । \newline
40. इत्या॑हा॒हे तीत्या॑ह । \newline
41. आ॒हा॒ दि॒त्यस्या॑ दि॒त्यस्या॑ हाहा दि॒त्यस्य॑ । \newline
42. आ॒दि॒त्यस्य॒ वै वा आ॑दि॒त्यस्या॑ दि॒त्यस्य॒ वै । \newline
43. वै र॒श्मयो॑ र॒श्मयो॒ वै वै र॒श्मयः॑ । \newline
44. र॒श्मयो॑ दे॒वा दे॒वा र॒श्मयो॑ र॒श्मयो॑ दे॒वाः । \newline
45. दे॒वा म॑रीचि॒पा म॑रीचि॒पा दे॒वा दे॒वा म॑रीचि॒पाः । \newline
46. म॒री॒चि॒पा स्तेषा॒म् तेषा᳚म् मरीचि॒पा म॑रीचि॒पा स्तेषा᳚म् । \newline
47. म॒री॒चि॒पा इति॑ मरीचि - पाः । \newline
48. तेषा॒म् तत् तत् तेषा॒म् तेषा॒म् तत् । \newline
49. तद् भा॑ग॒धेय॑म् भाग॒धेय॒म् तत् तद् भा॑ग॒धेय᳚म् । \newline
50. भा॒ग॒धेय॒म् ताꣳ स्तान् भा॑ग॒धेय॑म् भाग॒धेय॒म् तान् । \newline
51. भा॒ग॒धेय॒मिति॑ भाग - धेय᳚म् । \newline
52. ताने॒वैव ताꣳ स्ता ने॒व । \newline
53. ए॒वते न॒ते नै॒वैव तेन॑ । \newline
54. तेन॑ प्रीणाति प्रीणाति॒ तेन॒ तेन॑ प्रीणाति । \newline
55. प्री॒णा॒ति॒ यदि॒ यदि॑ प्रीणाति प्रीणाति॒ यदि॑ । \newline
56. यदि॑ का॒मये॑त का॒मये॑त॒ यदि॒ यदि॑ का॒मये॑त । \newline
57. का॒मये॑त॒ वर्.षु॑को॒ वर्.षु॑कः का॒मये॑त का॒मये॑त॒ वर्.षु॑कः । \newline
58. वर्.षु॑कः प॒र्जन्यः॑ प॒र्जन्यो॒ वर्.षु॑को॒ वर्.षु॑कः प॒र्जन्यः॑ । \newline
59. प॒र्जन्यः॑ स्याथ् स्यात् प॒र्जन्यः॑ प॒र्जन्यः॑ स्यात् । \newline

\textbf{Ghana Paata } \newline

1. अ॒ष्ट् विती त्य॑ष्ट् व॒ष्ट् वित्या॑हा॒हे त्य॑ष्ट् व॒ष्ट् वित्या॑ह । \newline
2. इत्या॑हा॒हे तीत्या॑ह॒ मनो॒ मन॑ आ॒हे तीत्या॑ह॒ मनः॑ । \newline
3. आ॒ह॒ मनो॒ मन॑ आहाह॒ मन॑ ए॒वैव मन॑ आहाह॒ मन॑ ए॒व । \newline
4. मन॑ ए॒वैव मनो॒ मन॑ ए॒वा श्ञु॑ते ऽश्ञुत ए॒व मनो॒ मन॑ ए॒वा श्ञु॑ते । \newline
5. ए॒वा श्ञु॑ते ऽश्ञुत ए॒वैवा श्ञु॑त उ॒रू᳚(1॒)र्व॑श्ञुत ए॒वैवा श्ञु॑त उ॒रु । \newline
6. अ॒श्ञु॒त॒ उ॒रू᳚(1॒)र्व॑श्ञुते ऽश्ञुत उ॒र्व॑न्तरि॑क्ष म॒न्तरि॑क्ष मु॒र्व॑श्ञुते ऽश्ञुत उ॒र्व॑न्तरि॑क्षम् । \newline
7. उ॒र्व॑न्तरि॑क्ष म॒न्तरि॑क्ष मु॒रू᳚(1॒)र्व॑न्तरि॑क्ष॒ मन्वन् व॒न्तरि॑क्ष मु॒रू᳚(1॒)र्व॑न्तरि॑क्ष॒ मनु॑ । \newline
8. अ॒न्तरि॑क्ष॒ मन्वन् व॒न्तरि॑क्ष म॒न्तरि॑क्ष॒ मन् वि॑ही॒ ह्यन् व॒न्तरि॑क्ष म॒न्तरि॑क्ष॒ मन् वि॑हि । \newline
9. अन् वि॑ही॒ ह्यन् वन् वि॒ही तीती॒ ह्यन् वन् वि॒हीति॑ । \newline
10. इ॒ही तीती॑ ही॒ही त्या॑हा॒हेती॑ ही॒ही त्या॑ह । \newline
11. इत्या॑हा॒हे तीत्या॑ हान्तरिक्षदेव॒त्यो᳚ ऽन्तरिक्षदेव॒त्य॑ आ॒हे तीत्या॑ हान्तरिक्षदेव॒त्यः॑ । \newline
12. आ॒हा॒न्त॒रि॒क्ष॒दे॒व॒त्यो᳚ ऽन्तरिक्षदेव॒त्य॑ आहा हान्तरिक्षदेव॒त्यो॑ हि ह्य॑न्तरिक्षदेव॒त्य॑ आहा हान्तरिक्षदेव॒त्यो॑ हि । \newline
13. अ॒न्त॒रि॒क्ष॒दे॒व॒त्यो॑ हि ह्य॑न्तरिक्षदेव॒त्यो᳚ ऽन्तरिक्षदेव॒त्यो॑ हि प्रा॒णः प्रा॒णो ह्य॑न्तरिक्षदेव॒त्यो᳚ ऽन्तरिक्षदेव॒त्यो॑ हि प्रा॒णः । \newline
14. अ॒न्त॒रि॒क्ष॒दे॒व॒त्य॑ इत्य॑न्तरिक्ष - दे॒व॒त्यः॑ । \newline
15. हि प्रा॒णः प्रा॒णो हि हि प्रा॒णः स्वाहा॒ स्वाहा᳚ प्रा॒णो हि हि प्रा॒णः स्वाहा᳚ । \newline
16. प्रा॒णः स्वाहा॒ स्वाहा᳚ प्रा॒णः प्रा॒णः स्वाहा᳚ त्वा त्वा॒ स्वाहा᳚ प्रा॒णः प्रा॒णः स्वाहा᳚ त्वा । \newline
17. प्रा॒ण इति॑ प्र - अ॒नः । \newline
18. स्वाहा᳚ त्वा त्वा॒ स्वाहा॒ स्वाहा᳚ त्वा सुभवः सुभव स्त्वा॒ स्वाहा॒ स्वाहा᳚ त्वा सुभवः । \newline
19. त्वा॒ सु॒भ॒वः॒ सु॒भ॒व॒ स्त्वा॒ त्वा॒ सु॒भ॒वः॒ सूर्या॑य॒ सूर्या॑य सुभव स्त्वा त्वा सुभवः॒ सूर्या॑य । \newline
20. सु॒भ॒वः॒ सूर्या॑य॒ सूर्या॑य सुभवः सुभवः॒ सूर्या॒येतीति॒ सूर्या॑य सुभवः सुभवः॒ सूर्या॒येति॑ । \newline
21. सु॒भ॒व॒ इति॑ सु - भ॒वः॒ । \newline
22. सूर्या॒ येतीति॒ सूर्या॑य॒ सूर्या॒ येत्या॑हा॒हेति॒ सूर्या॑य॒ सूर्या॒ येत्या॑ह । \newline
23. इत्या॑हा॒हे तीत्या॑ह प्रा॒णाः प्रा॒णा आ॒हे तीत्या॑ह प्रा॒णाः । \newline
24. आ॒ह॒ प्रा॒णाः प्रा॒णा आ॑हाह प्रा॒णा वै वै प्रा॒णा आ॑हाह प्रा॒णा वै । \newline
25. प्रा॒णा वै वै प्रा॒णाः प्रा॒णा वै स्वभ॑वसः॒ स्वभ॑वसो॒ वै प्रा॒णाः प्रा॒णा वै स्वभ॑वसः । \newline
26. प्रा॒णा इति॑ प्र - अ॒नाः । \newline
27. वै स्वभ॑वसः॒ स्वभ॑वसो॒ वै वै स्वभ॑वसो दे॒वा दे॒वाः स्वभ॑वसो॒ वै वै स्वभ॑वसो दे॒वाः । \newline
28. स्वभ॑वसो दे॒वा दे॒वाः स्वभ॑वसः॒ स्वभ॑वसो दे॒वा स्तेषु॒ तेषु॑ दे॒वाः स्वभ॑वसः॒ स्वभ॑वसो दे॒वा स्तेषु॑ । \newline
29. स्वभ॑वस॒ इति॒ स्व - भ॒व॒सः॒ । \newline
30. दे॒वा स्तेषु॒ तेषु॑ दे॒वा दे॒वा स्तेष्वे॒वैव तेषु॑ दे॒वा दे॒वा स्तेष्वे॒व । \newline
31. तेष्वे॒वैव तेषु॒ तेष्वे॒व प॒रोक्ष॑म् प॒रोक्ष॑ मे॒व तेषु॒ तेष्वे॒व प॒रोक्ष᳚म् । \newline
32. ए॒व प॒रोक्ष॑म् प॒रोक्ष॑ मे॒वैव प॒रोक्ष॑म् जुहोति जुहोति प॒रोक्ष॑ मे॒वैव प॒रोक्ष॑म् जुहोति । \newline
33. प॒रोक्ष॑म् जुहोति जुहोति प॒रोक्ष॑म् प॒रोक्ष॑म् जुहोति दे॒वेभ्यो॑ दे॒वेभ्यो॑ जुहोति प॒रोक्ष॑म् प॒रोक्ष॑म् जुहोति दे॒वेभ्यः॑ । \newline
34. प॒रोक्ष॒मिति॑ परः - अक्ष᳚म् । \newline
35. जु॒हो॒ति॒ दे॒वेभ्यो॑ दे॒वेभ्यो॑ जुहोति जुहोति दे॒वेभ्य॑ स्त्वा त्वा दे॒वेभ्यो॑ जुहोति जुहोति दे॒वेभ्य॑ स्त्वा । \newline
36. दे॒वेभ्य॑ स्त्वा त्वा दे॒वेभ्यो॑ दे॒वेभ्य॑ स्त्वा मरीचि॒पेभ्यो॑ मरीचि॒पेभ्य॑ स्त्वा दे॒वेभ्यो॑ दे॒वेभ्य॑ स्त्वा मरीचि॒पेभ्यः॑ । \newline
37. त्वा॒ म॒री॒चि॒पेभ्यो॑ मरीचि॒पेभ्य॑ स्त्वा त्वा मरीचि॒पेभ्य॒ इतीति॑ मरीचि॒पेभ्य॑ स्त्वा त्वा मरीचि॒पेभ्य॒ इति॑ । \newline
38. म॒री॒चि॒पेभ्य॒ इतीति॑ मरीचि॒पेभ्यो॑ मरीचि॒पेभ्य॒ इत्या॑हा॒हेति॑ मरीचि॒पेभ्यो॑ मरीचि॒पेभ्य॒ इत्या॑ह । \newline
39. म॒री॒चि॒पेभ्य॒ इति॑ मरीचि - पेभ्यः॑ । \newline
40. इत्या॑हा॒हेती त्या॑हादि॒त्यस्या॑ दि॒त्यस्या॒ हेतीत्या॑हा दि॒त्यस्य॑ । \newline
41. आ॒हा॒दि॒त्यस्या॑ दि॒त्यस्या॑ हाहा दि॒त्यस्य॒ वै वा आ॑दि॒त्यस्या॑ हाहा दि॒त्यस्य॒ वै । \newline
42. आ॒दि॒त्यस्य॒ वै वा आ॑दि॒त्यस्या॑ दि॒त्यस्य॒ वै र॒श्मयो॑ र॒श्मयो॒ वा आ॑दि॒त्यस्या॑ दि॒त्यस्य॒ वै र॒श्मयः॑ । \newline
43. वै र॒श्मयो॑ र॒श्मयो॒ वै वै र॒श्मयो॑ दे॒वा दे॒वा र॒श्मयो॒ वै वै र॒श्मयो॑ दे॒वाः । \newline
44. र॒श्मयो॑ दे॒वा दे॒वा र॒श्मयो॑ र॒श्मयो॑ दे॒वा म॑रीचि॒पा म॑रीचि॒पा दे॒वा र॒श्मयो॑ र॒श्मयो॑ दे॒वा म॑रीचि॒पाः । \newline
45. दे॒वा म॑रीचि॒पा म॑रीचि॒पा दे॒वा दे॒वा म॑रीचि॒पा स्तेषा॒म् तेषा᳚म् मरीचि॒पा दे॒वा दे॒वा म॑रीचि॒पा स्तेषा᳚म् । \newline
46. म॒री॒चि॒पा स्तेषा॒म् तेषा᳚म् मरीचि॒पा म॑रीचि॒पा स्तेषा॒म् तत् तत् तेषा᳚म् मरीचि॒पा म॑रीचि॒पा स्तेषा॒म् तत् । \newline
47. म॒री॒चि॒पा इति॑ मरीचि - पाः । \newline
48. तेषा॒म् तत् तत् तेषा॒म् तेषा॒म् तद् भा॑ग॒धेय॑म् भाग॒धेय॒म् तत् तेषा॒म् तेषा॒म् तद् भा॑ग॒धेय᳚म् । \newline
49. तद् भा॑ग॒धेय॑म् भाग॒धेय॒म् तत् तद् भा॑ग॒धेय॒म् ताꣳ स्तान् भा॑ग॒धेय॒म् तत् तद् भा॑ग॒धेय॒म् तान् । \newline
50. भा॒ग॒धेय॒म् ताꣳ स्तान् भा॑ग॒धेय॑म् भाग॒धेय॒म् ताने॒ वैव तान् भा॑ग॒धेय॑म् भाग॒धेय॒म् ताने॒व । \newline
51. भा॒ग॒धेय॒मिति॑ भाग - धेय᳚म् । \newline
52. ताने॒ वैव ताꣳ स्ताने॒व तेन॒ तेनै॒व ताꣳ स्ताने॒व तेन॑ । \newline
53. ए॒व तेन॒ तेनै॒ वैव तेन॑ प्रीणाति प्रीणाति॒ तेनै॒ वैव तेन॑ प्रीणाति । \newline
54. तेन॑ प्रीणाति प्रीणाति॒ तेन॒ तेन॑ प्रीणाति॒ यदि॒ यदि॑ प्रीणाति॒ तेन॒ तेन॑ प्रीणाति॒ यदि॑ । \newline
55. प्री॒णा॒ति॒ यदि॒ यदि॑ प्रीणाति प्रीणाति॒ यदि॑ का॒मये॑त का॒मये॑त॒ यदि॑ प्रीणाति प्रीणाति॒ यदि॑ का॒मये॑त । \newline
56. यदि॑ का॒मये॑त का॒मये॑त॒ यदि॒ यदि॑ का॒मये॑त॒ वर्.षु॑को॒ वर्.षु॑कः का॒मये॑त॒ यदि॒ यदि॑ का॒मये॑त॒ वर्.षु॑कः । \newline
57. का॒मये॑त॒ वर्.षु॑को॒ वर्.षु॑कः का॒मये॑त का॒मये॑त॒ वर्.षु॑कः प॒र्जन्यः॑ प॒र्जन्यो॒ वर्.षु॑कः का॒मये॑त का॒मये॑त॒ वर्.षु॑कः प॒र्जन्यः॑ । \newline
58. वर्.षु॑कः प॒र्जन्यः॑ प॒र्जन्यो॒ वर्.षु॑को॒ वर्.षु॑कः प॒र्जन्यः॑ स्याथ् स्यात् प॒र्जन्यो॒ वर्.षु॑को॒ वर्.षु॑कः प॒र्जन्यः॑ स्यात् । \newline
59. प॒र्जन्यः॑ स्याथ् स्यात् प॒र्जन्यः॑ प॒र्जन्यः॑ स्या॒दितीति॑ स्यात् प॒र्जन्यः॑ प॒र्जन्यः॑ स्या॒दिति॑ । \newline
\pagebreak
\markright{ TS 6.4.5.6  \hfill https://www.vedavms.in \hfill}

\section{ TS 6.4.5.6 }

\textbf{TS 6.4.5.6 } \newline
\textbf{Samhita Paata} \newline

स्या॒दिति॒ नीचा॒ हस्ते॑न॒ नि मृ॑ज्या॒द्-वृष्टि॑मे॒व नि य॑च्छति॒ यदि॑ का॒मये॒ताव॑र्.षुकः स्या॒दित्यु॑त्ता॒नेन॒ नि मृ॑ज्या॒द्-वृष्टि॑मे॒वोद्-य॑च्छति॒ यद्य॑भि॒चरे॑द॒मुं ज॒ह्यथ॑ त्वा होष्या॒मीति॑ ब्रूया॒दाहु॑तिमे॒वैनं॑ प्रे॒फ्सन्. ह॑न्ति॒ यदि॑ दू॒रे स्यादा तमि॑तोस्तिष्ठेत् प्रा॒णमे॒वास्या॑नु॒गत्य॑ हन्ति॒ यद्य॑भि॒चरे॑द॒मुष्य॑- [  ] \newline

\textbf{Pada Paata} \newline

स्या॒त् । इति॑ । नीचा᳚ । हस्ते॑न । नीति॑ । मृ॒ज्या॒त् । वृष्टि᳚म् । ए॒व । नीति॑ । य॒च्छ॒ति॒ । यदि॑ । का॒मये॑त । अव॑र्.षुकः । स्या॒त् । इति॑ । उ॒त्ता॒नेनेत्यु॑त् - ता॒नेन॑ । नीति॑ । मृ॒ज्या॒त् । वृष्टि᳚म् । ए॒व । उदिति॑ । य॒च्छ॒ति॒ । यदि॑ । अ॒भि॒चरे॒दित्य॑भि - चरे᳚त् । अ॒मुम् । ज॒हि॒ । अथ॑ । त्वा॒ । हो॒ष्या॒मि॒ । इति॑ । ब्रू॒या॒त् । आहु॑ति॒मित्या - हु॒ति॒म् । ए॒व । ए॒न॒म् । प्रे॒फ्सन्निति॑ प्र-ई॒फ्सन्न् । ह॒न्ति॒ । यदि॑ । दू॒रे । स्यात् । एति॑ । तमि॑तोः । ति॒ष्ठे॒त् । प्रा॒णमिति॑ प्र - अ॒नम् । ए॒व । अ॒स्य॒ । अ॒नु॒गत्येत्य॑नु - गत्य॑ । ह॒न्ति॒ । यदि॑ । अ॒भि॒चरे॒दित्य॑भि - चरे᳚त् । अ॒मुष्य॑ ।  \newline


\textbf{Krama Paata} \newline

स्या॒दिति॑ । इति॒ नीचा᳚ । नीचा॒ हस्ते॑न । हस्ते॑न॒ नि । नि मृ॑ज्यात् । मृ॒ज्या॒द् वृष्टि᳚म् । वृष्टि॑मे॒व । ए॒व नि । नि य॑च्छति । य॒च्छ॒ति॒ यदि॑ । यदि॑ का॒मये॑त । का॒मये॒ताव॑र्.षुकः । अव॑र्.षुकः स्यात् । स्या॒दिति॑ । इत्यु॑त्ता॒नेन॑ । उ॒त्ता॒नेन॒ नि । उ॒त्ता॒नेनेत्यु॑त् - ता॒नेन॑ । नि मृ॑ज्यात् । मृ॒ज्या॒द् वृष्टि᳚म् । वृष्टि॑मे॒व । ए॒वोत् । उद् य॑च्छति । य॒च्छ॒ति॒ यदि॑ । यद्य॑भि॒चरे᳚त् । अ॒भि॒चरे॑द॒मुम् । अ॒भि॒चरे॒दित्य॑भि - चरे᳚त् । अ॒मुम् ज॑हि । ज॒ह्यथ॑ । अथ॑ त्वा । त्वा॒ हो॒ष्या॒मि॒ । हो॒ष्या॒मीति॑ । इति॑ ब्रूयात् । ब्रू॒या॒दाहु॑तिम् । आहु॑तिमे॒व । आहु॑ति॒मित्या - हु॒ति॒म् । ए॒वैन᳚म् । ए॒न॒म् प्रे॒फ्सन्न् । प्रे॒फ्सन्. ह॑न्ति । प्रे॒फ्सन्निति॑ प्र - ई॒फ्सन्न् । ह॒न्ति॒ यदि॑ । यदि॑ दू॒रे । दू॒रे स्यात् । स्यादा । आ तमि॑तोः । तमि॑तोस्तिष्ठेत् । ति॒ष्ठे॒त् प्रा॒णम् । प्रा॒णमे॒व । प्रा॒णमिति॑ प्र - अ॒नम् । ए॒वास्य॑ । अ॒स्या॒नु॒गत्य॑ । अ॒नु॒गत्य॑ हन्ति । अ॒नु॒गत्येत्य॑नु - गत्य॑ । ह॒न्ति॒ यदि॑ । यद्य॑भि॒चरे᳚त् । अ॒भि॒चरे॑द॒मुष्य॑ । अ॒भि॒चरे॒दित्य॑भि - चरे᳚त् । अ॒मुष्य॑ त्वा \newline

\textbf{Jatai Paata} \newline

1. स्या॒ दितीति॑ स्याथ् स्या॒ दिति॑ । \newline
2. इति॒ नीचा॒ नीचे तीति॒ नीचा᳚ । \newline
3. नीचा॒ हस्ते॑न॒ हस्ते॑न॒ नीचा॒ नीचा॒ हस्ते॑न । \newline
4. हस्ते॑न॒ नि नि हस्ते॑न॒ हस्ते॑न॒ नि । \newline
5. नि मृ॑ज्यान् मृज्या॒न् नि नि मृ॑ज्यात् । \newline
6. मृ॒ज्या॒द् वृष्टिं॒ ॅवृष्टि॑म् मृज्यान् मृज्या॒द् वृष्टि᳚म् । \newline
7. वृष्टि॑ मे॒वैव वृष्टिं॒ ॅवृष्टि॑ मे॒व । \newline
8. ए॒व नि न्ये॑वैव नि । \newline
9. नि य॑च्छति यच्छति॒ नि नि य॑च्छति । \newline
10. य॒च्छ॒ति॒ यदि॒ यदि॑ यच्छति यच्छति॒ यदि॑ । \newline
11. यदि॑ का॒मये॑त का॒मये॑त॒ यदि॒ यदि॑ का॒मये॑त । \newline
12. का॒मये॒ता व॑र्.षु॒को ऽव॑र्.षुकः का॒मये॑त का॒मये॒ता व॑र्.षुकः । \newline
13. अव॑र्.षुकः स्याथ् स्या॒ दव॑र्.षु॒को ऽव॑र्.षुकः स्यात् । \newline
14. स्या॒दि तीति॑ स्याथ् स्या॒ दिति॑ । \newline
15. इत्यु॑त्ता॒नेनो᳚ त्ता॒नेने तीत्यु॑त्ता॒नेन॑ । \newline
16. उ॒त्ता॒नेन॒ नि न्यु॑त्ता॒ नेनो᳚त्ता॒नेन॒ नि । \newline
17. उ॒त्ता॒नेनेत्यु॑त् - ता॒नेन॑ । \newline
18. नि मृ॑ज्यान् मृज्या॒न् नि नि मृ॑ज्यात् । \newline
19. मृ॒ज्या॒द् वृष्टिं॒ ॅवृष्टि॑म् मृज्यान् मृज्या॒द् वृष्टि᳚म् । \newline
20. वृष्टि॑ मे॒वैव वृष्टिं॒ ॅवृष्टि॑ मे॒व । \newline
21. ए॒वोदु दे॒वैवोत् । \newline
22. उद् य॑च्छति यच्छ॒ त्युदुद् य॑च्छति । \newline
23. य॒च्छ॒ति॒ यदि॒ यदि॑ यच्छति यच्छति॒ यदि॑ । \newline
24. यद्य॑भि॒चरे॑ दभि॒चरे॒द् यदि॒ यद्य॑भि॒चरे᳚त् । \newline
25. अ॒भि॒चरे॑ द॒मु म॒मु म॑भि॒चरे॑ दभि॒चरे॑ द॒मुम् । \newline
26. अ॒भि॒चरे॒दित्य॑भि - चरे᳚त् । \newline
27. अ॒मुम् ज॑हि जह्य॒मु म॒मुम् ज॑हि । \newline
28. ज॒ह्य थाथ॑ जहि ज॒ह्यथ॑ । \newline
29. अथ॑ त्वा॒ त्वा ऽथाथ॑ त्वा । \newline
30. त्वा॒ हो॒ष्या॒मि॒ हो॒ष्या॒मि॒ त्वा॒ त्वा॒ हो॒ष्या॒मि॒ । \newline
31. हो॒ष्या॒मी तीति॑ होष्यामि होष्या॒ मीति॑ । \newline
32. इति॑ ब्रूयाद् ब्रूया॒ दितीति॑ ब्रूयात् । \newline
33. ब्रू॒या॒ दाहु॑ति॒ माहु॑तिम् ब्रूयाद् ब्रूया॒ दाहु॑तिम् । \newline
34. आहु॑ति मे॒वै वाहु॑ति॒ माहु॑ति मे॒व । \newline
35. आहु॑ति॒मित्या - हु॒ति॒म् । \newline
36. ए॒वैन॑ मेन मे॒वै वैन᳚म् । \newline
37. ए॒न॒म् प्रे॒फ्सन् प्रे॒फ्सन् ने॑न मेनम् प्रे॒फ्सन्न् । \newline
38. प्रे॒फ्सन्. ह॑न्ति हन्ति प्रे॒फ्सन् प्रे॒फ्सन्. ह॑न्ति । \newline
39. प्रे॒फ्सन्निति॑ प्र - ई॒फ्सन्न् । \newline
40. ह॒न्ति॒ यदि॒ यदि॑ हन्ति हन्ति॒ यदि॑ । \newline
41. यदि॑ दू॒रे दू॒रे यदि॒ यदि॑ दू॒रे । \newline
42. दू॒रे स्याथ् स्याद् दू॒रे दू॒रे स्यात् । \newline
43. स्यादा स्याथ् स्यादा । \newline
44. आ तमि॑तो॒ स्तमि॑तो॒रा तमि॑तोः । \newline
45. तमि॑तो स्तिष्ठेत् तिष्ठे॒त् तमि॑तो॒ स्तमि॑तो स्तिष्ठेत् । \newline
46. ति॒ष्ठे॒त् प्रा॒णम् प्रा॒णम् ति॑ष्ठेत् तिष्ठेत् प्रा॒णम् । \newline
47. प्रा॒ण मे॒वैव प्रा॒णम् प्रा॒ण मे॒व । \newline
48. प्रा॒णमिति॑ प्र - अ॒नम् । \newline
49. ए॒वास्या᳚ स्यै॒वै वास्य॑ । \newline
50. अ॒स्या॒ नु॒गत्या॑ नु॒गत्या᳚ स्यास्या नु॒गत्य॑ । \newline
51. अ॒नु॒गत्य॑ हन्ति हन्त्य नु॒गत्या॑ नु॒गत्य॑ हन्ति । \newline
52. अ॒नु॒गत्येत्य॑नु - गत्य॑ । \newline
53. ह॒न्ति॒ यदि॒ यदि॑ हन्ति हन्ति॒ यदि॑ । \newline
54. यद्य॑भि॒चरे॑ दभि॒चरे॒द् यदि॒ यद्य॑भि॒चरे᳚त् । \newline
55. अ॒भि॒चरे॑ द॒मुष्या॒ मुष्या॑ भि॒चरे॑ दभि॒चरे॑ द॒मुष्य॑ । \newline
56. अ॒भि॒चरे॒दित्य॑भि - चरे᳚त् । \newline
57. अ॒मुष्य॑ त्वा त्वा॒ ऽमुष्या॒ मुष्य॑ त्वा । \newline

\textbf{Ghana Paata } \newline

1. स्या॒दि तीति॑ स्याथ् स्या॒ दिति॒ नीचा॒ नीचेति॑ स्याथ् स्या॒ दिति॒ नीचा᳚ । \newline
2. इति॒ नीचा॒ नीचेतीति॒ नीचा॒ हस्ते॑न॒ हस्ते॑न॒ नीचेतीति॒ नीचा॒ हस्ते॑न । \newline
3. नीचा॒ हस्ते॑न॒ हस्ते॑न॒ नीचा॒ नीचा॒ हस्ते॑न॒ नि नि हस्ते॑न॒ नीचा॒ नीचा॒ हस्ते॑न॒ नि । \newline
4. हस्ते॑न॒ नि नि हस्ते॑न॒ हस्ते॑न॒ नि मृ॑ज्यान् मृज्या॒न् नि हस्ते॑न॒ हस्ते॑न॒ नि मृ॑ज्यात् । \newline
5. नि मृ॑ज्यान् मृज्या॒न् नि नि मृ॑ज्या॒द् वृष्टिं॒ ॅवृष्टि॑म् मृज्या॒न् नि नि मृ॑ज्या॒द् वृष्टि᳚म् । \newline
6. मृ॒ज्या॒द् वृष्टिं॒ ॅवृष्टि॑म् मृज्यान् मृज्या॒द् वृष्टि॑ मे॒वैव वृष्टि॑म् मृज्यान् मृज्या॒द् वृष्टि॑ मे॒व । \newline
7. वृष्टि॑ मे॒वैव वृष्टिं॒ ॅवृष्टि॑ मे॒व नि न्ये॑व वृष्टिं॒ ॅवृष्टि॑ मे॒व नि । \newline
8. ए॒व नि न्ये॑वैव नि य॑च्छति यच्छति॒ न्ये॑वैव नि य॑च्छति । \newline
9. नि य॑च्छति यच्छति॒ नि नि य॑च्छति॒ यदि॒ यदि॑ यच्छति॒ नि नि य॑च्छति॒ यदि॑ । \newline
10. य॒च्छ॒ति॒ यदि॒ यदि॑ यच्छति यच्छति॒ यदि॑ का॒मये॑त का॒मये॑त॒ यदि॑ यच्छति यच्छति॒ यदि॑ का॒मये॑त । \newline
11. यदि॑ का॒मये॑त का॒मये॑त॒ यदि॒ यदि॑ का॒मये॒ता व॑र्.षु॒को ऽव॑र्.षुकः का॒मये॑त॒ यदि॒ यदि॑ का॒मये॒ता व॑र्.षुकः । \newline
12. का॒मये॒ता व॑र्.षु॒को ऽव॑र्.षुकः का॒मये॑त का॒मये॒ता व॑र्.षुकः स्याथ् स्या॒ दव॑र्.षुकः का॒मये॑त का॒मये॒ता व॑र्.षुकः स्यात् । \newline
13. अव॑र्.षुकः स्याथ् स्या॒ दव॑र्.षु॒को ऽव॑र्.षुकः स्या॒दितीति॑ स्या॒ दव॑र्.षु॒को ऽव॑र्.षुकः स्या॒ दिति॑ । \newline
14. स्या॒दि तीति॑ स्याथ् स्या॒ दित्यु॑त्ता॒ने नो᳚त्ता॒नेनेति॑ स्याथ् स्या॒ दित्यु॑त्ता॒नेन॑ । \newline
15. इत्यु॑त्ता॒ने नो᳚त्ता॒नेने तीत्यु॑त्ता॒नेन॒ नि न्यु॑त्ता॒नेने तीत्यु॑त्ता॒नेन॒ नि । \newline
16. उ॒त्ता॒नेन॒ नि न्यु॑त्ता॒नेनो᳚ त्ता॒नेन॒ नि मृ॑ज्यान् मृज्या॒न् न्यु॑त्ता॒ने नो᳚त्ता॒नेन॒ नि मृ॑ज्यात् । \newline
17. उ॒त्ता॒नेनेत्यु॑त् - ता॒नेन॑ । \newline
18. नि मृ॑ज्यान् मृज्या॒न् नि नि मृ॑ज्या॒द् वृष्टिं॒ ॅवृष्टि॑म् मृज्या॒न् नि नि मृ॑ज्या॒द् वृष्टि᳚म् । \newline
19. मृ॒ज्या॒द् वृष्टिं॒ ॅवृष्टि॑म् मृज्यान् मृज्या॒द् वृष्टि॑ मे॒वैव वृष्टि॑म् मृज्यान् मृज्या॒द् वृष्टि॑ मे॒व । \newline
20. वृष्टि॑ मे॒वैव वृष्टिं॒ ॅवृष्टि॑ मे॒वो दुदे॒व वृष्टिं॒ ॅवृष्टि॑ मे॒वोत् । \newline
21. ए॒वोदुदे॒ वैवोद् य॑च्छति यच्छ॒ त्युदे॒वैवोद् य॑च्छति । \newline
22. उद् य॑च्छति यच्छ॒ त्युदुद् य॑च्छति॒ यदि॒ यदि॑ यच्छ॒ त्युदुद् य॑च्छति॒ यदि॑ । \newline
23. य॒च्छ॒ति॒ यदि॒ यदि॑ यच्छति यच्छति॒ यद्य॑भि॒चरे॑ दभि॒चरे॒द् यदि॑ यच्छति यच्छति॒ यद्य॑भि॒चरे᳚त् । \newline
24. यद्य॑भि॒चरे॑ दभि॒चरे॒द् यदि॒ यद्य॑भि॒चरे॑ द॒मु म॒मु म॑भि॒चरे॒द् यदि॒ यद्य॑भि॒चरे॑ द॒मुम् । \newline
25. अ॒भि॒चरे॑ द॒मु म॒मु म॑भि॒चरे॑ दभि॒चरे॑ द॒मुम् ज॑हि जह्य॒मु म॑भि॒चरे॑ दभि॒चरे॑ द॒मुम् ज॑हि । \newline
26. अ॒भि॒चरे॒दित्य॑भि - चरे᳚त् । \newline
27. अ॒मुम् ज॑हि जह्य॒मु म॒मुम् ज॒ह्य थाथ॑ जह्य॒मु म॒मुम् ज॒ह्यथ॑ । \newline
28. ज॒ह्य थाथ॑ जहि ज॒ह्यथ॑ त्वा॒ त्वा ऽथ॑ जहि ज॒ह्यथ॑ त्वा । \newline
29. अथ॑ त्वा॒ त्वा ऽथाथ॑ त्वा होष्यामि होष्यामि॒ त्वा ऽथाथ॑ त्वा होष्यामि । \newline
30. त्वा॒ हो॒ष्या॒मि॒ हो॒ष्या॒मि॒ त्वा॒ त्वा॒ हो॒ष्या॒मी तीति॑ होष्यामि त्वा त्वा होष्या॒ मीति॑ । \newline
31. हो॒ष्या॒मी तीति॑ होष्यामि होष्या॒ मीति॑ ब्रूयाद् ब्रूया॒ दिति॑ होष्यामि होष्या॒ मीति॑ ब्रूयात् । \newline
32. इति॑ ब्रूयाद् ब्रूया॒ दितीति॑ ब्रूया॒ दाहु॑ति॒ माहु॑तिम् ब्रूया॒ दितीति॑ ब्रूया॒ दाहु॑तिम् । \newline
33. ब्रू॒या॒ दाहु॑ति॒ माहु॑तिम् ब्रूयाद् ब्रूया॒ दाहु॑ति मे॒वै वाहु॑तिम् ब्रूयाद् ब्रूया॒ दाहु॑ति मे॒व । \newline
34. आहु॑ति मे॒वै वाहु॑ति॒ माहु॑ति मे॒वैन॑ मेन मे॒वा हु॑ति॒ माहु॑ति मे॒वैन᳚म् । \newline
35. आहु॑ति॒मित्या - हु॒ति॒म् । \newline
36. ए॒वैन॑ मेन मे॒वै वैन॑म् प्रे॒फ्सन् प्रे॒फ्सन् ने॑न मे॒वै वैन॑म् प्रे॒फ्सन्न् । \newline
37. ए॒न॒म् प्रे॒फ्सन् प्रे॒फ्सन् ने॑न मेनम् प्रे॒फ्सन्. ह॑न्ति हन्ति प्रे॒फ्सन् ने॑न मेनम् प्रे॒फ्सन्. ह॑न्ति । \newline
38. प्रे॒फ्सन्. ह॑न्ति हन्ति प्रे॒फ्सन् प्रे॒फ्सन्. ह॑न्ति॒ यदि॒ यदि॑ हन्ति प्रे॒फ्सन् प्रे॒फ्सन्. ह॑न्ति॒ यदि॑ । \newline
39. प्रे॒फ्सन्निति॑ प्र - ई॒फ्सन्न् । \newline
40. ह॒न्ति॒ यदि॒ यदि॑ हन्ति हन्ति॒ यदि॑ दू॒रे दू॒रे यदि॑ हन्ति हन्ति॒ यदि॑ दू॒रे । \newline
41. यदि॑ दू॒रे दू॒रे यदि॒ यदि॑ दू॒रे स्याथ् स्याद् दू॒रे यदि॒ यदि॑ दू॒रे स्यात् । \newline
42. दू॒रे स्याथ् स्याद् दू॒रे दू॒रे स्यादा स्याद् दू॒रे दू॒रे स्यादा । \newline
43. स्यादा स्याथ् स्यादा तमि॑तो॒ स्तमि॑तो॒रा स्याथ् स्यादा तमि॑तोः । \newline
44. आ तमि॑तो॒ स्तमि॑तो॒रा तमि॑तो स्तिष्ठेत् तिष्ठे॒त् तमि॑तो॒रा तमि॑तो स्तिष्ठेत् । \newline
45. तमि॑तो स्तिष्ठेत् तिष्ठे॒त् तमि॑तो॒ स्तमि॑तो स्तिष्ठेत् प्रा॒णम् प्रा॒णम् ति॑ष्ठे॒त् तमि॑तो॒ स्तमि॑तो स्तिष्ठेत् प्रा॒णम् । \newline
46. ति॒ष्ठे॒त् प्रा॒णम् प्रा॒णम् ति॑ष्ठेत् तिष्ठेत् प्रा॒ण मे॒वैव प्रा॒णम् ति॑ष्ठेत् तिष्ठेत् प्रा॒ण मे॒व । \newline
47. प्रा॒ण मे॒वैव प्रा॒णम् प्रा॒ण मे॒वास्या᳚ स्यै॒व प्रा॒णम् प्रा॒ण मे॒वास्य॑ । \newline
48. प्रा॒णमिति॑ प्र - अ॒नम् । \newline
49. ए॒वास्या᳚ स्यै॒वै वास्या॑ नु॒गत्या॑ नु॒गत्या᳚ स्यै॒वै वास्या॑ नु॒गत्य॑ । \newline
50. अ॒स्या॒ नु॒गत्या॑ नु॒गत्या᳚ स्यास्या नु॒गत्य॑ हन्ति हन्त्य नु॒गत्या᳚ स्यास्या नु॒गत्य॑ हन्ति । \newline
51. अ॒नु॒गत्य॑ हन्ति हन्त्य नु॒गत्या॑ नु॒गत्य॑ हन्ति॒ यदि॒ यदि॑ हन्त्य नु॒गत्या॑ नु॒गत्य॑ हन्ति॒ यदि॑ । \newline
52. अ॒नु॒गत्येत्य॑नु - गत्य॑ । \newline
53. ह॒न्ति॒ यदि॒ यदि॑ हन्ति हन्ति॒ यद्य॑भि॒चरे॑ दभि॒चरे॒द् यदि॑ हन्ति हन्ति॒ यद्य॑भि॒चरे᳚त् । \newline
54. यद्य॑भि॒चरे॑ दभि॒चरे॒द् यदि॒ यद्य॑भि॒चरे॑ द॒मुष्या॒ मुष्या॑ भि॒चरे॒द् यदि॒ यद्य॑भि॒चरे॑ द॒मुष्य॑ । \newline
55. अ॒भि॒चरे॑ द॒मुष्या॒ मुष्या॑ भि॒चरे॑ दभि॒चरे॑ द॒मुष्य॑ त्वा त्वा॒ ऽमुष्या॑ भि॒चरे॑ दभि॒चरे॑ द॒मुष्य॑ त्वा । \newline
56. अ॒भि॒चरे॒दित्य॑भि - चरे᳚त् । \newline
57. अ॒मुष्य॑ त्वा त्वा॒ ऽमुष्या॒ मुष्य॑ त्वा प्रा॒णे प्रा॒णे त्वा॒ ऽमुष्या॒ मुष्य॑ त्वा प्रा॒णे । \newline
\pagebreak
\markright{ TS 6.4.5.7  \hfill https://www.vedavms.in \hfill}

\section{ TS 6.4.5.7 }

\textbf{TS 6.4.5.7 } \newline
\textbf{Samhita Paata} \newline

त्वा प्रा॒णे सा॑दया॒मीति॑ सादये॒दस॑न्नो॒ वै प्रा॒णः प्रा॒णमे॒वास्य॑ सादयति ष॒ड्भिरꣳ॒॒शुभिः॑ पवयति॒ षड् वा ऋ॒तव॑ ऋ॒तुभि॑रे॒वैनं॑ पवयति॒ त्रिः प॑वयति॒ त्रय॑ इ॒मे लो॒का ए॒भिरे॒वैनं॑ ॅलो॒कैः प॑वयति ब्रह्मवा॒दिनो॑ वदन्ति॒ कस्मा᳚थ् स॒त्यात् त्रयः॑ पशू॒नाꣳ हस्ता॑दाना॒ इति॒ यत् त्रिरु॑पाꣳ॒॒ शुꣳ हस्ते॑न विगृ॒ह्णाति॒ तस्मा॒त् त्रयः॑ पशू॒नाꣳ हस्ता॑दानाः॒ पुरु॑षो ( ) ह॒स्ती म॒र्कटः॑ ॥ \newline

\textbf{Pada Paata} \newline

त्वा॒ । प्रा॒ण इति॑ प्र - अ॒ने । सा॒द॒या॒मि॒ । इति॑ । सा॒द॒ये॒त् । अस॑न्नः । वै । प्रा॒ण इति॑ प्र - अ॒नः । प्रा॒णमिति॑ प्र - अ॒नम् । ए॒व । अ॒स्य॒ । सा॒द॒य॒ति॒ । ष॒ड्भिरिति॑ षट् - भिः । अꣳ॒॒शुभि॒रित्यꣳ॒॒शु - भिः॒ । प॒व॒य॒ति॒ । षट् । वै । ऋ॒तवः॑ । ऋ॒तुभि॒रित्यृ॒तु - भिः॒ । ए॒व । ए॒न॒म् । प॒व॒य॒ति॒ । त्रिः । प॒व॒य॒ति॒ । त्रयः॑ । इ॒मे । लो॒काः । ए॒भिः । ए॒व । ए॒न॒म् । लो॒कैः । प॒व॒य॒ति॒ । ब्र॒ह्म॒वा॒दिन॒ इति॑ ब्रह्म-वा॒दिनः॑ । व॒द॒न्ति॒ । कस्मा᳚त् । स॒त्यात् । त्रयः॑ । प॒शू॒नाम् । हस्ता॑दाना॒ इति॒ हस्त॑ -आ॒दा॒नाः॒ । इति॑ । यत् । त्रिः । उ॒पाꣳ॒॒शुमित्यु॑प - अꣳ॒॒शुम् । हस्ते॑न । वि॒गृ॒ह्णातीति॑ वि - गृ॒ह्णाति॑ । तस्मा᳚त् । त्रयः॑ । प॒शू॒नाम् । हस्ता॑दाना॒ इति॒ हस्त॑ - आ॒दा॒नाः॒ । पुरु॑षः ( ) । ह॒स्ती । म॒र्कटः॑ ॥  \newline


\textbf{Krama Paata} \newline

त्वा॒ प्रा॒णे । प्रा॒णे सा॑दयामि । प्रा॒ण इति॑ प्र - अ॒ने । सा॒द॒या॒मीति॑ । इति॑ सादयेत् । सा॒द॒ये॒दस॑न्नः । अस॑न्नो॒ वै । वै प्रा॒णः । प्रा॒णः प्रा॒णम् । प्रा॒ण इति॑ प्र - अ॒नः । प्रा॒णमे॒व । प्रा॒णमिति॑ प्र - अ॒नम् । ए॒वास्य॑ । अ॒स्य॒ सा॒द॒य॒ति॒ । सा॒द॒य॒ति॒ ष॒ड्भिः । ष॒ड्भिरꣳ॒॒शुभिः॑ । ष॒ड्भिरिति॑ षट् - भिः । अꣳ॒॒शुभिः॑ पवयति । अꣳ॒॒शुभि॒रित्यꣳ॒॒शु - भिः॒ । प॒व॒य॒ति॒ षट् । षड् वै । वा ऋ॒तवः॑ । ऋ॒तव॑ ऋ॒तुभिः॑ । ऋ॒तुभि॑रे॒व । ऋ॒तुभि॒रित्यृ॒तु - भिः॒ । ए॒वैन᳚म् । ए॒न॒म् प॒व॒य॒ति॒ । प॒व॒य॒ति॒ त्रिः । त्रिः प॑वयति । प॒व॒य॒ति॒ त्रयः॑ । त्रय॑ इ॒मे । इ॒मे लो॒काः । लो॒का ए॒भिः । ए॒भिरे॒व । ए॒वैन᳚म् । ए॒न॒म् ॅलो॒कैः । लो॒कैः प॑वयति । प॒व॒य॒ति॒ ब्र॒ह्म॒वा॒दिनः॑ । ब्र॒ह्म॒वा॒दिनो॑ वदन्ति । ब्र॒ह्म॒वा॒दिन॒ इति॑ ब्रह्म - वा॒दिनः॑ । व॒द॒न्ति॒ कस्मा᳚त् । कस्मा᳚थ् स॒त्यात् । स॒त्यात् त्रयः॑ । त्रयः॑ पशू॒नाम् । प॒शू॒नाꣳ हस्ता॑दानाः । हस्ता॑दाना॒ इति॑ । हस्ता॑दाना॒ इति॒ हस्त॑ - आ॒दा॒नाः॒ । इति॒ यत् । यत् त्रिः । त्रिरु॑पाꣳ॒॒शुम् । उ॒पाꣳ॒॒शुꣳ हस्ते॑न । उ॒पाꣳ॒॒शुमित्यु॑प - अꣳ॒॒शुम् । हस्ते॑न विगृ॒ह्णाति॑ । वि॒गृ॒ह्णाति॒ तस्मा᳚त् । वि॒गृ॒ह्णातीति॑ वि - गृ॒ह्णाति॑ । तस्मा॒त् त्रयः॑ । त्रयः॑ पशू॒नाम् । प॒शू॒नाꣳ हस्ता॑दानाः । हस्ता॑दानाः॒ पुरु॑षः ( ) । हस्ता॑दाना॒ इति॒ हस्त॑ - आ॒दा॒नाः॒ । पुरु॑षो ह॒स्ती । ह॒स्ती म॒र्कटः॑ । म॒र्कट॒ इति॑ म॒र्कटः॑ । \newline

\textbf{Jatai Paata} \newline

1. त्वा॒ प्रा॒णे प्रा॒णे त्वा᳚ त्वा प्रा॒णे । \newline
2. प्रा॒णे सा॑दयामि सादयामि प्रा॒णे प्रा॒णे सा॑दयामि । \newline
3. प्रा॒ण इति॑ प्र - अ॒ने । \newline
4. सा॒द॒या॒मी तीति॑ सादयामि सादया॒ मीति॑ । \newline
5. इति॑ सादयेथ् सादये॒ दितीति॑ सादयेत् । \newline
6. सा॒द॒ये॒ दस॒न्नो ऽस॑न्नः सादयेथ् सादये॒ दस॑न्नः । \newline
7. अस॑न्नो॒ वै वा अस॒न्नो ऽस॑न्नो॒ वै । \newline
8. वै प्रा॒णः प्रा॒णो वै वै प्रा॒णः । \newline
9. प्रा॒णः प्रा॒णम् प्रा॒णम् प्रा॒णः प्रा॒णः प्रा॒णम् । \newline
10. प्रा॒ण इति॑ प्र - अ॒नः । \newline
11. प्रा॒ण मे॒वैव प्रा॒णम् प्रा॒ण मे॒व । \newline
12. प्रा॒णमिति॑ प्र - अ॒नम् । \newline
13. ए॒वास्या᳚ स्यै॒वै वास्य॑ । \newline
14. अ॒स्य॒ सा॒द॒य॒ति॒ सा॒द॒य॒ त्य॒स्या॒स्य॒ सा॒द॒य॒ति॒ । \newline
15. सा॒द॒य॒ति॒ ष॒ड्भि ष्ष॒ड्भिः सा॑दयति सादयति ष॒ड्भिः । \newline
16. ष॒ड्भि रꣳ॒॒शुभि॑ रꣳ॒॒शुभि॑ ष्ष॒ड्भि ष्ष॒ड्भि रꣳ॒॒शुभिः॑ । \newline
17. ष॒ड्भिरिति॑ षट् - भिः । \newline
18. अꣳ॒॒शुभिः॑ पवयति पवय त्यꣳ॒॒शुभि॑ रꣳ॒॒शुभिः॑ पवयति । \newline
19. अꣳ॒॒शुभि॒रित्यꣳ॒॒शु - भिः॒ । \newline
20. प॒व॒य॒ति॒ षट् थ्षट् प॑वयति पवयति॒ षट् । \newline
21. षड् वै वै षट् थ्षड् वै । \newline
22. वा ऋ॒तव॑ ऋ॒तवो॒ वै वा ऋ॒तवः॑ । \newline
23. ऋ॒तव॑ ऋ॒तुभिर्॑. ऋ॒तुभिर्॑. ऋ॒तव॑ ऋ॒तव॑ ऋ॒तुभिः॑ । \newline
24. ऋ॒तुभि॑ रे॒वैव र्‌तुभिर्॑. ऋ॒तुभि॑ रे॒व । \newline
25. ऋ॒तुभि॒रित्यृ॒तु - भिः॒ । \newline
26. ए॒वैन॑ मेन मे॒वै वैन᳚म् । \newline
27. ए॒न॒म् प॒व॒य॒ति॒ प॒व॒य॒ त्ये॒न॒ मे॒न॒म् प॒व॒य॒ति॒ । \newline
28. प॒व॒य॒ति॒ त्रि स्त्रिः प॑वयति पवयति॒ त्रिः । \newline
29. त्रिः प॑वयति पवयति॒ त्रि स्त्रिः प॑वयति । \newline
30. प॒व॒य॒ति॒ त्रय॒ स्त्रयः॑ पवयति पवयति॒ त्रयः॑ । \newline
31. त्रय॑ इ॒म इ॒मे त्रय॒ स्त्रय॑ इ॒मे । \newline
32. इ॒मे लो॒का लो॒का इ॒म इ॒मे लो॒काः । \newline
33. लो॒का ए॒भि रे॒भिर् लो॒का लो॒का ए॒भिः । \newline
34. ए॒भि रे॒वै वैभि रे॒भि रे॒व । \newline
35. ए॒वैन॑ मेन मे॒वै वैन᳚म् । \newline
36. ए॒न॒म् ॅलो॒कैर् लो॒कै रे॑न मेनम् ॅलो॒कैः । \newline
37. लो॒कैः प॑वयति पवयति लो॒कैर् लो॒कैः प॑वयति । \newline
38. प॒व॒य॒ति॒ ब्र॒ह्म॒वा॒दिनो᳚ ब्रह्मवा॒दिनः॑ पवयति पवयति ब्रह्मवा॒दिनः॑ । \newline
39. ब्र॒ह्म॒वा॒दिनो॑ वदन्ति वदन्ति ब्रह्मवा॒दिनो᳚ ब्रह्मवा॒दिनो॑ वदन्ति । \newline
40. ब्र॒ह्म॒वा॒दिन॒ इति॑ ब्रह्म - वा॒दिनः॑ । \newline
41. व॒द॒न्ति॒ कस्मा॒त् कस्मा᳚द् वदन्ति वदन्ति॒ कस्मा᳚त् । \newline
42. कस्मा᳚थ् स॒त्याथ् स॒त्यात् कस्मा॒त् कस्मा᳚थ् स॒त्यात् । \newline
43. स॒त्यात् त्रय॒ स्त्रयः॑ स॒त्याथ् स॒त्यात् त्रयः॑ । \newline
44. त्रयः॑ पशू॒नाम् प॑शू॒नाम् त्रय॒ स्त्रयः॑ पशू॒नाम् । \newline
45. प॒शू॒नाꣳ हस्ता॑दाना॒ हस्ता॑दानाः पशू॒नाम् प॑शू॒नाꣳ हस्ता॑दानाः । \newline
46. हस्ता॑दाना॒ इतीति॒ हस्ता॑दाना॒ हस्ता॑दाना॒ इति॑ । \newline
47. हस्ता॑दाना॒ इति॒ हस्त॑ - आ॒दा॒नाः॒ । \newline
48. इति॒ यद् यदि तीति॒ यत् । \newline
49. यत् त्रि स्त्रिर् यद् यत् त्रिः । \newline
50. त्रि रु॑पाꣳ॒॒शु मु॑पाꣳ॒॒शुम् त्रि स्त्रि रु॑पाꣳ॒॒शुम् । \newline
51. उ॒पाꣳ॒॒शुꣳ हस्ते॑न॒ हस्ते॑नोपाꣳ॒॒शु मु॑पाꣳ॒॒शुꣳ हस्ते॑न । \newline
52. उ॒पाꣳ॒॒शुमित्यु॑प - अꣳ॒॒शुम् । \newline
53. हस्ते॑न विगृ॒ह्णाति॑ विगृ॒ह्णाति॒ हस्ते॑न॒ हस्ते॑न विगृ॒ह्णाति॑ । \newline
54. वि॒गृ॒ह्णाति॒ तस्मा॒त् तस्मा᳚द् विगृ॒ह्णाति॑ विगृ॒ह्णाति॒ तस्मा᳚त् । \newline
55. वि॒गृ॒ह्णातीति॑ वि - गृ॒ह्णाति॑ । \newline
56. तस्मा॒त् त्रय॒ स्त्रय॒ स्तस्मा॒त् तस्मा॒त् त्रयः॑ । \newline
57. त्रयः॑ पशू॒नाम् प॑शू॒नाम् त्रय॒ स्त्रयः॑ पशू॒नाम् । \newline
58. प॒शू॒नाꣳ हस्ता॑दाना॒ हस्ता॑दानाः पशू॒नाम् प॑शू॒नाꣳ हस्ता॑दानाः । \newline
59. हस्ता॑दानाः॒ पुरु॑षः॒ पुरु॑षो॒ हस्ता॑दाना॒ हस्ता॑दानाः॒ पुरु॑षः । \newline
60. हस्ता॑दाना॒ इति॒ हस्त॑ - आ॒दा॒नाः॒ । \newline
61. पुरु॑षो ह॒स्ती ह॒स्ती पुरु॑षः॒ पुरु॑षो ह॒स्ती । \newline
62. ह॒स्ती म॒र्कटो॑ म॒र्कटो॑ ह॒स्ती ह॒स्ती म॒र्कटः॑ । \newline
63. म॒र्कट॒ इति॑ म॒र्कटः॑ । \newline

\textbf{Ghana Paata } \newline

1. त्वा॒ प्रा॒णे प्रा॒णे त्वा᳚ त्वा प्रा॒णे सा॑दयामि सादयामि प्रा॒णे त्वा᳚ त्वा प्रा॒णे सा॑दयामि । \newline
2. प्रा॒णे सा॑दयामि सादयामि प्रा॒णे प्रा॒णे सा॑दया॒मी तीति॑ सादयामि प्रा॒णे प्रा॒णे सा॑दया॒ मीति॑ । \newline
3. प्रा॒ण इति॑ प्र - अ॒ने । \newline
4. सा॒द॒या॒मी तीति॑ सादयामि सादया॒ मीति॑ सादयेथ् सादये॒ दिति॑ सादयामि सादया॒ मीति॑ सादयेत् । \newline
5. इति॑ सादयेथ् सादये॒दि तीति॑ सादये॒ दस॒न्नो ऽस॑न्नः सादये॒ दितीति॑ सादये॒ दस॑न्नः । \newline
6. सा॒द॒ये॒ दस॒न्नो ऽस॑न्नः सादयेथ् सादये॒ दस॑न्नो॒ वै वा अस॑न्नः सादयेथ् सादये॒ दस॑न्नो॒ वै । \newline
7. अस॑न्नो॒ वै वा अस॒न्नो ऽस॑न्नो॒ वै प्रा॒णः प्रा॒णो वा अस॒न्नो ऽस॑न्नो॒ वै प्रा॒णः । \newline
8. वै प्रा॒णः प्रा॒णो वै वै प्रा॒णः प्रा॒णम् प्रा॒णम् प्रा॒णो वै वै प्रा॒णः प्रा॒णम् । \newline
9. प्रा॒णः प्रा॒णम् प्रा॒णम् प्रा॒णः प्रा॒णः प्रा॒ण मे॒वैव प्रा॒णम् प्रा॒णः प्रा॒णः प्रा॒ण मे॒व । \newline
10. प्रा॒ण इति॑ प्र - अ॒नः । \newline
11. प्रा॒ण मे॒वैव प्रा॒णम् प्रा॒ण मे॒वास्या᳚ स्यै॒व प्रा॒णम् प्रा॒ण मे॒वास्य॑ । \newline
12. प्रा॒णमिति॑ प्र - अ॒नम् । \newline
13. ए॒वास्या᳚ स्यै॒वै वास्य॑ सादयति सादय त्यस्यै॒वै वास्य॑ सादयति । \newline
14. अ॒स्य॒ सा॒द॒य॒ति॒ सा॒द॒य॒ त्य॒स्या॒स्य॒ सा॒द॒य॒ति॒ ष॒ड्भि ष्ष॒ड्भिः सा॑दय त्यस्यास्य सादयति ष॒ड्भिः । \newline
15. सा॒द॒य॒ति॒ ष॒ड्भि ष्ष॒ड्भिः सा॑दयति सादयति ष॒ड्भि रꣳ॒॒शुभि॑ रꣳ॒॒शुभि॑ ष्ष॒ड्भिः सा॑दयति सादयति ष॒ड्भि रꣳ॒॒शुभिः॑ । \newline
16. ष॒ड्भि रꣳ॒॒शुभि॑ रꣳ॒॒शुभि॑ ष्ष॒ड्भि ष्ष॒ड्भि रꣳ॒॒शुभिः॑ पवयति पवय त्यꣳ॒॒शुभि॑ ष्ष॒ड्भि ष्ष॒ड्भि रꣳ॒॒शुभिः॑ पवयति । \newline
17. ष॒ड्भिरिति॑ षट् - भिः । \newline
18. अꣳ॒॒शुभिः॑ पवयति पवय त्यꣳ॒॒शुभि॑ रꣳ॒॒शुभिः॑ पवयति॒ षट् थ्षट् प॑वय त्यꣳ॒॒शुभि॑ रꣳ॒॒शुभिः॑ पवयति॒ षट् । \newline
19. अꣳ॒॒शुभि॒रित्यꣳ॒॒शु - भिः॒ । \newline
20. प॒व॒य॒ति॒ षट् थ्षट् प॑वयति पवयति॒ षड् वै वै षट् प॑वयति पवयति॒ षड् वै । \newline
21. षड् वै वै षट् थ्षड् वा ऋ॒तव॑ ऋ॒तवो॒ वै षट् थ्षड् वा ऋ॒तवः॑ । \newline
22. वा ऋ॒तव॑ ऋ॒तवो॒ वै वा ऋ॒तव॑ ऋ॒तुभिर्॑. ऋ॒तुभिर्॑. ऋ॒तवो॒ वै वा ऋ॒तव॑ ऋ॒तुभिः॑ । \newline
23. ऋ॒तव॑ ऋ॒तुभिर्॑. ऋ॒तुभिर्॑. ऋ॒तव॑ ऋ॒तव॑ ऋ॒तुभि॑ रे॒वैव र्‌तुभिर्॑. ऋ॒तव॑ ऋ॒तव॑ ऋ॒तुभि॑ रे॒व । \newline
24. ऋ॒तुभि॑ रे॒वैव र्‌तुभिर्॑. ऋ॒तुभि॑ रे॒वैन॑ मेन मे॒व र्‌तुभिर्॑. ऋ॒तुभि॑ रे॒वैन᳚म् । \newline
25. ऋ॒तुभि॒रित्यृ॒तु - भिः॒ । \newline
26. ए॒वैन॑ मेन मे॒वै वैन॑म् पवयति पवय त्येन मे॒वै वैन॑म् पवयति । \newline
27. ए॒न॒म् प॒व॒य॒ति॒ प॒व॒य॒ त्ये॒न॒ मे॒न॒म् प॒व॒य॒ति॒ त्रि स्त्रिः प॑वय त्येन मेनम् पवयति॒ त्रिः । \newline
28. प॒व॒य॒ति॒ त्रि स्त्रिः प॑वयति पवयति॒ त्रिः प॑वयति पवयति॒ त्रिः प॑वयति पवयति॒ त्रिः प॑वयति । \newline
29. त्रिः प॑वयति पवयति॒ त्रि स्त्रिः प॑वयति॒ त्रय॒ स्त्रयः॑ पवयति॒ त्रि स्त्रिः प॑वयति॒ त्रयः॑ । \newline
30. प॒व॒य॒ति॒ त्रय॒ स्त्रयः॑ पवयति पवयति॒ त्रय॑ इ॒म इ॒मे त्रयः॑ पवयति पवयति॒ त्रय॑ इ॒मे । \newline
31. त्रय॑ इ॒म इ॒मे त्रय॒ स्त्रय॑ इ॒मे लो॒का लो॒का इ॒मे त्रय॒ स्त्रय॑ इ॒मे लो॒काः । \newline
32. इ॒मे लो॒का लो॒का इ॒म इ॒मे लो॒का ए॒भि रे॒भिर् लो॒का इ॒म इ॒मे लो॒का ए॒भिः । \newline
33. लो॒का ए॒भि रे॒भिर् लो॒का लो॒का ए॒भि रे॒वै वैभिर् लो॒का लो॒का ए॒भि रे॒व । \newline
34. ए॒भि रे॒वैवैभि रे॒भि रे॒वैन॑ मेन मे॒वैभि रे॒भि रे॒वैन᳚म् । \newline
35. ए॒वैन॑ मेन मे॒वै वैन॑म् ॅलो॒कैर् लो॒कै रे॑न मे॒वै वैन॑म् ॅलो॒कैः । \newline
36. ए॒न॒म् ॅलो॒कैर् लो॒कै रे॑न मेनम् ॅलो॒कैः प॑वयति पवयति लो॒कै रे॑न मेनम् ॅलो॒कैः प॑वयति । \newline
37. लो॒कैः प॑वयति पवयति लो॒कैर् लो॒कैः प॑वयति ब्रह्मवा॒दिनो᳚ ब्रह्मवा॒दिनः॑ पवयति लो॒कैर् लो॒कैः प॑वयति ब्रह्मवा॒दिनः॑ । \newline
38. प॒व॒य॒ति॒ ब्र॒ह्म॒वा॒दिनो᳚ ब्रह्मवा॒दिनः॑ पवयति पवयति ब्रह्मवा॒दिनो॑ वदन्ति वदन्ति ब्रह्मवा॒दिनः॑ पवयति पवयति ब्रह्मवा॒दिनो॑ वदन्ति । \newline
39. ब्र॒ह्म॒वा॒दिनो॑ वदन्ति वदन्ति ब्रह्मवा॒दिनो᳚ ब्रह्मवा॒दिनो॑ वदन्ति॒ कस्मा॒त् कस्मा᳚द् वदन्ति ब्रह्मवा॒दिनो᳚ ब्रह्मवा॒दिनो॑ वदन्ति॒ कस्मा᳚त् । \newline
40. ब्र॒ह्म॒वा॒दिन॒ इति॑ ब्रह्म - वा॒दिनः॑ । \newline
41. व॒द॒न्ति॒ कस्मा॒त् कस्मा᳚द् वदन्ति वदन्ति॒ कस्मा᳚थ् स॒त्याथ् स॒त्यात् कस्मा᳚द् वदन्ति वदन्ति॒ कस्मा᳚थ् स॒त्यात् । \newline
42. कस्मा᳚थ् स॒त्याथ् स॒त्यात् कस्मा॒त् कस्मा᳚थ् स॒त्यात् त्रय॒ स्त्रयः॑ स॒त्यात् कस्मा॒त् कस्मा᳚थ् स॒त्यात् त्रयः॑ । \newline
43. स॒त्यात् त्रय॒ स्त्रयः॑ स॒त्याथ् स॒त्यात् त्रयः॑ पशू॒नाम् प॑शू॒नाम् त्रयः॑ स॒त्याथ् स॒त्यात् त्रयः॑ पशू॒नाम् । \newline
44. त्रयः॑ पशू॒नाम् प॑शू॒नाम् त्रय॒ स्त्रयः॑ पशू॒नाꣳ हस्ता॑दाना॒ हस्ता॑दानाः पशू॒नाम् त्रय॒ स्त्रयः॑ पशू॒नाꣳ हस्ता॑दानाः । \newline
45. प॒शू॒नाꣳ हस्ता॑दाना॒ हस्ता॑दानाः पशू॒नाम् प॑शू॒नाꣳ हस्ता॑दाना॒ इतीति॒ हस्ता॑दानाः पशू॒नाम् प॑शू॒नाꣳ हस्ता॑दाना॒ इति॑ । \newline
46. हस्ता॑दाना॒ इतीति॒ हस्ता॑दाना॒ हस्ता॑दाना॒ इति॒ यद् यदिति॒ हस्ता॑दाना॒ हस्ता॑दाना॒ इति॒ यत् । \newline
47. हस्ता॑दाना॒ इति॒ हस्त॑ - आ॒दा॒नाः॒ । \newline
48. इति॒ यद् यदि तीति॒ यत् त्रि स्त्रिर् यदितीति॒ यत् त्रिः । \newline
49. यत् त्रि स्त्रिर् यद् यत् त्रि रु॑पाꣳ॒॒शु मु॑पाꣳ॒॒शुम् त्रिर् यद् यत् त्रि रु॑पाꣳ॒॒शुम् । \newline
50. त्रि रु॑पाꣳ॒॒शु मु॑पाꣳ॒॒शुम् त्रि स्त्रि रु॑पाꣳ॒॒शुꣳ हस्ते॑न॒ हस्ते॑ नोपाꣳ॒॒शुम् त्रि स्त्रि रु॑पाꣳ॒॒शुꣳ हस्ते॑न । \newline
51. उ॒पाꣳ॒॒शुꣳ हस्ते॑न॒ हस्ते॑ नोपाꣳ॒॒शु मु॑पाꣳ॒॒शुꣳ हस्ते॑न विगृ॒ह्णाति॑ विगृ॒ह्णाति॒ हस्ते॑
नोपाꣳ॒॒शु मु॑पाꣳ॒॒शुꣳ हस्ते॑न विगृ॒ह्णाति॑ । \newline
52. उ॒पाꣳ॒॒शुमित्यु॑प - अꣳ॒॒शुम् । \newline
53. हस्ते॑न विगृ॒ह्णाति॑ विगृ॒ह्णाति॒ हस्ते॑न॒ हस्ते॑न विगृ॒ह्णाति॒ तस्मा॒त् तस्मा᳚द् विगृ॒ह्णाति॒ हस्ते॑न॒ हस्ते॑न विगृ॒ह्णाति॒ तस्मा᳚त् । \newline
54. वि॒गृ॒ह्णाति॒ तस्मा॒त् तस्मा᳚द् विगृ॒ह्णाति॑ विगृ॒ह्णाति॒ तस्मा॒त् त्रय॒ स्त्रय॒ स्तस्मा᳚द् विगृ॒ह्णाति॑ विगृ॒ह्णाति॒ तस्मा॒त् त्रयः॑ । \newline
55. वि॒गृ॒ह्णातीति॑ वि - गृ॒ह्णाति॑ । \newline
56. तस्मा॒त् त्रय॒ स्त्रय॒ स्तस्मा॒त् तस्मा॒त् त्रयः॑ पशू॒नाम् प॑शू॒नाम् त्रय॒ स्तस्मा॒त् तस्मा॒त् त्रयः॑ पशू॒नाम् । \newline
57. त्रयः॑ पशू॒नाम् प॑शू॒नाम् त्रय॒ स्त्रयः॑ पशू॒नाꣳ हस्ता॑दाना॒ हस्ता॑दानाः पशू॒नाम् त्रय॒ स्त्रयः॑ पशू॒नाꣳ हस्ता॑दानाः । \newline
58. प॒शू॒नाꣳ हस्ता॑दाना॒ हस्ता॑दानाः पशू॒नाम् प॑शू॒नाꣳ हस्ता॑दानाः॒ पुरु॑षः॒ पुरु॑षो॒ हस्ता॑दानाः पशू॒नाम् प॑शू॒नाꣳ हस्ता॑दानाः॒ पुरु॑षः । \newline
59. हस्ता॑दानाः॒ पुरु॑षः॒ पुरु॑षो॒ हस्ता॑दाना॒ हस्ता॑दानाः॒ पुरु॑षो ह॒स्ती ह॒स्ती पुरु॑षो॒ हस्ता॑दाना॒ हस्ता॑दानाः॒ पुरु॑षो ह॒स्ती । \newline
60. हस्ता॑दाना॒ इति॒ हस्त॑ - आ॒दा॒नाः॒ । \newline
61. पुरु॑षो ह॒स्ती ह॒स्ती पुरु॑षः॒ पुरु॑षो ह॒स्ती म॒र्कटो॑ म॒र्कटो॑ ह॒स्ती पुरु॑षः॒ पुरु॑षो ह॒स्ती म॒र्कटः॑ । \newline
62. ह॒स्ती म॒र्कटो॑ म॒र्कटो॑ ह॒स्ती ह॒स्ती म॒र्कटः॑ । \newline
63. म॒र्कट॒ इति॑ म॒र्कटः॑ । \newline
\pagebreak
\markright{ TS 6.4.6.1  \hfill https://www.vedavms.in \hfill}

\section{ TS 6.4.6.1 }

\textbf{TS 6.4.6.1 } \newline
\textbf{Samhita Paata} \newline

दे॒वा वै यद् य॒ज्ञेऽकु॑र्वत॒ तदसु॑रा अकुर्वत॒ ते दे॒वा उ॑पाꣳ॒॒शौ य॒ज्ञ्ꣳ सꣳ॒॒स्थाप्य॑मपश्य॒न् तमु॑पाꣳ॒॒शौ सम॑स्थापय॒न् तेऽसु॑रा॒ वज्र॑मु॒द्यत्य॑ दे॒वान॒भ्या॑यन्त॒ ते दे॒वा बिभ्य॑त॒ इन्द्र॒मुपा॑धाव॒न् तानिन्द्रो᳚ऽन्तर्या॒मेणा॒न्तर॑धत्त॒ तद॑न्तर्या॒मस्या᳚न्तर्याम॒त्वं ॅयद॑न्तर्या॒मो गृ॒ह्यते॒ भ्रातृ॑व्याने॒व तद् यज॑मानो॒ऽन्तर्द्ध॑त्ते॒ ऽन्तस्ते॑- [  ] \newline

\textbf{Pada Paata} \newline

दे॒वाः । वै । यत् । य॒ज्ञे । अकु॑र्वत । तत् । असु॑राः । अ॒कु॒र्व॒त॒ । ते । दे॒वाः । उ॒पाꣳ॒॒शावित्यु॑प - अꣳ॒॒शौ । य॒ज्ञ्म् । सꣳ॒॒स्थाप्य॒मिति॑ सं - स्थाप्य᳚म् । अ॒प॒श्य॒न्न् । तम् । उ॒पाꣳ॒॒शावित्यु॑प - अꣳ॒॒शौ । समिति॑ । अ॒स्था॒प॒य॒न्न् । ते । असु॑राः । वज्र᳚म् । उ॒द्यत्येत्यु॑त् - यत्य॑ । दे॒वान् । अ॒भीति॑ । आ॒य॒न्त॒ । ते । दे॒वाः । बिभ्य॑तः । इन्द्र᳚म् । उपेति॑ । अ॒धा॒व॒न्न् । तान् । इन्द्रः॑ । अ॒न्त॒र्या॒मेणेत्य॑न्तः - या॒मेन॑ । अ॒न्तः । अ॒ध॒त्त॒ । तत् । अ॒न्त॒र्या॒मस्येत्य॑न्तः - या॒मस्य॑ । अ॒न्त॒र्या॒म॒त्वमित्य॑न्तर्याम - त्वम् । यत् । अ॒न्त॒र्या॒म इत्य॑न्तः-या॒मः । गृ॒ह्यते᳚ । भ्रातृ॑व्यान् । ए॒व । तत् । यज॑मानः । अ॒न्तः । ध॒त्ते॒ । अ॒न्तः । ते॒ ।  \newline


\textbf{Krama Paata} \newline

दे॒वा वै । वै यत् । यद् य॒ज्ञे । य॒ज्ञेऽकु॑र्वत । अकु॑र्वत॒ तत् । तदसु॑राः । असु॑रा अकुर्वत । अ॒कु॒र्व॒त॒ ते । ते दे॒वाः । दे॒वा उ॑पाꣳ॒॒शौ । उ॒पाꣳ॒॒शौ य॒ज्ञ्म् । उ॒पाꣳ॒॒शावित्यु॑प - अꣳ॒॒शौ । य॒ज्ञ्ꣳ सꣳ॒॒स्थाप्य᳚म् । सꣳ॒॒स्थाप्य॑मपश्यन्न् । सꣳ॒॒स्थाप्य॒मिति॑ सम् - स्थाप्य᳚म् । अ॒प॒श्य॒न् तम् । तमु॑पाꣳ॒॒शौ । उ॒पाꣳ॒॒शौ सम् । उ॒पाꣳ॒॒शावित्यु॑प - अꣳ॒॒शौ । सम॑स्थापयन्न् । अ॒स्था॒प॒य॒न् ते । तेऽसु॑राः । असु॑रा॒ वज्र᳚म् । वज्र॑मु॒द्यत्य॑ । उ॒द्यत्य॑ दे॒वान् । उ॒द्यत्येत्यु॑त् - यत्य॑ । दे॒वान॒भि । अ॒भ्या॑यन्त । आ॒य॒न्त॒ ते । ते दे॒वाः । दे॒वा बिभ्य॑तः । बिभ्य॑त॒ इन्द्र᳚म् । इन्द्र॒मुप॑ । उपा॑धावन्न् । अ॒धा॒व॒न् तान् । तानिन्द्रः॑ । इन्द्रो᳚ऽन्तर्या॒मेण॑ । अ॒न्त॒र्या॒मेणा॒न्तः । अ॒न्त॒र्या॒मेणेत्य॑न्तः - या॒मेन॑ । अ॒न्तर॑धत्त । अ॒ध॒त्त॒ तत् । तद॑न्तर्या॒मस्य॑ । अ॒न्त॒र्या॒मस्या᳚न्तर्याम॒त्वम् । अ॒न्त॒र्या॒मस्येत्य॑न्तः - या॒मस्य॑ । अ॒न्त॒र्या॒म॒त्वम् ॅयत् । अ॒न्त॒र्या॒म॒त्वमित्य॑न्तर्याम - त्वम् । यद॑न्तर्या॒मः । अ॒न्त॒र्या॒मो गृ॒ह्यते᳚ । अ॒न्त॒र्या॒म इत्य॑न्तः - या॒मः । गृ॒ह्यते॒ भ्रातृ॑व्यान् । भ्रातृ॑व्याने॒व । ए॒व तत् । तद् यज॑मानः । यज॑मानो॒ऽन्तः । अ॒न्तर् ध॑त्ते । ध॒त्ते॒ऽन्तः । अ॒न्तस्ते᳚ । ते॒ द॒धा॒मि॒ \newline

\textbf{Jatai Paata} \newline

1. दे॒वा वै वै दे॒वा दे॒वा वै । \newline
2. वै यद् यद् वै वै यत् । \newline
3. यद् य॒ज्ञे य॒ज्ञे यद् यद् य॒ज्ञे । \newline
4. य॒ज्ञे ऽकु॑र्व॒ता कु॑र्वत य॒ज्ञे य॒ज्ञे ऽकु॑र्वत । \newline
5. अकु॑र्वत॒ तत् तदकु॑र्व॒ता कु॑र्वत॒ तत् । \newline
6. तदसु॑रा॒ असु॑रा॒ स्तत् तदसु॑राः । \newline
7. असु॑रा अकुर्वता कुर्व॒ता सु॑रा॒ असु॑रा अकुर्वत । \newline
8. अ॒कु॒र्व॒त॒ ते ते॑ ऽकुर्वता कुर्वत॒ ते । \newline
9. ते दे॒वा दे॒वा स्ते ते दे॒वाः । \newline
10. दे॒वा उ॑पाꣳ॒॒शा वु॑पाꣳ॒॒शौ दे॒वा दे॒वा उ॑पाꣳ॒॒शौ । \newline
11. उ॒पाꣳ॒॒शौ य॒ज्ञ्ं ॅय॒ज्ञ् मु॑पाꣳ॒॒शा वु॑पाꣳ॒॒शौ य॒ज्ञ्म् । \newline
12. उ॒पाꣳ॒॒शावित्यु॑प - अꣳ॒॒शौ । \newline
13. य॒ज्ञ्ꣳ सꣳ॒॒स्थाप्यꣳ॑ सꣳ॒॒स्थाप्यं॑ ॅय॒ज्ञ्ं ॅय॒ज्ञ्ꣳ सꣳ॒॒स्थाप्य᳚म् । \newline
14. सꣳ॒॒स्थाप्य॑ मपश्यन् नपश्यन् थ्सꣳ॒॒स्थाप्यꣳ॑ सꣳ॒॒स्थाप्य॑ मपश्यन्न् । \newline
15. सꣳ॒॒स्थाप्य॒मिति॑ सं - स्थाप्य᳚म् । \newline
16. अ॒प॒श्य॒न् तम् त म॑पश्यन् नपश्य॒न् तम् । \newline
17. त मु॑पाꣳ॒॒शा वु॑पाꣳ॒॒शौ तम् त मु॑पाꣳ॒॒शौ । \newline
18. उ॒पाꣳ॒॒शौ सꣳ स मु॑पाꣳ॒॒शा वु॑पाꣳ॒॒शौ सम् । \newline
19. उ॒पाꣳ॒॒शावित्यु॑प - अꣳ॒॒शौ । \newline
20. स म॑स्थापयन् नस्थापय॒न् थ्सꣳ स म॑स्थापयन्न् । \newline
21. अ॒स्था॒प॒य॒न् ते ते᳚ ऽस्थापयन् नस्थापय॒न् ते । \newline
22. ते ऽसु॑रा॒ असु॑रा॒ स्ते ते ऽसु॑राः । \newline
23. असु॑रा॒ वज्रं॒ ॅवज्र॒ मसु॑रा॒ असु॑रा॒ वज्र᳚म् । \newline
24. वज्र॑ मु॒द्यत्यो॒ द्यत्य॒ वज्रं॒ ॅवज्र॑ मु॒द्यत्य॑ । \newline
25. उ॒द्यत्य॑ दे॒वान् दे॒वा नु॒द्य त्यो॒द्यत्य॑ दे॒वान् । \newline
26. उ॒द्यत्येत्यु॑त् - यत्य॑ । \newline
27. दे॒वान॒ भ्य॑भि दे॒वान् दे॒वान॒भि । \newline
28. अ॒भ्या॑ यन्ता यन्ता॒ भ्या᳚(1॒)भ्या॑ यन्त । \newline
29. आ॒य॒न्त॒ ते त आ॑यन्ता यन्त॒ ते । \newline
30. ते दे॒वा दे॒वा स्ते ते दे॒वाः । \newline
31. दे॒वा बिभ्य॑तो॒ बिभ्य॑तो दे॒वा दे॒वा बिभ्य॑तः । \newline
32. बिभ्य॑त॒ इन्द्र॒ मिन्द्र॒म् बिभ्य॑तो॒ बिभ्य॑त॒ इन्द्र᳚म् । \newline
33. इन्द्र॒ मुपोपेन्द्र॒ मिन्द्र॒ मुप॑ । \newline
34. उपा॑धावन् नधाव॒न् नुपोपा॑ धावन्न् । \newline
35. अ॒धा॒व॒न् ताꣳ स्तान॑ धावन् नधाव॒न् तान् । \newline
36. तानिन्द्र॒ इन्द्र॒ स्ताꣳ स्तानिन्द्रः॑ । \newline
37. इन्द्रो᳚ ऽन्तर्या॒मेणा᳚ न्तर्या॒मेणेन्द्र॒ इन्द्रो᳚ ऽन्तर्या॒मेण॑ । \newline
38. अ॒न्त॒र्या॒मेणा॒न्त र॒न्त र॑न्तर्या॒मेणा᳚ न्तर्या॒मेणा॒न्तः । \newline
39. अ॒न्त॒र्या॒मेणेत्य॑न्तः - या॒मेन॑ । \newline
40. अ॒न्त र॑धत्ता धत्ता॒न्त र॒न्त र॑धत्त । \newline
41. अ॒ध॒त्त॒ तत् तद॑धत्ता धत्त॒ तत् । \newline
42. तद॑न्तर्या॒मस्या᳚ न्तर्या॒मस्य॒ तत् तद॑न्तर्या॒मस्य॑ । \newline
43. अ॒न्त॒र्या॒मस्या᳚ न्तर्याम॒त्व म॑न्तर्याम॒त्व म॑न्तर्या॒मस्या᳚ न्तर्या॒मस्या᳚ न्तर्याम॒त्वम् । \newline
44. अ॒न्त॒र्या॒मस्येत्य॑न्तः - या॒मस्य॑ । \newline
45. अ॒न्त॒र्या॒म॒त्वं ॅयद् यद॑न्तर्याम॒त्व म॑न्तर्याम॒त्वं ॅयत् । \newline
46. अ॒न्त॒र्या॒म॒त्वमित्य॑न्तर्याम - त्वम् । \newline
47. यद॑न्तर्या॒मो᳚ ऽन्तर्या॒मो यद् यद॑न्तर्या॒मः । \newline
48. अ॒न्त॒र्या॒मो गृ॒ह्यते॑ गृ॒ह्यते᳚ ऽन्तर्या॒मो᳚ ऽन्तर्या॒मो गृ॒ह्यते᳚ । \newline
49. अ॒न्त॒र्या॒म इत्य॑न्तः - या॒मः । \newline
50. गृ॒ह्यते॒ भ्रातृ॑व्या॒न् भ्रातृ॑व्यान् गृ॒ह्यते॑ गृ॒ह्यते॒ भ्रातृ॑व्यान् । \newline
51. भ्रातृ॑व्याने॒ वैव भ्रातृ॑व्या॒न् भ्रातृ॑व्याने॒व । \newline
52. ए॒व तत् तदे॒ वैव तत् । \newline
53. तद् यज॑मानो॒ यज॑मान॒ स्तत् तद् यज॑मानः । \newline
54. यज॑मानो॒ ऽन्त र॒न्तर् यज॑मानो॒ यज॑मानो॒ ऽन्तः । \newline
55. अ॒न्तर् ध॑त्ते धत्ते॒ ऽन्त र॒न्तर् ध॑त्ते । \newline
56. ध॒त्ते॒ ऽन्त र॒न्तर् ध॑त्ते धत्ते॒ ऽन्तः । \newline
57. अ॒न्त स्ते॑ ते॒ ऽन्त र॒न्त स्ते᳚ । \newline
58. ते॒ द॒धा॒मि॒ द॒धा॒मि॒ ते॒ ते॒ द॒धा॒मि॒ । \newline

\textbf{Ghana Paata } \newline

1. दे॒वा वै वै दे॒वा दे॒वा वै यद् यद् वै दे॒वा दे॒वा वै यत् । \newline
2. वै यद् यद् वै वै यद् य॒ज्ञे य॒ज्ञे यद् वै वै यद् य॒ज्ञे । \newline
3. यद् य॒ज्ञे य॒ज्ञे यद् यद् य॒ज्ञे ऽकु॑र्व॒ता कु॑र्वत य॒ज्ञे यद् यद् य॒ज्ञे ऽकु॑र्वत । \newline
4. य॒ज्ञे ऽकु॑र्व॒ता कु॑र्वत य॒ज्ञे य॒ज्ञे ऽकु॑र्वत॒ तत् तदकु॑र्वत य॒ज्ञे य॒ज्ञे ऽकु॑र्वत॒ तत् । \newline
5. अकु॑र्वत॒ तत् तदकु॑र्व॒ता कु॑र्वत॒ तदसु॑रा॒ असु॑रा॒ स्तदकु॑र्व॒ता कु॑र्वत॒ तदसु॑राः । \newline
6. तदसु॑रा॒ असु॑रा॒ स्तत् तदसु॑रा अकुर्वता कुर्व॒ता सु॑रा॒ स्तत् तदसु॑रा अकुर्वत । \newline
7. असु॑रा अकुर्वता कुर्व॒ता सु॑रा॒ असु॑रा अकुर्वत॒ ते ते॑ ऽकुर्व॒ता सु॑रा॒ असु॑रा अकुर्वत॒ ते । \newline
8. अ॒कु॒र्व॒त॒ ते ते॑ ऽकुर्वता कुर्वत॒ ते दे॒वा दे॒वा स्ते॑ ऽकुर्वता कुर्वत॒ ते दे॒वाः । \newline
9. ते दे॒वा दे॒वा स्ते ते दे॒वा उ॑पाꣳ॒॒शा वु॑पाꣳ॒॒शौ दे॒वा स्ते ते दे॒वा उ॑पाꣳ॒॒शौ । \newline
10. दे॒वा उ॑पाꣳ॒॒शा वु॑पाꣳ॒॒शौ दे॒वा दे॒वा उ॑पाꣳ॒॒शौ य॒ज्ञ्ं ॅय॒ज्ञ् मु॑पाꣳ॒॒शौ दे॒वा दे॒वा उ॑पाꣳ॒॒शौ य॒ज्ञ्म् । \newline
11. उ॒पाꣳ॒॒शौ य॒ज्ञ्ं ॅय॒ज्ञ् मु॑पाꣳ॒॒शा वु॑पाꣳ॒॒शौ य॒ज्ञ्ꣳ सꣳ॒॒स्थाप्यꣳ॑ सꣳ॒॒स्थाप्यं॑ ॅय॒ज्ञ् मु॑पाꣳ॒॒शा वु॑पाꣳ॒॒शौ य॒ज्ञ्ꣳ सꣳ॒॒स्थाप्य᳚म् । \newline
12. उ॒पाꣳ॒॒शावित्यु॑प - अꣳ॒॒शौ । \newline
13. य॒ज्ञ्ꣳ सꣳ॒॒स्थाप्यꣳ॑ सꣳ॒॒स्थाप्यं॑ ॅय॒ज्ञ्ं ॅय॒ज्ञ्ꣳ सꣳ॒॒स्थाप्य॑ मपश्यन् नपश्यन् थ्सꣳ॒॒स्थाप्यं॑ ॅय॒ज्ञ्ं ॅय॒ज्ञ्ꣳ सꣳ॒॒स्थाप्य॑ मपश्यन्न् । \newline
14. सꣳ॒॒स्थाप्य॑ मपश्यन् नपश्यन् थ्सꣳ॒॒स्थाप्यꣳ॑ सꣳ॒॒स्थाप्य॑ मपश्य॒न् तम् तम॑पश्यन् थ्सꣳ॒॒स्थाप्यꣳ॑ सꣳ॒॒स्थाप्य॑ मपश्य॒न् तम् । \newline
15. सꣳ॒॒स्थाप्य॒मिति॑ सं - स्थाप्य᳚म् । \newline
16. अ॒प॒श्य॒न् तम् त म॑पश्यन् नपश्य॒न् त मु॑पाꣳ॒॒शा वु॑पाꣳ॒॒शौ त म॑पश्यन् नपश्य॒न् त मु॑पाꣳ॒॒शौ । \newline
17. त मु॑पाꣳ॒॒शा वु॑पाꣳ॒॒शौ तम् त मु॑पाꣳ॒॒शौ सꣳ स मु॑पाꣳ॒॒शौ तम् त मु॑पाꣳ॒॒शौ सम् । \newline
18. उ॒पाꣳ॒॒शौ सꣳ स मु॑पाꣳ॒॒शा वु॑पाꣳ॒॒शौ स म॑स्थापयन् नस्थापय॒न् थ्स मु॑पाꣳ॒॒शा वु॑पाꣳ॒॒शौ स म॑स्थापयन्न् । \newline
19. उ॒पाꣳ॒॒शावित्यु॑प - अꣳ॒॒शौ । \newline
20. स म॑स्थापयन् नस्थापय॒न् थ्सꣳ स म॑स्थापय॒न् ते ते᳚ ऽस्थापय॒न् थ्सꣳ स म॑स्थापय॒न् ते । \newline
21. अ॒स्था॒प॒य॒न् ते ते᳚ ऽस्थापयन् नस्थापय॒न् ते ऽसु॑रा॒ असु॑रा॒ स्ते᳚ ऽस्थापयन् नस्थापय॒न् ते ऽसु॑राः । \newline
22. ते ऽसु॑रा॒ असु॑रा॒ स्ते ते ऽसु॑रा॒ वज्रं॒ ॅवज्र॒ मसु॑रा॒ स्ते ते ऽसु॑रा॒ वज्र᳚म् । \newline
23. असु॑रा॒ वज्रं॒ ॅवज्र॒ मसु॑रा॒ असु॑रा॒ वज्र॑ मु॒द्यत्यो॒ द्यत्य॒ वज्र॒ मसु॑रा॒ असु॑रा॒ वज्र॑ मु॒द्यत्य॑ । \newline
24. वज्र॑ मु॒द्यत्यो॒ द्यत्य॒ वज्रं॒ ॅवज्र॑ मु॒द्यत्य॑ दे॒वान् दे॒वा नु॒द्यत्य॒ वज्रं॒ ॅवज्र॑ मु॒द्यत्य॑ दे॒वान् । \newline
25. उ॒द्यत्य॑ दे॒वान् दे॒वा नु॒द्यत्यो॒ द्यत्य॑ दे॒वान॒भ्य॑भि दे॒वा नु॒द्यत्यो॒ द्यत्य॑ दे॒वान॒भि । \newline
26. उ॒द्यत्येत्यु॑त् - यत्य॑ । \newline
27. दे॒वा न॒भ्य॑भि दे॒वान् दे॒वा न॒भ्या॑यन्ता यन्ता॒भि दे॒वान् दे॒वा न॒भ्या॑यन्त । \newline
28. अ॒भ्या॑ यन्ता यन्ता॒भ्या᳚(1॒)भ्या॑ यन्त॒ ते त आ॑यन्ता॒ भ्या᳚(1॒)भ्या॑यन्त॒ ते । \newline
29. आ॒य॒न्त॒ ते त आ॑यन्ता यन्त॒ ते दे॒वा दे॒वा स्त आ॑यन्ता यन्त॒ ते दे॒वाः । \newline
30. ते दे॒वा दे॒वा स्ते ते दे॒वा बिभ्य॑तो॒ बिभ्य॑तो दे॒वा स्ते ते दे॒वा बिभ्य॑तः । \newline
31. दे॒वा बिभ्य॑तो॒ बिभ्य॑तो दे॒वा दे॒वा बिभ्य॑त॒ इन्द्र॒ मिन्द्र॒म् बिभ्य॑तो दे॒वा दे॒वा बिभ्य॑त॒ इन्द्र᳚म् । \newline
32. बिभ्य॑त॒ इन्द्र॒ मिन्द्र॒म् बिभ्य॑तो॒ बिभ्य॑त॒ इन्द्र॒ मुपोपेन्द्र॒म् बिभ्य॑तो॒ बिभ्य॑त॒ इन्द्र॒ मुप॑ । \newline
33. इन्द्र॒ मुपोपेन्द्र॒ मिन्द्र॒ मुपा॑धावन् नधाव॒न् नुपेन्द्र॒ मिन्द्र॒ मुपा॑धावन्न् । \newline
34. उपा॑धावन् नधाव॒न् नुपोपा॑ धाव॒न् ताꣳ स्ता न॑धाव॒न् नुपोपा॑ धाव॒न् तान् । \newline
35. अ॒धा॒व॒न् ताꣳ स्तान॑धावन् नधाव॒न् तानिन्द्र॒ इन्द्र॒ स्ता न॑धावन् नधाव॒न् तानिन्द्रः॑ । \newline
36. तानिन्द्र॒ इन्द्र॒ स्ताꣳ स्ता निन्द्रो᳚ ऽन्तर्या॒मेणा᳚ न्तर्या॒मे णेन्द्र॒ स्ताꣳ स्ता निन्द्रो᳚ ऽन्तर्या॒मेण॑ । \newline
37. इन्द्रो᳚ ऽन्तर्या॒मेणा᳚ न्तर्या॒मेणेन्द्र॒ इन्द्रो᳚ ऽन्तर्या॒मेणा॒न्त र॒न्त र॑न्तर्या॒मेणेन्द्र॒ इन्द्रो᳚ ऽन्तर्या॒मे णा॒न्तः । \newline
38. अ॒न्त॒र्या॒मेणा॒ न्त र॒न्त र॑न्तर्या॒मेणा᳚ न्तर्या॒मेणा॒न्त र॑धत्ता धत्ता॒ न्त र॑न्तर्या॒मेणा᳚ न्तर्या॒मेणा॒ 
न्तर॑धत्त । \newline
39. अ॒न्त॒र्या॒मेणेत्य॑न्तः - या॒मेन॑ । \newline
40. अ॒न्त र॑धत्ता धत्ता॒ न्तर॒न्त र॑धत्त॒ तत् तद॑धत्ता॒ न्तर॒न्त र॑धत्त॒ तत् । \newline
41. अ॒ध॒त्त॒ तत् तद॑धत्ता धत्त॒ तद॑न्तर्या॒मस्या᳚ न्तर्या॒मस्य॒ तद॑धत्ता धत्त॒ तद॑न्तर्या॒मस्य॑ । \newline
42. तद॑न्तर्या॒मस्या᳚ न्तर्या॒मस्य॒ तत् तद॑न्तर्या॒मस्या᳚ न्तर्याम॒त्व म॑न्तर्याम॒त्व म॑न्तर्या॒मस्य॒ तत् तद॑न्तर्या॒मस्या᳚ न्तर्याम॒त्वम् । \newline
43. अ॒न्त॒र्या॒मस्या᳚ न्तर्याम॒त्व म॑न्तर्याम॒त्व म॑न्तर्या॒मस्या᳚ न्तर्या॒मस्या᳚ न्तर्याम॒त्वं ॅयद् यद॑न्तर्याम॒त्व म॑न्तर्या॒मस्या᳚ न्तर्या॒मस्या᳚ न्तर्याम॒त्वं ॅयत् । \newline
44. अ॒न्त॒र्या॒मस्येत्य॑न्तः - या॒मस्य॑ । \newline
45. अ॒न्त॒र्या॒म॒त्वं ॅयद् यद॑न्तर्याम॒त्व म॑न्तर्याम॒त्वं ॅयद॑न्तर्या॒मो᳚ ऽन्तर्या॒मो यद॑न्तर्याम॒त्व म॑न्तर्याम॒त्वं ॅयद॑न्तर्या॒मः । \newline
46. अ॒न्त॒र्या॒म॒त्वमित्य॑न्तर्याम - त्वम् । \newline
47. यद॑न्तर्या॒मो᳚ ऽन्तर्या॒मो यद् यद॑न्तर्या॒मो गृ॒ह्यते॑ गृ॒ह्यते᳚ ऽन्तर्या॒मो यद् यद॑न्तर्या॒मो गृ॒ह्यते᳚ । \newline
48. अ॒न्त॒र्या॒मो गृ॒ह्यते॑ गृ॒ह्यते᳚ ऽन्तर्या॒मो᳚ ऽन्तर्या॒मो गृ॒ह्यते॒ भ्रातृ॑व्या॒न् भ्रातृ॑व्यान् गृ॒ह्यते᳚ ऽन्तर्या॒मो᳚ ऽन्तर्या॒मो गृ॒ह्यते॒ भ्रातृ॑व्यान् । \newline
49. अ॒न्त॒र्या॒म इत्य॑न्तः - या॒मः । \newline
50. गृ॒ह्यते॒ भ्रातृ॑व्या॒न् भ्रातृ॑व्यान् गृ॒ह्यते॑ गृ॒ह्यते॒ भ्रातृ॑व्या ने॒वैव भ्रातृ॑व्यान् गृ॒ह्यते॑ गृ॒ह्यते॒ भ्रातृ॑व्याने॒व । \newline
51. भ्रातृ॑व्या ने॒वैव भ्रातृ॑व्या॒न् भ्रातृ॑व्याने॒व तत् तदे॒व भ्रातृ॑व्या॒न् भ्रातृ॑व्याने॒व तत् । \newline
52. ए॒व तत् तदे॒ वैव तद् यज॑मानो॒ यज॑मान॒ स्तदे॒वैव तद् यज॑मानः । \newline
53. तद् यज॑मानो॒ यज॑मान॒ स्तत् तद् यज॑मानो॒ ऽन्त र॒न्तर् यज॑मान॒ स्तत् तद् यज॑मानो॒ ऽन्तः । \newline
54. यज॑मानो॒ ऽन्त र॒न्तर् यज॑मानो॒ यज॑मानो॒ ऽन्तर् ध॑त्ते धत्ते॒ ऽन्तर् यज॑मानो॒ यज॑मानो॒ ऽन्तर् ध॑त्ते । \newline
55. अ॒न्तर् ध॑त्ते धत्ते॒ ऽन्त र॒न्तर् ध॑त्ते॒ ऽन्त र॒न्तर् ध॑त्ते॒ ऽन्त र॒न्तर् ध॑त्ते॒ ऽन्तः । \newline
56. ध॒त्ते॒ ऽन्त र॒न्तर् ध॑त्ते धत्ते॒ ऽन्त स्ते॑ ते॒ ऽन्तर् ध॑त्ते धत्ते॒ ऽन्त स्ते᳚ । \newline
57. अ॒न्त स्ते॑ ते॒ ऽन्त र॒न्त स्ते॑ दधामि दधामि ते॒ ऽन्त र॒न्त स्ते॑ दधामि । \newline
58. ते॒ द॒धा॒मि॒ द॒धा॒मि॒ ते॒ ते॒ द॒धा॒मि॒ द्यावा॑पृथि॒वी द्यावा॑पृथि॒वी द॑धामि ते ते दधामि॒ द्यावा॑पृथि॒वी । \newline
\pagebreak
\markright{ TS 6.4.6.2  \hfill https://www.vedavms.in \hfill}

\section{ TS 6.4.6.2 }

\textbf{TS 6.4.6.2 } \newline
\textbf{Samhita Paata} \newline

दधामि॒ द्यावा॑पृथि॒वी अ॒न्तरु॒-र्व॑न्तरि॑क्ष॒-मित्या॑है॒भिरे॒व लो॒कैर्यज॑मानो॒ भ्रातृ॑व्यान॒न्तर्द्ध॑त्ते॒ ते दे॒वा अ॑मन्य॒न्तेन्द्रो॒ वा इ॒दम॑भू॒द्यद् व॒यꣳ स्म इति॒ ते᳚ऽब्रुव॒न् मघ॑व॒न्ननु॑ न॒ आ भ॒जेति॑ स॒जोषा॑ दे॒वैरव॑रैः॒ परै॒श्चेत्य॑ब्रवी॒द्ये चै॒व दे॒वाः परे॒ ये चाव॑रे॒ तानु॒भया॑- [  ] \newline

\textbf{Pada Paata} \newline

द॒धा॒मि॒ । द्यावा॑पृथि॒वी इति॒ द्यावा᳚ - पृ॒थि॒वी । अ॒न्तः । उ॒रु । अ॒न्तरि॑क्षम् । इति॑ । आ॒ह॒ । ए॒भिः । ए॒व । लो॒कैः । यज॑मानः । भ्रातृ॑व्यान् । अ॒न्तः । ध॒त्ते॒ । ते । दे॒वाः । अ॒म॒न्य॒न्त॒ । इन्द्रः॑ । वै । इ॒दम् । अ॒भू॒त् । यत् । व॒यम् । स्मः । इति॑ । ते । अ॒ब्रु॒व॒न्न् । मघ॑व॒न्निति॒ मघ॑ - व॒न्न् । अन्विति॑ । नः॒ । एति॑ । भ॒ज॒ । इति॑ । स॒जोषा॒ इति॑ स - जोषाः᳚ । दे॒वैः । अव॑रैः । परैः᳚ । च॒ । इति॑ । अ॒ब्र॒वी॒त् । ये । च॒ । ए॒व । दे॒वाः । परे᳚ । ये । च॒ । अव॑रे । तान् । उ॒भयान्॑ ।  \newline


\textbf{Krama Paata} \newline

द॒धा॒मि॒ द्यावा॑पृथि॒वी । द्यावा॑पृथि॒वी अ॒न्तः । द्यावा॑पृथि॒वी इति॒ द्यावा᳚ - पृ॒थि॒वी । अ॒न्तरु॒रु । उ॒र्व॑न्तरि॑क्षम् । अ॒न्तरि॑क्ष॒मिति॑ । इत्या॑ह । आ॒है॒भिः । ए॒भिरे॒व । ए॒व लो॒कैः । लो॒कैर् यज॑मानः । यज॑मानो॒ भ्रातृ॑व्यान् । भ्रातृ॑व्यान॒न्तः । अ॒न्तर् ध॑त्ते । ध॒त्ते॒ ते । ते दे॒वाः । दे॒वा अ॑मन्यन्त । अ॒म॒न्य॒न्तेन्द्रः॑ । इन्द्रो॒ वै । वा इ॒दम् । इ॒दम॑भूत् । अ॒भू॒द् यत् । यद् व॒यम् । व॒यꣳ स्मः । स्म इति॑ । इति॒ ते । ते᳚ऽब्रुवन्न् । अ॒ब्रु॒व॒न् मघ॑वन्न् । मघ॑व॒न्ननु॑ । मघ॑व॒न्निति॒ मघ॑ - व॒न्न्॒ । अनु॑ नः । न॒ आ । आ भ॑ज । भ॒जेति॑ । इति॑ स॒जोषाः᳚ । स॒जोषा॑ दे॒वैः । स॒जोषा॒ इति॑ स - जोषाः᳚ । दे॒वैरव॑रैः । अव॑रैः॒ परैः᳚ । परै᳚श्च । चेति॑ । इत्य॑ब्रवीत् । अ॒ब्र॒वी॒द् ये । ये च॑ । चै॒व । ए॒व दे॒वाः । दे॒वाः परे᳚ । परे॒ ये । ये च॑ । चाव॑रे । अव॑रे॒ तान् । तानु॒भयान्॑ । उ॒भया॑न॒न्वाभ॑जत् \newline

\textbf{Jatai Paata} \newline

1. द॒धा॒मि॒ द्यावा॑पृथि॒वी द्यावा॑पृथि॒वी द॑धामि दधामि॒ द्यावा॑पृथि॒वी । \newline
2. द्यावा॑पृथि॒वी अ॒न्त र॒न्तर् द्यावा॑पृथि॒वी द्यावा॑पृथि॒वी अ॒न्तः । \newline
3. द्यावा॑पृथि॒वी इति॒ द्यावा᳚ - पृ॒थि॒वी । \newline
4. अ॒न्त रु॒रू᳚(1॒) र्व॑न्त र॒न्त रु॒रु । \newline
5. उ॒र्व॑न्तरि॑क्ष म॒न्तरि॑क्ष मु॒रू᳚(1॒) र्व॑न्तरि॑क्षम् । \newline
6. अ॒न्तरि॑क्ष॒ मिती त्य॒न्तरि॑क्ष म॒न्तरि॑क्ष॒ मिति॑ । \newline
7. इत्या॑हा॒हे तीत्या॑ह । \newline
8. आ॒है॒भि रे॒भि रा॑हा है॒भिः । \newline
9. ए॒भि रे॒वै वैभि रे॒भि रे॒व । \newline
10. ए॒व लो॒कैर् लो॒कै रे॒वैव लो॒कैः । \newline
11. लो॒कैर् यज॑मानो॒ यज॑मानो लो॒कैर् लो॒कैर् यज॑मानः । \newline
12. यज॑मानो॒ भ्रातृ॑व्या॒न् भ्रातृ॑व्या॒न्॒. यज॑मानो॒ यज॑मानो॒ भ्रातृ॑व्यान् । \newline
13. भ्रातृ॑व्या न॒न्त र॒न्तर् भ्रातृ॑व्या॒न् भ्रातृ॑व्या न॒न्तः । \newline
14. अ॒न्तर् ध॑त्ते धत्ते॒ ऽन्त र॒न्तर् ध॑त्ते । \newline
15. ध॒त्ते॒ ते ते ध॑त्ते धत्ते॒ ते । \newline
16. ते दे॒वा दे॒वा स्ते ते दे॒वाः । \newline
17. दे॒वा अ॑मन्यन्ता मन्यन्त दे॒वा दे॒वा अ॑मन्यन्त । \newline
18. अ॒म॒न्य॒न्तेन्द्र॒ इन्द्रो॑ ऽमन्यन्ता मन्य॒न्तेन्द्रः॑ । \newline
19. इन्द्रो॒ वै वा इन्द्र॒ इन्द्रो॒ वै । \newline
20. वा इ॒द मि॒दं ॅवै वा इ॒दम् । \newline
21. इ॒द म॑भू दभू दि॒द मि॒द म॑भूत् । \newline
22. अ॒भू॒द् यद् यद॑भू दभू॒द् यत् । \newline
23. यद् व॒यं ॅव॒यं ॅयद् यद् व॒यम् । \newline
24. व॒यꣳ स्मः स्मो व॒यं ॅव॒यꣳ स्मः । \newline
25. स्म इतीति॒ स्मः स्म इति॑ । \newline
26. इति॒ ते त इतीति॒ ते । \newline
27. ते᳚ ऽब्रुवन् नब्रुव॒न् ते ते᳚ ऽब्रुवन्न् । \newline
28. अ॒ब्रु॒व॒न् मघ॑व॒न् मघ॑वन् नब्रुवन् नब्रुव॒न् मघ॑वन्न् । \newline
29. मघ॑व॒न् नन्वनु॒ मघ॑व॒न् मघ॑व॒न् ननु॑ । \newline
30. मघ॑व॒न्निति॒ मघ॑ - व॒न्न् । \newline
31. अनु॑ नो॒ नो ऽन्वनु॑ नः । \newline
32. न॒ आ नो॑ न॒ आ । \newline
33. आ भ॑ज भ॒जा भ॑ज । \newline
34. भ॒जेतीति॑ भज भ॒जेति॑ । \newline
35. इति॑ स॒जोषाः᳚ स॒जोषा॒ इतीति॑ स॒जोषाः᳚ । \newline
36. स॒जोषा॑ दे॒वैर् दे॒वैः स॒जोषाः᳚ स॒जोषा॑ दे॒वैः । \newline
37. स॒जोषा॒ इति॑ स - जोषाः᳚ । \newline
38. दे॒वै रव॑रै॒ रव॑रैर् दे॒वैर् दे॒वै रव॑रैः । \newline
39. अव॑रैः॒ परैः॒ परै॒ रव॑रै॒ रव॑रैः॒ परैः᳚ । \newline
40. परै᳚ श्च च॒ परैः॒ परै᳚ श्च । \newline
41. चेतीति॑ च॒ चेति॑ । \newline
42. इत्य॑ब्रवी दब्रवी॒ दिती त्य॑ब्रवीत् । \newline
43. अ॒ब्र॒वी॒द् ये ये᳚ ऽब्रवी दब्रवी॒द् ये । \newline
44. ये च॑ च॒ ये ये च॑ । \newline
45. चै॒वैव च॑ चै॒व । \newline
46. ए॒व दे॒वा दे॒वा ए॒वैव दे॒वाः । \newline
47. दे॒वाः परे॒ परे॑ दे॒वा दे॒वाः परे᳚ । \newline
48. परे॒ ये ये परे॒ परे॒ ये । \newline
49. ये च॑ च॒ ये ये च॑ । \newline
50. चाव॒रे ऽव॑रे च॒ चाव॑रे । \newline
51. अव॑रे॒ ताꣳ स्ता नव॒रे ऽव॑रे॒ तान् । \newline
52. तानु॒भया॑ नु॒भया॒न् ताꣳ स्ता नु॒भयान्॑ । \newline
53. उ॒भया॑ न॒न्वाभ॑ज द॒न्वाभ॑ज दु॒भया॑ नु॒भया॑ न॒न्वाभ॑जत् । \newline

\textbf{Ghana Paata } \newline

1. द॒धा॒मि॒ द्यावा॑पृथि॒वी द्यावा॑पृथि॒वी द॑धामि दधामि॒ द्यावा॑पृथि॒वी अ॒न्त र॒न्तर् द्यावा॑पृथि॒वी द॑धामि दधामि॒ द्यावा॑पृथि॒वी अ॒न्तः । \newline
2. द्यावा॑पृथि॒वी अ॒न्त र॒न्तर् द्यावा॑पृथि॒वी द्यावा॑पृथि॒वी अ॒न्त रु॒रू᳚(1॒)र्व॑न्तर् द्यावा॑पृथि॒वी द्यावा॑पृथि॒वी अ॒न्त रु॒रु । \newline
3. द्यावा॑पृथि॒वी इति॒ द्यावा᳚ - पृ॒थि॒वी । \newline
4. अ॒न्त रु॒रू᳚(1॒)र्व॑न्त र॒न्त रु॒र्व॑न्तरि॑क्ष म॒न्तरि॑क्ष मु॒र्व॑न्त र॒न्त रु॒र्व॑न्तरि॑क्षम् । \newline
5. उ॒र्व॑न्तरि॑क्ष म॒न्तरि॑क्ष मु॒रू᳚(1॒)र्व॑न्तरि॑क्ष॒ मिती त्य॒न्तरि॑क्ष मु॒रू᳚(1॒)र्व॑न्तरि॑क्ष॒ मिति॑ । \newline
6. अ॒न्तरि॑क्ष॒ मिती त्य॒न्तरि॑क्ष म॒न्तरि॑क्ष॒ मित्या॑हा॒हे त्य॒न्तरि॑क्ष म॒न्तरि॑क्ष॒ मित्या॑ह । \newline
7. इत्या॑हा॒हे तीत्या॑ है॒भि रे॒भि रा॒हे तीत्या॑ है॒भिः । \newline
8. आ॒है॒भि रे॒भि रा॑हा है॒भि रे॒वै वैभि रा॑हा है॒भि रे॒व । \newline
9. ए॒भि रे॒वै वैभि रे॒भि रे॒व लो॒कैर् लो॒कै रे॒वैभि रे॒भि रे॒व लो॒कैः । \newline
10. ए॒व लो॒कैर् लो॒कै रे॒वैव लो॒कैर् यज॑मानो॒ यज॑मानो लो॒कै रे॒वैव लो॒कैर् यज॑मानः । \newline
11. लो॒कैर् यज॑मानो॒ यज॑मानो लो॒कैर् लो॒कैर् यज॑मानो॒ भ्रातृ॑व्या॒न् भ्रातृ॑व्या॒न्॒. यज॑मानो लो॒कैर् लो॒कैर् यज॑मानो॒ भ्रातृ॑व्यान् । \newline
12. यज॑मानो॒ भ्रातृ॑व्या॒न् भ्रातृ॑व्या॒न्॒. यज॑मानो॒ यज॑मानो॒ भ्रातृ॑व्या न॒न्त र॒न्तर् भ्रातृ॑व्या॒न्॒. यज॑मानो॒ यज॑मानो॒ भ्रातृ॑व्या न॒न्तः । \newline
13. भ्रातृ॑व्या न॒न्त र॒न्तर् भ्रातृ॑व्या॒न् भ्रातृ॑व्या न॒न्तर् ध॑त्ते धत्ते॒ ऽन्तर् भ्रातृ॑व्या॒न् भ्रातृ॑व्या न॒न्तर् ध॑त्ते । \newline
14. अ॒न्तर् ध॑त्ते धत्ते॒ ऽन्त र॒न्तर् ध॑त्ते॒ ते ते ध॑त्ते॒ ऽन्त र॒न्तर् ध॑त्ते॒ ते । \newline
15. ध॒त्ते॒ ते ते ध॑त्ते धत्ते॒ ते दे॒वा दे॒वा स्ते ध॑त्ते धत्ते॒ ते दे॒वाः । \newline
16. ते दे॒वा दे॒वा स्ते ते दे॒वा अ॑मन्यन्ता मन्यन्त दे॒वा स्ते ते दे॒वा अ॑मन्यन्त । \newline
17. दे॒वा अ॑मन्यन्ता मन्यन्त दे॒वा दे॒वा अ॑मन्य॒न्तेन्द्र॒ इन्द्रो॑ ऽमन्यन्त दे॒वा दे॒वा अ॑मन्य॒न्तेन्द्रः॑ । \newline
18. अ॒म॒न्य॒न्तेन्द्र॒ इन्द्रो॑ ऽमन्यन्ता मन्य॒न्तेन्द्रो॒ वै वा इन्द्रो॑ ऽमन्यन्ता मन्य॒न्तेन्द्रो॒ वै । \newline
19. इन्द्रो॒ वै वा इन्द्र॒ इन्द्रो॒ वा इ॒द मि॒दं ॅवा इन्द्र॒ इन्द्रो॒ वा इ॒दम् । \newline
20. वा इ॒द मि॒दं ॅवै वा इ॒द म॑भू दभू दि॒दं ॅवै वा इ॒द म॑भूत् । \newline
21. इ॒द म॑भू दभू दि॒द मि॒द म॑भू॒द् यद् यद॑भू दि॒द मि॒द म॑भू॒द् यत् । \newline
22. अ॒भू॒द् यद् यद॑भू दभू॒द् यद् व॒यं ॅव॒यं ॅयद॑भू दभू॒द् यद् व॒यम् । \newline
23. यद् व॒यं ॅव॒यं ॅयद् यद् व॒यꣳ स्मः स्मो व॒यं ॅयद् यद् व॒यꣳ स्मः । \newline
24. व॒यꣳ स्मः स्मो व॒यं ॅव॒यꣳ स्म इतीति॒ स्मो व॒यं ॅव॒यꣳ स्म इति॑ । \newline
25. स्म इतीति॒ स्मः स्म इति॒ ते त इति॒ स्मः स्म इति॒ ते । \newline
26. इति॒ ते त इतीति॒ ते᳚ ऽब्रुवन् नब्रुव॒न् त इतीति॒ ते᳚ ऽब्रुवन्न् । \newline
27. ते᳚ ऽब्रुवन् नब्रुव॒न् ते ते᳚ ऽब्रुव॒न् मघ॑व॒न् मघ॑वन् नब्रुव॒न् ते ते᳚ ऽब्रुव॒न् मघ॑वन्न् । \newline
28. अ॒ब्रु॒व॒न् मघ॑व॒न् मघ॑वन् नब्रुवन् नब्रुव॒न् मघ॑व॒न् नन्वनु॒ मघ॑वन् नब्रुवन् नब्रुव॒न् मघ॑व॒न् ननु॑ । \newline
29. मघ॑व॒न् नन् वनु॒ मघ॑व॒न् मघ॑व॒न् ननु॑ नो॒ नो ऽनु॒ मघ॑व॒न् मघ॑व॒न् ननु॑ नः । \newline
30. मघ॑व॒न्निति॒ मघ॑ - व॒न्न् । \newline
31. अनु॑ नो॒ नो ऽन्वनु॑ न॒ आ नो ऽन्वनु॑ न॒ आ । \newline
32. न॒ आ नो॑ न॒ आ भ॑ज भ॒जा नो॑ न॒ आ भ॑ज । \newline
33. आ भ॑ज भ॒जा भ॒जे तीति॑ भ॒जा भ॒जेति॑ । \newline
34. भ॒जे तीति॑ भज भ॒जेति॑ स॒जोषाः᳚ स॒जोषा॒ इति॑ भज भ॒जेति॑ स॒जोषाः᳚ । \newline
35. इति॑ स॒जोषाः᳚ स॒जोषा॒ इतीति॑ स॒जोषा॑ दे॒वैर् दे॒वैः स॒जोषा॒ इतीति॑ स॒जोषा॑ दे॒वैः । \newline
36. स॒जोषा॑ दे॒वैर् दे॒वैः स॒जोषाः᳚ स॒जोषा॑ दे॒वै रव॑रै॒ रव॑रैर् दे॒वैः स॒जोषाः᳚ स॒जोषा॑ दे॒वै रव॑रैः । \newline
37. स॒जोषा॒ इति॑ स - जोषाः᳚ । \newline
38. दे॒वै रव॑रै॒ रव॑रैर् दे॒वैर् दे॒वै रव॑रैः॒ परैः॒ परै॒ रव॑रैर् दे॒वैर् दे॒वै रव॑रैः॒ परैः᳚ । \newline
39. अव॑रैः॒ परैः॒ परै॒ रव॑रै॒ रव॑रैः॒ परै᳚श्च च॒ परै॒ रव॑रै॒ रव॑रैः॒ परै᳚श्च । \newline
40. परै᳚श्च च॒ परैः॒ परै॒ श्चेतीति॑ च॒ परैः॒ परै॒श्चेति॑ । \newline
41. चेतीति॑ च॒ चेत्य॑ब्रवी दब्रवी॒ दिति॑ च॒ चेत्य॑ब्रवीत् । \newline
42. इत्य॑ब्रवी दब्रवी॒ दिती त्य॑ब्रवी॒द् ये ये᳚ ऽब्रवी॒ दिती त्य॑ब्रवी॒द् ये । \newline
43. अ॒ब्र॒वी॒द् ये ये᳚ ऽब्रवी दब्रवी॒द् ये च॑ च॒ ये᳚ ऽब्रवी दब्रवी॒द् ये च॑ । \newline
44. ये च॑ च॒ ये ये चै॒वैव च॒ ये ये चै॒व । \newline
45. चै॒वैव च॑ चै॒व दे॒वा दे॒वा ए॒व च॑ चै॒व दे॒वाः । \newline
46. ए॒व दे॒वा दे॒वा ए॒वैव दे॒वाः परे॒ परे॑ दे॒वा ए॒वैव दे॒वाः परे᳚ । \newline
47. दे॒वाः परे॒ परे॑ दे॒वा दे॒वाः परे॒ ये ये परे॑ दे॒वा दे॒वाः परे॒ ये । \newline
48. परे॒ ये ये परे॒ परे॒ ये च॑ च॒ ये परे॒ परे॒ ये च॑ । \newline
49. ये च॑ च॒ ये ये चाव॒रे ऽव॑रे च॒ ये ये चाव॑रे । \newline
50. चाव॒रे ऽव॑रे च॒ चाव॑रे॒ ताꣳ स्ता नव॑रे च॒ चाव॑रे॒ तान् । \newline
51. अव॑रे॒ ताꣳ स्ता नव॒रे ऽव॑रे॒ तानु॒भया॑ नु॒भया॒न् तानव॒रे ऽव॑रे॒ तानु॒भयान्॑ । \newline
52. तानु॒भया॑ नु॒भया॒न् ताꣳ स्तानु॒भया॑ न॒न्वाभ॑ज द॒न्वाभ॑ज दु॒भया॒न् ताꣳ स्ता नु॒भया॑ न॒न्वाभ॑जत् । \newline
53. उ॒भया॑ न॒न्वाभ॑ज द॒न्वाभ॑ज दु॒भया॑ नु॒भया॑ न॒न्वाभ॑जथ् स॒जोषाः᳚ स॒जोषा॑ अ॒न्वाभ॑ज दु॒भया॑ नु॒भया॑ न॒न्वाभ॑जथ् स॒जोषाः᳚ । \newline
\pagebreak
\markright{ TS 6.4.6.3  \hfill https://www.vedavms.in \hfill}

\section{ TS 6.4.6.3 }

\textbf{TS 6.4.6.3 } \newline
\textbf{Samhita Paata} \newline

न॒न्वाभ॑जथ् स॒जोषा॑ दे॒वैरव॑रैः॒ परै॒श्चेत्या॑ह॒ ये चै॒व दे॒वाः परे॒ ये चाव॑रे॒ तानु॒भया॑-न॒न्वाभ॑ज-त्यन्तर्या॒मे म॑घवन् मादय॒स्वेत्या॑ह य॒ज्ञादे॒व यज॑मानं॒ नान्तरे᳚त्युपया॒म-गृ॑हीतो॒ ऽसीत्या॑हापा॒नस्य॒ धृत्यै॒ यदु॒भाव॑पवि॒त्रौ गृ॒ह्येया॑तां प्रा॒णम॑पा॒नोऽनु॒ न्यृ॑च्छेत् प्र॒मायु॑कः स्यात् प॒वित्र॑वानन्तर्या॒मो गृ॑ह्यते- [  ] \newline

\textbf{Pada Paata} \newline

अ॒न्वाभ॑ज॒दित्य॑नु - आभ॑जत् । स॒जोषा॒ इति॑ स - जोषाः᳚ । दे॒वैः । अव॑रैः । परैः᳚ । च॒ । इति॑ । आ॒ह॒ । ये । च॒ । ए॒व । दे॒वाः । परे᳚ । ये । च॒ । अव॑रे । तान् । उ॒भयान्॑ । अ॒न्वाभ॑ज॒तीत्य॑नु- आभ॑जति । अ॒न्त॒र्या॒म इत्य॑न्तः - या॒मे । म॒घ॒व॒न्निति॑ मघ - व॒न्न् । मा॒द॒य॒स्व॒ । इति॑ । आ॒ह॒ । य॒ज्ञात् । ए॒व । यज॑मानम् । न । अ॒न्तः । ए॒ति॒ । उ॒प॒या॒मगृ॑हीत॒ इत्यु॑पया॒म-गृ॒ही॒तः॒ । अ॒सि॒ । इति॑ । आ॒ह॒ । अ॒पा॒नस्येत्य॑प - अ॒नस्य॑ । धृत्यै᳚ । यत् । उ॒भौ । अ॒प॒वि॒त्रौ । गृ॒ह्येया॑ताम् । प्रा॒णमिति॑ प्र-अ॒नम् । अ॒पा॒न इत्य॑प - अ॒नः । अनु॑ । नीति॑ । ऋ॒च्छे॒त् । प्र॒मायु॑क॒ इति॑ प्र-मायु॑कः । स्या॒त् । प॒वित्र॑वा॒निति॑ प॒वित्र॑ - वा॒न् । अ॒न्त॒र्या॒म इत्य॑न्तः - या॒मः । गृ॒ह्य॒ते॒ ।  \newline


\textbf{Krama Paata} \newline

अ॒न्वाभ॑जथ् स॒जोषाः᳚ । अ॒न्वाभ॑ज॒दित्य॑नु - आभ॑जत् । स॒जोषा॑ दे॒वैः । स॒जोषा॒ इति॑ स - जोषाः᳚ । दे॒वैरव॑रैः । अव॑रैः॒ परैः᳚ । परै᳚श्च । चेति॑ । इत्या॑ह । आ॒ह॒ ये । ये च॑ । चै॒व । ए॒व दे॒वाः । दे॒वाः परे᳚ । परे॒ ये । ये च॑ । चाव॑रे । अव॑रे॒ तान् । तानु॒भयान्॑ । उ॒भया॑न॒न्वाभ॑जति । अ॒न्वाभ॑जत्यन्तर्या॒मे । अ॒न्वाभ॑ज॒तीत्य॑नु - आभ॑जति । अ॒न्त॒र्या॒मे म॑घवन्न् । अ॒न्त॒र्या॒म इत्य॑न्तः - या॒मे । म॒घ॒व॒न् मा॒द॒य॒स्व॒ । म॒घ॒व॒न्निति॑ मघ - व॒न्न्॒ । मा॒द॒य॒स्वेति॑ । इत्या॑ह । आ॒ह॒ य॒ज्ञात् । य॒ज्ञादे॒व । ए॒व यज॑मानम् । यज॑मान॒म् न । नान्तः । अ॒न्तरे॑ति । ए॒त्यु॒प॒या॒मगृ॑हीतः । उ॒प॒या॒मगृ॑हीतोऽसि । उ॒प॒या॒मगृ॑हीत॒ इत्यु॑पया॒म - गृ॒ही॒तः॒ । अ॒सीति॑ । इत्या॑ह । आ॒हा॒पा॒नस्य॑ । अ॒पा॒नस्य॒ धृत्यै᳚ । अ॒पा॒नस्येत्य॑प - अ॒नस्य॑ । धृत्यै॒ यत् । यदु॒भौ । उ॒भाव॑पवि॒त्रौ । अ॒प॒वि॒त्रौ गृ॒ह्येया॑ताम् । गृ॒ह्येया॑ताम् प्रा॒णम् । प्रा॒णम॑पा॒नः । प्रा॒णमिति॑ प्र - अ॒नम् । अ॒पा॒नोऽनु॑ । अ॒पा॒न इत्य॑प - अ॒नः । अनु॒ नि । न्यृ॑च्छेत् । ऋ॒च्छे॒त् प्र॒मायु॑कः । प्र॒मायु॑कः स्यात् । प्र॒मायु॑क॒ इति॑ प्र - मायु॑कः । स्या॒त् प॒वित्र॑वान् । प॒वित्र॑वानन्तर्या॒मः । प॒वित्र॑वा॒निति॑ प॒वित्र॑ - वा॒न्॒ । अ॒न्त॒र्या॒मो गृ॑ह्यते ( ) । अ॒न्त॒र्या॒म इत्य॑न्तः - या॒मः । गृ॒ह्य॒ते॒ प्रा॒णा॒पा॒नयोः᳚ \newline

\textbf{Jatai Paata} \newline

1. अ॒न्वाभ॑जथ् स॒जोषाः᳚ स॒जोषा॑ अ॒न्वाभ॑ज द॒न्वाभ॑जथ् स॒जोषाः᳚ । \newline
2. अ॒न्वाभ॑ज॒दित्य॑नु - आभ॑जत् । \newline
3. स॒जोषा॑ दे॒वैर् दे॒वैः स॒जोषाः᳚ स॒जोषा॑ दे॒वैः । \newline
4. स॒जोषा॒ इति॑ स - जोषाः᳚ । \newline
5. दे॒वै रव॑रै॒ रव॑रैर् दे॒वैर् दे॒वै रव॑रैः । \newline
6. अव॑रैः॒ परैः॒ परै॒ रव॑रै॒ रव॑रैः॒ परैः᳚ । \newline
7. परै᳚ श्च च॒ परैः॒ परै᳚ श्च । \newline
8. चेतीति॑ च॒ चेति॑ । \newline
9. इत्या॑हा॒हे तीत्या॑ह । \newline
10. आ॒ह॒ ये य आ॑हाह॒ ये । \newline
11. ये च॑ च॒ ये ये च॑ । \newline
12. चै॒वैव च॑ चै॒व । \newline
13. ए॒व दे॒वा दे॒वा ए॒वैव दे॒वाः । \newline
14. दे॒वाः परे॒ परे॑ दे॒वा दे॒वाः परे᳚ । \newline
15. परे॒ ये ये परे॒ परे॒ ये । \newline
16. ये च॑ च॒ ये ये च॑ । \newline
17. चाव॒रे ऽव॑रे च॒ चाव॑रे । \newline
18. अव॑रे॒ ताꣳ स्तान व॒रे ऽव॑रे॒ तान् । \newline
19. तानु॒भया॑ नु॒भया॒न् ताꣳ स्ता नु॒भयान्॑ । \newline
20. उ॒भया॑ न॒न्वाभ॑ज त्य॒न्वाभ॑ज त्यु॒भया॑ नु॒भया॑ न॒न्वाभ॑जति । \newline
21. अ॒न्वाभ॑ज त्यन्तर्या॒मे᳚ ऽन्तर्या॒मे᳚ ऽन्वाभ॑ज त्य॒न्वाभ॑ज त्यन्तर्या॒मे । \newline
22. अ॒न्वाभ॑ज॒तीत्य॑नु - आभ॑जति । \newline
23. अ॒न्त॒र्या॒मे म॑घवन् मघवन् नन्तर्या॒मे᳚ ऽन्तर्या॒मे म॑घवन्न् । \newline
24. अ॒न्त॒र्या॒म इत्य॑न्तः - या॒मे । \newline
25. म॒घ॒व॒न् मा॒द॒य॒स्व॒ मा॒द॒य॒स्व॒ म॒घ॒व॒न् म॒घ॒व॒न् मा॒द॒य॒स्व॒ । \newline
26. म॒घ॒व॒न्निति॑ मघ - व॒न्न् । \newline
27. मा॒द॒य॒ स्वेतीति॑ मादयस्व मादय॒स्वेति॑ । \newline
28. इत्या॑हा॒हे तीत्या॑ह । \newline
29. आ॒ह॒ य॒ज्ञाद् य॒ज्ञा दा॑हाह य॒ज्ञात् । \newline
30. य॒ज्ञा दे॒वैव य॒ज्ञाद् य॒ज्ञा दे॒व । \newline
31. ए॒व यज॑मानं॒ ॅयज॑मान मे॒वैव यज॑मानम् । \newline
32. यज॑मान॒न् न न यज॑मानं॒ ॅयज॑मान॒न् न । \newline
33. नान्त र॒न्तर् न नान्तः । \newline
34. अ॒न्त रे᳚त्ये त्य॒न्त र॒न्त रे॑ति । \newline
35. ए॒त्यु॒प॒या॒मगृ॑हीत उपया॒मगृ॑हीत एत्येत्युपया॒मगृ॑हीतः । \newline
36. उ॒प॒या॒मगृ॑हीतो ऽस्य स्युपया॒मगृ॑हीत उपया॒मगृ॑हीतो ऽसि । \newline
37. उ॒प॒या॒मगृ॑हीत॒ इत्यु॑पया॒म - गृ॒ही॒तः॒ । \newline
38. अ॒सीती त्य॑स्य॒ सीति॑ । \newline
39. इत्या॑हा॒हे तीत्या॑ह । \newline
40. आ॒हा॒ पा॒नस्या॑ पा॒नस्या॑ हाहा पा॒नस्य॑ । \newline
41. अ॒पा॒नस्य॒ धृत्यै॒ धृत्या॑ अपा॒नस्या॑ पा॒नस्य॒ धृत्यै᳚ । \newline
42. अ॒पा॒नस्येत्य॑प - अ॒नस्य॑ । \newline
43. धृत्यै॒ यद् यद् धृत्यै॒ धृत्यै॒ यत् । \newline
44. यदु॒भा वु॒भौ यद् यदु॒भौ । \newline
45. उ॒भा व॑पवि॒त्रा व॑पवि॒त्रा वु॒भा वु॒भा व॑पवि॒त्रौ । \newline
46. अ॒प॒वि॒त्रौ गृ॒ह्येया॑ताम् गृ॒ह्येया॑ता मपवि॒त्रा व॑पवि॒त्रौ गृ॒ह्येया॑ताम् । \newline
47. गृ॒ह्येया॑ताम् प्रा॒णम् प्रा॒णम् गृ॒ह्येया॑ताम् गृ॒ह्येया॑ताम् प्रा॒णम् । \newline
48. प्रा॒ण म॑पा॒नो॑ ऽपा॒नः प्रा॒णम् प्रा॒ण म॑पा॒नः । \newline
49. प्रा॒णमिति॑ प्र - अ॒नम् । \newline
50. अ॒पा॒नो ऽन्वन् व॑पा॒नो॑ ऽपा॒नो ऽनु॑ । \newline
51. अ॒पा॒न इत्य॑प - अ॒नः । \newline
52. अनु॒ नि न्यन् वनु॒ नि । \newline
53. न्यृ॑च्छे दृच्छे॒न् नि न्यृ॑च्छेत् । \newline
54. ऋ॒च्छे॒त् प्र॒मायु॑कः प्र॒मायु॑क ऋच्छे दृच्छेत् प्र॒मायु॑कः । \newline
55. प्र॒मायु॑कः स्याथ् स्यात् प्र॒मायु॑कः प्र॒मायु॑कः स्यात् । \newline
56. प्र॒मायु॑क॒ इति॑ प्र - मायु॑कः । \newline
57. स्या॒त् प॒वित्र॑वान् प॒वित्र॑वान् थ्स्याथ् स्यात् प॒वित्र॑वान् । \newline
58. प॒वित्र॑वा नन्तर्या॒मो᳚ ऽन्तर्या॒मः प॒वित्र॑वान् प॒वित्र॑वा नन्तर्या॒मः । \newline
59. प॒वित्र॑वा॒निति॑ प॒वित्र॑ - वा॒न् । \newline
60. अ॒न्त॒र्या॒मो गृ॑ह्यते गृह्यते ऽन्तर्या॒मो᳚ ऽन्तर्या॒मो गृ॑ह्यते । \newline
61. अ॒न्त॒र्या॒म इत्य॑न्तः - या॒मः । \newline
62. गृ॒ह्य॒ते॒ प्रा॒णा॒पा॒नयोः᳚ प्राणापा॒नयो᳚र् गृह्यते गृह्यते प्राणापा॒नयोः᳚ । \newline

\textbf{Ghana Paata } \newline

1. अ॒न्वाभ॑जथ् स॒जोषाः᳚ स॒जोषा॑ अ॒न्वाभ॑ज द॒न्वाभ॑जथ् स॒जोषा॑ दे॒वैर् दे॒वैः स॒जोषा॑ अ॒न्वाभ॑ज द॒न्वाभ॑जथ् स॒जोषा॑ दे॒वैः । \newline
2. अ॒न्वाभ॑ज॒दित्य॑नु - आभ॑जत् । \newline
3. स॒जोषा॑ दे॒वैर् दे॒वैः स॒जोषाः᳚ स॒जोषा॑ दे॒वै रव॑रै॒ रव॑रैर् दे॒वैः स॒जोषाः᳚ स॒जोषा॑ दे॒वै रव॑रैः । \newline
4. स॒जोषा॒ इति॑ स - जोषाः᳚ । \newline
5. दे॒वै रव॑रै॒ रव॑रैर् दे॒वैर् दे॒वै रव॑रैः॒ परैः॒ परै॒ रव॑रैर् दे॒वैर् दे॒वै रव॑रैः॒ परैः᳚ । \newline
6. अव॑रैः॒ परैः॒ परै॒ रव॑रै॒ रव॑रैः॒ परै᳚श्च च॒ परै॒ रव॑रै॒ रव॑रैः॒ परै᳚श्च । \newline
7. परै᳚श्च च॒ परैः॒ परै॒ श्चेतीति॑ च॒ परैः॒ परै॒ श्चेति॑ । \newline
8. चेतीति॑ च॒ चे त्या॑हा॒ हेति॑ च॒ चे त्या॑ह । \newline
9. इत्या॑हा॒हे तीत्या॑ह॒ ये य आ॒हे तीत्या॑ह॒ ये । \newline
10. आ॒ह॒ ये य आ॑हाह॒ ये च॑ च॒ य आ॑हाह॒ ये च॑ । \newline
11. ये च॑ च॒ ये ये चै॒वैव च॒ ये ये चै॒व । \newline
12. चै॒वैव च॑ चै॒व दे॒वा दे॒वा ए॒व च॑ चै॒व दे॒वाः । \newline
13. ए॒व दे॒वा दे॒वा ए॒वैव दे॒वाः परे॒ परे॑ दे॒वा ए॒वैव दे॒वाः परे᳚ । \newline
14. दे॒वाः परे॒ परे॑ दे॒वा दे॒वाः परे॒ ये ये परे॑ दे॒वा दे॒वाः परे॒ ये । \newline
15. परे॒ ये ये परे॒ परे॒ ये च॑ च॒ ये परे॒ परे॒ ये च॑ । \newline
16. ये च॑ च॒ ये ये चाव॒रे ऽव॑रे च॒ ये ये चाव॑रे । \newline
17. चाव॒रे ऽव॑रे च॒ चाव॑रे॒ ताꣳ स्ता नव॑रे च॒ चाव॑रे॒ तान् । \newline
18. अव॑रे॒ ताꣳ स्ता नव॒रे ऽव॑रे॒ तानु॒भया॑ नु॒भया॒न् तानव॒रे ऽव॑रे॒ तानु॒भयान्॑ । \newline
19. तानु॒भया॑ नु॒भया॒न् ताꣳ स्तानु॒भया॑ न॒न्वाभ॑ज त्य॒न्वाभ॑ज त्यु॒भया॒न् ताꣳ स्तानु॒भया॑ न॒न्वाभ॑जति । \newline
20. उ॒भया॑ न॒न्वाभ॑ज त्य॒न्वाभ॑ज त्यु॒भया॑ नु॒भया॑ न॒न्वाभ॑ज त्यन्तर्या॒मे᳚ ऽन्तर्या॒मे᳚ ऽन्वाभ॑ज त्यु॒भया॑ नु॒भया॑ न॒न्वाभ॑ज त्यन्तर्या॒मे । \newline
21. अ॒न्वाभ॑ज त्यन्तर्या॒मे᳚ ऽन्तर्या॒मे᳚ ऽन्वाभ॑ज त्य॒न्वाभ॑ज त्यन्तर्या॒मे म॑घवन् मघवन् नन्तर्या॒मे᳚ ऽन्वाभ॑ज त्य॒न्वाभ॑ज त्यन्तर्या॒मे म॑घवन्न् । \newline
22. अ॒न्वाभ॑ज॒तीत्य॑नु - आभ॑जति । \newline
23. अ॒न्त॒र्या॒मे म॑घवन् मघवन् नन्तर्या॒मे᳚ ऽन्तर्या॒मे म॑घवन् मादयस्व मादयस्व मघवन् नन्तर्या॒मे᳚ ऽन्तर्या॒मे म॑घवन् मादयस्व । \newline
24. अ॒न्त॒र्या॒म इत्य॑न्तः - या॒मे । \newline
25. म॒घ॒व॒न् मा॒द॒य॒स्व॒ मा॒द॒य॒स्व॒ म॒घ॒व॒न् म॒घ॒व॒न् मा॒द॒य॒स्वेतीति॑ मादयस्व मघवन् मघवन् मादय॒स्वेति॑ । \newline
26. म॒घ॒व॒न्निति॑ मघ - व॒न्न् । \newline
27. मा॒द॒य॒स्वेतीति॑ मादयस्व मादय॒स्वे त्या॑हा॒हेति॑ मादयस्व मादय॒स्वे त्या॑ह । \newline
28. इत्या॑हा॒हे तीत्या॑ह य॒ज्ञाद् य॒ज्ञा दा॒हे तीत्या॑ह य॒ज्ञात् । \newline
29. आ॒ह॒ य॒ज्ञाद् य॒ज्ञा दा॑हाह य॒ज्ञा दे॒वैव य॒ज्ञा दा॑हाह य॒ज्ञा दे॒व । \newline
30. य॒ज्ञा दे॒वैव य॒ज्ञाद् य॒ज्ञा दे॒व यज॑मानं॒ ॅयज॑मान मे॒व य॒ज्ञाद् य॒ज्ञा दे॒व यज॑मानम् । \newline
31. ए॒व यज॑मानं॒ ॅयज॑मान मे॒वैव यज॑मान॒न् न न यज॑मान मे॒वैव यज॑मान॒न् न । \newline
32. यज॑मान॒न् न न यज॑मानं॒ ॅयज॑मान॒न् नान्त र॒न्तर् न यज॑मानं॒ ॅयज॑मान॒न् नान्तः । \newline
33. नान्त र॒न्तर् न नान्त रे᳚त्ये त्य॒न्तर् न नान्त रे॑ति । \newline
34. अ॒न्त रे᳚त्ये त्य॒न्त र॒न्त रे᳚त्युपया॒मगृ॑हीत उपया॒मगृ॑हीत एत्य॒न्त र॒न्त रे᳚त्युपया॒मगृ॑हीतः । \newline
35. ए॒त्यु॒प॒या॒मगृ॑हीत उपया॒मगृ॑हीत एत्ये त्युपया॒मगृ॑हीतो ऽस्य स्युपया॒मगृ॑हीत एत्ये त्युपया॒मगृ॑हीतो ऽसि । \newline
36. उ॒प॒या॒मगृ॑हीतो ऽस्य स्युपया॒मगृ॑हीत उपया॒मगृ॑हीतो॒ ऽसीती त्य॑स्युपया॒मगृ॑हीत उपया॒मगृ॑हीतो॒ ऽसीति॑ । \newline
37. उ॒प॒या॒मगृ॑हीत॒ इत्यु॑पया॒म - गृ॒ही॒तः॒ । \newline
38. अ॒सीती त्य॑स्य॒ सीत्या॑हा॒हे त्य॑स्य॒ सीत्या॑ह । \newline
39. इत्या॑हा॒हे तीत्या॑हा पा॒नस्या॑ पा॒नस्या॒हे तीत्या॑हा पा॒नस्य॑ । \newline
40. आ॒हा॒ पा॒नस्या॑ पा॒नस्या॑ हाहा पा॒नस्य॒ धृत्यै॒ धृत्या॑ अपा॒नस्या॑ हाहा पा॒नस्य॒ धृत्यै᳚ । \newline
41. अ॒पा॒नस्य॒ धृत्यै॒ धृत्या॑ अपा॒नस्या॑ पा॒नस्य॒ धृत्यै॒ यद् यद् धृत्या॑ अपा॒नस्या॑ पा॒नस्य॒ धृत्यै॒ यत् । \newline
42. अ॒पा॒नस्येत्य॑प - अ॒नस्य॑ । \newline
43. धृत्यै॒ यद् यद् धृत्यै॒ धृत्यै॒ यदु॒भा वु॒भौ यद् धृत्यै॒ धृत्यै॒ यदु॒भौ । \newline
44. यदु॒भा वु॒भौ यद् यदु॒भा व॑पवि॒त्रा व॑पवि॒त्रा वु॒भौ यद् यदु॒भा व॑पवि॒त्रौ । \newline
45. उ॒भा व॑पवि॒त्रा व॑पवि॒त्रा वु॒भा वु॒भा व॑पवि॒त्रौ गृ॒ह्येया॑ताम् गृ॒ह्येया॑ता मपवि॒त्रा वु॒भा वु॒भा व॑पवि॒त्रौ गृ॒ह्येया॑ताम् । \newline
46. अ॒प॒वि॒त्रौ गृ॒ह्येया॑ताम् गृ॒ह्येया॑ता मपवि॒त्रा व॑पवि॒त्रौ गृ॒ह्येया॑ताम् प्रा॒णम् प्रा॒णम् गृ॒ह्येया॑ता मपवि॒त्रा व॑पवि॒त्रौ गृ॒ह्येया॑ताम् प्रा॒णम् । \newline
47. गृ॒ह्येया॑ताम् प्रा॒णम् प्रा॒णम् गृ॒ह्येया॑ताम् गृ॒ह्येया॑ताम् प्रा॒ण म॑पा॒नो॑ ऽपा॒नः प्रा॒णम् गृ॒ह्येया॑ताम् गृ॒ह्येया॑ताम् प्रा॒ण म॑पा॒नः । \newline
48. प्रा॒ण म॑पा॒नो॑ ऽपा॒नः प्रा॒णम् प्रा॒ण म॑पा॒नो ऽन्वन् व॑पा॒नः प्रा॒णम् प्रा॒ण म॑पा॒नो ऽनु॑ । \newline
49. प्रा॒णमिति॑ प्र - अ॒नम् । \newline
50. अ॒पा॒नो ऽन्वन् व॑पा॒नो॑ ऽपा॒नो ऽनु॒ नि न्यन् व॑पा॒नो॑ ऽपा॒नो ऽनु॒ नि । \newline
51. अ॒पा॒न इत्य॑प - अ॒नः । \newline
52. अनु॒ नि न्यन् वनु॒ न्यृ॑च्छे दृच्छे॒न् न्यन् वनु॒ न्यृ॑च्छेत् । \newline
53. न्यृ॑च्छे दृच्छे॒न् नि न्यृ॑च्छेत् प्र॒मायु॑कः प्र॒मायु॑क ऋच्छे॒न् नि न्यृ॑च्छेत् प्र॒मायु॑कः । \newline
54. ऋ॒च्छे॒त् प्र॒मायु॑कः प्र॒मायु॑क ऋच्छे दृच्छेत् प्र॒मायु॑कः स्याथ् स्यात् प्र॒मायु॑क ऋच्छे दृच्छेत् प्र॒मायु॑कः स्यात् । \newline
55. प्र॒मायु॑कः स्याथ् स्यात् प्र॒मायु॑कः प्र॒मायु॑कः स्यात् प॒वित्र॑वान् प॒वित्र॑वान् थ्स्यात् प्र॒मायु॑कः प्र॒मायु॑कः स्यात् प॒वित्र॑वान् । \newline
56. प्र॒मायु॑क॒ इति॑ प्र - मायु॑कः । \newline
57. स्या॒त् प॒वित्र॑वान् प॒वित्र॑वान् थ्स्याथ् स्यात् प॒वित्र॑वा नन्तर्या॒मो᳚ ऽन्तर्या॒मः प॒वित्र॑वान् थ्स्याथ् स्यात् प॒वित्र॑वा नन्तर्या॒मः । \newline
58. प॒वित्र॑वा नन्तर्या॒मो᳚ ऽन्तर्या॒मः प॒वित्र॑वान् प॒वित्र॑वा नन्तर्या॒मो गृ॑ह्यते गृह्यते ऽन्तर्या॒मः प॒वित्र॑वान् प॒वित्र॑वा नन्तर्या॒मो गृ॑ह्यते । \newline
59. प॒वित्र॑वा॒निति॑ प॒वित्र॑ - वा॒न् । \newline
60. अ॒न्त॒र्या॒मो गृ॑ह्यते गृह्यते ऽन्तर्या॒मो᳚ ऽन्तर्या॒मो गृ॑ह्यते प्राणापा॒नयोः᳚ प्राणापा॒नयो᳚र् गृह्यते ऽन्तर्या॒मो᳚ ऽन्तर्या॒मो गृ॑ह्यते प्राणापा॒नयोः᳚ । \newline
61. अ॒न्त॒र्या॒म इत्य॑न्तः - या॒मः । \newline
62. गृ॒ह्य॒ते॒ प्रा॒णा॒पा॒नयोः᳚ प्राणापा॒नयो᳚र् गृह्यते गृह्यते प्राणापा॒नयो॒र् विधृ॑त्यै॒ विधृ॑त्यै प्राणापा॒नयो᳚र् गृह्यते गृह्यते प्राणापा॒नयो॒र् विधृ॑त्यै । \newline
\pagebreak
\markright{ TS 6.4.6.4  \hfill https://www.vedavms.in \hfill}

\section{ TS 6.4.6.4 }

\textbf{TS 6.4.6.4 } \newline
\textbf{Samhita Paata} \newline

प्राणापा॒नयो॒-र्विधृ॑त्यै प्राणापा॒नौ वा ए॒तौ यदु॑पाꣳश्वन्तर्या॒मौ व्या॒न उ॑पाꣳशु॒ सव॑नो॒ यं का॒मये॑त प्र॒मायु॑कः स्या॒दित्यसꣳ॑ स्पृष्टौ॒ तस्य॑ सादयेद्-व्या॒नेनै॒वास्य॑ प्राणापा॒नौ वि च्छि॑नत्ति ता॒जक् प्र मी॑यते॒ यं का॒मये॑त॒ सर्व॒मायु॑रिया॒दिति॒ सꣳ स्पृ॑ष्टौ॒ तस्य॑ सादयेद्-व्या॒नेनै॒वास्य॑ प्राणापा॒नौ सं त॑नोति॒ सर्व॒मायु॑रेति ॥ \newline

\textbf{Pada Paata} \newline

प्रा॒णा॒पा॒नयो॒रिति॑ प्राण - अ॒पा॒नयोः᳚ । विधृ॑त्या॒ इति॒ वि - धृ॒त्यै॒ । प्रा॒णा॒पा॒नाविति॑ प्राण - अ॒पा॒नौ । वै । ए॒तौ । यत् । उ॒पाꣳ॒॒श्व॒न्त॒र्या॒मावित्यु॑पाꣳशु - अ॒न्त॒र्या॒मौ । व्या॒न इति॑ वि - अ॒नः । उ॒पाꣳ॒॒शु॒सव॑न॒ इत्यु॑पाꣳशु - सव॑नः । यम् । का॒मये॑त । प्र॒मायु॑क॒ इति॑ प्र - मायु॑कः । स्या॒त् । इति॑ । असꣳ॑स्पृष्टा॒वित्यसं᳚ - स्पृ॒ष्टौ॒ । तस्य॑ । सा॒द॒ये॒त् । व्या॒नेनेति॑ वि - अ॒नेन॑ । ए॒व । अ॒स्य॒ । प्रा॒णा॒पा॒नाविति॑ प्राण - अ॒पा॒नौ । वीति॑ । छि॒न॒त्ति॒ । ता॒जक् । प्रेति॑ । मी॒य॒ते॒ । यम् । का॒मये॑त । सर्व᳚म् । आयुः॑ । इ॒या॒त् । इति॑ । सꣳस्पृ॑ष्टा॒विति॒ सं-स्पृ॒ष्टौ॒ । तस्य॑ । सा॒द॒ये॒त् । व्या॒नेनेति॑ वि-अ॒नेन॑ । ए॒व । अ॒स्य॒ । प्रा॒णा॒पा॒नाविति॑ प्राण - अ॒पा॒नौ । समिति॑ । त॒नो॒ति॒ । सर्व᳚म् । आयुः॑ । ए॒ति॒ ॥  \newline


\textbf{Krama Paata} \newline

प्रा॒णा॒पा॒नयो॒र् विधृ॑त्यै । प्रा॒णा॒पा॒नयो॒रिति॑ प्राण - अ॒पा॒नयोः᳚ । विधृ॑त्यै प्राणापा॒नौ । विधृ॑त्या॒ इति॒ वि - धृ॒त्यै॒ । प्रा॒णा॒पा॒नौ वै । प्रा॒णा॒पा॒नाविति॑ प्राण - अ॒पा॒नौ । वा ए॒तौ । ए॒तौ यत् । यदु॑पाꣳश्वन्तर्या॒मौ । उ॒पाꣳ॒॒श्व॒न्त॒र्या॒मौ व्या॒नः । उ॒पाꣳ॒॒श्व॒न्त॒र्या॒मावित्यु॑पाꣳशु - अ॒न्त॒र्या॒मौ । व्या॒न उ॑पाꣳशु॒सव॑नः । व्या॒न इति॑ वि - अ॒नः । उ॒पाꣳ॒॒शु॒सव॑नो॒ यम् । उ॒पाꣳ॒॒शु॒सव॑न॒ इत्यु॑पाꣳशु - सव॑नः । यम् का॒मये॑त । का॒मये॑त प्र॒मायु॑कः । प्र॒मायु॑कः स्यात् । प्र॒मायु॑क॒ इति॑ प्र - मायु॑कः । स्या॒दिति॑ । इत्यसꣳ॑स्पृष्टौ । असꣳ॑स्पृष्टौ॒ तस्य॑ । असꣳ॑स्पृष्टा॒वित्यस᳚म् - स्पृ॒ष्टौ॒ । तस्य॑ सादयेत् । सा॒द॒ये॒द् व्या॒नेन॑ । व्या॒नेनै॒व । व्या॒नेनेति॑ वि - अ॒नेन॑ । ए॒वास्य॑ । अ॒स्य॒ प्रा॒णा॒पा॒नौ । प्रा॒णा॒पा॒नौ वि । प्रा॒णा॒पा॒नाविति॑ प्राण - अ॒पा॒नौ । विच्छि॑नत्ति । छि॒न॒त्ति॒ ता॒जक् । ता॒जक् प्र । प्र मी॑यते । मी॒य॒ते॒ यम् । यम् का॒मये॑त । का॒मये॑त॒ सर्व᳚म् । सर्व॒मायुः॑ । आयु॑रियात् । इ॒या॒दिति॑ । इति॒ सꣳस्पृ॑ष्टौ । सꣳस्प॑ष्टौ॒ तस्य॑ । सꣳस्पृ॑ष्टा॒विति॒ सम् - स्पृ॒ष्टौ॒ । तस्य॑ सादयेत् । सा॒द॒ये॒द् व्या॒नेन॑ । व्या॒नेनै॒व । व्या॒नेनेति॑ वि - अ॒नेन॑ । ए॒वास्य॑ । अ॒स्य॒ प्रा॒णा॒पा॒नौ । प्रा॒णा॒पा॒नौ सम् । प्रा॒णा॒पा॒नाविति॑ प्राण - अ॒पा॒नौ । सम् त॑नोति । त॒नो॒ति॒ सर्व᳚म् । सर्व॒मायुः॑ । आयु॑रेति । ए॒तीत्ये॑ति । \newline

\textbf{Jatai Paata} \newline

1. प्रा॒णा॒पा॒नयो॒र् विधृ॑त्यै॒ विधृ॑त्यै प्राणापा॒नयोः᳚ प्राणापा॒नयो॒र् विधृ॑त्यै । \newline
2. प्रा॒णा॒पा॒नयो॒रिति॑ प्राण - अ॒पा॒नयोः᳚ । \newline
3. विधृ॑त्यै प्राणापा॒नौ प्रा॑णापा॒नौ विधृ॑त्यै॒ विधृ॑त्यै प्राणापा॒नौ । \newline
4. विधृ॑त्या॒ इति॒ वि - धृ॒त्यै॒ । \newline
5. प्रा॒णा॒पा॒नौ वै वै प्रा॑णापा॒नौ प्रा॑णापा॒नौ वै । \newline
6. प्रा॒णा॒पा॒नाविति॑ प्राण - अ॒पा॒नौ । \newline
7. वा ए॒ता वे॒तौ वै वा ए॒तौ । \newline
8. ए॒तौ यद् यदे॒ता वे॒तौ यत् । \newline
9. यदु॑पाꣳश्वन्तर्या॒मा वु॑पाꣳश्वन्तर्या॒मौ यद् यदु॑पाꣳश्वन्तर्या॒मौ । \newline
10. उ॒पाꣳ॒॒श्व॒न्त॒र्या॒मौ व्या॒नो व्या॒न उ॑पाꣳश्वन्तर्या॒मा वु॑पाꣳश्वन्तर्या॒मौ व्या॒नः । \newline
11. उ॒पाꣳ॒॒श्व॒न्त॒र्या॒मावित्यु॑पाꣳशु - अ॒न्त॒र्या॒मौ । \newline
12. व्या॒न उ॑पाꣳशु॒सव॑न उपाꣳशु॒सव॑नो व्या॒नो व्या॒न उ॑पाꣳशु॒सव॑नः । \newline
13. व्या॒न इति॑ वि - अ॒नः । \newline
14. उ॒पाꣳ॒॒शु॒सव॑नो॒ यं ॅय मु॑पाꣳशु॒सव॑न उपाꣳशु॒सव॑नो॒ यम् । \newline
15. उ॒पाꣳ॒॒शु॒सव॑न॒ इत्यु॑पाꣳशु - सव॑नः । \newline
16. यम् का॒मये॑त का॒मये॑त॒ यं ॅयम् का॒मये॑त । \newline
17. का॒मये॑त प्र॒मायु॑कः प्र॒मायु॑कः का॒मये॑त का॒मये॑त प्र॒मायु॑कः । \newline
18. प्र॒मायु॑कः स्याथ् स्यात् प्र॒मायु॑कः प्र॒मायु॑कः स्यात् । \newline
19. प्र॒मायु॑क॒ इति॑ प्र - मायु॑कः । \newline
20. स्या॒ दितीति॑ स्याथ् स्या॒ दिति॑ । \newline
21. इत्य सꣳ॑स्पृष्टा॒ वसꣳ॑स्पृष्टा॒ विती त्यसꣳ॑स्पृष्टौ । \newline
22. असꣳ॑स्पृष्टौ॒ तस्य॒ तस्या सꣳ॑स्पृष्टा॒ वसꣳ॑स्पृष्टौ॒ तस्य॑ । \newline
23. असꣳ॑स्पृष्टा॒वित्यसं᳚ - स्पृ॒ष्टौ॒ । \newline
24. तस्य॑ सादयेथ् सादये॒त् तस्य॒ तस्य॑ सादयेत् । \newline
25. सा॒द॒ये॒द् व्या॒नेन॑ व्या॒नेन॑ सादयेथ् सादयेद् व्या॒नेन॑ । \newline
26. व्या॒ने नै॒वैव व्या॒नेन॑ व्या॒ने नै॒व । \newline
27. व्या॒नेनेति॑ वि - अ॒नेन॑ । \newline
28. ए॒वास्या᳚ स्यै॒वै वास्य॑ । \newline
29. अ॒स्य॒ प्रा॒णा॒पा॒नौ प्रा॑णापा॒ना व॑स्यास्य प्राणापा॒नौ । \newline
30. प्रा॒णा॒पा॒नौ वि वि प्रा॑णापा॒नौ प्रा॑णापा॒नौ वि । \newline
31. प्रा॒णा॒पा॒नाविति॑ प्राण - अ॒पा॒नौ । \newline
32. वि च्छि॑नत्ति छिनत्ति॒ वि वि च्छि॑नत्ति । \newline
33. छि॒न॒त्ति॒ ता॒जक् ता॒जक् छि॑नत्ति छिनत्ति ता॒जक् । \newline
34. ता॒जक् प्र प्र ता॒जक् ता॒जक् प्र । \newline
35. प्र मी॑यते मीयते॒ प्र प्र मी॑यते । \newline
36. मी॒य॒ते॒ यं ॅयम् मी॑यते मीयते॒ यम् । \newline
37. यम् का॒मये॑त का॒मये॑त॒ यं ॅयम् का॒मये॑त । \newline
38. का॒मये॑त॒ सर्वꣳ॒॒ सर्व॑म् का॒मये॑त का॒मये॑त॒ सर्व᳚म् । \newline
39. सर्व॒ मायु॒ रायुः॒ सर्वꣳ॒॒ सर्व॒ मायुः॑ । \newline
40. आयु॑ रिया दिया॒ दायु॒ रायु॑ रियात् । \newline
41. इ॒या॒ दिती ती॑या दिया॒ दिति॑ । \newline
42. इति॒ सꣳस्पृ॑ष्टौ॒ सꣳस्पृ॑ष्टा॒ वितीति॒ सꣳस्पृ॑ष्टौ । \newline
43. सꣳस्पृ॑ष्टौ॒ तस्य॒ तस्य॒ सꣳस्पृ॑ष्टौ॒ सꣳस्पृ॑ष्टौ॒ तस्य॑ । \newline
44. सꣳस्पृ॑ष्टा॒विति॒ सं - स्पृ॒ष्टौ॒ । \newline
45. तस्य॑ सादयेथ् सादये॒त् तस्य॒ तस्य॑ सादयेत् । \newline
46. सा॒द॒ये॒द् व्या॒नेन॑ व्या॒नेन॑ सादयेथ् सादयेद् व्या॒नेन॑ । \newline
47. व्या॒ने नै॒वैव व्या॒नेन॑ व्या॒ने नै॒व । \newline
48. व्या॒नेनेति॑ वि - अ॒नेन॑ । \newline
49. ए॒वास्या᳚ स्यै॒वै वास्य॑ । \newline
50. अ॒स्य॒ प्रा॒णा॒पा॒नौ प्रा॑णापा॒ना व॑स्यास्य प्राणापा॒नौ । \newline
51. प्रा॒णा॒पा॒नौ सꣳ सम् प्रा॑णापा॒नौ प्रा॑णापा॒नौ सम् । \newline
52. प्रा॒णा॒पा॒नाविति॑ प्राण - अ॒पा॒नौ । \newline
53. सम् त॑नोति तनोति॒ सꣳ सम् त॑नोति । \newline
54. त॒नो॒ति॒ सर्वꣳ॒॒ सर्व॑म् तनोति तनोति॒ सर्व᳚म् । \newline
55. सर्व॒ मायु॒ रायुः॒ सर्वꣳ॒॒ सर्व॒ मायुः॑ । \newline
56. आयु॑ रेत्ये॒ त्यायु॒ रायु॑रेति । \newline
57. ए॒तीत्ये॑ति । \newline

\textbf{Ghana Paata } \newline

1. प्रा॒णा॒पा॒नयो॒र् विधृ॑त्यै॒ विधृ॑त्यै प्राणापा॒नयोः᳚ प्राणापा॒नयो॒र् विधृ॑त्यै प्राणापा॒नौ प्रा॑णापा॒नौ विधृ॑त्यै प्राणापा॒नयोः᳚ प्राणापा॒नयो॒र् विधृ॑त्यै प्राणापा॒नौ । \newline
2. प्रा॒णा॒पा॒नयो॒रिति॑ प्राण - अ॒पा॒नयोः᳚ । \newline
3. विधृ॑त्यै प्राणापा॒नौ प्रा॑णापा॒नौ विधृ॑त्यै॒ विधृ॑त्यै प्राणापा॒नौ वै वै प्रा॑णापा॒नौ विधृ॑त्यै॒ विधृ॑त्यै प्राणापा॒नौ वै । \newline
4. विधृ॑त्या॒ इति॒ वि - धृ॒त्यै॒ । \newline
5. प्रा॒णा॒पा॒नौ वै वै प्रा॑णापा॒नौ प्रा॑णापा॒नौ वा ए॒ता वे॒तौ वै प्रा॑णापा॒नौ प्रा॑णापा॒नौ वा ए॒तौ । \newline
6. प्रा॒णा॒पा॒नाविति॑ प्राण - अ॒पा॒नौ । \newline
7. वा ए॒ता वे॒तौ वै वा ए॒तौ यद् यदे॒तौ वै वा ए॒तौ यत् । \newline
8. ए॒तौ यद् यदे॒ता वे॒तौ यदु॑पाꣳश्वन्तर्या॒मा वु॑पाꣳश्वन्तर्या॒मौ यदे॒ता वे॒तौ यदु॑पाꣳश्वन्तर्या॒मौ । \newline
9. यदु॑पाꣳश्वन्तर्या॒मा वु॑पाꣳश्वन्तर्या॒मौ यद् यदु॑पाꣳश्वन्तर्या॒मौ व्या॒नो व्या॒न उ॑पाꣳश्वन्तर्या॒मौ यद् यदु॑पाꣳश्वन्तर्या॒मौ व्या॒नः । \newline
10. उ॒पाꣳ॒॒श्व॒न्त॒र्या॒मौ व्या॒नो व्या॒न उ॑पाꣳश्वन्तर्या॒मा वु॑पाꣳश्वन्तर्या॒मौ व्या॒न उ॑पाꣳशु॒सव॑न उपाꣳशु॒सव॑नो व्या॒न उ॑पाꣳश्वन्तर्या॒मा वु॑पाꣳश्वन्तर्या॒मौ व्या॒न उ॑पाꣳशु॒सव॑नः । \newline
11. उ॒पाꣳ॒॒श्व॒न्त॒र्या॒मावित्यु॑पाꣳशु - अ॒न्त॒र्या॒मौ । \newline
12. व्या॒न उ॑पाꣳशु॒सव॑न उपाꣳशु॒सव॑नो व्या॒नो व्या॒न उ॑पाꣳशु॒सव॑नो॒ यं ॅय मु॑पाꣳशु॒सव॑नो व्या॒नो व्या॒न उ॑पाꣳशु॒सव॑नो॒ यम् । \newline
13. व्या॒न इति॑ वि - अ॒नः । \newline
14. उ॒पाꣳ॒॒शु॒सव॑नो॒ यं ॅय मु॑पाꣳशु॒सव॑न उपाꣳशु॒सव॑नो॒ यम् का॒मये॑त का॒मये॑त॒ य मु॑पाꣳशु॒सव॑न उपाꣳशु॒सव॑नो॒ यम् का॒मये॑त । \newline
15. उ॒पाꣳ॒॒शु॒सव॑न॒ इत्यु॑पाꣳशु - सव॑नः । \newline
16. यम् का॒मये॑त का॒मये॑त॒ यं ॅयम् का॒मये॑त प्र॒मायु॑कः प्र॒मायु॑कः का॒मये॑त॒ यं ॅयम् का॒मये॑त प्र॒मायु॑कः । \newline
17. का॒मये॑त प्र॒मायु॑कः प्र॒मायु॑कः का॒मये॑त का॒मये॑त प्र॒मायु॑कः स्याथ् स्यात् प्र॒मायु॑कः का॒मये॑त का॒मये॑त प्र॒मायु॑कः स्यात् । \newline
18. प्र॒मायु॑कः स्याथ् स्यात् प्र॒मायु॑कः प्र॒मायु॑कः स्या॒दि तीति॑ स्यात् प्र॒मायु॑कः प्र॒मायु॑कः स्या॒ दिति॑ । \newline
19. प्र॒मायु॑क॒ इति॑ प्र - मायु॑कः । \newline
20. स्या॒दि तीति॑ स्याथ् स्या॒ दित्यसꣳ॑स्पृष्टा॒ वसꣳ॑स्पृष्टा॒ विति॑ स्याथ् स्या॒ दित्यसꣳ॑स्पृष्टौ । \newline
21. इत्य सꣳ॑स्पृष्टा॒ वसꣳ॑स्पृष्टा॒ विती त्यसꣳ॑स्पृष्टौ॒ तस्य॒ तस्या सꣳ॑स्पृष्टा॒ विती
त्यसꣳ॑स्पृष्टौ॒ तस्य॑ । \newline
22. असꣳ॑स्पृष्टौ॒ तस्य॒ तस्या सꣳ॑स्पृष्टा॒ वसꣳ॑स्पृष्टौ॒ तस्य॑ सादयेथ् सादये॒त् तस्या सꣳ॑स्पृष्टा॒ वसꣳ॑स्पृष्टौ॒ तस्य॑ सादयेत् । \newline
23. असꣳ॑स्पृष्टा॒वित्यसं᳚ - स्पृ॒ष्टौ॒ । \newline
24. तस्य॑ सादयेथ् सादये॒त् तस्य॒ तस्य॑ सादयेद् व्या॒नेन॑ व्या॒नेन॑ सादये॒त् तस्य॒ तस्य॑ सादयेद् व्या॒नेन॑ । \newline
25. सा॒द॒ये॒द् व्या॒नेन॑ व्या॒नेन॑ सादयेथ् सादयेद् व्या॒नेनै॒ वैव व्या॒नेन॑ सादयेथ् सादयेद् व्या॒ने नै॒व । \newline
26. व्या॒ने नै॒वैव व्या॒नेन॑ व्या॒ने नै॒वास्या᳚ स्यै॒व व्या॒नेन॑ व्या॒ने नै॒वास्य॑ । \newline
27. व्या॒नेनेति॑ वि - अ॒नेन॑ । \newline
28. ए॒वास्या᳚ स्यै॒वै वास्य॑ प्राणापा॒नौ प्रा॑णापा॒ना व॑स्यै॒ वैवास्य॑ प्राणापा॒नौ । \newline
29. अ॒स्य॒ प्रा॒णा॒पा॒नौ प्रा॑णापा॒ना व॑स्यास्य प्राणापा॒नौ वि वि प्रा॑णापा॒ना व॑स्यास्य प्राणापा॒नौ वि । \newline
30. प्रा॒णा॒पा॒नौ वि वि प्रा॑णापा॒नौ प्रा॑णापा॒नौ वि च्छि॑नत्ति छिनत्ति॒ वि प्रा॑णापा॒नौ प्रा॑णापा॒नौ वि च्छि॑नत्ति । \newline
31. प्रा॒णा॒पा॒नाविति॑ प्राण - अ॒पा॒नौ । \newline
32. वि च्छि॑नत्ति छिनत्ति॒ वि वि च्छि॑नत्ति ता॒जक् ता॒जक् छि॑नत्ति॒ वि वि च्छि॑नत्ति ता॒जक् । \newline
33. छि॒न॒त्ति॒ ता॒जक् ता॒जक् छि॑नत्ति छिनत्ति ता॒जक् प्र प्र ता॒जक् छि॑नत्ति छिनत्ति ता॒जक् प्र । \newline
34. ता॒जक् प्र प्र ता॒जक् ता॒जक् प्र मी॑यते मीयते॒ प्र ता॒जक् ता॒जक् प्र मी॑यते । \newline
35. प्र मी॑यते मीयते॒ प्र प्र मी॑यते॒ यं ॅयम् मी॑यते॒ प्र प्र मी॑यते॒ यम् । \newline
36. मी॒य॒ते॒ यं ॅयम् मी॑यते मीयते॒ यम् का॒मये॑त का॒मये॑त॒ यम् मी॑यते मीयते॒ यम् का॒मये॑त । \newline
37. यम् का॒मये॑त का॒मये॑त॒ यं ॅयम् का॒मये॑त॒ सर्वꣳ॒॒ सर्व॑म् का॒मये॑त॒ यं ॅयम् का॒मये॑त॒ सर्व᳚म् । \newline
38. का॒मये॑त॒ सर्वꣳ॒॒ सर्व॑म् का॒मये॑त का॒मये॑त॒ सर्व॒ मायु॒ रायुः॒ सर्व॑म् का॒मये॑त का॒मये॑त॒ सर्व॒ मायुः॑ । \newline
39. सर्व॒ मायु॒ रायुः॒ सर्वꣳ॒॒ सर्व॒ मायु॑ रिया दिया॒ दायुः॒ सर्वꣳ॒॒ सर्व॒ मायु॑ रियात् । \newline
40. आयु॑ रिया दिया॒ दायु॒ रायु॑ रिया॒ दितीती॑या॒ दायु॒ रायु॑ रिया॒ दिति॑ । \newline
41. इ॒या॒ दितीती॑या दिया॒ दिति॒ सꣳस्पृ॑ष्टौ॒ सꣳस्पृ॑ष्टा॒ विती॑या दिया॒ दिति॒ सꣳस्पृ॑ष्टौ । \newline
42. इति॒ सꣳस्पृ॑ष्टौ॒ सꣳस्पृ॑ष्टा॒ वितीति॒ सꣳस्पृ॑ष्टौ॒ तस्य॒ तस्य॒ सꣳस्पृ॑ष्टा॒ वितीति॒ सꣳस्पृ॑ष्टौ॒ तस्य॑ । \newline
43. सꣳस्पृ॑ष्टौ॒ तस्य॒ तस्य॒ सꣳस्पृ॑ष्टौ॒ सꣳस्पृ॑ष्टौ॒ तस्य॑ सादयेथ् सादये॒त् तस्य॒ सꣳस्पृ॑ष्टौ॒ सꣳस्पृ॑ष्टौ॒ तस्य॑ सादयेत् । \newline
44. सꣳस्पृ॑ष्टा॒विति॒ सं - स्पृ॒ष्टौ॒ । \newline
45. तस्य॑ सादयेथ् सादये॒त् तस्य॒ तस्य॑ सादयेद् व्या॒नेन॑ व्या॒नेन॑ सादये॒त् तस्य॒ तस्य॑ सादयेद् व्या॒नेन॑ । \newline
46. सा॒द॒ये॒द् व्या॒नेन॑ व्या॒नेन॑ सादयेथ् सादयेद् व्या॒ने नै॒वैव व्या॒नेन॑ सादयेथ् सादयेद् व्या॒ने नै॒व । \newline
47. व्या॒ने नै॒वैव व्या॒नेन॑ व्या॒ने नै॒वास्या᳚ स्यै॒व व्या॒नेन॑ व्या॒ने नै॒वास्य॑ । \newline
48. व्या॒नेनेति॑ वि - अ॒नेन॑ । \newline
49. ए॒वास्या᳚ स्यै॒वैवास्य॑ प्राणापा॒नौ प्रा॑णापा॒ना व॑स्यै॒ वैवास्य॑ प्राणापा॒नौ । \newline
50. अ॒स्य॒ प्रा॒णा॒पा॒नौ प्रा॑णापा॒ना व॑स्यास्य प्राणापा॒नौ सꣳ सम् प्रा॑णापा॒ना व॑स्यास्य प्राणापा॒नौ सम् । \newline
51. प्रा॒णा॒पा॒नौ सꣳ सम् प्रा॑णापा॒नौ प्रा॑णापा॒नौ सम् त॑नोति तनोति॒ सम् प्रा॑णापा॒नौ प्रा॑णापा॒नौ सम् त॑नोति । \newline
52. प्रा॒णा॒पा॒नाविति॑ प्राण - अ॒पा॒नौ । \newline
53. सम् त॑नोति तनोति॒ सꣳ सम् त॑नोति॒ सर्वꣳ॒॒ सर्व॑म् तनोति॒ सꣳ सम् त॑नोति॒ सर्व᳚म् । \newline
54. त॒नो॒ति॒ सर्वꣳ॒॒ सर्व॑म् तनोति तनोति॒ सर्व॒ मायु॒ रायुः॒ सर्व॑म् तनोति तनोति॒ सर्व॒ मायुः॑ । \newline
55. सर्व॒ मायु॒ रायुः॒ सर्वꣳ॒॒ सर्व॒ मायु॑ रेत्ये॒ त्यायुः॒ सर्वꣳ॒॒ सर्व॒ मायु॑रेति । \newline
56. आयु॑रे त्ये॒ त्यायु॒ रायु॑रेति । \newline
57. ए॒तीत्ये॑ति । \newline
\pagebreak
\markright{ TS 6.4.7.1  \hfill https://www.vedavms.in \hfill}

\section{ TS 6.4.7.1 }

\textbf{TS 6.4.7.1 } \newline
\textbf{Samhita Paata} \newline

वाग्वा ए॒षा यदै᳚न्द्रवाय॒वो यदै᳚न्द्रवाय॒वाग्रा॒ ग्रहा॑ गृ॒ह्यन्ते॒ वाच॑मे॒वानु॒ प्र य॑न्ति वा॒युं दे॒वा अ॑ब्रुव॒न्थ् सोमꣳ॒॒ राजा॑नꣳ हना॒मेति॒ सो᳚ऽब्रवी॒द् वरं॑ ॅवृणै॒ मद॑ग्रा ए॒व वो॒ ग्रहा॑ गृह्यान्ता॒ इति॒ तस्मा॑दैन्द्रवाय॒वाग्रा॒ ग्रहा॑ गृह्यन्ते॒ तम॑घ्न॒न्थ् सो॑ऽपूय॒त् तं दे॒वा नोपा॑धृष्णुव॒न् ते वा॒ युम॑ब्रुवन्नि॒मं नः॑ स्वद॒ये- [  ] \newline

\textbf{Pada Paata} \newline

वाक् । वै । ए॒षा । यत् । ऐ॒न्द्र॒वा॒य॒व इत्यै᳚न्द्र - वा॒य॒वः । यत् । ऐ॒न्द्र॒वा॒य॒वाग्रा॒ इत्यै᳚न्द्रवाय॒व - अ॒ग्राः॒ । ग्रहाः᳚ । गृ॒ह्यन्ते᳚ । वाच᳚म् । ए॒व । अनु॑ । प्रेति॑ । य॒न्ति॒ । वा॒युम् । दे॒वाः । अ॒ब्रु॒व॒न्न् । सोम᳚म् । राजा॑नम् । ह॒ना॒म॒ । इति॑ । सः । अ॒ब्र॒वी॒त् । वर᳚म् । वृ॒णै॒ । मद॑ग्रा॒ इति॒ मत् - अ॒ग्राः॒ । ए॒व । वः॒ । ग्रहाः᳚ । गृ॒ह्या॒न्तै॒ । इति॑ । तस्मा᳚त् । ऐ॒न्द्र॒वा॒य॒वाग्रा॒ इत्यै᳚न्द्रवाय॒व - अ॒ग्राः॒ । ग्रहाः᳚ । गृ॒ह्य॒न्ते॒ । तम् । अ॒घ्न॒न्न् । सः । अ॒पू॒य॒त् । तम् । दे॒वाः । न । उपेति॑ । अ॒धृ॒ष्णु॒व॒न्न् । ते । वा॒युम् । अ॒ब्रु॒व॒न्न् । इ॒मम् । नः॒ । स्व॒द॒य॒ ।  \newline


\textbf{Krama Paata} \newline

वाग् वै । वा ए॒षा । ए॒षा यत् । यदै᳚न्द्रवाय॒वः । ऐ॒न्द्र॒वा॒य॒वो यत् । ऐ॒न्द्र॒वा॒य॒व इत्यै᳚न्द्र - वा॒य॒वः । यदै᳚न्द्रवाय॒वाग्राः᳚ । ऐ॒न्द॒वा॒य॒वाग्रा॒ ग्रहाः᳚ । ऐ॒न्द॒वा॒य॒वाग्रा॒ इत्यै᳚न्द्रवाय॒व - अ॒ग्राः॒ । ग्रहा॑ गृ॒ह्यन्ते᳚ । गृ॒ह्यन्ते॒ वाच᳚म् । वाच॑मे॒व । ए॒वानु॑ । अनु॒ प्र । प्र य॑न्ति । य॒न्ति॒ वा॒युम् । वा॒युम् दे॒वाः । दे॒वा अ॑ब्रुवन्न् । अ॒ब्रु॒व॒न्थ् सोम᳚म् । सोमꣳ॒॒ राजा॑नम् । राजा॑नꣳ हनाम । ह॒ना॒मेति॑ । इति॒ सः । सो᳚ऽब्रवीत् । अ॒ब्र॒वी॒द् वर᳚म् । वर॑म् ॅवृणै । वृ॒णै॒ मद॑ग्राः । मद॑ग्रा ए॒व । मद॑ग्रा॒ इति॒ मत् - अ॒ग्राः॒ । ए॒व वः॑ । वो॒ ग्रहाः᳚ । ग्रहा॑ गृह्यान्तै । गृ॒ह्या॒न्ता॒ इति॑ । इति॒ तस्मा᳚त् । तस्मा॑दैन्द्रवाय॒वाग्राः᳚ । ऐ॒न्द॒वा॒य॒वाग्रा॒ ग्रहाः᳚ । ऐ॒न्द॒वा॒य॒वाग्रा॒ इत्यै᳚न्द्रवाय॒व - अ॒ग्राः॒ । ग्रहा॑ गृह्यन्ते । गृ॒ह्य॒न्ते॒ तम् । तम॑घ्नन्न् । अ॒घ्न॒न्थ् सः । सो॑ऽपूयत् । अ॒पू॒य॒त् तम् । तम् दे॒वाः । दे॒वा न । नोप॑ । उपा॑धृष्णुवन्न् । अ॒धृ॒ष्णु॒व॒न् ते । ते वा॒युम् । वा॒युम॑ब्रुवन्न् । अ॒ब्रु॒व॒न्नि॒मम् । इ॒मम् नः॑ । नः॒ स्व॒द॒य॒ । स्व॒द॒येति॑ \newline

\textbf{Jatai Paata} \newline

1. वाग् वै वै वाग् वाग् वै । \newline
2. वा ए॒षैषा वै वा ए॒षा । \newline
3. ए॒षा यद् यदे॒षैषा यत् । \newline
4. यदै᳚न्द्रवाय॒व ऐ᳚न्द्रवाय॒वो यद् यदै᳚न्द्रवाय॒वः । \newline
5. ऐ॒न्द्र॒वा॒य॒वो यद् यदै᳚न्द्रवाय॒व ऐ᳚न्द्रवाय॒वो यत् । \newline
6. ऐ॒न्द्र॒वा॒य॒व इत्यै᳚न्द्र - वा॒य॒वः । \newline
7. यदै᳚न्द्रवाय॒वाग्रा॑ ऐन्द्रवाय॒वाग्रा॒ यद् यदै᳚न्द्रवाय॒वाग्राः᳚ । \newline
8. ऐ॒न्द्र॒वा॒य॒वाग्रा॒ ग्रहा॒ ग्रहा॑ ऐन्द्रवाय॒वाग्रा॑ ऐन्द्रवाय॒वाग्रा॒ ग्रहाः᳚ । \newline
9. ऐ॒न्द्र॒वा॒य॒वाग्रा॒ इत्यै᳚न्द्रवाय॒व - अ॒ग्राः॒ । \newline
10. ग्रहा॑ गृ॒ह्यन्ते॑ गृ॒ह्यन्ते॒ ग्रहा॒ ग्रहा॑ गृ॒ह्यन्ते᳚ । \newline
11. गृ॒ह्यन्ते॒ वाचं॒ ॅवाच॑म् गृ॒ह्यन्ते॑ गृ॒ह्यन्ते॒ वाच᳚म् । \newline
12. वाच॑ मे॒वैव वाचं॒ ॅवाच॑ मे॒व । \newline
13. ए॒वान् वन् वे॒वैवानु॑ । \newline
14. अनु॒ प्र प्राण् वनु॒ प्र । \newline
15. प्र य॑न्ति यन्ति॒ प्र प्र य॑न्ति । \newline
16. य॒न्ति॒ वा॒युं ॅवा॒युं ॅय॑न्ति यन्ति वा॒युम् । \newline
17. वा॒युम् दे॒वा दे॒वा वा॒युं ॅवा॒युम् दे॒वाः । \newline
18. दे॒वा अ॑ब्रुवन् नब्रुवन् दे॒वा दे॒वा अ॑ब्रुवन्न् । \newline
19. अ॒ब्रु॒व॒न् थ्सोमꣳ॒॒ सोम॑ मब्रुवन् नब्रुव॒न् थ्सोम᳚म् । \newline
20. सोमꣳ॒॒ राजा॑नꣳ॒॒ राजा॑नꣳ॒॒ सोमꣳ॒॒ सोमꣳ॒॒ राजा॑नम् । \newline
21. राजा॑नꣳ हनाम हनाम॒ राजा॑नꣳ॒॒ राजा॑नꣳ हनाम । \newline
22. ह॒ना॒मे तीति॑ हनाम हना॒ मेति॑ । \newline
23. इति॒ स स इतीति॒ सः । \newline
24. सो᳚ ऽब्रवी दब्रवी॒थ् स सो᳚ ऽब्रवीत् । \newline
25. अ॒ब्र॒वी॒द् वरं॒ ॅवर॑ मब्रवी दब्रवी॒द् वर᳚म् । \newline
26. वरं॑ ॅवृणै वृणै॒ वरं॒ ॅवरं॑ ॅवृणै । \newline
27. वृ॒णै॒ मद॑ग्रा॒ मद॑ग्रा वृणै वृणै॒ मद॑ग्राः । \newline
28. मद॑ग्रा ए॒वैव मद॑ग्रा॒ मद॑ग्रा ए॒व । \newline
29. मद॑ग्रा॒ इति॒ मत् - अ॒ग्राः॒ । \newline
30. ए॒व वो॑ व ए॒वैव वः॑ । \newline
31. वो॒ ग्रहा॒ ग्रहा॑ वो वो॒ ग्रहाः᳚ । \newline
32. ग्रहा॑ गृह्यान्तै गृह्यान्तै॒ ग्रहा॒ ग्रहा॑ गृह्यान्तै । \newline
33. गृ॒ह्या॒न्ता॒ इतीति॑ गृह्यान्तै गृह्यान्ता॒ इति॑ । \newline
34. इति॒ तस्मा॒त् तस्मा॒ दितीति॒ तस्मा᳚त् । \newline
35. तस्मा॑ दैन्द्रवाय॒वाग्रा॑ ऐन्द्रवाय॒वाग्रा॒ स्तस्मा॒त् तस्मा॑ दैन्द्रवाय॒वाग्राः᳚ । \newline
36. ऐ॒न्द्र॒वा॒य॒वाग्रा॒ ग्रहा॒ ग्रहा॑ ऐन्द्रवाय॒वाग्रा॑ ऐन्द्रवाय॒वाग्रा॒ ग्रहाः᳚ । \newline
37. ऐ॒न्द्र॒वा॒य॒वाग्रा॒ इत्यै᳚न्द्रवाय॒व - अ॒ग्राः॒ । \newline
38. ग्रहा॑ गृह्यन्ते गृह्यन्ते॒ ग्रहा॒ ग्रहा॑ गृह्यन्ते । \newline
39. गृ॒ह्य॒न्ते॒ तम् तम् गृ॑ह्यन्ते गृह्यन्ते॒ तम् । \newline
40. त म॑घ्नन् नघ्न॒न् तम् त म॑घ्नन्न् । \newline
41. अ॒घ्न॒न् थ्स सो᳚ ऽघ्नन् नघ्न॒न् थ्सः । \newline
42. सो॑ ऽपूय दपूय॒थ् स सो॑ ऽपूयत् । \newline
43. अ॒पू॒य॒त् तम् त म॑पूय दपूय॒त् तम् । \newline
44. तम् दे॒वा दे॒वा स्तम् तम् दे॒वाः । \newline
45. दे॒वा न न दे॒वा दे॒वा न । \newline
46. नोपोप॒ न नोप॑ । \newline
47. उपा॑ धृष्णुवन् नधृष्णुव॒न् नुपोपा॑ धृष्णुवन्न् । \newline
48. अ॒धृ॒ष्णु॒व॒न् ते ते॑ ऽधृष्णुवन् नधृष्णुव॒न् ते । \newline
49. ते वा॒युं ॅवा॒युम् ते ते वा॒युम् । \newline
50. वा॒यु म॑ब्रुवन् नब्रुवन्. वा॒युं ॅवा॒यु म॑ब्रुवन्न् । \newline
51. अ॒ब्रु॒व॒न् नि॒म मि॒म म॑ब्रुवन् नब्रुवन् नि॒मम् । \newline
52. इ॒मन् नो॑ न इ॒म मि॒मन् नः॑ । \newline
53. नः॒ स्व॒द॒य॒ स्व॒द॒य॒ नो॒ नः॒ स्व॒द॒य॒ । \newline
54. स्व॒द॒येतीति॑ स्वदय स्वद॒येति॑ । \newline

\textbf{Ghana Paata } \newline

1. वाग् वै वै वाग् वाग् वा ए॒षैषा वै वाग् वाग् वा ए॒षा । \newline
2. वा ए॒षैषा वै वा ए॒षा यद् यदे॒षा वै वा ए॒षा यत् । \newline
3. ए॒षा यद् यदे॒षैषा यदै᳚न्द्रवाय॒व ऐ᳚न्द्रवाय॒वो यदे॒षैषा यदै᳚न्द्रवाय॒वः । \newline
4. यदै᳚न्द्रवाय॒व ऐ᳚न्द्रवाय॒वो यद् यदै᳚न्द्रवाय॒वो यद् यदै᳚न्द्रवाय॒वो यद् यदै᳚न्द्रवाय॒वो यत् । \newline
5. ऐ॒न्द्र॒वा॒य॒वो यद् यदै᳚न्द्रवाय॒व ऐ᳚न्द्रवाय॒वो यदै᳚न्द्रवाय॒वाग्रा॑ ऐन्द्रवाय॒वाग्रा॒ 
यदै᳚न्द्रवाय॒व ऐ᳚न्द्रवाय॒वो यदै᳚न्द्रवाय॒वाग्राः᳚ । \newline
6. ऐ॒न्द्र॒वा॒य॒व इत्यै᳚न्द्र - वा॒य॒वः । \newline
7. यदै᳚न्द्रवाय॒वाग्रा॑ ऐन्द्रवाय॒वाग्रा॒ यद् यदै᳚न्द्रवाय॒वाग्रा॒ ग्रहा॒ ग्रहा॑ ऐन्द्रवाय॒वाग्रा॒ यद् यदै᳚न्द्रवाय॒वाग्रा॒ ग्रहाः᳚ । \newline
8. ऐ॒न्द्र॒वा॒य॒वाग्रा॒ ग्रहा॒ ग्रहा॑ ऐन्द्रवाय॒वाग्रा॑ ऐन्द्रवाय॒वाग्रा॒ ग्रहा॑ गृ॒ह्यन्ते॑ गृ॒ह्यन्ते॒ ग्रहा॑ ऐन्द्रवाय॒वाग्रा॑ ऐन्द्रवाय॒वाग्रा॒ ग्रहा॑ गृ॒ह्यन्ते᳚ । \newline
9. ऐ॒न्द्र॒वा॒य॒वाग्रा॒ इत्यै᳚न्द्रवाय॒व - अ॒ग्राः॒ । \newline
10. ग्रहा॑ गृ॒ह्यन्ते॑ गृ॒ह्यन्ते॒ ग्रहा॒ ग्रहा॑ गृ॒ह्यन्ते॒ वाचं॒ ॅवाच॑म् गृ॒ह्यन्ते॒ ग्रहा॒ ग्रहा॑ गृ॒ह्यन्ते॒ वाच᳚म् । \newline
11. गृ॒ह्यन्ते॒ वाचं॒ ॅवाच॑म् गृ॒ह्यन्ते॑ गृ॒ह्यन्ते॒ वाच॑ मे॒वैव वाच॑म् गृ॒ह्यन्ते॑ गृ॒ह्यन्ते॒ वाच॑ मे॒व । \newline
12. वाच॑ मे॒वैव वाचं॒ ॅवाच॑ मे॒वान् वन् वे॒व वाचं॒ ॅवाच॑ मे॒वानु॑ । \newline
13. ए॒वान् वन् वे॒ वैवानु॒ प्र प्राण्वे॒ वैवानु॒ प्र । \newline
14. अनु॒ प्र प्राण् वनु॒ प्र य॑न्ति यन्ति॒ प्राण् वनु॒ प्र य॑न्ति । \newline
15. प्र य॑न्ति यन्ति॒ प्र प्र य॑न्ति वा॒युं ॅवा॒युं ॅय॑न्ति॒ प्र प्र य॑न्ति वा॒युम् । \newline
16. य॒न्ति॒ वा॒युं ॅवा॒युं ॅय॑न्ति यन्ति वा॒युम् दे॒वा दे॒वा वा॒युं ॅय॑न्ति यन्ति वा॒युम् दे॒वाः । \newline
17. वा॒युम् दे॒वा दे॒वा वा॒युं ॅवा॒युम् दे॒वा अ॑ब्रुवन् नब्रुवन् दे॒वा वा॒युं ॅवा॒युम् दे॒वा अ॑ब्रुवन्न् । \newline
18. दे॒वा अ॑ब्रुवन् नब्रुवन् दे॒वा दे॒वा अ॑ब्रुव॒न् थ्सोमꣳ॒॒ सोम॑ मब्रुवन् दे॒वा दे॒वा अ॑ब्रुव॒न् थ्सोम᳚म् । \newline
19. अ॒ब्रु॒व॒न् थ्सोमꣳ॒॒ सोम॑ मब्रुवन् नब्रुव॒न् थ्सोमꣳ॒॒ राजा॑नꣳ॒॒ राजा॑नꣳ॒॒ सोम॑ मब्रुवन् नब्रुव॒न् थ्सोमꣳ॒॒ राजा॑नम् । \newline
20. सोमꣳ॒॒ राजा॑नꣳ॒॒ राजा॑नꣳ॒॒ सोमꣳ॒॒ सोमꣳ॒॒ राजा॑नꣳ हनाम हनाम॒ राजा॑नꣳ॒॒ सोमꣳ॒॒ सोमꣳ॒॒ राजा॑नꣳ हनाम । \newline
21. राजा॑नꣳ हनाम हनाम॒ राजा॑नꣳ॒॒ राजा॑नꣳ हना॒मेतीति॑ हनाम॒ राजा॑नꣳ॒॒ राजा॑नꣳ हना॒मेति॑ । \newline
22. ह॒ना॒मेतीति॑ हनाम हना॒मेति॒ स स इति॑ हनाम हना॒मेति॒ सः । \newline
23. इति॒ स स इतीति॒ सो᳚ ऽब्रवी दब्रवी॒थ् स इतीति॒ सो᳚ ऽब्रवीत् । \newline
24. सो᳚ ऽब्रवी दब्रवी॒थ् स सो᳚ ऽब्रवी॒द् वरं॒ ॅवर॑ मब्रवी॒थ् स सो᳚ ऽब्रवी॒द् वर᳚म् । \newline
25. अ॒ब्र॒वी॒द् वरं॒ ॅवर॑ मब्रवी दब्रवी॒द् वरं॑ ॅवृणै वृणै॒ वर॑ मब्रवी दब्रवी॒द् वरं॑ ॅवृणै । \newline
26. वरं॑ ॅवृणै वृणै॒ वरं॒ ॅवरं॑ ॅवृणै॒ मद॑ग्रा॒ मद॑ग्रा वृणै॒ वरं॒ ॅवरं॑ ॅवृणै॒ मद॑ग्राः । \newline
27. वृ॒णै॒ मद॑ग्रा॒ मद॑ग्रा वृणै वृणै॒ मद॑ग्रा ए॒वैव मद॑ग्रा वृणै वृणै॒ मद॑ग्रा ए॒व । \newline
28. मद॑ग्रा ए॒वैव मद॑ग्रा॒ मद॑ग्रा ए॒व वो॑ व ए॒व मद॑ग्रा॒ मद॑ग्रा ए॒व वः॑ । \newline
29. मद॑ग्रा॒ इति॒ मत् - अ॒ग्राः॒ । \newline
30. ए॒व वो॑ व ए॒वैव वो॒ ग्रहा॒ ग्रहा॑ व ए॒वैव वो॒ ग्रहाः᳚ । \newline
31. वो॒ ग्रहा॒ ग्रहा॑ वो वो॒ ग्रहा॑ गृह्यान्तै गृह्यान्तै॒ ग्रहा॑ वो वो॒ ग्रहा॑ गृह्यान्तै । \newline
32. ग्रहा॑ गृह्यान्तै गृह्यान्तै॒ ग्रहा॒ ग्रहा॑ गृह्यान्ता॒ इतीति॑ गृह्यान्तै॒ ग्रहा॒ ग्रहा॑ गृह्यान्ता॒ इति॑ । \newline
33. गृ॒ह्या॒न्ता॒ इतीति॑ गृह्यान्तै गृह्यान्ता॒ इति॒ तस्मा॒त् तस्मा॒ दिति॑ गृह्यान्तै गृह्यान्ता॒ इति॒ तस्मा᳚त् । \newline
34. इति॒ तस्मा॒त् तस्मा॒ दितीति॒ तस्मा॑ दैन्द्रवाय॒वाग्रा॑ ऐन्द्रवाय॒वाग्रा॒ स्तस्मा॒ दितीति॒ तस्मा॑ दैन्द्रवाय॒वाग्राः᳚ । \newline
35. तस्मा॑ दैन्द्रवाय॒वाग्रा॑ ऐन्द्रवाय॒वाग्रा॒ स्तस्मा॒त् तस्मा॑ दैन्द्रवाय॒वाग्रा॒ ग्रहा॒ ग्रहा॑ ऐन्द्रवाय॒वाग्रा॒ स्तस्मा॒त् तस्मा॑ दैन्द्रवाय॒वाग्रा॒ ग्रहाः᳚ । \newline
36. ऐ॒न्द्र॒वा॒य॒वाग्रा॒ ग्रहा॒ ग्रहा॑ ऐन्द्रवाय॒वाग्रा॑ ऐन्द्रवाय॒वाग्रा॒ ग्रहा॑ गृह्यन्ते गृह्यन्ते॒ ग्रहा॑ ऐन्द्रवाय॒वाग्रा॑ ऐन्द्रवाय॒वाग्रा॒ ग्रहा॑ गृह्यन्ते । \newline
37. ऐ॒न्द्र॒वा॒य॒वाग्रा॒ इत्यै᳚न्द्रवाय॒व - अ॒ग्राः॒ । \newline
38. ग्रहा॑ गृह्यन्ते गृह्यन्ते॒ ग्रहा॒ ग्रहा॑ गृह्यन्ते॒ तम् तम् गृ॑ह्यन्ते॒ ग्रहा॒ ग्रहा॑ गृह्यन्ते॒ तम् । \newline
39. गृ॒ह्य॒न्ते॒ तम् तम् गृ॑ह्यन्ते गृह्यन्ते॒ त म॑घ्नन् नघ्न॒न् तम् गृ॑ह्यन्ते गृह्यन्ते॒ त म॑घ्नन्न् । \newline
40. त म॑घ्नन् नघ्न॒न् तम् त म॑घ्न॒न् थ्स सो᳚ ऽघ्न॒न् तम् त म॑घ्न॒न् थ्सः । \newline
41. अ॒घ्न॒न् थ्स सो᳚ ऽघ्नन् नघ्न॒न् थ्सो॑ ऽपूय दपूय॒थ् सो᳚ ऽघ्नन् नघ्न॒न् थ्सो॑ ऽपूयत् । \newline
42. सो॑ ऽपूय दपूय॒थ् स सो॑ ऽपूय॒त् तम् त म॑पूय॒थ् स सो॑ ऽपूय॒त् तम् । \newline
43. अ॒पू॒य॒त् तम् त म॑पूय दपूय॒त् तम् दे॒वा दे॒वा स्त म॑पूय दपूय॒त् तम् दे॒वाः । \newline
44. तम् दे॒वा दे॒वा स्तम् तम् दे॒वा न न दे॒वा स्तम् तम् दे॒वा न । \newline
45. दे॒वा न न दे॒वा दे॒वा नोपोप॒ न दे॒वा दे॒वा नोप॑ । \newline
46. नोपोप॒ न नोपा॑ धृष्णुवन् नधृष्णुव॒न् नुप॒ न नोपा॑ धृष्णुवन्न् । \newline
47. उपा॑धृष्णुवन् नधृष्णुव॒न् नुपोपा॑ धृष्णुव॒न् ते ते॑ ऽधृष्णुव॒न् नुपोपा॑ धृष्णुव॒न् ते । \newline
48. अ॒धृ॒ष्णु॒व॒न् ते ते॑ ऽधृष्णुवन् नधृष्णुव॒न् ते वा॒युं ॅवा॒युम् ते॑ ऽधृष्णुवन् नधृष्णुव॒न् ते वा॒युम् । \newline
49. ते वा॒युं ॅवा॒युम् ते ते वा॒यु म॑ब्रुवन् नब्रुवन्. वा॒युम् ते ते वा॒यु म॑ब्रुवन्न् । \newline
50. वा॒यु म॑ब्रुवन् नब्रुवन्. वा॒युं ॅवा॒यु म॑ब्रुवन् नि॒म मि॒म म॑ब्रुवन्. वा॒युं ॅवा॒यु म॑ब्रुवन् नि॒मम् । \newline
51. अ॒ब्रु॒व॒न् नि॒म मि॒म म॑ब्रुवन् नब्रुवन् नि॒मन् नो॑ न इ॒म म॑ब्रुवन् नब्रुवन् नि॒मन् नः॑ । \newline
52. इ॒मन् नो॑ न इ॒म मि॒मन् नः॑ स्वदय स्वदय न इ॒म मि॒मन् नः॑ स्वदय । \newline
53. नः॒ स्व॒द॒य॒ स्व॒द॒य॒ नो॒ नः॒ स्व॒द॒येतीति॑ स्वदय नो नः स्वद॒येति॑ । \newline
54. स्व॒द॒येतीति॑ स्वदय स्वद॒येति॒ स स इति॑ स्वदय स्वद॒येति॒ सः । \newline
\pagebreak
\markright{ TS 6.4.7.2  \hfill https://www.vedavms.in \hfill}

\section{ TS 6.4.7.2 }

\textbf{TS 6.4.7.2 } \newline
\textbf{Samhita Paata} \newline

-ति॒ सो᳚ऽब्रवी॒द् वरं॑ ॅवृणै मद्देव॒त्या᳚न्ये॒व वः॒ पात्रा᳚ण्युच्यान्ता॒ इति॒ तस्मा᳚न्नानादेव॒त्या॑नि॒ सन्ति॑ वाय॒व्या᳚न्युच्यन्ते॒ तमे᳚भ्यो वा॒युरे॒वास्व॑दय॒त् तस्मा॒द्यत् पूय॑ति॒ तत् प्र॑वा॒ते वि ष॑जन्ति वा॒युर्.हि तस्य॑ पवयि॒ता स्व॑दयि॒ता तस्य॑ वि॒ग्रह॑णं॒ नावि॑न्द॒न्थ् सादि॑तिरब्रवी॒द् वरं॑ ॅवृणा॒ अथ॒ मया॒ वि गृ॑ह्णीद्ध्वं मद्देव॒त्या॑ ए॒व वः॒ सोमाः᳚ - [  ] \newline

\textbf{Pada Paata} \newline

इति॑ । सः । अ॒ब्र॒वी॒त् । वर᳚म् । वृ॒णै॒ । म॒द्दे॒व॒त्या॑नीति॑ मत्-दे॒व॒त्या॑नि । ए॒व । वः॒ । पात्रा॑णि । उ॒च्या॒न्तै॒ । इति॑ । तस्मा᳚त् । ना॒ना॒दे॒व॒त्या॑नीति॑ नाना - दे॒व॒त्या॑नि । सन्ति॑ । वा॒य॒व्या॑नि । उ॒च्य॒न्ते॒ । तम् । ए॒भ्यः॒ । वा॒युः । ए॒व । अ॒स्व॒द॒य॒त् । तस्मा᳚त् । यत् । पूय॑ति । तत् । प्र॒वा॒त इति॑ प्र - वा॒ते । वीति॑ । स॒ज॒न्ति॒ । वा॒युः । हि । तस्य॑ । प॒व॒यि॒ता । स्व॒द॒यि॒ता । तस्य॑ । वि॒ग्रह॑ण॒मिति॑ वि - ग्रह॑णम् । न । अ॒वि॒न्द॒न्न् । सा । अदि॑तिः । अ॒ब्र॒वी॒त् । वर᳚म् । वृ॒णै॒ । अथ॑ । मया᳚ । वीति॑ । गृ॒ह्णी॒द्ध्व॒म् । म॒द्दे॒व॒त्या॑ इति॑ मत् - दे॒व॒त्याः᳚ । ए॒व । वः॒ । सोमाः᳚ ।  \newline


\textbf{Krama Paata} \newline

इति॒ सः । सो᳚ऽब्रवीत् । अ॒ब्र॒वी॒द् वर᳚म् । वर॑म् ॅवृणै । वृ॒णै॒ म॒द्दे॒व॒त्या॑नि । म॒द्दे॒व॒त्या᳚न्ये॒व । म॒द्दे॒व॒त्या॑नीति॑ मत् - दे॒व॒त्या॑नि । ए॒व वः॑ । वः॒ पात्रा॑णि । पात्रा᳚ण्युच्यान्तै । उ॒च्या॒न्ता॒ इति॑ । इति॒ तस्मा᳚त् । तस्मा᳚न् नानादेव॒त्या॑नि । ना॒ना॒दे॒व॒त्या॑नि॒ सन्ति॑ । ना॒ना॒दे॒व॒त्या॑नीति॑ नाना - दे॒व॒त्या॑नि । सन्ति॑ वाय॒व्या॑नि । वा॒य॒व्या᳚न्युच्यन्ते । उ॒च्य॒न्ते॒ तम् । तमे᳚भ्यः । ए॒भ्यो॒ वा॒युः । वा॒युरे॒व । ए॒वास्व॑दयत् । अ॒स्व॒द॒य॒त् तस्मा᳚त् । तस्मा॒द् यत् । यत् पूय॑ति । पूय॑ति॒ तत् । तत् प्र॑वा॒ते । प्र॒वा॒ते वि । प्र॒वा॒त इति॑ प्र - वा॒ते । वि ष॑जन्ति । स॒ज॒न्ति॒ वा॒युः । वा॒युर्. हि । हि तस्य॑ । तस्य॑ पवयि॒ता । प॒व॒यि॒ता स्व॑दयि॒ता । स्व॒द॒यि॒ता तस्य॑ । तस्य॑ वि॒ग्रह॑णम् । वि॒ग्रह॑ण॒म् न । वि॒ग्रह॑ण॒मिति॑ वि - ग्रह॑णम् । नावि॑न्दन्न् । अ॒वि॒न्द॒न्थ् सा । साऽदि॑तिः । अदि॑तिरब्रवीत् । अ॒ब्र॒वी॒द् वर᳚म् । वर॑म् ॅवृणै । वृ॒णा॒ अथ॑ । अथ॒ मया᳚ । मया॒ वि । वि गृ॑ह्णीद्ध्वम् । गृ॒ह्णी॒द्ध्व॒म् म॒द्दे॒व॒त्याः᳚ । म॒द्दे॒व॒त्या॑ ए॒व । म॒द्दे॒व॒त्या॑ इति॑ मत् - दे॒व॒त्याः᳚ । ए॒व वः॑ । वः॒ सोमाः᳚ । सोमाः᳚ स॒न्नाः \newline

\textbf{Jatai Paata} \newline

1. इति॒ स स इतीति॒ सः । \newline
2. सो᳚ ऽब्रवी दब्रवी॒थ् स सो᳚ ऽब्रवीत् । \newline
3. अ॒ब्र॒वी॒द् वरं॒ ॅवर॑ मब्रवी दब्रवी॒द् वर᳚म् । \newline
4. वरं॑ ॅवृणै वृणै॒ वरं॒ ॅवरं॑ ॅवृणै । \newline
5. वृ॒णै॒ म॒द्दे॒व॒त्या॑नि मद्देव॒त्या॑नि वृणै वृणै मद्देव॒त्या॑नि । \newline
6. म॒द्दे॒व॒त्या᳚ न्ये॒वैव म॑द्देव॒त्या॑नि मद्देव॒त्या᳚ न्ये॒व । \newline
7. म॒द्दे॒व॒त्या॑नीति॑ मत् - दे॒व॒त्या॑नि । \newline
8. ए॒व वो॑ व ए॒वैव वः॑ । \newline
9. वः॒ पात्रा॑णि॒ पात्रा॑णि वो वः॒ पात्रा॑णि । \newline
10. पात्रा᳚ ण्युच्यान्ता उच्यान्तै॒ पात्रा॑णि॒ पात्रा᳚ ण्युच्यान्तै । \newline
11. उ॒च्या॒न्ता॒ इती त्यु॑च्यान्ता उच्यान्ता॒ इति॑ । \newline
12. इति॒ तस्मा॒त् तस्मा॒ दितीति॒ तस्मा᳚त् । \newline
13. तस्मा᳚न् नानादेव॒त्या॑नि नानादेव॒त्या॑नि॒ तस्मा॒त् तस्मा᳚न् नानादेव॒त्या॑नि । \newline
14. ना॒ना॒दे॒व॒त्या॑नि॒ सन्ति॒ सन्ति॑ नानादेव॒त्या॑नि नानादेव॒त्या॑नि॒ सन्ति॑ । \newline
15. ना॒ना॒दे॒व॒त्या॑नीति॑ नाना - दे॒व॒त्या॑नि । \newline
16. सन्ति॑ वाय॒व्या॑नि वाय॒व्या॑नि॒ सन्ति॒ सन्ति॑ वाय॒व्या॑नि । \newline
17. वा॒य॒व्या᳚ न्युच्यन्त उच्यन्ते वाय॒व्या॑नि वाय॒व्या᳚ न्युच्यन्ते । \newline
18. उ॒च्य॒न्ते॒ तम् त मु॑च्यन्त उच्यन्ते॒ तम् । \newline
19. त मे᳚भ्य एभ्य॒ स्तम् त मे᳚भ्यः । \newline
20. ए॒भ्यो॒ वा॒युर् वा॒यु रे᳚भ्य एभ्यो वा॒युः । \newline
21. वा॒यु रे॒वैव वा॒युर् वा॒यु रे॒व । \newline
22. ए॒वा स्व॑दय दस्वदय दे॒वै वास्व॑दयत् । \newline
23. अ॒स्व॒द॒य॒त् तस्मा॒त् तस्मा॑ दस्वदय दस्वदय॒त् तस्मा᳚त् । \newline
24. तस्मा॒द् यद् यत् तस्मा॒त् तस्मा॒द् यत् । \newline
25. यत् पूय॑ति॒ पूय॑ति॒ यद् यत् पूय॑ति । \newline
26. पूय॑ति॒ तत् तत् पूय॑ति॒ पूय॑ति॒ तत् । \newline
27. तत् प्र॑वा॒ते प्र॑वा॒ते तत् तत् प्र॑वा॒ते । \newline
28. प्र॒वा॒ते वि वि प्र॑वा॒ते प्र॑वा॒ते वि । \newline
29. प्र॒वा॒त इति॑ प्र - वा॒ते । \newline
30. वि ष॑जन्ति सजन्ति॒ वि वि ष॑जन्ति । \newline
31. स॒ज॒न्ति॒ वा॒युर् वा॒युः स॑जन्ति सजन्ति वा॒युः । \newline
32. वा॒युर्. हि हि वा॒युर् वा॒युर्. हि । \newline
33. हि तस्य॒ तस्य॒ हि हि तस्य॑ । \newline
34. तस्य॑ पवयि॒ता प॑वयि॒ता तस्य॒ तस्य॑ पवयि॒ता । \newline
35. प॒व॒यि॒ता स्व॑दयि॒ता स्व॑दयि॒ता प॑वयि॒ता प॑वयि॒ता स्व॑दयि॒ता । \newline
36. स्व॒द॒यि॒ता तस्य॒ तस्य॑ स्वदयि॒ता स्व॑दयि॒ता तस्य॑ । \newline
37. तस्य॑ वि॒ग्रह॑णं ॅवि॒ग्रह॑ण॒म् तस्य॒ तस्य॑ वि॒ग्रह॑णम् । \newline
38. वि॒ग्रह॑ण॒न् न न वि॒ग्रह॑णं ॅवि॒ग्रह॑ण॒न् न । \newline
39. वि॒ग्रह॑ण॒मिति॑ वि - ग्रह॑णम् । \newline
40. नावि॑न्दन् नविन्द॒न् न नावि॑न्दन्न् । \newline
41. अ॒वि॒न्द॒न् थ्सा सा ऽवि॑न्दन् नविन्द॒न् थ्सा । \newline
42. सा ऽदि॑ति॒ रदि॑तिः॒ सा सा ऽदि॑तिः । \newline
43. अदि॑ति रब्रवी दब्रवी॒ ददि॑ति॒ रदि॑ति रब्रवीत् । \newline
44. अ॒ब्र॒वी॒द् वरं॒ ॅवर॑ मब्रवी दब्रवी॒द् वर᳚म् । \newline
45. वरं॑ ॅवृणै वृणै॒ वरं॒ ॅवरं॑ ॅवृणै । \newline
46. वृ॒णा॒ अथाथ॑ वृणै वृणा॒ अथ॑ । \newline
47. अथ॒ मया॒ मया ऽथाथ॒ मया᳚ । \newline
48. मया॒ वि वि मया॒ मया॒ वि । \newline
49. वि गृ॑ह्णीद्ध्वम् गृह्णीद्ध्वं॒ ॅवि वि गृ॑ह्णीद्ध्वम् । \newline
50. गृ॒ह्णी॒द्ध्व॒म् म॒द्दे॒व॒त्या॑ मद्देव॒त्या॑ गृह्णीद्ध्वम् गृह्णीद्ध्वम् मद्देव॒त्याः᳚ । \newline
51. म॒द्दे॒व॒त्या॑ ए॒वैव म॑द्देव॒त्या॑ मद्देव॒त्या॑ ए॒व । \newline
52. म॒द्दे॒व॒त्या॑ इति॑ मत् - दे॒व॒त्याः᳚ । \newline
53. ए॒व वो॑ व ए॒वैव वः॑ । \newline
54. वः॒ सोमाः॒ सोमा॑ वो वः॒ सोमाः᳚ । \newline
55. सोमाः᳚ स॒न्नाः स॒न्नाः सोमाः॒ सोमाः᳚ स॒न्नाः । \newline

\textbf{Ghana Paata } \newline

1. इति॒ स स इतीति॒ सो᳚ ऽब्रवी दब्रवी॒थ् स इतीति॒ सो᳚ ऽब्रवीत् । \newline
2. सो᳚ ऽब्रवी दब्रवी॒थ् स सो᳚ ऽब्रवी॒द् वरं॒ ॅवर॑ मब्रवी॒थ् स सो᳚ ऽब्रवी॒द् वर᳚म् । \newline
3. अ॒ब्र॒वी॒द् वरं॒ ॅवर॑ मब्रवी दब्रवी॒द् वरं॑ ॅवृणै वृणै॒ वर॑ मब्रवी दब्रवी॒द् वरं॑ ॅवृणै । \newline
4. वरं॑ ॅवृणै वृणै॒ वरं॒ ॅवरं॑ ॅवृणै मद्देव॒त्या॑नि मद्देव॒त्या॑नि वृणै॒ वरं॒ ॅवरं॑ ॅवृणै मद्देव॒त्या॑नि । \newline
5. वृ॒णै॒ म॒द्दे॒व॒त्या॑नि मद्देव॒त्या॑नि वृणै वृणै मद्देव॒त्या᳚ न्ये॒वैव म॑द्देव॒त्या॑नि वृणै वृणै मद्देव॒त्या᳚ न्ये॒व । \newline
6. म॒द्दे॒व॒त्या᳚ न्ये॒वैव म॑द्देव॒त्या॑नि मद्देव॒त्या᳚ न्ये॒व वो॑ व ए॒व म॑द्देव॒त्या॑नि मद्देव॒त्या᳚ न्ये॒व वः॑ । \newline
7. म॒द्दे॒व॒त्या॑नीति॑ मत् - दे॒व॒त्या॑नि । \newline
8. ए॒व वो॑ व ए॒वैव वः॒ पात्रा॑णि॒ पात्रा॑णि व ए॒वैव वः॒ पात्रा॑णि । \newline
9. वः॒ पात्रा॑णि॒ पात्रा॑णि वो वः॒ पात्रा᳚ ण्युच्यान्ता उच्यान्तै॒ पात्रा॑णि वो वः॒ पात्रा᳚ ण्युच्यान्तै । \newline
10. पात्रा᳚ ण्युच्यान्ता उच्यान्तै॒ पात्रा॑णि॒ पात्रा᳚ ण्युच्यान्ता॒ इती त्यु॑च्यान्तै॒ पात्रा॑णि॒ पात्रा᳚ ण्युच्यान्ता॒ इति॑ । \newline
11. उ॒च्या॒न्ता॒ इती त्यु॑च्यान्ता उच्यान्ता॒ इति॒ तस्मा॒त् तस्मा॒ दित्यु॑च्यान्ता उच्यान्ता॒ इति॒ तस्मा᳚त् । \newline
12. इति॒ तस्मा॒त् तस्मा॒दि तीति॒ तस्मा᳚न् नानादेव॒त्या॑नि नानादेव॒त्या॑नि॒ तस्मा॒ दितीति॒ तस्मा᳚न् नानादेव॒त्या॑नि । \newline
13. तस्मा᳚न् नानादेव॒त्या॑नि नानादेव॒त्या॑नि॒ तस्मा॒त् तस्मा᳚न् नानादेव॒त्या॑नि॒ सन्ति॒ सन्ति॑ नानादेव॒त्या॑नि॒ तस्मा॒त् तस्मा᳚न् नानादेव॒त्या॑नि॒ सन्ति॑ । \newline
14. ना॒ना॒दे॒व॒त्या॑नि॒ सन्ति॒ सन्ति॑ नानादेव॒त्या॑नि नानादेव॒त्या॑नि॒ सन्ति॑ वाय॒व्या॑नि वाय॒व्या॑नि॒ सन्ति॑ नानादेव॒त्या॑नि नानादेव॒त्या॑नि॒ सन्ति॑ वाय॒व्या॑नि । \newline
15. ना॒ना॒दे॒व॒त्या॑नीति॑ नाना - दे॒व॒त्या॑नि । \newline
16. सन्ति॑ वाय॒व्या॑नि वाय॒व्या॑नि॒ सन्ति॒ सन्ति॑ वाय॒व्या᳚ न्युच्यन्त उच्यन्ते वाय॒व्या॑नि॒ सन्ति॒ सन्ति॑ वाय॒व्या᳚ न्युच्यन्ते । \newline
17. वा॒य॒व्या᳚ न्युच्यन्त उच्यन्ते वाय॒व्या॑नि वाय॒व्या᳚ न्युच्यन्ते॒ तम् त मु॑च्यन्ते वाय॒व्या॑नि वाय॒व्या᳚ न्युच्यन्ते॒ तम् । \newline
18. उ॒च्य॒न्ते॒ तम् त मु॑च्यन्त उच्यन्ते॒ त मे᳚भ्य एभ्य॒ स्त मु॑च्यन्त उच्यन्ते॒ त मे᳚भ्यः । \newline
19. त मे᳚भ्य एभ्य॒ स्तम् त मे᳚भ्यो वा॒युर् वा॒यु रे᳚भ्य॒ स्तम् त मे᳚भ्यो वा॒युः । \newline
20. ए॒भ्यो॒ वा॒युर् वा॒यु रे᳚भ्य एभ्यो वा॒यु रे॒वैव वा॒यु रे᳚भ्य एभ्यो वा॒यु रे॒व । \newline
21. वा॒यु रे॒वैव वा॒युर् वा॒यु रे॒वा स्व॑दय दस्वदय दे॒व वा॒युर् वा॒यु रे॒वा स्व॑दयत् । \newline
22. ए॒वा स्व॑दय दस्वदय दे॒वैवा स्व॑दय॒त् तस्मा॒त् तस्मा॑ दस्वदय दे॒वैवा स्व॑दय॒त् तस्मा᳚त् । \newline
23. अ॒स्व॒द॒य॒त् तस्मा॒त् तस्मा॑ दस्वदय दस्वदय॒त् तस्मा॒द् यद् यत् तस्मा॑ दस्वदय दस्वदय॒त् तस्मा॒द् यत् । \newline
24. तस्मा॒द् यद् यत् तस्मा॒त् तस्मा॒द् यत् पूय॑ति॒ पूय॑ति॒ यत् तस्मा॒त् तस्मा॒द् यत् पूय॑ति । \newline
25. यत् पूय॑ति॒ पूय॑ति॒ यद् यत् पूय॑ति॒ तत् तत् पूय॑ति॒ यद् यत् पूय॑ति॒ तत् । \newline
26. पूय॑ति॒ तत् तत् पूय॑ति॒ पूय॑ति॒ तत् प्र॑वा॒ते प्र॑वा॒ते तत् पूय॑ति॒ पूय॑ति॒ तत् प्र॑वा॒ते । \newline
27. तत् प्र॑वा॒ते प्र॑वा॒ते तत् तत् प्र॑वा॒ते वि वि प्र॑वा॒ते तत् तत् प्र॑वा॒ते वि । \newline
28. प्र॒वा॒ते वि वि प्र॑वा॒ते प्र॑वा॒ते वि ष॑जन्ति सजन्ति॒ वि प्र॑वा॒ते प्र॑वा॒ते वि ष॑जन्ति । \newline
29. प्र॒वा॒त इति॑ प्र - वा॒ते । \newline
30. वि ष॑जन्ति सजन्ति॒ वि वि ष॑जन्ति वा॒युर् वा॒युः स॑जन्ति॒ वि वि ष॑जन्ति वा॒युः । \newline
31. स॒ज॒न्ति॒ वा॒युर् वा॒युः स॑जन्ति सजन्ति वा॒युर्. हि हि वा॒युः स॑जन्ति सजन्ति वा॒युर्. हि । \newline
32. वा॒युर्. हि हि वा॒युर् वा॒युर्. हि तस्य॒ तस्य॒ हि वा॒युर् वा॒युर्. हि तस्य॑ । \newline
33. हि तस्य॒ तस्य॒ हि हि तस्य॑ पवयि॒ता प॑वयि॒ता तस्य॒ हि हि तस्य॑ पवयि॒ता । \newline
34. तस्य॑ पवयि॒ता प॑वयि॒ता तस्य॒ तस्य॑ पवयि॒ता स्व॑दयि॒ता स्व॑दयि॒ता प॑वयि॒ता तस्य॒ तस्य॑ पवयि॒ता स्व॑दयि॒ता । \newline
35. प॒व॒यि॒ता स्व॑दयि॒ता स्व॑दयि॒ता प॑वयि॒ता प॑वयि॒ता स्व॑दयि॒ता तस्य॒ तस्य॑ स्वदयि॒ता प॑वयि॒ता प॑वयि॒ता स्व॑दयि॒ता तस्य॑ । \newline
36. स्व॒द॒यि॒ता तस्य॒ तस्य॑ स्वदयि॒ता स्व॑दयि॒ता तस्य॑ वि॒ग्रह॑णं ॅवि॒ग्रह॑ण॒म् तस्य॑ स्वदयि॒ता स्व॑दयि॒ता तस्य॑ वि॒ग्रह॑णम् । \newline
37. तस्य॑ वि॒ग्रह॑णं ॅवि॒ग्रह॑ण॒म् तस्य॒ तस्य॑ वि॒ग्रह॑ण॒न् न न वि॒ग्रह॑ण॒म् तस्य॒ तस्य॑ वि॒ग्रह॑ण॒न् न । \newline
38. वि॒ग्रह॑ण॒न् न न वि॒ग्रह॑णं ॅवि॒ग्रह॑ण॒न् नावि॑न्दन् नविन्द॒न् न वि॒ग्रह॑णं ॅवि॒ग्रह॑ण॒न् नावि॑न्दन्न् । \newline
39. वि॒ग्रह॑ण॒मिति॑ वि - ग्रह॑णम् । \newline
40. नावि॑न्दन् नविन्द॒न् न नावि॑न्द॒न् थ्सा सा ऽवि॑न्द॒न् न नावि॑न्द॒न् थ्सा । \newline
41. अ॒वि॒न्द॒न् थ्सा सा ऽवि॑न्दन् नविन्द॒न् थ्सा ऽदि॑ति॒ रदि॑तिः॒ सा ऽवि॑न्दन् नविन्द॒न् थ्सा ऽदि॑तिः । \newline
42. सा ऽदि॑ति॒ रदि॑तिः॒ सा सा ऽदि॑ति रब्रवी दब्रवी॒ ददि॑तिः॒ सा सा ऽदि॑ति रब्रवीत् । \newline
43. अदि॑ति रब्रवी दब्रवी॒ ददि॑ति॒ रदि॑ति रब्रवी॒द् वरं॒ ॅवर॑ मब्रवी॒ ददि॑ति॒ रदि॑ति रब्रवी॒द् वर᳚म् । \newline
44. अ॒ब्र॒वी॒द् वरं॒ ॅवर॑ मब्रवी दब्रवी॒द् वरं॑ ॅवृणै वृणै॒ वर॑ मब्रवी दब्रवी॒द् वरं॑ ॅवृणै । \newline
45. वरं॑ ॅवृणै वृणै॒ वरं॒ ॅवरं॑ ॅवृणा॒ अथाथ॑ वृणै॒ वरं॒ ॅवरं॑ ॅवृणा॒ अथ॑ । \newline
46. वृ॒णा॒ अथाथ॑ वृणै वृणा॒ अथ॒ मया॒ मया ऽथ॑ वृणै वृणा॒ अथ॒ मया᳚ । \newline
47. अथ॒ मया॒ मया ऽथाथ॒ मया॒ वि वि मया ऽथाथ॒ मया॒ वि । \newline
48. मया॒ वि वि मया॒ मया॒ वि गृ॑ह्णीद्ध्वम् गृह्णीद्ध्वं॒ ॅवि मया॒ मया॒ वि गृ॑ह्णीद्ध्वम् । \newline
49. वि गृ॑ह्णीद्ध्वम् गृह्णीद्ध्वं॒ ॅवि वि गृ॑ह्णीद्ध्वम् मद्देव॒त्या॑ मद्देव॒त्या॑ गृह्णीद्ध्वं॒ ॅवि वि गृ॑ह्णीद्ध्वम् मद्देव॒त्याः᳚ । \newline
50. गृ॒ह्णी॒द्ध्व॒म् म॒द्दे॒व॒त्या॑ मद्देव॒त्या॑ गृह्णीद्ध्वम् गृह्णीद्ध्वम् मद्देव॒त्या॑ ए॒वैव म॑द्देव॒त्या॑ गृह्णीद्ध्वम् गृह्णीद्ध्वम् मद्देव॒त्या॑ ए॒व । \newline
51. म॒द्दे॒व॒त्या॑ ए॒वैव म॑द्देव॒त्या॑ मद्देव॒त्या॑ ए॒व वो॑ व ए॒व म॑द्देव॒त्या॑ मद्देव॒त्या॑ ए॒व वः॑ । \newline
52. म॒द्दे॒व॒त्या॑ इति॑ मत् - दे॒व॒त्याः᳚ । \newline
53. ए॒व वो॑ व ए॒वैव वः॒ सोमाः॒ सोमा॑ व ए॒वैव वः॒ सोमाः᳚ । \newline
54. वः॒ सोमाः॒ सोमा॑ वो वः॒ सोमाः᳚ स॒न्नाः स॒न्नाः सोमा॑ वो वः॒ सोमाः᳚ स॒न्नाः । \newline
55. सोमाः᳚ स॒न्नाः स॒न्नाः सोमाः॒ सोमाः᳚ स॒न्ना अ॑सन् नसन् थ्स॒न्नाः सोमाः॒ सोमाः᳚ स॒न्ना अ॑सन्न् । \newline
\pagebreak
\markright{ TS 6.4.7.3  \hfill https://www.vedavms.in \hfill}

\section{ TS 6.4.7.3 }

\textbf{TS 6.4.7.3 } \newline
\textbf{Samhita Paata} \newline

स॒न्ना अ॑स॒-न्नित्यु॑पया॒मगृ॑हीतो॒ऽसी-त्या॑हा-दितिदेव॒त्या᳚स्तेन॒ यानि॒ हि दा॑रु॒मया॑णि॒ पात्रा᳚ण्य॒स्यै तानि॒ योनेः॒ संभू॑तानि॒ यानि॑ मृ॒न्मया॑नि सा॒क्षात् तान्य॒स्यै तस्मा॑दे॒वमा॑ह॒ वाग्वै परा॒च्य-व्या॑कृताऽवद॒त् ते दे॒वा इन्द्र॑मब्रुवन्नि॒मां नो॒ वाचं॒ ॅव्याकु॒र्विति॒ सो᳚ऽब्रवी॒द् वरं॑ ॅवृणै॒ मह्यं॑ चै॒वैष वा॒यवे॑ च स॒ह ( ) गृ॑ह्याता॒ इति॒ तस्मा॑दैन्द्रवाय॒वः स॒ह गृ॑ह्यते॒ तामिन्द्रो॑ मद्ध्य॒तो॑ऽव॒क्रम्य॒ व्याक॑रो॒त् तस्मा॑दि॒यं ॅव्याकृ॑ता॒ वागु॑द्यते॒ तस्मा᳚थ् स॒कृदिन्द्रा॑य मद्ध्य॒तो गृ॑ह्यते॒ द्विर्वा॒यवे॒ द्वौ हि स वरा॒ववृ॑णीत ॥ \newline

\textbf{Pada Paata} \newline

स॒न्नाः । अ॒स॒न्न् । इति॑ । उ॒प॒या॒मगृ॑हीत॒ इत्यु॑पया॒म-गृ॒ही॒तः॒ । अ॒सि॒ । इति॑ । आ॒ह॒ । अ॒दि॒ति॒दे॒व॒त्या॑ इत्य॑दिति - दे॒व॒त्याः᳚ । तेन॑ । यानि॑ । हि । दा॒रु॒मया॒णीति॑ दारु-मया॑नि । पात्रा॑णि । अ॒स्यै । तानि॑ । योनेः᳚ । संभू॑ता॒नीति॒ सं - भू॒ता॒नि॒ । यानि॑ । मृ॒न्मया॒नीति॑ मृत् - मया॑नि । सा॒क्षादिति॑ स - अ॒क्षात् । तानि॑ । अ॒स्यै । तस्मा᳚त् । ए॒वम् । आ॒ह॒ । वाक् । वै । परा॑ची । अव्या॑कृ॒तेत्यवि॑ - आ॒कृ॒ता॒ । अ॒व॒द॒त् । ते । दे॒वाः । इन्द्र᳚म् । अ॒ब्रु॒व॒न्न् । इ॒माम् । नः॒ । वाच᳚म् । व्याकु॒र्विति॑ वि - आकु॑रु । इति॑ । सः । अ॒ब्र॒वी॒त् । वर᳚म् । वृ॒णै॒ । मह्य᳚म् । च॒ । ए॒व । ए॒षः । वा॒यवे᳚ । च॒ । स॒ह ( ) । गृ॒ह्या॒तै॒ । इति॑ । तस्मा᳚त् । ऐ॒न्द्र॒वा॒य॒व इत्यै᳚न्द्र - वा॒य॒वः । स॒ह । गृ॒ह्य॒ते॒ । ताम् । इन्द्रः॑ । म॒द्ध्य॒तः । अ॒व॒क्रम्येत्य॑व - क्रम्य॑ । व्याक॑रो॒दिति॑ वि - आक॑रोत् । तस्मा᳚त् । इ॒यम् । व्याकृ॒तेति॑ वि-आकृ॑ता । वाक् । उ॒द्य॒ते॒ । तस्मा᳚त् । स॒कृत् । इन्द्रा॑य । म॒द्ध्य॒तः । गृ॒ह्य॒ते॒ । द्विः । वा॒यवे᳚ । द्वौ । हि । सः । वरौ᳚ । अवृ॑णीत ॥  \newline


\textbf{Krama Paata} \newline

स॒न्ना अ॑सन्न् । अ॒स॒न्निति॑ । इत्यु॑पया॒मगृ॑हीतः । उ॒प॒या॒मगृ॑हीतोऽसि । उ॒प॒या॒मगृ॑हीत॒ इत्यु॑पया॒म - गृ॒ही॒तः॒ । अ॒सीति॑ । इत्या॑ह । आ॒हा॒दि॒ति॒दे॒व॒त्याः᳚ । अ॒दि॒ति॒दे॒व॒त्या᳚स्तेन॑ । अ॒दि॒ति॒दे॒व॒त्या॑ इत्य॑दिति - दे॒व॒त्याः᳚ । तेन॒ यानि॑ । यानि॒ हि । हि दा॑रु॒मया॑णि । दा॒रु॒मया॑णि॒ पात्रा॑णि । दा॒रु॒मया॒णीति॑ दारु - मया॑नि । पात्रा᳚ण्य॒स्यै । अ॒स्यै तानि॑ । तानि॒ योनेः᳚ । योनेः॒ सम्भू॑तानि । सम्भू॑तानि॒ यानि॑ । सम्भू॑ता॒नीति॒ सम् - भू॒ता॒नि॒ । यानि॑ मृ॒न्मया॑नि । मृ॒न्मया॑नि सा॒क्षात् । मृ॒न्मया॒नीति॑ मृत् - मया॑नि । सा॒क्षात् तानि॑ । सा॒क्षादिति॑ स - अ॒क्षात् । तान्य॒स्यै । अ॒स्यै तस्मा᳚त् । तस्मा॑दे॒वम् । ए॒वमा॑ह । आ॒ह॒ वाक् । वाग् वै । वै परा॑ची । परा॒च्यव्या॑कृता । अव्या॑कृताऽवदत् । अव्या॑कृ॒तेत्यवि॑ - आ॒कृ॒ता॒ । अ॒व॒द॒त् ते । ते दे॒वाः । दे॒वा इन्द्र᳚म् । इन्द्र॑मब्रुवन्न् । अ॒ब्रु॒व॒न्नि॒माम् । इ॒माम् नः॑ । नो॒ वाच᳚म् । वाच॒म् ॅव्याकु॑रु । व्याकु॒र्विति॑ । व्याकु॒र्विति॑ वि - आकु॑रु । इति॒ सः । सो᳚ऽब्रवीत् । अ॒ब्र॒वी॒द् वर᳚म् । वर॑म् ॅवृणै । वृ॒णै॒ मह्य᳚म् । मह्य॑म् च । चै॒व । ए॒वैषः । ए॒ष वा॒यवे᳚ । वा॒यवे॑ च । च॒ स॒ह ( ) । स॒ह गृ॑ह्यातै । गृ॒ह्या॒ता॒ इति॑ । इति॒ तस्मा᳚त् । तस्मा॑दैन्द्रवाय॒वः । ऐ॒न्द्र॒वा॒य॒वः स॒ह । ऐ॒न्द्र॒वा॒य॒व इत्यै᳚न्द्र - वा॒य॒वः । स॒ह गृ॑ह्यते । गृ॒ह्य॒ते॒ ताम् । तामिन्द्रः॑ । इन्द्रो॑ मद्ध्य॒तः । म॒द्ध्य॒तो॑ऽव॒क्रम्य॑ । अ॒व॒क्रम्य॒ व्याक॑रोत् । अ॒व॒क्रम्येत्य॑व - क्रम्य॑ । व्याक॑रो॒त् तस्मा᳚त् । व्याक॑रो॒दिति॑ वि - आक॑रोत् । तस्मा॑दि॒यम् । इ॒यम् ॅव्याकृ॑ता । व्याकृ॑ता॒ वाक् । व्याकृ॒तेति॑ वि - आकृ॑ता । वागु॑द्यते । उ॒द्य॒ते॒ तस्मा᳚त् । तस्मा᳚थ् स॒कृत् । स॒कृदिन्द्रा॑य । इन्द्रा॑य मद्ध्य॒तः । म॒द्ध्य॒तो गृ॑ह्यते । गृ॒ह्य॒ते॒ द्विः । द्विर् वा॒यवे᳚ । वा॒यवे॒ द्वौ । द्वौ हि । हि सः । स वरौ᳚ । वरा॒ववृ॑णीत । अवृ॑णी॒तेत्यवृ॑णीत । \newline

\textbf{Jatai Paata} \newline

1. स॒न्ना अ॑सन् नसन् थ्स॒न्नाः स॒न्ना अ॑सन्न् । \newline
2. अ॒स॒न् निती त्य॑सन् नस॒न् निति॑ । \newline
3. इत्यु॑पया॒मगृ॑हीत उपया॒मगृ॑हीत॒ इती त्यु॑पया॒मगृ॑हीतः । \newline
4. उ॒प॒या॒मगृ॑हीतो ऽस्य स्युपया॒मगृ॑हीत उपया॒मगृ॑हीतो ऽसि । \newline
5. उ॒प॒या॒मगृ॑हीत॒ इत्यु॑पया॒म - गृ॒ही॒तः॒ । \newline
6. अ॒सीती त्य॑स्य॒ सीति॑ । \newline
7. इत्या॑हा॒हे तीत्या॑ह । \newline
8. आ॒हा॒ दि॒ति॒दे॒व॒त्या॑ अदितिदेव॒त्या॑ आहाहा दितिदेव॒त्याः᳚ । \newline
9. अ॒दि॒ति॒दे॒व॒त्या᳚ स्तेन॒ तेना॑ दितिदेव॒त्या॑ अदितिदेव॒त्या᳚ स्तेन॑ । \newline
10. अ॒दि॒ति॒दे॒व॒त्या॑ इत्य॑दिति - दे॒व॒त्याः᳚ । \newline
11. तेन॒ यानि॒ यानि॒ तेन॒ तेन॒ यानि॑ । \newline
12. यानि॒ हि हि यानि॒ यानि॒ हि । \newline
13. हि दा॑रु॒मया॑णि दारु॒मया॑णि॒ हि हि दा॑रु॒मया॑णि । \newline
14. दा॒रु॒मया॑णि॒ पात्रा॑णि॒ पात्रा॑णि दारु॒मया॑णि दारु॒मया॑णि॒ पात्रा॑णि । \newline
15. दा॒रु॒मया॒णीति॑ दारु - मया॑नि । \newline
16. पात्रा᳚ ण्य॒स्या अ॒स्यै पात्रा॑णि॒ पात्रा᳚ ण्य॒स्यै । \newline
17. अ॒स्यै तानि॒ तान्य॒स्या अ॒स्यै तानि॑ । \newline
18. तानि॒ योने॒र् योने॒ स्तानि॒ तानि॒ योनेः᳚ । \newline
19. योनेः॒ संभू॑तानि॒ संभू॑तानि॒ योने॒र् योनेः॒ संभू॑तानि । \newline
20. संभू॑तानि॒ यानि॒ यानि॒ संभू॑तानि॒ संभू॑तानि॒ यानि॑ । \newline
21. संभू॑ता॒नीति॒ सं - भू॒ता॒नि॒ । \newline
22. यानि॑ मृ॒न्मया॑नि मृ॒न्मया॑नि॒ यानि॒ यानि॑ मृ॒न्मया॑नि । \newline
23. मृ॒न्मया॑नि सा॒क्षाथ् सा॒क्षान् मृ॒न्मया॑नि मृ॒न्मया॑नि सा॒क्षात् । \newline
24. मृ॒न्मया॒नीति॑ मृत् - मया॑नि । \newline
25. सा॒क्षात् तानि॒ तानि॑ सा॒क्षाथ् सा॒क्षात् तानि॑ । \newline
26. सा॒क्षादिति॑ स - अ॒क्षात् । \newline
27. तान्य॒स्या अ॒स्यै तानि॒ तान्य॒स्यै । \newline
28. अ॒स्यै तस्मा॒त् तस्मा॑ द॒स्या अ॒स्यै तस्मा᳚त् । \newline
29. तस्मा॑ दे॒व मे॒वम् तस्मा॒त् तस्मा॑ दे॒वम् । \newline
30. ए॒व मा॑हा है॒व मे॒व मा॑ह । \newline
31. आ॒ह॒ वाग् वागा॑ हाह॒ वाक् । \newline
32. वाग् वै वै वाग् वाग् वै । \newline
33. वै परा॑ची॒ परा॑ची॒ वै वै परा॑ची । \newline
34. परा॒ च्यव्या॑कृ॒ता ऽव्या॑कृता॒ परा॑ची॒ परा॒ च्यव्या॑कृता । \newline
35. अव्या॑कृता ऽवद दवद॒ दव्या॑कृ॒ता ऽव्या॑कृता ऽवदत् । \newline
36. अव्या॑कृ॒तेत्यवि॑ - आ॒कृ॒ता॒ । \newline
37. अ॒व॒द॒त् ते ते॑ ऽवद दवद॒त् ते । \newline
38. ते दे॒वा दे॒वा स्ते ते दे॒वाः । \newline
39. दे॒वा इन्द्र॒ मिन्द्र॑म् दे॒वा दे॒वा इन्द्र᳚म् । \newline
40. इन्द्र॑ मब्रुवन् नब्रुव॒न् निन्द्र॒ मिन्द्र॑ मब्रुवन्न् । \newline
41. अ॒ब्रु॒व॒न् नि॒मा मि॒मा म॑ब्रुवन् नब्रुवन् नि॒माम् । \newline
42. इ॒मान् नो॑ न इ॒मा मि॒मान् नः॑ । \newline
43. नो॒ वाचं॒ ॅवाच॑न् नो नो॒ वाच᳚म् । \newline
44. वाचं॒ ॅव्याकु॑रु॒ व्याकु॑रु॒ वाचं॒ ॅवाचं॒ ॅव्याकु॑रु । \newline
45. व्याकु॒र् वितीति॒ व्याकु॑रु॒ व्याकु॒र् विति॑ । \newline
46. व्याकु॒र्विति॑ वि - आकु॑रु । \newline
47. इति॒ स स इतीति॒ सः । \newline
48. सो᳚ ऽब्रवी दब्रवी॒थ् स सो᳚ ऽब्रवीत् । \newline
49. अ॒ब्र॒वी॒द् वरं॒ ॅवर॑ मब्रवी दब्रवी॒द् वर᳚म् । \newline
50. वरं॑ ॅवृणै वृणै॒ वरं॒ ॅवरं॑ ॅवृणै । \newline
51. वृ॒णै॒ मह्य॒म् मह्यं॑ ॅवृणै वृणै॒ मह्य᳚म् । \newline
52. मह्य॑म् च च॒ मह्य॒म् मह्य॑म् च । \newline
53. चै॒वैव च॑ चै॒व । \newline
54. ए॒वैष ए॒ष ए॒वै वैषः । \newline
55. ए॒ष वा॒यवे॑ वा॒यव॑ ए॒ष ए॒ष वा॒यवे᳚ । \newline
56. वा॒यवे॑ च च वा॒यवे॑ वा॒यवे॑ च । \newline
57. च॒ स॒ह स॒ह च॑ च स॒ह । \newline
58. स॒ह गृ॑ह्यातै गृह्यातै स॒ह स॒ह गृ॑ह्यातै । \newline
59. गृ॒ह्या॒ता॒ इतीति॑ गृह्यातै गृह्याता॒ इति॑ । \newline
60. इति॒ तस्मा॒त् तस्मा॒ दितीति॒ तस्मा᳚त् । \newline
61. तस्मा॑ दैन्द्रवाय॒व ऐ᳚न्द्रवाय॒व स्तस्मा॒त् तस्मा॑ दैन्द्रवाय॒वः । \newline
62. ऐ॒न्द्र॒वा॒य॒वः स॒ह स॒हैन्द्र॑वाय॒व ऐ᳚न्द्रवाय॒वः स॒ह । \newline
63. ऐ॒न्द्र॒वा॒य॒व इत्यै᳚न्द्र - वा॒य॒वः । \newline
64. स॒ह गृ॑ह्यते गृह्यते स॒ह स॒ह गृ॑ह्यते । \newline
65. गृ॒ह्य॒ते॒ ताम् ताम् गृ॑ह्यते गृह्यते॒ ताम् । \newline
66. ता मिन्द्र॒ इन्द्र॒ स्ताम् ता मिन्द्रः॑ । \newline
67. इन्द्रो॑ मद्ध्य॒तो म॑द्ध्य॒त इन्द्र॒ इन्द्रो॑ मद्ध्य॒तः । \newline
68. म॒द्ध्य॒तो॑ ऽव॒क्रम्या॑ व॒क्रम्य॑ मद्ध्य॒तो म॑द्ध्य॒तो॑ ऽव॒क्रम्य॑ । \newline
69. अ॒व॒क्रम्य॒ व्याक॑रो॒द् व्याक॑रो दव॒क्रम्या॑ व॒क्रम्य॒ व्याक॑रोत् । \newline
70. अ॒व॒क्रम्येत्य॑व - क्रम्य॑ । \newline
71. व्याक॑रो॒त् तस्मा॒त् तस्मा॒द् व्याक॑रो॒द् व्याक॑रो॒त् तस्मा᳚त् । \newline
72. व्याक॑रो॒दिति॑ वि - आक॑रोत् । \newline
73. तस्मा॑ दि॒य मि॒यम् तस्मा॒त् तस्मा॑ दि॒यम् । \newline
74. इ॒यं ॅव्याकृ॑ता॒ व्याकृ॑ते॒य मि॒यं ॅव्याकृ॑ता । \newline
75. व्याकृ॑ता॒ वाग् वाग् व्याकृ॑ता॒ व्याकृ॑ता॒ वाक् । \newline
76. व्याकृ॒तेति॑ वि - आकृ॑ता । \newline
77. वागु॑द्यत उद्यते॒ वाग् वागु॑द्यते । \newline
78. उ॒द्य॒ते॒ तस्मा॒त् तस्मा॑ दुद्यत उद्यते॒ तस्मा᳚त् । \newline
79. तस्मा᳚थ् स॒कृथ् स॒कृत् तस्मा॒त् तस्मा᳚थ् स॒कृत् । \newline
80. स॒कृ दिन्द्रा॒ येन्द्रा॑य स॒कृथ् स॒कृ दिन्द्रा॑य । \newline
81. इन्द्रा॑य मद्ध्य॒तो म॑द्ध्य॒त इन्द्रा॒ येन्द्रा॑य मद्ध्य॒तः । \newline
82. म॒द्ध्य॒तो गृ॑ह्यते गृह्यते मद्ध्य॒तो म॑द्ध्य॒तो गृ॑ह्यते । \newline
83. गृ॒ह्य॒ते॒ द्विर् द्विर् गृ॑ह्यते गृह्यते॒ द्विः । \newline
84. द्विर् वा॒यवे॑ वा॒यवे॒ द्विर् द्विर् वा॒यवे᳚ । \newline
85. वा॒यवे॒ द्वौ द्वौ वा॒यवे॑ वा॒यवे॒ द्वौ । \newline
86. द्वौ हि हि द्वौ द्वौ हि । \newline
87. हि स स हि हि सः । \newline
88. स वरौ॒ वरौ॒ स स वरौ᳚ । \newline
89. वरा॒ ववृ॑णी॒ता वृ॑णीत॒ वरौ॒ वरा॒ ववृ॑णीत । \newline
90. अवृ॑णी॒तेत्यवृ॑णीत । \newline

\textbf{Ghana Paata } \newline

1. स॒न्ना अ॑सन् नसन् थ्स॒न्नाः स॒न्ना अ॑स॒न् नितीत्य॑सन् थ्स॒न्नाः स॒न्ना अ॑स॒न् निति॑ । \newline
2. अ॒स॒न् नितीत्य॑सन् नस॒न् नित्यु॑पया॒मगृ॑हीत उपया॒मगृ॑हीत॒ इत्य॑सन् नस॒न् नित्यु॑पया॒मगृ॑हीतः । \newline
3. इत्यु॑पया॒मगृ॑हीत उपया॒मगृ॑हीत॒ इतीत्यु॑पया॒मगृ॑हीतो ऽस्य स्युपया॒मगृ॑हीत॒ इतीत्यु॑पया॒मगृ॑हीतो ऽसि । \newline
4. उ॒प॒या॒मगृ॑हीतो ऽस्य स्युपया॒मगृ॑हीत उपया॒मगृ॑हीतो॒ ऽसीती त्य॑स्युपया॒मगृ॑हीत उपया॒मगृ॑हीतो॒ ऽसीति॑ । \newline
5. उ॒प॒या॒मगृ॑हीत॒ इत्यु॑पया॒म - गृ॒ही॒तः॒ । \newline
6. अ॒सीती त्य॑स्य॒ सीत्या॑ हा॒हे त्य॑स्य॒ सीत्या॑ह । \newline
7. इत्या॑हा॒हे तीत्या॑हा दितिदेव॒त्या॑ अदितिदेव॒त्या॑ आ॒हे तीत्या॑ हादितिदेव॒त्याः᳚ । \newline
8. आ॒हा॒ दि॒ति॒दे॒व॒त्या॑ अदितिदेव॒त्या॑ आहाहा दितिदेव॒त्या᳚ स्तेन॒ तेना॑ दितिदेव॒त्या॑ आहाहा दितिदेव॒त्या᳚ स्तेन॑ । \newline
9. अ॒दि॒ति॒दे॒व॒त्या᳚ स्तेन॒ तेना॑ दितिदेव॒त्या॑ अदितिदेव॒त्या᳚ स्तेन॒ यानि॒ यानि॒ तेना॑ दितिदेव॒त्या॑ अदितिदेव॒त्या᳚ स्तेन॒ यानि॑ । \newline
10. अ॒दि॒ति॒दे॒व॒त्या॑ इत्य॑दिति - दे॒व॒त्याः᳚ । \newline
11. तेन॒ यानि॒ यानि॒ तेन॒ तेन॒ यानि॒ हि हि यानि॒ तेन॒ तेन॒ यानि॒ हि । \newline
12. यानि॒ हि हि यानि॒ यानि॒ हि दा॑रु॒मया॑णि दारु॒मया॑णि॒ हि यानि॒ यानि॒ हि दा॑रु॒मया॑णि । \newline
13. हि दा॑रु॒मया॑णि दारु॒मया॑णि॒ हि हि दा॑रु॒मया॑णि॒ पात्रा॑णि॒ पात्रा॑णि दारु॒मया॑णि॒ हि हि दा॑रु॒मया॑णि॒ पात्रा॑णि । \newline
14. दा॒रु॒मया॑णि॒ पात्रा॑णि॒ पात्रा॑णि दारु॒मया॑णि दारु॒मया॑णि॒ पात्रा᳚ण्य॒स्या अ॒स्यै पात्रा॑णि दारु॒मया॑णि दारु॒मया॑णि॒ पात्रा᳚ण्य॒स्यै । \newline
15. दा॒रु॒मया॒णीति॑ दारु - मया॑नि । \newline
16. पात्रा᳚ ण्य॒स्या अ॒स्यै पात्रा॑णि॒ पात्रा᳚ ण्य॒स्यै तानि॒ तान्य॒स्यै पात्रा॑णि॒ पात्रा᳚ ण्य॒स्यै तानि॑ । \newline
17. अ॒स्यै तानि॒ तान्य॒स्या अ॒स्यै तानि॒ योने॒र् योने॒ स्ता न्य॒स्या अ॒स्यै तानि॒ योनेः᳚ । \newline
18. तानि॒ योने॒र् योने॒ स्तानि॒ तानि॒ योनेः॒ संभू॑तानि॒ संभू॑तानि॒ योने॒ स्तानि॒ तानि॒ योनेः॒ संभू॑तानि । \newline
19. योनेः॒ संभू॑तानि॒ संभू॑तानि॒ योने॒र् योनेः॒ संभू॑तानि॒ यानि॒ यानि॒ संभू॑तानि॒ योने॒र् योनेः॒ संभू॑तानि॒ यानि॑ । \newline
20. संभू॑तानि॒ यानि॒ यानि॒ संभू॑तानि॒ संभू॑तानि॒ यानि॑ मृ॒न्मया॑नि मृ॒न्मया॑नि॒ यानि॒ संभू॑तानि॒ संभू॑तानि॒ यानि॑ मृ॒न्मया॑नि । \newline
21. संभू॑ता॒नीति॒ सं - भू॒ता॒नि॒ । \newline
22. यानि॑ मृ॒न्मया॑नि मृ॒न्मया॑नि॒ यानि॒ यानि॑ मृ॒न्मया॑नि सा॒क्षाथ् सा॒क्षान् मृ॒न्मया॑नि॒ यानि॒ यानि॑ मृ॒न्मया॑नि सा॒क्षात् । \newline
23. मृ॒न्मया॑नि सा॒क्षाथ् सा॒क्षान् मृ॒न्मया॑नि मृ॒न्मया॑नि सा॒क्षात् तानि॒ तानि॑ सा॒क्षान् मृ॒न्मया॑नि मृ॒न्मया॑नि सा॒क्षात् तानि॑ । \newline
24. मृ॒न्मया॒नीति॑ मृत् - मया॑नि । \newline
25. सा॒क्षात् तानि॒ तानि॑ सा॒क्षाथ् सा॒क्षात् तान्य॒स्या अ॒स्यै तानि॑ सा॒क्षाथ् सा॒क्षात् तान्य॒स्यै । \newline
26. सा॒क्षादिति॑ स - अ॒क्षात् । \newline
27. तान्य॒स्या अ॒स्यै तानि॒ तान्य॒स्यै तस्मा॒त् तस्मा॑ द॒स्यै तानि॒ तान्य॒स्यै तस्मा᳚त् । \newline
28. अ॒स्यै तस्मा॒त् तस्मा॑ द॒स्या अ॒स्यै तस्मा॑ दे॒व मे॒वम् तस्मा॑ द॒स्या अ॒स्यै तस्मा॑ दे॒वम् । \newline
29. तस्मा॑ दे॒व मे॒वम् तस्मा॒त् तस्मा॑ दे॒व मा॑हा है॒वम् तस्मा॒त् तस्मा॑ दे॒व मा॑ह । \newline
30. ए॒व मा॑हा है॒व मे॒व मा॑ह॒ वाग् वागा॑ है॒व मे॒व मा॑ह॒ वाक् । \newline
31. आ॒ह॒ वाग् वागा॑ हाह॒ वाग् वै वै वागा॑ हाह॒ वाग् वै । \newline
32. वाग् वै वै वाग् वाग् वै परा॑ची॒ परा॑ची॒ वै वाग् वाग् वै परा॑ची । \newline
33. वै परा॑ची॒ परा॑ची॒ वै वै परा॒ च्यव्या॑कृ॒ता ऽव्या॑कृता॒ परा॑ची॒ वै वै परा॒ च्यव्या॑कृता । \newline
34. परा॒ च्यव्या॑कृ॒ता ऽव्या॑कृता॒ परा॑ची॒ परा॒ च्यव्या॑कृता ऽवद दवद॒ दव्या॑कृता॒ परा॑ची॒ परा॒
च्यव्या॑कृता ऽवदत् । \newline
35. अव्या॑कृता ऽवद दवद॒ दव्या॑कृ॒ता ऽव्या॑कृता ऽवद॒त् ते ते॑ ऽवद॒ दव्या॑कृ॒ता ऽव्या॑कृता ऽवद॒त् ते । \newline
36. अव्या॑कृ॒तेत्यवि॑ - आ॒कृ॒ता॒ । \newline
37. अ॒व॒द॒त् ते ते॑ ऽवद दवद॒त् ते दे॒वा दे॒वा स्ते॑ ऽवद दवद॒त् ते दे॒वाः । \newline
38. ते दे॒वा दे॒वा स्ते ते दे॒वा इन्द्र॒ मिन्द्र॑म् दे॒वा स्ते ते दे॒वा इन्द्र᳚म् । \newline
39. दे॒वा इन्द्र॒ मिन्द्र॑म् दे॒वा दे॒वा इन्द्र॑ मब्रुवन् नब्रुव॒न् निन्द्र॑म् दे॒वा दे॒वा इन्द्र॑ मब्रुवन्न् । \newline
40. इन्द्र॑ मब्रुवन् नब्रुव॒न् निन्द्र॒ मिन्द्र॑ मब्रुवन् नि॒मा मि॒मा म॑ब्रुव॒न् निन्द्र॒ मिन्द्र॑ मब्रुवन् नि॒माम् । \newline
41. अ॒ब्रु॒व॒न् नि॒मा मि॒मा म॑ब्रुवन् नब्रुवन् नि॒मान् नो॑ न इ॒मा म॑ब्रुवन् नब्रुवन् नि॒मान् नः॑ । \newline
42. इ॒मान् नो॑ न इ॒मा मि॒मान् नो॒ वाचं॒ ॅवाच॑न् न इ॒मा मि॒मान् नो॒ वाच᳚म् । \newline
43. नो॒ वाचं॒ ॅवाच॑न् नो नो॒ वाचं॒ ॅव्याकु॑रु॒ व्याकु॑रु॒ वाच॑न् नो नो॒ वाचं॒ ॅव्याकु॑रु । \newline
44. वाचं॒ ॅव्याकु॑रु॒ व्याकु॑रु॒ वाचं॒ ॅवाचं॒ ॅव्याकु॒र्वितीति॒ व्याकु॑रु॒ वाचं॒ ॅवाचं॒ ॅव्याकु॒र्विति॑ । \newline
45. व्याकु॒र्वितीति॒ व्याकु॑रु॒ व्याकु॒र्विति॒ स स इति॒ व्याकु॑रु॒ व्याकु॒र्विति॒ सः । \newline
46. व्याकु॒र्विति॑ वि - आकु॑रु । \newline
47. इति॒ स स इतीति॒ सो᳚ ऽब्रवी दब्रवी॒थ् स इतीति॒ सो᳚ ऽब्रवीत् । \newline
48. सो᳚ ऽब्रवी दब्रवी॒थ् स सो᳚ ऽब्रवी॒द् वरं॒ ॅवर॑ मब्रवी॒थ् स सो᳚ ऽब्रवी॒द् वर᳚म् । \newline
49. अ॒ब्र॒वी॒द् वरं॒ ॅवर॑ मब्रवी दब्रवी॒द् वरं॑ ॅवृणै वृणै॒ वर॑ मब्रवी दब्रवी॒द् वरं॑ ॅवृणै । \newline
50. वरं॑ ॅवृणै वृणै॒ वरं॒ ॅवरं॑ ॅवृणै॒ मह्य॒म् मह्यं॑ ॅवृणै॒ वरं॒ ॅवरं॑ ॅवृणै॒ मह्य᳚म् । \newline
51. वृ॒णै॒ मह्य॒म् मह्यं॑ ॅवृणै वृणै॒ मह्य॑म् च च॒ मह्यं॑ ॅवृणै वृणै॒ मह्य॑म् च । \newline
52. मह्य॑म् च च॒ मह्य॒म् मह्य॑म् चै॒वैव च॒ मह्य॒म् मह्य॑म् चै॒व । \newline
53. चै॒वैव च॑ चै॒वैष ए॒ष ए॒व च॑ चै॒वैषः । \newline
54. ए॒वैष ए॒ष ए॒वै वैष वा॒यवे॑ वा॒यव॑ ए॒ष ए॒वै वैष वा॒यवे᳚ । \newline
55. ए॒ष वा॒यवे॑ वा॒यव॑ ए॒ष ए॒ष वा॒यवे॑ च च वा॒यव॑ ए॒ष ए॒ष वा॒यवे॑ च । \newline
56. वा॒यवे॑ च च वा॒यवे॑ वा॒यवे॑ च स॒ह स॒ह च॑ वा॒यवे॑ वा॒यवे॑ च स॒ह । \newline
57. च॒ स॒ह स॒ह च॑ च स॒ह गृ॑ह्यातै गृह्यातै स॒ह च॑ च स॒ह गृ॑ह्यातै । \newline
58. स॒ह गृ॑ह्यातै गृह्यातै स॒ह स॒ह गृ॑ह्याता॒ इतीति॑ गृह्यातै स॒ह स॒ह गृ॑ह्याता॒ इति॑ । \newline
59. गृ॒ह्या॒ता॒ इतीति॑ गृह्यातै गृह्याता॒ इति॒ तस्मा॒त् तस्मा॒ दिति॑ गृह्यातै गृह्याता॒ इति॒ तस्मा᳚त् । \newline
60. इति॒ तस्मा॒त् तस्मा॒ दितीति॒ तस्मा॑ दैन्द्रवाय॒व ऐ᳚न्द्रवाय॒व स्तस्मा॒ दितीति॒ तस्मा॑ दैन्द्रवाय॒वः । \newline
61. तस्मा॑ दैन्द्रवाय॒व ऐ᳚न्द्रवाय॒व स्तस्मा॒त् तस्मा॑ दैन्द्रवाय॒वः स॒ह स॒हैन्द्र॑वाय॒व स्तस्मा॒त् तस्मा॑ दैन्द्रवाय॒वः स॒ह । \newline
62. ऐ॒न्द्र॒वा॒य॒वः स॒ह स॒हैन्द्र॑वाय॒व ऐ᳚न्द्रवाय॒वः स॒ह गृ॑ह्यते गृह्यते स॒हैन्द्र॑वाय॒व ऐ᳚न्द्रवाय॒वः स॒ह गृ॑ह्यते । \newline
63. ऐ॒न्द्र॒वा॒य॒व इत्यै᳚न्द्र - वा॒य॒वः । \newline
64. स॒ह गृ॑ह्यते गृह्यते स॒ह स॒ह गृ॑ह्यते॒ ताम् ताम् गृ॑ह्यते स॒ह स॒ह गृ॑ह्यते॒ ताम् । \newline
65. गृ॒ह्य॒ते॒ ताम् ताम् गृ॑ह्यते गृह्यते॒ ता मिन्द्र॒ इन्द्र॒ स्ताम् गृ॑ह्यते गृह्यते॒ ता मिन्द्रः॑ । \newline
66. ता मिन्द्र॒ इन्द्र॒ स्ताम् ता मिन्द्रो॑ मद्ध्य॒तो म॑द्ध्य॒त इन्द्र॒ स्ताम् ता मिन्द्रो॑ मद्ध्य॒तः । \newline
67. इन्द्रो॑ मद्ध्य॒तो म॑द्ध्य॒त इन्द्र॒ इन्द्रो॑ मद्ध्य॒तो॑ ऽव॒क्रम्या॑ व॒क्रम्य॑ मद्ध्य॒त इन्द्र॒ इन्द्रो॑ मद्ध्य॒तो॑ ऽव॒क्रम्य॑ । \newline
68. म॒द्ध्य॒तो॑ ऽव॒क्रम्या॑ व॒क्रम्य॑ मद्ध्य॒तो म॑द्ध्य॒तो॑ ऽव॒क्रम्य॒ व्याक॑रो॒द् व्याक॑रो दव॒क्रम्य॑ मद्ध्य॒तो म॑द्ध्य॒तो॑ ऽव॒क्रम्य॒ व्याक॑रोत् । \newline
69. अ॒व॒क्रम्य॒ व्याक॑रो॒द् व्याक॑रो दव॒क्रम्या॑ व॒क्रम्य॒ व्याक॑रो॒त् तस्मा॒त् तस्मा॒द् व्याक॑रो दव॒क्रम्या॑ व॒क्रम्य॒ व्याक॑रो॒त् तस्मा᳚त् । \newline
70. अ॒व॒क्रम्येत्य॑व - क्रम्य॑ । \newline
71. व्याक॑रो॒त् तस्मा॒त् तस्मा॒द् व्याक॑रो॒द् व्याक॑रो॒त् तस्मा॑ दि॒य मि॒यम् तस्मा॒द् व्याक॑रो॒द् व्याक॑रो॒त् तस्मा॑ दि॒यम् । \newline
72. व्याक॑रो॒दिति॑ वि - आक॑रोत् । \newline
73. तस्मा॑ दि॒य मि॒यम् तस्मा॒त् तस्मा॑ दि॒यं ॅव्याकृ॑ता॒ व्याकृ॑ ते॒यम् तस्मा॒त् तस्मा॑ दि॒यं ॅव्याकृ॑ता । \newline
74. इ॒यं ॅव्याकृ॑ता॒ व्याकृ॑ ते॒य मि॒यं ॅव्याकृ॑ता॒ वाग् वाग् व्याकृ॑ ते॒य मि॒यं ॅव्याकृ॑ता॒ वाक् । \newline
75. व्याकृ॑ता॒ वाग् वाग् व्याकृ॑ता॒ व्याकृ॑ता॒ वागु॑द्यत उद्यते॒ वाग् व्याकृ॑ता॒ व्याकृ॑ता॒ वागु॑द्यते । \newline
76. व्याकृ॒तेति॑ वि - आकृ॑ता । \newline
77. वागु॑द्यत उद्यते॒ वाग् वागु॑द्यते॒ तस्मा॒त् तस्मा॑ दुद्यते॒ वाग् वागु॑द्यते॒ तस्मा᳚त् । \newline
78. उ॒द्य॒ते॒ तस्मा॒त् तस्मा॑ दुद्यत उद्यते॒ तस्मा᳚थ् स॒कृथ् स॒कृत् तस्मा॑ दुद्यत उद्यते॒ तस्मा᳚थ् स॒कृत् । \newline
79. तस्मा᳚थ् स॒कृथ् स॒कृत् तस्मा॒त् तस्मा᳚थ् स॒कृ दिन्द्रा॒ येन्द्रा॑य स॒कृत् तस्मा॒त् तस्मा᳚थ् स॒कृ दिन्द्रा॑य । \newline
80. स॒कृ दिन्द्रा॒ येन्द्रा॑य स॒कृथ् स॒कृ दिन्द्रा॑य मद्ध्य॒तो म॑द्ध्य॒त इन्द्रा॑य स॒कृथ् स॒कृ दिन्द्रा॑य मद्ध्य॒तः । \newline
81. इन्द्रा॑य मद्ध्य॒तो म॑द्ध्य॒त इन्द्रा॒ येन्द्रा॑य मद्ध्य॒तो गृ॑ह्यते गृह्यते मद्ध्य॒त इन्द्रा॒ येन्द्रा॑य मद्ध्य॒तो गृ॑ह्यते । \newline
82. म॒द्ध्य॒तो गृ॑ह्यते गृह्यते मद्ध्य॒तो म॑द्ध्य॒तो गृ॑ह्यते॒ द्विर् द्विर् गृ॑ह्यते मद्ध्य॒तो म॑द्ध्य॒तो गृ॑ह्यते॒ द्विः । \newline
83. गृ॒ह्य॒ते॒ द्विर् द्विर् गृ॑ह्यते गृह्यते॒ द्विर् वा॒यवे॑ वा॒यवे॒ द्विर् गृ॑ह्यते गृह्यते॒ द्विर् वा॒यवे᳚ । \newline
84. द्विर् वा॒यवे॑ वा॒यवे॒ द्विर् द्विर् वा॒यवे॒ द्वौ द्वौ वा॒यवे॒ द्विर् द्विर् वा॒यवे॒ द्वौ । \newline
85. वा॒यवे॒ द्वौ द्वौ वा॒यवे॑ वा॒यवे॒ द्वौ हि हि द्वौ वा॒यवे॑ वा॒यवे॒ द्वौ हि । \newline
86. द्वौ हि हि द्वौ द्वौ हि स स हि द्वौ द्वौ हि सः । \newline
87. हि स स हि हि स वरौ॒ वरौ॒ स हि हि स वरौ᳚ । \newline
88. स वरौ॒ वरौ॒ स स वरा॒ ववृ॑णी॒ता वृ॑णीत॒ वरौ॒ स स वरा॒ ववृ॑णीत । \newline
89. वरा॒ ववृ॑णी॒ता वृ॑णीत॒ वरौ॒ वरा॒ ववृ॑णीत । \newline
90. अवृ॑णी॒तेत्यवृ॑णीत । \newline
\pagebreak
\markright{ TS 6.4.8.1  \hfill https://www.vedavms.in \hfill}

\section{ TS 6.4.8.1 }

\textbf{TS 6.4.8.1 } \newline
\textbf{Samhita Paata} \newline

मि॒त्रं दे॒वा अ॑ब्रुव॒न्थ् सोमꣳ॒॒ राजा॑नꣳ हना॒मेति॒ सो᳚ऽब्रवी॒न्नाहꣳ सर्व॑स्य॒ वा अ॒हं मि॒त्रम॒स्मीति॒ तम॑ब्रुव॒न्॒. हना॑मै॒वेति॒ सो᳚ऽब्रवी॒द् वरं॑ ॅवृणै॒ पय॑सै॒व मे॒ सोमꣳ॑ श्रीण॒न्निति॒ तस्मा᳚न्-मैत्रावरु॒णं पय॑सा श्रीणन्ति॒ तस्मा᳚त् प॒शवोऽपा᳚क्रामन् मि॒त्रः सन् क्रू॒रम॑क॒रिति॑ क्रू॒रमि॑व॒ खलु॒ वा ए॒षः - [  ] \newline

\textbf{Pada Paata} \newline

मि॒त्रम् । दे॒वाः । अ॒ब्रु॒व॒न्न् । सोम᳚म् । राजा॑नम् । ह॒ना॒म॒ । इति॑ । सः । अ॒ब्र॒वी॒त् । न । अ॒हम् । सर्व॑स्य । वै । अ॒हम् । मि॒त्रम् । अ॒स्मि॒ । इति॑ । तम् । अ॒ब्रु॒व॒न्न् । हना॑म । ए॒व । इति॑ । सः । अ॒ब्र॒वी॒त् । वर᳚म् । वृ॒णै॒ । पय॑सा । ए॒व । मे॒ । सोम᳚म् । श्री॒ण॒न्न् । इति॑ । तस्मा᳚त् । मै॒त्रा॒व॒रु॒णमिति॑ मैत्रा - व॒रु॒णम् । पय॑सा । श्री॒ण॒न्ति॒ । तस्मा᳚त् । प॒शवः॑ । अपेति॑ । अ॒क्रा॒म॒न्न् । मि॒त्रः । सन्न् । क्रू॒रम् । अ॒कः॒ । इति॑ । क्रू॒रम् । इ॒व॒ । खलु॑ । वै । ए॒षः ।  \newline


\textbf{Krama Paata} \newline

मि॒त्रम् दे॒वाः । दे॒वा अ॑ब्रुवन्न् । अ॒ब्रु॒व॒न्थ् सोम᳚म् । सोमꣳ॒॒ राजा॑नम् । राजा॑नꣳ हनाम । ह॒ना॒मेति॑ । इति॒ सः । सो᳚ऽब्रवीत् । अ॒ब्र॒वी॒न् न । नाहम् । अ॒हꣳ सर्व॑स्य । सर्व॑स्य॒ वै । वा अ॒हम् । अ॒हम् मि॒त्रम् । मि॒त्रम॑स्मि । अ॒स्मीति॑ । इति॒ तम् । तम॑ब्रुवन्न् । अ॒ब्रु॒व॒न्॒. हना॑म । हना॑मै॒व । ए॒वेति॑ । इति॒ सः । सो᳚ऽब्रवीत् । अ॒ब्र॒वी॒द् वर᳚म् । वर॑म् ॅवृणै । वृ॒णै॒ पय॑सा । पय॑सै॒व । ए॒व मे᳚ । मे॒ सोम᳚म् । सोमꣳ॑ श्रीणान् । श्री॒णा॒निति॑ । इति॒ तस्मा᳚त् । तस्मा᳚न् मैत्रावरु॒णम् । मै॒त्रा॒व॒रु॒णम् पय॑सा । मै॒त्रा॒व॒रु॒णमिति॑ मैत्रा - व॒रु॒णम् । पय॑सा श्रीणन्ति । श्री॒ण॒न्ति॒ तस्मा᳚त् । तस्मा᳚त् प॒शवः॑ । प॒शवोऽप॑ । अपा᳚क्रामन्न् । अ॒क्रा॒म॒न् मि॒त्रः । मि॒त्रः सन्न् । सन् क्रू॒रम् । क्रू॒रम॑कः । अ॒क॒रिति॑ । इति॑ क्रू॒रम् । क्रू॒रमि॑व । इ॒व॒ खलु॑ । खलु॒ वै । वा ए॒षः । ए॒ष क॑रोति \newline

\textbf{Jatai Paata} \newline

1. मि॒त्रम् दे॒वा दे॒वा मि॒त्रम् मि॒त्रम् दे॒वाः । \newline
2. दे॒वा अ॑ब्रुवन् नब्रुवन् दे॒वा दे॒वा अ॑ब्रुवन्न् । \newline
3. अ॒ब्रु॒व॒न् थ्सोमꣳ॒॒ सोम॑ मब्रुवन् नब्रुव॒न् थ्सोम᳚म् । \newline
4. सोमꣳ॒॒ राजा॑नꣳ॒॒ राजा॑नꣳ॒॒ सोमꣳ॒॒ सोमꣳ॒॒ राजा॑नम् । \newline
5. राजा॑नꣳ हनाम हनाम॒ राजा॑नꣳ॒॒ राजा॑नꣳ हनाम । \newline
6. ह॒ना॒ मेतीति॑ हनाम हना॒मेति॑ । \newline
7. इति॒ स स इतीति॒ सः । \newline
8. सो᳚ ऽब्रवी दब्रवी॒थ् स सो᳚ ऽब्रवीत् । \newline
9. अ॒ब्र॒वी॒न् न नाब्र॑वी दब्रवी॒न् न । \newline
10. नाह म॒हन् न नाहम् । \newline
11. अ॒हꣳ सर्व॑स्य॒ सर्व॑स्या॒ह म॒हꣳ सर्व॑स्य । \newline
12. सर्व॑स्य॒ वै वै सर्व॑स्य॒ सर्व॑स्य॒ वै । \newline
13. वा अ॒ह म॒हं ॅवै वा अ॒हम् । \newline
14. अ॒हम् मि॒त्रम् मि॒त्र म॒ह म॒हम् मि॒त्रम् । \newline
15. मि॒त्र म॑स्म्यस्मि मि॒त्रम् मि॒त्र म॑स्मि । \newline
16. अ॒स्मी तीत्य॑स्म्य॒ स्मीति॑ । \newline
17. इति॒ तम् त मितीति॒ तम् । \newline
18. त म॑ब्रुवन् नब्रुव॒न् तम् त म॑ब्रुवन्न् । \newline
19. अ॒ब्रु॒व॒न्॒. हना॑म॒ हना॑मा ब्रुवन् नब्रुव॒न्॒. हना॑म । \newline
20. हना॑ मै॒वैव हना॑म॒ हना॑ मै॒व । \newline
21. ए॒वेती त्ये॒ वैवेति॑ । \newline
22. इति॒ स स इतीति॒ सः । \newline
23. सो᳚ ऽब्रवी दब्रवी॒थ् स सो᳚ ऽब्रवीत् । \newline
24. अ॒ब्र॒वी॒द् वरं॒ ॅवर॑ मब्रवी दब्रवी॒द् वर᳚म् । \newline
25. वरं॑ ॅवृणै वृणै॒ वरं॒ ॅवरं॑ ॅवृणै । \newline
26. वृ॒णै॒ पय॑सा॒ पय॑सा वृणै वृणै॒ पय॑सा । \newline
27. पय॑ सै॒वैव पय॑सा॒ पय॑ सै॒व । \newline
28. ए॒व मे॑ म ए॒वैव मे᳚ । \newline
29. मे॒ सोमꣳ॒॒ सोम॑म् मे मे॒ सोम᳚म् । \newline
30. सोमꣳ॑ श्रीणञ् छ्रीण॒न् थ्सोमꣳ॒॒ सोमꣳ॑ श्रीणन्न् । \newline
31. श्री॒ण॒न् नितीति॑ श्रीणञ् छ्रीण॒न् निति॑ । \newline
32. इति॒ तस्मा॒त् तस्मा॒ दितीति॒ तस्मा᳚त् । \newline
33. तस्मा᳚न् मैत्रावरु॒णम् मै᳚त्रावरु॒णम् तस्मा॒त् तस्मा᳚न् मैत्रावरु॒णम् । \newline
34. मै॒त्रा॒व॒रु॒णम् पय॑सा॒ पय॑सा मैत्रावरु॒णम् मै᳚त्रावरु॒णम् पय॑सा । \newline
35. मै॒त्रा॒व॒रु॒णमिति॑ मैत्रा - व॒रु॒णम् । \newline
36. पय॑सा श्रीणन्ति श्रीणन्ति॒ पय॑सा॒ पय॑सा श्रीणन्ति । \newline
37. श्री॒ण॒न्ति॒ तस्मा॒त् तस्मा᳚च् छ्रीणन्ति श्रीणन्ति॒ तस्मा᳚त् । \newline
38. तस्मा᳚त् प॒शवः॑ प॒शव॒ स्तस्मा॒त् तस्मा᳚त् प॒शवः॑ । \newline
39. प॒शवो ऽपाप॑ प॒शवः॑ प॒शवो ऽप॑ । \newline
40. अपा᳚ क्रामन् नक्राम॒न् नपापा᳚ क्रामन्न् । \newline
41. अ॒क्रा॒म॒न् मि॒त्रो मि॒त्रो᳚ ऽक्रामन् नक्रामन् मि॒त्रः । \newline
42. मि॒त्रः सन् थ्सन् मि॒त्रो मि॒त्रः सन्न् । \newline
43. सन् क्रू॒रम् क्रू॒रꣳ सन् थ्सन् क्रू॒रम् । \newline
44. क्रू॒र म॑क रकः क्रू॒रम् क्रू॒र म॑कः । \newline
45. अ॒क॒ रिती त्य॑क रक॒ रिति॑ । \newline
46. इति॑ क्रू॒रम् क्रू॒र मितीति॑ क्रू॒रम् । \newline
47. क्रू॒र मि॑वेव क्रू॒रम् क्रू॒र मि॑व । \newline
48. इ॒व॒ खलु॒ खल्वि॑वेव॒ खलु॑ । \newline
49. खलु॒ वै वै खलु॒ खलु॒ वै । \newline
50. वा ए॒ष ए॒ष वै वा ए॒षः । \newline
51. ए॒ष क॑रोति करो त्ये॒ष ए॒ष क॑रोति । \newline

\textbf{Ghana Paata } \newline

1. मि॒त्रम् दे॒वा दे॒वा मि॒त्रम् मि॒त्रम् दे॒वा अ॑ब्रुवन् नब्रुवन् दे॒वा मि॒त्रम् मि॒त्रम् दे॒वा अ॑ब्रुवन्न् । \newline
2. दे॒वा अ॑ब्रुवन् नब्रुवन् दे॒वा दे॒वा अ॑ब्रुव॒न् थ्सोमꣳ॒॒ सोम॑ मब्रुवन् दे॒वा दे॒वा अ॑ब्रुव॒न् थ्सोम᳚म् । \newline
3. अ॒ब्रु॒व॒न् थ्सोमꣳ॒॒ सोम॑ मब्रुवन् नब्रुव॒न् थ्सोमꣳ॒॒ राजा॑नꣳ॒॒ राजा॑नꣳ॒॒ सोम॑ मब्रुवन् नब्रुव॒न् थ्सोमꣳ॒॒ राजा॑नम् । \newline
4. सोमꣳ॒॒ राजा॑नꣳ॒॒ राजा॑नꣳ॒॒ सोमꣳ॒॒ सोमꣳ॒॒ राजा॑नꣳ हनाम हनाम॒ राजा॑नꣳ॒॒ सोमꣳ॒॒ सोमꣳ॒॒ राजा॑नꣳ हनाम । \newline
5. राजा॑नꣳ हनाम हनाम॒ राजा॑नꣳ॒॒ राजा॑नꣳ हना॒मेतीति॑ हनाम॒ राजा॑नꣳ॒॒ राजा॑नꣳ हना॒मेति॑ । \newline
6. ह॒ना॒मेतीति॑ हनाम हना॒मेति॒ स स इति॑ हनाम हना॒मेति॒ सः । \newline
7. इति॒ स स इतीति॒ सो᳚ ऽब्रवी दब्रवी॒ थ्स इतीति॒ सो᳚ ऽब्रवीत् । \newline
8. सो᳚ ऽब्रवी दब्रवी॒ थ्स सो᳚ ऽब्रवी॒न् न नाब्र॑वी॒ थ्स सो᳚ ऽब्रवी॒न् न । \newline
9. अ॒ब्र॒वी॒न् न नाब्र॑वी दब्रवी॒न् नाह म॒हम् नाब्र॑वी दब्रवी॒न् नाहम् । \newline
10. नाह म॒हम् न नाहꣳ सर्व॑स्य॒ सर्व॑ स्या॒हन् न नाहꣳ सर्व॑स्य । \newline
11. अ॒हꣳ सर्व॑स्य॒ सर्व॑ स्या॒ह म॒हꣳ सर्व॑स्य॒ वै वै सर्व॑ स्या॒ह म॒हꣳ सर्व॑स्य॒ वै । \newline
12. सर्व॑स्य॒ वै वै सर्व॑स्य॒ सर्व॑स्य॒ वा अ॒ह म॒हं ॅवै सर्व॑स्य॒ सर्व॑स्य॒ वा अ॒हम् । \newline
13. वा अ॒ह म॒हं ॅवै वा अ॒हम् मि॒त्रम् मि॒त्र म॒हं ॅवै वा अ॒हम् मि॒त्रम् । \newline
14. अ॒हम् मि॒त्रम् मि॒त्र म॒ह म॒हम् मि॒त्र म॑स्म्यस्मि मि॒त्र म॒ह म॒हम् मि॒त्र म॑स्मि । \newline
15. मि॒त्र म॑स्म्यस्मि मि॒त्रम् मि॒त्र म॒स्मीती त्य॑स्मि मि॒त्रम् मि॒त्र म॒स्मीति॑ । \newline
16. अ॒स्मीती त्य॑स्म्य॒ स्मीति॒ तम् त मित्य॑स्म्य॒ स्मीति॒ तम् । \newline
17. इति॒ तम् त मितीति॒ त म॑ब्रुवन् नब्रुव॒न् त मितीति॒ त म॑ब्रुवन्न् । \newline
18. त म॑ब्रुवन् नब्रुव॒न् तम् त म॑ब्रुव॒न्॒. हना॑म॒ हना॑मा ब्रुव॒न् तम् त म॑ब्रुव॒न्॒. हना॑म । \newline
19. अ॒ब्रु॒व॒न्॒. हना॑म॒ हना॑मा ब्रुवन् नब्रुव॒न्॒. हना॑मै॒वैव हना॑मा ब्रुवन् नब्रुव॒न्॒. हना॑मै॒व । \newline
20. हना॑मै॒वैव हना॑म॒ हना॑मै॒वे तीत्ये॒व हना॑म॒ हना॑मै॒वेति॑ । \newline
21. ए॒वे तीत्ये॒ वैवेति॒ स स इत्ये॒ वैवेति॒ सः । \newline
22. इति॒ स स इतीति॒ सो᳚ ऽब्रवी दब्रवी॒थ् स इतीति॒ सो᳚ ऽब्रवीत् । \newline
23. सो᳚ ऽब्रवी दब्रवी॒थ् स सो᳚ ऽब्रवी॒द् वरं॒ ॅवर॑ मब्रवी॒थ् स सो᳚ ऽब्रवी॒द् वर᳚म् । \newline
24. अ॒ब्र॒वी॒द् वरं॒ ॅवर॑ मब्रवी दब्रवी॒द् वरं॑ ॅवृणै वृणै॒ वर॑ मब्रवी दब्रवी॒द् वरं॑ ॅवृणै । \newline
25. वरं॑ ॅवृणै वृणै॒ वरं॒ ॅवरं॑ ॅवृणै॒ पय॑सा॒ पय॑सा वृणै॒ वरं॒ ॅवरं॑ ॅवृणै॒ पय॑सा । \newline
26. वृ॒णै॒ पय॑सा॒ पय॑सा वृणै वृणै॒ पय॑सै॒वैव पय॑सा वृणै वृणै॒ पय॑सै॒व । \newline
27. पय॑सै॒वैव पय॑सा॒ पय॑सै॒व मे॑ म ए॒व पय॑सा॒ पय॑सै॒व मे᳚ । \newline
28. ए॒व मे॑ म ए॒वैव मे॒ सोमꣳ॒॒ सोम॑म् म ए॒वैव मे॒ सोम᳚म् । \newline
29. मे॒ सोमꣳ॒॒ सोम॑म् मे मे॒ सोमꣳ॑ श्रीणञ् छ्रीण॒न् थ्सोम॑म् मे मे॒ सोमꣳ॑ श्रीणन्न् । \newline
30. सोमꣳ॑ श्रीणञ् छ्रीण॒न् थ्सोमꣳ॒॒ सोमꣳ॑ श्रीण॒न् नितीति॑ श्रीण॒न् थ्सोमꣳ॒॒ सोमꣳ॑ श्रीण॒न् निति॑ । \newline
31. श्री॒ण॒न् नितीति॑ श्रीणञ् छ्रीण॒न् निति॒ तस्मा॒त् तस्मा॒ दिति॑ श्रीणञ् छ्रीण॒न् निति॒ तस्मा᳚त् । \newline
32. इति॒ तस्मा॒त् तस्मा॒ दितीति॒ तस्मा᳚न् मैत्रावरु॒णम् मै᳚त्रावरु॒णम् तस्मा॒ दितीति॒ तस्मा᳚न् मैत्रावरु॒णम् । \newline
33. तस्मा᳚न् मैत्रावरु॒णम् मै᳚त्रावरु॒णम् तस्मा॒त् तस्मा᳚न् मैत्रावरु॒णम् पय॑सा॒ पय॑सा मैत्रावरु॒णम् तस्मा॒त् तस्मा᳚न् मैत्रावरु॒णम् पय॑सा । \newline
34. मै॒त्रा॒व॒रु॒णम् पय॑सा॒ पय॑सा मैत्रावरु॒णम् मै᳚त्रावरु॒णम् पय॑सा श्रीणन्ति श्रीणन्ति॒ पय॑सा मैत्रावरु॒णम् मै᳚त्रावरु॒णम् पय॑सा श्रीणन्ति । \newline
35. मै॒त्रा॒व॒रु॒णमिति॑ मैत्रा - व॒रु॒णम् । \newline
36. पय॑सा श्रीणन्ति श्रीणन्ति॒ पय॑सा॒ पय॑सा श्रीणन्ति॒ तस्मा॒त् तस्मा᳚च् छ्रीणन्ति॒ पय॑सा॒ पय॑सा श्रीणन्ति॒ तस्मा᳚त् । \newline
37. श्री॒ण॒न्ति॒ तस्मा॒त् तस्मा᳚च् छ्रीणन्ति श्रीणन्ति॒ तस्मा᳚त् प॒शवः॑ प॒शव॒ स्तस्मा᳚च् छ्रीणन्ति श्रीणन्ति॒ तस्मा᳚त् प॒शवः॑ । \newline
38. तस्मा᳚त् प॒शवः॑ प॒शव॒ स्तस्मा॒त् तस्मा᳚त् प॒शवो ऽपाप॑ प॒शव॒ स्तस्मा॒त् तस्मा᳚त् प॒शवो ऽप॑ । \newline
39. प॒शवो ऽपाप॑ प॒शवः॑ प॒शवो ऽपा᳚क्रामन् नक्राम॒न् नप॑ प॒शवः॑ प॒शवो ऽपा᳚क्रामन्न् । \newline
40. अपा᳚क्रामन् नक्राम॒न् नपापा᳚ क्रामन् मि॒त्रो मि॒त्रो᳚ ऽक्राम॒न् नपापा᳚ क्रामन् मि॒त्रः । \newline
41. अ॒क्रा॒म॒न् मि॒त्रो मि॒त्रो᳚ ऽक्रामन् नक्रामन् मि॒त्रः सन् थ्सन् मि॒त्रो᳚ ऽक्रामन् नक्रामन् मि॒त्रः सन्न् । \newline
42. मि॒त्रः सन् थ्सन् मि॒त्रो मि॒त्रः सन् क्रू॒रम् क्रू॒रꣳ सन् मि॒त्रो मि॒त्रः सन् क्रू॒रम् । \newline
43. सन् क्रू॒रम् क्रू॒रꣳ सन् थ्सन् क्रू॒र म॑क रकः क्रू॒रꣳ सन् थ्सन् क्रू॒र म॑कः । \newline
44. क्रू॒र म॑क रकः क्रू॒रम् क्रू॒र म॑क॒ रिती त्य॑कः क्रू॒रम् क्रू॒र म॑क॒ रिति॑ । \newline
45. अ॒क॒ रिती त्य॑क रक॒ रिति॑ क्रू॒रम् क्रू॒र मित्य॑क रक॒ रिति॑ क्रू॒रम् । \newline
46. इति॑ क्रू॒रम् क्रू॒र मितीति॑ क्रू॒र मि॑वेव क्रू॒र मितीति॑ क्रू॒र मि॑व । \newline
47. क्रू॒र मि॑वेव क्रू॒रम् क्रू॒र मि॑व॒ खलु॒ खल्वि॑व क्रू॒रम् क्रू॒र मि॑व॒ खलु॑ । \newline
48. इ॒व॒ खलु॒ खल्वि॑ वेव॒ खलु॒ वै वै खल्वि॑ वेव॒ खलु॒ वै । \newline
49. खलु॒ वै वै खलु॒ खलु॒ वा ए॒ष ए॒ष वै खलु॒ खलु॒ वा ए॒षः । \newline
50. वा ए॒ष ए॒ष वै वा ए॒ष क॑रोति करो त्ये॒ष वै वा ए॒ष क॑रोति । \newline
51. ए॒ष क॑रोति करो त्ये॒ष ए॒ष क॑रोति॒ यो यः क॑रो त्ये॒ष ए॒ष क॑रोति॒ यः । \newline
\pagebreak
\markright{ TS 6.4.8.2  \hfill https://www.vedavms.in \hfill}

\section{ TS 6.4.8.2 }

\textbf{TS 6.4.8.2 } \newline
\textbf{Samhita Paata} \newline

क॑रोति॒ यः सोमे॑न॒ यज॑ते॒ तस्मा᳚त् प॒शवोऽप॑ क्रामन्ति॒ यन्मै᳚त्रावरु॒णं पय॑सा श्री॒णाति॑ प॒शुभि॑रे॒व तन्मि॒त्रꣳ स॑म॒र्द्धय॑ति प॒शुभि॒र्यज॑मानं पु॒रा खलु॒ वावैवं मि॒त्रो॑ऽवे॒दप॒ मत् क्रू॒रं च॒क्रुषः॑ प॒शवः॑ क्रमिष्य॒न्तीति॒ तस्मा॑दे॒वम॑वृणीत॒ वरु॑णं दे॒वा अ॑ब्रुव॒न् त्वयाऽꣳ॑श॒भुवा॒ सोमꣳ॒॒ राजा॑नꣳ हना॒मेति॒ सो᳚ऽब्रवी॒द् वरं॑ ॅवृणै॒ मह्यं॑ चै॒- [  ] \newline

\textbf{Pada Paata} \newline

क॒रो॒ति॒ । यः । सोमे॑न । यज॑ते । तस्मा᳚त् । प॒शवः॑ । अपेति॑ । क्रा॒म॒न्ति॒ । यत् । मै॒त्रा॒व॒रु॒णमिति॑ मैत्रा-व॒रु॒णम् । पय॑सा । श्री॒णाति॑ । प॒शुभि॒रिति॑ प॒शु - भिः॒ । ए॒व । तत् । मि॒त्रम् । स॒म॒द्‌र्धय॒तीति॑ सं - अ॒द्‌र्धय॑ति । प॒शुभि॒रिति॑ प॒शु - भिः॒ । यज॑मानम् । पु॒रा । खलु॑ । वाव । ए॒वम् । मि॒त्रः । अ॒वे॒त् । अपेति॑ । मत् । क्रू॒रम् । च॒क्रुषः॑ । प॒शवः॑ । क्र॒मि॒ष्य॒न्ति॒ । इति॑ । तस्मा᳚त् । ए॒वम् । अ॒वृ॒णी॒त॒ । वरु॑णम् । दे॒वाः । अ॒ब्रु॒व॒न्न् । त्वया᳚ । अꣳ॒॒श॒भुवेत्यꣳ॑श - भुवा᳚ । सोम᳚म् । राजा॑नम् । ह॒ना॒म॒ । इति॑ । सः । अ॒ब्र॒वी॒त् । वर᳚म् । वृ॒णै॒ । मह्य᳚म् । च॒ ।  \newline


\textbf{Krama Paata} \newline

क॒रो॒ति॒ यः । यः सोमे॑न । सोमे॑न॒ यज॑ते । यज॑ते॒ तस्मा᳚त् । तस्मा᳚त् प॒शवः॑ । प॒शवोऽप॑ । अप॑ क्रामन्ति । क्रा॒म॒न्ति॒ यत् । यन् मै᳚त्रावरु॒णम् । मै॒त्रा॒व॒रु॒णम् पय॑सा । मै॒त्रा॒व॒रु॒णमिति॑ मैत्रा - व॒रु॒णम् । पय॑सा श्री॒णाति॑ । श्री॒णाति॑ प॒शुभिः॑ । प॒शुभि॑रे॒व । प॒शुभि॒रिति॑ प॒शु - भिः॒ । ए॒व तत् । तन् मि॒त्रम् । मि॒त्रꣳ स॑म॒र्द्धय॑ति । स॒म॒र्द्धय॑ति प॒शुभिः॑ । स॒म॒र्द्धय॒तीति॑ सम् - अ॒र्द्धय॑ति । प॒शुभि॒र् यज॑मानम् । प॒शुभि॒रिति॑ प॒शु - भिः॒ । यज॑मानम् पु॒रा । पु॒रा खलु॑ । खलु॒ वाव । वावैवम् । ए॒वम् मि॒त्रः । मि॒त्रो॑ऽवेत् । अ॒वे॒दप॑ । अप॒ मत् । मत् क्रू॒रम् । क्रू॒रम् च॒क्रुषः॑ । च॒क्रुषः॑ प॒शवः॑ । प॒शवः॑ क्रमिष्यन्ति । क्र॒मि॒ष्य॒न्तीति॑ । इति॒ तस्मा᳚त् । तस्मा॑दे॒वम् । ए॒वम॑वृणीत । अ॒वृ॒णी॒त॒ वरु॑णम् । वरु॑णम् दे॒वाः । दे॒वा अ॑ब्रुवन्न् । अ॒ब्रु॒व॒न् त्वया᳚ । त्वयाऽꣳ॑श॒भुवा᳚ । अꣳ॒॒श॒भुवा॒ सोम᳚म् । अꣳ॒॒श॒भुवेत्यꣳ॑श - भुवा᳚ । सोमꣳ॒॒ राजा॑नम् । राजा॑नꣳ हनाम । ह॒ना॒मेति॑ । इति॒ सः । सो᳚ऽब्रवीत् । अ॒ब्र॒वी॒द् वर᳚म् । वर॑म् ॅवृणै । वृ॒णै॒ मह्य᳚म् । मह्य॑म् च । चै॒व \newline

\textbf{Jatai Paata} \newline

1. क॒रो॒ति॒ यो यः क॑रोति करोति॒ यः । \newline
2. यः सोमे॑न॒ सोमे॑न॒ यो यः सोमे॑न । \newline
3. सोमे॑न॒ यज॑ते॒ यज॑ते॒ सोमे॑न॒ सोमे॑न॒ यज॑ते । \newline
4. यज॑ते॒ तस्मा॒त् तस्मा॒द् यज॑ते॒ यज॑ते॒ तस्मा᳚त् । \newline
5. तस्मा᳚त् प॒शवः॑ प॒शव॒ स्तस्मा॒त् तस्मा᳚त् प॒शवः॑ । \newline
6. प॒शवो ऽपाप॑ प॒शवः॑ प॒शवो ऽप॑ । \newline
7. अप॑ क्रामन्ति क्राम॒ न्त्यपाप॑ क्रामन्ति । \newline
8. क्रा॒म॒न्ति॒ यद् यत् क्रा॑मन्ति क्रामन्ति॒ यत् । \newline
9. यन् मै᳚त्रावरु॒णम् मै᳚त्रावरु॒णं ॅयद् यन् मै᳚त्रावरु॒णम् । \newline
10. मै॒त्रा॒व॒रु॒णम् पय॑सा॒ पय॑सा मैत्रावरु॒णम् मै᳚त्रावरु॒णम् पय॑सा । \newline
11. मै॒त्रा॒व॒रु॒णमिति॑ मैत्रा - व॒रु॒णम् । \newline
12. पय॑सा श्री॒णाति॑ श्री॒णाति॒ पय॑सा॒ पय॑सा श्री॒णाति॑ । \newline
13. श्री॒णाति॑ प॒शुभिः॑ प॒शुभिः॑ श्री॒णाति॑ श्री॒णाति॑ प॒शुभिः॑ । \newline
14. प॒शुभि॑ रे॒वैव प॒शुभिः॑ प॒शुभि॑ रे॒व । \newline
15. प॒शुभि॒रिति॑ प॒शु - भिः॒ । \newline
16. ए॒व तत् तदे॒ वैव तत् । \newline
17. तन् मि॒त्रम् मि॒त्रम् तत् तन् मि॒त्रम् । \newline
18. मि॒त्रꣳ स॑म॒र्द्धय॑ति सम॒र्द्धय॑ति मि॒त्रम् मि॒त्रꣳ स॑म॒र्द्धय॑ति । \newline
19. स॒म॒र्द्धय॑ति प॒शुभिः॑ प॒शुभिः॑ सम॒र्द्धय॑ति सम॒र्द्धय॑ति प॒शुभिः॑ । \newline
20. स॒म॒र्द्धय॒तीति॑ सं - अ॒र्द्धय॑ति । \newline
21. प॒शुभि॒र् यज॑मानं॒ ॅयज॑मानम् प॒शुभिः॑ प॒शुभि॒र् यज॑मानम् । \newline
22. प॒शुभि॒रिति॑ प॒शु - भिः॒ । \newline
23. यज॑मानम् पु॒रा पु॒रा यज॑मानं॒ ॅयज॑मानम् पु॒रा । \newline
24. पु॒रा खलु॒ खलु॑ पु॒रा पु॒रा खलु॑ । \newline
25. खलु॒ वाव वाव खलु॒ खलु॒ वाव । \newline
26. वावैव मे॒वं ॅवाव वावैवम् । \newline
27. ए॒वम् मि॒त्रो मि॒त्र ए॒व मे॒वम् मि॒त्रः । \newline
28. मि॒त्रो॑ ऽवे दवेन् मि॒त्रो मि॒त्रो॑ ऽवेत् । \newline
29. अ॒वे॒ दपापा॑ वे दवे॒ दप॑ । \newline
30. अप॒ मन् मद पाप॒ मत् । \newline
31. मत् क्रू॒रम् क्रू॒रम् मन् मत् क्रू॒रम् । \newline
32. क्रू॒रम् च॒क्रुष॑ श्च॒क्रुषः॑ क्रू॒रम् क्रू॒रम् च॒क्रुषः॑ । \newline
33. च॒क्रुषः॑ प॒शवः॑ प॒शव॑ श्च॒क्रुष॑ श्च॒क्रुषः॑ प॒शवः॑ । \newline
34. प॒शवः॑ क्रमिष्यन्ति क्रमिष्यन्ति प॒शवः॑ प॒शवः॑ क्रमिष्यन्ति । \newline
35. क्र॒मि॒ष्य॒न्ती तीति॑ क्रमिष्यन्ति क्रमिष्य॒न्तीति॑ । \newline
36. इति॒ तस्मा॒त् तस्मा॒ दितीति॒ तस्मा᳚त् । \newline
37. तस्मा॑ दे॒व मे॒वम् तस्मा॒त् तस्मा॑ दे॒वम् । \newline
38. ए॒व म॑वृणीता वृणीतै॒व मे॒व म॑वृणीत । \newline
39. अ॒वृ॒णी॒त॒ वरु॑णं॒ ॅवरु॑ण मवृणीता वृणीत॒ वरु॑णम् । \newline
40. वरु॑णम् दे॒वा दे॒वा वरु॑णं॒ ॅवरु॑णम् दे॒वाः । \newline
41. दे॒वा अ॑ब्रुवन् नब्रुवन् दे॒वा दे॒वा अ॑ब्रुवन्न् । \newline
42. अ॒ब्रु॒व॒न् त्वया॒ त्वया᳚ ऽब्रुवन् नब्रुव॒न् त्वया᳚ । \newline
43. त्वया ऽꣳ॑श॒भुवा ऽꣳ॑श॒भुवा॒ त्वया॒ त्वया ऽꣳ॑श॒भुवा᳚ । \newline
44. अꣳ॒॒श॒भुवा॒ सोमꣳ॒॒ सोम॑ मꣳश॒भुवा ऽꣳ॑श॒भुवा॒ सोम᳚म् । \newline
45. अꣳ॒॒श॒भुवेत्यꣳ॑श - भुवा᳚ । \newline
46. सोमꣳ॒॒ राजा॑नꣳ॒॒ राजा॑नꣳ॒॒ सोमꣳ॒॒ सोमꣳ॒॒ राजा॑नम् । \newline
47. राजा॑नꣳ हनाम हनाम॒ राजा॑नꣳ॒॒ राजा॑नꣳ हनाम । \newline
48. ह॒ना॒मेतीति॑ हनाम हना॒मेति॑ । \newline
49. इति॒ स स इतीति॒ सः । \newline
50. सो᳚ ऽब्रवी दब्रवी॒थ् स सो᳚ ऽब्रवीत् । \newline
51. अ॒ब्र॒वी॒द् वरं॒ ॅवर॑ मब्रवी दब्रवी॒द् वर᳚म् । \newline
52. वरं॑ ॅवृणै वृणै॒ वरं॒ ॅवरं॑ ॅवृणै । \newline
53. वृ॒णै॒ मह्य॒म् मह्यं॑ ॅवृणै वृणै॒ मह्य᳚म् । \newline
54. मह्य॑म् च च॒ मह्य॒म् मह्य॑म् च । \newline
55. चै॒वैव च॑ चै॒व । \newline

\textbf{Ghana Paata } \newline

1. क॒रो॒ति॒ यो यः क॑रोति करोति॒ यः सोमे॑न॒ सोमे॑न॒ यः क॑रोति करोति॒ यः सोमे॑न । \newline
2. यः सोमे॑न॒ सोमे॑न॒ यो यः सोमे॑न॒ यज॑ते॒ यज॑ते॒ सोमे॑न॒ यो यः सोमे॑न॒ यज॑ते । \newline
3. सोमे॑न॒ यज॑ते॒ यज॑ते॒ सोमे॑न॒ सोमे॑न॒ यज॑ते॒ तस्मा॒त् तस्मा॒द् यज॑ते॒ सोमे॑न॒ सोमे॑न॒ यज॑ते॒ तस्मा᳚त् । \newline
4. यज॑ते॒ तस्मा॒त् तस्मा॒द् यज॑ते॒ यज॑ते॒ तस्मा᳚त् प॒शवः॑ प॒शव॒ स्तस्मा॒द् यज॑ते॒ यज॑ते॒ तस्मा᳚त् प॒शवः॑ । \newline
5. तस्मा᳚त् प॒शवः॑ प॒शव॒ स्तस्मा॒त् तस्मा᳚त् प॒शवो ऽपाप॑ प॒शव॒ स्तस्मा॒त् तस्मा᳚त् प॒शवो ऽप॑ । \newline
6. प॒शवो ऽपाप॑ प॒शवः॑ प॒शवो ऽप॑ क्रामन्ति क्राम॒ न्त्यप॑ प॒शवः॑ प॒शवो ऽप॑ क्रामन्ति । \newline
7. अप॑ क्रामन्ति क्राम॒ न्त्यपाप॑ क्रामन्ति॒ यद् यत् क्रा॑म॒ न्त्यपाप॑ क्रामन्ति॒ यत् । \newline
8. क्रा॒म॒न्ति॒ यद् यत् क्रा॑मन्ति क्रामन्ति॒ यन् मै᳚त्रावरु॒णम् मै᳚त्रावरु॒णं ॅयत् क्रा॑मन्ति क्रामन्ति॒ यन् मै᳚त्रावरु॒णम् । \newline
9. यन् मै᳚त्रावरु॒णम् मै᳚त्रावरु॒णं ॅयद् यन् मै᳚त्रावरु॒णम् पय॑सा॒ पय॑सा मैत्रावरु॒णं ॅयद् यन् मै᳚त्रावरु॒णम् पय॑सा । \newline
10. मै॒त्रा॒व॒रु॒णम् पय॑सा॒ पय॑सा मैत्रावरु॒णम् मै᳚त्रावरु॒णम् पय॑सा श्री॒णाति॑ श्री॒णाति॒ पय॑सा मैत्रावरु॒णम् मै᳚त्रावरु॒णम् पय॑सा श्री॒णाति॑ । \newline
11. मै॒त्रा॒व॒रु॒णमिति॑ मैत्रा - व॒रु॒णम् । \newline
12. पय॑सा श्री॒णाति॑ श्री॒णाति॒ पय॑सा॒ पय॑सा श्री॒णाति॑ प॒शुभिः॑ प॒शुभिः॑ श्री॒णाति॒ पय॑सा॒ पय॑सा श्री॒णाति॑ प॒शुभिः॑ । \newline
13. श्री॒णाति॑ प॒शुभिः॑ प॒शुभिः॑ श्री॒णाति॑ श्री॒णाति॑ प॒शुभि॑ रे॒वैव प॒शुभिः॑ श्री॒णाति॑ श्री॒णाति॑ प॒शुभि॑ रे॒व । \newline
14. प॒शुभि॑ रे॒वैव प॒शुभिः॑ प॒शुभि॑ रे॒व तत् तदे॒व प॒शुभिः॑ प॒शुभि॑ रे॒व तत् । \newline
15. प॒शुभि॒रिति॑ प॒शु - भिः॒ । \newline
16. ए॒व तत् तदे॒ वैव तन् मि॒त्रम् मि॒त्रम् तदे॒ वैव तन् मि॒त्रम् । \newline
17. तन् मि॒त्रम् मि॒त्रम् तत् तन् मि॒त्रꣳ स॑म॒र्द्धय॑ति सम॒र्द्धय॑ति मि॒त्रम् तत् तन् मि॒त्रꣳ स॑म॒र्द्धय॑ति । \newline
18. मि॒त्रꣳ स॑म॒र्द्धय॑ति सम॒र्द्धय॑ति मि॒त्रम् मि॒त्रꣳ स॑म॒र्द्धय॑ति प॒शुभिः॑ प॒शुभिः॑ सम॒र्द्धय॑ति मि॒त्रम् मि॒त्रꣳ स॑म॒र्द्धय॑ति प॒शुभिः॑ । \newline
19. स॒म॒र्द्धय॑ति प॒शुभिः॑ प॒शुभिः॑ सम॒र्द्धय॑ति सम॒र्द्धय॑ति प॒शुभि॒र् यज॑मानं॒ ॅयज॑मानम् प॒शुभिः॑ सम॒र्द्धय॑ति सम॒र्द्धय॑ति प॒शुभि॒र् यज॑मानम् । \newline
20. स॒म॒र्द्धय॒तीति॑ सं - अ॒र्द्धय॑ति । \newline
21. प॒शुभि॒र् यज॑मानं॒ ॅयज॑मानम् प॒शुभिः॑ प॒शुभि॒र् यज॑मानम् पु॒रा पु॒रा यज॑मानम् प॒शुभिः॑ प॒शुभि॒र् यज॑मानम् पु॒रा । \newline
22. प॒शुभि॒रिति॑ प॒शु - भिः॒ । \newline
23. यज॑मानम् पु॒रा पु॒रा यज॑मानं॒ ॅयज॑मानम् पु॒रा खलु॒ खलु॑ पु॒रा यज॑मानं॒ ॅयज॑मानम् पु॒रा खलु॑ । \newline
24. पु॒रा खलु॒ खलु॑ पु॒रा पु॒रा खलु॒ वाव वाव खलु॑ पु॒रा पु॒रा खलु॒ वाव । \newline
25. खलु॒ वाव वाव खलु॒ खलु॒ वावैव मे॒वं ॅवाव खलु॒ खलु॒ वावैवम् । \newline
26. वावैव मे॒वं ॅवाव वावैवम् मि॒त्रो मि॒त्र ए॒वं ॅवाव वावैवम् मि॒त्रः । \newline
27. ए॒वम् मि॒त्रो मि॒त्र ए॒व मे॒वम् मि॒त्रो॑ ऽवे दवेन् मि॒त्र ए॒व मे॒वम् मि॒त्रो॑ ऽवेत् । \newline
28. मि॒त्रो॑ ऽवे दवेन् मि॒त्रो मि॒त्रो॑ ऽवे॒ दपा पा॑वेन् मि॒त्रो मि॒त्रो॑ ऽवे॒ दप॑ । \newline
29. अ॒वे॒ दपा पा॑वे दवे॒ दप॒ मन् म दपा॑वे दवे॒ दप॒ मत् । \newline
30. अप॒ मन् मदपाप॒ मत् क्रू॒रम् क्रू॒रम् मदपाप॒ मत् क्रू॒रम् । \newline
31. मत् क्रू॒रम् क्रू॒रम् मन् मत् क्रू॒रम् च॒क्रुष॑ श्च॒क्रुषः॑ क्रू॒रम् मन् मत् क्रू॒रम् च॒क्रुषः॑ । \newline
32. क्रू॒रम् च॒क्रुष॑ श्च॒क्रुषः॑ क्रू॒रम् क्रू॒रम् च॒क्रुषः॑ प॒शवः॑ प॒शव॑ श्च॒क्रुषः॑ क्रू॒रम् क्रू॒रम् च॒क्रुषः॑ प॒शवः॑ । \newline
33. च॒क्रुषः॑ प॒शवः॑ प॒शव॑ श्च॒क्रुष॑ श्च॒क्रुषः॑ प॒शवः॑ क्रमिष्यन्ति क्रमिष्यन्ति प॒शव॑ श्च॒क्रुष॑ श्च॒क्रुषः॑ प॒शवः॑ क्रमिष्यन्ति । \newline
34. प॒शवः॑ क्रमिष्यन्ति क्रमिष्यन्ति प॒शवः॑ प॒शवः॑ क्रमिष्य॒न्ती तीति॑ क्रमिष्यन्ति प॒शवः॑ प॒शवः॑ क्रमिष्य॒न्तीति॑ । \newline
35. क्र॒मि॒ष्य॒न्ती तीति॑ क्रमिष्यन्ति क्रमिष्य॒न्तीति॒ तस्मा॒त् तस्मा॒ दिति॑ क्रमिष्यन्ति क्रमिष्य॒न्तीति॒ तस्मा᳚त् । \newline
36. इति॒ तस्मा॒त् तस्मा॒ दितीति॒ तस्मा॑ दे॒व मे॒वम् तस्मा॒ दितीति॒ तस्मा॑ दे॒वम् । \newline
37. तस्मा॑ दे॒व मे॒वम् तस्मा॒त् तस्मा॑ दे॒व म॑वृणीता वृणीतै॒वम् तस्मा॒त् तस्मा॑ दे॒व म॑वृणीत । \newline
38. ए॒व म॑वृणीता वृणीतै॒व मे॒व म॑वृणीत॒ वरु॑णं॒ ॅवरु॑ण मवृणीतै॒व मे॒व म॑वृणीत॒ वरु॑णम् । \newline
39. अ॒वृ॒णी॒त॒ वरु॑णं॒ ॅवरु॑ण मवृणीता वृणीत॒ वरु॑णम् दे॒वा दे॒वा वरु॑ण मवृणीता वृणीत॒ वरु॑णम् दे॒वाः । \newline
40. वरु॑णम् दे॒वा दे॒वा वरु॑णं॒ ॅवरु॑णम् दे॒वा अ॑ब्रुवन् नब्रुवन् दे॒वा वरु॑णं॒ ॅवरु॑णम् दे॒वा अ॑ब्रुवन्न् । \newline
41. दे॒वा अ॑ब्रुवन् नब्रुवन् दे॒वा दे॒वा अ॑ब्रुव॒न् त्वया॒ त्वया᳚ ऽब्रुवन् दे॒वा दे॒वा अ॑ब्रुव॒न् त्वया᳚ । \newline
42. अ॒ब्रु॒व॒न् त्वया॒ त्वया᳚ ऽब्रुवन् नब्रुव॒न् त्वया ऽꣳ॑श॒भुवा ऽꣳ॑श॒भुवा॒ त्वया᳚ ऽब्रुवन् नब्रुव॒न् 
त्वया ऽꣳ॑श॒भुवा᳚ । \newline
43. त्वया ऽꣳ॑श॒भुवा ऽꣳ॑श॒भुवा॒ त्वया॒ त्वया ऽꣳ॑श॒भुवा॒ सोमꣳ॒॒ सोम॑ मꣳश॒भुवा॒ 
त्वया॒ त्वया ऽꣳ॑श॒भुवा॒ सोम᳚म् । \newline
44. अꣳ॒॒श॒भुवा॒ सोमꣳ॒॒ सोम॑ मꣳश॒भुवा ऽꣳ॑श॒भुवा॒ सोमꣳ॒॒ राजा॑नꣳ॒॒ राजा॑नꣳ॒॒ सोम॑ मꣳश॒भुवा ऽꣳ॑श॒भुवा॒ सोमꣳ॒॒ राजा॑नम् । \newline
45. अꣳ॒॒श॒भुवेत्यꣳ॑श - भुवा᳚ । \newline
46. सोमꣳ॒॒ राजा॑नꣳ॒॒ राजा॑नꣳ॒॒ सोमꣳ॒॒ सोमꣳ॒॒ राजा॑नꣳ हनाम हनाम॒ राजा॑नꣳ॒॒ सोमꣳ॒॒ सोमꣳ॒॒ राजा॑नꣳ हनाम । \newline
47. राजा॑नꣳ हनाम हनाम॒ राजा॑नꣳ॒॒ राजा॑नꣳ हना॒मेतीति॑ हनाम॒ राजा॑नꣳ॒॒ राजा॑नꣳ हना॒मेति॑ । \newline
48. ह॒ना॒ मेतीति॑ हनाम हना॒ मेति॒ स स इति॑ हनाम हना॒ मेति॒ सः । \newline
49. इति॒ स स इतीति॒ सो᳚ ऽब्रवी दब्रवी॒थ् स इतीति॒ सो᳚ ऽब्रवीत् । \newline
50. सो᳚ ऽब्रवी दब्रवी॒थ् स सो᳚ ऽब्रवी॒द् वरं॒ ॅवर॑ मब्रवी॒थ् स सो᳚ ऽब्रवी॒द् वर᳚म् । \newline
51. अ॒ब्र॒वी॒द् वरं॒ ॅवर॑ मब्रवी दब्रवी॒द् वरं॑ ॅवृणै वृणै॒ वर॑ मब्रवी दब्रवी॒द् वरं॑ ॅवृणै । \newline
52. वरं॑ ॅवृणै वृणै॒ वरं॒ ॅवरं॑ ॅवृणै॒ मह्य॒म् मह्यं॑ ॅवृणै॒ वरं॒ ॅवरं॑ ॅवृणै॒ मह्य᳚म् । \newline
53. वृ॒णै॒ मह्य॒म् मह्यं॑ ॅवृणै वृणै॒ मह्य॑म् च च॒ मह्यं॑ ॅवृणै वृणै॒ मह्य॑म् च । \newline
54. मह्य॑म् च च॒ मह्य॒म् मह्य॑म् चै॒वैव च॒ मह्य॒म् मह्य॑म् चै॒व । \newline
55. चै॒वैव च॑ चै॒वैष ए॒ष ए॒व च॑ चै॒वैषः । \newline
\pagebreak
\markright{ TS 6.4.8.3  \hfill https://www.vedavms.in \hfill}

\section{ TS 6.4.8.3 }

\textbf{TS 6.4.8.3 } \newline
\textbf{Samhita Paata} \newline

वैष मि॒त्राय॑ च स॒ह गृ॑ह्याता॒ इति॒ तस्मा᳚न्मैत्रावरु॒णः स॒ह गृ॑ह्यते॒ तस्मा॒द् राज्ञा॒ राजा॑नमꣳश॒भुवा᳚ घ्नन्ति॒ वैश्ये॑न॒ वैश्यꣳ॑ शू॒द्रेण॑ शू॒द्रं न वा इ॒दं दिवा॒ न नक्त॑मासी॒दव्या॑वृत्तं॒ ते दे॒वा मि॒त्रावरु॑णावब्रुवन्नि॒दं नो॒ विवा॑सयत॒मिति॒ ताव॑ब्रूतां॒ ॅवरं॑ ॅवृणावहा॒ एक॑ ए॒वाऽऽ*वत् पूर्वो॒ ग्रहो॑ गृह्याता॒ इति॒ तस्मा॑दैन्द्रवाय॒वः ( ) पूर्वो॑ मैत्रावरु॒णाद्-गृ॑ह्यते प्राणापा॒नौ ह्ये॑तौ यदु॑पाꣳ-श्वन्तर्या॒मौ मि॒त्रोऽह॒रज॑नय॒द्-वरु॑णो॒ रात्रिं॒ ततो॒ वा इ॒दं ॅव्यौ᳚च्छ॒द्यन्-मै᳚त्रावरु॒णो गृ॒ह्यते॒ व्यु॑ष्ट्यै ॥ \newline

\textbf{Pada Paata} \newline

ए॒व । ए॒षः । मि॒त्राय॑ । च॒ । स॒ह । गृ॒ह्या॒तै॒ । इति॑ । तस्मा᳚त् । मै॒त्रा॒व॒रु॒ण इति॑ मैत्रा - व॒रु॒णः । स॒ह । गृ॒ह्य॒ते॒ । तस्मा᳚त् । राज्ञा᳚ । राजा॑नम् । अꣳ॒॒श॒भुवेत्यꣳ॑श - भुवा᳚ । घ्न॒न्ति॒ । वैश्ये॑न । वैश्य᳚म् । शू॒द्रेण॑ । शू॒द्रम् । न । वै । इ॒दम् । दिवा᳚ । न । नक्त᳚म् । आ॒सी॒त् । अव्या॑वृत्त॒मित्यवि॑ - आ॒वृ॒त्त॒म् । ते । दे॒वाः । मि॒त्रावरु॑णा॒विति॑ मि॒त्रा - वरु॑णौ । अ॒ब्रु॒व॒न्न् । इ॒दम् । नः॒ । वीति॑ । वा॒स॒य॒त॒म् । इति॑ । तौ । अ॒ब्रू॒ता॒म् । वर᳚म् । वृ॒णा॒व॒है॒ । एकः॑ । ए॒व । आ॒वत् । पूर्वः॑ । ग्रहः॑ । गृ॒ह्या॒तै॒ । इति॑ । तस्मा᳚त् । ऐ॒न्द्र॒वा॒य॒व इत्यै᳚न्द्र - वा॒य॒वः ( ) । पूर्वः॑ । मै॒त्रा॒व॒रु॒णादिति॑ मैत्रा - व॒रु॒णात् । गृ॒ह्य॒ते॒ । प्रा॒णा॒पा॒नाविति॑ प्राण - अ॒पा॒नौ । हि । ए॒तौ । यत् । उ॒पाꣳ॒॒श्व॒न्त॒र्या॒मावित्यु॑पाꣳशु-अ॒न्त॒र्या॒मौ । मि॒त्रः । अहः॑ । अज॑नयत् । वरु॑णः । रात्रि᳚म् । ततः॑ । वै । इ॒दम् । वीति॑ । औ॒च्छ॒त् । यत् । मै॒त्रा॒व॒रु॒ण इति॑ मैत्रा-व॒रु॒णः । गृ॒ह्यते᳚ । व्यु॑ष्ट्या॒ इति॒ वि - उ॒ष्ट्यै॒ ॥  \newline


\textbf{Krama Paata} \newline

ए॒वैषः । ए॒ष मि॒त्राय॑ । मि॒त्राय॑ च । च॒ स॒ह । स॒ह गृ॑ह्यातै । गृ॒ह्या॒ता॒ इति॑ । इति॒ तस्मा᳚त् । तस्मा᳚न् मैत्रावरु॒णः । मै॒त्रा॒व॒रु॒णः स॒ह । मै॒त्रा॒व॒रु॒ण इति॑ मैत्रा - व॒रु॒णः । स॒ह गृ॑ह्यते । गृ॒ह्य॒ते॒ तस्मा᳚त् । तस्मा॒द् राज्ञा᳚ । राज्ञा॒ राजा॑नम् । राजा॑नमꣳश॒भुवा᳚ । अꣳ॒॒श॒भुवा᳚ घ्नन्ति । अꣳ॒॒श॒भुवेत्यꣳ॑श - भुवा᳚ । घ्न॒न्ति॒ वैश्ये॑न । वैश्ये॑न॒ वैश्य᳚म् । वैश्यꣳ॑ शू॒द्रेण॑ । शू॒द्रेण॑ शू॒द्रम् । शू॒द्रम् न । न वै । वा इ॒दम् । इ॒दम् दिवा᳚ । दिवा॒ न । न नक्त᳚म् । नक्त॑मासीत् । आ॒सी॒दव्या॑वृत्तम् । अव्या॑वृत्त॒म् ते । अव्या॑वृत्त॒मित्यवि॑ - आ॒वृ॒त्त॒म् । ते दे॒वाः । दे॒वा मि॒त्रावरु॑णौ । मि॒त्रावरु॑णावब्रुवन्न् । मि॒त्रावरु॑णा॒विति॑ मि॒त्रा - वरु॑णौ । अ॒ब्रु॒व॒न्नि॒दम् । इ॒दम् नः॑ । नो॒ वि । वि वा॑सयतम् । वा॒स॒य॒त॒मिति॑ । इति॒ तौ । ताव॑ब्रूताम् । अ॒ब्रू॒ता॒म् ॅवर᳚म् । वर॑म् ॅवृणावहै । वृ॒णा॒व॒हा॒ एकः॑ । एक॑ ए॒व । ए॒वावत् । आ॒वत् पूर्वः॑ । पूर्वो॒ ग्रहः॑ । ग्रहो॑ गृह्यातै । गृ॒ह्या॒ता॒ इति॑ । इति॒ तस्मा᳚त् । तस्मा॑दैन्द्रवाय॒वः ( ) । ऐ॒न्द्र॒वा॒य॒वः पूर्वः॑ । ऐ॒न्द्र॒वा॒य॒व इत्यै᳚न्द्र - वा॒य॒वः । पूर्वो॑ मैत्रावरु॒णात् । मै॒त्रा॒व॒रु॒णाद् गृ॑ह्यते । मै॒त्रा॒व॒रु॒णादिति॑ मैत्रा - व॒रु॒णात् । गृ॒ह्य॒ते॒ प्रा॒णा॒पा॒नौ । प्रा॒णा॒पा॒नौ हि । प्रा॒णा॒पा॒नाविति॑ प्राण - अ॒पा॒नौ । ह्ये॑तौ । ए॒तौ यत् । यदु॑पाꣳश्वन्तर्या॒मौ । उ॒पाꣳ॒॒श्व॒न्त॒र्या॒मौ मि॒त्रः । उ॒पाꣳ॒॒श्व॒न्त॒र्या॒मावित्यु॑पाꣳशु - अ॒न्त॒र्या॒मौ । मि॒त्रोऽहः॑ । अह॒रज॑नयत् । अज॑नय॒द् वरु॑णः । वरु॑णो॒ रात्रि᳚म् । रात्रि॒म् ततः॑ । ततो॒ वै । वा इ॒दम् । इ॒दम् ॅवि । व्यौ᳚च्छत् । औ॒च्छ॒द् यत् । यन् मै᳚त्रावरु॒णः । मै॒त्रा॒व॒रु॒णो गृ॒ह्यते᳚ । मै॒त्रा॒व॒रु॒ण इति॑ मैत्रा - व॒रु॒णः । गृ॒ह्यते॒ व्यु॑ष्ट्‍यै । व्यु॑ष्ट्‍या॒ इति॒ वि - उ॒ष्ट्‍यै॒ । \newline

\textbf{Jatai Paata} \newline

1. ए॒वैष ए॒ष ए॒वै वैषः । \newline
2. ए॒ष मि॒त्राय॑ मि॒त्रा यै॒ष ए॒ष मि॒त्राय॑ । \newline
3. मि॒त्राय॑ च च मि॒त्राय॑ मि॒त्राय॑ च । \newline
4. च॒ स॒ह स॒ह च॑ च स॒ह । \newline
5. स॒ह गृ॑ह्यातै गृह्यातै स॒ह स॒ह गृ॑ह्यातै । \newline
6. गृ॒ह्या॒ता॒ इतीति॑ गृह्यातै गृह्याता॒ इति॑ । \newline
7. इति॒ तस्मा॒त् तस्मा॒ दितीति॒ तस्मा᳚त् । \newline
8. तस्मा᳚न् मैत्रावरु॒णो मै᳚त्रावरु॒ण स्तस्मा॒त् तस्मा᳚न् मैत्रावरु॒णः । \newline
9. मै॒त्रा॒व॒रु॒णः स॒ह स॒ह मै᳚त्रावरु॒णो मै᳚त्रावरु॒णः स॒ह । \newline
10. मै॒त्रा॒व॒रु॒ण इति॑ मैत्रा - व॒रु॒णः । \newline
11. स॒ह गृ॑ह्यते गृह्यते स॒ह स॒ह गृ॑ह्यते । \newline
12. गृ॒ह्य॒ते॒ तस्मा॒त् तस्मा᳚द् गृह्यते गृह्यते॒ तस्मा᳚त् । \newline
13. तस्मा॒द् राज्ञा॒ राज्ञा॒ तस्मा॒त् तस्मा॒द् राज्ञा᳚ । \newline
14. राज्ञा॒ राजा॑नꣳ॒॒ राजा॑नꣳ॒॒ राज्ञा॒ राज्ञा॒ राजा॑नम् । \newline
15. राजा॑न मꣳश॒भुवा ऽꣳ॑श॒भुवा॒ राजा॑नꣳ॒॒ राजा॑न मꣳश॒भुवा᳚ । \newline
16. अꣳ॒॒श॒भुवा᳚ घ्नन्ति घ्नन्त्यꣳश॒भुवा ऽꣳ॑श॒भुवा᳚ घ्नन्ति । \newline
17. अꣳ॒॒श॒भुवेत्यꣳ॑श - भुवा᳚ । \newline
18. घ्न॒न्ति॒ वैश्ये॑न॒ वैश्ये॑न घ्नन्ति घ्नन्ति॒ वैश्ये॑न । \newline
19. वैश्ये॑न॒ वैश्यं॒ ॅवैश्यं॒ ॅवैश्ये॑न॒ वैश्ये॑न॒ वैश्य᳚म् । \newline
20. वैश्यꣳ॑ शू॒द्रेण॑ शू॒द्रेण॒ वैश्यं॒ ॅवैश्यꣳ॑ शू॒द्रेण॑ । \newline
21. शू॒द्रेण॑ शू॒द्रꣳ शू॒द्रꣳ शू॒द्रेण॑ शू॒द्रेण॑ शू॒द्रम् । \newline
22. शू॒द्रन् न न शू॒द्रꣳ शू॒द्रन् न । \newline
23. न वै वै न न वै । \newline
24. वा इ॒द मि॒दं ॅवै वा इ॒दम् । \newline
25. इ॒दम् दिवा॒ दिवे॒द मि॒दम् दिवा᳚ । \newline
26. दिवा॒ न न दिवा॒ दिवा॒ न । \newline
27. न नक्त॒म् नक्त॒म् न न नक्त᳚म् । \newline
28. नक्त॑ मासी दासी॒न् नक्त॒म् नक्त॑ मासीत् । \newline
29. आ॒सी॒ दव्या॑वृत्त॒ मव्या॑वृत्त मासी दासी॒ दव्या॑वृत्तम् । \newline
30. अव्या॑वृत्त॒म् ते ते ऽव्या॑वृत्त॒ मव्या॑वृत्त॒म् ते । \newline
31. अव्या॑वृत्त॒मित्यवि॑ - आ॒वृ॒त्त॒म् । \newline
32. ते दे॒वा दे॒वा स्ते ते दे॒वाः । \newline
33. दे॒वा मि॒त्रावरु॑णौ मि॒त्रावरु॑णौ दे॒वा दे॒वा मि॒त्रावरु॑णौ । \newline
34. मि॒त्रावरु॑णा वब्रुवन् नब्रुवन् मि॒त्रावरु॑णौ मि॒त्रावरु॑णा वब्रुवन्न् । \newline
35. मि॒त्रावरु॑णा॒विति॑ मि॒त्रा - वरु॑णौ । \newline
36. अ॒ब्रु॒व॒न् नि॒द मि॒द म॑ब्रुवन् नब्रुवन् नि॒दम् । \newline
37. इ॒दन् नो॑ न इ॒द मि॒दन् नः॑ । \newline
38. नो॒ वि वि नो॑ नो॒ वि । \newline
39. वि वा॑सयतं ॅवासयतं॒ ॅवि वि वा॑सयतम् । \newline
40. वा॒स॒य॒त॒ मितीति॑ वासयतं ॅवासयत॒ मिति॑ । \newline
41. इति॒ तौ ता वितीति॒ तौ । \newline
42. ता व॑ब्रूता मब्रूता॒म् तौ ता व॑ब्रूताम् । \newline
43. अ॒ब्रू॒तां॒ ॅवरं॒ ॅवर॑ मब्रूता मब्रूतां॒ ॅवर᳚म् । \newline
44. वरं॑ ॅवृणावहै वृणावहै॒ वरं॒ ॅवरं॑ ॅवृणावहै । \newline
45. वृ॒णा॒व॒हा॒ एक॒ एको॑ वृणावहै वृणावहा॒ एकः॑ । \newline
46. एक॑ ए॒वै वैक॒ एक॑ ए॒व । \newline
47. ए॒वा व दा॒व दे॒वै वावत् । \newline
48. आ॒वत् पूर्वः॒ पूर्व॑ आ॒व दा॒वत् पूर्वः॑ । \newline
49. पूर्वो॒ ग्रहो॒ ग्रहः॒ पूर्वः॒ पूर्वो॒ ग्रहः॑ । \newline
50. ग्रहो॑ गृह्यातै गृह्यातै॒ ग्रहो॒ ग्रहो॑ गृह्यातै । \newline
51. गृ॒ह्या॒ता॒ इतीति॑ गृह्यातै गृह्याता॒ इति॑ । \newline
52. इति॒ तस्मा॒त् तस्मा॒ दितीति॒ तस्मा᳚त् । \newline
53. तस्मा॑ दैन्द्रवाय॒व ऐ᳚न्द्रवाय॒व स्तस्मा॒त् तस्मा॑ दैन्द्रवाय॒वः । \newline
54. ऐ॒न्द्र॒वा॒य॒वः पूर्वः॒ पूर्व॑ ऐन्द्रवाय॒व ऐ᳚न्द्रवाय॒वः पूर्वः॑ । \newline
55. ऐ॒न्द्र॒वा॒य॒व इत्यै᳚न्द्र - वा॒य॒वः । \newline
56. पूर्वो॑ मैत्रावरु॒णान् मै᳚त्रावरु॒णात् पूर्वः॒ पूर्वो॑ मैत्रावरु॒णात् । \newline
57. मै॒त्रा॒व॒रु॒णाद् गृ॑ह्यते गृह्यते मैत्रावरु॒णान् मै᳚त्रावरु॒णाद् गृ॑ह्यते । \newline
58. मै॒त्रा॒व॒रु॒णादिति॑ मैत्रा - व॒रु॒णात् । \newline
59. गृ॒ह्य॒ते॒ प्रा॒णा॒पा॒नौ प्रा॑णापा॒नौ गृ॑ह्यते गृह्यते प्राणापा॒नौ । \newline
60. प्रा॒णा॒पा॒नौ हि हि प्रा॑णापा॒नौ प्रा॑णापा॒नौ हि । \newline
61. प्रा॒णा॒पा॒नाविति॑ प्राण - अ॒पा॒नौ । \newline
62. ह्ये॑ता वे॒तौ हि ह्ये॑तौ । \newline
63. ए॒तौ यद् यदे॒ता वे॒तौ यत् । \newline
64. यदु॑पाꣳश्वन्तर्या॒मा वु॑पाꣳश्वन्तर्या॒मौ यद् यदु॑पाꣳश्वन्तर्या॒मौ । \newline
65. उ॒पाꣳ॒॒श्व॒न्त॒र्या॒मौ मि॒त्रो मि॒त्र उ॑पाꣳश्वन्तर्या॒मा वु॑पाꣳश्वन्तर्या॒मौ मि॒त्रः । \newline
66. उ॒पाꣳ॒॒श्व॒न्त॒र्या॒मावित्यु॑पाꣳशु - अ॒न्त॒र्या॒मौ । \newline
67. मि॒त्रो ऽह॒ रह॑र् मि॒त्रो मि॒त्रो ऽहः॑ । \newline
68. अह॒ रज॑नय॒ दज॑नय॒ दह॒ रह॒ रज॑नयत् । \newline
69. अज॑नय॒द् वरु॑णो॒ वरु॒णो ऽज॑नय॒ दज॑नय॒द् वरु॑णः । \newline
70. वरु॑णो॒ रात्रिꣳ॒॒ रात्रिं॒ ॅवरु॑णो॒ वरु॑णो॒ रात्रि᳚म् । \newline
71. रात्रि॒म् तत॒ स्ततो॒ रात्रिꣳ॒॒ रात्रि॒म् ततः॑ । \newline
72. ततो॒ वै वै तत॒ स्ततो॒ वै । \newline
73. वा इ॒द मि॒दं ॅवै वा इ॒दम् । \newline
74. इ॒दं ॅवि वीद मि॒दं ॅवि । \newline
75. व्यौ᳚च्छ दौच्छ॒द् वि व्यौ᳚च्छत् । \newline
76. औ॒च्छ॒द् यद् यदौ᳚च्छ दौच्छ॒द् यत् । \newline
77. यन् मै᳚त्रावरु॒णो मै᳚त्रावरु॒णो यद् यन् मै᳚त्रावरु॒णः । \newline
78. मै॒त्रा॒व॒रु॒णो गृ॒ह्यते॑ गृ॒ह्यते॑ मैत्रावरु॒णो मै᳚त्रावरु॒णो गृ॒ह्यते᳚ । \newline
79. मै॒त्रा॒व॒रु॒ण इति॑ मैत्रा - व॒रु॒णः । \newline
80. गृ॒ह्यते॒ व्यु॑ष्ट्यै॒ व्यु॑ष्ट्यै गृ॒ह्यते॑ गृ॒ह्यते॒ व्यु॑ष्ट्यै । \newline
81. व्यु॑ष्ट्या॒ इति॒ वि - उ॒ष्ट्यै॒ । \newline

\textbf{Ghana Paata } \newline

1. ए॒वैष ए॒ष ए॒वै वैष मि॒त्राय॑ मि॒त्रायै॒ष ए॒वै वैष मि॒त्राय॑ । \newline
2. ए॒ष मि॒त्राय॑ मि॒त्रा यै॒ष ए॒ष मि॒त्राय॑ च च मि॒त्रा यै॒ष ए॒ष मि॒त्राय॑ च । \newline
3. मि॒त्राय॑ च च मि॒त्राय॑ मि॒त्राय॑ च स॒ह स॒ह च॑ मि॒त्राय॑ मि॒त्राय॑ च स॒ह । \newline
4. च॒ स॒ह स॒ह च॑ च स॒ह गृ॑ह्यातै गृह्यातै स॒ह च॑ च स॒ह गृ॑ह्यातै । \newline
5. स॒ह गृ॑ह्यातै गृह्यातै स॒ह स॒ह गृ॑ह्याता॒ इतीति॑ गृह्यातै स॒ह स॒ह गृ॑ह्याता॒ इति॑ । \newline
6. गृ॒ह्या॒ता॒ इतीति॑ गृह्यातै गृह्याता॒ इति॒ तस्मा॒त् तस्मा॒ दिति॑ गृह्यातै गृह्याता॒ इति॒ तस्मा᳚त् । \newline
7. इति॒ तस्मा॒त् तस्मा॒ दितीति॒ तस्मा᳚न् मैत्रावरु॒णो मै᳚त्रावरु॒ण स्तस्मा॒ दितीति॒ तस्मा᳚न् मैत्रावरु॒णः । \newline
8. तस्मा᳚न् मैत्रावरु॒णो मै᳚त्रावरु॒ण स्तस्मा॒त् तस्मा᳚न् मैत्रावरु॒णः स॒ह स॒ह मै᳚त्रावरु॒ण स्तस्मा॒त् तस्मा᳚न् मैत्रावरु॒णः स॒ह । \newline
9. मै॒त्रा॒व॒रु॒णः स॒ह स॒ह मै᳚त्रावरु॒णो मै᳚त्रावरु॒णः स॒ह गृ॑ह्यते गृह्यते स॒ह मै᳚त्रावरु॒णो मै᳚त्रावरु॒णः स॒ह गृ॑ह्यते । \newline
10. मै॒त्रा॒व॒रु॒ण इति॑ मैत्रा - व॒रु॒णः । \newline
11. स॒ह गृ॑ह्यते गृह्यते स॒ह स॒ह गृ॑ह्यते॒ तस्मा॒त् तस्मा᳚द् गृह्यते स॒ह स॒ह गृ॑ह्यते॒ तस्मा᳚त् । \newline
12. गृ॒ह्य॒ते॒ तस्मा॒त् तस्मा᳚द् गृह्यते गृह्यते॒ तस्मा॒द् राज्ञा॒ राज्ञा॒ तस्मा᳚द् गृह्यते गृह्यते॒ तस्मा॒द् राज्ञा᳚ । \newline
13. तस्मा॒द् राज्ञा॒ राज्ञा॒ तस्मा॒त् तस्मा॒द् राज्ञा॒ राजा॑नꣳ॒॒ राजा॑नꣳ॒॒ राज्ञा॒ तस्मा॒त् तस्मा॒द् राज्ञा॒ राजा॑नम् । \newline
14. राज्ञा॒ राजा॑नꣳ॒॒ राजा॑नꣳ॒॒ राज्ञा॒ राज्ञा॒ राजा॑न मꣳश॒भुवा ऽꣳ॑श॒भुवा॒ राजा॑नꣳ॒॒ राज्ञा॒ राज्ञा॒ राजा॑न मꣳश॒भुवा᳚ । \newline
15. राजा॑न मꣳश॒भुवा ऽꣳ॑श॒भुवा॒ राजा॑नꣳ॒॒ राजा॑न मꣳश॒भुवा᳚ घ्नन्ति घ्नन्त्यꣳश॒भुवा॒ राजा॑नꣳ॒॒ राजा॑न मꣳश॒भुवा᳚ घ्नन्ति । \newline
16. अꣳ॒॒श॒भुवा᳚ घ्नन्ति घ्नन्त्यꣳश॒भुवा ऽꣳ॑श॒भुवा᳚ घ्नन्ति॒ वैश्ये॑न॒ वैश्ये॑न 
घ्नन्त्यꣳश॒भुवा ऽꣳ॑श॒भुवा᳚ घ्नन्ति॒ वैश्ये॑न । \newline
17. अꣳ॒॒श॒भुवेत्यꣳ॑श - भुवा᳚ । \newline
18. घ्न॒न्ति॒ वैश्ये॑न॒ वैश्ये॑न घ्नन्ति घ्नन्ति॒ वैश्ये॑न॒ वैश्यं॒ ॅवैश्यं॒ ॅवैश्ये॑न घ्नन्ति घ्नन्ति॒ वैश्ये॑न॒ वैश्य᳚म् । \newline
19. वैश्ये॑न॒ वैश्यं॒ ॅवैश्यं॒ ॅवैश्ये॑न॒ वैश्ये॑न॒ वैश्यꣳ॑ शू॒द्रेण॑ शू॒द्रेण॒ वैश्यं॒ ॅवैश्ये॑न॒ वैश्ये॑न॒ वैश्यꣳ॑ शू॒द्रेण॑ । \newline
20. वैश्यꣳ॑ शू॒द्रेण॑ शू॒द्रेण॒ वैश्यं॒ ॅवैश्यꣳ॑ शू॒द्रेण॑ शू॒द्रꣳ शू॒द्रꣳ शू॒द्रेण॒ वैश्यं॒ ॅवैश्यꣳ॑ शू॒द्रेण॑ शू॒द्रम् । \newline
21. शू॒द्रेण॑ शू॒द्रꣳ शू॒द्रꣳ शू॒द्रेण॑ शू॒द्रेण॑ शू॒द्रन् न न शू॒द्रꣳ शू॒द्रेण॑ शू॒द्रेण॑ शू॒द्रन् न । \newline
22. शू॒द्रन् न न शू॒द्रꣳ शू॒द्रन् न वै वै न शू॒द्रꣳ शू॒द्रन् न वै । \newline
23. न वै वै न न वा इ॒द मि॒दं ॅवै न न वा इ॒दम् । \newline
24. वा इ॒द मि॒दं ॅवै वा इ॒दम् दिवा॒ दिवे॒दं ॅवै वा इ॒दम् दिवा᳚ । \newline
25. इ॒दम् दिवा॒ दिवे॒द मि॒दम् दिवा॒ न न दिवे॒द मि॒दम् दिवा॒ न । \newline
26. दिवा॒ न न दिवा॒ दिवा॒ न नक्त॒म् नक्त॒न् न दिवा॒ दिवा॒ न नक्त᳚म् । \newline
27. न नक्त॒म् नक्त॒न् न न नक्त॑ मासी दासी॒न् नक्त॒न् न न नक्त॑ मासीत् । \newline
28. नक्त॑ मासी दासी॒न् नक्त॒न् नक्त॑ मासी॒ दव्या॑वृत्त॒ मव्या॑वृत्त मासी॒न् नक्त॒म् नक्त॑ मासी॒ दव्या॑वृत्तम् । \newline
29. आ॒सी॒ दव्या॑वृत्त॒ मव्या॑वृत्त मासी दासी॒ दव्या॑वृत्त॒म् ते ते ऽव्या॑वृत्त मासी दासी॒ दव्या॑वृत्त॒म् ते । \newline
30. अव्या॑वृत्त॒म् ते ते ऽव्या॑वृत्त॒ मव्या॑वृत्त॒म् ते दे॒वा दे॒वा स्ते ऽव्या॑वृत्त॒ मव्या॑वृत्त॒म् ते दे॒वाः । \newline
31. अव्या॑वृत्त॒मित्यवि॑ - आ॒वृ॒त्त॒म् । \newline
32. ते दे॒वा दे॒वा स्ते ते दे॒वा मि॒त्रावरु॑णौ मि॒त्रावरु॑णौ दे॒वा स्ते ते दे॒वा मि॒त्रावरु॑णौ । \newline
33. दे॒वा मि॒त्रावरु॑णौ मि॒त्रावरु॑णौ दे॒वा दे॒वा मि॒त्रावरु॑णा वब्रुवन् नब्रुवन् मि॒त्रावरु॑णौ दे॒वा दे॒वा मि॒त्रावरु॑णा वब्रुवन्न् । \newline
34. मि॒त्रावरु॑णा वब्रुवन् नब्रुवन् मि॒त्रावरु॑णौ मि॒त्रावरु॑णा वब्रुवन् नि॒द मि॒द म॑ब्रुवन् मि॒त्रावरु॑णौ मि॒त्रावरु॑णा वब्रुवन् नि॒दम् । \newline
35. मि॒त्रावरु॑णा॒विति॑ मि॒त्रा - वरु॑णौ । \newline
36. अ॒ब्रु॒व॒न् नि॒द मि॒द म॑ब्रुवन् नब्रुवन् नि॒दन् नो॑ न इ॒द म॑ब्रुवन् नब्रुवन् नि॒दन् नः॑ । \newline
37. इ॒दन् नो॑ न इ॒द मि॒दन् नो॒ वि वि न॑ इ॒द मि॒दन् नो॒ वि । \newline
38. नो॒ वि वि नो॑ नो॒ वि वा॑सयतं ॅवासयतं॒ ॅवि नो॑ नो॒ वि वा॑सयतम् । \newline
39. वि वा॑सयतं ॅवासयतं॒ ॅवि वि वा॑सयत॒ मितीति॑ वासयतं॒ ॅवि वि वा॑सयत॒ मिति॑ । \newline
40. वा॒स॒य॒त॒ मितीति॑ वासयतं ॅवासयत॒ मिति॒ तौ ता विति॑ वासयतं ॅवासयत॒ मिति॒ तौ । \newline
41. इति॒ तौ ता वितीति॒ ता व॑ब्रूता मब्रूता॒म् ता वितीति॒ ता व॑ब्रूताम् । \newline
42. ता व॑ब्रूता मब्रूता॒म् तौ ता व॑ब्रूतां॒ ॅवरं॒ ॅवर॑ मब्रूता॒म् तौ ता व॑ब्रूतां॒ ॅवर᳚म् । \newline
43. अ॒ब्रू॒तां॒ ॅवरं॒ ॅवर॑ मब्रूता मब्रूतां॒ ॅवरं॑ ॅवृणावहै वृणावहै॒ वर॑ मब्रूता मब्रूतां॒ ॅवरं॑ ॅवृणावहै । \newline
44. वरं॑ ॅवृणावहै वृणावहै॒ वरं॒ ॅवरं॑ ॅवृणावहा॒ एक॒ एको॑ वृणावहै॒ वरं॒ ॅवरं॑ ॅवृणावहा॒ एकः॑ । \newline
45. वृ॒णा॒व॒हा॒ एक॒ एको॑ वृणावहै वृणावहा॒ एक॑ ए॒वै वैको॑ वृणावहै वृणावहा॒ एक॑ ए॒व । \newline
46. एक॑ ए॒वै वैक॒ एक॑ ए॒वाव दा॒व दे॒वैक॒ एक॑ ए॒वावत् । \newline
47. ए॒वाव दा॒व दे॒वै वावत् पूर्वः॒ पूर्व॑ आ॒व दे॒वै वावत् पूर्वः॑ । \newline
48. आ॒वत् पूर्वः॒ पूर्व॑ आ॒व दा॒वत् पूर्वो॒ ग्रहो॒ ग्रहः॒ पूर्व॑ आ॒व दा॒वत् पूर्वो॒ ग्रहः॑ । \newline
49. पूर्वो॒ ग्रहो॒ ग्रहः॒ पूर्वः॒ पूर्वो॒ ग्रहो॑ गृह्यातै गृह्यातै॒ ग्रहः॒ पूर्वः॒ पूर्वो॒ ग्रहो॑ गृह्यातै । \newline
50. ग्रहो॑ गृह्यातै गृह्यातै॒ ग्रहो॒ ग्रहो॑ गृह्याता॒ इतीति॑ गृह्यातै॒ ग्रहो॒ ग्रहो॑ गृह्याता॒ इति॑ । \newline
51. गृ॒ह्या॒ता॒ इतीति॑ गृह्यातै गृह्याता॒ इति॒ तस्मा॒त् तस्मा॒ दिति॑ गृह्यातै गृह्याता॒ इति॒ तस्मा᳚त् । \newline
52. इति॒ तस्मा॒त् तस्मा॒ दितीति॒ तस्मा॑ दैन्द्रवाय॒व ऐ᳚न्द्रवाय॒व स्तस्मा॒ दितीति॒ तस्मा॑ दैन्द्रवाय॒वः । \newline
53. तस्मा॑ दैन्द्रवाय॒व ऐ᳚न्द्रवाय॒व स्तस्मा॒त् तस्मा॑ दैन्द्रवाय॒वः पूर्वः॒ पूर्व॑ ऐन्द्रवाय॒व स्तस्मा॒त् तस्मा॑ दैन्द्रवाय॒वः पूर्वः॑ । \newline
54. ऐ॒न्द्र॒वा॒य॒वः पूर्वः॒ पूर्व॑ ऐन्द्रवाय॒व ऐ᳚न्द्रवाय॒वः पूर्वो॑ मैत्रावरु॒णान् मै᳚त्रावरु॒णात् पूर्व॑ ऐन्द्रवाय॒व ऐ᳚न्द्रवाय॒वः पूर्वो॑ मैत्रावरु॒णात् । \newline
55. ऐ॒न्द्र॒वा॒य॒व इत्यै᳚न्द्र - वा॒य॒वः । \newline
56. पूर्वो॑ मैत्रावरु॒णान् मै᳚त्रावरु॒णात् पूर्वः॒ पूर्वो॑ मैत्रावरु॒णाद् गृ॑ह्यते गृह्यते मैत्रावरु॒णात् पूर्वः॒ पूर्वो॑ मैत्रावरु॒णाद् गृ॑ह्यते । \newline
57. मै॒त्रा॒व॒रु॒णाद् गृ॑ह्यते गृह्यते मैत्रावरु॒णान् मै᳚त्रावरु॒णाद् गृ॑ह्यते प्राणापा॒नौ प्रा॑णापा॒नौ गृ॑ह्यते मैत्रावरु॒णान् मै᳚त्रावरु॒णाद् गृ॑ह्यते प्राणापा॒नौ । \newline
58. मै॒त्रा॒व॒रु॒णादिति॑ मैत्रा - व॒रु॒णात् । \newline
59. गृ॒ह्य॒ते॒ प्रा॒णा॒पा॒नौ प्रा॑णापा॒नौ गृ॑ह्यते गृह्यते प्राणापा॒नौ हि हि प्रा॑णापा॒नौ गृ॑ह्यते गृह्यते प्राणापा॒नौ हि । \newline
60. प्रा॒णा॒पा॒नौ हि हि प्रा॑णापा॒नौ प्रा॑णापा॒नौ ह्ये॑ता वे॒तौ हि प्रा॑णापा॒नौ प्रा॑णापा॒नौ ह्ये॑तौ । \newline
61. प्रा॒णा॒पा॒नाविति॑ प्राण - अ॒पा॒नौ । \newline
62. ह्ये॑ता वे॒तौ हि ह्ये॑तौ यद् यदे॒तौ हि ह्ये॑तौ यत् । \newline
63. ए॒तौ यद् यदे॒ता वे॒तौ यदु॑पाꣳश्वन्तर्या॒मा वु॑पाꣳश्वन्तर्या॒मौ यदे॒ता वे॒तौ यदु॑पाꣳश्वन्तर्या॒मौ । \newline
64. यदु॑पाꣳश्वन्तर्या॒मा वु॑पाꣳश्वन्तर्या॒मौ यद् यदु॑पाꣳश्वन्तर्या॒मौ मि॒त्रो मि॒त्र उ॑पाꣳश्वन्तर्या॒मौ यद् यदु॑पाꣳश्वन्तर्या॒मौ मि॒त्रः । \newline
65. उ॒पाꣳ॒॒श्व॒न्त॒र्या॒मौ मि॒त्रो मि॒त्र उ॑पाꣳश्वन्तर्या॒मा वु॑पाꣳश्वन्तर्या॒मौ मि॒त्रो ऽह॒ रह॑र् मि॒त्र उ॑पाꣳश्वन्तर्या॒मा वु॑पाꣳश्वन्तर्या॒मौ मि॒त्रो ऽहः॑ । \newline
66. उ॒पाꣳ॒॒श्व॒न्त॒र्या॒मावित्यु॑पाꣳशु - अ॒न्त॒र्या॒मौ । \newline
67. मि॒त्रो ऽह॒ रह॑र् मि॒त्रो मि॒त्रो ऽह॒ रज॑नय॒ दज॑नय॒ दह॑र् मि॒त्रो मि॒त्रो ऽह॒ रज॑नयत् । \newline
68. अह॒ रज॑नय॒ दज॑नय॒ दह॒ रह॒ रज॑नय॒द् वरु॑णो॒ वरु॒णो ऽज॑नय॒ दह॒ रह॒ रज॑नय॒द् वरु॑णः । \newline
69. अज॑नय॒द् वरु॑णो॒ वरु॒णो ऽज॑नय॒ दज॑नय॒द् वरु॑णो॒ रात्रिꣳ॒॒ रात्रिं॒ ॅवरु॒णो ऽज॑नय॒ दज॑नय॒द् वरु॑णो॒ रात्रि᳚म् । \newline
70. वरु॑णो॒ रात्रिꣳ॒॒ रात्रिं॒ ॅवरु॑णो॒ वरु॑णो॒ रात्रि॒म् तत॒ स्ततो॒ रात्रिं॒ ॅवरु॑णो॒ वरु॑णो॒ रात्रि॒म् ततः॑ । \newline
71. रात्रि॒म् तत॒ स्ततो॒ रात्रिꣳ॒॒ रात्रि॒म् ततो॒ वै वै ततो॒ रात्रिꣳ॒॒ रात्रि॒म् ततो॒ वै । \newline
72. ततो॒ वै वै तत॒ स्ततो॒ वा इ॒द मि॒दं ॅवै तत॒ स्ततो॒ वा इ॒दम् । \newline
73. वा इ॒द मि॒दं ॅवै वा इ॒दं ॅवि वीदं ॅवै वा इ॒दं ॅवि । \newline
74. इ॒दं ॅवि वीद मि॒दं ॅव्यौ᳚च्छ दौच्छ॒द् वीद मि॒दं ॅव्यौ᳚च्छत् । \newline
75. व्यौ᳚च्छ दौच्छ॒द् वि व्यौ᳚च्छ॒द् यद् यदौ᳚च्छ॒द् वि व्यौ᳚च्छ॒द् यत् । \newline
76. औ॒च्छ॒द् यद् यदौ᳚च्छ दौच्छ॒द् यन् मै᳚त्रावरु॒णो मै᳚त्रावरु॒णो यदौ᳚च्छ दौच्छ॒द् यन् मै᳚त्रावरु॒णः । \newline
77. यन् मै᳚त्रावरु॒णो मै᳚त्रावरु॒णो यद् यन् मै᳚त्रावरु॒णो गृ॒ह्यते॑ गृ॒ह्यते॑ मैत्रावरु॒णो यद् यन् मै᳚त्रावरु॒णो गृ॒ह्यते᳚ । \newline
78. मै॒त्रा॒व॒रु॒णो गृ॒ह्यते॑ गृ॒ह्यते॑ मैत्रावरु॒णो मै᳚त्रावरु॒णो गृ॒ह्यते॒ व्यु॑ष्ट्यै॒ व्यु॑ष्ट्यै गृ॒ह्यते॑ मैत्रावरु॒णो मै᳚त्रावरु॒णो गृ॒ह्यते॒ व्यु॑ष्ट्यै । \newline
79. मै॒त्रा॒व॒रु॒ण इति॑ मैत्रा - व॒रु॒णः । \newline
80. गृ॒ह्यते॒ व्यु॑ष्ट्यै॒ व्यु॑ष्ट्यै गृ॒ह्यते॑ गृ॒ह्यते॒ व्यु॑ष्ट्यै । \newline
81. व्यु॑ष्ट्या॒ इति॒ वि - उ॒ष्ट्यै॒ । \newline
\pagebreak
\markright{ TS 6.4.9.1  \hfill https://www.vedavms.in \hfill}

\section{ TS 6.4.9.1 }

\textbf{TS 6.4.9.1 } \newline
\textbf{Samhita Paata} \newline

य॒ज्ञ्स्य॒ शिरो᳚ऽच्छिद्यत॒ ते दे॒वा अ॒श्विना॑वब्रुवन् भि॒षजौ॒ वै स्थ॑ इ॒दं ॅय॒ज्ञ्स्य॒ शिरः॒ प्रति॑ धत्त॒मिति॒ ताव॑ब्रूतां॒ ॅवरं॑ ॅवृणावहै॒ ग्रह॑ ए॒व ना॒वत्रापि॑ गृह्यता॒मिति॒ ताभ्या॑-मे॒तमा᳚श्वि॒न-म॑गृह्ण॒न् ततो॒ वै तौ य॒ज्ञ्स्य॒ शिरः॒ प्रत्य॑धत्तां॒ ॅयदा᳚श्वि॒नो गृ॒ह्यते॑ य॒ज्ञ्स्य॒ निष्कृ॑त्यै॒ तौ दे॒वा अ॑ब्रुव॒न्नपू॑तौ॒ वा इ॒मौ म॑नुष्यच॒रौ- [  ] \newline

\textbf{Pada Paata} \newline

य॒ज्ञ्स्य॑ । शिरः॑ । अ॒च्छि॒द्य॒त॒ । ते । दे॒वाः । अ॒श्विनौ᳚ । अ॒ब्रु॒व॒न्न् । भि॒षजौ᳚ । वै । स्थः॒ । इ॒दम् । य॒ज्ञ्स्य॑ । शिरः॑ । प्रतीति॑ । ध॒त्त॒म् । इति॑ । तौ । अ॒ब्रू॒ता॒म् । वर᳚म् । वृ॒णा॒व॒है॒ । ग्रहः॑ । ए॒व । नौ॒ । अत्र॑ । अपीति॑ । गृ॒ह्य॒ता॒म् । इति॑ । ताभ्या᳚म् । ए॒तम् । आ॒श्वि॒नम् । अ॒गृ॒ह्ण॒न्न् । ततः॑ । वै । तौ । य॒ज्ञ्स्य॑ । शिरः॑ । प्रतीति॑ । अ॒ध॒त्ता॒म् । यत् । आ॒श्वि॒नः । गृ॒ह्यते᳚ । य॒ज्ञ्स्य॑ । निष्कृ॑त्या॒ इति॒ निः - कृ॒त्यै॒ । तौ । दे॒वाः । अ॒ब्रु॒व॒न्न् । अपू॑तौ । वै । इ॒मौ । म॒नु॒ष्य॒च॒राविति॑ मनुष्य - च॒रौ ।  \newline


\textbf{Krama Paata} \newline

य॒ज्ञ्स्य॒ शिरः॑ । शिरो᳚ऽच्छिद्यत । अ॒च्छि॒द्य॒त॒ ते । ते दे॒वाः । दे॒वा अ॒श्विनौ᳚ । अ॒श्विना॑वब्रुवन्न् । अ॒ब्रु॒व॒न् भि॒षजौ᳚ । भि॒षजौ॒ वै । वै स्थः॑ । स्थ॒ इ॒दम् । इ॒दम् ॅय॒ज्ञ्स्य॑ । य॒ज्ञ्स्य॒ शिरः॑ । शिरः॒ प्रति॑ । प्रति॑ धत्तम् । ध॒त्त॒मिति॑ । इति॒ तौ । ताव॑ब्रूताम् । अ॒ब्रू॒ता॒म् ॅवर᳚म् । वर॑म् ॅवृणावहै । वृ॒णा॒व॒है॒ ग्रहः॑ । ग्रह॑ ए॒व । ए॒व नौ᳚ । ना॒वत्र॑ । अत्रापि॑ । अपि॑ गृह्यताम् । गृ॒ह्य॒ता॒मिति॑ । इति॒ ताभ्या᳚म् । ताभ्या॑मे॒तम् । ए॒तमा᳚श्वि॒नम् । आ॒श्वि॒नम॑गृह्णन्न् । अ॒गृ॒ह्ण॒न् ततः॑ । ततो॒ वै । वै तौ । तौ य॒ज्ञ्स्य॑ । य॒ज्ञ्स्य॒ शिरः॑ । शिरः॒ प्रति॑ । प्रत्य॑धत्ताम् । अ॒ध॒त्ता॒म् ॅयत् । यदा᳚श्वि॒नः । आ॒श्वि॒नो गृ॒ह्यते᳚ । गृ॒ह्यते॑ य॒ज्ञ्स्य॑ । य॒ज्ञ्स्य॒ निष्कृ॑त्यै । निष्कृ॑त्यै॒ तौ । निष्कृ॑त्या॒ इति॒ निः - कृ॒त्यै॒ । तौ दे॒वाः । दे॒वा अ॑ब्रुवन्न् । अ॒ब्रु॒व॒न्नपू॑तौ । अपू॑तौ॒ वै । वा इ॒मौ । इ॒मौ म॑नुष्यच॒रौ । म॒नु॒ष्य॒च॒रौ भि॒षजौ᳚ । म॒नु॒ष्य॒च॒राविति॑ मनुष्य - च॒रौ \newline

\textbf{Jatai Paata} \newline

1. य॒ज्ञ्स्य॒ शिरः॒ शिरो॑ य॒ज्ञ्स्य॑ य॒ज्ञ्स्य॒ शिरः॑ । \newline
2. शिरो᳚ ऽच्छिद्यता च्छिद्यत॒ शिरः॒ शिरो᳚ ऽच्छिद्यत । \newline
3. अ॒च्छि॒द्य॒त॒ ते ते᳚ ऽच्छिद्यता च्छिद्यत॒ ते । \newline
4. ते दे॒वा दे॒वा स्ते ते दे॒वाः । \newline
5. दे॒वा अ॒श्विना॑ व॒श्विनौ॑ दे॒वा दे॒वा अ॒श्विनौ᳚ । \newline
6. अ॒श्विना॑ वब्रुवन् नब्रुवन् न॒श्विना॑ व॒श्विना॑ वब्रुवन्न् । \newline
7. अ॒ब्रु॒व॒न् भि॒षजौ॑ भि॒षजा॑ वब्रुवन् नब्रुवन् भि॒षजौ᳚ । \newline
8. भि॒षजौ॒ वै वै भि॒षजौ॑ भि॒षजौ॒ वै । \newline
9. वै स्थः॑ स्थो॒ वै वै स्थः॑ । \newline
10. स्थ॒ इ॒द मि॒दꣳ स्थः॑ स्थ इ॒दम् । \newline
11. इ॒दं ॅय॒ज्ञ्स्य॑ य॒ज्ञ्स्ये॒द मि॒दं ॅय॒ज्ञ्स्य॑ । \newline
12. य॒ज्ञ्स्य॒ शिरः॒ शिरो॑ य॒ज्ञ्स्य॑ य॒ज्ञ्स्य॒ शिरः॑ । \newline
13. शिरः॒ प्रति॒ प्रति॒ शिरः॒ शिरः॒ प्रति॑ । \newline
14. प्रति॑ धत्तम् धत्त॒म् प्रति॒ प्रति॑ धत्तम् । \newline
15. ध॒त्त॒ मितीति॑ धत्तम् धत्त॒ मिति॑ । \newline
16. इति॒ तौ ता वितीति॒ तौ । \newline
17. ता व॑ब्रूता मब्रूता॒म् तौ ता व॑ब्रूताम् । \newline
18. अ॒ब्रू॒तां॒ ॅवरं॒ ॅवर॑ मब्रूता मब्रूतां॒ ॅवर᳚म् । \newline
19. वरं॑ ॅवृणावहै वृणावहै॒ वरं॒ ॅवरं॑ ॅवृणावहै । \newline
20. वृ॒णा॒व॒है॒ ग्रहो॒ ग्रहो॑ वृणावहै वृणावहै॒ ग्रहः॑ । \newline
21. ग्रह॑ ए॒वैव ग्रहो॒ ग्रह॑ ए॒व । \newline
22. ए॒व नौ॑ ना वे॒वैव नौ᳚ । \newline
23. ना॒ वत्रात्र॑ नौ ना॒ वत्र॑ । \newline
24. अत्रा प्यप्य त्रा त्रापि॑ । \newline
25. अपि॑ गृह्यताम् गृह्यता॒ मप्यपि॑ गृह्यताम् । \newline
26. गृ॒ह्य॒ता॒ मितीति॑ गृह्यताम् गृह्यता॒ मिति॑ । \newline
27. इति॒ ताभ्या॒म् ताभ्या॒ मितीति॒ ताभ्या᳚म् । \newline
28. ताभ्या॑ मे॒त मे॒तम् ताभ्या॒म् ताभ्या॑ मे॒तम् । \newline
29. ए॒त मा᳚श्वि॒न मा᳚श्वि॒न मे॒त मे॒त मा᳚श्वि॒नम् । \newline
30. आ॒श्वि॒न म॑गृह्णन् नगृह्णन् नाश्वि॒न मा᳚श्वि॒न म॑गृह्णन्न् । \newline
31. अ॒गृ॒ह्ण॒न् तत॒ स्ततो॑ ऽगृह्णन् नगृह्ण॒न् ततः॑ । \newline
32. ततो॒ वै वै तत॒ स्ततो॒ वै । \newline
33. वै तौ तौ वै वै तौ । \newline
34. तौ य॒ज्ञ्स्य॑ य॒ज्ञ्स्य॒ तौ तौ य॒ज्ञ्स्य॑ । \newline
35. य॒ज्ञ्स्य॒ शिरः॒ शिरो॑ य॒ज्ञ्स्य॑ य॒ज्ञ्स्य॒ शिरः॑ । \newline
36. शिरः॒ प्रति॒ प्रति॒ शिरः॒ शिरः॒ प्रति॑ । \newline
37. प्रत्य॑ धत्ता मधत्ता॒म् प्रति॒ प्रत्य॑ धत्ताम् । \newline
38. अ॒ध॒त्तां॒ ॅयद् यद॑धत्ता मधत्तां॒ ॅयत् । \newline
39. यदा᳚श्वि॒न आ᳚श्वि॒नो यद् यदा᳚श्वि॒नः । \newline
40. आ॒श्वि॒नो गृ॒ह्यते॑ गृ॒ह्यत॑ आश्वि॒न आ᳚श्वि॒नो गृ॒ह्यते᳚ । \newline
41. गृ॒ह्यते॑ य॒ज्ञ्स्य॑ य॒ज्ञ्स्य॑ गृ॒ह्यते॑ गृ॒ह्यते॑ य॒ज्ञ्स्य॑ । \newline
42. य॒ज्ञ्स्य॒ निष्कृ॑त्यै॒ निष्कृ॑त्यै य॒ज्ञ्स्य॑ य॒ज्ञ्स्य॒ निष्कृ॑त्यै । \newline
43. निष्कृ॑त्यै॒ तौ तौ निष्कृ॑त्यै॒ निष्कृ॑त्यै॒ तौ । \newline
44. निष्कृ॑त्या॒ इति॒ निः - कृ॒त्यै॒ । \newline
45. तौ दे॒वा दे॒वा स्तौ तौ दे॒वाः । \newline
46. दे॒वा अ॑ब्रुवन् नब्रुवन् दे॒वा दे॒वा अ॑ब्रुवन्न् । \newline
47. अ॒ब्रु॒व॒न् नपू॑ता॒ वपू॑ता वब्रुवन् नब्रुव॒न् नपू॑तौ । \newline
48. अपू॑तौ॒ वै वा अपू॑ता॒ वपू॑तौ॒ वै । \newline
49. वा इ॒मा वि॒मौ वै वा इ॒मौ । \newline
50. इ॒मौ म॑नुष्यच॒रौ म॑नुष्यच॒रा वि॒मा वि॒मौ म॑नुष्यच॒रौ । \newline
51. म॒नु॒ष्य॒च॒रौ भि॒षजौ॑ भि॒षजौ॑ मनुष्यच॒रौ म॑नुष्यच॒रौ भि॒षजौ᳚ । \newline
52. म॒नु॒ष्य॒च॒राविति॑ मनुष्य - च॒रौ । \newline

\textbf{Ghana Paata } \newline

1. य॒ज्ञ्स्य॒ शिरः॒ शिरो॑ य॒ज्ञ्स्य॑ य॒ज्ञ्स्य॒ शिरो᳚ ऽच्छिद्यता च्छिद्यत॒ शिरो॑ य॒ज्ञ्स्य॑ य॒ज्ञ्स्य॒ शिरो᳚ ऽच्छिद्यत । \newline
2. शिरो᳚ ऽच्छिद्यता च्छिद्यत॒ शिरः॒ शिरो᳚ ऽच्छिद्यत॒ ते ते᳚ ऽच्छिद्यत॒ शिरः॒ शिरो᳚ ऽच्छिद्यत॒ ते । \newline
3. अ॒च्छि॒द्य॒त॒ ते ते᳚ ऽच्छिद्यता च्छिद्यत॒ ते दे॒वा दे॒वा स्ते᳚ ऽच्छिद्यता च्छिद्यत॒ ते दे॒वाः । \newline
4. ते दे॒वा दे॒वा स्ते ते दे॒वा अ॒श्विना॑ व॒श्विनौ॑ दे॒वा स्ते ते दे॒वा अ॒श्विनौ᳚ । \newline
5. दे॒वा अ॒श्विना॑ व॒श्विनौ॑ दे॒वा दे॒वा अ॒श्विना॑ वब्रुवन् नब्रुवन् न॒श्विनौ॑ दे॒वा दे॒वा अ॒श्विना॑ वब्रुवन्न् । \newline
6. अ॒श्विना॑ वब्रुवन् नब्रुवन् न॒श्विना॑ व॒श्विना॑ वब्रुवन् भि॒षजौ॑ भि॒षजा॑ वब्रुवन् न॒श्विना॑ व॒श्विना॑ वब्रुवन् भि॒षजौ᳚ । \newline
7. अ॒ब्रु॒व॒न् भि॒षजौ॑ भि॒षजा॑ वब्रुवन् नब्रुवन् भि॒षजौ॒ वै वै भि॒षजा॑ वब्रुवन् नब्रुवन् भि॒षजौ॒ वै । \newline
8. भि॒षजौ॒ वै वै भि॒षजौ॑ भि॒षजौ॒ वै स्थः॑ स्थो॒ वै भि॒षजौ॑ भि॒षजौ॒ वै स्थः॑ । \newline
9. वै स्थः॑ स्थो॒ वै वै स्थ॑ इ॒द मि॒दꣳ स्थो॒ वै वै स्थ॑ इ॒दम् । \newline
10. स्थ॒ इ॒द मि॒दꣳ स्थः॑ स्थ इ॒दं ॅय॒ज्ञ्स्य॑ य॒ज्ञ् स्ये॒दꣳ स्थः॑ स्थ इ॒दं ॅय॒ज्ञ्स्य॑ । \newline
11. इ॒दं ॅय॒ज्ञ्स्य॑ य॒ज्ञ्स्ये॒द मि॒दं ॅय॒ज्ञ्स्य॒ शिरः॒ शिरो॑ य॒ज्ञ्स्ये॒द मि॒दं ॅय॒ज्ञ्स्य॒ शिरः॑ । \newline
12. य॒ज्ञ्स्य॒ शिरः॒ शिरो॑ य॒ज्ञ्स्य॑ य॒ज्ञ्स्य॒ शिरः॒ प्रति॒ प्रति॒ शिरो॑ य॒ज्ञ्स्य॑ य॒ज्ञ्स्य॒ शिरः॒ प्रति॑ । \newline
13. शिरः॒ प्रति॒ प्रति॒ शिरः॒ शिरः॒ प्रति॑ धत्तम् धत्त॒म् प्रति॒ शिरः॒ शिरः॒ प्रति॑ धत्तम् । \newline
14. प्रति॑ धत्तम् धत्त॒म् प्रति॒ प्रति॑ धत्त॒ मितीति॑ धत्त॒म् प्रति॒ प्रति॑ धत्त॒ मिति॑ । \newline
15. ध॒त्त॒ मितीति॑ धत्तम् धत्त॒ मिति॒ तौ ता विति॑ धत्तम् धत्त॒ मिति॒ तौ । \newline
16. इति॒ तौ ता वितीति॒ ता व॑ब्रूता मब्रूता॒म् ता वितीति॒ ता व॑ब्रूताम् । \newline
17. ता व॑ब्रूता मब्रूता॒म् तौ ता व॑ब्रूतां॒ ॅवरं॒ ॅवर॑ मब्रूता॒म् तौ ता व॑ब्रूतां॒ ॅवर᳚म् । \newline
18. अ॒ब्रू॒तां॒ ॅवरं॒ ॅवर॑ मब्रूता मब्रूतां॒ ॅवरं॑ ॅवृणावहै वृणावहै॒ वर॑ मब्रूता मब्रूतां॒ ॅवरं॑ ॅवृणावहै । \newline
19. वरं॑ ॅवृणावहै वृणावहै॒ वरं॒ ॅवरं॑ ॅवृणावहै॒ ग्रहो॒ ग्रहो॑ वृणावहै॒ वरं॒ ॅवरं॑ ॅवृणावहै॒ ग्रहः॑ । \newline
20. वृ॒णा॒व॒है॒ ग्रहो॒ ग्रहो॑ वृणावहै वृणावहै॒ ग्रह॑ ए॒वैव ग्रहो॑ वृणावहै वृणावहै॒ ग्रह॑ ए॒व । \newline
21. ग्रह॑ ए॒वैव ग्रहो॒ ग्रह॑ ए॒व नौ॑ ना वे॒व ग्रहो॒ ग्रह॑ ए॒व नौ᳚ । \newline
22. ए॒व नौ॑ ना वे॒वैव ना॒ वत्रात्र॑ ना वे॒वैव ना॒ वत्र॑ । \newline
23. ना॒ वत्रात्र॑ नौ ना॒ वत्रा प्यप्यत्र॑ नौ ना॒ वत्रापि॑ । \newline
24. अत्रा प्यप्य त्रात्रापि॑ गृह्यताम् गृह्यता॒ मप्य त्रात्रापि॑ गृह्यताम् । \newline
25. अपि॑ गृह्यताम् गृह्यता॒ मप्यपि॑ गृह्यता॒ मितीति॑ गृह्यता॒ मप्यपि॑ गृह्यता॒ मिति॑ । \newline
26. गृ॒ह्य॒ता॒ मितीति॑ गृह्यताम् गृह्यता॒ मिति॒ ताभ्या॒म् ताभ्या॒ मिति॑ गृह्यताम् गृह्यता॒ मिति॒ ताभ्या᳚म् । \newline
27. इति॒ ताभ्या॒म् ताभ्या॒ मितीति॒ ताभ्या॑ मे॒त मे॒तम् ताभ्या॒ मितीति॒ ताभ्या॑ मे॒तम् । \newline
28. ताभ्या॑ मे॒त मे॒तम् ताभ्या॒म् ताभ्या॑ मे॒त मा᳚श्वि॒न मा᳚श्वि॒न मे॒तम् ताभ्या॒म् ताभ्या॑ मे॒त मा᳚श्वि॒नम् । \newline
29. ए॒त मा᳚श्वि॒न मा᳚श्वि॒न मे॒त मे॒त मा᳚श्वि॒न म॑गृह्णन् नगृह्णन् नाश्वि॒न मे॒त मे॒त मा᳚श्वि॒न म॑गृह्णन्न् । \newline
30. आ॒श्वि॒न म॑गृह्णन् नगृह्णन् नाश्वि॒न मा᳚श्वि॒न म॑गृह्ण॒न् तत॒ स्ततो॑ ऽगृह्णन् नाश्वि॒न मा᳚श्वि॒न म॑गृह्ण॒न् ततः॑ । \newline
31. अ॒गृ॒ह्ण॒न् तत॒ स्ततो॑ ऽगृह्णन् नगृह्ण॒न् ततो॒ वै वै ततो॑ ऽगृह्णन् नगृह्ण॒न् ततो॒ वै । \newline
32. ततो॒ वै वै तत॒ स्ततो॒ वै तौ तौ वै तत॒ स्ततो॒ वै तौ । \newline
33. वै तौ तौ वै वै तौ य॒ज्ञ्स्य॑ य॒ज्ञ्स्य॒ तौ वै वै तौ य॒ज्ञ्स्य॑ । \newline
34. तौ य॒ज्ञ्स्य॑ य॒ज्ञ्स्य॒ तौ तौ य॒ज्ञ्स्य॒ शिरः॒ शिरो॑ य॒ज्ञ्स्य॒ तौ तौ य॒ज्ञ्स्य॒ शिरः॑ । \newline
35. य॒ज्ञ्स्य॒ शिरः॒ शिरो॑ य॒ज्ञ्स्य॑ य॒ज्ञ्स्य॒ शिरः॒ प्रति॒ प्रति॒ शिरो॑ य॒ज्ञ्स्य॑ य॒ज्ञ्स्य॒ शिरः॒ प्रति॑ । \newline
36. शिरः॒ प्रति॒ प्रति॒ शिरः॒ शिरः॒ प्रत्य॑ धत्ता मधत्ता॒म् प्रति॒ शिरः॒ शिरः॒ प्रत्य॑ धत्ताम् । \newline
37. प्रत्य॑ धत्ता मधत्ता॒म् प्रति॒ प्रत्य॑ धत्तां॒ ॅयद् यद॑धत्ता॒म् प्रति॒ प्रत्य॑ धत्तां॒ ॅयत् । \newline
38. अ॒ध॒त्तां॒ ॅयद् यद॑धत्ता मधत्तां॒ ॅयदा᳚श्वि॒न आ᳚श्वि॒नो यद॑धत्ता मधत्तां॒ ॅयदा᳚श्वि॒नः । \newline
39. यदा᳚श्वि॒न आ᳚श्वि॒नो यद् यदा᳚श्वि॒नो गृ॒ह्यते॑ गृ॒ह्यत॑ आश्वि॒नो यद् यदा᳚श्वि॒नो गृ॒ह्यते᳚ । \newline
40. आ॒श्वि॒नो गृ॒ह्यते॑ गृ॒ह्यत॑ आश्वि॒न आ᳚श्वि॒नो गृ॒ह्यते॑ य॒ज्ञ्स्य॑ य॒ज्ञ्स्य॑ गृ॒ह्यत॑ आश्वि॒न आ᳚श्वि॒नो गृ॒ह्यते॑ य॒ज्ञ्स्य॑ । \newline
41. गृ॒ह्यते॑ य॒ज्ञ्स्य॑ य॒ज्ञ्स्य॑ गृ॒ह्यते॑ गृ॒ह्यते॑ य॒ज्ञ्स्य॒ निष्कृ॑त्यै॒ निष्कृ॑त्यै य॒ज्ञ्स्य॑ गृ॒ह्यते॑ गृ॒ह्यते॑ य॒ज्ञ्स्य॒ निष्कृ॑त्यै । \newline
42. य॒ज्ञ्स्य॒ निष्कृ॑त्यै॒ निष्कृ॑त्यै य॒ज्ञ्स्य॑ य॒ज्ञ्स्य॒ निष्कृ॑त्यै॒ तौ तौ निष्कृ॑त्यै य॒ज्ञ्स्य॑ य॒ज्ञ्स्य॒ निष्कृ॑त्यै॒ तौ । \newline
43. निष्कृ॑त्यै॒ तौ तौ निष्कृ॑त्यै॒ निष्कृ॑त्यै॒ तौ दे॒वा दे॒वा स्तौ निष्कृ॑त्यै॒ निष्कृ॑त्यै॒ तौ दे॒वाः । \newline
44. निष्कृ॑त्या॒ इति॒ निः - कृ॒त्यै॒ । \newline
45. तौ दे॒वा दे॒वा स्तौ तौ दे॒वा अ॑ब्रुवन् नब्रुवन् दे॒वा स्तौ तौ दे॒वा अ॑ब्रुवन्न् । \newline
46. दे॒वा अ॑ब्रुवन् नब्रुवन् दे॒वा दे॒वा अ॑ब्रुव॒न् नपू॑ता॒ वपू॑ता वब्रुवन् दे॒वा दे॒वा अ॑ब्रुव॒न् नपू॑तौ । \newline
47. अ॒ब्रु॒व॒न् नपू॑ता॒ वपू॑ता वब्रुवन् नब्रुव॒न् नपू॑तौ॒ वै वा अपू॑ता वब्रुवन् नब्रुव॒न् नपू॑तौ॒ वै । \newline
48. अपू॑तौ॒ वै वा अपू॑ता॒ वपू॑तौ॒ वा इ॒मा वि॒मौ वा अपू॑ता॒ वपू॑तौ॒ वा इ॒मौ । \newline
49. वा इ॒मा वि॒मौ वै वा इ॒मौ म॑नुष्यच॒रौ म॑नुष्यच॒रा वि॒मौ वै वा इ॒मौ म॑नुष्यच॒रौ । \newline
50. इ॒मौ म॑नुष्यच॒रौ म॑नुष्यच॒रा वि॒मा वि॒मौ म॑नुष्यच॒रौ भि॒षजौ॑ भि॒षजौ॑ मनुष्यच॒रा वि॒मा वि॒मौ म॑नुष्यच॒रौ भि॒षजौ᳚ । \newline
51. म॒नु॒ष्य॒च॒रौ भि॒षजौ॑ भि॒षजौ॑ मनुष्यच॒रौ म॑नुष्यच॒रौ भि॒षजा॒ वितीति॑ भि॒षजौ॑ मनुष्यच॒रौ म॑नुष्यच॒रौ भि॒षजा॒ विति॑ । \newline
52. म॒नु॒ष्य॒च॒राविति॑ मनुष्य - च॒रौ । \newline
\pagebreak
\markright{ TS 6.4.9.2  \hfill https://www.vedavms.in \hfill}

\section{ TS 6.4.9.2 }

\textbf{TS 6.4.9.2 } \newline
\textbf{Samhita Paata} \newline

भि॒षजा॒विति॒ तस्मा᳚द् ब्राह्म॒णेन॑ भेष॒जं न का॒र्य॑मपू॑तो॒ ह्ये᳚(1॒)षो॑ ऽमे॒द्ध्यो यो भि॒षक्तौ ब॑हिष्पवमा॒नेन॑ पवयि॒त्वा ताभ्या॑-मे॒तमा᳚श्वि॒न-म॑गृह्ण॒न् तस्मा᳚द् बहिष्पवमा॒ने स्तु॒त आ᳚श्वि॒नो गृ॑ह्यते॒ तस्मा॑दे॒वं ॅवि॒दुषा॑ बहिष्पवमा॒न उ॑प॒सद्यः॑ प॒वित्रं॒ ॅवै ब॑हिष्पवमा॒न आ॒त्मान॑मे॒व प॑वयते॒ तयो᳚ऽस्त्रे॒धा भैष॑ज्यं॒ ॅवि न्य॑दधुर॒ग्नौ तृती॑यम॒फ्सु तृती॑यं ब्राह्म॒णे तृती॑यं॒ तस्मा॑दुदपा॒त्र- [  ] \newline

\textbf{Pada Paata} \newline

भि॒षजौ᳚ । इति॑ । तस्मा᳚त् । ब्रा॒ह्म॒णेन॑ । भे॒ष॒जम् । न । का॒र्य᳚म् । अपू॑तः । हि । ए॒षः । अ॒मे॒द्ध्यः । यः । भि॒षक् । तौ । ब॒हि॒ष्प॒व॒मा॒नेनेति॑ बहिः - प॒व॒मा॒नेन॑ । प॒व॒यि॒त्वा । ताभ्या᳚म् । ए॒तम् । आ॒श्वि॒नम् । अ॒गृ॒ह्ण॒न्न् । तस्मा᳚त् । ब॒हि॒ष्प॒व॒मा॒न इति॑ बहिः-प॒व॒मा॒ने । स्तु॒ते । आ॒श्वि॒नः । गृ॒ह्य॒ते॒ । तस्मा᳚त् । ए॒वम् । वि॒दुषा᳚ । ब॒हि॒ष्प॒व॒मा॒न इति॑ बहिः - प॒व॒मा॒नः । उ॒प॒सद्य॒ इत्यु॑प - सद्यः॑ । प॒वित्र᳚म् । वै । ब॒हि॒ष्प॒व॒मा॒न इति॑ बहिः - प॒व॒मा॒नः । आ॒त्मान᳚म् । ए॒व । प॒व॒य॒ते॒ । तयोः᳚ । त्रे॒धा । भैष॑ज्यम् । वि । नीति॑ । अ॒द॒धुः॒ । अ॒ग्नौ । तृती॑यम् । अ॒फ्स्वित्य॑प्-सु । तृती॑यम् । ब्रा॒ह्म॒णे । तृती॑यम् । तस्मा᳚त् । उ॒द॒पा॒त्रमित्यु॑द - पा॒त्रम् ।  \newline


\textbf{Krama Paata} \newline

भि॒षजा॒विति॑ । इति॒ तस्मा᳚त् । तस्मा᳚द् ब्राह्म॒णेन॑ । ब्रा॒ह्म॒णेन॑ भेष॒जम् । भे॒ष॒जम् न । न का॒र्य᳚म् । का॒र्य॑मपू॑तः । अपू॑तो॒ हि । ह्ये॑षः । ए॒षो॑ऽमे॒द्ध्यः । अ॒मे॒द्ध्यो यः । यो भि॒षक् । भि॒षक् तौ । तौ ब॑हिष्पवमा॒नेन॑ । ब॒हि॒ष्प॒मा॒नेन॑ पवयि॒त्वा । ब॒हि॒ष्प॒व॒मा॒नेनेति॑ बहिः - प॒व॒मा॒नेन॑ । प॒व॒यि॒त्वा ताभ्या᳚म् । ताभ्या॑मे॒तम् । ए॒तमा᳚श्वि॒नम् । आ॒श्वि॒नम॑गृह्णन्न् । अ॒गृ॒ह्ण॒न् तस्मा᳚त् । तस्मा᳚द् बहिष्पवमा॒ने । ब॒हि॒ष्प॒व॒मा॒ने स्तु॒ते । ब॒हि॒ष्प॒व॒मा॒न इति॑ बहिः - प॒व॒मा॒ने । स्तु॒त आ᳚श्वि॒नः । आ॒श्वि॒नो गृ॑ह्यते । गृ॒ह्य॒ते॒ तस्मा᳚त् । तस्मा॑दे॒वम् । ए॒वम् ॅवि॒दुषा᳚ । वि॒दुषा॑ बहिष्पवमा॒नः । ब॒हि॒ष्प॒व॒मा॒न उ॑प॒सद्यः॑ । ब॒हि॒ष्प॒व॒मा॒न इति॑ बहिः - प॒व॒मा॒नः । उ॒प॒सद्यः॑ प॒वित्र᳚म् । उ॒प॒सद्य॒ इत्यु॑प - सद्यः॑ । प॒वित्र॒म् ॅवै । वै ब॑हिष्पवमा॒नः । ब॒हि॒ष्प॒व॒मा॒न आ॒त्मान᳚म् । ब॒हि॒ष्प॒व॒मा॒न इति॑ बहिः - प॒व॒मा॒नः । आ॒त्मान॑मे॒व । ए॒व प॑वयते । प॒व॒य॒ते॒ तयोः᳚ । तयो᳚स्त्रे॒धा । त्रे॒धा भैष॑ज्यम् । भैष॑ज्य॒म् ॅवि । वि नि । न्य॑दधुः । अ॒द॒धु॒र॒ग्नौ । अ॒ग्नौ तृती॑यम् । तृती॑यम॒फ्सु । अ॒फ्सु तृती॑यम् । अ॒फ्स्वित्य॑प् - सु । तृती॑यम् ब्राह्म॒णे । ब्रा॒ह्म॒णे तृती॑यम् । तृती॑य॒म् तस्मा᳚त् । तस्मा॑दुदपा॒त्रम् । उ॒द॒पा॒त्रमु॑पनि॒धाय॑ । उ॒द॒पा॒त्रमित्यु॑द - पा॒त्रम् \newline

\textbf{Jatai Paata} \newline

1. भि॒षजा॒ वितीति॑ भि॒षजौ॑ भि॒षजा॒ विति॑ । \newline
2. इति॒ तस्मा॒त् तस्मा॒ दितीति॒ तस्मा᳚त् । \newline
3. तस्मा᳚द् ब्राह्म॒णेन॑ ब्राह्म॒णेन॒ तस्मा॒त् तस्मा᳚द् ब्राह्म॒णेन॑ । \newline
4. ब्रा॒ह्म॒णेन॑ भेष॒जम् भे॑ष॒जम् ब्रा᳚ह्म॒णेन॑ ब्राह्म॒णेन॑ भेष॒जम् । \newline
5. भे॒ष॒जन् न न भे॑ष॒जम् भे॑ष॒जन् न । \newline
6. न का॒र्य॑म् का॒र्य॑न् न न का॒र्य᳚म् । \newline
7. का॒र्य॑ मपू॒तो ऽपू॑तः का॒र्य॑म् का॒र्य॑ मपू॑तः । \newline
8. अपू॑तो॒ हि ह्यपू॒तो ऽपू॑तो॒ हि । \newline
9. ह्ये॑ष ए॒ष हि ह्ये॑षः । \newline
10. ए॒षो॑ ऽमे॒द्ध्यो॑ ऽमे॒द्ध्य ए॒ष ए॒षो॑ ऽमे॒द्ध्यः । \newline
11. अ॒मे॒द्ध्यो यो यो॑ ऽमे॒द्ध्यो॑ ऽमे॒द्ध्यो यः । \newline
12. यो भि॒षग् भि॒षग् यो यो भि॒षक् । \newline
13. भि॒षक् तौ तौ भि॒षग् भि॒षक् तौ । \newline
14. तौ ब॑हिष्पवमा॒नेन॑ बहिष्पवमा॒नेन॒ तौ तौ ब॑हिष्पवमा॒नेन॑ । \newline
15. ब॒हि॒ष्प॒व॒मा॒नेन॑ पवयि॒त्वा प॑वयि॒त्वा ब॑हिष्पवमा॒नेन॑ बहिष्पवमा॒नेन॑ पवयि॒त्वा । \newline
16. ब॒हि॒ष्प॒व॒मा॒नेनेति॑ बहिः - प॒व॒मा॒नेन॑ । \newline
17. प॒व॒यि॒त्वा ताभ्या॒म् ताभ्या᳚म् पवयि॒त्वा प॑वयि॒त्वा ताभ्या᳚म् । \newline
18. ताभ्या॑ मे॒त मे॒तम् ताभ्या॒म् ताभ्या॑ मे॒तम् । \newline
19. ए॒त मा᳚श्वि॒न मा᳚श्वि॒न मे॒त मे॒त मा᳚श्वि॒नम् । \newline
20. आ॒श्वि॒न म॑गृह्णन् नगृह्णन् नाश्वि॒न मा᳚श्वि॒न म॑गृह्णन्न् । \newline
21. अ॒गृ॒ह्ण॒न् तस्मा॒त् तस्मा॑ दगृह्णन् नगृह्ण॒न् तस्मा᳚त् । \newline
22. तस्मा᳚द् बहिष्पवमा॒ने ब॑हिष्पवमा॒ने तस्मा॒त् तस्मा᳚द् बहिष्पवमा॒ने । \newline
23. ब॒हि॒ष्प॒व॒मा॒ने स्तु॒ते स्तु॒ते ब॑हिष्पवमा॒ने ब॑हिष्पवमा॒ने स्तु॒ते । \newline
24. ब॒हि॒ष्प॒व॒मा॒न इति॑ बहिः - प॒व॒मा॒ने । \newline
25. स्तु॒त आ᳚श्वि॒न आ᳚श्वि॒नः स्तु॒ते स्तु॒त आ᳚श्वि॒नः । \newline
26. आ॒श्वि॒नो गृ॑ह्यते गृह्यत आश्वि॒न आ᳚श्वि॒नो गृ॑ह्यते । \newline
27. गृ॒ह्य॒ते॒ तस्मा॒त् तस्मा᳚द् गृह्यते गृह्यते॒ तस्मा᳚त् । \newline
28. तस्मा॑ दे॒व मे॒वम् तस्मा॒त् तस्मा॑ दे॒वम् । \newline
29. ए॒वं ॅवि॒दुषा॑ वि॒दुषै॒व मे॒वं ॅवि॒दुषा᳚ । \newline
30. वि॒दुषा॑ बहिष्पवमा॒नो ब॑हिष्पवमा॒नो वि॒दुषा॑ वि॒दुषा॑ बहिष्पवमा॒नः । \newline
31. ब॒हि॒ष्प॒व॒मा॒न उ॑प॒सद्य॑ उप॒सद्यो॑ बहिष्पवमा॒नो ब॑हिष्पवमा॒न उ॑प॒सद्यः॑ । \newline
32. ब॒हि॒ष्प॒व॒मा॒न इति॑ बहिः - प॒व॒मा॒नः । \newline
33. उ॒प॒सद्यः॑ प॒वित्र॑म् प॒वित्र॑ मुप॒सद्य॑ उप॒सद्यः॑ प॒वित्र᳚म् । \newline
34. उ॒प॒सद्य॒ इत्यु॑प - सद्यः॑ । \newline
35. प॒वित्रं॒ ॅवै वै प॒वित्र॑म् प॒वित्रं॒ ॅवै । \newline
36. वै ब॑हिष्पवमा॒नो ब॑हिष्पवमा॒नो वै वै ब॑हिष्पवमा॒नः । \newline
37. ब॒हि॒ष्प॒व॒मा॒न आ॒त्मान॑ मा॒त्मान॑म् बहिष्पवमा॒नो ब॑हिष्पवमा॒न आ॒त्मान᳚म् । \newline
38. ब॒हि॒ष्प॒व॒मा॒न इति॑ बहिः - प॒व॒मा॒नः । \newline
39. आ॒त्मान॑ मे॒वै वात्मान॑ मा॒त्मान॑ मे॒व । \newline
40. ए॒व प॑वयते पवयत ए॒वैव प॑वयते । \newline
41. प॒व॒य॒ते॒ तयो॒ स्तयोः᳚ पवयते पवयते॒ तयोः᳚ । \newline
42. तयो᳚ स्त्रे॒धा त्रे॒धा तयो॒ स्तयो᳚ स्त्रे॒धा । \newline
43. त्रे॒धा भैष॑ज्य॒म् भैष॑ज्यम् त्रे॒धा त्रे॒धा भैष॑ज्यम् । \newline
44. भैष॑ज्यं॒ ॅवि वि भैष॑ज्य॒म् भैष॑ज्यं॒ ॅवि । \newline
45. वि नि नि वि वि नि । \newline
46. न्य॑दधु रदधु॒र् नि न्य॑दधुः । \newline
47. अ॒द॒धु॒ र॒ग्ना व॒ग्ना व॑दधु रदधु र॒ग्नौ । \newline
48. अ॒ग्नौ तृती॑य॒म् तृती॑य म॒ग्ना व॒ग्नौ तृती॑यम् । \newline
49. तृती॑य म॒फ्स्व॑फ्सु तृती॑य॒म् तृती॑य म॒फ्सु । \newline
50. अ॒फ्सु तृती॑य॒म् तृती॑य म॒फ्स्व॑फ्सु तृती॑यम् । \newline
51. अ॒फ्स्वित्य॑प् - सु । \newline
52. तृती॑यम् ब्राह्म॒णे ब्रा᳚ह्म॒णे तृती॑य॒म् तृती॑यम् ब्राह्म॒णे । \newline
53. ब्रा॒ह्म॒णे तृती॑य॒म् तृती॑यम् ब्राह्म॒णे ब्रा᳚ह्म॒णे तृती॑यम् । \newline
54. तृती॑य॒म् तस्मा॒त् तस्मा॒त् तृती॑य॒म् तृती॑य॒म् तस्मा᳚त् । \newline
55. तस्मा॑ दुदपा॒त्र मु॑दपा॒त्रम् तस्मा॒त् तस्मा॑ दुदपा॒त्रम् । \newline
56. उ॒द॒पा॒त्र मु॑पनि॒धा यो॑पनि॒धा यो॑दपा॒त्र मु॑दपा॒त्र मु॑पनि॒धाय॑ । \newline
57. उ॒द॒पा॒त्रमित्यु॑द - पा॒त्रम् । \newline

\textbf{Ghana Paata } \newline

1. भि॒षजा॒ वितीति॑ भि॒षजौ॑ भि॒षजा॒ विति॒ तस्मा॒त् तस्मा॒ दिति॑ भि॒षजौ॑ भि॒षजा॒ विति॒ तस्मा᳚त् । \newline
2. इति॒ तस्मा॒त् तस्मा॒ दितीति॒ तस्मा᳚द् ब्राह्म॒णेन॑ ब्राह्म॒णेन॒ तस्मा॒ दितीति॒ तस्मा᳚द् ब्राह्म॒णेन॑ । \newline
3. तस्मा᳚द् ब्राह्म॒णेन॑ ब्राह्म॒णेन॒ तस्मा॒त् तस्मा᳚द् ब्राह्म॒णेन॑ भेष॒जम् भे॑ष॒जम् ब्रा᳚ह्म॒णेन॒ तस्मा॒त् तस्मा᳚द् ब्राह्म॒णेन॑ भेष॒जम् । \newline
4. ब्रा॒ह्म॒णेन॑ भेष॒जम् भे॑ष॒जम् ब्रा᳚ह्म॒णेन॑ ब्राह्म॒णेन॑ भेष॒जन् न न भे॑ष॒जम् ब्रा᳚ह्म॒णेन॑ ब्राह्म॒णेन॑ भेष॒जन् न । \newline
5. भे॒ष॒जन् न न भे॑ष॒जम् भे॑ष॒जन् न का॒र्य॑म् का॒र्य॑न् न भे॑ष॒जम् भे॑ष॒जन् न का॒र्य᳚म् । \newline
6. न का॒र्य॑म् का॒र्य॑न् न न का॒र्य॑ मपू॒तो ऽपू॑तः का॒र्य॑न् न न का॒र्य॑ मपू॑तः । \newline
7. का॒र्य॑ मपू॒तो ऽपू॑तः का॒र्य॑म् का॒र्य॑ मपू॑तो॒ हि ह्यपू॑तः का॒र्य॑म् का॒र्य॑ मपू॑तो॒ हि । \newline
8. अपू॑तो॒ हि ह्यपू॒तो ऽपू॑तो॒ ह्ये॑ष ए॒ष ह्यपू॒तो ऽपू॑तो॒ ह्ये॑षः । \newline
9. ह्ये॑ष ए॒ष हि ह्ये᳚(1॒)षो॑ ऽमे॒द्ध्यो॑ ऽमे॒द्ध्य ए॒ष हि ह्ये᳚(1॒)षो॑ ऽमे॒द्ध्यः । \newline
10. ए॒षो॑ ऽमे॒द्ध्यो॑ ऽमे॒द्ध्य ए॒ष ए॒षो॑ ऽमे॒द्ध्यो यो यो॑ ऽमे॒द्ध्य ए॒ष ए॒षो॑ ऽमे॒द्ध्यो यः । \newline
11. अ॒मे॒द्ध्यो यो यो॑ ऽमे॒द्ध्यो॑ ऽमे॒द्ध्यो यो भि॒षग् भि॒षग् यो॑ ऽमे॒द्ध्यो॑ ऽमे॒द्ध्यो यो भि॒षक् । \newline
12. यो भि॒षग् भि॒षग् यो यो भि॒षक् तौ तौ भि॒षग् यो यो भि॒षक् तौ । \newline
13. भि॒षक् तौ तौ भि॒षग् भि॒षक् तौ ब॑हिष्पवमा॒नेन॑ बहिष्पवमा॒नेन॒ तौ भि॒षग् भि॒षक् तौ ब॑हिष्पवमा॒नेन॑ । \newline
14. तौ ब॑हिष्पवमा॒नेन॑ बहिष्पवमा॒नेन॒ तौ तौ ब॑हिष्पवमा॒नेन॑ पवयि॒त्वा प॑वयि॒त्वा ब॑हिष्पवमा॒नेन॒ तौ तौ ब॑हिष्पवमा॒नेन॑ पवयि॒त्वा । \newline
15. ब॒हि॒ष्प॒व॒मा॒नेन॑ पवयि॒त्वा प॑वयि॒त्वा ब॑हिष्पवमा॒नेन॑ बहिष्पवमा॒नेन॑ पवयि॒त्वा ताभ्या॒म् ताभ्या᳚म् पवयि॒त्वा ब॑हिष्पवमा॒नेन॑ बहिष्पवमा॒नेन॑ पवयि॒त्वा ताभ्या᳚म् । \newline
16. ब॒हि॒ष्प॒व॒मा॒नेनेति॑ बहिः - प॒व॒मा॒नेन॑ । \newline
17. प॒व॒यि॒त्वा ताभ्या॒म् ताभ्या᳚म् पवयि॒त्वा प॑वयि॒त्वा ताभ्या॑ मे॒त मे॒तम् ताभ्या᳚म् पवयि॒त्वा प॑वयि॒त्वा ताभ्या॑ मे॒तम् । \newline
18. ताभ्या॑ मे॒त मे॒तम् ताभ्या॒म् ताभ्या॑ मे॒त मा᳚श्वि॒न मा᳚श्वि॒न मे॒तम् ताभ्या॒म् ताभ्या॑ मे॒त मा᳚श्वि॒नम् । \newline
19. ए॒त मा᳚श्वि॒न मा᳚श्वि॒न मे॒त मे॒त मा᳚श्वि॒न म॑गृह्णन् नगृह्णन् नाश्वि॒न मे॒त मे॒त मा᳚श्वि॒न म॑गृह्णन्न् । \newline
20. आ॒श्वि॒न म॑गृह्णन् नगृह्णन् नाश्वि॒न मा᳚श्वि॒न म॑गृह्ण॒न् तस्मा॒त् तस्मा॑ दगृह्णन् नाश्वि॒न मा᳚श्वि॒न म॑गृह्ण॒न् तस्मा᳚त् । \newline
21. अ॒गृ॒ह्ण॒न् तस्मा॒त् तस्मा॑ दगृह्णन् नगृह्ण॒न् तस्मा᳚द् बहिष्पवमा॒ने ब॑हिष्पवमा॒ने तस्मा॑ दगृह्णन् नगृह्ण॒न् तस्मा᳚द् बहिष्पवमा॒ने । \newline
22. तस्मा᳚द् बहिष्पवमा॒ने ब॑हिष्पवमा॒ने तस्मा॒त् तस्मा᳚द् बहिष्पवमा॒ने स्तु॒ते स्तु॒ते ब॑हिष्पवमा॒ने तस्मा॒त् तस्मा᳚द् बहिष्पवमा॒ने स्तु॒ते । \newline
23. ब॒हि॒ष्प॒व॒मा॒ने स्तु॒ते स्तु॒ते ब॑हिष्पवमा॒ने ब॑हिष्पवमा॒ने स्तु॒त आ᳚श्वि॒न आ᳚श्वि॒नः स्तु॒ते ब॑हिष्पवमा॒ने ब॑हिष्पवमा॒ने स्तु॒त आ᳚श्वि॒नः । \newline
24. ब॒हि॒ष्प॒व॒मा॒न इति॑ बहिः - प॒व॒मा॒ने । \newline
25. स्तु॒त आ᳚श्वि॒न आ᳚श्वि॒नः स्तु॒ते स्तु॒त आ᳚श्वि॒नो गृ॑ह्यते गृह्यत आश्वि॒नः स्तु॒ते स्तु॒त आ᳚श्वि॒नो गृ॑ह्यते । \newline
26. आ॒श्वि॒नो गृ॑ह्यते गृह्यत आश्वि॒न आ᳚श्वि॒नो गृ॑ह्यते॒ तस्मा॒त् तस्मा᳚द् गृह्यत आश्वि॒न आ᳚श्वि॒नो गृ॑ह्यते॒ तस्मा᳚त् । \newline
27. गृ॒ह्य॒ते॒ तस्मा॒त् तस्मा᳚द् गृह्यते गृह्यते॒ तस्मा॑ दे॒व मे॒वम् तस्मा᳚द् गृह्यते गृह्यते॒ तस्मा॑ दे॒वम् । \newline
28. तस्मा॑ दे॒व मे॒वम् तस्मा॒त् तस्मा॑ दे॒वं ॅवि॒दुषा॑ वि॒दुषै॒वम् तस्मा॒त् तस्मा॑ दे॒वं ॅवि॒दुषा᳚ । \newline
29. ए॒वं ॅवि॒दुषा॑ वि॒दुषै॒व मे॒वं ॅवि॒दुषा॑ बहिष्पवमा॒नो ब॑हिष्पवमा॒नो वि॒दुषै॒व मे॒वं ॅवि॒दुषा॑ बहिष्पवमा॒नः । \newline
30. वि॒दुषा॑ बहिष्पवमा॒नो ब॑हिष्पवमा॒नो वि॒दुषा॑ वि॒दुषा॑ बहिष्पवमा॒न उ॑प॒सद्य॑ उप॒सद्यो॑ बहिष्पवमा॒नो वि॒दुषा॑ वि॒दुषा॑ बहिष्पवमा॒न उ॑प॒सद्यः॑ । \newline
31. ब॒हि॒ष्प॒व॒मा॒न उ॑प॒सद्य॑ उप॒सद्यो॑ बहिष्पवमा॒नो ब॑हिष्पवमा॒न उ॑प॒सद्यः॑ प॒वित्र॑म् प॒वित्र॑ मुप॒सद्यो॑ बहिष्पवमा॒नो ब॑हिष्पवमा॒न उ॑प॒सद्यः॑ प॒वित्र᳚म् । \newline
32. ब॒हि॒ष्प॒व॒मा॒न इति॑ बहिः - प॒व॒मा॒नः । \newline
33. उ॒प॒सद्यः॑ प॒वित्र॑म् प॒वित्र॑ मुप॒सद्य॑ उप॒सद्यः॑ प॒वित्रं॒ ॅवै वै प॒वित्र॑ मुप॒सद्य॑ उप॒सद्यः॑ प॒वित्रं॒ ॅवै । \newline
34. उ॒प॒सद्य॒ इत्यु॑प - सद्यः॑ । \newline
35. प॒वित्रं॒ ॅवै वै प॒वित्र॑म् प॒वित्रं॒ ॅवै ब॑हिष्पवमा॒नो ब॑हिष्पवमा॒नो वै प॒वित्र॑म् प॒वित्रं॒ ॅवै ब॑हिष्पवमा॒नः । \newline
36. वै ब॑हिष्पवमा॒नो ब॑हिष्पवमा॒नो वै वै ब॑हिष्पवमा॒न आ॒त्मान॑ मा॒त्मान॑म् बहिष्पवमा॒नो वै वै ब॑हिष्पवमा॒न आ॒त्मान᳚म् । \newline
37. ब॒हि॒ष्प॒व॒मा॒न आ॒त्मान॑ मा॒त्मान॑म् बहिष्पवमा॒नो ब॑हिष्पवमा॒न आ॒त्मान॑ मे॒वै वात्मान॑म् बहिष्पवमा॒नो ब॑हिष्पवमा॒न आ॒त्मान॑ मे॒व । \newline
38. ब॒हि॒ष्प॒व॒मा॒न इति॑ बहिः - प॒व॒मा॒नः । \newline
39. आ॒त्मान॑ मे॒वै वात्मान॑ मा॒त्मान॑ मे॒व प॑वयते पवयत ए॒वात्मान॑ मा॒त्मान॑ मे॒व प॑वयते । \newline
40. ए॒व प॑वयते पवयत ए॒वैव प॑वयते॒ तयो॒ स्तयोः᳚ पवयत ए॒वैव प॑वयते॒ तयोः᳚ । \newline
41. प॒व॒य॒ते॒ तयो॒ स्तयोः᳚ पवयते पवयते॒ तयो᳚ स्त्रे॒धा त्रे॒धा तयोः᳚ पवयते पवयते॒ तयो᳚ स्त्रे॒धा । \newline
42. तयो᳚ स्त्रे॒धा त्रे॒धा तयो॒ स्तयो᳚ स्त्रे॒धा भैष॑ज्य॒म् भैष॑ज्यम् त्रे॒धा तयो॒ स्तयो᳚ स्त्रे॒धा भैष॑ज्यम् । \newline
43. त्रे॒धा भैष॑ज्य॒म् भैष॑ज्यम् त्रे॒धा त्रे॒धा भैष॑ज्यं॒ ॅवि वि भैष॑ज्यम् त्रे॒धा त्रे॒धा भैष॑ज्यं॒ ॅवि । \newline
44. भैष॑ज्यं॒ ॅवि वि भैष॑ज्य॒म् भैष॑ज्यं॒ ॅवि नि नि वि भैष॑ज्य॒म् भैष॑ज्यं॒ ॅवि नि । \newline
45. वि नि नि वि वि न्य॑दधु रदधु॒र् नि वि वि न्य॑दधुः । \newline
46. न्य॑दधु रदधु॒र् नि न्य॑दधु र॒ग्ना व॒ग्ना व॑दधु॒र् नि न्य॑दधु र॒ग्नौ । \newline
47. अ॒द॒धु॒ र॒ग्ना व॒ग्ना व॑दधु रदधु र॒ग्नौ तृती॑य॒म् तृती॑य म॒ग्ना व॑दधु रदधु र॒ग्नौ तृती॑यम् । \newline
48. अ॒ग्नौ तृती॑य॒म् तृती॑य म॒ग्ना व॒ग्नौ तृती॑य म॒फ्स्व॑फ्सु तृती॑य म॒ग्ना व॒ग्नौ तृती॑य म॒फ्सु । \newline
49. तृती॑य म॒फ्स्व॑फ्सु तृती॑य॒म् तृती॑य म॒फ्सु तृती॑य॒म् तृती॑य म॒फ्सु तृती॑य॒म् तृती॑य म॒फ्सु तृती॑यम् । \newline
50. अ॒फ्सु तृती॑य॒म् तृती॑य म॒फ्स्व॑फ्सु तृती॑यम् ब्राह्म॒णे ब्रा᳚ह्म॒णे तृती॑य म॒फ्स्व॑फ्सु तृती॑यम् ब्राह्म॒णे । \newline
51. अ॒फ्स्वित्य॑प् - सु । \newline
52. तृती॑यम् ब्राह्म॒णे ब्रा᳚ह्म॒णे तृती॑य॒म् तृती॑यम् ब्राह्म॒णे तृती॑य॒म् तृती॑यम् ब्राह्म॒णे तृती॑य॒म् तृती॑यम् ब्राह्म॒णे तृती॑यम् । \newline
53. ब्रा॒ह्म॒णे तृती॑य॒म् तृती॑यम् ब्राह्म॒णे ब्रा᳚ह्म॒णे तृती॑य॒म् तस्मा॒त् तस्मा॒त् तृती॑यम् ब्राह्म॒णे ब्रा᳚ह्म॒णे तृती॑य॒म् तस्मा᳚त् । \newline
54. तृती॑य॒म् तस्मा॒त् तस्मा॒त् तृती॑य॒म् तृती॑य॒म् तस्मा॑ दुदपा॒त्र मु॑दपा॒त्रम् तस्मा॒त् तृती॑य॒म् तृती॑य॒म् तस्मा॑ दुदपा॒त्रम् । \newline
55. तस्मा॑ दुदपा॒त्र मु॑दपा॒त्रम् तस्मा॒त् तस्मा॑ दुदपा॒त्र मु॑पनि॒धायो॑ पनि॒धायो॑ दपा॒त्रम् तस्मा॒त् तस्मा॑ दुदपा॒त्र मु॑पनि॒धाय॑ । \newline
56. उ॒द॒पा॒त्र मु॑पनि॒धायो॑ पनि॒धायो॑ दपा॒त्र मु॑दपा॒त्र मु॑पनि॒धाय॑ ब्राह्म॒णम् ब्रा᳚ह्म॒ण मु॑पनि॒धायो॑ दपा॒त्र मु॑दपा॒त्र मु॑पनि॒धाय॑ ब्राह्म॒णम् । \newline
57. उ॒द॒पा॒त्रमित्यु॑द - पा॒त्रम् । \newline
\pagebreak
\markright{ TS 6.4.9.3  \hfill https://www.vedavms.in \hfill}

\section{ TS 6.4.9.3 }

\textbf{TS 6.4.9.3 } \newline
\textbf{Samhita Paata} \newline

-मु॑पनि॒धाय॑ ब्राह्म॒णं द॑क्षिण॒तो नि॒षाद्य॑ भेष॒जं कु॑र्या॒द् याव॑दे॒व भे॑ष॒जं तेन॑ करोति स॒मर्द्धु॑कमस्य कृ॒तं भ॑वति ब्रह्मवा॒दिनो॑ वदन्ति॒  कस्मा᳚थ् स॒त्यादेक॑पात्रा द्विदेव॒त्या॑ गृ॒ह्यन्ते᳚ द्वि॒पात्रा॑ हूयन्त॒ इति॒ यदेक॑पात्रा गृ॒ह्यन्ते॒ तस्मा॒देको᳚ऽन्तर॒तः प्रा॒णो द्वि॒पात्रा॑ हूयन्ते॒ तस्मा॒द् द्वौद्वौ॑ ब॒हिष्टा᳚त् प्रा॒णाः प्रा॒णा वा ए॒ते यद् द्वि॑देव॒त्याः᳚ प॒शव॒ इडा॒ यदिडां॒ पूर्वां᳚ द्विदेव॒त्ये᳚भ्य उप॒ह्वये॑त- [  ] \newline

\textbf{Pada Paata} \newline

उ॒प॒नि॒धायेत्यु॑प - नि॒धाय॑ । ब्रा॒ह्म॒णम् । द॒क्षि॒ण॒तः । नि॒षाद्येति॑ नि - साद्य॑ । भे॒ष॒जम् । कु॒र्या॒त् । याव॑त् । ए॒व । भे॒ष॒जम् । तेन॑ । क॒रो॒ति॒ । स॒मद्‌र्धु॑क॒मिति॑ सं - अद्‌र्धु॑कम् । अ॒स्य॒ । कृ॒तम् । भ॒व॒ति॒ । ब्र॒ह्म॒वा॒दिन॒ इति॑ ब्रह्म - वा॒दिनः॑ । व॒द॒न्ति॒ । कस्मा᳚त् । स॒त्यात् । एक॑पात्रा॒ इत्येक॑ - पा॒त्राः॒ । द्वि॒दे॒व॒त्या॑ इति॑ द्वि-दे॒व॒त्याः᳚ । गृ॒ह्यन्ते᳚ । द्वि॒पात्रा॒ इति॑ द्वि - पात्राः᳚ । हू॒य॒न्ते॒ । इति॑ । यत् । एक॑पात्रा॒ इत्येक॑ - पा॒त्राः॒ । गृ॒ह्यन्ते᳚ । तस्मा᳚त् । एकः॑ । अ॒न्त॒र॒तः । प्रा॒ण इति॑ प्र - अ॒नः । द्वि॒पात्रा॒ इति॑ द्वि - पात्राः᳚ । हू॒य॒न्ते॒ । तस्मा᳚त् । द्वौद्वा॒विति॒ द्वौ - द्वौ॒ । ब॒हिष्टा᳚त् । प्रा॒णा इति॑ प्र - अ॒नाः । प्रा॒णा इति॑ प्र - अ॒नाः । वै । ए॒ते । यत् । द्वि॒दे॒व॒त्या॑ इति॑ द्वि - दे॒व॒त्याः᳚ । प॒शवः॑ । इडा᳚ । यत् । इडा᳚म् । पूर्वा᳚म् । द्वि॒दे॒व॒त्ये᳚भ्य॒ इति॑ द्वि - दे॒व॒त्ये᳚भ्यः । उ॒प॒ह्वये॒तेत्यु॑प - ह्वये॑त ।  \newline


\textbf{Krama Paata} \newline

उ॒प॒नि॒धाय॑ ब्राह्म॒णम् । उ॒प॒नि॒धायेत्यु॑प - नि॒धाय॑ । ब्रा॒ह्म॒णम् द॑क्षिण॒तः । द॒क्षि॒ण॒तो नि॒षाद्य॑ । नि॒षाद्य॑ भेष॒जम् । नि॒षाद्येति॑ नि - साद्य॑ । भे॒ष॒जम् कु॑र्यात् । कु॒र्या॒द् याव॑त् । याव॑दे॒व । ए॒व भे॑ष॒जम् । भे॒ष॒जम् तेन॑ । तेन॑ करोति । क॒रो॒ति॒ स॒मर्द्धु॑कम् । स॒मर्द्धु॑कमस्य । स॒मर्द्धु॑क॒मिति॑ सम् - अर्द्धु॑कम् । अ॒स्य॒ कृ॒तम् । कृ॒तम् भ॑वति । भ॒व॒ति॒ ब्र॒ह्म॒वा॒दिनः॑ । ब्र॒ह्म॒वा॒दिनो॑ वदन्ति । ब्र॒ह्म॒वा॒दिन॒ इति॑ ब्रह्म - वा॒दिनः॑ । व॒द॒न्ति॒ कस्मा᳚त् । कस्मा᳚थ् स॒त्यात् । स॒त्यादेक॑पात्राः । एक॑पात्रा द्विदेव॒त्याः᳚ । एक॑पात्रा॒ इत्येक॑ - पा॒त्राः॒ । द्वि॒दे॒व॒त्या॑ गृ॒ह्यन्ते᳚ । द्वि॒दे॒व॒त्या॑ इति॑ द्वि - दे॒व॒त्याः᳚ । गृ॒ह्यन्ते᳚ द्वि॒पात्राः᳚ । द्वि॒पात्रा॑ हूयन्ते । द्वि॒पात्रा॒ इति॑ द्वि - पात्राः᳚ । हू॒य॒न्त॒ इति॑ । इति॒ यत् । यदेक॑पात्राः । एक॑पात्रा गृ॒ह्यन्ते᳚ । एक॑पात्रा॒ इत्येक॑ - पा॒त्राः॒ । गृ॒ह्यन्ते॒ तस्मा᳚त् । तस्मा॒देकः॑ । एको᳚ऽन्तर॒तः । अ॒न्त॒र॒तः प्रा॒णः । प्रा॒णो द्वि॒पात्राः᳚ । प्रा॒ण इति॑ प्र - अ॒नः । द्वि॒पात्रा॑ हूयन्ते । द्वि॒पात्रा॒ इति॑ द्वि - पात्राः᳚ । हू॒य॒न्ते॒ तस्मा᳚त् । तस्मा॒द् द्वौद्वौ᳚ । द्वौद्वौ॑ ब॒हिष्टा᳚त् । द्वौद्वा॒विति॒ द्वौ - द्वौ॒ । ब॒हिष्टा᳚त् प्रा॒णाः । प्रा॒णाः प्रा॒णाः । प्रा॒णा इति॑ प्र - अ॒नाः । प्रा॒णा वै । प्रा॒णा इति॑ प्र - अ॒नाः । वा ए॒ते । ए॒ते यत् । यद् द्वि॑देव॒त्याः᳚ । द्वि॒दे॒व॒त्याः᳚ प॒शवः॑ । द्वि॒दे॒व॒त्या॑ इति॑ द्वि - दे॒व॒त्याः᳚ । प॒शव॒ इडा᳚ । इडा॒ यत् । यदिडा᳚म् । इडा॒म् पूर्वा᳚म् । पूर्वा᳚म् द्विदेव॒त्ये᳚भ्यः । द्वि॒दे॒व॒त्ये᳚भ्य उप॒ह्वये॑त । द्वि॒दे॒व॒त्ये᳚भ्य॒ इति॑ द्वि - दे॒व॒त्ये᳚भ्यः । उ॒प॒ह्वये॑त प॒शुभिः॑ । उ॒प॒ह्वये॒तेत्यु॑प - ह्वये॑त \newline

\textbf{Jatai Paata} \newline

1. उ॒प॒नि॒धाय॑ ब्राह्म॒णम् ब्रा᳚ह्म॒ण मु॑पनि॒धा यो॑पनि॒धाय॑ ब्राह्म॒णम् । \newline
2. उ॒प॒नि॒धायेत्यु॑प - नि॒धाय॑ । \newline
3. ब्रा॒ह्म॒णम् द॑क्षिण॒तो द॑क्षिण॒तो ब्रा᳚ह्म॒णम् ब्रा᳚ह्म॒णम् द॑क्षिण॒तः । \newline
4. द॒क्षि॒ण॒तो नि॒षाद्य॑ नि॒षाद्य॑ दक्षिण॒तो द॑क्षिण॒तो नि॒षाद्य॑ । \newline
5. नि॒षाद्य॑ भेष॒जम् भे॑ष॒जन् नि॒षाद्य॑ नि॒षाद्य॑ भेष॒जम् । \newline
6. नि॒षाद्येति॑ नि - साद्य॑ । \newline
7. भे॒ष॒जम् कु॑र्यात् कुर्याद् भेष॒जम् भे॑ष॒जम् कु॑र्यात् । \newline
8. कु॒र्या॒द् याव॒द् याव॑त् कुर्यात् कुर्या॒द् याव॑त् । \newline
9. याव॑दे॒ वैव याव॒द् याव॑ दे॒व । \newline
10. ए॒व भे॑ष॒जम् भे॑ष॒ज मे॒वैव भे॑ष॒जम् । \newline
11. भे॒ष॒जम् तेन॒ तेन॑ भेष॒जम् भे॑ष॒जम् तेन॑ । \newline
12. तेन॑ करोति करोति॒ तेन॒ तेन॑ करोति । \newline
13. क॒रो॒ति॒ स॒मर्द्धु॑कꣳ स॒मर्द्धु॑कम् करोति करोति स॒मर्द्धु॑कम् । \newline
14. स॒मर्द्धु॑क मस्यास्य स॒मर्द्धु॑कꣳ स॒मर्द्धु॑क मस्य । \newline
15. स॒मर्द्धु॑क॒मिति॑ सं - अर्द्धु॑कम् । \newline
16. अ॒स्य॒ कृ॒तम् कृ॒त म॑स्यास्य कृ॒तम् । \newline
17. कृ॒तम् भ॑वति भवति कृ॒तम् कृ॒तम् भ॑वति । \newline
18. भ॒व॒ति॒ ब्र॒ह्म॒वा॒दिनो᳚ ब्रह्मवा॒दिनो॑ भवति भवति ब्रह्मवा॒दिनः॑ । \newline
19. ब्र॒ह्म॒वा॒दिनो॑ वदन्ति वदन्ति ब्रह्मवा॒दिनो᳚ ब्रह्मवा॒दिनो॑ वदन्ति । \newline
20. ब्र॒ह्म॒वा॒दिन॒ इति॑ ब्रह्म - वा॒दिनः॑ । \newline
21. व॒द॒न्ति॒ कस्मा॒त् कस्मा᳚द् वदन्ति वदन्ति॒ कस्मा᳚त् । \newline
22. कस्मा᳚थ् स॒त्याथ् स॒त्यात् कस्मा॒त् कस्मा᳚थ् स॒त्यात् । \newline
23. स॒त्या देक॑पात्रा॒ एक॑पात्राः स॒त्याथ् स॒त्या देक॑पात्राः । \newline
24. एक॑पात्रा द्विदेव॒त्या᳚ द्विदेव॒त्या॑ एक॑पात्रा॒ एक॑पात्रा द्विदेव॒त्याः᳚ । \newline
25. एक॑पात्रा॒ इत्येक॑ - पा॒त्राः॒ । \newline
26. द्वि॒दे॒व॒त्या॑ गृ॒ह्यन्ते॑ गृ॒ह्यन्ते᳚ द्विदेव॒त्या᳚ द्विदेव॒त्या॑ गृ॒ह्यन्ते᳚ । \newline
27. द्वि॒दे॒व॒त्या॑ इति॑ द्वि - दे॒व॒त्याः᳚ । \newline
28. गृ॒ह्यन्ते᳚ द्वि॒पात्रा᳚ द्वि॒पात्रा॑ गृ॒ह्यन्ते॑ गृ॒ह्यन्ते᳚ द्वि॒पात्राः᳚ । \newline
29. द्वि॒पात्रा॑ हूयन्ते हूयन्ते द्वि॒पात्रा᳚ द्वि॒पात्रा॑ हूयन्ते । \newline
30. द्वि॒पात्रा॒ इति॑ द्वि - पात्राः᳚ । \newline
31. हू॒य॒न्त॒ इतीति॑ हूयन्ते हूयन्त॒ इति॑ । \newline
32. इति॒ यद् यदितीति॒ यत् । \newline
33. यदेक॑पात्रा॒ एक॑पात्रा॒ यद् यदेक॑पात्राः । \newline
34. एक॑पात्रा गृ॒ह्यन्ते॑ गृ॒ह्यन्त॒ एक॑पात्रा॒ एक॑पात्रा गृ॒ह्यन्ते᳚ । \newline
35. एक॑पात्रा॒ इत्येक॑ - पा॒त्राः॒ । \newline
36. गृ॒ह्यन्ते॒ तस्मा॒त् तस्मा᳚द् गृ॒ह्यन्ते॑ गृ॒ह्यन्ते॒ तस्मा᳚त् । \newline
37. तस्मा॒ देक॒ एक॒ स्तस्मा॒त् तस्मा॒ देकः॑ । \newline
38. एको᳚ ऽन्तर॒तो᳚ ऽन्तर॒त एक॒ एको᳚ ऽन्तर॒तः । \newline
39. अ॒न्त॒र॒तः प्रा॒णः प्रा॒णो᳚ ऽन्तर॒तो᳚ ऽन्तर॒तः प्रा॒णः । \newline
40. प्रा॒णो द्वि॒पात्रा᳚ द्वि॒पात्राः᳚ प्रा॒णः प्रा॒णो द्वि॒पात्राः᳚ । \newline
41. प्रा॒ण इति॑ प्र - अ॒नः । \newline
42. द्वि॒पात्रा॑ हूयन्ते हूयन्ते द्वि॒पात्रा᳚ द्वि॒पात्रा॑ हूयन्ते । \newline
43. द्वि॒पात्रा॒ इति॑ द्वि - पात्राः᳚ । \newline
44. हू॒य॒न्ते॒ तस्मा॒त् तस्मा᳚ द्धूयन्ते हूयन्ते॒ तस्मा᳚त् । \newline
45. तस्मा॒द् द्वौद्वौ॒ द्वौद्वौ॒ तस्मा॒त् तस्मा॒द् द्वौद्वौ᳚ । \newline
46. द्वौद्वौ॑ ब॒हिष्टा᳚द् ब॒हिष्टा॒द् द्वौद्वौ॒ द्वौद्वौ॑ ब॒हिष्टा᳚त् । \newline
47. द्वौद्वा॒विति॒ द्वौ - द्वौ॒ । \newline
48. ब॒हिष्टा᳚त् प्रा॒णाः प्रा॒णा ब॒हिष्टा᳚द् ब॒हिष्टा᳚त् प्रा॒णाः । \newline
49. प्रा॒णाः प्रा॒णाः । \newline
50. प्रा॒णा इति॑ प्र - अ॒नाः । \newline
51. प्रा॒णा वै वै प्रा॒णाः प्रा॒णा वै । \newline
52. प्रा॒णा इति॑ प्र - अ॒नाः । \newline
53. वा ए॒त ए॒ते वै वा ए॒ते । \newline
54. ए॒ते यद् यदे॒त ए॒ते यत् । \newline
55. यद् द्वि॑देव॒त्या᳚ द्विदेव॒त्या॑ यद् यद् द्वि॑देव॒त्याः᳚ । \newline
56. द्वि॒दे॒व॒त्याः᳚ प॒शवः॑ प॒शवो᳚ द्विदेव॒त्या᳚ द्विदेव॒त्याः᳚ प॒शवः॑ । \newline
57. द्वि॒दे॒व॒त्या॑ इति॑ द्वि - दे॒व॒त्याः᳚ । \newline
58. प॒शव॒ इडेडा॑ प॒शवः॑ प॒शव॒ इडा᳚ । \newline
59. इडा॒ यद् यदिडेडा॒ यत् । \newline
60. यदिडा॒ मिडां॒ ॅयद् यदिडा᳚म् । \newline
61. इडा॒म् पूर्वा॒म् पूर्वा॒ मिडा॒ मिडा॒म् पूर्वा᳚म् । \newline
62. पूर्वा᳚म् द्विदेव॒त्ये᳚भ्यो द्विदेव॒त्ये᳚भ्यः॒ पूर्वा॒म् पूर्वा᳚म् द्विदेव॒त्ये᳚भ्यः । \newline
63. द्वि॒दे॒व॒त्ये᳚भ्य उप॒ह्वये॑तो प॒ह्वये॑त द्विदेव॒त्ये᳚भ्यो द्विदेव॒त्ये᳚भ्य उप॒ह्वये॑त । \newline
64. द्वि॒दे॒व॒त्ये᳚भ्य॒ इति॑ द्वि - दे॒व॒त्ये᳚भ्यः । \newline
65. उ॒प॒ह्वये॑त प॒शुभिः॑ प॒शुभि॑ रुप॒ह्वये॑तो प॒ह्वये॑त प॒शुभिः॑ । \newline
66. उ॒प॒ह्वये॒तेत्यु॑प - ह्वये॑त । \newline

\textbf{Ghana Paata } \newline

1. उ॒प॒नि॒धाय॑ ब्राह्म॒णम् ब्रा᳚ह्म॒ण मु॑पनि॒धायो॑ पनि॒धाय॑ ब्राह्म॒णम् द॑क्षिण॒तो द॑क्षिण॒तो ब्रा᳚ह्म॒ण मु॑पनि॒धायो॑ पनि॒धाय॑ ब्राह्म॒णम् द॑क्षिण॒तः । \newline
2. उ॒प॒नि॒धायेत्यु॑प - नि॒धाय॑ । \newline
3. ब्रा॒ह्म॒णम् द॑क्षिण॒तो द॑क्षिण॒तो ब्रा᳚ह्म॒णम् ब्रा᳚ह्म॒णम् द॑क्षिण॒तो नि॒षाद्य॑ नि॒षाद्य॑ दक्षिण॒तो ब्रा᳚ह्म॒णम् ब्रा᳚ह्म॒णम् द॑क्षिण॒तो नि॒षाद्य॑ । \newline
4. द॒क्षि॒ण॒तो नि॒षाद्य॑ नि॒षाद्य॑ दक्षिण॒तो द॑क्षिण॒तो नि॒षाद्य॑ भेष॒जम् भे॑ष॒जन् नि॒षाद्य॑ दक्षिण॒तो द॑क्षिण॒तो नि॒षाद्य॑ भेष॒जम् । \newline
5. नि॒षाद्य॑ भेष॒जम् भे॑ष॒जन् नि॒षाद्य॑ नि॒षाद्य॑ भेष॒जम् कु॑र्यात् कुर्याद् भेष॒जन् नि॒षाद्य॑ नि॒षाद्य॑ भेष॒जम् कु॑र्यात् । \newline
6. नि॒षाद्येति॑ नि - साद्य॑ । \newline
7. भे॒ष॒जम् कु॑र्यात् कुर्याद् भेष॒जम् भे॑ष॒जम् कु॑र्या॒द् याव॒द् याव॑त् कुर्याद् भेष॒जम् भे॑ष॒जम् कु॑र्या॒द् याव॑त् । \newline
8. कु॒र्या॒द् याव॒द् याव॑त् कुर्यात् कुर्या॒द् याव॑ दे॒वैव याव॑त् कुर्यात् कुर्या॒द् याव॑ दे॒व । \newline
9. याव॑ दे॒वैव याव॒द् याव॑ दे॒व भे॑ष॒जम् भे॑ष॒ज मे॒व याव॒द् याव॑ दे॒व भे॑ष॒जम् । \newline
10. ए॒व भे॑ष॒जम् भे॑ष॒ज मे॒वैव भे॑ष॒जम् तेन॒ तेन॑ भेष॒ज मे॒वैव भे॑ष॒जम् तेन॑ । \newline
11. भे॒ष॒जम् तेन॒ तेन॑ भेष॒जम् भे॑ष॒जम् तेन॑ करोति करोति॒ तेन॑ भेष॒जम् भे॑ष॒जम् तेन॑ करोति । \newline
12. तेन॑ करोति करोति॒ तेन॒ तेन॑ करोति स॒मर्द्धु॑कꣳ स॒मर्द्धु॑कम् करोति॒ तेन॒ तेन॑ करोति स॒मर्द्धु॑कम् । \newline
13. क॒रो॒ति॒ स॒मर्द्धु॑कꣳ स॒मर्द्धु॑कम् करोति करोति स॒मर्द्धु॑क मस्यास्य स॒मर्द्धु॑कम् करोति करोति स॒मर्द्धु॑क मस्य । \newline
14. स॒मर्द्धु॑क मस्यास्य स॒मर्द्धु॑कꣳ स॒मर्द्धु॑क मस्य कृ॒तम् कृ॒त म॑स्य स॒मर्द्धु॑कꣳ स॒मर्द्धु॑क मस्य कृ॒तम् । \newline
15. स॒मर्द्धु॑क॒मिति॑ सं - अर्द्धु॑कम् । \newline
16. अ॒स्य॒ कृ॒तम् कृ॒त म॑स्यास्य कृ॒तम् भ॑वति भवति कृ॒त म॑स्यास्य कृ॒तम् भ॑वति । \newline
17. कृ॒तम् भ॑वति भवति कृ॒तम् कृ॒तम् भ॑वति ब्रह्मवा॒दिनो᳚ ब्रह्मवा॒दिनो॑ भवति कृ॒तम् कृ॒तम् भ॑वति ब्रह्मवा॒दिनः॑ । \newline
18. भ॒व॒ति॒ ब्र॒ह्म॒वा॒दिनो᳚ ब्रह्मवा॒दिनो॑ भवति भवति ब्रह्मवा॒दिनो॑ वदन्ति वदन्ति ब्रह्मवा॒दिनो॑ भवति भवति ब्रह्मवा॒दिनो॑ वदन्ति । \newline
19. ब्र॒ह्म॒वा॒दिनो॑ वदन्ति वदन्ति ब्रह्मवा॒दिनो᳚ ब्रह्मवा॒दिनो॑ वदन्ति॒ कस्मा॒त् कस्मा᳚द् वदन्ति ब्रह्मवा॒दिनो᳚ ब्रह्मवा॒दिनो॑ वदन्ति॒ कस्मा᳚त् । \newline
20. ब्र॒ह्म॒वा॒दिन॒ इति॑ ब्रह्म - वा॒दिनः॑ । \newline
21. व॒द॒न्ति॒ कस्मा॒त् कस्मा᳚द् वदन्ति वदन्ति॒ कस्मा᳚थ् स॒त्याथ् स॒त्यात् कस्मा᳚द् वदन्ति वदन्ति॒ कस्मा᳚थ् स॒त्यात् । \newline
22. कस्मा᳚थ् स॒त्याथ् स॒त्यात् कस्मा॒त् कस्मा᳚थ् स॒त्या देक॑पात्रा॒ एक॑पात्राः स॒त्यात् कस्मा॒त् कस्मा᳚थ् स॒त्या देक॑पात्राः । \newline
23. स॒त्या देक॑पात्रा॒ एक॑पात्राः स॒त्याथ् स॒त्या देक॑पात्रा द्विदेव॒त्या᳚ द्विदेव॒त्या॑ एक॑पात्राः स॒त्याथ् स॒त्या देक॑पात्रा द्विदेव॒त्याः᳚ । \newline
24. एक॑पात्रा द्विदेव॒त्या᳚ द्विदेव॒त्या॑ एक॑पात्रा॒ एक॑पात्रा द्विदेव॒त्या॑ गृ॒ह्यन्ते॑ गृ॒ह्यन्ते᳚ द्विदेव॒त्या॑ एक॑पात्रा॒ एक॑पात्रा द्विदेव॒त्या॑ गृ॒ह्यन्ते᳚ । \newline
25. एक॑पात्रा॒ इत्येक॑ - पा॒त्राः॒ । \newline
26. द्वि॒दे॒व॒त्या॑ गृ॒ह्यन्ते॑ गृ॒ह्यन्ते᳚ द्विदेव॒त्या᳚ द्विदेव॒त्या॑ गृ॒ह्यन्ते᳚ द्वि॒पात्रा᳚ द्वि॒पात्रा॑ गृ॒ह्यन्ते᳚ द्विदेव॒त्या᳚ द्विदेव॒त्या॑ गृ॒ह्यन्ते᳚ द्वि॒पात्राः᳚ । \newline
27. द्वि॒दे॒व॒त्या॑ इति॑ द्वि - दे॒व॒त्याः᳚ । \newline
28. गृ॒ह्यन्ते᳚ द्वि॒पात्रा᳚ द्वि॒पात्रा॑ गृ॒ह्यन्ते॑ गृ॒ह्यन्ते᳚ द्वि॒पात्रा॑ हूयन्ते हूयन्ते द्वि॒पात्रा॑ गृ॒ह्यन्ते॑ गृ॒ह्यन्ते᳚ द्वि॒पात्रा॑ हूयन्ते । \newline
29. द्वि॒पात्रा॑ हूयन्ते हूयन्ते द्वि॒पात्रा᳚ द्वि॒पात्रा॑ हूयन्त॒ इतीति॑ हूयन्ते द्वि॒पात्रा᳚ द्वि॒पात्रा॑ हूयन्त॒ इति॑ । \newline
30. द्वि॒पात्रा॒ इति॑ द्वि - पात्राः᳚ । \newline
31. हू॒य॒न्त॒ इतीति॑ हूयन्ते हूयन्त॒ इति॒ यद् यदिति॑ हूयन्ते हूयन्त॒ इति॒ यत् । \newline
32. इति॒ यद् यदितीति॒ यदेक॑पात्रा॒ एक॑पात्रा॒ यदि तीति॒ यदेक॑पात्राः । \newline
33. यदेक॑पात्रा॒ एक॑पात्रा॒ यद् यदेक॑पात्रा गृ॒ह्यन्ते॑ गृ॒ह्यन्त॒ एक॑पात्रा॒ यद् यदेक॑पात्रा गृ॒ह्यन्ते᳚ । \newline
34. एक॑पात्रा गृ॒ह्यन्ते॑ गृ॒ह्यन्त॒ एक॑पात्रा॒ एक॑पात्रा गृ॒ह्यन्ते॒ तस्मा॒त् तस्मा᳚द् गृ॒ह्यन्त॒ एक॑पात्रा॒ एक॑पात्रा गृ॒ह्यन्ते॒ तस्मा᳚त् । \newline
35. एक॑पात्रा॒ इत्येक॑ - पा॒त्राः॒ । \newline
36. गृ॒ह्यन्ते॒ तस्मा॒त् तस्मा᳚द् गृ॒ह्यन्ते॑ गृ॒ह्यन्ते॒ तस्मा॒ देक॒ एक॒ स्तस्मा᳚द् गृ॒ह्यन्ते॑ गृ॒ह्यन्ते॒ तस्मा॒ देकः॑ । \newline
37. तस्मा॒ देक॒ एक॒ स्तस्मा॒त् तस्मा॒ देको᳚ ऽन्तर॒तो᳚ ऽन्तर॒त एक॒ स्तस्मा॒त् तस्मा॒ देको᳚ ऽन्तर॒तः । \newline
38. एको᳚ ऽन्तर॒तो᳚ ऽन्तर॒त एक॒ एको᳚ ऽन्तर॒तः प्रा॒णः प्रा॒णो᳚ ऽन्तर॒त एक॒ एको᳚ ऽन्तर॒तः प्रा॒णः । \newline
39. अ॒न्त॒र॒तः प्रा॒णः प्रा॒णो᳚ ऽन्तर॒तो᳚ ऽन्तर॒तः प्रा॒णो द्वि॒पात्रा᳚ द्वि॒पात्राः᳚ प्रा॒णो᳚ ऽन्तर॒तो᳚ ऽन्तर॒तः प्रा॒णो द्वि॒पात्राः᳚ । \newline
40. प्रा॒णो द्वि॒पात्रा᳚ द्वि॒पात्राः᳚ प्रा॒णः प्रा॒णो द्वि॒पात्रा॑ हूयन्ते हूयन्ते द्वि॒पात्राः᳚ प्रा॒णः प्रा॒णो द्वि॒पात्रा॑ हूयन्ते । \newline
41. प्रा॒ण इति॑ प्र - अ॒नः । \newline
42. द्वि॒पात्रा॑ हूयन्ते हूयन्ते द्वि॒पात्रा᳚ द्वि॒पात्रा॑ हूयन्ते॒ तस्मा॒त् तस्मा᳚ द्धूयन्ते द्वि॒पात्रा᳚ द्वि॒पात्रा॑ हूयन्ते॒ तस्मा᳚त् । \newline
43. द्वि॒पात्रा॒ इति॑ द्वि - पात्राः᳚ । \newline
44. हू॒य॒न्ते॒ तस्मा॒त् तस्मा᳚ द्धूयन्ते हूयन्ते॒ तस्मा॒द् द्वौद्वौ॒ द्वौद्वौ॒ तस्मा᳚ द्धूयन्ते हूयन्ते॒ तस्मा॒द् द्वौद्वौ᳚ । \newline
45. तस्मा॒द् द्वौद्वौ॒ द्वौद्वौ॒ तस्मा॒त् तस्मा॒द् द्वौद्वौ॑ ब॒हिष्टा᳚द् ब॒हिष्टा॒द् द्वौद्वौ॒ तस्मा॒त् तस्मा॒द् द्वौद्वौ॑ ब॒हिष्टा᳚त् । \newline
46. द्वौद्वौ॑ ब॒हिष्टा᳚द् ब॒हिष्टा॒द् द्वौद्वौ॒ द्वौद्वौ॑ ब॒हिष्टा᳚त् प्रा॒णाः प्रा॒णा ब॒हिष्टा॒द् द्वौद्वौ॒ द्वौद्वौ॑ ब॒हिष्टा᳚त् प्रा॒णाः । \newline
47. द्वौद्वा॒विति॒ द्वौ - द्वौ॒ । \newline
48. ब॒हिष्टा᳚त् प्रा॒णाः प्रा॒णा ब॒हिष्टा᳚द् ब॒हिष्टा᳚त् प्रा॒णाः । \newline
49. प्रा॒णाः प्रा॒णाः । \newline
50. प्रा॒णा इति॑ प्र - अ॒नाः । \newline
51. प्रा॒णा वै वै प्रा॒णाः प्रा॒णा वा ए॒त ए॒ते वै प्रा॒णाः प्रा॒णा वा ए॒ते । \newline
52. प्रा॒णा इति॑ प्र - अ॒नाः । \newline
53. वा ए॒त ए॒ते वै वा ए॒ते यद् यदे॒ते वै वा ए॒ते यत् । \newline
54. ए॒ते यद् यदे॒त ए॒ते यद् द्वि॑देव॒त्या᳚ द्विदेव॒त्या॑ यदे॒त ए॒ते यद् द्वि॑देव॒त्याः᳚ । \newline
55. यद् द्वि॑देव॒त्या᳚ द्विदेव॒त्या॑ यद् यद् द्वि॑देव॒त्याः᳚ प॒शवः॑ प॒शवो᳚ द्विदेव॒त्या॑ यद् यद् द्वि॑देव॒त्याः᳚ प॒शवः॑ । \newline
56. द्वि॒दे॒व॒त्याः᳚ प॒शवः॑ प॒शवो᳚ द्विदेव॒त्या᳚ द्विदेव॒त्याः᳚ प॒शव॒ इडेडा॑ प॒शवो᳚ द्विदेव॒त्या᳚ द्विदेव॒त्याः᳚ प॒शव॒ इडा᳚ । \newline
57. द्वि॒दे॒व॒त्या॑ इति॑ द्वि - दे॒व॒त्याः᳚ । \newline
58. प॒शव॒ इडेडा॑ प॒शवः॑ प॒शव॒ इडा॒ यद् यदिडा॑ प॒शवः॑ प॒शव॒ इडा॒ यत् । \newline
59. इडा॒ यद् यदिडेडा॒ यदिडा॒ मिडां॒ ॅयदिडेडा॒ यदिडा᳚म् । \newline
60. यदिडा॒ मिडां॒ ॅयद् यदिडा॒म् पूर्वा॒म् पूर्वा॒ मिडां॒ ॅयद् यदिडा॒म् पूर्वा᳚म् । \newline
61. इडा॒म् पूर्वा॒म् पूर्वा॒ मिडा॒ मिडा॒म् पूर्वा᳚म् द्विदेव॒त्ये᳚भ्यो द्विदेव॒त्ये᳚भ्यः॒ पूर्वा॒ मिडा॒ मिडा॒म् पूर्वा᳚म् द्विदेव॒त्ये᳚भ्यः । \newline
62. पूर्वा᳚म् द्विदेव॒त्ये᳚भ्यो द्विदेव॒त्ये᳚भ्यः॒ पूर्वा॒म् पूर्वा᳚म् द्विदेव॒त्ये᳚भ्य उप॒ह्वये॑ तोप॒ह्वये॑त द्विदेव॒त्ये᳚भ्यः॒ पूर्वा॒म् पूर्वा᳚म् द्विदेव॒त्ये᳚भ्य उप॒ह्वये॑त । \newline
63. द्वि॒दे॒व॒त्ये᳚भ्य उप॒ह्वये॑ तोप॒ह्वये॑त द्विदेव॒त्ये᳚भ्यो द्विदेव॒त्ये᳚भ्य उप॒ह्वये॑त प॒शुभिः॑ प॒शुभि॑ रुप॒ह्वये॑त द्विदेव॒त्ये᳚भ्यो द्विदेव॒त्ये᳚भ्य उप॒ह्वये॑त प॒शुभिः॑ । \newline
64. द्वि॒दे॒व॒त्ये᳚भ्य॒ इति॑ द्वि - दे॒व॒त्ये᳚भ्यः । \newline
65. उ॒प॒ह्वये॑त प॒शुभिः॑ प॒शुभि॑ रुप॒ह्वये॑ तोप॒ह्वये॑त प॒शुभिः॑ प्रा॒णान् प्रा॒णान् प॒शुभि॑ रुप॒ह्वये॑
तोप॒ह्वये॑त प॒शुभिः॑ प्रा॒णान् । \newline
66. उ॒प॒ह्वये॒तेत्यु॑प - ह्वये॑त । \newline
\pagebreak
\markright{ TS 6.4.9.4  \hfill https://www.vedavms.in \hfill}

\section{ TS 6.4.9.4 }

\textbf{TS 6.4.9.4 } \newline
\textbf{Samhita Paata} \newline

प॒शुभिः॑ प्रा॒णान॒न्तर्द॑धीत प्र॒मायु॑कः स्याद् द्विदेव॒त्या᳚न् भक्षयि॒त्वेडा॒मुप॑ ह्वयते प्रा॒णाने॒वाऽऽ*त्मन् धि॒त्वा प॒शूनुप॑ ह्वयते॒ वाग्वा ऐ᳚न्द्रवाय॒वश्चक्षु॑-र्मैत्रावरु॒णः श्रोत्र॑माश्वि॒नः पु॒रस्ता॑दैन्द्रवाय॒वं भ॑क्षयति॒ तस्मा᳚त् पु॒रस्ता᳚द् वा॒चा व॑दति पु॒रस्ता᳚न्मैत्रावरु॒णं तस्मा᳚त् पु॒रस्ता॒च्चक्षु॑षा पश्यति स॒र्वतः॑ परि॒हार॑माश्वि॒नं तस्मा᳚थ् स॒र्वतः॒ श्रोत्रे॑ण शृणोति प्रा॒णा वा ए॒ते यद् द्वि॑देव॒त्या॑- [  ] \newline

\textbf{Pada Paata} \newline

प॒शुभि॒रिति॑ प॒शु - भिः॒ । प्रा॒णानिति॑ प्र-अ॒नान् । अ॒न्तः । द॒धी॒त॒ । प्र॒मायु॑क॒ इति॑ प्र - मायु॑कः । स्या॒त् । द्वि॒दे॒व॒त्या॑निति॑ द्वि-दे॒व॒त्यान्॑ । भ॒क्ष॒यि॒त्वा । इडा᳚म् । उपेति॑ । ह्व॒य॒ते॒ । प्रा॒णानिति॑ प्र - अ॒नान् । ए॒व । आ॒त्मन्न् । धि॒त्वा । प॒शून् । उपेति॑ । ह्व॒य॒ते॒ । वाक् । वै । ऐ॒न्द्र॒वा॒य॒व इत्यै᳚न्द्र-वा॒य॒वः । चक्षुः॑ । मै॒त्रा॒व॒रु॒ण इति॑ मैत्रा-व॒रु॒णः । श्रोत्र᳚म् । आ॒श्वि॒नः । पु॒रस्ता᳚त् । ऐ॒न्द्र॒वा॒य॒वमित्यै᳚न्द्र - वा॒य॒वम् । भ॒क्ष॒य॒ति॒ । तस्मा᳚त् । पु॒रस्ता᳚त् । वा॒चा । व॒द॒ति॒ । पु॒रस्ता᳚त् । मै॒त्रा॒व॒रु॒णमिति॑ मैत्रा - व॒रु॒णम् । तस्मा᳚त् । पु॒रस्ता᳚त् । चक्षु॑षा । प॒श्य॒ति॒ । स॒र्वतः॑ । प॒रि॒हार॒मिति॑ परि-हार᳚म् । आ॒श्वि॒नम् । तस्मा᳚त् । स॒र्वतः॑ । श्रोत्रे॑ण । शृ॒णो॒ति॒ । प्रा॒णा इति॑ प्र - अ॒नाः । वै । ए॒ते । यत् । द्वि॒दे॒व॒त्या॑ इति॑ द्वि - दे॒व॒त्याः᳚ ।  \newline


\textbf{Krama Paata} \newline

प॒शुभिः॑ प्रा॒णान् । प॒शुभि॒रिति॑ प॒शु - भिः॒ । प्रा॒णान॒न्तः । प्रा॒णानि॑ति प्र - अ॒नान् । अ॒न्तर् द॑धीत । द॒धी॒त॒ प्र॒मायु॑कः । प्र॒मायु॑कः स्यात् । प्र॒मायु॑क॒ इति॑ प्र - मायु॑कः । स्या॒द् द्वि॒दे॒व॒त्यान्॑ । द्वि॒दे॒व॒त्या᳚न् भक्षयि॒त्वा । द्वि॒दे॒व॒त्या॑निति॑ द्वि - दे॒व॒त्यान्॑ । भ॒क्ष॒यि॒त्वेडा᳚म् । इडा॒मुप॑ । उप॑ ह्वयते । ह्व॒य॒ते॒ प्रा॒णान् । प्रा॒णाने॒व । प्रा॒णानिति॑ प्र - अ॒नान् । ए॒वात्मन्न् । आ॒त्मन् धि॒त्वा । धि॒त्वा प॒शून् । प॒शूनुप॑ । उप॑ ह्वयते । ह्व॒य॒ते॒ वाक् । वाग् वै । वा ऐ᳚न्द्रवाय॒वः । ऐ॒न्द्र॒वा॒य॒वश्चक्षुः॑ । ऐ॒न्द॒वा॒य॒व इत्यै᳚न्द्र - वा॒य॒वः । चक्षु॑र् मैत्रावरु॒णः । मै॒त्र॒व॒रु॒णः श्रोत्र᳚म् । मै॒त्रा॒व॒रु॒ण इति॑ मैत्रा - व॒रु॒णः । श्रोत्र॑माश्वि॒नः । आ॒श्वि॒नः पु॒रस्ता᳚त् । पु॒रस्ता॑दैन्द्रवाय॒वम् । ऐ॒न्द्र॒वा॒य॒वम् भ॑क्षयति । ऐ॒न्द्र॒वा॒य॒वमित्यै᳚न्द्र - वा॒य॒वम् । भ॒क्ष॒य॒ति॒ तस्मा᳚त् । तस्मा᳚त् पु॒रस्ता᳚त् । पु॒रस्ता᳚द् वा॒चा । वा॒चा व॑दति । व॒द॒ति॒ पु॒रस्ता᳚त् । पु॒रस्ता᳚न् मैत्रावरु॒णम् । मै॒त्रा॒व॒रु॒णम् तस्मा᳚त् । मै॒त्रा॒व॒रु॒णमिति॑ मैत्रा - व॒रु॒णम् । तस्मा᳚त् पु॒रस्ता᳚त् । पु॒रस्ता॒च् चक्षु॑षा । चक्षु॑षा पश्यति । प॒श्य॒ति॒ स॒र्वतः॑ । स॒र्वतः॑ परि॒हार᳚म् । प॒रि॒हार॑माश्वि॒नम् । प॒रि॒हार॒मिति॑ परि - हार᳚म् । आ॒श्वि॒नम् तस्मा᳚त् । तस्मा᳚थ् स॒र्वतः॑ । स॒र्वतः॒ श्रोत्रे॑ण । श्रोत्रे॑ण शृणोति । शृ॒णो॒ति॒ प्रा॒णाः । प्रा॒णा वै । प्रा॒णा इति॑ प्र - अ॒नाः । वा ए॒ते । ए॒ते यत् । यद् द्वि॑देव॒त्याः᳚ ( ) । द्वि॒दे॒व॒त्या॑ अरि॑क्तानि । द्वि॒दे॒व॒त्या॑ इति॑ द्वि - दे॒वत्याः᳚ \newline

\textbf{Jatai Paata} \newline

1. प॒शुभिः॑ प्रा॒णान् प्रा॒णान् प॒शुभिः॑ प॒शुभिः॑ प्रा॒णान् । \newline
2. प॒शुभि॒रिति॑ प॒शु - भिः॒ । \newline
3. प्रा॒णा न॒न्त र॒न्तः प्रा॒णान् प्रा॒णा न॒न्तः । \newline
4. प्रा॒णानिति॑ प्र - अ॒नान् । \newline
5. अ॒न्तर् द॑धीत दधीता॒न्त र॒न्तर् द॑धीत । \newline
6. द॒धी॒त॒ प्र॒मायु॑कः प्र॒मायु॑को दधीत दधीत प्र॒मायु॑कः । \newline
7. प्र॒मायु॑कः स्याथ् स्यात् प्र॒मायु॑कः प्र॒मायु॑कः स्यात् । \newline
8. प्र॒मायु॑क॒ इति॑ प्र - मायु॑कः । \newline
9. स्या॒द् द्वि॒दे॒व॒त्या᳚न् द्विदेव॒त्या᳚न् थ्स्याथ् स्याद् द्विदेव॒त्यान्॑ । \newline
10. द्वि॒दे॒व॒त्या᳚न् भक्षयि॒त्वा भ॑क्षयि॒त्वा द्वि॑देव॒त्या᳚न् द्विदेव॒त्या᳚न् भक्षयि॒त्वा । \newline
11. द्वि॒दे॒व॒त्या॑निति॑ द्वि - दे॒व॒त्यान्॑ । \newline
12. भ॒क्ष॒यि॒त्वेडा॒ मिडा᳚म् भक्षयि॒त्वा भ॑क्षयि॒त्वेडा᳚म् । \newline
13. इडा॒ मुपोपेडा॒ मिडा॒ मुप॑ । \newline
14. उप॑ ह्वयते ह्वयत॒ उपोप॑ ह्वयते । \newline
15. ह्व॒य॒ते॒ प्रा॒णान् प्रा॒णान् ह्व॑यते ह्वयते प्रा॒णान् । \newline
16. प्रा॒णा ने॒वैव प्रा॒णान् प्रा॒णाने॒व । \newline
17. प्रा॒णानिति॑ प्र - अ॒नान् । \newline
18. ए॒वात्मन् ना॒त्मन् ने॒वै वात्मन्न् । \newline
19. आ॒त्मन् धि॒त्वा धि॒त्वा ऽऽत्मन् ना॒त्मन् धि॒त्वा । \newline
20. धि॒त्वा प॒शून् प॒शून् धि॒त्वा धि॒त्वा प॒शून् । \newline
21. प॒शूनु पोप॑ प॒शून् प॒शूनुप॑ । \newline
22. उप॑ ह्वयते ह्वयत॒ उपोप॑ ह्वयते । \newline
23. ह्व॒य॒ते॒ वाग् वाग्घ्व॑यते ह्वयते॒ वाक् । \newline
24. वाग् वै वै वाग् वाग् वै । \newline
25. वा ऐ᳚न्द्रवाय॒व ऐ᳚न्द्रवाय॒वो वै वा ऐ᳚न्द्रवाय॒वः । \newline
26. ऐ॒न्द्र॒वा॒य॒व श्चक्षु॒ श्चक्षु॑ रैन्द्रवाय॒व ऐ᳚न्द्रवाय॒व श्चक्षुः॑ । \newline
27. ऐ॒न्द्र॒वा॒य॒व इत्यै᳚न्द्र - वा॒य॒वः । \newline
28. चक्षु॑र् मैत्रावरु॒णो मै᳚त्रावरु॒ण श्चक्षु॒ श्चक्षु॑र् मैत्रावरु॒णः । \newline
29. मै॒त्रा॒व॒रु॒णः श्रोत्रꣳ॒॒ श्रोत्र॑म् मैत्रावरु॒णो मै᳚त्रावरु॒णः श्रोत्र᳚म् । \newline
30. मै॒त्रा॒व॒रु॒ण इति॑ मैत्रा - व॒रु॒णः । \newline
31. श्रोत्र॑ माश्वि॒न आ᳚श्वि॒नः श्रोत्रꣳ॒॒ श्रोत्र॑ माश्वि॒नः । \newline
32. आ॒श्वि॒नः पु॒रस्ता᳚त् पु॒रस्ता॑ दाश्वि॒न आ᳚श्वि॒नः पु॒रस्ता᳚त् । \newline
33. पु॒रस्ता॑ दैन्द्रवाय॒व मै᳚न्द्रवाय॒वम् पु॒रस्ता᳚त् पु॒रस्ता॑ दैन्द्रवाय॒वम् । \newline
34. ऐ॒न्द्र॒वा॒य॒वम् भ॑क्षयति भक्षय त्यैन्द्रवाय॒व मै᳚न्द्रवाय॒वम् भ॑क्षयति । \newline
35. ऐ॒न्द्र॒वा॒य॒वमित्यै᳚न्द्र - वा॒य॒वम् । \newline
36. भ॒क्ष॒य॒ति॒ तस्मा॒त् तस्मा᳚द् भक्षयति भक्षयति॒ तस्मा᳚त् । \newline
37. तस्मा᳚त् पु॒रस्ता᳚त् पु॒रस्ता॒त् तस्मा॒त् तस्मा᳚त् पु॒रस्ता᳚त् । \newline
38. पु॒रस्ता᳚द् वा॒चा वा॒चा पु॒रस्ता᳚त् पु॒रस्ता᳚द् वा॒चा । \newline
39. वा॒चा व॑दति वदति वा॒चा वा॒चा व॑दति । \newline
40. व॒द॒ति॒ पु॒रस्ता᳚त् पु॒रस्ता᳚द् वदति वदति पु॒रस्ता᳚त् । \newline
41. पु॒रस्ता᳚न् मैत्रावरु॒णम् मै᳚त्रावरु॒णम् पु॒रस्ता᳚त् पु॒रस्ता᳚न् मैत्रावरु॒णम् । \newline
42. मै॒त्रा॒व॒रु॒णम् तस्मा॒त् तस्मा᳚न् मैत्रावरु॒णम् मै᳚त्रावरु॒णम् तस्मा᳚त् । \newline
43. मै॒त्रा॒व॒रु॒णमिति॑ मैत्रा - व॒रु॒णम् । \newline
44. तस्मा᳚त् पु॒रस्ता᳚त् पु॒रस्ता॒त् तस्मा॒त् तस्मा᳚त् पु॒रस्ता᳚त् । \newline
45. पु॒रस्ता॒च् चक्षु॑षा॒ चक्षु॑षा पु॒रस्ता᳚त् पु॒रस्ता॒च् चक्षु॑षा । \newline
46. चक्षु॑षा पश्यति पश्यति॒ चक्षु॑षा॒ चक्षु॑षा पश्यति । \newline
47. प॒श्य॒ति॒ स॒र्वतः॑ स॒र्वतः॑ पश्यति पश्यति स॒र्वतः॑ । \newline
48. स॒र्वतः॑ परि॒हार॑म् परि॒हारꣳ॑ स॒र्वतः॑ स॒र्वतः॑ परि॒हार᳚म् । \newline
49. प॒रि॒हार॑ माश्वि॒न मा᳚श्वि॒नम् प॑रि॒हार॑म् परि॒हार॑ माश्वि॒नम् । \newline
50. प॒रि॒हार॒मिति॑ परि - हार᳚म् । \newline
51. आ॒श्वि॒नम् तस्मा॒त् तस्मा॑ दाश्वि॒न मा᳚श्वि॒नम् तस्मा᳚त् । \newline
52. तस्मा᳚थ् स॒र्वतः॑ स॒र्वत॒ स्तस्मा॒त् तस्मा᳚थ् स॒र्वतः॑ । \newline
53. स॒र्वतः॒ श्रोत्रे॑ण॒ श्रोत्रे॑ण स॒र्वतः॑ स॒र्वतः॒ श्रोत्रे॑ण । \newline
54. श्रोत्रे॑ण शृणोति शृणोति॒ श्रोत्रे॑ण॒ श्रोत्रे॑ण शृणोति । \newline
55. शृ॒णो॒ति॒ प्रा॒णाः प्रा॒णाः शृ॑णोति शृणोति प्रा॒णाः । \newline
56. प्रा॒णा वै वै प्रा॒णाः प्रा॒णा वै । \newline
57. प्रा॒णा इति॑ प्र - अ॒नाः । \newline
58. वा ए॒त ए॒ते वै वा ए॒ते । \newline
59. ए॒ते यद् यदे॒त ए॒ते यत् । \newline
60. यद् द्वि॑देव॒त्या᳚ द्विदेव॒त्या॑ यद् यद् द्वि॑देव॒त्याः᳚ । \newline
61. द्वि॒दे॒व॒त्या॑ अरि॑क्ता॒ न्यरि॑क्तानि द्विदेव॒त्या᳚ द्विदेव॒त्या॑ अरि॑क्तानि । \newline
62. द्वि॒दे॒व॒त्या॑ इति॑ द्वि - दे॒व॒त्याः᳚ । \newline

\textbf{Ghana Paata } \newline

1. प॒शुभिः॑ प्रा॒णान् प्रा॒णान् प॒शुभिः॑ प॒शुभिः॑ प्रा॒णा न॒न्त र॒न्तः प्रा॒णान् प॒शुभिः॑ प॒शुभिः॑ प्रा॒णा न॒न्तः । \newline
2. प॒शुभि॒रिति॑ प॒शु - भिः॒ । \newline
3. प्रा॒णा न॒न्त र॒न्तः प्रा॒णान् प्रा॒णा न॒न्तर् द॑धीत दधीता॒न्तः प्रा॒णान् प्रा॒णा न॒न्तर् द॑धीत । \newline
4. प्रा॒णानिति॑ प्र - अ॒नान् । \newline
5. अ॒न्तर् द॑धीत दधीता॒न्त र॒न्तर् द॑धीत प्र॒मायु॑कः प्र॒मायु॑को दधीता॒न्त र॒न्तर् द॑धीत प्र॒मायु॑कः । \newline
6. द॒धी॒त॒ प्र॒मायु॑कः प्र॒मायु॑को दधीत दधीत प्र॒मायु॑कः स्याथ् स्यात् प्र॒मायु॑को दधीत दधीत प्र॒मायु॑कः स्यात् । \newline
7. प्र॒मायु॑कः स्याथ् स्यात् प्र॒मायु॑कः प्र॒मायु॑कः स्याद् द्विदेव॒त्या᳚न् द्विदेव॒त्या᳚न् थ्स्यात् प्र॒मायु॑कः प्र॒मायु॑कः स्याद् द्विदेव॒त्यान्॑ । \newline
8. प्र॒मायु॑क॒ इति॑ प्र - मायु॑कः । \newline
9. स्या॒द् द्वि॒दे॒व॒त्या᳚न् द्विदेव॒त्या᳚न् थ्स्याथ् स्याद् द्विदेव॒त्या᳚न् भक्षयि॒त्वा भ॑क्षयि॒त्वा द्वि॑देव॒त्या᳚न् थ्स्याथ् स्याद् द्विदेव॒त्या᳚न् भक्षयि॒त्वा । \newline
10. द्वि॒दे॒व॒त्या᳚न् भक्षयि॒त्वा भ॑क्षयि॒त्वा द्वि॑देव॒त्या᳚न् द्विदेव॒त्या᳚न् भक्षयि॒ त्वेडा॒ मिडा᳚म् भक्षयि॒त्वा द्वि॑देव॒त्या᳚न् द्विदेव॒त्या᳚न् भक्षयि॒ त्वेडा᳚म् । \newline
11. द्वि॒दे॒व॒त्या॑निति॑ द्वि - दे॒व॒त्यान्॑ । \newline
12. भ॒क्ष॒यि॒ त्वेडा॒ मिडा᳚म् भक्षयि॒त्वा भ॑क्षयि॒ त्वेडा॒ मुपोपेडा᳚म् भक्षयि॒त्वा भ॑क्षयि॒ त्वेडा॒ मुप॑ । \newline
13. इडा॒ मुपोपेडा॒ मिडा॒ मुप॑ ह्वयते ह्वयत॒ उपेडा॒ मिडा॒ मुप॑ ह्वयते । \newline
14. उप॑ ह्वयते ह्वयत॒ उपोप॑ ह्वयते प्रा॒णान् प्रा॒णान् ह्व॑यत॒ उपोप॑ ह्वयते प्रा॒णान् । \newline
15. ह्व॒य॒ते॒ प्रा॒णान् प्रा॒णान् ह्व॑यते ह्वयते प्रा॒णा ने॒वैव प्रा॒णान् ह्व॑यते ह्वयते प्रा॒णाने॒व । \newline
16. प्रा॒णा ने॒वैव प्रा॒णान् प्रा॒णा ने॒वात्मन् ना॒त्मन् ने॒व प्रा॒णान् प्रा॒णा ने॒वात्मन्न् । \newline
17. प्रा॒णानिति॑ प्र - अ॒नान् । \newline
18. ए॒वात्मन् ना॒त्मन् ने॒वैवात्मन् धि॒त्वा धि॒त्वा ऽऽत्मन् ने॒वैवात्मन् धि॒त्वा । \newline
19. आ॒त्मन् धि॒त्वा धि॒त्वा ऽऽत्मन् ना॒त्मन् धि॒त्वा प॒शून् प॒शून् धि॒त्वा ऽऽत्मन् ना॒त्मन् धि॒त्वा प॒शून् । \newline
20. धि॒त्वा प॒शून् प॒शून् धि॒त्वा धि॒त्वा प॒शूनुपोप॑ प॒शून् धि॒त्वा धि॒त्वा प॒शूनुप॑ । \newline
21. प॒शूनुपोप॑ प॒शून् प॒शू नुप॑ह्वयते ह्वयत॒ उप॑ प॒शून् प॒शू नुप॑ह्वयते । \newline
22. उप॑ ह्वयते ह्वयत॒ उपोप॑ ह्वयते॒ वाग् वाग्घ्व॑यत॒ उपोप॑ ह्वयते॒ वाक् । \newline
23. ह्व॒य॒ते॒ वाग् वाग् घ्व॑यते ह्वयते॒ वाग् वै वै वाग् घ्व॑यते ह्वयते॒ वाग् वै । \newline
24. वाग् वै वै वाग् वाग् वा ऐ᳚न्द्रवाय॒व ऐ᳚न्द्रवाय॒वो वै वाग् वाग् वा ऐ᳚न्द्रवाय॒वः । \newline
25. वा ऐ᳚न्द्रवाय॒व ऐ᳚न्द्रवाय॒वो वै वा ऐ᳚न्द्रवाय॒व श्चक्षु॒ श्चक्षु॑ रैन्द्रवाय॒वो वै वा ऐ᳚न्द्रवाय॒व श्चक्षुः॑ । \newline
26. ऐ॒न्द्र॒वा॒य॒व श्चक्षु॒ श्चक्षु॑ रैन्द्रवाय॒व ऐ᳚न्द्रवाय॒व श्चक्षु॑र् मैत्रावरु॒णो मै᳚त्रावरु॒ण श्चक्षु॑ रैन्द्रवाय॒व ऐ᳚न्द्रवाय॒व श्चक्षु॑र् मैत्रावरु॒णः । \newline
27. ऐ॒न्द्र॒वा॒य॒व इत्यै᳚न्द्र - वा॒य॒वः । \newline
28. चक्षु॑र् मैत्रावरु॒णो मै᳚त्रावरु॒ण श्चक्षु॒ श्चक्षु॑र् मैत्रावरु॒णः श्रोत्रꣳ॒॒ श्रोत्र॑म् मैत्रावरु॒ण श्चक्षु॒ श्चक्षु॑र् मैत्रावरु॒णः श्रोत्र᳚म् । \newline
29. मै॒त्रा॒व॒रु॒णः श्रोत्रꣳ॒॒ श्रोत्र॑म् मैत्रावरु॒णो मै᳚त्रावरु॒णः श्रोत्र॑ माश्वि॒न आ᳚श्वि॒नः श्रोत्र॑म् मैत्रावरु॒णो मै᳚त्रावरु॒णः श्रोत्र॑ माश्वि॒नः । \newline
30. मै॒त्रा॒व॒रु॒ण इति॑ मैत्रा - व॒रु॒णः । \newline
31. श्रोत्र॑ माश्वि॒न आ᳚श्वि॒नः श्रोत्रꣳ॒॒ श्रोत्र॑ माश्वि॒नः पु॒रस्ता᳚त् पु॒रस्ता॑ दाश्वि॒नः श्रोत्रꣳ॒॒ श्रोत्र॑ माश्वि॒नः पु॒रस्ता᳚त् । \newline
32. आ॒श्वि॒नः पु॒रस्ता᳚त् पु॒रस्ता॑ दाश्वि॒न आ᳚श्वि॒नः पु॒रस्ता॑ दैन्द्रवाय॒व मै᳚न्द्रवाय॒वम् पु॒रस्ता॑ दाश्वि॒न आ᳚श्वि॒नः पु॒रस्ता॑ दैन्द्रवाय॒वम् । \newline
33. पु॒रस्ता॑ दैन्द्रवाय॒व मै᳚न्द्रवाय॒वम् पु॒रस्ता᳚त् पु॒रस्ता॑ दैन्द्रवाय॒वम् भ॑क्षयति भक्षय त्यैन्द्रवाय॒वम् पु॒रस्ता᳚त् पु॒रस्ता॑ दैन्द्रवाय॒वम् भ॑क्षयति । \newline
34. ऐ॒न्द्र॒वा॒य॒वम् भ॑क्षयति भक्षय त्यैन्द्रवाय॒व मै᳚न्द्रवाय॒वम् भ॑क्षयति॒ तस्मा॒त् तस्मा᳚द् भक्षय त्यैन्द्रवाय॒व मै᳚न्द्रवाय॒वम् भ॑क्षयति॒ तस्मा᳚त् । \newline
35. ऐ॒न्द्र॒वा॒य॒वमित्यै᳚न्द्र - वा॒य॒वम् । \newline
36. भ॒क्ष॒य॒ति॒ तस्मा॒त् तस्मा᳚द् भक्षयति भक्षयति॒ तस्मा᳚त् पु॒रस्ता᳚त् पु॒रस्ता॒त् तस्मा᳚द् भक्षयति भक्षयति॒ तस्मा᳚त् पु॒रस्ता᳚त् । \newline
37. तस्मा᳚त् पु॒रस्ता᳚त् पु॒रस्ता॒त् तस्मा॒त् तस्मा᳚त् पु॒रस्ता᳚द् वा॒चा वा॒चा पु॒रस्ता॒त् तस्मा॒त् तस्मा᳚त् पु॒रस्ता᳚द् वा॒चा । \newline
38. पु॒रस्ता᳚द् वा॒चा वा॒चा पु॒रस्ता᳚त् पु॒रस्ता᳚द् वा॒चा व॑दति वदति वा॒चा पु॒रस्ता᳚त् पु॒रस्ता᳚द् वा॒चा व॑दति । \newline
39. वा॒चा व॑दति वदति वा॒चा वा॒चा व॑दति पु॒रस्ता᳚त् पु॒रस्ता᳚द् वदति वा॒चा वा॒चा व॑दति पु॒रस्ता᳚त् । \newline
40. व॒द॒ति॒ पु॒रस्ता᳚त् पु॒रस्ता᳚द् वदति वदति पु॒रस्ता᳚न् मैत्रावरु॒णम् मै᳚त्रावरु॒णम् पु॒रस्ता᳚द् वदति वदति पु॒रस्ता᳚न् मैत्रावरु॒णम् । \newline
41. पु॒रस्ता᳚न् मैत्रावरु॒णम् मै᳚त्रावरु॒णम् पु॒रस्ता᳚त् पु॒रस्ता᳚न् मैत्रावरु॒णम् तस्मा॒त् तस्मा᳚न् मैत्रावरु॒णम् पु॒रस्ता᳚त् पु॒रस्ता᳚न् मैत्रावरु॒णम् तस्मा᳚त् । \newline
42. मै॒त्रा॒व॒रु॒णम् तस्मा॒त् तस्मा᳚न् मैत्रावरु॒णम् मै᳚त्रावरु॒णम् तस्मा᳚त् पु॒रस्ता᳚त् पु॒रस्ता॒त् तस्मा᳚न् मैत्रावरु॒णम् मै᳚त्रावरु॒णम् तस्मा᳚त् पु॒रस्ता᳚त् । \newline
43. मै॒त्रा॒व॒रु॒णमिति॑ मैत्रा - व॒रु॒णम् । \newline
44. तस्मा᳚त् पु॒रस्ता᳚त् पु॒रस्ता॒त् तस्मा॒त् तस्मा᳚त् पु॒रस्ता॒च् चक्षु॑षा॒ चक्षु॑षा पु॒रस्ता॒त् तस्मा॒त् तस्मा᳚त् पु॒रस्ता॒च् चक्षु॑षा । \newline
45. पु॒रस्ता॒च् चक्षु॑षा॒ चक्षु॑षा पु॒रस्ता᳚त् पु॒रस्ता॒च् चक्षु॑षा पश्यति पश्यति॒ चक्षु॑षा पु॒रस्ता᳚त् पु॒रस्ता॒च् चक्षु॑षा पश्यति । \newline
46. चक्षु॑षा पश्यति पश्यति॒ चक्षु॑षा॒ चक्षु॑षा पश्यति स॒र्वतः॑ स॒र्वतः॑ पश्यति॒ चक्षु॑षा॒ चक्षु॑षा पश्यति स॒र्वतः॑ । \newline
47. प॒श्य॒ति॒ स॒र्वतः॑ स॒र्वतः॑ पश्यति पश्यति स॒र्वतः॑ परि॒हार॑म् परि॒हारꣳ॑ स॒र्वतः॑ पश्यति पश्यति स॒र्वतः॑ परि॒हार᳚म् । \newline
48. स॒र्वतः॑ परि॒हार॑म् परि॒हारꣳ॑ स॒र्वतः॑ स॒र्वतः॑ परि॒हार॑ माश्वि॒न मा᳚श्वि॒नम् प॑रि॒हारꣳ॑ स॒र्वतः॑ स॒र्वतः॑ परि॒हार॑ माश्वि॒नम् । \newline
49. प॒रि॒हार॑ माश्वि॒न मा᳚श्वि॒नम् प॑रि॒हार॑म् परि॒हार॑ माश्वि॒नम् तस्मा॒त् तस्मा॑ दाश्वि॒नम् प॑रि॒हार॑म् परि॒हार॑ माश्वि॒नम् तस्मा᳚त् । \newline
50. प॒रि॒हार॒मिति॑ परि - हार᳚म् । \newline
51. आ॒श्वि॒नम् तस्मा॒त् तस्मा॑ दाश्वि॒न मा᳚श्वि॒नम् तस्मा᳚थ् स॒र्वतः॑ स॒र्वत॒ स्तस्मा॑ दाश्वि॒न मा᳚श्वि॒नम् तस्मा᳚थ् स॒र्वतः॑ । \newline
52. तस्मा᳚थ् स॒र्वतः॑ स॒र्वत॒ स्तस्मा॒त् तस्मा᳚थ् स॒र्वतः॒ श्रोत्रे॑ण॒ श्रोत्रे॑ण स॒र्वत॒ स्तस्मा॒त् तस्मा᳚थ् स॒र्वतः॒ श्रोत्रे॑ण । \newline
53. स॒र्वतः॒ श्रोत्रे॑ण॒ श्रोत्रे॑ण स॒र्वतः॑ स॒र्वतः॒ श्रोत्रे॑ण शृणोति शृणोति॒ श्रोत्रे॑ण स॒र्वतः॑ स॒र्वतः॒ श्रोत्रे॑ण शृणोति । \newline
54. श्रोत्रे॑ण शृणोति शृणोति॒ श्रोत्रे॑ण॒ श्रोत्रे॑ण शृणोति प्रा॒णाः प्रा॒णाः शृ॑णोति॒ श्रोत्रे॑ण॒ श्रोत्रे॑ण शृणोति प्रा॒णाः । \newline
55. शृ॒णो॒ति॒ प्रा॒णाः प्रा॒णाः शृ॑णोति शृणोति प्रा॒णा वै वै प्रा॒णाः शृ॑णोति शृणोति प्रा॒णा वै । \newline
56. प्रा॒णा वै वै प्रा॒णाः प्रा॒णा वा ए॒त ए॒ते वै प्रा॒णाः प्रा॒णा वा ए॒ते । \newline
57. प्रा॒णा इति॑ प्र - अ॒नाः । \newline
58. वा ए॒त ए॒ते वै वा ए॒ते यद् यदे॒ते वै वा ए॒ते यत् । \newline
59. ए॒ते यद् यदे॒त ए॒ते यद् द्वि॑देव॒त्या᳚ द्विदेव॒त्या॑ यदे॒त ए॒ते यद् द्वि॑देव॒त्याः᳚ । \newline
60. यद् द्वि॑देव॒त्या᳚ द्विदेव॒त्या॑ यद् यद् द्वि॑देव॒त्या॑ अरि॑क्ता॒ न्यरि॑क्तानि द्विदेव॒त्या॑ यद् यद् द्वि॑देव॒त्या॑ अरि॑क्तानि । \newline
61. द्वि॒दे॒व॒त्या॑ अरि॑क्ता॒ न्यरि॑क्तानि द्विदेव॒त्या᳚ द्विदेव॒त्या॑ अरि॑क्तानि॒ पात्रा॑णि॒ पात्रा॒ ण्यरि॑क्तानि द्विदेव॒त्या᳚ द्विदेव॒त्या॑ अरि॑क्तानि॒ पात्रा॑णि । \newline
62. द्वि॒दे॒व॒त्या॑ इति॑ द्वि - दे॒व॒त्याः᳚ । \newline
\pagebreak
\markright{ TS 6.4.9.5  \hfill https://www.vedavms.in \hfill}

\section{ TS 6.4.9.5 }

\textbf{TS 6.4.9.5 } \newline
\textbf{Samhita Paata} \newline

अरि॑क्तानि॒ पात्रा॑णि सादयति॒ तस्मा॒दरि॑क्ता अन्तर॒तः प्रा॒णा यतः॒ खलु॒ वै य॒ज्ञ्स्य॒ वित॑तस्य॒ न क्रि॒यते॒ तदनु॑ य॒ज्ञ्ꣳ रक्षाꣳ॒॒स्यव॑ चरन्ति॒ यदरि॑क्तानि॒ पात्रा॑णि सा॒दय॑ति क्रि॒यमा॑णमे॒व तद् य॒ज्ञ्स्य॑ शये॒ रक्ष॑सा॒ -मन॑न्ववचाराय॒ दक्षि॑णस्य हवि॒र्द्धान॒स्योत्त॑रस्यां ॅवर्त॒न्याꣳ सा॑दयति वा॒च्ये॑व वाचं॑ दधा॒त्या तृ॑तीयसव॒नात् परि॑ शेरे य॒ज्ञ्स्य॒ सन्त॑त्यै ॥ \newline

\textbf{Pada Paata} \newline

अरि॑क्तानि । पात्रा॑णि । सा॒द॒य॒ति॒ । तस्मा᳚त् । अरि॑क्ताः । अ॒न्त॒र॒तः । प्रा॒णा इति॑ प्र - अ॒नाः । यतः॑ । खलु॑ । वै । य॒ज्ञ्स्य॑ । वित॑त॒स्येति॒ वि - त॒त॒स्य॒ । न । क्रि॒यते᳚ । तत् । अन्विति॑ । य॒ज्ञ्म् । रक्षाꣳ॑सि । अवेति॑ । च॒र॒न्ति॒ । यत् । अरि॑क्तानि । पात्रा॑णि । सा॒दय॑ति । क्रि॒यमा॑णम् । ए॒व । तत् । य॒ज्ञ्स्य॑ । श॒ये॒ । रक्ष॑साम् । अन॑न्ववचारा॒येत्यन॑नु - अ॒व॒चा॒रा॒य॒ । दक्षि॑णस्य । ह॒वि॒द्‌र्धान॒स्येति॑ हविः - धान॑स्य । उत्त॑रस्या॒मित्युत् - त॒र॒स्या॒म् । व॒र्त॒न्याम् । सा॒द॒य॒ति॒ । वा॒चि । ए॒व । वाच᳚म् । द॒धा॒ति॒ । एति॑ । तृ॒ती॒य॒स॒व॒नादिति॑ तृतीय - स॒व॒नात् । परीति॑ । शे॒रे॒ । य॒ज्ञ्स्य॑ । सन्त॑त्या॒ इति॒ सं - त॒त्यै॒ ॥  \newline


\textbf{Krama Paata} \newline

अरि॑क्तानि॒ पात्रा॑णि । पात्रा॑णि सादयति । सा॒द॒य॒ति॒ तस्मा᳚त् । तस्मा॒दरि॑क्ताः । अरि॑क्ता अन्तर॒तः । अ॒न्त॒र॒तः प्रा॒णाः । प्रा॒ण यतः॑ । प्रा॒णा इति॑ प्र - अ॒नाः । यतः॒ खलु॑ । खलु॒ वै । वै य॒ज्ञ्स्य॑ । य॒ज्ञ्स्य॒ वित॑तस्य । वित॑तस्य॒ न । वित॑त॒स्येति॒ वि - त॒त॒स्य॒ । न क्रि॒यते᳚ । क्रि॒यते॒ तत् । तदनु॑ । अनु॑ य॒ज्ञ्म् । य॒ज्ञ्ꣳ रक्षाꣳ॑सि । रक्षाꣳ॒॒स्यव॑ । अव॑ चरन्ति । च॒र॒न्ति॒ यत् । यदरि॑क्तानि । अरि॑क्तानि॒ पात्रा॑णि । पात्रा॑णि सा॒दय॑ति । सा॒दय॑ति क्रि॒यमा॑णम् । क्रि॒यमा॑णमे॒व । ए॒व तत् । तद् य॒ज्ञ्स्य॑ । य॒ज्ञ्स्य॑ शये । श॒ये॒ रक्ष॑साम् । रक्ष॑सा॒मन॑न्ववचाराय । अन॑न्ववचाराय॒ दक्षि॑णस्य । अन॑न्ववचारा॒येत्यन॑नु - अ॒व॒चा॒रा॒य॒ । दक्षि॑णस्य हवि॒र्द्धान॑स्य । ह॒वि॒र्द्धान॒स्योत्त॑रस्याम् । ह॒वि॒र्द्धान॒स्येति॑ हविः - धान॑स्य । उत्त॑रस्याम् ॅवर्त॒न्याम् । उत्त॑रस्या॒मित्युत् - त॒र॒स्या॒म् । व॒र्त॒न्याꣳ सा॑दयति । सा॒द॒य॒ति॒ वा॒चि । वा॒च्ये॑व । ए॒व वाच᳚म् । वाच॑म् दधाति । द॒धा॒त्या । आ तृ॑तीयसव॒नात् । तृ॒ती॒य॒स॒व॒नात् परि॑ । तृ॒ती॒य॒स॒व॒नादिति॑ तृतीय - स॒व॒नात् । परि॑ शेरे । शे॒रे॒ य॒ज्ञ्स्य॑ । य॒ज्ञ्स्य॒ सन्त॑त्यै । सन्त॑त्या॒ इति॒ सम् - त॒त्यै॒ । \newline

\textbf{Jatai Paata} \newline

1. अरि॑क्तानि॒ पात्रा॑णि॒ पात्रा॒ ण्यरि॑क्ता॒ न्यरि॑क्तानि॒ पात्रा॑णि । \newline
2. पात्रा॑णि सादयति सादयति॒ पात्रा॑णि॒ पात्रा॑णि सादयति । \newline
3. सा॒द॒य॒ति॒ तस्मा॒त् तस्मा᳚थ् सादयति सादयति॒ तस्मा᳚त् । \newline
4. तस्मा॒ दरि॑क्ता॒ अरि॑क्ता॒ स्तस्मा॒त् तस्मा॒ दरि॑क्ताः । \newline
5. अरि॑क्ता अन्तर॒तो᳚ ऽन्तर॒तो ऽरि॑क्ता॒ अरि॑क्ता अन्तर॒तः । \newline
6. अ॒न्त॒र॒तः प्रा॒णाः प्रा॒णा अ॑न्तर॒तो᳚ ऽन्तर॒तः प्रा॒णाः । \newline
7. प्रा॒णा यतो॒ यतः॑ प्रा॒णाः प्रा॒णा यतः॑ । \newline
8. प्रा॒णा इति॑ प्र - अ॒नाः । \newline
9. यतः॒ खलु॒ खलु॒ यतो॒ यतः॒ खलु॑ । \newline
10. खलु॒ वै वै खलु॒ खलु॒ वै । \newline
11. वै य॒ज्ञ्स्य॑ य॒ज्ञ्स्य॒ वै वै य॒ज्ञ्स्य॑ । \newline
12. य॒ज्ञ्स्य॒ वित॑तस्य॒ वित॑तस्य य॒ज्ञ्स्य॑ य॒ज्ञ्स्य॒ वित॑तस्य । \newline
13. वित॑तस्य॒ न न वित॑तस्य॒ वित॑तस्य॒ न । \newline
14. वित॑त॒स्येति॒ वि - त॒त॒स्य॒ । \newline
15. न क्रि॒यते᳚ क्रि॒यते॒ न न क्रि॒यते᳚ । \newline
16. क्रि॒यते॒ तत् तत् क्रि॒यते᳚ क्रि॒यते॒ तत् । \newline
17. तदन् वनु॒ तत् तदनु॑ । \newline
18. अनु॑ य॒ज्ञ्ं ॅय॒ज्ञ् मन् वनु॑ य॒ज्ञ्म् । \newline
19. य॒ज्ञ्ꣳ रक्षाꣳ॑सि॒ रक्षाꣳ॑सि य॒ज्ञ्ं ॅय॒ज्ञ्ꣳ रक्षाꣳ॑सि । \newline
20. रक्षाꣳ॒॒ स्यवाव॒ रक्षाꣳ॑सि॒ रक्षाꣳ॒॒ स्यव॑ । \newline
21. अव॑ चरन्ति चर॒ न्त्यवाव॑ चरन्ति । \newline
22. च॒र॒न्ति॒ यद् यच् च॑रन्ति चरन्ति॒ यत् । \newline
23. यदरि॑क्ता॒ न्यरि॑क्तानि॒ यद् यदरि॑क्तानि । \newline
24. अरि॑क्तानि॒ पात्रा॑णि॒ पात्रा॒ ण्यरि॑क्ता॒ न्यरि॑क्तानि॒ पात्रा॑णि । \newline
25. पात्रा॑णि सा॒दय॑ति सा॒दय॑ति॒ पात्रा॑णि॒ पात्रा॑णि सा॒दय॑ति । \newline
26. सा॒दय॑ति क्रि॒यमा॑णम् क्रि॒यमा॑णꣳ सा॒दय॑ति सा॒दय॑ति क्रि॒यमा॑णम् । \newline
27. क्रि॒यमा॑ण मे॒वैव क्रि॒यमा॑णम् क्रि॒यमा॑ण मे॒व । \newline
28. ए॒व तत् तदे॒ वैव तत् । \newline
29. तद् य॒ज्ञ्स्य॑ य॒ज्ञ्स्य॒ तत् तद् य॒ज्ञ्स्य॑ । \newline
30. य॒ज्ञ्स्य॑ शये शये य॒ज्ञ्स्य॑ य॒ज्ञ्स्य॑ शये । \newline
31. श॒ये॒ रक्ष॑साꣳ॒॒ रक्ष॑साꣳ शये शये॒ रक्ष॑साम् । \newline
32. रक्ष॑सा॒ मन॑न्ववचारा॒या न॑न्ववचाराय॒ रक्ष॑साꣳ॒॒ रक्ष॑सा॒ मन॑न्ववचाराय । \newline
33. अन॑न्ववचाराय॒ दक्षि॑णस्य॒ दक्षि॑ण॒स्या न॑न्ववचारा॒या न॑न्ववचाराय॒ दक्षि॑णस्य । \newline
34. अन॑न्ववचारा॒येत्यन॑नु - अ॒व॒चा॒रा॒य॒ । \newline
35. दक्षि॑णस्य हवि॒र्द्धान॑स्य हवि॒र्द्धान॑स्य॒ दक्षि॑णस्य॒ दक्षि॑णस्य हवि॒र्द्धान॑स्य । \newline
36. ह॒वि॒र्द्धान॒स्यो त्त॑रस्या॒ मुत्त॑रस्याꣳ हवि॒र्द्धान॑स्य हवि॒र्द्धान॒स्यो त्त॑रस्याम् । \newline
37. ह॒वि॒र्द्धान॒स्येति॑ हविः - धान॑स्य । \newline
38. उत्त॑रस्यां ॅवर्त॒न्यां ॅव॑र्त॒न्या मुत्त॑रस्या॒ मुत्त॑रस्यां ॅवर्त॒न्याम् । \newline
39. उत्त॑रस्या॒मित्युत् - त॒र॒स्या॒म् । \newline
40. व॒र्त॒न्याꣳ सा॑दयति सादयति वर्त॒न्यां ॅव॑र्त॒न्याꣳ सा॑दयति । \newline
41. सा॒द॒य॒ति॒ वा॒चि वा॒चि सा॑दयति सादयति वा॒चि । \newline
42. वा॒च्ये॑वैव वा॒चि वा॒च्ये॑व । \newline
43. ए॒व वाचं॒ ॅवाच॑ मे॒वैव वाच᳚म् । \newline
44. वाच॑म् दधाति दधाति॒ वाचं॒ ॅवाच॑म् दधाति । \newline
45. द॒धा॒त्या द॑धाति दधा॒त्या । \newline
46. आ तृ॑तीयसव॒नात् तृ॑तीयसव॒नादा तृ॑तीयसव॒नात् । \newline
47. तृ॒ती॒य॒स॒व॒नात् परि॒ परि॑ तृतीयसव॒नात् तृ॑तीयसव॒नात् परि॑ । \newline
48. तृ॒ती॒य॒स॒व॒नादिति॑ तृतीय - स॒व॒नात् । \newline
49. परि॑ शेरे शेरे॒ परि॒ परि॑ शेरे । \newline
50. शे॒रे॒ य॒ज्ञ्स्य॑ य॒ज्ञ्स्य॑ शेरे शेरे य॒ज्ञ्स्य॑ । \newline
51. य॒ज्ञ्स्य॒ सन्त॑त्यै॒ सन्त॑त्यै य॒ज्ञ्स्य॑ य॒ज्ञ्स्य॒ सन्त॑त्यै । \newline
52. सन्त॑त्या॒ इति॒ सं - त॒त्यै॒ । \newline

\textbf{Ghana Paata } \newline

1. अरि॑क्तानि॒ पात्रा॑णि॒ पात्रा॒ ण्यरि॑क्ता॒ न्यरि॑क्तानि॒ पात्रा॑णि सादयति सादयति॒ पात्रा॒ ण्यरि॑क्ता॒ न्यरि॑क्तानि॒ पात्रा॑णि सादयति । \newline
2. पात्रा॑णि सादयति सादयति॒ पात्रा॑णि॒ पात्रा॑णि सादयति॒ तस्मा॒त् तस्मा᳚थ् सादयति॒ पात्रा॑णि॒ पात्रा॑णि सादयति॒ तस्मा᳚त् । \newline
3. सा॒द॒य॒ति॒ तस्मा॒त् तस्मा᳚थ् सादयति सादयति॒ तस्मा॒ दरि॑क्ता॒ अरि॑क्ता॒ स्तस्मा᳚थ् सादयति सादयति॒ तस्मा॒ दरि॑क्ताः । \newline
4. तस्मा॒ दरि॑क्ता॒ अरि॑क्ता॒ स्तस्मा॒त् तस्मा॒ दरि॑क्ता अन्तर॒तो᳚ ऽन्तर॒तो ऽरि॑क्ता॒ स्तस्मा॒त् तस्मा॒ दरि॑क्ता अन्तर॒तः । \newline
5. अरि॑क्ता अन्तर॒तो᳚ ऽन्तर॒तो ऽरि॑क्ता॒ अरि॑क्ता अन्तर॒तः प्रा॒णाः प्रा॒णा अ॑न्तर॒तो ऽरि॑क्ता॒ अरि॑क्ता अन्तर॒तः प्रा॒णाः । \newline
6. अ॒न्त॒र॒तः प्रा॒णाः प्रा॒णा अ॑न्तर॒तो᳚ ऽन्तर॒तः प्रा॒णा यतो॒ यतः॑ प्रा॒णा अ॑न्तर॒तो᳚ ऽन्तर॒तः प्रा॒णा यतः॑ । \newline
7. प्रा॒णा यतो॒ यतः॑ प्रा॒णाः प्रा॒णा यतः॒ खलु॒ खलु॒ यतः॑ प्रा॒णाः प्रा॒णा यतः॒ खलु॑ । \newline
8. प्रा॒णा इति॑ प्र - अ॒नाः । \newline
9. यतः॒ खलु॒ खलु॒ यतो॒ यतः॒ खलु॒ वै वै खलु॒ यतो॒ यतः॒ खलु॒ वै । \newline
10. खलु॒ वै वै खलु॒ खलु॒ वै य॒ज्ञ्स्य॑ य॒ज्ञ्स्य॒ वै खलु॒ खलु॒ वै य॒ज्ञ्स्य॑ । \newline
11. वै य॒ज्ञ्स्य॑ य॒ज्ञ्स्य॒ वै वै य॒ज्ञ्स्य॒ वित॑तस्य॒ वित॑तस्य य॒ज्ञ्स्य॒ वै वै य॒ज्ञ्स्य॒ वित॑तस्य । \newline
12. य॒ज्ञ्स्य॒ वित॑तस्य॒ वित॑तस्य य॒ज्ञ्स्य॑ य॒ज्ञ्स्य॒ वित॑तस्य॒ न न वित॑तस्य य॒ज्ञ्स्य॑ य॒ज्ञ्स्य॒ वित॑तस्य॒ न । \newline
13. वित॑तस्य॒ न न वित॑तस्य॒ वित॑तस्य॒ न क्रि॒यते᳚ क्रि॒यते॒ न वित॑तस्य॒ वित॑तस्य॒ न क्रि॒यते᳚ । \newline
14. वित॑त॒स्येति॒ वि - त॒त॒स्य॒ । \newline
15. न क्रि॒यते᳚ क्रि॒यते॒ न न क्रि॒यते॒ तत् तत् क्रि॒यते॒ न न क्रि॒यते॒ तत् । \newline
16. क्रि॒यते॒ तत् तत् क्रि॒यते᳚ क्रि॒यते॒ तदन् वनु॒ तत् क्रि॒यते᳚ क्रि॒यते॒ तदनु॑ । \newline
17. तदन् वनु॒ तत् तदनु॑ य॒ज्ञ्ं ॅय॒ज्ञ् मनु॒ तत् तदनु॑ य॒ज्ञ्म् । \newline
18. अनु॑ य॒ज्ञ्ं ॅय॒ज्ञ् मन् वनु॑ य॒ज्ञ्ꣳ रक्षाꣳ॑सि॒ रक्षाꣳ॑सि य॒ज्ञ् मन् वनु॑ य॒ज्ञ्ꣳ रक्षाꣳ॑सि । \newline
19. य॒ज्ञ्ꣳ रक्षाꣳ॑सि॒ रक्षाꣳ॑सि य॒ज्ञ्ं ॅय॒ज्ञ्ꣳ रक्षाꣳ॒॒ स्यवाव॒ रक्षाꣳ॑सि य॒ज्ञ्ं ॅय॒ज्ञ्ꣳ रक्षाꣳ॒॒ स्यव॑ । \newline
20. रक्षाꣳ॒॒ स्यवाव॒ रक्षाꣳ॑सि॒ रक्षाꣳ॒॒ स्यव॑ चरन्ति चर॒न्त्यव॒ रक्षाꣳ॑सि॒ रक्षाꣳ॒॒ स्यव॑ चरन्ति । \newline
21. अव॑ चरन्ति चर॒ न्त्यवाव॑ चरन्ति॒ यद् यच् च॑र॒ न्त्यवाव॑ चरन्ति॒ यत् । \newline
22. च॒र॒न्ति॒ यद् यच् च॑रन्ति चरन्ति॒ यदरि॑क्ता॒ न्यरि॑क्तानि॒ यच् च॑रन्ति चरन्ति॒ यदरि॑क्तानि । \newline
23. यदरि॑क्ता॒ न्यरि॑क्तानि॒ यद् यदरि॑क्तानि॒ पात्रा॑णि॒ पात्रा॒ ण्यरि॑क्तानि॒ यद् यदरि॑क्तानि॒ पात्रा॑णि । \newline
24. अरि॑क्तानि॒ पात्रा॑णि॒ पात्रा॒ ण्यरि॑क्ता॒ न्यरि॑क्तानि॒ पात्रा॑णि सा॒दय॑ति सा॒दय॑ति॒ पात्रा॒ ण्यरि॑क्ता॒ न्यरि॑क्तानि॒ पात्रा॑णि सा॒दय॑ति । \newline
25. पात्रा॑णि सा॒दय॑ति सा॒दय॑ति॒ पात्रा॑णि॒ पात्रा॑णि सा॒दय॑ति क्रि॒यमा॑णम् क्रि॒यमा॑णꣳ सा॒दय॑ति॒ पात्रा॑णि॒ पात्रा॑णि सा॒दय॑ति क्रि॒यमा॑णम् । \newline
26. सा॒दय॑ति क्रि॒यमा॑णम् क्रि॒यमा॑णꣳ सा॒दय॑ति सा॒दय॑ति क्रि॒यमा॑ण मे॒वैव क्रि॒यमा॑णꣳ सा॒दय॑ति सा॒दय॑ति क्रि॒यमा॑ण मे॒व । \newline
27. क्रि॒यमा॑ण मे॒वैव क्रि॒यमा॑णम् क्रि॒यमा॑ण मे॒व तत् तदे॒व क्रि॒यमा॑णम् क्रि॒यमा॑ण मे॒व तत् । \newline
28. ए॒व तत् तदे॒ वैव तद् य॒ज्ञ्स्य॑ य॒ज्ञ्स्य॒ तदे॒ वैव तद् य॒ज्ञ्स्य॑ । \newline
29. तद् य॒ज्ञ्स्य॑ य॒ज्ञ्स्य॒ तत् तद् य॒ज्ञ्स्य॑ शये शये य॒ज्ञ्स्य॒ तत् तद् य॒ज्ञ्स्य॑ शये । \newline
30. य॒ज्ञ्स्य॑ शये शये य॒ज्ञ्स्य॑ य॒ज्ञ्स्य॑ शये॒ रक्ष॑साꣳ॒॒ रक्ष॑साꣳ शये य॒ज्ञ्स्य॑ य॒ज्ञ्स्य॑ शये॒ रक्ष॑साम् । \newline
31. श॒ये॒ रक्ष॑साꣳ॒॒ रक्ष॑साꣳ शये शये॒ रक्ष॑सा॒ मन॑न्ववचारा॒या न॑न्ववचाराय॒ रक्ष॑साꣳ शये शये॒ रक्ष॑सा॒ मन॑न्ववचाराय । \newline
32. रक्ष॑सा॒ मन॑न्ववचारा॒या न॑न्ववचाराय॒ रक्ष॑साꣳ॒॒ रक्ष॑सा॒ मन॑न्ववचाराय॒ दक्षि॑णस्य॒ दक्षि॑ण॒स्या न॑न्ववचाराय॒ रक्ष॑साꣳ॒॒ रक्ष॑सा॒ मन॑न्ववचाराय॒ दक्षि॑णस्य । \newline
33. अन॑न्ववचाराय॒ दक्षि॑णस्य॒ दक्षि॑ण॒स्या न॑न्ववचारा॒या न॑न्ववचाराय॒ दक्षि॑णस्य हवि॒र्द्धान॑स्य हवि॒र्द्धान॑स्य॒ दक्षि॑ण॒स्या न॑न्ववचारा॒या न॑न्ववचाराय॒ दक्षि॑णस्य हवि॒र्द्धान॑स्य । \newline
34. अन॑न्ववचारा॒येत्यन॑नु - अ॒व॒चा॒रा॒य॒ । \newline
35. दक्षि॑णस्य हवि॒र्द्धान॑स्य हवि॒र्द्धान॑स्य॒ दक्षि॑णस्य॒ दक्षि॑णस्य हवि॒र्द्धान॒ स्योत्त॑रस्या॒ मुत्त॑रस्याꣳ हवि॒र्द्धान॑स्य॒ दक्षि॑णस्य॒ दक्षि॑णस्य हवि॒र्द्धान॒ स्योत्त॑रस्याम् । \newline
36. ह॒वि॒र्द्धान॒ स्योत्त॑रस्या॒ मुत्त॑रस्याꣳ हवि॒र्द्धान॑स्य हवि॒र्द्धान॒ स्योत्त॑रस्यां ॅवर्त॒न्यां ॅव॑र्त॒न्या मुत्त॑रस्याꣳ हवि॒र्द्धान॑स्य हवि॒र्द्धान॒ स्योत्त॑रस्यां ॅवर्त॒न्याम् । \newline
37. ह॒वि॒र्द्धान॒स्येति॑ हविः - धान॑स्य । \newline
38. उत्त॑रस्यां ॅवर्त॒न्यां ॅव॑र्त॒न्या मुत्त॑रस्या॒ मुत्त॑रस्यां ॅवर्त॒न्याꣳ सा॑दयति सादयति वर्त॒न्या मुत्त॑रस्या॒ मुत्त॑रस्यां ॅवर्त॒न्याꣳ सा॑दयति । \newline
39. उत्त॑रस्या॒मित्युत् - त॒र॒स्या॒म् । \newline
40. व॒र्त॒न्याꣳ सा॑दयति सादयति वर्त॒न्यां ॅव॑र्त॒न्याꣳ सा॑दयति वा॒चि वा॒चि सा॑दयति वर्त॒न्यां ॅव॑र्त॒न्याꣳ सा॑दयति वा॒चि । \newline
41. सा॒द॒य॒ति॒ वा॒चि वा॒चि सा॑दयति सादयति वा॒च्ये॑ वैव वा॒चि सा॑दयति सादयति वा॒च्ये॑व । \newline
42. वा॒च्ये॑ वैव वा॒चि वा॒च्ये॑व वाचं॒ ॅवाच॑ मे॒व वा॒चि वा॒च्ये॑व वाच᳚म् । \newline
43. ए॒व वाचं॒ ॅवाच॑ मे॒वैव वाच॑म् दधाति दधाति॒ वाच॑ मे॒वैव वाच॑म् दधाति । \newline
44. वाच॑म् दधाति दधाति॒ वाचं॒ ॅवाच॑म् दधा॒त्या द॑धाति॒ वाचं॒ ॅवाच॑म् दधा॒त्या । \newline
45. द॒धा॒त्या द॑धाति दधा॒त्या तृ॑तीयसव॒नात् तृ॑तीयसव॒नादा द॑धाति दधा॒त्या तृ॑तीयसव॒नात् । \newline
46. आ तृ॑तीयसव॒नात् तृ॑तीयसव॒नादा तृ॑तीयसव॒नात् परि॒ परि॑ तृतीयसव॒नादा तृ॑तीयसव॒नात् परि॑ । \newline
47. तृ॒ती॒य॒स॒व॒नात् परि॒ परि॑ तृतीयसव॒नात् तृ॑तीयसव॒नात् परि॑ शेरे शेरे॒ परि॑ तृतीयसव॒नात् तृ॑तीयसव॒नात् परि॑ शेरे । \newline
48. तृ॒ती॒य॒स॒व॒नादिति॑ तृतीय - स॒व॒नात् । \newline
49. परि॑ शेरे शेरे॒ परि॒ परि॑ शेरे य॒ज्ञ्स्य॑ य॒ज्ञ्स्य॑ शेरे॒ परि॒ परि॑ शेरे य॒ज्ञ्स्य॑ । \newline
50. शे॒रे॒ य॒ज्ञ्स्य॑ य॒ज्ञ्स्य॑ शेरे शेरे य॒ज्ञ्स्य॒ सन्त॑त्यै॒ सन्त॑त्यै य॒ज्ञ्स्य॑ शेरे शेरे य॒ज्ञ्स्य॒ सन्त॑त्यै । \newline
51. य॒ज्ञ्स्य॒ सन्त॑त्यै॒ सन्त॑त्यै य॒ज्ञ्स्य॑ य॒ज्ञ्स्य॒ सन्त॑त्यै । \newline
52. सन्त॑त्या॒ इति॒ सं - त॒त्यै॒ । \newline
\pagebreak
\markright{ TS 6.4.10.1  \hfill https://www.vedavms.in \hfill}

\section{ TS 6.4.10.1 }

\textbf{TS 6.4.10.1 } \newline
\textbf{Samhita Paata} \newline

बृह॒स्पति॑र्दे॒वानां᳚ पु॒रोहि॑त॒ आसी॒-च्छण्डा॒मर्का॒-वसु॑राणां॒ ब्रह्म॑ण् वन्तो दे॒वा आस॒न् ब्रह्म॑ण् व॒न्तोऽसु॑रा॒स्ते᳚(1॒) ऽन्यो᳚ऽन्यं नाश॑क्नुव-न्न॒भिभ॑वितुं॒ ते दे॒वाः शण्डा॒मर्का॒-वुपा॑मन्त्रयन्त॒ ता व॑ब्रूतां॒ ॅवरं॑ ॅवृणावहै॒ ग्रहा॑वे॒व ना॒वत्रापि॑ गृह्येता॒मिति॒ ताभ्या॑मे॒तौ शु॒क्राम॒न्थिना॑-वगृह्ण॒न् ततो॑ दे॒वा अभ॑व॒न् पराऽसु॑रा॒ यस्यै॒वं ॅवि॒दुषः॑ शु॒क्राम॒न्थिनौ॑ गृ॒ह्येते॒ भ॑वत्या॒त्मना॒ परा᳚- [  ] \newline

\textbf{Pada Paata} \newline

बृह॒स्पतिः॑ । दे॒वाना᳚म् । पु॒रोहि॑त॒ इति॑ पु॒रः - हि॒तः॒ । आसी᳚त् । शण्डा॒मर्का॒विति॒ शण्डा᳚ - मर्कौ᳚ । असु॑राणाम् । ब्रह्म॑ण्वन्त॒ इति॒ ब्रह्मण्ण्॑ - व॒न्तः॒ । दे॒वाः । आसन्न्॑ । ब्रह्म॑ण्वन्त॒ इति॒ ब्रह्मण्ण्॑ - व॒न्तः॒ । असु॑राः । ते । अ॒न्यः । अ॒न्यम् । न । अ॒श॒क्नु॒व॒न्न् । अ॒भिभ॑वितु॒मित्य॒भि - भ॒वि॒तु॒म् । ते । दे॒वाः । शण्डा॒मर्का॒विति॒ शण्डा᳚ - मर्कौ᳚ । उपेति॑ । अ॒म॒न्त्र॒य॒न्त॒ । तौ । अ॒ब्रू॒ता॒म् । वर᳚म् । वृ॒णा॒व॒है॒ । ग्रहौ᳚ । ए॒व । नौ॒ । अत्र॑ । अपीति॑ । गृ॒ह्ये॒ता॒म् । इति॑ । ताभ्या᳚म् । ए॒तौ । शु॒क्राम॒न्थिना॒विति॑ शु॒क्रा - म॒न्थिनौ᳚ । अ॒गृ॒ह्ण॒न्न् । ततः॑ । दे॒वाः । अभ॑वन्न् । परेति॑ । असु॑राः । यस्य॑ । ए॒वम् । वि॒दुषः॑ । शु॒क्राम॒न्थिना॒विति॑ शु॒क्रा - म॒न्थिनौ᳚ । गृ॒ह्येते॒ इति॑ । भव॑ति । आ॒त्मना᳚ । परेति॑ ।  \newline


\textbf{Krama Paata} \newline

बृह॒स्पति॑र् दे॒वाना᳚म् । दे॒वाना᳚म् पु॒रोहि॑तः । पु॒रोहि॑त॒ आसी᳚त् । पु॒रोहि॑त॒ इति॑ पु॒रः - हि॒तः॒ । आसी॒च्छण्डा॒मर्कौ᳚ । शण्डा॒मर्का॒वसु॑राणाम् । शण्डा॒मर्का॒विति॒ शण्डा᳚ - मर्कौ᳚ । असु॑राणा॒म् ब्रह्म॑ण्वन्तः । ब्रह्म॑ण्वन्तो दे॒वाः । ब्रह्म॑ण्वन्त॒ इति॒ ब्रह्मण्ण्॑ - व॒न्तः॒ । दे॒वा आसन्न्॑ । आस॒न् ब्रह्म॑ण्वन्तः । ब्रह्म॑ण्व॒न्तोऽसु॑राः । ब्रह्म॑ण्वन्त॒ इति॒ ब्रह्मण्ण्॑ - व॒न्तः॒ । असु॑रा॒स्ते । ते᳚ऽन्यः । अ॒न्यो᳚ऽन्यम् । अ॒न्यम् न । नाश॑क्नुवन्न् । अ॒श॒क्नु॒व॒न्न॒भिभ॑वितुम् । अ॒भिभ॑वितु॒म् ते । अ॒भिभ॑वितु॒मित्य॒भि - भ॒वि॒तु॒म् । ते दे॒वाः । दे॒वाः शण्डा॒मर्कौ᳚ । शण्डा॒मर्का॒वुप॑ । शण्डा॒मर्का॒विति॒ शण्डा᳚ - मर्कौ᳚ । उपा॑मन्त्रयन्त । अ॒म॒न्त्र॒य॒न्त॒ तौ । ताव॑ब्रूताम् । अ॒ब्रू॒ता॒म् ॅवर᳚म् । वर॑म् ॅवृणावहै । वृ॒णा॒व॒है॒ ग्रहौ᳚ । ग्रहा॑वे॒व । ए॒व नौ । ना॒वत्र॑ । अत्रापि॑ । अपि॑ गृह्येताम् । गृ॒ह्ये॒ता॒मिति॑ । इति॒ ताभ्या᳚म् । ताभ्या॑मे॒तौ । ए॒तौ शु॒क्राम॒न्थिनौ᳚ । शु॒क्राम॒न्थिना॑वगृह्णन्न् । शु॒क्राम॒न्थिना॒विति॑ शु॒क्रा - म॒न्थिनौ᳚ । अ॒गृ॒ह्णन् ततः॑ । ततो॑ दे॒वाः । दे॒वा अभ॑वन्न् । अभ॑व॒न् परा᳚ । पराऽसु॑राः । असु॑रा॒ यस्य॑ । यस्यै॒वम् । ए॒वम् ॅवि॒दुषः॑ । वि॒दुषः॑ शु॒क्राम॒न्थिनौ᳚ । शु॒क्राम॒न्थिनौ॑ गृ॒ह्येते᳚ । शु॒क्राम॒न्थिना॒विति॑ शु॒क्रा - म॒न्थिनौ᳚ । गृ॒ह्येते॒ भव॑ति । गृ॒ह्येते॒ इति॑ गृ॒ह्येते᳚ । भव॑त्या॒त्मना᳚ । आ॒त्मना॒ परा᳚ । परा᳚ऽस्य \newline

\textbf{Jatai Paata} \newline

1. बृह॒स्पति॑र् दे॒वाना᳚म् दे॒वाना॒म् बृह॒स्पति॒र् बृह॒स्पति॑र् दे॒वाना᳚म् । \newline
2. दे॒वाना᳚म् पु॒रोहि॑तः पु॒रोहि॑तो दे॒वाना᳚म् दे॒वाना᳚म् पु॒रोहि॑तः । \newline
3. पु॒रोहि॑त॒ आसी॒ दासी᳚त् पु॒रोहि॑तः पु॒रोहि॑त॒ आसी᳚त् । \newline
4. पु॒रोहि॑त॒ इति॑ पु॒रः - हि॒तः॒ । \newline
5. आसी॒च् छण्डा॒मर्कौ॒ शण्डा॒मर्का॒ वासी॒ दासी॒च् छण्डा॒मर्कौ᳚ । \newline
6. शण्डा॒मर्का॒ वसु॑राणा॒ मसु॑राणाꣳ॒॒ शण्डा॒मर्कौ॒ शण्डा॒मर्का॒ वसु॑राणाम् । \newline
7. शण्डा॒मर्का॒विति॒ शण्डा᳚ - मर्कौ᳚ । \newline
8. असु॑राणा॒म् ब्रह्म॑ण्वन्तो॒ ब्रह्म॑ण्व॒न्तो ऽसु॑राणा॒ मसु॑राणा॒म् ब्रह्म॑ण्वन्तः । \newline
9. ब्रह्म॑ण्वन्तो दे॒वा दे॒वा ब्रह्म॑ण्वन्तो॒ ब्रह्म॑ण्वन्तो दे॒वाः । \newline
10. ब्रह्म॑ण्वन्त॒ इति॒ ब्रह्मण्ण्॑ - व॒न्तः॒ । \newline
11. दे॒वा आस॒न् नास॑न् दे॒वा दे॒वा आसन्न्॑ । \newline
12. आस॒न् ब्रह्म॑ण्वन्तो॒ ब्रह्म॑ण्वन्त॒ आस॒न् नास॒न् ब्रह्म॑ण्वन्तः । \newline
13. ब्रह्म॑ण्व॒न्तो ऽसु॑रा॒ असु॑रा॒ ब्रह्म॑ण्वन्तो॒ ब्रह्म॑ण्व॒न्तो ऽसु॑राः । \newline
14. ब्रह्म॑ण्वन्त॒ इति॒ ब्रह्मण्ण्॑ - व॒न्तः॒ । \newline
15. असु॑रा॒ स्ते ते ऽसु॑रा॒ असु॑रा॒ स्ते । \newline
16. ते᳚(1॒) ऽन्यो᳚ ऽन्यस्ते ते᳚ ऽन्यः । \newline
17. अ॒न्यो᳚ ऽन्य म॒न्य म॒न्यो᳚(1॒) ऽन्यो᳚ ऽन्यम् । \newline
18. अ॒न्यन् न नान्य म॒न्यन् न । \newline
19. नाश॑क्नुवन् नशक्नुव॒न् न नाश॑क्नुवन्न् । \newline
20. अ॒श॒क्नु॒व॒न् न॒भिभ॑वितु म॒भिभ॑वितु मशक्नुवन् नशक्नुवन् न॒भिभ॑वितुम् । \newline
21. अ॒भिभ॑वितु॒म् ते ते॑ ऽभिभ॑वितु म॒भिभ॑वितु॒म् ते । \newline
22. अ॒भिभ॑वितु॒मित्य॒भि - भ॒वि॒तु॒म् । \newline
23. ते दे॒वा दे॒वा स्ते ते दे॒वाः । \newline
24. दे॒वाः शण्डा॒मर्कौ॒ शण्डा॒मर्कौ॑ दे॒वा दे॒वाः शण्डा॒मर्कौ᳚ । \newline
25. शण्डा॒मर्का॒ वुपोप॒ शण्डा॒मर्कौ॒ शण्डा॒मर्का॒ वुप॑ । \newline
26. शण्डा॒मर्का॒विति॒ शण्डा᳚ - मर्कौ᳚ । \newline
27. उपा॑ मन्त्रयन्त् आमन्त्रय॒ न्तोपोपा॑ मन्त्रयन्त । \newline
28. अ॒म॒न्त्र॒य॒न्त॒ तौ ता व॑मन्त्रयन्ता मन्त्रयन्त॒ तौ । \newline
29. ता व॑ब्रूता मब्रूता॒म् तौ ता व॑ब्रूताम् । \newline
30. अ॒ब्रू॒तां॒ ॅवरं॒ ॅवर॑ मब्रूता मब्रूतां॒ ॅवर᳚म् । \newline
31. वरं॑ ॅवृणावहै वृणावहै॒ वरं॒ ॅवरं॑ ॅवृणावहै । \newline
32. वृ॒णा॒व॒है॒ ग्रहौ॒ ग्रहौ॑ वृणावहै वृणावहै॒ ग्रहौ᳚ । \newline
33. ग्रहा॑ वे॒वैव ग्रहौ॒ ग्रहा॑ वे॒व । \newline
34. ए॒व नौ॑ ना वे॒वैव नौ᳚ । \newline
35. ना॒ वत्रात्र॑ नौ ना॒ वत्र॑ । \newline
36. अत्रा प्यप्य त्रा त्रापि॑ । \newline
37. अपि॑ गृह्येताम् गृह्येता॒ मप्यपि॑ गृह्येताम् । \newline
38. गृ॒ह्ये॒ता॒ मितीति॑ गृह्येताम् गृह्येता॒ मिति॑ । \newline
39. इति॒ ताभ्या॒म् ताभ्या॒ मितीति॒ ताभ्या᳚म् । \newline
40. ताभ्या॑ मे॒ता वे॒तौ ताभ्या॒म् ताभ्या॑ मे॒तौ । \newline
41. ए॒तौ शु॒क्राम॒न्थिनौ॑ शु॒क्राम॒न्थिना॑ वे॒ता वे॒तौ शु॒क्राम॒न्थिनौ᳚ । \newline
42. शु॒क्राम॒न्थिना॑ वगृह्णन् नगृह्णञ् छु॒क्राम॒न्थिनौ॑ शु॒क्राम॒न्थिना॑ वगृह्णन्न् । \newline
43. शु॒क्राम॒न्थिना॒विति॑ शु॒क्रा - म॒न्थिनौ᳚ । \newline
44. अ॒गृ॒ह्ण॒न् तत॒ स्ततो॑ ऽगृह्णन् नगृह्ण॒न् ततः॑ । \newline
45. ततो॑ दे॒वा दे॒वा स्तत॒ स्ततो॑ दे॒वाः । \newline
46. दे॒वा अभ॑व॒न् नभ॑वन् दे॒वा दे॒वा अभ॑वन्न् । \newline
47. अभ॑व॒न् परा॒ परा ऽभ॑व॒न् नभ॑व॒न् परा᳚ । \newline
48. परा ऽसु॑रा॒ असु॑राः॒ परा॒ परा ऽसु॑राः । \newline
49. असु॑रा॒ यस्य॒ यस्या सु॑रा॒ असु॑रा॒ यस्य॑ । \newline
50. यस्यै॒व मे॒वं ॅयस्य॒ यस्यै॒वम् । \newline
51. ए॒वं ॅवि॒दुषो॑ वि॒दुष॑ ए॒व मे॒वं ॅवि॒दुषः॑ । \newline
52. वि॒दुषः॑ शु॒क्राम॒न्थिनौ॑ शु॒क्राम॒न्थिनौ॑ वि॒दुषो॑ वि॒दुषः॑ शु॒क्राम॒न्थिनौ᳚ । \newline
53. शु॒क्राम॒न्थिनौ॑ गृ॒ह्येते॑ गृ॒ह्येते॑ शु॒क्राम॒न्थिनौ॑ शु॒क्राम॒न्थिनौ॑ गृ॒ह्येते᳚ । \newline
54. शु॒क्राम॒न्थिना॒विति॑ शु॒क्रा - म॒न्थिनौ᳚ । \newline
55. गृ॒ह्येते॒ भव॑ति॒ भव॑ति गृ॒ह्येते॑ गृ॒ह्येते॒ भव॑ति । \newline
56. गृ॒ह्येते॒ इति॑ गृ॒ह्येते᳚ । \newline
57. भव॑ त्या॒त्मना॒ ऽऽत्मना॒ भव॑ति॒ भव॑ त्या॒त्मना᳚ । \newline
58. आ॒त्मना॒ परा॒ परा॒ ऽऽत्मना॒ ऽऽत्मना॒ परा᳚ । \newline
59. परा᳚ ऽस्यास्य॒ परा॒ परा᳚ ऽस्य । \newline

\textbf{Ghana Paata } \newline

1. बृह॒स्पति॑र् दे॒वाना᳚म् दे॒वाना॒म् बृह॒स्पति॒र् बृह॒स्पति॑र् दे॒वाना᳚म् पु॒रोहि॑तः पु॒रोहि॑तो दे॒वाना॒म् बृह॒स्पति॒र् बृह॒स्पति॑र् दे॒वाना᳚म् पु॒रोहि॑तः । \newline
2. दे॒वाना᳚म् पु॒रोहि॑तः पु॒रोहि॑तो दे॒वाना᳚म् दे॒वाना᳚म् पु॒रोहि॑त॒ आसी॒ दासी᳚त् पु॒रोहि॑तो दे॒वाना᳚म् दे॒वाना᳚म् पु॒रोहि॑त॒ आसी᳚त् । \newline
3. पु॒रोहि॑त॒ आसी॒ दासी᳚त् पु॒रोहि॑तः पु॒रोहि॑त॒ आसी॒च् छण्डा॒मर्कौ॒ शण्डा॒मर्का॒ वासी᳚त् पु॒रोहि॑तः पु॒रोहि॑त॒ आसी॒च् छण्डा॒मर्कौ᳚ । \newline
4. पु॒रोहि॑त॒ इति॑ पु॒रः - हि॒तः॒ । \newline
5. आसी॒च् छण्डा॒मर्कौ॒ शण्डा॒मर्का॒ वासी॒ दासी॒च् छण्डा॒मर्का॒ वसु॑राणा॒ मसु॑राणाꣳ॒॒ शण्डा॒मर्का॒ वासी॒ दासी॒च् छण्डा॒मर्का॒ वसु॑राणाम् । \newline
6. शण्डा॒मर्का॒ वसु॑राणा॒ मसु॑राणाꣳ॒॒ शण्डा॒मर्कौ॒ शण्डा॒मर्का॒ वसु॑राणा॒म् ब्रह्म॑ण्वन्तो॒ ब्रह्म॑ण्व॒न्तो ऽसु॑राणाꣳ॒॒ शण्डा॒मर्कौ॒ शण्डा॒मर्का॒ वसु॑राणा॒म् ब्रह्म॑ण्वन्तः । \newline
7. शण्डा॒मर्का॒विति॒ शण्डा᳚ - मर्कौ᳚ । \newline
8. असु॑राणा॒म् ब्रह्म॑ण्वन्तो॒ ब्रह्म॑ण्व॒न्तो ऽसु॑राणा॒ मसु॑राणा॒म् ब्रह्म॑ण्वन्तो दे॒वा दे॒वा ब्रह्म॑ण्व॒न्तो ऽसु॑राणा॒ मसु॑राणा॒म् ब्रह्म॑ण्वन्तो दे॒वाः । \newline
9. ब्रह्म॑ण्वन्तो दे॒वा दे॒वा ब्रह्म॑ण्वन्तो॒ ब्रह्म॑ण्वन्तो दे॒वा आस॒न् नास॑न् दे॒वा ब्रह्म॑ण्वन्तो॒ ब्रह्म॑ण्वन्तो दे॒वा आसन्न्॑ । \newline
10. ब्रह्म॑ण्वन्त॒ इति॒ ब्रह्मण्ण्॑ - व॒न्तः॒ । \newline
11. दे॒वा आस॒न् नास॑न् दे॒वा दे॒वा आस॒न् ब्रह्म॑ण्वन्तो॒ ब्रह्म॑ण्वन्त॒ आस॑न् दे॒वा दे॒वा आस॒न् ब्रह्म॑ण्वन्तः । \newline
12. आस॒न् ब्रह्म॑ण्वन्तो॒ ब्रह्म॑ण्वन्त॒ आस॒न् नास॒न् ब्रह्म॑ण्व॒न्तो ऽसु॑रा॒ असु॑रा॒ ब्रह्म॑ण्वन्त॒ आस॒न् नास॒न् ब्रह्म॑ण्व॒न्तो ऽसु॑राः । \newline
13. ब्रह्म॑ण्व॒न्तो ऽसु॑रा॒ असु॑रा॒ ब्रह्म॑ण्वन्तो॒ ब्रह्म॑ण्व॒न्तो ऽसु॑रा॒ स्ते ते ऽसु॑रा॒ ब्रह्म॑ण्वन्तो॒ ब्रह्म॑ण्व॒न्तो ऽसु॑रा॒ स्ते । \newline
14. ब्रह्म॑ण्वन्त॒ इति॒ ब्रह्मण्ण्॑ - व॒न्तः॒ । \newline
15. असु॑रा॒ स्ते ते ऽसु॑रा॒ असु॑रा॒ स्ते᳚(1॒) ऽन्यो᳚ ऽन्य स्ते ऽसु॑रा॒ असु॑रा॒ स्ते᳚ ऽन्यः । \newline
16. ते᳚(1॒) ऽन्यो᳚ ऽन्य स्ते ते᳚(1॒) ऽन्यो᳚ ऽन्य म॒न्य म॒न्य स्ते ते᳚(1॒) ऽन्यो᳚ ऽन्यम् । \newline
17. अ॒न्यो᳚ ऽन्य म॒न्य म॒न्यो᳚(1॒) ऽन्यो᳚ ऽन्यन् न नान्य म॒न्यो᳚(1॒) ऽन्यो᳚ ऽन्यन्न । \newline
18. अ॒न्यन् न नान्य म॒न्यन् नाश॑क्नुवन् नशक्नुव॒न् नान्य म॒न्यन् नाश॑क्नुवन्न् । \newline
19. नाश॑क्नुवन् नशक्नुव॒न् न नाश॑क्नुवन् न॒भिभ॑वितु म॒भिभ॑वितु मशक्नुव॒न् न नाश॑क्नुवन् न॒भिभ॑वितुम् । \newline
20. अ॒श॒क्नु॒व॒न् न॒भिभ॑वितु म॒भिभ॑वितु मशक्नुवन् नशक्नुवन् न॒भिभ॑वितु॒म् ते ते॑ ऽभिभ॑वितु मशक्नुवन् नशक्नुवन् न॒भिभ॑वितु॒म् ते । \newline
21. अ॒भिभ॑वितु॒म् ते ते॑ ऽभिभ॑वितु म॒भिभ॑वितु॒म् ते दे॒वा दे॒वा स्ते॑ ऽभिभ॑वितु म॒भिभ॑वितु॒म् ते दे॒वाः । \newline
22. अ॒भिभ॑वितु॒मित्य॒भि - भ॒वि॒तु॒म् । \newline
23. ते दे॒वा दे॒वा स्ते ते दे॒वाः शण्डा॒मर्कौ॒ शण्डा॒मर्कौ॑ दे॒वा स्ते ते दे॒वाः शण्डा॒मर्कौ᳚ । \newline
24. दे॒वाः शण्डा॒मर्कौ॒ शण्डा॒मर्कौ॑ दे॒वा दे॒वाः शण्डा॒मर्का॒ वुपोप॒ शण्डा॒मर्कौ॑ दे॒वा दे॒वाः शण्डा॒मर्का॒ वुप॑ । \newline
25. शण्डा॒मर्का॒ वुपोप॒ शण्डा॒मर्कौ॒ शण्डा॒मर्का॒ वुपा॑मन्त्रयन्ता मन्त्रय॒न्तोप॒ शण्डा॒मर्कौ॒ शण्डा॒मर्का॒ वुपा॑मन्त्रयन्त । \newline
26. शण्डा॒मर्का॒विति॒ शण्डा᳚ - मर्कौ᳚ । \newline
27. उपा॑मन्त्रयन्ता मन्त्रय॒न्तो पोपा॑ मन्त्रयन्त॒ तौ ता व॑मन्त्रय॒न्तो पोपा॑ मन्त्रयन्त॒ तौ । \newline
28. अ॒म॒न्त्र॒य॒न्त॒ तौ ता व॑मन्त्रयन्ता मन्त्रयन्त॒ ता व॑ब्रूता मब्रूता॒म् ता व॑मन्त्रयन्ता मन्त्रयन्त॒ ता व॑ब्रूताम् । \newline
29. ता व॑ब्रूता मब्रूता॒म् तौ ता व॑ब्रूतां॒ ॅवरं॒ ॅवर॑ मब्रूता॒म् तौ ता व॑ब्रूतां॒ ॅवर᳚म् । \newline
30. अ॒ब्रू॒तां॒ ॅवरं॒ ॅवर॑ मब्रूता मब्रूतां॒ ॅवरं॑ ॅवृणावहै वृणावहै॒ वर॑ मब्रूता मब्रूतां॒ ॅवरं॑ ॅवृणावहै । \newline
31. वरं॑ ॅवृणावहै वृणावहै॒ वरं॒ ॅवरं॑ ॅवृणावहै॒ ग्रहौ॒ ग्रहौ॑ वृणावहै॒ वरं॒ ॅवरं॑ ॅवृणावहै॒ ग्रहौ᳚ । \newline
32. वृ॒णा॒व॒है॒ ग्रहौ॒ ग्रहौ॑ वृणावहै वृणावहै॒ ग्रहा॑ वे॒वैव ग्रहौ॑ वृणावहै वृणावहै॒ ग्रहा॑ वे॒व । \newline
33. ग्रहा॑ वे॒वैव ग्रहौ॒ ग्रहा॑ वे॒व नौ॑ ना वे॒व ग्रहौ॒ ग्रहा॑ वे॒व नौ᳚ । \newline
34. ए॒व नौ॑ ना वे॒वैव ना॒ वत्रात्र॑ ना वे॒वैव ना॒ वत्र॑ । \newline
35. ना॒ वत्रात्र॑ नौ ना॒ वत्राप्य प्यत्र॑ नौ ना॒ वत्रापि॑ । \newline
36. अत्राप्य प्यत्रा त्रापि॑ गृह्येताम् गृह्येता॒ मप्यत्रा त्रापि॑ गृह्येताम् । \newline
37. अपि॑ गृह्येताम् गृह्येता॒ मप्यपि॑ गृह्येता॒ मितीति॑ गृह्येता॒ मप्यपि॑ गृह्येता॒ मिति॑ । \newline
38. गृ॒ह्ये॒ता॒ मितीति॑ गृह्येताम् गृह्येता॒ मिति॒ ताभ्या॒म् ताभ्या॒ मिति॑ गृह्येताम् गृह्येता॒ मिति॒ ताभ्या᳚म् । \newline
39. इति॒ ताभ्या॒म् ताभ्या॒ मितीति॒ ताभ्या॑ मे॒ता वे॒तौ ताभ्या॒ मितीति॒ ताभ्या॑ मे॒तौ । \newline
40. ताभ्या॑ मे॒ता वे॒तौ ताभ्या॒म् ताभ्या॑ मे॒तौ शु॒क्राम॒न्थिनौ॑ शु॒क्राम॒न्थिना॑ वे॒तौ ताभ्या॒म् ताभ्या॑ मे॒तौ शु॒क्राम॒न्थिनौ᳚ । \newline
41. ए॒तौ शु॒क्राम॒न्थिनौ॑ शु॒क्राम॒न्थिना॑ वे॒ता वे॒तौ शु॒क्राम॒न्थिना॑ वगृह्णन् नगृह्णञ् छु॒क्राम॒न्थिना॑ वे॒ता वे॒तौ शु॒क्राम॒न्थिना॑ वगृह्णन्न् । \newline
42. शु॒क्राम॒न्थिना॑ वगृह्णन् नगृह्णञ् छु॒क्राम॒न्थिनौ॑ शु॒क्राम॒न्थिना॑ वगृह्ण॒न् तत॒ स्ततो॑ ऽगृह्णञ् छु॒क्राम॒न्थिनौ॑ शु॒क्राम॒न्थिना॑ वगृह्ण॒न् ततः॑ । \newline
43. शु॒क्राम॒न्थिना॒विति॑ शु॒क्रा - म॒न्थिनौ᳚ । \newline
44. अ॒गृ॒ह्ण॒न् तत॒ स्ततो॑ ऽगृह्णन् नगृह्ण॒न् ततो॑ दे॒वा दे॒वा स्ततो॑ ऽगृह्णन् नगृह्ण॒न् ततो॑ दे॒वाः । \newline
45. ततो॑ दे॒वा दे॒वा स्तत॒ स्ततो॑ दे॒वा अभ॑व॒न् नभ॑वन् दे॒वा स्तत॒ स्ततो॑ दे॒वा अभ॑वन्न् । \newline
46. दे॒वा अभ॑व॒न् नभ॑वन् दे॒वा दे॒वा अभ॑व॒न् परा॒ परा ऽभ॑वन् दे॒वा दे॒वा अभ॑व॒न् परा᳚ । \newline
47. अभ॑व॒न् परा॒ परा ऽभ॑व॒न् नभ॑व॒न् परा ऽसु॑रा॒ असु॑राः॒ परा ऽभ॑व॒न् नभ॑व॒न् परा ऽसु॑राः । \newline
48. परा ऽसु॑रा॒ असु॑राः॒ परा॒ परा ऽसु॑रा॒ यस्य॒ यस्या सु॑राः॒ परा॒ परा ऽसु॑रा॒ यस्य॑ । \newline
49. असु॑रा॒ यस्य॒ यस्या सु॑रा॒ असु॑रा॒ यस्यै॒व मे॒वं ॅयस्या सु॑रा॒ असु॑रा॒ यस्यै॒वम् । \newline
50. यस्यै॒व मे॒वं ॅयस्य॒ यस्यै॒वं ॅवि॒दुषो॑ वि॒दुष॑ ए॒वं ॅयस्य॒ यस्यै॒वं ॅवि॒दुषः॑ । \newline
51. ए॒वं ॅवि॒दुषो॑ वि॒दुष॑ ए॒व मे॒वं ॅवि॒दुषः॑ शु॒क्राम॒न्थिनौ॑ शु॒क्राम॒न्थिनौ॑ वि॒दुष॑ ए॒व मे॒वं ॅवि॒दुषः॑ शु॒क्राम॒न्थिनौ᳚ । \newline
52. वि॒दुषः॑ शु॒क्राम॒न्थिनौ॑ शु॒क्राम॒न्थिनौ॑ वि॒दुषो॑ वि॒दुषः॑ शु॒क्राम॒न्थिनौ॑ गृ॒ह्येते॑ गृ॒ह्येते॑ शु॒क्राम॒न्थिनौ॑ वि॒दुषो॑ वि॒दुषः॑ शु॒क्राम॒न्थिनौ॑ गृ॒ह्येते᳚ । \newline
53. शु॒क्राम॒न्थिनौ॑ गृ॒ह्येते॑ गृ॒ह्येते॑ शु॒क्राम॒न्थिनौ॑ शु॒क्राम॒न्थिनौ॑ गृ॒ह्येते॒ भव॑ति॒ भव॑ति गृ॒ह्येते॑ शु॒क्राम॒न्थिनौ॑ शु॒क्राम॒न्थिनौ॑ गृ॒ह्येते॒ भव॑ति । \newline
54. शु॒क्राम॒न्थिना॒विति॑ शु॒क्रा - म॒न्थिनौ᳚ । \newline
55. गृ॒ह्येते॒ भव॑ति॒ भव॑ति गृ॒ह्येते॑ गृ॒ह्येते॒ भव॑ त्या॒त्मना॒ ऽऽत्मना॒ भव॑ति गृ॒ह्येते॑ गृ॒ह्येते॒ भव॑ त्या॒त्मना᳚ । \newline
56. गृ॒ह्येते॒ इति॑ गृ॒ह्येते᳚ । \newline
57. भव॑ त्या॒त्मना॒ ऽऽत्मना॒ भव॑ति॒ भव॑ त्या॒त्मना॒ परा॒ परा॒ ऽऽत्मना॒ भव॑ति॒ भव॑ त्या॒त्मना॒ परा᳚ । \newline
58. आ॒त्मना॒ परा॒ परा॒ ऽऽत्मना॒ ऽऽत्मना॒ परा᳚ ऽस्यास्य॒ परा॒ ऽऽत्मना॒ ऽऽत्मना॒ परा᳚ ऽस्य । \newline
59. परा᳚ ऽस्यास्य॒ परा॒ परा᳚ ऽस्य॒ भ्रातृ॑व्यो॒ भ्रातृ॑व्यो ऽस्य॒ परा॒ परा᳚ ऽस्य॒ भ्रातृ॑व्यः । \newline
\pagebreak
\markright{ TS 6.4.10.2  \hfill https://www.vedavms.in \hfill}

\section{ TS 6.4.10.2 }

\textbf{TS 6.4.10.2 } \newline
\textbf{Samhita Paata} \newline

ऽस्य॒ भ्रातृ॑व्यो भवति॒ तौ दे॒वा अ॑प॒नुद्या॒ऽऽ*त्मन॒ इन्द्रा॑याजुहवु॒-रप॑नुत्तौ॒ शण्डा॒मर्कौ॑ स॒हामुनेति॑ ब्रूया॒द्यं द्वि॒ष्याद्यमे॒व द्वेष्टि॒ तेनै॑नौ स॒हाप॑ नुदते॒ स प्र॑थ॒मः संकृ॑ति-र्वि॒श्वक॒र्मेत्ये॒वैना॑-वा॒त्मन॒ इन्द्रा॑या-जुहवु॒रिन्द्रो॒ ह्ये॑तानि॑ रू॒पाणि॒ करि॑क्र॒दच॑रद॒सौ वा आ॑दि॒त्यः शु॒क्रश्च॒न्द्रमा॑ म॒न्थ्य॑पि॒-गृह्य॒ प्राञ्चौ॒ नि- [  ] \newline

\textbf{Pada Paata} \newline

अ॒स्य॒ । भ्रातृ॑व्यः । भ॒व॒ति॒ । तौ । दे॒वाः । अ॒प॒नुद्येत्य॑प - नुद्य॑ । आ॒त्मने᳚ । इन्द्रा॑य । अ॒जु॒ह॒वुः॒ । अप॑नुत्ता॒वित्यप॑ - नु॒त्तौ॒ । शण्डा॒मर्का॒विति॒ शण्डा᳚ - मर्कौ᳚ । स॒ह । अ॒मुना᳚ । इति॑ । ब्रू॒या॒त् । यम् । द्वि॒ष्यात् । यम् । ए॒व । द्वेष्टि॑ । तेन॑ । ए॒ना॒ । स॒ह । अपेति॑ । नु॒द॒ते॒ । सः । प्र॒थ॒मः । संकृ॑ति॒रिति॒ सं - कृ॒तिः॒ । वि॒श्वक॒र्मेति॑ वि॒श्व - क॒र्मा॒ । इति॑ । ए॒व । ए॒नौ॒ । आ॒त्मने᳚ । इन्द्रा॑य । अ॒जु॒ह॒वुः॒ । इन्द्रः॑ । हि । ए॒तानि॑ । रू॒पाणि॑ । करि॑क्रत् । अच॑रत् । अ॒सौ । वै । आ॒दि॒त्यः । शु॒क्रः । च॒न्द्रमाः᳚ । म॒न्थी । अ॒पि॒गृह्येत्य॑पि - गृह्य॑ । प्राञ्चौ᳚ । निरिति॑ ।  \newline


\textbf{Krama Paata} \newline

अ॒स्य॒ भ्रातृ॑व्यः । भ्रातृ॑व्यो भवति । भ॒व॒ति॒ तौ । तौ दे॒वाः । दे॒वा अ॑प॒नुद्य॑ । अ॒प॒नुद्या॒त्मने᳚ । अ॒प॒नुद्येत्य॑प - नुद्य॑ । आ॒त्मन॒ इन्द्रा॑य । इन्द्रा॑याजुहवुः । अ॒जु॒ह॒वु॒रप॑नुत्तौ । अप॑नुत्तौ॒ शण्डा॒मर्कौ᳚ । अप॑नुत्ता॒वित्यप॑ - नु॒त्तौ॒ । शण्डा॒मर्कौ॑ स॒ह । शण्डा॒मर्का॒विति॒ शण्डा᳚ - मर्कौ᳚ । स॒हामुना᳚ । अ॒मुनेति॑ । इति॑ ब्रूयात् । ब्रू॒या॒द् यम् । यम् द्वि॒ष्यात् । द्वि॒ष्याद् यम् । यमे॒व । ए॒व द्वेष्टि॑ । द्वेष्टि॒ तेन॑ । तेनै॑नौ । ए॒नौ॒ स॒ह । स॒हाप॑ । अप॑ नुदते । नु॒द॒ते॒ सः । स प्र॑थ॒मः । प्र॒थ॒मः सङ्‍कृ॑तिः । सङ्‍कृ॑तिर् वि॒श्वक॑र्मा । सङ्‍कृ॑ति॒रिति॒ सम् - कृ॒तिः॒ । वि॒श्वक॒र्मेति॑ । वि॒श्वक॒र्मेति॑ वि॒श्व - क॒र्मा॒ । इत्ये॒व । ए॒वैनौ᳚ । ए॒ना॒वा॒त्मने᳚ । आ॒त्मन॒ इन्द्रा॑य । इन्द्रा॑याजुहवुः । अ॒जु॒ह॒वु॒रिन्द्रः॑ । इन्द्रो॒ हि । ह्ये॑तानि॑ । ए॒तानि॑ रू॒पाणि॑ । रू॒पाणि॒ करि॑क्रत् । करि॑क्र॒दच॑रत् । अच॑रद॒सौ । अ॒सौ वै । वा आ॑दि॒त्यः । आ॒दि॒त्यः शु॒क्रः । शु॒क्रश्च॒न्द्रमाः᳚ । च॒न्द्रमा॑ म॒न्थी । म॒न्थ्य॑पि॒गृह्य॑ । अ॒पि॒गृह्य॒ प्राञ्चौ᳚ । अ॒पि॒गृह्येत्य॑पि - गृह्य॑ । प्राञ्चौ॒ निः । निष्क्रा॑मतः \newline

\textbf{Jatai Paata} \newline

1. अ॒स्य॒ भ्रातृ॑व्यो॒ भ्रातृ॑व्यो ऽस्यास्य॒ भ्रातृ॑व्यः । \newline
2. भ्रातृ॑व्यो भवति भवति॒ भ्रातृ॑व्यो॒ भ्रातृ॑व्यो भवति । \newline
3. भ॒व॒ति॒ तौ तौ भ॑वति भवति॒ तौ । \newline
4. तौ दे॒वा दे॒वा स्तौ तौ दे॒वाः । \newline
5. दे॒वा अ॑प॒नुद्या॑ प॒नुद्य॑ दे॒वा दे॒वा अ॑प॒नुद्य॑ । \newline
6. अ॒प॒नुद्या॒ त्मन॑ आ॒त्मने॑ ऽप॒नुद्या॑ प॒नुद्या॒ त्मने᳚ । \newline
7. अ॒प॒नुद्येत्य॑प - नुद्य॑ । \newline
8. आ॒त्मन॒ इन्द्रा॒ येन्द्रा॑या॒ त्मन॑ आ॒त्मन॒ इन्द्रा॑य । \newline
9. इन्द्रा॑या जुहवु रजुहवु॒ रिन्द्रा॒ येन्द्रा॑या जुहवुः । \newline
10. अ॒जु॒ह॒वु॒ रप॑नुत्ता॒ वप॑नुत्ता वजुहवु रजुहवु॒ रप॑नुत्तौ । \newline
11. अप॑नुत्तौ॒ शण्डा॒मर्कौ॒ शण्डा॒मर्का॒ वप॑नुत्ता॒ वप॑नुत्तौ॒ शण्डा॒मर्कौ᳚ । \newline
12. अप॑नुत्ता॒वित्यप॑ - नु॒त्तौ॒ । \newline
13. शण्डा॒मर्कौ॑ स॒ह स॒ह शण्डा॒मर्कौ॒ शण्डा॒मर्कौ॑ स॒ह । \newline
14. शण्डा॒मर्का॒विति॒ शण्डा᳚ - मर्कौ᳚ । \newline
15. स॒हामुना॒ ऽमुना॑ स॒ह स॒हामुना᳚ । \newline
16. अ॒मुनेती त्य॒मुना॒ ऽमुनेति॑ । \newline
17. इति॑ ब्रूयाद् ब्रूया॒ दितीति॑ ब्रूयात् । \newline
18. ब्रू॒या॒द् यं ॅयम् ब्रू॑याद् ब्रूया॒द् यम् । \newline
19. यम् द्वि॒ष्याद् द्वि॒ष्याद् यं ॅयम् द्वि॒ष्यात् । \newline
20. द्वि॒ष्याद् यं ॅयम् द्वि॒ष्याद् द्वि॒ष्याद् यम् । \newline
21. य मे॒वैव यं ॅय मे॒व । \newline
22. ए॒व द्वेष्टि॒ द्वेष्ट्ये॒ वैव द्वेष्टि॑ । \newline
23. द्वेष्टि॒ तेन॒ तेन॒ द्वेष्टि॒ द्वेष्टि॒ तेन॑ । \newline
24. तेनै॑ना वेनौ॒ तेन॒ तेनै॑नौ । \newline
25. ए॒नौ॒ स॒ह स॒हैना॑ वेनौ स॒हा । \newline
26. स॒हा पाप॑ स॒ह स॒हाप॑ । \newline
27. अप॑ नुदते नुद॒ते ऽपाप॑ नुदते । \newline
28. नु॒द॒ते॒ स स नु॑दते नुदते॒ सः । \newline
29. स प्र॑थ॒मः प्र॑थ॒मः स स प्र॑थ॒मः । \newline
30. प्र॒थ॒मः सङ्कृ॑तिः॒ सङ्कृ॑तिः प्रथ॒मः प्र॑थ॒मः सङ्कृ॑तिः । \newline
31. सङ्कृ॑तिर् वि॒श्वक॑र्मा वि॒श्वक॑र्मा॒ सङ्कृ॑तिः॒ सङ्कृ॑तिर् वि॒श्वक॑र्मा । \newline
32. सङ्कृ॑ति॒रिति॒ सं - कृ॒तिः॒ । \newline
33. वि॒श्वक॒र्मेतीति॑ वि॒श्वक॑र्मा वि॒श्वक॒र्मेति॑ । \newline
34. वि॒श्वक॒र्मेति॑ वि॒श्व - क॒र्मा॒ । \newline
35. इत्ये॒ वैवे तीत्ये॒व । \newline
36. ए॒वैना॑ वेना वे॒वै वैनौ᳚ । \newline
37. ए॒ना॒ वा॒त्मन॑ आ॒त्मन॑ एना वेना वा॒त्मने᳚ । \newline
38. आ॒त्मन॒ इन्द्रा॒ येन्द्रा॑या॒ त्मन॑ आ॒त्मन॒ इन्द्रा॑य । \newline
39. इन्द्रा॑या जुहवु रजुहवु॒ रिन्द्रा॒ येन्द्रा॑या जुहवुः । \newline
40. अ॒जु॒ह॒वु॒ रिन्द्र॒ इन्द्रो॑ ऽजुहवु रजुहवु॒ रिन्द्रः॑ । \newline
41. इन्द्रो॒ हि हीन्द्र॒ इन्द्रो॒ हि । \newline
42. ह्ये॑ता न्ये॒तानि॒ हि ह्ये॑तानि॑ । \newline
43. ए॒तानि॑ रू॒पाणि॑ रू॒पा ण्ये॒ता न्ये॒तानि॑ रू॒पाणि॑ । \newline
44. रू॒पाणि॒ करि॑क्र॒त् करि॑क्रद् रू॒पाणि॑ रू॒पाणि॒ करि॑क्रत् । \newline
45. करि॑क्र॒ दच॑र॒ दच॑र॒त् करि॑क्र॒त् करि॑क्र॒ दच॑रत् । \newline
46. अच॑र द॒सा व॒सा वच॑र॒ दच॑र द॒सौ । \newline
47. अ॒सौ वै वा अ॒सा व॒सौ वै । \newline
48. वा आ॑दि॒त्य आ॑दि॒त्यो वै वा आ॑दि॒त्यः । \newline
49. आ॒दि॒त्यः शु॒क्रः शु॒क्र आ॑दि॒त्य आ॑दि॒त्यः शु॒क्रः । \newline
50. शु॒क्र श्च॒न्द्रमा᳚ श्च॒न्द्रमाः᳚ शु॒क्रः शु॒क्र श्च॒न्द्रमाः᳚ । \newline
51. च॒न्द्रमा॑ म॒न्थी म॒न्थी च॒न्द्रमा᳚ श्च॒न्द्रमा॑ म॒न्थी । \newline
52. म॒न्थ्य॑ पि॒गृह्या॑ पि॒गृह्य॑ म॒न्थी म॒न्थ्य॑ पि॒गृह्य॑ । \newline
53. अ॒पि॒गृह्य॒ प्राञ्चौ॒ प्राञ्चा॑ वपि॒गृह्या॑ पि॒गृह्य॒ प्राञ्चौ᳚ । \newline
54. अ॒पि॒गृह्येत्य॑पि - गृह्य॑ । \newline
55. प्राञ्चौ॒ निः णिष् प्राञ्चौ॒ प्राञ्चौ॒ निः । \newline
56. नि ष्क्रा॑मतः क्रामतो॒ निर् णिष्क्रा॑मतः । \newline

\textbf{Ghana Paata } \newline

1. अ॒स्य॒ भ्रातृ॑व्यो॒ भ्रातृ॑व्यो ऽस्यास्य॒ भ्रातृ॑व्यो भवति भवति॒ भ्रातृ॑व्यो ऽस्यास्य॒ भ्रातृ॑व्यो भवति । \newline
2. भ्रातृ॑व्यो भवति भवति॒ भ्रातृ॑व्यो॒ भ्रातृ॑व्यो भवति॒ तौ तौ भ॑वति॒ भ्रातृ॑व्यो॒ भ्रातृ॑व्यो भवति॒ तौ । \newline
3. भ॒व॒ति॒ तौ तौ भ॑वति भवति॒ तौ दे॒वा दे॒वा स्तौ भ॑वति भवति॒ तौ दे॒वाः । \newline
4. तौ दे॒वा दे॒वा स्तौ तौ दे॒वा अ॑प॒नुद्या॑ प॒नुद्य॑ दे॒वा स्तौ तौ दे॒वा अ॑प॒नुद्य॑ । \newline
5. दे॒वा अ॑प॒नुद्या॑ प॒नुद्य॑ दे॒वा दे॒वा अ॑प॒नु द्या॒त्मन॑ आ॒त्मने॑ ऽप॒नुद्य॑ दे॒वा दे॒वा अ॑प॒नु द्या॒त्मने᳚ । \newline
6. अ॒प॒नु द्या॒त्मन॑ आ॒त्मने॑ ऽप॒नुद्या॑ प॒नु द्या॒त्मन॒ इन्द्रा॒ येन्द्रा॑या॒त्मने॑ ऽप॒नुद्या॑ प॒नु द्या॒त्मन॒ इन्द्रा॑य । \newline
7. अ॒प॒नुद्येत्य॑प - नुद्य॑ । \newline
8. आ॒त्मन॒ इन्द्रा॒ येन्द्रा॑या॒त्मन॑ आ॒त्मन॒ इन्द्रा॑या जुहवु रजुहवु॒ रिन्द्रा॑ या॒त्मन॑ आ॒त्मन॒ इन्द्रा॑या जुहवुः । \newline
9. इन्द्रा॑या जुहवु रजुहवु॒ रिन्द्रा॒ येन्द्रा॑या जुहवु॒ रप॑नुत्ता॒ वप॑नुत्ता वजुहवु॒ रिन्द्रा॒ येन्द्रा॑या जुहवु॒ रप॑नुत्तौ । \newline
10. अ॒जु॒ह॒वु॒ रप॑नुत्ता॒ वप॑नुत्ता वजुहवु रजुहवु॒ रप॑नुत्तौ॒ शण्डा॒मर्कौ॒ शण्डा॒मर्का॒ वप॑नुत्ता वजुहवु रजुहवु॒ रप॑नुत्तौ॒ शण्डा॒मर्कौ᳚ । \newline
11. अप॑नुत्तौ॒ शण्डा॒मर्कौ॒ शण्डा॒मर्का॒ वप॑नुत्ता॒ वप॑नुत्तौ॒ शण्डा॒मर्कौ॑ स॒ह स॒ह शण्डा॒मर्का॒ वप॑नुत्ता॒ वप॑नुत्तौ॒ शण्डा॒मर्कौ॑ स॒ह । \newline
12. अप॑नुत्ता॒वित्यप॑ - नु॒त्तौ॒ । \newline
13. शण्डा॒मर्कौ॑ स॒ह स॒ह शण्डा॒मर्कौ॒ शण्डा॒मर्कौ॑ स॒हामुना॒ ऽमुना॑ स॒ह शण्डा॒मर्कौ॒ शण्डा॒मर्कौ॑ स॒हामुना᳚ । \newline
14. शण्डा॒मर्का॒विति॒ शण्डा᳚ - मर्कौ᳚ । \newline
15. स॒हामुना॒ ऽमुना॑ स॒ह स॒हा मुने तीत्य॒ मुना॑ स॒ह स॒हामुनेति॑ । \newline
16. अ॒मुने तीत्य॒मुना॒ ऽमुनेति॑ ब्रूयाद् ब्रूया॒ दित्य॒मुना॒ ऽमुनेति॑ ब्रूयात् । \newline
17. इति॑ ब्रूयाद् ब्रूया॒दि तीति॑ ब्रूया॒द् यं ॅयम् ब्रू॑या॒ दितीति॑ ब्रूया॒द् यम् । \newline
18. ब्रू॒या॒द् यं ॅयम् ब्रू॑याद् ब्रूया॒द् यम् द्वि॒ष्याद् द्वि॒ष्याद् यम् ब्रू॑याद् ब्रूया॒द् यम् द्वि॒ष्यात् । \newline
19. यम् द्वि॒ष्याद् द्वि॒ष्याद् यं ॅयम् द्वि॒ष्याद् यं ॅयम् द्वि॒ष्याद् यं ॅयम् द्वि॒ष्याद् यम् । \newline
20. द्वि॒ष्याद् यं ॅयम् द्वि॒ष्याद् द्वि॒ष्याद् यमे॒वैव यम् द्वि॒ष्याद् द्वि॒ष्याद् यमे॒व । \newline
21. यमे॒वैव यं ॅयमे॒व द्वेष्टि॒ द्वेष्ट्ये॒व यं ॅयमे॒व द्वेष्टि॑ । \newline
22. ए॒व द्वेष्टि॒ द्वेष्ट्ये॒ वैव द्वेष्टि॒ तेन॒ तेन॒ द्वेष्ट्ये॒ वैव द्वेष्टि॒ तेन॑ । \newline
23. द्वेष्टि॒ तेन॒ तेन॒ द्वेष्टि॒ द्वेष्टि॒ तेनै॑ना वेनौ॒ तेन॒ द्वेष्टि॒ द्वेष्टि॒ तेनै॑नौ । \newline
24. तेनै॑ना वेनौ॒ तेन॒ तेनै॑नौ स॒ह स॒हैनौ॒ तेन॒ तेनै॑नौ स॒ह । \newline
25. ए॒नौ॒ स॒ह स॒हैना॑ वेनौ स॒हा पाप॑ स॒हैना॑ वेनौ स॒हाप॑ । \newline
26. स॒हा पाप॑ स॒ह स॒हाप॑ नुदते नुद॒ते ऽप॑ स॒ह स॒हाप॑ नुदते । \newline
27. अप॑ नुदते नुद॒ते ऽपाप॑ नुदते॒ स स नु॑द॒ते ऽपाप॑ नुदते॒ सः । \newline
28. नु॒द॒ते॒ स स नु॑दते नुदते॒ स प्र॑थ॒मः प्र॑थ॒मः स नु॑दते नुदते॒ स प्र॑थ॒मः । \newline
29. स प्र॑थ॒मः प्र॑थ॒मः स स प्र॑थ॒मः सङ्कृ॑तिः॒ सङ्कृ॑तिः प्रथ॒मः स स प्र॑थ॒मः सङ्कृ॑तिः । \newline
30. प्र॒थ॒मः सङ्कृ॑तिः॒ सङ्कृ॑तिः प्रथ॒मः प्र॑थ॒मः सङ्कृ॑तिर् वि॒श्वक॑र्मा वि॒श्वक॑र्मा॒ सङ्कृ॑तिः प्रथ॒मः प्र॑थ॒मः सङ्कृ॑तिर् वि॒श्वक॑र्मा । \newline
31. सङ्कृ॑तिर् वि॒श्वक॑र्मा वि॒श्वक॑र्मा॒ सङ्कृ॑तिः॒ सङ्कृ॑तिर् वि॒श्वक॒र्मेतीति॑ वि॒श्वक॑र्मा॒ सङ्कृ॑तिः॒ सङ्कृ॑तिर् वि॒श्वक॒र्मेति॑ । \newline
32. सङ्कृ॑ति॒रिति॒ सं - कृ॒तिः॒ । \newline
33. वि॒श्वक॒र्मेतीति॑ वि॒श्वक॑र्मा वि॒श्वक॒र्मे त्ये॒वैवेति॑ वि॒श्वक॑र्मा वि॒श्वक॒र्मे त्ये॒व । \newline
34. वि॒श्वक॒र्मेति॑ वि॒श्व - क॒र्मा॒ । \newline
35. इत्ये॒ वैवे तीत्ये॒ वैना॑ वेना वे॒वे तीत्ये॒ वैनौ᳚ । \newline
36. ए॒वैना॑ वेना वे॒वै वैना॑ वा॒त्मन॑ आ॒त्मन॑ एना वे॒वै वैना॑ वा॒त्मने᳚ । \newline
37. ए॒ना॒ वा॒त्मन॑ आ॒त्मन॑ एना वेना वा॒त्मन॒ इन्द्रा॒ येन्द्रा॑या॒ त्मन॑ एना वेना वा॒त्मन॒ इन्द्रा॑य । \newline
38. आ॒त्मन॒ इन्द्रा॒ येन्द्रा॑या॒ त्मन॑ आ॒त्मन॒ इन्द्रा॑या जुहवु रजुहवु॒ रिन्द्रा॑या॒ त्मन॑ आ॒त्मन॒ इन्द्रा॑या जुहवुः । \newline
39. इन्द्रा॑या जुहवु रजुहवु॒ रिन्द्रा॒ येन्द्रा॑या जुहवु॒ रिन्द्र॒ इन्द्रो॑ ऽजुहवु॒ रिन्द्रा॒ येन्द्रा॑या जुहवु॒ रिन्द्रः॑ । \newline
40. अ॒जु॒ह॒वु॒ रिन्द्र॒ इन्द्रो॑ ऽजुहवु रजुहवु॒ रिन्द्रो॒ हि हीन्द्रो॑ ऽजुहवु रजुहवु॒ रिन्द्रो॒ हि । \newline
41. इन्द्रो॒ हि हीन्द्र॒ इन्द्रो॒ ह्ये॑ता न्ये॒तानि॒ हीन्द्र॒ इन्द्रो॒ ह्ये॑तानि॑ । \newline
42. ह्ये॑ता न्ये॒तानि॒ हि ह्ये॑तानि॑ रू॒पाणि॑ रू॒पा ण्ये॒तानि॒ हि ह्ये॑तानि॑ रू॒पाणि॑ । \newline
43. ए॒तानि॑ रू॒पाणि॑ रू॒पा ण्ये॒ता न्ये॒तानि॑ रू॒पाणि॒ करि॑क्र॒त् करि॑क्रद् रू॒पा ण्ये॒ता न्ये॒तानि॑ रू॒पाणि॒ करि॑क्रत् । \newline
44. रू॒पाणि॒ करि॑क्र॒त् करि॑क्रद् रू॒पाणि॑ रू॒पाणि॒ करि॑क्र॒ दच॑र॒ दच॑र॒त् करि॑क्रद् रू॒पाणि॑ रू॒पाणि॒ करि॑क्र॒ दच॑रत् । \newline
45. करि॑क्र॒ दच॑र॒ दच॑र॒त् करि॑क्र॒त् करि॑क्र॒ दच॑र द॒सा व॒सा वच॑र॒त् करि॑क्र॒त् करि॑क्र॒ दच॑र द॒सौ । \newline
46. अच॑र द॒सा व॒सा वच॑र॒ दच॑र द॒सौ वै वा अ॒सा वच॑र॒ दच॑र द॒सौ वै । \newline
47. अ॒सौ वै वा अ॒सा व॒सौ वा आ॑दि॒त्य आ॑दि॒त्यो वा अ॒सा व॒सौ वा आ॑दि॒त्यः । \newline
48. वा आ॑दि॒त्य आ॑दि॒त्यो वै वा आ॑दि॒त्यः शु॒क्रः शु॒क्र आ॑दि॒त्यो वै वा आ॑दि॒त्यः शु॒क्रः । \newline
49. आ॒दि॒त्यः शु॒क्रः शु॒क्र आ॑दि॒त्य आ॑दि॒त्यः शु॒क्र श्च॒न्द्रमा᳚ श्च॒न्द्रमाः᳚ शु॒क्र आ॑दि॒त्य आ॑दि॒त्यः शु॒क्र श्च॒न्द्रमाः᳚ । \newline
50. शु॒क्र श्च॒न्द्रमा᳚ श्च॒न्द्रमाः᳚ शु॒क्रः शु॒क्र श्च॒न्द्रमा॑ म॒न्थी म॒न्थी च॒न्द्रमाः᳚ शु॒क्रः शु॒क्र श्च॒न्द्रमा॑ म॒न्थी । \newline
51. च॒न्द्रमा॑ म॒न्थी म॒न्थी च॒न्द्रमा᳚ श्च॒न्द्रमा॑ म॒न्थ्य॑ पि॒गृह्या॑ पि॒गृह्य॑ म॒न्थी च॒न्द्रमा᳚ श्च॒न्द्रमा॑ म॒न्थ्य॑ पि॒गृह्य॑ । \newline
52. म॒न्थ्य॑ पि॒गृह्या॑ पि॒गृह्य॑ म॒न्थी म॒न्थ्य॑ पि॒गृह्य॒ प्राञ्चौ॒ प्राञ्चा॑ वपि॒गृह्य॑ म॒न्थी म॒न्थ्य॑ पि॒गृह्य॒ प्राञ्चौ᳚ । \newline
53. अ॒पि॒गृह्य॒ प्राञ्चौ॒ प्राञ्चा॑ वपि॒गृह्या॑ पि॒गृह्य॒ प्राञ्चौ॒ निर् णिष् प्राञ्चा॑ वपि॒गृह्या॑ पि॒गृह्य॒ प्राञ्चौ॒ निः । \newline
54. अ॒पि॒गृह्येत्य॑पि - गृह्य॑ । \newline
55. प्राञ्चौ॒ निर् णिष् प्राञ्चौ॒  प्राञ्चौ॒ नि ष्क्रा॑मतः क्रामतो॒ निष् प्राञ्चौ॒ प्राञ्चौ॒ नि ष्क्रा॑मतः । \newline
56. नि ष्क्रा॑मतः क्रामतो॒ निर् णिष्क्रा॑मत॒ स्तस्मा॒त् तस्मा᳚त् क्रामतो॒ निर् णि ष्क्रा॑मत॒ स्तस्मा᳚त् । \newline
\pagebreak
\markright{ TS 6.4.10.3  \hfill https://www.vedavms.in \hfill}

\section{ TS 6.4.10.3 }

\textbf{TS 6.4.10.3 } \newline
\textbf{Samhita Paata} \newline

-ष्क्रा॑मत॒-स्तस्मा॒त् प्राञ्चौ॒ यन्तौ॒ न प॑श्यन्ति प्र॒त्यञ्चा॑वा॒वृत्य॑ जुहुत॒स्तस्मा᳚त् प्र॒त्यञ्चौ॒ यन्तौ॑ पश्यन्ति॒ चक्षु॑षी॒ वा ए॒ते य॒ज्ञ्स्य॒ यच्छु॒क्राम॒न्थिनौ॒ नासि॑कोत्तरवे॒दिर॒भितः॑ परि॒क्रम्य॑ जुहुत॒स्तस्मा॑द॒भितो॒ नासि॑कां॒ चक्षु॑षी॒ तस्मा॒न्नासि॑कया॒ चक्षु॑षी॒ विधृ॑ते स॒र्वतः॒ परि॑ क्रामतो॒ रक्ष॑सा॒मप॑हत्यै दे॒वा वै याः प्राची॒राहु॑ती॒रजु॑हवु॒र्ये पु॒रस्ता॒दसु॑रा॒ आस॒न् ताꣳस्ताभिः॒ प्रा- [  ] \newline

\textbf{Pada Paata} \newline

क्रा॒म॒तः॒ । तस्मा᳚त् । प्राञ्चौ᳚ । यन्तौ᳚ । न । प॒श्य॒न्ति॒ । प्र॒त्यञ्चौ᳚ । आ॒वृत्येत्या᳚ - वृत्य॑ । जु॒हु॒तः॒ । तस्मा᳚त् । प्र॒त्यञ्चौ᳚ । यन्तौ᳚ । प॒श्य॒न्ति॒ । चक्षु॑षी॒ इति॑ । वै । ए॒ते इति॑ । य॒ज्ञ्स्य॑ । यत् । शु॒क्राम॒न्थिना॒विति॑ शु॒क्रा - म॒न्थिनौ᳚ । नासि॑का । उ॒त्त॒र॒वे॒दिरित्यु॑त्तर- वे॒दिः । अ॒भितः॑ । प॒रि॒क्रम्येति॑ परि - क्रम्य॑ । जु॒हु॒तः॒ । तस्मा᳚त् । अ॒भितः॑ । नासि॑काम् । चक्षु॑षी॒ इति॑ । तस्मा᳚त् । नासि॑कया । चक्षु॑षी॒ इति॑ । विधृ॑ते॒ इति॒ वि-धृ॒ते॒ । स॒र्वतः॑ । परीति॑ । क्रा॒म॒तः॒ । रक्ष॑साम् । अप॑हत्या॒ इत्यप॑ - ह॒त्यै॒ । दे॒वाः । वै । याः । प्राचीः᳚ । आहु॑ती॒रित्या - हु॒तीः॒ । अजु॑हवुः । ये । पु॒रस्ता᳚त् । असु॑राः । आसन्न्॑ । तान् । ताभिः॑ । प्रेति॑ ।  \newline


\textbf{Krama Paata} \newline

क्रा॒म॒त॒स्तस्मा᳚त् । तस्मा॒त् प्राञ्चौ᳚ । प्राञ्चौ॒ यन्तौ᳚ । यन्तौ॒ न । न प॑श्यन्ति । प॒श्य॒न्ति॒ प्र॒त्यञ्चौ᳚ । प्र॒त्यञ्चा॑वा॒वृत्य॑ । आ॒वृत्य॑ जुहुतः । आ॒वृत्येत्या᳚ - वृत्य॑ । जु॒हु॒त॒स्तस्मा᳚त् । तस्मा᳚त् प्र॒त्यञ्चौ᳚ । प्र॒त्यञ्चौ॒ यन्तौ᳚ । यन्तौ॑ पश्यन्ति । प॒श्य॒न्ति॒ चक्षु॑षी । चक्षु॑षी॒ वै । चक्षु॑षी॒ इति॒ चक्षु॑षी । वा ए॒ते । ए॒ते य॒ज्ञ्स्य॑ । ए॒ते इत्ये॒ते । य॒ज्ञ्स्य॒ यत् । यच्छु॒क्राम॒न्थिनौ᳚ । शु॒क्राम॒न्थिनौ॒ नासि॑का । शु॒क्राम॒न्थिना॒विति॑ शु॒क्रा - म॒न्थिनौ᳚ । नासि॑कोत्तरवे॒दिः । उ॒त्त॒र॒वे॒दिर॒भितः॑ । उ॒त्त॒र॒वे॒दिरित्यु॑त्तर - वे॒दिः । अ॒भितः॑ परि॒क्रम्य॑ । प॒रि॒क्रम्य॑ जुहतः । प॒रि॒क्रम्येति॑ परि - क्रम्य॑ । जु॒हु॒त॒स्तस्मा᳚त् । तस्मा॑द॒भितः॑ । अ॒भितो॒ नासि॑काम् । नासि॑का॒म् चक्षु॑षी । चक्षु॑षी॒ तस्मा᳚त् । चक्षु॑षी॒ इति॒ चक्षु॑षी । तस्मा॒न् नासि॑कया । नासि॑कया॒ चक्षु॑षी । चक्षु॑षी॒ विधृ॑ते । चक्षु॑षी॒ इति॒ चक्षु॑षी । विधृ॑ते स॒र्वतः॑ । विधृ॑ते॒ इति॒ वि - धृ॒ते॒ । स॒र्वतः॒ परि॑ । परि॑ क्रामतः । क्रा॒म॒तो॒ रक्ष॑साम् । रक्ष॑सा॒मप॑हत्यै । अप॑हत्यै दे॒वाः । अप॑हत्या॒ इत्यप॑ - ह॒त्यै॒ । दे॒वा वै । वै याः । याः प्राचीः᳚ । प्राची॒राहु॑तीः । आहु॑ती॒रजु॑हवुः । आहु॑ती॒रित्या - हु॒तीः॒ । अजु॑हवु॒र् ये । ये पु॒रस्ता᳚त् । पु॒रस्ता॒दसु॑राः । असु॑रा॒ आसन्न्॑ । आस॒न् तान् । ताꣳस्ताभिः॑ । ताभिः॒ प्र । प्राणु॑दन्त \newline

\textbf{Jatai Paata} \newline

1. क्रा॒म॒त॒ स्तस्मा॒त् तस्मा᳚त् क्रामतः क्रामत॒ स्तस्मा᳚त् । \newline
2. तस्मा॒त् प्राञ्चौ॒ प्राञ्चौ॒ तस्मा॒त् तस्मा॒त् प्राञ्चौ᳚ । \newline
3. प्राञ्चौ॒ यन्तौ॒ यन्तौ॒ प्राञ्चौ॒ प्राञ्चौ॒ यन्तौ᳚ । \newline
4. यन्तौ॒ न न यन्तौ॒ यन्तौ॒ न । \newline
5. न प॑श्यन्ति पश्यन्ति॒ न न प॑श्यन्ति । \newline
6. प॒श्य॒न्ति॒ प्र॒त्यञ्चौ᳚ प्र॒त्यञ्चौ॑ पश्यन्ति पश्यन्ति प्र॒त्यञ्चौ᳚ । \newline
7. प्र॒त्यञ्चा॑ वा॒वृत्या॒ वृत्य॑ प्र॒त्यञ्चौ᳚ प्र॒त्यञ्चा॑ वा॒वृत्य॑ । \newline
8. आ॒वृत्य॑ जुहुतो जुहुत आ॒वृत्या॒ वृत्य॑ जुहुतः । \newline
9. आ॒वृत्येत्या᳚ - वृत्य॑ । \newline
10. जु॒हु॒त॒ स्तस्मा॒त् तस्मा᳚ज् जुहुतो जुहुत॒ स्तस्मा᳚त् । \newline
11. तस्मा᳚त् प्र॒त्यञ्चौ᳚ प्र॒त्यञ्चौ॒ तस्मा॒त् तस्मा᳚त् प्र॒त्यञ्चौ᳚ । \newline
12. प्र॒त्यञ्चौ॒ यन्तौ॒ यन्तौ᳚ प्र॒त्यञ्चौ᳚ प्र॒त्यञ्चौ॒ यन्तौ᳚ । \newline
13. यन्तौ॑ पश्यन्ति पश्यन्ति॒ यन्तौ॒ यन्तौ॑ पश्यन्ति । \newline
14. प॒श्य॒न्ति॒ चक्षु॑षी॒ चक्षु॑षी पश्यन्ति पश्यन्ति॒ चक्षु॑षी । \newline
15. चक्षु॑षी॒ वै वै चक्षु॑षी॒ चक्षु॑षी॒ वै । \newline
16. चक्षु॑षी॒ इति॒ चक्षु॑षी । \newline
17. वा ए॒ते ए॒ते वै वा ए॒ते । \newline
18. ए॒ते य॒ज्ञ्स्य॑ य॒ज्ञ्स्यै॒ते ए॒ते य॒ज्ञ्स्य॑ । \newline
19. ए॒ते इत्ये॒ते । \newline
20. य॒ज्ञ्स्य॒ यद् यद् य॒ज्ञ्स्य॑ य॒ज्ञ्स्य॒ यत् । \newline
21. यच् छु॒क्राम॒न्थिनौ॑ शु॒क्राम॒न्थिनौ॒ यद् यच् छु॒क्राम॒न्थिनौ᳚ । \newline
22. शु॒क्राम॒न्थिनौ॒ नासि॑का॒ नासि॑का शु॒क्राम॒न्थिनौ॑ शु॒क्राम॒न्थिनौ॒ नासि॑का । \newline
23. शु॒क्राम॒न्थिना॒विति॑ शु॒क्रा - म॒न्थिनौ᳚ । \newline
24. नासि॑ कोत्तरवे॒दि रु॑त्तरवे॒दिर् नासि॑का॒ नासि॑ कोत्तरवे॒दिः । \newline
25. उ॒त्त॒र॒वे॒दि र॒भितो॒ ऽभित॑ उत्तरवे॒दि रु॑त्तरवे॒दि र॒भितः॑ । \newline
26. उ॒त्त॒र॒वे॒दिरित्यु॑त्तर - वे॒दिः । \newline
27. अ॒भितः॑ परि॒क्रम्य॑ परि॒क्र म्या॒भितो॒ ऽभितः॑ परि॒क्रम्य॑ । \newline
28. प॒रि॒क्रम्य॑ जुहुतो जुहुतः परि॒क्रम्य॑ परि॒क्रम्य॑ जुहुतः । \newline
29. प॒रि॒क्रम्येति॑ परि - क्रम्य॑ । \newline
30. जु॒हु॒त॒ स्तस्मा॒त् तस्मा᳚ज् जुहुतो जुहुत॒ स्तस्मा᳚त् । \newline
31. तस्मा॑ द॒भितो॒ ऽभित॒ स्तस्मा॒त् तस्मा॑ द॒भितः॑ । \newline
32. अ॒भितो॒ नासि॑का॒म् नासि॑का म॒भितो॒ ऽभितो॒ नासि॑काम् । \newline
33. नासि॑का॒म् चक्षु॑षी॒ चक्षु॑षी॒ नासि॑का॒म् नासि॑का॒म् चक्षु॑षी । \newline
34. चक्षु॑षी॒ तस्मा॒त् तस्मा॒च् चक्षु॑षी॒ चक्षु॑षी॒ तस्मा᳚त् । \newline
35. चक्षु॑षी॒ इति॒ चक्षु॑षी । \newline
36. तस्मा॒न् नासि॑कया॒ नासि॑कया॒ तस्मा॒त् तस्मा॒न् नासि॑कया । \newline
37. नासि॑कया॒ चक्षु॑षी॒ चक्षु॑षी॒ नासि॑कया॒ नासि॑कया॒ चक्षु॑षी । \newline
38. चक्षु॑षी॒ विधृ॑ते॒ विधृ॑ते॒ चक्षु॑षी॒ चक्षु॑षी॒ विधृ॑ते । \newline
39. चक्षु॑षी॒ इति॒ चक्षु॑षी । \newline
40. विधृ॑ते स॒र्वतः॑ स॒र्वतो॒ विधृ॑ते॒ विधृ॑ते स॒र्वतः॑ । \newline
41. विधृ॑ते॒ इति॒ वि - धृ॒ते॒ । \newline
42. स॒र्वतः॒ परि॒ परि॑ स॒र्वतः॑ स॒र्वतः॒ परि॑ । \newline
43. परि॑ क्रामतः क्रामतः॒ परि॒ परि॑ क्रामतः । \newline
44. क्रा॒म॒तो॒ रक्ष॑साꣳ॒॒ रक्ष॑साम् क्रामतः क्रामतो॒ रक्ष॑साम् । \newline
45. रक्ष॑सा॒ मप॑हत्या॒ अप॑हत्यै॒ रक्ष॑साꣳ॒॒ रक्ष॑सा॒ मप॑हत्यै । \newline
46. अप॑हत्यै दे॒वा दे॒वा अप॑हत्या॒ अप॑हत्यै दे॒वाः । \newline
47. अप॑हत्या॒ इत्यप॑ - ह॒त्यै॒ । \newline
48. दे॒वा वै वै दे॒वा दे॒वा वै । \newline
49. वै या या वै वै याः । \newline
50. याः प्राचीः॒ प्राची॒र् या याः प्राचीः᳚ । \newline
51. प्राची॒ राहु॑ती॒ राहु॑तीः॒ प्राचीः॒ प्राची॒ राहु॑तीः । \newline
52. आहु॑ती॒ रजु॑हवु॒ रजु॑हवु॒ राहु॑ती॒ राहु॑ती॒ रजु॑हवुः । \newline
53. आहु॑ती॒रित्या - हु॒तीः॒ । \newline
54. अजु॑हवु॒र् ये ये ऽजु॑हवु॒ रजु॑हवु॒र् ये । \newline
55. ये पु॒रस्ता᳚त् पु॒रस्ता॒द् ये ये पु॒रस्ता᳚त् । \newline
56. पु॒रस्ता॒ दसु॑रा॒ असु॑राः पु॒रस्ता᳚त् पु॒रस्ता॒ दसु॑राः । \newline
57. असु॑रा॒ आस॒न् नास॒न् नसु॑रा॒ असु॑रा॒ आसन्न्॑ । \newline
58. आस॒न् ताꣳ स्ता नास॒न् नास॒न् तान् । \newline
59. ताꣳ स्ताभि॒ स्ताभि॒ स्ताꣳ स्ताꣳ स्ताभिः॑ । \newline
60. ताभिः॒ प्र प्र ताभि॒ स्ताभिः॒ प्र । \newline
61. प्राणु॑दन्ता नुदन्त॒ प्र प्राणु॑दन्त । \newline

\textbf{Ghana Paata } \newline

1. क्रा॒म॒त॒ स्तस्मा॒त् तस्मा᳚त् क्रामतः क्रामत॒ स्तस्मा॒त् प्राञ्चौ॒ प्राञ्चौ॒ तस्मा᳚त् क्रामतः क्रामत॒ स्तस्मा॒त् प्राञ्चौ᳚ । \newline
2. तस्मा॒त् प्राञ्चौ॒ प्राञ्चौ॒ तस्मा॒त् तस्मा॒त् प्राञ्चौ॒ यन्तौ॒ यन्तौ॒ प्राञ्चौ॒ तस्मा॒त् तस्मा॒त् प्राञ्चौ॒ यन्तौ᳚ । \newline
3. प्राञ्चौ॒ यन्तौ॒ यन्तौ॒ प्राञ्चौ॒ प्राञ्चौ॒ यन्तौ॒ न न यन्तौ॒ प्राञ्चौ॒ प्राञ्चौ॒ यन्तौ॒ न । \newline
4. यन्तौ॒ न न यन्तौ॒ यन्तौ॒ न प॑श्यन्ति पश्यन्ति॒ न यन्तौ॒ यन्तौ॒ न प॑श्यन्ति । \newline
5. न प॑श्यन्ति पश्यन्ति॒ न न प॑श्यन्ति प्र॒त्यञ्चौ᳚ प्र॒त्यञ्चौ॑ पश्यन्ति॒ न न प॑श्यन्ति प्र॒त्यञ्चौ᳚ । \newline
6. प॒श्य॒न्ति॒ प्र॒त्यञ्चौ᳚ प्र॒त्यञ्चौ॑ पश्यन्ति पश्यन्ति प्र॒त्यञ्चा॑ वा॒वृत्या॒ वृत्य॑ प्र॒त्यञ्चौ॑ पश्यन्ति पश्यन्ति प्र॒त्यञ्चा॑ वा॒वृत्य॑ । \newline
7. प्र॒त्यञ्चा॑ वा॒वृत्या॒ वृत्य॑ प्र॒त्यञ्चौ᳚ प्र॒त्यञ्चा॑ वा॒वृत्य॑ जुहुतो जुहुत आ॒वृत्य॑ प्र॒त्यञ्चौ᳚ प्र॒त्यञ्चा॑ वा॒वृत्य॑ जुहुतः । \newline
8. आ॒वृत्य॑ जुहुतो जुहुत आ॒वृत्या॒ वृत्य॑ जुहुत॒ स्तस्मा॒त् तस्मा᳚ज् जुहुत आ॒वृत्या॒ वृत्य॑ जुहुत॒ स्तस्मा᳚त् । \newline
9. आ॒वृत्येत्या᳚ - वृत्य॑ । \newline
10. जु॒हु॒त॒ स्तस्मा॒त् तस्मा᳚ज् जुहुतो जुहुत॒ स्तस्मा᳚त् प्र॒त्यञ्चौ᳚ प्र॒त्यञ्चौ॒ तस्मा᳚ज् जुहुतो जुहुत॒ स्तस्मा᳚त् प्र॒त्यञ्चौ᳚ । \newline
11. तस्मा᳚त् प्र॒त्यञ्चौ᳚ प्र॒त्यञ्चौ॒ तस्मा॒त् तस्मा᳚त् प्र॒त्यञ्चौ॒ यन्तौ॒ यन्तौ᳚ प्र॒त्यञ्चौ॒ तस्मा॒त् तस्मा᳚त् प्र॒त्यञ्चौ॒ यन्तौ᳚ । \newline
12. प्र॒त्यञ्चौ॒ यन्तौ॒ यन्तौ᳚ प्र॒त्यञ्चौ᳚ प्र॒त्यञ्चौ॒ यन्तौ॑ पश्यन्ति पश्यन्ति॒ यन्तौ᳚ प्र॒त्यञ्चौ᳚ प्र॒त्यञ्चौ॒ यन्तौ॑ पश्यन्ति । \newline
13. यन्तौ॑ पश्यन्ति पश्यन्ति॒ यन्तौ॒ यन्तौ॑ पश्यन्ति॒ चक्षु॑षी॒ चक्षु॑षी पश्यन्ति॒ यन्तौ॒ यन्तौ॑ पश्यन्ति॒ चक्षु॑षी । \newline
14. प॒श्य॒न्ति॒ चक्षु॑षी॒ चक्षु॑षी पश्यन्ति पश्यन्ति॒ चक्षु॑षी॒ वै वै चक्षु॑षी पश्यन्ति पश्यन्ति॒ चक्षु॑षी॒ वै । \newline
15. चक्षु॑षी॒ वै वै चक्षु॑षी॒ चक्षु॑षी॒ वा ए॒ते ए॒ते वै चक्षु॑षी॒ चक्षु॑षी॒ वा ए॒ते । \newline
16. चक्षु॑षी॒ इति॒ चक्षु॑षी । \newline
17. वा ए॒ते ए॒ते वै वा ए॒ते य॒ज्ञ्स्य॑ य॒ज्ञ्स्यै॒ते वै वा ए॒ते य॒ज्ञ्स्य॑ । \newline
18. ए॒ते य॒ज्ञ्स्य॑ य॒ज्ञ्स्यै॒ते ए॒ते य॒ज्ञ्स्य॒ यद् यद् य॒ज्ञ्स्यै॒ते ए॒ते य॒ज्ञ्स्य॒ यत् । \newline
19. ए॒ते इत्ये॒ते । \newline
20. य॒ज्ञ्स्य॒ यद् यद् य॒ज्ञ्स्य॑ य॒ज्ञ्स्य॒ यच् छु॒क्राम॒न्थिनौ॑ शु॒क्राम॒न्थिनौ॒ यद् य॒ज्ञ्स्य॑ य॒ज्ञ्स्य॒ यच् छु॒क्राम॒न्थिनौ᳚ । \newline
21. यच् छु॒क्राम॒न्थिनौ॑ शु॒क्राम॒न्थिनौ॒ यद् यच् छु॒क्राम॒न्थिनौ॒ नासि॑का॒ नासि॑का शु॒क्राम॒न्थिनौ॒ यद् यच् छु॒क्राम॒न्थिनौ॒ नासि॑का । \newline
22. शु॒क्राम॒न्थिनौ॒ नासि॑का॒ नासि॑का शु॒क्राम॒न्थिनौ॑ शु॒क्राम॒न्थिनौ॒ नासि॑को त्तरवे॒दि रु॑त्तरवे॒दिर् नासि॑का शु॒क्राम॒न्थिनौ॑ शु॒क्राम॒न्थिनौ॒ नासि॑कोत्तरवे॒दिः । \newline
23. शु॒क्राम॒न्थिना॒विति॑ शु॒क्रा - म॒न्थिनौ᳚ । \newline
24. नासि॑को त्तरवे॒दि रु॑त्तरवे॒दिर् नासि॑का॒ नासि॑कोत्तरवे॒दि र॒भितो॒ ऽभित॑ उत्तरवे॒दिर् नासि॑का॒ नासि॑को त्तरवे॒दि र॒भितः॑ । \newline
25. उ॒त्त॒र॒वे॒दि र॒भितो॒ ऽभित॑ उत्तरवे॒दि रु॑त्तरवे॒दि र॒भितः॑ परि॒क्रम्य॑ परि॒क्र म्या॒भित॑ उत्तरवे॒दि रु॑त्तरवे॒दि र॒भितः॑ परि॒क्रम्य॑ । \newline
26. उ॒त्त॒र॒वे॒दिरित्यु॑त्तर - वे॒दिः । \newline
27. अ॒भितः॑ परि॒क्रम्य॑ परि॒क्रम्या॒ भितो॒ ऽभितः॑ परि॒क्रम्य॑ जुहुतो जुहुतः परि॒क्रम्या॒ भितो॒ ऽभितः॑ परि॒क्रम्य॑ जुहुतः । \newline
28. प॒रि॒क्रम्य॑ जुहुतो जुहुतः परि॒क्रम्य॑ परि॒क्रम्य॑ जुहुत॒ स्तस्मा॒त् तस्मा᳚ज् जुहुतः परि॒क्रम्य॑ परि॒क्रम्य॑ जुहुत॒ स्तस्मा᳚त् । \newline
29. प॒रि॒क्रम्येति॑ परि - क्रम्य॑ । \newline
30. जु॒हु॒त॒ स्तस्मा॒त् तस्मा᳚ज् जुहुतो जुहुत॒ स्तस्मा॑ द॒भितो॒ ऽभित॒ स्तस्मा᳚ज् जुहुतो जुहुत॒ स्तस्मा॑ द॒भितः॑ । \newline
31. तस्मा॑ द॒भितो॒ ऽभित॒ स्तस्मा॒त् तस्मा॑ द॒भितो॒ नासि॑का॒म् नासि॑का म॒भित॒ स्तस्मा॒त् तस्मा॑ द॒भितो॒ नासि॑काम् । \newline
32. अ॒भितो॒ नासि॑का॒म् नासि॑का म॒भितो॒ ऽभितो॒ नासि॑का॒म् चक्षु॑षी॒ चक्षु॑षी॒ नासि॑का म॒भितो॒ ऽभितो॒ नासि॑का॒म् चक्षु॑षी । \newline
33. नासि॑का॒म् चक्षु॑षी॒ चक्षु॑षी॒ नासि॑का॒म् नासि॑का॒म् चक्षु॑षी॒ तस्मा॒त् तस्मा॒च् चक्षु॑षी॒ नासि॑का॒म् नासि॑का॒म् चक्षु॑षी॒ तस्मा᳚त् । \newline
34. चक्षु॑षी॒ तस्मा॒त् तस्मा॒च् चक्षु॑षी॒ चक्षु॑षी॒ तस्मा॒न् नासि॑कया॒ नासि॑कया॒ तस्मा॒च् चक्षु॑षी॒ चक्षु॑षी॒ तस्मा॒न् नासि॑कया । \newline
35. चक्षु॑षी॒ इति॒ चक्षु॑षी । \newline
36. तस्मा॒न् नासि॑कया॒ नासि॑कया॒ तस्मा॒त् तस्मा॒न् नासि॑कया॒ चक्षु॑षी॒ चक्षु॑षी॒ नासि॑कया॒ तस्मा॒त् तस्मा॒न् नासि॑कया॒ चक्षु॑षी । \newline
37. नासि॑कया॒ चक्षु॑षी॒ चक्षु॑षी॒ नासि॑कया॒ नासि॑कया॒ चक्षु॑षी॒ विधृ॑ते॒ विधृ॑ते॒ चक्षु॑षी॒ नासि॑कया॒ नासि॑कया॒ चक्षु॑षी॒ विधृ॑ते । \newline
38. चक्षु॑षी॒ विधृ॑ते॒ विधृ॑ते॒ चक्षु॑षी॒ चक्षु॑षी॒ विधृ॑ते स॒र्वतः॑ स॒र्वतो॒ विधृ॑ते॒ चक्षु॑षी॒ चक्षु॑षी॒ विधृ॑ते स॒र्वतः॑ । \newline
39. चक्षु॑षी॒ इति॒ चक्षु॑षी । \newline
40. विधृ॑ते स॒र्वतः॑ स॒र्वतो॒ विधृ॑ते॒ विधृ॑ते स॒र्वतः॒ परि॒ परि॑ स॒र्वतो॒ विधृ॑ते॒ विधृ॑ते स॒र्वतः॒ परि॑ । \newline
41. विधृ॑ते॒ इति॒ वि - धृ॒ते॒ । \newline
42. स॒र्वतः॒ परि॒ परि॑ स॒र्वतः॑ स॒र्वतः॒ परि॑ क्रामतः क्रामतः॒ परि॑ स॒र्वतः॑ स॒र्वतः॒ परि॑ क्रामतः । \newline
43. परि॑ क्रामतः क्रामतः॒ परि॒ परि॑ क्रामतो॒ रक्ष॑साꣳ॒॒ रक्ष॑साम् क्रामतः॒ परि॒ परि॑ क्रामतो॒ रक्ष॑साम् । \newline
44. क्रा॒म॒तो॒ रक्ष॑साꣳ॒॒ रक्ष॑साम् क्रामतः क्रामतो॒ रक्ष॑सा॒ मप॑हत्या॒ अप॑हत्यै॒ रक्ष॑साम् क्रामतः क्रामतो॒ रक्ष॑सा॒ मप॑हत्यै । \newline
45. रक्ष॑सा॒ मप॑हत्या॒ अप॑हत्यै॒ रक्ष॑साꣳ॒॒ रक्ष॑सा॒ मप॑हत्यै दे॒वा दे॒वा अप॑हत्यै॒ रक्ष॑साꣳ॒॒ रक्ष॑सा॒ मप॑हत्यै दे॒वाः । \newline
46. अप॑हत्यै दे॒वा दे॒वा अप॑हत्या॒ अप॑हत्यै दे॒वा वै वै दे॒वा अप॑हत्या॒ अप॑हत्यै दे॒वा वै । \newline
47. अप॑हत्या॒ इत्यप॑ - ह॒त्यै॒ । \newline
48. दे॒वा वै वै दे॒वा दे॒वा वै या या वै दे॒वा दे॒वा वै याः । \newline
49. वै या या वै वै याः प्राचीः॒ प्राची॒र् या वै वै याः प्राचीः᳚ । \newline
50. याः प्राचीः॒ प्राची॒र् या याः प्राची॒ राहु॑ती॒ राहु॑तीः॒ प्राची॒र् या याः प्राची॒ राहु॑तीः । \newline
51. प्राची॒ राहु॑ती॒ राहु॑तीः॒ प्राचीः॒ प्राची॒ राहु॑ती॒ रजु॑हवु॒ रजु॑हवु॒ राहु॑तीः॒ प्राचीः॒ प्राची॒ राहु॑ती॒ रजु॑हवुः । \newline
52. आहु॑ती॒ रजु॑हवु॒ रजु॑हवु॒ राहु॑ती॒ राहु॑ती॒ रजु॑हवु॒र् ये ये ऽजु॑हवु॒ राहु॑ती॒ राहु॑ती॒ रजु॑हवु॒र् ये । \newline
53. आहु॑ती॒रित्या - हु॒तीः॒ । \newline
54. अजु॑हवु॒र् ये ये ऽजु॑हवु॒ रजु॑हवु॒र् ये पु॒रस्ता᳚त् पु॒रस्ता॒द् ये ऽजु॑हवु॒ रजु॑हवु॒र् ये पु॒रस्ता᳚त् । \newline
55. ये पु॒रस्ता᳚त् पु॒रस्ता॒द् ये ये पु॒रस्ता॒ दसु॑रा॒ असु॑राः पु॒रस्ता॒द् ये ये पु॒रस्ता॒ दसु॑राः । \newline
56. पु॒रस्ता॒ दसु॑रा॒ असु॑राः पु॒रस्ता᳚त् पु॒रस्ता॒ दसु॑रा॒ आस॒न् नास॒न् नसु॑राः पु॒रस्ता᳚त् पु॒रस्ता॒ दसु॑रा॒ आसन्न्॑ । \newline
57. असु॑रा॒ आस॒न् नास॒न् नसु॑रा॒ असु॑रा॒ आस॒न् ताꣳ स्ता नास॒न् नसु॑रा॒ असु॑रा॒ आस॒न् तान् । \newline
58. आस॒न् ताꣳ स्ता नास॒न् नास॒न् ताꣳ स्ताभि॒ स्ताभि॒ स्ता नास॒न् नास॒न् ताꣳ स्ताभिः॑ । \newline
59. ताꣳ स्ताभि॒ स्ताभि॒ स्ताꣳ स्ताꣳ स्ताभिः॒ प्र प्र ताभि॒ स्ताꣳ स्ताꣳ स्ताभिः॒ प्र । \newline
60. ताभिः॒ प्र प्र ताभि॒ स्ताभिः॒ प्राणु॑दन्ता नुदन्त॒ प्र ताभि॒ स्ताभिः॒ प्राणु॑दन्त । \newline
61. प्राणु॑दन्ता नुदन्त॒ प्र प्राणु॑दन्त॒ या या अ॑नुदन्त॒ प्र प्राणु॑दन्त॒ याः । \newline
\pagebreak
\markright{ TS 6.4.10.4  \hfill https://www.vedavms.in \hfill}

\section{ TS 6.4.10.4 }

\textbf{TS 6.4.10.4 } \newline
\textbf{Samhita Paata} \newline

-णु॑दन्त॒ याः प्र॒तीची॒र्ये प॒श्चादसु॑रा॒ आस॒न् ताꣳस्ताभि॒रपा॑नुदन्त॒ प्राची॑र॒न्या आहु॑तयो हू॒यन्ते᳚ प्र॒त्यञ्चौ॑ शु॒क्राम॒न्थिनौ॑ प॒श्चाच्चै॒व पु॒रस्ता᳚च्च॒ यज॑मानो॒ भ्रातृ॑व्या॒न् प्र णु॑दते॒ तस्मा॒त् परा॑चीः प्र॒जाः प्र वी॑यन्ते प्र॒तीची᳚र्जायन्ते शु॒क्राम॒न्थिनौ॒ वा अनु॑ प्र॒जाः प्र जा॑यन्ते॒ऽत्त्रीश्चा॒द्या᳚श्च सु॒वीराः᳚ प्र॒जाः प्र॑ज॒नय॒न् परी॑हि शु॒क्रः शु॒क्रशो॑चिषा- [  ] \newline

\textbf{Pada Paata} \newline

अ॒नु॒द॒न्त॒ । याः । प्र॒तीचीः᳚ । ये । प॒श्चात् । असु॑राः । आसन्न्॑ । तान् । ताभिः॑ । अपेति॑ । अ॒नु॒द॒न्त॒ । प्राचीः᳚ । अ॒न्याः । आहु॑तय॒ इत्या-हु॒त॒यः॒ । हू॒यन्ते᳚ । प्र॒त्यञ्चौ᳚ । शु॒क्राम॒न्थिना॒विति॑ शु॒क्रा - म॒न्थिनौ᳚ । प॒श्चात् । च॒ । ए॒व । पु॒रस्ता᳚त् । च॒ । यज॑मानः । भ्रातृ॑व्यान् । प्रेति॑ । नु॒द॒ते॒ । तस्मा᳚त् । परा॑चीः । प्र॒जा इति॑ प्र - जाः । प्रेति॑ । वी॒य॒न्ते॒ । प्र॒तीचीः᳚ । जा॒य॒न्ते॒ । शु॒क्राम॒न्थिना॒विति॑ शु॒क्रा - म॒न्थिनौ᳚ । वै । अन्विति॑ । प्र॒जा इति॑ प्र - जाः । प्रेति॑ । जा॒य॒न्ते॒ । अ॒त्त्रीः । च॒ । आ॒द्याः᳚ । च॒ । सु॒वीरा॒ इति॑ सु - वीराः᳚ । प्र॒जा इति॑ प्र - जाः । प्र॒ज॒नय॒न्निति॑ प्र - ज॒नयन्न्॑ । परीति॑ । इ॒हि॒ । शु॒क्रः । शु॒क्रशो॑चि॒षेति॑ शु॒क्र - शो॒चि॒षा॒ ।  \newline


\textbf{Krama Paata} \newline

अ॒नु॒द॒न्त॒ याः । याः प्र॒तीचीः᳚ । प्र॒तीची॒र् ये । ये प॒श्चात् । प॒श्चादसु॑राः । असु॑रा॒ आसन्न्॑ । आस॒न् तान् । ताꣳस्ताभिः॑ । ताभि॒रप॑ । अपा॑नुदन्त । अ॒नु॒द॒न्त॒ प्राचीः᳚ । प्राची॑र॒न्याः । अ॒न्या आहु॑तयः । आहु॑तयो हू॒यन्ते᳚ । आहु॑तय॒ इत्या - हु॒त॒यः॒ । हू॒यन्ते᳚ प्र॒त्यञ्चौ᳚ । प्र॒त्यञ्चौ॑ शु॒क्राम॒न्थिनौ᳚ । शु॒क्राम॒न्थिनौ॑ प॒श्चात् । शु॒क्राम॒न्थिना॒विति॑ शु॒क्रा - म॒न्थिनौ᳚ । प॒श्चाच् च॑ । चै॒व । ए॒व पु॒रस्ता᳚त् । पु॒रस्ता᳚च् च । च॒ यज॑मानः । यज॑मानो॒ भ्रातृ॑व्यान् । भ्रातृ॑व्या॒न् प्र । प्र णु॑दते । नु॒द॒ते॒ तस्मा᳚त् । तस्मा॒त् परा॑चीः । परा॑चीः प्र॒जाः । प्र॒जाः प्र । प्र॒जा इति॑ प्र - जाः । प्र वी॑यन्ते । वी॒य॒न्ते॒ प्र॒तीचीः᳚ । प्र॒तीची᳚र् जायन्ते । जा॒य॒न्ते॒ शु॒क्राम॒न्थिनौ᳚ । शु॒क्राम॒न्थिनौ॒ वै । शु॒क्राम॒न्थिना॒विति॑ शु॒क्रा - म॒न्थिनौ᳚ । वा अनु॑ । अनु॑ प्र॒जाः । प्र॒जाः प्र । प्र॒जा इति॑ प्र - जाः । प्र जा॑यन्ते । जा॒य॒न्ते॒ऽत्रीः । अ॒त्रीश्च॑ । चा॒द्याः᳚ । आ॒द्या᳚श्च । च॒ सु॒वीराः᳚ । सु॒वीराः᳚ प्र॒जाः । सु॒वीरा॒ इति॑ सु - वीराः᳚ । प्र॒जाः प्र॑ज॒नयन्न्॑ । प्र॒जा इति॑ प्र - जाः । प्र॒ज॒नय॒न् परि॑ । प्र॒ज॒नय॒न्निति॑ प्र - ज॒नयन्न्॑ । परी॑हि । इ॒हि॒ शु॒क्रः । शु॒क्रः शु॒क्रशो॑चिषा । शु॒क्रशो॑चिषा सुप्र॒जाः । शु॒क्रशो॑चि॒षेति॑ शु॒क्र - शो॒चि॒षा॒ \newline

\textbf{Jatai Paata} \newline

1. अ॒नु॒द॒न्त॒ या या अ॑नुदन्ता नुदन्त॒ याः । \newline
2. याः प्र॒तीचीः᳚ प्र॒तीची॒र् या याः प्र॒तीचीः᳚ । \newline
3. प्र॒तीची॒र् ये ये प्र॒तीचीः᳚ प्र॒तीची॒र् ये । \newline
4. ये प॒श्चात् प॒श्चाद् ये ये प॒श्चात् । \newline
5. प॒श्चा दसु॑रा॒ असु॑राः प॒श्चात् प॒श्चा दसु॑राः । \newline
6. असु॑रा॒ आस॒न् नास॒न् नसु॑रा॒ असु॑रा॒ आसन्न्॑ । \newline
7. आस॒न् ताꣳ स्ता नास॒न् नास॒न् तान् । \newline
8. ताꣳ स्ताभि॒ स्ताभि॒ स्ताꣳ स्ताꣳ स्ताभिः॑ । \newline
9. ताभि॒ रपाप॒ ताभि॒ स्ताभि॒ रप॑ । \newline
10. अपा॑नुदन्ता नुद॒न्ता पापा॑ नुदन्त । \newline
11. अ॒नु॒द॒न्त॒ प्राचीः॒ प्राची॑ रनुदन्ता नुदन्त॒ प्राचीः᳚ । \newline
12. प्राची॑ र॒न्या अ॒न्याः प्राचीः॒ प्राची॑ र॒न्याः । \newline
13. अ॒न्या आहु॑तय॒ आहु॑तयो॒ ऽन्या अ॒न्या आहु॑तयः । \newline
14. आहु॑तयो हू॒यन्ते॑ हू॒यन्त॒ आहु॑तय॒ आहु॑तयो हू॒यन्ते᳚ । \newline
15. आहु॑तय॒ इत्या - हु॒त॒यः॒ । \newline
16. हू॒यन्ते᳚ प्र॒त्यञ्चौ᳚ प्र॒त्यञ्चौ॑ हू॒यन्ते॑ हू॒यन्ते᳚ प्र॒त्यञ्चौ᳚ । \newline
17. प्र॒त्यञ्चौ॑ शु॒क्राम॒न्थिनौ॑ शु॒क्राम॒न्थिनौ᳚ प्र॒त्यञ्चौ᳚ प्र॒त्यञ्चौ॑ शु॒क्राम॒न्थिनौ᳚ । \newline
18. शु॒क्राम॒न्थिनौ॑ प॒श्चात् प॒श्चाच् छु॒क्राम॒न्थिनौ॑ शु॒क्राम॒न्थिनौ॑ प॒श्चात् । \newline
19. शु॒क्राम॒न्थिना॒विति॑ शु॒क्रा - म॒न्थिनौ᳚ । \newline
20. प॒श्चाच् च॑ च प॒श्चात् प॒श्चाच् च॑ । \newline
21. चै॒वैव च॑ चै॒व । \newline
22. ए॒व पु॒रस्ता᳚त् पु॒रस्ता॑ दे॒वैव पु॒रस्ता᳚त् । \newline
23. पु॒रस्ता᳚च् च च पु॒रस्ता᳚त् पु॒रस्ता᳚च् च । \newline
24. च॒ यज॑मानो॒ यज॑मानश्च च॒ यज॑मानः । \newline
25. यज॑मानो॒ भ्रातृ॑व्या॒न् भ्रातृ॑व्या॒न्॒. यज॑मानो॒ यज॑मानो॒ भ्रातृ॑व्यान् । \newline
26. भ्रातृ॑व्या॒न् प्र प्र भ्रातृ॑व्या॒न् भ्रातृ॑व्या॒न् प्र । \newline
27. प्र णु॑दते नुदते॒ प्र प्र णु॑दते । \newline
28. नु॒द॒ते॒ तस्मा॒त् तस्मा᳚न् नुदते नुदते॒ तस्मा᳚त् । \newline
29. तस्मा॒त् परा॑चीः॒ परा॑ची॒ स्तस्मा॒त् तस्मा॒त् परा॑चीः । \newline
30. परा॑चीः प्र॒जाः प्र॒जाः परा॑चीः॒ परा॑चीः प्र॒जाः । \newline
31. प्र॒जाः प्र प्र प्र॒जाः प्र॒जाः प्र । \newline
32. प्र॒जा इति॑ प्र - जाः । \newline
33. प्र वी॑यन्ते वीयन्ते॒ प्र प्र वी॑यन्ते । \newline
34. वी॒य॒न्ते॒ प्र॒तीचीः᳚ प्र॒तीची᳚र् वीयन्ते वीयन्ते प्र॒तीचीः᳚ । \newline
35. प्र॒तीची᳚र् जायन्ते जायन्ते प्र॒तीचीः᳚ प्र॒तीची᳚र् जायन्ते । \newline
36. जा॒य॒न्ते॒ शु॒क्राम॒न्थिनौ॑ शु॒क्राम॒न्थिनौ॑ जायन्ते जायन्ते शु॒क्राम॒न्थिनौ᳚ । \newline
37. शु॒क्राम॒न्थिनौ॒ वै वै शु॒क्राम॒न्थिनौ॑ शु॒क्राम॒न्थिनौ॒ वै । \newline
38. शु॒क्राम॒न्थिना॒विति॑ शु॒क्रा - म॒न्थिनौ᳚ । \newline
39. वा अन् वनु॒ वै वा अनु॑ । \newline
40. अनु॑ प्र॒जाः प्र॒जा अन् वनु॑ प्र॒जाः । \newline
41. प्र॒जाः प्र प्र प्र॒जाः प्र॒जाः प्र । \newline
42. प्र॒जा इति॑ प्र - जाः । \newline
43. प्र जा॑यन्ते जायन्ते॒ प्र प्र जा॑यन्ते । \newline
44. जा॒य॒न्ते॒ ऽत्री र॒त्रीर् जा॑यन्ते जायन्ते॒ ऽत्रीः । \newline
45. अ॒त्री श्च॑ चा॒त्री र॒त्री श्च॑ । \newline
46. चा॒द्या॑ आ॒द्या᳚ श्च चा॒द्याः᳚ । \newline
47. आ॒द्या᳚ श्च चा॒द्या॑ आ॒द्या᳚ श्च । \newline
48. च॒ सु॒वीराः᳚ सु॒वीरा᳚ श्च च सु॒वीराः᳚ । \newline
49. सु॒वीराः᳚ प्र॒जाः प्र॒जाः सु॒वीराः᳚ सु॒वीराः᳚ प्र॒जाः । \newline
50. सु॒वीरा॒ इति॑ सु - वीराः᳚ । \newline
51. प्र॒जाः प्र॑ज॒नय॑न् प्रज॒नय॑न् प्र॒जाः प्र॒जाः प्र॑ज॒नयन्न्॑ । \newline
52. प्र॒जा इति॑ प्र - जाः । \newline
53. प्र॒ज॒नय॒न् परि॒ परि॑ प्रज॒नय॑न् प्रज॒नय॒न् परि॑ । \newline
54. प्र॒ज॒नय॒न्निति॑ प्र - ज॒नयन्न्॑ । \newline
55. परी॑ हीहि॒ परि॒ परी॑हि । \newline
56. इ॒हि॒ शु॒क्रः शु॒क्र इ॑हीहि शु॒क्रः । \newline
57. शु॒क्रः शु॒क्रशो॑चिषा शु॒क्रशो॑चिषा शु॒क्रः शु॒क्रः शु॒क्रशो॑चिषा । \newline
58. शु॒क्रशो॑चिषा सुप्र॒जाः सु॑प्र॒जाः शु॒क्रशो॑चिषा शु॒क्रशो॑चिषा सुप्र॒जाः । \newline
59. शु॒क्रशो॑चि॒षेति॑ शु॒क्र - शो॒चि॒षा॒ । \newline

\textbf{Ghana Paata } \newline

1. अ॒नु॒द॒न्त॒ या या अ॑नुदन्ता नुदन्त॒ याः प्र॒तीचीः᳚ प्र॒तीची॒र् या अ॑नुदन्ता नुदन्त॒ याः प्र॒तीचीः᳚ । \newline
2. याः प्र॒तीचीः᳚ प्र॒तीची॒र् या याः प्र॒तीची॒र् ये ये प्र॒तीची॒र् या याः प्र॒तीची॒र् ये । \newline
3. प्र॒तीची॒र् ये ये प्र॒तीचीः᳚ प्र॒तीची॒र् ये प॒श्चात् प॒श्चाद् ये प्र॒तीचीः᳚ प्र॒तीची॒र् ये प॒श्चात् । \newline
4. ये प॒श्चात् प॒श्चाद् ये ये प॒श्चा दसु॑रा॒ असु॑राः प॒श्चाद् ये ये प॒श्चा दसु॑राः । \newline
5. प॒श्चा दसु॑रा॒ असु॑राः प॒श्चात् प॒श्चा दसु॑रा॒ आस॒न् नास॒न् नसु॑राः प॒श्चात् प॒श्चा दसु॑रा॒ आसन्न्॑ । \newline
6. असु॑रा॒ आस॒न् नास॒न् नसु॑रा॒ असु॑रा॒ आस॒न् ताꣳ स्ता नास॒न् नसु॑रा॒ असु॑रा॒ आस॒न् तान् । \newline
7. आस॒न् ताꣳ स्ता नास॒न् नास॒न् ताꣳ स्ताभि॒ स्ताभि॒ स्ता नास॒न् नास॒न् ताꣳ स्ताभिः॑ । \newline
8. ताꣳ स्ताभि॒ स्ताभि॒ स्ताꣳ स्ताꣳ स्ताभि॒ रपाप॒ ताभि॒ स्ताꣳ स्ताꣳ स्ताभि॒ रप॑ । \newline
9. ताभि॒ रपाप॒ ताभि॒ स्ताभि॒ रपा॑नुदन्ता नुद॒न्ताप॒ ताभि॒ स्ताभि॒ रपा॑नुदन्त । \newline
10. अपा॑नुदन्ता नुद॒न्ता पापा॑ नुदन्त॒ प्राचीः॒ प्राची॑ रनुद॒न्ता पापा॑ नुदन्त॒ प्राचीः᳚ । \newline
11. अ॒नु॒द॒न्त॒ प्राचीः॒ प्राची॑ रनुदन्ता नुदन्त॒ प्राची॑ र॒न्या अ॒न्याः प्राची॑ रनुदन्ता नुदन्त॒ प्राची॑ र॒न्याः । \newline
12. प्राची॑ र॒न्या अ॒न्याः प्राचीः॒ प्राची॑ र॒न्या आहु॑तय॒ आहु॑तयो॒ ऽन्याः प्राचीः॒ प्राची॑ र॒न्या आहु॑तयः । \newline
13. अ॒न्या आहु॑तय॒ आहु॑तयो॒ ऽन्या अ॒न्या आहु॑तयो हू॒यन्ते॑ हू॒यन्त॒ आहु॑तयो॒ ऽन्या अ॒न्या आहु॑तयो हू॒यन्ते᳚ । \newline
14. आहु॑तयो हू॒यन्ते॑ हू॒यन्त॒ आहु॑तय॒ आहु॑तयो हू॒यन्ते᳚ प्र॒त्यञ्चौ᳚ प्र॒त्यञ्चौ॑ हू॒यन्त॒ आहु॑तय॒ आहु॑तयो हू॒यन्ते᳚ प्र॒त्यञ्चौ᳚ । \newline
15. आहु॑तय॒ इत्या - हु॒त॒यः॒ । \newline
16. हू॒यन्ते᳚ प्र॒त्यञ्चौ᳚ प्र॒त्यञ्चौ॑ हू॒यन्ते॑ हू॒यन्ते᳚ प्र॒त्यञ्चौ॑ शु॒क्राम॒न्थिनौ॑ शु॒क्राम॒न्थिनौ᳚ प्र॒त्यञ्चौ॑ हू॒यन्ते॑ हू॒यन्ते᳚ प्र॒त्यञ्चौ॑ शु॒क्राम॒न्थिनौ᳚ । \newline
17. प्र॒त्यञ्चौ॑ शु॒क्राम॒न्थिनौ॑ शु॒क्राम॒न्थिनौ᳚ प्र॒त्यञ्चौ᳚ प्र॒त्यञ्चौ॑ शु॒क्राम॒न्थिनौ॑ प॒श्चात् प॒श्चाच् छु॒क्राम॒न्थिनौ᳚ प्र॒त्यञ्चौ᳚ प्र॒त्यञ्चौ॑ शु॒क्राम॒न्थिनौ॑ प॒श्चात् । \newline
18. शु॒क्राम॒न्थिनौ॑ प॒श्चात् प॒श्चाच् छु॒क्राम॒न्थिनौ॑ शु॒क्राम॒न्थिनौ॑ प॒श्चाच् च॑ च प॒श्चाच् छु॒क्राम॒न्थिनौ॑ शु॒क्राम॒न्थिनौ॑ प॒श्चाच् च॑ । \newline
19. शु॒क्राम॒न्थिना॒विति॑ शु॒क्रा - म॒न्थिनौ᳚ । \newline
20. प॒श्चाच् च॑ च प॒श्चात् प॒श्चाच् चै॒वैव च॑ प॒श्चात् प॒श्चाच् चै॒व । \newline
21. चै॒वैव च॑ चै॒व पु॒रस्ता᳚त् पु॒रस्ता॑ दे॒व च॑ चै॒व पु॒रस्ता᳚त् । \newline
22. ए॒व पु॒रस्ता᳚त् पु॒रस्ता॑ दे॒वैव पु॒रस्ता᳚च् च च पु॒रस्ता॑ दे॒वैव पु॒रस्ता᳚च् च । \newline
23. पु॒रस्ता᳚च् च च पु॒रस्ता᳚त् पु॒रस्ता᳚च् च॒ यज॑मानो॒ यज॑मान श्च पु॒रस्ता᳚त् पु॒रस्ता᳚च् च॒ यज॑मानः । \newline
24. च॒ यज॑मानो॒ यज॑मान श्च च॒ यज॑मानो॒ भ्रातृ॑व्या॒न् भ्रातृ॑व्या॒न्॒. यज॑मान श्च च॒ यज॑मानो॒ भ्रातृ॑व्यान् । \newline
25. यज॑मानो॒ भ्रातृ॑व्या॒न् भ्रातृ॑व्या॒न्॒. यज॑मानो॒ यज॑मानो॒ भ्रातृ॑व्या॒न् प्र प्र भ्रातृ॑व्या॒न्॒. यज॑मानो॒ यज॑मानो॒ भ्रातृ॑व्या॒न् प्र । \newline
26. भ्रातृ॑व्या॒न् प्र प्र भ्रातृ॑व्या॒न् भ्रातृ॑व्या॒न् प्र णु॑दते नुदते॒ प्र भ्रातृ॑व्या॒न् भ्रातृ॑व्या॒न् प्र णु॑दते । \newline
27. प्र णु॑दते नुदते॒ प्र प्र णु॑दते॒ तस्मा॒त् तस्मा᳚न् नुदते॒ प्र प्र णु॑दते॒ तस्मा᳚त् । \newline
28. नु॒द॒ते॒ तस्मा॒त् तस्मा᳚न् नुदते नुदते॒ तस्मा॒त् परा॑चीः॒ परा॑ची॒ स्तस्मा᳚न् नुदते नुदते॒ तस्मा॒त् परा॑चीः । \newline
29. तस्मा॒त् परा॑चीः॒ परा॑ची॒ स्तस्मा॒त् तस्मा॒त् परा॑चीः प्र॒जाः प्र॒जाः परा॑ची॒ स्तस्मा॒त् तस्मा॒त् परा॑चीः प्र॒जाः । \newline
30. परा॑चीः प्र॒जाः प्र॒जाः परा॑चीः॒ परा॑चीः प्र॒जाः प्र प्र प्र॒जाः परा॑चीः॒ परा॑चीः प्र॒जाः प्र । \newline
31. प्र॒जाः प्र प्र प्र॒जाः प्र॒जाः प्र वी॑यन्ते वीयन्ते॒ प्र प्र॒जाः प्र॒जाः प्र वी॑यन्ते । \newline
32. प्र॒जा इति॑ प्र - जाः । \newline
33. प्र वी॑यन्ते वीयन्ते॒ प्र प्र वी॑यन्ते प्र॒तीचीः᳚ प्र॒तीची᳚र् वीयन्ते॒ प्र प्र वी॑यन्ते प्र॒तीचीः᳚ । \newline
34. वी॒य॒न्ते॒ प्र॒तीचीः᳚ प्र॒तीची᳚र् वीयन्ते वीयन्ते प्र॒तीची᳚र् जायन्ते जायन्ते प्र॒तीची᳚र् वीयन्ते वीयन्ते प्र॒तीची᳚र् जायन्ते । \newline
35. प्र॒तीची᳚र् जायन्ते जायन्ते प्र॒तीचीः᳚ प्र॒तीची᳚र् जायन्ते शु॒क्राम॒न्थिनौ॑ शु॒क्राम॒न्थिनौ॑ जायन्ते प्र॒तीचीः᳚ प्र॒तीची᳚र् जायन्ते शु॒क्राम॒न्थिनौ᳚ । \newline
36. जा॒य॒न्ते॒ शु॒क्राम॒न्थिनौ॑ शु॒क्राम॒न्थिनौ॑ जायन्ते जायन्ते शु॒क्राम॒न्थिनौ॒ वै वै शु॒क्राम॒न्थिनौ॑ जायन्ते जायन्ते शु॒क्राम॒न्थिनौ॒ वै । \newline
37. शु॒क्राम॒न्थिनौ॒ वै वै शु॒क्राम॒न्थिनौ॑ शु॒क्राम॒न्थिनौ॒ वा अन् वनु॒ वै शु॒क्राम॒न्थिनौ॑ शु॒क्राम॒न्थिनौ॒ वा अनु॑ । \newline
38. शु॒क्राम॒न्थिना॒विति॑ शु॒क्रा - म॒न्थिनौ᳚ । \newline
39. वा अन् वनु॒ वै वा अनु॑ प्र॒जाः प्र॒जा अनु॒ वै वा अनु॑ प्र॒जाः । \newline
40. अनु॑ प्र॒जाः प्र॒जा अन् वनु॑ प्र॒जाः प्र प्र प्र॒जा अन् वनु॑ प्र॒जाः प्र । \newline
41. प्र॒जाः प्र प्र प्र॒जाः प्र॒जाः प्र जा॑यन्ते जायन्ते॒ प्र प्र॒जाः प्र॒जाः प्र जा॑यन्ते । \newline
42. प्र॒जा इति॑ प्र - जाः । \newline
43. प्र जा॑यन्ते जायन्ते॒ प्र प्र जा॑यन्ते॒ ऽत्री र॒त्रीर् जा॑यन्ते॒ प्र प्र जा॑यन्ते॒ ऽत्रीः । \newline
44. जा॒य॒न्ते॒ ऽत्री र॒त्रीर् जा॑यन्ते जायन्ते॒ ऽत्री श्च॑ चा॒त्रीर् जा॑यन्ते जायन्ते॒ ऽत्री श्च॑ । \newline
45. अ॒त्री श्च॑ चा॒त्री र॒त्री श्चा॒द्या॑ आ॒द्या᳚ श्चा॒त्री र॒त्री श्चा॒द्याः᳚ । \newline
46. चा॒द्या॑ आ॒द्या᳚ श्च चा॒द्या᳚ श्च चा॒द्या᳚ श्च चा॒द्या᳚ श्च । \newline
47. आ॒द्या᳚ श्च चा॒द्या॑ आ॒द्या᳚ श्च सु॒वीराः᳚ सु॒वीरा᳚ श्चा॒द्या॑ आ॒द्या᳚ श्च सु॒वीराः᳚ । \newline
48. च॒ सु॒वीराः᳚ सु॒वीरा᳚ श्च च सु॒वीराः᳚ प्र॒जाः प्र॒जाः सु॒वीरा᳚ श्च च सु॒वीराः᳚ प्र॒जाः । \newline
49. सु॒वीराः᳚ प्र॒जाः प्र॒जाः सु॒वीराः᳚ सु॒वीराः᳚ प्र॒जाः प्र॑ज॒नय॑न् प्रज॒नय॑न् प्र॒जाः सु॒वीराः᳚ सु॒वीराः᳚ प्र॒जाः प्र॑ज॒नयन्न्॑ । \newline
50. सु॒वीरा॒ इति॑ सु - वीराः᳚ । \newline
51. प्र॒जाः प्र॑ज॒नय॑न् प्रज॒नय॑न् प्र॒जाः प्र॒जाः प्र॑ज॒नय॒न् परि॒ परि॑ प्रज॒नय॑न् प्र॒जाः प्र॒जाः प्र॑ज॒नय॒न् परि॑ । \newline
52. प्र॒जा इति॑ प्र - जाः । \newline
53. प्र॒ज॒नय॒न् परि॒ परि॑ प्रज॒नय॑न् प्रज॒नय॒न् परी॑ हीहि॒ परि॑ प्रज॒नय॑न् प्रज॒नय॒न् परी॑हि । \newline
54. प्र॒ज॒नय॒न्निति॑ प्र - ज॒नयन्न्॑ । \newline
55. परी॑ हीहि॒ परि॒ परी॑हि शु॒क्रः शु॒क्र इ॑हि॒ परि॒ परी॑हि शु॒क्रः । \newline
56. इ॒हि॒ शु॒क्रः शु॒क्र इ॑हीहि शु॒क्रः शु॒क्रशो॑चिषा शु॒क्रशो॑चिषा शु॒क्र इ॑हीहि शु॒क्रः शु॒क्रशो॑चिषा । \newline
57. शु॒क्रः शु॒क्रशो॑चिषा शु॒क्रशो॑चिषा शु॒क्रः शु॒क्रः शु॒क्रशो॑चिषा सुप्र॒जाः सु॑प्र॒जाः शु॒क्रशो॑चिषा शु॒क्रः शु॒क्रः शु॒क्रशो॑चिषा सुप्र॒जाः । \newline
58. शु॒क्रशो॑चिषा सुप्र॒जाः सु॑प्र॒जाः शु॒क्रशो॑चिषा शु॒क्रशो॑चिषा सुप्र॒जाः प्र॒जाः प्र॒जाः सु॑प्र॒जाः शु॒क्रशो॑चिषा शु॒क्रशो॑चिषा सुप्र॒जाः प्र॒जाः । \newline
59. शु॒क्रशो॑चि॒षेति॑ शु॒क्र - शो॒चि॒षा॒ । \newline
\pagebreak
\markright{ TS 6.4.10.5  \hfill https://www.vedavms.in \hfill}

\section{ TS 6.4.10.5 }

\textbf{TS 6.4.10.5 } \newline
\textbf{Samhita Paata} \newline

सुप्र॒जाः प्र॒जाः प्र॑ज॒नय॒न् परी॑हि म॒न्थी म॒न्थिशो॑चि॒षेत्या॑है॒ता वै सु॒वीरा॒ या अ॒त्त्रीरे॒ताः सु॑प्र॒जा या आ॒द्या॑ य ए॒वं ॅवेदा॒त्र्य॑स्य प्र॒जा जा॑यते॒ नाऽऽद्या᳚ प्र॒जाप॑ते॒रक्ष्य॑श्वय॒त् तत् परा॑ऽऽ*पत॒त् तद् विक॑ङ्कतं॒ प्रावि॑श॒त् तद् विक॑ङ्कते॒ नार॑मत॒ तद् यवं॒ प्रावि॑श॒त् तद् यवे॑ऽरमत॒ तद् यव॑स्य- [  ] \newline

\textbf{Pada Paata} \newline

सु॒प्र॒जा इति॑ सु - प्र॒जाः । प्र॒जा इति॑ प्र - जाः । प्र॒ज॒नय॒न्निति॑ प्र - ज॒नयन्न्॑ । परीति॑ । इ॒हि॒ । म॒न्थी । म॒न्थिशो॑चि॒षेति॑ म॒न्थि - शो॒चि॒षा॒ । इति॑ । आ॒ह॒ । ए॒ताः । वै । सु॒वीरा॒ इति॑ सु - वीराः᳚ । या । अ॒त्त्रीः । ए॒ताः । सु॒प्र॒जा इति॑ सु - प्र॒जाः । याः । आ॒द्याः᳚ । यः । ए॒वम् । वेद॑ । अ॒त्त्री । अ॒स्य॒ । प्र॒जेति॑ प्र - जा । जा॒य॒ते॒ । न । आ॒द्या᳚ । प्र॒जाप॑ते॒रिति॑ प्र॒जा-प॒तेः॒ । अक्षि॑ । अ॒श्व॒य॒त् । तत् । परेति॑ । अ॒प॒त॒त् । तत् । विक॑ङ्कत॒मिति॒ वि-क॒ङ्क॒त॒म् । प्रेति॑ । अ॒वि॒श॒त् । तत् । विक॑ङ्कत॒ इति॑ वि - क॒ङ्क॒ते॒ । न । अ॒र॒म॒त॒ । तत् । यव᳚म् । प्रेति॑ । अ॒वि॒श॒त् । तत् । यवे᳚ । अ॒र॒म॒त॒ । तत् । यव॑स्य ।  \newline


\textbf{Krama Paata} \newline

सु॒प्र॒जाः प्र॒जाः । सु॒प्र॒जा इति॑ सु - प्र॒जाः । प्र॒जाः प्र॑ज॒नयन्न्॑ । प्र॒जा इति॑ प्र - जाः । प्र॒ज॒नय॒न् परि॑ । प्र॒ज॒नय॒न्निति॑ प्र - ज॒नयन्न्॑ । परी॑हि । इ॒हि॒ म॒न्थी । म॒न्थी म॒न्थिशो॑चिषा । म॒न्थिशो॑चि॒षेति॑ । म॒न्थिशो॑चि॒षेति॑ म॒न्थि - शो॒चि॒षा॒ । इत्या॑ह । आ॒है॒ताः । ए॒ता वै । वै सु॒वीराः᳚ । सु॒वीरा॒ याः । सु॒वीरा॒ इति॑ सु - वीराः᳚ । या अ॒त्रीः । अ॒त्रीरे॒ताः । ए॒ताः सु॑प्र॒जाः । सु॒प्र॒जा याः । सु॒प्र॒जा इति॑ सु - प्र॒जाः । या आ॒द्याः᳚ । आ॒द्या॑ यः । य ए॒वम् । ए॒वम् ॅवेद॑ । वेदा॒त्री । अ॒त्र्य॑स्य । अ॒स्य॒ प्र॒जा । प्र॒जा जा॑यते । प्र॒जेति॑ प्र - जा । जा॒य॒ते॒ न । नाद्या᳚ । आ॒द्या᳚ प्र॒जाप॑तेः । प्र॒जाप॑ते॒रक्षि॑ । प्र॒जाप॑ते॒रिति॑ प्र॒जा - प॒तेः॒ । अक्ष्य॑श्वयत् । अ॒श्व॒य॒त् तत् । तत् परा᳚ । परा॑ऽपतत् । अ॒प॒त॒त् तत् । तद् विक॑ङ्‍कतम् । विक॑ङ्‍कत॒म् प्र । विक॑ङ्‍कत॒मिति॒ वि - क॒ङ्‍क॒त॒म् । प्रावि॑शत् । अ॒वि॒श॒त् तत् । तद् विक॑ङ्‍कते । विक॑ङ्‍कते॒ न । विक॑ङ्‍कत॒ इति॒ वि - क॒ङ्‍क॒ते॒ । नार॑मत । अ॒र॒म॒त॒ तत् । तद् यव᳚म् । यव॒म् प्र । प्रावि॑शत् । अ॒वि॒श॒त् तत् । तद् यवे᳚ । यवे॑ऽरमत । अ॒र॒म॒त॒ तत् । तद् यव॑स्य ( ) । यव॑स्य यव॒त्वम् \newline

\textbf{Jatai Paata} \newline

1. सु॒प्र॒जाः प्र॒जाः प्र॒जाः सु॑प्र॒जाः सु॑प्र॒जाः प्र॒जाः । \newline
2. सु॒प्र॒जा इति॑ सु - प्र॒जाः । \newline
3. प्र॒जाः प्र॑ज॒नय॑न् प्रज॒नय॑न् प्र॒जाः प्र॒जाः प्र॑ज॒नयन्न्॑ । \newline
4. प्र॒जा इति॑ प्र - जाः । \newline
5. प्र॒ज॒नय॒न् परि॒ परि॑ प्रज॒नय॑न् प्रज॒नय॒न् परि॑ । \newline
6. प्र॒ज॒नय॒न्निति॑ प्र - ज॒नयन्न्॑ । \newline
7. परी॑ हीहि॒ परि॒ परी॑हि । \newline
8. इ॒हि॒ म॒न्थी म॒न्थी ही॑हि म॒न्थी । \newline
9. म॒न्थी म॒न्थिशो॑चिषा म॒न्थिशो॑चिषा म॒न्थी म॒न्थी म॒न्थिशो॑चिषा । \newline
10. म॒न्थिशो॑चि॒षेतीति॑ म॒न्थिशो॑चिषा म॒न्थिशो॑चि॒षेति॑ । \newline
11. म॒न्थिशो॑चि॒षेति॑ म॒न्थि - शो॒चि॒षा॒ । \newline
12. इत्या॑हा॒हे तीत्या॑ह । \newline
13. आ॒है॒ता ए॒ता आ॑हा है॒ताः । \newline
14. ए॒ता वै वा ए॒ता ए॒ता वै । \newline
15. वै सु॒वीराः᳚ सु॒वीरा॒ वै वै सु॒वीराः᳚ । \newline
16. सु॒वीरा॒ या याः सु॒वीराः᳚ सु॒वीरा॒ याः । \newline
17. सु॒वीरा॒ इति॑ सु - वीराः᳚ । \newline
18. या अ॒त्री र॒त्रीर् या या अ॒त्रीः । \newline
19. अ॒त्री रे॒ता ए॒ता अ॒त्री र॒त्री रे॒ताः । \newline
20. ए॒ताः सु॑प्र॒जाः सु॑प्र॒जा ए॒ता ए॒ताः सु॑प्र॒जाः । \newline
21. सु॒प्र॒जा या याः सु॑प्र॒जाः सु॑प्र॒जा याः । \newline
22. सु॒प्र॒जा इति॑ सु - प्र॒जाः । \newline
23. या आ॒द्या॑ आ॒द्या॑ या या आ॒द्याः᳚ । \newline
24. आ॒द्या॑ यो य आ॒द्या॑ आ॒द्या॑ यः । \newline
25. य ए॒व मे॒वं ॅयो य ए॒वम् । \newline
26. ए॒वं ॅवेद॒ वेदै॒व मे॒वं ॅवेद॑ । \newline
27. वेदा॒ त्र्य॑त्री वेद॒ वेदा॒त्री । \newline
28. अ॒त्र्य॑ स्यास्या॒ त्र्या᳚(1॒)त्र्य॑स्य । \newline
29. अ॒स्य॒ प्र॒जा प्र॒जा ऽस्या᳚स्य प्र॒जा । \newline
30. प्र॒जा जा॑यते जायते प्र॒जा प्र॒जा जा॑यते । \newline
31. प्र॒जेति॑ प्र - जा । \newline
32. जा॒य॒ते॒ न न जा॑यते जायते॒ न । \newline
33. नाद्या᳚(1॒) ऽऽद्या॑ न नाद्या᳚ । \newline
34. आ॒द्या᳚ प्र॒जाप॑तेः प्र॒जाप॑ते रा॒द्या᳚(1॒) ऽऽद्या᳚ प्र॒जाप॑तेः । \newline
35. प्र॒जाप॑ते॒ रक्ष्यक्षि॑ प्र॒जाप॑तेः प्र॒जाप॑ते॒ रक्षि॑ । \newline
36. प्र॒जाप॑ते॒रिति॑ प्र॒जा - प॒तेः॒ । \newline
37. अक्ष्य॑ श्वय दश्वय॒ दक्ष्यक्ष्य॑ श्वयत् । \newline
38. अ॒श्व॒य॒त् तत् तद॑श्वय दश्वय॒त् तत् । \newline
39. तत् परा॒ परा॒ तत् तत् परा᳚ । \newline
40. परा॑ ऽपत दपत॒त् परा॒ परा॑ ऽपतत् । \newline
41. अ॒प॒त॒त् तत् तद॑पत दपत॒त् तत् । \newline
42. तद् विक॑ङ्कतं॒ ॅविक॑ङ्कत॒म् तत् तद् विक॑ङ्कतम् । \newline
43. विक॑ङ्कत॒म् प्र प्र विक॑ङ्कतं॒ ॅविक॑ङ्कत॒म् प्र । \newline
44. विक॑ङ्कत॒मिति॒ वि - क॒ङ्क॒त॒म् । \newline
45. प्रावि॑श दविश॒त् प्र प्रावि॑शत् । \newline
46. अ॒वि॒श॒त् तत् तद॑विश दविश॒त् तत् । \newline
47. तद् विक॑ङ्कते॒ विक॑ङ्कते॒ तत् तद् विक॑ङ्कते । \newline
48. विक॑ङ्कते॒ न न विक॑ङ्कते॒ विक॑ङ्कते॒ न । \newline
49. विक॑ङ्कत॒ इति॒ वि - क॒ङ्क॒ते॒ । \newline
50. नार॑मता रमत॒ न नार॑मत । \newline
51. अ॒र॒म॒त॒ तत् तद॑रमता रमत॒ तत् । \newline
52. तद् यवं॒ ॅयव॒म् तत् तद् यव᳚म् । \newline
53. यव॒म् प्र प्र यवं॒ ॅयव॒म् प्र । \newline
54. प्रावि॑श दविश॒त् प्र प्रावि॑शत् । \newline
55. अ॒वि॒श॒त् तत् तद॑विश दविश॒त् तत् । \newline
56. तद् यवे॒ यवे॒ तत् तद् यवे᳚ । \newline
57. यवे॑ ऽरमता रमत॒ यवे॒ यवे॑ ऽरमत । \newline
58. अ॒र॒म॒त॒ तत् तद॑रमता रमत॒ तत् । \newline
59. तद् यव॑स्य॒ यव॑स्य॒ तत् तद् यव॑स्य । \newline
60. यव॑स्य यव॒त्वं ॅय॑व॒त्वं ॅयव॑स्य॒ यव॑स्य यव॒त्वम् । \newline

\textbf{Ghana Paata } \newline

1. सु॒प्र॒जाः प्र॒जाः प्र॒जाः सु॑प्र॒जाः सु॑प्र॒जाः प्र॒जाः प्र॑ज॒नय॑न् प्रज॒नय॑न् प्र॒जाः सु॑प्र॒जाः सु॑प्र॒जाः प्र॒जाः प्र॑ज॒नयन्न्॑ । \newline
2. सु॒प्र॒जा इति॑ सु - प्र॒जाः । \newline
3. प्र॒जाः प्र॑ज॒नय॑न् प्रज॒नय॑न् प्र॒जाः प्र॒जाः प्र॑ज॒नय॒न् परि॒ परि॑ प्रज॒नय॑न् प्र॒जाः प्र॒जाः प्र॑ज॒नय॒न् परि॑ । \newline
4. प्र॒जा इति॑ प्र - जाः । \newline
5. प्र॒ज॒नय॒न् परि॒ परि॑ प्रज॒नय॑न् प्रज॒नय॒न् परी॑ हीहि॒ परि॑ प्रज॒नय॑न् प्रज॒नय॒न् परी॑हि । \newline
6. प्र॒ज॒नय॒न्निति॑ प्र - ज॒नयन्न्॑ । \newline
7. परी॑हीहि॒ परि॒ परी॑हि म॒न्थी म॒न्थीहि॒ परि॒ परी॑हि म॒न्थी । \newline
8. इ॒हि॒ म॒न्थी म॒न्थीही॑हि म॒न्थी म॒न्थिशो॑चिषा म॒न्थिशो॑चिषा म॒न्थीही॑हि म॒न्थी म॒न्थिशो॑चिषा । \newline
9. म॒न्थी म॒न्थिशो॑चिषा म॒न्थिशो॑चिषा म॒न्थी म॒न्थी म॒न्थिशो॑चि॒ षेतीति॑ म॒न्थिशो॑चिषा म॒न्थी म॒न्थी म॒न्थिशो॑चि॒षेति॑ । \newline
10. म॒न्थिशो॑चि॒षेतीति॑ म॒न्थिशो॑चिषा म॒न्थिशो॑चि॒षे त्या॑हा॒हेति॑ म॒न्थिशो॑चिषा म॒न्थिशो॑चि॒षे त्या॑ह । \newline
11. म॒न्थिशो॑चि॒षेति॑ म॒न्थि - शो॒चि॒षा॒ । \newline
12. इत्या॑हा॒हे तीत्या॑ है॒ता ए॒ता आ॒हे तीत्या॑ है॒ताः । \newline
13. आ॒है॒ता ए॒ता आ॑हा है॒ता वै वा ए॒ता आ॑हा है॒ता वै । \newline
14. ए॒ता वै वा ए॒ता ए॒ता वै सु॒वीराः᳚ सु॒वीरा॒ वा ए॒ता ए॒ता वै सु॒वीराः᳚ । \newline
15. वै सु॒वीराः᳚ सु॒वीरा॒ वै वै सु॒वीरा॒ या याः सु॒वीरा॒ वै वै सु॒वीरा॒ याः । \newline
16. सु॒वीरा॒ या याः सु॒वीराः᳚ सु॒वीरा॒ या अ॒त्री र॒त्रीर् याः सु॒वीराः᳚ सु॒वीरा॒ या अ॒त्रीः । \newline
17. सु॒वीरा॒ इति॑ सु - वीराः᳚ । \newline
18. या अ॒त्री र॒त्रीर् या या अ॒त्री रे॒ता ए॒ता अ॒त्रीर् या या अ॒त्री रे॒ताः । \newline
19. अ॒त्री रे॒ता ए॒ता अ॒त्री र॒त्री रे॒ताः सु॑प्र॒जाः सु॑प्र॒जा ए॒ता अ॒त्री र॒त्री रे॒ताः सु॑प्र॒जाः । \newline
20. ए॒ताः सु॑प्र॒जाः सु॑प्र॒जा ए॒ता ए॒ताः सु॑प्र॒जा या याः सु॑प्र॒जा ए॒ता ए॒ताः सु॑प्र॒जा याः । \newline
21. सु॒प्र॒जा या याः सु॑प्र॒जाः सु॑प्र॒जा या आ॒द्या॑ आ॒द्या॑ याः सु॑प्र॒जाः सु॑प्र॒जा या आ॒द्याः᳚ । \newline
22. सु॒प्र॒जा इति॑ सु - प्र॒जाः । \newline
23. या आ॒द्या॑ आ॒द्या॑ या या आ॒द्या॑ यो य आ॒द्या॑ या या आ॒द्या॑ यः । \newline
24. आ॒द्या॑ यो य आ॒द्या॑ आ॒द्या॑ य ए॒व मे॒वं ॅय आ॒द्या॑ आ॒द्या॑ य ए॒वम् । \newline
25. य ए॒व मे॒वं ॅयो य ए॒वं ॅवेद॒ वेदै॒वं ॅयो य ए॒वं ॅवेद॑ । \newline
26. ए॒वं ॅवेद॒ वेदै॒व मे॒वं ॅवेदा॒ त्र्य॑त्री वेदै॒व मे॒वं ॅवेदा॒त्री । \newline
27. वेदा॒ त्र्य॑त्री वेद॒ वेदा॒ त्र्य॑स्या स्या॒त्री वेद॒ वेदा॒ त्र्य॑स्य । \newline
28. अ॒त्र्य॑स्या स्या॒त्र्या᳚(1॒)त्र्य॑स्य प्र॒जा प्र॒जा ऽस्या॒त्र्या᳚(1॒)त्र्य॑स्य प्र॒जा । \newline
29. अ॒स्य॒ प्र॒जा प्र॒जा ऽस्या᳚स्य प्र॒जा जा॑यते जायते प्र॒जा ऽस्या᳚स्य प्र॒जा जा॑यते । \newline
30. प्र॒जा जा॑यते जायते प्र॒जा प्र॒जा जा॑यते॒ न न जा॑यते प्र॒जा प्र॒जा जा॑यते॒ न । \newline
31. प्र॒जेति॑ प्र - जा । \newline
32. जा॒य॒ते॒ न न जा॑यते जायते॒ नाद्या᳚(1॒) ऽऽद्या॑ न जा॑यते जायते॒ नाद्या᳚ । \newline
33. नाद्या᳚(1॒) ऽऽद्या॑ न नाद्या᳚ प्र॒जाप॑तेः प्र॒जाप॑ते रा॒द्या॑ न नाद्या᳚ प्र॒जाप॑तेः । \newline
34. आ॒द्या᳚ प्र॒जाप॑तेः प्र॒जाप॑ते रा॒द्या᳚(1॒) ऽऽद्या᳚ प्र॒जाप॑ते॒ रक्ष्यक्षि॑ प्र॒जाप॑ते रा॒द्या᳚(1॒) ऽऽद्या᳚ प्र॒जाप॑ते॒ रक्षि॑ । \newline
35. प्र॒जाप॑ते॒ रक्ष्यक्षि॑ प्र॒जाप॑तेः प्र॒जाप॑ते॒ रक्ष्य॑ श्वय दश्वय॒ दक्षि॑ प्र॒जाप॑तेः प्र॒जाप॑ते॒ रक्ष्य॑ श्वयत् । \newline
36. प्र॒जाप॑ते॒रिति॑ प्र॒जा - प॒तेः॒ । \newline
37. अक्ष्य॑ श्वय दश्वय॒ दक्ष्य क्ष्य॑श्वय॒त् तत् तद॑श्वय॒ दक्ष्य क्ष्य॑ श्वय॒त् तत् । \newline
38. अ॒श्व॒य॒त् तत् तद॑श्वय दश्वय॒त् तत् परा॒ परा॒ तद॑श्वय दश्वय॒त् तत् परा᳚ । \newline
39. तत् परा॒ परा॒ तत् तत् परा॑ ऽपत दपत॒त् परा॒ तत् तत् परा॑ ऽपतत् । \newline
40. परा॑ ऽपत दपत॒त् परा॒ परा॑ ऽपत॒त् तत् तद॑पत॒त् परा॒ परा॑ ऽपत॒त् तत् । \newline
41. अ॒प॒त॒त् तत् तद॑पत दपत॒त् तद् विक॑ङ्कतं॒ ॅविक॑ङ्कत॒म् तद॑पत दपत॒त् तद् विक॑ङ्कतम् । \newline
42. तद् विक॑ङ्कतं॒ ॅविक॑ङ्कत॒म् तत् तद् विक॑ङ्कत॒म् प्र प्र विक॑ङ्कत॒म् तत् तद् विक॑ङ्कत॒म् प्र । \newline
43. विक॑ङ्कत॒म् प्र प्र विक॑ङ्कतं॒ ॅविक॑ङ्कत॒म् प्रावि॑श दविश॒त् प्र विक॑ङ्कतं॒ ॅविक॑ङ्कत॒म् प्रावि॑शत् । \newline
44. विक॑ङ्कत॒मिति॒ वि - क॒ङ्क॒त॒म् । \newline
45. प्रावि॑श दविश॒त् प्र प्रावि॑श॒त् तत् तद॑विश॒त् प्र प्रावि॑श॒त् तत् । \newline
46. अ॒वि॒श॒त् तत् तद॑विश दविश॒त् तद् विक॑ङ्कते॒ विक॑ङ्कते॒ तद॑विश दविश॒त् तद् विक॑ङ्कते । \newline
47. तद् विक॑ङ्कते॒ विक॑ङ्कते॒ तत् तद् विक॑ङ्कते॒ न न विक॑ङ्कते॒ तत् तद् विक॑ङ्कते॒ न । \newline
48. विक॑ङ्कते॒ न न विक॑ङ्कते॒ विक॑ङ्कते॒ नार॑मता रमत॒ न विक॑ङ्कते॒ विक॑ङ्कते॒ नार॑मत । \newline
49. विक॑ङ्कत॒ इति॒ वि - क॒ङ्क॒ते॒ । \newline
50. नार॑मता रमत॒ न नार॑मत॒ तत् तद॑रमत॒ न नार॑मत॒ तत् । \newline
51. अ॒र॒म॒त॒ तत् तद॑रमता रमत॒ तद् यवं॒ ॅयव॒म् तद॑रमता रमत॒ तद् यव᳚म् । \newline
52. तद् यवं॒ ॅयव॒म् तत् तद् यव॒म् प्र प्र यव॒म् तत् तद् यव॒म् प्र । \newline
53. यव॒म् प्र प्र यवं॒ ॅयव॒म् प्रावि॑श दविश॒त् प्र यवं॒ ॅयव॒म् प्रावि॑शत् । \newline
54. प्रावि॑श दविश॒त् प्र प्रावि॑श॒त् तत् तद॑विश॒त् प्र प्रावि॑श॒त् तत् । \newline
55. अ॒वि॒श॒त् तत् तद॑विश दविश॒त् तद् यवे॒ यवे॒ तद॑विश दविश॒त् तद् यवे᳚ । \newline
56. तद् यवे॒ यवे॒ तत् तद् यवे॑ ऽरमता रमत॒ यवे॒ तत् तद् यवे॑ ऽरमत । \newline
57. यवे॑ ऽरमता रमत॒ यवे॒ यवे॑ ऽरमत॒ तत् तद॑रमत॒ यवे॒ यवे॑ ऽरमत॒ तत् । \newline
58. अ॒र॒म॒त॒ तत् तद॑रमता रमत॒ तद् यव॑स्य॒ यव॑स्य॒ तद॑रमता रमत॒ तद् यव॑स्य । \newline
59. तद् यव॑स्य॒ यव॑स्य॒ तत् तद् यव॑स्य यव॒त्वं ॅय॑व॒त्वं ॅयव॑स्य॒ तत् तद् यव॑स्य यव॒त्वम् । \newline
60. यव॑स्य यव॒त्वं ॅय॑व॒त्वं ॅयव॑स्य॒ यव॑स्य यव॒त्वं ॅयद् यद् य॑व॒त्वं ॅयव॑स्य॒ यव॑स्य यव॒त्वं ॅयत् । \newline
\pagebreak
\markright{ TS 6.4.10.6  \hfill https://www.vedavms.in \hfill}

\section{ TS 6.4.10.6 }

\textbf{TS 6.4.10.6 } \newline
\textbf{Samhita Paata} \newline

यव॒त्वं ॅयद् वैक॑ङ्कतं मन्थिपा॒त्रं भव॑ति॒ सक्तु॑भिः श्री॒णाति॑ प्र॒जाप॑तेरे॒व तच्चक्षुः॒ सं भ॑रति ब्रह्मवा॒दिनो॑ वदन्ति॒ कस्मा᳚थ् स॒त्यान्म॑न्थिपा॒त्रꣳ सदो॒ नाश्नु॑त॒ इत्या᳚र्तपा॒त्रꣳ हीति॑ ब्रूया॒द्-यद॑श्नुवी॒तान्धो᳚ऽद्ध्व॒र्युः स्या॒दार्ति॒मार्च्छे॒त् तस्मा॒न्नाश्नु॑ते ॥ \newline

\textbf{Pada Paata} \newline

य॒व॒त्वमिति॑ यव - त्वम् । यत् । वैक॑ङ्कतम् । म॒न्थि॒पा॒त्रमिति॑ मन्थि - पा॒त्रम् । भव॑ति । सक्तु॑भि॒रिति॒ सक्तु॑ - भिः॒ । श्री॒णाति॑ । प्र॒जाप॑ते॒रिति॑ प्र॒जा - प॒तेः॒ । ए॒व । तत् । चक्षुः॑ । समिति॑ । भ॒र॒ति॒ । ब्र॒ह्म॒वा॒दिन॒ इति॑ ब्रह्म - वा॒दिनः॑ । व॒द॒न्ति॒ । कस्मा᳚त् । स॒त्यात् । म॒न्थि॒पा॒त्रमिति॑ मन्थि - पा॒त्रम् । सदः॑ । न । अ॒श्नु॒ते॒ । इति॑ । आ॒र्त॒पा॒त्रमित्या᳚र्त - पा॒त्रम् । हि । इति॑ । ब्रू॒या॒त् । यत् । अ॒श्नु॒वी॒त । अ॒न्धः । अ॒द्ध्व॒र्युः । स्या॒त् । आर्ति᳚म् । एति॑ । ऋ॒च्छे॒त् । तस्मा᳚त् । न । अ॒श्नु॒ते॒ ॥  \newline


\textbf{Krama Paata} \newline

य॒व॒त्वम् ॅयत् । य॒व॒त्वमिति॑ यव - त्वम् । यद् वैक॑ङ्‍कतम् । वैक॑ङ्‍कतम् मन्थिपा॒त्रम् । म॒न्थि॒पा॒त्रम् भव॑ति । म॒न्थि॒पा॒त्रमिति॑ मन्थि - पा॒त्रम् । भव॑ति॒ सक्तु॑भिः । सक्तु॑भिः श्री॒णाति॑ । सक्तु॑भि॒रिति॒ सक्तु॑ - भिः॒ । श्री॒णाति॑ प्र॒जाप॑तेः । प्र॒जाप॑तेरे॒व । प्र॒जाप॑ते॒रिति॑ प्र॒जा - प॒तेः॒ । ए॒व तत् । तच् चक्षुः॑ । चक्षुः॒ सम् । सम् भ॑रति । भ॒र॒ति॒ ब्र॒ह्म॒वा॒दिनः॑ । ब्र॒ह्म॒वा॒दिनो॑ वदन्ति । ब्र॒ह्म॒वा॒दिन॒ इति॑ ब्रह्म - वा॒दिनः॑ । व॒द॒न्ति॒ कस्मा᳚त् । कस्मा᳚थ् स॒त्यात् । स॒त्यान् म॑न्थिपा॒त्रम् । म॒न्थि॒पा॒त्रꣳ सदः॑ । म॒न्थि॒पा॒त्रमिति॑ मन्थि - पा॒त्रम् । सदो॒ न । नाश्ञु॑ते । अ॒श्ञु॒त॒ इति॑ । इत्या᳚र्तपा॒त्रम् । आ॒र्त॒पा॒त्रꣳ हि । आ॒र्त॒पा॒त्रमित्या᳚र्त - पा॒त्रम् । हीति॑ । इति॑ ब्रूयात् । ब्रू॒या॒द् यत् । यद॑श्ञुवी॒त । अ॒श्ञु॒वी॒तान्धः । अ॒न्धो᳚ऽद्ध्व॒र्युः । अ॒द्ध्व॒र्युः स्या᳚त् । स्या॒दार्ति᳚म् । आर्ति॒मा । आर्च्छे᳚त् । ऋ॒च्छे॒त् तस्मा᳚त् । तस्मा॒न् न । नाश्ञु॑ते । अ॒श्ञु॒त॒ इत्य॑श्ञ्नुते । \newline

\textbf{Jatai Paata} \newline

1. य॒व॒त्वं ॅयद् यद् य॑व॒त्वं ॅय॑व॒त्वं ॅयत् । \newline
2. य॒व॒त्वमिति॑ यव - त्वम् । \newline
3. यद् वैक॑ङ्कतं॒ ॅवैक॑ङ्कतं॒ ॅयद् यद् वैक॑ङ्कतम् । \newline
4. वैक॑ङ्कतम् मन्थिपा॒त्रम् म॑न्थिपा॒त्रं ॅवैक॑ङ्कतं॒ ॅवैक॑ङ्कतम् मन्थिपा॒त्रम् । \newline
5. म॒न्थि॒पा॒त्रम् भव॑ति॒ भव॑ति मन्थिपा॒त्रम् म॑न्थिपा॒त्रम् भव॑ति । \newline
6. म॒न्थि॒पा॒त्रमिति॑ मन्थि - पा॒त्रम् । \newline
7. भव॑ति॒ सक्तु॑भिः॒ सक्तु॑भि॒र् भव॑ति॒ भव॑ति॒ सक्तु॑भिः । \newline
8. सक्तु॑भिः श्री॒णाति॑ श्री॒णाति॒ सक्तु॑भिः॒ सक्तु॑भिः श्री॒णाति॑ । \newline
9. सक्तु॑भि॒रिति॒ सक्तु॑ - भिः॒ । \newline
10. श्री॒णाति॑ प्र॒जाप॑तेः प्र॒जाप॑तेः श्री॒णाति॑ श्री॒णाति॑ प्र॒जाप॑तेः । \newline
11. प्र॒जाप॑ते रे॒वैव प्र॒जाप॑तेः प्र॒जाप॑ते रे॒व । \newline
12. प्र॒जाप॑ते॒रिति॑ प्र॒जा - प॒तेः॒ । \newline
13. ए॒व तत् तदे॒ वैव तत् । \newline
14. तच् चक्षु॒ श्चक्षु॒ स्तत् तच् चक्षुः॑ । \newline
15. चक्षुः॒ सꣳ सम् चक्षु॒ श्चक्षुः॒ सम् । \newline
16. सम् भ॑रति भरति॒ सꣳ सम् भ॑रति । \newline
17. भ॒र॒ति॒ ब्र॒ह्म॒वा॒दिनो᳚ ब्रह्मवा॒दिनो॑ भरति भरति ब्रह्मवा॒दिनः॑ । \newline
18. ब्र॒ह्म॒वा॒दिनो॑ वदन्ति वदन्ति ब्रह्मवा॒दिनो᳚ ब्रह्मवा॒दिनो॑ वदन्ति । \newline
19. ब्र॒ह्म॒वा॒दिन॒ इति॑ ब्रह्म - वा॒दिनः॑ । \newline
20. व॒द॒न्ति॒ कस्मा॒त् कस्मा᳚द् वदन्ति वदन्ति॒ कस्मा᳚त् । \newline
21. कस्मा᳚थ् स॒त्याथ् स॒त्यात् कस्मा॒त् कस्मा᳚थ् स॒त्यात् । \newline
22. स॒त्यान् म॑न्थिपा॒त्रम् म॑न्थिपा॒त्रꣳ स॒त्याथ् स॒त्यान् म॑न्थिपा॒त्रम् । \newline
23. म॒न्थि॒पा॒त्रꣳ सदः॒ सदो॑ मन्थिपा॒त्रम् म॑न्थिपा॒त्रꣳ सदः॑ । \newline
24. म॒न्थि॒पा॒त्रमिति॑ मन्थि - पा॒त्रम् । \newline
25. सदो॒ न न सदः॒ सदो॒ न । \newline
26. नाश्ञु॑ते ऽश्ञुते॒ न नाश्ञु॑ते । \newline
27. अ॒श्ञु॒त॒ इती त्य॑श्ञुते ऽश्ञुत॒ इति॑ । \newline
28. इत्या᳚र्तपा॒त्र मा᳚र्तपा॒त्र मिती त्या᳚र्तपा॒त्रम् । \newline
29. आ॒र्त॒पा॒त्रꣳ हि ह्या᳚र्तपा॒त्र मा᳚र्तपा॒त्रꣳ हि । \newline
30. आ॒र्त॒पा॒त्रमित्या᳚र्त - पा॒त्रम् । \newline
31. हीतीति॒ हि हीति॑ । \newline
32. इति॑ ब्रूयाद् ब्रूया॒ दितीति॑ ब्रूयात् । \newline
33. ब्रू॒या॒द् यद् यद् ब्रू॑याद् ब्रूया॒द् यत् । \newline
34. यद॑श्ञुवी॒ता श्ञु॑वी॒त यद् यद॑श्ञुवी॒त । \newline
35. अ॒श्ञु॒वी॒ता न्धो᳚(1॒) ऽन्धो᳚ ऽश्ञुवी॒ता श्ञु॑वी॒ तान्धः । \newline
36. अ॒न्धो᳚ ऽद्ध्व॒र्यु र॑द्ध्व॒र्यु र॒न्धो᳚(1॒) ऽन्धो᳚ ऽद्ध्व॒र्युः । \newline
37. अ॒द्ध्व॒र्युः स्या᳚थ् स्या दद्ध्व॒र्यु र॑द्ध्व॒र्युः स्या᳚त् । \newline
38. स्या॒ दार्ति॒ मार्तिꣳ॑ स्याथ् स्या॒ दार्ति᳚म् । \newline
39. आर्ति॒ मा ऽऽर्ति॒ मार्ति॒ मा । \newline
40. आर्च्छे॑ दृच्छे॒ दार्च्छे᳚त् । \newline
41. ऋ॒च्छे॒त् तस्मा॒त् तस्मा॑ दृच्छे दृच्छे॒त् तस्मा᳚त् । \newline
42. तस्मा॒न् न न तस्मा॒त् तस्मा॒न् न । \newline
43. नाश्ञु॑ते ऽश्ञुते॒ न नाश्ञु॑ते । \newline
44. अ॒श्ञु॒त॒ इत्य॑श्ञ्नुते । \newline

\textbf{Ghana Paata } \newline

1. य॒व॒त्वं ॅयद् यद् य॑व॒त्वं ॅय॑व॒त्वं ॅयद् वैक॑ङ्कतं॒ ॅवैक॑ङ्कतं॒ ॅयद् य॑व॒त्वं ॅय॑व॒त्वं ॅयद् वैक॑ङ्कतम् । \newline
2. य॒व॒त्वमिति॑ यव - त्वम् । \newline
3. यद् वैक॑ङ्कतं॒ ॅवैक॑ङ्कतं॒ ॅयद् यद् वैक॑ङ्कतम् मन्थिपा॒त्रम् म॑न्थिपा॒त्रं ॅवैक॑ङ्कतं॒ ॅयद् यद् वैक॑ङ्कतम् मन्थिपा॒त्रम् । \newline
4. वैक॑ङ्कतम् मन्थिपा॒त्रम् म॑न्थिपा॒त्रं ॅवैक॑ङ्कतं॒ ॅवैक॑ङ्कतम् मन्थिपा॒त्रम् भव॑ति॒ भव॑ति मन्थिपा॒त्रं ॅवैक॑ङ्कतं॒ ॅवैक॑ङ्कतम् मन्थिपा॒त्रम् भव॑ति । \newline
5. म॒न्थि॒पा॒त्रम् भव॑ति॒ भव॑ति मन्थिपा॒त्रम् म॑न्थिपा॒त्रम् भव॑ति॒ सक्तु॑भिः॒ सक्तु॑भि॒र् भव॑ति मन्थिपा॒त्रम् म॑न्थिपा॒त्रम् भव॑ति॒ सक्तु॑भिः । \newline
6. म॒न्थि॒पा॒त्रमिति॑ मन्थि - पा॒त्रम् । \newline
7. भव॑ति॒ सक्तु॑भिः॒ सक्तु॑भि॒र् भव॑ति॒ भव॑ति॒ सक्तु॑भिः श्री॒णाति॑ श्री॒णाति॒ सक्तु॑भि॒र् भव॑ति॒ भव॑ति॒ सक्तु॑भिः श्री॒णाति॑ । \newline
8. सक्तु॑भिः श्री॒णाति॑ श्री॒णाति॒ सक्तु॑भिः॒ सक्तु॑भिः श्री॒णाति॑ प्र॒जाप॑तेः प्र॒जाप॑तेः श्री॒णाति॒ सक्तु॑भिः॒ सक्तु॑भिः श्री॒णाति॑ प्र॒जाप॑तेः । \newline
9. सक्तु॑भि॒रिति॒ सक्तु॑ - भिः॒ । \newline
10. श्री॒णाति॑ प्र॒जाप॑तेः प्र॒जाप॑तेः श्री॒णाति॑ श्री॒णाति॑ प्र॒जाप॑ते रे॒वैव प्र॒जाप॑तेः श्री॒णाति॑ श्री॒णाति॑ प्र॒जाप॑ते रे॒व । \newline
11. प्र॒जाप॑ते रे॒वैव प्र॒जाप॑तेः प्र॒जाप॑ते रे॒व तत् तदे॒व प्र॒जाप॑तेः प्र॒जाप॑ते रे॒व तत् । \newline
12. प्र॒जाप॑ते॒रिति॑ प्र॒जा - प॒तेः॒ । \newline
13. ए॒व तत् तदे॒ वैव तच् चक्षु॒ श्चक्षु॒ स्तदे॒ वैव तच् चक्षुः॑ । \newline
14. तच् चक्षु॒ श्चक्षु॒ स्तत् तच् चक्षुः॒ सꣳ सम् चक्षु॒ स्तत् तच् चक्षुः॒ सम् । \newline
15. चक्षुः॒ सꣳ सम् चक्षु॒ श्चक्षुः॒ सम् भ॑रति भरति॒ सम् चक्षु॒ श्चक्षुः॒ सम् भ॑रति । \newline
16. सम् भ॑रति भरति॒ सꣳ सम् भ॑रति ब्रह्मवा॒दिनो᳚ ब्रह्मवा॒दिनो॑ भरति॒ सꣳ सम् भ॑रति ब्रह्मवा॒दिनः॑ । \newline
17. भ॒र॒ति॒ ब्र॒ह्म॒वा॒दिनो᳚ ब्रह्मवा॒दिनो॑ भरति भरति ब्रह्मवा॒दिनो॑ वदन्ति वदन्ति ब्रह्मवा॒दिनो॑ भरति भरति ब्रह्मवा॒दिनो॑ वदन्ति । \newline
18. ब्र॒ह्म॒वा॒दिनो॑ वदन्ति वदन्ति ब्रह्मवा॒दिनो᳚ ब्रह्मवा॒दिनो॑ वदन्ति॒ कस्मा॒त् कस्मा᳚द् वदन्ति ब्रह्मवा॒दिनो᳚ ब्रह्मवा॒दिनो॑ वदन्ति॒ कस्मा᳚त् । \newline
19. ब्र॒ह्म॒वा॒दिन॒ इति॑ ब्रह्म - वा॒दिनः॑ । \newline
20. व॒द॒न्ति॒ कस्मा॒त् कस्मा᳚द् वदन्ति वदन्ति॒ कस्मा᳚थ् स॒त्याथ् स॒त्यात् कस्मा᳚द् वदन्ति वदन्ति॒ कस्मा᳚थ् स॒त्यात् । \newline
21. कस्मा᳚थ् स॒त्याथ् स॒त्यात् कस्मा॒त् कस्मा᳚थ् स॒त्यान् म॑न्थिपा॒त्रम् म॑न्थिपा॒त्रꣳ स॒त्यात् कस्मा॒त् कस्मा᳚थ् स॒त्यान् म॑न्थिपा॒त्रम् । \newline
22. स॒त्यान् म॑न्थिपा॒त्रम् म॑न्थिपा॒त्रꣳ स॒त्याथ् स॒त्यान् म॑न्थिपा॒त्रꣳ सदः॒ सदो॑ मन्थिपा॒त्रꣳ स॒त्याथ् स॒त्यान् म॑न्थिपा॒त्रꣳ सदः॑ । \newline
23. म॒न्थि॒पा॒त्रꣳ सदः॒ सदो॑ मन्थिपा॒त्रम् म॑न्थिपा॒त्रꣳ सदो॒ न न सदो॑ मन्थिपा॒त्रम् म॑न्थिपा॒त्रꣳ सदो॒ न । \newline
24. म॒न्थि॒पा॒त्रमिति॑ मन्थि - पा॒त्रम् । \newline
25. सदो॒ न न सदः॒ सदो॒ नाश्ञु॑ते ऽश्ञुते॒ न सदः॒ सदो॒ नाश्ञु॑ते । \newline
26. नाश्ञु॑ते ऽश्ञुते॒ न नाश्ञु॑त॒ इतीत्य॑श्ञुते॒ न नाश्ञु॑त॒ इति॑ । \newline
27. अ॒श्ञु॒त॒ इती त्य॑श्ञुते ऽश्ञुत॒ इत्या᳚र्तपा॒त्र मा᳚र्तपा॒त्र मित्य॑श्ञुते ऽश्ञुत॒ इत्या᳚र्तपा॒त्रम् । \newline
28. इत्या᳚र्तपा॒त्र मा᳚र्तपा॒त्र मिती त्या᳚र्तपा॒त्रꣳ हि ह्या᳚र्तपा॒त्र मिती त्या᳚र्तपा॒त्रꣳ हि । \newline
29. आ॒र्त॒पा॒त्रꣳ हि ह्या᳚र्तपा॒त्र मा᳚र्तपा॒त्रꣳ हीतीति॒ ह्या᳚र्तपा॒त्र मा᳚र्तपा॒त्रꣳ हीति॑ । \newline
30. आ॒र्त॒पा॒त्रमित्या᳚र्त - पा॒त्रम् । \newline
31. हीतीति॒ हि हीति॑ ब्रूयाद् ब्रूया॒ दिति॒ हि हीति॑ ब्रूयात् । \newline
32. इति॑ ब्रूयाद् ब्रूया॒ दितीति॑ ब्रूया॒द् यद् यद् ब्रू॑या॒ दितीति॑ ब्रूया॒द् यत् । \newline
33. ब्रू॒या॒द् यद् यद् ब्रू॑याद् ब्रूया॒द् यद॑श्ञुवी॒ता श्ञु॑वी॒त यद् ब्रू॑याद् ब्रूया॒द् यद॑श्ञुवी॒त । \newline
34. यद॑श्ञुवी॒ता श्ञु॑वी॒त यद् यद॑श्ञुवी॒ता न्धो᳚(1॒) ऽन्धो᳚ ऽश्ञुवी॒त यद् यद॑ श्ञुवी॒तान्धः । \newline
35. अ॒श्ञु॒वी॒ तान्धो᳚(1॒) ऽन्धो᳚ ऽश्ञुवी॒ता श्ञु॑वी॒ तान्धो᳚ ऽद्ध्व॒र्यु र॑द्ध्व॒र्यु र॒न्धो᳚ ऽश्ञुवी॒ता श्ञु॑वी॒ तान्धो᳚ ऽद्ध्व॒र्युः । \newline
36. अ॒न्धो᳚ ऽद्ध्व॒र्यु र॑द्ध्व॒र्यु र॒न्धो᳚(1॒) ऽन्धो᳚ ऽद्ध्व॒र्युः स्या᳚थ् स्या दद्ध्व॒र्यु र॒न्धो᳚(1॒) ऽन्धो᳚ ऽद्ध्व॒र्युः स्या᳚त् । \newline
37. अ॒द्ध्व॒र्युः स्या᳚थ् स्या दद्ध्व॒र्यु र॑द्ध्व॒र्युः स्या॒ दार्ति॒ मार्तिꣳ॑ स्या दद्ध्व॒र्यु र॑द्ध्व॒र्युः स्या॒ दार्ति᳚म् । \newline
38. स्या॒ दार्ति॒ मार्तिꣳ॑ स्याथ् स्या॒ दार्ति॒ मा ऽऽर्तिꣳ॑ स्याथ् स्या॒ दार्ति॒ मा । \newline
39. आर्ति॒ मा ऽऽर्ति॒ मार्ति॒ मार्च्छे॑ दृच्छे॒दा ऽऽर्ति॒ मार्ति॒ मार्च्छे᳚त् । \newline
40. आर्च्छे॑ दृच्छे॒ दार्च्छे॒त् तस्मा॒त् तस्मा॑ दृच्छे॒ दार्च्छे॒त् तस्मा᳚त् । \newline
41. ऋ॒च्छे॒त् तस्मा॒त् तस्मा॑ दृच्छे दृच्छे॒त् तस्मा॒न् न न तस्मा॑ दृच्छे दृच्छे॒त् तस्मा॒न् न । \newline
42. तस्मा॒न् न न तस्मा॒त् तस्मा॒न् नाश्ञु॑ते ऽश्ञुते॒ न तस्मा॒त् तस्मा॒न् नाश्ञु॑ते । \newline
43. नाश्ञु॑ते ऽश्ञुते॒ न नाश्ञु॑ते । \newline
44. अ॒श्ञु॒त॒ इत्य॑श्ञ्नुते । \newline
\pagebreak
\markright{ TS 6.4.11.1  \hfill https://www.vedavms.in \hfill}

\section{ TS 6.4.11.1 }

\textbf{TS 6.4.11.1 } \newline
\textbf{Samhita Paata} \newline

दे॒वा वै यद् य॒ज्ञेऽकु॑र्वत॒ तदसु॑रा अकुर्वत॒ ते दे॒वा आ᳚ग्रय॒णाग्रा॒न् ग्रहा॑नपश्य॒न् तान॑गृह्णत॒ ततो॒ वै तेऽग्रं॒ पर्या॑य॒न॒. यस्यै॒वं ॅवि॒दुष॑ आग्रय॒णाग्रा॒ ग्रहा॑ गृ॒ह्यन्तेऽग्र॑मे॒व स॑मा॒नानां॒ पर्ये॑ति रु॒ग्णव॑त्य॒र्चा भ्रातृ॑व्यवतो गृह्णीया॒द्-भ्रातृ॑व्यस्यै॒व रु॒क्त्वाग्रꣳ॑ समा॒नानां॒ पर्ये॑ति॒ ये दे॑वा दि॒व्येका॑दश॒ स्थेत्या॑है॒- [  ] \newline

\textbf{Pada Paata} \newline

दे॒वाः । वै । यत् । य॒ज्ञे । अकु॑र्वत । तत् । असु॑राः । अ॒कु॒र्व॒त॒ । ते । दे॒वाः । आ॒ग्र॒य॒णाग्रा॒नित्या᳚ग्रय॒ण - अ॒ग्रा॒न् । ग्रहान्॑ । अ॒प॒श्य॒न्न् । तान् । अ॒गृ॒ह्ण॒त॒ । ततः॑ । वै । ते । अग्र᳚म् । परीति॑ । आ॒य॒न्न् । यस्य॑ । ए॒वम् । वि॒दुषः॑ । आ॒ग्र॒य॒णाग्रा॒ इत्या᳚ग्रय॒ण-अ॒ग्राः॒ । ग्रहाः᳚ । गृ॒ह्यन्त᳚ । अग्र᳚म् । ए॒व । स॒मा॒नाना᳚म् । परीति॑ । ए॒ति॒ । रु॒ग्णव॒त्येति॑ रु॒ग्ण - व॒त्या॒ । ऋ॒चा । भ्रातृ॑व्यवत॒ इति॒ भ्रातृ॑व्य-व॒तः॒ । गृ॒ह्णी॒या॒त् । भ्रातृ॑व्यस्य । ए॒व । रु॒क्त्वा । अग्र᳚म् । स॒मा॒नाना᳚म् । परीति॑ । ए॒ति॒ । ये । दे॒वाः॒ । दि॒वि । एका॑दश । स्थ । इति॑ । आ॒ह॒ ।  \newline


\textbf{Krama Paata} \newline

दे॒वा वै । वै यत् । यद् य॒ज्ञे । य॒ज्ञेऽकु॑र्वत । अकु॑र्वत॒ तत् । तदसु॑राः । असु॑रा अकुर्वत । अ॒कु॒र्व॒त॒ ते । 
ते दे॒वाः । दे॒वा आ᳚ग्रय॒णाग्रान् । आ॒ग्र॒य॒णाग्रा॒न् ग्रहान्॑ । आ॒ग्र॒य॒णाग्रा॒नित्या᳚ग्रय॒ण - अ॒ग्रा॒न्॒ । ग्रहा॑नपश्यन्न् । अ॒प॒श्य॒न् तान् । तान॑गृह्णत । अ॒गृ॒ह्ण॒त॒ ततः॑ । ततो॒ वै । वै ते । तेऽग्र᳚म् । अग्र॒म् परि॑ । पर्या॑यन्न् । आ॒य॒न्॒. यस्य॑ । यस्यै॒वम् । ए॒वम् ॅवि॒दुषः॑ । वि॒दुष॑ आग्रय॒णाग्राः᳚ । आ॒ग्र॒य॒णाग्रा॒ ग्रहाः᳚ । आ॒ग्र॒य॒णाग्रा॒ इत्या᳚ग्रय॒ण - अ॒ग्राः॒ । गृहा॑ गृ॒ह्यन्ते᳚ । गृ॒ह्यन्तेऽग्र᳚म् । अग्र॑मे॒व । ए॒व स॑मा॒नाना᳚म् । स॒मा॒नाना॒म् परि॑ । पर्ये॑ति । ए॒ति॒ रु॒ग्णव॑त्या । रु॒ग्णव॑त्य॒र्चा । रु॒ग्णव॒त्येति॑ रु॒ग्ण - व॒त्या॒ । ऋ॒चा भ्रातृ॑व्यवतः । भ्रातृ॑व्यवतो गृह्णीयात् । भ्रातृ॑व्यवत॒ इति॒ भ्रातृ॑व्य - व॒तः॒ । गृ॒ह्णी॒या॒द् भ्रातृ॑व्यस्य । भ्रातृ॑व्यस्यै॒व । ए॒व रु॒क्त्वा । रु॒क्त्वाऽग्र᳚म् । अग्रꣳ॑ समा॒नाना᳚म् । स॒मा॒नाना॒म् परि॑ । पर्ये॑ति । ए॒ति॒ ये । ये दे॑वाः । दे॒वा॒ दि॒वि । दि॒व्येका॑दश । एका॑दश॒ स्थ । स्थेति॑ । इत्या॑ह । आ॒है॒ताव॑तीः \newline

\textbf{Jatai Paata} \newline

1. दे॒वा वै वै दे॒वा दे॒वा वै । \newline
2. वै यद् यद् वै वै यत् । \newline
3. यद् य॒ज्ञे य॒ज्ञे यद् यद् य॒ज्ञे । \newline
4. य॒ज्ञे ऽकु॑र्व॒ता कु॑र्वत य॒ज्ञे य॒ज्ञे ऽकु॑र्वत । \newline
5. अकु॑र्वत॒ तत् तदकु॑र्व॒ता कु॑र्वत॒ तत् । \newline
6. तदसु॑रा॒ असु॑रा॒ स्तत् तदसु॑राः । \newline
7. असु॑रा अकुर्वता कुर्व॒ता सु॑रा॒ असु॑रा अकुर्वत । \newline
8. अ॒कु॒र्व॒त॒ ते ते॑ ऽकुर्वता कुर्वत॒ ते । \newline
9. ते दे॒वा दे॒वा स्ते ते दे॒वाः । \newline
10. दे॒वा आ᳚ग्रय॒णाग्रा॑ नाग्रय॒णाग्रा᳚न् दे॒वा दे॒वा आ᳚ग्रय॒णाग्रान्॑ । \newline
11. आ॒ग्र॒य॒णाग्रा॒न् ग्रहा॒न् ग्रहा॑ नाग्रय॒णाग्रा॑ नाग्रय॒णाग्रा॒न् ग्रहान्॑ । \newline
12. आ॒ग्र॒य॒णाग्रा॒नित्या᳚ग्रय॒ण - अ॒ग्रा॒न् । \newline
13. ग्रहा॑ नपश्यन् नपश्य॒न् ग्रहा॒न् ग्रहा॑ नपश्यन्न् । \newline
14. अ॒प॒श्य॒न् ताꣳ स्तान॑ पश्यन्न पश्य॒न् तान् । \newline
15. तान॑ गृह्णता गृह्णत॒ ताꣳ स्तान॑ गृह्णत । \newline
16. अ॒गृ॒ह्ण॒त॒ तत॒ स्ततो॑ ऽगृह्णता गृह्णत॒ ततः॑ । \newline
17. ततो॒ वै वै तत॒ स्ततो॒ वै । \newline
18. वै ते ते वै वै ते । \newline
19. ते ऽग्र॒ मग्र॒म् ते ते ऽग्र᳚म् । \newline
20. अग्र॒म् परि॒ पर्यग्र॒ मग्र॒म् परि॑ । \newline
21. पर्या॑यन् नाय॒न् परि॒ पर्या॑यन्न् । \newline
22. आ॒य॒न्॒. यस्य॒ यस्या॑यन् नाय॒न्॒. यस्य॑ । \newline
23. यस्यै॒व मे॒वं ॅयस्य॒ यस्यै॒वम् । \newline
24. ए॒वं ॅवि॒दुषो॑ वि॒दुष॑ ए॒व मे॒वं ॅवि॒दुषः॑ । \newline
25. वि॒दुष॑ आग्रय॒णाग्रा॑ आग्रय॒णाग्रा॑ वि॒दुषो॑ वि॒दुष॑ आग्रय॒णाग्राः᳚ । \newline
26. आ॒ग्र॒य॒णाग्रा॒ ग्रहा॒ ग्रहा॑ आग्रय॒णाग्रा॑ आग्रय॒णाग्रा॒ ग्रहाः᳚ । \newline
27. आ॒ग्र॒य॒णाग्रा॒ इत्या᳚ग्रय॒ण - अ॒ग्राः॒ । \newline
28. ग्रहा॑ गृ॒ह्यन्ते॑ गृ॒ह्यन्ते॒ ग्रहा॒ ग्रहा॑ गृ॒ह्यन्ते᳚ । \newline
29. गृ॒ह्यन्ते ऽग्र॒ मग्र॑म् गृ॒ह्यन्ते॑ गृ॒ह्यन्ते ऽग्र᳚म् । \newline
30. अग्र॑ मे॒वै वाग्र॒ मग्र॑ मे॒व । \newline
31. ए॒व स॑मा॒नानाꣳ॑ समा॒नाना॑ मे॒वैव स॑मा॒नाना᳚म् । \newline
32. स॒मा॒नाना॒म् परि॒ परि॑ समा॒नानाꣳ॑ समा॒नाना॒म् परि॑ । \newline
33. पर्ये᳚ त्येति॒ परि॒ पर्ये॑ति । \newline
34. ए॒ति॒ रु॒ग्णव॑त्या रु॒ग्णव॑ त्यैत्येति रु॒ग्णव॑त्या । \newline
35. रु॒ग्णव॑त्य॒ र्‌च र्‌चा रु॒ग्णव॑त्या रु॒ग्णव॑त्य॒ र्‌चा । \newline
36. रु॒ग्णव॒त्येति॑ रु॒ग्ण - व॒त्या॒ । \newline
37. ऋ॒चा भ्रातृ॑व्यवतो॒ भ्रातृ॑व्यवत ऋ॒च र्‌चा भ्रातृ॑व्यवतः । \newline
38. भ्रातृ॑व्यवतो गृह्णीयाद् गृह्णीया॒द् भ्रातृ॑व्यवतो॒ भ्रातृ॑व्यवतो गृह्णीयात् । \newline
39. भ्रातृ॑व्यवत॒ इति॒ भ्रातृ॑व्य - व॒तः॒ । \newline
40. गृ॒ह्णी॒या॒द् भ्रातृ॑व्यस्य॒ भ्रातृ॑व्यस्य गृह्णीयाद् गृह्णीया॒द् भ्रातृ॑व्यस्य । \newline
41. भ्रातृ॑व्य स्यै॒वैव भ्रातृ॑व्यस्य॒ भ्रातृ॑व्य स्यै॒व । \newline
42. ए॒व रु॒क्त्वा रु॒क्त्वै वैव रु॒क्त्वा । \newline
43. रु॒क्त्वा ऽग्र॒ मग्रꣳ॑ रु॒क्त्वा रु॒क्त्वा ऽग्र᳚म् । \newline
44. अग्रꣳ॑ समा॒नानाꣳ॑ समा॒नाना॒ मग्र॒ मग्रꣳ॑ समा॒नाना᳚म् । \newline
45. स॒मा॒नाना॒म् परि॒ परि॑ समा॒नानाꣳ॑ समा॒नाना॒म् परि॑ । \newline
46. पर्ये᳚ त्येति॒ परि॒ पर्ये॑ति । \newline
47. ए॒ति॒ ये य ए᳚त्येति॒ ये । \newline
48. ये दे॑वा देवा॒ ये ये दे॑वाः । \newline
49. दे॒वा॒ दि॒वि दि॒वि दे॑वा देवा दि॒वि । \newline
50. दि॒व्येका॑द॒ शैका॑दश दि॒वि दि॒व्येका॑दश । \newline
51. एका॑दश॒ स्थ स्थैका॑द॒ शैका॑दश॒ स्थ । \newline
52. स्थेतीति॒ स्थ स्थेति॑ । \newline
53. इत्या॑हा॒हे तीत्या॑ह । \newline
54. आ॒है॒ताव॑ती रे॒ताव॑ती राहा है॒ताव॑तीः । \newline

\textbf{Ghana Paata } \newline

1. दे॒वा वै वै दे॒वा दे॒वा वै यद् यद् वै दे॒वा दे॒वा वै यत् । \newline
2. वै यद् यद् वै वै यद् य॒ज्ञे य॒ज्ञे यद् वै वै यद् य॒ज्ञे । \newline
3. यद् य॒ज्ञे य॒ज्ञे यद् यद् य॒ज्ञे ऽकु॑र्व॒ता कु॑र्वत य॒ज्ञे यद् यद् य॒ज्ञे ऽकु॑र्वत । \newline
4. य॒ज्ञे ऽकु॑र्व॒ता कु॑र्वत य॒ज्ञे य॒ज्ञे ऽकु॑र्वत॒ तत् तदकु॑र्वत य॒ज्ञे य॒ज्ञे ऽकु॑र्वत॒ तत् । \newline
5. अकु॑र्वत॒ तत् तदकु॑र्व॒ता कु॑र्वत॒ तदसु॑रा॒ असु॑रा॒ स्त दकु॑र्व॒ता कु॑र्वत॒ तदसु॑राः । \newline
6. तदसु॑रा॒ असु॑रा॒ स्तत् तदसु॑रा अकुर्वता कुर्व॒ता सु॑रा॒ स्तत् तदसु॑रा अकुर्वत । \newline
7. असु॑रा अकुर्वता कुर्व॒ता सु॑रा॒ असु॑रा अकुर्वत॒ ते ते॑ ऽकुर्व॒ता सु॑रा॒ असु॑रा अकुर्वत॒ ते । \newline
8. अ॒कु॒र्व॒त॒ ते ते॑ ऽकुर्वता कुर्वत॒ ते दे॒वा दे॒वा स्ते॑ ऽकुर्वता कुर्वत॒ ते दे॒वाः । \newline
9. ते दे॒वा दे॒वा स्ते ते दे॒वा आ᳚ग्रय॒णाग्रा॑ नाग्रय॒णाग्रा᳚न् दे॒वा स्ते ते दे॒वा आ᳚ग्रय॒णाग्रान्॑ । \newline
10. दे॒वा आ᳚ग्रय॒णाग्रा॑ नाग्रय॒णाग्रा᳚न् दे॒वा दे॒वा आ᳚ग्रय॒णाग्रा॒न् ग्रहा॒न् ग्रहा॑ नाग्रय॒णाग्रा᳚न् दे॒वा दे॒वा आ᳚ग्रय॒णाग्रा॒न् ग्रहान्॑ । \newline
11. आ॒ग्र॒य॒णाग्रा॒न् ग्रहा॒न् ग्रहा॑ नाग्रय॒णाग्रा॑ नाग्रय॒णाग्रा॒न् ग्रहा॑ नपश्यन् नपश्य॒न् ग्रहा॑ नाग्रय॒णाग्रा॑ नाग्रय॒णाग्रा॒न् ग्रहा॑ नपश्यन्न् । \newline
12. आ॒ग्र॒य॒णाग्रा॒नित्या᳚ग्रय॒ण - अ॒ग्रा॒न् । \newline
13. ग्रहा॑ नपश्यन् नपश्य॒न् ग्रहा॒न् ग्रहा॑ नपश्य॒न् ताꣳ स्तान॑ पश्य॒न् ग्रहा॒न् ग्रहा॑ नपश्य॒न् तान् । \newline
14. अ॒प॒श्य॒न् ताꣳ स्तान॑ पश्यन् नपश्य॒न् तान॑ गृह्णता गृह्णत॒ तान॑ पश्यन् नपश्य॒न् तान॑ गृह्णत । \newline
15. तान॑गृह्णता गृह्णत॒ ताꣳ स्तान॑ गृह्णत॒ तत॒ स्ततो॑ ऽगृह्णत॒ ताꣳ स्तान॑ गृह्णत॒ ततः॑ । \newline
16. अ॒गृ॒ह्ण॒त॒ तत॒ स्ततो॑ ऽगृह्णता गृह्णत॒ ततो॒ वै वै ततो॑ ऽगृह्णता गृह्णत॒ ततो॒ वै । \newline
17. ततो॒ वै वै तत॒ स्ततो॒ वै ते ते वै तत॒ स्ततो॒ वै ते । \newline
18. वै ते ते वै वै ते ऽग्र॒ मग्र॒म् ते वै वै ते ऽग्र᳚म् । \newline
19. ते ऽग्र॒ मग्र॒म् ते ते ऽग्र॒म् परि॒ पर्यग्र॒म् ते ते ऽग्र॒म् परि॑ । \newline
20. अग्र॒म् परि॒ पर्यग्र॒ मग्र॒म् पर्या॑यन् नाय॒न् पर्यग्र॒ मग्र॒म् पर्या॑यन्न् । \newline
21. पर्या॑यन् नाय॒न् परि॒ पर्या॑य॒न्॒. यस्य॒ यस्या॑य॒न् परि॒ पर्या॑य॒न्॒. यस्य॑ । \newline
22. आ॒य॒न्॒. यस्य॒ यस्या॑यन् नाय॒न्॒. यस्यै॒व मे॒वं ॅयस्या॑यन् नाय॒न्॒. यस्यै॒वम् । \newline
23. यस्यै॒व मे॒वं ॅयस्य॒ यस्यै॒वं ॅवि॒दुषो॑ वि॒दुष॑ ए॒वं ॅयस्य॒ यस्यै॒वं ॅवि॒दुषः॑ । \newline
24. ए॒वं ॅवि॒दुषो॑ वि॒दुष॑ ए॒व मे॒वं ॅवि॒दुष॑ आग्रय॒णाग्रा॑ आग्रय॒णाग्रा॑ वि॒दुष॑ ए॒व मे॒वं ॅवि॒दुष॑ आग्रय॒णाग्राः᳚ । \newline
25. वि॒दुष॑ आग्रय॒णाग्रा॑ आग्रय॒णाग्रा॑ वि॒दुषो॑ वि॒दुष॑ आग्रय॒णाग्रा॒ ग्रहा॒ ग्रहा॑ आग्रय॒णाग्रा॑ वि॒दुषो॑ वि॒दुष॑ आग्रय॒णाग्रा॒ ग्रहाः᳚ । \newline
26. आ॒ग्र॒य॒णाग्रा॒ ग्रहा॒ ग्रहा॑ आग्रय॒णाग्रा॑ आग्रय॒णाग्रा॒ ग्रहा॑ गृ॒ह्यन्ते॑ गृ॒ह्यन्ते॒ ग्रहा॑ आग्रय॒णाग्रा॑ आग्रय॒णाग्रा॒ ग्रहा॑ गृ॒ह्यन्ते᳚ । \newline
27. आ॒ग्र॒य॒णाग्रा॒ इत्या᳚ग्रय॒ण - अ॒ग्राः॒ । \newline
28. ग्रहा॑ गृ॒ह्यन्ते॑ गृ॒ह्यन्ते॒ ग्रहा॒ ग्रहा॑ गृ॒ह्यन्ते ऽग्र॒ मग्र॑म् गृ॒ह्यन्ते॒ ग्रहा॒ ग्रहा॑ गृ॒ह्यन्ते ऽग्र᳚म् । \newline
29. गृ॒ह्यन्ते ऽग्र॒ मग्र॑म् गृ॒ह्यन्ते॑ गृ॒ह्यन्ते ऽग्र॑ मे॒वै वाग्र॑म् गृ॒ह्यन्ते॑ गृ॒ह्यन्ते ऽग्र॑ मे॒व । \newline
30. अग्र॑ मे॒वै वाग्र॒ मग्र॑ मे॒व स॑मा॒नानाꣳ॑ समा॒नाना॑ मे॒वाग्र॒ मग्र॑ मे॒व स॑मा॒नाना᳚म् । \newline
31. ए॒व स॑मा॒नानाꣳ॑ समा॒नाना॑ मे॒वैव स॑मा॒नाना॒म् परि॒ परि॑ समा॒नाना॑ मे॒वैव स॑मा॒नाना॒म् परि॑ । \newline
32. स॒मा॒नाना॒म् परि॒ परि॑ समा॒नानाꣳ॑ समा॒नाना॒म् पर्ये᳚ त्येति॒ परि॑ समा॒नानाꣳ॑ समा॒नाना॒म् पर्ये॑ति । \newline
33. पर्ये᳚ त्येति॒ परि॒ पर्ये॑ति रु॒ग्णव॑त्या रु॒ग्णव॑ त्यैति॒ परि॒ पर्ये॑ति रु॒ग्णव॑त्या । \newline
34. ए॒ति॒ रु॒ग्णव॑त्या रु॒ग्णव॑ त्यैत्येति रु॒ग्णव॑त्य॒ र्‌च र्‌चा रु॒ग्णव॑ त्यैत्येति रु॒ग्णव॑त्य॒ र्‌चा । \newline
35. रु॒ग्णव॑त्य॒ र्‌च र्‌चा रु॒ग्णव॑त्या रु॒ग्णव॑त्य॒ र्‌चा भ्रातृ॑व्यवतो॒ भ्रातृ॑व्यवत ऋ॒चा रु॒ग्णव॑त्या रु॒ग्णव॑त्य॒ र्‌चा भ्रातृ॑व्यवतः । \newline
36. रु॒ग्णव॒त्येति॑ रु॒ग्ण - व॒त्या॒ । \newline
37. ऋ॒चा भ्रातृ॑व्यवतो॒ भ्रातृ॑व्यवत ऋ॒च र्‌चा भ्रातृ॑व्यवतो गृह्णीयाद् गृह्णीया॒द् भ्रातृ॑व्यवत ऋ॒च र्‌चा भ्रातृ॑व्यवतो गृह्णीयात् । \newline
38. भ्रातृ॑व्यवतो गृह्णीयाद् गृह्णीया॒द् भ्रातृ॑व्यवतो॒ भ्रातृ॑व्यवतो गृह्णीया॒द् भ्रातृ॑व्यस्य॒ भ्रातृ॑व्यस्य गृह्णीया॒द् भ्रातृ॑व्यवतो॒ भ्रातृ॑व्यवतो गृह्णीया॒द् भ्रातृ॑व्यस्य । \newline
39. भ्रातृ॑व्यवत॒ इति॒ भ्रातृ॑व्य - व॒तः॒ । \newline
40. गृ॒ह्णी॒या॒द् भ्रातृ॑व्यस्य॒ भ्रातृ॑व्यस्य गृह्णीयाद् गृह्णीया॒द् भ्रातृ॑व्य स्यै॒वैव भ्रातृ॑व्यस्य गृह्णीयाद् गृह्णीया॒द् भ्रातृ॑व्य स्यै॒व । \newline
41. भ्रातृ॑व्य स्यै॒वैव भ्रातृ॑व्यस्य॒ भ्रातृ॑व्य स्यै॒व रु॒क्त्वा रु॒क्त्वैव भ्रातृ॑व्यस्य॒ भ्रातृ॑व्य स्यै॒व रु॒क्त्वा । \newline
42. ए॒व रु॒क्त्वा रु॒क्त्वै वैव रु॒क्त्वा ऽग्र॒ मग्रꣳ॑ रु॒क्त्वै वैव रु॒क्त्वा ऽग्र᳚म् । \newline
43. रु॒क्त्वा ऽग्र॒ मग्रꣳ॑ रु॒क्त्वा रु॒क्त्वा ऽग्रꣳ॑ समा॒नानाꣳ॑ समा॒नाना॒ मग्रꣳ॑ रु॒क्त्वा रु॒क्त्वा ऽग्रꣳ॑ समा॒नाना᳚म् । \newline
44. अग्रꣳ॑ समा॒नानाꣳ॑ समा॒नाना॒ मग्र॒ मग्रꣳ॑ समा॒नाना॒म् परि॒ परि॑ समा॒नाना॒ मग्र॒ मग्रꣳ॑ समा॒नाना॒म् परि॑ । \newline
45. स॒मा॒नाना॒म् परि॒ परि॑ समा॒नानाꣳ॑ समा॒नाना॒म् पर्ये᳚ त्येति॒ परि॑ समा॒नानाꣳ॑ समा॒नाना॒म् पर्ये॑ति । \newline
46. पर्ये᳚ त्येति॒ परि॒ पर्ये॑ति॒ ये य ए॑ति॒ परि॒ पर्ये॑ति॒ ये । \newline
47. ए॒ति॒ ये य ए᳚त्येति॒ ये दे॑वा देवा॒ य ए᳚त्येति॒ ये दे॑वाः । \newline
48. ये दे॑वा देवा॒ ये ये दे॑वा दि॒वि दि॒वि दे॑वा॒ ये ये दे॑वा दि॒वि । \newline
49. दे॒वा॒ दि॒वि दि॒वि दे॑वा देवा दि॒व्येका॑द॒ शैका॑दश दि॒वि दे॑वा देवा दि॒व्येका॑दश । \newline
50. दि॒व्येका॑द॒ शैका॑दश दि॒वि दि॒व्येका॑दश॒ स्थ स्थैका॑दश दि॒वि दि॒व्येका॑दश॒ स्थ । \newline
51. एका॑दश॒ स्थ स्थैका॑द॒ शैका॑दश॒ स्थेतीति॒ स्थैका॑द॒ शैका॑दश॒ स्थेति॑ । \newline
52. स्थेतीति॒ स्थ स्थे त्या॑हा॒ हेति॒ स्थ स्थेत्या॑ह । \newline
53. इत्या॑हा॒हे तीत्या॑ है॒ताव॑ती रे॒ताव॑ तीरा॒हे तीत्या॑ है॒ताव॑तीः । \newline
54. आ॒है॒ताव॑ती रे॒ताव॑ती राहा है॒ताव॑ती॒र् वै वा ए॒ताव॑ती राहा है॒ताव॑ती॒र् वै । \newline
\pagebreak
\markright{ TS 6.4.11.2  \hfill https://www.vedavms.in \hfill}

\section{ TS 6.4.11.2 }

\textbf{TS 6.4.11.2 } \newline
\textbf{Samhita Paata} \newline

-ताव॑ती॒र्वै दे॒वता॒स्ताभ्य॑ ए॒वैनꣳ॒॒ सर्वा᳚भ्यो गृह्णात्ये॒ष ते॒ योनि॒ र्विश्वे᳚भ्यस्त्वा दे॒वेभ्य॒ इत्या॑ह वैश्वदे॒वो ह्ये॑ष दे॒वत॑या॒ वाग्वै दे॒वेभ्यो-ऽपा᳚क्रामद्-य॒ज्ञायाति॑ष्ठमाना॒ ते दे॒वा वा॒च्यप॑क्रान्तायां तू॒ष्णीं ग्रहा॑नगृह्णत॒ साम॑न्यत॒ वाग॒न्तर्य॑न्ति॒ वै मेति॒ साऽऽग्र॑य॒णं प्रत्याग॑च्छ॒त् तदा᳚ग्रय॒णस्या᳚ऽऽग्रयण॒त्वं- [  ] \newline

\textbf{Pada Paata} \newline

ए॒ताव॑तीः । वै । दे॒वताः᳚ । ताभ्यः॑ । ए॒व । ए॒न॒म् । सर्वा᳚भ्यः । गृ॒ह्णा॒ति॒ । ए॒षः । ते॒ । योनिः॑ । विश्वे᳚भ्यः । त्वा॒ । दे॒वेभ्यः॑ । इति॑ । आ॒ह॒ । वै॒श्व॒दे॒व इति॑ वैश्व - दे॒वः । हि । ए॒षः । दे॒वत॑या । वाक् । वै । दे॒वेभ्यः॑ । अपेति॑ । अ॒क्रा॒म॒त् । य॒ज्ञाय॑ । अति॑ष्ठमाना । ते । दे॒वाः । वा॒चि । अप॑क्रान्ताया॒मित्यप॑-क्रा॒न्ता॒या॒म् । तू॒ष्णीम् । ग्रहान्॑ । अ॒गृ॒ह्ण॒त॒ । सा । अ॒म॒न्य॒त॒ । वाक् । अ॒न्तः । य॒न्ति॒ । वै । मा॒ । इति॑ । सा । आ॒ग्र॒य॒णम् । प्रति॑ । एति॑ । अ॒ग॒च्छ॒त् । तत् । आ॒ग्र॒य॒णस्य॑ । आ॒ग्र॒य॒ण॒त्वमित्या᳚ग्रयण - त्वम् ।  \newline


\textbf{Krama Paata} \newline

ए॒ताव॑ती॒र् वै । वै दे॒वताः᳚ । दे॒वता॒स्ताभ्यः॑ । ताभ्य॑ ए॒व । ए॒वैन᳚म् । ए॒नꣳ॒॒ सर्वा᳚भ्यः । सर्वा᳚भ्यो गृह्णाति । गृ॒ह्णा॒त्ये॒षः । ए॒ष ते᳚ । ते॒ योनिः॑ । योनि॒र् विश्वे᳚भ्यः । विश्वे᳚भ्यस्त्वा । त्वा॒ दे॒वेभ्यः॑ । दे॒वेभ्य॒ इति॑ । इत्या॑ह । आ॒ह॒ वै॒श्व॒दे॒वः । वै॒श्व॒दे॒वो हि । वै॒श्व॒दे॒व इति॑ वैश्व - दे॒वः । ह्ये॑षः । ए॒ष दे॒वत॑या । दे॒वत॑या॒ वाक् । वाग् वै । वै दे॒वेभ्यः॑ । दे॒वेभ्योऽप॑ । अपा᳚क्रामत् । अ॒क्रा॒म॒द् य॒ज्ञाय॑ । य॒ज्ञ्याति॑ष्ठमाना । अति॑ष्ठमाना॒ ते । ते दे॒वाः । दे॒वा वा॒चि । वा॒च्यप॑क्रान्तायाम् । अप॑क्रान्तायाम् तू॒ष्णीम् । अप॑क्रान्ताया॒मित्यप॑ - क्रा॒न्ता॒या॒म् । तू॒ष्णीम् ग्रहान्॑ । गृहा॑नगृह्णत । अ॒गृ॒ह्ण॒त॒ सा । साऽम॑न्यत । अ॒म॒न्य॒त॒ वाक् । वाग॒न्तः । अ॒न्तर् य॑न्ति । य॒न्ति॒ वै । वै मा᳚ । मेति॑ । इति॒ सा । साऽऽग्र॑य॒णम् । आ॒ग्र॒य॒णम् प्रति॑ । प्रत्या । आऽग॑च्छत् । अ॒ग॒च्छ॒त् तत् । तदा᳚ग्रय॒णस्य॑ । आ॒ग्र॒य॒णस्या᳚ग्रयण॒त्वम् । आ॒ग्र॒य॒ण॒त्वम् तस्मा᳚त् । आ॒ग्र॒य॒ण॒त्वमित्या᳚ग्रयण - त्वम् \newline

\textbf{Jatai Paata} \newline

1. ए॒ताव॑ती॒र् वै वा ए॒ताव॑ती रे॒ताव॑ती॒र् वै । \newline
2. वै दे॒वता॑ दे॒वता॒ वै वै दे॒वताः᳚ । \newline
3. दे॒वता॒ स्ताभ्य॒ स्ताभ्यो॑ दे॒वता॑ दे॒वता॒ स्ताभ्यः॑ । \newline
4. ताभ्य॑ ए॒वैव ताभ्य॒ स्ताभ्य॑ ए॒व । \newline
5. ए॒वैन॑ मेन मे॒वै वैन᳚म् । \newline
6. ए॒नꣳ॒॒ सर्वा᳚भ्यः॒ सर्वा᳚भ्य एन मेनꣳ॒॒ सर्वा᳚भ्यः । \newline
7. सर्वा᳚भ्यो गृह्णाति गृह्णाति॒ सर्वा᳚भ्यः॒ सर्वा᳚भ्यो गृह्णाति । \newline
8. गृ॒ह्णा॒ त्ये॒ष ए॒ष गृ॑ह्णाति गृह्णा त्ये॒षः । \newline
9. ए॒ष ते॑ त ए॒ष ए॒ष ते᳚ । \newline
10. ते॒ योनि॒र् योनि॑ स्ते ते॒ योनिः॑ । \newline
11. योनि॒र् विश्वे᳚भ्यो॒ विश्वे᳚भ्यो॒ योनि॒र् योनि॒र् विश्वे᳚भ्यः । \newline
12. विश्वे᳚भ्य स्त्वा त्वा॒ विश्वे᳚भ्यो॒ विश्वे᳚भ्य स्त्वा । \newline
13. त्वा॒ दे॒वेभ्यो॑ दे॒वेभ्य॑ स्त्वा त्वा दे॒वेभ्यः॑ । \newline
14. दे॒वेभ्य॒ इतीति॑ दे॒वेभ्यो॑ दे॒वेभ्य॒ इति॑ । \newline
15. इत्या॑हा॒हे तीत्या॑ह । \newline
16. आ॒ह॒ वै॒श्व॒दे॒वो वै᳚श्वदे॒व आ॑हाह वैश्वदे॒वः । \newline
17. वै॒श्व॒दे॒वो हि हि वै᳚श्वदे॒वो वै᳚श्वदे॒वो हि । \newline
18. वै॒श्व॒दे॒व इति॑ वैश्व - दे॒वः । \newline
19. ह्ये॑ष ए॒ष हि ह्ये॑षः । \newline
20. ए॒ष दे॒वत॑या दे॒वत॑ यै॒ष ए॒ष दे॒वत॑या । \newline
21. दे॒वत॑या॒ वाग् वाग् दे॒वत॑या दे॒वत॑या॒ वाक् । \newline
22. वाग् वै वै वाग् वाग् वै । \newline
23. वै दे॒वेभ्यो॑ दे॒वेभ्यो॒ वै वै दे॒वेभ्यः॑ । \newline
24. दे॒वेभ्यो ऽपाप॑ दे॒वेभ्यो॑ दे॒वेभ्यो ऽप॑ । \newline
25. अपा᳚ क्राम दक्राम॒ दपापा᳚ क्रामत् । \newline
26. अ॒क्रा॒म॒द् य॒ज्ञाय॑ य॒ज्ञाया᳚ क्राम दक्रामद् य॒ज्ञाय॑ । \newline
27. य॒ज्ञाया ति॑ष्ठमा॒ना ऽति॑ष्ठमाना य॒ज्ञाय॑ य॒ज्ञाया ति॑ष्ठमाना । \newline
28. अति॑ष्ठमाना॒ ते ते ऽति॑ष्ठमा॒ना ऽति॑ष्ठमाना॒ ते । \newline
29. ते दे॒वा दे॒वा स्ते ते दे॒वाः । \newline
30. दे॒वा वा॒चि वा॒चि दे॒वा दे॒वा वा॒चि । \newline
31. वा॒च्यप॑क्रान्ताया॒ मप॑क्रान्तायां ॅवा॒चि वा॒च्यप॑क्रान्तायाम् । \newline
32. अप॑क्रान्तायाम् तू॒ष्णीम् तू॒ष्णी मप॑क्रान्ताया॒ मप॑क्रान्तायाम् तू॒ष्णीम् । \newline
33. अप॑क्रान्ताया॒मित्यप॑ - क्रा॒न्ता॒या॒म् । \newline
34. तू॒ष्णीम् ग्रहा॒न् ग्रहा᳚न् तू॒ष्णीम् तू॒ष्णीम् ग्रहान्॑ । \newline
35. ग्रहा॑ नगृह्णता गृह्णत॒ ग्रहा॒न् ग्रहा॑ नगृह्णत । \newline
36. अ॒गृ॒ह्ण॒त॒ सा सा ऽगृ॑ह्णता गृह्णत॒ सा । \newline
37. सा ऽम॑न्यता मन्यत॒ सा सा ऽम॑न्यत । \newline
38. अ॒म॒न्य॒त॒ वाग् वा ग॑मन्यता मन्यत॒ वाक् । \newline
39. वाग॒न्त र॒न्तर् वाग् वाग॒न्तः । \newline
40. अ॒न्तर् य॑न्ति यन्त्य॒न्त र॒न्तर् य॑न्ति । \newline
41. य॒न्ति॒ वै वै य॑न्ति यन्ति॒ वै । \newline
42. वै मा॑ मा॒ वै वै मा᳚ । \newline
43. मेतीति॑ मा॒ मेति॑ । \newline
44. इति॒ सा सेतीति॒ सा । \newline
45. सा ऽऽग्र॑य॒ण मा᳚ग्रय॒णꣳ सा सा ऽऽग्र॑य॒णम् । \newline
46. आ॒ग्र॒य॒णम् प्रति॒ प्रत्या᳚ग्रय॒ण मा᳚ग्रय॒णम् प्रति॑ । \newline
47. प्रत्या प्रति॒ प्रत्या । \newline
48. आ ऽग॑च्छ दगच्छ॒दा ऽग॑च्छत् । \newline
49. अ॒ग॒च्छ॒त् तत् तद॑गच्छ दगच्छ॒त् तत् । \newline
50. तदा᳚ग्रय॒णस्या᳚ ग्रय॒णस्य॒ तत् तदा᳚ग्रय॒णस्य॑ । \newline
51. आ॒ग्र॒य॒णस्या᳚ ग्रयण॒त्व मा᳚ग्रयण॒त्व मा᳚ग्रय॒णस्या᳚ ग्रय॒णस्या᳚ ग्रयण॒त्वम् । \newline
52. आ॒ग्र॒य॒ण॒त्वम् तस्मा॒त् तस्मा॑ दाग्रयण॒त्व मा᳚ग्रयण॒त्वम् तस्मा᳚त् । \newline
53. आ॒ग्र॒य॒ण॒त्वमित्या᳚ग्रयण - त्वम् । \newline

\textbf{Ghana Paata } \newline

1. ए॒ताव॑ती॒र् वै वा ए॒ताव॑ती रे॒ताव॑ती॒र् वै दे॒वता॑ दे॒वता॒ वा ए॒ताव॑ती रे॒ताव॑ती॒र् वै दे॒वताः᳚ । \newline
2. वै दे॒वता॑ दे॒वता॒ वै वै दे॒वता॒ स्ताभ्य॒ स्ताभ्यो॑ दे॒वता॒ वै वै दे॒वता॒ स्ताभ्यः॑ । \newline
3. दे॒वता॒ स्ताभ्य॒ स्ताभ्यो॑ दे॒वता॑ दे॒वता॒ स्ताभ्य॑ ए॒वैव ताभ्यो॑ दे॒वता॑ दे॒वता॒ स्ताभ्य॑ ए॒व । \newline
4. ताभ्य॑ ए॒वैव ताभ्य॒ स्ताभ्य॑ ए॒वैन॑ मेन मे॒व ताभ्य॒ स्ताभ्य॑ ए॒वैन᳚म् । \newline
5. ए॒वैन॑ मेन मे॒वै वैनꣳ॒॒ सर्वा᳚भ्यः॒ सर्वा᳚भ्य एन मे॒वै वैनꣳ॒॒ सर्वा᳚भ्यः । \newline
6. ए॒नꣳ॒॒ सर्वा᳚भ्यः॒ सर्वा᳚भ्य एन मेनꣳ॒॒ सर्वा᳚भ्यो गृह्णाति गृह्णाति॒ सर्वा᳚भ्य एन मेनꣳ॒॒ सर्वा᳚भ्यो गृह्णाति । \newline
7. सर्वा᳚भ्यो गृह्णाति गृह्णाति॒ सर्वा᳚भ्यः॒ सर्वा᳚भ्यो गृह्णा त्ये॒ष ए॒ष गृ॑ह्णाति॒ सर्वा᳚भ्यः॒ सर्वा᳚भ्यो गृह्णा त्ये॒षः । \newline
8. गृ॒ह्णा॒ त्ये॒ष ए॒ष गृ॑ह्णाति गृह्णा त्ये॒ष ते॑ त ए॒ष गृ॑ह्णाति गृह्णा त्ये॒ष ते᳚ । \newline
9. ए॒ष ते॑ त ए॒ष ए॒ष ते॒ योनि॒र् योनि॑ स्त ए॒ष ए॒ष ते॒ योनिः॑ । \newline
10. ते॒ योनि॒र् योनि॑ स्ते ते॒ योनि॒र् विश्वे᳚भ्यो॒ विश्वे᳚भ्यो॒ योनि॑ स्ते ते॒ योनि॒र् विश्वे᳚भ्यः । \newline
11. योनि॒र् विश्वे᳚भ्यो॒ विश्वे᳚भ्यो॒ योनि॒र् योनि॒र् विश्वे᳚भ्य स्त्वा त्वा॒ विश्वे᳚भ्यो॒ योनि॒र् योनि॒र् विश्वे᳚भ्य स्त्वा । \newline
12. विश्वे᳚भ्य स्त्वा त्वा॒ विश्वे᳚भ्यो॒ विश्वे᳚भ्य स्त्वा दे॒वेभ्यो॑ दे॒वेभ्य॑ स्त्वा॒ विश्वे᳚भ्यो॒ विश्वे᳚भ्य स्त्वा दे॒वेभ्यः॑ । \newline
13. त्वा॒ दे॒वेभ्यो॑ दे॒वेभ्य॑ स्त्वा त्वा दे॒वेभ्य॒ इतीति॑ दे॒वेभ्य॑ स्त्वा त्वा दे॒वेभ्य॒ इति॑ । \newline
14. दे॒वेभ्य॒ इतीति॑ दे॒वेभ्यो॑ दे॒वेभ्य॒ इत्या॑हा॒ हेति॑ दे॒वेभ्यो॑ दे॒वेभ्य॒ इत्या॑ह । \newline
15. इत्या॑हा॒हे तीत्या॑ह वैश्वदे॒वो वै᳚श्वदे॒व आ॒हे तीत्या॑ह वैश्वदे॒वः । \newline
16. आ॒ह॒ वै॒श्व॒दे॒वो वै᳚श्वदे॒व आ॑हाह वैश्वदे॒वो हि हि वै᳚श्वदे॒व आ॑हाह वैश्वदे॒वो हि । \newline
17. वै॒श्व॒दे॒वो हि हि वै᳚श्वदे॒वो वै᳚श्वदे॒वो ह्ये॑ष ए॒ष हि वै᳚श्वदे॒वो वै᳚श्वदे॒वो ह्ये॑षः । \newline
18. वै॒श्व॒दे॒व इति॑ वैश्व - दे॒वः । \newline
19. ह्ये॑ष ए॒ष हि ह्ये॑ष दे॒वत॑या दे॒वत॑ यै॒ष हि ह्ये॑ष दे॒वत॑या । \newline
20. ए॒ष दे॒वत॑या दे॒वत॑ यै॒ष ए॒ष दे॒वत॑या॒ वाग् वाग् दे॒वत॑ यै॒ष ए॒ष दे॒वत॑या॒ वाक् । \newline
21. दे॒वत॑या॒ वाग् वाग् दे॒वत॑या दे॒वत॑या॒ वाग् वै वै वाग् दे॒वत॑या दे॒वत॑या॒ वाग् वै । \newline
22. वाग् वै वै वाग् वाग् वै दे॒वेभ्यो॑ दे॒वेभ्यो॒ वै वाग् वाग् वै दे॒वेभ्यः॑ । \newline
23. वै दे॒वेभ्यो॑ दे॒वेभ्यो॒ वै वै दे॒वेभ्यो ऽपाप॑ दे॒वेभ्यो॒ वै वै दे॒वेभ्यो ऽप॑ । \newline
24. दे॒वेभ्यो ऽपाप॑ दे॒वेभ्यो॑ दे॒वेभ्यो ऽपा᳚ क्रामद क्राम॒ दप॑ दे॒वेभ्यो॑ दे॒वेभ्यो ऽपा᳚क्रामत् । \newline
25. अपा᳚ क्रामद क्राम॒ दपापा᳚ क्रामद् य॒ज्ञाय॑ य॒ज्ञाया᳚ क्राम॒ दपापा᳚ क्रामद् य॒ज्ञाय॑ । \newline
26. अ॒क्रा॒म॒द् य॒ज्ञाय॑ य॒ज्ञाया᳚ क्राम दक्रामद् य॒ज्ञाया ति॑ष्ठमा॒ना ऽति॑ष्ठमाना य॒ज्ञाया᳚ क्राम दक्रामद् य॒ज्ञाया ति॑ष्ठमाना । \newline
27. य॒ज्ञाया ति॑ष्ठमा॒ना ऽति॑ष्ठमाना य॒ज्ञाय॑ य॒ज्ञाया ति॑ष्ठमाना॒ ते ते ऽति॑ष्ठमाना य॒ज्ञाय॑ य॒ज्ञाया ति॑ष्ठमाना॒ ते । \newline
28. अति॑ष्ठमाना॒ ते ते ऽति॑ष्ठमा॒ना ऽति॑ष्ठमाना॒ ते दे॒वा दे॒वा स्ते ऽति॑ष्ठमा॒ना ऽति॑ष्ठमाना॒ ते दे॒वाः । \newline
29. ते दे॒वा दे॒वा स्ते ते दे॒वा वा॒चि वा॒चि दे॒वा स्ते ते दे॒वा वा॒चि । \newline
30. दे॒वा वा॒चि वा॒चि दे॒वा दे॒वा वा॒च्य प॑क्रान्ताया॒ मप॑क्रान्तायां ॅवा॒चि दे॒वा दे॒वा वा॒च्य प॑क्रान्तायाम् । \newline
31. वा॒च्य प॑क्रान्ताया॒ मप॑क्रान्तायां ॅवा॒चि वा॒च्य प॑क्रान्तायाम् तू॒ष्णीम् तू॒ष्णी मप॑क्रान्तायां ॅवा॒चि वा॒च्य प॑क्रान्तायाम् तू॒ष्णीम् । \newline
32. अप॑क्रान्तायाम् तू॒ष्णीम् तू॒ष्णी मप॑क्रान्ताया॒ मप॑क्रान्तायाम् तू॒ष्णीम् ग्रहा॒न् ग्रहा᳚न् तू॒ष्णी मप॑क्रान्ताया॒ मप॑क्रान्तायाम् तू॒ष्णीम् ग्रहान्॑ । \newline
33. अप॑क्रान्ताया॒मित्यप॑ - क्रा॒न्ता॒या॒म् । \newline
34. तू॒ष्णीम् ग्रहा॒न् ग्रहा᳚न् तू॒ष्णीम् तू॒ष्णीम् ग्रहा॑ नगृह्णता गृह्णत॒ ग्रहा᳚न् तू॒ष्णीम् तू॒ष्णीम् ग्रहा॑ नगृह्णत । \newline
35. ग्रहा॑ नगृह्णता गृह्णत॒ ग्रहा॒न् ग्रहा॑ नगृह्णत॒ सा सा ऽगृ॑ह्णत॒ ग्रहा॒न् ग्रहा॑ नगृह्णत॒ सा । \newline
36. अ॒गृ॒ह्ण॒त॒ सा सा ऽगृ॑ह्णता गृह्णत॒ सा ऽम॑न्यता मन्यत॒ सा ऽगृ॑ह्णता गृह्णत॒ सा ऽम॑न्यत । \newline
37. सा ऽम॑न्यता मन्यत॒ सा सा ऽम॑न्यत॒ वाग् वाग॑मन्यत॒ सा सा ऽम॑न्यत॒ वाक् । \newline
38. अ॒म॒न्य॒त॒ वाग् वाग॑मन्यता मन्यत॒ वाग॒न्त र॒न्तर् वाग॑मन्यता मन्यत॒ वाग॒न्तः । \newline
39. वाग॒न्त र॒न्तर् वाग् वाग॒न्तर् य॑न्ति यन्त्य॒न्तर् वाग् वाग॒न्तर् य॑न्ति । \newline
40. अ॒न्तर् य॑न्ति यन्त्य॒न्त र॒न्तर् य॑न्ति॒ वै वै य॑न्त्य॒न्त र॒न्तर् य॑न्ति॒ वै । \newline
41. य॒न्ति॒ वै वै य॑न्ति यन्ति॒ वै मा॑ मा॒ वै य॑न्ति यन्ति॒ वै मा᳚ । \newline
42. वै मा॑ मा॒ वै वै मेतीति॑ मा॒ वै वै मेति॑ । \newline
43. मेतीति॑ मा॒ मेति॒ सा सेति॑ मा॒ मेति॒ सा । \newline
44. इति॒ सा सेतीति॒ सा ऽऽग्र॑य॒ण मा᳚ग्रय॒णꣳ सेतीति॒ सा ऽऽग्र॑य॒णम् । \newline
45. सा ऽऽग्र॑य॒ण मा᳚ग्रय॒णꣳ सा सा ऽऽग्र॑य॒णम् प्रति॒ प्रत्या᳚ ग्रय॒णꣳ सा सा ऽऽग्र॑य॒णम् प्रति॑ । \newline
46. आ॒ग्र॒य॒णम् प्रति॒ प्रत्या᳚ ग्रय॒ण मा᳚ग्रय॒णम् प्रत्या प्रत्या᳚ ग्रय॒ण मा᳚ग्रय॒णम् प्रत्या । \newline
47. प्रत्या प्रति॒ प्रत्या ऽग॑च्छ दगच्छ॒दा प्रति॒ प्रत्या ऽग॑च्छत् । \newline
48. आ ऽग॑च्छ दगच्छ॒दा ऽग॑च्छ॒त् तत् तद॑गच्छ॒दा ऽग॑च्छ॒त् तत् । \newline
49. अ॒ग॒च्छ॒त् तत् तद॑गच्छ दगच्छ॒त् तदा᳚ग्रय॒णस्या᳚ ग्रय॒णस्य॒ तद॑गच्छ दगच्छ॒त् तदा᳚ग्रय॒णस्य॑ । \newline
50. तदा᳚ग्रय॒णस्या᳚ ग्रय॒णस्य॒ तत् तदा᳚ग्रय॒णस्या᳚ ग्रयण॒त्व मा᳚ग्रयण॒त्व मा᳚ग्रय॒णस्य॒ तत् तदा᳚ग्रय॒णस्या᳚ ग्रयण॒त्वम् । \newline
51. आ॒ग्र॒य॒णस्या᳚ ग्रयण॒त्व मा᳚ग्रयण॒त्व मा᳚ग्रय॒णस्या᳚ ग्रय॒णस्या᳚ ग्रयण॒त्वम् तस्मा॒त् तस्मा॑ दाग्रयण॒त्व मा᳚ग्रय॒णस्या᳚ ग्रय॒णस्या᳚ ग्रयण॒त्वम् तस्मा᳚त् । \newline
52. आ॒ग्र॒य॒ण॒त्वम् तस्मा॒त् तस्मा॑ दाग्रयण॒त्व मा᳚ग्रयण॒त्वम् तस्मा॑ दाग्रय॒ण आ᳚ग्रय॒णे तस्मा॑ दाग्रयण॒त्व मा᳚ग्रयण॒त्वम् तस्मा॑ दाग्रय॒णे । \newline
53. आ॒ग्र॒य॒ण॒त्वमित्या᳚ग्रयण - त्वम् । \newline
\pagebreak
\markright{ TS 6.4.11.3  \hfill https://www.vedavms.in \hfill}

\section{ TS 6.4.11.3 }

\textbf{TS 6.4.11.3 } \newline
\textbf{Samhita Paata} \newline

तस्मा॑दाग्रय॒णे वाग्वि सृ॑ज्यते॒ यत् तू॒ष्णीं पूर्वे॒ ग्रहा॑ गृ॒ह्यन्ते॒ यथा᳚थ्सा॒रीय॑ति म॒ आख॒ इय॑ति॒ नाप॑ राथ्स्या॒-मीत्यु॑पावसृ॒जत्ये॒वमे॒व तद॑द्ध्व॒र्युरा᳚ग्रय॒णं गृ॑ही॒त्वा य॒ज्ञ्मा॒रभ्य॒ वाचं॒ ॅवि सृ॑जते॒ त्रिर्.हिं क॑रोत्युद्गा॒तॄ-ने॒व तद् वृ॑णीते प्र॒जाप॑ति॒र्वा ए॒ष यदा᳚ग्रय॒णो यदा᳚ग्रय॒णं गृ॑ही॒त्वा हिं॑ क॒रोति॑ प्र॒जाप॑तिरे॒व- [  ] \newline

\textbf{Pada Paata} \newline

तस्मा᳚त् । आ॒ग्र॒य॒णे । वाक् । वीति॑ । सृ॒ज्य॒ते॒ । यत् । तू॒ष्णीम् । पूर्वे᳚ । ग्रहाः᳚ । गृ॒ह्यन्ते᳚ । यथा᳚ । थ्सा॒री । इय॑ति । मे॒ । आखः॑ । इय॑ति । न । अपेति॑ । रा॒थ्स्या॒मि॒ । इति॑ । उ॒पा॒व॒सृ॒जतीत्यु॑प - अ॒व॒सृ॒जति॑ । ए॒वम् । ए॒व । तत् । अ॒द्ध्व॒र्युः । आ॒ग्र॒य॒णम् । गृ॒ही॒त्वा । य॒ज्ञ्म् । आ॒रभ्येत्या᳚-रभ्य॑ । वाच᳚म् । वीति॑ । सृ॒ज॒ते॒ । त्रिः । हिम् । क॒रो॒ति॒ । उ॒द्गा॒तॄनित्यु॑त् - गा॒तॄन् । ए॒व । तत् । वृ॒णी॒ते॒ । प्र॒जाप॑ति॒रिति॑ प्र॒जा - प॒तिः॒ । वै । ए॒षः । यत् । आ॒ग्र॒य॒णः । यत् । आ॒ग्र॒य॒णम् । गृ॒ही॒त्वा । हि॒कं॒रोतीति॑ हिम् - क॒रोति॑ । प्र॒जाप॑ति॒रिति॑ प्र॒जा-प॒तिः॒ । ए॒व ।  \newline


\textbf{Krama Paata} \newline

तस्मा॑दाग्रय॒णे । आ॒ग्र॒य॒णे वाक् । वाग् वि । वि सृ॑ज्यते । सृ॒ज्य॒ते॒ यत् । यत् तू॒ष्णीम् । तू॒ष्णीम् पूर्वे᳚ । पूर्वे॒ ग्रहाः᳚ । ग्रहा॑ गृ॒ह्यन्ते᳚ । गृ॒ह्यन्ते॒ यथा᳚ । यथा᳚ थ्सा॒री । थ्सा॒रीय॑ति । इय॑ति॒ मे । म॒ आखः॑ । आख॒ इय॑ति । इय॑ति॒ न । नाप॑ । अप॑ राथ्स्यामि । रा॒थ्स्या॒मीति॑ । इत्यु॑पावसृ॒जति॑ । उ॒पा॒व॒सृ॒जत्ये॒वम् । उ॒पा॒व॒सृ॒जतीत्यु॑प - अ॒व॒सृ॒जति॑ । ए॒वमे॒व । ए॒व तत् । तद॑द्ध्व॒र्युः । अ॒द्ध्व॒र्युरा᳚ग्रय॒णम् । आ॒ग्र॒य॒णम् गृ॑ही॒त्वा । गृ॒ही॒त्वा य॒ज्ञ्म् । य॒ज्ञ्मा॒रभ्य॑ । आ॒रभ्य॒ वाच᳚म् । आ॒रभ्येत्या᳚ - रभ्य॑ । वाच॒म् ॅवि । वि सृ॑जते । सृ॒ज॒ते॒ त्रिः । त्रिर्. हिम् । हिम् क॑रोति । क॒रो॒त्यु॒द्‌गा॒तॄन् । उ॒द्‌गा॒तॄने॒व । उ॒द्‌गा॒तॄनित्यु॑त् - गा॒तॄन् । ए॒व तत् । तद् वृ॑णीते । वृ॒णी॒ते॒ प्र॒जाप॑तिः । प्र॒जाप॑ति॒र् वै । प्र॒जाप॑ति॒रिति॑ प्र॒जा - प॒तिः॒ । वा ए॒षः । ए॒ष यत् । यदा᳚ग्रय॒णः । आ॒ग्र॒य॒णो यत् । यदा᳚ग्रय॒णम् । आ॒ग्र॒य॒णम् गृ॑ही॒त्वा । गृ॒ही॒त्वा हि॑ङ्‍क॒रोति॑ । हि॒ङ्‍क॒रोति॑ प्र॒जाप॑तिः । हि॒ङ्‍क॒रोतीति॑ हिम् - क॒रोति॑ । प्र॒जाप॑तिरे॒व । प्र॒जाप॑ति॒रिति॑ प्र॒जा - प॒तिः॒ । ए॒व तत् \newline

\textbf{Jatai Paata} \newline

1. तस्मा॑ दाग्रय॒ण आ᳚ग्रय॒णे तस्मा॒त् तस्मा॑ दाग्रय॒णे । \newline
2. आ॒ग्र॒य॒णे वाग् वागा᳚ग्रय॒ण आ᳚ग्रय॒णे वाक् । \newline
3. वाग् वि वि वाग् वाग् वि । \newline
4. वि सृ॑ज्यते सृज्यते॒ वि वि सृ॑ज्यते । \newline
5. सृ॒ज्य॒ते॒ यद् यथ् सृ॑ज्यते सृज्यते॒ यत् । \newline
6. यत् तू॒ष्णीम् तू॒ष्णीं ॅयद् यत् तू॒ष्णीम् । \newline
7. तू॒ष्णीम् पूर्वे॒ पूर्वे॑ तू॒ष्णीम् तू॒ष्णीम् पूर्वे᳚ । \newline
8. पूर्वे॒ ग्रहा॒ ग्रहाः॒ पूर्वे॒ पूर्वे॒ ग्रहाः᳚ । \newline
9. ग्रहा॑ गृ॒ह्यन्ते॑ गृ॒ह्यन्ते॒ ग्रहा॒ ग्रहा॑ गृ॒ह्यन्ते᳚ । \newline
10. गृ॒ह्यन्ते॒ यथा॒ यथा॑ गृ॒ह्यन्ते॑ गृ॒ह्यन्ते॒ यथा᳚ । \newline
11. यथा᳚ थ्सा॒री थ्सा॒री यथा॒ यथा᳚ थ्सा॒री । \newline
12. थ्सा॒री य॒ती य॑ति थ्सा॒री थ्सा॒री य॑ति । \newline
13. इय॑ति मे म॒ इय॒ती य॑ति मे । \newline
14. म॒ आख॒ आखो॑ मे म॒ आखः॑ । \newline
15. आख॒ इय॒ती य॒ त्याख॒ आख॒ इय॑ति । \newline
16. इय॑ति॒ न नेय॒ती य॑ति॒ न । \newline
17. नापाप॒ न नाप॑ । \newline
18. अप॑ राथ्स्यामि राथ्स्या॒ म्यपाप॑ राथ्स्यामि । \newline
19. रा॒थ्स्या॒मी तीति॑ राथ्स्यामि राथ्स्या॒ मीति॑ । \newline
20. इत्यु॑पावसृ॒ज त्यु॑पावसृ॒ज तीती त्यु॑पावसृ॒जति॑ । \newline
21. उ॒पा॒व॒सृ॒ज त्ये॒व मे॒व मु॑पावसृ॒ज त्यु॑पावसृ॒ज त्ये॒वम् । \newline
22. उ॒पा॒व॒सृ॒जतीत्यु॑प - अ॒व॒सृ॒जति॑ । \newline
23. ए॒व मे॒वै वैव मे॒व मे॒व । \newline
24. ए॒व तत् तदे॒ वैव तत् । \newline
25. तद॑द्ध्व॒र्यु र॑द्ध्व॒र्यु स्तत् तद॑द्ध्व॒र्युः । \newline
26. अ॒द्ध्व॒र्यु रा᳚ग्रय॒ण मा᳚ग्रय॒ण म॑द्ध्व॒र्यु र॑द्ध्व॒र्यु रा᳚ग्रय॒णम् । \newline
27. आ॒ग्र॒य॒णम् गृ॑ही॒त्वा गृ॑ही॒त्वा ऽऽग्र॑य॒ण मा᳚ग्रय॒णम् गृ॑ही॒त्वा । \newline
28. गृ॒ही॒त्वा य॒ज्ञ्ं ॅय॒ज्ञ्म् गृ॑ही॒त्वा गृ॑ही॒त्वा य॒ज्ञ्म् । \newline
29. य॒ज्ञ् मा॒रभ्या॒ रभ्य॑ य॒ज्ञ्ं ॅय॒ज्ञ् मा॒रभ्य॑ । \newline
30. आ॒रभ्य॒ वाचं॒ ॅवाच॑ मा॒रभ्या॒ रभ्य॒ वाच᳚म् । \newline
31. आ॒रभ्येत्या᳚ - रभ्य॑ । \newline
32. वाचं॒ ॅवि वि वाचं॒ ॅवाचं॒ ॅवि । \newline
33. वि सृ॑जते सृजते॒ वि वि सृ॑जते । \newline
34. सृ॒ज॒ते॒ त्रि स्त्रिः सृ॑जते सृजते॒ त्रिः । \newline
35. त्रिर्. हिꣳ हिम् त्रि स्त्रिर्. हिम् । \newline
36. हिम् क॑रोति करोति॒ हिꣳ हिम् क॑रोति । \newline
37. क॒रो॒ त्यु॒द्‍गा॒तॄ नु॑द्‍गा॒तॄन् क॑रोति करो त्युद्‍गा॒तॄन् । \newline
38. उ॒द्‍गा॒तॄ ने॒वै वोद्‍गा॒तॄ नु॑द्‍गा॒तॄने॒व । \newline
39. उ॒द्‍गा॒तॄनित्यु॑त् - गा॒तॄन् । \newline
40. ए॒व तत् तदे॒ वैव तत् । \newline
41. तद् वृ॑णीते वृणीते॒ तत् तद् वृ॑णीते । \newline
42. वृ॒णी॒ते॒ प्र॒जाप॑तिः प्र॒जाप॑तिर् वृणीते वृणीते प्र॒जाप॑तिः । \newline
43. प्र॒जाप॑ति॒र् वै वै प्र॒जाप॑तिः प्र॒जाप॑ति॒र् वै । \newline
44. प्र॒जाप॑ति॒रिति॑ प्र॒जा - प॒तिः॒ । \newline
45. वा ए॒ष ए॒ष वै वा ए॒षः । \newline
46. ए॒ष यद् यदे॒ष ए॒ष यत् । \newline
47. यदा᳚ग्रय॒ण आ᳚ग्रय॒णो यद् यदा᳚ग्रय॒णः । \newline
48. आ॒ग्र॒य॒णो यद् यदा᳚ग्रय॒ण आ᳚ग्रय॒णो यत् । \newline
49. यदा᳚ग्रय॒ण मा᳚ग्रय॒णं ॅयद् यदा᳚ग्रय॒णम् । \newline
50. आ॒ग्र॒य॒णम् गृ॑ही॒त्वा गृ॑ही॒त्वा ऽऽग्र॑य॒ण मा᳚ग्रय॒णम् गृ॑ही॒त्वा । \newline
51. गृ॒ही॒त्वा हि॑ङ्क॒रोति॑ हिङ्क॒रोति॑ गृही॒त्वा गृ॑ही॒त्वा हि॑ङ्क॒रोति॑ । \newline
52. हि॒ङ्क॒रोति॑ प्र॒जाप॑तिः प्र॒जाप॑तिर्. हिङ्क॒रोति॑ हिङ्क॒रोति॑ प्र॒जाप॑तिः । \newline
53. हि॒ङ्क॒रोतीति॑ हिम् - क॒रोति॑ । \newline
54. प्र॒जाप॑ति रे॒वैव प्र॒जाप॑तिः प्र॒जाप॑ति रे॒व । \newline
55. प्र॒जाप॑ति॒रिति॑ प्र॒जा - प॒तिः॒ । \newline
56. ए॒व तत् तदे॒ वैव तत् । \newline

\textbf{Ghana Paata } \newline

1. तस्मा॑ दाग्रय॒ण आ᳚ग्रय॒णे तस्मा॒त् तस्मा॑ दाग्रय॒णे वाग् वागा᳚ ग्रय॒णे तस्मा॒त् तस्मा॑ दाग्रय॒णे वाक् । \newline
2. आ॒ग्र॒य॒णे वाग् वागा᳚ ग्रय॒ण आ᳚ग्रय॒णे वाग् वि वि वागा᳚ ग्रय॒ण आ᳚ग्रय॒णे वाग् वि । \newline
3. वाग् वि वि वाग् वाग् वि सृ॑ज्यते सृज्यते॒ वि वाग् वाग् वि सृ॑ज्यते । \newline
4. वि सृ॑ज्यते सृज्यते॒ वि वि सृ॑ज्यते॒ यद् यथ् सृ॑ज्यते॒ वि वि सृ॑ज्यते॒ यत् । \newline
5. सृ॒ज्य॒ते॒ यद् यथ् सृ॑ज्यते सृज्यते॒ यत् तू॒ष्णीम् तू॒ष्णीं ॅयथ् सृ॑ज्यते सृज्यते॒ यत् तू॒ष्णीम् । \newline
6. यत् तू॒ष्णीम् तू॒ष्णीं ॅयद् यत् तू॒ष्णीम् पूर्वे॒ पूर्वे॑ तू॒ष्णीं ॅयद् यत् तू॒ष्णीम् पूर्वे᳚ । \newline
7. तू॒ष्णीम् पूर्वे॒ पूर्वे॑ तू॒ष्णीम् तू॒ष्णीम् पूर्वे॒ ग्रहा॒ ग्रहाः॒ पूर्वे॑ तू॒ष्णीम् तू॒ष्णीम् पूर्वे॒ ग्रहाः᳚ । \newline
8. पूर्वे॒ ग्रहा॒ ग्रहाः॒ पूर्वे॒ पूर्वे॒ ग्रहा॑ गृ॒ह्यन्ते॑ गृ॒ह्यन्ते॒ ग्रहाः॒ पूर्वे॒ पूर्वे॒ ग्रहा॑ गृ॒ह्यन्ते᳚ । \newline
9. ग्रहा॑ गृ॒ह्यन्ते॑ गृ॒ह्यन्ते॒ ग्रहा॒ ग्रहा॑ गृ॒ह्यन्ते॒ यथा॒ यथा॑ गृ॒ह्यन्ते॒ ग्रहा॒ ग्रहा॑ गृ॒ह्यन्ते॒ यथा᳚ । \newline
10. गृ॒ह्यन्ते॒ यथा॒ यथा॑ गृ॒ह्यन्ते॑ गृ॒ह्यन्ते॒ यथा᳚ थ्सा॒री थ्सा॒री यथा॑ गृ॒ह्यन्ते॑ गृ॒ह्यन्ते॒ यथा᳚ थ्सा॒री । \newline
11. यथा᳚ थ्सा॒री थ्सा॒री यथा॒ यथा᳚ थ्सा॒री य॒ती य॑ति थ्सा॒री यथा॒ यथा᳚ थ्सा॒री य॑ति । \newline
12. थ्सा॒री य॒ती य॑ति थ्सा॒री थ्सा॒री य॑ति मे म॒ इय॑ति थ्सा॒री थ्सा॒री य॑ति मे । \newline
13. इय॑ति मे म॒ इय॒ती य॑ति म॒ आख॒ आखो॑ म॒ इय॒ती य॑ति म॒ आखः॑ । \newline
14. म॒ आख॒ आखो॑ मे म॒ आख॒ इय॒ तीय॒ त्याखो॑ मे म॒ आख॒ इय॑ति । \newline
15. आख॒ इय॒ तीय॒ त्याख॒ आख॒ इय॑ति॒ न नेय॒त्याख॒ आख॒ इय॑ति॒ न । \newline
16. इय॑ति॒ न नेय॒ती य॑ति॒ नापाप॒ नेय॒ती य॑ति॒ नाप॑ । \newline
17. नापाप॒ न नाप॑ राथ्स्यामि राथ्स्या॒ म्यप॒ न नाप॑ राथ्स्यामि । \newline
18. अप॑ राथ्स्यामि राथ्स्या॒ म्यपाप॑ राथ्स्या॒मी तीति॑ राथ्स्या॒ म्यपाप॑ राथ्स्या॒ मीति॑ । \newline
19. रा॒थ्स्या॒मी तीति॑ राथ्स्यामि राथ्स्या॒मी त्यु॑पावसृ॒ज त्यु॑पावसृ॒जतीति॑ राथ्स्यामि राथ्स्या॒मी त्यु॑पावसृ॒जति॑ । \newline
20. इत्यु॑पावसृ॒ज त्यु॑पावसृ॒जतीती त्यु॑पावसृ॒जत्ये॒व मे॒व मु॑पावसृ॒जतीती त्यु॑पावसृ॒ज त्ये॒वम् । \newline
21. उ॒पा॒व॒सृ॒ज त्ये॒व मे॒व मु॑पावसृ॒ज त्यु॑पावसृ॒ज त्ये॒व मे॒वै वैव मु॑पावसृ॒ज त्यु॑पावसृ॒ज त्ये॒व मे॒व । \newline
22. उ॒पा॒व॒सृ॒जतीत्यु॑प - अ॒व॒सृ॒जति॑ । \newline
23. ए॒व मे॒वै वैव मे॒व मे॒व तत् तदे॒ वैव मे॒व मे॒व तत् । \newline
24. ए॒व तत् तदे॒ वैव तद॑द्ध्व॒र्यु र॑द्ध्व॒र्यु स्तदे॒ वैव तद॑द्ध्व॒र्युः । \newline
25. तद॑द्ध्व॒र्यु र॑द्ध्व॒र्यु स्तत् तद॑द्ध्व॒र्यु रा᳚ग्रय॒ण मा᳚ग्रय॒ण म॑द्ध्व॒र्यु स्तत् तद॑द्ध्व॒र्यु रा᳚ग्रय॒णम् । \newline
26. अ॒द्ध्व॒र्यु रा᳚ग्रय॒ण मा᳚ग्रय॒ण म॑द्ध्व॒र्यु र॑द्ध्व॒र्यु रा᳚ग्रय॒णम् गृ॑ही॒त्वा गृ॑ही॒त्वा ऽऽग्र॑य॒ण म॑द्ध्व॒र्यु र॑द्ध्व॒र्यु रा᳚ग्रय॒णम् गृ॑ही॒त्वा । \newline
27. आ॒ग्र॒य॒णम् गृ॑ही॒त्वा गृ॑ही॒त्वा ऽऽग्र॑य॒ण मा᳚ग्रय॒णम् गृ॑ही॒त्वा य॒ज्ञ्ं ॅय॒ज्ञ्म् गृ॑ही॒त्वा ऽऽग्र॑य॒ण मा᳚ग्रय॒णम् गृ॑ही॒त्वा य॒ज्ञ्म् । \newline
28. गृ॒ही॒त्वा य॒ज्ञ्ं ॅय॒ज्ञ्म् गृ॑ही॒त्वा गृ॑ही॒त्वा य॒ज्ञ् मा॒रभ्या॒ रभ्य॑ य॒ज्ञ्म् गृ॑ही॒त्वा गृ॑ही॒त्वा य॒ज्ञ् मा॒रभ्य॑ । \newline
29. य॒ज्ञ् मा॒रभ्या॒ रभ्य॑ य॒ज्ञ्ं ॅय॒ज्ञ् मा॒रभ्य॒ वाचं॒ ॅवाच॑ मा॒रभ्य॑ य॒ज्ञ्ं ॅय॒ज्ञ् मा॒रभ्य॒ वाच᳚म् । \newline
30. आ॒रभ्य॒ वाचं॒ ॅवाच॑ मा॒रभ्या॒ रभ्य॒ वाचं॒ ॅवि वि वाच॑ मा॒रभ्या॒ रभ्य॒ वाचं॒ ॅवि । \newline
31. आ॒रभ्येत्या᳚ - रभ्य॑ । \newline
32. वाचं॒ ॅवि वि वाचं॒ ॅवाचं॒ ॅवि सृ॑जते सृजते॒ वि वाचं॒ ॅवाचं॒ ॅवि सृ॑जते । \newline
33. वि सृ॑जते सृजते॒ वि वि सृ॑जते॒ त्रि स्त्रिः सृ॑जते॒ वि वि सृ॑जते॒ त्रिः । \newline
34. सृ॒ज॒ते॒ त्रि स्त्रिः सृ॑जते सृजते॒ त्रिर्. हिꣳ हिम् त्रिः सृ॑जते सृजते॒ त्रिर्. हिम् । \newline
35. त्रिर्. हिꣳ हिम् त्रि स्त्रिर्. हिम् क॑रोति करोति॒ हिम् त्रि स्त्रिर्. हिम् क॑रोति । \newline
36. हिम् क॑रोति करोति॒ हिꣳ हिम् क॑रो त्युद्‌गा॒तॄ नु॑द्‌गा॒तॄन् क॑रोति॒ हिꣳ हिम् क॑रो त्युद्‌गा॒तॄन् । \newline
37. क॒रो॒ त्यु॒द्‌गा॒तॄ नु॑द्‌गा॒तॄन् क॑रोति करो त्युद्‌गा॒तॄ ने॒वै वोद्‌गा॒तॄन् क॑रोति करो त्युद्‌गा॒तॄ ने॒व । \newline
38. उ॒द्‌गा॒तॄ ने॒वै वोद्‌गा॒तॄ नु॑द्‌गा॒तॄ ने॒व तत् तदे॒ वोद्‌गा॒तॄ नु॑द्‌गा॒तॄने॒व तत् । \newline
39. उ॒द्‌गा॒तॄनित्यु॑त् - गा॒तॄन् । \newline
40. ए॒व तत् तदे॒ वैव तद् वृ॑णीते वृणीते॒ तदे॒ वैव तद् वृ॑णीते । \newline
41. तद् वृ॑णीते वृणीते॒ तत् तद् वृ॑णीते प्र॒जाप॑तिः प्र॒जाप॑तिर् वृणीते॒ तत् तद् वृ॑णीते प्र॒जाप॑तिः । \newline
42. वृ॒णी॒ते॒ प्र॒जाप॑तिः प्र॒जाप॑तिर् वृणीते वृणीते प्र॒जाप॑ति॒र् वै वै प्र॒जाप॑तिर् वृणीते वृणीते प्र॒जाप॑ति॒र् वै । \newline
43. प्र॒जाप॑ति॒र् वै वै प्र॒जाप॑तिः प्र॒जाप॑ति॒र् वा ए॒ष ए॒ष वै प्र॒जाप॑तिः प्र॒जाप॑ति॒र् वा ए॒षः । \newline
44. प्र॒जाप॑ति॒रिति॑ प्र॒जा - प॒तिः॒ । \newline
45. वा ए॒ष ए॒ष वै वा ए॒ष यद् यदे॒ष वै वा ए॒ष यत् । \newline
46. ए॒ष यद् यदे॒ष ए॒ष यदा᳚ग्रय॒ण आ᳚ग्रय॒णो यदे॒ष ए॒ष यदा᳚ग्रय॒णः । \newline
47. यदा᳚ग्रय॒ण आ᳚ग्रय॒णो यद् यदा᳚ग्रय॒णो यद् यदा᳚ग्रय॒णो यद् यदा᳚ग्रय॒णो यत् । \newline
48. आ॒ग्र॒य॒णो यद् यदा᳚ग्रय॒ण आ᳚ग्रय॒णो यदा᳚ग्रय॒ण मा᳚ग्रय॒णं ॅयदा᳚ग्रय॒ण आ᳚ग्रय॒णो यदा᳚ग्रय॒णम् । \newline
49. यदा᳚ग्रय॒ण मा᳚ग्रय॒णं ॅयद् यदा᳚ग्रय॒णम् गृ॑ही॒त्वा गृ॑ही॒त्वा ऽऽग्र॑य॒णं ॅयद् यदा᳚ग्रय॒णम् गृ॑ही॒त्वा । \newline
50. आ॒ग्र॒य॒णम् गृ॑ही॒त्वा गृ॑ही॒त्वा ऽऽग्र॑य॒ण मा᳚ग्रय॒णम् गृ॑ही॒त्वा हि॑ङ्क॒रोति॑ हिङ्क॒रोति॑ गृही॒त्वा ऽऽग्र॑य॒ण मा᳚ग्रय॒णम् गृ॑ही॒त्वा हि॑ङ्क॒रोति॑ । \newline
51. गृ॒ही॒त्वा हि॑ङ्क॒रोति॑ हिङ्क॒रोति॑ गृही॒त्वा गृ॑ही॒त्वा हि॑ङ्क॒रोति॑ प्र॒जाप॑तिः प्र॒जाप॑तिर्. हिङ्क॒रोति॑ गृही॒त्वा गृ॑ही॒त्वा हि॑ङ्क॒रोति॑ प्र॒जाप॑तिः । \newline
52. हि॒ङ्क॒रोति॑ प्र॒जाप॑तिः प्र॒जाप॑तिर्. हिङ्क॒रोति॑ हिङ्क॒रोति॑ प्र॒जाप॑ति रे॒वैव प्र॒जाप॑तिर्. हिङ्क॒रोति॑ हिङ्क॒रोति॑ प्र॒जाप॑ति रे॒व । \newline
53. हि॒ङ्क॒रोतीति॑ हिम् - क॒रोति॑ । \newline
54. प्र॒जाप॑ति रे॒वैव प्र॒जाप॑तिः प्र॒जाप॑ति रे॒व तत् तदे॒व प्र॒जाप॑तिः प्र॒जाप॑ति रे॒व तत् । \newline
55. प्र॒जाप॑ति॒रिति॑ प्र॒जा - प॒तिः॒ । \newline
56. ए॒व तत् तदे॒ वैव तत् प्र॒जाः प्र॒जा स्त दे॒वैव तत् प्र॒जाः । \newline
\pagebreak
\markright{ TS 6.4.11.4  \hfill https://www.vedavms.in \hfill}

\section{ TS 6.4.11.4 }

\textbf{TS 6.4.11.4 } \newline
\textbf{Samhita Paata} \newline

तत् प्र॒जा अ॒भि जि॑घ्रति॒ तस्मा᳚द् व॒थ्सं जा॒तं गौर॒भि जि॑घ्रत्या॒त्मा वा ए॒ष य॒ज्ञ्स्य॒ यदा᳚ग्रय॒णः सव॑नेसवने॒ऽभि गृ॑ह्णात्या॒त्मन्ने॒व य॒ज्ञ्ꣳ सं त॑नोत्यु॒परि॑ष्टा॒दा न॑यति॒ रेत॑ ए॒व तद् द॑धात्य॒धस्ता॒दुप॑ गृह्णाति॒ प्र ज॑नयत्ये॒व तद्ब्र॑ह्मवा॒दिनो॑ वदन्ति॒ कस्मा᳚थ् स॒त्याद्-गा॑य॒त्री कनि॑ष्ठा॒ छन्द॑साꣳ स॒ती सर्वा॑णि॒ सव॑नानि वह॒तीत्ये॒ ( )-ष वै गा॑यत्रि॒यै व॒थ्सो यदा᳚ग्रय॒णस्तमे॒व तद॑भिनि॒वर्तꣳ॒॒ सर्वा॑णि॒ सव॑नानि वहति॒ तस्मा᳚द् व॒थ्सम॒पाकृ॑तं॒ गौर॒भि नि व॑र्तते ॥ \newline

\textbf{Pada Paata} \newline

तत् । प्र॒जा इति॑ प्र - जाः । अ॒भीति॑ । जि॒घ्र॒ति॒ । तस्मा᳚त् । व॒थ्सम् । जा॒तम् । गौः । अ॒भीति॑ । जि॒घ्र॒ति॒ । आ॒त्मा । वै । ए॒षः । य॒ज्ञ्स्य॑ । यत् । आ॒ग्र॒य॒णः । सव॑नेसवन॒ इति॒ सव॑ने - स॒व॒ने॒ । अ॒भीति॑ । गृ॒ह्णा॒ति॒ । आ॒त्मन्न् । ए॒व । य॒ज्ञ्म् । समिति॑ । त॒नो॒ति॒ । उ॒परि॑ष्टात् । एति॑ । न॒य॒ति॒ । रेतः॑ । ए॒व । तत् । द॒धा॒ति॒ । अ॒धस्ता᳚त् । उपेति॑ । गृ॒ह्णा॒ति॒ । प्रेति॑ । ज॒न॒य॒ति॒ । ए॒व । तत् । ब्र॒ह्म॒वा॒दिन॒ इति॑ ब्रह्म - वा॒दिनः॑ । व॒द॒न्ति॒ । कस्मा᳚त् । स॒त्यात् । गा॒य॒त्री । कनि॑ष्ठा । छन्द॑साम् । स॒ती । सर्वा॑णि । सव॑नानि । व॒ह॒ति॒ । इति॑ ( ) । ए॒षः । वै । गा॒य॒त्रि॒यै । व॒थ्सः । यत् । आ॒ग्र॒य॒णः । तम् । ए॒व । तत् । अ॒भि॒नि॒वर्त॒मित्य॑भि - नि॒वर्त᳚म् । सर्वा॑णि । सव॑नानि । व॒ह॒ति॒ । तस्मा᳚त् । व॒थ्सम् । अ॒पाकृ॑त॒मित्य॑प - आकृ॑तम् । गौः । अ॒भ । नीति॑ । व॒र्त॒ते॒ ॥  \newline


\textbf{Krama Paata} \newline

तत् प्र॒जाः । प्र॒जा अ॒भि । प्र॒जा इति॑ प्र - जाः । अ॒भि जि॑घ्रति । जि॒घ्र॒ति॒ तस्मा᳚त् । तस्मा᳚द् व॒थ्सम् । व॒थ्सम् जा॒तम् । जा॒तम् गौः । गौर॒भि । अ॒भि जि॑घ्रति । जि॒घ्र॒त्या॒त्मा । आ॒त्मा वै । वा ए॒षः । ए॒ष य॒ज्ञ्स्य॑ । य॒ज्ञ्स्य॒ यत् । यदा᳚ग्रय॒णः । आ॒ग्र॒य॒णः सव॑नेसवने । सव॑नेसवने॒ऽभि । सव॑नेसवन॒ इति॒ सव॑ने - स॒व॒ने॒ । अ॒भि गृ॑ह्णाति । गृ॒ह्णा॒त्या॒त्मन्न् । आ॒त्मन्ने॒व । ए॒व य॒ज्ञ्म् । य॒ज्ञ्ꣳ सम् । सम् त॑नोति । त॒नो॒त्यु॒परि॑ष्टात् । उ॒परि॑ष्टा॒दा । आ न॑यति । न॒य॒ति॒ रेतः॑ । रेत॑ ए॒व । ए॒व तत् । तद् द॑धाति । द॒धा॒त्य॒धस्ता᳚त् । अ॒धस्ता॒दुप॑ । उप॑ गृह्णाति । गृ॒ह्णा॒ति॒ प्र । प्र ज॑नयति । ज॒न॒य॒त्ये॒व । ए॒व तत् । तद् ब्र॑ह्मवा॒दिनः॑ । ब्र॒ह्म॒वा॒दिनो॑ वदन्ति । ब्र॒ह्म॒वा॒दिन॒ इति॑ ब्रह्म - वा॒दिनः॑ । व॒द॒न्ति॒ कस्मा᳚त् । कस्मा᳚थ् स॒त्यात् । स॒त्याद् गा॑य॒त्री । गा॒य॒त्री कनि॑ष्ठा । कनि॑ष्ठा॒ छन्द॑साम् । छन्द॑साꣳ स॒ती । स॒ती सर्वा॑णि । सर्वा॑णि॒ सव॑नानि । सव॑नानि वहति । व॒ह॒तीति॑ ( ) । इत्ये॒षः । ए॒ष वै । वै गा॑यत्रि॒यै । गा॒य॒त्रि॒यै व॒थ्सः । व॒थ्सो यत् । यदा᳚ग्रय॒णः । आ॒ग्र॒य॒णस्तम् । तमे॒व । ए॒व तत् । तद॑भिनि॒वर्त᳚म् । अ॒भि॒नि॒वर्तꣳ॒॒ सर्वा॑णि । अ॒भि॒नि॒वर्त॒मित्य॑भि - नि॒वर्त᳚म् । सर्वा॑णि॒ सव॑नानि । सव॑नानि वहति । व॒ह॒ति॒ तस्मा᳚त् । तस्मा᳚द् व॒थ्सम् । व॒थ्सम॒पाकृ॑तम् । अ॒पाकृ॑त॒म् गौः । अ॒पाकृ॑त॒मित्य॑प - आकृ॑तम् । गौर॒भि । अ॒भि नि । नि व॑र्तते । व॒र्त॒त॒ इति॑ वर्तते । \newline

\textbf{Jatai Paata} \newline

1. तत् प्र॒जाः प्र॒जा स्तत् तत् प्र॒जाः । \newline
2. प्र॒जा अ॒भ्य॑भि प्र॒जाः प्र॒जा अ॒भि । \newline
3. प्र॒जा इति॑ प्र - जाः । \newline
4. अ॒भि जि॑घ्रति जिघ्र त्य॒भ्य॑भि जि॑घ्रति । \newline
5. जि॒घ्र॒ति॒ तस्मा॒त् तस्मा᳚ज् जिघ्रति जिघ्रति॒ तस्मा᳚त् । \newline
6. तस्मा᳚द् व॒थ्सं ॅव॒थ्सम् तस्मा॒त् तस्मा᳚द् व॒थ्सम् । \newline
7. व॒थ्सम् जा॒तम् जा॒तं ॅव॒थ्सं ॅव॒थ्सम् जा॒तम् । \newline
8. जा॒तम् गौर् गौर् जा॒तम् जा॒तम् गौः । \newline
9. गौर॒ भ्य॑भि गौर् गौर॒भि । \newline
10. अ॒भि जि॑घ्रति जिघ्र त्य॒भ्य॑भि जि॑घ्रति । \newline
11. जि॒घ्र॒ त्या॒त्मा ऽऽत्मा जि॑घ्रति जिघ्र त्या॒त्मा । \newline
12. आ॒त्मा वै वा आ॒त्मा ऽऽत्मा वै । \newline
13. वा ए॒ष ए॒ष वै वा ए॒षः । \newline
14. ए॒ष य॒ज्ञ्स्य॑ य॒ज्ञ्स्यै॒ष ए॒ष य॒ज्ञ्स्य॑ । \newline
15. य॒ज्ञ्स्य॒ यद् यद् य॒ज्ञ्स्य॑ य॒ज्ञ्स्य॒ यत् । \newline
16. यदा᳚ग्रय॒ण आ᳚ग्रय॒णो यद् यदा᳚ग्रय॒णः । \newline
17. आ॒ग्र॒य॒णः सव॑नेसवने॒ सव॑नेसवन आग्रय॒ण आ᳚ग्रय॒णः सव॑नेसवने । \newline
18. सव॑नेसवने॒ ऽभ्य॑भि सव॑नेसवने॒ सव॑नेसवने॒ ऽभि । \newline
19. सव॑नेसवन॒ इति॒ सव॑ने - स॒व॒ने॒ । \newline
20. अ॒भि गृ॑ह्णाति गृह्णा त्य॒भ्य॑भि गृ॑ह्णाति । \newline
21. गृ॒ह्णा॒ त्या॒त्मन् ना॒त्मन् गृ॑ह्णाति गृह्णा त्या॒त्मन्न् । \newline
22. आ॒त्मन् ने॒वै वात्मन् ना॒त्मन् ने॒व । \newline
23. ए॒व य॒ज्ञ्ं ॅय॒ज्ञ् मे॒वैव य॒ज्ञ्म् । \newline
24. य॒ज्ञ्ꣳ सꣳ सं ॅय॒ज्ञ्ं ॅय॒ज्ञ्ꣳ सम् । \newline
25. सम् त॑नोति तनोति॒ सꣳ सम् त॑नोति । \newline
26. त॒नो॒ त्यु॒परि॑ष्टा दु॒परि॑ष्टात् तनोति तनो त्यु॒परि॑ष्टात् । \newline
27. उ॒परि॑ष्टा॒दो परि॑ष्टा दु॒परि॑ष्टा॒दा । \newline
28. आ न॑यति नय॒त्या न॑यति । \newline
29. न॒य॒ति॒ रेतो॒ रेतो॑ नयति नयति॒ रेतः॑ । \newline
30. रेत॑ ए॒वैव रेतो॒ रेत॑ ए॒व । \newline
31. ए॒व तत् तदे॒ वैव तत् । \newline
32. तद् द॑धाति दधाति॒ तत् तद् द॑धाति । \newline
33. द॒धा॒ त्य॒धस्ता॑ द॒धस्ता᳚द् दधाति दधा त्य॒धस्ता᳚त् । \newline
34. अ॒धस्ता॒ दुपोपा॒ धस्ता॑ द॒धस्ता॒ दुप॑ । \newline
35. उप॑ गृह्णाति गृह्णा॒ त्युपोप॑ गृह्णाति । \newline
36. गृ॒ह्णा॒ति॒ प्र प्र गृ॑ह्णाति गृह्णाति॒ प्र । \newline
37. प्र ज॑नयति जनयति॒ प्र प्र ज॑नयति । \newline
38. ज॒न॒य॒त्ये॒ वैव ज॑नयति जनय त्ये॒व । \newline
39. ए॒व तत् तदे॒ वैव तत् । \newline
40. तद् ब्र॑ह्मवा॒दिनो᳚ ब्रह्मवा॒दिन॒ स्तत् तद् ब्र॑ह्मवा॒दिनः॑ । \newline
41. ब्र॒ह्म॒वा॒दिनो॑ वदन्ति वदन्ति ब्रह्मवा॒दिनो᳚ ब्रह्मवा॒दिनो॑ वदन्ति । \newline
42. ब्र॒ह्म॒वा॒दिन॒ इति॑ ब्रह्म - वा॒दिनः॑ । \newline
43. व॒द॒न्ति॒ कस्मा॒त् कस्मा᳚द् वदन्ति वदन्ति॒ कस्मा᳚त् । \newline
44. कस्मा᳚थ् स॒त्याथ् स॒त्यात् कस्मा॒त् कस्मा᳚थ् स॒त्यात् । \newline
45. स॒त्याद् गा॑य॒त्री गा॑य॒त्री स॒त्याथ् स॒त्याद् गा॑य॒त्री । \newline
46. गा॒य॒त्री कनि॑ष्ठा॒ कनि॑ष्ठा गाय॒त्री गा॑य॒त्री कनि॑ष्ठा । \newline
47. कनि॑ष्ठा॒ छन्द॑सा॒म् छन्द॑सा॒म् कनि॑ष्ठा॒ कनि॑ष्ठा॒ छन्द॑साम् । \newline
48. छन्द॑साꣳ स॒ती स॒ती छन्द॑सा॒म् छन्द॑साꣳ स॒ती । \newline
49. स॒ती सर्वा॑णि॒ सर्वा॑णि स॒ती स॒ती सर्वा॑णि । \newline
50. सर्वा॑णि॒ सव॑नानि॒ सव॑नानि॒ सर्वा॑णि॒ सर्वा॑णि॒ सव॑नानि । \newline
51. सव॑नानि वहति वहति॒ सव॑नानि॒ सव॑नानि वहति । \newline
52. व॒ह॒ती तीति॑ वहति वह॒ तीति॑ । \newline
53. इत्ये॒ष ए॒ष इतीत्ये॒षः । \newline
54. ए॒ष वै वा ए॒ष ए॒ष वै । \newline
55. वै गा॑यत्रि॒यै गा॑यत्रि॒यै वै वै गा॑यत्रि॒यै । \newline
56. गा॒य॒त्रि॒यै व॒थ्सो व॒थ्सो गा॑यत्रि॒यै गा॑यत्रि॒यै व॒थ्सः । \newline
57. व॒थ्सो यद् यद् व॒थ्सो व॒थ्सो यत् । \newline
58. यदा᳚ग्रय॒ण आ᳚ग्रय॒णो यद् यदा᳚ग्रय॒णः । \newline
59. आ॒ग्र॒य॒णस्तम् त मा᳚ग्रय॒ण आ᳚ग्रय॒णस्तम् । \newline
60. त मे॒वैव तम् त मे॒व । \newline
61. ए॒व तत् तदे॒ वैव तत् । \newline
62. तद॑भिनि॒वर्त॑ मभिनि॒वर्त॒म् तत् तद॑भिनि॒वर्त᳚म् । \newline
63. अ॒भि॒नि॒वर्तꣳ॒॒ सर्वा॑णि॒ सर्वा᳚ ण्यभिनि॒वर्त॑ मभिनि॒वर्तꣳ॒॒ सर्वा॑णि । \newline
64. अ॒भि॒नि॒वर्त॒मित्य॑भि - नि॒वर्त᳚म् । \newline
65. सर्वा॑णि॒ सव॑नानि॒ सव॑नानि॒ सर्वा॑णि॒ सर्वा॑णि॒ सव॑नानि । \newline
66. सव॑नानि वहति वहति॒ सव॑नानि॒ सव॑नानि वहति । \newline
67. व॒ह॒ति॒ तस्मा॒त् तस्मा᳚द् वहति वहति॒ तस्मा᳚त् । \newline
68. तस्मा᳚द् व॒थ्सं ॅव॒थ्सम् तस्मा॒त् तस्मा᳚द् व॒थ्सम् । \newline
69. व॒थ्स म॒पाकृ॑त म॒पाकृ॑तं ॅव॒थ्सं ॅव॒थ्स म॒पाकृ॑तम् । \newline
70. अ॒पाकृ॑त॒म् गौर् गौ र॒पाकृ॑त म॒पाकृ॑त॒म् गौः । \newline
71. अ॒पाकृ॑त॒मित्य॑प - आकृ॑तम् । \newline
72. गौर॒ भ्य॑भि गौर् गौर॒भि । \newline
73. अ॒भि नि न्या᳚(1॒)भ्य॑भि नि । \newline
74. नि व॑र्तते वर्तते॒ नि नि व॑र्तते । \newline
75. व॒र्त॒त॒ इति॑ वर्तते । \newline

\textbf{Ghana Paata } \newline

1. तत् प्र॒जाः प्र॒जा स्तत् तत् प्र॒जा अ॒भ्य॑भि प्र॒जा स्तत् तत् प्र॒जा अ॒भि । \newline
2. प्र॒जा अ॒भ्य॑भि प्र॒जाः प्र॒जा अ॒भि जि॑घ्रति जिघ्र त्य॒भि प्र॒जाः प्र॒जा अ॒भि जि॑घ्रति । \newline
3. प्र॒जा इति॑ प्र - जाः । \newline
4. अ॒भि जि॑घ्रति जिघ्र त्य॒भ्य॑भि जि॑घ्रति॒ तस्मा॒त् तस्मा᳚ज् जिघ्र त्य॒भ्य॑भि जि॑घ्रति॒ तस्मा᳚त् । \newline
5. जि॒घ्र॒ति॒ तस्मा॒त् तस्मा᳚ज् जिघ्रति जिघ्रति॒ तस्मा᳚द् व॒थ्सं ॅव॒थ्सम् तस्मा᳚ज् जिघ्रति जिघ्रति॒ तस्मा᳚द् व॒थ्सम् । \newline
6. तस्मा᳚द् व॒थ्सं ॅव॒थ्सम् तस्मा॒त् तस्मा᳚द् व॒थ्सम् जा॒तम् जा॒तं ॅव॒थ्सम् तस्मा॒त् तस्मा᳚द् व॒थ्सम् जा॒तम् । \newline
7. व॒थ्सम् जा॒तम् जा॒तं ॅव॒थ्सं ॅव॒थ्सम् जा॒तम् गौर् गौर् जा॒तं ॅव॒थ्सं ॅव॒थ्सम् जा॒तम् गौः । \newline
8. जा॒तम् गौर् गौर् जा॒तम् जा॒तम् गौर॒भ्य॑भि गौर् जा॒तम् जा॒तम् गौर॒भि । \newline
9. गौर॒ भ्य॑भि गौर् गौर॒भि जि॑घ्रति जिघ्र त्य॒भि गौर् गौर॒भि जि॑घ्रति । \newline
10. अ॒भि जि॑घ्रति जिघ्र त्य॒भ्य॑भि जि॑घ्र त्या॒त्मा ऽऽत्मा जि॑घ्र त्य॒भ्य॑भि जि॑घ्र त्या॒त्मा । \newline
11. जि॒घ्र॒ त्या॒त्मा ऽऽत्मा जि॑घ्रति जिघ्र त्या॒त्मा वै वा आ॒त्मा जि॑घ्रति जिघ्र त्या॒त्मा वै । \newline
12. आ॒त्मा वै वा आ॒त्मा ऽऽत्मा वा ए॒ष ए॒ष वा आ॒त्मा ऽऽत्मा वा ए॒षः । \newline
13. वा ए॒ष ए॒ष वै वा ए॒ष य॒ज्ञ्स्य॑ य॒ज्ञ् स्यै॒ष वै वा ए॒ष य॒ज्ञ्स्य॑ । \newline
14. ए॒ष य॒ज्ञ्स्य॑ य॒ज्ञ् स्यै॒ष ए॒ष य॒ज्ञ्स्य॒ यद् यद् य॒ज्ञ् स्यै॒ष ए॒ष य॒ज्ञ्स्य॒ यत् । \newline
15. य॒ज्ञ्स्य॒ यद् यद् य॒ज्ञ्स्य॑ य॒ज्ञ्स्य॒ यदा᳚ग्रय॒ण आ᳚ग्रय॒णो यद् य॒ज्ञ्स्य॑ य॒ज्ञ्स्य॒ यदा᳚ग्रय॒णः । \newline
16. यदा᳚ग्रय॒ण आ᳚ग्रय॒णो यद् यदा᳚ग्रय॒णः सव॑नेसवने॒ सव॑नेसवन आग्रय॒णो यद् यदा᳚ग्रय॒णः सव॑नेसवने । \newline
17. आ॒ग्र॒य॒णः सव॑नेसवने॒ सव॑नेसवन आग्रय॒ण आ᳚ग्रय॒णः सव॑नेसवने॒ ऽभ्य॑भि सव॑नेसवन आग्रय॒ण आ᳚ग्रय॒णः सव॑नेसवने॒ ऽभि । \newline
18. सव॑नेसवने॒ ऽभ्य॑भि सव॑नेसवने॒ सव॑नेसवने॒ ऽभि गृ॑ह्णाति गृह्णा त्य॒भि सव॑नेसवने॒ सव॑नेसवने॒ ऽभि गृ॑ह्णाति । \newline
19. सव॑नेसवन॒ इति॒ सव॑ने - स॒व॒ने॒ । \newline
20. अ॒भि गृ॑ह्णाति गृह्णा त्य॒भ्य॑भि गृ॑ह्णा त्या॒त्मन् ना॒त्मन् गृ॑ह्णा त्य॒भ्य॑भि गृ॑ह्णा त्या॒त्मन्न् । \newline
21. गृ॒ह्णा॒ त्या॒त्मन् ना॒त्मन् गृ॑ह्णाति गृह्णा त्या॒त्मन् ने॒वै वात्मन् गृ॑ह्णाति गृह्णा त्या॒त्मन् ने॒व । \newline
22. आ॒त्मन् ने॒वै वात्मन् ना॒त्मन् ने॒व य॒ज्ञ्ं ॅय॒ज्ञ् मे॒वात्मन् ना॒त्मन् ने॒व य॒ज्ञ्म् । \newline
23. ए॒व य॒ज्ञ्ं ॅय॒ज्ञ् मे॒वैव य॒ज्ञ्ꣳ सꣳ सं ॅय॒ज्ञ् मे॒वैव य॒ज्ञ्ꣳ सम् । \newline
24. य॒ज्ञ्ꣳ सꣳ सं ॅय॒ज्ञ्ं ॅय॒ज्ञ्ꣳ सम् त॑नोति तनोति॒ सं ॅय॒ज्ञ्ं ॅय॒ज्ञ्ꣳ सम् त॑नोति । \newline
25. सम् त॑नोति तनोति॒ सꣳ सम् त॑नो त्यु॒परि॑ष्टा दु॒परि॑ष्टात् तनोति॒ सꣳ सम् त॑नो त्यु॒परि॑ष्टात् । \newline
26. त॒नो॒ त्यु॒परि॑ष्टा दु॒परि॑ष्टात् तनोति तनो त्यु॒परि॑ष्टा॒ दोपरि॑ष्टात् तनोति तनो त्यु॒परि॑ष्टा॒दा । \newline
27. उ॒परि॑ष्टा॒ दोपरि॑ष्टा दु॒परि॑ष्टा॒दा न॑यति नय॒ त्योपरि॑ष्टा दु॒परि॑ष्टा॒दा न॑यति । \newline
28. आ न॑यति नय॒त्या न॑यति॒ रेतो॒ रेतो॑ नय॒त्या न॑यति॒ रेतः॑ । \newline
29. न॒य॒ति॒ रेतो॒ रेतो॑ नयति नयति॒ रेत॑ ए॒वैव रेतो॑ नयति नयति॒ रेत॑ ए॒व । \newline
30. रेत॑ ए॒वैव रेतो॒ रेत॑ ए॒व तत् तदे॒व रेतो॒ रेत॑ ए॒व तत् । \newline
31. ए॒व तत् तदे॒ वैव तद् द॑धाति दधाति॒ तदे॒ वैव तद् द॑धाति । \newline
32. तद् द॑धाति दधाति॒ तत् तद् द॑धा त्य॒धस्ता॑ द॒धस्ता᳚द् दधाति॒ तत् तद् द॑धा त्य॒धस्ता᳚त् । \newline
33. द॒धा॒ त्य॒धस्ता॑ द॒धस्ता᳚द् दधाति दधा त्य॒धस्ता॒ दुपो पा॒धस्ता᳚द् दधाति दधा त्य॒धस्ता॒ दुप॑ । \newline
34. अ॒धस्ता॒ दुपो पा॒धस्ता॑ द॒धस्ता॒ दुप॑ गृह्णाति गृह्णा॒ त्युपा॒धस्ता॑ द॒ध स्ता॒दुप॑ गृह्णाति । \newline
35. उप॑ गृह्णाति गृह्णा॒ त्युपोप॑ गृह्णाति॒ प्र प्र गृ॑ह्णा॒ त्युपोप॑ गृह्णाति॒ प्र । \newline
36. गृ॒ह्णा॒ति॒ प्र प्र गृ॑ह्णाति गृह्णाति॒ प्र ज॑नयति जनयति॒ प्र गृ॑ह्णाति गृह्णाति॒ प्र ज॑नयति । \newline
37. प्र ज॑नयति जनयति॒ प्र प्र ज॑नय त्ये॒वैव ज॑नयति॒ प्र प्र ज॑नय त्ये॒व । \newline
38. ज॒न॒य॒ त्ये॒वैव ज॑नयति जनय त्ये॒व तत् तदे॒व ज॑नयति जनय त्ये॒व तत् । \newline
39. ए॒व तत् तदे॒वैव तद् ब्र॑ह्मवा॒दिनो᳚ ब्रह्मवा॒दिन॒ स्तदे॒वैव तद् ब्र॑ह्मवा॒दिनः॑ । \newline
40. तद् ब्र॑ह्मवा॒दिनो᳚ ब्रह्मवा॒दिन॒ स्तत् तद् ब्र॑ह्मवा॒दिनो॑ वदन्ति वदन्ति ब्रह्मवा॒दिन॒ स्तत् तद् ब्र॑ह्मवा॒दिनो॑ वदन्ति । \newline
41. ब्र॒ह्म॒वा॒दिनो॑ वदन्ति वदन्ति ब्रह्मवा॒दिनो᳚ ब्रह्मवा॒दिनो॑ वदन्ति॒ कस्मा॒त् कस्मा᳚द् वदन्ति ब्रह्मवा॒दिनो᳚ ब्रह्मवा॒दिनो॑ वदन्ति॒ कस्मा᳚त् । \newline
42. ब्र॒ह्म॒वा॒दिन॒ इति॑ ब्रह्म - वा॒दिनः॑ । \newline
43. व॒द॒न्ति॒ कस्मा॒त् कस्मा᳚द् वदन्ति वदन्ति॒ कस्मा᳚थ् स॒त्याथ् स॒त्यात् कस्मा᳚द् वदन्ति वदन्ति॒ कस्मा᳚थ् स॒त्यात् । \newline
44. कस्मा᳚थ् स॒त्याथ् स॒त्यात् कस्मा॒त् कस्मा᳚थ् स॒त्याद् गा॑य॒त्री गा॑य॒त्री स॒त्यात् कस्मा॒त् कस्मा᳚थ् स॒त्याद् गा॑य॒त्री । \newline
45. स॒त्याद् गा॑य॒त्री गा॑य॒त्री स॒त्याथ् स॒त्याद् गा॑य॒त्री कनि॑ष्ठा॒ कनि॑ष्ठा गाय॒त्री स॒त्याथ् स॒त्याद् गा॑य॒त्री कनि॑ष्ठा । \newline
46. गा॒य॒त्री कनि॑ष्ठा॒ कनि॑ष्ठा गाय॒त्री गा॑य॒त्री कनि॑ष्ठा॒ छन्द॑सा॒म् छन्द॑सा॒म् कनि॑ष्ठा गाय॒त्री गा॑य॒त्री कनि॑ष्ठा॒ छन्द॑साम् । \newline
47. कनि॑ष्ठा॒ छन्द॑सा॒म् छन्द॑सा॒म् कनि॑ष्ठा॒ कनि॑ष्ठा॒ छन्द॑साꣳ स॒ती स॒ती छन्द॑सा॒म् कनि॑ष्ठा॒ कनि॑ष्ठा॒ छन्द॑साꣳ स॒ती । \newline
48. छन्द॑साꣳ स॒ती स॒ती छन्द॑सा॒म् छन्द॑साꣳ स॒ती सर्वा॑णि॒ सर्वा॑णि स॒ती छन्द॑सा॒म् छन्द॑साꣳ स॒ती सर्वा॑णि । \newline
49. स॒ती सर्वा॑णि॒ सर्वा॑णि स॒ती स॒ती सर्वा॑णि॒ सव॑नानि॒ सव॑नानि॒ सर्वा॑णि स॒ती स॒ती सर्वा॑णि॒ सव॑नानि । \newline
50. सर्वा॑णि॒ सव॑नानि॒ सव॑नानि॒ सर्वा॑णि॒ सर्वा॑णि॒ सव॑नानि वहति वहति॒ सव॑नानि॒ सर्वा॑णि॒ सर्वा॑णि॒ सव॑नानि वहति । \newline
51. सव॑नानि वहति वहति॒ सव॑नानि॒ सव॑नानि वह॒ती तीति॑ वहति॒ सव॑नानि॒ सव॑नानि वह॒ तीति॑ । \newline
52. व॒ह॒ती तीति॑ वहति वह॒ती त्ये॒ष ए॒ष इति॑ वहति वह॒ती त्ये॒षः । \newline
53. इत्ये॒ष ए॒ष इती त्ये॒ष वै वा ए॒ष इती त्ये॒ष वै । \newline
54. ए॒ष वै वा ए॒ष ए॒ष वै गा॑यत्रि॒यै गा॑यत्रि॒यै वा ए॒ष ए॒ष वै गा॑यत्रि॒यै । \newline
55. वै गा॑यत्रि॒यै गा॑यत्रि॒यै वै वै गा॑यत्रि॒यै व॒थ्सो व॒थ्सो गा॑यत्रि॒यै वै वै गा॑यत्रि॒यै व॒थ्सः । \newline
56. गा॒य॒त्रि॒यै व॒थ्सो व॒थ्सो गा॑यत्रि॒यै गा॑यत्रि॒यै व॒थ्सो यद् यद् व॒थ्सो गा॑यत्रि॒यै गा॑यत्रि॒यै व॒थ्सो यत् । \newline
57. व॒थ्सो यद् यद् व॒थ्सो व॒थ्सो यदा᳚ग्रय॒ण आ᳚ग्रय॒णो यद् व॒थ्सो व॒थ्सो यदा᳚ग्रय॒णः । \newline
58. यदा᳚ग्रय॒ण आ᳚ग्रय॒णो यद् यदा᳚ग्रय॒ण स्तम् त मा᳚ग्रय॒णो यद् यदा᳚ग्रय॒ण स्तम् । \newline
59. आ॒ग्र॒य॒ण स्तम् त मा᳚ग्रय॒ण आ᳚ग्रय॒ण स्त मे॒वैव त मा᳚ग्रय॒ण आ᳚ग्रय॒ण स्त मे॒व । \newline
60. त मे॒वैव तम् त मे॒व तत् तदे॒व तम् त मे॒व तत् । \newline
61. ए॒व तत् तदे॒ वैव तद॑भिनि॒वर्त॑ मभिनि॒वर्त॒म् तदे॒ वैव तद॑भिनि॒वर्त᳚म् । \newline
62. तद॑भिनि॒वर्त॑ मभिनि॒वर्त॒म् तत् तद॑भिनि॒वर्तꣳ॒॒ सर्वा॑णि॒ सर्वा᳚ ण्यभिनि॒वर्त॒म् तत् तद॑भिनि॒वर्तꣳ॒॒ सर्वा॑णि । \newline
63. अ॒भि॒नि॒वर्तꣳ॒॒ सर्वा॑णि॒ सर्वा᳚ ण्यभिनि॒वर्त॑ मभिनि॒वर्तꣳ॒॒ सर्वा॑णि॒ सव॑नानि॒ सव॑नानि॒ सर्वा᳚
ण्यभिनि॒वर्त॑ मभिनि॒वर्तꣳ॒॒ सर्वा॑णि॒ सव॑नानि । \newline
64. अ॒भि॒नि॒वर्त॒मित्य॑भि - नि॒वर्त᳚म् । \newline
65. सर्वा॑णि॒ सव॑नानि॒ सव॑नानि॒ सर्वा॑णि॒ सर्वा॑णि॒ सव॑नानि वहति वहति॒ सव॑नानि॒ सर्वा॑णि॒ सर्वा॑णि॒ सव॑नानि वहति । \newline
66. सव॑नानि वहति वहति॒ सव॑नानि॒ सव॑नानि वहति॒ तस्मा॒त् तस्मा᳚द् वहति॒ सव॑नानि॒ सव॑नानि वहति॒ तस्मा᳚त् । \newline
67. व॒ह॒ति॒ तस्मा॒त् तस्मा᳚द् वहति वहति॒ तस्मा᳚द् व॒थ्सं ॅव॒थ्सम् तस्मा᳚द् वहति वहति॒ तस्मा᳚द् व॒थ्सम् । \newline
68. तस्मा᳚द् व॒थ्सं ॅव॒थ्सम् तस्मा॒त् तस्मा᳚द् व॒थ्स म॒पाकृ॑त म॒पाकृ॑तं ॅव॒थ्सम् तस्मा॒त् तस्मा᳚द् व॒थ्स म॒पाकृ॑तम् । \newline
69. व॒थ्स म॒पाकृ॑त म॒पाकृ॑तं ॅव॒थ्सं ॅव॒थ्स म॒पाकृ॑त॒म् गौर् गौ र॒पाकृ॑तं ॅव॒थ्सं ॅव॒थ्स म॒पाकृ॑त॒म् गौः । \newline
70. अ॒पाकृ॑त॒म् गौर् गौ र॒पाकृ॑त म॒पाकृ॑त॒म् गौ र॒भ्य॑भि गौ र॒पाकृ॑त म॒पाकृ॑त॒म् गौ र॒भि । \newline
71. अ॒पाकृ॑त॒मित्य॑प - आकृ॑तम् । \newline
72. गौ र॒भ्य॑भि गौर् गौ र॒भि नि न्य॑भि गौर् गौ र॒भि नि । \newline
73. अ॒भि नि न्या᳚(1॒)भ्य॑भि नि व॑र्तते वर्तते॒ न्या᳚(1॒)भ्य॑भि नि व॑र्तते । \newline
74. नि व॑र्तते वर्तते॒ नि नि व॑र्तते । \newline
75. व॒र्त॒त॒ इति॑ वर्तते । \newline
\pagebreak


\end{document}