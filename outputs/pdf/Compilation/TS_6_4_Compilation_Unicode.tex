\documentclass[17pt]{extarticle}
\usepackage{babel}
\usepackage{fontspec}
\usepackage{polyglossia}
\usepackage{extsizes}



\setmainlanguage{sanskrit}
\setotherlanguages{english} %% or other languages
\setlength{\parindent}{0pt}
\pagestyle{myheadings}
\newfontfamily\devanagarifont[Script=Devanagari]{AdishilaVedic}


\newcommand{\VAR}[1]{}
\newcommand{\BLOCK}[1]{}




\begin{document}
\begin{titlepage}
    \begin{center}
 
\begin{sanskrit}
    { \Huge
    कृष्ण यजुर्वेदीय तैत्तिरीय संहिता,पद,जटा,घन पाठः 
    }
    \\
    \vspace{2.5cm}
    \mbox{ \Huge
    6.4       षष्ठकाण्डे चतुर्थः प्रश्नः - सोममन्त्रब्राह्मणनिरूपणं   }
\end{sanskrit}
\end{center}

\end{titlepage}
\tableofcontents
\pagebreak

\markright{ TS 6.4.1.1  \hfill https://www.vedavms.in \hfill}
\addcontentsline{toc}{section}{ TS 6.4.1.1 }
\section*{ TS 6.4.1.1 }

\textbf{TS 6.4.1.1 } \newline
\textbf{Samhita Paata} \newline

य॒ज्ञेन॒ वै प्र॒जाप॑तिः प्र॒जा अ॑सृजत॒ ता उ॑प॒यड्भि॑-रे॒वासृ॑जत॒ यदु॑प॒यज॑ उप॒यज॑ति प्र॒जा ए॒व तद् यज॑मानः सृजते जघना॒र्द्धादव॑ द्यति जघना॒र्द्धाद्धि प्र॒जाः प्र॒जाय॑न्ते स्थविम॒तोऽव॑ द्यति स्थविम॒तो हि प्र॒जाः प्र॒जाय॒न्ते ऽस॑भिंन्द॒न्नव॑ द्यति प्रा॒णाना॒-मस॑भेंदाय॒ न प॒र्याव॑र्तयति॒ यत् प॑र्याव॒र्तये॑दुदाव॒र्तः प्र॒जा ग्राहु॑कः स्याथ् समु॒द्रं ग॑च्छ॒ स्वाहेत्या॑ह॒ रेत॑- [  ] \newline

\textbf{Pada Paata} \newline

य॒ज्ञेन॑ । वै । प्र॒जाप॑ति॒रिति॑ प्र॒जा - प॒तिः॒ । प्र॒जा इति॑ प्र - जाः । अ॒सृ॒ज॒त॒ । ताः । उ॒प॒यड्भि॒रित्यु॑प॒यट्-भिः॒ । ए॒व । अ॒सृ॒ज॒त॒ । यत् । उ॒प॒यज॒ इत्यु॑प - यजः॑ । उ॒प॒यज॒तीत्यु॑प - यज॑ति । प्र॒जा इति॑ प्र - जाः । ए॒व । तत् । यज॑मानः । सृ॒ज॒ते॒ । ज॒घ॒ना॒द्‌र्धादिति॑ जघन - अ॒द्‌र्धात् । अवेति॑ । द्य॒ति॒ । ज॒घ॒ना॒द्‌र्धादिति॑ जघन-अ॒द्‌र्धात् । हि । प्र॒जा इति॑ प्र - जाः । प्र॒जाय॑न्त॒ इति॑ प्र - जाय॑न्ते । स्थ॒वि॒म॒तः । अवेति॑ । द्य॒ति॒ । स्थ॒वि॒म॒तः । हि । प्र॒जा इति॑ प्र-जाः । प्र॒जाय॑न्त॒ इति॑ प्र-जाय॑न्ते । अस॑भिंन्द॒न्नित्यसं᳚ - भि॒न्द॒न्न् । अवेति॑ । द्य॒ति॒ । प्रा॒णाना॒मिति॑ प्र - अ॒नाना᳚म् । अस॑भेंदा॒येत्यसं᳚ - भे॒दा॒य॒ । न । प॒र्याव॑र्तय॒तीति॑ परि - आव॑र्तयति । यत् । प॒र्या॒व॒र्तये॒दिति॑ परि - आ॒व॒र्तये᳚त् । उ॒दा॒व॒र्तः । प्र॒जा इति॑ प्र - जाः । ग्राहु॑कः । स्या॒त् । स॒मु॒द्रम् । ग॒च्छ॒ । स्वाहा᳚ । इति॑ । आ॒ह॒ । रेतः॑ ।  \newline




\markright{ TS 6.4.1.2  \hfill https://www.vedavms.in \hfill}
\addcontentsline{toc}{section}{ TS 6.4.1.2 }
\section*{ TS 6.4.1.2 }

\textbf{TS 6.4.1.2 } \newline
\textbf{Samhita Paata} \newline

ए॒व तद् द॑धात्य॒न्तरि॑क्षं गच्छ॒ स्वाहेत्या॑हा॒ऽन्तरि॑क्षेणै॒वास्मै᳚ प्र॒जाः प्र ज॑नयत्य॒न्तरि॑क्षꣳ॒॒ ह्यनु॑ प्र॒जाः प्र॒जाय॑न्ते दे॒वꣳ स॑वि॒तारं॑ गच्छ॒ स्वाहेत्या॑ह सवि॒तृप्र॑सूत ए॒वास्मै᳚ प्र॒जाः प्र ज॑नयत्य-होरा॒त्रे ग॑च्छ॒ स्वाहेत्या॑हा-होरा॒त्राभ्या॑-मे॒वास्मै᳚ प्र॒जाः प्र ज॑नयत्य-होरा॒त्रे ह्यनु॑ प्र॒जाः प्र॒जाय॑न्ते मि॒त्रावरु॑णौ गच्छ॒ स्वाहे- [  ] \newline

\textbf{Pada Paata} \newline

ए॒व । तत् । द॒धा॒ति॒ । अ॒न्तरि॑क्षम् । ग॒च्छ॒ । स्वाहा᳚ । इति॑ । आ॒ह॒ । अ॒न्तरि॑क्षेण । ए॒व । अ॒स्मै॒ । प्र॒जा इति॑ प्र - जाः । प्रेति॑ । ज॒न॒य॒ति॒ । अ॒न्तरि॑क्षम् । हि । अन्विति॑ । प्र॒जा इति॑ प्र - जाः । प्र॒जाय॑न्त॒ इति॑ प्र - जाय॑न्ते । दे॒वम् । स॒वि॒तार᳚म् । ग॒च्छ॒ । स्वाहा᳚ । इति॑ । आ॒ह॒ । स॒वि॒तृप्र॑सूत॒ इति॑ सवि॒तृ - प्र॒सू॒तः॒ । ए॒व । अ॒स्मै॒ । प्र॒जा इति॑ प्र - जाः । प्रेति॑ । ज॒न॒य॒ति॒ । अ॒हो॒रा॒त्रे इत्य॑हः - रा॒त्रे । ग॒च्छ॒ । स्वाहा᳚ । इति॑ । आ॒ह॒ । अ॒हो॒रा॒त्राभ्या॒मित्य॑हः - रा॒त्राभ्या᳚म् । ए॒व । अ॒स्मै॒ । प्र॒जा इति॑ प्र-जाः । प्रेति॑ । ज॒न॒य॒ति॒ । अ॒हो॒रा॒त्रे इत्य॑हः-रा॒त्रे । हि । अन्विति॑ । प्र॒जा इति॑ प्र - जाः । प्र॒जाय॑न्त॒ इति॑ प्र-जाय॑न्ते । मि॒त्रावरु॑णा॒विति॑ मि॒त्रा-वरु॑णौ । ग॒च्छ॒ । स्वाहा᳚ ।  \newline




\markright{ TS 6.4.1.3  \hfill https://www.vedavms.in \hfill}
\addcontentsline{toc}{section}{ TS 6.4.1.3 }
\section*{ TS 6.4.1.3 }

\textbf{TS 6.4.1.3 } \newline
\textbf{Samhita Paata} \newline

-त्या॑ह प्र॒जास्वे॒व प्रजा॑तासु प्राणापा॒नौ द॑धाति॒ सोमं॑ गच्छ॒ स्वाहेत्या॑ह सौ॒म्या हि दे॒वत॑या प्र॒जा य॒ज्ञ्ं ग॑च्छ॒ स्वाहेत्या॑ह प्र॒जा ए॒व य॒ज्ञियाः᳚ करोति॒ छन्दाꣳ॑सि गच्छ॒ स्वाहेत्या॑ह प॒शवो॒ वै छन्दाꣳ॑सि प॒शूने॒वाव॑ रुन्धे॒ द्यावा॑पृथि॒वी ग॑च्छ॒ स्वाहेत्या॑ह प्र॒जा ए॒व प्रजा॑ता॒ द्यावा॑पृथि॒वीभ्या॑मुभ॒यतः॒ परि॑ गृह्णाति॒ नभो॑- [  ] \newline

\textbf{Pada Paata} \newline

इति॑ । आ॒ह॒ । प्र॒जास्विति॑ प्र - जासु॑ । ए॒व । प्रजा॑ता॒स्विति॒ प्र - जा॒ता॒सु॒ । प्रा॒णा॒पा॒नाविति॑ प्राण - अ॒पा॒नौ । द॒धा॒ति॒ । सोम᳚म् । ग॒च्छ॒ । स्वाहा᳚ । इति॑ । आ॒ह॒ । सौ॒म्याः । हि । दे॒वत॑या । प्र॒जा इति॑ प्र - जाः । य॒ज्ञ्म् । ग॒च्छ॒ । स्वाहा᳚ । इति॑ । आ॒ह॒ । प्र॒जा इति॑ प्र-जाः । ए॒व । य॒ज्ञियाः᳚ । क॒रो॒ति॒ । छन्दाꣳ॑सि । ग॒च्छ॒ । स्वाहा᳚ । इति॑ । आ॒ह॒ । प॒शवः॑ । वै । छन्दाꣳ॑सि । प॒शून् । ए॒व । अवेति॑ । रु॒न्धे॒ । द्यावा॑पृथि॒वी इति॒ द्यावा᳚-पृ॒थि॒वी । ग॒च्छ॒ । स्वाहा᳚ । इति॑ । आ॒ह॒ । प्र॒जा इति॑ प्र - जाः । ए॒व । प्रजा॑ता॒ इति॒ प्र-जा॒ताः॒ । द्यावा॑पृथि॒वीभ्या॒मिति॒ द्यावा᳚ - पृ॒थि॒वीभ्या᳚म् । उ॒भ॒यतः॑ । परीति॑ । गृ॒ह्णा॒ति॒ । नभः॑ ।  \newline




\markright{ TS 6.4.1.4  \hfill https://www.vedavms.in \hfill}
\addcontentsline{toc}{section}{ TS 6.4.1.4 }
\section*{ TS 6.4.1.4 }

\textbf{TS 6.4.1.4 } \newline
\textbf{Samhita Paata} \newline

दि॒व्यं ग॑च्छ॒ स्वाहेत्या॑ह प्र॒जाभ्य॑ ए॒व प्रजा॑ताभ्यो॒ वृष्टिं॒ निय॑च्छत्य॒ग्निं ॅवै᳚श्वान॒रं ग॑च्छ॒ स्वाहेत्या॑ह प्र॒जा ए॒व प्रजा॑ता अ॒स्यां प्रति॑ ष्ठापयति प्रा॒णानां॒ ॅवा ए॒षोऽव॑ द्यति॒ यो॑ऽव॒द्यति॑ गु॒दस्य॒ मनो॑ मे॒ हार्दि॑ य॒च्छेत्या॑ह प्रा॒णाने॒व य॑थास्था॒नमुप॑ ह्वयते प॒शोर्वा आल॑ब्धस्य॒ हृद॑यꣳ॒॒ शुगृ॑च्छति॒ सा हृ॑दयशू॒ल- [  ] \newline

\textbf{Pada Paata} \newline

दि॒व्यम् । ग॒च्छ॒ । स्वाहा᳚ । इति॑ । आ॒ह॒ । प्र॒जाभ्य॒ इति॑ प्र-जाभ्यः॑ । ए॒व । प्रजा॑ताभ्य॒ इति॒ प्र - जा॒ता॒भ्यः॒ । वृष्टि᳚म् । नीति॑ । य॒च्छ॒ति॒ । अ॒ग्निम् । वै॒श्वा॒न॒रम् । ग॒च्छ॒ । स्वाहा᳚ । इति॑ । आ॒ह॒ । प्र॒जा इति॑ प्र - जाः । ए॒व । प्रजा॑ता॒ इति॒ प्र - जा॒ताः॒ । अ॒स्याम् । प्रतीति॑ । स्था॒प॒य॒ति॒ । प्रा॒णाना॒मिति॑ प्र-अ॒नाना᳚म् । वै । ए॒षः । अवेति॑ । द्य॒ति॒ । यः । अ॒व॒द्यतीय॑व - द्यति॑ । गु॒दस्य॑ । मनः॑ । मे॒ । हार्दि॑ । य॒च्छ॒ । इति॑ । आ॒ह॒ । प्रा॒णानिति॑ प्र - अ॒नान् । ए॒व । य॒था॒स्था॒नमिति॑ यथा - स्था॒नम् । उपेति॑ । ह्व॒य॒ते॒ । प॒शोः । वै । आल॑ब्ध॒स्येत्या- ल॒ब्ध॒स्य॒ । हृद॑यम् । शुक् । ऋ॒च्छ॒ति॒ । सा । हृ॒द॒य॒शू॒लमिति॑ हृदय - शू॒लम् ।  \newline




\markright{ TS 6.4.1.5  \hfill https://www.vedavms.in \hfill}
\addcontentsline{toc}{section}{ TS 6.4.1.5 }
\section*{ TS 6.4.1.5 }

\textbf{TS 6.4.1.5 } \newline
\textbf{Samhita Paata} \newline

-म॒भि समे॑ति॒ यत् पृ॑थि॒व्याꣳ हृ॑दयशू॒ल-मु॑द्वा॒सये᳚त् पृथि॒वीꣳ शु॒चाऽर्प॑ये॒द् यद॒फ्स्व॑पः शु॒चाऽर्प॑ये॒च्छुष्क॑स्य चा॒ऽऽ*र्द्रस्य॑ च स॒न्धावुद्वा॑सयत्यु॒भय॑स्य॒ शान्त्यै॒ यं द्वि॒ष्यात् तं ध्या॑ये-च्छु॒चैवैन॑-मर्पयति ॥ \newline

\textbf{Pada Paata} \newline

अ॒भि । समिति॑ । ए॒ति॒ । यत् । पृ॒थि॒व्याम् । हृ॒द॒य॒शू॒लमिति॑ हृदय - शू॒लम् । उ॒द्वा॒सये॒दित्यु॑त् - वा॒सये᳚त् । पृ॒थि॒वीम् । शु॒चा । अ॒र्प॒ये॒त् । यत् । अ॒फ्स्वित्य॑प्- सु । अ॒पः । शु॒चा । अ॒र्प॒ये॒त् । शुष्क॑स्य । च॒ । आ॒र्द्रस्य॑ । च॒ । स॒न्धाविति॑ सं - धौ । उदिति॑ । वा॒स॒य॒ति॒ । उ॒भय॑स्य । शान्त्यै᳚ । यम् । द्वि॒ष्यात् । तम् । ध्या॒ये॒त् । शु॒चा । ए॒व । ए॒न॒म् । अ॒र्प॒य॒ति॒ ॥  \newline




\markright{ TS 6.4.2.1  \hfill https://www.vedavms.in \hfill}
\addcontentsline{toc}{section}{ TS 6.4.2.1 }
\section*{ TS 6.4.2.1 }

\textbf{TS 6.4.2.1 } \newline
\textbf{Samhita Paata} \newline

दे॒वा वै य॒ज्ञ्माग्नी᳚द्ध्रे॒ व्य॑भजन्त॒ ततो॒ यद॒त्यशि॑ष्यत॒ तद॑ब्रुव॒न् वस॑तु॒ नु न॑ इ॒दमिति॒ तद् व॑सती॒वरी॑णां ॅवसती वरि॒त्वं तस्मि॑न् प्रा॒तर्न सम॑शक्नुव॒न् तद॒फ्सु प्रावे॑शय॒न् ता व॑सती॒ वरी॑रभवन् वसती॒वरी᳚र्गृह्णाति य॒ज्ञो वै व॑सती॒ वरी᳚र्य॒ज्ञ्मे॒वाऽऽ*रभ्य॑ गृही॒त्वोप॑ वसति॒ यस्यागृ॑हीता अ॒भि नि॒म्रोचे॒-दना॑रब्धोऽस्य य॒ज्ञ्ः स्या᳚- [  ] \newline

\textbf{Pada Paata} \newline

दे॒वाः । वै । य॒ज्ञ्म् । आग्नी᳚द्ध्र॒ इत्याग्नि॑ - इ॒द्ध्रे॒ । वीति॑ । अ॒भ॒ज॒न्त॒ । ततः॑ । यत् । अ॒त्यशि॑ष्य॒तेत्य॑ति-अशि॑ष्यत । तत् । अ॒ब्रु॒व॒न्न् । वस॑तु । नु । नः॒ । इ॒दम् । इति॑ । तत् । व॒स॒ती॒वरी॑णाम् । व॒स॒ती॒व॒रि॒त्वमिति॑ वसतीवरि - त्वम् । तस्मिन्न्॑ । प्रा॒तः । न । समिति॑ । अ॒श॒क्नु॒व॒न्न् । तत् । अ॒फ्स्वित्य॑प् - सु । प्रेति॑ । अ॒वे॒श॒य॒न्न् । ताः । व॒स॒ती॒वरीः᳚ । अ॒भ॒व॒न्न् । व॒स॒ती॒वरीः᳚ । गृ॒ह्णा॒ति॒ । य॒ज्ञ्ः । वै । व॒स॒ती॒वरीः᳚ । य॒ज्ञ्म् । ए॒व । आ॒रभ्येत्या᳚ - रभ्य॑ । गृ॒ही॒त्वा । उपेति॑ । व॒स॒ति॒ । यस्य॑ । अगृ॑हीताः । अ॒भीति॑ । नि॒म्रोचे॒दिति॑ नि - म्रोचे᳚त् । अना॑रब्ध॒ इत्यना᳚ - र॒ब्धः॒ । अ॒स्य॒ । य॒ज्ञ्ः । स्या॒त् ।  \newline




\markright{ TS 6.4.2.2  \hfill https://www.vedavms.in \hfill}
\addcontentsline{toc}{section}{ TS 6.4.2.2 }
\section*{ TS 6.4.2.2 }

\textbf{TS 6.4.2.2 } \newline
\textbf{Samhita Paata} \newline

द्य॒ज्ञ्ं ॅवि च्छि॑न्द्या-ज्ज्योति॒ष्या॑ वा गृह्णी॒याद्धिर॑ण्यं ॅवा ऽव॒धाय॒ सशु॑क्राणामे॒व गृ॑ह्णाति॒ यो वा᳚ ब्राह्म॒णो ब॑हुया॒जी तस्य॒ कुंभ्या॑नां गृह्णीया॒थ् स हि गृ॑ही॒त व॑सतीवरीको वसती॒वरी᳚र्गृह्णाति प॒शवो॒ वै व॑सती॒वरीः᳚ प॒शूने॒वाऽऽ*रभ्य॑ गृही॒त्वोप॑ वसति॒ यद॑न्वी॒पं तिष्ठ॑न् गृह्णी॒यान्नि॒र्मार्गु॑का अस्मात् प॒शवः॑ स्युः प्रती॒पं तिष्ठ॑न् गृह्णाति प्रति॒रुद्ध्यै॒वास्मै॑ प॒शून् गृ॑ह्णा॒तीन्द्रो॑- [  ] \newline

\textbf{Pada Paata} \newline

य॒ज्ञ्म् । वीति॑ । छि॒न्द्या॒त् । ज्यो॒ति॒ष्या᳚ । वा॒ । गृ॒ह्णी॒यात् । हिर॑ण्यम् । वा॒ । अ॒व॒धायेत्य॑व - धाय॑ । सशु॑क्राणा॒मिति॒ स-शु॒क्रा॒णा॒म् । ए॒व । गृ॒ह्णा॒ति॒ । यः । वा॒ । ब्रा॒ह्म॒णः । ब॒हु॒या॒जीति॑ बहु - या॒जी । तस्य॑ । कुंभ्या॑नाम् । गृ॒ह्णी॒या॒त् । सः । हि । गृ॒ही॒तव॑सतीवरीक॒ इति॑ गृही॒त - व॒स॒ती॒व॒री॒कः॒ । व॒स॒ती॒वरीः᳚ । गृ॒ह्णा॒ति॒ । प॒शवः॑ । वै । व॒स॒ती॒वरीः᳚ । प॒शून् । ए॒व । आ॒रभ्येत्या᳚- रभ्य॑ । गृ॒ही॒त्वा । उपेति॑ । व॒स॒ति॒ । यत् । अ॒न्वी॒पम् । तिष्ठन्न्॑ । गृ॒ह्णी॒यात् । नि॒र्मार्गु॑का॒ इति॑ निः - मार्गु॑काः । अ॒स्मा॒त् । प॒शवः॑ । स्युः॒ । प्र॒ती॒पम् । तिष्ठन्न्॑ । गृ॒ह्णा॒ति॒ । प्र॒ति॒रुद्ध्येति॑ प्रति - रुद्ध्य॑ । ए॒व । अ॒स्मै॒ । प॒शून् । गृ॒ह्णा॒ति॒ । इन्द्रः॑ ।  \newline




\markright{ TS 6.4.2.3  \hfill https://www.vedavms.in \hfill}
\addcontentsline{toc}{section}{ TS 6.4.2.3 }
\section*{ TS 6.4.2.3 }

\textbf{TS 6.4.2.3 } \newline
\textbf{Samhita Paata} \newline

वृ॒त्रम॑ह॒न्थ् सो᳚ऽ(1॒)पो᳚ऽ(1॒)भ्य॑म्रियत॒ तासां॒ ॅयन्मेद्ध्यं॑ ॅय॒ज्ञियꣳ॒॒ सदे॑व॒मासी॒त् तदत्य॑मुच्यत॒ ता वह॑न्तीरभव॒न् वह॑न्तीनां गृह्णाति॒ या ए॒व मेद्ध्या॑ य॒ज्ञियाः॒ सदे॑वा॒ आप॒स्ता सा॑मे॒व गृ॑ह्णाति॒ नान्त॒मा वह॑न्ती॒रती॑या॒द्- यद॑न्त॒मा वह॑न्तीरती॒याद् य॒ज्ञ्मति॑ मन्येत॒ न स्था॑व॒राणां᳚ गृह्णीया॒द् वरु॑णगृहीता॒ वै स्था॑व॒रा यथ् स्था॑व॒राणां᳚ं गृह्णी॒याद्- [  ] \newline

\textbf{Pada Paata} \newline

वृ॒त्रम् । अ॒ह॒न्न् । सः । अ॒पः । अ॒भीति॑ । अ॒म्रि॒य॒त॒ । तासा᳚म् । यत् । मेद्ध्य᳚म् । य॒ज्ञिय᳚म् । सदे॑व॒मिति॒ स - दे॒व॒म् । आसी᳚त् । तत् । अतीति॑ । अ॒मु॒च्य॒त॒ । ताः । वह॑न्तीः । अ॒भ॒व॒न्न् । वह॑न्तीनाम् । गृ॒ह्णा॒ति॒ । याः । ए॒व । मेद्ध्याः᳚ । य॒ज्ञियाः᳚ । सदे॑वा॒ इति॒ स - दे॒वाः॒ । आपः॑ । तासा᳚म् । ए॒व । गृ॒ह्णा॒ति॒ । न । अ॒न्त॒माः । वह॑न्तीः । अतीति॑ । इ॒या॒त् । यत् । अ॒न्त॒माः । वह॑न्तीः । अ॒ती॒यादित्य॑ति-इ॒यात् । य॒ज्ञ्म् । अतीति॑ । म॒न्ये॒त॒ । न । स्था॒व॒राणा᳚म् । गृ॒ह्णी॒या॒त् । वरु॑णगृहीता॒ इति॒ वरु॑ण - गृ॒ही॒ताः॒ । वै । स्था॒व॒राः । यत् । स्था॒व॒राणा᳚म् । गृ॒ह्णी॒यात् ।  \newline




\markright{ TS 6.4.2.4  \hfill https://www.vedavms.in \hfill}
\addcontentsline{toc}{section}{ TS 6.4.2.4 }
\section*{ TS 6.4.2.4 }

\textbf{TS 6.4.2.4 } \newline
\textbf{Samhita Paata} \newline

-वरु॑णेनास्य य॒ज्ञ्ं ग्रा॑हये॒द् यद्वै दिवा॒ भव॑त्य॒पो रात्रिः॒ प्र वि॑शति॒ तस्मा᳚त् ता॒म्रा आपो॒ दिवा॑ ददृश्रे॒ यन्नक्तं॒ भव॑त्य॒पोऽहः॒ प्र वि॑शति॒ तस्मा᳚च्च॒न्द्रा आपो॒ नक्तं॑ ददृश्रे छा॒यायै॑ चा॒ऽऽ*तप॑तश्च स॒धौं गृ॑ह्णात्य-होरा॒त्रयो॑रे॒वास्मै॒ वर्णं॑ गृह्णाति ह॒विष्म॑तीरि॒मा आप॒ इत्या॑ह ह॒विष्कृ॑तानामे॒व गृ॑ह्णाति ह॒विष्माꣳ॑ अस्तु॒- [  ] \newline

\textbf{Pada Paata} \newline

वरु॑णेन । अ॒स्य॒ । य॒ज्ञ्म् । ग्रा॒ह॒ये॒त् । यत् । वै । दिवा᳚ । भव॑ति । अ॒पः । रात्रिः॑ । प्रेति॑ । वि॒श॒ति॒ । तस्मा᳚त् । ता॒म्राः । आपः॑ । दिवा᳚ । द॒दृ॒श्रे॒ । यत् । नक्त᳚म् । भव॑ति । अ॒पः । अहः॑ । प्रेति॑ । वि॒श॒ति॒ । तस्मा᳚त् । च॒न्द्राः । आपः॑ । नक्त᳚म् । द॒दृ॒श्रे॒ । छा॒यायै᳚ । च॒ । आ॒तप॑त॒ इत्या᳚ - तप॑तः । च॒ । स॒न्धाविति॑ सं - धौ । गृ॒ह्णा॒ति॒ । अ॒हो॒रा॒त्रयो॒रित्य॑हः - रा॒त्रयोः᳚ । ए॒व । अ॒स्मै॒ । वर्ण᳚म् । गृ॒ह्णा॒ति॒ । ह॒विष्म॑तीः । इ॒माः । आपः॑ । इति॑ । आ॒ह॒ । ह॒विष्कृ॑ताना॒मिति॑ ह॒विः - कृ॒ता॒ना॒म् । ए॒व । गृ॒ह्णा॒ति॒ । ह॒विष्मान्॑ । अ॒स्तु॒ ।  \newline




\markright{ TS 6.4.2.5  \hfill https://www.vedavms.in \hfill}
\addcontentsline{toc}{section}{ TS 6.4.2.5 }
\section*{ TS 6.4.2.5 }

\textbf{TS 6.4.2.5 } \newline
\textbf{Samhita Paata} \newline

सूर्य॒ इत्या॑ह॒ सशु॑क्राणामे॒व गृ॑ह्णात्यनु॒ष्टुभा॑ गृह्णाति॒ वाग्वा अ॑नु॒ष्टुग् वा॒चैवैनाः॒ सर्व॑या गृह्णाति॒ चतु॑ष्पदय॒र्चा गृ॑ह्णाति॒ त्रिः सा॑दयति स॒प्त सं प॑द्यन्ते स॒प्तप॑दा॒ शक्व॑री प॒शवः॒ शक्व॑री प॒शूने॒वाव॑ रुन्धे॒ ऽस्मै वै लो॒काय॒ गार्.ह॑पत्य॒ आ धी॑यते॒ऽमुष्मा॑ आहव॒नीयो॒ यद् गार्.ह॑पत्य उपसा॒दये॑द॒स्मिन् ॅलो॒के प॑शु॒मान्थ् स्या॒द् यदा॑हव॒नीये॒ऽमुष्मि॑न्- [  ] \newline

\textbf{Pada Paata} \newline

सूर्यः॑ । इति॑ । आ॒ह॒ । सशु॑क्राणा॒मिति॒ स - शु॒क्रा॒णा॒म् । ए॒व । गृ॒ह्णा॒ति॒ । अ॒नु॒ष्टुभेत्य॑नु - स्तुभा᳚ । गृ॒ह्णा॒ति॒ । वाक् । वै । अ॒नु॒ष्टुगित्य॑नु - स्तुक् । वा॒चा । ए॒व । ए॒नाः॒ । सर्व॑या । गृ॒ह्णा॒ति॒ । चतु॑ष्पद॒येति॒ चतुः॑ - प॒द॒या॒ । ऋ॒चा । गृ॒ह्णा॒ति॒ । त्रिः । सा॒द॒य॒ति॒ । स॒प्त । समिति॑ । प॒द्य॒न्ते॒ । स॒प्तप॒देति॑ स॒प्त - प॒दा॒ । शक्व॑री । प॒शवः॑ । शक्व॑री । प॒शून् । ए॒व । अवेति॑ । रु॒न्धे॒ । अ॒स्मै । वै । लो॒काय॑ । गार्.ह॑पत्य॒ इति॒ गार्.ह॑ - प॒त्यः॒ । एति॑ । धी॒य॒ते॒ । अ॒मुष्मै᳚ । आ॒ह॒व॒नीय॒ इत्या᳚ - ह॒व॒नीयः॑ । यत् । गार्.ह॑पत्य॒ इति॒ गार्.ह॑ - प॒त्ये॒ । उ॒प॒सा॒दये॒दित्यु॑प - सा॒दये᳚त् । अ॒स्मिन्न् । लो॒के । प॒शु॒मानिति॑ पशु - मान् । स्या॒त् । यत् । आ॒ह॒व॒नीय॒ इत्या᳚ - ह॒व॒नीये᳚ । अ॒मुष्मिन्न्॑ ।  \newline




\markright{ TS 6.4.2.6  \hfill https://www.vedavms.in \hfill}
\addcontentsline{toc}{section}{ TS 6.4.2.6 }
\section*{ TS 6.4.2.6 }

\textbf{TS 6.4.2.6 } \newline
\textbf{Samhita Paata} \newline

ॅलो॒के प॑शु॒मान्थ् स्या॑दु॒भयो॒रुप॑ सादयत्यु॒भयो॑रे॒वैनं॑ ॅलो॒कयोः᳚ पशु॒मन्तं॑ करोति स॒र्वतः॒ परि॑ हरति॒ रक्ष॑सा॒मप॑हत्या इन्द्राग्नि॒योर्भा॑ग॒धेयीः॒ स्थेत्या॑ह यथाय॒जुरे॒वैतदाग्नी᳚द्ध्र॒ उप॑ वासयत्ये॒तद्वै य॒ज्ञ्स्याप॑राजितं॒ ॅयदाग्नी᳚द्ध्रं॒ ॅयदे॒व य॒ज्ञ्स्याप॑राजितं॒ तदे॒वैना॒ उप॑ वासयति॒ यतः॒ खलु॒ वै य॒ज्ञ्स्य॒ वित॑तस्य॒ न क्रि॒यते॒ ( ) तदनु॑ य॒ज्ञ्ꣳ रक्षाꣳ॒॒स्यव॑ चरन्ति॒ यद् वह॑न्तीनां गृ॒ह्णाति॑ क्रि॒यमा॑णमे॒व तद् य॒ज्ञ्स्य॑ शये॒ रक्ष॑सा॒-मन॑न्ववचाराय॒ न ह्ये॑ता ई॒लय॒न्त्या तृ॑तीयसव॒नात् परि॑ शेरे य॒ज्ञ्स्य॒ संत॑त्यै ॥ \newline

\textbf{Pada Paata} \newline

लो॒के । प॒शु॒मानिति॑ पशु - मान् । स्या॒त् । उ॒भयोः᳚ । उपेति॑ । सा॒द॒य॒ति॒ । उ॒भयोः᳚ । ए॒व । ए॒न॒म् । लो॒कयोः᳚ । प॒शु॒मन्त॒मिति॑ पशु - मन्त᳚म् । क॒रो॒ति॒ । स॒र्वतः॑ । परीति॑ । ह॒र॒ति॒ । रक्ष॑साम् । अप॑हत्या॒ इत्यप॑ - ह॒त्यै॒ । इ॒न्द्रा॒ग्नि॒योरिती᳚न्द्र - अ॒ग्नि॒योः । भा॒ग॒धेयी॒रिति॑ भाग - धेयीः᳚ । स्थ॒ । इति॑ । आ॒ह॒ । य॒था॒य॒जुरिति॑ यथा-य॒जुः । ए॒व । ए॒तत् । आग्नी᳚द्ध्र॒ इत्याग्नि॑ - इ॒द्ध्रे॒ । उपेति॑ । वा॒स॒य॒ति॒ । ए॒तत् । वै । य॒ज्ञ्स्य॑ । अप॑राजित॒मित्यप॑रा-जि॒त॒म् । यत् । आग्नी᳚द्ध्र॒मित्याग्नि॑ - इ॒द्ध्र॒म् । यत् । ए॒व । य॒ज्ञ्स्य॑ । अप॑राजित॒मित्यप॑रा - जि॒त॒म् । तत् । ए॒व । ए॒नाः॒ । उपेति॑ । वा॒स॒य॒ति॒ । यतः॑ । खलु॑ । वै । य॒ज्ञ्स्य॑ । वित॑त॒स्येति॒ वि - त॒त॒स्य॒ । न । क्रि॒यते᳚ ( ) । तत् । अन्विति॑ । य॒ज्ञ्म् । रक्षाꣳ॑सि । अवेति॑ । च॒र॒न्ति॒ । यत् । वह॑न्तीनाम् । गृ॒ह्णाति॑ । क्रि॒यमा॑णम् । ए॒व । तत् । य॒ज्ञ्स्य॑ । श॒ये॒ । रक्ष॑साम् । अन॑न्ववचारा॒येत्यन॑नु - अ॒व॒चा॒रा॒य॒ । न । हि । ए॒ताः । ई॒लय॑न्ति । एति॑ । तृ॒ती॒य॒स॒व॒नादिति॑ तृतीय - स॒व॒नात् । परीति॑ । शे॒रे॒ । य॒ज्ञ्स्य॑ । सन्त॑त्या॒ इति॒ सं - त॒त्यै॒ ॥  \newline




\markright{ TS 6.4.3.1  \hfill https://www.vedavms.in \hfill}
\addcontentsline{toc}{section}{ TS 6.4.3.1 }
\section*{ TS 6.4.3.1 }

\textbf{TS 6.4.3.1 } \newline
\textbf{Samhita Paata} \newline

ब्र॒ह्म॒वा॒दिनो॑ वदन्ति॒ स त्वा अ॑द्ध्व॒र्युः स्या॒द्यः सोम॑मुपाव॒हर॒न्थ् सर्वा᳚भ्यो दे॒वता᳚भ्य उपाव॒हरे॒दिति॑ हृ॒दे त्वेत्या॑ह मनु॒ष्ये᳚भ्य ए॒वैतेन॑ करोति॒ मन॑से॒ त्वेत्या॑ह पि॒तृभ्य॑ ए॒वैतेन॑ करोति दि॒वे त्वा॒ सूर्या॑य॒ त्वेत्या॑ह दे॒वेभ्य॑ ए॒वैतेन॑ करोत्ये॒ताव॑ती॒र्वै दे॒वता॒स्ताभ्य॑ ए॒वैनꣳ॒॒ सर्वा᳚भ्य उ॒पाव॑हरति पु॒रा वा॒चः- [  ] \newline

\textbf{Pada Paata} \newline

ब्र॒ह्म॒वा॒दिन॒ इति॑ ब्रह्म-वा॒दिनः॑ । व॒द॒न्ति॒ । सः । तु । वै । अ॒द्ध्व॒र्युः । स्या॒त् । यः । सोम᳚म् । उ॒पा॒व॒हर॒न्नित्यु॑प - अ॒व॒हरन्न्॑ । सर्वा᳚भ्यः । दे॒वता᳚भ्यः । उ॒पा॒व॒हरे॒दित्यु॑प - अ॒व॒हरे᳚त् । इति॑ । हृ॒दे । त्वा॒ । इति॑ । आ॒ह॒ । म॒नु॒ष्ये᳚भ्यः । ए॒व । ए॒तेन॑ । क॒रो॒ति॒ । मन॑से । त्वा॒ । इति॑ । आ॒ह॒ । पि॒तृभ्य॒ इति॑ पि॒तृ - भ्यः॒ । ए॒व । ए॒तेन॑ । क॒रो॒ति॒ । दि॒वे । त्वा॒ । सूर्या॑य । त्वा॒ । इति॑ । आ॒ह॒ । दे॒वेभ्यः॑ । ए॒व । ए॒तेन॑ । क॒रो॒ति॒ । ए॒ताव॑तीः । वै । दे॒वताः᳚ । ताभ्यः॑ । ए॒व । ए॒न॒म् । सर्वा᳚भ्यः । उ॒पाव॑हर॒तीत्यु॑प - अव॑हरति । पु॒रा । वा॒चः ।  \newline




\markright{ TS 6.4.3.2  \hfill https://www.vedavms.in \hfill}
\addcontentsline{toc}{section}{ TS 6.4.3.2 }
\section*{ TS 6.4.3.2 }

\textbf{TS 6.4.3.2 } \newline
\textbf{Samhita Paata} \newline

प्रव॑दितोः प्रातरनुवा॒कमु॒पाक॑रोति॒ याव॑त्ये॒व वाक् तामव॑ रुन्धे॒ ऽपोऽग्रे॑ऽभि॒व्याह॑रति य॒ज्ञो वा आपो॑ य॒ज्ञ्मे॒वाभि वाचं॒ ॅविसृ॑जति॒ सर्वा॑णि॒ छन्दाꣳ॒॒स्यन्वा॑ह प॒शवो॒ वै छन्दाꣳ॑सि प॒शूने॒वाव॑ रुन्धे गायत्रि॒या तेज॑स्कामस्य॒ परि॑ दद्ध्यात् त्रि॒ष्टुभे᳚न्द्रि॒य का॑मस्य॒ जग॑त्या प॒शुका॑मस्यानु॒ष्टुभा᳚ प्रति॒ष्ठाका॑मस्य प॒ङ्क्त्या य॒ज्ञ्का॑मस्य वि॒राजाऽन्न॑कामस्य शृ॒णोत्व॒ग्निः स॒मिधा॒ हवं॑- [  ] \newline

\textbf{Pada Paata} \newline

प्रव॑दितो॒रिति॒ प्र - व॒दि॒तोः॒ । प्रा॒त॒र॒नु॒वा॒कमिति॑ प्रातः - अ॒नु॒वा॒कम् । उ॒पाक॑रो॒तीत्यु॑प-आक॑रोति । याव॑ती । ए॒व । वाक् । ताम् । अवेति॑ । रु॒न्धे॒ । अ॒पः । अग्रे᳚ । अ॒भि॒व्याह॑र॒तीत्य॑भि - व्याह॑रति । य॒ज्ञ्ः । वै । आपः॑ । य॒ज्ञ्म् । ए॒व । अ॒भीति॑ । वाच᳚म् । वीति॑ । सृ॒ज॒ति॒ । सर्वा॑णि । छन्दाꣳ॑सि । अन्विति॑ । आ॒ह॒ । प॒शवः॑ । वै । छन्दाꣳ॑सि । प॒शून् । ए॒व । अवेति॑ । रु॒न्धे॒ । गा॒य॒त्रि॒या । तेज॑स्काम॒स्येति॒ तेजः॑ - का॒म॒स्य॒ । परीति॑ । द॒द्ध्या॒त् । त्रि॒ष्टुभा᳚ । इ॒न्द्रि॒यका॑म॒स्येती᳚न्द्रि॒य - का॒म॒स्य॒ । जग॑त्या । प॒शुका॑म॒स्येति॑ प॒शु - का॒म॒स्य॒ । अ॒नु॒ष्टुभेत्य॑नु - स्तुभा᳚ । प्र॒ति॒ष्ठाका॑म॒स्येति॑ प्रति॒ष्ठा - का॒म॒स्य॒ । प॒ङ्क्त्या । य॒ज्ञ्का॑म॒स्येति॑ य॒ज्ञ् - का॒म॒स्य॒ । वि॒राजेति॑ वि - राजा᳚ । अन्न॑काम॒स्येत्यन्न॑ - का॒म॒स्य॒ । शृ॒णोतु॑ । अ॒ग्निः । स॒मिधेति॑ सं - इधा᳚ । हव᳚म् ।  \newline




\markright{ TS 6.4.3.3  \hfill https://www.vedavms.in \hfill}
\addcontentsline{toc}{section}{ TS 6.4.3.3 }
\section*{ TS 6.4.3.3 }

\textbf{TS 6.4.3.3 } \newline
\textbf{Samhita Paata} \newline

म॒ इत्या॑ह सवि॒तृप्र॑सूत ए॒व दे॒वता᳚भ्यो नि॒वेद्या॒पोऽच्छै᳚त्य॒प इ॑ष्य होत॒रित्या॑हेषि॒तꣳ हि कर्म॑ क्रि॒यते॒ मैत्रा॑वरुणस्य चमसाद्ध्वर्य॒वा द्र॒वेत्या॑ह मि॒त्रावरु॑णौ॒ वा अ॒पां ने॒तारौ॒ ताभ्या॑मे॒वैना॒ अच्छै॑ति॒ देवी॑रापो अपां नपा॒दित्या॒हाऽऽ*हु॑त्यै॒वैना॑ नि॒ष्क्रीय॑ गृह्णा॒त्यथो॑ ह॒विष्कृ॑ताना-मे॒वाभिघृ॑तानां गृह्णाति॒- [  ] \newline

\textbf{Pada Paata} \newline

मे॒ । इति॑ । आ॒ह॒ । स॒वि॒तृप्र॑सूत॒ इति॑ सवि॒तृ - प्र॒सू॒तः॒ । ए॒व । दे॒वता᳚भ्यः । नि॒वेद्येति॑ नि - वेद्य॑ । अ॒पः । अच्छ॑ । ए॒ति॒ । अ॒पः । इ॒ष्य॒ । हो॒तः॒ । इति॑ । आ॒ह॒ । इ॒षि॒तम् । हि । कर्म॑ । क्रि॒यते᳚ । मैत्रा॑वरुण॒स्येति॒ मैत्रा᳚-व॒रु॒ण॒स्य॒ । च॒म॒सा॒द्ध्व॒र्य॒विति॑ चमस-अ॒द्ध्व॒र्यो॒ । एति॑ । द्र॒व॒ । इति॑ । आ॒ह॒ । मि॒त्रावरु॑णा॒विति॑ मि॒त्रा - वरु॑णौ । वै । अ॒पाम् । ने॒तारौ᳚ । ताभ्या᳚म् । ए॒व । ए॒नाः॒ । अच्छ॑ । ए॒ति॒ । देवीः᳚ । आ॒पः॒ । अ॒पा॒म् । न॒पा॒त् । इति॑ । आ॒ह॒ । आह॒त्येत्या - हु॒त्या॒ । ए॒व । ए॒नाः॒ । नि॒ष्क्रीयेति॑ निः - क्रीय॑ । गृ॒ह्णा॒ति॒ । अथो॒ इति॑ । ह॒विष्कृ॑ताना॒मिति॑ ह॒विः - कृ॒ता॒ना॒म् । ए॒व । अ॒भिघृ॑ताना॒मित्य॒भि - घृ॒ता॒ना॒म् । गृ॒ह्णा॒ति॒ ।  \newline




\markright{ TS 6.4.3.4  \hfill https://www.vedavms.in \hfill}
\addcontentsline{toc}{section}{ TS 6.4.3.4 }
\section*{ TS 6.4.3.4 }

\textbf{TS 6.4.3.4 } \newline
\textbf{Samhita Paata} \newline

कार्.षि॑र॒सीत्या॑ह॒ शम॑लमे॒वाऽऽ*सा॒मप॑ प्लावयति समु॒द्रस्य॒ वोऽक्षि॑त्या॒ उन्न॑य॒ इत्या॑ह॒ तस्मा॑द॒द्यमा॑नाः पी॒यमा॑ना॒ आपो॒ न क्षी॑यन्ते॒ योनि॒र्वै य॒ज्ञ्स्य॒ चात्वा॑लं ॅय॒ज्ञो व॑सती॒वरीर्॑.होतृचम॒सं च॑ मैत्रावरुणचम॒सं च॑ सꣳ॒॒स्पर्श्य॑ वसती॒वरी॒र्व्यान॑यति य॒ज्ञ्स्य॑ सयोनि॒त्वायाथो॒ स्वादे॒वैना॒ योनेः॒ प्र ज॑नय॒त्यद्ध्व॒र्योऽवे॑र॒पा(3) इत्या॑हो॒ते ( ) -म॑नन्नमुरु॒तेमाः प॒श्येति॒ वावैतदा॑ह॒ यद्य॑ग्निष्टो॒मो जु॒होति॒ यद् यु॒क्थ्यः॑ परि॒धौ नि मा᳚र्ष्टि॒ यद् य॑तिरा॒त्रो यजु॒र्वद॒न् प्र प॑द्यते यज्ञ्क्रतू॒नां ॅव्यावृ॑त्त्यै ॥ \newline

\textbf{Pada Paata} \newline

कार्.षिः॑ । अ॒सि॒ । इति॑ । आ॒ह॒ । शम॑लम् । ए॒व । आ॒सा॒म् । अपेति॑ । प्ला॒व॒य॒ति॒ । स॒मु॒द्रस्य॑ । वः॒ । अक्षि॑त्यै । उदिति॑ । न॒ये॒ । इति॑ । आ॒ह॒ । तस्मा᳚त् । अ॒द्यमा॑नाः । पी॒यमा॑नाः । आपः॑ । न । क्षी॒य॒न्ते॒ । योनिः॑ । वै । य॒ज्ञ्स्य॑ । चात्वा॑लम् । य॒ज्ञ्ः । व॒स॒ती॒वरीः᳚ । हो॒तृ॒च॒म॒समिति॑ होतृ - च॒म॒सम् । च॒ । मै॒त्रा॒व॒रु॒ण॒च॒म॒समिति॑ मैत्रावरुण - च॒म॒सम् । च॒ । सꣳ॒॒स्पर्श्येति॑ सं - स्पर्श्य॑ । व॒स॒ती॒वरीः᳚ । व्यान॑य॒तीति॑ वि - आन॑यति । य॒ज्ञ्स्य॑ । स॒यो॒नि॒त्वायेति॑ सयोनि - त्वाय॑ । अथो॒ इति॑ । स्वात् । ए॒व । ए॒नाः॒ । योनेः᳚ । प्रेति॑ । ज॒न॒य॒ति॒ । अद्ध्व॑र्यो॒ इति॑ । अवेः᳚ । अ॒पा(3)ः । इति॑ । आ॒ह॒ । उ॒त ( ) । ई॒म् । अ॒न॒न्न॒मुः॒ । उ॒त । इ॒माः । प॒श्य॒ । इति॑ । वाव । ए॒तत् । आ॒ह॒ । यदि॑ । अ॒ग्नि॒ष्टो॒म इत्य॑ग्नि - स्तो॒मः । जु॒होति॑ । यदि॑ । उ॒क्थ्यः॑ । प॒रि॒धाविति॑ परि - धौ । नीति॑ । मा॒र्ष्टि॒ । यदि॑ । अ॒ति॒रा॒त्र इत्य॑ति - रा॒त्रः । यजुः॑ । वदन्न्॑ । प्रेति॑ । प॒द्य॒ते॒ । य॒ज्ञ्॒क्र॒तू॒नामिति॑ यज्ञ् - क्र॒तू॒नाम् । व्यावृ॑त्त्या॒ इति॑ वि - आवृ॑त्त्यै ॥  \newline




\markright{ TS 6.4.4.1  \hfill https://www.vedavms.in \hfill}
\addcontentsline{toc}{section}{ TS 6.4.4.1 }
\section*{ TS 6.4.4.1 }

\textbf{TS 6.4.4.1 } \newline
\textbf{Samhita Paata} \newline

दे॒वस्य॑ त्वा सवि॒तुः प्रस॑व॒ इति॒ ग्रावा॑ण॒मा द॑त्ते॒ प्रसू᳚त्या अ॒श्विनो᳚-र्बा॒हुभ्या॒मित्या॑हा॒श्विनौ॒ हि दे॒वाना॑मद्ध्व॒र्यू आस्तां᳚ पू॒ष्णो हस्ता᳚भ्या॒मित्या॑ह॒ यत्यै॑ प॒शवो॒ वै सोमो᳚ व्या॒न उ॑पाꣳशु॒सव॑नो॒ यदु॑पाꣳशु॒सव॑न-म॒भि मिमी॑ते व्या॒नमे॒व प॒शुषु॑ दधा॒तीन्द्रा॑य॒ त्वेन्द्रा॑य॒ त्वेति॑ मिमीत॒ इन्द्रा॑य॒ हि सोम॑ आह्रि॒यते॒ पञ्च॒ कृत्वो॒ यजु॑षा मिमीते॒- [  ] \newline

\textbf{Pada Paata} \newline

दे॒वस्य॑ । त्वा॒ । स॒वि॒तुः । प्र॒स॒व इति॑ प्र - स॒वे । इति॑ । ग्रावा॑णम् । एति॑ । द॒त्ते॒ । प्रसू᳚त्या॒ इति॒ प्र - सू॒त्यै॒ । अ॒श्विनोः᳚ । बा॒हुभ्या॒मिति॑ बा॒हु-भ्या॒म् । इति॑ । आ॒ह॒ । अ॒श्विनौ᳚ । हि । दे॒वाना᳚म् । अ॒द्ध्व॒र्यू इति॑ । आस्ता᳚म् । पू॒ष्णः । हस्ता᳚भ्याम् । इति॑ । आ॒ह॒ । यत्यै᳚ । प॒शवः॑ । वै । सोमः॑ । व्या॒न इति॑ वि - अ॒नः । उ॒पाꣳ॒॒शु॒सव॑न॒ इत्यु॑पाꣳशु - सव॑नः । यत् । उ॒पाꣳ॒॒शु॒सव॑न॒मित्यु॑पाꣳशु - सव॑नम् । अ॒भीति॑ । मिमी॑ते । व्या॒नमिति॑ वि - अ॒नम् । ए॒व । प॒शुषु॑ । द॒धा॒ति॒ । इन्द्रा॑य । त्वा॒ । इन्द्रा॑य । त्वा॒ । इति॑ । मि॒मी॒ते॒ । इन्द्रा॑य । हि । सोमः॑ । आ॒ह्रि॒यत॒ इत्या᳚ - ह्रि॒यते᳚ । पञ्च॑ । कृत्वः॑ । यजु॑षा । मि॒मी॒ते॒ ।  \newline




\markright{ TS 6.4.4.2  \hfill https://www.vedavms.in \hfill}
\addcontentsline{toc}{section}{ TS 6.4.4.2 }
\section*{ TS 6.4.4.2 }

\textbf{TS 6.4.4.2 } \newline
\textbf{Samhita Paata} \newline

पञ्चा᳚क्षरा प॒ङ्क्तिः पाङ्क्तो॑ य॒ज्ञो य॒ज्ञ्मे॒वाव॑ रुन्धे॒ पञ्च॒ कृत्व॑स्तू॒ष्णीं दश॒ सं प॑द्यन्ते॒ दशा᳚क्षरा वि॒राडन्नं॑ ॅवि॒राड् वि॒राजै॒वान्नाद्य॒मव॑ रुन्धे श्वा॒त्राः स्थ॑ वृत्र॒तुर॒ इत्या॑है॒ष वा अ॒पाꣳ सो॑मपी॒थो य ए॒वं ॅवेद॒ नाफ्स्वार्ति॒मार्च्छ॑ति॒ यत्ते॑ सोम दि॒वि ज्योति॒रित्या॑है॒भ्य ए॒वैनं॑- [  ] \newline

\textbf{Pada Paata} \newline

पञ्चा᳚क्ष॒रेति॒ पञ्च॑-अ॒क्ष॒रा॒ । प॒ङ्क्तिः । पाङ्क्तः॑ । य॒ज्ञ्ः । य॒ज्ञ्म् । ए॒व । अवेति॑ । रु॒न्धे॒ । पञ्च॑ । कृत्वः॑ । तू॒ष्णीम् । दश॑ । समिति॑ । प॒द्य॒न्ते॒ । दशा᳚क्ष॒रेति॒ दश॑ - अ॒क्ष॒रा॒ । वि॒राडिति॑ वि - राट् । अन्न᳚म् । वि॒राडिति॑ वि - राट् । वि॒राजेति॑ वि - राजा᳚ । ए॒व । अ॒न्नाद्य॒मित्य॑न्न -अद्य᳚म् । अवेति॑ । रु॒न्धे॒ । श्वा॒त्राः । स्थ॒ । वृ॒त्र॒तुर॒ इति॑ वृत्र - तुरः॑ । इति॑ । आ॒ह॒ । ए॒षः । वै । अ॒पाम् । सो॒म॒पी॒थ इति॑ सोम - पी॒थः । यः । ए॒वम् । वेद॑ । न । अ॒फ्स्वित्य॑प् - सु । आर्ति᳚म् । एति॑ । ऋ॒च्छ॒ति॒ । यत् । ते॒ । सो॒म॒ । दि॒वि । ज्योतिः॑ । इति॑ । आ॒ह॒ । ए॒भ्यः । ए॒व । ए॒न॒म् ।  \newline




\markright{ TS 6.4.4.3  \hfill https://www.vedavms.in \hfill}
\addcontentsline{toc}{section}{ TS 6.4.4.3 }
\section*{ TS 6.4.4.3 }

\textbf{TS 6.4.4.3 } \newline
\textbf{Samhita Paata} \newline

ॅलो॒केभ्यः॒ सं भ॑रति॒ सोमो॒ वै राजा॒ दिशो॒ऽभ्य॑द्ध्याय॒थ् स दिशोऽनु॒ प्रावि॑श॒त् प्रागपा॒गुद॑गध॒रागित्या॑ह दि॒ग्भ्य ए॒वैनꣳ॒॒ सं भ॑र॒त्यथो॒ दिश॑ ए॒वास्मा॒ अव॑ रु॒न्धे ऽम्ब॒ नि ष्व॒रेत्या॑ह॒ कामु॑का एनꣳ॒॒ स्त्रियो॑ भवन्ति॒ य ए॒वं ॅवेद॒ यत् ते॑ सो॒मादा᳚भ्यं॒ नाम॒ जागृ॒वीत्या॑- [  ] \newline

\textbf{Pada Paata} \newline

लो॒केभ्यः॑ । समिति॑ । भ॒र॒ति॒ । सोमः॑ । वै । राजा᳚ । दिशः॑ । अ॒भीति॑ । अ॒द्ध्या॒य॒त् । सः । दिशः॑ । अनु॑ । प्रेति॑ । अ॒वि॒श॒त् । प्राक् । अपा᳚क् । उद॑क् । अ॒ध॒राक् । इति॑ । आ॒ह॒ । दि॒ग्भ्य इति॑ दिक् - भ्यः । ए॒व । ए॒न॒म् । समिति॑ । भ॒र॒ति॒ । अथो॒ इति॑ । दिशः॑ । ए॒व । अ॒स्मै॒ । अवेति॑ । रु॒न्धे॒ । अम्ब॑ । नीति॑ । स्व॒र॒ । इति॑ । आ॒ह॒ । कामु॑काः । ए॒न॒म् । स्त्रियः॑ । भ॒व॒न्ति॒ । यः । ए॒वम् । वेद॑ । यत् । ते॒ । सो॒म॒ । अदा᳚भ्यम् । नाम॑ । जागृ॑वि । इति॑ ।  \newline




\markright{ TS 6.4.4.4  \hfill https://www.vedavms.in \hfill}
\addcontentsline{toc}{section}{ TS 6.4.4.4 }
\section*{ TS 6.4.4.4 }

\textbf{TS 6.4.4.4 } \newline
\textbf{Samhita Paata} \newline

-है॒ष वै सोम॑स्य सोमपी॒थो य ए॒वं ॅवेद॒ न सौ॒म्यामार्ति॒मार्च्छ॑ति॒ घ्नन्ति॒ वा ए॒तथ् सोमं॒ ॅयद॑भिषु॒ण्वन्त्यꣳ॒॒ शूनप॑ गृह्णाति॒ त्राय॑त ए॒वैनं॑ प्रा॒णा वा अꣳ॒॒शवः॑ प॒शवः॒ सोमो॒ ऽꣳ॒शून् पुन॒रपि॑ सृजति प्रा॒णाने॒व प॒शुषु॑ दधाति॒ द्वौद्वा॒वपि॑ सृजति॒ तस्मा॒द् द्वौद्वौ᳚ प्रा॒णाः ॥ \newline

\textbf{Pada Paata} \newline

आ॒ह॒ । ए॒षः । वै । सोम॑स्य । सो॒म॒पी॒थ इति॑ सोम - पी॒थः । यः । ए॒वम् । वेद॑ । न । सौ॒म्याम् । आर्ति᳚म् । एति॑ । ऋ॒च्छ॒ति॒ । घ्नन्ति॑ । वै । ए॒तत् । सोम᳚म् । यत् । अ॒भि॒षु॒ण्वन्तीत्य॑भि - सु॒न्वन्ति॑ । अꣳ॒॒शून् । अपेति॑ । गृ॒ह्णा॒ति॒ । त्राय॑ते । ए॒व । ए॒न॒म् । प्रा॒णा इति॑ प्र - अ॒नाः । वै । अꣳ॒॒शवः॑ । प॒शवः॑ । सोमः॑ । अꣳ॒॒शून् । पुनः॑ । अपीति॑ । सृ॒ज॒ति॒ । प्रा॒णानिति॑ प्र - अ॒नान् । ए॒व । प॒शुषु॑ । द॒धा॒ति॒ । द्वौद्वा॒विति॒ द्वौ - द्वौ॒ । अपीति॑ । सृ॒ज॒ति॒ । तस्मा᳚त् । द्वौद्वा॒विति॒ द्वौ - द्वौ॒ । प्रा॒णा इति॑ प्र - अ॒नाः ॥  \newline




\markright{ TS 6.4.5.1  \hfill https://www.vedavms.in \hfill}
\addcontentsline{toc}{section}{ TS 6.4.5.1 }
\section*{ TS 6.4.5.1 }

\textbf{TS 6.4.5.1 } \newline
\textbf{Samhita Paata} \newline

प्रा॒णो वा ए॒ष यदु॑पाꣳ॒॒शु र्यदुपाꣳ॒॒श्व॑ग्रा॒ ग्रहा॑ गृ॒ह्यन्ते᳚ प्रा॒णमे॒वानु॒ प्र य॑न्त्यरु॒णो ह॑ स्मा॒ऽऽ*हौप॑वेशिः प्रातस्सव॒न ए॒वाहं ॅय॒ज्ञ्ꣳ सꣳ स्था॑पयामि॒ तेन॒ ततः॒ सꣳस्थि॑तेन चरा॒मीत्य॒ष्टौ कृत्वोऽग्रे॒-ऽभिषु॑णो-त्य॒ष्टाक्ष॑रा गाय॒त्री गा॑य॒त्रं प्रा॑तस्सव॒नं प्रा॑तस्सव॒नमे॒व तेना᳚ ऽऽ*प्नो॒त्येका॑दश॒ कृत्वो᳚ द्वि॒तीय॒-मेका॑दशाक्षरा त्रि॒ष्टुप् त्रैष्टु॑भं॒ माद्ध्य॑दिंनꣳ॒॒- [  ] \newline

\textbf{Pada Paata} \newline

प्रा॒ण इति॑ प्र - अ॒नः । वै । ए॒षः । यत् । उ॒पाꣳ॒॒शुरित्यु॑प-अ॒शुः । यत् । उ॒पाꣳ॒॒श्व॑ग्रा॒ इत्यु॑पाꣳ॒॒शु - अ॒ग्राः॒ । ग्रहाः᳚ । गृ॒ह्यन्ते᳚ । प्रा॒णमिति॑ प्र - अ॒नम् । ए॒व । अनु॑ । प्रेति॑ । य॒न्ति॒ । अ॒रु॒णः । ह॒ । स्म॒ । आ॒ह॒ । औप॑वेशि॒रित्यौप॑ - वे॒शिः॒ । प्रा॒त॒स्स॒व॒न इति॑ प्रातः-स॒व॒ने । ए॒व । अ॒हम् । य॒ज्ञ्म् । समिति॑ । स्था॒प॒या॒मि॒ । तेन॑ । ततः॑ । सꣳस्थि॑ते॒नेति॒ सं-स्थि॒ते॒न॒ । च॒रा॒मि॒ । इति॑ । अ॒ष्टौ । कृत्वः॑ । अग्रे᳚ । अ॒भीति॑ । सु॒नो॒ति॒ । अ॒ष्टाक्ष॒रेत्य॒ष्टा - अ॒क्ष॒रा॒ । गा॒य॒त्री । गा॒य॒त्रम् । प्रा॒त॒स्स॒व॒नमिति॑ प्रातः - स॒व॒नम् । प्रा॒त॒स्स॒व॒नमिति॑ प्रातः - स॒व॒नम् । ए॒व । तेन॑ । आ॒प्नो॒ति॒ । एका॑दश । कृत्वः॑ । द्वि॒तीय᳚म् । एका॑दशाक्ष॒रेत्येका॑दश-अ॒क्ष॒रा॒ । त्रि॒ष्टुप् । त्रैष्टु॑भम् । माद्ध्य॑न्दिनम् ।  \newline




\markright{ TS 6.4.5.2  \hfill https://www.vedavms.in \hfill}
\addcontentsline{toc}{section}{ TS 6.4.5.2 }
\section*{ TS 6.4.5.2 }

\textbf{TS 6.4.5.2 } \newline
\textbf{Samhita Paata} \newline

सव॑नं॒ माद्ध्य॑न्दिनमे॒व सव॑नं॒ तेना᳚ऽऽ*प्नोति॒ द्वाद॑श॒ कृत्व॑स्तृ॒तीयं॒ द्वाद॑शाक्षरा॒ जग॑ती॒ जाग॑तं तृतीयसव॒नं तृ॑तीयसव॒नमे॒व तेना᳚ ऽऽ*प्नोत्ये॒ताꣳ ह॒ वाव स य॒ज्ञ्स्य॒ सꣳस्थि॑तिमुवा॒चा स्क॑न्दा॒यास्क॑न्नꣳ॒॒ हि तद्-यद्-य॒ज्ञ्स्य॒ सꣳस्थि॑तस्य॒ स्कन्द॒त्यथो॒ खल्वा॑हुर्गाय॒त्री वाव प्रा॑तस्सव॒ने नाति॒वाद॒ इत्यन॑तिवादुक एनं॒ भ्रातृ॑व्यो भवति॒ य ए॒वं ॅवेद॒ तस्मा॑द॒ष्टाव॑ष्टौ॒- [  ] \newline

\textbf{Pada Paata} \newline

सव॑नम् । माद्ध्य॑न्दिनम् । ए॒व । सव॑नम् । तेन॑ । आ॒प्नो॒ति॒ । द्वाद॑श । कृत्वः॑ । तृ॒तीय᳚म् । द्वाद॑शाक्ष॒रेति॒ द्वाद॑श - अ॒क्ष॒रा॒ । जग॑ती । जाग॑तम् । तृ॒ती॒य॒स॒व॒नमिति॑ तृतीय - स॒व॒नम् । तृ॒ती॒य॒स॒व॒नमिति॑ तृतीय-स॒व॒नम् । ए॒व । तेन॑ । आ॒प्नो॒ति॒ । ए॒ताम् । ह॒ । वाव । सः । य॒ज्ञ्स्य॑ । सꣳस्थि॑ति॒मिति॒ सं - स्थि॒ति॒म् । उ॒वा॒च॒ । अस्क॑न्दाय । अस्क॑न्नम् । हि । तत् । यत् । य॒ज्ञ्स्य॑ । सꣳस्थि॑त॒स्येति॒ सं - स्थि॒त॒स्य॒ । स्कन्द॑ति । अथो॒ इति॑ । खलु॑ । आ॒हुः॒ । गा॒य॒त्री । वाव । प्रा॒त॒स्स॒व॒न इति॑ प्रातः - स॒व॒ने । न । अ॒ति॒वाद॒ इत्य॑ति - वादे᳚ । इति॑ । अन॑तिवादुक॒ इत्यन॑ति - वा॒दु॒कः॒ । ए॒न॒म् । भ्रातृ॑व्यः । भ॒व॒ति॒ । यः । ए॒वम् । वेद॑ । तस्मा᳚त् । अ॒ष्टाव॑ष्टा॒वित्य॒ष्टौ - अ॒ष्टौ॒ ।  \newline




\markright{ TS 6.4.5.3  \hfill https://www.vedavms.in \hfill}
\addcontentsline{toc}{section}{ TS 6.4.5.3 }
\section*{ TS 6.4.5.3 }

\textbf{TS 6.4.5.3 } \newline
\textbf{Samhita Paata} \newline

कृत्वो॑ऽभि॒षुत्यं॑ ब्रह्मवा॒दिनो॑ वदन्ति प॒वित्र॑वन्तो॒ऽन्ये ग्रहा॑ गृ॒ह्यन्ते॒ किंप॑वित्र उपाꣳ॒॒शुरिति॒ वाक्प॑वित्र॒ इति॑ ब्रूयाद्-वा॒चस्पत॑ये पवस्व वाजि॒न्नित्या॑ह वा॒चैवैनं॑ पवयति॒ वृष्णो॑ अꣳ॒॒शुभ्या॒मित्या॑ह॒ वृष्णो॒ ह्ये॑तावꣳ॒॒शू यौ सोम॑स्य॒ गभ॑स्तिपूत॒ इत्या॑ह॒ गभ॑स्तिना॒ ह्ये॑नं प॒वय॑ति दे॒वो दे॒वानां᳚ प॒वित्र॑म॒सीत्या॑ह दे॒वो ह्ये॑ष- [  ] \newline

\textbf{Pada Paata} \newline

कृत्वः॑ । अ॒भि॒षुत्य॒मित्य॑भि-सुत्य᳚म् । ब्र॒ह्म॒वा॒दिन॒ इति॑ ब्रह्म-वा॒दिनः॑ । व॒द॒न्ति॒ । प॒वित्र॑वन्त॒ इति॑ प॒वित्र॑ - व॒न्तः॒ । अ॒न्ये । ग्रहाः᳚ । गृ॒ह्यन्ते᳚ । किम्प॑वित्र॒ इति॒ किम् - प॒वि॒त्रः॒ । उ॒पाꣳ॒॒शुरित्यु॑प - अꣳ॒॒शुः । इति॑ । वाक्प॑वित्र॒ इति॒ वाक्-प॒वि॒त्र॒ ः । इति॑ । ब्रू॒या॒त् । वा॒चः । पत॑ये । प॒व॒स्व॒ । वा॒जि॒न्न् । इति॑ । आ॒ह॒ । वा॒चा । ए॒व । ए॒न॒म् । प॒व॒य॒ति॒ । वृषः॑ । अꣳ॒॒शुभ्या॒मित्यꣳ॒॒शु - भ्या॒म् । इति॑ । आ॒ह॒ । वृष्णः॑ । हि । ए॒तौ । अꣳ॒॒शू इति॑ । यौ । सोम॑स्य । गभ॑स्तिपूत॒ इति॒ गभ॑स्ति - पू॒तः॒ । इति॑ । आ॒ह॒ । गभ॑स्तिना । हि । ए॒न॒म् । प॒वय॑ति । दे॒वः । दे॒वाना᳚म् । प॒वित्र᳚म् । अ॒सि॒ । इति॑ । आ॒ह॒ । दे॒वः । हि । ए॒षः ।  \newline




\markright{ TS 6.4.5.4  \hfill https://www.vedavms.in \hfill}
\addcontentsline{toc}{section}{ TS 6.4.5.4 }
\section*{ TS 6.4.5.4 }

\textbf{TS 6.4.5.4 } \newline
\textbf{Samhita Paata} \newline

सन् दे॒वानां᳚ प॒वित्रं॒ ॅयेषां᳚ भा॒गोऽसि॒ तेभ्य॒स्त्वेत्या॑ह॒ येषाꣳ॒॒ ह्ये॑ष भा॒गस्तेभ्य॑ एनं गृ॒ह्णाति॒ स्वां कृ॑तो॒ऽसीत्या॑ह प्रा॒णमे॒व स्वम॑कृत॒ मधु॑मतीर्न॒ इष॑स्कृ॒धीत्या॑ह॒ सर्व॑मे॒वास्मा॑ इ॒दꣳ स्व॑दयति॒ विश्वे᳚भ्य-स्त्वेन्द्रि॒येभ्यो॑ दि॒व्येभ्यः॒ पार्थि॑वेभ्य॒ इत्या॑हो॒भये᳚ष्वे॒व दे॑वमनु॒ष्येषु॑ प्रा॒णान् द॑धाति॒ मन॑स्त्वा॒- [  ] \newline

\textbf{Pada Paata} \newline

सन्न् । दे॒वाना᳚म् । प॒वित्र᳚म् । येषा᳚म् । भा॒गः । असि॑ । तेभ्यः॑ । त्वा॒ । इति॑ । आ॒ह॒ । येषा᳚म् । हि । ए॒षः । भा॒गः । तेभ्यः॑ । ए॒न॒म् । गृ॒ह्णाति॑ । स्वाङ्कृ॑तः । अ॒सि॒ । इति॑ । आ॒ह॒ । प्रा॒णमिति॑ प्र - अ॒नम् । ए॒व । स्वम् । अ॒कृ॒त॒ । मधु॑मती॒रिति॒ मधु॑ - म॒तीः॒ । नः॒ । इषः॑ । कृ॒धि॒ । इति॑ । आ॒ह॒ । सर्व᳚म् । ए॒व । अ॒स्मै॒ । इ॒दम् । स्व॒द॒य॒ति॒ । विश्वे᳚भ्यः । त्वा॒ । इ॒न्द्रि॒येभ्यः॑ । दि॒व्येभ्यः॑ । पार्थि॑वेभ्यः । इति॑ । आ॒ह॒ । उ॒भये॑षु । ए॒व । दे॒व॒म॒नु॒ष्येष्विति॑ देव - म॒नु॒ष्येषु॑ । प्रा॒णानिति॑ प्र - अ॒नान् । द॒धा॒ति॒ । मनः॑ । त्वा॒ ।  \newline




\markright{ TS 6.4.5.5  \hfill https://www.vedavms.in \hfill}
\addcontentsline{toc}{section}{ TS 6.4.5.5 }
\section*{ TS 6.4.5.5 }

\textbf{TS 6.4.5.5 } \newline
\textbf{Samhita Paata} \newline

ऽष्ट्वित्या॑ह॒ मन॑ ए॒वाश्नु॑त उ॒र्व॑न्तरि॑क्ष॒-मन्वि॒हीत्या॑हा-न्तरिक्षदेव॒त्यो॑ हि प्रा॒णः स्वाहा᳚ त्वा सुभवः॒ सूर्या॒येत्या॑ह प्रा॒णा वै स्वभ॑वसो दे॒वास्तेष्वे॒व प॒रोक्षं॑ जुहोति दे॒वेभ्य॑स्त्वा मरीचि॒पेभ्य॒ इत्या॑हा ऽऽ*दि॒त्यस्य॒ वै र॒श्मयो॑ दे॒वा म॑रीचि॒पास्तेषां॒ तद् भा॑ग॒धेयं॒ ताने॒व तेन॑ प्रीणाति॒ यदि॑ का॒मये॑त॒ वर्.षु॑कः प॒र्जन्यः॑- [  ] \newline

\textbf{Pada Paata} \newline

अ॒ष्टु॒ । इति॑ । आ॒ह॒ । मनः॑ । ए॒व । आ॒श्नु॒ते॒ । उ॒रु । अ॒न्तरि॑क्षम् । अन्विति॑ । इ॒हि॒ । इति॑ । आ॒ह॒ । अ॒न्त॒रि॒क्ष॒दे॒व॒त्य॑ इत्य॑न्तरिक्ष-दे॒व॒त्यः॑ । हि । प्रा॒ण इति॑ प्र - अ॒नः । स्वाहा᳚ । त्वा॒ । सु॒भ॒व॒ इति॑ सु - भ॒वः॒ । सूर्या॑य । इति॑ । आ॒ह॒ । प्रा॒णा इति॑ प्र - अ॒नाः । वै । स्वभ॑वस॒ इति॒ स्व - भ॒व॒सः॒ । दे॒वाः । तेषु॑ । ए॒व । प॒रोक्ष॒मिति॑ परः - अक्ष᳚म् । जु॒हो॒ति॒ । दे॒वेभ्यः॑ । त्वा॒ । म॒री॒चि॒पेभ्य॒ इति॑ मरीचि - पेभ्यः॑ । इति॑ । आ॒ह॒ । आ॒दि॒त्यस्य॑ । वै । र॒श्मयः॑ । दे॒वाः । म॒री॒चि॒पा इति॑ मरीचि-पाः । तेषा᳚म् । तत् । भा॒ग॒धेय॒मिति॑ भाग-धेय᳚म् । तान् । ए॒व । तेन॑ । प्री॒णा॒ति॒ । यदि॑ । का॒मये॑त । वर्.षु॑कः । प॒र्जन्यः॑ ।  \newline




\markright{ TS 6.4.5.6  \hfill https://www.vedavms.in \hfill}
\addcontentsline{toc}{section}{ TS 6.4.5.6 }
\section*{ TS 6.4.5.6 }

\textbf{TS 6.4.5.6 } \newline
\textbf{Samhita Paata} \newline

स्या॒दिति॒ नीचा॒ हस्ते॑न॒ नि मृ॑ज्या॒द्-वृष्टि॑मे॒व नि य॑च्छति॒ यदि॑ का॒मये॒ताव॑र्.षुकः स्या॒दित्यु॑त्ता॒नेन॒ नि मृ॑ज्या॒द्-वृष्टि॑मे॒वोद्-य॑च्छति॒ यद्य॑भि॒चरे॑द॒मुं ज॒ह्यथ॑ त्वा होष्या॒मीति॑ ब्रूया॒दाहु॑तिमे॒वैनं॑ प्रे॒फ्सन्. ह॑न्ति॒ यदि॑ दू॒रे स्यादा तमि॑तोस्तिष्ठेत् प्रा॒णमे॒वास्या॑नु॒गत्य॑ हन्ति॒ यद्य॑भि॒चरे॑द॒मुष्य॑- [  ] \newline

\textbf{Pada Paata} \newline

स्या॒त् । इति॑ । नीचा᳚ । हस्ते॑न । नीति॑ । मृ॒ज्या॒त् । वृष्टि᳚म् । ए॒व । नीति॑ । य॒च्छ॒ति॒ । यदि॑ । का॒मये॑त । अव॑र्.षुकः । स्या॒त् । इति॑ । उ॒त्ता॒नेनेत्यु॑त् - ता॒नेन॑ । नीति॑ । मृ॒ज्या॒त् । वृष्टि᳚म् । ए॒व । उदिति॑ । य॒च्छ॒ति॒ । यदि॑ । अ॒भि॒चरे॒दित्य॑भि - चरे᳚त् । अ॒मुम् । ज॒हि॒ । अथ॑ । त्वा॒ । हो॒ष्या॒मि॒ । इति॑ । ब्रू॒या॒त् । आहु॑ति॒मित्या - हु॒ति॒म् । ए॒व । ए॒न॒म् । प्रे॒फ्सन्निति॑ प्र-ई॒फ्सन्न् । ह॒न्ति॒ । यदि॑ । दू॒रे । स्यात् । एति॑ । तमि॑तोः । ति॒ष्ठे॒त् । प्रा॒णमिति॑ प्र - अ॒नम् । ए॒व । अ॒स्य॒ । अ॒नु॒गत्येत्य॑नु - गत्य॑ । ह॒न्ति॒ । यदि॑ । अ॒भि॒चरे॒दित्य॑भि - चरे᳚त् । अ॒मुष्य॑ ।  \newline




\markright{ TS 6.4.5.7  \hfill https://www.vedavms.in \hfill}
\addcontentsline{toc}{section}{ TS 6.4.5.7 }
\section*{ TS 6.4.5.7 }

\textbf{TS 6.4.5.7 } \newline
\textbf{Samhita Paata} \newline

त्वा प्रा॒णे सा॑दया॒मीति॑ सादये॒दस॑न्नो॒ वै प्रा॒णः प्रा॒णमे॒वास्य॑ सादयति ष॒ड्भिरꣳ॒॒शुभिः॑ पवयति॒ षड् वा ऋ॒तव॑ ऋ॒तुभि॑रे॒वैनं॑ पवयति॒ त्रिः प॑वयति॒ त्रय॑ इ॒मे लो॒का ए॒भिरे॒वैनं॑ ॅलो॒कैः प॑वयति ब्रह्मवा॒दिनो॑ वदन्ति॒ कस्मा᳚थ् स॒त्यात् त्रयः॑ पशू॒नाꣳ हस्ता॑दाना॒ इति॒ यत् त्रिरु॑पाꣳ॒॒ शुꣳ हस्ते॑न विगृ॒ह्णाति॒ तस्मा॒त् त्रयः॑ पशू॒नाꣳ हस्ता॑दानाः॒ पुरु॑षो ( ) ह॒स्ती म॒र्कटः॑ ॥ \newline

\textbf{Pada Paata} \newline

त्वा॒ । प्रा॒ण इति॑ प्र - अ॒ने । सा॒द॒या॒मि॒ । इति॑ । सा॒द॒ये॒त् । अस॑न्नः । वै । प्रा॒ण इति॑ प्र - अ॒नः । प्रा॒णमिति॑ प्र - अ॒नम् । ए॒व । अ॒स्य॒ । सा॒द॒य॒ति॒ । ष॒ड्भिरिति॑ षट् - भिः । अꣳ॒॒शुभि॒रित्यꣳ॒॒शु - भिः॒ । प॒व॒य॒ति॒ । षट् । वै । ऋ॒तवः॑ । ऋ॒तुभि॒रित्यृ॒तु - भिः॒ । ए॒व । ए॒न॒म् । प॒व॒य॒ति॒ । त्रिः । प॒व॒य॒ति॒ । त्रयः॑ । इ॒मे । लो॒काः । ए॒भिः । ए॒व । ए॒न॒म् । लो॒कैः । प॒व॒य॒ति॒ । ब्र॒ह्म॒वा॒दिन॒ इति॑ ब्रह्म-वा॒दिनः॑ । व॒द॒न्ति॒ । कस्मा᳚त् । स॒त्यात् । त्रयः॑ । प॒शू॒नाम् । हस्ता॑दाना॒ इति॒ हस्त॑ -आ॒दा॒नाः॒ । इति॑ । यत् । त्रिः । उ॒पाꣳ॒॒शुमित्यु॑प - अꣳ॒॒शुम् । हस्ते॑न । वि॒गृ॒ह्णातीति॑ वि - गृ॒ह्णाति॑ । तस्मा᳚त् । त्रयः॑ । प॒शू॒नाम् । हस्ता॑दाना॒ इति॒ हस्त॑ - आ॒दा॒नाः॒ । पुरु॑षः ( ) । ह॒स्ती । म॒र्कटः॑ ॥  \newline




\markright{ TS 6.4.6.1  \hfill https://www.vedavms.in \hfill}
\addcontentsline{toc}{section}{ TS 6.4.6.1 }
\section*{ TS 6.4.6.1 }

\textbf{TS 6.4.6.1 } \newline
\textbf{Samhita Paata} \newline

दे॒वा वै यद् य॒ज्ञेऽकु॑र्वत॒ तदसु॑रा अकुर्वत॒ ते दे॒वा उ॑पाꣳ॒॒शौ य॒ज्ञ्ꣳ सꣳ॒॒स्थाप्य॑मपश्य॒न् तमु॑पाꣳ॒॒शौ सम॑स्थापय॒न् तेऽसु॑रा॒ वज्र॑मु॒द्यत्य॑ दे॒वान॒भ्या॑यन्त॒ ते दे॒वा बिभ्य॑त॒ इन्द्र॒मुपा॑धाव॒न् तानिन्द्रो᳚ऽन्तर्या॒मेणा॒न्तर॑धत्त॒ तद॑न्तर्या॒मस्या᳚न्तर्याम॒त्वं ॅयद॑न्तर्या॒मो गृ॒ह्यते॒ भ्रातृ॑व्याने॒व तद् यज॑मानो॒ऽन्तर्द्ध॑त्ते॒ ऽन्तस्ते॑- [  ] \newline

\textbf{Pada Paata} \newline

दे॒वाः । वै । यत् । य॒ज्ञे । अकु॑र्वत । तत् । असु॑राः । अ॒कु॒र्व॒त॒ । ते । दे॒वाः । उ॒पाꣳ॒॒शावित्यु॑प - अꣳ॒॒शौ । य॒ज्ञ्म् । सꣳ॒॒स्थाप्य॒मिति॑ सं - स्थाप्य᳚म् । अ॒प॒श्य॒न्न् । तम् । उ॒पाꣳ॒॒शावित्यु॑प - अꣳ॒॒शौ । समिति॑ । अ॒स्था॒प॒य॒न्न् । ते । असु॑राः । वज्र᳚म् । उ॒द्यत्येत्यु॑त् - यत्य॑ । दे॒वान् । अ॒भीति॑ । आ॒य॒न्त॒ । ते । दे॒वाः । बिभ्य॑तः । इन्द्र᳚म् । उपेति॑ । अ॒धा॒व॒न्न् । तान् । इन्द्रः॑ । अ॒न्त॒र्या॒मेणेत्य॑न्तः - या॒मेन॑ । अ॒न्तः । अ॒ध॒त्त॒ । तत् । अ॒न्त॒र्या॒मस्येत्य॑न्तः - या॒मस्य॑ । अ॒न्त॒र्या॒म॒त्वमित्य॑न्तर्याम - त्वम् । यत् । अ॒न्त॒र्या॒म इत्य॑न्तः-या॒मः । गृ॒ह्यते᳚ । भ्रातृ॑व्यान् । ए॒व । तत् । यज॑मानः । अ॒न्तः । ध॒त्ते॒ । अ॒न्तः । ते॒ ।  \newline




\markright{ TS 6.4.6.2  \hfill https://www.vedavms.in \hfill}
\addcontentsline{toc}{section}{ TS 6.4.6.2 }
\section*{ TS 6.4.6.2 }

\textbf{TS 6.4.6.2 } \newline
\textbf{Samhita Paata} \newline

दधामि॒ द्यावा॑पृथि॒वी अ॒न्तरु॒-र्व॑न्तरि॑क्ष॒-मित्या॑है॒भिरे॒व लो॒कैर्यज॑मानो॒ भ्रातृ॑व्यान॒न्तर्द्ध॑त्ते॒ ते दे॒वा अ॑मन्य॒न्तेन्द्रो॒ वा इ॒दम॑भू॒द्यद् व॒यꣳ स्म इति॒ ते᳚ऽब्रुव॒न् मघ॑व॒न्ननु॑ न॒ आ भ॒जेति॑ स॒जोषा॑ दे॒वैरव॑रैः॒ परै॒श्चेत्य॑ब्रवी॒द्ये चै॒व दे॒वाः परे॒ ये चाव॑रे॒ तानु॒भया॑- [  ] \newline

\textbf{Pada Paata} \newline

द॒धा॒मि॒ । द्यावा॑पृथि॒वी इति॒ द्यावा᳚ - पृ॒थि॒वी । अ॒न्तः । उ॒रु । अ॒न्तरि॑क्षम् । इति॑ । आ॒ह॒ । ए॒भिः । ए॒व । लो॒कैः । यज॑मानः । भ्रातृ॑व्यान् । अ॒न्तः । ध॒त्ते॒ । ते । दे॒वाः । अ॒म॒न्य॒न्त॒ । इन्द्रः॑ । वै । इ॒दम् । अ॒भू॒त् । यत् । व॒यम् । स्मः । इति॑ । ते । अ॒ब्रु॒व॒न्न् । मघ॑व॒न्निति॒ मघ॑ - व॒न्न् । अन्विति॑ । नः॒ । एति॑ । भ॒ज॒ । इति॑ । स॒जोषा॒ इति॑ स - जोषाः᳚ । दे॒वैः । अव॑रैः । परैः᳚ । च॒ । इति॑ । अ॒ब्र॒वी॒त् । ये । च॒ । ए॒व । दे॒वाः । परे᳚ । ये । च॒ । अव॑रे । तान् । उ॒भयान्॑ ।  \newline




\markright{ TS 6.4.6.3  \hfill https://www.vedavms.in \hfill}
\addcontentsline{toc}{section}{ TS 6.4.6.3 }
\section*{ TS 6.4.6.3 }

\textbf{TS 6.4.6.3 } \newline
\textbf{Samhita Paata} \newline

न॒न्वाभ॑जथ् स॒जोषा॑ दे॒वैरव॑रैः॒ परै॒श्चेत्या॑ह॒ ये चै॒व दे॒वाः परे॒ ये चाव॑रे॒ तानु॒भया॑-न॒न्वाभ॑ज-त्यन्तर्या॒मे म॑घवन् मादय॒स्वेत्या॑ह य॒ज्ञादे॒व यज॑मानं॒ नान्तरे᳚त्युपया॒म-गृ॑हीतो॒ ऽसीत्या॑हापा॒नस्य॒ धृत्यै॒ यदु॒भाव॑पवि॒त्रौ गृ॒ह्येया॑तां प्रा॒णम॑पा॒नोऽनु॒ न्यृ॑च्छेत् प्र॒मायु॑कः स्यात् प॒वित्र॑वानन्तर्या॒मो गृ॑ह्यते- [  ] \newline

\textbf{Pada Paata} \newline

अ॒न्वाभ॑ज॒दित्य॑नु - आभ॑जत् । स॒जोषा॒ इति॑ स - जोषाः᳚ । दे॒वैः । अव॑रैः । परैः᳚ । च॒ । इति॑ । आ॒ह॒ । ये । च॒ । ए॒व । दे॒वाः । परे᳚ । ये । च॒ । अव॑रे । तान् । उ॒भयान्॑ । अ॒न्वाभ॑ज॒तीत्य॑नु- आभ॑जति । अ॒न्त॒र्या॒म इत्य॑न्तः - या॒मे । म॒घ॒व॒न्निति॑ मघ - व॒न्न् । मा॒द॒य॒स्व॒ । इति॑ । आ॒ह॒ । य॒ज्ञात् । ए॒व । यज॑मानम् । न । अ॒न्तः । ए॒ति॒ । उ॒प॒या॒मगृ॑हीत॒ इत्यु॑पया॒म-गृ॒ही॒तः॒ । अ॒सि॒ । इति॑ । आ॒ह॒ । अ॒पा॒नस्येत्य॑प - अ॒नस्य॑ । धृत्यै᳚ । यत् । उ॒भौ । अ॒प॒वि॒त्रौ । गृ॒ह्येया॑ताम् । प्रा॒णमिति॑ प्र-अ॒नम् । अ॒पा॒न इत्य॑प - अ॒नः । अनु॑ । नीति॑ । ऋ॒च्छे॒त् । प्र॒मायु॑क॒ इति॑ प्र-मायु॑कः । स्या॒त् । प॒वित्र॑वा॒निति॑ प॒वित्र॑ - वा॒न् । अ॒न्त॒र्या॒म इत्य॑न्तः - या॒मः । गृ॒ह्य॒ते॒ ।  \newline




\markright{ TS 6.4.6.4  \hfill https://www.vedavms.in \hfill}
\addcontentsline{toc}{section}{ TS 6.4.6.4 }
\section*{ TS 6.4.6.4 }

\textbf{TS 6.4.6.4 } \newline
\textbf{Samhita Paata} \newline

प्राणापा॒नयो॒-र्विधृ॑त्यै प्राणापा॒नौ वा ए॒तौ यदु॑पाꣳश्वन्तर्या॒मौ व्या॒न उ॑पाꣳशु॒ सव॑नो॒ यं का॒मये॑त प्र॒मायु॑कः स्या॒दित्यसꣳ॑ स्पृष्टौ॒ तस्य॑ सादयेद्-व्या॒नेनै॒वास्य॑ प्राणापा॒नौ वि च्छि॑नत्ति ता॒जक् प्र मी॑यते॒ यं का॒मये॑त॒ सर्व॒मायु॑रिया॒दिति॒ सꣳ स्पृ॑ष्टौ॒ तस्य॑ सादयेद्-व्या॒नेनै॒वास्य॑ प्राणापा॒नौ सं त॑नोति॒ सर्व॒मायु॑रेति ॥ \newline

\textbf{Pada Paata} \newline

प्रा॒णा॒पा॒नयो॒रिति॑ प्राण - अ॒पा॒नयोः᳚ । विधृ॑त्या॒ इति॒ वि - धृ॒त्यै॒ । प्रा॒णा॒पा॒नाविति॑ प्राण - अ॒पा॒नौ । वै । ए॒तौ । यत् । उ॒पाꣳ॒॒श्व॒न्त॒र्या॒मावित्यु॑पाꣳशु - अ॒न्त॒र्या॒मौ । व्या॒न इति॑ वि - अ॒नः । उ॒पाꣳ॒॒शु॒सव॑न॒ इत्यु॑पाꣳशु - सव॑नः । यम् । का॒मये॑त । प्र॒मायु॑क॒ इति॑ प्र - मायु॑कः । स्या॒त् । इति॑ । असꣳ॑स्पृष्टा॒वित्यसं᳚ - स्पृ॒ष्टौ॒ । तस्य॑ । सा॒द॒ये॒त् । व्या॒नेनेति॑ वि - अ॒नेन॑ । ए॒व । अ॒स्य॒ । प्रा॒णा॒पा॒नाविति॑ प्राण - अ॒पा॒नौ । वीति॑ । छि॒न॒त्ति॒ । ता॒जक् । प्रेति॑ । मी॒य॒ते॒ । यम् । का॒मये॑त । सर्व᳚म् । आयुः॑ । इ॒या॒त् । इति॑ । सꣳस्पृ॑ष्टा॒विति॒ सं-स्पृ॒ष्टौ॒ । तस्य॑ । सा॒द॒ये॒त् । व्या॒नेनेति॑ वि-अ॒नेन॑ । ए॒व । अ॒स्य॒ । प्रा॒णा॒पा॒नाविति॑ प्राण - अ॒पा॒नौ । समिति॑ । त॒नो॒ति॒ । सर्व᳚म् । आयुः॑ । ए॒ति॒ ॥  \newline




\markright{ TS 6.4.7.1  \hfill https://www.vedavms.in \hfill}
\addcontentsline{toc}{section}{ TS 6.4.7.1 }
\section*{ TS 6.4.7.1 }

\textbf{TS 6.4.7.1 } \newline
\textbf{Samhita Paata} \newline

वाग्वा ए॒षा यदै᳚न्द्रवाय॒वो यदै᳚न्द्रवाय॒वाग्रा॒ ग्रहा॑ गृ॒ह्यन्ते॒ वाच॑मे॒वानु॒ प्र य॑न्ति वा॒युं दे॒वा अ॑ब्रुव॒न्थ् सोमꣳ॒॒ राजा॑नꣳ हना॒मेति॒ सो᳚ऽब्रवी॒द् वरं॑ ॅवृणै॒ मद॑ग्रा ए॒व वो॒ ग्रहा॑ गृह्यान्ता॒ इति॒ तस्मा॑दैन्द्रवाय॒वाग्रा॒ ग्रहा॑ गृह्यन्ते॒ तम॑घ्न॒न्थ् सो॑ऽपूय॒त् तं दे॒वा नोपा॑धृष्णुव॒न् ते वा॒ युम॑ब्रुवन्नि॒मं नः॑ स्वद॒ये- [  ] \newline

\textbf{Pada Paata} \newline

वाक् । वै । ए॒षा । यत् । ऐ॒न्द्र॒वा॒य॒व इत्यै᳚न्द्र - वा॒य॒वः । यत् । ऐ॒न्द्र॒वा॒य॒वाग्रा॒ इत्यै᳚न्द्रवाय॒व - अ॒ग्राः॒ । ग्रहाः᳚ । गृ॒ह्यन्ते᳚ । वाच᳚म् । ए॒व । अनु॑ । प्रेति॑ । य॒न्ति॒ । वा॒युम् । दे॒वाः । अ॒ब्रु॒व॒न्न् । सोम᳚म् । राजा॑नम् । ह॒ना॒म॒ । इति॑ । सः । अ॒ब्र॒वी॒त् । वर᳚म् । वृ॒णै॒ । मद॑ग्रा॒ इति॒ मत् - अ॒ग्राः॒ । ए॒व । वः॒ । ग्रहाः᳚ । गृ॒ह्या॒न्तै॒ । इति॑ । तस्मा᳚त् । ऐ॒न्द्र॒वा॒य॒वाग्रा॒ इत्यै᳚न्द्रवाय॒व - अ॒ग्राः॒ । ग्रहाः᳚ । गृ॒ह्य॒न्ते॒ । तम् । अ॒घ्न॒न्न् । सः । अ॒पू॒य॒त् । तम् । दे॒वाः । न । उपेति॑ । अ॒धृ॒ष्णु॒व॒न्न् । ते । वा॒युम् । अ॒ब्रु॒व॒न्न् । इ॒मम् । नः॒ । स्व॒द॒य॒ ।  \newline




\markright{ TS 6.4.7.2  \hfill https://www.vedavms.in \hfill}
\addcontentsline{toc}{section}{ TS 6.4.7.2 }
\section*{ TS 6.4.7.2 }

\textbf{TS 6.4.7.2 } \newline
\textbf{Samhita Paata} \newline

-ति॒ सो᳚ऽब्रवी॒द् वरं॑ ॅवृणै मद्देव॒त्या᳚न्ये॒व वः॒ पात्रा᳚ण्युच्यान्ता॒ इति॒ तस्मा᳚न्नानादेव॒त्या॑नि॒ सन्ति॑ वाय॒व्या᳚न्युच्यन्ते॒ तमे᳚भ्यो वा॒युरे॒वास्व॑दय॒त् तस्मा॒द्यत् पूय॑ति॒ तत् प्र॑वा॒ते वि ष॑जन्ति वा॒युर्.हि तस्य॑ पवयि॒ता स्व॑दयि॒ता तस्य॑ वि॒ग्रह॑णं॒ नावि॑न्द॒न्थ् सादि॑तिरब्रवी॒द् वरं॑ ॅवृणा॒ अथ॒ मया॒ वि गृ॑ह्णीद्ध्वं मद्देव॒त्या॑ ए॒व वः॒ सोमाः᳚ - [  ] \newline

\textbf{Pada Paata} \newline

इति॑ । सः । अ॒ब्र॒वी॒त् । वर᳚म् । वृ॒णै॒ । म॒द्दे॒व॒त्या॑नीति॑ मत्-दे॒व॒त्या॑नि । ए॒व । वः॒ । पात्रा॑णि । उ॒च्या॒न्तै॒ । इति॑ । तस्मा᳚त् । ना॒ना॒दे॒व॒त्या॑नीति॑ नाना - दे॒व॒त्या॑नि । सन्ति॑ । वा॒य॒व्या॑नि । उ॒च्य॒न्ते॒ । तम् । ए॒भ्यः॒ । वा॒युः । ए॒व । अ॒स्व॒द॒य॒त् । तस्मा᳚त् । यत् । पूय॑ति । तत् । प्र॒वा॒त इति॑ प्र - वा॒ते । वीति॑ । स॒ज॒न्ति॒ । वा॒युः । हि । तस्य॑ । प॒व॒यि॒ता । स्व॒द॒यि॒ता । तस्य॑ । वि॒ग्रह॑ण॒मिति॑ वि - ग्रह॑णम् । न । अ॒वि॒न्द॒न्न् । सा । अदि॑तिः । अ॒ब्र॒वी॒त् । वर᳚म् । वृ॒णै॒ । अथ॑ । मया᳚ । वीति॑ । गृ॒ह्णी॒द्ध्व॒म् । म॒द्दे॒व॒त्या॑ इति॑ मत् - दे॒व॒त्याः᳚ । ए॒व । वः॒ । सोमाः᳚ ।  \newline




\markright{ TS 6.4.7.3  \hfill https://www.vedavms.in \hfill}
\addcontentsline{toc}{section}{ TS 6.4.7.3 }
\section*{ TS 6.4.7.3 }

\textbf{TS 6.4.7.3 } \newline
\textbf{Samhita Paata} \newline

स॒न्ना अ॑स॒-न्नित्यु॑पया॒मगृ॑हीतो॒ऽसी-त्या॑हा-दितिदेव॒त्या᳚स्तेन॒ यानि॒ हि दा॑रु॒मया॑णि॒ पात्रा᳚ण्य॒स्यै तानि॒ योनेः॒ संभू॑तानि॒ यानि॑ मृ॒न्मया॑नि सा॒क्षात् तान्य॒स्यै तस्मा॑दे॒वमा॑ह॒ वाग्वै परा॒च्य-व्या॑कृताऽवद॒त् ते दे॒वा इन्द्र॑मब्रुवन्नि॒मां नो॒ वाचं॒ ॅव्याकु॒र्विति॒ सो᳚ऽब्रवी॒द् वरं॑ ॅवृणै॒ मह्यं॑ चै॒वैष वा॒यवे॑ च स॒ह ( ) गृ॑ह्याता॒ इति॒ तस्मा॑दैन्द्रवाय॒वः स॒ह गृ॑ह्यते॒ तामिन्द्रो॑ मद्ध्य॒तो॑ऽव॒क्रम्य॒ व्याक॑रो॒त् तस्मा॑दि॒यं ॅव्याकृ॑ता॒ वागु॑द्यते॒ तस्मा᳚थ् स॒कृदिन्द्रा॑य मद्ध्य॒तो गृ॑ह्यते॒ द्विर्वा॒यवे॒ द्वौ हि स वरा॒ववृ॑णीत ॥ \newline

\textbf{Pada Paata} \newline

स॒न्नाः । अ॒स॒न्न् । इति॑ । उ॒प॒या॒मगृ॑हीत॒ इत्यु॑पया॒म-गृ॒ही॒तः॒ । अ॒सि॒ । इति॑ । आ॒ह॒ । अ॒दि॒ति॒दे॒व॒त्या॑ इत्य॑दिति - दे॒व॒त्याः᳚ । तेन॑ । यानि॑ । हि । दा॒रु॒मया॒णीति॑ दारु-मया॑नि । पात्रा॑णि । अ॒स्यै । तानि॑ । योनेः᳚ । संभू॑ता॒नीति॒ सं - भू॒ता॒नि॒ । यानि॑ । मृ॒न्मया॒नीति॑ मृत् - मया॑नि । सा॒क्षादिति॑ स - अ॒क्षात् । तानि॑ । अ॒स्यै । तस्मा᳚त् । ए॒वम् । आ॒ह॒ । वाक् । वै । परा॑ची । अव्या॑कृ॒तेत्यवि॑ - आ॒कृ॒ता॒ । अ॒व॒द॒त् । ते । दे॒वाः । इन्द्र᳚म् । अ॒ब्रु॒व॒न्न् । इ॒माम् । नः॒ । वाच᳚म् । व्याकु॒र्विति॑ वि - आकु॑रु । इति॑ । सः । अ॒ब्र॒वी॒त् । वर᳚म् । वृ॒णै॒ । मह्य᳚म् । च॒ । ए॒व । ए॒षः । वा॒यवे᳚ । च॒ । स॒ह ( ) । गृ॒ह्या॒तै॒ । इति॑ । तस्मा᳚त् । ऐ॒न्द्र॒वा॒य॒व इत्यै᳚न्द्र - वा॒य॒वः । स॒ह । गृ॒ह्य॒ते॒ । ताम् । इन्द्रः॑ । म॒द्ध्य॒तः । अ॒व॒क्रम्येत्य॑व - क्रम्य॑ । व्याक॑रो॒दिति॑ वि - आक॑रोत् । तस्मा᳚त् । इ॒यम् । व्याकृ॒तेति॑ वि-आकृ॑ता । वाक् । उ॒द्य॒ते॒ । तस्मा᳚त् । स॒कृत् । इन्द्रा॑य । म॒द्ध्य॒तः । गृ॒ह्य॒ते॒ । द्विः । वा॒यवे᳚ । द्वौ । हि । सः । वरौ᳚ । अवृ॑णीत ॥  \newline




\markright{ TS 6.4.8.1  \hfill https://www.vedavms.in \hfill}
\addcontentsline{toc}{section}{ TS 6.4.8.1 }
\section*{ TS 6.4.8.1 }

\textbf{TS 6.4.8.1 } \newline
\textbf{Samhita Paata} \newline

मि॒त्रं दे॒वा अ॑ब्रुव॒न्थ् सोमꣳ॒॒ राजा॑नꣳ हना॒मेति॒ सो᳚ऽब्रवी॒न्नाहꣳ सर्व॑स्य॒ वा अ॒हं मि॒त्रम॒स्मीति॒ तम॑ब्रुव॒न्॒. हना॑मै॒वेति॒ सो᳚ऽब्रवी॒द् वरं॑ ॅवृणै॒ पय॑सै॒व मे॒ सोमꣳ॑ श्रीण॒न्निति॒ तस्मा᳚न्-मैत्रावरु॒णं पय॑सा श्रीणन्ति॒ तस्मा᳚त् प॒शवोऽपा᳚क्रामन् मि॒त्रः सन् क्रू॒रम॑क॒रिति॑ क्रू॒रमि॑व॒ खलु॒ वा ए॒षः - [  ] \newline

\textbf{Pada Paata} \newline

मि॒त्रम् । दे॒वाः । अ॒ब्रु॒व॒न्न् । सोम᳚म् । राजा॑नम् । ह॒ना॒म॒ । इति॑ । सः । अ॒ब्र॒वी॒त् । न । अ॒हम् । सर्व॑स्य । वै । अ॒हम् । मि॒त्रम् । अ॒स्मि॒ । इति॑ । तम् । अ॒ब्रु॒व॒न्न् । हना॑म । ए॒व । इति॑ । सः । अ॒ब्र॒वी॒त् । वर᳚म् । वृ॒णै॒ । पय॑सा । ए॒व । मे॒ । सोम᳚म् । श्री॒ण॒न्न् । इति॑ । तस्मा᳚त् । मै॒त्रा॒व॒रु॒णमिति॑ मैत्रा - व॒रु॒णम् । पय॑सा । श्री॒ण॒न्ति॒ । तस्मा᳚त् । प॒शवः॑ । अपेति॑ । अ॒क्रा॒म॒न्न् । मि॒त्रः । सन्न् । क्रू॒रम् । अ॒कः॒ । इति॑ । क्रू॒रम् । इ॒व॒ । खलु॑ । वै । ए॒षः ।  \newline




\markright{ TS 6.4.8.2  \hfill https://www.vedavms.in \hfill}
\addcontentsline{toc}{section}{ TS 6.4.8.2 }
\section*{ TS 6.4.8.2 }

\textbf{TS 6.4.8.2 } \newline
\textbf{Samhita Paata} \newline

क॑रोति॒ यः सोमे॑न॒ यज॑ते॒ तस्मा᳚त् प॒शवोऽप॑ क्रामन्ति॒ यन्मै᳚त्रावरु॒णं पय॑सा श्री॒णाति॑ प॒शुभि॑रे॒व तन्मि॒त्रꣳ स॑म॒र्द्धय॑ति प॒शुभि॒र्यज॑मानं पु॒रा खलु॒ वावैवं मि॒त्रो॑ऽवे॒दप॒ मत् क्रू॒रं च॒क्रुषः॑ प॒शवः॑ क्रमिष्य॒न्तीति॒ तस्मा॑दे॒वम॑वृणीत॒ वरु॑णं दे॒वा अ॑ब्रुव॒न् त्वयाऽꣳ॑श॒भुवा॒ सोमꣳ॒॒ राजा॑नꣳ हना॒मेति॒ सो᳚ऽब्रवी॒द् वरं॑ ॅवृणै॒ मह्यं॑ चै॒- [  ] \newline

\textbf{Pada Paata} \newline

क॒रो॒ति॒ । यः । सोमे॑न । यज॑ते । तस्मा᳚त् । प॒शवः॑ । अपेति॑ । क्रा॒म॒न्ति॒ । यत् । मै॒त्रा॒व॒रु॒णमिति॑ मैत्रा-व॒रु॒णम् । पय॑सा । श्री॒णाति॑ । प॒शुभि॒रिति॑ प॒शु - भिः॒ । ए॒व । तत् । मि॒त्रम् । स॒म॒द्‌र्धय॒तीति॑ सं - अ॒द्‌र्धय॑ति । प॒शुभि॒रिति॑ प॒शु - भिः॒ । यज॑मानम् । पु॒रा । खलु॑ । वाव । ए॒वम् । मि॒त्रः । अ॒वे॒त् । अपेति॑ । मत् । क्रू॒रम् । च॒क्रुषः॑ । प॒शवः॑ । क्र॒मि॒ष्य॒न्ति॒ । इति॑ । तस्मा᳚त् । ए॒वम् । अ॒वृ॒णी॒त॒ । वरु॑णम् । दे॒वाः । अ॒ब्रु॒व॒न्न् । त्वया᳚ । अꣳ॒॒श॒भुवेत्यꣳ॑श - भुवा᳚ । सोम᳚म् । राजा॑नम् । ह॒ना॒म॒ । इति॑ । सः । अ॒ब्र॒वी॒त् । वर᳚म् । वृ॒णै॒ । मह्य᳚म् । च॒ ।  \newline




\markright{ TS 6.4.8.3  \hfill https://www.vedavms.in \hfill}
\addcontentsline{toc}{section}{ TS 6.4.8.3 }
\section*{ TS 6.4.8.3 }

\textbf{TS 6.4.8.3 } \newline
\textbf{Samhita Paata} \newline

वैष मि॒त्राय॑ च स॒ह गृ॑ह्याता॒ इति॒ तस्मा᳚न्मैत्रावरु॒णः स॒ह गृ॑ह्यते॒ तस्मा॒द् राज्ञा॒ राजा॑नमꣳश॒भुवा᳚ घ्नन्ति॒ वैश्ये॑न॒ वैश्यꣳ॑ शू॒द्रेण॑ शू॒द्रं न वा इ॒दं दिवा॒ न नक्त॑मासी॒दव्या॑वृत्तं॒ ते दे॒वा मि॒त्रावरु॑णावब्रुवन्नि॒दं नो॒ विवा॑सयत॒मिति॒ ताव॑ब्रूतां॒ ॅवरं॑ ॅवृणावहा॒ एक॑ ए॒वाऽऽ*वत् पूर्वो॒ ग्रहो॑ गृह्याता॒ इति॒ तस्मा॑दैन्द्रवाय॒वः ( ) पूर्वो॑ मैत्रावरु॒णाद्-गृ॑ह्यते प्राणापा॒नौ ह्ये॑तौ यदु॑पाꣳ-श्वन्तर्या॒मौ मि॒त्रोऽह॒रज॑नय॒द्-वरु॑णो॒ रात्रिं॒ ततो॒ वा इ॒दं ॅव्यौ᳚च्छ॒द्यन्-मै᳚त्रावरु॒णो गृ॒ह्यते॒ व्यु॑ष्ट्यै ॥ \newline

\textbf{Pada Paata} \newline

ए॒व । ए॒षः । मि॒त्राय॑ । च॒ । स॒ह । गृ॒ह्या॒तै॒ । इति॑ । तस्मा᳚त् । मै॒त्रा॒व॒रु॒ण इति॑ मैत्रा - व॒रु॒णः । स॒ह । गृ॒ह्य॒ते॒ । तस्मा᳚त् । राज्ञा᳚ । राजा॑नम् । अꣳ॒॒श॒भुवेत्यꣳ॑श - भुवा᳚ । घ्न॒न्ति॒ । वैश्ये॑न । वैश्य᳚म् । शू॒द्रेण॑ । शू॒द्रम् । न । वै । इ॒दम् । दिवा᳚ । न । नक्त᳚म् । आ॒सी॒त् । अव्या॑वृत्त॒मित्यवि॑ - आ॒वृ॒त्त॒म् । ते । दे॒वाः । मि॒त्रावरु॑णा॒विति॑ मि॒त्रा - वरु॑णौ । अ॒ब्रु॒व॒न्न् । इ॒दम् । नः॒ । वीति॑ । वा॒स॒य॒त॒म् । इति॑ । तौ । अ॒ब्रू॒ता॒म् । वर᳚म् । वृ॒णा॒व॒है॒ । एकः॑ । ए॒व । आ॒वत् । पूर्वः॑ । ग्रहः॑ । गृ॒ह्या॒तै॒ । इति॑ । तस्मा᳚त् । ऐ॒न्द्र॒वा॒य॒व इत्यै᳚न्द्र - वा॒य॒वः ( ) । पूर्वः॑ । मै॒त्रा॒व॒रु॒णादिति॑ मैत्रा - व॒रु॒णात् । गृ॒ह्य॒ते॒ । प्रा॒णा॒पा॒नाविति॑ प्राण - अ॒पा॒नौ । हि । ए॒तौ । यत् । उ॒पाꣳ॒॒श्व॒न्त॒र्या॒मावित्यु॑पाꣳशु-अ॒न्त॒र्या॒मौ । मि॒त्रः । अहः॑ । अज॑नयत् । वरु॑णः । रात्रि᳚म् । ततः॑ । वै । इ॒दम् । वीति॑ । औ॒च्छ॒त् । यत् । मै॒त्रा॒व॒रु॒ण इति॑ मैत्रा-व॒रु॒णः । गृ॒ह्यते᳚ । व्यु॑ष्ट्या॒ इति॒ वि - उ॒ष्ट्यै॒ ॥  \newline




\markright{ TS 6.4.9.1  \hfill https://www.vedavms.in \hfill}
\addcontentsline{toc}{section}{ TS 6.4.9.1 }
\section*{ TS 6.4.9.1 }

\textbf{TS 6.4.9.1 } \newline
\textbf{Samhita Paata} \newline

य॒ज्ञ्स्य॒ शिरो᳚ऽच्छिद्यत॒ ते दे॒वा अ॒श्विना॑वब्रुवन् भि॒षजौ॒ वै स्थ॑ इ॒दं ॅय॒ज्ञ्स्य॒ शिरः॒ प्रति॑ धत्त॒मिति॒ ताव॑ब्रूतां॒ ॅवरं॑ ॅवृणावहै॒ ग्रह॑ ए॒व ना॒वत्रापि॑ गृह्यता॒मिति॒ ताभ्या॑-मे॒तमा᳚श्वि॒न-म॑गृह्ण॒न् ततो॒ वै तौ य॒ज्ञ्स्य॒ शिरः॒ प्रत्य॑धत्तां॒ ॅयदा᳚श्वि॒नो गृ॒ह्यते॑ य॒ज्ञ्स्य॒ निष्कृ॑त्यै॒ तौ दे॒वा अ॑ब्रुव॒न्नपू॑तौ॒ वा इ॒मौ म॑नुष्यच॒रौ- [  ] \newline

\textbf{Pada Paata} \newline

य॒ज्ञ्स्य॑ । शिरः॑ । अ॒च्छि॒द्य॒त॒ । ते । दे॒वाः । अ॒श्विनौ᳚ । अ॒ब्रु॒व॒न्न् । भि॒षजौ᳚ । वै । स्थः॒ । इ॒दम् । य॒ज्ञ्स्य॑ । शिरः॑ । प्रतीति॑ । ध॒त्त॒म् । इति॑ । तौ । अ॒ब्रू॒ता॒म् । वर᳚म् । वृ॒णा॒व॒है॒ । ग्रहः॑ । ए॒व । नौ॒ । अत्र॑ । अपीति॑ । गृ॒ह्य॒ता॒म् । इति॑ । ताभ्या᳚म् । ए॒तम् । आ॒श्वि॒नम् । अ॒गृ॒ह्ण॒न्न् । ततः॑ । वै । तौ । य॒ज्ञ्स्य॑ । शिरः॑ । प्रतीति॑ । अ॒ध॒त्ता॒म् । यत् । आ॒श्वि॒नः । गृ॒ह्यते᳚ । य॒ज्ञ्स्य॑ । निष्कृ॑त्या॒ इति॒ निः - कृ॒त्यै॒ । तौ । दे॒वाः । अ॒ब्रु॒व॒न्न् । अपू॑तौ । वै । इ॒मौ । म॒नु॒ष्य॒च॒राविति॑ मनुष्य - च॒रौ ।  \newline




\markright{ TS 6.4.9.2  \hfill https://www.vedavms.in \hfill}
\addcontentsline{toc}{section}{ TS 6.4.9.2 }
\section*{ TS 6.4.9.2 }

\textbf{TS 6.4.9.2 } \newline
\textbf{Samhita Paata} \newline

भि॒षजा॒विति॒ तस्मा᳚द् ब्राह्म॒णेन॑ भेष॒जं न का॒र्य॑मपू॑तो॒ ह्ये᳚(1॒)षो॑ ऽमे॒द्ध्यो यो भि॒षक्तौ ब॑हिष्पवमा॒नेन॑ पवयि॒त्वा ताभ्या॑-मे॒तमा᳚श्वि॒न-म॑गृह्ण॒न् तस्मा᳚द् बहिष्पवमा॒ने स्तु॒त आ᳚श्वि॒नो गृ॑ह्यते॒ तस्मा॑दे॒वं ॅवि॒दुषा॑ बहिष्पवमा॒न उ॑प॒सद्यः॑ प॒वित्रं॒ ॅवै ब॑हिष्पवमा॒न आ॒त्मान॑मे॒व प॑वयते॒ तयो᳚ऽस्त्रे॒धा भैष॑ज्यं॒ ॅवि न्य॑दधुर॒ग्नौ तृती॑यम॒फ्सु तृती॑यं ब्राह्म॒णे तृती॑यं॒ तस्मा॑दुदपा॒त्र- [  ] \newline

\textbf{Pada Paata} \newline

भि॒षजौ᳚ । इति॑ । तस्मा᳚त् । ब्रा॒ह्म॒णेन॑ । भे॒ष॒जम् । न । का॒र्य᳚म् । अपू॑तः । हि । ए॒षः । अ॒मे॒द्ध्यः । यः । भि॒षक् । तौ । ब॒हि॒ष्प॒व॒मा॒नेनेति॑ बहिः - प॒व॒मा॒नेन॑ । प॒व॒यि॒त्वा । ताभ्या᳚म् । ए॒तम् । आ॒श्वि॒नम् । अ॒गृ॒ह्ण॒न्न् । तस्मा᳚त् । ब॒हि॒ष्प॒व॒मा॒न इति॑ बहिः-प॒व॒मा॒ने । स्तु॒ते । आ॒श्वि॒नः । गृ॒ह्य॒ते॒ । तस्मा᳚त् । ए॒वम् । वि॒दुषा᳚ । ब॒हि॒ष्प॒व॒मा॒न इति॑ बहिः - प॒व॒मा॒नः । उ॒प॒सद्य॒ इत्यु॑प - सद्यः॑ । प॒वित्र᳚म् । वै । ब॒हि॒ष्प॒व॒मा॒न इति॑ बहिः - प॒व॒मा॒नः । आ॒त्मान᳚म् । ए॒व । प॒व॒य॒ते॒ । तयोः᳚ । त्रे॒धा । भैष॑ज्यम् । वि । नीति॑ । अ॒द॒धुः॒ । अ॒ग्नौ । तृती॑यम् । अ॒फ्स्वित्य॑प्-सु । तृती॑यम् । ब्रा॒ह्म॒णे । तृती॑यम् । तस्मा᳚त् । उ॒द॒पा॒त्रमित्यु॑द - पा॒त्रम् ।  \newline




\markright{ TS 6.4.9.3  \hfill https://www.vedavms.in \hfill}
\addcontentsline{toc}{section}{ TS 6.4.9.3 }
\section*{ TS 6.4.9.3 }

\textbf{TS 6.4.9.3 } \newline
\textbf{Samhita Paata} \newline

-मु॑पनि॒धाय॑ ब्राह्म॒णं द॑क्षिण॒तो नि॒षाद्य॑ भेष॒जं कु॑र्या॒द् याव॑दे॒व भे॑ष॒जं तेन॑ करोति स॒मर्द्धु॑कमस्य कृ॒तं भ॑वति ब्रह्मवा॒दिनो॑ वदन्ति॒  कस्मा᳚थ् स॒त्यादेक॑पात्रा द्विदेव॒त्या॑ गृ॒ह्यन्ते᳚ द्वि॒पात्रा॑ हूयन्त॒ इति॒ यदेक॑पात्रा गृ॒ह्यन्ते॒ तस्मा॒देको᳚ऽन्तर॒तः प्रा॒णो द्वि॒पात्रा॑ हूयन्ते॒ तस्मा॒द् द्वौद्वौ॑ ब॒हिष्टा᳚त् प्रा॒णाः प्रा॒णा वा ए॒ते यद् द्वि॑देव॒त्याः᳚ प॒शव॒ इडा॒ यदिडां॒ पूर्वां᳚ द्विदेव॒त्ये᳚भ्य उप॒ह्वये॑त- [  ] \newline

\textbf{Pada Paata} \newline

उ॒प॒नि॒धायेत्यु॑प - नि॒धाय॑ । ब्रा॒ह्म॒णम् । द॒क्षि॒ण॒तः । नि॒षाद्येति॑ नि - साद्य॑ । भे॒ष॒जम् । कु॒र्या॒त् । याव॑त् । ए॒व । भे॒ष॒जम् । तेन॑ । क॒रो॒ति॒ । स॒मद्‌र्धु॑क॒मिति॑ सं - अद्‌र्धु॑कम् । अ॒स्य॒ । कृ॒तम् । भ॒व॒ति॒ । ब्र॒ह्म॒वा॒दिन॒ इति॑ ब्रह्म - वा॒दिनः॑ । व॒द॒न्ति॒ । कस्मा᳚त् । स॒त्यात् । एक॑पात्रा॒ इत्येक॑ - पा॒त्राः॒ । द्वि॒दे॒व॒त्या॑ इति॑ द्वि-दे॒व॒त्याः᳚ । गृ॒ह्यन्ते᳚ । द्वि॒पात्रा॒ इति॑ द्वि - पात्राः᳚ । हू॒य॒न्ते॒ । इति॑ । यत् । एक॑पात्रा॒ इत्येक॑ - पा॒त्राः॒ । गृ॒ह्यन्ते᳚ । तस्मा᳚त् । एकः॑ । अ॒न्त॒र॒तः । प्रा॒ण इति॑ प्र - अ॒नः । द्वि॒पात्रा॒ इति॑ द्वि - पात्राः᳚ । हू॒य॒न्ते॒ । तस्मा᳚त् । द्वौद्वा॒विति॒ द्वौ - द्वौ॒ । ब॒हिष्टा᳚त् । प्रा॒णा इति॑ प्र - अ॒नाः । प्रा॒णा इति॑ प्र - अ॒नाः । वै । ए॒ते । यत् । द्वि॒दे॒व॒त्या॑ इति॑ द्वि - दे॒व॒त्याः᳚ । प॒शवः॑ । इडा᳚ । यत् । इडा᳚म् । पूर्वा᳚म् । द्वि॒दे॒व॒त्ये᳚भ्य॒ इति॑ द्वि - दे॒व॒त्ये᳚भ्यः । उ॒प॒ह्वये॒तेत्यु॑प - ह्वये॑त ।  \newline




\markright{ TS 6.4.9.4  \hfill https://www.vedavms.in \hfill}
\addcontentsline{toc}{section}{ TS 6.4.9.4 }
\section*{ TS 6.4.9.4 }

\textbf{TS 6.4.9.4 } \newline
\textbf{Samhita Paata} \newline

प॒शुभिः॑ प्रा॒णान॒न्तर्द॑धीत प्र॒मायु॑कः स्याद् द्विदेव॒त्या᳚न् भक्षयि॒त्वेडा॒मुप॑ ह्वयते प्रा॒णाने॒वाऽऽ*त्मन् धि॒त्वा प॒शूनुप॑ ह्वयते॒ वाग्वा ऐ᳚न्द्रवाय॒वश्चक्षु॑-र्मैत्रावरु॒णः श्रोत्र॑माश्वि॒नः पु॒रस्ता॑दैन्द्रवाय॒वं भ॑क्षयति॒ तस्मा᳚त् पु॒रस्ता᳚द् वा॒चा व॑दति पु॒रस्ता᳚न्मैत्रावरु॒णं तस्मा᳚त् पु॒रस्ता॒च्चक्षु॑षा पश्यति स॒र्वतः॑ परि॒हार॑माश्वि॒नं तस्मा᳚थ् स॒र्वतः॒ श्रोत्रे॑ण शृणोति प्रा॒णा वा ए॒ते यद् द्वि॑देव॒त्या॑- [  ] \newline

\textbf{Pada Paata} \newline

प॒शुभि॒रिति॑ प॒शु - भिः॒ । प्रा॒णानिति॑ प्र-अ॒नान् । अ॒न्तः । द॒धी॒त॒ । प्र॒मायु॑क॒ इति॑ प्र - मायु॑कः । स्या॒त् । द्वि॒दे॒व॒त्या॑निति॑ द्वि-दे॒व॒त्यान्॑ । भ॒क्ष॒यि॒त्वा । इडा᳚म् । उपेति॑ । ह्व॒य॒ते॒ । प्रा॒णानिति॑ प्र - अ॒नान् । ए॒व । आ॒त्मन्न् । धि॒त्वा । प॒शून् । उपेति॑ । ह्व॒य॒ते॒ । वाक् । वै । ऐ॒न्द्र॒वा॒य॒व इत्यै᳚न्द्र-वा॒य॒वः । चक्षुः॑ । मै॒त्रा॒व॒रु॒ण इति॑ मैत्रा-व॒रु॒णः । श्रोत्र᳚म् । आ॒श्वि॒नः । पु॒रस्ता᳚त् । ऐ॒न्द्र॒वा॒य॒वमित्यै᳚न्द्र - वा॒य॒वम् । भ॒क्ष॒य॒ति॒ । तस्मा᳚त् । पु॒रस्ता᳚त् । वा॒चा । व॒द॒ति॒ । पु॒रस्ता᳚त् । मै॒त्रा॒व॒रु॒णमिति॑ मैत्रा - व॒रु॒णम् । तस्मा᳚त् । पु॒रस्ता᳚त् । चक्षु॑षा । प॒श्य॒ति॒ । स॒र्वतः॑ । प॒रि॒हार॒मिति॑ परि-हार᳚म् । आ॒श्वि॒नम् । तस्मा᳚त् । स॒र्वतः॑ । श्रोत्रे॑ण । शृ॒णो॒ति॒ । प्रा॒णा इति॑ प्र - अ॒नाः । वै । ए॒ते । यत् । द्वि॒दे॒व॒त्या॑ इति॑ द्वि - दे॒व॒त्याः᳚ ।  \newline




\markright{ TS 6.4.9.5  \hfill https://www.vedavms.in \hfill}
\addcontentsline{toc}{section}{ TS 6.4.9.5 }
\section*{ TS 6.4.9.5 }

\textbf{TS 6.4.9.5 } \newline
\textbf{Samhita Paata} \newline

अरि॑क्तानि॒ पात्रा॑णि सादयति॒ तस्मा॒दरि॑क्ता अन्तर॒तः प्रा॒णा यतः॒ खलु॒ वै य॒ज्ञ्स्य॒ वित॑तस्य॒ न क्रि॒यते॒ तदनु॑ य॒ज्ञ्ꣳ रक्षाꣳ॒॒स्यव॑ चरन्ति॒ यदरि॑क्तानि॒ पात्रा॑णि सा॒दय॑ति क्रि॒यमा॑णमे॒व तद् य॒ज्ञ्स्य॑ शये॒ रक्ष॑सा॒ -मन॑न्ववचाराय॒ दक्षि॑णस्य हवि॒र्द्धान॒स्योत्त॑रस्यां ॅवर्त॒न्याꣳ सा॑दयति वा॒च्ये॑व वाचं॑ दधा॒त्या तृ॑तीयसव॒नात् परि॑ शेरे य॒ज्ञ्स्य॒ सन्त॑त्यै ॥ \newline

\textbf{Pada Paata} \newline

अरि॑क्तानि । पात्रा॑णि । सा॒द॒य॒ति॒ । तस्मा᳚त् । अरि॑क्ताः । अ॒न्त॒र॒तः । प्रा॒णा इति॑ प्र - अ॒नाः । यतः॑ । खलु॑ । वै । य॒ज्ञ्स्य॑ । वित॑त॒स्येति॒ वि - त॒त॒स्य॒ । न । क्रि॒यते᳚ । तत् । अन्विति॑ । य॒ज्ञ्म् । रक्षाꣳ॑सि । अवेति॑ । च॒र॒न्ति॒ । यत् । अरि॑क्तानि । पात्रा॑णि । सा॒दय॑ति । क्रि॒यमा॑णम् । ए॒व । तत् । य॒ज्ञ्स्य॑ । श॒ये॒ । रक्ष॑साम् । अन॑न्ववचारा॒येत्यन॑नु - अ॒व॒चा॒रा॒य॒ । दक्षि॑णस्य । ह॒वि॒द्‌र्धान॒स्येति॑ हविः - धान॑स्य । उत्त॑रस्या॒मित्युत् - त॒र॒स्या॒म् । व॒र्त॒न्याम् । सा॒द॒य॒ति॒ । वा॒चि । ए॒व । वाच᳚म् । द॒धा॒ति॒ । एति॑ । तृ॒ती॒य॒स॒व॒नादिति॑ तृतीय - स॒व॒नात् । परीति॑ । शे॒रे॒ । य॒ज्ञ्स्य॑ । सन्त॑त्या॒ इति॒ सं - त॒त्यै॒ ॥  \newline




\markright{ TS 6.4.10.1  \hfill https://www.vedavms.in \hfill}
\addcontentsline{toc}{section}{ TS 6.4.10.1 }
\section*{ TS 6.4.10.1 }

\textbf{TS 6.4.10.1 } \newline
\textbf{Samhita Paata} \newline

बृह॒स्पति॑र्दे॒वानां᳚ पु॒रोहि॑त॒ आसी॒-च्छण्डा॒मर्का॒-वसु॑राणां॒ ब्रह्म॑ण् वन्तो दे॒वा आस॒न् ब्रह्म॑ण् व॒न्तोऽसु॑रा॒स्ते᳚(1॒) ऽन्यो᳚ऽन्यं नाश॑क्नुव-न्न॒भिभ॑वितुं॒ ते दे॒वाः शण्डा॒मर्का॒-वुपा॑मन्त्रयन्त॒ ता व॑ब्रूतां॒ ॅवरं॑ ॅवृणावहै॒ ग्रहा॑वे॒व ना॒वत्रापि॑ गृह्येता॒मिति॒ ताभ्या॑मे॒तौ शु॒क्राम॒न्थिना॑-वगृह्ण॒न् ततो॑ दे॒वा अभ॑व॒न् पराऽसु॑रा॒ यस्यै॒वं ॅवि॒दुषः॑ शु॒क्राम॒न्थिनौ॑ गृ॒ह्येते॒ भ॑वत्या॒त्मना॒ परा᳚- [  ] \newline

\textbf{Pada Paata} \newline

बृह॒स्पतिः॑ । दे॒वाना᳚म् । पु॒रोहि॑त॒ इति॑ पु॒रः - हि॒तः॒ । आसी᳚त् । शण्डा॒मर्का॒विति॒ शण्डा᳚ - मर्कौ᳚ । असु॑राणाम् । ब्रह्म॑ण्वन्त॒ इति॒ ब्रह्मण्ण्॑ - व॒न्तः॒ । दे॒वाः । आसन्न्॑ । ब्रह्म॑ण्वन्त॒ इति॒ ब्रह्मण्ण्॑ - व॒न्तः॒ । असु॑राः । ते । अ॒न्यः । अ॒न्यम् । न । अ॒श॒क्नु॒व॒न्न् । अ॒भिभ॑वितु॒मित्य॒भि - भ॒वि॒तु॒म् । ते । दे॒वाः । शण्डा॒मर्का॒विति॒ शण्डा᳚ - मर्कौ᳚ । उपेति॑ । अ॒म॒न्त्र॒य॒न्त॒ । तौ । अ॒ब्रू॒ता॒म् । वर᳚म् । वृ॒णा॒व॒है॒ । ग्रहौ᳚ । ए॒व । नौ॒ । अत्र॑ । अपीति॑ । गृ॒ह्ये॒ता॒म् । इति॑ । ताभ्या᳚म् । ए॒तौ । शु॒क्राम॒न्थिना॒विति॑ शु॒क्रा - म॒न्थिनौ᳚ । अ॒गृ॒ह्ण॒न्न् । ततः॑ । दे॒वाः । अभ॑वन्न् । परेति॑ । असु॑राः । यस्य॑ । ए॒वम् । वि॒दुषः॑ । शु॒क्राम॒न्थिना॒विति॑ शु॒क्रा - म॒न्थिनौ᳚ । गृ॒ह्येते॒ इति॑ । भव॑ति । आ॒त्मना᳚ । परेति॑ ।  \newline




\markright{ TS 6.4.10.2  \hfill https://www.vedavms.in \hfill}
\addcontentsline{toc}{section}{ TS 6.4.10.2 }
\section*{ TS 6.4.10.2 }

\textbf{TS 6.4.10.2 } \newline
\textbf{Samhita Paata} \newline

ऽस्य॒ भ्रातृ॑व्यो भवति॒ तौ दे॒वा अ॑प॒नुद्या॒ऽऽ*त्मन॒ इन्द्रा॑याजुहवु॒-रप॑नुत्तौ॒ शण्डा॒मर्कौ॑ स॒हामुनेति॑ ब्रूया॒द्यं द्वि॒ष्याद्यमे॒व द्वेष्टि॒ तेनै॑नौ स॒हाप॑ नुदते॒ स प्र॑थ॒मः संकृ॑ति-र्वि॒श्वक॒र्मेत्ये॒वैना॑-वा॒त्मन॒ इन्द्रा॑या-जुहवु॒रिन्द्रो॒ ह्ये॑तानि॑ रू॒पाणि॒ करि॑क्र॒दच॑रद॒सौ वा आ॑दि॒त्यः शु॒क्रश्च॒न्द्रमा॑ म॒न्थ्य॑पि॒-गृह्य॒ प्राञ्चौ॒ नि- [  ] \newline

\textbf{Pada Paata} \newline

अ॒स्य॒ । भ्रातृ॑व्यः । भ॒व॒ति॒ । तौ । दे॒वाः । अ॒प॒नुद्येत्य॑प - नुद्य॑ । आ॒त्मने᳚ । इन्द्रा॑य । अ॒जु॒ह॒वुः॒ । अप॑नुत्ता॒वित्यप॑ - नु॒त्तौ॒ । शण्डा॒मर्का॒विति॒ शण्डा᳚ - मर्कौ᳚ । स॒ह । अ॒मुना᳚ । इति॑ । ब्रू॒या॒त् । यम् । द्वि॒ष्यात् । यम् । ए॒व । द्वेष्टि॑ । तेन॑ । ए॒ना॒ । स॒ह । अपेति॑ । नु॒द॒ते॒ । सः । प्र॒थ॒मः । संकृ॑ति॒रिति॒ सं - कृ॒तिः॒ । वि॒श्वक॒र्मेति॑ वि॒श्व - क॒र्मा॒ । इति॑ । ए॒व । ए॒नौ॒ । आ॒त्मने᳚ । इन्द्रा॑य । अ॒जु॒ह॒वुः॒ । इन्द्रः॑ । हि । ए॒तानि॑ । रू॒पाणि॑ । करि॑क्रत् । अच॑रत् । अ॒सौ । वै । आ॒दि॒त्यः । शु॒क्रः । च॒न्द्रमाः᳚ । म॒न्थी । अ॒पि॒गृह्येत्य॑पि - गृह्य॑ । प्राञ्चौ᳚ । निरिति॑ ।  \newline




\markright{ TS 6.4.10.3  \hfill https://www.vedavms.in \hfill}
\addcontentsline{toc}{section}{ TS 6.4.10.3 }
\section*{ TS 6.4.10.3 }

\textbf{TS 6.4.10.3 } \newline
\textbf{Samhita Paata} \newline

-ष्क्रा॑मत॒-स्तस्मा॒त् प्राञ्चौ॒ यन्तौ॒ न प॑श्यन्ति प्र॒त्यञ्चा॑वा॒वृत्य॑ जुहुत॒स्तस्मा᳚त् प्र॒त्यञ्चौ॒ यन्तौ॑ पश्यन्ति॒ चक्षु॑षी॒ वा ए॒ते य॒ज्ञ्स्य॒ यच्छु॒क्राम॒न्थिनौ॒ नासि॑कोत्तरवे॒दिर॒भितः॑ परि॒क्रम्य॑ जुहुत॒स्तस्मा॑द॒भितो॒ नासि॑कां॒ चक्षु॑षी॒ तस्मा॒न्नासि॑कया॒ चक्षु॑षी॒ विधृ॑ते स॒र्वतः॒ परि॑ क्रामतो॒ रक्ष॑सा॒मप॑हत्यै दे॒वा वै याः प्राची॒राहु॑ती॒रजु॑हवु॒र्ये पु॒रस्ता॒दसु॑रा॒ आस॒न् ताꣳस्ताभिः॒ प्रा- [  ] \newline

\textbf{Pada Paata} \newline

क्रा॒म॒तः॒ । तस्मा᳚त् । प्राञ्चौ᳚ । यन्तौ᳚ । न । प॒श्य॒न्ति॒ । प्र॒त्यञ्चौ᳚ । आ॒वृत्येत्या᳚ - वृत्य॑ । जु॒हु॒तः॒ । तस्मा᳚त् । प्र॒त्यञ्चौ᳚ । यन्तौ᳚ । प॒श्य॒न्ति॒ । चक्षु॑षी॒ इति॑ । वै । ए॒ते इति॑ । य॒ज्ञ्स्य॑ । यत् । शु॒क्राम॒न्थिना॒विति॑ शु॒क्रा - म॒न्थिनौ᳚ । नासि॑का । उ॒त्त॒र॒वे॒दिरित्यु॑त्तर- वे॒दिः । अ॒भितः॑ । प॒रि॒क्रम्येति॑ परि - क्रम्य॑ । जु॒हु॒तः॒ । तस्मा᳚त् । अ॒भितः॑ । नासि॑काम् । चक्षु॑षी॒ इति॑ । तस्मा᳚त् । नासि॑कया । चक्षु॑षी॒ इति॑ । विधृ॑ते॒ इति॒ वि-धृ॒ते॒ । स॒र्वतः॑ । परीति॑ । क्रा॒म॒तः॒ । रक्ष॑साम् । अप॑हत्या॒ इत्यप॑ - ह॒त्यै॒ । दे॒वाः । वै । याः । प्राचीः᳚ । आहु॑ती॒रित्या - हु॒तीः॒ । अजु॑हवुः । ये । पु॒रस्ता᳚त् । असु॑राः । आसन्न्॑ । तान् । ताभिः॑ । प्रेति॑ ।  \newline




\markright{ TS 6.4.10.4  \hfill https://www.vedavms.in \hfill}
\addcontentsline{toc}{section}{ TS 6.4.10.4 }
\section*{ TS 6.4.10.4 }

\textbf{TS 6.4.10.4 } \newline
\textbf{Samhita Paata} \newline

-णु॑दन्त॒ याः प्र॒तीची॒र्ये प॒श्चादसु॑रा॒ आस॒न् ताꣳस्ताभि॒रपा॑नुदन्त॒ प्राची॑र॒न्या आहु॑तयो हू॒यन्ते᳚ प्र॒त्यञ्चौ॑ शु॒क्राम॒न्थिनौ॑ प॒श्चाच्चै॒व पु॒रस्ता᳚च्च॒ यज॑मानो॒ भ्रातृ॑व्या॒न् प्र णु॑दते॒ तस्मा॒त् परा॑चीः प्र॒जाः प्र वी॑यन्ते प्र॒तीची᳚र्जायन्ते शु॒क्राम॒न्थिनौ॒ वा अनु॑ प्र॒जाः प्र जा॑यन्ते॒ऽत्त्रीश्चा॒द्या᳚श्च सु॒वीराः᳚ प्र॒जाः प्र॑ज॒नय॒न् परी॑हि शु॒क्रः शु॒क्रशो॑चिषा- [  ] \newline

\textbf{Pada Paata} \newline

अ॒नु॒द॒न्त॒ । याः । प्र॒तीचीः᳚ । ये । प॒श्चात् । असु॑राः । आसन्न्॑ । तान् । ताभिः॑ । अपेति॑ । अ॒नु॒द॒न्त॒ । प्राचीः᳚ । अ॒न्याः । आहु॑तय॒ इत्या-हु॒त॒यः॒ । हू॒यन्ते᳚ । प्र॒त्यञ्चौ᳚ । शु॒क्राम॒न्थिना॒विति॑ शु॒क्रा - म॒न्थिनौ᳚ । प॒श्चात् । च॒ । ए॒व । पु॒रस्ता᳚त् । च॒ । यज॑मानः । भ्रातृ॑व्यान् । प्रेति॑ । नु॒द॒ते॒ । तस्मा᳚त् । परा॑चीः । प्र॒जा इति॑ प्र - जाः । प्रेति॑ । वी॒य॒न्ते॒ । प्र॒तीचीः᳚ । जा॒य॒न्ते॒ । शु॒क्राम॒न्थिना॒विति॑ शु॒क्रा - म॒न्थिनौ᳚ । वै । अन्विति॑ । प्र॒जा इति॑ प्र - जाः । प्रेति॑ । जा॒य॒न्ते॒ । अ॒त्त्रीः । च॒ । आ॒द्याः᳚ । च॒ । सु॒वीरा॒ इति॑ सु - वीराः᳚ । प्र॒जा इति॑ प्र - जाः । प्र॒ज॒नय॒न्निति॑ प्र - ज॒नयन्न्॑ । परीति॑ । इ॒हि॒ । शु॒क्रः । शु॒क्रशो॑चि॒षेति॑ शु॒क्र - शो॒चि॒षा॒ ।  \newline




\markright{ TS 6.4.10.5  \hfill https://www.vedavms.in \hfill}
\addcontentsline{toc}{section}{ TS 6.4.10.5 }
\section*{ TS 6.4.10.5 }

\textbf{TS 6.4.10.5 } \newline
\textbf{Samhita Paata} \newline

सुप्र॒जाः प्र॒जाः प्र॑ज॒नय॒न् परी॑हि म॒न्थी म॒न्थिशो॑चि॒षेत्या॑है॒ता वै सु॒वीरा॒ या अ॒त्त्रीरे॒ताः सु॑प्र॒जा या आ॒द्या॑ य ए॒वं ॅवेदा॒त्र्य॑स्य प्र॒जा जा॑यते॒ नाऽऽद्या᳚ प्र॒जाप॑ते॒रक्ष्य॑श्वय॒त् तत् परा॑ऽऽ*पत॒त् तद् विक॑ङ्कतं॒ प्रावि॑श॒त् तद् विक॑ङ्कते॒ नार॑मत॒ तद् यवं॒ प्रावि॑श॒त् तद् यवे॑ऽरमत॒ तद् यव॑स्य- [  ] \newline

\textbf{Pada Paata} \newline

सु॒प्र॒जा इति॑ सु - प्र॒जाः । प्र॒जा इति॑ प्र - जाः । प्र॒ज॒नय॒न्निति॑ प्र - ज॒नयन्न्॑ । परीति॑ । इ॒हि॒ । म॒न्थी । म॒न्थिशो॑चि॒षेति॑ म॒न्थि - शो॒चि॒षा॒ । इति॑ । आ॒ह॒ । ए॒ताः । वै । सु॒वीरा॒ इति॑ सु - वीराः᳚ । या । अ॒त्त्रीः । ए॒ताः । सु॒प्र॒जा इति॑ सु - प्र॒जाः । याः । आ॒द्याः᳚ । यः । ए॒वम् । वेद॑ । अ॒त्त्री । अ॒स्य॒ । प्र॒जेति॑ प्र - जा । जा॒य॒ते॒ । न । आ॒द्या᳚ । प्र॒जाप॑ते॒रिति॑ प्र॒जा-प॒तेः॒ । अक्षि॑ । अ॒श्व॒य॒त् । तत् । परेति॑ । अ॒प॒त॒त् । तत् । विक॑ङ्कत॒मिति॒ वि-क॒ङ्क॒त॒म् । प्रेति॑ । अ॒वि॒श॒त् । तत् । विक॑ङ्कत॒ इति॑ वि - क॒ङ्क॒ते॒ । न । अ॒र॒म॒त॒ । तत् । यव᳚म् । प्रेति॑ । अ॒वि॒श॒त् । तत् । यवे᳚ । अ॒र॒म॒त॒ । तत् । यव॑स्य ।  \newline




\markright{ TS 6.4.10.6  \hfill https://www.vedavms.in \hfill}
\addcontentsline{toc}{section}{ TS 6.4.10.6 }
\section*{ TS 6.4.10.6 }

\textbf{TS 6.4.10.6 } \newline
\textbf{Samhita Paata} \newline

यव॒त्वं ॅयद् वैक॑ङ्कतं मन्थिपा॒त्रं भव॑ति॒ सक्तु॑भिः श्री॒णाति॑ प्र॒जाप॑तेरे॒व तच्चक्षुः॒ सं भ॑रति ब्रह्मवा॒दिनो॑ वदन्ति॒ कस्मा᳚थ् स॒त्यान्म॑न्थिपा॒त्रꣳ सदो॒ नाश्नु॑त॒ इत्या᳚र्तपा॒त्रꣳ हीति॑ ब्रूया॒द्-यद॑श्नुवी॒तान्धो᳚ऽद्ध्व॒र्युः स्या॒दार्ति॒मार्च्छे॒त् तस्मा॒न्नाश्नु॑ते ॥ \newline

\textbf{Pada Paata} \newline

य॒व॒त्वमिति॑ यव - त्वम् । यत् । वैक॑ङ्कतम् । म॒न्थि॒पा॒त्रमिति॑ मन्थि - पा॒त्रम् । भव॑ति । सक्तु॑भि॒रिति॒ सक्तु॑ - भिः॒ । श्री॒णाति॑ । प्र॒जाप॑ते॒रिति॑ प्र॒जा - प॒तेः॒ । ए॒व । तत् । चक्षुः॑ । समिति॑ । भ॒र॒ति॒ । ब्र॒ह्म॒वा॒दिन॒ इति॑ ब्रह्म - वा॒दिनः॑ । व॒द॒न्ति॒ । कस्मा᳚त् । स॒त्यात् । म॒न्थि॒पा॒त्रमिति॑ मन्थि - पा॒त्रम् । सदः॑ । न । अ॒श्नु॒ते॒ । इति॑ । आ॒र्त॒पा॒त्रमित्या᳚र्त - पा॒त्रम् । हि । इति॑ । ब्रू॒या॒त् । यत् । अ॒श्नु॒वी॒त । अ॒न्धः । अ॒द्ध्व॒र्युः । स्या॒त् । आर्ति᳚म् । एति॑ । ऋ॒च्छे॒त् । तस्मा᳚त् । न । अ॒श्नु॒ते॒ ॥  \newline




\markright{ TS 6.4.11.1  \hfill https://www.vedavms.in \hfill}
\addcontentsline{toc}{section}{ TS 6.4.11.1 }
\section*{ TS 6.4.11.1 }

\textbf{TS 6.4.11.1 } \newline
\textbf{Samhita Paata} \newline

दे॒वा वै यद् य॒ज्ञेऽकु॑र्वत॒ तदसु॑रा अकुर्वत॒ ते दे॒वा आ᳚ग्रय॒णाग्रा॒न् ग्रहा॑नपश्य॒न् तान॑गृह्णत॒ ततो॒ वै तेऽग्रं॒ पर्या॑य॒न॒. यस्यै॒वं ॅवि॒दुष॑ आग्रय॒णाग्रा॒ ग्रहा॑ गृ॒ह्यन्तेऽग्र॑मे॒व स॑मा॒नानां॒ पर्ये॑ति रु॒ग्णव॑त्य॒र्चा भ्रातृ॑व्यवतो गृह्णीया॒द्-भ्रातृ॑व्यस्यै॒व रु॒क्त्वाग्रꣳ॑ समा॒नानां॒ पर्ये॑ति॒ ये दे॑वा दि॒व्येका॑दश॒ स्थेत्या॑है॒- [  ] \newline

\textbf{Pada Paata} \newline

दे॒वाः । वै । यत् । य॒ज्ञे । अकु॑र्वत । तत् । असु॑राः । अ॒कु॒र्व॒त॒ । ते । दे॒वाः । आ॒ग्र॒य॒णाग्रा॒नित्या᳚ग्रय॒ण - अ॒ग्रा॒न् । ग्रहान्॑ । अ॒प॒श्य॒न्न् । तान् । अ॒गृ॒ह्ण॒त॒ । ततः॑ । वै । ते । अग्र᳚म् । परीति॑ । आ॒य॒न्न् । यस्य॑ । ए॒वम् । वि॒दुषः॑ । आ॒ग्र॒य॒णाग्रा॒ इत्या᳚ग्रय॒ण-अ॒ग्राः॒ । ग्रहाः᳚ । गृ॒ह्यन्त᳚ । अग्र᳚म् । ए॒व । स॒मा॒नाना᳚म् । परीति॑ । ए॒ति॒ । रु॒ग्णव॒त्येति॑ रु॒ग्ण - व॒त्या॒ । ऋ॒चा । भ्रातृ॑व्यवत॒ इति॒ भ्रातृ॑व्य-व॒तः॒ । गृ॒ह्णी॒या॒त् । भ्रातृ॑व्यस्य । ए॒व । रु॒क्त्वा । अग्र᳚म् । स॒मा॒नाना᳚म् । परीति॑ । ए॒ति॒ । ये । दे॒वाः॒ । दि॒वि । एका॑दश । स्थ । इति॑ । आ॒ह॒ ।  \newline




\markright{ TS 6.4.11.2  \hfill https://www.vedavms.in \hfill}
\addcontentsline{toc}{section}{ TS 6.4.11.2 }
\section*{ TS 6.4.11.2 }

\textbf{TS 6.4.11.2 } \newline
\textbf{Samhita Paata} \newline

-ताव॑ती॒र्वै दे॒वता॒स्ताभ्य॑ ए॒वैनꣳ॒॒ सर्वा᳚भ्यो गृह्णात्ये॒ष ते॒ योनि॒ र्विश्वे᳚भ्यस्त्वा दे॒वेभ्य॒ इत्या॑ह वैश्वदे॒वो ह्ये॑ष दे॒वत॑या॒ वाग्वै दे॒वेभ्यो-ऽपा᳚क्रामद्-य॒ज्ञायाति॑ष्ठमाना॒ ते दे॒वा वा॒च्यप॑क्रान्तायां तू॒ष्णीं ग्रहा॑नगृह्णत॒ साम॑न्यत॒ वाग॒न्तर्य॑न्ति॒ वै मेति॒ साऽऽग्र॑य॒णं प्रत्याग॑च्छ॒त् तदा᳚ग्रय॒णस्या᳚ऽऽग्रयण॒त्वं- [  ] \newline

\textbf{Pada Paata} \newline

ए॒ताव॑तीः । वै । दे॒वताः᳚ । ताभ्यः॑ । ए॒व । ए॒न॒म् । सर्वा᳚भ्यः । गृ॒ह्णा॒ति॒ । ए॒षः । ते॒ । योनिः॑ । विश्वे᳚भ्यः । त्वा॒ । दे॒वेभ्यः॑ । इति॑ । आ॒ह॒ । वै॒श्व॒दे॒व इति॑ वैश्व - दे॒वः । हि । ए॒षः । दे॒वत॑या । वाक् । वै । दे॒वेभ्यः॑ । अपेति॑ । अ॒क्रा॒म॒त् । य॒ज्ञाय॑ । अति॑ष्ठमाना । ते । दे॒वाः । वा॒चि । अप॑क्रान्ताया॒मित्यप॑-क्रा॒न्ता॒या॒म् । तू॒ष्णीम् । ग्रहान्॑ । अ॒गृ॒ह्ण॒त॒ । सा । अ॒म॒न्य॒त॒ । वाक् । अ॒न्तः । य॒न्ति॒ । वै । मा॒ । इति॑ । सा । आ॒ग्र॒य॒णम् । प्रति॑ । एति॑ । अ॒ग॒च्छ॒त् । तत् । आ॒ग्र॒य॒णस्य॑ । आ॒ग्र॒य॒ण॒त्वमित्या᳚ग्रयण - त्वम् ।  \newline




\markright{ TS 6.4.11.3  \hfill https://www.vedavms.in \hfill}
\addcontentsline{toc}{section}{ TS 6.4.11.3 }
\section*{ TS 6.4.11.3 }

\textbf{TS 6.4.11.3 } \newline
\textbf{Samhita Paata} \newline

तस्मा॑दाग्रय॒णे वाग्वि सृ॑ज्यते॒ यत् तू॒ष्णीं पूर्वे॒ ग्रहा॑ गृ॒ह्यन्ते॒ यथा᳚थ्सा॒रीय॑ति म॒ आख॒ इय॑ति॒ नाप॑ राथ्स्या॒-मीत्यु॑पावसृ॒जत्ये॒वमे॒व तद॑द्ध्व॒र्युरा᳚ग्रय॒णं गृ॑ही॒त्वा य॒ज्ञ्मा॒रभ्य॒ वाचं॒ ॅवि सृ॑जते॒ त्रिर्.हिं क॑रोत्युद्गा॒तॄ-ने॒व तद् वृ॑णीते प्र॒जाप॑ति॒र्वा ए॒ष यदा᳚ग्रय॒णो यदा᳚ग्रय॒णं गृ॑ही॒त्वा हिं॑ क॒रोति॑ प्र॒जाप॑तिरे॒व- [  ] \newline

\textbf{Pada Paata} \newline

तस्मा᳚त् । आ॒ग्र॒य॒णे । वाक् । वीति॑ । सृ॒ज्य॒ते॒ । यत् । तू॒ष्णीम् । पूर्वे᳚ । ग्रहाः᳚ । गृ॒ह्यन्ते᳚ । यथा᳚ । थ्सा॒री । इय॑ति । मे॒ । आखः॑ । इय॑ति । न । अपेति॑ । रा॒थ्स्या॒मि॒ । इति॑ । उ॒पा॒व॒सृ॒जतीत्यु॑प - अ॒व॒सृ॒जति॑ । ए॒वम् । ए॒व । तत् । अ॒द्ध्व॒र्युः । आ॒ग्र॒य॒णम् । गृ॒ही॒त्वा । य॒ज्ञ्म् । आ॒रभ्येत्या᳚-रभ्य॑ । वाच᳚म् । वीति॑ । सृ॒ज॒ते॒ । त्रिः । हिम् । क॒रो॒ति॒ । उ॒द्गा॒तॄनित्यु॑त् - गा॒तॄन् । ए॒व । तत् । वृ॒णी॒ते॒ । प्र॒जाप॑ति॒रिति॑ प्र॒जा - प॒तिः॒ । वै । ए॒षः । यत् । आ॒ग्र॒य॒णः । यत् । आ॒ग्र॒य॒णम् । गृ॒ही॒त्वा । हि॒कं॒रोतीति॑ हिम् - क॒रोति॑ । प्र॒जाप॑ति॒रिति॑ प्र॒जा-प॒तिः॒ । ए॒व ।  \newline




\markright{ TS 6.4.11.4  \hfill https://www.vedavms.in \hfill}
\addcontentsline{toc}{section}{ TS 6.4.11.4 }
\section*{ TS 6.4.11.4 }

\textbf{TS 6.4.11.4 } \newline
\textbf{Samhita Paata} \newline

तत् प्र॒जा अ॒भि जि॑घ्रति॒ तस्मा᳚द् व॒थ्सं जा॒तं गौर॒भि जि॑घ्रत्या॒त्मा वा ए॒ष य॒ज्ञ्स्य॒ यदा᳚ग्रय॒णः सव॑नेसवने॒ऽभि गृ॑ह्णात्या॒त्मन्ने॒व य॒ज्ञ्ꣳ सं त॑नोत्यु॒परि॑ष्टा॒दा न॑यति॒ रेत॑ ए॒व तद् द॑धात्य॒धस्ता॒दुप॑ गृह्णाति॒ प्र ज॑नयत्ये॒व तद्ब्र॑ह्मवा॒दिनो॑ वदन्ति॒ कस्मा᳚थ् स॒त्याद्-गा॑य॒त्री कनि॑ष्ठा॒ छन्द॑साꣳ स॒ती सर्वा॑णि॒ सव॑नानि वह॒तीत्ये॒ ( )-ष वै गा॑यत्रि॒यै व॒थ्सो यदा᳚ग्रय॒णस्तमे॒व तद॑भिनि॒वर्तꣳ॒॒ सर्वा॑णि॒ सव॑नानि वहति॒ तस्मा᳚द् व॒थ्सम॒पाकृ॑तं॒ गौर॒भि नि व॑र्तते ॥ \newline

\textbf{Pada Paata} \newline

तत् । प्र॒जा इति॑ प्र - जाः । अ॒भीति॑ । जि॒घ्र॒ति॒ । तस्मा᳚त् । व॒थ्सम् । जा॒तम् । गौः । अ॒भीति॑ । जि॒घ्र॒ति॒ । आ॒त्मा । वै । ए॒षः । य॒ज्ञ्स्य॑ । यत् । आ॒ग्र॒य॒णः । सव॑नेसवन॒ इति॒ सव॑ने - स॒व॒ने॒ । अ॒भीति॑ । गृ॒ह्णा॒ति॒ । आ॒त्मन्न् । ए॒व । य॒ज्ञ्म् । समिति॑ । त॒नो॒ति॒ । उ॒परि॑ष्टात् । एति॑ । न॒य॒ति॒ । रेतः॑ । ए॒व । तत् । द॒धा॒ति॒ । अ॒धस्ता᳚त् । उपेति॑ । गृ॒ह्णा॒ति॒ । प्रेति॑ । ज॒न॒य॒ति॒ । ए॒व । तत् । ब्र॒ह्म॒वा॒दिन॒ इति॑ ब्रह्म - वा॒दिनः॑ । व॒द॒न्ति॒ । कस्मा᳚त् । स॒त्यात् । गा॒य॒त्री । कनि॑ष्ठा । छन्द॑साम् । स॒ती । सर्वा॑णि । सव॑नानि । व॒ह॒ति॒ । इति॑ ( ) । ए॒षः । वै । गा॒य॒त्रि॒यै । व॒थ्सः । यत् । आ॒ग्र॒य॒णः । तम् । ए॒व । तत् । अ॒भि॒नि॒वर्त॒मित्य॑भि - नि॒वर्त᳚म् । सर्वा॑णि । सव॑नानि । व॒ह॒ति॒ । तस्मा᳚त् । व॒थ्सम् । अ॒पाकृ॑त॒मित्य॑प - आकृ॑तम् । गौः । अ॒भ । नीति॑ । व॒र्त॒ते॒ ॥  \newline






\end{document}