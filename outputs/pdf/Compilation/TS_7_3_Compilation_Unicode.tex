\documentclass[17pt]{extarticle}
\usepackage{babel}
\usepackage{fontspec}
\usepackage{polyglossia}
\usepackage{extsizes}

\usepackage{color}   %May be necessary if you want to color links
\usepackage{hyperref}
\hypersetup{
    colorlinks=true, %set true if you want colored links
    linktoc=all,     %set to all if you want both sections and subsections linked
    linkcolor=black,  %choose some color if you want links to stand out
}

\setmainlanguage{sanskrit}
\setotherlanguages{english} %% or other languages
\setlength{\parindent}{0pt}
\pagestyle{myheadings}
\newfontfamily\devanagarifont[Script=Devanagari]{AdishilaVedic}
\renewcommand{\theHsection}{\thepart.section.\thesection}

\newcommand{\VAR}[1]{}
\newcommand{\BLOCK}[1]{}




\begin{document}
\begin{titlepage}
    \begin{center}
 
\begin{sanskrit}
    { \Large
    कृष्ण यजुर्वेदीय तैत्तिरीय संहिता,पद,जटा,घन पाठः 
    }
    \\
    \vspace{2.5cm}
    \mbox{ \Large
    7.3      सप्तमकाण्डे तृतीयः प्रश्नः - सत्रजातनिरूपणं   }
\end{sanskrit}
\end{center}

\end{titlepage}
\tableofcontents
\phantomsection
\pagebreak

\markright{ TS 7.3.1.1  \hfill https://www.vedavms.in \hfill}

\section{ TS 7.3.1.1 }

\textbf{TS 7.3.1.1 } \newline
\textbf{Samhita Paata} \newline

प्र॒जवं॒ ॅवा ए॒तेन॑ यन्ति॒ यद्-द॑श॒ममहः॑ पापाव॒हीयं॒ ॅवा ए॒तेन॑ भवन्ति॒ यद्-द॑श॒ममह॒र्यो वै प्र॒जवं॑ ॅय॒तामप॑थेन प्रति॒पद्य॑ते॒ यः स्था॒णुꣳ हन्ति॒ यो भ्रेषं॒ न्येति॒ स ही॑यते॒ स यो वै द॑श॒मेऽह॑न्नविवा॒क्य उ॑पह॒न्यते॒ स ही॑यते॒ तस्मै॒ य उप॑हताय॒ व्याह॒ तमे॒वान्वा॒रभ्य॒ सम॑श्नु॒तेऽथ॒ यो व्याह॒ स - [  ] \newline

\textbf{Pada Paata} \newline

प्र॒जव॒मिति॑ प्र - जव᳚म् । वै । ए॒तेन॑ । य॒न्ति॒ । यत् । द॒श॒मम् । अहः॑ । पा॒पा॒व॒हीय॒मिति॑ पाप - अ॒व॒हीय᳚म् । वै । ए॒तेन॑ । भ॒व॒न्ति॒ । यत् । द॒श॒मम् । अहः॑ । यः । वै । प्र॒जव॒मिति॑ प्र - जव᳚म् । य॒ताम् । अप॑थेन । प्र॒ति॒पद्य॑त॒ इति॑ प्रति - पद्य॑ते । यः । स्था॒णुम् । हन्ति॑ । यः । भ्रेष᳚म् । न्येतीति॑ नि - एति॑ । सः । ही॒य॒ते॒ । सः । यः । वै । द॒श॒मे । अहन्न्॑ । अ॒वि॒वा॒क्य इत्य॑वि - वा॒क्ये । उ॒प॒ह॒न्यत॒ इत्यु॑प-ह॒न्यते᳚ । सः । ही॒य॒ते॒ । तस्मै᳚ । यः । उप॑हता॒येत्युप॑-ह॒ता॒य॒ । व्याहेति॑ वि - आह॑ । तम् । ए॒व । अ॒न्वा॒रभ्येत्य॑नु - आ॒रभ्य॑ । समिति॑ । अ॒श्नु॒ते॒ । अथ॑ । यः । व्याहेति॑ वि - आह॑ । सः । 1 ( 50)  \newline


\textbf{Krama Paata} \newline

प्र॒जव॒म् ॅवै । प्र॒जव॒मिति॑ प्र - जव᳚म् । वा ए॒तेन॑ । ए॒तेन॑ यन्ति । य॒न्ति॒ यत् । यद् द॑श॒मम् । द॒श॒ममहः॑ । अहः॑ पापाव॒हीय᳚म् । पा॒पा॒व॒हीय॒म् ॅवै । पा॒पा॒व॒हीय॒मिति॑ पाप - अ॒व॒हीय᳚म् । वा ए॒तेन॑ । ए॒तेन॑ भवन्ति । भ॒व॒न्ति॒ यत् । यद् द॑श॒मम् । द॒श॒ममहः॑ । अह॒र् यः । यो वै । वै प्र॒जव᳚म् । प्र॒जव॑म् ॅय॒ताम् । प्र॒जव॒मिति॑ प्र - जव᳚म् । य॒तामप॑थेन । अप॑थेन प्रति॒पद्य॑ते । प्र॒ति॒पद्य॑ते॒ यः । प्र॒ति॒पद्य॑त॒ इति॑ प्रति - पद्य॑ते । यः स्था॒णुम् । स्था॒णुꣳ हन्ति॑ । हन्ति॒ यः । यो भ्रेष᳚म् । भ्रेष॒म् न्येति॑ । न्येति॒ सः । न्येतीति॑ नि - एति॑ । स ही॑यते । ही॒य॒ते॒ सः । स यः । यो वै । वै द॑श॒मे । द॒श॒मेऽहन्न्॑ । अह॑न्नविवा॒क्ये । अ॒वि॒वा॒क्य उ॑पह॒न्यते᳚ । अ॒वि॒वा॒क्य इत्य॑वि - वा॒क्ये । उ॒प॒ह॒न्यते॒ सः । उ॒प॒ह॒न्यत॒ इत्यु॑प - ह॒न्यते᳚ । स ही॑यते । ही॒य॒ते॒ तस्मै᳚ । तस्मै॒ यः । य उप॑हताय । उप॑हताय॒ व्याह॑ । उप॑हता॒येत्युप॑ - ह॒ता॒य॒ । व्याह॒ तम् । व्याहेति॑ वि - आह॑ । तमे॒व । ए॒वान्वा॒रभ्य॑ । अ॒न्वा॒रभ्य॒ सम् । अ॒न्वा॒रभ्येत्य॑नु - आ॒रभ्य॑ । सम॑श्ञुते । अ॒श्ञु॒तेऽथ॑ । अथ॒ यः । यो व्याह॑ । व्याह॒ सः । व्याहेति॑ वि - आह॑ । स ही॑यते \newline

\textbf{Jatai Paata} \newline

1. प्र॒जवं॒ ॅवै वै प्र॒जव॑म् प्र॒जवं॒ ॅवै । \newline
2. प्र॒जव॒मिति॑ प्र - जव᳚म् । \newline
3. वा ए॒ते नै॒तेन॒ वै वा ए॒तेन॑ । \newline
4. ए॒तेन॑ यन्ति यन्त्ये॒ते नै॒तेन॑ यन्ति । \newline
5. य॒न्ति॒ यद् यद् य॑न्ति यन्ति॒ यत् । \newline
6. यद् द॑श॒मम् द॑श॒मं ॅयद् यद् द॑श॒मम् । \newline
7. द॒श॒म मह॒ रह॑र् दश॒मम् द॑श॒म महः॑ । \newline
8. अहः॑ पापाव॒हीय॑म् पापाव॒हीय॒ मह॒ रहः॑ पापाव॒हीय᳚म् । \newline
9. पा॒पा॒व॒हीयं॒ ॅवै वै पा॑पाव॒हीय॑म् पापाव॒हीयं॒ ॅवै । \newline
10. पा॒पा॒व॒हीय॒मिति॑ पाप - अ॒व॒हीय᳚म् । \newline
11. वा ए॒ते नै॒तेन॒ वै वा ए॒तेन॑ । \newline
12. ए॒तेन॑ भवन्ति भव न्त्ये॒ते नै॒तेन॑ भवन्ति । \newline
13. भ॒व॒न्ति॒ यद् यद् भ॑वन्ति भवन्ति॒ यत् । \newline
14. यद् द॑श॒मम् द॑श॒मं ॅयद् यद् द॑श॒मम् । \newline
15. द॒श॒म मह॒ रह॑र् दश॒मम् द॑श॒म महः॑ । \newline
16. अह॒र् यो यो ऽह॒ रह॒र् यः । \newline
17. यो वै वै यो यो वै । \newline
18. वै प्र॒जव॑म् प्र॒जवं॒ ॅवै वै प्र॒जव᳚म् । \newline
19. प्र॒जवं॑ ॅय॒तां ॅय॒ताम् प्र॒जव॑म् प्र॒जवं॑ ॅय॒ताम् । \newline
20. प्र॒जव॒मिति॑ प्र - जव᳚म् । \newline
21. य॒ता मप॑थे॒ना प॑थेन य॒तां ॅय॒ता मप॑थेन । \newline
22. अप॑थेन प्रति॒पद्य॑ते प्रति॒पद्य॒ते ऽप॑थे॒ना प॑थेन प्रति॒पद्य॑ते । \newline
23. प्र॒ति॒पद्य॑ते॒ यो यः प्र॑ति॒पद्य॑ते प्रति॒पद्य॑ते॒ यः । \newline
24. प्र॒ति॒पद्य॑त॒ इति॑ प्रति - पद्य॑ते । \newline
25. यः स्था॒णुꣳ स्था॒णुं ॅयो यः स्था॒णुम् । \newline
26. स्था॒णुꣳ हन्ति॒ हन्ति॑ स्था॒णुꣳ स्था॒णुꣳ हन्ति॑ । \newline
27. हन्ति॒ यो यो हन्ति॒ हन्ति॒ यः । \newline
28. यो भ्रेष॒म् भ्रेषं॒ ॅयो यो भ्रेष᳚म् । \newline
29. भ्रेष॒न् न्येति॒ न्येति॒ भ्रेष॒म् भ्रेष॒न् न्येति॑ । \newline
30. न्येति॒ स स न्येति॒ न्येति॒ सः । \newline
31. न्येतीति॑ नि - एति॑ । \newline
32. स ही॑यते हीयते॒ स स ही॑यते । \newline
33. ही॒य॒ते॒ स स ही॑यते हीयते॒ सः । \newline
34. स यो यः स स यः । \newline
35. यो वै वै यो यो वै । \newline
36. वै द॑श॒मे द॑श॒मे वै वै द॑श॒मे । \newline
37. द॒श॒मे ऽह॒न् नह॑न् दश॒मे द॑श॒मे ऽहन्न्॑ । \newline
38. अह॑न् नविवा॒क्ये॑ ऽविवा॒क्ये ऽह॒न् नह॑न् नविवा॒क्ये । \newline
39. अ॒वि॒वा॒क्य उ॑पह॒न्यत॑ उपह॒न्यते॑ ऽविवा॒क्ये॑ ऽविवा॒क्य उ॑पह॒न्यते᳚ । \newline
40. अ॒वि॒वा॒क्य इत्य॑वि - वा॒क्ये । \newline
41. उ॒प॒ह॒न्यते॒ स स उ॑पह॒न्यत॑ उपह॒न्यते॒ सः । \newline
42. उ॒प॒ह॒न्यत॒ इत्यु॑प - ह॒न्यते᳚ । \newline
43. स ही॑यते हीयते॒ स स ही॑यते । \newline
44. ही॒य॒ते॒ तस्मै॒ तस्मै॑ हीयते हीयते॒ तस्मै᳚ । \newline
45. तस्मै॒ यो य स्तस्मै॒ तस्मै॒ यः । \newline
46. य उप॑हता॒यो प॑हताय॒ यो य उप॑हताय । \newline
47. उप॑हताय॒ व्याह॒ व्याहोप॑हता॒ योप॑हताय॒ व्याह॑ । \newline
48. उप॑हता॒येत्युप॑ - ह॒ता॒य॒ । \newline
49. व्याह॒ तम् तं ॅव्याह॒ व्याह॒ तम् । \newline
50. व्याहेति॑ वि - आह॑ । \newline
51. त मे॒वैव तम् त मे॒व । \newline
52. ए॒वा न्वा॒रभ्या᳚ न्वा॒रभ्यै॒ वैवा न्वा॒रभ्य॑ । \newline
53. अ॒न्वा॒रभ्य॒ सꣳ स म॑न्वा॒रभ्या᳚ न्वा॒रभ्य॒ सम् । \newline
54. अ॒न्वा॒रभ्येत्य॑नु - आ॒रभ्य॑ । \newline
55. स म॑श्ञुते ऽश्ञुते॒ सꣳ स म॑श्ञुते । \newline
56. अ॒श्ञु॒ते ऽथा था᳚श्ञुते ऽश्ञु॒ते ऽथ॑ । \newline
57. अथ॒ यो यो ऽथाथ॒ यः । \newline
58. यो व्याह॒ व्याह॒ यो यो व्याह॑ । \newline
59. व्याह॒ स स व्याह॒ व्याह॒ सः । \newline
60. व्याहेति॑ वि - आह॑ । \newline
61. स ही॑यते हीयते॒ स स ही॑यते । \newline

\textbf{Ghana Paata } \newline

1. प्र॒जवं॒ ॅवै वै प्र॒जव॑म् प्र॒जवं॒ ॅवा ए॒ते नै॒तेन॒ वै प्र॒जव॑म् प्र॒जवं॒ ॅवा ए॒तेन॑ । \newline
2. प्र॒जव॒मिति॑ प्र - जव᳚म् । \newline
3. वा ए॒ते नै॒तेन॒ वै वा ए॒तेन॑ यन्ति यन्त्ये॒तेन॒ वै वा ए॒तेन॑ यन्ति । \newline
4. ए॒तेन॑ यन्ति यन्त्ये॒ते नै॒तेन॑ यन्ति॒ यद् यद् य॑न्त्ये॒ते नै॒तेन॑ यन्ति॒ यत् । \newline
5. य॒न्ति॒ यद् यद् य॑न्ति यन्ति॒ यद् द॑श॒मम् द॑श॒मं ॅयद् य॑न्ति यन्ति॒ यद् द॑श॒मम् । \newline
6. यद् द॑श॒मम् द॑श॒मं ॅयद् यद् द॑श॒म मह॒ रह॑र् दश॒मं ॅयद् यद् द॑श॒म महः॑ । \newline
7. द॒श॒म मह॒ रह॑र् दश॒मम् द॑श॒म महः॑ पापाव॒हीय॑म् पापाव॒हीय॒ मह॑र् दश॒मम् द॑श॒म महः॑ पापाव॒हीय᳚म् । \newline
8. अहः॑ पापाव॒हीय॑म् पापाव॒हीय॒ मह॒ रहः॑ पापाव॒हीयं॒ ॅवै वै पा॑पाव॒हीय॒ मह॒ रहः॑ पापाव॒हीयं॒ ॅवै । \newline
9. पा॒पा॒व॒हीयं॒ ॅवै वै पा॑पाव॒हीय॑म् पापाव॒हीयं॒ ॅवा ए॒ते नै॒तेन॒ वै पा॑पाव॒हीय॑म् पापाव॒हीयं॒ ॅवा ए॒तेन॑ । \newline
10. पा॒पा॒व॒हीय॒मिति॑ पाप - अ॒व॒हीय᳚म् । \newline
11. वा ए॒ते नै॒तेन॒ वै वा ए॒तेन॑ भवन्ति भव न्त्ये॒तेन॒ वै वा ए॒तेन॑ भवन्ति । \newline
12. ए॒तेन॑ भवन्ति भव न्त्ये॒ते नै॒तेन॑ भवन्ति॒ यद् यद् भ॑व न्त्ये॒ते नै॒तेन॑ भवन्ति॒ यत् । \newline
13. भ॒व॒न्ति॒ यद् यद् भ॑वन्ति भवन्ति॒ यद् द॑श॒मम् द॑श॒मं ॅयद् भ॑वन्ति भवन्ति॒ यद् द॑श॒मम् । \newline
14. यद् द॑श॒मम् द॑श॒मं ॅयद् यद् द॑श॒म मह॒ रह॑र् दश॒मं ॅयद् यद् द॑श॒म महः॑ । \newline
15. द॒श॒म मह॒ रह॑र् दश॒मम् द॑श॒म मह॒र् यो यो ऽह॑र् दश॒मम् द॑श॒म मह॒र् यः । \newline
16. अह॒र् यो यो ऽह॒ रह॒र् यो वै वै यो ऽह॒ रह॒र् यो वै । \newline
17. यो वै वै यो यो वै प्र॒जव॑म् प्र॒जवं॒ ॅवै यो यो वै प्र॒जव᳚म् । \newline
18. वै प्र॒जव॑म् प्र॒जवं॒ ॅवै वै प्र॒जवं॑ ॅय॒तां ॅय॒ताम् प्र॒जवं॒ ॅवै वै प्र॒जवं॑ ॅय॒ताम् । \newline
19. प्र॒जवं॑ ॅय॒तां ॅय॒ताम् प्र॒जव॑म् प्र॒जवं॑ ॅय॒ता मप॑थे॒ना प॑थेन य॒ताम् प्र॒जव॑म् प्र॒जवं॑ ॅय॒ता मप॑थेन । \newline
20. प्र॒जव॒मिति॑ प्र - जव᳚म् । \newline
21. य॒ता मप॑थे॒ना प॑थेन य॒तां ॅय॒ता मप॑थेन प्रति॒पद्य॑ते प्रति॒पद्य॒ते ऽप॑थेन य॒तां ॅय॒ता मप॑थेन प्रति॒पद्य॑ते । \newline
22. अप॑थेन प्रति॒पद्य॑ते प्रति॒पद्य॒ते ऽप॑थे॒ना प॑थेन प्रति॒पद्य॑ते॒ यो यः प्र॑ति॒पद्य॒ते ऽप॑थे॒ना प॑थेन प्रति॒पद्य॑ते॒ यः । \newline
23. प्र॒ति॒पद्य॑ते॒ यो यः प्र॑ति॒पद्य॑ते प्रति॒पद्य॑ते॒ यः स्था॒णुꣳ स्था॒णुं ॅयः प्र॑ति॒पद्य॑ते प्रति॒पद्य॑ते॒ यः स्था॒णुम् । \newline
24. प्र॒ति॒पद्य॑त॒ इति॑ प्रति - पद्य॑ते । \newline
25. यः स्था॒णुꣳ स्था॒णुं ॅयो यः स्था॒णुꣳ हन्ति॒ हन्ति॑ स्था॒णुं ॅयो यः स्था॒णुꣳ हन्ति॑ । \newline
26. स्था॒णुꣳ हन्ति॒ हन्ति॑ स्था॒णुꣳ स्था॒णुꣳ हन्ति॒ यो यो हन्ति॑ स्था॒णुꣳ स्था॒णुꣳ हन्ति॒ यः । \newline
27. हन्ति॒ यो यो हन्ति॒ हन्ति॒ यो भ्रेष॒म् भ्रेषं॒ ॅयो हन्ति॒ हन्ति॒ यो भ्रेष᳚म् । \newline
28. यो भ्रेष॒म् भ्रेषं॒ ॅयो यो भ्रेष॒न् न्येति॒ न्येति॒ भ्रेषं॒ ॅयो यो भ्रेष॒न् न्येति॑ । \newline
29. भ्रेष॒न् न्येति॒ न्येति॒ भ्रेष॒म् भ्रेष॒न् न्येति॒ स स न्येति॒ भ्रेष॒म् भ्रेष॒न् न्येति॒ सः । \newline
30. न्येति॒ स स न्येति॒ न्येति॒ स ही॑यते हीयते॒ स न्येति॒ न्येति॒ स ही॑यते । \newline
31. न्येतीति॑ नि - एति॑ । \newline
32. स ही॑यते हीयते॒ स स ही॑यते॒ स स ही॑यते॒ स स ही॑यते॒ सः । \newline
33. ही॒य॒ते॒ स स ही॑यते हीयते॒ स यो यः स ही॑यते हीयते॒ स यः । \newline
34. स यो यः स स यो वै वै यः स स यो वै । \newline
35. यो वै वै यो यो वै द॑श॒मे द॑श॒मे वै यो यो वै द॑श॒मे । \newline
36. वै द॑श॒मे द॑श॒मे वै वै द॑श॒मे ऽह॒न् नह॑न् दश॒मे वै वै द॑श॒मे ऽहन्न्॑ । \newline
37. द॒श॒मे ऽह॒न् नह॑न् दश॒मे द॑श॒मे ऽह॑न् नविवा॒क्ये॑ ऽविवा॒क्ये ऽह॑न् दश॒मे द॑श॒मे ऽह॑न् नविवा॒क्ये । \newline
38. अह॑न् नविवा॒क्ये॑ ऽविवा॒क्ये ऽह॒न् नह॑न् नविवा॒क्य उ॑पह॒न्यत॑ उपह॒न्यते॑ ऽविवा॒क्ये ऽह॒न् नह॑न् नविवा॒क्य उ॑पह॒न्यते᳚ । \newline
39. अ॒वि॒वा॒क्य उ॑पह॒न्यत॑ उपह॒न्यते॑ ऽविवा॒क्ये॑ ऽविवा॒क्य उ॑पह॒न्यते॒ स स उ॑पह॒न्यते॑ ऽविवा॒क्ये॑ ऽविवा॒क्य उ॑पह॒न्यते॒ सः । \newline
40. अ॒वि॒वा॒क्य इत्य॑वि - वा॒क्ये । \newline
41. उ॒प॒ह॒न्यते॒ स स उ॑पह॒न्यत॑ उपह॒न्यते॒ स ही॑यते हीयते॒ स उ॑पह॒न्यत॑ उपह॒न्यते॒ स ही॑यते । \newline
42. उ॒प॒ह॒न्यत॒ इत्यु॑प - ह॒न्यते᳚ । \newline
43. स ही॑यते हीयते॒ स स ही॑यते॒ तस्मै॒ तस्मै॑ हीयते॒ स स ही॑यते॒ तस्मै᳚ । \newline
44. ही॒य॒ते॒ तस्मै॒ तस्मै॑ हीयते हीयते॒ तस्मै॒ यो य स्तस्मै॑ हीयते हीयते॒ तस्मै॒ यः । \newline
45. तस्मै॒ यो य स्तस्मै॒ तस्मै॒ य उप॑हता॒ योप॑हताय॒ य स्तस्मै॒ तस्मै॒ य उप॑हताय । \newline
46. य उप॑हता॒ योप॑हताय॒ यो य उप॑हताय॒ व्याह॒ व्याहोप॑हताय॒ यो य उप॑हताय॒ व्याह॑ । \newline
47. उप॑हताय॒ व्याह॒ व्याहोप॑हता॒ योप॑हताय॒ व्याह॒ तम् तं ॅव्याहोप॑हता॒ योप॑हताय॒ व्याह॒ तम् । \newline
48. उप॑हता॒येत्युप॑ - ह॒ता॒य॒ । \newline
49. व्याह॒ तम् तं ॅव्याह॒ व्याह॒ त मे॒वैव तं ॅव्याह॒ व्याह॒ त मे॒व । \newline
50. व्याहेति॑ वि - आह॑ । \newline
51. त मे॒वैव तम् त मे॒वा न्वा॒रभ्या᳚ न्वा॒रभ्यै॒व तम् त मे॒वा न्वा॒रभ्य॑ । \newline
52. ए॒वा न्वा॒रभ्या᳚ न्वा॒रभ्यै॒वैवा न्वा॒रभ्य॒ सꣳ स म॑न्वा॒रभ्यै॒वैवा न्वा॒रभ्य॒ सम् । \newline
53. अ॒न्वा॒रभ्य॒ सꣳ स म॑न्वा॒रभ्या᳚ न्वा॒रभ्य॒ स म॑श्ञुते ऽश्ञुते॒ स म॑न्वा॒रभ्या᳚ न्वा॒रभ्य॒ स म॑श्ञुते । \newline
54. अ॒न्वा॒रभ्येत्य॑नु - आ॒रभ्य॑ । \newline
55. स म॑श्ञुते ऽश्ञुते॒ सꣳ स म॑श्ञु॒ते ऽथा था᳚श्ञुते॒ सꣳ स म॑श्ञु॒ते ऽथ॑ । \newline
56. अ॒श्ञु॒ते ऽथा था᳚श्ञुते ऽश्ञु॒ते ऽथ॒ यो यो ऽथा᳚श्ञुते ऽश्ञु॒ते ऽथ॒ यः । \newline
57. अथ॒ यो यो ऽथाथ॒ यो व्याह॒ व्याह॒ यो ऽथाथ॒ यो व्याह॑ । \newline
58. यो व्याह॒ व्याह॒ यो यो व्याह॒ स स व्याह॒ यो यो व्याह॒ सः । \newline
59. व्याह॒ स स व्याह॒ व्याह॒ स ही॑यते हीयते॒ स व्याह॒ व्याह॒ स ही॑यते । \newline
60. व्याहेति॑ वि - आह॑ । \newline
61. स ही॑यते हीयते॒ स स ही॑यते॒ तस्मा॒त् तस्मा᳚ द्धीयते॒ स स ही॑यते॒ तस्मा᳚त् । \newline
\pagebreak
\markright{ TS 7.3.1.2  \hfill https://www.vedavms.in \hfill}

\section{ TS 7.3.1.2 }

\textbf{TS 7.3.1.2 } \newline
\textbf{Samhita Paata} \newline

ही॑यते॒ तस्मा᳚द्-दश॒मे ऽह॑न्नविवा॒क्य उप॑हताय॒ न व्युच्य॒मथो॒ खल्वा॑हुर्य॒ज्ञ्स्य॒ वै समृ॑द्धेन दे॒वाः सु॑व॒र्गं ॅलो॒कमा॑यन् य॒ज्ञ्स्य॒ व्यृ॑द्धे॒नासु॑रा॒न् परा॑ऽभावय॒न्निति॒ यत् खलु॒ वै य॒ज्ञ्स्य॒ समृ॑द्धं॒ तद्-यज॑मानस्य॒ यद्व्यृ॑द्धं॒ तद्भ्रातृ॑व्यस्य॒ स यो वै द॑श॒मेऽह॑न्नविवा॒क्य उ॑पह॒न्यते॒ स ए॒वाति॑ रेचयति॒ ते ये बाह्या॑ दृशी॒कवः॒ - [  ] \newline

\textbf{Pada Paata} \newline

ही॒य॒ते॒ । तस्मा᳚त् । द॒श॒मे । अहन्न्॑ । अ॒वि॒वा॒क्य इत्य॑वि - वा॒क्ये । उप॑हता॒येत्युप॑ - ह॒ता॒य॒ । न । व्युच्य॒मिति॑ वि-उच्य᳚म् । अथो॒ इति॑ । खलु॑ । आ॒हुः॒ । य॒ज्ञ्स्य॑ । वै । समृ॑द्धे॒नेति॒ सं - ऋ॒द्धे॒न॒ । दे॒वाः । सु॒व॒र्गमिति॑ सुवः - गम् । लो॒कम् । आ॒य॒न्न् । य॒ज्ञ्स्य॑ । व्यृ॑द्धे॒नेति॒ वि - ऋ॒द्धे॒न॒ । असु॑रान् । परेति॑ । अ॒भा॒व॒य॒न्न् । इति॑ । यत् । खलु॑ । वै । य॒ज्ञ्स्य॑ । समृ॑द्ध॒मिति॒ सं - ऋ॒द्ध॒म् । तत् । यज॑मानस्य । यत् । व्यृ॑द्ध॒मिति॒ वि-ऋ॒द्ध॒म् । तत् । भ्रातृ॑व्यस्य । सः । यः । वै । द॒श॒मे । अहन्न्॑ । अ॒वि॒वा॒क्य इत्य॑वि - वा॒क्ये । उ॒प॒ह॒न्यत॒ इत्यु॑प - ह॒न्यते᳚ । सः । ए॒व । अतीति॑ । रे॒च॒य॒ति॒ । ते । ये । बाह्याः᳚ । दृ॒शी॒कवः॑ ।  \newline


\textbf{Krama Paata} \newline

ही॒य॒ते॒ तस्मा᳚त् । तस्मा᳚द् दश॒मे । द॒श॒मेऽहन्न्॑ । अह॑न्नविवा॒क्ये । अ॒वि॒वा॒क्य उप॑हताय । अ॒वि॒वा॒क्य इत्य॑वि - वा॒क्ये । उप॑हताय॒ न । उप॑हता॒येत्युप॑ - ह॒ता॒य॒ । न व्युच्यम्᳚ । व्युच्य॒मथो᳚ । व्युच्य॒मिति॑ वि - उच्य᳚म् । अथो॒ खलु॑ । अथो॒ इत्यथो᳚ । खल्वा॑हुः । आ॒हु॒र् य॒ज्ञ्स्य॑ । य॒ज्ञ्स्य॒ वै । वै समृ॑द्धेन । समृ॑द्धेन दे॒वाः । समृ॑द्धे॒नेति॒ सम् - ऋ॒द्धे॒न॒ । दे॒वाः सु॑व॒र्गम् । सु॒व॒र्गम् ॅलो॒कम् । सु॒व॒र्गमिति॑ सुवः - गम् । लो॒कमा॑यन्न् । आ॒य॒न्.॒ य॒ज्ञ्स्य॑ । य॒ज्ञ्स्य॒ व्यृ॑द्धेन । व्यृ॑द्धे॒नासु॑रान् । व्यृ॑द्धे॒नेति॒ वि - ऋ॒द्धे॒न॒ । असु॑रा॒न् परा᳚ । परा॑ऽभावयन्न् । अ॒भा॒व॒य॒न्निति॑ । इति॒ यत् । यत् खलु॑ । खलु॒ वै । वै य॒ज्ञ्स्य॑ । य॒ज्ञ्स्य॒ समृ॑द्धम् । समृ॑द्ध॒म् तत् । समृ॑द्ध॒मिति॒ सम् - ऋ॒द्ध॒म् । तद् यज॑मानस्य । यज॑मानस्य॒ यत् । यद् व्यृ॑द्धम् । व्यृ॑द्ध॒म् तत् । व्यृ॑द्ध॒मिति॒ वि - ऋ॒द्ध॒म् । तद् भ्रातृ॑व्यस्य । भ्रातृ॑व्यस्य॒ सः । स यः । यो वै । वै द॑श॒मे । द॒श॒मेऽहन्न्॑ । अह॑न्नविवा॒क्ये । अ॒वि॒वा॒क्य उ॑पह॒न्यते᳚ । अ॒वि॒वा॒क्य इत्य॑वि - वा॒क्ये । उ॒प॒ह॒न्यते॒ सः । उ॒प॒ह॒न्यत॒ इत्यु॑प - ह॒न्यते᳚ । स ए॒व । ए॒वाति॑ । अति॑ रेचयति । रे॒च॒य॒ति॒ ते । ते ये । ये बाह्याः᳚ । बाह्या॑ दृशी॒कवः॑ । दृ॒शी॒कवः॒ स्युः \newline

\textbf{Jatai Paata} \newline

1. ही॒य॒ते॒ तस्मा॒त् तस्मा᳚द् धीयते हीयते॒ तस्मा᳚त् । \newline
2. तस्मा᳚द् दश॒मे द॑श॒मे तस्मा॒त् तस्मा᳚द् दश॒मे । \newline
3. द॒श॒मे ऽह॒न् नह॑न् दश॒मे द॑श॒मे ऽहन्न्॑ । \newline
4. अह॑न् नविवा॒क्ये॑ ऽविवा॒क्ये ऽह॒न् नह॑न् नविवा॒क्ये । \newline
5. अ॒वि॒वा॒क्य उप॑हता॒यो प॑हताया विवा॒क्ये॑ ऽविवा॒क्य उप॑हताय । \newline
6. अ॒वि॒वा॒क्य इत्य॑वि - वा॒क्ये । \newline
7. उप॑हताय॒ न नोप॑हता॒यो प॑हताय॒ न । \newline
8. उप॑हता॒येत्युप॑ - ह॒ता॒य॒ । \newline
9. न व्युच्यं॒ ॅव्युच्य॒न् न न व्युच्य᳚म् । \newline
10. व्युच्य॒ मथो॒ अथो॒ व्युच्यं॒ ॅव्युच्य॒ मथो᳚ । \newline
11. व्युच्य॒मिति॑ वि - उच्य᳚म् । \newline
12. अथो॒ खलु॒ खल्वथो॒ अथो॒ खलु॑ । \newline
13. अथो॒ इत्यथो᳚ । \newline
14. खल्वा॑हु राहुः॒ खलु॒ खल्वा॑हुः । \newline
15. आ॒हु॒र् य॒ज्ञ्स्य॑ य॒ज्ञ् स्या॑हु राहुर् य॒ज्ञ्स्य॑ । \newline
16. य॒ज्ञ्स्य॒ वै वै य॒ज्ञ्स्य॑ य॒ज्ञ्स्य॒ वै । \newline
17. वै समृ॑द्धेन॒ समृ॑द्धेन॒ वै वै समृ॑द्धेन । \newline
18. समृ॑द्धेन दे॒वा दे॒वाः समृ॑द्धेन॒ समृ॑द्धेन दे॒वाः । \newline
19. समृ॑द्धे॒नेति॒ सं - ऋ॒द्धे॒न॒ । \newline
20. दे॒वाः सु॑व॒र्गꣳ सु॑व॒र्गम् दे॒वा दे॒वाः सु॑व॒र्गम् । \newline
21. सु॒व॒र्गम् ॅलो॒कम् ॅलो॒कꣳ सु॑व॒र्गꣳ सु॑व॒र्गम् ॅलो॒कम् । \newline
22. सु॒व॒र्गमिति॑ सुवः - गम् । \newline
23. लो॒क मा॑यन् नायन् ॅलो॒कम् ॅलो॒क मा॑यन्न् । \newline
24. आ॒य॒न्॒. य॒ज्ञ्स्य॑ य॒ज्ञ्स्या॑यन् नायन्. य॒ज्ञ्स्य॑ । \newline
25. य॒ज्ञ्स्य॒ व्यृ॑द्धेन॒ व्यृ॑द्धेन य॒ज्ञ्स्य॑ य॒ज्ञ्स्य॒ व्यृ॑द्धेन । \newline
26. व्यृ॑द्धे॒ना सु॑रा॒ नसु॑रा॒न् व्यृ॑द्धेन॒ व्यृ॑द्धे॒ना सु॑रान् । \newline
27. व्यृ॑द्धे॒नेति॒ वि - ऋ॒द्धे॒न॒ । \newline
28. असु॑रा॒न् परा॒ परा ऽसु॑रा॒ नसु॑रा॒न् परा᳚ । \newline
29. परा॑ ऽभावयन् नभावय॒न् परा॒ परा॑ ऽभावयन्न् । \newline
30. अ॒भा॒व॒य॒न् निती त्य॑भावयन् नभावय॒न् निति॑ । \newline
31. इति॒ यद् यदितीति॒ यत् । \newline
32. यत् खलु॒ खलु॒ यद् यत् खलु॑ । \newline
33. खलु॒ वै वै खलु॒ खलु॒ वै । \newline
34. वै य॒ज्ञ्स्य॑ य॒ज्ञ्स्य॒ वै वै य॒ज्ञ्स्य॑ । \newline
35. य॒ज्ञ्स्य॒ समृ॑द्धꣳ॒॒ समृ॑द्धं ॅय॒ज्ञ्स्य॑ य॒ज्ञ्स्य॒ समृ॑द्धम् । \newline
36. समृ॑द्ध॒म् तत् तथ् समृ॑द्धꣳ॒॒ समृ॑द्ध॒म् तत् । \newline
37. समृ॑द्ध॒मिति॒ सं - ऋ॒द्ध॒म् । \newline
38. तद् यज॑मानस्य॒ यज॑मानस्य॒ तत् तद् यज॑मानस्य । \newline
39. यज॑मानस्य॒ यद् यद् यज॑मानस्य॒ यज॑मानस्य॒ यत् । \newline
40. यद् व्यृ॑द्धं॒ ॅव्यृ॑द्धं॒ ॅयद् यद् व्यृ॑द्धम् । \newline
41. व्यृ॑द्ध॒म् तत् तद् व्यृ॑द्धं॒ ॅव्यृ॑द्ध॒म् तत् । \newline
42. व्यृ॑द्ध॒मिति॒ वि - ऋ॒द्ध॒म् । \newline
43. तद् भ्रातृ॑व्यस्य॒ भ्रातृ॑व्यस्य॒ तत् तद् भ्रातृ॑व्यस्य । \newline
44. भ्रातृ॑व्यस्य॒ स स भ्रातृ॑व्यस्य॒ भ्रातृ॑व्यस्य॒ सः । \newline
45. स यो यः स स यः । \newline
46. यो वै वै यो यो वै । \newline
47. वै द॑श॒मे द॑श॒मे वै वै द॑श॒मे । \newline
48. द॒श॒मे ऽह॒न् नह॑न् दश॒मे द॑श॒मे ऽहन्न्॑ । \newline
49. अह॑न् नविवा॒क्ये॑ ऽविवा॒क्ये ऽह॒न् नह॑न् नविवा॒क्ये । \newline
50. अ॒वि॒वा॒क्य उ॑पह॒न्यत॑ उपह॒न्यते॑ ऽविवा॒क्ये॑ ऽविवा॒क्य उ॑पह॒न्यते᳚ । \newline
51. अ॒वि॒वा॒क्य इत्य॑वि - वा॒क्ये । \newline
52. उ॒प॒ह॒न्यते॒ स स उ॑पह॒न्यत॑ उपह॒न्यते॒ सः । \newline
53. उ॒प॒ह॒न्यत॒ इत्यु॑प - ह॒न्यते᳚ । \newline
54. स ए॒वैव स स ए॒व । \newline
55. ए॒वा त्य त्ये॒वैवाति॑ । \newline
56. अति॑ रेचयति रेचय॒ त्य त्यति॑ रेचयति । \newline
57. रे॒च॒य॒ति॒ ते ते रे॑चयति रेचयति॒ ते । \newline
58. ते ये ये ते ते ये । \newline
59. ये बाह्या॒ बाह्या॒ ये ये बाह्याः᳚ । \newline
60. बाह्या॑ दृशी॒कवो॑ दृशी॒कवो॒ बाह्या॒ बाह्या॑ दृशी॒कवः॑ । \newline
61. दृ॒शी॒कवः॒ स्युः स्युर् दृ॑शी॒कवो॑ दृशी॒कवः॒ स्युः । \newline

\textbf{Ghana Paata } \newline

1. ही॒य॒ते॒ तस्मा॒त् तस्मा᳚ द्धीयते हीयते॒ तस्मा᳚द् दश॒मे द॑श॒मे तस्मा᳚ द्धीयते हीयते॒ तस्मा᳚द् दश॒मे । \newline
2. तस्मा᳚द् दश॒मे द॑श॒मे तस्मा॒त् तस्मा᳚द् दश॒मे ऽह॒न् नह॑न् दश॒मे तस्मा॒त् तस्मा᳚द् दश॒मे ऽहन्न्॑ । \newline
3. द॒श॒मे ऽह॒न् नह॑न् दश॒मे द॑श॒मे ऽह॑न् नविवा॒क्ये॑ ऽविवा॒क्ये ऽह॑न् दश॒मे द॑श॒मे ऽह॑न् नविवा॒क्ये । \newline
4. अह॑न् नविवा॒क्ये॑ ऽविवा॒क्ये ऽह॒न् नह॑न् नविवा॒क्य उप॑हता॒यो प॑हताया विवा॒क्ये ऽह॒न् नह॑न् नविवा॒क्य उप॑हताय । \newline
5. अ॒वि॒वा॒क्य उप॑हता॒ योप॑हताया विवा॒क्ये॑ ऽविवा॒क्य उप॑हताय॒ न नोप॑हताया विवा॒क्ये॑ ऽविवा॒क्य उप॑हताय॒ न । \newline
6. अ॒वि॒वा॒क्य इत्य॑वि - वा॒क्ये । \newline
7. उप॑हताय॒ न नोप॑हता॒ योप॑हताय॒ न व्युच्यं॒ ॅव्युच्य॒म् नोप॑हता॒ योप॑हताय॒ न व्युच्य᳚म् । \newline
8. उप॑हता॒येत्युप॑ - ह॒ता॒य॒ । \newline
9. न व्युच्यं॒ ॅव्युच्य॒न् न न व्युच्य॒ मथो॒ अथो॒ व्युच्य॒न् न न व्युच्य॒ मथो᳚ । \newline
10. व्युच्य॒ मथो॒ अथो॒ व्युच्यं॒ ॅव्युच्य॒ मथो॒ खलु॒ खल्वथो॒ व्युच्यं॒ ॅव्युच्य॒ मथो॒ खलु॑ । \newline
11. व्युच्य॒मिति॑ वि - उच्य᳚म् । \newline
12. अथो॒ खलु॒ खल्वथो॒ अथो॒ खल्वा॑हु राहुः॒ खल्वथो॒ अथो॒ खल्वा॑हुः । \newline
13. अथो॒ इत्यथो᳚ । \newline
14. खल्वा॑हु राहुः॒ खलु॒ खल्वा॑हुर् य॒ज्ञ्स्य॑ य॒ज्ञ् स्या॑हुः॒ खलु॒ खल्वा॑हुर् य॒ज्ञ्स्य॑ । \newline
15. आ॒हु॒र् य॒ज्ञ्स्य॑ य॒ज्ञ् स्या॑हु राहुर् य॒ज्ञ्स्य॒ वै वै य॒ज्ञ् स्या॑हु राहुर् य॒ज्ञ्स्य॒ वै । \newline
16. य॒ज्ञ्स्य॒ वै वै य॒ज्ञ्स्य॑ य॒ज्ञ्स्य॒ वै समृ॑द्धेन॒ समृ॑द्धेन॒ वै य॒ज्ञ्स्य॑ य॒ज्ञ्स्य॒ वै समृ॑द्धेन । \newline
17. वै समृ॑द्धेन॒ समृ॑द्धेन॒ वै वै समृ॑द्धेन दे॒वा दे॒वाः समृ॑द्धेन॒ वै वै समृ॑द्धेन दे॒वाः । \newline
18. समृ॑द्धेन दे॒वा दे॒वाः समृ॑द्धेन॒ समृ॑द्धेन दे॒वाः सु॑व॒र्गꣳ सु॑व॒र्गम् दे॒वाः समृ॑द्धेन॒ समृ॑द्धेन दे॒वाः सु॑व॒र्गम् । \newline
19. समृ॑द्धे॒नेति॒ सं - ऋ॒द्धे॒न॒ । \newline
20. दे॒वाः सु॑व॒र्गꣳ सु॑व॒र्गम् दे॒वा दे॒वाः सु॑व॒र्गम् ॅलो॒कम् ॅलो॒कꣳ सु॑व॒र्गम् दे॒वा दे॒वाः सु॑व॒र्गम् ॅलो॒कम् । \newline
21. सु॒व॒र्गम् ॅलो॒कम् ॅलो॒कꣳ सु॑व॒र्गꣳ सु॑व॒र्गम् ॅलो॒क मा॑यन् नायन् ॅलो॒कꣳ सु॑व॒र्गꣳ सु॑व॒र्गम् ॅलो॒क मा॑यन्न् । \newline
22. सु॒व॒र्गमिति॑ सुवः - गम् । \newline
23. लो॒क मा॑यन् नायन् ॅलो॒कम् ॅलो॒क मा॑यन्. य॒ज्ञ्स्य॑ य॒ज्ञ्स्या॑यन् ॅलो॒कम् ॅलो॒क मा॑यन्. य॒ज्ञ्स्य॑ । \newline
24. आ॒य॒न्॒. य॒ज्ञ्स्य॑ य॒ज्ञ्स्या॑यन् नायन्. य॒ज्ञ्स्य॒ व्यृ॑द्धेन॒ व्यृ॑द्धेन य॒ज्ञ् स्या॑यन् नायन्. य॒ज्ञ्स्य॒ व्यृ॑द्धेन । \newline
25. य॒ज्ञ्स्य॒ व्यृ॑द्धेन॒ व्यृ॑द्धेन य॒ज्ञ्स्य॑ य॒ज्ञ्स्य॒ व्यृ॑द्धे॒ना सु॑रा॒ नसु॑रा॒न् व्यृ॑द्धेन य॒ज्ञ्स्य॑ य॒ज्ञ्स्य॒ व्यृ॑द्धे॒ना सु॑रान् । \newline
26. व्यृ॑द्धे॒ना सु॑रा॒ नसु॑रा॒न् व्यृ॑द्धेन॒ व्यृ॑द्धे॒ना सु॑रा॒न् परा॒ परा ऽसु॑रा॒न् व्यृ॑द्धेन॒ व्यृ॑द्धे॒ना सु॑रा॒न् परा᳚ । \newline
27. व्यृ॑द्धे॒नेति॒ वि - ऋ॒द्धे॒न॒ । \newline
28. असु॑रा॒न् परा॒ परा ऽसु॑रा॒ नसु॑रा॒न् परा॑ ऽभावयन् नभावय॒न् परा ऽसु॑रा॒ नसु॑रा॒न् परा॑ ऽभावयन्न् । \newline
29. परा॑ ऽभावयन् नभावय॒न् परा॒ परा॑ ऽभावय॒न् निती त्य॑भावय॒न् परा॒ परा॑ ऽभावय॒न् निति॑ । \newline
30. अ॒भा॒व॒य॒न् निती त्य॑भावयन् नभावय॒न् निति॒ यद् यदि त्य॑भावयन् नभावय॒न् निति॒ यत् । \newline
31. इति॒ यद् यदितीति॒ यत् खलु॒ खलु॒ यदितीति॒ यत् खलु॑ । \newline
32. यत् खलु॒ खलु॒ यद् यत् खलु॒ वै वै खलु॒ यद् यत् खलु॒ वै । \newline
33. खलु॒ वै वै खलु॒ खलु॒ वै य॒ज्ञ्स्य॑ य॒ज्ञ्स्य॒ वै खलु॒ खलु॒ वै य॒ज्ञ्स्य॑ । \newline
34. वै य॒ज्ञ्स्य॑ य॒ज्ञ्स्य॒ वै वै य॒ज्ञ्स्य॒ समृ॑द्धꣳ॒॒ समृ॑द्धं ॅय॒ज्ञ्स्य॒ वै वै य॒ज्ञ्स्य॒ समृ॑द्धम् । \newline
35. य॒ज्ञ्स्य॒ समृ॑द्धꣳ॒॒ समृ॑द्धं ॅय॒ज्ञ्स्य॑ य॒ज्ञ्स्य॒ समृ॑द्ध॒म् तत् तथ् समृ॑द्धं ॅय॒ज्ञ्स्य॑ य॒ज्ञ्स्य॒ समृ॑द्ध॒म् तत् । \newline
36. समृ॑द्ध॒म् तत् तथ् समृ॑द्धꣳ॒॒ समृ॑द्ध॒म् तद् यज॑मानस्य॒ यज॑मानस्य॒ तथ् समृ॑द्धꣳ॒॒ समृ॑द्ध॒म् तद् यज॑मानस्य । \newline
37. समृ॑द्ध॒मिति॒ सं - ऋ॒द्ध॒म् । \newline
38. तद् यज॑मानस्य॒ यज॑मानस्य॒ तत् तद् यज॑मानस्य॒ यद् यद् यज॑मानस्य॒ तत् तद् यज॑मानस्य॒ यत् । \newline
39. यज॑मानस्य॒ यद् यद् यज॑मानस्य॒ यज॑मानस्य॒ यद् व्यृ॑द्धं॒ ॅव्यृ॑द्धं॒ ॅयद् यज॑मानस्य॒ यज॑मानस्य॒ यद् व्यृ॑द्धम् । \newline
40. यद् व्यृ॑द्धं॒ ॅव्यृ॑द्धं॒ ॅयद् यद् व्यृ॑द्ध॒म् तत् तद् व्यृ॑द्धं॒ ॅयद् यद् व्यृ॑द्ध॒म् तत् । \newline
41. व्यृ॑द्ध॒म् तत् तद् व्यृ॑द्धं॒ ॅव्यृ॑द्ध॒म् तद् भ्रातृ॑व्यस्य॒ भ्रातृ॑व्यस्य॒ तद् व्यृ॑द्धं॒ ॅव्यृ॑द्ध॒म् तद् भ्रातृ॑व्यस्य । \newline
42. व्यृ॑द्ध॒मिति॒ वि - ऋ॒द्ध॒म् । \newline
43. तद् भ्रातृ॑व्यस्य॒ भ्रातृ॑व्यस्य॒ तत् तद् भ्रातृ॑व्यस्य॒ स स भ्रातृ॑व्यस्य॒ तत् तद् भ्रातृ॑व्यस्य॒ सः । \newline
44. भ्रातृ॑व्यस्य॒ स स भ्रातृ॑व्यस्य॒ भ्रातृ॑व्यस्य॒ स यो यः स भ्रातृ॑व्यस्य॒ भ्रातृ॑व्यस्य॒ स यः । \newline
45. स यो यः स स यो वै वै यः स स यो वै । \newline
46. यो वै वै यो यो वै द॑श॒मे द॑श॒मे वै यो यो वै द॑श॒मे । \newline
47. वै द॑श॒मे द॑श॒मे वै वै द॑श॒मे ऽह॒न् नह॑न् दश॒मे वै वै द॑श॒मे ऽहन्न्॑ । \newline
48. द॒श॒मे ऽह॒न् नह॑न् दश॒मे द॑श॒मे ऽह॑न् नविवा॒क्ये॑ ऽविवा॒क्ये ऽह॑न् दश॒मे द॑श॒मे ऽह॑न् नविवा॒क्ये । \newline
49. अह॑न् नविवा॒क्ये॑ ऽविवा॒क्ये ऽह॒न् नह॑न् नविवा॒क्य उ॑पह॒न्यत॑ उपह॒न्यते॑ ऽविवा॒क्ये ऽह॒न् नह॑न् नविवा॒क्य उ॑पह॒न्यते᳚ । \newline
50. अ॒वि॒वा॒क्य उ॑पह॒न्यत॑ उपह॒न्यते॑ ऽविवा॒क्ये॑ ऽविवा॒क्य उ॑पह॒न्यते॒ स स उ॑पह॒न्यते॑ ऽविवा॒क्ये॑ ऽविवा॒क्य उ॑पह॒न्यते॒ सः । \newline
51. अ॒वि॒वा॒क्य इत्य॑वि - वा॒क्ये । \newline
52. उ॒प॒ह॒न्यते॒ स स उ॑पह॒न्यत॑ उपह॒न्यते॒ स ए॒वैव स उ॑पह॒न्यत॑ उपह॒न्यते॒ स ए॒व । \newline
53. उ॒प॒ह॒न्यत॒ इत्यु॑प - ह॒न्यते᳚ । \newline
54. स ए॒वैव स स ए॒वा त्य त्ये॒व स स ए॒वाति॑ । \newline
55. ए॒वात्य त्ये॒वैवाति॑ रेचयति रेचय॒त्य त्ये॒वैवाति॑ रेचयति । \newline
56. अति॑ रेचयति रेचय॒ त्यत्यति॑ रेचयति॒ ते ते रे॑चय॒ त्य त्यति॑ रेचयति॒ ते । \newline
57. रे॒च॒य॒ति॒ ते ते रे॑चयति रेचयति॒ ते ये ये ते रे॑चयति रेचयति॒ ते ये । \newline
58. ते ये ये ते ते ये बाह्या॒ बाह्या॒ ये ते ते ये बाह्याः᳚ । \newline
59. ये बाह्या॒ बाह्या॒ ये ये बाह्या॑ दृशी॒कवो॑ दृशी॒कवो॒ बाह्या॒ ये ये बाह्या॑ दृशी॒कवः॑ । \newline
60. बाह्या॑ दृशी॒कवो॑ दृशी॒कवो॒ बाह्या॒ बाह्या॑ दृशी॒कवः॒ स्युः स्युर् दृ॑शी॒कवो॒ बाह्या॒ बाह्या॑ दृशी॒कवः॒ स्युः । \newline
61. दृ॒शी॒कवः॒ स्युः स्युर् दृ॑शी॒कवो॑ दृशी॒कवः॒ स्यु स्ते ते स्युर् दृ॑शी॒कवो॑ दृशी॒कवः॒ स्यु स्ते । \newline
\pagebreak
\markright{ TS 7.3.1.3  \hfill https://www.vedavms.in \hfill}

\section{ TS 7.3.1.3 }

\textbf{TS 7.3.1.3 } \newline
\textbf{Samhita Paata} \newline

स्युस्ते वि ब्रू॑यु॒र्यदि॒ तत्र॒ न वि॒न्देयु॑रन्तस्सद॒साद्-व्युच्यं॒ ॅयदि॒ तत्र॒ न वि॒न्देयु॑र्गृ॒हप॑तिना॒ व्युच्यं॒ तद्व्युच्य॑मे॒वाथ॒ वा ए॒तथ् स॑र्परा॒ज्ञिया॑ ऋ॒ग्भिः स्तु॑वन्त॒यं ॅवै सर्प॑तो॒ राज्ञी॒ यद्वा अ॒स्यां किं चार्च॑न्ति॒ यदा॑नृ॒चुस्तेने॒यꣳ स॑र्परा॒ज्ञी ते यदे॒व किं च॑ वा॒चा ऽऽनृ॒चुर्यद॒तोऽद्ध्य॑र्चि॒तार॒ - [  ] \newline

\textbf{Pada Paata} \newline

स्युः । ते । वीति॑ । ब्रू॒युः॒ । यदि॑ । तत्र॑ । न । वि॒न्देयुः॑ । अ॒न्त॒स्स॒द॒सादित्य॑न्तः - स॒द॒सात् । व्युच्य॒मिति॑ वि - उच्य᳚म् । यदि॑ । तत्र॑ । न । वि॒न्देयुः॑ । गृ॒हप॑ति॒नेति॑ गृ॒ह - प॒ति॒ना॒ । व्युच्य॒मिति॑ वि- उच्य᳚म् । तत् । व्युच्य॒मिति॑ वि - उच्य᳚म् । ए॒व । अथ॑ । वै । ए॒तत् । स॒र्प॒रा॒ज्ञिया॒ इति॑ सर्प - रा॒ज्ञियाः᳚ । ऋ॒ग्भिरित्यृ॑क् - भिः । स्तु॒व॒न्ति॒ । इ॒यम् । वै । सर्प॑तः । राज्ञी᳚ । यत् । वै । अ॒स्याम् । किम् । च॒ । अर्च॑न्ति । यत् । आ॒नृ॒चुः । तेन॑ । इ॒यम् । स॒र्प॒रा॒ज्ञीति॑ सर्प-रा॒ज्ञी । ते । यत् । ए॒व । किम् । च॒ । वा॒चा । आ॒नृ॒चुः । यत् । अ॒तोधि॑ । अ॒र्चि॒तारः॑ ।  \newline


\textbf{Krama Paata} \newline

स्युस्ते । ते वि । वि ब्रू॑युः । ब्रू॒यु॒र् यदि॑ । यदि॒ तत्र॑ । तत्र॒ न । न वि॒न्देयुः॑ । वि॒न्देयु॑रन्तस्सद॒सात् । अ॒न्त॒स्स॒द॒साद् व्युच्य᳚म् । अ॒न्त॒स्स॒द॒सादित्य॑न्तः - स॒द॒सात् । व्युच्य॒म् ॅयदि॑ । व्युच्य॒मिति॑ वि - उच्य᳚म् । यदि॒ तत्र॑ । तत्र॒ न । न वि॒न्देयुः॑ । वि॒न्देयु॑र् गृ॒हप॑तिना । गृ॒हप॑तिना॒ व्युच्य᳚म् । गृ॒हप॑ति॒नेति॑ गृ॒ह - प॒ति॒ना॒ । व्युच्य॒म् तत् । व्युच्य॒मिति॑ वि - उच्य᳚म् । तद् व्युच्य᳚म् । व्युच्य॑मे॒व । व्युच्य॒मिति॑ वि - उच्य᳚म् । ए॒वाथ॑ । अथ॒ वै । वा ए॒तत् । ए॒तथ् स॑र्परा॒ज्ञियाः᳚ । स॒र्प॒रा॒ज्ञिया॑ ऋ॒ग्भिः । स॒र्प॒रा॒ज्ञिया॒ इति॑ सर्प - रा॒ज्ञियाः᳚ । ऋ॒ग्भिः स्तु॑वन्ति । ऋ॒ग्भिरित्यृ॑क् - भिः । स्तु॒व॒न्ती॒यम् । इ॒यम् ॅवै । वै सर्प॑तः । सर्प॑तो॒ राज्ञी᳚ । राज्ञी॒ यत् । यद् वै । वा अ॒स्याम् । अ॒स्याम् किम् । किम् च॑ । चार्च॑न्ति । अर्च॑न्ति॒ यत् । यदा॑नृ॒चुः । आ॒नृ॒चुस्तेन॑ । तेने॒यम् । इ॒यꣳ स॑र्परा॒ज्ञी । स॒र्प॒रा॒ज्ञी ते । स॒र्प॒रा॒ज्ञीति॑ सर्प - रा॒ज्ञी । ते यत् । यदे॒व । ए॒व किम् । किम् च॑ । च॒ वा॒चा । वा॒चाऽनृ॒चुः । अ॒नृ॒चुर् यत् । यद॒तोधि॑ । अ॒तोद्ध्य॑र्चि॒तारः॑ । अ॒र्चि॒तार॒स्तत् \newline

\textbf{Jatai Paata} \newline

1. स्यु स्ते ते स्युः स्यु स्ते । \newline
2. ते वि वि ते ते वि । \newline
3. वि ब्रू॑युर् ब्रूयु॒र् वि वि ब्रू॑युः । \newline
4. ब्रू॒यु॒र् यदि॒ यदि॑ ब्रूयुर् ब्रूयु॒र् यदि॑ । \newline
5. यदि॒ तत्र॒ तत्र॒ यदि॒ यदि॒ तत्र॑ । \newline
6. तत्र॒ न न तत्र॒ तत्र॒ न । \newline
7. न वि॒न्देयु॑र् वि॒न्देयु॒र् न न वि॒न्देयुः॑ । \newline
8. वि॒न्देयु॑ रन्तस्सद॒सा द॑न्तस्सद॒साद् वि॒न्देयु॑र् वि॒न्देयु॑ रन्तस्सद॒सात् । \newline
9. अ॒न्त॒स्स॒द॒साद् व्युच्यं॒ ॅव्युच्य॑ मन्तस्सद॒सा द॑न्तस्सद॒साद् व्युच्य᳚म् । \newline
10. अ॒न्त॒स्स॒द॒सादित्य॑न्तः - स॒द॒सात् । \newline
11. व्युच्यं॒ ॅयदि॒ यदि॒ व्युच्यं॒ ॅव्युच्यं॒ ॅयदि॑ । \newline
12. व्युच्य॒मिति॑ वि - उच्य᳚म् । \newline
13. यदि॒ तत्र॒ तत्र॒ यदि॒ यदि॒ तत्र॑ । \newline
14. तत्र॒ न न तत्र॒ तत्र॒ न । \newline
15. न वि॒न्देयु॑र् वि॒न्देयु॒र् न न वि॒न्देयुः॑ । \newline
16. वि॒न्देयु॑र् गृ॒हप॑तिना गृ॒हप॑तिना वि॒न्देयु॑र् वि॒न्देयु॑र् गृ॒हप॑तिना । \newline
17. गृ॒हप॑तिना॒ व्युच्यं॒ ॅव्युच्य॑म् गृ॒हप॑तिना गृ॒हप॑तिना॒ व्युच्य᳚म् । \newline
18. गृ॒हप॑ति॒नेति॑ गृ॒ह - प॒ति॒ना॒ । \newline
19. व्युच्य॒म् तत् तद् व्युच्यं॒ ॅव्युच्य॒म् तत् । \newline
20. व्युच्य॒मिति॑ वि - उच्य᳚म् । \newline
21. तद् व्युच्यं॒ ॅव्युच्य॒म् तत् तद् व्युच्य᳚म् । \newline
22. व्युच्य॑ मे॒वैव व्युच्यं॒ ॅव्युच्य॑ मे॒व । \newline
23. व्युच्य॒मिति॑ वि - उच्य᳚म् । \newline
24. ए॒वाथा थै॒वैवाथ॑ । \newline
25. अथ॒ वै वा अथाथ॒ वै । \newline
26. वा ए॒त दे॒तद् वै वा ए॒तत् । \newline
27. ए॒तथ् स॑र्परा॒ज्ञियाः᳚ सर्परा॒ज्ञिया॑ ए॒त दे॒तथ् स॑र्परा॒ज्ञियाः᳚ । \newline
28. स॒र्प॒रा॒ज्ञिया॑ ऋ॒ग्भिर्. ऋ॒ग्भिः स॑र्परा॒ज्ञियाः᳚ सर्परा॒ज्ञिया॑ ऋ॒ग्भिः । \newline
29. स॒र्प॒रा॒ज्ञिया॒ इति॑ सर्प - रा॒ज्ञियाः᳚ । \newline
30. ऋ॒ग्भिः स्तु॑वन्ति स्तुवन् त्यृ॒ग्भिर्. ऋ॒ग्भिः स्तु॑वन्ति । \newline
31. ऋ॒ग्भिरित्यृ॑क् - भिः । \newline
32. स्तु॒व॒न्ती॒य मि॒यꣳ स्तु॑वन्ति स्तुवन्ती॒यम् । \newline
33. इ॒यं ॅवै वा इ॒य मि॒यं ॅवै । \newline
34. वै सर्प॑तः॒ सर्प॑तो॒ वै वै सर्प॑तः । \newline
35. सर्प॑तो॒ राज्ञी॒ राज्ञी॒ सर्प॑तः॒ सर्प॑तो॒ राज्ञी᳚ । \newline
36. राज्ञी॒ यद् यद् राज्ञी॒ राज्ञी॒ यत् । \newline
37. यद् वै वै यद् यद् वै । \newline
38. वा अ॒स्या म॒स्यां ॅवै वा अ॒स्याम् । \newline
39. अ॒स्याम् किम् कि म॒स्या म॒स्याम् किम् । \newline
40. किम् च॑ च॒ किम् किम् च॑ । \newline
41. चार्च॒ न्त्यर्च॑न्ति च॒ चार्च॑न्ति । \newline
42. अर्च॑न्ति॒ यद् य दर्च॒ न्त्यर्च॑न्ति॒ यत् । \newline
43. य दा॑नृ॒चु रा॑नृ॒चुर् यद् य दा॑नृ॒चुः । \newline
44. आ॒नृ॒चु स्तेन॒ तेना॑ नृ॒चुरा॑ नृ॒चु स्तेन॑ । \newline
45. तेने॒य मि॒यम् तेन॒ तेने॒यम् । \newline
46. इ॒यꣳ स॑र्परा॒ज्ञी स॑र्परा॒ज्ञीय मि॒यꣳ स॑र्परा॒ज्ञी । \newline
47. स॒र्प॒रा॒ज्ञी ते ते स॑र्परा॒ज्ञी स॑र्परा॒ज्ञी ते । \newline
48. स॒र्प॒रा॒ज्ञीति॑ सर्प - रा॒ज्ञी । \newline
49. ते यद् यत् ते ते यत् । \newline
50. यदे॒वैव यद् यदे॒व । \newline
51. ए॒व किम् कि मे॒वैव किम् । \newline
52. किम् च॑ च॒ किम् किम् च॑ । \newline
53. च॒ वा॒चा वा॒चा च॑ च वा॒चा । \newline
54. वा॒चा ऽऽनृ॒चु रा॑नृ॒चुर् वा॒चा वा॒चा ऽऽनृ॒चुः । \newline
55. आ॒नृ॒चुर् यद् यदा॑नृ॒चु रा॑नृ॒चुर् यत् । \newline
56. यद॒ तोध्य॒ तोधि॒ यद् यद॒तोधि॑ । \newline
57. अ॒तो ध्य॑र्चि॒तारो᳚ ऽर्चि॒तारो॒ ऽतोध्य॒ तोध्य॑र्चि॒तारः॑ । \newline
58. अ॒र्चि॒तार॒ स्तत् तद॑र्चि॒तारो᳚ ऽर्चि॒तार॒ स्तत् । \newline

\textbf{Ghana Paata } \newline

1. स्यु स्ते ते स्युः स्यु स्ते वि वि ते स्युः स्यु स्ते वि । \newline
2. ते वि वि ते ते वि ब्रू॑युर् ब्रूयु॒र् वि ते ते वि ब्रू॑युः । \newline
3. वि ब्रू॑युर् ब्रूयु॒र् वि वि ब्रू॑यु॒र् यदि॒ यदि॑ ब्रूयु॒र् वि वि ब्रू॑यु॒र् यदि॑ । \newline
4. ब्रू॒यु॒र् यदि॒ यदि॑ ब्रूयुर् ब्रूयु॒र् यदि॒ तत्र॒ तत्र॒ यदि॑ ब्रूयुर् ब्रूयु॒र् यदि॒ तत्र॑ । \newline
5. यदि॒ तत्र॒ तत्र॒ यदि॒ यदि॒ तत्र॒ न न तत्र॒ यदि॒ यदि॒ तत्र॒ न । \newline
6. तत्र॒ न न तत्र॒ तत्र॒ न वि॒न्देयु॑र् वि॒न्देयु॒र् न तत्र॒ तत्र॒ न वि॒न्देयुः॑ । \newline
7. न वि॒न्देयु॑र् वि॒न्देयु॒र् न न वि॒न्देयु॑ रन्तस्सद॒सा द॑न्तस्सद॒साद् वि॒न्देयु॒र् न न वि॒न्देयु॑ रन्तस्सद॒सात् । \newline
8. वि॒न्देयु॑ रन्तस्सद॒सा द॑न्तस्सद॒साद् वि॒न्देयु॑र् वि॒न्देयु॑ रन्तस्सद॒साद् व्युच्यं॒ ॅव्युच्य॑ मन्तस्सद॒साद् वि॒न्देयु॑र् वि॒न्देयु॑ रन्तस्सद॒साद् व्युच्य᳚म् । \newline
9. अ॒न्त॒स्स॒द॒साद् व्युच्यं॒ ॅव्युच्य॑ मन्तस्सद॒सा द॑न्तस्सद॒साद् व्युच्यं॒ ॅयदि॒ यदि॒ व्युच्य॑ मन्तस्सद॒सा द॑न्तस्सद॒साद् व्युच्यं॒ ॅयदि॑ । \newline
10. अ॒न्त॒स्स॒द॒सादित्य॑न्तः - स॒द॒सात् । \newline
11. व्युच्यं॒ ॅयदि॒ यदि॒ व्युच्यं॒ ॅव्युच्यं॒ ॅयदि॒ तत्र॒ तत्र॒ यदि॒ व्युच्यं॒ ॅव्युच्यं॒ ॅयदि॒ तत्र॑ । \newline
12. व्युच्य॒मिति॑ वि - उच्य᳚म् । \newline
13. यदि॒ तत्र॒ तत्र॒ यदि॒ यदि॒ तत्र॒ न न तत्र॒ यदि॒ यदि॒ तत्र॒ न । \newline
14. तत्र॒ न न तत्र॒ तत्र॒ न वि॒न्देयु॑र् वि॒न्देयु॒र् न तत्र॒ तत्र॒ न वि॒न्देयुः॑ । \newline
15. न वि॒न्देयु॑र् वि॒न्देयु॒र् न न वि॒न्देयु॑र् गृ॒हप॑तिना गृ॒हप॑तिना वि॒न्देयु॒र् न न वि॒न्देयु॑र् गृ॒हप॑तिना । \newline
16. वि॒न्देयु॑र् गृ॒हप॑तिना गृ॒हप॑तिना वि॒न्देयु॑र् वि॒न्देयु॑र् गृ॒हप॑तिना॒ व्युच्यं॒ ॅव्युच्य॑म् गृ॒हप॑तिना वि॒न्देयु॑र् वि॒न्देयु॑र् गृ॒हप॑तिना॒ व्युच्य᳚म् । \newline
17. गृ॒हप॑तिना॒ व्युच्यं॒ ॅव्युच्य॑म् गृ॒हप॑तिना गृ॒हप॑तिना॒ व्युच्य॒म् तत् तद् व्युच्य॑म् गृ॒हप॑तिना गृ॒हप॑तिना॒ व्युच्य॒म् तत् । \newline
18. गृ॒हप॑ति॒नेति॑ गृ॒ह - प॒ति॒ना॒ । \newline
19. व्युच्य॒म् तत् तद् व्युच्यं॒ ॅव्युच्य॒म् तद् व्युच्यं॒ ॅव्युच्य॒म् तद् व्युच्यं॒ ॅव्युच्य॒म् तद् व्युच्य᳚म् । \newline
20. व्युच्य॒मिति॑ वि - उच्य᳚म् । \newline
21. तद् व्युच्यं॒ ॅव्युच्य॒म् तत् तद् व्युच्य॑ मे॒वैव व्युच्य॒म् तत् तद् व्युच्य॑ मे॒व । \newline
22. व्युच्य॑ मे॒वैव व्युच्यं॒ ॅव्युच्य॑ मे॒वाथा थै॒व व्युच्यं॒ ॅव्युच्य॑ मे॒वाथ॑ । \newline
23. व्युच्य॒मिति॑ वि - उच्य᳚म् । \newline
24. ए॒वाथा थै॒वै वाथ॒ वै वा अथै॒वै वाथ॒ वै । \newline
25. अथ॒ वै वा अथाथ॒ वा ए॒त दे॒तद् वा अथाथ॒ वा ए॒तत् । \newline
26. वा ए॒त दे॒तद् वै वा ए॒तथ् स॑र्परा॒ज्ञियाः᳚ सर्परा॒ज्ञिया॑ ए॒तद् वै वा ए॒तथ् स॑र्परा॒ज्ञियाः᳚ । \newline
27. ए॒तथ् स॑र्परा॒ज्ञियाः᳚ सर्परा॒ज्ञिया॑ ए॒त दे॒तथ् स॑र्परा॒ज्ञिया॑ ऋ॒ग्भिर्. ऋ॒ग्भिः स॑र्परा॒ज्ञिया॑ ए॒त दे॒तथ् स॑र्परा॒ज्ञिया॑ ऋ॒ग्भिः । \newline
28. स॒र्प॒रा॒ज्ञिया॑ ऋ॒ग्भिर्. ऋ॒ग्भिः स॑र्परा॒ज्ञियाः᳚ सर्परा॒ज्ञिया॑ ऋ॒ग्भिः स्तु॑वन्ति स्तुव न्त्यृ॒ग्भिः स॑र्परा॒ज्ञियाः᳚ सर्परा॒ज्ञिया॑ ऋ॒ग्भिः स्तु॑वन्ति । \newline
29. स॒र्प॒रा॒ज्ञिया॒ इति॑ सर्प - रा॒ज्ञियाः᳚ । \newline
30. ऋ॒ग्भिः स्तु॑वन्ति स्तुव न्त्यृ॒ग्भिर्. ऋ॒ग्भिः स्तु॑वन्ती॒य मि॒यꣳ स्तु॑व न्त्यृ॒ग्भिर्. ऋ॒ग्भिः स्तु॑वन्ती॒यम् । \newline
31. ऋ॒ग्भिरित्यृ॑क् - भिः । \newline
32. स्तु॒व॒ न्ती॒य मि॒यꣳ स्तु॑वन्ति स्तुव न्ती॒यं ॅवै वा इ॒यꣳ स्तु॑वन्ति स्तुव न्ती॒यं ॅवै । \newline
33. इ॒यं ॅवै वा इ॒य मि॒यं ॅवै सर्प॑तः॒ सर्प॑तो॒ वा इ॒य मि॒यं ॅवै सर्प॑तः । \newline
34. वै सर्प॑तः॒ सर्प॑तो॒ वै वै सर्प॑तो॒ राज्ञी॒ राज्ञी॒ सर्प॑तो॒ वै वै सर्प॑तो॒ राज्ञी᳚ । \newline
35. सर्प॑तो॒ राज्ञी॒ राज्ञी॒ सर्प॑तः॒ सर्प॑तो॒ राज्ञी॒ यद् यद् राज्ञी॒ सर्प॑तः॒ सर्प॑तो॒ राज्ञी॒ यत् । \newline
36. राज्ञी॒ यद् यद् राज्ञी॒ राज्ञी॒ यद् वै वै यद् राज्ञी॒ राज्ञी॒ यद् वै । \newline
37. यद् वै वै यद् यद् वा अ॒स्या म॒स्यां ॅवै यद् यद् वा अ॒स्याम् । \newline
38. वा अ॒स्या म॒स्यां ॅवै वा अ॒स्याम् किम् कि म॒स्यां ॅवै वा अ॒स्याम् किम् । \newline
39. अ॒स्याम् किम् कि म॒स्या म॒स्याम् किम् च॑ च॒ कि म॒स्या म॒स्याम् किम् च॑ । \newline
40. किम् च॑ च॒ किम् किम् चार्च॒ न्त्यर्च॑न्ति च॒ किम् किम् चार्च॑न्ति । \newline
41. चार्च॒ न्त्यर्च॑न्ति च॒ चार्च॑न्ति॒ यद् यदर्च॑न्ति च॒ चार्च॑न्ति॒ यत् । \newline
42. अर्च॑न्ति॒ यद् यदर्च॒ न्त्यर्च॑न्ति॒ यदा॑नृ॒चु रा॑नृ॒चुर् यदर्च॒ न्त्यर्च॑न्ति॒ यदा॑नृ॒चुः । \newline
43. यदा॑नृ॒चु रा॑नृ॒चुर् यद् यदा॑नृ॒चु स्तेन॒ तेना॑ नृ॒चुर् यद् यदा॑नृ॒चु स्तेन॑ । \newline
44. आ॒नृ॒चु स्तेन॒ तेना॑ नृ॒चु रा॑नृ॒चु स्तेने॒य मि॒यम् तेना॑ नृ॒चु रा॑नृ॒चु स्तेने॒यम् । \newline
45. तेने॒य मि॒यम् तेन॒ तेने॒यꣳ स॑र्परा॒ज्ञी स॑र्परा॒ज्ञीयम् तेन॒ तेने॒यꣳ स॑र्परा॒ज्ञी । \newline
46. इ॒यꣳ स॑र्परा॒ज्ञी स॑र्परा॒ज्ञीय मि॒यꣳ स॑र्परा॒ज्ञी ते ते स॑र्परा॒ज्ञीय मि॒यꣳ स॑र्परा॒ज्ञी ते । \newline
47. स॒र्प॒रा॒ज्ञी ते ते स॑र्परा॒ज्ञी स॑र्परा॒ज्ञी ते यद् यत् ते स॑र्परा॒ज्ञी स॑र्परा॒ज्ञी ते यत् । \newline
48. स॒र्प॒रा॒ज्ञीति॑ सर्प - रा॒ज्ञी । \newline
49. ते यद् यत् ते ते यदे॒वैव यत् ते ते यदे॒व । \newline
50. यदे॒वैव यद् यदे॒व किम् कि मे॒व यद् यदे॒व किम् । \newline
51. ए॒व किम् कि मे॒वैव किम् च॑ च॒ कि मे॒वैव किम् च॑ । \newline
52. किम् च॑ च॒ किम् किम् च॑ वा॒चा वा॒चा च॒ किम् किम् च॑ वा॒चा । \newline
53. च॒ वा॒चा वा॒चा च॑ च वा॒चा ऽऽनृ॒चु रा॑नृ॒चुर् वा॒चा च॑ च वा॒चा ऽऽनृ॒चुः । \newline
54. वा॒चा ऽऽनृ॒चु रा॑नृ॒चुर् वा॒चा वा॒चा ऽऽनृ॒चुर् यद् यदा॑नृ॒चुर् वा॒चा वा॒चा ऽऽनृ॒चुर् यत् । \newline
55. आ॒नृ॒चुर् यद् यदा॑नृ॒चु रा॑नृ॒चुर् यद॒तोध्य॒ तोधि॒ यदा॑नृ॒चु रा॑नृ॒चुर् यद॒तोधि॑ । \newline
56. यद॒तो ध्य॒तोधि॒ यद् यद॒तो ध्य॑र्चि॒तारो᳚ ऽर्चि॒तारो॒ ऽतोधि॒ यद् यद॒तोध्य॑र्चि॒तारः॑ । \newline
57. अ॒तोध्य॑ र्चि॒तारो᳚ ऽर्चि॒तारो॒ ऽतोध्य॒तो ध्य॑र्चि॒तार॒ स्तत् तद॑र्चि॒तारो॒ ऽतोध्य॒तो ध्य॑र्चि॒तार॒ स्तत् । \newline
58. अ॒र्चि॒तार॒ स्तत् तद॑र्चि॒तारो᳚ ऽर्चि॒तार॒ स्तदु॒भय॑ मु॒भय॒म् तद॑र्चि॒तारो᳚ ऽर्चि॒तार॒ स्तदु॒भय᳚म् । \newline
\pagebreak
\markright{ TS 7.3.1.4  \hfill https://www.vedavms.in \hfill}

\section{ TS 7.3.1.4 }

\textbf{TS 7.3.1.4 } \newline
\textbf{Samhita Paata} \newline

-स्तदु॒भय॑मा॒प्त्वा ऽव॒रुद्ध्योत् ति॑ष्ठा॒मेति॒ ताभि॒र्मन॑सा स्तुवते॒ न वा इ॒माम॑श्वर॒थो नाऽश्व॑तरीर॒थः स॒द्यः पर्या᳚प्तुमर्.हति॒ मनो॒ वा इ॒माꣳ स॒द्यः पर्या᳚प्तुमर्.हति॒ मनः॒ परि॑भवितु॒मथ॒ ब्रह्म॑ वदन्ति॒ परि॑मिता॒ वा ऋचः॒ परि॑मितानि॒ सामा॑नि॒ परि॑मितानि॒ यजूꣳ॒॒ष्यथै॒तस्यै॒वान्तो॒ नास्ति॒ यद्-ब्रह्म॒ तत् प्र॑तिगृण॒त आ च॑क्षीत॒ स ( ) प्र॑तिग॒रः ॥ \newline

\textbf{Pada Paata} \newline

तत् । उ॒भय᳚म् । आ॒प्त्वा । अ॒व॒रुद्ध्येत्य॑व - रुद्ध्य॑ । उदिति॑ । ति॒ष्ठा॒म॒ । इति॑ । ताभिः॑ । मन॑सा । स्तु॒व॒ते॒ । न । वै । इ॒माम् । अ॒श्व॒र॒थ इत्य॑श्व - र॒थः । न । अ॒श्व॒त॒री॒र॒थ इत्यश्व॑तरी - र॒थः । स॒द्यः । पर्या᳚प्तु॒मिति॒ परि॑-आ॒प्तु॒म् । अ॒र्.॒ह॒ति॒ । मनः॑ । वै । इ॒माम् । स॒द्यः । पर्या᳚प्तु॒मिति॒ परि॑ - आ॒प्तु॒म् । अ॒र्.॒ह॒ति॒ । मनः॑ । परि॑भवितु॒मिति॒ परि॑-भ॒वि॒तु॒म् । अथ॑ । ब्रह्म॑ । व॒द॒न्ति॒ । परि॑मिता॒ इति॒ परि॑-मि॒ताः॒ । वै । ऋचः॑ । परि॑मिता॒नीति॒ परि॑ - मि॒ता॒नि॒ । सामा॑नि । परि॑मिता॒नीति॒ परि॑ - मि॒ता॒नि॒ । यजूꣳ॑षि । अथ॑ । ए॒तस्य॑ । ए॒व । अन्तः॑ । न । अ॒स्ति॒ । यत् । ब्रह्म॑ । तत् । प्र॒ति॒गृ॒ण॒त इति॑ प्रति - गृ॒ण॒ते । एति॑ । च॒क्षी॒त॒ । सः ( ) । प्र॒ति॒ग॒र इति॑ प्रति - ग॒रः ॥  \newline


\textbf{Krama Paata} \newline

तदु॒भय᳚म् । उ॒भय॑मा॒प्त्वा । आ॒प्त्वाऽव॒रुद्ध्य॑ । अ॒व॒रुद्ध्योत् । अ॒व॒रुद्ध्येत्य॑व - रुद्ध्य॑ । उत् ति॑ष्ठाम । ति॒ष्ठा॒मेति॑ । इति॒ ताभिः॑ । ताभि॒र् मन॑सा । मन॑सा स्तुवते । स्तु॒व॒ते॒ न । न वै । वा इ॒माम् । इ॒माम॑श्वर॒थः । अ॒श्व॒र॒थो न । अ॒श्व॒र॒थ इत्य॑श्व - र॒थः । नाश्व॑तरीर॒थः । अ॒श्व॒त॒री॒र॒थः स॒द्यः । अ॒श्व॒त॒री॒र॒थ इत्य॑श्वतरी - र॒थः । स॒द्यः पर्या᳚प्तुम् । पर्या᳚प्तुमर्.हति । पर्या᳚प्तु॒मिति॒ परि॑ - आ॒प्तु॒म् । अ॒र्.॒ह॒ति॒ मनः॑ । मनो॒ वै । वा इ॒माम् । इ॒माꣳ स॒द्यः । स॒द्यः पर्या᳚प्तुम् । पर्या᳚प्तुमर्.हति । पर्या᳚प्तुमिति॒ परि॑ - आ॒प्तु॒म् । अ॒र्.॒ह॒ति॒ मनः॑ । मनः॒ परि॑भवितुम् । परि॑भवितु॒मथ॑ । परि॑भवितु॒मिति॒ परि॑ - भ॒वि॒तु॒म् । अथ॒ ब्रह्म॑ । ब्रह्म॑ वदन्ति । व॒द॒न्ति॒ परि॑मिताः । परि॑मिता॒ वै । परि॑मिता॒ इति॒ परि॑ - मि॒ताः॒ । वा ऋचः॑ । ऋचः॒ परि॑मितानि । परि॑मितानि॒ सामा॑नि । परि॑मिता॒नीति॒ परि॑ - मि॒ता॒नि॒ । सामा॑नि॒ परि॑मितानि । परि॑मितानि॒ यजूꣳ॑षि । परि॑मिता॒नीति॒ परि॑ - मि॒ता॒नि॒ । यजूꣳ॒॒ष्यथ॑ । अथै॒तस्य॑ । ए॒तस्यै॒व । ए॒वान्तः॑ । अन्तो॒ न । नास्ति॑ । अ॒स्ति॒ यत् । यद् ब्रह्म॑ । ब्रह्म॒ तत् । तत् प्र॑तिगृण॒ते । प्र॒ति॒गृ॒ण॒त आ । प्र॒ति॒गृ॒ण॒त इति॑ प्रति - गृ॒ण॒ते । आ च॑क्षीत । च॒क्षी॒त॒ सः ( ) । स प्र॑तिग॒रः । प्र॒ति॒ग॒र इति॑ प्रति - ग॒रः । \newline

\textbf{Jatai Paata} \newline

1. तदु॒भय॑ मु॒भय॒म् तत् तदु॒भय᳚म् । \newline
2. उ॒भय॑ मा॒प्त्वा ऽऽप्त्वोभय॑ मु॒भय॑ मा॒प्त्वा । \newline
3. आ॒प्त्वा ऽव॒रुद्ध्या॑ व॒रुद्ध्या॒ प्त्वा ऽऽप्त्वा ऽव॒रुद्ध्य॑ । \newline
4. अ॒व॒रुद्ध्यो दुद॑ व॒रुद्ध्या॑ व॒रुद्ध्योत् । \newline
5. अ॒व॒रुद्ध्येत्य॑व - रुद्ध्य॑ । \newline
6. उत् ति॑ष्ठाम तिष्ठा॒मोदुत् ति॑ष्ठाम । \newline
7. ति॒ष्ठा॒मेतीति॑ तिष्ठाम तिष्ठा॒ मेति॑ । \newline
8. इति॒ ताभि॒ स्ता भि॒रितीति॒ ताभिः॑ । \newline
9. ताभि॒र् मन॑सा॒ मन॑सा॒ ताभि॒ स्ताभि॒र् मन॑सा । \newline
10. मन॑सा स्तुवते स्तुवते॒ मन॑सा॒ मन॑सा स्तुवते । \newline
11. स्तु॒व॒ते॒ न न स्तु॑वते स्तुवते॒ न । \newline
12. न वै वै न न वै । \newline
13. वा इ॒मा मि॒मां ॅवै वा इ॒माम् । \newline
14. इ॒मा म॑श्वर॒थो᳚ ऽश्वर॒थ इ॒मा मि॒मा म॑श्वर॒थः । \newline
15. अ॒श्व॒र॒थो न ना श्व॑र॒थो᳚ ऽश्वर॒थो न । \newline
16. अ॒श्व॒र॒थ इत्य॑श्व - र॒थः । \newline
17. ना श्व॑तरीर॒थो᳚ ऽश्वतरीर॒थो न ना श्व॑तरीर॒थः । \newline
18. अ॒श्व॒त॒री॒र॒थः स॒द्यः स॒द्यो᳚ ऽश्वतरीर॒थो᳚ ऽश्वतरीर॒थः स॒द्यः । \newline
19. अ॒श्व॒त॒री॒र॒थ इत्य॑श्वतरी - र॒थः । \newline
20. स॒द्यः पर्या᳚प्तु॒म् पर्या᳚प्तुꣳ स॒द्यः स॒द्यः पर्या᳚प्तुम् । \newline
21. पर्या᳚प्तु मर्.ह त्यर्.हति॒ पर्या᳚प्तु॒म् पर्या᳚प्तु मर्.हति । \newline
22. पर्या᳚प्तु॒मिति॒ परि॑ - आ॒प्तु॒म् । \newline
23. अ॒र्॒.ह॒ति॒ मनो॒ मनो॑ ऽर्.ह त्यर्.हति॒ मनः॑ । \newline
24. मनो॒ वै वै मनो॒ मनो॒ वै । \newline
25. वा इ॒मा मि॒मां ॅवै वा इ॒माम् । \newline
26. इ॒माꣳ स॒द्यः स॒द्य इ॒मा मि॒माꣳ स॒द्यः । \newline
27. स॒द्यः पर्या᳚प्तु॒म् पर्या᳚प्तुꣳ स॒द्यः स॒द्यः पर्या᳚प्तुम् । \newline
28. पर्या᳚प्तु मर्.ह त्यर्.हति॒ पर्या᳚प्तु॒म् पर्या᳚प्तु मर्.हति । \newline
29. पर्या᳚प्तु॒मिति॒ परि॑ - आ॒प्तु॒म् । \newline
30. अ॒र्॒.ह॒ति॒ मनो॒ मनो॑ ऽर्.ह त्यर्.हति॒ मनः॑ । \newline
31. मनः॒ परि॑भवितु॒म् परि॑भवितु॒म् मनो॒ मनः॒ परि॑भवितुम् । \newline
32. परि॑भवितु॒ मथाथ॒ परि॑भवितु॒म् परि॑भवितु॒ मथ॑ । \newline
33. परि॑भवितु॒मिति॒ परि॑ - भ॒वि॒तु॒म् । \newline
34. अथ॒ ब्रह्म॒ ब्रह्मा थाथ॒ ब्रह्म॑ । \newline
35. ब्रह्म॑ वदन्ति वदन्ति॒ ब्रह्म॒ ब्रह्म॑ वदन्ति । \newline
36. व॒द॒न्ति॒ परि॑मिताः॒ परि॑मिता वदन्ति वदन्ति॒ परि॑मिताः । \newline
37. परि॑मिता॒ वै वै परि॑मिताः॒ परि॑मिता॒ वै । \newline
38. परि॑मिता॒ इति॒ परि॑ - मि॒ताः॒ । \newline
39. वा ऋच॒ ऋचो॒ वै वा ऋचः॑ । \newline
40. ऋचः॒ परि॑मितानि॒ परि॑मिता॒ न्यृच॒ ऋचः॒ परि॑मितानि । \newline
41. परि॑मितानि॒ सामा॑नि॒ सामा॑नि॒ परि॑मितानि॒ परि॑मितानि॒ सामा॑नि । \newline
42. परि॑मिता॒नीति॒ परि॑ - मि॒ता॒नि॒ । \newline
43. सामा॑नि॒ परि॑मितानि॒ परि॑मितानि॒ सामा॑नि॒ सामा॑नि॒ परि॑मितानि । \newline
44. परि॑मितानि॒ यजूꣳ॑षि॒ यजूꣳ॑षि॒ परि॑मितानि॒ परि॑मितानि॒ यजूꣳ॑षि । \newline
45. परि॑मिता॒नीति॒ परि॑ - मि॒ता॒नि॒ । \newline
46. यजूꣳ॒॒ ष्यथाथ॒ यजूꣳ॑षि॒ यजूꣳ॒॒ ष्यथ॑ । \newline
47. अथै॒ तस्यै॒ तस्या थाथै॒तस्य॑ । \newline
48. ए॒तस्यै॒वैवै तस्यै॒ तस्यै॒व । \newline
49. ए॒वान्तो ऽन्त॑ ए॒वै वान्तः॑ । \newline
50. अन्तो॒ न नान्तो ऽन्तो॒ न । \newline
51. ना स्त्य॑स्ति॒ न नास्ति॑ । \newline
52. अ॒स्ति॒ यद् यद॑ स्त्यस्ति॒ यत् । \newline
53. यद् ब्रह्म॒ ब्रह्म॒ यद् यद् ब्रह्म॑ । \newline
54. ब्रह्म॒ तत् तद् ब्रह्म॒ ब्रह्म॒ तत् । \newline
55. तत् प्र॑तिगृण॒ते प्र॑तिगृण॒ते तत् तत् प्र॑तिगृण॒ते । \newline
56. प्र॒ति॒गृ॒ण॒त आ प्र॑तिगृण॒ते प्र॑तिगृण॒त आ । \newline
57. प्र॒ति॒गृ॒ण॒त इति॑ प्रति - गृ॒ण॒ते । \newline
58. आ च॑क्षीत चक्षी॒ता च॑क्षीत । \newline
59. च॒क्षी॒त॒ स स च॑क्षीत चक्षीत॒ सः । \newline
60. स प्र॑तिग॒रः प्र॑तिग॒रः स स प्र॑तिग॒रः । \newline
61. प्र॒ति॒ग॒र इति॑ प्रति - ग॒रः । \newline

\textbf{Ghana Paata } \newline

1. तदु॒भय॑ मु॒भय॒म् तत् तदु॒भय॑ मा॒प्त्वा ऽऽप्त्वोभय॒म् तत् तदु॒भय॑ मा॒प्त्वा । \newline
2. उ॒भय॑ मा॒प्त्वा ऽऽप्त्वोभय॑ मु॒भय॑ मा॒प्त्वा ऽव॒रुद्ध्या॑ व॒रुद्ध्या॒ प्त्वोभय॑ मु॒भय॑ मा॒प्त्वा ऽव॒रुद्ध्य॑ । \newline
3. आ॒प्त्वा ऽव॒रुद्ध्या॑ व॒रुद्ध्या॒प्त्वा ऽऽप्त्वा ऽव॒रुद्ध्यो दुद॑व॒रुद्ध्या॒ प्त्वा ऽऽप्त्वा ऽव॒रुद्ध्योत् । \newline
4. अ॒व॒रुद्ध्यो दुद॑व॒रुद्ध्या॑ व॒रुद्ध्योत् ति॑ष्ठाम तिष्ठा॒मो द॑व॒रुद्ध्या॑ व॒रुद्ध्योत् ति॑ष्ठाम । \newline
5. अ॒व॒रुद्ध्येत्य॑व - रुद्ध्य॑ । \newline
6. उत् ति॑ष्ठाम तिष्ठा॒मोदुत् ति॑ष्ठा॒मेतीति॑ तिष्ठा॒मोदुत् ति॑ष्ठा॒मेति॑ । \newline
7. ति॒ष्ठा॒मेतीति॑ तिष्ठाम तिष्ठा॒मेति॒ ताभि॒ स्ताभि॒ रिति॑ तिष्ठाम तिष्ठा॒मेति॒ ताभिः॑ । \newline
8. इति॒ ताभि॒ स्ताभि॒ रितीति॒ ताभि॒र् मन॑सा॒ मन॑सा॒ ताभि॒ रितीति॒ ताभि॒र् मन॑सा । \newline
9. ताभि॒र् मन॑सा॒ मन॑सा॒ ताभि॒ स्ताभि॒र् मन॑सा स्तुवते स्तुवते॒ मन॑सा॒ ताभि॒ स्ताभि॒र् मन॑सा स्तुवते । \newline
10. मन॑सा स्तुवते स्तुवते॒ मन॑सा॒ मन॑सा स्तुवते॒ न न स्तु॑वते॒ मन॑सा॒ मन॑सा स्तुवते॒ न । \newline
11. स्तु॒व॒ते॒ न न स्तु॑वते स्तुवते॒ न वै वै न स्तु॑वते स्तुवते॒ न वै । \newline
12. न वै वै न न वा इ॒मा मि॒मां ॅवै न न वा इ॒माम् । \newline
13. वा इ॒मा मि॒मां ॅवै वा इ॒मा म॑श्वर॒थो᳚ ऽश्वर॒थ इ॒मां ॅवै वा इ॒मा म॑श्वर॒थः । \newline
14. इ॒मा म॑श्वर॒थो᳚ ऽश्वर॒थ इ॒मा मि॒मा म॑श्वर॒थो न नाश्व॑र॒थ इ॒मा मि॒मा म॑श्वर॒थो न । \newline
15. अ॒श्व॒र॒थो न नाश्व॑र॒थो᳚ ऽश्वर॒थो नाश्व॑तरीर॒थो᳚ ऽश्वतरीर॒थो नाश्व॑र॒थो᳚ ऽश्वर॒थो नाश्व॑तरीर॒थः । \newline
16. अ॒श्व॒र॒थ इत्य॑श्व - र॒थः । \newline
17. नाश्व॑तरीर॒थो᳚ ऽश्वतरीर॒थो न नाश्व॑तरीर॒थः स॒द्यः स॒द्यो᳚ ऽश्वतरीर॒थो न नाश्व॑तरीर॒थः स॒द्यः । \newline
18. अ॒श्व॒त॒री॒र॒थः स॒द्यः स॒द्यो᳚ ऽश्वतरीर॒थो᳚ ऽश्वतरीर॒थः स॒द्यः पर्या᳚प्तु॒म् पर्या᳚प्तुꣳ स॒द्यो᳚ ऽश्वतरीर॒थो᳚ ऽश्वतरीर॒थः स॒द्यः पर्या᳚प्तुम् । \newline
19. अ॒श्व॒त॒री॒र॒थ इत्य॑श्वतरी - र॒थः । \newline
20. स॒द्यः पर्या᳚प्तु॒म् पर्या᳚प्तुꣳ स॒द्यः स॒द्यः पर्या᳚प्तु मर्.ह त्यर्.हति॒ पर्या᳚प्तुꣳ स॒द्यः स॒द्यः पर्या᳚प्तु मर्.हति । \newline
21. पर्या᳚प्तु मर्.ह त्यर्.हति॒ पर्या᳚प्तु॒म् पर्या᳚प्तु मर्.हति॒ मनो॒ मनो॑ ऽर्.हति॒ पर्या᳚प्तु॒म् पर्या᳚प्तु मर्.हति॒ मनः॑ । \newline
22. पर्या᳚प्तु॒मिति॒ परि॑ - आ॒प्तु॒म् । \newline
23. अ॒र्॒.ह॒ति॒ मनो॒ मनो॑ ऽर्.ह त्यर्.हति॒ मनो॒ वै वै मनो॑ ऽर्.ह त्यर्.हति॒ मनो॒ वै । \newline
24. मनो॒ वै वै मनो॒ मनो॒ वा इ॒मा मि॒मां ॅवै मनो॒ मनो॒ वा इ॒माम् । \newline
25. वा इ॒मा मि॒मां ॅवै वा इ॒माꣳ स॒द्यः स॒द्य इ॒मां ॅवै वा इ॒माꣳ स॒द्यः । \newline
26. इ॒माꣳ स॒द्यः स॒द्य इ॒मा मि॒माꣳ स॒द्यः पर्या᳚प्तु॒म् पर्या᳚प्तुꣳ स॒द्य इ॒मा मि॒माꣳ स॒द्यः पर्या᳚प्तुम् । \newline
27. स॒द्यः पर्या᳚प्तु॒म् पर्या᳚प्तुꣳ स॒द्यः स॒द्यः पर्या᳚प्तु मर्.ह त्यर्.हति॒ पर्या᳚प्तुꣳ स॒द्यः स॒द्यः पर्या᳚प्तु मर्.हति । \newline
28. पर्या᳚प्तु मर्.ह त्यर्.हति॒ पर्या᳚प्तु॒म् पर्या᳚प्तु मर्.हति॒ मनो॒ मनो॑ ऽर्.हति॒ पर्या᳚प्तु॒म् पर्या᳚प्तु मर्.हति॒ मनः॑ । \newline
29. पर्या᳚प्तु॒मिति॒ परि॑ - आ॒प्तु॒म् । \newline
30. अ॒र्॒.ह॒ति॒ मनो॒ मनो॑ ऽर्.ह त्यर्.हति॒ मनः॒ परि॑भवितु॒म् परि॑भवितु॒म् मनो॑ ऽर्.ह त्यर्.हति॒ मनः॒ परि॑भवितुम् । \newline
31. मनः॒ परि॑भवितु॒म् परि॑भवितु॒म् मनो॒ मनः॒ परि॑भवितु॒ मथाथ॒ परि॑भवितु॒म् मनो॒ मनः॒ परि॑भवितु॒ मथ॑ । \newline
32. परि॑भवितु॒ मथाथ॒ परि॑भवितु॒म् परि॑भवितु॒ मथ॒ ब्रह्म॒ ब्रह्माथ॒ परि॑भवितु॒म् परि॑भवितु॒ मथ॒ ब्रह्म॑ । \newline
33. परि॑भवितु॒मिति॒ परि॑ - भ॒वि॒तु॒म् । \newline
34. अथ॒ ब्रह्म॒ ब्रह्मा थाथ॒ ब्रह्म॑ वदन्ति वदन्ति॒ ब्रह्मा थाथ॒ ब्रह्म॑ वदन्ति । \newline
35. ब्रह्म॑ वदन्ति वदन्ति॒ ब्रह्म॒ ब्रह्म॑ वदन्ति॒ परि॑मिताः॒ परि॑मिता वदन्ति॒ ब्रह्म॒ ब्रह्म॑ वदन्ति॒ परि॑मिताः । \newline
36. व॒द॒न्ति॒ परि॑मिताः॒ परि॑मिता वदन्ति वदन्ति॒ परि॑मिता॒ वै वै परि॑मिता वदन्ति वदन्ति॒ परि॑मिता॒ वै । \newline
37. परि॑मिता॒ वै वै परि॑मिताः॒ परि॑मिता॒ वा ऋच॒ ऋचो॒ वै परि॑मिताः॒ परि॑मिता॒ वा ऋचः॑ । \newline
38. परि॑मिता॒ इति॒ परि॑ - मि॒ताः॒ । \newline
39. वा ऋच॒ ऋचो॒ वै वा ऋचः॒ परि॑मितानि॒ परि॑मिता॒ न्यृचो॒ वै वा ऋचः॒ परि॑मितानि । \newline
40. ऋचः॒ परि॑मितानि॒ परि॑मिता॒ न्यृच॒ ऋचः॒ परि॑मितानि॒ सामा॑नि॒ सामा॑नि॒ परि॑मिता॒ न्यृच॒ ऋचः॒ परि॑मितानि॒ सामा॑नि । \newline
41. परि॑मितानि॒ सामा॑नि॒ सामा॑नि॒ परि॑मितानि॒ परि॑मितानि॒ सामा॑नि॒ परि॑मितानि॒ परि॑मितानि॒ सामा॑नि॒ परि॑मितानि॒ परि॑मितानि॒ सामा॑नि॒ परि॑मितानि । \newline
42. परि॑मिता॒नीति॒ परि॑ - मि॒ता॒नि॒ । \newline
43. सामा॑नि॒ परि॑मितानि॒ परि॑मितानि॒ सामा॑नि॒ सामा॑नि॒ परि॑मितानि॒ यजूꣳ॑षि॒ यजूꣳ॑षि॒ परि॑मितानि॒ सामा॑नि॒ सामा॑नि॒ परि॑मितानि॒ यजूꣳ॑षि । \newline
44. परि॑मितानि॒ यजूꣳ॑षि॒ यजूꣳ॑षि॒ परि॑मितानि॒ परि॑मितानि॒ यजूꣳ॒॒ ष्यथाथ॒ यजूꣳ॑षि॒ परि॑मितानि॒ परि॑मितानि॒ यजूꣳ॒॒ ष्यथ॑ । \newline
45. परि॑मिता॒नीति॒ परि॑ - मि॒ता॒नि॒ । \newline
46. यजूꣳ॒॒ ष्यथाथ॒ यजूꣳ॑षि॒ यजूꣳ॒॒ ष्यथै॒त स्यै॒तस्याथ॒ यजूꣳ॑षि॒ यजूꣳ॒॒ ष्यथै॒तस्य॑ । \newline
47. अथै॒ तस्यै॒त स्याथा थै॒त स्यै॒वैवै तस्याथा थै॒तस्यै॒व । \newline
48. ए॒तस्यै॒ वैवैत स्यै॒त स्यै॒वान्तो ऽन्त॑ ए॒वै तस्यै॒त स्यै॒वान्तः॑ । \newline
49. ए॒वान्तो ऽन्त॑ ए॒वैवान्तो॒ न नान्त॑ ए॒वै वान्तो॒ न । \newline
50. अन्तो॒ न नान्तो ऽन्तो॒ नास्त्य॑स्ति॒ नान्तो ऽन्तो॒ नास्ति॑ । \newline
51. नास्त्य॑स्ति॒ न नास्ति॒ यद् यद॑स्ति॒ न नास्ति॒ यत् । \newline
52. अ॒स्ति॒ यद् यद॑ स्त्यस्ति॒ यद् ब्रह्म॒ ब्रह्म॒ यद॑ स्त्यस्ति॒ यद् ब्रह्म॑ । \newline
53. यद् ब्रह्म॒ ब्रह्म॒ यद् यद् ब्रह्म॒ तत् तद् ब्रह्म॒ यद् यद् ब्रह्म॒ तत् । \newline
54. ब्रह्म॒ तत् तद् ब्रह्म॒ ब्रह्म॒ तत् प्र॑तिगृण॒ते प्र॑तिगृण॒ते तद् ब्रह्म॒ ब्रह्म॒ तत् प्र॑तिगृण॒ते । \newline
55. तत् प्र॑तिगृण॒ते प्र॑तिगृण॒ते तत् तत् प्र॑तिगृण॒त आ प्र॑तिगृण॒ते तत् तत् प्र॑तिगृण॒त आ । \newline
56. प्र॒ति॒गृ॒ण॒त आ प्र॑तिगृण॒ते प्र॑तिगृण॒त आ च॑क्षीत चक्षी॒ता प्र॑तिगृण॒ते प्र॑तिगृण॒त आ च॑क्षीत । \newline
57. प्र॒ति॒गृ॒ण॒त इति॑ प्रति - गृ॒ण॒ते । \newline
58. आ च॑क्षीत चक्षी॒ता च॑क्षीत॒ स स च॑क्षी॒ता च॑क्षीत॒ सः । \newline
59. च॒क्षी॒त॒ स स च॑क्षीत चक्षीत॒ स प्र॑तिग॒रः प्र॑तिग॒रः स च॑क्षीत चक्षीत॒ स प्र॑तिग॒रः । \newline
60. स प्र॑तिग॒रः प्र॑तिग॒रः स स प्र॑तिग॒रः । \newline
61. प्र॒ति॒ग॒र इति॑ प्रति - ग॒रः । \newline
\pagebreak
\markright{ TS 7.3.2.1  \hfill https://www.vedavms.in \hfill}

\section{ TS 7.3.2.1 }

\textbf{TS 7.3.2.1 } \newline
\textbf{Samhita Paata} \newline

ब्र॒ह्म॒वा॒दिनो॑ वदन्ति॒ किं द्वा॑दशा॒हस्य॑ प्रथ॒मेनाह्न॒र्त्विजां॒ ॅयज॑मानो वृङ्क्त॒ इति॒ तेज॑ इन्द्रि॒यमिति॒ किं द्वि॒तीये॒नेति॑ प्रा॒णान॒न्नाद्य॒मिति॒ किं तृ॒तीये॒नेति॒ त्रीनि॒मान् ॅलो॒कानिति॒ किं च॑तु॒र्त्थेनेति॒ चतु॑ष्पदः प॒शूनिति॒ किं प॑ञ्च॒मेनेति॒ पञ्चा᳚क्षरां प॒ङ्क्तिमिति॒ किꣳ ष॒ष्ठेनेति॒ षड् ऋ॒तूनिति॒ किꣳ स॑प्त॒मेनेति॑ स॒प्तप॑दाꣳ॒॒ शक्व॑री॒मिति॒ - [  ] \newline

\textbf{Pada Paata} \newline

ब्र॒ह्म॒वा॒दिन॒ इति॑ ब्रह्म - वा॒दिनः॑ । व॒द॒न्ति॒ । किम् । द्वा॒द॒शा॒हस्येति॑ द्वादश - अ॒हस्य॑ । प्र॒थ॒मेन॑ । अह्ना᳚ । ऋ॒त्विजा᳚म् । यज॑मानः । वृ॒ङ्क्ते॒ । इति॑ । तेजः॑ । इ॒न्द्रि॒यम् । इति॑ । किम् । द्वि॒तीये॑न । इति॑ । प्रा॒णानिति॑ प्र - अ॒नान् । अ॒न्नाद्य॒मित्य॑न्न - अद्य᳚म् । इति॑ । किम् । तृ॒तीये॑न । इति॑ । त्रीन् । इ॒मान् । लो॒कान् । इति॑ । किम् । च॒तु॒र्त्थेन॑ । इति॑ । चतु॑ष्पद॒ इति॒ चतुः॑-प॒दः॒ । प॒शून् । इति॑ । किम् । प॒ञ्च॒मेन॑ । इति॑ । पञ्चा᳚क्षरा॒मिति॒ पञ्च॑ - अ॒क्ष॒रा॒म् । प॒ङ्क्तिम् । इति॑ । किम् । ष॒ष्ठेन॑ । इति॑ । षट् । ऋ॒तून् । इति॑ । किम् । स॒प्त॒मेन॑ । इति॑ । स॒प्तप॑दा॒मिति॑ स॒प्त - प॒दा॒म् । शक्व॑रीम् । इति॑ ।  \newline


\textbf{Krama Paata} \newline

ब्र॒ह्म॒वा॒दिनो॑ वदन्ति । ब्र॒ह्म॒वा॒दिन॒ इति॑ ब्रह्म - वा॒दिनः॑ । व॒द॒न्ति॒ किम् । किम् द्वा॑दशा॒हस्य॑ । द्वा॒द॒शा॒हस्य॑ प्रथ॒मेन॑ । द्वा॒द॒शा॒हस्येति॑ द्वादश - अ॒हस्य॑ । प्र॒थ॒मेनाह्ना᳚ । अह्न॒र्त्विजा᳚म् । ऋ॒त्विजा॒म् ॅयज॑मानः । यज॑मानो वृङ्‍क्ते । वृ॒ङ्‍क्त॒ इति॑ । इति॒ तेजः॑ । तेज॑ इन्द्रि॒यम् । इ॒न्द्रि॒यमिति॑ । इति॒ किम् । किम् द्वि॒तीये॑न । द्वि॒तीये॒नेति॑ । इति॑ प्रा॒णान् । प्रा॒णान॒न्नाद्य᳚म् । प्रा॒णानिति॑ प्र - अ॒नान् । अ॒न्नाद्य॒मिति॑ । अ॒न्नाद्य॒मित्य॑न्न - अद्य᳚म् । इति॒ किम् । किम् तृ॒तीये॑न । तृ॒तीये॒नेति॑ । इति॒ त्रीन् । त्रीनि॒मान् । इ॒माम् ॅलो॒कान् । लो॒कानिति॑ । इति॒ किम् । किम् च॑तु॒र्थेन॑ । च॒तु॒र्थेनेति॑ । इति॒ चतु॑ष्पदः । चतु॑ष्पदः प॒शून् । चतु॑ष्पद॒ इति॒ चतुः॑ - प॒दः॒ । प॒शूनिति॑ । इति॒ किम् । किम् प॑ञ्च॒मेन॑ । प॒ञ्च॒मेनेति॑ । इति॒ पञ्चा᳚क्षराम् । पञ्चा᳚क्षराम् प॒ङ्‍क्तिम् । पञ्चा᳚क्षरा॒मिति॒ पञ्च॑ - अ॒क्ष॒रा॒म् । प॒ङ्‍क्तिमिति॑ । इति॒ किम् । किꣳ ष॒ष्ठेन॑ । ष॒ष्ठेनेति॑ । इति॒ षट् । षडृ॒तून् । ऋ॒तूनिति॑ । इति॒ किम् । किꣳ स॑प्त॒मेन॑ । स॒प्त॒मेनेति॑ । इति॑ स॒प्तप॑दाम् । स॒प्तप॑दाꣳ॒॒ शक्व॑रीम् ( ) । स॒प्तप॑दा॒मिति॑ स॒प्त - प॒दा॒म् । शक्व॑री॒मिति॑ । इति॒ किम् \newline

\textbf{Jatai Paata} \newline

1. ब्र॒ह्म॒वा॒दिनो॑ वदन्ति वदन्ति ब्रह्मवा॒दिनो᳚ ब्रह्मवा॒दिनो॑ वदन्ति । \newline
2. ब्र॒ह्म॒वा॒दिन॒ इति॑ ब्रह्म - वा॒दिनः॑ । \newline
3. व॒द॒न्ति॒ किम् किं ॅव॑दन्ति वदन्ति॒ किम् । \newline
4. किम् द्वा॑दशा॒हस्य॑ द्वादशा॒हस्य॒ किम् किम् द्वा॑दशा॒हस्य॑ । \newline
5. द्वा॒द॒शा॒हस्य॑ प्रथ॒मेन॑ प्रथ॒मेन॑ द्वादशा॒हस्य॑ द्वादशा॒हस्य॑ प्रथ॒मेन॑ । \newline
6. द्वा॒द॒शा॒हस्येति॑ द्वादश - अ॒हस्य॑ । \newline
7. प्र॒थ॒मे नाह्ना ऽह्ना᳚ प्रथ॒मेन॑ प्रथ॒मे नाह्ना᳚ । \newline
8. अह्न॒ र्‌त्विजा॑ मृ॒त्विजा॒ मह्ना ऽह्न॒ र्‌त्विजा᳚म् । \newline
9. ऋ॒त्विजां॒ ॅयज॑मानो॒ यज॑मान ऋ॒त्विजा॑ मृ॒त्विजां॒ ॅयज॑मानः । \newline
10. यज॑मानो वृङ्क्ते वृङ्क्ते॒ यज॑मानो॒ यज॑मानो वृङ्क्ते । \newline
11. वृ॒ङ्क्त॒ इतीति॑ वृङ्क्ते वृङ्क्त॒ इति॑ । \newline
12. इति॒ तेज॒ स्तेज॒ इतीति॒ तेजः॑ । \newline
13. तेज॑ इन्द्रि॒य मि॑न्द्रि॒यम् तेज॒ स्तेज॑ इन्द्रि॒यम् । \newline
14. इ॒न्द्रि॒य मिती ती᳚न्द्रि॒य मि॑न्द्रि॒य मिति॑ । \newline
15. इति॒ किम् कि मितीति॒ किम् । \newline
16. किम् द्वि॒तीये॑न द्वि॒तीये॑न॒ किम् किम् द्वि॒तीये॑न । \newline
17. द्वि॒तीये॒नेतीति॑ द्वि॒तीये॑न द्वि॒तीये॒ नेति॑ । \newline
18. इति॑ प्रा॒णान् प्रा॒णानितीति॑ प्रा॒णान् । \newline
19. प्रा॒णा न॒न्नाद्य॑ म॒न्नाद्य॑म् प्रा॒णान् प्रा॒णा न॒न्नाद्य᳚म् । \newline
20. प्रा॒णानिति॑ प्र - अ॒नान् । \newline
21. अ॒न्नाद्य॒ मिती त्य॒न्नाद्य॑ म॒न्नाद्य॒ मिति॑ । \newline
22. अ॒न्नाद्य॒मित्य॑न्न - अद्य᳚म् । \newline
23. इति॒ किम् कि मितीति॒ किम् । \newline
24. किम् तृ॒तीये॑न तृ॒तीये॑न॒ किम् किम् तृ॒तीये॑न । \newline
25. तृ॒तीये॒ नेतीति॑ तृ॒तीये॑न तृ॒तीये॒ नेति॑ । \newline
26. इति॒ त्रीꣳ स्त्री नितीति॒ त्रीन् । \newline
27. त्री नि॒मा नि॒मान् त्रीꣳ स्त्री नि॒मान् । \newline
28. इ॒मान् ॅलो॒कान् ॅलो॒का नि॒मा नि॒मान् ॅलो॒कान् । \newline
29. लो॒कानितीति॑ लो॒कान् ॅलो॒कानिति॑ । \newline
30. इति॒ किम् कि मितीति॒ किम् । \newline
31. किम् च॑तु॒र्त्थेन॑ चतु॒र्त्थेन॒ किम् किम् च॑तु॒र्त्थेन॑ । \newline
32. च॒तु॒र्त्थे नेतीति॑ चतु॒र्त्थेन॑ चतु॒र्त्थे नेति॑ । \newline
33. इति॒ चतु॑ष्पद॒ श्चतु॑ष्पद॒ इतीति॒ चतु॑ष्पदः । \newline
34. चतु॑ष्पदः प॒शून् प॒शूꣳ श्चतु॑ष्पद॒ श्चतु॑ष्पदः प॒शून् । \newline
35. चतु॑ष्पद॒ इति॒ चतुः॑ - प॒दः॒ । \newline
36. प॒शूनितीति॑ प॒शून् प॒शूनिति॑ । \newline
37. इति॒ किम् कि मितीति॒ किम् । \newline
38. किम् प॑ञ्च॒मेन॑ पञ्च॒मेन॒ किम् किम् प॑ञ्च॒मेन॑ । \newline
39. प॒ञ्च॒मे नेतीति॑ पञ्च॒मेन॑ पञ्च॒मे नेति॑ । \newline
40. इति॒ पञ्चा᳚क्षरा॒म् पञ्चा᳚क्षरा॒ मितीति॒ पञ्चा᳚क्षराम् । \newline
41. पञ्चा᳚क्षराम् प॒ङ्क्तिम् प॒ङ्क्तिम् पञ्चा᳚क्षरा॒म् पञ्चा᳚क्षराम् प॒ङ्क्तिम् । \newline
42. पञ्चा᳚क्षरा॒मिति॒ पञ्च॑ - अ॒क्ष॒रा॒म् । \newline
43. प॒ङ्क्ति मितीति॑ प॒ङ्क्तिम् प॒ङ्क्ति मिति॑ । \newline
44. इति॒ किम् किमि तीति॒ किम् । \newline
45. किꣳ ष॒ष्ठेन॑ ष॒ष्ठेन॒ किम् किꣳ ष॒ष्ठेन॑ । \newline
46. ष॒ष्ठे नेतीति॑ ष॒ष्ठेन॑ ष॒ष्ठे नेति॑ । \newline
47. इति॒ षट् थ्षडितीति॒ षट् । \newline
48. षडृ॒तू नृ॒तून् षट् थ्षडृ॒तून् । \newline
49. ऋ॒तूनिती त्यृ॒तू नृ॒तूनिति॑ । \newline
50. इति॒ किम् किमितीति॒ किम् । \newline
51. किꣳ स॑प्त॒मेन॑ सप्त॒मेन॒ किम् किꣳ स॑प्त॒मेन॑ । \newline
52. स॒प्त॒मे नेतीति॑ सप्त॒मेन॑ सप्त॒मे नेति॑ । \newline
53. इति॑ स॒प्तप॑दाꣳ स॒प्तप॑दा॒ मितीति॑ स॒प्तप॑दाम् । \newline
54. स॒प्तप॑दाꣳ॒॒ शक्व॑रीꣳ॒॒ शक्व॑रीꣳ स॒प्तप॑दाꣳ स॒प्तप॑दाꣳ॒॒ शक्व॑रीम् । \newline
55. स॒प्तप॑दा॒मिति॑ स॒प्त - प॒दा॒म् । \newline
56. शक्व॑री॒ मितीति॒ शक्व॑रीꣳ॒॒ शक्व॑री॒ मिति॑ । \newline
57. इति॒ किम् किमि तीति॒ किम् । \newline

\textbf{Ghana Paata } \newline

1. ब्र॒ह्म॒वा॒दिनो॑ वदन्ति वदन्ति ब्रह्मवा॒दिनो᳚ ब्रह्मवा॒दिनो॑ वदन्ति॒ किम् किं ॅव॑दन्ति ब्रह्मवा॒दिनो᳚ ब्रह्मवा॒दिनो॑ वदन्ति॒ किम् । \newline
2. ब्र॒ह्म॒वा॒दिन॒ इति॑ ब्रह्म - वा॒दिनः॑ । \newline
3. व॒द॒न्ति॒ किम् किं ॅव॑दन्ति वदन्ति॒ किम् द्वा॑दशा॒हस्य॑ द्वादशा॒हस्य॒ किं ॅव॑दन्ति वदन्ति॒ किम् द्वा॑दशा॒हस्य॑ । \newline
4. किम् द्वा॑दशा॒हस्य॑ द्वादशा॒हस्य॒ किम् किम् द्वा॑दशा॒हस्य॑ प्रथ॒मेन॑ प्रथ॒मेन॑ द्वादशा॒हस्य॒ किम् किम् द्वा॑दशा॒हस्य॑ प्रथ॒मेन॑ । \newline
5. द्वा॒द॒शा॒हस्य॑ प्रथ॒मेन॑ प्रथ॒मेन॑ द्वादशा॒हस्य॑ द्वादशा॒हस्य॑ प्रथ॒मेनाह्ना ऽह्ना᳚ प्रथ॒मेन॑ द्वादशा॒हस्य॑ द्वादशा॒हस्य॑ प्रथ॒मेनाह्ना᳚ । \newline
6. द्वा॒द॒शा॒हस्येति॑ द्वादश - अ॒हस्य॑ । \newline
7. प्र॒थ॒मे नाह्ना ऽह्ना᳚ प्रथ॒मेन॑ प्रथ॒मे नाह्न॒ र्‌त्विजा॑ मृ॒त्विजा॒ मह्ना᳚ प्रथ॒मेन॑ प्रथ॒मे नाह्न॒ र्‌त्विजा᳚म् । \newline
8. अह्न॒ र्‌त्विजा॑ मृ॒त्विजा॒ मह्ना ऽह्न॒ र्‌त्विजां॒ ॅयज॑मानो॒ यज॑मान ऋ॒त्विजा॒ मह्ना ऽह्न॒ र्‌त्विजां॒ ॅयज॑मानः । \newline
9. ऋ॒त्विजां॒ ॅयज॑मानो॒ यज॑मान ऋ॒त्विजा॑ मृ॒त्विजां॒ ॅयज॑मानो वृङ्क्ते वृङ्क्ते॒ यज॑मान ऋ॒त्विजा॑ मृ॒त्विजां॒ ॅयज॑मानो वृङ्क्ते । \newline
10. यज॑मानो वृङ्क्ते वृङ्क्ते॒ यज॑मानो॒ यज॑मानो वृङ्क्त॒ इतीति॑ वृङ्क्ते॒ यज॑मानो॒ यज॑मानो वृङ्क्त॒ इति॑ । \newline
11. वृ॒ङ्क्त॒ इतीति॑ वृङ्क्ते वृङ्क्त॒ इति॒ तेज॒ स्तेज॒ इति॑ वृङ्क्ते वृङ्क्त॒ इति॒ तेजः॑ । \newline
12. इति॒ तेज॒ स्तेज॒ इतीति॒ तेज॑ इन्द्रि॒य मि॑न्द्रि॒यम् तेज॒ इतीति॒ तेज॑ इन्द्रि॒यम् । \newline
13. तेज॑ इन्द्रि॒य मि॑न्द्रि॒यम् तेज॒ स्तेज॑ इन्द्रि॒य मितीती᳚न्द्रि॒यम् तेज॒ स्तेज॑ इन्द्रि॒य मिति॑ । \newline
14. इ॒न्द्रि॒य मिती ती᳚न्द्रि॒य मि॑न्द्रि॒य मिति॒ किम् कि मिती᳚न्द्रि॒य मि॑न्द्रि॒य मिति॒ किम् । \newline
15. इति॒ किम् कि मितीति॒ किम् द्वि॒तीये॑न द्वि॒तीये॑न॒ कि मितीति॒ किम् द्वि॒तीये॑न । \newline
16. किम् द्वि॒तीये॑न द्वि॒तीये॑न॒ किम् किम् द्वि॒तीये॒नेतीति॑ द्वि॒तीये॑न॒ किम् किम् द्वि॒तीये॒नेति॑ । \newline
17. द्वि॒तीये॒नेतीति॑ द्वि॒तीये॑न द्वि॒तीये॒नेति॑ प्रा॒णान् प्रा॒णा निति॑ द्वि॒तीये॑न द्वि॒तीये॒नेति॑ प्रा॒णान् । \newline
18. इति॑ प्रा॒णान् प्रा॒णा नितीति॑ प्रा॒णा न॒न्नाद्य॑ म॒न्नाद्य॑म् प्रा॒णा नितीति॑ प्रा॒णा न॒न्नाद्य᳚म् । \newline
19. प्रा॒णा न॒न्नाद्य॑ म॒न्नाद्य॑म् प्रा॒णान् प्रा॒णा न॒न्नाद्य॒ मिती त्य॒न्नाद्य॑म् प्रा॒णान् प्रा॒णा न॒न्नाद्य॒ मिति॑ । \newline
20. प्रा॒णानिति॑ प्र - अ॒नान् । \newline
21. अ॒न्नाद्य॒ मिती त्य॒न्नाद्य॑ म॒न्नाद्य॒ मिति॒ किम् कि मित्य॒न्नाद्य॑ म॒न्नाद्य॒ मिति॒ किम् । \newline
22. अ॒न्नाद्य॒मित्य॑न्न - अद्य᳚म् । \newline
23. इति॒ किम् कि मितीति॒ किम् तृ॒तीये॑न तृ॒तीये॑न॒ कि मितीति॒ किम् तृ॒तीये॑न । \newline
24. किम् तृ॒तीये॑न तृ॒तीये॑न॒ किम् किम् तृ॒तीये॒नेतीति॑ तृ॒तीये॑न॒ किम् किम् तृ॒तीये॒नेति॑ । \newline
25. तृ॒तीये॒नेतीति॑ तृ॒तीये॑न तृ॒तीये॒नेति॒ त्रीꣳ स्त्री निति॑ तृ॒तीये॑न तृ॒तीये॒नेति॒ त्रीन् । \newline
26. इति॒ त्रीꣳ स्त्री नितीति॒ त्री नि॒मा नि॒मा न्त्री नितीति॒ त्री नि॒मान् । \newline
27. त्री नि॒मा नि॒मान् त्रीꣳ स्त्री नि॒मान् ॅलो॒कान् ॅलो॒का नि॒मान् त्रीꣳ स्त्री नि॒मान् ॅलो॒कान् । \newline
28. इ॒मान् ॅलो॒कान् ॅलो॒का नि॒मा नि॒मान् ॅलो॒का नितीति॑ लो॒का नि॒मा नि॒मान् ॅलो॒का निति॑ । \newline
29. लो॒का नितीति॑ लो॒कान् ॅलो॒का निति॒ किम् कि मिति॑ लो॒कान् ॅलो॒का निति॒ किम् । \newline
30. इति॒ किम् कि मितीति॒ किम् च॑तु॒र्त्थेन॑ चतु॒र्त्थेन॒ कि मितीति॒ किम् च॑तु॒र्त्थेन॑ । \newline
31. किम् च॑तु॒र्त्थेन॑ चतु॒र्त्थेन॒ किम् किम् च॑तु॒र्त्थेनेतीति॑ चतु॒र्त्थेन॒ किम् किम् च॑तु॒र्त्थेनेति॑ । \newline
32. च॒तु॒र्त्थेनेतीति॑ चतु॒र्त्थेन॑ चतु॒र्त्थेनेति॒ चतु॑ष्पद॒ श्चतु॑ष्पद॒ इति॑ चतु॒र्त्थेन॑ चतु॒र्त्थेनेति॒ चतु॑ष्पदः । \newline
33. इति॒ चतु॑ष्पद॒ श्चतु॑ष्पद॒ इतीति॒ चतु॑ष्पदः प॒शून् प॒शूꣳ श्चतु॑ष्पद॒ इतीति॒ चतु॑ष्पदः प॒शून् । \newline
34. चतु॑ष्पदः प॒शून् प॒शूꣳ श्चतु॑ष्पद॒ श्चतु॑ष्पदः प॒शू नितीति॑ प॒शूꣳ श्चतु॑ष्पद॒ श्चतु॑ष्पदः प॒शू निति॑ । \newline
35. चतु॑ष्पद॒ इति॒ चतुः॑ - प॒दः॒ । \newline
36. प॒शू नितीति॑ प॒शून् प॒शू निति॒ किम् कि मिति॑ प॒शून् प॒शू निति॒ किम् । \newline
37. इति॒ किम् कि मितीति॒ किम् प॑ञ्च॒मेन॑ पञ्च॒मेन॒ कि मितीति॒ किम् प॑ञ्च॒मेन॑ । \newline
38. किम् प॑ञ्च॒मेन॑ पञ्च॒मेन॒ किम् किम् प॑ञ्च॒मेनेतीति॑ पञ्च॒मेन॒ किम् किम् प॑ञ्च॒मेनेति॑ । \newline
39. प॒ञ्च॒मेनेतीति॑ पञ्च॒मेन॑ पञ्च॒मेनेति॒ पञ्चा᳚क्षरा॒म् पञ्चा᳚क्षरा॒ मिति॑ पञ्च॒मेन॑ पञ्च॒मेनेति॒ पञ्चा᳚क्षराम् । \newline
40. इति॒ पञ्चा᳚क्षरा॒म् पञ्चा᳚क्षरा॒ मितीति॒ पञ्चा᳚क्षराम् प॒ङ्क्तिम् प॒ङ्क्तिम् पञ्चा᳚क्षरा॒ मितीति॒ पञ्चा᳚क्षराम् प॒ङ्क्तिम् । \newline
41. पञ्चा᳚क्षराम् प॒ङ्क्तिम् प॒ङ्क्तिम् पञ्चा᳚क्षरा॒म् पञ्चा᳚क्षराम् प॒ङ्क्ति मितीति॑ प॒ङ्क्तिम् पञ्चा᳚क्षरा॒म् पञ्चा᳚क्षराम् प॒ङ्क्ति मिति॑ । \newline
42. पञ्चा᳚क्षरा॒मिति॒ पञ्च॑ - अ॒क्ष॒रा॒म् । \newline
43. प॒ङ्क्ति मितीति॑ प॒ङ्क्तिम् प॒ङ्क्ति मिति॒ किम् कि मिति॑ प॒ङ्क्तिम् प॒ङ्क्ति मिति॒ किम् । \newline
44. इति॒ किम् कि मितीति॒ किꣳ ष॒ष्ठेन॑ ष॒ष्ठेन॒ कि मितीति॒ किꣳ ष॒ष्ठेन॑ । \newline
45. किꣳ ष॒ष्ठेन॑ ष॒ष्ठेन॒ किम् किꣳ ष॒ष्ठेनेतीति॑ ष॒ष्ठेन॒ किम् किꣳ ष॒ष्ठेनेति॑ । \newline
46. ष॒ष्ठेनेतीति॑ ष॒ष्ठेन॑ ष॒ष्ठेनेति॒ षट् थ्षडिति॑ ष॒ष्ठेन॑ ष॒ष्ठेनेति॒ षट् । \newline
47. इति॒ षट् थ्षडितीति॒ षडृ॒तू नृ॒तून् षडितीति॒ षडृ॒तून् । \newline
48. षडृ॒तू नृ॒तून् षट् थ्षडृ॒तू निती त्यृ॒तून् षट् थ्षडृ॒तू निति॑ । \newline
49. ऋ॒तू निती त्यृ॒तू नृ॒तू निति॒ किम् कि मित्यृ॒तू नृ॒तू निति॒ किम् । \newline
50. इति॒ किम् कि मितीति॒ किꣳ स॑प्त॒मेन॑ सप्त॒मेन॒ कि मितीति॒ किꣳ स॑प्त॒मेन॑ । \newline
51. किꣳ स॑प्त॒मेन॑ सप्त॒मेन॒ किम् किꣳ स॑प्त॒मेनेतीति॑ सप्त॒मेन॒ किम् किꣳ स॑प्त॒मेनेति॑ । \newline
52. स॒प्त॒मेनेतीति॑ सप्त॒मेन॑ सप्त॒मेनेति॑ स॒प्तप॑दाꣳ स॒प्तप॑दा॒ मिति॑ सप्त॒मेन॑ सप्त॒मेनेति॑ स॒प्तप॑दाम् । \newline
53. इति॑ स॒प्तप॑दाꣳ स॒प्तप॑दा॒ मितीति॑ स॒प्तप॑दाꣳ॒॒ शक्व॑रीꣳ॒॒ शक्व॑रीꣳ स॒प्तप॑दा॒ मितीति॑ स॒प्तप॑दाꣳ॒॒ शक्व॑रीम् । \newline
54. स॒प्तप॑दाꣳ॒॒ शक्व॑रीꣳ॒॒ शक्व॑रीꣳ स॒प्तप॑दाꣳ स॒प्तप॑दाꣳ॒॒ शक्व॑री॒ मितीति॒ शक्व॑रीꣳ स॒प्तप॑दाꣳ स॒प्तप॑दाꣳ॒॒ शक्व॑री॒ मिति॑ । \newline
55. स॒प्तप॑दा॒मिति॑ स॒प्त - प॒दा॒म् । \newline
56. शक्व॑री॒ मितीति॒ शक्व॑रीꣳ॒॒ शक्व॑री॒ मिति॒ किम् कि मिति॒ शक्व॑रीꣳ॒॒ शक्व॑री॒ मिति॒ किम् । \newline
57. इति॒ किम् कि मितीति॒ कि म॑ष्ट॒मेना᳚ ष्ट॒मेन॒ कि मितीति॒ कि म॑ष्ट॒मेन॑ । \newline
\pagebreak
\markright{ TS 7.3.2.2  \hfill https://www.vedavms.in \hfill}

\section{ TS 7.3.2.2 }

\textbf{TS 7.3.2.2 } \newline
\textbf{Samhita Paata} \newline

किम॑ष्ट॒मेनेत्य॒ष्टाक्ष॑रां गाय॒त्रीमिति॒ किं न॑व॒मेनेति॑ त्रि॒वृतꣳ॒॒ स्तोम॒मिति॒ किं द॑श॒मेनेति॒ दशा᳚क्षरां ॅवि॒राज॒मिति॒ किमे॑काद॒शेनेत्येका॑दशाक्षरां त्रि॒ष्टुभ॒मिति॒ किं द्वा॑द॒शेनेति॒ द्वाद॑शाक्षरां॒ जग॑ती॒मित्ये॒ताव॒द्वा अ॑स्ति॒ याव॑दे॒तद्-याव॑दे॒वास्ति॒ तदे॑षां ॅवृङ्क्ते ॥ \newline

\textbf{Pada Paata} \newline

किम् । अ॒ष्ट॒मेन॑ । इति॑ । अ॒ष्टाक्ष॑रा॒मित्य॒ष्टा - अ॒क्ष॒रा॒म् । गा॒य॒त्रीम् । इति॑ । किम् । न॒व॒मेन॑ । इति॑ । त्रि॒वृत॒मिति॑ त्रि - वृत᳚म् । स्तोम᳚म् । इति॑ । किम् । द॒श॒मेन॑ । इति॑ । दशा᳚क्षरा॒मिति॒ दश॑ - अ॒क्ष॒रा॒म् । वि॒राज॒मिति॑ वि - राज᳚म् । इति॑ । किम् । ए॒का॒द॒शेन॑ । इति॑ । एका॑दशाक्षरा॒मित्येका॑दश - अ॒क्ष॒रा॒म् । त्रि॒ष्टुभ᳚म् । इति॑ । किम् । द्वा॒द॒शेन॑ । इति॑ । द्वाद॑शाक्षरा॒मिति॒ द्वाद॑श - अ॒क्ष॒रा॒म् । जग॑तीम् । इति॑ । ए॒ताव॑त् । वै । अ॒स्ति॒ । याव॑त् । ए॒तत् । याव॑त् । ए॒व । अस्ति॑ । तत् । ए॒षा॒म् । वृ॒ङ्क्ते॒ ॥  \newline


\textbf{Krama Paata} \newline

किम॑ष्ट॒मेन॑ । अ॒ष्ट॒मेनेति॑ । इत्य॒ष्टाक्ष॑राम् । अ॒ष्टाक्ष॑राम् गाय॒त्रीम् । अ॒ष्टाक्ष॑रा॒मित्य॒ष्टा - अ॒क्ष॒रा॒म् । गा॒य॒त्रीमिति॑ । इति॒ किम् । किम् न॑व॒मेन॑ । न॒व॒मेनेति॑ । इति॑ त्रि॒वृत᳚म् । त्रि॒वृतꣳ॒॒ स्तोम᳚म् । त्रि॒वृत॒मिति॑ त्रि - वृत᳚म् । स्तोम॒मिति॑ । इति॒ किम् । किम् द॑श॒मेन॑ । द॒श॒मेनेति॑ । इति॒ दशा᳚क्षराम् । दशा᳚क्षराम् ॅवि॒राज᳚म् । दशा᳚क्षरा॒मिति॒ दश॑ - अ॒क्ष॒रा॒म् । वि॒राज॒मिति॑ । वि॒राज॒मिति॑ वि - राज᳚म् । इति॒ किम् । किमे॑काद॒शेन॑ । ए॒का॒द॒शेनेति॑ । इत्येका॑दशाक्षराम् । एका॑दशाक्षराम् त्रि॒ष्टुभ᳚म् । एका॑दशाक्षरा॒मित्येका॑दश - अ॒क्ष॒रा॒म् । त्रि॒ष्टुभ॒मिति॑ । इति॒ किम् । किम् द्वा॑द॒शेन॑ । द्वा॒द॒शेनेति॑ । इति॒ द्वाद॑शाक्षराम् । द्वाद॑शाक्षरा॒म् जग॑तीम् । द्वाद॑शाक्षरा॒मिति॒ द्वाद॑श - अ॒क्ष॒रा॒म् । जग॑ती॒मिति॑ । इत्ये॒ताव॑त् । ए॒ताव॒द् वै । वा अ॑स्ति । अ॒स्ति॒ याव॑त् । याव॑दे॒तत् । ए॒तद् याव॑त् । याव॑दे॒व । ए॒वास्ति॑ । अस्ति॒ तत् । तदे॑षाम् । ए॒षा॒म् ॅवृ॒ङ्‍क्ते॒ । वृ॒ङ्‍क्ते॒ इति॑ वृङ्‍क्ते । \newline

\textbf{Jatai Paata} \newline

1. कि म॑ष्ट॒मेना᳚ ष्ट॒मेन॒ किम् कि म॑ष्ट॒मेन॑ । \newline
2. अ॒ष्ट॒मेने तीत्य॑ष्ट॒मेना᳚ ष्ट॒मेनेति॑ । \newline
3. इत्य॒ष्टाक्ष॑रा म॒ष्टाक्ष॑रा॒ मिती त्य॒ष्टाक्ष॑राम् । \newline
4. अ॒ष्टाक्ष॑राम् गाय॒त्रीम् गा॑य॒त्री म॒ष्टाक्ष॑रा म॒ष्टाक्ष॑राम् गाय॒त्रीम् । \newline
5. अ॒ष्टाक्ष॑रा॒मित्य॒ष्टा - अ॒क्ष॒रा॒म् । \newline
6. गा॒य॒त्री मितीति॑ गाय॒त्रीम् गा॑य॒त्री मिति॑ । \newline
7. इति॒ किम् किमितीति॒ किम् । \newline
8. किन् न॑व॒मेन॑ नव॒मेन॒ किम् किन् न॑व॒मेन॑ । \newline
9. न॒व॒मे नेतीति॑ नव॒मेन॑ नव॒मे नेति॑ । \newline
10. इति॑ त्रि॒वृत॑म् त्रि॒वृत॒ मितीति॑ त्रि॒वृत᳚म् । \newline
11. त्रि॒वृतꣳ॒॒ स्तोमꣳ॒॒ स्तोम॑म् त्रि॒वृत॑म् त्रि॒वृतꣳ॒॒ स्तोम᳚म् । \newline
12. त्रि॒वृत॒मिति॑ त्रि - वृत᳚म् । \newline
13. स्तोम॒ मितीति॒ स्तोमꣳ॒॒ स्तोम॒ मिति॑ । \newline
14. इति॒ किम् किमि तीति॒ किम् । \newline
15. किम् द॑श॒मेन॑ दश॒मेन॒ किम् किम् द॑श॒मेन॑ । \newline
16. द॒श॒मे नेतीति॑ दश॒मेन॑ दश॒मे नेति॑ । \newline
17. इति॒ दशा᳚क्षरा॒म् दशा᳚क्षरा॒ मितीति॒ दशा᳚क्षराम् । \newline
18. दशा᳚क्षरां ॅवि॒राजं॑ ॅवि॒राज॒म् दशा᳚क्षरा॒म् दशा᳚क्षरां ॅवि॒राज᳚म् । \newline
19. दशा᳚क्षरा॒मिति॒ दश॑ - अ॒क्ष॒रा॒म् । \newline
20. वि॒राज॒ मितीति॑ वि॒राजं॑ ॅवि॒राज॒ मिति॑ । \newline
21. वि॒राज॒मिति॑ वि - राज᳚म् । \newline
22. इति॒ किम् किमितीति॒ किम् । \newline
23. किमे॑काद॒शे नै॑काद॒शेन॒ किम् किमे॑काद॒शेन॑ । \newline
24. ए॒का॒द॒शेने तीत्ये॑काद॒शे नै॑काद॒शे नेति॑ । \newline
25. इत्येका॑दशाक्षरा॒ मेका॑दशाक्षरा॒ मिती त्येका॑दशाक्षराम् । \newline
26. एका॑दशाक्षराम् त्रि॒ष्टुभ॑म् त्रि॒ष्टुभ॒ मेका॑दशाक्षरा॒ मेका॑दशाक्षराम् त्रि॒ष्टुभ᳚म् । \newline
27. एका॑दशाक्षरा॒मित्येका॑दश - अ॒क्ष॒रा॒म् । \newline
28. त्रि॒ष्टुभ॒ मितीति॑ त्रि॒ष्टुभ॑म् त्रि॒ष्टुभ॒ मिति॑ । \newline
29. इति॒ किम् किमितीति॒ किम् । \newline
30. किम् द्वा॑द॒शेन॑ द्वाद॒शेन॒ किम् किम् द्वा॑द॒शेन॑ । \newline
31. द्वा॒द॒शे नेतीति॑ द्वाद॒शेन॑ द्वाद॒शे नेति॑ । \newline
32. इति॒ द्वाद॑शाक्षरा॒म् द्वाद॑शाक्षरा॒ मितीति॒ द्वाद॑शाक्षराम् । \newline
33. द्वाद॑शाक्षरा॒म् जग॑ती॒म् जग॑ती॒म् द्वाद॑शाक्षरा॒म् द्वाद॑शाक्षरा॒म् जग॑तीम् । \newline
34. द्वाद॑शाक्षरा॒मिति॒ द्वाद॑श - अ॒क्ष॒रा॒म् । \newline
35. जग॑ती॒ मितीति॒ जग॑ती॒म् जग॑ती॒ मिति॑ । \newline
36. इत्ये॒ताव॑ दे॒ताव॒ दितीत्ये॒ताव॑त् । \newline
37. ए॒ताव॒द् वै वा ए॒ताव॑ दे॒ताव॒द् वै । \newline
38. वा अ॑स्त्यस्ति॒ वै वा अ॑स्ति । \newline
39. अ॒स्ति॒ याव॒द् याव॑द स्त्यस्ति॒ याव॑त् । \newline
40. याव॑ दे॒त दे॒तद् याव॒द् याव॑ दे॒तत् । \newline
41. ए॒तद् याव॒द् याव॑ दे॒त दे॒तद् याव॑त् । \newline
42. याव॑ दे॒वैव याव॒द् याव॑ दे॒व । \newline
43. ए॒वा स्त्य स्त्ये॒वै वास्ति॑ । \newline
44. अस्ति॒ तत् तद स्त्यस्ति॒ तत् । \newline
45. तदे॑षा मेषा॒म् तत् तदे॑षाम् । \newline
46. ए॒षां॒ ॅवृ॒ङ्क्ते॒ वृ॒ङ्क्त॒ ए॒षा॒ मे॒षां॒ ॅवृ॒ङ्क्ते॒ । \newline
47. वृ॒ङ्‍क्ते॒ इति॑ वृङ्‍क्ते । \newline

\textbf{Ghana Paata } \newline

1. कि म॑ष्ट॒मेना᳚ ष्ट॒मेन॒ किम् कि म॑ष्ट॒मेने तीत्य॑ष्ट॒मेन॒ किम् कि म॑ष्ट॒मेनेति॑ । \newline
2. अ॒ष्ट॒मेने तीत्य॑ष्ट॒मेना᳚ ष्ट॒मेने त्य॒ष्टाक्ष॑रा म॒ष्टाक्ष॑रा॒ मित्य॑ष्ट॒मेना᳚ ष्ट॒मेने त्य॒ष्टाक्ष॑राम् । \newline
3. इत्य॒ष्टाक्ष॑रा म॒ष्टाक्ष॑रा॒ मिती त्य॒ष्टाक्ष॑राम् गाय॒त्रीम् गा॑य॒त्री म॒ष्टाक्ष॑रा॒ मिती त्य॒ष्टाक्ष॑राम् गाय॒त्रीम् । \newline
4. अ॒ष्टाक्ष॑राम् गाय॒त्रीम् गा॑य॒त्री म॒ष्टाक्ष॑रा म॒ष्टाक्ष॑राम् गाय॒त्री मितीति॑ गाय॒त्री म॒ष्टाक्ष॑रा म॒ष्टाक्ष॑राम् गाय॒त्री मिति॑ । \newline
5. अ॒ष्टाक्ष॑रा॒मित्य॒ष्टा - अ॒क्ष॒रा॒म् । \newline
6. गा॒य॒त्री मितीति॑ गाय॒त्रीम् गा॑य॒त्री मिति॒ किम् कि मिति॑ गाय॒त्रीम् गा॑य॒त्री मिति॒ किम् । \newline
7. इति॒ किम् कि मितीति॒ किम् न॑व॒मेन॑ नव॒मेन॒ कि मितीति॒ किम् न॑व॒मेन॑ । \newline
8. किम् न॑व॒मेन॑ नव॒मेन॒ किम् किम् न॑व॒मेनेतीति॑ नव॒मेन॒ किम् किम् न॑व॒मेनेति॑ । \newline
9. न॒व॒मेनेतीति॑ नव॒मेन॑ नव॒मेनेति॑ त्रि॒वृत॑म् त्रि॒वृत॒ मिति॑ नव॒मेन॑ नव॒मेनेति॑ त्रि॒वृत᳚म् । \newline
10. इति॑ त्रि॒वृत॑म् त्रि॒वृत॒ मितीति॑ त्रि॒वृतꣳ॒॒ स्तोमꣳ॒॒ स्तोम॑म् त्रि॒वृत॒ मितीति॑ त्रि॒वृतꣳ॒॒ स्तोम᳚म् । \newline
11. त्रि॒वृतꣳ॒॒ स्तोमꣳ॒॒ स्तोम॑म् त्रि॒वृत॑म् त्रि॒वृतꣳ॒॒ स्तोम॒ मितीति॒ स्तोम॑म् त्रि॒वृत॑म् त्रि॒वृतꣳ॒॒ स्तोम॒ मिति॑ । \newline
12. त्रि॒वृत॒मिति॑ त्रि - वृत᳚म् । \newline
13. स्तोम॒ मितीति॒ स्तोमꣳ॒॒ स्तोम॒ मिति॒ किम् कि मिति॒ स्तोमꣳ॒॒ स्तोम॒ मिति॒ किम् । \newline
14. इति॒ किम् कि मितीति॒ किम् द॑श॒मेन॑ दश॒मेन॒ कि मितीति॒ किम् द॑श॒मेन॑ । \newline
15. किम् द॑श॒मेन॑ दश॒मेन॒ किम् किम् द॑श॒मेनेतीति॑ दश॒मेन॒ किम् किम् द॑श॒मेनेति॑ । \newline
16. द॒श॒मेनेतीति॑ दश॒मेन॑ दश॒मेनेति॒ दशा᳚क्षरा॒म् दशा᳚क्षरा॒ मिति॑ दश॒मेन॑ दश॒मेनेति॒ दशा᳚क्षराम् । \newline
17. इति॒ दशा᳚क्षरा॒म् दशा᳚क्षरा॒ मितीति॒ दशा᳚क्षरां ॅवि॒राजं॑ ॅवि॒राज॒म् दशा᳚क्षरा॒ मितीति॒ दशा᳚क्षरां ॅवि॒राज᳚म् । \newline
18. दशा᳚क्षरां ॅवि॒राजं॑ ॅवि॒राज॒म् दशा᳚क्षरा॒म् दशा᳚क्षरां ॅवि॒राज॒ मितीति॑ वि॒राज॒म् दशा᳚क्षरा॒म् दशा᳚क्षरां ॅवि॒राज॒ मिति॑ । \newline
19. दशा᳚क्षरा॒मिति॒ दश॑ - अ॒क्ष॒रा॒म् । \newline
20. वि॒राज॒ मितीति॑ वि॒राजं॑ ॅवि॒राज॒ मिति॒ किम् कि मिति॑ वि॒राजं॑ ॅवि॒राज॒ मिति॒ किम् । \newline
21. वि॒राज॒मिति॑ वि - राज᳚म् । \newline
22. इति॒ किम् कि मितीति॒ कि मे॑काद॒शे नै॑काद॒शेन॒ कि मितीति॒ कि मे॑काद॒शेन॑ । \newline
23. कि मे॑काद॒शे नै॑काद॒शेन॒ किम् कि मे॑काद॒शेने तीत्ये॑काद॒शेन॒ किम् कि मे॑काद॒शेनेति॑ । \newline
24. ए॒का॒द॒शेने तीत्ये॑काद॒शे नै॑काद॒शेने त्येका॑दशाक्षरा॒ मेका॑दशाक्षरा॒ मित्ये॑काद॒शे नै॑काद॒शेने त्येका॑दशाक्षराम् । \newline
25. इत्येका॑दशाक्षरा॒ मेका॑दशाक्षरा॒ मिती त्येका॑दशाक्षराम् त्रि॒ष्टुभ॑म् त्रि॒ष्टुभ॒ मेका॑दशाक्षरा॒ मिती त्येका॑दशाक्षराम् त्रि॒ष्टुभ᳚म् । \newline
26. एका॑दशाक्षराम् त्रि॒ष्टुभ॑म् त्रि॒ष्टुभ॒ मेका॑दशाक्षरा॒ मेका॑दशाक्षराम् त्रि॒ष्टुभ॒ मितीति॑ त्रि॒ष्टुभ॒ मेका॑दशाक्षरा॒ मेका॑दशाक्षराम् त्रि॒ष्टुभ॒ मिति॑ । \newline
27. एका॑दशाक्षरा॒मित्येका॑दश - अ॒क्ष॒रा॒म् । \newline
28. त्रि॒ष्टुभ॒ मितीति॑ त्रि॒ष्टुभ॑म् त्रि॒ष्टुभ॒ मिति॒ किम् कि मिति॑ त्रि॒ष्टुभ॑म् त्रि॒ष्टुभ॒ मिति॒ किम् । \newline
29. इति॒ किम् कि मितीति॒ किम् द्वा॑द॒शेन॑ द्वाद॒शेन॒ कि मितीति॒ किम् द्वा॑द॒शेन॑ । \newline
30. किम् द्वा॑द॒शेन॑ द्वाद॒शेन॒ किम् किम् द्वा॑द॒शेनेतीति॑ द्वाद॒शेन॒ किम् किम् द्वा॑द॒शेनेति॑ । \newline
31. द्वा॒द॒शेनेतीति॑ द्वाद॒शेन॑ द्वाद॒शेनेति॒ द्वाद॑शाक्षरा॒म् द्वाद॑शाक्षरा॒ मिति॑ द्वाद॒शेन॑ द्वाद॒शेनेति॒ द्वाद॑शाक्षराम् । \newline
32. इति॒ द्वाद॑शाक्षरा॒म् द्वाद॑शाक्षरा॒ मितीति॒ द्वाद॑शाक्षरा॒म् जग॑ती॒म् जग॑ती॒म् द्वाद॑शाक्षरा॒ मितीति॒ द्वाद॑शाक्षरा॒म् जग॑तीम् । \newline
33. द्वाद॑शाक्षरा॒म् जग॑ती॒म् जग॑ती॒म् द्वाद॑शाक्षरा॒म् द्वाद॑शाक्षरा॒म् जग॑ती॒ मितीति॒ जग॑ती॒म् द्वाद॑शाक्षरा॒म् द्वाद॑शाक्षरा॒म् जग॑ती॒ मिति॑ । \newline
34. द्वाद॑शाक्षरा॒मिति॒ द्वाद॑श - अ॒क्ष॒रा॒म् । \newline
35. जग॑ती॒ मितीति॒ जग॑ती॒म् जग॑ती॒ मित्ये॒ताव॑ दे॒ताव॒ दिति॒ जग॑ती॒म् जग॑ती॒ मित्ये॒ताव॑त् । \newline
36. इत्ये॒ताव॑ दे॒ताव॒ दिती त्ये॒ताव॒द् वै वा ए॒ताव॒ दिती त्ये॒ताव॒द् वै । \newline
37. ए॒ताव॒द् वै वा ए॒ताव॑ दे॒ताव॒द् वा अ॑स्त्यस्ति॒ वा ए॒ताव॑ दे॒ताव॒द् वा अ॑स्ति । \newline
38. वा अ॑स्त्यस्ति॒ वै वा अ॑स्ति॒ याव॒द् याव॑दस्ति॒ वै वा अ॑स्ति॒ याव॑त् । \newline
39. अ॒स्ति॒ याव॒द् याव॑ दस्त्यस्ति॒ याव॑ दे॒त दे॒तद् याव॑ दस्त्यस्ति॒ याव॑ दे॒तत् । \newline
40. याव॑ दे॒त दे॒तद् याव॒द् याव॑ दे॒तद् याव॒द् याव॑ दे॒तद् याव॒द् याव॑ दे॒तद् याव॑त् । \newline
41. ए॒तद् याव॒द् याव॑ दे॒त दे॒तद् याव॑ दे॒वैव याव॑ दे॒त दे॒तद् याव॑दे॒व । \newline
42. याव॑ दे॒वैव याव॒द् याव॑ दे॒वास्त्य स्त्ये॒व याव॒द् याव॑ दे॒वास्ति॑ । \newline
43. ए॒वा स्त्य स्त्ये॒वैवास्ति॒ तत् तद स्त्ये॒वैवास्ति॒ तत् । \newline
44. अस्ति॒ तत् तद स्त्यस्ति॒ तदे॑षा मेषा॒म् तद स्त्यस्ति॒ तदे॑षाम् । \newline
45. तदे॑षा मेषा॒म् तत् तदे॑षां ॅवृङ्क्ते वृङ्क्त एषा॒म् तत् तदे॑षां ॅवृङ्क्ते । \newline
46. ए॒षां॒ ॅवृ॒ङ्क्ते॒ वृ॒ङ्क्त॒ ए॒षा॒ मे॒षां॒ ॅवृ॒ङ्क्ते॒ । \newline
47. वृ॒ङ्क्ते॒ इति॑ वृङ्क्ते । \newline
\pagebreak
\markright{ TS 7.3.3.1  \hfill https://www.vedavms.in \hfill}

\section{ TS 7.3.3.1 }

\textbf{TS 7.3.3.1 } \newline
\textbf{Samhita Paata} \newline

ए॒ष वा आ॒प्तो द्वा॑दशा॒हो यत् त्र॑योदशरा॒त्रः स॑मा॒नꣳ ह्ये॑तदह॒र्यत् प्रा॑य॒णीय॑श्चोदय॒नीय॑श्च॒ त्र्य॑तिरात्रो भवति॒ त्रय॑ इ॒मे लो॒का ए॒षां ॅलो॒काना॒माप्त्यै᳚ प्रा॒णो वै प्र॑थ॒मो॑ऽतिरा॒त्रो व्या॒नो द्वि॒तीयो॑ ऽपा॒नस्तृ॒तीयः॑ प्राणापानो-दा॒नेष्वे॒वान्नाद्ये॒ प्रति॑ तिष्ठन्ति॒ सर्व॒मायु॑र्यन्ति॒ य ए॒वं ॅवि॒द्वाꣳस॑-स्त्रयोदशरा॒त्र-मास॑ते॒ तदा॑हु॒र्वाग्वा ए॒षा वित॑ता॒ - [  ] \newline

\textbf{Pada Paata} \newline

ए॒षः । वै । आ॒प्तः । द्वा॒द॒शा॒ह इति॑ द्वादश - अ॒हः । यत् । त्र॒यो॒द॒श॒रा॒त्र इति॑ त्रयोदश - रा॒त्रः । स॒मा॒नम् । हि । ए॒तत् । अहः॑ । यत् । प्रा॒य॒णीय॒ इति॑ प्र - अ॒य॒नीयः॑ । च॒ । उ॒द॒य॒नीय॒ इत्यु॑त् - अ॒य॒नीयः॑ । च॒ । त्र्य॑तिरात्र॒ इति॒ त्रि - अ॒ति॒रा॒त्रः॒ । भ॒व॒ति॒ । त्रयः॑ । इ॒मे । लो॒काः । ए॒षाम् । लो॒काना᳚म् । आप्त्यै᳚ । प्रा॒ण इति॑ प्र - अ॒नः । वै । प्र॒थ॒मः । अ॒ति॒रा॒त्र इत्य॑ति - रा॒त्रः । व्या॒न इति॑ वि - अ॒नः । द्वि॒तीयः॑ । अ॒पा॒न इत्य॑प-अ॒नः । तृ॒तीयः॑ । प्रा॒णा॒पा॒नो॒दा॒नेष्विति॑ प्राणापान - उ॒दा॒नेषु॑ । ए॒व । अ॒न्नाद्य॒ इत्य॑न्न - अद्ये᳚ । प्रतीति॑ । ति॒ष्ठ॒न्ति॒ । सर्व᳚म् । आयुः॑ । य॒न्ति॒ । ये । ए॒वम् । वि॒द्वाꣳसः॑ । त्र॒यो॒द॒श॒रा॒त्रमिति॑ त्रयोदश - रा॒त्रम् । आस॑ते । तत् । आ॒हुः॒ । वाक् । वै । ए॒षा । वित॒तेति॒ वि - त॒ता॒ ।  \newline


\textbf{Krama Paata} \newline

ए॒ष वै । वा आ॒प्तः । आ॒प्तो द्वा॑दशा॒हः । द्वा॒द॒शा॒हो यत् । द्वा॒द॒शा॒ह इति॑ द्वादश - अ॒हः । यत् त्र॑योदशरा॒त्रः । त्र॒यो॒द॒श॒रा॒त्रः स॑मा॒नम् । त्र॒यो॒द॒श॒रा॒त्र इति॑ त्रयोदश - रा॒त्रः । स॒मा॒नꣳ हि । ह्ये॑तत् । ए॒तदहः॑ । अह॒र् यत् । यत् प्रा॑य॒णीयः॑ । प्रा॒य॒णीय॑श्च । प्रा॒य॒णीय॒ इति॑ प्र - अ॒य॒नीयः॑ । चो॒द॒य॒नीयः॑ । उ॒द॒य॒नीय॑श्च । उ॒द॒य॒नीय॒ इत्यु॑त् - अ॒य॒नीयः॑ । च॒ त्र्य॑तिरात्रः । त्र्य॑तिरात्रो भवति । त्र्य॑तिरात्र॒ इति॒ त्रि - अ॒ति॒रा॒त्रः॒ । भ॒व॒ति॒ त्रयः॑ । त्रय॑ इ॒मे । इ॒मे लो॒काः । लो॒का ए॒षाम् । ए॒षाम् ॅलो॒काना᳚म् । लो॒काना॒माप्त्यै᳚ । आप्त्यै᳚ प्रा॒णः । प्रा॒णो वै । प्रा॒ण इति॑ प्र - अ॒नः । वै प्र॑थ॒मः । प्र॒थ॒मो॑ऽतिरा॒त्रः । अ॒ति॒रा॒त्रो व्या॒नः । अ॒ति॒रा॒त्र इत्य॑ति - रा॒त्रः । व्या॒नो द्वि॒तीयः॑ । व्या॒न इति॑ वि - अ॒नः । द्वि॒तीयो॑ऽपा॒नः । अ॒पा॒नस्तृ॒तीयः॑ । अ॒पा॒न इत्य॑प - अ॒नः । तृ॒तीयः॑ प्राणापानोदा॒नेषु॑ । प्रा॒णा॒पा॒,नो॒दा॒नेष्वे॒व । प्रा॒णा॒पा॒नो॒दा॒नेष्विति॑ प्राणापान - उ॒दा॒नेषु॑ । ए॒वान्नाद्ये᳚ । अ॒न्नाद्ये॒ प्रति॑ । अ॒न्नाद्य॒ इत्य॑न्न - अद्ये᳚ । प्रति॑ तिष्ठन्ति । ति॒ष्ठ॒न्ति॒ सर्व᳚म् । सर्व॒मायुः॑ । आयु॑र् यन्ति । य॒न्ति॒ ये । य ए॒वम् । ए॒वम् ॅवि॒द्वाꣳसः॑ । वि॒द्वाꣳस॑स्त्रयोदशरा॒त्रम् । त्र॒यो॒द॒श॒रा॒त्रमास॑ते । त्र॒यो॒द॒श॒रा॒त्रमिति॑ त्रयोदश - रा॒त्रम् । आस॑ते॒ तत् । तदा॑हुः । आ॒हु॒र् वाक् । वाग् वै । वा ए॒षा । ए॒षा वित॑ता ( ) । वित॑ता॒ यत् । वित॒तेति॒ वि - त॒ता॒ \newline

\textbf{Jatai Paata} \newline

1. ए॒ष वै वा ए॒ष ए॒ष वै । \newline
2. वा आ॒प्त आ॒प्तो वै वा आ॒प्तः । \newline
3. आ॒प्तो द्वा॑दशा॒हो द्वा॑दशा॒ह आ॒प्त आ॒प्तो द्वा॑दशा॒हः । \newline
4. द्वा॒द॒शा॒हो यद् यद् द्वा॑दशा॒हो द्वा॑दशा॒हो यत् । \newline
5. द्वा॒द॒शा॒ह इति॑ द्वादश - अ॒हः । \newline
6. यत् त्र॑योदशरा॒त्र स्त्र॑योदशरा॒त्रो यद् यत् त्र॑योदशरा॒त्रः । \newline
7. त्र॒यो॒द॒श॒रा॒त्रः स॑मा॒नꣳ स॑मा॒नम् त्र॑योदशरा॒त्र स्त्र॑योदशरा॒त्रः स॑मा॒नम् । \newline
8. त्र॒यो॒द॒श॒रा॒त्र इति॑ त्रयोदश - रा॒त्रः । \newline
9. स॒मा॒नꣳ हि हि स॑मा॒नꣳ स॑मा॒नꣳ हि । \newline
10. ह्ये॑त दे॒तद्धि ह्ये॑तत् । \newline
11. ए॒त दह॒ रह॑ रे॒त दे॒त दहः॑ । \newline
12. अह॒र् यद् यदह॒ रह॒र् यत् । \newline
13. यत् प्रा॑य॒णीयः॑ प्राय॒णीयो॒ यद् यत् प्रा॑य॒णीयः॑ । \newline
14. प्रा॒य॒णीय॑ श्च च प्राय॒णीयः॑ प्राय॒णीय॑ श्च । \newline
15. प्रा॒य॒णीय॒ इति॑ प्र - अ॒य॒नीयः॑ । \newline
16. चो॒द॒य॒नीय॑ उदय॒नीय॑ श्च चोदय॒नीयः॑ । \newline
17. उ॒द॒य॒नीय॑ श्च चोदय॒नीय॑ उदय॒नीय॑ श्च । \newline
18. उ॒द॒य॒नीय॒ इत्यु॑त् - अ॒य॒नीयः॑ । \newline
19. च॒ त्र्य॑तिरात्र॒ स्त्र्य॑तिरात्र श्च च॒ त्र्य॑तिरात्रः । \newline
20. त्र्य॑तिरात्रो भवति भवति॒ त्र्य॑तिरात्र॒ स्त्र्य॑तिरात्रो भवति । \newline
21. त्र्य॑तिरात्र॒ इति॒ त्रि - अ॒ति॒रा॒त्रः॒ । \newline
22. भ॒व॒ति॒ त्रय॒ स्त्रयो॑ भवति भवति॒ त्रयः॑ । \newline
23. त्रय॑ इ॒म इ॒मे त्रय॒ स्त्रय॑ इ॒मे । \newline
24. इ॒मे लो॒का लो॒का इ॒म इ॒मे लो॒काः । \newline
25. लो॒का ए॒षा मे॒षाम् ॅलो॒का लो॒का ए॒षाम् । \newline
26. ए॒षाम् ॅलो॒काना᳚म् ॅलो॒काना॑ मे॒षा मे॒षाम् ॅलो॒काना᳚म् । \newline
27. लो॒काना॒ माप्त्या॒ आप्त्यै॑ लो॒काना᳚म् ॅलो॒काना॒ माप्त्यै᳚ । \newline
28. आप्त्यै᳚ प्रा॒णः प्रा॒ण आप्त्या॒ आप्त्यै᳚ प्रा॒णः । \newline
29. प्रा॒णो वै वै प्रा॒णः प्रा॒णो वै । \newline
30. प्रा॒ण इति॑ प्र - अ॒नः । \newline
31. वै प्र॑थ॒मः प्र॑थ॒मो वै वै प्र॑थ॒मः । \newline
32. प्र॒थ॒मो॑ ऽतिरा॒त्रो॑ ऽतिरा॒त्रः प्र॑थ॒मः प्र॑थ॒मो॑ ऽतिरा॒त्रः । \newline
33. अ॒ति॒रा॒त्रो व्या॒नो व्या॒नो॑ ऽतिरा॒त्रो॑ ऽतिरा॒त्रो व्या॒नः । \newline
34. अ॒ति॒रा॒त्र इत्य॑ति - रा॒त्रः । \newline
35. व्या॒नो द्वि॒तीयो᳚ द्वि॒तीयो᳚ व्या॒नो व्या॒नो द्वि॒तीयः॑ । \newline
36. व्या॒न इति॑ वि - अ॒नः । \newline
37. द्वि॒तीयो॑ ऽपा॒नो॑ ऽपा॒नो द्वि॒तीयो᳚ द्वि॒तीयो॑ ऽपा॒नः । \newline
38. अ॒पा॒न स्तृ॒तीय॑ स्तृ॒तीयो॑ ऽपा॒नो॑ ऽपा॒न स्तृ॒तीयः॑ । \newline
39. अ॒पा॒न इत्य॑प - अ॒नः । \newline
40. तृ॒तीयः॑ प्राणापानोदा॒नेषु॑ प्राणापानोदा॒नेषु॑ तृ॒तीय॑ स्तृ॒तीयः॑ प्राणापानोदा॒नेषु॑ । \newline
41. प्रा॒णा॒पा॒नो॒दा॒ने ष्वे॒वैव प्रा॑णापानोदा॒नेषु॑ प्राणापानोदा॒ने ष्वे॒व । \newline
42. प्रा॒णा॒पा॒नो॒दा॒नेष्विति॑ प्राणापान - उ॒दा॒नेषु॑ । \newline
43. ए॒वा न्नाद्ये॒ ऽन्नाद्य॑ ए॒वैवा न्नाद्ये᳚ । \newline
44. अ॒न्नाद्ये॒ प्रति॒ प्रत्य॒न्नाद्ये॒ ऽन्नाद्ये॒ प्रति॑ । \newline
45. अ॒न्नाद्य॒ इत्य॑न्न - अद्ये᳚ । \newline
46. प्रति॑ तिष्ठन्ति तिष्ठन्ति॒ प्रति॒ प्रति॑ तिष्ठन्ति । \newline
47. ति॒ष्ठ॒न्ति॒ सर्वꣳ॒॒ सर्व॑म् तिष्ठन्ति तिष्ठन्ति॒ सर्व᳚म् । \newline
48. सर्व॒ मायु॒ रायुः॒ सर्वꣳ॒॒ सर्व॒ मायुः॑ । \newline
49. आयु॑र् यन्ति य॒न्त्यायु॒ रायु॑र् यन्ति । \newline
50. य॒न्ति॒ ये ये य॑न्ति यन्ति॒ ये । \newline
51. य ए॒व मे॒वं ॅये य ए॒वम् । \newline
52. ए॒वं ॅवि॒द्वाꣳसो॑ वि॒द्वाꣳस॑ ए॒व मे॒वं ॅवि॒द्वाꣳसः॑ । \newline
53. वि॒द्वाꣳस॑ स्त्रयोदशरा॒त्रम् त्र॑योदशरा॒त्रं ॅवि॒द्वाꣳसो॑ वि॒द्वाꣳस॑ स्त्रयोदशरा॒त्रम् । \newline
54. त्र॒यो॒द॒श॒रा॒त्र मास॑त॒ आस॑ते त्रयोदशरा॒त्रम् त्र॑योदशरा॒त्र मास॑ते । \newline
55. त्र॒यो॒द॒श॒रा॒त्रमिति॑ त्रयोदश - रा॒त्रम् । \newline
56. आस॑ते॒ तत् तदास॑त॒ आस॑ते॒ तत् । \newline
57. तदा॑हु राहु॒ स्तत् तदा॑हुः । \newline
58. आ॒हु॒र् वाग् वागा॑हु राहु॒र् वाक् । \newline
59. वाग् वै वै वाग् वाग् वै । \newline
60. वा ए॒षैषा वै वा ए॒षा । \newline
61. ए॒षा वित॑ता॒ वित॑तै॒षैषा वित॑ता । \newline
62. वित॑ता॒ यद् यद् वित॑ता॒ वित॑ता॒ यत् । \newline
63. वित॒तेति॒ वि - त॒ता॒ । \newline

\textbf{Ghana Paata } \newline

1. ए॒ष वै वा ए॒ष ए॒ष वा आ॒प्त आ॒प्तो वा ए॒ष ए॒ष वा आ॒प्तः । \newline
2. वा आ॒प्त आ॒प्तो वै वा आ॒प्तो द्वा॑दशा॒हो द्वा॑दशा॒ह आ॒प्तो वै वा आ॒प्तो द्वा॑दशा॒हः । \newline
3. आ॒प्तो द्वा॑दशा॒हो द्वा॑दशा॒ह आ॒प्त आ॒प्तो द्वा॑दशा॒हो यद् यद् द्वा॑दशा॒ह आ॒प्त आ॒प्तो द्वा॑दशा॒हो यत् । \newline
4. द्वा॒द॒शा॒हो यद् यद् द्वा॑दशा॒हो द्वा॑दशा॒हो यत् त्र॑योदशरा॒त्र स्त्र॑योदशरा॒त्रो यद् द्वा॑दशा॒हो द्वा॑दशा॒हो यत् त्र॑योदशरा॒त्रः । \newline
5. द्वा॒द॒शा॒ह इति॑ द्वादश - अ॒हः । \newline
6. यत् त्र॑योदशरा॒त्र स्त्र॑योदशरा॒त्रो यद् यत् त्र॑योदशरा॒त्रः स॑मा॒नꣳ स॑मा॒नम् त्र॑योदशरा॒त्रो यद् यत् त्र॑योदशरा॒त्रः स॑मा॒नम् । \newline
7. त्र॒यो॒द॒श॒रा॒त्रः स॑मा॒नꣳ स॑मा॒नम् त्र॑योदशरा॒त्र स्त्र॑योदशरा॒त्रः स॑मा॒नꣳ हि हि स॑मा॒नम् त्र॑योदशरा॒त्र स्त्र॑योदशरा॒त्रः स॑मा॒नꣳ हि । \newline
8. त्र॒यो॒द॒श॒रा॒त्र इति॑ त्रयोदश - रा॒त्रः । \newline
9. स॒मा॒नꣳ हि हि स॑मा॒नꣳ स॑मा॒नꣳ ह्ये॑त दे॒तद्धि स॑मा॒नꣳ स॑मा॒नꣳ ह्ये॑तत् । \newline
10. ह्ये॑त दे॒तद्धि ह्ये॑त दह॒ रह॑ रे॒तद्धि ह्ये॑त दहः॑ । \newline
11. ए॒त दह॒ रह॑ रे॒त दे॒त दह॒र् यद् यदह॑ रे॒त दे॒त दह॒र् यत् । \newline
12. अह॒र् यद् यदह॒ रह॒र् यत् प्रा॑य॒णीयः॑ प्राय॒णीयो॒ यदह॒ रह॒र् यत् प्रा॑य॒णीयः॑ । \newline
13. यत् प्रा॑य॒णीयः॑ प्राय॒णीयो॒ यद् यत् प्रा॑य॒णीय॑श्च च प्राय॒णीयो॒ यद् यत् प्रा॑य॒णीय॑श्च । \newline
14. प्रा॒य॒णीय॑श्च च प्राय॒णीयः॑ प्राय॒णीय॑श्चो दय॒नीय॑ उदय॒नीय॑श्च प्राय॒णीयः॑ प्राय॒णीय॑श्चो दय॒नीयः॑ । \newline
15. प्रा॒य॒णीय॒ इति॑ प्र - अ॒य॒नीयः॑ । \newline
16. चो॒द॒य॒नीय॑ उदय॒नीय॑ श्च चोदय॒नीय॑ श्च चोदय॒नीय॑ श्च चोदय॒नीय॑ श्च । \newline
17. उ॒द॒य॒नीय॑ श्च चोदय॒नीय॑ उदय॒नीय॑ श्च॒ त्र्य॑तिरात्र॒ स्त्र्य॑तिरात्र श्चोदय॒नीय॑ उदय॒नीय॑श्च॒ त्र्य॑तिरात्रः । \newline
18. उ॒द॒य॒नीय॒ इत्यु॑त् - अ॒य॒नीयः॑ । \newline
19. च॒ त्र्य॑तिरात्र॒ स्त्र्य॑तिरात्र श्च च॒ त्र्य॑तिरात्रो भवति भवति॒ त्र्य॑तिरात्र श्च च॒ त्र्य॑तिरात्रो भवति । \newline
20. त्र्य॑तिरात्रो भवति भवति॒ त्र्य॑तिरात्र॒ स्त्र्य॑तिरात्रो भवति॒ त्रय॒ स्त्रयो॑ भवति॒ त्र्य॑तिरात्र॒ स्त्र्य॑तिरात्रो भवति॒ त्रयः॑ । \newline
21. त्र्य॑तिरात्र॒ इति॒ त्रि - अ॒ति॒रा॒त्रः॒ । \newline
22. भ॒व॒ति॒ त्रय॒ स्त्रयो॑ भवति भवति॒ त्रय॑ इ॒म इ॒मे त्रयो॑ भवति भवति॒ त्रय॑ इ॒मे । \newline
23. त्रय॑ इ॒म इ॒मे त्रय॒ स्त्रय॑ इ॒मे लो॒का लो॒का इ॒मे त्रय॒ स्त्रय॑ इ॒मे लो॒काः । \newline
24. इ॒मे लो॒का लो॒का इ॒म इ॒मे लो॒का ए॒षा मे॒षाम् ॅलो॒का इ॒म इ॒मे लो॒का ए॒षाम् । \newline
25. लो॒का ए॒षा मे॒षाम् ॅलो॒का लो॒का ए॒षाम् ॅलो॒काना᳚म् ॅलो॒काना॑ मे॒षाम् ॅलो॒का लो॒का ए॒षाम् ॅलो॒काना᳚म् । \newline
26. ए॒षाम् ॅलो॒काना᳚म् ॅलो॒काना॑ मे॒षा मे॒षाम् ॅलो॒काना॒ माप्त्या॒ आप्त्यै॑ लो॒काना॑ मे॒षा मे॒षाम् ॅलो॒काना॒ माप्त्यै᳚ । \newline
27. लो॒काना॒ माप्त्या॒ आप्त्यै॑ लो॒काना᳚म् ॅलो॒काना॒ माप्त्यै᳚ प्रा॒णः प्रा॒ण आप्त्यै॑ लो॒काना᳚म् ॅलो॒काना॒ माप्त्यै᳚ प्रा॒णः । \newline
28. आप्त्यै᳚ प्रा॒णः प्रा॒ण आप्त्या॒ आप्त्यै᳚ प्रा॒णो वै वै प्रा॒ण आप्त्या॒ आप्त्यै᳚ प्रा॒णो वै । \newline
29. प्रा॒णो वै वै प्रा॒णः प्रा॒णो वै प्र॑थ॒मः प्र॑थ॒मो वै प्रा॒णः प्रा॒णो वै प्र॑थ॒मः । \newline
30. प्रा॒ण इति॑ प्र - अ॒नः । \newline
31. वै प्र॑थ॒मः प्र॑थ॒मो वै वै प्र॑थ॒मो॑ ऽतिरा॒त्रो॑ ऽतिरा॒त्रः प्र॑थ॒मो वै वै प्र॑थ॒मो॑ ऽतिरा॒त्रः । \newline
32. प्र॒थ॒मो॑ ऽतिरा॒त्रो॑ ऽतिरा॒त्रः प्र॑थ॒मः प्र॑थ॒मो॑ ऽतिरा॒त्रो व्या॒नो व्या॒नो॑ ऽतिरा॒त्रः प्र॑थ॒मः प्र॑थ॒मो॑ ऽतिरा॒त्रो व्या॒नः । \newline
33. अ॒ति॒रा॒त्रो व्या॒नो व्या॒नो॑ ऽतिरा॒त्रो॑ ऽतिरा॒त्रो व्या॒नो द्वि॒तीयो᳚ द्वि॒तीयो᳚ व्या॒नो॑ ऽतिरा॒त्रो॑ ऽतिरा॒त्रो व्या॒नो द्वि॒तीयः॑ । \newline
34. अ॒ति॒रा॒त्र इत्य॑ति - रा॒त्रः । \newline
35. व्या॒नो द्वि॒तीयो᳚ द्वि॒तीयो᳚ व्या॒नो व्या॒नो द्वि॒तीयो॑ ऽपा॒नो॑ ऽपा॒नो द्वि॒तीयो᳚ व्या॒नो व्या॒नो द्वि॒तीयो॑ ऽपा॒नः । \newline
36. व्या॒न इति॑ वि - अ॒नः । \newline
37. द्वि॒तीयो॑ ऽपा॒नो॑ ऽपा॒नो द्वि॒तीयो᳚ द्वि॒तीयो॑ ऽपा॒न स्तृ॒तीय॑ स्तृ॒तीयो॑ ऽपा॒नो द्वि॒तीयो᳚ द्वि॒तीयो॑ ऽपा॒न स्तृ॒तीयः॑ । \newline
38. अ॒पा॒न स्तृ॒तीय॑ स्तृ॒तीयो॑ ऽपा॒नो॑ ऽपा॒न स्तृ॒तीयः॑ प्राणापानोदा॒नेषु॑ प्राणापानोदा॒नेषु॑ तृ॒तीयो॑ ऽपा॒नो॑ ऽपा॒न स्तृ॒तीयः॑ प्राणापानोदा॒नेषु॑ । \newline
39. अ॒पा॒न इत्य॑प - अ॒नः । \newline
40. तृ॒तीयः॑ प्राणापानोदा॒नेषु॑ प्राणापानोदा॒नेषु॑ तृ॒तीय॑ स्तृ॒तीयः॑ प्राणापानोदा॒ने ष्वे॒वैव प्रा॑णापानोदा॒नेषु॑ तृ॒तीय॑ स्तृ॒तीयः॑ प्राणापानोदा॒ने ष्वे॒व । \newline
41. प्रा॒णा॒पा॒नो॒दा॒ने ष्वे॒वैव प्रा॑णापानोदा॒नेषु॑ प्राणापानोदा॒ने ष्वे॒वा न्नाद्ये॒ ऽन्नाद्य॑ ए॒व प्रा॑णापानोदा॒नेषु॑ प्राणापानोदा॒ने ष्वे॒वा न्नाद्ये᳚ । \newline
42. प्रा॒णा॒पा॒नो॒दा॒नेष्विति॑ प्राणापान - उ॒दा॒नेषु॑ । \newline
43. ए॒वान्नाद्ये॒ ऽन्नाद्य॑ ए॒वै वान्नाद्ये॒ प्रति॒ प्रत्य॒न्नाद्य॑ ए॒वै वान्नाद्ये॒ प्रति॑ । \newline
44. अ॒न्नाद्ये॒ प्रति॒ प्रत्य॒न्नाद्ये॒ ऽन्नाद्ये॒ प्रति॑ तिष्ठन्ति तिष्ठन्ति॒ प्रत्य॒न्नाद्ये॒ ऽन्नाद्ये॒ प्रति॑ तिष्ठन्ति । \newline
45. अ॒न्नाद्य॒ इत्य॑न्न - अद्ये᳚ । \newline
46. प्रति॑ तिष्ठन्ति तिष्ठन्ति॒ प्रति॒ प्रति॑ तिष्ठन्ति॒ सर्वꣳ॒॒ सर्व॑म् तिष्ठन्ति॒ प्रति॒ प्रति॑ तिष्ठन्ति॒ सर्व᳚म् । \newline
47. ति॒ष्ठ॒न्ति॒ सर्वꣳ॒॒ सर्व॑म् तिष्ठन्ति तिष्ठन्ति॒ सर्व॒ मायु॒ रायुः॒ सर्व॑म् तिष्ठन्ति तिष्ठन्ति॒ सर्व॒ मायुः॑ । \newline
48. सर्व॒ मायु॒ रायुः॒ सर्वꣳ॒॒ सर्व॒ मायु॑र् यन्ति य॒न्त्यायुः॒ सर्वꣳ॒॒ सर्व॒ मायु॑र् यन्ति । \newline
49. आयु॑र् यन्ति य॒न्त्यायु॒ रायु॑र् यन्ति॒ ये ये य॒न्त्यायु॒ रायु॑र् यन्ति॒ ये । \newline
50. य॒न्ति॒ ये ये य॑न्ति यन्ति॒ य ए॒व मे॒वं ॅये य॑न्ति यन्ति॒ य ए॒वम् । \newline
51. य ए॒व मे॒वं ॅये य ए॒वं ॅवि॒द्वाꣳसो॑ वि॒द्वाꣳस॑ ए॒वं ॅये य ए॒वं ॅवि॒द्वाꣳसः॑ । \newline
52. ए॒वं ॅवि॒द्वाꣳसो॑ वि॒द्वाꣳस॑ ए॒व मे॒वं ॅवि॒द्वाꣳस॑ स्त्रयोदशरा॒त्रम् त्र॑योदशरा॒त्रं ॅवि॒द्वाꣳस॑ ए॒व मे॒वं ॅवि॒द्वाꣳस॑ स्त्रयोदशरा॒त्रम् । \newline
53. वि॒द्वाꣳस॑ स्त्रयोदशरा॒त्रम् त्र॑योदशरा॒त्रं ॅवि॒द्वाꣳसो॑ वि॒द्वाꣳस॑ स्त्रयोदशरा॒त्र मास॑त॒ आस॑ते त्रयोदशरा॒त्रं ॅवि॒द्वाꣳसो॑ वि॒द्वाꣳस॑ स्त्रयोदशरा॒त्र मास॑ते । \newline
54. त्र॒यो॒द॒श॒रा॒त्र मास॑त॒ आस॑ते त्रयोदशरा॒त्रम् त्र॑योदशरा॒त्र मास॑ते॒ तत् तदास॑ते त्रयोदशरा॒त्रम् त्र॑योदशरा॒त्र मास॑ते॒ तत् । \newline
55. त्र॒यो॒द॒श॒रा॒त्रमिति॑ त्रयोदश - रा॒त्रम् । \newline
56. आस॑ते॒ तत् तदा स॑त॒ आस॑ते॒ तदा॑हु राहु॒ स्तदास॑त॒ आस॑ते॒ तदा॑हुः । \newline
57. तदा॑हु राहु॒ स्तत् तदा॑हु॒र् वाग् वागा॑हु॒ स्तत् तदा॑हु॒र् वाक् । \newline
58. आ॒हु॒र् वाग् वागा॑हु राहु॒र् वाग् वै वै वागा॑हु राहु॒र् वाग् वै । \newline
59. वाग् वै वै वाग् वाग् वा ए॒षैषा वै वाग् वाग् वा ए॒षा । \newline
60. वा ए॒षैषा वै वा ए॒षा वित॑ता॒ वित॑ तै॒षा वै वा ए॒षा वित॑ता । \newline
61. ए॒षा वित॑ता॒ वित॑ तै॒षैषा वित॑ता॒ यद् यद् वित॑ तै॒षैषा वित॑ता॒ यत् । \newline
62. वित॑ता॒ यद् यद् वित॑ता॒ वित॑ता॒ यद् द्वा॑दशा॒हो द्वा॑दशा॒हो यद् वित॑ता॒ वित॑ता॒ यद् द्वा॑दशा॒हः । \newline
63. वित॒तेति॒ वि - त॒ता॒ । \newline
\pagebreak
\markright{ TS 7.3.3.2  \hfill https://www.vedavms.in \hfill}

\section{ TS 7.3.3.2 }

\textbf{TS 7.3.3.2 } \newline
\textbf{Samhita Paata} \newline

यद् द्वा॑दशा॒हस्तां ॅविच्छि॑न्द्यु॒र्यन्मद्ध्ये॑ ऽतिरा॒त्रं कु॒र्युरु॑प॒दासु॑का गृ॒हप॑ते॒र्वाख् स्या॑-दु॒परि॑ष्टाच्छन्दो॒मानां᳚ महाव्र॒तं कु॑र्वन्ति॒ संत॑तामे॒व वाच॒मव॑ रुन्ध॒तेऽनु॑पदासुका गृ॒हप॑ते॒र्वाग्-भ॑वति प॒शवो॒ वै छ॑न्दो॒मा अन्नं॑ महाव्र॒तं ॅयदु॒परि॑ष्टाच्छन्दो॒मानां᳚ महाव्र॒तं कु॒र्वन्ति॑ प॒शुषु॑ चै॒वान्नाद्ये॑ च॒ प्रति॑ तिष्ठन्ति ॥ \newline

\textbf{Pada Paata} \newline

यत् । द्वा॒द॒शा॒ह इति॑ द्वादश - अ॒हः । ताम् । वीति॑ । छि॒न्द्युः॒ । यत् । मद्ध्ये᳚ । अ॒ति॒रा॒त्रमित्य॑ति-रा॒त्रम् । कु॒र्युः । उ॒प॒दासु॒केत्यु॑प-दासु॑का । गृ॒हप॑ते॒रिति॑ गृ॒ह - प॒तेः॒ । वाक् । स्या॒त् । उ॒परि॑ष्टात् । छ॒न्दो॒माना॒मिति॑ छन्दः - माना᳚म् । म॒हा॒व्र॒तमिति॑ महा - व्र॒तम् । कु॒र्व॒न्ति॒ । संत॑ता॒मिति॒ सं-त॒ता॒म् । ए॒व । वाच᳚म् । अवेति॑ । रु॒न्ध॒ते॒ । अनु॑पदासु॒केत्यनु॑प - दा॒सु॒का॒ । गृ॒हप॑ते॒रिति॑ गृ॒ह - प॒तेः॒ । वाक् । भ॒व॒ति॒ । प॒शवः॑ । वै । छ॒न्दो॒मा इति॑ छन्दः - माः । अन्न᳚म् । म॒हा॒व्र॒तमिति॑ महा - व्र॒तम् । यत् । उ॒परि॑ष्टात् । छ॒न्दो॒माना॒मिति॑ छन्दः - माना᳚म् । म॒हा॒व्र॒तमिति॑ महा - व्र॒तम् । कु॒र्वन्ति॑ । प॒शुषु॑ । च॒ । ए॒व । अ॒न्नाद्य॒ इत्य॑न्न - अद्ये᳚ । च॒ । प्रतीति॑ । ति॒ष्ठ॒न्ति॒ ॥  \newline


\textbf{Krama Paata} \newline

यद् द्वा॑दशा॒हः । द्वा॒द॒शा॒हस्ताम् । द्वा॒द॒शा॒ह इति॑ द्वादश - अ॒हः । ताम् ॅवि । विच्छि॑न्द्युः । छि॒न्द्यु॒र् यत् । यन् मद्ध्ये᳚ । मद्ध्ये॑ऽतिरा॒त्रम् । अ॒ति॒रा॒त्रम् कु॒र्युः । अ॒ति॒रा॒त्रमित्य॑ति - रा॒त्रम् । कु॒र्युरु॑प॒दासु॑का । उ॒प॒दासु॑का गृ॒हप॑तेः । उ॒प॒दासु॒केत्यु॑प - दासु॑का । गृ॒हप॑ते॒र् वाक् । गृ॒हप॑ते॒रिति॑ गृ॒ह - प॒तेः॒ । वाख् स्या᳚त् । स्या॒दु॒परि॑ष्टात् । उ॒परि॑ष्टाच् छन्दो॒माना᳚म् । छ॒न्दो॒माना᳚म् महाव्र॒तम् । छ॒न्दो॒माना॒मिति॑ छन्दः - माना᳚म् । म॒हा॒व्र॒तम् कु॑र्वन्ति । म॒हा॒व्र॒तमिति॑ महा - व्र॒तम् । कु॒र्व॒न्ति॒ सन्त॑ताम् । सन्ता॑तामे॒व । सन्त॑ता॒मिति॒ सम् - त॒ता॒म् । ए॒व वाच᳚म् । वाच॒मव॑ । अव॑ रुन्धते । रु॒न्ध॒तेऽनु॑पदासुका । अनु॑पदासुका गृ॒हप॑तेः । अनु॑पदासु॒केत्यनु॑प - दा॒सु॒का॒ । गृ॒हप॑ते॒र् वाक् । गृ॒हप॑ते॒रिति॑ गृ॒ह - प॒तेः॒ । वाग् भ॑वति । भ॒व॒ति॒ प॒शवः॑ । प॒शवो॒ वै । वै छ॑न्दो॒माः । छ॒न्दो॒मा अन्न᳚म् । छ॒न्दो॒मा इति॑ छन्दः - माः । अन्न॑म् महाव्र॒तम् । म॒हा॒व्र॒तम् ॅयत् । म॒हा॒व्र॒तमिति॑ महा - व्र॒तम् । यदु॒परि॑ष्टात् । उ॒परि॑ष्टाच् छन्दो॒माना᳚म् । छ॒न्दो॒माना᳚म् महाव्र॒तम् । छ॒न्दो॒माना॒मिति॑ छन्दः - माना᳚म् । म॒हा॒व्र॒तम् कु॒र्वन्ति॑ । म॒हा॒व्र॒तमिति॑ महा - व्र॒तम् । कु॒र्वन्ति॑ प॒शुषु॑ । प॒शुषु॑ च । चै॒व । ए॒वान्नाद्ये᳚ । अ॒न्नाद्ये॑ च । अ॒न्नाद्य॒ इत्य॑न्न - अद्ये᳚ । च॒ प्रति॑ । प्रति॑ तिष्ठन्ति । ति॒ष्ठ॒न्तीति॑ तिष्ठन्ति । \newline

\textbf{Jatai Paata} \newline

1. यद् द्वा॑दशा॒हो द्वा॑दशा॒हो यद् यद् द्वा॑दशा॒हः । \newline
2. द्वा॒द॒शा॒ह स्ताम् ताम् द्वा॑दशा॒हो द्वा॑दशा॒ह स्ताम् । \newline
3. द्वा॒द॒शा॒ह इति॑ द्वादश - अ॒हः । \newline
4. तां ॅवि वि ताम् तां ॅवि । \newline
5. वि च्छि॑न्द्यु श्छिन्द्यु॒र् वि वि च्छि॑न्द्युः । \newline
6. छि॒न्द्यु॒र् यद् यच् छि॑न्द्यु श्छिन्द्यु॒र् यत् । \newline
7. यन् मद्ध्ये॒ मद्ध्ये॒ यद् यन् मद्ध्ये᳚ । \newline
8. मद्ध्ये॑ ऽतिरा॒त्र म॑तिरा॒त्रम् मद्ध्ये॒ मद्ध्ये॑ ऽतिरा॒त्रम् । \newline
9. अ॒ति॒रा॒त्रम् कु॒र्युः कु॒र्यु र॑तिरा॒त्र म॑तिरा॒त्रम् कु॒र्युः । \newline
10. अ॒ति॒रा॒त्रमित्य॑ति - रा॒त्रम् । \newline
11. कु॒र्यु रु॑प॒दासु॑ कोप॒दासु॑का कु॒र्युः कु॒र्यु रु॑प॒दासु॑का । \newline
12. उ॒प॒दासु॑का गृ॒हप॑तेर् गृ॒हप॑ते रुप॒दासु॑ कोप॒दासु॑का गृ॒हप॑तेः । \newline
13. उ॒प॒दासु॒केत्यु॑प - दासु॑का । \newline
14. गृ॒हप॑ते॒र् वाग् वाग् गृ॒हप॑तेर् गृ॒हप॑ते॒र् वाक् । \newline
15. गृ॒हप॑ते॒रिति॑ गृ॒ह - प॒तेः॒ । \newline
16. वाख् स्या᳚थ् स्या॒द् वाग् वाख् स्या᳚त् । \newline
17. स्या॒ दु॒परि॑ष्टा दु॒परि॑ष्टाथ् स्याथ् स्या दु॒परि॑ष्टात् । \newline
18. उ॒परि॑ष्टाच् छन्दो॒माना᳚म् छन्दो॒माना॑ मु॒परि॑ष्टा दु॒परि॑ष्टाच् छन्दो॒माना᳚म् । \newline
19. छ॒न्दो॒माना᳚म् महाव्र॒तम् म॑हाव्र॒तम् छ॑न्दो॒माना᳚म् छन्दो॒माना᳚म् महाव्र॒तम् । \newline
20. छ॒न्दो॒माना॒मिति॑ छन्दः - माना᳚म् । \newline
21. म॒हा॒व्र॒तम् कु॑र्वन्ति कुर्वन्ति महाव्र॒तम् म॑हाव्र॒तम् कु॑र्वन्ति । \newline
22. म॒हा॒व्र॒तमिति॑ महा - व्र॒तम् । \newline
23. कु॒र्व॒न्ति॒ सन्त॑ताꣳ॒॒ सन्त॑ताम् कुर्वन्ति कुर्वन्ति॒ सन्त॑ताम् । \newline
24. सन्त॑ता मे॒वैव सन्त॑ताꣳ॒॒ सन्त॑ता मे॒व । \newline
25. सन्त॑ता॒मिति॒ सं - त॒ता॒म् । \newline
26. ए॒व वाचं॒ ॅवाच॑ मे॒वैव वाच᳚म् । \newline
27. वाच॒ मवाव॒ वाचं॒ ॅवाच॒ मव॑ । \newline
28. अव॑ रुन्धते रुन्ध॒ते ऽवाव॑ रुन्धते । \newline
29. रु॒न्ध॒ते ऽनु॑पदासु॒का ऽनु॑पदासुका रुन्धते रुन्ध॒ते ऽनु॑पदासुका । \newline
30. अनु॑पदासुका गृ॒हप॑तेर् गृ॒हप॑ते॒ रनु॑पदासु॒का ऽनु॑पदासुका गृ॒हप॑तेः । \newline
31. अनु॑पदासु॒केत्यनु॑प - दा॒सु॒का॒ । \newline
32. गृ॒हप॑ते॒र् वाग् वाग् गृ॒हप॑तेर् गृ॒हप॑ते॒र् वाक् । \newline
33. गृ॒हप॑ते॒रिति॑ गृ॒ह - प॒तेः॒ । \newline
34. वाग् भ॑वति भवति॒ वाग् वाग् भ॑वति । \newline
35. भ॒व॒ति॒ प॒शवः॑ प॒शवो॑ भवति भवति प॒शवः॑ । \newline
36. प॒शवो॒ वै वै प॒शवः॑ प॒शवो॒ वै । \newline
37. वै छ॑न्दो॒मा श्छ॑न्दो॒मा वै वै छ॑न्दो॒माः । \newline
38. छ॒न्दो॒मा अन्न॒ मन्न॑म् छन्दो॒मा श्छ॑न्दो॒मा अन्न᳚म् । \newline
39. छ॒न्दो॒मा इति॑ छन्दः - माः । \newline
40. अन्न॑म् महाव्र॒तम् म॑हाव्र॒त मन्न॒ मन्न॑म् महाव्र॒तम् । \newline
41. म॒हा॒व्र॒तं ॅयद् यन् म॑हाव्र॒तम् म॑हाव्र॒तं ॅयत् । \newline
42. म॒हा॒व्र॒तमिति॑ महा - व्र॒तम् । \newline
43. यदु॒परि॑ष्टा दु॒परि॑ष्टा॒द् यद् यदु॒परि॑ष्टात् । \newline
44. उ॒परि॑ष्टाच् छन्दो॒माना᳚म् छन्दो॒माना॑ मु॒परि॑ष्टा दु॒परि॑ष्टाच् छन्दो॒माना᳚म् । \newline
45. छ॒न्दो॒माना᳚म् महाव्र॒तम् म॑हाव्र॒तम् छ॑न्दो॒माना᳚म् छन्दो॒माना᳚म् महाव्र॒तम् । \newline
46. छ॒न्दो॒माना॒मिति॑ छन्दः - माना᳚म् । \newline
47. म॒हा॒व्र॒तम् कु॒र्वन्ति॑ कु॒र्वन्ति॑ महाव्र॒तम् म॑हाव्र॒तम् कु॒र्वन्ति॑ । \newline
48. म॒हा॒व्र॒तमिति॑ महा - व्र॒तम् । \newline
49. कु॒र्वन्ति॑ प॒शुषु॑ प॒शुषु॑ कु॒र्वन्ति॑ कु॒र्वन्ति॑ प॒शुषु॑ । \newline
50. प॒शुषु॑ च च प॒शुषु॑ प॒शुषु॑ च । \newline
51. चै॒वैव च॑ चै॒व । \newline
52. ए॒वा न्नाद्ये॒ ऽन्नाद्य॑ ए॒वैवा न्नाद्ये᳚ । \newline
53. अ॒न्नाद्ये॑ च चा॒न्नाद्ये॒ ऽन्नाद्ये॑ च । \newline
54. अ॒न्नाद्य॒ इत्य॑न्न - अद्ये᳚ । \newline
55. च॒ प्रति॒ प्रति॑ च च॒ प्रति॑ । \newline
56. प्रति॑ तिष्ठन्ति तिष्ठन्ति॒ प्रति॒ प्रति॑ तिष्ठन्ति । \newline
57. ति॒ष्ठ॒न्तीति॑ तिष्ठन्ति । \newline

\textbf{Ghana Paata } \newline

1. यद् द्वा॑दशा॒हो द्वा॑दशा॒हो यद् यद् द्वा॑दशा॒ह स्ताम् ताम् द्वा॑दशा॒हो यद् यद् द्वा॑दशा॒ह स्ताम् । \newline
2. द्वा॒द॒शा॒ह स्ताम् ताम् द्वा॑दशा॒हो द्वा॑दशा॒ह स्तां ॅवि वि ताम् द्वा॑दशा॒हो द्वा॑दशा॒ह स्तां ॅवि । \newline
3. द्वा॒द॒शा॒ह इति॑ द्वादश - अ॒हः । \newline
4. तां ॅवि वि ताम् तां ॅवि च्छि॑न्द्यु श्छिन्द्यु॒र् वि ताम् तां ॅवि च्छि॑न्द्युः । \newline
5. वि च्छि॑न्द्यु श्छिन्द्यु॒र् वि विच् छि॑न्द्यु॒र् यद् यच् छि॑न्द्यु॒र् वि वि च्छि॑न्द्यु॒र् यत् । \newline
6. छि॒न्द्यु॒र् यद् यच् छि॑न्द्यु श्छिन्द्यु॒र् यन् मद्ध्ये॒ मद्ध्ये॒ यच् छि॑न्द्यु श्छिन्द्यु॒र् यन् मद्ध्ये᳚ । \newline
7. यन् मद्ध्ये॒ मद्ध्ये॒ यद् यन् मद्ध्ये॑ ऽतिरा॒त्र म॑तिरा॒त्रम् मद्ध्ये॒ यद् यन् मद्ध्ये॑ ऽतिरा॒त्रम् । \newline
8. मद्ध्ये॑ ऽतिरा॒त्र म॑तिरा॒त्रम् मद्ध्ये॒ मद्ध्ये॑ ऽतिरा॒त्रम् कु॒र्युः कु॒र्यु र॑तिरा॒त्रम् मद्ध्ये॒ मद्ध्ये॑ ऽतिरा॒त्रम् कु॒र्युः । \newline
9. अ॒ति॒रा॒त्रम् कु॒र्युः कु॒र्यु र॑तिरा॒त्र म॑तिरा॒त्रम् कु॒र्यु रु॑प॒दासु॑ कोप॒दासु॑का कु॒र्यु र॑तिरा॒त्र म॑तिरा॒त्रम् कु॒र्यु रु॑प॒दासु॑का । \newline
10. अ॒ति॒रा॒त्रमित्य॑ति - रा॒त्रम् । \newline
11. कु॒र्यु रु॑प॒दासु॑ कोप॒दासु॑का कु॒र्युः कु॒र्यु रु॑प॒दासु॑का गृ॒हप॑तेर् गृ॒हप॑ते रुप॒दासु॑का कु॒र्युः कु॒र्यु रु॑प॒दासु॑का गृ॒हप॑तेः । \newline
12. उ॒प॒दासु॑का गृ॒हप॑तेर् गृ॒हप॑ते रुप॒दासु॑ कोप॒दासु॑का गृ॒हप॑ते॒र् वाग् वाग् गृ॒हप॑ते रुप॒दासु॑को प॒दासु॑का गृ॒हप॑ते॒र् वाक् । \newline
13. उ॒प॒दासु॒केत्यु॑प - दासु॑का । \newline
14. गृ॒हप॑ते॒र् वाग् वाग् गृ॒हप॑तेर् गृ॒हप॑ते॒र् वाख् स्या᳚थ् स्या॒द् वाग् गृ॒हप॑तेर् गृ॒हप॑ते॒र् वाख् स्या᳚त् । \newline
15. गृ॒हप॑ते॒रिति॑ गृ॒ह - प॒तेः॒ । \newline
16. वाख् स्या᳚थ् स्या॒द् वाग् वाख् स्या॑ दु॒परि॑ष्टा दु॒परि॑ष्टाथ् स्या॒द् वाग् वाख् स्या॑ दु॒परि॑ष्टात् । \newline
17. स्या॒ दु॒परि॑ष्टा दु॒परि॑ष्टाथ् स्याथ् स्या दु॒परि॑ष्टाच् छन्दो॒माना᳚म् छन्दो॒माना॑ मु॒परि॑ष्टाथ् स्याथ् स्या दु॒परि॑ष्टाच् छन्दो॒माना᳚म् । \newline
18. उ॒परि॑ष्टाच् छन्दो॒माना᳚म् छन्दो॒माना॑ मु॒परि॑ष्टा दु॒परि॑ष्टाच् छन्दो॒माना᳚म् महाव्र॒तम् म॑हाव्र॒तम् छ॑न्दो॒माना॑ मु॒परि॑ष्टा दु॒परि॑ष्टाच् छन्दो॒माना᳚म् महाव्र॒तम् । \newline
19. छ॒न्दो॒माना᳚म् महाव्र॒तम् म॑हाव्र॒तम् छ॑न्दो॒माना᳚म् छन्दो॒माना᳚म् महाव्र॒तम् कु॑र्वन्ति कुर्वन्ति महाव्र॒तम् छ॑न्दो॒माना᳚म् छन्दो॒माना᳚म् महाव्र॒तम् कु॑र्वन्ति । \newline
20. छ॒न्दो॒माना॒मिति॑ छन्दः - माना᳚म् । \newline
21. म॒हा॒व्र॒तम् कु॑र्वन्ति कुर्वन्ति महाव्र॒तम् म॑हाव्र॒तम् कु॑र्वन्ति॒ सन्त॑ताꣳ॒॒ सन्त॑ताम् कुर्वन्ति महाव्र॒तम् म॑हाव्र॒तम् कु॑र्वन्ति॒ सन्त॑ताम् । \newline
22. म॒हा॒व्र॒तमिति॑ महा - व्र॒तम् । \newline
23. कु॒र्व॒न्ति॒ सन्त॑ताꣳ॒॒ सन्त॑ताम् कुर्वन्ति कुर्वन्ति॒ सन्त॑ता मे॒वैव सन्त॑ताम् कुर्वन्ति कुर्वन्ति॒ सन्त॑ता मे॒व । \newline
24. सन्त॑ता मे॒वैव सन्त॑ताꣳ॒॒ सन्त॑ता मे॒व वाचं॒ ॅवाच॑ मे॒व सन्त॑ताꣳ॒॒ सन्त॑ता मे॒व वाच᳚म् । \newline
25. सन्त॑ता॒मिति॒ सं - त॒ता॒म् । \newline
26. ए॒व वाचं॒ ॅवाच॑ मे॒वैव वाच॒ मवाव॒ वाच॑ मे॒वैव वाच॒ मव॑ । \newline
27. वाच॒ मवाव॒ वाचं॒ ॅवाच॒ मव॑ रुन्धते रुन्ध॒ते ऽव॒ वाचं॒ ॅवाच॒ मव॑ रुन्धते । \newline
28. अव॑ रुन्धते रुन्ध॒ते ऽवाव॑ रुन्ध॒ते ऽनु॑पदासु॒का ऽनु॑पदासुका रुन्ध॒ते ऽवाव॑ रुन्ध॒ते ऽनु॑पदासुका । \newline
29. रु॒न्ध॒ते ऽनु॑पदासु॒का ऽनु॑पदासुका रुन्धते रुन्ध॒ते ऽनु॑पदासुका गृ॒हप॑तेर् गृ॒हप॑ते॒ रनु॑पदासुका रुन्धते रुन्ध॒ते ऽनु॑पदासुका गृ॒हप॑तेः । \newline
30. अनु॑पदासुका गृ॒हप॑तेर् गृ॒हप॑ते॒ रनु॑पदासु॒का ऽनु॑पदासुका गृ॒हप॑ते॒र् वाग् वाग् गृ॒हप॑ते॒ रनु॑पदासु॒का ऽनु॑पदासुका गृ॒हप॑ते॒र् वाक् । \newline
31. अनु॑पदासु॒केत्यनु॑प - दा॒सु॒का॒ । \newline
32. गृ॒हप॑ते॒र् वाग् वाग् गृ॒हप॑तेर् गृ॒हप॑ते॒र् वाग् भ॑वति भवति॒ वाग् गृ॒हप॑तेर् गृ॒हप॑ते॒र् वाग् भ॑वति । \newline
33. गृ॒हप॑ते॒रिति॑ गृ॒ह - प॒तेः॒ । \newline
34. वाग् भ॑वति भवति॒ वाग् वाग् भ॑वति प॒शवः॑ प॒शवो॑ भवति॒ वाग् वाग् भ॑वति प॒शवः॑ । \newline
35. भ॒व॒ति॒ प॒शवः॑ प॒शवो॑ भवति भवति प॒शवो॒ वै वै प॒शवो॑ भवति भवति प॒शवो॒ वै । \newline
36. प॒शवो॒ वै वै प॒शवः॑ प॒शवो॒ वै छ॑न्दो॒मा श्छ॑न्दो॒मा वै प॒शवः॑ प॒शवो॒ वै छ॑न्दो॒माः । \newline
37. वै छ॑न्दो॒मा श्छ॑न्दो॒मा वै वै छ॑न्दो॒मा अन्न॒ मन्न॑म् छन्दो॒मा वै वै छ॑न्दो॒मा अन्न᳚म् । \newline
38. छ॒न्दो॒मा अन्न॒ मन्न॑म् छन्दो॒मा श्छ॑न्दो॒मा अन्न॑म् महाव्र॒तम् म॑हाव्र॒त मन्न॑म् छन्दो॒मा श्छ॑न्दो॒मा अन्न॑म् महाव्र॒तम् । \newline
39. छ॒न्दो॒मा इति॑ छन्दः - माः । \newline
40. अन्न॑म् महाव्र॒तम् म॑हाव्र॒त मन्न॒ मन्न॑म् महाव्र॒तं ॅयद् यन् म॑हाव्र॒त मन्न॒ मन्न॑म् महाव्र॒तं ॅयत् । \newline
41. म॒हा॒व्र॒तं ॅयद् यन् म॑हाव्र॒तम् म॑हाव्र॒तं ॅयदु॒परि॑ष्टा दु॒परि॑ष्टा॒द् यन् म॑हाव्र॒तम् म॑हाव्र॒तं ॅयदु॒परि॑ष्टात् । \newline
42. म॒हा॒व्र॒तमिति॑ महा - व्र॒तम् । \newline
43. यदु॒परि॑ष्टा दु॒परि॑ष्टा॒द् यद् यदु॒परि॑ष्टाच् छन्दो॒माना᳚म् छन्दो॒माना॑ मु॒परि॑ष्टा॒द् यद् यदु॒परि॑ष्टाच् छन्दो॒माना᳚म् । \newline
44. उ॒परि॑ष्टाच् छन्दो॒माना᳚म् छन्दो॒माना॑ मु॒परि॑ष्टा दु॒परि॑ष्टाच् छन्दो॒माना᳚म् महाव्र॒तम् म॑हाव्र॒तम् छ॑न्दो॒माना॑ मु॒परि॑ष्टा दु॒परि॑ष्टाच् छन्दो॒माना᳚म् महाव्र॒तम् । \newline
45. छ॒न्दो॒माना᳚म् महाव्र॒तम् म॑हाव्र॒तम् छ॑न्दो॒माना᳚म् छन्दो॒माना᳚म् महाव्र॒तम् कु॒र्वन्ति॑ कु॒र्वन्ति॑ महाव्र॒तम् छ॑न्दो॒माना᳚म् छन्दो॒माना᳚म् महाव्र॒तम् कु॒र्वन्ति॑ । \newline
46. छ॒न्दो॒माना॒मिति॑ छन्दः - माना᳚म् । \newline
47. म॒हा॒व्र॒तम् कु॒र्वन्ति॑ कु॒र्वन्ति॑ महाव्र॒तम् म॑हाव्र॒तम् कु॒र्वन्ति॑ प॒शुषु॑ प॒शुषु॑ कु॒र्वन्ति॑ महाव्र॒तम् म॑हाव्र॒तम् कु॒र्वन्ति॑ प॒शुषु॑ । \newline
48. म॒हा॒व्र॒तमिति॑ महा - व्र॒तम् । \newline
49. कु॒र्वन्ति॑ प॒शुषु॑ प॒शुषु॑ कु॒र्वन्ति॑ कु॒र्वन्ति॑ प॒शुषु॑ च च प॒शुषु॑ कु॒र्वन्ति॑ कु॒र्वन्ति॑ प॒शुषु॑ च । \newline
50. प॒शुषु॑ च च प॒शुषु॑ प॒शुषु॑ चै॒वैव च॑ प॒शुषु॑ प॒शुषु॑ चै॒व । \newline
51. चै॒वैव च॑ चै॒वान्नाद्ये॒ ऽन्नाद्य॑ ए॒व च॑ चै॒वान्नाद्ये᳚ । \newline
52. ए॒वान्नाद्ये॒ ऽन्नाद्य॑ ए॒वै वान्नाद्ये॑ च चा॒न्नाद्य॑ ए॒वै वान्नाद्ये॑ च । \newline
53. अ॒न्नाद्ये॑ च चा॒न्नाद्ये॒ ऽन्नाद्ये॑ च॒ प्रति॒ प्रति॑ चा॒न्नाद्ये॒ ऽन्नाद्ये॑ च॒ प्रति॑ । \newline
54. अ॒न्नाद्य॒ इत्य॑न्न - अद्ये᳚ । \newline
55. च॒ प्रति॒ प्रति॑ च च॒ प्रति॑ तिष्ठन्ति तिष्ठन्ति॒ प्रति॑ च च॒ प्रति॑ तिष्ठन्ति । \newline
56. प्रति॑ तिष्ठन्ति तिष्ठन्ति॒ प्रति॒ प्रति॑ तिष्ठन्ति । \newline
57. ति॒ष्ठ॒न्तीति॑ तिष्ठन्ति । \newline
\pagebreak
\markright{ TS 7.3.4.1  \hfill https://www.vedavms.in \hfill}

\section{ TS 7.3.4.1 }

\textbf{TS 7.3.4.1 } \newline
\textbf{Samhita Paata} \newline

आ॒दि॒त्या अ॑कामयन्तो॒-भयो᳚र्लो॒कयोर्॑ ऋद्ध्नुया॒मेति॒ त ए॒तं च॑तुर्दशरा॒त्र- म॑पश्य॒न् तमाऽह॑र॒न् तेना॑यजन्त॒ ततो॒ वै त उ॒भयो᳚-र्लो॒कयो॑-रार्द्ध्नुवन्न॒स्मिꣳश्चा॒-मुष्मिꣳ॑श्च॒ य ए॒वं ॅवि॒द्वाꣳस॑-श्चतुर्दशरा॒त्रमास॑त उ॒भयो॑रे॒व लो॒कयोर्॑. ऋद्ध्नुवन्त्य॒स्मिꣳश्चा॒-मुष्मिꣳ॑श्च चतुर्दशरा॒त्रो भ॑वति स॒प्त ग्रा॒म्या ओष॑धयः स॒प्ताऽऽ*र॒ण्या उ॒भयी॑षा॒मव॑रुद्ध्यै॒ यत् प॑रा॒चीना॑नि पृ॒ष्ठानि॒ - [  ] \newline

\textbf{Pada Paata} \newline

आ॒दि॒त्याः । अ॒का॒म॒य॒न्त॒ । उ॒भयोः᳚ । लो॒कयोः᳚ । ऋ॒द्ध्नु॒या॒म॒ । इति॑ । ते । ए॒तम् । च॒तु॒र्द॒श॒रा॒त्रमिति॑ चतुर्दश - रा॒त्रम् । अ॒प॒श्य॒न्न् । तम् । एति॑ । अ॒ह॒र॒न्न् । तेन॑ । अ॒य॒ज॒न्त॒ । ततः॑ । वै । ते । उ॒भयोः᳚ । लो॒कयोः᳚ । आ॒द्‌र्ध्नु॒व॒न्न् । अ॒स्मिन्न् । च॒ । अ॒मुष्मिन्न्॑ । च॒ । ये । ए॒वम् । वि॒द्वाꣳसः॑ । च॒तु॒र्द॒श॒रा॒त्रमिति॑ चतुर्दश - रा॒त्रम् । आस॑ते । उ॒भयोः᳚ । ए॒व । लो॒कयोः᳚ । ऋ॒द्ध्नु॒व॒न्ति॒ । अ॒स्मिन्न् । च॒ । अ॒मुष्मिन्न्॑ । च॒ । च॒तु॒र्द॒श॒रा॒त्र इति॑ चतुर्दश - रा॒त्रः । भ॒व॒ति॒ । स॒प्त । ग्रा॒म्याः । ओष॑धयः । स॒प्त । आ॒र॒ण्याः । उ॒भयी॑षाम् । अव॑रुद्ध्या॒ इत्यव॑-रु॒द्ध्यै॒ । यत् । प॒रा॒चीना॑नि । पृ॒ष्ठानि॑ ।  \newline


\textbf{Krama Paata} \newline

आ॒दि॒त्या अ॑कामयन्त । अ॒का॒म॒य॒न्तो॒भयोः᳚ । उ॒भयो᳚र् लो॒कयोः᳚ । लो॒कयोर्॑. ऋद्ध्नुयाम । ऋ॒द्ध्नु॒या॒मेति॑ । इति॒ ते । त ए॒तम् । ए॒तम् च॑तुर्दशरा॒त्रम् । च॒तु॒र्द॒श॒रा॒त्रम॑पश्यन्न् । च॒तु॒र्द॒श॒रा॒त्रमिति॑ चतुर्दश - रा॒त्रम् । अ॒प॒श्य॒न् तम् । तमा । आऽह॑रन्न् । अ॒ह॒र॒न् तेन॑ । तेना॑यजन्त । अ॒य॒ज॒न्त॒ ततः॑ । ततो॒ वै । वै ते । त उ॒भयोः᳚ । उ॒भयो᳚र् लो॒कयोः᳚ । लो॒कयो॑रार्द्ध्नुवन्न् । आ॒र्द्ध्नु॒व॒न्न॒स्मिन्न् । अ॒स्मिꣳश्च॑ । चा॒मुष्मिन्न्॑ । अ॒मुष्मिꣳ॑श्च । च॒ ये । य ए॒वम् । ए॒वम् ॅवि॒द्वाꣳसः॑ । वि॒द्वाꣳस॑श्चतुर्दशरा॒त्रम् । च॒तु॒र्द॒श॒रा॒त्रमास॑ते । च॒तु॒र्द॒श॒रा॒त्रमिति॑ चतुर्दश - रा॒त्रम् । आस॑त उ॒भयोः᳚ । उ॒भयो॑रे॒व । ए॒व लो॒कयोः᳚ । लो॒कयोर्॑. ऋद्ध्नुवन्ति । ऋ॒द्ध्नु॒व॒न्त्य॒स्मिन्न् । अ॒स्मिꣳश्च॑ । चा॒मुष्मिन्न्॑ । अ॒मुष्मिꣳ॑श्च । च॒ च॒तु॒र्द॒श॒रा॒त्रः । च॒तु॒र्द॒श॒रा॒त्रो भ॑वति । च॒तु॒र्द॒श॒रा॒त्र इति॑ चतुर्दश - रा॒त्रः । भ॒व॒ति॒ स॒प्त । स॒प्त ग्रा॒म्याः । ग्रा॒म्या ओष॑धयः । ओष॑धयः स॒प्त । स॒प्तार॒ण्याः । आ॒र॒ण्या उ॒भयी॑षाम् । उ॒भयी॑षा॒मव॑रुद्ध्यै । अव॑रुद्ध्यै॒ यत् । अव॑रुद्ध्या॒ इत्यव॑ - रु॒द्ध्यै॒ । यत् प॑रा॒चीना॑नि । प॒रा॒चीना॑नि पृ॒ष्ठानि॑ ( ) । पृ॒ष्ठानि॒ भव॑न्ति \newline

\textbf{Jatai Paata} \newline

1. आ॒दि॒त्या अ॑कामयन्ता कामयन्ता दि॒त्या आ॑दि॒त्या अ॑कामयन्त । \newline
2. अ॒का॒म॒य॒न्तो॒भयो॑ रु॒भयो॑ रकामयन्ता कामयन्तो॒भयोः᳚ । \newline
3. उ॒भयो᳚र् लो॒कयो᳚र् लो॒कयो॑ रु॒भयो॑ रु॒भयो᳚र् लो॒कयोः᳚ । \newline
4. लो॒कयोर्॑. ऋद्ध्नुयाम र्द्ध्नुयाम लो॒कयो᳚र् लो॒कयोर्॑. ऋद्ध्नुयाम । \newline
5. ऋ॒द्ध्नु॒या॒मे तीत्यृ॑द्ध्नुयाम र्द्ध्नुया॒मेति॑ । \newline
6. इति॒ ते त इतीति॒ ते । \newline
7. त ए॒त मे॒तम् ते त ए॒तम् । \newline
8. ए॒तम् च॑तुर्दशरा॒त्रम् च॑तुर्दशरा॒त्र मे॒त मे॒तम् च॑तुर्दशरा॒त्रम् । \newline
9. च॒तु॒र्द॒श॒रा॒त्र म॑पश्यन् नपश्यꣳ श्चतुर्दशरा॒त्रम् च॑तुर्दशरा॒त्र म॑पश्यन्न् । \newline
10. च॒तु॒र्द॒श॒रा॒त्रमिति॑ चतुर्दश - रा॒त्रम् । \newline
11. अ॒प॒श्य॒न् तम् त म॑पश्यन् नपश्य॒न् तम् । \newline
12. त मा तम् त मा । \newline
13. आ ऽह॑रन् नहर॒न् ना ऽह॑रन्न् । \newline
14. अ॒ह॒र॒न् तेन॒ तेना॑ हरन् नहर॒न् तेन॑ । \newline
15. तेना॑ यजन्ता यजन्त॒ तेन॒ तेना॑ यजन्त । \newline
16. अ॒य॒ज॒न्त॒ तत॒ स्ततो॑ ऽयजन्ता यजन्त॒ ततः॑ । \newline
17. ततो॒ वै वै तत॒ स्ततो॒ वै । \newline
18. वै ते ते वै वै ते । \newline
19. त उ॒भयो॑ रु॒भयो॒ स्ते त उ॒भयोः᳚ । \newline
20. उ॒भयो᳚र् लो॒कयो᳚र् लो॒कयो॑ रु॒भयो॑ रु॒भयो᳚र् लो॒कयोः᳚ । \newline
21. लो॒कयो॑ रार्द्ध्नुवन् नार्द्ध्नुवन् ॅलो॒कयो᳚र् लो॒कयो॑ रार्द्ध्नुवन्न् । \newline
22. आ॒र्द्ध्नु॒व॒न् न॒स्मिन् न॒स्मिन् ना᳚र्द्ध्नुवन् नार्द्ध्नुवन् न॒स्मिन्न् । \newline
23. अ॒स्मिꣳ श्च॑ चा॒स्मिन् न॒स्मिꣳ श्च॑ । \newline
24. चा॒मुष्मि॑न् न॒मुष्मिꣳ॑श्च चा॒मुष्मिन्न्॑ । \newline
25. अ॒मुष्मिꣳ॑श्च चा॒मुष्मि॑न् न॒मुष्मिꣳ॑श्च । \newline
26. च॒ ये ये च॑ च॒ ये । \newline
27. य ए॒व मे॒वं ॅये य ए॒वम् । \newline
28. ए॒वं ॅवि॒द्वाꣳसो॑ वि॒द्वाꣳस॑ ए॒व मे॒वं ॅवि॒द्वाꣳसः॑ । \newline
29. वि॒द्वाꣳस॑ श्चतुर्दशरा॒त्रम् च॑तुर्दशरा॒त्रं ॅवि॒द्वाꣳसो॑ वि॒द्वाꣳस॑ श्चतुर्दशरा॒त्रम् । \newline
30. च॒तु॒र्द॒श॒रा॒त्र मास॑त॒ आस॑ते चतुर्दशरा॒त्रम् च॑तुर्दशरा॒त्र मास॑ते । \newline
31. च॒तु॒र्द॒श॒रा॒त्रमिति॑ चतुर्दश - रा॒त्रम् । \newline
32. आस॑त उ॒भयो॑ रु॒भयो॒ रास॑त॒ आस॑त उ॒भयोः᳚ । \newline
33. उ॒भयो॑ रे॒वै वोभयो॑ रु॒भयो॑ रे॒व । \newline
34. ए॒व लो॒कयो᳚र् लो॒कयो॑ रे॒वैव लो॒कयोः᳚ । \newline
35. लो॒कयोर्॑. ऋद्ध्नुवन् त्यृद्ध्नुवन्ति लो॒कयो᳚र् लो॒कयोर्॑. ऋद्ध्नुवन्ति । \newline
36. ऋ॒द्ध्नु॒व॒ न्त्य॒स्मिन् न॒स्मिन् नृ॑द्ध्नुव न्त्यृद्ध्नुव न्त्य॒स्मिन्न् । \newline
37. अ॒स्मिꣳश्च॑ चा॒स्मिन् न॒स्मिꣳश्च॑ । \newline
38. चा॒मुष्मि॑न् न॒मुष्मिꣳ॑श्च चा॒मुष्मिन्न्॑ । \newline
39. अ॒मुष्मिꣳ॑श्च चा॒मुष्मि॑न् न॒मुष्मिꣳ॑श्च । \newline
40. च॒ च॒तु॒र्द॒श॒रा॒त्र श्च॑तुर्दशरा॒त्र श्च॑ च चतुर्दशरा॒त्रः । \newline
41. च॒तु॒र्द॒श॒रा॒त्रो भ॑वति भवति चतुर्दशरा॒त्र श्च॑तुर्दशरा॒त्रो भ॑वति । \newline
42. च॒तु॒र्द॒श॒रा॒त्र इति॑ चतुर्दश - रा॒त्रः । \newline
43. भ॒व॒ति॒ स॒प्त स॒प्त भ॑वति भवति स॒प्त । \newline
44. स॒प्त ग्रा॒म्या ग्रा॒म्याः स॒प्त स॒प्त ग्रा॒म्याः । \newline
45. ग्रा॒म्या ओष॑धय॒ ओष॑धयो ग्रा॒म्या ग्रा॒म्या ओष॑धयः । \newline
46. ओष॑धयः स॒प्त स॒प्तौष॑धय॒ ओष॑धयः स॒प्त । \newline
47. स॒प्ता र॒ण्या आ॑र॒ण्याः स॒प्त स॒प्ता र॒ण्याः । \newline
48. आ॒र॒ण्या उ॒भयी॑षा मु॒भयी॑षा मार॒ण्या आ॑र॒ण्या उ॒भयी॑षाम् । \newline
49. उ॒भयी॑षा॒ मव॑रुद्ध्या॒ अव॑रुद्ध्या उ॒भयी॑षा मु॒भयी॑षा॒ मव॑रुद्ध्यै । \newline
50. अव॑रुद्ध्यै॒ यद् यदव॑रुद्ध्या॒ अव॑रुद्ध्यै॒ यत् । \newline
51. अव॑रुद्ध्या॒ इत्यव॑ - रु॒द्ध्यै॒ । \newline
52. यत् प॑रा॒चीना॑नि परा॒चीना॑नि॒ यद् यत् प॑रा॒चीना॑नि । \newline
53. प॒रा॒चीना॑नि पृ॒ष्ठानि॑ पृ॒ष्ठानि॑ परा॒चीना॑नि परा॒चीना॑नि पृ॒ष्ठानि॑ । \newline
54. पृ॒ष्ठानि॒ भव॑न्ति॒ भव॑न्ति पृ॒ष्ठानि॑ पृ॒ष्ठानि॒ भव॑न्ति । \newline

\textbf{Ghana Paata } \newline

1. आ॒दि॒त्या अ॑कामयन्ता कामयन्ता दि॒त्या आ॑दि॒त्या अ॑कामयन्तो॒ भयो॑ रु॒भयो॑ रकामयन्ता दि॒त्या आ॑दि॒त्या अ॑कामयन्तो॒ भयोः᳚ । \newline
2. अ॒का॒म॒य॒न्तो॒ भयो॑ रु॒भयो॑ रकामयन्ता कामयन्तो॒ भयो᳚र् लो॒कयो᳚र् लो॒कयो॑ रु॒भयो॑ रकामयन्ता 
कामयन्तो॒ भयो᳚र् लो॒कयोः᳚ । \newline
3. उ॒भयो᳚र् लो॒कयो᳚र् लो॒कयो॑ रु॒भयो॑ रु॒भयो᳚र् लो॒कयोर्॑. ऋद्ध्नुयाम र्‌द्ध्नुयाम लो॒कयो॑ रु॒भयो॑ रु॒भयो᳚र् लो॒कयोर्॑. ऋद्ध्नुयाम । \newline
4. लो॒कयोर्॑. ऋद्ध्नुयाम र्‌द्ध्नुयाम लो॒कयो᳚र् लो॒कयोर्॑. ऋद्ध्नुया॒ मेतीत्यृ॑द्ध्नुयाम लो॒कयो᳚र् लो॒कयोर्॑. ऋद्ध्नुया॒मेति॑ । \newline
5. ऋ॒द्ध्नु॒या॒ मेतीत्यृ॑द्ध्नुयाम र्‌द्ध्नुया॒मेति॒ ते त इत्यृ॑द्ध्नुयाम र्‌द्ध्नुया॒मेति॒ ते । \newline
6. इति॒ ते त इतीति॒ त ए॒त मे॒तम् त इतीति॒ त ए॒तम् । \newline
7. त ए॒त मे॒तम् ते त ए॒तम् च॑तुर्दशरा॒त्रम् च॑तुर्दशरा॒त्र मे॒तम् ते त ए॒तम् च॑तुर्दशरा॒त्रम् । \newline
8. ए॒तम् च॑तुर्दशरा॒त्रम् च॑तुर्दशरा॒त्र मे॒त मे॒तम् च॑तुर्दशरा॒त्र म॑पश्यन् नपश्यꣳ श्चतुर्दशरा॒त्र मे॒त मे॒तम् च॑तुर्दशरा॒त्र म॑पश्यन्न् । \newline
9. च॒तु॒र्द॒श॒रा॒त्र म॑पश्यन् नपश्यꣳ श्चतुर्दशरा॒त्रम् च॑तुर्दशरा॒त्र म॑पश्य॒न् तम् त म॑पश्यꣳ श्चतुर्दशरा॒त्रम् च॑तुर्दशरा॒त्र म॑पश्य॒न् तम् । \newline
10. च॒तु॒र्द॒श॒रा॒त्रमिति॑ चतुर्दश - रा॒त्रम् । \newline
11. अ॒प॒श्य॒न् तम् त म॑पश्यन् नपश्य॒न् त मा त म॑पश्यन् नपश्य॒न् त मा । \newline
12. त मा तम् त मा ऽह॑रन् नहर॒न् ना तम् त मा ऽह॑रन्न् । \newline
13. आ ऽह॑रन् नहर॒न् ना ऽह॑र॒न् तेन॒ तेना॑हर॒न् ना ऽह॑र॒न् तेन॑ । \newline
14. अ॒ह॒र॒न् तेन॒ तेना॑हरन् नहर॒न् तेना॑ यजन्ता यजन्त॒ तेना॑हरन् नहर॒न् तेना॑ यजन्त । \newline
15. तेना॑ यजन्ता यजन्त॒ तेन॒ तेना॑ यजन्त॒ तत॒ स्ततो॑ ऽयजन्त॒ तेन॒ तेना॑ यजन्त॒ ततः॑ । \newline
16. अ॒य॒ज॒न्त॒ तत॒ स्ततो॑ ऽयजन्ता यजन्त॒ ततो॒ वै वै ततो॑ ऽयजन्ता यजन्त॒ ततो॒ वै । \newline
17. ततो॒ वै वै तत॒ स्ततो॒ वै ते ते वै तत॒ स्ततो॒ वै ते । \newline
18. वै ते ते वै वै त उ॒भयो॑ रु॒भयो॒ स्ते वै वै त उ॒भयोः᳚ । \newline
19. त उ॒भयो॑ रु॒भयो॒ स्ते त उ॒भयो᳚र् लो॒कयो᳚र् लो॒कयो॑ रु॒भयो॒ स्ते त उ॒भयो᳚र् लो॒कयोः᳚ । \newline
20. उ॒भयो᳚र् लो॒कयो᳚र् लो॒कयो॑ रु॒भयो॑ रु॒भयो᳚र् लो॒कयो॑ रार्द्ध्नुवन् नार्द्ध्नुवन् ॅलो॒कयो॑ रु॒भयो॑ रु॒भयो᳚र् लो॒कयो॑ रार्द्ध्नुवन्न् । \newline
21. लो॒कयो॑ रार्द्ध्नुवन् नार्द्ध्नुवन् ॅलो॒कयो᳚र् लो॒कयो॑ रार्द्ध्नुवन् न॒स्मिन् न॒स्मिन् ना᳚र्द्ध्नुवन् ॅलो॒कयो᳚र् लो॒कयो॑ रार्द्ध्नुवन् न॒स्मिन्न् । \newline
22. आ॒र्द्ध्नु॒व॒न् न॒स्मिन् न॒स्मिन् ना᳚र्द्ध्नुवन् नार्द्ध्नुवन् न॒स्मिꣳश्च॑ चा॒स्मिन् ना᳚र्द्ध्नुवन् नार्द्ध्नुवन् न॒स्मिꣳ श्च॑ । \newline
23. अ॒स्मिꣳश्च॑ चा॒स्मिन् न॒स्मिꣳ श्चा॒मुष्मि॑न् न॒मुष्मिꣳ॑ श्चा॒स्मिन् न॒स्मिꣳ
श्चा॒मुष्मिन्न्॑ । \newline
24. चा॒मुष्मि॑न् न॒मुष्मिꣳ॑श्च चा॒मुष्मिꣳ॑श्च चा॒मुष्मिꣳ॑श्च चा॒मुष्मिꣳ॑श्च । \newline
25. अ॒मुष्मिꣳ॑श्च चा॒मुष्मि॑न् न॒मुष्मिꣳ॑श्च॒ ये ये चा॒मुष्मि॑न् न॒मुष्मिꣳ॑श्च॒ ये । \newline
26. च॒ ये ये च॑ च॒ य ए॒व मे॒वं ॅये च॑ च॒ य ए॒वम् । \newline
27. य ए॒व मे॒वं ॅये य ए॒वं ॅवि॒द्वाꣳसो॑ वि॒द्वाꣳस॑ ए॒वं ॅये य ए॒वं ॅवि॒द्वाꣳसः॑ । \newline
28. ए॒वं ॅवि॒द्वाꣳसो॑ वि॒द्वाꣳस॑ ए॒व मे॒वं ॅवि॒द्वाꣳस॑ श्चतुर्दशरा॒त्रम् च॑तुर्दशरा॒त्रं ॅवि॒द्वाꣳस॑ ए॒व मे॒वं ॅवि॒द्वाꣳस॑ श्चतुर्दशरा॒त्रम् । \newline
29. वि॒द्वाꣳस॑ श्चतुर्दशरा॒त्रम् च॑तुर्दशरा॒त्रं ॅवि॒द्वाꣳसो॑ वि॒द्वाꣳस॑ श्चतुर्दशरा॒त्र मास॑त॒ आस॑ते चतुर्दशरा॒त्रं ॅवि॒द्वाꣳसो॑ वि॒द्वाꣳस॑ श्चतुर्दशरा॒त्र मास॑ते । \newline
30. च॒तु॒र्द॒श॒रा॒त्र मास॑त॒ आस॑ते चतुर्दशरा॒त्रम् च॑तुर्दशरा॒त्र मास॑त उ॒भयो॑ रु॒भयो॒ रास॑ते चतुर्दशरा॒त्रम् च॑तुर्दशरा॒त्र मास॑त उ॒भयोः᳚ । \newline
31. च॒तु॒र्द॒श॒रा॒त्रमिति॑ चतुर्दश - रा॒त्रम् । \newline
32. आस॑त उ॒भयो॑ रु॒भयो॒ रास॑त॒ आस॑त उ॒भयो॑ रे॒वैवो भयो॒ रास॑त॒ आस॑त उ॒भयो॑ रे॒व । \newline
33. उ॒भयो॑ रे॒वैवो भयो॑ रु॒भयो॑ रे॒व लो॒कयो᳚र् लो॒कयो॑ रे॒वो भयो॑ रु॒भयो॑ रे॒व लो॒कयोः᳚ । \newline
34. ए॒व लो॒कयो᳚र् लो॒कयो॑ रे॒वैव लो॒कयोर्॑. ऋद्ध्नुव न्त्यृद्ध्नुवन्ति लो॒कयो॑ रे॒वैव लो॒कयोर्॑. ऋद्ध्नुवन्ति । \newline
35. लो॒कयोर्॑. ऋद्ध्नुव न्त्यृद्ध्नुवन्ति लो॒कयो᳚र् लो॒कयोर्॑. ऋद्ध्नुव न्त्य॒स्मिन् न॒स्मिन् नृ॑द्ध्नुवन्ति लो॒कयो᳚र् लो॒कयोर्॑. ऋद्ध्नुव न्त्य॒स्मिन्न् । \newline
36. ऋ॒द्ध्नु॒व॒ न्त्य॒स्मिन् न॒स्मिन् नृ॑द्ध्नुव न्त्यृद्ध्नुव न्त्य॒स्मिꣳश्च॑ चा॒स्मिन् नृ॑द्ध्नुव न्त्यृद्ध्नुव न्त्य॒स्मिꣳश्च॑ । \newline
37. अ॒स्मिꣳश्च॑ चा॒स्मिन् न॒स्मिꣳ श्चा॒मुष्मि॑न् न॒मुष्मिꣳ॑ श्चा॒स्मिन् न॒स्मिꣳ श्चा॒मुष्मिन्न्॑ । \newline
38. चा॒मुष्मि॑न् न॒मुष्मिꣳ॑श्च चा॒मुष्मिꣳ॑श्च चा॒मुष्मिꣳ॑श्च चा॒मुष्मिꣳ॑श्च । \newline
39. अ॒मुष्मिꣳ॑श्च चा॒मुष्मि॑न् न॒मुष्मिꣳ॑श्च चतुर्दशरा॒त्र श्च॑तुर्दशरा॒त्र श्चा॒मुष्मि॑न् न॒मुष्मिꣳ॑ श्च चतुर्दशरा॒त्रः । \newline
40. च॒ च॒तु॒र्द॒श॒रा॒त्र श्च॑तुर्दशरा॒त्रश्च॑ च चतुर्दशरा॒त्रो भ॑वति भवति चतुर्दशरा॒त्र श्च॑ च चतुर्दशरा॒त्रो भ॑वति । \newline
41. च॒तु॒र्द॒श॒रा॒त्रो भ॑वति भवति चतुर्दशरा॒त्र श्च॑तुर्दशरा॒त्रो भ॑वति स॒प्त स॒प्त भ॑वति चतुर्दशरा॒त्र श्च॑तुर्दशरा॒त्रो भ॑वति स॒प्त । \newline
42. च॒तु॒र्द॒श॒रा॒त्र इति॑ चतुर्दश - रा॒त्रः । \newline
43. भ॒व॒ति॒ स॒प्त स॒प्त भ॑वति भवति स॒प्त ग्रा॒म्या ग्रा॒म्याः स॒प्त भ॑वति भवति स॒प्त ग्रा॒म्याः । \newline
44. स॒प्त ग्रा॒म्या ग्रा॒म्याः स॒प्त स॒प्त ग्रा॒म्या ओष॑धय॒ ओष॑धयो ग्रा॒म्याः स॒प्त स॒प्त ग्रा॒म्या ओष॑धयः । \newline
45. ग्रा॒म्या ओष॑धय॒ ओष॑धयो ग्रा॒म्या ग्रा॒म्या ओष॑धयः स॒प्त स॒प्तौष॑धयो ग्रा॒म्या ग्रा॒म्या ओष॑धयः स॒प्त । \newline
46. ओष॑धयः स॒प्त स॒प्तौष॑धय॒ ओष॑धयः स॒प्तार॒ण्या आ॑र॒ण्याः स॒प्तौष॑धय॒ ओष॑धयः स॒प्तार॒ण्याः । \newline
47. स॒प्तार॒ण्या आ॑र॒ण्याः स॒प्त स॒प्तार॒ण्या उ॒भयी॑षा मु॒भयी॑षा मार॒ण्याः स॒प्त स॒प्तार॒ण्या उ॒भयी॑षाम् । \newline
48. आ॒र॒ण्या उ॒भयी॑षा मु॒भयी॑षा मार॒ण्या आ॑र॒ण्या उ॒भयी॑षा॒ मव॑रुद्ध्या॒ अव॑रुद्ध्या उ॒भयी॑षा मार॒ण्या आ॑र॒ण्या उ॒भयी॑षा॒ मव॑रुद्ध्यै । \newline
49. उ॒भयी॑षा॒ मव॑रुद्ध्या॒ अव॑रुद्ध्या उ॒भयी॑षा मु॒भयी॑षा॒ मव॑रुद्ध्यै॒ यद् यदव॑रुद्ध्या उ॒भयी॑षा मु॒भयी॑षा॒ मव॑रुद्ध्यै॒ यत् । \newline
50. अव॑रुद्ध्यै॒ यद् यदव॑रुद्ध्या॒ अव॑रुद्ध्यै॒ यत् प॑रा॒चीना॑नि परा॒चीना॑नि॒ यदव॑रुद्ध्या॒ अव॑रुद्ध्यै॒ यत् प॑रा॒चीना॑नि । \newline
51. अव॑रुद्ध्या॒ इत्यव॑ - रु॒द्ध्यै॒ । \newline
52. यत् प॑रा॒चीना॑नि परा॒चीना॑नि॒ यद् यत् प॑रा॒चीना॑नि पृ॒ष्ठानि॑ पृ॒ष्ठानि॑ परा॒चीना॑नि॒ यद् यत् प॑रा॒चीना॑नि पृ॒ष्ठानि॑ । \newline
53. प॒रा॒चीना॑नि पृ॒ष्ठानि॑ पृ॒ष्ठानि॑ परा॒चीना॑नि परा॒चीना॑नि पृ॒ष्ठानि॒ भव॑न्ति॒ भव॑न्ति पृ॒ष्ठानि॑ परा॒चीना॑नि परा॒चीना॑नि पृ॒ष्ठानि॒ भव॑न्ति । \newline
54. पृ॒ष्ठानि॒ भव॑न्ति॒ भव॑न्ति पृ॒ष्ठानि॑ पृ॒ष्ठानि॒ भव॑ न्त्य॒मु म॒मुम् भव॑न्ति पृ॒ष्ठानि॑ पृ॒ष्ठानि॒ भव॑ न्त्य॒मुम् । \newline
\pagebreak
\markright{ TS 7.3.4.2  \hfill https://www.vedavms.in \hfill}

\section{ TS 7.3.4.2 }

\textbf{TS 7.3.4.2 } \newline
\textbf{Samhita Paata} \newline

भव॑न्त्य॒मुमे॒व तैर्लो॒कम॒भि ज॑यन्ति॒ यत् प्र॑ती॒चीना॑नि पृ॒ष्ठानि॒ भव॑न्ती॒ममे॒व तैर्लो॒कम॒भि ज॑यन्ति त्रयस्त्रिꣳ॒॒शौ म॑द्ध्य॒तः स्तोमौ॑ भवतः॒ साम्रा᳚ज्यमे॒व ग॑च्छन्त्यधिरा॒जौ भ॑वतोऽधिरा॒जा ए॒व स॑मा॒नानां᳚ भवन्त्यतिरा॒त्राव॒भितो॑ भवतः॒ परि॑गृहीत्यै ॥ \newline

\textbf{Pada Paata} \newline

भव॑न्ति । अ॒मुम् । ए॒व । तैः । लो॒कम् । अ॒भीति॑ । ज॒य॒न्ति॒ । यत् । प्र॒ती॒चीना॑नि । पृ॒ष्ठानि॑ । भव॑न्ति । इ॒मम् । ए॒व । तैः । लो॒कम् । अ॒भीति॑ । ज॒य॒न्ति॒ । त्र॒य॒स्त्रिꣳ॒॒शाविति॑ त्रयः - त्रिꣳ॒॒शौ । म॒द्ध्य॒तः । स्तोमौ᳚ । भ॒व॒तः॒ । साम्रा᳚ज्य॒मिति॒ सां - रा॒ज्य॒म् । ए॒व । ग॒च्छ॒न्ति॒ । अ॒धि॒रा॒जावित्य॑धि - रा॒जौ । भ॒व॒तः॒ । अ॒धि॒रा॒जा इत्य॑धि - रा॒जाः । ए॒व । स॒मा॒नाना᳚म् । भ॒व॒न्ति॒ । अ॒ति॒रा॒त्रावित्य॑ति - रा॒त्रौ । अ॒भितः॑ । भ॒व॒तः॒ । परि॑गृहीत्या॒ इति॒ परि॑ - गृ॒ही॒त्यै॒ ॥  \newline


\textbf{Krama Paata} \newline

भव॑न्त्य॒मुम् । अ॒मुमे॒व । ए॒व तैः । तैर् लो॒कम् । लो॒कम॒भि । अ॒भि ज॑यन्ति । ज॒य॒न्ति॒ यत् । यत् प्र॑ती॒चीना॑नि । प्र॒ती॒चीना॑नि पृ॒ष्ठानि॑ । पृ॒ष्ठानि॒ भव॑न्ति । भव॑न्ती॒मम् । इ॒ममे॒व । ए॒व तैः । तैर् लो॒कम् । लो॒कम॒भि । अ॒भि ज॑यन्ति । ज॒य॒न्ति॒ त्र॒य॒स्त्रिꣳ॒॒शौ । त्र॒य॒स्त्रिꣳ॒॒शौ म॑द्ध्य॒तः । त्र॒य॒स्त्रिꣳ॒॒शाविति॑ त्रयः - त्रिꣳ॒॒शौ । म॒द्ध्य॒तः स्तौमौ᳚ । स्तौमौ॑ भवतः । भ॒व॒तः॒ साम्रा᳚ज्यम् । साम्रा᳚ज्यमे॒व । साम्रा᳚ज्य॒मिति॒ साम् - रा॒ज्य॒म् । ए॒व ग॑च्छन्ति । ग॒च्छ॒न्त्य॒धि॒रा॒जौ । अ॒धि॒रा॒जौ भ॑वतः । अ॒धि॒रा॒जावित्य॑धि - रा॒जौ । भ॒व॒तो॒ऽधि॒रा॒जाः । अ॒धि॒रा॒जा ए॒व । अ॒धि॒रा॒जा इत्य॑धि - रा॒जाः । ए॒व स॑मा॒नाना᳚म् । स॒मा॒नाना᳚म् भवन्ति । भ॒व॒न्त्य॒ति॒रा॒त्रौ । अ॒ति॒रा॒त्राव॒भितः॑ । अ॒ति॒रा॒त्रावित्य॑ति - रा॒त्रौ । अ॒भितो॑ भवतः । भ॒व॒तः॒ परि॑गृहीत्यै । परि॑गृहीत्या॒ इति॒ परि॑ - गृ॒ही॒त्यै॒ । \newline

\textbf{Jatai Paata} \newline

1. भव॑ न्त्य॒मु म॒मुम् भव॑न्ति॒ भव॑ न्त्य॒मुम् । \newline
2. अ॒मु मे॒वै वामु म॒मु मे॒व । \newline
3. ए॒व तै स्तै रे॒वैव तैः । \newline
4. तैर् लो॒कम् ॅलो॒कम् तै स्तैर् लो॒कम् । \newline
5. लो॒क म॒भ्य॑भि लो॒कम् ॅलो॒क म॒भि । \newline
6. अ॒भि ज॑यन्ति जय न्त्य॒भ्य॑भि ज॑यन्ति । \newline
7. ज॒य॒न्ति॒ यद् यज् ज॑यन्ति जयन्ति॒ यत् । \newline
8. यत् प्र॑ती॒चीना॑नि प्रती॒चीना॑नि॒ यद् यत् प्र॑ती॒चीना॑नि । \newline
9. प्र॒ती॒चीना॑नि पृ॒ष्ठानि॑ पृ॒ष्ठानि॑ प्रती॒चीना॑नि प्रती॒चीना॑नि पृ॒ष्ठानि॑ । \newline
10. पृ॒ष्ठानि॒ भव॑न्ति॒ भव॑न्ति पृ॒ष्ठानि॑ पृ॒ष्ठानि॒ भव॑न्ति । \newline
11. भव॑न्ती॒म मि॒मम् भव॑न्ति॒ भव॑न्ती॒मम् । \newline
12. इ॒म मे॒वैवेम मि॒म मे॒व । \newline
13. ए॒व तै स्तै रे॒वैव तैः । \newline
14. तैर् लो॒कम् ॅलो॒कम् तै स्तैर् लो॒कम् । \newline
15. लो॒क म॒भ्य॑भि लो॒कम् ॅलो॒क म॒भि । \newline
16. अ॒भि ज॑यन्ति जय न्त्य॒भ्य॑भि ज॑यन्ति । \newline
17. ज॒य॒न्ति॒ त्र॒य॒स्त्रिꣳ॒॒शौ त्र॑यस्त्रिꣳ॒॒शौ ज॑यन्ति जयन्ति त्रयस्त्रिꣳ॒॒शौ । \newline
18. त्र॒य॒स्त्रिꣳ॒॒शौ म॑द्ध्य॒तो म॑द्ध्य॒त स्त्र॑यस्त्रिꣳ॒॒शौ त्र॑यस्त्रिꣳ॒॒शौ म॑द्ध्य॒तः । \newline
19. त्र॒य॒स्त्रिꣳ॒॒शाविति॑ त्रयः - त्रिꣳ॒॒शौ । \newline
20. म॒द्ध्य॒तः स्तोमौ॒ स्तोमौ॑ मद्ध्य॒तो म॑द्ध्य॒तः स्तोमौ᳚ । \newline
21. स्तोमौ॑ भवतो भवतः॒ स्तोमौ॒ स्तोमौ॑ भवतः । \newline
22. भ॒व॒तः॒ साम्रा᳚ज्यꣳ॒॒ साम्रा᳚ज्यम् भवतो भवतः॒ साम्रा᳚ज्यम् । \newline
23. साम्रा᳚ज्य मे॒वैव साम्रा᳚ज्यꣳ॒॒ साम्रा᳚ज्य मे॒व । \newline
24. साम्रा᳚ज्य॒मिति॒ सां - रा॒ज्य॒म् । \newline
25. ए॒व ग॑च्छन्ति गच्छ न्त्ये॒वैव ग॑च्छन्ति । \newline
26. ग॒च्छ॒ न्त्य॒धि॒रा॒जा व॑धिरा॒जौ ग॑च्छन्ति गच्छ न्त्यधिरा॒जौ । \newline
27. अ॒धि॒रा॒जौ भ॑वतो भवतो ऽधिरा॒जा व॑धिरा॒जौ भ॑वतः । \newline
28. अ॒धि॒रा॒जावित्य॑धि - रा॒जौ । \newline
29. भ॒व॒तो॒ ऽधि॒रा॒जा अ॑धिरा॒जा भ॑वतो भवतो ऽधिरा॒जाः । \newline
30. अ॒धि॒रा॒जा ए॒वैवा धि॑रा॒जा अ॑धिरा॒जा ए॒व । \newline
31. अ॒धि॒रा॒जा इत्य॑धि - रा॒जाः । \newline
32. ए॒व स॑मा॒नानाꣳ॑ समा॒नाना॑ मे॒वैव स॑मा॒नाना᳚म् । \newline
33. स॒मा॒नाना᳚म् भवन्ति भवन्ति समा॒नानाꣳ॑ समा॒नाना᳚म् भवन्ति । \newline
34. भ॒व॒ न्त्य॒ति॒रा॒त्रा व॑तिरा॒त्रौ भ॑वन्ति भव न्त्यतिरा॒त्रौ । \newline
35. अ॒ति॒रा॒त्रा व॒भितो॒ ऽभितो॑ ऽतिरा॒त्रा व॑तिरा॒त्रा व॒भितः॑ । \newline
36. अ॒ति॒रा॒त्रावित्य॑ति - रा॒त्रौ । \newline
37. अ॒भितो॑ भवतो भवतो॒ ऽभितो॒ ऽभितो॑ भवतः । \newline
38. भ॒व॒तः॒ परि॑गृहीत्यै॒ परि॑गृहीत्यै भवतो भवतः॒ परि॑गृहीत्यै । \newline
39. परि॑गृहीत्या॒ इति॒ परि॑ - गृ॒ही॒त्यै॒ । \newline

\textbf{Ghana Paata } \newline

1. भव॑ न्त्य॒मु म॒मुम् भव॑न्ति॒ भव॑ न्त्य॒मु मे॒वैवामुम् भव॑न्ति॒ भव॑ न्त्य॒मु मे॒व । \newline
2. अ॒मु मे॒वै वामु म॒मु मे॒व तै स्तै रे॒वामु म॒मु मे॒व तैः । \newline
3. ए॒व तै स्तै रे॒वैव तैर् लो॒कम् ॅलो॒कम् तै रे॒वैव तैर् लो॒कम् । \newline
4. तैर् लो॒कम् ॅलो॒कम् तै स्तैर् लो॒क म॒भ्य॑भि लो॒कम् तै स्तैर् लो॒क म॒भि । \newline
5. लो॒क म॒भ्य॑भि लो॒कम् ॅलो॒क म॒भि ज॑यन्ति जयन्त्य॒भि लो॒कम् ॅलो॒क म॒भि ज॑यन्ति । \newline
6. अ॒भि ज॑यन्ति जयन्त्य॒भ्य॑भि ज॑यन्ति॒ यद् यज् ज॑यन्त्य॒भ्य॑भि ज॑यन्ति॒ यत् । \newline
7. ज॒य॒न्ति॒ यद् यज् ज॑यन्ति जयन्ति॒ यत् प्र॑ती॒चीना॑नि प्रती॒चीना॑नि॒ यज् ज॑यन्ति जयन्ति॒ यत् प्र॑ती॒चीना॑नि । \newline
8. यत् प्र॑ती॒चीना॑नि प्रती॒चीना॑नि॒ यद् यत् प्र॑ती॒चीना॑नि पृ॒ष्ठानि॑ पृ॒ष्ठानि॑ प्रती॒चीना॑नि॒ यद् यत् प्र॑ती॒चीना॑नि पृ॒ष्ठानि॑ । \newline
9. प्र॒ती॒चीना॑नि पृ॒ष्ठानि॑ पृ॒ष्ठानि॑ प्रती॒चीना॑नि प्रती॒चीना॑नि पृ॒ष्ठानि॒ भव॑न्ति॒ भव॑न्ति पृ॒ष्ठानि॑ प्रती॒चीना॑नि प्रती॒चीना॑नि पृ॒ष्ठानि॒ भव॑न्ति । \newline
10. पृ॒ष्ठानि॒ भव॑न्ति॒ भव॑न्ति पृ॒ष्ठानि॑ पृ॒ष्ठानि॒ भव॑न्ती॒म मि॒मम् भव॑न्ति पृ॒ष्ठानि॑ पृ॒ष्ठानि॒ भव॑न्ती॒मम् । \newline
11. भव॑न्ती॒म मि॒मम् भव॑न्ति॒ भव॑न्ती॒म मे॒वैवेमम् भव॑न्ति॒ भव॑न्ती॒म मे॒व । \newline
12. इ॒म मे॒वैवेम मि॒म मे॒व तै स्तै रे॒वेम मि॒म मे॒व तैः । \newline
13. ए॒व तै स्तै रे॒वैव तैर् लो॒कम् ॅलो॒कम् तै रे॒वैव तैर् लो॒कम् । \newline
14. तैर् लो॒कम् ॅलो॒कम् तै स्तैर् लो॒क म॒भ्य॑भि लो॒कम् तै स्तैर् लो॒क म॒भि । \newline
15. लो॒क म॒भ्य॑भि लो॒कम् ॅलो॒क म॒भि ज॑यन्ति जयन्त्य॒भि लो॒कम् ॅलो॒क म॒भि ज॑यन्ति । \newline
16. अ॒भि ज॑यन्ति जय न्त्य॒भ्य॑भि ज॑यन्ति त्रयस्त्रिꣳ॒॒शौ त्र॑यस्त्रिꣳ॒॒शौ ज॑यन्त्य॒भ्य॑भि ज॑यन्ति त्रयस्त्रिꣳ॒॒शौ । \newline
17. ज॒य॒न्ति॒ त्र॒य॒स्त्रिꣳ॒॒शौ त्र॑यस्त्रिꣳ॒॒शौ ज॑यन्ति जयन्ति त्रयस्त्रिꣳ॒॒शौ म॑द्ध्य॒तो म॑द्ध्य॒त स्त्र॑यस्त्रिꣳ॒॒शौ ज॑यन्ति जयन्ति त्रयस्त्रिꣳ॒॒शौ म॑द्ध्य॒तः । \newline
18. त्र॒य॒स्त्रिꣳ॒॒शौ म॑द्ध्य॒तो म॑द्ध्य॒त स्त्र॑यस्त्रिꣳ॒॒शौ त्र॑यस्त्रिꣳ॒॒शौ म॑द्ध्य॒तः स्तोमौ॒ स्तोमौ॑ मद्ध्य॒त स्त्र॑यस्त्रिꣳ॒॒शौ त्र॑यस्त्रिꣳ॒॒शौ म॑द्ध्य॒तः स्तोमौ᳚ । \newline
19. त्र॒य॒स्त्रिꣳ॒॒शाविति॑ त्रयः - त्रिꣳ॒॒शौ । \newline
20. म॒द्ध्य॒तः स्तोमौ॒ स्तोमौ॑ मद्ध्य॒तो म॑द्ध्य॒तः स्तोमौ॑ भवतो भवतः॒ स्तोमौ॑ मद्ध्य॒तो म॑द्ध्य॒तः स्तोमौ॑ भवतः । \newline
21. स्तोमौ॑ भवतो भवतः॒ स्तोमौ॒ स्तोमौ॑ भवतः॒ साम्रा᳚ज्यꣳ॒॒ साम्रा᳚ज्यम् भवतः॒ स्तोमौ॒ स्तोमौ॑ भवतः॒ साम्रा᳚ज्यम् । \newline
22. भ॒व॒तः॒ साम्रा᳚ज्यꣳ॒॒ साम्रा᳚ज्यम् भवतो भवतः॒ साम्रा᳚ज्य मे॒वैव साम्रा᳚ज्यम् भवतो भवतः॒ साम्रा᳚ज्य मे॒व । \newline
23. साम्रा᳚ज्य मे॒वैव साम्रा᳚ज्यꣳ॒॒ साम्रा᳚ज्य मे॒व ग॑च्छन्ति गच्छन्त्ये॒व साम्रा᳚ज्यꣳ॒॒ साम्रा᳚ज्य मे॒व ग॑च्छन्ति । \newline
24. साम्रा᳚ज्य॒मिति॒ सां - रा॒ज्य॒म् । \newline
25. ए॒व ग॑च्छन्ति गच्छ न्त्ये॒वैव ग॑च्छ न्त्यधिरा॒जा व॑धिरा॒जौ ग॑च्छ न्त्ये॒वैव ग॑च्छ न्त्यधिरा॒जौ । \newline
26. ग॒च्छ॒ न्त्य॒धि॒रा॒जा व॑धिरा॒जौ ग॑च्छन्ति गच्छ न्त्यधिरा॒जौ भ॑वतो भवतो ऽधिरा॒जौ ग॑च्छन्ति गच्छ
न्त्यधिरा॒जौ भ॑वतः । \newline
27. अ॒धि॒रा॒जौ भ॑वतो भवतो ऽधिरा॒जा व॑धिरा॒जौ भ॑वतो ऽधिरा॒जा अ॑धिरा॒जा भ॑वतो ऽधिरा॒जा व॑धिरा॒जौ भ॑वतो ऽधिरा॒जाः । \newline
28. अ॒धि॒रा॒जावित्य॑धि - रा॒जौ । \newline
29. भ॒व॒तो॒ ऽधि॒रा॒जा अ॑धिरा॒जा भ॑वतो भवतो ऽधिरा॒जा ए॒वै वाधि॑रा॒जा भ॑वतो भवतो ऽधिरा॒जा ए॒व । \newline
30. अ॒धि॒रा॒जा ए॒वै वाधि॑रा॒जा अ॑धिरा॒जा ए॒व स॑मा॒नानाꣳ॑ समा॒नाना॑ मे॒वाधि॑रा॒जा अ॑धिरा॒जा ए॒व स॑मा॒नाना᳚म् । \newline
31. अ॒धि॒रा॒जा इत्य॑धि - रा॒जाः । \newline
32. ए॒व स॑मा॒नानाꣳ॑ समा॒नाना॑ मे॒वैव स॑मा॒नाना᳚म् भवन्ति भवन्ति समा॒नाना॑ मे॒वैव स॑मा॒नाना᳚म् भवन्ति । \newline
33. स॒मा॒नाना᳚म् भवन्ति भवन्ति समा॒नानाꣳ॑ समा॒नाना᳚म् भव न्त्यतिरा॒त्रा व॑तिरा॒त्रौ भ॑वन्ति समा॒नानाꣳ॑ समा॒नाना᳚म् भव न्त्यतिरा॒त्रौ । \newline
34. भ॒व॒ न्त्य॒ति॒रा॒त्रा व॑तिरा॒त्रौ भ॑वन्ति भव न्त्यतिरा॒त्रा व॒भितो॒ ऽभितो॑ ऽतिरा॒त्रौ भ॑वन्ति भव न्त्यतिरा॒त्रा व॒भितः॑ । \newline
35. अ॒ति॒रा॒त्रा व॒भितो॒ ऽभितो॑ ऽतिरा॒त्रा व॑तिरा॒त्रा व॒भितो॑ भवतो भवतो॒ ऽभितो॑ ऽतिरा॒त्रा व॑तिरा॒त्रा व॒भितो॑ भवतः । \newline
36. अ॒ति॒रा॒त्रावित्य॑ति - रा॒त्रौ । \newline
37. अ॒भितो॑ भवतो भवतो॒ ऽभितो॒ ऽभितो॑ भवतः॒ परि॑गृहीत्यै॒ परि॑गृहीत्यै भवतो॒ ऽभितो॒ ऽभितो॑ भवतः॒ परि॑गृहीत्यै । \newline
38. भ॒व॒तः॒ परि॑गृहीत्यै॒ परि॑गृहीत्यै भवतो भवतः॒ परि॑गृहीत्यै । \newline
39. परि॑गृहीत्या॒ इति॒ परि॑ - गृ॒ही॒त्यै॒ । \newline
\pagebreak
\markright{ TS 7.3.5.1  \hfill https://www.vedavms.in \hfill}

\section{ TS 7.3.5.1 }

\textbf{TS 7.3.5.1 } \newline
\textbf{Samhita Paata} \newline

प्र॒जाप॑तिः सुव॒र्गं ॅलो॒कमै॒त् तं दे॒वा अन्वा॑य॒न् ताना॑दि॒त्याश्च॑ प॒शव॒श्चान्वा॑य॒न् ते दे॒वा अ॑ब्रुव॒न॒. यान् प॒शूनु॒पाजी॑विष्म॒ त इ॒मे᳚ ऽन्वाग्म॒न्निति॒ तेभ्य॑ ए॒तं च॑तुर्दशरा॒त्रं प्रत्यौ॑ह॒न् त आ॑दि॒त्याः पृ॒ष्ठैः सु॑व॒र्गं ॅलो॒कमाऽरो॑हन् त्र्य॒हाभ्या॑म॒स्मिन् ॅलो॒के प॒शून् प्रत्यौ॑हन् पृ॒ष्ठैरा॑दि॒त्या अ॒मुष्मि॑न् ॅलो॒क आर्द्ध्नु॑वन् त्र्य॒हाभ्या॑म॒स्मिन् - [  ] \newline

\textbf{Pada Paata} \newline

प्र॒जाप॑ति॒रिति॑ प्र॒जा - प॒तिः॒ । सु॒व॒र्गमिति॑ सुवः - गम् । लो॒कम् । ऐ॒त् । तम् । दे॒वाः । अन्विति॑ । आ॒य॒न्न् । तान् । आ॒दि॒त्याः । च॒ । प॒शवः॑ । च॒ । अन्विति॑ । आ॒य॒न्न् । ते । दे॒वाः । अ॒ब्रु॒व॒न्. । यान् । प॒शून् । उ॒पाजी॑वि॒ष्मेत्यु॑प - अजी॑विष्म । ते । इ॒मे । अ॒न्वाग्म॒न्नित्य॑नु-आग्मन्न्॑ । इति॑ । तेभ्यः॑ । ए॒तम् । च॒तु॒र्द॒श॒रा॒त्रमिति॑ चतुर्दश-रा॒त्रम् । प्रतीति॑ । औ॒ह॒न्न् । ते । आ॒दि॒त्याः । पृ॒ष्ठैः । सु॒व॒र्गमिति॑ सुवः - गम् । लो॒कम् । एति॑ । अ॒रो॒ह॒न्न् । त्र्य॒हाभ्या॒मिति॑ त्रि - अ॒हाभ्या᳚म् । अ॒स्मिन्न् । लो॒के । प॒शून् । प्रतीति॑ । औ॒ह॒न्न् । पृ॒ष्ठैः । आ॒दि॒त्याः । अ॒मुष्मिन्न्॑ । लो॒के । आद्‌र्ध्नु॑वन्न् । त्र्य॒हाभ्या॒मिति॑ त्रि - अ॒हाभ्या᳚म् । अ॒स्मिन्न् ।  \newline


\textbf{Krama Paata} \newline

प्र॒जाप॑तिः सुव॒र्गम् । प्र॒जाप॑ति॒रिति॑ प्र॒जा - प॒तिः॒ । सु॒व॒र्गम् ॅलो॒कम् । सु॒व॒र्गमिति॑ सुवः - गम् । लो॒कमै᳚त् । ऐ॒त् तम् । तम् दे॒वाः । दे॒वा अनु॑ । अन्वा॑यन्न् । आ॒य॒न् तान् । ताना॑दि॒त्याः । आ॒दि॒त्याश्च॑ । च॒ प॒शवः॑ । प॒शव॑श्च । चानु॑ । अन्वा॑यन्न् । आ॒य॒न् ते । ते दे॒वाः । दे॒वा अ॑ब्रुवन्न् । अ॒ब्रु॒व॒न्॒. यान् । यान् प॒शून् । प॒शूनु॒पाजी॑विष्म । उ॒पाजी॑विष्म॒ ते । उ॒पाजी॑वि॒ष्मेत्यु॑प - अजी॑विष्म । त इ॒मे । इ॒मे᳚ऽन्वाग्मन्न्॑ । अ॒न्वाग्म॒न्निति॑ । अ॒न्वाग्म॒न्नित्य॑नु - आग्मन्न्॑ । इति॒ तेभ्यः॑ । तेभ्य॑ ए॒तम् । ए॒तम् च॑तुर्दशरा॒त्रम् । च॒तु॒र्द॒श॒रा॒त्रम् प्रति॑ । च॒तु॒र्द॒श॒रा॒त्रमिति॑ चतुर्दश - रा॒त्रम् । प्रत्यौ॑हन्न् । औ॒ह॒न् ते । त आ॑दि॒त्याः । आ॒दि॒त्याः पृ॒ष्ठैः । पृ॒ष्ठैः सु॑व॒र्गम् । सु॒व॒र्गम् ॅलो॒कम् । सु॒व॒र्गमिति॑ सुवः - गम् । लो॒कमा । आऽरो॑हन्न् । अ॒रो॒ह॒न् त्र्य॒हाभ्या᳚म् । त्र्य॒हाभ्या॑म॒स्मिन्न् । त्र्य॒हाभ्या॒मिति॑ त्रि - अ॒हाभ्या᳚म् । अ॒स्मिन् ॅलो॒के । लो॒के प॒शून् । प॒शून् प्रति॑ । प्रत्यौ॑हन्न् । औ॒ह॒न् पृ॒ष्ठैः । पृ॒ष्ठैरा॑दि॒त्याः । आ॒दि॒त्या अ॒मुष्मिन्न्॑ । अ॒मुष्मि॑न् ॅलो॒के । लो॒क आर्द्ध्नु॑वन्न् । आर्द्ध्नु॑वन् त्र्य॒हाभ्या᳚म् । त्र्य॒हाभ्या॑म॒स्मिन्न् । त्र्य॒हाभ्या॒मिति॑ त्रि - अ॒हाभ्या᳚म् । अ॒स्मिन् ॅलो॒के \newline

\textbf{Jatai Paata} \newline

1. प्र॒जाप॑तिः सुव॒र्गꣳ सु॑व॒र्गम् प्र॒जाप॑तिः प्र॒जाप॑तिः सुव॒र्गम् । \newline
2. प्र॒जाप॑ति॒रिति॑ प्र॒जा - प॒तिः॒ । \newline
3. सु॒व॒र्गम् ॅलो॒कम् ॅलो॒कꣳ सु॑व॒र्गꣳ सु॑व॒र्गम् ॅलो॒कम् । \newline
4. सु॒व॒र्गमिति॑ सुवः - गम् । \newline
5. लो॒क मै॑दै ल्लो॒कम् ॅलो॒क मै᳚त् । \newline
6. ऐ॒त् तम् त मै॑दै॒त् तम् । \newline
7. तम् दे॒वा दे॒वा स्तम् तम् दे॒वाः । \newline
8. दे॒वा अन्वनु॑ दे॒वा दे॒वा अनु॑ । \newline
9. अन्वा॑यन् नाय॒न् नन्व न्वा॑यन्न् । \newline
10. आ॒य॒न् ताꣳ स्ता ना॑यन् नाय॒न् तान् । \newline
11. ताना॑दि॒त्या आ॑दि॒त्या स्ताꣳ स्ताना॑दि॒त्याः । \newline
12. आ॒दि॒त्या श्च॑ चादि॒त्या आ॑दि॒त्या श्च॑ । \newline
13. च॒ प॒शवः॑ प॒शव॑ श्च च प॒शवः॑ । \newline
14. प॒शव॑ श्च च प॒शवः॑ प॒शव॑ श्च । \newline
15. चान्वनु॑ च॒ चानु॑ । \newline
16. अन्वा॑यन् नाय॒न् नन्व न्वा॑यन्न् । \newline
17. आ॒य॒न् ते त आ॑यन् नाय॒न् ते । \newline
18. ते दे॒वा दे॒वा स्ते ते दे॒वाः । \newline
19. दे॒वा अ॑ब्रुवन् नब्रुवन् दे॒वा दे॒वा अ॑ब्रुवन्न् । \newline
20. अ॒ब्रु॒व॒न्॒. यान्. या न॑ब्रुवन् नब्रुव॒न्॒. यान् । \newline
21. यान् प॒शून् प॒शून्. यान्. यान् प॒शून् । \newline
22. प॒शू नु॒पाजी॑विष् मो॒पाजी॑विष्म प॒शून् प॒शू नु॒पाजी॑विष्म । \newline
23. उ॒पाजी॑विष्म॒ ते त उ॒पाजी॑वि ष्मो॒पाजी॑विष्म॒ ते । \newline
24. उ॒पाजी॑वि॒ष्मेत्यु॑प - अजी॑विष्म । \newline
25. त इ॒म इ॒मे ते त इ॒मे । \newline
26. इ॒मे᳚ ऽन्वाग्म॑न् न॒न्वाग्म॑न् नि॒म इ॒मे᳚ ऽन्वाग्मन्न्॑ । \newline
27. अ॒न्वाग्म॒न् निती त्य॒न्वाग्म॑न् न॒न्वाग्म॒न् निति॑ । \newline
28. अ॒न्वाग्म॒न्नित्य॑नु - आग्मन्न्॑ । \newline
29. इति॒ तेभ्य॒ स्तेभ्य॒ इतीति॒ तेभ्यः॑ । \newline
30. तेभ्य॑ ए॒त मे॒तम् तेभ्य॒ स्तेभ्य॑ ए॒तम् । \newline
31. ए॒तम् च॑तुर्दशरा॒त्रम् च॑तुर्दशरा॒त्र मे॒त मे॒तम् च॑तुर्दशरा॒त्रम् । \newline
32. च॒तु॒र्द॒श॒रा॒त्रम् प्रति॒ प्रति॑ चतुर्दशरा॒त्रम् च॑तुर्दशरा॒त्रम् प्रति॑ । \newline
33. च॒तु॒र्द॒श॒रा॒त्रमिति॑ चतुर्दश - रा॒त्रम् । \newline
34. प्रत्यौ॑हन् नौह॒न् प्रति॒ प्रत्यौ॑हन्न् । \newline
35. औ॒ह॒न् ते त औ॑हन् नौह॒न् ते । \newline
36. त आ॑दि॒त्या आ॑दि॒त्या स्ते त आ॑दि॒त्याः । \newline
37. आ॒दि॒त्याः पृ॒ष्ठैः पृ॒ष्ठै रा॑दि॒त्या आ॑दि॒त्याः पृ॒ष्ठैः । \newline
38. पृ॒ष्ठैः सु॑व॒र्गꣳ सु॑व॒र्गम् पृ॒ष्ठैः पृ॒ष्ठैः सु॑व॒र्गम् । \newline
39. सु॒व॒र्गम् ॅलो॒कम् ॅलो॒कꣳ सु॑व॒र्गꣳ सु॑व॒र्गम् ॅलो॒कम् । \newline
40. सु॒व॒र्गमिति॑ सुवः - गम् । \newline
41. लो॒क मा लो॒कम् ॅलो॒क मा । \newline
42. आ ऽरो॑हन् नरोह॒न् ना ऽरो॑हन्न् । \newline
43. अ॒रो॒ह॒न् त्र्य॒हाभ्या᳚म् त्र्य॒हाभ्या॑ मरोहन् नरोहन् त्र्य॒हाभ्या᳚म् । \newline
44. त्र्य॒हाभ्या॑ म॒स्मिन् न॒स्मिन् त्र्य॒हाभ्या᳚म् त्र्य॒हाभ्या॑ म॒स्मिन्न् । \newline
45. त्र्य॒हाभ्या॒मिति॑ त्रि - अ॒हाभ्या᳚म् । \newline
46. अ॒स्मिन् ॅलो॒के लो॒के᳚ ऽस्मिन् न॒स्मिन् ॅलो॒के । \newline
47. लो॒के प॒शून् प॒शून् ॅलो॒के लो॒के प॒शून् । \newline
48. प॒शून् प्रति॒ प्रति॑ प॒शून् प॒शून् प्रति॑ । \newline
49. प्रत्यौ॑हन् नौह॒न् प्रति॒ प्रत्यौ॑हन्न् । \newline
50. औ॒ह॒न् पृ॒ष्ठैः पृ॒ष्ठै रौ॑हन् नौहन् पृ॒ष्ठैः । \newline
51. पृ॒ष्ठै रा॑दि॒त्या आ॑दि॒त्याः पृ॒ष्ठैः पृ॒ष्ठै रा॑दि॒त्याः । \newline
52. आ॒दि॒त्या अ॒मुष्मि॑न् न॒मुष्मि॑न् नादि॒त्या आ॑दि॒त्या अ॒मुष्मिन्न्॑ । \newline
53. अ॒मुष्मि॑न् ॅलो॒के लो॒के॑ ऽमुष्मि॑न् न॒मुष्मि॑न् ॅलो॒के । \newline
54. लो॒क आर्द्ध्नु॑व॒न् नार्द्ध्नु॑वन् ॅलो॒के लो॒क आर्द्ध्नु॑वन्न् । \newline
55. आर्द्ध्नु॑वन् त्र्य॒हाभ्या᳚म् त्र्य॒हाभ्या॒ मार्द्ध्नु॑व॒न् नार्द्ध्नु॑वन् त्र्य॒हाभ्या᳚म् । \newline
56. त्र्य॒हाभ्या॑ म॒स्मिन् न॒स्मिन् त्र्य॒हाभ्या᳚म् त्र्य॒हाभ्या॑ म॒स्मिन्न् । \newline
57. त्र्य॒हाभ्या॒मिति॑ त्रि - अ॒हाभ्या᳚म् । \newline
58. अ॒स्मिन् ॅलो॒के लो॒के᳚ ऽस्मिन् न॒स्मिन् ॅलो॒के । \newline

\textbf{Ghana Paata } \newline

1. प्र॒जाप॑तिः सुव॒र्गꣳ सु॑व॒र्गम् प्र॒जाप॑तिः प्र॒जाप॑तिः सुव॒र्गम् ॅलो॒कम् ॅलो॒कꣳ सु॑व॒र्गम् प्र॒जाप॑तिः प्र॒जाप॑तिः सुव॒र्गम् ॅलो॒कम् । \newline
2. प्र॒जाप॑ति॒रिति॑ प्र॒जा - प॒तिः॒ । \newline
3. सु॒व॒र्गम् ॅलो॒कम् ॅलो॒कꣳ सु॑व॒र्गꣳ सु॑व॒र्गम् ॅलो॒क मै॑दै ल्लो॒कꣳ सु॑व॒र्गꣳ सु॑व॒र्गम् ॅलो॒क मै᳚त् । \newline
4. सु॒व॒र्गमिति॑ सुवः - गम् । \newline
5. लो॒क मै॑दै ल्लो॒कम् ॅलो॒क मै॒त् तम् तमै᳚ ल्लो॒कम् ॅलो॒क मै॒त् तम् । \newline
6. ऐ॒त् तम् त मै॑दै॒त् तम् दे॒वा दे॒वा स्त मै॑दै॒त् तम् दे॒वाः । \newline
7. तम् दे॒वा दे॒वा स्तम् तम् दे॒वा अन्वनु॑ दे॒वा स्तम् तम् दे॒वा अनु॑ । \newline
8. दे॒वा अन्वनु॑ दे॒वा दे॒वा अन्वा॑यन् नाय॒न् ननु॑ दे॒वा दे॒वा अन्वा॑यन्न् । \newline
9. अन्वा॑यन् नाय॒न् नन् वन् वा॑य॒न् ताꣳ स्ता ना॑य॒न् नन् वन् वा॑य॒न् तान् । \newline
10. आ॒य॒न् ताꣳ स्ता ना॑यन् नाय॒न् ताना॑दि॒त्या आ॑दि॒त्या स्ता ना॑यन् नाय॒न् ताना॑दि॒त्याः । \newline
11. ताना॑दि॒त्या आ॑दि॒त्या स्ताꣳ स्ता ना॑दि॒त्या श्च॑ चादि॒त्या स्ताꣳ स्ता ना॑दि॒त्या श्च॑ । \newline
12. आ॒दि॒त्या श्च॑ चादि॒त्या आ॑दि॒त्या श्च॑ प॒शवः॑ प॒शव॑ श्चादि॒त्या आ॑दि॒त्या श्च॑ प॒शवः॑ । \newline
13. च॒ प॒शवः॑ प॒शव॑ श्च च प॒शव॑ श्च च प॒शव॑ श्च च प॒शव॑ श्च । \newline
14. प॒शव॑ श्च च प॒शवः॑ प॒शव॒ श्चान्वनु॑ च प॒शवः॑ प॒शव॒ श्चानु॑ । \newline
15. चान्वनु॑ च॒ चान्वा॑यन् नाय॒न् ननु॑ च॒ चान्वा॑यन्न् । \newline
16. अन्वा॑यन् नाय॒न् नन् वन् वा॑य॒न् ते त आ॑य॒न् नन् वन् वा॑य॒न् ते । \newline
17. आ॒य॒न् ते त आ॑यन् नाय॒न् ते दे॒वा दे॒वा स्त आ॑यन् नाय॒न् ते दे॒वाः । \newline
18. ते दे॒वा दे॒वा स्ते ते दे॒वा अ॑ब्रुवन् नब्रुवन् दे॒वा स्ते ते दे॒वा अ॑ब्रुवन्न् । \newline
19. दे॒वा अ॑ब्रुवन् नब्रुवन् दे॒वा दे॒वा अ॑ब्रुव॒न्॒. यान्. यान॑ब्रुवन् दे॒वा दे॒वा अ॑ब्रुव॒न्॒. यान् । \newline
20. अ॒ब्रु॒व॒न्॒. यान्. यान॑ब्रुवन् नब्रुव॒न्॒. यान् प॒शून् प॒शून्. यान॑ब्रुवन् नब्रुव॒न्॒. यान् प॒शून् । \newline
21. यान् प॒शून् प॒शून्. यान्. यान् प॒शू नु॒पाजी॑वि ष्मो॒पाजी॑विष्म प॒शून्. यान्. यान् प॒शू नु॒पाजी॑विष्म । \newline
22. प॒शू नु॒पाजी॑वि ष्मो॒पाजी॑विष्म प॒शून् प॒शू नु॒पाजी॑विष्म॒ ते त उ॒पाजी॑विष्म प॒शून् प॒शू नु॒पाजी॑विष्म॒ ते । \newline
23. उ॒पाजी॑विष्म॒ ते त उ॒पाजी॑वि ष्मो॒पाजी॑विष्म॒ त इ॒म इ॒मे त उ॒पाजी॑वि ष्मो॒पाजी॑विष्म॒ त इ॒मे । \newline
24. उ॒पाजी॑वि॒ष्मेत्यु॑प - अजी॑विष्म । \newline
25. त इ॒म इ॒मे ते त इ॒मे᳚ ऽन्वाग्म॑न् न॒न्वाग्म॑न् नि॒मे ते त इ॒मे᳚ ऽन्वाग्मन्न्॑ । \newline
26. इ॒मे᳚ ऽन्वाग्म॑न् न॒न्वाग्म॑न् नि॒म इ॒मे᳚ ऽन्वाग्म॒न् निती त्य॒न्वाग्म॑न् नि॒म इ॒मे᳚ ऽन्वाग्म॒न् निति॑ । \newline
27. अ॒न्वाग्म॒न् निती त्य॒न्वाग्म॑न् न॒न्वाग्म॒न् निति॒ तेभ्य॒ स्तेभ्य॒ इत्य॒ न्वाग्म॑न् न॒न्वाग्म॒न् निति॒ तेभ्यः॑ । \newline
28. अ॒न्वाग्म॒न्नित्य॑नु - आग्मन्न्॑ । \newline
29. इति॒ तेभ्य॒ स्तेभ्य॒ इतीति॒ तेभ्य॑ ए॒त मे॒तम् तेभ्य॒ इतीति॒ तेभ्य॑ ए॒तम् । \newline
30. तेभ्य॑ ए॒त मे॒तम् तेभ्य॒ स्तेभ्य॑ ए॒तम् च॑तुर्दशरा॒त्रम् च॑तुर्दशरा॒त्र मे॒तम् तेभ्य॒ स्तेभ्य॑ ए॒तम् च॑तुर्दशरा॒त्रम् । \newline
31. ए॒तम् च॑तुर्दशरा॒त्रम् च॑तुर्दशरा॒त्र मे॒त मे॒तम् च॑तुर्दशरा॒त्रम् प्रति॒ प्रति॑ चतुर्दशरा॒त्र मे॒त मे॒तम् च॑तुर्दशरा॒त्रम् प्रति॑ । \newline
32. च॒तु॒र्द॒श॒रा॒त्रम् प्रति॒ प्रति॑ चतुर्दशरा॒त्रम् च॑तुर्दशरा॒त्रम् प्रत्यौ॑हन् नौह॒न् प्रति॑ चतुर्दशरा॒त्रम् च॑तुर्दशरा॒त्रम् प्रत्यौ॑हन्न् । \newline
33. च॒तु॒र्द॒श॒रा॒त्रमिति॑ चतुर्दश - रा॒त्रम् । \newline
34. प्रत्यौ॑हन् नौह॒न् प्रति॒ प्रत्यौ॑ह॒न् ते त औ॑ह॒न् प्रति॒ प्रत्यौ॑ह॒न् ते । \newline
35. औ॒ह॒न् ते त औ॑हन् नौह॒न् त आ॑दि॒त्या आ॑दि॒त्या स्त औ॑हन् नौह॒न् त आ॑दि॒त्याः । \newline
36. त आ॑दि॒त्या आ॑दि॒त्या स्ते त आ॑दि॒त्याः पृ॒ष्ठैः पृ॒ष्ठै रा॑दि॒त्या स्ते त आ॑दि॒त्याः पृ॒ष्ठैः । \newline
37. आ॒दि॒त्याः पृ॒ष्ठैः पृ॒ष्ठै रा॑दि॒त्या आ॑दि॒त्याः पृ॒ष्ठैः सु॑व॒र्गꣳ सु॑व॒र्गम् पृ॒ष्ठै रा॑दि॒त्या आ॑दि॒त्याः पृ॒ष्ठैः सु॑व॒र्गम् । \newline
38. पृ॒ष्ठैः सु॑व॒र्गꣳ सु॑व॒र्गम् पृ॒ष्ठैः पृ॒ष्ठैः सु॑व॒र्गम् ॅलो॒कम् ॅलो॒कꣳ सु॑व॒र्गम् पृ॒ष्ठैः पृ॒ष्ठैः सु॑व॒र्गम् ॅलो॒कम् । \newline
39. सु॒व॒र्गम् ॅलो॒कम् ॅलो॒कꣳ सु॑व॒र्गꣳ सु॑व॒र्गम् ॅलो॒क मा लो॒कꣳ सु॑व॒र्गꣳ सु॑व॒र्गम् ॅलो॒क मा । \newline
40. सु॒व॒र्गमिति॑ सुवः - गम् । \newline
41. लो॒क मा लो॒कम् ॅलो॒क मा ऽरो॑हन् नरोह॒न् ना लो॒कम् ॅलो॒क मा ऽरो॑हन्न् । \newline
42. आ ऽरो॑हन् नरोह॒न् ना ऽरो॑हन् त्र्य॒हाभ्या᳚म् त्र्य॒हाभ्या॑ मरोह॒न् ना ऽरो॑हन् त्र्य॒हाभ्या᳚म् । \newline
43. अ॒रो॒ह॒न् त्र्य॒हाभ्या᳚म् त्र्य॒हाभ्या॑ मरोहन् नरोहन् त्र्य॒हाभ्या॑ म॒स्मिन् न॒स्मिन् त्र्य॒हाभ्या॑ मरोहन् नरोहन् त्र्य॒हाभ्या॑ म॒स्मिन्न् । \newline
44. त्र्य॒हाभ्या॑ म॒स्मिन् न॒स्मिन् त्र्य॒हाभ्या᳚म् त्र्य॒हाभ्या॑ म॒स्मिन् ॅलो॒के लो॒के᳚ ऽस्मिन् त्र्य॒हाभ्या᳚म् त्र्य॒हाभ्या॑ म॒स्मिन् ॅलो॒के । \newline
45. त्र्य॒हाभ्या॒मिति॑ त्रि - अ॒हाभ्या᳚म् । \newline
46. अ॒स्मिन् ॅलो॒के लो॒के᳚ ऽस्मिन् न॒स्मिन् ॅलो॒के प॒शून् प॒शून् ॅलो॒के᳚ ऽस्मिन् न॒स्मिन् ॅलो॒के प॒शून् । \newline
47. लो॒के प॒शून् प॒शून् ॅलो॒के लो॒के प॒शून् प्रति॒ प्रति॑ प॒शून् ॅलो॒के लो॒के प॒शून् प्रति॑ । \newline
48. प॒शून् प्रति॒ प्रति॑ प॒शून् प॒शून् प्रत्यौ॑हन् नौह॒न् प्रति॑ प॒शून् प॒शून् प्रत्यौ॑हन्न् । \newline
49. प्रत्यौ॑हन् नौह॒न् प्रति॒ प्रत्यौ॑हन् पृ॒ष्ठैः पृ॒ष्ठै रौ॑ह॒न् प्रति॒ प्रत्यौ॑हन् पृ॒ष्ठैः । \newline
50. औ॒ह॒न् पृ॒ष्ठैः पृ॒ष्ठै रौ॑हन् नौहन् पृ॒ष्ठै रा॑दि॒त्या आ॑दि॒त्याः पृ॒ष्ठै रौ॑हन् नौहन् पृ॒ष्ठै रा॑दि॒त्याः । \newline
51. पृ॒ष्ठै रा॑दि॒त्या आ॑दि॒त्याः पृ॒ष्ठैः पृ॒ष्ठै रा॑दि॒त्या अ॒मुष्मि॑न् न॒मुष्मि॑न् नादि॒त्याः पृ॒ष्ठैः पृ॒ष्ठै रा॑दि॒त्या अ॒मुष्मिन्न्॑ । \newline
52. आ॒दि॒त्या अ॒मुष्मि॑न् न॒मुष्मि॑न् नादि॒त्या आ॑दि॒त्या अ॒मुष्मि॑न् ॅलो॒के लो॒के॑ ऽमुष्मि॑न् नादि॒त्या आ॑दि॒त्या अ॒मुष्मि॑न् ॅलो॒के । \newline
53. अ॒मुष्मि॑न् ॅलो॒के लो॒के॑ ऽमुष्मि॑न् न॒मुष्मि॑न् ॅलो॒क आर्द्ध्नु॑व॒न् नार्द्ध्नु॑वन् ॅलो॒के॑ ऽमुष्मि॑न् न॒मुष्मि॑न् ॅलो॒क आर्द्ध्नु॑वन्न् । \newline
54. लो॒क आर्द्ध्नु॑व॒न् नार्द्ध्नु॑वन् ॅलो॒के लो॒क आर्द्ध्नु॑वन् त्र्य॒हाभ्या᳚म् त्र्य॒हाभ्या॒ मार्द्ध्नु॑वन् ॅलो॒के लो॒क आर्द्ध्नु॑वन् त्र्य॒हाभ्या᳚म् । \newline
55. आर्द्ध्नु॑वन् त्र्य॒हाभ्या᳚म् त्र्य॒हाभ्या॒ मार्द्ध्नु॑व॒न् नार्द्ध्नु॑वन् त्र्य॒हाभ्या॑ म॒स्मिन् न॒स्मिन् त्र्य॒हाभ्या॒ मार्द्ध्नु॑व॒न् नार्द्ध्नु॑वन् त्र्य॒हाभ्या॑ म॒स्मिन्न् । \newline
56. त्र्य॒हाभ्या॑ म॒स्मिन् न॒स्मिन् त्र्य॒हाभ्या᳚म् त्र्य॒हाभ्या॑ म॒स्मिन् ॅलो॒के लो॒के᳚ ऽस्मिन् त्र्य॒हाभ्या᳚म् त्र्य॒हाभ्या॑ म॒स्मिन् ॅलो॒के । \newline
57. त्र्य॒हाभ्या॒मिति॑ त्रि - अ॒हाभ्या᳚म् । \newline
58. अ॒स्मिन् ॅलो॒के लो॒के᳚ ऽस्मिन् न॒स्मिन् ॅलो॒के प॒शवः॑ प॒शवो॑ लो॒के᳚ ऽस्मिन् न॒स्मिन् ॅलो॒के प॒शवः॑ । \newline
\pagebreak
\markright{ TS 7.3.5.2  \hfill https://www.vedavms.in \hfill}

\section{ TS 7.3.5.2 }

\textbf{TS 7.3.5.2 } \newline
\textbf{Samhita Paata} \newline

ॅलो॒के प॒शवो॒ य ए॒वं ॅवि॒द्वाꣳस॑-श्चतुर्दशरा॒त्रमास॑त उ॒भयो॑रे॒व लो॒कयोर्॑ ऋद्ध्नुवन्त्य॒स्मिꣳश्चा॒मुष्मिꣳ॑श्च पृ॒ष्ठैरे॒वाऽमुष्मि॑न् ॅलो॒क ऋ॑द्ध्नु॒वन्ति॑ त्र्य॒हाभ्या॑म॒स्मिन् ॅलो॒के ज्योति॒र्गौरायु॒रिति॑ त्र्य॒हो भ॑वती॒यं ॅवाव ज्योति॑र॒न्तरि॑क्षं॒ गौर॒सावायु॑रि॒माने॒व लो॒कान॒भ्यारो॑हन्ति॒ यद॒न्यतः॑ पृ॒ष्ठानि॒ स्युर्विवि॑वधꣳ स्या॒न्मद्ध्ये॑ पृ॒ष्ठानि॑ भवन्ति सविवध॒त्वायौ - [  ] \newline

\textbf{Pada Paata} \newline

लो॒के । प॒शवः॑ । ये । ए॒वम् । वि॒द्वाꣳसः॑ । च॒तु॒र्द॒श॒रा॒त्रमिति॑ चतुर्दश - रा॒त्रम् । आस॑ते । उ॒भयोः᳚ । ए॒व । लो॒कयोः᳚ । ऋ॒द्ध्नु॒व॒न्ति॒ । अ॒स्मिन्न् । च॒ । अ॒मुष्मिन्न्॑ । च॒ । पृ॒ष्ठैः । ए॒व । अ॒मुष्मिन्न्॑ । लो॒के । ऋ॒द्ध्नु॒वन्ति॑ । त्र्य॒हाभ्या॒मिति॑ त्रि - अ॒हाभ्या᳚म् । अ॒स्मिन्न् । लो॒के । ज्योतिः॑ । गौः । आयुः॑ । इति॑ । त्र्य॒ह इति॑ त्रि-अ॒हः । भ॒व॒ति॒ । इ॒यम् । वाव । ज्योतिः॑ । अ॒न्तरि॑क्षम् । गौः । अ॒सौ । आयुः॑ । इ॒मान् । ए॒व । लो॒कान् । अ॒भ्यारो॑ह॒न्तीत्य॑भि - आरो॑हन्ति । यत् । अ॒न्यतः॑ । पृ॒ष्ठानि॑ । स्युः । विवि॑वध॒मिति॒ वि-वि॒व॒ध॒म् । स्या॒त् । मद्ध्ये᳚ । पृ॒ष्ठानि॑ । भ॒व॒न्ति॒ । स॒वि॒व॒ध॒त्वायेति॑ सविवध - त्वाय॑ ।  \newline


\textbf{Krama Paata} \newline

लो॒के प॒शवः॑ । प॒शवो॒ ये । य ए॒वम् । ए॒वम् ॅवि॒द्वाꣳसः॑ । वि॒द्वाꣳस॑श्चतुर्दशरा॒त्रम् । च॒तु॒र्द॒श॒रा॒त्रमास॑ते । च॒तु॒र्द॒श॒रा॒त्रमिति॑ चतुर्दश - रा॒त्रम् । आस॑त उ॒भयोः᳚ । उ॒भयो॑रे॒व । ए॒व लो॒कयोः᳚ । लो॒कयोर्॑. ऋद्ध्नुवन्ति । ऋ॒द्ध्नु॒व॒न्त्य॒स्मिन्न् । अ॒स्मिꣳश्च॑ । चा॒मुष्मिन्न्॑ । अ॒मुष्मिꣳ॑श्च । च॒ पृ॒ष्ठैः । पृ॒ष्ठैरे॒व । ए॒वामुष्मिन्न्॑ । अ॒मुष्मि॑न् ॅलो॒के । लो॒क ऋ॑द्ध्नु॒वन्ति॑ । ऋ॒द्ध्नु॒वन्ति॑ त्र्य॒हाभ्या᳚म् । त्र्य॒हाभ्या॑म॒स्मिन्न् । त्र्य॒हाभ्या॒मिति॑ त्रि - अ॒हाभ्या᳚म् । अ॒स्मिन् ॅलो॒के । लो॒के ज्योतिः॑ । ज्योति॒र् गौः । गौरायुः॑ । आयु॒रिति॑ । इति॑ त्र्य॒हः । त्र्य॒हो भ॑वति । त्र्य॒ह इति॑ त्रि - अ॒हः । भ॒व॒ती॒यम् । इ॒यम् ॅवाव । वाव ज्योतिः॑ । ज्योति॑र॒न्तरि॑क्षम् । अ॒न्तरि॑क्ष॒म् गौः । गौर॒सौ । अ॒सावायुः॑ । आयु॑रि॒मान् । इ॒माने॒व । ए॒व लो॒कान् । लो॒कान॒भ्यारो॑हन्ति । अ॒भ्यारो॑हन्ति॒ यत् । अ॒भ्यारो॑ह॒न्तीत्य॑भि - आरो॑हन्ति । यद॒न्यतः॑ । अ॒न्यतः॑ पृ॒ष्ठानि॑ । पृ॒ष्ठानि॒ स्युः । स्युर् विवि॑वधम् । विवि॑वधꣳ स्यात् । विवि॑वध॒मिति॒ वि - वि॒व॒ध॒म् । स्या॒न् मद्ध्ये᳚ । मद्ध्ये॑ पृ॒ष्ठानि॑ । पृ॒ष्ठानि॑ भवन्ति । भ॒व॒न्ति॒ स॒वि॒व॒ध॒त्वाय॑ । स॒वि॒व॒ध॒त्वायौजः॑ । स॒वि॒व॒ध॒त्वायेति॑ सविवध - त्वाय॑ \newline

\textbf{Jatai Paata} \newline

1. लो॒के प॒शवः॑ प॒शवो॑ लो॒के लो॒के प॒शवः॑ । \newline
2. प॒शवो॒ ये ये प॒शवः॑ प॒शवो॒ ये । \newline
3. य ए॒व मे॒वं ॅये य ए॒वम् । \newline
4. ए॒वं ॅवि॒द्वाꣳसो॑ वि॒द्वाꣳस॑ ए॒व मे॒वं ॅवि॒द्वाꣳसः॑ । \newline
5. वि॒द्वाꣳस॑ श्चतुर्दशरा॒त्रम् च॑तुर्दशरा॒त्रं ॅवि॒द्वाꣳसो॑ वि॒द्वाꣳस॑ श्चतुर्दशरा॒त्रम् । \newline
6. च॒तु॒र्द॒श॒रा॒त्र मास॑त॒ आस॑ते चतुर्दशरा॒त्रम् च॑तुर्दशरा॒त्र मास॑ते । \newline
7. च॒तु॒र्द॒श॒रा॒त्रमिति॑ चतुर्दश - रा॒त्रम् । \newline
8. आस॑त उ॒भयो॑ रु॒भयो॒ रास॑त॒ आस॑त उ॒भयोः᳚ । \newline
9. उ॒भयो॑ रे॒वैवोभयो॑ रु॒भयो॑ रे॒व । \newline
10. ए॒व लो॒कयो᳚र् लो॒कयो॑ रे॒वैव लो॒कयोः᳚ । \newline
11. लो॒कयोर्॑. ऋद्ध्नुव न्त्यृद्ध्नुवन्ति लो॒कयो᳚र् लो॒कयोर्॑. ऋद्ध्नुवन्ति । \newline
12. ऋ॒द्ध्नु॒व॒ न्त्य॒स्मिन् न॒स्मिन् नृ॑द्ध्नुव न्त्यृद्ध्नुव न्त्य॒स्मिन्न् । \newline
13. अ॒स्मिꣳश्च॑ चा॒स्मिन् न॒स्मिꣳश्च॑ । \newline
14. चा॒मुष्मि॑न् न॒मुष्मिꣳ॑श्च चा॒मुष्मिन्न्॑ । \newline
15. अ॒मुष्मिꣳ॑श्च चा॒मुष्मि॑न् न॒मुष्मिꣳ॑श्च । \newline
16. च॒ पृ॒ष्ठैः पृ॒ष्ठैश्च॑ च पृ॒ष्ठैः । \newline
17. पृ॒ष्ठै रे॒वैव पृ॒ष्ठैः पृ॒ष्ठै रे॒व । \newline
18. ए॒वा मुष्मि॑न् न॒मुष्मि॑न् ने॒वैवा मुष्मिन्न्॑ । \newline
19. अ॒मुष्मि॑न् ॅलो॒के लो॒के॑ ऽमुष्मि॑न् न॒मुष्मि॑न् ॅलो॒के । \newline
20. लो॒क ऋ॑द्ध्नु॒व न्त्यृ॑द्ध्नु॒वन्ति॑ लो॒के लो॒क ऋ॑द्ध्नु॒वन्ति॑ । \newline
21. ऋ॒द्ध्नु॒वन्ति॑ त्र्य॒हाभ्या᳚म् त्र्य॒हाभ्या॑ मृद्ध्नु॒व न्त्यृ॑द्ध्नु॒वन्ति॑ त्र्य॒हाभ्या᳚म् । \newline
22. त्र्य॒हाभ्या॑ म॒स्मिन् न॒स्मिन् त्र्य॒हाभ्या᳚म् त्र्य॒हाभ्या॑ म॒स्मिन्न् । \newline
23. त्र्य॒हाभ्या॒मिति॑ त्रि - अ॒हाभ्या᳚म् । \newline
24. अ॒स्मिन् ॅलो॒के लो॒के᳚ ऽस्मिन् न॒स्मिन् ॅलो॒के । \newline
25. लो॒के ज्योति॒र् ज्योति॑र् लो॒के लो॒के ज्योतिः॑ । \newline
26. ज्योति॒र् गौर् गौर् ज्योति॒र् ज्योति॒र् गौः । \newline
27. गौरायु॒ रायु॒र् गौर् गौरायुः॑ । \newline
28. आयु॒ रिती त्यायु॒ रायु॒ रिति॑ । \newline
29. इति॑ त्र्य॒ह स्त्र्य॒ह इतीति॑ त्र्य॒हः । \newline
30. त्र्य॒हो भ॑वति भवति त्र्य॒ह स्त्र्य॒हो भ॑वति । \newline
31. त्र्य॒ह इति॑ त्रि - अ॒हः । \newline
32. भ॒व॒ती॒य मि॒यम् भ॑वति भवती॒यम् । \newline
33. इ॒यं ॅवाव वावेय मि॒यं ॅवाव । \newline
34. वाव ज्योति॒र् ज्योति॒र् वाव वाव ज्योतिः॑ । \newline
35. ज्योति॑ र॒न्तरि॑क्ष म॒न्तरि॑क्ष॒म् ज्योति॒र् ज्योति॑ र॒न्तरि॑क्षम् । \newline
36. अ॒न्तरि॑क्ष॒म् गौर् गौर॒न्तरि॑क्ष म॒न्तरि॑क्ष॒म् गौः । \newline
37. गौ र॒सा व॒सौ गौर् गौ र॒सौ । \newline
38. अ॒सा वायु॒ रायु॑ र॒सा व॒सा वायुः॑ । \newline
39. आयु॑ रि॒मा नि॒मा नायु॒ रायु॑ रि॒मान् । \newline
40. इ॒मा ने॒वैवेमा नि॒मा ने॒व । \newline
41. ए॒व लो॒कान् ॅलो॒का ने॒वैव लो॒कान् । \newline
42. लो॒का न॒भ्यारो॑ह न्त्य॒भ्यारो॑हन्ति लो॒कान् ॅलो॒का न॒भ्यारो॑हन्ति । \newline
43. अ॒भ्यारो॑हन्ति॒ यद् यद॒भ्यारो॑ह न्त्य॒भ्यारो॑हन्ति॒ यत् । \newline
44. अ॒भ्यारो॑ह॒न्तीत्य॑भि - आरो॑हन्ति । \newline
45. यद॒न्यतो॒ ऽन्यतो॒ यद् यद॒न्यतः॑ । \newline
46. अ॒न्यतः॑ पृ॒ष्ठानि॑ पृ॒ष्ठा न्य॒न्यतो॒ ऽन्यतः॑ पृ॒ष्ठानि॑ । \newline
47. पृ॒ष्ठानि॒ स्युः स्युः पृ॒ष्ठानि॑ पृ॒ष्ठानि॒ स्युः । \newline
48. स्युर् विवि॑वधं॒ ॅविवि॑वधꣳ॒॒ स्युः स्युर् विवि॑वधम् । \newline
49. विवि॑वधꣳ स्याथ् स्या॒द् विवि॑वधं॒ ॅविवि॑वधꣳ स्यात् । \newline
50. विवि॑वध॒मिति॒ वि - वि॒व॒ध॒म् । \newline
51. स्या॒न् मद्ध्ये॒ मद्ध्ये᳚ स्याथ् स्या॒न् मद्ध्ये᳚ । \newline
52. मद्ध्ये॑ पृ॒ष्ठानि॑ पृ॒ष्ठानि॒ मद्ध्ये॒ मद्ध्ये॑ पृ॒ष्ठानि॑ । \newline
53. पृ॒ष्ठानि॑ भवन्ति भवन्ति पृ॒ष्ठानि॑ पृ॒ष्ठानि॑ भवन्ति । \newline
54. भ॒व॒न्ति॒ स॒वि॒व॒ध॒त्वाय॑ सविवध॒त्वाय॑ भवन्ति भवन्ति सविवध॒त्वाय॑ । \newline
55. स॒वि॒व॒ध॒त्वा यौज॒ ओजः॑ सविवध॒त्वाय॑ सविवध॒त्वा यौजः॑ । \newline
56. स॒वि॒व॒ध॒त्वायेति॑ सविवध - त्वाय॑ । \newline

\textbf{Ghana Paata } \newline

1. लो॒के प॒शवः॑ प॒शवो॑ लो॒के लो॒के प॒शवो॒ ये ये प॒शवो॑ लो॒के लो॒के प॒शवो॒ ये । \newline
2. प॒शवो॒ ये ये प॒शवः॑ प॒शवो॒ य ए॒व मे॒वं ॅये प॒शवः॑ प॒शवो॒ य ए॒वम् । \newline
3. य ए॒व मे॒वं ॅये य ए॒वं ॅवि॒द्वाꣳसो॑ वि॒द्वाꣳस॑ ए॒वं ॅये य ए॒वं ॅवि॒द्वाꣳसः॑ । \newline
4. ए॒वं ॅवि॒द्वाꣳसो॑ वि॒द्वाꣳस॑ ए॒व मे॒वं ॅवि॒द्वाꣳस॑ श्चतुर्दशरा॒त्रम् च॑तुर्दशरा॒त्रं ॅवि॒द्वाꣳस॑ ए॒व मे॒वं ॅवि॒द्वाꣳस॑ श्चतुर्दशरा॒त्रम् । \newline
5. वि॒द्वाꣳस॑ श्चतुर्दशरा॒त्रम् च॑तुर्दशरा॒त्रं ॅवि॒द्वाꣳसो॑ वि॒द्वाꣳस॑ श्चतुर्दशरा॒त्र मास॑त॒ आस॑ते चतुर्दशरा॒त्रं ॅवि॒द्वाꣳसो॑ वि॒द्वाꣳस॑ श्चतुर्दशरा॒त्र मास॑ते । \newline
6. च॒तु॒र्द॒श॒रा॒त्र मास॑त॒ आस॑ते चतुर्दशरा॒त्रम् च॑तुर्दशरा॒त्र मास॑त उ॒भयो॑ रु॒भयो॒ रास॑ते चतुर्दशरा॒त्रम् च॑तुर्दशरा॒त्र मास॑त उ॒भयोः᳚ । \newline
7. च॒तु॒र्द॒श॒रा॒त्रमिति॑ चतुर्दश - रा॒त्रम् । \newline
8. आस॑त उ॒भयो॑ रु॒भयो॒ रास॑त॒ आस॑त उ॒भयो॑ रे॒वैवोभयो॒ रास॑त॒ आस॑त उ॒भयो॑ रे॒व । \newline
9. उ॒भयो॑ रे॒वैवोभयो॑ रु॒भयो॑ रे॒व लो॒कयो᳚र् लो॒कयो॑ रे॒वोभयो॑ रु॒भयो॑ रे॒व लो॒कयोः᳚ । \newline
10. ए॒व लो॒कयो᳚र् लो॒कयो॑ रे॒वैव लो॒कयोर्॑. ऋद्ध्नुव न्त्यृद्ध्नुवन्ति लो॒कयो॑ रे॒वैव लो॒कयोर्॑. ऋद्ध्नुवन्ति । \newline
11. लो॒कयोर्॑. ऋद्ध्नुव न्त्यृद्ध्नुवन्ति लो॒कयो᳚र् लो॒कयोर्॑. ऋद्ध्नुव न्त्य॒स्मिन् न॒स्मिन् नृ॑द्ध्नुवन्ति लो॒कयो᳚र् लो॒कयोर्॑. ऋद्ध्नुव न्त्य॒स्मिन्न् । \newline
12. ऋ॒द्ध्नु॒व॒ न्त्य॒स्मिन् न॒स्मिन् नृ॑द्ध्नुव न्त्यृद्ध्नुव न्त्य॒स्मिꣳश्च॑ चा॒स्मिन् नृ॑द्ध्नुव न्त्यृद्ध्नुव
न्त्य॒स्मिꣳश्च॑ । \newline
13. अ॒स्मिꣳश्च॑ चा॒स्मिन् न॒स्मिꣳ श्चा॒मुष्मि॑न् न॒मुष्मिꣳ॑ श्चा॒स्मिन् न॒स्मिꣳ
श्चा॒मुष्मिन्न्॑ । \newline
14. चा॒मुष्मि॑न् न॒मुष्मिꣳ॑श्च चा॒मुष्मिꣳ॑श्च चा॒मुष्मिꣳ॑श्च चा॒मुष्मिꣳ॑श्च । \newline
15. अ॒मुष्मिꣳ॑श्च चा॒मुष्मि॑न् न॒मुष्मिꣳ॑श्च पृ॒ष्ठैः पृ॒ष्ठै श्चा॒मुष्मि॑न् न॒मुष्मिꣳ॑श्च पृ॒ष्ठैः । \newline
16. च॒ पृ॒ष्ठैः पृ॒ष्ठै श्च॑ च पृ॒ष्ठै रे॒वैव पृ॒ष्ठै श्च॑ च पृ॒ष्ठै रे॒व । \newline
17. पृ॒ष्ठै रे॒वैव पृ॒ष्ठैः पृ॒ष्ठै रे॒वा मुष्मि॑न् न॒मुष्मि॑न् ने॒व पृ॒ष्ठैः पृ॒ष्ठै रे॒वा मुष्मिन्न्॑ । \newline
18. ए॒वा मुष्मि॑न् न॒मुष्मि॑न् ने॒वैवा मुष्मि॑न् ॅलो॒के लो॒के॑ ऽमुष्मि॑न् ने॒वैवा मुष्मि॑न् ॅलो॒के । \newline
19. अ॒मुष्मि॑न् ॅलो॒के लो॒के॑ ऽमुष्मि॑न् न॒मुष्मि॑न् ॅलो॒क ऋ॑द्ध्नु॒व न्त्यृ॑द्ध्नु॒वन्ति॑ लो॒के॑ ऽमुष्मि॑न् न॒मुष्मि॑न् ॅलो॒क ऋ॑द्ध्नु॒वन्ति॑ । \newline
20. लो॒क ऋ॑द्ध्नु॒व न्त्यृ॑द्ध्नु॒वन्ति॑ लो॒के लो॒क ऋ॑द्ध्नु॒वन्ति॑ त्र्य॒हाभ्या᳚म् त्र्य॒हाभ्या॑ मृद्ध्नु॒वन्ति॑ लो॒के लो॒क ऋ॑द्ध्नु॒वन्ति॑ त्र्य॒हाभ्या᳚म् । \newline
21. ऋ॒द्ध्नु॒वन्ति॑ त्र्य॒हाभ्या᳚म् त्र्य॒हाभ्या॑ मृद्ध्नु॒व न्त्यृ॑द्ध्नु॒वन्ति॑ त्र्य॒हाभ्या॑ म॒स्मिन् न॒स्मिन् त्र्य॒हाभ्या॑ मृद्ध्नु॒व न्त्यृ॑द्ध्नु॒वन्ति॑ त्र्य॒हाभ्या॑ म॒स्मिन्न् । \newline
22. त्र्य॒हाभ्या॑ म॒स्मिन् न॒स्मिन् त्र्य॒हाभ्या᳚म् त्र्य॒हाभ्या॑ म॒स्मिन् ॅलो॒के लो॒के᳚ ऽस्मिन् त्र्य॒हाभ्या᳚म् त्र्य॒हाभ्या॑ म॒स्मिन् ॅलो॒के । \newline
23. त्र्य॒हाभ्या॒मिति॑ त्रि - अ॒हाभ्या᳚म् । \newline
24. अ॒स्मिन् ॅलो॒के लो॒के᳚ ऽस्मिन् न॒स्मिन् ॅलो॒के ज्योति॒र् ज्योति॑र् लो॒के᳚ ऽस्मिन् न॒स्मिन् ॅलो॒के ज्योतिः॑ । \newline
25. लो॒के ज्योति॒र् ज्योति॑र् लो॒के लो॒के ज्योति॒र् गौर् गौर् ज्योति॑र् लो॒के लो॒के ज्योति॒र् गौः । \newline
26. ज्योति॒र् गौर् गौर् ज्योति॒र् ज्योति॒र् गौ रायु॒ रायु॒र् गौर् ज्योति॒र् ज्योति॒र् गौ रायुः॑ । \newline
27. गौ रायु॒ रायु॒र् गौर् गौ रायु॒ रिती त्यायु॒र् गौर् गौ रायु॒ रिति॑ । \newline
28. आयु॒ रिती त्यायु॒ रायु॒ रिति॑ त्र्य॒ह स्त्र्य॒ह इत्यायु॒ रायु॒ रिति॑ त्र्य॒हः । \newline
29. इति॑ त्र्य॒ह स्त्र्य॒ह इतीति॑ त्र्य॒हो भ॑वति भवति त्र्य॒ह इतीति॑ त्र्य॒हो भ॑वति । \newline
30. त्र्य॒हो भ॑वति भवति त्र्य॒ह स्त्र्य॒हो भ॑वती॒य मि॒यम् भ॑वति त्र्य॒ह स्त्र्य॒हो भ॑वती॒यम् । \newline
31. त्र्य॒ह इति॑ त्रि - अ॒हः । \newline
32. भ॒व॒ती॒य मि॒यम् भ॑वति भवती॒यं ॅवाव वावेयम् भ॑वति भवती॒यं ॅवाव । \newline
33. इ॒यं ॅवाव वावेय मि॒यं ॅवाव ज्योति॒र् ज्योति॒र् वावेय मि॒यं ॅवाव ज्योतिः॑ । \newline
34. वाव ज्योति॒र् ज्योति॒र् वाव वाव ज्योति॑ र॒न्तरि॑क्ष म॒न्तरि॑क्ष॒म् ज्योति॒र् वाव वाव ज्योति॑ र॒न्तरि॑क्षम् । \newline
35. ज्योति॑ र॒न्तरि॑क्ष म॒न्तरि॑क्ष॒म् ज्योति॒र् ज्योति॑ र॒न्तरि॑क्ष॒म् गौर् गौ र॒न्तरि॑क्ष॒म् ज्योति॒र् ज्योति॑ र॒न्तरि॑क्ष॒म् गौः । \newline
36. अ॒न्तरि॑क्ष॒म् गौर् गौ र॒न्तरि॑क्ष म॒न्तरि॑क्ष॒म् गौ र॒सा व॒सौ गौ र॒न्तरि॑क्ष म॒न्तरि॑क्ष॒म् गौ र॒सौ । \newline
37. गौर॒सा व॒सौ गौर् गौर॒सा वायु॒ रायु॑ र॒सौ गौर् गौर॒सा वायुः॑ । \newline
38. अ॒सा वायु॒ रायु॑ र॒सा व॒सा वायु॑ रि॒मा नि॒मा नायु॑ र॒सा व॒सा वायु॑ रि॒मान् । \newline
39. आयु॑ रि॒मा नि॒मा नायु॒ रायु॑ रि॒मा ने॒वैवेमा नायु॒ रायु॑ रि॒मा ने॒व । \newline
40. इ॒मा ने॒वैवेमा नि॒मा ने॒व लो॒कान् ॅलो॒का ने॒वेमा नि॒मा ने॒व लो॒कान् । \newline
41. ए॒व लो॒कान् ॅलो॒का ने॒वैव लो॒का न॒भ्यारो॑ह न्त्य॒भ्यारो॑हन्ति लो॒का ने॒वैव लो॒का न॒भ्यारो॑हन्ति । \newline
42. लो॒का न॒भ्यारो॑ह न्त्य॒भ्यारो॑हन्ति लो॒कान् ॅलो॒का न॒भ्यारो॑हन्ति॒ यद् यद॒भ्यारो॑हन्ति लो॒कान् ॅलो॒का न॒भ्यारो॑हन्ति॒ यत् । \newline
43. अ॒भ्यारो॑हन्ति॒ यद् यद॒भ्यारो॑ह न्त्य॒भ्यारो॑हन्ति॒ यद॒न्यतो॒ ऽन्यतो॒ यद॒भ्यारो॑ह न्त्य॒भ्यारो॑हन्ति॒ यद॒न्यतः॑ । \newline
44. अ॒भ्यारो॑ह॒न्तीत्य॑भि - आरो॑हन्ति । \newline
45. यद॒न्यतो॒ ऽन्यतो॒ यद् यद॒न्यतः॑ पृ॒ष्ठानि॑ पृ॒ष्ठा न्य॒न्यतो॒ यद् यद॒न्यतः॑ पृ॒ष्ठानि॑ । \newline
46. अ॒न्यतः॑ पृ॒ष्ठानि॑ पृ॒ष्ठा न्य॒न्यतो॒ ऽन्यतः॑ पृ॒ष्ठानि॒ स्युः स्युः पृ॒ष्ठा न्य॒न्यतो॒ ऽन्यतः॑ पृ॒ष्ठानि॒ स्युः । \newline
47. पृ॒ष्ठानि॒ स्युः स्युः पृ॒ष्ठानि॑ पृ॒ष्ठानि॒ स्युर् विवि॑वधं॒ ॅविवि॑वधꣳ॒॒ स्युः पृ॒ष्ठानि॑ पृ॒ष्ठानि॒ स्युर् विवि॑वधम् । \newline
48. स्युर् विवि॑वधं॒ ॅविवि॑वधꣳ॒॒ स्युः स्युर् विवि॑वधꣳ स्याथ् स्या॒द् विवि॑वधꣳ॒॒ स्युः स्युर् विवि॑वधꣳ स्यात् । \newline
49. विवि॑वधꣳ स्याथ् स्या॒द् विवि॑वधं॒ ॅविवि॑वधꣳ स्या॒न् मद्ध्ये॒ मद्ध्ये᳚ स्या॒द् विवि॑वधं॒ ॅविवि॑वधꣳ स्या॒न् मद्ध्ये᳚ । \newline
50. विवि॑वध॒मिति॒ वि - वि॒व॒ध॒म् । \newline
51. स्या॒न् मद्ध्ये॒ मद्ध्ये᳚ स्याथ् स्या॒न् मद्ध्ये॑ पृ॒ष्ठानि॑ पृ॒ष्ठानि॒ मद्ध्ये᳚ स्याथ् स्या॒न् मद्ध्ये॑ पृ॒ष्ठानि॑ । \newline
52. मद्ध्ये॑ पृ॒ष्ठानि॑ पृ॒ष्ठानि॒ मद्ध्ये॒ मद्ध्ये॑ पृ॒ष्ठानि॑ भवन्ति भवन्ति पृ॒ष्ठानि॒ मद्ध्ये॒ मद्ध्ये॑ पृ॒ष्ठानि॑ भवन्ति । \newline
53. पृ॒ष्ठानि॑ भवन्ति भवन्ति पृ॒ष्ठानि॑ पृ॒ष्ठानि॑ भवन्ति सविवध॒त्वाय॑ सविवध॒त्वाय॑ भवन्ति पृ॒ष्ठानि॑ पृ॒ष्ठानि॑ भवन्ति सविवध॒त्वाय॑ । \newline
54. भ॒व॒न्ति॒ स॒वि॒व॒ध॒त्वाय॑ सविवध॒त्वाय॑ भवन्ति भवन्ति सविवध॒त्वा यौज॒ ओजः॑ सविवध॒त्वाय॑ भवन्ति भवन्ति सविवध॒त्वा यौजः॑ । \newline
55. स॒वि॒व॒ध॒त्वा यौज॒ ओजः॑ सविवध॒त्वाय॑ सविवध॒त्वा यौजो॒ वै वा ओजः॑ सविवध॒त्वाय॑ सविवध॒त्वा यौजो॒ वै । \newline
56. स॒वि॒व॒ध॒त्वायेति॑ सविवध - त्वाय॑ । \newline
\pagebreak
\markright{ TS 7.3.5.3  \hfill https://www.vedavms.in \hfill}

\section{ TS 7.3.5.3 }

\textbf{TS 7.3.5.3 } \newline
\textbf{Samhita Paata} \newline

-जो॒ वै वी॒र्यं॑ पृ॒ष्ठान्योज॑ ए॒व वी॒र्यं॑ मद्ध्य॒तो द॑धते बृहद्-रथन्त॒राभ्यां᳚ ॅयन्ती॒यं ॅवाव र॑थन्त॒रम॒सौ बृ॒हदा॒भ्यामे॒व य॒न्त्यथो॑ अ॒नयो॑रे॒व प्रति॑ तिष्ठन्त्ये॒ते वै य॒ज्ञ्स्या᳚ञ्ज॒साय॑नी स्रु॒ती ताभ्या॑मे॒व सु॑व॒र्गं ॅलो॒कं ॅय॑न्ति॒ परा᳚ञ्चो॒ वा ए॒ते सु॑व॒र्गं ॅलो॒कम॒भ्यारो॑हन्ति॒ ये प॑रा॒चीना॑नि पृ॒ष्ठान्यु॑प॒यन्ति॑ प्र॒त्यङ् त्र्य॒हो भ॑वति प्र॒त्यव॑रूढ्या॒ अथो॒ प्रति॑ष्ठित्या ( ) उ॒भयो᳚र्लो॒कयोर्॑ ऋ॒द्ध्वोत् ति॑ष्ठन्ति॒ चतु॑र्दशै॒तास्तासां॒ ॅया दश॒ दशा᳚क्षरा वि॒राडन्नं॑ ॅवि॒राड् वि॒राजै॒वान्नाद्य॒मव॑ रुन्धते॒ याश्चत॑स्र॒श्चत॑स्रो॒ दिशो॑ दि॒क्ष्वे॑व प्रति॑ तिष्ठन्त्यतिरा॒त्राव॒भितो॑ भवतः॒ परि॑गृहीत्यै ॥ \newline

\textbf{Pada Paata} \newline

ओजः॑ । वै । वी॒र्य᳚म् । पृ॒ष्ठानि॑ । ओजः॑ । ए॒व । वी॒र्य᳚म् । म॒द्ध्य॒तः । द॒ध॒ते॒ । बृ॒ह॒द्र॒थ॒न्त॒राभ्या॒मिति॑ बृहत् - र॒थ॒न्त॒राभ्या᳚म् । य॒न्ति॒ । इ॒यम् । वाव । र॒थ॒न्त॒रमिति॑ रथं - त॒रम् । अ॒सौ । बृ॒हत् । आ॒भ्याम् । ए॒व । य॒न्ति॒ । अथो॒ इति॑ । अ॒नयोः᳚ । ए॒व । प्रतीति॑ । ति॒ष्ठ॒न्ति॒ । ए॒ते इति॑ । वै । य॒ज्ञ्स्य॑ । अ॒ञ्ज॒साय॑नी॒ इत्य॑ञ्जसा - अय॑नी । स्रु॒ती इति॑ । ताभ्या᳚म् । ए॒व । सु॒व॒र्गमिति॑ सुवः - गम् । लो॒कम् । य॒न्ति॒ । परा᳚ञ्चः । वै । ए॒ते । सु॒व॒र्गमिति॑ सुवः - गम् । लो॒कम् । अ॒भ्यारो॑ह॒न्तीत्य॑भि - आरो॑हन्ति । ये । प॒रा॒चीना॑नि । पृ॒ष्ठानि॑ । उ॒प॒यन्तीत्यु॑प - यन्ति॑ । प्र॒त्यङ् । त्र्य॒ह इति॑ त्रि - अ॒हः । भ॒व॒ति॒ । प्र॒त्यव॑रूढ्या॒ इति॑ प्रति - अव॑रूढ्यै । अथो॒ इति॑ । प्रति॑ष्ठित्या॒ इति॒ प्रति॑-स्थि॒त्यै॒ ( ) । उ॒भयोः᳚ । लो॒कयोः᳚ । ऋ॒द्ध्वा । उदिति॑ । ति॒ष्ठ॒न्ति॒ । चतु॑र्द॒शेति॒ चतुः॑ - द॒श॒ । ए॒ताः । तासा᳚म् । याः । दश॑ । दशा᳚क्ष॒रेति॒ दश॑ - अ॒क्ष॒रा॒ । वि॒राडिति॑ वि - राट् । अन्न᳚म् । वि॒राडिति॑ वि - राट् । वि॒राजेति॑ वि - राजा᳚ । ए॒व । अ॒न्नाद्य॒मित्य॑न्न - अद्य᳚म् । अवेति॑ । रु॒न्ध॒ते॒ । याः । चत॑स्रः । चत॑स्रः । दिशः॑ । दि॒क्षु । ए॒व । प्रतीति॑ । ति॒ष्ठ॒न्ति॒ । अ॒ति॒रा॒त्रावित्य॑ति - रा॒त्रौ । अ॒भितः॑ । भ॒व॒तः॒ । परि॑गृहीत्या॒ इति॒ परि॑ - गृ॒ही॒त्यै॒ ॥  \newline


\textbf{Krama Paata} \newline

ओजो॒ वै । वै वी॒र्य᳚म् । वी॒र्य॑म् पृ॒ष्ठानि॑ । पृ॒ष्ठान्योजः॑ । ओज॑ ए॒व । ए॒व वी॒र्य᳚म् । वी॒र्य॑म् मद्ध्य॒तः । म॒द्ध्य॒तो द॑धते । द॒ध॒ते॒ बृ॒ह॒द्‍र॒थ॒न्त॒राभ्या᳚म् । बृ॒ह॒द्‍र॒थ॒न्त॒राभ्या᳚म् ॅयन्ति । बृ॒ह॒द्‍र॒थ॒न्त॒राभ्या॒मिति॑ बृहत् - र॒थ॒न्त॒राभ्या᳚म् । य॒न्ती॒यम् । इ॒यम् ॅवाव । वाव र॑थन्त॒रम् । र॒थ॒न्त॒रम॒सौ । र॒थ॒न्त॒रमिति॑ रथम् - त॒रम् । अ॒सौ बृ॒हत् । बृ॒हदा॒भ्याम् । आ॒भ्यामे॒व । ए॒व य॑न्ति । य॒न्त्यथो᳚ । अथो॑ अ॒नयोः᳚ । अथो॒ इत्यथो᳚ । अ॒नयो॑रे॒व । ए॒व प्रति॑ । प्रति॑ तिष्ठन्ति । ति॒ष्ठ॒न्त्ये॒ते । ए॒ते वै । ए॒ते इत्ये॒ते । वै य॒ज्ञ्स्य॑ । य॒ज्ञ्स्या᳚ञ्ज॒साय॑नी । अ॒ञ्ज॒साय॑नी स्रु॒ती । अ॒ञ्ज॒साय॑नी॒ इत्य॑ञ्जसा - अय॑नी । स्रु॒ती ताभ्या᳚म् । स्रु॒ती इति॑ स्रु॒ती । ताभ्या॑मे॒व । ए॒व सु॑व॒र्गम् । सु॒व॒र्गम् ॅलो॒कम् । सु॒व॒र्गमिति॑ सुवः - गम् । लो॒कम् ॅय॑न्ति । य॒न्ति॒ परा᳚ञ्चः । परा᳚ञ्चो॒ वै । वा ए॒ते । ए॒ते सु॑व॒र्गम् । सु॒व॒र्गम् ॅलो॒कम् । सु॒व॒र्गमिति॑ सुवः - गम् । लो॒कम॒भ्यारो॑हन्ति । अ॒भ्यारो॑हन्ति॒ ये । अ॒भ्यारो॑ह॒न्तीत्य॑भि - आरो॑हन्ति । ये प॑रा॒चीना॑नि । प॒रा॒चीना॑नि पृ॒ष्ठानि॑ । पृ॒ष्ठान्यु॑प॒यन्ति॑ । उ॒प॒यन्ति॑ प्र॒त्यङ्‍ङ् । उ॒प॒यन्तीत्यु॑प - यन्ति॑ । प्र॒त्यङ् त्र्य॒हः । त्र्य॒हो भ॑वति । त्र्य॒ह इति॑ त्रि - अ॒हः । भ॒व॒ति॒ प्र॒त्यव॑रूढ्‍यै । प्र॒त्यव॑रूढ्‍या॒ अथो᳚ । प्र॒त्यव॑रूढ्‍या॒ इति॑ प्रति - अव॑रूढ्‍यै । अथो॒ प्रति॑ष्ठित्यै ( ) । अथो॒ इत्यथो᳚ । प्रति॑ष्ठित्या उ॒भयोः᳚ । प्रति॑ष्ठित्या॒ इति॒ प्रति॑ - स्थि॒त्यै॒ । उ॒भयो᳚र् लो॒कयोः᳚ । लो॒कयोर्॑. ऋ॒द्ध्वा । ऋ॒द्ध्वोत् । उत् ति॑ष्ठन्ति । ति॒ष्ठ॒न्ति॒ चतु॑र्दश । चतु॑र्दशै॒ताः । चतु॑र्द॒शेति॒ चतुः॑ - द॒श॒ । ए॒तास्तासा᳚म् । तासा॒म् ॅयाः । या दश॑ । दश॒ दशा᳚क्षरा । दशा᳚क्षरा वि॒राट् । दशा᳚क्ष॒रेति॒ दश॑ - अ॒क्ष॒रा॒ । वि॒राडन्न᳚म् । वि॒राडिति॑ वि - राट् । अन्न॑म् ॅवि॒राट् । वि॒राड् वि॒राजा᳚ । वि॒राडिति॑ वि - राट् । वि॒राजै॒व । वि॒राजेति॑ वि - राजा᳚ । ए॒वान्नाद्य᳚म् । अ॒न्नाद्य॒मव॑ । अ॒न्नाद्य॒मित्य॑न्न - अद्य᳚म् । अव॑ रुन्धते । रु॒न्ध॒ते॒ याः । याश्चत॑स्रः । चत॑स्र॒श्चत॑स्रः । चत॑स्रो॒ दिशः॑ । दिशो॑ दि॒क्षु । दि॒क्ष्वे॑व । ए॒व प्रति॑ । प्रति॑ तिष्ठन्ति । ति॒ष्ठ॒न्त्य॒ति॒रा॒त्रौ । अ॒ति॒रा॒त्राव॒भितः॑ । अ॒ति॒रा॒त्रा॒वित्य॑ति - रा॒त्रौ । अ॒भितो॑ भवतः । भ॒व॒तः॒ परि॑गृहीत्यै । परि॑गृहीत्या॒ इति॒ परि॑ - गृ॒ही॒त्यै॒ । \newline

\textbf{Jatai Paata} \newline

1. ओजो॒ वै वा ओज॒ ओजो॒ वै । \newline
2. वै वी॒र्यं॑ ॅवी॒र्यं॑ ॅवै वै वी॒र्य᳚म् । \newline
3. वी॒र्य॑म् पृ॒ष्ठानि॑ पृ॒ष्ठानि॑ वी॒र्यं॑ ॅवी॒र्य॑म् पृ॒ष्ठानि॑ । \newline
4. पृ॒ष्ठा न्योज॒ ओजः॑ पृ॒ष्ठानि॑ पृ॒ष्ठा न्योजः॑ । \newline
5. ओज॑ ए॒वै वौज॒ ओज॑ ए॒व । \newline
6. ए॒व वी॒र्यं॑ ॅवी॒र्य॑ मे॒वैव वी॒र्य᳚म् । \newline
7. वी॒र्य॑म् मद्ध्य॒तो म॑द्ध्य॒तो वी॒र्यं॑ ॅवी॒र्य॑म् मद्ध्य॒तः । \newline
8. म॒द्ध्य॒तो द॑धते दधते मद्ध्य॒तो म॑द्ध्य॒तो द॑धते । \newline
9. द॒ध॒ते॒ बृ॒ह॒द्र॒थ॒न्त॒राभ्या᳚म् बृहद्रथन्त॒राभ्या᳚म् दधते दधते बृहद्रथन्त॒राभ्या᳚म् । \newline
10. बृ॒ह॒द्र॒थ॒न्त॒राभ्यां᳚ ॅयन्ति यन्ति बृहद्रथन्त॒राभ्या᳚म् बृहद्रथन्त॒राभ्यां᳚ ॅयन्ति । \newline
11. बृ॒ह॒द्र॒थ॒न्त॒राभ्या॒मिति॑ बृहत् - र॒थ॒न्त॒राभ्या᳚म् । \newline
12. य॒न्ती॒य मि॒यं ॅय॑न्ति यन्ती॒यम् । \newline
13. इ॒यं ॅवाव वावेय मि॒यं ॅवाव । \newline
14. वाव र॑थन्त॒रꣳ र॑थन्त॒रं ॅवाव वाव र॑थन्त॒रम् । \newline
15. र॒थ॒न्त॒र म॒सा व॒सौ र॑थन्त॒रꣳ र॑थन्त॒र म॒सौ । \newline
16. र॒थ॒न्त॒रमिति॑ रथं - त॒रम् । \newline
17. अ॒सौ बृ॒हद् बृ॒ह द॒सा व॒सौ बृ॒हत् । \newline
18. बृ॒ह दा॒भ्या मा॒भ्याम् बृ॒हद् बृ॒ह दा॒भ्याम् । \newline
19. आ॒भ्या मे॒वै वाभ्या मा॒भ्या मे॒व । \newline
20. ए॒व य॑न्ति यन्त्ये॒वैव य॑न्ति । \newline
21. य॒न्त्यथो॒ अथो॑ यन्ति य॒न्त्यथो᳚ । \newline
22. अथो॑ अ॒नयो॑ र॒नयो॒ रथो॒ अथो॑ अ॒नयोः᳚ । \newline
23. अथो॒ इत्यथो᳚ । \newline
24. अ॒नयो॑ रे॒वै वानयो॑ र॒नयो॑ रे॒व । \newline
25. ए॒व प्रति॒ प्रत्ये॒वैव प्रति॑ । \newline
26. प्रति॑ तिष्ठन्ति तिष्ठन्ति॒ प्रति॒ प्रति॑ तिष्ठन्ति । \newline
27. ति॒ष्ठ॒न्त्ये॒ते ए॒ते ति॑ष्ठन्ति तिष्ठन्त्ये॒ते । \newline
28. ए॒ते वै वा ए॒ते ए॒ते वै । \newline
29. ए॒ते इत्ये॒ते । \newline
30. वै य॒ज्ञ्स्य॑ य॒ज्ञ्स्य॒ वै वै य॒ज्ञ्स्य॑ । \newline
31. य॒ज्ञ्स्या᳚ ञ्ज॒साय॑नी अञ्ज॒साय॑नी य॒ज्ञ्स्य॑ य॒ज्ञ्स्या᳚ ञ्ज॒साय॑नी । \newline
32. अ॒ञ्ज॒साय॑नी स्रु॒ती स्रु॒ती अ॑ञ्ज॒साय॑नी अञ्ज॒साय॑नी स्रु॒ती । \newline
33. अ॒ञ्ज॒साय॑नी॒ इत्य॑ञ्जसा - अय॑नी । \newline
34. स्रु॒ती ताभ्या॒म् ताभ्याꣳ॑ स्रु॒ती स्रु॒ती ताभ्या᳚म् । \newline
35. स्रु॒ती इति॑ स्रु॒ती । \newline
36. ताभ्या॑ मे॒वैव ताभ्या॒म् ताभ्या॑ मे॒व । \newline
37. ए॒व सु॑व॒र्गꣳ सु॑व॒र्ग मे॒वैव सु॑व॒र्गम् । \newline
38. सु॒व॒र्गम् ॅलो॒कम् ॅलो॒कꣳ सु॑व॒र्गꣳ सु॑व॒र्गम् ॅलो॒कम् । \newline
39. सु॒व॒र्गमिति॑ सुवः - गम् । \newline
40. लो॒कं ॅय॑न्ति यन्ति लो॒कम् ॅलो॒कं ॅय॑न्ति । \newline
41. य॒न्ति॒ परा᳚ञ्चः॒ परा᳚ञ्चो यन्ति यन्ति॒ परा᳚ञ्चः । \newline
42. परा᳚ञ्चो॒ वै वै परा᳚ञ्चः॒ परा᳚ञ्चो॒ वै । \newline
43. वा ए॒त ए॒ते वै वा ए॒ते । \newline
44. ए॒ते सु॑व॒र्गꣳ सु॑व॒र्ग मे॒त ए॒ते सु॑व॒र्गम् । \newline
45. सु॒व॒र्गम् ॅलो॒कम् ॅलो॒कꣳ सु॑व॒र्गꣳ सु॑व॒र्गम् ॅलो॒कम् । \newline
46. सु॒व॒र्गमिति॑ सुवः - गम् । \newline
47. लो॒क म॒भ्यारो॑ह न्त्य॒भ्यारो॑हन्ति लो॒कम् ॅलो॒क म॒भ्यारो॑हन्ति । \newline
48. अ॒भ्यारो॑हन्ति॒ ये ये᳚ ऽभ्यारो॑ह न्त्य॒भ्यारो॑हन्ति॒ ये । \newline
49. अ॒भ्यारो॑ह॒न्तीत्य॑भि - आरो॑हन्ति । \newline
50. ये प॑रा॒चीना॑नि परा॒चीना॑नि॒ ये ये प॑रा॒चीना॑नि । \newline
51. प॒रा॒चीना॑नि पृ॒ष्ठानि॑ पृ॒ष्ठानि॑ परा॒चीना॑नि परा॒चीना॑नि पृ॒ष्ठानि॑ । \newline
52. पृ॒ष्ठा न्यु॑प॒य न्त्यु॑प॒यन्ति॑ पृ॒ष्ठानि॑ पृ॒ष्ठा न्यु॑प॒यन्ति॑ । \newline
53. उ॒प॒यन्ति॑ प्र॒त्यङ् प्र॒त्यङ् ङु॑प॒य न्त्यु॑प॒यन्ति॑ प्र॒त्यङ् । \newline
54. उ॒प॒यन्तीत्यु॑प - यन्ति॑ । \newline
55. प्र॒त्यङ् त्र्य॒ह स्त्र्य॒हः प्र॒त्यङ् प्र॒त्यङ् त्र्य॒हः । \newline
56. त्र्य॒हो भ॑वति भवति त्र्य॒ह स्त्र्य॒हो भ॑वति । \newline
57. त्र्य॒ह इति॑ त्रि - अ॒हः । \newline
58. भ॒व॒ति॒ प्र॒त्यव॑रूढ्यै प्र॒त्यव॑रूढ्यै भवति भवति प्र॒त्यव॑रूढ्यै । \newline
59. प्र॒त्यव॑रूढ्या॒ अथो॒ अथो᳚ प्र॒त्यव॑रूढ्यै प्र॒त्यव॑रूढ्या॒ अथो᳚ । \newline
60. प्र॒त्यव॑रूढ्या॒ इति॑ प्रति - अव॑रूढ्यै । \newline
61. अथो॒ प्रति॑ष्ठित्यै॒ प्रति॑ष्ठित्या॒ अथो॒ अथो॒ प्रति॑ष्ठित्यै । \newline
62. अथो॒ इत्यथो᳚ । \newline
63. प्रति॑ष्ठित्या उ॒भयो॑ रु॒भयोः॒ प्रति॑ष्ठित्यै॒ प्रति॑ष्ठित्या उ॒भयोः᳚ । \newline
64. प्रति॑ष्ठित्या॒ इति॒ प्रति॑ - स्थि॒त्यै॒ । \newline
65. उ॒भयो᳚र् लो॒कयो᳚र् लो॒कयो॑ रु॒भयो॑ रु॒भयो᳚र् लो॒कयोः᳚ । \newline
66. लो॒कयोर्॑. ऋ॒द्ध्व र्‌द्ध्वा लो॒कयो᳚र् लो॒कयोर्॑. ऋ॒द्ध्वा । \newline
67. ऋ॒द्ध्वो दुदृ॒द्ध्व र्‌द्ध्वोत् । \newline
68. उत् ति॑ष्ठन्ति तिष्ठ॒ न्त्युदुत् ति॑ष्ठन्ति । \newline
69. ति॒ष्ठ॒न्ति॒ चतु॑र्दश॒ चतु॑र्दश तिष्ठन्ति तिष्ठन्ति॒ चतु॑र्दश । \newline
70. चतु॑र्दशै॒ता ए॒ता श्चतु॑र्दश॒ चतु॑र्दशै॒ताः । \newline
71. चतु॑र्द॒शेति॒ चतुः॑ - द॒श॒ । \newline
72. ए॒ता स्तासा॒म् तासा॑ मे॒ता ए॒ता स्तासा᳚म् । \newline
73. तासां॒ ॅया या स्तासा॒म् तासां॒ ॅयाः । \newline
74. या दश॒ दश॒ या या दश॑ । \newline
75. दश॒ दशा᳚क्षरा॒ दशा᳚क्षरा॒ दश॒ दश॒ दशा᳚क्षरा । \newline
76. दशा᳚क्षरा वि॒राड् वि॒राड् दशा᳚क्षरा॒ दशा᳚क्षरा वि॒राट् । \newline
77. दशा᳚क्ष॒रेति॒ दश॑ - अ॒क्ष॒रा॒ । \newline
78. वि॒रा डन्न॒ मन्नं॑ ॅवि॒राड् वि॒रा डन्न᳚म् । \newline
79. वि॒राडिति॑ वि - राट् । \newline
80. अन्नं॑ ॅवि॒राड् वि॒रा डन्न॒ मन्नं॑ ॅवि॒राट् । \newline
81. वि॒राड् वि॒राजा॑ वि॒राजा॑ वि॒राड् वि॒राड् वि॒राजा᳚ । \newline
82. वि॒राडिति॑ वि - राट् । \newline
83. वि॒राजै॒वैव वि॒राजा॑ वि॒रा जै॒व । \newline
84. वि॒राजेति॑ वि - राजा᳚ । \newline
85. ए॒वा न्नाद्य॑ म॒न्नाद्य॑ मे॒वैवा न्नाद्य᳚म् । \newline
86. अ॒न्नाद्य॒ मवावा॒ न्नाद्य॑ म॒न्नाद्य॒ मव॑ । \newline
87. अ॒न्नाद्य॒मित्य॑न्न - अद्य᳚म् । \newline
88. अव॑ रुन्धते रुन्ध॒ते ऽवाव॑ रुन्धते । \newline
89. रु॒न्ध॒ते॒ या या रु॑न्धते रुन्धते॒ याः । \newline
90. या श्चत॑स्र॒ श्चत॑स्रो॒ या या श्चत॑स्रः । \newline
91. चत॑स्र॒ श्चत॑स्रः । \newline
92. चत॑स्रो॒ दिशो॒ दिश॒ श्चत॑स्र॒ श्चत॑स्रो॒ दिशः॑ । \newline
93. दिशो॑ दि॒क्षु दि॒क्षु दिशो॒ दिशो॑ दि॒क्षु । \newline
94. दि॒क्ष्वे॑वैव दि॒क्षु दि॒क्ष्वे॑व । \newline
95. ए॒व प्रति॒ प्रत्ये॒वैव प्रति॑ । \newline
96. प्रति॑ तिष्ठन्ति तिष्ठन्ति॒ प्रति॒ प्रति॑ तिष्ठन्ति । \newline
97. ति॒ष्ठ॒ न्त्य॒ति॒रा॒त्रा व॑तिरा॒त्रौ ति॑ष्ठन्ति तिष्ठ न्त्यतिरा॒त्रौ । \newline
98. अ॒ति॒रा॒त्रा व॒भितो॒ ऽभितो॑ ऽतिरा॒त्रा व॑तिरा॒त्रा व॒भितः॑ । \newline
99. अ॒ति॒रा॒त्रावित्य॑ति - रा॒त्रौ । \newline
100. अ॒भितो॑ भवतो भवतो॒ ऽभितो॒ ऽभितो॑ भवतः । \newline
101. भ॒व॒तः॒ परि॑गृहीत्यै॒ परि॑गृहीत्यै भवतो भवतः॒ परि॑गृहीत्यै । \newline
102. परि॑गृहीत्या॒ इति॒ परि॑ - गृ॒ही॒त्यै॒ । \newline

\textbf{Ghana Paata } \newline

1. ओजो॒ वै वा ओज॒ ओजो॒ वै वी॒र्यं॑ ॅवी॒र्यं॑ ॅवा ओज॒ ओजो॒ वै वी॒र्य᳚म् । \newline
2. वै वी॒र्यं॑ ॅवी॒र्यं॑ ॅवै वै वी॒र्य॑म् पृ॒ष्ठानि॑ पृ॒ष्ठानि॑ वी॒र्यं॑ ॅवै वै वी॒र्य॑म् पृ॒ष्ठानि॑ । \newline
3. वी॒र्य॑म् पृ॒ष्ठानि॑ पृ॒ष्ठानि॑ वी॒र्यं॑ ॅवी॒र्य॑म् पृ॒ष्ठा न्योज॒ ओजः॑ पृ॒ष्ठानि॑ वी॒र्यं॑ ॅवी॒र्य॑म् पृ॒ष्ठा न्योजः॑ । \newline
4. पृ॒ष्ठा न्योज॒ ओजः॑ पृ॒ष्ठानि॑ पृ॒ष्ठा न्योज॑ ए॒वै वौजः॑ पृ॒ष्ठानि॑ पृ॒ष्ठा न्योज॑ ए॒व । \newline
5. ओज॑ ए॒वै वौज॒ ओज॑ ए॒व वी॒र्यं॑ ॅवी॒र्य॑ मे॒वौज॒ ओज॑ ए॒व वी॒र्य᳚म् । \newline
6. ए॒व वी॒र्यं॑ ॅवी॒र्य॑ मे॒वैव वी॒र्य॑म् मद्ध्य॒तो म॑द्ध्य॒तो वी॒र्य॑ मे॒वैव वी॒र्य॑म् मद्ध्य॒तः । \newline
7. वी॒र्य॑म् मद्ध्य॒तो म॑द्ध्य॒तो वी॒र्यं॑ ॅवी॒र्य॑म् मद्ध्य॒तो द॑धते दधते मद्ध्य॒तो वी॒र्यं॑ ॅवी॒र्य॑म् मद्ध्य॒तो द॑धते । \newline
8. म॒द्ध्य॒तो द॑धते दधते मद्ध्य॒तो म॑द्ध्य॒तो द॑धते बृहद्रथन्त॒राभ्या᳚म् बृहद्रथन्त॒राभ्या᳚म् दधते मद्ध्य॒तो म॑द्ध्य॒तो द॑धते बृहद्रथन्त॒राभ्या᳚म् । \newline
9. द॒ध॒ते॒ बृ॒ह॒द्र॒थ॒न्त॒राभ्या᳚म् बृहद्रथन्त॒राभ्या᳚म् दधते दधते बृहद्रथन्त॒राभ्यां᳚ ॅयन्ति यन्ति बृहद्रथन्त॒राभ्या᳚म् दधते दधते बृहद्रथन्त॒राभ्यां᳚ ॅयन्ति । \newline
10. बृ॒ह॒द्र॒थ॒न्त॒राभ्यां᳚ ॅयन्ति यन्ति बृहद्रथन्त॒राभ्या᳚म् बृहद्रथन्त॒राभ्यां᳚ ॅयन्ती॒य मि॒यं ॅय॑न्ति बृहद्रथन्त॒राभ्या᳚म् बृहद्रथन्त॒राभ्यां᳚ ॅयन्ती॒यम् । \newline
11. बृ॒ह॒द्र॒थ॒न्त॒राभ्या॒मिति॑ बृहत् - र॒थ॒न्त॒राभ्या᳚म् । \newline
12. य॒न्ती॒य मि॒यं ॅय॑न्ति यन्ती॒यं ॅवाव वावे यं ॅय॑न्ति यन्ती॒यं ॅवाव । \newline
13. इ॒यं ॅवाव वावेय मि॒यं ॅवाव र॑थन्त॒रꣳ र॑थन्त॒रं ॅवावेय मि॒यं ॅवाव र॑थन्त॒रम् । \newline
14. वाव र॑थन्त॒रꣳ र॑थन्त॒रं ॅवाव वाव र॑थन्त॒र म॒सा व॒सौ र॑थन्त॒रं ॅवाव वाव र॑थन्त॒र म॒सौ । \newline
15. र॒थ॒न्त॒र म॒सा व॒सौ र॑थन्त॒रꣳ र॑थन्त॒र म॒सौ बृ॒हद् बृ॒हद॒सौ र॑थन्त॒रꣳ र॑थन्त॒र म॒सौ बृ॒हत् । \newline
16. र॒थ॒न्त॒रमिति॑ रथं - त॒रम् । \newline
17. अ॒सौ बृ॒हद् बृ॒ह द॒सा व॒सौ बृ॒ह दा॒भ्या मा॒भ्याम् बृ॒ह द॒सा व॒सौ बृ॒ह दा॒भ्याम् । \newline
18. बृ॒ह दा॒भ्या मा॒भ्याम् बृ॒हद् बृ॒ह दा॒भ्या मे॒वै वाभ्याम् बृ॒हद् बृ॒ह दा॒भ्या मे॒व । \newline
19. आ॒भ्या मे॒वै वाभ्या मा॒भ्या मे॒व य॑न्ति यन्त्ये॒वाभ्या मा॒भ्या मे॒व य॑न्ति । \newline
20. ए॒व य॑न्ति यन्त्ये॒ वैव य॒न्त्यथो॒ अथो॑ यन्त्ये॒ वैव य॒न्त्यथो᳚ । \newline
21. य॒न्त्यथो॒ अथो॑ यन्ति य॒न्त्यथो॑ अ॒नयो॑ र॒नयो॒ रथो॑ यन्ति य॒न्त्यथो॑ अ॒नयोः᳚ । \newline
22. अथो॑ अ॒नयो॑ र॒नयो॒ रथो॒ अथो॑ अ॒नयो॑ रे॒वै वानयो॒ रथो॒ अथो॑ अ॒नयो॑ रे॒व । \newline
23. अथो॒ इत्यथो᳚ । \newline
24. अ॒नयो॑ रे॒वै वानयो॑ र॒नयो॑ रे॒व प्रति॒ प्रत्ये॒ वानयो॑ र॒नयो॑ रे॒व प्रति॑ । \newline
25. ए॒व प्रति॒ प्रत्ये॒ वैव प्रति॑ तिष्ठन्ति तिष्ठन्ति॒ प्रत्ये॒ वैव प्रति॑ तिष्ठन्ति । \newline
26. प्रति॑ तिष्ठन्ति तिष्ठन्ति॒ प्रति॒ प्रति॑ तिष्ठ न्त्ये॒ते ए॒ते ति॑ष्ठन्ति॒ प्रति॒ प्रति॑ तिष्ठ न्त्ये॒ते । \newline
27. ति॒ष्ठ॒ न्त्ये॒ते ए॒ते ति॑ष्ठन्ति तिष्ठ न्त्ये॒ते वै वा ए॒ते ति॑ष्ठन्ति तिष्ठ न्त्ये॒ते वै । \newline
28. ए॒ते वै वा ए॒ते ए॒ते वै य॒ज्ञ्स्य॑ य॒ज्ञ्स्य॒ वा ए॒ते ए॒ते वै य॒ज्ञ्स्य॑ । \newline
29. ए॒ते इत्ये॒ते । \newline
30. वै य॒ज्ञ्स्य॑ य॒ज्ञ्स्य॒ वै वै य॒ज्ञ् स्या᳚ञ्ज॒साय॑नी अञ्ज॒साय॑नी य॒ज्ञ्स्य॒ वै वै य॒ज्ञ् स्या᳚ञ्ज॒साय॑नी । \newline
31. य॒ज्ञ् स्या᳚ञ्ज॒साय॑नी अञ्ज॒साय॑नी य॒ज्ञ्स्य॑ य॒ज्ञ् स्या᳚ञ्ज॒साय॑नी स्रु॒ती स्रु॒ती अ॑ञ्ज॒साय॑नी य॒ज्ञ्स्य॑ य॒ज्ञ् स्या᳚ञ्ज॒साय॑नी स्रु॒ती । \newline
32. अ॒ञ्ज॒साय॑नी स्रु॒ती स्रु॒ती अ॑ञ्ज॒साय॑नी अञ्ज॒साय॑नी स्रु॒ती ताभ्या॒म् ताभ्याꣳ॑ स्रु॒ती अ॑ञ्ज॒साय॑नी अञ्ज॒साय॑नी स्रु॒ती ताभ्या᳚म् । \newline
33. अ॒ञ्ज॒साय॑नी॒ इत्य॑ञ्जसा - अय॑नी । \newline
34. स्रु॒ती ताभ्या॒म् ताभ्याꣳ॑ स्रु॒ती स्रु॒ती ताभ्या॑ मे॒वैव ताभ्याꣳ॑ स्रु॒ती स्रु॒ती ताभ्या॑ मे॒व । \newline
35. स्रु॒ती इति॑ स्रु॒ती । \newline
36. ताभ्या॑ मे॒वैव ताभ्या॒म् ताभ्या॑ मे॒व सु॑व॒र्गꣳ सु॑व॒र्ग मे॒व ताभ्या॒म् ताभ्या॑ मे॒व सु॑व॒र्गम् । \newline
37. ए॒व सु॑व॒र्गꣳ सु॑व॒र्ग मे॒वैव सु॑व॒र्गम् ॅलो॒कम् ॅलो॒कꣳ सु॑व॒र्ग मे॒वैव सु॑व॒र्गम् ॅलो॒कम् । \newline
38. सु॒व॒र्गम् ॅलो॒कम् ॅलो॒कꣳ सु॑व॒र्गꣳ सु॑व॒र्गम् ॅलो॒कं ॅय॑न्ति यन्ति लो॒कꣳ सु॑व॒र्गꣳ सु॑व॒र्गम् ॅलो॒कं ॅय॑न्ति । \newline
39. सु॒व॒र्गमिति॑ सुवः - गम् । \newline
40. लो॒कं ॅय॑न्ति यन्ति लो॒कम् ॅलो॒कं ॅय॑न्ति॒ परा᳚ञ्चः॒ परा᳚ञ्चो यन्ति लो॒कम् ॅलो॒कं ॅय॑न्ति॒ परा᳚ञ्चः । \newline
41. य॒न्ति॒ परा᳚ञ्चः॒ परा᳚ञ्चो यन्ति यन्ति॒ परा᳚ञ्चो॒ वै वै परा᳚ञ्चो यन्ति यन्ति॒ परा᳚ञ्चो॒ वै । \newline
42. परा᳚ञ्चो॒ वै वै परा᳚ञ्चः॒ परा᳚ञ्चो॒ वा ए॒त ए॒ते वै परा᳚ञ्चः॒ परा᳚ञ्चो॒ वा ए॒ते । \newline
43. वा ए॒त ए॒ते वै वा ए॒ते सु॑व॒र्गꣳ सु॑व॒र्ग मे॒ते वै वा ए॒ते सु॑व॒र्गम् । \newline
44. ए॒ते सु॑व॒र्गꣳ सु॑व॒र्ग मे॒त ए॒ते सु॑व॒र्गम् ॅलो॒कम् ॅलो॒कꣳ सु॑व॒र्ग मे॒त ए॒ते सु॑व॒र्गम् ॅलो॒कम् । \newline
45. सु॒व॒र्गम् ॅलो॒कम् ॅलो॒कꣳ सु॑व॒र्गꣳ सु॑व॒र्गम् ॅलो॒क म॒भ्यारो॑ह न्त्य॒भ्यारो॑हन्ति लो॒कꣳ सु॑व॒र्गꣳ सु॑व॒र्गम् ॅलो॒क म॒भ्यारो॑हन्ति । \newline
46. सु॒व॒र्गमिति॑ सुवः - गम् । \newline
47. लो॒क म॒भ्यारो॑ह न्त्य॒भ्यारो॑हन्ति लो॒कम् ॅलो॒क म॒भ्यारो॑हन्ति॒ ये ये᳚ ऽभ्यारो॑हन्ति लो॒कम् ॅलो॒क म॒भ्यारो॑हन्ति॒ ये । \newline
48. अ॒भ्यारो॑हन्ति॒ ये ये᳚ ऽभ्यारो॑ह न्त्य॒भ्यारो॑हन्ति॒ ये प॑रा॒चीना॑नि परा॒चीना॑नि॒ ये᳚ ऽभ्यारो॑ह न्त्य॒भ्यारो॑हन्ति॒ ये प॑रा॒चीना॑नि । \newline
49. अ॒भ्यारो॑ह॒न्तीत्य॑भि - आरो॑हन्ति । \newline
50. ये प॑रा॒चीना॑नि परा॒चीना॑नि॒ ये ये प॑रा॒चीना॑नि पृ॒ष्ठानि॑ पृ॒ष्ठानि॑ परा॒चीना॑नि॒ ये ये प॑रा॒चीना॑नि पृ॒ष्ठानि॑ । \newline
51. प॒रा॒चीना॑नि पृ॒ष्ठानि॑ पृ॒ष्ठानि॑ परा॒चीना॑नि परा॒चीना॑नि पृ॒ष्ठा न्यु॑प॒य न्त्यु॑प॒यन्ति॑ पृ॒ष्ठानि॑ परा॒चीना॑नि परा॒चीना॑नि पृ॒ष्ठा न्यु॑प॒यन्ति॑ । \newline
52. पृ॒ष्ठा न्यु॑प॒य न्त्यु॑प॒यन्ति॑ पृ॒ष्ठानि॑ पृ॒ष्ठा न्यु॑प॒यन्ति॑ प्र॒त्यङ् प्र॒त्यङ् ङु॑प॒यन्ति॑ पृ॒ष्ठानि॑ पृ॒ष्ठा न्यु॑प॒यन्ति॑ प्र॒त्यङ् । \newline
53. उ॒प॒यन्ति॑ प्र॒त्यङ् प्र॒त्यङ् ङु॑प॒य न्त्यु॑प॒यन्ति॑ प्र॒त्यङ् त्र्य॒ह स्त्र्य॒हः प्र॒त्यङ् ङु॑प॒य न्त्यु॑प॒यन्ति॑ प्र॒त्यङ् त्र्य॒हः । \newline
54. उ॒प॒यन्तीत्यु॑प - यन्ति॑ । \newline
55. प्र॒त्यङ् त्र्य॒ह स्त्र्य॒हः प्र॒त्यङ् प्र॒त्यङ् त्र्य॒हो भ॑वति भवति त्र्य॒हः प्र॒त्यङ् प्र॒त्यङ् त्र्य॒हो भ॑वति । \newline
56. त्र्य॒हो भ॑वति भवति त्र्य॒ह स्त्र्य॒हो भ॑वति प्र॒त्यव॑रूढ्यै प्र॒त्यव॑रूढ्यै भवति त्र्य॒हस्त्र्य॒हो भ॑वति प्र॒त्यव॑रूढ्यै । \newline
57. त्र्य॒ह इति॑ त्रि - अ॒हः । \newline
58. भ॒व॒ति॒ प्र॒त्यव॑रूढ्यै प्र॒त्यव॑रूढ्यै भवति भवति प्र॒त्यव॑रूढ्या॒ अथो॒ अथो᳚ प्र॒त्यव॑रूढ्यै भवति भवति प्र॒त्यव॑रूढ्या॒ अथो᳚ । \newline
59. प्र॒त्यव॑रूढ्या॒ अथो॒ अथो᳚ प्र॒त्यव॑रूढ्यै प्र॒त्यव॑रूढ्या॒ अथो॒ प्रति॑ष्ठित्यै॒ प्रति॑ष्ठित्या॒ अथो᳚ प्र॒त्यव॑रूढ्यै प्र॒त्यव॑रूढ्या॒ अथो॒ प्रति॑ष्ठित्यै । \newline
60. प्र॒त्यव॑रूढ्या॒ इति॑ प्रति - अव॑रूढ्यै । \newline
61. अथो॒ प्रति॑ष्ठित्यै॒ प्रति॑ष्ठित्या॒ अथो॒ अथो॒ प्रति॑ष्ठित्या उ॒भयो॑ रु॒भयोः॒ प्रति॑ष्ठित्या॒ अथो॒ अथो॒ प्रति॑ष्ठित्या उ॒भयोः᳚ । \newline
62. अथो॒ इत्यथो᳚ । \newline
63. प्रति॑ष्ठित्या उ॒भयो॑ रु॒भयोः॒ प्रति॑ष्ठित्यै॒ प्रति॑ष्ठित्या उ॒भयो᳚र् लो॒कयो᳚र् लो॒कयो॑ रु॒भयोः॒ प्रति॑ष्ठित्यै॒ प्रति॑ष्ठित्या उ॒भयो᳚र् लो॒कयोः᳚ । \newline
64. प्रति॑ष्ठित्या॒ इति॒ प्रति॑ - स्थि॒त्यै॒ । \newline
65. उ॒भयो᳚र् लो॒कयो᳚र् लो॒कयो॑ रु॒भयो॑ रु॒भयो᳚र् लो॒कयोर्॑. ऋ॒द्ध्व र्‌द्ध्वा लो॒कयो॑ रु॒भयो॑ रु॒भयो᳚र् लो॒कयोर्॑. ऋ॒द्ध्वा । \newline
66. लो॒कयोर्॑. ऋ॒द्ध्व र्‌द्ध्वा लो॒कयो᳚र् लो॒कयोर्॑. ऋ॒द्ध्वोदु दृ॒द्ध्वा लो॒कयो᳚र् लो॒कयोर्॑. ऋ॒द्ध्वोत् । \newline
67. ऋ॒द्ध्वोदु दृ॒द्ध्व र्‌द्ध्वोत् ति॑ष्ठन्ति तिष्ठ॒ न्त्युदृ॒द्ध्व र्‌द्ध्वोत् ति॑ष्ठन्ति । \newline
68. उत् ति॑ष्ठन्ति तिष्ठ॒ न्त्युदुत् ति॑ष्ठन्ति॒ चतु॑र्दश॒ चतु॑र्दश तिष्ठ॒ न्त्युदुत् ति॑ष्ठन्ति॒ चतु॑र्दश । \newline
69. ति॒ष्ठ॒न्ति॒ चतु॑र्दश॒ चतु॑र्दश तिष्ठन्ति तिष्ठन्ति॒ चतु॑र्दशै॒ता ए॒ता श्चतु॑र्दश तिष्ठन्ति तिष्ठन्ति॒ चतु॑र्दशै॒ताः । \newline
70. चतु॑र्दशै॒ता ए॒ता श्चतु॑र्दश॒ चतु॑र्दशै॒ता स्तासा॒म् तासा॑ मे॒ता श्चतु॑र्दश॒ चतु॑र्दशै॒ता स्तासा᳚म् । \newline
71. चतु॑र्द॒शेति॒ चतुः॑ - द॒श॒ । \newline
72. ए॒ता स्तासा॒म् तासा॑ मे॒ता ए॒ता स्तासां॒ ॅया या स्तासा॑ मे॒ता ए॒ता स्तासां॒ ॅयाः । \newline
73. तासां॒ ॅया या स्तासा॒म् तासां॒ ॅया दश॒ दश॒ या स्तासा॒म् तासां॒ ॅया दश॑ । \newline
74. या दश॒ दश॒ या या दश॒ दशा᳚क्षरा॒ दशा᳚क्षरा॒ दश॒ या या दश॒ दशा᳚क्षरा । \newline
75. दश॒ दशा᳚क्षरा॒ दशा᳚क्षरा॒ दश॒ दश॒ दशा᳚क्षरा वि॒राड् वि॒राड् दशा᳚क्षरा॒ दश॒ दश॒ दशा᳚क्षरा वि॒राट् । \newline
76. दशा᳚क्षरा वि॒राड् वि॒राड् दशा᳚क्षरा॒ दशा᳚क्षरा वि॒रा डन्न॒ मन्नं॑ ॅवि॒राड् दशा᳚क्षरा॒ दशा᳚क्षरा वि॒रा डन्न᳚म् । \newline
77. दशा᳚क्ष॒रेति॒ दश॑ - अ॒क्ष॒रा॒ । \newline
78. वि॒रा डन्न॒ मन्नं॑ ॅवि॒राड् वि॒रा डन्नं॑ ॅवि॒राड् वि॒रा डन्नं॑ ॅवि॒राड् वि॒रा डन्नं॑ ॅवि॒राट् । \newline
79. वि॒राडिति॑ वि - राट् । \newline
80. अन्नं॑ ॅवि॒राड् वि॒रा डन्न॒ मन्नं॑ ॅवि॒राड् वि॒राजा॑ वि॒राजा॑ वि॒रा डन्न॒ मन्नं॑ ॅवि॒राड् वि॒राजा᳚ । \newline
81. वि॒राड् वि॒राजा॑ वि॒राजा॑ वि॒राड् वि॒राड् वि॒राजै॒वैव वि॒राजा॑ वि॒राड् वि॒राड् वि॒रा जै॒व । \newline
82. वि॒राडिति॑ वि - राट् । \newline
83. वि॒राजै॒वैव वि॒राजा॑ वि॒राजै॒ वान्नाद्य॑ म॒न्नाद्य॑ मे॒व वि॒राजा॑ वि॒राजै॒ वान्नाद्य᳚म् । \newline
84. वि॒राजेति॑ वि - राजा᳚ । \newline
85. ए॒वान्नाद्य॑ म॒न्नाद्य॑ मे॒वै वान्नाद्य॒ मवावा॒न्नाद्य॑ मे॒वै वान्नाद्य॒ मव॑ । \newline
86. अ॒न्नाद्य॒ मवा वा॒न्नाद्य॑ म॒न्नाद्य॒ मव॑ रुन्धते रुन्ध॒ते ऽवा॒न्नाद्य॑ म॒न्नाद्य॒ मव॑ रुन्धते । \newline
87. अ॒न्नाद्य॒मित्य॑न्न - अद्य᳚म् । \newline
88. अव॑ रुन्धते रुन्ध॒ते ऽवाव॑ रुन्धते॒ या या रु॑न्ध॒ते ऽवाव॑ रुन्धते॒ याः । \newline
89. रु॒न्ध॒ते॒ या या रु॑न्धते रुन्धते॒ या श्चत॑स्र॒ श्चत॑स्रो॒ या रु॑न्धते रुन्धते॒ या श्चत॑स्रः । \newline
90. या श्चत॑स्र॒ श्चत॑स्रो॒ या या श्चत॑स्रः । \newline
91. चत॑स्र॒ श्चत॑स्रः । \newline
92. चत॑स्रो॒ दिशो॒ दिश॒ श्चत॑स्र॒ श्चत॑स्रो॒ दिशो॑ दि॒क्षु दि॒क्षु दिश॒ श्चत॑स्र॒ श्चत॑स्रो॒ दिशो॑ दि॒क्षु । \newline
93. दिशो॑ दि॒क्षु दि॒क्षु दिशो॒ दिशो॑ दि॒क्ष्वे॑वैव दि॒क्षु दिशो॒ दिशो॑ दि॒क्ष्वे॑व । \newline
94. दि॒क्ष्वे॑वैव दि॒क्षु दि॒क्ष्वे॑व प्रति॒ प्रत्ये॒व दि॒क्षु दि॒क्ष्वे॑व प्रति॑ । \newline
95. ए॒व प्रति॒ प्रत्ये॒ वैव प्रति॑ तिष्ठन्ति तिष्ठन्ति॒ प्रत्ये॒ वैव प्रति॑ तिष्ठन्ति । \newline
96. प्रति॑ तिष्ठन्ति तिष्ठन्ति॒ प्रति॒ प्रति॑ तिष्ठ न्त्यतिरा॒त्रा व॑तिरा॒त्रौ ति॑ष्ठन्ति॒ प्रति॒ प्रति॑ तिष्ठ न्त्यतिरा॒त्रौ । \newline
97. ति॒ष्ठ॒ न्त्य॒ति॒रा॒त्रा व॑तिरा॒त्रौ ति॑ष्ठन्ति तिष्ठ न्त्यतिरा॒त्रा व॒भितो॒ ऽभितो॑ ऽतिरा॒त्रौ ति॑ष्ठन्ति तिष्ठ न्त्यतिरा॒त्रा व॒भितः॑ । \newline
98. अ॒ति॒रा॒त्रा व॒भितो॒ ऽभितो॑ ऽतिरा॒त्रा व॑तिरा॒त्रा व॒भितो॑ भवतो भवतो॒ ऽभितो॑ ऽतिरा॒त्रा व॑तिरा॒त्रा व॒भितो॑ भवतः । \newline
99. अ॒ति॒रा॒त्रावित्य॑ति - रा॒त्रौ । \newline
100. अ॒भितो॑ भवतो भवतो॒ ऽभितो॒ ऽभितो॑ भवतः॒ परि॑गृहीत्यै॒ परि॑गृहीत्यै भवतो॒ ऽभितो॒ ऽभितो॑ भवतः॒ परि॑गृहीत्यै । \newline
101. भ॒व॒तः॒ परि॑गृहीत्यै॒ परि॑गृहीत्यै भवतो भवतः॒ परि॑गृहीत्यै । \newline
102. परि॑गृहीत्या॒ इति॒ परि॑ - गृ॒ही॒त्यै॒ । \newline
\pagebreak
\markright{ TS 7.3.6.1  \hfill https://www.vedavms.in \hfill}

\section{ TS 7.3.6.1 }

\textbf{TS 7.3.6.1 } \newline
\textbf{Samhita Paata} \newline

इन्द्रो॒ वै स॒दृङ् दे॒वता॑भिरासी॒थ् स न व्या॒वृत॑मगच्छ॒थ् स प्र॒जाप॑ति॒मुपा॑धाव॒त् तस्मा॑ ए॒तं प॑ञ्चदशरा॒त्रं प्राय॑च्छ॒त् तमाऽह॑र॒त् तेना॑यजत॒ ततो॒ वै सो᳚ऽन्याभि॑-र्दे॒वता॑भि-र्व्या॒वृत॑मगच्छ॒द्य ए॒वं ॅवि॒द्वाꣳसः॑ पञ्चदशरा॒त्रमास॑ते व्या॒वृत॑मे॒व पा॒प्मना॒ भ्रातृ॑व्येण गच्छन्ति॒ ज्योति॒र्गौरायु॒रिति॑ त्र्य॒हो भ॑वती॒यं ॅवाव ज्योति॑र॒न्तरि॑क्षं॒ - [  ] \newline

\textbf{Pada Paata} \newline

इन्द्रः॑ । वै । स॒दृङ्ङिति॑ स - दृङ् । दे॒वता॑भिः । आ॒सी॒त् । सः । न । व्या॒वृत॒मिति॑ वि - आ॒वृत᳚म् । अ॒ग॒च्छ॒त् । सः । प्र॒जाप॑ति॒मिति॑ प्र॒जा - प॒ति॒म् । उपेति॑ । अ॒धा॒व॒त् । तस्मै᳚ । ए॒तम् । प॒ञ्च॒द॒श॒रा॒त्रमिति॑ पञ्चदश-रा॒त्रम् । प्रेति॑ । अ॒य॒च्छ॒त् । तम् । एति॑ । अ॒ह॒र॒त् । तेन॑ । अ॒य॒ज॒त॒ । ततः॑ । वै । सः । अ॒न्याभिः॑ । दे॒वता॑भिः । व्या॒वृत॒मिति॑ वि - आ॒वृत᳚म् । अ॒ग॒च्छ॒त् । ये । ए॒वम् । वि॒द्वाꣳसः॑ । प॒ञ्च॒द॒श॒रा॒त्रमिति॑ पञ्चदश - रा॒त्रम् । आस॑ते । व्या॒वृत॒मिति॑ वि - आ॒वृत᳚म् । ए॒व । पा॒प्मना᳚ । भ्रातृ॑व्येण । ग॒च्छ॒न्ति॒ । ज्योतिः॑ । गौः । आयुः॑ । इति॑ । त्र्य॒ह इति॑ त्रि - अ॒हः । भ॒व॒ति॒ । इ॒यम् । वाव । ज्योतिः॑ । अ॒न्तरि॑क्षम् ।  \newline


\textbf{Krama Paata} \newline

इन्द्रो॒ वै । वै स॒दृङ्‍ङ् । स॒दृङ् दे॒वता॑भिः । स॒दृङ्‍ङिति॑ स - दृङ्‍ङ् । दे॒वता॑भिरासीत् । आ॒सी॒थ् सः । स न । न व्या॒वृत᳚म् । व्या॒वृत॑मगच्छत् । व्या॒वृत॒मिति॑ वि - आ॒वृत᳚म् । अ॒ग॒च्छ॒थ् सः । स प्र॒जाप॑तिम् । प्र॒जाप॑ति॒मुप॑ । प्र॒जाप॑ति॒मिति॑ प्र॒जा - प॒ति॒म् । उपा॑धावत् । अ॒धा॒व॒त् तस्मै᳚ । तस्मा॑ ए॒तम् । ए॒तम् प॑ञ्चदशरा॒त्रम् । प॒ञ्च॒द॒श॒रा॒त्रम् प्र । प॒ञ्च॒द॒श॒रा॒त्रमिति॑ पञ्चदश - रा॒त्रम् । प्राय॑च्छत् । अ॒य॒च्छ॒त् तम् । तमा । आऽह॑रत् । अ॒ह॒र॒त् तेन॑ । तेना॑यजत । अ॒य॒ज॒त॒ ततः॑ । ततो॒ वै । वै सः । सो᳚ऽन्याभिः॑ । अ॒न्याभि॑र् दे॒वता॑भिः । दे॒वता॑भिर् व्या॒वृत᳚म् । व्या॒वृत॑मगच्छत् । व्या॒वृत॒मिति॑ वि - आ॒वृत᳚म् । अ॒ग॒च्छ॒द् ये । य ए॒वम् । ए॒वम् ॅवि॒द्वाꣳसः॑ । वि॒द्वाꣳसः॑ पञ्चदशरा॒त्रम् । प॒ञ्च॒द॒श॒,रा॒त्रमास॑ते । प॒ञ्च॒द॒श॒रा॒त्रमिति॑ पञ्चदश - रा॒त्रम् । आस॑ते व्या॒वृत᳚म् । व्या॒वृत॑मे॒व । व्या॒वृत॒मिति॑ वि - आ॒वृत᳚म् । ए॒व पा॒प्मना᳚ । पा॒प्मना॒ भ्रातृ॑व्येण । भ्रातृ॑व्येण गच्छन्ति । ग॒च्छ॒न्ति॒ ज्योतिः॑ । ज्योति॒र् गौः । गौरायुः॑ । आयु॒रिति॑ । इति॑ त्र्य॒हः । त्र्य॒हो भ॑वति । त्र्य॒ह इति॑ त्रि - अ॒हः । भ॒व॒ती॒यम् । इ॒यम् ॅवाव । वाव ज्योतिः॑ । ज्योति॑र॒न्तरि॑क्षम् । अ॒न्तरि॑क्ष॒म् गौः \newline

\textbf{Jatai Paata} \newline

1. इन्द्रो॒ वै वा इन्द्र॒ इन्द्रो॒ वै । \newline
2. वै स॒दृङ् ख्स॒दृङ्. वै वै स॒दृङ् । \newline
3. स॒दृङ् दे॒वता॑भिर् दे॒वता॑भिः स॒दृङ् ख्स॒दृङ् दे॒वता॑भिः । \newline
4. स॒दृङ्ङिति॑ स - दृङ् । \newline
5. दे॒वता॑भि रासी दासीद् दे॒वता॑भिर् दे॒वता॑भि रासीत् । \newline
6. आ॒सी॒थ् स स आ॑सी दासी॒थ् सः । \newline
7. स न न स स न । \newline
8. न व्या॒वृतं॑ ॅव्या॒वृत॒न् न न व्या॒वृत᳚म् । \newline
9. व्या॒वृत॑ मगच्छ दगच्छद् व्या॒वृतं॑ ॅव्या॒वृत॑ मगच्छत् । \newline
10. व्या॒वृत॒मिति॑ वि - आ॒वृत᳚म् । \newline
11. अ॒ग॒च्छ॒थ् स सो॑ ऽगच्छ दगच्छ॒थ् सः । \newline
12. स प्र॒जाप॑तिम् प्र॒जाप॑तिꣳ॒॒ स स प्र॒जाप॑तिम् । \newline
13. प्र॒जाप॑ति॒ मुपोप॑ प्र॒जाप॑तिम् प्र॒जाप॑ति॒ मुप॑ । \newline
14. प्र॒जाप॑ति॒मिति॑ प्र॒जा - प॒ति॒म् । \newline
15. उपा॑ धाव दधाव॒ दुपोपा॑ धावत् । \newline
16. अ॒धा॒व॒त् तस्मै॒ तस्मा॑ अधाव दधाव॒त् तस्मै᳚ । \newline
17. तस्मा॑ ए॒त मे॒तम् तस्मै॒ तस्मा॑ ए॒तम् । \newline
18. ए॒तम् प॑ञ्चदशरा॒त्रम् प॑ञ्चदशरा॒त्र मे॒त मे॒तम् प॑ञ्चदशरा॒त्रम् । \newline
19. प॒ञ्च॒द॒श॒रा॒त्रम् प्र प्र प॑ञ्चदशरा॒त्रम् प॑ञ्चदशरा॒त्रम् प्र । \newline
20. प॒ञ्च॒द॒श॒रा॒त्रमिति॑ पञ्चदश - रा॒त्रम् । \newline
21. प्रा य॑च्छ दयच्छ॒त् प्र प्रा य॑च्छत् । \newline
22. अ॒य॒च्छ॒त् तम् त म॑यच्छ दयच्छ॒त् तम् । \newline
23. त मा तम् त मा । \newline
24. आ ऽह॑र दहर॒दा ऽह॑रत् । \newline
25. अ॒ह॒र॒त् तेन॒ तेना॑ हर दहर॒त् तेन॑ । \newline
26. तेना॑ यजता यजत॒ तेन॒ तेना॑ यजत । \newline
27. अ॒य॒ज॒त॒ तत॒ स्ततो॑ ऽयजता यजत॒ ततः॑ । \newline
28. ततो॒ वै वै तत॒ स्ततो॒ वै । \newline
29. वै स स वै वै सः । \newline
30. सो᳚ ऽन्याभि॑ र॒न्याभिः॒ स सो᳚ ऽन्याभिः॑ । \newline
31. अ॒न्याभि॑र् दे॒वता॑भिर् दे॒वता॑भि र॒न्याभि॑ र॒न्याभि॑र् दे॒वता॑भिः । \newline
32. दे॒वता॑भिर् व्या॒वृतं॑ ॅव्या॒वृत॑म् दे॒वता॑भिर् दे॒वता॑भिर् व्या॒वृत᳚म् । \newline
33. व्या॒वृत॑ मगच्छ दगच्छद् व्या॒वृतं॑ ॅव्या॒वृत॑ मगच्छत् । \newline
34. व्या॒वृत॒मिति॑ वि - आ॒वृत᳚म् । \newline
35. अ॒ग॒च्छ॒द् ये ये॑ ऽगच्छ दगच्छ॒द् ये । \newline
36. य ए॒व मे॒वं ॅये य ए॒वम् । \newline
37. ए॒वं ॅवि॒द्वाꣳसो॑ वि॒द्वाꣳस॑ ए॒व मे॒वं ॅवि॒द्वाꣳसः॑ । \newline
38. वि॒द्वाꣳसः॑ पञ्चदशरा॒त्रम् प॑ञ्चदशरा॒त्रं ॅवि॒द्वाꣳसो॑ वि॒द्वाꣳसः॑ पञ्चदशरा॒त्रम् । \newline
39. प॒ञ्च॒द॒श॒रा॒त्र मास॑त॒ आस॑ते पञ्चदशरा॒त्रम् प॑ञ्चदशरा॒त्र मास॑ते । \newline
40. प॒ञ्च॒द॒श॒रा॒त्रमिति॑ पञ्चदश - रा॒त्रम् । \newline
41. आस॑ते व्या॒वृतं॑ ॅव्या॒वृत॒ मास॑त॒ आस॑ते व्या॒वृत᳚म् । \newline
42. व्या॒वृत॑ मे॒वैव व्या॒वृतं॑ ॅव्या॒वृत॑ मे॒व । \newline
43. व्या॒वृत॒मिति॑ वि - आ॒वृत᳚म् । \newline
44. ए॒व पा॒प्मना॑ पा॒प्मनै॒वैव पा॒प्मना᳚ । \newline
45. पा॒प्मना॒ भ्रातृ॑व्येण॒ भ्रातृ॑व्येण पा॒प्मना॑ पा॒प्मना॒ भ्रातृ॑व्येण । \newline
46. भ्रातृ॑व्येण गच्छन्ति गच्छन्ति॒ भ्रातृ॑व्येण॒ भ्रातृ॑व्येण गच्छन्ति । \newline
47. ग॒च्छ॒न्ति॒ ज्योति॒र् ज्योति॑र् गच्छन्ति गच्छन्ति॒ ज्योतिः॑ । \newline
48. ज्योति॒र् गौर् गौर् ज्योति॒र् ज्योति॒र् गौः । \newline
49. गौ रायु॒ रायु॒र् गौर् गौ रायुः॑ । \newline
50. आयु॒ रिती त्यायु॒ रायु॒ रिति॑ । \newline
51. इति॑ त्र्य॒ह स्त्र्य॒ह इतीति॑ त्र्य॒हः । \newline
52. त्र्य॒हो भ॑वति भवति त्र्य॒ह स्त्र्य॒हो भ॑वति । \newline
53. त्र्य॒ह इति॑ त्रि - अ॒हः । \newline
54. भ॒व॒ती॒य मि॒यम् भ॑वति भवती॒यम् । \newline
55. इ॒यं ॅवाव वावेय मि॒यं ॅवाव । \newline
56. वाव ज्योति॒र् ज्योति॒र् वाव वाव ज्योतिः॑ । \newline
57. ज्योति॑ र॒न्तरि॑क्ष म॒न्तरि॑क्ष॒म् ज्योति॒र् ज्योति॑ र॒न्तरि॑क्षम् । \newline
58. अ॒न्तरि॑क्ष॒म् गौर् गौर॒न्तरि॑क्ष म॒न्तरि॑क्ष॒म् गौः । \newline

\textbf{Ghana Paata } \newline

1. इन्द्रो॒ वै वा इन्द्र॒ इन्द्रो॒ वै स॒दृङ् ख्स॒दृङ्. वा इन्द्र॒ इन्द्रो॒ वै स॒दृङ् । \newline
2. वै स॒दृङ् ख्स॒दृङ्. वै वै स॒दृङ् दे॒वता॑भिर् दे॒वता॑भिः स॒दृङ्. वै वै स॒दृङ् दे॒वता॑भिः । \newline
3. स॒दृङ् दे॒वता॑भिर् दे॒वता॑भिः स॒दृङ् ख्स॒दृङ् दे॒वता॑भि रासी दासीद् दे॒वता॑भिः स॒दृङ् ख्स॒दृङ् दे॒वता॑भि रासीत् । \newline
4. स॒दृङ्ङिति॑ स - दृङ् । \newline
5. दे॒वता॑भि रासी दासीद् दे॒वता॑भिर् दे॒वता॑भि रासी॒थ् स स आ॑सीद् दे॒वता॑भिर् दे॒वता॑भि रासी॒थ् सः । \newline
6. आ॒सी॒थ् स स आ॑सी दासी॒थ् स न न स आ॑सी दासी॒थ् स न । \newline
7. स न न स स न व्या॒वृतं॑ ॅव्या॒वृत॒न् न स स न व्या॒वृत᳚म् । \newline
8. न व्या॒वृतं॑ ॅव्या॒वृत॒न् न न व्या॒वृत॑ मगच्छ दगच्छद् व्या॒वृत॒न् न न व्या॒वृत॑ मगच्छत् । \newline
9. व्या॒वृत॑ मगच्छ दगच्छद् व्या॒वृतं॑ ॅव्या॒वृत॑ मगच्छ॒थ् स सो॑ ऽगच्छद् व्या॒वृतं॑ ॅव्या॒वृत॑ मगच्छ॒थ् सः । \newline
10. व्या॒वृत॒मिति॑ वि - आ॒वृत᳚म् । \newline
11. अ॒ग॒च्छ॒थ् स सो॑ ऽगच्छ दगच्छ॒थ् स प्र॒जाप॑तिम् प्र॒जाप॑तिꣳ॒॒ सो॑ ऽगच्छ दगच्छ॒थ् स प्र॒जाप॑तिम् । \newline
12. स प्र॒जाप॑तिम् प्र॒जाप॑तिꣳ॒॒ स स प्र॒जाप॑ति॒ मुपोप॑ प्र॒जाप॑तिꣳ॒॒ स स प्र॒जाप॑ति॒ मुप॑ । \newline
13. प्र॒जाप॑ति॒ मुपोप॑ प्र॒जाप॑तिम् प्र॒जाप॑ति॒ मुपा॑धाव दधाव॒ दुप॑ प्र॒जाप॑तिम् प्र॒जाप॑ति॒ मुपा॑धावत् । \newline
14. प्र॒जाप॑ति॒मिति॑ प्र॒जा - प॒ति॒म् । \newline
15. उपा॑धा वदधाव॒ दुपोपा॑ धाव॒त् तस्मै॒ तस्मा॑ अधाव॒ दुपोपा॑ धाव॒त् तस्मै᳚ । \newline
16. अ॒धा॒व॒त् तस्मै॒ तस्मा॑ अधाव दधाव॒त् तस्मा॑ ए॒त मे॒तम् तस्मा॑ अधाव दधाव॒त् तस्मा॑ ए॒तम् । \newline
17. तस्मा॑ ए॒त मे॒तम् तस्मै॒ तस्मा॑ ए॒तम् प॑ञ्चदशरा॒त्रम् प॑ञ्चदशरा॒त्र मे॒तम् तस्मै॒ तस्मा॑ ए॒तम् प॑ञ्चदशरा॒त्रम् । \newline
18. ए॒तम् प॑ञ्चदशरा॒त्रम् प॑ञ्चदशरा॒त्र मे॒त मे॒तम् प॑ञ्चदशरा॒त्रम् प्र प्र प॑ञ्चदशरा॒त्र मे॒त मे॒तम् प॑ञ्चदशरा॒त्रम् प्र । \newline
19. प॒ञ्च॒द॒श॒रा॒त्रम् प्र प्र प॑ञ्चदशरा॒त्रम् प॑ञ्चदशरा॒त्रम् प्राय॑च्छ दयच्छ॒त् प्र प॑ञ्चदशरा॒त्रम् प॑ञ्चदशरा॒त्रम् प्राय॑च्छत् । \newline
20. प॒ञ्च॒द॒श॒रा॒त्रमिति॑ पञ्चदश - रा॒त्रम् । \newline
21. प्राय॑च्छ दयच्छ॒त् प्र प्राय॑च्छ॒त् तम् त म॑यच्छ॒त् प्र प्राय॑च्छ॒त् तम् । \newline
22. अ॒य॒च्छ॒त् तम् त म॑यच्छ दयच्छ॒त् त मा त म॑यच्छ दयच्छ॒त् त मा । \newline
23. त मा तम् त मा ऽह॑र दहर॒दा तम् त मा ऽह॑रत् । \newline
24. आ ऽह॑र दहर॒दा ऽह॑र॒त् तेन॒ तेना॑ हर॒दा ऽह॑र॒त् तेन॑ । \newline
25. अ॒ह॒र॒त् तेन॒ तेना॑ हरद हर॒त् तेना॑ यजता यजत॒ तेना॑ हरद हर॒त् तेना॑ यजत । \newline
26. तेना॑ यजता यजत॒ तेन॒ तेना॑ यजत॒ तत॒ स्ततो॑ ऽयजत॒ तेन॒ तेना॑ यजत॒ ततः॑ । \newline
27. अ॒य॒ज॒त॒ तत॒ स्ततो॑ ऽयजता यजत॒ ततो॒ वै वै ततो॑ ऽयजता यजत॒ ततो॒ वै । \newline
28. ततो॒ वै वै तत॒ स्ततो॒ वै स स वै तत॒ स्ततो॒ वै सः । \newline
29. वै स स वै वै सो᳚ ऽन्याभि॑ र॒न्याभिः॒ स वै वै सो᳚ ऽन्याभिः॑ । \newline
30. सो᳚ ऽन्याभि॑ र॒न्याभिः॒ स सो᳚ ऽन्याभि॑र् दे॒वता॑भिर् दे॒वता॑भि र॒न्याभिः॒ स सो᳚ ऽन्याभि॑र् दे॒वता॑भिः । \newline
31. अ॒न्याभि॑र् दे॒वता॑भिर् दे॒वता॑भि र॒न्याभि॑ र॒न्याभि॑र् दे॒वता॑भिर् व्या॒वृतं॑ ॅव्या॒वृत॑म् दे॒वता॑भि र॒न्याभि॑ र॒न्याभि॑र् दे॒वता॑भिर् व्या॒वृत᳚म् । \newline
32. दे॒वता॑भिर् व्या॒वृतं॑ ॅव्या॒वृत॑म् दे॒वता॑भिर् दे॒वता॑भिर् व्या॒वृत॑ मगच्छ दगच्छद् व्या॒वृत॑म् दे॒वता॑भिर् दे॒वता॑भिर् व्या॒वृत॑ मगच्छत् । \newline
33. व्या॒वृत॑ मगच्छ दगच्छद् व्या॒वृतं॑ ॅव्या॒वृत॑ मगच्छ॒द् ये ये॑ ऽगच्छद् व्या॒वृतं॑ ॅव्या॒वृत॑ मगच्छ॒द् ये । \newline
34. व्या॒वृत॒मिति॑ वि - आ॒वृत᳚म् । \newline
35. अ॒ग॒च्छ॒द् ये ये॑ ऽगच्छ दगच्छ॒द् य ए॒व मे॒वं ॅये॑ ऽगच्छ दगच्छ॒द् य ए॒वम् । \newline
36. य ए॒व मे॒वं ॅये य ए॒वं ॅवि॒द्वाꣳसो॑ वि॒द्वाꣳस॑ ए॒वं ॅये य ए॒वं ॅवि॒द्वाꣳसः॑ । \newline
37. ए॒वं ॅवि॒द्वाꣳसो॑ वि॒द्वाꣳस॑ ए॒व मे॒वं ॅवि॒द्वाꣳसः॑ पञ्चदशरा॒त्रम् प॑ञ्चदशरा॒त्रं ॅवि॒द्वाꣳस॑ ए॒व मे॒वं ॅवि॒द्वाꣳसः॑ पञ्चदशरा॒त्रम् । \newline
38. वि॒द्वाꣳसः॑ पञ्चदशरा॒त्रम् प॑ञ्चदशरा॒त्रं ॅवि॒द्वाꣳसो॑ वि॒द्वाꣳसः॑ पञ्चदशरा॒त्र मास॑त॒ आस॑ते पञ्चदशरा॒त्रं ॅवि॒द्वाꣳसो॑ वि॒द्वाꣳसः॑ पञ्चदशरा॒त्र मास॑ते । \newline
39. प॒ञ्च॒द॒श॒रा॒त्र मास॑त॒ आस॑ते पञ्चदशरा॒त्रम् प॑ञ्चदशरा॒त्र मास॑ते व्या॒वृतं॑ ॅव्या॒वृत॒ मास॑ते पञ्चदशरा॒त्रम् प॑ञ्चदशरा॒त्र मास॑ते व्या॒वृत᳚म् । \newline
40. प॒ञ्च॒द॒श॒रा॒त्रमिति॑ पञ्चदश - रा॒त्रम् । \newline
41. आस॑ते व्या॒वृतं॑ ॅव्या॒वृत॒ मास॑त॒ आस॑ते व्या॒वृत॑ मे॒वैव व्या॒वृत॒ मास॑त॒ आस॑ते व्या॒वृत॑ मे॒व । \newline
42. व्या॒वृत॑ मे॒वैव व्या॒वृतं॑ ॅव्या॒वृत॑ मे॒व पा॒प्मना॑ पा॒प्मनै॒व व्या॒वृतं॑ ॅव्या॒वृत॑ मे॒व पा॒प्मना᳚ । \newline
43. व्या॒वृत॒मिति॑ वि - आ॒वृत᳚म् । \newline
44. ए॒व पा॒प्मना॑ पा॒प्मनै॒वैव पा॒प्मना॒ भ्रातृ॑व्येण॒ भ्रातृ॑व्येण पा॒प्मनै॒वैव पा॒प्मना॒ भ्रातृ॑व्येण । \newline
45. पा॒प्मना॒ भ्रातृ॑व्येण॒ भ्रातृ॑व्येण पा॒प्मना॑ पा॒प्मना॒ भ्रातृ॑व्येण गच्छन्ति गच्छन्ति॒ भ्रातृ॑व्येण पा॒प्मना॑ पा॒प्मना॒ भ्रातृ॑व्येण गच्छन्ति । \newline
46. भ्रातृ॑व्येण गच्छन्ति गच्छन्ति॒ भ्रातृ॑व्येण॒ भ्रातृ॑व्येण गच्छन्ति॒ ज्योति॒र् ज्योति॑र् गच्छन्ति॒ भ्रातृ॑व्येण॒ भ्रातृ॑व्येण गच्छन्ति॒ ज्योतिः॑ । \newline
47. ग॒च्छ॒न्ति॒ ज्योति॒र् ज्योति॑र् गच्छन्ति गच्छन्ति॒ ज्योति॒र् गौर् गौर् ज्योति॑र् गच्छन्ति गच्छन्ति॒ ज्योति॒र् गौः । \newline
48. ज्योति॒र् गौर् गौर् ज्योति॒र् ज्योति॒र् गौ रायु॒ रायु॒र् गौर् ज्योति॒र् ज्योति॒र् गौ रायुः॑ । \newline
49. गौ रायु॒ रायु॒र् गौर् गौ रायु॒ रिती त्यायु॒र् गौर् गौ रायु॒ रिति॑ । \newline
50. आयु॒ रिती त्यायु॒ रायु॒ रिति॑ त्र्य॒ह स्त्र्य॒ह इत्यायु॒ रायु॒ रिति॑ त्र्य॒हः । \newline
51. इति॑ त्र्य॒ह स्त्र्य॒ह इतीति॑ त्र्य॒हो भ॑वति भवति त्र्य॒ह इतीति॑ त्र्य॒हो भ॑वति । \newline
52. त्र्य॒हो भ॑वति भवति त्र्य॒ह स्त्र्य॒हो भ॑व ती॒य मि॒यम् भ॑वति त्र्य॒ह स्त्र्य॒हो भ॑व ती॒यम् । \newline
53. त्र्य॒ह इति॑ त्रि - अ॒हः । \newline
54. भ॒व॒ती॒य मि॒यम् भ॑वति भवती॒यं ॅवाव वावेयम् भ॑वति भवती॒यं ॅवाव । \newline
55. इ॒यं ॅवाव वावेय मि॒यं ॅवाव ज्योति॒र् ज्योति॒र् वावेय मि॒यं ॅवाव ज्योतिः॑ । \newline
56. वाव ज्योति॒र् ज्योति॒र् वाव वाव ज्योति॑ र॒न्तरि॑क्ष म॒न्तरि॑क्ष॒म् ज्योति॒र् वाव वाव ज्योति॑ र॒न्तरि॑क्षम् । \newline
57. ज्योति॑ र॒न्तरि॑क्ष म॒न्तरि॑क्ष॒म् ज्योति॒र् ज्योति॑ र॒न्तरि॑क्ष॒म् गौर् गौ र॒न्तरि॑क्ष॒म् ज्योति॒र् ज्योति॑ र॒न्तरि॑क्ष॒म् गौः । \newline
58. अ॒न्तरि॑क्ष॒म् गौर् गौ र॒न्तरि॑क्ष म॒न्तरि॑क्ष॒म् गौर॒सा व॒सौ गौ र॒न्तरि॑क्ष म॒न्तरि॑क्ष॒म् गौर॒सौ । \newline
\pagebreak
\markright{ TS 7.3.6.2  \hfill https://www.vedavms.in \hfill}

\section{ TS 7.3.6.2 }

\textbf{TS 7.3.6.2 } \newline
\textbf{Samhita Paata} \newline

गौर॒सावायु॑रे॒ष्वे॑व लो॒केषु॒ प्रति॑ तिष्ठ॒न्त्यस॑त्रं॒ ॅवा ए॒तद्-यद॑छन्दो॒मं ॅयच्छ॑न्दो॒मा भव॑न्ति॒ तेन॑ स॒त्रं दे॒वता॑ ए॒व पृ॒ष्ठैरव॑ रुन्धते प॒शूञ्छ॑न्दो॒मैरोजो॒ वै वी॒र्यं॑ पृ॒ष्ठानि॑ प॒शवः॑ छन्दो॒मा ओज॑स्ये॒व वी॒र्ये॑ प॒शुषु॒ प्रति॑ तिष्ठन्ति पञ्चदशरा॒त्रो भ॑वति पञ्चद॒शो वज्रो॒ वज्र॑मे॒व भ्रातृ॑व्येभ्यः॒ प्र ह॑रन्त्यतिरा॒त्राव॒भितो॑ भवत इन्द्रि॒यस्य॒ ( ) परि॑गृहीत्यै ॥ \newline

\textbf{Pada Paata} \newline

गौः । अ॒सौ । आयुः॑ । ए॒षु । ए॒व । लो॒केषु॑ । प्रतीति॑ । ति॒ष्ठ॒न्ति॒ । अस॑त्रम् । वै । ए॒तत् । यत् । अ॒छ॒न्दो॒ममित्य॑छन्दः - मम् । यत् । छ॒न्दो॒मा इति॑ छन्दः - माः । भव॑न्ति । तेन॑ । स॒त्रम् । दे॒वताः᳚ । ए॒व । पृ॒ष्ठैः । अवेति॑ । रु॒न्ध॒ते॒ । प॒शून् । छ॒न्दो॒मैरिति॑ छन्दः-मैः । ओजः॑ । वै । वी॒र्य᳚म् । पृ॒ष्ठानि॑ । प॒शवः॑ । छ॒न्दो॒मा इति॑ छन्दः - माः । ओज॑सि । ए॒व । वी॒र्ये᳚ । प॒शुषु॑ । प्रतीति॑ । ति॒ष्ठ॒न्ति॒ । प॒ञ्च॒द॒श॒रा॒त्र इति॑ पञ्चदश - रा॒त्रः । भ॒व॒ति॒ । प॒ञ्च॒द॒श इति॑ पञ्च - द॒शः । वज्रः॑ । वज्र᳚म् । ए॒व । भ्रातृ॑व्येभ्यः । प्रेति॑ । ह॒र॒न्ति॒ । अ॒ति॒रा॒त्रावित्य॑ति - रा॒त्रौ । अ॒भितः॑ । भ॒व॒तः॒ । इ॒न्द्रि॒यस्य॑ ( ) । परि॑गृहीत्या॒ इति॒ परि॑ - गृ॒ही॒त्यै॒ ॥  \newline


\textbf{Krama Paata} \newline

गौर॒सौ । अ॒सावायुः॑ । आयु॑रे॒षु । ए॒ष्वे॑व । ए॒व लो॒केषु॑ । लो॒केषु॒ प्रति॑ । प्रति॑ तिष्ठन्ति । ति॒ष्ठ॒न्त्यस॑त्रम् । अस॑त्र॒म् ॅवै । वा ए॒तत् । ए॒तद् यत् । यद॑छन्दो॒मम् । अ॒छ॒न्दो॒मम् ॅयत् । अ॒छ॒न्दो॒ममित्य॑छन्दः - मम् । यच् छ॑न्दो॒माः । छ॒न्दो॒मा भव॑न्ति । छ॒न्दो॒मा इति॑ छन्दः - माः । भव॑न्ति॒ तेन॑ । तेन॑ स॒त्रम् । स॒त्रम् दे॒वताः᳚ । दे॒वता॑ ए॒व । ए॒व पृ॒ष्ठैः । पृ॒ष्ठैरव॑ । अव॑ रुन्धते । रु॒न्ध॒ते॒ प॒शून् । प॒शूञ्छ॑न्दो॒मैः । छ॒न्दो॒मैरोजः॑ । छ॒न्दो॒मैरिति॑ छन्दः - मैः । ओजो॒ वै । वै वी॒र्य᳚म् । वी॒र्य॑म् पृ॒ष्ठानि॑ । पृ॒ष्ठानि॑ प॒शवः॑ । प॒शव॑श्छन्दो॒माः । छ॒न्दो॒मा ओज॑सि । छ॒न्दो॒मा इति॑ छन्दः - माः । ओज॑स्ये॒व । ए॒व वी॒र्ये᳚ । वी॒र्ये॑ प॒शुषु॑ । प॒शुषु॒ प्रति॑ । प्रति॑ तिष्ठन्ति । ति॒ष्ठ॒न्ति॒ प॒ञ्च॒द॒श॒रा॒त्रः । प॒ञ्च॒द॒श॒रा॒त्रो भ॑वति । प॒ञ्च॒द॒श॒रा॒त्र इति॑ पञ्चदश - रा॒त्रः । भ॒व॒ति॒ प॒ञ्च॒द॒शः । प॒ञ्च॒द॒शो वज्रः॑ । प॒ञ्च॒द॒श इति॑ पञ्च - द॒शः । वज्रो॒ वज्र᳚म् । वज्र॑मे॒व । ए॒व भ्रातृ॑व्येभ्यः । भ्रातृ॑व्येभ्यः॒ प्र । प्र ह॑रन्ति । ह॒र॒न्त्य॒ति॒रा॒त्रौ । अ॒ति॒रा॒त्राव॒भितः॑ । अ॒ति॒रा॒त्रावित्य॑ति - रा॒त्रौ । अ॒भितो॑ भवतः । भ॒व॒त॒ इ॒न्द्रि॒यस्य॑ ( ) । इ॒न्द्रि॒यस्य॒ परि॑गृहीत्यै । परि॑गृहीत्या॒ इति॒ परि॑ - गृ॒ही॒त्यै॒ । \newline

\textbf{Jatai Paata} \newline

1. गौ र॒सा व॒सौ गौर् गौ र॒सौ । \newline
2. अ॒सा वायु॒ रायु॑ र॒सा व॒सा वायुः॑ । \newline
3. आयु॑ रे॒ष्वे᳚ ष्वायु॒ रायु॑ रे॒षु । \newline
4. ए॒ष्वे॑ वैवै ष्वे᳚(1॒)ष्वे॑व । \newline
5. ए॒व लो॒केषु॑ लो॒के ष्वे॒वैव लो॒केषु॑ । \newline
6. लो॒केषु॒ प्रति॒ प्रति॑ लो॒केषु॑ लो॒केषु॒ प्रति॑ । \newline
7. प्रति॑ तिष्ठन्ति तिष्ठन्ति॒ प्रति॒ प्रति॑ तिष्ठन्ति । \newline
8. ति॒ष्ठ॒ न्त्यस॑त्र॒ मस॑त्रम् तिष्ठन्ति तिष्ठ॒ न्त्यस॑त्रम् । \newline
9. अस॑त्रं॒ ॅवै वा अस॑त्र॒ मस॑त्रं॒ ॅवै । \newline
10. वा ए॒त दे॒तद् वै वा ए॒तत् । \newline
11. ए॒तद् यद् यदे॒त दे॒तद् यत् । \newline
12. यद॑छन्दो॒म म॑छन्दो॒मं ॅयद् यद॑छन्दो॒मम् । \newline
13. अ॒छ॒न्दो॒मं ॅयद् यद॑छन्दो॒म म॑छन्दो॒मं ॅयत् । \newline
14. अ॒छ॒न्दो॒ममित्य॑छन्दः - मम् । \newline
15. यच् छ॑न्दो॒मा श्छ॑न्दो॒मा यद् यच् छ॑न्दो॒माः । \newline
16. छ॒न्दो॒मा भव॑न्ति॒ भव॑न्ति छन्दो॒मा श्छ॑न्दो॒मा भव॑न्ति । \newline
17. छ॒न्दो॒मा इति॑ छन्दः - माः । \newline
18. भव॑न्ति॒ तेन॒ तेन॒ भव॑न्ति॒ भव॑न्ति॒ तेन॑ । \newline
19. तेन॑ स॒त्रꣳ स॒त्रम् तेन॒ तेन॑ स॒त्रम् । \newline
20. स॒त्रम् दे॒वता॑ दे॒वताः᳚ स॒त्रꣳ स॒त्रम् दे॒वताः᳚ । \newline
21. दे॒वता॑ ए॒वैव दे॒वता॑ दे॒वता॑ ए॒व । \newline
22. ए॒व पृ॒ष्ठैः पृ॒ष्ठै रे॒वैव पृ॒ष्ठैः । \newline
23. पृ॒ष्ठै रवाव॑ पृ॒ष्ठैः पृ॒ष्ठै रव॑ । \newline
24. अव॑ रुन्धते रुन्ध॒ते ऽवाव॑ रुन्धते । \newline
25. रु॒न्ध॒ते॒ प॒शून् प॒शून् रु॑न्धते रुन्धते प॒शून् । \newline
26. प॒शूञ् छ॑न्दो॒मै श्छ॑न्दो॒मैः प॒शून् प॒शूञ् छ॑न्दो॒मैः । \newline
27. छ॒न्दो॒मै रोज॒ ओज॑ श्छन्दो॒मै श्छ॑न्दो॒मै रोजः॑ । \newline
28. छ॒न्दो॒मैरिति॑ छन्दः - मैः । \newline
29. ओजो॒ वै वा ओज॒ ओजो॒ वै । \newline
30. वै वी॒र्यं॑ ॅवी॒र्यं॑ ॅवै वै वी॒र्य᳚म् । \newline
31. वी॒र्य॑म् पृ॒ष्ठानि॑ पृ॒ष्ठानि॑ वी॒र्यं॑ ॅवी॒र्य॑म् पृ॒ष्ठानि॑ । \newline
32. पृ॒ष्ठानि॑ प॒शवः॑ प॒शवः॑ पृ॒ष्ठानि॑ पृ॒ष्ठानि॑ प॒शवः॑ । \newline
33. प॒शव॑ श्छन्दो॒मा श्छ॑न्दो॒माः प॒शवः॑ प॒शव॑ श्छन्दो॒माः । \newline
34. छ॒न्दो॒मा ओज॒ स्योज॑सि छन्दो॒मा श्छ॑न्दो॒मा ओज॑सि । \newline
35. छ॒न्दो॒मा इति॑ छन्दः - माः । \newline
36. ओज॑ स्ये॒वै वौज॒ स्योज॑ स्ये॒व । \newline
37. ए॒व वी॒र्ये॑ वी॒र्य॑ ए॒वैव वी॒र्ये᳚ । \newline
38. वी॒र्ये॑ प॒शुषु॑ प॒शुषु॑ वी॒र्ये॑ वी॒र्ये॑ प॒शुषु॑ । \newline
39. प॒शुषु॒ प्रति॒ प्रति॑ प॒शुषु॑ प॒शुषु॒ प्रति॑ । \newline
40. प्रति॑ तिष्ठन्ति तिष्ठन्ति॒ प्रति॒ प्रति॑ तिष्ठन्ति । \newline
41. ति॒ष्ठ॒न्ति॒ प॒ञ्च॒द॒श॒रा॒त्रः प॑ञ्चदशरा॒ त्रस्ति॑ष्ठन्ति तिष्ठन्ति पञ्चदशरा॒त्रः । \newline
42. प॒ञ्च॒द॒श॒रा॒त्रो भ॑वति भवति पञ्चदशरा॒त्रः प॑ञ्चदशरा॒त्रो भ॑वति । \newline
43. प॒ञ्च॒द॒श॒रा॒त्र इति॑ पञ्चदश - रा॒त्रः । \newline
44. भ॒व॒ति॒ प॒ञ्च॒द॒शः प॑ञ्चद॒शो भ॑वति भवति पञ्चद॒शः । \newline
45. प॒ञ्च॒द॒शो वज्रो॒ वज्रः॑ पञ्चद॒शः प॑ञ्चद॒शो वज्रः॑ । \newline
46. प॒ञ्च॒द॒श इति॑ पञ्च - द॒शः । \newline
47. वज्रो॒ वज्रं॒ ॅवज्रं॒ ॅवज्रो॒ वज्रो॒ वज्र᳚म् । \newline
48. वज्र॑ मे॒वैव वज्रं॒ ॅवज्र॑ मे॒व । \newline
49. ए॒व भ्रातृ॑व्येभ्यो॒ भ्रातृ॑व्येभ्य ए॒वैव भ्रातृ॑व्येभ्यः । \newline
50. भ्रातृ॑व्येभ्यः॒ प्र प्र भ्रातृ॑व्येभ्यो॒ भ्रातृ॑व्येभ्यः॒ प्र । \newline
51. प्र ह॑रन्ति हरन्ति॒ प्र प्र ह॑रन्ति । \newline
52. ह॒र॒ न्त्य॒ति॒रा॒त्रा व॑तिरा॒त्रौ ह॑रन्ति हर न्त्यतिरा॒त्रौ । \newline
53. अ॒ति॒रा॒त्रा व॒भितो॒ ऽभितो॑ ऽतिरा॒त्रा व॑तिरा॒त्रा व॒भितः॑ । \newline
54. अ॒ति॒रा॒त्रावित्य॑ति - रा॒त्रौ । \newline
55. अ॒भितो॑ भवतो भवतो॒ ऽभितो॒ ऽभितो॑ भवतः । \newline
56. भ॒व॒त॒ इ॒न्द्रि॒य स्ये᳚न्द्रि॒यस्य॑ भवतो भवत इन्द्रि॒यस्य॑ । \newline
57. इ॒न्द्रि॒यस्य॒ परि॑गृहीत्यै॒ परि॑गृहीत्या इन्द्रि॒य स्ये᳚न्द्रि॒यस्य॒ परि॑गृहीत्यै । \newline
58. परि॑गृहीत्या॒ इति॒ परि॑ - गृ॒ही॒त्यै॒ । \newline

\textbf{Ghana Paata } \newline

1. गौ र॒सा व॒सौ गौर् गौ र॒सा वायु॒ रायु॑ र॒सौ गौर् गौ र॒सा वायुः॑ । \newline
2. अ॒सा वायु॒ रायु॑ र॒सा व॒सा वायु॑ रे॒ष्वे᳚ ष्वायु॑ र॒सा व॒सा वायु॑ रे॒षु । \newline
3. आयु॑ रे॒ष्वे᳚ ष्वायु॒ रायु॑ रे॒ष्वे॑ वैवै ष्वायु॒ रायु॑ रे॒ष्वे॑व । \newline
4. ए॒ष्वे॑ वैवै ष्वे᳚(1॒)ष्वे॑व लो॒केषु॑ लो॒केष्वे॒ वैष्वे᳚(1॒)ष्वे॑व लो॒केषु॑ । \newline
5. ए॒व लो॒केषु॑ लो॒के ष्वे॒वैव लो॒केषु॒ प्रति॒ प्रति॑ लो॒के ष्वे॒वैव लो॒केषु॒ प्रति॑ । \newline
6. लो॒केषु॒ प्रति॒ प्रति॑ लो॒केषु॑ लो॒केषु॒ प्रति॑ तिष्ठन्ति तिष्ठन्ति॒ प्रति॑ लो॒केषु॑ लो॒केषु॒ प्रति॑ तिष्ठन्ति । \newline
7. प्रति॑ तिष्ठन्ति तिष्ठन्ति॒ प्रति॒ प्रति॑ तिष्ठ॒ न्त्यस॑त्र॒ मस॑त्रम् तिष्ठन्ति॒ प्रति॒ प्रति॑ तिष्ठ॒ न्त्यस॑त्रम् । \newline
8. ति॒ष्ठ॒ न्त्यस॑त्र॒ मस॑त्रम् तिष्ठन्ति तिष्ठ॒ न्त्यस॑त्रं॒ ॅवै वा अस॑त्रम् तिष्ठन्ति तिष्ठ॒ न्त्यस॑त्रं॒ ॅवै । \newline
9. अस॑त्रं॒ ॅवै वा अस॑त्र॒ मस॑त्रं॒ ॅवा ए॒त दे॒तद् वा अस॑त्र॒ मस॑त्रं॒ ॅवा ए॒तत् । \newline
10. वा ए॒त दे॒तद् वै वा ए॒तद् यद् यदे॒तद् वै वा ए॒तद् यत् । \newline
11. ए॒तद् यद् यदे॒त दे॒तद् यद॑छन्दो॒म म॑छन्दो॒मं ॅयदे॒त दे॒तद् यद॑छन्दो॒मम् । \newline
12. यद॑छन्दो॒म म॑छन्दो॒मं ॅयद् यद॑छन्दो॒मं ॅयद् यद॑छन्दो॒मं ॅयद् यद॑छन्दो॒मं ॅयत् । \newline
13. अ॒छ॒न्दो॒मं ॅयद् यद॑छन्दो॒म म॑छन्दो॒मं ॅयच् छ॑न्दो॒मा श्छ॑न्दो॒मा यद॑छन्दो॒म म॑छन्दो॒मं ॅयच् छ॑न्दो॒माः । \newline
14. अ॒छ॒न्दो॒ममित्य॑छन्दः - मम् । \newline
15. यच् छ॑न्दो॒मा श्छ॑न्दो॒मा यद् यच् छ॑न्दो॒मा भव॑न्ति॒ भव॑न्ति छन्दो॒मा यद् यच् छ॑न्दो॒मा भव॑न्ति । \newline
16. छ॒न्दो॒मा भव॑न्ति॒ भव॑न्ति छन्दो॒मा श्छ॑न्दो॒मा भव॑न्ति॒ तेन॒ तेन॒ भव॑न्ति छन्दो॒मा श्छ॑न्दो॒मा भव॑न्ति॒ तेन॑ । \newline
17. छ॒न्दो॒मा इति॑ छन्दः - माः । \newline
18. भव॑न्ति॒ तेन॒ तेन॒ भव॑न्ति॒ भव॑न्ति॒ तेन॑ स॒त्रꣳ स॒त्रम् तेन॒ भव॑न्ति॒ भव॑न्ति॒ तेन॑ स॒त्रम् । \newline
19. तेन॑ स॒त्रꣳ स॒त्रम् तेन॒ तेन॑ स॒त्रम् दे॒वता॑ दे॒वताः᳚ स॒त्रम् तेन॒ तेन॑ स॒त्रम् दे॒वताः᳚ । \newline
20. स॒त्रम् दे॒वता॑ दे॒वताः᳚ स॒त्रꣳ स॒त्रम् दे॒वता॑ ए॒वैव दे॒वताः᳚ स॒त्रꣳ स॒त्रम् दे॒वता॑ ए॒व । \newline
21. दे॒वता॑ ए॒वैव दे॒वता॑ दे॒वता॑ ए॒व पृ॒ष्ठैः पृ॒ष्ठै रे॒व दे॒वता॑ दे॒वता॑ ए॒व पृ॒ष्ठैः । \newline
22. ए॒व पृ॒ष्ठैः पृ॒ष्ठै रे॒वैव पृ॒ष्ठै रवाव॑ पृ॒ष्ठै रे॒वैव पृ॒ष्ठै रव॑ । \newline
23. पृ॒ष्ठै रवाव॑ पृ॒ष्ठैः पृ॒ष्ठै रव॑ रुन्धते रुन्ध॒ते ऽव॑ पृ॒ष्ठैः पृ॒ष्ठै रव॑ रुन्धते । \newline
24. अव॑ रुन्धते रुन्ध॒ते ऽवाव॑ रुन्धते प॒शून् प॒शून् रु॑न्ध॒ते ऽवाव॑ रुन्धते प॒शून् । \newline
25. रु॒न्ध॒ते॒ प॒शून् प॒शून् रु॑न्धते रुन्धते प॒शूञ् छ॑न्दो॒मै श्छ॑न्दो॒मैः प॒शून् रु॑न्धते रुन्धते प॒शूञ् छ॑न्दो॒मैः । \newline
26. प॒शूञ् छ॑न्दो॒मै श्छ॑न्दो॒मैः प॒शून् प॒शूञ् छ॑न्दो॒मै रोज॒ ओज॑ श्छन्दो॒मैः प॒शून् प॒शूञ् छ॑न्दो॒मै रोजः॑ । \newline
27. छ॒न्दो॒मै रोज॒ ओज॑ श्छन्दो॒मै श्छ॑न्दो॒मै रोजो॒ वै वा ओज॑ श्छन्दो॒मै श्छ॑न्दो॒मै रोजो॒ वै । \newline
28. छ॒न्दो॒मैरिति॑ छन्दः - मैः । \newline
29. ओजो॒ वै वा ओज॒ ओजो॒ वै वी॒र्यं॑ ॅवी॒र्यं॑ ॅवा ओज॒ ओजो॒ वै वी॒र्य᳚म् । \newline
30. वै वी॒र्यं॑ ॅवी॒र्यं॑ ॅवै वै वी॒र्य॑म् पृ॒ष्ठानि॑ पृ॒ष्ठानि॑ वी॒र्यं॑ ॅवै वै वी॒र्य॑म् पृ॒ष्ठानि॑ । \newline
31. वी॒र्य॑म् पृ॒ष्ठानि॑ पृ॒ष्ठानि॑ वी॒र्यं॑ ॅवी॒र्य॑म् पृ॒ष्ठानि॑ प॒शवः॑ प॒शवः॑ पृ॒ष्ठानि॑ वी॒र्यं॑ ॅवी॒र्य॑म् पृ॒ष्ठानि॑ प॒शवः॑ । \newline
32. पृ॒ष्ठानि॑ प॒शवः॑ प॒शवः॑ पृ॒ष्ठानि॑ पृ॒ष्ठानि॑ प॒शव॑ श्छन्दो॒मा श्छ॑न्दो॒माः प॒शवः॑ पृ॒ष्ठानि॑ पृ॒ष्ठानि॑ प॒शव॑ श्छन्दो॒माः । \newline
33. प॒शव॑ श्छन्दो॒मा श्छ॑न्दो॒माः प॒शवः॑ प॒शव॑ श्छन्दो॒मा ओज॒ स्योज॑सि छन्दो॒माः प॒शवः॑ प॒शव॑ श्छन्दो॒मा ओज॑सि । \newline
34. छ॒न्दो॒मा ओज॒ स्योज॑सि छन्दो॒मा श्छ॑न्दो॒मा ओज॑ स्ये॒वै वौज॑सि छन्दो॒मा श्छ॑न्दो॒मा ओज॑स्ये॒व । \newline
35. छ॒न्दो॒मा इति॑ छन्दः - माः । \newline
36. ओज॑ स्ये॒वै वौज॒ स्योज॑ स्ये॒व वी॒र्ये॑ वी॒र्य॑ ए॒वौज॒ स्योज॑ स्ये॒व वी॒र्ये᳚ । \newline
37. ए॒व वी॒र्ये॑ वी॒र्य॑ ए॒वैव वी॒र्ये॑ प॒शुषु॑ प॒शुषु॑ वी॒र्य॑ ए॒वैव वी॒र्ये॑ प॒शुषु॑ । \newline
38. वी॒र्ये॑ प॒शुषु॑ प॒शुषु॑ वी॒र्ये॑ वी॒र्ये॑ प॒शुषु॒ प्रति॒ प्रति॑ प॒शुषु॑ वी॒र्ये॑ वी॒र्ये॑ प॒शुषु॒ प्रति॑ । \newline
39. प॒शुषु॒ प्रति॒ प्रति॑ प॒शुषु॑ प॒शुषु॒ प्रति॑ तिष्ठन्ति तिष्ठन्ति॒ प्रति॑ प॒शुषु॑ प॒शुषु॒ प्रति॑ तिष्ठन्ति । \newline
40. प्रति॑ तिष्ठन्ति तिष्ठन्ति॒ प्रति॒ प्रति॑ तिष्ठन्ति पञ्चदशरा॒त्रः प॑ञ्चदशरा॒त्र स्ति॑ष्ठन्ति॒ प्रति॒ प्रति॑ तिष्ठन्ति पञ्चदशरा॒त्रः । \newline
41. ति॒ष्ठ॒न्ति॒ प॒ञ्च॒द॒श॒रा॒त्रः प॑ञ्चदशरा॒त्र स्ति॑ष्ठन्ति तिष्ठन्ति पञ्चदशरा॒त्रो भ॑वति भवति पञ्चदशरा॒त्र स्ति॑ष्ठन्ति तिष्ठन्ति पञ्चदशरा॒त्रो भ॑वति । \newline
42. प॒ञ्च॒द॒श॒रा॒त्रो भ॑वति भवति पञ्चदशरा॒त्रः प॑ञ्चदशरा॒त्रो भ॑वति पञ्चद॒शः प॑ञ्चद॒शो भ॑वति पञ्चदशरा॒त्रः प॑ञ्चदशरा॒त्रो भ॑वति पञ्चद॒शः । \newline
43. प॒ञ्च॒द॒श॒रा॒त्र इति॑ पञ्चदश - रा॒त्रः । \newline
44. भ॒व॒ति॒ प॒ञ्च॒द॒शः प॑ञ्चद॒शो भ॑वति भवति पञ्चद॒शो वज्रो॒ वज्रः॑ पञ्चद॒शो भ॑वति भवति पञ्चद॒शो वज्रः॑ । \newline
45. प॒ञ्च॒द॒शो वज्रो॒ वज्रः॑ पञ्चद॒शः प॑ञ्चद॒शो वज्रो॒ वज्रं॒ ॅवज्रं॒ ॅवज्रः॑ पञ्चद॒शः प॑ञ्चद॒शो वज्रो॒ वज्र᳚म् । \newline
46. प॒ञ्च॒द॒श इति॑ पञ्च - द॒शः । \newline
47. वज्रो॒ वज्रं॒ ॅवज्रं॒ ॅवज्रो॒ वज्रो॒ वज्र॑ मे॒वैव वज्रं॒ ॅवज्रो॒ वज्रो॒ वज्र॑ मे॒व । \newline
48. वज्र॑ मे॒वैव वज्रं॒ ॅवज्र॑ मे॒व भ्रातृ॑व्येभ्यो॒ भ्रातृ॑व्येभ्य ए॒व वज्रं॒ ॅवज्र॑ मे॒व भ्रातृ॑व्येभ्यः । \newline
49. ए॒व भ्रातृ॑व्येभ्यो॒ भ्रातृ॑व्येभ्य ए॒वैव भ्रातृ॑व्येभ्यः॒ प्र प्र भ्रातृ॑व्येभ्य ए॒वैव भ्रातृ॑व्येभ्यः॒ प्र । \newline
50. भ्रातृ॑व्येभ्यः॒ प्र प्र भ्रातृ॑व्येभ्यो॒ भ्रातृ॑व्येभ्यः॒ प्र ह॑रन्ति हरन्ति॒ प्र भ्रातृ॑व्येभ्यो॒ भ्रातृ॑व्येभ्यः॒ प्र ह॑रन्ति । \newline
51. प्र ह॑रन्ति हरन्ति॒ प्र प्र ह॑र न्त्यतिरा॒त्रा व॑तिरा॒त्रौ ह॑रन्ति॒ प्र प्र ह॑र न्त्यतिरा॒त्रौ । \newline
52. ह॒र॒ न्त्य॒ति॒रा॒त्रा व॑तिरा॒त्रौ ह॑रन्ति हर न्त्यतिरा॒त्रा व॒भितो॒ ऽभितो॑ ऽतिरा॒त्रौ ह॑रन्ति हर न्त्यतिरा॒त्रा व॒भितः॑ । \newline
53. अ॒ति॒रा॒त्रा व॒भितो॒ ऽभितो॑ ऽतिरा॒त्रा व॑तिरा॒त्रा व॒भितो॑ भवतो भवतो॒ ऽभितो॑ ऽतिरा॒त्रा व॑तिरा॒त्रा व॒भितो॑ भवतः । \newline
54. अ॒ति॒रा॒त्रावित्य॑ति - रा॒त्रौ । \newline
55. अ॒भितो॑ भवतो भवतो॒ ऽभितो॒ ऽभितो॑ भवत इन्द्रि॒य स्ये᳚न्द्रि॒यस्य॑ भवतो॒ ऽभितो॒ ऽभितो॑ भवत इन्द्रि॒यस्य॑ । \newline
56. भ॒व॒त॒ इ॒न्द्रि॒य स्ये᳚न्द्रि॒यस्य॑ भवतो भवत इन्द्रि॒यस्य॒ परि॑गृहीत्यै॒ परि॑गृहीत्या इन्द्रि॒यस्य॑ भवतो भवत इन्द्रि॒यस्य॒ परि॑गृहीत्यै । \newline
57. इ॒न्द्रि॒यस्य॒ परि॑गृहीत्यै॒ परि॑गृहीत्या इन्द्रि॒य स्ये᳚न्द्रि॒यस्य॒ परि॑गृहीत्यै । \newline
58. परि॑गृहीत्या॒ इति॒ परि॑ - गृ॒ही॒त्यै॒ । \newline
\pagebreak
\markright{ TS 7.3.7.1  \hfill https://www.vedavms.in \hfill}

\section{ TS 7.3.7.1 }

\textbf{TS 7.3.7.1 } \newline
\textbf{Samhita Paata} \newline

इन्द्रो॒ वै शि॑थि॒ल इ॒वाऽप्र॑तिष्ठित आसी॒थ् सोऽसु॑रेभ्योऽबिभे॒थ् स प्र॒जाप॑ति॒मुपा॑ऽधाव॒त् तस्मा॑ ए॒तं प॑ञ्चदशरा॒त्रं ॅवज्रं॒ प्राय॑च्छ॒त् तेनासु॑रान् परा॒भाव्य॑ वि॒जित्य॒ श्रिय॑मगच्छदग्नि॒ष्टुता॑ पा॒प्मानं॒ निर॑दहत पञ्चदशरा॒त्रेणौजो॒ बल॑मिन्द्रि॒यं ॅवी॒र्य॑मा॒त्मन्न॑धत्त॒ य ए॒वं ॅवि॒द्वाꣳसः॑ पञ्चदशरा॒त्रमास॑ते॒ भ्रातृ॑व्याने॒व प॑रा॒भाव्य॑ वि॒जित्य॒ श्रियं॑ गच्छन्त्यग्नि॒ष्टुता॑ पा॒प्मानं॒ नि - [  ] \newline

\textbf{Pada Paata} \newline

इन्द्रः॑ । वै । शि॒थि॒लः । इ॒व॒ । अप्र॑तिष्ठित॒ इत्यप्र॑ति - स्थि॒तः॒ । आ॒सी॒त् । सः । असु॑रेभ्यः । अ॒बि॒भे॒त् । सः । प्र॒जाप॑ति॒मिति॑ प्र॒जा - प॒ति॒म् । उपेति॑ । अ॒धा॒व॒त् । तस्मै᳚ । ए॒तम् । प॒ञ्च॒द॒श॒रा॒त्रमिति॑ पञ्चदश - रा॒त्रम् । वज्र᳚म् । प्रेति॑ । अ॒य॒च्छ॒त् । तेन॑ । असु॑रान् । प॒रा॒भाव्येति॑ परा - भाव्य॑ । वि॒जित्येति॑ वि-जित्य॑ । श्रिय᳚म् । अ॒ग॒च्छ॒त् । अ॒ग्नि॒ष्टुतेत्य॑ग्नि - स्तुता᳚ । पा॒प्मान᳚म् । निरिति॑ । अ॒द॒ह॒त॒ । प॒ञ्च॒द॒श॒रा॒त्रेणेति॑ पञ्चदश - रा॒त्रेण॑ । ओजः॑ । बल᳚म् । इ॒न्द्रि॒यम् । वी॒र्य᳚म् । आ॒त्मन्न् । अ॒ध॒त्त॒ । ये । ए॒वम् । वि॒द्वाꣳसः॑ । प॒ञ्च॒द॒श॒रा॒त्रमिति॑ पञ्चदश - रा॒त्रम् । आस॑ते । भ्रातृ॑व्यान् । ए॒व । प॒रा॒भाव्येति॑ परा - भाव्य॑ । वि॒जित्येति॑ वि - जित्य॑ । श्रिय᳚म् । ग॒च्छ॒न्ति॒ । अ॒ग्नि॒ष्टुतेत्य॑ग्नि - स्तुता᳚ । पा॒प्मान᳚म् । निरिति॑ ।  \newline


\textbf{Krama Paata} \newline

इन्द्रो॒ वै । वै शि॑थि॒लः । शि॒थि॒ल इ॑व । इ॒वाप्र॑तिष्ठितः । अप्र॑तिष्ठित आसीत् । अप्र॑तिष्ठित॒ इत्यप्र॑ति - स्थि॒तः॒ । आ॒सी॒थ् सः । सोऽसु॑रेभ्यः । असु॑रेभ्योऽबिभेत् । अ॒बि॒भे॒थ् सः । स प्र॒जाप॑तिम् । प्र॒जाप॑ति॒मुप॑ । प्र॒जाप॑ति॒मिति॑ प्र॒जा - प॒ति॒म् । उपा॑धावत् । अ॒धा॒व॒त् तस्मै᳚ । तस्मा॑ ए॒तम् । ए॒तम् प॑ञ्चदशरा॒त्रम् । प॒ञ्च॒द॒श॒रा॒त्रम् ॅवज्र᳚म् । प॒ञ्च॒द॒श॒रा॒त्रमिति॑ पञ्चदश - रा॒त्रम् । वज्र॒म् प्र । प्राय॑च्छत् । अ॒य॒च्छ॒त् तेन॑ । तेनासु॑रान् । असु॑रान् परा॒भाव्य॑ । प॒रा॒भाव्य॑ वि॒जित्य॑ । प॒रा॒भाव्येति॑ परा - भाव्य॑ । वि॒जित्य॒ श्रिय᳚म् । वि॒जित्येति॑ वि - जित्य॑ । श्रिय॑मगच्छत् । अ॒ग॒च्छ॒द॒ग्नि॒ष्टुता᳚ । अ॒ग्नि॒ष्टुता॑ पा॒प्मान᳚म् । अ॒ग्नि॒ष्टुतेत्य॑ग्नि - स्तुता᳚ । पा॒प्मान॒म् निः । निर॑दहत । अ॒द॒ह॒त॒ प॒ञ्च॒द॒श॒रा॒त्रेण॑ । प॒ञ्च॒द॒श॒रा॒त्रेणौजः॑ । प॒ञ्च॒द॒श॒रा॒त्रेणेति॑ पञ्चदश - रा॒त्रेण॑ । ओजो॒ बल᳚म् । बल॑मिन्द्रि॒यम् । इ॒न्द्रि॒यम् ॅवी॒र्य᳚म् । वी॒र्य॑मा॒त्मन्न् । आ॒त्मन्न॑धत्त । अ॒ध॒त्त॒ ये । य ए॒वम् । ए॒वम् ॅवि॒द्वाꣳसः॑ । वि॒द्वाꣳसः॑ पञ्चदशरा॒त्रम् । प॒ञ्च॒द॒श॒रा॒त्रमास॑ते । प॒ञ्च॒द॒श॒रा॒त्रमिति॑ पञ्चदश - रा॒त्रम् । आस॑ते॒ भ्रातृ॑व्यान् । भ्रातृ॑व्याने॒व । ए॒व प॑रा॒भाव्य॑ । प॒रा॒भाव्य॑ वि॒जित्य॑ । प॒रा॒भाव्येति॑ परा - भाव्य॑ । वि॒जित्य॒ श्रिय᳚म् । वि॒जित्येति॑ वि - जित्य॑ । श्रिय॑म् गच्छन्ति । ग॒च्छ॒न्त्य॒ग्नि॒ष्टुता᳚ । अ॒ग्नि॒ष्टुता॑ पा॒प्मान᳚म् । अ॒ग्नि॒ष्टुतेत्य॑ग्नि - स्तुता᳚ । पा॒प्मान॒म् निः । निर् द॑हन्ते \newline

\textbf{Jatai Paata} \newline

1. इन्द्रो॒ वै वा इन्द्र॒ इन्द्रो॒ वै । \newline
2. वै शि॑थि॒लः शि॑थि॒लो वै वै शि॑थि॒लः । \newline
3. शि॒थि॒ल इ॑वेव शिथि॒लः शि॑थि॒ल इ॑व । \newline
4. इ॒वा प्र॑तिष्ठि॒तो ऽप्र॑तिष्ठित इवे॒वा प्र॑तिष्ठितः । \newline
5. अप्र॑तिष्ठित आसी दासी॒ दप्र॑तिष्ठि॒तो ऽप्र॑तिष्ठित आसीत् । \newline
6. अप्र॑तिष्ठित॒ इत्यप्र॑ति - स्थि॒तः॒ । \newline
7. आ॒सी॒थ् स स आ॑सी दासी॒थ् सः । \newline
8. सो ऽसु॑रे॒भ्यो ऽसु॑रेभ्यः॒ स सो ऽसु॑रेभ्यः । \newline
9. असु॑रेभ्यो ऽबिभे दबिभे॒ दसु॑रे॒भ्यो ऽसु॑रेभ्यो ऽबिभेत् । \newline
10. अ॒बि॒भे॒थ् स सो॑ ऽबिभे दबिभे॒थ् सः । \newline
11. स प्र॒जाप॑तिम् प्र॒जाप॑तिꣳ॒॒ स स प्र॒जाप॑तिम् । \newline
12. प्र॒जाप॑ति॒ मुपोप॑ प्र॒जाप॑तिम् प्र॒जाप॑ति॒ मुप॑ । \newline
13. प्र॒जाप॑ति॒मिति॑ प्र॒जा - प॒ति॒म् । \newline
14. उपा॑ धाव दधाव॒ दुपोपा॑ धावत् । \newline
15. अ॒धा॒व॒त् तस्मै॒ तस्मा॑ अधाव दधाव॒त् तस्मै᳚ । \newline
16. तस्मा॑ ए॒त मे॒तम् तस्मै॒ तस्मा॑ ए॒तम् । \newline
17. ए॒तम् प॑ञ्चदशरा॒त्रम् प॑ञ्चदशरा॒त्र मे॒त मे॒तम् प॑ञ्चदशरा॒त्रम् । \newline
18. प॒ञ्च॒द॒श॒रा॒त्रं ॅवज्रं॒ ॅवज्र॑म् पञ्चदशरा॒त्रम् प॑ञ्चदशरा॒त्रं ॅवज्र᳚म् । \newline
19. प॒ञ्च॒द॒श॒रा॒त्रमिति॑ पञ्चदश - रा॒त्रम् । \newline
20. वज्र॒म् प्र प्र वज्रं॒ ॅवज्र॒म् प्र । \newline
21. प्रा य॑च्छ दयच्छ॒त् प्र प्राय॑च्छत् । \newline
22. अ॒य॒च्छ॒त् तेन॒ तेना॑ यच्छ दयच्छ॒त् तेन॑ । \newline
23. तेना सु॑रा॒ नसु॑रा॒न् तेन॒ तेना सु॑रान् । \newline
24. असु॑रान् परा॒भाव्य॑ परा॒भाव्या सु॑रा॒ नसु॑रान् परा॒भाव्य॑ । \newline
25. प॒रा॒भाव्य॑ वि॒जित्य॑ वि॒जित्य॑ परा॒भाव्य॑ परा॒भाव्य॑ वि॒जित्य॑ । \newline
26. प॒रा॒भाव्येति॑ परा - भाव्य॑ । \newline
27. वि॒जित्य॒ श्रियꣳ॒॒ श्रियं॑ ॅवि॒जित्य॑ वि॒जित्य॒ श्रिय᳚म् । \newline
28. वि॒जित्येति॑ वि - जित्य॑ । \newline
29. श्रिय॑ मगच्छ दगच्छ॒च् छ्रियꣳ॒॒ श्रिय॑ मगच्छत् । \newline
30. अ॒ग॒च्छ॒ द॒ग्नि॒ष्टुता᳚ ऽग्नि॒ष्टुता॑ ऽगच्छ दगच्छ दग्नि॒ष्टुता᳚ । \newline
31. अ॒ग्नि॒ष्टुता॑ पा॒प्मान॑म् पा॒प्मान॑ मग्नि॒ष्टुता᳚ ऽग्नि॒ष्टुता॑ पा॒प्मान᳚म् । \newline
32. अ॒ग्नि॒ष्टुतेत्य॑ग्नि - स्तुता᳚ । \newline
33. पा॒प्मान॒न् निर् णिष् पा॒प्मान॑म् पा॒प्मान॒न् निः । \newline
34. निर॑दहता दहत॒ निर् णिर॑दहत । \newline
35. अ॒द॒ह॒त॒ प॒ञ्च॒द॒श॒रा॒त्रेण॑ पञ्चदशरा॒त्रेणा॑ दहता दहत पञ्चदशरा॒त्रेण॑ । \newline
36. प॒ञ्च॒द॒श॒रा॒त्रे णौज॒ ओजः॑ पञ्चदशरा॒त्रेण॑ पञ्चदशरा॒त्रे णौजः॑ । \newline
37. प॒ञ्च॒द॒श॒रा॒त्रेणेति॑ पञ्चदश - रा॒त्रेण॑ । \newline
38. ओजो॒ बल॒म् बल॒ मोज॒ ओजो॒ बल᳚म् । \newline
39. बल॑ मिन्द्रि॒य मि॑न्द्रि॒यम् बल॒म् बल॑ मिन्द्रि॒यम् । \newline
40. इ॒न्द्रि॒यं ॅवी॒र्यं॑ ॅवी॒र्य॑ मिन्द्रि॒य मि॑न्द्रि॒यं ॅवी॒र्य᳚म् । \newline
41. वी॒र्य॑ मा॒त्मन् ना॒त्मन्. वी॒र्यं॑ ॅवी॒र्य॑ मा॒त्मन्न् । \newline
42. आ॒त्मन् न॑धत्ता धत्ता॒त्मन् ना॒त्मन् न॑धत्त । \newline
43. अ॒ध॒त्त॒ ये ये॑ ऽधत्ता धत्त॒ ये । \newline
44. य ए॒व मे॒वं ॅये य ए॒वम् । \newline
45. ए॒वं ॅवि॒द्वाꣳसो॑ वि॒द्वाꣳस॑ ए॒व मे॒वं ॅवि॒द्वाꣳसः॑ । \newline
46. वि॒द्वाꣳसः॑ पञ्चदशरा॒त्रम् प॑ञ्चदशरा॒त्रं ॅवि॒द्वाꣳसो॑ वि॒द्वाꣳसः॑ पञ्चदशरा॒त्रम् । \newline
47. प॒ञ्च॒द॒श॒रा॒त्र मास॑त॒ आस॑ते पञ्चदशरा॒त्रम् प॑ञ्चदशरा॒त्र मास॑ते । \newline
48. प॒ञ्च॒द॒श॒रा॒त्रमिति॑ पञ्चदश - रा॒त्रम् । \newline
49. आस॑ते॒ भ्रातृ॑व्या॒न् भ्रातृ॑व्या॒ नास॑त॒ आस॑ते॒ भ्रातृ॑व्यान् । \newline
50. भ्रातृ॑व्या ने॒वैव भ्रातृ॑व्या॒न् भ्रातृ॑व्याने॒व । \newline
51. ए॒व प॑रा॒भाव्य॑ परा॒भा व्यै॒वैव प॑रा॒भाव्य॑ । \newline
52. प॒रा॒भाव्य॑ वि॒जित्य॑ वि॒जित्य॑ परा॒भाव्य॑ परा॒भाव्य॑ वि॒जित्य॑ । \newline
53. प॒रा॒भाव्येति॑ परा - भाव्य॑ । \newline
54. वि॒जित्य॒ श्रियꣳ॒॒ श्रियं॑ ॅवि॒जित्य॑ वि॒जित्य॒ श्रिय᳚म् । \newline
55. वि॒जित्येति॑ वि - जित्य॑ । \newline
56. श्रिय॑म् गच्छन्ति गच्छन्ति॒ श्रियꣳ॒॒ श्रिय॑म् गच्छन्ति । \newline
57. ग॒च्छ॒न् त्य॒ग्नि॒ष्टुता᳚ ऽग्नि॒ष्टुता॑ गच्छन्ति गच्छन् त्यग्नि॒ष्टुता᳚ । \newline
58. अ॒ग्नि॒ष्टुता॑ पा॒प्मान॑म् पा॒प्मान॑ मग्नि॒ष्टुता᳚ ऽग्नि॒ष्टुता॑ पा॒प्मान᳚म् । \newline
59. अ॒ग्नि॒ष्टुतेत्य॑ग्नि - स्तुता᳚ । \newline
60. पा॒प्मान॒न् निर् णिष् पा॒प्मान॑म् पा॒प्मान॒न् निः । \newline
61. निर् द॑हन्ते दहन्ते॒ निर् णिर् द॑हन्ते । \newline

\textbf{Ghana Paata } \newline

1. इन्द्रो॒ वै वा इन्द्र॒ इन्द्रो॒ वै शि॑थि॒लः शि॑थि॒लो वा इन्द्र॒ इन्द्रो॒ वै शि॑थि॒लः । \newline
2. वै शि॑थि॒लः शि॑थि॒लो वै वै शि॑थि॒ल इ॑वेव शिथि॒लो वै वै शि॑थि॒ल इ॑व । \newline
3. शि॒थि॒ल इ॑वेव शिथि॒लः शि॑थि॒ल इ॒वा प्र॑तिष्ठि॒तो ऽप्र॑तिष्ठित इव शिथि॒लः शि॑थि॒ल इ॒वा प्र॑तिष्ठितः । \newline
4. इ॒वा प्र॑तिष्ठि॒तो ऽप्र॑तिष्ठित इवे॒वा प्र॑तिष्ठित आसी दासी॒ दप्र॑तिष्ठित इवे॒वा प्र॑तिष्ठित आसीत् । \newline
5. अप्र॑तिष्ठित आसी दासी॒ दप्र॑तिष्ठि॒तो ऽप्र॑तिष्ठित आसी॒थ् स स आ॑सी॒ दप्र॑तिष्ठि॒तो ऽप्र॑तिष्ठित आसी॒थ् सः । \newline
6. अप्र॑तिष्ठित॒ इत्यप्र॑ति - स्थि॒तः॒ । \newline
7. आ॒सी॒थ् स स आ॑सी दासी॒थ् सो ऽसु॑रे॒भ्यो ऽसु॑रेभ्यः॒ स आ॑सी दासी॒थ् सो ऽसु॑रेभ्यः । \newline
8. सो ऽसु॑रे॒भ्यो ऽसु॑रेभ्यः॒ स सो ऽसु॑रेभ्यो ऽबिभे दबिभे॒ दसु॑रेभ्यः॒ स सो ऽसु॑रेभ्यो ऽबिभेत् । \newline
9. असु॑रेभ्यो ऽबिभे दबिभे॒ दसु॑रे॒भ्यो ऽसु॑रेभ्यो ऽबिभे॒थ् स सो॑ ऽबिभे॒ दसु॑रे॒भ्यो ऽसु॑रेभ्यो ऽबिभे॒थ् सः । \newline
10. अ॒बि॒भे॒थ् स सो॑ ऽबिभे दबिभे॒थ् स प्र॒जाप॑तिम् प्र॒जाप॑तिꣳ॒॒ सो॑ ऽबिभे दबिभे॒थ् स प्र॒जाप॑तिम् । \newline
11. स प्र॒जाप॑तिम् प्र॒जाप॑तिꣳ॒॒ स स प्र॒जाप॑ति॒ मुपोप॑ प्र॒जाप॑तिꣳ॒॒ स स प्र॒जाप॑ति॒ मुप॑ । \newline
12. प्र॒जाप॑ति॒ मुपोप॑ प्र॒जाप॑तिम् प्र॒जाप॑ति॒ मुपा॑धाव दधाव॒ दुप॑ प्र॒जाप॑तिम् प्र॒जाप॑ति॒ मुपा॑ धावत् । \newline
13. प्र॒जाप॑ति॒मिति॑ प्र॒जा - प॒ति॒म् । \newline
14. उपा॑ धाव दधाव॒ दुपोपा॑ धाव॒त् तस्मै॒ तस्मा॑ अधाव॒ दुपोपा॑ धाव॒त् तस्मै᳚ । \newline
15. अ॒धा॒व॒त् तस्मै॒ तस्मा॑ अधाव दधाव॒त् तस्मा॑ ए॒त मे॒तम् तस्मा॑ अधाव दधाव॒त् तस्मा॑ ए॒तम् । \newline
16. तस्मा॑ ए॒त मे॒तम् तस्मै॒ तस्मा॑ ए॒तम् प॑ञ्चदशरा॒त्रम् प॑ञ्चदशरा॒त्र मे॒तम् तस्मै॒ तस्मा॑ ए॒तम् प॑ञ्चदशरा॒त्रम् । \newline
17. ए॒तम् प॑ञ्चदशरा॒त्रम् प॑ञ्चदशरा॒त्र मे॒त मे॒तम् प॑ञ्चदशरा॒त्रं ॅवज्रं॒ ॅवज्र॑म् पञ्चदशरा॒त्र मे॒त मे॒तम् प॑ञ्चदशरा॒त्रं ॅवज्र᳚म् । \newline
18. प॒ञ्च॒द॒श॒रा॒त्रं ॅवज्रं॒ ॅवज्र॑म् पञ्चदशरा॒त्रम् प॑ञ्चदशरा॒त्रं ॅवज्र॒म् प्र प्र वज्र॑म् पञ्चदशरा॒त्रम् प॑ञ्चदशरा॒त्रं ॅवज्र॒म् प्र । \newline
19. प॒ञ्च॒द॒श॒रा॒त्रमिति॑ पञ्चदश - रा॒त्रम् । \newline
20. वज्र॒म् प्र प्र वज्रं॒ ॅवज्र॒म् प्राय॑च्छ दयच्छ॒त् प्र वज्रं॒ ॅवज्र॒म् प्राय॑च्छत् । \newline
21. प्राय॑च्छ दयच्छ॒त् प्र प्रा य॑च्छ॒त् तेन॒ तेना॑ यच्छ॒त् प्र प्राय॑च्छ॒त् तेन॑ । \newline
22. अ॒य॒च्छ॒त् तेन॒ तेना॑ यच्छ दयच्छ॒त् तेना सु॑रा॒ नसु॑रा॒न् तेना॑ यच्छ दयच्छ॒त् तेना सु॑रान् । \newline
23. तेना सु॑रा॒ नसु॑रा॒न् तेन॒ तेना सु॑रान् परा॒भाव्य॑ परा॒भाव्या सु॑रा॒न् तेन॒ तेना सु॑रान् परा॒भाव्य॑ । \newline
24. असु॑रान् परा॒भाव्य॑ परा॒भाव्या सु॑रा॒ नसु॑रान् परा॒भाव्य॑ वि॒जित्य॑ वि॒जित्य॑ परा॒भाव्या सु॑रा॒ नसु॑रान् परा॒भाव्य॑ वि॒जित्य॑ । \newline
25. प॒रा॒भाव्य॑ वि॒जित्य॑ वि॒जित्य॑ परा॒भाव्य॑ परा॒भाव्य॑ वि॒जित्य॒ श्रियꣳ॒॒ श्रियं॑ ॅवि॒जित्य॑ परा॒भाव्य॑ परा॒भाव्य॑ वि॒जित्य॒ श्रिय᳚म् । \newline
26. प॒रा॒भाव्येति॑ परा - भाव्य॑ । \newline
27. वि॒जित्य॒ श्रियꣳ॒॒ श्रियं॑ ॅवि॒जित्य॑ वि॒जित्य॒ श्रिय॑ मगच्छ दगच्छ॒च् छ्रियं॑ ॅवि॒जित्य॑ वि॒जित्य॒ श्रिय॑ मगच्छत् । \newline
28. वि॒जित्येति॑ वि - जित्य॑ । \newline
29. श्रिय॑ मगच्छ दगच्छ॒च् छ्रियꣳ॒॒ श्रिय॑ मगच्छ दग्नि॒ष्टुता᳚ ऽग्नि॒ष्टुता॑ ऽगच्छ॒च् छ्रियꣳ॒॒ श्रिय॑ मगच्छ दग्नि॒ष्टुता᳚ । \newline
30. अ॒ग॒च्छ॒ द॒ग्नि॒ष्टुता᳚ ऽग्नि॒ष्टुता॑ ऽगच्छ दगच्छ दग्नि॒ष्टुता॑ पा॒प्मान॑म् पा॒प्मान॑ मग्नि॒ष्टुता॑ ऽगच्छ दगच्छ दग्नि॒ष्टुता॑ पा॒प्मान᳚म् । \newline
31. अ॒ग्नि॒ष्टुता॑ पा॒प्मान॑म् पा॒प्मान॑ मग्नि॒ष्टुता᳚ ऽग्नि॒ष्टुता॑ पा॒प्मान॒न् निर् णिष् पा॒प्मान॑ मग्नि॒ष्टुता᳚ ऽग्नि॒ष्टुता॑ पा॒प्मान॒न् निः । \newline
32. अ॒ग्नि॒ष्टुतेत्य॑ग्नि - स्तुता᳚ । \newline
33. पा॒प्मान॒न् निर् णिष् पा॒प्मान॑म् पा॒प्मान॒न् निर॑दहता दहत॒ निष् पा॒प्मान॑म् पा॒प्मान॒न् निर॑दहत । \newline
34. निर॑दहता दहत॒ निर् णिर॑दहत पञ्चदशरा॒त्रेण॑ पञ्चदशरा॒त्रेणा॑ दहत॒ निर् णिर॑दहत पञ्चदशरा॒त्रेण॑ । \newline
35. अ॒द॒ह॒त॒ प॒ञ्च॒द॒श॒रा॒त्रेण॑ पञ्चदशरा॒त्रेणा॑ दहता दहत पञ्चदशरा॒त्रे णौज॒ ओजः॑ पञ्चदशरा॒त्रेणा॑ दहता दहत पञ्चदशरा॒त्रे णौजः॑ । \newline
36. प॒ञ्च॒द॒श॒रा॒त्रे णौज॒ ओजः॑ पञ्चदशरा॒त्रेण॑ पञ्चदशरा॒त्रे णौजो॒ बल॒म् बल॒ मोजः॑ पञ्चदशरा॒त्रेण॑ पञ्चदशरा॒त्रे णौजो॒ बल᳚म् । \newline
37. प॒ञ्च॒द॒श॒रा॒त्रेणेति॑ पञ्चदश - रा॒त्रेण॑ । \newline
38. ओजो॒ बल॒म् बल॒ मोज॒ ओजो॒ बल॑ मिन्द्रि॒य मि॑न्द्रि॒यम् बल॒ मोज॒ ओजो॒ बल॑ मिन्द्रि॒यम् । \newline
39. बल॑ मिन्द्रि॒य मि॑न्द्रि॒यम् बल॒म् बल॑ मिन्द्रि॒यं ॅवी॒र्यं॑ ॅवी॒र्य॑ मिन्द्रि॒यम् बल॒म् बल॑ मिन्द्रि॒यं ॅवी॒र्य᳚म् । \newline
40. इ॒न्द्रि॒यं ॅवी॒र्यं॑ ॅवी॒र्य॑ मिन्द्रि॒य मि॑न्द्रि॒यं ॅवी॒र्य॑ मा॒त्मन् ना॒त्मन्. वी॒र्य॑ मिन्द्रि॒य मि॑न्द्रि॒यं ॅवी॒र्य॑ मा॒त्मन्न् । \newline
41. वी॒र्य॑ मा॒त्मन् ना॒त्मन्. वी॒र्यं॑ ॅवी॒र्य॑ मा॒त्मन् न॑धत्ता धत्ता॒त्मन्. वी॒र्यं॑ ॅवी॒र्य॑ मा॒त्मन् न॑धत्त । \newline
42. आ॒त्मन् न॑धत्ता धत्ता॒त्मन् ना॒त्मन् न॑धत्त॒ ये ये॑ ऽधत्ता॒त्मन् ना॒त्मन् न॑धत्त॒ ये । \newline
43. अ॒ध॒त्त॒ ये ये॑ ऽधत्ता धत्त॒ य ए॒व मे॒वं ॅये॑ ऽधत्ता धत्त॒ य ए॒वम् । \newline
44. य ए॒व मे॒वं ॅये य ए॒वं ॅवि॒द्वाꣳसो॑ वि॒द्वाꣳस॑ ए॒वं ॅये य ए॒वं ॅवि॒द्वाꣳसः॑ । \newline
45. ए॒वं ॅवि॒द्वाꣳसो॑ वि॒द्वाꣳस॑ ए॒व मे॒वं ॅवि॒द्वाꣳसः॑ पञ्चदशरा॒त्रम् प॑ञ्चदशरा॒त्रं ॅवि॒द्वाꣳस॑ ए॒व मे॒वं ॅवि॒द्वाꣳसः॑ पञ्चदशरा॒त्रम् । \newline
46. वि॒द्वाꣳसः॑ पञ्चदशरा॒त्रम् प॑ञ्चदशरा॒त्रं ॅवि॒द्वाꣳसो॑ वि॒द्वाꣳसः॑ पञ्चदशरा॒त्र मास॑त॒ आस॑ते पञ्चदशरा॒त्रं ॅवि॒द्वाꣳसो॑ वि॒द्वाꣳसः॑ पञ्चदशरा॒त्र मास॑ते । \newline
47. प॒ञ्च॒द॒श॒रा॒त्र मास॑त॒ आस॑ते पञ्चदशरा॒त्रम् प॑ञ्चदशरा॒त्र मास॑ते॒ भ्रातृ॑व्या॒न् भ्रातृ॑व्या॒ नास॑ते पञ्चदशरा॒त्रम् प॑ञ्चदशरा॒त्र मास॑ते॒ भ्रातृ॑व्यान् । \newline
48. प॒ञ्च॒द॒श॒रा॒त्रमिति॑ पञ्चदश - रा॒त्रम् । \newline
49. आस॑ते॒ भ्रातृ॑व्या॒न् भ्रातृ॑व्या॒ नास॑त॒ आस॑ते॒ भ्रातृ॑व्या ने॒वैव भ्रातृ॑व्या॒ नास॑त॒ आस॑ते॒ भ्रातृ॑व्या ने॒व । \newline
50. भ्रातृ॑व्या ने॒वैव भ्रातृ॑व्या॒न् भ्रातृ॑व्या ने॒व प॑रा॒भाव्य॑ परा॒भाव्यै॒व भ्रातृ॑व्या॒न् भ्रातृ॑व्या ने॒व प॑रा॒भाव्य॑ । \newline
51. ए॒व प॑रा॒भाव्य॑ परा॒भाव्यै॒ वैव प॑रा॒भाव्य॑ वि॒जित्य॑ वि॒जित्य॑ परा॒भाव्यै॒ वैव प॑रा॒भाव्य॑ वि॒जित्य॑ । \newline
52. प॒रा॒भाव्य॑ वि॒जित्य॑ वि॒जित्य॑ परा॒भाव्य॑ परा॒भाव्य॑ वि॒जित्य॒ श्रियꣳ॒॒ श्रियं॑ ॅवि॒जित्य॑ परा॒भाव्य॑ परा॒भाव्य॑ वि॒जित्य॒ श्रिय᳚म् । \newline
53. प॒रा॒भाव्येति॑ परा - भाव्य॑ । \newline
54. वि॒जित्य॒ श्रियꣳ॒॒ श्रियं॑ ॅवि॒जित्य॑ वि॒जित्य॒ श्रिय॑म् गच्छन्ति गच्छन्ति॒ श्रियं॑ ॅवि॒जित्य॑ वि॒जित्य॒ श्रिय॑म् गच्छन्ति । \newline
55. वि॒जित्येति॑ वि - जित्य॑ । \newline
56. श्रिय॑म् गच्छन्ति गच्छन्ति॒ श्रियꣳ॒॒ श्रिय॑म् गच्छ न्त्यग्नि॒ष्टुता᳚ ऽग्नि॒ष्टुता॑ गच्छन्ति॒ श्रियꣳ॒॒ श्रिय॑म् गच्छ न्त्यग्नि॒ष्टुता᳚ । \newline
57. ग॒च्छ॒ न्त्य॒ग्नि॒ष्टुता᳚ ऽग्नि॒ष्टुता॑ गच्छन्ति गच्छ न्त्यग्नि॒ष्टुता॑ पा॒प्मान॑म् पा॒प्मान॑ मग्नि॒ष्टुता॑ गच्छन्ति गच्छ न्त्यग्नि॒ष्टुता॑ पा॒प्मान᳚म् । \newline
58. अ॒ग्नि॒ष्टुता॑ पा॒प्मान॑म् पा॒प्मान॑ मग्नि॒ष्टुता᳚ ऽग्नि॒ष्टुता॑ पा॒प्मान॒न् निर् णिष् पा॒प्मान॑ मग्नि॒ष्टुता᳚ ऽग्नि॒ष्टुता॑ पा॒प्मान॒न् निः । \newline
59. अ॒ग्नि॒ष्टुतेत्य॑ग्नि - स्तुता᳚ । \newline
60. पा॒प्मान॒न् निर् णिष् पा॒प्मान॑म् पा॒प्मान॒न् निर् द॑हन्ते दहन्ते॒ निष् पा॒प्मान॑म् पा॒प्मान॒न् निर् द॑हन्ते । \newline
61. निर् द॑हन्ते दहन्ते॒ निर् णिर् द॑हन्ते पञ्चदशरा॒त्रेण॑ पञ्चदशरा॒त्रेण॑ दहन्ते॒ निर् णिर् द॑हन्ते पञ्चदशरा॒त्रेण॑ । \newline
\pagebreak
\markright{ TS 7.3.7.2  \hfill https://www.vedavms.in \hfill}

\section{ TS 7.3.7.2 }

\textbf{TS 7.3.7.2 } \newline
\textbf{Samhita Paata} \newline

र्द॑हन्ते पञ्चदशरा॒त्रेणौजो॒ बल॑मिन्द्रि॒यं ॅवी॒र्य॑मा॒त्मन् द॑धत ए॒ता ए॒व प॑श॒व्याः᳚ पञ्च॑दश॒ वा अ॑र्द्धमा॒सस्य॒ रात्र॑योऽर्द्धमास॒शः सं॑ॅवथ्स॒र आ᳚प्यते संॅवथ्स॒रं प॒शवोऽनु॒ प्र जा॑यन्ते॒ तस्मा᳚त् पश॒व्या॑ ए॒ता ए॒व सु॑व॒र्ग्याः᳚ पञ्च॑दश॒ वा अ॑र्द्धमा॒सस्य॒ रात्र॑योऽर्द्धमास॒शः सं॑ॅवथ्स॒र आ᳚प्यते संॅवथ्स॒रः सु॑व॒र्गो लो॒कस्तस्मा᳚थ् सुव॒र्ग्या᳚ ज्योति॒र्गौरायु॒रिति॑ त्र्य॒हो भ॑वती॒यं ॅवाव ज्योति॑र॒न्तरि॑क्षं॒ - [  ] \newline

\textbf{Pada Paata} \newline

द॒ह॒न्ते॒ । प॒ञ्च॒द॒श॒रा॒त्रेणेति॑ पञ्चदश - रा॒त्रेण॑ । ओजः॑ । बल᳚म् । इ॒न्द्रि॒यम् । वी॒र्य᳚म् । आ॒त्मन्न् । द॒ध॒ते॒ । ए॒ताः । ए॒व । प॒श॒व्याः᳚ । पञ्च॑द॒शेति॒ पञ्च॑ - द॒श॒ । वै । अ॒द्‌र्ध॒मा॒सस्येत्य॒॑द्‌र्ध - मा॒सस्य॑ । रात्र॑यः । अ॒द्‌र्ध॒मा॒स॒श इत्य॑द्‌र्धमास - शः । सं॒ॅव॒थ्स॒र इति॑ सं-व॒थ्स॒रः । आ॒प्य॒ते॒ । सं॒ॅव॒थ्स॒रमिति॑ सं-व॒थ्स॒रम् । प॒शवः॑ । अनु॑ । प्रेति॑ । जा॒य॒न्ते॒ । तस्मा᳚त् । प॒श॒व्याः᳚ । ए॒ताः । ए॒व । सु॒व॒र्ग्या॑ इति॑ सुवः - ग्याः᳚ । पञ्च॑द॒शेति॒ पञ्च॑ - द॒श॒ । वै । अ॒द्‌र्ध॒मा॒सस्येत्य॑द्‌र्ध -मा॒सस्य॑ । रात्र॑यः । अ॒द्‌र्ध॒मा॒स॒श इत्य॑द्‌र्धमास - शः । सं॒ॅव॒थ्स॒र इति॑ सं - व॒थ्स॒रः । आ॒प्य॒ते॒ । सं॒ॅव॒थ्स॒र इति॑ सं - व॒थ्स॒रः । सु॒व॒र्ग इति॑ सुवः - गः । लो॒कः । तस्मा᳚त् । सु॒व॒र्ग्या॑ इति॑ सुवः - ग्याः᳚ । ज्योतिः॑ । गौः । आयुः॑ । इति॑ । त्र्य॒ह इति॑ त्रि - अ॒हः । भ॒व॒ति॒ । इ॒यम् । वाव । ज्योतिः॑ । अ॒न्तरि॑क्षम् ।  \newline


\textbf{Krama Paata} \newline

द॒ह॒न्ते॒ प॒ञ्च॒द॒श॒रा॒त्रेण॑ । प॒ञ्च॒द॒श॒रा॒त्रेणौजः॑ । प॒ञ्च॒द॒श॒रा॒त्रेणेति॑ पञ्चदश - रा॒त्रेण॑ । ओजो॒ बल᳚म् । बल॑मिन्द्रि॒यम् । इ॒न्द्रि॒यम् ॅवी॒र्य᳚म् । वी॒र्य॑मा॒त्मन्न् । आ॒त्मन् द॑धते । द॒ध॒त॒ ए॒ताः । ए॒ता ए॒व । ए॒व प॑श॒व्याः᳚ । प॒श॒व्याः᳚ पञ्च॑दश । पञ्च॑दश॒ वै । पञ्च॑द॒शेति॒ पञ्च॑ - द॒श॒ । वा अ॑र्द्धमा॒सस्य॑ । अ॒र्द्ध॒मा॒सस्य॒ रात्र॑यः । अ॒र्द्ध॒मा॒सस्येत्य॑र्ध - मा॒सस्य॑ । रात्र॑योऽर्द्धमास॒शः । अ॒र्द्ध॒मा॒स॒शः स॑म्ॅवथ्स॒रः । अ॒र्द्ध॒मा॒स॒श इत्य॑र्द्धमास - शः । स॒म्ॅव॒थ्स॒र आ᳚प्यते । स॒म्ॅव॒थ्स॒र इति॑ सम् - व॒थ्स॒रः । आ॒प्य॒ते॒ स॒म्ॅव॒थ्स॒रम् । स॒म्ॅव॒थ्स॒रम् प॒शवः॑ । स॒म्ॅव॒थ्स॒रमिति॑ सम् - व॒थ्स॒रम् । प॒शवोऽनु॑ । अनु॒ प्र । प्र जा॑यन्ते । जा॒य॒न्ते॒ तस्मा᳚त् । तस्मा᳚त् पश॒व्या᳚ । प॒श॒व्या॑ ए॒ताः । ए॒ता ए॒व । ए॒व सु॑व॒र्ग्याः᳚ । सु॒व॒र्ग्याः᳚ पञ्च॑दश । सु॒व॒र्ग्या॑ इति॑ सुवः - ग्याः᳚ । पञ्च॑दश॒ वै । पञ्च॑द॒शेति॒ पञ्च॑ - द॒श॒ । वा अ॑र्द्धमा॒सस्य॑ । अ॒र्द्ध॒मा॒सस्य॒ रात्र॑यः । अ॒र्द्ध॒मा॒सस्येत्य॑र्द्ध - मा॒सस्य॑ । रात्र॑योऽर्द्धमास॒शः । अ॒र्द्ध॒मा॒स॒शः स॑म्ॅवथ्स॒रः । अ॒र्द्ध॒मा॒स॒श इत्य॑र्द्धमास - शः । स॒म्ॅव॒थ्स॒र आ᳚प्यते । स॒म्ॅव॒थ्स॒र इति॑ सम् - व॒थ्स॒रः । आ॒प्य॒ते॒ स॒म्ॅव॒थ्स॒रः । स॒म्ॅव॒थ्स॒रः सु॑व॒र्गः । स॒म्ॅव॒थ्स॒र इति॑ सम् - व॒थ्स॒रः । सु॒व॒र्गो लो॒कः । सु॒व॒र्ग इति॑ सुवः - गः । लो॒कस्तस्मा᳚त् । तस्मा᳚थ् सुव॒र्ग्याः᳚ । सु॒व॒र्ग्या᳚ ज्योतिः॑ । सु॒व॒र्ग्या॑ इति॑ सुवः - ग्याः᳚ । ज्योति॒र् गौः । गौरायुः॑ । आयु॒रिति॑ । इति॑ त्र्य॒हः । त्र्य॒हो भ॑वति । त्र्य॒ह इति॑ त्रि - अ॒हः । भ॒व॒ती॒यम् । इ॒यम् ॅवाव । वाव ज्योतिः॑ । ज्योति॑र॒न्तरि॑क्षम् । अ॒न्तरि॑क्ष॒म् गौः \newline

\textbf{Jatai Paata} \newline

1. द॒ह॒न्ते॒ प॒ञ्च॒द॒श॒रा॒त्रेण॑ पञ्चदशरा॒त्रेण॑ दहन्ते दहन्ते पञ्चदशरा॒त्रेण॑ । \newline
2. प॒ञ्च॒द॒श॒रा॒त्रे णौज॒ ओजः॑ पञ्चदशरा॒त्रेण॑ पञ्चदशरा॒त्रे णौजः॑ । \newline
3. प॒ञ्च॒द॒श॒रा॒त्रेणेति॑ पञ्चदश - रा॒त्रेण॑ । \newline
4. ओजो॒ बल॒म् बल॒ मोज॒ ओजो॒ बल᳚म् । \newline
5. बल॑ मिन्द्रि॒य मि॑न्द्रि॒यम् बल॒म् बल॑ मिन्द्रि॒यम् । \newline
6. इ॒न्द्रि॒यं ॅवी॒र्यं॑ ॅवी॒र्य॑ मिन्द्रि॒य मि॑न्द्रि॒यं ॅवी॒र्य᳚म् । \newline
7. वी॒र्य॑ मा॒त्मन् ना॒त्मन्. वी॒र्यं॑ ॅवी॒र्य॑ मा॒त्मन्न् । \newline
8. आ॒त्मन् द॑धते दधत आ॒त्मन् ना॒त्मन् द॑धते । \newline
9. द॒ध॒त॒ ए॒ता ए॒ता द॑धते दधत ए॒ताः । \newline
10. ए॒ता ए॒वैवैता ए॒ता ए॒व । \newline
11. ए॒व प॑श॒व्याः᳚ पश॒व्या॑ ए॒वैव प॑श॒व्याः᳚ । \newline
12. प॒श॒व्याः᳚ पञ्च॑दश॒ पञ्च॑दश पश॒व्याः᳚ पश॒व्याः᳚ पञ्च॑दश । \newline
13. पञ्च॑दश॒ वै वै पञ्च॑दश॒ पञ्च॑दश॒ वै । \newline
14. पञ्च॑द॒शेति॒ पञ्च॑ - द॒श॒ । \newline
15. वा अ॑र्द्धमा॒सस्या᳚ र्द्धमा॒सस्य॒ वै वा अ॑र्द्धमा॒सस्य॑ । \newline
16. अ॒र्द्ध॒मा॒सस्य॒ रात्र॑यो॒ रात्र॑यो ऽर्द्धमा॒सस्या᳚ र्द्धमा॒सस्य॒ रात्र॑यः । \newline
17. अ॒र्द्ध॒मा॒सस्येत्य॑र्द्ध - मा॒सस्य॑ । \newline
18. रात्र॑यो ऽर्द्धमास॒शो᳚ ऽर्द्धमास॒शो रात्र॑यो॒ रात्र॑यो ऽर्द्धमास॒शः । \newline
19. अ॒र्द्ध॒मा॒स॒शः सं॑ॅवथ्स॒रः सं॑ॅवथ्स॒रो᳚ ऽर्द्धमास॒शो᳚ ऽर्द्धमास॒शः सं॑ॅवथ्स॒रः । \newline
20. अ॒र्द्ध॒मा॒स॒श इत्य॑र्द्धमास - शः । \newline
21. सं॒ॅव॒थ्स॒र आ᳚प्यत आप्यते संॅवथ्स॒रः सं॑ॅवथ्स॒र आ᳚प्यते । \newline
22. सं॒ॅव॒थ्स॒र इति॑ सं - व॒थ्स॒रः । \newline
23. आ॒प्य॒ते॒ सं॒ॅव॒थ्स॒रꣳ सं॑ॅवथ्स॒र मा᳚प्यत आप्यते संॅवथ्स॒रम् । \newline
24. सं॒ॅव॒थ्स॒रम् प॒शवः॑ प॒शवः॑ संॅवथ्स॒रꣳ सं॑ॅवथ्स॒रम् प॒शवः॑ । \newline
25. सं॒ॅव॒थ्स॒रमिति॑ सं - व॒थ्स॒रम् । \newline
26. प॒शवो ऽन्वनु॑ प॒शवः॑ प॒शवो ऽनु॑ । \newline
27. अनु॒ प्र प्राण्वनु॒ प्र । \newline
28. प्र जा॑यन्ते जायन्ते॒ प्र प्र जा॑यन्ते । \newline
29. जा॒य॒न्ते॒ तस्मा॒त् तस्मा᳚ज् जायन्ते जायन्ते॒ तस्मा᳚त् । \newline
30. तस्मा᳚त् पश॒व्याः᳚ पश॒व्या᳚ स्तस्मा॒त् तस्मा᳚त् पश॒व्याः᳚ । \newline
31. प॒श॒व्या॑ ए॒ता ए॒ताः प॑श॒व्याः᳚ पश॒व्या॑ ए॒ताः । \newline
32. ए॒ता ए॒वैवैता ए॒ता ए॒व । \newline
33. ए॒व सु॑व॒र्ग्याः᳚ सुव॒र्ग्या॑ ए॒वैव सु॑व॒र्ग्याः᳚ । \newline
34. सु॒व॒र्ग्याः᳚ पञ्च॑दश॒ पञ्च॑दश सुव॒र्ग्याः᳚ सुव॒र्ग्याः᳚ पञ्च॑दश । \newline
35. सु॒व॒र्ग्या॑ इति॑ सुवः - ग्याः᳚ । \newline
36. पञ्च॑दश॒ वै वै पञ्च॑दश॒ पञ्च॑दश॒ वै । \newline
37. पञ्च॑द॒शेति॒ पञ्च॑ - द॒श॒ । \newline
38. वा अ॑र्द्धमा॒सस्या᳚ र्द्धमा॒सस्य॒ वै वा अ॑र्द्धमा॒सस्य॑ । \newline
39. अ॒र्द्ध॒मा॒सस्य॒ रात्र॑यो॒ रात्र॑यो ऽर्द्धमा॒सस्या᳚ र्द्धमा॒सस्य॒ रात्र॑यः । \newline
40. अ॒र्द्ध॒मा॒सस्येत्य॑र्द्ध - मा॒सस्य॑ । \newline
41. रात्र॑यो ऽर्द्धमास॒शो᳚ ऽर्द्धमास॒शो रात्र॑यो॒ रात्र॑यो ऽर्द्धमास॒शः । \newline
42. अ॒र्द्ध॒मा॒स॒शः सं॑ॅवथ्स॒रः सं॑ॅवथ्स॒रो᳚ ऽर्द्धमास॒शो᳚ ऽर्द्धमास॒शः सं॑ॅवथ्स॒रः । \newline
43. अ॒र्द्ध॒मा॒स॒श इत्य॑र्द्धमास - शः । \newline
44. सं॒ॅव॒थ्स॒र आ᳚प्यत आप्यते संॅवथ्स॒रः सं॑ॅवथ्स॒र आ᳚प्यते । \newline
45. सं॒ॅव॒थ्स॒र इति॑ सं - व॒थ्स॒रः । \newline
46. आ॒प्य॒ते॒ सं॒ॅव॒थ्स॒रः सं॑ॅवथ्स॒र आ᳚प्यत आप्यते संॅवथ्स॒रः । \newline
47. सं॒ॅव॒थ्स॒रः सु॑व॒र्गः सु॑व॒र्गः सं॑ॅवथ्स॒रः सं॑ॅवथ्स॒रः सु॑व॒र्गः । \newline
48. सं॒ॅव॒थ्स॒र इति॑ सं - व॒थ्स॒रः । \newline
49. सु॒व॒र्गो लो॒को लो॒कः सु॑व॒र्गः सु॑व॒र्गो लो॒कः । \newline
50. सु॒व॒र्ग इति॑ सुवः - गः । \newline
51. लो॒क स्तस्मा॒त् तस्मा᳚ ल्लो॒को लो॒क स्तस्मा᳚त् । \newline
52. तस्मा᳚थ् सुव॒र्ग्याः᳚ सुव॒र्ग्या᳚ स्तस्मा॒त् तस्मा᳚थ् सुव॒र्ग्याः᳚ । \newline
53. सु॒व॒र्ग्या᳚ ज्योति॒र् ज्योतिः॑ सुव॒र्ग्याः᳚ सुव॒र्ग्या᳚ ज्योतिः॑ । \newline
54. सु॒व॒र्ग्या॑ इति॑ सुवः - ग्याः᳚ । \newline
55. ज्योति॒र् गौर् गौर् ज्योति॒र् ज्योति॒र् गौः । \newline
56. गौ रायु॒ रायु॒र् गौर् गौ रायुः॑ । \newline
57. आयु॒ रिती त्यायु॒ रायु॒ रिति॑ । \newline
58. इति॑ त्र्य॒ह स्त्र्य॒ह इतीति॑ त्र्य॒हः । \newline
59. त्र्य॒हो भ॑वति भवति त्र्य॒ह स्त्र्य॒हो भ॑वति । \newline
60. त्र्य॒ह इति॑ त्रि - अ॒हः । \newline
61. भ॒व॒ती॒य मि॒यम् भ॑वति भवती॒यम् । \newline
62. इ॒यं ॅवाव वावेय मि॒यं ॅवाव । \newline
63. वाव ज्योति॒र् ज्योति॒र् वाव वाव ज्योतिः॑ । \newline
64. ज्योति॑ र॒न्तरि॑क्ष म॒न्तरि॑क्ष॒म् ज्योति॒र् ज्योति॑ र॒न्तरि॑क्षम् । \newline
65. अ॒न्तरि॑क्ष॒म् गौर् गौ र॒न्तरि॑क्ष म॒न्तरि॑क्ष॒म् गौः । \newline

\textbf{Ghana Paata } \newline

1. द॒ह॒न्ते॒ प॒ञ्च॒द॒श॒रा॒त्रेण॑ पञ्चदशरा॒त्रेण॑ दहन्ते दहन्ते पञ्चदशरा॒त्रे णौज॒ ओजः॑ पञ्चदशरा॒त्रेण॑ दहन्ते दहन्ते पञ्चदशरा॒त्रे णौजः॑ । \newline
2. प॒ञ्च॒द॒श॒रा॒त्रे णौज॒ ओजः॑ पञ्चदशरा॒त्रेण॑ पञ्चदशरा॒त्रे णौजो॒ बल॒म् बल॒ मोजः॑ पञ्चदशरा॒त्रेण॑ पञ्चदशरा॒त्रे णौजो॒ बल᳚म् । \newline
3. प॒ञ्च॒द॒श॒रा॒त्रेणेति॑ पञ्चदश - रा॒त्रेण॑ । \newline
4. ओजो॒ बल॒म् बल॒ मोज॒ ओजो॒ बल॑ मिन्द्रि॒य मि॑न्द्रि॒यम् बल॒ मोज॒ ओजो॒ बल॑ मिन्द्रि॒यम् । \newline
5. बल॑ मिन्द्रि॒य मि॑न्द्रि॒यम् बल॒म् बल॑ मिन्द्रि॒यं ॅवी॒र्यं॑ ॅवी॒र्य॑ मिन्द्रि॒यम् बल॒म् बल॑ मिन्द्रि॒यं ॅवी॒र्य᳚म् । \newline
6. इ॒न्द्रि॒यं ॅवी॒र्यं॑ ॅवी॒र्य॑ मिन्द्रि॒य मि॑न्द्रि॒यं ॅवी॒र्य॑ मा॒त्मन् ना॒त्मन्. वी॒र्य॑ मिन्द्रि॒य मि॑न्द्रि॒यं ॅवी॒र्य॑ मा॒त्मन्न् । \newline
7. वी॒र्य॑ मा॒त्मन् ना॒त्मन्. वी॒र्यं॑ ॅवी॒र्य॑ मा॒त्मन् द॑धते दधत आ॒त्मन्. वी॒र्यं॑ ॅवी॒र्य॑ मा॒त्मन् द॑धते । \newline
8. आ॒त्मन् द॑धते दधत आ॒त्मन् ना॒त्मन् द॑धत ए॒ता ए॒ता द॑धत आ॒त्मन् ना॒त्मन् द॑धत ए॒ताः । \newline
9. द॒ध॒त॒ ए॒ता ए॒ता द॑धते दधत ए॒ता ए॒वैवैता द॑धते दधत ए॒ता ए॒व । \newline
10. ए॒ता ए॒वैवैता ए॒ता ए॒व प॑श॒व्याः᳚ पश॒व्या॑ ए॒वैता ए॒ता ए॒व प॑श॒व्याः᳚ । \newline
11. ए॒व प॑श॒व्याः᳚ पश॒व्या॑ ए॒वैव प॑श॒व्याः᳚ पञ्च॑दश॒ पञ्च॑दश पश॒व्या॑ ए॒वैव प॑श॒व्याः᳚ पञ्च॑दश । \newline
12. प॒श॒व्याः᳚ पञ्च॑दश॒ पञ्च॑दश पश॒व्याः᳚ पश॒व्याः᳚ पञ्च॑दश॒ वै वै पञ्च॑दश पश॒व्याः᳚ पश॒व्याः᳚ पञ्च॑दश॒ वै । \newline
13. पञ्च॑दश॒ वै वै पञ्च॑दश॒ पञ्च॑दश॒ वा अ॑र्द्धमा॒सस्या᳚ र्द्धमा॒सस्य॒ वै पञ्च॑दश॒ पञ्च॑दश॒ वा अ॑र्द्धमा॒सस्य॑ । \newline
14. पञ्च॑द॒शेति॒ पञ्च॑ - द॒श॒ । \newline
15. वा अ॑र्द्धमा॒सस्या᳚ र्द्धमा॒सस्य॒ वै वा अ॑र्द्धमा॒सस्य॒ रात्र॑यो॒ रात्र॑यो ऽर्द्धमा॒सस्य॒ वै वा अ॑र्द्धमा॒सस्य॒ रात्र॑यः । \newline
16. अ॒र्द्ध॒मा॒सस्य॒ रात्र॑यो॒ रात्र॑यो ऽर्द्धमा॒सस्या᳚ र्द्धमा॒सस्य॒ रात्र॑यो ऽर्द्धमास॒शो᳚ ऽर्द्धमास॒शो रात्र॑यो ऽर्द्धमा॒सस्या᳚ र्द्धमा॒सस्य॒ रात्र॑यो ऽर्द्धमास॒शः । \newline
17. अ॒र्द्ध॒मा॒सस्येत्य॑र्द्ध - मा॒सस्य॑ । \newline
18. रात्र॑यो ऽर्द्धमास॒शो᳚ ऽर्द्धमास॒शो रात्र॑यो॒ रात्र॑यो ऽर्द्धमास॒शः सं॑ॅवथ्स॒रः सं॑ॅवथ्स॒रो᳚ ऽर्द्धमास॒शो रात्र॑यो॒ रात्र॑यो ऽर्द्धमास॒शः सं॑ॅवथ्स॒रः । \newline
19. अ॒र्द्ध॒मा॒स॒शः सं॑ॅवथ्स॒रः सं॑ॅवथ्स॒रो᳚ ऽर्द्धमास॒शो᳚ ऽर्द्धमास॒शः सं॑ॅवथ्स॒र आ᳚प्यत आप्यते संॅवथ्स॒रो᳚ ऽर्द्धमास॒शो᳚ ऽर्द्धमास॒शः सं॑ॅवथ्स॒र आ᳚प्यते । \newline
20. अ॒र्द्ध॒मा॒स॒श इत्य॑र्द्धमास - शः । \newline
21. सं॒ॅव॒थ्स॒र आ᳚प्यत आप्यते संॅवथ्स॒रः सं॑ॅवथ्स॒र आ᳚प्यते संॅवथ्स॒रꣳ सं॑ॅवथ्स॒र मा᳚प्यते संॅवथ्स॒रः सं॑ॅवथ्स॒र आ᳚प्यते संॅवथ्स॒रम् । \newline
22. सं॒ॅव॒थ्स॒र इति॑ सं - व॒थ्स॒रः । \newline
23. आ॒प्य॒ते॒ सं॒ॅव॒थ्स॒रꣳ सं॑ॅवथ्स॒र मा᳚प्यत आप्यते संॅवथ्स॒रम् प॒शवः॑ प॒शवः॑ संॅवथ्स॒र मा᳚प्यत आप्यते संॅवथ्स॒रम् प॒शवः॑ । \newline
24. सं॒ॅव॒थ्स॒रम् प॒शवः॑ प॒शवः॑ संॅवथ्स॒रꣳ सं॑ॅवथ्स॒रम् प॒शवो ऽन्वनु॑ प॒शवः॑ संॅवथ्स॒रꣳ सं॑ॅवथ्स॒रम् प॒शवो ऽनु॑ । \newline
25. सं॒ॅव॒थ्स॒रमिति॑ सं - व॒थ्स॒रम् । \newline
26. प॒शवो ऽन्वनु॑ प॒शवः॑ प॒शवो ऽनु॒ प्र प्राणु॑ प॒शवः॑ प॒शवो ऽनु॒ प्र । \newline
27. अनु॒ प्र प्राण्वनु॒ प्र जा॑यन्ते जायन्ते॒ प्राण्वनु॒ प्र जा॑यन्ते । \newline
28. प्र जा॑यन्ते जायन्ते॒ प्र प्र जा॑यन्ते॒ तस्मा॒त् तस्मा᳚ज् जायन्ते॒ प्र प्र जा॑यन्ते॒ तस्मा᳚त् । \newline
29. जा॒य॒न्ते॒ तस्मा॒त् तस्मा᳚ज् जायन्ते जायन्ते॒ तस्मा᳚त् पश॒व्याः᳚ पश॒व्या᳚ स्तस्मा᳚ज् जायन्ते जायन्ते॒ तस्मा᳚त् पश॒व्याः᳚ । \newline
30. तस्मा᳚त् पश॒व्याः᳚ पश॒व्या᳚ स्तस्मा॒त् तस्मा᳚त् पश॒व्या॑ ए॒ता ए॒ताः प॑श॒व्या᳚ स्तस्मा॒त् तस्मा᳚त् पश॒व्या॑ ए॒ताः । \newline
31. प॒श॒व्या॑ ए॒ता ए॒ताः प॑श॒व्याः᳚ पश॒व्या॑ ए॒ता ए॒वै वैताः प॑श॒व्याः᳚ पश॒व्या॑ ए॒ता ए॒व । \newline
32. ए॒ता ए॒वै वैता ए॒ता ए॒व सु॑व॒र्ग्याः᳚ सुव॒र्ग्या॑ ए॒वैता ए॒ता ए॒व सु॑व॒र्ग्याः᳚ । \newline
33. ए॒व सु॑व॒र्ग्याः᳚ सुव॒र्ग्या॑ ए॒वैव सु॑व॒र्ग्याः᳚ पञ्च॑दश॒ पञ्च॑दश सुव॒र्ग्या॑ ए॒वैव सु॑व॒र्ग्याः᳚ पञ्च॑दश । \newline
34. सु॒व॒र्ग्याः᳚ पञ्च॑दश॒ पञ्च॑दश सुव॒र्ग्याः᳚ सुव॒र्ग्याः᳚ पञ्च॑दश॒ वै वै पञ्च॑दश सुव॒र्ग्याः᳚ सुव॒र्ग्याः᳚ पञ्च॑दश॒ वै । \newline
35. सु॒व॒र्ग्या॑ इति॑ सुवः - ग्याः᳚ । \newline
36. पञ्च॑दश॒ वै वै पञ्च॑दश॒ पञ्च॑दश॒ वा अ॑र्द्धमा॒सस्या᳚ र्द्धमा॒सस्य॒ वै पञ्च॑दश॒ पञ्च॑दश॒ वा अ॑र्द्धमा॒सस्य॑ । \newline
37. पञ्च॑द॒शेति॒ पञ्च॑ - द॒श॒ । \newline
38. वा अ॑र्द्धमा॒सस्या᳚ र्द्धमा॒सस्य॒ वै वा अ॑र्द्धमा॒सस्य॒ रात्र॑यो॒ रात्र॑यो ऽर्द्धमा॒सस्य॒ वै वा अ॑र्द्धमा॒सस्य॒ रात्र॑यः । \newline
39. अ॒र्द्ध॒मा॒सस्य॒ रात्र॑यो॒ रात्र॑यो ऽर्द्धमा॒सस्या᳚ र्द्धमा॒सस्य॒ रात्र॑यो ऽर्द्धमास॒शो᳚ ऽर्द्धमास॒शो रात्र॑यो ऽर्द्धमा॒सस्या᳚ र्द्धमा॒सस्य॒ रात्र॑यो ऽर्द्धमास॒शः । \newline
40. अ॒र्द्ध॒मा॒सस्येत्य॑र्द्ध - मा॒सस्य॑ । \newline
41. रात्र॑यो ऽर्द्धमास॒शो᳚ ऽर्द्धमास॒शो रात्र॑यो॒ रात्र॑यो ऽर्द्धमास॒शः सं॑ॅवथ्स॒रः सं॑ॅवथ्स॒रो᳚ ऽर्द्धमास॒शो रात्र॑यो॒ रात्र॑यो ऽर्द्धमास॒शः सं॑ॅवथ्स॒रः । \newline
42. अ॒र्द्ध॒मा॒स॒शः सं॑ॅवथ्स॒रः सं॑ॅवथ्स॒रो᳚ ऽर्द्धमास॒शो᳚ ऽर्द्धमास॒शः सं॑ॅवथ्स॒र आ᳚प्यत आप्यते संॅवथ्स॒रो᳚ ऽर्द्धमास॒शो᳚ ऽर्द्धमास॒शः सं॑ॅवथ्स॒र आ᳚प्यते । \newline
43. अ॒र्द्ध॒मा॒स॒श इत्य॑र्द्धमास - शः । \newline
44. सं॒ॅव॒थ्स॒र आ᳚प्यत आप्यते संॅवथ्स॒रः सं॑ॅवथ्स॒र आ᳚प्यते संॅवथ्स॒रः सं॑ॅवथ्स॒र आ᳚प्यते संॅवथ्स॒रः सं॑ॅवथ्स॒र आ᳚प्यते संॅवथ्स॒रः । \newline
45. सं॒ॅव॒थ्स॒र इति॑ सं - व॒थ्स॒रः । \newline
46. आ॒प्य॒ते॒ सं॒ॅव॒थ्स॒रः सं॑ॅवथ्स॒र आ᳚प्यत आप्यते संॅवथ्स॒रः सु॑व॒र्गः सु॑व॒र्गः सं॑ॅवथ्स॒र आ᳚प्यत आप्यते संॅवथ्स॒रः सु॑व॒र्गः । \newline
47. सं॒ॅव॒थ्स॒रः सु॑व॒र्गः सु॑व॒र्गः सं॑ॅवथ्स॒रः सं॑ॅवथ्स॒रः सु॑व॒र्गो लो॒को लो॒कः सु॑व॒र्गः सं॑ॅवथ्स॒रः सं॑ॅवथ्स॒रः सु॑व॒र्गो लो॒कः । \newline
48. सं॒ॅव॒थ्स॒र इति॑ सं - व॒थ्स॒रः । \newline
49. सु॒व॒र्गो लो॒को लो॒कः सु॑व॒र्गः सु॑व॒र्गो लो॒क स्तस्मा॒त् तस्मा᳚ ल्लो॒कः सु॑व॒र्गः सु॑व॒र्गो लो॒क स्तस्मा᳚त् । \newline
50. सु॒व॒र्ग इति॑ सुवः - गः । \newline
51. लो॒क स्तस्मा॒त् तस्मा᳚ ल्लो॒को लो॒क स्तस्मा᳚थ् सुव॒र्ग्याः᳚ सुव॒र्ग्या᳚ स्तस्मा᳚ ल्लो॒को लो॒क स्तस्मा᳚थ् सुव॒र्ग्याः᳚ । \newline
52. तस्मा᳚थ् सुव॒र्ग्याः᳚ सुव॒र्ग्या᳚ स्तस्मा॒त् तस्मा᳚थ् सुव॒र्ग्या᳚ ज्योति॒र् ज्योतिः॑ सुव॒र्ग्या᳚ स्तस्मा॒त् तस्मा᳚थ् सुव॒र्ग्या᳚ ज्योतिः॑ । \newline
53. सु॒व॒र्ग्या᳚ ज्योति॒र् ज्योतिः॑ सुव॒र्ग्याः᳚ सुव॒र्ग्या᳚ ज्योति॒र् गौर् गौर् ज्योतिः॑ सुव॒र्ग्याः᳚ सुव॒र्ग्या᳚ ज्योति॒र् गौः । \newline
54. सु॒व॒र्ग्या॑ इति॑ सुवः - ग्याः᳚ । \newline
55. ज्योति॒र् गौर् गौर् ज्योति॒र् ज्योति॒र् गौ रायु॒ रायु॒र् गौर् ज्योति॒र् ज्योति॒र् गौ रायुः॑ । \newline
56. गौ रायु॒ रायु॒र् गौर् गौ रायु॒ रिती त्यायु॒र् गौर् गौ रायु॒ रिति॑ । \newline
57. आयु॒ रिती त्यायु॒ रायु॒ रिति॑ त्र्य॒ह स्त्र्य॒ह इत्यायु॒ रायु॒ रिति॑ त्र्य॒हः । \newline
58. इति॑ त्र्य॒ह स्त्र्य॒ह इतीति॑ त्र्य॒हो भ॑वति भवति त्र्य॒ह इतीति॑ त्र्य॒हो भ॑वति । \newline
59. त्र्य॒हो भ॑वति भवति त्र्य॒ह स्त्र्य॒हो भ॑वती॒य मि॒यम् भ॑वति त्र्य॒ह स्त्र्य॒हो भ॑वती॒यम् । \newline
60. त्र्य॒ह इति॑ त्रि - अ॒हः । \newline
61. भ॒व॒ती॒य मि॒यम् भ॑वति भवती॒यं ॅवाव वावेयम् भ॑वति भवती॒यं ॅवाव । \newline
62. इ॒यं ॅवाव वावेय मि॒यं ॅवाव ज्योति॒र् ज्योति॒र् वावेय मि॒यं ॅवाव ज्योतिः॑ । \newline
63. वाव ज्योति॒र् ज्योति॒र् वाव वाव ज्योति॑ र॒न्तरि॑क्ष म॒न्तरि॑क्ष॒म् ज्योति॒र् वाव वाव ज्योति॑ र॒न्तरि॑क्षम् । \newline
64. ज्योति॑ र॒न्तरि॑क्ष म॒न्तरि॑क्ष॒म् ज्योति॒र् ज्योति॑ र॒न्तरि॑क्ष॒म् गौर् गौर॒न्तरि॑क्ष॒म् ज्योति॒र् ज्योति॑ र॒न्तरि॑क्ष॒म् गौः । \newline
65. अ॒न्तरि॑क्ष॒म् गौर् गौर॒न्तरि॑क्ष म॒न्तरि॑क्ष॒म् गौर॒सा व॒सौ गौर॒न्तरि॑क्ष म॒न्तरि॑क्ष॒म् गौर॒सौ । \newline
\pagebreak
\markright{ TS 7.3.7.3  \hfill https://www.vedavms.in \hfill}

\section{ TS 7.3.7.3 }

\textbf{TS 7.3.7.3 } \newline
\textbf{Samhita Paata} \newline

गौर॒सावायु॑रि॒माने॒व लो॒कान॒भ्यारो॑हन्ति॒ यद॒न्यतः॑ पृ॒ष्ठानि॒ स्युर्विवि॑वधꣳ स्या॒न्मद्ध्ये॑ पृ॒ष्ठानि॑ भवन्ति सविवध॒त्वायौजो॒ वै वी॒र्यं॑ पृ॒ष्ठान्योज॑ ए॒व वी॒र्यं॑ मद्ध्य॒तो द॑धते बृहद्-रथन्त॒राभ्यां᳚ ॅयन्ती॒यं ॅवाव र॑थन्त॒रम॒सौ बृ॒हदा॒भ्यामे॒व य॒न्त्यथो॑ अ॒नयो॑रे॒व प्रति॑ तिष्ठन्त्ये॒ते वै य॒ज्ञ्स्या᳚ञ्ज॒साय॑नी स्रु॒ती ताभ्या॑मे॒व सु॑व॒र्गं ॅलो॒कं - [  ] \newline

\textbf{Pada Paata} \newline

गौः । अ॒सौ । आयुः॑ । इ॒मान् । ए॒व । लो॒कान् । अ॒भ्यारो॑ह॒न्तीत्य॑भि - आरो॑हन्ति । यत् । अ॒न्यतः॑ । पृ॒ष्ठानि॑ । स्युः । विवि॑वध॒मिति॒ वि - वि॒व॒ध॒म् । स्या॒त् । मद्ध्ये᳚ । पृ॒ष्ठानि॑ । भ॒व॒न्ति॒ । स॒वि॒व॒ध॒त्वायेति॑ सविवध - त्वाय॑ । ओजः॑ । वै । वी॒र्य᳚म् । पृ॒ष्ठानि॑ । ओजः॑ । ए॒व । वी॒र्य᳚म् । म॒द्ध्य॒तः । द॒ध॒ते॒ । बृ॒ह॒द्र॒थ॒न्त॒राभ्या॒मिति॑ बृहत्-र॒थ॒न्त॒राभ्या᳚म् । य॒न्ति॒ । इ॒यम् । वाव । र॒थ॒न्त॒रमिति॑ रथं - त॒रम् । अ॒सौ । बृ॒हत् । आ॒भ्याम् । ए॒व । य॒न्ति॒ । अथो॒ इति॑ । अ॒नयोः᳚ । ए॒व । प्रतीति॑ । ति॒ष्ठ॒न्ति॒ । ए॒ते इति॑ । वै । य॒ज्ञ्स्य॑ । अ॒ञ्ज॒साय॑नी॒ इत्य॑ञ्जसा-अय॑नी । स्रु॒ती इति॑ । ताभ्या᳚म् । ए॒व । सु॒व॒र्गमिति॑ सुवः - गम् । लो॒कम् ।  \newline


\textbf{Krama Paata} \newline

गौर॒सौ । अ॒सावायुः॑ । आयु॑रि॒मान् । इ॒माने॒व । ए॒व लो॒कान् । लो॒कान॒भ्यारो॑हन्ति । अ॒भ्यारो॑हन्ति॒ यत् । अ॒भ्यारो॑ह॒न्तीत्य॑भि - आरो॑हन्ति । यद॒न्यतः॑ । अ॒न्यतः॑ पृ॒ष्ठानि॑ । पृ॒ष्ठानि॒ स्युः । स्युर् विवि॑वधम् । विवि॑वधꣳ स्यात् । विवि॑वध॒मिति॒ वि - वि॒व॒ध॒म् । स्या॒न् मद्ध्ये᳚ । मद्ध्ये॑ पृ॒ष्ठानि॑ । पृ॒ष्ठानि॑ भवन्ति । भ॒व॒न्ति॒ स॒वि॒व॒ध॒त्वाय॑ । स॒वि॒व॒ध॒त्वायौजः॑ । स॒वि॒व॒ध॒त्वायेति॑ सविवध - त्वाय॑ । ओजो॒ वै । वै वी॒र्य᳚म् । वी॒र्य॑म् पृ॒ष्ठानि॑ । पृ॒ष्ठान्योजः॑ । ओज॑ ए॒व । ए॒व वी॒र्य᳚म् । वी॒र्य॑म् मद्ध्य॒तः । म॒द्ध्य॒तो द॑धते । द॒ध॒ते॒ बृ॒ह॒द्‍र॒थ॒न्त॒राभ्या᳚म् । बृ॒ह॒द्‍र॒थ॒न्त॒राभ्या᳚म् ॅयन्ति । बृ॒ह॒द्‍र॒थ॒न्त॒राभ्या॒मिति॑ बृहत् - र॒थ॒न्त॒राभ्या᳚म् । य॒न्ती॒यम् । इ॒यम् ॅवाव । वाव र॑थन्त॒रम् । र॒थ॒न्त॒रम॒सौ । र॒थ॒न्त॒रमिति॑रथम् - त॒रम् । अ॒सौ बृ॒हत् । बृ॒हदा॒भ्याम् । आ॒भ्यामे॒व । ए॒व य॑न्ति । य॒न्त्यथो᳚ । अथो॑ अ॒नयोः᳚ । अथो॒ इत्यथो᳚ । अ॒नयो॑रे॒व । ए॒व प्रति॑ । प्रति॑ तिष्ठन्ति । ति॒ष्ठ॒न्त्ये॒ते । ए॒ते वै । ए॒ते इत्ये॒ते । वै य॒ज्ञ्स्य॑ । य॒ज्ञ्स्या᳚ञ्ज॒साय॑नी । अ॒ञ्ज॒साय॑नी स्रु॒ती । अ॒ञ्ज॒साय॑नी॒ इत्य॑ञ्जसा - अय॑नी । स्रु॒ती ताभ्या᳚म् । स्रु॒ती इति॑ स्रु॒ती । ताभ्या॑मे॒व । ए॒व सु॑व॒र्गम् । सु॒व॒र्गम् ॅलो॒कम् । सु॒व॒र्गमिति॑ सुवः - गम् । लो॒कम् ॅय॑न्ति \newline

\textbf{Jatai Paata} \newline

1. गौ र॒सा व॒सौ गौर् गौ र॒सौ । \newline
2. अ॒सा वायु॒ रायु॑ र॒सा व॒सा वायुः॑ । \newline
3. आयु॑ रि॒मा नि॒मा नायु॒ रायु॑ रि॒मान् । \newline
4. इ॒मा ने॒वैवेमा नि॒माने॒व । \newline
5. ए॒व लो॒कान् ॅलो॒का ने॒वैव लो॒कान् । \newline
6. लो॒का न॒भ्यारो॑ह न्त्य॒भ्यारो॑हन्ति लो॒कान् ॅलो॒का न॒भ्यारो॑हन्ति । \newline
7. अ॒भ्यारो॑हन्ति॒ यद् यद॒भ्यारो॑ह न्त्य॒भ्यारो॑हन्ति॒ यत् । \newline
8. अ॒भ्यारो॑ह॒न्तीत्य॑भि - आरो॑हन्ति । \newline
9. यद॒न्यतो॒ ऽन्यतो॒ यद् यद॒न्यतः॑ । \newline
10. अ॒न्यतः॑ पृ॒ष्ठानि॑ पृ॒ष्ठा न्य॒न्यतो॒ ऽन्यतः॑ पृ॒ष्ठानि॑ । \newline
11. पृ॒ष्ठानि॒ स्युः स्युः पृ॒ष्ठानि॑ पृ॒ष्ठानि॒ स्युः । \newline
12. स्युर् विवि॑वधं॒ ॅविवि॑वधꣳ॒॒ स्युः स्युर् विवि॑वधम् । \newline
13. विवि॑वधꣳ स्याथ् स्या॒द् विवि॑वधं॒ ॅविवि॑वधꣳ स्यात् । \newline
14. विवि॑वध॒मिति॒ वि - वि॒व॒ध॒म् । \newline
15. स्या॒न् मद्ध्ये॒ मद्ध्ये᳚ स्याथ् स्या॒न् मद्ध्ये᳚ । \newline
16. मद्ध्ये॑ पृ॒ष्ठानि॑ पृ॒ष्ठानि॒ मद्ध्ये॒ मद्ध्ये॑ पृ॒ष्ठानि॑ । \newline
17. पृ॒ष्ठानि॑ भवन्ति भवन्ति पृ॒ष्ठानि॑ पृ॒ष्ठानि॑ भवन्ति । \newline
18. भ॒व॒न्ति॒ स॒वि॒व॒ध॒त्वाय॑ सविवध॒त्वाय॑ भवन्ति भवन्ति सविवध॒त्वाय॑ । \newline
19. स॒वि॒व॒ध॒त्वा यौज॒ ओजः॑ सविवध॒त्वाय॑ सविवध॒त्वा यौजः॑ । \newline
20. स॒वि॒व॒ध॒त्वायेति॑ सविवध - त्वाय॑ । \newline
21. ओजो॒ वै वा ओज॒ ओजो॒ वै । \newline
22. वै वी॒र्यं॑ ॅवी॒र्यं॑ ॅवै वै वी॒र्य᳚म् । \newline
23. वी॒र्य॑म् पृ॒ष्ठानि॑ पृ॒ष्ठानि॑ वी॒र्यं॑ ॅवी॒र्य॑म् पृ॒ष्ठानि॑ । \newline
24. पृ॒ष्ठा न्योज॒ ओजः॑ पृ॒ष्ठानि॑ पृ॒ष्ठा न्योजः॑ । \newline
25. ओज॑ ए॒वै वौज॒ ओज॑ ए॒व । \newline
26. ए॒व वी॒र्यं॑ ॅवी॒र्य॑ मे॒वैव वी॒र्य᳚म् । \newline
27. वी॒र्य॑म् मद्ध्य॒तो म॑द्ध्य॒तो वी॒र्यं॑ ॅवी॒र्य॑म् मद्ध्य॒तः । \newline
28. म॒द्ध्य॒तो द॑धते दधते मद्ध्य॒तो म॑द्ध्य॒तो द॑धते । \newline
29. द॒ध॒ते॒ बृ॒ह॒द्र॒थ॒न्त॒राभ्या᳚म् बृहद्रथन्त॒राभ्या᳚म् दधते दधते बृहद्रथन्त॒राभ्या᳚म् । \newline
30. बृ॒ह॒द्र॒थ॒न्त॒राभ्यां᳚ ॅयन्ति यन्ति बृहद्रथन्त॒राभ्या᳚म् बृहद्रथन्त॒राभ्यां᳚ ॅयन्ति । \newline
31. बृ॒ह॒द्र॒थ॒न्त॒राभ्या॒मिति॑ बृहत् - र॒थ॒न्त॒राभ्या᳚म् । \newline
32. य॒न्ती॒य मि॒यं ॅय॑न्ति यन्ती॒यम् । \newline
33. इ॒यं ॅवाव वावेय मि॒यं ॅवाव । \newline
34. वाव र॑थन्त॒रꣳ र॑थन्त॒रं ॅवाव वाव र॑थन्त॒रम् । \newline
35. र॒थ॒न्त॒र म॒सा व॒सौ र॑थन्त॒रꣳ र॑थन्त॒र म॒सौ । \newline
36. र॒थ॒न्त॒रमिति॑ रथं - त॒रम् । \newline
37. अ॒सौ बृ॒हद् बृ॒ह द॒सा व॒सौ बृ॒हत् । \newline
38. बृ॒ह दा॒भ्या मा॒भ्याम् बृ॒हद् बृ॒ह दा॒भ्याम् । \newline
39. आ॒भ्या मे॒वै वाभ्या मा॒भ्या मे॒व । \newline
40. ए॒व य॑न्ति यन्त्ये॒ वैव य॑न्ति । \newline
41. य॒न्त्यथो॒ अथो॑ यन्ति य॒न्त्यथो᳚ । \newline
42. अथो॑ अ॒नयो॑ र॒नयो॒ रथो॒ अथो॑ अ॒नयोः᳚ । \newline
43. अथो॒ इत्यथो᳚ । \newline
44. अ॒नयो॑ रे॒वैवा नयो॑ र॒नयो॑ रे॒व । \newline
45. ए॒व प्रति॒ प्रत्ये॒वैव प्रति॑ । \newline
46. प्रति॑ तिष्ठन्ति तिष्ठन्ति॒ प्रति॒ प्रति॑ तिष्ठन्ति । \newline
47. ति॒ष्ठ॒न्त्ये॒ते ए॒ते ति॑ष्ठन्ति तिष्ठन्त्ये॒ते । \newline
48. ए॒ते वै वा ए॒ते ए॒ते वै । \newline
49. ए॒ते इत्ये॒ते । \newline
50. वै य॒ज्ञ्स्य॑ य॒ज्ञ्स्य॒ वै वै य॒ज्ञ्स्य॑ । \newline
51. य॒ज्ञ्स्या᳚ ञ्ज॒साय॑नी अञ्ज॒साय॑नी य॒ज्ञ्स्य॑ य॒ज्ञ्स्या᳚ ञ्ज॒साय॑नी । \newline
52. अ॒ञ्ज॒साय॑नी स्रु॒ती स्रु॒ती अ॑ञ्ज॒साय॑नी अञ्ज॒साय॑नी स्रु॒ती । \newline
53. अ॒ञ्ज॒साय॑नी॒ इत्य॑ञ्जसा - अय॑नी । \newline
54. स्रु॒ती ताभ्या॒म् ताभ्याꣳ॑ स्रु॒ती स्रु॒ती ताभ्या᳚म् । \newline
55. स्रु॒ती इति॑ स्रु॒ती । \newline
56. ताभ्या॑ मे॒वैव ताभ्या॒म् ताभ्या॑ मे॒व । \newline
57. ए॒व सु॑व॒र्गꣳ सु॑व॒र्ग मे॒वैव सु॑व॒र्गम् । \newline
58. सु॒व॒र्गम् ॅलो॒कम् ॅलो॒कꣳ सु॑व॒र्गꣳ सु॑व॒र्गम् ॅलो॒कम् । \newline
59. सु॒व॒र्गमिति॑ सुवः - गम् । \newline
60. लो॒कं ॅय॑न्ति यन्ति लो॒कम् ॅलो॒कं ॅय॑न्ति । \newline

\textbf{Ghana Paata } \newline

1. गौ र॒सा व॒सौ गौर् गौ र॒सा वायु॒ रायु॑ र॒सौ गौर् गौ र॒सा वायुः॑ । \newline
2. अ॒सा वायु॒ रायु॑ र॒सा व॒सा वायु॑ रि॒मा नि॒मा नायु॑ र॒सा व॒सा वायु॑ रि॒मान् । \newline
3. आयु॑ रि॒मा नि॒मा नायु॒ रायु॑ रि॒मा ने॒वैवेमा नायु॒ रायु॑ रि॒मा ने॒व । \newline
4. इ॒मा ने॒वैवेमा नि॒मा ने॒व लो॒कान् ॅलो॒का ने॒वेमा नि॒मा ने॒व लो॒कान् । \newline
5. ए॒व लो॒कान् ॅलो॒का ने॒वैव लो॒का न॒भ्यारो॑ह न्त्य॒भ्यारो॑हन्ति लो॒का ने॒वैव लो॒का न॒भ्यारो॑हन्ति । \newline
6. लो॒का न॒भ्यारो॑ह न्त्य॒भ्यारो॑हन्ति लो॒कान् ॅलो॒का न॒भ्यारो॑हन्ति॒ यद् यद॒भ्यारो॑हन्ति लो॒कान् ॅलो॒का न॒भ्यारो॑हन्ति॒ यत् । \newline
7. अ॒भ्यारो॑हन्ति॒ यद् यद॒भ्यारो॑ह न्त्य॒भ्यारो॑हन्ति॒ यद॒न्यतो॒ ऽन्यतो॒ यद॒भ्यारो॑ह न्त्य॒भ्यारो॑हन्ति॒ यद॒न्यतः॑ । \newline
8. अ॒भ्यारो॑ह॒न्तीत्य॑भि - आरो॑हन्ति । \newline
9. यद॒न्यतो॒ ऽन्यतो॒ यद् यद॒न्यतः॑ पृ॒ष्ठानि॑ पृ॒ष्ठा न्य॒ न्यतो॒ यद् यद॒न्यतः॑ पृ॒ष्ठानि॑ । \newline
10. अ॒न्यतः॑ पृ॒ष्ठानि॑ पृ॒ष्ठा न्य॒ न्यतो॒ ऽन्यतः॑ पृ॒ष्ठानि॒ स्युः स्युः पृ॒ष्ठा न्य॒न्यतो॒ ऽन्यतः॑ पृ॒ष्ठानि॒ स्युः । \newline
11. पृ॒ष्ठानि॒ स्युः स्युः पृ॒ष्ठानि॑ पृ॒ष्ठानि॒ स्युर् विवि॑वधं॒ ॅविवि॑वधꣳ॒॒ स्युः पृ॒ष्ठानि॑ पृ॒ष्ठानि॒ स्युर् विवि॑वधम् । \newline
12. स्युर् विवि॑वधं॒ ॅविवि॑वधꣳ॒॒ स्युः स्युर् विवि॑वधꣳ स्याथ् स्या॒द् विवि॑वधꣳ॒॒ स्युः स्युर् विवि॑वधꣳ स्यात् । \newline
13. विवि॑वधꣳ स्याथ् स्या॒द् विवि॑वधं॒ ॅविवि॑वधꣳ स्या॒न् मद्ध्ये॒ मद्ध्ये᳚ स्या॒द् विवि॑वधं॒ ॅविवि॑वधꣳ स्या॒न् मद्ध्ये᳚ । \newline
14. विवि॑वध॒मिति॒ वि - वि॒व॒ध॒म् । \newline
15. स्या॒न् मद्ध्ये॒ मद्ध्ये᳚ स्याथ् स्या॒न् मद्ध्ये॑ पृ॒ष्ठानि॑ पृ॒ष्ठानि॒ मद्ध्ये᳚ स्याथ् स्या॒न् मद्ध्ये॑ पृ॒ष्ठानि॑ । \newline
16. मद्ध्ये॑ पृ॒ष्ठानि॑ पृ॒ष्ठानि॒ मद्ध्ये॒ मद्ध्ये॑ पृ॒ष्ठानि॑ भवन्ति भवन्ति पृ॒ष्ठानि॒ मद्ध्ये॒ मद्ध्ये॑ पृ॒ष्ठानि॑ भवन्ति । \newline
17. पृ॒ष्ठानि॑ भवन्ति भवन्ति पृ॒ष्ठानि॑ पृ॒ष्ठानि॑ भवन्ति सविवध॒त्वाय॑ सविवध॒त्वाय॑ भवन्ति पृ॒ष्ठानि॑ पृ॒ष्ठानि॑ भवन्ति सविवध॒त्वाय॑ । \newline
18. भ॒व॒न्ति॒ स॒वि॒व॒ध॒त्वाय॑ सविवध॒त्वाय॑ भवन्ति भवन्ति सविवध॒त्वा यौज॒ ओजः॑ सविवध॒त्वाय॑ भवन्ति भवन्ति सविवध॒त्वा यौजः॑ । \newline
19. स॒वि॒व॒ध॒त्वा यौज॒ ओजः॑ सविवध॒त्वाय॑ सविवध॒त्वा यौजो॒ वै वा ओजः॑ सविवध॒त्वाय॑ सविवध॒त्वा यौजो॒ वै । \newline
20. स॒वि॒व॒ध॒त्वायेति॑ सविवध - त्वाय॑ । \newline
21. ओजो॒ वै वा ओज॒ ओजो॒ वै वी॒र्यं॑ ॅवी॒र्यं॑ ॅवा ओज॒ ओजो॒ वै वी॒र्य᳚म् । \newline
22. वै वी॒र्यं॑ ॅवी॒र्यं॑ ॅवै वै वी॒र्य॑म् पृ॒ष्ठानि॑ पृ॒ष्ठानि॑ वी॒र्यं॑ ॅवै वै वी॒र्य॑म् पृ॒ष्ठानि॑ । \newline
23. वी॒र्य॑म् पृ॒ष्ठानि॑ पृ॒ष्ठानि॑ वी॒र्यं॑ ॅवी॒र्य॑म् पृ॒ष्ठा न्योज॒ ओजः॑ पृ॒ष्ठानि॑ वी॒र्यं॑ ॅवी॒र्य॑म् पृ॒ष्ठा न्योजः॑ । \newline
24. पृ॒ष्ठा न्योज॒ ओजः॑ पृ॒ष्ठानि॑ पृ॒ष्ठा न्योज॑ ए॒वै वौजः॑ पृ॒ष्ठानि॑ पृ॒ष्ठा न्योज॑ ए॒व । \newline
25. ओज॑ ए॒वै वौज॒ ओज॑ ए॒व वी॒र्यं॑ ॅवी॒र्य॑ मे॒वौज॒ ओज॑ ए॒व वी॒र्य᳚म् । \newline
26. ए॒व वी॒र्यं॑ ॅवी॒र्य॑ मे॒वैव वी॒र्य॑म् मद्ध्य॒तो म॑द्ध्य॒तो वी॒र्य॑ मे॒वैव वी॒र्य॑म् मद्ध्य॒तः । \newline
27. वी॒र्य॑म् मद्ध्य॒तो म॑द्ध्य॒तो वी॒र्यं॑ ॅवी॒र्य॑म् मद्ध्य॒तो द॑धते दधते मद्ध्य॒तो वी॒र्यं॑ ॅवी॒र्य॑म् मद्ध्य॒तो द॑धते । \newline
28. म॒द्ध्य॒तो द॑धते दधते मद्ध्य॒तो म॑द्ध्य॒तो द॑धते बृहद्रथन्त॒राभ्या᳚म् बृहद्रथन्त॒राभ्या᳚म् दधते मद्ध्य॒तो म॑द्ध्य॒तो द॑धते बृहद्रथन्त॒राभ्या᳚म् । \newline
29. द॒ध॒ते॒ बृ॒ह॒द्र॒थ॒न्त॒राभ्या᳚म् बृहद्रथन्त॒राभ्या᳚म् दधते दधते बृहद्रथन्त॒राभ्यां᳚ ॅयन्ति यन्ति बृहद्रथन्त॒राभ्या᳚म् दधते दधते बृहद्रथन्त॒राभ्यां᳚ ॅयन्ति । \newline
30. बृ॒ह॒द्र॒थ॒न्त॒राभ्यां᳚ ॅयन्ति यन्ति बृहद्रथन्त॒राभ्या᳚म् बृहद्रथन्त॒राभ्यां᳚ ॅयन्ती॒य मि॒यं ॅय॑न्ति बृहद्रथन्त॒राभ्या᳚म् बृहद्रथन्त॒राभ्यां᳚ ॅयन्ती॒यम् । \newline
31. बृ॒ह॒द्र॒थ॒न्त॒राभ्या॒मिति॑ बृहत् - र॒थ॒न्त॒राभ्या᳚म् । \newline
32. य॒न्ती॒य मि॒यं ॅय॑न्ति यन्ती॒यं ॅवाव वावेयं ॅय॑न्ति यन्ती॒यं ॅवाव । \newline
33. इ॒यं ॅवाव वावेय मि॒यं ॅवाव र॑थन्त॒रꣳ र॑थन्त॒रं ॅवावेय मि॒यं ॅवाव र॑थन्त॒रम् । \newline
34. वाव र॑थन्त॒रꣳ र॑थन्त॒रं ॅवाव वाव र॑थन्त॒र म॒सा व॒सौ र॑थन्त॒रं ॅवाव वाव र॑थन्त॒र म॒सौ । \newline
35. र॒थ॒न्त॒र म॒सा व॒सौ र॑थन्त॒रꣳ र॑थन्त॒र म॒सौ बृ॒हद् बृ॒ह द॒सौ र॑थन्त॒रꣳ र॑थन्त॒र म॒सौ बृ॒हत् । \newline
36. र॒थ॒न्त॒रमिति॑ रथं - त॒रम् । \newline
37. अ॒सौ बृ॒हद् बृ॒ह द॒सा व॒सौ बृ॒ह दा॒भ्या मा॒भ्याम् बृ॒ह द॒सा व॒सौ बृ॒ह दा॒भ्याम् । \newline
38. बृ॒ह दा॒भ्या मा॒भ्याम् बृ॒हद् बृ॒ह दा॒भ्या मे॒वै वाभ्याम् बृ॒हद् बृ॒ह दा॒भ्या मे॒व । \newline
39. आ॒भ्या मे॒वै वाभ्या मा॒भ्या मे॒व य॑न्ति यन्त्ये॒वाभ्या मा॒भ्या मे॒व य॑न्ति । \newline
40. ए॒व य॑न्ति यन्त्ये॒वैव य॒न्त्यथो॒ अथो॑ यन्त्ये॒वैव य॒न्त्यथो᳚ । \newline
41. य॒न्त्यथो॒ अथो॑ यन्ति य॒न्त्यथो॑ अ॒नयो॑ र॒नयो॒ रथो॑ यन्ति य॒न्त्यथो॑ अ॒नयोः᳚ । \newline
42. अथो॑ अ॒नयो॑ र॒नयो॒ रथो॒ अथो॑ अ॒नयो॑ रे॒वैवा नयो॒ रथो॒ अथो॑ अ॒नयो॑ रे॒व । \newline
43. अथो॒ इत्यथो᳚ । \newline
44. अ॒नयो॑ रे॒वैवा नयो॑ र॒नयो॑ रे॒व प्रति॒ प्रत्ये॒वा नयो॑ र॒नयो॑ रे॒व प्रति॑ । \newline
45. ए॒व प्रति॒ प्रत्ये॒वैव प्रति॑ तिष्ठन्ति तिष्ठन्ति॒ प्रत्ये॒वैव प्रति॑ तिष्ठन्ति । \newline
46. प्रति॑ तिष्ठन्ति तिष्ठन्ति॒ प्रति॒ प्रति॑ तिष्ठ न्त्ये॒ते ए॒ते ति॑ष्ठन्ति॒ प्रति॒ प्रति॑ तिष्ठ न्त्ये॒ते । \newline
47. ति॒ष्ठ॒ न्त्ये॒ते ए॒ते ति॑ष्ठन्ति तिष्ठ न्त्ये॒ते वै वा ए॒ते ति॑ष्ठन्ति तिष्ठ न्त्ये॒ते वै । \newline
48. ए॒ते वै वा ए॒ते ए॒ते वै य॒ज्ञ्स्य॑ य॒ज्ञ्स्य॒ वा ए॒ते ए॒ते वै य॒ज्ञ्स्य॑ । \newline
49. ए॒ते इत्ये॒ते । \newline
50. वै य॒ज्ञ्स्य॑ य॒ज्ञ्स्य॒ वै वै य॒ज्ञ्स्या᳚ ञ्ज॒साय॑नी अञ्ज॒साय॑नी य॒ज्ञ्स्य॒ वै वै य॒ज्ञ्स्या᳚ ञ्ज॒साय॑नी । \newline
51. य॒ज्ञ्स्या᳚ ञ्ज॒साय॑नी अञ्ज॒साय॑नी य॒ज्ञ्स्य॑ य॒ज्ञ्स्या᳚ ञ्ज॒साय॑नी स्रु॒ती स्रु॒ती अ॑ञ्ज॒साय॑नी य॒ज्ञ्स्य॑ य॒ज्ञ्स्या᳚ ञ्ज॒साय॑नी स्रु॒ती । \newline
52. अ॒ञ्ज॒साय॑नी स्रु॒ती स्रु॒ती अ॑ञ्ज॒साय॑नी अञ्ज॒साय॑नी स्रु॒ती ताभ्या॒म् ताभ्याꣳ॑ स्रु॒ती अ॑ञ्ज॒साय॑नी अञ्ज॒साय॑नी स्रु॒ती ताभ्या᳚म् । \newline
53. अ॒ञ्ज॒साय॑नी॒ इत्य॑ञ्जसा - अय॑नी । \newline
54. स्रु॒ती ताभ्या॒म् ताभ्याꣳ॑ स्रु॒ती स्रु॒ती ताभ्या॑ मे॒वैव ताभ्याꣳ॑ स्रु॒ती स्रु॒ती ताभ्या॑ मे॒व । \newline
55. स्रु॒ती इति॑ स्रु॒ती । \newline
56. ताभ्या॑ मे॒वैव ताभ्या॒म् ताभ्या॑ मे॒व सु॑व॒र्गꣳ सु॑व॒र्ग मे॒व ताभ्या॒म् ताभ्या॑ मे॒व सु॑व॒र्गम् । \newline
57. ए॒व सु॑व॒र्गꣳ सु॑व॒र्ग मे॒वैव सु॑व॒र्गम् ॅलो॒कम् ॅलो॒कꣳ सु॑व॒र्ग मे॒वैव सु॑व॒र्गम् ॅलो॒कम् । \newline
58. सु॒व॒र्गम् ॅलो॒कम् ॅलो॒कꣳ सु॑व॒र्गꣳ सु॑व॒र्गम् ॅलो॒कं ॅय॑न्ति यन्ति लो॒कꣳ सु॑व॒र्गꣳ सु॑व॒र्गम् ॅलो॒कं ॅय॑न्ति । \newline
59. सु॒व॒र्गमिति॑ सुवः - गम् । \newline
60. लो॒कं ॅय॑न्ति यन्ति लो॒कम् ॅलो॒कं ॅय॑न्ति॒ परा᳚ञ्चः॒ परा᳚ञ्चो यन्ति लो॒कम् ॅलो॒कं ॅय॑न्ति॒ परा᳚ञ्चः । \newline
\pagebreak
\markright{ TS 7.3.7.4  \hfill https://www.vedavms.in \hfill}

\section{ TS 7.3.7.4 }

\textbf{TS 7.3.7.4 } \newline
\textbf{Samhita Paata} \newline

ॅय॑न्ति॒ परा᳚ञ्चो॒ वा ए॒ते सु॑व॒र्गं ॅलो॒कम॒भ्यारो॑हन्ति॒ ये प॑रा॒चीना॑नि पृ॒ष्ठान्यु॑प॒यन्ति॑ प्र॒त्यङ् त्र्य॒हो भ॑वति प्र॒त्यव॑रूढ्या॒ अथो॒ प्रति॑ष्ठित्या उ॒भयो᳚र्लो॒कयोर्॑ ऋ॒द्ध्वोत् ति॑ष्ठन्ति॒ पञ्च॑दशै॒तास्तासां॒ ॅया दश॒ दशा᳚क्षरा वि॒राडन्नं॑ ॅवि॒राड् वि॒राजै॒वान्नाद्य॒मव॑ रुन्धते॒ याः पञ्च॒ पञ्च॒ दिशो॑ दि॒क्ष्वे॑व प्रति॑ तिष्ठन्त्यतिरा॒त्राव॒भितो॑ भवत इन्द्रि॒यस्य॑ वी॒र्य॑स्य प्र॒जायै॑ ( ) पशू॒नां परि॑गृहीत्यै ॥ \newline

\textbf{Pada Paata} \newline

य॒न्ति॒ । परा᳚ञ्चः । वै । ए॒ते । सु॒व॒र्गमिति॑ सुवः - गम् । लो॒कम् । अ॒भ्यारो॑ह॒न्तीत्य॑भि - आरो॑हन्ति । ये । प॒रा॒चीना॑नि । पृ॒ष्ठानि॑ । उ॒प॒यन्तीत्यु॑प - यन्ति॑ । प्र॒त्यङ् । त्र्य॒ह इति॑ त्रि - अ॒हः । भ॒व॒ति॒ । प्र॒त्यव॑रूढ्या॒ इति॑ प्रति - अव॑रूढ्यै । अथो॒ इति॑ । प्रति॑ष्ठित्या॒ इति॒ प्रति॑ - स्थि॒त्यै॒ । उ॒भयोः᳚ । लो॒कयोः᳚ । ऋ॒द्ध्वा । उदिति॑ । ति॒ष्ठ॒न्ति॒ । पञ्च॑द॒शेति॒ पञ्च॑ - द॒श॒ । ए॒ताः । तासा᳚म् । याः । दश॑ । दशा᳚क्ष॒रेति॒ दश॑ - अ॒क्ष॒रा॒ । वि॒राडिति॑ वि - राट् । अन्न᳚म् । वि॒राडिति॑ वि - राट् । वि॒राजेति॑ वि - राजा᳚ । ए॒व । अ॒न्नाद्य॒मित्य॑न्न - अद्य᳚म् । अवेति॑ । रु॒न्ध॒ते॒ । याः । पञ्च॑ । पञ्च॑ । दिशः॑ । दि॒क्षु । ए॒व । प्रतीति॑ । ति॒ष्ठ॒न्ति॒ । अ॒ति॒रा॒त्रावित्य॑ति - रा॒त्रौ । अ॒भितः॑ । भ॒व॒तः॒ । इ॒न्द्रि॒यस्य॑ । वी॒र्य॑स्य । प्र॒जाया॒ इति॑ प्र - जायै᳚ ( ) । प॒शू॒नाम् । परि॑गृहीत्या॒ इति॒ परि॑ - गृ॒ही॒त्यै॒ ॥  \newline


\textbf{Krama Paata} \newline

य॒न्ति॒ परा᳚ञ्चः । परा᳚ञ्चो॒ वै । वा ए॒ते । ए॒ते सु॑व॒र्गम् । सु॒व॒र्गम् ॅलो॒कम् । सु॒व॒र्ग॒मिति॑ सुवः - गम् । लो॒कम॒भ्यारो॑हन्ति । अ॒भ्यारो॑हन्ति॒ ये । अ॒भ्यारो॑ह॒न्तीत्य॑भि - आरो॑हन्ति । ये प॑रा॒चीना॑नि । प॒रा॒चीना॑नि पृ॒ष्ठानि॑ । पृ॒ष्ठान्यु॑प॒यन्ति॑ । उ॒प॒यन्ति॑ प्र॒त्यङ्‍ङ् । उ॒प॒यन्तीत्यु॑प - यन्ति॑ । प्र॒त्यङ्‍ त्र्य॒हः । त्र्य॒हो भ॑वति । त्र्य॒ह इति॑ त्रि - अ॒हः । भ॒व॒ति॒ प्र॒त्यव॑रूढ्यै । प्र॒त्यव॑रूढ्या॒ अथो᳚ । प्र॒त्यव॑रूढ्या॒ इति॑ प्रति - अव॑रूढ्यै । अथो॒ प्रति॑ष्ठित्यै । अथो॒ इत्यथो᳚ । प्रति॑ष्ठित्या उ॒भयोः᳚ । प्रति॑ष्ठित्या॒ इति॒ प्रति॑ - स्थि॒त्यै॒ । उ॒भयो᳚र् लो॒कयोः᳚ । लो॒कयोर्॑. ऋ॒द्ध्वा । ऋ॒द्ध्वोत् । उत् ति॑ष्ठन्ति । ति॒ष्ठ॒न्ति॒ पञ्च॑दश । पञ्च॑दशै॒ताः । पञ्च॑द॒शेति॒ पञ्च॑ - द॒श॒ । ए॒तास्तासा᳚म् । तासा॒म् ॅयाः । या दश॑ । दश॒ दशा᳚क्षरा । दशा᳚क्षरा वि॒राट् । दशा᳚क्ष॒रेति॒ दश॑ - अ॒क्ष॒रा॒ । वि॒राडन्न᳚म् । वि॒राडिति॑ वि - राट् । अन्न॑म् ॅवि॒राट् । वि॒राड् वि॒राजा᳚ । वि॒राडिति॑ वि - राट् । वि॒राजै॒व । वि॒राजेति॑ वि - राजा᳚ । ए॒वान्नाद्य᳚म् । अ॒न्नाद्य॒मव॑ । अ॒न्नाद्य॒मित्य॑न्न - अद्य᳚म् । अव॑ रुन्धते । रु॒न्ध॒ते॒ याः । याः पञ्च॑ । पञ्च॒ पञ्च॑ । पञ्च॒ दिशः॑ । दिशो॑ दि॒क्षु । दि॒क्ष्वे॑व । ए॒व प्रति॑ । प्रति॑ तिष्ठन्ति । ति॒ष्ठ॒न्त्य॒ति॒रा॒त्रौ । अ॒ति॒रा॒त्राव॒भितः॑ । अ॒ति॒रा॒त्रावित्य॑ति - रा॒त्रौ । अ॒भितो॑ भवतः । भ॒व॒त॒ इ॒न्द्रि॒यस्य॑ । इ॒न्द्रि॒यस्य॑ वी॒र्य॑स्य । वी॒र्य॑स्य प्र॒जायै᳚ ( ) । प्र॒जायै॑ पशू॒नाम् । प्र॒जाया॒ इति॑ प्र - जायै᳚ । प॒शू॒नाम् परि॑गृहीत्यै । परि॑गृहीत्या॒ इति॒ परि॑ - गृ॒ही॒त्यै॒ । \newline

\textbf{Jatai Paata} \newline

1. य॒न्ति॒ परा᳚ञ्चः॒ परा᳚ञ्चो यन्ति यन्ति॒ परा᳚ञ्चः । \newline
2. परा᳚ञ्चो॒ वै वै परा᳚ञ्चः॒ परा᳚ञ्चो॒ वै । \newline
3. वा ए॒त ए॒ते वै वा ए॒ते । \newline
4. ए॒ते सु॑व॒र्गꣳ सु॑व॒र्ग मे॒त ए॒ते सु॑व॒र्गम् । \newline
5. सु॒व॒र्गम् ॅलो॒कम् ॅलो॒कꣳ सु॑व॒र्गꣳ सु॑व॒र्गम् ॅलो॒कम् । \newline
6. सु॒व॒र्गमिति॑ सुवः - गम् । \newline
7. लो॒क म॒भ्यारो॑ह न्त्य॒भ्यारो॑हन्ति लो॒कम् ॅलो॒क म॒भ्यारो॑हन्ति । \newline
8. अ॒भ्यारो॑हन्ति॒ ये ये᳚ ऽभ्यारो॑ह न्त्य॒भ्यारो॑हन्ति॒ ये । \newline
9. अ॒भ्यारो॑ह॒न्तीत्य॑भि - आरो॑हन्ति । \newline
10. ये प॑रा॒चीना॑नि परा॒चीना॑नि॒ ये ये प॑रा॒चीना॑नि । \newline
11. प॒रा॒चीना॑नि पृ॒ष्ठानि॑ पृ॒ष्ठानि॑ परा॒चीना॑नि परा॒चीना॑नि पृ॒ष्ठानि॑ । \newline
12. पृ॒ष्ठा न्यु॑प॒य न्त्यु॑प॒यन्ति॑ पृ॒ष्ठानि॑ पृ॒ष्ठा न्यु॑प॒यन्ति॑ । \newline
13. उ॒प॒यन्ति॑ प्र॒त्यङ् प्र॒त्यङ् ङु॑प॒य न्त्यु॑प॒यन्ति॑ प्र॒त्यङ् । \newline
14. उ॒प॒यन्तीत्यु॑प - यन्ति॑ । \newline
15. प्र॒त्यङ् त्र्य॒ह स्त्र्य॒हः प्र॒त्यङ् प्र॒त्यङ् त्र्य॒हः । \newline
16. त्र्य॒हो भ॑वति भवति त्र्य॒ह स्त्र्य॒हो भ॑वति । \newline
17. त्र्य॒ह इति॑ त्रि - अ॒हः । \newline
18. भ॒व॒ति॒ प्र॒त्यव॑रूढ्यै प्र॒त्यव॑रूढ्यै भवति भवति प्र॒त्यव॑रूढ्यै । \newline
19. प्र॒त्यव॑रूढ्या॒ अथो॒ अथो᳚ प्र॒त्यव॑रूढ्यै प्र॒त्यव॑रूढ्या॒ अथो᳚ । \newline
20. प्र॒त्यव॑रूढ्या॒ इति॑ प्रति - अव॑रूढ्यै । \newline
21. अथो॒ प्रति॑ष्ठित्यै॒ प्रति॑ष्ठित्या॒ अथो॒ अथो॒ प्रति॑ष्ठित्यै । \newline
22. अथो॒ इत्यथो᳚ । \newline
23. प्रति॑ष्ठित्या उ॒भयो॑ रु॒भयोः॒ प्रति॑ष्ठित्यै॒ प्रति॑ष्ठित्या उ॒भयोः᳚ । \newline
24. प्रति॑ष्ठित्या॒ इति॒ प्रति॑ - स्थि॒त्यै॒ । \newline
25. उ॒भयो᳚र् लो॒कयो᳚र् लो॒कयो॑ रु॒भयो॑ रु॒भयो᳚र् लो॒कयोः᳚ । \newline
26. लो॒कयोर्॑. ऋ॒द्ध्व र्‌द्ध्वा लो॒कयो᳚र् लो॒कयोर्॑. ऋ॒द्ध्वा । \newline
27. ऋ॒द्ध्वो दुदृ॒द्ध्व र्‌द्ध्वोत् । \newline
28. उत् ति॑ष्ठन्ति तिष्ठ॒ न्त्युदुत् ति॑ष्ठन्ति । \newline
29. ति॒ष्ठ॒न्ति॒ पञ्च॑दश॒ पञ्च॑दश तिष्ठन्ति तिष्ठन्ति॒ पञ्च॑दश । \newline
30. पञ्च॑दशै॒ता ए॒ताः पञ्च॑दश॒ पञ्च॑दशै॒ताः । \newline
31. पञ्च॑द॒शेति॒ पञ्च॑ - द॒श॒ । \newline
32. ए॒ता स्तासा॒म् तासा॑ मे॒ता ए॒ता स्तासा᳚म् । \newline
33. तासां॒ ॅया या स्तासा॒म् तासां॒ ॅयाः । \newline
34. या दश॒ दश॒ या या दश॑ । \newline
35. दश॒ दशा᳚क्षरा॒ दशा᳚क्षरा॒ दश॒ दश॒ दशा᳚क्षरा । \newline
36. दशा᳚क्षरा वि॒राड् वि॒राड् दशा᳚क्षरा॒ दशा᳚क्षरा वि॒राट् । \newline
37. दशा᳚क्ष॒रेति॒ दश॑ - अ॒क्ष॒रा॒ । \newline
38. वि॒रा डन्न॒ मन्नं॑ ॅवि॒राड् वि॒रा डन्न᳚म् । \newline
39. वि॒राडिति॑ वि - राट् । \newline
40. अन्नं॑ ॅवि॒राड् वि॒रा डन्न॒ मन्नं॑ ॅवि॒राट् । \newline
41. वि॒राड् वि॒राजा॑ वि॒राजा॑ वि॒राड् वि॒राड् वि॒राजा᳚ । \newline
42. वि॒राडिति॑ वि - राट् । \newline
43. वि॒राजै॒वैव वि॒राजा॑ वि॒राजै॒व । \newline
44. वि॒राजेति॑ वि - राजा᳚ । \newline
45. ए॒वा न्नाद्य॑ म॒न्नाद्य॑ मे॒वैवा न्नाद्य᳚म् । \newline
46. अ॒न्नाद्य॒ मवावा॒ न्नाद्य॑ म॒न्नाद्य॒ मव॑ । \newline
47. अ॒न्नाद्य॒मित्य॑न्न - अद्य᳚म् । \newline
48. अव॑ रुन्धते रुन्ध॒ते ऽवाव॑ रुन्धते । \newline
49. रु॒न्ध॒ते॒ या या रु॑न्धते रुन्धते॒ याः । \newline
50. याः पञ्च॒ पञ्च॒ या याः पञ्च॑ । \newline
51. पञ्च॒ पञ्च॑ । \newline
52. पञ्च॒ दिशो॒ दिशः॒ पञ्च॒ पञ्च॒ दिशः॑ । \newline
53. दिशो॑ दि॒क्षु दि॒क्षु दिशो॒ दिशो॑ दि॒क्षु । \newline
54. दि॒क्ष्वे॑वैव दि॒क्षु दि॒क्ष्वे॑व । \newline
55. ए॒व प्रति॒ प्रत्ये॒वैव प्रति॑ । \newline
56. प्रति॑ तिष्ठन्ति तिष्ठन्ति॒ प्रति॒ प्रति॑ तिष्ठन्ति । \newline
57. ति॒ष्ठ॒ न्त्य॒ति॒रा॒त्रा व॑तिरा॒त्रौ ति॑ष्ठन्ति तिष्ठ न्त्यतिरा॒त्रौ । \newline
58. अ॒ति॒रा॒त्रा व॒भितो॒ ऽभितो॑ ऽतिरा॒त्रा व॑तिरा॒त्रा व॒भितः॑ । \newline
59. अ॒ति॒रा॒त्रावित्य॑ति - रा॒त्रौ । \newline
60. अ॒भितो॑ भवतो भवतो॒ ऽभितो॒ ऽभितो॑ भवतः । \newline
61. भ॒व॒त॒ इ॒न्द्रि॒य स्ये᳚न्द्रि॒यस्य॑ भवतो भवत इन्द्रि॒यस्य॑ । \newline
62. इ॒न्द्रि॒यस्य॑ वी॒र्य॑स्य वी॒र्य॑ स्येन्द्रि॒य स्ये᳚न्द्रि॒यस्य॑ वी॒र्य॑स्य । \newline
63. वी॒र्य॑स्य प्र॒जायै᳚ प्र॒जायै॑ वी॒र्य॑स्य वी॒र्य॑स्य प्र॒जायै᳚ । \newline
64. प्र॒जायै॑ पशू॒नाम् प॑शू॒नाम् प्र॒जायै᳚ प्र॒जायै॑ पशू॒नाम् । \newline
65. प्र॒जाया॒ इति॑ प्र - जायै᳚ । \newline
66. प॒शू॒नाम् परि॑गृहीत्यै॒ परि॑गृहीत्यै पशू॒नाम् प॑शू॒नाम् परि॑गृहीत्यै । \newline
67. परि॑गृहीत्या॒ इति॒ परि॑ - गृ॒ही॒त्यै॒ । \newline

\textbf{Ghana Paata } \newline

1. य॒न्ति॒ परा᳚ञ्चः॒ परा᳚ञ्चो यन्ति यन्ति॒ परा᳚ञ्चो॒ वै वै परा᳚ञ्चो यन्ति यन्ति॒ परा᳚ञ्चो॒ वै । \newline
2. परा᳚ञ्चो॒ वै वै परा᳚ञ्चः॒ परा᳚ञ्चो॒ वा ए॒त ए॒ते वै परा᳚ञ्चः॒ परा᳚ञ्चो॒ वा ए॒ते । \newline
3. वा ए॒त ए॒ते वै वा ए॒ते सु॑व॒र्गꣳ सु॑व॒र्ग मे॒ते वै वा ए॒ते सु॑व॒र्गम् । \newline
4. ए॒ते सु॑व॒र्गꣳ सु॑व॒र्ग मे॒त ए॒ते सु॑व॒र्गम् ॅलो॒कम् ॅलो॒कꣳ सु॑व॒र्ग मे॒त ए॒ते सु॑व॒र्गम् ॅलो॒कम् । \newline
5. सु॒व॒र्गम् ॅलो॒कम् ॅलो॒कꣳ सु॑व॒र्गꣳ सु॑व॒र्गम् ॅलो॒क म॒भ्यारो॑ह न्त्य॒भ्यारो॑हन्ति लो॒कꣳ सु॑व॒र्गꣳ सु॑व॒र्गम् ॅलो॒क म॒भ्यारो॑हन्ति । \newline
6. सु॒व॒र्गमिति॑ सुवः - गम् । \newline
7. लो॒क म॒भ्यारो॑ह न्त्य॒भ्यारो॑हन्ति लो॒कम् ॅलो॒क म॒भ्यारो॑हन्ति॒ ये ये᳚ ऽभ्यारो॑हन्ति लो॒कम् ॅलो॒क म॒भ्यारो॑हन्ति॒ ये । \newline
8. अ॒भ्यारो॑हन्ति॒ ये ये᳚ ऽभ्यारो॑ह न्त्य॒भ्यारो॑हन्ति॒ ये प॑रा॒चीना॑नि परा॒चीना॑नि॒ ये᳚ ऽभ्यारो॑ह न्त्य॒भ्यारो॑हन्ति॒ ये प॑रा॒चीना॑नि । \newline
9. अ॒भ्यारो॑ह॒न्तीत्य॑भि - आरो॑हन्ति । \newline
10. ये प॑रा॒चीना॑नि परा॒चीना॑नि॒ ये ये प॑रा॒चीना॑नि पृ॒ष्ठानि॑ पृ॒ष्ठानि॑ परा॒चीना॑नि॒ ये ये प॑रा॒चीना॑नि पृ॒ष्ठानि॑ । \newline
11. प॒रा॒चीना॑नि पृ॒ष्ठानि॑ पृ॒ष्ठानि॑ परा॒चीना॑नि परा॒चीना॑नि पृ॒ष्ठा न्यु॑प॒य न्त्यु॑प॒यन्ति॑ पृ॒ष्ठानि॑ परा॒चीना॑नि परा॒चीना॑नि पृ॒ष्ठा न्यु॑प॒यन्ति॑ । \newline
12. पृ॒ष्ठा न्यु॑प॒य न्त्यु॑प॒यन्ति॑ पृ॒ष्ठानि॑ पृ॒ष्ठा न्यु॑प॒यन्ति॑ प्र॒त्यङ् प्र॒त्यङ् ङु॑प॒यन्ति॑ पृ॒ष्ठानि॑ पृ॒ष्ठा न्यु॑प॒यन्ति॑ प्र॒त्यङ् । \newline
13. उ॒प॒यन्ति॑ प्र॒त्यङ् प्र॒त्यङ् ङु॑प॒य न्त्यु॑प॒यन्ति॑ प्र॒त्यङ् त्र्य॒ह स्त्र्य॒हः प्र॒त्यङ् ङु॑प॒य न्त्यु॑प॒यन्ति॑ प्र॒त्यङ् त्र्य॒हः । \newline
14. उ॒प॒यन्तीत्यु॑प - यन्ति॑ । \newline
15. प्र॒त्यङ् त्र्य॒ह स्त्र्य॒हः प्र॒त्यङ् प्र॒त्यङ् त्र्य॒हो भ॑वति भवति त्र्य॒हः प्र॒त्यङ् प्र॒त्यङ् त्र्य॒हो भ॑वति । \newline
16. त्र्य॒हो भ॑वति भवति त्र्य॒ह स्त्र्य॒हो भ॑वति प्र॒त्यव॑रूढ्यै प्र॒त्यव॑रूढ्यै भवति त्र्य॒ह स्त्र्य॒हो भ॑वति प्र॒त्यव॑रूढ्यै । \newline
17. त्र्य॒ह इति॑ त्रि - अ॒हः । \newline
18. भ॒व॒ति॒ प्र॒त्यव॑रूढ्यै प्र॒त्यव॑रूढ्यै भवति भवति प्र॒त्यव॑रूढ्या॒ अथो॒ अथो᳚ प्र॒त्यव॑रूढ्यै भवति भवति प्र॒त्यव॑रूढ्या॒ अथो᳚ । \newline
19. प्र॒त्यव॑रूढ्या॒ अथो॒ अथो᳚ प्र॒त्यव॑रूढ्यै प्र॒त्यव॑रूढ्या॒ अथो॒ प्रति॑ष्ठित्यै॒ प्रति॑ष्ठित्या॒ अथो᳚ प्र॒त्यव॑रूढ्यै प्र॒त्यव॑रूढ्या॒ अथो॒ प्रति॑ष्ठित्यै । \newline
20. प्र॒त्यव॑रूढ्या॒ इति॑ प्रति - अव॑रूढ्यै । \newline
21. अथो॒ प्रति॑ष्ठित्यै॒ प्रति॑ष्ठित्या॒ अथो॒ अथो॒ प्रति॑ष्ठित्या उ॒भयो॑ रु॒भयोः॒ प्रति॑ष्ठित्या॒ अथो॒ अथो॒ प्रति॑ष्ठित्या उ॒भयोः᳚ । \newline
22. अथो॒ इत्यथो᳚ । \newline
23. प्रति॑ष्ठित्या उ॒भयो॑ रु॒भयोः॒ प्रति॑ष्ठित्यै॒ प्रति॑ष्ठित्या उ॒भयो᳚र् लो॒कयो᳚र् लो॒कयो॑ रु॒भयोः॒ प्रति॑ष्ठित्यै॒ प्रति॑ष्ठित्या उ॒भयो᳚र् लो॒कयोः᳚ । \newline
24. प्रति॑ष्ठित्या॒ इति॒ प्रति॑ - स्थि॒त्यै॒ । \newline
25. उ॒भयो᳚र् लो॒कयो᳚र् लो॒कयो॑ रु॒भयो॑ रु॒भयो᳚र् लो॒कयोर्॑. ऋ॒द्ध्व र्‌द्ध्वा लो॒कयो॑ रु॒भयो॑ रु॒भयो᳚र् लो॒कयोर्॑. ऋ॒द्ध्वा । \newline
26. लो॒कयोर्॑. ऋ॒द्ध्व र्‌द्ध्वा लो॒कयो᳚र् लो॒कयोर्॑. ऋ॒द्ध्वो दुदृ॒द्ध्वा लो॒कयो᳚र् लो॒कयोर्॑. ऋ॒द्ध्वोत् । \newline
27. ऋ॒द्ध्वो दुदृ॒द्ध्व र्‌द्ध्वोत् ति॑ष्ठन्ति तिष्ठ॒ न्त्युदृ॒द्ध्व र्‌द्ध्वोत् ति॑ष्ठन्ति । \newline
28. उत् ति॑ष्ठन्ति तिष्ठ॒ न्त्युदुत् ति॑ष्ठन्ति॒ पञ्च॑दश॒ पञ्च॑दश तिष्ठ॒ न्त्युदुत् ति॑ष्ठन्ति॒ पञ्च॑दश । \newline
29. ति॒ष्ठ॒न्ति॒ पञ्च॑दश॒ पञ्च॑दश तिष्ठन्ति तिष्ठन्ति॒ पञ्च॑दशै॒ता ए॒ताः पञ्च॑दश तिष्ठन्ति तिष्ठन्ति॒ पञ्च॑दशै॒ताः । \newline
30. पञ्च॑दशै॒ता ए॒ताः पञ्च॑दश॒ पञ्च॑दशै॒ता स्तासा॒म् तासा॑ मे॒ताः पञ्च॑दश॒ पञ्च॑दशै॒ता स्तासा᳚म् । \newline
31. पञ्च॑द॒शेति॒ पञ्च॑ - द॒श॒ । \newline
32. ए॒ता स्तासा॒म् तासा॑ मे॒ता ए॒ता स्तासां॒ ॅया या स्तासा॑ मे॒ता ए॒ता स्तासां॒ ॅयाः । \newline
33. तासां॒ ॅया या स्तासा॒म् तासां॒ ॅया दश॒ दश॒ या स्तासा॒म् तासां॒ ॅया दश॑ । \newline
34. या दश॒ दश॒ या या दश॒ दशा᳚क्षरा॒ दशा᳚क्षरा॒ दश॒ या या दश॒ दशा᳚क्षरा । \newline
35. दश॒ दशा᳚क्षरा॒ दशा᳚क्षरा॒ दश॒ दश॒ दशा᳚क्षरा वि॒राड् वि॒राड् दशा᳚क्षरा॒ दश॒ दश॒ दशा᳚क्षरा वि॒राट् । \newline
36. दशा᳚क्षरा वि॒राड् वि॒राड् दशा᳚क्षरा॒ दशा᳚क्षरा वि॒रा डन्न॒ मन्नं॑ ॅवि॒राड् दशा᳚क्षरा॒ दशा᳚क्षरा वि॒रा डन्न᳚म् । \newline
37. दशा᳚क्ष॒रेति॒ दश॑ - अ॒क्ष॒रा॒ । \newline
38. वि॒रा डन्न॒ मन्नं॑ ॅवि॒राड् वि॒रा डन्नं॑ ॅवि॒राड् वि॒रा डन्नं॑ ॅवि॒राड् वि॒रा डन्नं॑ ॅवि॒राट् । \newline
39. वि॒राडिति॑ वि - राट् । \newline
40. अन्नं॑ ॅवि॒राड् वि॒रा डन्न॒ मन्नं॑ ॅवि॒राड् वि॒राजा॑ वि॒राजा॑ वि॒रा डन्न॒ मन्नं॑ ॅवि॒राड् वि॒राजा᳚ । \newline
41. वि॒राड् वि॒राजा॑ वि॒राजा॑ वि॒राड् वि॒राड् वि॒राजै॒वैव वि॒राजा॑ वि॒राड् वि॒राड् वि॒राजै॒व । \newline
42. वि॒राडिति॑ वि - राट् । \newline
43. वि॒राजै॒वैव वि॒राजा॑ वि॒राजै॒ वान्नाद्य॑ म॒न्नाद्य॑ मे॒व वि॒राजा॑ वि॒राजै॒ वान्नाद्य᳚म् । \newline
44. वि॒राजेति॑ वि - राजा᳚ । \newline
45. ए॒वान्नाद्य॑ म॒न्नाद्य॑ मे॒वै वान्नाद्य॒ मवा वा॒न्नाद्य॑ मे॒वै वान्नाद्य॒ मव॑ । \newline
46. अ॒न्नाद्य॒ मवा वा॒न्नाद्य॑ म॒न्नाद्य॒ मव॑ रुन्धते रुन्ध॒ते ऽवा॒न्नाद्य॑ म॒न्नाद्य॒ मव॑ रुन्धते । \newline
47. अ॒न्नाद्य॒मित्य॑न्न - अद्य᳚म् । \newline
48. अव॑ रुन्धते रुन्ध॒ते ऽवाव॑ रुन्धते॒ या या रु॑न्ध॒ते ऽवाव॑ रुन्धते॒ याः । \newline
49. रु॒न्ध॒ते॒ या या रु॑न्धते रुन्धते॒ याः पञ्च॒ पञ्च॒ या रु॑न्धते रुन्धते॒ याः पञ्च॑ । \newline
50. याः पञ्च॒ पञ्च॒ या याः पञ्च॑ । \newline
51. पञ्च॒ पञ्च॑ । \newline
52. पञ्च॒ दिशो॒ दिशः॒ पञ्च॒ पञ्च॒ दिशो॑ दि॒क्षु दि॒क्षु दिशः॒ पञ्च॒ पञ्च॒ दिशो॑ दि॒क्षु । \newline
53. दिशो॑ दि॒क्षु दि॒क्षु दिशो॒ दिशो॑ दि॒क्ष्वे॑वैव दि॒क्षु दिशो॒ दिशो॑ दि॒क्ष्वे॑व । \newline
54. दि॒क्ष्वे॑वैव दि॒क्षु दि॒क्ष्वे॑व प्रति॒ प्रत्ये॒व दि॒क्षु दि॒क्ष्वे॑व प्रति॑ । \newline
55. ए॒व प्रति॒ प्रत्ये॒वैव प्रति॑ तिष्ठन्ति तिष्ठन्ति॒ प्रत्ये॒ वैव प्रति॑ तिष्ठन्ति । \newline
56. प्रति॑ तिष्ठन्ति तिष्ठन्ति॒ प्रति॒ प्रति॑ तिष्ठ न्त्यतिरा॒त्रा व॑तिरा॒त्रौ ति॑ष्ठन्ति॒ प्रति॒ प्रति॑ तिष्ठ न्त्यतिरा॒त्रौ । \newline
57. ति॒ष्ठ॒ न्त्य॒ति॒रा॒त्रा व॑तिरा॒त्रौ ति॑ष्ठन्ति तिष्ठ न्त्यतिरा॒त्रा व॒भितो॒ ऽभितो॑ ऽतिरा॒त्रौ ति॑ष्ठन्ति तिष्ठ न्त्यतिरा॒त्रा व॒भितः॑ । \newline
58. अ॒ति॒रा॒त्रा व॒भितो॒ ऽभितो॑ ऽतिरा॒त्रा व॑तिरा॒त्रा व॒भितो॑ भवतो भवतो॒ ऽभितो॑ ऽतिरा॒त्रा व॑तिरा॒त्रा व॒भितो॑ भवतः । \newline
59. अ॒ति॒रा॒त्रावित्य॑ति - रा॒त्रौ । \newline
60. अ॒भितो॑ भवतो भवतो॒ ऽभितो॒ ऽभितो॑ भवत इन्द्रि॒य स्ये᳚न्द्रि॒यस्य॑ भवतो॒ ऽभितो॒ ऽभितो॑ भवत इन्द्रि॒यस्य॑ । \newline
61. भ॒व॒त॒ इ॒न्द्रि॒य स्ये᳚न्द्रि॒यस्य॑ भवतो भवत इन्द्रि॒यस्य॑ वी॒र्य॑स्य वी॒र्य॑ स्येन्द्रि॒यस्य॑ भवतो भवत इन्द्रि॒यस्य॑ वी॒र्य॑स्य । \newline
62. इ॒न्द्रि॒यस्य॑ वी॒र्य॑स्य वी॒र्य॑ स्येन्द्रि॒य स्ये᳚न्द्रि॒यस्य॑ वी॒र्य॑स्य प्र॒जायै᳚ प्र॒जायै॑ वी॒र्य॑ स्येन्द्रि॒य
स्ये᳚न्द्रि॒यस्य॑ वी॒र्य॑स्य प्र॒जायै᳚ । \newline
63. वी॒र्य॑स्य प्र॒जायै᳚ प्र॒जायै॑ वी॒र्य॑स्य वी॒र्य॑स्य प्र॒जायै॑ पशू॒नाम् प॑शू॒नाम् प्र॒जायै॑ वी॒र्य॑स्य वी॒र्य॑स्य प्र॒जायै॑ पशू॒नाम् । \newline
64. प्र॒जायै॑ पशू॒नाम् प॑शू॒नाम् प्र॒जायै᳚ प्र॒जायै॑ पशू॒नाम् परि॑गृहीत्यै॒ परि॑गृहीत्यै पशू॒नाम् प्र॒जायै᳚ प्र॒जायै॑ पशू॒नाम् परि॑गृहीत्यै । \newline
65. प्र॒जाया॒ इति॑ प्र - जायै᳚ । \newline
66. प॒शू॒नाम् परि॑गृहीत्यै॒ परि॑गृहीत्यै पशू॒नाम् प॑शू॒नाम् परि॑गृहीत्यै । \newline
67. परि॑गृहीत्या॒ इति॒ परि॑ - गृ॒ही॒त्यै॒ । \newline
\pagebreak
\markright{ TS 7.3.8.1  \hfill https://www.vedavms.in \hfill}

\section{ TS 7.3.8.1 }

\textbf{TS 7.3.8.1 } \newline
\textbf{Samhita Paata} \newline

प्र॒जाप॑ति-रकामयतान्ना॒दः स्या॒मिति॒ स ए॒तꣳ स॑प्तदशरा॒त्र-म॑पश्य॒त् तमाऽह॑र॒त् तेना॑यजत॒ ततो॒ वै सो᳚ऽन्ना॒दो॑ऽभव॒द्य ए॒वं ॅवि॒द्वाꣳसः॑ सप्तदश-रा॒त्रमास॑ते ऽन्ना॒दा ए॒व भ॑वन्ति पञ्चा॒हो भ॑वति॒ पञ्च॒ वा ऋ॒तवः॑ संॅवथ्स॒र ऋ॒तुष्वे॒व सं॑ॅवथ्स॒रे प्रति॑ तिष्ठ॒न्त्यथो॒ पञ्चा᳚क्षरा प॒ङ्क्तिः पाङ्क्तो॑ य॒ज्ञो य॒ज्ञ्मे॒वाऽव॑ रुन्ध॒ते ऽस॑त्रं॒ ॅवा ए॒त- [  ] \newline

\textbf{Pada Paata} \newline

प्र॒जाप॑ति॒रिति॑ प्र॒जा - प॒तिः॒ । अ॒का॒म॒य॒त॒ । अ॒न्ना॒द इत्य॑न्न - अ॒दः । स्या॒म् । इति॑ । सः । ए॒तम् । स॒प्त॒द॒श॒रा॒त्रमिति॑ सप्तदश - रा॒त्रम् । अ॒प॒श्य॒त् । तम् । एति॑ । अ॒ह॒र॒त् । तेन॑ । अ॒य॒ज॒त॒ । ततः॑ । वै । सः । अ॒न्ना॒द इत्य॑न्न - अ॒दः । अ॒भ॒व॒त् । ये । ए॒वम् । वि॒द्वाꣳसः॑ । स॒प्त॒द॒श॒रा॒त्रमिति॑ सप्तदश-रा॒त्रम् । आस॑ते । अ॒न्ना॒दा इत्य॑न्न-अ॒दाः । ए॒व । भ॒व॒न्ति॒ । प॒ञ्चा॒ह इति॑ पञ्च - अ॒हः । भ॒व॒ति॒ । पञ्च॑ । वै । ऋ॒तवः॑ । सं॒ॅव॒थ्स॒र इति॑ सं - व॒थ्स॒रः । ऋ॒तुषु॑ । ए॒व । सं॒ॅव॒थ्स॒र इति॑ सं - व॒थ्स॒रे । प्रतीति॑ । ति॒ष्ठ॒न्ति॒ । अथा॒ इति॑ । पञ्चा᳚क्ष॒रेति॒ पञ्च॑ - अ॒क्ष॒रा॒ । प॒ङ्क्तिः । पाङ्क्तः॑ । य॒ज्ञ्ः । य॒ज्ञ्म् । ए॒व । अवेति॑ । रु॒न्ध॒ते॒ । अस॑त्रम् । वै । ए॒तत् ।  \newline


\textbf{Krama Paata} \newline

प्र॒जाप॑तिरकामयत । प्र॒जाप॑ति॒रिति॑ प्र॒जा - प॒तिः॒ । अ॒का॒म॒य॒ता॒न्ना॒दः । अ॒न्ना॒दः स्या᳚म् । अ॒न्ना॒द इत्य॑न्न - अ॒दः । स्या॒मिति॑ । इति॒ सः । स ए॒तम् । ए॒तꣳ स॑प्तदशरा॒त्रम् । स॒प्त॒द॒श॒रा॒त्रम॑पश्यत् । स॒प्त॒द॒श॒रा॒त्रमिति॑ सप्तदश - रा॒त्रम् । अ॒प॒श्य॒त् तम् । तमा । आऽह॑रत् । अ॒ह॒र॒त् तेन॑ । तेना॑यजत । अ॒य॒ज॒त॒ ततः॑ । ततो॒ वै । वै सः । सो᳚ऽन्ना॒दः । अ॒न्ना॒दो॑ऽभवत् । अ॒न्ना॒द इत्य॑न्न - अ॒दः । अ॒भ॒व॒द् ये । य ए॒वम् । ए॒वम् ॅवि॒द्वाꣳसः॑ । वि॒द्वाꣳसः॑ सप्तदशरा॒त्रम् । स॒प्त॒द॒श॒रा॒त्रमास॑ते । स॒प्त॒द॒श॒रा॒त्रमिति॑ सप्तदश - रा॒त्रम् । आस॑तेऽन्ना॒दाः । अ॒न्ना॒दा ए॒व । अ॒न्ना॒दा इत्य॑न्न - अ॒दाः । ए॒व भ॑वन्ति । भ॒व॒न्ति॒ प॒ञ्चा॒हः । प॒ञ्चा॒हो भ॑वति । प॒ञ्चा॒ह इति॑ पञ्च - अ॒हः । भ॒व॒ति॒ पञ्च॑ । पञ्च॒ वै । वा ऋ॒तवः॑ । ऋ॒तवः॑ सम्ॅवथ्स॒रः । स॒म्ॅव॒थ्स॒र ऋ॒तुषु॑ । स॒म्ॅव॒थ्स॒र इति॑ सम् - व॒थ्स॒रः । ऋ॒तुष्वे॒व । ए॒व स॑म्ॅवथ्स॒रे । स॒म्ॅव॒थ्स॒रे प्रति॑ । स॒म्ॅव॒थ्स॒र इति॑ सम् - व॒थ्स॒रे । प्रति॑ तिष्ठन्ति । ति॒ष्ठ॒न्त्यथो᳚ । अथो॒ पञ्चा᳚क्षरा । अथो॒ इत्यथो᳚ । पञ्चा᳚क्षरा प॒ङ्‍क्तिः । पञ्चा᳚क्ष॒रेति॒ पञ्च॑ - अ॒क्ष॒रा॒ । प॒ङ्‍क्तिः पाङ्‍क्तः॑ । पाङ्‍क्तो॑ य॒ज्ञ्ः । य॒ज्ञो य॒ज्ञ्म् । य॒ज्ञ्मे॒व । ए॒वाव॑ । अव॑ रुन्धते । रु॒न्ध॒तेऽस॑त्रम् । अस॑त्र॒म् ॅवै । वा ए॒तत् ( ) । ए॒तद् यत् \newline

\textbf{Jatai Paata} \newline

1. प्र॒जाप॑ति रकामयता कामयत प्र॒जाप॑तिः प्र॒जाप॑ति रकामयत । \newline
2. प्र॒जाप॑ति॒रिति॑ प्र॒जा - प॒तिः॒ । \newline
3. अ॒का॒म॒य॒ता॒ न्ना॒दो᳚ ऽन्ना॒दो॑ ऽकामयता कामयता न्ना॒दः । \newline
4. अ॒न्ना॒दः स्याꣳ॑ स्या मन्ना॒दो᳚ ऽन्ना॒दः स्या᳚म् । \newline
5. अ॒न्ना॒द इत्य॑न्न - अ॒दः । \newline
6. स्या॒ मितीति॑ स्याꣳ स्या॒ मिति॑ । \newline
7. इति॒ स स इतीति॒ सः । \newline
8. स ए॒त मे॒तꣳ स स ए॒तम् । \newline
9. ए॒तꣳ स॑प्तदशरा॒त्रꣳ स॑प्तदशरा॒त्र मे॒त मे॒तꣳ स॑प्तदशरा॒त्रम् । \newline
10. स॒प्त॒द॒श॒रा॒त्र म॑पश्य दपश्यथ् सप्तदशरा॒त्रꣳ स॑प्तदशरा॒त्र म॑पश्यत् । \newline
11. स॒प्त॒द॒श॒रा॒त्रमिति॑ सप्तदश - रा॒त्रम् । \newline
12. अ॒प॒श्य॒त् तम् त म॑पश्य दपश्य॒त् तम् । \newline
13. त मा तम् त मा । \newline
14. आ ऽह॑र दहर॒दा ऽह॑रत् । \newline
15. अ॒ह॒र॒त् तेन॒ तेना॑ हर दहर॒त् तेन॑ । \newline
16. तेना॑ यजता यजत॒ तेन॒ तेना॑ यजत । \newline
17. अ॒य॒ज॒त॒ तत॒ स्ततो॑ ऽयजता यजत॒ ततः॑ । \newline
18. ततो॒ वै वै तत॒ स्ततो॒ वै । \newline
19. वै स स वै वै सः । \newline
20. सो᳚ ऽन्ना॒दो᳚ ऽन्ना॒दः स सो᳚ ऽन्ना॒दः । \newline
21. अ॒न्ना॒दो॑ ऽभव दभव दन्ना॒दो᳚ ऽन्ना॒दो॑ ऽभवत् । \newline
22. अ॒न्ना॒द इत्य॑न्न - अ॒दः । \newline
23. अ॒भ॒व॒द् ये ये॑ ऽभव दभव॒द् ये । \newline
24. य ए॒व मे॒वं ॅये य ए॒वम् । \newline
25. ए॒वं ॅवि॒द्वाꣳसो॑ वि॒द्वाꣳस॑ ए॒व मे॒वं ॅवि॒द्वाꣳसः॑ । \newline
26. वि॒द्वाꣳसः॑ सप्तदशरा॒त्रꣳ स॑प्तदशरा॒त्रं ॅवि॒द्वाꣳसो॑ वि॒द्वाꣳसः॑ सप्तदशरा॒त्रम् । \newline
27. स॒प्त॒द॒श॒रा॒त्र मास॑त॒ आस॑ते सप्तदशरा॒त्रꣳ स॑प्तदशरा॒त्र मास॑ते । \newline
28. स॒प्त॒द॒श॒रा॒त्रमिति॑ सप्तदश - रा॒त्रम् । \newline
29. आस॑ते ऽन्ना॒दा अ॑न्ना॒दा आस॑त॒ आस॑ते ऽन्ना॒दाः । \newline
30. अ॒न्ना॒दा ए॒वैवा न्ना॒दा अ॑न्ना॒दा ए॒व । \newline
31. अ॒न्ना॒दा इत्य॑न्न - अ॒दाः । \newline
32. ए॒व भ॑वन्ति भवन् त्ये॒वैव भ॑वन्ति । \newline
33. भ॒व॒न्ति॒ प॒ञ्चा॒हः प॑ञ्चा॒हो भ॑वन्ति भवन्ति पञ्चा॒हः । \newline
34. प॒ञ्चा॒हो भ॑वति भवति पञ्चा॒हः प॑ञ्चा॒हो भ॑वति । \newline
35. प॒ञ्चा॒ह इति॑ पञ्च - अ॒हः । \newline
36. भ॒व॒ति॒ पञ्च॒ पञ्च॑ भवति भवति॒ पञ्च॑ । \newline
37. पञ्च॒ वै वै पञ्च॒ पञ्च॒ वै । \newline
38. वा ऋ॒तव॑ ऋ॒तवो॒ वै वा ऋ॒तवः॑ । \newline
39. ऋ॒तवः॑ संॅवथ्स॒रः सं॑ॅवथ्स॒र ऋ॒तव॑ ऋ॒तवः॑ संॅवथ्स॒रः । \newline
40. सं॒ॅव॒थ्स॒र ऋ॒तुष् वृ॒तुषु॑ संॅवथ्स॒रः सं॑ॅवथ्स॒र ऋ॒तुषु॑ । \newline
41. सं॒ॅव॒थ्स॒र इति॑ सं - व॒थ्स॒रः । \newline
42. ऋ॒तुष् वे॒वैव र्तुष्व् ऋ॒तुष् वे॒व । \newline
43. ए॒व सं॑ॅवथ्स॒रे सं॑ॅवथ्स॒र ए॒वैव सं॑ॅवथ्स॒रे । \newline
44. सं॒ॅव॒थ्स॒रे प्रति॒ प्रति॑ संॅवथ्स॒रे सं॑ॅवथ्स॒रे प्रति॑ । \newline
45. सं॒ॅव॒थ्स॒र इति॑ सं - व॒थ्स॒रे । \newline
46. प्रति॑ तिष्ठन्ति तिष्ठन्ति॒ प्रति॒ प्रति॑ तिष्ठन्ति । \newline
47. ति॒ष्ठ॒न् त्यथो॒ अथो॑ तिष्ठन्ति तिष्ठ॒न् त्यथो᳚ । \newline
48. अथो॒ पञ्चा᳚क्षरा॒ पञ्चा᳚क्ष॒रा ऽथो॒ अथो॒ पञ्चा᳚क्षरा । \newline
49. अथो॒ इत्यथो᳚ । \newline
50. पञ्चा᳚क्षरा प॒ङ्क्तिः प॒ङ्क्तिः पञ्चा᳚क्षरा॒ पञ्चा᳚क्षरा प॒ङ्क्तिः । \newline
51. पञ्चा᳚क्ष॒रेति॒ पञ्च॑ - अ॒क्ष॒रा॒ । \newline
52. प॒ङ्क्तिः पाङ्क्तः॒ पाङ्क्तः॑ प॒ङ्क्तिः प॒ङ्क्तिः पाङ्क्तः॑ । \newline
53. पाङ्क्तो॑ य॒ज्ञो य॒ज्ञ्ः पाङ्क्तः॒ पाङ्क्तो॑ य॒ज्ञ्ः । \newline
54. य॒ज्ञो य॒ज्ञ्ं ॅय॒ज्ञ्ं ॅय॒ज्ञो य॒ज्ञो य॒ज्ञ्म् । \newline
55. य॒ज्ञ् मे॒वैव य॒ज्ञ्ं ॅय॒ज्ञ् मे॒व । \newline
56. ए॒वावा वै॒वै वाव॑ । \newline
57. अव॑ रुन्धते रुन्ध॒ते ऽवाव॑ रुन्धते । \newline
58. रु॒न्ध॒ते ऽस॑त्र॒ मस॑त्रꣳ रुन्धते रुन्ध॒ते ऽस॑त्रम् । \newline
59. अस॑त्रं॒ ॅवै वा अस॑त्र॒ मस॑त्रं॒ ॅवै । \newline
60. वा ए॒त दे॒तद् वै वा ए॒तत् । \newline
61. ए॒तद् यद् यदे॒त दे॒तद् यत् । \newline

\textbf{Ghana Paata } \newline

1. प्र॒जाप॑ति रकामयता कामयत प्र॒जाप॑तिः प्र॒जाप॑ति रकामयता न्ना॒दो᳚ ऽन्ना॒दो॑ ऽकामयत प्र॒जाप॑तिः प्र॒जाप॑ति रकामयता न्ना॒दः । \newline
2. प्र॒जाप॑ति॒रिति॑ प्र॒जा - प॒तिः॒ । \newline
3. अ॒का॒म॒य॒ता॒ न्ना॒दो᳚ ऽन्ना॒दो॑ ऽकामयता कामयता न्ना॒दः स्याꣳ॑ स्या मन्ना॒दो॑ ऽकामयता कामयता न्ना॒दः स्या᳚म् । \newline
4. अ॒न्ना॒दः स्याꣳ॑ स्या मन्ना॒दो᳚ ऽन्ना॒दः स्या॒ मितीति॑ स्या मन्ना॒दो᳚ ऽन्ना॒दः स्या॒ मिति॑ । \newline
5. अ॒न्ना॒द इत्य॑न्न - अ॒दः । \newline
6. स्या॒ मितीति॑ स्याꣳ स्या॒ मिति॒ स स इति॑ स्याꣳ स्या॒ मिति॒ सः । \newline
7. इति॒ स स इतीति॒ स ए॒त मे॒तꣳ स इतीति॒ स ए॒तम् । \newline
8. स ए॒त मे॒तꣳ स स ए॒तꣳ स॑प्तदशरा॒त्रꣳ स॑प्तदशरा॒त्र मे॒तꣳ स स ए॒तꣳ स॑प्तदशरा॒त्रम् । \newline
9. ए॒तꣳ स॑प्तदशरा॒त्रꣳ स॑प्तदशरा॒त्र मे॒त मे॒तꣳ स॑प्तदशरा॒त्र म॑पश्य दपश्यथ् सप्तदशरा॒त्र मे॒त मे॒तꣳ स॑प्तदशरा॒त्र म॑पश्यत् । \newline
10. स॒प्त॒द॒श॒रा॒त्र म॑पश्य दपश्यथ् सप्तदशरा॒त्रꣳ स॑प्तदशरा॒त्र म॑पश्य॒त् तम् त म॑पश्यथ् सप्तदशरा॒त्रꣳ स॑प्तदशरा॒त्र म॑पश्य॒त् तम् । \newline
11. स॒प्त॒द॒श॒रा॒त्रमिति॑ सप्तदश - रा॒त्रम् । \newline
12. अ॒प॒श्य॒त् तम् त म॑पश्य दपश्य॒त् त मा त म॑पश्य दपश्य॒त् त मा । \newline
13. त मा तम् त मा ऽह॑र दहर॒दा तम् त मा ऽह॑रत् । \newline
14. आ ऽह॑र दहर॒दा ऽह॑र॒त् तेन॒ तेना॑ हर॒दा ऽह॑र॒त् तेन॑ । \newline
15. अ॒ह॒र॒त् तेन॒ तेना॑ हर दहर॒त् तेना॑ यजता यजत॒ तेना॑ हर दहर॒त् तेना॑ यजत । \newline
16. तेना॑ यजता यजत॒ तेन॒ तेना॑ यजत॒ तत॒ स्ततो॑ ऽयजत॒ तेन॒ तेना॑ यजत॒ ततः॑ । \newline
17. अ॒य॒ज॒त॒ तत॒ स्ततो॑ ऽयजता यजत॒ ततो॒ वै वै ततो॑ ऽयजता यजत॒ ततो॒ वै । \newline
18. ततो॒ वै वै तत॒ स्ततो॒ वै स स वै तत॒ स्ततो॒ वै सः । \newline
19. वै स स वै वै सो᳚ ऽन्ना॒दो᳚ ऽन्ना॒दः स वै वै सो᳚ ऽन्ना॒दः । \newline
20. सो᳚ ऽन्ना॒दो᳚ ऽन्ना॒दः स सो᳚ ऽन्ना॒दो॑ ऽभव दभव दन्ना॒दः स सो᳚ ऽन्ना॒दो॑ ऽभवत् । \newline
21. अ॒न्ना॒दो॑ ऽभव दभव दन्ना॒दो᳚ ऽन्ना॒दो॑ ऽभव॒द् ये ये॑ ऽभव दन्ना॒दो᳚ ऽन्ना॒दो॑ ऽभव॒द् ये । \newline
22. अ॒न्ना॒द इत्य॑न्न - अ॒दः । \newline
23. अ॒भ॒व॒द् ये ये॑ ऽभव दभव॒द् य ए॒व मे॒वं ॅये॑ ऽभव दभव॒द् य ए॒वम् । \newline
24. य ए॒व मे॒वं ॅये य ए॒वं ॅवि॒द्वाꣳसो॑ वि॒द्वाꣳस॑ ए॒वं ॅये य ए॒वं ॅवि॒द्वाꣳसः॑ । \newline
25. ए॒वं ॅवि॒द्वाꣳसो॑ वि॒द्वाꣳस॑ ए॒व मे॒वं ॅवि॒द्वाꣳसः॑ सप्तदशरा॒त्रꣳ स॑प्तदशरा॒त्रं ॅवि॒द्वाꣳस॑ ए॒व मे॒वं ॅवि॒द्वाꣳसः॑ सप्तदशरा॒त्रम् । \newline
26. वि॒द्वाꣳसः॑ सप्तदशरा॒त्रꣳ स॑प्तदशरा॒त्रं ॅवि॒द्वाꣳसो॑ वि॒द्वाꣳसः॑ सप्तदशरा॒त्र मास॑त॒ आस॑ते सप्तदशरा॒त्रं ॅवि॒द्वाꣳसो॑ वि॒द्वाꣳसः॑ सप्तदशरा॒त्र मास॑ते । \newline
27. स॒प्त॒द॒श॒रा॒त्र मास॑त॒ आस॑ते सप्तदशरा॒त्रꣳ स॑प्तदशरा॒त्र मास॑ते ऽन्ना॒दा अ॑न्ना॒दा आस॑ते सप्तदशरा॒त्रꣳ स॑प्तदशरा॒त्र मास॑ते ऽन्ना॒दाः । \newline
28. स॒प्त॒द॒श॒रा॒त्रमिति॑ सप्तदश - रा॒त्रम् । \newline
29. आस॑ते ऽन्ना॒दा अ॑न्ना॒दा आस॑त॒ आस॑ते ऽन्ना॒दा ए॒वै वान्ना॒दा आस॑त॒ आस॑ते ऽन्ना॒दा ए॒व । \newline
30. अ॒न्ना॒दा ए॒वै वान्ना॒दा अ॑न्ना॒दा ए॒व भ॑वन्ति भव न्त्ये॒वान्ना॒दा अ॑न्ना॒दा ए॒व भ॑वन्ति । \newline
31. अ॒न्ना॒दा इत्य॑न्न - अ॒दाः । \newline
32. ए॒व भ॑वन्ति भव न्त्ये॒वैव भ॑वन्ति पञ्चा॒हः प॑ञ्चा॒हो भ॑व न्त्ये॒वैव भ॑वन्ति पञ्चा॒हः । \newline
33. भ॒व॒न्ति॒ प॒ञ्चा॒हः प॑ञ्चा॒हो भ॑वन्ति भवन्ति पञ्चा॒हो भ॑वति भवति पञ्चा॒हो भ॑वन्ति भवन्ति पञ्चा॒हो भ॑वति । \newline
34. प॒ञ्चा॒हो भ॑वति भवति पञ्चा॒हः प॑ञ्चा॒हो भ॑वति॒ पञ्च॒ पञ्च॑ भवति पञ्चा॒हः प॑ञ्चा॒हो भ॑वति॒ पञ्च॑ । \newline
35. प॒ञ्चा॒ह इति॑ पञ्च - अ॒हः । \newline
36. भ॒व॒ति॒ पञ्च॒ पञ्च॑ भवति भवति॒ पञ्च॒ वै वै पञ्च॑ भवति भवति॒ पञ्च॒ वै । \newline
37. पञ्च॒ वै वै पञ्च॒ पञ्च॒ वा ऋ॒तव॑ ऋ॒तवो॒ वै पञ्च॒ पञ्च॒ वा ऋ॒तवः॑ । \newline
38. वा ऋ॒तव॑ ऋ॒तवो॒ वै वा ऋ॒तवः॑ संॅवथ्स॒रः सं॑ॅवथ्स॒र ऋ॒तवो॒ वै वा ऋ॒तवः॑ संॅवथ्स॒रः । \newline
39. ऋ॒तवः॑ संॅवथ्स॒रः सं॑ॅवथ्स॒र ऋ॒तव॑ ऋ॒तवः॑ संॅवथ्स॒र ऋ॒तुष् वृ॒तुषु॑ संॅवथ्स॒र ऋ॒तव॑ ऋ॒तवः॑ संॅवथ्स॒र ऋ॒तुषु॑ । \newline
40. सं॒ॅव॒थ्स॒र ऋ॒तुष् वृ॒तुषु॑ संॅवथ्स॒रः सं॑ॅवथ्स॒र ऋ॒तु ष्वे॒वैव र्‌तुषु॑ संॅवथ्स॒रः सं॑ॅवथ्स॒र ऋ॒तुष्वे॒व । \newline
41. सं॒ॅव॒थ्स॒र इति॑ सं - व॒थ्स॒रः । \newline
42. ऋ॒तु ष्वे॒वैव र्‌तुष् वृ॒तु ष्वे॒व सं॑ॅवथ्स॒रे सं॑ॅवथ्स॒र ए॒व र्‌तुष् वृ॒तु ष्वे॒व सं॑ॅवथ्स॒रे । \newline
43. ए॒व सं॑ॅवथ्स॒रे सं॑ॅवथ्स॒र ए॒वैव सं॑ॅवथ्स॒रे प्रति॒ प्रति॑ संॅवथ्स॒र ए॒वैव सं॑ॅवथ्स॒रे प्रति॑ । \newline
44. सं॒ॅव॒थ्स॒रे प्रति॒ प्रति॑ संॅवथ्स॒रे सं॑ॅवथ्स॒रे प्रति॑ तिष्ठन्ति तिष्ठन्ति॒ प्रति॑ संॅवथ्स॒रे सं॑ॅवथ्स॒रे प्रति॑ तिष्ठन्ति । \newline
45. सं॒ॅव॒थ्स॒र इति॑ सं - व॒थ्स॒रे । \newline
46. प्रति॑ तिष्ठन्ति तिष्ठन्ति॒ प्रति॒ प्रति॑ तिष्ठ॒ न्त्यथो॒ अथो॑ तिष्ठन्ति॒ प्रति॒ प्रति॑ तिष्ठ॒ न्त्यथो᳚ । \newline
47. ति॒ष्ठ॒ न्त्यथो॒ अथो॑ तिष्ठन्ति तिष्ठ॒ न्त्यथो॒ पञ्चा᳚क्षरा॒ पञ्चा᳚क्ष॒रा ऽथो॑ तिष्ठन्ति तिष्ठ॒ न्त्यथो॒ पञ्चा᳚क्षरा । \newline
48. अथो॒ पञ्चा᳚क्षरा॒ पञ्चा᳚क्ष॒रा ऽथो॒ अथो॒ पञ्चा᳚क्षरा प॒ङ्क्तिः प॒ङ्क्तिः पञ्चा᳚क्ष॒रा ऽथो॒ अथो॒ पञ्चा᳚क्षरा प॒ङ्क्तिः । \newline
49. अथो॒ इत्यथो᳚ । \newline
50. पञ्चा᳚क्षरा प॒ङ्क्तिः प॒ङ्क्तिः पञ्चा᳚क्षरा॒ पञ्चा᳚क्षरा प॒ङ्क्तिः पाङ्क्तः॒ पाङ्क्तः॑ प॒ङ्क्तिः पञ्चा᳚क्षरा॒ पञ्चा᳚क्षरा प॒ङ्क्तिः पाङ्क्तः॑ । \newline
51. पञ्चा᳚क्ष॒रेति॒ पञ्च॑ - अ॒क्ष॒रा॒ । \newline
52. प॒ङ्क्तिः पाङ्क्तः॒ पाङ्क्तः॑ प॒ङ्क्तिः प॒ङ्क्तिः पाङ्क्तो॑ य॒ज्ञो य॒ज्ञ्ः पाङ्क्तः॑ प॒ङ्क्तिः प॒ङ्क्तिः पाङ्क्तो॑ य॒ज्ञ्ः । \newline
53. पाङ्क्तो॑ य॒ज्ञो य॒ज्ञ्ः पाङ्क्तः॒ पाङ्क्तो॑ य॒ज्ञो य॒ज्ञ्ं ॅय॒ज्ञ्ं ॅय॒ज्ञ्ः पाङ्क्तः॒ पाङ्क्तो॑ य॒ज्ञो य॒ज्ञ्म् । \newline
54. य॒ज्ञो य॒ज्ञ्ं ॅय॒ज्ञ्ं ॅय॒ज्ञो य॒ज्ञो य॒ज्ञ् मे॒वैव य॒ज्ञ्ं ॅय॒ज्ञो य॒ज्ञो य॒ज्ञ् मे॒व । \newline
55. य॒ज्ञ् मे॒वैव य॒ज्ञ्ं ॅय॒ज्ञ् मे॒वावा वै॒व य॒ज्ञ्ं ॅय॒ज्ञ् मे॒वाव॑ । \newline
56. ए॒वावा वै॒वै वाव॑ रुन्धते रुन्ध॒ते ऽवै॒वै वाव॑ रुन्धते । \newline
57. अव॑ रुन्धते रुन्ध॒ते ऽवाव॑ रुन्ध॒ते ऽस॑त्र॒ मस॑त्रꣳ रुन्ध॒ते ऽवाव॑ रुन्ध॒ते ऽस॑त्रम् । \newline
58. रु॒न्ध॒ते ऽस॑त्र॒ मस॑त्रꣳ रुन्धते रुन्ध॒ते ऽस॑त्रं॒ ॅवै वा अस॑त्रꣳ रुन्धते रुन्ध॒ते ऽस॑त्रं॒ ॅवै । \newline
59. अस॑त्रं॒ ॅवै वा अस॑त्र॒ मस॑त्रं॒ ॅवा ए॒त दे॒तद् वा अस॑त्र॒ मस॑त्रं॒ ॅवा ए॒तत् । \newline
60. वा ए॒त दे॒तद् वै वा ए॒तद् यद् यदे॒तद् वै वा ए॒तद् यत् । \newline
61. ए॒तद् यद् यदे॒त दे॒तद् यद॑छन्दो॒म म॑छन्दो॒मं ॅयदे॒त दे॒तद् यद॑छन्दो॒मम् । \newline
\pagebreak
\markright{ TS 7.3.8.2  \hfill https://www.vedavms.in \hfill}

\section{ TS 7.3.8.2 }

\textbf{TS 7.3.8.2 } \newline
\textbf{Samhita Paata} \newline

-द्यद॑छन्दो॒मं ॅयच्छ॑न्दो॒मा भव॑न्ति॒ तेन॑ स॒त्रं दे॒वता॑ ए॒व पृ॒ष्ठैरव॑ रुन्धते प॒शूञ्छ॑न्दो॒मैरोजो॒ वै वी॒र्यं॑ पृ॒ष्ठानि॑ प॒शवः॑ छन्दो॒मा ओज॑स्ये॒व वी॒र्ये॑ प॒शुषु॒ प्रति॑ तिष्ठन्ति सप्तदशरा॒त्रो भ॑वति सप्तद॒शः प्र॒जाप॑तिः प्र॒जाप॑ते॒राप्त्या॑ अतिरा॒त्राव॒भितो॑ भवतो॒ऽन्नाद्य॑स्य॒ परि॑गृहीत्यै ॥ \newline

\textbf{Pada Paata} \newline

यत् । अ॒छ॒न्दो॒ममित्य॑छन्दः-मम् । यत् । छ॒न्दो॒मा इति॑ छन्दः - माः । भव॑न्ति । तेन॑ । स॒त्रम् । दे॒वताः᳚ । ए॒व । पृ॒ष्ठैः । अवेति॑ । रु॒न्ध॒ते॒ । प॒शून् । छ॒न्दो॒मैरिति॑ छन्दः - मैः । ओजः॑ । वै । वी॒र्य᳚म् । पृ॒ष्ठानि॑ । प॒शवः॑ । छ॒न्दो॒मा इति॑ छन्दः - माः । ओज॑सि । ए॒व । वी॒र्ये᳚ । प॒शुषु॑ । प्रतीति॑ । ति॒ष्ठ॒न्ति॒ । स॒प्त॒द॒श॒रा॒त्र इति॑ सप्तदश - रा॒त्रः । भ॒व॒ति॒ । स॒प्त॒द॒श इति॑ सप्त - द॒शः । प्र॒जाप॑ति॒रिति॑ प्र॒जा-प॒तिः॒ । प्र॒जाप॑ते॒रिति॑ प्र॒जा - प॒तेः॒ । आप्त्यै᳚ । अ॒ति॒रा॒त्रावित्य॑ति - रा॒त्रौ । अ॒भितः॑ । भ॒व॒तः॒ । अ॒न्नाद्य॒स्येत्य॑न्न - अद्य॑स्य । परि॑गृहीत्या॒ इति॒ परि॑ - गृ॒ही॒त्यै॒ ॥  \newline


\textbf{Krama Paata} \newline

यद॑छन्दो॒मम् । अ॒छ॒न्दो॒मम् ॅयत् । अ॒छ॒न्दो॒ममित्य॑छन्दः - मम् । यच्छ॑न्दो॒माः । छ॒न्दो॒मा भव॑न्ति । छ॒न्दो॒मा इति॑ छन्दः - माः । भव॑न्ति॒ तेन॑ । तेन॑ स॒त्रम् । स॒त्रम् दे॒वताः᳚ । दे॒वता॑ ए॒व । ए॒व पृ॒ष्ठैः । पृ॒ष्ठैरव॑ । अव॑ रुन्धते । रु॒न्ध॒ते॒ प॒शून् । प॒शूञ्छ॑न्दो॒मैः । छ॒न्दो॒मैरोजः॑ । छ॒न्दो॒मैरिति॑ छन्दः - मै । ओजो॒ वै । वै वी॒र्य᳚म् । वी॒र्य॑म् पृ॒ष्ठानि॑ । पृ॒ष्ठानि॑ प॒शवः॑ । प॒शव॑श्छन्दो॒माः । छ॒न्दो॒मा ओज॑सि । छ॒न्दो॒मा इति॑ छन्दः - माः । ओज॑स्ये॒व । ए॒व वी॒र्ये᳚ । वी॒र्ये॑ प॒शुषु॑ । प॒शुषु॒ प्रति॑ । प्रति॑ तिष्ठन्ति । ति॒ष्ठ॒न्ति॒ स॒प्त॒द॒श॒रा॒त्रः । स॒प्त॒द॒श॒रा॒त्रो भ॑वति । स॒प्त॒द॒श॒रा॒त्र इति॑ सप्तदश - रा॒त्रः । भ॒व॒ति॒ स॒प्त॒द॒शः । स॒प्त॒द॒शः प्र॒जाप॑तिः । स॒प्त॒द॒श इति॑ सप्त - द॒शः । प्र॒जाप॑तिः प्र॒जाप॑तेः । प्र॒जाप॑ति॒रिति॑ प्र॒जा - प॒तिः॒ । प्र॒जाप॑ते॒राप्त्यै᳚ । प्र॒जाप॑ते॒रिति॑ प्र॒जा - प॒तेः॒ । आप्त्या॑ अतिरा॒त्रौ । अ॒ति॒रा॒त्राव॒भितः॑ । अ॒ति॒रा॒त्रावित्य॑ति - रा॒त्रौ । अ॒भितो॑ भवतः । भ॒व॒तो॒ऽन्नाद्य॑स्य । अ॒न्नाद्य॑स्य॒ परि॑गृहीत्यै । अ॒न्नाद्य॒स्येत्य॑न्न - अद्य॑स्य । परि॑गृहीत्या॒ इति॒ परि॑ - गृ॒ही॒त्यै॒ । \newline

\textbf{Jatai Paata} \newline

1. यद॑छन्दो॒म म॑छन्दो॒मं ॅयद् यद॑छन्दो॒मम् । \newline
2. अ॒छ॒न्दो॒मं ॅयद् यद॑छन्दो॒म म॑छन्दो॒मं ॅयत् । \newline
3. अ॒छ॒न्दो॒ममित्य॑छन्दः - मम् । \newline
4. यच् छ॑न्दो॒मा श्छ॑न्दो॒मा यद् यच् छ॑न्दो॒माः । \newline
5. छ॒न्दो॒मा भव॑न्ति॒ भव॑न्ति छन्दो॒मा श्छ॑न्दो॒मा भव॑न्ति । \newline
6. छ॒न्दो॒मा इति॑ छन्दः - माः । \newline
7. भव॑न्ति॒ तेन॒ तेन॒ भव॑न्ति॒ भव॑न्ति॒ तेन॑ । \newline
8. तेन॑ स॒त्रꣳ स॒त्रम् तेन॒ तेन॑ स॒त्रम् । \newline
9. स॒त्रम् दे॒वता॑ दे॒वताः᳚ स॒त्रꣳ स॒त्रम् दे॒वताः᳚ । \newline
10. दे॒वता॑ ए॒वैव दे॒वता॑ दे॒वता॑ ए॒व । \newline
11. ए॒व पृ॒ष्ठैः पृ॒ष्ठै रे॒वैव पृ॒ष्ठैः । \newline
12. पृ॒ष्ठै रवाव॑ पृ॒ष्ठैः पृ॒ष्ठै रव॑ । \newline
13. अव॑ रुन्धते रुन्ध॒ते ऽवाव॑ रुन्धते । \newline
14. रु॒न्ध॒ते॒ प॒शून् प॒शून् रु॑न्धते रुन्धते प॒शून् । \newline
15. प॒शूञ् छ॑न्दो॒मै श्छ॑न्दो॒मैः प॒शून् प॒शूञ् छ॑न्दो॒मैः । \newline
16. छ॒न्दो॒मै रोज॒ ओज॑ श्छन्दो॒मै श्छ॑न्दो॒मै रोजः॑ । \newline
17. छ॒न्दो॒मैरिति॑ छन्दः - मैः । \newline
18. ओजो॒ वै वा ओज॒ ओजो॒ वै । \newline
19. वै वी॒र्यं॑ ॅवी॒र्यं॑ ॅवै वै वी॒र्य᳚म् । \newline
20. वी॒र्य॑म् पृ॒ष्ठानि॑ पृ॒ष्ठानि॑ वी॒र्यं॑ ॅवी॒र्य॑म् पृ॒ष्ठानि॑ । \newline
21. पृ॒ष्ठानि॑ प॒शवः॑ प॒शवः॑ पृ॒ष्ठानि॑ पृ॒ष्ठानि॑ प॒शवः॑ । \newline
22. प॒शव॑ श्छन्दो॒मा श्छ॑न्दो॒माः प॒शवः॑ प॒शव॑ श्छन्दो॒माः । \newline
23. छ॒न्दो॒मा ओज॒ स्योज॑सि छन्दो॒मा श्छ॑न्दो॒मा ओज॑सि । \newline
24. छ॒न्दो॒मा इति॑ छन्दः - माः । \newline
25. ओज॑ स्ये॒वैवौज॒ स्योज॑ स्ये॒व । \newline
26. ए॒व वी॒र्ये॑ वी॒र्य॑ ए॒वैव वी॒र्ये᳚ । \newline
27. वी॒र्ये॑ प॒शुषु॑ प॒शुषु॑ वी॒र्ये॑ वी॒र्ये॑ प॒शुषु॑ । \newline
28. प॒शुषु॒ प्रति॒ प्रति॑ प॒शुषु॑ प॒शुषु॒ प्रति॑ । \newline
29. प्रति॑ तिष्ठन्ति तिष्ठन्ति॒ प्रति॒ प्रति॑ तिष्ठन्ति । \newline
30. ति॒ष्ठ॒न्ति॒ स॒प्त॒द॒श॒रा॒त्रः स॑प्तदशरा॒त्र स्ति॑ष्ठन्ति तिष्ठन्ति सप्तदशरा॒त्रः । \newline
31. स॒प्त॒द॒श॒रा॒त्रो भ॑वति भवति सप्तदशरा॒त्रः स॑प्तदशरा॒त्रो भ॑वति । \newline
32. स॒प्त॒द॒श॒रा॒त्र इति॑ सप्तदश - रा॒त्रः । \newline
33. भ॒व॒ति॒ स॒प्त॒द॒शः स॑प्तद॒शो भ॑वति भवति सप्तद॒शः । \newline
34. स॒प्त॒द॒शः प्र॒जाप॑तिः प्र॒जाप॑तिः सप्तद॒शः स॑प्तद॒शः प्र॒जाप॑तिः । \newline
35. स॒प्त॒द॒श इति॑ सप्त - द॒शः । \newline
36. प्र॒जाप॑तिः प्र॒जाप॑तेः प्र॒जाप॑तेः प्र॒जाप॑तिः प्र॒जाप॑तिः प्र॒जाप॑तेः । \newline
37. प्र॒जाप॑ति॒रिति॑ प्र॒जा - प॒तिः॒ । \newline
38. प्र॒जाप॑ते॒ राप्त्या॒ आप्त्यै᳚ प्र॒जाप॑तेः प्र॒जाप॑ते॒ राप्त्यै᳚ । \newline
39. प्र॒जाप॑ते॒रिति॑ प्र॒जा - प॒तेः॒ । \newline
40. आप्त्या॑ अतिरा॒त्रा व॑तिरा॒त्रा वाप्त्या॒ आप्त्या॑ अतिरा॒त्रौ । \newline
41. अ॒ति॒रा॒त्रा व॒भितो॒ ऽभितो॑ ऽतिरा॒त्रा व॑तिरा॒त्रा व॒भितः॑ । \newline
42. अ॒ति॒रा॒त्रावित्य॑ति - रा॒त्रौ । \newline
43. अ॒भितो॑ भवतो भवतो॒ ऽभितो॒ ऽभितो॑ भवतः । \newline
44. भ॒व॒तो॒ ऽन्नाद्य॑स्या॒ न्नाद्य॑स्य भवतो भवतो॒ ऽन्नाद्य॑स्य । \newline
45. अ॒न्नाद्य॑स्य॒ परि॑गृहीत्यै॒ परि॑गृहीत्या अ॒न्नाद्य॑स्या॒ न्नाद्य॑स्य॒ परि॑गृहीत्यै । \newline
46. अ॒न्नाद्य॒स्येत्य॑न्न - अद्य॑स्य । \newline
47. परि॑गृहीत्या॒ इति॒ परि॑ - गृ॒ही॒त्यै॒ । \newline

\textbf{Ghana Paata } \newline

1. यद॑छन्दो॒म म॑छन्दो॒मं ॅयद् यद॑छन्दो॒मं ॅयद् यद॑छन्दो॒मं ॅयद् यद॑छन्दो॒मं ॅयत् । \newline
2. अ॒छ॒न्दो॒मं ॅयद् यद॑छन्दो॒म म॑छन्दो॒मं ॅयच् छ॑न्दो॒मा श्छ॑न्दो॒मा यद॑छन्दो॒म म॑छन्दो॒मं ॅयच् छ॑न्दो॒माः । \newline
3. अ॒छ॒न्दो॒ममित्य॑छन्दः - मम् । \newline
4. यच् छ॑न्दो॒मा श्छ॑न्दो॒मा यद् यच् छ॑न्दो॒मा भव॑न्ति॒ भव॑न्ति छन्दो॒मा यद् यच् छ॑न्दो॒मा भव॑न्ति । \newline
5. छ॒न्दो॒मा भव॑न्ति॒ भव॑न्ति छन्दो॒मा श्छ॑न्दो॒मा भव॑न्ति॒ तेन॒ तेन॒ भव॑न्ति छन्दो॒मा श्छ॑न्दो॒मा भव॑न्ति॒ तेन॑ । \newline
6. छ॒न्दो॒मा इति॑ छन्दः - माः । \newline
7. भव॑न्ति॒ तेन॒ तेन॒ भव॑न्ति॒ भव॑न्ति॒ तेन॑ स॒त्रꣳ स॒त्रम् तेन॒ भव॑न्ति॒ भव॑न्ति॒ तेन॑ स॒त्रम् । \newline
8. तेन॑ स॒त्रꣳ स॒त्रम् तेन॒ तेन॑ स॒त्रम् दे॒वता॑ दे॒वताः᳚ स॒त्रम् तेन॒ तेन॑ स॒त्रम् दे॒वताः᳚ । \newline
9. स॒त्रम् दे॒वता॑ दे॒वताः᳚ स॒त्रꣳ स॒त्रम् दे॒वता॑ ए॒वैव दे॒वताः᳚ स॒त्रꣳ स॒त्रम् दे॒वता॑ ए॒व । \newline
10. दे॒वता॑ ए॒वैव दे॒वता॑ दे॒वता॑ ए॒व पृ॒ष्ठैः पृ॒ष्ठै रे॒व दे॒वता॑ दे॒वता॑ ए॒व पृ॒ष्ठैः । \newline
11. ए॒व पृ॒ष्ठैः पृ॒ष्ठै रे॒वैव पृ॒ष्ठै रवाव॑ पृ॒ष्ठै रे॒वैव पृ॒ष्ठै रव॑ । \newline
12. पृ॒ष्ठै रवाव॑ पृ॒ष्ठैः पृ॒ष्ठै रव॑ रुन्धते रुन्ध॒ते ऽव॑ पृ॒ष्ठैः पृ॒ष्ठै रव॑ रुन्धते । \newline
13. अव॑ रुन्धते रुन्ध॒ते ऽवाव॑ रुन्धते प॒शून् प॒शून् रु॑न्ध॒ते ऽवाव॑ रुन्धते प॒शून् । \newline
14. रु॒न्ध॒ते॒ प॒शून् प॒शून् रु॑न्धते रुन्धते प॒शूञ् छ॑न्दो॒मै श्छ॑न्दो॒मैः प॒शून् रु॑न्धते रुन्धते प॒शूञ् छ॑न्दो॒मैः । \newline
15. प॒शूञ् छ॑न्दो॒मै श्छ॑न्दो॒मैः प॒शून् प॒शूञ् छ॑न्दो॒मै रोज॒ ओज॑ श्छन्दो॒मैः प॒शून् प॒शूञ् छ॑न्दो॒मै रोजः॑ । \newline
16. छ॒न्दो॒मै रोज॒ ओज॑ श्छन्दो॒मै श्छ॑न्दो॒मै रोजो॒ वै वा ओज॑ श्छन्दो॒मै श्छ॑न्दो॒मै रोजो॒ वै । \newline
17. छ॒न्दो॒मैरिति॑ छन्दः - मैः । \newline
18. ओजो॒ वै वा ओज॒ ओजो॒ वै वी॒र्यं॑ ॅवी॒र्यं॑ ॅवा ओज॒ ओजो॒ वै वी॒र्य᳚म् । \newline
19. वै वी॒र्यं॑ ॅवी॒र्यं॑ ॅवै वै वी॒र्य॑म् पृ॒ष्ठानि॑ पृ॒ष्ठानि॑ वी॒र्यं॑ ॅवै वै वी॒र्य॑म् पृ॒ष्ठानि॑ । \newline
20. वी॒र्य॑म् पृ॒ष्ठानि॑ पृ॒ष्ठानि॑ वी॒र्यं॑ ॅवी॒र्य॑म् पृ॒ष्ठानि॑ प॒शवः॑ प॒शवः॑ पृ॒ष्ठानि॑ वी॒र्यं॑ ॅवी॒र्य॑म् पृ॒ष्ठानि॑ प॒शवः॑ । \newline
21. पृ॒ष्ठानि॑ प॒शवः॑ प॒शवः॑ पृ॒ष्ठानि॑ पृ॒ष्ठानि॑ प॒शव॑ श्छन्दो॒मा श्छ॑न्दो॒माः प॒शवः॑ पृ॒ष्ठानि॑ पृ॒ष्ठानि॑ प॒शव॑ श्छन्दो॒माः । \newline
22. प॒शव॑ श्छन्दो॒मा श्छ॑न्दो॒माः प॒शवः॑ प॒शव॑ श्छन्दो॒मा ओज॒ स्योज॑सि छन्दो॒माः प॒शवः॑ प॒शव॑ श्छन्दो॒मा ओज॑सि । \newline
23. छ॒न्दो॒मा ओज॒ स्योज॑सि छन्दो॒मा श्छ॑न्दो॒मा ओज॑ स्ये॒वै वौज॑सि छन्दो॒मा श्छ॑न्दो॒मा ओज॑ स्ये॒व । \newline
24. छ॒न्दो॒मा इति॑ छन्दः - माः । \newline
25. ओज॑ स्ये॒वै वौज॒ स्योज॑ स्ये॒व वी॒र्ये॑ वी॒र्य॑ ए॒वौज॒ स्योज॑ स्ये॒व वी॒र्ये᳚ । \newline
26. ए॒व वी॒र्ये॑ वी॒र्य॑ ए॒वैव वी॒र्ये॑ प॒शुषु॑ प॒शुषु॑ वी॒र्य॑ ए॒वैव वी॒र्ये॑ प॒शुषु॑ । \newline
27. वी॒र्ये॑ प॒शुषु॑ प॒शुषु॑ वी॒र्ये॑ वी॒र्ये॑ प॒शुषु॒ प्रति॒ प्रति॑ प॒शुषु॑ वी॒र्ये॑ वी॒र्ये॑ प॒शुषु॒ प्रति॑ । \newline
28. प॒शुषु॒ प्रति॒ प्रति॑ प॒शुषु॑ प॒शुषु॒ प्रति॑ तिष्ठन्ति तिष्ठन्ति॒ प्रति॑ प॒शुषु॑ प॒शुषु॒ प्रति॑ तिष्ठन्ति । \newline
29. प्रति॑ तिष्ठन्ति तिष्ठन्ति॒ प्रति॒ प्रति॑ तिष्ठन्ति सप्तदशरा॒त्रः स॑प्तदशरा॒त्र स्ति॑ष्ठन्ति॒ प्रति॒ प्रति॑ तिष्ठन्ति सप्तदशरा॒त्रः । \newline
30. ति॒ष्ठ॒न्ति॒ स॒प्त॒द॒श॒रा॒त्रः स॑प्तदशरा॒त्र स्ति॑ष्ठन्ति तिष्ठन्ति सप्तदशरा॒त्रो भ॑वति भवति सप्तदशरा॒त्र स्ति॑ष्ठन्ति तिष्ठन्ति सप्तदशरा॒त्रो भ॑वति । \newline
31. स॒प्त॒द॒श॒रा॒त्रो भ॑वति भवति सप्तदशरा॒त्रः स॑प्तदशरा॒त्रो भ॑वति सप्तद॒शः स॑प्तद॒शो भ॑वति सप्तदशरा॒त्रः स॑प्तदशरा॒त्रो भ॑वति सप्तद॒शः । \newline
32. स॒प्त॒द॒श॒रा॒त्र इति॑ सप्तदश - रा॒त्रः । \newline
33. भ॒व॒ति॒ स॒प्त॒द॒शः स॑प्तद॒शो भ॑वति भवति सप्तद॒शः प्र॒जाप॑तिः प्र॒जाप॑तिः सप्तद॒शो भ॑वति भवति सप्तद॒शः प्र॒जाप॑तिः । \newline
34. स॒प्त॒द॒शः प्र॒जाप॑तिः प्र॒जाप॑तिः सप्तद॒शः स॑प्तद॒शः प्र॒जाप॑तिः प्र॒जाप॑तेः प्र॒जाप॑तेः प्र॒जाप॑तिः सप्तद॒शः स॑प्तद॒शः प्र॒जाप॑तिः प्र॒जाप॑तेः । \newline
35. स॒प्त॒द॒श इति॑ सप्त - द॒शः । \newline
36. प्र॒जाप॑तिः प्र॒जाप॑तेः प्र॒जाप॑तेः प्र॒जाप॑तिः प्र॒जाप॑तिः प्र॒जाप॑ते॒ राप्त्या॒ आप्त्यै᳚ प्र॒जाप॑तेः प्र॒जाप॑तिः प्र॒जाप॑तिः प्र॒जाप॑ते॒ राप्त्यै᳚ । \newline
37. प्र॒जाप॑ति॒रिति॑ प्र॒जा - प॒तिः॒ । \newline
38. प्र॒जाप॑ते॒ राप्त्या॒ आप्त्यै᳚ प्र॒जाप॑तेः प्र॒जाप॑ते॒ राप्त्या॑ अतिरा॒त्रा व॑तिरा॒त्रा वाप्त्यै᳚ प्र॒जाप॑तेः प्र॒जाप॑ते॒ राप्त्या॑ अतिरा॒त्रौ । \newline
39. प्र॒जाप॑ते॒रिति॑ प्र॒जा - प॒तेः॒ । \newline
40. आप्त्या॑ अतिरा॒त्रा व॑तिरा॒त्रा वाप्त्या॒ आप्त्या॑ अतिरा॒त्रा व॒भितो॒ ऽभितो॑ ऽतिरा॒त्रा वाप्त्या॒ आप्त्या॑ अतिरा॒त्रा व॒भितः॑ । \newline
41. अ॒ति॒रा॒त्रा व॒भितो॒ ऽभितो॑ ऽतिरा॒त्रा व॑तिरा॒त्रा व॒भितो॑ भवतो भवतो॒ ऽभितो॑ ऽतिरा॒त्रा व॑तिरा॒त्रा व॒भितो॑ भवतः । \newline
42. अ॒ति॒रा॒त्रावित्य॑ति - रा॒त्रौ । \newline
43. अ॒भितो॑ भवतो भवतो॒ ऽभितो॒ ऽभितो॑ भवतो॒ ऽन्नाद्य॑स्या॒ न्नाद्य॑स्य भवतो॒ ऽभितो॒ ऽभितो॑ भवतो॒ ऽन्नाद्य॑स्य । \newline
44. भ॒व॒तो॒ ऽन्नाद्य॑स्या॒ न्नाद्य॑स्य भवतो भवतो॒ ऽन्नाद्य॑स्य॒ परि॑गृहीत्यै॒ परि॑गृहीत्या अ॒न्नाद्य॑स्य भवतो भवतो॒ ऽन्नाद्य॑स्य॒ परि॑गृहीत्यै । \newline
45. अ॒न्नाद्य॑स्य॒ परि॑गृहीत्यै॒ परि॑गृहीत्या अ॒न्नाद्य॑स्या॒ न्नाद्य॑स्य॒ परि॑गृहीत्यै । \newline
46. अ॒न्नाद्य॒स्येत्य॑न्न - अद्य॑स्य । \newline
47. परि॑गृहीत्या॒ इति॒ परि॑ - गृ॒ही॒त्यै॒ । \newline
\pagebreak
\markright{ TS 7.3.9.1  \hfill https://www.vedavms.in \hfill}

\section{ TS 7.3.9.1 }

\textbf{TS 7.3.9.1 } \newline
\textbf{Samhita Paata} \newline

सा वि॒राड् वि॒क्रम्या॑तिष्ठ॒द्-ब्रह्म॑णा दे॒वेष्वन्ने॒ना-सु॑रेषु॒ ते दे॒वा अ॑कामयन्तो॒भयꣳ॒॒ सं ॅवृ॑ञ्जीमहि॒ ब्रह्म॒ चान्नं॒ चेति॒ त ए॒ता विꣳ॑श॒तिꣳ रात्री॑रपश्य॒न् ततो॒ वै त उ॒भयꣳ॒॒ सम॑वृञ्जत॒ ब्रह्म॒ चान्नं॑ च ब्रह्मवर्च॒सिनो᳚ऽन्ना॒दा अ॑भव॒न्॒. य ए॒वं ॅवि॒द्वाꣳस॑ ए॒ता आस॑त उ॒भय॑मे॒व संॅ वृ॑ञ्जते॒ ब्रह्म॒ चान्नं॑ च - [  ] \newline

\textbf{Pada Paata} \newline

सा । वि॒राडिति॑ वि - राट् । वि॒क्रम्येति॑ वि - क्रम्य॑ । अ॒ति॒ष्ठ॒त् । ब्रह्म॑णा । दे॒वेषु॑ । अन्ने॑न । असु॑रेषु । ते । दे॒वाः । अ॒का॒म॒य॒न्त॒ । उ॒भय᳚म् । समिति॑ । वृ॒ञ्जी॒म॒हि॒ । ब्रह्म॑ । च॒ । अन्न᳚म् । च॒ । इति॑ । ते । ए॒ताः । विꣳ॒॒श॒तिम् । रात्रीः᳚ । अ॒प॒श्य॒न्न् । ततः॑ । वै । ते । उ॒भय᳚म् । समिति॑ । अ॒वृ॒ञ्ज॒त॒ । ब्रह्म॑ । च॒ । अन्न᳚म् । च॒ । ब्र॒ह्म॒व॒र्च॒सिन॒ इति॑ ब्रह्म - व॒र्च॒सिनः॑ । अ॒न्ना॒दा इत्य॑न्न - अ॒दाः । अ॒भ॒व॒न्न् । ये । ए॒वम् । वि॒द्वाꣳसः॑ । ए॒ताः । आस॑ते । उ॒भय᳚म् । ए॒व । समिति॑ । वृ॒ञ्ज॒ते॒ । ब्रह्म॑ । च॒ । अन्न᳚म् । च॒ ।  \newline


\textbf{Krama Paata} \newline

सा वि॒राट् । वि॒राड् वि॒क्रम्य॑ । वि॒राडिति॑ वि - राट् । वि॒क्रम्या॑तिष्ठत् । वि॒क्रम्येति॑ वि - क्रम्य॑ । अ॒ति॒ष्ठ॒द् ब्रह्म॑णा । ब्रह्म॑णा दे॒वेषु॑ । दे॒वेष्वन्ने॑न । अन्ने॒नासु॑रेषु । असु॑रेषु॒ ते । ते दे॒वाः । दे॒वा अ॑कामयन्त । अ॒का॒म॒य॒न्तो॒भय᳚म् । उ॒भयꣳ॒॒ सम् । सम् ॅवृ॑ञ्जीमहि । वृ॒ञ्जी॒म॒हि॒ ब्रह्म॑ । ब्रह्म॑ च । चान्न᳚म् । अन्न॑म् च । चेति॑ । इति॒ ते । त ए॒ताः । ए॒ता विꣳ॑श॒तिम् । विꣳ॒॒श॒तिꣳ रात्रीः᳚ । रात्री॑रपश्यन्न् । अ॒प॒श्य॒न् ततः॑ । ततो॒ वै । वै ते । त उ॒भय᳚म् । उ॒भयꣳ॒॒ सम् । सम॑वृञ्जत । अ॒वृ॒ञ्ज॒त॒ ब्रह्म॑ । ब्रह्म॑ च । चान्न᳚म् । अन्न॑म् च । च॒ ब्र॒ह्म॒व॒र्च॒सिनः॑ । ब्र॒ह्म॒व॒र्च॒सिनो᳚ऽन्ना॒दाः । ब्र॒ह्म॒व॒र्च॒सिन॒ इति॑ ब्रह्म - व॒र्च॒सिनः॑ । अ॒न्ना॒दा अ॑भवन्न् । अ॒न्ना॒दा इत्य॑न्न - अ॒दाः । अ॒भ॒व॒न्॒. ये । य ए॒वम् । ए॒वम् ॅवि॒द्वाꣳसः॑ । वि॒द्वाꣳस॑ ए॒ताः । ए॒ता आस॑ते । आस॑त उ॒भय᳚म् । उ॒भय॑मे॒व । ए॒व सम् । सम् ॅवृ॑ञ्जते । वृ॒ञ्ज॒ते॒ ब्रह्म॑ । ब्रह्म॑ च । चान्न᳚म् । अन्न॑म् च । च॒ ब्र॒ह्म॒व॒र्च॒सिनः॑ \newline

\textbf{Jatai Paata} \newline

1. सा वि॒राड् वि॒राट् थ्सा सा वि॒राट् । \newline
2. वि॒राड् वि॒क्रम्य॑ वि॒क्रम्य॑ वि॒राड् वि॒राड् वि॒क्रम्य॑ । \newline
3. वि॒राडिति॑ वि - राट् । \newline
4. वि॒क्रम्या॑ तिष्ठ दतिष्ठद् वि॒क्रम्य॑ वि॒क्रम्या॑ तिष्ठत् । \newline
5. वि॒क्रम्येति॑ वि - क्रम्य॑ । \newline
6. अ॒ति॒ष्ठ॒द् ब्रह्म॑णा॒ ब्रह्म॑णा ऽतिष्ठ दतिष्ठ॒द् ब्रह्म॑णा । \newline
7. ब्रह्म॑णा दे॒वेषु॑ दे॒वेषु॒ ब्रह्म॑णा॒ ब्रह्म॑णा दे॒वेषु॑ । \newline
8. दे॒वे ष्वन्ने॒ना न्ने॑न दे॒वेषु॑ दे॒वेष्वन्ने॑न । \newline
9. अन्ने॒ना सु॑रे॒ ष्वसु॑रे॒ ष्वन्ने॒ना न्ने॒ना सु॑रेषु । \newline
10. असु॑रेषु॒ ते ते ऽसु॑रे॒ ष्वसु॑रेषु॒ ते । \newline
11. ते दे॒वा दे॒वा स्ते ते दे॒वाः । \newline
12. दे॒वा अ॑कामयन्ता कामयन्त दे॒वा दे॒वा अ॑कामयन्त । \newline
13. अ॒का॒म॒य॒न्तो॒भय॑ मु॒भय॑ मकामयन्ता कामयन्तो॒भय᳚म् । \newline
14. उ॒भयꣳ॒॒ सꣳ स मु॒भय॑ मु॒भयꣳ॒॒ सम् । \newline
15. सं ॅवृ॑ञ्जीमहि वृञ्जीमहि॒ सꣳ सं ॅवृ॑ञ्जीमहि । \newline
16. वृ॒ञ्जी॒म॒हि॒ ब्रह्म॒ ब्रह्म॑ वृञ्जीमहि वृञ्जीमहि॒ ब्रह्म॑ । \newline
17. ब्रह्म॑ च च॒ ब्रह्म॒ ब्रह्म॑ च । \newline
18. चान्न॒ मन्न॑म् च॒ चान्न᳚म् । \newline
19. अन्न॑म् च॒ चान्न॒ मन्न॑म् च । \newline
20. चेतीति॑ च॒ चेति॑ । \newline
21. इति॒ ते त इतीति॒ ते । \newline
22. त ए॒ता ए॒ता स्ते त ए॒ताः । \newline
23. ए॒ता विꣳ॑श॒तिं ॅविꣳ॑श॒ति मे॒ता ए॒ता विꣳ॑श॒तिम् । \newline
24. विꣳ॒॒श॒तिꣳ रात्री॒ रात्री᳚र् विꣳश॒तिं ॅविꣳ॑श॒तिꣳ रात्रीः᳚ । \newline
25. रात्री॑ रपश्यन् नपश्य॒न् रात्री॒ रात्री॑ रपश्यन्न् । \newline
26. अ॒प॒श्य॒न् तत॒ स्ततो॑ ऽपश्यन् नपश्य॒न् ततः॑ । \newline
27. ततो॒ वै वै तत॒ स्ततो॒ वै । \newline
28. वै ते ते वै वै ते । \newline
29. त उ॒भय॑ मु॒भय॒म् ते त उ॒भय᳚म् । \newline
30. उ॒भयꣳ॒॒ सꣳ स मु॒भय॑ मु॒भयꣳ॒॒ सम् । \newline
31. स म॑वृञ्जता वृञ्जत॒ सꣳ स म॑वृञ्जत । \newline
32. अ॒वृ॒ञ्ज॒त॒ ब्रह्म॒ ब्रह्मा॑ वृञ्जता वृञ्जत॒ ब्रह्म॑ । \newline
33. ब्रह्म॑ च च॒ ब्रह्म॒ ब्रह्म॑ च । \newline
34. चान्न॒ मन्न॑म् च॒ चान्न᳚म् । \newline
35. अन्न॑म् च॒ चान्न॒ मन्न॑म् च । \newline
36. च॒ ब्र॒ह्म॒व॒र्च॒सिनो᳚ ब्रह्मवर्च॒सिन॑ श्च च ब्रह्मवर्च॒सिनः॑ । \newline
37. ब्र॒ह्म॒व॒र्च॒सिनो᳚ ऽन्ना॒दा अ॑न्ना॒दा ब्र॑ह्मवर्च॒सिनो᳚ ब्रह्मवर्च॒सिनो᳚ ऽन्ना॒दाः । \newline
38. ब्र॒ह्म॒व॒र्च॒सिन॒ इति॑ ब्रह्म - व॒र्च॒सिनः॑ । \newline
39. अ॒न्ना॒दा अ॑भवन् नभवन् नन्ना॒दा अ॑न्ना॒दा अ॑भवन्न् । \newline
40. अ॒न्ना॒दा इत्य॑न्न - अ॒दाः । \newline
41. अ॒भ॒व॒न्॒. ये ये॑ ऽभवन् नभव॒न्॒. ये । \newline
42. य ए॒व मे॒वं ॅये य ए॒वम् । \newline
43. ए॒वं ॅवि॒द्वाꣳसो॑ वि॒द्वाꣳस॑ ए॒व मे॒वं ॅवि॒द्वाꣳसः॑ । \newline
44. वि॒द्वाꣳस॑ ए॒ता ए॒ता वि॒द्वाꣳसो॑ वि॒द्वाꣳस॑ ए॒ताः । \newline
45. ए॒ता आस॑त॒ आस॑त ए॒ता ए॒ता आस॑ते । \newline
46. आस॑त उ॒भय॑ मु॒भय॒ मास॑त॒ आस॑त उ॒भय᳚म् । \newline
47. उ॒भय॑ मे॒वै वोभय॑ मु॒भय॑ मे॒व । \newline
48. ए॒व सꣳ स मे॒वैव सम् । \newline
49. सं ॅवृ॑ञ्जते वृञ्जते॒ सꣳ सं ॅवृ॑ञ्जते । \newline
50. वृ॒ञ्ज॒ते॒ ब्रह्म॒ ब्रह्म॑ वृञ्जते वृञ्जते॒ ब्रह्म॑ । \newline
51. ब्रह्म॑ च च॒ ब्रह्म॒ ब्रह्म॑ च । \newline
52. चान्न॒ मन्न॑म् च॒ चान्न᳚म् । \newline
53. अन्न॑म् च॒ चान्न॒ मन्न॑म् च । \newline
54. च॒ ब्र॒ह्म॒व॒र्च॒सिनो᳚ ब्रह्मवर्च॒सिन॑श्च च ब्रह्मवर्च॒सिनः॑ । \newline

\textbf{Ghana Paata } \newline

1. सा वि॒राड् वि॒राट् थ्सा सा वि॒राड् वि॒क्रम्य॑ वि॒क्रम्य॑ वि॒राट् थ्सा सा वि॒राड् वि॒क्रम्य॑ । \newline
2. वि॒राड् वि॒क्रम्य॑ वि॒क्रम्य॑ वि॒राड् वि॒राड् वि॒क्रम्या॑ तिष्ठ दतिष्ठद् वि॒क्रम्य॑ वि॒राड् वि॒राड् वि॒क्रम्या॑ तिष्ठत् । \newline
3. वि॒राडिति॑ वि - राट् । \newline
4. वि॒क्रम्या॑ तिष्ठ दतिष्ठद् वि॒क्रम्य॑ वि॒क्रम्या॑ तिष्ठ॒द् ब्रह्म॑णा॒ ब्रह्म॑णा ऽतिष्ठद् वि॒क्रम्य॑ वि॒क्रम्या॑ तिष्ठ॒द् ब्रह्म॑णा । \newline
5. वि॒क्रम्येति॑ वि - क्रम्य॑ । \newline
6. अ॒ति॒ष्ठ॒द् ब्रह्म॑णा॒ ब्रह्म॑णा ऽतिष्ठ दतिष्ठ॒द् ब्रह्म॑णा दे॒वेषु॑ दे॒वेषु॒ ब्रह्म॑णा ऽतिष्ठ दतिष्ठ॒द् ब्रह्म॑णा दे॒वेषु॑ । \newline
7. ब्रह्म॑णा दे॒वेषु॑ दे॒वेषु॒ ब्रह्म॑णा॒ ब्रह्म॑णा दे॒वे ष्वन्ने॒ना न्ने॑न दे॒वेषु॒ ब्रह्म॑णा॒ ब्रह्म॑णा दे॒वेष्वन्ने॑न । \newline
8. दे॒वेष्वन्ने॒ना न्ने॑न दे॒वेषु॑ दे॒वेष्वन्ने॒ना सु॑रे॒ ष्वसु॑रे॒ ष्वन्ने॑न दे॒वेषु॑ दे॒वे ष्वन्ने॒ना सु॑रेषु । \newline
9. अन्ने॒ना सु॑रे॒ ष्वसु॑रे॒ ष्वन्ने॒ना न्ने॒ना सु॑रेषु॒ ते ते ऽसु॑रे॒ ष्वन्ने॒ना न्ने॒ना सु॑रेषु॒ ते । \newline
10. असु॑रेषु॒ ते ते ऽसु॑रे॒ ष्वसु॑रेषु॒ ते दे॒वा दे॒वा स्ते ऽसु॑रे॒ ष्वसु॑रेषु॒ ते दे॒वाः । \newline
11. ते दे॒वा दे॒वा स्ते ते दे॒वा अ॑कामयन्ता कामयन्त दे॒वा स्ते ते दे॒वा अ॑कामयन्त । \newline
12. दे॒वा अ॑कामयन्ता कामयन्त दे॒वा दे॒वा अ॑कामयन्तो॒ भय॑ मु॒भय॑ मकामयन्त दे॒वा दे॒वा अ॑कामयन्तो॒ भय᳚म् । \newline
13. अ॒का॒म॒य॒न्तो॒ भय॑ मु॒भय॑ मकामयन्ता कामयन्तो॒ भयꣳ॒॒ सꣳ स मु॒भय॑ मकामयन्ता कामयन्तो॒ भयꣳ॒॒ सम् । \newline
14. उ॒भयꣳ॒॒ सꣳ स मु॒भय॑ मु॒भयꣳ॒॒ सं ॅवृ॑ञ्जीमहि वृञ्जीमहि॒ स मु॒भय॑ मु॒भयꣳ॒॒ सं ॅवृ॑ञ्जीमहि । \newline
15. सं ॅवृ॑ञ्जीमहि वृञ्जीमहि॒ सꣳ सं ॅवृ॑ञ्जीमहि॒ ब्रह्म॒ ब्रह्म॑ वृञ्जीमहि॒ सꣳ सं ॅवृ॑ञ्जीमहि॒ ब्रह्म॑ । \newline
16. वृ॒ञ्जी॒म॒हि॒ ब्रह्म॒ ब्रह्म॑ वृञ्जीमहि वृञ्जीमहि॒ ब्रह्म॑ च च॒ ब्रह्म॑ वृञ्जीमहि वृञ्जीमहि॒ ब्रह्म॑ च । \newline
17. ब्रह्म॑ च च॒ ब्रह्म॒ ब्रह्म॒ चान्न॒ मन्न॑म् च॒ ब्रह्म॒ ब्रह्म॒ चान्न᳚म् । \newline
18. चान्न॒ मन्न॑म् च॒ चान्न॑म् च॒ चान्न॑म् च॒ चान्न॑म् च । \newline
19. अन्न॑म् च॒ चान्न॒ मन्न॒म् चेतीति॒ चान्न॒ मन्न॒म् चेति॑ । \newline
20. चेतीति॑ च॒ चेति॒ ते त इति॑ च॒ चेति॒ ते । \newline
21. इति॒ ते त इतीति॒ त ए॒ता ए॒ता स्त इतीति॒ त ए॒ताः । \newline
22. त ए॒ता ए॒ता स्ते त ए॒ता विꣳ॑श॒तिं ॅविꣳ॑श॒ति मे॒ता स्ते त ए॒ता विꣳ॑श॒तिम् । \newline
23. ए॒ता विꣳ॑श॒तिं ॅविꣳ॑श॒ति मे॒ता ए॒ता विꣳ॑श॒तिꣳ रात्री॒ रात्री᳚र् विꣳश॒ति मे॒ता ए॒ता विꣳ॑श॒तिꣳ रात्रीः᳚ । \newline
24. विꣳ॒॒श॒तिꣳ रात्री॒ रात्री᳚र् विꣳश॒तिं ॅविꣳ॑श॒तिꣳ रात्री॑ रपश्यन् नपश्य॒न् रात्री᳚र् विꣳश॒तिं ॅविꣳ॑श॒तिꣳ रात्री॑ रपश्यन्न् । \newline
25. रात्री॑ रपश्यन् नपश्य॒न् रात्री॒ रात्री॑ रपश्य॒न् तत॒ स्ततो॑ ऽपश्य॒न् रात्री॒ रात्री॑ रपश्य॒न् ततः॑ । \newline
26. अ॒प॒श्य॒न् तत॒ स्ततो॑ ऽपश्यन् नपश्य॒न् ततो॒ वै वै ततो॑ ऽपश्यन् नपश्य॒न् ततो॒ वै । \newline
27. ततो॒ वै वै तत॒ स्ततो॒ वै ते ते वै तत॒ स्ततो॒ वै ते । \newline
28. वै ते ते वै वै त उ॒भय॑ मु॒भय॒म् ते वै वै त उ॒भय᳚म् । \newline
29. त उ॒भय॑ मु॒भय॒म् ते त उ॒भयꣳ॒॒ सꣳ स मु॒भय॒म् ते त उ॒भयꣳ॒॒ सम् । \newline
30. उ॒भयꣳ॒॒ सꣳ स मु॒भय॑ मु॒भयꣳ॒॒ स म॑वृञ्जता वृञ्जत॒ स मु॒भय॑ मु॒भयꣳ॒॒ स म॑वृञ्जत । \newline
31. स म॑वृञ्जता वृञ्जत॒ सꣳ स म॑वृञ्जत॒ ब्रह्म॒ ब्रह्मा॑ वृञ्जत॒ सꣳ स म॑वृञ्जत॒ ब्रह्म॑ । \newline
32. अ॒वृ॒ञ्ज॒त॒ ब्रह्म॒ ब्रह्मा॑ वृञ्जता वृञ्जत॒ ब्रह्म॑ च च॒ ब्रह्मा॑ वृञ्जता वृञ्जत॒ ब्रह्म॑ च । \newline
33. ब्रह्म॑ च च॒ ब्रह्म॒ ब्रह्म॒ चान्न॒ मन्न॑म् च॒ ब्रह्म॒ ब्रह्म॒ चान्न᳚म् । \newline
34. चान्न॒ मन्न॑म् च॒ चान्न॑म् च॒ चान्न॑म् च॒ चान्न॑म् च । \newline
35. अन्न॑म् च॒ चान्न॒ मन्न॑म् च ब्रह्मवर्च॒सिनो᳚ ब्रह्मवर्च॒सिन॒ श्चान्न॒ मन्न॑म् च ब्रह्मवर्च॒सिनः॑ । \newline
36. च॒ ब्र॒ह्म॒व॒र्च॒सिनो᳚ ब्रह्मवर्च॒सिन॑श्च च ब्रह्मवर्च॒सिनो᳚ ऽन्ना॒दा अ॑न्ना॒दा ब्र॑ह्मवर्च॒सिन॑श्च च ब्रह्मवर्च॒सिनो᳚ ऽन्ना॒दाः । \newline
37. ब्र॒ह्म॒व॒र्च॒सिनो᳚ ऽन्ना॒दा अ॑न्ना॒दा ब्र॑ह्मवर्च॒सिनो᳚ ब्रह्मवर्च॒सिनो᳚ ऽन्ना॒दा अ॑भवन् नभवन् नन्ना॒दा ब्र॑ह्मवर्च॒सिनो᳚ ब्रह्मवर्च॒सिनो᳚ ऽन्ना॒दा अ॑भवन्न् । \newline
38. ब्र॒ह्म॒व॒र्च॒सिन॒ इति॑ ब्रह्म - व॒र्च॒सिनः॑ । \newline
39. अ॒न्ना॒दा अ॑भवन् नभवन् नन्ना॒दा अ॑न्ना॒दा अ॑भव॒न्॒. ये ये॑ ऽभवन् नन्ना॒दा अ॑न्ना॒दा अ॑भव॒न्॒. ये । \newline
40. अ॒न्ना॒दा इत्य॑न्न - अ॒दाः । \newline
41. अ॒भ॒व॒न्॒. ये ये॑ ऽभवन् नभव॒न्॒. य ए॒व मे॒वं ॅये॑ ऽभवन् नभव॒न्॒. य ए॒वम् । \newline
42. य ए॒व मे॒वं ॅये य ए॒वं ॅवि॒द्वाꣳसो॑ वि॒द्वाꣳस॑ ए॒वं ॅये य ए॒वं ॅवि॒द्वाꣳसः॑ । \newline
43. ए॒वं ॅवि॒द्वाꣳसो॑ वि॒द्वाꣳस॑ ए॒व मे॒वं ॅवि॒द्वाꣳस॑ ए॒ता ए॒ता वि॒द्वाꣳस॑ ए॒व मे॒वं ॅवि॒द्वाꣳस॑ ए॒ताः । \newline
44. वि॒द्वाꣳस॑ ए॒ता ए॒ता वि॒द्वाꣳसो॑ वि॒द्वाꣳस॑ ए॒ता आस॑त॒ आस॑त ए॒ता वि॒द्वाꣳसो॑ वि॒द्वाꣳस॑ ए॒ता आस॑ते । \newline
45. ए॒ता आस॑त॒ आस॑त ए॒ता ए॒ता आस॑त उ॒भय॑ मु॒भय॒ मास॑त ए॒ता ए॒ता आस॑त उ॒भय᳚म् । \newline
46. आस॑त उ॒भय॑ मु॒भय॒ मास॑त॒ आस॑त उ॒भय॑ मे॒वैवोभय॒ मास॑त॒ आस॑त उ॒भय॑ मे॒व । \newline
47. उ॒भय॑ मे॒वैवोभय॑ मु॒भय॑ मे॒व सꣳ स मे॒वोभय॑ मु॒भय॑ मे॒व सम् । \newline
48. ए॒व सꣳ स मे॒वैव सं ॅवृ॑ञ्जते वृञ्जते॒ स मे॒वैव सं ॅवृ॑ञ्जते । \newline
49. सं ॅवृ॑ञ्जते वृञ्जते॒ सꣳ सं ॅवृ॑ञ्जते॒ ब्रह्म॒ ब्रह्म॑ वृञ्जते॒ सꣳ सं ॅवृ॑ञ्जते॒ ब्रह्म॑ । \newline
50. वृ॒ञ्ज॒ते॒ ब्रह्म॒ ब्रह्म॑ वृञ्जते वृञ्जते॒ ब्रह्म॑ च च॒ ब्रह्म॑ वृञ्जते वृञ्जते॒ ब्रह्म॑ च । \newline
51. ब्रह्म॑ च च॒ ब्रह्म॒ ब्रह्म॒ चान्न॒ मन्न॑म् च॒ ब्रह्म॒ ब्रह्म॒ चान्न᳚म् । \newline
52. चान्न॒ मन्न॑म् च॒ चान्न॑म् च॒ चान्न॑म् च॒ चान्न॑म् च । \newline
53. अन्न॑म् च॒ चान्न॒ मन्न॑म् च ब्रह्मवर्च॒सिनो᳚ ब्रह्मवर्च॒सिन॒ श्चान्न॒ मन्न॑म् च ब्रह्मवर्च॒सिनः॑ । \newline
54. च॒ ब्र॒ह्म॒व॒र्च॒सिनो᳚ ब्रह्मवर्च॒सिन॑श्च च ब्रह्मवर्च॒सिनो᳚ ऽन्ना॒दा अ॑न्ना॒दा ब्र॑ह्मवर्च॒सिन॑श्च च ब्रह्मवर्च॒सिनो᳚ ऽन्ना॒दाः । \newline
\pagebreak
\markright{ TS 7.3.9.2  \hfill https://www.vedavms.in \hfill}

\section{ TS 7.3.9.2 }

\textbf{TS 7.3.9.2 } \newline
\textbf{Samhita Paata} \newline

ब्रह्मवर्च॒सिनो᳚ऽन्ना॒दा भ॑वन्ति॒ द्वे वा ए॒ते वि॒राजौ॒ तयो॑रे॒व नाना॒ प्रति॑ तिष्ठन्ति विꣳ॒॒शो वै पुरु॑षो॒ दश॒ हस्त्या॑ अ॒ङ्गुल॑यो॒ दश॒ पद्या॒ यावा॑ने॒व पुरु॑ष॒स्तमा॒प्त्वोत् ति॑ष्ठन्ति॒ ज्योति॒र्गौरायु॒रिति॑ त्र्य॒हा भ॑वन्ती॒यं ॅवाव ज्योति॑र॒न्तरि॑क्षं॒ गौर॒सावायु॑रि॒माने॒व लो॒का-न॒भ्यारो॑हन्त्यभिपू॒र्वं त्र्य॒हा भ॑वन्त्यभिपू॒र्वमे॒व सु॑व॒र्गं - [  ] \newline

\textbf{Pada Paata} \newline

ब्र॒ह्म॒व॒र्च॒सिन॒ इति॑ ब्रह्म - व॒र्च॒सिनः॑ । अ॒न्ना॒दा इत्य॑न्न - अ॒दाः । भ॒व॒न्ति॒ । द्वे इति॑ । वै । ए॒ते इति॑ । वि॒राजा॒विति॑ वि-राजौ᳚ । तयोः᳚ । ए॒व । नाना᳚ । प्रतीति॑ । ति॒ष्ठ॒न्ति॒ । विꣳ॒॒शः । वै । पुरु॑षः । दश॑ । हस्त्याः᳚ । अ॒ङ्गुल॑यः । दश॑ । पद्याः᳚ । यावान्॑ । ए॒व । पुरु॑षः । तम् । आ॒प्त्वा । उदिति॑ । ति॒ष्ठ॒न्ति॒ । ज्योतिः॑ । गौः । आयुः॑ । इति॑ । त्र्य॒हा इति॑ त्रि - अ॒हाः । भ॒व॒न्ति॒ । इ॒यम् । वाव । ज्योतिः॑ । अ॒न्तरि॑क्षम् । गौः । अ॒सौ । आयुः॑ । इ॒मान् । ए॒व । लो॒कान् । अ॒भ्यारो॑ह॒न्तीत्य॑भि - आरो॑हन्ति । अ॒भि॒पू॒र्वमित्य॑भि - पू॒र्वम् । त्र्य॒हा इति॑ त्रि - अ॒हाः । भ॒व॒न्ति॒ । अ॒भि॒पू॒र्वमित्य॑भि - पू॒र्वम् । ए॒व । सु॒व॒र्गमिति॑ सुवः - गम् ।  \newline


\textbf{Krama Paata} \newline

ब्र॒ह्म॒व॒र्च॒सिनो᳚ऽन्ना॒दाः । ब्र॒ह्म॒व॒र्च॒सिन॒ इति॑ ब्रह्म - व॒र्च॒सिनः॑ । अ॒न्ना॒दा भ॑वन्ति । अ॒न्ना॒दा इत्य॑न्न - अ॒दाः । भ॒व॒न्ति॒ द्वे । द्वे वै । द्वे इति॒ द्वे । वा ए॒ते । ए॒ते वि॒राजौ᳚ । ए॒ते इत्ये॒ते । वि॒राजौ॒ तयोः᳚ । वि॒राजा॒विति॑ वि - राजौ᳚ । तयो॑रे॒व । ए॒व नाना᳚ । नाना॒ प्रति॑ । प्रति॑ तिष्ठन्ति । ति॒ष्ठ॒न्ति॒ विꣳ॒॒शः । विꣳ॒॒शो वै । वै पुरु॑षः । पुरु॑षो॒ दश॑ । दश॒ हस्त्याः᳚ । हस्त्या॑ अ॒ङ्‍गुल॑यः । अ॒ङ्‍गुल॑यो॒ दश॑ । दश॒ पद्याः᳚ । पद्या॒ यावान्॑ । यावा॑ने॒व । ए॒व पुरु॑षः । पुरु॑ष॒स्तम् । तमा॒प्त्वा । आ॒प्त्वोत् । उत् ति॑ष्ठन्ति । ति॒ष्ठ॒न्ति॒ ज्योतिः॑ । ज्योति॒र् गौः । गौरायुः॑ । आयु॒रिति॑ । इति॑ त्र्य॒हाः । त्र्य॒हा भ॑वन्ति । त्र्य॒हा इति॑ त्रि - अ॒हाः । भ॒व॒न्ती॒यम् । इ॒यम् ॅवाव । वाव ज्योतिः॑ । ज्योति॑र॒न्तरि॑क्षम् । अ॒न्तरि॑क्ष॒म् गौः । गौर॒सौ । अ॒सावायुः॑ । आयु॑रि॒मान् । इ॒माने॒व । ए॒व लो॒कान् । लो॒कान॒भ्यारो॑हन्ति । अ॒भ्यारो॑हन्त्यभिपू॒र्वम् । अ॒भ्यारो॑ह॒न्तीत्य॑भि - आरो॑हन्ति । अ॒भि॒पू॒र्वम् त्र्य॒हाः । अ॒भि॒पू॒र्वमित्य॑भि - पू॒र्वम् । त्र्य॒हा भ॑वन्ति । त्र्य॒हा इति॑ त्रि - अ॒हाः । भ॒व॒न्त्य॒भि॒पू॒र्वम् । अ॒भि॒पू॒र्वमे॒व । अ॒भि॒पू॒र्वमित्य॑भि - पू॒र्वम् । ए॒व सु॑व॒र्गम् । सु॒व॒र्गम् ॅलो॒कम् । सु॒व॒र्गमिति॑ सुवः - गम् \newline

\textbf{Jatai Paata} \newline

1. ब्र॒ह्म॒व॒र्च॒सिनो᳚ ऽन्ना॒दा अ॑न्ना॒दा ब्र॑ह्मवर्च॒सिनो᳚ ब्रह्मवर्च॒सिनो᳚ ऽन्ना॒दाः । \newline
2. ब्र॒ह्म॒व॒र्च॒सिन॒ इति॑ ब्रह्म - व॒र्च॒सिनः॑ । \newline
3. अ॒न्ना॒दा भ॑वन्ति भव न्त्यन्ना॒दा अ॑न्ना॒दा भ॑वन्ति । \newline
4. अ॒न्ना॒दा इत्य॑न्न - अ॒दाः । \newline
5. भ॒व॒न्ति॒ द्वे द्वे भ॑वन्ति भवन्ति॒ द्वे । \newline
6. द्वे वै वै द्वे द्वे वै । \newline
7. द्वे इति॒ द्वे । \newline
8. वा ए॒ते ए॒ते वै वा ए॒ते । \newline
9. ए॒ते वि॒राजौ॑ वि॒राजा॑ वे॒ते ए॒ते वि॒राजौ᳚ । \newline
10. ए॒ते इत्ये॒ते । \newline
11. वि॒राजौ॒ तयो॒ स्तयो᳚र् वि॒राजौ॑ वि॒राजौ॒ तयोः᳚ । \newline
12. वि॒राजा॒विति॑ वि - राजौ᳚ । \newline
13. तयो॑ रे॒वैव तयो॒ स्तयो॑ रे॒व । \newline
14. ए॒व नाना॒ नानै॒वैव नाना᳚ । \newline
15. नाना॒ प्रति॒ प्रति॒ नाना॒ नाना॒ प्रति॑ । \newline
16. प्रति॑ तिष्ठन्ति तिष्ठन्ति॒ प्रति॒ प्रति॑ तिष्ठन्ति । \newline
17. ति॒ष्ठ॒न्ति॒ विꣳ॒॒शो विꣳ॒॒श स्ति॑ष्ठन्ति तिष्ठन्ति विꣳ॒॒शः । \newline
18. विꣳ॒॒शो वै वै विꣳ॒॒शो विꣳ॒॒शो वै । \newline
19. वै पुरु॑षः॒ पुरु॑षो॒ वै वै पुरु॑षः । \newline
20. पुरु॑षो॒ दश॒ दश॒ पुरु॑षः॒ पुरु॑षो॒ दश॑ । \newline
21. दश॒ हस्त्या॒ हस्त्या॒ दश॒ दश॒ हस्त्याः᳚ । \newline
22. हस्त्या॑ अ॒ङ्गुल॑यो॒ ऽङ्गुल॑यो॒ हस्त्या॒ हस्त्या॑ अ॒ङ्गुल॑यः । \newline
23. अ॒ङ्गुल॑यो॒ दश॒ दशा॒ ङ्गुल॑यो॒ ऽङ्गुल॑यो॒ दश॑ । \newline
24. दश॒ पद्याः॒ पद्या॒ दश॒ दश॒ पद्याः᳚ । \newline
25. पद्या॒ यावा॒न्॒. यावा॒न् पद्याः॒ पद्या॒ यावान्॑ । \newline
26. यावा॑ने॒वैव यावा॒न्॒. यावा॑ने॒व । \newline
27. ए॒व पुरु॑षः॒ पुरु॑ष ए॒वैव पुरु॑षः । \newline
28. पुरु॑ष॒ स्तम् तम् पुरु॑षः॒ पुरु॑ष॒ स्तम् । \newline
29. त मा॒प्त्वा ऽऽप्त्वा तम् त मा॒प्त्वा । \newline
30. आ॒प्त्वोदु दा॒प्त्वा ऽऽप्त्वोत् । \newline
31. उत् ति॑ष्ठन्ति तिष्ठ॒ न्त्युदुत् ति॑ष्ठन्ति । \newline
32. ति॒ष्ठ॒न्ति॒ ज्योति॒र् ज्योति॑ स्तिष्ठन्ति तिष्ठन्ति॒ ज्योतिः॑ । \newline
33. ज्योति॒र् गौर् गौर् ज्योति॒र् ज्योति॒र् गौः । \newline
34. गौ रायु॒ रायु॒र् गौर् गौ रायुः॑ । \newline
35. आयु॒ रिती त्यायु॒ रायु॒ रिति॑ । \newline
36. इति॑ त्र्य॒हा स्त्र्य॒हा इतीति॑ त्र्य॒हाः । \newline
37. त्र्य॒हा भ॑वन्ति भवन्ति त्र्य॒हा स्त्र्य॒हा भ॑वन्ति । \newline
38. त्र्य॒हा इति॑ त्रि - अ॒हाः । \newline
39. भ॒व॒न्ती॒य मि॒यम् भ॑वन्ति भवन्ती॒यम् । \newline
40. इ॒यं ॅवाव वावेय मि॒यं ॅवाव । \newline
41. वाव ज्योति॒र् ज्योति॒र् वाव वाव ज्योतिः॑ । \newline
42. ज्योति॑ र॒न्तरि॑क्ष म॒न्तरि॑क्ष॒म् ज्योति॒र् ज्योति॑ र॒न्तरि॑क्षम् । \newline
43. अ॒न्तरि॑क्ष॒म् गौर् गौ र॒न्तरि॑क्ष म॒न्तरि॑क्ष॒म् गौः । \newline
44. गौ र॒सा व॒सौ गौर् गौ र॒सौ । \newline
45. अ॒सा वायु॒ रायु॑ र॒सा व॒सा वायुः॑ । \newline
46. आयु॑ रि॒मा नि॒मा नायु॒ रायु॑ रि॒मान् । \newline
47. इ॒मा ने॒वैवेमा नि॒माने॒व । \newline
48. ए॒व लो॒कान् ॅलो॒का ने॒वैव लो॒कान् । \newline
49. लो॒का न॒भ्यारो॑ह न्त्य॒भ्यारो॑हन्ति लो॒कान् ॅलो॒का न॒भ्यारो॑हन्ति । \newline
50. अ॒भ्यारो॑ह न्त्यभिपू॒र्व म॑भिपू॒र्व म॒भ्यारो॑ह न्त्य॒भ्यारो॑ह न्त्यभिपू॒र्वम् । \newline
51. अ॒भ्यारो॑ह॒न्तीत्य॑भि - आरो॑हन्ति । \newline
52. अ॒भि॒पू॒र्वम् त्र्य॒हा स्त्र्य॒हा अ॑भिपू॒र्व म॑भिपू॒र्वम् त्र्य॒हाः । \newline
53. अ॒भि॒पू॒र्वमित्य॑भि - पू॒र्वम् । \newline
54. त्र्य॒हा भ॑वन्ति भवन्ति त्र्य॒हा स्त्र्य॒हा भ॑वन्ति । \newline
55. त्र्य॒हा इति॑ त्रि - अ॒हाः । \newline
56. भ॒व॒ न्त्य॒भि॒पू॒र्व म॑भिपू॒र्वम् भ॑वन्ति भव न्त्यभिपू॒र्वम् । \newline
57. अ॒भि॒पू॒र्व मे॒वैवाभि॑पू॒र्व म॑भिपू॒र्व मे॒व । \newline
58. अ॒भि॒पू॒र्वमित्य॑भि - पू॒र्वम् । \newline
59. ए॒व सु॑व॒र्गꣳ सु॑व॒र्ग मे॒वैव सु॑व॒र्गम् । \newline
60. सु॒व॒र्गम् ॅलो॒कम् ॅलो॒कꣳ सु॑व॒र्गꣳ सु॑व॒र्गम् ॅलो॒कम् । \newline
61. सु॒व॒र्गमिति॑ सुवः - गम् । \newline

\textbf{Ghana Paata } \newline

1. ब्र॒ह्म॒व॒र्च॒सिनो᳚ ऽन्ना॒दा अ॑न्ना॒दा ब्र॑ह्मवर्च॒सिनो᳚ ब्रह्मवर्च॒सिनो᳚ ऽन्ना॒दा भ॑वन्ति भव न्त्यन्ना॒दा ब्र॑ह्मवर्च॒सिनो᳚ ब्रह्मवर्च॒सिनो᳚ ऽन्ना॒दा भ॑वन्ति । \newline
2. ब्र॒ह्म॒व॒र्च॒सिन॒ इति॑ ब्रह्म - व॒र्च॒सिनः॑ । \newline
3. अ॒न्ना॒दा भ॑वन्ति भव न्त्यन्ना॒दा अ॑न्ना॒दा भ॑वन्ति॒ द्वे द्वे भ॑व न्त्यन्ना॒दा अ॑न्ना॒दा भ॑वन्ति॒ द्वे । \newline
4. अ॒न्ना॒दा इत्य॑न्न - अ॒दाः । \newline
5. भ॒व॒न्ति॒ द्वे द्वे भ॑वन्ति भवन्ति॒ द्वे वै वै द्वे भ॑वन्ति भवन्ति॒ द्वे वै । \newline
6. द्वे वै वै द्वे द्वे वा ए॒ते ए॒ते वै द्वे द्वे वा ए॒ते । \newline
7. द्वे इति॒ द्वे । \newline
8. वा ए॒ते ए॒ते वै वा ए॒ते वि॒राजौ॑ वि॒राजा॑ वे॒ते वै वा ए॒ते वि॒राजौ᳚ । \newline
9. ए॒ते वि॒राजौ॑ वि॒राजा॑ वे॒ते ए॒ते वि॒राजौ॒ तयो॒ स्तयो᳚र् वि॒राजा॑ वे॒ते ए॒ते वि॒राजौ॒ तयोः᳚ । \newline
10. ए॒ते इत्ये॒ते । \newline
11. वि॒राजौ॒ तयो॒ स्तयो᳚र् वि॒राजौ॑ वि॒राजौ॒ तयो॑ रे॒वैव तयो᳚र् वि॒राजौ॑ वि॒राजौ॒ तयो॑ रे॒व । \newline
12. वि॒राजा॒विति॑ वि - राजौ᳚ । \newline
13. तयो॑ रे॒वैव तयो॒ स्तयो॑ रे॒व नाना॒ नानै॒व तयो॒ स्तयो॑ रे॒व नाना᳚ । \newline
14. ए॒व नाना॒ नानै॒ वैव नाना॒ प्रति॒ प्रति॒ नानै॒ वैव नाना॒ प्रति॑ । \newline
15. नाना॒ प्रति॒ प्रति॒ नाना॒ नाना॒ प्रति॑ तिष्ठन्ति तिष्ठन्ति॒ प्रति॒ नाना॒ नाना॒ प्रति॑ तिष्ठन्ति । \newline
16. प्रति॑ तिष्ठन्ति तिष्ठन्ति॒ प्रति॒ प्रति॑ तिष्ठन्ति विꣳ॒॒शो विꣳ॒॒श स्ति॑ष्ठन्ति॒ प्रति॒ प्रति॑ तिष्ठन्ति विꣳ॒॒शः । \newline
17. ति॒ष्ठ॒न्ति॒ विꣳ॒॒शो विꣳ॒॒श स्ति॑ष्ठन्ति तिष्ठन्ति विꣳ॒॒शो वै वै विꣳ॒॒श स्ति॑ष्ठन्ति तिष्ठन्ति विꣳ॒॒शो वै । \newline
18. विꣳ॒॒शो वै वै विꣳ॒॒शो विꣳ॒॒शो वै पुरु॑षः॒ पुरु॑षो॒ वै विꣳ॒॒शो विꣳ॒॒शो वै पुरु॑षः । \newline
19. वै पुरु॑षः॒ पुरु॑षो॒ वै वै पुरु॑षो॒ दश॒ दश॒ पुरु॑षो॒ वै वै पुरु॑षो॒ दश॑ । \newline
20. पुरु॑षो॒ दश॒ दश॒ पुरु॑षः॒ पुरु॑षो॒ दश॒ हस्त्या॒ हस्त्या॒ दश॒ पुरु॑षः॒ पुरु॑षो॒ दश॒ हस्त्याः᳚ । \newline
21. दश॒ हस्त्या॒ हस्त्या॒ दश॒ दश॒ हस्त्या॑ अ॒ङ्गुल॑यो॒ ऽङ्गुल॑यो॒ हस्त्या॒ दश॒ दश॒ हस्त्या॑ अ॒ङ्गुल॑यः । \newline
22. हस्त्या॑ अ॒ङ्गुल॑यो॒ ऽङ्गुल॑यो॒ हस्त्या॒ हस्त्या॑ अ॒ङ्गुल॑यो॒ दश॒ दशा॒ ङ्गुल॑यो॒ हस्त्या॒ हस्त्या॑ अ॒ङ्गुल॑यो॒ दश॑ । \newline
23. अ॒ङ्गुल॑यो॒ दश॒ दशा॒ ङ्गुल॑यो॒ ऽङ्गुल॑यो॒ दश॒ पद्याः॒ पद्या॒ दशा॒ ङ्गुल॑यो॒ ऽङ्गुल॑यो॒ दश॒ पद्याः᳚ । \newline
24. दश॒ पद्याः॒ पद्या॒ दश॒ दश॒ पद्या॒ यावा॒न्॒. यावा॒न् पद्या॒ दश॒ दश॒ पद्या॒ यावान्॑ । \newline
25. पद्या॒ यावा॒न्॒. यावा॒न् पद्याः॒ पद्या॒ यावा॑ ने॒वैव यावा॒न् पद्याः॒ पद्या॒ यावा॑ने॒व । \newline
26. यावा॑ ने॒वैव यावा॒न्॒. यावा॑ने॒व पुरु॑षः॒ पुरु॑ष ए॒व यावा॒न्॒. यावा॑ने॒व पुरु॑षः । \newline
27. ए॒व पुरु॑षः॒ पुरु॑ष ए॒वैव पुरु॑ष॒ स्तम् तम् पुरु॑ष ए॒वैव पुरु॑ष॒ स्तम् । \newline
28. पुरु॑ष॒ स्तम् तम् पुरु॑षः॒ पुरु॑ष॒ स्त मा॒प्त्वा ऽऽप्त्वा तम् पुरु॑षः॒ पुरु॑ष॒ स्त मा॒प्त्वा । \newline
29. त मा॒प्त्वा ऽऽप्त्वा तम् त मा॒प्त्वो दुदाप्त्वा तम् त मा॒प्त्वोत् । \newline
30. आ॒प्त्वो दुदा॒प्त्वा ऽऽप्त्वोत् ति॑ष्ठन्ति तिष्ठ॒ न्त्युदा॒प्त्वा ऽऽप्त्वोत् ति॑ष्ठन्ति । \newline
31. उत् ति॑ष्ठन्ति तिष्ठ॒ न्त्युदुत् ति॑ष्ठन्ति॒ ज्योति॒र् ज्योति॑ स्तिष्ठ॒ न्त्युदुत् ति॑ष्ठन्ति॒ ज्योतिः॑ । \newline
32. ति॒ष्ठ॒न्ति॒ ज्योति॒र् ज्योति॑ स्तिष्ठन्ति तिष्ठन्ति॒ ज्योति॒र् गौर् गौर् ज्योति॑ स्तिष्ठन्ति तिष्ठन्ति॒ ज्योति॒र् गौः । \newline
33. ज्योति॒र् गौर् गौर् ज्योति॒र् ज्योति॒र् गौरायु॒ रायु॒र् गौर् ज्योति॒र् ज्योति॒र् गौ रायुः॑ । \newline
34. गौ रायु॒ रायु॒र् गौर् गौ रायु॒ रिती त्यायु॒र् गौर् गौ रायु॒ रिति॑ । \newline
35. आयु॒रिती त्यायु॒ रायु॒ रिति॑ त्र्य॒हा स्त्र्य॒हा इत्यायु॒ रायु॒ रिति॑ त्र्य॒हाः । \newline
36. इति॑ त्र्य॒हा स्त्र्य॒हा इतीति॑ त्र्य॒हा भ॑वन्ति भवन्ति त्र्य॒हा इतीति॑ त्र्य॒हा भ॑वन्ति । \newline
37. त्र्य॒हा भ॑वन्ति भवन्ति त्र्य॒हा स्त्र्य॒हा भ॑वन्ती॒य मि॒यम् भ॑वन्ति त्र्य॒हा स्त्र्य॒हा भ॑वन्ती॒यम् । \newline
38. त्र्य॒हा इति॑ त्रि - अ॒हाः । \newline
39. भ॒व॒न्ती॒य मि॒यम् भ॑वन्ति भवन्ती॒यं ॅवाव वावेयम् भ॑वन्ति भवन्ती॒यं ॅवाव । \newline
40. इ॒यं ॅवाव वावेय मि॒यं ॅवाव ज्योति॒र् ज्योति॒र् वावेय मि॒यं ॅवाव ज्योतिः॑ । \newline
41. वाव ज्योति॒र् ज्योति॒र् वाव वाव ज्योति॑ र॒न्तरि॑क्ष म॒न्तरि॑क्ष॒म् ज्योति॒र् वाव वाव ज्योति॑ र॒न्तरि॑क्षम् । \newline
42. ज्योति॑ र॒न्तरि॑क्ष म॒न्तरि॑क्ष॒म् ज्योति॒र् ज्योति॑ र॒न्तरि॑क्ष॒म् गौर् गौ र॒न्तरि॑क्ष॒म् ज्योति॒र् ज्योति॑ र॒न्तरि॑क्ष॒म् गौः । \newline
43. अ॒न्तरि॑क्ष॒म् गौर् गौ र॒न्तरि॑क्ष म॒न्तरि॑क्ष॒म् गौ र॒सा व॒सौ गौ र॒न्तरि॑क्ष म॒न्तरि॑क्ष॒म् गौ र॒सौ । \newline
44. गौ र॒सा व॒सौ गौर् गौ र॒सा वायु॒ रायु॑ र॒सौ गौर् गौ र॒सा वायुः॑ । \newline
45. अ॒सा वायु॒ रायु॑ र॒सा व॒सा वायु॑ रि॒मा नि॒मा नायु॑ र॒सा व॒सा वायु॑ रि॒मान् । \newline
46. आयु॑ रि॒मा नि॒मा नायु॒ रायु॑ रि॒मा ने॒वैवेमा नायु॒ रायु॑ रि॒मा ने॒व । \newline
47. इ॒मा ने॒वैवेमा नि॒मा ने॒व लो॒कान् ॅलो॒का ने॒वेमा नि॒मा ने॒व लो॒कान् । \newline
48. ए॒व लो॒कान् ॅलो॒का ने॒वैव लो॒का न॒भ्यारो॑ह न्त्य॒भ्यारो॑हन्ति लो॒का ने॒वैव लो॒का न॒भ्यारो॑हन्ति । \newline
49. लो॒का न॒भ्यारो॑ह न्त्य॒भ्यारो॑हन्ति लो॒कान् ॅलो॒का न॒भ्यारो॑ह न्त्यभिपू॒र्व म॑भिपू॒र्व म॒भ्यारो॑हन्ति लो॒कान् ॅलो॒का न॒भ्यारो॑ह न्त्यभिपू॒र्वम् । \newline
50. अ॒भ्यारो॑ह न्त्यभिपू॒र्व म॑भिपू॒र्व म॒भ्यारो॑ह न्त्य॒भ्यारो॑ह न्त्यभिपू॒र्वम् त्र्य॒हा स्त्र्य॒हा अ॑भिपू॒र्व म॒भ्यारो॑ह न्त्य॒भ्यारो॑ह न्त्यभिपू॒र्वम् त्र्य॒हाः । \newline
51. अ॒भ्यारो॑ह॒न्तीत्य॑भि - आरो॑हन्ति । \newline
52. अ॒भि॒पू॒र्वम् त्र्य॒हा स्त्र्य॒हा अ॑भिपू॒र्व म॑भिपू॒र्वम् त्र्य॒हा भ॑वन्ति भवन्ति त्र्य॒हा अ॑भिपू॒र्व म॑भिपू॒र्वम् त्र्य॒हा भ॑वन्ति । \newline
53. अ॒भि॒पू॒र्वमित्य॑भि - पू॒र्वम् । \newline
54. त्र्य॒हा भ॑वन्ति भवन्ति त्र्य॒हा स्त्र्य॒हा भ॑व न्त्यभिपू॒र्व म॑भिपू॒र्वम् भ॑वन्ति त्र्य॒हा स्त्र्य॒हा भ॑व न्त्यभिपू॒र्वम् । \newline
55. त्र्य॒हा इति॑ त्रि - अ॒हाः । \newline
56. भ॒व॒ न्त्य॒भि॒पू॒र्व म॑भिपू॒र्वम् भ॑वन्ति भव न्त्यभिपू॒र्व मे॒वै वाभि॑पू॒र्वम् भ॑वन्ति भव न्त्यभिपू॒र्व मे॒व । \newline
57. अ॒भि॒पू॒र्व मे॒वै वाभि॑पू॒र्व म॑भिपू॒र्व मे॒व सु॑व॒र्गꣳ सु॑व॒र्ग मे॒वा भि॑पू॒र्व म॑भिपू॒र्व मे॒व सु॑व॒र्गम् । \newline
58. अ॒भि॒पू॒र्वमित्य॑भि - पू॒र्वम् । \newline
59. ए॒व सु॑व॒र्गꣳ सु॑व॒र्ग मे॒वैव सु॑व॒र्गम् ॅलो॒कम् ॅलो॒कꣳ सु॑व॒र्ग मे॒वैव सु॑व॒र्गम् ॅलो॒कम् । \newline
60. सु॒व॒र्गम् ॅलो॒कम् ॅलो॒कꣳ सु॑व॒र्गꣳ सु॑व॒र्गम् ॅलो॒क म॒भ्यारो॑ह न्त्य॒भ्यारो॑हन्ति लो॒कꣳ सु॑व॒र्गꣳ सु॑व॒र्गम् ॅलो॒क म॒भ्यारो॑हन्ति । \newline
61. सु॒व॒र्गमिति॑ सुवः - गम् । \newline
\pagebreak
\markright{ TS 7.3.9.3  \hfill https://www.vedavms.in \hfill}

\section{ TS 7.3.9.3 }

\textbf{TS 7.3.9.3 } \newline
\textbf{Samhita Paata} \newline

ॅलो॒कम॒भ्यारो॑हन्ति॒ यद॒न्यतः॑ पृ॒ष्ठानि॒ स्युर्विवि॑वधꣳ स्या॒न्मद्ध्ये॑ पृ॒ष्ठानि॑ भवन्ति सविवध॒त्वायौजो॒ वै वी॒र्यं॑ पृ॒ष्ठान्योज॑ ए॒व वी॒र्यं॑ मद्ध्य॒तो द॑धते बृहद्-रथन्त॒राभ्यां᳚ ॅयन्ती॒यं ॅवाव र॑थन्त॒रम॒सौ बृ॒हदा॒भ्यामे॒व य॒न्त्यथो॑ अ॒नयो॑रे॒व प्रति॑ तिष्ठन्त्ये॒ते वै य॒ज्ञ्स्या᳚ञ्ज॒साय॑नी स्रु॒ती ताभ्या॑मे॒व सु॑व॒र्गं ॅलो॒कं ॅय॑न्ति॒ परा᳚ञ्चो॒ वा ए॒ते सु॑व॒र्गं ( ) ॅलो॒कम॒भ्यारो॑हन्ति॒ ये प॑रा॒चीना॑नि पृ॒ष्ठान्यु॑प॒यन्ति॑ प्र॒त्यङ् त्र्य॒हो भ॑वति प्र॒त्यव॑रूढ्या॒ अथो॒ प्रति॑ष्ठित्या उ॒भयो᳚र्लो॒कयोर्॑. ऋ॒द्ध्वोत् ति॑ष्ठन्त्यतिरा॒त्राव॒भितो॑ भवतो ब्रह्मवर्च॒स-स्या॒न्नाद्य॑स्य॒ परि॑गृहीत्यै ॥ \newline

\textbf{Pada Paata} \newline

लो॒कम् । अ॒भ्यारो॑ह॒न्तीत्य॑भि-आरो॑हन्ति । यत् । अ॒न्यतः॑ । पृ॒ष्ठानि॑ । स्युः । विवि॑वध॒मिति॒ वि - वि॒व॒ध॒म् । स्या॒त् । मद्ध्ये᳚ । पृ॒ष्ठानि॑ । भ॒व॒न्ति॒ । स॒वि॒व॒ध॒त्वायेति॑ सविवध - त्वाय॑ । ओजः॑ । वै । वी॒र्य᳚म् । पृ॒ष्ठानि॑ । ओजः॑ । ए॒व । वी॒र्य᳚म् । म॒द्ध्य॒तः । द॒ध॒ते॒ । बृ॒ह॒द्र॒थ॒न्त॒राभ्या॒मिति॑ बृहत् - र॒थ॒न्त॒राभ्या᳚म् । य॒न्ति॒ । इ॒यम् । वाव । र॒थ॒न्त॒रमिति॑ रथं - त॒रम् । अ॒सौ । बृ॒हत् । आ॒भ्याम् । ए॒व । य॒न्ति॒ । अथो॒ इति॑ । अ॒नयोः᳚ । ए॒व । प्रतीति॑ । ति॒ष्ठ॒न्ति॒ । ए॒ते इति॑ । वै । य॒ज्ञ्स्य॑ । अ॒ञ्ज॒साय॑नी॒ इत्य॑ञ्जसा - अय॑नी । स्रु॒ती इति॑ । ताभ्या᳚म् । ए॒व । सु॒व॒र्गमिति॑ सुवः - गम् । लो॒कम् । य॒न्ति॒ । परा᳚ञ्चः । वै । ए॒ते । सु॒व॒र्गमिति॑ सुवः - गम् ( ) । लो॒कम् । अ॒भ्यारो॑ह॒न्तीत्य॑भि - आरो॑हन्ति । ये । प॒रा॒चीना॑नि । पृ॒ष्ठानि॑ । उ॒प॒यन्तीत्यु॑प - यन्ति॑ । प्र॒त्यङ् । त्र्य॒ह इति॑ त्रि - अ॒हः । भ॒व॒ति॒ । प्र॒त्यव॑रूढ्या॒ इति॑ प्रति - अव॑रूढ्यै । अथो॒ इति॑ । प्रति॑ष्ठित्या॒ इति॒ प्रति॑ - स्थि॒त्यै॒ । उ॒भयोः᳚ । लो॒कयोः᳚ । ऋ॒द्ध्वा । उदिति॑ । ति॒ष्ठ॒न्ति॒ । अ॒ति॒रा॒त्रावित्य॑ति - रा॒त्रौ । अ॒भितः॑ । भ॒व॒तः॒ । ब्र॒ह्म॒व॒र्च॒सस्येति॑ ब्रह्म - व॒र्च॒सस्य॑ । अ॒न्नाद्य॒स्येत्य॑न्न - अद्य॑स्य । परि॑गृहीत्या॒ इति॒ परि॑ - गृ॒ही॒त्यै॒ ॥  \newline


\textbf{Krama Paata} \newline

लो॒कम॒भ्यारो॑हन्ति । अ॒भ्यारो॑हन्ति॒ यत् । अ॒भ्यारो॑ह॒न्तीत्य॑भि - आरो॑हन्ति । यद॒न्यतः॑ । अ॒न्यतः॑ पृ॒ष्ठानि॑ । पृ॒ष्ठानि॒ स्युः । स्युर् विवि॑वधम् । विवि॑वधꣳ स्यात् । विवि॑वध॒मिति॒ वि - वि॒व॒ध॒म् । स्या॒न् मद्ध्ये᳚ । मद्ध्ये॑ पृ॒ष्ठानि॑ । पृ॒ष्ठानि॑ भवन्ति । भ॒व॒न्ति॒ स॒वि॒व॒ध॒त्वाय॑ । स॒वि॒व॒ध॒त्वायौजः॑ । स॒वि॒व॒ध॒त्वायेति॑ सविवध - त्वाय॑ । ओजो॒ वै । वै वी॒र्य᳚म् । वी॒र्य॑म् पृ॒ष्ठानि॑ । पृ॒ष्ठान्योजः॑ । ओज॑ ए॒व । ए॒व वी॒र्य᳚म् । वी॒र्य॑म् मद्ध्य॒तः । म॒द्ध्य॒तो द॑धते । द॒ध॒ते॒ बृ॒ह॒द्‍र॒थ॒न्त॒राभ्या᳚म् । बृ॒ह॒द्‍र॒थ॒न्त॒राभ्या᳚म् ॅयन्ति । बृ॒ह॒द्‍र॒थ॒न्त॒राभ्या॒मिति॑ बृहत् - र॒थ॒न्त॒राभ्या᳚म् । य॒न्ती॒यम् । इ॒यम् ॅवाव । वाव र॑थन्त॒रम् । र॒थ॒न्त॒रम॒सौ । र॒थ॒न्त॒रमिति॑ रथम् - त॒रम् । अ॒सौ बृ॒हत् । बृ॒हदा॒भ्याम् । आ॒भ्यामे॒व । ए॒व य॑न्ति । य॒न्त्यथो᳚ । अथो॑ अ॒नयोः᳚ । अथो॒ इत्यथो᳚ । अ॒नयो॑रे॒व । ए॒व प्रति॑ । प्रति॑ तिष्ठन्ति । ति॒ष्ठ॒न्त्ये॒ते । ए॒ते वै । ए॒ते इत्ये॒ते । वै य॒ज्ञ्स्य॑ । य॒ज्ञ्स्या᳚ञ्ज॒साय॑नी । अ॒ञ्ज॒साय॑नी स्रु॒ती । अ॒ञ्ज॒साय॑नी॒ इत्य॑ञ्जसा - अय॑नी । स्रु॒ती ताभ्या᳚म् । स्रु॒ती इति॑ स्रु॒ती । ताभ्या॑मे॒व । ए॒व सु॑व॒र्गम् । सु॒व॒र्गम् ॅलो॒कम् । सु॒व॒र्गमिति॑ सुवः - गम् । लो॒कम् ॅय॑न्ति । य॒न्ति॒ परा᳚ञ्चः । परा᳚ञ्चो॒ वै । वा ए॒ते । ए॒ते सु॑व॒र्गम् ( ) । सु॒व॒र्गम् ॅलो॒कम् । सु॒व॒र्गमिति॑ सुवः - गम् । लो॒कम॒भ्यारो॑हन्ति । अ॒भ्यारो॑हन्ति॒ ये । अ॒भ्यारो॑ह॒न्तीत्य॑भि - आरो॑हन्ति । ये प॑रा॒चीना॑नि । प॒रा॒चीना॑नि पृ॒ष्ठानि॑ । पृ॒ष्ठान्यु॑प॒यन्ति॑ । उ॒प॒यन्ति॑ प्र॒त्यङ्‍ङ् । उ॒प॒यन्तीत्यु॑प - यन्ति॑ । प्र॒त्यङ् त्र्य॒हः । त्र्य॒हो भ॑वति । त्र्य॒ह इति॑ त्रि - अ॒हः । भ॒व॒ति॒ प्र॒त्यव॑रूढ्‍यै । प्र॒त्यव॑रूढ्‍या॒ अथो᳚ । प्र॒त्यव॑रूढ्‍या॒ इति॑ प्रति - अव॑रूढ्‍यै । अथो॒ प्रति॑ष्ठित्यै । अथो॒ इत्यथो᳚ । प्रति॑ष्ठित्या उ॒भयोः᳚ । प्रति॑ष्ठित्या॒ इति॒ प्रति॑ - स्थि॒त्यै॒ । उ॒भयो᳚र् लो॒कयोः᳚ । लो॒कयोर्॑. ऋ॒द्ध्वा । ऋ॒द्ध्वोत् । उत् ति॑ष्ठन्ति । ति॒ष्ठ॒न्त्य॒ति॒रा॒त्रौ । अ॒ति॒रा॒त्राव॒भितः॑ । अ॒ति॒रा॒त्रावित्य॑ति - रा॒त्रौ । अ॒भितो॑ भवतः । भ॒व॒तो॒ ब्र॒ह्म॒व॒र्च॒सस्य॑ । ब्र॒ह्म॒व॒र्च॒सस्या॒न्नाद्य॑स्य । ब्र॒ह्म॒व॒र्च॒सस्येति॑ ब्रह्म - व॒र्च॒सस्य॑ । अ॒न्नाद्य॑स्य॒ परि॑गृहीत्यै । अ॒न्नाद्य॒स्येत्य॑न्न - अद्य॑स्य । परि॑गृहीत्या॒ इति॒ परि॑ - गृ॒ही॒त्यै॒ । \newline

\textbf{Jatai Paata} \newline

1. लो॒क म॒भ्यारो॑ह न्त्य॒भ्यारो॑हन्ति लो॒कम् ॅलो॒क म॒भ्यारो॑हन्ति । \newline
2. अ॒भ्यारो॑हन्ति॒ यद् यद॒भ्यारो॑ह न्त्य॒भ्यारो॑हन्ति॒ यत् । \newline
3. अ॒भ्यारो॑ह॒न्तीत्य॑भि - आरो॑हन्ति । \newline
4. यद॒न्यतो॒ ऽन्यतो॒ यद् यद॒न्यतः॑ । \newline
5. अ॒न्यतः॑ पृ॒ष्ठानि॑ पृ॒ष्ठा न्य॒न्यतो॒ ऽन्यतः॑ पृ॒ष्ठानि॑ । \newline
6. पृ॒ष्ठानि॒ स्युः स्युः पृ॒ष्ठानि॑ पृ॒ष्ठानि॒ स्युः । \newline
7. स्युर् विवि॑वधं॒ ॅविवि॑वधꣳ॒॒ स्युः स्युर् विवि॑वधम् । \newline
8. विवि॑वधꣳ स्याथ् स्या॒द् विवि॑वधं॒ ॅविवि॑वधꣳ स्यात् । \newline
9. विवि॑वध॒मिति॒ वि - वि॒व॒ध॒म् । \newline
10. स्या॒न् मद्ध्ये॒ मद्ध्ये᳚ स्याथ् स्या॒न् मद्ध्ये᳚ । \newline
11. मद्ध्ये॑ पृ॒ष्ठानि॑ पृ॒ष्ठानि॒ मद्ध्ये॒ मद्ध्ये॑ पृ॒ष्ठानि॑ । \newline
12. पृ॒ष्ठानि॑ भवन्ति भवन्ति पृ॒ष्ठानि॑ पृ॒ष्ठानि॑ भवन्ति । \newline
13. भ॒व॒न्ति॒ स॒वि॒व॒ध॒त्वाय॑ सविवध॒त्वाय॑ भवन्ति भवन्ति सविवध॒त्वाय॑ । \newline
14. स॒वि॒व॒ध॒त्वा यौज॒ ओजः॑ सविवध॒त्वाय॑ सविवध॒त्वा यौजः॑ । \newline
15. स॒वि॒व॒ध॒त्वायेति॑ सविवध - त्वाय॑ । \newline
16. ओजो॒ वै वा ओज॒ ओजो॒ वै । \newline
17. वै वी॒र्यं॑ ॅवी॒र्यं॑ ॅवै वै वी॒र्य᳚म् । \newline
18. वी॒र्य॑म् पृ॒ष्ठानि॑ पृ॒ष्ठानि॑ वी॒र्यं॑ ॅवी॒र्य॑म् पृ॒ष्ठानि॑ । \newline
19. पृ॒ष्ठा न्योज॒ ओजः॑ पृ॒ष्ठानि॑ पृ॒ष्ठा न्योजः॑ । \newline
20. ओज॑ ए॒वै वौज॒ ओज॑ ए॒व । \newline
21. ए॒व वी॒र्यं॑ ॅवी॒र्य॑ मे॒वैव वी॒र्य᳚म् । \newline
22. वी॒र्य॑म् मद्ध्य॒तो म॑द्ध्य॒तो वी॒र्यं॑ ॅवी॒र्य॑म् मद्ध्य॒तः । \newline
23. म॒द्ध्य॒तो द॑धते दधते मद्ध्य॒तो म॑द्ध्य॒तो द॑धते । \newline
24. द॒ध॒ते॒ बृ॒ह॒द्र॒थ॒न्त॒राभ्या᳚म् बृहद्रथन्त॒राभ्या᳚म् दधते दधते बृहद्रथन्त॒राभ्या᳚म् । \newline
25. बृ॒ह॒द्र॒थ॒न्त॒राभ्यां᳚ ॅयन्ति यन्ति बृहद्रथन्त॒राभ्या᳚म् बृहद्रथन्त॒राभ्यां᳚ ॅयन्ति । \newline
26. बृ॒ह॒द्र॒थ॒न्त॒राभ्या॒मिति॑ बृहत् - र॒थ॒न्त॒राभ्या᳚म् । \newline
27. य॒न्ती॒य मि॒यं ॅय॑न्ति यन्ती॒यम् । \newline
28. इ॒यं ॅवाव वावेय मि॒यं ॅवाव । \newline
29. वाव र॑थन्त॒रꣳ र॑थन्त॒रं ॅवाव वाव र॑थन्त॒रम् । \newline
30. र॒थ॒न्त॒र म॒सा व॒सौ र॑थन्त॒रꣳ र॑थन्त॒र म॒सौ । \newline
31. र॒थ॒न्त॒रमिति॑ रथं - त॒रम् । \newline
32. अ॒सौ बृ॒हद् बृ॒ह द॒सा व॒सौ बृ॒हत् । \newline
33. बृ॒ह दा॒भ्या मा॒भ्याम् बृ॒हद् बृ॒ह दा॒भ्याम् । \newline
34. आ॒भ्या मे॒वै वाभ्या मा॒भ्या मे॒व । \newline
35. ए॒व य॑न्ति यन्त्ये॒ वैव य॑न्ति । \newline
36. य॒न्त्यथो॒ अथो॑ यन्ति य॒न्त्यथो᳚ । \newline
37. अथो॑ अ॒नयो॑ र॒नयो॒ रथो॒ अथो॑ अ॒नयोः᳚ । \newline
38. अथो॒ इत्यथो᳚ । \newline
39. अ॒नयो॑ रे॒वै वानयो॑ र॒नयो॑ रे॒व । \newline
40. ए॒व प्रति॒ प्रत्ये॒वैव प्रति॑ । \newline
41. प्रति॑ तिष्ठन्ति तिष्ठन्ति॒ प्रति॒ प्रति॑ तिष्ठन्ति । \newline
42. ति॒ष्ठ॒न्त्ये॒ते ए॒ते ति॑ष्ठन्ति तिष्ठन्त्ये॒ते । \newline
43. ए॒ते वै वा ए॒ते ए॒ते वै । \newline
44. ए॒ते इत्ये॒ते । \newline
45. वै य॒ज्ञ्स्य॑ य॒ज्ञ्स्य॒ वै वै य॒ज्ञ्स्य॑ । \newline
46. य॒ज्ञ्स्या᳚ ञ्ज॒साय॑नी अञ्ज॒साय॑नी य॒ज्ञ्स्य॑ य॒ज्ञ्स्या᳚ ञ्ज॒साय॑नी । \newline
47. अ॒ञ्ज॒साय॑नी स्रु॒ती स्रु॒ती अ॑ञ्ज॒साय॑नी अञ्ज॒साय॑नी स्रु॒ती । \newline
48. अ॒ञ्ज॒साय॑नी॒ इत्य॑ञ्जसा - अय॑नी । \newline
49. स्रु॒ती ताभ्या॒म् ताभ्याꣳ॑ स्रु॒ती स्रु॒ती ताभ्या᳚म् । \newline
50. स्रु॒ती इति॑ स्रु॒ती । \newline
51. ताभ्या॑ मे॒वैव ताभ्या॒म् ताभ्या॑ मे॒व । \newline
52. ए॒व सु॑व॒र्गꣳ सु॑व॒र्ग मे॒वैव सु॑व॒र्गम् । \newline
53. सु॒व॒र्गम् ॅलो॒कम् ॅलो॒कꣳ सु॑व॒र्गꣳ सु॑व॒र्गम् ॅलो॒कम् । \newline
54. सु॒व॒र्गमिति॑ सुवः - गम् । \newline
55. लो॒कं ॅय॑न्ति यन्ति लो॒कम् ॅलो॒कं ॅय॑न्ति । \newline
56. य॒न्ति॒ परा᳚ञ्चः॒ परा᳚ञ्चो यन्ति यन्ति॒ परा᳚ञ्चः । \newline
57. परा᳚ञ्चो॒ वै वै परा᳚ञ्चः॒ परा᳚ञ्चो॒ वै । \newline
58. वा ए॒त ए॒ते वै वा ए॒ते । \newline
59. ए॒ते सु॑व॒र्गꣳ सु॑व॒र्ग मे॒त ए॒ते सु॑व॒र्गम् । \newline
60. सु॒व॒र्गम् ॅलो॒कम् ॅलो॒कꣳ सु॑व॒र्गꣳ सु॑व॒र्गम् ॅलो॒कम् । \newline
61. सु॒व॒र्गमिति॑ सुवः - गम् । \newline
62. लो॒क म॒भ्यारो॑ह न्त्य॒भ्यारो॑हन्ति लो॒कम् ॅलो॒क म॒भ्यारो॑हन्ति । \newline
63. अ॒भ्यारो॑हन्ति॒ ये ये᳚ ऽभ्यारो॑ह न्त्य॒भ्यारो॑हन्ति॒ ये । \newline
64. अ॒भ्यारो॑ह॒न्तीत्य॑भि - आरो॑हन्ति । \newline
65. ये प॑रा॒चीना॑नि परा॒चीना॑नि॒ ये ये प॑रा॒चीना॑नि । \newline
66. प॒रा॒चीना॑नि पृ॒ष्ठानि॑ पृ॒ष्ठानि॑ परा॒चीना॑नि परा॒चीना॑नि पृ॒ष्ठानि॑ । \newline
67. पृ॒ष्ठा न्यु॑प॒य न्त्यु॑प॒यन्ति॑ पृ॒ष्ठानि॑ पृ॒ष्ठा न्यु॑प॒यन्ति॑ । \newline
68. उ॒प॒यन्ति॑ प्र॒त्यङ् प्र॒त्यङ् ङु॑प॒य न्त्यु॑प॒यन्ति॑ प्र॒त्यङ् । \newline
69. उ॒प॒यन्तीत्यु॑प - यन्ति॑ । \newline
70. प्र॒त्यङ् त्र्य॒ह स्त्र्य॒हः प्र॒त्यङ् प्र॒त्यङ् त्र्य॒हः । \newline
71. त्र्य॒हो भ॑वति भवति त्र्य॒ह स्त्र्य॒हो भ॑वति । \newline
72. त्र्य॒ह इति॑ त्रि - अ॒हः । \newline
73. भ॒व॒ति॒ प्र॒त्यव॑रूढ्यै प्र॒त्यव॑रूढ्यै भवति भवति प्र॒त्यव॑रूढ्यै । \newline
74. प्र॒त्यव॑रूढ्या॒ अथो॒ अथो᳚ प्र॒त्यव॑रूढ्यै प्र॒त्यव॑रूढ्या॒ अथो᳚ । \newline
75. प्र॒त्यव॑रूढ्या॒ इति॑ प्रति - अव॑रूढ्यै । \newline
76. अथो॒ प्रति॑ष्ठित्यै॒ प्रति॑ष्ठित्या॒ अथो॒ अथो॒ प्रति॑ष्ठित्यै । \newline
77. अथो॒ इत्यथो᳚ । \newline
78. प्रति॑ष्ठित्या उ॒भयो॑ रु॒भयोः॒ प्रति॑ष्ठित्यै॒ प्रति॑ष्ठित्या उ॒भयोः᳚ । \newline
79. प्रति॑ष्ठित्या॒ इति॒ प्रति॑ - स्थि॒त्यै॒ । \newline
80. उ॒भयो᳚र् लो॒कयो᳚र् लो॒कयो॑ रु॒भयो॑ रु॒भयो᳚र् लो॒कयोः᳚ । \newline
81. लो॒कयोर्॑. ऋ॒द्ध्व र्द्ध्वा लो॒कयो᳚र् लो॒कयोर्॑. ऋ॒द्ध्वा । \newline
82. ऋ॒द्ध्वोदु दृ॒द्ध्व र्‌द्ध्वोत् । \newline
83. उत् ति॑ष्ठन्ति तिष्ठ॒ न्त्युदुत् ति॑ष्ठन्ति । \newline
84. ति॒ष्ठ॒ न्त्य॒ति॒रा॒त्रा व॑तिरा॒त्रौ ति॑ष्ठन्ति तिष्ठ न्त्यतिरा॒त्रौ । \newline
85. अ॒ति॒रा॒त्रा व॒भितो॒ ऽभितो॑ ऽतिरा॒त्रा व॑तिरा॒त्रा व॒भितः॑ । \newline
86. अ॒ति॒रा॒त्रावित्य॑ति - रा॒त्रौ । \newline
87. अ॒भितो॑ भवतो भवतो॒ ऽभितो॒ ऽभितो॑ भवतः । \newline
88. भ॒व॒तो॒ ब्र॒ह्म॒व॒र्च॒सस्य॑ ब्रह्मवर्च॒सस्य॑ भवतो भवतो ब्रह्मवर्च॒सस्य॑ । \newline
89. ब्र॒ह्म॒व॒र्च॒सस्या॒ न्नाद्य॑स्या॒ न्नाद्य॑स्य ब्रह्मवर्च॒सस्य॑ ब्रह्मवर्च॒सस्या॒ न्नाद्य॑स्य । \newline
90. ब्र॒ह्म॒व॒र्च॒सस्येति॑ ब्रह्म - व॒र्च॒सस्य॑ । \newline
91. अ॒न्नाद्य॑स्य॒ परि॑गृहीत्यै॒ परि॑गृहीत्या अ॒न्नाद्य॑स्या॒ न्नाद्य॑स्य॒ परि॑गृहीत्यै । \newline
92. अ॒न्नाद्य॒स्येत्य॑न्न - अद्य॑स्य । \newline
93. परि॑गृहीत्या॒ इति॒ परि॑ - गृ॒ही॒त्यै॒ । \newline

\textbf{Ghana Paata } \newline

1. लो॒क म॒भ्यारो॑ह न्त्य॒भ्यारो॑हन्ति लो॒कम् ॅलो॒क म॒भ्यारो॑हन्ति॒ यद् यद॒भ्यारो॑हन्ति लो॒कम् ॅलो॒क म॒भ्यारो॑हन्ति॒ यत् । \newline
2. अ॒भ्यारो॑हन्ति॒ यद् यद॒भ्यारो॑ह न्त्य॒भ्यारो॑हन्ति॒ यद॒न्यतो॒ ऽन्यतो॒ यद॒भ्यारो॑ह न्त्य॒भ्यारो॑हन्ति॒ यद॒न्यतः॑ । \newline
3. अ॒भ्यारो॑ह॒न्तीत्य॑भि - आरो॑हन्ति । \newline
4. यद॒न्यतो॒ ऽन्यतो॒ यद् यद॒न्यतः॑ पृ॒ष्ठानि॑ पृ॒ष्ठा न्य॒न्यतो॒ यद् यद॒न्यतः॑ पृ॒ष्ठानि॑ । \newline
5. अ॒न्यतः॑ पृ॒ष्ठानि॑ पृ॒ष्ठा न्य॒न्यतो॒ ऽन्यतः॑ पृ॒ष्ठानि॒ स्युः स्युः पृ॒ष्ठा न्य॒न्यतो॒ ऽन्यतः॑ पृ॒ष्ठानि॒ स्युः । \newline
6. पृ॒ष्ठानि॒ स्युः स्युः पृ॒ष्ठानि॑ पृ॒ष्ठानि॒ स्युर् विवि॑वधं॒ ॅविवि॑वधꣳ॒॒ स्युः पृ॒ष्ठानि॑ पृ॒ष्ठानि॒ स्युर् विवि॑वधम् । \newline
7. स्युर् विवि॑वधं॒ ॅविवि॑वधꣳ॒॒ स्युः स्युर् विवि॑वधꣳ स्याथ् स्या॒द् विवि॑वधꣳ॒॒ स्युः स्युर् विवि॑वधꣳ स्यात् । \newline
8. विवि॑वधꣳ स्याथ् स्या॒द् विवि॑वधं॒ ॅविवि॑वधꣳ स्या॒न् मद्ध्ये॒ मद्ध्ये᳚ स्या॒द् विवि॑वधं॒ ॅविवि॑वधꣳ स्या॒न् मद्ध्ये᳚ । \newline
9. विवि॑वध॒मिति॒ वि - वि॒व॒ध॒म् । \newline
10. स्या॒न् मद्ध्ये॒ मद्ध्ये᳚ स्याथ् स्या॒न् मद्ध्ये॑ पृ॒ष्ठानि॑ पृ॒ष्ठानि॒ मद्ध्ये᳚ स्याथ् स्या॒न् मद्ध्ये॑ पृ॒ष्ठानि॑ । \newline
11. मद्ध्ये॑ पृ॒ष्ठानि॑ पृ॒ष्ठानि॒ मद्ध्ये॒ मद्ध्ये॑ पृ॒ष्ठानि॑ भवन्ति भवन्ति पृ॒ष्ठानि॒ मद्ध्ये॒ मद्ध्ये॑ पृ॒ष्ठानि॑ भवन्ति । \newline
12. पृ॒ष्ठानि॑ भवन्ति भवन्ति पृ॒ष्ठानि॑ पृ॒ष्ठानि॑ भवन्ति सविवध॒त्वाय॑ सविवध॒त्वाय॑ भवन्ति पृ॒ष्ठानि॑ पृ॒ष्ठानि॑ भवन्ति सविवध॒त्वाय॑ । \newline
13. भ॒व॒न्ति॒ स॒वि॒व॒ध॒त्वाय॑ सविवध॒त्वाय॑ भवन्ति भवन्ति सविवध॒त्वा यौज॒ ओजः॑ सविवध॒त्वाय॑ भवन्ति भवन्ति सविवध॒त्वा यौजः॑ । \newline
14. स॒वि॒व॒ध॒त्वा यौज॒ ओजः॑ सविवध॒त्वाय॑ सविवध॒त्वा यौजो॒ वै वा ओजः॑ सविवध॒त्वाय॑ सविवध॒त्वा यौजो॒ वै । \newline
15. स॒वि॒व॒ध॒त्वायेति॑ सविवध - त्वाय॑ । \newline
16. ओजो॒ वै वा ओज॒ ओजो॒ वै वी॒र्यं॑ ॅवी॒र्यं॑ ॅवा ओज॒ ओजो॒ वै वी॒र्य᳚म् । \newline
17. वै वी॒र्यं॑ ॅवी॒र्यं॑ ॅवै वै वी॒र्य॑म् पृ॒ष्ठानि॑ पृ॒ष्ठानि॑ वी॒र्यं॑ ॅवै वै वी॒र्य॑म् पृ॒ष्ठानि॑ । \newline
18. वी॒र्य॑म् पृ॒ष्ठानि॑ पृ॒ष्ठानि॑ वी॒र्यं॑ ॅवी॒र्य॑म् पृ॒ष्ठा न्योज॒ ओजः॑ पृ॒ष्ठानि॑ वी॒र्यं॑ ॅवी॒र्य॑म् पृ॒ष्ठा न्योजः॑ । \newline
19. पृ॒ष्ठा न्योज॒ ओजः॑ पृ॒ष्ठानि॑ पृ॒ष्ठा न्योज॑ ए॒वैवौजः॑ पृ॒ष्ठानि॑ पृ॒ष्ठा न्योज॑ ए॒व । \newline
20. ओज॑ ए॒वै वौज॒ ओज॑ ए॒व वी॒र्यं॑ ॅवी॒र्य॑ मे॒वौज॒ ओज॑ ए॒व वी॒र्य᳚म् । \newline
21. ए॒व वी॒र्यं॑ ॅवी॒र्य॑ मे॒वैव वी॒र्य॑म् मद्ध्य॒तो म॑द्ध्य॒तो वी॒र्य॑ मे॒वैव वी॒र्य॑म् मद्ध्य॒तः । \newline
22. वी॒र्य॑म् मद्ध्य॒तो म॑द्ध्य॒तो वी॒र्यं॑ ॅवी॒र्य॑म् मद्ध्य॒तो द॑धते दधते मद्ध्य॒तो वी॒र्यं॑ ॅवी॒र्य॑म् मद्ध्य॒तो द॑धते । \newline
23. म॒द्ध्य॒तो द॑धते दधते मद्ध्य॒तो म॑द्ध्य॒तो द॑धते बृहद्रथन्त॒राभ्या᳚म् बृहद्रथन्त॒राभ्या᳚म् दधते मद्ध्य॒तो म॑द्ध्य॒तो द॑धते बृहद्रथन्त॒राभ्या᳚म् । \newline
24. द॒ध॒ते॒ बृ॒ह॒द्र॒थ॒न्त॒राभ्या᳚म् बृहद्रथन्त॒राभ्या᳚म् दधते दधते बृहद्रथन्त॒राभ्यां᳚ ॅयन्ति यन्ति बृहद्रथन्त॒राभ्या᳚म् दधते दधते बृहद्रथन्त॒राभ्यां᳚ ॅयन्ति । \newline
25. बृ॒ह॒द्र॒थ॒न्त॒राभ्यां᳚ ॅयन्ति यन्ति बृहद्रथन्त॒राभ्या᳚म् बृहद्रथन्त॒राभ्यां᳚ ॅयन्ती॒य मि॒यं ॅय॑न्ति बृहद्रथन्त॒राभ्या᳚म् बृहद्रथन्त॒राभ्यां᳚ ॅयन्ती॒यम् । \newline
26. बृ॒ह॒द्र॒थ॒न्त॒राभ्या॒मिति॑ बृहत् - र॒थ॒न्त॒राभ्या᳚म् । \newline
27. य॒न्ती॒य मि॒यं ॅय॑न्ति यन्ती॒यं ॅवाव वावेयं ॅय॑न्ति यन्ती॒यं ॅवाव । \newline
28. इ॒यं ॅवाव वावेय मि॒यं ॅवाव र॑थन्त॒रꣳ र॑थन्त॒रं ॅवावेय मि॒यं ॅवाव र॑थन्त॒रम् । \newline
29. वाव र॑थन्त॒रꣳ र॑थन्त॒रं ॅवाव वाव र॑थन्त॒र म॒सा व॒सौ र॑थन्त॒रं ॅवाव वाव र॑थन्त॒र म॒सौ । \newline
30. र॒थ॒न्त॒र म॒सा व॒सौ र॑थन्त॒रꣳ र॑थन्त॒र म॒सौ बृ॒हद् बृ॒ह द॒सौ र॑थन्त॒रꣳ र॑थन्त॒र म॒सौ बृ॒हत् । \newline
31. र॒थ॒न्त॒रमिति॑ रथं - त॒रम् । \newline
32. अ॒सौ बृ॒हद् बृ॒ह द॒सा व॒सौ बृ॒ह दा॒भ्या मा॒भ्याम् बृ॒ह द॒सा व॒सौ बृ॒ह दा॒भ्याम् । \newline
33. बृ॒ह दा॒भ्या मा॒भ्याम् बृ॒हद् बृ॒ह दा॒भ्या मे॒वै वाभ्याम् बृ॒हद् बृ॒ह दा॒भ्या मे॒व । \newline
34. आ॒भ्या मे॒वै वाभ्या मा॒भ्या मे॒व य॑न्ति यन्त्ये॒ वाभ्या मा॒भ्या मे॒व य॑न्ति । \newline
35. ए॒व य॑न्ति यन्त्ये॒वैव य॒ न्त्यथो॒ अथो॑ यन्त्ये॒वैव य॒ न्त्यथो᳚ । \newline
36. य॒ न्त्यथो॒ अथो॑ यन्ति य॒ न्त्यथो॑ अ॒नयो॑ र॒नयो॒ रथो॑ यन्ति य॒ न्त्यथो॑ अ॒नयोः᳚ । \newline
37. अथो॑ अ॒नयो॑ र॒नयो॒ रथो॒ अथो॑ अ॒नयो॑ रे॒वैवा नयो॒ रथो॒ अथो॑ अ॒नयो॑ रे॒व । \newline
38. अथो॒ इत्यथो᳚ । \newline
39. अ॒नयो॑ रे॒वै वानयो॑ र॒नयो॑ रे॒व प्रति॒ प्रत्ये॒ वानयो॑ र॒नयो॑ रे॒व प्रति॑ । \newline
40. ए॒व प्रति॒ प्रत्ये॒ वैव प्रति॑ तिष्ठन्ति तिष्ठन्ति॒ प्रत्ये॒ वैव प्रति॑ तिष्ठन्ति । \newline
41. प्रति॑ तिष्ठन्ति तिष्ठन्ति॒ प्रति॒ प्रति॑ तिष्ठ न्त्ये॒ते ए॒ते ति॑ष्ठन्ति॒ प्रति॒ प्रति॑ तिष्ठ न्त्ये॒ते । \newline
42. ति॒ष्ठ॒ न्त्ये॒ते ए॒ते ति॑ष्ठन्ति तिष्ठ न्त्ये॒ते वै वा ए॒ते ति॑ष्ठन्ति तिष्ठ न्त्ये॒ते वै । \newline
43. ए॒ते वै वा ए॒ते ए॒ते वै य॒ज्ञ्स्य॑ य॒ज्ञ्स्य॒ वा ए॒ते ए॒ते वै य॒ज्ञ्स्य॑ । \newline
44. ए॒ते इत्ये॒ते । \newline
45. वै य॒ज्ञ्स्य॑ य॒ज्ञ्स्य॒ वै वै य॒ज्ञ् स्या᳚ञ्ज॒साय॑नी अञ्ज॒साय॑नी य॒ज्ञ्स्य॒ वै वै य॒ज्ञ् स्या᳚ञ्ज॒साय॑नी । \newline
46. य॒ज्ञ् स्या᳚ञ्ज॒साय॑नी अञ्ज॒साय॑नी य॒ज्ञ्स्य॑ य॒ज्ञ् स्या᳚ञ्ज॒साय॑नी स्रु॒ती स्रु॒ती अ॑ञ्ज॒साय॑नी य॒ज्ञ्स्य॑ य॒ज्ञ् स्या᳚ञ्ज॒साय॑नी स्रु॒ती । \newline
47. अ॒ञ्ज॒साय॑नी स्रु॒ती स्रु॒ती अ॑ञ्ज॒साय॑नी अञ्ज॒साय॑नी स्रु॒ती ताभ्या॒म् ताभ्याꣳ॑ स्रु॒ती अ॑ञ्ज॒साय॑नी अञ्ज॒साय॑नी स्रु॒ती ताभ्या᳚म् । \newline
48. अ॒ञ्ज॒साय॑नी॒ इत्य॑ञ्जसा - अय॑नी । \newline
49. स्रु॒ती ताभ्या॒म् ताभ्याꣳ॑ स्रु॒ती स्रु॒ती ताभ्या॑ मे॒वैव ताभ्याꣳ॑ स्रु॒ती स्रु॒ती ताभ्या॑ मे॒व । \newline
50. स्रु॒ती इति॑ स्रु॒ती । \newline
51. ताभ्या॑ मे॒वैव ताभ्या॒म् ताभ्या॑ मे॒व सु॑व॒र्गꣳ सु॑व॒र्ग मे॒व ताभ्या॒म् ताभ्या॑ मे॒व सु॑व॒र्गम् । \newline
52. ए॒व सु॑व॒र्गꣳ सु॑व॒र्ग मे॒वैव सु॑व॒र्गम् ॅलो॒कम् ॅलो॒कꣳ सु॑व॒र्ग मे॒वैव सु॑व॒र्गम् ॅलो॒कम् । \newline
53. सु॒व॒र्गम् ॅलो॒कम् ॅलो॒कꣳ सु॑व॒र्गꣳ सु॑व॒र्गम् ॅलो॒कं ॅय॑न्ति यन्ति लो॒कꣳ सु॑व॒र्गꣳ सु॑व॒र्गम् ॅलो॒कं ॅय॑न्ति । \newline
54. सु॒व॒र्गमिति॑ सुवः - गम् । \newline
55. लो॒कं ॅय॑न्ति यन्ति लो॒कम् ॅलो॒कं ॅय॑न्ति॒ परा᳚ञ्चः॒ परा᳚ञ्चो यन्ति लो॒कम् ॅलो॒कं ॅय॑न्ति॒ परा᳚ञ्चः । \newline
56. य॒न्ति॒ परा᳚ञ्चः॒ परा᳚ञ्चो यन्ति यन्ति॒ परा᳚ञ्चो॒ वै वै परा᳚ञ्चो यन्ति यन्ति॒ परा᳚ञ्चो॒ वै । \newline
57. परा᳚ञ्चो॒ वै वै परा᳚ञ्चः॒ परा᳚ञ्चो॒ वा ए॒त ए॒ते वै परा᳚ञ्चः॒ परा᳚ञ्चो॒ वा ए॒ते । \newline
58. वा ए॒त ए॒ते वै वा ए॒ते सु॑व॒र्गꣳ सु॑व॒र्ग मे॒ते वै वा ए॒ते सु॑व॒र्गम् । \newline
59. ए॒ते सु॑व॒र्गꣳ सु॑व॒र्ग मे॒त ए॒ते सु॑व॒र्गम् ॅलो॒कम् ॅलो॒कꣳ सु॑व॒र्ग मे॒त ए॒ते सु॑व॒र्गम् ॅलो॒कम् । \newline
60. सु॒व॒र्गम् ॅलो॒कम् ॅलो॒कꣳ सु॑व॒र्गꣳ सु॑व॒र्गम् ॅलो॒क म॒भ्यारो॑ह न्त्य॒भ्यारो॑हन्ति लो॒कꣳ सु॑व॒र्गꣳ सु॑व॒र्गम् ॅलो॒क म॒भ्यारो॑हन्ति । \newline
61. सु॒व॒र्गमिति॑ सुवः - गम् । \newline
62. लो॒क म॒भ्यारो॑ह न्त्य॒भ्यारो॑हन्ति लो॒कम् ॅलो॒क म॒भ्यारो॑हन्ति॒ ये ये᳚ ऽभ्यारो॑हन्ति लो॒कम् ॅलो॒क म॒भ्यारो॑हन्ति॒ ये । \newline
63. अ॒भ्यारो॑हन्ति॒ ये ये᳚ ऽभ्यारो॑ह न्त्य॒भ्यारो॑हन्ति॒ ये प॑रा॒चीना॑नि परा॒चीना॑नि॒ ये᳚ ऽभ्यारो॑ह न्त्य॒भ्यारो॑हन्ति॒ ये प॑रा॒चीना॑नि । \newline
64. अ॒भ्यारो॑ह॒न्तीत्य॑भि - आरो॑हन्ति । \newline
65. ये प॑रा॒चीना॑नि परा॒चीना॑नि॒ ये ये प॑रा॒चीना॑नि पृ॒ष्ठानि॑ पृ॒ष्ठानि॑ परा॒चीना॑नि॒ ये ये प॑रा॒चीना॑नि पृ॒ष्ठानि॑ । \newline
66. प॒रा॒चीना॑नि पृ॒ष्ठानि॑ पृ॒ष्ठानि॑ परा॒चीना॑नि परा॒चीना॑नि पृ॒ष्ठा न्यु॑प॒य न्त्यु॑प॒यन्ति॑ पृ॒ष्ठानि॑ परा॒चीना॑नि परा॒चीना॑नि पृ॒ष्ठा न्यु॑प॒यन्ति॑ । \newline
67. पृ॒ष्ठा न्यु॑प॒य न्त्यु॑प॒यन्ति॑ पृ॒ष्ठानि॑ पृ॒ष्ठा न्यु॑प॒यन्ति॑ प्र॒त्यङ् प्र॒त्यङ् ङु॑प॒यन्ति॑ पृ॒ष्ठानि॑ पृ॒ष्ठा न्यु॑प॒यन्ति॑ प्र॒त्यङ् । \newline
68. उ॒प॒यन्ति॑ प्र॒त्यङ् प्र॒त्यङ् ङु॑प॒य न्त्यु॑प॒यन्ति॑ प्र॒त्यङ् त्र्य॒ह स्त्र्य॒हः प्र॒त्यङ् 
ङु॑प॒य न्त्यु॑प॒यन्ति॑ प्र॒त्यङ् त्र्य॒हः । \newline
69. उ॒प॒यन्तीत्यु॑प - यन्ति॑ । \newline
70. प्र॒त्यङ् त्र्य॒ह स्त्र्य॒हः प्र॒त्यङ् प्र॒त्यङ् त्र्य॒हो भ॑वति भवति त्र्य॒हः प्र॒त्यङ् प्र॒त्यङ् त्र्य॒हो भ॑वति । \newline
71. त्र्य॒हो भ॑वति भवति त्र्य॒ह स्त्र्य॒हो भ॑वति प्र॒त्यव॑रूढ्यै प्र॒त्यव॑रूढ्यै भवति त्र्य॒ह स्त्र्य॒हो भ॑वति प्र॒त्यव॑रूढ्यै । \newline
72. त्र्य॒ह इति॑ त्रि - अ॒हः । \newline
73. भ॒व॒ति॒ प्र॒त्यव॑रूढ्यै प्र॒त्यव॑रूढ्यै भवति भवति प्र॒त्यव॑रूढ्या॒ अथो॒ अथो᳚ प्र॒त्यव॑रूढ्यै भवति भवति प्र॒त्यव॑रूढ्या॒ अथो᳚ । \newline
74. प्र॒त्यव॑रूढ्या॒ अथो॒ अथो᳚ प्र॒त्यव॑रूढ्यै प्र॒त्यव॑रूढ्या॒ अथो॒ प्रति॑ष्ठित्यै॒ प्रति॑ष्ठित्या॒ अथो᳚ प्र॒त्यव॑रूढ्यै प्र॒त्यव॑रूढ्या॒ अथो॒ प्रति॑ष्ठित्यै । \newline
75. प्र॒त्यव॑रूढ्या॒ इति॑ प्रति - अव॑रूढ्यै । \newline
76. अथो॒ प्रति॑ष्ठित्यै॒ प्रति॑ष्ठित्या॒ अथो॒ अथो॒ प्रति॑ष्ठित्या उ॒भयो॑ रु॒भयोः॒ प्रति॑ष्ठित्या॒ अथो॒ अथो॒ प्रति॑ष्ठित्या उ॒भयोः᳚ । \newline
77. अथो॒ इत्यथो᳚ । \newline
78. प्रति॑ष्ठित्या उ॒भयो॑ रु॒भयोः॒ प्रति॑ष्ठित्यै॒ प्रति॑ष्ठित्या उ॒भयो᳚र् लो॒कयो᳚र् लो॒कयो॑ रु॒भयोः॒ प्रति॑ष्ठित्यै॒ प्रति॑ष्ठित्या उ॒भयो᳚र् लो॒कयोः᳚ । \newline
79. प्रति॑ष्ठित्या॒ इति॒ प्रति॑ - स्थि॒त्यै॒ । \newline
80. उ॒भयो᳚र् लो॒कयो᳚र् लो॒कयो॑ रु॒भयो॑ रु॒भयो᳚र् लो॒कयोर्॑. ऋ॒द्ध्व र्‌द्ध्वा लो॒कयो॑ रु॒भयो॑ रु॒भयो᳚र् लो॒कयोर्॑. ऋ॒द्ध्वा । \newline
81. लो॒कयोर्॑. ऋ॒द्ध्व र्‌द्ध्वा लो॒कयो᳚र् लो॒कयोर्॑. ऋ॒द्ध्वोदु दृ॒द्ध्वा लो॒कयो᳚र् लो॒कयोर्॑. ऋ॒द्ध्वोत् । \newline
82. ऋ॒द्ध्वो दुदृ॒द्ध्व र्‌द्ध्वोत् ति॑ष्ठन्ति तिष्ठ॒ न्त्युदृ॒द्ध्व र्‌द्ध्वोत् ति॑ष्ठन्ति । \newline
83. उत् ति॑ष्ठन्ति तिष्ठ॒ न्त्युदुत् ति॑ष्ठ न्त्यतिरा॒त्रा व॑तिरा॒त्रौ ति॑ष्ठ॒ न्त्युदुत् ति॑ष्ठ न्त्यतिरा॒त्रौ । \newline
84. ति॒ष्ठ॒ न्त्य॒ति॒रा॒त्रा व॑तिरा॒त्रौ ति॑ष्ठन्ति तिष्ठ न्त्यतिरा॒त्रा व॒भितो॒ ऽभितो॑ ऽतिरा॒त्रौ ति॑ष्ठन्ति तिष्ठ न्त्यतिरा॒त्रा व॒भितः॑ । \newline
85. अ॒ति॒रा॒त्रा व॒भितो॒ ऽभितो॑ ऽतिरा॒त्रा व॑तिरा॒त्रा व॒भितो॑ भवतो भवतो॒ ऽभितो॑ ऽतिरा॒त्रा व॑तिरा॒त्रा व॒भितो॑ भवतः । \newline
86. अ॒ति॒रा॒त्रावित्य॑ति - रा॒त्रौ । \newline
87. अ॒भितो॑ भवतो भवतो॒ ऽभितो॒ ऽभितो॑ भवतो ब्रह्मवर्च॒सस्य॑ ब्रह्मवर्च॒सस्य॑ भवतो॒ ऽभितो॒ ऽभितो॑ भवतो ब्रह्मवर्च॒सस्य॑ । \newline
88. भ॒व॒तो॒ ब्र॒ह्म॒व॒र्च॒सस्य॑ ब्रह्मवर्च॒सस्य॑ भवतो भवतो ब्रह्मवर्च॒सस्या॒ न्नाद्य॑स्या॒ न्नाद्य॑स्य ब्रह्मवर्च॒सस्य॑ भवतो भवतो ब्रह्मवर्च॒सस्या॒ न्नाद्य॑स्य । \newline
89. ब्र॒ह्म॒व॒र्च॒सस्या॒ न्नाद्य॑स्या॒ न्नाद्य॑स्य ब्रह्मवर्च॒सस्य॑ ब्रह्मवर्च॒सस्या॒ न्नाद्य॑स्य॒ परि॑गृहीत्यै॒ परि॑गृहीत्या अ॒न्नाद्य॑स्य ब्रह्मवर्च॒सस्य॑ ब्रह्मवर्च॒सस्या॒ न्नाद्य॑स्य॒ परि॑गृहीत्यै । \newline
90. ब्र॒ह्म॒व॒र्च॒सस्येति॑ ब्रह्म - व॒र्च॒सस्य॑ । \newline
91. अ॒न्नाद्य॑स्य॒ परि॑गृहीत्यै॒ परि॑गृहीत्या अ॒न्नाद्य॑स्या॒ न्नाद्य॑स्य॒ परि॑गृहीत्यै । \newline
92. अ॒न्नाद्य॒स्येत्य॑न्न - अद्य॑स्य । \newline
93. परि॑गृहीत्या॒ इति॒ परि॑ - गृ॒ही॒त्यै॒ । \newline
\pagebreak
\markright{ TS 7.3.10.1  \hfill https://www.vedavms.in \hfill}

\section{ TS 7.3.10.1 }

\textbf{TS 7.3.10.1 } \newline
\textbf{Samhita Paata} \newline

अ॒सावा॑दि॒त्यो᳚ऽस्मिन् ॅलो॒क आ॑सी॒त् तं दे॒वाः पृ॒ष्ठैः प॑रि॒गृह्य॑ सुव॒र्गं ॅलो॒कम॑गमय॒न् परै॑र॒वस्ता॒त् पर्य॑गृह्णन् दिवाकी॒र्त्ये॑न सुव॒र्गे लो॒के प्रत्य॑स्थापय॒न् परैः᳚ प॒रस्ता॒त् पर्य॑गृह्णन् पृ॒ष्ठैरु॒पावा॑रोह॒न्थ्स वा अ॒सावा॑दि॒त्यो॑ऽमुष्मि॑न् ॅलो॒के परै॑रुभ॒यतः॒ परि॑गृहीतो॒ यत् पृ॒ष्ठानि॒ भव॑न्ति सुव॒र्गमे॒व तैर्लो॒कं ॅयज॑माना यन्ति॒ परै॑र॒वस्ता॒त् परि॑ गृह्णन्ति दिवाकी॒र्त्ये॑न - [  ] \newline

\textbf{Pada Paata} \newline

अ॒सौ । आ॒दि॒त्यः । अ॒स्मिन्न् । लो॒के । आ॒सी॒त् । तम् । दे॒वाः । पृ॒ष्ठैः । प॒रि॒गृह्येति॑ परि - गृह्य॑ । सु॒व॒र्गमिति॑ सुवः - गम् । लो॒कम् । अ॒ग॒म॒य॒न्न् । परैः᳚ । अ॒वस्ता᳚त् । परीति॑ । अ॒गृ॒ह्ण॒न्न् । दि॒वा॒की॒र्त्ये॑नेति॑ दिवा - की॒र्त्ये॑न । सु॒व॒र्ग इति॑ सुवः - गे । लो॒के । प्रतीति॑ । अ॒स्था॒प॒य॒न्न् । परैः᳚ । प॒रस्ता᳚त् । परीति॑ । अ॒गृ॒ह्ण॒न्न् । पृ॒ष्ठैः । उ॒पावा॑रोह॒न्नित्यु॑प - अवा॑रोहन्न् । सः । वै । अ॒सौ । आ॒दि॒त्यः । अ॒मुष्मिन्न्॑ । लो॒के । परैः᳚ । उ॒भ॒यतः॑ । परि॑गृहीत॒ इति॒ परि॑-गृ॒ही॒तः॒ । यत् । पृ॒ष्ठानि॑ । भव॑न्ति । सु॒व॒र्गमिति॑ सुवः - गम् । ए॒व । तैः । लो॒कम् । यज॑मानाः । य॒न्ति॒ । परैः᳚ । अ॒वस्ता᳚त् । परीति॑ । गृ॒ह्ण॒न्ति॒ । दि॒वा॒की॒र्त्ये॑नेति॑ दिवा - की॒र्त्ये॑न ।  \newline


\textbf{Krama Paata} \newline

अ॒सावा॑दि॒त्यः । आ॒दि॒त्यो᳚ऽस्मिन्न् । अ॒स्मिन् ॅलो॒के । लो॒क आ॑सीत् । आ॒सी॒त् तम् । तम् दे॒वाः । दे॒वाः पृ॒ष्ठैः । पृ॒ष्ठैः प॑रि॒गृह्य॑ । प॒रि॒गृह्य॑ सुव॒र्गम् । प॒रि॒गृह्येति॑ परि - गृह्य॑ । सु॒व॒र्गम् ॅलो॒कम् । सु॒व॒र्गमिति॑ सुवः - गम् । लो॒कम॑गमयन्न् । अ॒ग॒म॒य॒न् परैः᳚ । परै॑र॒वस्ता᳚त् । अ॒वस्ता॒त् परि॑ । पर्य॑गृह्णन्न् । अ॒गृ॒ह्ण॒न् दि॒वा॒की॒र्त्ये॑न । दि॒वा॒की॒र्त्ये॑न सुव॒र्गे । दि॒वा॒की॒र्त्ये॑नेति॑ दिवा - की॒र्त्ये॑न । सु॒व॒र्गे लो॒के । सु॒व॒र्ग इति॑ सुवः - गे । लो॒के प्रति॑ । प्रत्य॑स्थापयन्न् । अ॒स्था॒प॒य॒न् परैः᳚ । परैः᳚ प॒रस्ता᳚त् । प॒रस्ता॒त् परि॑ । पर्य॑गृह्णन्न् । अ॒गृ॒ह्ण॒न् पृ॒ष्ठैः । पृ॒ष्ठै,रु॒पावा॑रोहन्न् । उ॒पावा॑रोह॒न्थ् सः । उ॒पावा॑रोह॒न्नित्यु॑प - अवा॑रोहन्न् । स वै । वा अ॒सौ । अ॒सावा॑दि॒त्यः । आ॒दि॒त्यो॑ऽमुष्मिन्न्॑ । अ॒मुष्मि॑न् ॅलो॒के । लो॒के परैः᳚ । परै॑रुभ॒यतः॑ । उ॒भ॒यतः॒ परि॑गृहीतः । परि॑गृहीतो॒ यत् । परि॑गृहीत॒ इति॒ परि॑ - गृ॒ही॒तः॒ । यत् पृ॒ष्ठानि॑ । पृ॒ष्ठानि॒ भव॑न्ति । भव॑न्ति सुव॒र्गम् । सु॒व॒र्गमे॒व । सु॒व॒र्गमिति॑ सुवः - गम् । ए॒व तैः । तैर् लो॒कम् । लो॒कम् ॅयज॑मानाः । यज॑माना यन्ति । य॒न्ति॒ परैः᳚ । परै॑र॒वस्ता᳚त् । अ॒वस्ता॒त् परि॑ । परि॑ गृह्णन्ति । गृ॒ह्ण॒न्ति॒ दि॒वा॒की॒र्त्ये॑न । दि॒वा॒की॒र्त्ये॑न सुव॒र्गे । दि॒वा॒की॒र्त्ये॑नेति॑ दिवा - की॒र्त्ये॑न \newline

\textbf{Jatai Paata} \newline

1. अ॒सा वा॑दि॒त्य आ॑दि॒त्यो॑ ऽसा व॒सा वा॑दि॒त्यः । \newline
2. आ॒दि॒त्यो᳚ ऽस्मिन् न॒स्मिन् ना॑दि॒त्य आ॑दि॒त्यो᳚ ऽस्मिन्न् । \newline
3. अ॒स्मिन् ॅलो॒के लो॒के᳚ ऽस्मिन् न॒स्मिन् ॅलो॒के । \newline
4. लो॒क आ॑सी दासी ल्लो॒के लो॒क आ॑सीत् । \newline
5. आ॒सी॒त् तम् त मा॑सी दासी॒त् तम् । \newline
6. तम् दे॒वा दे॒वा स्तम् तम् दे॒वाः । \newline
7. दे॒वाः पृ॒ष्ठैः पृ॒ष्ठैर् दे॒वा दे॒वाः पृ॒ष्ठैः । \newline
8. पृ॒ष्ठैः प॑रि॒गृह्य॑ परि॒गृह्य॑ पृ॒ष्ठैः पृ॒ष्ठैः प॑रि॒गृह्य॑ । \newline
9. प॒रि॒गृह्य॑ सुव॒र्गꣳ सु॑व॒र्गम् प॑रि॒गृह्य॑ परि॒गृह्य॑ सुव॒र्गम् । \newline
10. प॒रि॒गृह्येति॑ परि - गृह्य॑ । \newline
11. सु॒व॒र्गम् ॅलो॒कम् ॅलो॒कꣳ सु॑व॒र्गꣳ सु॑व॒र्गम् ॅलो॒कम् । \newline
12. सु॒व॒र्गमिति॑ सुवः - गम् । \newline
13. लो॒क म॑गमयन् नगमयन् ॅलो॒कम् ॅलो॒क म॑गमयन्न् । \newline
14. अ॒ग॒म॒य॒न् परैः॒ परै॑ रगमयन् नगमय॒न् परैः᳚ । \newline
15. परै॑ र॒वस्ता॑ द॒वस्ता॒त् परैः॒ परै॑ र॒वस्ता᳚त् । \newline
16. अ॒वस्ता॒त् परि॒ पर्य॒वस्ता॑ द॒वस्ता॒त् परि॑ । \newline
17. पर्य॑गृह्णन् नगृह्ण॒न् परि॒ पर्य॑गृह्णन्न् । \newline
18. अ॒गृ॒ह्ण॒न् दि॒वा॒की॒र्त्ये॑न दिवाकी॒र्त्ये॑ना गृह्णन् नगृह्णन् दिवाकी॒र्त्ये॑न । \newline
19. दि॒वा॒की॒र्त्ये॑न सुव॒र्गे सु॑व॒र्गे दि॑वाकी॒र्त्ये॑न दिवाकी॒र्त्ये॑न सुव॒र्गे । \newline
20. दि॒वा॒की॒र्त्ये॑नेति॑ दिवा - की॒र्त्ये॑न । \newline
21. सु॒व॒र्गे लो॒के लो॒के सु॑व॒र्गे सु॑व॒र्गे लो॒के । \newline
22. सु॒व॒र्ग इति॑ सुवः - गे । \newline
23. लो॒के प्रति॒ प्रति॑ लो॒के लो॒के प्रति॑ । \newline
24. प्रत्य॑ स्थापयन् नस्थापय॒न् प्रति॒ प्रत्य॑ स्थापयन्न् । \newline
25. अ॒स्था॒प॒य॒न् परैः॒ परै॑ रस्थापयन् नस्थापय॒न् परैः᳚ । \newline
26. परैः᳚ प॒रस्ता᳚त् प॒रस्ता॒त् परैः॒ परैः᳚ प॒रस्ता᳚त् । \newline
27. प॒रस्ता॒त् परि॒ परि॑ प॒रस्ता᳚त् प॒रस्ता॒त् परि॑ । \newline
28. पर्य॑गृह्णन् नगृह्ण॒न् परि॒ पर्य॑गृह्णन्न् । \newline
29. अ॒गृ॒ह्ण॒न् पृ॒ष्ठैः पृ॒ष्ठै र॑गृह्णन् नगृह्णन् पृ॒ष्ठैः । \newline
30. पृ॒ष्ठै रु॒पावा॑रोहन् नु॒पावा॑रोहन् पृ॒ष्ठैः पृ॒ष्ठै रु॒पावा॑रोहन्न् । \newline
31. उ॒पावा॑रोह॒न् थ्स स उ॒पावा॑रोहन् नु॒पावा॑रोह॒न् थ्सः । \newline
32. उ॒पावा॑रोह॒न्नित्यु॑प - अवा॑रोहन्न् । \newline
33. स वै वै स स वै । \newline
34. वा अ॒सा व॒सौ वै वा अ॒सौ । \newline
35. अ॒सा वा॑दि॒त्य आ॑दि॒त्यो॑ ऽसा व॒सा वा॑दि॒त्यः । \newline
36. आ॒दि॒त्यो॑ ऽमुष्मि॑न् न॒मुष्मि॑न् नादि॒त्य आ॑दि॒त्यो॑ ऽमुष्मिन्न्॑ । \newline
37. अ॒मुष्मि॑न् ॅलो॒के लो॒के॑ ऽमुष्मि॑न् न॒मुष्मि॑न् ॅलो॒के । \newline
38. लो॒के परैः॒ परै᳚र् लो॒के लो॒के परैः᳚ । \newline
39. परै॑ रुभ॒यत॑ उभ॒यतः॒ परैः॒ परै॑ रुभ॒यतः॑ । \newline
40. उ॒भ॒यतः॒ परि॑गृहीतः॒ परि॑गृहीत उभ॒यत॑ उभ॒यतः॒ परि॑गृहीतः । \newline
41. परि॑गृहीतो॒ यद् यत् परि॑गृहीतः॒ परि॑गृहीतो॒ यत् । \newline
42. परि॑गृहीत॒ इति॒ परि॑ - गृ॒ही॒तः॒ । \newline
43. यत् पृ॒ष्ठानि॑ पृ॒ष्ठानि॒ यद् यत् पृ॒ष्ठानि॑ । \newline
44. पृ॒ष्ठानि॒ भव॑न्ति॒ भव॑न्ति पृ॒ष्ठानि॑ पृ॒ष्ठानि॒ भव॑न्ति । \newline
45. भव॑न्ति सुव॒र्गꣳ सु॑व॒र्गम् भव॑न्ति॒ भव॑न्ति सुव॒र्गम् । \newline
46. सु॒व॒र्ग मे॒वैव सु॑व॒र्गꣳ सु॑व॒र्ग मे॒व । \newline
47. सु॒व॒र्गमिति॑ सुवः - गम् । \newline
48. ए॒व तै स्तै रे॒वैव तैः । \newline
49. तैर् लो॒कम् ॅलो॒कम् तै स्तैर् लो॒कम् । \newline
50. लो॒कं ॅयज॑माना॒ यज॑माना लो॒कम् ॅलो॒कं ॅयज॑मानाः । \newline
51. यज॑माना यन्ति यन्ति॒ यज॑माना॒ यज॑माना यन्ति । \newline
52. य॒न्ति॒ परैः॒ परै᳚र् यन्ति यन्ति॒ परैः᳚ । \newline
53. परै॑ र॒वस्ता॑ द॒वस्ता॒त् परैः॒ परै॑ र॒वस्ता᳚त् । \newline
54. अ॒वस्ता॒त् परि॒ पर्य॒वस्ता॑ द॒वस्ता॒त् परि॑ । \newline
55. परि॑ गृह्णन्ति गृह्णन्ति॒ परि॒ परि॑ गृह्णन्ति । \newline
56. गृ॒ह्ण॒न्ति॒ दि॒वा॒की॒र्त्ये॑न दिवाकी॒र्त्ये॑न गृह्णन्ति गृह्णन्ति दिवाकी॒र्त्ये॑न । \newline
57. दि॒वा॒की॒र्त्ये॑न सुव॒र्गे सु॑व॒र्गे दि॑वाकी॒र्त्ये॑न दिवाकी॒र्त्ये॑न सुव॒र्गे । \newline
58. दि॒वा॒की॒र्त्ये॑नेति॑ दिवा - की॒र्त्ये॑न । \newline

\textbf{Ghana Paata } \newline

1. अ॒सा वा॑दि॒त्य आ॑दि॒त्यो॑ ऽसा व॒सा वा॑दि॒त्यो᳚ ऽस्मिन् न॒स्मिन् ना॑दि॒त्यो॑ ऽसा व॒सा वा॑दि॒त्यो᳚ ऽस्मिन्न् । \newline
2. आ॒दि॒त्यो᳚ ऽस्मिन् न॒स्मिन् ना॑दि॒त्य आ॑दि॒त्यो᳚ ऽस्मिन् ॅलो॒के लो॒के᳚ ऽस्मिन् ना॑दि॒त्य आ॑दि॒त्यो᳚ ऽस्मिन् ॅलो॒के । \newline
3. अ॒स्मिन् ॅलो॒के लो॒के᳚ ऽस्मिन् न॒स्मिन् ॅलो॒क आ॑सी दासी ल्लो॒के᳚ ऽस्मिन् न॒स्मिन् ॅलो॒क आ॑सीत् । \newline
4. लो॒क आ॑सी दासी ल्लो॒के लो॒क आ॑सी॒त् तम् त मा॑सी ल्लो॒के लो॒क आ॑सी॒त् तम् । \newline
5. आ॒सी॒त् तम् त मा॑सी दासी॒त् तम् दे॒वा दे॒वा स्त मा॑सी दासी॒त् तम् दे॒वाः । \newline
6. तम् दे॒वा दे॒वा स्तम् तम् दे॒वाः पृ॒ष्ठैः पृ॒ष्ठैर् दे॒वा स्तम् तम् दे॒वाः पृ॒ष्ठैः । \newline
7. दे॒वाः पृ॒ष्ठैः पृ॒ष्ठैर् दे॒वा दे॒वाः पृ॒ष्ठैः प॑रि॒गृह्य॑ परि॒गृह्य॑ पृ॒ष्ठैर् दे॒वा दे॒वाः पृ॒ष्ठैः प॑रि॒गृह्य॑ । \newline
8. पृ॒ष्ठैः प॑रि॒गृह्य॑ परि॒गृह्य॑ पृ॒ष्ठैः पृ॒ष्ठैः प॑रि॒गृह्य॑ सुव॒र्गꣳ सु॑व॒र्गम् प॑रि॒गृह्य॑ पृ॒ष्ठैः पृ॒ष्ठैः प॑रि॒गृह्य॑ सुव॒र्गम् । \newline
9. प॒रि॒गृह्य॑ सुव॒र्गꣳ सु॑व॒र्गम् प॑रि॒गृह्य॑ परि॒गृह्य॑ सुव॒र्गम् ॅलो॒कम् ॅलो॒कꣳ सु॑व॒र्गम् प॑रि॒गृह्य॑ परि॒गृह्य॑ सुव॒र्गम् ॅलो॒कम् । \newline
10. प॒रि॒गृह्येति॑ परि - गृह्य॑ । \newline
11. सु॒व॒र्गम् ॅलो॒कम् ॅलो॒कꣳ सु॑व॒र्गꣳ सु॑व॒र्गम् ॅलो॒क म॑गमयन् नगमयन् ॅलो॒कꣳ सु॑व॒र्गꣳ सु॑व॒र्गम् ॅलो॒क म॑गमयन्न् । \newline
12. सु॒व॒र्गमिति॑ सुवः - गम् । \newline
13. लो॒क म॑गमयन् नगमयन् ॅलो॒कम् ॅलो॒क म॑गमय॒न् परैः॒ परै॑ रगमयन् ॅलो॒कम् ॅलो॒क म॑गमय॒न् परैः᳚ । \newline
14. अ॒ग॒म॒य॒न् परैः॒ परै॑ रगमयन् नगमय॒न् परै॑ र॒वस्ता॑ द॒वस्ता॒त् परै॑ रगमयन् नगमय॒न् परै॑ र॒वस्ता᳚त् । \newline
15. परै॑ र॒वस्ता॑ द॒वस्ता॒त् परैः॒ परै॑ र॒वस्ता॒त् परि॒ पर्य॒वस्ता॒त् परैः॒ परै॑ र॒वस्ता॒त् परि॑ । \newline
16. अ॒वस्ता॒त् परि॒ पर्य॒वस्ता॑ द॒वस्ता॒त् पर्य॑गृह्णन् नगृह्ण॒न् पर्य॒वस्ता॑ द॒वस्ता॒त् पर्य॑गृह्णन्न् । \newline
17. पर्य॑गृह्णन् नगृह्ण॒न् परि॒ पर्य॑गृह्णन् दिवाकी॒र्त्ये॑न दिवाकी॒र्त्ये॑ना गृह्ण॒न् परि॒ पर्य॑गृह्णन् दिवाकी॒र्त्ये॑न । \newline
18. अ॒गृ॒ह्ण॒न् दि॒वा॒की॒र्त्ये॑न दिवाकी॒र्त्ये॑ना गृह्णन् नगृह्णन् दिवाकी॒र्त्ये॑न सुव॒र्गे सु॑व॒र्गे दि॑वाकी॒र्त्ये॑ नागृह्णन् नगृह्णन् दिवाकी॒र्त्ये॑न सुव॒र्गे । \newline
19. दि॒वा॒की॒र्त्ये॑न सुव॒र्गे सु॑व॒र्गे दि॑वाकी॒र्त्ये॑न दिवाकी॒र्त्ये॑न सुव॒र्गे लो॒के लो॒के सु॑व॒र्गे दि॑वाकी॒र्त्ये॑न दिवाकी॒र्त्ये॑न सुव॒र्गे लो॒के । \newline
20. दि॒वा॒की॒र्त्ये॑नेति॑ दिवा - की॒र्त्ये॑न । \newline
21. सु॒व॒र्गे लो॒के लो॒के सु॑व॒र्गे सु॑व॒र्गे लो॒के प्रति॒ प्रति॑ लो॒के सु॑व॒र्गे सु॑व॒र्गे लो॒के प्रति॑ । \newline
22. सु॒व॒र्ग इति॑ सुवः - गे । \newline
23. लो॒के प्रति॒ प्रति॑ लो॒के लो॒के प्रत्य॑ स्थापयन् नस्थापय॒न् प्रति॑ लो॒के लो॒के प्रत्य॑ स्थापयन्न् । \newline
24. प्रत्य॑ स्थापयन् नस्थापय॒न् प्रति॒ प्रत्य॑ स्थापय॒न् परैः॒ परै॑ रस्थापय॒न् प्रति॒ प्रत्य॑ स्थापय॒न् परैः᳚ । \newline
25. अ॒स्था॒प॒य॒न् परैः॒ परै॑ रस्थापयन् नस्थापय॒न् परैः᳚ प॒रस्ता᳚त् प॒रस्ता॒त् परै॑ रस्थापयन् नस्थापय॒न् परैः᳚ प॒रस्ता᳚त् । \newline
26. परैः᳚ प॒रस्ता᳚त् प॒रस्ता॒त् परैः॒ परैः᳚ प॒रस्ता॒त् परि॒ परि॑ प॒रस्ता॒त् परैः॒ परैः᳚ प॒रस्ता॒त् परि॑ । \newline
27. प॒रस्ता॒त् परि॒ परि॑ प॒रस्ता᳚त् प॒रस्ता॒त् पर्य॑गृह्णन् नगृह्ण॒न् परि॑ प॒रस्ता᳚त् प॒रस्ता॒त् पर्य॑गृह्णन्न् । \newline
28. पर्य॑गृह्णन् नगृह्ण॒न् परि॒ पर्य॑गृह्णन् पृ॒ष्ठैः पृ॒ष्ठै र॑गृह्ण॒न् परि॒ पर्य॑गृह्णन् पृ॒ष्ठैः । \newline
29. अ॒गृ॒ह्ण॒न् पृ॒ष्ठैः पृ॒ष्ठै र॑गृह्णन् नगृह्णन् पृ॒ष्ठै रु॒पावा॑रोहन् नु॒पावा॑रोहन् पृ॒ष्ठै र॑गृह्णन् नगृह्णन् पृ॒ष्ठै रु॒पावा॑रोहन्न् । \newline
30. पृ॒ष्ठै रु॒पावा॑रोहन् नु॒पावा॑रोहन् पृ॒ष्ठैः पृ॒ष्ठै रु॒पावा॑रोह॒न् थ्स स उ॒पावा॑रोहन् पृ॒ष्ठैः पृ॒ष्ठै रु॒पावा॑रोह॒न् थ्सः । \newline
31. उ॒पावा॑रोह॒न् थ्स स उ॒पावा॑रोहन् नु॒पावा॑रोह॒न् थ्स वै वै स उ॒पावा॑रोहन् नु॒पावा॑रोह॒न् थ्स वै । \newline
32. उ॒पावा॑रोह॒न्नित्यु॑प - अवा॑रोहन्न् । \newline
33. स वै वै स स वा अ॒सा व॒सौ वै स स वा अ॒सौ । \newline
34. वा अ॒सा व॒सौ वै वा अ॒सा वा॑दि॒त्य आ॑दि॒त्यो॑ ऽसौ वै वा अ॒सा वा॑दि॒त्यः । \newline
35. अ॒सा वा॑दि॒त्य आ॑दि॒त्यो॑ ऽसा व॒सा वा॑दि॒त्यो॑ ऽमुष्मि॑न् न॒मुष्मि॑न् नादि॒त्यो॑ ऽसा व॒सा वा॑दि॒त्यो॑ ऽमुष्मिन्न्॑ । \newline
36. आ॒दि॒त्यो॑ ऽमुष्मि॑न् न॒मुष्मि॑न् नादि॒त्य आ॑दि॒त्यो॑ ऽमुष्मि॑न् ॅलो॒के लो॒के॑ ऽमुष्मि॑न् नादि॒त्य आ॑दि॒त्यो॑ ऽमुष्मि॑न् ॅलो॒के । \newline
37. अ॒मुष्मि॑न् ॅलो॒के लो॒के॑ ऽमुष्मि॑न् न॒मुष्मि॑न् ॅलो॒के परैः॒ परै᳚र् लो॒के॑ ऽमुष्मि॑न् न॒मुष्मि॑न् ॅलो॒के परैः᳚ । \newline
38. लो॒के परैः॒ परै᳚र् लो॒के लो॒के परै॑ रुभ॒यत॑ उभ॒यतः॒ परै᳚र् लो॒के लो॒के परै॑ रुभ॒यतः॑ । \newline
39. परै॑ रुभ॒यत॑ उभ॒यतः॒ परैः॒ परै॑ रुभ॒यतः॒ परि॑गृहीतः॒ परि॑गृहीत उभ॒यतः॒ परैः॒ परै॑ रुभ॒यतः॒ परि॑गृहीतः । \newline
40. उ॒भ॒यतः॒ परि॑गृहीतः॒ परि॑गृहीत उभ॒यत॑ उभ॒यतः॒ परि॑गृहीतो॒ यद् यत् परि॑गृहीत उभ॒यत॑ उभ॒यतः॒ परि॑गृहीतो॒ यत् । \newline
41. परि॑गृहीतो॒ यद् यत् परि॑गृहीतः॒ परि॑गृहीतो॒ यत् पृ॒ष्ठानि॑ पृ॒ष्ठानि॒ यत् परि॑गृहीतः॒ परि॑गृहीतो॒ यत् पृ॒ष्ठानि॑ । \newline
42. परि॑गृहीत॒ इति॒ परि॑ - गृ॒ही॒तः॒ । \newline
43. यत् पृ॒ष्ठानि॑ पृ॒ष्ठानि॒ यद् यत् पृ॒ष्ठानि॒ भव॑न्ति॒ भव॑न्ति पृ॒ष्ठानि॒ यद् यत् पृ॒ष्ठानि॒ भव॑न्ति । \newline
44. पृ॒ष्ठानि॒ भव॑न्ति॒ भव॑न्ति पृ॒ष्ठानि॑ पृ॒ष्ठानि॒ भव॑न्ति सुव॒र्गꣳ सु॑व॒र्गम् भव॑न्ति पृ॒ष्ठानि॑ पृ॒ष्ठानि॒ भव॑न्ति सुव॒र्गम् । \newline
45. भव॑न्ति सुव॒र्गꣳ सु॑व॒र्गम् भव॑न्ति॒ भव॑न्ति सुव॒र्ग मे॒वैव सु॑व॒र्गम् भव॑न्ति॒ भव॑न्ति सुव॒र्ग मे॒व । \newline
46. सु॒व॒र्ग मे॒वैव सु॑व॒र्गꣳ सु॑व॒र्ग मे॒व तै स्तै रे॒व सु॑व॒र्गꣳ सु॑व॒र्ग मे॒व तैः । \newline
47. सु॒व॒र्गमिति॑ सुवः - गम् । \newline
48. ए॒व तै स्तै रे॒वैव तैर् लो॒कम् ॅलो॒कम् तै रे॒वैव तैर् लो॒कम् । \newline
49. तैर् लो॒कम् ॅलो॒कम् तै स्तैर् लो॒कं ॅयज॑माना॒ यज॑माना लो॒कम् तै स्तैर् लो॒कं ॅयज॑मानाः । \newline
50. लो॒कं ॅयज॑माना॒ यज॑माना लो॒कम् ॅलो॒कं ॅयज॑माना यन्ति यन्ति॒ यज॑माना लो॒कम् ॅलो॒कं ॅयज॑माना यन्ति । \newline
51. यज॑माना यन्ति यन्ति॒ यज॑माना॒ यज॑माना यन्ति॒ परैः॒ परै᳚र् यन्ति॒ यज॑माना॒ यज॑माना यन्ति॒ परैः᳚ । \newline
52. य॒न्ति॒ परैः॒ परै᳚र् यन्ति यन्ति॒ परै॑ र॒वस्ता॑ द॒वस्ता॒त् परै᳚र् यन्ति यन्ति॒ परै॑ र॒वस्ता᳚त् । \newline
53. परै॑ र॒वस्ता॑ द॒वस्ता॒त् परैः॒ परै॑ र॒वस्ता॒त् परि॒ पर्य॒वस्ता॒त् परैः॒ परै॑ र॒वस्ता॒त् परि॑ । \newline
54. अ॒वस्ता॒त् परि॒ पर्य॒वस्ता॑ द॒वस्ता॒त् परि॑ गृह्णन्ति गृह्णन्ति॒ पर्य॒वस्ता॑ द॒वस्ता॒त् परि॑ गृह्णन्ति । \newline
55. परि॑ गृह्णन्ति गृह्णन्ति॒ परि॒ परि॑ गृह्णन्ति दिवाकी॒र्त्ये॑न दिवाकी॒र्त्ये॑न गृह्णन्ति॒ परि॒ परि॑ गृह्णन्ति दिवाकी॒र्त्ये॑न । \newline
56. गृ॒ह्ण॒न्ति॒ दि॒वा॒की॒र्त्ये॑न दिवाकी॒र्त्ये॑न गृह्णन्ति गृह्णन्ति दिवाकी॒र्त्ये॑न सुव॒र्गे सु॑व॒र्गे दि॑वाकी॒र्त्ये॑न गृह्णन्ति गृह्णन्ति दिवाकी॒र्त्ये॑न सुव॒र्गे । \newline
57. दि॒वा॒की॒र्त्ये॑न सुव॒र्गे सु॑व॒र्गे दि॑वाकी॒र्त्ये॑न दिवाकी॒र्त्ये॑न सुव॒र्गे लो॒के लो॒के सु॑व॒र्गे दि॑वाकी॒र्त्ये॑न दिवाकी॒र्त्ये॑न सुव॒र्गे लो॒के । \newline
58. दि॒वा॒की॒र्त्ये॑नेति॑ दिवा - की॒र्त्ये॑न । \newline
\pagebreak
\markright{ TS 7.3.10.2  \hfill https://www.vedavms.in \hfill}

\section{ TS 7.3.10.2 }

\textbf{TS 7.3.10.2 } \newline
\textbf{Samhita Paata} \newline

सुव॒र्गे लो॒के प्रति॑ तिष्ठन्ति॒ परैः᳚ प॒रस्ता॒त् परि॑ गृह्णन्ति पृ॒ष्ठैरु॒पाव॑रोहन्ति॒ यत् परे॑ प॒रस्ता॒न्न स्युः परा᳚ञ्चः सुव॒र्गा-ल्लो॒कान्निष्प॑द्येर॒न्॒. यद॒वस्ता॒न्न स्युः प्र॒जा निर्द॑हेयुर॒भितो॑ दिवाकी॒र्त्यं॑ पर॑स्सामानो भवन्ति सुव॒र्ग ए॒वैनां᳚ ॅलो॒क उ॑भ॒यतः॒ परि॑ गृह्णन्ति॒ यज॑माना॒ वै दि॑वाकी॒र्त्यꣳ॑ संॅवथ्स॒रः पर॑स्सामानो॒ऽभितो॑ दिवाकी॒र्त्यं॑ पर॑स्सामानो भवन्ति संॅवथ्स॒र ए॒वोभ॒यतः॒ - [  ] \newline

\textbf{Pada Paata} \newline

सु॒व॒र्ग इति॑ सुवः-गे । लो॒के । प्रतीति॑ । ति॒ष्ठ॒न्ति॒ । परैः᳚ । प॒रस्ता᳚त् । परीति॑ । गृ॒ह्ण॒न्ति॒ । पृ॒ष्ठैः । उ॒पाव॑रोह॒न्तीत्यु॑प - अव॑रोहन्ति । यत् । परे᳚ । प॒रस्ता᳚त् । न । स्युः । परा᳚ञ्चः । सु॒व॒र्गादिति॑ सुवः - गात् । लो॒कात् । निरिति॑ । प॒द्ये॒र॒न्न् । यत् । अ॒वस्ता᳚त् । न । स्युः । प्र॒जा इति॑ प्र - जाः । निरिति॑ । द॒हे॒युः॒ । अ॒भितः॑ । दि॒वा॒की॒र्त्य॑मिति॑ दिवा - की॒र्त्य᳚म् । पर॑स्सामान॒ इति॒ परः॑ - सा॒मा॒नः॒ । भ॒व॒न्ति॒ । सु॒व॒र्ग इति॑ सुवः - गे । ए॒व । ए॒ना॒न्न् । लो॒के । उ॒भ॒यतः॑ । परीति॑ । गृ॒ह्ण॒न्ति॒ । यज॑मानाः । वै । दि॒वा॒की᳚र्त्यमिति॑ दिवा - की॒र्त्य᳚म् । सं॒ॅव॒थ्स॒र इति॑ सं - व॒थ्स॒रः । पर॑स्सामान॒ इति॒ परः॑ - सा॒मा॒नः॒ । अ॒भितः॑ । दि॒वा॒की॒र्त्य॑मिति॑ दिवा - की॒र्त्य᳚म् । पर॑स्सामान॒ इति॒ परः॑ - सा॒मा॒नः॒ । भ॒व॒न्ति॒ । सं॒ॅव॒थ्स॒र इति॑ सं - व॒थ्स॒रे । ए॒व । उ॒भ॒यतः॑ ।  \newline


\textbf{Krama Paata} \newline

सु॒व॒र्गे लो॒के । सु॒व॒र्ग इति॑ सुवः - गे । लो॒के प्रति॑ । प्रति॑ तिष्ठन्ति । ति॒ष्ठ॒न्ति॒ परैः᳚ । परैः᳚ प॒रस्ता᳚त् । प॒रस्ता॒त् परि॑ । परि॑ गृह्णन्ति । गृ॒ह्ण॒न्ति॒ पृ॒ष्ठैः । पृ॒ष्ठै,रु॒पाव॑रोहन्ति । उ॒पाव॑रोहन्ति॒ यत् । उ॒पाव॑रोह॒न्तीत्यु॑प - अव॑रोहन्ति । यत् परे᳚ । परे॑ प॒रस्ता᳚त् । प॒रस्ता॒न् न । न स्युः । स्युः परा᳚ञ्चः । परा᳚ञ्चः सुव॒र्गात् । सु॒व॒र्गाल्लो॒कात् । सु॒व॒र्गादिति॑ सुवः - गात् । लो॒कान् निः । निष्प॑द्येरन्न् । प॒द्ये॒र॒न्॒. यत् । यद॒वस्ता᳚त् । अ॒वस्ता॒न् न । न स्युः । स्युः प्र॒जाः । प्र॒जा निः । प्र॒जा इति॑ प्र - जाः । निर् द॑हेयुः । द॒हे॒यु॒र॒भितः॑ । अ॒भितो॑ दिवाकी॒र्त्य᳚म् । दि॒वा॒की॒र्त्य॑म् पर॑स्सामानः । दि॒वा॒की॒र्त्यमिति॑ दिवा - की॒र्त्य᳚म् । पर॑स्सामानो भवन्ति । पर॑स्सामान॒ इति॒ परः॑ - सा॒मा॒नः॒ । भ॒व॒न्ति॒ सु॒व॒र्गे । सु॒व॒र्ग ए॒व । सु॒व॒र्ग इति॑ सुवः - गे । ए॒वैनान्॑ । ए॒ना॒न् ॅलो॒के । लो॒क उ॑भ॒यतः॑ । उ॒भ॒यतः॒ परि॑ । परि॑ गृह्णन्ति । गृ॒ह्ण॒न्ति॒ यज॑मानाः । यज॑माना॒ वै । वै दि॑वाकी॒र्त्य᳚म् । दि॒वा॒की॒र्त्यꣳ॑ सम्ॅवथ्स॒रः । दि॒वा॒की॒र्त्य॑मिति॑ दिवा - की॒र्त्य᳚म् । स॒म्ॅव॒थ्स॒रः पर॑स्सामानः । स॒म्ॅव॒थ्स॒र इति॑ सम् - व॒थ्स॒रः । पर॑स्सामानो॒ऽभितः॑ । पर॑स्सामान॒ इति॒ परः॑ - सा॒मा॒नः॒ । अ॒भितो॑ दिवाकी॒र्त्य᳚म् । दि॒वा॒की॒र्त्य॑म् पर॑स्सामानः । दि॒वा॒की॒र्त्य॑मिति॑ दिवा - की॒र्त्य᳚म् । पर॑स्सामानो भवन्ति । पर॑स्सामान॒ इति॒ परः॑ - सा॒मा॒नः॒ । भ॒व॒न्ति॒ स॒म्ॅव॒थ्स॒रे ।? स॒म्ॅव॒थ्स॒र ए॒व । स॒म्ॅव॒थ्स॒र इति॑ सम् - व॒थ्स॒रे । ए॒वोभ॒यतः॑ । उ॒भ॒यतः॒ प्रति॑ \newline

\textbf{Jatai Paata} \newline

1. सु॒व॒र्गे लो॒के लो॒के सु॑व॒र्गे सु॑व॒र्गे लो॒के । \newline
2. सु॒व॒र्ग इति॑ सुवः - गे । \newline
3. लो॒के प्रति॒ प्रति॑ लो॒के लो॒के प्रति॑ । \newline
4. प्रति॑ तिष्ठन्ति तिष्ठन्ति॒ प्रति॒ प्रति॑ तिष्ठन्ति । \newline
5. ति॒ष्ठ॒न्ति॒ परैः॒ परै᳚ स्तिष्ठन्ति तिष्ठन्ति॒ परैः᳚ । \newline
6. परैः᳚ प॒रस्ता᳚त् प॒रस्ता॒त् परैः॒ परैः᳚ प॒रस्ता᳚त् । \newline
7. प॒रस्ता॒त् परि॒ परि॑ प॒रस्ता᳚त् प॒रस्ता॒त् परि॑ । \newline
8. परि॑ गृह्णन्ति गृह्णन्ति॒ परि॒ परि॑ गृह्णन्ति । \newline
9. गृ॒ह्ण॒न्ति॒ पृ॒ष्ठैः पृ॒ष्ठैर् गृ॑ह्णन्ति गृह्णन्ति पृ॒ष्ठैः । \newline
10. पृ॒ष्ठै रु॒पाव॑रोह न्त्यु॒पाव॑रोहन्ति पृ॒ष्ठैः पृ॒ष्ठै रु॒पाव॑रोहन्ति । \newline
11. उ॒पाव॑रोहन्ति॒ यद् यदु॒पाव॑रोह न्त्यु॒पाव॑रोहन्ति॒ यत् । \newline
12. उ॒पाव॑रोह॒न्तीत्यु॑प - अव॑रोहन्ति । \newline
13. यत् परे॒ परे॒ यद् यत् परे᳚ । \newline
14. परे॑ प॒रस्ता᳚त् प॒रस्ता॒त् परे॒ परे॑ प॒रस्ता᳚त् । \newline
15. प॒रस्ता॒न् न न प॒रस्ता᳚त् प॒रस्ता॒न् न । \newline
16. न स्युः स्युर् न न स्युः । \newline
17. स्युः परा᳚ञ्चः॒ परा᳚ञ्चः॒ स्युः स्युः परा᳚ञ्चः । \newline
18. परा᳚ञ्चः सुव॒र्गाथ् सु॑व॒र्गात् परा᳚ञ्चः॒ परा᳚ञ्चः सुव॒र्गात् । \newline
19. सु॒व॒र्गाल् लो॒काल् लो॒काथ् सु॑व॒र्गाथ् सु॑व॒र्गाल् लो॒कात् । \newline
20. सु॒व॒र्गादिति॑ सुवः - गात् । \newline
21. लो॒कान् निर् णिर् लो॒काल् लो॒कान् निः । \newline
22. निष् प॑द्येरन् पद्येर॒न् निर् णिष् प॑द्येरन्न् । \newline
23. प॒द्ये॒र॒न्॒. यद् यत् प॑द्येरन् पद्येर॒न्॒. यत् । \newline
24. यद॒वस्ता॑ द॒वस्ता॒द् यद् यद॒वस्ता᳚त् । \newline
25. अ॒वस्ता॒न् न नावस्ता॑ द॒वस्ता॒न् न । \newline
26. न स्युः स्युर् न न स्युः । \newline
27. स्युः प्र॒जाः प्र॒जाः स्युः स्युः प्र॒जाः । \newline
28. प्र॒जा निर् णिष् प्र॒जाः प्र॒जा निः । \newline
29. प्र॒जा इति॑ प्र - जाः । \newline
30. निर् द॑हेयुर् दहेयु॒र् निर् णिर् द॑हेयुः । \newline
31. द॒हे॒यु॒ र॒भितो॒ ऽभितो॑ दहेयुर् दहेयु र॒भितः॑ । \newline
32. अ॒भितो॑ दिवाकी॒र्त्य॑म् दिवाकी॒र्त्य॑ म॒भितो॒ ऽभितो॑ दिवाकी॒र्त्य᳚म् । \newline
33. दि॒वा॒की॒र्त्य॑म् पर॑स्सामानः॒ पर॑स्सामानो दिवाकी॒र्त्य॑म् दिवाकी॒र्त्य॑म् पर॑स्सामानः । \newline
34. दि॒वा॒की॒र्त्य॑मिति॑ दिवा - की॒र्त्य᳚म् । \newline
35. पर॑स्सामानो भवन्ति भवन्ति॒ पर॑स्सामानः॒ पर॑स्सामानो भवन्ति । \newline
36. पर॑स्सामान॒ इति॒ परः॑ - सा॒मा॒नः॒ । \newline
37. भ॒व॒न्ति॒ सु॒व॒र्गे सु॑व॒र्गे भ॑वन्ति भवन्ति सुव॒र्गे । \newline
38. सु॒व॒र्ग ए॒वैव सु॑व॒र्गे सु॑व॒र्ग ए॒व । \newline
39. सु॒व॒र्ग इति॑ सुवः - गे । \newline
40. ए॒वैना॑ नेना ने॒वै वैनान्॑ । \newline
41. ए॒ना॒न् ॅलो॒के लो॒क ए॑ना नेनान् ॅलो॒के । \newline
42. लो॒क उ॑भ॒यत॑ उभ॒यतो॑ लो॒के लो॒क उ॑भ॒यतः॑ । \newline
43. उ॒भ॒यतः॒ परि॒ पर्यु॑भ॒यत॑ उभ॒यतः॒ परि॑ । \newline
44. परि॑ गृह्णन्ति गृह्णन्ति॒ परि॒ परि॑ गृह्णन्ति । \newline
45. गृ॒ह्ण॒न्ति॒ यज॑माना॒ यज॑माना गृह्णन्ति गृह्णन्ति॒ यज॑मानाः । \newline
46. यज॑माना॒ वै वै यज॑माना॒ यज॑माना॒ वै । \newline
47. वै दि॑वाकी॒र्त्य॑म् दिवाकी॒र्त्यं॑ ॅवै वै दि॑वाकी॒र्त्य᳚म् । \newline
48. दि॒वा॒की॒र्त्यꣳ॑ संॅवथ्स॒रः सं॑ॅवथ्स॒रो दि॑वाकी॒र्त्य॑म् दिवाकी॒र्त्यꣳ॑ संॅवथ्स॒रः । \newline
49. दि॒वा॒की᳚र्त्यमिति॑ दिवा - की॒र्त्य᳚म् । \newline
50. सं॒ॅव॒थ्स॒रः पर॑स्सामानः॒ पर॑स्सामानः संॅवथ्स॒रः सं॑ॅवथ्स॒रः पर॑स्सामानः । \newline
51. सं॒ॅव॒थ्स॒र इति॑ सं - व॒थ्स॒रः । \newline
52. पर॑स्सामानो॒ ऽभितो॒ ऽभितः॒ पर॑स्सामानः॒ पर॑स्सामानो॒ ऽभितः॑ । \newline
53. पर॑स्सामान॒ इति॒ परः॑ - सा॒मा॒नः॒ । \newline
54. अ॒भितो॑ दिवाकी॒र्त्य॑म् दिवाकी॒र्त्य॑ म॒भितो॒ ऽभितो॑ दिवाकी॒र्त्य᳚म् । \newline
55. दि॒वा॒की॒र्त्य॑म् पर॑स्सामानः॒ पर॑स्सामानो दिवाकी॒र्त्य॑म् दिवाकी॒र्त्य॑म् पर॑स्सामानः । \newline
56. दि॒वा॒की॒र्त्य॑मिति॑ दिवा - की॒र्त्य᳚म् । \newline
57. पर॑स्सामानो भवन्ति भवन्ति॒ पर॑स्सामानः॒ पर॑स्सामानो भवन्ति । \newline
58. पर॑स्सामान॒ इति॒ परः॑ - सा॒मा॒नः॒ । \newline
59. भ॒व॒न्ति॒ सं॒ॅव॒थ्स॒रे सं॑ॅवथ्स॒रे भ॑वन्ति भवन्ति संॅवथ्स॒रे । \newline
60. सं॒ॅव॒थ्स॒र ए॒वैव सं॑ॅवथ्स॒रे सं॑ॅवथ्स॒र ए॒व । \newline
61. सं॒ॅव॒थ्स॒र इति॑ सं - व॒थ्स॒रे । \newline
62. ए॒वो भ॒यत॑ उभ॒यत॑ ए॒वैवो भ॒यतः॑ । \newline
63. उ॒भ॒यतः॒ प्रति॒ प्रत्यु॑भ॒यत॑ उभ॒यतः॒ प्रति॑ । \newline

\textbf{Ghana Paata } \newline

1. सु॒व॒र्गे लो॒के लो॒के सु॑व॒र्गे सु॑व॒र्गे लो॒के प्रति॒ प्रति॑ लो॒के सु॑व॒र्गे सु॑व॒र्गे लो॒के प्रति॑ । \newline
2. सु॒व॒र्ग इति॑ सुवः - गे । \newline
3. लो॒के प्रति॒ प्रति॑ लो॒के लो॒के प्रति॑ तिष्ठन्ति तिष्ठन्ति॒ प्रति॑ लो॒के लो॒के प्रति॑ तिष्ठन्ति । \newline
4. प्रति॑ तिष्ठन्ति तिष्ठन्ति॒ प्रति॒ प्रति॑ तिष्ठन्ति॒ परैः॒ परै᳚ स्तिष्ठन्ति॒ प्रति॒ प्रति॑ तिष्ठन्ति॒ परैः᳚ । \newline
5. ति॒ष्ठ॒न्ति॒ परैः॒ परै᳚ स्तिष्ठन्ति तिष्ठन्ति॒ परैः᳚ प॒रस्ता᳚त् प॒रस्ता॒त् परै᳚ स्तिष्ठन्ति तिष्ठन्ति॒ परैः᳚ प॒रस्ता᳚त् । \newline
6. परैः᳚ प॒रस्ता᳚त् प॒रस्ता॒त् परैः॒ परैः᳚ प॒रस्ता॒त् परि॒ परि॑ प॒रस्ता॒त् परैः॒ परैः᳚ प॒रस्ता॒त् परि॑ । \newline
7. प॒रस्ता॒त् परि॒ परि॑ प॒रस्ता᳚त् प॒रस्ता॒त् परि॑ गृह्णन्ति गृह्णन्ति॒ परि॑ प॒रस्ता᳚त् प॒रस्ता॒त् परि॑ गृह्णन्ति । \newline
8. परि॑ गृह्णन्ति गृह्णन्ति॒ परि॒ परि॑ गृह्णन्ति पृ॒ष्ठैः पृ॒ष्ठैर् गृ॑ह्णन्ति॒ परि॒ परि॑ गृह्णन्ति पृ॒ष्ठैः । \newline
9. गृ॒ह्ण॒न्ति॒ पृ॒ष्ठैः पृ॒ष्ठैर् गृ॑ह्णन्ति गृह्णन्ति पृ॒ष्ठै रु॒पाव॑रोह न्त्यु॒पाव॑रोहन्ति पृ॒ष्ठैर् गृ॑ह्णन्ति गृह्णन्ति पृ॒ष्ठै रु॒पाव॑रोहन्ति । \newline
10. पृ॒ष्ठै रु॒पाव॑रोह न्त्यु॒पाव॑रोहन्ति पृ॒ष्ठैः पृ॒ष्ठै रु॒पाव॑रोहन्ति॒ यद् यदु॒पाव॑रोहन्ति पृ॒ष्ठैः पृ॒ष्ठै रु॒पाव॑रोहन्ति॒ यत् । \newline
11. उ॒पाव॑रोहन्ति॒ यद् यदु॒पाव॑रोह न्त्यु॒पाव॑रोहन्ति॒ यत् परे॒ परे॒ यदु॒पाव॑रोह न्त्यु॒पाव॑रोहन्ति॒ यत् परे᳚ । \newline
12. उ॒पाव॑रोह॒न्तीत्यु॑प - अव॑रोहन्ति । \newline
13. यत् परे॒ परे॒ यद् यत् परे॑ प॒रस्ता᳚त् प॒रस्ता॒त् परे॒ यद् यत् परे॑ प॒रस्ता᳚त् । \newline
14. परे॑ प॒रस्ता᳚त् प॒रस्ता॒त् परे॒ परे॑ प॒रस्ता॒न् न न प॒रस्ता॒त् परे॒ परे॑ प॒रस्ता॒न् न । \newline
15. प॒रस्ता॒न् न न प॒रस्ता᳚त् प॒रस्ता॒न् न स्युः स्युर् न प॒रस्ता᳚त् प॒रस्ता॒न् न स्युः । \newline
16. न स्युः स्युर् न न स्युः परा᳚ञ्चः॒ परा᳚ञ्चः॒ स्युर् न न स्युः परा᳚ञ्चः । \newline
17. स्युः परा᳚ञ्चः॒ परा᳚ञ्चः॒ स्युः स्युः परा᳚ञ्चः सुव॒र्गाथ् सु॑व॒र्गात् परा᳚ञ्चः॒ स्युः स्युः परा᳚ञ्चः सुव॒र्गात् । \newline
18. परा᳚ञ्चः सुव॒र्गाथ् सु॑व॒र्गात् परा᳚ञ्चः॒ परा᳚ञ्चः सुव॒र्गा ल्लो॒का ल्लो॒काथ् सु॑व॒र्गात् परा᳚ञ्चः॒ परा᳚ञ्चः सुव॒र्गा ल्लो॒कात् । \newline
19. सु॒व॒र्गा ल्लो॒का ल्लो॒काथ् सु॑व॒र्गाथ् सु॑व॒र्गा ल्लो॒कान् निर् णिर् लो॒काथ् सु॑व॒र्गाथ् सु॑व॒र्गा ल्लो॒कान् निः । \newline
20. सु॒व॒र्गादिति॑ सुवः - गात् । \newline
21. लो॒कान् निर् णिर् लो॒का ल्लो॒कान् निष् प॑द्येरन् पद्येर॒न् निर् लो॒का ल्लो॒कान् निष् प॑द्येरन्न् । \newline
22. निष् प॑द्येरन् पद्येर॒न् निर् णिष् प॑द्येर॒न्॒. यद् यत् प॑द्येर॒न् निर् णिष् प॑द्येर॒न्॒. यत् । \newline
23. प॒द्ये॒र॒न्॒. यद् यत् प॑द्येरन् पद्येर॒न्॒. यद॒वस्ता॑ द॒वस्ता॒द् यत् प॑द्येरन् पद्येर॒न्॒. यद॒वस्ता᳚त् । \newline
24. यद॒वस्ता॑ द॒वस्ता॒द् यद् यद॒वस्ता॒न् न नावस्ता॒द् यद् यद॒वस्ता॒न् न । \newline
25. अ॒वस्ता॒न् न नावस्ता॑ द॒वस्ता॒न् न स्युः स्युर् नावस्ता॑ द॒वस्ता॒न् न स्युः । \newline
26. न स्युः स्युर् न न स्युः प्र॒जाः प्र॒जाः स्युर् न न स्युः प्र॒जाः । \newline
27. स्युः प्र॒जाः प्र॒जाः स्युः स्युः प्र॒जा निर् णिष् प्र॒जाः स्युः स्युः प्र॒जा निः । \newline
28. प्र॒जा निर् णिष् प्र॒जाः प्र॒जा निर् द॑हेयुर् दहेयु॒र् निष् प्र॒जाः प्र॒जा निर् द॑हेयुः । \newline
29. प्र॒जा इति॑ प्र - जाः । \newline
30. निर् द॑हेयुर् दहेयु॒र् निर् णिर् द॑हेयु र॒भितो॒ ऽभितो॑ दहेयु॒र् निर् णिर् द॑हेयु र॒भितः॑ । \newline
31. द॒हे॒यु॒ र॒भितो॒ ऽभितो॑ दहेयुर् दहेयु र॒भितो॑ दिवाकी॒र्त्य॑म् दिवाकी॒र्त्य॑ म॒भितो॑ दहेयुर् दहेयु र॒भितो॑ दिवाकी॒र्त्य᳚म् । \newline
32. अ॒भितो॑ दिवाकी॒र्त्य॑म् दिवाकी॒र्त्य॑ म॒भितो॒ ऽभितो॑ दिवाकी॒र्त्य॑म् पर॑स्सामानः॒ पर॑स्सामानो दिवाकी॒र्त्य॑ म॒भितो॒ ऽभितो॑ दिवाकी॒र्त्य॑म् पर॑स्सामानः । \newline
33. दि॒वा॒की॒र्त्य॑म् पर॑स्सामानः॒ पर॑स्सामानो दिवाकी॒र्त्य॑म् दिवाकी॒र्त्य॑म् पर॑स्सामानो भवन्ति भवन्ति॒ पर॑स्सामानो दिवाकी॒र्त्य॑म् दिवाकी॒र्त्य॑म् पर॑स्सामानो भवन्ति । \newline
34. दि॒वा॒की॒र्त्य॑मिति॑ दिवा - की॒र्त्य᳚म् । \newline
35. पर॑स्सामानो भवन्ति भवन्ति॒ पर॑स्सामानः॒ पर॑स्सामानो भवन्ति सुव॒र्गे सु॑व॒र्गे भ॑वन्ति॒ पर॑स्सामानः॒ पर॑स्सामानो भवन्ति सुव॒र्गे । \newline
36. पर॑स्सामान॒ इति॒ परः॑ - सा॒मा॒नः॒ । \newline
37. भ॒व॒न्ति॒ सु॒व॒र्गे सु॑व॒र्गे भ॑वन्ति भवन्ति सुव॒र्ग ए॒वैव सु॑व॒र्गे भ॑वन्ति भवन्ति सुव॒र्ग ए॒व । \newline
38. सु॒व॒र्ग ए॒वैव सु॑व॒र्गे सु॑व॒र्ग ए॒वैना॑ नेना ने॒व सु॑व॒र्गे सु॑व॒र्ग ए॒वैनान्॑ । \newline
39. सु॒व॒र्ग इति॑ सुवः - गे । \newline
40. ए॒वैना॑ नेना ने॒वै वैना᳚न् ॅलो॒के लो॒क ए॑ना ने॒वै वैना᳚न् ॅलो॒के । \newline
41. ए॒ना॒न् ॅलो॒के लो॒क ए॑ना नेनान् ॅलो॒क उ॑भ॒यत॑ उभ॒यतो॑ लो॒क ए॑ना नेनान् ॅलो॒क उ॑भ॒यतः॑ । \newline
42. लो॒क उ॑भ॒यत॑ उभ॒यतो॑ लो॒के लो॒क उ॑भ॒यतः॒ परि॒ पर्यु॑भ॒यतो॑ लो॒के लो॒क उ॑भ॒यतः॒ परि॑ । \newline
43. उ॒भ॒यतः॒ परि॒ पर्यु॑भ॒यत॑ उभ॒यतः॒ परि॑ गृह्णन्ति गृह्णन्ति॒ पर्यु॑भ॒यत॑ उभ॒यतः॒ परि॑ गृह्णन्ति । \newline
44. परि॑ गृह्णन्ति गृह्णन्ति॒ परि॒ परि॑ गृह्णन्ति॒ यज॑माना॒ यज॑माना गृह्णन्ति॒ परि॒ परि॑ गृह्णन्ति॒ यज॑मानाः । \newline
45. गृ॒ह्ण॒न्ति॒ यज॑माना॒ यज॑माना गृह्णन्ति गृह्णन्ति॒ यज॑माना॒ वै वै यज॑माना गृह्णन्ति गृह्णन्ति॒ यज॑माना॒ वै । \newline
46. यज॑माना॒ वै वै यज॑माना॒ यज॑माना॒ वै दि॑वाकी॒र्त्य॑म् दिवाकी॒र्त्यं॑ ॅवै यज॑माना॒ यज॑माना॒ वै दि॑वाकी॒र्त्य᳚म् । \newline
47. वै दि॑वाकी॒र्त्य॑म् दिवाकी॒र्त्यं॑ ॅवै वै दि॑वाकी॒र्त्यꣳ॑ संॅवथ्स॒रः सं॑ॅवथ्स॒रो दि॑वाकी॒र्त्यं॑ ॅवै वै दि॑वाकी॒र्त्यꣳ॑ संॅवथ्स॒रः । \newline
48. दि॒वा॒की॒र्त्यꣳ॑ संॅवथ्स॒रः सं॑ॅवथ्स॒रो दि॑वाकी॒र्त्य॑म् दिवाकी॒र्त्यꣳ॑ संॅवथ्स॒रः पर॑स्सामानः॒ पर॑स्सामानः संॅवथ्स॒रो दि॑वाकी॒र्त्य॑म् दिवाकी॒र्त्यꣳ॑ संॅवथ्स॒रः पर॑स्सामानः । \newline
49. दि॒वा॒की᳚र्त्यमिति॑ दिवा - की॒र्त्य᳚म् । \newline
50. सं॒ॅव॒थ्स॒रः पर॑स्सामानः॒ पर॑स्सामानः संॅवथ्स॒रः सं॑ॅवथ्स॒रः पर॑स्सामानो॒ ऽभितो॒ ऽभितः॒ पर॑स्सामानः संॅवथ्स॒रः सं॑ॅवथ्स॒रः पर॑स्सामानो॒ ऽभितः॑ । \newline
51. सं॒ॅव॒थ्स॒र इति॑ सं - व॒थ्स॒रः । \newline
52. पर॑स्सामानो॒ ऽभितो॒ ऽभितः॒ पर॑स्सामानः॒ पर॑स्सामानो॒ ऽभितो॑ दिवाकी॒र्त्य॑म् दिवाकी॒र्त्य॑ म॒भितः॒ पर॑स्सामानः॒ पर॑स्सामानो॒ ऽभितो॑ दिवाकी॒र्त्य᳚म् । \newline
53. पर॑स्सामान॒ इति॒ परः॑ - सा॒मा॒नः॒ । \newline
54. अ॒भितो॑ दिवाकी॒र्त्य॑म् दिवाकी॒र्त्य॑ म॒भितो॒ ऽभितो॑ दिवाकी॒र्त्य॑म् पर॑स्सामानः॒ पर॑स्सामानो दिवाकी॒र्त्य॑ म॒भितो॒ ऽभितो॑ दिवाकी॒र्त्य॑म् पर॑स्सामानः । \newline
55. दि॒वा॒की॒र्त्य॑म् पर॑स्सामानः॒ पर॑स्सामानो दिवाकी॒र्त्य॑म् दिवाकी॒र्त्य॑म् पर॑स्सामानो भवन्ति भवन्ति॒ पर॑स्सामानो दिवाकी॒र्त्य॑म् दिवाकी॒र्त्य॑म् पर॑स्सामानो भवन्ति । \newline
56. दि॒वा॒की॒र्त्य॑मिति॑ दिवा - की॒र्त्य᳚म् । \newline
57. पर॑स्सामानो भवन्ति भवन्ति॒ पर॑स्सामानः॒ पर॑स्सामानो भवन्ति संॅवथ्स॒रे सं॑ॅवथ्स॒रे भ॑वन्ति॒ पर॑स्सामानः॒ पर॑स्सामानो भवन्ति संॅवथ्स॒रे । \newline
58. पर॑स्सामान॒ इति॒ परः॑ - सा॒मा॒नः॒ । \newline
59. भ॒व॒न्ति॒ सं॒ॅव॒थ्स॒रे सं॑ॅवथ्स॒रे भ॑वन्ति भवन्ति संॅवथ्स॒र ए॒वैव सं॑ॅवथ्स॒रे भ॑वन्ति भवन्ति संॅवथ्स॒र ए॒व । \newline
60. सं॒ॅव॒थ्स॒र ए॒वैव सं॑ॅवथ्स॒रे सं॑ॅवथ्स॒र ए॒वोभ॒यत॑ उभ॒यत॑ ए॒व सं॑ॅवथ्स॒रे सं॑ॅवथ्स॒र ए॒वोभ॒यतः॑ । \newline
61. सं॒ॅव॒थ्स॒र इति॑ सं - व॒थ्स॒रे । \newline
62. ए॒वोभ॒यत॑ उभ॒यत॑ ए॒वैवोभ॒यतः॒ प्रति॒ प्रत्यु॑भ॒यत॑ ए॒वैवोभ॒यतः॒ प्रति॑ । \newline
63. उ॒भ॒यतः॒ प्रति॒ प्रत्यु॑भ॒यत॑ उभ॒यतः॒ प्रति॑ तिष्ठन्ति तिष्ठन्ति॒ प्रत्यु॑भ॒यत॑ उभ॒यतः॒ प्रति॑ तिष्ठन्ति । \newline
\pagebreak
\markright{ TS 7.3.10.3  \hfill https://www.vedavms.in \hfill}

\section{ TS 7.3.10.3 }

\textbf{TS 7.3.10.3 } \newline
\textbf{Samhita Paata} \newline

प्रति॑ तिष्ठन्ति पृ॒ष्ठं ॅवै दि॑वाकी॒र्त्यं॑ पा॒र्श्वे पर॑स्सामानो॒ ऽभितो॑ दिवाकी॒र्त्यं॑ पर॑स्सामानो भवन्ति॒ तस्मा॑द॒भितः॑ पृ॒ष्ठं पा॒र्श्वे भूयि॑ष्ठा॒ ग्रहा॑ गृह्यन्ते॒ भूयि॑ष्ठꣳ शस्यते य॒ज्ञ्स्यै॒व तन्म॑द्ध्य॒तो ग्र॒न्थिं ग्र॑थ्न॒न्त्यवि॑स्रꣳसाय स॒प्त गृ॑ह्यन्ते स॒प्त वै शी॑र्.ष॒ण्याः᳚ प्रा॒णाः प्रा॒णाने॒व यज॑मानेषु दधति॒ यत् प॑रा॒चीना॑नि पृ॒ष्ठानि॒ भव॑न्त्य॒मुमे॒व तै-र्लो॒कम॒भ्यारो॑हन्ति॒ यदि॒मं ॅलो॒कं न - [  ] \newline

\textbf{Pada Paata} \newline

प्रतीति॑ । ति॒ष्ठ॒न्ति॒ । पृ॒ष्ठम् । वै । दि॒वा॒की॒र्त्य॑मिति॑ दिवा - की॒र्त्य᳚म् । पा॒र्श्वे इति॑ । पर॑स्सामान॒ इति॒ परः॑ - सा॒मा॒नः॒ । अ॒भितः॑ । दि॒वा॒की॒र्त्य॑मिति॑ दिवा - की॒र्त्य᳚म् । पर॑स्सामान॒ इति॒ परः॑-सा॒मा॒नः॒ । भ॒व॒न्ति॒ । तस्मा᳚त् । अ॒भितः॑ । पृ॒ष्ठम् । पा॒र्श्वे इति॑ । भूयि॑ष्ठाः । ग्रहाः᳚ । गृ॒ह्य॒न्ते॒ । भूयि॑ष्ठम् । श॒स्य॒ते॒ । य॒ज्ञ्स्य॑ । ए॒व । तत् । म॒द्ध्य॒तः । ग्र॒न्थिम् । ग्र॒थ्न॒न्ति॒ । अवि॑स्रꣳसा॒येत्यवि॑ - स्रꣳ॒॒सा॒य॒ । स॒प्त । गृ॒ह्य॒न्ते॒ । स॒प्त । वै । शी॒र्.॒ष॒ण्याः᳚ । प्रा॒णा इति॑ प्र - अ॒नाः । प्रा॒णानिति॑ प्र - अ॒नान् । ए॒व । यज॑मानेषु । द॒ध॒ति॒ । यत् । प॒रा॒चीना॑नि । पृ॒ष्ठानि॑ । भव॑न्ति । अ॒मुम् । ए॒व । तैः । लो॒कम् । अ॒भ्यारो॑ह॒न्तीत्य॑भि-आरो॑हन्ति । यत् । इ॒मम् । लो॒कम् । न ।  \newline


\textbf{Krama Paata} \newline

प्रति॑ तिष्ठन्ति । ति॒ष्ठ॒न्ति॒ पृ॒ष्ठम् । पृ॒ष्ठम् ॅवै । वै दि॑वाकी॒र्त्य᳚म् । दि॒वा॒की॒र्त्य॑म् पा॒र्श्वे । दि॒वा॒की॒र्त्य॑मिति॑ दिवा - की॒र्त्य᳚म् । पा॒र्श्वे पर॑स्सामानः । पा॒र्श्वे इति॑ पा॒र्श्वे । पर॑स्सामानो॒ऽभितः॑ । पर॑स्सामान॒ इति॒ परः॑ - सा॒मा॒नः॒ । अ॒भितो॑ दिवाकी॒र्त्य᳚म् । दि॒वा॒की॒र्त्य॑म् पर॑स्सामानः । दि॒वा॒की॒र्त्य॑मिति॑ दिवा - की॒र्त्य᳚म् । पर॑स्सामानो भवन्ति । पर॑स्सामान॒ इति॒ परः॑ - सा॒मा॒नः॒ । भ॒व॒न्ति॒ तस्मा᳚त् । तस्मा॑द॒भितः॑ । अ॒भितः॑ पृ॒ष्ठम् । पृ॒ष्ठम् पा॒र्श्वे । पा॒र्श्वे भूयि॑ष्ठाः । पा॒र्श्वे इति॑ पा॒र्श्वे । भूयि॑ष्ठा॒ ग्रहाः᳚ । ग्रहा॑ गृह्यन्ते । गृ॒ह्य॒न्ते॒ भूयि॑ष्ठम् । भूयि॑ष्ठꣳ शस्यते । श॒स्य॒ते॒ य॒ज्ञ्स्य॑ । य॒ज्ञ्स्यै॒व । ए॒व तत् । तन् म॑द्ध्य॒तः । म॒द्ध्य॒तो ग्र॒न्थिम् । ग्र॒न्थिम् ग्र॑थ्नन्ति । ग॒थ्न॒न्त्यवि॑स्रꣳसाय । अवि॑स्रꣳसाय स॒प्त । अवि॑स्रꣳसा॒येत्यवि॑ - स्रꣳ॒॒सा॒य॒ । स॒प्त गृ॑ह्यन्ते । गृ॒ह्य॒न्ते॒ स॒प्त । स॒प्त वै । वै शी॑र्.ष॒ण्याः᳚ । शी॒र्॒.ष॒ण्याः᳚ प्रा॒णाः । प्रा॒णाः प्रा॒णान् । प्रा॒णा इति॑ प्र - अ॒नाः । प्रा॒णाने॒व । प्रा॒णानिति॑ प्र - अ॒नान् । ए॒व यज॑मानेषु । यज॑मानेषु दधति । द॒ध॒ति॒ यत् । यत् प॑रा॒चीना॑नि । प॒रा॒चीना॑नि पृ॒ष्ठानि॑ । पृ॒ष्ठानि॒ भव॑न्ति । भव॑न्त्य॒मुम् । अ॒मुमे॒व । ए॒व तैः । तैर् लो॒कम् । लो॒कम॒भ्यारो॑हन्ति । अ॒भ्यारो॑हन्ति॒ यत् । अ॒भ्यारो॑ह॒न्तीत्य॑भि - आरो॑हन्ति । यदि॒मम् । इ॒मम् ॅलो॒कम् । लो॒कम् न । न प्र॑त्यव॒रोहे॑युः \newline

\textbf{Jatai Paata} \newline

1. प्रति॑ तिष्ठन्ति तिष्ठन्ति॒ प्रति॒ प्रति॑ तिष्ठन्ति । \newline
2. ति॒ष्ठ॒न्ति॒ पृ॒ष्ठम् पृ॒ष्ठम् ति॑ष्ठन्ति तिष्ठन्ति पृ॒ष्ठम् । \newline
3. पृ॒ष्ठं ॅवै वै पृ॒ष्ठम् पृ॒ष्ठं ॅवै । \newline
4. वै दि॑वाकी॒र्त्य॑म् दिवाकी॒र्त्यं॑ ॅवै वै दि॑वाकी॒र्त्य᳚म् । \newline
5. दि॒वा॒की॒र्त्य॑म् पा॒र्श्वे पा॒र्श्वे दि॑वाकी॒र्त्य॑म् दिवाकी॒र्त्य॑म् पा॒र्श्वे । \newline
6. दि॒वा॒की॒र्त्य॑मिति॑ दिवा - की॒र्त्य᳚म् । \newline
7. पा॒र्श्वे पर॑स्सामानः॒ पर॑स्सामानः पा॒र्श्वे पा॒र्श्वे पर॑स्सामानः । \newline
8. पा॒र्श्वे इति॑ पा॒र्श्वे । \newline
9. पर॑स्सामानो॒ ऽभितो॒ ऽभितः॒ पर॑स्सामानः॒ पर॑स्सामानो॒ ऽभितः॑ । \newline
10. पर॑स्सामान॒ इति॒ परः॑ - सा॒मा॒नः॒ । \newline
11. अ॒भितो॑ दिवाकी॒र्त्य॑म् दिवाकी॒र्त्य॑ म॒भितो॒ ऽभितो॑ दिवाकी॒र्त्य᳚म् । \newline
12. दि॒वा॒की॒र्त्य॑म् पर॑स्सामानः॒ पर॑स्सामानो दिवाकी॒र्त्य॑म् दिवाकी॒र्त्य॑म् पर॑स्सामानः । \newline
13. दि॒वा॒की॒र्त्य॑मिति॑ दिवा - की॒र्त्य᳚म् । \newline
14. पर॑स्सामानो भवन्ति भवन्ति॒ पर॑स्सामानः॒ पर॑स्सामानो भवन्ति । \newline
15. पर॑स्सामान॒ इति॒ परः॑ - सा॒मा॒नः॒ । \newline
16. भ॒व॒न्ति॒ तस्मा॒त् तस्मा᳚द् भवन्ति भवन्ति॒ तस्मा᳚त् । \newline
17. तस्मा॑ द॒भितो॒ ऽभित॒ स्तस्मा॒त् तस्मा॑ द॒भितः॑ । \newline
18. अ॒भितः॑ पृ॒ष्ठम् पृ॒ष्ठ म॒भितो॒ ऽभितः॑ पृ॒ष्ठम् । \newline
19. पृ॒ष्ठम् पा॒र्श्वे पा॒र्श्वे पृ॒ष्ठम् पृ॒ष्ठम् पा॒र्श्वे । \newline
20. पा॒र्श्वे भूयि॑ष्ठा॒ भूयि॑ष्ठाः पा॒र्श्वे पा॒र्श्वे भूयि॑ष्ठाः । \newline
21. पा॒र्श्वे इति॑ पा॒र्श्वे । \newline
22. भूयि॑ष्ठा॒ ग्रहा॒ ग्रहा॒ भूयि॑ष्ठा॒ भूयि॑ष्ठा॒ ग्रहाः᳚ । \newline
23. ग्रहा॑ गृह्यन्ते गृह्यन्ते॒ ग्रहा॒ ग्रहा॑ गृह्यन्ते । \newline
24. गृ॒ह्य॒न्ते॒ भूयि॑ष्ठ॒म् भूयि॑ष्ठम् गृह्यन्ते गृह्यन्ते॒ भूयि॑ष्ठम् । \newline
25. भूयि॑ष्ठꣳ शस्यते शस्यते॒ भूयि॑ष्ठ॒म् भूयि॑ष्ठꣳ शस्यते । \newline
26. श॒स्य॒ते॒ य॒ज्ञ्स्य॑ य॒ज्ञ्स्य॑ शस्यते शस्यते य॒ज्ञ्स्य॑ । \newline
27. य॒ज्ञ्स्यै॒वैव य॒ज्ञ्स्य॑ य॒ज्ञ्स्यै॒व । \newline
28. ए॒व तत् तदे॒ वैव तत् । \newline
29. तन् म॑द्ध्य॒तो म॑द्ध्य॒त स्तत् तन् म॑द्ध्य॒तः । \newline
30. म॒द्ध्य॒तो ग्र॒न्थिम् ग्र॒न्थिम् म॑द्ध्य॒तो म॑द्ध्य॒तो ग्र॒न्थिम् । \newline
31. ग्र॒न्थिम् ग्र॑थ्नन्ति ग्रथ्नन्ति ग्र॒न्थिम् ग्र॒न्थिम् ग्र॑थ्नन्ति । \newline
32. ग्र॒थ्न॒ न्त्यवि॑स्रꣳसा॒या वि॑स्रꣳसाय ग्रथ्नन्ति ग्रथ्न॒ न्त्यवि॑स्रꣳसाय । \newline
33. अवि॑स्रꣳसाय स॒प्त स॒प्ता वि॑स्रꣳसा॒या वि॑स्रꣳसाय स॒प्त । \newline
34. अवि॑स्रꣳसा॒येत्यवि॑ - स्रꣳ॒॒सा॒य॒ । \newline
35. स॒प्त गृ॑ह्यन्ते गृह्यन्ते स॒प्त स॒प्त गृ॑ह्यन्ते । \newline
36. गृ॒ह्य॒न्ते॒ स॒प्त स॒प्त गृ॑ह्यन्ते गृह्यन्ते स॒प्त । \newline
37. स॒प्त वै वै स॒प्त स॒प्त वै । \newline
38. वै शी॑र्.ष॒ण्याः᳚ शीर्.ष॒ण्या॑ वै वै शी॑र्.ष॒ण्याः᳚ । \newline
39. शी॒र्॒.ष॒ण्याः᳚ प्रा॒णाः प्रा॒णाः शी॑र्.ष॒ण्याः᳚ शीर्.ष॒ण्याः᳚ प्रा॒णाः । \newline
40. प्रा॒णाः प्रा॒णान् प्रा॒णान् प्रा॒णाः प्रा॒णाः प्रा॒णान् । \newline
41. प्रा॒णा इति॑ प्र - अ॒नाः । \newline
42. प्रा॒णा ने॒वैव प्रा॒णान् प्रा॒णा ने॒व । \newline
43. प्रा॒णानिति॑ प्र - अ॒नान् । \newline
44. ए॒व यज॑मानेषु॒ यज॑माने ष्वे॒वैव यज॑मानेषु । \newline
45. यज॑मानेषु दधति दधति॒ यज॑मानेषु॒ यज॑मानेषु दधति । \newline
46. द॒ध॒ति॒ यद् यद् द॑धति दधति॒ यत् । \newline
47. यत् प॑रा॒चीना॑नि परा॒चीना॑नि॒ यद् यत् प॑रा॒चीना॑नि । \newline
48. प॒रा॒चीना॑नि पृ॒ष्ठानि॑ पृ॒ष्ठानि॑ परा॒चीना॑नि परा॒चीना॑नि पृ॒ष्ठानि॑ । \newline
49. पृ॒ष्ठानि॒ भव॑न्ति॒ भव॑न्ति पृ॒ष्ठानि॑ पृ॒ष्ठानि॒ भव॑न्ति । \newline
50. भव॑ न्त्य॒मु म॒मुम् भव॑न्ति॒ भव॑ न्त्य॒मुम् । \newline
51. अ॒मु मे॒वै वामु म॒मु मे॒व । \newline
52. ए॒व तै स्तै रे॒वैव तैः । \newline
53. तैर् लो॒कम् ॅलो॒कम् तै स्तैर् लो॒कम् । \newline
54. लो॒क म॒भ्यारो॑ह न्त्य॒भ्यारो॑हन्ति लो॒कम् ॅलो॒क म॒भ्यारो॑हन्ति । \newline
55. अ॒भ्यारो॑हन्ति॒ यद् यद॒भ्यारो॑ह न्त्य॒भ्यारो॑हन्ति॒ यत् । \newline
56. अ॒भ्यारो॑ह॒न्तीत्य॑भि - आरो॑हन्ति । \newline
57. यदि॒म मि॒मं ॅयद् यदि॒मम् । \newline
58. इ॒मम् ॅलो॒कम् ॅलो॒क मि॒म मि॒मम् ॅलो॒कम् । \newline
59. लो॒कन् न न लो॒कम् ॅलो॒कन् न । \newline
60. न प्र॑त्यव॒रोहे॑युः प्रत्यव॒रोहे॑यु॒र् न न प्र॑त्यव॒रोहे॑युः । \newline

\textbf{Ghana Paata } \newline

1. प्रति॑ तिष्ठन्ति तिष्ठन्ति॒ प्रति॒ प्रति॑ तिष्ठन्ति पृ॒ष्ठम् पृ॒ष्ठम् ति॑ष्ठन्ति॒ प्रति॒ प्रति॑ तिष्ठन्ति पृ॒ष्ठम् । \newline
2. ति॒ष्ठ॒न्ति॒ पृ॒ष्ठम् पृ॒ष्ठम् ति॑ष्ठन्ति तिष्ठन्ति पृ॒ष्ठं ॅवै वै पृ॒ष्ठम् ति॑ष्ठन्ति तिष्ठन्ति पृ॒ष्ठं ॅवै । \newline
3. पृ॒ष्ठं ॅवै वै पृ॒ष्ठम् पृ॒ष्ठं ॅवै दि॑वाकी॒र्त्य॑म् दिवाकी॒र्त्यं॑ ॅवै पृ॒ष्ठम् पृ॒ष्ठं ॅवै दि॑वाकी॒र्त्य᳚म् । \newline
4. वै दि॑वाकी॒र्त्य॑म् दिवाकी॒र्त्यं॑ ॅवै वै दि॑वाकी॒र्त्य॑म् पा॒र्श्वे पा॒र्श्वे दि॑वाकी॒र्त्यं॑ ॅवै वै दि॑वाकी॒र्त्य॑म् पा॒र्श्वे । \newline
5. दि॒वा॒की॒र्त्य॑म् पा॒र्श्वे पा॒र्श्वे दि॑वाकी॒र्त्य॑म् दिवाकी॒र्त्य॑म् पा॒र्श्वे पर॑स्सामानः॒ पर॑स्सामानः पा॒र्श्वे दि॑वाकी॒र्त्य॑म् दिवाकी॒र्त्य॑म् पा॒र्श्वे पर॑स्सामानः । \newline
6. दि॒वा॒की॒र्त्य॑मिति॑ दिवा - की॒र्त्य᳚म् । \newline
7. पा॒र्श्वे पर॑स्सामानः॒ पर॑स्सामानः पा॒र्श्वे पा॒र्श्वे पर॑स्सामानो॒ ऽभितो॒ ऽभितः॒ पर॑स्सामानः पा॒र्श्वे पा॒र्श्वे पर॑स्सामानो॒ ऽभितः॑ । \newline
8. पा॒र्श्वे इति॑ पा॒र्श्वे । \newline
9. पर॑स्सामानो॒ ऽभितो॒ ऽभितः॒ पर॑स्सामानः॒ पर॑स्सामानो॒ ऽभितो॑ दिवाकी॒र्त्य॑म् दिवाकी॒र्त्य॑ म॒भितः॒ पर॑स्सामानः॒ पर॑स्सामानो॒ ऽभितो॑ दिवाकी॒र्त्य᳚म् । \newline
10. पर॑स्सामान॒ इति॒ परः॑ - सा॒मा॒नः॒ । \newline
11. अ॒भितो॑ दिवाकी॒र्त्य॑म् दिवाकी॒र्त्य॑ म॒भितो॒ ऽभितो॑ दिवाकी॒र्त्य॑म् पर॑स्सामानः॒ पर॑स्सामानो दिवाकी॒र्त्य॑ म॒भितो॒ ऽभितो॑ दिवाकी॒र्त्य॑म् पर॑स्सामानः । \newline
12. दि॒वा॒की॒र्त्य॑म् पर॑स्सामानः॒ पर॑स्सामानो दिवाकी॒र्त्य॑म् दिवाकी॒र्त्य॑म् पर॑स्सामानो भवन्ति भवन्ति॒ पर॑स्सामानो दिवाकी॒र्त्य॑म् दिवाकी॒र्त्य॑म् पर॑स्सामानो भवन्ति । \newline
13. दि॒वा॒की॒र्त्य॑मिति॑ दिवा - की॒र्त्य᳚म् । \newline
14. पर॑स्सामानो भवन्ति भवन्ति॒ पर॑स्सामानः॒ पर॑स्सामानो भवन्ति॒ तस्मा॒त् तस्मा᳚द् भवन्ति॒ पर॑स्सामानः॒ पर॑स्सामानो भवन्ति॒ तस्मा᳚त् । \newline
15. पर॑स्सामान॒ इति॒ परः॑ - सा॒मा॒नः॒ । \newline
16. भ॒व॒न्ति॒ तस्मा॒त् तस्मा᳚द् भवन्ति भवन्ति॒ तस्मा॑ द॒भितो॒ ऽभित॒ स्तस्मा᳚द् भवन्ति भवन्ति॒ तस्मा॑ द॒भितः॑ । \newline
17. तस्मा॑ द॒भितो॒ ऽभित॒ स्तस्मा॒त् तस्मा॑ द॒भितः॑ पृ॒ष्ठम् पृ॒ष्ठ म॒भित॒ स्तस्मा॒त् तस्मा॑ द॒भितः॑ पृ॒ष्ठम् । \newline
18. अ॒भितः॑ पृ॒ष्ठम् पृ॒ष्ठ म॒भितो॒ ऽभितः॑ पृ॒ष्ठम् पा॒र्श्वे पा॒र्श्वे पृ॒ष्ठ म॒भितो॒ ऽभितः॑ पृ॒ष्ठम् पा॒र्श्वे । \newline
19. पृ॒ष्ठम् पा॒र्श्वे पा॒र्श्वे पृ॒ष्ठम् पृ॒ष्ठम् पा॒र्श्वे भूयि॑ष्ठा॒ भूयि॑ष्ठाः पा॒र्श्वे पृ॒ष्ठम् पृ॒ष्ठम् पा॒र्श्वे भूयि॑ष्ठाः । \newline
20. पा॒र्श्वे भूयि॑ष्ठा॒ भूयि॑ष्ठाः पा॒र्श्वे पा॒र्श्वे भूयि॑ष्ठा॒ ग्रहा॒ ग्रहा॒ भूयि॑ष्ठाः पा॒र्श्वे पा॒र्श्वे भूयि॑ष्ठा॒ ग्रहाः᳚ । \newline
21. पा॒र्श्वे इति॑ पा॒र्श्वे । \newline
22. भूयि॑ष्ठा॒ ग्रहा॒ ग्रहा॒ भूयि॑ष्ठा॒ भूयि॑ष्ठा॒ ग्रहा॑ गृह्यन्ते गृह्यन्ते॒ ग्रहा॒ भूयि॑ष्ठा॒ भूयि॑ष्ठा॒ ग्रहा॑ गृह्यन्ते । \newline
23. ग्रहा॑ गृह्यन्ते गृह्यन्ते॒ ग्रहा॒ ग्रहा॑ गृह्यन्ते॒ भूयि॑ष्ठ॒म् भूयि॑ष्ठम् गृह्यन्ते॒ ग्रहा॒ ग्रहा॑ गृह्यन्ते॒ भूयि॑ष्ठम् । \newline
24. गृ॒ह्य॒न्ते॒ भूयि॑ष्ठ॒म् भूयि॑ष्ठम् गृह्यन्ते गृह्यन्ते॒ भूयि॑ष्ठꣳ शस्यते शस्यते॒ भूयि॑ष्ठम् गृह्यन्ते गृह्यन्ते॒ भूयि॑ष्ठꣳ शस्यते । \newline
25. भूयि॑ष्ठꣳ शस्यते शस्यते॒ भूयि॑ष्ठ॒म् भूयि॑ष्ठꣳ शस्यते य॒ज्ञ्स्य॑ य॒ज्ञ्स्य॑ शस्यते॒ भूयि॑ष्ठ॒म् भूयि॑ष्ठꣳ शस्यते य॒ज्ञ्स्य॑ । \newline
26. श॒स्य॒ते॒ य॒ज्ञ्स्य॑ य॒ज्ञ्स्य॑ शस्यते शस्यते य॒ज्ञ्स्यै॒वैव य॒ज्ञ्स्य॑ शस्यते शस्यते य॒ज्ञ्स्यै॒व । \newline
27. य॒ज्ञ्स्यै॒वैव य॒ज्ञ्स्य॑ य॒ज्ञ्स्यै॒व तत् तदे॒व य॒ज्ञ्स्य॑ य॒ज्ञ्स्यै॒व तत् । \newline
28. ए॒व तत् तदे॒ वैव तन् म॑द्ध्य॒तो म॑द्ध्य॒त स्तदे॒ वैव तन् म॑द्ध्य॒तः । \newline
29. तन् म॑द्ध्य॒तो म॑द्ध्य॒त स्तत् तन् म॑द्ध्य॒तो ग्र॒न्थिम् ग्र॒न्थिम् म॑द्ध्य॒त स्तत् तन् म॑द्ध्य॒तो ग्र॒न्थिम् । \newline
30. म॒द्ध्य॒तो ग्र॒न्थिम् ग्र॒न्थिम् म॑द्ध्य॒तो म॑द्ध्य॒तो ग्र॒न्थिम् ग्र॑थ्नन्ति ग्रथ्नन्ति ग्र॒न्थिम् म॑द्ध्य॒तो म॑द्ध्य॒तो ग्र॒न्थिम् ग्र॑थ्नन्ति । \newline
31. ग्र॒न्थिम् ग्र॑थ्नन्ति ग्रथ्नन्ति ग्र॒न्थिम् ग्र॒न्थिम् ग्र॑थ्न॒ न्त्यवि॑स्रꣳसा॒या वि॑स्रꣳसाय ग्रथ्नन्ति ग्र॒न्थिम् ग्र॒न्थिम् ग्र॑थ्न॒ न्त्यवि॑स्रꣳसाय । \newline
32. ग्र॒थ्न॒ न्त्यवि॑स्रꣳसा॒या वि॑स्रꣳसाय ग्रथ्नन्ति ग्रथ्न॒ न्त्यवि॑स्रꣳसाय स॒प्त स॒प्ता वि॑स्रꣳसाय ग्रथ्नन्ति ग्रथ्न॒ न्त्यवि॑स्रꣳसाय स॒प्त । \newline
33. अवि॑स्रꣳसाय स॒प्त स॒प्ता वि॑स्रꣳसा॒या वि॑स्रꣳसाय स॒प्त गृ॑ह्यन्ते गृह्यन्ते स॒प्ता वि॑स्रꣳसा॒या वि॑स्रꣳसाय स॒प्त गृ॑ह्यन्ते । \newline
34. अवि॑स्रꣳसा॒येत्यवि॑ - स्रꣳ॒॒सा॒य॒ । \newline
35. स॒प्त गृ॑ह्यन्ते गृह्यन्ते स॒प्त स॒प्त गृ॑ह्यन्ते स॒प्त स॒प्त गृ॑ह्यन्ते स॒प्त स॒प्त गृ॑ह्यन्ते स॒प्त । \newline
36. गृ॒ह्य॒न्ते॒ स॒प्त स॒प्त गृ॑ह्यन्ते गृह्यन्ते स॒प्त वै वै स॒प्त गृ॑ह्यन्ते गृह्यन्ते स॒प्त वै । \newline
37. स॒प्त वै वै स॒प्त स॒प्त वै शी॑र्.ष॒ण्याः᳚ शीर्.ष॒ण्या॑ वै स॒प्त स॒प्त वै शी॑र्.ष॒ण्याः᳚ । \newline
38. वै शी॑र्.ष॒ण्याः᳚ शीर्.ष॒ण्या॑ वै वै शी॑र्.ष॒ण्याः᳚ प्रा॒णाः प्रा॒णाः शी॑र्.ष॒ण्या॑ वै वै शी॑र्.ष॒ण्याः᳚ प्रा॒णाः । \newline
39. शी॒र्॒.ष॒ण्याः᳚ प्रा॒णाः प्रा॒णाः शी॑र्.ष॒ण्याः᳚ शीर्.ष॒ण्याः᳚ प्रा॒णाः प्रा॒णान् प्रा॒णान् प्रा॒णाः शी॑र्.ष॒ण्याः᳚ शीर्.ष॒ण्याः᳚ प्रा॒णाः प्रा॒णान् । \newline
40. प्रा॒णाः प्रा॒णान् प्रा॒णान् प्रा॒णाः प्रा॒णाः प्रा॒णा ने॒वैव प्रा॒णान् प्रा॒णाः प्रा॒णाः प्रा॒णा ने॒व । \newline
41. प्रा॒णा इति॑ प्र - अ॒नाः । \newline
42. प्रा॒णा ने॒वैव प्रा॒णान् प्रा॒णा ने॒व यज॑मानेषु॒ यज॑मा नेष्वे॒व प्रा॒णान् प्रा॒णा ने॒व यज॑मानेषु । \newline
43. प्रा॒णानिति॑ प्र - अ॒नान् । \newline
44. ए॒व यज॑मानेषु॒ यज॑माने ष्वे॒वैव यज॑मानेषु दधति दधति॒ यज॑माने ष्वे॒वैव यज॑मानेषु दधति । \newline
45. यज॑मानेषु दधति दधति॒ यज॑मानेषु॒ यज॑मानेषु दधति॒ यद् यद् द॑धति॒ यज॑मानेषु॒ यज॑मानेषु दधति॒ यत् । \newline
46. द॒ध॒ति॒ यद् यद् द॑धति दधति॒ यत् प॑रा॒चीना॑नि परा॒चीना॑नि॒ यद् द॑धति दधति॒ यत् प॑रा॒चीना॑नि । \newline
47. यत् प॑रा॒चीना॑नि परा॒चीना॑नि॒ यद् यत् प॑रा॒चीना॑नि पृ॒ष्ठानि॑ पृ॒ष्ठानि॑ परा॒चीना॑नि॒ यद् यत् प॑रा॒चीना॑नि पृ॒ष्ठानि॑ । \newline
48. प॒रा॒चीना॑नि पृ॒ष्ठानि॑ पृ॒ष्ठानि॑ परा॒चीना॑नि परा॒चीना॑नि पृ॒ष्ठानि॒ भव॑न्ति॒ भव॑न्ति पृ॒ष्ठानि॑ परा॒चीना॑नि परा॒चीना॑नि पृ॒ष्ठानि॒ भव॑न्ति । \newline
49. पृ॒ष्ठानि॒ भव॑न्ति॒ भव॑न्ति पृ॒ष्ठानि॑ पृ॒ष्ठानि॒ भव॑ न्त्य॒मु म॒मुम् भव॑न्ति पृ॒ष्ठानि॑ पृ॒ष्ठानि॒ भव॑ न्त्य॒मुम् । \newline
50. भव॑ न्त्य॒मु म॒मुम् भव॑न्ति॒ भव॑ न्त्य॒मु मे॒वै वामुम् भव॑न्ति॒ भव॑ न्त्य॒मु मे॒व । \newline
51. अ॒मु मे॒वै वामु म॒मु मे॒व तै स्तै रे॒वामु म॒मु मे॒व तैः । \newline
52. ए॒व तै स्तै रे॒वैव तैर् लो॒कम् ॅलो॒कम् तै रे॒वैव तैर् लो॒कम् । \newline
53. तैर् लो॒कम् ॅलो॒कम् तै स्तैर् लो॒क म॒भ्यारो॑ह न्त्य॒भ्यारो॑हन्ति लो॒कम् तै स्तैर् लो॒क म॒भ्यारो॑हन्ति । \newline
54. लो॒क म॒भ्यारो॑ह न्त्य॒भ्यारो॑हन्ति लो॒कम् ॅलो॒क म॒भ्यारो॑हन्ति॒ यद् यद॒भ्यारो॑हन्ति लो॒कम् ॅलो॒क म॒भ्यारो॑हन्ति॒ यत् । \newline
55. अ॒भ्यारो॑हन्ति॒ यद् यद॒भ्यारो॑ह न्त्य॒भ्यारो॑हन्ति॒ यदि॒म मि॒मं ॅयद॒भ्यारो॑ह न्त्य॒भ्यारो॑हन्ति॒ यदि॒मम् । \newline
56. अ॒भ्यारो॑ह॒न्तीत्य॑भि - आरो॑हन्ति । \newline
57. यदि॒म मि॒मं ॅयद् यदि॒मम् ॅलो॒कम् ॅलो॒क मि॒मं ॅयद् यदि॒मम् ॅलो॒कम् । \newline
58. इ॒मम् ॅलो॒कम् ॅलो॒क मि॒म मि॒मम् ॅलो॒कन् न न लो॒क मि॒म मि॒मम् ॅलो॒कन् न । \newline
59. लो॒कन् न न लो॒कम् ॅलो॒कन् न प्र॑त्यव॒रोहे॑युः प्रत्यव॒रोहे॑यु॒र् न लो॒कम् ॅलो॒कन् न प्र॑त्यव॒रोहे॑युः । \newline
60. न प्र॑त्यव॒रोहे॑युः प्रत्यव॒रोहे॑यु॒र् न न प्र॑त्यव॒रोहे॑यु॒ रुदुत् प्र॑त्यव॒रोहे॑यु॒र् न न प्र॑त्यव॒रोहे॑यु॒ रुत् । \newline
\pagebreak
\markright{ TS 7.3.10.4  \hfill https://www.vedavms.in \hfill}

\section{ TS 7.3.10.4 }

\textbf{TS 7.3.10.4 } \newline
\textbf{Samhita Paata} \newline

प्र॑त्यव॒-रोहे॑यु॒रुद्वा॒ माद्ये॑यु॒र्यज॑मानाः॒ प्र वा॑ मीयेर॒न्॒. यत् प्र॑ती॒चीना॑नि पृ॒ष्ठानि॒ भव॑न्ती॒ममे॒व तैर्लो॒कं प्र॒त्यव॑रोह॒न्त्यथो॑ अ॒स्मिन्ने॒व लो॒के प्रति॑ तिष्ठ॒न्त्यनु॑न्मादा॒येन्द्रो॒ वा अप्र॑तिष्ठित आसी॒थ् स प्र॒जाप॑ति॒-मुपा॑धाव॒त् तस्मा॑ ए॒तमे॑कविꣳशतिरा॒त्रं प्राय॑च्छ॒त् तमाऽह॑र॒त् तेना॑यजत॒ ततो॒ वै स प्रत्य॑तिष्ठ॒द्ये ब॑हुया॒जिनो ऽप्र॑तिष्ठिताः॒ - [  ] \newline

\textbf{Pada Paata} \newline

प्र॒त्य॒व॒रोहे॑यु॒रिति॑ प्रति - अ॒व॒रोहे॑युः । उदिति॑ । वा॒ । माद्ये॑युः । यज॑मानाः । प्रेति॑ । वा॒ । मी॒ये॒र॒न्न् । यत् । प्र॒ती॒चीना॑नि । पृ॒ष्ठानि॑ । भव॑न्ति । इ॒मम् । ए॒व । तैः । लो॒कम् । प्र॒त्यव॑रोह॒न्तीति॑ प्रति - अव॑रोहन्ति । अथो॒ इति॑ । अ॒स्मिन्न् । ए॒व । लो॒के । प्रतीति॑ । ति॒ष्ठ॒न्ति॒ । अनु॑न्मादा॒येत्यनु॑त् - मा॒दा॒य॒ । इन्द्रः॑ । वै । अप्र॑तिष्ठित॒ इत्यप्र॑ति - स्थि॒तः॒ । आ॒सी॒त् । सः । प्र॒जाप॑ति॒मिति॑ प्र॒जा-प॒ति॒म् । उपेति॑ । अ॒धा॒व॒त् । तस्मै᳚ । ए॒तम् । ए॒क॒विꣳ॒॒श॒ति॒रा॒त्रमित्ये॑कविꣳशति - रा॒त्रम् । प्रेति॑ । अ॒य॒च्छ॒त् । तम् । एति॑ । अ॒ह॒र॒त् । तेन॑ । अ॒य॒ज॒त॒ । ततः॑ । वै । सः । प्रतीति॑ । अ॒ति॒ष्ठ॒त् । ये । ब॒हु॒या॒जिन॒ इति॑ बहु - या॒जिनः॑ । अप्र॑तिष्ठिता॒ इत्यप्र॑ति - स्थि॒ताः॒ ।  \newline


\textbf{Krama Paata} \newline

प्र॒त्य॒व॒रोहे॑यु॒रुत् । प्र॒त्य॒व॒रोहे॑यु॒रिति॑ प्रति - अ॒व॒रोहे॑युः । उद् वा᳚ । वा॒ माद्ये॑युः । माद्ये॑यु॒र् यज॑मानाः । यज॑मानाः॒ प्र । प्र वा᳚ । वा॒ मी॒ये॒र॒न्न्॒ । मी॒ये॒र॒न्॒. यत् । यत् प्र॑ती॒चीना॑नि । प्र॒ती॒चीना॑नि पृ॒ष्ठानि॑ । पृ॒ष्ठानि॒ भव॑न्ति । भव॑न्ती॒मम् । इ॒ममे॒व । ए॒व तैः । तैर् लो॒कम् । लो॒कम् प्र॒त्यव॑रोहन्ति । प्र॒त्यव॑रोह॒न्त्यथो᳚ । प्र॒त्यव॑रोह॒न्तीति॑ प्रति - अव॑रोहन्ति । अथो॑ अ॒स्मिन्न् । अथो॒ इत्यथो᳚ । अ॒स्मिन्ने॒व । ए॒व लो॒के । लो॒के प्रति॑ । प्रति॑ तिष्ठन्ति । ति॒ष्ठ॒न्त्यनु॑न्मादाय । अनु॑न्मादा॒येन्द्रः॑ । अनु॑न्मादा॒येत्यनु॑त् - मा॒दा॒य॒ । इन्द्रो॒ वै । वा अप्र॑तिष्ठितः । अप्र॑तिष्ठित आसीत् । अप्र॑तिष्ठित॒ इत्यप्र॑ति - स्थि॒तः॒ । आ॒सी॒थ् सः । स प्र॒जाप॑तिम् । प्र॒जाप॑ति॒मुप॑ । प्र॒जाप॑ति॒मिति॑ प्र॒जा - प॒ति॒म् । उपा॑धावत् । अ॒धा॒व॒त् तस्मै᳚ । तस्मा॑ ए॒तम् । ए॒तमे॑कविꣳशतिरा॒त्रम् । ए॒क॒विꣳ॒॒श॒ति॒रा॒त्रम् प्र । ए॒क॒विꣳ॒॒श॒ति॒रा॒त्रमित्ये॑कविꣳशति - रा॒त्रम् । प्राय॑च्छत् । अ॒य॒च्छ॒त् तम् । तमा । आऽह॑रत् । अ॒ह॒र॒त् तेन॑ । तेना॑यजत । अ॒य॒ज॒त॒ ततः॑ । ततो॒ वै । वै सः । स प्रति॑ । प्रत्य॑तिष्ठत् । अ॒ति॒ष्ठ॒द् ये । ये ब॑हुया॒जिनः॑ । ब॒हु॒या॒जिनोऽप्र॑तिष्ठिताः । ब॒हु॒या॒जिन॒ इति॑ बहु - या॒जिनः॑ । अप्र॑तिष्ठिताः॒ स्युः । अप्र॑तिष्ठिता॒ इत्यप्र॑ति - स्थि॒ताः॒ \newline

\textbf{Jatai Paata} \newline

1. प्र॒त्य॒व॒रोहे॑यु॒ रुदुत् प्र॑त्यव॒रोहे॑युः प्रत्यव॒रोहे॑यु॒ रुत् । \newline
2. प्र॒त्य॒व॒रोहे॑यु॒रिति॑ प्रति - अ॒व॒रोहे॑युः । \newline
3. उद् वा॒ वोदुद् वा᳚ । \newline
4. वा॒ माद्ये॑यु॒र् माद्ये॑युर् वा वा॒ माद्ये॑युः । \newline
5. माद्ये॑यु॒र् यज॑माना॒ यज॑माना॒ माद्ये॑यु॒र् माद्ये॑यु॒र् यज॑मानाः । \newline
6. यज॑मानाः॒ प्र प्र यज॑माना॒ यज॑मानाः॒ प्र । \newline
7. प्र वा॑ वा॒ प्र प्र वा᳚ । \newline
8. वा॒ मी॒ये॒र॒न् मी॒ये॒र॒न्॒. वा॒ वा॒ मी॒ये॒र॒न्न् । \newline
9. मी॒ये॒र॒न्॒. यद् यन् मी॑येरन् मीयेर॒न्॒. यत् । \newline
10. यत् प्र॑ती॒चीना॑नि प्रती॒चीना॑नि॒ यद् यत् प्र॑ती॒चीना॑नि । \newline
11. प्र॒ती॒चीना॑नि पृ॒ष्ठानि॑ पृ॒ष्ठानि॑ प्रती॒चीना॑नि प्रती॒चीना॑नि पृ॒ष्ठानि॑ । \newline
12. पृ॒ष्ठानि॒ भव॑न्ति॒ भव॑न्ति पृ॒ष्ठानि॑ पृ॒ष्ठानि॒ भव॑न्ति । \newline
13. भव॑ न्ती॒म मि॒मम् भव॑न्ति॒ भव॑ न्ती॒मम् । \newline
14. इ॒म मे॒वैवेम मि॒म मे॒व । \newline
15. ए॒व तै स्तै रे॒वैव तैः । \newline
16. तैर् लो॒कम् ॅलो॒कम् तै स्तैर् लो॒कम् । \newline
17. लो॒कम् प्र॒त्यव॑रोहन्ति प्र॒त्यव॑रोहन्ति लो॒कम् ॅलो॒कम् प्र॒त्यव॑रोहन्ति । \newline
18. प्र॒त्यव॑रोह॒ न्त्यथो॒ अथो᳚ प्र॒त्यव॑रोहन्ति प्र॒त्यव॑रोह॒ न्त्यथो᳚ । \newline
19. प्र॒त्यव॑रोह॒न्तीति॑ प्रति - अव॑रोहन्ति । \newline
20. अथो॑ अ॒स्मिन् न॒स्मिन् नथो॒ अथो॑ अ॒स्मिन्न् । \newline
21. अथो॒ इत्यथो᳚ । \newline
22. अ॒स्मिन् ने॒वै वास्मिन् न॒स्मिन् ने॒व । \newline
23. ए॒व लो॒के लो॒क ए॒वैव लो॒के । \newline
24. लो॒के प्रति॒ प्रति॑ लो॒के लो॒के प्रति॑ । \newline
25. प्रति॑ तिष्ठन्ति तिष्ठन्ति॒ प्रति॒ प्रति॑ तिष्ठन्ति । \newline
26. ति॒ष्ठ॒ न्त्यनु॑न्मादा॒या नु॑न्मादाय तिष्ठन्ति तिष्ठ॒ न्त्यनु॑न्मादाय । \newline
27. अनु॑न्मादा॒ येन्द्र॒ इन्द्रो ऽनु॑न्मादा॒या नु॑न्मादा॒येन्द्रः॑ । \newline
28. अनु॑न्मादा॒येत्यनु॑त् - मा॒दा॒य॒ । \newline
29. इन्द्रो॒ वै वा इन्द्र॒ इन्द्रो॒ वै । \newline
30. वा अप्र॑तिष्ठि॒तो ऽप्र॑तिष्ठितो॒ वै वा अप्र॑तिष्ठितः । \newline
31. अप्र॑तिष्ठित आसी दासी॒ दप्र॑तिष्ठि॒तो ऽप्र॑तिष्ठित आसीत् । \newline
32. अप्र॑तिष्ठित॒ इत्यप्र॑ति - स्थि॒तः॒ । \newline
33. आ॒सी॒थ् स स आ॑सी दासी॒थ् सः । \newline
34. स प्र॒जाप॑तिम् प्र॒जाप॑तिꣳ॒॒ स स प्र॒जाप॑तिम् । \newline
35. प्र॒जाप॑ति॒ मुपोप॑ प्र॒जाप॑तिम् प्र॒जाप॑ति॒ मुप॑ । \newline
36. प्र॒जाप॑ति॒मिति॑ प्र॒जा - प॒ति॒म् । \newline
37. उपा॑ धाव दधाव॒ दुपोपा॑ धावत् । \newline
38. अ॒धा॒व॒त् तस्मै॒ तस्मा॑ अधाव दधाव॒त् तस्मै᳚ । \newline
39. तस्मा॑ ए॒त मे॒तम् तस्मै॒ तस्मा॑ ए॒तम् । \newline
40. ए॒त मे॑कविꣳशतिरा॒त्र मे॑कविꣳशतिरा॒त्र मे॒त मे॒त मे॑कविꣳशतिरा॒त्रम् । \newline
41. ए॒क॒विꣳ॒॒श॒ति॒रा॒त्रम् प्र प्रैक॑विꣳशतिरा॒त्र मे॑कविꣳशतिरा॒त्रम् प्र । \newline
42. ए॒क॒विꣳ॒॒श॒ति॒रा॒त्रमित्ये॑कविꣳशति - रा॒त्रम् । \newline
43. प्रा य॑च्छ दयच्छ॒त् प्र प्रा य॑च्छत् । \newline
44. अ॒य॒च्छ॒त् तम् त म॑यच्छ दयच्छ॒त् तम् । \newline
45. त मा तम् त मा । \newline
46. आ ऽह॑र दहर॒दा ऽह॑रत् । \newline
47. अ॒ह॒र॒त् तेन॒ तेना॑ हर दहर॒त् तेन॑ । \newline
48. तेना॑ यजता यजत॒ तेन॒ तेना॑ यजत । \newline
49. अ॒य॒ज॒त॒ तत॒ स्ततो॑ ऽयजता यजत॒ ततः॑ । \newline
50. ततो॒ वै वै तत॒ स्ततो॒ वै । \newline
51. वै स स वै वै सः । \newline
52. स प्रति॒ प्रति॒ स स प्रति॑ । \newline
53. प्रत्य॑ तिष्ठ दतिष्ठ॒त् प्रति॒ प्रत्य॑ तिष्ठत् । \newline
54. अ॒ति॒ष्ठ॒द् ये ये॑ ऽतिष्ठ दतिष्ठ॒द् ये । \newline
55. ये ब॑हुया॒जिनो॑ बहुया॒जिनो॒ ये ये ब॑हुया॒जिनः॑ । \newline
56. ब॒हु॒या॒जिनो ऽप्र॑तिष्ठिता॒ अप्र॑तिष्ठिता बहुया॒जिनो॑ बहुया॒जिनो ऽप्र॑तिष्ठिताः । \newline
57. ब॒हु॒या॒जिन॒ इति॑ बहु - या॒जिनः॑ । \newline
58. अप्र॑तिष्ठिताः॒ स्युः स्यु रप्र॑तिष्ठिता॒ अप्र॑तिष्ठिताः॒ स्युः । \newline
59. अप्र॑तिष्ठिता॒ इत्यप्र॑ति - स्थि॒ताः॒ । \newline

\textbf{Ghana Paata } \newline

1. प्र॒त्य॒व॒रोहे॑यु॒ रुदुत् प्र॑त्यव॒रोहे॑युः प्रत्यव॒रोहे॑यु॒ रुद् वा॒ वोत् प्र॑त्यव॒रोहे॑युः प्रत्यव॒रोहे॑यु॒ रुद् वा᳚ । \newline
2. प्र॒त्य॒व॒रोहे॑यु॒रिति॑ प्रति - अ॒व॒रोहे॑युः । \newline
3. उद् वा॒ वोदुद् वा॒ माद्ये॑यु॒र् माद्ये॑यु॒र् वोदुद् वा॒ माद्ये॑युः । \newline
4. वा॒ माद्ये॑यु॒र् माद्ये॑युर् वा वा॒ माद्ये॑यु॒र् यज॑माना॒ यज॑माना॒ माद्ये॑युर् वा वा॒ माद्ये॑यु॒र् यज॑मानाः । \newline
5. माद्ये॑यु॒र् यज॑माना॒ यज॑माना॒ माद्ये॑यु॒र् माद्ये॑यु॒र् यज॑मानाः॒ प्र प्र यज॑माना॒ माद्ये॑यु॒र् माद्ये॑यु॒र् यज॑मानाः॒ प्र । \newline
6. यज॑मानाः॒ प्र प्र यज॑माना॒ यज॑मानाः॒ प्र वा॑ वा॒ प्र यज॑माना॒ यज॑मानाः॒ प्र वा᳚ । \newline
7. प्र वा॑ वा॒ प्र प्र वा॑ मीयेरन् मीयेरन्. वा॒ प्र प्र वा॑ मीयेरन्न् । \newline
8. वा॒ मी॒ये॒र॒न् मी॒ये॒र॒न्॒. वा॒ वा॒ मी॒ये॒र॒न्॒. यद् यन् मी॑येरन्. वा वा मीयेर॒न्॒. यत् । \newline
9. मी॒ये॒र॒न्॒. यद् यन् मी॑येरन् मीयेर॒न्॒. यत् प्र॑ती॒चीना॑नि प्रती॒चीना॑नि॒ यन् मी॑येरन् मीयेर॒न्॒. यत् प्र॑ती॒चीना॑नि । \newline
10. यत् प्र॑ती॒चीना॑नि प्रती॒चीना॑नि॒ यद् यत् प्र॑ती॒चीना॑नि पृ॒ष्ठानि॑ पृ॒ष्ठानि॑ प्रती॒चीना॑नि॒ यद् यत् प्र॑ती॒चीना॑नि पृ॒ष्ठानि॑ । \newline
11. प्र॒ती॒चीना॑नि पृ॒ष्ठानि॑ पृ॒ष्ठानि॑ प्रती॒चीना॑नि प्रती॒चीना॑नि पृ॒ष्ठानि॒ भव॑न्ति॒ भव॑न्ति पृ॒ष्ठानि॑ प्रती॒चीना॑नि प्रती॒चीना॑नि पृ॒ष्ठानि॒ भव॑न्ति । \newline
12. पृ॒ष्ठानि॒ भव॑न्ति॒ भव॑न्ति पृ॒ष्ठानि॑ पृ॒ष्ठानि॒ भव॑न्ती॒म मि॒मम् भव॑न्ति पृ॒ष्ठानि॑ पृ॒ष्ठानि॒ भव॑न्ती॒मम् । \newline
13. भव॑न्ती॒म मि॒मम् भव॑न्ति॒ भव॑न्ती॒म मे॒वैवेमम् भव॑न्ति॒ भव॑न्ती॒म मे॒व । \newline
14. इ॒म मे॒वैवेम मि॒म मे॒व तै स्तै रे॒वेम मि॒म मे॒व तैः । \newline
15. ए॒व तै स्तै रे॒वैव तैर् लो॒कम् ॅलो॒कम् तैरे॒वैव तैर् लो॒कम् । \newline
16. तैर् लो॒कम् ॅलो॒कम् तै स्तैर् लो॒कम् प्र॒त्यव॑रोहन्ति प्र॒त्यव॑रोहन्ति लो॒कम् तै स्तैर् लो॒कम् प्र॒त्यव॑रोहन्ति । \newline
17. लो॒कम् प्र॒त्यव॑रोहन्ति प्र॒त्यव॑रोहन्ति लो॒कम् ॅलो॒कम् प्र॒त्यव॑रोह॒ न्त्यथो॒ अथो᳚ प्र॒त्यव॑रोहन्ति लो॒कम् ॅलो॒कम् प्र॒त्यव॑रोह॒ न्त्यथो᳚ । \newline
18. प्र॒त्यव॑रोह॒ न्त्यथो॒ अथो᳚ प्र॒त्यव॑रोहन्ति प्र॒त्यव॑रोह॒ न्त्यथो॑ अ॒स्मिन् न॒स्मिन् नथो᳚ प्र॒त्यव॑रोहन्ति प्र॒त्यव॑रोह॒ न्त्यथो॑ अ॒स्मिन्न् । \newline
19. प्र॒त्यव॑रोह॒न्तीति॑ प्रति - अव॑रोहन्ति । \newline
20. अथो॑ अ॒स्मिन् न॒स्मिन् नथो॒ अथो॑ अ॒स्मिन् ने॒वै वास्मिन् नथो॒ अथो॑ अ॒स्मिन् ने॒व । \newline
21. अथो॒ इत्यथो᳚ । \newline
22. अ॒स्मिन् ने॒वै वास्मिन् न॒स्मिन् ने॒व लो॒के लो॒क ए॒वास्मिन् न॒स्मिन् ने॒व लो॒के । \newline
23. ए॒व लो॒के लो॒क ए॒वैव लो॒के प्रति॒ प्रति॑ लो॒क ए॒वैव लो॒के प्रति॑ । \newline
24. लो॒के प्रति॒ प्रति॑ लो॒के लो॒के प्रति॑ तिष्ठन्ति तिष्ठन्ति॒ प्रति॑ लो॒के लो॒के प्रति॑ तिष्ठन्ति । \newline
25. प्रति॑ तिष्ठन्ति तिष्ठन्ति॒ प्रति॒ प्रति॑ तिष्ठ॒ न्त्यनु॑न्मादा॒या नु॑न्मादाय तिष्ठन्ति॒ प्रति॒ प्रति॑ तिष्ठ॒ न्त्यनु॑न्मादाय । \newline
26. ति॒ष्ठ॒ न्त्यनु॑न्मादा॒या नु॑न्मादाय तिष्ठन्ति तिष्ठ॒ न्त्यनु॑न्मादा॒ येन्द्र॒ इन्द्रो ऽनु॑न्मादाय तिष्ठन्ति तिष्ठ॒ न्त्यनु॑न्मादा॒येन्द्रः॑ । \newline
27. अनु॑न्मादा॒येन्द्र॒ इन्द्रो ऽनु॑न्मादा॒या नु॑न्मादा॒ येन्द्रो॒ वै वा इन्द्रो ऽनु॑न्मादा॒या नु॑न्मादा॒ येन्द्रो॒ वै । \newline
28. अनु॑न्मादा॒येत्यनु॑त् - मा॒दा॒य॒ । \newline
29. इन्द्रो॒ वै वा इन्द्र॒ इन्द्रो॒ वा अप्र॑तिष्ठि॒तो ऽप्र॑तिष्ठितो॒ वा इन्द्र॒ इन्द्रो॒ वा अप्र॑तिष्ठितः । \newline
30. वा अप्र॑तिष्ठि॒तो ऽप्र॑तिष्ठितो॒ वै वा अप्र॑तिष्ठित आसी दासी॒ दप्र॑तिष्ठितो॒ वै वा अप्र॑तिष्ठित आसीत् । \newline
31. अप्र॑तिष्ठित आसी दासी॒ दप्र॑तिष्ठि॒तो ऽप्र॑तिष्ठित आसी॒थ् स स आ॑सी॒ दप्र॑तिष्ठि॒तो ऽप्र॑तिष्ठित आसी॒थ् सः । \newline
32. अप्र॑तिष्ठित॒ इत्यप्र॑ति - स्थि॒तः॒ । \newline
33. आ॒सी॒थ् स स आ॑सी दासी॒थ् स प्र॒जाप॑तिम् प्र॒जाप॑तिꣳ॒॒ स आ॑सी दासी॒थ् स प्र॒जाप॑तिम् । \newline
34. स प्र॒जाप॑तिम् प्र॒जाप॑तिꣳ॒॒ स स प्र॒जाप॑ति॒ मुपोप॑ प्र॒जाप॑तिꣳ॒॒ स स प्र॒जाप॑ति॒ मुप॑ । \newline
35. प्र॒जाप॑ति॒ मुपोप॑ प्र॒जाप॑तिम् प्र॒जाप॑ति॒ मुपा॑ धाव दधाव॒ दुप॑ प्र॒जाप॑तिम् प्र॒जाप॑ति॒ मुपा॑ धावत् । \newline
36. प्र॒जाप॑ति॒मिति॑ प्र॒जा - प॒ति॒म् । \newline
37. उपा॑धाव दधाव॒ दुपोपा॑ धाव॒त् तस्मै॒ तस्मा॑ अधाव॒ दुपोपा॑ धाव॒त् तस्मै᳚ । \newline
38. अ॒धा॒व॒त् तस्मै॒ तस्मा॑ अधाव दधाव॒त् तस्मा॑ ए॒त मे॒तम् तस्मा॑ अधाव दधाव॒त् तस्मा॑ ए॒तम् । \newline
39. तस्मा॑ ए॒त मे॒तम् तस्मै॒ तस्मा॑ ए॒त मे॑कविꣳशतिरा॒त्र मे॑कविꣳशतिरा॒त्र मे॒तम् तस्मै॒ तस्मा॑ ए॒त मे॑कविꣳशतिरा॒त्रम् । \newline
40. ए॒त मे॑कविꣳशतिरा॒त्र मे॑कविꣳशतिरा॒त्र मे॒त मे॒त मे॑कविꣳशतिरा॒त्रम् प्र प्रैक॑विꣳशतिरा॒त्र मे॒त मे॒त मे॑कविꣳशतिरा॒त्रम् प्र । \newline
41. ए॒क॒विꣳ॒॒श॒ति॒रा॒त्रम् प्र प्रैक॑विꣳशतिरा॒त्र मे॑कविꣳशतिरा॒त्रम् प्राय॑च्छ दयच्छ॒त् प्रैक॑विꣳशतिरा॒त्र मे॑कविꣳशतिरा॒त्रम् प्राय॑च्छत् । \newline
42. ए॒क॒विꣳ॒॒श॒ति॒रा॒त्रमित्ये॑कविꣳशति - रा॒त्रम् । \newline
43. प्राय॑च्छ दयच्छ॒त् प्र प्राय॑च्छ॒त् तम् त म॑यच्छ॒त् प्र प्राय॑च्छ॒त् तम् । \newline
44. अ॒य॒च्छ॒त् तम् त म॑यच्छ दयच्छ॒त् त मा त म॑यच्छ दयच्छ॒त् त मा । \newline
45. त मा तम् त मा ऽह॑र दहर॒दा तम् त मा ऽह॑रत् । \newline
46. आ ऽह॑र दहर॒दा ऽह॑र॒त् तेन॒ तेना॑ हर॒दा ऽह॑र॒त् तेन॑ । \newline
47. अ॒ह॒र॒त् तेन॒ तेना॑ हर दहर॒त् तेना॑ यजता यजत॒ तेना॑ हर दहर॒त् तेना॑ यजत । \newline
48. तेना॑ यजता यजत॒ तेन॒ तेना॑ यजत॒ तत॒ स्ततो॑ ऽयजत॒ तेन॒ तेना॑ यजत॒ ततः॑ । \newline
49. अ॒य॒ज॒त॒ तत॒ स्ततो॑ ऽयजता यजत॒ ततो॒ वै वै ततो॑ ऽयजता यजत॒ ततो॒ वै । \newline
50. ततो॒ वै वै तत॒ स्ततो॒ वै स स वै तत॒ स्ततो॒ वै सः । \newline
51. वै स स वै वै स प्रति॒ प्रति॒ स वै वै स प्रति॑ । \newline
52. स प्रति॒ प्रति॒ स स प्रत्य॑तिष्ठ दतिष्ठ॒त् प्रति॒ स स प्रत्य॑तिष्ठत् । \newline
53. प्रत्य॑तिष्ठ दतिष्ठ॒त् प्रति॒ प्रत्य॑तिष्ठ॒द् ये ये॑ ऽतिष्ठ॒त् प्रति॒ प्रत्य॑तिष्ठ॒द् ये । \newline
54. अ॒ति॒ष्ठ॒द् ये ये॑ ऽतिष्ठ दतिष्ठ॒द् ये ब॑हुया॒जिनो॑ बहुया॒जिनो॒ ये॑ ऽतिष्ठ दतिष्ठ॒द् ये ब॑हुया॒जिनः॑ । \newline
55. ये ब॑हुया॒जिनो॑ बहुया॒जिनो॒ ये ये ब॑हुया॒जिनो ऽप्र॑तिष्ठिता॒ अप्र॑तिष्ठिता बहुया॒जिनो॒ ये ये 
ब॑हुया॒जिनो ऽप्र॑तिष्ठिताः । \newline
56. ब॒हु॒या॒जिनो ऽप्र॑तिष्ठिता॒ अप्र॑तिष्ठिता बहुया॒जिनो॑ बहुया॒जिनो ऽप्र॑तिष्ठिताः॒ स्युः स्यु रप्र॑तिष्ठिता बहुया॒जिनो॑ बहुया॒जिनो ऽप्र॑तिष्ठिताः॒ स्युः । \newline
57. ब॒हु॒या॒जिन॒ इति॑ बहु - या॒जिनः॑ । \newline
58. अप्र॑तिष्ठिताः॒ स्युः स्यु रप्र॑तिष्ठिता॒ अप्र॑तिष्ठिताः॒ स्यु स्ते ते स्यु रप्र॑तिष्ठिता॒ अप्र॑तिष्ठिताः॒ स्यु स्ते । \newline
59. अप्र॑तिष्ठिता॒ इत्यप्र॑ति - स्थि॒ताः॒ । \newline
\pagebreak
\markright{ TS 7.3.10.5  \hfill https://www.vedavms.in \hfill}

\section{ TS 7.3.10.5 }

\textbf{TS 7.3.10.5 } \newline
\textbf{Samhita Paata} \newline

स्युस्त ए॑कविꣳशति-रा॒त्रमा॑सीर॒न् द्वाद॑श॒ मासाः॒ पञ्च॒र्तव॒स्त्रय॑ इ॒मे लो॒का अ॒सावा॑दि॒त्य ए॑कविꣳ॒॒श ए॒ताव॑न्तो॒ वै दे॑वलो॒कास्तेष्वे॒व य॑था पू॒र्वं प्रति॑ तिष्ठन्त्य॒सावा॑दि॒त्यो न व्य॑रोचत॒ स प्र॒जाप॑ति॒मुपा॑धाव॒त् तस्मा॑ ए॒तमे॑कविꣳशतिरा॒त्रं प्राय॑च्छ॒त् तमाऽह॑र॒त् तेना॑यजत॒ ततो॒ वै सो॑ ऽरोचत॒ य ए॒वं वि॒द्वाꣳस॑ एकविꣳशतिरा॒त्रमास॑ते॒ ( ) रोच॑न्त ए॒वैक॑विꣳशतिरा॒त्रो भ॑वति॒ रुग्वा ए॑कविꣳ॒॒शो रुच॑मे॒व ग॑च्छ॒न्त्यथो᳚ प्रति॒ष्ठामे॒व प्र॑ति॒ष्ठा ह्ये॑कविꣳ॒॒शो॑ ऽतिरा॒त्राव॒भितो॑ भवतो ब्रह्मवर्च॒सस्य॒ परि॑गृहीत्यै ॥ \newline

\textbf{Pada Paata} \newline

स्युः । ते । ए॒क॒विꣳ॒॒श॒ति॒रा॒त्रमित्ये॑कविꣳशति - रा॒त्रम् । आ॒सी॒र॒न्न् । द्वाद॑श । मासाः᳚ । पञ्च॑ । ऋ॒तवः॑ । त्रयः॑ । इ॒मे । लो॒काः । अ॒सौ । आ॒दि॒त्यः । ए॒क॒विꣳ॒॒श इत्ये॑क - विꣳ॒॒शः । ए॒ताव॑न्तः । वै । दे॒व॒लो॒का इति॑ देव - लो॒काः । तेषु॑ । ए॒व । य॒था॒पू॒र्वमिति॑ यथा - पू॒र्वम् । प्रतीति॑ । ति॒ष्ठ॒न्ति॒ । अ॒सौ । आ॒दि॒त्यः । न । वीति॑ । अ॒रो॒च॒त॒ । सः । प्र॒जाप॑ति॒मिति॑ प्र॒जा - प॒ति॒म् । उपेति॑ । अ॒धा॒व॒त् । तस्मै᳚ । ए॒तम् । ए॒क॒विꣳ॒॒श॒ति॒रा॒त्रमित्ये॑कविꣳशति - रा॒त्रम् । प्रेति॑ । अ॒य॒च्छ॒त् । तम् । एति॑ । अ॒ह॒र॒त् । तेन॑ । अ॒य॒ज॒त॒ । ततः॑ । वै । सः । अ॒रो॒च॒त॒ । ये । ए॒वम् । वि॒द्वाꣳसः॑ । ए॒क॒विꣳ॒॒श॒ति॒रा॒त्रमित्ये॑कविꣳशति - रा॒त्रम् । आस॑ते ( ) । रोच॑न्ते । ए॒व । ए॒क॒विꣳ॒॒श॒ति॒रा॒त्र इत्येक॑विꣳशति - रा॒त्रः । भ॒व॒ति॒ । रुक् । वै । ए॒क॒विꣳ॒॒श इत्ये॑क - विꣳ॒॒शः । रुच᳚म् । ए॒व । ग॒च्छ॒न्ति॒ । अथो॒ इति॑ । प्र॒ति॒ष्ठामिति॑ प्रति - स्थाम् । ए॒व । प्र॒ति॒ष्ठेति॑ प्रति - स्था । हि । ए॒क॒विꣳ॒॒श इत्ये॑क - विꣳ॒॒शः । अ॒ति॒रा॒त्रावित्य॑ति - रा॒त्रौ । अ॒भितः॑ । भ॒व॒तः॒ । ब्र॒ह्म॒व॒र्च॒सस्येति॑ ब्रह्म - व॒र्च॒सस्य॑ । परि॑गृहीत्या॒ इति॒ परि॑ - गृ॒ही॒त्यै॒ ॥  \newline


\textbf{Krama Paata} \newline

स्युस्ते । त ए॑कविꣳशतिरा॒त्रम् । ए॒क॒विꣳ॒॒श॒ति॒रा॒त्रमा॑सीरन्न् । ए॒क॒विꣳ॒॒श॒ति॒रा॒त्रमित्ये॑कविꣳशति - रा॒त्रम् । आ॒सी॒र॒न् द्वाद॑श । द्वाद॑श॒मासाः᳚ । मासाः॒ पञ्च॑ । पञ्च॒र्तवः॑ । ऋ॒तव॒स्त्रयः॑ । त्रय॑ इ॒मे । इ॒मे लो॒काः । लो॒का अ॒सौ । अ॒सावा॑दि॒त्यः । आ॒दि॒त्य ए॑कविꣳ॒॒शः । ए॒क॒विꣳ॒॒श ए॒ताव॑न्तः । ए॒क॒विꣳ॒॒श इत्ये॑क - विꣳ॒॒शः । ए॒ताव॑न्तो॒ वै । वै दे॑वलो॒काः । दे॒व॒लो॒कास्तेषु॑ । दे॒व॒लो॒का इति॑ देव - लो॒काः । तेष्वे॒व । ए॒व य॑थापू॒र्वम् । य॒था॒पू॒र्वम् प्रति॑ । य॒था॒पू॒र्वमिति॑ यथा - पू॒र्वम् । प्रति॑ तिष्ठन्ति । ति॒ष्ठ॒न्त्य॒सौ । अ॒सावा॑दि॒त्यः । आ॒दि॒त्यो न । न वि । व्य॑रोचत । अ॒रो॒च॒त॒ सः । स प्र॒जाप॑तिम् । प्र॒जाप॑ति॒मुप॑ । प्र॒जाप॑ति॒मिति॑ प्र॒जा - प॒ति॒म् । उपा॑धावत् । अ॒धा॒व॒त् तस्मै᳚ । तस्मा॑ ए॒तम् । ए॒तमे॑कविꣳशतिरा॒त्रम् । ए॒क॒विꣳ॒॒श॒ति॒रा॒त्रम् प्र । ए॒क॒विꣳ॒॒श॒ति॒रा॒त्रमित्ये॑कविꣳशति - रा॒त्रम् । प्राय॑च्छत् । अ॒य॒च्छ॒त् तम् । तमा । आऽह॑रत् । अ॒ह॒र॒त् तेन॑ । तेना॑यजत । अ॒य॒ज॒त॒ ततः॑ । ततो॒ वै । वै सः । सो॑ऽरोचत । अ॒रो॒च॒त॒ ये । य ए॒वम् । ए॒वम् ॅवि॒द्वाꣳसः॑ । वि॒द्वाꣳस॑ एकविꣳशतिरा॒त्रम् । ए॒क॒विꣳ॒॒श॒ति॒रा॒त्रमास॑ते ( ) । ए॒क॒विꣳ॒॒श॒ति॒रा॒त्रमित्ये॑कविꣳशति - रा॒त्रम् । आस॑ते॒ रोच॑न्ते । रोच॑न्त ए॒व । ए॒वैक॑विꣳशतिरा॒त्रः । ए॒क॒विꣳ॒॒श॒ति॒रा॒त्रो भ॑वति । ए॒क॒विꣳ॒॒श॒रा॒त्र इत्ये॑क विꣳशति - रा॒त्रः । भ॒व॒ति॒ रुक् । रुग् वै । वा ए॑कविꣳ॒॒शः । ए॒क॒विꣳ॒॒शो रुच᳚म् । ए॒क॒विꣳ॒॒श इत्ये॑क - विꣳ॒॒शः । रुच॑मे॒व । ए॒व ग॑च्छन्ति । ग॒च्छ॒न्त्यथो᳚ । अथो᳚ प्रति॒ष्ठाम् । अथो॒ इत्यथो᳚ । प्र॒ति॒ष्ठामे॒व । प्र॒ति॒ष्ठामिति॑ प्रति - स्थाम् । ए॒व प्र॑ति॒ष्ठा । प्र॒ति॒ष्ठा हि । प्र॒ति॒ष्ठेति॑ प्रति - स्था । ह्ये॑कविꣳ॒॒शः । ए॒क॒विꣳ॒॒शो॑ऽतिरा॒त्रौ । ए॒क॒विꣳ॒॒श इत्ये॑क - विꣳ॒॒शः । अ॒ति॒रा॒त्राव॒भितः॑ । अ॒ति॒रा॒त्रावित्य॑ति - रा॒त्रौ । अ॒भितो॑ भवतः । भ॒व॒तो॒ ब्र॒ह्म॒व॒र्च॒सस्य॑ । ब्र॒ह्म॒व॒र्च॒सस्य॒ परि॑गृहीत्यै । ब्र॒ह्म॒व॒र्च॒सस्येति॑ ब्रह्म - व॒र्च॒सस्य॑ । परि॑गृहीत्या॒ इति॒ परि॑ - गृ॒ही॒त्यै॒ । \newline

\textbf{Jatai Paata} \newline

1. स्यु स्ते ते स्युः स्यु स्ते । \newline
2. त ए॑कविꣳशतिरा॒त्र मे॑कविꣳशतिरा॒त्रम् ते त ए॑कविꣳशतिरा॒त्रम् । \newline
3. ए॒क॒विꣳ॒॒श॒ति॒रा॒त्र मा॑सीरन् नासीरन् नेकविꣳशतिरा॒त्र मे॑कविꣳशतिरा॒त्र मा॑सीरन्न् । \newline
4. ए॒क॒विꣳ॒॒श॒ति॒रा॒त्रमित्ये॑कविꣳशति - रा॒त्रम् । \newline
5. आ॒सी॒र॒न् द्वाद॑श॒ द्वाद॑शासीरन् नासीर॒न् द्वाद॑श । \newline
6. द्वाद॑श॒ मासा॒ मासा॒ द्वाद॑श॒ द्वाद॑श॒ मासाः᳚ । \newline
7. मासाः॒ पञ्च॒ पञ्च॒ मासा॒ मासाः॒ पञ्च॑ । \newline
8. पञ्च॒ र्‌तव॑ ऋ॒तवः॒ पञ्च॒ पञ्च॒ र्‌तवः॑ । \newline
9. ऋ॒तव॒ स्त्रय॒ स्त्रय॑ ऋ॒तव॑ ऋ॒तव॒ स्त्रयः॑ । \newline
10. त्रय॑ इ॒म इ॒मे त्रय॒ स्त्रय॑ इ॒मे । \newline
11. इ॒मे लो॒का लो॒का इ॒म इ॒मे लो॒काः । \newline
12. लो॒का अ॒सा व॒सौ लो॒का लो॒का अ॒सौ । \newline
13. अ॒सा वा॑दि॒त्य आ॑दि॒त्यो॑ ऽसा व॒सा वा॑दि॒त्यः । \newline
14. आ॒दि॒त्य ए॑कविꣳ॒॒श ए॑कविꣳ॒॒श आ॑दि॒त्य आ॑दि॒त्य ए॑कविꣳ॒॒शः । \newline
15. ए॒क॒विꣳ॒॒श ए॒ताव॑न्त ए॒ताव॑न्त एकविꣳ॒॒श ए॑कविꣳ॒॒श ए॒ताव॑न्तः । \newline
16. ए॒क॒विꣳ॒॒श इत्ये॑क - विꣳ॒॒शः । \newline
17. ए॒ताव॑न्तो॒ वै वा ए॒ताव॑न्त ए॒ताव॑न्तो॒ वै । \newline
18. वै दे॑वलो॒का दे॑वलो॒का वै वै दे॑वलो॒काः । \newline
19. दे॒व॒लो॒का स्तेषु॒ तेषु॑ देवलो॒का दे॑वलो॒का स्तेषु॑ । \newline
20. दे॒व॒लो॒का इति॑ देव - लो॒काः । \newline
21. तेष्वे॒वैव तेषु॒ तेष्वे॒व । \newline
22. ए॒व य॑थापू॒र्वं ॅय॑थापू॒र्व मे॒वैव य॑थापू॒र्वम् । \newline
23. य॒था॒पू॒र्वम् प्रति॒ प्रति॑ यथापू॒र्वं ॅय॑थापू॒र्वम् प्रति॑ । \newline
24. य॒था॒पू॒र्वमिति॑ यथा - पू॒र्वम् । \newline
25. प्रति॑ तिष्ठन्ति तिष्ठन्ति॒ प्रति॒ प्रति॑ तिष्ठन्ति । \newline
26. ति॒ष्ठ॒ न्त्य॒सा व॒सौ ति॑ष्ठन्ति तिष्ठ न्त्य॒सौ । \newline
27. अ॒सा वा॑दि॒त्य आ॑दि॒त्यो॑ ऽसा व॒सा वा॑दि॒त्यः । \newline
28. आ॒दि॒त्यो न नादि॒त्य आ॑दि॒त्यो न । \newline
29. न वि वि न न वि । \newline
30. व्य॑रोचता रोचत॒ वि व्य॑रोचत । \newline
31. अ॒रो॒च॒त॒ स सो॑ ऽरोचता रोचत॒ सः । \newline
32. स प्र॒जाप॑तिम् प्र॒जाप॑तिꣳ॒॒ स स प्र॒जाप॑तिम् । \newline
33. प्र॒जाप॑ति॒ मुपोप॑ प्र॒जाप॑तिम् प्र॒जाप॑ति॒ मुप॑ । \newline
34. प्र॒जाप॑ति॒मिति॑ प्र॒जा - प॒ति॒म् । \newline
35. उपा॑ धाव दधाव॒ दुपोपा॑ धावत् । \newline
36. अ॒धा॒व॒त् तस्मै॒ तस्मा॑ अधाव दधाव॒त् तस्मै᳚ । \newline
37. तस्मा॑ ए॒त मे॒तम् तस्मै॒ तस्मा॑ ए॒तम् । \newline
38. ए॒त मे॑कविꣳशतिरा॒त्र मे॑कविꣳशतिरा॒त्र मे॒त मे॒त मे॑कविꣳशतिरा॒त्रम् । \newline
39. ए॒क॒विꣳ॒॒श॒ति॒रा॒त्रम् प्र प्रैक॑विꣳशतिरा॒त्र मे॑कविꣳशतिरा॒त्रम् प्र । \newline
40. ए॒क॒विꣳ॒॒श॒ति॒रा॒त्रमित्ये॑कविꣳशति - रा॒त्रम् । \newline
41. प्रा य॑च्छ दयच्छ॒त् प्र प्राय॑च्छत् । \newline
42. अ॒य॒च्छ॒त् तम् त म॑यच्छ दयच्छ॒त् तम् । \newline
43. त मा तम् त मा । \newline
44. आ ऽह॑र दहर॒दा ऽह॑रत् । \newline
45. अ॒ह॒र॒त् तेन॒ तेना॑ हर दहर॒त् तेन॑ । \newline
46. तेना॑ यजता यजत॒ तेन॒ तेना॑ यजत । \newline
47. अ॒य॒ज॒त॒ तत॒ स्ततो॑ ऽयजता यजत॒ ततः॑ । \newline
48. ततो॒ वै वै तत॒ स्ततो॒ वै । \newline
49. वै स स वै वै सः । \newline
50. सो॑ ऽरोचता रोचत॒ स सो॑ ऽरोचत । \newline
51. अ॒रो॒च॒त॒ ये ये॑ ऽरोचता रोचत॒ ये । \newline
52. य ए॒व मे॒वं ॅये य ए॒वम् । \newline
53. ए॒वं ॅवि॒द्वाꣳसो॑ वि॒द्वाꣳस॑ ए॒व मे॒वं ॅवि॒द्वाꣳसः॑ । \newline
54. वि॒द्वाꣳस॑ एकविꣳशतिरा॒त्र मे॑कविꣳशतिरा॒त्रं ॅवि॒द्वाꣳसो॑ वि॒द्वाꣳस॑ एकविꣳशतिरा॒त्रम् । \newline
55. ए॒क॒विꣳ॒॒श॒ति॒रा॒त्र मास॑त॒ आस॑त एकविꣳशतिरा॒त्र मे॑कविꣳशतिरा॒त्र मास॑ते । \newline
56. ए॒क॒विꣳ॒॒श॒ति॒रा॒त्रमित्ये॑कविꣳशति - रा॒त्रम् । \newline
57. आस॑ते॒ रोच॑न्ते॒ रोच॑न्त॒ आस॑त॒ आस॑ते॒ रोच॑न्ते । \newline
58. रोच॑न्त ए॒वैव रोच॑न्ते॒ रोच॑न्त ए॒व । \newline
59. ए॒वैक॑विꣳशतिरा॒त्र ए॑कविꣳशतिरा॒त्र ए॒वैवैक॑विꣳशतिरा॒त्रः । \newline
60. ए॒क॒विꣳ॒॒श॒ति॒रा॒त्रो भ॑वति भव त्येकविꣳशतिरा॒त्र ए॑कविꣳशतिरा॒त्रो भ॑वति । \newline
61. ए॒क॒विꣳ॒॒श॒ति॒रा॒त्र इत्येक॑विꣳशति - रा॒त्रः । \newline
62. भ॒व॒ति॒ रुग् रुग् भ॑वति भवति॒ रुक् । \newline
63. रुग् वै वै रुग् रुग् वै । \newline
64. वा ए॑कविꣳ॒॒श ए॑कविꣳ॒॒शो वै वा ए॑कविꣳ॒॒शः । \newline
65. ए॒क॒विꣳ॒॒शो रुचꣳ॒॒ रुच॑ मेकविꣳ॒॒श ए॑कविꣳ॒॒शो रुच᳚म् । \newline
66. ए॒क॒विꣳ॒॒श इत्ये॑क - विꣳ॒॒शः । \newline
67. रुच॑ मे॒वैव रुचꣳ॒॒ रुच॑ मे॒व । \newline
68. ए॒व ग॑च्छन्ति गच्छ न्त्ये॒वैव ग॑च्छन्ति । \newline
69. ग॒च्छ॒ न्त्यथो॒ अथो॑ गच्छन्ति गच्छ॒ न्त्यथो᳚ । \newline
70. अथो᳚ प्रति॒ष्ठाम् प्र॑ति॒ष्ठा मथो॒ अथो᳚ प्रति॒ष्ठाम् । \newline
71. अथो॒ इत्यथो᳚ । \newline
72. प्र॒ति॒ष्ठा मे॒वैव प्र॑ति॒ष्ठाम् प्र॑ति॒ष्ठा मे॒व । \newline
73. प्र॒ति॒ष्ठामिति॑ प्रति - स्थाम् । \newline
74. ए॒व प्र॑ति॒ष्ठा प्र॑ति॒ष्ठैवैव प्र॑ति॒ष्ठा । \newline
75. प्र॒ति॒ष्ठा हि हि प्र॑ति॒ष्ठा प्र॑ति॒ष्ठा हि । \newline
76. प्र॒ति॒ष्ठेति॑ प्रति - स्था । \newline
77. ह्ये॑कविꣳ॒॒श ए॑कविꣳ॒॒शो हि ह्ये॑कविꣳ॒॒शः । \newline
78. ए॒क॒विꣳ॒॒शो॑ ऽतिरा॒त्रा व॑तिरा॒त्रा वे॑कविꣳ॒॒श ए॑कविꣳ॒॒शो॑ ऽतिरा॒त्रौ । \newline
79. ए॒क॒विꣳ॒॒श इत्ये॑क - विꣳ॒॒शः । \newline
80. अ॒ति॒रा॒त्रा व॒भितो॒ ऽभितो॑ ऽतिरा॒त्रा व॑तिरा॒त्रा व॒भितः॑ । \newline
81. अ॒ति॒रा॒त्रावित्य॑ति - रा॒त्रौ । \newline
82. अ॒भितो॑ भवतो भवतो॒ ऽभितो॒ ऽभितो॑ भवतः । \newline
83. भ॒व॒तो॒ ब्र॒ह्म॒व॒र्च॒सस्य॑ ब्रह्मवर्च॒सस्य॑ भवतो भवतो ब्रह्मवर्च॒सस्य॑ । \newline
84. ब्र॒ह्म॒व॒र्च॒सस्य॒ परि॑गृहीत्यै॒ परि॑गृहीत्यै ब्रह्मवर्च॒सस्य॑ ब्रह्मवर्च॒सस्य॒ परि॑गृहीत्यै । \newline
85. ब्र॒ह्म॒व॒र्च॒सस्येति॑ ब्रह्म - व॒र्च॒सस्य॑ । \newline
86. परि॑गृहीत्या॒ इति॒ परि॑ - गृ॒ही॒त्यै॒ । \newline

\textbf{Ghana Paata } \newline

1. स्यु स्ते ते स्युः स्यु स्त ए॑कविꣳशतिरा॒त्र मे॑कविꣳशतिरा॒त्रम् ते स्युः स्यु स्त ए॑कविꣳशतिरा॒त्रम् । \newline
2. त ए॑कविꣳशतिरा॒त्र मे॑कविꣳशतिरा॒त्रम् ते त ए॑कविꣳशतिरा॒त्र मा॑सीरन् नासीरन् नेकविꣳशतिरा॒त्रम् ते त ए॑कविꣳशतिरा॒त्र मा॑सीरन्न् । \newline
3. ए॒क॒विꣳ॒॒श॒ति॒रा॒त्र मा॑सीरन् नासीरन् नेकविꣳशतिरा॒त्र मे॑कविꣳशतिरा॒त्र मा॑सीर॒न् द्वाद॑श॒ द्वाद॑शासीरन् नेकविꣳशतिरा॒त्र मे॑कविꣳशतिरा॒त्र मा॑सीर॒न् द्वाद॑श । \newline
4. ए॒क॒विꣳ॒॒श॒ति॒रा॒त्रमित्ये॑कविꣳशति - रा॒त्रम् । \newline
5. आ॒सी॒र॒न् द्वाद॑श॒ द्वाद॑शा सीरन् नासीर॒न् द्वाद॑श॒ मासा॒ मासा॒ द्वाद॑शा सीरन् नासीर॒न् द्वाद॑श॒ मासाः᳚ । \newline
6. द्वाद॑श॒ मासा॒ मासा॒ द्वाद॑श॒ द्वाद॑श॒ मासाः॒ पञ्च॒ पञ्च॒ मासा॒ द्वाद॑श॒ द्वाद॑श॒ मासाः॒ पञ्च॑ । \newline
7. मासाः॒ पञ्च॒ पञ्च॒ मासा॒ मासाः॒ पञ्च॒ र्‌तव॑ ऋ॒तवः॒ पञ्च॒ मासा॒ मासाः॒ पञ्च॒ र्‌तवः॑ । \newline
8. पञ्च॒ र्‌तव॑ ऋ॒तवः॒ पञ्च॒ पञ्च॒ र्‌तव॒ स्त्रय॒ स्त्रय॑ ऋ॒तवः॒ पञ्च॒ पञ्च॒ र्‌तव॒ स्त्रयः॑ । \newline
9. ऋ॒तव॒ स्त्रय॒ स्त्रय॑ ऋ॒तव॑ ऋ॒तव॒ स्त्रय॑ इ॒म इ॒मे त्रय॑ ऋ॒तव॑ ऋ॒तव॒ स्त्रय॑ इ॒मे । \newline
10. त्रय॑ इ॒म इ॒मे त्रय॒ स्त्रय॑ इ॒मे लो॒का लो॒का इ॒मे त्रय॒ स्त्रय॑ इ॒मे लो॒काः । \newline
11. इ॒मे लो॒का लो॒का इ॒म इ॒मे लो॒का अ॒सा व॒सौ लो॒का इ॒म इ॒मे लो॒का अ॒सौ । \newline
12. लो॒का अ॒सा व॒सौ लो॒का लो॒का अ॒सा वा॑दि॒त्य आ॑दि॒त्यो॑ ऽसौ लो॒का लो॒का अ॒सा वा॑दि॒त्यः । \newline
13. अ॒सा वा॑दि॒त्य आ॑दि॒त्यो॑ ऽसा व॒सा वा॑दि॒त्य ए॑कविꣳ॒॒श ए॑कविꣳ॒॒श आ॑दि॒त्यो॑ ऽसा व॒सा वा॑दि॒त्य ए॑कविꣳ॒॒शः । \newline
14. आ॒दि॒त्य ए॑कविꣳ॒॒श ए॑कविꣳ॒॒श आ॑दि॒त्य आ॑दि॒त्य ए॑कविꣳ॒॒श ए॒ताव॑न्त ए॒ताव॑न्त एकविꣳ॒॒श आ॑दि॒त्य आ॑दि॒त्य ए॑कविꣳ॒॒श ए॒ताव॑न्तः । \newline
15. ए॒क॒विꣳ॒॒श ए॒ताव॑न्त ए॒ताव॑न्त एकविꣳ॒॒श ए॑कविꣳ॒॒श ए॒ताव॑न्तो॒ वै वा ए॒ताव॑न्त एकविꣳ॒॒श ए॑कविꣳ॒॒श ए॒ताव॑न्तो॒ वै । \newline
16. ए॒क॒विꣳ॒॒श इत्ये॑क - विꣳ॒॒शः । \newline
17. ए॒ताव॑न्तो॒ वै वा ए॒ताव॑न्त ए॒ताव॑न्तो॒ वै दे॑वलो॒का दे॑वलो॒का वा ए॒ताव॑न्त ए॒ताव॑न्तो॒ वै दे॑वलो॒काः । \newline
18. वै दे॑वलो॒का दे॑वलो॒का वै वै दे॑वलो॒का स्तेषु॒ तेषु॑ देवलो॒का वै वै दे॑वलो॒का स्तेषु॑ । \newline
19. दे॒व॒लो॒का स्तेषु॒ तेषु॑ देवलो॒का दे॑वलो॒का स्तेष्वे॒वैव तेषु॑ देवलो॒का दे॑वलो॒का स्तेष्वे॒व । \newline
20. दे॒व॒लो॒का इति॑ देव - लो॒काः । \newline
21. तेष्वे॒वैव तेषु॒ तेष्वे॒व य॑थापू॒र्वं ॅय॑थापू॒र्व मे॒व तेषु॒ तेष्वे॒व य॑थापू॒र्वम् । \newline
22. ए॒व य॑थापू॒र्वं ॅय॑थापू॒र्व मे॒वैव य॑थापू॒र्वम् प्रति॒ प्रति॑ यथापू॒र्व मे॒वैव य॑थापू॒र्वम् प्रति॑ । \newline
23. य॒था॒पू॒र्वम् प्रति॒ प्रति॑ यथापू॒र्वं ॅय॑थापू॒र्वम् प्रति॑ तिष्ठन्ति तिष्ठन्ति॒ प्रति॑ यथापू॒र्वं ॅय॑थापू॒र्वम् प्रति॑ तिष्ठन्ति । \newline
24. य॒था॒पू॒र्वमिति॑ यथा - पू॒र्वम् । \newline
25. प्रति॑ तिष्ठन्ति तिष्ठन्ति॒ प्रति॒ प्रति॑ तिष्ठ न्त्य॒सा व॒सौ ति॑ष्ठन्ति॒ प्रति॒ प्रति॑ तिष्ठ न्त्य॒सौ । \newline
26. ति॒ष्ठ॒ न्त्य॒सा व॒सौ ति॑ष्ठन्ति तिष्ठ न्त्य॒सा वा॑दि॒त्य आ॑दि॒त्यो॑ ऽसौ ति॑ष्ठन्ति तिष्ठ न्त्य॒सा वा॑दि॒त्यः । \newline
27. अ॒सा वा॑दि॒त्य आ॑दि॒त्यो॑ ऽसा व॒सा वा॑दि॒त्यो न नादि॒त्यो॑ ऽसा व॒सा वा॑दि॒त्यो न । \newline
28. आ॒दि॒त्यो न नादि॒त्य आ॑दि॒त्यो न वि वि नादि॒त्य आ॑दि॒त्यो न वि । \newline
29. न वि वि न न व्य॑रोचता रोचत॒ वि न न व्य॑रोचत । \newline
30. व्य॑रोचता रोचत॒ वि व्य॑रोचत॒ स सो॑ ऽरोचत॒ वि व्य॑रोचत॒ सः । \newline
31. अ॒रो॒च॒त॒ स सो॑ ऽरोचता रोचत॒ स प्र॒जाप॑तिम् प्र॒जाप॑तिꣳ॒॒ सो॑ ऽरोचता रोचत॒ स प्र॒जाप॑तिम् । \newline
32. स प्र॒जाप॑तिम् प्र॒जाप॑तिꣳ॒॒ स स प्र॒जाप॑ति॒ मुपोप॑ प्र॒जाप॑तिꣳ॒॒ स स प्र॒जाप॑ति॒ मुप॑ । \newline
33. प्र॒जाप॑ति॒ मुपोप॑ प्र॒जाप॑तिम् प्र॒जाप॑ति॒ मुपा॑धाव दधाव॒ दुप॑ प्र॒जाप॑तिम् प्र॒जाप॑ति॒ मुपा॑धावत् । \newline
34. प्र॒जाप॑ति॒मिति॑ प्र॒जा - प॒ति॒म् । \newline
35. उपा॑धाव दधाव॒ दुपोपा॑ धाव॒त् तस्मै॒ तस्मा॑ अधाव॒ दुपोपा॑ धाव॒त् तस्मै᳚ । \newline
36. अ॒धा॒व॒त् तस्मै॒ तस्मा॑ अधाव दधाव॒त् तस्मा॑ ए॒त मे॒तम् तस्मा॑ अधाव दधाव॒त् तस्मा॑ ए॒तम् । \newline
37. तस्मा॑ ए॒त मे॒तम् तस्मै॒ तस्मा॑ ए॒त मे॑कविꣳशतिरा॒त्र मे॑कविꣳशतिरा॒त्र मे॒तम् तस्मै॒ तस्मा॑ ए॒त मे॑कविꣳशतिरा॒त्रम् । \newline
38. ए॒त मे॑कविꣳशतिरा॒त्र मे॑कविꣳशतिरा॒त्र मे॒त मे॒त मे॑कविꣳशतिरा॒त्रम् प्र प्रैक॑विꣳशतिरा॒त्र मे॒त मे॒त मे॑कविꣳशतिरा॒त्रम् प्र । \newline
39. ए॒क॒विꣳ॒॒श॒ति॒रा॒त्रम् प्र प्रैक॑विꣳशतिरा॒त्र मे॑कविꣳशतिरा॒त्रम् प्रा य॑च्छ दयच्छ॒त् प्रैक॑विꣳशतिरा॒त्र मे॑कविꣳशतिरा॒त्रम् प्रा य॑च्छत् । \newline
40. ए॒क॒विꣳ॒॒श॒ति॒रा॒त्रमित्ये॑कविꣳशति - रा॒त्रम् । \newline
41. प्रा य॑च्छ दयच्छ॒त् प्र प्रा य॑च्छ॒त् तम् त म॑यच्छ॒त् प्र प्रा य॑च्छ॒त् तम् । \newline
42. अ॒य॒च्छ॒त् तम् त म॑यच्छ दयच्छ॒त् त मा त म॑यच्छ दयच्छ॒त् त मा । \newline
43. त मा तम् त मा ऽह॑र दहर॒दा तम् त मा ऽह॑रत् । \newline
44. आ ऽह॑र दहर॒दा ऽह॑र॒त् तेन॒ तेना॑ हर॒दा ऽह॑र॒त् तेन॑ । \newline
45. अ॒ह॒र॒त् तेन॒ तेना॑ हर दहर॒त् तेना॑ यजता यजत॒ तेना॑ हर दहर॒त् तेना॑ यजत । \newline
46. तेना॑ यजता यजत॒ तेन॒ तेना॑ यजत॒ तत॒ स्ततो॑ ऽयजत॒ तेन॒ तेना॑ यजत॒ ततः॑ । \newline
47. अ॒य॒ज॒त॒ तत॒ स्ततो॑ ऽयजता यजत॒ ततो॒ वै वै ततो॑ ऽयजता यजत॒ ततो॒ वै । \newline
48. ततो॒ वै वै तत॒ स्ततो॒ वै स स वै तत॒ स्ततो॒ वै सः । \newline
49. वै स स वै वै सो॑ ऽरोचता रोचत॒ स वै वै सो॑ ऽरोचत । \newline
50. सो॑ ऽरोचता रोचत॒ स सो॑ ऽरोचत॒ ये ये॑ ऽरोचत॒ स सो॑ ऽरोचत॒ ये । \newline
51. अ॒रो॒च॒त॒ ये ये॑ ऽरोचता रोचत॒ य ए॒व मे॒वं ॅये॑ ऽरोचता रोचत॒ य ए॒वम् । \newline
52. य ए॒व मे॒वं ॅये य ए॒वं ॅवि॒द्वाꣳसो॑ वि॒द्वाꣳस॑ ए॒वं ॅये य ए॒वं ॅवि॒द्वाꣳसः॑ । \newline
53. ए॒वं ॅवि॒द्वाꣳसो॑ वि॒द्वाꣳस॑ ए॒व मे॒वं ॅवि॒द्वाꣳस॑ एकविꣳशतिरा॒त्र मे॑कविꣳशतिरा॒त्रं ॅवि॒द्वाꣳस॑ ए॒व मे॒वं ॅवि॒द्वाꣳस॑ एकविꣳशतिरा॒त्रम् । \newline
54. वि॒द्वाꣳस॑ एकविꣳशतिरा॒त्र मे॑कविꣳशतिरा॒त्रं ॅवि॒द्वाꣳसो॑ वि॒द्वाꣳस॑ एकविꣳशतिरा॒त्र मास॑त॒ आस॑त एकविꣳशतिरा॒त्रं ॅवि॒द्वाꣳसो॑ वि॒द्वाꣳस॑ एकविꣳशतिरा॒त्र मास॑ते । \newline
55. ए॒क॒विꣳ॒॒श॒ति॒रा॒त्र मास॑त॒ आस॑त एकविꣳशतिरा॒त्र मे॑कविꣳशतिरा॒त्र मास॑ते॒ रोच॑न्ते॒ रोच॑न्त॒ आस॑त एकविꣳशतिरा॒त्र मे॑कविꣳशतिरा॒त्र मास॑ते॒ रोच॑न्ते । \newline
56. ए॒क॒विꣳ॒॒श॒ति॒रा॒त्रमित्ये॑कविꣳशति - रा॒त्रम् । \newline
57. आस॑ते॒ रोच॑न्ते॒ रोच॑न्त॒ आस॑त॒ आस॑ते॒ रोच॑न्त ए॒वैव रोच॑न्त॒ आस॑त॒ आस॑ते॒ रोच॑न्त ए॒व । \newline
58. रोच॑न्त ए॒वैव रोच॑न्ते॒ रोच॑न्त ए॒वैक॑विꣳशतिरा॒त्र ए॑कविꣳशतिरा॒त्र ए॒व रोच॑न्ते॒ रोच॑न्त ए॒वैक॑विꣳशतिरा॒त्रः । \newline
59. ए॒वैक॑विꣳशतिरा॒त्र ए॑कविꣳशतिरा॒त्र ए॒वैवैक॑विꣳशतिरा॒त्रो भ॑वति भव त्येकविꣳशतिरा॒त्र 
ए॒वैवैक॑विꣳशतिरा॒त्रो भ॑वति । \newline
60. ए॒क॒विꣳ॒॒श॒ति॒रा॒त्रो भ॑वति भव त्येकविꣳशतिरा॒त्र ए॑कविꣳशतिरा॒त्रो भ॑वति॒ रुग् रुग् भ॑व त्येकविꣳशतिरा॒त्र ए॑कविꣳशतिरा॒त्रो भ॑वति॒ रुक् । \newline
61. ए॒क॒विꣳ॒॒श॒ति॒रा॒त्र इत्येक॑विꣳशति - रा॒त्रः । \newline
62. भ॒व॒ति॒ रुग् रुग् भ॑वति भवति॒ रुग् वै वै रुग् भ॑वति भवति॒ रुग् वै । \newline
63. रुग् वै वै रुग् रुग् वा ए॑कविꣳ॒॒श ए॑कविꣳ॒॒शो वै रुग् रुग् वा ए॑कविꣳ॒॒शः । \newline
64. वा ए॑कविꣳ॒॒श ए॑कविꣳ॒॒शो वै वा ए॑कविꣳ॒॒शो रुचꣳ॒॒ रुच॑ मेकविꣳ॒॒शो वै वा ए॑कविꣳ॒॒शो रुच᳚म् । \newline
65. ए॒क॒विꣳ॒॒शो रुचꣳ॒॒ रुच॑ मेकविꣳ॒॒श ए॑कविꣳ॒॒शो रुच॑ मे॒वैव रुच॑ मेकविꣳ॒॒श ए॑कविꣳ॒॒शो रुच॑ मे॒व । \newline
66. ए॒क॒विꣳ॒॒श इत्ये॑क - विꣳ॒॒शः । \newline
67. रुच॑ मे॒वैव रुचꣳ॒॒ रुच॑ मे॒व ग॑च्छन्ति गच्छ न्त्ये॒व रुचꣳ॒॒ रुच॑ मे॒व ग॑च्छन्ति । \newline
68. ए॒व ग॑च्छन्ति गच्छ न्त्ये॒वैव ग॑च्छ॒ न्त्यथो॒ अथो॑ गच्छ न्त्ये॒वैव ग॑च्छ॒ न्त्यथो᳚ । \newline
69. ग॒च्छ॒ न्त्यथो॒ अथो॑ गच्छन्ति गच्छ॒ न्त्यथो᳚ प्रति॒ष्ठाम् प्र॑ति॒ष्ठा मथो॑ गच्छन्ति गच्छ॒ न्त्यथो᳚ प्रति॒ष्ठाम् । \newline
70. अथो᳚ प्रति॒ष्ठाम् प्र॑ति॒ष्ठा मथो॒ अथो᳚ प्रति॒ष्ठा मे॒वैव प्र॑ति॒ष्ठा मथो॒ अथो᳚ प्रति॒ष्ठा मे॒व । \newline
71. अथो॒ इत्यथो᳚ । \newline
72. प्र॒ति॒ष्ठा मे॒वैव प्र॑ति॒ष्ठाम् प्र॑ति॒ष्ठा मे॒व प्र॑ति॒ष्ठा प्र॑ति॒ष्ठैव प्र॑ति॒ष्ठाम् प्र॑ति॒ष्ठा मे॒व प्र॑ति॒ष्ठा । \newline
73. प्र॒ति॒ष्ठामिति॑ प्रति - स्थाम् । \newline
74. ए॒व प्र॑ति॒ष्ठा प्र॑ति॒ष्ठैवैव प्र॑ति॒ष्ठा हि हि प्र॑ति॒ष्ठैवैव प्र॑ति॒ष्ठा हि । \newline
75. प्र॒ति॒ष्ठा हि हि प्र॑ति॒ष्ठा प्र॑ति॒ष्ठा ह्ये॑कविꣳ॒॒श ए॑कविꣳ॒॒शो हि प्र॑ति॒ष्ठा प्र॑ति॒ष्ठा ह्ये॑कविꣳ॒॒शः । \newline
76. प्र॒ति॒ष्ठेति॑ प्रति - स्था । \newline
77. ह्ये॑कविꣳ॒॒श ए॑कविꣳ॒॒शो हि ह्ये॑कविꣳ॒॒शो॑ ऽतिरा॒त्रा व॑तिरा॒त्रा वे॑कविꣳ॒॒शो हि ह्ये॑कविꣳ॒॒शो॑ ऽतिरा॒त्रौ । \newline
78. ए॒क॒विꣳ॒॒शो॑ ऽतिरा॒त्रा व॑तिरा॒त्रा वे॑कविꣳ॒॒श ए॑कविꣳ॒॒शो॑ ऽतिरा॒त्रा व॒भितो॒ ऽभितो॑ ऽतिरा॒त्रा वे॑कविꣳ॒॒श ए॑कविꣳ॒॒शो॑ ऽतिरा॒त्रा व॒भितः॑ । \newline
79. ए॒क॒विꣳ॒॒श इत्ये॑क - विꣳ॒॒शः । \newline
80. अ॒ति॒रा॒त्रा व॒भितो॒ ऽभितो॑ ऽतिरा॒त्रा व॑तिरा॒त्रा व॒भितो॑ भवतो भवतो॒ ऽभितो॑ ऽतिरा॒त्रा व॑तिरा॒त्रा व॒भितो॑ भवतः । \newline
81. अ॒ति॒रा॒त्रावित्य॑ति - रा॒त्रौ । \newline
82. अ॒भितो॑ भवतो भवतो॒ ऽभितो॒ ऽभितो॑ भवतो ब्रह्मवर्च॒सस्य॑ ब्रह्मवर्च॒सस्य॑ भवतो॒ ऽभितो॒ ऽभितो॑ भवतो ब्रह्मवर्च॒सस्य॑ । \newline
83. भ॒व॒तो॒ ब्र॒ह्म॒व॒र्च॒सस्य॑ ब्रह्मवर्च॒सस्य॑ भवतो भवतो ब्रह्मवर्च॒सस्य॒ परि॑गृहीत्यै॒ परि॑गृहीत्यै ब्रह्मवर्च॒सस्य॑ भवतो भवतो ब्रह्मवर्च॒सस्य॒ परि॑गृहीत्यै । \newline
84. ब्र॒ह्म॒व॒र्च॒सस्य॒ परि॑गृहीत्यै॒ परि॑गृहीत्यै ब्रह्मवर्च॒सस्य॑ ब्रह्मवर्च॒सस्य॒ परि॑गृहीत्यै । \newline
85. ब्र॒ह्म॒व॒र्च॒सस्येति॑ ब्रह्म - व॒र्च॒सस्य॑ । \newline
86. परि॑गृहीत्या॒ इति॒ परि॑ - गृ॒ही॒त्यै॒ । \newline
\pagebreak
\markright{ TS 7.3.11.1  \hfill https://www.vedavms.in \hfill}

\section{ TS 7.3.11.1 }

\textbf{TS 7.3.11.1 } \newline
\textbf{Samhita Paata} \newline

अ॒र्वाङ् य॒ज्ञ्ः सं क्रा॑मत्व॒मुष्मा॒दधि॒ माम॒भि । ऋषी॑णां॒ ॅयः पु॒रोहि॑तः ॥ निर्दे॑वं॒ निर्वी॑रं कृ॒त्वा विष्क॑न्धं॒ तस्मि॑न् हीयतां॒ ॅयो᳚ऽस्मान् द्वेष्टि॑ । शरी॑रं ॅयज्ञ्शम॒लं कुसी॑दं॒ तस्मिन्᳚थ्सीदतु॒ यो᳚ऽस्मान् द्वेष्टि॑ ॥यज्ञ्॑ य॒ज्ञ्स्य॒ यत् तेज॒स्तेन॒ संक्रा॑म॒ माम॒भि । ब्रा॒ह्म॒णानृ॒त्विजो॑ दे॒वान् य॒ज्ञ्स्य॒ तप॑सा ते सवा॒हमा हु॑वे ॥इ॒ष्टेन॑ प॒क्वमुप॑ - [  ] \newline

\textbf{Pada Paata} \newline

अ॒र्वाङ् । य॒ज्ञ्ः । समिति॑ । क्रा॒म॒तु॒ । अ॒मुष्मा᳚त् । अधीति॑ । माम् । अ॒भि ॥ ऋषी॑णाम् । यः । पु॒रोहि॑त॒ इति॑ पु॒रः - हि॒तः॒ ॥ निर्दे॑व॒मिति॒ निः - दे॒व॒म् । निर्वी॑र॒मिति॒ निः - वी॒र॒म् । कृ॒त्वा । विष्क॑न्ध॒मिति॒ वि - स्क॒न्ध॒म् । तस्मिन्न्॑ । ही॒य॒ता॒म् । यः । अ॒स्मान् । द्वेष्टि॑ ॥ शरी॑रम् । य॒ज्ञ्॒श॒म॒लमिति॑ यज्ञ् - श॒म॒लम् । कुसी॑दम् । तस्मिन्न्॑ । सी॒द॒तु॒ । यः । अ॒स्मान् । द्वेष्टि॑ ॥ यज्ञ्॑ । य॒ज्ञ्स्य॑ । यत् । तेजः॑ । तेन॑ । समिति॑ । क्रा॒म॒ । माम् । अ॒भि ॥ ब्रा॒ह्म॒णान् । ऋ॒त्विजः॑ । दे॒वान् । य॒ज्ञ्स्य॑ । तप॑सा । ते॒ । स॒व॒ । अ॒हम् । एति॑ । हु॒वे॒ ॥ इ॒ष्टेन॑ । प॒क्वम् । उपेति॑ ।  \newline


\textbf{Krama Paata} \newline

अ॒र्वाङ् य॒ज्ञ्ः । य॒ज्ञ्ः सम् । सम् क्रा॑मतु । क्रा॒म॒त्व॒मुष्मा᳚त् । अ॒मुष्मा॒दधि॑ । अधि॒ माम् । माम॒भि । अ॒भीत्य॒भि ॥ ऋषी॑णा॒म् ॅयः । यः पु॒रोहि॑तः । पु॒रोहि॑त॒ इति॑ पु॒रः - हि॒तः॒ ॥ निर्दे॑व॒म् निर्वी॑रम् । निर्दे॑व॒मिति॒ निः - दे॒व॒म् । निर्वी॑रम् कृ॒त्वा । निर्वी॑र॒मिति॒ निः - वी॒र॒म् । कृ॒त्वा विष्क॑न्धम् । विष्क॑न्ध॒म् तस्मिन्न्॑ । विष्क॑न्ध॒मिति॒ वि - स्क॒न्ध॒म् । तस्मि॑न्. हीयताम् । ही॒य॒ता॒म् ॅयः । यो᳚ऽस्मान् । अ॒स्मान् द्वेष्टि॑ । द्वेष्टीति॒ द्वेष्टि॑ ॥ शरी॑रम् ॅयज्ञ्शम॒लम् । य॒ज्ञ्॒श॒म॒लम् कुसी॑दम् । य॒ज्ञ्॒श॒म॒लमिति॑ यज्ञ् - श॒म॒लम् । कुसी॑द॒म् तस्मिन्न्॑ । तस्मि᳚न्थ् सीदतु । सी॒द॒तु॒ यः । यो᳚ऽस्मान् । अ॒स्मान् द्वेष्टि॑ । द्वेष्टीति॒ द्वेष्टि॑ ॥ यज्ञ्॑ य॒ज्ञ्स्य॑ । य॒ज्ञ्स्य॒ यत् । यत् तेजः॑ । तेज॒स्तेन॑ । तेन॒ सम् । सम् क्रा॑म । क्रा॒म॒ माम् । माम॒भि । अ॒भीत्य॒भि ॥ ब्रा॒ह्म॒णानृ॒त्विजः॑ । ऋ॒त्विजो॑ दे॒वान् । दे॒वान्. य॒ज्ञ्स्य॑ । य॒ज्ञ्स्य॒ तप॑सा । तप॑सा ते । ते॒ स॒व॒ । स॒वा॒हम् । अ॒हमा । आ हु॑वे । हु॒व॒ इति॑ हुवे ॥ इ॒ष्टेन॑ प॒क्वम् । प॒क्वमुप॑ । उप॑ ते \newline

\textbf{Jatai Paata} \newline

1. अ॒र्वाङ् य॒ज्ञो य॒ज्ञो᳚ ऽर्वाङ् ङ॒र्वाङ् य॒ज्ञ्ः । \newline
2. य॒ज्ञ्ः सꣳ सं ॅय॒ज्ञो य॒ज्ञ्ः सम् । \newline
3. सम् क्रा॑मतु क्रामतु॒ सꣳ सम् क्रा॑मतु । \newline
4. क्रा॒म॒ त्व॒मुष्मा॑द॒ मुष्मा᳚त् क्रामतु क्राम त्व॒मुष्मा᳚त् । \newline
5. अ॒मुष्मा॒ दध्य ध्य॒मुष्मा॑ द॒मुष्मा॒ दधि॑ । \newline
6. अधि॒ माम् मा मध्यधि॒ माम् । \newline
7. मा म॒भ्य॑भि माम् मा म॒भि । \newline
8. अ॒भीत्य॒भि । \newline
9. ऋषी॑णां॒ ॅयो य ऋषी॑णा॒ मृषी॑णां॒ ॅयः । \newline
10. यः पु॒रोहि॑तः पु॒रोहि॑तो॒ यो यः पु॒रोहि॑तः । \newline
11. पु॒रोहि॑त॒ इति॑ पु॒रः - हि॒तः॒ । \newline
12. निर्दे॑व॒न् निर्वी॑र॒न् निर्वी॑र॒न् निर्दे॑व॒न् निर्दे॑व॒न् निर्वी॑रम् । \newline
13. निर्दे॑व॒मिति॒ निः - दे॒व॒म् । \newline
14. निर्वी॑रम् कृ॒त्वा कृ॒त्वा निर्वी॑र॒न् निर्वी॑रम् कृ॒त्वा । \newline
15. निर्वी॑र॒मिति॒ निः - वी॒र॒म् । \newline
16. कृ॒त्वा विष्क॑न्धं॒ ॅविष्क॑न्धम् कृ॒त्वा कृ॒त्वा विष्क॑न्धम् । \newline
17. विष्क॑न्ध॒म् तस्मिꣳ॒॒ स्तस्मि॒न्॒. विष्क॑न्धं॒ ॅविष्क॑न्ध॒म् तस्मिन्न्॑ । \newline
18. विष्क॑न्ध॒मिति॒ वि - स्क॒न्ध॒म् । \newline
19. तस्मि॑न्. हीयताꣳ हीयता॒म् तस्मिꣳ॒॒ स्तस्मि॑न्. हीयताम् । \newline
20. ही॒य॒तां॒ ॅयो यो ही॑यताꣳ हीयतां॒ ॅयः । \newline
21. यो᳚ ऽस्मा न॒स्मान्. यो यो᳚ ऽस्मान् । \newline
22. अ॒स्मान् द्वेष्टि॒ द्वेष्ट्य॒स्मा न॒स्मान् द्वेष्टि॑ । \newline
23. द्वेष्टीति॒ द्वेष्टि॑ । \newline
24. शरी॑रं ॅयज्ञ्शम॒लं ॅय॑ज्ञ्शम॒लꣳ शरी॑रꣳ॒॒ शरी॑रं ॅयज्ञ्शम॒लम् । \newline
25. य॒ज्ञ्॒श॒म॒लम् कुसी॑द॒म् कुसी॑दं ॅयज्ञ्शम॒लं ॅय॑ज्ञ्शम॒लम् कुसी॑दम् । \newline
26. य॒ज्ञ्॒श॒म॒लमिति॑ यज्ञ् - श॒म॒लम् । \newline
27. कुसी॑द॒म् तस्मिꣳ॒॒ स्तस्मि॒न् कुसी॑द॒म् कुसी॑द॒म् तस्मिन्न्॑ । \newline
28. तस्मिन्᳚ थ्सीदतु सीदतु॒ तस्मिꣳ॒॒ स्तस्मिन्᳚ थ्सीदतु । \newline
29. सी॒द॒तु॒ यो यः सी॑दतु सीदतु॒ यः । \newline
30. यो᳚ ऽस्मा न॒स्मान्. यो यो᳚ ऽस्मान् । \newline
31. अ॒स्मान् द्वेष्टि॒ द्वेष्ट्य॒स्मा न॒स्मान् द्वेष्टि॑ । \newline
32. द्वेष्टीति॒ द्वेष्टि॑ । \newline
33. यज्ञ्॑ य॒ज्ञ्स्य॑ य॒ज्ञ्स्य॒ यज्ञ्॒ यज्ञ्॑ य॒ज्ञ्स्य॑ । \newline
34. य॒ज्ञ्स्य॒ यद् यद् य॒ज्ञ्स्य॑ य॒ज्ञ्स्य॒ यत् । \newline
35. यत् तेज॒ स्तेजो॒ यद् यत् तेजः॑ । \newline
36. तेज॒ स्तेन॒ तेन॒ तेज॒ स्तेज॒ स्तेन॑ । \newline
37. तेन॒ सꣳ सम् तेन॒ तेन॒ सम् । \newline
38. सम् क्रा॑म क्राम॒ सꣳ सम् क्रा॑म । \newline
39. क्रा॒म॒ माम् माम् क्रा॑म क्राम॒ माम् । \newline
40. मा म॒भ्य॑भि माम् मा म॒भि । \newline
41. अ॒भीत्य॒भि । \newline
42. ब्रा॒ह्म॒णा नृ॒त्विज॑ ऋ॒त्विजो᳚ ब्राह्म॒णान् ब्रा᳚ह्म॒णा नृ॒त्विजः॑ । \newline
43. ऋ॒त्विजो॑ दे॒वान् दे॒वा नृ॒त्विज॑ ऋ॒त्विजो॑ दे॒वान् । \newline
44. दे॒वान्. य॒ज्ञ्स्य॑ य॒ज्ञ्स्य॑ दे॒वान् दे॒वान्. य॒ज्ञ्स्य॑ । \newline
45. य॒ज्ञ्स्य॒ तप॑सा॒ तप॑सा य॒ज्ञ्स्य॑ य॒ज्ञ्स्य॒ तप॑सा । \newline
46. तप॑सा ते ते॒ तप॑सा॒ तप॑सा ते । \newline
47. ते॒ स॒व॒ स॒व॒ ते॒ ते॒ स॒व॒ । \newline
48. स॒वा॒ह म॒हꣳ स॑व सवा॒हम् । \newline
49. अ॒ह मा ऽह म॒ह मा । \newline
50. आ हु॑वे हुव॒ आ हु॑वे । \newline
51. हु॒व॒ इति॑ हुवे । \newline
52. इ॒ष्टेन॑ प॒क्वम् प॒क्व मि॒ष्टे ने॒ष्टेन॑ प॒क्वम् । \newline
53. प॒क्व मुपोप॑ प॒क्वम् प॒क्व मुप॑ । \newline
54. उप॑ ते त॒ उपोप॑ ते । \newline

\textbf{Ghana Paata } \newline

1. अ॒र्वाङ् य॒ज्ञो य॒ज्ञो᳚ ऽर्वाङ् ङ॒र्वाङ् य॒ज्ञ्ः सꣳ सं ॅय॒ज्ञो᳚ ऽर्वाङ् ङ॒र्वाङ् य॒ज्ञ्ः सम् । \newline
2. य॒ज्ञ्ः सꣳ सं ॅय॒ज्ञो य॒ज्ञ्ः सम् क्रा॑मतु क्रामतु॒ सं ॅय॒ज्ञो य॒ज्ञ्ः सम् क्रा॑मतु । \newline
3. सम् क्रा॑मतु क्रामतु॒ सꣳ सम् क्रा॑म त्व॒मुष्मा॑ द॒मुष्मा᳚त् क्रामतु॒ सꣳ सम् क्रा॑म त्व॒मुष्मा᳚त् । \newline
4. क्रा॒म॒ त्व॒मुष्मा॑ द॒मुष्मा᳚त् क्रामतु क्राम त्व॒मुष्मा॒ दध्य ध्य॒मुष्मा᳚त् क्रामतु क्राम त्व॒मुष्मा॒ दधि॑ । \newline
5. अ॒मुष्मा॒ दध्यध्य॒ मुष्मा॑ द॒मुष्मा॒ दधि॒ माम् मा मध्य॒ मुष्मा॑ द॒मुष्मा॒ दधि॒ माम् । \newline
6. अधि॒ माम् मा मध्यधि॒ मा म॒भ्य॑भि मा मध्यधि॒ मा म॒भि । \newline
7. मा म॒भ्य॑भि माम् मा म॒भि । \newline
8. अ॒भीत्य॒भि । \newline
9. ऋषी॑णां॒ ॅयो य ऋषी॑णा॒ मृषी॑णां॒ ॅयः पु॒रोहि॑तः पु॒रोहि॑तो॒ य ऋषी॑णा॒ मृषी॑णां॒ ॅयः पु॒रोहि॑तः । \newline
10. यः पु॒रोहि॑तः पु॒रोहि॑तो॒ यो यः पु॒रोहि॑तः । \newline
11. पु॒रोहि॑त॒ इति॑ पु॒रः - हि॒तः॒ । \newline
12. निर्दे॑व॒न् निर्वी॑र॒न् निर्वी॑र॒न् निर्दे॑व॒न् निर्दे॑व॒न् निर्वी॑रम् कृ॒त्वा कृ॒त्वा निर्वी॑र॒न् निर्दे॑व॒न् निर्दे॑व॒न् निर्वी॑रम् कृ॒त्वा । \newline
13. निर्दे॑व॒मिति॒ निः - दे॒व॒म् । \newline
14. निर्वी॑रम् कृ॒त्वा कृ॒त्वा निर्वी॑र॒न् निर्वी॑रम् कृ॒त्वा विष्क॑न्धं॒ ॅविष्क॑न्धम् कृ॒त्वा निर्वी॑र॒न् निर्वी॑रम् कृ॒त्वा विष्क॑न्धम् । \newline
15. निर्वी॑र॒मिति॒ निः - वी॒र॒म् । \newline
16. कृ॒त्वा विष्क॑न्धं॒ ॅविष्क॑न्धम् कृ॒त्वा कृ॒त्वा विष्क॑न्ध॒म् तस्मिꣳ॒॒ स्तस्मि॒न्॒. विष्क॑न्धम् कृ॒त्वा कृ॒त्वा विष्क॑न्ध॒म् तस्मिन्न्॑ । \newline
17. विष्क॑न्ध॒म् तस्मिꣳ॒॒ स्तस्मि॒न्॒. विष्क॑न्धं॒ ॅविष्क॑न्ध॒म् तस्मि॑न्. हीयताꣳ हीयता॒म् तस्मि॒न्॒. विष्क॑न्धं॒ ॅविष्क॑न्ध॒म् तस्मि॑न्. हीयताम् । \newline
18. विष्क॑न्ध॒मिति॒ वि - स्क॒न्ध॒म् । \newline
19. तस्मि॑न्. हीयताꣳ हीयता॒म् तस्मिꣳ॒॒ स्तस्मि॑न्. हीयतां॒ ॅयो यो ही॑यता॒म् तस्मिꣳ॒॒ स्तस्मि॑न्. हीयतां॒ ॅयः । \newline
20. ही॒य॒तां॒ ॅयो यो ही॑यताꣳ हीयतां॒ ॅयो᳚ ऽस्मा न॒स्मान्. यो ही॑यताꣳ हीयतां॒ ॅयो᳚ ऽस्मान् । \newline
21. यो᳚ ऽस्मा न॒स्मान्. यो यो᳚ ऽस्मान् द्वेष्टि॒ द्वेष्ट्य॒स्मान्. यो यो᳚ ऽस्मान् द्वेष्टि॑ । \newline
22. अ॒स्मान् द्वेष्टि॒ द्वेष्ट्य॒स्मा न॒स्मान् द्वेष्टि॑ । \newline
23. द्वेष्टीति॒ द्वेष्टि॑ । \newline
24. शरी॑रं ॅयज्ञ्शम॒लं ॅय॑ज्ञ्शम॒लꣳ शरी॑रꣳ॒॒ शरी॑रं ॅयज्ञ्शम॒लम् कुसी॑द॒म् कुसी॑दं ॅयज्ञ्शम॒लꣳ शरी॑रꣳ॒॒ शरी॑रं ॅयज्ञ्शम॒लम् कुसी॑दम् । \newline
25. य॒ज्ञ्॒श॒म॒लम् कुसी॑द॒म् कुसी॑दं ॅयज्ञ्शम॒लं ॅय॑ज्ञ्शम॒लम् कुसी॑द॒म् तस्मिꣳ॒॒ स्तस्मि॒न् कुसी॑दं ॅयज्ञ्शम॒लं ॅय॑ज्ञ्शम॒लम् कुसी॑द॒म् तस्मिन्न्॑ । \newline
26. य॒ज्ञ्॒श॒म॒लमिति॑ यज्ञ् - श॒म॒लम् । \newline
27. कुसी॑द॒म् तस्मिꣳ॒॒ स्तस्मि॒न् कुसी॑द॒म् कुसी॑द॒म् तस्मिन्᳚ थ्सीदतु सीदतु॒ तस्मि॒न् कुसी॑द॒म् कुसी॑द॒म् तस्मिन्᳚ थ्सीदतु । \newline
28. तस्मिन्᳚ थ्सीदतु सीदतु॒ तस्मिꣳ॒॒ स्तस्मिन्᳚ थ्सीदतु॒ यो यः सी॑दतु॒ तस्मिꣳ॒॒ स्तस्मिन्᳚ थ्सीदतु॒ यः । \newline
29. सी॒द॒तु॒ यो यः सी॑दतु सीदतु॒ यो᳚ ऽस्मा न॒स्मान्. यः सी॑दतु सीदतु॒ यो᳚ ऽस्मान् । \newline
30. यो᳚ ऽस्मा न॒स्मान्. यो यो᳚ ऽस्मान् द्वेष्टि॒ द्वेष्ट्य॒स्मान्. यो यो᳚ ऽस्मान् द्वेष्टि॑ । \newline
31. अ॒स्मान् द्वेष्टि॒ द्वेष्ट्य॒स्मा न॒स्मान् द्वेष्टि॑ । \newline
32. द्वेष्टीति॒ द्वेष्टि॑ । \newline
33. यज्ञ्॑ य॒ज्ञ्स्य॑ य॒ज्ञ्स्य॒ यज्ञ्॒ यज्ञ्॑ य॒ज्ञ्स्य॒ यद् यद् य॒ज्ञ्स्य॒ यज्ञ्॒ यज्ञ्॑ य॒ज्ञ्स्य॒ यत् । \newline
34. य॒ज्ञ्स्य॒ यद् यद् य॒ज्ञ्स्य॑ य॒ज्ञ्स्य॒ यत् तेज॒ स्तेजो॒ यद् य॒ज्ञ्स्य॑ य॒ज्ञ्स्य॒ यत् तेजः॑ । \newline
35. यत् तेज॒ स्तेजो॒ यद् यत् तेज॒ स्तेन॒ तेन॒ तेजो॒ यद् यत् तेज॒ स्तेन॑ । \newline
36. तेज॒ स्तेन॒ तेन॒ तेज॒ स्तेज॒ स्तेन॒ सꣳ सम् तेन॒ तेज॒ स्तेज॒ स्तेन॒ सम् । \newline
37. तेन॒ सꣳ सम् तेन॒ तेन॒ सम् क्रा॑म क्राम॒ सम् तेन॒ तेन॒ सम् क्रा॑म । \newline
38. सम् क्रा॑म क्राम॒ सꣳ सम् क्रा॑म॒ माम् माम् क्रा॑म॒ सꣳ सम् क्रा॑म॒ माम् । \newline
39. क्रा॒म॒ माम् माम् क्रा॑म क्राम॒ मा म॒भ्य॑भि माम् क्रा॑म क्राम॒ मा म॒भि । \newline
40. मा म॒भ्य॑भि माम् मा म॒भि । \newline
41. अ॒भीत्य॒भि । \newline
42. ब्रा॒ह्म॒णा नृ॒त्विज॑ ऋ॒त्विजो᳚ ब्राह्म॒णान् ब्रा᳚ह्म॒णा नृ॒त्विजो॑ दे॒वान् दे॒वा नृ॒त्विजो᳚ ब्राह्म॒णान् ब्रा᳚ह्म॒णा नृ॒त्विजो॑ दे॒वान् । \newline
43. ऋ॒त्विजो॑ दे॒वान् दे॒वा नृ॒त्विज॑ ऋ॒त्विजो॑ दे॒वान्. य॒ज्ञ्स्य॑ य॒ज्ञ्स्य॑ दे॒वा नृ॒त्विज॑ ऋ॒त्विजो॑ दे॒वान्. य॒ज्ञ्स्य॑ । \newline
44. दे॒वान्. य॒ज्ञ्स्य॑ य॒ज्ञ्स्य॑ दे॒वान् दे॒वान्. य॒ज्ञ्स्य॒ तप॑सा॒ तप॑सा य॒ज्ञ्स्य॑ दे॒वान् दे॒वान्. य॒ज्ञ्स्य॒ तप॑सा । \newline
45. य॒ज्ञ्स्य॒ तप॑सा॒ तप॑सा य॒ज्ञ्स्य॑ य॒ज्ञ्स्य॒ तप॑सा ते ते॒ तप॑सा य॒ज्ञ्स्य॑ य॒ज्ञ्स्य॒ तप॑सा ते । \newline
46. तप॑सा ते ते॒ तप॑सा॒ तप॑सा ते सव सव ते॒ तप॑सा॒ तप॑सा ते सव । \newline
47. ते॒ स॒व॒ स॒व॒ ते॒ ते॒ स॒वा॒ह म॒हꣳ स॑व ते ते सवा॒हम् । \newline
48. स॒वा॒ह म॒हꣳ स॑व सवा॒ह मा ऽहꣳ स॑व सवा॒ह मा । \newline
49. अ॒ह मा ऽह म॒ह मा हु॑वे हुव॒ आ ऽह म॒ह मा हु॑वे । \newline
50. आ हु॑वे हुव॒ आ हु॑वे । \newline
51. हु॒व॒ इति॑ हुवे । \newline
52. इ॒ष्टेन॑ प॒क्वम् प॒क्व मि॒ष्टे ने॒ष्टेन॑ प॒क्व मुपोप॑ प॒क्व मि॒ष्टे ने॒ष्टेन॑ प॒क्व मुप॑ । \newline
53. प॒क्व मुपोप॑ प॒क्वम् प॒क्व मुप॑ ते त॒ उप॑ प॒क्वम् प॒क्व मुप॑ ते । \newline
54. उप॑ ते त॒ उपोप॑ ते हुवे हुवे त॒ उपोप॑ ते हुवे । \newline
\pagebreak
\markright{ TS 7.3.11.2  \hfill https://www.vedavms.in \hfill}

\section{ TS 7.3.11.2 }

\textbf{TS 7.3.11.2 } \newline
\textbf{Samhita Paata} \newline

ते हुवे सवा॒हं । सं ते॑ वृञ्जे सुकृ॒तꣳ सं प्र॒जां प॒शून् ॥प्रै॒षान्थ्-सा॑मिधे॒नी-रा॑घा॒रावाज्य॑भागा॒वा-श्रु॑तं प्र॒त्याश्रु॑त॒मा शृ॑णामि ते । प्र॒या॒जा॒नू॒या॒जान्थ्-स्वि॑ष्ट॒कृत॒-मिडा॑मा॒शिष॒ आ वृ॑ञ्जे॒ सुवः॑ ॥ अ॒ग्निनेन्द्रे॑ण॒ सोमे॑न॒ सर॑स्वत्या॒ विष्णु॑ना दे॒वता॑भिः । या॒ज्या॒नु॒वा॒क्या᳚भ्या॒मुप॑ ते हुवे सवा॒हं ॅय॒ज्ञ्मा द॑दे ते॒ वष॑ट्कृतं ॥ स्तु॒तꣳ श॒स्त्रं प्र॑तिग॒रं ग्रह॒मिडा॑मा॒शिष॒ - [  ] \newline

\textbf{Pada Paata} \newline

ते॒ । हु॒वे॒ । स॒व॒ । अ॒हम् ॥ समिति॑ । ते॒ । वृ॒ञ्जे॒ । सु॒कृ॒तमिति॑ सु - कृ॒तम् । समिति॑ । प्र॒जामिति॑ प्र - जाम् । प॒शून् ॥ प्रै॒षानिति॑ प्र - ए॒षान् । सा॒मि॒धे॒नीरिति॑ सां - इ॒धे॒नीः । आ॒घा॒रावित्या᳚ - घा॒रौ । आज्य॑भागा॒वित्याज्य॑ - भा॒गौ॒ । आश्रु॑त॒मित्या - श्रु॒त॒म् । प्र॒त्याश्रु॑त॒मिति॑ प्रति - आश्रु॑तम् । एति॑ । शृ॒णा॒मि॒ । ते॒ ॥ प्र॒या॒जा॒नू॒या॒जानिति॑ प्रयाज - अ॒नू॒या॒जान् । स्वि॒ष्ट॒कृत॒मिति॑ स्विष्ट - कृत᳚म् । इडा᳚म् । आ॒शिष॒ इत्या᳚ - शिषः॑ । एति॑ । वृ॒ञ्जे॒ । सुवः॑ ॥ अ॒ग्निना᳚ । इन्द्रे॑ण । सोमे॑न । सर॑स्वत्या । विष्णु॑ना । दे॒वता॑भिः ॥ या॒ज्या॒नु॒वा॒क्या᳚भ्या॒मिति॑ याज्या - अ॒नु॒वा॒क्या᳚भ्याम् । उपेति॑ । ते॒ । हु॒वे॒ । स॒व॒ । अ॒हम् । य॒ज्ञ्म् । एति॑ । द॒दे॒ । ते॒ । वष॑ट्कृत॒मिति॒ वष॑ट् - कृ॒त॒म् ॥ स्तु॒तम् । श॒स्त्रम् । प्र॒ति॒ग॒रमिति॑ प्रति - ग॒रम् । ग्रह᳚म् । इडा᳚म् । आ॒शिष॒ इत्या᳚ - शिषः॑ ।  \newline


\textbf{Krama Paata} \newline

ते॒ हु॒वे॒ । हु॒वे॒ स॒व॒ । स॒वा॒हम् । अ॒हमित्य॒हम् ॥ सम् ते᳚ । ते॒ वृ॒ञ्जे॒ । वृ॒ञ्जे॒ सु॒कृ॒तम् । सु॒कृ॒तꣳ सम् । सु॒कृ॒तमिति॑ सु - कृ॒तम् । सम् प्र॒जाम् । प्र॒जाम् प॒शून् । प्र॒जामिति॑ प्र - जाम् । प॒शूनिति॑ प॒शून् ॥ प्रै॒षान्थ् सा॑मिधे॒नीः । प्रै॒षानिति॑ प्र - ए॒षान् । सा॒मि॒धे॒नीरा॑घा॒रौ । सा॒मि॒धे॒नीरिति॑ साम् - इ॒धे॒नीः । आ॒घा॒रावाज्य॑भागौ । आ॒घा॒रावित्या᳚ - घा॒रौ । आज्य॑भागा॒वाश्रु॑तम् । आज्य॑भागा॒वित्याज्य॑ - भा॒गौ॒ । आश्रु॑तम् प्र॒त्याश्रु॑तम् । आश्रु॑त॒मित्या - श्रु॒त॒म् । प्र॒त्याश्रु॑त॒मा । प्र॒त्याश्रु॑त॒मिति॑ प्रति - आश्रु॑तम् । आ शृ॑णामि । शृ॒णा॒मि॒ ते॒ । त॒ इति॑ ते ॥ प्र॒या॒जा॒नू॒या॒जान्थ् स्वि॑ष्ट॒कृत᳚म् । प्र॒या॒जा॒नू॒या॒जानिति॑ प्रयाज - अ॒नू॒या॒जान् । स्वि॒ष्ट॒कृत॒मिडा᳚म् । स्वि॒ष्ट॒कृत॒मिति॑ स्विष्ट - कृत᳚म् । इडा॑मा॒शिषः॑ । आ॒शिष॒ आ । आ॒शिष॒ इत्या᳚ - शिषः॑ । आ वृ॑ञ्जे । वृ॒ञ्जे॒ सुवः॑ । सुव॒रिति॒ सुवः॑ ॥ अ॒ग्निनेन्द्रे॑ण । इन्द्रे॑ण॒ सोमे॑न । सोमे॑न॒ सर॑स्वत्या । सर॑स्वत्या॒ विष्णु॑ना । विष्णु॑ना दे॒वता॑भिः । दे॒वता॑भि॒रिति॑ दे॒वता॑भिः ॥ या॒ज्या॒नु॒वा॒क्या᳚भ्या॒मुप॑ । या॒ज्या॒नु॒वा॒क्या᳚भ्या॒मिति॑ याज्या - अ॒नु॒वा॒क्या᳚भ्याम् । उप॑ ते । ते॒ हु॒वे॒ । हु॒वे॒ स॒व॒ । स॒वा॒हम् । अ॒हम् ॅय॒ज्ञ्म् । य॒ज्ञ्मा । आ द॑दे । द॒दे॒ ते॒ । ते॒ वष॑ट्कृतम् । वष॑ट्कृत॒मिति॒ वष॑ट् - कृ॒त॒म् ॥ स्तु॒तꣳ श॒स्त्रम् । श॒स्त्रम् प्र॑तिग॒रम् । प्र॒ति॒ग॒रम् ग्रह᳚म् । प्र॒ति॒ग॒रमिति॑ प्रति - ग॒रम् । ग्रह॒मिडा᳚म् । इडा॑मा॒शिषः॑ ( ) । आ॒शिष॒ आ । आ॒शिष॒ इत्या᳚ - शिषः॑ \newline

\textbf{Jatai Paata} \newline

1. ते॒ हु॒वे॒ हु॒वे॒ ते॒ ते॒ हु॒वे॒ । \newline
2. हु॒वे॒ स॒व॒ स॒व॒ हु॒वे॒ हु॒वे॒ स॒व॒ । \newline
3. स॒वा॒ह म॒हꣳ स॑व सवा॒हम् । \newline
4. अ॒ह॒मित्य॒हम् । \newline
5. सम् ते॑ ते॒ सꣳ सम् ते᳚ । \newline
6. ते॒ वृ॒ञ्जे॒ वृ॒ञ्जे॒ ते॒ ते॒ वृ॒ञ्जे॒ । \newline
7. वृ॒ञ्जे॒ सु॒कृ॒तꣳ सु॑कृ॒तं ॅवृ॑ञ्जे वृञ्जे सुकृ॒तम् । \newline
8. सु॒कृ॒तꣳ सꣳ सꣳ सु॑कृ॒तꣳ सु॑कृ॒तꣳ सम् । \newline
9. सु॒कृ॒तमिति॑ सु - कृ॒तम् । \newline
10. सम् प्र॒जाम् प्र॒जाꣳ सꣳ सम् प्र॒जाम् । \newline
11. प्र॒जाम् प॒शून् प॒शून् प्र॒जाम् प्र॒जाम् प॒शून् । \newline
12. प्र॒जामिति॑ प्र - जाम् । \newline
13. प॒शूनिति॑ प॒शून् । \newline
14. प्रै॒षान् थ्सा॑मिधे॒नीः सा॑मिधे॒नीः प्रै॒षान् प्रै॒षान् थ्सा॑मिधे॒नीः । \newline
15. प्रै॒षानिति॑ प्र - ए॒षान् । \newline
16. सा॒मि॒धे॒नी रा॑घा॒रा वा॑घा॒रौ सा॑मिधे॒नीः सा॑मिधे॒नी रा॑घा॒रौ । \newline
17. सा॒मि॒धे॒नीरिति॑ सां - इ॒धे॒नीः । \newline
18. आ॒घा॒रा वाज्य॑भागा॒ वाज्य॑भागा वाघा॒रा वा॑घा॒रा वाज्य॑भागौ । \newline
19. आ॒घा॒रावित्या᳚ - घा॒रौ । \newline
20. आज्य॑भागा॒ वाश्रु॑त॒ माश्रु॑त॒ माज्य॑भागा॒ वाज्य॑भागा॒ वाश्रु॑तम् । \newline
21. आज्य॑भागा॒वित्याज्य॑ - भा॒गौ॒ । \newline
22. आश्रु॑तम् प्र॒त्याश्रु॑तम् प्र॒त्याश्रु॑त॒ माश्रु॑त॒ माश्रु॑तम् प्र॒त्याश्रु॑तम् । \newline
23. आश्रु॑त॒मित्या - श्रु॒त॒म् । \newline
24. प्र॒त्याश्रु॑त॒ मा प्र॒त्याश्रु॑तम् प्र॒त्याश्रु॑त॒ मा । \newline
25. प्र॒त्याश्रु॑त॒मिति॑ प्रति - आश्रु॑तम् । \newline
26. आ शृ॑णामि शृणा॒म्या शृ॑णामि । \newline
27. शृ॒णा॒मि॒ ते॒ ते॒ शृ॒णा॒मि॒ शृ॒णा॒मि॒ ते॒ । \newline
28. त॒ इति॑ ते । \newline
29. प्र॒या॒जा॒नू॒या॒जान् थ्स्वि॑ष्ट॒कृतꣳ॑ स्विष्ट॒कृत॑म् प्रयाजानूया॒जान् प्र॑याजानूया॒जान् थ्स्वि॑ष्ट॒कृत᳚म् । \newline
30. प्र॒या॒जा॒नू॒या॒जानिति॑ प्रयाज - अ॒नू॒या॒जान् । \newline
31. स्वि॒ष्ट॒कृत॒ मिडा॒ मिडाꣳ॑ स्विष्ट॒कृतꣳ॑ स्विष्ट॒कृत॒ मिडा᳚म् । \newline
32. स्वि॒ष्ट॒कृत॒मिति॑ स्विष्ट - कृत᳚म् । \newline
33. इडा॑ मा॒शिष॑ आ॒शिष॒ इडा॒ मिडा॑ मा॒शिषः॑ । \newline
34. आ॒शिष॒ आ ऽऽशिष॑ आ॒शिष॒ आ । \newline
35. आ॒शिष॒ इत्या᳚ - शिषः॑ । \newline
36. आ वृ॑ञ्जे वृञ्ज॒ आ वृ॑ञ्जे । \newline
37. वृ॒ञ्जे॒ सुवः॒ सुव॑र् वृञ्जे वृञ्जे॒ सुवः॑ । \newline
38. सुव॒रिति॒ सुवः॑ । \newline
39. अ॒ग्नि नेन्द्रे॒ णेन्द्रे॑णा॒ग्निना॒ ऽग्नि नेन्द्रे॑ण । \newline
40. इन्द्रे॑ण॒ सोमे॑न॒ सोमे॒ नेन्द्रे॒ णेन्द्रे॑ण॒ सोमे॑न । \newline
41. सोमे॑न॒ सर॑स्वत्या॒ सर॑स्वत्या॒ सोमे॑न॒ सोमे॑न॒ सर॑स्वत्या । \newline
42. सर॑स्वत्या॒ विष्णु॑ना॒ विष्णु॑ना॒ सर॑स्वत्या॒ सर॑स्वत्या॒ विष्णु॑ना । \newline
43. विष्णु॑ना दे॒वता॑भिर् दे॒वता॑भि॒र् विष्णु॑ना॒ विष्णु॑ना दे॒वता॑भिः । \newline
44. दे॒वता॑भि॒रिति॑ दे॒वता॑भिः । \newline
45. या॒ज्या॒नु॒वा॒क्या᳚भ्या॒ मुपोप॑ याज्यानुवा॒क्या᳚भ्यां ॅयाज्यानुवा॒क्या᳚भ्या॒ मुप॑ । \newline
46. या॒ज्या॒नु॒वा॒क्या᳚भ्या॒मिति॑ याज्या - अ॒नु॒वा॒क्या᳚भ्याम् । \newline
47. उप॑ ते त॒ उपोप॑ ते । \newline
48. ते॒ हु॒वे॒ हु॒वे॒ ते॒ ते॒ हु॒वे॒ । \newline
49. हु॒वे॒ स॒व॒ स॒व॒ हु॒वे॒ हु॒वे॒ स॒व॒ । \newline
50. स॒वा॒ह म॒हꣳ स॑व सवा॒हम् । \newline
51. अ॒हं ॅय॒ज्ञ्ं ॅय॒ज्ञ् म॒ह म॒हं ॅय॒ज्ञ्म् । \newline
52. य॒ज्ञ् मा य॒ज्ञ्ं ॅय॒ज्ञ् मा । \newline
53. आ द॑दे दद॒ आ द॑दे । \newline
54. द॒दे॒ ते॒ ते॒ द॒दे॒ द॒दे॒ ते॒ । \newline
55. ते॒ वष॑ट्कृतं॒ ॅवष॑ट्कृतम् ते ते॒ वष॑ट्कृतम् । \newline
56. वष॑ट्कृत॒मिति॒ वष॑ट् - कृ॒त॒म् । \newline
57. स्तु॒तꣳ श॒स्त्रꣳ श॒स्त्रꣳ स्तु॒तꣳ स्तु॒तꣳ श॒स्त्रम् । \newline
58. श॒स्त्रम् प्र॑तिग॒रम् प्र॑तिग॒रꣳ श॒स्त्रꣳ श॒स्त्रम् प्र॑तिग॒रम् । \newline
59. प्र॒ति॒ग॒रम् ग्रह॒म् ग्रह॑म् प्रतिग॒रम् प्र॑तिग॒रम् ग्रह᳚म् । \newline
60. प्र॒ति॒ग॒रमिति॑ प्रति - ग॒रम् । \newline
61. ग्रह॒ मिडा॒ मिडा॒म् ग्रह॒म् ग्रह॒ मिडा᳚म् । \newline
62. इडा॑ मा॒शिष॑ आ॒शिष॒ इडा॒ मिडा॑ मा॒शिषः॑ । \newline
63. आ॒शिष॒ आ ऽऽशिष॑ आ॒शिष॒ आ । \newline
64. आ॒शिष॒ इत्या᳚ - शिषः॑ । \newline

\textbf{Ghana Paata } \newline

1. ते॒ हु॒वे॒ हु॒वे॒ ते॒ ते॒ हु॒वे॒ स॒व॒ स॒व॒ हु॒वे॒ ते॒ ते॒ हु॒वे॒ स॒व॒ । \newline
2. हु॒वे॒ स॒व॒ स॒व॒ हु॒वे॒ हु॒वे॒ स॒वा॒ह म॒हꣳ स॑व हुवे हुवे सवा॒हम् । \newline
3. स॒वा॒ह म॒हꣳ स॑व सवा॒हम् । \newline
4. अ॒ह॒मित्य॒हम् । \newline
5. सम् ते॑ ते॒ सꣳ सम् ते॑ वृञ्जे वृञ्जे ते॒ सꣳ सम् ते॑ वृञ्जे । \newline
6. ते॒ वृ॒ञ्जे॒ वृ॒ञ्जे॒ ते॒ ते॒ वृ॒ञ्जे॒ सु॒कृ॒तꣳ सु॑कृ॒तं ॅवृ॑ञ्जे ते ते वृञ्जे सुकृ॒तम् । \newline
7. वृ॒ञ्जे॒ सु॒कृ॒तꣳ सु॑कृ॒तं ॅवृ॑ञ्जे वृञ्जे सुकृ॒तꣳ सꣳ सꣳ सु॑कृ॒तं ॅवृ॑ञ्जे वृञ्जे सुकृ॒तꣳ सम् । \newline
8. सु॒कृ॒तꣳ सꣳ सꣳ सु॑कृ॒तꣳ सु॑कृ॒तꣳ सम् प्र॒जाम् प्र॒जाꣳ सꣳ सु॑कृ॒तꣳ सु॑कृ॒तꣳ सम् प्र॒जाम् । \newline
9. सु॒कृ॒तमिति॑ सु - कृ॒तम् । \newline
10. सम् प्र॒जाम् प्र॒जाꣳ सꣳ सम् प्र॒जाम् प॒शून् प॒शून् प्र॒जाꣳ सꣳ सम् प्र॒जाम् प॒शून् । \newline
11. प्र॒जाम् प॒शून् प॒शून् प्र॒जाम् प्र॒जाम् प॒शून् । \newline
12. प्र॒जामिति॑ प्र - जाम् । \newline
13. प॒शूनिति॑ प॒शून् । \newline
14. प्रै॒षान् थ्सा॑मिधे॒नीः सा॑मिधे॒नीः प्रै॒षान् प्रै॒षान् थ्सा॑मिधे॒नी रा॑घा॒रा वा॑घा॒रौ सा॑मिधे॒नीः प्रै॒षान् प्रै॒षान् थ्सा॑मिधे॒नी रा॑घा॒रौ । \newline
15. प्रै॒षानिति॑ प्र - ए॒षान् । \newline
16. सा॒मि॒धे॒नी रा॑घा॒रा वा॑घा॒रौ सा॑मिधे॒नीः सा॑मिधे॒नी रा॑घा॒रा वाज्य॑भागा॒ वाज्य॑भागा वाघा॒रौ सा॑मिधे॒नीः सा॑मिधे॒नी रा॑घा॒रा वाज्य॑भागौ । \newline
17. सा॒मि॒धे॒नीरिति॑ सां - इ॒धे॒नीः । \newline
18. आ॒घा॒रा वाज्य॑भागा॒ वाज्य॑भागा वाघा॒रा वा॑घा॒रा वाज्य॑भागा॒ वाश्रु॑त॒ माश्रु॑त॒ माज्य॑भागा वाघा॒रा वा॑घा॒रा वाज्य॑भागा॒ वाश्रु॑तम् । \newline
19. आ॒घा॒रावित्या᳚ - घा॒रौ । \newline
20. आज्य॑भागा॒ वाश्रु॑त॒ माश्रु॑त॒ माज्य॑भागा॒ वाज्य॑भागा॒ वाश्रु॑तम् प्र॒त्याश्रु॑तम् प्र॒त्याश्रु॑त॒ माश्रु॑त॒ माज्य॑भागा॒ वाज्य॑भागा॒ वाश्रु॑तम् प्र॒त्याश्रु॑तम् । \newline
21. आज्य॑भागा॒वित्याज्य॑ - भा॒गौ॒ । \newline
22. आश्रु॑तम् प्र॒त्याश्रु॑तम् प्र॒त्याश्रु॑त॒ माश्रु॑त॒ माश्रु॑तम् प्र॒त्याश्रु॑त॒ मा प्र॒त्याश्रु॑त॒ माश्रु॑त॒ माश्रु॑तम् प्र॒त्याश्रु॑त॒ मा । \newline
23. आश्रु॑त॒मित्या - श्रु॒त॒म् । \newline
24. प्र॒त्याश्रु॑त॒ मा प्र॒त्याश्रु॑तम् प्र॒त्याश्रु॑त॒ मा शृ॑णामि शृणा॒म्या प्र॒त्याश्रु॑तम् प्र॒त्याश्रु॑त॒ मा शृ॑णामि । \newline
25. प्र॒त्याश्रु॑त॒मिति॑ प्रति - आश्रु॑तम् । \newline
26. आ शृ॑णामि शृणा॒म्या शृ॑णामि ते ते शृणा॒म्या शृ॑णामि ते । \newline
27. शृ॒णा॒मि॒ ते॒ ते॒ शृ॒णा॒मि॒ शृ॒णा॒मि॒ ते॒ । \newline
28. त॒ इति॑ ते । \newline
29. प्र॒या॒जा॒नू॒या॒जान् थ्स्वि॑ष्ट॒कृतꣳ॑ स्विष्ट॒कृत॑म् प्रयाजानूया॒जान् प्र॑याजानूया॒जान् थ्स्वि॑ष्ट॒कृत॒ मिडा॒ मिडाꣳ॑ स्विष्ट॒कृत॑म् प्रयाजानूया॒जान् प्र॑याजानूया॒जान् थ्स्वि॑ष्ट॒कृत॒ मिडा᳚म् । \newline
30. प्र॒या॒जा॒नू॒या॒जानिति॑ प्रयाज - अ॒नू॒या॒जान् । \newline
31. स्वि॒ष्ट॒कृत॒ मिडा॒ मिडाꣳ॑ स्विष्ट॒कृतꣳ॑ स्विष्ट॒कृत॒ मिडा॑ मा॒शिष॑ आ॒शिष॒ इडाꣳ॑ स्विष्ट॒कृतꣳ॑ स्विष्ट॒कृत॒ मिडा॑ मा॒शिषः॑ । \newline
32. स्वि॒ष्ट॒कृत॒मिति॑ स्विष्ट - कृत᳚म् । \newline
33. इडा॑ मा॒शिष॑ आ॒शिष॒ इडा॒ मिडा॑ मा॒शिष॒ आ ऽऽशिष॒ इडा॒ मिडा॑ मा॒शिष॒ आ । \newline
34. आ॒शिष॒ आ ऽऽशिष॑ आ॒शिष॒ आ वृ॑ञ्जे वृञ्ज॒ आ ऽऽशिष॑ आ॒शिष॒ आ वृ॑ञ्जे । \newline
35. आ॒शिष॒ इत्या᳚ - शिषः॑ । \newline
36. आ वृ॑ञ्जे वृञ्ज॒ आ वृ॑ञ्जे॒ सुवः॒ सुव॑र् वृञ्ज॒ आ वृ॑ञ्जे॒ सुवः॑ । \newline
37. वृ॒ञ्जे॒ सुवः॒ सुव॑र् वृञ्जे वृञ्जे॒ सुवः॑ । \newline
38. सुव॒रिति॒ सुवः॑ । \newline
39. अ॒ग्नि नेन्द्रे॒ णेन्द्रे॑णा॒ग्निना॒ ऽग्निनेन्द्रे॑ण॒ सोमे॑न॒ सोमे॒ नेन्द्रे॑णा॒ ग्निना॒ ऽग्नि नेन्द्रे॑ण॒ सोमे॑न । \newline
40. इन्द्रे॑ण॒ सोमे॑न॒ सोमे॒ नेन्द्रे॒ णेन्द्रे॑ण॒ सोमे॑न॒ सर॑स्वत्या॒ सर॑स्वत्या॒ सोमे॒ नेन्द्रे॒ णेन्द्रे॑ण॒ सोमे॑न॒ सर॑स्वत्या । \newline
41. सोमे॑न॒ सर॑स्वत्या॒ सर॑स्वत्या॒ सोमे॑न॒ सोमे॑न॒ सर॑स्वत्या॒ विष्णु॑ना॒ विष्णु॑ना॒ सर॑स्वत्या॒ सोमे॑न॒ सोमे॑न॒ सर॑स्वत्या॒ विष्णु॑ना । \newline
42. सर॑स्वत्या॒ विष्णु॑ना॒ विष्णु॑ना॒ सर॑स्वत्या॒ सर॑स्वत्या॒ विष्णु॑ना दे॒वता॑भिर् दे॒वता॑भि॒र् विष्णु॑ना॒ सर॑स्वत्या॒ सर॑स्वत्या॒ विष्णु॑ना दे॒वता॑भिः । \newline
43. विष्णु॑ना दे॒वता॑भिर् दे॒वता॑भि॒र् विष्णु॑ना॒ विष्णु॑ना दे॒वता॑भिः । \newline
44. दे॒वता॑भि॒रिति॑ दे॒वता॑भिः । \newline
45. या॒ज्या॒नु॒वा॒क्या᳚भ्या॒ मुपोप॑ याज्यानुवा॒क्या᳚भ्यां ॅयाज्यानुवा॒क्या᳚भ्या॒ मुप॑ ते त॒ उप॑ याज्यानुवा॒क्या᳚भ्यां ॅयाज्यानुवा॒क्या᳚भ्या॒ मुप॑ ते । \newline
46. या॒ज्या॒नु॒वा॒क्या᳚भ्या॒मिति॑ याज्या - अ॒नु॒वा॒क्या᳚भ्याम् । \newline
47. उप॑ ते त॒ उपोप॑ ते हुवे हुवे त॒ उपोप॑ ते हुवे । \newline
48. ते॒ हु॒वे॒ हु॒वे॒ ते॒ ते॒ हु॒वे॒ स॒व॒ स॒व॒ हु॒वे॒ ते॒ ते॒ हु॒वे॒ स॒व॒ । \newline
49. हु॒वे॒ स॒व॒ स॒व॒ हु॒वे॒ हु॒वे॒ स॒वा॒ह म॒हꣳ स॑व हुवे हुवे सवा॒हम् । \newline
50. स॒वा॒ह म॒हꣳ स॑व सवा॒हं ॅय॒ज्ञ्ं ॅय॒ज्ञ् म॒हꣳ स॑व सवा॒हं ॅय॒ज्ञ्म् । \newline
51. अ॒हं ॅय॒ज्ञ्ं ॅय॒ज्ञ् म॒ह म॒हं ॅय॒ज्ञ् मा य॒ज्ञ् म॒ह म॒हं ॅय॒ज्ञ् मा । \newline
52. य॒ज्ञ् मा य॒ज्ञ्ं ॅय॒ज्ञ् मा द॑दे दद॒ आ य॒ज्ञ्ं ॅय॒ज्ञ् मा द॑दे । \newline
53. आ द॑दे दद॒ आ द॑दे ते ते दद॒ आ द॑दे ते । \newline
54. द॒दे॒ ते॒ ते॒ द॒दे॒ द॒दे॒ ते॒ वष॑ट्कृतं॒ ॅवष॑ट्कृतम् ते ददे ददे ते॒ वष॑ट्कृतम् । \newline
55. ते॒ वष॑ट्कृतं॒ ॅवष॑ट्कृतम् ते ते॒ वष॑ट्कृतम् । \newline
56. वष॑ट्कृत॒मिति॒ वष॑ट् - कृ॒त॒म् । \newline
57. स्तु॒तꣳ श॒स्त्रꣳ श॒स्त्रꣳ स्तु॒तꣳ स्तु॒तꣳ श॒स्त्रम् प्र॑तिग॒रम् प्र॑तिग॒रꣳ श॒स्त्रꣳ स्तु॒तꣳ स्तु॒तꣳ श॒स्त्रम् प्र॑तिग॒रम् । \newline
58. श॒स्त्रम् प्र॑तिग॒रम् प्र॑तिग॒रꣳ श॒स्त्रꣳ श॒स्त्रम् प्र॑तिग॒रम् ग्रह॒म् ग्रह॑म् प्रतिग॒रꣳ श॒स्त्रꣳ श॒स्त्रम् प्र॑तिग॒रम् ग्रह᳚म् । \newline
59. प्र॒ति॒ग॒रम् ग्रह॒म् ग्रह॑म् प्रतिग॒रम् प्र॑तिग॒रम् ग्रह॒ मिडा॒ मिडा॒म् ग्रह॑म् प्रतिग॒रम् प्र॑तिग॒रम् ग्रह॒ मिडा᳚म् । \newline
60. प्र॒ति॒ग॒रमिति॑ प्रति - ग॒रम् । \newline
61. ग्रह॒ मिडा॒ मिडा॒म् ग्रह॒म् ग्रह॒ मिडा॑ मा॒शिष॑ आ॒शिष॒ इडा॒म् ग्रह॒म् ग्रह॒ मिडा॑ मा॒शिषः॑ । \newline
62. इडा॑ मा॒शिष॑ आ॒शिष॒ इडा॒ मिडा॑ मा॒शिष॒ आ ऽऽशिष॒ इडा॒ मिडा॑ मा॒शिष॒ आ । \newline
63. आ॒शिष॒ आ ऽऽशिष॑ आ॒शिष॒ आ वृ॑ञ्जे वृञ्ज॒ आ ऽऽशिष॑ आ॒शिष॒ आ वृ॑ञ्जे । \newline
64. आ॒शिष॒ इत्या᳚ - शिषः॑ । \newline
\pagebreak
\markright{ TS 7.3.11.3  \hfill https://www.vedavms.in \hfill}

\section{ TS 7.3.11.3 }

\textbf{TS 7.3.11.3 } \newline
\textbf{Samhita Paata} \newline

आ वृ॑ञ्जे॒ सुवः॑ । प॒त्नी॒सं॒ॅया॒जानुप॑ ते हुवे सवा॒हꣳ स॑मिष्टय॒जुरा द॑दे॒ तव॑ ॥ प॒शून्थ्-सु॒तं पु॑रो॒डाशा॒न्थ्-सव॑ना॒न्योत य॒ज्ञ्ं । दे॒वान्थ्-सेन्द्रा॒नुप॑ ते हुवे सवा॒ह-म॒ग्निमु॑खा॒न्थ्-सोम॑वतो॒ ये च॒ विश्वे᳚ ॥ \newline

\textbf{Pada Paata} \newline

एति॑ । वृ॒ञ्जे॒ । सुवः॑ ॥ प॒त्नी॒सं॒ॅया॒जानिति॑ पत्नी-सं॒ॅया॒जान् । उपेति॑ । ते॒ । हु॒वे॒ । स॒व॒ । अ॒हम् । स॒मि॒ष्ट॒य॒जुरिति॑ समिष्ट - य॒जुः । एति॑ । द॒दे॒ । तव॑ ॥ प॒शून् । सु॒तम् । पु॒रो॒डाशान्॑ । सव॑नानि । एति॑ । उ॒त । य॒ज्ञ्म् ॥ दे॒वान् । सेन्द्रा॒निति॒ स-इ॒न्द्रा॒न् । उपेति॑ । ते॒ । हु॒वे॒ । स॒व॒ । अ॒हम् । अ॒ग्निमु॑खा॒नित्य॒ग्नि - मु॒खा॒न् । सोम॑वत॒ इति॒ सोम॑-व॒तः॒ । ये । च॒ । विश्वे᳚ ॥  \newline


\textbf{Krama Paata} \newline

आ वृ॑ञ्जे । वृ॒ञ्जे॒ सुवः॑ । सुव॒रिति॒ सुवः॑ ॥ प॒त्नी॒स॒म्ॅया॒जानुप॑ । प॒त्नी॒स॒म्ॅया॒जानिति॑ पत्नी - स॒म्ॅया॒जान् । उप॑ ते । ते॒ हु॒वे॒ । हु॒वे॒ स॒व॒ । स॒वा॒हम् । अ॒हꣳ स॑मिष्टय॒जुः । स॒मि॒ष्ट॒य॒जुरा । स॒मि॒ष्ट॒य॒जुरिति॑ समिष्ट - य॒जुः । आ द॑दे । द॒दे॒ तव॑ । तवेति॒ तव॑ ॥ प॒शून्थ् सु॒तम् । सु॒तम् पु॑रो॒डाशान्॑ । पु॒रो॒डाशा॒न्थ् सव॑नानि । सव॑ना॒न्या । ओत । उ॒त य॒ज्ञ्म् । य॒ज्ञ्मिति॑ य॒ज्ञ्म् ॥ दे॒वान्थ् सेन्द्रान्॑ । सेन्द्रा॒नुप॑ । सेन्द्रा॒निति॒ स - इ॒न्द्रा॒न्॒ । उप॑ ते । ते॒ हु॒वे॒ । हु॒वे॒ स॒व॒ । स॒वा॒हम् । अ॒हम॒ग्निमु॑खान् । अ॒ग्निमु॑खा॒न्थ् सोम॑वतः । अ॒ग्निमु॑खा॒नित्य॒ग्नि - मु॒खा॒न्॒ । सोम॑वतो॒ ये । सोम॑वत॒ इति॒ सोम॑ - व॒तः॒ । ये च॑ । च॒ विश्वे᳚ । विश्व॒ इति॒ विश्वे᳚ । \newline

\textbf{Jatai Paata} \newline

1. आ वृ॑ञ्जे वृञ्ज॒ आ वृ॑ञ्जे । \newline
2. वृ॒ञ्जे॒ सुवः॒ सुव॑र् वृञ्जे वृञ्जे॒ सुवः॑ । \newline
3. सुव॒रिति॒ सुवः॑ । \newline
4. प॒त्नी॒सं॒ॅया॒जा नुपोप॑ पत्नीसंॅया॒जान् प॑त्नीसंॅया॒जा नुप॑ । \newline
5. प॒त्नी॒सं॒ॅया॒जानिति॑ पत्नी - सं॒ॅया॒जान् । \newline
6. उप॑ ते त॒ उपोप॑ ते । \newline
7. ते॒ हु॒वे॒ हु॒वे॒ ते॒ ते॒ हु॒वे॒ । \newline
8. हु॒वे॒ स॒व॒ स॒व॒ हु॒वे॒ हु॒वे॒ स॒व॒ । \newline
9. स॒वा॒ह म॒हꣳ स॑व सवा॒हम् । \newline
10. अ॒हꣳ स॑मिष्टय॒जुः स॑मिष्टय॒जु र॒ह म॒हꣳ स॑मिष्टय॒जुः । \newline
11. स॒मि॒ष्ट॒य॒जुरा स॑मिष्टय॒जुः स॑मिष्टय॒जुरा । \newline
12. स॒मि॒ष्ट॒य॒जुरिति॑ समिष्ट - य॒जुः । \newline
13. आ द॑दे दद॒ आ द॑दे । \newline
14. द॒दे॒ तव॒ तव॑ ददे ददे॒ तव॑ । \newline
15. तवेति॒ तव॑ । \newline
16. प॒शून् थ्सु॒तꣳ सु॒तम् प॒शून् प॒शून् थ्सु॒तम् । \newline
17. सु॒तम् पु॑रो॒डाशा᳚न् पुरो॒डाशा᳚न् थ्सु॒तꣳ सु॒तम् पु॑रो॒डाशान्॑ । \newline
18. पु॒रो॒डाशा॒न् थ्सव॑नानि॒ सव॑नानि पुरो॒डाशा᳚न् पुरो॒डाशा॒न् थ्सव॑नानि । \newline
19. सव॑ना॒न्या सव॑नानि॒ सव॑ना॒न्या । \newline
20. ओतो तो त । \newline
21. उ॒त य॒ज्ञ्ं ॅय॒ज्ञ् मु॒तोत य॒ज्ञ्म् । \newline
22. य॒ज्ञ्मिति॑ य॒ज्ञ्म् । \newline
23. दे॒वान् थ्सेन्द्रा॒न् थ्सेन्द्रा᳚न् दे॒वान् दे॒वान् थ्सेन्द्रान्॑ । \newline
24. सेन्द्रा॒ नुपोप॒ सेन्द्रा॒न् थ्सेन्द्रा॒ नुप॑ । \newline
25. सेन्द्रा॒निति॒ स - इ॒न्द्रा॒न् । \newline
26. उप॑ ते त॒ उपोप॑ ते । \newline
27. ते॒ हु॒वे॒ हु॒वे॒ ते॒ ते॒ हु॒वे॒ । \newline
28. हु॒वे॒ स॒व॒ स॒व॒ हु॒वे॒ हु॒वे॒ स॒व॒ । \newline
29. स॒वा॒ह म॒हꣳ स॑व सवा॒हम् । \newline
30. अ॒ह म॒ग्निमु॑खा न॒ग्निमु॑खा न॒ह म॒ह म॒ग्निमु॑खान् । \newline
31. अ॒ग्निमु॑खा॒न् थ्सोम॑वतः॒ सोम॑वतो॒ ऽग्निमु॑खा न॒ग्निमु॑खा॒न् थ्सोम॑वतः । \newline
32. अ॒ग्निमु॑खा॒नित्य॒ग्नि - मु॒खा॒न् । \newline
33. सोम॑वतो॒ ये ये सोम॑वतः॒ सोम॑वतो॒ ये । \newline
34. सोम॑वत॒ इति॒ सोम॑ - व॒तः॒ । \newline
35. ये च॑ च॒ ये ये च॑ । \newline
36. च॒ विश्वे॒ विश्वे॑ च च॒ विश्वे᳚ । \newline
37. विश्व॒ इति॒ विश्वे᳚ । \newline

\textbf{Ghana Paata } \newline

1. आ वृ॑ञ्जे वृञ्ज॒ आ वृ॑ञ्जे॒ सुवः॒ सुव॑र् वृञ्ज॒ आ वृ॑ञ्जे॒ सुवः॑ । \newline
2. वृ॒ञ्जे॒ सुवः॒ सुव॑र् वृञ्जे वृञ्जे॒ सुवः॑ । \newline
3. सुव॒रिति॒ सुवः॑ । \newline
4. प॒त्नी॒सं॒ॅया॒जा नुपोप॑ पत्नीसंॅया॒जान् प॑त्नीसंॅया॒जा नुप॑ ते त॒ उप॑ पत्नीसंॅया॒जान् प॑त्नीसंॅया॒जा नुप॑ ते । \newline
5. प॒त्नी॒सं॒ॅया॒जानिति॑ पत्नी - सं॒ॅया॒जान् । \newline
6. उप॑ ते त॒ उपोप॑ ते हुवे हुवे त॒ उपोप॑ ते हुवे । \newline
7. ते॒ हु॒वे॒ हु॒वे॒ ते॒ ते॒ हु॒वे॒ स॒व॒ स॒व॒ हु॒वे॒ ते॒ ते॒ हु॒वे॒ स॒व॒ । \newline
8. हु॒वे॒ स॒व॒ स॒व॒ हु॒वे॒ हु॒वे॒ स॒वा॒ह म॒हꣳ स॑व हुवे हुवे सवा॒हम् । \newline
9. स॒वा॒ह म॒हꣳ स॑व सवा॒हꣳ स॑मिष्टय॒जुः स॑मिष्टय॒जु र॒हꣳ स॑व सवा॒हꣳ स॑मिष्टय॒जुः । \newline
10. अ॒हꣳ स॑मिष्टय॒जुः स॑मिष्टय॒जु र॒ह म॒हꣳ स॑मिष्टय॒जुरा स॑मिष्टय॒जु र॒ह म॒हꣳ स॑मिष्टय॒जुरा । \newline
11. स॒मि॒ष्ट॒य॒जुरा स॑मिष्टय॒जुः स॑मिष्टय॒जुरा द॑दे दद॒ आ स॑मिष्टय॒जुः स॑मिष्टय॒जुरा द॑दे । \newline
12. स॒मि॒ष्ट॒य॒जुरिति॑ समिष्ट - य॒जुः । \newline
13. आ द॑दे दद॒ आ द॑दे॒ तव॒ तव॑ दद॒ आ द॑दे॒ तव॑ । \newline
14. द॒दे॒ तव॒ तव॑ ददे ददे॒ तव॑ । \newline
15. तवेति॒ तव॑ । \newline
16. प॒शून् थ्सु॒तꣳ सु॒तम् प॒शून् प॒शून् थ्सु॒तम् पु॑रो॒डाशा᳚न् पुरो॒डाशा᳚न् थ्सु॒तम् प॒शून् प॒शून् थ्सु॒तम् पु॑रो॒डाशान्॑ । \newline
17. सु॒तम् पु॑रो॒डाशा᳚न् पुरो॒डाशा᳚न् थ्सु॒तꣳ सु॒तम् पु॑रो॒डाशा॒न् थ्सव॑नानि॒ सव॑नानि पुरो॒डाशा᳚न् थ्सु॒तꣳ सु॒तम् पु॑रो॒डाशा॒न् थ्सव॑नानि । \newline
18. पु॒रो॒डाशा॒न् थ्सव॑नानि॒ सव॑नानि पुरो॒डाशा᳚न् पुरो॒डाशा॒न् थ्सव॑ना॒न्या सव॑नानि पुरो॒डाशा᳚न् पुरो॒डाशा॒न् थ्सव॑ना॒न्या । \newline
19. सव॑ना॒न्या सव॑नानि॒ सव॑ना॒ न्योतोता सव॑नानि॒ सव॑ना॒न्योत । \newline
20. ओतोतोत य॒ज्ञ्ं ॅय॒ज्ञ् मु॒तोत य॒ज्ञ्म् । \newline
21. उ॒त य॒ज्ञ्ं ॅय॒ज्ञ् मु॒तोत य॒ज्ञ्म् । \newline
22. य॒ज्ञ्मिति॑ य॒ज्ञ्म् । \newline
23. दे॒वान् थ्सेन्द्रा॒न् थ्सेन्द्रा᳚न् दे॒वान् दे॒वान् थ्सेन्द्रा॒ नुपोप॒ सेन्द्रा᳚न् दे॒वान् दे॒वान् थ्सेन्द्रा॒ नुप॑ । \newline
24. सेन्द्रा॒ नुपोप॒ सेन्द्रा॒न् थ्सेन्द्रा॒ नुप॑ ते त॒ उप॒ सेन्द्रा॒न् थ्सेन्द्रा॒ नुप॑ ते । \newline
25. सेन्द्रा॒निति॒ स - इ॒न्द्रा॒न् । \newline
26. उप॑ ते त॒ उपोप॑ ते हुवे हुवे त॒ उपोप॑ ते हुवे । \newline
27. ते॒ हु॒वे॒ हु॒वे॒ ते॒ ते॒ हु॒वे॒ स॒व॒ स॒व॒ हु॒वे॒ ते॒ ते॒ हु॒वे॒ स॒व॒ । \newline
28. हु॒वे॒ स॒व॒ स॒व॒ हु॒वे॒ हु॒वे॒ स॒वा॒ह म॒हꣳ स॑व हुवे हुवे सवा॒हम् । \newline
29. स॒वा॒ह म॒हꣳ स॑व सवा॒ह म॒ग्निमु॑खा न॒ग्निमु॑खा न॒हꣳ स॑व सवा॒ह म॒ग्निमु॑खान् । \newline
30. अ॒ह म॒ग्निमु॑खा न॒ग्निमु॑खा न॒ह म॒ह म॒ग्निमु॑खा॒न् थ्सोम॑वतः॒ सोम॑वतो॒ ऽग्निमु॑खा न॒ह म॒ह म॒ग्निमु॑खा॒न् थ्सोम॑वतः । \newline
31. अ॒ग्निमु॑खा॒न् थ्सोम॑वतः॒ सोम॑वतो॒ ऽग्निमु॑खा न॒ग्निमु॑खा॒न् थ्सोम॑वतो॒ ये ये सोम॑वतो॒ ऽग्निमु॑खा न॒ग्निमु॑खा॒न् थ्सोम॑वतो॒ ये । \newline
32. अ॒ग्निमु॑खा॒नित्य॒ग्नि - मु॒खा॒न् । \newline
33. सोम॑वतो॒ ये ये सोम॑वतः॒ सोम॑वतो॒ ये च॑ च॒ ये सोम॑वतः॒ सोम॑वतो॒ ये च॑ । \newline
34. सोम॑वत॒ इति॒ सोम॑ - व॒तः॒ । \newline
35. ये च॑ च॒ ये ये च॒ विश्वे॒ विश्वे॑ च॒ ये ये च॒ विश्वे᳚ । \newline
36. च॒ विश्वे॒ विश्वे॑ च च॒ विश्वे᳚ । \newline
37. विश्व॒ इति॒ विश्वे᳚ । \newline
\pagebreak
\markright{ TS 7.3.12.1  \hfill https://www.vedavms.in \hfill}

\section{ TS 7.3.12.1 }

\textbf{TS 7.3.12.1 } \newline
\textbf{Samhita Paata} \newline

भू॒तं भव्यं॑ भवि॒ष्यद्वष॒ट्थ् स्वाहा॒ नम॒ ऋख् साम॒ यजु॒र्वष॒ट्थ् स्वाहा॒ नमो॑ गाय॒त्री त्रि॒ष्टुब् जग॑ती॒ वष॒ट्थ् स्वाहा॒ नमः॑ पृथि॒व्य॑न्तरि॑क्षं॒ द्यौ र्वष॒ट्थ् स्वाहा॒ नमो॒ ऽग्निर्वा॒युः सूर्यो॒ वष॒ट्थ् स्वाहा॒ नमः॑ प्रा॒णो-व्या॒नो॑-ऽपा॒नो वष॒ट्थ् स्वाहा॒ नमो ऽन्नं॑ कृ॒षि-र्वृष्टि॒-र्वष॒ट्थ् स्वाहा॒ नमः॑ पि॒तापु॒त्रः पौत्रो॒ वष॒ट्थ् स्वाहा॒ नमो॒ भूर्भुवः॒ ( ) सुव॒ र्वष॒ट्थ् स्वाहा॒ नमः॑ ॥ \newline

\textbf{Pada Paata} \newline

भू॒तम् । भव्य᳚म् । भ॒वि॒ष्यत् । वष॑ट् । स्वाहा᳚ । नमः॑ । ऋक् । साम॑ । यजुः॑ । वष॑ट् । स्वाहा᳚ । नमः॑ । गा॒य॒त्री । त्रि॒ष्टुप् । जग॑ती । वष॑ट् । स्वाहा᳚ । नमः॑ । पृ॒थिवी । अ॒न्तरि॑क्षम् । द्यौः । वष॑ट् । स्वाहा᳚ । नमः॑ । अ॒ग्निः । वा॒युः । सूर्यः॑ । वष॑ट् । स्वाहा᳚ । नमः॑ । प्रा॒ण इति॑ प्र - अ॒नः । व्या॒न इति॑ वि - अ॒नः । अ॒पा॒न इत्य॑प-अ॒नः । वष॑ट् । स्वाहा᳚ । नमः॑ । अन्न᳚म् । कृ॒षिः । वृष्टिः॑ । वष॑ट् । स्वाहा᳚ । नमः॑ । पि॒ता । पु॒त्रः । पौत्रः॑ । वष॑ट् । स्वाहा᳚ । नमः॑ । भूः । भुवः॑ ( ) । सुवः॑ । वष॑ट् । स्वाहा᳚ । नमः॑ ॥  \newline


\textbf{Krama Paata} \newline

भू॒तम् भव्य᳚म् । भव्य॑म् भवि॒ष्यत् । भ॒वि॒ष्यद् वष॑ट् । वष॒ट्थ् स्वाहा᳚ । स्वाहा॒ नमः॑ । नम॒ ऋक् । ऋख् साम॑ । साम॒ यजुः॑ । यजु॒र् वष॑ट् । वष॒ट्थ् स्वाहा᳚ । स्वाहा॒ नमः॑ । नमो॑ गाय॒त्री । गा॒य॒त्री त्रि॒ष्टुप् । त्रि॒ष्टुब् जग॑ती । जग॑ती॒ वष॑ट् । वष॒ट्थ् स्वाहा᳚ । स्वाहा॒ नमः॑ । नमः॑ पृथि॒वी । पृ॒थि॒व्य॑न्तरि॑क्षम् । अ॒न्तरि॑क्ष॒म् द्यौः । द्यौर् वष॑ट् । वष॒ट्थ् स्वाहा᳚ । स्वाहा॒ नमः॑ । नमो॒ऽग्निः । अ॒ग्निर् वा॒युः । वा॒युः सूर्यः॑ । सूर्यो॒ वष॑ट् । वष॒ट्थ् स्वाहा᳚ । स्वाहा॒ नमः॑ । नमः॑ प्रा॒णः । प्रा॒णो व्या॒नः । प्रा॒ण इति॑ प्र - अ॒नः । व्या॒नो॑ऽपा॒नः । व्या॒न इति॑ वि - अ॒नः । अ॒पा॒नो वष॑ट् । अ॒पा॒न इत्य॑प - अ॒नः । वष॒ट्थ् स्वाहा᳚ । स्वाहा॒ नमः॑ । नमोऽन्न᳚म् । अन्न॑म् कृ॒षिः । कृ॒षिर् वृष्टिः॑ । वृष्टि॒र् वष॑ट् । वष॒ट्थ् स्वाहा᳚ । स्वाहा॒ नमः॑ । नमः॑ पि॒ता । पि॒ता पु॒त्रः । पु॒त्रः पौत्रः॑ । पौत्रो॒ वष॑ट् । वष॒ट्थ् स्वाहा᳚ । स्वाहा॒ नमः॑ । नमो॒ भूः । भूर् भुवः॑ ( ) । भुवः॒ सुवः॑ । सुव॒र् वष॑ट् । वष॒ट्थ् स्वाहा᳚ । स्वाहा॒ नमः॑ । नम॒ इति॒ नमः॑ । \newline

\textbf{Jatai Paata} \newline

1. भू॒तम् भव्य॒म् भव्य॑म् भू॒तम् भू॒तम् भव्य᳚म् । \newline
2. भव्य॑म् भवि॒ष्यद् भ॑वि॒ष्यद् भव्य॒म् भव्य॑म् भवि॒ष्यत् । \newline
3. भ॒वि॒ष्यद् वष॒ड् वष॑ड् भवि॒ष्यद् भ॑वि॒ष्यद् वष॑ट् । \newline
4. वष॒ट् थ्स्वाहा॒ स्वाहा॒ वष॒ड् वष॒ट् थ्स्वाहा᳚ । \newline
5. स्वाहा॒ नमो॒ नमः॒ स्वाहा॒ स्वाहा॒ नमः॑ । \newline
6. नम॒ ऋगृङ् नमो॒ नम॒ ऋक् । \newline
7. ऋख् साम॒ साम र्‌ग् ऋख् साम॑ । \newline
8. साम॒ यजु॒र् यजुः॒ साम॒ साम॒ यजुः॑ । \newline
9. यजु॒र् वष॒ड् वष॒ड् यजु॒र् यजु॒र् वष॑ट् । \newline
10. वष॒ट् थ्स्वाहा॒ स्वाहा॒ वष॒ड् वष॒ट् थ्स्वाहा᳚ । \newline
11. स्वाहा॒ नमो॒ नमः॒ स्वाहा॒ स्वाहा॒ नमः॑ । \newline
12. नमो॑ गाय॒त्री गा॑य॒त्री नमो॒ नमो॑ गाय॒त्री । \newline
13. गा॒य॒त्री त्रि॒ष्टुप् त्रि॒ष्टुब् गा॑य॒त्री गा॑य॒त्री त्रि॒ष्टुप् । \newline
14. त्रि॒ष्टुब् जग॑ती॒ जग॑ती त्रि॒ष्टुप् त्रि॒ष्टुब् जग॑ती । \newline
15. जग॑ती॒ वष॒ड् वष॒ड् जग॑ती॒ जग॑ती॒ वष॑ट् । \newline
16. वष॒ट् थ्स्वाहा॒ स्वाहा॒ वष॒ड् वष॒ट् थ्वाहा᳚ । \newline
17. स्वाहा॒ नमो॒ नमः॒ स्वाहा॒ स्वाहा॒ नमः॑ । \newline
18. नमः॑ पृथि॒वी पृ॑थि॒वी नमो॒ नमः॑ पृथि॒वी । \newline
19. पृ॒थि॒ व्य॑न्तरि॑क्ष म॒न्तरि॑क्षम् पृथि॒वी पृ॑थि॒ व्य॑न्तरि॑क्षम् । \newline
20. अ॒न्तरि॑क्ष॒म् द्यौर् द्यौ र॒न्तरि॑क्ष म॒न्तरि॑क्ष॒म् द्यौः । \newline
21. द्यौर् वष॒ड् वष॒ड् द्यौर् द्यौर् वष॑ट् । \newline
22. वष॒ट् थ्स्वाहा॒ स्वाहा॒ वष॒ड् वष॒ट् थ्स्वाहा᳚ । \newline
23. स्वाहा॒ नमो॒ नमः॒ स्वाहा॒ स्वाहा॒ नमः॑ । \newline
24. नमो॒ ऽग्नि र॒ग्निर् नमो॒ नमो॒ ऽग्निः । \newline
25. अ॒ग्निर् वा॒युर् वा॒यु र॒ग्नि र॒ग्निर् वा॒युः । \newline
26. वा॒युः सूर्यः॒ सूर्यो॑ वा॒युर् वा॒युः सूर्यः॑ । \newline
27. सूर्यो॒ वष॒ड् वष॒ट् थ्सूर्यः॒ सूर्यो॒ वष॑ट् । \newline
28. वष॒ट् थ्स्वाहा॒ स्वाहा॒ वष॒ड् वष॒ट् थ्स्वाहा᳚ । \newline
29. स्वाहा॒ नमो॒ नमः॒ स्वाहा॒ स्वाहा॒ नमः॑ । \newline
30. नमः॑ प्रा॒णः प्रा॒णो नमो॒ नमः॑ प्रा॒णः । \newline
31. प्रा॒णो व्या॒नो व्या॒नः प्रा॒णः प्रा॒णो व्या॒नः । \newline
32. प्रा॒ण इति॑ प्र - अ॒नः । \newline
33. व्या॒नो॑ ऽपा॒नो॑ ऽपा॒नो व्या॒नो व्या॒नो॑ ऽपा॒नः । \newline
34. व्या॒न इति॑ वि - अ॒नः । \newline
35. अ॒पा॒नो वष॒ड् वष॑ डपा॒नो॑ ऽपा॒नो वष॑ट् । \newline
36. अ॒पा॒न इत्य॑प - अ॒नः । \newline
37. वष॒ट् थ्स्वाहा॒ स्वाहा॒ वष॒ड् वष॒ट् थ्स्वाहा᳚ । \newline
38. स्वाहा॒ नमो॒ नमः॒ स्वाहा॒ स्वाहा॒ नमः॑ । \newline
39. नमो ऽन्न॒ मन्न॒न् नमो॒ नमो ऽन्न᳚म् । \newline
40. अन्न॑म् कृ॒षिः कृ॒षि रन्न॒ मन्न॑म् कृ॒षिः । \newline
41. कृ॒षिर् वृष्टि॒र् वृष्टिः॑ कृ॒षिः कृ॒षिर् वृष्टिः॑ । \newline
42. वृष्टि॒र् वष॒ड् वष॒ड् वृष्टि॒र् वृष्टि॒र् वष॑ट् । \newline
43. वष॒ट् थ्स्वाहा॒ स्वाहा॒ वष॒ड् वष॒ट् थ्स्वाहा᳚ । \newline
44. स्वाहा॒ नमो॒ नमः॒ स्वाहा॒ स्वाहा॒ नमः॑ । \newline
45. नमः॑ पि॒ता पि॒ता नमो॒ नमः॑ पि॒ता । \newline
46. पि॒ता पु॒त्रः पु॒त्रः पि॒ता पि॒ता पु॒त्रः । \newline
47. पु॒त्रः पौत्रः॒ पौत्रः॑ पु॒त्रः पु॒त्रः पौत्रः॑ । \newline
48. पौत्रो॒ वष॒ड् वष॒ट् पौत्रः॒ पौत्रो॒ वष॑ट् । \newline
49. वष॒ट् थ्स्वाहा॒ स्वाहा॒ वष॒ड् वष॒ट् थ्स्वाहा᳚ । \newline
50. स्वाहा॒ नमो॒ नमः॒ स्वाहा॒ स्वाहा॒ नमः॑ । \newline
51. नमो॒ भूर् भूर् नमो॒ नमो॒ भूः । \newline
52. भूर् भुवो॒ भुवो॒ भूर् भूर् भुवः॑ । \newline
53. भुवः॒ सुवः॒ सुव॒र् भुवो॒ भुवः॒ सुवः॑ । \newline
54. सुव॒र् वष॒ड् वष॒ट् थ्सुवः॒ सुव॒र् वष॑ट् । \newline
55. वष॒ट् थ्स्वाहा॒ स्वाहा॒ वष॒ड् वष॒ट् थ्स्वाहा᳚ । \newline
56. स्वाहा॒ नमो॒ नमः॒ स्वाहा॒ स्वाहा॒ नमः॑ । \newline
57. नम॒ इति॒ नमः॑ । \newline

\textbf{Ghana Paata } \newline

1. भू॒तम् भव्य॒म् भव्य॑म् भू॒तम् भू॒तम् भव्य॑म् भवि॒ष्यद् भ॑वि॒ष्यद् भव्य॑म् भू॒तम् भू॒तम् भव्य॑म् भवि॒ष्यत् । \newline
2. भव्य॑म् भवि॒ष्यद् भ॑वि॒ष्यद् भव्य॒म् भव्य॑म् भवि॒ष्यद् वष॒ड् वष॑ड् भवि॒ष्यद् भव्य॒म् भव्य॑म् भवि॒ष्यद् वष॑ट् । \newline
3. भ॒वि॒ष्यद् वष॒ड् वष॑ड् भवि॒ष्यद् भ॑वि॒ष्यद् वष॒ट् थ्स्वाहा॒ स्वाहा॒ वष॑ड् भवि॒ष्यद् भ॑वि॒ष्यद् वष॒ट् थ्स्वाहा᳚ । \newline
4. वष॒ट् थ्स्वाहा॒ स्वाहा॒ वष॒ड् वष॒ट् थ्स्वाहा॒ नमो॒ नमः॒ स्वाहा॒ वष॒ड् वष॒ट् थ्स्वाहा॒ नमः॑ । \newline
5. स्वाहा॒ नमो॒ नमः॒ स्वाहा॒ स्वाहा॒ नम॒ ऋग् ऋङ् नमः॒ स्वाहा॒ स्वाहा॒ नम॒ ऋक् । \newline
6. नम॒ ऋग् ऋङ् नमो॒ नम॒ ऋख् साम॒ साम र्‌ङ् नमो॒ नम॒ ऋख् साम॑ । \newline
7. ऋख् साम॒ साम र्‌गृख् साम॒ यजु॒र् यजुः॒ साम र्‌गृख् साम॒ यजुः॑ । \newline
8. साम॒ यजु॒र् यजुः॒ साम॒ साम॒ यजु॒र् वष॒ड् वष॒ड् यजुः॒ साम॒ साम॒ यजु॒र् वष॑ट् । \newline
9. यजु॒र् वष॒ड् वष॒ड् यजु॒र् यजु॒र् वष॒ट् थ्स्वाहा॒ स्वाहा॒ वष॒ड् यजु॒र् यजु॒र् वष॒ट् थ्स्वाहा᳚ । \newline
10. वष॒ट् थ्स्वाहा॒ स्वाहा॒ वष॒ड् वष॒ट् थ्स्वाहा॒ नमो॒ नमः॒ स्वाहा॒ वष॒ड् वष॒ट् थ्स्वाहा॒ नमः॑ । \newline
11. स्वाहा॒ नमो॒ नमः॒ स्वाहा॒ स्वाहा॒ नमो॑ गाय॒त्री गा॑य॒त्री नमः॒ स्वाहा॒ स्वाहा॒ नमो॑ गाय॒त्री । \newline
12. नमो॑ गाय॒त्री गा॑य॒त्री नमो॒ नमो॑ गाय॒त्री त्रि॒ष्टुप् त्रि॒ष्टुब् गा॑य॒त्री नमो॒ नमो॑ गाय॒त्री त्रि॒ष्टुप् । \newline
13. गा॒य॒त्री त्रि॒ष्टुप् त्रि॒ष्टुब् गा॑य॒त्री गा॑य॒त्री त्रि॒ष्टुब् जग॑ती॒ जग॑ती त्रि॒ष्टुब् गा॑य॒त्री गा॑य॒त्री त्रि॒ष्टुब् जग॑ती । \newline
14. त्रि॒ष्टुब् जग॑ती॒ जग॑ती त्रि॒ष्टुप् त्रि॒ष्टुब् जग॑ती॒ वष॒ड् वष॒ड् जग॑ती त्रि॒ष्टुप् त्रि॒ष्टुब् जग॑ती॒ वष॑ट् । \newline
15. जग॑ती॒ वष॒ड् वष॒ड् जग॑ती॒ जग॑ती॒ वष॒ट् थ्स्वाहा॒ स्वाहा॒ वष॒ड् जग॑ती॒ जग॑ती॒ वष॒ट् थ्स्वाहा᳚ । \newline
16. वष॒ट् थ्स्वाहा॒ स्वाहा॒ वष॒ड् वष॒ट् थ्स्वाहा॒ नमो॒ नमः॒ स्वाहा॒ वष॒ड् वष॒ट् थ्स्वाहा॒ नमः॑ । \newline
17. स्वाहा॒ नमो॒ नमः॒ स्वाहा॒ स्वाहा॒ नमः॑ पृथि॒वी पृ॑थि॒वी नमः॒ स्वाहा॒ स्वाहा॒ नमः॑ पृथि॒वी । \newline
18. नमः॑ पृथि॒वी पृ॑थि॒वी नमो॒ नमः॑ पृथि॒ व्य॑न्तरि॑क्ष म॒न्तरि॑क्षम् पृथि॒वी नमो॒ नमः॑ पृथि॒ व्य॑न्तरि॑क्षम् । \newline
19. पृ॒थि॒ व्य॑न्तरि॑क्ष म॒न्तरि॑क्षम् पृथि॒वी पृ॑थि॒ व्य॑न्तरि॑क्ष॒म् द्यौर् द्यौ र॒न्तरि॑क्षम् पृथि॒वी पृ॑थि॒ व्य॑न्तरि॑क्ष॒म् द्यौः । \newline
20. अ॒न्तरि॑क्ष॒म् द्यौर् द्यौ र॒न्तरि॑क्ष म॒न्तरि॑क्ष॒म् द्यौर् वष॒ड् वष॒ड् द्यौ र॒न्तरि॑क्ष म॒न्तरि॑क्ष॒म् द्यौर् वष॑ट् । \newline
21. द्यौर् वष॒ड् वष॒ड् द्यौर् द्यौर् वष॒ट् थ्स्वाहा॒ स्वाहा॒ वष॒ड् द्यौर् द्यौर् वष॒ट् थ्स्वाहा᳚ । \newline
22. वष॒ट् थ्स्वाहा॒ स्वाहा॒ वष॒ड् वष॒ट् थ्स्वाहा॒ नमो॒ नमः॒ स्वाहा॒ वष॒ड् वष॒ट् थ्स्वाहा॒ नमः॑ । \newline
23. स्वाहा॒ नमो॒ नमः॒ स्वाहा॒ स्वाहा॒ नमो॒ ऽग्नि र॒ग्निर् नमः॒ स्वाहा॒ स्वाहा॒ नमो॒ ऽग्निः । \newline
24. नमो॒ ऽग्नि र॒ग्निर् नमो॒ नमो॒ ऽग्निर् वा॒युर् वा॒यु र॒ग्निर् नमो॒ नमो॒ ऽग्निर् वा॒युः । \newline
25. अ॒ग्निर् वा॒युर् वा॒यु र॒ग्नि र॒ग्निर् वा॒युः सूर्यः॒ सूर्यो॑ वा॒यु र॒ग्नि र॒ग्निर् वा॒युः सूर्यः॑ । \newline
26. वा॒युः सूर्यः॒ सूर्यो॑ वा॒युर् वा॒युः सूर्यो॒ वष॒ड् वष॒ट् थ्सूर्यो॑ वा॒युर् वा॒युः सूर्यो॒ वष॑ट् । \newline
27. सूर्यो॒ वष॒ड् वष॒ट् थ्सूर्यः॒ सूर्यो॒ वष॒ट् थ्स्वाहा॒ स्वाहा॒ वष॒ट् थ्सूर्यः॒ सूर्यो॒ वष॒ट् थ्स्वाहा᳚ । \newline
28. वष॒ट् थ्स्वाहा॒ स्वाहा॒ वष॒ड् वष॒ट् थ्स्वाहा॒ नमो॒ नमः॒ स्वाहा॒ वष॒ड् वष॒ट् थ्स्वाहा॒ नमः॑ । \newline
29. स्वाहा॒ नमो॒ नमः॒ स्वाहा॒ स्वाहा॒ नमः॑ प्रा॒णः प्रा॒णो नमः॒ स्वाहा॒ स्वाहा॒ नमः॑ प्रा॒णः । \newline
30. नमः॑ प्रा॒णः प्रा॒णो नमो॒ नमः॑ प्रा॒णो व्या॒नो व्या॒नः प्रा॒णो नमो॒ नमः॑ प्रा॒णो व्या॒नः । \newline
31. प्रा॒णो व्या॒नो व्या॒नः प्रा॒णः प्रा॒णो व्या॒नो॑ ऽपा॒नो॑ ऽपा॒नो व्या॒नः प्रा॒णः प्रा॒णो व्या॒नो॑ ऽपा॒नः । \newline
32. प्रा॒ण इति॑ प्र - अ॒नः । \newline
33. व्या॒नो॑ ऽपा॒नो॑ ऽपा॒नो व्या॒नो व्या॒नो॑ ऽपा॒नो वष॒ड् वष॑ डपा॒नो व्या॒नो व्या॒नो॑ ऽपा॒नो वष॑ट् । \newline
34. व्या॒न इति॑ वि - अ॒नः । \newline
35. अ॒पा॒नो वष॒ड् वष॑ डपा॒नो॑ ऽपा॒नो वष॒ट् थ्स्वाहा॒ स्वाहा॒ वष॑ डपा॒नो॑ ऽपा॒नो वष॒ट् थ्स्वाहा᳚ । \newline
36. अ॒पा॒न इत्य॑प - अ॒नः । \newline
37. वष॒ट् थ्स्वाहा॒ स्वाहा॒ वष॒ड् वष॒ट् थ्स्वाहा॒ नमो॒ नमः॒ स्वाहा॒ वष॒ड् वष॒ट् थ्स्वाहा॒ नमः॑ । \newline
38. स्वाहा॒ नमो॒ नमः॒ स्वाहा॒ स्वाहा॒ नमो ऽन्न॒ मन्न॒न् नमः॒ स्वाहा॒ स्वाहा॒ नमो ऽन्न᳚म् । \newline
39. नमो ऽन्न॒ मन्न॒न् नमो॒ नमो ऽन्न॑म् कृ॒षिः कृ॒षि रन्न॒न् नमो॒ नमो ऽन्न॑म् कृ॒षिः । \newline
40. अन्न॑म् कृ॒षिः कृ॒षि रन्न॒ मन्न॑म् कृ॒षिर् वृष्टि॒र् वृष्टिः॑ कृ॒षि रन्न॒ मन्न॑म् कृ॒षिर् वृष्टिः॑ । \newline
41. कृ॒षिर् वृष्टि॒र् वृष्टिः॑ कृ॒षिः कृ॒षिर् वृष्टि॒र् वष॒ड् वष॒ड् वृष्टिः॑ कृ॒षिः कृ॒षिर् वृष्टि॒र् वष॑ट् । \newline
42. वृष्टि॒र् वष॒ड् वष॒ड् वृष्टि॒र् वृष्टि॒र् वष॒ट् थ्स्वाहा॒ स्वाहा॒ वष॒ड् वृष्टि॒र् वृष्टि॒र् वष॒ट् थ्स्वाहा᳚ । \newline
43. वष॒ट् थ्स्वाहा॒ स्वाहा॒ वष॒ड् वष॒ट् थ्स्वाहा॒ नमो॒ नमः॒ स्वाहा॒ वष॒ड् वष॒ट् थ्स्वाहा॒ नमः॑ । \newline
44. स्वाहा॒ नमो॒ नमः॒ स्वाहा॒ स्वाहा॒ नमः॑ पि॒ता पि॒ता नमः॒ स्वाहा॒ स्वाहा॒ नमः॑ पि॒ता । \newline
45. नमः॑ पि॒ता पि॒ता नमो॒ नमः॑ पि॒ता पु॒त्रः पु॒त्रः पि॒ता नमो॒ नमः॑ पि॒ता पु॒त्रः । \newline
46. पि॒ता पु॒त्रः पु॒त्रः पि॒ता पि॒ता पु॒त्रः पौत्रः॒ पौत्रः॑ पु॒त्रः पि॒ता पि॒ता पु॒त्रः पौत्रः॑ । \newline
47. पु॒त्रः पौत्रः॒ पौत्रः॑ पु॒त्रः पु॒त्रः पौत्रो॒ वष॒ड् वष॒ट् पौत्रः॑ पु॒त्रः पु॒त्रः पौत्रो॒ वष॑ट् । \newline
48. पौत्रो॒ वष॒ड् वष॒ट् पौत्रः॒ पौत्रो॒ वष॒ट् थ्स्वाहा॒ स्वाहा॒ वष॒ट् पौत्रः॒ पौत्रो॒ वष॒ट् थ्स्वाहा᳚ । \newline
49. वष॒ट् थ्स्वाहा॒ स्वाहा॒ वष॒ड् वष॒ट् थ्स्वाहा॒ नमो॒ नमः॒ स्वाहा॒ वष॒ड् वष॒ट् थ्स्वाहा॒ नमः॑ । \newline
50. स्वाहा॒ नमो॒ नमः॒ स्वाहा॒ स्वाहा॒ नमो॒ भूर् भूर् नमः॒ स्वाहा॒ स्वाहा॒ नमो॒ भूः । \newline
51. नमो॒ भूर् भूर् नमो॒ नमो॒ भूर् भुवो॒ भुवो॒ भूर् नमो॒ नमो॒ भूर् भुवः॑ । \newline
52. भूर् भुवो॒ भुवो॒ भूर् भूर् भुवः॒ सुवः॒ सुव॒र् भुवो॒ भूर् भूर् भुवः॒ सुवः॑ । \newline
53. भुवः॒ सुवः॒ सुव॒र् भुवो॒ भुवः॒ सुव॒र् वष॒ड् वष॒ट् थ्सुव॒र् भुवो॒ भुवः॒ सुव॒र् वष॑ट् । \newline
54. सुव॒र् वष॒ड् वष॒ट् थ्सुवः॒ सुव॒र् वष॒ट् थ्स्वाहा॒ स्वाहा॒ वष॒ट् थ्सुवः॒ सुव॒र् वष॒ट् थ्स्वाहा᳚ । \newline
55. वष॒ट् थ्स्वाहा॒ स्वाहा॒ वष॒ड् वष॒ट् थ्स्वाहा॒ नमो॒ नमः॒ स्वाहा॒ वष॒ड् वष॒ट् थ्स्वाहा॒ नमः॑ । \newline
56. स्वाहा॒ नमो॒ नमः॒ स्वाहा॒ स्वाहा॒ नमः॑ । \newline
57. नम॒ इति॒ नमः॑ । \newline
\pagebreak
\markright{ TS 7.3.13.1  \hfill https://www.vedavms.in \hfill}

\section{ TS 7.3.13.1 }

\textbf{TS 7.3.13.1 } \newline
\textbf{Samhita Paata} \newline

आ मे॑ गृ॒हा भ॑व॒न्त्वा प्र॒जा म॒ आ मा॑ य॒ज्ञो वि॑शतु वी॒र्या॑वान् । आपो॑ दे॒वीर्य॒ज्ञिया॒ माऽऽवि॑शन्तु स॒हस्र॑स्य मा भू॒मा मा प्र हा॑सीत् ॥ आ मे॒ ग्रहो॑ भव॒त्वा पु॑रो॒रुख् स्तु॑तश॒स्त्रे मा ऽऽ वि॑शताꣳ स॒मीची᳚ । आ॒दि॒त्या रु॒द्रा वस॑वो मे सद॒स्याः᳚ स॒हस्र॑स्य मा भू॒मा मा प्र हा॑सीत् ॥आ मा᳚ऽग्निष्टो॒मो वि॑शतू॒ ( ) क्थ्य॑श्चातिरा॒त्रो माऽऽ वि॑शत्वापिशर्व॒रः ।ति॒रोअ॑ह्निया मा॒ सुहु॑ता॒ आ वि॑शन्तु स॒हस्र॑स्य मा भू॒मा मा प्र हा॑सीत् ॥ \newline

\textbf{Pada Paata} \newline

एति॑ । मे॒ । गृ॒हाः । भ॒व॒न्तु॒ । एति॑ । प्र॒जेति॑ प्र - जा । मे॒ । एति॑ । मा॒ । य॒ज्ञ्ः । वि॒श॒तु॒ । वी॒र्या॑वा॒निति॑ वी॒र्य॑ - वा॒न् ॥ आपः॑ । दे॒वीः । य॒ज्ञियाः᳚ । मा॒ । एति॑ । वि॒श॒न्तु॒ । स॒हस्र॑स्य । मा॒ । भू॒मा । मा । प्रेति॑ । हा॒सी॒त् ॥ एति॑ । मे॒ । ग्रहः॑ । भ॒व॒तु॒ । एति॑ । पु॒रो॒रुगिति॑ पुरः - रुक् । स्तु॒त॒श॒स्त्रे इति॑ स्तुत - श॒स्त्रे । मा॒ । एति॑ । वि॒श॒ता॒म् । स॒मीची॒ इति॑ ॥ आ॒दि॒त्याः । रु॒द्राः । वस॑वः । मे॒ । स॒द॒स्याः᳚ । स॒हस्र॑स्य । मा॒ । भू॒मा । मा । प्रेति॑ । हा॒सी॒त् ॥ एति॑ । मा॒ । अ॒ग्नि॒ष्टो॒म इत्य॑ग्नि - स्तो॒मः । वि॒श॒तु॒ ( ) । उ॒क्थ्यः॑ । च॒ । अ॒ति॒रा॒त्र इत्य॑ति - रा॒त्रः । मा॒ । एति॑ । वि॒श॒तु॒ । आ॒पि॒श॒र्व॒र इत्या॑पि - श॒र्व॒रः ॥ ति॒रो‌अ॑ह्निया॒ इति॑ ति॒रः - अ॒ह्नि॒याः॒ । मा॒ । सुहु॑ता॒ इति॒ सु - हु॒ताः॒ । एति॑ । वि॒श॒न्तु॒ । स॒हस्र॑स्य । मा॒ । भू॒मा । मा । प्रेति॑ । हा॒सी॒त् ॥  \newline


\textbf{Krama Paata} \newline

आ मे᳚ । मे॒ गृ॒हाः । गृ॒हा भ॑वन्तु । भ॒व॒न्त्वा । आ प्र॒जा । प्र॒जा मे᳚ । प्र॒जेति॑ प्र - जा । म॒ आ । आ मा᳚ । मा॒ य॒ज्ञ्ः । य॒ज्ञो वि॑शतु । वि॒श॒तु॒ वी॒र्या॑वान् । वी॒र्या॑वा॒निति॑ वी॒र्य॑ - वा॒न्॒ ॥ आपो॑ दे॒वीः । दे॒वीर् य॒ज्ञियाः᳚ । य॒ज्ञिया॑ मा । 
मा । आ वि॑शन्तु । वि॒श॒न्तु॒ स॒हस्र॑स्य । स॒हस्र॑स्य मा । मा॒ भू॒मा । भू॒मा मा । मा प्र । प्र हा॑सीत् । हा॒सी॒दिति॑ हासीत् ॥ आ मे᳚ । मे॒ ग्रहः॑ । ग्रहो॑ भवतु । भ॒व॒त्वा । आ पु॑रो॒रुक् । पु॒रो॒रुख् स्तु॑तश॒स्त्रे । पु॒रो॒रुगिति॑ पुरः - रुक् । स्तु॒त॒श॒स्त्रे मा᳚ । स्तु॒त॒श॒स्त्रे इति॑ स्तुत - श॒स्त्रे । मा । आ वि॑शताम् । वि॒श॒ताꣳ॒॒ स॒मीची᳚ । स॒मीची॒ इति॑ स॒मीची᳚ ॥ आ॒दि॒त्या रु॒द्राः । रु॒द्रा वस॑वः । वस॑वो मे । मे॒ स॒द॒स्याः᳚ । स॒द॒स्याः᳚ स॒हस्र॑स्य । स॒हस्र॑स्य मा । मा॒ भू॒मा । भू॒मा मा । मा प्र । प्र हा॑सीत् । हा॒सी॒दिति॑ हासीत् ॥ आ मा᳚ । मा॒ऽग्नि॒ष्टो॒मः । अ॒ग्नि॒ष्टो॒मो वि॑शतु ( ) । अ॒ग्नि॒ष्टो॒म इत्य॑ग्नि - स्तो॒मः । वि॒श॒तू॒क्थ्यः॑ । उ॒क्थ्य॑श्च । चा॒ति॒रा॒त्रः । अ॒ति॒रा॒त्रो मा᳚ । अ॒ति॒रा॒त्र इत्य॑ति - रा॒त्रः । मा । आ वि॑शतु । वि॒श॒त्वा॒पि॒श॒र्व॒रः । आ॒पि॒श॒र्व॒र इत्या॑पि - श॒र्व॒रः ॥ ति॒रोअ॑ह्निया मा । ति॒रोअ॑ह्निया॒ इति॑ ति॒रः - अ॒ह्नि॒याः॒ । मा॒ सुहु॑ताः । सुहु॑ता॒ आ । सुहु॑ता॒ इति॒ सु - हु॒ताः॒ । आ वि॑शन्तु । वि॒श॒न्तु॒ स॒हस्र॑स्य । स॒हस्र॑स्य मा । मा॒ भू॒मा । भू॒मा मा । मा प्र । प्र हा॑सीत् । हा॒सी॒दिति॑ हासीत् । \newline

\textbf{Jatai Paata} \newline

1. आ मे॑ म॒ आ मे᳚ । \newline
2. मे॒ गृ॒हा गृ॒हा मे॑ मे गृ॒हाः । \newline
3. गृ॒हा भ॑वन्तु भवन्तु गृ॒हा गृ॒हा भ॑वन्तु । \newline
4. भ॒व॒न्त्वा भ॑वन्तु भव॒न्त्वा । \newline
5. आ प्र॒जा प्र॒जा ऽऽप्र॒जा । \newline
6. प्र॒जा मे॑ मे प्र॒जा प्र॒जा मे᳚ । \newline
7. प्र॒जेति॑ प्र - जा । \newline
8. म॒ आ मे॑ म॒ आ । \newline
9. आ मा॒ मा ऽऽमा᳚ । \newline
10. मा॒ य॒ज्ञो य॒ज्ञो मा॑ मा य॒ज्ञ्ः । \newline
11. य॒ज्ञो वि॑शतु विशतु य॒ज्ञो य॒ज्ञो वि॑शतु । \newline
12. वि॒श॒तु॒ वी॒र्या॑वान्. वी॒र्या॑वान्. विशतु विशतु वी॒र्या॑वान् । \newline
13. वी॒र्या॑वा॒निति॑ वी॒र्य॑ - वा॒न् । \newline
14. आपो॑ दे॒वीर् दे॒वी राप॒ आपो॑ दे॒वीः । \newline
15. दे॒वीर् य॒ज्ञिया॑ य॒ज्ञिया॑ दे॒वीर् दे॒वीर् य॒ज्ञियाः᳚ । \newline
16. य॒ज्ञिया॑ मा मा य॒ज्ञिया॑ य॒ज्ञिया॑ मा । \newline
17. मा ऽऽमा॒ मा । \newline
18. आ वि॑शन्तु विश॒न्त्वा वि॑शन्तु । \newline
19. वि॒श॒न्तु॒ स॒हस्र॑स्य स॒हस्र॑स्य विशन्तु विशन्तु स॒हस्र॑स्य । \newline
20. स॒हस्र॑स्य मा मा स॒हस्र॑स्य स॒हस्र॑स्य मा । \newline
21. मा॒ भू॒मा भू॒मा मा॑ मा भू॒मा । \newline
22. भू॒मा मा मा भू॒मा भू॒मा मा । \newline
23. मा प्र प्र मा मा प्र । \newline
24. प्र हा॑सी द्धासी॒त् प्र प्र हा॑सीत् । \newline
25. हा॒सी॒ दिति॑ हासीत् । \newline
26. आ मे॑ म॒ आ मे᳚ । \newline
27. मे॒ ग्रहो॒ ग्रहो॑ मे मे॒ ग्रहः॑ । \newline
28. ग्रहो॑ भवतु भवतु॒ ग्रहो॒ ग्रहो॑ भवतु । \newline
29. भ॒व॒त्वा भ॑वतु भव॒त्वा । \newline
30. आ पु॑रो॒रुक् पु॑रो॒रुगा पु॑रो॒रुक् । \newline
31. पु॒रो॒रुख् स्तु॑तश॒स्त्रे स्तु॑तश॒स्त्रे पु॑रो॒रुक् पु॑रो॒रुख् स्तु॑तश॒स्त्रे । \newline
32. पु॒रो॒रुगिति॑ पुरः - रुक् । \newline
33. स्तु॒त॒श॒स्त्रे मा॑ मा स्तुतश॒स्त्रे स्तु॑तश॒स्त्रे मा᳚ । \newline
34. स्तु॒त॒श॒स्त्रे इति॑ स्तुत - श॒स्त्रे । \newline
35. मा ऽऽमा॒ मा । \newline
36. आ वि॑शतां ॅविशता॒ मा वि॑शताम् । \newline
37. वि॒श॒ताꣳ॒॒ स॒मीची॑ स॒मीची॑ विशतां ॅविशताꣳ स॒मीची᳚ । \newline
38. स॒मीची॒ इति॑ स॒मीची᳚ । \newline
39. आ॒दि॒त्या रु॒द्रा रु॒द्रा आ॑दि॒त्या आ॑दि॒त्या रु॒द्राः । \newline
40. रु॒द्रा वस॑वो॒ वस॑वो रु॒द्रा रु॒द्रा वस॑वः । \newline
41. वस॑वो मे मे॒ वस॑वो॒ वस॑वो मे । \newline
42. मे॒ स॒द॒स्याः᳚ सद॒स्या॑ मे मे सद॒स्याः᳚ । \newline
43. स॒द॒स्याः᳚ स॒हस्र॑स्य स॒हस्र॑स्य सद॒स्याः᳚ सद॒स्याः᳚ स॒हस्र॑स्य । \newline
44. स॒हस्र॑स्य मा मा स॒हस्र॑स्य स॒हस्र॑स्य मा । \newline
45. मा॒ भू॒मा भू॒मा मा॑ मा भू॒मा । \newline
46. भू॒मा मा मा भू॒मा भू॒मा मा । \newline
47. मा प्र प्र मा मा प्र । \newline
48. प्र हा॑सी द्धासी॒त् प्र प्र हा॑सीत् । \newline
49. हा॒सी॒ दिति॑ हासीत् । \newline
50. आ मा॒ मा ऽऽमा᳚ । \newline
51. मा॒ ऽग्नि॒ष्टो॒मो᳚ ऽग्निष्टो॒मो मा॑ मा ऽग्निष्टो॒मः । \newline
52. अ॒ग्नि॒ष्टो॒मो वि॑शतु विशत्वग्निष्टो॒मो᳚ ऽग्निष्टो॒मो वि॑शतु । \newline
53. अ॒ग्नि॒ष्टो॒म इत्य॑ग्नि - स्तो॒मः । \newline
54. वि॒श॒तू॒क्थ्य॑ उ॒क्थ्यो॑ विशतु विशतू॒क्थ्यः॑ । \newline
55. उ॒क्थ्य॑ श्च चो॒क्थ्य॑ उ॒क्थ्य॑ श्च । \newline
56. चा॒ति॒रा॒त्रो॑ ऽतिरा॒त्र श्च॑ चातिरा॒त्रः । \newline
57. अ॒ति॒रा॒त्रो मा॑ मा ऽतिरा॒त्रो॑ ऽतिरा॒त्रो मा᳚ । \newline
58. अ॒ति॒रा॒त्र इत्य॑ति - रा॒त्रः । \newline
59. मा ऽऽमा॒ मा । \newline
60. आ वि॑शतु विश॒त्वा वि॑शतु । \newline
61. वि॒श॒ त्वा॒पि॒श॒र्व॒र आ॑पिशर्व॒रो वि॑शतु विश त्वापिशर्व॒रः । \newline
62. आ॒पि॒श॒र्व॒र इत्या॑पि - श॒र्व॒रः । \newline
63. ति॒रोअ॑ह्निया मा मा ति॒रोअ॑ह्निया स्ति॒रोअ॑ह्निया मा । \newline
64. ति॒रोअ॑ह्निया॒ इति॑ ति॒रः - अ॒ह्नि॒याः॒ । \newline
65. मा॒ सुहु॑ताः॒ सुहु॑ता मा मा॒ सुहु॑ताः । \newline
66. सुहु॑ता॒ आ सुहु॑ताः॒ सुहु॑ता॒ आ । \newline
67. सुहु॑ता॒ इति॒ सु - हु॒ताः॒ । \newline
68. आ वि॑शन्तु विश॒न्त्वा वि॑शन्तु । \newline
69. वि॒श॒न्तु॒ स॒हस्र॑स्य स॒हस्र॑स्य विशन्तु विशन्तु स॒हस्र॑स्य । \newline
70. स॒हस्र॑स्य मा मा स॒हस्र॑स्य स॒हस्र॑स्य मा । \newline
71. मा॒ भू॒मा भू॒मा मा॑ मा भू॒मा । \newline
72. भू॒मा मा मा भू॒मा भू॒मा मा । \newline
73. मा प्र प्र मा मा प्र । \newline
74. प्र हा॑सी द्धासी॒त् प्र प्र हा॑सीत् । \newline
75. हा॒सी॒ दिति॑ हासीत् । \newline

\textbf{Ghana Paata } \newline

1. आ मे॑ म॒ आ मे॑ गृ॒हा गृ॒हा म॒ आ मे॑ गृ॒हाः । \newline
2. मे॒ गृ॒हा गृ॒हा मे॑ मे गृ॒हा भ॑वन्तु भवन्तु गृ॒हा मे॑ मे गृ॒हा भ॑वन्तु । \newline
3. गृ॒हा भ॑वन्तु भवन्तु गृ॒हा गृ॒हा भ॑व॒न्त्वा भ॑वन्तु गृ॒हा गृ॒हा भ॑व॒न्त्वा । \newline
4. भ॒व॒न्त्वा भ॑वन्तु भव॒न्त्वा प्र॒जा प्र॒जा ऽऽभ॑वन्तु भव॒न्त्वा प्र॒जा । \newline
5. आ प्र॒जा प्र॒जा ऽऽप्र॒जा मे॑ मे प्र॒जा ऽऽप्र॒जा मे᳚ । \newline
6. प्र॒जा मे॑ मे प्र॒जा प्र॒जा म॒ आ मे᳚ प्र॒जा प्र॒जा म॒ आ । \newline
7. प्र॒जेति॑ प्र - जा । \newline
8. म॒ आ मे॑ म॒ आ मा॒ मा ऽऽमे॑ म॒ आ मा᳚ । \newline
9. आ मा॒ मा ऽऽमा॑ य॒ज्ञो य॒ज्ञो मा ऽऽमा॑ य॒ज्ञ्ः । \newline
10. मा॒ य॒ज्ञो य॒ज्ञो मा॑ मा य॒ज्ञो वि॑शतु विशतु य॒ज्ञो मा॑ मा य॒ज्ञो वि॑शतु । \newline
11. य॒ज्ञो वि॑शतु विशतु य॒ज्ञो य॒ज्ञो वि॑शतु वी॒र्या॑वान्. वी॒र्या॑वान्. विशतु य॒ज्ञो य॒ज्ञो वि॑शतु वी॒र्या॑वान् । \newline
12. वि॒श॒तु॒ वी॒र्या॑वान्. वी॒र्या॑वान्. विशतु विशतु वी॒र्या॑वान् । \newline
13. वी॒र्या॑वा॒निति॑ वी॒र्य॑ - वा॒न् । \newline
14. आपो॑ दे॒वीर् दे॒वी राप॒ आपो॑ दे॒वीर् य॒ज्ञिया॑ य॒ज्ञिया॑ दे॒वी राप॒ आपो॑ दे॒वीर् य॒ज्ञियाः᳚ । \newline
15. दे॒वीर् य॒ज्ञिया॑ य॒ज्ञिया॑ दे॒वीर् दे॒वीर् य॒ज्ञिया॑ मा मा य॒ज्ञिया॑ दे॒वीर् दे॒वीर् य॒ज्ञिया॑ मा । \newline
16. य॒ज्ञिया॑ मा मा य॒ज्ञिया॑ य॒ज्ञिया॒ मा ऽऽमा॑ य॒ज्ञिया॑ य॒ज्ञिया॒ मा । \newline
17. मा ऽऽमा॒ मा ऽऽवि॑शन्तु विश॒न्त्वा मा॒ मा ऽऽवि॑शन्तु । \newline
18. आ वि॑शन्तु विश॒न्त्वा वि॑शन्तु स॒हस्र॑स्य स॒हस्र॑स्य विश॒न्त्वा वि॑शन्तु स॒हस्र॑स्य । \newline
19. वि॒श॒न्तु॒ स॒हस्र॑स्य स॒हस्र॑स्य विशन्तु विशन्तु स॒हस्र॑स्य मा मा स॒हस्र॑स्य विशन्तु विशन्तु स॒हस्र॑स्य मा । \newline
20. स॒हस्र॑स्य मा मा स॒हस्र॑स्य स॒हस्र॑स्य मा भू॒मा भू॒मा मा॑ स॒हस्र॑स्य स॒हस्र॑स्य मा भू॒मा । \newline
21. मा॒ भू॒मा भू॒मा मा॑ मा भू॒मा मा मा भू॒मा मा॑ मा भू॒मा मा । \newline
22. भू॒मा मा मा भू॒मा भू॒मा मा प्र प्र मा भू॒मा भू॒मा मा प्र । \newline
23. मा प्र प्र मा मा प्र हा॑सी द्धासी॒त् प्र मा मा प्र हा॑सीत् । \newline
24. प्र हा॑सी द्धासी॒त् प्र प्र हा॑सीत् । \newline
25. हा॒सी॒दिति॑ हासीत् । \newline
26. आ मे॑ म॒ आ मे॒ ग्रहो॒ ग्रहो॑ म॒ आ मे॒ ग्रहः॑ । \newline
27. मे॒ ग्रहो॒ ग्रहो॑ मे मे॒ ग्रहो॑ भवतु भवतु॒ ग्रहो॑ मे मे॒ ग्रहो॑ भवतु । \newline
28. ग्रहो॑ भवतु भवतु॒ ग्रहो॒ ग्रहो॑ भव॒त्वा भ॑वतु॒ ग्रहो॒ ग्रहो॑ भव॒त्वा । \newline
29. भ॒व॒त्वा भ॑वतु भव॒त्वा पु॑रो॒रुक् पु॑रो॒रुगा भ॑वतु भव॒त्वा पु॑रो॒रुक् । \newline
30. आ पु॑रो॒रुक् पु॑रो॒रुगा पु॑रो॒रुख् स्तु॑तश॒स्त्रे स्तु॑तश॒स्त्रे पु॑रो॒रुगा पु॑रो॒रुख् स्तु॑तश॒स्त्रे । \newline
31. पु॒रो॒रुख् स्तु॑तश॒स्त्रे स्तु॑तश॒स्त्रे पु॑रो॒रुक् पु॑रो॒रुख् स्तु॑तश॒स्त्रे मा॑ मा स्तुतश॒स्त्रे पु॑रो॒रुक् पु॑रो॒रुख् स्तु॑तश॒स्त्रे मा᳚ । \newline
32. पु॒रो॒रुगिति॑ पुरः - रुक् । \newline
33. स्तु॒त॒श॒स्त्रे मा॑ मा स्तुतश॒स्त्रे स्तु॑तश॒स्त्रे मा ऽऽमा᳚ स्तुतश॒स्त्रे स्तु॑तश॒स्त्रे मा । \newline
34. स्तु॒त॒श॒स्त्रे इति॑ स्तुत - श॒स्त्रे । \newline
35. मा ऽऽमा॒ मा ऽऽवि॑शतां ॅविशता॒ मा मा॒ मा ऽऽवि॑शताम् । \newline
36. आ वि॑शतां ॅविशता॒ मा वि॑शताꣳ स॒मीची॑ स॒मीची॑ विशता॒ मा वि॑शताꣳ स॒मीची᳚ । \newline
37. वि॒श॒ताꣳ॒॒ स॒मीची॑ स॒मीची॑ विशतां ॅविशताꣳ स॒मीची᳚ । \newline
38. स॒मीची॒ इति॑ स॒मीची᳚ । \newline
39. आ॒दि॒त्या रु॒द्रा रु॒द्रा आ॑दि॒त्या आ॑दि॒त्या रु॒द्रा वस॑वो॒ वस॑वो रु॒द्रा आ॑दि॒त्या आ॑दि॒त्या रु॒द्रा वस॑वः । \newline
40. रु॒द्रा वस॑वो॒ वस॑वो रु॒द्रा रु॒द्रा वस॑वो मे मे॒ वस॑वो रु॒द्रा रु॒द्रा वस॑वो मे । \newline
41. वस॑वो मे मे॒ वस॑वो॒ वस॑वो मे सद॒स्याः᳚ सद॒स्या॑ मे॒ वस॑वो॒ वस॑वो मे सद॒स्याः᳚ । \newline
42. मे॒ स॒द॒स्याः᳚ सद॒स्या॑ मे मे सद॒स्याः᳚ स॒हस्र॑स्य स॒हस्र॑स्य सद॒स्या॑ मे मे सद॒स्याः᳚ स॒हस्र॑स्य । \newline
43. स॒द॒स्याः᳚ स॒हस्र॑स्य स॒हस्र॑स्य सद॒स्याः᳚ सद॒स्याः᳚ स॒हस्र॑स्य मा मा स॒हस्र॑स्य सद॒स्याः᳚ सद॒स्याः᳚ स॒हस्र॑स्य मा । \newline
44. स॒हस्र॑स्य मा मा स॒हस्र॑स्य स॒हस्र॑स्य मा भू॒मा भू॒मा मा॑ स॒हस्र॑स्य स॒हस्र॑स्य मा भू॒मा । \newline
45. मा॒ भू॒मा भू॒मा मा॑ मा भू॒मा मा मा भू॒मा मा॑ मा भू॒मा मा । \newline
46. भू॒मा मा मा भू॒मा भू॒मा मा प्र प्र मा भू॒मा भू॒मा मा प्र । \newline
47. मा प्र प्र मा मा प्र हा॑सी द्धासी॒त् प्र मा मा प्र हा॑सीत् । \newline
48. प्र हा॑सी द्धासी॒त् प्र प्र हा॑सीत् । \newline
49. हा॒सी॒दिति॑ हासीत् । \newline
50. आ मा॒ मा ऽऽमा᳚ ऽग्निष्टो॒मो᳚ ऽग्निष्टो॒मो मा ऽऽमा᳚ ऽग्निष्टो॒मः । \newline
51. मा॒ ऽग्नि॒ष्टो॒मो᳚ ऽग्निष्टो॒मो मा॑ मा ऽग्निष्टो॒मो वि॑शतु विश त्वग्निष्टो॒मो मा॑ मा ऽग्निष्टो॒मो वि॑शतु । \newline
52. अ॒ग्नि॒ष्टो॒मो वि॑शतु विश त्वग्निष्टो॒मो᳚ ऽग्निष्टो॒मो वि॑शतू॒क्थ्य॑ उ॒क्थ्यो॑ विश त्वग्निष्टो॒मो᳚ ऽग्निष्टो॒मो वि॑शतू॒क्थ्यः॑ । \newline
53. अ॒ग्नि॒ष्टो॒म इत्य॑ग्नि - स्तो॒मः । \newline
54. वि॒श॒तू॒क्थ्य॑ उ॒क्थ्यो॑ विशतु विशतू॒क्थ्य॑ श्च चो॒क्थ्यो॑ विशतु विशतू॒क्थ्य॑ श्च । \newline
55. उ॒क्थ्य॑ श्च चो॒क्थ्य॑ उ॒क्थ्य॑ श्चातिरा॒त्रो॑ ऽतिरा॒त्र श्चो॒क्थ्य॑ उ॒क्थ्य॑ श्चातिरा॒त्रः । \newline
56. चा॒ ति॒रा॒त्रो॑ ऽतिरा॒त्रश्च॑ चा तिरा॒त्रो मा॑ मा ऽतिरा॒त्र श्च॑ चा तिरा॒त्रो मा᳚ । \newline
57. अ॒ति॒रा॒त्रो मा॑ मा ऽतिरा॒त्रो॑ ऽतिरा॒त्रो मा ऽऽमा॑ ऽतिरा॒त्रो॑ ऽतिरा॒त्रो मा । \newline
58. अ॒ति॒रा॒त्र इत्य॑ति - रा॒त्रः । \newline
59. मा ऽऽमा॒ मा ऽऽवि॑शतु विश॒त्वा मा॒ मा ऽऽवि॑शतु । \newline
60. आ वि॑शतु विश॒त्वा वि॑श त्वापिशर्व॒र आ॑पिशर्व॒रो वि॑श॒त्वा वि॑श त्वापिशर्व॒रः । \newline
61. वि॒श॒ त्वा॒पि॒श॒र्व॒र आ॑पिशर्व॒रो वि॑शतु विश त्वापिशर्व॒रः । \newline
62. आ॒पि॒श॒र्व॒र इत्या॑पि - श॒र्व॒रः । \newline
63. ति॒रोअ॑ह्निया मा मा ति॒रोअ॑ह्निया स्ति॒रोअ॑ह्निया मा॒ सुहु॑ताः॒ सुहु॑ता मा ति॒रोअ॑ह्निया स्ति॒रोअ॑ह्निया मा॒ सुहु॑ताः । \newline
64. ति॒रोअ॑ह्निया॒ इति॑ ति॒रः - अ॒ह्नि॒याः॒ । \newline
65. मा॒ सुहु॑ताः॒ सुहु॑ता मा मा॒ सुहु॑ता॒ आ सुहु॑ता मा मा॒ सुहु॑ता॒ आ । \newline
66. सुहु॑ता॒ आ सुहु॑ताः॒ सुहु॑ता॒ आ वि॑शन्तु विश॒न्त्वा सुहु॑ताः॒ सुहु॑ता॒ आ वि॑शन्तु । \newline
67. सुहु॑ता॒ इति॒ सु - हु॒ताः॒ । \newline
68. आ वि॑शन्तु विश॒न्त्वा वि॑शन्तु स॒हस्र॑स्य स॒हस्र॑स्य विश॒न्त्वा वि॑शन्तु स॒हस्र॑स्य । \newline
69. वि॒श॒न्तु॒ स॒हस्र॑स्य स॒हस्र॑स्य विशन्तु विशन्तु स॒हस्र॑स्य मा मा स॒हस्र॑स्य विशन्तु विशन्तु स॒हस्र॑स्य मा । \newline
70. स॒हस्र॑स्य मा मा स॒हस्र॑स्य स॒हस्र॑स्य मा भू॒मा भू॒मा मा॑ स॒हस्र॑स्य स॒हस्र॑स्य मा भू॒मा । \newline
71. मा॒ भू॒मा भू॒मा मा॑ मा भू॒मा मा मा भू॒मा मा॑ मा भू॒मा मा । \newline
72. भू॒मा मा मा भू॒मा भू॒मा मा प्र प्र मा भू॒मा भू॒मा मा प्र । \newline
73. मा प्र प्र मा मा प्र हा॑सी द्धासी॒त् प्र मा मा प्र हा॑सीत् । \newline
74. प्र हा॑सी द्धासी॒त् प्र प्र हा॑सीत् । \newline
75. हा॒सी॒दिति॑ हासीत् । \newline
\pagebreak
\markright{ TS 7.3.14.1  \hfill https://www.vedavms.in \hfill}

\section{ TS 7.3.14.1 }

\textbf{TS 7.3.14.1 } \newline
\textbf{Samhita Paata} \newline

अ॒ग्निना॒ तपोऽन्व॑भवद्-वा॒चा ब्रह्म॑ म॒णिना॑ रू॒पाणीन्द्रे॑ण दे॒वान् वाते॑न प्रा॒णान्थ् सूर्ये॑ण॒ द्यां च॒न्द्रम॑सा॒ नक्ष॑त्राणि य॒मेन॑ पि॒तॄन् राज्ञा᳚ मनु॒ष्या᳚न् फ॒लेन॑ नादे॒यान॑जग॒रेण॑ स॒र्पान् व्या॒घ्रेणा॑ऽऽ*र॒ण्यान् प॒शूञ्छ्ये॒नेन॑ पत॒त्रिणो॒ वृष्णाऽश्वा॑नृष॒भेण॒ गा ब॒स्तेना॒जा वृ॒ष्णिनाऽवी᳚र्व्री॒हिणाऽन्ना॑नि॒ यवे॒नौष॑धीर्न्य॒ग्रोधे॑न॒ वन॒स्पती॑नुदु॒बंरे॒णोर्जं॑ गायत्रि॒या छन्दाꣳ॑सि त्रि॒वृता॒ स्तोमा᳚न् ब्राह्म॒णेन॒ ( ) वाचं᳚ ॥ \newline

\textbf{Pada Paata} \newline

अ॒ग्निना᳚ । तपः॑ । अन्विति॑ । अ॒भ॒व॒त् । वा॒चा । ब्रह्म॑ । म॒णिना᳚ । रू॒पाणि॑ । इन्द्रे॑ण । दे॒वान् । वाते॑न । प्रा॒णानिति॑ प्र - अ॒नान् । सूर्ये॑ण । द्याम् । च॒न्द्रम॑सा । नक्ष॑त्राणि । य॒मेन॑ । पि॒तॄन् । राज्ञा᳚ । म॒नु॒ष्यान्॑ । फ॒लेन॑ । ना॒दे॒यान् । अ॒ज॒ग॒रेण॑ । स॒र्पान् । व्या॒घ्रेण॑ । आ॒र॒ण्यान् । प॒शून् । श्ये॒नेन॑ । प॒त॒त्रिणः॑ । वृष्णा᳚ । अश्वान्॑ । ऋ॒ष॒भेण॑ । गाः । ब॒स्तेन॑ । अ॒जाः । वृ॒ष्णिना᳚ । अवीः᳚ । व्री॒हिणा᳚ । अन्ना॑नि । यवे॑न । ओष॑धीः । न्य॒ग्रोधे॑न । वन॒स्पतीन्॑ । उ॒दु॒म्बरे॑ण । ऊर्ज᳚म् । गा॒य॒त्रि॒या । छन्दाꣳ॑सि । त्रि॒वृतेति॑ त्रि - वृता᳚ । स्तोमान्॑ । ब्रा॒ह्म॒णेन॑ ( ) । वाच᳚म् ॥  \newline


\textbf{Krama Paata} \newline

अ॒ग्निना॒ तपः॑ । तपोऽनु॑ । अन्व॑भवत् । अ॒भ॒व॒द् वा॒चा । वा॒चा ब्रह्म॑ । ब्रह्म॑ म॒णिना᳚ । म॒णिना॑ रू॒पाणि॑ । रू॒पाणीन्द्रे॑ण । इन्द्रे॑ण दे॒वान् । दे॒वान्. वाते॑न । वाते॑न प्रा॒णान् । प्रा॒णान्थ् सूर्ये॑ण । प्रा॒णानिति॑ प्र - अ॒नान् । सूर्ये॑ण॒ द्याम् । द्याम् च॒न्द्रम॑सा । च॒न्द्रम॑सा॒ नक्ष॑त्राणि । नक्ष॑त्राणि य॒मेन॑ । य॒मेन॑ पि॒तॄन् । पि॒तॄन् राज्ञा᳚ । राज्ञा॑ मनु॒ष्यान्॑ । म॒नु॒ष्या᳚न् फ॒लेन॑ । फ॒लेन॑ नादे॒यान् । ना॒दे॒यान॑जग॒रेण॑ । अ॒ज॒ग॒रेण॑ स॒र्पान् । स॒र्पान् व्या॒घ्रेण॑ । व्या॒घ्रेणा॑र॒ण्यान् । आ॒र॒ण्यान् प॒शून् । प॒शूञ्छ्ये॒नेन॑ । श्ये॒नेन॑ पत॒त्रिणः॑ । प॒त॒त्रिणो॒ वृष्णा᳚ । वृष्णाऽश्वान्॑ । अश्वा॑नृष॒भेण॑ । ऋ॒ष॒भेण॒ गाः । गा ब॒स्तेन॑ । ब॒स्तेना॒जाः । अ॒जा वृ॒ष्णिना᳚ । वृ॒ष्णिनाऽवीः᳚ । अवी᳚र् व्री॒हिणा᳚ । व्री॒हिणाऽन्ना॑नि । अन्ना॑नि॒ यवे॑न । यवे॒नौष॑धीः । ओष॑धीर् न्य॒ग्रोधे॑न । न्य॒ग्रोधे॑न॒ वन॒स्पतीन्॑ । वन॒स्पती॑नुदु॒म्बरे॑ण । उ॒दु॒म्बरे॒णोर्ज᳚म् । ऊर्ज॑म् गायत्रि॒या । गा॒य॒त्रि॒या छन्दाꣳ॑सि । छन्दाꣳ॑सि त्रि॒वृता᳚ । त्रि॒वृता॒ स्तोमान्॑ । त्रि॒वृतेति॑ त्रि - वृता᳚ । स्तोमा᳚न् ब्राह्म॒णेन॑ ( ) । ब्रा॒ह्म॒णेन॒ वाच᳚म् । वाच॒मिति॒ वाच᳚म् । \newline

\textbf{Jatai Paata} \newline

1. अ॒ग्निना॒ तप॒ स्तपो॒ ऽग्निना॒ ऽग्निना॒ तपः॑ । \newline
2. तपो ऽन्वनु॒ तप॒ स्तपो ऽनु॑ । \newline
3. अन्व॑भव दभव॒ दन् वन् व॑भवत् । \newline
4. अ॒भ॒व॒द् वा॒चा वा॒चा ऽभ॑व दभवद् वा॒चा । \newline
5. वा॒चा ब्रह्म॒ ब्रह्म॑ वा॒चा वा॒चा ब्रह्म॑ । \newline
6. ब्रह्म॑ म॒णिना॑ म॒णिना॒ ब्रह्म॒ ब्रह्म॑ म॒णिना᳚ । \newline
7. म॒णिना॑ रू॒पाणि॑ रू॒पाणि॑ म॒णिना॑ म॒णिना॑ रू॒पाणि॑ । \newline
8. रू॒पाणीन्द्रे॒ णेन्द्रे॑ण रू॒पाणि॑ रू॒पाणीन्द्रे॑ण । \newline
9. इन्द्रे॑ण दे॒वान् दे॒वा निन्द्रे॒ णेन्द्रे॑ण दे॒वान् । \newline
10. दे॒वान्. वाते॑न॒ वाते॑न दे॒वान् दे॒वान्. वाते॑न । \newline
11. वाते॑न प्रा॒णान् प्रा॒णान्. वाते॑न॒ वाते॑न प्रा॒णान् । \newline
12. प्रा॒णान् थ्सूर्ये॑ण॒ सूर्ये॑ण प्रा॒णान् प्रा॒णान् थ्सूर्ये॑ण । \newline
13. प्रा॒णानिति॑ प्र - अ॒नान् । \newline
14. सूर्ये॑ण॒ द्याम् द्याꣳ सूर्ये॑ण॒ सूर्ये॑ण॒ द्याम् । \newline
15. द्याम् च॒न्द्रम॑सा च॒न्द्रम॑सा॒ द्याम् द्याम् च॒न्द्रम॑सा । \newline
16. च॒न्द्रम॑सा॒ नक्ष॑त्राणि॒ नक्ष॑त्राणि च॒न्द्रम॑सा च॒न्द्रम॑सा॒ नक्ष॑त्राणि । \newline
17. नक्ष॑त्राणि य॒मेन॑ य॒मेन॒ नक्ष॑त्राणि॒ नक्ष॑त्राणि य॒मेन॑ । \newline
18. य॒मेन॑ पि॒तॄन् पि॒तॄन्. य॒मेन॑ य॒मेन॑ पि॒तॄन् । \newline
19. पि॒तॄन् राज्ञा॒ राज्ञा॑ पि॒तॄन् पि॒तॄन् राज्ञा᳚ । \newline
20. राज्ञा॑ मनु॒ष्या᳚न् मनु॒ष्या᳚न् राज्ञा॒ राज्ञा॑ मनु॒ष्यान्॑ । \newline
21. म॒नु॒ष्या᳚न् फ॒लेन॑ फ॒लेन॑ मनु॒ष्या᳚न् मनु॒ष्या᳚न् फ॒लेन॑ । \newline
22. फ॒लेन॑ नादे॒यान् ना॑दे॒यान् फ॒लेन॑ फ॒लेन॑ नादे॒यान् । \newline
23. ना॒दे॒या न॑जग॒रेणा॑ जग॒रेण॑ नादे॒यान् ना॑दे॒या न॑जग॒रेण॑ । \newline
24. अ॒ज॒ग॒रेण॑ स॒र्पान् थ्स॒र्पा न॑जग॒रेणा॑ जग॒रेण॑ स॒र्पान् । \newline
25. स॒र्पान् व्या॒घ्रेण॑ व्या॒घ्रेण॑ स॒र्पान् थ्स॒र्पान् व्या॒घ्रेण॑ । \newline
26. व्या॒घ्रेणा॑ र॒ण्या ना॑र॒ण्यान् व्या॒घ्रेण॑ व्या॒घ्रेणा॑ र॒ण्यान् । \newline
27. आ॒र॒ण्यान् प॒शून् प॒शू ना॑र॒ण्या ना॑र॒ण्यान् प॒शून् । \newline
28. प॒शूञ् छ्ये॒नेन॑ श्ये॒नेन॑ प॒शून् प॒शूञ् छ्ये॒नेन॑ । \newline
29. श्ये॒नेन॑ पत॒त्रिणः॑ पत॒त्रिणः॑ श्ये॒नेन॑ श्ये॒नेन॑ पत॒त्रिणः॑ । \newline
30. प॒त॒त्रिणो॒ वृष्णा॒ वृष्णा॑ पत॒त्रिणः॑ पत॒त्रिणो॒ वृष्णा᳚ । \newline
31. वृष्णा ऽश्वा॒ नश्वा॒न् वृष्णा॒ वृष्णा ऽश्वान्॑ । \newline
32. अश्वा॑ नृष॒भेण॑ र्.ष॒भेणाश्वा॒ नश्वा॑ नृष॒भेण॑ । \newline
33. ऋ॒ष॒भेण॒ गा गा ऋ॑ष॒भेण॑ र्.ष॒भेण॒ गाः । \newline
34. गा ब॒स्तेन॑ ब॒स्तेन॒ गा गा ब॒स्तेन॑ । \newline
35. ब॒स्तेना॒ जा अ॒जा ब॒स्तेन॑ ब॒स्तेना॒ जाः । \newline
36. अ॒जा वृ॒ष्णिना॑ वृ॒ष्णिना॒ ऽजा अ॒जा वृ॒ष्णिना᳚ । \newline
37. वृ॒ष्णिना ऽवी॒ रवी᳚र् वृ॒ष्णिना॑ वृ॒ष्णिना ऽवीः᳚ । \newline
38. अवी᳚र् व्री॒हिणा᳚ व्री॒हिणा ऽवी॒ रवी᳚र् व्री॒हिणा᳚ । \newline
39. व्री॒हिणा ऽन्ना॒ न्यन्ना॑नि व्री॒हिणा᳚ व्री॒हिणा ऽन्ना॑नि । \newline
40. अन्ना॑नि॒ यवे॑न॒ यवे॒ना न्ना॒ न्यन्ना॑नि॒ यवे॑न । \newline
41. यवे॒नौष॑धी॒ रोष॑धी॒र् यवे॑न॒ यवे॒नौष॑धीः । \newline
42. ओष॑धीर् न्य॒ग्रोधे॑न न्य॒ग्रोधे॒ नौष॑धी॒ रोष॑धीर् न्य॒ग्रोधे॑न । \newline
43. न्य॒ग्रोधे॑न॒ वन॒स्पती॒न्॒. वन॒स्पती᳚न् न्य॒ग्रोधे॑न न्य॒ग्रोधे॑न॒ वन॒स्पतीन्॑ । \newline
44. वन॒स्पती॑ नुदु॒म्बरे॑ णोदु॒म्बरे॑ण॒ वन॒स्पती॒न्॒. वन॒स्पती॑ नुदु॒म्बरे॑ण । \newline
45. उ॒दु॒म्बरे॒ णोर्ज॒ मूर्ज॑ मुदु॒म्बरे॑ णोदु॒म्बरे॒ णोर्ज᳚म् । \newline
46. ऊर्ज॑म् गायत्रि॒या गा॑यत्रि॒ योर्ज॒ मूर्ज॑म् गायत्रि॒या । \newline
47. गा॒य॒त्रि॒या छन्दाꣳ॑सि॒ छन्दाꣳ॑सि गायत्रि॒या गा॑यत्रि॒या छन्दाꣳ॑सि । \newline
48. छन्दाꣳ॑सि त्रि॒वृता᳚ त्रि॒वृता॒ छन्दाꣳ॑सि॒ छन्दाꣳ॑सि त्रि॒वृता᳚ । \newline
49. त्रि॒वृता॒ स्तोमा॒न् थ्स्तोमा᳚न् त्रि॒वृता᳚ त्रि॒वृता॒ स्तोमान्॑ । \newline
50. त्रि॒वृतेति॑ त्रि - वृता᳚ । \newline
51. स्तोमा᳚न् ब्राह्म॒णेन॑ ब्राह्म॒णेन॒ स्तोमा॒न् थ्स्तोमा᳚न् ब्राह्म॒णेन॑ । \newline
52. ब्रा॒ह्म॒णेन॒ वाचं॒ ॅवाच॑म् ब्राह्म॒णेन॑ ब्राह्म॒णेन॒ वाच᳚म् । \newline
53. वाच॒मिति॒ वाच᳚म् । \newline

\textbf{Ghana Paata } \newline

1. अ॒ग्निना॒ तप॒ स्तपो॒ ऽग्निना॒ ऽग्निना॒ तपो ऽन्वनु॒ तपो॒ ऽग्निना॒ ऽग्निना॒ तपो ऽनु॑ । \newline
2. तपो ऽन्वनु॒ तप॒ स्तपो ऽन्व॑भव दभव॒ दनु॒ तप॒ स्तपो ऽन्व॑भवत् । \newline
3. अन्व॑भव दभव॒दन् वन् व॑भवद् वा॒चा वा॒चा ऽभ॑व॒ दन् वन् व॑भवद् वा॒चा । \newline
4. अ॒भ॒व॒द् वा॒चा वा॒चा ऽभ॑व दभवद् वा॒चा ब्रह्म॒ ब्रह्म॑ वा॒चा ऽभ॑व दभवद् वा॒चा ब्रह्म॑ । \newline
5. वा॒चा ब्रह्म॒ ब्रह्म॑ वा॒चा वा॒चा ब्रह्म॑ म॒णिना॑ म॒णिना॒ ब्रह्म॑ वा॒चा वा॒चा ब्रह्म॑ म॒णिना᳚ । \newline
6. ब्रह्म॑ म॒णिना॑ म॒णिना॒ ब्रह्म॒ ब्रह्म॑ म॒णिना॑ रू॒पाणि॑ रू॒पाणि॑ म॒णिना॒ ब्रह्म॒ ब्रह्म॑ म॒णिना॑ रू॒पाणि॑ । \newline
7. म॒णिना॑ रू॒पाणि॑ रू॒पाणि॑ म॒णिना॑ म॒णिना॑ रू॒पाणीन्द्रे॒ णेन्द्रे॑ण रू॒पाणि॑ म॒णिना॑ म॒णिना॑ रू॒पाणीन्द्रे॑ण । \newline
8. रू॒पाणीन्द्रे॒ णेन्द्रे॑ण रू॒पाणि॑ रू॒पाणीन्द्रे॑ण दे॒वान् दे॒वा निन्द्रे॑ण रू॒पाणि॑ रू॒पाणीन्द्रे॑ण दे॒वान् । \newline
9. इन्द्रे॑ण दे॒वान् दे॒वा निन्द्रे॒ णेन्द्रे॑ण दे॒वान्. वाते॑न॒ वाते॑न दे॒वा निन्द्रे॒ णेन्द्रे॑ण दे॒वान्. वाते॑न । \newline
10. दे॒वान्. वाते॑न॒ वाते॑न दे॒वान् दे॒वान्. वाते॑न प्रा॒णान् प्रा॒णान्. वाते॑न दे॒वान् दे॒वान्. वाते॑न प्रा॒णान् । \newline
11. वाते॑न प्रा॒णान् प्रा॒णान्. वाते॑न॒ वाते॑न प्रा॒णान् थ्सूर्ये॑ण॒ सूर्ये॑ण प्रा॒णान्. वाते॑न॒ वाते॑न प्रा॒णान् थ्सूर्ये॑ण । \newline
12. प्रा॒णान् थ्सूर्ये॑ण॒ सूर्ये॑ण प्रा॒णान् प्रा॒णान् थ्सूर्ये॑ण॒ द्याम् द्याꣳ सूर्ये॑ण प्रा॒णान् प्रा॒णान् थ्सूर्ये॑ण॒ द्याम् । \newline
13. प्रा॒णानिति॑ प्र - अ॒नान् । \newline
14. सूर्ये॑ण॒ द्याम् द्याꣳ सूर्ये॑ण॒ सूर्ये॑ण॒ द्याम् च॒न्द्रम॑सा च॒न्द्रम॑सा॒ द्याꣳ सूर्ये॑ण॒ सूर्ये॑ण॒ द्याम् च॒न्द्रम॑सा । \newline
15. द्याम् च॒न्द्रम॑सा च॒न्द्रम॑सा॒ द्याम् द्याम् च॒न्द्रम॑सा॒ नक्ष॑त्राणि॒ नक्ष॑त्राणि च॒न्द्रम॑सा॒ द्याम् द्याम् च॒न्द्रम॑सा॒ नक्ष॑त्राणि । \newline
16. च॒न्द्रम॑सा॒ नक्ष॑त्राणि॒ नक्ष॑त्राणि च॒न्द्रम॑सा च॒न्द्रम॑सा॒ नक्ष॑त्राणि य॒मेन॑ य॒मेन॒ नक्ष॑त्राणि च॒न्द्रम॑सा च॒न्द्रम॑सा॒ नक्ष॑त्राणि य॒मेन॑ । \newline
17. नक्ष॑त्राणि य॒मेन॑ य॒मेन॒ नक्ष॑त्राणि॒ नक्ष॑त्राणि य॒मेन॑ पि॒तॄन् पि॒तॄन्. य॒मेन॒ नक्ष॑त्राणि॒ नक्ष॑त्राणि य॒मेन॑ पि॒तॄन् । \newline
18. य॒मेन॑ पि॒तॄन् पि॒तॄन्. य॒मेन॑ य॒मेन॑ पि॒तॄन् राज्ञा॒ राज्ञा॑ पि॒तॄन्. य॒मेन॑ य॒मेन॑ पि॒तॄन् राज्ञा᳚ । \newline
19. पि॒तॄन् राज्ञा॒ राज्ञा॑ पि॒तॄन् पि॒तॄन् राज्ञा॑ मनु॒ष्या᳚न् मनु॒ष्या᳚न् राज्ञा॑ पि॒तॄन् पि॒तॄन् राज्ञा॑ मनु॒ष्यान्॑ । \newline
20. राज्ञा॑ मनु॒ष्या᳚न् मनु॒ष्या᳚न् राज्ञा॒ राज्ञा॑ मनु॒ष्या᳚न् फ॒लेन॑ फ॒लेन॑ मनु॒ष्या᳚न् राज्ञा॒ राज्ञा॑ मनु॒ष्या᳚न् फ॒लेन॑ । \newline
21. म॒नु॒ष्या᳚न् फ॒लेन॑ फ॒लेन॑ मनु॒ष्या᳚न् मनु॒ष्या᳚न् फ॒लेन॑ नादे॒यान् ना॑दे॒यान् फ॒लेन॑ मनु॒ष्या᳚न् मनु॒ष्या᳚न् फ॒लेन॑ नादे॒यान् । \newline
22. फ॒लेन॑ नादे॒यान् ना॑दे॒यान् फ॒लेन॑ फ॒लेन॑ नादे॒या न॑जग॒रेणा॑ जग॒रेण॑ नादे॒यान् फ॒लेन॑ फ॒लेन॑ नादे॒या न॑जग॒रेण॑ । \newline
23. ना॒दे॒या न॑जग॒रे णा॑जग॒रेण॑ नादे॒यान् ना॑दे॒या न॑जग॒रेण॑ स॒र्पान् थ्स॒र्पा न॑जग॒रेण॑ नादे॒यान् ना॑दे॒या न॑जग॒रेण॑ स॒र्पान् । \newline
24. अ॒ज॒ग॒रेण॑ स॒र्पान् थ्स॒र्पा न॑जग॒रेणा॑ जग॒रेण॑ स॒र्पान् व्या॒घ्रेण॑ व्या॒घ्रेण॑ स॒र्पा न॑जग॒रेणा॑ जग॒रेण॑ स॒र्पान् व्या॒घ्रेण॑ । \newline
25. स॒र्पान् व्या॒घ्रेण॑ व्या॒घ्रेण॑ स॒र्पान् थ्स॒र्पान् व्या॒घ्रेणा॑ र॒ण्या ना॑र॒ण्यान् व्या॒घ्रेण॑ स॒र्पान् थ्स॒र्पान् व्या॒घ्रेणा॑ र॒ण्यान् । \newline
26. व्या॒घ्रेणा॑ र॒ण्या ना॑र॒ण्यान् व्या॒घ्रेण॑ व्या॒घ्रेणा॑ र॒ण्यान् प॒शून् प॒शू ना॑र॒ण्यान् व्या॒घ्रेण॑ व्या॒घ्रेणा॑ र॒ण्यान् प॒शून् । \newline
27. आ॒र॒ण्यान् प॒शून् प॒शू ना॑र॒ण्या ना॑र॒ण्यान् प॒शू ञ्छ्‌ये॒नेन॑ श्ये॒नेन॑ प॒शूना॑ र॒ण्या ना॑र॒ण्यान् प॒शूञ् छ्‌ये॒नेन॑ । \newline
28. प॒शूञ् छ्‌ये॒नेन॑ श्ये॒नेन॑ प॒शून् प॒शूञ् छ्‌ये॒नेन॑ पत॒त्रिणः॑ पत॒त्रिणः॑ श्ये॒नेन॑ प॒शून् प॒शूञ् छ्‌ये॒नेन॑ पत॒त्रिणः॑ । \newline
29. श्ये॒नेन॑ पत॒त्रिणः॑ पत॒त्रिणः॑ श्ये॒नेन॑ श्ये॒नेन॑ पत॒त्रिणो॒ वृष्णा॒ वृष्णा॑ पत॒त्रिणः॑ श्ये॒नेन॑ श्ये॒नेन॑ पत॒त्रिणो॒ वृष्णा᳚ । \newline
30. प॒त॒त्रिणो॒ वृष्णा॒ वृष्णा॑ पत॒त्रिणः॑ पत॒त्रिणो॒ वृष्णा ऽश्वा॒ नश्वा॒न् वृष्णा॑ पत॒त्रिणः॑ पत॒त्रिणो॒ वृष्णा ऽश्वान्॑ । \newline
31. वृष्णा ऽश्वा॒ नश्वा॒न् वृष्णा॒ वृष्णा ऽश्वा॑ नृष॒भेण॑ र्.ष॒भेणा श्वा॒न् वृष्णा॒ वृष्णा ऽश्वा॑ नृष॒भेण॑ । \newline
32. अश्वा॑ नृष॒भेण॑ र्.ष॒भेणाश्वा॒ नश्वा॑ नृष॒भेण॒ गा गा ऋ॑ष॒भेणा श्वा॒ नश्वा॑ नृष॒भेण॒ गाः । \newline
33. ऋ॒ष॒भेण॒ गा गा ऋ॑ष॒भेण॑ र्.ष॒भेण॒ गा ब॒स्तेन॑ ब॒स्तेन॒ गा ऋ॑ष॒भेण॑ र्.ष॒भेण॒ गा ब॒स्तेन॑ । \newline
34. गा ब॒स्तेन॑ ब॒स्तेन॒ गा गा ब॒स्तेना॒जा अ॒जा ब॒स्तेन॒ गा गा ब॒स्तेना॒जाः । \newline
35. ब॒स्तेना॒जा अ॒जा ब॒स्तेन॑ ब॒स्तेना॒जा वृ॒ष्णिना॑ वृ॒ष्णिना॒ ऽजा ब॒स्तेन॑ ब॒स्तेना॒जा वृ॒ष्णिना᳚ । \newline
36. अ॒जा वृ॒ष्णिना॑ वृ॒ष्णिना॒ ऽजा अ॒जा वृ॒ष्णिना ऽवी॒ रवी᳚र् वृ॒ष्णिना॒ ऽजा अ॒जा वृ॒ष्णिना ऽवीः᳚ । \newline
37. वृ॒ष्णिना ऽवी॒ रवी᳚र् वृ॒ष्णिना॑ वृ॒ष्णिना ऽवी᳚र् व्री॒हिणा᳚ व्री॒हिणा ऽवी᳚र् वृ॒ष्णिना॑ वृ॒ष्णिना ऽवी᳚र् व्री॒हिणा᳚ । \newline
38. अवी᳚र् व्री॒हिणा᳚ व्री॒हिणा ऽवी॒ रवी᳚र् व्री॒हिणा ऽन्ना॒ न्यन्ना॑नि व्री॒हिणा ऽवी॒ रवी᳚र् व्री॒हिणा ऽन्ना॑नि । \newline
39. व्री॒हिणा ऽन्ना॒ न्यन्ना॑नि व्री॒हिणा᳚ व्री॒हिणा ऽन्ना॑नि॒ यवे॑न॒ यवे॒ना न्ना॑नि व्री॒हिणा᳚ व्री॒हिणा ऽन्ना॑नि॒ यवे॑न । \newline
40. अन्ना॑नि॒ यवे॑न॒ यवे॒ना न्ना॒ न्यन्ना॑नि॒ यवे॒नौष॑धी॒ रोष॑धी॒र् यवे॒ना न्ना॒ न्यन्ना॑नि॒ यवे॒नौष॑धीः । \newline
41. यवे॒नौष॑धी॒ रोष॑धी॒र् यवे॑न॒ यवे॒नौष॑धीर् न्य॒ग्रोधे॑न न्य॒ग्रोधे॒ नौष॑धी॒र् यवे॑न॒ यवे॒नौ ष॑धीर् न्य॒ग्रोधे॑न । \newline
42. ओष॑धीर् न्य॒ग्रोधे॑न न्य॒ग्रोधे॒ नौष॑धी॒ रोष॑धीर् न्य॒ग्रोधे॑न॒ वन॒स्पती॒न्॒. 
वन॒स्पती᳚न् न्य॒ग्रोधे॒ नौष॑धी॒ रोष॑धीर् न्य॒ग्रोधे॑न॒ वन॒स्पतीन्॑ । \newline
43. न्य॒ग्रोधे॑न॒ वन॒स्पती॒न्॒. वन॒स्पती᳚न् न्य॒ग्रोधे॑न न्य॒ग्रोधे॑न॒ वन॒स्पती॑ नुदु॒म्बरे॑ णोदु॒म्बरे॑ण॒ वन॒स्पती᳚न् न्य॒ग्रोधे॑न न्य॒ग्रोधे॑न॒ वन॒स्पती॑ नुदु॒म्बरे॑ण । \newline
44. वन॒स्पती॑ नुदु॒म्बरे॑णो दु॒म्बरे॑ण॒ वन॒स्पती॒न्॒. वन॒स्पती॑ नुदु॒म्बरे॒ णोर्ज॒ मूर्ज॑ मुदु॒म्बरे॑ण॒ वन॒स्पती॒न्॒. वन॒स्पती॑ नुदु॒म्बरे॒ णोर्ज᳚म् । \newline
45. उ॒दु॒म्बरे॒ णोर्ज॒ मूर्ज॑ मुदु॒म्बरे॑ णोदु॒म्बरे॒ णोर्ज॑म् गायत्रि॒या गा॑यत्रि॒ योर्ज॑ मुदु॒म्बरे॑णो दु॒म्बरे॒ णोर्ज॑म् गायत्रि॒या । \newline
46. ऊर्ज॑म् गायत्रि॒या गा॑यत्रि॒ योर्ज॒ मूर्ज॑म् गायत्रि॒या छन्दाꣳ॑सि॒ छन्दाꣳ॑सि गायत्रि॒ योर्ज॒ मूर्ज॑म् गायत्रि॒या छन्दाꣳ॑सि । \newline
47. गा॒य॒त्रि॒या छन्दाꣳ॑सि॒ छन्दाꣳ॑सि गायत्रि॒या गा॑यत्रि॒या छन्दाꣳ॑सि त्रि॒वृता᳚ त्रि॒वृता॒ छन्दाꣳ॑सि गायत्रि॒या गा॑यत्रि॒या छन्दाꣳ॑सि त्रि॒वृता᳚ । \newline
48. छन्दाꣳ॑सि त्रि॒वृता᳚ त्रि॒वृता॒ छन्दाꣳ॑सि॒ छन्दाꣳ॑सि त्रि॒वृता॒ स्तोमा॒न् थ्स्तोमा᳚न् त्रि॒वृता॒ छन्दाꣳ॑सि॒ छन्दाꣳ॑सि त्रि॒वृता॒ स्तोमान्॑ । \newline
49. त्रि॒वृता॒ स्तोमा॒न् थ्स्तोमा᳚न् त्रि॒वृता᳚ त्रि॒वृता॒ स्तोमा᳚न् ब्राह्म॒णेन॑ ब्राह्म॒णेन॒ स्तोमा᳚न् त्रि॒वृता᳚ त्रि॒वृता॒ स्तोमा᳚न् ब्राह्म॒णेन॑ । \newline
50. त्रि॒वृतेति॑ त्रि - वृता᳚ । \newline
51. स्तोमा᳚न् ब्राह्म॒णेन॑ ब्राह्म॒णेन॒ स्तोमा॒न् थ्स्तोमा᳚न् ब्राह्म॒णेन॒ वाचं॒ ॅवाच॑म् ब्राह्म॒णेन॒ स्तोमा॒न् थ्स्तोमा᳚न् ब्राह्म॒णेन॒ वाच᳚म् । \newline
52. ब्रा॒ह्म॒णेन॒ वाचं॒ ॅवाच॑म् ब्राह्म॒णेन॑ ब्राह्म॒णेन॒ वाच᳚म् । \newline
53. वाच॒मिति॒ वाच᳚म् । \newline
\pagebreak
\markright{ TS 7.3.15.1  \hfill https://www.vedavms.in \hfill}

\section{ TS 7.3.15.1 }

\textbf{TS 7.3.15.1 } \newline
\textbf{Samhita Paata} \newline

स्वाहा॒ऽऽधिमाधी॑ताय॒ स्वाहा॒ स्वाहाऽऽधी॑तं॒ मन॑से॒ स्वाहा॒ स्वाहा॒ मनः॑ प्र॒जाप॑तये॒ स्वाहा॒ काय॒ स्वाहा॒ कस्मै॒ स्वाहा॑ कत॒मस्मै॒ स्वाहा ऽदि॑त्यै॒ स्वाहा ऽदि॑त्यै म॒ह्यै᳚ स्वाहाऽदि॑त्यै सुमृडी॒कायै॒ स्वाहा॒ सर॑स्वत्यै॒ स्वाहा॒ सर॑स्वत्यै बृह॒त्यै᳚ स्वाहा॒ सर॑स्वत्यै पाव॒कायै॒ स्वाहा॑ पू॒ष्णे स्वाहा॑ पू॒ष्णे प्र॑प॒थ्या॑य॒ स्वाहा॑ पू॒ष्णे न॒रन्धि॑षाय॒ स्वाहा॒ त्वष्ट्रे॒ स्वाहा॒ त्वष्ट्रे॑ तु॒रीपा॑य॒ स्वाहा॒ त्वष्ट्रे॑ पुरु॒रूपा॑य॒ स्वाहा॒ ( ) विष्ण॑वे॒ स्वाहा॒ विष्ण॑वे निखुर्य॒पाय॒ स्वाहा॒ विष्ण॑वे निभूय॒पाय॒ स्वाहा॒ सर्व॑स्मै॒ स्वाहा᳚ ॥ \newline

\textbf{Pada Paata} \newline

स्वाहा᳚ । आ॒धिमित्या᳚ - धिम् । आधी॑ता॒येत्या - धी॒ता॒य॒ । स्वाहा᳚ । स्वाहा᳚ । आधी॑त॒मित्या - धी॒त॒म् । मन॑से । स्वाहा᳚ । स्वाहा᳚ । मनः॑ । प्र॒जाप॑तय॒ इति॑ प्र॒जा - प॒त॒ये॒ । स्वाहा᳚ । काय॑ । स्वाहा᳚ । कस्मै᳚ । स्वाहा᳚ । क॒त॒मस्मै᳚ । स्वाहा᳚ । अदि॑त्यै । स्वाहा᳚ । अदि॑त्यै । म॒ह्यै᳚ । स्वाहा᳚ । अदि॑त्यै । सु॒मृ॒डी॒काया॒ इति॑ सु - मृ॒डी॒कायै᳚ । स्वाहा᳚ । सर॑स्वत्यै । स्वाहा᳚ । सर॑स्वत्यै । बृ॒ह॒त्यै᳚ । स्वाहा᳚ । सर॑स्वत्यै । पा॒व॒कायै᳚ । स्वाहा᳚ । पू॒ष्णे । स्वाहा᳚ । पू॒ष्णे । प्र॒प॒थ्या॑येति॑ प्र-प॒थ्या॑य । स्वाहा᳚ । पू॒ष्णे । न॒रन्धि॑षाय । स्वाहा᳚ । त्वष्ट्रे᳚ । स्वाहा᳚ । त्वष्ट्रे᳚ । तु॒रीपा॑य । स्वाहा᳚ । त्वष्ट्रे᳚ । पु॒रु॒रूपा॒येति॑ पुरु - रूपा॑य । स्वाहा᳚ ( ) । विष्ण॑वे । स्वाहा᳚ । विष्ण॑वे । नि॒खु॒र्य॒पायेति॑ निखुर्य - पाय॑ । स्वाहा᳚ । विष्ण॑वे । नि॒भू॒य॒पायेति॑ निभूय - पाय॑ । स्वाहा᳚ । सर्व॑स्मै । स्वाहा᳚ ॥  \newline


\textbf{Krama Paata} \newline

स्वाहा॒ऽऽधिम् । आ॒धिमाधी॑ताय । आ॒धिमित्या᳚ - धिम् । आधी॑ताय॒ स्वाहा᳚ । आधी॑ता॒येत्या - धी॒ता॒य॒ । स्वाहा॒ स्वाहा᳚ । स्वाहाऽऽधी॑तम् । आधी॑त॒म् मन॑से । आधी॑त॒मित्या - धी॒त॒म् । मन॑से॒ स्वाहा᳚ । स्वाहा॒ स्वाहा᳚ । स्वाहा॒ मनः॑ । मनः॑ प्र॒जाप॑तये । प्र॒जाप॑तये॒ स्वाहा᳚ । प्र॒जाप॑तय॒ इति॑ प्र॒जा - प॒त॒ये॒ । स्वाहा॒ काय॑ । काय॒ स्वाहा᳚ । स्वाहा॒ कस्मै᳚ । कस्मै॒ स्वाहा᳚ । स्वाहा॑ कत॒मस्मै᳚ । क॒त॒मस्मै॒ स्वाहा᳚ । स्वाहाऽदि॑त्यै । अदि॑त्यै॒ स्वाहा᳚ । स्वाहाऽदि॑त्यै । अदि॑त्यै म॒ह्यै᳚ । म॒ह्यै᳚ स्वाहा᳚ । स्वाहाऽदि॑त्यै । अदि॑त्यै सुमृडी॒कायै᳚ । सु॒मृ॒डी॒कायै॒ स्वाहा᳚ । सु॒मृ॒डी॒काया॒ इति॑ सु - मृ॒डी॒कायै᳚ । स्वाहा॒ सर॑स्वत्यै । सर॑स्वत्यै॒ स्वाहा᳚ । स्वाहा॒ सर॑स्वत्यै । सर॑स्वत्यै बृह॒त्यै᳚ । बृ॒ह॒त्यै᳚ स्वाहा᳚ । स्वाहा॒ सर॑स्वत्यै । सर॑स्वत्यै पाव॒कायै᳚ । पा॒व॒कायै॒ स्वाहा᳚ । स्वाहा॑ पू॒ष्णे । पू॒ष्णे स्वाहा᳚ । स्वाहा॑ पू॒ष्णे । पू॒ष्णे प्र॑प॒थ्या॑य । प्र॒प॒थ्या॑य॒ स्वाहा᳚ । प्र॒प॒थ्या॑येति॑ प्र - प॒थ्या॑य । स्वाहा॑ पू॒ष्णे । पू॒ष्णे न॒रन्धि॑षाय । न॒रन्धि॑षाय॒ स्वाहा᳚ । स्वाहा॒ त्वष्ट्रे᳚ । त्वष्ट्रे॒ स्वाहा᳚ । स्वाहा॒ त्वष्ट्रे᳚ । त्वष्ट्रे॑ तु॒रीपा॑य । तु॒रीपा॑य॒ स्वाहा᳚ । स्वाहा॒ त्वष्ट्रे᳚ । त्वष्ट्रे॑ पुरु॒रूपा॑य । पु॒रु॒रूपा॑य॒ स्वाहा᳚ ( ) । पु॒रु॒रूपा॒येति॑ पुरु - रूपा॑य । स्वाहा॒ विष्ण॑वे । विष्ण॑वे॒ स्वाहा᳚ । स्वाहा॒ विष्ण॑वे । विष्ण॑वे निखुर्य॒पाय॑ । नि॒खु॒र्य॒पाय॒ स्वाहा᳚ । नि॒खु॒र्य॒पायेति॑ निखुर्य - पाय॑ । स्वाहा॒ विष्ण॑वे । विष्ण॑वे निभूय॒पाय॑ । नि॒भू॒य॒पाय॒ स्वाहा᳚ । नि॒भू॒य॒पायेति॑ निभूय - पाय॑ । स्वाहा॒ सर्व॑स्मै । सर्व॑स्मै॒ स्वाहा᳚ । स्वाहेति॒ स्वाहा᳚ । \newline

\textbf{Jatai Paata} \newline

1. स्वाहा॒ ऽऽधि मा॒धिꣳ स्वाहा॒ स्वाहा॒ ऽऽधिम् । \newline
2. आ॒धि माधी॑ता॒या धी॑ताया॒ धि मा॒धि माधी॑ताय । \newline
3. आ॒धिमित्या᳚ - धिम् । \newline
4. आधी॑ताय॒ स्वाहा॒ स्वाहा ऽऽधी॑ता॒या धी॑ताय॒ स्वाहा᳚ । \newline
5. आधी॑ता॒येत्या - धी॒ता॒य॒ । \newline
6. स्वाहा॒ स्वाहा᳚ । \newline
7. स्वाहा ऽऽधी॑त॒ माधी॑तꣳ॒॒ स्वाहा॒ स्वाहा ऽऽधी॑तम् । \newline
8. आधी॑त॒म् मन॑से॒ मन॑स॒ आधी॑त॒ माधी॑त॒म् मन॑से । \newline
9. आधी॑त॒मित्या - धी॒त॒म् । \newline
10. मन॑से॒ स्वाहा॒ स्वाहा॒ मन॑से॒ मन॑से॒ स्वाहा᳚ । \newline
11. स्वाहा॒ स्वाहा᳚ । \newline
12. स्वाहा॒ मनो॒ मनः॒ स्वाहा॒ स्वाहा॒ मनः॑ । \newline
13. मनः॑ प्र॒जाप॑तये प्र॒जाप॑तये॒ मनो॒ मनः॑ प्र॒जाप॑तये । \newline
14. प्र॒जाप॑तये॒ स्वाहा॒ स्वाहा᳚ प्र॒जाप॑तये प्र॒जाप॑तये॒ स्वाहा᳚ । \newline
15. प्र॒जाप॑तय॒ इति॑ प्र॒जा - प॒त॒ये॒ । \newline
16. स्वाहा॒ काय॒ काय॒ स्वाहा॒ स्वाहा॒ काय॑ । \newline
17. काय॒ स्वाहा॒ स्वाहा॒ काय॒ काय॒ स्वाहा᳚ । \newline
18. स्वाहा॒ कस्मै॒ कस्मै॒ स्वाहा॒ स्वाहा॒ कस्मै᳚ । \newline
19. कस्मै॒ स्वाहा॒ स्वाहा॒ कस्मै॒ कस्मै॒ स्वाहा᳚ । \newline
20. स्वाहा॑ कत॒मस्मै॑ कत॒मस्मै॒ स्वाहा॒ स्वाहा॑ कत॒मस्मै᳚ । \newline
21. क॒त॒मस्मै॒ स्वाहा॒ स्वाहा॑ कत॒मस्मै॑ कत॒मस्मै॒ स्वाहा᳚ । \newline
22. स्वाहा ऽदि॑त्या॒ अदि॑त्यै॒ स्वाहा॒ स्वाहा ऽदि॑त्यै । \newline
23. अदि॑त्यै॒ स्वाहा॒ स्वाहा ऽदि॑त्या॒ अदि॑त्यै॒ स्वाहा᳚ । \newline
24. स्वाहा ऽदि॑त्या॒ अदि॑त्यै॒ स्वाहा॒ स्वाहा ऽदि॑त्यै । \newline
25. अदि॑त्यै म॒ह्यै॑ म॒ह्या॑ अदि॑त्या॒ अदि॑त्यै म॒ह्यै᳚ । \newline
26. म॒ह्यै᳚ स्वाहा॒ स्वाहा॑ म॒ह्यै॑ म॒ह्यै᳚ स्वाहा᳚ । \newline
27. स्वाहा ऽदि॑त्या॒ अदि॑त्यै॒ स्वाहा॒ स्वाहा ऽदि॑त्यै । \newline
28. अदि॑त्यै सुमृडी॒कायै॑ सुमृडी॒काया॒ अदि॑त्या॒ अदि॑त्यै सुमृडी॒कायै᳚ । \newline
29. सु॒मृ॒डी॒कायै॒ स्वाहा॒ स्वाहा॑ सुमृडी॒कायै॑ सुमृडी॒कायै॒ स्वाहा᳚ । \newline
30. सु॒मृ॒डी॒काया॒ इति॑ सु - मृ॒डी॒कायै᳚ । \newline
31. स्वाहा॒ सर॑स्वत्यै॒ सर॑स्वत्यै॒ स्वाहा॒ स्वाहा॒ सर॑स्वत्यै । \newline
32. सर॑स्वत्यै॒ स्वाहा॒ स्वाहा॒ सर॑स्वत्यै॒ सर॑स्वत्यै॒ स्वाहा᳚ । \newline
33. स्वाहा॒ सर॑स्वत्यै॒ सर॑स्वत्यै॒ स्वाहा॒ स्वाहा॒ सर॑स्वत्यै । \newline
34. सर॑स्वत्यै बृह॒त्यै॑ बृह॒त्यै॑ सर॑स्वत्यै॒ सर॑स्वत्यै बृह॒त्यै᳚ । \newline
35. बृ॒ह॒त्यै᳚ स्वाहा॒ स्वाहा॑ बृह॒त्यै॑ बृह॒त्यै᳚ स्वाहा᳚ । \newline
36. स्वाहा॒ सर॑स्वत्यै॒ सर॑स्वत्यै॒ स्वाहा॒ स्वाहा॒ सर॑स्वत्यै । \newline
37. सर॑स्वत्यै पाव॒कायै॑ पाव॒कायै॒ सर॑स्वत्यै॒ सर॑स्वत्यै पाव॒कायै᳚ । \newline
38. पा॒व॒कायै॒ स्वाहा॒ स्वाहा॑ पाव॒कायै॑ पाव॒कायै॒ स्वाहा᳚ । \newline
39. स्वाहा॑ पू॒ष्णे पू॒ष्णे स्वाहा॒ स्वाहा॑ पू॒ष्णे । \newline
40. पू॒ष्णे स्वाहा॒ स्वाहा॑ पू॒ष्णे पू॒ष्णे स्वाहा᳚ । \newline
41. स्वाहा॑ पू॒ष्णे पू॒ष्णे स्वाहा॒ स्वाहा॑ पू॒ष्णे । \newline
42. पू॒ष्णे प्र॑प॒थ्या॑य प्रप॒थ्या॑य पू॒ष्णे पू॒ष्णे प्र॑प॒थ्या॑य । \newline
43. प्र॒प॒थ्या॑य॒ स्वाहा॒ स्वाहा᳚ प्रप॒थ्या॑य प्रप॒थ्या॑य॒ स्वाहा᳚ । \newline
44. प्र॒प॒थ्या॑येति॑ प्र - प॒थ्या॑य । \newline
45. स्वाहा॑ पू॒ष्णे पू॒ष्णे स्वाहा॒ स्वाहा॑ पू॒ष्णे । \newline
46. पू॒ष्णे न॒रन्धि॑षाय न॒रन्धि॑षाय पू॒ष्णे पू॒ष्णे न॒रन्धि॑षाय । \newline
47. न॒रन्धि॑षाय॒ स्वाहा॒ स्वाहा॑ न॒रन्धि॑षाय न॒रन्धि॑षाय॒ स्वाहा᳚ । \newline
48. स्वाहा॒ त्वष्ट्रे॒ त्वष्ट्रे॒ स्वाहा॒ स्वाहा॒ त्वष्ट्रे᳚ । \newline
49. त्वष्ट्रे॒ स्वाहा॒ स्वाहा॒ त्वष्ट्रे॒ त्वष्ट्रे॒ स्वाहा᳚ । \newline
50. स्वाहा॒ त्वष्ट्रे॒ त्वष्ट्रे॒ स्वाहा॒ स्वाहा॒ त्वष्ट्रे᳚ । \newline
51. त्वष्ट्रे॑ तु॒रीपा॑य तु॒रीपा॑य॒ त्वष्ट्रे॒ त्वष्ट्रे॑ तु॒रीपा॑य । \newline
52. तु॒रीपा॑य॒ स्वाहा॒ स्वाहा॑ तु॒रीपा॑य तु॒रीपा॑य॒ स्वाहा᳚ । \newline
53. स्वाहा॒ त्वष्ट्रे॒ त्वष्ट्रे॒ स्वाहा॒ स्वाहा॒ त्वष्ट्रे᳚ । \newline
54. त्वष्ट्रे॑ पुरु॒रूपा॑य पुरु॒रूपा॑य॒ त्वष्ट्रे॒ त्वष्ट्रे॑ पुरु॒रूपा॑य । \newline
55. पु॒रु॒रूपा॑य॒ स्वाहा॒ स्वाहा॑ पुरु॒रूपा॑य पुरु॒रूपा॑य॒ स्वाहा᳚ । \newline
56. पु॒रु॒रूपा॒येति॑ पुरु - रूपा॑य । \newline
57. स्वाहा॒ विष्ण॑वे॒ विष्ण॑वे॒ स्वाहा॒ स्वाहा॒ विष्ण॑वे । \newline
58. विष्ण॑वे॒ स्वाहा॒ स्वाहा॒ विष्ण॑वे॒ विष्ण॑वे॒ स्वाहा᳚ । \newline
59. स्वाहा॒ विष्ण॑वे॒ विष्ण॑वे॒ स्वाहा॒ स्वाहा॒ विष्ण॑वे । \newline
60. विष्ण॑वे निखुर्य॒पाय॑ निखुर्य॒पाय॒ विष्ण॑वे॒ विष्ण॑वे निखुर्य॒पाय॑ । \newline
61. नि॒खु॒र्य॒पाय॒ स्वाहा॒ स्वाहा॑ निखुर्य॒पाय॑ निखुर्य॒पाय॒ स्वाहा᳚ । \newline
62. नि॒खु॒र्य॒पायेति॑ निखुर्य - पाय॑ । \newline
63. स्वाहा॒ विष्ण॑वे॒ विष्ण॑वे॒ स्वाहा॒ स्वाहा॒ विष्ण॑वे । \newline
64. विष्ण॑वे निभूय॒पाय॑ निभूय॒पाय॒ विष्ण॑वे॒ विष्ण॑वे निभूय॒पाय॑ । \newline
65. नि॒भू॒य॒पाय॒ स्वाहा॒ स्वाहा॑ निभूय॒पाय॑ निभूय॒पाय॒ स्वाहा᳚ । \newline
66. नि॒भू॒य॒पायेति॑ निभूय - पाय॑ । \newline
67. स्वाहा॒ सर्व॑स्मै॒ सर्व॑स्मै॒ स्वाहा॒ स्वाहा॒ सर्व॑स्मै । \newline
68. सर्व॑स्मै॒ स्वाहा॒ स्वाहा॒ सर्व॑स्मै॒ सर्व॑स्मै॒ स्वाहा᳚ । \newline
69. स्वाहेति॒ स्वाहा᳚ । \newline

\textbf{Ghana Paata } \newline

1. स्वाहा॒ ऽऽधि मा॒धिꣳ स्वाहा॒ स्वाहा॒ ऽऽधि माधी॑ता॒या धी॑ताया॒ धिꣳ स्वाहा॒ स्वाहा॒ ऽऽधि माधी॑ताय । \newline
2. आ॒धि माधी॑ता॒या धी॑ताया॒धि मा॒धि माधी॑ताय॒ स्वाहा॒ स्वाहा ऽऽधी॑ताया॒ धि मा॒धि माधी॑ताय॒ स्वाहा᳚ । \newline
3. आ॒धिमित्या᳚ - धिम् । \newline
4. आधी॑ताय॒ स्वाहा॒ स्वाहा ऽऽधी॑ता॒या धी॑ताय॒ स्वाहा᳚ । \newline
5. आधी॑ता॒येत्या - धी॒ता॒य॒ । \newline
6. स्वाहा॒ स्वाहा᳚ । \newline
7. स्वाहा ऽऽधी॑त॒ माधी॑तꣳ॒॒ स्वाहा॒ स्वाहा ऽऽधी॑त॒म् मन॑से॒ मन॑स॒ आधी॑तꣳ॒॒ स्वाहा॒ स्वाहा ऽऽधी॑त॒म् मन॑से । \newline
8. आधी॑त॒म् मन॑से॒ मन॑स॒ आधी॑त॒ माधी॑त॒म् मन॑से॒ स्वाहा॒ स्वाहा॒ मन॑स॒ आधी॑त॒ माधी॑त॒म् मन॑से॒ स्वाहा᳚ । \newline
9. आधी॑त॒मित्या - धी॒त॒म् । \newline
10. मन॑से॒ स्वाहा॒ स्वाहा॒ मन॑से॒ मन॑से॒ स्वाहा᳚ । \newline
11. स्वाहा॒ स्वाहा᳚ । \newline
12. स्वाहा॒ मनो॒ मनः॒ स्वाहा॒ स्वाहा॒ मनः॑ प्र॒जाप॑तये प्र॒जाप॑तये॒ मनः॒ स्वाहा॒ स्वाहा॒ मनः॑ प्र॒जाप॑तये । \newline
13. मनः॑ प्र॒जाप॑तये प्र॒जाप॑तये॒ मनो॒ मनः॑ प्र॒जाप॑तये॒ स्वाहा॒ स्वाहा᳚ प्र॒जाप॑तये॒ मनो॒ मनः॑ प्र॒जाप॑तये॒ स्वाहा᳚ । \newline
14. प्र॒जाप॑तये॒ स्वाहा॒ स्वाहा᳚ प्र॒जाप॑तये प्र॒जाप॑तये॒ स्वाहा॒ काय॒ काय॒ स्वाहा᳚ प्र॒जाप॑तये प्र॒जाप॑तये॒ स्वाहा॒ काय॑ । \newline
15. प्र॒जाप॑तय॒ इति॑ प्र॒जा - प॒त॒ये॒ । \newline
16. स्वाहा॒ काय॒ काय॒ स्वाहा॒ स्वाहा॒ काय॒ स्वाहा॒ स्वाहा॒ काय॒ स्वाहा॒ स्वाहा॒ काय॒ स्वाहा᳚ । \newline
17. काय॒ स्वाहा॒ स्वाहा॒ काय॒ काय॒ स्वाहा॒ कस्मै॒ कस्मै॒ स्वाहा॒ काय॒ काय॒ स्वाहा॒ कस्मै᳚ । \newline
18. स्वाहा॒ कस्मै॒ कस्मै॒ स्वाहा॒ स्वाहा॒ कस्मै॒ स्वाहा॒ स्वाहा॒ कस्मै॒ स्वाहा॒ स्वाहा॒ कस्मै॒ स्वाहा᳚ । \newline
19. कस्मै॒ स्वाहा॒ स्वाहा॒ कस्मै॒ कस्मै॒ स्वाहा॑ कत॒मस्मै॑ कत॒मस्मै॒ स्वाहा॒ कस्मै॒ कस्मै॒ स्वाहा॑ कत॒मस्मै᳚ । \newline
20. स्वाहा॑ कत॒मस्मै॑ कत॒मस्मै॒ स्वाहा॒ स्वाहा॑ कत॒मस्मै॒ स्वाहा॒ स्वाहा॑ कत॒मस्मै॒ स्वाहा॒ स्वाहा॑ कत॒मस्मै॒ स्वाहा᳚ । \newline
21. क॒त॒मस्मै॒ स्वाहा॒ स्वाहा॑ कत॒मस्मै॑ कत॒मस्मै॒ स्वाहा ऽदि॑त्या॒ अदि॑त्यै॒ स्वाहा॑ कत॒मस्मै॑ कत॒मस्मै॒ स्वाहा ऽदि॑त्यै । \newline
22. स्वाहा ऽदि॑त्या॒ अदि॑त्यै॒ स्वाहा॒ स्वाहा ऽदि॑त्यै॒ स्वाहा॒ स्वाहा ऽदि॑त्यै॒ स्वाहा॒ स्वाहा ऽदि॑त्यै॒ स्वाहा᳚ । \newline
23. अदि॑त्यै॒ स्वाहा॒ स्वाहा ऽदि॑त्या॒ अदि॑त्यै॒ स्वाहा ऽदि॑त्या॒ अदि॑त्यै॒ स्वाहा ऽदि॑त्या॒ अदि॑त्यै॒ स्वाहा ऽदि॑त्यै । \newline
24. स्वाहा ऽदि॑त्या॒ अदि॑त्यै॒ स्वाहा॒ स्वाहा ऽदि॑त्यै म॒ह्यै॑ म॒ह्या॑ अदि॑त्यै॒ स्वाहा॒ स्वाहा ऽदि॑त्यै म॒ह्यै᳚ । \newline
25. अदि॑त्यै म॒ह्यै॑ म॒ह्या॑ अदि॑त्या॒ अदि॑त्यै म॒ह्यै᳚ स्वाहा॒ स्वाहा॑ म॒ह्या॑ अदि॑त्या॒ अदि॑त्यै म॒ह्यै᳚ स्वाहा᳚ । \newline
26. म॒ह्यै᳚ स्वाहा॒ स्वाहा॑ म॒ह्यै॑ म॒ह्यै᳚ स्वाहा ऽदि॑त्या॒ अदि॑त्यै॒ स्वाहा॑ म॒ह्यै॑ म॒ह्यै᳚ स्वाहा ऽदि॑त्यै । \newline
27. स्वाहा ऽदि॑त्या॒ अदि॑त्यै॒ स्वाहा॒ स्वाहा ऽदि॑त्यै सुमृडी॒कायै॑ सुमृडी॒काया॒ अदि॑त्यै॒ स्वाहा॒ स्वाहा ऽदि॑त्यै सुमृडी॒कायै᳚ । \newline
28. अदि॑त्यै सुमृडी॒कायै॑ सुमृडी॒काया॒ अदि॑त्या॒ अदि॑त्यै सुमृडी॒कायै॒ स्वाहा॒ स्वाहा॑ सुमृडी॒काया॒ अदि॑त्या॒ अदि॑त्यै सुमृडी॒कायै॒ स्वाहा᳚ । \newline
29. सु॒मृ॒डी॒कायै॒ स्वाहा॒ स्वाहा॑ सुमृडी॒कायै॑ सुमृडी॒कायै॒ स्वाहा॒ सर॑स्वत्यै॒ सर॑स्वत्यै॒ स्वाहा॑ सुमृडी॒कायै॑ सुमृडी॒कायै॒ स्वाहा॒ सर॑स्वत्यै । \newline
30. सु॒मृ॒डी॒काया॒ इति॑ सु - मृ॒डी॒कायै᳚ । \newline
31. स्वाहा॒ सर॑स्वत्यै॒ सर॑स्वत्यै॒ स्वाहा॒ स्वाहा॒ सर॑स्वत्यै॒ स्वाहा॒ स्वाहा॒ सर॑स्वत्यै॒ स्वाहा॒ स्वाहा॒ सर॑स्वत्यै॒ स्वाहा᳚ । \newline
32. सर॑स्वत्यै॒ स्वाहा॒ स्वाहा॒ सर॑स्वत्यै॒ सर॑स्वत्यै॒ स्वाहा॒ सर॑स्वत्यै॒ सर॑स्वत्यै॒ स्वाहा॒ सर॑स्वत्यै॒ सर॑स्वत्यै॒ स्वाहा॒ सर॑स्वत्यै । \newline
33. स्वाहा॒ सर॑स्वत्यै॒ सर॑स्वत्यै॒ स्वाहा॒ स्वाहा॒ सर॑स्वत्यै बृह॒त्यै॑ बृह॒त्यै॑ सर॑स्वत्यै॒ स्वाहा॒ स्वाहा॒ सर॑स्वत्यै बृह॒त्यै᳚ । \newline
34. सर॑स्वत्यै बृह॒त्यै॑ बृह॒त्यै॑ सर॑स्वत्यै॒ सर॑स्वत्यै बृह॒त्यै᳚ स्वाहा॒ स्वाहा॑ बृह॒त्यै॑ सर॑स्वत्यै॒ सर॑स्वत्यै बृह॒त्यै᳚ स्वाहा᳚ । \newline
35. बृ॒ह॒त्यै᳚ स्वाहा॒ स्वाहा॑ बृह॒त्यै॑ बृह॒त्यै᳚ स्वाहा॒ सर॑स्वत्यै॒ सर॑स्वत्यै॒ स्वाहा॑ बृह॒त्यै॑ बृह॒त्यै᳚ स्वाहा॒ सर॑स्वत्यै । \newline
36. स्वाहा॒ सर॑स्वत्यै॒ सर॑स्वत्यै॒ स्वाहा॒ स्वाहा॒ सर॑स्वत्यै पाव॒कायै॑ पाव॒कायै॒ सर॑स्वत्यै॒ स्वाहा॒ स्वाहा॒ सर॑स्वत्यै पाव॒कायै᳚ । \newline
37. सर॑स्वत्यै पाव॒कायै॑ पाव॒कायै॒ सर॑स्वत्यै॒ सर॑स्वत्यै पाव॒कायै॒ स्वाहा॒ स्वाहा॑ पाव॒कायै॒ सर॑स्वत्यै॒ सर॑स्वत्यै पाव॒कायै॒ स्वाहा᳚ । \newline
38. पा॒व॒कायै॒ स्वाहा॒ स्वाहा॑ पाव॒कायै॑ पाव॒कायै॒ स्वाहा॑ पू॒ष्णे पू॒ष्णे स्वाहा॑ पाव॒कायै॑ पाव॒कायै॒ स्वाहा॑ पू॒ष्णे । \newline
39. स्वाहा॑ पू॒ष्णे पू॒ष्णे स्वाहा॒ स्वाहा॑ पू॒ष्णे स्वाहा॒ स्वाहा॑ पू॒ष्णे स्वाहा॒ स्वाहा॑ पू॒ष्णे स्वाहा᳚ । \newline
40. पू॒ष्णे स्वाहा॒ स्वाहा॑ पू॒ष्णे पू॒ष्णे स्वाहा॑ पू॒ष्णे पू॒ष्णे स्वाहा॑ पू॒ष्णे पू॒ष्णे स्वाहा॑ पू॒ष्णे । \newline
41. स्वाहा॑ पू॒ष्णे पू॒ष्णे स्वाहा॒ स्वाहा॑ पू॒ष्णे प्र॑प॒थ्या॑य प्रप॒थ्या॑य पू॒ष्णे स्वाहा॒ स्वाहा॑ पू॒ष्णे प्र॑प॒थ्या॑य । \newline
42. पू॒ष्णे प्र॑प॒थ्या॑य प्रप॒थ्या॑य पू॒ष्णे पू॒ष्णे प्र॑प॒थ्या॑य॒ स्वाहा॒ स्वाहा᳚ प्रप॒थ्या॑य पू॒ष्णे पू॒ष्णे प्र॑प॒थ्या॑य॒ स्वाहा᳚ । \newline
43. प्र॒प॒थ्या॑य॒ स्वाहा॒ स्वाहा᳚ प्रप॒थ्या॑य प्रप॒थ्या॑य॒ स्वाहा॑ पू॒ष्णे पू॒ष्णे स्वाहा᳚ प्रप॒थ्या॑य प्रप॒थ्या॑य॒ स्वाहा॑ पू॒ष्णे । \newline
44. प्र॒प॒थ्या॑येति॑ प्र - प॒थ्या॑य । \newline
45. स्वाहा॑ पू॒ष्णे पू॒ष्णे स्वाहा॒ स्वाहा॑ पू॒ष्णे न॒रन्धि॑षाय न॒रन्धि॑षाय पू॒ष्णे स्वाहा॒ स्वाहा॑ पू॒ष्णे न॒रन्धि॑षाय । \newline
46. पू॒ष्णे न॒रन्धि॑षाय न॒रन्धि॑षाय पू॒ष्णे पू॒ष्णे न॒रन्धि॑षाय॒ स्वाहा॒ स्वाहा॑ न॒रन्धि॑षाय पू॒ष्णे पू॒ष्णे न॒रन्धि॑षाय॒ स्वाहा᳚ । \newline
47. न॒रन्धि॑षाय॒ स्वाहा॒ स्वाहा॑ न॒रन्धि॑षाय न॒रन्धि॑षाय॒ स्वाहा॒ त्वष्ट्रे॒ त्वष्ट्रे॒ स्वाहा॑ न॒रन्धि॑षाय न॒रन्धि॑षाय॒ स्वाहा॒ त्वष्ट्रे᳚ । \newline
48. स्वाहा॒ त्वष्ट्रे॒ त्वष्ट्रे॒ स्वाहा॒ स्वाहा॒ त्वष्ट्रे॒ स्वाहा॒ स्वाहा॒ त्वष्ट्रे॒ स्वाहा॒ स्वाहा॒ त्वष्ट्रे॒ स्वाहा᳚ । \newline
49. त्वष्ट्रे॒ स्वाहा॒ स्वाहा॒ त्वष्ट्रे॒ त्वष्ट्रे॒ स्वाहा॒ त्वष्ट्रे॒ त्वष्ट्रे॒ स्वाहा॒ त्वष्ट्रे॒ त्वष्ट्रे॒ स्वाहा॒ त्वष्ट्रे᳚ । \newline
50. स्वाहा॒ त्वष्ट्रे॒ त्वष्ट्रे॒ स्वाहा॒ स्वाहा॒ त्वष्ट्रे॑ तु॒रीपा॑य तु॒रीपा॑य॒ त्वष्ट्रे॒ स्वाहा॒ स्वाहा॒ त्वष्ट्रे॑ तु॒रीपा॑य । \newline
51. त्वष्ट्रे॑ तु॒रीपा॑य तु॒रीपा॑य॒ त्वष्ट्रे॒ त्वष्ट्रे॑ तु॒रीपा॑य॒ स्वाहा॒ स्वाहा॑ तु॒रीपा॑य॒ त्वष्ट्रे॒ त्वष्ट्रे॑ तु॒रीपा॑य॒ स्वाहा᳚ । \newline
52. तु॒रीपा॑य॒ स्वाहा॒ स्वाहा॑ तु॒रीपा॑य तु॒रीपा॑य॒ स्वाहा॒ त्वष्ट्रे॒ त्वष्ट्रे॒ स्वाहा॑ तु॒रीपा॑य तु॒रीपा॑य॒ स्वाहा॒ त्वष्ट्रे᳚ । \newline
53. स्वाहा॒ त्वष्ट्रे॒ त्वष्ट्रे॒ स्वाहा॒ स्वाहा॒ त्वष्ट्रे॑ पुरु॒रूपा॑य पुरु॒रूपा॑य॒ त्वष्ट्रे॒ स्वाहा॒ स्वाहा॒ त्वष्ट्रे॑ पुरु॒रूपा॑य । \newline
54. त्वष्ट्रे॑ पुरु॒रूपा॑य पुरु॒रूपा॑य॒ त्वष्ट्रे॒ त्वष्ट्रे॑ पुरु॒रूपा॑य॒ स्वाहा॒ स्वाहा॑ पुरु॒रूपा॑य॒ त्वष्ट्रे॒ त्वष्ट्रे॑ पुरु॒रूपा॑य॒ स्वाहा᳚ । \newline
55. पु॒रु॒रूपा॑य॒ स्वाहा॒ स्वाहा॑ पुरु॒रूपा॑य पुरु॒रूपा॑य॒ स्वाहा॒ विष्ण॑वे॒ विष्ण॑वे॒ स्वाहा॑ पुरु॒रूपा॑य पुरु॒रूपा॑य॒ स्वाहा॒ विष्ण॑वे । \newline
56. पु॒रु॒रूपा॒येति॑ पुरु - रूपा॑य । \newline
57. स्वाहा॒ विष्ण॑वे॒ विष्ण॑वे॒ स्वाहा॒ स्वाहा॒ विष्ण॑वे॒ स्वाहा॒ स्वाहा॒ विष्ण॑वे॒ स्वाहा॒ स्वाहा॒ विष्ण॑वे॒ स्वाहा᳚ । \newline
58. विष्ण॑वे॒ स्वाहा॒ स्वाहा॒ विष्ण॑वे॒ विष्ण॑वे॒ स्वाहा॒ विष्ण॑वे॒ विष्ण॑वे॒ स्वाहा॒ विष्ण॑वे॒ विष्ण॑वे॒ स्वाहा॒ विष्ण॑वे । \newline
59. स्वाहा॒ विष्ण॑वे॒ विष्ण॑वे॒ स्वाहा॒ स्वाहा॒ विष्ण॑वे निखुर्य॒पाय॑ निखुर्य॒पाय॒ विष्ण॑वे॒ स्वाहा॒ स्वाहा॒ विष्ण॑वे निखुर्य॒पाय॑ । \newline
60. विष्ण॑वे निखुर्य॒पाय॑ निखुर्य॒पाय॒ विष्ण॑वे॒ विष्ण॑वे निखुर्य॒पाय॒ स्वाहा॒ स्वाहा॑ निखुर्य॒पाय॒ विष्ण॑वे॒ विष्ण॑वे निखुर्य॒पाय॒ स्वाहा᳚ । \newline
61. नि॒खु॒र्य॒पाय॒ स्वाहा॒ स्वाहा॑ निखुर्य॒पाय॑ निखुर्य॒पाय॒ स्वाहा॒ विष्ण॑वे॒ विष्ण॑वे॒ स्वाहा॑ निखुर्य॒पाय॑ निखुर्य॒पाय॒ स्वाहा॒ विष्ण॑वे । \newline
62. नि॒खु॒र्य॒पायेति॑ निखुर्य - पाय॑ । \newline
63. स्वाहा॒ विष्ण॑वे॒ विष्ण॑वे॒ स्वाहा॒ स्वाहा॒ विष्ण॑वे निभूय॒पाय॑ निभूय॒पाय॒ विष्ण॑वे॒ स्वाहा॒ स्वाहा॒ विष्ण॑वे निभूय॒पाय॑ । \newline
64. विष्ण॑वे निभूय॒पाय॑ निभूय॒पाय॒ विष्ण॑वे॒ विष्ण॑वे निभूय॒पाय॒ स्वाहा॒ स्वाहा॑ निभूय॒पाय॒ विष्ण॑वे॒ विष्ण॑वे निभूय॒पाय॒ स्वाहा᳚ । \newline
65. नि॒भू॒य॒पाय॒ स्वाहा॒ स्वाहा॑ निभूय॒पाय॑ निभूय॒पाय॒ स्वाहा॒ सर्व॑स्मै॒ सर्व॑स्मै॒ स्वाहा॑ निभूय॒पाय॑ निभूय॒पाय॒ स्वाहा॒ सर्व॑स्मै । \newline
66. नि॒भू॒य॒पायेति॑ निभूय - पाय॑ । \newline
67. स्वाहा॒ सर्व॑स्मै॒ सर्व॑स्मै॒ स्वाहा॒ स्वाहा॒ सर्व॑स्मै॒ स्वाहा॒ स्वाहा॒ सर्व॑स्मै॒ स्वाहा॒ स्वाहा॒ सर्व॑स्मै॒ स्वाहा᳚ । \newline
68. सर्व॑स्मै॒ स्वाहा॒ स्वाहा॒ सर्व॑स्मै॒ सर्व॑स्मै॒ स्वाहा᳚ । \newline
69. स्वाहेति॒ स्वाहा᳚ । \newline
\pagebreak
\markright{ TS 7.3.16.1  \hfill https://www.vedavms.in \hfill}

\section{ TS 7.3.16.1 }

\textbf{TS 7.3.16.1 } \newline
\textbf{Samhita Paata} \newline

द॒द्भ्यः स्वाहा॒ हनू᳚भ्याꣳ॒॒ स्वाहोष्ठा᳚भ्याꣳ॒॒ स्वाहा॒ मुखा॑य॒ स्वाहा॒ नासि॑काभ्याꣳ॒॒ स्वाहा॒ ऽक्षीभ्याꣳ॒॒ स्वाहा॒ कर्णा᳚भ्याꣳ॒॒ स्वाहा॑ पा॒र इ॒क्षवो॑ऽवा॒र्ये᳚भ्यः॒ पक्ष्म॑भ्यः॒ स्वाहा॑ ऽवा॒र इ॒क्षवः॑ पा॒र्ये᳚भ्यः॒ पक्ष्म॑भ्यः॒ स्वाहा॑ शी॒र्ष्णे स्वाहा᳚ भ्रू॒भ्याꣳ स्वाहा॑ ल॒लाटा॑य॒ स्वाहा॑ मू॒र्द्ध्ने स्वाहा॑ म॒स्तिष्का॑य॒ स्वाहा॒ केशे᳚भ्यः॒ स्वाहा॒ वहा॑य॒ स्वाहा᳚ ग्री॒वाभ्यः॒ स्वाहा᳚ स्क॒न्धेभ्यः॒ स्वाहा॒ कीक॑साभ्यः॒ स्वाहा॑ पृ॒ष्टीभ्यः॒ स्वाहा॑ पाज॒स्या॑य॒ स्वाहा॑ पा॒र्श्वाभ्याꣳ॒॒ स्वाहा- [  ] \newline

\textbf{Pada Paata} \newline

द॒द्भ्य इति॑ दत् - भ्यः । स्वाहा᳚ । हनू᳚भ्या॒मिति॒ हनु॑ - भ्या॒म् । स्वाहा᳚ । ओष्ठा᳚भ्याम् । स्वाहा᳚ । मुखा॑य । स्वाहा᳚ । नासि॑काभ्याम् । स्वाहा᳚ । अ॒क्षीभ्या᳚म् । स्वाहा᳚ । कर्णा᳚भ्याम् । स्वाहा᳚ । पा॒रे । इ॒क्षवः॑ । अ॒वा॒र्ये᳚भ्यः । पक्ष्म॑भ्य॒ इति॒ पक्ष्म॑ - भ्यः॒ । स्वाहा᳚ । अ॒वा॒रे । इ॒क्षवः॑ । पा॒र्ये᳚भ्यः । पक्ष्म॑भ्य॒ इति॒ पक्ष्म॑ - भ्यः॒ । स्वाहा᳚ । शी॒र्ष्णे । स्वाहा᳚ । भ्रू॒भ्याम् । स्वाहा᳚ । ल॒लाटा॑य । स्वाहा᳚ । मू॒द्‌र्ध्ने । स्वाहा᳚ । म॒स्तिष्का॑य । स्वाहा᳚ । केशे᳚भ्यः । स्वाहा᳚ । वहा॑य । स्वाहा᳚ । ग्री॒वाभ्यः॑ । स्वाहा᳚ । स्क॒न्धेभ्यः॑ । स्वाहा᳚ । कीक॑साभ्यः । स्वाहा᳚ । पृ॒ष्टीभ्य॒ इति॑ पृ॒ष्टि - भ्यः॒ । स्वाहा᳚ । पा॒ज॒स्या॑य । स्वाहा᳚ । पा॒र्श्वाभ्या᳚म् । स्वाहा᳚ ।  \newline


\textbf{Krama Paata} \newline

द॒द्भ्यः स्वाहा᳚ । द॒द्भ्य इति॑ दत् - भ्यः । स्वाहा॒ हनू᳚भ्याम् । हनू᳚भ्याꣳ॒॒ स्वाहा᳚ । हनू᳚भ्या॒मिति॒ हनु॑ - भ्या॒म् । स्वाहोष्ठा᳚भ्याम् । ओष्ठा᳚भ्याꣳ॒॒ स्वाहा᳚ । स्वाहा॒ मुखा॑य । मुखा॑य॒ स्वाहा᳚ । स्वाहा॒ नासि॑काभ्याम् । नासि॑काभ्याꣳ॒॒ स्वाहा᳚ । स्वाहा॒ऽक्षीभ्या᳚म् । अ॒क्षीभ्याꣳ॒॒ स्वाहा᳚ । स्वाहा॒ कर्णा᳚भ्याम् । कर्णा᳚भ्याꣳ॒॒ स्वाहा᳚ । स्वाहा॑ पा॒रे । पा॒र इ॒क्षवः॑ । इ॒क्षवो॑ऽवा॒र्ये᳚भ्यः । अ॒वा॒र्ये᳚भ्यः॒ पक्ष्म॑भ्यः । पक्ष्म॑भ्यः॒ स्वाहा᳚ । पक्ष्म॑भ्य॒ इति॒ पक्ष्म॑ - भ्यः॒ । स्वाहा॑ऽवा॒रे । अ॒वा॒र इ॒क्षवः॑ । इ॒क्षवः॑ पा॒र्ये᳚भ्यः । पा॒र्ये᳚भ्यः॒ पक्ष्म॑भ्यः । पक्ष्म॑भ्यः॒ स्वाहा᳚ । पक्ष्म॑भ्य॒ इति॒ पक्ष्म॑ - भ्यः॒ । स्वाहा॑ शी॒र्ष्णे । शी॒र्ष्णे स्वाहा᳚ । स्वाहा᳚ भ्रू॒भ्याम् । भ्रू॒भ्याꣳ स्वाहा᳚ । स्वाहा॑ ल॒लाटा॑य । ल॒लाटा॑य॒ स्वाहा᳚ । स्वाहा॑ मू॒र्द्ध्ने । मू॒र्द्ध्ने स्वाहा᳚ । स्वाहा॑ म॒स्तिष्का॑य । म॒स्तिष्का॑य॒ स्वाहा᳚ । स्वाहा॒ केशे᳚भ्यः । केशे᳚भ्यः॒ स्वाहा᳚ । स्वाहा॒ वहा॑य । वहा॑य॒ स्वाहा᳚ । स्वाहा᳚ ग्री॒वाभ्यः॑ । ग्री॒वाभ्यः॒ स्वाहा᳚ । स्वाहा᳚ स्क॒न्धेभ्यः॑ । स्क॒न्धेभ्यः॒ स्वाहा᳚ । स्वाहा॒ कीक॑साभ्यः । कीक॑साभ्यः॒ स्वाहा᳚ । स्वाहा॑ पृ॒ष्टीभ्यः॑ । पृ॒ष्टीभ्यः॒ स्वाहा᳚ । पृ॒ष्टीभ्य॒ इति॑ पृ॒ष्टि - भ्यः॒ । स्वाहा॑ पाज॒स्या॑य । पा॒ज॒स्या॑य॒ स्वाहा᳚ । स्वाहा॑ पा॒र्श्वाभ्या᳚म् । पा॒र्श्वाभ्याꣳ॒॒ स्वाहा᳚ । स्वाहाऽꣳसा᳚भ्याम् \newline

\textbf{Jatai Paata} \newline

1. द॒द्भ्यः स्वाहा॒ स्वाहा॑ द॒द्भ्यो द॒द्भ्यः स्वाहा᳚ । \newline
2. द॒द्भ्य इति॑ दत् - भ्यः । \newline
3. स्वाहा॒ हनू᳚भ्याꣳ॒॒ हनू᳚भ्याꣳ॒॒ स्वाहा॒ स्वाहा॒ हनू᳚भ्याम् । \newline
4. हनू᳚भ्याꣳ॒॒ स्वाहा॒ स्वाहा॒ हनू᳚भ्याꣳ॒॒ हनू᳚भ्याꣳ॒॒ स्वाहा᳚ । \newline
5. हनू᳚भ्या॒मिति॒ हनु॑ - भ्या॒म् । \newline
6. स्वाहोष्ठा᳚भ्या॒ मोष्ठा᳚भ्याꣳ॒॒ स्वाहा॒ स्वाहोष्ठा᳚भ्याम् । \newline
7. ओष्ठा᳚भ्याꣳ॒॒ स्वाहा॒ स्वाहौष्ठा᳚भ्या॒ मोष्ठा᳚भ्याꣳ॒॒ स्वाहा᳚ । \newline
8. स्वाहा॒ मुखा॑य॒ मुखा॑य॒ स्वाहा॒ स्वाहा॒ मुखा॑य । \newline
9. मुखा॑य॒ स्वाहा॒ स्वाहा॒ मुखा॑य॒ मुखा॑य॒ स्वाहा᳚ । \newline
10. स्वाहा॒ नासि॑काभ्या॒न् नासि॑काभ्याꣳ॒॒ स्वाहा॒ स्वाहा॒ नासि॑काभ्याम् । \newline
11. नासि॑काभ्याꣳ॒॒ स्वाहा॒ स्वाहा॒ नासि॑काभ्या॒न् नासि॑काभ्याꣳ॒॒ स्वाहा᳚ । \newline
12. स्वाहा॒ ऽक्षीभ्या॑ म॒क्षीभ्याꣳ॒॒ स्वाहा॒ स्वाहा॒ ऽक्षीभ्या᳚म् । \newline
13. अ॒क्षीभ्याꣳ॒॒ स्वाहा॒ स्वाहा॒ ऽक्षीभ्या॑ म॒क्षीभ्याꣳ॒॒ स्वाहा᳚ । \newline
14. स्वाहा॒ कर्णा᳚भ्या॒म् कर्णा᳚भ्याꣳ॒॒ स्वाहा॒ स्वाहा॒ कर्णा᳚भ्याम् । \newline
15. कर्णा᳚भ्याꣳ॒॒ स्वाहा॒ स्वाहा॒ कर्णा᳚भ्या॒म् कर्णा᳚भ्याꣳ॒॒ स्वाहा᳚ । \newline
16. स्वाहा॑ पा॒रे पा॒रे स्वाहा॒ स्वाहा॑ पा॒रे । \newline
17. पा॒र इ॒क्षव॑ इ॒क्षवः॑ पा॒रे पा॒र इ॒क्षवः॑ । \newline
18. इ॒क्षवो॑ ऽवा॒र्ये᳚भ्यो ऽवा॒र्ये᳚भ्य इ॒क्षव॑ इ॒क्षवो॑ ऽवा॒र्ये᳚भ्यः । \newline
19. अ॒वा॒र्ये᳚भ्यः॒ पक्ष्म॑भ्यः॒ पक्ष्म॑भ्यो ऽवा॒र्ये᳚भ्यो ऽवा॒र्ये᳚भ्यः॒ पक्ष्म॑भ्यः । \newline
20. पक्ष्म॑भ्यः॒ स्वाहा॒ स्वाहा॒ पक्ष्म॑भ्यः॒ पक्ष्म॑भ्यः॒ स्वाहा᳚ । \newline
21. पक्ष्म॑भ्य॒ इति॒ पक्ष्म॑ - भ्यः॒ । \newline
22. स्वाहा॑ ऽवा॒रे॑ ऽवा॒रे स्वाहा॒ स्वाहा॑ ऽवा॒रे । \newline
23. अ॒वा॒र इ॒क्षव॑ इ॒क्षवो॑ ऽवा॒रे॑ ऽवा॒र इ॒क्षवः॑ । \newline
24. इ॒क्षवः॑ पा॒र्ये᳚भ्यः पा॒र्ये᳚भ्य इ॒क्षव॑ इ॒क्षवः॑ पा॒र्ये᳚भ्यः । \newline
25. पा॒र्ये᳚भ्यः॒ पक्ष्म॑भ्यः॒ पक्ष्म॑भ्यः पा॒र्ये᳚भ्यः पा॒र्ये᳚भ्यः॒ पक्ष्म॑भ्यः । \newline
26. पक्ष्म॑भ्यः॒ स्वाहा॒ स्वाहा॒ पक्ष्म॑भ्यः॒ पक्ष्म॑भ्यः॒ स्वाहा᳚ । \newline
27. पक्ष्म॑भ्य॒ इति॒ पक्ष्म॑ - भ्यः॒ । \newline
28. स्वाहा॑ शी॒र्ष्णे शी॒र्ष्णे स्वाहा॒ स्वाहा॑ शी॒र्ष्णे । \newline
29. शी॒र्ष्णे स्वाहा॒ स्वाहा॑ शी॒र्ष्णे शी॒र्ष्णे स्वाहा᳚ । \newline
30. स्वाहा᳚ भ्रू॒भ्याम् भ्रू॒भ्याꣳ स्वाहा॒ स्वाहा᳚ भ्रू॒भ्याम् । \newline
31. भ्रू॒भ्याꣳ स्वाहा॒ स्वाहा᳚ भ्रू॒भ्याम् भ्रू॒भ्याꣳ स्वाहा᳚ । \newline
32. स्वाहा॑ ल॒लाटा॑य ल॒लाटा॑य॒ स्वाहा॒ स्वाहा॑ ल॒लाटा॑य । \newline
33. ल॒लाटा॑य॒ स्वाहा॒ स्वाहा॑ ल॒लाटा॑य ल॒लाटा॑य॒ स्वाहा᳚ । \newline
34. स्वाहा॑ मू॒र्द्ध्ने मू॒र्द्ध्ने स्वाहा॒ स्वाहा॑ मू॒र्द्ध्ने । \newline
35. मू॒र्द्ध्ने स्वाहा॒ स्वाहा॑ मू॒र्द्ध्ने मू॒र्द्ध्ने स्वाहा᳚ । \newline
36. स्वाहा॑ म॒स्तिष्का॑य म॒स्तिष्का॑य॒ स्वाहा॒ स्वाहा॑ म॒स्तिष्का॑य । \newline
37. म॒स्तिष्का॑य॒ स्वाहा॒ स्वाहा॑ म॒स्तिष्का॑य म॒स्तिष्का॑य॒ स्वाहा᳚ । \newline
38. स्वाहा॒ केशे᳚भ्यः॒ केशे᳚भ्यः॒ स्वाहा॒ स्वाहा॒ केशे᳚भ्यः । \newline
39. केशे᳚भ्यः॒ स्वाहा॒ स्वाहा॒ केशे᳚भ्यः॒ केशे᳚भ्यः॒ स्वाहा᳚ । \newline
40. स्वाहा॒ वहा॑य॒ वहा॑य॒ स्वाहा॒ स्वाहा॒ वहा॑य । \newline
41. वहा॑य॒ स्वाहा॒ स्वाहा॒ वहा॑य॒ वहा॑य॒ स्वाहा᳚ । \newline
42. स्वाहा᳚ ग्री॒वाभ्यो᳚ ग्री॒वाभ्यः॒ स्वाहा॒ स्वाहा᳚ ग्री॒वाभ्यः॑ । \newline
43. ग्री॒वाभ्यः॒ स्वाहा॒ स्वाहा᳚ ग्री॒वाभ्यो᳚ ग्री॒वाभ्यः॒ स्वाहा᳚ । \newline
44. स्वाहा᳚ स्क॒न्धेभ्यः॑ स्क॒न्धेभ्यः॒ स्वाहा॒ स्वाहा᳚ स्क॒न्धेभ्यः॑ । \newline
45. स्क॒न्धेभ्यः॒ स्वाहा॒ स्वाहा᳚ स्क॒न्धेभ्यः॑ स्क॒न्धेभ्यः॒ स्वाहा᳚ । \newline
46. स्वाहा॒ कीक॑साभ्यः॒ कीक॑साभ्यः॒ स्वाहा॒ स्वाहा॒ कीक॑साभ्यः । \newline
47. कीक॑साभ्यः॒ स्वाहा॒ स्वाहा॒ कीक॑साभ्यः॒ कीक॑साभ्यः॒ स्वाहा᳚ । \newline
48. स्वाहा॑ पृ॒ष्टीभ्यः॑ पृ॒ष्टीभ्यः॒ स्वाहा॒ स्वाहा॑ पृ॒ष्टीभ्यः॑ । \newline
49. पृ॒ष्टीभ्यः॒ स्वाहा॒ स्वाहा॑ पृ॒ष्टीभ्यः॑ पृ॒ष्टीभ्यः॒ स्वाहा᳚ । \newline
50. पृ॒ष्टीभ्य॒ इति॑ पृ॒ष्टि - भ्यः॒ । \newline
51. स्वाहा॑ पाज॒स्या॑य पाज॒स्या॑य॒ स्वाहा॒ स्वाहा॑ पाज॒स्या॑य । \newline
52. पा॒ज॒स्या॑य॒ स्वाहा॒ स्वाहा॑ पाज॒स्या॑य पाज॒स्या॑य॒ स्वाहा᳚ । \newline
53. स्वाहा॑ पा॒र्श्वाभ्या᳚म् पा॒र्श्वाभ्याꣳ॒॒ स्वाहा॒ स्वाहा॑ पा॒र्श्वाभ्या᳚म् । \newline
54. पा॒र्श्वाभ्याꣳ॒॒ स्वाहा॒ स्वाहा॑ पा॒र्श्वाभ्या᳚म् पा॒र्श्वाभ्याꣳ॒॒ स्वाहा᳚ । \newline
55. स्वाहा ऽꣳसा᳚भ्या॒ मꣳसा᳚भ्याꣳ॒॒ स्वाहा॒ स्वाहा ऽꣳसा᳚भ्याम् । \newline

\textbf{Ghana Paata } \newline

1. द॒द्भ्यः स्वाहा॒ स्वाहा॑ द॒द्भ्यो द॒द्भ्यः स्वाहा॒ हनू᳚भ्याꣳ॒॒ हनू᳚भ्याꣳ॒॒ स्वाहा॑ द॒द्भ्यो द॒द्भ्यः स्वाहा॒ हनू᳚भ्याम् । \newline
2. द॒द्भ्य इति॑ दत् - भ्यः । \newline
3. स्वाहा॒ हनू᳚भ्याꣳ॒॒ हनू᳚भ्याꣳ॒॒ स्वाहा॒ स्वाहा॒ हनू᳚भ्याꣳ॒॒ स्वाहा॒ स्वाहा॒ हनू᳚भ्याꣳ॒॒ स्वाहा॒ स्वाहा॒ हनू᳚भ्याꣳ॒॒ स्वाहा᳚ । \newline
4. हनू᳚भ्याꣳ॒॒ स्वाहा॒ स्वाहा॒ हनू᳚भ्याꣳ॒॒ हनू᳚भ्याꣳ॒॒ स्वाहोष्ठा᳚भ्या॒ मोष्ठा᳚भ्याꣳ॒॒ स्वाहा॒ हनू᳚भ्याꣳ॒॒ हनू᳚भ्याꣳ॒॒ स्वाहोष्ठा᳚भ्याम् । \newline
5. हनू᳚भ्या॒मिति॒ हनु॑ - भ्या॒म् । \newline
6. स्वाहोष्ठा᳚भ्या॒ मोष्ठा᳚भ्याꣳ॒॒ स्वाहा॒ स्वाहोष्ठा᳚भ्याꣳ॒॒ स्वाहा॒ स्वाहोष्ठा᳚भ्याꣳ॒॒ स्वाहा॒ 
स्वाहोष्ठा᳚भ्याꣳ॒॒ स्वाहा᳚ । \newline
7. ओष्ठा᳚भ्याꣳ॒॒ स्वाहा॒ स्वाहोष्ठा᳚भ्या॒ मोष्ठा᳚भ्याꣳ॒॒ स्वाहा॒ मुखा॑य॒ मुखा॑य॒ स्वाहोष्ठा᳚भ्या॒ मोष्ठा᳚भ्याꣳ॒॒ स्वाहा॒ मुखा॑य । \newline
8. स्वाहा॒ मुखा॑य॒ मुखा॑य॒ स्वाहा॒ स्वाहा॒ मुखा॑य॒ स्वाहा॒ स्वाहा॒ मुखा॑य॒ स्वाहा॒ स्वाहा॒ मुखा॑य॒ स्वाहा᳚ । \newline
9. मुखा॑य॒ स्वाहा॒ स्वाहा॒ मुखा॑य॒ मुखा॑य॒ स्वाहा॒ नासि॑काभ्या॒न् नासि॑काभ्याꣳ॒॒ स्वाहा॒ मुखा॑य॒ मुखा॑य॒ स्वाहा॒ नासि॑काभ्याम् । \newline
10. स्वाहा॒ नासि॑काभ्या॒न् नासि॑काभ्याꣳ॒॒ स्वाहा॒ स्वाहा॒ नासि॑काभ्याꣳ॒॒ स्वाहा॒ स्वाहा॒ नासि॑काभ्याꣳ॒॒ स्वाहा॒ स्वाहा॒ नासि॑काभ्याꣳ॒॒ स्वाहा᳚ । \newline
11. नासि॑काभ्याꣳ॒॒ स्वाहा॒ स्वाहा॒ नासि॑काभ्या॒न् नासि॑काभ्याꣳ॒॒ स्वाहा॒ ऽक्षीभ्या॑ म॒क्षीभ्याꣳ॒॒ स्वाहा॒ नासि॑काभ्या॒न् नासि॑काभ्याꣳ॒॒ स्वाहा॒ ऽक्षीभ्या᳚म् । \newline
12. स्वाहा॒ ऽक्षीभ्या॑ म॒क्षीभ्याꣳ॒॒ स्वाहा॒ स्वाहा॒ ऽक्षीभ्याꣳ॒॒ स्वाहा॒ स्वाहा॒ ऽक्षीभ्याꣳ॒॒ स्वाहा॒ स्वाहा॒ ऽक्षीभ्याꣳ॒॒ स्वाहा᳚ । \newline
13. अ॒क्षीभ्याꣳ॒॒ स्वाहा॒ स्वाहा॒ ऽक्षीभ्या॑ म॒क्षीभ्याꣳ॒॒ स्वाहा॒ कर्णा᳚भ्या॒म् कर्णा᳚भ्याꣳ॒॒ स्वाहा॒ ऽक्षीभ्या॑ म॒क्षीभ्याꣳ॒॒ स्वाहा॒ कर्णा᳚भ्याम् । \newline
14. स्वाहा॒ कर्णा᳚भ्या॒म् कर्णा᳚भ्याꣳ॒॒ स्वाहा॒ स्वाहा॒ कर्णा᳚भ्याꣳ॒॒ स्वाहा॒ स्वाहा॒ कर्णा᳚भ्याꣳ॒॒ स्वाहा॒ स्वाहा॒ कर्णा᳚भ्याꣳ॒॒ स्वाहा᳚ । \newline
15. कर्णा᳚भ्याꣳ॒॒ स्वाहा॒ स्वाहा॒ कर्णा᳚भ्या॒म् कर्णा᳚भ्याꣳ॒॒ स्वाहा॑ पा॒रे पा॒रे स्वाहा॒ कर्णा᳚भ्या॒म् कर्णा᳚भ्याꣳ॒॒ स्वाहा॑ पा॒रे । \newline
16. स्वाहा॑ पा॒रे पा॒रे स्वाहा॒ स्वाहा॑ पा॒र इ॒क्षव॑ इ॒क्षवः॑ पा॒रे स्वाहा॒ स्वाहा॑ पा॒र इ॒क्षवः॑ । \newline
17. पा॒र इ॒क्षव॑ इ॒क्षवः॑ पा॒रे पा॒र इ॒क्षवो॑ ऽवा॒र्ये᳚भ्यो ऽवा॒र्ये᳚भ्य इ॒क्षवः॑ पा॒रे पा॒र इ॒क्षवो॑ ऽवा॒र्ये᳚भ्यः । \newline
18. इ॒क्षवो॑ ऽवा॒र्ये᳚भ्यो ऽवा॒र्ये᳚भ्य इ॒क्षव॑ इ॒क्षवो॑ ऽवा॒र्ये᳚भ्यः॒ पक्ष्म॑भ्यः॒ पक्ष्म॑भ्यो ऽवा॒र्ये᳚भ्य इ॒क्षव॑ इ॒क्षवो॑ ऽवा॒र्ये᳚भ्यः॒ पक्ष्म॑भ्यः । \newline
19. अ॒वा॒र्ये᳚भ्यः॒ पक्ष्म॑भ्यः॒ पक्ष्म॑भ्यो ऽवा॒र्ये᳚भ्यो ऽवा॒र्ये᳚भ्यः॒ पक्ष्म॑भ्यः॒ स्वाहा॒ स्वाहा॒ पक्ष्म॑भ्यो ऽवा॒र्ये᳚भ्यो ऽवा॒र्ये᳚भ्यः॒ पक्ष्म॑भ्यः॒ स्वाहा᳚ । \newline
20. पक्ष्म॑भ्यः॒ स्वाहा॒ स्वाहा॒ पक्ष्म॑भ्यः॒ पक्ष्म॑भ्यः॒ स्वाहा॑ ऽवा॒रे॑ ऽवा॒रे स्वाहा॒ पक्ष्म॑भ्यः॒ पक्ष्म॑भ्यः॒ स्वाहा॑ ऽवा॒रे । \newline
21. पक्ष्म॑भ्य॒ इति॒ पक्ष्म॑ - भ्यः॒ । \newline
22. स्वाहा॑ ऽवा॒रे॑ ऽवा॒रे स्वाहा॒ स्वाहा॑ ऽवा॒र इ॒क्षव॑ इ॒क्षवो॑ ऽवा॒रे स्वाहा॒ स्वाहा॑ ऽवा॒र इ॒क्षवः॑ । \newline
23. अ॒वा॒र इ॒क्षव॑ इ॒क्षवो॑ ऽवा॒रे॑ ऽवा॒र इ॒क्षवः॑ पा॒र्ये᳚भ्यः पा॒र्ये᳚भ्य इ॒क्षवो॑ ऽवा॒रे॑ ऽवा॒र इ॒क्षवः॑ पा॒र्ये᳚भ्यः । \newline
24. इ॒क्षवः॑ पा॒र्ये᳚भ्यः पा॒र्ये᳚भ्य इ॒क्षव॑ इ॒क्षवः॑ पा॒र्ये᳚भ्यः॒ पक्ष्म॑भ्यः॒ पक्ष्म॑भ्यः पा॒र्ये᳚भ्य इ॒क्षव॑ इ॒क्षवः॑ पा॒र्ये᳚भ्यः॒ पक्ष्म॑भ्यः । \newline
25. पा॒र्ये᳚भ्यः॒ पक्ष्म॑भ्यः॒ पक्ष्म॑भ्यः पा॒र्ये᳚भ्यः पा॒र्ये᳚भ्यः॒ पक्ष्म॑भ्यः॒ स्वाहा॒ स्वाहा॒ पक्ष्म॑भ्यः पा॒र्ये᳚भ्यः पा॒र्ये᳚भ्यः॒ पक्ष्म॑भ्यः॒ स्वाहा᳚ । \newline
26. पक्ष्म॑भ्यः॒ स्वाहा॒ स्वाहा॒ पक्ष्म॑भ्यः॒ पक्ष्म॑भ्यः॒ स्वाहा॑ शी॒र्ष्णे शी॒र्ष्णे स्वाहा॒ पक्ष्म॑भ्यः॒ पक्ष्म॑भ्यः॒ स्वाहा॑ शी॒र्ष्णे । \newline
27. पक्ष्म॑भ्य॒ इति॒ पक्ष्म॑ - भ्यः॒ । \newline
28. स्वाहा॑ शी॒र्ष्णे शी॒र्ष्णे स्वाहा॒ स्वाहा॑ शी॒र्ष्णे स्वाहा॒ स्वाहा॑ शी॒र्ष्णे स्वाहा॒ स्वाहा॑ शी॒र्ष्णे स्वाहा᳚ । \newline
29. शी॒र्ष्णे स्वाहा॒ स्वाहा॑ शी॒र्ष्णे शी॒र्ष्णे स्वाहा᳚ भ्रू॒भ्याम् भ्रू॒भ्याꣳ स्वाहा॑ शी॒र्ष्णे शी॒र्ष्णे स्वाहा᳚ भ्रू॒भ्याम् । \newline
30. स्वाहा᳚ भ्रू॒भ्याम् भ्रू॒भ्याꣳ स्वाहा॒ स्वाहा᳚ भ्रू॒भ्याꣳ स्वाहा॒ स्वाहा᳚ भ्रू॒भ्याꣳ स्वाहा॒ स्वाहा᳚ भ्रू॒भ्याꣳ स्वाहा᳚ । \newline
31. भ्रू॒भ्याꣳ स्वाहा॒ स्वाहा᳚ भ्रू॒भ्याम् भ्रू॒भ्याꣳ स्वाहा॑ ल॒लाटा॑य ल॒लाटा॑य॒ स्वाहा᳚ भ्रू॒भ्याम् भ्रू॒भ्याꣳ स्वाहा॑ ल॒लाटा॑य । \newline
32. स्वाहा॑ ल॒लाटा॑य ल॒लाटा॑य॒ स्वाहा॒ स्वाहा॑ ल॒लाटा॑य॒ स्वाहा॒ स्वाहा॑ ल॒लाटा॑य॒ स्वाहा॒ स्वाहा॑ ल॒लाटा॑य॒ स्वाहा᳚ । \newline
33. ल॒लाटा॑य॒ स्वाहा॒ स्वाहा॑ ल॒लाटा॑य ल॒लाटा॑य॒ स्वाहा॑ मू॒र्द्ध्ने मू॒र्द्ध्ने स्वाहा॑ ल॒लाटा॑य ल॒लाटा॑य॒ स्वाहा॑ मू॒र्द्ध्ने । \newline
34. स्वाहा॑ मू॒र्द्ध्ने मू॒र्द्ध्ने स्वाहा॒ स्वाहा॑ मू॒र्द्ध्ने स्वाहा॒ स्वाहा॑ मू॒र्द्ध्ने स्वाहा॒ स्वाहा॑ मू॒र्द्ध्ने स्वाहा᳚ । \newline
35. मू॒र्द्ध्ने स्वाहा॒ स्वाहा॑ मू॒र्द्ध्ने मू॒र्द्ध्ने स्वाहा॑ म॒स्तिष्का॑य म॒स्तिष्का॑य॒ स्वाहा॑ मू॒र्द्ध्ने मू॒र्द्ध्ने स्वाहा॑ म॒स्तिष्का॑य । \newline
36. स्वाहा॑ म॒स्तिष्का॑य म॒स्तिष्का॑य॒ स्वाहा॒ स्वाहा॑ म॒स्तिष्का॑य॒ स्वाहा॒ स्वाहा॑ म॒स्तिष्का॑य॒ स्वाहा॒ स्वाहा॑ म॒स्तिष्का॑य॒ स्वाहा᳚ । \newline
37. म॒स्तिष्का॑य॒ स्वाहा॒ स्वाहा॑ म॒स्तिष्का॑य म॒स्तिष्का॑य॒ स्वाहा॒ केशे᳚भ्यः॒ केशे᳚भ्यः॒ स्वाहा॑ म॒स्तिष्का॑य म॒स्तिष्का॑य॒ स्वाहा॒ केशे᳚भ्यः । \newline
38. स्वाहा॒ केशे᳚भ्यः॒ केशे᳚भ्यः॒ स्वाहा॒ स्वाहा॒ केशे᳚भ्यः॒ स्वाहा॒ स्वाहा॒ केशे᳚भ्यः॒ स्वाहा॒ स्वाहा॒ केशे᳚भ्यः॒ स्वाहा᳚ । \newline
39. केशे᳚भ्यः॒ स्वाहा॒ स्वाहा॒ केशे᳚भ्यः॒ केशे᳚भ्यः॒ स्वाहा॒ वहा॑य॒ वहा॑य॒ स्वाहा॒ केशे᳚भ्यः॒ केशे᳚भ्यः॒ स्वाहा॒ वहा॑य । \newline
40. स्वाहा॒ वहा॑य॒ वहा॑य॒ स्वाहा॒ स्वाहा॒ वहा॑य॒ स्वाहा॒ स्वाहा॒ वहा॑य॒ स्वाहा॒ स्वाहा॒ वहा॑य॒ स्वाहा᳚ । \newline
41. वहा॑य॒ स्वाहा॒ स्वाहा॒ वहा॑य॒ वहा॑य॒ स्वाहा᳚ ग्री॒वाभ्यो᳚ ग्री॒वाभ्यः॒ स्वाहा॒ वहा॑य॒ वहा॑य॒ स्वाहा᳚ ग्री॒वाभ्यः॑ । \newline
42. स्वाहा᳚ ग्री॒वाभ्यो᳚ ग्री॒वाभ्यः॒ स्वाहा॒ स्वाहा᳚ ग्री॒वाभ्यः॒ स्वाहा॒ स्वाहा᳚ ग्री॒वाभ्यः॒ स्वाहा॒ स्वाहा᳚ ग्री॒वाभ्यः॒ स्वाहा᳚ । \newline
43. ग्री॒वाभ्यः॒ स्वाहा॒ स्वाहा᳚ ग्री॒वाभ्यो᳚ ग्री॒वाभ्यः॒ स्वाहा᳚ स्क॒न्धेभ्यः॑ स्क॒न्धेभ्यः॒ स्वाहा᳚ ग्री॒वाभ्यो᳚ ग्री॒वाभ्यः॒ स्वाहा᳚ स्क॒न्धेभ्यः॑ । \newline
44. स्वाहा᳚ स्क॒न्धेभ्यः॑ स्क॒न्धेभ्यः॒ स्वाहा॒ स्वाहा᳚ स्क॒न्धेभ्यः॒ स्वाहा॒ स्वाहा᳚ स्क॒न्धेभ्यः॒ स्वाहा॒ स्वाहा᳚ स्क॒न्धेभ्यः॒ स्वाहा᳚ । \newline
45. स्क॒न्धेभ्यः॒ स्वाहा॒ स्वाहा᳚ स्क॒न्धेभ्यः॑ स्क॒न्धेभ्यः॒ स्वाहा॒ कीक॑साभ्यः॒ कीक॑साभ्यः॒ स्वाहा᳚ स्क॒न्धेभ्यः॑ स्क॒न्धेभ्यः॒ स्वाहा॒ कीक॑साभ्यः । \newline
46. स्वाहा॒ कीक॑साभ्यः॒ कीक॑साभ्यः॒ स्वाहा॒ स्वाहा॒ कीक॑साभ्यः॒ स्वाहा॒ स्वाहा॒ कीक॑साभ्यः॒ स्वाहा॒ स्वाहा॒ कीक॑साभ्यः॒ स्वाहा᳚ । \newline
47. कीक॑साभ्यः॒ स्वाहा॒ स्वाहा॒ कीक॑साभ्यः॒ कीक॑साभ्यः॒ स्वाहा॑ पृ॒ष्टीभ्यः॑ पृ॒ष्टीभ्यः॒ स्वाहा॒ कीक॑साभ्यः॒ कीक॑साभ्यः॒ स्वाहा॑ पृ॒ष्टीभ्यः॑ । \newline
48. स्वाहा॑ पृ॒ष्टीभ्यः॑ पृ॒ष्टीभ्यः॒ स्वाहा॒ स्वाहा॑ पृ॒ष्टीभ्यः॒ स्वाहा॒ स्वाहा॑ पृ॒ष्टीभ्यः॒ स्वाहा॒ स्वाहा॑ पृ॒ष्टीभ्यः॒ स्वाहा᳚ । \newline
49. पृ॒ष्टीभ्यः॒ स्वाहा॒ स्वाहा॑ पृ॒ष्टीभ्यः॑ पृ॒ष्टीभ्यः॒ स्वाहा॑ पाज॒स्या॑य पाज॒स्या॑य॒ स्वाहा॑ पृ॒ष्टीभ्यः॑ पृ॒ष्टीभ्यः॒ स्वाहा॑ पाज॒स्या॑य । \newline
50. पृ॒ष्टीभ्य॒ इति॑ पृ॒ष्टि - भ्यः॒ । \newline
51. स्वाहा॑ पाज॒स्या॑य पाज॒स्या॑य॒ स्वाहा॒ स्वाहा॑ पाज॒स्या॑य॒ स्वाहा॒ स्वाहा॑ पाज॒स्या॑य॒ स्वाहा॒ स्वाहा॑ पाज॒स्या॑य॒ स्वाहा᳚ । \newline
52. पा॒ज॒स्या॑य॒ स्वाहा॒ स्वाहा॑ पाज॒स्या॑य पाज॒स्या॑य॒ स्वाहा॑ पा॒र्श्वाभ्या᳚म् पा॒र्श्वाभ्याꣳ॒॒ स्वाहा॑ पाज॒स्या॑य पाज॒स्या॑य॒ स्वाहा॑ पा॒र्श्वाभ्या᳚म् । \newline
53. स्वाहा॑ पा॒र्श्वाभ्या᳚म् पा॒र्श्वाभ्याꣳ॒॒ स्वाहा॒ स्वाहा॑ पा॒र्श्वाभ्याꣳ॒॒ स्वाहा॒ स्वाहा॑ पा॒र्श्वाभ्याꣳ॒॒ स्वाहा॒ स्वाहा॑ पा॒र्श्वाभ्याꣳ॒॒ स्वाहा᳚ । \newline
54. पा॒र्श्वाभ्याꣳ॒॒ स्वाहा॒ स्वाहा॑ पा॒र्श्वाभ्या᳚म् पा॒र्श्वाभ्याꣳ॒॒ स्वाहा ऽꣳसा᳚भ्या॒ मꣳसा᳚भ्याꣳ॒॒ स्वाहा॑ पा॒र्श्वाभ्या᳚म् पा॒र्श्वाभ्याꣳ॒॒ स्वाहा ऽꣳसा᳚भ्याम् । \newline
55. स्वाहा ऽꣳसा᳚भ्या॒ मꣳसा᳚भ्याꣳ॒॒ स्वाहा॒ स्वाहा ऽꣳसा᳚भ्याꣳ॒॒ स्वाहा॒ स्वाहा ऽꣳसा᳚भ्याꣳ॒॒ स्वाहा॒ स्वाहा ऽꣳसा᳚भ्याꣳ॒॒ स्वाहा᳚ । \newline
\pagebreak
\markright{ TS 7.3.16.2  \hfill https://www.vedavms.in \hfill}

\section{ TS 7.3.16.2 }

\textbf{TS 7.3.16.2 } \newline
\textbf{Samhita Paata} \newline

ऽꣳसा᳚भ्याꣳ॒॒ स्वाहा॑ दो॒षभ्याꣳ॒॒ स्वाहा॑ बा॒हुभ्याꣳ॒॒ स्वाहा॒ जङ्घा᳚भ्याꣳ॒॒ स्वाहा॒ श्रोणी᳚भ्याꣳ॒॒ स्वाहो॒रुभ्याꣳ॒॒ स्वाहा᳚ ऽष्ठी॒वद्भ्याꣳ॒॒ स्वाहा॒ जङ्घा᳚भ्याꣳ॒॒ स्वाहा॑ भ॒सदे॒ स्वाहा॑ शिख॒ण्डेभ्यः॒ स्वाहा॑ वाल॒धाना॑य॒ स्वाहा॒ ऽऽण्डाभ्याꣳ॒॒ स्वाहा॒ शेपा॑य॒ स्वाहा॒ रेत॑से॒ स्वाहा᳚ प्र॒जाभ्यः॒ स्वाहा᳚ प्र॒जन॑नाय॒ स्वाहा॑ प॒द्भ्यः स्वाहा॑ श॒फेभ्यः॒ स्वाहा॒ लोम॑भ्यः॒ स्वाहा᳚ त्व॒चे स्वाहा॒ लोहि॑ताय॒ स्वाहा॑ माꣳ॒॒साय॒ स्वाहा॒ स्नाव॑भ्यः॒ स्वाहा॒ ऽस्थभ्यः॒ स्वाहा॑ म॒ज्जभ्यः॒ स्वाहा ( ) ऽङ्गे᳚भ्यः॒ स्वाहा॒ ऽऽत्मने॒ स्वाहा॒ सर्व॑स्मै॒ स्वाहा᳚ ॥ \newline

\textbf{Pada Paata} \newline

अꣳसा᳚भ्याम् । स्वाहा᳚ । दो॒षभ्या॒मिति॑ दो॒ष - भ्या॒म् । स्वाहा᳚ । बा॒हुभ्या॒मिति॑ बा॒हु - भ्या॒म् । स्वाहा᳚ । जङ्घा᳚भ्याम् । स्वाहा᳚ । श्रोणी᳚भ्या॒मिति॒ श्राणि॑ - भ्या॒म् । स्वाहा᳚ । ऊ॒रुभ्या॒मित्यू॒रु - भ्या॒म् । स्वाहा᳚ । अ॒ष्ठी॒वद्भ्या॒मित्य॑ष्ठी॒वत् - भ्या॒म् । स्वाहा᳚ । जङ्घा᳚भ्याम् । स्वाहा᳚ । भ॒सदे᳚ । स्वाहा᳚ । शि॒ख॒ण्डेभ्यः॑ । स्वाहा᳚ । वा॒ल॒धाना॒येति॑ वाल - धाना॑य । स्वाहा᳚ । अ॒ण्डाभ्या᳚म् । स्वाहा᳚ । शेपा॑य । स्वाहा᳚ । रेत॑से । स्वाहा᳚ । प्र॒जाभ्य॒ इति॑ प्र - जाभ्यः॑ । स्वाहा᳚ । प्र॒जन॑ना॒येति॑ प्र - जन॑नाय । स्वाहा᳚ । प॒द्भ्य इति॑ पत् - भ्यः । स्वाहा᳚ । श॒फेभ्यः॑ । स्वाहा᳚ । लोम॑भ्य॒ इति॒ लोम॑ - भ्यः॒ । स्वाहा᳚ । त्व॒चे । स्वाहा᳚ । लोहि॑ताय । स्वाहा᳚ । माꣳ॒॒साय॑ । स्वाहा᳚ । स्नाव॑भ्य॒ इति॒ स्नाव॑ - भ्यः॒ । स्वाहा᳚ । अ॒स्थभ्य॒ इत्य॒स्थ - भ्यः॒ । स्वाहा᳚ । म॒ज्जभ्य॒ इति॑ म॒ज्ज - भ्यः॒ । स्वाहा᳚ ( ) । अङ्गे᳚भ्यः । स्वाहा᳚ । आ॒त्मने᳚ । स्वाहा᳚ । सर्व॑स्मै । स्वाहा᳚ ॥  \newline


\textbf{Krama Paata} \newline

अꣳसा᳚भ्याꣳ॒॒ स्वाहा᳚ । स्वाहा॑ दो॒षभ्या᳚म् । दो॒षभ्याꣳ॒॒ स्वाहा᳚ । दो॒षभ्या॒मिति॑ दो॒ष - भ्या॒म् । स्वाहा॑ बा॒हुभ्या᳚म् । बा॒हुभ्याꣳ॒॒ स्वाहा᳚ । बा॒हुभ्या॒मिति॑ बा॒हु - भ्या॒म् । स्वाहा॒ जङ्‍घा᳚भ्याम् । जङ्‍घा᳚भ्याꣳ॒॒ स्वाहा᳚ । स्वाहा॒ श्रोणी᳚भ्याम् । श्रोणी᳚भ्याꣳ॒॒ स्वाहा᳚ । श्रोणी᳚भ्या॒मिति॒ श्रोणि॑ - भ्या॒म् । स्वाहो॒रुभ्या᳚म् । ऊ॒रुभ्याꣳ॒॒ स्वाहा᳚ । ऊ॒रुभ्या॒मित्यू॒रु - भ्या॒म् । स्वाहा᳚ऽष्ठी॒वद्भ्या᳚म् । अ॒ष्ठी॒वद्भ्याꣳ॒॒ स्वाहा᳚ । अ॒ष्ठी॒वद्भ्या॒मित्य॑ष्ठी॒वत् - भ्या॒म् । स्वाहा॒ जङ्‍घा᳚भ्याम् । जङ्‍घा᳚भ्याꣳ॒॒ स्वाहा᳚ । स्वाहा॑ भ॒सदे᳚ । भ॒सदे॒ स्वाहा᳚ । स्वाहा॑ शिख॒ण्डेभ्यः॑ । शि॒ख॒ण्डेभ्यः॒ स्वाहा᳚ । स्वाहा॑ वाल॒धाना॑य । वा॒ल॒धाना॑य॒ स्वाहा᳚ । वा॒ल॒धाना॒येति॑ वाल - धाना॑य । स्वाहा॒ऽऽण्डाभ्या᳚म् । आ॒ण्डाभ्याꣳ॒॒ स्वाहा᳚ । स्वाहा॒ शेपा॑य । शेपा॑य॒ स्वाहा᳚ । स्वाहा॒ रेत॑से । रेत॑से॒ स्वाहा᳚ । स्वाहा᳚ प्र॒जाभ्यः॑ । प्र॒जाभ्यः॒ स्वाहा᳚ । प्र॒जाभ्य॒ इति॑ प्र - जाभ्यः॑ । स्वाहा᳚ प्र॒जन॑नाय । प्र॒जन॑नाय॒ स्वाहा᳚ । प्र॒जन॑ना॒येति॑ प्र - जन॑नाय । स्वाहा॑ प॒द्भ्यः । प॒द्भ्यः स्वाहा᳚ । प॒द्भ्य इति॑ पत् - भ्यः । स्वाहा॑ श॒फेभ्यः॑ । श॒फेभ्यः॒ स्वाहा᳚ । स्वाहा॒ लोम॑भ्यः । लोम॑भ्यः॒ स्वाहा᳚ । लोम॑भ्य॒ इति॒ लोम॑ - भ्यः॒ । स्वाहा᳚ त्व॒चे । त्व॒चे स्वाहा᳚ । स्वाहा॒ लोहि॑ताय । लोहि॑ताय॒ स्वाहा᳚ । स्वाहा॑ माꣳ॒॒साय॑ । माꣳ॒॒साय॒ स्वाहा᳚ । स्वाहा॒ स्नाव॑भ्यः । स्नाव॑भ्यः॒ स्वाहा᳚ । स्नाव॑भ्य॒ इति॒ स्नाव॑ - भ्यः॒ । स्वाहा॒ऽस्थभ्यः॑ । अ॒स्थभ्यः॒ स्वाहा᳚ । अ॒स्थभ्य॒ इत्य॒स्थ - भ्यः॒ । स्वाहा॑ म॒ज्जभ्यः॑ । म॒ज्जभ्यः॒ स्वाहा᳚ ( ) । म॒ज्जभ्य॒ इति॑ म॒ज्ज - भ्यः॒ । स्वाहाऽङ्‍गे᳚भ्यः । अङ्‍गे᳚भ्यः॒ स्वाहा᳚ । स्वाहा॒ऽऽत्मने᳚ । आ॒त्मने॒ स्वाहा᳚ । स्वाहा॒ सर्व॑स्मै । सर्व॑स्मै॒ स्वाहा᳚ । स्वाहेति॒ स्वाहा᳚ । \newline

\textbf{Jatai Paata} \newline

1. अꣳसा᳚भ्याꣳ॒॒ स्वाहा॒ स्वाहा ऽꣳसा᳚भ्या॒ मꣳसा᳚भ्याꣳ॒॒ स्वाहा᳚ । \newline
2. स्वाहा॑ दो॒षभ्या᳚म् दो॒षभ्याꣳ॒॒ स्वाहा॒ स्वाहा॑ दो॒षभ्या᳚म् । \newline
3. दो॒षभ्याꣳ॒॒ स्वाहा॒ स्वाहा॑ दो॒षभ्या᳚म् दो॒षभ्याꣳ॒॒ स्वाहा᳚ । \newline
4. दो॒षभ्या॒मिति॑ दो॒ष - भ्या॒म् । \newline
5. स्वाहा॑ बा॒हुभ्या᳚म् बा॒हुभ्याꣳ॒॒ स्वाहा॒ स्वाहा॑ बा॒हुभ्या᳚म् । \newline
6. बा॒हुभ्याꣳ॒॒ स्वाहा॒ स्वाहा॑ बा॒हुभ्या᳚म् बा॒हुभ्याꣳ॒॒ स्वाहा᳚ । \newline
7. बा॒हुभ्या॒मिति॑ बा॒हु - भ्या॒म् । \newline
8. स्वाहा॒ जङ्घा᳚भ्या॒म् जङ्घा᳚भ्याꣳ॒॒ स्वाहा॒ स्वाहा॒ जङ्घा᳚भ्याम् । \newline
9. जङ्घा᳚भ्याꣳ॒॒ स्वाहा॒ स्वाहा॒ जङ्घा᳚भ्या॒म् जङ्घा᳚भ्याꣳ॒॒ स्वाहा᳚ । \newline
10. स्वाहा॒ श्रोणी᳚भ्याꣳ॒॒ श्रोणी᳚भ्याꣳ॒॒ स्वाहा॒ स्वाहा॒ श्रोणी᳚भ्याम् । \newline
11. श्रोणी᳚भ्याꣳ॒॒ स्वाहा॒ स्वाहा॒ श्रोणी᳚भ्याꣳ॒॒ श्रोणी᳚भ्याꣳ॒॒ स्वाहा᳚ । \newline
12. श्रोणी᳚भ्या॒मिति॒ श्रोणि॑ - भ्या॒म् । \newline
13. स्वाहो॒ रुभ्या॑ मू॒रुभ्याꣳ॒॒ स्वाहा॒ स्वाहो॒ रुभ्या᳚म् । \newline
14. ऊ॒रुभ्याꣳ॒॒ स्वाहा॒ स्वाहो॒ रुभ्या॑ मू॒रुभ्याꣳ॒॒ स्वाहा᳚ । \newline
15. ऊ॒रुभ्या॒मित्यू॒रु - भ्या॒म् । \newline
16. स्वाहा᳚ ऽष्ठी॒वद्भ्या॑ मष्ठी॒वद्भ्याꣳ॒॒ स्वाहा॒ स्वाहा᳚ ऽष्ठी॒वद्भ्या᳚म् । \newline
17. अ॒ष्ठी॒वद्भ्याꣳ॒॒ स्वाहा॒ स्वाहा᳚ ऽष्ठी॒वद्भ्या॑ मष्ठी॒वद्भ्याꣳ॒॒ स्वाहा᳚ । \newline
18. अ॒ष्ठी॒वद्भ्या॒मित्य॑ष्ठी॒वत् - भ्या॒म् । \newline
19. स्वाहा॒ जङ्घा᳚भ्या॒म् जङ्घा᳚भ्याꣳ॒॒ स्वाहा॒ स्वाहा॒ जङ्घा᳚भ्याम् । \newline
20. जङ्घा᳚भ्याꣳ॒॒ स्वाहा॒ स्वाहा॒ जङ्घा᳚भ्या॒म् जङ्घा᳚भ्याꣳ॒॒ स्वाहा᳚ । \newline
21. स्वाहा॑ भ॒सदे॑ भ॒सदे॒ स्वाहा॒ स्वाहा॑ भ॒सदे᳚ । \newline
22. भ॒सदे॒ स्वाहा॒ स्वाहा॑ भ॒सदे॑ भ॒सदे॒ स्वाहा᳚ । \newline
23. स्वाहा॑ शिख॒ण्डेभ्यः॑ शिख॒ण्डेभ्यः॒ स्वाहा॒ स्वाहा॑ शिख॒ण्डेभ्यः॑ । \newline
24. शि॒ख॒ण्डेभ्यः॒ स्वाहा॒ स्वाहा॑ शिख॒ण्डेभ्यः॑ शिख॒ण्डेभ्यः॒ स्वाहा᳚ । \newline
25. स्वाहा॑ वाल॒धाना॑य वाल॒धाना॑य॒ स्वाहा॒ स्वाहा॑ वाल॒धाना॑य । \newline
26. वा॒ल॒धाना॑य॒ स्वाहा॒ स्वाहा॑ वाल॒धाना॑य वाल॒धाना॑य॒ स्वाहा᳚ । \newline
27. वा॒ल॒धाना॒येति॑ वाल - धाना॑य । \newline
28. स्वाहा॒ ऽऽण्डाभ्या॑ मा॒ण्डाभ्याꣳ॒॒ स्वाहा॒ स्वाहा॒ ऽऽण्डाभ्या᳚म् । \newline
29. आ॒ण्डाभ्याꣳ॒॒ स्वाहा॒ स्वाहा॒ ऽऽण्डाभ्या॑ मा॒ण्डाभ्याꣳ॒॒ स्वाहा᳚ । \newline
30. स्वाहा॒ शेपा॑य॒ शेपा॑य॒ स्वाहा॒ स्वाहा॒ शेपा॑य । \newline
31. शेपा॑य॒ स्वाहा॒ स्वाहा॒ शेपा॑य॒ शेपा॑य॒ स्वाहा᳚ । \newline
32. स्वाहा॒ रेत॑से॒ रेत॑से॒ स्वाहा॒ स्वाहा॒ रेत॑से । \newline
33. रेत॑से॒ स्वाहा॒ स्वाहा॒ रेत॑से॒ रेत॑से॒ स्वाहा᳚ । \newline
34. स्वाहा᳚ प्र॒जाभ्यः॑ प्र॒जाभ्यः॒ स्वाहा॒ स्वाहा᳚ प्र॒जाभ्यः॑ । \newline
35. प्र॒जाभ्यः॒ स्वाहा॒ स्वाहा᳚ प्र॒जाभ्यः॑ प्र॒जाभ्यः॒ स्वाहा᳚ । \newline
36. प्र॒जाभ्य॒ इति॑ प्र - जाभ्यः॑ । \newline
37. स्वाहा᳚ प्र॒जन॑नाय प्र॒जन॑नाय॒ स्वाहा॒ स्वाहा᳚ प्र॒जन॑नाय । \newline
38. प्र॒जन॑नाय॒ स्वाहा॒ स्वाहा᳚ प्र॒जन॑नाय प्र॒जन॑नाय॒ स्वाहा᳚ । \newline
39. प्र॒जन॑ना॒येति॑ प्र - जन॑नाय । \newline
40. स्वाहा॑ प॒द्भ्यः प॒द्भ्यः स्वाहा॒ स्वाहा॑ प॒द्भ्यः । \newline
41. प॒द्भ्यः स्वाहा॒ स्वाहा॑ प॒द्भ्यः प॒द्भ्यः स्वाहा᳚ । \newline
42. प॒द्भ्य इति॑ पत् - भ्यः । \newline
43. स्वाहा॑ श॒फेभ्यः॑ श॒फेभ्यः॒ स्वाहा॒ स्वाहा॑ श॒फेभ्यः॑ । \newline
44. श॒फेभ्यः॒ स्वाहा॒ स्वाहा॑ श॒फेभ्यः॑ श॒फेभ्यः॒ स्वाहा᳚ । \newline
45. स्वाहा॒ लोम॑भ्यो॒ लोम॑भ्यः॒ स्वाहा॒ स्वाहा॒ लोम॑भ्यः । \newline
46. लोम॑भ्यः॒ स्वाहा॒ स्वाहा॒ लोम॑भ्यो॒ लोम॑भ्यः॒ स्वाहा᳚ । \newline
47. लोम॑भ्य॒ इति॒ लोम॑ - भ्यः॒ । \newline
48. स्वाहा᳚ त्व॒चे त्व॒चे स्वाहा॒ स्वाहा᳚ त्व॒चे । \newline
49. त्व॒चे स्वाहा॒ स्वाहा᳚ त्व॒चे त्व॒चे स्वाहा᳚ । \newline
50. स्वाहा॒ लोहि॑ताय॒ लोहि॑ताय॒ स्वाहा॒ स्वाहा॒ लोहि॑ताय । \newline
51. लोहि॑ताय॒ स्वाहा॒ स्वाहा॒ लोहि॑ताय॒ लोहि॑ताय॒ स्वाहा᳚ । \newline
52. स्वाहा॑ माꣳ॒॒साय॑ माꣳ॒॒साय॒ स्वाहा॒ स्वाहा॑ माꣳ॒॒साय॑ । \newline
53. माꣳ॒॒साय॒ स्वाहा॒ स्वाहा॑ माꣳ॒॒साय॑ माꣳ॒॒साय॒ स्वाहा᳚ । \newline
54. स्वाहा॒ स्नाव॑भ्यः॒ स्नाव॑भ्यः॒ स्वाहा॒ स्वाहा॒ स्नाव॑भ्यः । \newline
55. स्नाव॑भ्यः॒ स्वाहा॒ स्वाहा॒ स्नाव॑भ्यः॒ स्नाव॑भ्यः॒ स्वाहा᳚ । \newline
56. स्नाव॑भ्य॒ इति॒ स्नाव॑ - भ्यः॒ । \newline
57. स्वाहा॒ ऽस्थभ्यो॒ ऽस्थभ्यः॒ स्वाहा॒ स्वाहा॒ ऽस्थभ्यः॑ । \newline
58. अ॒स्थभ्यः॒ स्वाहा॒ स्वाहा॒ ऽस्थभ्यो॒ ऽस्थभ्यः॒ स्वाहा᳚ । \newline
59. अ॒स्थभ्य॒ इत्य॒स्थ - भ्यः॒ । \newline
60. स्वाहा॑ म॒ज्जभ्यो॑ म॒ज्जभ्यः॒ स्वाहा॒ स्वाहा॑ म॒ज्जभ्यः॑ । \newline
61. म॒ज्जभ्यः॒ स्वाहा॒ स्वाहा॑ म॒ज्जभ्यो॑ म॒ज्जभ्यः॒ स्वाहा᳚ । \newline
62. म॒ज्जभ्य॒ इति॑ म॒ज्ज - भ्यः॒ । \newline
63. स्वाहा ऽङ्गे॒भ्यो ऽङ्गे᳚भ्यः॒ स्वाहा॒ स्वाहा ऽङ्गे᳚भ्यः । \newline
64. अङ्गे᳚भ्यः॒ स्वाहा॒ स्वाहा ऽङ्गे॒भ्यो ऽङ्गे᳚भ्यः॒ स्वाहा᳚ । \newline
65. स्वाहा॒ ऽऽत्मन॑ आ॒त्मने॒ स्वाहा॒ स्वाहा॒ ऽऽत्मने᳚ । \newline
66. आ॒त्मने॒ स्वाहा॒ स्वाहा॒ ऽऽत्मन॑ आ॒त्मने॒ स्वाहा᳚ । \newline
67. स्वाहा॒ सर्व॑स्मै॒ सर्व॑स्मै॒ स्वाहा॒ स्वाहा॒ सर्व॑स्मै । \newline
68. सर्व॑स्मै॒ स्वाहा॒ स्वाहा॒ सर्व॑स्मै॒ सर्व॑स्मै॒ स्वाहा᳚ । \newline
69. स्वाहेति॒ स्वाहा᳚ । \newline

\textbf{Ghana Paata } \newline

1. अꣳसा᳚भ्याꣳ॒॒ स्वाहा॒ स्वाहा ऽꣳसा᳚भ्या॒ मꣳसा᳚भ्याꣳ॒॒ स्वाहा॑ दो॒षभ्या᳚म् दो॒षभ्याꣳ॒॒ स्वाहा ऽꣳसा᳚भ्या॒ मꣳसा᳚भ्याꣳ॒॒ स्वाहा॑ दो॒षभ्या᳚म् । \newline
2. स्वाहा॑ दो॒षभ्या᳚म् दो॒षभ्याꣳ॒॒ स्वाहा॒ स्वाहा॑ दो॒षभ्याꣳ॒॒ स्वाहा॒ स्वाहा॑ दो॒षभ्याꣳ॒॒ स्वाहा॒ स्वाहा॑ दो॒षभ्याꣳ॒॒ स्वाहा᳚ । \newline
3. दो॒षभ्याꣳ॒॒ स्वाहा॒ स्वाहा॑ दो॒षभ्या᳚म् दो॒षभ्याꣳ॒॒ स्वाहा॑ बा॒हुभ्या᳚म् बा॒हुभ्याꣳ॒॒ स्वाहा॑ दो॒षभ्या᳚म् दो॒षभ्याꣳ॒॒ स्वाहा॑ बा॒हुभ्या᳚म् । \newline
4. दो॒षभ्या॒मिति॑ दो॒ष - भ्या॒म् । \newline
5. स्वाहा॑ बा॒हुभ्या᳚म् बा॒हुभ्याꣳ॒॒ स्वाहा॒ स्वाहा॑ बा॒हुभ्याꣳ॒॒ स्वाहा॒ स्वाहा॑ बा॒हुभ्याꣳ॒॒ स्वाहा॒ स्वाहा॑ बा॒हुभ्याꣳ॒॒ स्वाहा᳚ । \newline
6. बा॒हुभ्याꣳ॒॒ स्वाहा॒ स्वाहा॑ बा॒हुभ्या᳚म् बा॒हुभ्याꣳ॒॒ स्वाहा॒ जङ्घा᳚भ्या॒म् जङ्घा᳚भ्याꣳ॒॒ स्वाहा॑ बा॒हुभ्या᳚म् बा॒हुभ्याꣳ॒॒ स्वाहा॒ जङ्घा᳚भ्याम् । \newline
7. बा॒हुभ्या॒मिति॑ बा॒हु - भ्या॒म् । \newline
8. स्वाहा॒ जङ्घा᳚भ्या॒म् जङ्घा᳚भ्याꣳ॒॒ स्वाहा॒ स्वाहा॒ जङ्घा᳚भ्याꣳ॒॒ स्वाहा॒ स्वाहा॒ जङ्घा᳚भ्याꣳ॒॒ स्वाहा॒ स्वाहा॒ जङ्घा᳚भ्याꣳ॒॒ स्वाहा᳚ । \newline
9. जङ्घा᳚भ्याꣳ॒॒ स्वाहा॒ स्वाहा॒ जङ्घा᳚भ्या॒म् जङ्घा᳚भ्याꣳ॒॒ स्वाहा॒ श्रोणी᳚भ्याꣳ॒॒ श्रोणी᳚भ्याꣳ॒॒ स्वाहा॒ जङ्घा᳚भ्या॒म् जङ्घा᳚भ्याꣳ॒॒ स्वाहा॒ श्रोणी᳚भ्याम् । \newline
10. स्वाहा॒ श्रोणी᳚भ्याꣳ॒॒ श्रोणी᳚भ्याꣳ॒॒ स्वाहा॒ स्वाहा॒ श्रोणी᳚भ्याꣳ॒॒ स्वाहा॒ स्वाहा॒ श्रोणी᳚भ्याꣳ॒॒ स्वाहा॒ स्वाहा॒ श्रोणी᳚भ्याꣳ॒॒ स्वाहा᳚ । \newline
11. श्रोणी᳚भ्याꣳ॒॒ स्वाहा॒ स्वाहा॒ श्रोणी᳚भ्याꣳ॒॒ श्रोणी᳚भ्याꣳ॒॒ स्वाहो॒रुभ्या॑ मू॒रुभ्याꣳ॒॒ स्वाहा॒ श्रोणी᳚भ्याꣳ॒॒ श्रोणी᳚भ्याꣳ॒॒ स्वाहो॒रुभ्या᳚म् । \newline
12. श्रोणी᳚भ्या॒मिति॒ श्रोणि॑ - भ्या॒म् । \newline
13. स्वाहो॒रुभ्या॑ मू॒रुभ्याꣳ॒॒ स्वाहा॒ स्वाहो॒ रुभ्याꣳ॒॒ स्वाहा॒ स्वाहो॒ रुभ्याꣳ॒॒ स्वाहा॒ स्वाहो॒ रुभ्याꣳ॒॒ स्वाहा᳚ । \newline
14. ऊ॒रुभ्याꣳ॒॒ स्वाहा॒ स्वाहो॒रुभ्या॑ मू॒रुभ्याꣳ॒॒ स्वाहा᳚ ऽष्ठी॒वद्भ्या॑ मष्ठी॒वद्भ्याꣳ॒॒ स्वाहो॒रुभ्या॑ मू॒रुभ्याꣳ॒॒ स्वाहा᳚ ऽष्ठी॒वद्भ्या᳚म् । \newline
15. ऊ॒रुभ्या॒मित्यू॒रु - भ्या॒म् । \newline
16. स्वाहा᳚ ऽष्ठी॒वद्भ्या॑ मष्ठी॒वद्भ्याꣳ॒॒ स्वाहा॒ स्वाहा᳚ ऽष्ठी॒वद्भ्याꣳ॒॒ स्वाहा॒ स्वाहा᳚ ऽष्ठी॒वद्भ्याꣳ॒॒ स्वाहा॒ स्वाहा᳚ ऽष्ठी॒वद्भ्याꣳ॒॒ स्वाहा᳚ । \newline
17. अ॒ष्ठी॒वद्भ्याꣳ॒॒ स्वाहा॒ स्वाहा᳚ ऽष्ठी॒वद्भ्या॑ मष्ठी॒वद्भ्याꣳ॒॒ स्वाहा॒ जङ्घा᳚भ्या॒म् जङ्घा᳚भ्याꣳ॒॒ स्वाहा᳚ ऽष्ठी॒वद्भ्या॑ मष्ठी॒वद्भ्याꣳ॒॒ स्वाहा॒ जङ्घा᳚भ्याम् । \newline
18. अ॒ष्ठी॒वद्भ्या॒मित्य॑ष्ठी॒वत् - भ्या॒म् । \newline
19. स्वाहा॒ जङ्घा᳚भ्या॒म् जङ्घा᳚भ्याꣳ॒॒ स्वाहा॒ स्वाहा॒ जङ्घा᳚भ्याꣳ॒॒ स्वाहा॒ स्वाहा॒ जङ्घा᳚भ्याꣳ॒॒ स्वाहा॒ स्वाहा॒ जङ्घा᳚भ्याꣳ॒॒ स्वाहा᳚ । \newline
20. जङ्घा᳚भ्याꣳ॒॒ स्वाहा॒ स्वाहा॒ जङ्घा᳚भ्या॒म् जङ्घा᳚भ्याꣳ॒॒ स्वाहा॑ भ॒सदे॑ भ॒सदे॒ स्वाहा॒ जङ्घा᳚भ्या॒म् जङ्घा᳚भ्याꣳ॒॒ स्वाहा॑ भ॒सदे᳚ । \newline
21. स्वाहा॑ भ॒सदे॑ भ॒सदे॒ स्वाहा॒ स्वाहा॑ भ॒सदे॒ स्वाहा॒ स्वाहा॑ भ॒सदे॒ स्वाहा॒ स्वाहा॑ भ॒सदे॒ स्वाहा᳚ । \newline
22. भ॒सदे॒ स्वाहा॒ स्वाहा॑ भ॒सदे॑ भ॒सदे॒ स्वाहा॑ शिख॒ण्डेभ्यः॑ शिख॒ण्डेभ्यः॒ स्वाहा॑ भ॒सदे॑ भ॒सदे॒ स्वाहा॑ शिख॒ण्डेभ्यः॑ । \newline
23. स्वाहा॑ शिख॒ण्डेभ्यः॑ शिख॒ण्डेभ्यः॒ स्वाहा॒ स्वाहा॑ शिख॒ण्डेभ्यः॒ स्वाहा॒ स्वाहा॑ शिख॒ण्डेभ्यः॒ स्वाहा॒ स्वाहा॑ शिख॒ण्डेभ्यः॒ स्वाहा᳚ । \newline
24. शि॒ख॒ण्डेभ्यः॒ स्वाहा॒ स्वाहा॑ शिख॒ण्डेभ्यः॑ शिख॒ण्डेभ्यः॒ स्वाहा॑ वाल॒धाना॑य वाल॒धाना॑य॒ स्वाहा॑ शिख॒ण्डेभ्यः॑ शिख॒ण्डेभ्यः॒ स्वाहा॑ वाल॒धाना॑य । \newline
25. स्वाहा॑ वाल॒धाना॑य वाल॒धाना॑य॒ स्वाहा॒ स्वाहा॑ वाल॒धाना॑य॒ स्वाहा॒ स्वाहा॑ वाल॒धाना॑य॒ स्वाहा॒ स्वाहा॑ वाल॒धाना॑य॒ स्वाहा᳚ । \newline
26. वा॒ल॒धाना॑य॒ स्वाहा॒ स्वाहा॑ वाल॒धाना॑य वाल॒धाना॑य॒ स्वाहा॒ ऽऽण्डाभ्या॑ मा॒ण्डाभ्याꣳ॒॒ स्वाहा॑ वाल॒धाना॑य वाल॒धाना॑य॒ स्वाहा॒ ऽऽण्डाभ्या᳚म् । \newline
27. वा॒ल॒धाना॒येति॑ वाल - धाना॑य । \newline
28. स्वाहा॒ ऽऽण्डाभ्या॑ मा॒ण्डाभ्याꣳ॒॒ स्वाहा॒ स्वाहा॒ ऽऽण्डाभ्याꣳ॒॒ स्वाहा॒ स्वाहा॒ ऽऽण्डाभ्याꣳ॒॒ स्वाहा॒ स्वाहा॒ ऽऽण्डाभ्याꣳ॒॒ स्वाहा᳚ । \newline
29. आ॒ण्डाभ्याꣳ॒॒ स्वाहा॒ स्वाहा॒ ऽऽण्डाभ्या॑ मा॒ण्डाभ्याꣳ॒॒ स्वाहा॒ शेपा॑य॒ शेपा॑य॒ स्वाहा॒ ऽऽण्डाभ्या॑ मा॒ण्डाभ्याꣳ॒॒ स्वाहा॒ शेपा॑य । \newline
30. स्वाहा॒ शेपा॑य॒ शेपा॑य॒ स्वाहा॒ स्वाहा॒ शेपा॑य॒ स्वाहा॒ स्वाहा॒ शेपा॑य॒ स्वाहा॒ स्वाहा॒ शेपा॑य॒ स्वाहा᳚ । \newline
31. शेपा॑य॒ स्वाहा॒ स्वाहा॒ शेपा॑य॒ शेपा॑य॒ स्वाहा॒ रेत॑से॒ रेत॑से॒ स्वाहा॒ शेपा॑य॒ शेपा॑य॒ स्वाहा॒ रेत॑से । \newline
32. स्वाहा॒ रेत॑से॒ रेत॑से॒ स्वाहा॒ स्वाहा॒ रेत॑से॒ स्वाहा॒ स्वाहा॒ रेत॑से॒ स्वाहा॒ स्वाहा॒ रेत॑से॒ स्वाहा᳚ । \newline
33. रेत॑से॒ स्वाहा॒ स्वाहा॒ रेत॑से॒ रेत॑से॒ स्वाहा᳚ प्र॒जाभ्यः॑ प्र॒जाभ्यः॒ स्वाहा॒ रेत॑से॒ रेत॑से॒ स्वाहा᳚ प्र॒जाभ्यः॑ । \newline
34. स्वाहा᳚ प्र॒जाभ्यः॑ प्र॒जाभ्यः॒ स्वाहा॒ स्वाहा᳚ प्र॒जाभ्यः॒ स्वाहा॒ स्वाहा᳚ प्र॒जाभ्यः॒ स्वाहा॒ स्वाहा᳚ प्र॒जाभ्यः॒ स्वाहा᳚ । \newline
35. प्र॒जाभ्यः॒ स्वाहा॒ स्वाहा᳚ प्र॒जाभ्यः॑ प्र॒जाभ्यः॒ स्वाहा᳚ प्र॒जन॑नाय प्र॒जन॑नाय॒ स्वाहा᳚ प्र॒जाभ्यः॑ प्र॒जाभ्यः॒ स्वाहा᳚ प्र॒जन॑नाय । \newline
36. प्र॒जाभ्य॒ इति॑ प्र - जाभ्यः॑ । \newline
37. स्वाहा᳚ प्र॒जन॑नाय प्र॒जन॑नाय॒ स्वाहा॒ स्वाहा᳚ प्र॒जन॑नाय॒ स्वाहा॒ स्वाहा᳚ प्र॒जन॑नाय॒ स्वाहा॒ स्वाहा᳚ प्र॒जन॑नाय॒ स्वाहा᳚ । \newline
38. प्र॒जन॑नाय॒ स्वाहा॒ स्वाहा᳚ प्र॒जन॑नाय प्र॒जन॑नाय॒ स्वाहा॑ प॒द्भ्यः प॒द्भ्यः स्वाहा᳚ प्र॒जन॑नाय प्र॒जन॑नाय॒ स्वाहा॑ प॒द्भ्यः । \newline
39. प्र॒जन॑ना॒येति॑ प्र - जन॑नाय । \newline
40. स्वाहा॑ प॒द्भ्यः प॒द्भ्यः स्वाहा॒ स्वाहा॑ प॒द्भ्यः स्वाहा॒ स्वाहा॑ प॒द्भ्यः स्वाहा॒ स्वाहा॑ प॒द्भ्यः स्वाहा᳚ । \newline
41. प॒द्भ्यः स्वाहा॒ स्वाहा॑ प॒द्भ्यः प॒द्भ्यः स्वाहा॑ श॒फेभ्यः॑ श॒फेभ्यः॒ स्वाहा॑ प॒द्भ्यः प॒द्भ्यः स्वाहा॑ श॒फेभ्यः॑ । \newline
42. प॒द्भ्य इति॑ पत् - भ्यः । \newline
43. स्वाहा॑ श॒फेभ्यः॑ श॒फेभ्यः॒ स्वाहा॒ स्वाहा॑ श॒फेभ्यः॒ स्वाहा॒ स्वाहा॑ श॒फेभ्यः॒ स्वाहा॒ स्वाहा॑ श॒फेभ्यः॒ स्वाहा᳚ । \newline
44. श॒फेभ्यः॒ स्वाहा॒ स्वाहा॑ श॒फेभ्यः॑ श॒फेभ्यः॒ स्वाहा॒ लोम॑भ्यो॒ लोम॑भ्यः॒ स्वाहा॑ श॒फेभ्यः॑ श॒फेभ्यः॒ स्वाहा॒ लोम॑भ्यः । \newline
45. स्वाहा॒ लोम॑भ्यो॒ लोम॑भ्यः॒ स्वाहा॒ स्वाहा॒ लोम॑भ्यः॒ स्वाहा॒ स्वाहा॒ लोम॑भ्यः॒ स्वाहा॒ स्वाहा॒ लोम॑भ्यः॒ स्वाहा᳚ । \newline
46. लोम॑भ्यः॒ स्वाहा॒ स्वाहा॒ लोम॑भ्यो॒ लोम॑भ्यः॒ स्वाहा᳚ त्व॒चे त्व॒चे स्वाहा॒ लोम॑भ्यो॒ लोम॑भ्यः॒ स्वाहा᳚ त्व॒चे । \newline
47. लोम॑भ्य॒ इति॒ लोम॑ - भ्यः॒ । \newline
48. स्वाहा᳚ त्व॒चे त्व॒चे स्वाहा॒ स्वाहा᳚ त्व॒चे स्वाहा॒ स्वाहा᳚ त्व॒चे स्वाहा॒ स्वाहा᳚ त्व॒चे स्वाहा᳚ । \newline
49. त्व॒चे स्वाहा॒ स्वाहा᳚ त्व॒चे त्व॒चे स्वाहा॒ लोहि॑ताय॒ लोहि॑ताय॒ स्वाहा᳚ त्व॒चे त्व॒चे स्वाहा॒ लोहि॑ताय । \newline
50. स्वाहा॒ लोहि॑ताय॒ लोहि॑ताय॒ स्वाहा॒ स्वाहा॒ लोहि॑ताय॒ स्वाहा॒ स्वाहा॒ लोहि॑ताय॒ स्वाहा॒ स्वाहा॒ लोहि॑ताय॒ स्वाहा᳚ । \newline
51. लोहि॑ताय॒ स्वाहा॒ स्वाहा॒ लोहि॑ताय॒ लोहि॑ताय॒ स्वाहा॑ माꣳ॒॒साय॑ माꣳ॒॒साय॒ स्वाहा॒ लोहि॑ताय॒ लोहि॑ताय॒ स्वाहा॑ माꣳ॒॒साय॑ । \newline
52. स्वाहा॑ माꣳ॒॒साय॑ माꣳ॒॒साय॒ स्वाहा॒ स्वाहा॑ माꣳ॒॒साय॒ स्वाहा॒ स्वाहा॑ माꣳ॒॒साय॒ स्वाहा॒ स्वाहा॑ माꣳ॒॒साय॒ स्वाहा᳚ । \newline
53. माꣳ॒॒साय॒ स्वाहा॒ स्वाहा॑ माꣳ॒॒साय॑ माꣳ॒॒साय॒ स्वाहा॒ स्नाव॑भ्यः॒ स्नाव॑भ्यः॒ स्वाहा॑ माꣳ॒॒साय॑ माꣳ॒॒साय॒ स्वाहा॒ स्नाव॑भ्यः । \newline
54. स्वाहा॒ स्नाव॑भ्यः॒ स्नाव॑भ्यः॒ स्वाहा॒ स्वाहा॒ स्नाव॑भ्यः॒ स्वाहा॒ स्वाहा॒ स्नाव॑भ्यः॒ स्वाहा॒ स्वाहा॒ स्नाव॑भ्यः॒ स्वाहा᳚ । \newline
55. स्नाव॑भ्यः॒ स्वाहा॒ स्वाहा॒ स्नाव॑भ्यः॒ स्नाव॑भ्यः॒ स्वाहा॒ ऽस्थभ्यो॒ ऽस्थभ्यः॒ स्वाहा॒ स्नाव॑भ्यः॒ स्नाव॑भ्यः॒ स्वाहा॒ ऽस्थभ्यः॑ । \newline
56. स्नाव॑भ्य॒ इति॒ स्नाव॑ - भ्यः॒ । \newline
57. स्वाहा॒ ऽस्थभ्यो॒ ऽस्थभ्यः॒ स्वाहा॒ स्वाहा॒ ऽस्थभ्यः॒ स्वाहा॒ स्वाहा॒ ऽस्थभ्यः॒ स्वाहा॒ स्वाहा॒ ऽस्थभ्यः॒ स्वाहा᳚ । \newline
58. अ॒स्थभ्यः॒ स्वाहा॒ स्वाहा॒ ऽस्थभ्यो॒ ऽस्थभ्यः॒ स्वाहा॑ म॒ज्जभ्यो॑ म॒ज्जभ्यः॒ स्वाहा॒ ऽस्थभ्यो॒ ऽस्थभ्यः॒ स्वाहा॑ म॒ज्जभ्यः॑ । \newline
59. अ॒स्थभ्य॒ इत्य॒स्थ - भ्यः॒ । \newline
60. स्वाहा॑ म॒ज्जभ्यो॑ म॒ज्जभ्यः॒ स्वाहा॒ स्वाहा॑ म॒ज्जभ्यः॒ स्वाहा॒ स्वाहा॑ म॒ज्जभ्यः॒ स्वाहा॒ स्वाहा॑ म॒ज्जभ्यः॒ स्वाहा᳚ । \newline
61. म॒ज्जभ्यः॒ स्वाहा॒ स्वाहा॑ म॒ज्जभ्यो॑ म॒ज्जभ्यः॒ स्वाहा ऽङ्गे॒भ्यो ऽङ्गे᳚भ्यः॒ स्वाहा॑ म॒ज्जभ्यो॑ म॒ज्जभ्यः॒ स्वाहा ऽङ्गे᳚भ्यः । \newline
62. म॒ज्जभ्य॒ इति॑ म॒ज्ज - भ्यः॒ । \newline
63. स्वाहा ऽङ्गे॒भ्यो ऽङ्गे᳚भ्यः॒ स्वाहा॒ स्वाहा ऽङ्गे᳚भ्यः॒ स्वाहा॒ स्वाहा ऽङ्गे᳚भ्यः॒ स्वाहा॒ स्वाहा ऽङ्गे᳚भ्यः॒ स्वाहा᳚ । \newline
64. अङ्गे᳚भ्यः॒ स्वाहा॒ स्वाहा ऽङ्गे॒भ्यो ऽङ्गे᳚भ्यः॒ स्वाहा॒ ऽऽत्मन॑ आ॒त्मने॒ स्वाहा ऽङ्गे॒भ्यो ऽङ्गे᳚भ्यः॒ स्वाहा॒ ऽऽत्मने᳚ । \newline
65. स्वाहा॒ ऽऽत्मन॑ आ॒त्मने॒ स्वाहा॒ स्वाहा॒ ऽऽत्मने॒ स्वाहा॒ स्वाहा॒ ऽऽत्मने॒ स्वाहा॒ स्वाहा॒ ऽऽत्मने॒ स्वाहा᳚ । \newline
66. आ॒त्मने॒ स्वाहा॒ स्वाहा॒ ऽऽत्मन॑ आ॒त्मने॒ स्वाहा॒ सर्व॑स्मै॒ सर्व॑स्मै॒ स्वाहा॒ ऽऽत्मन॑ आ॒त्मने॒ स्वाहा॒ सर्व॑स्मै । \newline
67. स्वाहा॒ सर्व॑स्मै॒ सर्व॑स्मै॒ स्वाहा॒ स्वाहा॒ सर्व॑स्मै॒ स्वाहा॒ स्वाहा॒ सर्व॑स्मै॒ स्वाहा॒ स्वाहा॒ सर्व॑स्मै॒ स्वाहा᳚ । \newline
68. सर्व॑स्मै॒ स्वाहा॒ स्वाहा॒ सर्व॑स्मै॒ सर्व॑स्मै॒ स्वाहा᳚ । \newline
69. स्वाहेति॒ स्वाहा᳚ । \newline
\pagebreak
\markright{ TS 7.3.17.1  \hfill https://www.vedavms.in \hfill}

\section{ TS 7.3.17.1 }

\textbf{TS 7.3.17.1 } \newline
\textbf{Samhita Paata} \newline

अ॒ञ्ज्ये॒ताय॒ स्वाहा᳚ ऽञ्जिस॒क्थाय॒ स्वाहा॑ शिति॒पदे॒ स्वाहा॒ शिति॑ककुदे॒ स्वाहा॑ शिति॒रन्ध्रा॑य॒ स्वाहा॑ शितिपृ॒ष्ठाय॒ स्वाहा॑ शि॒त्यꣳसा॑य॒ स्वाहा॑ पुष्प॒कर्णा॑य॒ स्वाहा॑ शि॒त्योष्ठा॑य॒ स्वाहा॑ शिति॒भ्रवे॒ स्वाहा॒ शिति॑भसदे॒ स्वाहा᳚ श्वे॒तानू॑काशाय॒ स्वाहा॒ ऽञ्जये॒ स्वाहा॑ ल॒लामा॑य॒ स्वाहा ऽसि॑तज्ञ्वे॒ स्वाहा॑ कृष्णै॒ताय॒ स्वाहा॑ रोहितै॒ताय॒ स्वाहा॑ ऽरुणै॒ताय॒ स्वाहे॒दृशा॑य॒ स्वाहा॑ की॒दृशा॑य॒ स्वाहा॑ ता॒दृशा॑य॒ स्वाहा॑ स॒दृशा॑य॒ स्वाहा॒ विस॑दृशाय॒ स्वाहा॒ सुस॑दृशाय॒ स्वाहा॑ रू॒पाय॒ स्वाहा॒ ( ) सर्व॑स्मै॒ स्वाहा᳚ ॥ \newline

\textbf{Pada Paata} \newline

अ॒ञ्ज्ये॒तायेत्य॑ञ्जि - ए॒ताय॑ । स्वाहा᳚ । अ॒ञ्जि॒स॒क्थायेत्य॑ञ्जि-स॒क्थाय॑ । स्वाहा᳚ । शि॒ति॒पद॒ इति॑ शिति - पदे᳚ । स्वाहा᳚ । शिति॑ककुद॒ इति॒ शिति॑ - क॒कु॒दे॒ । स्वाहा᳚ । शि॒ति॒रन्ध्रा॒येति॑ शिति - रन्ध्रा॑य । स्वाहा᳚ । शि॒ति॒पृ॒ष्ठायेति॑ शिति - पृ॒ष्ठाय॑ । स्वाहा᳚ । शि॒त्यꣳसा॒येति॑ शिति - अꣳसा॑य । स्वाहा᳚ । पु॒ष्प॒कर्णा॒येति॑ पुष्प - कर्णा॑य । स्वाहा᳚ । शि॒त्योष्ठा॒येति॑ शिति - ओष्ठा॑य । स्वाहा᳚ । शि॒ति॒भ्रव॒ इति॑ शिति - भ्रवे᳚ । स्वाहा᳚ । शिति॑भसद॒ इति॒ शिति॑ - भ॒स॒दे॒ । स्वाहा᳚ । श्वे॒तानू॑काशा॒येति॑ श्वे॒त - अ॒नू॒का॒शा॒य॒ । स्वाहा᳚ । अ॒ञ्जये᳚ । स्वाहा᳚ । ल॒लामा॑य । स्वाहा᳚ । असि॑तज्ञ्व॒ इत्यसि॑त - ज्ञ्॒वे॒ । स्वाहा᳚ । कृ॒ष्णै॒तायेति॑ कृष्ण - ए॒ताय॑ । स्वाहा᳚ । रो॒हि॒तै॒तायेति॑ रोहित - ए॒ताय॑ । स्वाहा᳚ । अ॒रु॒णै॒तायेत्य॑रुण - ए॒ताय॑ । स्वाहा᳚ । ई॒दृशा॑य । स्वाहा᳚ । की॒दृशा॑य । स्वाहा᳚ । ता॒दृशा॑य । स्वाहा᳚ । स॒दृशा॑य । स्वाहा᳚ । विस॑दृशा॒येति॒ वि-स॒दृ॒शा॒य॒ । स्वाहा᳚ । सुस॑दृशा॒येति॒ सु - स॒दृ॒शा॒य॒ । स्वाहा᳚ । रू॒पाय॑ । स्वाहा᳚ ( ) । सर्व॑स्मै । स्वाहा᳚ ॥  \newline


\textbf{Krama Paata} \newline

अ॒ञ्ज्ये॒ताय॒ स्वाहा᳚ । अ॒ञ्ज्ये॒तायेत्य॑ञ्जि - ए॒ताय॑ । स्वाहा᳚ऽञ्जिस॒क्थाय॑ । अ॒ञ्जि॒स॒क्थाय॒ स्वाहा᳚ । अ॒ञ्जि॒स॒क्थायेत्य॑ञ्जि - स॒क्थाय॑ । स्वाहा॑ शिति॒पदे᳚ । शि॒ति॒पदे॒ स्वाहा᳚ । शि॒ति॒पद॒ इति॑ शिति - पदे᳚ । स्वाहा॒ शिति॑ककुदे । शिति॑ककुदे॒ स्वाहा᳚ । शिति॑ककुद॒ इति॒ शिति॑ - क॒कु॒दे॒ । स्वाहा॑ शिति॒रन्ध्रा॑य । शि॒ति॒रन्ध्रा॑य॒ स्वाहा᳚ । शि॒ति॒रन्ध्रा॒येति॑ शिति - रन्ध्रा॑य । स्वाहा॑ शितिपृ॒ष्ठाय॑ । शि॒ति॒पृ॒ष्ठाय॒ स्वाहा᳚ । शि॒ति॒पृ॒ष्ठायेति॑ शिति - पृ॒ष्ठाय॑ । स्वाहा॑ शि॒त्यꣳसा॑य । शि॒त्यꣳसा॑य॒ स्वाहा᳚ । शि॒त्यꣳसा॒येति॑ शिति - अꣳसा॑य । स्वाहा॑ पुष्प॒कर्णा॑य । पु॒ष्प॒कर्णा॑य॒ स्वाहा᳚ । पु॒ष्प॒कर्णा॒येति॑ पुष्प - कर्णा॑य । स्वाहा॑ शि॒त्योष्ठा॑य । शि॒त्योष्ठा॑य॒ स्वाहा᳚ । शि॒त्योष्ठा॒येति॑ शिति - ओष्ठा॑य । स्वाहा॑ शिति॒भ्रवे᳚ । शि॒ति॒भ्रवे॒ स्वाहा᳚ । शि॒ति॒भ्रव॒ इति॑ शिति - भ्रवे᳚ । स्वाहा॒ शिति॑भसदे । शिति॑भसदे॒ स्वाहा᳚ । शिति॑भसद॒ इति॒ शिति॑ - भ॒स॒दे॒ । स्वाहा᳚ श्वे॒तानू॑काशाय । श्वे॒तानू॑काशाय॒ स्वाहा᳚ । श्वे॒तानू॑काशा॒येति॑ श्वे॒त - अ॒नू॒का॒शा॒य॒ । स्वाहा॒ञ्जये᳚ । अ॒ञ्जये॒ स्वाहा᳚ । स्वाहा॑ ल॒लामा॑य । ल॒लामा॑य॒ स्वाहा᳚ । स्वाहाऽसि॑तज्ञ्वे । असि॑तज्ञ्वे॒ स्वाहा᳚ । असि॑तज्ञ्व॒ इत्यसि॑त - ज्ञ्॒वे॒ । स्वाहा॑ कृष्णै॒ताय॑ । कृ॒ष्णै॒ताय॒ स्वाहा᳚ । कृ॒ष्णै॒तायेति॑ कृष्ण - ए॒ताय॑ । स्वाहा॑ रोहितै॒ताय॑ । रो॒हि॒तै॒ताय॒ स्वाहा᳚ । रो॒हि॒तै॒तायेति॑ रोहित - ए॒ताय॑ । स्वाहा॑ऽरुणै॒ताय॑ । अ॒रु॒णै॒ताय॒ स्वाहा᳚ । अ॒रु॒णै॒तायेत्य॑रुण - ए॒ताय॑ । स्वाहे॒दृशा॑य । ई॒दृशा॑य॒ स्वाहा᳚ । स्वाहा॑ की॒दृशा॑य । की॒दृशा॑य॒ स्वाहा᳚ । स्वाहा॑ ता॒दृशा॑य । ता॒दृशा॑य॒ स्वाहा᳚ । स्वाहा॑ स॒दृशा॑य । स॒दृशा॑य॒ स्वाहा᳚ । स्वाहा॒ विस॑दृशाय । विस॑दृशाय॒ स्वाहा᳚ । विसृ॑दृशा॒येति॒वि - स॒दृ॒शा॒य॒ । स्वाहा॒ सुस॑दृशाय । सुस॑दृशाय॒ स्वाहा᳚ । सुस॑दृशा॒येति॒ सु - स॒दृ॒शा॒य॒ । स्वाहा॑ रू॒पाय॑ । रू॒पाय॒ स्वाहा᳚ ( ) । स्वाहा॒ सर्व॑स्मै । सर्व॑स्मै॒ स्वाहा᳚ । 
स्वाहेति॒ स्वाहा᳚ । \newline

\textbf{Jatai Paata} \newline

1. अ॒ञ्ज्ये॒ताय॒ स्वाहा॒ स्वाहा᳚ ऽञ्ज्ये॒ताया᳚ ञ्ज्ये॒ताय॒ स्वाहा᳚ । \newline
2. अ॒ञ्ज्ये॒तायेत्य॑ञ्जि - ए॒ताय॑ । \newline
3. स्वाहा᳚ ऽञ्जिस॒क्थाया᳚ ञ्जिस॒क्थाय॒ स्वाहा॒ स्वाहा᳚ ऽञ्जिस॒क्थाय॑ । \newline
4. अ॒ञ्जि॒स॒क्थाय॒ स्वाहा॒ स्वाहा᳚ ऽञ्जिस॒क्थाया᳚ ञ्जिस॒क्थाय॒ स्वाहा᳚ । \newline
5. अ॒ञ्जि॒स॒क्थायेत्य॑ञ्जि - स॒क्थाय॑ । \newline
6. स्वाहा॑ शिति॒पदे॑ शिति॒पदे॒ स्वाहा॒ स्वाहा॑ शिति॒पदे᳚ । \newline
7. शि॒ति॒पदे॒ स्वाहा॒ स्वाहा॑ शिति॒पदे॑ शिति॒पदे॒ स्वाहा᳚ । \newline
8. शि॒ति॒पद॒ इति॑ शिति - पदे᳚ । \newline
9. स्वाहा॒ शिति॑ककुदे॒ शिति॑ककुदे॒ स्वाहा॒ स्वाहा॒ शिति॑ककुदे । \newline
10. शिति॑ककुदे॒ स्वाहा॒ स्वाहा॒ शिति॑ककुदे॒ शिति॑ककुदे॒ स्वाहा᳚ । \newline
11. शिति॑ककुद॒ इति॒ शिति॑ - क॒कु॒दे॒ । \newline
12. स्वाहा॑ शिति॒रन्ध्रा॑य शिति॒रन्ध्रा॑य॒ स्वाहा॒ स्वाहा॑ शिति॒रन्ध्रा॑य । \newline
13. शि॒ति॒रन्ध्रा॑य॒ स्वाहा॒ स्वाहा॑ शिति॒रन्ध्रा॑य शिति॒रन्ध्रा॑य॒ स्वाहा᳚ । \newline
14. शि॒ति॒रन्ध्रा॒येति॑ शिति - रन्ध्रा॑य । \newline
15. स्वाहा॑ शितिपृ॒ष्ठाय॑ शितिपृ॒ष्ठाय॒ स्वाहा॒ स्वाहा॑ शितिपृ॒ष्ठाय॑ । \newline
16. शि॒ति॒पृ॒ष्ठाय॒ स्वाहा॒ स्वाहा॑ शितिपृ॒ष्ठाय॑ शितिपृ॒ष्ठाय॒ स्वाहा᳚ । \newline
17. शि॒ति॒पृ॒ष्ठायेति॑ शिति - पृ॒ष्ठाय॑ । \newline
18. स्वाहा॑ शि॒त्यꣳसा॑य शि॒त्यꣳसा॑य॒ स्वाहा॒ स्वाहा॑ शि॒त्यꣳसा॑य । \newline
19. शि॒त्यꣳसा॑य॒ स्वाहा॒ स्वाहा॑ शि॒त्यꣳसा॑य शि॒त्यꣳसा॑य॒ स्वाहा᳚ । \newline
20. शि॒त्यꣳसा॒येति॑ शिति - अꣳसा॑य । \newline
21. स्वाहा॑ पुष्प॒कर्णा॑य पुष्प॒कर्णा॑य॒ स्वाहा॒ स्वाहा॑ पुष्प॒कर्णा॑य । \newline
22. पु॒ष्प॒कर्णा॑य॒ स्वाहा॒ स्वाहा॑ पुष्प॒कर्णा॑य पुष्प॒कर्णा॑य॒ स्वाहा᳚ । \newline
23. पु॒ष्प॒कर्णा॒येति॑ पुष्प - कर्णा॑य । \newline
24. स्वाहा॑ शि॒त्योष्ठा॑य शि॒त्योष्ठा॑य॒ स्वाहा॒ स्वाहा॑ शि॒त्योष्ठा॑य । \newline
25. शि॒त्योष्ठा॑य॒ स्वाहा॒ स्वाहा॑ शि॒त्योष्ठा॑य शि॒त्योष्ठा॑य॒ स्वाहा᳚ । \newline
26. शि॒त्योष्ठा॒येति॑ शिति - ओष्ठा॑य । \newline
27. स्वाहा॑ शिति॒भ्रवे॑ शिति॒भ्रवे॒ स्वाहा॒ स्वाहा॑ शिति॒भ्रवे᳚ । \newline
28. शि॒ति॒भ्रवे॒ स्वाहा॒ स्वाहा॑ शिति॒भ्रवे॑ शिति॒भ्रवे॒ स्वाहा᳚ । \newline
29. शि॒ति॒भ्रव॒ इति॑ शिति - भ्रवे᳚ । \newline
30. स्वाहा॒ शिति॑भसदे॒ शिति॑भसदे॒ स्वाहा॒ स्वाहा॒ शिति॑भसदे । \newline
31. शिति॑भसदे॒ स्वाहा॒ स्वाहा॒ शिति॑भसदे॒ शिति॑भसदे॒ स्वाहा᳚ । \newline
32. शिति॑भसद॒ इति॒ शिति॑ - भ॒स॒दे॒ । \newline
33. स्वाहा᳚ श्वे॒तानू॑काशाय श्वे॒तानू॑काशाय॒ स्वाहा॒ स्वाहा᳚ श्वे॒तानू॑काशाय । \newline
34. श्वे॒तानू॑काशाय॒ स्वाहा॒ स्वाहा᳚ श्वे॒तानू॑काशाय श्वे॒तानू॑काशाय॒ स्वाहा᳚ । \newline
35. श्वे॒तानू॑काशा॒येति॑ श्वे॒त - अ॒नू॒का॒शा॒य॒ । \newline
36. स्वाहा॒ ऽञ्जये॒ ऽञ्जये॒ स्वाहा॒ स्वाहा॒ ऽञ्जये᳚ । \newline
37. अ॒ञ्जये॒ स्वाहा॒ स्वाहा॒ ऽञ्जये॒ ऽञ्जये॒ स्वाहा᳚ । \newline
38. स्वाहा॑ ल॒लामा॑य ल॒लामा॑य॒ स्वाहा॒ स्वाहा॑ ल॒लामा॑य । \newline
39. ल॒लामा॑य॒ स्वाहा॒ स्वाहा॑ ल॒लामा॑य ल॒लामा॑य॒ स्वाहा᳚ । \newline
40. स्वाहा ऽसि॑तज्ञ्॒वे ऽसि॑तज्ञ्वे॒ स्वाहा॒ स्वाहा ऽसि॑तज्ञ्वे । \newline
41. असि॑तज्ञ्वे॒ स्वाहा॒ स्वाहा ऽसि॑तज्ञ्॒वे ऽसि॑तज्ञ्वे॒ स्वाहा᳚ । \newline
42. असि॑तज्ञ्व॒ इत्यसि॑त - ज्ञ्॒वे॒ । \newline
43. स्वाहा॑ कृष्णै॒ताय॑ कृष्णै॒ताय॒ स्वाहा॒ स्वाहा॑ कृष्णै॒ताय॑ । \newline
44. कृ॒ष्णै॒ताय॒ स्वाहा॒ स्वाहा॑ कृष्णै॒ताय॑ कृष्णै॒ताय॒ स्वाहा᳚ । \newline
45. कृ॒ष्णै॒तायेति॑ कृष्ण - ए॒ताय॑ । \newline
46. स्वाहा॑ रोहितै॒ताय॑ रोहितै॒ताय॒ स्वाहा॒ स्वाहा॑ रोहितै॒ताय॑ । \newline
47. रो॒हि॒तै॒ताय॒ स्वाहा॒ स्वाहा॑ रोहितै॒ताय॑ रोहितै॒ताय॒ स्वाहा᳚ । \newline
48. रो॒हि॒तै॒तायेति॑ रोहित - ए॒ताय॑ । \newline
49. स्वाहा॑ ऽरुणै॒ताया॑ रुणै॒ताय॒ स्वाहा॒ स्वाहा॑ ऽरुणै॒ताय॑ । \newline
50. अ॒रु॒णै॒ताय॒ स्वाहा॒ स्वाहा॑ ऽरुणै॒ताया॑ रुणै॒ताय॒ स्वाहा᳚ । \newline
51. अ॒रु॒णै॒तायेत्य॑रुण - ए॒ताय॑ । \newline
52. स्वाहे॒दृशा॑ये॒ दृशा॑य॒ स्वाहा॒ स्वाहे॒दृशा॑य । \newline
53. ई॒दृशा॑य॒ स्वाहा॒ स्वाहे॒दृशा॑ ये॒दृशा॑य॒ स्वाहा᳚ । \newline
54. स्वाहा॑ की॒दृशा॑य की॒दृशा॑य॒ स्वाहा॒ स्वाहा॑ की॒दृशा॑य । \newline
55. की॒दृशा॑य॒ स्वाहा॒ स्वाहा॑ की॒दृशा॑य की॒दृशा॑य॒ स्वाहा᳚ । \newline
56. स्वाहा॑ ता॒दृशा॑य ता॒दृशा॑य॒ स्वाहा॒ स्वाहा॑ ता॒दृशा॑य । \newline
57. ता॒दृशा॑य॒ स्वाहा॒ स्वाहा॑ ता॒दृशा॑य ता॒दृशा॑य॒ स्वाहा᳚ । \newline
58. स्वाहा॑ स॒दृशा॑य स॒दृशा॑य॒ स्वाहा॒ स्वाहा॑ स॒दृशा॑य । \newline
59. स॒दृशा॑य॒ स्वाहा॒ स्वाहा॑ स॒दृशा॑य स॒दृशा॑य॒ स्वाहा᳚ । \newline
60. स्वाहा॒ विस॑दृशाय॒ विस॑दृशाय॒ स्वाहा॒ स्वाहा॒ विस॑दृशाय । \newline
61. विस॑दृशाय॒ स्वाहा॒ स्वाहा॒ विस॑दृशाय॒ विस॑दृशाय॒ स्वाहा᳚ । \newline
62. विस॑दृशा॒येति॒ वि - स॒दृ॒शा॒य॒ । \newline
63. स्वाहा॒ सुस॑दृशाय॒ सुस॑दृशाय॒ स्वाहा॒ स्वाहा॒ सुस॑दृशाय । \newline
64. सुस॑दृशाय॒ स्वाहा॒ स्वाहा॒ सुस॑दृशाय॒ सुस॑दृशाय॒ स्वाहा᳚ । \newline
65. सुस॑दृशा॒येति॒ सु - स॒दृ॒शा॒य॒ । \newline
66. स्वाहा॑ रू॒पाय॑ रू॒पाय॒ स्वाहा॒ स्वाहा॑ रू॒पाय॑ । \newline
67. रू॒पाय॒ स्वाहा॒ स्वाहा॑ रू॒पाय॑ रू॒पाय॒ स्वाहा᳚ । \newline
68. स्वाहा॒ सर्व॑स्मै॒ सर्व॑स्मै॒ स्वाहा॒ स्वाहा॒ सर्व॑स्मै । \newline
69. सर्व॑स्मै॒ स्वाहा॒ स्वाहा॒ सर्व॑स्मै॒ सर्व॑स्मै॒ स्वाहा᳚ । \newline
70. स्वाहेति॒ स्वाहा᳚ । \newline

\textbf{Ghana Paata } \newline

1. अ॒ञ्ज्ये॒ताय॒ स्वाहा॒ स्वाहा᳚ ऽञ्ज्ये॒ताया᳚ ञ्ज्ये॒ताय॒ स्वाहा᳚ ऽञ्जिस॒क्थाया᳚ ञ्जिस॒क्थाय॒ स्वाहा᳚ ऽञ्ज्ये॒ताया᳚ञ्ज्ये॒ताय॒ स्वाहा᳚ ऽञ्जिस॒क्थाय॑ । \newline
2. अ॒ञ्ज्ये॒तायेत्य॑ञ्जि - ए॒ताय॑ । \newline
3. स्वाहा᳚ ऽञ्जिस॒क्थाया᳚ ञ्जिस॒क्थाय॒ स्वाहा॒ स्वाहा᳚ ऽञ्जिस॒क्थाय॒ स्वाहा॒ स्वाहा᳚ ऽञ्जिस॒क्थाय॒ स्वाहा॒ स्वाहा᳚ ऽञ्जिस॒क्थाय॒ स्वाहा᳚ । \newline
4. अ॒ञ्जि॒स॒क्थाय॒ स्वाहा॒ स्वाहा᳚ ऽञ्जिस॒क्थाया᳚ ञ्जिस॒क्थाय॒ स्वाहा॑ शिति॒पदे॑ शिति॒पदे॒ स्वाहा᳚ ऽञ्जिस॒क्थाया᳚ ञ्जिस॒क्थाय॒ स्वाहा॑ शिति॒पदे᳚ । \newline
5. अ॒ञ्जि॒स॒क्थायेत्य॑ञ्जि - स॒क्थाय॑ । \newline
6. स्वाहा॑ शिति॒पदे॑ शिति॒पदे॒ स्वाहा॒ स्वाहा॑ शिति॒पदे॒ स्वाहा॒ स्वाहा॑ शिति॒पदे॒ स्वाहा॒ स्वाहा॑ शिति॒पदे॒ स्वाहा᳚ । \newline
7. शि॒ति॒पदे॒ स्वाहा॒ स्वाहा॑ शिति॒पदे॑ शिति॒पदे॒ स्वाहा॒ शिति॑ककुदे॒ शिति॑ककुदे॒ स्वाहा॑ शिति॒पदे॑ शिति॒पदे॒ स्वाहा॒ शिति॑ककुदे । \newline
8. शि॒ति॒पद॒ इति॑ शिति - पदे᳚ । \newline
9. स्वाहा॒ शिति॑ककुदे॒ शिति॑ककुदे॒ स्वाहा॒ स्वाहा॒ शिति॑ककुदे॒ स्वाहा॒ स्वाहा॒ शिति॑ककुदे॒ स्वाहा॒ स्वाहा॒ शिति॑ककुदे॒ स्वाहा᳚ । \newline
10. शिति॑ककुदे॒ स्वाहा॒ स्वाहा॒ शिति॑ककुदे॒ शिति॑ककुदे॒ स्वाहा॑ शिति॒रन्ध्रा॑य शिति॒रन्ध्रा॑य॒ स्वाहा॒ शिति॑ककुदे॒ शिति॑ककुदे॒ स्वाहा॑ शिति॒रन्ध्रा॑य । \newline
11. शिति॑ककुद॒ इति॒ शिति॑ - क॒कु॒दे॒ । \newline
12. स्वाहा॑ शिति॒रन्ध्रा॑य शिति॒रन्ध्रा॑य॒ स्वाहा॒ स्वाहा॑ शिति॒रन्ध्रा॑य॒ स्वाहा॒ स्वाहा॑ शिति॒रन्ध्रा॑य॒ स्वाहा॒ स्वाहा॑ शिति॒रन्ध्रा॑य॒ स्वाहा᳚ । \newline
13. शि॒ति॒रन्ध्रा॑य॒ स्वाहा॒ स्वाहा॑ शिति॒रन्ध्रा॑य शिति॒रन्ध्रा॑य॒ स्वाहा॑ शितिपृ॒ष्ठाय॑ शितिपृ॒ष्ठाय॒ स्वाहा॑ शिति॒रन्ध्रा॑य शिति॒रन्ध्रा॑य॒ स्वाहा॑ शितिपृ॒ष्ठाय॑ । \newline
14. शि॒ति॒रन्ध्रा॒येति॑ शिति - रन्ध्रा॑य । \newline
15. स्वाहा॑ शितिपृ॒ष्ठाय॑ शितिपृ॒ष्ठाय॒ स्वाहा॒ स्वाहा॑ शितिपृ॒ष्ठाय॒ स्वाहा॒ स्वाहा॑ शितिपृ॒ष्ठाय॒ स्वाहा॒ स्वाहा॑ शितिपृ॒ष्ठाय॒ स्वाहा᳚ । \newline
16. शि॒ति॒पृ॒ष्ठाय॒ स्वाहा॒ स्वाहा॑ शितिपृ॒ष्ठाय॑ शितिपृ॒ष्ठाय॒ स्वाहा॑ शि॒त्यꣳसा॑य शि॒त्यꣳसा॑य॒ स्वाहा॑ शितिपृ॒ष्ठाय॑ शितिपृ॒ष्ठाय॒ स्वाहा॑ शि॒त्यꣳसा॑य । \newline
17. शि॒ति॒पृ॒ष्ठायेति॑ शिति - पृ॒ष्ठाय॑ । \newline
18. स्वाहा॑ शि॒त्यꣳसा॑य शि॒त्यꣳसा॑य॒ स्वाहा॒ स्वाहा॑ शि॒त्यꣳसा॑य॒ स्वाहा॒ स्वाहा॑ शि॒त्यꣳसा॑य॒ स्वाहा॒ स्वाहा॑ शि॒त्यꣳसा॑य॒ स्वाहा᳚ । \newline
19. शि॒त्यꣳसा॑य॒ स्वाहा॒ स्वाहा॑ शि॒त्यꣳसा॑य शि॒त्यꣳसा॑य॒ स्वाहा॑ पुष्प॒कर्णा॑य पुष्प॒कर्णा॑य॒ स्वाहा॑ शि॒त्यꣳसा॑य शि॒त्यꣳसा॑य॒ स्वाहा॑ पुष्प॒कर्णा॑य । \newline
20. शि॒त्यꣳसा॒येति॑ शिति - अꣳसा॑य । \newline
21. स्वाहा॑ पुष्प॒कर्णा॑य पुष्प॒कर्णा॑य॒ स्वाहा॒ स्वाहा॑ पुष्प॒कर्णा॑य॒ स्वाहा॒ स्वाहा॑ पुष्प॒कर्णा॑य॒ स्वाहा॒ स्वाहा॑ पुष्प॒कर्णा॑य॒ स्वाहा᳚ । \newline
22. पु॒ष्प॒कर्णा॑य॒ स्वाहा॒ स्वाहा॑ पुष्प॒कर्णा॑य पुष्प॒कर्णा॑य॒ स्वाहा॑ शि॒त्योष्ठा॑य शि॒त्योष्ठा॑य॒ स्वाहा॑ पुष्प॒कर्णा॑य पुष्प॒कर्णा॑य॒ स्वाहा॑ शि॒त्योष्ठा॑य । \newline
23. पु॒ष्प॒कर्णा॒येति॑ पुष्प - कर्णा॑य । \newline
24. स्वाहा॑ शि॒त्योष्ठा॑य शि॒त्योष्ठा॑य॒ स्वाहा॒ स्वाहा॑ शि॒त्योष्ठा॑य॒ स्वाहा॒ स्वाहा॑ शि॒त्योष्ठा॑य॒ स्वाहा॒ स्वाहा॑ शि॒त्योष्ठा॑य॒ स्वाहा᳚ । \newline
25. शि॒त्योष्ठा॑य॒ स्वाहा॒ स्वाहा॑ शि॒त्योष्ठा॑य शि॒त्योष्ठा॑य॒ स्वाहा॑ शिति॒भ्रवे॑ शिति॒भ्रवे॒ स्वाहा॑ शि॒त्योष्ठा॑य शि॒त्योष्ठा॑य॒ स्वाहा॑ शिति॒भ्रवे᳚ । \newline
26. शि॒त्योष्ठा॒येति॑ शिति - ओष्ठा॑य । \newline
27. स्वाहा॑ शिति॒भ्रवे॑ शिति॒भ्रवे॒ स्वाहा॒ स्वाहा॑ शिति॒भ्रवे॒ स्वाहा॒ स्वाहा॑ शिति॒भ्रवे॒ स्वाहा॒ स्वाहा॑ शिति॒भ्रवे॒ स्वाहा᳚ । \newline
28. शि॒ति॒भ्रवे॒ स्वाहा॒ स्वाहा॑ शिति॒भ्रवे॑ शिति॒भ्रवे॒ स्वाहा॒ शिति॑भसदे॒ शिति॑भसदे॒ स्वाहा॑ शिति॒भ्रवे॑ शिति॒भ्रवे॒ स्वाहा॒ शिति॑भसदे । \newline
29. शि॒ति॒भ्रव॒ इति॑ शिति - भ्रवे᳚ । \newline
30. स्वाहा॒ शिति॑भसदे॒ शिति॑भसदे॒ स्वाहा॒ स्वाहा॒ शिति॑भसदे॒ स्वाहा॒ स्वाहा॒ शिति॑भसदे॒ स्वाहा॒ स्वाहा॒ शिति॑भसदे॒ स्वाहा᳚ । \newline
31. शिति॑भसदे॒ स्वाहा॒ स्वाहा॒ शिति॑भसदे॒ शिति॑भसदे॒ स्वाहा᳚ श्वे॒तानू॑काशाय श्वे॒तानू॑काशाय॒ स्वाहा॒ शिति॑भसदे॒ शिति॑भसदे॒ स्वाहा᳚ श्वे॒तानू॑काशाय । \newline
32. शिति॑भसद॒ इति॒ शिति॑ - भ॒स॒दे॒ । \newline
33. स्वाहा᳚ श्वे॒तानू॑काशाय श्वे॒तानू॑काशाय॒ स्वाहा॒ स्वाहा᳚ श्वे॒तानू॑काशाय॒ स्वाहा॒ स्वाहा᳚ श्वे॒तानू॑काशाय॒ स्वाहा॒ स्वाहा᳚ श्वे॒तानू॑काशाय॒ स्वाहा᳚ । \newline
34. श्वे॒तानू॑काशाय॒ स्वाहा॒ स्वाहा᳚ श्वे॒तानू॑काशाय श्वे॒तानू॑काशाय॒ स्वाहा॒ ऽञ्जये॒ ऽञ्जये॒ स्वाहा᳚ श्वे॒तानू॑काशाय श्वे॒तानू॑काशाय॒ स्वाहा॒ ऽञ्जये᳚ । \newline
35. श्वे॒तानू॑काशा॒येति॑ श्वे॒त - अ॒नू॒का॒शा॒य॒ । \newline
36. स्वाहा॒ ऽञ्जये॒ ऽञ्जये॒ स्वाहा॒ स्वाहा॒ ऽञ्जये॒ स्वाहा॒ स्वाहा॒ ऽञ्जये॒ स्वाहा॒ स्वाहा॒ ऽञ्जये॒ स्वाहा᳚ । \newline
37. अ॒ञ्जये॒ स्वाहा॒ स्वाहा॒ ऽञ्जये॒ ऽञ्जये॒ स्वाहा॑ ल॒लामा॑य ल॒लामा॑य॒ स्वाहा॒ ऽञ्जये॒ ऽञ्जये॒ स्वाहा॑ ल॒लामा॑य । \newline
38. स्वाहा॑ ल॒लामा॑य ल॒लामा॑य॒ स्वाहा॒ स्वाहा॑ ल॒लामा॑य॒ स्वाहा॒ स्वाहा॑ ल॒लामा॑य॒ स्वाहा॒ स्वाहा॑ ल॒लामा॑य॒ स्वाहा᳚ । \newline
39. ल॒लामा॑य॒ स्वाहा॒ स्वाहा॑ ल॒लामा॑य ल॒लामा॑य॒ स्वाहा ऽसि॑तज्ञ्॒वे ऽसि॑तज्ञ्वे॒ स्वाहा॑ ल॒लामा॑य ल॒लामा॑य॒ स्वाहा ऽसि॑तज्ञ्वे । \newline
40. स्वाहा ऽसि॑तज्ञ्॒वे ऽसि॑तज्ञ्वे॒ स्वाहा॒ स्वाहा ऽसि॑तज्ञ्वे॒ स्वाहा॒ स्वाहा ऽसि॑तज्ञ्वे॒ स्वाहा॒ स्वाहा ऽसि॑तज्ञ्वे॒ स्वाहा᳚ । \newline
41. असि॑तज्ञ्वे॒ स्वाहा॒ स्वाहा ऽसि॑तज्ञ्॒वे ऽसि॑तज्ञ्वे॒ स्वाहा॑ कृष्णै॒ताय॑ कृष्णै॒ताय॒ स्वाहा ऽसि॑तज्ञ्॒वे ऽसि॑तज्ञ्वे॒ स्वाहा॑ कृष्णै॒ताय॑ । \newline
42. असि॑तज्ञ्व॒ इत्यसि॑त - ज्ञ्॒वे॒ । \newline
43. स्वाहा॑ कृष्णै॒ताय॑ कृष्णै॒ताय॒ स्वाहा॒ स्वाहा॑ कृष्णै॒ताय॒ स्वाहा॒ स्वाहा॑ कृष्णै॒ताय॒ स्वाहा॒ स्वाहा॑ कृष्णै॒ताय॒ स्वाहा᳚ । \newline
44. कृ॒ष्णै॒ताय॒ स्वाहा॒ स्वाहा॑ कृष्णै॒ताय॑ कृष्णै॒ताय॒ स्वाहा॑ रोहितै॒ताय॑ रोहितै॒ताय॒ स्वाहा॑ कृष्णै॒ताय॑ कृष्णै॒ताय॒ स्वाहा॑ रोहितै॒ताय॑ । \newline
45. कृ॒ष्णै॒तायेति॑ कृष्ण - ए॒ताय॑ । \newline
46. स्वाहा॑ रोहितै॒ताय॑ रोहितै॒ताय॒ स्वाहा॒ स्वाहा॑ रोहितै॒ताय॒ स्वाहा॒ स्वाहा॑ रोहितै॒ताय॒ स्वाहा॒ स्वाहा॑ रोहितै॒ताय॒ स्वाहा᳚ । \newline
47. रो॒हि॒तै॒ताय॒ स्वाहा॒ स्वाहा॑ रोहितै॒ताय॑ रोहितै॒ताय॒ स्वाहा॑ ऽरुणै॒ताया॑ रुणै॒ताय॒ स्वाहा॑ रोहितै॒ताय॑ रोहितै॒ताय॒ स्वाहा॑ ऽरुणै॒ताय॑ । \newline
48. रो॒हि॒तै॒तायेति॑ रोहित - ए॒ताय॑ । \newline
49. स्वाहा॑ ऽरुणै॒ताया॑ रुणै॒ताय॒ स्वाहा॒ स्वाहा॑ ऽरुणै॒ताय॒ स्वाहा॒ स्वाहा॑ ऽरुणै॒ताय॒ स्वाहा॒ स्वाहा॑ ऽरुणै॒ताय॒ स्वाहा᳚ । \newline
50. अ॒रु॒णै॒ताय॒ स्वाहा॒ स्वाहा॑ ऽरुणै॒ताया॑ रुणै॒ताय॒ स्वाहे॒दृशा॑ये॒ दृशा॑य॒ स्वाहा॑ ऽरुणै॒ताया॑ रुणै॒ताय॒ स्वाहे॒दृशा॑य । \newline
51. अ॒रु॒णै॒तायेत्य॑रुण - ए॒ताय॑ । \newline
52. स्वाहे॒दृशा॑ ये॒दृशा॑य॒ स्वाहा॒ स्वाहे॒दृशा॑य॒ स्वाहा॒ स्वाहे॒ दृशा॑य॒ स्वाहा॒ स्वाहे॒दृशा॑य॒ स्वाहा᳚ । \newline
53. ई॒दृशा॑य॒ स्वाहा॒ स्वाहे॒ दृशा॑ये॒ दृशा॑य॒ स्वाहा॑ की॒दृशा॑य की॒दृशा॑य॒ स्वाहे॒ दृशा॑ये॒ दृशा॑य॒ स्वाहा॑ की॒दृशा॑य । \newline
54. स्वाहा॑ की॒दृशा॑य की॒दृशा॑य॒ स्वाहा॒ स्वाहा॑ की॒दृशा॑य॒ स्वाहा॒ स्वाहा॑ की॒दृशा॑य॒ स्वाहा॒ स्वाहा॑ की॒दृशा॑य॒ स्वाहा᳚ । \newline
55. की॒दृशा॑य॒ स्वाहा॒ स्वाहा॑ की॒दृशा॑य की॒दृशा॑य॒ स्वाहा॑ ता॒दृशा॑य ता॒दृशा॑य॒ स्वाहा॑ की॒दृशा॑य की॒दृशा॑य॒ स्वाहा॑ ता॒दृशा॑य । \newline
56. स्वाहा॑ ता॒दृशा॑य ता॒दृशा॑य॒ स्वाहा॒ स्वाहा॑ ता॒दृशा॑य॒ स्वाहा॒ स्वाहा॑ ता॒दृशा॑य॒ स्वाहा॒ स्वाहा॑ ता॒दृशा॑य॒ स्वाहा᳚ । \newline
57. ता॒दृशा॑य॒ स्वाहा॒ स्वाहा॑ ता॒दृशा॑य ता॒दृशा॑य॒ स्वाहा॑ स॒दृशा॑य स॒दृशा॑य॒ स्वाहा॑ ता॒दृशा॑य ता॒दृशा॑य॒ स्वाहा॑ स॒दृशा॑य । \newline
58. स्वाहा॑ स॒दृशा॑य स॒दृशा॑य॒ स्वाहा॒ स्वाहा॑ स॒दृशा॑य॒ स्वाहा॒ स्वाहा॑ स॒दृशा॑य॒ स्वाहा॒ स्वाहा॑ स॒दृशा॑य॒ स्वाहा᳚ । \newline
59. स॒दृशा॑य॒ स्वाहा॒ स्वाहा॑ स॒दृशा॑य स॒दृशा॑य॒ स्वाहा॒ विस॑दृशाय॒ विस॑दृशाय॒ स्वाहा॑ स॒दृशा॑य स॒दृशा॑य॒ स्वाहा॒ विस॑दृशाय । \newline
60. स्वाहा॒ विस॑दृशाय॒ विस॑दृशाय॒ स्वाहा॒ स्वाहा॒ विस॑दृशाय॒ स्वाहा॒ स्वाहा॒ विस॑दृशाय॒ स्वाहा॒ स्वाहा॒ विस॑दृशाय॒ स्वाहा᳚ । \newline
61. विस॑दृशाय॒ स्वाहा॒ स्वाहा॒ विस॑दृशाय॒ विस॑दृशाय॒ स्वाहा॒ सुस॑दृशाय॒ सुस॑दृशाय॒ स्वाहा॒ विस॑दृशाय॒ विस॑दृशाय॒ स्वाहा॒ सुस॑दृशाय । \newline
62. विस॑दृशा॒येति॒ वि - स॒दृ॒शा॒य॒ । \newline
63. स्वाहा॒ सुस॑दृशाय॒ सुस॑दृशाय॒ स्वाहा॒ स्वाहा॒ सुस॑दृशाय॒ स्वाहा॒ स्वाहा॒ सुस॑दृशाय॒ स्वाहा॒ स्वाहा॒ सुस॑दृशाय॒ स्वाहा᳚ । \newline
64. सुस॑दृशाय॒ स्वाहा॒ स्वाहा॒ सुस॑दृशाय॒ सुस॑दृशाय॒ स्वाहा॑ रू॒पाय॑ रू॒पाय॒ स्वाहा॒ सुस॑दृशाय॒ सुस॑दृशाय॒ स्वाहा॑ रू॒पाय॑ । \newline
65. सुस॑दृशा॒येति॒ सु - स॒दृ॒शा॒य॒ । \newline
66. स्वाहा॑ रू॒पाय॑ रू॒पाय॒ स्वाहा॒ स्वाहा॑ रू॒पाय॒ स्वाहा॒ स्वाहा॑ रू॒पाय॒ स्वाहा॒ स्वाहा॑ रू॒पाय॒ स्वाहा᳚ । \newline
67. रू॒पाय॒ स्वाहा॒ स्वाहा॑ रू॒पाय॑ रू॒पाय॒ स्वाहा॒ सर्व॑स्मै॒ सर्व॑स्मै॒ स्वाहा॑ रू॒पाय॑ रू॒पाय॒ स्वाहा॒ सर्व॑स्मै । \newline
68. स्वाहा॒ सर्व॑स्मै॒ सर्व॑स्मै॒ स्वाहा॒ स्वाहा॒ सर्व॑स्मै॒ स्वाहा॒ स्वाहा॒ सर्व॑स्मै॒ स्वाहा॒ स्वाहा॒ सर्व॑स्मै॒ स्वाहा᳚ । \newline
69. सर्व॑स्मै॒ स्वाहा॒ स्वाहा॒ सर्व॑स्मै॒ सर्व॑स्मै॒ स्वाहा᳚ । \newline
70. स्वाहेति॒ स्वाहा᳚ । \newline
\pagebreak
\markright{ TS 7.3.18.1  \hfill https://www.vedavms.in \hfill}

\section{ TS 7.3.18.1 }

\textbf{TS 7.3.18.1 } \newline
\textbf{Samhita Paata} \newline

कृ॒ष्णाय॒ स्वाहा᳚ श्वे॒ताय॒ स्वाहा॑ पि॒शङ्गा॑य॒ स्वाहा॑ सा॒रङ्गा॑य॒ स्वाहा॑ ऽरु॒णाय॒ स्वाहा॑ गौ॒राय॒ स्वाहा॑ ब॒भ्रवे॒ स्वाहा॑ नकु॒लाय॒ स्वाहा॒ रोहि॑ताय॒ स्वाहा॒ शोणा॑य॒ स्वाहा᳚ श्या॒वाय॒ स्वाहा᳚ श्या॒माय॒ स्वाहा॑ पाक॒लाय॒ स्वाहा॑ सुरू॒पाय॒ स्वाहा ऽनु॑रूपाय॒ स्वाहा॒ विरू॑पाय॒ स्वाहा॒ सरू॑पाय॒ स्वाहा॒ प्रति॑रूपाय॒ स्वाहा॑ श॒बला॑य॒ स्वाहा॑ कम॒लाय॒ स्वाहा॒ पृश्न॑ये॒ स्वाहा॑ पृश्निस॒क्थाय॒ स्वाहा॒ सर्व॑स्मै॒ स्वाहा᳚ ॥ \newline

\textbf{Pada Paata} \newline

कृ॒ष्णाय॑ । स्वाहा᳚ । श्वे॒ताय॑ । स्वाहा᳚ । पि॒शङ्गा॑य । स्वाहा᳚ । सा॒रङ्गा॑य । स्वाहा᳚ । अ॒रु॒णाय॑ । स्वाहा᳚ । गौ॒राय॑ । स्वाहा᳚ । ब॒भ्रवे᳚ । स्वाहा᳚ । न॒कु॒लाय॑ । स्वाहा᳚ । रोहि॑ताय । स्वाहा᳚ । शोणा॑य । स्वाहा᳚ । श्या॒वाय॑ । स्वाहा᳚ । श्या॒माय॑ । स्वाहा᳚ । पा॒क॒लाय॑ । स्वाहा᳚ । सु॒रू॒पायेति॑ सु - रू॒पाय॑ । स्वाहा᳚ । अनु॑रूपा॒येत्यनु॑ - रू॒पा॒य॒ । स्वाहा᳚ । विरू॑पा॒येति॒ वि-रू॒पा॒य॒ । स्वाहा᳚ । सरू॑पा॒येति॒ स-रू॒पा॒य॒ । स्वाहा᳚ । प्रति॑रूपा॒येति॒ प्रति॑ - रू॒पा॒य॒ । स्वाहा᳚ । श॒बला॑य । स्वाहा᳚ । क॒म॒लाय॑ । स्वाहा᳚ । पृश्न॑ये । स्वाहा᳚ । पृ॒श्नि॒स॒क्थायेति॑ पृश्नि - स॒क्थाय॑ । स्वाहा᳚ । सर्व॑स्मै । स्वाहा᳚ ॥  \newline


\textbf{Krama Paata} \newline

कृ॒ष्णाय॒ स्वाहा᳚ । स्वाहा᳚ श्वे॒ताय॑ । श्वे॒ताय॒ स्वाहा᳚ । स्वाहा॑ पि॒शङ्‍गा॑य । पि॒शङ्‍गा॑य॒ स्वाहा᳚ । स्वाहा॑ सा॒रङ्‍गा॑य । सा॒रङ्‍गा॑य॒ स्वाहा᳚ । स्वाहा॑ऽरु॒णाय॑ । अ॒रु॒णाय॒ स्वाहा᳚ । स्वाहा॑ गौ॒राय॑ । गौ॒राय॒ स्वाहा᳚ । स्वाहा॑ ब॒भ्रवे᳚ । ब॒भ्रवे॒ स्वाहा᳚ । स्वाहा॑ नकु॒लाय॑ । न॒कु॒लाय॒ स्वाहा᳚ । स्वाहा॒ रोहि॑ताय । रोहि॑ताय॒ स्वाहा᳚ । स्वाहा॒ शोणा॑य । शोणा॑य॒ स्वाहा᳚ । स्वाहा᳚ श्या॒वाय॑ । श्या॒वाय॒ स्वाहा᳚ । स्वाहा᳚ श्या॒माय॑ । श्या॒माय॒ स्वाहा᳚ । स्वाहा॑ पाक॒लाय॑ । पा॒क॒लाय॒ स्वाहा᳚ । स्वाहा॑ सुरू॒पाय॑ । सु॒रू॒पाय॒ स्वाहा᳚ । सु॒रू॒पायेति॑ सु - रू॒पाय॑ । स्वाहाऽनु॑रूपाय । अनु॑रूपाय॒ स्वाहा᳚ । अनु॑रूपा॒येत्यनु॑ - रू॒पा॒य॒ । स्वाहा॒ विरू॑पाय । विरू॑पाय॒ स्वाहा᳚ । विरू॑पा॒येति॒ वि - रू॒पा॒य॒ । स्वाहा॒ सरू॑पाय । सरू॑पाय॒ स्वाहा᳚ । सरू॑पा॒येति॒ स - रू॒पा॒य॒ । स्वाहा॒ प्रति॑रूपाय । प्रति॑रूपाय॒ स्वाहा᳚ । प्रति॑रूपा॒येति॒ प्रति॑ - रू॒पा॒य॒ । स्वाहा॑ श॒बला॑य । श॒बला॑य॒ स्वाहा᳚ । स्वाहा॑ कम॒लाय॑ । क॒म॒लाय॒ स्वाहा᳚ । स्वाहा॒ पृश्ञ॑ये । पृश्ञ॑ये॒ स्वाहा᳚ । स्वाहा॑ पृश्ञिस॒क्थाय॑ । पृ॒श्ञि॒स॒क्थाय॒ स्वाहा᳚ । पृ॒श्ञि॒स॒क्थायेति॑ पृश्ञि - स॒क्थाय॑ । स्वाहा॒ सर्व॑स्मै । सर्व॑स्मै॒ स्वाहा᳚ । स्वाहेति॒ स्वाहा᳚ । \newline

\textbf{Jatai Paata} \newline

1. कृ॒ष्णाय॒ स्वाहा॒ स्वाहा॑ कृ॒ष्णाय॑ कृ॒ष्णाय॒ स्वाहा᳚ । \newline
2. स्वाहा᳚ श्वे॒ताय॑ श्वे॒ताय॒ स्वाहा॒ स्वाहा᳚ श्वे॒ताय॑ । \newline
3. श्वे॒ताय॒ स्वाहा॒ स्वाहा᳚ श्वे॒ताय॑ श्वे॒ताय॒ स्वाहा᳚ । \newline
4. स्वाहा॑ पि॒शङ्गा॑य पि॒शङ्गा॑य॒ स्वाहा॒ स्वाहा॑ पि॒शङ्गा॑य । \newline
5. पि॒शङ्गा॑य॒ स्वाहा॒ स्वाहा॑ पि॒शङ्गा॑य पि॒शङ्गा॑य॒ स्वाहा᳚ । \newline
6. स्वाहा॑ सा॒रङ्गा॑य सा॒रङ्गा॑य॒ स्वाहा॒ स्वाहा॑ सा॒रङ्गा॑य । \newline
7. सा॒रङ्गा॑य॒ स्वाहा॒ स्वाहा॑ सा॒रङ्गा॑य सा॒रङ्गा॑य॒ स्वाहा᳚ । \newline
8. स्वाहा॑ ऽरु॒णाया॑ रु॒णाय॒ स्वाहा॒ स्वाहा॑ ऽरु॒णाय॑ । \newline
9. अ॒रु॒णाय॒ स्वाहा॒ स्वाहा॑ ऽरु॒णाया॑ रु॒णाय॒ स्वाहा᳚ । \newline
10. स्वाहा॑ गौ॒राय॑ गौ॒राय॒ स्वाहा॒ स्वाहा॑ गौ॒राय॑ । \newline
11. गौ॒राय॒ स्वाहा॒ स्वाहा॑ गौ॒राय॑ गौ॒राय॒ स्वाहा᳚ । \newline
12. स्वाहा॑ ब॒भ्रवे॑ ब॒भ्रवे॒ स्वाहा॒ स्वाहा॑ ब॒भ्रवे᳚ । \newline
13. ब॒भ्रवे॒ स्वाहा॒ स्वाहा॑ ब॒भ्रवे॑ ब॒भ्रवे॒ स्वाहा᳚ । \newline
14. स्वाहा॑ नकु॒लाय॑ नकु॒लाय॒ स्वाहा॒ स्वाहा॑ नकु॒लाय॑ । \newline
15. न॒कु॒लाय॒ स्वाहा॒ स्वाहा॑ नकु॒लाय॑ नकु॒लाय॒ स्वाहा᳚ । \newline
16. स्वाहा॒ रोहि॑ताय॒ रोहि॑ताय॒ स्वाहा॒ स्वाहा॒ रोहि॑ताय । \newline
17. रोहि॑ताय॒ स्वाहा॒ स्वाहा॒ रोहि॑ताय॒ रोहि॑ताय॒ स्वाहा᳚ । \newline
18. स्वाहा॒ शोणा॑य॒ शोणा॑य॒ स्वाहा॒ स्वाहा॒ शोणा॑य । \newline
19. शोणा॑य॒ स्वाहा॒ स्वाहा॒ शोणा॑य॒ शोणा॑य॒ स्वाहा᳚ । \newline
20. स्वाहा᳚ श्या॒वाय॑ श्या॒वाय॒ स्वाहा॒ स्वाहा᳚ श्या॒वाय॑ । \newline
21. श्या॒वाय॒ स्वाहा॒ स्वाहा᳚ श्या॒वाय॑ श्या॒वाय॒ स्वाहा᳚ । \newline
22. स्वाहा᳚ श्या॒माय॑ श्या॒माय॒ स्वाहा॒ स्वाहा᳚ श्या॒माय॑ । \newline
23. श्या॒माय॒ स्वाहा॒ स्वाहा᳚ श्या॒माय॑ श्या॒माय॒ स्वाहा᳚ । \newline
24. स्वाहा॑ पाक॒लाय॑ पाक॒लाय॒ स्वाहा॒ स्वाहा॑ पाक॒लाय॑ । \newline
25. पा॒क॒लाय॒ स्वाहा॒ स्वाहा॑ पाक॒लाय॑ पाक॒लाय॒ स्वाहा᳚ । \newline
26. स्वाहा॑ सुरू॒पाय॑ सुरू॒पाय॒ स्वाहा॒ स्वाहा॑ सुरू॒पाय॑ । \newline
27. सु॒रू॒पाय॒ स्वाहा॒ स्वाहा॑ सुरू॒पाय॑ सुरू॒पाय॒ स्वाहा᳚ । \newline
28. सु॒रू॒पायेति॑ सु - रू॒पाय॑ । \newline
29. स्वाहा ऽनु॑रूपा॒या नु॑रूपाय॒ स्वाहा॒ स्वाहा ऽनु॑रूपाय । \newline
30. अनु॑रूपाय॒ स्वाहा॒ स्वाहा ऽनु॑रूपा॒या नु॑रूपाय॒ स्वाहा᳚ । \newline
31. अनु॑रूपा॒येत्यनु॑ - रू॒पा॒य॒ । \newline
32. स्वाहा॒ विरू॑पाय॒ विरू॑पाय॒ स्वाहा॒ स्वाहा॒ विरू॑पाय । \newline
33. विरू॑पाय॒ स्वाहा॒ स्वाहा॒ विरू॑पाय॒ विरू॑पाय॒ स्वाहा᳚ । \newline
34. विरू॑पा॒येति॒ वि - रू॒पा॒य॒ । \newline
35. स्वाहा॒ सरू॑पाय॒ सरू॑पाय॒ स्वाहा॒ स्वाहा॒ सरू॑पाय । \newline
36. सरू॑पाय॒ स्वाहा॒ स्वाहा॒ सरू॑पाय॒ सरू॑पाय॒ स्वाहा᳚ । \newline
37. सरू॑पा॒येति॒ स - रू॒पा॒य॒ । \newline
38. स्वाहा॒ प्रति॑रूपाय॒ प्रति॑रूपाय॒ स्वाहा॒ स्वाहा॒ प्रति॑रूपाय । \newline
39. प्रति॑रूपाय॒ स्वाहा॒ स्वाहा॒ प्रति॑रूपाय॒ प्रति॑रूपाय॒ स्वाहा᳚ । \newline
40. प्रति॑रूपा॒येति॒ प्रति॑ - रू॒पा॒य॒ । \newline
41. स्वाहा॑ श॒बला॑य श॒बला॑य॒ स्वाहा॒ स्वाहा॑ श॒बला॑य । \newline
42. श॒बला॑य॒ स्वाहा॒ स्वाहा॑ श॒बला॑य श॒बला॑य॒ स्वाहा᳚ । \newline
43. स्वाहा॑ कम॒लाय॑ कम॒लाय॒ स्वाहा॒ स्वाहा॑ कम॒लाय॑ । \newline
44. क॒म॒लाय॒ स्वाहा॒ स्वाहा॑ कम॒लाय॑ कम॒लाय॒ स्वाहा᳚ । \newline
45. स्वाहा॒ पृश्ञ॑ये॒ पृश्ञ॑ये॒ स्वाहा॒ स्वाहा॒ पृश्ञ॑ये । \newline
46. पृश्ञ॑ये॒ स्वाहा॒ स्वाहा॒ पृश्ञ॑ये॒ पृश्ञ॑ये॒ स्वाहा᳚ । \newline
47. स्वाहा॑ पृश्ञिस॒क्थाय॑ पृश्ञिस॒क्थाय॒ स्वाहा॒ स्वाहा॑ पृश्ञिस॒क्थाय॑ । \newline
48. पृ॒श्ञि॒स॒क्थाय॒ स्वाहा॒ स्वाहा॑ पृश्ञिस॒क्थाय॑ पृश्ञिस॒क्थाय॒ स्वाहा᳚ । \newline
49. पृ॒श्ञि॒स॒क्थायेति॑ पृश्ञि - स॒क्थाय॑ । \newline
50. स्वाहा॒ सर्व॑स्मै॒ सर्व॑स्मै॒ स्वाहा॒ स्वाहा॒ सर्व॑स्मै । \newline
51. सर्व॑स्मै॒ स्वाहा॒ स्वाहा॒ सर्व॑स्मै॒ सर्व॑स्मै॒ स्वाहा᳚ । \newline
52. स्वाहेति॒ स्वाहा᳚ । \newline

\textbf{Ghana Paata } \newline

1. कृ॒ष्णाय॒ स्वाहा॒ स्वाहा॑ कृ॒ष्णाय॑ कृ॒ष्णाय॒ स्वाहा᳚ श्वे॒ताय॑ श्वे॒ताय॒ स्वाहा॑ कृ॒ष्णाय॑ कृ॒ष्णाय॒ स्वाहा᳚ श्वे॒ताय॑ । \newline
2. स्वाहा᳚ श्वे॒ताय॑ श्वे॒ताय॒ स्वाहा॒ स्वाहा᳚ श्वे॒ताय॒ स्वाहा॒ स्वाहा᳚ श्वे॒ताय॒ स्वाहा॒ स्वाहा᳚ श्वे॒ताय॒ स्वाहा᳚ । \newline
3. श्वे॒ताय॒ स्वाहा॒ स्वाहा᳚ श्वे॒ताय॑ श्वे॒ताय॒ स्वाहा॑ पि॒शङ्गा॑य पि॒शङ्गा॑य॒ स्वाहा᳚ श्वे॒ताय॑ श्वे॒ताय॒ स्वाहा॑ पि॒शङ्गा॑य । \newline
4. स्वाहा॑ पि॒शङ्गा॑य पि॒शङ्गा॑य॒ स्वाहा॒ स्वाहा॑ पि॒शङ्गा॑य॒ स्वाहा॒ स्वाहा॑ पि॒शङ्गा॑य॒ स्वाहा॒ स्वाहा॑ पि॒शङ्गा॑य॒ स्वाहा᳚ । \newline
5. पि॒शङ्गा॑य॒ स्वाहा॒ स्वाहा॑ पि॒शङ्गा॑य पि॒शङ्गा॑य॒ स्वाहा॑ सा॒रङ्गा॑य सा॒रङ्गा॑य॒ स्वाहा॑ पि॒शङ्गा॑य पि॒शङ्गा॑य॒ स्वाहा॑ सा॒रङ्गा॑य । \newline
6. स्वाहा॑ सा॒रङ्गा॑य सा॒रङ्गा॑य॒ स्वाहा॒ स्वाहा॑ सा॒रङ्गा॑य॒ स्वाहा॒ स्वाहा॑ सा॒रङ्गा॑य॒ स्वाहा॒ स्वाहा॑ सा॒रङ्गा॑य॒ स्वाहा᳚ । \newline
7. सा॒रङ्गा॑य॒ स्वाहा॒ स्वाहा॑ सा॒रङ्गा॑य सा॒रङ्गा॑य॒ स्वाहा॑ ऽरु॒णाया॑ रु॒णाय॒ स्वाहा॑ सा॒रङ्गा॑य सा॒रङ्गा॑य॒ स्वाहा॑ ऽरु॒णाय॑ । \newline
8. स्वाहा॑ ऽरु॒णाया॑ रु॒णाय॒ स्वाहा॒ स्वाहा॑ ऽरु॒णाय॒ स्वाहा॒ स्वाहा॑ ऽरु॒णाय॒ स्वाहा॒ स्वाहा॑ ऽरु॒णाय॒ स्वाहा᳚ । \newline
9. अ॒रु॒णाय॒ स्वाहा॒ स्वाहा॑ ऽरु॒णाया॑ रु॒णाय॒ स्वाहा॑ गौ॒राय॑ गौ॒राय॒ स्वाहा॑ ऽरु॒णाया॑ रु॒णाय॒ स्वाहा॑ गौ॒राय॑ । \newline
10. स्वाहा॑ गौ॒राय॑ गौ॒राय॒ स्वाहा॒ स्वाहा॑ गौ॒राय॒ स्वाहा॒ स्वाहा॑ गौ॒राय॒ स्वाहा॒ स्वाहा॑ गौ॒राय॒ स्वाहा᳚ । \newline
11. गौ॒राय॒ स्वाहा॒ स्वाहा॑ गौ॒राय॑ गौ॒राय॒ स्वाहा॑ ब॒भ्रवे॑ ब॒भ्रवे॒ स्वाहा॑ गौ॒राय॑ गौ॒राय॒ स्वाहा॑ ब॒भ्रवे᳚ । \newline
12. स्वाहा॑ ब॒भ्रवे॑ ब॒भ्रवे॒ स्वाहा॒ स्वाहा॑ ब॒भ्रवे॒ स्वाहा॒ स्वाहा॑ ब॒भ्रवे॒ स्वाहा॒ स्वाहा॑ ब॒भ्रवे॒ स्वाहा᳚ । \newline
13. ब॒भ्रवे॒ स्वाहा॒ स्वाहा॑ ब॒भ्रवे॑ ब॒भ्रवे॒ स्वाहा॑ नकु॒लाय॑ नकु॒लाय॒ स्वाहा॑ ब॒भ्रवे॑ ब॒भ्रवे॒ स्वाहा॑ नकु॒लाय॑ । \newline
14. स्वाहा॑ नकु॒लाय॑ नकु॒लाय॒ स्वाहा॒ स्वाहा॑ नकु॒लाय॒ स्वाहा॒ स्वाहा॑ नकु॒लाय॒ स्वाहा॒ स्वाहा॑ नकु॒लाय॒ स्वाहा᳚ । \newline
15. न॒कु॒लाय॒ स्वाहा॒ स्वाहा॑ नकु॒लाय॑ नकु॒लाय॒ स्वाहा॒ रोहि॑ताय॒ रोहि॑ताय॒ स्वाहा॑ नकु॒लाय॑ नकु॒लाय॒ स्वाहा॒ रोहि॑ताय । \newline
16. स्वाहा॒ रोहि॑ताय॒ रोहि॑ताय॒ स्वाहा॒ स्वाहा॒ रोहि॑ताय॒ स्वाहा॒ स्वाहा॒ रोहि॑ताय॒ स्वाहा॒ स्वाहा॒ रोहि॑ताय॒ स्वाहा᳚ । \newline
17. रोहि॑ताय॒ स्वाहा॒ स्वाहा॒ रोहि॑ताय॒ रोहि॑ताय॒ स्वाहा॒ शोणा॑य॒ शोणा॑य॒ स्वाहा॒ रोहि॑ताय॒ रोहि॑ताय॒ स्वाहा॒ शोणा॑य । \newline
18. स्वाहा॒ शोणा॑य॒ शोणा॑य॒ स्वाहा॒ स्वाहा॒ शोणा॑य॒ स्वाहा॒ स्वाहा॒ शोणा॑य॒ स्वाहा॒ स्वाहा॒ शोणा॑य॒ स्वाहा᳚ । \newline
19. शोणा॑य॒ स्वाहा॒ स्वाहा॒ शोणा॑य॒ शोणा॑य॒ स्वाहा᳚ श्या॒वाय॑ श्या॒वाय॒ स्वाहा॒ शोणा॑य॒ शोणा॑य॒ स्वाहा᳚ श्या॒वाय॑ । \newline
20. स्वाहा᳚ श्या॒वाय॑ श्या॒वाय॒ स्वाहा॒ स्वाहा᳚ श्या॒वाय॒ स्वाहा॒ स्वाहा᳚ श्या॒वाय॒ स्वाहा॒ स्वाहा᳚ श्या॒वाय॒ स्वाहा᳚ । \newline
21. श्या॒वाय॒ स्वाहा॒ स्वाहा᳚ श्या॒वाय॑ श्या॒वाय॒ स्वाहा᳚ श्या॒माय॑ श्या॒माय॒ स्वाहा᳚ श्या॒वाय॑ श्या॒वाय॒ स्वाहा᳚ श्या॒माय॑ । \newline
22. स्वाहा᳚ श्या॒माय॑ श्या॒माय॒ स्वाहा॒ स्वाहा᳚ श्या॒माय॒ स्वाहा॒ स्वाहा᳚ श्या॒माय॒ स्वाहा॒ स्वाहा᳚ श्या॒माय॒ स्वाहा᳚ । \newline
23. श्या॒माय॒ स्वाहा॒ स्वाहा᳚ श्या॒माय॑ श्या॒माय॒ स्वाहा॑ पाक॒लाय॑ पाक॒लाय॒ स्वाहा᳚ श्या॒माय॑ श्या॒माय॒ स्वाहा॑ पाक॒लाय॑ । \newline
24. स्वाहा॑ पाक॒लाय॑ पाक॒लाय॒ स्वाहा॒ स्वाहा॑ पाक॒लाय॒ स्वाहा॒ स्वाहा॑ पाक॒लाय॒ स्वाहा॒ स्वाहा॑ पाक॒लाय॒ स्वाहा᳚ । \newline
25. पा॒क॒लाय॒ स्वाहा॒ स्वाहा॑ पाक॒लाय॑ पाक॒लाय॒ स्वाहा॑ सुरू॒पाय॑ सुरू॒पाय॒ स्वाहा॑ पाक॒लाय॑ पाक॒लाय॒ स्वाहा॑ सुरू॒पाय॑ । \newline
26. स्वाहा॑ सुरू॒पाय॑ सुरू॒पाय॒ स्वाहा॒ स्वाहा॑ सुरू॒पाय॒ स्वाहा॒ स्वाहा॑ सुरू॒पाय॒ स्वाहा॒ स्वाहा॑ सुरू॒पाय॒ स्वाहा᳚ । \newline
27. सु॒रू॒पाय॒ स्वाहा॒ स्वाहा॑ सुरू॒पाय॑ सुरू॒पाय॒ स्वाहा ऽनु॑रूपा॒या नु॑रूपाय॒ स्वाहा॑ सुरू॒पाय॑ सुरू॒पाय॒ स्वाहा ऽनु॑रूपाय । \newline
28. सु॒रू॒पायेति॑ सु - रू॒पाय॑ । \newline
29. स्वाहा ऽनु॑रूपा॒या नु॑रूपाय॒ स्वाहा॒ स्वाहा ऽनु॑रूपाय॒ स्वाहा॒ स्वाहा ऽनु॑रूपाय॒ स्वाहा॒ स्वाहा ऽनु॑रूपाय॒ स्वाहा᳚ । \newline
30. अनु॑रूपाय॒ स्वाहा॒ स्वाहा ऽनु॑रूपा॒या नु॑रूपाय॒ स्वाहा॒ विरू॑पाय॒ विरू॑पाय॒ स्वाहा ऽनु॑रूपा॒या नु॑रूपाय॒ स्वाहा॒ विरू॑पाय । \newline
31. अनु॑रूपा॒येत्यनु॑ - रू॒पा॒य॒ । \newline
32. स्वाहा॒ विरू॑पाय॒ विरू॑पाय॒ स्वाहा॒ स्वाहा॒ विरू॑पाय॒ स्वाहा॒ स्वाहा॒ विरू॑पाय॒ स्वाहा॒ स्वाहा॒ विरू॑पाय॒ स्वाहा᳚ । \newline
33. विरू॑पाय॒ स्वाहा॒ स्वाहा॒ विरू॑पाय॒ विरू॑पाय॒ स्वाहा॒ सरू॑पाय॒ सरू॑पाय॒ स्वाहा॒ विरू॑पाय॒ विरू॑पाय॒ स्वाहा॒ सरू॑पाय । \newline
34. विरू॑पा॒येति॒ वि - रू॒पा॒य॒ । \newline
35. स्वाहा॒ सरू॑पाय॒ सरू॑पाय॒ स्वाहा॒ स्वाहा॒ सरू॑पाय॒ स्वाहा॒ स्वाहा॒ सरू॑पाय॒ स्वाहा॒ स्वाहा॒ सरू॑पाय॒ स्वाहा᳚ । \newline
36. सरू॑पाय॒ स्वाहा॒ स्वाहा॒ सरू॑पाय॒ सरू॑पाय॒ स्वाहा॒ प्रति॑रूपाय॒ प्रति॑रूपाय॒ स्वाहा॒ सरू॑पाय॒ सरू॑पाय॒ स्वाहा॒ प्रति॑रूपाय । \newline
37. सरू॑पा॒येति॒ स - रू॒पा॒य॒ । \newline
38. स्वाहा॒ प्रति॑रूपाय॒ प्रति॑रूपाय॒ स्वाहा॒ स्वाहा॒ प्रति॑रूपाय॒ स्वाहा॒ स्वाहा॒ प्रति॑रूपाय॒ स्वाहा॒ स्वाहा॒ प्रति॑रूपाय॒ स्वाहा᳚ । \newline
39. प्रति॑रूपाय॒ स्वाहा॒ स्वाहा॒ प्रति॑रूपाय॒ प्रति॑रूपाय॒ स्वाहा॑ श॒बला॑य श॒बला॑य॒ स्वाहा॒ प्रति॑रूपाय॒ प्रति॑रूपाय॒ स्वाहा॑ श॒बला॑य । \newline
40. प्रति॑रूपा॒येति॒ प्रति॑ - रू॒पा॒य॒ । \newline
41. स्वाहा॑ श॒बला॑य श॒बला॑य॒ स्वाहा॒ स्वाहा॑ श॒बला॑य॒ स्वाहा॒ स्वाहा॑ श॒बला॑य॒ स्वाहा॒ स्वाहा॑ श॒बला॑य॒ स्वाहा᳚ । \newline
42. श॒बला॑य॒ स्वाहा॒ स्वाहा॑ श॒बला॑य श॒बला॑य॒ स्वाहा॑ कम॒लाय॑ कम॒लाय॒ स्वाहा॑ श॒बला॑य श॒बला॑य॒ स्वाहा॑ कम॒लाय॑ । \newline
43. स्वाहा॑ कम॒लाय॑ कम॒लाय॒ स्वाहा॒ स्वाहा॑ कम॒लाय॒ स्वाहा॒ स्वाहा॑ कम॒लाय॒ स्वाहा॒ स्वाहा॑ कम॒लाय॒ स्वाहा᳚ । \newline
44. क॒म॒लाय॒ स्वाहा॒ स्वाहा॑ कम॒लाय॑ कम॒लाय॒ स्वाहा॒ पृश्ञ॑ये॒ पृश्ञ॑ये॒ स्वाहा॑ कम॒लाय॑ कम॒लाय॒ स्वाहा॒ पृश्ञ॑ये । \newline
45. स्वाहा॒ पृश्ञ॑ये॒ पृश्ञ॑ये॒ स्वाहा॒ स्वाहा॒ पृश्ञ॑ये॒ स्वाहा॒ स्वाहा॒ पृश्ञ॑ये॒ स्वाहा॒ स्वाहा॒ पृश्ञ॑ये॒ स्वाहा᳚ । \newline
46. पृश्ञ॑ये॒ स्वाहा॒ स्वाहा॒ पृश्ञ॑ये॒ पृश्ञ॑ये॒ स्वाहा॑ पृश्ञिस॒क्थाय॑ पृश्ञिस॒क्थाय॒ स्वाहा॒ पृश्ञ॑ये॒ पृश्ञ॑ये॒ स्वाहा॑ पृश्ञिस॒क्थाय॑ । \newline
47. स्वाहा॑ पृश्ञिस॒क्थाय॑ पृश्ञिस॒क्थाय॒ स्वाहा॒ स्वाहा॑ पृश्ञिस॒क्थाय॒ स्वाहा॒ स्वाहा॑ पृश्ञिस॒क्थाय॒ स्वाहा॒ स्वाहा॑ पृश्ञिस॒क्थाय॒ स्वाहा᳚ । \newline
48. पृ॒श्ञि॒स॒क्थाय॒ स्वाहा॒ स्वाहा॑ पृश्ञिस॒क्थाय॑ पृश्ञिस॒क्थाय॒ स्वाहा॒ सर्व॑स्मै॒ सर्व॑स्मै॒ स्वाहा॑ पृश्ञिस॒क्थाय॑ पृश्ञिस॒क्थाय॒ स्वाहा॒ सर्व॑स्मै । \newline
49. पृ॒श्ञि॒स॒क्थायेति॑ पृश्ञि - स॒क्थाय॑ । \newline
50. स्वाहा॒ सर्व॑स्मै॒ सर्व॑स्मै॒ स्वाहा॒ स्वाहा॒ सर्व॑स्मै॒ स्वाहा॒ स्वाहा॒ सर्व॑स्मै॒ स्वाहा॒ स्वाहा॒ सर्व॑स्मै॒ स्वाहा᳚ । \newline
51. सर्व॑स्मै॒ स्वाहा॒ स्वाहा॒ सर्व॑स्मै॒ सर्व॑स्मै॒ स्वाहा᳚ । \newline
52. स्वाहेति॒ स्वाहा᳚ । \newline
\pagebreak
\markright{ TS 7.3.19.1  \hfill https://www.vedavms.in \hfill}

\section{ TS 7.3.19.1 }

\textbf{TS 7.3.19.1 } \newline
\textbf{Samhita Paata} \newline

ओष॑धीभ्यः॒ स्वाहा॒ मूले᳚भ्यः॒ स्वाहा॒ तूले᳚भ्यः॒ स्वाहा॒ काण्डे᳚भ्यः॒ स्वाहा॒ वल्.शे᳚भ्यः॒ स्वाहा॒ पुष्पे᳚भ्यः॒ स्वाहा॒ फले᳚भ्यः॒ स्वाहा॑ गृही॒तेभ्यः॒ स्वाहा ऽगृ॑हीतेभ्यः॒ स्वाहा ऽव॑पन्नेभ्यः॒ स्वाहा॒ शया॑नेभ्यः॒ स्वाहा॒ सर्व॑स्मै॒ स्वाहा᳚ ॥ \newline

\textbf{Pada Paata} \newline

ओष॑धीभ्य॒ इत्योष॑धि-भ्यः॒ । स्वाहा᳚ । मूले᳚भ्यः । स्वाहा᳚ । तूले᳚भ्यः । स्वाहा᳚ । काण्डे᳚भ्यः । स्वाहा᳚ । वल्.शे᳚भ्यः । स्वाहा᳚ । पुष्पे᳚भ्यः । स्वाहा᳚ । फले᳚भ्यः । स्वाहा᳚ । गृ॒ही॒तेभ्यः॑ । स्वाहा᳚ । अगृ॑हीतेभ्यः । स्वाहा᳚ । अव॑पन्नेभ्य॒ इत्यव॑ - प॒न्ने॒भ्यः॒ । स्वाहा᳚ । शया॑नेभ्यः । स्वाहा᳚ । सर्व॑स्मै । स्वाहा᳚ ॥  \newline


\textbf{Krama Paata} \newline

ओष॑धीभ्यः॒ स्वाहा᳚ । ओष॑धीभ्य॒ इत्योष॑धि - भ्यः॒ । स्वाहा॒ मूले᳚भ्यः । मूले᳚भ्यः॒ स्वाहा᳚ । स्वाहा॒ तूले᳚भ्यः । तूले᳚भ्यः॒ स्वाहा᳚ । स्वाहा॒ काण्डे᳚भ्यः । काण्डे᳚भ्यः॒ स्वाहा᳚ । स्वाहा॒ वल्.शे᳚भ्यः । वल्.शे᳚भ्यः॒ स्वाहा᳚ । स्वाहा॒ पुष्पे᳚भ्यः । पुष्पे᳚भ्यः॒ स्वाहा᳚ । स्वाहा॒ फले᳚भ्यः । फले᳚भ्यः॒ स्वाहा᳚ । स्वाहा॑ गृही॒तेभ्यः॑ । गृ॒ही॒तेभ्यः॒ स्वाहा᳚ । स्वाहाऽगृ॑हीतेभ्यः । अगृ॑हीतेभ्यः॒ स्वाहा᳚ । स्वाहाऽव॑पन्नेभ्यः । अव॑पन्नेभ्यः॒ स्वाहा᳚ । अव॑पन्नेभ्य॒ इत्यव॑ - प॒न्ने॒भ्यः॒ । स्वाहा॒ शया॑नेभ्यः । शया॑नेभ्यः॒ स्वाहा᳚ । स्वाहा॒ सर्व॑स्मै । सर्व॑स्मै॒ स्वाहा᳚ । स्वाहेति॒ स्वाहा᳚ । \newline

\textbf{Jatai Paata} \newline

1. ओष॑धीभ्यः॒ स्वाहा॒ स्वाहौष॑धीभ्य॒ ओष॑धीभ्यः॒ स्वाहा᳚ । \newline
2. ओष॑धीभ्य॒ इत्योष॑धि - भ्यः॒ । \newline
3. स्वाहा॒ मूले᳚भ्यो॒ मूले᳚भ्यः॒ स्वाहा॒ स्वाहा॒ मूले᳚भ्यः । \newline
4. मूले᳚भ्यः॒ स्वाहा॒ स्वाहा॒ मूले᳚भ्यो॒ मूले᳚भ्यः॒ स्वाहा᳚ । \newline
5. स्वाहा॒ तूले᳚भ्य॒ स्तूले᳚भ्यः॒ स्वाहा॒ स्वाहा॒ तूले᳚भ्यः । \newline
6. तूले᳚भ्यः॒ स्वाहा॒ स्वाहा॒ तूले᳚भ्य॒ स्तूले᳚भ्यः॒ स्वाहा᳚ । \newline
7. स्वाहा॒ काण्डे᳚भ्यः॒ काण्डे᳚भ्यः॒ स्वाहा॒ स्वाहा॒ काण्डे᳚भ्यः । \newline
8. काण्डे᳚भ्यः॒ स्वाहा॒ स्वाहा॒ काण्डे᳚भ्यः॒ काण्डे᳚भ्यः॒ स्वाहा᳚ । \newline
9. स्वाहा॒ वल्.शे᳚भ्यो॒ वल्.शे᳚भ्यः॒ स्वाहा॒ स्वाहा॒ वल्.शे᳚भ्यः । \newline
10. वल्.शे᳚भ्यः॒ स्वाहा॒ स्वाहा॒ वल्.शे᳚भ्यो॒ वल्.शे᳚भ्यः॒ स्वाहा᳚ । \newline
11. स्वाहा॒ पुष्पे᳚भ्यः॒ पुष्पे᳚भ्यः॒ स्वाहा॒ स्वाहा॒ पुष्पे᳚भ्यः । \newline
12. पुष्पे᳚भ्यः॒ स्वाहा॒ स्वाहा॒ पुष्पे᳚भ्यः॒ पुष्पे᳚भ्यः॒ स्वाहा᳚ । \newline
13. स्वाहा॒ फले᳚भ्यः॒ फले᳚भ्यः॒ स्वाहा॒ स्वाहा॒ फले᳚भ्यः । \newline
14. फले᳚भ्यः॒ स्वाहा॒ स्वाहा॒ फले᳚भ्यः॒ फले᳚भ्यः॒ स्वाहा᳚ । \newline
15. स्वाहा॑ गृही॒तेभ्यो॑ गृही॒तेभ्यः॒ स्वाहा॒ स्वाहा॑ गृही॒तेभ्यः॑ । \newline
16. गृ॒ही॒तेभ्यः॒ स्वाहा॒ स्वाहा॑ गृही॒तेभ्यो॑ गृही॒तेभ्यः॒ स्वाहा᳚ । \newline
17. स्वाहा ऽगृ॑हीते॒भ्यो ऽगृ॑हीतेभ्यः॒ स्वाहा॒ स्वाहा ऽगृ॑हीतेभ्यः । \newline
18. अगृ॑हीतेभ्यः॒ स्वाहा॒ स्वाहा ऽगृ॑हीते॒भ्यो ऽगृ॑हीतेभ्यः॒ स्वाहा᳚ । \newline
19. स्वाहा ऽव॑पन्ने॒भ्यो ऽव॑पन्नेभ्यः॒ स्वाहा॒ स्वाहा ऽव॑पन्नेभ्यः । \newline
20. अव॑पन्नेभ्यः॒ स्वाहा॒ स्वाहा ऽव॑पन्ने॒भ्यो ऽव॑पन्नेभ्यः॒ स्वाहा᳚ । \newline
21. अव॑पन्नेभ्य॒ इत्यव॑ - प॒न्ने॒भ्यः॒ । \newline
22. स्वाहा॒ शया॑नेभ्यः॒ शया॑नेभ्यः॒ स्वाहा॒ स्वाहा॒ शया॑नेभ्यः । \newline
23. शया॑नेभ्यः॒ स्वाहा॒ स्वाहा॒ शया॑नेभ्यः॒ शया॑नेभ्यः॒ स्वाहा᳚ । \newline
24. स्वाहा॒ सर्व॑स्मै॒ सर्व॑स्मै॒ स्वाहा॒ स्वाहा॒ सर्व॑स्मै । \newline
25. सर्व॑स्मै॒ स्वाहा॒ स्वाहा॒ सर्व॑स्मै॒ सर्व॑स्मै॒ स्वाहा᳚ । \newline
26. स्वाहेति॒ स्वाहा᳚ । \newline

\textbf{Ghana Paata } \newline

1. ओष॑धीभ्यः॒ स्वाहा॒ स्वाहौष॑धीभ्य॒ ओष॑धीभ्यः॒ स्वाहा॒ मूले᳚भ्यो॒ मूले᳚भ्यः॒ स्वाहौष॑धीभ्य॒ ओष॑धीभ्यः॒ स्वाहा॒ मूले᳚भ्यः । \newline
2. ओष॑धीभ्य॒ इत्योष॑धि - भ्यः॒ । \newline
3. स्वाहा॒ मूले᳚भ्यो॒ मूले᳚भ्यः॒ स्वाहा॒ स्वाहा॒ मूले᳚भ्यः॒ स्वाहा॒ स्वाहा॒ मूले᳚भ्यः॒ स्वाहा॒ स्वाहा॒ मूले᳚भ्यः॒ स्वाहा᳚ । \newline
4. मूले᳚भ्यः॒ स्वाहा॒ स्वाहा॒ मूले᳚भ्यो॒ मूले᳚भ्यः॒ स्वाहा॒ तूले᳚भ्य॒ स्तूले᳚भ्यः॒ स्वाहा॒ मूले᳚भ्यो॒ मूले᳚भ्यः॒ स्वाहा॒ तूले᳚भ्यः । \newline
5. स्वाहा॒ तूले᳚भ्य॒ स्तूले᳚भ्यः॒ स्वाहा॒ स्वाहा॒ तूले᳚भ्यः॒ स्वाहा॒ स्वाहा॒ तूले᳚भ्यः॒ स्वाहा॒ स्वाहा॒ तूले᳚भ्यः॒ स्वाहा᳚ । \newline
6. तूले᳚भ्यः॒ स्वाहा॒ स्वाहा॒ तूले᳚भ्य॒ स्तूले᳚भ्यः॒ स्वाहा॒ काण्डे᳚भ्यः॒ काण्डे᳚भ्यः॒ स्वाहा॒ तूले᳚भ्य॒ स्तूले᳚भ्यः॒ स्वाहा॒ काण्डे᳚भ्यः । \newline
7. स्वाहा॒ काण्डे᳚भ्यः॒ काण्डे᳚भ्यः॒ स्वाहा॒ स्वाहा॒ काण्डे᳚भ्यः॒ स्वाहा॒ स्वाहा॒ काण्डे᳚भ्यः॒ स्वाहा॒ स्वाहा॒ काण्डे᳚भ्यः॒ स्वाहा᳚ । \newline
8. काण्डे᳚भ्यः॒ स्वाहा॒ स्वाहा॒ काण्डे᳚भ्यः॒ काण्डे᳚भ्यः॒ स्वाहा॒ वल्.शे᳚भ्यो॒ वल्.शे᳚भ्यः॒ स्वाहा॒ काण्डे᳚भ्यः॒ काण्डे᳚भ्यः॒ स्वाहा॒ वल्.शे᳚भ्यः । \newline
9. स्वाहा॒ वल्.शे᳚भ्यो॒ वल्.शे᳚भ्यः॒ स्वाहा॒ स्वाहा॒ वल्.शे᳚भ्यः॒ स्वाहा॒ स्वाहा॒ वल्.शे᳚भ्यः॒ स्वाहा॒ स्वाहा॒ वल्.शे᳚भ्यः॒ स्वाहा᳚ । \newline
10. वल्.शे᳚भ्यः॒ स्वाहा॒ स्वाहा॒ वल्.शे᳚भ्यो॒ वल्.शे᳚भ्यः॒ स्वाहा॒ पुष्पे᳚भ्यः॒ पुष्पे᳚भ्यः॒ स्वाहा॒ वल्.शे᳚भ्यो॒ वल्.शे᳚भ्यः॒ स्वाहा॒ पुष्पे᳚भ्यः । \newline
11. स्वाहा॒ पुष्पे᳚भ्यः॒ पुष्पे᳚भ्यः॒ स्वाहा॒ स्वाहा॒ पुष्पे᳚भ्यः॒ स्वाहा॒ स्वाहा॒ पुष्पे᳚भ्यः॒ स्वाहा॒ स्वाहा॒ पुष्पे᳚भ्यः॒ स्वाहा᳚ । \newline
12. पुष्पे᳚भ्यः॒ स्वाहा॒ स्वाहा॒ पुष्पे᳚भ्यः॒ पुष्पे᳚भ्यः॒ स्वाहा॒ फले᳚भ्यः॒ फले᳚भ्यः॒ स्वाहा॒ पुष्पे᳚भ्यः॒ पुष्पे᳚भ्यः॒ स्वाहा॒ फले᳚भ्यः । \newline
13. स्वाहा॒ फले᳚भ्यः॒ फले᳚भ्यः॒ स्वाहा॒ स्वाहा॒ फले᳚भ्यः॒ स्वाहा॒ स्वाहा॒ फले᳚भ्यः॒ स्वाहा॒ स्वाहा॒ फले᳚भ्यः॒ स्वाहा᳚ । \newline
14. फले᳚भ्यः॒ स्वाहा॒ स्वाहा॒ फले᳚भ्यः॒ फले᳚भ्यः॒ स्वाहा॑ गृही॒तेभ्यो॑ गृही॒तेभ्यः॒ स्वाहा॒ फले᳚भ्यः॒ फले᳚भ्यः॒ स्वाहा॑ गृही॒तेभ्यः॑ । \newline
15. स्वाहा॑ गृही॒तेभ्यो॑ गृही॒तेभ्यः॒ स्वाहा॒ स्वाहा॑ गृही॒तेभ्यः॒ स्वाहा॒ स्वाहा॑ गृही॒तेभ्यः॒ स्वाहा॒ स्वाहा॑ गृही॒तेभ्यः॒ स्वाहा᳚ । \newline
16. गृ॒ही॒तेभ्यः॒ स्वाहा॒ स्वाहा॑ गृही॒तेभ्यो॑ गृही॒तेभ्यः॒ स्वाहा ऽगृ॑हीते॒भ्यो ऽगृ॑हीतेभ्यः॒ स्वाहा॑ गृही॒तेभ्यो॑ गृही॒तेभ्यः॒ स्वाहा ऽगृ॑हीतेभ्यः । \newline
17. स्वाहा ऽगृ॑हीते॒भ्यो ऽगृ॑हीतेभ्यः॒ स्वाहा॒ स्वाहा ऽगृ॑हीतेभ्यः॒ स्वाहा॒ स्वाहा ऽगृ॑हीतेभ्यः॒ स्वाहा॒ स्वाहा ऽगृ॑हीतेभ्यः॒ स्वाहा᳚ । \newline
18. अगृ॑हीतेभ्यः॒ स्वाहा॒ स्वाहा ऽगृ॑हीते॒भ्यो ऽगृ॑हीतेभ्यः॒ स्वाहा ऽव॑पन्ने॒भ्यो ऽव॑पन्नेभ्यः॒ स्वाहा ऽगृ॑हीते॒भ्यो ऽगृ॑हीतेभ्यः॒ स्वाहा ऽव॑पन्नेभ्यः । \newline
19. स्वाहा ऽव॑पन्ने॒भ्यो ऽव॑पन्नेभ्यः॒ स्वाहा॒ स्वाहा ऽव॑पन्नेभ्यः॒ स्वाहा॒ स्वाहा ऽव॑पन्नेभ्यः॒ स्वाहा॒ स्वाहा ऽव॑पन्नेभ्यः॒ स्वाहा᳚ । \newline
20. अव॑पन्नेभ्यः॒ स्वाहा॒ स्वाहा ऽव॑पन्ने॒भ्यो ऽव॑पन्नेभ्यः॒ स्वाहा॒ शया॑नेभ्यः॒ शया॑नेभ्यः॒ स्वाहा ऽव॑पन्ने॒भ्यो ऽव॑पन्नेभ्यः॒ स्वाहा॒ शया॑नेभ्यः । \newline
21. अव॑पन्नेभ्य॒ इत्यव॑ - प॒न्ने॒भ्यः॒ । \newline
22. स्वाहा॒ शया॑नेभ्यः॒ शया॑नेभ्यः॒ स्वाहा॒ स्वाहा॒ शया॑नेभ्यः॒ स्वाहा॒ स्वाहा॒ शया॑नेभ्यः॒ स्वाहा॒ स्वाहा॒ शया॑नेभ्यः॒ स्वाहा᳚ । \newline
23. शया॑नेभ्यः॒ स्वाहा॒ स्वाहा॒ शया॑नेभ्यः॒ शया॑नेभ्यः॒ स्वाहा॒ सर्व॑स्मै॒ सर्व॑स्मै॒ स्वाहा॒ शया॑नेभ्यः॒ शया॑नेभ्यः॒ स्वाहा॒ सर्व॑स्मै । \newline
24. स्वाहा॒ सर्व॑स्मै॒ सर्व॑स्मै॒ स्वाहा॒ स्वाहा॒ सर्व॑स्मै॒ स्वाहा॒ स्वाहा॒ सर्व॑स्मै॒ स्वाहा॒ स्वाहा॒ सर्व॑स्मै॒ स्वाहा᳚ । \newline
25. सर्व॑स्मै॒ स्वाहा॒ स्वाहा॒ सर्व॑स्मै॒ सर्व॑स्मै॒ स्वाहा᳚ । \newline
26. स्वाहेति॒ स्वाहा᳚ । \newline
\pagebreak
\markright{ TS 7.3.20.1  \hfill https://www.vedavms.in \hfill}

\section{ TS 7.3.20.1 }

\textbf{TS 7.3.20.1 } \newline
\textbf{Samhita Paata} \newline

वन॒स्पति॑भ्यः॒ स्वाहा॒ मूले᳚भ्यः॒ स्वाहा॒ तूले᳚भ्यः॒ स्वाहा॒ स्कन्धो᳚भ्यः॒ स्वाहा॒ शाखा᳚भ्यः॒ स्वाहा॑ प॒र्णेभ्यः॒ स्वाहा॒ पुष्पे᳚भ्यः॒ स्वाहा॒ फले᳚भ्यः॒ स्वाहा॑ गृही॒तेभ्यः॒ स्वाहा ऽगृ॑हीतेभ्यः॒ स्वाहा ऽव॑पन्नेभ्यः॒ स्वाहा॒ शया॑नेभ्यः॒ स्वाहा॑ शि॒ष्टाय॒ स्वाहा ऽति॑शिष्टाय॒ स्वाहा॒ परि॑शिष्टाय॒ स्वाहा॒ सꣳशि॑ष्टाय॒ स्वाहो-च्छि॑ष्टाय॒ स्वाहा॑ रि॒क्ताय॒ स्वाहा ऽरि॑क्ताय॒ स्वाहा॒ प्ररि॑क्ताय॒ स्वाहा॒ सꣳरि॑क्ताय॒ स्वाहो -द्रि॑क्ताय॒ स्वाहा॒ सर्व॑स्मै॒ स्वाहा᳚ ॥ \newline

\textbf{Pada Paata} \newline

वन॒स्पति॑भ्य॒ इति॒ वन॒स्पति॑ - भ्यः॒ । स्वाहा᳚ । मूले᳚भ्यः । स्वाहा᳚ । तूले᳚भ्यः । स्वाहा᳚ । स्कन्धो᳚भ्य॒ इति॒ स्कन्धः॑ - भ्यः॒ । स्वाहा᳚ । शाखा᳚भ्यः । स्वाहा᳚ । प॒र्णेभ्यः॑ । स्वाहा᳚ । पुष्पे᳚भ्यः । स्वाहा᳚ । फले᳚भ्यः । स्वाहा᳚ । गृ॒ही॒तेभ्यः॑ । स्वाहा᳚ । अगृ॑हीतेभ्यः । स्वाहा᳚ । अव॑पन्नेभ्य॒ इत्यव॑ - प॒न्ने॒भ्यः॒ । स्वाहा᳚ । शया॑नेभ्यः । स्वाहा᳚ । शि॒ष्टाय॑ । स्वाहा᳚ । अति॑शिष्टा॒येत्यति॑ - शि॒ष्टा॒य॒ । स्वाहा᳚ । परि॑शिष्टा॒येति॒ परि॑-शि॒ष्टा॒य॒ । स्वाहा᳚ । सꣳशि॑ष्टा॒येति॒ सं - शि॒ष्टा॒य॒ । स्वाहा᳚ । उच्छि॑ष्टा॒येत्युत् - शि॒ष्टा॒य॒ । स्वाहा᳚ । रि॒क्ताय॑ । स्वाहा᳚ । अरि॑क्ताय । स्वाहा᳚ । प्ररि॑क्ता॒येति॒ प्र - रि॒क्ता॒य॒ । स्वाहा᳚ । सꣳरि॑क्ता॒येति॒ सं - रि॒क्ता॒य॒ । स्वाहा᳚ । उद्रि॑क्ता॒येत्युत् - रि॒क्ता॒य॒ । स्वाहा᳚ । सर्व॑स्मै । स्वाहा᳚ ॥  \newline


\textbf{Krama Paata} \newline

वन॒स्पति॑भ्यः॒ स्वाहा᳚ । वन॒स्पति॑भ्य॒ इति॒ वन॒स्पति॑ - भ्यः॒ । स्वाहा॒ मूले᳚भ्यः । मूले᳚भ्यः॒ स्वाहाः᳚ । स्वाहा॒ तूले᳚भ्यः । तूले᳚भ्यः॒ स्वाहा᳚ । स्वाहा॒ स्कन्धो᳚भ्यः । स्कन्धो᳚भ्यः॒ स्वाहा᳚ । स्कन्धो᳚भ्य॒ इति॒ स्कन्धः॑ - भ्यः॒ । स्वाहा॒ शाखा᳚भ्यः । शाखा᳚भ्यः॒ स्वाहा᳚ । स्वाहा॑ प॒र्णेभ्यः॑ । प॒र्णेभ्यः॒ स्वाहा᳚ । स्वाहा॒ पुष्पे᳚भ्यः । पुष्पे᳚भ्यः॒ स्वाहा᳚ । स्वाहा॒ फले᳚भ्यः । फले᳚भ्यः॒ स्वाहा᳚ । स्वाहा॑ गृही॒तेभ्यः॑ । गृ॒ही॒तेभ्यः॒ स्वाहा᳚ । स्वाहाऽगृ॑हीतेभ्यः । अगृ॑हीतेभ्यः॒ स्वाहा᳚ । स्वाहाऽव॑पन्नेभ्यः । अव॑पन्नेभ्यः॒ स्वाहा᳚ । अव॑पन्नेभ्य॒ इत्यव॑ - प॒न्ने॒भ्यः॒ । स्वाहा॒ शया॑नेभ्यः । शया॑नेभ्यः॒ स्वाहा᳚ । स्वाहा॑ शि॒ष्टाय॑ । शि॒ष्टाय॒ स्वाहा᳚ । स्वाहाऽति॑शिष्टाय । अति॑शिष्टाय॒ स्वाहा᳚ । अति॑शिष्टा॒येत्यति॑ - शि॒ष्टा॒य॒ । स्वाहा॒ परि॑शिष्टाय । परि॑शिष्टाय॒ स्वाहा᳚ । परि॑शिष्टा॒येति॒ परि॑ - शि॒ष्टा॒य॒ । स्वाहा॒ सꣳशि॑ष्टाय । सꣳशि॑ष्टाय॒ स्वाहा᳚ । सꣳशि॑ष्टा॒येति॒ सम् - शि॒ष्टा॒य॒ । स्वाहोच्छि॑ष्टाय । उच्छि॑ष्टाय॒ स्वाहा᳚ । उच्छि॑ष्टा॒येत्युत् - शि॒ष्टा॒य॒ । स्वाहा॑ रि॒क्ताय॑ । रि॒क्ताय॒ स्वाहा᳚ । स्वाहाऽरि॑क्ताय । अरि॑क्ताय॒ स्वाहा᳚ । स्वाहा॒ प्ररि॑क्ताय । प्ररि॑क्ताय॒ स्वाहा᳚ । प्ररि॑क्ता॒येति॒ प्र - रि॒क्ता॒य॒ । स्वाहा॒ सꣳरि॑क्ताय । सꣳरि॑क्ताय॒ स्वाहा᳚ । सꣳरि॑क्ता॒येति॒ सम् - रि॒क्ता॒य॒ । स्वाहोद्रि॑क्ताय । उद्रि॑क्ताय॒ स्वाहा᳚ । उद्रि॑क्ता॒येत्युत् - रि॒क्ता॒य॒ । स्वाहा॒ सर्व॑स्मै । सर्व॑स्मै॒ स्वाहा᳚ । स्वाहेति॒ स्वाहा᳚ । \newline

\textbf{Jatai Paata} \newline

1. वन॒स्पति॑भ्यः॒ स्वाहा॒ स्वाहा॒ वन॒स्पति॑भ्यो॒ वन॒स्पति॑भ्यः॒ स्वाहा᳚ । \newline
2. वन॒स्पति॑भ्य॒ इति॒ वन॒स्पति॑ - भ्यः॒ । \newline
3. स्वाहा॒ मूले᳚भ्यो॒ मूले᳚भ्यः॒ स्वाहा॒ स्वाहा॒ मूले᳚भ्यः । \newline
4. मूले᳚भ्यः॒ स्वाहा॒ स्वाहा॒ मूले᳚भ्यो॒ मूले᳚भ्यः॒ स्वाहा᳚ । \newline
5. स्वाहा॒ तूले᳚भ्य॒ स्तूले᳚भ्यः॒ स्वाहा॒ स्वाहा॒ तूले᳚भ्यः । \newline
6. तूले᳚भ्यः॒ स्वाहा॒ स्वाहा॒ तूले᳚भ्य॒ स्तूले᳚भ्यः॒ स्वाहा᳚ । \newline
7. स्वाहा॒ स्कन्धो᳚भ्यः॒ स्कन्धो᳚भ्यः॒ स्वाहा॒ स्वाहा॒ स्कन्धो᳚भ्यः । \newline
8. स्कन्धो᳚भ्यः॒ स्वाहा॒ स्वाहा॒ स्कन्धो᳚भ्यः॒ स्कन्धो᳚भ्यः॒ स्वाहा᳚ । \newline
9. स्कन्धो᳚भ्य॒ इति॒ स्कन्धः॑ - भ्यः॒ । \newline
10. स्वाहा॒ शाखा᳚भ्यः॒ शाखा᳚भ्यः॒ स्वाहा॒ स्वाहा॒ शाखा᳚भ्यः । \newline
11. शाखा᳚भ्यः॒ स्वाहा॒ स्वाहा॒ शाखा᳚भ्यः॒ शाखा᳚भ्यः॒ स्वाहा᳚ । \newline
12. स्वाहा॑ प॒र्णेभ्यः॑ प॒र्णेभ्यः॒ स्वाहा॒ स्वाहा॑ प॒र्णेभ्यः॑ । \newline
13. प॒र्णेभ्यः॒ स्वाहा॒ स्वाहा॑ प॒र्णेभ्यः॑ प॒र्णेभ्यः॒ स्वाहा᳚ । \newline
14. स्वाहा॒ पुष्पे᳚भ्यः॒ पुष्पे᳚भ्यः॒ स्वाहा॒ स्वाहा॒ पुष्पे᳚भ्यः । \newline
15. पुष्पे᳚भ्यः॒ स्वाहा॒ स्वाहा॒ पुष्पे᳚भ्यः॒ पुष्पे᳚भ्यः॒ स्वाहा᳚ । \newline
16. स्वाहा॒ फले᳚भ्यः॒ फले᳚भ्यः॒ स्वाहा॒ स्वाहा॒ फले᳚भ्यः । \newline
17. फले᳚भ्यः॒ स्वाहा॒ स्वाहा॒ फले᳚भ्यः॒ फले᳚भ्यः॒ स्वाहा᳚ । \newline
18. स्वाहा॑ गृही॒तेभ्यो॑ गृही॒तेभ्यः॒ स्वाहा॒ स्वाहा॑ गृही॒तेभ्यः॑ । \newline
19. गृ॒ही॒तेभ्यः॒ स्वाहा॒ स्वाहा॑ गृही॒तेभ्यो॑ गृही॒तेभ्यः॒ स्वाहा᳚ । \newline
20. स्वाहा ऽगृ॑हीते॒भ्यो ऽगृ॑हीतेभ्यः॒ स्वाहा॒ स्वाहा ऽगृ॑हीतेभ्यः । \newline
21. अगृ॑हीतेभ्यः॒ स्वाहा॒ स्वाहा ऽगृ॑हीते॒भ्यो ऽगृ॑हीतेभ्यः॒ स्वाहा᳚ । \newline
22. स्वाहा ऽव॑पन्ने॒भ्यो ऽव॑पन्नेभ्यः॒ स्वाहा॒ स्वाहा ऽव॑पन्नेभ्यः । \newline
23. अव॑पन्नेभ्यः॒ स्वाहा॒ स्वाहा ऽव॑पन्ने॒भ्यो ऽव॑पन्नेभ्यः॒ स्वाहा᳚ । \newline
24. अव॑पन्नेभ्य॒ इत्यव॑ - प॒न्ने॒भ्यः॒ । \newline
25. स्वाहा॒ शया॑नेभ्यः॒ शया॑नेभ्यः॒ स्वाहा॒ स्वाहा॒ शया॑नेभ्यः । \newline
26. शया॑नेभ्यः॒ स्वाहा॒ स्वाहा॒ शया॑नेभ्यः॒ शया॑नेभ्यः॒ स्वाहा᳚ । \newline
27. स्वाहा॑ शि॒ष्टाय॑ शि॒ष्टाय॒ स्वाहा॒ स्वाहा॑ शि॒ष्टाय॑ । \newline
28. शि॒ष्टाय॒ स्वाहा॒ स्वाहा॑ शि॒ष्टाय॑ शि॒ष्टाय॒ स्वाहा᳚ । \newline
29. स्वाहा ऽति॑शिष्टा॒या ति॑शिष्टाय॒ स्वाहा॒ स्वाहा ऽति॑शिष्टाय । \newline
30. अति॑शिष्टाय॒ स्वाहा॒ स्वाहा ऽति॑शिष्टा॒ याति॑शिष्टाय॒ स्वाहा᳚ । \newline
31. अति॑शिष्टा॒येत्यति॑ - शि॒ष्टा॒य॒ । \newline
32. स्वाहा॒ परि॑शिष्टाय॒ परि॑शिष्टाय॒ स्वाहा॒ स्वाहा॒ परि॑शिष्टाय । \newline
33. परि॑शिष्टाय॒ स्वाहा॒ स्वाहा॒ परि॑शिष्टाय॒ परि॑शिष्टाय॒ स्वाहा᳚ । \newline
34. परि॑शिष्टा॒येति॒ परि॑ - शि॒ष्टा॒य॒ । \newline
35. स्वाहा॒ सꣳशि॑ष्टाय॒ सꣳशि॑ष्टाय॒ स्वाहा॒ स्वाहा॒ सꣳशि॑ष्टाय । \newline
36. सꣳशि॑ष्टाय॒ स्वाहा॒ स्वाहा॒ सꣳशि॑ष्टाय॒ सꣳशि॑ष्टाय॒ स्वाहा᳚ । \newline
37. सꣳशि॑ष्टा॒येति॒ सं - शि॒ष्टा॒य॒ । \newline
38. स्वाहोच्छि॑ष्टा॒ योच्छि॑ष्टाय॒ स्वाहा॒ स्वाहोच्छि॑ष्टाय । \newline
39. उच्छि॑ष्टाय॒ स्वाहा॒ स्वाहोच्छि॑ष्टा॒ योच्छि॑ष्टाय॒ स्वाहा᳚ । \newline
40. उच्छि॑ष्टा॒येत्युत् - शि॒ष्टा॒य॒ । \newline
41. स्वाहा॑ रि॒क्ताय॑ रि॒क्ताय॒ स्वाहा॒ स्वाहा॑ रि॒क्ताय॑ । \newline
42. रि॒क्ताय॒ स्वाहा॒ स्वाहा॑ रि॒क्ताय॑ रि॒क्ताय॒ स्वाहा᳚ । \newline
43. स्वाहा ऽरि॑क्ता॒या रि॑क्ताय॒ स्वाहा॒ स्वाहा ऽरि॑क्ताय । \newline
44. अरि॑क्ताय॒ स्वाहा॒ स्वाहा ऽरि॑क्ता॒या रि॑क्ताय॒ स्वाहा᳚ । \newline
45. स्वाहा॒ प्ररि॑क्ताय॒ प्ररि॑क्ताय॒ स्वाहा॒ स्वाहा॒ प्ररि॑क्ताय । \newline
46. प्ररि॑क्ताय॒ स्वाहा॒ स्वाहा॒ प्ररि॑क्ताय॒ प्ररि॑क्ताय॒ स्वाहा᳚ । \newline
47. प्ररि॑क्ता॒येति॒ प्र - रि॒क्ता॒य॒ । \newline
48. स्वाहा॒ सꣳरि॑क्ताय॒ सꣳरि॑क्ताय॒ स्वाहा॒ स्वाहा॒ सꣳरि॑क्ताय । \newline
49. सꣳरि॑क्ताय॒ स्वाहा॒ स्वाहा॒ सꣳरि॑क्ताय॒ सꣳरि॑क्ताय॒ स्वाहा᳚ । \newline
50. सꣳरि॑क्ता॒येति॒ सं - रि॒क्ता॒य॒ । \newline
51. स्वाहोद्रि॑क्ता॒ योद्रि॑क्ताय॒ स्वाहा॒ स्वाहोद्रि॑क्ताय । \newline
52. उद्रि॑क्ताय॒ स्वाहा॒ स्वाहोद्रि॑क्ता॒ योद्रि॑क्ताय॒ स्वाहा᳚ । \newline
53. उद्रि॑क्ता॒येत्युत् - रि॒क्ता॒य॒ । \newline
54. स्वाहा॒ सर्व॑स्मै॒ सर्व॑स्मै॒ स्वाहा॒ स्वाहा॒ सर्व॑स्मै । \newline
55. सर्व॑स्मै॒ स्वाहा॒ स्वाहा॒ सर्व॑स्मै॒ सर्व॑स्मै॒ स्वाहा᳚ । \newline
56. स्वाहेति॒ स्वाहा᳚ । \newline

\textbf{Ghana Paata } \newline

1. वन॒स्पति॑भ्यः॒ स्वाहा॒ स्वाहा॒ वन॒स्पति॑भ्यो॒ वन॒स्पति॑भ्यः॒ स्वाहा॒ मूले᳚भ्यो॒ मूले᳚भ्यः॒ स्वाहा॒ वन॒स्पति॑भ्यो॒ वन॒स्पति॑भ्यः॒ स्वाहा॒ मूले᳚भ्यः । \newline
2. वन॒स्पति॑भ्य॒ इति॒ वन॒स्पति॑ - भ्यः॒ । \newline
3. स्वाहा॒ मूले᳚भ्यो॒ मूले᳚भ्यः॒ स्वाहा॒ स्वाहा॒ मूले᳚भ्यः॒ स्वाहा॒ स्वाहा॒ मूले᳚भ्यः॒ स्वाहा॒ स्वाहा॒ मूले᳚भ्यः॒ स्वाहा᳚ । \newline
4. मूले᳚भ्यः॒ स्वाहा॒ स्वाहा॒ मूले᳚भ्यो॒ मूले᳚भ्यः॒ स्वाहा॒ तूले᳚भ्य॒ स्तूले᳚भ्यः॒ स्वाहा॒ मूले᳚भ्यो॒ मूले᳚भ्यः॒ स्वाहा॒ तूले᳚भ्यः । \newline
5. स्वाहा॒ तूले᳚भ्य॒ स्तूले᳚भ्यः॒ स्वाहा॒ स्वाहा॒ तूले᳚भ्यः॒ स्वाहा॒ स्वाहा॒ तूले᳚भ्यः॒ स्वाहा॒ स्वाहा॒ तूले᳚भ्यः॒ स्वाहा᳚ । \newline
6. तूले᳚भ्यः॒ स्वाहा॒ स्वाहा॒ तूले᳚भ्य॒ स्तूले᳚भ्यः॒ स्वाहा॒ स्कन्धो᳚भ्यः॒ स्कन्धो᳚भ्यः॒ स्वाहा॒ तूले᳚भ्य॒ स्तूले᳚भ्यः॒ स्वाहा॒ स्कन्धो᳚भ्यः । \newline
7. स्वाहा॒ स्कन्धो᳚भ्यः॒ स्कन्धो᳚भ्यः॒ स्वाहा॒ स्वाहा॒ स्कन्धो᳚भ्यः॒ स्वाहा॒ स्वाहा॒ स्कन्धो᳚भ्यः॒ स्वाहा॒ स्वाहा॒ स्कन्धो᳚भ्यः॒ स्वाहा᳚ । \newline
8. स्कन्धो᳚भ्यः॒ स्वाहा॒ स्वाहा॒ स्कन्धो᳚भ्यः॒ स्कन्धो᳚भ्यः॒ स्वाहा॒ शाखा᳚भ्यः॒ शाखा᳚भ्यः॒ स्वाहा॒ स्कन्धो᳚भ्यः॒ स्कन्धो᳚भ्यः॒ स्वाहा॒ शाखा᳚भ्यः । \newline
9. स्कन्धो᳚भ्य॒ इति॒ स्कन्धः॑ - भ्यः॒ । \newline
10. स्वाहा॒ शाखा᳚भ्यः॒ शाखा᳚भ्यः॒ स्वाहा॒ स्वाहा॒ शाखा᳚भ्यः॒ स्वाहा॒ स्वाहा॒ शाखा᳚भ्यः॒ स्वाहा॒ स्वाहा॒ शाखा᳚भ्यः॒ स्वाहा᳚ । \newline
11. शाखा᳚भ्यः॒ स्वाहा॒ स्वाहा॒ शाखा᳚भ्यः॒ शाखा᳚भ्यः॒ स्वाहा॑ प॒र्णेभ्यः॑ प॒र्णेभ्यः॒ स्वाहा॒ शाखा᳚भ्यः॒ शाखा᳚भ्यः॒ स्वाहा॑ प॒र्णेभ्यः॑ । \newline
12. स्वाहा॑ प॒र्णेभ्यः॑ प॒र्णेभ्यः॒ स्वाहा॒ स्वाहा॑ प॒र्णेभ्यः॒ स्वाहा॒ स्वाहा॑ प॒र्णेभ्यः॒ स्वाहा॒ स्वाहा॑ प॒र्णेभ्यः॒ स्वाहा᳚ । \newline
13. प॒र्णेभ्यः॒ स्वाहा॒ स्वाहा॑ प॒र्णेभ्यः॑ प॒र्णेभ्यः॒ स्वाहा॒ पुष्पे᳚भ्यः॒ पुष्पे᳚भ्यः॒ स्वाहा॑ प॒र्णेभ्यः॑ प॒र्णेभ्यः॒ स्वाहा॒ पुष्पे᳚भ्यः । \newline
14. स्वाहा॒ पुष्पे᳚भ्यः॒ पुष्पे᳚भ्यः॒ स्वाहा॒ स्वाहा॒ पुष्पे᳚भ्यः॒ स्वाहा॒ स्वाहा॒ पुष्पे᳚भ्यः॒ स्वाहा॒ स्वाहा॒ पुष्पे᳚भ्यः॒ स्वाहा᳚ । \newline
15. पुष्पे᳚भ्यः॒ स्वाहा॒ स्वाहा॒ पुष्पे᳚भ्यः॒ पुष्पे᳚भ्यः॒ स्वाहा॒ फले᳚भ्यः॒ फले᳚भ्यः॒ स्वाहा॒ पुष्पे᳚भ्यः॒ पुष्पे᳚भ्यः॒ स्वाहा॒ फले᳚भ्यः । \newline
16. स्वाहा॒ फले᳚भ्यः॒ फले᳚भ्यः॒ स्वाहा॒ स्वाहा॒ फले᳚भ्यः॒ स्वाहा॒ स्वाहा॒ फले᳚भ्यः॒ स्वाहा॒ स्वाहा॒ फले᳚भ्यः॒ स्वाहा᳚ । \newline
17. फले᳚भ्यः॒ स्वाहा॒ स्वाहा॒ फले᳚भ्यः॒ फले᳚भ्यः॒ स्वाहा॑ गृही॒तेभ्यो॑ गृही॒तेभ्यः॒ स्वाहा॒ फले᳚भ्यः॒ फले᳚भ्यः॒ स्वाहा॑ गृही॒तेभ्यः॑ । \newline
18. स्वाहा॑ गृही॒तेभ्यो॑ गृही॒तेभ्यः॒ स्वाहा॒ स्वाहा॑ गृही॒तेभ्यः॒ स्वाहा॒ स्वाहा॑ गृही॒तेभ्यः॒ स्वाहा॒ स्वाहा॑ गृही॒तेभ्यः॒ स्वाहा᳚ । \newline
19. गृ॒ही॒तेभ्यः॒ स्वाहा॒ स्वाहा॑ गृही॒तेभ्यो॑ गृही॒तेभ्यः॒ स्वाहा ऽगृ॑हीते॒भ्यो ऽगृ॑हीतेभ्यः॒ स्वाहा॑ गृही॒तेभ्यो॑ गृही॒तेभ्यः॒ स्वाहा ऽगृ॑हीतेभ्यः । \newline
20. स्वाहा ऽगृ॑हीते॒भ्यो ऽगृ॑हीतेभ्यः॒ स्वाहा॒ स्वाहा ऽगृ॑हीतेभ्यः॒ स्वाहा॒ स्वाहा ऽगृ॑हीतेभ्यः॒ स्वाहा॒ स्वाहा ऽगृ॑हीतेभ्यः॒ स्वाहा᳚ । \newline
21. अगृ॑हीतेभ्यः॒ स्वाहा॒ स्वाहा ऽगृ॑हीते॒भ्यो ऽगृ॑हीतेभ्यः॒ स्वाहा ऽव॑पन्ने॒भ्यो ऽव॑पन्नेभ्यः॒ स्वाहा ऽगृ॑हीते॒भ्यो ऽगृ॑हीतेभ्यः॒ स्वाहा ऽव॑पन्नेभ्यः । \newline
22. स्वाहा ऽव॑पन्ने॒भ्यो ऽव॑पन्नेभ्यः॒ स्वाहा॒ स्वाहा ऽव॑पन्नेभ्यः॒ स्वाहा॒ स्वाहा ऽव॑पन्नेभ्यः॒ स्वाहा॒ स्वाहा ऽव॑पन्नेभ्यः॒ स्वाहा᳚ । \newline
23. अव॑पन्नेभ्यः॒ स्वाहा॒ स्वाहा ऽव॑पन्ने॒भ्यो ऽव॑पन्नेभ्यः॒ स्वाहा॒ शया॑नेभ्यः॒ शया॑नेभ्यः॒ स्वाहा ऽव॑पन्ने॒भ्यो ऽव॑पन्नेभ्यः॒ स्वाहा॒ शया॑नेभ्यः । \newline
24. अव॑पन्नेभ्य॒ इत्यव॑ - प॒न्ने॒भ्यः॒ । \newline
25. स्वाहा॒ शया॑नेभ्यः॒ शया॑नेभ्यः॒ स्वाहा॒ स्वाहा॒ शया॑नेभ्यः॒ स्वाहा॒ स्वाहा॒ शया॑नेभ्यः॒ स्वाहा॒ स्वाहा॒ शया॑नेभ्यः॒ स्वाहा᳚ । \newline
26. शया॑नेभ्यः॒ स्वाहा॒ स्वाहा॒ शया॑नेभ्यः॒ शया॑नेभ्यः॒ स्वाहा॑ शि॒ष्टाय॑ शि॒ष्टाय॒ स्वाहा॒ शया॑नेभ्यः॒ शया॑नेभ्यः॒ स्वाहा॑ शि॒ष्टाय॑ । \newline
27. स्वाहा॑ शि॒ष्टाय॑ शि॒ष्टाय॒ स्वाहा॒ स्वाहा॑ शि॒ष्टाय॒ स्वाहा॒ स्वाहा॑ शि॒ष्टाय॒ स्वाहा॒ स्वाहा॑ शि॒ष्टाय॒ स्वाहा᳚ । \newline
28. शि॒ष्टाय॒ स्वाहा॒ स्वाहा॑ शि॒ष्टाय॑ शि॒ष्टाय॒ स्वाहा ऽति॑शिष्टा॒या ति॑शिष्टाय॒ स्वाहा॑ शि॒ष्टाय॑ शि॒ष्टाय॒ स्वाहा ऽति॑शिष्टाय । \newline
29. स्वाहा ऽति॑शिष्टा॒या ति॑शिष्टाय॒ स्वाहा॒ स्वाहा ऽति॑शिष्टाय॒ स्वाहा॒ स्वाहा ऽति॑शिष्टाय॒ स्वाहा॒ स्वाहा ऽति॑शिष्टाय॒ स्वाहा᳚ । \newline
30. अति॑शिष्टाय॒ स्वाहा॒ स्वाहा ऽति॑शिष्टा॒या ति॑शिष्टाय॒ स्वाहा॒ परि॑शिष्टाय॒ परि॑शिष्टाय॒ स्वाहा ऽति॑शिष्टा॒या ति॑शिष्टाय॒ स्वाहा॒ परि॑शिष्टाय । \newline
31. अति॑शिष्टा॒येत्यति॑ - शि॒ष्टा॒य॒ । \newline
32. स्वाहा॒ परि॑शिष्टाय॒ परि॑शिष्टाय॒ स्वाहा॒ स्वाहा॒ परि॑शिष्टाय॒ स्वाहा॒ स्वाहा॒ परि॑शिष्टाय॒ स्वाहा॒ स्वाहा॒ परि॑शिष्टाय॒ स्वाहा᳚ । \newline
33. परि॑शिष्टाय॒ स्वाहा॒ स्वाहा॒ परि॑शिष्टाय॒ परि॑शिष्टाय॒ स्वाहा॒ सꣳशि॑ष्टाय॒ सꣳशि॑ष्टाय॒ स्वाहा॒ परि॑शिष्टाय॒ परि॑शिष्टाय॒ स्वाहा॒ सꣳशि॑ष्टाय । \newline
34. परि॑शिष्टा॒येति॒ परि॑ - शि॒ष्टा॒य॒ । \newline
35. स्वाहा॒ सꣳशि॑ष्टाय॒ सꣳशि॑ष्टाय॒ स्वाहा॒ स्वाहा॒ सꣳशि॑ष्टाय॒ स्वाहा॒ स्वाहा॒ सꣳशि॑ष्टाय॒ स्वाहा॒ स्वाहा॒ सꣳशि॑ष्टाय॒ स्वाहा᳚ । \newline
36. सꣳशि॑ष्टाय॒ स्वाहा॒ स्वाहा॒ सꣳशि॑ष्टाय॒ सꣳशि॑ष्टाय॒ स्वाहोच्छि॑ष्टा॒ योच्छि॑ष्टाय॒ स्वाहा॒ सꣳशि॑ष्टाय॒ सꣳशि॑ष्टाय॒ स्वाहोच्छि॑ष्टाय । \newline
37. सꣳशि॑ष्टा॒येति॒ सं - शि॒ष्टा॒य॒ । \newline
38. स्वाहोच्छि॑ष्टा॒ योच्छि॑ष्टाय॒ स्वाहा॒ स्वाहोच्छि॑ष्टाय॒ स्वाहा॒ स्वाहोच्छि॑ष्टाय॒ स्वाहा॒ स्वाहोच्छि॑ष्टाय॒ स्वाहा᳚ । \newline
39. उच्छि॑ष्टाय॒ स्वाहा॒ स्वाहोच्छि॑ष्टा॒ योच्छि॑ष्टाय॒ स्वाहा॑ रि॒क्ताय॑ रि॒क्ताय॒ स्वाहोच्छि॑ष्टा॒ योच्छि॑ष्टाय॒ स्वाहा॑ रि॒क्ताय॑ । \newline
40. उच्छि॑ष्टा॒येत्युत् - शि॒ष्टा॒य॒ । \newline
41. स्वाहा॑ रि॒क्ताय॑ रि॒क्ताय॒ स्वाहा॒ स्वाहा॑ रि॒क्ताय॒ स्वाहा॒ स्वाहा॑ रि॒क्ताय॒ स्वाहा॒ स्वाहा॑ रि॒क्ताय॒ स्वाहा᳚ । \newline
42. रि॒क्ताय॒ स्वाहा॒ स्वाहा॑ रि॒क्ताय॑ रि॒क्ताय॒ स्वाहा ऽरि॑क्ता॒या रि॑क्ताय॒ स्वाहा॑ रि॒क्ताय॑ रि॒क्ताय॒ स्वाहा ऽरि॑क्ताय । \newline
43. स्वाहा ऽरि॑क्ता॒या रि॑क्ताय॒ स्वाहा॒ स्वाहा ऽरि॑क्ताय॒ स्वाहा॒ स्वाहा ऽरि॑क्ताय॒ स्वाहा॒ स्वाहा ऽरि॑क्ताय॒ स्वाहा᳚ । \newline
44. अरि॑क्ताय॒ स्वाहा॒ स्वाहा ऽरि॑क्ता॒या रि॑क्ताय॒ स्वाहा॒ प्ररि॑क्ताय॒ प्ररि॑क्ताय॒ स्वाहा ऽरि॑क्ता॒या रि॑क्ताय॒ स्वाहा॒ प्ररि॑क्ताय । \newline
45. स्वाहा॒ प्ररि॑क्ताय॒ प्ररि॑क्ताय॒ स्वाहा॒ स्वाहा॒ प्ररि॑क्ताय॒ स्वाहा॒ स्वाहा॒ प्ररि॑क्ताय॒ स्वाहा॒ स्वाहा॒ प्ररि॑क्ताय॒ स्वाहा᳚ । \newline
46. प्ररि॑क्ताय॒ स्वाहा॒ स्वाहा॒ प्ररि॑क्ताय॒ प्ररि॑क्ताय॒ स्वाहा॒ सꣳरि॑क्ताय॒ सꣳरि॑क्ताय॒ स्वाहा॒ प्ररि॑क्ताय॒ प्ररि॑क्ताय॒ स्वाहा॒ सꣳरि॑क्ताय । \newline
47. प्ररि॑क्ता॒येति॒ प्र - रि॒क्ता॒य॒ । \newline
48. स्वाहा॒ सꣳरि॑क्ताय॒ सꣳरि॑क्ताय॒ स्वाहा॒ स्वाहा॒ सꣳरि॑क्ताय॒ स्वाहा॒ स्वाहा॒ सꣳरि॑क्ताय॒ स्वाहा॒ स्वाहा॒ सꣳरि॑क्ताय॒ स्वाहा᳚ । \newline
49. सꣳरि॑क्ताय॒ स्वाहा॒ स्वाहा॒ सꣳरि॑क्ताय॒ सꣳरि॑क्ताय॒ स्वाहोद्रि॑क्ता॒ योद्रि॑क्ताय॒ स्वाहा॒ सꣳरि॑क्ताय॒ सꣳरि॑क्ताय॒ स्वाहोद्रि॑क्ताय । \newline
50. सꣳरि॑क्ता॒येति॒ सं - रि॒क्ता॒य॒ । \newline
51. स्वाहोद्रि॑क्ता॒ योद्रि॑क्ताय॒ स्वाहा॒ स्वाहोद्रि॑क्ताय॒ स्वाहा॒ स्वाहोद्रि॑क्ताय॒ स्वाहा॒ स्वाहोद्रि॑क्ताय॒ स्वाहा᳚ । \newline
52. उद्रि॑क्ताय॒ स्वाहा॒ स्वाहोद्रि॑क्ता॒ योद्रि॑क्ताय॒ स्वाहा॒ सर्व॑स्मै॒ सर्व॑स्मै॒ स्वाहोद्रि॑क्ता॒ योद्रि॑क्ताय॒ स्वाहा॒ सर्व॑स्मै । \newline
53. उद्रि॑क्ता॒येत्युत् - रि॒क्ता॒य॒ । \newline
54. स्वाहा॒ सर्व॑स्मै॒ सर्व॑स्मै॒ स्वाहा॒ स्वाहा॒ सर्व॑स्मै॒ स्वाहा॒ स्वाहा॒ सर्व॑स्मै॒ स्वाहा॒ स्वाहा॒ सर्व॑स्मै॒ स्वाहा᳚ । \newline
55. सर्व॑स्मै॒ स्वाहा॒ स्वाहा॒ सर्व॑स्मै॒ सर्व॑स्मै॒ स्वाहा᳚ । \newline
56. स्वाहेति॒ स्वाहा᳚ । \newline
\pagebreak


\end{document}