\documentclass[17pt]{extarticle}
\usepackage{babel}
\usepackage{fontspec}
\usepackage{polyglossia}
\usepackage{extsizes}

\usepackage{color}   %May be necessary if you want to color links
\usepackage{hyperref}
\hypersetup{
    colorlinks=true, %set true if you want colored links
    linktoc=all,     %set to all if you want both sections and subsections linked
    linkcolor=black,  %choose some color if you want links to stand out
}

\setmainlanguage{sanskrit}
\setotherlanguages{english} %% or other languages
\setlength{\parindent}{0pt}
\pagestyle{myheadings}
\newfontfamily\devanagarifont[Script=Devanagari]{AdishilaVedic}
\renewcommand{\theHsection}{\thepart.section.\thesection}

\newcommand{\VAR}[1]{}
\newcommand{\BLOCK}[1]{}




\begin{document}
\begin{titlepage}
    \begin{center}
 
\begin{sanskrit}
    { \Large
    कृष्ण यजुर्वेदीय तैत्तिरीय संहिता,पद,जटा,घन पाठः 
    }
    \\
    \vspace{2.5cm}
    \mbox{ \Large
    5.1      पञ्चमकाण्डे प्रथमः प्रश्नः - उख्याग्निकथनं   }
\end{sanskrit}
\end{center}

\end{titlepage}
\tableofcontents
\phantomsection
\pagebreak

\markright{ TS 5.1.1.1  \hfill https://www.vedavms.in \hfill}

\section{ TS 5.1.1.1 }

\textbf{TS 5.1.1.1 } \newline
\textbf{Samhita Paata} \newline

सा॒वि॒त्राणि॑ जुहोति॒ प्रसू᳚त्यै चतुर्गृही॒तेन॑ जुहोति॒ चतु॑ष्पादः प॒शवः॑ प॒शूने॒वाऽव॑ रुन्धे॒ चत॑स्रो॒ दिशो॑ दि॒क्ष्वे॑व प्रति॑ तिष्ठति॒ छन्दाꣳ॑सि दे॒वेभ्यो ऽपा᳚ऽक्राम॒न् न वो॑ भा॒गानि॑ ह॒व्यं ॅव॑क्ष्याम॒ इति॒ तेभ्य॑ ए॒तच्च॑तु-र्गृही॒तम॑-धारयन् पुरोऽनु वा॒क्या॑यै या॒ज्या॑यै दे॒वता॑यै वषट्का॒राय॒ यच्च॑तुर्गृही॒तं जु॒होति॒ छन्दाꣳ॑स्ये॒व तत् प्री॑णाति॒ तान्य॑स्य प्री॒तानि॑ दे॒वेभ्यो॑ ह॒व्यं ॅव॑हन्ति॒ यं का॒मये॑त॒ - [  ] \newline

\textbf{Pada Paata} \newline

सा॒वि॒त्राणि॑ । जु॒हो॒ति॒ । प्रसू᳚त्या॒ इति॒ प्र - सू॒त्यै॒ । च॒तु॒र्गृ॒ही॒तेनेति॑ चतुः - गृ॒ही॒तेन॑ । जु॒हो॒ति॒ । चतु॑ष्पाद॒ इति॒ चतुः॑-पा॒दः॒ । प॒शवः॑ । प॒शून् । ए॒व । अवेति॑ । रु॒न्धे॒ । चत॑स्रः । दिशः॑ । दि॒क्षु । ए॒व । प्रतीति॑ । ति॒ष्ठ॒ति॒ । छन्दाꣳ॑सि । दे॒वेभ्यः॑ । अपेति॑ । अ॒क्रा॒म॒न्न् । न । वः॒ । अ॒भा॒गानि॑ । ह॒व्यम् । व॒क्ष्या॒मः॒ । इति॑ । तेभ्यः॑ । ए॒तत् । च॒तु॒र्गृ॒ही॒तमिति॑ चतुः - गृ॒ही॒तम् । अ॒धा॒र॒य॒न्न् । पु॒रो॒नु॒वा॒क्या॑या॒ इति॑ पुरः - अ॒नु॒वा॒क्या॑यै । या॒ज्या॑यै । दे॒वता॑यै । व॒ष॒ट्का॒रायेति॑ वषट् - का॒राय॑ । यत् । च॒तु॒र्गृ॒ही॒तमिति॑ चतुः - गृ॒ही॒तम् । जु॒होति॑ । छन्दाꣳ॑सि । ए॒व । तत् । प्री॒णा॒ति॒ । तामि॑ । अ॒स्य॒ । प्री॒तानि॑ । दे॒वेभ्यः॑ । ह॒व्यम् । व॒ह॒न्ति॒ । यम् । का॒मये॑त ।  \newline


\textbf{Krama Paata} \newline

सा॒वि॒त्राणि॑ जुहोति । जु॒हो॒ति॒ प्रसू᳚त्यै । प्रसू᳚त्यै चतुर्गृही॒तेन॑ । प्रसू᳚त्या॒ इति॒ प्र - सू॒त्यै॒ । च॒तु॒र्गृ॒ही॒तेन॑ जुहोति । च॒तु॒गृ॒ही॒तेनेति॑ चतुः - गृ॒ही॒तेन॑ । जु॒हो॒ति॒ चतु॑ष्पादः । चतु॑ष्पादः प॒शवः॑ । चतु॑ष्पाद॒ इति॒ चतुः॑ - पा॒दः॒ । प॒शवः॑ प॒शून् । प॒शूने॒व । ए॒वाव॑ । अव॑ रुन्धे । रु॒न्धे॒ चत॑स्रः । चत॑स्रो॒ दिशः॑ । दिशो॑ दि॒क्षु । दि॒क्ष्वे॑व । ए॒व प्रति॑ । प्रति॑ तिष्ठति । ति॒ष्ठ॒ति॒ छन्दाꣳ॑सि । छन्दाꣳ॑सि दे॒वेभ्यः॑ । दे॒वेभ्योऽप॑ । अपा᳚क्रामन्न् । अ॒क्रा॒म॒न् न । न वः॑ । वो॒ऽभा॒गानि॑ । अ॒भा॒गानि॑ ह॒व्यम् । ह॒व्यम् ॅव॑क्ष्यामः । व॒क्ष्या॒म॒ इति॑ । इति॒ तेभ्यः॑ । तेभ्य॑ ए॒तत् । ए॒तच् च॑तुर्गृही॒तम् । च॒तु॒र्गृ॒ही॒तम॑धारयन्न् । च॒तु॒र्गृ॒ही॒तमिति॑ चतुः - गृ॒ही॒तम् । अ॒धा॒र॒य॒न् पु॒रो॒नु॒वा॒क्या॑यै । पु॒रो॒नु॒वा॒क्या॑यै या॒ज्या॑यै । पु॒रो॒नु॒वा॒क्या॑या॒ इति॑ पुरः - अ॒नु॒वा॒क्या॑यै । या॒ज्या॑यै दे॒वता॑यै । दे॒वता॑यै वषट्का॒राय॑ । व॒ष॒ट्का॒राय॒ यत् । व॒ष॒ट्का॒रायेति॑ वषट् - का॒राय॑ । यच् च॑तुर्गृही॒तम् । च॒तु॒र्गृ॒ही॒तम् जु॒होति॑ । च॒तु॒र्गृ॒ही॒तमिति॑ चतुः - गृ॒ही॒तम् । जु॒होति॒ छन्दाꣳ॑सि । छन्दाꣳ॑स्ये॒व । ए॒व तत् । तत् प्री॑णाति । प्री॒णा॒ति॒ तानि॑ । तान्य॑स्य । अ॒स्य॒ प्री॒तानि॑ । प्री॒तानि॑ दे॒वेभ्यः॑ । दे॒वेभ्यो॑ ह॒व्यम् । ह॒व्यम् ॅव॑हन्ति । व॒ह॒न्ति॒ यम् । यम् का॒मये॑त । का॒मये॑त॒ पापी॑यान् \newline

\textbf{Jatai Paata} \newline

1. सा॒वि॒त्राणि॑ जुहोति जुहोति सावि॒त्राणि॑ सावि॒त्राणि॑ जुहोति । \newline
2. जु॒हो॒ति॒ प्रसू᳚त्यै॒ प्रसू᳚त्यै जुहोति जुहोति॒ प्रसू᳚त्यै । \newline
3. प्रसू᳚त्यै चतुर्गृही॒तेन॑ चतुर्गृही॒तेन॒ प्रसू᳚त्यै॒ प्रसू᳚त्यै चतुर्गृही॒तेन॑ । \newline
4. प्रसू᳚त्या॒ इति॒ प्र - सू॒त्यै॒ । \newline
5. च॒तु॒र्गृ॒ही॒तेन॑ जुहोति जुहोति चतुर्गृही॒तेन॑ चतुर्गृही॒तेन॑ जुहोति । \newline
6. च॒तु॒र्गृ॒ही॒तेनेति॑ चतुः - गृ॒ही॒तेन॑ । \newline
7. जु॒हो॒ति॒ चतु॑ष्पाद॒ श्चतु॑ष्पादो जुहोति जुहोति॒ चतु॑ष्पादः । \newline
8. चतु॑ष्पादः प॒शवः॑ प॒शव॒ श्चतु॑ष्पाद॒ श्चतु॑ष्पादः प॒शवः॑ । \newline
9. चतु॑ष्पाद॒ इति॒ चतुः॑ - पा॒दः॒ । \newline
10. प॒शवः॑ प॒शून् प॒शून् प॒शवः॑ प॒शवः॑ प॒शून् । \newline
11. प॒शू ने॒वैव प॒शून् प॒शू ने॒व । \newline
12. ए॒वावा वै॒वै वाव॑ । \newline
13. अव॑ रुन्धे रु॒न्धे ऽवाव॑ रुन्धे । \newline
14. रु॒न्धे॒ चत॑स्र॒ श्चत॑स्रो रुन्धे रुन्धे॒ चत॑स्रः । \newline
15. चत॑स्रो॒ दिशो॒ दिश॒ श्चत॑स्र॒ श्चत॑स्रो॒ दिशः॑ । \newline
16. दिशो॑ दि॒क्षु दि॒क्षु दिशो॒ दिशो॑ दि॒क्षु । \newline
17. दि॒क्ष्वे॑वैव दि॒क्षु दि॒क्ष्वे॑व । \newline
18. ए॒व प्रति॒ प्रत्ये॒वैव प्रति॑ । \newline
19. प्रति॑ तिष्ठति तिष्ठति॒ प्रति॒ प्रति॑ तिष्ठति । \newline
20. ति॒ष्ठ॒ति॒ छन्दाꣳ॑सि॒ छन्दाꣳ॑सि तिष्ठति तिष्ठति॒ छन्दाꣳ॑सि । \newline
21. छन्दाꣳ॑सि दे॒वेभ्यो॑ दे॒वेभ्य॒ श्छन्दाꣳ॑सि॒ छन्दाꣳ॑सि दे॒वेभ्यः॑ । \newline
22. दे॒वेभ्यो ऽपाप॑ दे॒वेभ्यो॑ दे॒वेभ्यो ऽप॑ । \newline
23. अपा᳚क्रामन् नक्राम॒न् नपापा᳚ क्रामन्न् । \newline
24. अ॒क्रा॒म॒न् न नाक्रा॑मन् नक्राम॒न् न । \newline
25. न वो॑ वो॒ न न वः॑ । \newline
26. वो॒ ऽभा॒गा न्य॑भा॒गानि॑ वो वो ऽभा॒गानि॑ । \newline
27. अ॒भा॒गानि॑ ह॒व्यꣳ ह॒व्य म॑भा॒गा न्य॑भा॒गानि॑ ह॒व्यम् । \newline
28. ह॒व्यं ॅव॑क्ष्यामो वक्ष्यामो ह॒व्यꣳ ह॒व्यं ॅव॑क्ष्यामः । \newline
29. व॒क्ष्या॒म॒ इतीति॑ वक्ष्यामो वक्ष्याम॒ इति॑ । \newline
30. इति॒ तेभ्य॒ स्तेभ्य॒ इतीति॒ तेभ्यः॑ । \newline
31. तेभ्य॑ ए॒त दे॒तत् तेभ्य॒ स्तेभ्य॑ ए॒तत् । \newline
32. ए॒तच् च॑तुर्गृही॒तम् च॑तुर्गृही॒त मे॒तदे॒तच् च॑तुर्गृही॒तम् । \newline
33. च॒तु॒र्गृ॒ही॒त म॑धारयन् नधारयꣳ श्चतुर्गृही॒तम् च॑तुर्गृही॒त म॑धारयन्न् । \newline
34. च॒तु॒र्गृ॒ही॒तमिति॑ चतुः - गृ॒ही॒तम् । \newline
35. अ॒धा॒र॒य॒न् पु॒रो॒नु॒वा॒क्या॑यै पुरोनुवा॒क्या॑या अधारयन् नधारयन् पुरोनुवा॒क्या॑यै । \newline
36. पु॒रो॒नु॒वा॒क्या॑यै या॒ज्या॑यै या॒ज्या॑यै पुरोनुवा॒क्या॑यै पुरोनुवा॒क्या॑यै या॒ज्या॑यै । \newline
37. पु॒रो॒नु॒वा॒क्या॑या॒ इति॑ पुरः - अ॒नु॒वा॒क्या॑यै । \newline
38. या॒ज्या॑यै दे॒वता॑यै दे॒वता॑यै या॒ज्या॑यै या॒ज्या॑यै दे॒वता॑यै । \newline
39. दे॒वता॑यै वषट्का॒राय॑ वषट्का॒राय॑ दे॒वता॑यै दे॒वता॑यै वषट्का॒राय॑ । \newline
40. व॒ष॒ट्का॒राय॒ यद् यद् व॑षट्का॒राय॑ वषट्का॒राय॒ यत् । \newline
41. व॒ष॒ट्का॒रायेति॑ वषट् - का॒राय॑ । \newline
42. यच् च॑तुर्गृही॒तम् च॑तुर्गृही॒तं ॅयद् यच् च॑तुर्गृही॒तम् । \newline
43. च॒तु॒र्गृ॒ही॒तम् जु॒होति॑ जु॒होति॑ चतुर्गृही॒तम् च॑तुर्गृही॒तम् जु॒होति॑ । \newline
44. च॒तु॒र्गृ॒ही॒तमिति॑ चतुः - गृ॒ही॒तम् । \newline
45. जु॒होति॒ छन्दाꣳ॑सि॒ छन्दाꣳ॑सि जु॒होति॑ जु॒होति॒ छन्दाꣳ॑सि । \newline
46. छन्दाꣳ॑ स्ये॒वैव छन्दाꣳ॑सि॒ छन्दाꣳ॑ स्ये॒व । \newline
47. ए॒व तत् तदे॒वैव तत् । \newline
48. तत् प्री॑णाति प्रीणाति॒ तत् तत् प्री॑णाति । \newline
49. प्री॒णा॒ति॒ तानि॒ तानि॑ प्रीणाति प्रीणाति॒ तानि॑ । \newline
50. तान्य॑ स्यास्य॒ तानि॒ तान्य॑स्य । \newline
51. अ॒स्य॒ प्री॒तानि॑ प्री॒ता न्य॑स्यास्य प्री॒तानि॑ । \newline
52. प्री॒तानि॑ दे॒वेभ्यो॑ दे॒वेभ्यः॑ प्री॒तानि॑ प्री॒तानि॑ दे॒वेभ्यः॑ । \newline
53. दे॒वेभ्यो॑ ह॒व्यꣳ ह॒व्यम् दे॒वेभ्यो॑ दे॒वेभ्यो॑ ह॒व्यम् । \newline
54. ह॒व्यं ॅव॑हन्ति वहन्ति ह॒व्यꣳ ह॒व्यं ॅव॑हन्ति । \newline
55. व॒ह॒न्ति॒ यं ॅयं ॅव॑हन्ति वहन्ति॒ यम् । \newline
56. यम् का॒मये॑त का॒मये॑त॒ यं ॅयम् का॒मये॑त । \newline
57. का॒मये॑त॒ पापी॑या॒न् पापी॑यान् का॒मये॑त का॒मये॑त॒ पापी॑यान् । \newline

\textbf{Ghana Paata } \newline

1. सा॒वि॒त्राणि॑ जुहोति जुहोति सावि॒त्राणि॑ सावि॒त्राणि॑ जुहोति॒ प्रसू᳚त्यै॒ प्रसू᳚त्यै जुहोति सावि॒त्राणि॑ सावि॒त्राणि॑ जुहोति॒ प्रसू᳚त्यै । \newline
2. जु॒हो॒ति॒ प्रसू᳚त्यै॒ प्रसू᳚त्यै जुहोति जुहोति॒ प्रसू᳚त्यै चतुर्गृही॒तेन॑ चतुर्गृही॒तेन॒ प्रसू᳚त्यै जुहोति जुहोति॒ प्रसू᳚त्यै चतुर्गृही॒तेन॑ । \newline
3. प्रसू᳚त्यै चतुर्गृही॒तेन॑ चतुर्गृही॒तेन॒ प्रसू᳚त्यै॒ प्रसू᳚त्यै चतुर्गृही॒तेन॑ जुहोति जुहोति चतुर्गृही॒तेन॒ प्रसू᳚त्यै॒ प्रसू᳚त्यै चतुर्गृही॒तेन॑ जुहोति । \newline
4. प्रसू᳚त्या॒ इति॒ प्र - सू॒त्यै॒ । \newline
5. च॒तु॒र्गृ॒ही॒तेन॑ जुहोति जुहोति चतुर्गृही॒तेन॑ चतुर्गृही॒तेन॑ जुहोति॒ चतु॑ष्पाद॒ श्चतु॑ष्पादो जुहोति चतुर्गृही॒तेन॑ चतुर्गृही॒तेन॑ जुहोति॒ चतु॑ष्पादः । \newline
6. च॒तु॒र्गृ॒ही॒तेनेति॑ चतुः - गृ॒ही॒तेन॑ । \newline
7. जु॒हो॒ति॒ चतु॑ष्पाद॒ श्चतु॑ष्पादो जुहोति जुहोति॒ चतु॑ष्पादः प॒शवः॑ प॒शव॒ श्चतु॑ष्पादो जुहोति जुहोति॒ चतु॑ष्पादः प॒शवः॑ । \newline
8. चतु॑ष्पादः प॒शवः॑ प॒शव॒ श्चतु॑ष्पाद॒ श्चतु॑ष्पादः प॒शवः॑ प॒शून् प॒शून् प॒शव॒ श्चतु॑ष्पाद॒ श्चतु॑ष्पादः प॒शवः॑ प॒शून् । \newline
9. चतु॑ष्पाद॒ इति॒ चतुः॑ - पा॒दः॒ । \newline
10. प॒शवः॑ प॒शून् प॒शून् प॒शवः॑ प॒शवः॑ प॒शू ने॒वैव प॒शून् प॒शवः॑ प॒शवः॑ प॒शू ने॒व । \newline
11. प॒शू ने॒वैव प॒शून् प॒शू ने॒वावा वै॒व प॒शून् प॒शू ने॒वाव॑ । \newline
12. ए॒वावा वै॒वै वाव॑ रुन्धे रु॒न्धे ऽवै॒वै वाव॑ रुन्धे । \newline
13. अव॑ रुन्धे रु॒न्धे ऽवाव॑ रुन्धे॒ चत॑स्र॒ श्चत॑स्रो रु॒न्धे ऽवाव॑ रुन्धे॒ चत॑स्रः । \newline
14. रु॒न्धे॒ चत॑स्र॒ श्चत॑स्रो रुन्धे रुन्धे॒ चत॑स्रो॒ दिशो॒ दिश॒ श्चत॑स्रो रुन्धे रुन्धे॒ चत॑स्रो॒ दिशः॑ । \newline
15. चत॑स्रो॒ दिशो॒ दिश॒ श्चत॑स्र॒ श्चत॑स्रो॒ दिशो॑ दि॒क्षु दि॒क्षु दिश॒ श्चत॑स्र॒ श्चत॑स्रो॒ दिशो॑ दि॒क्षु । \newline
16. दिशो॑ दि॒क्षु दि॒क्षु दिशो॒ दिशो॑ दि॒क्ष्वे॑वैव दि॒क्षु दिशो॒ दिशो॑ दि॒क्ष्वे॑व । \newline
17. दि॒क्ष्वे॑वैव दि॒क्षु दि॒क्ष्वे॑व प्रति॒ प्रत्ये॒व दि॒क्षु दि॒क्ष्वे॑व प्रति॑ । \newline
18. ए॒व प्रति॒ प्रत्ये॒वैव प्रति॑ तिष्ठति तिष्ठति॒ प्रत्ये॒वैव प्रति॑ तिष्ठति । \newline
19. प्रति॑ तिष्ठति तिष्ठति॒ प्रति॒ प्रति॑ तिष्ठति॒ छन्दाꣳ॑सि॒ छन्दाꣳ॑सि तिष्ठति॒ प्रति॒ प्रति॑ तिष्ठति॒ छन्दाꣳ॑सि । \newline
20. ति॒ष्ठ॒ति॒ छन्दाꣳ॑सि॒ छन्दाꣳ॑सि तिष्ठति तिष्ठति॒ छन्दाꣳ॑सि दे॒वेभ्यो॑ दे॒वेभ्य॒ श्छन्दाꣳ॑सि तिष्ठति तिष्ठति॒ छन्दाꣳ॑सि दे॒वेभ्यः॑ । \newline
21. छन्दाꣳ॑सि दे॒वेभ्यो॑ दे॒वेभ्य॒ श्छन्दाꣳ॑सि॒ छन्दाꣳ॑सि दे॒वेभ्यो ऽपाप॑ दे॒वेभ्य॒ श्छन्दाꣳ॑सि॒ छन्दाꣳ॑सि दे॒वेभ्यो ऽप॑ । \newline
22. दे॒वेभ्यो ऽपाप॑ दे॒वेभ्यो॑ दे॒वेभ्यो ऽपा᳚क्रामन् नक्राम॒न् नप॑ दे॒वेभ्यो॑ दे॒वेभ्यो ऽपा᳚क्रामन्न् । \newline
23. अपा᳚क्रामन् नक्राम॒न् नपापा᳚क्राम॒न् न नाक्रा॑म॒न् नपापा᳚क्राम॒न् न । \newline
24. अ॒क्रा॒म॒न् न नाक्रा॑मन् नक्राम॒न् न वो॑ वो॒ नाक्रा॑मन् नक्राम॒न् न वः॑ । \newline
25. न वो॑ वो॒ न न वो॑ ऽभा॒गा न्य॑भा॒गानि॑ वो॒ न न वो॑ ऽभा॒गानि॑ । \newline
26. वो॒ ऽभा॒गा न्य॑भा॒गानि॑ वो वो ऽभा॒गानि॑ ह॒व्यꣳ ह॒व्य म॑भा॒गानि॑ वो वो ऽभा॒गानि॑ ह॒व्यम् । \newline
27. अ॒भा॒गानि॑ ह॒व्यꣳ ह॒व्य म॑भा॒गा न्य॑भा॒गानि॑ ह॒व्यं ॅव॑क्ष्यामो वक्ष्यामो ह॒व्य म॑भा॒गा न्य॑भा॒गानि॑ ह॒व्यं ॅव॑क्ष्यामः । \newline
28. ह॒व्यं ॅव॑क्ष्यामो वक्ष्यामो ह॒व्यꣳ ह॒व्यं ॅव॑क्ष्याम॒ इतीति॑ वक्ष्यामो ह॒व्यꣳ ह॒व्यं ॅव॑क्ष्याम॒ इति॑ । \newline
29. व॒क्ष्या॒म॒ इतीति॑ वक्ष्यामो वक्ष्याम॒ इति॒ तेभ्य॒ स्तेभ्य॒ इति॑ वक्ष्यामो वक्ष्याम॒ इति॒ तेभ्यः॑ । \newline
30. इति॒ तेभ्य॒ स्तेभ्य॒ इतीति॒ तेभ्य॑ ए॒त दे॒तत् तेभ्य॒ इतीति॒ तेभ्य॑ ए॒तत् । \newline
31. तेभ्य॑ ए॒त दे॒तत् तेभ्य॒ स्तेभ्य॑ ए॒तच् च॑तुर्गृही॒तम् च॑तुर्गृही॒त मे॒तत् तेभ्य॒ स्तेभ्य॑ ए॒तच् च॑तुर्गृही॒तम् । \newline
32. ए॒तच् च॑तुर्गृही॒तम् च॑तुर्गृही॒त मे॒त दे॒तच् च॑तुर्गृही॒त म॑धारयन् नधारयꣳ श्चतुर्गृही॒त मे॒त दे॒तच् च॑तुर्गृही॒त म॑धारयन्न् । \newline
33. च॒तु॒र्गृ॒ही॒त म॑धारयन् नधारयꣳ श्चतुर्गृही॒तम् च॑तुर्गृही॒त म॑धारयन् पुरोनुवा॒क्या॑यै पुरोनुवा॒क्या॑या अधारयꣳ श्चतुर्गृही॒तम् च॑तुर्गृही॒त म॑धारयन् पुरोनुवा॒क्या॑यै । \newline
34. च॒तु॒र्गृ॒ही॒तमिति॑ चतुः - गृ॒ही॒तम् । \newline
35. अ॒धा॒र॒य॒न् पु॒रो॒नु॒वा॒क्या॑यै पुरोनुवा॒क्या॑या अधारयन् नधारयन् पुरोनुवा॒क्या॑यै या॒ज्या॑यै या॒ज्या॑यै पुरोनुवा॒क्या॑या अधारयन् नधारयन् पुरोनुवा॒क्या॑यै या॒ज्या॑यै । \newline
36. पु॒रो॒नु॒वा॒क्या॑यै या॒ज्या॑यै या॒ज्या॑यै पुरोनुवा॒क्या॑यै पुरोनुवा॒क्या॑यै या॒ज्या॑यै दे॒वता॑यै दे॒वता॑यै या॒ज्या॑यै पुरोनुवा॒क्या॑यै पुरोनुवा॒क्या॑यै या॒ज्या॑यै दे॒वता॑यै । \newline
37. पु॒रो॒नु॒वा॒क्या॑या॒ इति॑ पुरः - अ॒नु॒वा॒क्या॑यै । \newline
38. या॒ज्या॑यै दे॒वता॑यै दे॒वता॑यै या॒ज्या॑यै या॒ज्या॑यै दे॒वता॑यै वषट्का॒राय॑ वषट्का॒राय॑ दे॒वता॑यै या॒ज्या॑यै या॒ज्या॑यै दे॒वता॑यै वषट्का॒राय॑ । \newline
39. दे॒वता॑यै वषट्का॒राय॑ वषट्का॒राय॑ दे॒वता॑यै दे॒वता॑यै वषट्का॒राय॒ यद् यद् व॑षट्का॒राय॑ दे॒वता॑यै दे॒वता॑यै वषट्का॒राय॒ यत् । \newline
40. व॒ष॒ट्का॒राय॒ यद् यद् व॑षट्का॒राय॑ वषट्का॒राय॒ यच् च॑तुर्गृही॒तम् च॑तुर्गृही॒तं ॅयद् व॑षट्का॒राय॑ वषट्का॒राय॒ यच् च॑तुर्गृही॒तम् । \newline
41. व॒ष॒ट्का॒रायेति॑ वषट् - का॒राय॑ । \newline
42. यच् च॑तुर्गृही॒तम् च॑तुर्गृही॒तं ॅयद् यच् च॑तुर्गृही॒तम् जु॒होति॑ जु॒होति॑ चतुर्गृही॒तं ॅयद् यच् च॑तुर्गृही॒तम् जु॒होति॑ । \newline
43. च॒तु॒र्गृ॒ही॒तम् जु॒होति॑ जु॒होति॑ चतुर्गृही॒तम् च॑तुर्गृही॒तम् जु॒होति॒ छन्दाꣳ॑सि॒ छन्दाꣳ॑सि जु॒होति॑ चतुर्गृही॒तम् च॑तुर्गृही॒तम् जु॒होति॒ छन्दाꣳ॑सि । \newline
44. च॒तु॒र्गृ॒ही॒तमिति॑ चतुः - गृ॒ही॒तम् । \newline
45. जु॒होति॒ छन्दाꣳ॑सि॒ छन्दाꣳ॑सि जु॒होति॑ जु॒होति॒ छन्दाꣳ॑ स्ये॒वैव छन्दाꣳ॑सि जु॒होति॑ जु॒होति॒ छन्दाꣳ॑ स्ये॒व । \newline
46. छन्दाꣳ॑ स्ये॒वैव छन्दाꣳ॑सि॒ छन्दाꣳ॑ स्ये॒व तत् तदे॒व छन्दाꣳ॑सि॒ छन्दाꣳ॑ स्ये॒व तत् । \newline
47. ए॒व तत् तदे॒वैव तत् प्री॑णाति प्रीणाति॒ तदे॒वैव तत् प्री॑णाति । \newline
48. तत् प्री॑णाति प्रीणाति॒ तत् तत् प्री॑णाति॒ तानि॒ तानि॑ प्रीणाति॒ तत् तत् प्री॑णाति॒ तानि॑ । \newline
49. प्री॒णा॒ति॒ तानि॒ तानि॑ प्रीणाति प्रीणाति॒ तान्य॑स्या स्य॒ तानि॑ प्रीणाति प्रीणाति॒ तान्य॑स्य । \newline
50. तान्य॑स्या स्य॒ तानि॒ तान्य॑स्य प्री॒तानि॑ प्री॒तान्य॑स्य॒ तानि॒ तान्य॑स्य प्री॒तानि॑ । \newline
51. अ॒स्य॒ प्री॒तानि॑ प्री॒ता न्य॑स्यास्य प्री॒तानि॑ दे॒वेभ्यो॑ दे॒वेभ्यः॑ प्री॒ता न्य॑स्यास्य प्री॒तानि॑ दे॒वेभ्यः॑ । \newline
52. प्री॒तानि॑ दे॒वेभ्यो॑ दे॒वेभ्यः॑ प्री॒तानि॑ प्री॒तानि॑ दे॒वेभ्यो॑ ह॒व्यꣳ ह॒व्यम् दे॒वेभ्यः॑ प्री॒तानि॑ प्री॒तानि॑ दे॒वेभ्यो॑ ह॒व्यम् । \newline
53. दे॒वेभ्यो॑ ह॒व्यꣳ ह॒व्यम् दे॒वेभ्यो॑ दे॒वेभ्यो॑ ह॒व्यं ॅव॑हन्ति वहन्ति ह॒व्यम् दे॒वेभ्यो॑ दे॒वेभ्यो॑ ह॒व्यं ॅव॑हन्ति । \newline
54. ह॒व्यं ॅव॑हन्ति वहन्ति ह॒व्यꣳ ह॒व्यं ॅव॑हन्ति॒ यं ॅयं ॅव॑हन्ति ह॒व्यꣳ ह॒व्यं ॅव॑हन्ति॒ यम् । \newline
55. व॒ह॒न्ति॒ यं ॅयं ॅव॑हन्ति वहन्ति॒ यम् का॒मये॑त का॒मये॑त॒ यं ॅव॑हन्ति वहन्ति॒ यम् का॒मये॑त । \newline
56. यम् का॒मये॑त का॒मये॑त॒ यं ॅयम् का॒मये॑त॒ पापी॑या॒न् पापी॑यान् का॒मये॑त॒ यं ॅयम् का॒मये॑त॒ पापी॑यान् । \newline
57. का॒मये॑त॒ पापी॑या॒न् पापी॑यान् का॒मये॑त का॒मये॑त॒ पापी॑यान् थ्स्याथ् स्या॒त् पापी॑यान् का॒मये॑त का॒मये॑त॒ पापी॑यान् थ्स्यात् । \newline
\pagebreak
\markright{ TS 5.1.1.2  \hfill https://www.vedavms.in \hfill}

\section{ TS 5.1.1.2 }

\textbf{TS 5.1.1.2 } \newline
\textbf{Samhita Paata} \newline

पापी॑यान्थ् स्या॒दित्येकै॑कं॒ तस्य॑ जुहुया॒-दाहु॑तीभिरे॒वैन॒मप॑ गृह्णाति॒ पापी॑यान् भवति॒ यं का॒मये॑त॒ वसी॑यान्थ् स्या॒दिति॒ सर्वा॑णि॒ तस्या॑ऽनु॒द्रुत्य॑ जुहुया॒दाहु॑त्यै॒वैन॑म॒भि क्र॑मयति॒ वसी॑यान् भव॒त्यथो॑ य॒ज्ञ्स्यै॒वैषा-ऽभिक्रा᳚न्ति॒रेति॒ वा ए॒ष य॑ज्ञ्मु॒खा-दृद्ध्या॒ यो᳚ऽग्नेर्दे॒वता॑या॒ एत्य॒ष्टावे॒तानि॑ सावि॒त्राणि॑ भवन्त्य॒ष्टाक्ष॑रा गाय॒त्री गा॑य॒त्रो᳚ - [  ] \newline

\textbf{Pada Paata} \newline

पापी॑यान् । स्या॒त् । इति॑ । एकै॑क॒मित्येक᳚म् - ए॒क॒म् । तस्य॑ । जु॒हु॒या॒त् । आहु॑तीभि॒रित्याहु॑ति - भिः॒ । ए॒व । ए॒न॒म् । अवेति॑ । गृ॒ह्णा॒ति॒ । पापी॑यान् । भ॒व॒ति॒ । यम् । का॒मये॑त । वसी॑यान् । स्या॒त् । इति॑ । सर्वा॑णि । तस्य॑ । अ॒नु॒द्रुत्येत्य॑नु - द्रुत्य॑ । जु॒हु॒या॒त् । आहु॒त्येत्या - हु॒त्या॒ । ए॒व । ए॒न॒म् । अ॒भीति॑ । क्र॒म॒य॒ति॒ । वसी॑यान् । भ॒व॒ति॒ । अथो॒ इति॑ । य॒ज्ञ्स्य॑ । ए॒व । ए॒षा । अ॒भिक्रा᳚न्ति॒रित्य॒भि -  क्रा॒न्तिः॒ । एति॑ । वै । ए॒षः । य॒ज्ञ्॒मु॒खादिति॑ यज्ञ्-मु॒खात् । ऋद्ध्याः᳚ । यः । अ॒ग्नेः । दे॒वता॑याः । एति॑ । अ॒ष्टौ । ए॒तानि॑ । सा॒वि॒त्राणि॑ । भ॒व॒न्ति॒ । अ॒ष्टाक्ष॒रेत्य॒ष्टा - अ॒क्ष॒रा॒ । गा॒य॒त्री । गा॒य॒त्रः ।  \newline


\textbf{Krama Paata} \newline

पापी॑यान्थ् स्यात् । स्या॒दिति॑ । इत्येकै॑कम् । एकै॑क॒म् तस्य॑ । एकै॑क॒मित्येक᳚म् - ए॒क॒म् । तस्य॑ जुहुयात् । जु॒हु॒या॒दाहु॑तीभिः । आहु॑तीभिरे॒व । आहु॑तीभि॒रित्याहु॑ति - भिः॒ । ए॒वैन᳚म् । ए॒न॒मप॑ । अप॑ गृह्णाति । गृ॒ह्णा॒ति॒ पापी॑यान् । पापी॑यान् भवति । भ॒व॒ति॒ यम् । यम् का॒मये॑त । का॒मये॑त॒ वसी॑यान् । वसी॑यान्थ् स्यात् । स्या॒दिति॑ । इति॒ सर्वा॑णि । सर्वा॑णि॒ तस्य॑ । तस्या॑नु॒द्रुत्य॑ । अ॒नु॒द्रुत्य॑ जुहुयात् । अ॒नु॒द्रुत्येत्य॑नु - द्रुत्य॑ । जु॒हु॒या॒दाहु॑त्या । आहु॑त्यै॒व । आहु॒त्येत्या - हु॒त्या॒ । ए॒वैन᳚म् । ए॒न॒म॒भि । अ॒भि क्र॑मयति । क्र॒म॒य॒ति॒ वसी॑यान् । वसी॑यान् भवति । भ॒व॒त्यथो᳚ । अथो॑ य॒ज्ञ्स्य॑ । अथो॒ इत्यथो᳚ । य॒ज्ञ्स्यै॒व । ए॒वैषा । ए॒षाऽभिक्रा᳚न्तिः । अ॒भिक्रा᳚न्ति॒रेति॑ । अ॒भिक्रा᳚न्ति॒रित्य॒भि - क्रा॒न्तिः॒ । एति॒ वै । वा ए॒षः । ए॒ष य॑ज्ञ्मु॒खात् । य॒ज्ञ्॒मु॒खादृद्ध्याः᳚ । य॒ज्ञ्॒मु॒खादिति॑ यज्ञ् - मु॒खात् । ऋद्ध्या॒ यः । यो᳚ऽग्नेः । अ॒ग्नेर् दे॒वता॑याः । दे॒वता॑या॒ एति॑ । एत्य॒ष्टौ । अ॒ष्टावे॒तानि॑ । ए॒तानि॑ सावि॒त्राणि॑ । सा॒वि॒त्राणि॑ भवन्ति । भ॒व॒न्त्य॒ष्टाक्ष॑रा । अ॒ष्टाक्ष॑रा गाय॒त्री । अ॒ष्टाक्ष॒रेत्य॒ष्टा - अ॒क्ष॒रा॒ । गा॒य॒त्री गा॑य॒त्रः । गा॒य॒त्रो᳚ऽग्निः \newline

\textbf{Jatai Paata} \newline

1. पापी॑यान् थ्स्याथ् स्या॒त् पापी॑या॒न् पापी॑यान् थ्स्यात् । \newline
2. स्या॒दितीति॑ स्याथ् स्या॒दिति॑ । \newline
3. इत्येकै॑क॒ मेकै॑क॒ मिती त्येकै॑कम् । \newline
4. एकै॑क॒म् तस्य॒ तस्यैकै॑क॒ मेकै॑क॒म् तस्य॑ । \newline
5. एकै॑क॒मित्येक᳚म् - ए॒क॒म् । \newline
6. तस्य॑ जुहुयाज् जुहुया॒त् तस्य॒ तस्य॑ जुहुयात् । \newline
7. जु॒हु॒या॒ दाहु॑तीभि॒ राहु॑तीभिर् जुहुयाज् जुहुया॒ दाहु॑तीभिः । \newline
8. आहु॑तीभि रे॒वैवाहु॑तीभि॒ राहु॑तीभि रे॒व । \newline
9. आहु॑तीभि॒रित्याहु॑ति - भिः॒ । \newline
10. ए॒वैन॑ मेन मे॒वैवैन᳚म् । \newline
11. ए॒न॒ मपापै॑न मेन॒ मप॑ । \newline
12. अप॑ गृह्णाति गृह्णा॒ त्यपाप॑ गृह्णाति । \newline
13. गृ॒ह्णा॒ति॒ पापी॑या॒न् पापी॑यान् गृह्णाति गृह्णाति॒ पापी॑यान् । \newline
14. पापी॑यान् भवति भवति॒ पापी॑या॒न् पापी॑यान् भवति । \newline
15. भ॒व॒ति॒ यं ॅयम् भ॑वति भवति॒ यम् । \newline
16. यम् का॒मये॑त का॒मये॑त॒ यं ॅयम् का॒मये॑त । \newline
17. का॒मये॑त॒ वसी॑या॒न्॒. वसी॑यान् का॒मये॑त का॒मये॑त॒ वसी॑यान् । \newline
18. वसी॑यान् थ्स्याथ् स्या॒द् वसी॑या॒न्॒. वसी॑यान् थ्स्यात् । \newline
19. स्या॒दितीति॑ स्याथ् स्या॒दिति॑ । \newline
20. इति॒ सर्वा॑णि॒ सर्वा॒णीतीति॒ सर्वा॑णि । \newline
21. सर्वा॑णि॒ तस्य॒ तस्य॒ सर्वा॑णि॒ सर्वा॑णि॒ तस्य॑ । \newline
22. तस्या॑ नु॒द्रु त्या॑नु॒द्रुत्य॒ तस्य॒ तस्या॑ नु॒द्रुत्य॑ । \newline
23. अ॒नु॒द्रुत्य॑ जुहुयाज् जुहुया दनु॒द्रुत्या॑ नु॒द्रुत्य॑ जुहुयात् । \newline
24. अ॒नु॒द्रुत्येत्य॑नु - द्रुत्य॑ । \newline
25. जु॒हु॒या॒ दाहु॒त्या ऽऽहु॑त्या जुहुयाज् जुहुया॒ दाहु॑त्या । \newline
26. आहु॑ त्यै॒वैवा हु॒त्या ऽऽहु॑त्यै॒व । \newline
27. आहु॒त्येत्या - हु॒त्या॒ । \newline
28. ए॒वैन॑ मेन मे॒वैवैन᳚म् । \newline
29. ए॒न॒ म॒भ्या᳚(1॒)भ्ये॑न मेन म॒भि । \newline
30. अ॒भि क्र॑मयति क्रमय त्य॒भ्य॑भि क्र॑मयति । \newline
31. क्र॒म॒य॒ति॒ वसी॑या॒न्॒. वसी॑यान् क्रमयति क्रमयति॒ वसी॑यान् । \newline
32. वसी॑यान् भवति भवति॒ वसी॑या॒न्॒. वसी॑यान् भवति । \newline
33. भ॒व॒त्यथो॒ अथो॑ भवति भव॒त्यथो᳚ । \newline
34. अथो॑ य॒ज्ञ्स्य॑ य॒ज्ञ्स्याथो॒ अथो॑ य॒ज्ञ्स्य॑ । \newline
35. अथो॒ इत्यथो᳚ । \newline
36. य॒ज्ञ् स्यै॒वैव य॒ज्ञ्स्य॑ य॒ज्ञ् स्यै॒व । \newline
37. ए॒वै षैषै वैवैषा । \newline
38. ए॒षा ऽभिक्रा᳚न्ति र॒भिक्रा᳚न्ति रे॒षैषा ऽभिक्रा᳚न्तिः । \newline
39. अ॒भिक्रा᳚न्ति॒ रेत्ये त्य॒भिक्रा᳚न्ति र॒भिक्रा᳚न्ति॒ रेति॑ । \newline
40. अ॒भिक्रा᳚न्ति॒रित्य॒भि - क्रा॒न्तिः॒ । \newline
41. एति॒ वै वा एत्येति॒ वै । \newline
42. वा ए॒ष ए॒ष वै वा ए॒षः । \newline
43. ए॒ष य॑ज्ञ्मु॒खाद् य॑ज्ञ्मु॒खादे॒ष ए॒ष य॑ज्ञ्मु॒खात् । \newline
44. य॒ज्ञ्॒मु॒खा दृद्ध्या॒ ऋद्ध्या॑ यज्ञ्मु॒खाद् य॑ज्ञ्मु॒खा दृद्ध्याः᳚ । \newline
45. य॒ज्ञ्॒मु॒खादिति॑ यज्ञ् - मु॒खात् । \newline
46. ऋद्ध्या॒ यो य ऋद्ध्या॒ ऋद्ध्या॒ यः । \newline
47. यो᳚ ऽग्ने र॒ग्नेर् यो यो᳚ ऽग्नेः । \newline
48. अ॒ग्नेर् दे॒वता॑या दे॒वता॑या अ॒ग्ने र॒ग्नेर् दे॒वता॑याः । \newline
49. दे॒वता॑या॒ एत्येति॑ दे॒वता॑या दे॒वता॑या॒ एति॑ । \newline
50. एत्य॒ष्टा व॒ष्टा वेत्ये त्य॒ष्टौ । \newline
51. अ॒ष्टा वे॒ता न्ये॒ता न्य॒ष्टा व॒ष्टा वे॒तानि॑ । \newline
52. ए॒तानि॑ सावि॒त्राणि॑ सावि॒त्राण्ये॒ता न्ये॒तानि॑ सावि॒त्राणि॑ । \newline
53. सा॒वि॒त्राणि॑ भवन्ति भवन्ति सावि॒त्राणि॑ सावि॒त्राणि॑ भवन्ति । \newline
54. भ॒व॒ न्त्य॒ष्टाक्ष॑रा॒ ऽष्टाक्ष॑रा भवन्ति भव न्त्य॒ष्टाक्ष॑रा । \newline
55. अ॒ष्टाक्ष॑रा गाय॒त्री गा॑य॒ त्र्य॑ष्टाक्ष॑रा॒ ऽष्टाक्ष॑रा गाय॒त्री । \newline
56. अ॒ष्टाक्ष॒रेत्य॒ष्टा - अ॒क्ष॒रा॒ । \newline
57. गा॒य॒त्री गा॑य॒त्रो गा॑य॒त्रो गा॑य॒त्री गा॑य॒त्री गा॑य॒त्रः । \newline
58. गा॒य॒त्रो᳚ ऽग्नि र॒ग्निर् गा॑य॒त्रो गा॑य॒त्रो᳚ ऽग्निः । \newline

\textbf{Ghana Paata } \newline

1. पापी॑यान् थ्स्याथ् स्या॒त् पापी॑या॒न् पापी॑यान् थ्स्या॒ दितीति॑ स्या॒त् पापी॑या॒न् पापी॑यान् थ्स्या॒दिति॑ । \newline
2. स्या॒दितीति॑ स्याथ् स्या॒ दित्येकै॑क॒ मेकै॑क॒ मिति॑ स्याथ् स्या॒ दित्येकै॑कम् । \newline
3. इत्येकै॑क॒ मेकै॑क॒ मिती त्येकै॑क॒म् तस्य॒ तस्यैकै॑क॒ मिती त्येकै॑क॒म् तस्य॑ । \newline
4. एकै॑क॒म् तस्य॒ तस्यैकै॑क॒ मेकै॑क॒म् तस्य॑ जुहुयाज् जुहुया॒त् तस्यैकै॑क॒ मेकै॑क॒म् तस्य॑ जुहुयात् । \newline
5. एकै॑क॒मित्येक᳚म् - ए॒क॒म् । \newline
6. तस्य॑ जुहुयाज् जुहुया॒त् तस्य॒ तस्य॑ जुहुया॒ दाहु॑तीभि॒ राहु॑तीभिर् जुहुया॒त् तस्य॒ तस्य॑ जुहुया॒ दाहु॑तीभिः । \newline
7. जु॒हु॒या॒ दाहु॑तीभि॒ राहु॑तीभिर् जुहुयाज् जुहुया॒ दाहु॑तीभि रे॒वैवाहु॑तीभिर् जुहुयाज् जुहुया॒ दाहु॑तीभि रे॒व । \newline
8. आहु॑तीभि रे॒वैवाहु॑तीभि॒ राहु॑तीभि रे॒वैन॑ मेन मे॒वाहु॑तीभि॒ राहु॑तीभि रे॒वैन᳚म् । \newline
9. आहु॑तीभि॒रित्याहु॑ति - भिः॒ । \newline
10. ए॒वैन॑ मेन मे॒वैवैन॒ मपापै॑न मे॒वैवैन॒ मप॑ । \newline
11. ए॒न॒ मपापै॑न मेन॒ मप॑ गृह्णाति गृह्णा॒ त्यपै॑न मेन॒ मप॑ गृह्णाति । \newline
12. अप॑ गृह्णाति गृह्णा॒ त्यपाप॑ गृह्णाति॒ पापी॑या॒न् पापी॑यान् गृह्णा॒ त्यपाप॑ गृह्णाति॒ पापी॑यान् । \newline
13. गृ॒ह्णा॒ति॒ पापी॑या॒न् पापी॑यान् गृह्णाति गृह्णाति॒ पापी॑यान् भवति भवति॒ पापी॑यान् गृह्णाति गृह्णाति॒ पापी॑यान् भवति । \newline
14. पापी॑यान् भवति भवति॒ पापी॑या॒न् पापी॑यान् भवति॒ यं ॅयम् भ॑वति॒ पापी॑या॒न् पापी॑यान् भवति॒ यम् । \newline
15. भ॒व॒ति॒ यं ॅयम् भ॑वति भवति॒ यम् का॒मये॑त का॒मये॑त॒ यम् भ॑वति भवति॒ यम् का॒मये॑त । \newline
16. यम् का॒मये॑त का॒मये॑त॒ यं ॅयम् का॒मये॑त॒ वसी॑या॒न्॒. वसी॑यान् का॒मये॑त॒ यं ॅयम् का॒मये॑त॒ वसी॑यान् । \newline
17. का॒मये॑त॒ वसी॑या॒न्॒. वसी॑यान् का॒मये॑त का॒मये॑त॒ वसी॑यान् थ्स्याथ् स्या॒द् वसी॑यान् का॒मये॑त का॒मये॑त॒ वसी॑यान् थ्स्यात् । \newline
18. वसी॑यान् थ्स्याथ् स्या॒द् वसी॑या॒न्॒. वसी॑यान् थ्स्या॒दितीति॑ स्या॒द् वसी॑या॒न्॒. वसी॑यान् थ्स्या॒दिति॑ । \newline
19. स्या॒दितीति॑ स्याथ् स्या॒दिति॒ सर्वा॑णि॒ सर्वा॒णीति॑ स्याथ् स्या॒दिति॒ सर्वा॑णि । \newline
20. इति॒ सर्वा॑णि॒ सर्वा॒णीतीति॒ सर्वा॑णि॒ तस्य॒ तस्य॒ सर्वा॒णीतीति॒ सर्वा॑णि॒ तस्य॑ । \newline
21. सर्वा॑णि॒ तस्य॒ तस्य॒ सर्वा॑णि॒ सर्वा॑णि॒ तस्या॑ नु॒द्रुत्या॑ नु॒द्रुत्य॒ तस्य॒ सर्वा॑णि॒ सर्वा॑णि॒ तस्या॑ नु॒द्रुत्य॑ । \newline
22. तस्या॑ नु॒द्रुत्या॑ नु॒द्रुत्य॒ तस्य॒ तस्या॑ नु॒द्रुत्य॑ जुहुयाज् जुहुया दनु॒द्रुत्य॒ तस्य॒ तस्या॑ नु॒द्रुत्य॑ जुहुयात् । \newline
23. अ॒नु॒द्रुत्य॑ जुहुयाज् जुहुया दनु॒द्रुत्या॑ नु॒द्रुत्य॑ जुहुया॒ दाहु॒त्या ऽऽहु॑त्या जुहुया दनु॒द्रुत्या॑ नु॒द्रुत्य॑ जुहुया॒ दाहु॑त्या । \newline
24. अ॒नु॒द्रुत्येत्य॑नु - द्रुत्य॑ । \newline
25. जु॒हु॒या॒ दाहु॒त्या ऽऽहु॑त्या जुहुयाज् जुहुया॒ दाहु॑त्यै॒वै वाहु॑त्या जुहुयाज् जुहुया॒ दाहु॑त्यै॒व । \newline
26. आहु॑त्यै॒वै वाहु॒त्या ऽऽहु॑त्यै॒वैन॑ मेन मे॒वाहु॒त्या ऽऽहु॑त्यै॒वैन᳚म् । \newline
27. आहु॒त्येत्या - हु॒त्या॒ । \newline
28. ए॒वैन॑ मेन मे॒वैवैन॑ म॒भ्या᳚(1॒)भ्ये॑न मे॒वैवैन॑ म॒भि । \newline
29. ए॒न॒ म॒भ्या᳚(1॒)भ्ये॑न मेन म॒भि क्र॑मयति क्रमय त्य॒भ्ये॑न मेन म॒भि क्र॑मयति । \newline
30. अ॒भि क्र॑मयति क्रमय त्य॒भ्य॑भि क्र॑मयति॒ वसी॑या॒न्॒. वसी॑यान् क्रमय त्य॒भ्य॑भि क्र॑मयति॒ वसी॑यान् । \newline
31. क्र॒म॒य॒ति॒ वसी॑या॒न्॒. वसी॑यान् क्रमयति क्रमयति॒ वसी॑यान् भवति भवति॒ वसी॑यान् क्रमयति क्रमयति॒ वसी॑यान् भवति । \newline
32. वसी॑यान् भवति भवति॒ वसी॑या॒न्॒. वसी॑यान् भव॒त्यथो॒ अथो॑ भवति॒ वसी॑या॒न्॒. वसी॑यान् भव॒त्यथो᳚ । \newline
33. भ॒व॒ त्यथो॒ अथो॑ भवति भव॒त्यथो॑ य॒ज्ञ्स्य॑ य॒ज्ञ्स्याथो॑ भवति भव॒ त्यथो॑ य॒ज्ञ्स्य॑ । \newline
34. अथो॑ य॒ज्ञ्स्य॑ य॒ज्ञ्स्याथो॒ अथो॑ य॒ज्ञ् स्यै॒वैव य॒ज्ञ्स्याथो॒ अथो॑ य॒ज्ञ्स्यै॒व । \newline
35. अथो॒ इत्यथो᳚ । \newline
36. य॒ज्ञ् स्यै॒वैव य॒ज्ञ्स्य॑ य॒ज्ञ् स्यै॒वै षैषैव य॒ज्ञ्स्य॑ य॒ज्ञ्स्यै॒वैषा । \newline
37. ए॒वै षैषै वैवैषा ऽभिक्रा᳚न्ति र॒भिक्रा᳚न्ति रे॒षै वैवैषा ऽभिक्रा᳚न्तिः । \newline
38. ए॒षा ऽभिक्रा᳚न्ति र॒भिक्रा᳚न्ति रे॒षैषा ऽभिक्रा᳚न्ति॒ रेत्ये त्य॒भिक्रा᳚न्ति रे॒षैषा ऽभिक्रा᳚न्ति॒ रेति॑ । \newline
39. अ॒भिक्रा᳚न्ति॒ रेत्ये त्य॒भिक्रा᳚न्ति र॒भिक्रा᳚न्ति॒ रेति॒ वै वा एत्य॒भिक्रा᳚न्ति र॒भिक्रा᳚न्ति॒ रेति॒ वै । \newline
40. अ॒भिक्रा᳚न्ति॒रित्य॒भि - क्रा॒न्तिः॒ । \newline
41. एति॒ वै वा एत्येति॒ वा ए॒ष ए॒ष वा एत्येति॒ वा ए॒षः । \newline
42. वा ए॒ष ए॒ष वै वा ए॒ष य॑ज्ञ्मु॒खाद् य॑ज्ञ्मु॒खा दे॒ष वै वा ए॒ष य॑ज्ञ्मु॒खात् । \newline
43. ए॒ष य॑ज्ञ्मु॒खाद् य॑ज्ञ्मु॒खा दे॒ष ए॒ष य॑ज्ञ्मु॒खा दृद्ध्या॒ ऋद्ध्या॑ यज्ञ्मु॒खा दे॒ष ए॒ष य॑ज्ञ्मु॒खा दृद्ध्याः᳚ । \newline
44. य॒ज्ञ्॒मु॒खा दृद्ध्या॒ ऋद्ध्या॑ यज्ञ्मु॒खाद् य॑ज्ञ्मु॒खा दृद्ध्या॒ यो य ऋद्ध्या॑ यज्ञ्मु॒खाद् य॑ज्ञ्मु॒खा दृद्ध्या॒ यः । \newline
45. य॒ज्ञ्॒मु॒खादिति॑ यज्ञ् - मु॒खात् । \newline
46. ऋद्ध्या॒ यो य ऋद्ध्या॒ ऋद्ध्या॒ यो᳚ ऽग्ने र॒ग्नेर् य ऋद्ध्या॒ ऋद्ध्या॒ यो᳚ ऽग्नेः । \newline
47. यो᳚ ऽग्ने र॒ग्नेर् यो यो᳚ ऽग्नेर् दे॒वता॑या दे॒वता॑या अ॒ग्नेर् यो यो᳚ ऽग्नेर् दे॒वता॑याः । \newline
48. अ॒ग्नेर् दे॒वता॑या दे॒वता॑या अ॒ग्ने र॒ग्नेर् दे॒वता॑या॒ एत्येति॑ दे॒वता॑या अ॒ग्ने र॒ग्नेर् दे॒वता॑या॒ एति॑ । \newline
49. दे॒वता॑या॒ एत्येति॑ दे॒वता॑या दे॒वता॑या॒ एत्य॒ष्टा व॒ष्टा वेति॑ दे॒वता॑या दे॒वता॑या॒ एत्य॒ष्टौ । \newline
50. एत्य॒ष्टा व॒ष्टा वेत्येत्य॒ष्टा वे॒ता न्ये॒ता न्य॒ष्टा वेत्येत्य॒ष्टा वे॒तानि॑ । \newline
51. अ॒ष्टा वे॒ता न्ये॒ता न्य॒ष्टा व॒ष्टा वे॒तानि॑ सावि॒त्राणि॑ सावि॒त्रा ण्ये॒ता न्य॒ष्टा व॒ष्टा वे॒तानि॑ सावि॒त्राणि॑ । \newline
52. ए॒तानि॑ सावि॒त्राणि॑ सावि॒त्राण्ये॒ता न्ये॒तानि॑ सावि॒त्राणि॑ भवन्ति भवन्ति सावि॒त्रा ण्ये॒ता न्ये॒तानि॑ सावि॒त्राणि॑ भवन्ति । \newline
53. सा॒वि॒त्राणि॑ भवन्ति भवन्ति सावि॒त्राणि॑ सावि॒त्राणि॑ भव न्त्य॒ष्टाक्ष॑रा॒ ऽष्टाक्ष॑रा भवन्ति सावि॒त्राणि॑ सावि॒त्राणि॑ भव न्त्य॒ष्टाक्ष॑रा । \newline
54. भ॒व॒ न्त्य॒ष्टाक्ष॑रा॒ ऽष्टाक्ष॑रा भवन्ति भव न्त्य॒ष्टाक्ष॑रा गाय॒त्री गा॑य॒ त्र्य॑ष्टाक्ष॑रा भवन्ति भव न्त्य॒ष्टाक्ष॑रा गाय॒त्री । \newline
55. अ॒ष्टाक्ष॑रा गाय॒त्री गा॑य॒ त्र्य॑ष्टाक्ष॑रा॒ ऽष्टाक्ष॑रा गाय॒त्री गा॑य॒त्रो गा॑य॒त्रो गा॑य॒ त्र्य॑ष्टाक्ष॑रा॒ ऽष्टाक्ष॑रा गाय॒त्री गा॑य॒त्रः । \newline
56. अ॒ष्टाक्ष॒रेत्य॒ष्टा - अ॒क्ष॒रा॒ । \newline
57. गा॒य॒त्री गा॑य॒त्रो गा॑य॒त्रो गा॑य॒त्री गा॑य॒त्री गा॑य॒त्रो᳚ ऽग्नि र॒ग्निर् गा॑य॒त्रो गा॑य॒त्री गा॑य॒त्री गा॑य॒त्रो᳚ ऽग्निः । \newline
58. गा॒य॒त्रो᳚ ऽग्नि र॒ग्निर् गा॑य॒त्रो गा॑य॒त्रो᳚ ऽग्नि स्तेन॒ तेना॒ग्निर् गा॑य॒त्रो गा॑य॒त्रो᳚ ऽग्नि स्तेन॑ । \newline
\pagebreak
\markright{ TS 5.1.1.3  \hfill https://www.vedavms.in \hfill}

\section{ TS 5.1.1.3 }

\textbf{TS 5.1.1.3 } \newline
\textbf{Samhita Paata} \newline

ऽग्निस्तेनै॒व य॑ज्ञ्मु॒खादृद्ध्या॑ अ॒ग्नेर्दे॒वता॑यै॒ नैत्य॒ष्टौ सा॑वि॒त्राणि॑ भव॒न्त्याहु॑तिर्नव॒मी त्रि॒वृत॑मे॒व य॑ज्ञ्मु॒खे विया॑तयति॒ यदि॑ का॒मये॑त॒ छन्दाꣳ॑सि यज्ञ्यश॒सेना᳚ ऽर्पयेय॒मित्यृच॑मन्त॒मां कु॑र्या॒च्छन्दाꣳ॑स्ये॒व य॑ज्ञ्यश॒सेना᳚ ऽर्पयति॒ यदि॑ का॒मये॑त॒ यज॑मानं ॅयज्ञ्यश॒सेना᳚-ऽर्पयेय॒मिति॒ यजु॑रन्त॒मं कु॑र्या॒द्-यज॑मानमे॒व य॑ज्ञ्यश॒सेना᳚-ऽर्पयत्यृ॒चा स्तोमꣳ॒॒ सम॑र्द्ध॒येत्या॑ - [  ] \newline

\textbf{Pada Paata} \newline

अ॒ग्निः । तेन॑ । ए॒व । य॒ज्ञ्॒मु॒खादिति॑ यज्ञ् - मु॒खात् । ऋद्ध्याः᳚ । अ॒ग्नेः । दे॒वता॑यै । न । ए॒ति॒ । अ॒ष्टौ । सा॒वि॒त्राणि॑ । भ॒व॒न्ति॒ । आहु॑ति॒रित्या - हू॒तिः॒ । न॒व॒मी । त्रि॒वृत॒मिति॑ त्रि - वृत᳚म् । ए॒व । य॒ज्ञ्॒मु॒ख इति॑ यज्ञ् - मु॒खे । वीति॑ । या॒त॒य॒ति॒ । यदि॑ । का॒मये॑त । छन्दाꣳ॑सि । य॒ज्ञ्॒य॒श॒सेनेति॑ यज्ञ् - य॒श॒सेन॑ । अ॒र्प॒ये॒य॒म् । इति॑ । ऋच᳚म् । अ॒न्त॒माम् । कु॒र्या॒त् । छन्दाꣳ॑सि । ए॒व । य॒ज्ञ्॒य॒श॒सेनेति॑ यज्ञ् - य॒श॒सेन॑ । अ॒र्प॒य॒ति॒ । यदि॑ । का॒मये॑त । यज॑मानम् । य॒ज्ञ्॒य॒श॒सेनेति॑ यज्ञ् - य॒श॒सेन॑ । अ॒र्प॒ये॒य॒म् । इति॑ । यजुः॑ । अ॒न्त॒मम् । कु॒र्या॒त् । यज॑मानम् । ए॒व । य॒ज्ञ्॒य॒श॒सेनेति॑ यज्ञ् - य॒श॒सेन॑ । अ॒र्प॒य॒ति॒ । ऋ॒चा । स्तोम᳚म् । समिति॑ । अ॒द्‌र्ध॒य॒ । इति॑ ।  \newline


\textbf{Krama Paata} \newline

अ॒ग्निस्तेन॑ । तेनै॒व । ए॒व य॑ज्ञ्मु॒खात् । य॒ज्ञ्॒मु॒खादृद्ध्याः᳚ । य॒ज्ञ्॒मु॒खादिति॑ यज्ञ् - मु॒खात् । ऋद्ध्या॑ अ॒ग्नेः । अ॒ग्नेर् दे॒वता॑यै । दे॒वता॑यै॒ न । नैति॑ । एत्य॒ष्टौ । अ॒ष्टौ सा॑वि॒त्राणि॑ । सा॒वि॒त्राणि॑ भवन्ति । भ॒व॒न्त्याहु॑तिः । आहु॑तिर् नव॒मी । आहु॑ति॒रित्या - हु॒तिः॒ । न॒व॒मी त्रि॒वृत᳚म् । त्रि॒वृत॑मे॒व । त्रि॒वृत॒मिति॑ त्रि - वृत᳚म् । ए॒व य॑ज्ञ्मु॒खे । य॒ज्ञ्॒मु॒खे वि । य॒ज्ञ्॒मु॒ख इति॑ यज्ञ् - मु॒खे । वि या॑तयति । या॒त॒य॒ति॒ यदि॑ । यदि॑ का॒मये॑त । का॒मये॑त॒ छन्दाꣳ॑सि । छन्दाꣳ॑सि यज्ञ्यश॒सेन॑ । य॒ज्ञ्॒य॒श॒सेना᳚र्पयेयम् । य॒ज्ञ्॒य॒श॒सेनेति॑ यज्ञ् - य॒श॒सेन॑ । अ॒र्प॒ये॒य॒मिति॑ । इत्यृच᳚म् । ऋच॑मन्त॒माम् । अ॒न्त॒माम् कु॑र्यात् । कु॒र्या॒च्छन्दाꣳ॑सि । छन्दाꣳ॑स्ये॒व । ए॒व य॑ज्ञ्यश॒सेन॑ । य॒ज्ञ्॒य॒श॒सेना᳚र्पयति । य॒ज्ञ्॒य॒श॒सेनेति॑ यज्ञ् - य॒श॒सेन॑ । अ॒र्प॒य॒ति॒ यदि॑ । यदि॑ का॒मये॑त । का॒मये॑त॒ यज॑मानम् । यज॑मानम् ॅयज्ञ्यश॒सेन॑ । य॒ज्ञ्॒य॒श॒सेना᳚र्पयेयम् । य॒ज्ञ्॒य॒श॒सेनेति॑ यज्ञ् - य॒श॒सेन॑ । अ॒र्प॒ये॒य॒मिति॑ । इति॒ यजुः॑ । यजु॑रन्त॒मम् । अ॒न्त॒मम् कु॑र्यात् । कु॒र्या॒द् यज॑मानम् । यज॑मानमे॒व । ए॒व य॑ज्ञ्यश॒सेन॑ । य॒ज्ञ्॒य॒श॒सेना᳚र्पयति । य॒ज्ञ्॒य॒श॒सेनेति॑ यज्ञ् - य॒श॒सेन॑ । अ॒र्प॒य॒त्यृ॒चा । ऋ॒चा स्तोम᳚म् । स्तोमꣳ॒॒ सम् । सम॑र्द्धय । अ॒र्द्ध॒येति॑ । इत्या॑ह \newline

\textbf{Jatai Paata} \newline

1. अ॒ग्नि स्तेन॒ तेना॒ग्नि र॒ग्नि स्तेन॑ । \newline
2. तेनै॒ वैव तेन॒ तेनै॒व । \newline
3. ए॒व य॑ज्ञ्मु॒खाद् य॑ज्ञ्मु॒खा दे॒वैव य॑ज्ञ्मु॒खात् । \newline
4. य॒ज्ञ्॒मु॒खा दृद्ध्या॒ ऋद्ध्या॑ यज्ञ्मु॒खाद् य॑ज्ञ्मु॒खा दृद्ध्याः᳚ । \newline
5. य॒ज्ञ्॒मु॒खादिति॑ यज्ञ् - मु॒खात् । \newline
6. ऋद्ध्या॑ अ॒ग्ने र॒ग्नेर्. ऋद्ध्या॒ ऋद्ध्या॑ अ॒ग्नेः । \newline
7. अ॒ग्नेर् दे॒वता॑यै दे॒वता॑या अ॒ग्ने र॒ग्नेर् दे॒वता॑यै । \newline
8. दे॒वता॑यै॒ न न दे॒वता॑यै दे॒वता॑यै॒ न । \newline
9. नैत्ये॑ति॒ न नैति॑ । \newline
10. ए॒त्य॒ष्टा व॒ष्टा वे᳚त्ये त्य॒ष्टौ । \newline
11. अ॒ष्टौ सा॑वि॒त्राणि॑ सावि॒त्रा ण्य॒ष्टा व॒ष्टौ सा॑वि॒त्राणि॑ । \newline
12. सा॒वि॒त्राणि॑ भवन्ति भवन्ति सावि॒त्राणि॑ सावि॒त्राणि॑ भवन्ति । \newline
13. भ॒व॒ न्त्याहु॑ति॒ राहु॑तिर् भवन्ति भव॒ न्त्याहु॑तिः । \newline
14. आहु॑तिर् नव॒मी न॑व॒ म्याहु॑ति॒ राहु॑तिर् नव॒मी । \newline
15. आहु॑ति॒रित्या - हु॒तिः॒ । \newline
16. न॒व॒मी त्रि॒वृत॑म् त्रि॒वृत॑म् नव॒मी न॑व॒मी त्रि॒वृत᳚म् । \newline
17. त्रि॒वृत॑ मे॒वैव त्रि॒वृत॑म् त्रि॒वृत॑ मे॒व । \newline
18. त्रि॒वृत॒मिति॑ त्रि - वृत᳚म् । \newline
19. ए॒व य॑ज्ञ्मु॒खे य॑ज्ञ्मु॒ख ए॒वैव य॑ज्ञ्मु॒खे । \newline
20. य॒ज्ञ्॒मु॒खे वि वि य॑ज्ञ्मु॒खे य॑ज्ञ्मु॒खे वि । \newline
21. य॒ज्ञ्॒मु॒ख इति॑ यज्ञ् - मु॒खे । \newline
22. वि या॑तयति यातयति॒ वि वि या॑तयति । \newline
23. या॒त॒य॒ति॒ यदि॒ यदि॑ यातयति यातयति॒ यदि॑ । \newline
24. यदि॑ का॒मये॑त का॒मये॑त॒ यदि॒ यदि॑ का॒मये॑त । \newline
25. का॒मये॑त॒ छन्दाꣳ॑सि॒ छन्दाꣳ॑सि का॒मये॑त का॒मये॑त॒ छन्दाꣳ॑सि । \newline
26. छन्दाꣳ॑सि यज्ञ्यश॒सेन॑ यज्ञ्यश॒सेन॒ छन्दाꣳ॑सि॒ छन्दाꣳ॑सि यज्ञ्यश॒सेन॑ । \newline
27. य॒ज्ञ्॒य॒श॒सेना᳚ र्पयेय मर्पयेयं ॅयज्ञ्यश॒सेन॑ यज्ञ्यश॒सेना᳚ र्पयेयम् । \newline
28. य॒ज्ञ्॒य॒श॒सेनेति॑ यज्ञ् - य॒श॒सेन॑ । \newline
29. अ॒र्प॒ये॒य॒ मिती त्य॑र्पयेय मर्पयेय॒ मिति॑ । \newline
30. इत्यृच॒ मृच॒ मिती त्यृच᳚म् । \newline
31. ऋच॑ मन्त॒मा म॑न्त॒मा मृच॒ मृच॑ मन्त॒माम् । \newline
32. अ॒न्त॒माम् कु॑र्यात् कुर्यादन्त॒मा म॑न्त॒माम् कु॑र्यात् । \newline
33. कु॒र्या॒च् छन्दाꣳ॑सि॒ छन्दाꣳ॑सि कुर्यात् कुर्या॒च् छन्दाꣳ॑सि । \newline
34. छन्दाꣳ॑ स्ये॒वैव छन्दाꣳ॑सि॒ छन्दाꣳ॑ स्ये॒व । \newline
35. ए॒व य॑ज्ञ्यश॒सेन॑ यज्ञ्यश॒से नै॒वैव य॑ज्ञ्यश॒सेन॑ । \newline
36. य॒ज्ञ्॒य॒श॒सेना᳚ र्पय त्यर्पयति यज्ञ्यश॒सेन॑ यज्ञ्यश॒सेना᳚ र्पयति । \newline
37. य॒ज्ञ्॒य॒श॒सेनेति॑ यज्ञ् - य॒श॒सेन॑ । \newline
38. अ॒र्प॒य॒ति॒ यदि॒ यद्य॑र्पय त्यर्पयति॒ यदि॑ । \newline
39. यदि॑ का॒मये॑त का॒मये॑त॒ यदि॒ यदि॑ का॒मये॑त । \newline
40. का॒मये॑त॒ यज॑मानं॒ ॅयज॑मानम् का॒मये॑त का॒मये॑त॒ यज॑मानम् । \newline
41. यज॑मानं ॅयज्ञ्यश॒सेन॑ यज्ञ्यश॒सेन॒ यज॑मानं॒ ॅयज॑मानं ॅयज्ञ्यश॒सेन॑ । \newline
42. य॒ज्ञ्॒य॒श॒सेना᳚ र्पयेय मर्पयेयं ॅयज्ञ्यश॒सेन॑ यज्ञ्यश॒सेना᳚ र्पयेयम् । \newline
43. य॒ज्ञ्॒य॒श॒सेनेति॑ यज्ञ् - य॒श॒सेन॑ । \newline
44. अ॒र्प॒ये॒य॒ मिती त्य॑र्पयेय मर्पयेय॒ मिति॑ । \newline
45. इति॒ यजु॒र् यजु॒ रितीति॒ यजुः॑ । \newline
46. यजु॑ रन्त॒म म॑न्त॒मं ॅयजु॒र् यजु॑ रन्त॒मम् । \newline
47. अ॒न्त॒मम् कु॑र्यात् कुर्या दन्त॒म म॑न्त॒मम् कु॑र्यात् । \newline
48. कु॒र्या॒द् यज॑मानं॒ ॅयज॑मानम् कुर्यात् कुर्या॒द् यज॑मानम् । \newline
49. यज॑मान मे॒वैव यज॑मानं॒ ॅयज॑मान मे॒व । \newline
50. ए॒व य॑ज्ञ्यश॒सेन॑ यज्ञ्यश॒से नै॒वैव य॑ज्ञ्यश॒सेन॑ । \newline
51. य॒ज्ञ्॒य॒श॒सेना᳚ र्पय त्यर्पयति यज्ञ्यश॒सेन॑ यज्ञ्यश॒सेना᳚ र्पयति । \newline
52. य॒ज्ञ्॒य॒श॒सेनेति॑ यज्ञ् - य॒श॒सेन॑ । \newline
53. अ॒र्प॒य॒ त्यृ॒च र्‌चा ऽर्प॑य त्यर्पय त्यृ॒चा । \newline
54. ऋ॒चा स्तोमꣳ॒॒ स्तोम॑ मृ॒च र्‌चा स्तोम᳚म् । \newline
55. स्तोमꣳ॒॒ सꣳ सꣳ स्तोमꣳ॒॒ स्तोमꣳ॒॒ सम् । \newline
56. स म॑र्द्धया र्द्धय॒ सꣳ स म॑र्द्धय । \newline
57. अ॒र्द्ध॒ये तीत्य॑र्द्धया र्द्ध॒येति॑ । \newline
58. इत्या॑हा॒हे तीत्या॑ह । \newline

\textbf{Ghana Paata } \newline

1. अ॒ग्नि स्तेन॒ तेना॒ग्नि र॒ग्नि स्तेनै॒वैव तेना॒ग्नि र॒ग्नि स्तेनै॒व । \newline
2. तेनै॒वैव तेन॒ तेनै॒व य॑ज्ञ्मु॒खाद् य॑ज्ञ्मु॒खा दे॒व तेन॒ तेनै॒व य॑ज्ञ्मु॒खात् । \newline
3. ए॒व य॑ज्ञ्मु॒खाद् य॑ज्ञ्मु॒खा दे॒वैव य॑ज्ञ्मु॒खा दृद्ध्या॒ ऋद्ध्या॑ यज्ञ्मु॒खा दे॒वैव य॑ज्ञ्मु॒खा दृद्ध्याः᳚ । \newline
4. य॒ज्ञ्॒मु॒खा दृद्ध्या॒ ऋद्ध्या॑ यज्ञ्मु॒खाद् य॑ज्ञ्मु॒खा दृद्ध्या॑ अ॒ग्ने र॒ग्नेर्. ऋद्ध्या॑ यज्ञ्मु॒खाद् य॑ज्ञ्मु॒खा दृद्ध्या॑ अ॒ग्नेः । \newline
5. य॒ज्ञ्॒मु॒खादिति॑ यज्ञ् - मु॒खात् । \newline
6. ऋद्ध्या॑ अ॒ग्ने र॒ग्नेर्. ऋद्ध्या॒ ऋद्ध्या॑ अ॒ग्नेर् दे॒वता॑यै दे॒वता॑या अ॒ग्नेर्. ऋद्ध्या॒ ऋद्ध्या॑ अ॒ग्नेर् दे॒वता॑यै । \newline
7. अ॒ग्नेर् दे॒वता॑यै दे॒वता॑या अ॒ग्ने र॒ग्नेर् दे॒वता॑यै॒ न न दे॒वता॑या अ॒ग्ने र॒ग्नेर् दे॒वता॑यै॒ न । \newline
8. दे॒वता॑यै॒ न न दे॒वता॑यै दे॒वता॑यै॒ नैत्ये॑ति॒ न दे॒वता॑यै दे॒वता॑यै॒ नैति॑ । \newline
9. नैत्ये॑ति॒ न नैत्य॒ष्टा व॒ष्टा वे॑ति॒ न नैत्य॒ष्टौ । \newline
10. ए॒त्य॒ष्टा व॒ष्टा वे᳚त्ये त्य॒ष्टौ सा॑वि॒त्राणि॑ सावि॒त्रा ण्य॒ष्टा वे᳚त्ये त्य॒ष्टौ सा॑वि॒त्राणि॑ । \newline
11. अ॒ष्टौ सा॑वि॒त्राणि॑ सावि॒त्रा ण्य॒ष्टा व॒ष्टौ सा॑वि॒त्राणि॑ भवन्ति भवन्ति सावि॒त्रा ण्य॒ष्टा व॒ष्टौ सा॑वि॒त्राणि॑ भवन्ति । \newline
12. सा॒वि॒त्राणि॑ भवन्ति भवन्ति सावि॒त्राणि॑ सावि॒त्राणि॑ भव॒ न्त्याहु॑ति॒ राहु॑तिर् भवन्ति सावि॒त्राणि॑ सावि॒त्राणि॑ भव॒ न्त्याहु॑तिः । \newline
13. भ॒व॒ न्त्याहु॑ति॒ राहु॑तिर् भवन्ति भव॒ न्त्याहु॑तिर् नव॒मी न॑व॒ म्याहु॑तिर् भवन्ति भव॒ न्त्याहु॑तिर् नव॒मी । \newline
14. आहु॑तिर् नव॒मी न॑व॒ म्याहु॑ति॒ राहु॑तिर् नव॒मी त्रि॒वृत॑म् त्रि॒वृत॑म् नव॒म्याहु॑ति॒ राहु॑तिर् नव॒मी त्रि॒वृत᳚म् । \newline
15. आहु॑ति॒रित्या - हु॒तिः॒ । \newline
16. न॒व॒मी त्रि॒वृत॑म् त्रि॒वृत॑म् नव॒मी न॑व॒मी त्रि॒वृत॑ मे॒वैव त्रि॒वृत॑म् नव॒मी न॑व॒मी त्रि॒वृत॑ मे॒व । \newline
17. त्रि॒वृत॑ मे॒वैव त्रि॒वृत॑म् त्रि॒वृत॑ मे॒व य॑ज्ञ्मु॒खे य॑ज्ञ्मु॒ख ए॒व त्रि॒वृत॑म् त्रि॒वृत॑ मे॒व य॑ज्ञ्मु॒खे । \newline
18. त्रि॒वृत॒मिति॑ त्रि - वृत᳚म् । \newline
19. ए॒व य॑ज्ञ्मु॒खे य॑ज्ञ्मु॒ख ए॒वैव य॑ज्ञ्मु॒खे वि वि य॑ज्ञ्मु॒ख ए॒वैव य॑ज्ञ्मु॒खे वि । \newline
20. य॒ज्ञ्॒मु॒खे वि वि य॑ज्ञ्मु॒खे य॑ज्ञ्मु॒खे वि या॑तयति यातयति॒ वि य॑ज्ञ्मु॒खे य॑ज्ञ्मु॒खे वि या॑तयति । \newline
21. य॒ज्ञ्॒मु॒ख इति॑ यज्ञ् - मु॒खे । \newline
22. वि या॑तयति यातयति॒ वि वि या॑तयति॒ यदि॒ यदि॑ यातयति॒ वि वि या॑तयति॒ यदि॑ । \newline
23. या॒त॒य॒ति॒ यदि॒ यदि॑ यातयति यातयति॒ यदि॑ का॒मये॑त का॒मये॑त॒ यदि॑ यातयति यातयति॒ यदि॑ का॒मये॑त । \newline
24. यदि॑ का॒मये॑त का॒मये॑त॒ यदि॒ यदि॑ का॒मये॑त॒ छन्दाꣳ॑सि॒ छन्दाꣳ॑सि का॒मये॑त॒ यदि॒ यदि॑ का॒मये॑त॒ छन्दाꣳ॑सि । \newline
25. का॒मये॑त॒ छन्दाꣳ॑सि॒ छन्दाꣳ॑सि का॒मये॑त का॒मये॑त॒ छन्दाꣳ॑सि यज्ञ्यश॒सेन॑ यज्ञ्यश॒सेन॒ छन्दाꣳ॑सि का॒मये॑त का॒मये॑त॒ छन्दाꣳ॑सि यज्ञ्यश॒सेन॑ । \newline
26. छन्दाꣳ॑सि यज्ञ्यश॒सेन॑ यज्ञ्यश॒सेन॒ छन्दाꣳ॑सि॒ छन्दाꣳ॑सि यज्ञ्यश॒सेना᳚ र्पयेय मर्पयेयं ॅयज्ञ्यश॒सेन॒ छन्दाꣳ॑सि॒ छन्दाꣳ॑सि यज्ञ्यश॒सेना᳚ र्पयेयम् । \newline
27. य॒ज्ञ्॒य॒श॒सेना᳚ र्पयेय मर्पयेयं ॅयज्ञ्यश॒सेन॑ यज्ञ्यश॒सेना᳚ र्पयेय॒ मिती त्य॑र्पयेयं ॅयज्ञ्यश॒सेन॑ यज्ञ्यश॒सेना᳚ र्पयेय॒ मिति॑ । \newline
28. य॒ज्ञ्॒य॒श॒सेनेति॑ यज्ञ् - य॒श॒सेन॑ । \newline
29. अ॒र्प॒ये॒य॒ मिती त्य॑र्पयेय मर्पयेय॒ मित्यृच॒ मृच॒ मित्य॑र्पयेय मर्पयेय॒ मित्यृच᳚म् । \newline
30. इत्यृच॒ मृच॒ मिती त्यृच॑ मन्त॒मा म॑न्त॒मा मृच॒ मिती त्यृच॑ मन्त॒माम् । \newline
31. ऋच॑ मन्त॒मा म॑न्त॒मा मृच॒ मृच॑ मन्त॒माम् कु॑र्यात् कुर्या दन्त॒मा मृच॒ मृच॑ मन्त॒माम् कु॑र्यात् । \newline
32. अ॒न्त॒माम् कु॑र्यात् कुर्या दन्त॒मा म॑न्त॒माम् कु॑र्या॒च् छन्दाꣳ॑सि॒ छन्दाꣳ॑सि कुर्या दन्त॒मा म॑न्त॒माम् कु॑र्या॒च् छन्दाꣳ॑सि । \newline
33. कु॒र्या॒च् छन्दाꣳ॑सि॒ छन्दाꣳ॑सि कुर्यात् कुर्या॒च् छन्दाꣳ॑ स्ये॒वैव छन्दाꣳ॑सि कुर्यात् कुर्या॒च् छन्दाꣳ॑ स्ये॒व । \newline
34. छन्दाꣳ॑ स्ये॒वैव छन्दाꣳ॑सि॒ छन्दाꣳ॑ स्ये॒व य॑ज्ञ्यश॒सेन॑ यज्ञ्यश॒सेनै॒व छन्दाꣳ॑सि॒ छन्दाꣳ॑ स्ये॒व य॑ज्ञ्यश॒सेन॑ । \newline
35. ए॒व य॑ज्ञ्यश॒सेन॑ यज्ञ्यश॒से नै॒वैव य॑ज्ञ्यश॒सेना᳚ र्पय त्यर्पयति यज्ञ्यश॒से नै॒वैव य॑ज्ञ्यश॒सेना᳚ र्पयति । \newline
36. य॒ज्ञ्॒य॒श॒सेना᳚ र्पय त्यर्पयति यज्ञ्यश॒सेन॑ यज्ञ्यश॒सेना᳚ र्पयति॒ यदि॒ यद्य॑र्पयति यज्ञ्यश॒सेन॑ यज्ञ्यश॒सेना᳚ र्पयति॒ यदि॑ । \newline
37. य॒ज्ञ्॒य॒श॒सेनेति॑ यज्ञ् - य॒श॒सेन॑ । \newline
38. अ॒र्प॒य॒ति॒ यदि॒ यद्य॑र्पय त्यर्पयति॒ यदि॑ का॒मये॑त का॒मये॑त॒ यद्य॑र्पय त्यर्पयति॒ यदि॑ का॒मये॑त । \newline
39. यदि॑ का॒मये॑त का॒मये॑त॒ यदि॒ यदि॑ का॒मये॑त॒ यज॑मानं॒ ॅयज॑मानम् का॒मये॑त॒ यदि॒ यदि॑ का॒मये॑त॒ यज॑मानम् । \newline
40. का॒मये॑त॒ यज॑मानं॒ ॅयज॑मानम् का॒मये॑त का॒मये॑त॒ यज॑मानं ॅयज्ञ्यश॒सेन॑ यज्ञ्यश॒सेन॒ यज॑मानम् का॒मये॑त का॒मये॑त॒ यज॑मानं ॅयज्ञ्यश॒सेन॑ । \newline
41. यज॑मानं ॅयज्ञ्यश॒सेन॑ यज्ञ्यश॒सेन॒ यज॑मानं॒ ॅयज॑मानं ॅयज्ञ्यश॒सेना᳚ र्पयेय मर्पयेयं ॅयज्ञ्यश॒सेन॒ यज॑मानं॒ ॅयज॑मानं ॅयज्ञ्यश॒सेना᳚ र्पयेयम् । \newline
42. य॒ज्ञ्॒य॒श॒सेना᳚ र्पयेय मर्पयेयं ॅयज्ञ्यश॒सेन॑ यज्ञ्यश॒सेना᳚ र्पयेय॒ मिती त्य॑र्पयेयं ॅयज्ञ्यश॒सेन॑ यज्ञ्यश॒सेना᳚ र्पयेय॒ मिति॑ । \newline
43. य॒ज्ञ्॒य॒श॒सेनेति॑ यज्ञ् - य॒श॒सेन॑ । \newline
44. अ॒र्प॒ये॒य॒ मिती त्य॑र्पयेय मर्पयेय॒ मिति॒ यजु॒र् यजु॒रि त्य॑र्पयेय मर्पयेय॒ मिति॒ यजुः॑ । \newline
45. इति॒ यजु॒र् यजु॒ रितीति॒ यजु॑ रन्त॒म म॑न्त॒मं ॅयजु॒ रितीति॒ यजु॑ रन्त॒मम् । \newline
46. यजु॑ रन्त॒म म॑न्त॒मं ॅयजु॒र् यजु॑ रन्त॒मम् कु॑र्यात् कुर्या दन्त॒मं ॅयजु॒र् यजु॑ रन्त॒मम् कु॑र्यात् । \newline
47. अ॒न्त॒मम् कु॑र्यात् कुर्या दन्त॒म म॑न्त॒मम् कु॑र्या॒द् यज॑मानं॒ ॅयज॑मानम् कुर्या दन्त॒म म॑न्त॒मम् कु॑र्या॒द् यज॑मानम् । \newline
48. कु॒र्या॒द् यज॑मानं॒ ॅयज॑मानम् कुर्यात् कुर्या॒द् यज॑मान मे॒वैव यज॑मानम् कुर्यात् कुर्या॒द् यज॑मान मे॒व । \newline
49. यज॑मान मे॒वैव यज॑मानं॒ ॅयज॑मान मे॒व य॑ज्ञ्यश॒सेन॑ यज्ञ्यश॒सेनै॒व यज॑मानं॒ ॅयज॑मान मे॒व य॑ज्ञ्यश॒सेन॑ । \newline
50. ए॒व य॑ज्ञ्यश॒सेन॑ यज्ञ्यश॒से नै॒वैव य॑ज्ञ्यश॒सेना᳚ र्पय त्यर्पयति यज्ञ्यश॒से नै॒वैव य॑ज्ञ्यश॒सेना᳚ र्पयति । \newline
51. य॒ज्ञ्॒य॒श॒सेना᳚ र्पय त्यर्पयति यज्ञ्यश॒सेन॑ यज्ञ्यश॒सेना᳚ र्पय त्यृ॒च र्‌चा ऽर्प॑यति यज्ञ्यश॒सेन॑ यज्ञ्यश॒सेना᳚ र्पय त्यृ॒चा । \newline
52. य॒ज्ञ्॒य॒श॒सेनेति॑ यज्ञ् - य॒श॒सेन॑ । \newline
53. अ॒र्प॒य॒ त्यृ॒च र्‌चा ऽर्प॑य त्यर्पय त्यृ॒चा स्तोमꣳ॒॒ स्तोम॑ मृ॒चा ऽर्प॑य त्यर्पय त्यृ॒चा स्तोम᳚म् । \newline
54. ऋ॒चा स्तोमꣳ॒॒ स्तोम॑ मृ॒च र्‌चा स्तोमꣳ॒॒ सꣳ सꣳ स्तोम॑ मृ॒च र्‌चा स्तोमꣳ॒॒ सम् । \newline
55. स्तोमꣳ॒॒ सꣳ सꣳ स्तोमꣳ॒॒ स्तोमꣳ॒॒ सम॑र्द्धया र्द्धय॒ सꣳ स्तोमꣳ॒॒ स्तोमꣳ॒॒ 
सम॑र्द्धय । \newline
56. स म॑र्द्धया र्द्धय॒ सꣳ स म॑र्द्ध॒ये तीत्य॑र्द्धय॒ सꣳ स म॑र्द्ध॒येति॑ । \newline
57. अ॒र्द्ध॒येती त्य॑र्द्धया र्द्ध॒ये त्या॑हा॒हे त्य॑र्द्धया र्द्ध॒ये त्या॑ह । \newline
58. इत्या॑हा॒हे तीत्या॑ह॒ समृ॑द्ध्यै॒ समृ॑द्ध्या आ॒हे तीत्या॑ह॒ समृ॑द्ध्यै । \newline
\pagebreak
\markright{ TS 5.1.1.4  \hfill https://www.vedavms.in \hfill}

\section{ TS 5.1.1.4 }

\textbf{TS 5.1.1.4 } \newline
\textbf{Samhita Paata} \newline

-ह॒ समृ॑द्ध्यै च॒तुर्भि॒रभ्रि॒मा द॑त्ते च॒त्वारि॒ छन्दाꣳ॑सि॒ छन्दो॑भिरे॒व दे॒वस्य॑ त्वा सवि॒तुः प्र॑स॒व इत्या॑ह॒ प्रसू᳚त्या अ॒ग्निर्दे॒वेभ्यो॒ निला॑यत॒ स वेणुं॒ प्राऽवि॑श॒थ् स ए॒तामू॒तिमनु॒ सम॑चर॒द्-यद्-वेणोः᳚ सुषि॒रꣳ सु॑षि॒रा-ऽभ्रि॑र्भवति सयोनि॒त्वाय॒ स यत्र॑य॒त्राऽव॑स॒त् तत् कृ॒ष्णम॑भवत् कल्मा॒षी भ॑वति रू॒पस॑मृद्ध्या उभयतः॒ क्ष्णूर्भ॑वती॒तश्चा॒ऽ-( ) -मुत॑श्चा॒ऽर्कस्या-व॑रुद्ध्यै व्याममा॒त्री भ॑वत्ये॒ताव॒द्वै पुरु॑षे वी॒र्यं॑ ॅवी॒र्य॑सम्मि॒ता ऽप॑रिमिता भव॒त्य-प॑रिमित॒स्याऽ व॑रुद्ध्यै॒ यो वन॒स्पती॑नां फल॒ग्रहिः॒ स ए॑षां ॅवी॒र्या॑वान् फल॒ग्रहि॒र्वेणु॑-र्वैण॒वी भ॑वति वी॒र्य॑स्या व॑रुद्ध्यै ॥ \newline

\textbf{Pada Paata} \newline

आ॒ह॒ । समृ॑द्ध्या॒ इति॒ सं - ऋ॒द्ध्यै॒ । च॒तुभि॒रिति॑ च॒तुः - भिः॒ । अभ्रि᳚म् । एति॑ । द॒त्ते॒ । च॒त्वारि॑ । छन्दाꣳ॑सि । छन्दो॑भि॒रिति॒ छन्दः॑ - भिः॒ । ए॒व । दे॒वस्य॑ । त्वा॒ । स॒वि॒तुः । प्र॒स॒व इति॑ प्र - स॒वे । इति॑ । आ॒ह॒ । प्रसू᳚त्या॒ इति॒ प्र - सू॒त्यै॒ । अ॒ग्निः । दे॒वेभ्यः॑ । निला॑यत । सः । वेणु᳚म् । प्रेति॑ । अ॒वि॒श॒त् । सः । ए॒ताम् । ऊ॒तिम् । अनु॑ । समिति॑ । अ॒च॒र॒त् । यत् । वेणोः᳚ । सु॒षि॒रम् । सु॒षि॒रा । अभ्रिः॑ । भ॒व॒ति॒ । स॒यो॒नि॒त्वायेति॑ सयोनि - त्वाय॑ । सः । यत्र॑य॒त्रेति॒ यत्र॑ - य॒त्र॒ । अव॑सत् । तत् । कृ॒ष्णम् । अ॒भ॒व॒त् । क॒ल्मा॒षी । भ॒व॒ति॒ । रू॒पस॑मृद्ध्या॒ इति॑ रू॒प - स॒मृ॒द्ध्यै॒ । उ॒भ॒य॒तः॒ क्ष्णूरित्यु॑भयतः-क्ष्णूः । भ॒व॒ति॒ । इ॒तः । च॒ ( ) । अ॒मुतः॑ । च॒ । अ॒र्कस्य॑ । अव॑रुद्ध्या॒ इत्यव॑ - रु॒द्ध्यै॒ । व्या॒म॒मा॒त्रीति॑ व्याम - मा॒त्री । भ॒व॒ति॒ । ए॒ताव॑त् । वै । पुरु॑षे । वी॒र्य᳚म् । वी॒र्य॑सम्मि॒तेति॑ वी॒र्य॑ - स॒म्मि॒ता॒ । अप॑रिमि॒तेत्यप॑रि-मि॒ता॒ । भ॒व॒ति॒ । अप॑रिमित॒स्येत्यप॑रि - मि॒त॒स्य॒ । अव॑रुद्ध्या॒ इत्यव॑ - रु॒द्ध्यै॒ । यः । वन॒स्पती॑नाम् । फ॒ल॒ग्रहि॒रिति॑ फल - ग्रहिः॑ । सः । ए॒षा॒म् । वी॒र्या॑वा॒निति॑ वी॒र्य॑ - वा॒न् । फ॒ल॒ग्रहि॒रिति॑ फल - ग्रहिः॑ । वेणुः॑ । वै॒ण॒वी । भ॒व॒ति॒ । वी॒र्य॑स्य । अव॑रुद्ध्या॒ इत्यव॑ - रु॒द्ध्यै॒ ॥  \newline


\textbf{Krama Paata} \newline

आ॒ह॒ समृ॑द्ध्यै । समृ॑द्ध्यै च॒तुर्भिः॑ । समृ॑द्ध्या॒ इति॒ सम् - ऋ॒द्ध्यै॒ । च॒तुर्भि॒रभ्रि᳚म् । च॒तुर्भि॒रिति॑ च॒तुः - भिः॒ । अभ्रि॒मा । आ द॑त्ते । द॒त्ते॒ च॒त्वारि॑ । च॒त्वारि॒ छन्दाꣳ॑सि । छन्दाꣳ॑सि॒ छन्दो॑भिः । छन्दो॑भिरे॒व । छन्दो॑भि॒रिति॒ छन्दः॑ - भिः॒ । ए॒व दे॒वस्य॑ । दे॒वस्य॑ त्वा । त्वा॒ स॒वि॒तुः । स॒वि॒तुः प्र॑स॒वे । प्र॒स॒व इति॑ । प्र॒स॒व इति॑ प्र - स॒वे । इत्या॑ह । आ॒ह॒ प्रसू᳚त्यै । 
प्रसू᳚त्या अ॒ग्निः । प्रसू᳚त्या॒ इति॒ प्र - सू॒त्यै॒ । अ॒ग्निर् दे॒वेभ्यः॑ । दे॒वेभ्यो॒ निला॑यत । निला॑यत॒ सः । स वेणु᳚म् । वेणु॒म् प्र । प्रावि॑शत् । अ॒वि॒श॒थ् सः । स ए॒ताम् । ए॒तामू॒तिम् । ऊ॒तिमनु॑ । अनु॒ सम् । 
सम॑चरत् । अ॒च॒र॒द् यत् । यद् वेणोः᳚ । वेणोः᳚ सुषि॒रम् । सु॒षि॒रꣳ सु॑षि॒रा । सु॒षि॒राऽभ्रिः॑ । अभ्रि॑र् भवति । भ॒व॒ति॒ स॒यो॒नि॒त्वाय॑ । स॒यो॒नि॒त्वाय॒ सः । स॒यो॒नि॒त्वायेति॑ सयोनि - त्वाय॑ । स यत्र॑यत्र । यत्र॑य॒त्राव॑सत् । यत्र॑य॒त्रेति॒ यत्र॑ - य॒त्र॒ । अव॑स॒त् तत् । तत् कृ॒ष्णम् । कृ॒ष्णम॑भवत् । अ॒भ॒व॒त् क॒ल्मा॒षी । क॒ल्मा॒षी भ॑वति । भ॒व॒ति॒ रू॒पस॑मृद्ध्यै । रू॒पस॑मृद्ध्या उभयतः॒क्ष्णूः । रू॒पस॑मृद्ध्या॒ इति॑ रू॒प - स॒मृ॒द्ध्यै॒ । उ॒भ॒य॒तः॒क्ष्णूर् भ॑वति । उ॒भ॒य॒तः॒क्ष्णूरित्यु॑भयतः - क्ष्णूः । भ॒व॒ती॒तः । इ॒तश्च॑ ( ) । चा॒मुतः॑ । अ॒मुत॑श्च । चा॒र्कस्य॑ । अ॒र्कस्याव॑रुद्ध्यै । अव॑रुद्ध्यै व्याममा॒त्री । अव॑रुद्ध्या॒ इत्यव॑ - रु॒द्ध्यै॒ । व्या॒म॒मा॒त्री भ॑वति । व्या॒म॒मा॒त्रीति॑ व्याम - मा॒त्री । भ॒व॒त्ये॒ताव॑त् । ए॒ताव॒द् वै । वै पुरु॑षे । पुरु॑षे वी॒र्य᳚म् । वी॒र्य॑म् ॅवी॒र्य॑सम्मिता । वी॒र्य॑सम्मि॒ताऽप॑रिमिता । वी॒र्य॑सम्मि॒तेति॑ वी॒र्य॑ - स॒म्मि॒ता॒ । अप॑रिमिता भवति । अप॑रिमि॒तेत्यप॑रि - मि॒ता॒ । भ॒व॒त्यप॑रिमितस्य । अप॑रिमित॒स्याव॑रुद्ध्यै । अप॑रिमित॒स्येत्यप॑रि - मि॒त॒स्य॒ । अव॑रुद्ध्यै॒ यः । अव॑रुद्ध्या॒ इत्यव॑ - रु॒द्ध्यै॒ । यो वन॒स्पती॑नाम् । वन॒स्पती॑नाम् फल॒ग्रहिः॑ । फ॒ल॒ग्रहिः॒ सः । फ॒ल॒ग्रहि॒रिति॑ फल - ग्रहिः॑ । स ए॑षाम् । ए॒षा॒म् ॅवी॒र्या॑वान् । वी॒र्या॑वान् फल॒ग्रहिः॑ । वी॒र्या॑वा॒निति॑ वी॒र्य॑ - वा॒न्॒ । फ॒ल॒ग्रहि॒र् वेणुः॑ । फ॒ल॒ग्रहि॒रिति॑ फल - ग्रहिः॑ । वेणु॑र् वैण॒वी । वै॒ण॒वी भ॑वति । भ॒व॒ति॒ वी॒र्य॑स्य । वी॒र्य॑स्याव॑रुद्ध्यै । अव॑रुद्ध्या॒ इत्यव॑ - रु॒द्ध्यै॒ । \newline

\textbf{Jatai Paata} \newline

1. आ॒ह॒ समृ॑द्ध्यै॒ समृ॑द्ध्या आहाह॒ समृ॑द्ध्यै । \newline
2. समृ॑द्ध्यै च॒तुर्भि॑ श्च॒तुर्भिः॒ समृ॑द्ध्यै॒ समृ॑द्ध्यै च॒तुर्भिः॑ । \newline
3. समृ॑द्ध्या॒ इति॒ सं - ऋ॒द्ध्यै॒ । \newline
4. च॒तुर्भि॒ रभ्रि॒ मभ्रि॑म् च॒तुर्भि॑ श्च॒तुर्भि॒ रभ्रि᳚म् । \newline
5. च॒तुर्भि॒रिति॑ च॒तुः - भिः॒ । \newline
6. अभ्रि॒ मा ऽभ्रि॒ मभ्रि॒ मा । \newline
7. आ द॑त्ते दत्त॒ आ द॑त्ते । \newline
8. द॒त्ते॒ च॒त्वारि॑ च॒त्वारि॑ दत्ते दत्ते च॒त्वारि॑ । \newline
9. च॒त्वारि॒ छन्दाꣳ॑सि॒ छन्दाꣳ॑सि च॒त्वारि॑ च॒त्वारि॒ छन्दाꣳ॑सि । \newline
10. छन्दाꣳ॑सि॒ छन्दो॑भि॒ श्छन्दो॑भि॒ श्छन्दाꣳ॑सि॒ छन्दाꣳ॑सि॒ छन्दो॑भिः । \newline
11. छन्दो॑भि रे॒वैव छन्दो॑भि॒ श्छन्दो॑भि रे॒व । \newline
12. छन्दो॑भि॒रिति॒ छन्दः॑ - भिः॒ । \newline
13. ए॒व दे॒वस्य॑ दे॒वस्यै॒ वैव दे॒वस्य॑ । \newline
14. दे॒वस्य॑ त्वा त्वा दे॒वस्य॑ दे॒वस्य॑ त्वा । \newline
15. त्वा॒ स॒वि॒तुः स॑वि॒तु स्त्वा᳚ त्वा सवि॒तुः । \newline
16. स॒वि॒तुः प्र॑स॒वे प्र॑स॒वे स॑वि॒तुः स॑वि॒तुः प्र॑स॒वे । \newline
17. प्र॒स॒व इतीति॑ प्रस॒वे प्र॑स॒व इति॑ । \newline
18. प्र॒स॒व इति॑ प्र - स॒वे । \newline
19. इत्या॑हा॒हे तीत्या॑ह । \newline
20. आ॒ह॒ प्रसू᳚त्यै॒ प्रसू᳚त्या आहाह॒ प्रसू᳚त्यै । \newline
21. प्रसू᳚त्या अ॒ग्नि र॒ग्निः प्रसू᳚त्यै॒ प्रसू᳚त्या अ॒ग्निः । \newline
22. प्रसू᳚त्या॒ इति॒ प्र - सू॒त्यै॒ । \newline
23. अ॒ग्निर् दे॒वेभ्यो॑ दे॒वेभ्यो॒ ऽग्नि र॒ग्निर् दे॒वेभ्यः॑ । \newline
24. दे॒वेभ्यो॒ निला॑यत॒ निला॑यत दे॒वेभ्यो॑ दे॒वेभ्यो॒ निला॑यत । \newline
25. निला॑यत॒ स स निला॑यत॒ निला॑यत॒ सः । \newline
26. स वेणुं॒ ॅवेणुꣳ॒॒ स स वेणु᳚म् । \newline
27. वेणु॒म् प्र प्र वेणुं॒ ॅवेणु॒म् प्र । \newline
28. प्रावि॑श दविश॒त् प्र प्रावि॑शत् । \newline
29. अ॒वि॒श॒थ् स सो॑ ऽविश दविश॒थ् सः । \newline
30. स ए॒ता मे॒ताꣳ स स ए॒ताम् । \newline
31. ए॒ता मू॒ति मू॒ति मे॒ता मे॒ता मू॒तिम् । \newline
32. ऊ॒ति मन्वनू॒ति मू॒ति मनु॑ । \newline
33. अनु॒ सꣳ स मन्वनु॒ सम् । \newline
34. स म॑चर दचर॒थ् सꣳ स म॑चरत् । \newline
35. अ॒च॒र॒द् यद् यद॑चर दचर॒द् यत् । \newline
36. यद् वेणो॒र् वेणो॒र् यद् यद् वेणोः᳚ । \newline
37. वेणोः᳚ सुषि॒रꣳ सु॑षि॒रं ॅवेणो॒र् वेणोः᳚ सुषि॒रम् । \newline
38. सु॒षि॒रꣳ सु॑षि॒रा सु॑षि॒रा सु॑षि॒रꣳ सु॑षि॒रꣳ सु॑षि॒रा । \newline
39. सु॒षि॒रा ऽभ्रि॒ रभ्रिः॑ सुषि॒रा सु॑षि॒रा ऽभ्रिः॑ । \newline
40. अभ्रि॑र् भवति भव॒ त्यभ्रि॒ रभ्रि॑र् भवति । \newline
41. भ॒व॒ति॒ स॒यो॒नि॒त्वाय॑ सयोनि॒त्वाय॑ भवति भवति सयोनि॒त्वाय॑ । \newline
42. स॒यो॒नि॒त्वाय॒ स स स॑योनि॒त्वाय॑ सयोनि॒त्वाय॒ सः । \newline
43. स॒यो॒नि॒त्वायेति॑ सयोनि - त्वाय॑ । \newline
44. स यत्र॑यत्र॒ यत्र॑यत्र॒ स स यत्र॑यत्र । \newline
45. यत्र॑य॒त्रा व॑स॒ दव॑स॒द् यत्र॑यत्र॒ यत्र॑य॒त्रा व॑सत् । \newline
46. यत्र॑य॒त्रेति॒ यत्र॑ - य॒त्र॒ । \newline
47. अव॑स॒त् तत् तदव॑स॒ दव॑स॒त् तत् । \newline
48. तत् कृ॒ष्णम् कृ॒ष्णम् तत् तत् कृ॒ष्णम् । \newline
49. कृ॒ष्ण म॑भव दभवत् कृ॒ष्णम् कृ॒ष्ण म॑भवत् । \newline
50. अ॒भ॒व॒त् क॒ल्मा॒षी क॑ल्मा॒ ष्य॑भव दभवत् कल्मा॒षी । \newline
51. क॒ल्मा॒षी भ॑वति भवति कल्मा॒षी क॑ल्मा॒षी भ॑वति । \newline
52. भ॒व॒ति॒ रू॒पस॑मृद्ध्यै रू॒पस॑मृद्ध्यै भवति भवति रू॒पस॑मृद्ध्यै । \newline
53. रू॒पस॑मृद्ध्या उभयतः॒क्ष्णू रु॑भयतः॒क्ष्णू रू॒पस॑मृद्ध्यै रू॒पस॑मृद्ध्या उभयतः॒क्ष्णूः । \newline
54. रू॒पस॑मृद्ध्या॒ इति॑ रू॒प - स॒मृ॒द्ध्यै॒ । \newline
55. उ॒भ॒य॒तः॒क्ष्णूर् भ॑वति भव त्युभयतः॒क्ष्णू रु॑भयतः॒क्ष्णूर् भ॑वति । \newline
56. उ॒भ॒य॒तः॒क्ष्णूरित्यु॑भयतः - क्ष्णूः । \newline
57. भ॒व॒ती॒त इ॒तो भ॑वति भवती॒तः । \newline
58. इ॒तश्च॑ चे॒ त इ॒तश्च॑ । \newline
59. चा॒मुतो॒ ऽमुत॑श्च चा॒मुतः॑ । \newline
60. अ॒मुत॑श्च चा॒मुतो॒ ऽमुत॑श्च । \newline
61. चा॒र्क स्या॒र्कस्य॑ च चा॒र्कस्य॑ । \newline
62. अ॒र्कस्या व॑रुद्ध्या॒ अव॑रुद्ध्या अ॒र्क स्या॒र्कस्या व॑रुद्ध्यै । \newline
63. अव॑रुद्ध्यै व्याममा॒त्री व्या॑ममा॒ त्र्यव॑रुद्ध्या॒ अव॑रुद्ध्यै व्याममा॒त्री । \newline
64. अव॑रुद्ध्या॒ इत्यव॑ - रु॒द्ध्यै॒ । \newline
65. व्या॒म॒मा॒त्री भ॑वति भवति व्याममा॒त्री व्या॑ममा॒त्री भ॑वति । \newline
66. व्या॒म॒मा॒त्रीति॑ व्याम - मा॒त्री । \newline
67. भ॒व॒ त्ये॒ताव॑ दे॒ताव॑द् भवति भव त्ये॒ताव॑त् । \newline
68. ए॒ताव॒द् वै वा ए॒ताव॑ दे॒ताव॒द् वै । \newline
69. वै पुरु॑षे॒ पुरु॑षे॒ वै वै पुरु॑षे । \newline
70. पुरु॑षे वी॒र्यं॑ ॅवी॒र्य॑म् पुरु॑षे॒ पुरु॑षे वी॒र्य᳚म् । \newline
71. वी॒र्यं॑ ॅवी॒र्य॑सम्मिता वी॒र्य॑सम्मिता वी॒र्यं॑ ॅवी॒र्यं॑ ॅवी॒र्य॑सम्मिता । \newline
72. वी॒र्य॑सम्मि॒ता ऽप॑रिमि॒ता ऽप॑रिमिता वी॒र्य॑सम्मिता वी॒र्य॑सम्मि॒ता ऽप॑रिमिता । \newline
73. वी॒र्य॑सम्मि॒तेति॑ वी॒र्य॑ - स॒म्मि॒ता॒ । \newline
74. अप॑रिमिता भवति भव॒ त्यप॑रिमि॒ता ऽप॑रिमिता भवति । \newline
75. अप॑रिमि॒तेत्यप॑रि - मि॒ता॒ । \newline
76. भ॒व॒ त्यप॑रिमित॒स्या प॑रिमितस्य भवति भव॒ त्यप॑रिमितस्य । \newline
77. अप॑रिमित॒स्या व॑रुद्ध्या॒ अव॑रुद्ध्या॒ अप॑रिमित॒स्या प॑रिमित॒स्या व॑रुद्ध्यै । \newline
78. अप॑रिमित॒स्येत्यप॑रि - मि॒त॒स्य॒ । \newline
79. अव॑रुद्ध्यै॒ यो यो ऽव॑रुद्ध्या॒ अव॑रुद्ध्यै॒ यः । \newline
80. अव॑रुद्ध्या॒ इत्यव॑ - रु॒द्ध्यै॒ । \newline
81. यो वन॒स्पती॑नां॒ ॅवन॒स्पती॑नां॒ ॅयो यो वन॒स्पती॑नाम् । \newline
82. वन॒स्पती॑नाम् फल॒ग्रहिः॑ फल॒ग्रहि॒र् वन॒स्पती॑नां॒ ॅवन॒स्पती॑नाम् फल॒ग्रहिः॑ । \newline
83. फ॒ल॒ग्रहिः॒ स स फ॑ल॒ग्रहिः॑ फल॒ग्रहिः॒ सः । \newline
84. फ॒ल॒ग्रहि॒रिति॑ फल - ग्रहिः॑ । \newline
85. स ए॑षा मेषाꣳ॒॒ स स ए॑षाम् । \newline
86. ए॒षां॒ ॅवी॒र्या॑वान्. वी॒र्या॑वा नेषा मेषां ॅवी॒र्या॑वान् । \newline
87. वी॒र्या॑वान् फल॒ग्रहिः॑ फल॒ग्रहि॑र् वी॒र्या॑वान्. वी॒र्या॑वान् फल॒ग्रहिः॑ । \newline
88. वी॒र्या॑वा॒निति॑ वी॒र्य॑ - वा॒न् । \newline
89. फ॒ल॒ग्रहि॒र् वेणु॒र् वेणुः॑ फल॒ग्रहिः॑ फल॒ग्रहि॒र् वेणुः॑ । \newline
90. फ॒ल॒ग्रहि॒रिति॑ फल - ग्रहिः॑ । \newline
91. वेणु॑र् वैण॒वी वै॑ण॒वी वेणु॒र् वेणु॑र् वैण॒वी । \newline
92. वै॒ण॒वी भ॑वति भवति वैण॒वी वै॑ण॒वी भ॑वति । \newline
93. भ॒व॒ति॒ वी॒र्य॑स्य वी॒र्य॑स्य भवति भवति वी॒र्य॑स्य । \newline
94. वी॒र्य॑स्या व॑रुद्ध्या॒ अव॑रुद्ध्यै वी॒र्य॑स्य वी॒र्य॑स्या व॑रुद्ध्यै । \newline
95. अव॑रुद्ध्या॒ इत्यव॑ - रु॒द्ध्यै॒ । \newline

\textbf{Ghana Paata } \newline

1. आ॒ह॒ समृ॑द्ध्यै॒ समृ॑द्ध्या आहाह॒ समृ॑द्ध्यै च॒तुर्भि॑ श्च॒तुर्भिः॒ समृ॑द्ध्या आहाह॒ समृ॑द्ध्यै च॒तुर्भिः॑ । \newline
2. समृ॑द्ध्यै च॒तुर्भि॑ श्च॒तुर्भिः॒ समृ॑द्ध्यै॒ समृ॑द्ध्यै च॒तुर्भि॒ रभ्रि॒ मभ्रि॑म् च॒तुर्भिः॒ समृ॑द्ध्यै॒ समृ॑द्ध्यै च॒तुर्भि॒ रभ्रि᳚म् । \newline
3. समृ॑द्ध्या॒ इति॒ सं - ऋ॒द्ध्यै॒ । \newline
4. च॒तुर्भि॒ रभ्रि॒ मभ्रि॑म् च॒तुर्भि॑ श्च॒तुर्भि॒ रभ्रि॒ मा ऽभ्रि॑म् च॒तुर्भि॑ श्च॒तुर्भि॒ रभ्रि॒ मा । \newline
5. च॒तुर्भि॒रिति॑ च॒तुः - भिः॒ । \newline
6. अभ्रि॒ मा ऽभ्रि॒ मभ्रि॒ मा द॑त्ते दत्त॒ आ ऽभ्रि॒ मभ्रि॒ मा द॑त्ते । \newline
7. आ द॑त्ते दत्त॒ आ द॑त्ते च॒त्वारि॑ च॒त्वारि॑ दत्त॒ आ द॑त्ते च॒त्वारि॑ । \newline
8. द॒त्ते॒ च॒त्वारि॑ च॒त्वारि॑ दत्ते दत्ते च॒त्वारि॒ छन्दाꣳ॑सि॒ छन्दाꣳ॑सि च॒त्वारि॑ दत्ते दत्ते च॒त्वारि॒ छन्दाꣳ॑सि । \newline
9. च॒त्वारि॒ छन्दाꣳ॑सि॒ छन्दाꣳ॑सि च॒त्वारि॑ च॒त्वारि॒ छन्दाꣳ॑सि॒ छन्दो॑भि॒ श्छन्दो॑भि॒ श्छन्दाꣳ॑सि च॒त्वारि॑ च॒त्वारि॒ छन्दाꣳ॑सि॒ छन्दो॑भिः । \newline
10. छन्दाꣳ॑सि॒ छन्दो॑भि॒ श्छन्दो॑भि॒ श्छन्दाꣳ॑सि॒ छन्दाꣳ॑सि॒ छन्दो॑भि रे॒वैव छन्दो॑भि॒ श्छन्दाꣳ॑सि॒ छन्दाꣳ॑सि॒ छन्दो॑भि रे॒व । \newline
11. छन्दो॑भि रे॒वैव छन्दो॑भि॒ श्छन्दो॑भि रे॒व दे॒वस्य॑ दे॒वस्यै॒व छन्दो॑भि॒ श्छन्दो॑भि रे॒व दे॒वस्य॑ । \newline
12. छन्दो॑भि॒रिति॒ छन्दः॑ - भिः॒ । \newline
13. ए॒व दे॒वस्य॑ दे॒वस्यै॒वैव दे॒वस्य॑ त्वा त्वा दे॒वस्यै॒वैव दे॒वस्य॑ त्वा । \newline
14. दे॒वस्य॑ त्वा त्वा दे॒वस्य॑ दे॒वस्य॑ त्वा सवि॒तुः स॑वि॒तु स्त्वा॑ दे॒वस्य॑ दे॒वस्य॑ त्वा सवि॒तुः । \newline
15. त्वा॒ स॒वि॒तुः स॑वि॒तु स्त्वा᳚ त्वा सवि॒तुः प्र॑स॒वे प्र॑स॒वे स॑वि॒तु स्त्वा᳚ त्वा सवि॒तुः प्र॑स॒वे । \newline
16. स॒वि॒तुः प्र॑स॒वे प्र॑स॒वे स॑वि॒तुः स॑वि॒तुः प्र॑स॒व इतीति॑ प्रस॒वे स॑वि॒तुः स॑वि॒तुः प्र॑स॒व इति॑ । \newline
17. प्र॒स॒व इतीति॑ प्रस॒वे प्र॑स॒व इत्या॑ हा॒हेति॑ प्रस॒वे प्र॑स॒व इत्या॑ह । \newline
18. प्र॒स॒व इति॑ प्र - स॒वे । \newline
19. इत्या॑हा॒हे तीत्या॑ह॒ प्रसू᳚त्यै॒ प्रसू᳚त्या आ॒हे तीत्या॑ह॒ प्रसू᳚त्यै । \newline
20. आ॒ह॒ प्रसू᳚त्यै॒ प्रसू᳚त्या आहाह॒ प्रसू᳚त्या अ॒ग्नि र॒ग्निः प्रसू᳚त्या आहाह॒ प्रसू᳚त्या अ॒ग्निः । \newline
21. प्रसू᳚त्या अ॒ग्नि र॒ग्निः प्रसू᳚त्यै॒ प्रसू᳚त्या अ॒ग्निर् दे॒वेभ्यो॑ दे॒वेभ्यो॒ ऽग्निः प्रसू᳚त्यै॒ प्रसू᳚त्या अ॒ग्निर् दे॒वेभ्यः॑ । \newline
22. प्रसू᳚त्या॒ इति॒ प्र - सू॒त्यै॒ । \newline
23. अ॒ग्निर् दे॒वेभ्यो॑ दे॒वेभ्यो॒ ऽग्नि र॒ग्निर् दे॒वेभ्यो॒ निला॑यत॒ निला॑यत दे॒वेभ्यो॒ ऽग्नि र॒ग्निर् दे॒वेभ्यो॒ निला॑यत । \newline
24. दे॒वेभ्यो॒ निला॑यत॒ निला॑यत दे॒वेभ्यो॑ दे॒वेभ्यो॒ निला॑यत॒ स स निला॑यत दे॒वेभ्यो॑ दे॒वेभ्यो॒ निला॑यत॒ सः । \newline
25. निला॑यत॒ स स निला॑यत॒ निला॑यत॒ स वेणुं॒ ॅवेणुꣳ॒॒ स निला॑यत॒ निला॑यत॒ स वेणु᳚म् । \newline
26. स वेणुं॒ ॅवेणुꣳ॒॒ स स वेणु॒म् प्र प्र वेणुꣳ॒॒ स स वेणु॒म् प्र । \newline
27. वेणु॒म् प्र प्र वेणुं॒ ॅवेणु॒म् प्रावि॑श दविश॒त् प्र वेणुं॒ ॅवेणु॒म् प्रावि॑शत् । \newline
28. प्रावि॑श दविश॒त् प्र प्रावि॑श॒थ् स सो॑ ऽविश॒त् प्र प्रावि॑श॒थ् सः । \newline
29. अ॒वि॒श॒थ् स सो॑ ऽविश दविश॒थ् स ए॒ता मे॒ताꣳ सो॑ ऽविश दविश॒थ् स ए॒ताम् । \newline
30. स ए॒ता मे॒ताꣳ स स ए॒ता मू॒ति मू॒ति मे॒ताꣳ स स ए॒ता मू॒तिम् । \newline
31. ए॒ता मू॒ति मू॒ति मे॒ता मे॒ता मू॒ति मन्वनू॒ति मे॒ता मे॒ता मू॒ति मनु॑ । \newline
32. ऊ॒ति मन्वनू॒ति मू॒ति मनु॒ सꣳ स मनू॒ति मू॒ति मनु॒ सम् । \newline
33. अनु॒ सꣳ स मन्वनु॒ स म॑चर दचर॒थ् स मन्वनु॒ स म॑चरत् । \newline
34. स म॑चर दचर॒थ् सꣳ स म॑चर॒द् यद् यद॑चर॒थ् सꣳ स म॑चर॒द् यत् । \newline
35. अ॒च॒र॒द् यद् यद॑चर दचर॒द् यद् वेणो॒र् वेणो॒र् यद॑चर दचर॒द् यद् वेणोः᳚ । \newline
36. यद् वेणो॒र् वेणो॒र् यद् यद् वेणोः᳚ सुषि॒रꣳ सु॑षि॒रं ॅवेणो॒र् यद् यद् वेणोः᳚ सुषि॒रम् । \newline
37. वेणोः᳚ सुषि॒रꣳ सु॑षि॒रं ॅवेणो॒र् वेणोः᳚ सुषि॒रꣳ सु॑षि॒रा सु॑षि॒रा सु॑षि॒रं ॅवेणो॒र् वेणोः᳚ सुषि॒रꣳ सु॑षि॒रा । \newline
38. सु॒षि॒रꣳ सु॑षि॒रा सु॑षि॒रा सु॑षि॒रꣳ सु॑षि॒रꣳ सु॑षि॒रा ऽभ्रि॒ रभ्रिः॑ सुषि॒रा सु॑षि॒रꣳ सु॑षि॒रꣳ सु॑षि॒रा ऽभ्रिः॑ । \newline
39. सु॒षि॒रा ऽभ्रि॒ रभ्रिः॑ सुषि॒रा सु॑षि॒रा ऽभ्रि॑र् भवति भव॒ त्यभ्रिः॑ सुषि॒रा सु॑षि॒रा ऽभ्रि॑र् भवति । \newline
40. अभ्रि॑र् भवति भव॒ त्यभ्रि॒ रभ्रि॑र् भवति सयोनि॒त्वाय॑ सयोनि॒त्वाय॑ भव॒ त्यभ्रि॒ रभ्रि॑र् भवति सयोनि॒त्वाय॑ । \newline
41. भ॒व॒ति॒ स॒यो॒नि॒त्वाय॑ सयोनि॒त्वाय॑ भवति भवति सयोनि॒त्वाय॒ स स स॑योनि॒त्वाय॑ भवति भवति सयोनि॒त्वाय॒ सः । \newline
42. स॒यो॒नि॒त्वाय॒ स स स॑योनि॒त्वाय॑ सयोनि॒त्वाय॒ स यत्र॑यत्र॒ यत्र॑यत्र॒ स स॑योनि॒त्वाय॑ सयोनि॒त्वाय॒ स यत्र॑यत्र । \newline
43. स॒यो॒नि॒त्वायेति॑ सयोनि - त्वाय॑ । \newline
44. स यत्र॑यत्र॒ यत्र॑यत्र॒ स स यत्र॑य॒त्रा व॑स॒ दव॑स॒द् यत्र॑यत्र॒ स स यत्र॑य॒त्रा व॑सत् । \newline
45. यत्र॑य॒त्रा व॑स॒ दव॑स॒द् यत्र॑यत्र॒ यत्र॑य॒त्रा व॑स॒त् तत् तदव॑स॒द् यत्र॑यत्र॒ यत्र॑य॒त्रा व॑स॒त् तत् । \newline
46. यत्र॑य॒त्रेति॒ यत्र॑ - य॒त्र॒ । \newline
47. अव॑स॒त् तत् तदव॑स॒ दव॑स॒त् तत् कृ॒ष्णम् कृ॒ष्णम् तदव॑स॒ दव॑स॒त् तत् कृ॒ष्णम् । \newline
48. तत् कृ॒ष्णम् कृ॒ष्णम् तत् तत् कृ॒ष्ण म॑भव दभवत् कृ॒ष्णम् तत् तत् कृ॒ष्ण म॑भवत् । \newline
49. कृ॒ष्ण म॑भव दभवत् कृ॒ष्णम् कृ॒ष्ण म॑भवत् कल्मा॒षी क॑ल्मा॒ ष्य॑भवत् कृ॒ष्णम् कृ॒ष्ण म॑भवत् कल्मा॒षी । \newline
50. अ॒भ॒व॒त् क॒ल्मा॒षी क॑ल्मा॒ ष्य॑भव दभवत् कल्मा॒षी भ॑वति भवति कल्मा॒ ष्य॑भव दभवत् कल्मा॒षी भ॑वति । \newline
51. क॒ल्मा॒षी भ॑वति भवति कल्मा॒षी क॑ल्मा॒षी भ॑वति रू॒पस॑मृद्ध्यै रू॒पस॑मृद्ध्यै भवति कल्मा॒षी क॑ल्मा॒षी भ॑वति रू॒पस॑मृद्ध्यै । \newline
52. भ॒व॒ति॒ रू॒पस॑मृद्ध्यै रू॒पस॑मृद्ध्यै भवति भवति रू॒पस॑मृद्ध्या उभयतः॒क्ष्णू 
रु॑भयतः॒क्ष्णू रू॒पस॑मृद्ध्यै भवति भवति रू॒पस॑मृद्ध्या उभयतः॒क्ष्णूः । \newline
53. रू॒पस॑मृद्ध्या उभयतः॒क्ष्णू रु॑भयतः॒क्ष्णू रू॒पस॑मृद्ध्यै रू॒पस॑मृद्ध्या उभयतः॒क्ष्णूर् भ॑वति भव त्युभयतः॒क्ष्णू रू॒पस॑मृद्ध्यै रू॒पस॑मृद्ध्या उभयतः॒क्ष्णूर् भ॑वति । \newline
54. रू॒पस॑मृद्ध्या॒ इति॑ रू॒प - स॒मृ॒द्ध्यै॒ । \newline
55. उ॒भ॒य॒तः॒ष्णूर् भ॑वति भव त्युभयतः॒क्ष्णू रु॑भयतः॒क्ष्णूर् भ॑वती॒त इ॒तो भ॑व त्युभयतः॒क्ष्णू रु॑भयतः॒क्ष्णूर् भ॑वती॒तः । \newline
56. उ॒भ॒य॒तः॒क्ष्णूरित्यु॑भयतः - क्ष्णूः । \newline
57. भ॒व॒ती॒त इ॒तो भ॑वति भवती॒तश्च॑ चे॒तो भ॑वति भवती॒तश्च॑ । \newline
58. इ॒तश्च॑ चे॒ त इ॒तश्चा॒ मुतो॒ ऽमुत॑श्चे॒ त इ॒तश्चा॒ मुतः॑ । \newline
59. चा॒मुतो॒ ऽमुत॑श्च चा॒मुत॑श्च चा॒मुत॑श्च चा॒मुत॑श्च । \newline
60. अ॒मुत॑श्च चा॒मुतो॒ ऽमुत॑ श्चा॒र्कस्या॒ र्कस्य॑ चा॒मुतो॒ ऽमुत॑श्चा॒र्कस्य॑ । \newline
61. चा॒र्कस्या॒ र्कस्य॑ च चा॒र्कस्या व॑रुद्ध्या॒ अव॑रुद्ध्या अ॒र्कस्य॑ च चा॒र्कस्या व॑रुद्ध्यै । \newline
62. अ॒र्कस्या व॑रुद्ध्या॒ अव॑रुद्ध्या अ॒र्कस्या॒ र्कस्या व॑रुद्ध्यै व्याममा॒त्री व्या॑ममा॒ त्र्यव॑रुद्ध्या अ॒र्कस्या॒ र्कस्या व॑रुद्ध्यै व्याममा॒त्री । \newline
63. अव॑रुद्ध्यै व्याममा॒त्री व्या॑ममा॒ त्र्यव॑रुद्ध्या॒ अव॑रुद्ध्यै व्याममा॒त्री भ॑वति भवति व्याममा॒ त्र्यव॑रुद्ध्या॒ अव॑रुद्ध्यै व्याममा॒त्री भ॑वति । \newline
64. अव॑रुद्ध्या॒ इत्यव॑ - रु॒द्ध्यै॒ । \newline
65. व्या॒म॒मा॒त्री भ॑वति भवति व्याममा॒त्री व्या॑ममा॒त्री भ॑व त्ये॒ताव॑ दे॒ताव॑द् भवति व्याममा॒त्री व्या॑ममा॒त्री भ॑व त्ये॒ताव॑त् । \newline
66. व्या॒म॒मा॒त्रीति॑ व्याम - मा॒त्री । \newline
67. भ॒व॒ त्ये॒ताव॑ दे॒ताव॑द् भवति भव त्ये॒ताव॒द् वै वा ए॒ताव॑द् भवति भव त्ये॒ताव॒द् वै । \newline
68. ए॒ताव॒द् वै वा ए॒ताव॑ दे॒ताव॒द् वै पुरु॑षे॒ पुरु॑षे॒ वा ए॒ताव॑ दे॒ताव॒द् वै पुरु॑षे । \newline
69. वै पुरु॑षे॒ पुरु॑षे॒ वै वै पुरु॑षे वी॒र्यं॑ ॅवी॒र्य॑म् पुरु॑षे॒ वै वै पुरु॑षे वी॒र्य᳚म् । \newline
70. पुरु॑षे वी॒र्यं॑ ॅवी॒र्य॑म् पुरु॑षे॒ पुरु॑षे वी॒र्यं॑ ॅवी॒र्य॑सम्मिता वी॒र्य॑सम्मिता वी॒र्य॑म् पुरु॑षे॒ पुरु॑षे वी॒र्यं॑ ॅवी॒र्य॑सम्मिता । \newline
71. वी॒र्यं॑ ॅवी॒र्य॑सम्मिता वी॒र्य॑सम्मिता वी॒र्यं॑ ॅवी॒र्यं॑ ॅवी॒र्य॑सम्मि॒ता ऽप॑रिमि॒ता ऽप॑रिमिता वी॒र्य॑सम्मिता वी॒र्यं॑ ॅवी॒र्यं॑ ॅवी॒र्य॑सम्मि॒ता ऽप॑रिमिता । \newline
72. वी॒र्य॑सम्मि॒ता ऽप॑रिमि॒ता ऽप॑रिमिता वी॒र्य॑सम्मिता वी॒र्य॑सम्मि॒ता ऽप॑रिमिता भवति भव॒ त्यप॑रिमिता वी॒र्य॑सम्मिता वी॒र्य॑सम्मि॒ता ऽप॑रिमिता भवति । \newline
73. वी॒र्य॑सम्मि॒तेति॑ वी॒र्य॑ - स॒म्मि॒ता॒ । \newline
74. अप॑रिमिता भवति भव॒ त्यप॑रिमि॒ता ऽप॑रिमिता भव॒ त्यप॑रिमित॒स्या प॑रिमितस्य भव॒ त्यप॑रिमि॒ता ऽप॑रिमिता भव॒ त्यप॑रिमितस्य । \newline
75. अप॑रिमि॒तेत्यप॑रि - मि॒ता॒ । \newline
76. भ॒व॒ त्यप॑रिमित॒स्या प॑रिमितस्य भवति भव॒ त्यप॑रिमित॒स्या व॑रुद्ध्या॒ अव॑रुद्ध्या॒ अप॑रिमितस्य भवति भव॒ त्यप॑रिमित॒स्या व॑रुद्ध्यै । \newline
77. अप॑रिमित॒स्या व॑रुद्ध्या॒ अव॑रुद्ध्या॒ अप॑रिमित॒स्या प॑रिमित॒स्या व॑रुद्ध्यै॒ यो यो ऽव॑रुद्ध्या॒ अप॑रिमित॒स्या प॑रिमित॒स्या व॑रुद्ध्यै॒ यः । \newline
78. अप॑रिमित॒स्येत्यप॑रि - मि॒त॒स्य॒ । \newline
79. अव॑रुद्ध्यै॒ यो यो ऽव॑रुद्ध्या॒ अव॑रुद्ध्यै॒ यो वन॒स्पती॑नां॒ ॅवन॒स्पती॑नां॒ ॅयो ऽव॑रुद्ध्या॒ अव॑रुद्ध्यै॒ यो वन॒स्पती॑नाम् । \newline
80. अव॑रुद्ध्या॒ इत्यव॑ - रु॒द्ध्यै॒ । \newline
81. यो वन॒स्पती॑नां॒ ॅवन॒स्पती॑नां॒ ॅयो यो वन॒स्पती॑नाम् फल॒ग्रहिः॑ फल॒ग्रहि॒र् वन॒स्पती॑नां॒ ॅयो यो वन॒स्पती॑नाम् फल॒ग्रहिः॑ । \newline
82. वन॒स्पती॑नाम् फल॒ग्रहिः॑ फल॒ग्रहि॒र् वन॒स्पती॑नां॒ ॅवन॒स्पती॑नाम् फल॒ग्रहिः॒ स स फ॑ल॒ग्रहि॒र् वन॒स्पती॑नां॒ ॅवन॒स्पती॑नाम् फल॒ग्रहिः॒ सः । \newline
83. फ॒ल॒ग्रहिः॒ स स फ॑ल॒ग्रहिः॑ फल॒ग्रहिः॒ स ए॑षा मेषाꣳ॒॒ स फ॑ल॒ग्रहिः॑ फल॒ग्रहिः॒ स ए॑षाम् । \newline
84. फ॒ल॒ग्रहि॒रिति॑ फल - ग्रहिः॑ । \newline
85. स ए॑षा मेषाꣳ॒॒ स स ए॑षां ॅवी॒र्या॑वान्. वी॒र्या॑वा नेषाꣳ॒॒ स स ए॑षां ॅवी॒र्या॑वान् । \newline
86. ए॒षां॒ ॅवी॒र्या॑वान्. वी॒र्या॑वा नेषा मेषां ॅवी॒र्या॑वान् फल॒ग्रहिः॑ फल॒ग्रहि॑र् वी॒र्या॑वा नेषा मेषां ॅवी॒र्या॑वान् फल॒ग्रहिः॑ । \newline
87. वी॒र्या॑वान् फल॒ग्रहिः॑ फल॒ग्रहि॑र् वी॒र्या॑वान्. वी॒र्या॑वान् फल॒ग्रहि॒र् वेणु॒र् वेणुः॑ फल॒ग्रहि॑र् वी॒र्या॑वान्. वी॒र्या॑वान् फल॒ग्रहि॒र् वेणुः॑ । \newline
88. वी॒र्या॑वा॒निति॑ वी॒र्य॑ - वा॒न् । \newline
89. फ॒ल॒ग्रहि॒र् वेणु॒र् वेणुः॑ फल॒ग्रहिः॑ फल॒ग्रहि॒र् वेणु॑र् वैण॒वी वै॑ण॒वी वेणुः॑ फल॒ग्रहिः॑ फल॒ग्रहि॒र् वेणु॑र् वैण॒वी । \newline
90. फ॒ल॒ग्रहि॒रिति॑ फल - ग्रहिः॑ । \newline
91. वेणु॑र् वैण॒वी वै॑ण॒वी वेणु॒र् वेणु॑र् वैण॒वी भ॑वति भवति वैण॒वी वेणु॒र् वेणु॑र् वैण॒वी भ॑वति । \newline
92. वै॒ण॒वी भ॑वति भवति वैण॒वी वै॑ण॒वी भ॑वति वी॒र्य॑स्य वी॒र्य॑स्य भवति वैण॒वी वै॑ण॒वी भ॑वति वी॒र्य॑स्य । \newline
93. भ॒व॒ति॒ वी॒र्य॑स्य वी॒र्य॑स्य भवति भवति वी॒र्य॑स्या व॑रुद्ध्या॒ अव॑रुद्ध्यै वी॒र्य॑स्य भवति भवति वी॒र्य॑स्या व॑रुद्ध्यै । \newline
94. वी॒र्य॑स्या व॑रुद्ध्या॒ अव॑रुद्ध्यै वी॒र्य॑स्य वी॒र्य॑स्या व॑रुद्ध्यै । \newline
95. अव॑रुद्ध्या॒ इत्यव॑ - रु॒द्ध्यै॒ । \newline
\pagebreak
\markright{ TS 5.1.2.1  \hfill https://www.vedavms.in \hfill}

\section{ TS 5.1.2.1 }

\textbf{TS 5.1.2.1 } \newline
\textbf{Samhita Paata} \newline

व्यृ॑द्धं॒ ॅवा ए॒तद्-य॒ज्ञ्स्य॒ यद॑य॒जुष्के॑ण क्रि॒यत॑ इ॒माम॑गृभ्णन् रश॒ना-मृ॒तस्येत्य॑श्वाभि॒धानी॒मा द॑त्ते॒ यजु॑ष्कृत्यै य॒ज्ञ्स्य॒ समृ॑द्ध्यै॒ प्रतू᳚र्तं ॅवाजि॒न्ना द्र॒वेत्यश्व॑म॒भि द॑धाति रू॒पमे॒वास्यै॒तन् म॑हि॒मानं॒ ॅव्याच॑ष्टे यु॒ञ्जाथाꣳ॒॒ रास॑भं ॅयु॒वमिति॑ गर्द॒भमसे॑त्ये॒व ग॑र्द॒भं प्रति॑ ष्ठापयति॒ तस्मा॒दश्वा᳚द्-गर्द॒भोऽस॑त्तरो॒ योगे॑योगे त॒वस्त॑र॒मित्या॑ह॒ - [  ] \newline

\textbf{Pada Paata} \newline

व्यृ॑द्ध॒मिति॒ वि - ऋ॒द्ध॒म् । वै । ए॒तत् । य॒ज्ञ्स्य॑ । यत् । अ॒य॒जुष्के॒णेत्य॑य॒जुः-के॒न॒ । क्रि॒यते᳚ । इ॒माम् । अ॒गृ॒भ्ण॒न्न् । र॒श॒नाम् । ऋ॒तस्य॑ । इति॑ । अ॒श्वा॒भि॒धानी॒मित्य॑श्व - अ॒भि॒धानी᳚म् । एति॑ । द॒त्ते॒ । यजु॑ष्कृत्या॒ इति॒ यजुः॑-कृ॒त्यै॒ । य॒ज्ञ्स्य॑ । समृ॑द्ध्या॒ इति॒ सं-ऋ॒द्ध्यै॒ । प्रतू᳚र्त॒मिति॒ प्र - तू॒र्त॒म् । वा॒जि॒न्न् । एति॑ । द्र॒व॒ । इति॑ । अश्व᳚म् । अ॒भीति॑ । द॒धा॒ति॒ । रू॒पम् । ए॒व । अ॒स्य॒ । ए॒तत् । म॒हि॒मान᳚म् । व्याच॑ष्ट॒ इति॑ वि - आच॑ष्टे । यु॒ञ्जाथा᳚म् । रास॑भम् । यु॒वम् । इति॑ । ग॒र्द॒भम् । अस॑ति । ए॒व । ग॒र्द॒भम् । प्रतीति॑ । स्था॒प॒य॒ति॒ । तस्मा᳚त् । अश्वा᳚त् । ग॒र्द॒भः । अस॑त्तर॒ इत्यस॑त् - त॒रः॒ । योगे॑योग॒ इति॒ योगे᳚ - यो॒गे॒ । त॒वस्त॑र॒मिति॑ त॒वः - त॒र॒म् । इति॑ । आ॒ह॒ ।  \newline


\textbf{Krama Paata} \newline

व्यृ॑द्ध॒म् ॅवै । व्यृ॑द्ध॒मिति॒ वि - ऋ॒द्ध॒म् । वा ए॒तत् । ए॒तद् य॒ज्ञ्स्य॑ । य॒ज्ञ्स्य॒ यत् । यद॑य॒जुष्के॑ण । अ॒य॒जुष्के॑ण क्रि॒यते᳚ । अ॒य॒जुष्के॒णेत्य॑य॒जुः - के॒न॒ । क्रि॒यत॑ इ॒माम् । इ॒माम॑गृभ्णन्न् । अ॒गृ॒भ्ण॒न् रश॑नाम् । र॒श॒नामृ॒तस्य॑ । ऋ॒तस्येति॑ । इत्य॑श्वाभि॒धानी᳚म् । अ॒श्वा॒भि॒धानी॒मा । अ॒श्वा॒भि॒धानी॒मित्य॑श्व - अ॒भि॒धानी᳚म् । आ द॑त्ते । द॒त्ते॒ यजु॑ष्कृत्यै । यजु॑ष्कृत्यै य॒ज्ञ्स्य॑ । यजु॑ष्कृत्या॒ इति॒ यजुः॑ - कृ॒त्यै॒ । य॒ज्ञ्स्य॒ समृ॑द्ध्यै । समृ॑द्ध्यै॒ प्रतू᳚र्तम् । समृ॑द्ध्या॒ इति॒ सम् - ऋ॒द्ध्यै॒ । प्रतू᳚र्तम् ॅवाजिन्न् । प्रतू᳚र्त॒मिति॒ प्र - तू॒र्त॒म् । वा॒जि॒न्ना । आ द्र॑व । द्र॒वेति॑ । इत्यश्व᳚म् । अश्व॑म॒भि । अ॒भि द॑धाति । द॒धा॒ति॒ रू॒पम् । रू॒पमे॒व । ए॒वास्य॑ । अ॒स्यै॒तत् । ए॒तन् म॑हि॒मान᳚म् । म॒हि॒मान॒म् ॅव्याच॑ष्टे । व्याच॑ष्टे यु॒ञ्जाथा᳚म् । व्याच॑ष्ट॒ इति॑ वि - आच॑ष्टे । यु॒ञ्जाथाꣳ॒॒ रास॑भम् । रास॑भम् ॅयु॒वम् । यु॒वमिति॑ । इति॑ गर्द॒भम् । ग॒र्द॒भमस॑ति । अस॑त्ये॒व । ए॒व ग॑र्द॒भम् । ग॒र्द॒भम् प्रति॑ । प्रति॑ ष्ठापयति । स्था॒प॒य॒ति॒ तस्मा᳚त् । तस्मा॒दश्वा᳚त् । अश्वा᳚द् गर्द॒भः । ग॒र्द॒भोऽस॑त्तरः । अस॑त्तरो॒ योगे॑योगे । अस॑त्तर॒ इत्यस॑त् - त॒रः॒ । योगे॑योगे त॒वस्त॑रम् । योगे॑योग॒ इति॒ योगे᳚ - यो॒गे॒ । त॒वस्त॑र॒मिति॑ । त॒वस्त॑र॒मिति॑ त॒वः - त॒र॒म् । इत्या॑ह । आ॒ह॒ योगे॑योगे \newline

\textbf{Jatai Paata} \newline

1. व्यृ॑द्धं॒ ॅवै वै व्यृ॑द्धं॒ ॅव्यृ॑द्धं॒ ॅवै । \newline
2. व्यृ॑द्ध॒मिति॒ वि - ऋ॒द्ध॒म् । \newline
3. वा ए॒त दे॒तद् वै वा ए॒तत् । \newline
4. ए॒तद् य॒ज्ञ्स्य॑ य॒ज्ञ् स्यै॒त दे॒तद् य॒ज्ञ्स्य॑ । \newline
5. य॒ज्ञ्स्य॒ यद् यद् य॒ज्ञ्स्य॑ य॒ज्ञ्स्य॒ यत् । \newline
6. यद॑य॒जुष्के॑णा य॒जुष्के॑ण॒ यद् यद॑य॒जुष्के॑ण । \newline
7. अ॒य॒जुष्के॑ण क्रि॒यते᳚ क्रि॒यते॑ ऽय॒जुष्के॑णा य॒जुष्के॑ण क्रि॒यते᳚ । \newline
8. अ॒य॒जुष्के॒णेत्य॑य॒जुः - के॒न॒ । \newline
9. क्रि॒यत॑ इ॒मा मि॒माम् क्रि॒यते᳚ क्रि॒यत॑ इ॒माम् । \newline
10. इ॒मा म॑गृभ्णन् नगृभ्णन् नि॒मा मि॒मा म॑गृभ्णन्न् । \newline
11. अ॒गृ॒भ्ण॒न् र॒श॒नाꣳ र॑श॒ना म॑गृभ्णन् नगृभ्णन् रश॒नाम् । \newline
12. र॒श॒ना मृ॒तस्य॒ र्‌तस्य॑ रश॒नाꣳ र॑श॒ना मृ॒तस्य॑ । \newline
13. ऋ॒तस्येती त्यृ॒तस्य॒ र्‌तस्ये ति॑ । \newline
14. इत्य॑श्वाभि॒धानी॑ मश्वाभि॒धानी॒ मिती त्य॑श्वाभि॒धानी᳚म् । \newline
15. अ॒श्वा॒भि॒धानी॒ मा ऽश्वा॑भि॒धानी॑ मश्वाभि॒धानी॒ मा । \newline
16. अ॒श्वा॒भि॒धानी॒मित्य॑श्व - अ॒भि॒धानी᳚म् । \newline
17. आ द॑त्ते दत्त॒ आ द॑त्ते । \newline
18. द॒त्ते॒ यजु॑ष्कृत्यै॒ यजु॑ष्कृत्यै दत्ते दत्ते॒ यजु॑ष्कृत्यै । \newline
19. यजु॑ष्कृत्यै य॒ज्ञ्स्य॑ य॒ज्ञ्स्य॒ यजु॑ष्कृत्यै॒ यजु॑ष्कृत्यै य॒ज्ञ्स्य॑ । \newline
20. यजु॑ष्कृत्या॒ इति॒ यजुः॑ - कृ॒त्यै॒ । \newline
21. य॒ज्ञ्स्य॒ समृ॑द्ध्यै॒ समृ॑द्ध्यै य॒ज्ञ्स्य॑ य॒ज्ञ्स्य॒ समृ॑द्ध्यै । \newline
22. समृ॑द्ध्यै॒ प्रतू᳚र्त॒म् प्रतू᳚र्तꣳ॒॒ समृ॑द्ध्यै॒ समृ॑द्ध्यै॒ प्रतू᳚र्तम् । \newline
23. समृ॑द्ध्या॒ इति॒ सं - ऋ॒द्ध्यै॒ । \newline
24. प्रतू᳚र्तं ॅवाजिन्. वाजि॒न् प्रतू᳚र्त॒म् प्रतू᳚र्तं ॅवाजिन्न् । \newline
25. प्रतू᳚र्त॒मिति॒ प्र - तू॒र्त॒म् । \newline
26. वा॒जि॒न् ना वा॑जिन्. वाजि॒न् ना । \newline
27. आ द्र॑व द्र॒वा द्र॑व । \newline
28. द्र॒वेतीति॑ द्रव द्र॒वेति॑ । \newline
29. इत्यश्व॒ मश्व॒ मिती त्यश्व᳚म् । \newline
30. अश्व॑ म॒भ्य॑भ्यश्व॒ मश्व॑ म॒भि । \newline
31. अ॒भि द॑धाति दधा त्य॒भ्य॑भि द॑धाति । \newline
32. द॒धा॒ति॒ रू॒पꣳ रू॒पम् द॑धाति दधाति रू॒पम् । \newline
33. रू॒प मे॒वैव रू॒पꣳ रू॒प मे॒व । \newline
34. ए॒वास्या᳚ स्यै॒वै वास्य॑ । \newline
35. अ॒स्यै॒त दे॒त द॑स्या स्यै॒तत् । \newline
36. ए॒तन् म॑हि॒मान॑म् महि॒मान॑ मे॒त दे॒तन् म॑हि॒मान᳚म् । \newline
37. म॒हि॒मानं॒ ॅव्याच॑ष्टे॒ व्याच॑ष्टे महि॒मान॑म् महि॒मानं॒ ॅव्याच॑ष्टे । \newline
38. व्याच॑ष्टे यु॒ञ्जाथां᳚ ॅयु॒ञ्जाथां॒ ॅव्याच॑ष्टे॒ व्याच॑ष्टे यु॒ञ्जाथा᳚म् । \newline
39. व्याच॑ष्ट॒ इति॑ वि - आच॑ष्टे । \newline
40. यु॒ञ्जाथाꣳ॒॒ रास॑भꣳ॒॒ रास॑भं ॅयु॒ञ्जाथां᳚ ॅयु॒ञ्जाथाꣳ॒॒ रास॑भम् । \newline
41. रास॑भं ॅयु॒वं ॅयु॒वꣳ रास॑भꣳ॒॒ रास॑भं ॅयु॒वम् । \newline
42. यु॒व मितीति॑ यु॒वं ॅयु॒व मिति॑ । \newline
43. इति॑ गर्द॒भम् ग॑र्द॒भ मितीति॑ गर्द॒भम् । \newline
44. ग॒र्द॒भ मस॒ त्यस॑ति गर्द॒भम् ग॑र्द॒भ मस॑ति । \newline
45. अस॑ त्ये॒वैवा स॒त्य स॑त्ये॒व । \newline
46. ए॒व ग॑र्द॒भम् ग॑र्द॒भ मे॒वैव ग॑र्द॒भम् । \newline
47. ग॒र्द॒भम् प्रति॒ प्रति॑ गर्द॒भम् ग॑र्द॒भम् प्रति॑ । \newline
48. प्रति॑ ष्ठापयति स्थापयति॒ प्रति॒ प्रति॑ ष्ठापयति । \newline
49. स्था॒प॒य॒ति॒ तस्मा॒त् तस्मा᳚थ् स्थापयति स्थापयति॒ तस्मा᳚त् । \newline
50. तस्मा॒ दश्वा॒ दश्वा॒त् तस्मा॒त् तस्मा॒ दश्वा᳚त् । \newline
51. अश्वा᳚द् गर्द॒भो ग॑र्द॒भो ऽश्वा॒ दश्वा᳚द् गर्द॒भः । \newline
52. ग॒र्द॒भो ऽस॑त्त॒रो ऽस॑त्तरो गर्द॒भो ग॑र्द॒भो ऽस॑त्तरः । \newline
53. अस॑त्तरो॒ योगे॑योगे॒ योगे॑यो॒गे ऽस॑त्त॒रो ऽस॑त्तरो॒ योगे॑योगे । \newline
54. अस॑त्तर॒ इत्यस॑त् - त॒रः॒ । \newline
55. योगे॑योगे त॒वस्त॑रम् त॒वस्त॑रं॒ ॅयोगे॑योगे॒ योगे॑योगे त॒वस्त॑रम् । \newline
56. योगे॑योग॒ इति॒ योगे᳚ - यो॒गे॒ । \newline
57. त॒वस्त॑र॒ मितीति॑ त॒वस्त॑रम् त॒वस्त॑र॒ मिति॑ । \newline
58. त॒वस्त॑र॒मिति॑ त॒वः - त॒र॒म् । \newline
59. इत्या॑हा॒हे तीत्या॑ह । \newline
60. आ॒ह॒ योगे॑योगे॒ योगे॑योग आहाह॒ योगे॑योगे । \newline

\textbf{Ghana Paata } \newline

1. व्यृ॑द्धं॒ ॅवै वै व्यृ॑द्धं॒ ॅव्यृ॑द्धं॒ ॅवा ए॒त दे॒तद् वै व्यृ॑द्धं॒ ॅव्यृ॑द्धं॒ ॅवा ए॒तत् । \newline
2. व्यृ॑द्ध॒मिति॒ वि - ऋ॒द्ध॒म् । \newline
3. वा ए॒त दे॒तद् वै वा ए॒तद् य॒ज्ञ्स्य॑ य॒ज्ञ् स्यै॒तद् वै वा ए॒तद् य॒ज्ञ्स्य॑ । \newline
4. ए॒तद् य॒ज्ञ्स्य॑ य॒ज्ञ् स्यै॒त दे॒तद् य॒ज्ञ्स्य॒ यद् यद् य॒ज्ञ् स्यै॒त दे॒तद् य॒ज्ञ्स्य॒ यत् । \newline
5. य॒ज्ञ्स्य॒ यद् यद् य॒ज्ञ्स्य॑ य॒ज्ञ्स्य॒ यद॑य॒जुष्के॑णा य॒जुष्के॑ण॒ यद् य॒ज्ञ्स्य॑ य॒ज्ञ्स्य॒ यद॑य॒जुष्के॑ण । \newline
6. यद॑य॒जुष्के॑णा य॒जुष्के॑ण॒ यद् यद॑य॒जुष्के॑ण क्रि॒यते᳚ क्रि॒यते॑ ऽय॒जुष्के॑ण॒ यद् यद॑य॒जुष्के॑ण क्रि॒यते᳚ । \newline
7. अ॒य॒जुष्के॑ण क्रि॒यते᳚ क्रि॒यते॑ ऽय॒जुष्के॑णा य॒जुष्के॑ण क्रि॒यत॑ इ॒मा मि॒माम् क्रि॒यते॑ ऽय॒जुष्के॑णा य॒जुष्के॑ण क्रि॒यत॑ इ॒माम् । \newline
8. अ॒य॒जुष्के॒णेत्य॑य॒जुः - के॒न॒ । \newline
9. क्रि॒यत॑ इ॒मा मि॒माम् क्रि॒यते᳚ क्रि॒यत॑ इ॒मा म॑गृभ्णन् नगृभ्णन् नि॒माम् क्रि॒यते᳚ क्रि॒यत॑ इ॒मा म॑गृभ्णन्न् । \newline
10. इ॒मा म॑गृभ्णन् नगृभ्णन् नि॒मा मि॒मा म॑गृभ्णन् रश॒नाꣳ र॑श॒ना म॑गृभ्णन् नि॒मा मि॒मा म॑गृभ्णन् रश॒नाम् । \newline
11. अ॒गृ॒भ्ण॒न् र॒श॒नाꣳ र॑श॒ना म॑गृभ्णन् नगृभ्णन् रश॒ना मृ॒तस्य॒ र्‌तस्य॑ रश॒ना म॑गृभ्णन् नगृभ्णन् रश॒ना मृ॒तस्य॑ । \newline
12. र॒श॒ना मृ॒तस्य॒ र्‌तस्य॑ रश॒नाꣳ र॑श॒ना मृ॒तस्ये तीत्यृ॒तस्य॑ रश॒नाꣳ र॑श॒ना मृ॒तस्येति॑ । \newline
13. ऋ॒तस्ये तीत्यृ॒तस्य॒ र्‌तस्ये त्य॑श्वाभि॒धानी॑ मश्वाभि॒धानी॒ मित्यृ॒तस्य॒ र्‌तस्ये त्य॑श्वाभि॒धानी᳚म् । \newline
14. इत्य॑श्वाभि॒धानी॑ मश्वाभि॒धानी॒ मिती त्य॑श्वाभि॒धानी॒ मा ऽश्वा॑भि॒धानी॒ मिती त्य॑श्वाभि॒धानी॒ मा । \newline
15. अ॒श्वा॒भि॒धानी॒ मा ऽश्वा॑भि॒धानी॑ मश्वाभि॒धानी॒ मा द॑त्ते दत्त॒ आ ऽश्वा॑भि॒धानी॑ मश्वाभि॒धानी॒ मा द॑त्ते । \newline
16. अ॒श्वा॒भि॒धानी॒मित्य॑श्व - अ॒भि॒धानी᳚म् । \newline
17. आ द॑त्ते दत्त॒ आ द॑त्ते॒ यजु॑ष्कृत्यै॒ यजु॑ष्कृत्यै दत्त॒ आ द॑त्ते॒ यजु॑ष्कृत्यै । \newline
18. द॒त्ते॒ यजु॑ष्कृत्यै॒ यजु॑ष्कृत्यै दत्ते दत्ते॒ यजु॑ष्कृत्यै य॒ज्ञ्स्य॑ य॒ज्ञ्स्य॒ यजु॑ष्कृत्यै दत्ते दत्ते॒ यजु॑ष्कृत्यै य॒ज्ञ्स्य॑ । \newline
19. यजु॑ष्कृत्यै य॒ज्ञ्स्य॑ य॒ज्ञ्स्य॒ यजु॑ष्कृत्यै॒ यजु॑ष्कृत्यै य॒ज्ञ्स्य॒ समृ॑द्ध्यै॒ समृ॑द्ध्यै य॒ज्ञ्स्य॒ यजु॑ष्कृत्यै॒ यजु॑ष्कृत्यै य॒ज्ञ्स्य॒ समृ॑द्ध्यै । \newline
20. यजु॑ष्कृत्या॒ इति॒ यजुः॑ - कृ॒त्यै॒ । \newline
21. य॒ज्ञ्स्य॒ समृ॑द्ध्यै॒ समृ॑द्ध्यै य॒ज्ञ्स्य॑ य॒ज्ञ्स्य॒ समृ॑द्ध्यै॒ प्रतू᳚र्त॒म् प्रतू᳚र्तꣳ॒॒ समृ॑द्ध्यै य॒ज्ञ्स्य॑ य॒ज्ञ्स्य॒ समृ॑द्ध्यै॒ प्रतू᳚र्तम् । \newline
22. समृ॑द्ध्यै॒ प्रतू᳚र्त॒म् प्रतू᳚र्तꣳ॒॒ समृ॑द्ध्यै॒ समृ॑द्ध्यै॒ प्रतू᳚र्तं ॅवाजिन्. वाजि॒न् प्रतू᳚र्तꣳ॒॒ समृ॑द्ध्यै॒ समृ॑द्ध्यै॒ प्रतू᳚र्तं ॅवाजिन्न् । \newline
23. समृ॑द्ध्या॒ इति॒ सं - ऋ॒द्ध्यै॒ । \newline
24. प्रतू᳚र्तं ॅवाजिन्. वाजि॒न् प्रतू᳚र्त॒म् प्रतू᳚र्तं ॅवाजि॒न् ना वा॑जि॒न् प्रतू᳚र्त॒म् प्रतू᳚र्तं ॅवाजि॒न् ना । \newline
25. प्रतू᳚र्त॒मिति॒ प्र - तू॒र्त॒म् । \newline
26. वा॒जि॒न् ना वा॑जिन्. वाजि॒न् ना द्र॑व द्र॒वा वा॑जिन्. वाजि॒न् ना द्र॑व । \newline
27. आ द्र॑व द्र॒वा द्र॒वे तीति॑ द्र॒वा द्र॒वेति॑ । \newline
28. द्र॒वे तीति॑ द्रव द्र॒वे त्यश्व॒ मश्व॒ मिति॑ द्रव द्र॒वे त्यश्व᳚म् । \newline
29. इत्यश्व॒ मश्व॒ मिती त्यश्व॑ म॒भ्य॑भ्यश्व॒ मिती त्यश्व॑ म॒भि । \newline
30. अश्व॑ म॒भ्य॑भ्यश्व॒ मश्व॑ म॒भि द॑धाति दधा त्य॒भ्यश्व॒ मश्व॑ म॒भि द॑धाति । \newline
31. अ॒भि द॑धाति दधा त्य॒भ्य॑भि द॑धाति रू॒पꣳ रू॒पम् द॑धा त्य॒भ्य॑भि द॑धाति रू॒पम् । \newline
32. द॒धा॒ति॒ रू॒पꣳ रू॒पम् द॑धाति दधाति रू॒प मे॒वैव रू॒पम् द॑धाति दधाति रू॒प मे॒व । \newline
33. रू॒प मे॒वैव रू॒पꣳ रू॒प मे॒वास्या᳚ स्यै॒व रू॒पꣳ रू॒प मे॒वास्य॑ । \newline
34. ए॒वास्या᳚ स्यै॒वैवा स्यै॒त दे॒त द॑स्यै॒वै वास्यै॒तत् । \newline
35. अ॒स्यै॒त दे॒त द॑स्या स्यै॒तन् म॑हि॒मान॑म् महि॒मान॑ मे॒तद॑स्या स्यै॒तन् म॑हि॒मान᳚म् । \newline
36. ए॒तन् म॑हि॒मान॑म् महि॒मान॑ मे॒त दे॒तन् म॑हि॒मानं॒ ॅव्याच॑ष्टे॒ व्याच॑ष्टे महि॒मान॑ मे॒त दे॒तन् म॑हि॒मानं॒ ॅव्याच॑ष्टे । \newline
37. म॒हि॒मानं॒ ॅव्याच॑ष्टे॒ व्याच॑ष्टे महि॒मान॑म् महि॒मानं॒ ॅव्याच॑ष्टे यु॒ञ्जाथां᳚ ॅयु॒ञ्जाथां॒ ॅव्याच॑ष्टे महि॒मान॑म् महि॒मानं॒ ॅव्याच॑ष्टे यु॒ञ्जाथा᳚म् । \newline
38. व्याच॑ष्टे यु॒ञ्जाथां᳚ ॅयु॒ञ्जाथां॒ ॅव्याच॑ष्टे॒ व्याच॑ष्टे यु॒ञ्जाथाꣳ॒॒ रास॑भꣳ॒॒ रास॑भं ॅयु॒ञ्जाथां॒ ॅव्याच॑ष्टे॒ व्याच॑ष्टे यु॒ञ्जाथाꣳ॒॒ रास॑भम् । \newline
39. व्याच॑ष्ट॒ इति॑ वि - आच॑ष्टे । \newline
40. यु॒ञ्जाथाꣳ॒॒ रास॑भꣳ॒॒ रास॑भं ॅयु॒ञ्जाथां᳚ ॅयु॒ञ्जाथाꣳ॒॒ रास॑भं ॅयु॒वं ॅयु॒वꣳ रास॑भं ॅयु॒ञ्जाथां᳚ ॅयु॒ञ्जाथाꣳ॒॒ रास॑भं ॅयु॒वम् । \newline
41. रास॑भं ॅयु॒वं ॅयु॒वꣳ रास॑भꣳ॒॒ रास॑भं ॅयु॒व मितीति॑ यु॒वꣳ रास॑भꣳ॒॒ रास॑भं ॅयु॒व मिति॑ । \newline
42. यु॒व मितीति॑ यु॒वं ॅयु॒व मिति॑ गर्द॒भम् ग॑र्द॒भ मिति॑ यु॒वं ॅयु॒व मिति॑ गर्द॒भम् । \newline
43. इति॑ गर्द॒भम् ग॑र्द॒भ मितीति॑ गर्द॒भ मस॒ त्यस॑ति गर्द॒भ मितीति॑ गर्द॒भ मस॑ति । \newline
44. ग॒र्द॒भ मस॒ त्यस॑ति गर्द॒भम् ग॑र्द॒भ मस॑ त्ये॒वैवा स॑ति गर्द॒भम् ग॑र्द॒भ मस॑ त्ये॒व । \newline
45. अस॑ त्ये॒वैवास॒ त्यस॑ त्ये॒व ग॑र्द॒भम् ग॑र्द॒भ मे॒वास॒ त्यस॑ त्ये॒व ग॑र्द॒भम् । \newline
46. ए॒व ग॑र्द॒भम् ग॑र्द॒भ मे॒वैव ग॑र्द॒भम् प्रति॒ प्रति॑ गर्द॒भ मे॒वैव ग॑र्द॒भम् प्रति॑ । \newline
47. ग॒र्द॒भम् प्रति॒ प्रति॑ गर्द॒भम् ग॑र्द॒भम् प्रति॑ ष्ठापयति स्थापयति॒ प्रति॑ गर्द॒भम् ग॑र्द॒भम् प्रति॑ ष्ठापयति । \newline
48. प्रति॑ ष्ठापयति स्थापयति॒ प्रति॒ प्रति॑ ष्ठापयति॒ तस्मा॒त् तस्मा᳚थ् स्थापयति॒ प्रति॒ प्रति॑ ष्ठापयति॒ तस्मा᳚त् । \newline
49. स्था॒प॒य॒ति॒ तस्मा॒त् तस्मा᳚थ् स्थापयति स्थापयति॒ तस्मा॒ दश्वा॒ दश्वा॒त् तस्मा᳚थ् स्थापयति स्थापयति॒ तस्मा॒ दश्वा᳚त् । \newline
50. तस्मा॒ दश्वा॒ दश्वा॒त् तस्मा॒त् तस्मा॒ दश्वा᳚द् गर्द॒भो ग॑र्द॒भो ऽश्वा॒त् तस्मा॒त् तस्मा॒ दश्वा᳚द् गर्द॒भः । \newline
51. अश्वा᳚द् गर्द॒भो ग॑र्द॒भो ऽश्वा॒ दश्वा᳚द् गर्द॒भो ऽस॑त्त॒रो ऽस॑त्तरो गर्द॒भो ऽश्वा॒ दश्वा᳚द् गर्द॒भो ऽस॑त्तरः । \newline
52. ग॒र्द॒भो ऽस॑त्त॒रो ऽस॑त्तरो गर्द॒भो ग॑र्द॒भो ऽस॑त्तरो॒ योगे॑योगे॒ योगे॑यो॒गे ऽस॑त्तरो गर्द॒भो ग॑र्द॒भो ऽस॑त्तरो॒ योगे॑योगे । \newline
53. अस॑त्तरो॒ योगे॑योगे॒ योगे॑यो॒गे ऽस॑त्त॒रो ऽस॑त्तरो॒ योगे॑योगे त॒वस्त॑रम् त॒वस्त॑रं॒ ॅयोगे॑यो॒गे ऽस॑त्त॒रो ऽस॑त्तरो॒ योगे॑योगे त॒वस्त॑रम् । \newline
54. अस॑त्तर॒ इत्यस॑त् - त॒रः॒ । \newline
55. योगे॑योगे त॒वस्त॑रम् त॒वस्त॑रं॒ ॅयोगे॑योगे॒ योगे॑योगे त॒वस्त॑र॒ मितीति॑ त॒वस्त॑रं॒ ॅयोगे॑योगे॒ योगे॑योगे त॒वस्त॑र॒ मिति॑ । \newline
56. योगे॑योग॒ इति॒ योगे᳚ - यो॒गे॒ । \newline
57. त॒वस्त॑र॒ मितीति॑ त॒वस्त॑रम् त॒वस्त॑र॒ मित्या॑हा॒हेति॑ त॒वस्त॑रम् त॒वस्त॑र॒ मित्या॑ह । \newline
58. त॒वस्त॑र॒मिति॑ त॒वः - त॒र॒म् । \newline
59. इत्या॑हा॒हे तीत्या॑ह॒ योगे॑योगे॒ योगे॑योग आ॒हे तीत्या॑ह॒ योगे॑योगे । \newline
60. आ॒ह॒ योगे॑योगे॒ योगे॑योग आहाह॒ योगे॑योग ए॒वैव योगे॑योग आहाह॒ योगे॑योग ए॒व । \newline
\pagebreak
\markright{ TS 5.1.2.2  \hfill https://www.vedavms.in \hfill}

\section{ TS 5.1.2.2 }

\textbf{TS 5.1.2.2 } \newline
\textbf{Samhita Paata} \newline

योगे॑योग ए॒वैनं॑ ॅयुङ्क्ते॒ वाजे॑वाजे हवामह॒ इत्या॒हान्नं॒ ॅवै वाजो-ऽन्न॑मे॒वाव॑ रुन्धे॒ सखा॑य॒ इन्द्र॑मू॒तय॒ इत्या॑हेन्द्रि॒यमे॒वाव॑ रुन्धे॒ ऽग्निर्दे॒वेभ्यो॒ निला॑यत॒ तं प्र॒जाप॑ति॒रन्व॑विन्दत् प्राजाप॒त्योऽश्वो ऽश्वे॑न॒ सं भ॑र॒त्यनु॑वित्त्यै पापवस्य॒सं ॅवा ए॒तत् क्रि॑यते॒ यच्छ्रेय॑सा च॒ पापी॑यसा च समा॒नं कर्म॑ कु॒र्वन्ति॒ पापी॑या॒न्॒. - [  ] \newline

\textbf{Pada Paata} \newline

योगे॑योग॒ इति॒ योगे᳚ - यो॒गे॒ । ए॒व । ए॒न॒म् । यु॒ङ्क्ते॒ । वाजे॑वाज॒ इति॒ वाजे᳚- वा॒जे॒ । ह॒वा॒म॒हे॒ । इति॑ । आ॒ह॒ । अन्न᳚म् । वै । वाजः॑ । अन्न᳚म् । ए॒व । अवेति॑ । रु॒न्धे॒ । सखा॑यः । इन्द्र᳚म् । ऊ॒तये᳚ । इति॑ । आ॒ह॒ । इ॒न्द्रि॒यम् । ए॒व । अवेति॑ । रु॒न्धे॒ । अ॒ग्निः । दे॒वेभ्यः॑ । निला॑यत । तम् । प्र॒जाप॑ति॒रिति॑ प्र॒जा-प॒तिः॒ । अन्विति॑ । अ॒वि॒न्द॒त् । प्रा॒जा॒प॒त्य इति॑ प्राजा - प॒त्यः । अश्वः॑ । अश्वे॑न । समिति॑ । भ॒र॒ति॒ । अनु॑वित्त्या॒ इत्यनु॑ - वि॒त्त्यै॒ । पा॒प॒व॒स्य॒समिति॑ पाप - व॒स्य॒सम् । वै । ए॒तत् । क्रि॒य॒ते॒ । यत् । श्रेय॑सा । च॒ । पापी॑यसा । च॒ । स॒मा॒नम् । कर्म॑ । कु॒र्वन्ति॑ । पापी॑यान् ।  \newline


\textbf{Krama Paata} \newline

योगे॑योग ए॒व । योगे॑योग॒ इति॒ योगे᳚ - यो॒गे॒ । ए॒वैन᳚म् । ए॒न॒म् ॅयु॒ङ्क्ते॒ । यु॒ङ्क्ते॒ वाजे॑वाजे । वाजे॑वाजे हवामहे । वाजे॑वाज॒ इति॒ वाजे᳚ - वा॒जे॒ । ह॒वा॒म॒ह॒ इति॑ । इत्या॑ह । आ॒हान्न᳚म् । अन्न॒म् ॅवै । वै वाजः॑ । वाजोऽन्न᳚म् । अन्न॑मे॒व । ए॒वाव॑ । अव॑ रुन्धे । रु॒न्धे॒ सखा॑यः । सखा॑य॒ इन्द्र᳚म् । इन्द्र॑मू॒तये᳚ । ऊ॒तय॒ इति॑ । इत्या॑ह । आ॒हे॒न्द्रि॒यम् । इ॒न्द्रि॒यमे॒व । ए॒वाव॑ । अव॑ रुन्धे । रु॒न्धे॒ऽग्निः । अ॒ग्निर् दे॒वेभ्यः॑ । दे॒वेभ्यो॒ निला॑यत । निला॑यत॒ तम् । तम् प्र॒जाप॑तिः । प्र॒जाप॑ति॒रनु॑ । प्र॒जाप॑ति॒रिति॑ प्र॒जा - प॒तिः॒ । अन्व॑विन्दत् । अ॒वि॒न्द॒त् प्रा॒जा॒प॒त्यः । प्रा॒जा॒प॒त्योऽश्वः॑ । प्रा॒जा॒प॒त्य इति॑ प्राजा - प॒त्यः । अश्वोऽश्वे॑न । अश्वे॑न॒ सम् । सम् भ॑रति । भ॒र॒त्यनु॑वित्यै । अनु॑वित्यै पापवस्य॒सम् । अनु॑वित्या॒ इत्यनु॑ - वि॒त्यै॒ । पा॒प॒व॒स्य॒सम् ॅवै । पा॒प॒व॒स्य॒समिति॑ पाप - व॒स्य॒सम् । वा ए॒तत् । ए॒तत् क्रि॑यते । क्रि॒य॒ते॒ यत् । यच्छ्रेय॑सा । श्रेय॑सा च । च॒ पापी॑यसा । पापी॑यसा च । च॒ स॒मा॒नम् । स॒मा॒नम् कर्म॑ । कर्म॑ कु॒र्वन्ति॑ । कु॒र्वन्ति॒ पापी॑यान् । पापी॑या॒न्. हि \newline

\textbf{Jatai Paata} \newline

1. योगे॑योग ए॒वैव योगे॑योगे॒ योगे॑योग ए॒व । \newline
2. योगे॑योग॒ इति॒ योगे᳚ - यो॒गे॒ । \newline
3. ए॒वैन॑ मेन मे॒वैवैन᳚म् । \newline
4. ए॒नं॒ ॅयु॒ङ्क्ते॒ यु॒ङ्क्त॒ ए॒न॒ मे॒नं॒ ॅयु॒ङ्क्ते॒ । \newline
5. यु॒ङ्क्ते॒ वाजे॑वाजे॒ वाजे॑वाजे युङ्क्ते युङ्क्ते॒ वाजे॑वाजे । \newline
6. वाजे॑वाजे हवामहे हवामहे॒ वाजे॑वाजे॒ वाजे॑वाजे हवामहे । \newline
7. वाजे॑वाज॒ इति॒ वाजे᳚ - वा॒जे॒ । \newline
8. ह॒वा॒म॒ह॒ इतीति॑ हवामहे हवामह॒ इति॑ । \newline
9. इत्या॑हा॒हे तीत्या॑ह । \newline
10. आ॒हान्न॒ मन्न॑ माहा॒हा न्न᳚म् । \newline
11. अन्नं॒ ॅवै वा अन्न॒ मन्नं॒ ॅवै । \newline
12. वै वाजो॒ वाजो॒ वै वै वाजः॑ । \newline
13. वाजो ऽन्न॒ मन्नं॒ ॅवाजो॒ वाजो ऽन्न᳚म् । \newline
14. अन्न॑ मे॒वैवान्न॒ मन्न॑ मे॒व । \newline
15. ए॒वावा वै॒वै वाव॑ । \newline
16. अव॑ रुन्धे रु॒न्धे ऽवाव॑ रुन्धे । \newline
17. रु॒न्धे॒ सखा॑यः॒ सखा॑यो रुन्धे रुन्धे॒ सखा॑यः । \newline
18. सखा॑य॒ इन्द्र॒ मिन्द्रꣳ॒॒ सखा॑यः॒ सखा॑य॒ इन्द्र᳚म् । \newline
19. इन्द्र॑ मू॒तय॑ ऊ॒तय॒ इन्द्र॒ मिन्द्र॑ मू॒तये᳚ । \newline
20. ऊ॒तय॒ इतीत्यू॒तय॑ ऊ॒तय॒ इति॑ । \newline
21. इत्या॑हा॒हे तीत्या॑ह । \newline
22. आ॒हे॒न्द्रि॒य मि॑न्द्रि॒य मा॑हा हेन्द्रि॒यम् । \newline
23. इ॒न्द्रि॒य मे॒वै वेन्द्रि॒य मि॑न्द्रि॒य मे॒व । \newline
24. ए॒वावा वै॒वै वाव॑ । \newline
25. अव॑ रुन्धे रु॒न्धे ऽवाव॑ रुन्धे । \newline
26. रु॒न्धे॒ ऽग्नि र॒ग्नी रु॑न्धे रुन्धे॒ ऽग्निः । \newline
27. अ॒ग्निर् दे॒वेभ्यो॑ दे॒वेभ्यो॒ ऽग्नि र॒ग्निर् दे॒वेभ्यः॑ । \newline
28. दे॒वेभ्यो॒ निला॑यत॒ निला॑यत दे॒वेभ्यो॑ दे॒वेभ्यो॒ निला॑यत । \newline
29. निला॑यत॒ तम् तम् निला॑यत॒ निला॑यत॒ तम् । \newline
30. तम् प्र॒जाप॑तिः प्र॒जाप॑ति॒ स्तम् तम् प्र॒जाप॑तिः । \newline
31. प्र॒जाप॑ति॒ रन्वनु॑ प्र॒जाप॑तिः प्र॒जाप॑ति॒ रनु॑ । \newline
32. प्र॒जाप॑ति॒रिति॑ प्र॒जा - प॒तिः॒ । \newline
33. अन्व॑विन्द दविन्द॒ दन्वन् व॑विन्दत् । \newline
34. अ॒वि॒न्द॒त् प्रा॒जा॒प॒त्यः प्रा॑जाप॒त्यो॑ ऽविन्द दविन्दत् प्राजाप॒त्यः । \newline
35. प्रा॒जा॒प॒त्यो ऽश्वो ऽश्वः॑ प्राजाप॒त्यः प्रा॑जाप॒त्यो ऽश्वः॑ । \newline
36. प्रा॒जा॒प॒त्य इति॑ प्राजा - प॒त्यः । \newline
37. अश्वो ऽश्वे॒ना श्वे॒ना श्वो ऽश्वो ऽश्वे॑न । \newline
38. अश्वे॑न॒ सꣳ स मश्वे॒ना श्वे॑न॒ सम् । \newline
39. सम् भ॑रति भरति॒ सꣳ सम् भ॑रति । \newline
40. भ॒र॒ त्यनु॑वित्त्या॒ अनु॑वित्त्यै भरति भर॒ त्यनु॑वित्त्यै । \newline
41. अनु॑वित्त्यै पापवस्य॒सम् पा॑पवस्य॒स मनु॑वित्त्या॒ अनु॑वित्त्यै पापवस्य॒सम् । \newline
42. अनु॑वित्त्या॒ इत्यनु॑ - वि॒त्त्यै॒ । \newline
43. पा॒प॒व॒स्य॒सं ॅवै वै पा॑पवस्य॒सम् पा॑पवस्य॒सं ॅवै । \newline
44. पा॒प॒व॒स्य॒समिति॑ पाप - व॒स्य॒सम् । \newline
45. वा ए॒त दे॒तद् वै वा ए॒तत् । \newline
46. ए॒तत् क्रि॑यते क्रियत ए॒त दे॒तत् क्रि॑यते । \newline
47. क्रि॒य॒ते॒ यद् यत् क्रि॑यते क्रियते॒ यत् । \newline
48. यच्छ्रेय॑सा॒ श्रेय॑सा॒ यद् यच्छ्रेय॑सा । \newline
49. श्रेय॑सा च च॒ श्रेय॑सा॒ श्रेय॑सा च । \newline
50. च॒ पापी॑यसा॒ पापी॑यसा च च॒ पापी॑यसा । \newline
51. पापी॑यसा च च॒ पापी॑यसा॒ पापी॑यसा च । \newline
52. च॒ स॒मा॒नꣳ स॑मा॒नम् च॑ च समा॒नम् । \newline
53. स॒मा॒नम् कर्म॒ कर्म॑ समा॒नꣳ स॑मा॒नम् कर्म॑ । \newline
54. कर्म॑ कु॒र्वन्ति॑ कु॒र्वन्ति॒ कर्म॒ कर्म॑ कु॒र्वन्ति॑ । \newline
55. कु॒र्वन्ति॒ पापी॑या॒न् पापी॑यान् कु॒र्वन्ति॑ कु॒र्वन्ति॒ पापी॑यान् । \newline
56. पापी॑या॒न्॒. हि हि पापी॑या॒न् पापी॑या॒न्॒. हि । \newline

\textbf{Ghana Paata } \newline

1. योगे॑योग ए॒वैव योगे॑योगे॒ योगे॑योग ए॒वैन॑ मेन मे॒व योगे॑योगे॒ योगे॑योग ए॒वैन᳚म् । \newline
2. योगे॑योग॒ इति॒ योगे᳚ - यो॒गे॒ । \newline
3. ए॒वैन॑ मेन मे॒वैवैनं॑ ॅयुङ्क्ते युङ्क्त एन मे॒वैवैनं॑ ॅयुङ्क्ते । \newline
4. ए॒नं॒ ॅयु॒ङ्क्ते॒ यु॒ङ्क्त॒ ए॒न॒ मे॒नं॒ ॅयु॒ङ्क्ते॒ वाजे॑वाजे॒ वाजे॑वाजे युङ्क्त एन मेनं ॅयुङ्क्ते॒ वाजे॑वाजे । \newline
5. यु॒ङ्क्ते॒ वाजे॑वाजे॒ वाजे॑वाजे युङ्क्ते युङ्क्ते॒ वाजे॑वाजे हवामहे हवामहे॒ वाजे॑वाजे युङ्क्ते युङ्क्ते॒ वाजे॑वाजे हवामहे । \newline
6. वाजे॑वाजे हवामहे हवामहे॒ वाजे॑वाजे॒ वाजे॑वाजे हवामह॒ इतीति॑ हवामहे॒ वाजे॑वाजे॒ वाजे॑वाजे हवामह॒ इति॑ । \newline
7. वाजे॑वाज॒ इति॒ वाजे᳚ - वा॒जे॒ । \newline
8. ह॒वा॒म॒ह॒ इतीति॑ हवामहे हवामह॒ इत्या॑हा॒हेति॑ हवामहे हवामह॒ इत्या॑ह । \newline
9. इत्या॑हा॒हेती त्या॒हान्न॒ मन्न॑ मा॒हेती त्या॒हान्न᳚म् । \newline
10. आ॒हान्न॒ मन्न॑ माहा॒ हान्नं॒ ॅवै वा अन्न॑ माहा॒ हान्नं॒ ॅवै । \newline
11. अन्नं॒ ॅवै वा अन्न॒ मन्नं॒ ॅवै वाजो॒ वाजो॒ वा अन्न॒ मन्नं॒ ॅवै वाजः॑ । \newline
12. वै वाजो॒ वाजो॒ वै वै वाजो ऽन्न॒ मन्नं॒ ॅवाजो॒ वै वै वाजो ऽन्न᳚म् । \newline
13. वाजो ऽन्न॒ मन्नं॒ ॅवाजो॒ वाजो ऽन्न॑ मे॒वैवान्नं॒ ॅवाजो॒ वाजो ऽन्न॑ मे॒व । \newline
14. अन्न॑ मे॒वैवान्न॒ मन्न॑ मे॒वावा वै॒वान्न॒ मन्न॑ मे॒वाव॑ । \newline
15. ए॒वावा वै॒वै वाव॑ रुन्धे रु॒न्धे ऽवै॒वै वाव॑ रुन्धे । \newline
16. अव॑ रुन्धे रु॒न्धे ऽवाव॑ रुन्धे॒ सखा॑यः॒ सखा॑यो रु॒न्धे ऽवाव॑ रुन्धे॒ सखा॑यः । \newline
17. रु॒न्धे॒ सखा॑यः॒ सखा॑यो रुन्धे रुन्धे॒ सखा॑य॒ इन्द्र॒ मिन्द्रꣳ॒॒ सखा॑यो रुन्धे रुन्धे॒ सखा॑य॒ इन्द्र᳚म् । \newline
18. सखा॑य॒ इन्द्र॒ मिन्द्रꣳ॒॒ सखा॑यः॒ सखा॑य॒ इन्द्र॑ मू॒तय॑ ऊ॒तय॒ इन्द्रꣳ॒॒ सखा॑यः॒ सखा॑य॒ इन्द्र॑ मू॒तये᳚ । \newline
19. इन्द्र॑ मू॒तय॑ ऊ॒तय॒ इन्द्र॒ मिन्द्र॑ मू॒तय॒ इती त्यू॒तय॒ इन्द्र॒ मिन्द्र॑ मू॒तय॒ इति॑ । \newline
20. ऊ॒तय॒ इती त्यू॒तय॑ ऊ॒तय॒ इत्या॑हा॒हे त्यू॒तय॑ ऊ॒तय॒ इत्या॑ह । \newline
21. इत्या॑हा॒हेती त्या॑हेन्द्रि॒य मि॑न्द्रि॒य मा॒हेती त्या॑हेन्द्रि॒यम् । \newline
22. आ॒हे॒न्द्रि॒य मि॑न्द्रि॒य मा॑हा हेन्द्रि॒य मे॒वैवेन्द्रि॒य मा॑हा हेन्द्रि॒य मे॒व । \newline
23. इ॒न्द्रि॒य मे॒वैवे न्द्रि॒य मि॑न्द्रि॒य मे॒वावा वै॒वेन्द्रि॒य मि॑न्द्रि॒य मे॒वाव॑ । \newline
24. ए॒वावा वै॒वै वाव॑ रुन्धे रु॒न्धे ऽवै॒वैवाव॑ रुन्धे । \newline
25. अव॑ रुन्धे रु॒न्धे ऽवाव॑ रुन्धे॒ ऽग्नि र॒ग्नी रु॒न्धे ऽवाव॑ रुन्धे॒ ऽग्निः । \newline
26. रु॒न्धे॒ ऽग्नि र॒ग्नी रु॑न्धे रुन्धे॒ ऽग्निर् दे॒वेभ्यो॑ दे॒वेभ्यो॒ ऽग्नी रु॑न्धे रुन्धे॒ ऽग्निर् दे॒वेभ्यः॑ । \newline
27. अ॒ग्निर् दे॒वेभ्यो॑ दे॒वेभ्यो॒ ऽग्नि र॒ग्निर् दे॒वेभ्यो॒ निला॑यत॒ निला॑यत दे॒वेभ्यो॒ ऽग्नि र॒ग्निर् दे॒वेभ्यो॒ निला॑यत । \newline
28. दे॒वेभ्यो॒ निला॑यत॒ निला॑यत दे॒वेभ्यो॑ दे॒वेभ्यो॒ निला॑यत॒ तम् तम् निला॑यत दे॒वेभ्यो॑ दे॒वेभ्यो॒ निला॑यत॒ तम् । \newline
29. निला॑यत॒ तम् तम् निला॑यत॒ निला॑यत॒ तम् प्र॒जाप॑तिः प्र॒जाप॑ति॒ स्तम् निला॑यत॒ निला॑यत॒ तम् प्र॒जाप॑तिः । \newline
30. तम् प्र॒जाप॑तिः प्र॒जाप॑ति॒ स्तम् तम् प्र॒जाप॑ति॒ रन्वनु॑ प्र॒जाप॑ति॒ स्तम् तम् प्र॒जाप॑ति॒ रनु॑ । \newline
31. प्र॒जाप॑ति॒ रन्वनु॑ प्र॒जाप॑तिः प्र॒जाप॑ति॒ रन्व॑विन्द दविन्द॒ दनु॑ प्र॒जाप॑तिः प्र॒जाप॑ति॒ रन्व॑विन्दत् । \newline
32. प्र॒जाप॑ति॒रिति॑ प्र॒जा - प॒तिः॒ । \newline
33. अन्व॑विन्द दविन्द॒ दन्वन् व॑विन्दत् प्राजाप॒त्यः प्रा॑जाप॒त्यो॑ ऽविन्द॒ दन्वन् व॑विन्दत् प्राजाप॒त्यः । \newline
34. अ॒वि॒न्द॒त् प्रा॒जा॒प॒त्यः प्रा॑जाप॒त्यो॑ ऽविन्द दविन्दत् प्राजाप॒त्यो ऽश्वो ऽश्वः॑ प्राजाप॒त्यो॑ ऽविन्द दविन्दत् प्राजाप॒त्यो ऽश्वः॑ । \newline
35. प्रा॒जा॒प॒त्यो ऽश्वो ऽश्वः॑ प्राजाप॒त्यः प्रा॑जाप॒त्यो ऽश्वो ऽश्वे॒ना श्वे॒ना श्वः॑ प्राजाप॒त्यः प्रा॑जाप॒त्यो ऽश्वो ऽश्वे॑न । \newline
36. प्रा॒जा॒प॒त्य इति॑ प्राजा - प॒त्यः । \newline
37. अश्वो ऽश्वे॒ना श्वे॒ना श्वो ऽश्वो ऽश्वे॑न॒ सꣳ स मश्वे॒ना श्वो ऽश्वो ऽश्वे॑न॒ सम् । \newline
38. अश्वे॑न॒ सꣳ स मश्वे॒ना श्वे॑न॒ सम् भ॑रति भरति॒ स मश्वे॒ना श्वे॑न॒ सम् भ॑रति । \newline
39. सम् भ॑रति भरति॒ सꣳ सम् भ॑र॒ त्यनु॑वित्त्या॒ अनु॑वित्त्यै भरति॒ सꣳ सम् भ॑र॒ त्यनु॑वित्त्यै । \newline
40. भ॒र॒ त्यनु॑वित्त्या॒ अनु॑वित्त्यै भरति भर॒ त्यनु॑वित्त्यै पापवस्य॒सम् पा॑पवस्य॒स मनु॑वित्त्यै भरति भर॒ त्यनु॑वित्त्यै पापवस्य॒सम् । \newline
41. अनु॑वित्त्यै पापवस्य॒सम् पा॑पवस्य॒स मनु॑वित्त्या॒ अनु॑वित्त्यै पापवस्य॒सं ॅवै वै पा॑पवस्य॒स मनु॑वित्त्या॒ अनु॑वित्त्यै पापवस्य॒सं ॅवै । \newline
42. अनु॑वित्त्या॒ इत्यनु॑ - वि॒त्त्यै॒ । \newline
43. पा॒प॒व॒स्य॒सं ॅवै वै पा॑पवस्य॒सम् पा॑पवस्य॒सं ॅवा ए॒त दे॒तद् वै पा॑पवस्य॒सम् पा॑पवस्य॒सं ॅवा ए॒तत् । \newline
44. पा॒प॒व॒स्य॒समिति॑ पाप - व॒स्य॒सम् । \newline
45. वा ए॒त दे॒तद् वै वा ए॒तत् क्रि॑यते क्रियत ए॒तद् वै वा ए॒तत् क्रि॑यते । \newline
46. ए॒तत् क्रि॑यते क्रियत ए॒त दे॒तत् क्रि॑यते॒ यद् यत् क्रि॑यत ए॒त दे॒तत् क्रि॑यते॒ यत् । \newline
47. क्रि॒य॒ते॒ यद् यत् क्रि॑यते क्रियते॒ यच्छ्रेय॑सा॒ श्रेय॑सा॒ यत् क्रि॑यते क्रियते॒ यच्छ्रेय॑सा । \newline
48. यच्छ्रेय॑सा॒ श्रेय॑सा॒ यद् यच्छ्रेय॑सा च च॒ श्रेय॑सा॒ यद् यच्छ्रेय॑सा च । \newline
49. श्रेय॑सा च च॒ श्रेय॑सा॒ श्रेय॑सा च॒ पापी॑यसा॒ पापी॑यसा च॒ श्रेय॑सा॒ श्रेय॑सा च॒ पापी॑यसा । \newline
50. च॒ पापी॑यसा॒ पापी॑यसा च च॒ पापी॑यसा च च॒ पापी॑यसा च च॒ पापी॑यसा च । \newline
51. पापी॑यसा च च॒ पापी॑यसा॒ पापी॑यसा च समा॒नꣳ स॑मा॒नम् च॒ पापी॑यसा॒ पापी॑यसा च समा॒नम् । \newline
52. च॒ स॒मा॒नꣳ स॑मा॒नम् च॑ च समा॒नम् कर्म॒ कर्म॑ समा॒नम् च॑ च समा॒नम् कर्म॑ । \newline
53. स॒मा॒नम् कर्म॒ कर्म॑ समा॒नꣳ स॑मा॒नम् कर्म॑ कु॒र्वन्ति॑ कु॒र्वन्ति॒ कर्म॑ समा॒नꣳ स॑मा॒नम् कर्म॑ कु॒र्वन्ति॑ । \newline
54. कर्म॑ कु॒र्वन्ति॑ कु॒र्वन्ति॒ कर्म॒ कर्म॑ कु॒र्वन्ति॒ पापी॑या॒न् पापी॑यान् कु॒र्वन्ति॒ कर्म॒ कर्म॑ कु॒र्वन्ति॒ पापी॑यान् । \newline
55. कु॒र्वन्ति॒ पापी॑या॒न् पापी॑यान् कु॒र्वन्ति॑ कु॒र्वन्ति॒ पापी॑या॒न्॒. हि हि पापी॑यान् कु॒र्वन्ति॑ कु॒र्वन्ति॒ पापी॑या॒न्॒. हि । \newline
56. पापी॑या॒न्॒. हि हि पापी॑या॒न् पापी॑या॒न् ह्यश्वा॒ दश्वा॒ द्धि पापी॑या॒न् पापी॑या॒न् ह्यश्वा᳚त् । \newline
\pagebreak
\markright{ TS 5.1.2.3  \hfill https://www.vedavms.in \hfill}

\section{ TS 5.1.2.3 }

\textbf{TS 5.1.2.3 } \newline
\textbf{Samhita Paata} \newline

ह्यश्वा᳚द्-गर्द॒भोऽश्वं॒ पूर्वं॑ नयन्ति पापवस्य॒-सस्य॒ व्यावृ॑त्त्यै॒ तस्मा॒च्छ्रेयाꣳ॑सं॒ पापी॑यान् प॒श्चादन्वे॑ति ब॒हुर्वै भव॑तो॒ भ्रातृ॑व्यो॒ भव॑तीव॒ खलु॒ वा ए॒ष यो᳚ऽग्निं चि॑नु॒ते व॒ज्र्यश्वः॑ प्र॒तूर्व॒न्नेह्य॑व॒-क्राम॒न्न-श॑स्ती॒रित्या॑ह॒ वज्रे॑णै॒व पा॒प्मानं॒ भ्रातृ॑व्य॒मव॑ क्रामति रु॒द्रस्य॒ गाण॑पत्या॒दित्या॑ह रौ॒द्रा वै प॒शवो॑ रु॒द्रादे॒व - [  ] \newline

\textbf{Pada Paata} \newline

हि । अश्वा᳚त् । ग॒र्द॒भः । अश्व᳚म् । पूर्व᳚म् । न॒य॒न्ति॒ । पा॒प॒व॒स्य॒सस्येति॑ पाप - व॒स्य॒सस्य॑ । व्यावृ॑त्त्या॒ इति॑ वि - आवृ॑त्त्यै । तस्मा᳚त् । श्रेयाꣳ॑सम् । पापी॑यान् । प॒श्चात् । अन्विति॑ । ए॒ति॒ । ब॒हुः । वै । भव॑तः । भ्रातृ॑व्यः । भव॑ति । इ॒व॒ । खलु॑ । वै । ए॒षः । यः । अ॒ग्निम् । चि॒नु॒ते । व॒ज्री । अश्वः॑ । प्र॒तूर्व॒न्निति॑ प्र - तूर्वन्न्॑ । एति॑ । इ॒हि॒ । अ॒व॒क्राम॒न्नित्य॑व - क्रामन्न्॑ । अश॑स्तीः । इति॑ । आ॒ह॒ । वज्रे॑ण । ए॒व । पा॒प्मान᳚म् । भ्रातृ॑व्यम् । अवेति॑ । क्रा॒म॒ति॒ । रु॒द्रस्य॑ । गाण॑पत्या॒दिति॒ गाण॑ - प॒त्या॒त् । इति॑ । आ॒ह॒ । रौ॒द्राः । वै । प॒शवः॑ । रु॒द्रात् । ए॒व ।  \newline


\textbf{Krama Paata} \newline

ह्यश्वा᳚त् । अश्वा᳚द् गर्द॒भः । ग॒र्द॒भोऽश्व᳚म् । अश्व॒म् पूर्व᳚म् । पूर्व॑म् नयन्ति । न॒य॒न्ति॒ पा॒प॒व॒स्य॒सस्य॑ । पा॒प॒व॒स्य॒सस्य॒ व्यावृ॑त्त्यै । पा॒प॒व॒स्य॒सस्येति॑ पाप - व॒स्य॒सस्य॑ । व्यावृ॑त्त्यै॒ तस्मा᳚त् । व्यावृ॑त्त्या॒ इति॑ वि - आवृ॑त्त्यै । तस्मा॒च्छ्रेयाꣳ॑सम् । श्रेयाꣳ॑स॒म् पापी॑यान् । पापी॑यान् प॒श्चात् । प॒श्चादनु॑ । अन्वे॑ति । ए॒ति॒ ब॒हुः । ब॒हुर् वै । वै भव॑तः । भव॑तो॒ भ्रातृ॑व्यः । भ्रातृ॑व्यो॒ भव॑ति । भव॑तीव । इ॒व॒ खलु॑ । खलु॒ वै । वा ए॒षः । ए॒ष यः । यो᳚ऽग्निम् । अ॒ग्निम् चि॑नु॒ते । चि॒नु॒ते व॒ज्री । व॒ज्र्यश्वः॑ । अश्वः॑ प्र॒तूर्वन्न्॑ । प्र॒तूर्व॒न्ना । प्र॒तूर्व॒न्निति॑ प्र - तूर्वन्न्॑ । एहि॑ । इ॒ह्य॒व॒क्रामन्न्॑ । अ॒व॒क्राम॒न्नश॑स्तीः । अ॒व॒क्राम॒न्नित्य॑व - क्रामन्न्॑ । अश॑स्ती॒रिति॑ । इत्या॑ह । आ॒ह॒ वज्रे॑ण । वज्रे॑णै॒व । ए॒व पा॒प्मान᳚म् । पा॒प्मान॒म् भ्रातृ॑व्यम् । भ्रातृ॑व्य॒मव॑ । अव॑ क्रामति । क्रा॒म॒ति॒ रु॒द्रस्य॑ । रु॒द्रस्य॒ गाण॑पत्यात् । गाण॑पत्या॒दिति॑ । गाण॑पत्या॒दिति॒ गाण॑ - प॒त्या॒त्॒ । इत्या॑ह । आ॒ह॒ रौ॒द्राः । रौ॒द्रा वै । वै प॒शवः॑ । प॒शवो॑ रु॒द्रात् । रु॒द्रादे॒व \newline

\textbf{Jatai Paata} \newline

1. ह्यश्वा॒ दश्वा॒ द्धि ह्यश्वा᳚त् । \newline
2. अश्वा᳚द् गर्द॒भो ग॑र्द॒भो ऽश्वा॒ दश्वा᳚द् गर्द॒भः । \newline
3. ग॒र्द॒भो ऽश्व॒ मश्व॑म् गर्द॒भो ग॑र्द॒भो ऽश्व᳚म् । \newline
4. अश्व॒म् पूर्व॒म् पूर्व॒ मश्व॒ मश्व॒म् पूर्व᳚म् । \newline
5. पूर्व॑म् नयन्ति नयन्ति॒ पूर्व॒म् पूर्व॑म् नयन्ति । \newline
6. न॒य॒न्ति॒ पा॒प॒व॒स्य॒सस्य॑ पापवस्य॒सस्य॑ नयन्ति नयन्ति पापवस्य॒सस्य॑ । \newline
7. पा॒प॒व॒स्य॒सस्य॒ व्यावृ॑त्त्यै॒ व्यावृ॑त्त्यै पापवस्य॒सस्य॑ पापवस्य॒सस्य॒ व्यावृ॑त्त्यै । \newline
8. पा॒प॒व॒स्य॒सस्येति॑ पाप - व॒स्य॒सस्य॑ । \newline
9. व्यावृ॑त्त्यै॒ तस्मा॒त् तस्मा॒द् व्यावृ॑त्त्यै॒ व्यावृ॑त्त्यै॒ तस्मा᳚त् । \newline
10. व्यावृ॑त्त्या॒ इति॑ वि - आवृ॑त्त्यै । \newline
11. तस्मा॒ च्छ्रेयाꣳ॑सꣳ॒॒ श्रेयाꣳ॑स॒म् तस्मा॒त् तस्मा॒ च्छ्रेयाꣳ॑सम् । \newline
12. श्रेयाꣳ॑स॒म् पापी॑या॒न् पापी॑या॒ञ् छ्रेयाꣳ॑सꣳ॒॒ श्रेयाꣳ॑स॒म् पापी॑यान् । \newline
13. पापी॑यान् प॒श्चात् प॒श्चात् पापी॑या॒न् पापी॑यान् प॒श्चात् । \newline
14. प॒श्चा दन्वनु॑ प॒श्चात् प॒श्चा दनु॑ । \newline
15. अन्वे᳚ त्ये॒ त्यन् वन् वे॑ति । \newline
16. ए॒ति॒ ब॒हुर् ब॒हु रे᳚त्येति ब॒हुः । \newline
17. ब॒हुर् वै वै ब॒हुर् ब॒हुर् वै । \newline
18. वै भव॑तो॒ भव॑तो॒ वै वै भव॑तः । \newline
19. भव॑तो॒ भ्रातृ॑व्यो॒ भ्रातृ॑व्यो॒ भव॑तो॒ भव॑तो॒ भ्रातृ॑व्यः । \newline
20. भ्रातृ॑व्यो॒ भव॑ति॒ भव॑ति॒ भ्रातृ॑व्यो॒ भ्रातृ॑व्यो॒ भव॑ति । \newline
21. भव॑तीवेव॒ भव॑ति॒ भव॑तीव । \newline
22. इ॒व॒ खलु॒ खल्वि॑वेव॒ खलु॑ । \newline
23. खलु॒ वै वै खलु॒ खलु॒ वै । \newline
24. वा ए॒ष ए॒ष वै वा ए॒षः । \newline
25. ए॒ष यो य ए॒ष ए॒ष यः । \newline
26. यो᳚ ऽग्नि म॒ग्निं ॅयो यो᳚ ऽग्निम् । \newline
27. अ॒ग्निम् चि॑नु॒ते चि॑नु॒ते᳚ ऽग्नि म॒ग्निम् चि॑नु॒ते । \newline
28. चि॒नु॒ते व॒ज्री व॒ज्री चि॑नु॒ते चि॑नु॒ते व॒ज्री । \newline
29. व॒ज्र्यश्वो ऽश्वो॑ व॒ज्री व॒ज्र्यश्वः॑ । \newline
30. अश्वः॑ प्र॒तूर्व॑न् प्र॒तूर्व॒न् नश्वो ऽश्वः॑ प्र॒तूर्वन्न्॑ । \newline
31. प्र॒तूर्व॒न् ना प्र॒तूर्व॑न् प्र॒तूर्व॒न् ना । \newline
32. प्र॒तूर्व॒न्निति॑ प्र - तूर्वन्न्॑ । \newline
33. एही॒ह्येहि॑ । \newline
34. इ॒ह्य॒व॒क्राम॑न् नव॒क्राम॑न् निही ह्यव॒क्रामन्न्॑ । \newline
35. अ॒व॒क्राम॒न् नश॑स्ती॒ रश॑स्ती रव॒क्राम॑न् नव॒क्राम॒न् नश॑स्तीः । \newline
36. अ॒व॒क्राम॒न्नित्य॑व - क्रामन्न्॑ । \newline
37. अश॑स्ती॒ रिती त्यश॑स्ती॒ रश॑स्ती॒ रिति॑ । \newline
38. इत्या॑हा॒हे तीत्या॑ह । \newline
39. आ॒ह॒ वज्रे॑ण॒ वज्रे॑णा हाह॒ वज्रे॑ण । \newline
40. वज्रे॑ णै॒वैव वज्रे॑ण॒ वज्रे॑णै॒व । \newline
41. ए॒व पा॒प्मान॑म् पा॒प्मान॑ मे॒वैव पा॒प्मान᳚म् । \newline
42. पा॒प्मान॒म् भ्रातृ॑व्य॒म् भ्रातृ॑व्यम् पा॒प्मान॑म् पा॒प्मान॒म् भ्रातृ॑व्यम् । \newline
43. भ्रातृ॑व्य॒ मवाव॒ भ्रातृ॑व्य॒म् भ्रातृ॑व्य॒ मव॑ । \newline
44. अव॑ क्रामति क्राम॒ त्यवाव॑ क्रामति । \newline
45. क्रा॒म॒ति॒ रु॒द्रस्य॑ रु॒द्रस्य॑ क्रामति क्रामति रु॒द्रस्य॑ । \newline
46. रु॒द्रस्य॒ गाण॑पत्या॒द् गाण॑पत्याद् रु॒द्रस्य॑ रु॒द्रस्य॒ गाण॑पत्यात् । \newline
47. गाण॑पत्या॒ दितीति॒ गाण॑पत्या॒द् गाण॑पत्या॒ दिति॑ । \newline
48. गाण॑पत्या॒दिति॒ गाण॑ - प॒त्या॒त् । \newline
49. इत्या॑हा॒हे तीत्या॑ह । \newline
50. आ॒ह॒ रौ॒द्रा रौ॒द्रा आ॑हाह रौ॒द्राः । \newline
51. रौ॒द्रा वै वै रौ॒द्रा रौ॒द्रा वै । \newline
52. वै प॒शवः॑ प॒शवो॒ वै वै प॒शवः॑ । \newline
53. प॒शवो॑ रु॒द्राद् रु॒द्रात् प॒शवः॑ प॒शवो॑ रु॒द्रात् । \newline
54. रु॒द्रा दे॒वैव रु॒द्राद् रु॒द्रा दे॒व । \newline
55. ए॒व प॒शून् प॒शू ने॒वैव प॒शून् । \newline

\textbf{Ghana Paata } \newline

1. ह्यश्वा॒ दश्वा॒द् धि ह्यश्वा᳚द् गर्द॒भो ग॑र्द॒भो ऽश्वा॒ द्धि ह्यश्वा᳚द् गर्द॒भः । \newline
2. अश्वा᳚द् गर्द॒भो ग॑र्द॒भो ऽश्वा॒ दश्वा᳚द् गर्द॒भो ऽश्व॒ मश्व॑म् गर्द॒भो ऽश्वा॒ दश्वा᳚द् गर्द॒भो ऽश्व᳚म् । \newline
3. ग॒र्द॒भो ऽश्व॒ मश्व॑म् गर्द॒भो ग॑र्द॒भो ऽश्व॒म् पूर्व॒म् पूर्व॒ मश्व॑म् गर्द॒भो ग॑र्द॒भो ऽश्व॒म् पूर्व᳚म् । \newline
4. अश्व॒म् पूर्व॒म् पूर्व॒ मश्व॒ मश्व॒म् पूर्व॑म् नयन्ति नयन्ति॒ पूर्व॒ मश्व॒ मश्व॒म् पूर्व॑म् नयन्ति । \newline
5. पूर्व॑म् नयन्ति नयन्ति॒ पूर्व॒म् पूर्व॑म् नयन्ति पापवस्य॒सस्य॑ पापवस्य॒सस्य॑ नयन्ति॒ पूर्व॒म् पूर्व॑म् नयन्ति पापवस्य॒सस्य॑ । \newline
6. न॒य॒न्ति॒ पा॒प॒व॒स्य॒सस्य॑ पापवस्य॒सस्य॑ नयन्ति नयन्ति पापवस्य॒सस्य॒ व्यावृ॑त्त्यै॒ व्यावृ॑त्त्यै पापवस्य॒सस्य॑ नयन्ति नयन्ति पापवस्य॒सस्य॒ व्यावृ॑त्त्यै । \newline
7. पा॒प॒व॒स्य॒सस्य॒ व्यावृ॑त्त्यै॒ व्यावृ॑त्त्यै पापवस्य॒सस्य॑ पापवस्य॒सस्य॒ व्यावृ॑त्त्यै॒ तस्मा॒त् तस्मा॒द् व्यावृ॑त्त्यै पापवस्य॒सस्य॑ पापवस्य॒सस्य॒ व्यावृ॑त्त्यै॒ तस्मा᳚त् । \newline
8. पा॒प॒व॒स्य॒सस्येति॑ पाप - व॒स्य॒सस्य॑ । \newline
9. व्यावृ॑त्त्यै॒ तस्मा॒त् तस्मा॒द् व्यावृ॑त्त्यै॒ व्यावृ॑त्त्यै॒ तस्मा॒ च्छ्रेयाꣳ॑सꣳ॒॒ श्रेयाꣳ॑स॒म् तस्मा॒द् व्यावृ॑त्त्यै॒ व्यावृ॑त्त्यै॒ तस्मा॒ च्छ्रेयाꣳ॑सम् । \newline
10. व्यावृ॑त्त्या॒ इति॑ वि - आवृ॑त्त्यै । \newline
11. तस्मा॒ च्छ्रेयाꣳ॑सꣳ॒॒ श्रेयाꣳ॑स॒म् तस्मा॒त् तस्मा॒ च्छ्रेयाꣳ॑स॒म् पापी॑या॒न् पापी॑या॒ञ् छ्रेयाꣳ॑स॒म् तस्मा॒त् तस्मा॒ च्छ्रेयाꣳ॑स॒म् पापी॑यान् । \newline
12. श्रेयाꣳ॑स॒म् पापी॑या॒न् पापी॑या॒ञ् छ्रेयाꣳ॑सꣳ॒॒ श्रेयाꣳ॑स॒म् पापी॑यान् प॒श्चात् प॒श्चात् पापी॑या॒ञ् छ्रेयाꣳ॑सꣳ॒॒ श्रेयाꣳ॑स॒म् पापी॑यान् प॒श्चात् । \newline
13. पापी॑यान् प॒श्चात् प॒श्चात् पापी॑या॒न् पापी॑यान् प॒श्चादन्वनु॑ प॒श्चात् पापी॑या॒न् पापी॑यान् प॒श्चादनु॑ । \newline
14. प॒श्चा दन्वनु॑ प॒श्चात् प॒श्चा दन्वे᳚ त्ये॒त्यनु॑ प॒श्चात् प॒श्चा दन्वे॑ति । \newline
15. अन्वे᳚ त्ये॒त्यन् वन् वे॑ति ब॒हुर् ब॒हु रे॒त्यन्वन् वे॑ति ब॒हुः । \newline
16. ए॒ति॒ ब॒हुर् ब॒हु रे᳚त्येति ब॒हुर् वै वै ब॒हु रे᳚त्येति ब॒हुर् वै । \newline
17. ब॒हुर् वै वै ब॒हुर् ब॒हुर् वै भव॑तो॒ भव॑तो॒ वै ब॒हुर् ब॒हुर् वै भव॑तः । \newline
18. वै भव॑तो॒ भव॑तो॒ वै वै भव॑तो॒ भ्रातृ॑व्यो॒ भ्रातृ॑व्यो॒ भव॑तो॒ वै वै भव॑तो॒ भ्रातृ॑व्यः । \newline
19. भव॑तो॒ भ्रातृ॑व्यो॒ भ्रातृ॑व्यो॒ भव॑तो॒ भव॑तो॒ भ्रातृ॑व्यो॒ भव॑ति॒ भव॑ति॒ भ्रातृ॑व्यो॒ भव॑तो॒ भव॑तो॒ भ्रातृ॑व्यो॒ भव॑ति । \newline
20. भ्रातृ॑व्यो॒ भव॑ति॒ भव॑ति॒ भ्रातृ॑व्यो॒ भ्रातृ॑व्यो॒ भव॑तीवेव॒ भव॑ति॒ भ्रातृ॑व्यो॒ भ्रातृ॑व्यो॒ भव॑तीव । \newline
21. भव॑तीवेव॒ भव॑ति॒ भव॑तीव॒ खलु॒ खल्वि॑व॒ भव॑ति॒ भव॑तीव॒ खलु॑ । \newline
22. इ॒व॒ खलु॒ खल्वि॑वेव॒ खलु॒ वै वै खल्वि॑वे व॒ खलु॒ वै । \newline
23. खलु॒ वै वै खलु॒ खलु॒ वा ए॒ष ए॒ष वै खलु॒ खलु॒ वा ए॒षः । \newline
24. वा ए॒ष ए॒ष वै वा ए॒ष यो य ए॒ष वै वा ए॒ष यः । \newline
25. ए॒ष यो य ए॒ष ए॒ष यो᳚ ऽग्नि म॒ग्निं ॅय ए॒ष ए॒ष यो᳚ ऽग्निम् । \newline
26. यो᳚ ऽग्नि म॒ग्निं ॅयो यो᳚ ऽग्निम् चि॑नु॒ते चि॑नु॒ते᳚ ऽग्निं ॅयो यो᳚ ऽग्निम् चि॑नु॒ते । \newline
27. अ॒ग्निम् चि॑नु॒ते चि॑नु॒ते᳚ ऽग्नि म॒ग्निम् चि॑नु॒ते व॒ज्री व॒ज्री चि॑नु॒ते᳚ ऽग्नि म॒ग्निम् चि॑नु॒ते व॒ज्री । \newline
28. चि॒नु॒ते व॒ज्री व॒ज्री चि॑नु॒ते चि॑नु॒ते व॒ज्‌र्यश्वो ऽश्वो॑ व॒ज्री चि॑नु॒ते चि॑नु॒ते व॒ज्‌र्यश्वः॑ । \newline
29. व॒ज्‌र्यश्वो ऽश्वो॑ व॒ज्री व॒ज्‌र्यश्वः॑ प्र॒तूर्व॑न् प्र॒तूर्व॒न् नश्वो॑ व॒ज्री व॒ज्‌र्यश्वः॑ प्र॒तूर्वन्न्॑ । \newline
30. अश्वः॑ प्र॒तूर्व॑न् प्र॒तूर्व॒न् नश्वो ऽश्वः॑ प्र॒तूर्व॒न् ना प्र॒तूर्व॒न् नश्वो ऽश्वः॑ प्र॒तूर्व॒न् ना । \newline
31. प्र॒तूर्व॒न् ना प्र॒तूर्व॑न् प्र॒तूर्व॒न् नेही॒ह्या प्र॒तूर्व॑न् प्र॒तूर्व॒न् नेहि॑ । \newline
32. प्र॒तूर्व॒न्निति॑ प्र - तूर्वन्न्॑ । \newline
33. एही॒ह्ये ह्य॑व॒क्राम॑न् नव॒क्राम॑न् नि॒ह्ये ह्य॑व॒क्रामन्न्॑ । \newline
34. इ॒ह्य॒व॒क्राम॑न् नव॒क्राम॑न् निही ह्यव॒क्राम॒न् नश॑स्ती॒ रश॑स्ती रव॒क्राम॑न् निही ह्यव॒क्राम॒न् नश॑स्तीः । \newline
35. अ॒व॒क्राम॒न् नश॑स्ती॒ रश॑स्ती रव॒क्राम॑न् नव॒क्राम॒न् नश॑स्ती॒ रिती त्यश॑स्ती रव॒क्राम॑न् नव॒क्राम॒न् नश॑स्ती॒ रिति॑ । \newline
36. अ॒व॒क्राम॒न्नित्य॑व - क्रामन्न्॑ । \newline
37. अश॑स्ती॒ रिती त्यश॑स्ती॒ रश॑स्ती॒ रित्या॑हा॒हे त्यश॑स्ती॒ रश॑स्ती॒ रित्या॑ह । \newline
38. इत्या॑हा॒हे तीत्या॑ह॒ वज्रे॑ण॒ वज्रे॑णा॒हे तीत्या॑ह॒ वज्रे॑ण । \newline
39. आ॒ह॒ वज्रे॑ण॒ वज्रे॑णाहाह॒ वज्रे॑णै॒वैव वज्रे॑णाहाह॒ वज्रे॑णै॒व । \newline
40. वज्रे॑णै॒वैव वज्रे॑ण॒ वज्रे॑णै॒व पा॒प्मान॑म् पा॒प्मान॑ मे॒व वज्रे॑ण॒ वज्रे॑णै॒व पा॒प्मान᳚म् । \newline
41. ए॒व पा॒प्मान॑म् पा॒प्मान॑ मे॒वैव पा॒प्मान॒म् भ्रातृ॑व्य॒म् भ्रातृ॑व्यम् पा॒प्मान॑ मे॒वैव पा॒प्मान॒म् भ्रातृ॑व्यम् । \newline
42. पा॒प्मान॒म् भ्रातृ॑व्य॒म् भ्रातृ॑व्यम् पा॒प्मान॑म् पा॒प्मान॒म् भ्रातृ॑व्य॒ मवाव॒ भ्रातृ॑व्यम् पा॒प्मान॑म् पा॒प्मान॒म् भ्रातृ॑व्य॒ मव॑ । \newline
43. भ्रातृ॑व्य॒ मवाव॒ भ्रातृ॑व्य॒म् भ्रातृ॑व्य॒ मव॑ क्रामति क्राम॒ त्यव॒ भ्रातृ॑व्य॒म् भ्रातृ॑व्य॒ मव॑ क्रामति । \newline
44. अव॑ क्रामति क्राम॒ त्यवाव॑ क्रामति रु॒द्रस्य॑ रु॒द्रस्य॑ क्राम॒ त्यवाव॑ क्रामति रु॒द्रस्य॑ । \newline
45. क्रा॒म॒ति॒ रु॒द्रस्य॑ रु॒द्रस्य॑ क्रामति क्रामति रु॒द्रस्य॒ गाण॑पत्या॒द् गाण॑पत्याद् रु॒द्रस्य॑ क्रामति क्रामति रु॒द्रस्य॒ गाण॑पत्यात् । \newline
46. रु॒द्रस्य॒ गाण॑पत्या॒द् गाण॑पत्याद् रु॒द्रस्य॑ रु॒द्रस्य॒ गाण॑पत्या॒ दितीति॒ गाण॑पत्याद् रु॒द्रस्य॑ रु॒द्रस्य॒ गाण॑पत्या॒ दिति॑ । \newline
47. गाण॑पत्या॒ दितीति॒ गाण॑पत्या॒द् गाण॑पत्या॒ दित्या॑हा॒हेति॒ गाण॑पत्या॒द् गाण॑पत्या॒ दित्या॑ह । \newline
48. गाण॑पत्या॒दिति॒ गाण॑ - प॒त्या॒त् । \newline
49. इत्या॑ हा॒हेती त्या॑ह रौ॒द्रा रौ॒द्रा आ॒हेती त्या॑ह रौ॒द्राः । \newline
50. आ॒ह॒ रौ॒द्रा रौ॒द्रा आ॑हाह रौ॒द्रा वै वै रौ॒द्रा आ॑हाह रौ॒द्रा वै । \newline
51. रौ॒द्रा वै वै रौ॒द्रा रौ॒द्रा वै प॒शवः॑ प॒शवो॒ वै रौ॒द्रा रौ॒द्रा वै प॒शवः॑ । \newline
52. वै प॒शवः॑ प॒शवो॒ वै वै प॒शवो॑ रु॒द्राद् रु॒द्रात् प॒शवो॒ वै वै प॒शवो॑ रु॒द्रात् । \newline
53. प॒शवो॑ रु॒द्राद् रु॒द्रात् प॒शवः॑ प॒शवो॑ रु॒द्रा दे॒वैव रु॒द्रात् प॒शवः॑ प॒शवो॑ रु॒द्रा दे॒व । \newline
54. रु॒द्रा दे॒वैव रु॒द्राद् रु॒द्रा दे॒व प॒शून् प॒शू ने॒व रु॒द्राद् रु॒द्रा दे॒व प॒शून् । \newline
55. ए॒व प॒शून् प॒शू ने॒वैव प॒शून् नि॒र्याच्य॑ नि॒र्याच्य॑ प॒शू ने॒वैव प॒शून् नि॒र्याच्य॑ । \newline
\pagebreak
\markright{ TS 5.1.2.4  \hfill https://www.vedavms.in \hfill}

\section{ TS 5.1.2.4 }

\textbf{TS 5.1.2.4 } \newline
\textbf{Samhita Paata} \newline

प॒शून् नि॒र्याच्या॒ऽऽ*त्मने॒ कर्म॑ कुरुते पू॒ष्णा स॒युजा॑ स॒हेत्या॑ह पू॒षा वा अद्ध्व॑नाꣳ सन्ने॒ता सम॑ष्ट्यै॒ पुरी॑षायतनो॒ वा ए॒ष यद॒ग्निरङ्गि॑रसो॒ वा ए॒तमग्रे॑ दे॒वता॑नाꣳ॒॒ सम॑भरन् पृथि॒व्याः स॒धस्था॑द॒ग्निं पु॑री॒ष्य॑-मङ्गिर॒स्व-दच्छे॒हीत्या॑ह॒ साय॑तनमे॒वैनं॑ दे॒वता॑भिः॒ सं भ॑रत्य॒ग्निं पु॑री॒ष्य॑-मङ्गिर॒स्व- दच्छे॑म॒ इत्या॑ह॒ येन॑ - [  ] \newline

\textbf{Pada Paata} \newline

प॒शून् । नि॒र्याच्येति॑ निः - याच्य॑ । आ॒त्मने᳚ । कर्म॑ । कु॒रु॒ते॒ । पू॒ष्णा । स॒युजेति॑ स-युजा᳚ । स॒ह । इति॑ । आ॒ह॒ । पू॒षा । वै । अद्ध्व॑नाम् । स॒नें॒तेति॑ सं - ने॒ता । सम॑ष्ट्या॒ इति॒ सं - अ॒ष्ट्यै॒ । पुरी॑षायतन॒ इति॒ पुरी॑ष - आ॒य॒त॒नः॒ । वै । ए॒षः । यत् । अ॒ग्निः । अङ्गि॑रसः । वै । ए॒तम् । अग्रे᳚ । दे॒वता॑नाम् । समिति॑ । अ॒भ॒र॒न्न् । पृ॒थि॒व्याः । स॒धस्था॒दिति॑ स॒ध - स्था॒त् । अ॒ग्निम् । पु॒री॒ष्य᳚म् । अ॒ङ्गि॒र॒स्वत् । अच्छ॑ । इ॒हि॒ । इति॑ । आ॒ह॒ । साय॑तन॒मिति॒ स - आ॒य॒त॒न॒म् । ए॒व । ए॒न॒म् । दे॒वता॑भिः । समिति॑ । भ॒र॒ति॒ । अ॒ग्निम् । पु॒री॒ष्य᳚म् । अ॒ङ्गि॒र॒स्वत् । अच्छ॑ । इ॒मः॒ । इति॑ । आ॒ह॒ । येन॑ ।  \newline


\textbf{Krama Paata} \newline

ए॒व प॒शून् । प॒शुन् नि॒र्याच्य॑ । नि॒र्याच्या॒त्मने᳚ । नि॒र्याच्येति॑ निः - याच्य॑ । आ॒त्मने॒ कर्म॑ । कर्म॑ कुरुते । कु॒रु॒ते॒ पू॒ष्णा । पू॒ष्णा स॒युजा᳚ । स॒युजा॑ स॒ह । स॒युजेति॑ स - युजा᳚ । स॒हेति॑ । इत्या॑ह । आ॒ह॒ पू॒षा । पू॒षा वै । वा अद्ध्व॑नाम् । अद्ध्व॑नाꣳ सन्ने॒ता । स॒न्ने॒ता सम॑ष्ट्यै । स॒न्ने॒तेति॑ सम् - ने॒ता । सम॑ष्ट्यै॒ पुरी॑षायतनः । सम॑ष्ट्या॒ इति॒ सम् - अ॒ष्ट्यै॒ । पुरी॑षायतनो॒ वै । पुरी॑षायतन॒ इति॒ पुरी॑ष - आ॒य॒त॒नः॒ । वा ए॒षः । ए॒ष यत् । यद॒ग्निः । अ॒ग्निरङ्गि॑रसः । अङ्गि॑रसो॒ वै । वा ए॒तम् । ए॒तमग्रे᳚ । अग्रे॑ दे॒वता॑नाम् । दे॒वता॑नाꣳ॒॒ सम् । सम॑भरन्न् । अ॒भ॒र॒न् पृ॒थि॒व्याः । पृ॒थि॒व्याः स॒धस्था᳚त् । स॒धस्था॑द॒ग्निम् । स॒धस्था॒दिति॑ स॒ध - स्था॒त्॒ । अ॒ग्निम् पु॑री॒ष्य᳚म् । पु॒री॒ष्य॑मङ्गिर॒स्वत् । अ॒ङ्गि॒र॒स्वदच्छ॑ । अच्छे॑हि । एहीति॑ । इत्या॑ह । आ॒ह॒ साय॑तनम् । साय॑तनमे॒व । साय॑तन॒मिति॒ स - आ॒य॒त॒न॒म् । ए॒वैन᳚म् । ए॒न॒म् दे॒वता॑भिः । दे॒वता॑भिः॒ सम् । सम् भ॑रति । भ॒र॒त्य॒ग्निम् । अ॒ग्निम् पु॑री॒ष्य᳚म् । पु॒री॒ष्य॑मङ्गिर॒स्वत् । अ॒ङ्गि॒र॒स्वदच्छ॑ । अच्छे॑मः । इ॒म॒ इति॑ । इत्या॑ह । आ॒ह॒ येन॑ । येन॑ स॒ङ्गच्छ॑ते \newline

\textbf{Jatai Paata} \newline

1. प॒शून् नि॒र्याच्य॑ नि॒र्याच्य॑ प॒शून् प॒शून् नि॒र्याच्य॑ । \newline
2. नि॒र्याच्या॒त्मन॑ आ॒त्मने॑ नि॒र्याच्य॑ नि॒र्याच्या॒त्मने᳚ । \newline
3. नि॒र्याच्येति॑ निः - याच्य॑ । \newline
4. आ॒त्मने॒ कर्म॒ कर्मा॒त्मन॑ आ॒त्मने॒ कर्म॑ । \newline
5. कर्म॑ कुरुते कुरुते॒ कर्म॒ कर्म॑ कुरुते । \newline
6. कु॒रु॒ते॒ पू॒ष्णा पू॒ष्णा कु॑रुते कुरुते पू॒ष्णा । \newline
7. पू॒ष्णा स॒युजा॑ स॒युजा॑ पू॒ष्णा पू॒ष्णा स॒युजा᳚ । \newline
8. स॒युजा॑ स॒ह स॒ह स॒युजा॑ स॒युजा॑ स॒ह । \newline
9. स॒युजेति॑ स - युजा᳚ । \newline
10. स॒हे तीति॑ स॒ह स॒हेति॑ । \newline
11. इत्या॑हा॒हे तीत्या॑ह । \newline
12. आ॒ह॒ पू॒षा पू॒षा ऽऽहा॑ह पू॒षा । \newline
13. पू॒षा वै वै पू॒षा पू॒षा वै । \newline
14. वा अद्ध्व॑ना॒ मद्ध्व॑नां॒ ॅवै वा अद्ध्व॑नाम् । \newline
15. अद्ध्व॑नाꣳ सन्ने॒ता स॑न्ने॒ता ऽद्ध्व॑ना॒ मद्ध्व॑नाꣳ सन्ने॒ता । \newline
16. स॒न्ने॒ता सम॑ष्ट्यै॒ सम॑ष्ट्यै सन्ने॒ता स॑न्ने॒ता सम॑ष्ट्यै । \newline
17. स॒न्ने॒तेति॑ सं - ने॒ता । \newline
18. सम॑ष्ट्यै॒ पुरी॑षायतनः॒ पुरी॑षायतनः॒ सम॑ष्ट्यै॒ सम॑ष्ट्यै॒ पुरी॑षायतनः । \newline
19. सम॑ष्ट्या॒ इति॒ सं - अ॒ष्ट्यै॒ । \newline
20. पुरी॑षायतनो॒ वै वै पुरी॑षायतनः॒ पुरी॑षायतनो॒ वै । \newline
21. पुरी॑षायतन॒ इति॒ पुरी॑ष - आ॒य॒त॒नः॒ । \newline
22. वा ए॒ष ए॒ष वै वा ए॒षः । \newline
23. ए॒ष यद् यदे॒ष ए॒ष यत् । \newline
24. यद॒ग्नि र॒ग्निर् यद् यद॒ग्निः । \newline
25. अ॒ग्नि रङ्गि॑र॒सो ऽङ्गि॑रसो॒ ऽग्नि र॒ग्नि रङ्गि॑रसः । \newline
26. अङ्गि॑रसो॒ वै वा अङ्गि॑र॒सो ऽङ्गि॑रसो॒ वै । \newline
27. वा ए॒त मे॒तं ॅवै वा ए॒तम् । \newline
28. ए॒त मग्रे ऽग्र॑ ए॒त मे॒त मग्रे᳚ । \newline
29. अग्रे॑ दे॒वता॑नाम् दे॒वता॑ना॒ मग्रे ऽग्रे॑ दे॒वता॑नाम् । \newline
30. दे॒वता॑नाꣳ॒॒ सꣳ सम् दे॒वता॑नाम् दे॒वता॑नाꣳ॒॒ सम् । \newline
31. स म॑भरन् नभर॒न् थ्सꣳ स म॑भरन्न् । \newline
32. अ॒भ॒र॒न् पृ॒थि॒व्याः पृ॑थि॒व्या अ॑भरन् नभरन् पृथि॒व्याः । \newline
33. पृ॒थि॒व्याः स॒धस्था᳚थ् स॒धस्था᳚त् पृथि॒व्याः पृ॑थि॒व्याः स॒धस्था᳚त् । \newline
34. स॒धस्था॑ द॒ग्नि म॒ग्निꣳ स॒धस्था᳚थ् स॒धस्था॑ द॒ग्निम् । \newline
35. स॒धस्था॒दिति॑ स॒ध - स्था॒त् । \newline
36. अ॒ग्निम् पु॑री॒ष्य॑म् पुरी॒ष्य॑ म॒ग्नि म॒ग्निम् पु॑री॒ष्य᳚म् । \newline
37. पु॒री॒ष्य॑ मङ्गिर॒स्व द॑ङ्गिर॒स्वत् पु॑री॒ष्य॑म् पुरी॒ष्य॑ मङ्गिर॒स्वत् । \newline
38. अ॒ङ्गि॒र॒स्व दच्छाच्छा᳚ ङ्गिर॒स्व द॑ङ्गिर॒स्व दच्छ॑ । \newline
39. अच्छे॑ ही॒ ह्यच्छा च्छे॑हि । \newline
40. इ॒हीतीती॑ ही॒हीति॑ । \newline
41. इत्या॑हा॒हे तीत्या॑ह । \newline
42. आ॒ह॒ साय॑तनꣳ॒॒ साय॑तन माहाह॒ साय॑तनम् । \newline
43. साय॑तन मे॒वैव साय॑तनꣳ॒॒ साय॑तन मे॒व । \newline
44. साय॑तन॒मिति॒ स - आ॒य॒त॒न॒म् । \newline
45. ए॒वैन॑ मेन मे॒वैवैन᳚म् । \newline
46. ए॒न॒म् दे॒वता॑भिर् दे॒वता॑भि रेन मेनम् दे॒वता॑भिः । \newline
47. दे॒वता॑भिः॒ सꣳ सम् दे॒वता॑भिर् दे॒वता॑भिः॒ सम् । \newline
48. सम् भ॑रति भरति॒ सꣳ सम् भ॑रति । \newline
49. भ॒र॒ त्य॒ग्नि म॒ग्निम् भ॑रति भर त्य॒ग्निम् । \newline
50. अ॒ग्निम् पु॑री॒ष्य॑म् पुरी॒ष्य॑ म॒ग्नि म॒ग्निम् पु॑री॒ष्य᳚म् । \newline
51. पु॒री॒ष्य॑ मङ्गिर॒स्व द॑ङ्गिर॒स्वत् पु॑री॒ष्य॑म् पुरी॒ष्य॑ मङ्गिर॒स्वत् । \newline
52. अ॒ङ्गि॒र॒स्व दच्छाच्छा᳚ ङ्गिर॒स्व द॑ङ्गिर॒स्व दच्छ॑ । \newline
53. अच्छे॑म इमो॒ अच्छाच्छे॑मः । \newline
54. इ॒म॒ इतीती॑म इम॒ इति॑ । \newline
55. इत्या॑हा॒हे तीत्या॑ह । \newline
56. आ॒ह॒ येन॒ येना॑हाह॒ येन॑ । \newline
57. येन॑ स॒ङ्गच्छ॑ते स॒ङ्गच्छ॑ते॒ येन॒ येन॑ स॒ङ्गच्छ॑ते । \newline

\textbf{Ghana Paata } \newline

1. प॒शून् नि॒र्याच्य॑ नि॒र्याच्य॑ प॒शून् प॒शून् नि॒र्याच्या॒त्मन॑ आ॒त्मने॑ नि॒र्याच्य॑ प॒शून् प॒शून् नि॒र्याच्या॒त्मने᳚ । \newline
2. नि॒र्याच्या॒त्मन॑ आ॒त्मने॑ नि॒र्याच्य॑ नि॒र्याच्या॒त्मने॒ कर्म॒ कर्मा॒त्मने॑ नि॒र्याच्य॑ नि॒र्याच्या॒त्मने॒ कर्म॑ । \newline
3. नि॒र्याच्येति॑ निः - याच्य॑ । \newline
4. आ॒त्मने॒ कर्म॒ कर्मा॒त्मन॑ आ॒त्मने॒ कर्म॑ कुरुते कुरुते॒ कर्मा॒त्मन॑ आ॒त्मने॒ कर्म॑ कुरुते । \newline
5. कर्म॑ कुरुते कुरुते॒ कर्म॒ कर्म॑ कुरुते पू॒ष्णा पू॒ष्णा कु॑रुते॒ कर्म॒ कर्म॑ कुरुते पू॒ष्णा । \newline
6. कु॒रु॒ते॒ पू॒ष्णा पू॒ष्णा कु॑रुते कुरुते पू॒ष्णा स॒युजा॑ स॒युजा॑ पू॒ष्णा कु॑रुते कुरुते पू॒ष्णा स॒युजा᳚ । \newline
7. पू॒ष्णा स॒युजा॑ स॒युजा॑ पू॒ष्णा पू॒ष्णा स॒युजा॑ स॒ह स॒ह स॒युजा॑ पू॒ष्णा पू॒ष्णा स॒युजा॑ स॒ह । \newline
8. स॒युजा॑ स॒ह स॒ह स॒युजा॑ स॒युजा॑ स॒हे तीति॑ स॒ह स॒युजा॑ स॒युजा॑ स॒हेति॑ । \newline
9. स॒युजेति॑ स - युजा᳚ । \newline
10. स॒हे तीति॑ स॒ह स॒हे त्या॑हा॒हेति॑ स॒ह स॒हे त्या॑ह । \newline
11. इत्या॑ हा॒हेतीत्या॑ह पू॒षा पू॒षा ऽऽहेतीत्या॑ह पू॒षा । \newline
12. आ॒ह॒ पू॒षा पू॒षा ऽऽहा॑ह पू॒षा वै वै पू॒षा ऽऽहा॑ह पू॒षा वै । \newline
13. पू॒षा वै वै पू॒षा पू॒षा वा अद्ध्व॑ना॒ मद्ध्व॑नां॒ ॅवै पू॒षा पू॒षा वा अद्ध्व॑नाम् । \newline
14. वा अद्ध्व॑ना॒ मद्ध्व॑नां॒ ॅवै वा अद्ध्व॑नाꣳ सन्ने॒ता स॑न्ने॒ता ऽद्ध्व॑नां॒ ॅवै वा अद्ध्व॑नाꣳ सन्ने॒ता । \newline
15. अद्ध्व॑नाꣳ सन्ने॒ता स॑न्ने॒ता ऽद्ध्व॑ना॒ मद्ध्व॑नाꣳ सन्ने॒ता सम॑ष्ट्यै॒ सम॑ष्ट्यै सन्ने॒ता ऽद्ध्व॑ना॒ मद्ध्व॑नाꣳ सन्ने॒ता सम॑ष्ट्यै । \newline
16. स॒न्ने॒ता सम॑ष्ट्यै॒ सम॑ष्ट्यै सन्ने॒ता स॑न्ने॒ता सम॑ष्ट्यै॒ पुरी॑षायतनः॒ पुरी॑षायतनः॒ सम॑ष्ट्यै सन्ने॒ता स॑न्ने॒ता सम॑ष्ट्यै॒ पुरी॑षायतनः । \newline
17. स॒न्ने॒तेति॑ सं - ने॒ता । \newline
18. सम॑ष्ट्यै॒ पुरी॑षायतनः॒ पुरी॑षायतनः॒ सम॑ष्ट्यै॒ सम॑ष्ट्यै॒ पुरी॑षायतनो॒ वै वै पुरी॑षायतनः॒ सम॑ष्ट्यै॒ सम॑ष्ट्यै॒ पुरी॑षायतनो॒ वै । \newline
19. सम॑ष्ट्या॒ इति॒ सं - अ॒ष्ट्यै॒ । \newline
20. पुरी॑षायतनो॒ वै वै पुरी॑षायतनः॒ पुरी॑षायतनो॒ वा ए॒ष ए॒ष वै पुरी॑षायतनः॒ पुरी॑षायतनो॒ वा ए॒षः । \newline
21. पुरी॑षायतन॒ इति॒ पुरी॑ष - आ॒य॒त॒नः॒ । \newline
22. वा ए॒ष ए॒ष वै वा ए॒ष यद् यदे॒ष वै वा ए॒ष यत् । \newline
23. ए॒ष यद् यदे॒ष ए॒ष यद॒ग्नि र॒ग्निर् यदे॒ष ए॒ष यद॒ग्निः । \newline
24. यद॒ग्नि र॒ग्निर् यद् यद॒ग्नि रङ्गि॑र॒सो ऽङ्गि॑रसो॒ ऽग्निर् यद् यद॒ग्नि रङ्गि॑रसः । \newline
25. अ॒ग्नि रङ्गि॑र॒सो ऽङ्गि॑रसो॒ ऽग्नि र॒ग्नि रङ्गि॑रसो॒ वै वा अङ्गि॑रसो॒ ऽग्नि र॒ग्नि रङ्गि॑रसो॒ वै । \newline
26. अङ्गि॑रसो॒ वै वा अङ्गि॑र॒सो ऽङ्गि॑रसो॒ वा ए॒त मे॒तं ॅवा अङ्गि॑र॒सो ऽङ्गि॑रसो॒ वा ए॒तम् । \newline
27. वा ए॒त मे॒तं ॅवै वा ए॒त मग्रे ऽग्र॑ ए॒तं ॅवै वा ए॒त मग्रे᳚ । \newline
28. ए॒त मग्रे ऽग्र॑ ए॒त मे॒त मग्रे॑ दे॒वता॑नाम् दे॒वता॑ना॒ मग्र॑ ए॒त मे॒त मग्रे॑ दे॒वता॑नाम् । \newline
29. अग्रे॑ दे॒वता॑नाम् दे॒वता॑ना॒ मग्रे ऽग्रे॑ दे॒वता॑नाꣳ॒॒ सꣳ सम् दे॒वता॑ना॒ मग्रे ऽग्रे॑ दे॒वता॑नाꣳ॒॒ सम् । \newline
30. दे॒वता॑नाꣳ॒॒ सꣳ सम् दे॒वता॑नाम् दे॒वता॑नाꣳ॒॒ स म॑भरन् नभर॒न् थ्सम् दे॒वता॑नाम् दे॒वता॑नाꣳ॒॒ स म॑भरन्न् । \newline
31. स म॑भरन् नभर॒न् थ्सꣳ स म॑भरन् पृथि॒व्याः पृ॑थि॒व्या अ॑भर॒न् थ्सꣳ स म॑भरन् पृथि॒व्याः । \newline
32. अ॒भ॒र॒न् पृ॒थि॒व्याः पृ॑थि॒व्या अ॑भरन् नभरन् पृथि॒व्याः स॒धस्था᳚थ् स॒धस्था᳚त् पृथि॒व्या अ॑भरन् नभरन् पृथि॒व्याः स॒धस्था᳚त् । \newline
33. पृ॒थि॒व्याः स॒धस्था᳚थ् स॒धस्था᳚त् पृथि॒व्याः पृ॑थि॒व्याः स॒धस्था॑ द॒ग्नि म॒ग्निꣳ स॒धस्था᳚त् पृथि॒व्याः पृ॑थि॒व्याः स॒धस्था॑ द॒ग्निम् । \newline
34. स॒धस्था॑ द॒ग्नि म॒ग्निꣳ स॒धस्था᳚थ् स॒धस्था॑ द॒ग्निम् पु॑री॒ष्य॑म् पुरी॒ष्य॑ म॒ग्निꣳ स॒धस्था᳚थ् स॒धस्था॑ द॒ग्निम् पु॑री॒ष्य᳚म् । \newline
35. स॒धस्था॒दिति॑ स॒ध - स्था॒त् । \newline
36. अ॒ग्निम् पु॑री॒ष्य॑म् पुरी॒ष्य॑ म॒ग्नि म॒ग्निम् पु॑री॒ष्य॑ मङ्गिर॒स्व द॑ङ्गिर॒स्वत् पु॑री॒ष्य॑ म॒ग्नि म॒ग्निम् पु॑री॒ष्य॑ मङ्गिर॒स्वत् । \newline
37. पु॒री॒ष्य॑ मङ्गिर॒स्व द॑ङ्गिर॒स्वत् पु॑री॒ष्य॑म् पुरी॒ष्य॑ मङ्गिर॒स्व दच्छाच्छा᳚ ङ्गिर॒स्वत् पु॑री॒ष्य॑म् पुरी॒ष्य॑ मङ्गिर॒स्व दच्छ॑ । \newline
38. अ॒ङ्गि॒र॒स्व दच्छाच्छा᳚ ङ्गिर॒स्व द॑ङ्गिर॒स्व दच्छे॑ ही॒ह्यच्छा᳚ङ्गिर॒स्व द॑ङ्गिर॒स्व दच्छे॑हि । \newline
39. अच्छे॑ही॒ ह्यच्छाच्छे॒ हीतीती॒ ह्यच्छाच्छे॒ हीति॑ । \newline
40. इ॒हीतीती॑ ही॒ही त्या॑हा॒हेती॑ ही॒हीत्या॑ह । \newline
41. इत्या॑हा॒हेती त्या॑ह॒ साय॑तनꣳ॒॒ साय॑तन मा॒हेती त्या॑ह॒ साय॑तनम् । \newline
42. आ॒ह॒ साय॑तनꣳ॒॒ साय॑तन माहाह॒ साय॑तन मे॒वैव साय॑तन माहाह॒ साय॑तन मे॒व । \newline
43. साय॑तन मे॒वैव साय॑तनꣳ॒॒ साय॑तन मे॒वैन॑ मेन मे॒व साय॑तनꣳ॒॒ साय॑तन मे॒वैन᳚म् । \newline
44. साय॑तन॒मिति॒ स - आ॒य॒त॒न॒म् । \newline
45. ए॒वैन॑ मेन मे॒वैवैन॑म् दे॒वता॑भिर् दे॒वता॑भि रेन मे॒वैवैन॑म् दे॒वता॑भिः । \newline
46. ए॒न॒म् दे॒वता॑भिर् दे॒वता॑भि रेन मेनम् दे॒वता॑भिः॒ सꣳ सम् दे॒वता॑भि रेन मेनम् दे॒वता॑भिः॒ सम् । \newline
47. दे॒वता॑भिः॒ सꣳ सम् दे॒वता॑भिर् दे॒वता॑भिः॒ सम् भ॑रति भरति॒ सम् दे॒वता॑भिर् दे॒वता॑भिः॒ सम् भ॑रति । \newline
48. सम् भ॑रति भरति॒ सꣳ सम् भ॑र त्य॒ग्नि म॒ग्निम् भ॑रति॒ सꣳ सम् भ॑र त्य॒ग्निम् । \newline
49. भ॒र॒ त्य॒ग्नि म॒ग्निम् भ॑रति भर त्य॒ग्निम् पु॑री॒ष्य॑म् पुरी॒ष्य॑ म॒ग्निम् भ॑रति भर त्य॒ग्निम् पु॑री॒ष्य᳚म् । \newline
50. अ॒ग्निम् पु॑री॒ष्य॑म् पुरी॒ष्य॑ म॒ग्नि म॒ग्निम् पु॑री॒ष्य॑ मङ्गिर॒स्व द॑ङ्गिर॒स्वत् पु॑री॒ष्य॑ म॒ग्नि म॒ग्निम् पु॑री॒ष्य॑ मङ्गिर॒स्वत् । \newline
51. पु॒री॒ष्य॑ मङ्गिर॒स्व द॑ङ्गिर॒स्वत् पु॑री॒ष्य॑म् पुरी॒ष्य॑ मङ्गिर॒स्व दच्छाच्छा᳚ ङ्गिर॒स्वत् पु॑री॒ष्य॑म् पुरी॒ष्य॑ मङ्गिर॒स्व दच्छ॑ । \newline
52. अ॒ङ्गि॒र॒स्व दच्छाच्छा᳚ ङ्गिर॒स्व द॑ङ्गिर॒स्व दच्छे॑म इमो॒ अच्छा᳚ङ्गिर॒स्व द॑ङ्गिर॒स्व दच्छे॑मः । \newline
53. अच्छे॑म इमो॒ अच्छाच्छे॑म॒ इतीती॑मो॒ अच्छाच्छे॑म॒ इति॑ । \newline
54. इ॒म॒ इतीती॑म इम॒ इत्या॑हा॒हे ती॑म इम॒ इत्या॑ह । \newline
55. इत्या॑ हा॒हेती त्या॑ह॒ येन॒ येना॒हेती त्या॑ह॒ येन॑ । \newline
56. आ॒ह॒ येन॒ येना॑हाह॒ येन॑ स॒ङ्गच्छ॑ते स॒ङ्गच्छ॑ते॒ येना॑हाह॒ येन॑ स॒ङ्गच्छ॑ते । \newline
57. येन॑ स॒ङ्गच्छ॑ते स॒ङ्गच्छ॑ते॒ येन॒ येन॑ स॒ङ्गच्छ॑ते॒ वाजं॒ ॅवाजꣳ॑ स॒ङ्गच्छ॑ते॒ येन॒ येन॑ स॒ङ्गच्छ॑ते॒ वाज᳚म् । \newline
\pagebreak
\markright{ TS 5.1.2.5  \hfill https://www.vedavms.in \hfill}

\section{ TS 5.1.2.5 }

\textbf{TS 5.1.2.5 } \newline
\textbf{Samhita Paata} \newline

स॒ङ्गच्छ॑ते॒ वाज॑मे॒वास्य॑ वृङ्क्ते प्र॒जाप॑तये प्रति॒प्रोच्या॒ग्निः स॒म्भृत्य॒ इत्या॑हुरि॒यं ॅवै प्र॒जाप॑ति॒स्तस्या॑ ए॒तच्छ्रोत्रं॒ ॅयद्व॒ल्मीको॒ऽग्निं पु॑री॒ष्य॑-मङ्गिर॒स्वद्-भ॑रिष्याम॒ इति॑ वल्मीकव॒पामुप॑ तिष्ठते सा॒क्षादे॒व प्र॒जाप॑तये प्रति॒प्रोच्या॒ऽग्निꣳ सं भ॑रत्य॒ग्निं पु॑री॒ष्य॑-मङ्गिर॒स्वद्-भ॑राम॒ इत्या॑ह॒ येन॑ स॒गंच्छ॑ते॒ वाज॑मे॒वास्य॑ वृ॒ङ्क्ते ऽन्व॒ग्निरु॒षसा॒मग्र॑ - [  ] \newline

\textbf{Pada Paata} \newline

स॒गंच्छ॑त॒ इति॑ सं - गच्छ॑ते । वाज᳚म् । ए॒व । अ॒स्य॒ । वृ॒ङ्क्ते॒ । प्र॒जाप॑तय॒ इति॑ प्र॒जा - प॒त॒ये॒ । प्र॒ति॒प्रोच्येति॑ प्रति - प्रोच्य॑ । अ॒ग्निः । स॒भृंत्य॒ इति॑ सं - भृत्यः॑ । इति॑ । आ॒हुः॒ । इ॒यम् । वै । प्र॒जाप॑ति॒रिति॑ प्र॒जा-प॒तिः॒ । तस्याः᳚ । ए॒तत् । श्रोत्र᳚म् । यत् । व॒ल्मीकः॑ । अ॒ग्निम् । पु॒री॒ष्य᳚म् । अ॒ङ्गि॒र॒स्वत् । भ॒रि॒ष्या॒मः॒ । इति॑ । व॒ल्मी॒क॒व॒पामिति॑ वल्मीक - व॒पाम् । उपेति॑ । ति॒ष्ठ॒ते॒ । सा॒क्षादिति॑ स-अ॒क्षात् । ए॒व । प्र॒जाप॑तय॒ इति॑ प्र॒जा - प॒त॒ये॒ । प्र॒ति॒प्रोच्येति॑ प्रति - प्रोच्य॑ । अ॒ग्निम् । समिति॑ । भ॒र॒ति॒ । अ॒ग्निम् । पु॒री॒ष्य᳚म् । अ॒ङ्गि॒र॒स्वत् । भ॒रा॒मः॒ । इति॑ । आ॒ह॒ । येन॑ । स॒गंच्छ॑त॒ इति॑ सं-गच्छ॑ते । वाज᳚म् । ए॒व । अ॒स्य॒ । वृ॒ङ्क्ते॒ । अन्विति॑ । अ॒ग्निः । उ॒षसा᳚म् । अग्र᳚म् ।  \newline


\textbf{Krama Paata} \newline

स॒ङ्गच्छ॑ते॒ वाज᳚म् । स॒ङ्गच्छ॑त॒ इति॑ सम् - गच्छ॑ते । वाज॑मे॒व । ए॒वास्य॑ । अ॒स्य॒ वृ॒ङ्क्ते॒ । वृ॒ङ्क्ते॒ प्र॒जाप॑तये । प्र॒जाप॑तये प्रति॒प्रोच्य॑ । प्र॒जाप॑तय॒ इति॑ प्र॒जा - प॒त॒ये॒ । प्र॒ति॒प्रोच्या॒ग्निः । प्र॒ति॒प्रोच्येति॑ प्रति - प्रोच्य॑ । अ॒ग्निः स॒म्भृत्यः॑ । स॒म्भृत्य॒ इति॑ । स॒म्भृत्य॒ इति॑ सम् - भृत्यः॑ । इत्या॑हुः । आ॒हुरि॒यम् । इ॒यम् ॅवै । वै प्र॒जाप॑तिः । प्र॒जाप॑ति॒स्तस्याः᳚ । प्र॒जाप॑ति॒रिति॑ प्र॒जा - प॒तिः॒ । तस्या॑ ए॒तत् । ए॒तच्छ्रोत्र᳚म् । श्रोत्र॒म् ॅयत् । यद् व॒ल्मीकः॑ । व॒ल्मीको॒ऽग्निम् । अ॒ग्निम् पु॑री॒ष्य᳚म् । पु॒री॒ष्य॑मङ्गिर॒स्वत् । अ॒ङ्गि॒र॒स्वद् भ॑रिष्यामः । भ॒रि॒ष्या॒म॒ इति॑ । इति॑ वल्मीकव॒पाम् । व॒ल्मी॒क॒व॒पामुप॑ । व॒ल्मी॒क॒व॒पामिति॑ वल्मीक - व॒पाम् । उप॑ तिष्ठते । ति॒ष्ठ॒ते॒ सा॒क्षात् । सा॒क्षादे॒व । सा॒क्षादिति॑ स - अ॒क्षात् । ए॒व प्र॒जाप॑तये । प्र॒जाप॑तये प्रति॒प्रोच्य॑ । प्र॒जाप॑तय॒ इति॑ प्र॒जा - प॒त॒ये॒ । प्र॒ति॒प्रोच्या॒ग्निम् । प्र॒ति॒प्रोच्येति॑ प्रति - प्रोच्य॑ । अ॒ग्निꣳ सम् । सम् भ॑रति । भ॒र॒त्य॒ग्निम् । अ॒ग्निम् पु॑री॒ष्य᳚म् । पु॒री॒ष्य॑मङ्गिर॒स्वत् । अ॒ङ्गि॒र॒स्वद् भ॑रामः । भ॒रा॒म॒ इति॑ । इत्या॑ह । आ॒ह॒ येन॑ । येन॑ स॒ङ्गच्छ॑ते । स॒ङ्गच्छ॑ते॒ वाज᳚म् । स॒ङ्गच्छ॑त॒ इति॑ सम् - गच्छ॑ते । वाज॑मे॒व । ए॒वास्य॑ । अ॒स्य॒ वृ॒ङ्क्ते॒ । वृ॒ङ्क्तेऽनु॑ । अन्व॒ग्निः । अ॒ग्निरु॒षसा᳚म् । उ॒षसा॒मग्र᳚म् । अग्र॑मख्यत् \newline

\textbf{Jatai Paata} \newline

1. स॒ङ्गच्छ॑ते॒ वाजं॒ ॅवाजꣳ॑ स॒ङ्गच्छ॑ते स॒ङ्गच्छ॑ते॒ वाज᳚म् । \newline
2. स॒ङ्गच्छ॑त॒ इति॑ सं - गच्छ॑ते । \newline
3. वाज॑ मे॒वैव वाजं॒ ॅवाज॑ मे॒व । \newline
4. ए॒वास्या᳚ स्यै॒वैवास्य॑ । \newline
5. अ॒स्य॒ वृ॒ङ्क्ते॒ वृ॒ङ्क्ते॒ ऽस्या॒स्य॒ वृ॒ङ्क्ते॒ । \newline
6. वृ॒ङ्क्ते॒ प्र॒जाप॑तये प्र॒जाप॑तये वृङ्क्ते वृङ्क्ते प्र॒जाप॑तये । \newline
7. प्र॒जाप॑तये प्रति॒प्रोच्य॑ प्रति॒प्रोच्य॑ प्र॒जाप॑तये प्र॒जाप॑तये प्रति॒प्रोच्य॑ । \newline
8. प्र॒जाप॑तय॒ इति॑ प्र॒जा - प॒त॒ये॒ । \newline
9. प्र॒ति॒प्रोच्या॒ग्नि र॒ग्निः प्र॑ति॒प्रोच्य॑ प्रति॒प्रोच्या॒ग्निः । \newline
10. प्र॒ति॒प्रोच्येति॑ प्रति - प्रोच्य॑ । \newline
11. अ॒ग्निः सं॒भृत्यः॑ सं॒भृत्यो॒ ऽग्नि र॒ग्निः सं॒भृत्यः॑ । \newline
12. सं॒भृत्य॒ इतीति॑ सं॒भृत्यः॑ सं॒भृत्य॒ इति॑ । \newline
13. सं॒भृत्य॒ इति॑ सं - भृत्यः॑ । \newline
14. इत्या॑हु राहु॒ रिती त्या॑हुः । \newline
15. आ॒हु॒ रि॒य मि॒य मा॑हु राहु रि॒यम् । \newline
16. इ॒यं ॅवै वा इ॒य मि॒यं ॅवै । \newline
17. वै प्र॒जाप॑तिः प्र॒जाप॑ति॒र् वै वै प्र॒जाप॑तिः । \newline
18. प्र॒जाप॑ति॒ स्तस्या॒ स्तस्याः᳚ प्र॒जाप॑तिः प्र॒जाप॑ति॒ स्तस्याः᳚ । \newline
19. प्र॒जाप॑ति॒रिति॑ प्र॒जा - प॒तिः॒ । \newline
20. तस्या॑ ए॒त दे॒तत् तस्या॒ स्तस्या॑ ए॒तत् । \newline
21. ए॒त च्छ्रोत्रꣳ॒॒ श्रोत्र॑ मे॒त दे॒त च्छ्रोत्र᳚म् । \newline
22. श्रोत्रं॒ ॅयद् यच्छ्रोत्रꣳ॒॒ श्रोत्रं॒ ॅयत् । \newline
23. यद् व॒ल्मीको॑ व॒ल्मीको॒ यद् यद् व॒ल्मीकः॑ । \newline
24. व॒ल्मीको॒ ऽग्नि म॒ग्निं ॅव॒ल्मीको॑ व॒ल्मीको॒ ऽग्निम् । \newline
25. अ॒ग्निम् पु॑री॒ष्य॑म् पुरी॒ष्य॑ म॒ग्नि म॒ग्निम् पु॑री॒ष्य᳚म् । \newline
26. पु॒री॒ष्य॑ मङ्गिर॒स्व द॑ङ्गिर॒स्वत् पु॑री॒ष्य॑म् पुरी॒ष्य॑ मङ्गिर॒स्वत् । \newline
27. अ॒ङ्गि॒र॒स्वद् भ॑रिष्यामो भरिष्यामो ऽङ्गिर॒स्व द॑ङ्गिर॒स्वद् भ॑रिष्यामः । \newline
28. भ॒रि॒ष्या॒म॒ इतीति॑ भरिष्यामो भरिष्याम॒ इति॑ । \newline
29. इति॑ वल्मीकव॒पां ॅव॑ल्मीकव॒पा मितीति॑ वल्मीकव॒पाम् । \newline
30. व॒ल्मी॒क॒व॒पा मुपोप॑ वल्मीकव॒पां ॅव॑ल्मीकव॒पा मुप॑ । \newline
31. व॒ल्मी॒क॒व॒पामिति॑ वल्मीक - व॒पाम् । \newline
32. उप॑ तिष्ठते तिष्ठत॒ उपोप॑ तिष्ठते । \newline
33. ति॒ष्ठ॒ते॒ सा॒क्षाथ् सा॒क्षात् ति॑ष्ठते तिष्ठते सा॒क्षात् । \newline
34. सा॒क्षा दे॒वैव सा॒क्षाथ् सा॒क्षा दे॒व । \newline
35. सा॒क्षादिति॑ स - अ॒क्षात् । \newline
36. ए॒व प्र॒जाप॑तये प्र॒जाप॑तय ए॒वैव प्र॒जाप॑तये । \newline
37. प्र॒जाप॑तये प्रति॒प्रोच्य॑ प्रति॒प्रोच्य॑ प्र॒जाप॑तये प्र॒जाप॑तये प्रति॒प्रोच्य॑ । \newline
38. प्र॒जाप॑तय॒ इति॑ प्र॒जा - प॒त॒ये॒ । \newline
39. प्र॒ति॒प्रोच्या॒ग्नि म॒ग्निम् प्र॑ति॒प्रोच्य॑ प्रति॒प्रोच्या॒ग्निम् । \newline
40. प्र॒ति॒प्रोच्येति॑ प्रति - प्रोच्य॑ । \newline
41. अ॒ग्निꣳ सꣳ स म॒ग्नि म॒ग्निꣳ सम् । \newline
42. सम् भ॑रति भरति॒ सꣳ सम् भ॑रति । \newline
43. भ॒र॒ त्य॒ग्नि म॒ग्निम् भ॑रति भर त्य॒ग्निम् । \newline
44. अ॒ग्निम् पु॑री॒ष्य॑म् पुरी॒ष्य॑ म॒ग्नि म॒ग्निम् पु॑री॒ष्य᳚म् । \newline
45. पु॒री॒ष्य॑ मङ्गिर॒स्व द॑ङ्गिर॒स्वत् पु॑री॒ष्य॑म् पुरी॒ष्य॑ मङ्गिर॒स्वत् । \newline
46. अ॒ङ्गि॒र॒स्वद् भ॑रामो भरामो ऽङ्गिर॒स्व द॑ङ्गिर॒स्वद् भ॑रामः । \newline
47. भ॒रा॒म॒ इतीति॑ भरामो भराम॒ इति॑ । \newline
48. इत्या॑हा॒हे तीत्या॑ह । \newline
49. आ॒ह॒ येन॒ येना॑हाह॒ येन॑ । \newline
50. येन॑ स॒ङ्गच्छ॑ते स॒ङ्गच्छ॑ते॒ येन॒ येन॑ स॒ङ्गच्छ॑ते । \newline
51. स॒ङ्गच्छ॑ते॒ वाजं॒ ॅवाजꣳ॑ स॒ङ्गच्छ॑ते स॒ङ्गच्छ॑ते॒ वाज᳚म् । \newline
52. स॒ङ्गच्छ॑त॒ इति॑ सं - गच्छ॑ते । \newline
53. वाज॑ मे॒वैव वाजं॒ ॅवाज॑ मे॒व । \newline
54. ए॒वास्या᳚ स्यै॒वै वास्य॑ । \newline
55. अ॒स्य॒ वृ॒ङ्क्ते॒ वृ॒ङ्क्ते॒ ऽस्या॒स्य॒ वृ॒ङ्क्ते॒ । \newline
56. वृ॒ङ्क्ते ऽन्वनु॑ वृङ्क्ते वृ॒ङ्क्ते ऽनु॑ । \newline
57. अन्व॒ग्नि र॒ग्नि रन्वन् व॒ग्निः । \newline
58. अ॒ग्नि रु॒षसा॑ मु॒षसा॑ म॒ग्नि र॒ग्नि रु॒षसा᳚म् । \newline
59. उ॒षसा॒ मग्र॒ मग्र॑ मु॒षसा॑ मु॒षसा॒ मग्र᳚म् । \newline
60. अग्र॑ मख्य दख्य॒ दग्र॒ मग्र॑ मख्यत् । \newline

\textbf{Ghana Paata } \newline

1. स॒ङ्गच्छ॑ते॒ वाजं॒ ॅवाजꣳ॑ स॒ङ्गच्छ॑ते स॒ङ्गच्छ॑ते॒ वाज॑ मे॒वैव वाजꣳ॑ स॒ङ्गच्छ॑ते स॒ङ्गच्छ॑ते॒ वाज॑ मे॒व । \newline
2. स॒ङ्गच्छ॑त॒ इति॑ सं - गच्छ॑ते । \newline
3. वाज॑ मे॒वैव वाजं॒ ॅवाज॑ मे॒वास्या᳚ स्यै॒व वाजं॒ ॅवाज॑ मे॒वास्य॑ । \newline
4. ए॒वास्या᳚ स्यै॒वै वास्य॑ वृङ्क्ते वृङ्क्ते ऽस्यै॒वै वास्य॑ वृङ्क्ते । \newline
5. अ॒स्य॒ वृ॒ङ्क्ते॒ वृ॒ङ्क्ते॒ ऽस्या॒स्य॒ वृ॒ङ्क्ते॒ प्र॒जाप॑तये प्र॒जाप॑तये वृङ्क्ते ऽस्यास्य वृङ्क्ते प्र॒जाप॑तये । \newline
6. वृ॒ङ्क्ते॒ प्र॒जाप॑तये प्र॒जाप॑तये वृङ्क्ते वृङ्क्ते प्र॒जाप॑तये प्रति॒प्रोच्य॑ प्रति॒प्रोच्य॑ प्र॒जाप॑तये वृङ्क्ते वृङ्क्ते प्र॒जाप॑तये प्रति॒प्रोच्य॑ । \newline
7. प्र॒जाप॑तये प्रति॒प्रोच्य॑ प्रति॒प्रोच्य॑ प्र॒जाप॑तये प्र॒जाप॑तये प्रति॒प्रोच्या॒ ग्नि र॒ग्निः प्र॑ति॒प्रोच्य॑ प्र॒जाप॑तये प्र॒जाप॑तये प्रति॒प्रोच्या॒ग्निः । \newline
8. प्र॒जाप॑तय॒ इति॑ प्र॒जा - प॒त॒ये॒ । \newline
9. प्र॒ति॒प्रोच्या॒ ग्नि र॒ग्निः प्र॑ति॒प्रोच्य॑ प्रति॒प्रोच्या॒ग्निः सं॒भृत्यः॑ सं॒भृत्यो॒ ऽग्निः प्र॑ति॒प्रोच्य॑ प्रति॒प्रोच्या॒ग्निः सं॒भृत्यः॑ । \newline
10. प्र॒ति॒प्रोच्येति॑ प्रति - प्रोच्य॑ । \newline
11. अ॒ग्निः सं॒भृत्यः॑ सं॒भृत्यो॒ ऽग्नि र॒ग्निः सं॒भृत्य॒ इतीति॑ सं॒भृत्यो॒ ऽग्नि र॒ग्निः सं॒भृत्य॒ इति॑ । \newline
12. सं॒भृत्य॒ इतीति॑ सं॒भृत्यः॑ सं॒भृत्य॒ इत्या॑हु राहु॒ रिति॑ सं॒भृत्यः॑ सं॒भृत्य॒ इत्या॑हुः । \newline
13. सं॒भृत्य॒ इति॑ सं - भृत्यः॑ । \newline
14. इत्या॑हु राहु॒ रिती त्या॑हु रि॒य मि॒य मा॑हु॒ रिती त्या॑हु रि॒यम् । \newline
15. आ॒हु॒रि॒य मि॒य मा॑हु राहु रि॒यं ॅवै वा इ॒य मा॑हु राहु रि॒यं ॅवै । \newline
16. इ॒यं ॅवै वा इ॒य मि॒यं ॅवै प्र॒जाप॑तिः प्र॒जाप॑ति॒र् वा इ॒य मि॒यं ॅवै प्र॒जाप॑तिः । \newline
17. वै प्र॒जाप॑तिः प्र॒जाप॑ति॒र् वै वै प्र॒जाप॑ति॒ स्तस्या॒ स्तस्याः᳚ प्र॒जाप॑ति॒र् वै वै प्र॒जाप॑ति॒ स्तस्याः᳚ । \newline
18. प्र॒जाप॑ति॒ स्तस्या॒ स्तस्याः᳚ प्र॒जाप॑तिः प्र॒जाप॑ति॒ स्तस्या॑ ए॒त दे॒तत् तस्याः᳚ प्र॒जाप॑तिः प्र॒जाप॑ति॒ स्तस्या॑ ए॒तत् । \newline
19. प्र॒जाप॑ति॒रिति॑ प्र॒जा - प॒तिः॒ । \newline
20. तस्या॑ ए॒त दे॒तत् तस्या॒ स्तस्या॑ ए॒त च्छ्रोत्रꣳ॒॒ श्रोत्र॑ मे॒तत् तस्या॒ स्तस्या॑ ए॒त च्छ्रोत्र᳚म् । \newline
21. ए॒त च्छ्रोत्रꣳ॒॒ श्रोत्र॑ मे॒त दे॒त च्छ्रोत्रं॒ ॅयद् यच्छ्रोत्र॑ मे॒त दे॒त च्छ्रोत्रं॒ ॅयत् । \newline
22. श्रोत्रं॒ ॅयद् यच्छ्रोत्रꣳ॒॒ श्रोत्रं॒ ॅयद् व॒ल्मीको॑ व॒ल्मीको॒ यच्छ्रोत्रꣳ॒॒ श्रोत्रं॒ ॅयद् व॒ल्मीकः॑ । \newline
23. यद् व॒ल्मीको॑ व॒ल्मीको॒ यद् यद् व॒ल्मीको॒ ऽग्नि म॒ग्निं ॅव॒ल्मीको॒ यद् यद् व॒ल्मीको॒ ऽग्निम् । \newline
24. व॒ल्मीको॒ ऽग्नि म॒ग्निं ॅव॒ल्मीको॑ व॒ल्मीको॒ ऽग्निम् पु॑री॒ष्य॑म् पुरी॒ष्य॑ म॒ग्निं ॅव॒ल्मीको॑ व॒ल्मीको॒ ऽग्निम् पु॑री॒ष्य᳚म् । \newline
25. अ॒ग्निम् पु॑री॒ष्य॑म् पुरी॒ष्य॑ म॒ग्नि म॒ग्निम् पु॑री॒ष्य॑ मङ्गिर॒स्व द॑ङ्गिर॒स्वत् पु॑री॒ष्य॑ म॒ग्नि म॒ग्निम् पु॑री॒ष्य॑ मङ्गिर॒स्वत् । \newline
26. पु॒री॒ष्य॑ मङ्गिर॒स्व द॑ङ्गिर॒स्वत् पु॑री॒ष्य॑म् पुरी॒ष्य॑ मङ्गिर॒स्वद् भ॑रिष्यामो भरिष्यामो ऽङ्गिर॒स्वत् पु॑री॒ष्य॑म् पुरी॒ष्य॑ मङ्गिर॒स्वद् भ॑रिष्यामः । \newline
27. अ॒ङ्गि॒र॒स्वद् भ॑रिष्यामो भरिष्यामो ऽङ्गिर॒स्व द॑ङ्गिर॒स्वद् भ॑रिष्याम॒ इतीति॑ भरिष्यामो ऽङ्गिर॒स्व द॑ङ्गिर॒स्वद् भ॑रिष्याम॒ इति॑ । \newline
28. भ॒रि॒ष्या॒म॒ इतीति॑ भरिष्यामो भरिष्याम॒ इति॑ वल्मीकव॒पां ॅव॑ल्मीकव॒पा मिति॑ भरिष्यामो भरिष्याम॒ इति॑ वल्मीकव॒पाम् । \newline
29. इति॑ वल्मीकव॒पां ॅव॑ल्मीकव॒पा मितीति॑ वल्मीकव॒पा मुपोप॑ वल्मीकव॒पा मितीति॑ वल्मीकव॒पा मुप॑ । \newline
30. व॒ल्मी॒क॒व॒पा मुपोप॑ वल्मीकव॒पां ॅव॑ल्मीकव॒पा मुप॑ तिष्ठते तिष्ठत॒ उप॑ वल्मीकव॒पां ॅव॑ल्मीकव॒पा मुप॑ तिष्ठते । \newline
31. व॒ल्मी॒क॒व॒पामिति॑ वल्मीक - व॒पाम् । \newline
32. उप॑ तिष्ठते तिष्ठत॒ उपोप॑ तिष्ठते सा॒क्षाथ् सा॒क्षात् ति॑ष्ठत॒ उपोप॑ तिष्ठते सा॒क्षात् । \newline
33. ति॒ष्ठ॒ते॒ सा॒क्षाथ् सा॒क्षात् ति॑ष्ठते तिष्ठते सा॒क्षा दे॒वैव सा॒क्षात् ति॑ष्ठते तिष्ठते सा॒क्षा दे॒व । \newline
34. सा॒क्षा दे॒वैव सा॒क्षाथ् सा॒क्षा दे॒व प्र॒जाप॑तये प्र॒जाप॑तय ए॒व सा॒क्षाथ् सा॒क्षा दे॒व प्र॒जाप॑तये । \newline
35. सा॒क्षादिति॑ स - अ॒क्षात् । \newline
36. ए॒व प्र॒जाप॑तये प्र॒जाप॑तय ए॒वैव प्र॒जाप॑तये प्रति॒प्रोच्य॑ प्रति॒प्रोच्य॑ प्र॒जाप॑तय ए॒वैव प्र॒जाप॑तये प्रति॒प्रोच्य॑ । \newline
37. प्र॒जाप॑तये प्रति॒प्रोच्य॑ प्रति॒प्रोच्य॑ प्र॒जाप॑तये प्र॒जाप॑तये प्रति॒प्रोच्या॒ग्नि म॒ग्निम् प्र॑ति॒प्रोच्य॑ प्र॒जाप॑तये प्र॒जाप॑तये प्रति॒प्रोच्या॒ग्निम् । \newline
38. प्र॒जाप॑तय॒ इति॑ प्र॒जा - प॒त॒ये॒ । \newline
39. प्र॒ति॒प्रोच्या॒ग्नि म॒ग्निम् प्र॑ति॒प्रोच्य॑ प्रति॒प्रोच्या॒ग्निꣳ सꣳ स म॒ग्निम् प्र॑ति॒प्रोच्य॑ प्रति॒प्रोच्या॒ग्निꣳ सम् । \newline
40. प्र॒ति॒प्रोच्येति॑ प्रति - प्रोच्य॑ । \newline
41. अ॒ग्निꣳ सꣳ स म॒ग्नि म॒ग्निꣳ सम् भ॑रति भरति॒ स म॒ग्नि म॒ग्निꣳ सम् भ॑रति । \newline
42. सम् भ॑रति भरति॒ सꣳ सम् भ॑रत्य॒ग्नि म॒ग्निम् भ॑रति॒ सꣳ सम् भ॑र त्य॒ग्निम् । \newline
43. भ॒र॒त्य॒ग्नि म॒ग्निम् भ॑रति भर त्य॒ग्निम् पु॑री॒ष्य॑म् पुरी॒ष्य॑ म॒ग्निम् भ॑रति भर त्य॒ग्निम् पु॑री॒ष्य᳚म् । \newline
44. अ॒ग्निम् पु॑री॒ष्य॑म् पुरी॒ष्य॑ म॒ग्नि म॒ग्निम् पु॑री॒ष्य॑ मङ्गिर॒स्व द॑ङ्गिर॒स्वत् पु॑री॒ष्य॑ म॒ग्नि म॒ग्निम् पु॑री॒ष्य॑ मङ्गिर॒स्वत् । \newline
45. पु॒री॒ष्य॑ मङ्गिर॒स्व द॑ङ्गिर॒स्वत् पु॑री॒ष्य॑म् पुरी॒ष्य॑ मङ्गिर॒स्वद् भ॑रामो भरामो ऽङ्गिर॒स्वत् पु॑री॒ष्य॑म् पुरी॒ष्य॑ मङ्गिर॒स्वद् भ॑रामः । \newline
46. अ॒ङ्गि॒र॒स्वद् भ॑रामो भरामो ऽङ्गिर॒स्व द॑ङ्गिर॒स्वद् भ॑राम॒ इतीति॑ भरामो ऽङ्गिर॒स्व द॑ङ्गिर॒स्वद् भ॑राम॒ इति॑ । \newline
47. भ॒रा॒म॒ इतीति॑ भरामो भराम॒ इत्या॑हा॒हेति॑ भरामो भराम॒ इत्या॑ह । \newline
48. इत्या॑हा॒हेती त्या॑ह॒ येन॒ येना॒हेती त्या॑ह॒ येन॑ । \newline
49. आ॒ह॒ येन॒ येना॑हाह॒ येन॑ स॒ङ्गच्छ॑ते स॒ङ्गच्छ॑ते॒ येना॑हाह॒ येन॑ स॒ङ्गच्छ॑ते । \newline
50. येन॑ स॒ङ्गच्छ॑ते स॒ङ्गच्छ॑ते॒ येन॒ येन॑ स॒ङ्गच्छ॑ते॒ वाजं॒ ॅवाजꣳ॑ स॒ङ्गच्छ॑ते॒ येन॒ येन॑ स॒ङ्गच्छ॑ते॒ वाज᳚म् । \newline
51. स॒ङ्गच्छ॑ते॒ वाजं॒ ॅवाजꣳ॑ स॒ङ्गच्छ॑ते स॒ङ्गच्छ॑ते॒ वाज॑ मे॒वैव वाजꣳ॑ स॒ङ्गच्छ॑ते स॒ङ्गच्छ॑ते॒ वाज॑ मे॒व । \newline
52. स॒ङ्गच्छ॑त॒ इति॑ सं - गच्छ॑ते । \newline
53. वाज॑ मे॒वैव वाजं॒ ॅवाज॑ मे॒वास्या᳚ स्यै॒व वाजं॒ ॅवाज॑ मे॒वास्य॑ । \newline
54. ए॒वास्या᳚ स्यै॒वैवास्य॑ वृङ्क्ते वृङ्क्ते ऽस्यै॒वैवास्य॑ वृङ्क्ते । \newline
55. अ॒स्य॒ वृ॒ङ्क्ते॒ वृ॒ङ्क्ते॒ ऽस्या॒स्य॒ वृ॒ङ्क्ते ऽन्वनु॑ वृङ्क्ते ऽस्यास्य वृ॒ङ्क्ते ऽनु॑ । \newline
56. वृ॒ङ्क्ते ऽन्वनु॑ वृङ्क्ते वृ॒ङ्क्ते ऽन्व॒ग्नि र॒ग्नि रनु॑ वृङ्क्ते वृ॒ङ्क्ते ऽन्व॒ग्निः । \newline
57. अन्व॒ग्नि र॒ग्नि रन्वन् व॒ग्नि रु॒षसा॑ मु॒षसा॑ म॒ग्नि रन्वन् व॒ग्नि रु॒षसा᳚म् । \newline
58. अ॒ग्नि रु॒षसा॑ मु॒षसा॑ म॒ग्नि र॒ग्नि रु॒षसा॒ मग्र॒ मग्र॑ मु॒षसा॑ म॒ग्नि र॒ग्नि रु॒षसा॒ मग्र᳚म् । \newline
59. उ॒षसा॒ मग्र॒ मग्र॑ मु॒षसा॑ मु॒षसा॒ मग्र॑ मख्य दख्य॒ दग्र॑ मु॒षसा॑ मु॒षसा॒ मग्र॑ मख्यत् । \newline
60. अग्र॑ मख्य दख्य॒ दग्र॒ मग्र॑ मख्य॒ दिती त्य॑ख्य॒ दग्र॒ मग्र॑ मख्य॒ दिति॑ । \newline
\pagebreak
\markright{ TS 5.1.2.6  \hfill https://www.vedavms.in \hfill}

\section{ TS 5.1.2.6 }

\textbf{TS 5.1.2.6 } \newline
\textbf{Samhita Paata} \newline

मख्य॒दित्या॒हा-नु॑ख्यात्या आ॒गत्य॑ वा॒ज्यद्ध्व॑न आ॒क्रम्य॑ वाजिन् पृथि॒वीमित्या॑हे॒च्छत्ये॒वैनं॒ पूर्व॑या वि॒न्दत्युत्त॑रया॒ द्वाभ्या॒मा क्र॑मयति॒ प्रति॑ष्ठित्या॒ अनु॑रूपाभ्यां॒ तस्मा॒दनु॑रूपाः प॒शवः॒ प्रजा॑यन्ते॒ द्यौस्ते॑ पृ॒ष्ठं पृ॑थि॒वी स॒धस्थ॒मित्या॑है॒भ्यो वा ए॒तं ॅलो॒केभ्यः॑ प्र॒जाप॑तिः॒ समै॑रयद् रू॒पमे॒वास्यै॒-तन्म॑हि॒मानं॒ ॅव्याच॑ष्टे व॒ज्री वा ( ) ए॒ष यदश्वो॑ द॒द्-भिर॒न्यतो॑दद्भ्यो॒ भूयां॒ ॅलोम॑भिरुभ॒याद॑द्भ्यो॒ यं द्वि॒ष्यात् तम॑धस्प॒दं ध्या॑ये॒द्-वज्रे॑णै॒वैनꣳ॑ स्तृणुते ॥ \newline

\textbf{Pada Paata} \newline

अ॒ख्य॒त् । इति॑ । आ॒ह॒ । अनु॑ख्यात्या॒ इत्यनु॑ - ख्या॒त्यै॒ । आ॒गत्येत्या᳚ - गत्य॑ । वा॒जी । अद्ध्व॑नः । आ॒क्रम्येत्या᳚ - क्रम्य॑ । वा॒जि॒न्न् । पृ॒थि॒वीम् । इति॑ । आ॒ह॒ । इ॒च्छति॑ । ए॒व । ए॒न॒म् । पूर्व॑या । वि॒न्दति॑ । उत्त॑र॒येत्युत् - त॒र॒या॒ । द्वाभ्या᳚म् । एति॑ । क्र॒म॒य॒ति॒ । प्रति॑ष्ठित्या॒ इति॒ प्रति॑ - स्थि॒त्यै॒ । अनु॑रूपाभ्या॒मित्यनु॑ - रू॒पा॒भ्या॒म् । तस्मा᳚त् । अनु॑रूपा॒ इत्यनु॑-रू॒पाः॒ । प॒शवः॑ । प्रेति॑ । जा॒य॒न्ते॒ । द्यौः । ते॒ । पृ॒ष्ठम् । पृ॒थि॒वी । स॒धस्थ॒मिति॑ स॒ध - स्थ॒म् । इति॑ । आ॒ह॒ । ए॒भ्यः । वै । ए॒तम् । लो॒केभ्यः॑ । प्र॒जाप॑ति॒रिति॑ प्र॒जा - प॒तिः॒ । समिति॑ । ऐ॒र॒य॒त् । रू॒पम् । ए॒व । अ॒स्य॒ । ए॒तत् । म॒हि॒मान᳚म् । व्याच॑ष्ट॒ इति॑ वि - आच॑ष्टे । व॒ज्री । वै ( ) । ए॒षः । यत् । अश्वः॑ । द॒द्भिरिति॑ दत् - भिः । अ॒न्यतो॑दद्भ्य॒ इत्य॒न्यतो॑दत्-भ्यः॒ । भूयान्॑ । लोम॑भि॒रिति॒ लोम॑-भिः॒ । उ॒भ॒याद॑द्भ्य॒ इत्यु॑भ॒याद॑त् - भ्यः॒ । यम् । द्वि॒ष्यात् । तम् । अ॒ध॒स्प॒दमित्य॑धः - प॒दम् । ध्या॒ये॒त् । वज्रे॑ण । ए॒व । ए॒न॒म् । स्तृ॒णु॒ते॒ ॥  \newline


\textbf{Krama Paata} \newline

अ॒ख्य॒दिति॑ । इत्या॑ह । आ॒हानु॑ख्यात्यै । अनु॑ख्यात्या आ॒गत्य॑ । अनु॑ख्यात्या॒ इत्यनु॑ - ख्या॒त्यै॒ । आ॒गत्य॑ वा॒जी । आ॒गत्येत्या᳚ - गत्य॑ । वा॒ज्यद्ध्व॑नः । अद्ध्व॑न आ॒क्रम्य॑ । आ॒क्रम्य॑ वाजिन्न् । आ॒क्रम्येत्या᳚ - क्रम्य॑ । वा॒जि॒न् पृ॒थि॒वीम् । पृ॒थि॒वीमिति॑ । इत्या॑ह । आ॒हे॒च्छति॑ । इ॒च्छत्ये॒व । ए॒वैन᳚म् । ए॒न॒म् पूर्व॑या । पूर्व॑या वि॒न्दति॑ । वि॒न्दत्युत्त॑रया । उत्त॑रया॒ द्वाभ्या᳚म् । उत्त॑र॒येत्युत् - त॒र॒या॒ । द्वाभ्या॒मा । 
आ क्र॑मयति । क्र॒म॒य॒ति॒ प्रति॑ष्ठित्यै । प्रति॑ष्ठित्या॒ अनु॑रूपाभ्याम् । प्रति॑ष्ठित्या॒ इति॒ प्रति॑ - स्थि॒त्यै॒ । अनु॑रूपाभ्या॒म् तस्मा᳚त् । अनु॑रूपाभ्या॒मित्यनु॑ - रू॒पा॒भ्या॒म् । तस्मा॒दनु॑रूपाः । अनु॑रूपाः प॒शवः॑ । अनु॑रूपा॒ इत्यनु॑ - रू॒पाः॒ । प॒शवः॒ प्र । प्र जा॑यन्ते । जा॒य॒न्ते॒ द्यौः । द्यौस्ते᳚ । ते॒ पृ॒ष्ठम् । पृ॒ष्ठम् पृ॑थि॒वी । पृ॒थि॒वी स॒धस्थ᳚म् । स॒धस्थ॒मिति॑ । स॒धस्थ॒मिति॑ स॒ध - स्थ॒म् । इत्या॑ह । आ॒है॒भ्यः । ए॒भ्यो वै । वा ए॒तम् । ए॒तम् ॅलो॒केभ्यः॑ । लो॒केभ्यः॑ प्र॒जाप॑तिः । प्र॒जाप॑तिः॒ सम् । प्र॒जाप॑ति॒रिति॑ प्र॒जा - प॒तिः॒ । समै॑रयत् । ऐ॒र॒य॒द् रू॒पम् । रू॒पमे॒व । ए॒वास्य॑ । अ॒स्यै॒तत् । ए॒तन् म॑हि॒मान᳚म् । म॒हि॒मान॒म् ॅव्याच॑ष्टे । व्याच॑ष्टे व॒ज्री । व्याच॑ष्ट॒ इति॑ वि - आच॑ष्टे । व॒ज्री वै ( ) । वा ए॒षः । ए॒ष यत् । यदश्वः॑ । अश्वो॑ द॒द्भिः । द॒द्भिर॒न्यतो॑दद्भ्यः । द॒द्भिरिति॑ दत् - भिः । अ॒न्यतो॑दद्भ्यो॒ भूयान्॑ । अ॒न्यतो॑दद्भ्य॒ इत्य॒न्यतो॑दत् - भ्यः॒ । भूया॒न् लोम॑भिः । लोम॑भिरुभ॒याद॑द्भ्यः । लोम॑भि॒रिति॒ लोम॑ - भिः॒ । उ॒भ॒याद॑द्भ्यो॒ यम् । उ॒भ॒याद॑द्भ्य॒ इत्यु॑भ॒याद॑त् - भ्यः॒ । यम् द्वि॒ष्यात् । द्वि॒ष्यात् तम् । तम॑धस्प॒दम् । अ॒ध॒स्प॒दम् ध्या॑येत् । अ॒ध॒स्प॒दमित्य॑धः - प॒दम् । ध्या॒ये॒द् वज्रे॑ण । वज्रे॑णै॒व । ए॒वैन᳚म् । ए॒नꣳ॒॒ स्तृ॒णु॒ते॒ । स्तृ॒णु॒त॒ इति॑ स्तणुते । \newline

\textbf{Jatai Paata} \newline

1. अ॒ख्य॒ दिती त्य॑ख्य दख्य॒ दिति॑ । \newline
2. इत्या॑हा॒हे तीत्या॑ह । \newline
3. आ॒हानु॑ख्यात्या॒ अनु॑ख्यात्या आहा॒हा नु॑ख्यात्यै । \newline
4. अनु॑ख्यात्या आ॒गत्या॒ गत्या नु॑ख्यात्या॒ अनु॑ख्यात्या आ॒गत्य॑ । \newline
5. अनु॑ख्यात्या॒ इत्यनु॑ - ख्या॒त्यै॒ । \newline
6. आ॒गत्य॑ वा॒जी वा॒ज्या॑गत्या॒ गत्य॑ वा॒जी । \newline
7. आ॒गत्येत्या᳚ - गत्य॑ । \newline
8. वा॒ज्यद्ध्व॑नो॒ अद्ध्व॑नो वा॒जी वा॒ज्यद्ध्व॑नः । \newline
9. अद्ध्व॑न आ॒क्रम्या॒ क्रम्या द्ध्व॑नो॒ ऽद्ध्व॑न आ॒क्रम्य॑ । \newline
10. आ॒क्रम्य॑ वाजिन्. वाजिन् ना॒क्रम्या॒ क्रम्य॑ वाजिन्न् । \newline
11. आ॒क्रम्येत्या᳚ - क्रम्य॑ । \newline
12. वा॒जि॒न् पृ॒थि॒वीम् पृ॑थि॒वीं ॅवा॑जिन्. वाजिन् पृथि॒वीम् । \newline
13. पृ॒थि॒वी मितीति॑ पृथि॒वीम् पृ॑थि॒वी मिति॑ । \newline
14. इत्या॑हा॒हे तीत्या॑ह । \newline
15. आ॒हे॒ च्छती॒ च्छ त्या॑हाहे॒ च्छति॑ । \newline
16. इ॒च्छ त्ये॒वैवे च्छती॒ च्छत्ये॒व । \newline
17. ए॒वैन॑ मेन मे॒वैवैन᳚म् । \newline
18. ए॒न॒म् पूर्व॑या॒ पूर्व॑यैन मेन॒म् पूर्व॑या । \newline
19. पूर्व॑या वि॒न्दति॑ वि॒न्दति॒ पूर्व॑या॒ पूर्व॑या वि॒न्दति॑ । \newline
20. वि॒न्द त्युत्त॑र॒ योत्त॑रया वि॒न्दति॑ वि॒न्द त्युत्त॑रया । \newline
21. उत्त॑रया॒ द्वाभ्या॒म् द्वाभ्या॒ मुत्त॑र॒ योत्त॑रया॒ द्वाभ्या᳚म् । \newline
22. उत्त॑र॒येत्युत् - त॒र॒या॒ । \newline
23. द्वाभ्या॒ मा द्वाभ्या॒म् द्वाभ्या॒ मा । \newline
24. आ क्र॑मयति क्रमय॒त्या क्र॑मयति । \newline
25. क्र॒म॒य॒ति॒ प्रति॑ष्ठित्यै॒ प्रति॑ष्ठित्यै क्रमयति क्रमयति॒ प्रति॑ष्ठित्यै । \newline
26. प्रति॑ष्ठित्या॒ अनु॑रूपाभ्या॒ मनु॑रूपाभ्या॒म् प्रति॑ष्ठित्यै॒ प्रति॑ष्ठित्या॒ अनु॑रूपाभ्याम् । \newline
27. प्रति॑ष्ठित्या॒ इति॒ प्रति॑ - स्थि॒त्यै॒ । \newline
28. अनु॑रूपाभ्या॒म् तस्मा॒त् तस्मा॒ दनु॑रूपाभ्या॒ मनु॑रूपाभ्या॒म् तस्मा᳚त् । \newline
29. अनु॑रूपाभ्या॒मित्यनु॑ - रू॒पा॒भ्या॒म् । \newline
30. तस्मा॒ दनु॑रूपा॒ अनु॑रूपा॒ स्तस्मा॒त् तस्मा॒ दनु॑रूपाः । \newline
31. अनु॑रूपाः प॒शवः॑ प॒शवो ऽनु॑रूपा॒ अनु॑रूपाः प॒शवः॑ । \newline
32. अनु॑रूपा॒ इत्यनु॑ - रू॒पाः॒ । \newline
33. प॒शवः॒ प्र प्र प॒शवः॑ प॒शवः॒ प्र । \newline
34. प्र जा॑यन्ते जायन्ते॒ प्र प्र जा॑यन्ते । \newline
35. जा॒य॒न्ते॒ द्यौर् द्यौर् जा॑यन्ते जायन्ते॒ द्यौः । \newline
36. द्यौ स्ते॑ ते॒ द्यौर् द्यौ स्ते᳚ । \newline
37. ते॒ पृ॒ष्ठम् पृ॒ष्ठम् ते॑ ते पृ॒ष्ठम् । \newline
38. पृ॒ष्ठम् पृ॑थि॒वी पृ॑थि॒वी पृ॒ष्ठम् पृ॒ष्ठम् पृ॑थि॒वी । \newline
39. पृ॒थि॒वी स॒धस्थꣳ॑ स॒धस्थ॑म् पृथि॒वी पृ॑थि॒वी स॒धस्थ᳚म् । \newline
40. स॒धस्थ॒ मितीति॑ स॒धस्थꣳ॑ स॒धस्थ॒ मिति॑ । \newline
41. स॒धस्थ॒मिति॑ स॒ध - स्थ॒म् । \newline
42. इत्या॑हा॒हे तीत्या॑ह । \newline
43. आ॒है॒भ्य ए॒भ्य आ॑हा है॒भ्यः । \newline
44. ए॒भ्यो वै वा ए॒भ्य ए॒भ्यो वै । \newline
45. वा ए॒त मे॒तं ॅवै वा ए॒तम् । \newline
46. ए॒तम् ॅलो॒केभ्यो॑ लो॒केभ्य॑ ए॒त मे॒तम् ॅलो॒केभ्यः॑ । \newline
47. लो॒केभ्यः॑ प्र॒जाप॑तिः प्र॒जाप॑तिर् लो॒केभ्यो॑ लो॒केभ्यः॑ प्र॒जाप॑तिः । \newline
48. प्र॒जाप॑तिः॒ सꣳ सम् प्र॒जाप॑तिः प्र॒जाप॑तिः॒ सम् । \newline
49. प्र॒जाप॑ति॒रिति॑ प्र॒जा - प॒तिः॒ । \newline
50. स मै॑रय दैरय॒थ् सꣳ स मै॑रयत् । \newline
51. ऐ॒र॒य॒द् रू॒पꣳ रू॒प मै॑रय दैरयद् रू॒पम् । \newline
52. रू॒प मे॒वैव रू॒पꣳ रू॒प मे॒व । \newline
53. ए॒वास्या ᳚स्यै॒वै वास्य॑ । \newline
54. अ॒स्यै॒त दे॒त द॑स्या स्यै॒तत् । \newline
55. ए॒तन् म॑हि॒मान॑म् महि॒मान॑ मे॒त दे॒तन् म॑हि॒मान᳚म् । \newline
56. म॒हि॒मानं॒ ॅव्याच॑ष्टे॒ व्याच॑ष्टे महि॒मान॑म् महि॒मानं॒ ॅव्याच॑ष्टे । \newline
57. व्याच॑ष्टे व॒ज्री व॒ज्री व्याच॑ष्टे॒ व्याच॑ष्टे व॒ज्री । \newline
58. व्याच॑ष्ट॒ इति॑ वि - आच॑ष्टे । \newline
59. व॒ज्री वै वै व॒ज्री व॒ज्री वै । \newline
60. वा ए॒ष ए॒ष वै वा ए॒षः । \newline
61. ए॒ष यद् यदे॒ष ए॒ष यत् । \newline
62. यदश्वो ऽश्वो॒ यद् यदश्वः॑ । \newline
63. अश्वो॑ द॒द्भिर् द॒द्भि रश्वो ऽश्वो॑ द॒द्भिः । \newline
64. द॒द्भि र॒न्यतो॑दद्भ्यो॒ ऽन्यतो॑दद्भ्यो द॒द्भिर् द॒द्भि र॒न्यतो॑दद्भ्यः । \newline
65. द॒द्भिरिति॑ दत् - भिः । \newline
66. अ॒न्यतो॑दद्भ्यो॒ भूया॒न् भूया॑ न॒न्यतो॑दद्भ्यो॒ ऽन्यतो॑दद्भ्यो॒ भूयान्॑ । \newline
67. अ॒न्यतो॑दद्भ्य॒ इत्य॒न्यतो॑दत् - भ्यः॒ । \newline
68. भूया॒न् ॅलोम॑भि॒र् लोम॑भि॒र् भूया॒न् भूया॒न् ॅलोम॑भिः । \newline
69. लोम॑भि रुभ॒याद॑द्भ्य उभ॒याद॑द्भ्यो॒ लोम॑भि॒र् लोम॑भि रुभ॒याद॑द्भ्यः । \newline
70. लोम॑भि॒रिति॒ लोम॑ - भिः॒ । \newline
71. उ॒भ॒याद॑द्भ्यो॒ यं ॅय मु॑भ॒याद॑द्भ्य उभ॒याद॑द्भ्यो॒ यम् । \newline
72. उ॒भ॒याद॑द्भ्य॒ इत्यु॑भ॒याद॑त् - भ्यः॒ । \newline
73. यम् द्वि॒ष्याद् द्वि॒ष्याद् यं ॅयम् द्वि॒ष्यात् । \newline
74. द्वि॒ष्यात् तम् तम् द्वि॒ष्याद् द्वि॒ष्यात् तम् । \newline
75. त म॑धस्प॒द म॑धस्प॒दम् तम् त म॑धस्प॒दम् । \newline
76. अ॒ध॒स्प॒दम् ध्या॑येद् ध्याये दधस्प॒द म॑धस्प॒दम् ध्या॑येत् । \newline
77. अ॒ध॒स्प॒दमित्य॑धः - प॒दम् । \newline
78. ध्या॒ये॒द् वज्रे॑ण॒ वज्रे॑ण ध्यायेद् ध्याये॒द् वज्रे॑ण । \newline
79. वज्रे॑ णै॒वैव वज्रे॑ण॒ वज्रे॑ णै॒व । \newline
80. ए॒वैन॑ मेन मे॒वैवैन᳚म् । \newline
81. ए॒नꣳ॒॒ स्तृ॒णु॒ते॒ स्तृ॒णु॒त॒ ए॒न॒ मे॒नꣳ॒॒ स्तृ॒णु॒ते॒ । \newline
82. स्तृ॒णु॒त॒ इति॑ स्तणुते । \newline

\textbf{Ghana Paata } \newline

1. अ॒ख्य॒दिती त्य॑ख्य दख्य॒ दित्या॑हा॒हे त्य॑ख्य दख्य॒ दित्या॑ह । \newline
2. इत्या॑हा॒हेती त्या॒हानु॑ख्यात्या॒ अनु॑ख्यात्या आ॒हेती त्या॒हानु॑ख्यात्यै । \newline
3. आ॒हानु॑ख्यात्या॒ अनु॑ख्यात्या आहा॒हानु॑ख्यात्या आ॒गत्या॒ गत्या नु॑ख्यात्या आहा॒हानु॑ख्यात्या आ॒गत्य॑ । \newline
4. अनु॑ख्यात्या आ॒गत्या॒ गत्या नु॑ख्यात्या॒ अनु॑ख्यात्या आ॒गत्य॑ वा॒जी वा॒ज्या॑गत्या नु॑ख्यात्या॒ अनु॑ख्यात्या आ॒गत्य॑ वा॒जी । \newline
5. अनु॑ख्यात्या॒ इत्यनु॑ - ख्या॒त्यै॒ । \newline
6. आ॒गत्य॑ वा॒जी वा॒ज्या॑गत्या॒ गत्य॑ वा॒ज्यद्ध्व॑नो॒ अद्ध्व॑नो वा॒ज्या॑गत्या॒ गत्य॑ वा॒ज्यद्ध्व॑नः । \newline
7. आ॒गत्येत्या᳚ - गत्य॑ । \newline
8. वा॒ज्यद्ध्व॑नो॒ अद्ध्व॑नो वा॒जी वा॒ज्यद्ध्व॑न आ॒क्रम्या॒ क्रम्या द्ध्व॑नो वा॒जी वा॒ज्यद्ध्व॑न आ॒क्रम्य॑ । \newline
9. अद्ध्व॑न आ॒क्रम्या॒ क्रम्या द्ध्व॑नो॒ ऽद्ध्व॑न आ॒क्रम्य॑ वाजिन्. वाजिन् ना॒क्रम्या द्ध्व॑नो॒ ऽद्ध्व॑न आ॒क्रम्य॑ वाजिन्न् । \newline
10. आ॒क्रम्य॑ वाजिन्. वाजिन् ना॒क्रम्या॒ क्रम्य॑ वाजिन् पृथि॒वीम् पृ॑थि॒वीं ॅवा॑जिन् ना॒क्रम्या॒ क्रम्य॑ वाजिन् पृथि॒वीम् । \newline
11. आ॒क्रम्येत्या᳚ - क्रम्य॑ । \newline
12. वा॒जि॒न् पृ॒थि॒वीम् पृ॑थि॒वीं ॅवा॑जिन्. वाजिन् पृथि॒वी मितीति॑ पृथि॒वीं ॅवा॑जिन्. वाजिन् पृथि॒वी मिति॑ । \newline
13. पृ॒थि॒वी मितीति॑ पृथि॒वीम् पृ॑थि॒वी मित्या॑हा॒हेति॑ पृथि॒वीम् पृ॑थि॒वी मित्या॑ह । \newline
14. इत्या॑हा॒हेती त्या॑हे॒च्छती॒च्छ त्या॒हेती त्या॑हे॒च्छति॑ । \newline
15. आ॒हे॒ च्छती॒च्छ त्या॑हाहे॒च्छ त्ये॒वैवेच्छ त्या॑हाहे॒च्छ त्ये॒व । \newline
16. इ॒च्छ त्ये॒वैवे च्छती॒च्छ त्ये॒वैन॑ मेन मे॒वे च्छती॒ च्छ त्ये॒वैन᳚म् । \newline
17. ए॒वैन॑ मेन मे॒वै वैन॒म् पूर्व॑या॒ पूर्व॑यैन मे॒वै वैन॒म् पूर्व॑या । \newline
18. ए॒न॒म् पूर्व॑या॒ पूर्व॑यैन मेन॒म् पूर्व॑या वि॒न्दति॑ वि॒न्दति॒ पूर्व॑यैन मेन॒म् पूर्व॑या वि॒न्दति॑ । \newline
19. पूर्व॑या वि॒न्दति॑ वि॒न्दति॒ पूर्व॑या॒ पूर्व॑या वि॒न्द त्युत्त॑र॒ योत्त॑रया वि॒न्दति॒ पूर्व॑या॒ पूर्व॑या वि॒न्दत्युत्त॑रया । \newline
20. वि॒न्द त्युत्त॑र॒ योत्त॑रया वि॒न्दति॑ वि॒न्द त्युत्त॑रया॒ द्वाभ्या॒म् द्वाभ्या॒ मुत्त॑रया वि॒न्दति॑ वि॒न्द त्युत्त॑रया॒ द्वाभ्या᳚म् । \newline
21. उत्त॑रया॒ द्वाभ्या॒म् द्वाभ्या॒ मुत्त॑र॒ योत्त॑रया॒ द्वाभ्या॒ मा द्वाभ्या॒ मुत्त॑र॒ योत्त॑रया॒ द्वाभ्या॒ मा । \newline
22. उत्त॑र॒येत्युत् - त॒र॒या॒ । \newline
23. द्वाभ्या॒ मा द्वाभ्या॒म् द्वाभ्या॒ मा क्र॑मयति क्रमय॒त्या द्वाभ्या॒म् द्वाभ्या॒ मा क्र॑मयति । \newline
24. आ क्र॑मयति क्रमय॒त्या क्र॑मयति॒ प्रति॑ष्ठित्यै॒ प्रति॑ष्ठित्यै क्रमय॒त्या क्र॑मयति॒ प्रति॑ष्ठित्यै । \newline
25. क्र॒म॒य॒ति॒ प्रति॑ष्ठित्यै॒ प्रति॑ष्ठित्यै क्रमयति क्रमयति॒ प्रति॑ष्ठित्या॒ अनु॑रूपाभ्या॒ मनु॑रूपाभ्या॒म् प्रति॑ष्ठित्यै क्रमयति क्रमयति॒ प्रति॑ष्ठित्या॒ अनु॑रूपाभ्याम् । \newline
26. प्रति॑ष्ठित्या॒ अनु॑रूपाभ्या॒ मनु॑रूपाभ्या॒म् प्रति॑ष्ठित्यै॒ प्रति॑ष्ठित्या॒ अनु॑रूपाभ्या॒म् तस्मा॒त् तस्मा॒ दनु॑रूपाभ्या॒म् प्रति॑ष्ठित्यै॒ प्रति॑ष्ठित्या॒ अनु॑रूपाभ्या॒म् तस्मा᳚त् । \newline
27. प्रति॑ष्ठित्या॒ इति॒ प्रति॑ - स्थि॒त्यै॒ । \newline
28. अनु॑रूपाभ्या॒म् तस्मा॒त् तस्मा॒ दनु॑रूपाभ्या॒ मनु॑रूपाभ्या॒म् तस्मा॒ दनु॑रूपा॒ अनु॑रूपा॒ स्तस्मा॒ दनु॑रूपाभ्या॒ मनु॑रूपाभ्या॒म् तस्मा॒ दनु॑रूपाः । \newline
29. अनु॑रूपाभ्या॒मित्यनु॑ - रू॒पा॒भ्या॒म् । \newline
30. तस्मा॒ दनु॑रूपा॒ अनु॑रूपा॒ स्तस्मा॒त् तस्मा॒ दनु॑रूपाः प॒शवः॑ प॒शवो ऽनु॑रूपा॒ स्तस्मा॒त् तस्मा॒ दनु॑रूपाः प॒शवः॑ । \newline
31. अनु॑रूपाः प॒शवः॑ प॒शवो ऽनु॑रूपा॒ अनु॑रूपाः प॒शवः॒ प्र प्र प॒शवो ऽनु॑रूपा॒ अनु॑रूपाः प॒शवः॒ प्र । \newline
32. अनु॑रूपा॒ इत्यनु॑ - रू॒पाः॒ । \newline
33. प॒शवः॒ प्र प्र प॒शवः॑ प॒शवः॒ प्र जा॑यन्ते जायन्ते॒ प्र प॒शवः॑ प॒शवः॒ प्र जा॑यन्ते । \newline
34. प्र जा॑यन्ते जायन्ते॒ प्र प्र जा॑यन्ते॒ द्यौर् द्यौर् जा॑यन्ते॒ प्र प्र जा॑यन्ते॒ द्यौः । \newline
35. जा॒य॒न्ते॒ द्यौर् द्यौर् जा॑यन्ते जायन्ते॒ द्यौ स्ते॑ ते॒ द्यौर् जा॑यन्ते जायन्ते॒ द्यौ स्ते᳚ । \newline
36. द्यौ स्ते॑ ते॒ द्यौर् द्यौ स्ते॑ पृ॒ष्ठम् पृ॒ष्ठम् ते॒ द्यौर् द्यौ स्ते॑ पृ॒ष्ठम् । \newline
37. ते॒ पृ॒ष्ठम् पृ॒ष्ठम् ते॑ ते पृ॒ष्ठम् पृ॑थि॒वी पृ॑थि॒वी पृ॒ष्ठम् ते॑ ते पृ॒ष्ठम् पृ॑थि॒वी । \newline
38. पृ॒ष्ठम् पृ॑थि॒वी पृ॑थि॒वी पृ॒ष्ठम् पृ॒ष्ठम् पृ॑थि॒वी स॒धस्थꣳ॑ स॒धस्थ॑म् पृथि॒वी पृ॒ष्ठम् पृ॒ष्ठम् पृ॑थि॒वी स॒धस्थ᳚म् । \newline
39. पृ॒थि॒वी स॒धस्थꣳ॑ स॒धस्थ॑म् पृथि॒वी पृ॑थि॒वी स॒धस्थ॒ मितीति॑ स॒धस्थ॑म् पृथि॒वी पृ॑थि॒वी स॒धस्थ॒ मिति॑ । \newline
40. स॒धस्थ॒ मितीति॑ स॒धस्थꣳ॑ स॒धस्थ॒ मित्या॑ हा॒हेति॑ स॒धस्थꣳ॑ स॒धस्थ॒ मित्या॑ह । \newline
41. स॒धस्थ॒मिति॑ स॒ध - स्थ॒म् । \newline
42. इत्या॑हा॒हेती त्या॑है॒भ्य ए॒भ्य आ॒हेती त्या॑है॒भ्यः । \newline
43. आ॒है॒भ्य ए॒भ्य आ॑हाहै॒भ्यो वै वा ए॒भ्य आ॑हाहै॒भ्यो वै । \newline
44. ए॒भ्यो वै वा ए॒भ्य ए॒भ्यो वा ए॒त मे॒तं ॅवा ए॒भ्य ए॒भ्यो वा ए॒तम् । \newline
45. वा ए॒त मे॒तं ॅवै वा ए॒तम् ॅलो॒केभ्यो॑ लो॒केभ्य॑ ए॒तं ॅवै वा ए॒तम् ॅलो॒केभ्यः॑ । \newline
46. ए॒तम् ॅलो॒केभ्यो॑ लो॒केभ्य॑ ए॒त मे॒तम् ॅलो॒केभ्यः॑ प्र॒जाप॑तिः प्र॒जाप॑तिर् लो॒केभ्य॑ ए॒त मे॒तम् ॅलो॒केभ्यः॑ प्र॒जाप॑तिः । \newline
47. लो॒केभ्यः॑ प्र॒जाप॑तिः प्र॒जाप॑तिर् लो॒केभ्यो॑ लो॒केभ्यः॑ प्र॒जाप॑तिः॒ सꣳ सम् प्र॒जाप॑तिर् लो॒केभ्यो॑ लो॒केभ्यः॑ प्र॒जाप॑तिः॒ सम् । \newline
48. प्र॒जाप॑तिः॒ सꣳ सम् प्र॒जाप॑तिः प्र॒जाप॑तिः॒ स मै॑रय दैरय॒थ् सम् प्र॒जाप॑तिः प्र॒जाप॑तिः॒ स मै॑रयत् । \newline
49. प्र॒जाप॑ति॒रिति॑ प्र॒जा - प॒तिः॒ । \newline
50. स मै॑रय दैरय॒थ् सꣳ स मै॑रयद् रू॒पꣳ रू॒प मै॑रय॒थ् सꣳ स मै॑रयद् रू॒पम् । \newline
51. ऐ॒र॒य॒द् रू॒पꣳ रू॒प मै॑रय दैरयद् रू॒प मे॒वैव रू॒प मै॑रय दैरयद् रू॒प मे॒व । \newline
52. रू॒प मे॒वैव रू॒पꣳ रू॒प मे॒वास्या᳚ स्यै॒व रू॒पꣳ रू॒प मे॒वास्य॑ । \newline
53. ए॒वास्या᳚ स्यै॒वैवा स्यै॒त दे॒त द॑स्यै॒वैवा स्यै॒तत् । \newline
54. अ॒स्यै॒त दे॒त द॑स्या स्यै॒तन् म॑हि॒मान॑म् महि॒मान॑ मे॒त द॑स्या स्यै॒तन् म॑हि॒मान᳚म् । \newline
55. ए॒तन् म॑हि॒मान॑म् महि॒मान॑ मे॒त दे॒तन् म॑हि॒मानं॒ ॅव्याच॑ष्टे॒ व्याच॑ष्टे महि॒मान॑ मे॒त दे॒तन् म॑हि॒मानं॒ ॅव्याच॑ष्टे । \newline
56. म॒हि॒मानं॒ ॅव्याच॑ष्टे॒ व्याच॑ष्टे महि॒मान॑म् महि॒मानं॒ ॅव्याच॑ष्टे व॒ज्री व॒ज्री व्याच॑ष्टे महि॒मान॑म् महि॒मानं॒ ॅव्याच॑ष्टे व॒ज्री । \newline
57. व्याच॑ष्टे व॒ज्री व॒ज्री व्याच॑ष्टे॒ व्याच॑ष्टे व॒ज्री वै वै व॒ज्री व्याच॑ष्टे॒ व्याच॑ष्टे व॒ज्री वै । \newline
58. व्याच॑ष्ट॒ इति॑ वि - आच॑ष्टे । \newline
59. व॒ज्री वै वै व॒ज्री व॒ज्री वा ए॒ष ए॒ष वै व॒ज्री व॒ज्री वा ए॒षः । \newline
60. वा ए॒ष ए॒ष वै वा ए॒ष यद् यदे॒ष वै वा ए॒ष यत् । \newline
61. ए॒ष यद् यदे॒ष ए॒ष यदश्वो ऽश्वो॒ यदे॒ष ए॒ष यदश्वः॑ । \newline
62. यदश्वो ऽश्वो॒ यद् यदश्वो॑ द॒द्भिर् द॒द्भि रश्वो॒ यद् यदश्वो॑ द॒द्भिः । \newline
63. अश्वो॑ द॒द्भिर् द॒द्भि रश्वो ऽश्वो॑ द॒द्भि र॒न्यतो॑ दद्भ्यो॒ ऽन्यतो॑ दद्भ्यो द॒द्भि रश्वो ऽश्वो॑ द॒द्भि र॒न्यतो॑ दद्भ्यः । \newline
64. द॒द्भि र॒न्यतो॑ दद्भ्यो॒ ऽन्यतो॑ दद्भ्यो द॒द्भिर् द॒द्भि र॒न्यतो॑ दद्भ्यो॒ भूया॒न् भूया॑ न॒न्यतो॑ दद्भ्यो द॒द्भिर् द॒द्भि र॒न्यतो॑ दद्भ्यो॒ भूयान्॑ । \newline
65. द॒द्भिरिति॑ दत् - भिः । \newline
66. अ॒न्यतो॑दद्भ्यो॒ भूया॒न् भूया॑ न॒न्यतो॑दद्भ्यो॒ ऽन्यतो॑दद्भ्यो॒ भूया॒न् ॅलोम॑भि॒र् लोम॑भि॒र् भूया॑ न॒न्यतो॑दद्भ्यो॒ ऽन्यतो॑दद्भ्यो॒ भूया॒न् ॅलोम॑भिः । \newline
67. अ॒न्यतो॑दद्भ्य॒ इत्य॒न्यतो॑दत् - भ्यः॒ । \newline
68. भूया॒न् ॅलोम॑भि॒र् लोम॑भि॒र् भूया॒न् भूया॒न् ॅलोम॑भि रुभ॒याद॑द्भ्य उभ॒याद॑द्भ्यो॒ लोम॑भि॒र् भूया॒न् भूया॒न् ॅलोम॑भि रुभ॒याद॑द्भ्यः । \newline
69. लोम॑भि रुभ॒याद॑द्भ्य उभ॒याद॑द्भ्यो॒ लोम॑भि॒र् लोम॑भि रुभ॒याद॑द्भ्यो॒ यं ॅय मु॑भ॒याद॑द्भ्यो॒ लोम॑भि॒र् लोम॑भि रुभ॒याद॑द्भ्यो॒ यम् । \newline
70. लोम॑भि॒रिति॒ लोम॑ - भिः॒ । \newline
71. उ॒भ॒याद॑द्भ्यो॒ यं ॅय मु॑भ॒याद॑द्भ्य उभ॒याद॑द्भ्यो॒ यम् द्वि॒ष्याद् द्वि॒ष्याद् य मु॑भ॒याद॑द्भ्य उभ॒याद॑द्भ्यो॒ यम् द्वि॒ष्यात् । \newline
72. उ॒भ॒याद॑द्भ्य॒ इत्यु॑भ॒याद॑त् - भ्यः॒ । \newline
73. यम् द्वि॒ष्याद् द्वि॒ष्याद् यं ॅयम् द्वि॒ष्यात् तम् तम् द्वि॒ष्याद् यं ॅयम् द्वि॒ष्यात् तम् । \newline
74. द्वि॒ष्यात् तम् तम् द्वि॒ष्याद् द्वि॒ष्यात् त म॑धस्प॒द म॑धस्प॒दम् तम् द्वि॒ष्याद् द्वि॒ष्यात् त म॑धस्प॒दम् । \newline
75. त म॑धस्प॒द म॑धस्प॒दम् तम् त म॑धस्प॒दम् ध्या॑येद् ध्याये दधस्प॒दम् तम् त म॑धस्प॒दम् ध्या॑येत् । \newline
76. अ॒ध॒स्प॒दम् ध्या॑येद् ध्याये दधस्प॒द म॑धस्प॒दम् ध्या॑ये॒द् वज्रे॑ण॒ वज्रे॑ण ध्याये दधस्प॒द म॑धस्प॒दम् ध्या॑ये॒द् वज्रे॑ण । \newline
77. अ॒ध॒स्प॒दमित्य॑धः - प॒दम् । \newline
78. ध्या॒ये॒द् वज्रे॑ण॒ वज्रे॑ण ध्यायेद् ध्याये॒द् वज्रे॑णै॒वैव वज्रे॑ण ध्यायेद् ध्याये॒द् वज्रे॑णै॒व । \newline
79. वज्रे॑णै॒वैव वज्रे॑ण॒ वज्रे॑णै॒वैन॑ मेन मे॒व वज्रे॑ण॒ वज्रे॑णै॒वैन᳚म् । \newline
80. ए॒वैन॑ मेन मे॒वैवैनꣳ॑ स्तृणुते स्तृणुत एन मे॒वैवैनꣳ॑ स्तृणुते । \newline
81. ए॒नꣳ॒॒ स्तृ॒णु॒ते॒ स्तृ॒णु॒त॒ ए॒न॒ मे॒नꣳ॒॒ स्तृ॒णु॒ते॒ । \newline
82. स्तृ॒णु॒त॒ इति॑ स्तणुते । \newline
\pagebreak
\markright{ TS 5.1.3.1  \hfill https://www.vedavms.in \hfill}

\section{ TS 5.1.3.1 }

\textbf{TS 5.1.3.1 } \newline
\textbf{Samhita Paata} \newline

उत्क्रा॒मो-द॑क्रमी॒दिति॒ द्वाभ्या॒मुत्क्र॑मयति॒ प्रति॑ष्ठित्या॒ अनु॑रूपाभ्यां॒ तस्मा॒दनु॑रूपाः प॒शवः॒ प्रजा॑यन्ते॒ ऽप उप॑ सृजति॒ यत्र॒ वा आप॑ उप॒ गच्छ॑न्ति॒ तदोष॑धयः॒ प्रति॑ तिष्ठ॒न्त्योष॑धीः प्रति॒तिष्ठ॑न्तीः प॒शवोऽनु॒ प्रति॑ तिष्ठन्ति प॒शून्. य॒ज्ञो य॒ज्ञ्ं ॅयज॑मानो॒ यज॑मानं प्र॒जास्तस्मा॑द॒प उप॑ सृजति॒ प्रति॑ष्ठित्यै॒ यद॑द्ध्व॒र्यु-र॑न॒ग्नावाहु॑तिं जुहु॒याद॒न्धो᳚ ऽद्ध्व॒र्युः - [  ] \newline

\textbf{Pada Paata} \newline

उदिति॑ । क्रा॒म॒ । उदिति॑ । अ॒क्र॒मी॒त् । इति॑ । द्वाभ्या᳚म् । उदिति॑ । क्र॒म॒य॒ति॒ । प्रति॑ष्ठित्या॒ इति॒ प्रति॑ - स्थि॒त्यै॒ । अनु॑रूपाभ्या॒मित्यनु॑-रू॒पा॒भ्या॒म् । तस्मा᳚त् । अनु॑रूपा॒ इत्यनु॑-रू॒पाः॒ । प॒शवः॑ । प्रेति॑ । जा॒य॒न्ते॒ । अ॒पः । उपेति॑ । सृ॒ज॒ति॒ । यत्र॑ । वै । आपः॑ । उ॒प॒गच्छ॒न्तीत्यु॑प - गच्छ॑न्ति । तत् । ओष॑धयः । प्रतीति॑ । ति॒ष्ठ॒न्ति॒ । ओष॑धीः । प्र॒ति॒तिष्ठ॑न्ती॒रिति॑ प्रति - तिष्ठ॑न्तीः । प॒शवः॑ । अनु॑ । प्रतीति॑ । ति॒ष्ठ॒न्ति॒ । प॒शून् । य॒ज्ञ्ः । य॒ज्ञ्म् । यज॑मानः । यज॑मानम् । प्र॒जा इति॑ प्र-जाः । तस्मा᳚त् । अ॒पः । उपेति॑ । सृ॒ज॒ति॒ । प्रति॑ष्ठित्या॒ इति॒ प्रति॑ - स्थि॒त्यै॒ । यत् । अ॒द्ध्व॒र्युः । अ॒न॒ग्नौ । आहु॑ति॒मित्या - हु॒ति॒म् । जु॒हु॒यात् । अ॒न्धः । अ॒द्ध्व॒र्युः ।  \newline


\textbf{Krama Paata} \newline

उत् क्रा॑म । क्रा॒मोत् । उद॑क्रमीत् । अ॒क्र॒मी॒दिति॑ । इति॒ द्वाभ्या᳚म् । द्वाभ्या॒मुत् । उत् क्र॑मयति । क्र॒म॒य॒ति॒ प्रति॑ष्ठित्यै । प्रति॑ष्ठित्या॒ अनु॑रूपाभ्याम् । प्रति॑ष्ठित्या॒ इति॒ प्रति॑ - स्थि॒त्यै॒ । अनु॑रूपाभ्या॒म् तस्मा᳚त् । अनु॑रूपाभ्या॒मित्यनु॑ - रू॒पा॒भ्या॒म् । तस्मा॒दनु॑रूपाः । अनु॑रूपाः प॒शवः॑ । अनु॑रूपा॒ इत्यनु॑ - रू॒पाः॒ । प॒शवः॒ प्र । प्र जा॑यन्ते । जा॒य॒न्ते॒ऽपः । अ॒प उप॑ । उप॑ सृजति । सृ॒ज॒ति॒ यत्र॑ । यत्र॒ वै । वा आपः॑ । आप॑ उप॒गच्छ॑न्ति । उ॒प॒गच्छ॑न्ति॒ तत् । उ॒प॒गच्छ॒न्तीत्यु॑प - गच्छ॑न्ति । तदोष॑धयः । ओष॑धयः॒ प्रति॑ । प्रति॑ तिष्ठन्ति । ति॒ष्ठ॒न्त्योष॑धीः । ओष॑धीः प्रति॒तिष्ठ॑न्तीः । प्र॒ति॒तिष्ठ॑न्तीः प॒शवः॑ । प्र॒ति॒तिष्ठ॑न्ती॒रिति॑ प्रति - तिष्ठ॑न्तीः । प॒शवोऽनु॑ । अनु॒ प्रति॑ । प्रति॑ तिष्ठन्ति । ति॒ष्ठ॒न्ति॒ प॒शून् । प॒शून्. य॒ज्ञ्ः । य॒ज्ञो य॒ज्ञ्म् । य॒ज्ञ्म् ॅयज॑मानः । यज॑मानो॒ यज॑मानम् । यज॑मानम् प्र॒जाः । प्र॒जास्तस्मा᳚त् । प्र॒जा इति॑ प्र - जाः । तस्मा॑द॒पः । अ॒प उप॑ । उप॑ सृजति । सृ॒ज॒ति॒ प्रति॑ष्ठित्यै । प्रति॑ष्ठित्यै॒ यत् । प्रति॑ष्ठित्या॒ इति॒ प्रति॑ - स्थि॒त्यै॒ । यद॑द्ध्व॒र्युः । अ॒द्ध्व॒र्युर॑न॒ग्नौ । अ॒न॒ग्नावाहु॑तिम् । आहु॑तिम् जुहु॒यात् । आहु॑ति॒मित्या - हु॒ति॒म् । जु॒हु॒याद॒न्धः । अ॒न्धो᳚ऽद्ध्व॒र्युः । अ॒द्ध्व॒र्युः स्या᳚त् \newline

\textbf{Jatai Paata} \newline

1. उत् क्रा॑म क्रा॒मोदुत् क्रा॑म । \newline
2. क्रा॒मोदुत् क्रा॑म क्रा॒मोत् । \newline
3. उद॑क्रमी दक्रमी॒ दुदु द॑क्रमीत् । \newline
4. अ॒क्र॒मी॒ दिती त्य॑क्रमी दक्रमी॒ दिति॑ । \newline
5. इति॒ द्वाभ्या॒म् द्वाभ्या॒ मितीति॒ द्वाभ्या᳚म् । \newline
6. द्वाभ्या॒ मुदुद् द्वाभ्या॒म् द्वाभ्या॒ मुत् । \newline
7. उत् क्र॑मयति क्रमय॒ त्युदुत् क्र॑मयति । \newline
8. क्र॒म॒य॒ति॒ प्रति॑ष्ठित्यै॒ प्रति॑ष्ठित्यै क्रमयति क्रमयति॒ प्रति॑ष्ठित्यै । \newline
9. प्रति॑ष्ठित्या॒ अनु॑रूपाभ्या॒ मनु॑रूपाभ्या॒म् प्रति॑ष्ठित्यै॒ प्रति॑ष्ठित्या॒ अनु॑रूपाभ्याम् । \newline
10. प्रति॑ष्ठित्या॒ इति॒ प्रति॑ - स्थि॒त्यै॒ । \newline
11. अनु॑रूपाभ्या॒म् तस्मा॒त् तस्मा॒ दनु॑रूपाभ्या॒ मनु॑रूपाभ्या॒म् तस्मा᳚त् । \newline
12. अनु॑रूपाभ्या॒मित्यनु॑ - रू॒पा॒भ्या॒म् । \newline
13. तस्मा॒ दनु॑रूपा॒ अनु॑रूपा॒ स्तस्मा॒त् तस्मा॒ दनु॑रूपाः । \newline
14. अनु॑रूपाः प॒शवः॑ प॒शवो ऽनु॑रूपा॒ अनु॑रूपाः प॒शवः॑ । \newline
15. अनु॑रूपा॒ इत्यनु॑ - रू॒पाः॒ । \newline
16. प॒शवः॒ प्र प्र प॒शवः॑ प॒शवः॒ प्र । \newline
17. प्र जा॑यन्ते जायन्ते॒ प्र प्र जा॑यन्ते । \newline
18. जा॒य॒न्ते॒ ऽपो॑ ऽपो जा॑यन्ते जायन्ते॒ ऽपः । \newline
19. अ॒प उपो पा॒पो॑ ऽप उप॑ । \newline
20. उप॑ सृजति सृज॒ त्युपोप॑ सृजति । \newline
21. सृ॒ज॒ति॒ यत्र॒ यत्र॑ सृजति सृजति॒ यत्र॑ । \newline
22. यत्र॒ वै वै यत्र॒ यत्र॒ वै । \newline
23. वा आप॒ आपो॒ वै वा आपः॑ । \newline
24. आप॑ उप॒गच्छ॑ न्त्युप॒गच्छ॒ न्त्याप॒ आप॑ उप॒गच्छ॑न्ति । \newline
25. उ॒प॒गच्छ॑न्ति॒ तत् तदु॑प॒गच्छ॑ न्त्युप॒गच्छ॑न्ति॒ तत् । \newline
26. उ॒प॒गच्छ॒न्तीत्यु॑प - गच्छ॑न्ति । \newline
27. तदोष॑धय॒ ओष॑धय॒ स्तत् तदोष॑धयः । \newline
28. ओष॑धयः॒ प्रति॒ प्रत्योष॑धय॒ ओष॑धयः॒ प्रति॑ । \newline
29. प्रति॑ तिष्ठन्ति तिष्ठन्ति॒ प्रति॒ प्रति॑ तिष्ठन्ति । \newline
30. ति॒ष्ठ॒ न्त्योष॑धी॒ रोष॑धी स्तिष्ठन्ति तिष्ठ॒ न्त्योष॑धीः । \newline
31. ओष॑धीः प्रति॒तिष्ठ॑न्तीः प्रति॒तिष्ठ॑न्ती॒ रोष॑धी॒ रोष॑धीः प्रति॒तिष्ठ॑न्तीः । \newline
32. प्र॒ति॒तिष्ठ॑न्तीः प॒शवः॑ प॒शवः॑ प्रति॒तिष्ठ॑न्तीः प्रति॒तिष्ठ॑न्तीः प॒शवः॑ । \newline
33. प्र॒ति॒तिष्ठ॑न्ती॒रिति॑ प्रति - तिष्ठ॑न्तीः । \newline
34. प॒शवो ऽन्वनु॑ प॒शवः॑ प॒शवो ऽनु॑ । \newline
35. अनु॒ प्रति॒ प्रत्यन्वनु॒ प्रति॑ । \newline
36. प्रति॑ तिष्ठन्ति तिष्ठन्ति॒ प्रति॒ प्रति॑ तिष्ठन्ति । \newline
37. ति॒ष्ठ॒न्ति॒ प॒शून् प॒शून् ति॑ष्ठन्ति तिष्ठन्ति प॒शून् । \newline
38. प॒शून्. य॒ज्ञो य॒ज्ञ्ः प॒शून् प॒शून्. य॒ज्ञ्ः । \newline
39. य॒ज्ञो य॒ज्ञ्ं ॅय॒ज्ञ्ं ॅय॒ज्ञो य॒ज्ञो य॒ज्ञ्म् । \newline
40. य॒ज्ञ्ं ॅयज॑मानो॒ यज॑मानो य॒ज्ञ्ं ॅय॒ज्ञ्ं ॅयज॑मानः । \newline
41. यज॑मानो॒ यज॑मानं॒ ॅयज॑मानं॒ ॅयज॑मानो॒ यज॑मानो॒ यज॑मानम् । \newline
42. यज॑मानम् प्र॒जाः प्र॒जा यज॑मानं॒ ॅयज॑मानम् प्र॒जाः । \newline
43. प्र॒जा स्तस्मा॒त् तस्मा᳚त् प्र॒जाः प्र॒जा स्तस्मा᳚त् । \newline
44. प्र॒जा इति॑ प्र - जाः । \newline
45. तस्मा॑ द॒पो॑ ऽप स्तस्मा॒त् तस्मा॑ द॒पः । \newline
46. अ॒प उपो पा॒पो॑ ऽप उप॑ । \newline
47. उप॑ सृजति सृज॒ त्युपोप॑ सृजति । \newline
48. सृ॒ज॒ति॒ प्रति॑ष्ठित्यै॒ प्रति॑ष्ठित्यै सृजति सृजति॒ प्रति॑ष्ठित्यै । \newline
49. प्रति॑ष्ठित्यै॒ यद् यत् प्रति॑ष्ठित्यै॒ प्रति॑ष्ठित्यै॒ यत् । \newline
50. प्रति॑ष्ठित्या॒ इति॒ प्रति॑ - स्थि॒त्यै॒ । \newline
51. यद॑द्ध्व॒र्यु र॑द्ध्व॒र्युर् यद् यद॑द्ध्व॒र्युः । \newline
52. अ॒द्ध्व॒र्यु र॑न॒ग्ना व॑न॒ग्ना व॑द्ध्व॒र्यु र॑द्ध्व॒र्यु र॑न॒ग्नौ । \newline
53. अ॒न॒ग्ना वाहु॑ति॒ माहु॑ति मन॒ग्ना व॑न॒ग्ना वाहु॑तिम् । \newline
54. आहु॑तिम् जुहु॒याज् जु॑हु॒या दाहु॑ति॒ माहु॑तिम् जुहु॒यात् । \newline
55. आहु॑ति॒मित्या - हु॒ति॒म् । \newline
56. जु॒हु॒या द॒न्धो᳚ ऽन्धो जु॑हु॒याज् जु॑हु॒या द॒न्धः । \newline
57. अ॒न्धो᳚ ऽद्ध्व॒र्यु र॑द्ध्व॒र्यु र॒न्धो᳚(1॒) ऽन्धो᳚ ऽद्ध्व॒र्युः । \newline
58. अ॒द्ध्व॒र्युः स्या᳚थ् स्या दद्ध्व॒र्यु र॑द्ध्व॒र्युः स्या᳚त् । \newline

\textbf{Ghana Paata } \newline

1. उत् क्रा॑म क्रा॒मो दुत् क्रा॒मो दुत् क्रा॒मो दुत् क्रा॒मोत् । \newline
2. क्रा॒मो दुत् क्रा॑म क्रा॒मो द॑क्रमी दक्रमी॒ दुत् क्रा॑म क्रा॒मो द॑क्रमीत् । \newline
3. उद॑क्रमी दक्रमी॒ दुदु द॑क्रमी॒ दिती त्य॑क्रमी॒ दुदु द॑क्रमी॒ दिति॑ । \newline
4. अ॒क्र॒मी॒ दिती त्य॑क्रमी दक्रमी॒दिति॒ द्वाभ्या॒म् द्वाभ्या॒ मित्य॑क्रमी दक्रमी॒ दिति॒ द्वाभ्या᳚म् । \newline
5. इति॒ द्वाभ्या॒म् द्वाभ्या॒ मितीति॒ द्वाभ्या॒ मुदुद् द्वाभ्या॒ मितीति॒ द्वाभ्या॒ मुत् । \newline
6. द्वाभ्या॒ मुदुद् द्वाभ्या॒म् द्वाभ्या॒ मुत् क्र॑मयति क्रमय॒ त्युद् द्वाभ्या॒म् द्वाभ्या॒ मुत् क्र॑मयति । \newline
7. उत् क्र॑मयति क्रमय॒ त्युदुत् क्र॑मयति॒ प्रति॑ष्ठित्यै॒ प्रति॑ष्ठित्यै क्रमय॒ त्युदुत् क्र॑मयति॒ प्रति॑ष्ठित्यै । \newline
8. क्र॒म॒य॒ति॒ प्रति॑ष्ठित्यै॒ प्रति॑ष्ठित्यै क्रमयति क्रमयति॒ प्रति॑ष्ठित्या॒ अनु॑रूपाभ्या॒ मनु॑रूपाभ्या॒म् प्रति॑ष्ठित्यै क्रमयति क्रमयति॒ प्रति॑ष्ठित्या॒ अनु॑रूपाभ्याम् । \newline
9. प्रति॑ष्ठित्या॒ अनु॑रूपाभ्या॒ मनु॑रूपाभ्या॒म् प्रति॑ष्ठित्यै॒ प्रति॑ष्ठित्या॒ अनु॑रूपाभ्या॒म् तस्मा॒त् तस्मा॒ दनु॑रूपाभ्या॒म् प्रति॑ष्ठित्यै॒ प्रति॑ष्ठित्या॒ अनु॑रूपाभ्या॒म् तस्मा᳚त् । \newline
10. प्रति॑ष्ठित्या॒ इति॒ प्रति॑ - स्थि॒त्यै॒ । \newline
11. अनु॑रूपाभ्या॒म् तस्मा॒त् तस्मा॒ दनु॑रूपाभ्या॒ मनु॑रूपाभ्या॒म् तस्मा॒ दनु॑रूपा॒ अनु॑रूपा॒ स्तस्मा॒ दनु॑रूपाभ्या॒ मनु॑रूपाभ्या॒म् तस्मा॒ दनु॑रूपाः । \newline
12. अनु॑रूपाभ्या॒मित्यनु॑ - रू॒पा॒भ्या॒म् । \newline
13. तस्मा॒ दनु॑रूपा॒ अनु॑रूपा॒ स्तस्मा॒त् तस्मा॒ दनु॑रूपाः प॒शवः॑ प॒शवो ऽनु॑रूपा॒ स्तस्मा॒त् तस्मा॒ दनु॑रूपाः प॒शवः॑ । \newline
14. अनु॑रूपाः प॒शवः॑ प॒शवो ऽनु॑रूपा॒ अनु॑रूपाः प॒शवः॒ प्र प्र प॒शवो ऽनु॑रूपा॒ अनु॑रूपाः प॒शवः॒ प्र । \newline
15. अनु॑रूपा॒ इत्यनु॑ - रू॒पाः॒ । \newline
16. प॒शवः॒ प्र प्र प॒शवः॑ प॒शवः॒ प्र जा॑यन्ते जायन्ते॒ प्र प॒शवः॑ प॒शवः॒ प्र जा॑यन्ते । \newline
17. प्र जा॑यन्ते जायन्ते॒ प्र प्र जा॑यन्ते॒ ऽपो॑ ऽपो जा॑यन्ते॒ प्र प्र जा॑यन्ते॒ ऽपः । \newline
18. जा॒य॒न्ते॒ ऽपो॑ ऽपो जा॑यन्ते जायन्ते॒ ऽप उपोपा॒पो जा॑यन्ते जायन्ते॒ ऽप उप॑ । \newline
19. अ॒प उपोपा॒ पो॑ ऽप उप॑ सृजति सृज॒ त्युपा॒ पो॑ ऽप उप॑ सृजति । \newline
20. उप॑ सृजति सृज॒ त्युपोप॑ सृजति॒ यत्र॒ यत्र॑ सृज॒ त्युपोप॑ सृजति॒ यत्र॑ । \newline
21. सृ॒ज॒ति॒ यत्र॒ यत्र॑ सृजति सृजति॒ यत्र॒ वै वै यत्र॑ सृजति सृजति॒ यत्र॒ वै । \newline
22. यत्र॒ वै वै यत्र॒ यत्र॒ वा आप॒ आपो॒ वै यत्र॒ यत्र॒ वा आपः॑ । \newline
23. वा आप॒ आपो॒ वै वा आप॑ उप॒गच्छ॑ न्त्युप॒गच्छ॒ न्त्यापो॒ वै वा आप॑ उप॒गच्छ॑न्ति । \newline
24. आप॑ उप॒गच्छ॑ न्त्युप॒गच्छ॒ न्त्याप॒ आप॑ उप॒गच्छ॑न्ति॒ तत् तदु॑प॒गच्छ॒ न्त्याप॒ आप॑ उप॒गच्छ॑न्ति॒ तत् । \newline
25. उ॒प॒गच्छ॑न्ति॒ तत् तदु॑प॒गच्छ॑ न्त्युप॒गच्छ॑न्ति॒ तदोष॑धय॒ ओष॑धय॒ स्तदु॑प॒गच्छ॑ न्त्युप॒गच्छ॑न्ति॒ तदोष॑धयः । \newline
26. उ॒प॒गच्छ॒न्तीत्यु॑प - गच्छ॑न्ति । \newline
27. तदोष॑धय॒ ओष॑धय॒ स्तत् तदोष॑धयः॒ प्रति॒ प्रत्योष॑धय॒ स्तत् तदोष॑धयः॒ प्रति॑ । \newline
28. ओष॑धयः॒ प्रति॒ प्रत्योष॑धय॒ ओष॑धयः॒ प्रति॑ तिष्ठन्ति तिष्ठन्ति॒ प्रत्योष॑धय॒ ओष॑धयः॒ प्रति॑ तिष्ठन्ति । \newline
29. प्रति॑ तिष्ठन्ति तिष्ठन्ति॒ प्रति॒ प्रति॑ तिष्ठ॒ न्त्योष॑धी॒ रोष॑धी स्तिष्ठन्ति॒ प्रति॒ प्रति॑ तिष्ठ॒ न्त्योष॑धीः । \newline
30. ति॒ष्ठ॒ न्त्योष॑धी॒ रोष॑धी स्तिष्ठन्ति तिष्ठ॒ न्त्योष॑धीः प्रति॒तिष्ठ॑न्तीः प्रति॒तिष्ठ॑न्ती॒ रोष॑धी स्तिष्ठन्ति तिष्ठ॒ न्त्योष॑धीः प्रति॒तिष्ठ॑न्तीः । \newline
31. ओष॑धीः प्रति॒तिष्ठ॑न्तीः प्रति॒तिष्ठ॑न्ती॒ रोष॑धी॒ रोष॑धीः प्रति॒तिष्ठ॑न्तीः प॒शवः॑ प॒शवः॑ प्रति॒तिष्ठ॑न्ती॒ रोष॑धी॒ रोष॑धीः प्रति॒तिष्ठ॑न्तीः प॒शवः॑ । \newline
32. प्र॒ति॒तिष्ठ॑न्तीः प॒शवः॑ प॒शवः॑ प्रति॒तिष्ठ॑न्तीः प्रति॒तिष्ठ॑न्तीः प॒शवो ऽन्वनु॑ प॒शवः॑ प्रति॒तिष्ठ॑न्तीः प्रति॒तिष्ठ॑न्तीः प॒शवो ऽनु॑ । \newline
33. प्र॒ति॒तिष्ठ॑न्ती॒रिति॑ प्रति - तिष्ठ॑न्तीः । \newline
34. प॒शवो ऽन्वनु॑ प॒शवः॑ प॒शवो ऽनु॒ प्रति॒ प्रत्यनु॑ प॒शवः॑ प॒शवो ऽनु॒ प्रति॑ । \newline
35. अनु॒ प्रति॒ प्रत्यन्वनु॒ प्रति॑ तिष्ठन्ति तिष्ठन्ति॒ प्रत्यन्वनु॒ प्रति॑ तिष्ठन्ति । \newline
36. प्रति॑ तिष्ठन्ति तिष्ठन्ति॒ प्रति॒ प्रति॑ तिष्ठन्ति प॒शून् प॒शून् ति॑ष्ठन्ति॒ प्रति॒ प्रति॑ तिष्ठन्ति प॒शून् । \newline
37. ति॒ष्ठ॒न्ति॒ प॒शून् प॒शून् ति॑ष्ठन्ति तिष्ठन्ति प॒शून्. य॒ज्ञो य॒ज्ञ्ः प॒शून् ति॑ष्ठन्ति तिष्ठन्ति प॒शून्. य॒ज्ञ्ः । \newline
38. प॒शून्. य॒ज्ञो य॒ज्ञ्ः प॒शून् प॒शून्. य॒ज्ञो य॒ज्ञ्ं ॅय॒ज्ञ्ं ॅय॒ज्ञ्ः प॒शून् प॒शून्. य॒ज्ञो य॒ज्ञ्म् । \newline
39. य॒ज्ञो य॒ज्ञ्ं ॅय॒ज्ञ्ं ॅय॒ज्ञो य॒ज्ञो य॒ज्ञ्ं ॅयज॑मानो॒ यज॑मानो य॒ज्ञ्ं ॅय॒ज्ञो य॒ज्ञो य॒ज्ञ्ं ॅयज॑मानः । \newline
40. य॒ज्ञ्ं ॅयज॑मानो॒ यज॑मानो य॒ज्ञ्ं ॅय॒ज्ञ्ं ॅयज॑मानो॒ यज॑मानं॒ ॅयज॑मानं॒ ॅयज॑मानो य॒ज्ञ्ं ॅय॒ज्ञ्ं ॅयज॑मानो॒ यज॑मानम् । \newline
41. यज॑मानो॒ यज॑मानं॒ ॅयज॑मानं॒ ॅयज॑मानो॒ यज॑मानो॒ यज॑मानम् प्र॒जाः प्र॒जा यज॑मानं॒ ॅयज॑मानो॒ यज॑मानो॒ यज॑मानम् प्र॒जाः । \newline
42. यज॑मानम् प्र॒जाः प्र॒जा यज॑मानं॒ ॅयज॑मानम् प्र॒जा स्तस्मा॒त् तस्मा᳚त् प्र॒जा यज॑मानं॒ ॅयज॑मानम् प्र॒जा स्तस्मा᳚त् । \newline
43. प्र॒जा स्तस्मा॒त् तस्मा᳚त् प्र॒जाः प्र॒जा स्तस्मा॑ द॒पो॑ ऽप स्तस्मा᳚त् प्र॒जाः प्र॒जा स्तस्मा॑ द॒पः । \newline
44. प्र॒जा इति॑ प्र - जाः । \newline
45. तस्मा॑ द॒पो॑ ऽप स्तस्मा॒त् तस्मा॑ द॒प उपोपा॒प स्तस्मा॒त् तस्मा॑ द॒प उप॑ । \newline
46. अ॒प उपोपा॒पो॑ ऽप उप॑ सृजति सृज॒ त्युपा॒पो॑ ऽप उप॑ सृजति । \newline
47. उप॑ सृजति सृज॒ त्युपोप॑ सृजति॒ प्रति॑ष्ठित्यै॒ प्रति॑ष्ठित्यै सृज॒ त्युपोप॑ सृजति॒ प्रति॑ष्ठित्यै । \newline
48. सृ॒ज॒ति॒ प्रति॑ष्ठित्यै॒ प्रति॑ष्ठित्यै सृजति सृजति॒ प्रति॑ष्ठित्यै॒ यद् यत् प्रति॑ष्ठित्यै सृजति सृजति॒ प्रति॑ष्ठित्यै॒ यत् । \newline
49. प्रति॑ष्ठित्यै॒ यद् यत् प्रति॑ष्ठित्यै॒ प्रति॑ष्ठित्यै॒ यद॑द्ध्व॒र्यु र॑द्ध्व॒र्युर् यत् प्रति॑ष्ठित्यै॒ प्रति॑ष्ठित्यै॒ यद॑द्ध्व॒र्युः । \newline
50. प्रति॑ष्ठित्या॒ इति॒ प्रति॑ - स्थि॒त्यै॒ । \newline
51. यद॑द्ध्व॒र्यु र॑द्ध्व॒र्युर् यद् यद॑द्ध्व॒र्यु र॑न॒ग्ना व॑न॒ग्ना व॑द्ध्व॒र्युर् यद् यद॑द्ध्व॒र्यु र॑न॒ग्नौ । \newline
52. अ॒द्ध्व॒र्यु र॑न॒ग्ना व॑न॒ग्ना व॑द्ध्व॒र्यु र॑द्ध्व॒र्यु र॑न॒ग्ना वाहु॑ति॒ माहु॑ति मन॒ग्ना व॑द्ध्व॒र्यु र॑द्ध्व॒र्यु र॑न॒ग्ना वाहु॑तिम् । \newline
53. अ॒न॒ग्ना वाहु॑ति॒ माहु॑ति मन॒ग्ना व॑न॒ग्ना वाहु॑तिम् जुहु॒याज् जु॑हु॒या दाहु॑ति मन॒ग्ना व॑न॒ग्ना वाहु॑तिम् जुहु॒यात् । \newline
54. आहु॑तिम् जुहु॒याज् जु॑हु॒या दाहु॑ति॒ माहु॑तिम् जुहु॒या द॒न्धो᳚ ऽन्धो जु॑हु॒या दाहु॑ति॒ माहु॑तिम् जुहु॒या द॒न्धः । \newline
55. आहु॑ति॒मित्या - हु॒ति॒म् । \newline
56. जु॒हु॒या द॒न्धो᳚ ऽन्धो जु॑हु॒याज् जु॑हु॒या द॒न्धो᳚ ऽद्ध्व॒र्यु र॑द्ध्व॒र्यु र॒न्धो जु॑हु॒याज् जु॑हु॒या द॒न्धो᳚ ऽद्ध्व॒र्युः । \newline
57. अ॒न्धो᳚ ऽद्ध्व॒र्यु र॑द्ध्व॒र्यु र॒न्धो᳚(1॒) ऽन्धो᳚ ऽद्ध्व॒र्युः स्या᳚थ् स्या दद्ध्व॒र्यु र॒न्धो᳚(1॒) ऽन्धो᳚ ऽद्ध्व॒र्युः स्या᳚त् । \newline
58. अ॒द्ध्व॒र्युः स्या᳚थ् स्या दद्ध्व॒र्यु र॑द्ध्व॒र्युः स्या॒द् रक्षाꣳ॑सि॒ रक्षाꣳ॑सि स्या दद्ध्व॒र्यु र॑द्ध्व॒र्युः स्या॒द् रक्षाꣳ॑सि । \newline
\pagebreak
\markright{ TS 5.1.3.2  \hfill https://www.vedavms.in \hfill}

\section{ TS 5.1.3.2 }

\textbf{TS 5.1.3.2 } \newline
\textbf{Samhita Paata} \newline

स्या॒द्-रक्षाꣳ॑सि य॒ज्ञ्ꣳ ह॑न्यु॒र्॒.हिर॑ण्यमु॒पास्य॑ जुहोत्यग्नि॒वत्ये॒व जु॑होति॒ नान्धो᳚-ऽद्ध्व॒र्युर्भव॑ति॒ न य॒ज्ञ्ꣳ रक्षाꣳ॑सि घ्नन्ति॒ जिघ॑र्म्य॒ग्निं मन॑सा घृ॒तेनेत्या॑ह॒ मन॑सा॒ हि पुरु॑षो य॒ज्ञ्म॑भि॒गच्छ॑ति प्रति॒क्ष्यन्तं॒ भुव॑नानि॒ विश्वेत्या॑ह॒ सर्वꣳ॒॒ ह्ये॑ष प्र॒त्यङ् क्षेति॑ पृ॒थुं ति॑र॒श्चा वय॑सा बृ॒हन्त॒मित्या॒हाऽल्पो॒ ह्ये॑ष जा॒तो म॒हान् - [  ] \newline

\textbf{Pada Paata} \newline

स्या॒त् । रक्षाꣳ॑सि । य॒ज्ञ्म् । ह॒न्युः॒ । हिर॑ण्यम् । उ॒पास्येत्यु॑प - अस्य॑ । जु॒हो॒ति॒ । अ॒ग्नि॒वतीत्य॑ग्नि - वति॑ । ए॒व । जु॒हो॒ति॒ । न । अ॒न्धः । अ॒द्ध्व॒र्युः । भव॑ति । न । य॒ज्ञ्म् । रक्षाꣳ॑सि । घ्न॒न्ति॒ । जिघ॑र्मि । अ॒ग्निम् । मन॑सा । घृ॒तेन॑ । इति॑ । आ॒ह॒ । मन॑सा । हि । पुरु॑षः । य॒ज्ञ्म् । अ॒भि॒गच्छ॒तीत्य॑भि - गच्छ॑ति । प्र॒ति॒क्ष्यन्त॒मिति॑ प्रति - क्ष्यन्त᳚म् । भुव॑नानि । विश्वा᳚ । इति॑ । आ॒ह॒ । सर्व᳚म् । हि । ए॒षः । प्र॒त्यङ् । क्षेति॑ । पृ॒थुम् । ति॒र॒श्चा । वय॑सा । बृ॒हन्त᳚म् । इति॑ । आ॒ह॒ । अल्पः॑ । हि । ए॒षः । जा॒तः । म॒हान् ।  \newline


\textbf{Krama Paata} \newline

स्या॒द् रक्षाꣳ॑सि । रक्षाꣳ॑सि य॒ज्ञ्म् । य॒ज्ञ्ꣳ ह॑न्युः । ह॒न्यु॒र्॒. हिर॑ण्यम् । हिर॑ण्यमु॒पास्य॑ । उ॒पास्य॑ जुहोति । उ॒पास्येत्यु॑प - अस्य॑ । जु॒हो॒त्य॒ग्नि॒वति॑ । अ॒ग्नि॒वत्ये॒व । अ॒ग्नि॒वतीत्य॑ग्नि - वति॑ । ए॒व जु॑होति । जु॒हो॒ति॒ न । नान्धः॑ । अ॒न्धो᳚ऽद्ध्व॒र्युः । अ॒द्ध्व॒र्युर् भव॑ति । भव॑ति॒ न । न य॒ज्ञ्म् । य॒ज्ञ्ꣳ रक्षाꣳ॑सि । रक्षाꣳ॑सि घ्नन्ति । घ्न॒न्ति॒ जिघ॑र्मि । जिघ॑र्म्य॒ग्निम् । अ॒ग्निम् मन॑सा । मन॑सा घृ॒तेन॑ । घृ॒तेनेति॑ । इत्या॑ह । आ॒ह॒ मन॑सा । मन॑सा॒ हि । हि पुरु॑षः । पुरु॑षो य॒ज्ञ्म् । य॒ज्ञ्म॑भि॒गच्छ॑ति । अ॒भि॒गच्छ॑ति प्रति॒क्ष्यन्त᳚म् । अ॒भि॒गच्छ॒तीत्य॑भि - गच्छ॑ति । प्र॒ति॒क्ष्यन्त॒म् भुव॑नानि । प्र॒ति॒क्ष्यन्त॒मिति॑ प्रति - क्ष्यन्त᳚म् । भुव॑नानि॒ विश्वा᳚ । विश्वेति॑ । इत्या॑ह । आ॒ह॒ सर्व᳚म् । सर्वꣳ॒॒ हि । ह्ये॑षः । ए॒ष प्र॒त्यङ्ङ् । प्र॒त्यङ् क्षेति॑ । क्षेति॑ पृ॒थुम् । पृ॒थुम् ति॑र॒श्चा । ति॒र॒श्चा वय॑सा । वय॑सा बृ॒हन्त᳚म् । बृ॒हन्त॒मिति॑ । इत्या॑ह । आ॒हाल्पः॑ । अल्पो॒ हि । ह्ये॑षः । ए॒ष जा॒तः । जा॒तो म॒हान् । म॒हान् भव॑ति \newline

\textbf{Jatai Paata} \newline

1. स्या॒द् रक्षाꣳ॑सि॒ रक्षाꣳ॑सि स्याथ् स्या॒द् रक्षाꣳ॑सि । \newline
2. रक्षाꣳ॑सि य॒ज्ञ्ं ॅय॒ज्ञ्ꣳ रक्षाꣳ॑सि॒ रक्षाꣳ॑सि य॒ज्ञ्म् । \newline
3. य॒ज्ञ्ꣳ ह॑न्युर्. हन्युर् य॒ज्ञ्ं ॅय॒ज्ञ्ꣳ ह॑न्युः । \newline
4. ह॒न्यु॒र्॒. हिर॑ण्यꣳ॒॒ हिर॑ण्यꣳ हन्युर्. हन्यु॒र्॒. हिर॑ण्यम् । \newline
5. हिर॑ण्य मु॒पा स्यो॒पास्य॒ हिर॑ण्यꣳ॒॒ हिर॑ण्य मु॒पास्य॑ । \newline
6. उ॒पास्य॑ जुहोति जुहो त्यु॒पा स्यो॒पास्य॑ जुहोति । \newline
7. उ॒पास्येत्यु॑प - अस्य॑ । \newline
8. जु॒हो॒ त्य॒ग्नि॒व त्य॑ग्नि॒वति॑ जुहोति जुहो त्यग्नि॒वति॑ । \newline
9. अ॒ग्नि॒व त्ये॒वैवाग्नि॒व त्य॑ग्नि॒व त्ये॒व । \newline
10. अ॒ग्नि॒वतीत्य॑ग्नि - वति॑ । \newline
11. ए॒व जु॑होति जुहो त्ये॒वैव जु॑होति । \newline
12. जु॒हो॒ति॒ न न जु॑होति जुहोति॒ न । \newline
13. नान्धो᳚ ऽन्धो न नान्धः । \newline
14. अ॒न्धो᳚ ऽद्ध्व॒र्यु र॑द्ध्व॒र्यु र॒न्धो᳚(1॒) ऽन्धो᳚ ऽद्ध्व॒र्युः । \newline
15. अ॒द्ध्व॒र्युर् भव॑ति॒ भव॑ त्यद्ध्व॒र्यु र॑द्ध्व॒र्युर् भव॑ति । \newline
16. भव॑ति॒ न न भव॑ति॒ भव॑ति॒ न । \newline
17. न य॒ज्ञ्ं ॅय॒ज्ञ्म् न न य॒ज्ञ्म् । \newline
18. य॒ज्ञ्ꣳ रक्षाꣳ॑सि॒ रक्षाꣳ॑सि य॒ज्ञ्ं ॅय॒ज्ञ्ꣳ रक्षाꣳ॑सि । \newline
19. रक्षाꣳ॑सि घ्नन्ति घ्नन्ति॒ रक्षाꣳ॑सि॒ रक्षाꣳ॑सि घ्नन्ति । \newline
20. घ्न॒न्ति॒ जिघ॑र्मि॒ जिघ॑र्मि घ्नन्ति घ्नन्ति॒ जिघ॑र्मि । \newline
21. जिघ॑र् म्य॒ग्नि म॒ग्निम् जिघ॑र्मि॒ जिघ॑र् म्य॒ग्निम् । \newline
22. अ॒ग्निम् मन॑सा॒ मन॑सा॒ ऽग्नि म॒ग्निम् मन॑सा । \newline
23. मन॑सा घृ॒तेन॑ घृ॒तेन॒ मन॑सा॒ मन॑सा घृ॒तेन॑ । \newline
24. घृ॒तेने तीति॑ घृ॒तेन॑ घृ॒तेनेति॑ । \newline
25. इत्या॑हा॒हे तीत्या॑ह । \newline
26. आ॒ह॒ मन॑सा॒ मन॑सा ऽऽहाह॒ मन॑सा । \newline
27. मन॑सा॒ हि हि मन॑सा॒ मन॑सा॒ हि । \newline
28. हि पुरु॑षः॒ पुरु॑षो॒ हि हि पुरु॑षः । \newline
29. पुरु॑षो य॒ज्ञ्ं ॅय॒ज्ञ्म् पुरु॑षः॒ पुरु॑षो य॒ज्ञ्म् । \newline
30. य॒ज्ञ् म॑भि॒गच्छ॑ त्यभि॒गच्छ॑ति य॒ज्ञ्ं ॅय॒ज्ञ् म॑भि॒गच्छ॑ति । \newline
31. अ॒भि॒गच्छ॑ति प्रति॒क्ष्यन्त॑म् प्रति॒क्ष्यन्त॑ मभि॒गच्छ॑ त्यभि॒गच्छ॑ति प्रति॒क्ष्यन्त᳚म् । \newline
32. अ॒भि॒गच्छ॒तीत्य॑भि - गच्छ॑ति । \newline
33. प्र॒ति॒क्ष्यन्त॒म् भुव॑नानि॒ भुव॑नानि प्रति॒क्ष्यन्त॑म् प्रति॒क्ष्यन्त॒म् भुव॑नानि । \newline
34. प्र॒ति॒क्ष्यन्त॒मिति॑ प्रति - क्ष्यन्त᳚म् । \newline
35. भुव॑नानि॒ विश्वा॒ विश्वा॒ भुव॑नानि॒ भुव॑नानि॒ विश्वा᳚ । \newline
36. विश्वेतीति॒ विश्वा॒ विश्वेति॑ । \newline
37. इत्या॑हा॒हे तीत्या॑ह । \newline
38. आ॒ह॒ सर्वꣳ॒॒ सर्व॑ माहाह॒ सर्व᳚म् । \newline
39. सर्वꣳ॒॒ हि हि सर्वꣳ॒॒ सर्वꣳ॒॒ हि । \newline
40. ह्ये॑ष ए॒ष हि ह्ये॑षः । \newline
41. ए॒ष प्र॒त्यङ् प्र॒त्यङ् ङे॒ष ए॒ष प्र॒त्यङ् । \newline
42. प्र॒त्यङ् क्षेति॒ क्षेति॑ प्र॒त्यङ् प्र॒त्यङ् क्षेति॑ । \newline
43. क्षेति॑ पृ॒थुम् पृ॒थुम् क्षेति॒ क्षेति॑ पृ॒थुम् । \newline
44. पृ॒थुम् ति॑र॒श्चा ति॑र॒श्चा पृ॒थुम् पृ॒थुम् ति॑र॒श्चा । \newline
45. ति॒र॒श्चा वय॑सा॒ वय॑सा तिर॒श्चा ति॑र॒श्चा वय॑सा । \newline
46. वय॑सा बृ॒हन्त॑म् बृ॒हन्तं॒ ॅवय॑सा॒ वय॑सा बृ॒हन्त᳚म् । \newline
47. बृ॒हन्त॒ मितीति॑ बृ॒हन्त॑म् बृ॒हन्त॒ मिति॑ । \newline
48. इत्या॑हा॒हे तीत्या॑ह । \newline
49. आ॒हाल्पो ऽल्प॑ आहा॒ हाल्पः॑ । \newline
50. अल्पो॒ हि ह्यल्पो ऽल्पो॒ हि । \newline
51. ह्ये॑ष ए॒ष हि ह्ये॑षः । \newline
52. ए॒ष जा॒तो जा॒त ए॒ष ए॒ष जा॒तः । \newline
53. जा॒तो म॒हान् म॒हान् जा॒तो जा॒तो म॒हान् । \newline
54. म॒हान् भव॑ति॒ भव॑ति म॒हान् म॒हान् भव॑ति । \newline

\textbf{Ghana Paata } \newline

1. स्या॒द् रक्षाꣳ॑सि॒ रक्षाꣳ॑सि स्याथ् स्या॒द् रक्षाꣳ॑सि य॒ज्ञ्ं ॅय॒ज्ञ्ꣳ रक्षाꣳ॑सि स्याथ् स्या॒द् रक्षाꣳ॑सि य॒ज्ञ्म् । \newline
2. रक्षाꣳ॑सि य॒ज्ञ्ं ॅय॒ज्ञ्ꣳ रक्षाꣳ॑सि॒ रक्षाꣳ॑सि य॒ज्ञ्ꣳ ह॑न्युर्. हन्युर् य॒ज्ञ्ꣳ रक्षाꣳ॑सि॒ रक्षाꣳ॑सि य॒ज्ञ्ꣳ ह॑न्युः । \newline
3. य॒ज्ञ्ꣳ ह॑न्युर्. हन्युर् य॒ज्ञ्ं ॅय॒ज्ञ्ꣳ ह॑न्यु॒र्॒. हिर॑ण्यꣳ॒॒ हिर॑ण्यꣳ हन्युर् य॒ज्ञ्ं ॅय॒ज्ञ्ꣳ ह॑न्यु॒र्॒. हिर॑ण्यम् । \newline
4. ह॒न्यु॒र्॒. हिर॑ण्यꣳ॒॒ हिर॑ण्यꣳ हन्युर्. हन्यु॒र्॒. हिर॑ण्य मु॒पा स्यो॒पास्य॒ हिर॑ण्यꣳ हन्युर्. हन्यु॒र्॒. हिर॑ण्य मु॒पास्य॑ । \newline
5. हिर॑ण्य मु॒पा स्यो॒पास्य॒ हिर॑ण्यꣳ॒॒ हिर॑ण्य मु॒पास्य॑ जुहोति जुहो त्यु॒पास्य॒ हिर॑ण्यꣳ॒॒ हिर॑ण्य मु॒पास्य॑ जुहोति । \newline
6. उ॒पास्य॑ जुहोति जुहो त्यु॒पा स्यो॒पास्य॑ जुहो त्यग्नि॒व त्य॑ग्नि॒वति॑ जुहो त्यु॒पा स्यो॒पास्य॑ जुहो त्यग्नि॒वति॑ । \newline
7. उ॒पास्येत्यु॑प - अस्य॑ । \newline
8. जु॒हो॒ त्य॒ग्नि॒व त्य॑ग्नि॒वति॑ जुहोति जुहो त्यग्नि॒व त्ये॒वैवा ग्नि॒वति॑ जुहोति जुहो त्यग्नि॒व त्ये॒व । \newline
9. अ॒ग्नि॒व त्ये॒वैवा ग्नि॒व त्य॑ग्नि॒व त्ये॒व जु॑होति जुहो त्ये॒वा ग्नि॒व त्य॑ग्नि॒व त्ये॒व जु॑होति । \newline
10. अ॒ग्नि॒वतीत्य॑ग्नि - वति॑ । \newline
11. ए॒व जु॑होति जुहो त्ये॒वैव जु॑होति॒ न न जु॑हो त्ये॒वैव जु॑होति॒ न । \newline
12. जु॒हो॒ति॒ न न जु॑होति जुहोति॒ नान्धो᳚ ऽन्धो न जु॑होति जुहोति॒ नान्धः । \newline
13. नान्धो᳚ ऽन्धो न नान्धो᳚ ऽद्ध्व॒र्यु र॑द्ध्व॒र्यु र॒न्धो न नान्धो᳚ ऽद्ध्व॒र्युः । \newline
14. अ॒न्धो᳚ ऽद्ध्व॒र्यु र॑द्ध्व॒र्यु र॒न्धो᳚(1॒) ऽन्धो᳚ ऽद्ध्व॒र्युर् भव॑ति॒ भव॑ त्यद्ध्व॒र्यु र॒न्धो᳚(1॒) ऽन्धो᳚ ऽद्ध्व॒र्युर् भव॑ति । \newline
15. अ॒द्ध्व॒र्युर् भव॑ति॒ भव॑ त्यद्ध्व॒र्यु र॑द्ध्व॒र्युर् भव॑ति॒ न न भव॑ त्यद्ध्व॒र्यु र॑द्ध्व॒र्युर् भव॑ति॒ न । \newline
16. भव॑ति॒ न न भव॑ति॒ भव॑ति॒ न य॒ज्ञ्ं ॅय॒ज्ञ्म् न भव॑ति॒ भव॑ति॒ न य॒ज्ञ्म् । \newline
17. न य॒ज्ञ्ं ॅय॒ज्ञ्म् न न य॒ज्ञ्ꣳ रक्षाꣳ॑सि॒ रक्षाꣳ॑सि य॒ज्ञ्म् न न य॒ज्ञ्ꣳ रक्षाꣳ॑सि । \newline
18. य॒ज्ञ्ꣳ रक्षाꣳ॑सि॒ रक्षाꣳ॑सि य॒ज्ञ्ं ॅय॒ज्ञ्ꣳ रक्षाꣳ॑सि घ्नन्ति घ्नन्ति॒ रक्षाꣳ॑सि य॒ज्ञ्ं ॅय॒ज्ञ्ꣳ रक्षाꣳ॑सि घ्नन्ति । \newline
19. रक्षाꣳ॑सि घ्नन्ति घ्नन्ति॒ रक्षाꣳ॑सि॒ रक्षाꣳ॑सि घ्नन्ति॒ जिघ॑र्मि॒ जिघ॑र्मि घ्नन्ति॒ रक्षाꣳ॑सि॒ रक्षाꣳ॑सि घ्नन्ति॒ जिघ॑र्मि । \newline
20. घ्न॒न्ति॒ जिघ॑र्मि॒ जिघ॑र्मि घ्नन्ति घ्नन्ति॒ जिघ॑र् म्य॒ग्नि म॒ग्निम् जिघ॑र्मि घ्नन्ति घ्नन्ति॒ जिघ॑र् म्य॒ग्निम् । \newline
21. जिघ॑र् म्य॒ग्नि म॒ग्निम् जिघ॑र्मि॒ जिघ॑र् म्य॒ग्निम् मन॑सा॒ मन॑सा॒ ऽग्निम् जिघ॑र्मि॒ जिघ॑र् म्य॒ग्निम् मन॑सा । \newline
22. अ॒ग्निम् मन॑सा॒ मन॑सा॒ ऽग्नि म॒ग्निम् मन॑सा घृ॒तेन॑ घृ॒तेन॒ मन॑सा॒ ऽग्नि म॒ग्निम् मन॑सा घृ॒तेन॑ । \newline
23. मन॑सा घृ॒तेन॑ घृ॒तेन॒ मन॑सा॒ मन॑सा घृ॒तेने तीति॑ घृ॒तेन॒ मन॑सा॒ मन॑सा घृ॒तेनेति॑ । \newline
24. घृ॒तेने तीति॑ घृ॒तेन॑ घृ॒तेने त्या॑हा॒हेति॑ घृ॒तेन॑ घृ॒तेने त्या॑ह । \newline
25. इत्या॑हा॒हे तीत्या॑ह॒ मन॑सा॒ मन॑सा॒ ऽऽहे तीत्या॑ह॒ मन॑सा । \newline
26. आ॒ह॒ मन॑सा॒ मन॑सा ऽऽहाह॒ मन॑सा॒ हि हि मन॑सा ऽऽहाह॒ मन॑सा॒ हि । \newline
27. मन॑सा॒ हि हि मन॑सा॒ मन॑सा॒ हि पुरु॑षः॒ पुरु॑षो॒ हि मन॑सा॒ मन॑सा॒ हि पुरु॑षः । \newline
28. हि पुरु॑षः॒ पुरु॑षो॒ हि हि पुरु॑षो य॒ज्ञ्ं ॅय॒ज्ञ्म् पुरु॑षो॒ हि हि पुरु॑षो य॒ज्ञ्म् । \newline
29. पुरु॑षो य॒ज्ञ्ं ॅय॒ज्ञ्म् पुरु॑षः॒ पुरु॑षो य॒ज्ञ् म॑भि॒गच्छ॑ त्यभि॒गच्छ॑ति य॒ज्ञ्म् पुरु॑षः॒ पुरु॑षो य॒ज्ञ् म॑भि॒गच्छ॑ति । \newline
30. य॒ज्ञ् म॑भि॒गच्छ॑ त्यभि॒गच्छ॑ति य॒ज्ञ्ं ॅय॒ज्ञ् म॑भि॒गच्छ॑ति प्रति॒क्ष्यन्त॑म् प्रति॒क्ष्यन्त॑ मभि॒गच्छ॑ति य॒ज्ञ्ं ॅय॒ज्ञ् म॑भि॒गच्छ॑ति प्रति॒क्ष्यन्त᳚म् । \newline
31. अ॒भि॒गच्छ॑ति प्रति॒क्ष्यन्त॑म् प्रति॒क्ष्यन्त॑ मभि॒गच्छ॑ त्यभि॒गच्छ॑ति प्रति॒क्ष्यन्त॒म् भुव॑नानि॒ भुव॑नानि प्रति॒क्ष्यन्त॑ मभि॒गच्छ॑ त्यभि॒गच्छ॑ति प्रति॒क्ष्यन्त॒म् भुव॑नानि । \newline
32. अ॒भि॒गच्छ॒तीत्य॑भि - गच्छ॑ति । \newline
33. प्र॒ति॒क्ष्यन्त॒म् भुव॑नानि॒ भुव॑नानि प्रति॒क्ष्यन्त॑म् प्रति॒क्ष्यन्त॒म् भुव॑नानि॒ विश्वा॒ विश्वा॒ भुव॑नानि प्रति॒क्ष्यन्त॑म् प्रति॒क्ष्यन्त॒म् भुव॑नानि॒ विश्वा᳚ । \newline
34. प्र॒ति॒क्ष्यन्त॒मिति॑ प्रति - क्ष्यन्त᳚म् । \newline
35. भुव॑नानि॒ विश्वा॒ विश्वा॒ भुव॑नानि॒ भुव॑नानि॒ विश्वेतीति॒ विश्वा॒ भुव॑नानि॒ भुव॑नानि॒ विश्वेति॑ । \newline
36. विश्वेतीति॒ विश्वा॒ विश्वे त्या॑हा॒हेति॒ विश्वा॒ विश्वेत्या॑ह । \newline
37. इत्या॑हा॒हे तीत्या॑ह॒ सर्वꣳ॒॒ सर्व॑ मा॒हे तीत्या॑ह॒ सर्व᳚म् । \newline
38. आ॒ह॒ सर्वꣳ॒॒ सर्व॑ माहाह॒ सर्वꣳ॒॒ हि हि सर्व॑ माहाह॒ सर्वꣳ॒॒ हि । \newline
39. सर्वꣳ॒॒ हि हि सर्वꣳ॒॒ सर्वꣳ॒॒ ह्ये॑ष ए॒ष हि सर्वꣳ॒॒ सर्वꣳ॒॒ ह्ये॑षः । \newline
40. ह्ये॑ष ए॒ष हि ह्ये॑ष प्र॒त्यङ् प्र॒त्यङ् ङे॒ष हि ह्ये॑ष प्र॒त्यङ् । \newline
41. ए॒ष प्र॒त्यङ् प्र॒त्यङ् ङे॒ष ए॒ष प्र॒त्यङ् क्षेति॒ क्षेति॑ प्र॒त्यङ् ङे॒ष ए॒ष प्र॒त्यङ् क्षेति॑ । \newline
42. प्र॒त्यङ् क्षेति॒ क्षेति॑ प्र॒त्यङ् प्र॒त्यङ् क्षेति॑ पृ॒थुम् पृ॒थुम् क्षेति॑ प्र॒त्यङ् प्र॒त्यङ् क्षेति॑ पृ॒थुम् । \newline
43. क्षेति॑ पृ॒थुम् पृ॒थुम् क्षेति॒ क्षेति॑ पृ॒थुम् ति॑र॒श्चा ति॑र॒श्चा पृ॒थुम् क्षेति॒ क्षेति॑ पृ॒थुम् ति॑र॒श्चा । \newline
44. पृ॒थुम् ति॑र॒श्चा ति॑र॒श्चा पृ॒थुम् पृ॒थुम् ति॑र॒श्चा वय॑सा॒ वय॑सा तिर॒श्चा पृ॒थुम् पृ॒थुम् ति॑र॒श्चा वय॑सा । \newline
45. ति॒र॒श्चा वय॑सा॒ वय॑सा तिर॒श्चा ति॑र॒श्चा वय॑सा बृ॒हन्त॑म् बृ॒हन्तं॒ ॅवय॑सा तिर॒श्चा ति॑र॒श्चा वय॑सा बृ॒हन्त᳚म् । \newline
46. वय॑सा बृ॒हन्त॑म् बृ॒हन्तं॒ ॅवय॑सा॒ वय॑सा बृ॒हन्त॒ मितीति॑ बृ॒हन्तं॒ ॅवय॑सा॒ वय॑सा बृ॒हन्त॒ मिति॑ । \newline
47. बृ॒हन्त॒ मितीति॑ बृ॒हन्त॑म् बृ॒हन्त॒ मित्या॑हा॒हेति॑ बृ॒हन्त॑म् बृ॒हन्त॒ मित्या॑ह । \newline
48. इत्या॑हा॒हेती त्या॒हाल्पो ऽल्प॑ आ॒हेती त्या॒हाल्पः॑ । \newline
49. आ॒हाल्पो ऽल्प॑ आहा॒हाल्पो॒ हि ह्यल्प॑ आहा॒हाल्पो॒ हि । \newline
50. अल्पो॒ हि ह्यल्पो ऽल्पो॒ ह्ये॑ष ए॒ष ह्यल्पो ऽल्पो॒ ह्ये॑षः । \newline
51. ह्ये॑ष ए॒ष हि ह्ये॑ष जा॒तो जा॒त ए॒ष हि ह्ये॑ष जा॒तः । \newline
52. ए॒ष जा॒तो जा॒त ए॒ष ए॒ष जा॒तो म॒हान् म॒हान् जा॒त ए॒ष ए॒ष जा॒तो म॒हान् । \newline
53. जा॒तो म॒हान् म॒हान् जा॒तो जा॒तो म॒हान् भव॑ति॒ भव॑ति म॒हान् जा॒तो जा॒तो म॒हान् भव॑ति । \newline
54. म॒हान् भव॑ति॒ भव॑ति म॒हान् म॒हान् भव॑ति॒ व्यचि॑ष्ठं॒ ॅव्यचि॑ष्ठ॒म् भव॑ति म॒हान् म॒हान् भव॑ति॒ व्यचि॑ष्ठम् । \newline
\pagebreak
\markright{ TS 5.1.3.3  \hfill https://www.vedavms.in \hfill}

\section{ TS 5.1.3.3 }

\textbf{TS 5.1.3.3 } \newline
\textbf{Samhita Paata} \newline

भव॑ति॒ व्यचि॑ष्ठ॒मन्नꣳ॑ रभ॒सं ॅविदा॑न॒मित्या॒हा ऽन्न॑मे॒वाऽस्मै᳚ स्वदयति॒ सर्व॑मस्मै स्वदते॒ य ए॒वं ॅवेदा ऽऽ*त्वा॑ जिघर्मि॒ वच॑सा घृ॒तेनेत्या॑ह॒ तस्मा॒द्-यत् पुरु॑षो॒ मन॑सा-ऽभि॒गच्छ॑ति॒ तद्-वा॒चा व॑दत्य र॒क्षसेत्या॑ह॒ रक्ष॑सा॒मप॑हत्यै॒ मर्य॑श्रीः स्पृह॒यद्-व॑र्णो अ॒ग्निरित्या॒हा-प॑चितिमे॒वा-ऽस्मि॑न् दधा॒त्य-प॑चितिमान् भवति॒ य ए॒वं - [  ] \newline

\textbf{Pada Paata} \newline

भव॑ति । व्यचि॑ष्ठम् । अन्न᳚म् । र॒भ॒सम् । विदा॑नम् । इति॑ । आ॒ह॒ । अन्न᳚म् । ए॒व । अ॒स्मै॒ । स्व॒द॒य॒ति॒ । सर्व᳚म् । अ॒स्मै॒ । स्व॒द॒ते॒ । यः । ए॒वम् । वेद॑ । एति॑ । त्वा॒ । जि॒घ॒र्मि॒ । वच॑सा । घृ॒तेन॑ । इति॑ । आ॒ह॒ । तस्मा᳚त् । यत् । पुरु॑षः । मन॑सा । अ॒भि॒गच्छ॒तीत्य॑भि - गच्छ॑ति । तत् । वा॒चा । व॒द॒ति॒ । अ॒र॒क्षसा᳚ । इति॑ । आ॒ह॒ । रक्ष॑साम् । अप॑हत्या॒ इत्यप॑ - ह॒त्यै॒ । मर्य॑श्री॒रिति॒ मर्य॑ - श्रीः॒ । स्पृ॒ह॒यद्व॑र्ण॒ इति॑ स्पृह॒यत्-व॒र्णः॒ । अ॒ग्निः । इति॑ । आ॒ह॒ । अप॑चिति॒मित्यप॑-चि॒ति॒म् । ए॒व । अ॒स्मि॒न्न् । द॒धा॒ति॒ । अप॑चितिमा॒नित्यप॑चिति - मा॒न् । भ॒व॒ति॒ । यः । ए॒वम् ।  \newline


\textbf{Krama Paata} \newline

भव॑ति॒ व्यचि॑ष्ठम् । व्यचि॑ष्ठ॒मन्न᳚म् । अन्नꣳ॑ रभ॒सम् । र॒भ॒सम् ॅविदा॑नम् । विदा॑न॒मिति॑ । इत्या॑ह । आ॒हान्न᳚म् । अन्न॑मे॒व । ए॒वास्मै᳚ । अ॒स्मै॒ स्व॒द॒य॒ति॒ । स्व॒द॒य॒ति॒ सर्व᳚म् । सर्व॑मस्मै । अ॒स्मै॒ स्व॒द॒ते॒ । स्व॒द॒ते॒ यः । य ए॒वम् । ए॒वम् ॅवेद॑ । वेदा । आ त्वा᳚ । त्वा॒ जि॒घ॒र्मि॒ । जि॒घ॒र्मि॒ वच॑सा । वच॑सा घृ॒तेन॑ । घृ॒तेनेति॑ । इत्या॑ह । आ॒ह॒ तस्मा᳚त् । तस्मा॒द् यत् । यत् पुरु॑षः । पुरु॑षो॒ मन॑सा । मन॑साऽभि॒गच्छ॑ति । अ॒भि॒गच्छ॑ति॒ तत् । अ॒भि॒गच्छ॒तीत्य॑भि - गच्छ॑ति । तद् वा॒चा । वा॒चा व॑दति । व॒द॒त्य॒र॒क्षसा᳚ । अ॒र॒क्षसेति॑ । इत्या॑ह । आ॒ह॒ रक्ष॑साम् । रक्ष॑सा॒मप॑हत्यै । अप॑हत्यै॒ मर्य॑श्रीः । अप॑हत्या॒ इत्यप॑ - ह॒त्यै॒ । मर्य॑श्रीः स्पृह॒यद्व॑र्णः । मर्य॑श्री॒रिति॒ मर्य॑ - श्रीः॒ । स्पृ॒ह॒यद्व॑र्णो अ॒ग्निः । स्पृ॒ह॒यद्व॑र्ण॒ इति॑ स्पृह॒यत् - व॒र्णः॒ । अ॒ग्निरिति॑ । इत्या॑ह । आ॒हाप॑चितिम् । अप॑चितिमे॒व । अप॑चिति॒मित्यप॑ - चि॒ति॒म् । ए॒वास्मिन्न्॑ । अ॒स्मि॒न् द॒धा॒ति॒ । द॒धा॒त्यप॑चितिमान् । अप॑चितिमान् भवति । अप॑चितिमा॒नित्यप॑चिति - मा॒न्॒ । भ॒व॒ति॒ यः । य ए॒वम् । ए॒वम् ॅवेद॑ \newline

\textbf{Jatai Paata} \newline

1. भव॑ति॒ व्यचि॑ष्ठं॒ ॅव्यचि॑ष्ठ॒म् भव॑ति॒ भव॑ति॒ व्यचि॑ष्ठम् । \newline
2. व्यचि॑ष्ठ॒ मन्न॒ मन्नं॒ ॅव्यचि॑ष्ठं॒ ॅव्यचि॑ष्ठ॒ मन्न᳚म् । \newline
3. अन्नꣳ॑ रभ॒सꣳ र॑भ॒स मन्न॒ मन्नꣳ॑ रभ॒सम् । \newline
4. र॒भ॒सं ॅविदा॑नं॒ ॅविदा॑नꣳ रभ॒सꣳ र॑भ॒सं ॅविदा॑नम् । \newline
5. विदा॑न॒ मितीति॒ विदा॑नं॒ ॅविदा॑न॒ मिति॑ । \newline
6. इत्या॑हा॒हे तीत्या॑ह । \newline
7. आ॒हान्न॒ मन्न॑ माहा॒ हान्न᳚म् । \newline
8. अन्न॑ मे॒वै वान्न॒ मन्न॑ मे॒व । \newline
9. ए॒वास्मा॑ अस्मा ए॒वैवास्मै᳚ । \newline
10. अ॒स्मै॒ स्व॒द॒य॒ति॒ स्व॒द॒य॒ त्य॒स्मा॒ अ॒स्मै॒ स्व॒द॒य॒ति॒ । \newline
11. स्व॒द॒य॒ति॒ सर्वꣳ॒॒ सर्वꣳ॑ स्वदयति स्वदयति॒ सर्व᳚म् । \newline
12. सर्व॑ मस्मा अस्मै॒ सर्वꣳ॒॒ सर्व॑ मस्मै । \newline
13. अ॒स्मै॒ स्व॒द॒ते॒ स्व॒द॒ते॒ ऽस्मा॒ अ॒स्मै॒ स्व॒द॒ते॒ । \newline
14. स्व॒द॒ते॒ यो यः स्व॑दते स्वदते॒ यः । \newline
15. य ए॒व मे॒वं ॅयो य ए॒वम् । \newline
16. ए॒वं ॅवेद॒ वेदै॒व मे॒वं ॅवेद॑ । \newline
17. वेदा वेद॒ वेदा । \newline
18. आ त्वा॒ त्वा ऽऽत्वा᳚ । \newline
19. त्वा॒ जि॒घ॒र्मि॒ जि॒घ॒र्मि॒ त्वा॒ त्वा॒ जि॒घ॒र्मि॒ । \newline
20. जि॒घ॒र्मि॒ वच॑सा॒ वच॑सा जिघर्मि जिघर्मि॒ वच॑सा । \newline
21. वच॑सा घृ॒तेन॑ घृ॒तेन॒ वच॑सा॒ वच॑सा घृ॒तेन॑ । \newline
22. घृ॒तेने तीति॑ घृ॒तेन॑ घृ॒तेनेति॑ । \newline
23. इत्या॑हा॒हे तीत्या॑ह । \newline
24. आ॒ह॒ तस्मा॒त् तस्मा॑ दाहाह॒ तस्मा᳚त् । \newline
25. तस्मा॒द् यद् यत् तस्मा॒त् तस्मा॒द् यत् । \newline
26. यत् पुरु॑षः॒ पुरु॑षो॒ यद् यत् पुरु॑षः । \newline
27. पुरु॑षो॒ मन॑सा॒ मन॑सा॒ पुरु॑षः॒ पुरु॑षो॒ मन॑सा । \newline
28. मन॑सा ऽभि॒गच्छ॑ त्यभि॒गच्छ॑ति॒ मन॑सा॒ मन॑सा ऽभि॒गच्छ॑ति । \newline
29. अ॒भि॒गच्छ॑ति॒ तत् तद॑भि॒गच्छ॑ त्यभि॒गच्छ॑ति॒ तत् । \newline
30. अ॒भि॒गच्छ॒तीत्य॑भि - गच्छ॑ति । \newline
31. तद् वा॒चा वा॒चा तत् तद् वा॒चा । \newline
32. वा॒चा व॑दति वदति वा॒चा वा॒चा व॑दति । \newline
33. व॒द॒ त्य॒र॒क्षसा॑ ऽर॒क्षसा॑ वदति वद त्यर॒क्षसा᳚ । \newline
34. अ॒र॒क्षसेती त्य॑र॒क्षसा॑ ऽर॒क्षसेति॑ । \newline
35. इत्या॑हा॒हे तीत्या॑ह । \newline
36. आ॒ह॒ रक्ष॑साꣳ॒॒ रक्ष॑सा माहाह॒ रक्ष॑साम् । \newline
37. रक्ष॑सा॒ मप॑हत्या॒ अप॑हत्यै॒ रक्ष॑साꣳ॒॒ रक्ष॑सा॒ मप॑हत्यै । \newline
38. अप॑हत्यै॒ मर्य॑श्री॒र् मर्य॑श्री॒ रप॑हत्या॒ अप॑हत्यै॒ मर्य॑श्रीः । \newline
39. अप॑हत्या॒ इत्यप॑ - ह॒त्यै॒ । \newline
40. मर्य॑श्रीः स्पृह॒यद्व॑र्णः स्पृह॒यद्व॑र्णो॒ मर्य॑श्री॒र् मर्य॑श्रीः स्पृह॒यद्व॑र्णः । \newline
41. मर्य॑श्री॒रिति॒ मर्य॑ - श्रीः॒ । \newline
42. स्पृ॒ह॒यद्व॑र्णो अ॒ग्नि र॒ग्निः स्पृ॑ह॒यद्व॑र्णः स्पृह॒यद्व॑र्णो अ॒ग्निः । \newline
43. स्पृ॒ह॒यद्व॑र्ण॒ इति॑ स्पृह॒यत् - व॒र्णः॒ । \newline
44. अ॒ग्निरिती त्य॒ग्नि र॒ग्नि रिति॑ । \newline
45. इत्या॑हा॒हे तीत्या॑ह । \newline
46. आ॒हा प॑चिति॒ मप॑चिति माहा॒हा प॑चितिम् । \newline
47. अप॑चिति मे॒वैवा प॑चिति॒ मप॑चिति मे॒व । \newline
48. अप॑चिति॒मित्यप॑ - चि॒ति॒म् । \newline
49. ए॒वास्मि॑न् नस्मिन् ने॒वैवास्मिन्न्॑ । \newline
50. अ॒स्मि॒न् द॒धा॒ति॒ द॒धा॒ त्य॒स्मि॒न् न॒स्मि॒न् द॒धा॒ति॒ । \newline
51. द॒धा॒ त्यप॑चितिमा॒ नप॑चितिमान् दधाति दधा॒ त्यप॑चितिमान् । \newline
52. अप॑चितिमान् भवति भव॒ त्यप॑चितिमा॒ नप॑चितिमान् भवति । \newline
53. अप॑चितिमा॒नित्यप॑चिति - मा॒न् । \newline
54. भ॒व॒ति॒ यो यो भ॑वति भवति॒ यः । \newline
55. य ए॒व मे॒वं ॅयो य ए॒वम् । \newline
56. ए॒वं ॅवेद॒ वेदै॒व मे॒वं ॅवेद॑ । \newline

\textbf{Ghana Paata } \newline

1. भव॑ति॒ व्यचि॑ष्ठं॒ ॅव्यचि॑ष्ठ॒म् भव॑ति॒ भव॑ति॒ व्यचि॑ष्ठ॒ मन्न॒ मन्नं॒ ॅव्यचि॑ष्ठ॒म् भव॑ति॒ भव॑ति॒ व्यचि॑ष्ठ॒ मन्न᳚म् । \newline
2. व्यचि॑ष्ठ॒ मन्न॒ मन्नं॒ ॅव्यचि॑ष्ठं॒ ॅव्यचि॑ष्ठ॒ मन्नꣳ॑ रभ॒सꣳ र॑भ॒स मन्नं॒ ॅव्यचि॑ष्ठं॒ ॅव्यचि॑ष्ठ॒ मन्नꣳ॑ रभ॒सम् । \newline
3. अन्नꣳ॑ रभ॒सꣳ र॑भ॒स मन्न॒ मन्नꣳ॑ रभ॒सं ॅविदा॑नं॒ ॅविदा॑नꣳ रभ॒स मन्न॒ मन्नꣳ॑ रभ॒सं ॅविदा॑नम् । \newline
4. र॒भ॒सं ॅविदा॑नं॒ ॅविदा॑नꣳ रभ॒सꣳ र॑भ॒सं ॅविदा॑न॒ मितीति॒ विदा॑नꣳ रभ॒सꣳ र॑भ॒सं ॅविदा॑न॒ मिति॑ । \newline
5. विदा॑न॒ मितीति॒ विदा॑नं॒ ॅविदा॑न॒ मित्या॑ हा॒हेति॒ विदा॑नं॒ ॅविदा॑न॒ मित्या॑ह । \newline
6. इत्या॑हा॒हेती त्या॒हान्न॒ मन्न॑ मा॒हेती त्या॒हान्न᳚म् । \newline
7. आ॒हान्न॒ मन्न॑ माहा॒ हान्न॑ मे॒वैवान्न॑ माहा॒ हान्न॑ मे॒व । \newline
8. अन्न॑ मे॒वैवान्न॒ मन्न॑ मे॒वास्मा॑ अस्मा ए॒वान्न॒ मन्न॑ मे॒वास्मै᳚ । \newline
9. ए॒वास्मा॑ अस्मा ए॒वैवास्मै᳚ स्वदयति स्वदय त्यस्मा ए॒वैवास्मै᳚ स्वदयति । \newline
10. अ॒स्मै॒ स्व॒द॒य॒ति॒ स्व॒द॒य॒ त्य॒स्मा॒ अ॒स्मै॒ स्व॒द॒य॒ति॒ सर्वꣳ॒॒ सर्वꣳ॑ स्वदय त्यस्मा अस्मै स्वदयति॒ सर्व᳚म् । \newline
11. स्व॒द॒य॒ति॒ सर्वꣳ॒॒ सर्वꣳ॑ स्वदयति स्वदयति॒ सर्व॑ मस्मा अस्मै॒ सर्वꣳ॑ स्वदयति स्वदयति॒ सर्व॑ मस्मै । \newline
12. सर्व॑ मस्मा अस्मै॒ सर्वꣳ॒॒ सर्व॑ मस्मै स्वदते स्वदते ऽस्मै॒ सर्वꣳ॒॒ सर्व॑ मस्मै स्वदते । \newline
13. अ॒स्मै॒ स्व॒द॒ते॒ स्व॒द॒ते॒ ऽस्मा॒ अ॒स्मै॒ स्व॒द॒ते॒ यो यः स्व॑दते ऽस्मा अस्मै स्वदते॒ यः । \newline
14. स्व॒द॒ते॒ यो यः स्व॑दते स्वदते॒ य ए॒व मे॒वं ॅयः स्व॑दते स्वदते॒ य ए॒वम् । \newline
15. य ए॒व मे॒वं ॅयो य ए॒वं ॅवेद॒ वेदै॒वं ॅयो य ए॒वं ॅवेद॑ । \newline
16. ए॒वं ॅवेद॒ वेदै॒व मे॒वं ॅवेदा वेदै॒व मे॒वं ॅवेदा । \newline
17. वेदा वेद॒ वेदा त्वा॒ त्वा ऽऽवेद॒ वेदा त्वा᳚ । \newline
18. आ त्वा॒ त्वा ऽऽत्वा॑ जिघर्मि जिघर्मि॒ त्वा ऽऽत्वा॑ जिघर्मि । \newline
19. त्वा॒ जि॒घ॒र्मि॒ जि॒घ॒र्मि॒ त्वा॒ त्वा॒ जि॒घ॒र्मि॒ वच॑सा॒ वच॑सा जिघर्मि त्वा त्वा जिघर्मि॒ वच॑सा । \newline
20. जि॒घ॒र्मि॒ वच॑सा॒ वच॑सा जिघर्मि जिघर्मि॒ वच॑सा घृ॒तेन॑ घृ॒तेन॒ वच॑सा जिघर्मि जिघर्मि॒ वच॑सा घृ॒तेन॑ । \newline
21. वच॑सा घृ॒तेन॑ घृ॒तेन॒ वच॑सा॒ वच॑सा घृ॒तेने तीति॑ घृ॒तेन॒ वच॑सा॒ वच॑सा घृ॒तेनेति॑ । \newline
22. घृ॒तेने तीति॑ घृ॒तेन॑ घृ॒तेने त्या॑हा॒हेति॑ घृ॒तेन॑ घृ॒तेने त्या॑ह । \newline
23. इत्या॑ हा॒हेती त्या॑ह॒ तस्मा॒त् तस्मा॑ दा॒हेती त्या॑ह॒ तस्मा᳚त् । \newline
24. आ॒ह॒ तस्मा॒त् तस्मा॑ दाहाह॒ तस्मा॒द् यद् यत् तस्मा॑ दाहाह॒ तस्मा॒द् यत् । \newline
25. तस्मा॒द् यद् यत् तस्मा॒त् तस्मा॒द् यत् पुरु॑षः॒ पुरु॑षो॒ यत् तस्मा॒त् तस्मा॒द् यत् पुरु॑षः । \newline
26. यत् पुरु॑षः॒ पुरु॑षो॒ यद् यत् पुरु॑षो॒ मन॑सा॒ मन॑सा॒ पुरु॑षो॒ यद् यत् पुरु॑षो॒ मन॑सा । \newline
27. पुरु॑षो॒ मन॑सा॒ मन॑सा॒ पुरु॑षः॒ पुरु॑षो॒ मन॑सा ऽभि॒गच्छ॑ त्यभि॒गच्छ॑ति॒ मन॑सा॒ पुरु॑षः॒ पुरु॑षो॒ मन॑सा ऽभि॒गच्छ॑ति । \newline
28. मन॑सा ऽभि॒गच्छ॑ त्यभि॒गच्छ॑ति॒ मन॑सा॒ मन॑सा ऽभि॒गच्छ॑ति॒ तत् तद॑भि॒गच्छ॑ति॒ मन॑सा॒ मन॑सा ऽभि॒गच्छ॑ति॒ तत् । \newline
29. अ॒भि॒गच्छ॑ति॒ तत् तद॑भि॒गच्छ॑ त्यभि॒गच्छ॑ति॒ तद् वा॒चा वा॒चा तद॑भि॒गच्छ॑ त्यभि॒गच्छ॑ति॒ तद् वा॒चा । \newline
30. अ॒भि॒गच्छ॒तीत्य॑भि - गच्छ॑ति । \newline
31. तद् वा॒चा वा॒चा तत् तद् वा॒चा व॑दति वदति वा॒चा तत् तद् वा॒चा व॑दति । \newline
32. वा॒चा व॑दति वदति वा॒चा वा॒चा व॑द त्यर॒क्षसा॑ ऽर॒क्षसा॑ वदति वा॒चा वा॒चा व॑द त्यर॒क्षसा᳚ । \newline
33. व॒द॒ त्य॒र॒क्षसा॑ ऽर॒क्षसा॑ वदति वद त्यर॒क्षसेती त्य॑र॒क्षसा॑ वदति वद त्यर॒क्षसेति॑ । \newline
34. अ॒र॒क्षसेती त्य॑र॒क्षसा॑ ऽर॒क्षसे त्या॑हा॒हे त्य॑र॒क्षसा॑ ऽर॒क्षसे त्या॑ह । \newline
35. इत्या॑ हा॒हेती त्या॑ह॒ रक्ष॑साꣳ॒॒ रक्ष॑सा मा॒हेती त्या॑ह॒ रक्ष॑साम् । \newline
36. आ॒ह॒ रक्ष॑साꣳ॒॒ रक्ष॑सा माहाह॒ रक्ष॑सा॒ मप॑हत्या॒ अप॑हत्यै॒ रक्ष॑सा माहाह॒ रक्ष॑सा॒ मप॑हत्यै । \newline
37. रक्ष॑सा॒ मप॑हत्या॒ अप॑हत्यै॒ रक्ष॑साꣳ॒॒ रक्ष॑सा॒ मप॑हत्यै॒ मर्य॑श्री॒र् मर्य॑श्री॒ रप॑हत्यै॒ रक्ष॑साꣳ॒॒ रक्ष॑सा॒ मप॑हत्यै॒ मर्य॑श्रीः । \newline
38. अप॑हत्यै॒ मर्य॑श्री॒र् मर्य॑श्री॒ रप॑हत्या॒ अप॑हत्यै॒ मर्य॑श्रीः स्पृह॒यद्व॑र्णः स्पृह॒यद्व॑र्णो॒ मर्य॑श्री॒ रप॑हत्या॒ अप॑हत्यै॒ मर्य॑श्रीः स्पृह॒यद्व॑र्णः । \newline
39. अप॑हत्या॒ इत्यप॑ - ह॒त्यै॒ । \newline
40. मर्य॑श्रीः स्पृह॒यद्व॑र्णः स्पृह॒यद्व॑र्णो॒ मर्य॑श्री॒र् मर्य॑श्रीः स्पृह॒यद्व॑र्णो अ॒ग्नि र॒ग्निः स्पृ॑ह॒यद्व॑र्णो॒ मर्य॑श्री॒र् मर्य॑श्रीः स्पृह॒यद्व॑र्णो अ॒ग्निः । \newline
41. मर्य॑श्री॒रिति॒ मर्य॑ - श्रीः॒ । \newline
42. स्पृ॒ह॒यद्व॑र्णो अ॒ग्नि र॒ग्निः स्पृ॑ह॒यद्व॑र्णः स्पृह॒यद्व॑र्णो अ॒ग्नि रिती त्य॒ग्निः स्पृ॑ह॒यद्व॑र्णः स्पृह॒यद्व॑र्णो अ॒ग्नि रिति॑ । \newline
43. स्पृ॒ह॒यद्व॑र्ण॒ इति॑ स्पृह॒यत् - व॒र्णः॒ । \newline
44. अ॒ग्नि रिती त्य॒ग्नि र॒ग्नि रित्या॑हा॒हे त्य॒ग्नि र॒ग्नि रित्या॑ह । \newline
45. इत्या॑हा॒हेती त्या॒हा प॑चिति॒ मप॑चिति मा॒हेती त्या॒हा प॑चितिम् । \newline
46. आ॒हा प॑चिति॒ मप॑चिति माहा॒हा प॑चिति मे॒वैवा प॑चिति माहा॒हा प॑चिति मे॒व । \newline
47. अप॑चिति मे॒वैवा प॑चिति॒ मप॑चिति मे॒वास्मि॑न् नस्मिन् ने॒वा प॑चिति॒ मप॑चिति मे॒वास्मिन्न्॑ । \newline
48. अप॑चिति॒मित्यप॑ - चि॒ति॒म् । \newline
49. ए॒वास्मि॑न् नस्मिन् ने॒वैवास्मि॑न् दधाति दधा त्यस्मिन् ने॒वैवास्मि॑न् दधाति । \newline
50. अ॒स्मि॒न् द॒धा॒ति॒ द॒धा॒ त्य॒स्मि॒न् न॒स्मि॒न् द॒धा॒ त्यप॑चितिमा॒ नप॑चितिमान् दधा त्यस्मिन् नस्मिन् दधा॒ त्यप॑चितिमान् । \newline
51. द॒धा॒ त्यप॑चितिमा॒ नप॑चितिमान् दधाति दधा॒ त्यप॑चितिमान् भवति भव॒ त्यप॑चितिमान् दधाति दधा॒ त्यप॑चितिमान् भवति । \newline
52. अप॑चितिमान् भवति भव॒ त्यप॑चितिमा॒ नप॑चितिमान् भवति॒ यो यो भ॑व॒ त्यप॑चितिमा॒ नप॑चितिमान् भवति॒ यः । \newline
53. अप॑चितिमा॒नित्यप॑चिति - मा॒न् । \newline
54. भ॒व॒ति॒ यो यो भ॑वति भवति॒ य ए॒व मे॒वं ॅयो भ॑वति भवति॒ य ए॒वम् । \newline
55. य ए॒व मे॒वं ॅयो य ए॒वं ॅवेद॒ वेदै॒वं ॅयो य ए॒वं ॅवेद॑ । \newline
56. ए॒वं ॅवेद॒ वेदै॒व मे॒वं ॅवेद॒ मन॑सा॒ मन॑सा॒ वेदै॒व मे॒वं ॅवेद॒ मन॑सा । \newline
\pagebreak
\markright{ TS 5.1.3.4  \hfill https://www.vedavms.in \hfill}

\section{ TS 5.1.3.4 }

\textbf{TS 5.1.3.4 } \newline
\textbf{Samhita Paata} \newline

ॅवेद॒ मन॑सा॒ त्वै तामाप्तु॑मर्.हति॒ याम॑द्ध्व॒र्युर॑-न॒ग्नावाहु॑तिं जु॒होति॒ मन॑स्वतीभ्यां जुहो॒त्याहु॑त्यो॒राप्त्यै॒ द्वाभ्यां॒ प्रति॑ष्ठित्यै यज्ञ्मु॒खे य॑ज्ञ्मुखे॒ वै क्रि॒यमा॑णे य॒ज्ञ्ꣳ रक्षाꣳ॑सि जिघाꣳ सन्त्ये॒तर्.हि॒ खलु॒ वा ए॒तद्-य॑ज्ञ्मु॒खं ॅयर्.ह्ये॑न॒दाहु॑तिरश्नु॒ते परि॑ लिखति॒ रक्ष॑सा॒मप॑हत्यै ति॒सृभिः॒ परि॑ लिखति त्रि॒वृद्वा अ॒ग्निर्यावा॑ने॒वा-ऽग्निस्तस्मा॒द्-रक्षाꣳ॒॒स्यप॑ हन्ति - [  ] \newline

\textbf{Pada Paata} \newline

वेद॑ । मन॑सा । तु । वै । ताम् । आप्तु᳚म् । अ॒र्॒.ह॒ति॒ । याम् । अ॒द्ध्व॒र्युः । अ॒न॒ग्नौ । आहु॑ति॒मित्या - हु॒ति॒म् । जु॒होति॑ । मन॑स्वतीभ्याम् । जु॒हो॒ति॒ । आहु॑त्यो॒रित्या - हू॒त्योः॒ । आप्त्यै᳚ । द्वाभ्या᳚म् । प्रति॑ष्ठित्या॒ इति॒ प्रति॑ - स्थि॒त्यै॒ । य॒ज्ञ्॒मु॒खे य॑ज्ञ्मुख॒ इति॑ यज्ञ्मु॒खे - य॒ज्ञ्॒मु॒खे॒ । वै । क्रि॒यमा॑णे । य॒ज्ञ्म् । रक्षाꣳ॑सि । जि॒घाꣳ॒॒स॒न्ति॒ । ए॒तर्.हि॑ । खलु॑ । वै । ए॒तत् । य॒ज्ञ्॒मु॒खमिति॑ यज्ञ् - मु॒खम् । यर्.हि॑ । ए॒न॒त् । आहु॑ति॒रित्या - हु॒तिः॒ । अ॒श्नु॒ते । परीति॑ । लि॒ख॒ति॒ । रक्ष॑साम् । अप॑हत्या॒ इत्यप॑ - ह॒त्यै॒ । ति॒सृभि॒रिति॑ ति॒सृ - भिः॒ । परीति॑ । लि॒ख॒ति॒ । त्रि॒वृदिति॑ त्रि - वृत् । वै । अ॒ग्निः । यावान्॑ । ए॒व । अ॒ग्निः । तस्मा᳚त् । रक्षाꣳ॑सि । अपेति॑ । ह॒न्ति॒ ।  \newline


\textbf{Krama Paata} \newline

वेद॒ मन॑सा । मन॑सा॒ तु । त्वै । वै ताम् । तामाप्तु᳚म् । आप्तु॑मर्.हति । अ॒र्॒.ह॒ति॒ याम् । याम॑द्ध्व॒र्युः । अ॒द्ध्व॒र्युर॑न॒ग्नौ । अ॒न॒ग्नावाहु॑तिम् । आहु॑तिम् जु॒होति॑ । आहु॑ति॒मित्या - हु॒ति॒म् । जु॒होति॒ मन॑स्वतीभ्याम् । मन॑स्वतीभ्याम् जुहोति । जु॒हो॒त्याहु॑त्योः । आहु॑त्यो॒राप्त्यै᳚ । आहु॑त्यो॒रित्या - हु॒त्योः॒ । आप्त्यै॒ द्वाभ्या᳚म् । द्वाभ्या॒म् प्रति॑ष्ठित्यै । प्रति॑ष्ठित्यै यज्ञ्मु॒खेय॑ज्ञ्मुखे । प्रति॑ष्ठित्या॒ इति॒ प्रति॑ - स्थि॒त्यै॒ । य॒ज्ञ्॒मु॒खे य॑ज्ञ्मुखे॒ वै । य॒ज्ञ्॒मु॒खे य॑ज्ञ्मुख॒ इति॑ यज्ञ्मु॒खे - य॒ज्ञ्॒मु॒खे॒ । वै क्रि॒यमा॑णे । क्रि॒यमा॑णे य॒ज्ञ्म् । य॒ज्ञ्ꣳ रक्षाꣳ॑सि । रक्षाꣳ॑सि जिघाꣳसन्ति । जि॒घाꣳ॒॒स॒न्त्ये॒तर्.हि॑ । ए॒तर्.हि॒ खलु॑ । खलु॒ वै । वा ए॒तत् । ए॒तद् य॑ज्ञ्मु॒खम् । य॒ज्ञ्॒मु॒खम् ॅयर्.हि॑ । य॒ज्ञ्॒मु॒खमिति॑ यज्ञ् - मु॒खम् । यर्.ह्ये॑नत् । ए॒न॒दाहु॑तिः । आहु॑तिरश्ञु॒ते । आहु॑ति॒रित्या - हु॒तिः॒ । अ॒श्ञु॒ते परि॑ । परि॑ लिखति । लि॒ख॒ति॒ रक्ष॑साम् । रक्ष॑सा॒मप॑हत्यै । अप॑हत्यै ति॒सृभिः॑ । अप॑हत्या॒ इत्यप॑ - ह॒त्यै॒ । ति॒सृभिः॒ परि॑ । ति॒सृभि॒रिति॑ ति॒सृ - भिः॒ । परि॑ लिखति । लि॒ख॒ति॒ त्रि॒वृत् । त्रि॒वृद् वै । त्रि॒वृदिति॑ त्रि - वृत् । वा अ॒ग्निः । अ॒ग्निर् यावान्॑ । यावा॑ने॒व । ए॒वाग्निः । अ॒ग्निस्तस्मा᳚त् । तस्मा॒द् रक्षाꣳ॑सि । रक्षाꣳ॒॒स्यप॑ । अप॑ हन्ति । ह॒न्ति॒ गा॒य॒त्रि॒या \newline

\textbf{Jatai Paata} \newline

1. वेद॒ मन॑सा॒ मन॑सा॒ वेद॒ वेद॒ मन॑सा । \newline
2. मन॑सा॒ तु तु मन॑सा॒ मन॑सा॒ तु । \newline
3. त्वै वै तु त्वै । \newline
4. वै ताम् तां ॅवै वै ताम् । \newline
5. ता माप्तु॒ माप्तु॒म् ताम् ता माप्तु᳚म् । \newline
6. आप्तु॑ मर्.ह त्यर्.ह॒ त्याप्तु॒ माप्तु॑ मर्.हति । \newline
7. अ॒र्॒.ह॒ति॒ यां ॅया म॑र्.ह त्यर्.हति॒ याम् । \newline
8. या म॑द्ध्व॒र्यु र॑द्ध्व॒र्युर् यां ॅया म॑द्ध्व॒र्युः । \newline
9. अ॒द्ध्व॒र्यु र॑न॒ग्ना व॑न॒ग्ना व॑द्ध्व॒र्यु र॑द्ध्व॒र्यु र॑न॒ग्नौ । \newline
10. अ॒न॒ग्ना वाहु॑ति॒ माहु॑ति मन॒ग्ना व॑न॒ग्ना वाहु॑तिम् । \newline
11. आहु॑तिम् जु॒होति॑ जु॒हो त्याहु॑ति॒ माहु॑तिम् जु॒होति॑ । \newline
12. आहु॑ति॒मित्या - हु॒ति॒म् । \newline
13. जु॒होति॒ मन॑स्वतीभ्या॒म् मन॑स्वतीभ्याम् जु॒होति॑ जु॒होति॒ मन॑स्वतीभ्याम् । \newline
14. मन॑स्वतीभ्याम् जुहोति जुहोति॒ मन॑स्वतीभ्या॒म् मन॑स्वतीभ्याम् जुहोति । \newline
15. जु॒हो॒ त्याहु॑त्यो॒ राहु॑त्योर् जुहोति जुहो॒ त्याहु॑त्योः । \newline
16. आहु॑त्यो॒ राप्त्या॒ आप्त्या॒ आहु॑त्यो॒ राहु॑त्यो॒ राप्त्यै᳚ । \newline
17. आहु॑त्यो॒रित्या - हु॒त्योः॒ । \newline
18. आप्त्यै॒ द्वाभ्या॒म् द्वाभ्या॒ माप्त्या॒ आप्त्यै॒ द्वाभ्या᳚म् । \newline
19. द्वाभ्या॒म् प्रति॑ष्ठित्यै॒ प्रति॑ष्ठित्यै॒ द्वाभ्या॒म् द्वाभ्या॒म् प्रति॑ष्ठित्यै । \newline
20. प्रति॑ष्ठित्यै यज्ञ्मु॒खेय॑ज्ञ्मुखे यज्ञ्मु॒खेय॑ज्ञ्मुखे॒ प्रति॑ष्ठित्यै॒ प्रति॑ष्ठित्यै यज्ञ्मु॒खेय॑ज्ञ्मुखे । \newline
21. प्रति॑ष्ठित्या॒ इति॒ प्रति॑ - स्थि॒त्यै॒ । \newline
22. य॒ज्ञ्॒मु॒खेय॑ज्ञ्मुखे॒ वै वै य॑ज्ञ्मु॒खेय॑ज्ञ्मुखे यज्ञ्मु॒खेय॑ज्ञ्मुखे॒ वै । \newline
23. य॒ज्ञ्॒मु॒खेय॑ज्ञ्मुख॒ इति॑ यज्ञ्मु॒खे - य॒ज्ञ्॒मु॒खे॒ । \newline
24. वै क्रि॒यमा॑णे क्रि॒यमा॑णे॒ वै वै क्रि॒यमा॑णे । \newline
25. क्रि॒यमा॑णे य॒ज्ञ्ं ॅय॒ज्ञ्म् क्रि॒यमा॑णे क्रि॒यमा॑णे य॒ज्ञ्म् । \newline
26. य॒ज्ञ्ꣳ रक्षाꣳ॑सि॒ रक्षाꣳ॑सि य॒ज्ञ्ं ॅय॒ज्ञ्ꣳ रक्षाꣳ॑सि । \newline
27. रक्षाꣳ॑सि जिघाꣳसन्ति जिघाꣳसन्ति॒ रक्षाꣳ॑सि॒ रक्षाꣳ॑सि जिघाꣳसन्ति । \newline
28. जि॒घाꣳ॒॒स॒ न्त्ये॒तर् ह्ये॒तर्.हि॑ जिघाꣳसन्ति जिघाꣳस न्त्ये॒तर्.हि॑ । \newline
29. ए॒तर्.हि॒ खलु॒ खल्वे॒तर् ह्ये॒तर्.हि॒ खलु॑ । \newline
30. खलु॒ वै वै खलु॒ खलु॒ वै । \newline
31. वा ए॒त दे॒तद् वै वा ए॒तत् । \newline
32. ए॒तद् य॑ज्ञ्मु॒खं ॅय॑ज्ञ्मु॒ख मे॒त दे॒तद् य॑ज्ञ्मु॒खम् । \newline
33. य॒ज्ञ्॒मु॒खं ॅयर्.हि॒ यर्.हि॑ यज्ञ्मु॒खं ॅय॑ज्ञ्मु॒खं ॅयर्.हि॑ । \newline
34. य॒ज्ञ्॒मु॒खमिति॑ यज्ञ् - मु॒खम् । \newline
35. यर्ह्ये॑न देन॒द् यर्.हि॒ यर्ह्ये॑नत् । \newline
36. ए॒न॒ दाहु॑ति॒ राहु॑ति रेन देन॒ दाहु॑तिः । \newline
37. आहु॑ति रश्ञु॒ते᳚ ऽश्ञु॒त आहु॑ति॒ राहु॑ति रश्ञु॒ते । \newline
38. आहु॑ति॒रित्या - हु॒तिः॒ । \newline
39. अ॒श्ञु॒ते परि॒ पर्य॑श्ञु॒ते᳚ ऽश्ञु॒ते परि॑ । \newline
40. परि॑ लिखति लिखति॒ परि॒ परि॑ लिखति । \newline
41. लि॒ख॒ति॒ रक्ष॑साꣳ॒॒ रक्ष॑साम् ॅलिखति लिखति॒ रक्ष॑साम् । \newline
42. रक्ष॑सा॒ मप॑हत्या॒ अप॑हत्यै॒ रक्ष॑साꣳ॒॒ रक्ष॑सा॒ मप॑हत्यै । \newline
43. अप॑हत्यै ति॒सृभि॑ स्ति॒सृभि॒ रप॑हत्या॒ अप॑हत्यै ति॒सृभिः॑ । \newline
44. अप॑हत्या॒ इत्यप॑ - ह॒त्यै॒ । \newline
45. ति॒सृभिः॒ परि॒ परि॑ ति॒सृभि॑ स्ति॒सृभिः॒ परि॑ । \newline
46. ति॒सृभि॒रिति॑ ति॒सृ - भिः॒ । \newline
47. परि॑ लिखति लिखति॒ परि॒ परि॑ लिखति । \newline
48. लि॒ख॒ति॒ त्रि॒वृत् त्रि॒वृ ल्लि॑खति लिखति त्रि॒वृत् । \newline
49. त्रि॒वृद् वै वै त्रि॒वृत् त्रि॒वृद् वै । \newline
50. त्रि॒वृदिति॑ त्रि - वृत् । \newline
51. वा अ॒ग्नि र॒ग्निर् वै वा अ॒ग्निः । \newline
52. अ॒ग्निर् यावा॒न्॒. यावा॑ न॒ग्नि र॒ग्निर् यावान्॑ । \newline
53. यावा॑ ने॒वैव यावा॒न्॒. यावा॑ ने॒व । \newline
54. ए॒वाग्नि र॒ग्नि रे॒वैवाग्निः । \newline
55. अ॒ग्नि स्तस्मा॒त् तस्मा॑ द॒ग्नि र॒ग्नि स्तस्मा᳚त् । \newline
56. तस्मा॒द् रक्षाꣳ॑सि॒ रक्षाꣳ॑सि॒ तस्मा॒त् तस्मा॒द् रक्षाꣳ॑सि । \newline
57. रक्षाꣳ॒॒ स्यपाप॒ रक्षाꣳ॑सि॒ रक्षाꣳ॒॒ स्यप॑ । \newline
58. अप॑ हन्ति ह॒न्त्यपाप॑ हन्ति । \newline
59. ह॒न्ति॒ गा॒य॒त्रि॒या गा॑यत्रि॒या ह॑न्ति हन्ति गायत्रि॒या । \newline

\textbf{Ghana Paata } \newline

1. वेद॒ मन॑सा॒ मन॑सा॒ वेद॒ वेद॒ मन॑सा॒ तु तु मन॑सा॒ वेद॒ वेद॒ मन॑सा॒ तु । \newline
2. मन॑सा॒ तु तु मन॑सा॒ मन॑सा॒ त्वै वै तु मन॑सा॒ मन॑सा॒ त्वै । \newline
3. त्वै वै तु त्वै ताम् तां ॅवै तु त्वै ताम् । \newline
4. वै ताम् तां ॅवै वै ता माप्तु॒ माप्तु॒म् तां ॅवै वै ता माप्तु᳚म् । \newline
5. ता माप्तु॒ माप्तु॒म् ताम् ता माप्तु॑ मर्.ह त्यर्.ह॒ त्याप्तु॒म् ताम् ता माप्तु॑ मर्.हति । \newline
6. आप्तु॑ मर्.ह त्यर्.ह॒ त्याप्तु॒ माप्तु॑ मर्.हति॒ यां ॅया म॑र्.ह॒ त्याप्तु॒ माप्तु॑ मर्.हति॒ याम् । \newline
7. अ॒र्॒.ह॒ति॒ यां ॅया म॑र्.ह त्यर्.हति॒ या म॑द्ध्व॒र्यु र॑द्ध्व॒र्युर् या म॑र्.ह त्यर्.हति॒ या म॑द्ध्व॒र्युः । \newline
8. या म॑द्ध्व॒र्यु र॑द्ध्व॒र्युर् यां ॅया म॑द्ध्व॒र्यु र॑न॒ग्ना व॑न॒ग्ना व॑द्ध्व॒र्युर् यां ॅया म॑द्ध्व॒र्यु र॑न॒ग्नौ । \newline
9. अ॒द्ध्व॒र्यु र॑न॒ग्ना व॑न॒ग्ना व॑द्ध्व॒र्यु र॑द्ध्व॒र्यु र॑न॒ग्ना वाहु॑ति॒ माहु॑ति मन॒ग्ना व॑द्ध्व॒र्यु र॑द्ध्व॒र्यु र॑न॒ग्ना वाहु॑तिम् । \newline
10. अ॒न॒ग्ना वाहु॑ति॒ माहु॑ति मन॒ग्ना व॑न॒ग्ना वाहु॑तिम् जु॒होति॑ जु॒हो त्याहु॑ति मन॒ग्ना व॑न॒ग्ना वाहु॑तिम् जु॒होति॑ । \newline
11. आहु॑तिम् जु॒होति॑ जु॒हो त्याहु॑ति॒ माहु॑तिम् जु॒होति॒ मन॑स्वतीभ्या॒म् मन॑स्वतीभ्याम् जु॒हो त्याहु॑ति॒ माहु॑तिम् जु॒होति॒ मन॑स्वतीभ्याम् । \newline
12. आहु॑ति॒मित्या - हु॒ति॒म् । \newline
13. जु॒होति॒ मन॑स्वतीभ्या॒म् मन॑स्वतीभ्याम् जु॒होति॑ जु॒होति॒ मन॑स्वतीभ्याम् जुहोति जुहोति॒ मन॑स्वतीभ्याम् जु॒होति॑ जु॒होति॒ मन॑स्वतीभ्याम् जुहोति । \newline
14. मन॑स्वतीभ्याम् जुहोति जुहोति॒ मन॑स्वतीभ्या॒म् मन॑स्वतीभ्याम् जुहो॒ त्याहु॑त्यो॒ राहु॑त्योर् जुहोति॒ मन॑स्वतीभ्या॒म् मन॑स्वतीभ्याम् जुहो॒ त्याहु॑त्योः । \newline
15. जु॒हो॒ त्याहु॑त्यो॒ राहु॑त्योर् जुहोति जुहो॒ त्याहु॑त्यो॒ राप्त्या॒ आप्त्या॒ आहु॑त्योर् जुहोति जुहो॒ त्याहु॑त्यो॒ राप्त्यै᳚ । \newline
16. आहु॑त्यो॒ राप्त्या॒ आप्त्या॒ आहु॑त्यो॒ राहु॑त्यो॒ राप्त्यै॒ द्वाभ्या॒म् द्वाभ्या॒ माप्त्या॒ आहु॑त्यो॒ राहु॑त्यो॒ राप्त्यै॒ द्वाभ्या᳚म् । \newline
17. आहु॑त्यो॒रित्या - हु॒त्योः॒ । \newline
18. आप्त्यै॒ द्वाभ्या॒म् द्वाभ्या॒ माप्त्या॒ आप्त्यै॒ द्वाभ्या॒म् प्रति॑ष्ठित्यै॒ प्रति॑ष्ठित्यै॒ द्वाभ्या॒ माप्त्या॒ आप्त्यै॒ द्वाभ्या॒म् प्रति॑ष्ठित्यै । \newline
19. द्वाभ्या॒म् प्रति॑ष्ठित्यै॒ प्रति॑ष्ठित्यै॒ द्वाभ्या॒म् द्वाभ्या॒म् प्रति॑ष्ठित्यै यज्ञ्मु॒खेय॑ज्ञ्मुखे यज्ञ्मु॒खेय॑ज्ञ्मुखे॒ प्रति॑ष्ठित्यै॒ द्वाभ्या॒म् द्वाभ्या॒म् प्रति॑ष्ठित्यै यज्ञ्मु॒खेय॑ज्ञ्मुखे । \newline
20. प्रति॑ष्ठित्यै यज्ञ्मु॒खेय॑ज्ञ्मुखे यज्ञ्मु॒खेय॑ज्ञ्मुखे॒ प्रति॑ष्ठित्यै॒ प्रति॑ष्ठित्यै यज्ञ्मु॒खेय॑ज्ञ्मुखे॒ वै वै य॑ज्ञ्मु॒खेय॑ज्ञ्मुखे॒ प्रति॑ष्ठित्यै॒ प्रति॑ष्ठित्यै यज्ञ्मु॒खेय॑ज्ञ्मुखे॒ वै । \newline
21. प्रति॑ष्ठित्या॒ इति॒ प्रति॑ - स्थि॒त्यै॒ । \newline
22. य॒ज्ञ्॒मु॒खेय॑ज्ञ्मुखे॒ वै वै य॑ज्ञ्मु॒खेय॑ज्ञ्मुखे यज्ञ्मु॒खेय॑ज्ञ्मुखे॒ वै क्रि॒यमा॑णे क्रि॒यमा॑णे॒ वै य॑ज्ञ्मु॒खेय॑ज्ञ्मुखे यज्ञ्मु॒खेय॑ज्ञ्मुखे॒ वै क्रि॒यमा॑णे । \newline
23. य॒ज्ञ्॒मु॒खेय॑ज्ञ्मुख॒ इति॑ यज्ञ्मु॒खे - य॒ज्ञ्॒मु॒खे॒ । \newline
24. वै क्रि॒यमा॑णे क्रि॒यमा॑णे॒ वै वै क्रि॒यमा॑णे य॒ज्ञ्ं ॅय॒ज्ञ्म् क्रि॒यमा॑णे॒ वै वै क्रि॒यमा॑णे य॒ज्ञ्म् । \newline
25. क्रि॒यमा॑णे य॒ज्ञ्ं ॅय॒ज्ञ्म् क्रि॒यमा॑णे क्रि॒यमा॑णे य॒ज्ञ्ꣳ रक्षाꣳ॑सि॒ रक्षाꣳ॑सि य॒ज्ञ्म् क्रि॒यमा॑णे क्रि॒यमा॑णे य॒ज्ञ्ꣳ रक्षाꣳ॑सि । \newline
26. य॒ज्ञ्ꣳ रक्षाꣳ॑सि॒ रक्षाꣳ॑सि य॒ज्ञ्ं ॅय॒ज्ञ्ꣳ रक्षाꣳ॑सि जिघाꣳसन्ति जिघाꣳसन्ति॒ रक्षाꣳ॑सि य॒ज्ञ्ं ॅय॒ज्ञ्ꣳ रक्षाꣳ॑सि जिघाꣳसन्ति । \newline
27. रक्षाꣳ॑सि जिघाꣳसन्ति जिघाꣳसन्ति॒ रक्षाꣳ॑सि॒ रक्षाꣳ॑सि जिघाꣳस न्त्ये॒तर्. ह्ये॒तर्.हि॑ जिघाꣳसन्ति॒ रक्षाꣳ॑सि॒ रक्षाꣳ॑सि जिघाꣳस न्त्ये॒तर्.हि॑ । \newline
28. जि॒घाꣳ॒॒स॒ न्त्ये॒तर्. ह्ये॒तर्.हि॑ जिघाꣳसन्ति जिघाꣳस न्त्ये॒तर्.हि॒ खलु॒ खल्वे॒तर्.हि॑ जिघाꣳसन्ति जिघाꣳसन्त्ये॒तर्.हि॒ खलु॑ । \newline
29. ए॒तर्.हि॒ खलु॒ खल्वे॒तर्. ह्ये॒तर्.हि॒ खलु॒ वै वै खल्वे॒तर्. ह्ये॒तर्.हि॒ खलु॒ वै । \newline
30. खलु॒ वै वै खलु॒ खलु॒ वा ए॒त दे॒तद् वै खलु॒ खलु॒ वा ए॒तत् । \newline
31. वा ए॒त दे॒तद् वै वा ए॒तद् य॑ज्ञ्मु॒खं ॅय॑ज्ञ्मु॒ख मे॒तद् वै वा ए॒तद् य॑ज्ञ्मु॒खम् । \newline
32. ए॒तद् य॑ज्ञ्मु॒खं ॅय॑ज्ञ्मु॒ख मे॒त दे॒तद् य॑ज्ञ्मु॒खं ॅयर्.हि॒ यर्.हि॑ यज्ञ्मु॒ख मे॒त दे॒तद् य॑ज्ञ्मु॒खं ॅयर्.हि॑ । \newline
33. य॒ज्ञ्॒मु॒खं ॅयर्.हि॒ यर्.हि॑ यज्ञ्मु॒खं ॅय॑ज्ञ्मु॒खं ॅयर्ह्ये॑न देन॒द् यर्.हि॑ यज्ञ्मु॒खं ॅय॑ज्ञ्मु॒खं ॅयर्ह्ये॑नत् । \newline
34. य॒ज्ञ्॒मु॒खमिति॑ यज्ञ् - मु॒खम् । \newline
35. यर्ह्ये॑न देन॒द् यर्.हि॒ यर्ह्ये॑न॒ दाहु॑ति॒ राहु॑ति रेन॒द् यर्.हि॒ यर्ह्ये॑न॒ दाहु॑तिः । \newline
36. ए॒न॒ दाहु॑ति॒ राहु॑ति रेन देन॒ दाहु॑ति रश्ञु॒ते᳚ ऽश्ञु॒त आहु॑ति रेन देन॒ दाहु॑ति रश्ञु॒ते । \newline
37. आहु॑ति रश्ञु॒ते᳚ ऽश्ञु॒त आहु॑ति॒ राहु॑ति रश्ञु॒ते परि॒ पर्य॑श्ञु॒त आहु॑ति॒ राहु॑ति रश्ञु॒ते परि॑ । \newline
38. आहु॑ति॒रित्या - हु॒तिः॒ । \newline
39. अ॒श्ञु॒ते परि॒ पर्य॑श्ञु॒ते᳚ ऽश्ञु॒ते परि॑ लिखति लिखति॒ पर्य॑श्ञु॒ते᳚ ऽश्ञु॒ते परि॑ लिखति । \newline
40. परि॑ लिखति लिखति॒ परि॒ परि॑ लिखति॒ रक्ष॑साꣳ॒॒ रक्ष॑साम् ॅलिखति॒ परि॒ परि॑ लिखति॒ रक्ष॑साम् । \newline
41. लि॒ख॒ति॒ रक्ष॑साꣳ॒॒ रक्ष॑साम् ॅलिखति लिखति॒ रक्ष॑सा॒ मप॑हत्या॒ अप॑हत्यै॒ रक्ष॑साम् ॅलिखति लिखति॒ रक्ष॑सा॒ मप॑हत्यै । \newline
42. रक्ष॑सा॒ मप॑हत्या॒ अप॑हत्यै॒ रक्ष॑साꣳ॒॒ रक्ष॑सा॒ मप॑हत्यै ति॒सृभि॑ स्ति॒सृभि॒ रप॑हत्यै॒ रक्ष॑साꣳ॒॒ रक्ष॑सा॒ मप॑हत्यै ति॒सृभिः॑ । \newline
43. अप॑हत्यै ति॒सृभि॑ स्ति॒सृभि॒ रप॑हत्या॒ अप॑हत्यै ति॒सृभिः॒ परि॒ परि॑ ति॒सृभि॒ रप॑हत्या॒ अप॑हत्यै ति॒सृभिः॒ परि॑ । \newline
44. अप॑हत्या॒ इत्यप॑ - ह॒त्यै॒ । \newline
45. ति॒सृभिः॒ परि॒ परि॑ ति॒सृभि॑ स्ति॒सृभिः॒ परि॑ लिखति लिखति॒ परि॑ ति॒सृभि॑ स्ति॒सृभिः॒ परि॑ लिखति । \newline
46. ति॒सृभि॒रिति॑ ति॒सृ - भिः॒ । \newline
47. परि॑ लिखति लिखति॒ परि॒ परि॑ लिखति त्रि॒वृत् त्रि॒वृल् लि॑खति॒ परि॒ परि॑ लिखति त्रि॒वृत् । \newline
48. लि॒ख॒ति॒ त्रि॒वृत् त्रि॒वृल् लि॑खति लिखति त्रि॒वृद् वै वै त्रि॒वृल् लि॑खति लिखति त्रि॒वृद् वै । \newline
49. त्रि॒वृद् वै वै त्रि॒वृत् त्रि॒वृद् वा अ॒ग्नि र॒ग्निर् वै त्रि॒वृत् त्रि॒वृद् वा अ॒ग्निः । \newline
50. त्रि॒वृदिति॑ त्रि - वृत् । \newline
51. वा अ॒ग्नि र॒ग्निर् वै वा अ॒ग्निर् यावा॒न्॒. यावा॑ न॒ग्निर् वै वा अ॒ग्निर् यावान्॑ । \newline
52. अ॒ग्निर् यावा॒न्॒. यावा॑ न॒ग्नि र॒ग्निर् यावा॑ ने॒वैव यावा॑ न॒ग्नि र॒ग्निर् यावा॑ ने॒व । \newline
53. यावा॑ ने॒वैव यावा॒न्॒. यावा॑ ने॒वाग्नि र॒ग्नि रे॒व यावा॒न्॒. यावा॑ ने॒वाग्निः । \newline
54. ए॒वाग्नि र॒ग्नि रे॒वैवाग्नि स्तस्मा॒त् तस्मा॑ द॒ग्नि रे॒वैवाग्नि स्तस्मा᳚त् । \newline
55. अ॒ग्नि स्तस्मा॒त् तस्मा॑ द॒ग्नि र॒ग्नि स्तस्मा॒द् रक्षाꣳ॑सि॒ रक्षाꣳ॑सि॒ तस्मा॑ द॒ग्नि र॒ग्नि स्तस्मा॒द् रक्षाꣳ॑सि । \newline
56. तस्मा॒द् रक्षाꣳ॑सि॒ रक्षाꣳ॑सि॒ तस्मा॒त् तस्मा॒द् रक्षाꣳ॒॒ स्यपाप॒ रक्षाꣳ॑सि॒ तस्मा॒त् तस्मा॒द् रक्षाꣳ॒॒ स्यप॑ । \newline
57. रक्षाꣳ॒॒ स्यपाप॒ रक्षाꣳ॑सि॒ रक्षाꣳ॒॒ स्यप॑ हन्ति ह॒न्त्यप॒ रक्षाꣳ॑सि॒ रक्षाꣳ॒॒ स्यप॑ हन्ति । \newline
58. अप॑ हन्ति ह॒न्त्यपाप॑ हन्ति गायत्रि॒या गा॑यत्रि॒या ह॒न्त्यपाप॑ हन्ति गायत्रि॒या । \newline
59. ह॒न्ति॒ गा॒य॒त्रि॒या गा॑यत्रि॒या ह॑न्ति हन्ति गायत्रि॒या परि॒ परि॑ गायत्रि॒या ह॑न्ति हन्ति गायत्रि॒या परि॑ । \newline
\pagebreak
\markright{ TS 5.1.3.5  \hfill https://www.vedavms.in \hfill}

\section{ TS 5.1.3.5 }

\textbf{TS 5.1.3.5 } \newline
\textbf{Samhita Paata} \newline

गायत्रि॒या परि॑ लिखति॒ तेजो॒ वै गा॑य॒त्री तेज॑सै॒वैनं॒ परि॑गृह्णाति त्रि॒ष्टुभा॒ परि॑ लिखतीन्द्रि॒यं ॅवै त्रि॒ष्टुगि॑न्द्रि॒येणै॒वैनं॒ परि॑ गृह्णात्यनु॒ष्टुभा॒ परि॑ लिखत्यनु॒ष्टुफ् सर्वा॑णि॒ छन्दाꣳ॑सि परि॒भूः पर्या᳚प्त्यै मद्ध्य॒तो॑ऽनु॒ष्टुभा॒ वाग्वा अ॑नु॒ष्टुप् तस्मा᳚न् मद्ध्य॒तो वा॒चा व॑दामो गायत्रि॒या प्र॑थ॒मया॒ परि॑ लिख॒त्यथा॑-ऽनु॒ष्टुभाऽथ॑ त्रि॒ष्टुभा॒ तेजो॒ वै गा॑य॒त्री ( ) य॒ज्ञो॑ ऽनु॒ष्टुगि॑न्द्रि॒यं त्रि॒ष्टुप् तेज॑सा चै॒वेन्द्रि॒येण॑ चोभ॒यतो॑ य॒ज्ञ्ं परि॑ गृह्णाति ॥ \newline

\textbf{Pada Paata} \newline

गा॒य॒त्रि॒या । परीति॑ । लि॒ख॒ति॒ । तेजः॑ । वै । गा॒य॒त्री । तेज॑सा । ए॒व । ए॒न॒म् । परीति॑ । गृ॒ह्णा॒ति॒ । त्रि॒ष्टुभा᳚ । परीति॑ । लि॒ख॒ति॒ । इ॒न्द्रि॒यम् । वै । त्रि॒ष्टुक् । इ॒न्द्रि॒येण॑ । ए॒व । ए॒न॒म् । परीति॑ । गृ॒ह्णा॒ति॒ । अ॒नु॒ष्टुभेत्य॑नु - स्तुभा᳚ । परीति॑ । लि॒ख॒ति॒ । अ॒नु॒ष्टुबित्य॑नु - स्तुप् । सर्वा॑णि । छन्दाꣳ॑सि । प॒रि॒भूरिति॑ परि - भूः । पर्या᳚प्त्या॒ इति॒ परि॑ - आ॒प्त्यै॒ । म॒द्ध्य॒तः । अ॒नु॒ष्टुभेत्य॑नु - स्तुभा᳚ । वाक् । वै । अ॒नु॒ष्टुबित्य॑नु - स्तुप् । तस्मा᳚त् । म॒द्ध्य॒तः । वा॒चा । व॒दा॒मः॒ । गा॒य॒त्रि॒या । प्र॒थ॒मया᳚ । परीति॑ । लि॒ख॒ति॒ । अथ॑ । अ॒नु॒ष्टुभेत्य॑नु - स्तुभा᳚ । अथ॑ । त्रि॒ष्टुभा᳚ । तेजः॑ । वै । गा॒य॒त्री ( ) । य॒ज्ञ्ः । अ॒नु॒ष्टुगित्य॑नु - स्तुक् । इ॒न्द्रि॒यम् । त्रि॒ष्टुप् । तेज॑सा । च॒ । ए॒व । इ॒न्द्रि॒येण॑ । च॒ । उ॒भ॒यतः॑ । य॒ज्ञ्म् । परीति॑ । गृ॒ह्णा॒ति॒ ॥  \newline


\textbf{Krama Paata} \newline

गा॒य॒त्रि॒या परि॑ । परि॑ लिखति । लि॒ख॒ति॒ तेजः॑ । तेजो॒ वै । वै गा॑य॒त्री । गा॒य॒त्री तेज॑सा । तेज॑सै॒व । ए॒वैन᳚म् । ए॒न॒म् परि॑ । परि॑ गृह्णाति । गृ॒ह्णा॒ति॒ त्रि॒ष्टुभा᳚ । त्रि॒ष्टुभा॒ परि॑ । परि॑ लिखति । लि॒ख॒ती॒न्द्रि॒यम् । इ॒न्द्रि॒यम् ॅवै । वै त्रि॒ष्टुक् । त्रि॒ष्टुगि॑न्द्रि॒येण॑ । इ॒न्द्रि॒येणै॒व । ए॒वैन᳚म् । ए॒न॒म् परि॑ । परि॑ गृह्णाति । गृ॒ह्णा॒त्य॒नु॒ष्टुभा᳚ । अ॒नु॒ष्टुभा॒ परि॑ । अ॒नु॒ष्टुभेत्य॑नु - स्तुभा᳚ । परि॑ लिखति । लि॒ख॒त्य॒नु॒ष्टुप् । अ॒नु॒ष्टुफ् सर्वा॑णि । अ॒नु॒ष्टुबित्य॑नु - स्तुप् । सर्वा॑णि॒ छन्दाꣳ॑सि । छन्दाꣳ॑सि परि॒भूः । प॒रि॒भूः पर्या᳚प्त्यै । प॒रि॒भूरिति॑ परि - भूः । पर्या᳚प्त्यै मद्ध्य॒तः । पर्या᳚प्त्या॒ इति॒ परि॑ - आ॒प्त्यै॒ । म॒द्ध्य॒तो॑ऽनु॒ष्टुभा᳚ । अ॒नु॒ष्टुभा॒ वाक् । अ॒नु॒ष्टुभेत्य॑नु - स्तुभा᳚ । वाग् वै । वा अ॑नु॒ष्टुप् । अ॒नु॒ष्टुप् तस्मा᳚त् । अ॒नु॒ष्टुबित्य॑नु - स्तुप् । तस्मा᳚न् मद्ध्य॒तः । म॒द्ध्य॒तो वा॒चा । वा॒चा व॑दामः । व॒दा॒मो॒ गा॒य॒त्रि॒या । गा॒य॒त्रि॒या प्र॑थ॒मया᳚ । प्र॒थ॒मया॒ परि॑ । परि॑ लिखति । लि॒ख॒त्यथ॑ । अथा॑नु॒ष्टुभा᳚ । अ॒नु॒ष्टुभाऽथ॑ । अ॒नु॒ष्टुभेत्य॑नु - स्तुभा᳚ । अथ॑ त्रि॒ष्टुभा᳚ । त्रि॒ष्टुभा॒ तेजः॑ । तेजो॒ वै । वै गा॑य॒त्री ( ) । गा॒य॒त्री य॒ज्ञ्ः । य॒ज्ञो॑ऽनु॒ष्टुक् । अ॒नु॒ष्टुगि॑न्द्रि॒यम् । अ॒नु॒ष्टुगित्य॑नु - स्तुक् । इ॒न्द्रि॒यम् त्रि॒ष्टुप् । त्रि॒ष्टुप् तेज॑सा । तेज॑सा च । चै॒व । ए॒वेन्द्रि॒येण॑ । इ॒न्द्रि॒येण॑ च । चो॒भ॒यतः॑ । उ॒भ॒यतो॑ य॒ज्ञ्म् । य॒ज्ञ्म् परि॑ । परि॑ गृह्णाति । गृ॒ह्णा॒तीति॑ गृह्णाति । \newline

\textbf{Jatai Paata} \newline

1. गा॒य॒त्रि॒या परि॒ परि॑ गायत्रि॒या गा॑यत्रि॒या परि॑ । \newline
2. परि॑ लिखति लिखति॒ परि॒ परि॑ लिखति । \newline
3. लि॒ख॒ति॒ तेज॒ स्तेजो॑ लिखति लिखति॒ तेजः॑ । \newline
4. तेजो॒ वै वै तेज॒ स्तेजो॒ वै । \newline
5. वै गा॑य॒त्री गा॑य॒त्री वै वै गा॑य॒त्री । \newline
6. गा॒य॒त्री तेज॑सा॒ तेज॑सा गाय॒त्री गा॑य॒त्री तेज॑सा । \newline
7. तेज॑ सै॒वैव तेज॑सा॒ तेज॑ सै॒व । \newline
8. ए॒वैन॑ मेन मे॒वैवैन᳚म् । \newline
9. ए॒न॒म् परि॒ पर्ये॑न मेन॒म् परि॑ । \newline
10. परि॑ गृह्णाति गृह्णाति॒ परि॒ परि॑ गृह्णाति । \newline
11. गृ॒ह्णा॒ति॒ त्रि॒ष्टुभा᳚ त्रि॒ष्टुभा॑ गृह्णाति गृह्णाति त्रि॒ष्टुभा᳚ । \newline
12. त्रि॒ष्टुभा॒ परि॒ परि॑ त्रि॒ष्टुभा᳚ त्रि॒ष्टुभा॒ परि॑ । \newline
13. परि॑ लिखति लिखति॒ परि॒ परि॑ लिखति । \newline
14. लि॒ख॒ती॒न्द्रि॒य मि॑न्द्रि॒यम् ॅलि॑खति लिखतीन्द्रि॒यम् । \newline
15. इ॒न्द्रि॒यं ॅवै वा इ॑न्द्रि॒य मि॑न्द्रि॒यं ॅवै । \newline
16. वै त्रि॒ष्टुक् त्रि॒ष्टुग् वै वै त्रि॒ष्टुक् । \newline
17. त्रि॒ष्टु गि॑न्द्रि॒ये णे᳚न्द्रि॒येण॑ त्रि॒ष्टुक् त्रि॒ष्टु गि॑न्द्रि॒येण॑ । \newline
18. इ॒न्द्रि॒ये णै॒वै वेन्द्रि॒ये णे᳚न्द्रि॒ये णै॒व । \newline
19. ए॒वैन॑ मेन मे॒वैवैन᳚म् । \newline
20. ए॒न॒म् परि॒ पर्ये॑न मेन॒म् परि॑ । \newline
21. परि॑ गृह्णाति गृह्णाति॒ परि॒ परि॑ गृह्णाति । \newline
22. गृ॒ह्णा॒ त्य॒नु॒ष्टुभा॑ ऽनु॒ष्टुभा॑ गृह्णाति गृह्णा त्यनु॒ष्टुभा᳚ । \newline
23. अ॒नु॒ष्टुभा॒ परि॒ पर्य॑णु॒ष्टुभा॑ ऽनु॒ष्टुभा॒ परि॑ । \newline
24. अ॒नु॒ष्टुभेत्य॑नु - स्तुभा᳚ । \newline
25. परि॑ लिखति लिखति॒ परि॒ परि॑ लिखति । \newline
26. लि॒ख॒ त्य॒नु॒ष्टु ब॑नु॒ष्टुब् लि॑खति लिख त्यनु॒ष्टुप् । \newline
27. अ॒नु॒ष्टुफ् सर्वा॑णि॒ सर्वा᳚ ण्यनु॒ष्टु ब॑नु॒ष्टुफ् सर्वा॑णि । \newline
28. अ॒नु॒ष्टुबित्य॑नु - स्तुप् । \newline
29. सर्वा॑णि॒ छन्दाꣳ॑सि॒ छन्दाꣳ॑सि॒ सर्वा॑णि॒ सर्वा॑णि॒ छन्दाꣳ॑सि । \newline
30. छन्दाꣳ॑सि परि॒भूः प॑रि॒भू श्छन्दाꣳ॑सि॒ छन्दाꣳ॑सि परि॒भूः । \newline
31. प॒रि॒भूः पर्या᳚प्त्यै॒ पर्या᳚प्त्यै परि॒भूः प॑रि॒भूः पर्या᳚प्त्यै । \newline
32. प॒रि॒भूरिति॑ परि - भूः । \newline
33. पर्या᳚प्त्यै मद्ध्य॒तो म॑द्ध्य॒तः पर्या᳚प्त्यै॒ पर्या᳚प्त्यै मद्ध्य॒तः । \newline
34. पर्या᳚प्त्या॒ इति॒ परि॑ - आ॒प्त्यै॒ । \newline
35. म॒द्ध्य॒तो॑ ऽनु॒ष्टुभा॑ ऽनु॒ष्टुभा॑ मद्ध्य॒तो म॑द्ध्य॒तो॑ ऽनु॒ष्टुभा᳚ । \newline
36. अ॒नु॒ष्टुभा॒ वाग् वाग॑नु॒ष्टुभा॑ ऽनु॒ष्टुभा॒ वाक् । \newline
37. अ॒नु॒ष्टुभेत्य॑नु - स्तुभा᳚ । \newline
38. वाग् वै वै वाग् वाग् वै । \newline
39. वा अ॑नु॒ष्टु ब॑नु॒ष्टुब् वै वा अ॑नु॒ष्टुप् । \newline
40. अ॒नु॒ष्टुप् तस्मा॒त् तस्मा॑ दनु॒ष्टु ब॑नु॒ष्टुप् तस्मा᳚त् । \newline
41. अ॒नु॒ष्टुबित्य॑नु - स्तुप् । \newline
42. तस्मा᳚न् मद्ध्य॒तो म॑द्ध्य॒त स्तस्मा॒त् तस्मा᳚न् मद्ध्य॒तः । \newline
43. म॒द्ध्य॒तो वा॒चा वा॒चा म॑द्ध्य॒तो म॑द्ध्य॒तो वा॒चा । \newline
44. वा॒चा व॑दामो वदामो वा॒चा वा॒चा व॑दामः । \newline
45. व॒दा॒मो॒ गा॒य॒त्रि॒या गा॑यत्रि॒या व॑दामो वदामो गायत्रि॒या । \newline
46. गा॒य॒त्रि॒या प्र॑थ॒मया᳚ प्रथ॒मया॑ गायत्रि॒या गा॑यत्रि॒या प्र॑थ॒मया᳚ । \newline
47. प्र॒थ॒मया॒ परि॒ परि॑ प्रथ॒मया᳚ प्रथ॒मया॒ परि॑ । \newline
48. परि॑ लिखति लिखति॒ परि॒ परि॑ लिखति । \newline
49. लि॒ख॒ त्यथाथ॑ लिखति लिख॒ त्यथ॑ । \newline
50. अथा॑ नु॒ष्टुभा॑ ऽनु॒ष्टुभा ऽथाथा॑ नु॒ष्टुभा᳚ । \newline
51. अ॒नु॒ष्टुभा ऽथाथा॑ नु॒ष्टुभा॑ ऽनु॒ष्टुभा ऽथ॑ । \newline
52. अ॒नु॒ष्टुभेत्य॑नु - स्तुभा᳚ । \newline
53. अथ॑ त्रि॒ष्टुभा᳚ त्रि॒ष्टुभा ऽथाथ॑ त्रि॒ष्टुभा᳚ । \newline
54. त्रि॒ष्टुभा॒ तेज॒ स्तेज॑ स्त्रि॒ष्टुभा᳚ त्रि॒ष्टुभा॒ तेजः॑ । \newline
55. तेजो॒ वै वै तेज॒ स्तेजो॒ वै । \newline
56. वै गा॑य॒त्री गा॑य॒त्री वै वै गा॑य॒त्री । \newline
57. गा॒य॒त्री य॒ज्ञो य॒ज्ञो गा॑य॒त्री गा॑य॒त्री य॒ज्ञ्ः । \newline
58. य॒ज्ञो॑ ऽनु॒ष्टु ग॑नु॒ष्टुग् य॒ज्ञो य॒ज्ञो॑ ऽनु॒ष्टुक् । \newline
59. अ॒नु॒ष्टु गि॑न्द्रि॒य मि॑न्द्रि॒य म॑नु॒ष्टु ग॑नु॒ष्टु गि॑न्द्रि॒यम् । \newline
60. अ॒नु॒ष्टुगित्य॑नु - स्तुक् । \newline
61. इ॒न्द्रि॒यम् त्रि॒ष्टुप् त्रि॒ष्टु बि॑न्द्रि॒य मि॑न्द्रि॒यम् त्रि॒ष्टुप् । \newline
62. त्रि॒ष्टुप् तेज॑सा॒ तेज॑सा त्रि॒ष्टुप् त्रि॒ष्टुप् तेज॑सा । \newline
63. तेज॑सा च च॒ तेज॑सा॒ तेज॑सा च । \newline
64. चै॒वैव च॑ चै॒व । \newline
65. ए॒वेन्द्रि॒ये णे᳚न्द्रि॒ये णै॒वै वेन्द्रि॒येण॑ । \newline
66. इ॒न्द्रि॒येण॑ च चेन्द्रि॒ये णे᳚न्द्रि॒येण॑ च । \newline
67. चो॒भ॒यत॑ उभ॒यत॑श्च चोभ॒यतः॑ । \newline
68. उ॒भ॒यतो॑ य॒ज्ञ्ं ॅय॒ज्ञ् मु॑भ॒यत॑ उभ॒यतो॑ य॒ज्ञ्म् । \newline
69. य॒ज्ञ्म् परि॒ परि॑ य॒ज्ञ्ं ॅय॒ज्ञ्म् परि॑ । \newline
70. परि॑ गृह्णाति गृह्णाति॒ परि॒ परि॑ गृह्णाति । \newline
71. गृ॒ह्णा॒तीति॑ गृह्णाति । \newline

\textbf{Ghana Paata } \newline

1. गा॒य॒त्रि॒या परि॒ परि॑ गायत्रि॒या गा॑यत्रि॒या परि॑ लिखति लिखति॒ परि॑ गायत्रि॒या गा॑यत्रि॒या परि॑ लिखति । \newline
2. परि॑ लिखति लिखति॒ परि॒ परि॑ लिखति॒ तेज॒ स्तेजो॑ लिखति॒ परि॒ परि॑ लिखति॒ तेजः॑ । \newline
3. लि॒ख॒ति॒ तेज॒ स्तेजो॑ लिखति लिखति॒ तेजो॒ वै वै तेजो॑ लिखति लिखति॒ तेजो॒ वै । \newline
4. तेजो॒ वै वै तेज॒ स्तेजो॒ वै गा॑य॒त्री गा॑य॒त्री वै तेज॒ स्तेजो॒ वै गा॑य॒त्री । \newline
5. वै गा॑य॒त्री गा॑य॒त्री वै वै गा॑य॒त्री तेज॑सा॒ तेज॑सा गाय॒त्री वै वै गा॑य॒त्री तेज॑सा । \newline
6. गा॒य॒त्री तेज॑सा॒ तेज॑सा गाय॒त्री गा॑य॒त्री तेज॑सै॒वैव तेज॑सा गाय॒त्री गा॑य॒त्री तेज॑सै॒व । \newline
7. तेज॑सै॒वैव तेज॑सा॒ तेज॑सै॒वैन॑ मेन मे॒व तेज॑सा॒ तेज॑सै॒वैन᳚म् । \newline
8. ए॒वैन॑ मेन मे॒वैवैन॒म् परि॒ पर्ये॑न मे॒वैवैन॒म् परि॑ । \newline
9. ए॒न॒म् परि॒ पर्ये॑न मेन॒म् परि॑ गृह्णाति गृह्णाति॒ पर्ये॑न मेन॒म् परि॑ गृह्णाति । \newline
10. परि॑ गृह्णाति गृह्णाति॒ परि॒ परि॑ गृह्णाति त्रि॒ष्टुभा᳚ त्रि॒ष्टुभा॑ गृह्णाति॒ परि॒ परि॑ गृह्णाति त्रि॒ष्टुभा᳚ । \newline
11. गृ॒ह्णा॒ति॒ त्रि॒ष्टुभा᳚ त्रि॒ष्टुभा॑ गृह्णाति गृह्णाति त्रि॒ष्टुभा॒ परि॒ परि॑ त्रि॒ष्टुभा॑ गृह्णाति गृह्णाति त्रि॒ष्टुभा॒ परि॑ । \newline
12. त्रि॒ष्टुभा॒ परि॒ परि॑ त्रि॒ष्टुभा᳚ त्रि॒ष्टुभा॒ परि॑ लिखति लिखति॒ परि॑ त्रि॒ष्टुभा᳚ त्रि॒ष्टुभा॒ परि॑ लिखति । \newline
13. परि॑ लिखति लिखति॒ परि॒ परि॑ लिखतीन्द्रि॒य मि॑न्द्रि॒यम् ॅलि॑खति॒ परि॒ परि॑ लिखतीन्द्रि॒यम् । \newline
14. लि॒ख॒ती॒न्द्रि॒य मि॑न्द्रि॒यम् ॅलि॑खति लिखतीन्द्रि॒यं ॅवै वा इ॑न्द्रि॒यम् ॅलि॑खति लिखतीन्द्रि॒यं ॅवै । \newline
15. इ॒न्द्रि॒यं ॅवै वा इ॑न्द्रि॒य मि॑न्द्रि॒यं ॅवै त्रि॒ष्टुक् त्रि॒ष्टुग् वा इ॑न्द्रि॒य मि॑न्द्रि॒यं ॅवै त्रि॒ष्टुक् । \newline
16. वै त्रि॒ष्टुक् त्रि॒ष्टुग् वै वै त्रि॒ष्टु गि॑न्द्रि॒येणे᳚ न्द्रि॒येण॑ त्रि॒ष्टुग् वै वै त्रि॒ष्टु गि॑न्द्रि॒येण॑ । \newline
17. त्रि॒ष्टु गि॑न्द्रि॒ये णे᳚न्द्रि॒येण॑ त्रि॒ष्टुक् त्रि॒ष्टु गि॑न्द्रि॒ येणै॒वै वेन्द्रि॒येण॑ त्रि॒ष्टुक् त्रि॒ष्टु गि॑न्द्रि॒येणै॒व । \newline
18. इ॒न्द्रि॒ये णै॒वैवे न्द्रि॒ये णे᳚न्द्रि॒ये णै॒वैन॑ मेन मे॒वेन्द्रि॒ये णे᳚न्द्रि॒ये णै॒वैन᳚म् । \newline
19. ए॒वैन॑ मेन मे॒वैवैन॒म् परि॒ पर्ये॑न मे॒वैवैन॒म् परि॑ । \newline
20. ए॒न॒म् परि॒ पर्ये॑न मेन॒म् परि॑ गृह्णाति गृह्णाति॒ पर्ये॑न मेन॒म् परि॑ गृह्णाति । \newline
21. परि॑ गृह्णाति गृह्णाति॒ परि॒ परि॑ गृह्णा त्यनु॒ष्टुभा॑ ऽनु॒ष्टुभा॑ गृह्णाति॒ परि॒ परि॑ गृह्णा त्यनु॒ष्टुभा᳚ । \newline
22. गृ॒ह्णा॒ त्य॒नु॒ष्टुभा॑ ऽनु॒ष्टुभा॑ गृह्णाति गृह्णा त्यनु॒ष्टुभा॒ परि॒ पर्य॑णु॒ष्टुभा॑ गृह्णाति गृह्णा त्यनु॒ष्टुभा॒ परि॑ । \newline
23. अ॒नु॒ष्टुभा॒ परि॒ पर्य॑ णु॒ष्टुभा॑ ऽनु॒ष्टुभा॒ परि॑ लिखति लिखति॒ पर्य॑ णु॒ष्टुभा॑ ऽनु॒ष्टुभा॒ परि॑ लिखति । \newline
24. अ॒नु॒ष्टुभेत्य॑नु - स्तुभा᳚ । \newline
25. परि॑ लिखति लिखति॒ परि॒ परि॑ लिख त्यनु॒ष्टु ब॑नु॒ष्टुब् लि॑खति॒ परि॒ परि॑ लिख त्यनु॒ष्टुप् । \newline
26. लि॒ख॒ त्य॒नु॒ष्टु ब॑नु॒ष्टुब् लि॑खति लिख त्यनु॒ष्टुफ् सर्वा॑णि॒ सर्वा᳚ ण्यनु॒ष्टुब् लि॑खति लिख त्यनु॒ष्टुफ् सर्वा॑णि । \newline
27. अ॒नु॒ष्टुफ् सर्वा॑णि॒ सर्वा᳚ ण्यनु॒ष्टु ब॑नु॒ष्टुफ् सर्वा॑णि॒ छन्दाꣳ॑सि॒ छन्दाꣳ॑सि॒ सर्वा᳚ण्यनु॒ष्टु ब॑नु॒ष्टुफ् सर्वा॑णि॒ छन्दाꣳ॑सि । \newline
28. अ॒नु॒ष्टुबित्य॑नु - स्तुप् । \newline
29. सर्वा॑णि॒ छन्दाꣳ॑सि॒ छन्दाꣳ॑सि॒ सर्वा॑णि॒ सर्वा॑णि॒ छन्दाꣳ॑सि परि॒भूः प॑रि॒भू श्छन्दाꣳ॑सि॒ सर्वा॑णि॒ सर्वा॑णि॒ छन्दाꣳ॑सि परि॒भूः । \newline
30. छन्दाꣳ॑सि परि॒भूः प॑रि॒भू श्छन्दाꣳ॑सि॒ छन्दाꣳ॑सि परि॒भूः पर्या᳚प्त्यै॒ पर्या᳚प्त्यै परि॒भू श्छन्दाꣳ॑सि॒ छन्दाꣳ॑सि परि॒भूः पर्या᳚प्त्यै । \newline
31. प॒रि॒भूः पर्या᳚प्त्यै॒ पर्या᳚प्त्यै परि॒भूः प॑रि॒भूः पर्या᳚प्त्यै मद्ध्य॒तो म॑द्ध्य॒तः पर्या᳚प्त्यै परि॒भूः प॑रि॒भूः पर्या᳚प्त्यै मद्ध्य॒तः । \newline
32. प॒रि॒भूरिति॑ परि - भूः । \newline
33. पर्या᳚प्त्यै मद्ध्य॒तो म॑द्ध्य॒तः पर्या᳚प्त्यै॒ पर्या᳚प्त्यै मद्ध्य॒तो॑ ऽनु॒ष्टुभा॑ ऽनु॒ष्टुभा॑ मद्ध्य॒तः पर्या᳚प्त्यै॒ पर्या᳚प्त्यै मद्ध्य॒तो॑ ऽनु॒ष्टुभा᳚ । \newline
34. पर्या᳚प्त्या॒ इति॒ परि॑ - आ॒प्त्यै॒ । \newline
35. म॒द्ध्य॒तो॑ ऽनु॒ष्टुभा॑ ऽनु॒ष्टुभा॑ मद्ध्य॒तो म॑द्ध्य॒तो॑ ऽनु॒ष्टुभा॒ वाग् वाग॑नु॒ष्टुभा॑ मद्ध्य॒तो म॑द्ध्य॒तो॑ ऽनु॒ष्टुभा॒ वाक् । \newline
36. अ॒नु॒ष्टुभा॒ वाग् वाग॑नु॒ष्टुभा॑ ऽनु॒ष्टुभा॒ वाग् वै वै वाग॑नु॒ष्टुभा॑ ऽनु॒ष्टुभा॒ वाग् वै । \newline
37. अ॒नु॒ष्टुभेत्य॑नु - स्तुभा᳚ । \newline
38. वाग् वै वै वाग् वाग् वा अ॑नु॒ष्टु ब॑नु॒ष्टुब् वै वाग् वाग् वा अ॑नु॒ष्टुप् । \newline
39. वा अ॑नु॒ष्टु ब॑नु॒ष्टुब् वै वा अ॑नु॒ष्टुप् तस्मा॒त् तस्मा॑ दनु॒ष्टुब् वै वा अ॑नु॒ष्टुप् तस्मा᳚त् । \newline
40. अ॒नु॒ष्टुप् तस्मा॒त् तस्मा॑ दनु॒ष्टु ब॑नु॒ष्टुप् तस्मा᳚न् मद्ध्य॒तो म॑द्ध्य॒त स्तस्मा॑ दनु॒ष्टु ब॑नु॒ष्टुप् तस्मा᳚न् मद्ध्य॒तः । \newline
41. अ॒नु॒ष्टुबित्य॑नु - स्तुप् । \newline
42. तस्मा᳚न् मद्ध्य॒तो म॑द्ध्य॒त स्तस्मा॒त् तस्मा᳚न् मद्ध्य॒तो वा॒चा वा॒चा म॑द्ध्य॒त स्तस्मा॒त् तस्मा᳚न् मद्ध्य॒तो वा॒चा । \newline
43. म॒द्ध्य॒तो वा॒चा वा॒चा म॑द्ध्य॒तो म॑द्ध्य॒तो वा॒चा व॑दामो वदामो वा॒चा म॑द्ध्य॒तो म॑द्ध्य॒तो वा॒चा व॑दामः । \newline
44. वा॒चा व॑दामो वदामो वा॒चा वा॒चा व॑दामो गायत्रि॒या गा॑यत्रि॒या व॑दामो वा॒चा वा॒चा व॑दामो गायत्रि॒या । \newline
45. व॒दा॒मो॒ गा॒य॒त्रि॒या गा॑यत्रि॒या व॑दामो वदामो गायत्रि॒या प्र॑थ॒मया᳚ प्रथ॒मया॑ गायत्रि॒या व॑दामो वदामो गायत्रि॒या प्र॑थ॒मया᳚ । \newline
46. गा॒य॒त्रि॒या प्र॑थ॒मया᳚ प्रथ॒मया॑ गायत्रि॒या गा॑यत्रि॒या प्र॑थ॒मया॒ परि॒ परि॑ प्रथ॒मया॑ गायत्रि॒या गा॑यत्रि॒या प्र॑थ॒मया॒ परि॑ । \newline
47. प्र॒थ॒मया॒ परि॒ परि॑ प्रथ॒मया᳚ प्रथ॒मया॒ परि॑ लिखति लिखति॒ परि॑ प्रथ॒मया᳚ प्रथ॒मया॒ परि॑ लिखति । \newline
48. परि॑ लिखति लिखति॒ परि॒ परि॑ लिख॒ त्यथाथ॑ लिखति॒ परि॒ परि॑ लिख॒ त्यथ॑ । \newline
49. लि॒ख॒ त्यथाथ॑ लिखति लिख॒ त्यथा॑ नु॒ष्टुभा॑ ऽनु॒ष्टुभा ऽथ॑ लिखति लिख॒ त्यथा॑ नु॒ष्टुभा᳚ । \newline
50. अथा॑ नु॒ष्टुभा॑ ऽनु॒ष्टुभा ऽथाथा॑ नु॒ष्टुभा ऽथाथा॑ नु॒ष्टुभा ऽथाथा॑ नु॒ष्टुभा ऽथ॑ । \newline
51. अ॒नु॒ष्टुभा ऽथाथा॑ नु॒ष्टुभा॑ ऽनु॒ष्टुभा ऽथ॑ त्रि॒ष्टुभा᳚ त्रि॒ष्टुभा ऽथा॑ नु॒ष्टुभा॑ ऽनु॒ष्टुभा ऽथ॑ त्रि॒ष्टुभा᳚ । \newline
52. अ॒नु॒ष्टुभेत्य॑नु - स्तुभा᳚ । \newline
53. अथ॑ त्रि॒ष्टुभा᳚ त्रि॒ष्टुभा ऽथाथ॑ त्रि॒ष्टुभा॒ तेज॒ स्तेज॑ स्त्रि॒ष्टुभा ऽथाथ॑ त्रि॒ष्टुभा॒ तेजः॑ । \newline
54. त्रि॒ष्टुभा॒ तेज॒ स्तेज॑ स्त्रि॒ष्टुभा᳚ त्रि॒ष्टुभा॒ तेजो॒ वै वै तेज॑ स्त्रि॒ष्टुभा᳚ त्रि॒ष्टुभा॒ तेजो॒ वै । \newline
55. तेजो॒ वै वै तेज॒ स्तेजो॒ वै गा॑य॒त्री गा॑य॒त्री वै तेज॒ स्तेजो॒ वै गा॑य॒त्री । \newline
56. वै गा॑य॒त्री गा॑य॒त्री वै वै गा॑य॒त्री य॒ज्ञो य॒ज्ञो गा॑य॒त्री वै वै गा॑य॒त्री य॒ज्ञ्ः । \newline
57. गा॒य॒त्री य॒ज्ञो य॒ज्ञो गा॑य॒त्री गा॑य॒त्री य॒ज्ञो॑ ऽनु॒ष्टु ग॑नु॒ष्टुग् य॒ज्ञो गा॑य॒त्री गा॑य॒त्री य॒ज्ञो॑ ऽनु॒ष्टुक् । \newline
58. य॒ज्ञो॑ ऽनु॒ष्टु ग॑नु॒ष्टुग् य॒ज्ञो य॒ज्ञो॑ ऽनु॒ष्टु गि॑न्द्रि॒य मि॑न्द्रि॒य म॑नु॒ष्टुग् य॒ज्ञो य॒ज्ञो॑ ऽनु॒ष्टु गि॑न्द्रि॒यम् । \newline
59. अ॒नु॒ष्टु गि॑न्द्रि॒य मि॑न्द्रि॒य म॑नु॒ष्टु ग॑नु॒ष्टु गि॑न्द्रि॒यम् त्रि॒ष्टुप् त्रि॒ष्टु बि॑न्द्रि॒य म॑नु॒ष्टु ग॑नु॒ष्टु गि॑न्द्रि॒यम् त्रि॒ष्टुप् । \newline
60. अ॒नु॒ष्टुगित्य॑नु - स्तुक् । \newline
61. इ॒न्द्रि॒यम् त्रि॒ष्टुप् त्रि॒ष्टु बि॑न्द्रि॒य मि॑न्द्रि॒यम् त्रि॒ष्टुप् तेज॑सा॒ तेज॑सा त्रि॒ष्टु बि॑न्द्रि॒य मि॑न्द्रि॒यम् त्रि॒ष्टुप् तेज॑सा । \newline
62. त्रि॒ष्टुप् तेज॑सा॒ तेज॑सा त्रि॒ष्टुप् त्रि॒ष्टुप् तेज॑सा च च॒ तेज॑सा त्रि॒ष्टुप् त्रि॒ष्टुप् तेज॑सा च । \newline
63. तेज॑सा च च॒ तेज॑सा॒ तेज॑सा चै॒वैव च॒ तेज॑सा॒ तेज॑सा चै॒व । \newline
64. चै॒वैव च॑ चै॒वेन्द्रि॒ये णे᳚न्द्रि॒ये णै॒व च॑ चै॒वेन्द्रि॒येण॑ । \newline
65. ए॒वेन्द्रि॒ये णे᳚न्द्रि॒ये णै॒वै वेन्द्रि॒येण॑ च चेन्द्रि॒ये णै॒वै वेन्द्रि॒येण॑ च । \newline
66. इ॒न्द्रि॒येण॑ च चेन्द्रि॒ये णे᳚न्द्रि॒येण॑ चोभ॒यत॑ उभ॒यत॑ श्चेन्द्रि॒ये णे᳚न्द्रि॒येण॑ चोभ॒यतः॑ । \newline
67. चो॒भ॒यत॑ उभ॒यत॑श्च चोभ॒यतो॑ य॒ज्ञ्ं ॅय॒ज्ञ् मु॑भ॒यत॑श्च चोभ॒यतो॑ य॒ज्ञ्म् । \newline
68. उ॒भ॒यतो॑ य॒ज्ञ्ं ॅय॒ज्ञ् मु॑भ॒यत॑ उभ॒यतो॑ य॒ज्ञ्म् परि॒ परि॑ य॒ज्ञ् मु॑भ॒यत॑ उभ॒यतो॑ य॒ज्ञ्म् परि॑ । \newline
69. य॒ज्ञ्म् परि॒ परि॑ य॒ज्ञ्ं ॅय॒ज्ञ्म् परि॑ गृह्णाति गृह्णाति॒ परि॑ य॒ज्ञ्ं ॅय॒ज्ञ्म् परि॑ गृह्णाति । \newline
70. परि॑ गृह्णाति गृह्णाति॒ परि॒ परि॑ गृह्णाति । \newline
71. गृ॒ह्णा॒तीति॑ गृह्णाति । \newline
\pagebreak
\markright{ TS 5.1.4.1  \hfill https://www.vedavms.in \hfill}

\section{ TS 5.1.4.1 }

\textbf{TS 5.1.4.1 } \newline
\textbf{Samhita Paata} \newline

दे॒वस्य॑ त्वा सवि॒तुः प्र॑स॒व इति॑ खनति॒ प्रसू᳚त्या॒ अथो॑ धू॒ममे॒वैतेन॑ जनयति॒ ज्योति॑ष्मन्तं त्वाऽग्ने सु॒प्रती॑क॒मित्या॑ह॒ ज्योति॑रे॒वैतेन॑ जनयति॒ सो᳚ऽग्निर्जा॒तः प्र॒जाः शु॒चाऽऽर्प॑य॒त् तं दे॒वा अ॑र्द्ध॒र्चेना॑-शमयच्छि॒वं प्र॒जाभ्योऽहिꣳ॑ सन्त॒मित्या॑ह प्र॒जाभ्य॑ ए॒वैनꣳ॑ शमयति॒ द्वाभ्यां᳚ खनति॒ प्रति॑ष्ठित्या अ॒पां पृ॒ष्ठम॒सीति॑ पुष्करप॒र्णमा - [  ] \newline

\textbf{Pada Paata} \newline

दे॒वस्य॑ । त्वा॒ । स॒वि॒तुः । प्र॒स॒व इति॑ प्र - स॒वे । इति॑ । ख॒न॒ति॒ । प्रसू᳚त्या॒ इति॒ प्र - सू॒त्यै॒ । अथो॒ इति॑ । धू॒मम् । ए॒व । ए॒तेन॑ । ज॒न॒य॒ति॒ । ज्योति॑ष्मन्तम् । त्वा॒ । अ॒ग्ने॒ । सु॒प्रती॑क॒मिति॑ सु - प्रती॑कम् । इति॑ । आ॒ह॒ । ज्योतिः॑ । ए॒व । ए॒तेन॑ । ज॒न॒य॒ति॒ । सः । अ॒ग्निः । जा॒तः । प्र॒जा इति॑ प्र - जाः । शु॒चा । आ॒र्प॒य॒त् । तम् । दे॒वाः । अ॒द्‌र्ध॒र्चेनेत्य॒॑द्‌र्ध - ऋ॒चेन॑ । अ॒श॒म॒य॒न्न् । शि॒वम् । प्र॒जाभ्य॒ इति॑ प्र - जाभ्यः॑ । अहिꣳ॑सन्तम् । इति॑ । आ॒ह॒ । प्र॒जाभ्य॒ इति॑ प्र - जाभ्यः॑ । ए॒व । ए॒न॒म् । श॒म॒य॒ति॒ । द्वाभ्या᳚म् । ख॒न॒ति॒ । प्रति॑ष्ठित्या॒ इति॒ प्रति॑ - स्थि॒त्यै॒ । अ॒पाम् । पृ॒ष्ठम् । अ॒सि॒ । इति॑ । पु॒ष्क॒र॒प॒र्णमिति॑ पुष्कर - प॒र्णम् । एति॑ ।  \newline


\textbf{Krama Paata} \newline

दे॒वस्य॑ त्वा । त्वा॒ स॒वि॒तुः । स॒वि॒तुः प्र॑स॒वे । प्र॒स॒व इति॑ । प्र॒स॒व इति॑ प्र - स॒वे । इति॑ खनति । ख॒न॒ति॒ प्रसू᳚त्यै । प्रसू᳚त्या॒ अथो᳚ । प्रसू᳚त्या॒ इति॒ प्र - सू॒त्यै॒ । अथो॑ धू॒मम् । अथो॒ इत्यथो᳚ । धू॒ममे॒व । ए॒वैतेन॑ । ए॒तेन॑ जनयति । ज॒न॒य॒ति॒ ज्योति॑ष्मन्तम् । ज्योति॑ष्मन्तम् त्वा । त्वा॒ऽग्ने॒ । अ॒ग्ने॒ सु॒प्रती॑कम् । सु॒प्रती॑क॒मिति॑ । सु॒प्रती॑क॒मिति॑ सु - प्रती॑कम् । इत्या॑ह । आ॒ह॒ ज्योतिः॑ । ज्योति॑रे॒व । ए॒वैतेन॑ । ए॒तेन॑ जनयति । ज॒न॒य॒ति॒ सः । सो᳚ऽग्निः । अ॒ग्निर् जा॒तः । जा॒तः प्र॒जाः । प्र॒जाः शु॒चा । प्र॒जा इति॑ प्र - जाः । शु॒चाऽऽर्प॑यत् । आ॒र्प॒य॒त् तम् । तम् दे॒वाः । दे॒वा अ॑र्द्ध॒र्चेन॑ । अ॒र्द्ध॒र्चेना॑शमयन्न् । अ॒र्द्ध॒र्चेनेत्य॑र्द्ध - ऋ॒चेन॑ । अ॒श॒म॒य॒ञ्छि॒वम् । शि॒वम् प्र॒जाभ्यः॑ । प्र॒जाभ्योऽहिꣳ॑सन्तम् । प्र॒जाभ्य॒ इति॑ प्र - जाभ्यः॑ । अहिꣳ॑सन्त॒मिति॑ । इत्या॑ह । आ॒ह॒ प्र॒जाभ्यः॑ । प्र॒जाभ्य॑ ए॒व । प्र॒जाभ्य॒ इति॑ प्र - जाभ्यः॑ । ए॒वैन᳚म् । ए॒नꣳ॒॒ श॒म॒य॒ति॒ । श॒म॒य॒ति॒ द्वाभ्या᳚म् । द्वाभ्या᳚म् खनति । ख॒न॒ति॒ प्रति॑ष्ठित्यै । प्रति॑ष्ठित्या अ॒पाम् । प्रति॑ष्ठित्या॒ इति॒ प्रति॑ - स्थि॒त्यै॒ । अ॒पाम् पृ॒ष्ठम् । पृ॒ष्ठम॑सि । अ॒सीति॑ । इति॑ पुष्करप॒र्णम् । पु॒ष्क॒र॒प॒र्णमा । पु॒ष्क॒र॒प॒र्णमिति॑ पुष्कर - प॒र्णम् । आ ह॑रति \newline

\textbf{Jatai Paata} \newline

1. दे॒वस्य॑ त्वा त्वा दे॒वस्य॑ दे॒वस्य॑ त्वा । \newline
2. त्वा॒ स॒वि॒तुः स॑वि॒तु स्त्वा᳚ त्वा सवि॒तुः । \newline
3. स॒वि॒तुः प्र॑स॒वे प्र॑स॒वे स॑वि॒तुः स॑वि॒तुः प्र॑स॒वे । \newline
4. प्र॒स॒व इतीति॑ प्रस॒वे प्र॑स॒व इति॑ । \newline
5. प्र॒स॒व इति॑ प्र - स॒वे । \newline
6. इति॑ खनति खन॒तीतीति॑ खनति । \newline
7. ख॒न॒ति॒ प्रसू᳚त्यै॒ प्रसू᳚त्यै खनति खनति॒ प्रसू᳚त्यै । \newline
8. प्रसू᳚त्या॒ अथो॒ अथो॒ प्रसू᳚त्यै॒ प्रसू᳚त्या॒ अथो᳚ । \newline
9. प्रसू᳚त्या॒ इति॒ प्र - सू॒त्यै॒ । \newline
10. अथो॑ धू॒मम् धू॒म मथो॒ अथो॑ धू॒मम् । \newline
11. अथो॒ इत्यथो᳚ । \newline
12. धू॒म मे॒वैव धू॒मम् धू॒म मे॒व । \newline
13. ए॒वैते नै॒ते नै॒वै वैतेन॑ । \newline
14. ए॒तेन॑ जनयति जनय त्ये॒ते नै॒तेन॑ जनयति । \newline
15. ज॒न॒य॒ति॒ ज्योति॑ष्मन्त॒म् ज्योति॑ष्मन्तम् जनयति जनयति॒ ज्योति॑ष्मन्तम् । \newline
16. ज्योति॑ष्मन्तम् त्वा त्वा॒ ज्योति॑ष्मन्त॒म् ज्योति॑ष्मन्तम् त्वा । \newline
17. त्वा॒ ऽग्ने॒ अ॒ग्ने॒ त्वा॒ त्वा॒ ऽग्ने॒ । \newline
18. अ॒ग्ने॒ सु॒प्रती॑कꣳ सु॒प्रती॑क मग्ने अग्ने सु॒प्रती॑कम् । \newline
19. सु॒प्रती॑क॒ मितीति॑ सु॒प्रती॑कꣳ सु॒प्रती॑क॒ मिति॑ । \newline
20. सु॒प्रती॑क॒मिति॑ सु - प्रती॑कम् । \newline
21. इत्या॑हा॒हे तीत्या॑ह । \newline
22. आ॒ह॒ ज्योति॒र् ज्योति॑ राहाह॒ ज्योतिः॑ । \newline
23. ज्योति॑ रे॒वैव ज्योति॒र् ज्योति॑ रे॒व । \newline
24. ए॒वैते नै॒ते नै॒वै वैतेन॑ । \newline
25. ए॒तेन॑ जनयति जनय त्ये॒ते नै॒तेन॑ जनयति । \newline
26. ज॒न॒य॒ति॒ स स ज॑नयति जनयति॒ सः । \newline
27. सो᳚ ऽग्नि र॒ग्निः स सो᳚ ऽग्निः । \newline
28. अ॒ग्निर् जा॒तो जा॒तो᳚ ऽग्नि र॒ग्निर् जा॒तः । \newline
29. जा॒तः प्र॒जाः प्र॒जा जा॒तो जा॒तः प्र॒जाः । \newline
30. प्र॒जाः शु॒चा शु॒चा प्र॒जाः प्र॒जाः शु॒चा । \newline
31. प्र॒जा इति॑ प्र - जाः । \newline
32. शु॒चा ऽऽर्प॑य दार्पय च्छु॒चा शु॒चा ऽऽर्प॑यत् । \newline
33. आ॒र्प॒य॒त् तम् त मा᳚र्पय दार्पय॒त् तम् । \newline
34. तम् दे॒वा दे॒वा स्तम् तम् दे॒वाः । \newline
35. दे॒वा अ॑र्द्ध॒र्चे ना᳚र्द्ध॒र्चेन॑ दे॒वा दे॒वा अ॑र्द्ध॒र्चेन॑ । \newline
36. अ॒र्द्ध॒र्चे ना॑शमयन् नशमयन् नर्द्ध॒र्चेना᳚ र्द्ध॒र्चेना॑ शमयन्न् । \newline
37. अ॒र्द्ध॒र्चेनेत्य॑र्द्ध - ऋ॒चेन॑ । \newline
38. अ॒श॒म॒य॒ञ् छि॒वꣳ शि॒व म॑शमयन् नशमयञ् छि॒वम् । \newline
39. शि॒वम् प्र॒जाभ्यः॑ प्र॒जाभ्यः॑ शि॒वꣳ शि॒वम् प्र॒जाभ्यः॑ । \newline
40. प्र॒जाभ्यो ऽहिꣳ॑सन्त॒ महिꣳ॑सन्तम् प्र॒जाभ्यः॑ प्र॒जाभ्यो ऽहिꣳ॑सन्तम् । \newline
41. प्र॒जाभ्य॒ इति॑ प्र - जाभ्यः॑ । \newline
42. अहिꣳ॑सन्त॒ मिती त्यहिꣳ॑सन्त॒ महिꣳ॑सन्त॒ मिति॑ । \newline
43. इत्या॑हा॒हे तीत्या॑ह । \newline
44. आ॒ह॒ प्र॒जाभ्यः॑ प्र॒जाभ्य॑ आहाह प्र॒जाभ्यः॑ । \newline
45. प्र॒जाभ्य॑ ए॒वैव प्र॒जाभ्यः॑ प्र॒जाभ्य॑ ए॒व । \newline
46. प्र॒जाभ्य॒ इति॑ प्र - जाभ्यः॑ । \newline
47. ए॒वैन॑ मेन मे॒वैवैन᳚म् । \newline
48. ए॒नꣳ॒॒ श॒म॒य॒ति॒ श॒म॒य॒ त्ये॒न॒ मे॒नꣳ॒॒ श॒म॒य॒ति॒ । \newline
49. श॒म॒य॒ति॒ द्वाभ्या॒म् द्वाभ्याꣳ॑ शमयति शमयति॒ द्वाभ्या᳚म् । \newline
50. द्वाभ्या᳚म् खनति खनति॒ द्वाभ्या॒म् द्वाभ्या᳚म् खनति । \newline
51. ख॒न॒ति॒ प्रति॑ष्ठित्यै॒ प्रति॑ष्ठित्यै खनति खनति॒ प्रति॑ष्ठित्यै । \newline
52. प्रति॑ष्ठित्या अ॒पा म॒पाम् प्रति॑ष्ठित्यै॒ प्रति॑ष्ठित्या अ॒पाम् । \newline
53. प्रति॑ष्ठित्या॒ इति॒ प्रति॑ - स्थि॒त्यै॒ । \newline
54. अ॒पाम् पृ॒ष्ठम् पृ॒ष्ठ म॒पा म॒पाम् पृ॒ष्ठम् । \newline
55. पृ॒ष्ठ म॑स्यसि पृ॒ष्ठम् पृ॒ष्ठ म॑सि । \newline
56. अ॒सीती त्य॑स्य॒सीति॑ । \newline
57. इति॑ पुष्करप॒र्णम् पु॑ष्करप॒र्ण मितीति॑ पुष्करप॒र्णम् । \newline
58. पु॒ष्क॒र॒प॒र्ण मा पु॑ष्करप॒र्णम् पु॑ष्करप॒र्ण मा । \newline
59. पु॒ष्क॒र॒प॒र्णमिति॑ पुष्कर - प॒र्णम् । \newline
60. आ ह॑रति हर॒त्या ह॑रति । \newline

\textbf{Ghana Paata } \newline

1. दे॒वस्य॑ त्वा त्वा दे॒वस्य॑ दे॒वस्य॑ त्वा सवि॒तुः स॑वि॒तु स्त्वा॑ दे॒वस्य॑ दे॒वस्य॑ त्वा सवि॒तुः । \newline
2. त्वा॒ स॒वि॒तुः स॑वि॒तु स्त्वा᳚ त्वा सवि॒तुः प्र॑स॒वे प्र॑स॒वे स॑वि॒तु स्त्वा᳚ त्वा सवि॒तुः प्र॑स॒वे । \newline
3. स॒वि॒तुः प्र॑स॒वे प्र॑स॒वे स॑वि॒तुः स॑वि॒तुः प्र॑स॒व इतीति॑ प्रस॒वे स॑वि॒तुः स॑वि॒तुः प्र॑स॒व इति॑ । \newline
4. प्र॒स॒व इतीति॑ प्रस॒वे प्र॑स॒व इति॑ खनति खन॒तीति॑ प्रस॒वे प्र॑स॒व इति॑ खनति । \newline
5. प्र॒स॒व इति॑ प्र - स॒वे । \newline
6. इति॑ खनति खन॒तीतीति॑ खनति॒ प्रसू᳚त्यै॒ प्रसू᳚त्यै खन॒तीतीति॑ खनति॒ प्रसू᳚त्यै । \newline
7. ख॒न॒ति॒ प्रसू᳚त्यै॒ प्रसू᳚त्यै खनति खनति॒ प्रसू᳚त्या॒ अथो॒ अथो॒ प्रसू᳚त्यै खनति खनति॒ प्रसू᳚त्या॒ अथो᳚ । \newline
8. प्रसू᳚त्या॒ अथो॒ अथो॒ प्रसू᳚त्यै॒ प्रसू᳚त्या॒ अथो॑ धू॒मम् धू॒म मथो॒ प्रसू᳚त्यै॒ प्रसू᳚त्या॒ अथो॑ धू॒मम् । \newline
9. प्रसू᳚त्या॒ इति॒ प्र - सू॒त्यै॒ । \newline
10. अथो॑ धू॒मम् धू॒म मथो॒ अथो॑ धू॒म मे॒वैव धू॒म मथो॒ अथो॑ धू॒म मे॒व । \newline
11. अथो॒ इत्यथो᳚ । \newline
12. धू॒म मे॒वैव धू॒मम् धू॒म मे॒वैते नै॒तेनै॒व धू॒मम् धू॒म मे॒वैतेन॑ । \newline
13. ए॒वैते नै॒ते नै॒वैवैतेन॑ जनयति जनय त्ये॒ते नै॒वैवैतेन॑ जनयति । \newline
14. ए॒तेन॑ जनयति जनय त्ये॒ते नै॒तेन॑ जनयति॒ ज्योति॑ष्मन्त॒म् ज्योति॑ष्मन्तम् जनय त्ये॒ते नै॒तेन॑ जनयति॒ ज्योति॑ष्मन्तम् । \newline
15. ज॒न॒य॒ति॒ ज्योति॑ष्मन्त॒म् ज्योति॑ष्मन्तम् जनयति जनयति॒ ज्योति॑ष्मन्तम् त्वा त्वा॒ ज्योति॑ष्मन्तम् जनयति जनयति॒ ज्योति॑ष्मन्तम् त्वा । \newline
16. ज्योति॑ष्मन्तम् त्वा त्वा॒ ज्योति॑ष्मन्त॒म् ज्योति॑ष्मन्तम् त्वा ऽग्ने अग्ने त्वा॒ ज्योति॑ष्मन्त॒म् ज्योति॑ष्मन्तम् त्वा ऽग्ने । \newline
17. त्वा॒ ऽग्ने॒ अ॒ग्ने॒ त्वा॒ त्वा॒ ऽग्ने॒ सु॒प्रती॑कꣳ सु॒प्रती॑क मग्ने त्वा त्वा ऽग्ने सु॒प्रती॑कम् । \newline
18. अ॒ग्ने॒ सु॒प्रती॑कꣳ सु॒प्रती॑क मग्ने अग्ने सु॒प्रती॑क॒ मितीति॑ सु॒प्रती॑क मग्ने अग्ने सु॒प्रती॑क॒ मिति॑ । \newline
19. सु॒प्रती॑क॒ मितीति॑ सु॒प्रती॑कꣳ सु॒प्रती॑क॒ मित्या॑हा॒हेति॑ सु॒प्रती॑कꣳ सु॒प्रती॑क॒ मित्या॑ह । \newline
20. सु॒प्रती॑क॒मिति॑ सु - प्रती॑कम् । \newline
21. इत्या॑हा॒हेती त्या॑ह॒ ज्योति॒र् ज्योति॑ रा॒हेती त्या॑ह॒ ज्योतिः॑ । \newline
22. आ॒ह॒ ज्योति॒र् ज्योति॑ राहाह॒ ज्योति॑ रे॒वैव ज्योति॑ राहाह॒ ज्योति॑ रे॒व । \newline
23. ज्योति॑ रे॒वैव ज्योति॒र् ज्योति॑ रे॒वैते नै॒ते नै॒व ज्योति॒र् ज्योति॑ रे॒वैतेन॑ । \newline
24. ए॒वैते नै॒ते नै॒वैवैतेन॑ जनयति जनय त्ये॒ते नै॒वैवैतेन॑ जनयति । \newline
25. ए॒तेन॑ जनयति जनय त्ये॒ते नै॒तेन॑ जनयति॒ स स ज॑नय त्ये॒ते नै॒तेन॑ जनयति॒ सः । \newline
26. ज॒न॒य॒ति॒ स स ज॑नयति जनयति॒ सो᳚ ऽग्नि र॒ग्निः स ज॑नयति जनयति॒ सो᳚ ऽग्निः । \newline
27. सो᳚ ऽग्नि र॒ग्निः स सो᳚ ऽग्निर् जा॒तो जा॒तो᳚ ऽग्निः स सो᳚ ऽग्निर् जा॒तः । \newline
28. अ॒ग्निर् जा॒तो जा॒तो᳚ ऽग्नि र॒ग्निर् जा॒तः प्र॒जाः प्र॒जा जा॒तो᳚ ऽग्नि र॒ग्निर् जा॒तः प्र॒जाः । \newline
29. जा॒तः प्र॒जाः प्र॒जा जा॒तो जा॒तः प्र॒जाः शु॒चा शु॒चा प्र॒जा जा॒तो जा॒तः प्र॒जाः शु॒चा । \newline
30. प्र॒जाः शु॒चा शु॒चा प्र॒जाः प्र॒जाः शु॒चा ऽऽर्प॑यदा र्पय च्छु॒चा प्र॒जाः प्र॒जाः शु॒चा ऽऽर्प॑यत् । \newline
31. प्र॒जा इति॑ प्र - जाः । \newline
32. शु॒चा ऽऽर्प॑य दार्पय च्छु॒चा शु॒चा ऽऽर्प॑य॒त् तम् त मा᳚र्पय च्छु॒चा शु॒चा ऽऽर्प॑य॒त् तम् । \newline
33. आ॒र्प॒य॒त् तम् त मा᳚र्पय दार्पय॒त् तम् दे॒वा दे॒वा स्त मा᳚र्पय दार्पय॒त् तम् दे॒वाः । \newline
34. तम् दे॒वा दे॒वा स्तम् तम् दे॒वा अ॑र्द्ध॒र्चेना᳚ र्द्ध॒र्चेन॑ दे॒वा स्तम् तम् दे॒वा अ॑र्द्ध॒र्चेन॑ । \newline
35. दे॒वा अ॑र्द्ध॒र्चेना᳚ र्द्ध॒र्चेन॑ दे॒वा दे॒वा अ॑र्द्ध॒र्चेना॑ शमयन् नशमयन् नर्द्ध॒र्चेन॑ दे॒वा दे॒वा अ॑र्द्ध॒र्चेना॑ शमयन्न् । \newline
36. अ॒र्द्ध॒र्चेना॑ शमयन् नशमयन् नर्द्ध॒र्चेना᳚ र्द्ध॒र्चेना॑ शमयञ् छि॒वꣳ शि॒व म॑शमयन् नर्द्ध॒र्चेना᳚ र्द्ध॒र्चेना॑ शमयञ् छि॒वम् । \newline
37. अ॒र्द्ध॒र्चेनेत्य॑र्द्ध - ऋ॒चेन॑ । \newline
38. अ॒श॒म॒य॒ञ् छि॒वꣳ शि॒व म॑शमयन् नशमयञ् छि॒वम् प्र॒जाभ्यः॑ प्र॒जाभ्यः॑ शि॒व म॑शमयन् नशमयञ् छि॒वम् प्र॒जाभ्यः॑ । \newline
39. शि॒वम् प्र॒जाभ्यः॑ प्र॒जाभ्यः॑ शि॒वꣳ शि॒वम् प्र॒जाभ्यो ऽहिꣳ॑सन्त॒ महिꣳ॑सन्तम् प्र॒जाभ्यः॑ शि॒वꣳ शि॒वम् प्र॒जाभ्यो ऽहिꣳ॑सन्तम् । \newline
40. प्र॒जाभ्यो ऽहिꣳ॑सन्त॒ महिꣳ॑सन्तम् प्र॒जाभ्यः॑ प्र॒जाभ्यो ऽहिꣳ॑सन्त॒ मिती त्यहिꣳ॑सन्तम् प्र॒जाभ्यः॑ प्र॒जाभ्यो ऽहिꣳ॑सन्त॒ मिति॑ । \newline
41. प्र॒जाभ्य॒ इति॑ प्र - जाभ्यः॑ । \newline
42. अहिꣳ॑सन्त॒ मिती त्यहिꣳ॑सन्त॒ महिꣳ॑सन्त॒ मित्या॑हा॒हे त्यहिꣳ॑सन्त॒ महिꣳ॑सन्त॒ मित्या॑ह । \newline
43. इत्या॑हा॒हे तीत्या॑ह प्र॒जाभ्यः॑ प्र॒जाभ्य॑ आ॒हेती त्या॑ह प्र॒जाभ्यः॑ । \newline
44. आ॒ह॒ प्र॒जाभ्यः॑ प्र॒जाभ्य॑ आहाह प्र॒जाभ्य॑ ए॒वैव प्र॒जाभ्य॑ आहाह प्र॒जाभ्य॑ ए॒व । \newline
45. प्र॒जाभ्य॑ ए॒वैव प्र॒जाभ्यः॑ प्र॒जाभ्य॑ ए॒वैन॑ मेन मे॒व प्र॒जाभ्यः॑ प्र॒जाभ्य॑ ए॒वैन᳚म् । \newline
46. प्र॒जाभ्य॒ इति॑ प्र - जाभ्यः॑ । \newline
47. ए॒वैन॑ मेन मे॒वैवैनꣳ॑ शमयति शमय त्येन मे॒वैवैनꣳ॑ शमयति । \newline
48. ए॒नꣳ॒॒ श॒म॒य॒ति॒ श॒म॒य॒ त्ये॒न॒ मे॒नꣳ॒॒ श॒म॒य॒ति॒ द्वाभ्या॒म् द्वाभ्याꣳ॑ शमय त्येन मेनꣳ शमयति॒ द्वाभ्या᳚म् । \newline
49. श॒म॒य॒ति॒ द्वाभ्या॒म् द्वाभ्याꣳ॑ शमयति शमयति॒ द्वाभ्या᳚म् खनति खनति॒ द्वाभ्याꣳ॑ शमयति शमयति॒ द्वाभ्या᳚म् खनति । \newline
50. द्वाभ्या᳚म् खनति खनति॒ द्वाभ्या॒म् द्वाभ्या᳚म् खनति॒ प्रति॑ष्ठित्यै॒ प्रति॑ष्ठित्यै खनति॒ द्वाभ्या॒म् द्वाभ्या᳚म् खनति॒ प्रति॑ष्ठित्यै । \newline
51. ख॒न॒ति॒ प्रति॑ष्ठित्यै॒ प्रति॑ष्ठित्यै खनति खनति॒ प्रति॑ष्ठित्या अ॒पा म॒पाम् प्रति॑ष्ठित्यै खनति खनति॒ प्रति॑ष्ठित्या अ॒पाम् । \newline
52. प्रति॑ष्ठित्या अ॒पा म॒पाम् प्रति॑ष्ठित्यै॒ प्रति॑ष्ठित्या अ॒पाम् पृ॒ष्ठम् पृ॒ष्ठ म॒पाम् प्रति॑ष्ठित्यै॒ प्रति॑ष्ठित्या अ॒पाम् पृ॒ष्ठम् । \newline
53. प्रति॑ष्ठित्या॒ इति॒ प्रति॑ - स्थि॒त्यै॒ । \newline
54. अ॒पाम् पृ॒ष्ठम् पृ॒ष्ठ म॒पा म॒पाम् पृ॒ष्ठ म॑स्यसि पृ॒ष्ठ म॒पा म॒पाम् पृ॒ष्ठ म॑सि । \newline
55. पृ॒ष्ठ म॑स्यसि पृ॒ष्ठम् पृ॒ष्ठ म॒सीती त्य॑सि पृ॒ष्ठम् पृ॒ष्ठ म॒सीति॑ । \newline
56. अ॒सीती त्य॑स्य॒सीति॑ पुष्करप॒र्णम् पु॑ष्करप॒र्ण मित्य॑स्य॒सीति॑ पुष्करप॒र्णम् । \newline
57. इति॑ पुष्करप॒र्णम् पु॑ष्करप॒र्ण मितीति॑ पुष्करप॒र्ण मा पु॑ष्करप॒र्ण मितीति॑ पुष्करप॒र्ण मा । \newline
58. पु॒ष्क॒र॒प॒र्ण मा पु॑ष्करप॒र्णम् पु॑ष्करप॒र्ण मा ह॑रति हर॒त्या पु॑ष्करप॒र्णम् पु॑ष्करप॒र्ण मा ह॑रति । \newline
59. पु॒ष्क॒र॒प॒र्णमिति॑ पुष्कर - प॒र्णम् । \newline
60. आ ह॑रति हर॒त्या ह॑रत्य॒पा म॒पाꣳ ह॑र॒त्या ह॑रत्य॒पाम् । \newline
\pagebreak
\markright{ TS 5.1.4.2  \hfill https://www.vedavms.in \hfill}

\section{ TS 5.1.4.2 }

\textbf{TS 5.1.4.2 } \newline
\textbf{Samhita Paata} \newline

ह॑रत्य॒पां ॅवा ए॒तत् पृ॒ष्ठं ॅयत् पु॑ष्करप॒र्णꣳ रू॒पेणै॒वैन॒दा ह॑रति पुष्करप॒र्णेन॒ सं भ॑रति॒ योनि॒र्वा अ॒ग्नेः पु॑ष्करप॒र्णꣳ सयो॑निमे॒वाग्निꣳ संभ॑रति कृष्णाजि॒नेन॒ संभ॑रति य॒ज्ञो वै कृ॑ष्णाजि॒नं ॅय॒ज्ञेनै॒व य॒ज्ञ्ꣳ संभ॑रति॒ यद् ग्रा॒म्याणां᳚ पशू॒नां चर्म॑णा स॒भंरे᳚द् ग्रा॒म्यान् प॒शूञ्छु॒चाऽर्प॑येत् कृष्णाजि॒नेन॒ संभ॑रत्यार॒ण्याने॒व प॒शून् - [  ] \newline

\textbf{Pada Paata} \newline

ह॒र॒ति॒ । अ॒पाम् । वै । ए॒तत् । पृ॒ष्ठम् । यत् । पु॒ष्क॒र॒प॒र्णमिति॑ पुष्कर - प॒र्णम् । रू॒पेण॑ । ए॒व । ए॒न॒त् । एति॑ । ह॒र॒ति॒ । पु॒ष्क॒र॒प॒र्णेनेति॑ पुष्कर - प॒र्णेन॑ । समिति॑ । भ॒र॒ति॒ । योनिः॑ । वै । अ॒ग्नेः । पु॒ष्क॒र॒प॒र्णमिति॑ पुष्कर - प॒र्णम् । सयो॑नि॒मिति॒ स-यो॒नि॒म् । ए॒व । अ॒ग्निम् । समिति॑ । भ॒र॒ति॒ । कृ॒ष्णा॒जि॒नेनेति॑ कृष्ण-अ॒जि॒नेन॑ । समिति॑ । भ॒र॒ति॒ । य॒ज्ञ्ः । वै । कृ॒ष्णा॒जि॒नमिति॑ कृष्ण - अ॒जि॒नम् । य॒ज्ञेन॑ । ए॒व । य॒ज्ञ्म् । समिति॑ । भ॒र॒ति॒ । यत् । ग्रा॒म्याणा᳚म् । प॒शू॒नाम् । चर्म॑णा । स॒भंरे॒दिति॑ सं - भरे᳚त् । ग्रा॒म्यान् । प॒शून् । शु॒चा । अ॒र्प॒ये॒त् । कृ॒ष्णा॒जि॒नेनेति॑ कृष्ण - अ॒जि॒नेन॑ । समिति॑ । भ॒र॒ति॒ । आ॒र॒ण्यान् । ए॒व । प॒शून् ।  \newline


\textbf{Krama Paata} \newline

ह॒र॒त्य॒पाम् । अ॒पाम् ॅवै । वा ए॒तत् । ए॒तत् पृ॒ष्ठम् । पृ॒ष्ठम् ॅयत् । यत् पु॑ष्करप॒र्णम् । पु॒ष्क॒र॒प॒र्णꣳ रू॒पेण॑ । पु॒ष्क॒र॒प॒र्णमिति॑ पुष्कर - प॒र्णम् । रू॒पेणै॒व । ए॒वैन॑त् । ए॒न॒दा । आ ह॑रति । ह॒र॒ति॒ पु॒ष्क॒र॒प॒र्णेन॑ । पु॒ष्क॒र॒प॒र्णेन॒ सम् । पु॒ष्क॒र॒प॒र्णेनेति॑ पुष्कर - प॒र्णेन॑ । सम् भ॑रति । भ॒र॒ति॒ योनिः॑ । योनि॒र् वै । वा अ॒ग्नेः । अ॒ग्नेः पु॑ष्करप॒र्णम् । पु॒ष्क॒र॒प॒र्णꣳ सयो॑निम् । पु॒ष्क॒र॒प॒र्णमिति॑ पुष्कर - प॒र्णम् । सयो॑निमे॒व । सयो॑नि॒मिति॒ स - यो॒नि॒म् । ए॒वाग्निम् । अ॒ग्निꣳ सम् । 
सम् भ॑रति । भ॒र॒ति॒ कृ॒ष्णा॒जि॒नेन॑ । कृ॒ष्णा॒जि॒नेन॒ सम् । कृ॒ष्णा॒जि॒नेनेति॑ कृष्ण - अ॒जि॒नेन॑ । सम् भ॑रति । भ॒र॒ति॒ य॒ज्ञ्ः । य॒ज्ञो वै । वै कृ॑ष्णाजि॒नम् । कृ॒ष्णा॒जि॒नम् ॅय॒ज्ञेन॑ । कृ॒ष्णा॒जि॒नमिति॑ कृष्ण - अ॒जि॒नम् । य॒ज्ञेनै॒व । ए॒व य॒ज्ञ्म् । य॒ज्ञ्ꣳ सम् । सम् भ॑रति । भ॒र॒ति॒ यत् । यद् ग्रा॒म्याणा᳚म् । ग्रा॒म्याणा᳚म् पशू॒नाम् । प॒शू॒नाम् चर्म॑णा । चर्म॑णा स॒म्भरे᳚त् । स॒म्भरे᳚द् ग्रा॒म्यान् । स॒म्भरे॒दिति॑ सम् - भरे᳚त् । ग्रा॒म्यान् प॒शून् । प॒शूञ्छु॒चा । शु॒चाऽर्प॑येत् । अ॒र्प॒ये॒त् कृ॒ष्णा॒जि॒नेन॑ । कृ॒ष्णा॒जि॒नेन॒ सम् । कृ॒ष्णा॒जि॒नेनेति॑ कृष्ण - अ॒जि॒नेन॑ । 
सम् भ॑रति । भ॒र॒त्या॒र॒ण्यान् । आ॒र॒ण्याने॒व । ए॒व प॒शून् । 
प॒शूञ्छु॒चा \newline

\textbf{Jatai Paata} \newline

1. ह॒र॒ त्य॒पा म॒पाꣳ ह॑रति हर त्य॒पाम् । \newline
2. अ॒पां ॅवै वा अ॒पा म॒पां ॅवै । \newline
3. वा ए॒त दे॒तद् वै वा ए॒तत् । \newline
4. ए॒तत् पृ॒ष्ठम् पृ॒ष्ठ मे॒त दे॒तत् पृ॒ष्ठम् । \newline
5. पृ॒ष्ठं ॅयद् यत् पृ॒ष्ठम् पृ॒ष्ठं ॅयत् । \newline
6. यत् पु॑ष्करप॒र्णम् पु॑ष्करप॒र्णं ॅयद् यत् पु॑ष्करप॒र्णम् । \newline
7. पु॒ष्क॒र॒प॒र्णꣳ रू॒पेण॑ रू॒पेण॑ पुष्करप॒र्णम् पु॑ष्करप॒र्णꣳ रू॒पेण॑ । \newline
8. पु॒ष्क॒र॒प॒र्णमिति॑ पुष्कर - प॒र्णम् । \newline
9. रू॒पे णै॒वैव रू॒पेण॑ रू॒पे णै॒व । \newline
10. ए॒वैन॑ देन दे॒वै वैन॑त् । \newline
11. ए॒न॒ दैन॑ देन॒दा । \newline
12. आ ह॑रति हर॒त्या ह॑रति । \newline
13. ह॒र॒ति॒ पु॒ष्क॒र॒प॒र्णेन॑ पुष्करप॒र्णेन॑ हरति हरति पुष्करप॒र्णेन॑ । \newline
14. पु॒ष्क॒र॒प॒र्णेन॒ सꣳ सम् पु॑ष्करप॒र्णेन॑ पुष्करप॒र्णेन॒ सम् । \newline
15. पु॒ष्क॒र॒प॒र्णेनेति॑ पुष्कर - प॒र्णेन॑ । \newline
16. सम् भ॑रति भरति॒ सꣳ सम् भ॑रति । \newline
17. भ॒र॒ति॒ योनि॒र् योनि॑र् भरति भरति॒ योनिः॑ । \newline
18. योनि॒र् वै वै योनि॒र् योनि॒र् वै । \newline
19. वा अ॒ग्ने र॒ग्नेर् वै वा अ॒ग्नेः । \newline
20. अ॒ग्नेः पु॑ष्करप॒र्णम् पु॑ष्करप॒र्ण म॒ग्ने र॒ग्नेः पु॑ष्करप॒र्णम् । \newline
21. पु॒ष्क॒र॒प॒र्णꣳ सयो॑निꣳ॒॒ सयो॑निम् पुष्करप॒र्णम् पु॑ष्करप॒र्णꣳ सयो॑निम् । \newline
22. पु॒ष्क॒र॒प॒र्णमिति॑ पुष्कर - प॒र्णम् । \newline
23. सयो॑नि मे॒वैव सयो॑निꣳ॒॒ सयो॑नि मे॒व । \newline
24. सयो॑नि॒मिति॒ स - यो॒नि॒म् । \newline
25. ए॒वाग्नि म॒ग्नि मे॒वैवाग्निम् । \newline
26. अ॒ग्निꣳ सꣳ स म॒ग्नि म॒ग्निꣳ सम् । \newline
27. सम् भ॑रति भरति॒ सꣳ सम् भ॑रति । \newline
28. भ॒र॒ति॒ कृ॒ष्णा॒जि॒नेन॑ कृष्णाजि॒नेन॑ भरति भरति कृष्णाजि॒नेन॑ । \newline
29. कृ॒ष्णा॒जि॒नेन॒ सꣳ सम् कृ॑ष्णाजि॒नेन॑ कृष्णाजि॒नेन॒ सम् । \newline
30. कृ॒ष्णा॒जि॒नेनेति॑ कृष्ण - अ॒जि॒नेन॑ । \newline
31. सम् भ॑रति भरति॒ सꣳ सम् भ॑रति । \newline
32. भ॒र॒ति॒ य॒ज्ञो य॒ज्ञो भ॑रति भरति य॒ज्ञ्ः । \newline
33. य॒ज्ञो वै वै य॒ज्ञो य॒ज्ञो वै । \newline
34. वै कृ॑ष्णाजि॒नम् कृ॑ष्णाजि॒नं ॅवै वै कृ॑ष्णाजि॒नम् । \newline
35. कृ॒ष्णा॒जि॒नं ॅय॒ज्ञेन॑ य॒ज्ञेन॑ कृष्णाजि॒नम् कृ॑ष्णाजि॒नं ॅय॒ज्ञेन॑ । \newline
36. कृ॒ष्णा॒जि॒नमिति॑ कृष्ण - अ॒जि॒नम् । \newline
37. य॒ज्ञे नै॒वैव य॒ज्ञेन॑ य॒ज्ञे नै॒व । \newline
38. ए॒व य॒ज्ञ्ं ॅय॒ज्ञ् मे॒वैव य॒ज्ञ्म् । \newline
39. य॒ज्ञ्ꣳ सꣳ सं ॅय॒ज्ञ्ं ॅय॒ज्ञ्ꣳ सम् । \newline
40. सम् भ॑रति भरति॒ सꣳ सम् भ॑रति । \newline
41. भ॒र॒ति॒ यद् यद् भ॑रति भरति॒ यत् । \newline
42. यद् ग्रा॒म्याणा᳚म् ग्रा॒म्याणां॒ ॅयद् यद् ग्रा॒म्याणा᳚म् । \newline
43. ग्रा॒म्याणा᳚म् पशू॒नाम् प॑शू॒नाम् ग्रा॒म्याणा᳚म् ग्रा॒म्याणा᳚म् पशू॒नाम् । \newline
44. प॒शू॒नाम् चर्म॑णा॒ चर्म॑णा पशू॒नाम् प॑शू॒नाम् चर्म॑णा । \newline
45. चर्म॑णा सं॒भरे᳚थ् सं॒भरे॒च् चर्म॑णा॒ चर्म॑णा सं॒भरे᳚त् । \newline
46. सं॒भरे᳚द् ग्रा॒म्यान् ग्रा॒म्यान् थ्सं॒भरे᳚थ् सं॒भरे᳚द् ग्रा॒म्यान् । \newline
47. सं॒भरे॒दिति॑ सं - भरे᳚त् । \newline
48. ग्रा॒म्यान् प॒शून् प॒शून् ग्रा॒म्यान् ग्रा॒म्यान् प॒शून् । \newline
49. प॒शूञ् छु॒चा शु॒चा प॒शून् प॒शूञ् छु॒चा । \newline
50. शु॒चा ऽर्प॑ये दर्पये च्छु॒चा शु॒चा ऽर्प॑येत् । \newline
51. अ॒र्प॒ये॒त् कृ॒ष्णा॒जि॒नेन॑ कृष्णाजि॒नेना᳚ र्पये दर्पयेत् कृष्णाजि॒नेन॑ । \newline
52. कृ॒ष्णा॒जि॒नेन॒ सꣳ सम् कृ॑ष्णाजि॒नेन॑ कृष्णाजि॒नेन॒ सम् । \newline
53. कृ॒ष्णा॒जि॒नेनेति॑ कृष्ण - अ॒जि॒नेन॑ । \newline
54. सम् भ॑रति भरति॒ सꣳ सम् भ॑रति । \newline
55. भ॒र॒त्या॒ र॒ण्या ना॑र॒ण्यान् भ॑रति भरत्या र॒ण्यान् । \newline
56. आ॒र॒ण्या ने॒वै वार॒ण्या ना॑र॒ण्या ने॒व । \newline
57. ए॒व प॒शून् प॒शू ने॒वैव प॒शून् । \newline
58. प॒शूञ् छु॒चा शु॒चा प॒शून् प॒शूञ् छु॒चा । \newline

\textbf{Ghana Paata } \newline

1. ह॒र॒त्य॒पा म॒पाꣳ ह॑रति हरत्य॒पां ॅवै वा अ॒पाꣳ ह॑रति हरत्य॒पां ॅवै । \newline
2. अ॒पां ॅवै वा अ॒पा म॒पां ॅवा ए॒त दे॒तद् वा अ॒पा म॒पां ॅवा ए॒तत् । \newline
3. वा ए॒त दे॒तद् वै वा ए॒तत् पृ॒ष्ठम् पृ॒ष्ठ मे॒तद् वै वा ए॒तत् पृ॒ष्ठम् । \newline
4. ए॒तत् पृ॒ष्ठम् पृ॒ष्ठ मे॒त दे॒तत् पृ॒ष्ठं ॅयद् यत् पृ॒ष्ठ मे॒त दे॒तत् पृ॒ष्ठं ॅयत् । \newline
5. पृ॒ष्ठं ॅयद् यत् पृ॒ष्ठम् पृ॒ष्ठं ॅयत् पु॑ष्करप॒र्णम् पु॑ष्करप॒र्णं ॅयत् पृ॒ष्ठम् पृ॒ष्ठं ॅयत् पु॑ष्करप॒र्णम् । \newline
6. यत् पु॑ष्करप॒र्णम् पु॑ष्करप॒र्णं ॅयद् यत् पु॑ष्करप॒र्णꣳ रू॒पेण॑ रू॒पेण॑ पुष्करप॒र्णं ॅयद् यत् पु॑ष्करप॒र्णꣳ रू॒पेण॑ । \newline
7. पु॒ष्क॒र॒प॒र्णꣳ रू॒पेण॑ रू॒पेण॑ पुष्करप॒र्णम् पु॑ष्करप॒र्णꣳ रू॒पेणै॒वैव रू॒पेण॑ पुष्करप॒र्णम् पु॑ष्करप॒र्णꣳ रू॒पेणै॒व । \newline
8. पु॒ष्क॒र॒प॒र्णमिति॑ पुष्कर - प॒र्णम् । \newline
9. रू॒पे णै॒वैव रू॒पेण॑ रू॒पे णै॒वैन॑ देन दे॒व रू॒पेण॑ रू॒पे णै॒वैन॑त् । \newline
10. ए॒वैन॑ देन दे॒वैवैन॒ दैन॑ दे॒वैवैन॒दा । \newline
11. ए॒न॒ दैन॑देन॒दा ह॑रति हर॒ त्यैन॑ देन॒दा ह॑रति । \newline
12. आ ह॑रति हर॒त्या ह॑रति पुष्करप॒र्णेन॑ पुष्करप॒र्णेन॑ हर॒त्या ह॑रति पुष्करप॒र्णेन॑ । \newline
13. ह॒र॒ति॒ पु॒ष्क॒र॒प॒र्णेन॑ पुष्करप॒र्णेन॑ हरति हरति पुष्करप॒र्णेन॒ सꣳ सम् पु॑ष्करप॒र्णेन॑ हरति हरति पुष्करप॒र्णेन॒ सम् । \newline
14. पु॒ष्क॒र॒प॒र्णेन॒ सꣳ सम् पु॑ष्करप॒र्णेन॑ पुष्करप॒र्णेन॒ सम् भ॑रति भरति॒ सम् पु॑ष्करप॒र्णेन॑ पुष्करप॒र्णेन॒ सम् भ॑रति । \newline
15. पु॒ष्क॒र॒प॒र्णेनेति॑ पुष्कर - प॒र्णेन॑ । \newline
16. सम् भ॑रति भरति॒ सꣳ सम् भ॑रति॒ योनि॒र् योनि॑र् भरति॒ सꣳ सम् भ॑रति॒ योनिः॑ । \newline
17. भ॒र॒ति॒ योनि॒र् योनि॑र् भरति भरति॒ योनि॒र् वै वै योनि॑र् भरति भरति॒ योनि॒र् वै । \newline
18. योनि॒र् वै वै योनि॒र् योनि॒र् वा अ॒ग्ने र॒ग्नेर् वै योनि॒र् योनि॒र् वा अ॒ग्नेः । \newline
19. वा अ॒ग्ने र॒ग्नेर् वै वा अ॒ग्नेः पु॑ष्करप॒र्णम् पु॑ष्करप॒र्ण म॒ग्नेर् वै वा अ॒ग्नेः पु॑ष्करप॒र्णम् । \newline
20. अ॒ग्नेः पु॑ष्करप॒र्णम् पु॑ष्करप॒र्ण म॒ग्ने र॒ग्नेः पु॑ष्करप॒र्णꣳ सयो॑निꣳ॒॒ सयो॑निम् पुष्करप॒र्ण म॒ग्ने र॒ग्नेः पु॑ष्करप॒र्णꣳ सयो॑निम् । \newline
21. पु॒ष्क॒र॒प॒र्णꣳ सयो॑निꣳ॒॒ सयो॑निम् पुष्करप॒र्णम् पु॑ष्करप॒र्णꣳ सयो॑नि मे॒वैव सयो॑निम् पुष्करप॒र्णम् पु॑ष्करप॒र्णꣳ सयो॑नि मे॒व । \newline
22. पु॒ष्क॒र॒प॒र्णमिति॑ पुष्कर - प॒र्णम् । \newline
23. सयो॑नि मे॒वैव सयो॑निꣳ॒॒ सयो॑नि मे॒वाग्नि म॒ग्नि मे॒व सयो॑निꣳ॒॒ सयो॑नि मे॒वाग्निम् । \newline
24. सयो॑नि॒मिति॒ स - यो॒नि॒म् । \newline
25. ए॒वाग्नि म॒ग्नि मे॒वैवाग्निꣳ सꣳ स म॒ग्नि मे॒वैवाग्निꣳ सम् । \newline
26. अ॒ग्निꣳ सꣳ स म॒ग्नि म॒ग्निꣳ सम् भ॑रति भरति॒ स म॒ग्नि म॒ग्निꣳ सम् भ॑रति । \newline
27. सम् भ॑रति भरति॒ सꣳ सम् भ॑रति कृष्णाजि॒नेन॑ कृष्णाजि॒नेन॑ भरति॒ सꣳ सम् भ॑रति कृष्णाजि॒नेन॑ । \newline
28. भ॒र॒ति॒ कृ॒ष्णा॒जि॒नेन॑ कृष्णाजि॒नेन॑ भरति भरति कृष्णाजि॒नेन॒ सꣳ सम् कृ॑ष्णाजि॒नेन॑ भरति भरति कृष्णाजि॒नेन॒ सम् । \newline
29. कृ॒ष्णा॒जि॒नेन॒ सꣳ सम् कृ॑ष्णाजि॒नेन॑ कृष्णाजि॒नेन॒ सम् भ॑रति भरति॒ सम् कृ॑ष्णाजि॒नेन॑ कृष्णाजि॒नेन॒ सम् भ॑रति । \newline
30. कृ॒ष्णा॒जि॒नेनेति॑ कृष्ण - अ॒जि॒नेन॑ । \newline
31. सम् भ॑रति भरति॒ सꣳ सम् भ॑रति य॒ज्ञो य॒ज्ञो भ॑रति॒ सꣳ सम् भ॑रति य॒ज्ञ्ः । \newline
32. भ॒र॒ति॒ य॒ज्ञो य॒ज्ञो भ॑रति भरति य॒ज्ञो वै वै य॒ज्ञो भ॑रति भरति य॒ज्ञो वै । \newline
33. य॒ज्ञो वै वै य॒ज्ञो य॒ज्ञो वै कृ॑ष्णाजि॒नम् कृ॑ष्णाजि॒नं ॅवै य॒ज्ञो य॒ज्ञो वै कृ॑ष्णाजि॒नम् । \newline
34. वै कृ॑ष्णाजि॒नम् कृ॑ष्णाजि॒नं ॅवै वै कृ॑ष्णाजि॒नं ॅय॒ज्ञेन॑ य॒ज्ञेन॑ कृष्णाजि॒नं ॅवै वै कृ॑ष्णाजि॒नं ॅय॒ज्ञेन॑ । \newline
35. कृ॒ष्णा॒जि॒नं ॅय॒ज्ञेन॑ य॒ज्ञेन॑ कृष्णाजि॒नम् कृ॑ष्णाजि॒नं ॅय॒ज्ञे नै॒वैव य॒ज्ञेन॑ कृष्णाजि॒नम् कृ॑ष्णाजि॒नं ॅय॒ज्ञेनै॒व । \newline
36. कृ॒ष्णा॒जि॒नमिति॑ कृष्ण - अ॒जि॒नम् । \newline
37. य॒ज्ञे नै॒वैव य॒ज्ञेन॑ य॒ज्ञेनै॒व य॒ज्ञ्ं ॅय॒ज्ञ् मे॒व य॒ज्ञेन॑ य॒ज्ञेनै॒व य॒ज्ञ्म् । \newline
38. ए॒व य॒ज्ञ्ं ॅय॒ज्ञ् मे॒वैव य॒ज्ञ्ꣳ सꣳ सं ॅय॒ज्ञ् मे॒वैव य॒ज्ञ्ꣳ सम् । \newline
39. य॒ज्ञ्ꣳ सꣳ सं ॅय॒ज्ञ्ं ॅय॒ज्ञ्ꣳ सम् भ॑रति भरति॒ सं ॅय॒ज्ञ्ं ॅय॒ज्ञ्ꣳ सम् भ॑रति । \newline
40. सम् भ॑रति भरति॒ सꣳ सम् भ॑रति॒ यद् यद् भ॑रति॒ सꣳ सम् भ॑रति॒ यत् । \newline
41. भ॒र॒ति॒ यद् यद् भ॑रति भरति॒ यद् ग्रा॒म्याणा᳚म् ग्रा॒म्याणां॒ ॅयद् भ॑रति भरति॒ यद् ग्रा॒म्याणा᳚म् । \newline
42. यद् ग्रा॒म्याणा᳚म् ग्रा॒म्याणां॒ ॅयद् यद् ग्रा॒म्याणा᳚म् पशू॒नाम् प॑शू॒नाम् ग्रा॒म्याणां॒ ॅयद् यद् ग्रा॒म्याणा᳚म् पशू॒नाम् । \newline
43. ग्रा॒म्याणा᳚म् पशू॒नाम् प॑शू॒नाम् ग्रा॒म्याणा᳚म् ग्रा॒म्याणा᳚म् पशू॒नाम् चर्म॑णा॒ चर्म॑णा पशू॒नाम् ग्रा॒म्याणा᳚म् ग्रा॒म्याणा᳚म् पशू॒नाम् चर्म॑णा । \newline
44. प॒शू॒नाम् चर्म॑णा॒ चर्म॑णा पशू॒नाम् प॑शू॒नाम् चर्म॑णा सं॒भरे᳚थ् सं॒भरे॒च् चर्म॑णा पशू॒नाम् प॑शू॒नाम् चर्म॑णा सं॒भरे᳚त् । \newline
45. चर्म॑णा सं॒भरे᳚थ् सं॒भरे॒च् चर्म॑णा॒ चर्म॑णा सं॒भरे᳚द् ग्रा॒म्यान् ग्रा॒म्यान् थ्सं॒भरे॒च् चर्म॑णा॒ चर्म॑णा सं॒भरे᳚द् ग्रा॒म्यान् । \newline
46. सं॒भरे᳚द् ग्रा॒म्यान् ग्रा॒म्यान् थ्सं॒भरे᳚थ् सं॒भरे᳚द् ग्रा॒म्यान् प॒शून् प॒शून् ग्रा॒म्यान् थ्सं॒भरे᳚थ् सं॒भरे᳚द् ग्रा॒म्यान् प॒शून् । \newline
47. सं॒भरे॒दिति॑ सं - भरे᳚त् । \newline
48. ग्रा॒म्यान् प॒शून् प॒शून् ग्रा॒म्यान् ग्रा॒म्यान् प॒शूञ् छु॒चा शु॒चा प॒शून् ग्रा॒म्यान् ग्रा॒म्यान् प॒शूञ् छु॒चा । \newline
49. प॒शूञ् छु॒चा शु॒चा प॒शून् प॒शूञ् छु॒चा ऽर्प॑ये दर्पये च्छु॒चा प॒शून् प॒शूञ् छु॒चा ऽर्प॑येत् । \newline
50. शु॒चा ऽर्प॑ये दर्पये च्छु॒चा शु॒चा ऽर्प॑येत् कृष्णाजि॒नेन॑ कृष्णाजि॒नेना᳚ र्पये च्छु॒चा शु॒चा ऽर्प॑येत् कृष्णाजि॒नेन॑ । \newline
51. अ॒र्प॒ये॒त् कृ॒ष्णा॒जि॒नेन॑ कृष्णाजि॒नेना᳚ र्पये दर्पयेत् कृष्णाजि॒नेन॒ सꣳ सम् कृ॑ष्णाजि॒नेना᳚र्पये दर्पयेत् कृष्णाजि॒नेन॒ सम् । \newline
52. कृ॒ष्णा॒जि॒नेन॒ सꣳ सम् कृ॑ष्णाजि॒नेन॑ कृष्णाजि॒नेन॒ सम् भ॑रति भरति॒ सम् कृ॑ष्णाजि॒नेन॑ कृष्णाजि॒नेन॒ सम् भ॑रति । \newline
53. कृ॒ष्णा॒जि॒नेनेति॑ कृष्ण - अ॒जि॒नेन॑ । \newline
54. सम् भ॑रति भरति॒ सꣳ सम् भ॑र त्यार॒ण्या ना॑र॒ण्यान् भ॑रति॒ सꣳ सम् भ॑र त्यार॒ण्यान् । \newline
55. भ॒र॒ त्या॒र॒ण्या ना॑र॒ण्यान् भ॑रति भर त्यार॒ण्या ने॒वैवार॒ण्यान् भ॑रति भर त्यार॒ण्या ने॒व । \newline
56. आ॒र॒ण्या ने॒वैवा र॒ण्या ना॑र॒ण्या ने॒व प॒शून् प॒शू ने॒वार॒ण्या ना॑र॒ण्या ने॒व प॒शून् । \newline
57. ए॒व प॒शून् प॒शू ने॒वैव प॒शूञ् छु॒चा शु॒चा प॒शू ने॒वैव प॒शूञ् छु॒चा । \newline
58. प॒शूञ् छु॒चा शु॒चा प॒शून् प॒शूञ् छु॒चा ऽर्प॑य त्यर्पयति शु॒चा प॒शून् प॒शूञ् छु॒चा ऽर्प॑यति । \newline
\pagebreak
\markright{ TS 5.1.4.3  \hfill https://www.vedavms.in \hfill}

\section{ TS 5.1.4.3 }

\textbf{TS 5.1.4.3 } \newline
\textbf{Samhita Paata} \newline

शु॒चाऽर्प॑यति॒ तस्मा᳚थ् स॒माव॑त् पशू॒नां प्र॒जाय॑मानाना-मार॒ण्याः प॒शवः॒ कनी॑याꣳसः शु॒चा ह्यृ॑ता लो॑म॒तः संभ॑र॒त्यतो॒ ह्य॑स्य॒ मेद्ध्यं॑ कृष्णाजि॒नं च॑ पुष्करप॒र्णं च॒ सꣳ स्तृ॑णाती॒यं ॅवै कृ॑ष्णाजि॒नम॒सौ पु॑ष्करप॒र्ण-मा॒भ्या-मे॒वैन॑-मुभ॒यतः॒ परि॑गृह्णात्य॒-ग्निर्दे॒वेभ्यो॒ निला॑यत॒ तमथ॒र्वा-ऽन्व॑पश्य॒दथ॑र्वा त्वा प्रथ॒मो निर॑मन्थदग्न॒ इत्या॑ - [  ] \newline

\textbf{Pada Paata} \newline

शु॒चा । अ॒र्प॒य॒ति॒ । तस्मा᳚त् । स॒माव॑त् । प॒शू॒नाम् । प्र॒जाय॑मानाना॒मिति॑ प्र-जाय॑मानानाम् । आ॒र॒ण्याः । प॒शवः॑ । कनी॑याꣳसः । शु॒चा । हि । ऋ॒ताः । लो॒म॒तः । समिति॑ । भ॒र॒ति॒ । अतः॑ । हि । अ॒स्य॒ । मेद्ध्य᳚म् । कृ॒ष्णा॒जि॒नमिति॑ कृष्ण - अ॒जि॒नम् । च॒ । पु॒ष्क॒र॒प॒र्णमिति॑ पुष्कर - प॒र्णम् । च॒ । समिति॑ । स्तृ॒णा॒ति॒ । इ॒यम् । वै । कृ॒ष्णा॒जि॒नमिति॑ कृष्ण - अ॒जि॒नम् । अ॒सौ । पु॒ष्क॒र॒प॒र्णमिति॑ पुष्कर - प॒र्णम् । आ॒भ्याम् । ए॒व । ए॒न॒म् । उ॒भ॒यतः॑ । परीति॑ । गृ॒ह्णा॒ति॒ । अ॒ग्निः । दे॒वेभ्यः॑ । निला॑यत । तम् । अथ॑र्वा । अन्विति॑ । अ॒प॒श्य॒त् । अथ॑र्वा । त्वा॒ । प्र॒थ॒मः । निरिति॑ । अ॒म॒न्थ॒त् । अ॒ग्ने॒ । इति॑ ।  \newline


\textbf{Krama Paata} \newline

शु॒चाऽर्प॑यति । अ॒र्प॒य॒ति॒ तस्मा᳚त् । तस्मा᳚थ् स॒माव॑त् । स॒माव॑त् पशू॒नाम् । प॒शू॒नाम् प्र॒जाय॑मानानाम् । प्र॒जाय॑मानानामार॒ण्याः । प्र॒जाय॑मानाना॒मिति॑ प्र - जाय॑मानानाम् । आ॒र॒ण्याः प॒शवः॑ । 
प॒शवः॒ कनी॑याꣳसः । कनी॑याꣳसः शु॒चा । शु॒चा हि । ह्यृ॑ताः । ऋ॒ता लो॑म॒तः । लो॒म॒तः सम् । सम् भ॑रति । भ॒र॒त्यतः॑ । अतो॒ हि । ह्य॑स्य । अ॒स्य॒ मेद्ध्य᳚म् । मेद्ध्य॑म् कृष्णाजि॒नम् । कृ॒ष्णा॒जि॒नम् च॑ । कृ॒ष्णा॒जि॒नमिति॑ कृष्ण - अ॒जि॒नम् । च॒ पु॒ष्क॒र॒प॒र्णम् । पु॒ष्क॒र॒प॒र्णम् च॑ । पु॒ष्क॒र॒प॒र्णमिति॑ पुष्कर - प॒र्णम् । च॒ सम् । सꣳ स्तृ॑णाति । स्तृ॒णा॒ती॒यम् । 
इ॒यम् ॅवै । वै कृ॑ष्णाजि॒नम् । कृ॒ष्णा॒जि॒नम॒सौ । कृ॒ष्णा॒जि॒नमिति॑ कृष्ण - अ॒जि॒नम् । अ॒सौ पु॑ष्करप॒र्णम् । पु॒ष्क॒र॒प॒र्णमा॒भ्याम् । पु॒ष्क॒र॒प॒र्णमिति॑ पुष्कर - प॒र्णम् । आ॒भ्यामे॒व । ए॒वैन᳚म् । ए॒न॒मु॒भ॒यतः॑ । उ॒भ॒यतः॒ परि॑ । परि॑ गृह्णाति । गृ॒ह्णा॒त्य॒ग्निः । अ॒ग्निर् दे॒वेभ्यः॑ । दे॒वेभ्यो॒ निला॑यत । निला॑यत॒ तम् । तमथ॑र्वा । अथ॒र्वाऽनु॑ । अन्व॑पश्यत् । अ॒प॒श्य॒दथ॑र्वा । अथ॑र्वा त्वा । त्वा॒ प्र॒थ॒मः । प्र॒थ॒मो निः । निर॑मन्थत् । अ॒म॒न्थ॒द॒ग्ने॒ । अ॒ग्न॒ इति॑ । इत्या॑ह \newline

\textbf{Jatai Paata} \newline

1. शु॒चा ऽर्प॑य त्यर्पयति शु॒चा शु॒चा ऽर्प॑यति । \newline
2. अ॒र्प॒य॒ति॒ तस्मा॒त् तस्मा॑ दर्पय त्यर्पयति॒ तस्मा᳚त् । \newline
3. तस्मा᳚थ् स॒माव॑थ् स॒माव॒त् तस्मा॒त् तस्मा᳚थ् स॒माव॑त् । \newline
4. स॒माव॑त् पशू॒नाम् प॑शू॒नाꣳ स॒माव॑थ् स॒माव॑त् पशू॒नाम् । \newline
5. प॒शू॒नाम् प्र॒जाय॑मानानाम् प्र॒जाय॑मानानाम् पशू॒नाम् प॑शू॒नाम् प्र॒जाय॑मानानाम् । \newline
6. प्र॒जाय॑मानाना मार॒ण्या आ॑र॒ण्याः प्र॒जाय॑मानानाम् प्र॒जाय॑मानाना मार॒ण्याः । \newline
7. प्र॒जाय॑मानाना॒मिति॑ प्र - जाय॑मानानाम् । \newline
8. आ॒र॒ण्याः प॒शवः॑ प॒शव॑ आर॒ण्या आ॑र॒ण्याः प॒शवः॑ । \newline
9. प॒शवः॒ कनी॑याꣳसः॒ कनी॑याꣳसः प॒शवः॑ प॒शवः॒ कनी॑याꣳसः । \newline
10. कनी॑याꣳसः शु॒चा शु॒चा कनी॑याꣳसः॒ कनी॑याꣳसः शु॒चा । \newline
11. शु॒चा हि हि शु॒चा शु॒चा हि । \newline
12. ह्यृ॑ता ऋ॒ता हि ह्यृ॑ताः । \newline
13. ऋ॒ता लो॑म॒तो लो॑म॒त ऋ॒ता ऋ॒ता लो॑म॒तः । \newline
14. लो॒म॒तः सꣳ सम् ॅलो॑म॒तो लो॑म॒तः सम् । \newline
15. सम् भ॑रति भरति॒ सꣳ सम् भ॑रति । \newline
16. भ॒र॒ त्यतो ऽतो॑ भरति भर॒ त्यतः॑ । \newline
17. अतो॒ हि ह्यतो ऽतो॒ हि । \newline
18. ह्य॑स्या स्य॒ हि ह्य॑स्य । \newline
19. अ॒स्य॒ मेद्ध्य॒म् मेद्ध्य॑ मस्यास्य॒ मेद्ध्य᳚म् । \newline
20. मेद्ध्य॑म् कृष्णाजि॒नम् कृ॑ष्णाजि॒नम् मेद्ध्य॒म् मेद्ध्य॑म् कृष्णाजि॒नम् । \newline
21. कृ॒ष्णा॒जि॒नम् च॑ च कृष्णाजि॒नम् कृ॑ष्णाजि॒नम् च॑ । \newline
22. कृ॒ष्णा॒जि॒नमिति॑ कृष्ण - अ॒जि॒नम् । \newline
23. च॒ पु॒ष्क॒र॒प॒र्णम् पु॑ष्करप॒र्णम् च॑ च पुष्करप॒र्णम् । \newline
24. पु॒ष्क॒र॒प॒र्णम् च॑ च पुष्करप॒र्णम् पु॑ष्करप॒र्णम् च॑ । \newline
25. पु॒ष्क॒र॒प॒र्णमिति॑ पुष्कर - प॒र्णम् । \newline
26. च॒ सꣳ सम् च॑ च॒ सम् । \newline
27. सꣳ स्तृ॑णाति स्तृणाति॒ सꣳ सꣳ स्तृ॑णाति । \newline
28. स्तृ॒णा॒ती॒य मि॒यꣳ स्तृ॑णाति स्तृणाती॒यम् । \newline
29. इ॒यं ॅवै वा इ॒य मि॒यं ॅवै । \newline
30. वै कृ॑ष्णाजि॒नम् कृ॑ष्णाजि॒नं ॅवै वै कृ॑ष्णाजि॒नम् । \newline
31. कृ॒ष्णा॒जि॒न म॒सा व॒सौ कृ॑ष्णाजि॒नम् कृ॑ष्णाजि॒न म॒सौ । \newline
32. कृ॒ष्णा॒जि॒नमिति॑ कृष्ण - अ॒जि॒नम् । \newline
33. अ॒सौ पु॑ष्करप॒र्णम् पु॑ष्करप॒र्ण म॒सा व॒सौ पु॑ष्करप॒र्णम् । \newline
34. पु॒ष्क॒र॒प॒र्ण मा॒भ्या मा॒भ्याम् पु॑ष्करप॒र्णम् पु॑ष्करप॒र्ण मा॒भ्याम् । \newline
35. पु॒ष्क॒र॒प॒र्णमिति॑ पुष्कर - प॒र्णम् । \newline
36. आ॒भ्या मे॒वैवाभ्या मा॒भ्या मे॒व । \newline
37. ए॒वैन॑ मेन मे॒वैवैन᳚म् । \newline
38. ए॒न॒ मु॒भ॒यत॑ उभ॒यत॑ एन मेन मुभ॒यतः॑ । \newline
39. उ॒भ॒यतः॒ परि॒ पर्यु॑भ॒यत॑ उभ॒यतः॒ परि॑ । \newline
40. परि॑ गृह्णाति गृह्णाति॒ परि॒ परि॑ गृह्णाति । \newline
41. गृ॒ह्णा॒ त्य॒ग्नि र॒ग्निर् गृ॑ह्णाति गृह्णा त्य॒ग्निः । \newline
42. अ॒ग्निर् दे॒वेभ्यो॑ दे॒वेभ्यो॒ ऽग्नि र॒ग्निर् दे॒वेभ्यः॑ । \newline
43. दे॒वेभ्यो॒ निला॑यत॒ निला॑यत दे॒वेभ्यो॑ दे॒वेभ्यो॒ निला॑यत । \newline
44. निला॑यत॒ तम् तम् निला॑यत॒ निला॑यत॒ तम् । \newline
45. त मथ॒र्वा ऽथ॑र्वा॒ तम् त मथ॑र्वा । \newline
46. अथ॒र्वा ऽन्वन् वथ॒र्वा ऽथ॒र्वा ऽनु॑ । \newline
47. अन्व॑पश्य दपश्य॒ दन् वन् व॑पश्यत् । \newline
48. अ॒प॒श्य॒ दथ॒र्वा ऽथ॑र्वा ऽपश्य दपश्य॒ दथ॑र्वा । \newline
49. अथ॑र्वा त्वा॒ त्वा ऽथ॒र्वा ऽथ॑र्वा त्वा । \newline
50. त्वा॒ प्र॒थ॒मः प्र॑थ॒म स्त्वा᳚ त्वा प्रथ॒मः । \newline
51. प्र॒थ॒मो निर् णिष् प्र॑थ॒मः प्र॑थ॒मो निः । \newline
52. निर॑मन्थ दमन्थ॒न् निर् णि र॑मन्थत् । \newline
53. अ॒म॒न्थ॒ द॒ग्ने॒ अ॒ग्ने॒ अ॒म॒न्थ॒ द॒म॒न्थ॒ द॒ग्ने॒ । \newline
54. अ॒ग्न॒ इती त्य॑ग्ने ऽग्न॒ इति॑ । \newline
55. इत्या॑हा॒हे तीत्या॑ह । \newline

\textbf{Ghana Paata } \newline

1. शु॒चा ऽर्प॑य त्यर्पयति शु॒चा शु॒चा ऽर्प॑यति॒ तस्मा॒त् तस्मा॑ दर्पयति शु॒चा शु॒चा ऽर्प॑यति॒ तस्मा᳚त् । \newline
2. अ॒र्प॒य॒ति॒ तस्मा॒त् तस्मा॑ दर्पय त्यर्पयति॒ तस्मा᳚थ् स॒माव॑थ् स॒माव॒त् तस्मा॑ दर्पय त्यर्पयति॒ तस्मा᳚थ् स॒माव॑त् । \newline
3. तस्मा᳚थ् स॒माव॑थ् स॒माव॒त् तस्मा॒त् तस्मा᳚थ् स॒माव॑त् पशू॒नाम् प॑शू॒नाꣳ स॒माव॒त् तस्मा॒त् तस्मा᳚थ् स॒माव॑त् पशू॒नाम् । \newline
4. स॒माव॑त् पशू॒नाम् प॑शू॒नाꣳ स॒माव॑थ् स॒माव॑त् पशू॒नाम् प्र॒जाय॑मानानाम् प्र॒जाय॑मानानाम् पशू॒नाꣳ स॒माव॑थ् स॒माव॑त् पशू॒नाम् प्र॒जाय॑मानानाम् । \newline
5. प॒शू॒नाम् प्र॒जाय॑मानानाम् प्र॒जाय॑मानानाम् पशू॒नाम् प॑शू॒नाम् प्र॒जाय॑मानाना मार॒ण्या आ॑र॒ण्याः प्र॒जाय॑मानानाम् पशू॒नाम् प॑शू॒नाम् प्र॒जाय॑मानाना मार॒ण्याः । \newline
6. प्र॒जाय॑मानाना मार॒ण्या आ॑र॒ण्याः प्र॒जाय॑मानानाम् प्र॒जाय॑मानाना मार॒ण्याः प॒शवः॑ प॒शव॑ आर॒ण्याः प्र॒जाय॑मानानाम् प्र॒जाय॑मानाना मार॒ण्याः प॒शवः॑ । \newline
7. प्र॒जाय॑मानाना॒मिति॑ प्र - जाय॑मानानाम् । \newline
8. आ॒र॒ण्याः प॒शवः॑ प॒शव॑ आर॒ण्या आ॑र॒ण्याः प॒शवः॒ कनी॑याꣳसः॒ कनी॑याꣳसः प॒शव॑ आर॒ण्या आ॑र॒ण्याः प॒शवः॒ कनी॑याꣳसः । \newline
9. प॒शवः॒ कनी॑याꣳसः॒ कनी॑याꣳसः प॒शवः॑ प॒शवः॒ कनी॑याꣳसः शु॒चा शु॒चा कनी॑याꣳसः प॒शवः॑ प॒शवः॒ कनी॑याꣳसः शु॒चा । \newline
10. कनी॑याꣳसः शु॒चा शु॒चा कनी॑याꣳसः॒ कनी॑याꣳसः शु॒चा हि हि शु॒चा कनी॑याꣳसः॒ कनी॑याꣳसः शु॒चा हि । \newline
11. शु॒चा हि हि शु॒चा शु॒चा ह्यृ॑ता ऋ॒ता हि शु॒चा शु॒चा ह्यृ॑ताः । \newline
12. ह्यृ॑ता ऋ॒ता हि ह्यृ॑ता लो॑म॒तो लो॑म॒त ऋ॒ता हि ह्यृ॑ता लो॑म॒तः । \newline
13. ऋ॒ता लो॑म॒तो लो॑म॒त ऋ॒ता ऋ॒ता लो॑म॒तः सꣳ सम् ॅलो॑म॒त ऋ॒ता ऋ॒ता लो॑म॒तः सम् । \newline
14. लो॒म॒तः सꣳ सम् ॅलो॑म॒तो लो॑म॒तः सम् भ॑रति भरति॒ सम् ॅलो॑म॒तो लो॑म॒तः सम् भ॑रति । \newline
15. सम् भ॑रति भरति॒ सꣳ सम् भ॑र॒ त्यतो ऽतो॑ भरति॒ सꣳ सम् भ॑र॒ त्यतः॑ । \newline
16. भ॒र॒ त्यतो ऽतो॑ भरति भर॒ त्यतो॒ हि ह्यतो॑ भरति भर॒ त्यतो॒ हि । \newline
17. अतो॒ हि ह्यतो ऽतो॒ ह्य॑स्यास्य॒ ह्यतो ऽतो॒ ह्य॑स्य । \newline
18. ह्य॑स्यास्य॒ हि ह्य॑स्य॒ मेद्ध्य॒म् मेद्ध्य॑ मस्य॒ हि ह्य॑स्य॒ मेद्ध्य᳚म् । \newline
19. अ॒स्य॒ मेद्ध्य॒म् मेद्ध्य॑ मस्यास्य॒ मेद्ध्य॑म् कृष्णाजि॒नम् कृ॑ष्णाजि॒नम् मेद्ध्य॑ मस्यास्य॒ मेद्ध्य॑म् कृष्णाजि॒नम् । \newline
20. मेद्ध्य॑म् कृष्णाजि॒नम् कृ॑ष्णाजि॒नम् मेद्ध्य॒म् मेद्ध्य॑म् कृष्णाजि॒नम् च॑ च कृष्णाजि॒नम् मेद्ध्य॒म् मेद्ध्य॑म् कृष्णाजि॒नम् च॑ । \newline
21. कृ॒ष्णा॒जि॒नम् च॑ च कृष्णाजि॒नम् कृ॑ष्णाजि॒नम् च॑ पुष्करप॒र्णम् पु॑ष्करप॒र्णम् च॑ कृष्णाजि॒नम् कृ॑ष्णाजि॒नम् च॑ पुष्करप॒र्णम् । \newline
22. कृ॒ष्णा॒जि॒नमिति॑ कृष्ण - अ॒जि॒नम् । \newline
23. च॒ पु॒ष्क॒र॒प॒र्णम् पु॑ष्करप॒र्णम् च॑ च पुष्करप॒र्णम् च॑ च पुष्करप॒र्णम् च॑ च पुष्करप॒र्णम् च॑ । \newline
24. पु॒ष्क॒र॒प॒र्णम् च॑ च पुष्करप॒र्णम् पु॑ष्करप॒र्णम् च॒ सꣳ सम् च॑ पुष्करप॒र्णम् पु॑ष्करप॒र्णम् च॒ सम् । \newline
25. पु॒ष्क॒र॒प॒र्णमिति॑ पुष्कर - प॒र्णम् । \newline
26. च॒ सꣳ सम् च॑ च॒ सꣳ स्तृ॑णाति स्तृणाति॒ सम् च॑ च॒ सꣳ स्तृ॑णाति । \newline
27. सꣳ स्तृ॑णाति स्तृणाति॒ सꣳ सꣳ स्तृ॑णाती॒य मि॒यꣳ स्तृ॑णाति॒ सꣳ सꣳ स्तृ॑णाती॒यम् । \newline
28. स्तृ॒णा॒ती॒य मि॒यꣳ स्तृ॑णाति स्तृणाती॒यं ॅवै वा इ॒यꣳ स्तृ॑णाति स्तृणाती॒यं ॅवै । \newline
29. इ॒यं ॅवै वा इ॒य मि॒यं ॅवै कृ॑ष्णाजि॒नम् कृ॑ष्णाजि॒नं ॅवा इ॒य मि॒यं ॅवै कृ॑ष्णाजि॒नम् । \newline
30. वै कृ॑ष्णाजि॒नम् कृ॑ष्णाजि॒नं ॅवै वै कृ॑ष्णाजि॒न म॒सा व॒सौ कृ॑ष्णाजि॒नं ॅवै वै कृ॑ष्णाजि॒न म॒सौ । \newline
31. कृ॒ष्णा॒जि॒न म॒सा व॒सौ कृ॑ष्णाजि॒नम् कृ॑ष्णाजि॒न म॒सौ पु॑ष्करप॒र्णम् पु॑ष्करप॒र्ण म॒सौ कृ॑ष्णाजि॒नम् कृ॑ष्णाजि॒न म॒सौ पु॑ष्करप॒र्णम् । \newline
32. कृ॒ष्णा॒जि॒नमिति॑ कृष्ण - अ॒जि॒नम् । \newline
33. अ॒सौ पु॑ष्करप॒र्णम् पु॑ष्करप॒र्ण म॒सा व॒सौ पु॑ष्करप॒र्ण मा॒भ्या मा॒भ्याम् पु॑ष्करप॒र्ण म॒सा व॒सौ पु॑ष्करप॒र्ण मा॒भ्याम् । \newline
34. पु॒ष्क॒र॒प॒र्ण मा॒भ्या मा॒भ्याम् पु॑ष्करप॒र्णम् पु॑ष्करप॒र्ण मा॒भ्या मे॒वैवाभ्याम् पु॑ष्करप॒र्णम् पु॑ष्करप॒र्ण मा॒भ्या मे॒व । \newline
35. पु॒ष्क॒र॒प॒र्णमिति॑ पुष्कर - प॒र्णम् । \newline
36. आ॒भ्या मे॒वैवाभ्या मा॒भ्या मे॒वैन॑ मेन मे॒वाभ्या मा॒भ्या मे॒वैन᳚म् । \newline
37. ए॒वैन॑ मेन मे॒वैवैन॑ मुभ॒यत॑ उभ॒यत॑ एन मे॒वैवैन॑ मुभ॒यतः॑ । \newline
38. ए॒न॒ मु॒भ॒यत॑ उभ॒यत॑ एन मेन मुभ॒यतः॒ परि॒ पर्यु॑भ॒यत॑ एन मेन मुभ॒यतः॒ परि॑ । \newline
39. उ॒भ॒यतः॒ परि॒ पर्यु॑भ॒यत॑ उभ॒यतः॒ परि॑ गृह्णाति गृह्णाति॒ पर्यु॑भ॒यत॑ उभ॒यतः॒ परि॑ गृह्णाति । \newline
40. परि॑ गृह्णाति गृह्णाति॒ परि॒ परि॑ गृह्णा त्य॒ग्नि र॒ग्निर् गृ॑ह्णाति॒ परि॒ परि॑ गृह्णा त्य॒ग्निः । \newline
41. गृ॒ह्णा॒ त्य॒ग्नि र॒ग्निर् गृ॑ह्णाति गृह्णा त्य॒ग्निर् दे॒वेभ्यो॑ दे॒वेभ्यो॒ ऽग्निर् गृ॑ह्णाति गृह्णा त्य॒ग्निर् दे॒वेभ्यः॑ । \newline
42. अ॒ग्निर् दे॒वेभ्यो॑ दे॒वेभ्यो॒ ऽग्नि र॒ग्निर् दे॒वेभ्यो॒ निला॑यत॒ निला॑यत दे॒वेभ्यो॒ ऽग्नि र॒ग्निर् दे॒वेभ्यो॒ निला॑यत । \newline
43. दे॒वेभ्यो॒ निला॑यत॒ निला॑यत दे॒वेभ्यो॑ दे॒वेभ्यो॒ निला॑यत॒ तम् तम् निला॑यत दे॒वेभ्यो॑ दे॒वेभ्यो॒ निला॑यत॒ तम् । \newline
44. निला॑यत॒ तम् तम् निला॑यत॒ निला॑यत॒ त मथ॒र्वा ऽथ॑र्वा॒ तम् निला॑यत॒ निला॑यत॒ तमथ॑र्वा । \newline
45. त मथ॒र्वा ऽथ॑र्वा॒ तम् त मथ॒र्वा ऽन्वन् वथ॑र्वा॒ तम् त मथ॒र्वा ऽनु॑ । \newline
46. अथ॒र्वा ऽन्वन् वथ॒र्वा ऽथ॒र्वा ऽन्व॑पश्य दपश्य॒ दन्वथ॒र्वा ऽथ॒र्वा ऽन्व॑पश्यत् । \newline
47. अन्व॑पश्य दपश्य॒ दन् वन् व॑पश्य॒ दथ॒र्वा ऽथ॑र्वा ऽपश्य॒ दन् वन् व॑पश्य॒ दथ॑र्वा । \newline
48. अ॒प॒श्य॒ दथ॒र्वा ऽथ॑र्वा ऽपश्य दपश्य॒ दथ॑र्वा त्वा॒ त्वा ऽथ॑र्वा ऽपश्य दपश्य॒ दथ॑र्वा त्वा । \newline
49. अथ॑र्वा त्वा॒ त्वा ऽथ॒र्वा ऽथ॑र्वा त्वा प्रथ॒मः प्र॑थ॒म स्त्वा ऽथ॒र्वा ऽथ॑र्वा त्वा प्रथ॒मः । \newline
50. त्वा॒ प्र॒थ॒मः प्र॑थ॒म स्त्वा᳚ त्वा प्रथ॒मो निर् णिष् प्र॑थ॒म स्त्वा᳚ त्वा प्रथ॒मो निः । \newline
51. प्र॒थ॒मो निर् णिष् प्र॑थ॒मः प्र॑थ॒मो निर॑मन्थ दमन्थ॒न् निष् प्र॑थ॒मः प्र॑थ॒मो निर॑मन्थत् । \newline
52. निर॑मन्थ दमन्थ॒न् निर् णिर॑मन्थ दग्ने अग्ने अमन्थ॒न् निर् णिर॑मन्थ दग्ने । \newline
53. अ॒म॒न्थ॒ द॒ग्ने॒ अ॒ग्ने॒ अ॒म॒न्थ॒ द॒म॒न्थ॒ द॒ग्न॒ इतीत्य॑ग्ने अमन्थ दमन्थ दग्न॒ इति॑ । \newline
54. अ॒ग्न॒ इतीत्य॑ग्ने ऽग्न॒ इत्या॑हा॒हे त्य॑ग्ने ऽग्न॒ इत्या॑ह । \newline
55. इत्या॑हा॒हेती त्या॑ह॒ यो य आ॒हेती त्या॑ह॒ यः । \newline
\pagebreak
\markright{ TS 5.1.4.4  \hfill https://www.vedavms.in \hfill}

\section{ TS 5.1.4.4 }

\textbf{TS 5.1.4.4 } \newline
\textbf{Samhita Paata} \newline

-ह॒ य ए॒वैन॑म॒न्वप॑श्य॒त् तेनै॒वैनꣳ॒॒ संभ॑रति॒ त्वाम॑ग्ने॒ पुष्क॑रा॒दधीत्या॑ह पुष्करप॒र्णे ह्ये॑न॒मुप॑श्रित॒-मवि॑न्द॒त् तमु॑ त्वा द॒द्ध्यङ्ङृषि॒रित्या॑ह द॒द्ध्यङ्वा आ॑थर्व॒ण-स्ते॑ज॒स्व्या॑सी॒त् तेज॑ ए॒वास्मि॑न् दधाति॒ तमु॑ त्वा पा॒थ्यो वृषेत्या॑ह॒ पूर्व॑मे॒वोदि॒त-मुत्त॑रेणा॒भि गृ॑णाति - [  ] \newline

\textbf{Pada Paata} \newline

आ॒ह॒ । यः । ए॒व । ए॒न॒म् । अ॒न्वप॑श्य॒दित्य॑नु-अप॑श्यत् । तेन॑ । ए॒व । ए॒न॒म् । समिति॑ । भ॒र॒ति॒ । त्वाम् । अ॒ग्ने॒ । पुष्क॑रात् । अधीति॑ । इति॑ । आ॒ह॒ । पु॒ष्क॒रप॒र्ण इति॑ पुष्कर - प॒र्णे । हि । ए॒न॒म् । उप॑श्रित॒मित्युप॑ - श्रि॒त॒म् । अवि॑न्दत् । तम् । उ॒ । त्वा॒ । द॒द्ध्यङ् । ऋषिः॑ । इति॑ । आ॒ह॒ । द॒द्ध्यङ् । वै । आ॒थ॒र्व॒णः । ते॒ज॒स्वी । आ॒सी॒त् । तेजः॑ । ए॒व । अ॒स्मि॒न्न् । द॒धा॒ति॒ । तम् । उ॒ । त्वा॒ । पा॒थ्यः । वृषा᳚ । इति॑ । आ॒ह॒ । पूर्व᳚म् । ए॒व । उ॒दि॒तम् । उत्त॑रे॒णेत्युत् - त॒रे॒ण॒ । अ॒भीति॑ । गृ॒णा॒ति॒ ।  \newline


\textbf{Krama Paata} \newline

आ॒ह॒ यः । य ए॒व । ए॒वैन᳚म् । ए॒न॒म॒न्वप॑श्यत् । अ॒न्वप॑श्य॒त् तेन॑ । अ॒न्वप॑श्य॒दित्य॑नु - अप॑श्यत् । तेनै॒व । ए॒वैन᳚म् । ए॒नꣳ॒॒ सम् । सम् भ॑रति । भ॒र॒ति॒ त्वाम् । त्वाम॑ग्ने । अ॒ग्ने॒ पुष्क॑रात् । पुष्क॑रा॒दधि॑ । अधीति॑ । इत्या॑ह । आ॒ह॒ पु॒ष्क॒र॒प॒र्णे । पु॒ष्क॒र॒प॒र्णे हि । पु॒ष्क॒र॒प॒र्ण इति॑ पुष्कर - प॒र्णे । ह्ये॑नम् । ए॒न॒मुप॑श्रितम् । उप॑श्रित॒मवि॑न्दत् । उप॑श्रित॒मित्युप॑ - श्रि॒त॒म् । अवि॑न्द॒त् तम् । तमु॑ । उ॒ त्वा॒ । त्वा॒ द॒द्ध्यङ्ङ् । द॒द्ध्यङ्ङृषिः॑ । ऋषि॒रिति॑ । इत्या॑ह । आ॒ह॒ द॒द्ध्यङ्ङ् । द॒द्ध्यङ् वै । वा आ॑थर्व॒णः । आ॒थ॒र्व॒णस्ते॑ज॒स्वी । ते॒ज॒स्व्या॑सीत् । आ॒सी॒त् तेजः॑ । तेज॑ ए॒व । ए॒वास्मिन्न्॑ । अ॒स्मि॒न् द॒धा॒ति॒ । द॒धा॒ति॒ तम् । तमु॑ । उ॒ त्वा॒ । त्वा॒ पा॒थ्यः । पा॒थ्यो वृषा᳚ । वृषेति॑ । इत्या॑ह । आ॒ह॒ पूर्व᳚म् । पूर्व॑मे॒व । ए॒वोदि॒तम् । उ॒दि॒तमुत्त॑रेण । उत्त॑रेणा॒भि । उत्त॑रे॒णेत्युत् - त॒रे॒ण॒ । अ॒भि गृ॑णाति । गृ॒णा॒ति॒ च॒त॒सृभिः॑ \newline

\textbf{Jatai Paata} \newline

1. आ॒ह॒ यो य आ॑हाह॒ यः । \newline
2. य ए॒वैव यो य ए॒व । \newline
3. ए॒वैन॑ मेन मे॒वैवैन᳚म् । \newline
4. ए॒न॒ म॒न्वप॑श्य द॒न्वप॑श्य देन मेन म॒न्वप॑श्यत् । \newline
5. अ॒न्वप॑श्य॒त् तेन॒ तेना॒ न्वप॑श्य द॒न्वप॑श्य॒त् तेन॑ । \newline
6. अ॒न्वप॑श्य॒दित्य॑नु - अप॑श्यत् । \newline
7. तेनै॒वैव तेन॒ तेनै॒व । \newline
8. ए॒वैन॑ मेन मे॒वैवैन᳚म् । \newline
9. ए॒नꣳ॒॒ सꣳ स मे॑न मेनꣳ॒॒ सम् । \newline
10. सम् भ॑रति भरति॒ सꣳ सम् भ॑रति । \newline
11. भ॒र॒ति॒ त्वाम् त्वाम् भ॑रति भरति॒ त्वाम् । \newline
12. त्वा म॑ग्ने अग्ने॒ त्वाम् त्वा म॑ग्ने । \newline
13. अ॒ग्ने॒ पुष्क॑रा॒त् पुष्क॑रा दग्ने अग्ने॒ पुष्क॑रात् । \newline
14. पुष्क॑रा॒ दध्यधि॒ पुष्क॑रा॒त् पुष्क॑रा॒ दधि॑ । \newline
15. अधीती त्यध्य धीति॑ । \newline
16. इत्या॑हा॒हे तीत्या॑ह । \newline
17. आ॒ह॒ पु॒ष्क॒र॒प॒र्णे पु॑ष्करप॒र्ण आ॑हाह पुष्करप॒र्णे । \newline
18. पु॒ष्क॒र॒प॒र्णे हि हि पु॑ष्करप॒र्णे पु॑ष्करप॒र्णे हि । \newline
19. पु॒ष्क॒र॒प॒र्ण इति॑ पुष्कर - प॒र्णे । \newline
20. ह्ये॑न मेनꣳ॒॒ हि ह्ये॑नम् । \newline
21. ए॒न॒ मुप॑श्रित॒ मुप॑श्रित मेन मेन॒ मुप॑श्रितम् । \newline
22. उप॑श्रित॒ मवि॑न् द॒दवि॑न्द॒ दुप॑श्रित॒ मुप॑श्रित॒ मवि॑न्दत् । \newline
23. उप॑श्रित॒मित्युप॑ - श्रि॒त॒म् । \newline
24. अवि॑न्द॒त् तम् त मवि॑न्द॒ दवि॑न्द॒त् तम् । \newline
25. तमु॑ वु॒ तम् तमु॑ । \newline
26. उ॒ त्वा॒ त्व॒ वु॒ त्वा॒ । \newline
27. त्वा॒ द॒द्ध्यङ् द॒द्ध्यङ् त्वा᳚ त्वा द॒द्ध्यङ् । \newline
28. द॒द्ध्यङ् ङृषि॒र्॒. ऋषि॑र् द॒द्ध्यङ् द॒द्ध्यङ् ङृषिः॑ । \newline
29. ऋषि॒रिती त्यृषि॒र्॒. ऋषि॒ रिति॑ । \newline
30. इत्या॑हा॒हे तीत्या॑ह । \newline
31. आ॒ह॒ द॒द्ध्यङ् द॒द्ध्यङ् ङा॑हाह द॒द्ध्यङ् । \newline
32. द॒द्ध्यङ्. वै वै द॒द्ध्यङ् द॒द्ध्यङ्. वै । \newline
33. वा आ॑थर्व॒ण आ॑थर्व॒णो वै वा आ॑थर्व॒णः । \newline
34. आ॒थ॒र्व॒ण स्ते॑ज॒स्वी ते॑ज॒ स्व्या॑थर्व॒ण आ॑थर्व॒ण स्ते॑ज॒स्वी । \newline
35. ते॒ज॒ स्व्या॑सी दासीत् तेज॒स्वी ते॑ज॒ स्व्या॑सीत् । \newline
36. आ॒सी॒त् तेज॒ स्तेज॑ आसी दासी॒त् तेजः॑ । \newline
37. तेज॑ ए॒वैव तेज॒ स्तेज॑ ए॒व । \newline
38. ए॒वास्मि॑न् नस्मिन् ने॒वैवास्मिन्न्॑ । \newline
39. अ॒स्मि॒न् द॒धा॒ति॒ द॒धा॒ त्य॒स्मि॒न् न॒स्मि॒न् द॒धा॒ति॒ । \newline
40. द॒धा॒ति॒ तम् तम् द॑धाति दधाति॒ तम् । \newline
41. त मु॑ वु॒ तम् त मु॑ । \newline
42. उ॒ त्वा॒ त्व॒ वु॒ त्वा॒ । \newline
43. त्वा॒ पा॒थ्यः पा॒थ्य स्त्वा᳚ त्वा पा॒थ्यः । \newline
44. पा॒थ्यो वृषा॒ वृषा॑ पा॒थ्यः पा॒थ्यो वृषा᳚ । \newline
45. वृषेतीति॒ वृषा॒ वृषेति॑ । \newline
46. इत्या॑हा॒हे तीत्या॑ह । \newline
47. आ॒ह॒ पूर्व॒म् पूर्व॑ माहाह॒ पूर्व᳚म् । \newline
48. पूर्व॑ मे॒वैव पूर्व॒म् पूर्व॑ मे॒व । \newline
49. ए॒वोदि॒त मु॑दि॒त मे॒वै वोदि॒तम् । \newline
50. उ॒दि॒त मुत्त॑रे॒ णोत्त॑रे णोदि॒त मु॑दि॒त मुत्त॑रेण । \newline
51. उत्त॑रेणा॒ भ्य॑भ्युत्त॑रे॒ णोत्त॑रेणा॒भि । \newline
52. उत्त॑रे॒णेत्युत् - त॒रे॒ण॒ । \newline
53. अ॒भि गृ॑णाति गृणा त्य॒भ्य॑भि गृ॑णाति । \newline
54. गृ॒णा॒ति॒ च॒त॒सृभि॑ श्चत॒सृभि॑र् गृणाति गृणाति चत॒सृभिः॑ । \newline

\textbf{Ghana Paata } \newline

1. आ॒ह॒ यो य आ॑हाह॒ य ए॒वैव य आ॑हाह॒ य ए॒व । \newline
2. य ए॒वैव यो य ए॒वैन॑ मेन मे॒व यो य ए॒वैन᳚म् । \newline
3. ए॒वैन॑ मेन मे॒वैवैन॑ म॒न्वप॑श्य द॒न्वप॑श्य देन मे॒वैवैन॑ म॒न्वप॑श्यत् । \newline
4. ए॒न॒ म॒न्वप॑श्य द॒न्वप॑श्यदेन मेन म॒न्वप॑श्य॒त् तेन॒ तेना॒ न्वप॑श्यदेन मेन म॒न्वप॑श्य॒त् तेन॑ । \newline
5. अ॒न्वप॑श्य॒त् तेन॒ तेना॒ न्वप॑श्य द॒न्वप॑श्य॒त् तेनै॒वैव तेना॒ न्वप॑श्य द॒न्वप॑श्य॒त् तेनै॒व । \newline
6. अ॒न्वप॑श्य॒दित्य॑नु - अप॑श्यत् । \newline
7. तेनै॒वैव तेन॒ तेनै॒वैन॑ मेन मे॒व तेन॒ तेनै॒वैन᳚म् । \newline
8. ए॒वैन॑ मेन मे॒वैवैनꣳ॒॒ सꣳ स मे॑न मे॒वैवैनꣳ॒॒ सम् । \newline
9. ए॒नꣳ॒॒ सꣳ स मे॑न मेनꣳ॒॒ सम् भ॑रति भरति॒ स मे॑न मेनꣳ॒॒ सम् भ॑रति । \newline
10. सम् भ॑रति भरति॒ सꣳ सम् भ॑रति॒ त्वाम् त्वाम् भ॑रति॒ सꣳ सम् भ॑रति॒ त्वाम् । \newline
11. भ॒र॒ति॒ त्वाम् त्वाम् भ॑रति भरति॒ त्वा म॑ग्ने अग्ने॒ त्वाम् भ॑रति भरति॒ त्वा म॑ग्ने । \newline
12. त्वा म॑ग्ने अग्ने॒ त्वाम् त्वा म॑ग्ने॒ पुष्क॑रा॒त् पुष्क॑रा दग्ने॒ त्वाम् त्वा म॑ग्ने॒ पुष्क॑रात् । \newline
13. अ॒ग्ने॒ पुष्क॑रा॒त् पुष्क॑रा दग्ने अग्ने॒ पुष्क॑रा॒ दध्यधि॒ पुष्क॑रा दग्ने अग्ने॒ पुष्क॑रा॒ दधि॑ । \newline
14. पुष्क॑रा॒ दध्यधि॒ पुष्क॑रा॒त् पुष्क॑रा॒ दधीती त्यधि॒ पुष्क॑रा॒त् पुष्क॑रा॒ दधीति॑ । \newline
15. अधीती त्यध्यधी त्या॑हा॒हे त्यध्यधी त्या॑ह । \newline
16. इत्या॑हा॒हेती त्या॑ह पुष्करप॒र्णे पु॑ष्करप॒र्ण आ॒हेती त्या॑ह पुष्करप॒र्णे । \newline
17. आ॒ह॒ पु॒ष्क॒र॒प॒र्णे पु॑ष्करप॒र्ण आ॑हाह पुष्करप॒र्णे हि हि पु॑ष्करप॒र्ण आ॑हाह पुष्करप॒र्णे हि । \newline
18. पु॒ष्क॒र॒प॒र्णे हि हि पु॑ष्करप॒र्णे पु॑ष्करप॒र्णे ह्ये॑न मेनꣳ॒॒ हि पु॑ष्करप॒र्णे पु॑ष्करप॒र्णे ह्ये॑नम् । \newline
19. पु॒ष्क॒र॒प॒र्ण इति॑ पुष्कर - प॒र्णे । \newline
20. ह्ये॑न मेनꣳ॒॒ हि ह्ये॑न॒ मुप॑श्रित॒ मुप॑श्रित मेनꣳ॒॒ हि ह्ये॑न॒ मुप॑श्रितम् । \newline
21. ए॒न॒ मुप॑श्रित॒ मुप॑श्रित मेन मेन॒ मुप॑श्रित॒ मवि॑न्द॒ दवि॑न्द॒ दुप॑श्रित मेन मेन॒ मुप॑श्रित॒ मवि॑न्दत् । \newline
22. उप॑श्रित॒ मवि॑न्द॒ दवि॑न्द॒ दुप॑श्रित॒ मुप॑श्रित॒ मवि॑न्द॒त् तम् त मवि॑न्द॒ दुप॑श्रित॒ मुप॑श्रित॒ मवि॑न्द॒त् तम् । \newline
23. उप॑श्रित॒मित्युप॑ - श्रि॒त॒म् । \newline
24. अवि॑न्द॒त् तम् त मवि॑न्द॒ दवि॑न्द॒त् त मु॑ वु॒ त मवि॑न्द॒ दवि॑न्द॒त् त मु॑ । \newline
25. त मु॑ वु॒ तम् त मु॑ त्वा त्वो॒ तम् त मु॑ त्वा । \newline
26. उ॒ त्वा॒ त्व॒ वु॒ त्वा॒ द॒द्ध्यङ् द॒द्ध्यङ् त्व॑ वु त्वा द॒द्ध्यङ् । \newline
27. त्वा॒ द॒द्ध्यङ् द॒द्ध्यङ् त्वा᳚ त्वा द॒द्ध्यङ् ङृषि॒र्॒. ऋषि॑र् द॒द्ध्यङ् त्वा᳚ त्वा द॒द्ध्यङ् ङृषिः॑ । \newline
28. द॒द्ध्यङ् ङृषि॒र्॒. ऋषि॑र् द॒द्ध्यङ् द॒द्ध्यङ् ङृषि॒रिती त्यृषि॑र् द॒द्ध्यङ् द॒द्ध्यङ् ङृषि॒रिति॑ । \newline
29. ऋषि॒रिती त्यृषि॒र्॒. ऋषि॒रि त्या॑हा॒हे त्यृषि॒र्॒. ऋषि॒ रित्या॑ह । \newline
30. इत्या॑हा॒हेती त्या॑ह द॒द्ध्यङ् द॒द्ध्यङ् ङा॒हे तीत्या॑ह द॒द्ध्यङ् । \newline
31. आ॒ह॒ द॒द्ध्यङ् द॒द्ध्यङ् ङा॑हाह द॒द्ध्यङ्. वै वै द॒द्ध्यङ् ङा॑हाह द॒द्ध्यङ्. वै । \newline
32. द॒द्ध्यङ्. वै वै द॒द्ध्यङ् द॒द्ध्यङ्. वा आ॑थर्व॒ण आ॑थर्व॒णो वै द॒द्ध्यङ् द॒द्ध्यङ्. वा आ॑थर्व॒णः । \newline
33. वा आ॑थर्व॒ण आ॑थर्व॒णो वै वा आ॑थर्व॒ण स्ते॑ज॒स्वी ते॑ज॒ स्व्या॑थर्व॒णो वै वा आ॑थर्व॒ण स्ते॑ज॒स्वी । \newline
34. आ॒थ॒र्व॒ण स्ते॑ज॒स्वी ते॑ज॒ स्व्या॑थर्व॒ण आ॑थर्व॒ण स्ते॑ज॒ स्व्या॑सी दासीत् तेज॒ स्व्या॑थर्व॒ण आ॑थर्व॒ण स्ते॑ज॒ स्व्या॑सीत् । \newline
35. ते॒ज॒ स्व्या॑सी दासीत् तेज॒स्वी ते॑ज॒ स्व्या॑सी॒त् तेज॒ स्तेज॑ आसीत् तेज॒स्वी ते॑ज॒ स्व्या॑सी॒त् तेजः॑ । \newline
36. आ॒सी॒त् तेज॒ स्तेज॑ आसी दासी॒त् तेज॑ ए॒वैव तेज॑ आसी दासी॒त् तेज॑ ए॒व । \newline
37. तेज॑ ए॒वैव तेज॒ स्तेज॑ ए॒वास्मि॑न् नस्मिन् ने॒व तेज॒ स्तेज॑ ए॒वास्मिन्न्॑ । \newline
38. ए॒वास्मि॑न् नस्मिन् ने॒वैवास्मि॑न् दधाति दधा त्यस्मिन् ने॒वैवास्मि॑न् दधाति । \newline
39. अ॒स्मि॒न् द॒धा॒ति॒ द॒धा॒ त्य॒स्मि॒न् न॒स्मि॒न् द॒धा॒ति॒ तम् तम् द॑धा त्यस्मिन् नस्मिन् दधाति॒ तम् । \newline
40. द॒धा॒ति॒ तम् तम् द॑धाति दधाति॒ त मु॑ वु॒ तम् द॑धाति दधाति॒ त मु॑ । \newline
41. त मु॑ वु॒ तम् त मु॑ त्वा त्वो॒ तम् त मु॑ त्वा । \newline
42. उ॒ त्वा॒ त्व॒ वु॒ त्वा॒ पा॒थ्यः पा॒थ्य स्त्व॑ वु त्वा पा॒थ्यः । \newline
43. त्वा॒ पा॒थ्यः पा॒थ्य स्त्वा᳚ त्वा पा॒थ्यो वृषा॒ वृषा॑ पा॒थ्य स्त्वा᳚ त्वा पा॒थ्यो वृषा᳚ । \newline
44. पा॒थ्यो वृषा॒ वृषा॑ पा॒थ्यः पा॒थ्यो वृषेतीति॒ वृषा॑ पा॒थ्यः पा॒थ्यो वृषेति॑ । \newline
45. वृषेतीति॒ वृषा॒ वृषे त्या॑हा॒हेति॒ वृषा॒ वृषेत्या॑ह । \newline
46. इत्या॑हा॒हेती त्या॑ह॒ पूर्व॒म् पूर्व॑ मा॒हेती त्या॑ह॒ पूर्व᳚म् । \newline
47. आ॒ह॒ पूर्व॒म् पूर्व॑ माहाह॒ पूर्व॑ मे॒वैव पूर्व॑ माहाह॒ पूर्व॑ मे॒व । \newline
48. पूर्व॑ मे॒वैव पूर्व॒म् पूर्व॑ मे॒वोदि॒त मु॑दि॒त मे॒व पूर्व॒म् पूर्व॑ मे॒वोदि॒तम् । \newline
49. ए॒वोदि॒त मु॑दि॒त मे॒वैवोदि॒त मुत्त॑रे॒ णोत्त॑रे णोदि॒त मे॒वैवोदि॒त मुत्त॑रेण । \newline
50. उ॒दि॒त मुत्त॑रे॒ णोत्त॑रे णोदि॒त मु॑दि॒त मुत्त॑रेणा॒ भ्य॑ भ्युत्त॑रे णोदि॒त मु॑दि॒त मुत्त॑रेणा॒भि । \newline
51. उत्त॑रेणा॒ भ्य॑ भ्युत्त॑रे॒ णोत्त॑रेणा॒भि गृ॑णाति गृणा त्य॒भ्युत्त॑रे॒ णोत्त॑रेणा॒भि गृ॑णाति । \newline
52. उत्त॑रे॒णेत्युत् - त॒रे॒ण॒ । \newline
53. अ॒भि गृ॑णाति गृणा त्य॒भ्य॑भि गृ॑णाति चत॒सृभि॑ श्चत॒सृभि॑र् गृणा त्य॒भ्य॑भि गृ॑णाति चत॒सृभिः॑ । \newline
54. गृ॒णा॒ति॒ च॒त॒सृभि॑ श्चत॒सृभि॑र् गृणाति गृणाति चत॒सृभिः॒ सꣳ सम् च॑त॒सृभि॑र् गृणाति गृणाति चत॒सृभिः॒ सम् । \newline
\pagebreak
\markright{ TS 5.1.4.5  \hfill https://www.vedavms.in \hfill}

\section{ TS 5.1.4.5 }

\textbf{TS 5.1.4.5 } \newline

\textbf{Pada Paata} \newline

च॒त॒सृभि॒रिति॑ चत॒सृ-भिः॒ । समिति॑ । भ॒र॒ति॒ । च॒त्वारि॑ । छन्दाꣳ॑सि । छन्दो॑भि॒रिति॒ छन्दः॑-भिः॒ । ए॒व । गा॒य॒त्रीभिः॑ । ब्रा॒ह्म॒णस्य॑ । गा॒य॒त्रः । हि । ब्रा॒ह्म॒णः । त्रि॒ष्टुग्भि॒रिति॑ त्रि॒ष्टुक् - भिः॒ । रा॒ज॒न्य॑स्य । त्रैष्टु॑भः । हि । रा॒ज॒न्यः॑ । यम् । का॒मये॑त । वसी॑यान् । स्या॒त् । इति॑ । उ॒भयी॑भिः । तस्य॑ । समिति॑ । भ॒रे॒त् । तेजः॑ । च॒ । ए॒व । अ॒स्मै॒ । इ॒न्द्रि॒यम् । च॒ । स॒मीची॒ इति॑ । द॒धा॒ति॒ । अ॒ष्टा॒भिः । समिति॑ । भ॒र॒ति॒ । अ॒ष्टाक्ष॒रेत्य॒ष्टा- अ॒क्ष॒रा॒ । गा॒य॒त्री । गा॒य॒त्रः । अ॒ग्निः । यावान्॑ । ए॒व । अ॒ग्निः । तम् । समिति॑ । भ॒र॒ति॒ । सीद॑ । हो॒तः॒ । इति॑ ( ) । आ॒ह॒ । दे॒वताः᳚ । ए॒व । अ॒स्मै॒ । समिति॑ । सा॒द॒य॒ति॒ । नीति॑ । होता᳚ । इति॑ । म॒नु॒ष्यान्॑ । समिति॑ । सी॒द॒स्व॒ । इति॑ । वयाꣳ॑सि । जनि॑ष्व । हि । जेन्यः॑ । अग्रे᳚ । अह्ना᳚म् । इति॑ । आ॒ह॒ । दे॒व॒म॒नु॒ष्यानिति॑ देव - म॒नु॒ष्यान् । ए॒व । अ॒स्मै॒ । सꣳस॑न्ना॒निति॒ सं - स॒न्ना॒न् । प्रेति॑ । ज॒न॒य॒ति॒ ॥  \newline


\textbf{Krama Paata} \newline

च॒त॒सृभिः॒ सम् । च॒त॒सृभि॒रिति॑ चत॒सृ - भिः॒ । सम् भ॑रति । भ॒र॒ति॒ च॒त्वारि॑ । च॒त्वारि॒ छन्दाꣳ॑सि । छन्दाꣳ॑सि॒ छन्दो॑भिः । छन्दो॑भिरे॒व । छन्दो॑भि॒रिति॒ छन्दः॑ - भिः॒ । ए॒व गा॑य॒त्रीभिः॑ । गा॒य॒त्रीभि॑र् ब्राह्म॒णस्य॑ । ब्रा॒ह्म॒णस्य॑ गाय॒त्रः । गा॒य॒त्रो हि । हि ब्रा᳚ह्म॒णः । ब्रा॒ह्म॒णस्त्रि॒ष्टुग्भिः॑ । त्रि॒ष्टुग्भी॑ राज॒न्य॑स्य । त्रि॒ष्टुग्भि॒रिति॑ त्रि॒ष्टुक् - भिः॒ । रा॒ज॒न्य॑स्य॒ त्रैष्टु॑भः । त्रैष्टु॑भो॒ हि । हि रा॑ज॒न्यः॑ । रा॒ज॒न्यो॑ यम् । यम् का॒मये॑त । का॒मये॑त॒ वसी॑यान् । वसी॑यान्थ् स्यात् । स्या॒दिति॑ । इत्यु॒भयी॑भिः । उ॒भयी॑भि॒स्तस्य॑ । तस्य॒ सम् । सम् भ॑रेत् । भ॒रे॒त् तेजः॑ । तेज॑श्च । चै॒व । ए॒वास्मै᳚ । अ॒स्मा॒ इ॒न्द्रि॒यम् । इ॒न्द्रि॒यम् च॑ । च॒ स॒मीची᳚ । स॒मीची॑ दधाति । स॒मीची॒ इति॑ स॒मीची᳚ । द॒धा॒त्य॒ष्टा॒भिः । अ॒ष्टा॒भिः सम् । सम् भ॑रति । भ॒र॒त्य॒ष्टाक्ष॑रा । अ॒ष्टाक्ष॑रा गाय॒त्री । अ॒ष्टाक्ष॒रेत्य॒ष्टा - अ॒क्ष॒रा॒ । गा॒य॒त्री गा॑य॒त्रः । गा॒य॒त्रो᳚ऽग्निः । अ॒ग्निर् यावान्॑ । यावा॑ने॒व । ए॒वाग्निः । अ॒ग्निस्तम् । तꣳ सम् । सम् भ॑रति । भ॒र॒ति॒ सीद॑ । सीद॑ होतः । हो॒त॒रिति॑ ( ) । इत्या॑ह । आ॒ह॒ दे॒वताः᳚ । दे॒वता॑ ए॒व । ए॒वास्मै᳚ । अ॒स्मै॒ सम् । सꣳ सा॑दयति । सा॒द॒य॒ति॒ नि । नि होता᳚ । होतेति॑ । इति॑ मनु॒ष्यान्॑ । म॒नु॒ष्या᳚न्थ् सम् । सꣳ सी॑दस्व । सी॒द॒स्वेति॑ । इति॒ वयाꣳ॑सि । वयाꣳ॑सि॒ जनि॑ष्व । जनि॑ष्वा॒ हि । हि जेन्यः॑ । जेन्यो॒ अग्रे᳚ । अग्रे॒ अह्ना᳚म् । अह्ना॒मिति॑ । इत्या॑ह । आ॒ह॒ दे॒व॒म॒नु॒ष्यान् । दे॒व॒म॒नु॒ष्याने॒व । दे॒व॒म॒नु॒ष्यानिति॑ देव - म॒नु॒ष्यान् । ए॒वास्मै᳚ । अ॒स्मै॒ सꣳस॑न्नान् । सꣳस॑न्ना॒न् प्र । सꣳस॑न्ना॒निति॒ सम् - स॒न्ना॒न्॒ । प्र ज॑नयति । ज॒न॒य॒तीति॑ जनयति । \newline

\textbf{Jatai Paata} \newline

1. च॒त॒सृभिः॒ सꣳ सम् च॑त॒सृभि॑ श्चत॒सृभिः॒ सम् । \newline
2. च॒त॒सृभि॒रिति॑ चत॒सृ - भिः॒ । \newline
3. सम् भ॑रति भरति॒ सꣳ सम् भ॑रति । \newline
4. भ॒र॒ति॒ च॒त्वारि॑ च॒त्वारि॑ भरति भरति च॒त्वारि॑ । \newline
5. च॒त्वारि॒ छन्दाꣳ॑सि॒ छन्दाꣳ॑सि च॒त्वारि॑ च॒त्वारि॒ छन्दाꣳ॑सि । \newline
6. छन्दाꣳ॑सि॒ छन्दो॑भि॒ श्छन्दो॑भि॒ श्छन्दाꣳ॑सि॒ छन्दाꣳ॑सि॒ छन्दो॑भिः । \newline
7. छन्दो॑भि रे॒वैव छन्दो॑भि॒ श्छन्दो॑भि रे॒व । \newline
8. छन्दो॑भि॒रिति॒ छन्दः॑ - भिः॒ । \newline
9. ए॒व गा॑य॒त्रीभि॑र् गाय॒त्रीभि॑ रे॒वैव गा॑य॒त्रीभिः॑ । \newline
10. गा॒य॒त्रीभि॑र् ब्राह्म॒णस्य॑ ब्राह्म॒णस्य॑ गाय॒त्रीभि॑र् गाय॒त्रीभि॑र् ब्राह्म॒णस्य॑ । \newline
11. ब्रा॒ह्म॒णस्य॑ गाय॒त्रो गा॑य॒त्रो ब्रा᳚ह्म॒णस्य॑ ब्राह्म॒णस्य॑ गाय॒त्रः । \newline
12. गा॒य॒त्रो हि हि गा॑य॒त्रो गा॑य॒त्रो हि । \newline
13. हि ब्रा᳚ह्म॒णो ब्रा᳚ह्म॒णो हि हि ब्रा᳚ह्म॒णः । \newline
14. ब्रा॒ह्म॒ण स्त्रि॒ष्टुग्भि॑ स्त्रि॒ष्टुग्भि॑र् ब्राह्म॒णो ब्रा᳚ह्म॒ण स्त्रि॒ष्टुग्भिः॑ । \newline
15. त्रि॒ष्टुग्भी॑ राज॒न्य॑स्य राज॒न्य॑स्य त्रि॒ष्टुग्भि॑ स्त्रि॒ष्टुग्भी॑ राज॒न्य॑स्य । \newline
16. त्रि॒ष्टुग्भि॒रिति॑ त्रि॒ष्टुक् - भिः॒ । \newline
17. रा॒ज॒न्य॑स्य॒ त्रैष्टु॑भ॒ स्त्रैष्टु॑भो राज॒न्य॑स्य राज॒न्य॑स्य॒ त्रैष्टु॑भः । \newline
18. त्रैष्टु॑भो॒ हि हि त्रैष्टु॑भ॒ स्त्रैष्टु॑भो॒ हि । \newline
19. हि रा॑ज॒न्यो॑ राज॒न्यो॑ हि हि रा॑ज॒न्यः॑ । \newline
20. रा॒ज॒न्यो॑ यं ॅयꣳ रा॑ज॒न्यो॑ राज॒न्यो॑ यम् । \newline
21. यम् का॒मये॑त का॒मये॑त॒ यं ॅयम् का॒मये॑त । \newline
22. का॒मये॑त॒ वसी॑या॒न्॒. वसी॑यान् का॒मये॑त का॒मये॑त॒ वसी॑यान् । \newline
23. वसी॑यान् थ्स्याथ् स्या॒द् वसी॑या॒न्॒. वसी॑यान् थ्स्यात् । \newline
24. स्या॒दितीति॑ स्याथ् स्या॒दिति॑ । \newline
25. इत्यु॒भयी॑भि रु॒भयी॑भि॒ रिती त्यु॒भयी॑भिः । \newline
26. उ॒भयी॑भि॒ स्तस्य॒ तस्यो॒भयी॑भि रु॒भयी॑भि॒ स्तस्य॑ । \newline
27. तस्य॒ सꣳ सम् तस्य॒ तस्य॒ सम् । \newline
28. सम् भ॑रेद् भरे॒थ् सꣳ सम् भ॑रेत् । \newline
29. भ॒रे॒त् तेज॒ स्तेजो॑ भरेद् भरे॒त् तेजः॑ । \newline
30. तेज॑श्च च॒ तेज॒ स्तेज॑श्च । \newline
31. चै॒वैव च॑ चै॒व । \newline
32. ए॒वास्मा॑ अस्मा ए॒वैवास्मै᳚ । \newline
33. अ॒स्मा॒ इ॒न्द्रि॒य मि॑न्द्रि॒य म॑स्मा अस्मा इन्द्रि॒यम् । \newline
34. इ॒न्द्रि॒यम् च॑ चेन्द्रि॒य मि॑न्द्रि॒यम् च॑ । \newline
35. च॒ स॒मीची॑ स॒मीची॑ च च स॒मीची᳚ । \newline
36. स॒मीची॑ दधाति दधाति स॒मीची॑ स॒मीची॑ दधाति । \newline
37. स॒मीची॒ इति॑ स॒मीची᳚ । \newline
38. द॒धा॒ त्य॒ष्टा॒भि र॑ष्टा॒भिर् द॑धाति दधा त्यष्टा॒भिः । \newline
39. अ॒ष्टा॒भिः सꣳ स म॑ष्टा॒भि र॑ष्टा॒भिः सम् । \newline
40. सम् भ॑रति भरति॒ सꣳ सम् भ॑रति । \newline
41. भ॒र॒ त्य॒ष्टाक्ष॑रा॒ ऽष्टाक्ष॑रा भरति भर त्य॒ष्टाक्ष॑रा । \newline
42. अ॒ष्टाक्ष॑रा गाय॒त्री गा॑य॒ त्र्य॑ष्टाक्ष॑रा॒ ऽष्टाक्ष॑रा गाय॒त्री । \newline
43. अ॒ष्टाक्ष॒रेत्य॒ष्टा - अ॒क्ष॒रा॒ । \newline
44. गा॒य॒त्री गा॑य॒त्रो गा॑य॒त्रो गा॑य॒त्री गा॑य॒त्री गा॑य॒त्रः । \newline
45. गा॒य॒त्रो᳚ ऽग्नि र॒ग्निर् गा॑य॒त्रो गा॑य॒त्रो᳚ ऽग्निः । \newline
46. अ॒ग्निर् यावा॒न्॒. यावा॑ न॒ग्नि र॒ग्निर् यावान्॑ । \newline
47. यावा॑ ने॒वैव यावा॒न्॒. यावा॑ ने॒व । \newline
48. ए॒वाग्नि र॒ग्नि रे॒वैवाग्निः । \newline
49. अ॒ग्नि स्तम् त म॒ग्नि र॒ग्नि स्तम् । \newline
50. तꣳ सꣳ सम् तम् तꣳ सम् । \newline
51. सम् भ॑रति भरति॒ सꣳ सम् भ॑रति । \newline
52. भ॒र॒ति॒ सीद॒ सीद॑ भरति भरति॒ सीद॑ । \newline
53. सीद॑ होतर्. होतः॒ सीद॒ सीद॑ होतः । \newline
54. हो॒त॒ रितीति॑ होतर्. होत॒ रिति॑ । \newline
55. इत्या॑हा॒हे तीत्या॑ह । \newline
56. आ॒ह॒ दे॒वता॑ दे॒वता॑ आहाह दे॒वताः᳚ । \newline
57. दे॒वता॑ ए॒वैव दे॒वता॑ दे॒वता॑ ए॒व । \newline
58. ए॒वास्मा॑ अस्मा ए॒वैवास्मै᳚ । \newline
59. अ॒स्मै॒ सꣳ स म॑स्मा अस्मै॒ सम् । \newline
60. सꣳ सा॑दयति सादयति॒ सꣳ सꣳ सा॑दयति । \newline
61. सा॒द॒य॒ति॒ नि नि षा॑दयति सादयति॒ नि । \newline
62. नि होता॒ होता॒ नि नि होता᳚ । \newline
63. होतेतीति॒ होता॒ होतेति॑ । \newline
64. इति॑ मनु॒ष्या᳚न् मनु॒ष्या॑ नितीति॑ मनु॒ष्यान्॑ । \newline
65. म॒नु॒ष्या᳚न् थ्सꣳ सम् म॑नु॒ष्या᳚न् मनु॒ष्या᳚न् थ्सम् । \newline
66. सꣳ सी॑दस्व सीदस्व॒ सꣳ सꣳ सी॑दस्व । \newline
67. सी॒द॒स्वे तीति॑ सीदस्व सीद॒स्वेति॑ । \newline
68. इति॒ वयाꣳ॑सि॒ वयाꣳ॒॒सीतीति॒ वयाꣳ॑सि । \newline
69. वयाꣳ॑सि॒ जनि॑ष्व॒ जनि॑ष्व॒ वयाꣳ॑सि॒ वयाꣳ॑सि॒ जनि॑ष्व । \newline
70. जनि॑ष्वा॒ हि हि जनि॑ष्व॒ जनि॑ष्वा॒ हि । \newline
71. हि जेन्यो॒ जेन्यो॒ हि हि जेन्यः॑ । \newline
72. जेन्यो॒ अग्रे॒ अग्रे॒ जेन्यो॒ जेन्यो॒ अग्रे᳚ । \newline
73. अग्रे॒ अह्ना॒ मह्ना॒ मग्रे॒ अग्रे॒ अह्ना᳚म् । \newline
74. अह्ना॒ मितीत्यह्ना॒ मह्ना॒ मिति॑ । \newline
75. इत्या॑हा॒हे तीत्या॑ह । \newline
76. आ॒ह॒ दे॒व॒म॒नु॒ष्यान् दे॑वमनु॒ष्या ना॑हाह देवमनु॒ष्यान् । \newline
77. दे॒व॒म॒नु॒ष्या ने॒वैव दे॑वमनु॒ष्यान् दे॑वमनु॒ष्या ने॒व । \newline
78. दे॒व॒म॒नु॒ष्यानिति॑ देव - म॒नु॒ष्यान् । \newline
79. ए॒वास्मा॑ अस्मा ए॒वैवास्मै᳚ । \newline
80. अ॒स्मै॒ सꣳस॑न्ना॒न् थ्सꣳस॑न्ना नस्मा अस्मै॒ सꣳस॑न्नान् । \newline
81. सꣳस॑न्ना॒न् प्र प्र सꣳस॑न्ना॒न् थ्सꣳस॑न्ना॒न् प्र । \newline
82. सꣳस॑न्ना॒निति॒ सं - स॒न्ना॒न् । \newline
83. प्र ज॑नयति जनयति॒ प्र प्र ज॑नयति । \newline
84. ज॒न॒य॒तीति॑ जनयति । \newline

\textbf{Ghana Paata } \newline

1. च॒त॒सृभिः॒ सꣳ सम् च॑त॒सृभि॑ श्चत॒सृभिः॒ सम् भ॑रति भरति॒ सम् च॑त॒सृभि॑ श्चत॒सृभिः॒ सम् भ॑रति । \newline
2. च॒त॒सृभि॒रिति॑ चत॒सृ - भिः॒ । \newline
3. सम् भ॑रति भरति॒ सꣳ सम् भ॑रति च॒त्वारि॑ च॒त्वारि॑ भरति॒ सꣳ सम् भ॑रति च॒त्वारि॑ । \newline
4. भ॒र॒ति॒ च॒त्वारि॑ च॒त्वारि॑ भरति भरति च॒त्वारि॒ छन्दाꣳ॑सि॒ छन्दाꣳ॑सि च॒त्वारि॑ भरति भरति च॒त्वारि॒ छन्दाꣳ॑सि । \newline
5. च॒त्वारि॒ छन्दाꣳ॑सि॒ छन्दाꣳ॑सि च॒त्वारि॑ च॒त्वारि॒ छन्दाꣳ॑सि॒ छन्दो॑भि॒ श्छन्दो॑भि॒ श्छन्दाꣳ॑सि च॒त्वारि॑ च॒त्वारि॒ छन्दाꣳ॑सि॒ छन्दो॑भिः । \newline
6. छन्दाꣳ॑सि॒ छन्दो॑भि॒ श्छन्दो॑भि॒ श्छन्दाꣳ॑सि॒ छन्दाꣳ॑सि॒ छन्दो॑भि रे॒वैव छन्दो॑भि॒ श्छन्दाꣳ॑सि॒ छन्दाꣳ॑सि॒ छन्दो॑भि रे॒व । \newline
7. छन्दो॑भि रे॒वैव छन्दो॑भि॒ श्छन्दो॑भि रे॒व गा॑य॒त्रीभि॑र् गाय॒त्रीभि॑ रे॒व छन्दो॑भि॒ श्छन्दो॑भि रे॒व गा॑य॒त्रीभिः॑ । \newline
8. छन्दो॑भि॒रिति॒ छन्दः॑ - भिः॒ । \newline
9. ए॒व गा॑य॒त्रीभि॑र् गाय॒त्रीभि॑ रे॒वैव गा॑य॒त्रीभि॑र् ब्राह्म॒णस्य॑ ब्राह्म॒णस्य॑ गाय॒त्रीभि॑ रे॒वैव गा॑य॒त्रीभि॑र् ब्राह्म॒णस्य॑ । \newline
10. गा॒य॒त्रीभि॑र् ब्राह्म॒णस्य॑ ब्राह्म॒णस्य॑ गाय॒त्रीभि॑र् गाय॒त्रीभि॑र् ब्राह्म॒णस्य॑ गाय॒त्रो गा॑य॒त्रो ब्रा᳚ह्म॒णस्य॑ गाय॒त्रीभि॑र् गाय॒त्रीभि॑र् ब्राह्म॒णस्य॑ गाय॒त्रः । \newline
11. ब्रा॒ह्म॒णस्य॑ गाय॒त्रो गा॑य॒त्रो ब्रा᳚ह्म॒णस्य॑ ब्राह्म॒णस्य॑ गाय॒त्रो हि हि गा॑य॒त्रो ब्रा᳚ह्म॒णस्य॑ ब्राह्म॒णस्य॑ गाय॒त्रो हि । \newline
12. गा॒य॒त्रो हि हि गा॑य॒त्रो गा॑य॒त्रो हि ब्रा᳚ह्म॒णो ब्रा᳚ह्म॒णो हि गा॑य॒त्रो गा॑य॒त्रो हि ब्रा᳚ह्म॒णः । \newline
13. हि ब्रा᳚ह्म॒णो ब्रा᳚ह्म॒णो हि हि ब्रा᳚ह्म॒ण स्त्रि॒ष्टुग्भि॑ स्त्रि॒ष्टुग्भि॑र् ब्राह्म॒णो हि हि ब्रा᳚ह्म॒ण स्त्रि॒ष्टुग्भिः॑ । \newline
14. ब्रा॒ह्म॒ण स्त्रि॒ष्टुग्भि॑ स्त्रि॒ष्टुग्भि॑र् ब्राह्म॒णो ब्रा᳚ह्म॒ण स्त्रि॒ष्टुग्भी॑ राज॒न्य॑स्य राज॒न्य॑स्य त्रि॒ष्टुग्भि॑र् ब्राह्म॒णो ब्रा᳚ह्म॒ण स्त्रि॒ष्टुग्भी॑ राज॒न्य॑स्य । \newline
15. त्रि॒ष्टुग्भी॑ राज॒न्य॑स्य राज॒न्य॑स्य त्रि॒ष्टुग्भि॑ स्त्रि॒ष्टुग्भी॑ राज॒न्य॑स्य॒ त्रैष्टु॑भ॒ स्त्रैष्टु॑भो राज॒न्य॑स्य त्रि॒ष्टुग्भि॑ स्त्रि॒ष्टुग्भी॑ राज॒न्य॑स्य॒ त्रैष्टु॑भः । \newline
16. त्रि॒ष्टुग्भि॒रिति॑ त्रि॒ष्टुक् - भिः॒ । \newline
17. रा॒ज॒न्य॑स्य॒ त्रैष्टु॑भ॒ स्त्रैष्टु॑भो राज॒न्य॑स्य राज॒न्य॑स्य॒ त्रैष्टु॑भो॒ हि हि त्रैष्टु॑भो राज॒न्य॑स्य राज॒न्य॑स्य॒ त्रैष्टु॑भो॒ हि । \newline
18. त्रैष्टु॑भो॒ हि हि त्रैष्टु॑भ॒ स्त्रैष्टु॑भो॒ हि रा॑ज॒न्यो॑ राज॒न्यो॑ हि त्रैष्टु॑भ॒ स्त्रैष्टु॑भो॒ हि रा॑ज॒न्यः॑ । \newline
19. हि रा॑ज॒न्यो॑ राज॒न्यो॑ हि हि रा॑ज॒न्यो॑ यं ॅयꣳ रा॑ज॒न्यो॑ हि हि रा॑ज॒न्यो॑ यम् । \newline
20. रा॒ज॒न्यो॑ यं ॅयꣳ रा॑ज॒न्यो॑ राज॒न्यो॑ यम् का॒मये॑त का॒मये॑त॒ यꣳ रा॑ज॒न्यो॑ राज॒न्यो॑ यम् का॒मये॑त । \newline
21. यम् का॒मये॑त का॒मये॑त॒ यं ॅयम् का॒मये॑त॒ वसी॑या॒न्॒. वसी॑यान् का॒मये॑त॒ यं ॅयम् का॒मये॑त॒ वसी॑यान् । \newline
22. का॒मये॑त॒ वसी॑या॒न्॒. वसी॑यान् का॒मये॑त का॒मये॑त॒ वसी॑यान् थ्स्याथ् स्या॒द् वसी॑यान् का॒मये॑त का॒मये॑त॒ वसी॑यान् थ्स्यात् । \newline
23. वसी॑यान् थ्स्याथ् स्या॒द् वसी॑या॒न्॒. वसी॑यान् थ्स्या॒दितीति॑ स्या॒द् वसी॑या॒न्॒. वसी॑यान् थ्स्या॒दिति॑ । \newline
24. स्या॒दितीति॑ स्याथ् स्या॒ दित्यु॒भयी॑भि रु॒भयी॑भि॒ रिति॑ स्याथ् स्या॒दि त्यु॒भयी॑भिः । \newline
25. इत्यु॒भयी॑भि रु॒भयी॑भि॒ रिती त्यु॒भयी॑भि॒ स्तस्य॒ तस्यो॒भयी॑भि॒ रिती त्यु॒भयी॑भि॒ स्तस्य॑ । \newline
26. उ॒भयी॑भि॒ स्तस्य॒ तस्यो॒भयी॑भि रु॒भयी॑भि॒ स्तस्य॒ सꣳ सम् तस्यो॒भयी॑भि रु॒भयी॑भि॒ स्तस्य॒ सम् । \newline
27. तस्य॒ सꣳ सम् तस्य॒ तस्य॒ सम् भ॑रेद् भरे॒थ् सम् तस्य॒ तस्य॒ सम् भ॑रेत् । \newline
28. सम् भ॑रेद् भरे॒थ् सꣳ सम् भ॑रे॒त् तेज॒ स्तेजो॑ भरे॒थ् सꣳ सम् भ॑रे॒त् तेजः॑ । \newline
29. भ॒रे॒त् तेज॒ स्तेजो॑ भरेद् भरे॒त् तेज॑श्च च॒ तेजो॑ भरेद् भरे॒त् तेज॑श्च । \newline
30. तेज॑श्च च॒ तेज॒ स्तेज॑ श्चै॒वैव च॒ तेज॒ स्तेज॑ श्चै॒व । \newline
31. चै॒वैव च॑ चै॒वास्मा॑ अस्मा ए॒व च॑ चै॒वास्मै᳚ । \newline
32. ए॒वास्मा॑ अस्मा ए॒वैवास्मा॑ इन्द्रि॒य मि॑न्द्रि॒य म॑स्मा ए॒वैवास्मा॑ इन्द्रि॒यम् । \newline
33. अ॒स्मा॒ इ॒न्द्रि॒य मि॑न्द्रि॒य म॑स्मा अस्मा इन्द्रि॒यम् च॑ चेन्द्रि॒य म॑स्मा अस्मा इन्द्रि॒यम् च॑ । \newline
34. इ॒न्द्रि॒यम् च॑ चेन्द्रि॒य मि॑न्द्रि॒यम् च॑ स॒मीची॑ स॒मीची॑ चेन्द्रि॒य मि॑न्द्रि॒यम् च॑ स॒मीची᳚ । \newline
35. च॒ स॒मीची॑ स॒मीची॑ च च स॒मीची॑ दधाति दधाति स॒मीची॑ च च स॒मीची॑ दधाति । \newline
36. स॒मीची॑ दधाति दधाति स॒मीची॑ स॒मीची॑ दधा त्यष्टा॒भि र॑ष्टा॒भिर् द॑धाति स॒मीची॑ स॒मीची॑ दधा त्यष्टा॒भिः । \newline
37. स॒मीची॒ इति॑ स॒मीची᳚ । \newline
38. द॒धा॒ त्य॒ष्टा॒भि र॑ष्टा॒भिर् द॑धाति दधा त्यष्टा॒भिः सꣳ स म॑ष्टा॒भिर् द॑धाति दधा त्यष्टा॒भिः सम् । \newline
39. अ॒ष्टा॒भिः सꣳ स म॑ष्टा॒भि र॑ष्टा॒भिः सम् भ॑रति भरति॒ स म॑ष्टा॒भि र॑ष्टा॒भिः सम् भ॑रति । \newline
40. सम् भ॑रति भरति॒ सꣳ सम् भ॑र त्य॒ष्टाक्ष॑रा॒ ऽष्टाक्ष॑रा भरति॒ सꣳ सम् भ॑र त्य॒ष्टाक्ष॑रा । \newline
41. भ॒र॒ त्य॒ष्टाक्ष॑रा॒ ऽष्टाक्ष॑रा भरति भर त्य॒ष्टाक्ष॑रा गाय॒त्री गा॑य॒ त्र्य॑ष्टाक्ष॑रा भरति भर त्य॒ष्टाक्ष॑रा गाय॒त्री । \newline
42. अ॒ष्टाक्ष॑रा गाय॒त्री गा॑य॒ त्र्य॑ष्टाक्ष॑रा॒ ऽष्टाक्ष॑रा गाय॒त्री गा॑य॒त्रो गा॑य॒त्रो गा॑य॒ त्र्य॑ष्टाक्ष॑रा॒ ऽष्टाक्ष॑रा गाय॒त्री गा॑य॒त्रः । \newline
43. अ॒ष्टाक्ष॒रेत्य॒ष्टा - अ॒क्ष॒रा॒ । \newline
44. गा॒य॒त्री गा॑य॒त्रो गा॑य॒त्रो गा॑य॒त्री गा॑य॒त्री गा॑य॒त्रो᳚ ऽग्नि र॒ग्निर् गा॑य॒त्रो गा॑य॒त्री गा॑य॒त्री गा॑य॒त्रो᳚ ऽग्निः । \newline
45. गा॒य॒त्रो᳚ ऽग्नि र॒ग्निर् गा॑य॒त्रो गा॑य॒त्रो᳚ ऽग्निर् यावा॒न्॒. यावा॑ न॒ग्निर् गा॑य॒त्रो गा॑य॒त्रो᳚ ऽग्निर् यावान्॑ । \newline
46. अ॒ग्निर् यावा॒न्॒. यावा॑ न॒ग्नि र॒ग्निर् यावा॑ ने॒वैव यावा॑ न॒ग्नि र॒ग्निर् यावा॑ ने॒व । \newline
47. यावा॑ ने॒वैव यावा॒न्॒. यावा॑ ने॒वाग्नि र॒ग्नि रे॒व यावा॒न्॒. यावा॑ ने॒वाग्निः । \newline
48. ए॒वाग्नि र॒ग्नि रे॒वैवाग्नि स्तम् त म॒ग्नि रे॒वैवाग्नि स्तम् । \newline
49. अ॒ग्नि स्तम् त म॒ग्नि र॒ग्नि स्तꣳ सꣳ सम् त म॒ग्नि र॒ग्नि स्तꣳ सम् । \newline
50. तꣳ सꣳ सम् तम् तꣳ सम् भ॑रति भरति॒ सम् तम् तꣳ सम् भ॑रति । \newline
51. सम् भ॑रति भरति॒ सꣳ सम् भ॑रति॒ सीद॒ सीद॑ भरति॒ सꣳ सम् भ॑रति॒ सीद॑ । \newline
52. भ॒र॒ति॒ सीद॒ सीद॑ भरति भरति॒ सीद॑ होतर्. होतः॒ सीद॑ भरति भरति॒ सीद॑ होतः । \newline
53. सीद॑ होतर्. होतः॒ सीद॒ सीद॑ होत॒ रितीति॑ होतः॒ सीद॒ सीद॑ होत॒ रिति॑ । \newline
54. हो॒त॒ रितीति॑ होतर्. होत॒ रित्या॑हा॒हेति॑ होतर्. होत॒ रित्या॑ह । \newline
55. इत्या॑हा॒हेती त्या॑ह दे॒वता॑ दे॒वता॑ आ॒हेती त्या॑ह दे॒वताः᳚ । \newline
56. आ॒ह॒ दे॒वता॑ दे॒वता॑ आहाह दे॒वता॑ ए॒वैव दे॒वता॑ आहाह दे॒वता॑ ए॒व । \newline
57. दे॒वता॑ ए॒वैव दे॒वता॑ दे॒वता॑ ए॒वास्मा॑ अस्मा ए॒व दे॒वता॑ दे॒वता॑ ए॒वास्मै᳚ । \newline
58. ए॒वास्मा॑ अस्मा ए॒वैवास्मै॒ सꣳ स म॑स्मा ए॒वैवास्मै॒ सम् । \newline
59. अ॒स्मै॒ सꣳ स म॑स्मा अस्मै॒ सꣳ सा॑दयति सादयति॒ स म॑स्मा अस्मै॒ सꣳ सा॑दयति । \newline
60. सꣳ सा॑दयति सादयति॒ सꣳ सꣳ सा॑दयति॒ नि नि षा॑दयति॒ सꣳ सꣳ सा॑दयति॒ नि । \newline
61. सा॒द॒य॒ति॒ नि नि षा॑दयति सादयति॒ नि होता॒ होता॒ नि षा॑दयति सादयति॒ नि होता᳚ । \newline
62. नि होता॒ होता॒ नि नि होतेतीति॒ होता॒ नि नि होतेति॑ । \newline
63. होतेतीति॒ होता॒ होतेति॑ मनु॒ष्या᳚न् मनु॒ष्या॑ निति॒ होता॒ होतेति॑ मनु॒ष्यान्॑ । \newline
64. इति॑ मनु॒ष्या᳚न् मनु॒ष्या॑ नितीति॑ मनु॒ष्या᳚न् थ्सꣳ सम् म॑नु॒ष्या॑ नितीति॑ मनु॒ष्या᳚न् थ्सम् । \newline
65. म॒नु॒ष्या᳚न् थ्सꣳ सम् म॑नु॒ष्या᳚न् मनु॒ष्या᳚न् थ्सꣳ सी॑दस्व सीदस्व॒ सम् म॑नु॒ष्या᳚न् 
मनु॒ष्या᳚न् थ्सꣳ सी॑दस्व । \newline
66. सꣳ सी॑दस्व सीदस्व॒ सꣳ सꣳ सी॑द॒स्वेतीति॑ सीदस्व॒ सꣳ सꣳ सी॑द॒स्वेति॑ । \newline
67. सी॒द॒स्वेतीति॑ सीदस्व सीद॒स्वेति॒ वयाꣳ॑सि॒ वयाꣳ॒॒सीति॑ सीदस्व सीद॒स्वेति॒ वयाꣳ॑सि । \newline
68. इति॒ वयाꣳ॑सि॒ वयाꣳ॒॒सीतीति॒ वयाꣳ॑सि॒ जनि॑ष्व॒ जनि॑ष्व॒ वयाꣳ॒॒सीतीति॒ वयाꣳ॑सि॒ जनि॑ष्व । \newline
69. वयाꣳ॑सि॒ जनि॑ष्व॒ जनि॑ष्व॒ वयाꣳ॑सि॒ वयाꣳ॑सि॒ जनि॑ष्वा॒ हि हि जनि॑ष्व॒ वयाꣳ॑सि॒ वयाꣳ॑सि॒ जनि॑ष्वा॒ हि । \newline
70. जनि॑ष्वा॒ हि हि जनि॑ष्व॒ जनि॑ष्वा॒ हि जेन्यो॒ जेन्यो॒ हि जनि॑ष्व॒ जनि॑ष्वा॒ हि जेन्यः॑ । \newline
71. हि जेन्यो॒ जेन्यो॒ हि हि जेन्यो॒ अग्रे॒ अग्रे॒ जेन्यो॒ हि हि जेन्यो॒ अग्रे᳚ । \newline
72. जेन्यो॒ अग्रे॒ अग्रे॒ जेन्यो॒ जेन्यो॒ अग्रे॒ अह्ना॒ मह्ना॒ मग्रे॒ जेन्यो॒ जेन्यो॒ अग्रे॒ अह्ना᳚म् । \newline
73. अग्रे॒ अह्ना॒ मह्ना॒ मग्रे॒ अग्रे॒ अह्ना॒ मिती त्यह्ना॒ मग्रे॒ अग्रे॒ अह्ना॒ मिति॑ । \newline
74. अह्ना॒ मिती त्यह्ना॒ मह्ना॒ मित्या॑हा॒हे त्यह्ना॒ मह्ना॒ मित्या॑ह । \newline
75. इत्या॑हा॒हे तीत्या॑ह देवमनु॒ष्यान् दे॑वमनु॒ष्या ना॒हेती त्या॑ह देवमनु॒ष्यान् । \newline
76. आ॒ह॒ दे॒व॒म॒नु॒ष्यान् दे॑वमनु॒ष्या ना॑हाह देवमनु॒ष्या ने॒वैव दे॑वमनु॒ष्या ना॑हाह देवमनु॒ष्या ने॒व । \newline
77. दे॒व॒म॒नु॒ष्या ने॒वैव दे॑वमनु॒ष्यान् दे॑वमनु॒ष्या ने॒वास्मा॑ अस्मा ए॒व दे॑वमनु॒ष्यान् दे॑वमनु॒ष्या ने॒वास्मै᳚ । \newline
78. दे॒व॒म॒नु॒ष्यानिति॑ देव - म॒नु॒ष्यान् । \newline
79. ए॒वास्मा॑ अस्मा ए॒वैवास्मै॒ सꣳस॑न्ना॒न् थ्सꣳस॑न्ना नस्मा ए॒वैवास्मै॒ सꣳस॑न्नान् । \newline
80. अ॒स्मै॒ सꣳस॑न्ना॒न् थ्सꣳस॑न्ना नस्मा अस्मै॒ सꣳस॑न्ना॒न् प्र प्र सꣳस॑न्ना नस्मा अस्मै॒ सꣳस॑न्ना॒न् प्र । \newline
81. सꣳस॑न्ना॒न् प्र प्र सꣳस॑न्ना॒न् थ्सꣳस॑न्ना॒न् प्र ज॑नयति जनयति॒ प्र सꣳस॑न्ना॒न् थ्सꣳस॑न्ना॒न् प्र ज॑नयति । \newline
82. सꣳस॑न्ना॒निति॒ सं - स॒न्ना॒न् । \newline
83. प्र ज॑नयति जनयति॒ प्र प्र ज॑नयति । \newline
84. ज॒न॒य॒तीति॑ जनयति । \newline
\pagebreak
\markright{ TS 5.1.5.1  \hfill https://www.vedavms.in \hfill}

\section{ TS 5.1.5.1 }

\textbf{TS 5.1.5.1 } \newline
\textbf{Samhita Paata} \newline

क्रू॒रमि॑व॒ वा अ॑स्या ए॒तत् क॑रोति॒ यत् खन॑त्य॒प उप॑ सृज॒त्यापो॒ वै शा॒न्ताः शा॒न्ताभि॑रे॒वाऽस्यै॒ शुचꣳ॑ शमयति॒ सं ते॑ वा॒युर्मा॑त॒रिश्वा॑ दधा॒त्वित्या॑ह प्रा॒णो वै वा॒युः प्रा॒णेनै॒वास्यै᳚ प्रा॒णꣳ सं द॑धाति॒ सं ते॑ वा॒युरित्या॑ह॒ तस्मा᳚द्-वा॒युप्र॑च्युता दि॒वो वृष्टि॑रीर्ते॒ तस्मै॑ च देवि॒ वष॑डस्तु॒ - [  ] \newline

\textbf{Pada Paata} \newline

क्रू॒रम् । इ॒व॒ । वै । अ॒स्याः॒ । ए॒तत् । क॒रो॒ति॒ । यत् । खन॑ति । अ॒पः । उपेति॑ । सृ॒ज॒ति॒ । आपः॑ । वै । शा॒न्ताः । शा॒न्ताभिः॑ । ए॒व । अ॒स्यै॒ । शुच᳚म् । श॒म॒य॒ति॒ । समिति॑ । ते॒ । वा॒युः । मा॒त॒रिश्वा᳚ । द॒धा॒तु॒ । इति॑ । आ॒ह॒ । प्रा॒ण इति॑ प्र - अ॒नः । वै । वा॒युः । प्रा॒णेनेति॑ प्र - अ॒नेन॑ । ए॒व । अ॒स्यै॒ । प्रा॒णमिति॑ प्र - अ॒नम् । समिति॑ । द॒धा॒ति॒ । समिति॑ । ते॒ । वा॒युः । इति॑ । आ॒ह॒ । तस्मा᳚त् । वा॒युप्र॑च्यु॒तेति॑ वा॒यु - प्र॒च्यु॒ता॒ । दि॒वः । वृष्टिः॑ । ई॒र्ते॒ । तस्मै᳚ । च॒ । दे॒वि॒ । वष॑ट् । अ॒स्तु॒ ।  \newline


\textbf{Krama Paata} \newline

क्रू॒रमि॑व । इ॒व॒ वै । वा अ॑स्याः । अ॒स्या॒ ए॒तत् । ए॒तत् क॑रोति । क॒रो॒ति॒ यत् । यत् खन॑ति । खन॑त्य॒पः । अ॒प उप॑ । उप॑ सृजति । सृ॒ज॒त्यापः॑ । आपो॒ वै । वै शा॒न्ताः । शा॒न्ताः शा॒न्ताभिः॑ । शा॒न्ताभि॑रे॒व । ए॒वास्यै᳚ । अ॒स्यै॒ शुच᳚म् । शुचꣳ॑ शमयति । श॒म॒य॒ति॒ सम् । सम् ते᳚ । ते॒ वा॒युः । वा॒युर् मा॑त॒रिश्वा᳚ । मा॒त॒रिश्वा॑ दधातु । द॒धा॒त्विति॑ । इत्या॑ह । आ॒ह॒ प्रा॒णः । प्रा॒णो वै । प्रा॒ण इति॑ प्र - अ॒नः । वै वा॒युः । वा॒युः प्रा॒णेन॑ । प्रा॒णेनै॒व । प्रा॒णेनेति॑ प्र - अ॒नेन॑ । ए॒वास्यै᳚ । अ॒स्यै॒ प्रा॒णम् । प्रा॒णꣳ सम् । प्रा॒णमिति॑ प्र - अ॒नम् । सम् द॑धाति । द॒धा॒ति॒ सम् । सम् ते᳚ । ते॒ वा॒युः । वा॒युरिति॑ । इत्या॑ह । आ॒ह॒ तस्मा᳚त् । तस्मा᳚द् वा॒युप्र॑च्युता । वा॒युप्र॑च्युता दि॒वः । वा॒युप्र॑च्यु॒तेति॑ वा॒यु - प्र॒च्यु॒ता॒ । दि॒वो वृष्टिः॑ । वृष्टि॑रीर्ते । ई॒र्ते॒ तस्मै᳚ । तस्मै॑ च । च॒ दे॒वि॒ । दे॒वि॒ वष॑ट् । वष॑डस्तु । अ॒स्तु॒ तुभ्य᳚म् \newline

\textbf{Jatai Paata} \newline

1. क्रू॒र मि॑वे व क्रू॒रम् क्रू॒र मि॑व । \newline
2. इ॒व॒ वै वा इ॑वे व॒ वै । \newline
3. वा अ॑स्या अस्या॒ वै वा अ॑स्याः । \newline
4. अ॒स्या॒ ए॒त दे॒त द॑स्या अस्या ए॒तत् । \newline
5. ए॒तत् क॑रोति करो त्ये॒त दे॒तत् क॑रोति । \newline
6. क॒रो॒ति॒ यद् यत् क॑रोति करोति॒ यत् । \newline
7. यत् खन॑ति॒ खन॑ति॒ यद् यत् खन॑ति । \newline
8. खन॑ त्य॒पो॑ ऽपः खन॑ति॒ खन॑ त्य॒पः । \newline
9. अ॒प उपोपा॒पो॑ ऽप उप॑ । \newline
10. उप॑ सृजति सृज॒ त्युपोप॑ सृजति । \newline
11. सृ॒ज॒ त्याप॒ आपः॑ सृजति सृज॒ त्यापः॑ । \newline
12. आपो॒ वै वा आप॒ आपो॒ वै । \newline
13. वै शा॒न्ताः शा॒न्ता वै वै शा॒न्ताः । \newline
14. शा॒न्ताः शा॒न्ताभिः॑ शा॒न्ताभिः॑ शा॒न्ताः शा॒न्ताः शा॒न्ताभिः॑ । \newline
15. शा॒न्ताभि॑ रे॒वैव शा॒न्ताभिः॑ शा॒न्ताभि॑ रे॒व । \newline
16. ए॒वास्या॑ अस्या ए॒वैवास्यै᳚ । \newline
17. अ॒स्यै॒ शुचꣳ॒॒ शुच॑ मस्या अस्यै॒ शुच᳚म् । \newline
18. शुचꣳ॑ शमयति शमयति॒ शुचꣳ॒॒ शुचꣳ॑ शमयति । \newline
19. श॒म॒य॒ति॒ सꣳ सꣳ श॑मयति शमयति॒ सम् । \newline
20. सम् ते॑ ते॒ सꣳ सम् ते᳚ । \newline
21. ते॒ वा॒युर् वा॒यु स्ते॑ ते वा॒युः । \newline
22. वा॒युर् मा॑त॒रिश्वा॑ मात॒रिश्वा॑ वा॒युर् वा॒युर् मा॑त॒रिश्वा᳚ । \newline
23. मा॒त॒रिश्वा॑ दधातु दधातु मात॒रिश्वा॑ मात॒रिश्वा॑ दधातु । \newline
24. द॒धा॒त्वितीति॑ दधातु दधा॒त्विति॑ । \newline
25. इत्या॑हा॒हे तीत्या॑ह । \newline
26. आ॒ह॒ प्रा॒णः प्रा॒ण आ॑हाह प्रा॒णः । \newline
27. प्रा॒णो वै वै प्रा॒णः प्रा॒णो वै । \newline
28. प्रा॒ण इति॑ प्र - अ॒नः । \newline
29. वै वा॒युर् वा॒युर् वै वै वा॒युः । \newline
30. वा॒युः प्रा॒णेन॑ प्रा॒णेन॑ वा॒युर् वा॒युः प्रा॒णेन॑ । \newline
31. प्रा॒णे नै॒वैव प्रा॒णेन॑ प्रा॒णे नै॒व । \newline
32. प्रा॒णेनेति॑ प्र - अ॒नेन॑ । \newline
33. ए॒वास्या॑ अस्या ए॒वैवास्यै᳚ । \newline
34. अ॒स्यै॒ प्रा॒णम् प्रा॒ण म॑स्या अस्यै प्रा॒णम् । \newline
35. प्रा॒णꣳ सꣳ सम् प्रा॒णम् प्रा॒णꣳ सम् । \newline
36. प्रा॒णमिति॑ प्र - अ॒नम् । \newline
37. सम् द॑धाति दधाति॒ सꣳ सम् द॑धाति । \newline
38. द॒धा॒ति॒ सꣳ सम् द॑धाति दधाति॒ सम् । \newline
39. सम् ते॑ ते॒ सꣳ सम् ते᳚ । \newline
40. ते॒ वा॒युर् वा॒यु स्ते॑ ते वा॒युः । \newline
41. वा॒यु रितीति॑ वा॒युर् वा॒यु रिति॑ । \newline
42. इत्या॑हा॒हे तीत्या॑ह । \newline
43. आ॒ह॒ तस्मा॒त् तस्मा॑ दाहाह॒ तस्मा᳚त् । \newline
44. तस्मा᳚द् वा॒युप्र॑च्युता वा॒युप्र॑च्युता॒ तस्मा॒त् तस्मा᳚द् वा॒युप्र॑च्युता । \newline
45. वा॒युप्र॑च्युता दि॒वो दि॒वो वा॒युप्र॑च्युता वा॒युप्र॑च्युता दि॒वः । \newline
46. वा॒युप्र॑च्यु॒तेति॑ वा॒यु - प्र॒च्यु॒ता॒ । \newline
47. दि॒वो वृष्टि॒र् वृष्टि॑र् दि॒वो दि॒वो वृष्टिः॑ । \newline
48. वृष्टि॑ रीर्त ईर्ते॒ वृष्टि॒र् वृष्टि॑ रीर्ते । \newline
49. ई॒र्ते॒ तस्मै॒ तस्मा॑ ईर्त ईर्ते॒ तस्मै᳚ । \newline
50. तस्मै॑ च च॒ तस्मै॒ तस्मै॑ च । \newline
51. च॒ दे॒वि॒ दे॒वि॒ च॒ च॒ दे॒वि॒ । \newline
52. दे॒वि॒ वष॒ड् वष॑ड् देवि देवि॒ वष॑ट् । \newline
53. वष॑ड स्त्वस्तु॒ वष॒ड् वष॑ डस्तु । \newline
54. अ॒स्तु॒ तुभ्य॒म् तुभ्य॑ मस्त्वस्तु॒ तुभ्य᳚म् । \newline

\textbf{Ghana Paata } \newline

1. क्रू॒र मि॑वेव क्रू॒रम् क्रू॒र मि॑व॒ वै वा इ॑व क्रू॒रम् क्रू॒र मि॑व॒ वै । \newline
2. इ॒व॒ वै वा इ॑वेव॒ वा अ॑स्या अस्या॒ वा इ॑वेव॒ वा अ॑स्याः । \newline
3. वा अ॑स्या अस्या॒ वै वा अ॑स्या ए॒त दे॒त द॑स्या॒ वै वा अ॑स्या ए॒तत् । \newline
4. अ॒स्या॒ ए॒त दे॒त द॑स्या अस्या ए॒तत् क॑रोति करो त्ये॒त द॑स्या अस्या ए॒तत् क॑रोति । \newline
5. ए॒तत् क॑रोति करो त्ये॒त दे॒तत् क॑रोति॒ यद् यत् क॑रो त्ये॒त दे॒तत् क॑रोति॒ यत् । \newline
6. क॒रो॒ति॒ यद् यत् क॑रोति करोति॒ यत् खन॑ति॒ खन॑ति॒ यत् क॑रोति करोति॒ यत् खन॑ति । \newline
7. यत् खन॑ति॒ खन॑ति॒ यद् यत् खन॑ त्य॒पो॑ ऽपः खन॑ति॒ यद् यत् खन॑ त्य॒पः । \newline
8. खन॑ त्य॒पो॑ ऽपः खन॑ति॒ खन॑ त्य॒प उपो पा॒पः खन॑ति॒ खन॑ त्य॒प उप॑ । \newline
9. अ॒प उपो पा॒पो॑ ऽप उप॑ सृजति सृज॒ त्युपा॒पो॑ ऽप उप॑ सृजति । \newline
10. उप॑ सृजति सृज॒ त्युपोप॑ सृज॒त्याप॒ आपः॑ सृज॒ त्युपोप॑ सृज॒त्यापः॑ । \newline
11. सृ॒ज॒त्याप॒ आपः॑ सृजति सृज॒ त्यापो॒ वै वा आपः॑ सृजति सृज॒ त्यापो॒ वै । \newline
12. आपो॒ वै वा आप॒ आपो॒ वै शा॒न्ताः शा॒न्ता वा आप॒ आपो॒ वै शा॒न्ताः । \newline
13. वै शा॒न्ताः शा॒न्ता वै वै शा॒न्ताः शा॒न्ताभिः॑ शा॒न्ताभिः॑ शा॒न्ता वै वै शा॒न्ताः शा॒न्ताभिः॑ । \newline
14. शा॒न्ताः शा॒न्ताभिः॑ शा॒न्ताभिः॑ शा॒न्ताः शा॒न्ताः शा॒न्ताभि॑ रे॒वैव शा॒न्ताभिः॑ शा॒न्ताः शा॒न्ताः शा॒न्ताभि॑ रे॒व । \newline
15. शा॒न्ताभि॑ रे॒वैव शा॒न्ताभिः॑ शा॒न्ताभि॑ रे॒वास्या॑ अस्या ए॒व शा॒न्ताभिः॑ शा॒न्ताभि॑ रे॒वास्यै᳚ । \newline
16. ए॒वास्या॑ अस्या ए॒वैवास्यै॒ शुचꣳ॒॒ शुच॑ मस्या ए॒वैवास्यै॒ शुच᳚म् । \newline
17. अ॒स्यै॒ शुचꣳ॒॒ शुच॑ मस्या अस्यै॒ शुचꣳ॑ शमयति शमयति॒ शुच॑ मस्या अस्यै॒ शुचꣳ॑ शमयति । \newline
18. शुचꣳ॑ शमयति शमयति॒ शुचꣳ॒॒ शुचꣳ॑ शमयति॒ सꣳ सꣳ श॑मयति॒ शुचꣳ॒॒ शुचꣳ॑ शमयति॒ सम् । \newline
19. श॒म॒य॒ति॒ सꣳ सꣳ श॑मयति शमयति॒ सम् ते॑ ते॒ सꣳ श॑मयति शमयति॒ सम् ते᳚ । \newline
20. सम् ते॑ ते॒ सꣳ सम् ते॑ वा॒युर् वा॒यु स्ते॒ सꣳ सम् ते॑ वा॒युः । \newline
21. ते॒ वा॒युर् वा॒यु स्ते॑ ते वा॒युर् मा॑त॒रिश्वा॑ मात॒रिश्वा॑ वा॒यु स्ते॑ ते वा॒युर् मा॑त॒रिश्वा᳚ । \newline
22. वा॒युर् मा॑त॒रिश्वा॑ मात॒रिश्वा॑ वा॒युर् वा॒युर् मा॑त॒रिश्वा॑ दधातु दधातु मात॒रिश्वा॑ वा॒युर् वा॒युर् मा॑त॒रिश्वा॑ दधातु । \newline
23. मा॒त॒रिश्वा॑ दधातु दधातु मात॒रिश्वा॑ मात॒रिश्वा॑ दधा॒त्वितीति॑ दधातु मात॒रिश्वा॑ मात॒रिश्वा॑ दधा॒त्विति॑ । \newline
24. द॒धा॒त्वितीति॑ दधातु दधा॒त्वि त्या॑हा॒हेति॑ दधातु दधा॒त्वि त्या॑ह । \newline
25. इत्या॑हा॒हेती त्या॑ह प्रा॒णः प्रा॒ण आ॒हेती त्या॑ह प्रा॒णः । \newline
26. आ॒ह॒ प्रा॒णः प्रा॒ण आ॑हाह प्रा॒णो वै वै प्रा॒ण आ॑हाह प्रा॒णो वै । \newline
27. प्रा॒णो वै वै प्रा॒णः प्रा॒णो वै वा॒युर् वा॒युर् वै प्रा॒णः प्रा॒णो वै वा॒युः । \newline
28. प्रा॒ण इति॑ प्र - अ॒नः । \newline
29. वै वा॒युर् वा॒युर् वै वै वा॒युः प्रा॒णेन॑ प्रा॒णेन॑ वा॒युर् वै वै वा॒युः प्रा॒णेन॑ । \newline
30. वा॒युः प्रा॒णेन॑ प्रा॒णेन॑ वा॒युर् वा॒युः प्रा॒णेनै॒वैव प्रा॒णेन॑ वा॒युर् वा॒युः प्रा॒णेनै॒व । \newline
31. प्रा॒णेनै॒वैव प्रा॒णेन॑ प्रा॒णे नै॒वास्या॑ अस्या ए॒व प्रा॒णेन॑ प्रा॒णेनै॒वास्यै᳚ । \newline
32. प्रा॒णेनेति॑ प्र - अ॒नेन॑ । \newline
33. ए॒वास्या॑ अस्या ए॒वैवास्यै᳚ प्रा॒णम् प्रा॒ण म॑स्या ए॒वैवास्यै᳚ प्रा॒णम् । \newline
34. अ॒स्यै॒ प्रा॒णम् प्रा॒ण म॑स्या अस्यै प्रा॒णꣳ सꣳ सम् प्रा॒ण म॑स्या अस्यै प्रा॒णꣳ सम् । \newline
35. प्रा॒णꣳ सꣳ सम् प्रा॒णम् प्रा॒णꣳ सम् द॑धाति दधाति॒ सम् प्रा॒णम् प्रा॒णꣳ सम् द॑धाति । \newline
36. प्रा॒णमिति॑ प्र - अ॒नम् । \newline
37. सम् द॑धाति दधाति॒ सꣳ सम् द॑धाति॒ सꣳ सम् द॑धाति॒ सꣳ सम् द॑धाति॒ सम् । \newline
38. द॒धा॒ति॒ सꣳ सम् द॑धाति दधाति॒ सम् ते॑ ते॒ सम् द॑धाति दधाति॒ सम् ते᳚ । \newline
39. सम् ते॑ ते॒ सꣳ सम् ते॑ वा॒युर् वा॒यु स्ते॒ सꣳ सम् ते॑ वा॒युः । \newline
40. ते॒ वा॒युर् वा॒यु स्ते॑ ते वा॒यु रितीति॑ वा॒यु स्ते॑ ते वा॒युरिति॑ । \newline
41. वा॒यु रितीति॑ वा॒युर् वा॒यु रित्या॑ हा॒हेति॑ वा॒युर् वा॒यु रित्या॑ह । \newline
42. इत्या॑हा॒हेती त्या॑ह॒ तस्मा॒त् तस्मा॑ दा॒हेती त्या॑ह॒ तस्मा᳚त् । \newline
43. आ॒ह॒ तस्मा॒त् तस्मा॑ दाहाह॒ तस्मा᳚द् वा॒युप्र॑च्युता वा॒युप्र॑च्युता॒ तस्मा॑ दाहाह॒ तस्मा᳚द् वा॒युप्र॑च्युता । \newline
44. तस्मा᳚द् वा॒युप्र॑च्युता वा॒युप्र॑च्युता॒ तस्मा॒त् तस्मा᳚द् वा॒युप्र॑च्युता दि॒वो दि॒वो वा॒युप्र॑च्युता॒ तस्मा॒त् तस्मा᳚द् वा॒युप्र॑च्युता दि॒वः । \newline
45. वा॒युप्र॑च्युता दि॒वो दि॒वो वा॒युप्र॑च्युता वा॒युप्र॑च्युता दि॒वो वृष्टि॒र् वृष्टि॑र् दि॒वो वा॒युप्र॑च्युता वा॒युप्र॑च्युता दि॒वो वृष्टिः॑ । \newline
46. वा॒युप्र॑च्यु॒तेति॑ वा॒यु - प्र॒च्यु॒ता॒ । \newline
47. दि॒वो वृष्टि॒र् वृष्टि॑र् दि॒वो दि॒वो वृष्टि॑ रीर्त ईर्ते॒ वृष्टि॑र् दि॒वो दि॒वो वृष्टि॑ रीर्ते । \newline
48. वृष्टि॑रीर्त ईर्ते॒ वृष्टि॒र् वृष्टि॑ रीर्ते॒ तस्मै॒ तस्मा॑ ईर्ते॒ वृष्टि॒र् वृष्टि॑ रीर्ते॒ तस्मै᳚ । \newline
49. ई॒र्ते॒ तस्मै॒ तस्मा॑ ईर्त ईर्ते॒ तस्मै॑ च च॒ तस्मा॑ ईर्त ईर्ते॒ तस्मै॑ च । \newline
50. तस्मै॑ च च॒ तस्मै॒ तस्मै॑ च देवि देवि च॒ तस्मै॒ तस्मै॑ च देवि । \newline
51. च॒ दे॒वि॒ दे॒वि॒ च॒ च॒ दे॒वि॒ वष॒ड् वष॑ड् देवि च च देवि॒ वष॑ट् । \newline
52. दे॒वि॒ वष॒ड् वष॑ड् देवि देवि॒ वष॑ डस्त्वस्तु॒ वष॑ड् देवि देवि॒ वष॑डस्तु । \newline
53. वष॑ डस्त्वस्तु॒ वष॒ड् वष॑डस्तु॒ तुभ्य॒म् तुभ्य॑ मस्तु॒ वष॒ड् वष॑डस्तु॒ तुभ्य᳚म् । \newline
54. अ॒स्तु॒ तुभ्य॒म् तुभ्य॑ मस्त्वस्तु॒ तुभ्य॒ मितीति॒ तुभ्य॑ मस्त्वस्तु॒ तुभ्य॒ मिति॑ । \newline
\pagebreak
\markright{ TS 5.1.5.2  \hfill https://www.vedavms.in \hfill}

\section{ TS 5.1.5.2 }

\textbf{TS 5.1.5.2 } \newline
\textbf{Samhita Paata} \newline

तुभ्य॒मित्या॑ह॒ षड्वा ऋ॒तव॑ ऋ॒तुष्वे॒व वृष्टिं॑ दधाति॒ तस्मा॒थ् सर्वा॑नृ॒तून्. व॑र्.षति॒ यद्-व॑षट्कु॒र्याद्-या॒तया॑माऽस्य वषट्का॒रः स्या॒द्यन्न व॑षट्कु॒र्याद् रक्षाꣳ॑सि य॒ज्ञ्ꣳ ह॑न्यु॒र्वडित्या॑ह प॒रोक्ष॑मे॒व वष॑ट् करोति॒ नास्य॑ या॒तया॑मा वषट्का॒रो भव॑ति॒ न य॒ज्ञ्ꣳ रक्षाꣳ॑सि घ्नन्ति॒ सुजा॑तो॒ ज्योति॑षा स॒हेत्य॑नु॒ष्टुभोप॑ नह्यत्यनु॒ष्टु - [  ] \newline

\textbf{Pada Paata} \newline

तुभ्य᳚म् । इति॑ । आ॒ह॒ । षट् । वै । ऋ॒तवः॑ । ऋ॒तुषु॑ । ए॒व । वृष्टि᳚म् । द॒धा॒ति॒ । तस्मा᳚त् । सर्वान्॑ । ऋ॒तून् । व॒र्॒.ष॒ति॒ । यत् । व॒ष॒ट्कु॒र्यादिति॑ वषट् - कु॒र्यात् । या॒तया॒मेति॑ या॒त - या॒मा॒ । अ॒स्य॒ । व॒ष॒ट्का॒र इति॑ वषट् - का॒रः । स्या॒त् । यत् । न । व॒ष॒ट्कु॒र्यादिति॑ वषट् - कु॒र्यात् । रक्षाꣳ॑सि । य॒ज्ञ्म् । ह॒न्युः॒ । वट् । इति॑ । आ॒ह॒ । प॒रोक्ष॒मिति॑ परः - अक्ष᳚म् । ए॒व । वष॑ट् । क॒रो॒ति॒ । न । अ॒स्य॒ । या॒तया॒मेति॑ या॒त - या॒मा॒ । व॒ष॒ट्का॒र इति॑ वषट् - का॒रः । भव॑ति । न । य॒ज्ञ्म् । रक्षाꣳ॑सि । घ्न॒न्ति॒ । सुजा॑त॒ इति॒ सु - जा॒तः॒ । ज्योति॑षा । स॒ह । इति॑ । अ॒नु॒ष्टुभेत्य॑नु - स्तुभा᳚ । उपेति॑ । न॒ह्य॒ति॒ । अ॒नु॒ष्टुबित्य॑नु - स्तुप् ।  \newline


\textbf{Krama Paata} \newline

तुभ्य॒मिति॑ । इत्या॑ह । आ॒ह॒ षट् । षड् वै । वा ऋ॒तवः॑ । ऋ॒तव॑ ऋ॒तुषु॑ । ऋ॒तुष्वे॒व । ए॒व वृष्टि᳚म् । वृष्टि॑म् दधाति । द॒धा॒ति॒ तस्मा᳚त् । तस्मा॒थ् सर्वान्॑ । सर्वा॑नृ॒तून् । ऋ॒तून्. व॑र्.षति । व॒र्.॒ष॒ति॒ यत् । यद् व॑षट्कु॒र्यात् । व॒ष॒ट्कु॒र्याद् या॒तया॑मा । व॒ष॒ट्कु॒र्यादिति॑ वषट् - कु॒र्यात् । या॒तया॑माऽस्य । या॒तया॒मेति॑ या॒त - या॒मा॒ । अ॒स्य॒ व॒ष॒ट्का॒रः । व॒ष॒ट्का॒रः स्या᳚त् । व॒ष॒ट्का॒र इति॑ वषट् - का॒रः । स्या॒द् यत् । यन् न । न व॑षट्कु॒र्यात् । व॒ष॒ट्कु॒र्याद् रक्षाꣳ॑सि । व॒ष॒ट्कु॒र्यादिति॑ वषट् - कु॒र्यात् । रक्षाꣳ॑सि य॒ज्ञ्म् । य॒ज्ञ्ꣳ ह॑न्युः । ह॒न्यु॒र् वट् । वडिति॑ । इत्या॑ह । आ॒ह॒ प॒रोक्ष᳚म् । प॒रोक्ष॑मे॒व । प॒रोक्ष॒मिति॑ परः - अक्ष᳚म् । ए॒व वष॑ट् । वष॑ट् करोति । क॒रो॒ति॒ न । नास्य॑ । अ॒स्य॒ या॒तया॑मा । या॒तया॑मा वषट्का॒रः । या॒तया॒मेति॑ या॒त - या॒मा॒ । व॒ष॒ट्का॒रो भव॑ति । व॒ष॒ट्का॒र इति॑ वषट् - का॒रः । भव॑ति॒ न । न य॒ज्ञ्म् । य॒ज्ञ्ꣳ रक्षाꣳ॑सि । रक्षाꣳ॑सि घ्नन्ति । घ्न॒न्ति॒ सुजा॑तः । सुजा॑तो॒ ज्योति॑षा । सुजा॑त॒ इति॒ सु - जा॒तः॒ । ज्योति॑षा स॒ह । स॒हेति॑ । इत्य॑नु॒ष्टुभा᳚ । अ॒नु॒ष्टुभोप॑ । अ॒नु॒ष्टुभेत्य॑नु - स्तुभा᳚ । उप॑ नह्यति । न॒ह्य॒त्य॒नु॒ष्टुप् । अ॒नु॒ष्टुफ् सर्वा॑णि । अ॒नु॒ष्टुबित्य॑नु - स्तुप् \newline

\textbf{Jatai Paata} \newline

1. तुभ्य॒ मितीति॒ तुभ्य॒म् तुभ्य॒ मिति॑ । \newline
2. इत्या॑हा॒हे तीत्या॑ह । \newline
3. आ॒ह॒ षट् थ् षडा॑हाह॒ षट् । \newline
4. षड् वै वै षट् थ्षड् वै । \newline
5. वा ऋ॒तव॑ ऋ॒तवो॒ वै वा ऋ॒तवः॑ । \newline
6. ऋ॒तव॑ ऋ॒तु ष्वृ॒तु ष्वृ॒तव॑ ऋ॒तव॑ ऋ॒तुषु॑ । \newline
7. ऋ॒तु ष्वे॒वैव र्‌तुष्वृ॒तु ष्वे॒व । \newline
8. ए॒व वृष्टिं॒ ॅवृष्टि॑ मे॒वैव वृष्टि᳚म् । \newline
9. वृष्टि॑म् दधाति दधाति॒ वृष्टिं॒ ॅवृष्टि॑म् दधाति । \newline
10. द॒धा॒ति॒ तस्मा॒त् तस्मा᳚द् दधाति दधाति॒ तस्मा᳚त् । \newline
11. तस्मा॒थ् सर्वा॒न् थ्सर्वा॒न् तस्मा॒त् तस्मा॒थ् सर्वान्॑ । \newline
12. सर्वा॑ नृ॒तू नृ॒तून् थ्सर्वा॒न् थ्सर्वा॑ नृ॒तून् । \newline
13. ऋ॒तून्. व॑र्.षति वर्.षत्यृ॒तू नृ॒तून्. व॑र्.षति । \newline
14. व॒र्॒.ष॒ति॒ यद् यद् व॑र्.षति वर्.षति॒ यत् । \newline
15. यद् व॑षट्कु॒र्याद् व॑षट्कु॒र्याद् यद् यद् व॑षट्कु॒र्यात् । \newline
16. व॒ष॒ट्कु॒र्याद् या॒तया॑मा या॒तया॑मा वषट्कु॒र्याद् व॑षट्कु॒र्याद् या॒तया॑मा । \newline
17. व॒ष॒ट्कु॒र्यादिति॑ वषट् - कु॒र्यात् । \newline
18. या॒तया॑मा ऽस्यास्य या॒तया॑मा या॒तया॑मा ऽस्य । \newline
19. या॒तया॒मेति॑ या॒त - या॒मा॒ । \newline
20. अ॒स्य॒ व॒ष॒ट्का॒रो व॑षट्का॒रो᳚ ऽस्यास्य वषट्का॒रः । \newline
21. व॒ष॒ट्का॒रः स्या᳚थ् स्याद् वषट्का॒रो व॑षट्का॒रः स्या᳚त् । \newline
22. व॒ष॒ट्का॒र इति॑ वषट् - का॒रः । \newline
23. स्या॒द् यद् यथ् स्या᳚थ् स्या॒द् यत् । \newline
24. यन् न न यद् यन् न । \newline
25. न व॑षट्कु॒र्याद् व॑षट्कु॒र्यान् न न व॑षट्कु॒र्यात् । \newline
26. व॒ष॒ट्कु॒र्याद् रक्षाꣳ॑सि॒ रक्षाꣳ॑सि वषट्कु॒र्याद् व॑षट्कु॒र्याद् रक्षाꣳ॑सि । \newline
27. व॒ष॒ट्कु॒र्यादिति॑ वषट् - कु॒र्यात् । \newline
28. रक्षाꣳ॑सि य॒ज्ञ्ं ॅय॒ज्ञ्ꣳ रक्षाꣳ॑सि॒ रक्षाꣳ॑सि य॒ज्ञ्म् । \newline
29. य॒ज्ञ्ꣳ ह॑न्युर्. हन्युर् य॒ज्ञ्ं ॅय॒ज्ञ्ꣳ ह॑न्युः । \newline
30. ह॒न्यु॒र् वड् वड्ढ॑न्युर्. हन्यु॒र् वट् । \newline
31. वडितीति॒ वड् वडिति॑ । \newline
32. इत्या॑हा॒हे तीत्या॑ह । \newline
33. आ॒ह॒ प॒रोक्ष॑म् प॒रोक्ष॑ माहाह प॒रोक्ष᳚म् । \newline
34. प॒रोक्ष॑ मे॒वैव प॒रोक्ष॑म् प॒रोक्ष॑ मे॒व । \newline
35. प॒रोक्ष॒मिति॑ परः - अक्ष᳚म् । \newline
36. ए॒व वष॒ड् वष॑ डे॒वैव वष॑ट् । \newline
37. वष॑ट् करोति करोति॒ वष॒ड् वष॑ट् करोति । \newline
38. क॒रो॒ति॒ न न क॑रोति करोति॒ न । \newline
39. नास्या᳚स्य॒ न नास्य॑ । \newline
40. अ॒स्य॒ या॒तया॑मा या॒तया॑मा ऽस्यास्य या॒तया॑मा । \newline
41. या॒तया॑मा वषट्का॒रो व॑षट्का॒रो या॒तया॑मा या॒तया॑मा वषट्का॒रः । \newline
42. या॒तया॒मेति॑ या॒त - या॒मा॒ । \newline
43. व॒ष॒ट्का॒रो भव॑ति॒ भव॑ति वषट्का॒रो व॑षट्का॒रो भव॑ति । \newline
44. व॒ष॒ट्का॒र इति॑ वषट् - का॒रः । \newline
45. भव॑ति॒ न न भव॑ति॒ भव॑ति॒ न । \newline
46. न य॒ज्ञ्ं ॅय॒ज्ञ्म् न न य॒ज्ञ्म् । \newline
47. य॒ज्ञ्ꣳ रक्षाꣳ॑सि॒ रक्षाꣳ॑सि य॒ज्ञ्ं ॅय॒ज्ञ्ꣳ रक्षाꣳ॑सि । \newline
48. रक्षाꣳ॑सि घ्नन्ति घ्नन्ति॒ रक्षाꣳ॑सि॒ रक्षाꣳ॑सि घ्नन्ति । \newline
49. घ्न॒न्ति॒ सुजा॑तः॒ सुजा॑तो घ्नन्ति घ्नन्ति॒ सुजा॑तः । \newline
50. सुजा॑तो॒ ज्योति॑षा॒ ज्योति॑षा॒ सुजा॑तः॒ सुजा॑तो॒ ज्योति॑षा । \newline
51. सुजा॑त॒ इति॒ सु - जा॒तः॒ । \newline
52. ज्योति॑षा स॒ह स॒ह ज्योति॑षा॒ ज्योति॑षा स॒ह । \newline
53. स॒हे तीति॑ स॒ह स॒हेति॑ । \newline
54. इत्य॑नु॒ष्टुभा॑ ऽनु॒ष्टुभेती त्य॑नु॒ष्टुभा᳚ । \newline
55. अ॒नु॒ष्टुभो पोपा॑नु॒ष्टुभा॑ ऽनु॒ष्टुभोप॑ । \newline
56. अ॒नु॒ष्टुभेत्य॑नु - स्तुभा᳚ । \newline
57. उप॑ नह्यति नह्य॒ त्युपोप॑ नह्यति । \newline
58. न॒ह्य॒ त्य॒नु॒ष्टु ब॑नु॒ष्टुम् न॑ह्यति नह्य त्यनु॒ष्टुप् । \newline
59. अ॒नु॒ष्टुफ् सर्वा॑णि॒ सर्वा᳚ ण्यनु॒ष्टु ब॑नु॒ष्टुफ् सर्वा॑णि । \newline
60. अ॒नु॒ष्टुबित्य॑नु - स्तुप् । \newline

\textbf{Ghana Paata } \newline

1. तुभ्य॒ मितीति॒ तुभ्य॒म् तुभ्य॒ मित्या॑ हा॒हेति॒ तुभ्य॒म् तुभ्य॒ मित्या॑ह । \newline
2. इत्या॑ हा॒हेती त्या॑ह॒ षट् थ्षडा॒हेती त्या॑ह॒ षट् । \newline
3. आ॒ह॒ षट् थ्षडा॑हाह॒ षड् वै वै षडा॑हाह॒ षड् वै । \newline
4. षड् वै वै षट् थ्षड् वा ऋ॒तव॑ ऋ॒तवो॒ वै षट् थ्षड् वा ऋ॒तवः॑ । \newline
5. वा ऋ॒तव॑ ऋ॒तवो॒ वै वा ऋ॒तव॑ ऋ॒तु ष्वृ॒तु ष्वृ॒तवो॒ वै वा ऋ॒तव॑ ऋ॒तुषु॑ । \newline
6. ऋ॒तव॑ ऋ॒तु ष्वृ॒तु ष्वृ॒तव॑ ऋ॒तव॑ ऋ॒तु ष्वे॒वैव र्‌तु ष्वृ॒तव॑ ऋ॒तव॑ ऋ॒तु ष्वे॒व । \newline
7. ऋ॒तु ष्वे॒वैव र्‌तुष्वृ॒तु ष्वे॒व वृष्टिं॒ ॅवृष्टि॑ मे॒व र्‌तु ष्वृ॒तु ष्वे॒व वृष्टि᳚म् । \newline
8. ए॒व वृष्टिं॒ ॅवृष्टि॑ मे॒वैव वृष्टि॑म् दधाति दधाति॒ वृष्टि॑ मे॒वैव वृष्टि॑म् दधाति । \newline
9. वृष्टि॑म् दधाति दधाति॒ वृष्टिं॒ ॅवृष्टि॑म् दधाति॒ तस्मा॒त् तस्मा᳚द् दधाति॒ वृष्टिं॒ ॅवृष्टि॑म् दधाति॒ तस्मा᳚त् । \newline
10. द॒धा॒ति॒ तस्मा॒त् तस्मा᳚द् दधाति दधाति॒ तस्मा॒थ् सर्वा॒न् थ्सर्वा॒न् तस्मा᳚द् दधाति दधाति॒ तस्मा॒थ् सर्वान्॑ । \newline
11. तस्मा॒थ् सर्वा॒न् थ्सर्वा॒न् तस्मा॒त् तस्मा॒थ् सर्वा॑ नृ॒तू नृ॒तून् थ्सर्वा॒न् तस्मा॒त् तस्मा॒थ् सर्वा॑ नृ॒तून् । \newline
12. सर्वा॑ नृ॒तू नृ॒तून् थ्सर्वा॒न् थ्सर्वा॑ नृ॒तून्. व॑र्.षति वर्.ष त्यृ॒तून् थ्सर्वा॒न् थ्सर्वा॑ नृ॒तून्. व॑र्.षति । \newline
13. ऋ॒तून्. व॑र्.षति वर्.ष त्यृ॒तू नृ॒तून्. व॑र्.षति॒ यद् यद् व॑र्.ष त्यृ॒तू नृ॒तून्. व॑र्.षति॒ यत् । \newline
14. व॒र्॒.ष॒ति॒ यद् यद् व॑र्.षति वर्.षति॒ यद् व॑षट्कु॒र्याद् व॑षट्कु॒र्याद् यद् व॑र्.षति वर्.षति॒ यद् व॑षट्कु॒र्यात् । \newline
15. यद् व॑षट्कु॒र्याद् व॑षट्कु॒र्याद् यद् यद् व॑षट्कु॒र्याद् या॒तया॑मा या॒तया॑मा वषट्कु॒र्याद् यद् यद् व॑षट्कु॒र्याद् या॒तया॑मा । \newline
16. व॒ष॒ट्कु॒र्याद् या॒तया॑मा या॒तया॑मा वषट्कु॒र्याद् व॑षट्कु॒र्याद् या॒तया॑मा ऽस्यास्य या॒तया॑मा वषट्कु॒र्याद् व॑षट्कु॒र्याद् या॒तया॑मा ऽस्य । \newline
17. व॒ष॒ट्कु॒र्यादिति॑ वषट् - कु॒र्यात् । \newline
18. या॒तया॑मा ऽस्यास्य या॒तया॑मा या॒तया॑मा ऽस्य वषट्का॒रो व॑षट्का॒रो᳚ ऽस्य या॒तया॑मा या॒तया॑मा ऽस्य वषट्का॒रः । \newline
19. या॒तया॒मेति॑ या॒त - या॒मा॒ । \newline
20. अ॒स्य॒ व॒ष॒ट्का॒रो व॑षट्का॒रो᳚ ऽस्यास्य वषट्का॒रः स्या᳚थ् स्याद् वषट्का॒रो᳚ ऽस्यास्य वषट्का॒रः स्या᳚त् । \newline
21. व॒ष॒ट्का॒रः स्या᳚थ् स्याद् वषट्का॒रो व॑षट्का॒रः स्या॒द् यद् यथ् स्या᳚द् वषट्का॒रो व॑षट्का॒रः स्या॒द् यत् । \newline
22. व॒ष॒ट्का॒र इति॑ वषट् - का॒रः । \newline
23. स्या॒द् यद् यथ् स्या᳚थ् स्या॒द् यन् न न यथ् स्या᳚थ् स्या॒द् यन् न । \newline
24. यन् न न यद् यन् न व॑षट्कु॒र्याद् व॑षट्कु॒र्यान् न यद् यन् न व॑षट्कु॒र्यात् । \newline
25. न व॑षट्कु॒र्याद् व॑षट्कु॒र्यान् न न व॑षट्कु॒र्याद् रक्षाꣳ॑सि॒ रक्षाꣳ॑सि वषट्कु॒र्यान् न न व॑षट्कु॒र्याद् रक्षाꣳ॑सि । \newline
26. व॒ष॒ट्कु॒र्याद् रक्षाꣳ॑सि॒ रक्षाꣳ॑सि वषट्कु॒र्याद् व॑षट्कु॒र्याद् रक्षाꣳ॑सि य॒ज्ञ्ं ॅय॒ज्ञ्ꣳ रक्षाꣳ॑सि वषट्कु॒र्याद् व॑षट्कु॒र्याद् रक्षाꣳ॑सि य॒ज्ञ्म् । \newline
27. व॒ष॒ट्कु॒र्यादिति॑ वषट् - कु॒र्यात् । \newline
28. रक्षाꣳ॑सि य॒ज्ञ्ं ॅय॒ज्ञ्ꣳ रक्षाꣳ॑सि॒ रक्षाꣳ॑सि य॒ज्ञ्ꣳ ह॑न्युर्. हन्युर् य॒ज्ञ्ꣳ रक्षाꣳ॑सि॒ रक्षाꣳ॑सि य॒ज्ञ्ꣳ ह॑न्युः । \newline
29. य॒ज्ञ्ꣳ ह॑न्युर्. हन्युर् य॒ज्ञ्ं ॅय॒ज्ञ्ꣳ ह॑न्यु॒र् वड् वड्ढ॑न्युर् य॒ज्ञ्ं ॅय॒ज्ञ्ꣳ ह॑न्यु॒र् वट् । \newline
30. ह॒न्यु॒र् वड् वड्ढ॑न्युर्. हन्यु॒र् वडितीति॒ वड्ढ॑न्युर्. हन्यु॒र् वडिति॑ । \newline
31. वडितीति॒ वड् वडित्या॑ हा॒हेति॒ वड् वडित्या॑ह । \newline
32. इत्या॑हा॒हे तीत्या॑ह प॒रोक्ष॑म् प॒रोक्ष॑ मा॒हे तीत्या॑ह प॒रोक्ष᳚म् । \newline
33. आ॒ह॒ प॒रोक्ष॑म् प॒रोक्ष॑ माहाह प॒रोक्ष॑ मे॒वैव प॒रोक्ष॑ माहाह प॒रोक्ष॑ मे॒व । \newline
34. प॒रोक्ष॑ मे॒वैव प॒रोक्ष॑म् प॒रोक्ष॑ मे॒व वष॒ड् वष॑डे॒व प॒रोक्ष॑म् प॒रोक्ष॑ मे॒व वष॑ट् । \newline
35. प॒रोक्ष॒मिति॑ परः - अक्ष᳚म् । \newline
36. ए॒व वष॒ड् वष॑ डे॒वैव वष॑ट् करोति करोति॒ वष॑ डे॒वैव वष॑ट् करोति । \newline
37. वष॑ट् करोति करोति॒ वष॒ड् वष॑ट् करोति॒ न न क॑रोति॒ वष॒ड् वष॑ट् करोति॒ न । \newline
38. क॒रो॒ति॒ न न क॑रोति करोति॒ नास्या᳚स्य॒ न क॑रोति करोति॒ नास्य॑ । \newline
39. नास्या᳚स्य॒ न नास्य॑ या॒तया॑मा या॒तया॑मा ऽस्य॒ न नास्य॑ या॒तया॑मा । \newline
40. अ॒स्य॒ या॒तया॑मा या॒तया॑मा ऽस्यास्य या॒तया॑मा वषट्का॒रो व॑षट्का॒रो या॒तया॑मा ऽस्यास्य या॒तया॑मा वषट्का॒रः । \newline
41. या॒तया॑मा वषट्का॒रो व॑षट्का॒रो या॒तया॑मा या॒तया॑मा वषट्का॒रो भव॑ति॒ भव॑ति वषट्का॒रो या॒तया॑मा या॒तया॑मा वषट्का॒रो भव॑ति । \newline
42. या॒तया॒मेति॑ या॒त - या॒मा॒ । \newline
43. व॒ष॒ट्का॒रो भव॑ति॒ भव॑ति वषट्का॒रो व॑षट्का॒रो भव॑ति॒ न न भव॑ति वषट्का॒रो व॑षट्का॒रो भव॑ति॒ न । \newline
44. व॒ष॒ट्का॒र इति॑ वषट् - का॒रः । \newline
45. भव॑ति॒ न न भव॑ति॒ भव॑ति॒ न य॒ज्ञ्ं ॅय॒ज्ञ्म् न भव॑ति॒ भव॑ति॒ न य॒ज्ञ्म् । \newline
46. न य॒ज्ञ्ं ॅय॒ज्ञ्म् न न य॒ज्ञ्ꣳ रक्षाꣳ॑सि॒ रक्षाꣳ॑सि य॒ज्ञ्म् न न य॒ज्ञ्ꣳ रक्षाꣳ॑सि । \newline
47. य॒ज्ञ्ꣳ रक्षाꣳ॑सि॒ रक्षाꣳ॑सि य॒ज्ञ्ं ॅय॒ज्ञ्ꣳ रक्षाꣳ॑सि घ्नन्ति घ्नन्ति॒ रक्षाꣳ॑सि य॒ज्ञ्ं ॅय॒ज्ञ्ꣳ रक्षाꣳ॑सि घ्नन्ति । \newline
48. रक्षाꣳ॑सि घ्नन्ति घ्नन्ति॒ रक्षाꣳ॑सि॒ रक्षाꣳ॑सि घ्नन्ति॒ सुजा॑तः॒ सुजा॑तो घ्नन्ति॒ रक्षाꣳ॑सि॒ रक्षाꣳ॑सि घ्नन्ति॒ सुजा॑तः । \newline
49. घ्न॒न्ति॒ सुजा॑तः॒ सुजा॑तो घ्नन्ति घ्नन्ति॒ सुजा॑तो॒ ज्योति॑षा॒ ज्योति॑षा॒ सुजा॑तो घ्नन्ति घ्नन्ति॒ सुजा॑तो॒ ज्योति॑षा । \newline
50. सुजा॑तो॒ ज्योति॑षा॒ ज्योति॑षा॒ सुजा॑तः॒ सुजा॑तो॒ ज्योति॑षा स॒ह स॒ह ज्योति॑षा॒ सुजा॑तः॒ सुजा॑तो॒ ज्योति॑षा स॒ह । \newline
51. सुजा॑त॒ इति॒ सु - जा॒तः॒ । \newline
52. ज्योति॑षा स॒ह स॒ह ज्योति॑षा॒ ज्योति॑षा स॒हे तीति॑ स॒ह ज्योति॑षा॒ ज्योति॑षा स॒हेति॑ । \newline
53. स॒हे तीति॑ स॒ह स॒हे त्य॑नु॒ष्टुभा॑ ऽनु॒ष्टुभेति॑ स॒ह स॒हे त्य॑नु॒ष्टुभा᳚ । \newline
54. इत्य॑नु॒ष्टुभा॑ ऽनु॒ष्टुभेती त्य॑नु॒ष्टु भोपोपा॑ नु॒ष्टुभेती त्य॑नु॒ष्टुभोप॑ । \newline
55. अ॒नु॒ष्टु भोपोपा॑ नु॒ष्टुभा॑ ऽनु॒ष्टुभोप॑ नह्यति नह्य॒ त्युपा॑ नु॒ष्टुभा॑ ऽनु॒ष्टुभोप॑ नह्यति । \newline
56. अ॒नु॒ष्टुभेत्य॑नु - स्तुभा᳚ । \newline
57. उप॑ नह्यति नह्य॒ त्युपोप॑ नह्य त्यनु॒ष्टु ब॑नु॒ष्टुम् न॑ह्य॒ त्युपोप॑ नह्य त्यनु॒ष्टुप् । \newline
58. न॒ह्य॒ त्य॒नु॒ष्टु ब॑नु॒ष्टुम् न॑ह्यति नह्य त्यनु॒ष्टुफ् सर्वा॑णि॒ सर्वा᳚ ण्यनु॒ष्टुम् न॑ह्यति नह्य त्यनु॒ष्टुफ् सर्वा॑णि । \newline
59. अ॒नु॒ष्टुफ् सर्वा॑णि॒ सर्वा᳚ ण्यनु॒ष्टु ब॑नु॒ष्टुफ् सर्वा॑णि॒ छन्दाꣳ॑सि॒ छन्दाꣳ॑सि॒ सर्वा᳚ण्यनु॒ष्टु ब॑नु॒ष्टुफ् सर्वा॑णि॒ छन्दाꣳ॑सि । \newline
60. अ॒नु॒ष्टुबित्य॑नु - स्तुप् । \newline
\pagebreak
\markright{ TS 5.1.5.3  \hfill https://www.vedavms.in \hfill}

\section{ TS 5.1.5.3 }

\textbf{TS 5.1.5.3 } \newline
\textbf{Samhita Paata} \newline

-फ्सर्वा॑णि॒ छन्दाꣳ॑सि॒ छन्दाꣳ॑सि॒ खलु॒ वा अ॒ग्नेः प्रि॒या त॒नूः प्रि॒ययै॒वैनं॑ त॒नुवा॒ परि॑ दधाति॒ वेदु॑को॒ वासो॑ भवति॒य ए॒वं ॅवेद॑ वारु॒णो वा अ॒ग्निरुप॑नद्ध॒ उदु॑ तिष्ठ स्वद्ध्वरो॒र्द्ध्व ऊ॒ षुण॑ ऊ॒तय॒ इति॑ सावि॒त्रीभ्या॒मुत् ति॑ष्ठति सवि॒तृप्र॑सूत ए॒वास्यो॒र्द्ध्वां ॅव॑रुणमे॒निमुथ् सृ॑जति॒ द्वाभ्यां॒ प्रति॑ष्ठित्यै॒ स जा॒तो गर्भो॑ असि॒ - [  ] \newline

\textbf{Pada Paata} \newline

सर्वा॑णि । छन्दाꣳ॑सि । छन्दाꣳ॑सि । खलु॑ । वै । अ॒ग्नेः । प्रि॒या । त॒नूः । प्रि॒यया᳚ । ए॒व । ए॒न॒म् । त॒नुवा᳚ । परीति॑ । द॒धा॒ति॒ । वेदु॑कः । वासः॑ । भ॒व॒ति॒ । यः । ए॒वम् । वेद॑ । वा॒रु॒णः । वै । अ॒ग्निः । उप॑नद्ध॒ इत्युप॑ - न॒द्धः॒ । उदिति॑ । उ॒ । ति॒ष्ठ॒ । स्व॒द्ध्व॒रेति॑ सु - अ॒द्ध्व॒र॒ । ऊ॒द्‌र्ध्वः । उ॒ । स्विति॑ । नः॒ । ऊ॒तये᳚ । इति॑ । सा॒वि॒त्रीभ्या᳚म् । उदिति॑ । ति॒ष्ठ॒ति॒ । स॒वि॒तृप्र॑सूत॒ इति॑ सवि॒तृ - प्र॒सू॒तः॒ । ए॒व । अ॒स्य॒ । ऊ॒द्‌र्ध्वाम् । व॒रु॒ण॒मे॒निमिति॑ वरुण - मे॒निम् । उदिति॑ । सृ॒ज॒ति॒ । द्वाभ्या᳚म् । प्रति॑ष्ठित्या॒ इति॒ प्रति॑ - स्थि॒त्यै॒ । सः । जा॒तः । गर्भः॑ । अ॒सि॒ ।  \newline


\textbf{Krama Paata} \newline

सर्वा॑णि॒ छन्दाꣳ॑सि । छन्दाꣳ॑सि॒ छन्दाꣳ॑सि । छन्दाꣳ॑सि॒ खलु॑ । खलु॒ वै । वा अ॒ग्नेः । अ॒ग्नेः प्रि॒या । प्रि॒या त॒नूः । त॒नूः प्रि॒यया᳚ । प्रि॒ययै॒व । ए॒वैन᳚म् । ए॒न॒म् त॒नुवा᳚ । त॒नुवा॒ परि॑ । परि॑ दधाति । द॒धा॒ति॒ वेदु॑कः । वेदु॑को॒ वासः॑ । वासो॑ भवति । भ॒व॒ति॒ यः । य ए॒वम् । ए॒वम् ॅवेद॑ । वेद॑ वारु॒णः । वा॒रु॒णो वै । वा अ॒ग्निः । अ॒ग्निरुप॑नद्धः । उप॑नद्ध॒ उत् । उप॑नद्ध॒ इत्युप॑ - न॒द्धः॒ । उदु॑ । उ॒ ति॒ष्ठ॒ । ति॒ष्ठ॒ स्व॒द्ध्व॒र॒ । स्व॒द्ध्व॒रो॒र्द्ध्वः । स्व॒द्ध्व॒रेति॑ सु - अ॒द्ध्व॒र॒ । ऊ॒र्द्ध्व उ । ऊ॒ षु णः॑ । सु नः॑ । न॒ ऊ॒तये᳚ । ऊ॒तय॒ इति॑ । इति॑ सावि॒त्रीभ्या᳚म् । सा॒वि॒त्रीभ्या॒मुत् । उत् ति॑ष्ठति । ति॒ष्ठ॒ति॒ स॒वि॒तृप्र॑सूतः । स॒वि॒तृप्र॑सूत ए॒व । स॒वि॒तृप्र॑सूत॒ इति॑ सवि॒तृ - प्र॒सू॒तः॒ । ए॒वास्य॑ । अ॒स्यो॒र्द्ध्वाम् । ऊ॒र्द्ध्वाम् ॅव॑रुणमे॒निम् । व॒रु॒ण॒मे॒निमुत् । व॒रु॒ण॒मे॒निमिति॑ वरुण - मे॒निम् । उथ् सृ॑जति । सृ॒ज॒ति॒ द्वाभ्या᳚म् । द्वाभ्या॒म् प्रति॑ष्ठित्यै । प्रति॑ष्ठित्यै॒ सः । प्रति॑ष्ठित्या॒ इति॒ प्रति॑ - स्थि॒त्यै॒ । स जा॒तः । जा॒तो गर्भः॑ । गर्भो॑ असि । अ॒सि॒ रोद॑स्योः \newline

\textbf{Jatai Paata} \newline

1. सर्वा॑णि॒ छन्दाꣳ॑सि॒ छन्दाꣳ॑सि॒ सर्वा॑णि॒ सर्वा॑णि॒ छन्दाꣳ॑सि । \newline
2. छन्दाꣳ॑सि॒ छन्दाꣳ॑सि । \newline
3. छन्दाꣳ॑सि॒ खलु॒ खलु॒ छन्दाꣳ॑सि॒ छन्दाꣳ॑सि॒ खलु॑ । \newline
4. खलु॒ वै वै खलु॒ खलु॒ वै । \newline
5. वा अ॒ग्ने र॒ग्नेर् वै वा अ॒ग्नेः । \newline
6. अ॒ग्नेः प्रि॒या प्रि॒या ऽग्ने र॒ग्नेः प्रि॒या । \newline
7. प्रि॒या त॒नू स्त॒नूः प्रि॒या प्रि॒या त॒नूः । \newline
8. त॒नूः प्रि॒यया᳚ प्रि॒यया॑ त॒नू स्त॒नूः प्रि॒यया᳚ । \newline
9. प्रि॒य यै॒वैव प्रि॒यया᳚ प्रि॒य यै॒व । \newline
10. ए॒वैन॑ मेन मे॒वैवैन᳚म् । \newline
11. ए॒न॒म् त॒नुवा॑ त॒नुवै॑न मेनम् त॒नुवा᳚ । \newline
12. त॒नुवा॒ परि॒ परि॑ त॒नुवा॑ त॒नुवा॒ परि॑ । \newline
13. परि॑ दधाति दधाति॒ परि॒ परि॑ दधाति । \newline
14. द॒धा॒ति॒ वेदु॑को॒ वेदु॑को दधाति दधाति॒ वेदु॑कः । \newline
15. वेदु॑को॒ वासो॒ वासो॒ वेदु॑को॒ वेदु॑को॒ वासः॑ । \newline
16. वासो॑ भवति भवति॒ वासो॒ वासो॑ भवति । \newline
17. भ॒व॒ति॒ यो यो भ॑वति भवति॒ यः । \newline
18. य ए॒व मे॒वं ॅयो य ए॒वम् । \newline
19. ए॒वं ॅवेद॒ वेदै॒व मे॒वं ॅवेद॑ । \newline
20. वेद॑ वारु॒णो वा॑रु॒णो वेद॒ वेद॑ वारु॒णः । \newline
21. वा॒रु॒णो वै वै वा॑रु॒णो वा॑रु॒णो वै । \newline
22. वा अ॒ग्नि र॒ग्निर् वै वा अ॒ग्निः । \newline
23. अ॒ग्नि रुप॑नद्ध॒ उप॑नद्धो॒ ऽग्नि र॒ग्नि रुप॑नद्धः । \newline
24. उप॑नद्ध॒ उदु दुप॑नद्ध॒ उप॑नद्ध॒ उत् । \newline
25. उप॑नद्ध॒ इत्युप॑ - न॒द्धः॒ । \newline
26. उदु॑ वु॒वु दुदु॑ । \newline
27. उ॒ ति॒ष्ठ॒ ति॒ष्ठ॒ वु॒ ति॒ष्ठ॒ । \newline
28. ति॒ष्ठ॒ स्व॒द्ध्व॒र॒ स्व॒द्ध्व॒र॒ ति॒ष्ठ॒ ति॒ष्ठ॒ स्व॒द्ध्व॒र॒ । \newline
29. स्व॒द्ध्व॒ रो॒र्द्ध्व ऊ॒र्द्ध्वः स्व॑द्ध्वर स्वद्ध्व रो॒र्द्ध्वः । \newline
30. स्व॒द्ध्व॒रेति॑ सु - अ॒द्ध्व॒र॒ । \newline
31. ऊ॒र्द्ध्व उ॑ वु वू॒र्द्ध्व ऊ॒र्द्ध्व उ॑ । \newline
32. ऊ॒ षु णो॑ नः॒ सू॑ षु णः॑ । \newline
33. सु नो॑ नः॒ सु सु नः॑ । \newline
34. न॒ ऊ॒तय॑ ऊ॒तये॑ नो न ऊ॒तये᳚ । \newline
35. ऊ॒तय॒ इतीत्यू॒तय॑ ऊ॒तय॒ इति॑ । \newline
36. इति॑ सावि॒त्रीभ्याꣳ॑ सावि॒त्रीभ्या॒ मितीति॑ सावि॒त्रीभ्या᳚म् । \newline
37. सा॒वि॒त्रीभ्या॒ मुदुथ् सा॑वि॒त्रीभ्याꣳ॑ सावि॒त्रीभ्या॒ मुत् । \newline
38. उत् ति॑ष्ठति तिष्ठ॒ त्युदुत् ति॑ष्ठति । \newline
39. ति॒ष्ठ॒ति॒ स॒वि॒तृप्र॑सूतः सवि॒तृप्र॑सूत स्तिष्ठति तिष्ठति सवि॒तृप्र॑सूतः । \newline
40. स॒वि॒तृप्र॑सूत ए॒वैव स॑वि॒तृप्र॑सूतः सवि॒तृप्र॑सूत ए॒व । \newline
41. स॒वि॒तृप्र॑सूत॒ इति॑ सवि॒तृ - प्र॒सू॒तः॒ । \newline
42. ए॒वास्या᳚ स्यै॒ वैवास्य॑ । \newline
43. अ॒स्यो॒र्द्ध्वा मू॒र्द्ध्वा म॑स्या स्यो॒र्द्ध्वाम् । \newline
44. ऊ॒र्द्ध्वां ॅव॑रुणमे॒निं ॅव॑रुणमे॒नि मू॒र्द्ध्वा मू॒र्द्ध्वां ॅव॑रुणमे॒निम् । \newline
45. व॒रु॒ण॒मे॒नि मुदुद् व॑रुणमे॒निं ॅव॑रुणमे॒नि मुत् । \newline
46. व॒रु॒ण॒मे॒निमिति॑ वरुण - मे॒निम् । \newline
47. उथ् सृ॑जति सृज॒ त्युदुथ् सृ॑जति । \newline
48. सृ॒ज॒ति॒ द्वाभ्या॒म् द्वाभ्याꣳ॑ सृजति सृजति॒ द्वाभ्या᳚म् । \newline
49. द्वाभ्या॒म् प्रति॑ष्ठित्यै॒ प्रति॑ष्ठित्यै॒ द्वाभ्या॒म् द्वाभ्या॒म् प्रति॑ष्ठित्यै । \newline
50. प्रति॑ष्ठित्यै॒ स स प्रति॑ष्ठित्यै॒ प्रति॑ष्ठित्यै॒ सः । \newline
51. प्रति॑ष्ठित्या॒ इति॒ प्रति॑ - स्थि॒त्यै॒ । \newline
52. स जा॒तो जा॒तः स स जा॒तः । \newline
53. जा॒तो गर्भो॒ गर्भो॑ जा॒तो जा॒तो गर्भः॑ । \newline
54. गर्भो॑ अस्यसि॒ गर्भो॒ गर्भो॑ असि । \newline
55. अ॒सि॒ रोद॑स्यो॒ रोद॑स्यो रस्यसि॒ रोद॑स्योः । \newline

\textbf{Ghana Paata } \newline

1. सर्वा॑णि॒ छन्दाꣳ॑सि॒ छन्दाꣳ॑सि॒ सर्वा॑णि॒ सर्वा॑णि॒ छन्दाꣳ॑सि । \newline
2. छन्दाꣳ॑सि॒ छन्दाꣳ॑सि । \newline
3. छन्दाꣳ॑सि॒ खलु॒ खलु॒ छन्दाꣳ॑सि॒ छन्दाꣳ॑सि॒ खलु॒ वै वै खलु॒ छन्दाꣳ॑सि॒ छन्दाꣳ॑सि॒ खलु॒ वै । \newline
4. खलु॒ वै वै खलु॒ खलु॒ वा अ॒ग्ने र॒ग्नेर् वै खलु॒ खलु॒ वा अ॒ग्नेः । \newline
5. वा अ॒ग्ने र॒ग्नेर् वै वा अ॒ग्नेः प्रि॒या प्रि॒या ऽग्नेर् वै वा अ॒ग्नेः प्रि॒या । \newline
6. अ॒ग्नेः प्रि॒या प्रि॒या ऽग्ने र॒ग्नेः प्रि॒या त॒नू स्त॒नूः प्रि॒या ऽग्ने र॒ग्नेः प्रि॒या त॒नूः । \newline
7. प्रि॒या त॒नू स्त॒नूः प्रि॒या प्रि॒या त॒नूः प्रि॒यया᳚ प्रि॒यया॑ त॒नूः प्रि॒या प्रि॒या त॒नूः प्रि॒यया᳚ । \newline
8. त॒नूः प्रि॒यया᳚ प्रि॒यया॑ त॒नू स्त॒नूः प्रि॒ययै॒वैव प्रि॒यया॑ त॒नू स्त॒नूः प्रि॒ययै॒व । \newline
9. प्रि॒ययै॒वैव प्रि॒यया᳚ प्रि॒ययै॒वैन॑ मेन मे॒व प्रि॒यया᳚ प्रि॒य यै॒वैन᳚म् । \newline
10. ए॒वैन॑ मेन मे॒वैवैन॑म् त॒नुवा॑ त॒नुवै॑न मे॒वैवैन॑म् त॒नुवा᳚ । \newline
11. ए॒न॒म् त॒नुवा॑ त॒नुवै॑न मेनम् त॒नुवा॒ परि॒ परि॑ त॒नुवै॑न मेनम् त॒नुवा॒ परि॑ । \newline
12. त॒नुवा॒ परि॒ परि॑ त॒नुवा॑ त॒नुवा॒ परि॑ दधाति दधाति॒ परि॑ त॒नुवा॑ त॒नुवा॒ परि॑ दधाति । \newline
13. परि॑ दधाति दधाति॒ परि॒ परि॑ दधाति॒ वेदु॑को॒ वेदु॑को दधाति॒ परि॒ परि॑ दधाति॒ वेदु॑कः । \newline
14. द॒धा॒ति॒ वेदु॑को॒ वेदु॑को दधाति दधाति॒ वेदु॑को॒ वासो॒ वासो॒ वेदु॑को दधाति दधाति॒ वेदु॑को॒ वासः॑ । \newline
15. वेदु॑को॒ वासो॒ वासो॒ वेदु॑को॒ वेदु॑को॒ वासो॑ भवति भवति॒ वासो॒ वेदु॑को॒ वेदु॑को॒ वासो॑ भवति । \newline
16. वासो॑ भवति भवति॒ वासो॒ वासो॑ भवति॒ यो यो भ॑वति॒ वासो॒ वासो॑ भवति॒ यः । \newline
17. भ॒व॒ति॒ यो यो भ॑वति भवति॒ य ए॒व मे॒वं ॅयो भ॑वति भवति॒ य ए॒वम् । \newline
18. य ए॒व मे॒वं ॅयो य ए॒वं ॅवेद॒ वेदै॒वं ॅयो य ए॒वं ॅवेद॑ । \newline
19. ए॒वं ॅवेद॒ वेदै॒व मे॒वं ॅवेद॑ वारु॒णो वा॑रु॒णो वेदै॒व मे॒वं ॅवेद॑ वारु॒णः । \newline
20. वेद॑ वारु॒णो वा॑रु॒णो वेद॒ वेद॑ वारु॒णो वै वै वा॑रु॒णो वेद॒ वेद॑ वारु॒णो वै । \newline
21. वा॒रु॒णो वै वै वा॑रु॒णो वा॑रु॒णो वा अ॒ग्नि र॒ग्निर् वै वा॑रु॒णो वा॑रु॒णो वा अ॒ग्निः । \newline
22. वा अ॒ग्नि र॒ग्निर् वै वा अ॒ग्नि रुप॑नद्ध॒ उप॑नद्धो॒ ऽग्निर् वै वा अ॒ग्नि रुप॑नद्धः । \newline
23. अ॒ग्नि रुप॑नद्ध॒ उप॑नद्धो॒ ऽग्नि र॒ग्नि रुप॑नद्ध॒ उदु दुप॑नद्धो॒ ऽग्नि र॒ग्नि रुप॑नद्ध॒ उत् । \newline
24. उप॑नद्ध॒ उदु दुप॑नद्ध॒ उप॑नद्ध॒ उदु॑ वु॒ वुदुप॑नद्ध॒ उप॑नद्ध॒ उदु॑ । \newline
25. उप॑नद्ध॒ इत्युप॑ - न॒द्धः॒ । \newline
26. उदु॑ वु॒ वु दुदु॑ तिष्ठ तिष्ठ॒ वु दुदु॑ तिष्ठ । \newline
27. उ॒ ति॒ष्ठ॒ ति॒ष्ठ॒ वु॒ ति॒ष्ठ॒ स्व॒द्ध्व॒र॒ स्व॒द्ध्व॒र॒ ति॒ष्ठ॒ वु॒ ति॒ष्ठ॒ स्व॒द्ध्व॒र॒ । \newline
28. ति॒ष्ठ॒ स्व॒द्ध्व॒र॒ स्व॒द्ध्व॒र॒ ति॒ष्ठ॒ ति॒ष्ठ॒ स्व॒द्ध्व॒ रो॒र्द्ध्व ऊ॒र्द्ध्वः स्व॑द्ध्वर तिष्ठ तिष्ठ स्वद्ध्व रो॒र्द्ध्वः । \newline
29. स्व॒द्ध्व॒ रो॒र्द्ध्व ऊ॒र्द्ध्वः स्व॑द्ध्वर स्वद्ध्व रो॒र्द्ध्व उ॑ वु वू॒र्द्ध्वः स्व॑द्ध्वर स्वद्ध्व रो॒र्द्ध्व उ॑ । \newline
30. स्व॒द्ध्व॒रेति॑ सु - अ॒द्ध्व॒र॒ । \newline
31. ऊ॒र्द्ध्व उ॑ वु वू॒र्द्ध्व ऊ॒र्द्ध्व ऊ॒ षु णो॑ नः॒ सू᳚ र्द्ध्व ऊ॒र्द्ध्व ऊ॒ षु णः॑ । \newline
32. ऊ॒ षु णो॑ नः॒ सू॑ षु ण॑ ऊ॒तय॑ ऊ॒तये॑ नः॒ सू॑ षु ण॑ ऊ॒तये᳚ । \newline
33. सु नो॑ नः॒ सु सु न॑ ऊ॒तय॑ ऊ॒तये॑ नः॒ सु सु न॑ ऊ॒तये᳚ । \newline
34. न॒ ऊ॒तय॑ ऊ॒तये॑ नो न ऊ॒तय॒ इती त्यू॒तये॑ नो न ऊ॒तय॒ इति॑ । \newline
35. ऊ॒तय॒ इतीत्यू॒तय॑ ऊ॒तय॒ इति॑ सावि॒त्रीभ्याꣳ॑ सावि॒त्रीभ्या॒ मित्यू॒तय॑ ऊ॒तय॒ इति॑ सावि॒त्रीभ्या᳚म् । \newline
36. इति॑ सावि॒त्रीभ्याꣳ॑ सावि॒त्रीभ्या॒ मितीति॑ सावि॒त्रीभ्या॒ मुदुथ् सा॑वि॒त्रीभ्या॒ मितीति॑ सावि॒त्रीभ्या॒ मुत् । \newline
37. सा॒वि॒त्रीभ्या॒ मुदुथ् सा॑वि॒त्रीभ्याꣳ॑ सावि॒त्रीभ्या॒ मुत् ति॑ष्ठति तिष्ठ॒त्युथ् सा॑वि॒त्रीभ्याꣳ॑ सावि॒त्रीभ्या॒ मुत् ति॑ष्ठति । \newline
38. उत् ति॑ष्ठति तिष्ठ॒ त्युदुत् ति॑ष्ठति सवि॒तृप्र॑सूतः सवि॒तृप्र॑सूत स्तिष्ठ॒ त्युदुत् ति॑ष्ठति सवि॒तृप्र॑सूतः । \newline
39. ति॒ष्ठ॒ति॒ स॒वि॒तृप्र॑सूतः सवि॒तृप्र॑सूत स्तिष्ठति तिष्ठति सवि॒तृप्र॑सूत ए॒वैव स॑वि॒तृप्र॑सूत स्तिष्ठति तिष्ठति सवि॒तृप्र॑सूत ए॒व । \newline
40. स॒वि॒तृप्र॑सूत ए॒वैव स॑वि॒तृप्र॑सूतः सवि॒तृप्र॑सूत ए॒वास्या᳚ स्यै॒व स॑वि॒तृप्र॑सूतः सवि॒तृप्र॑सूत ए॒वास्य॑ । \newline
41. स॒वि॒तृप्र॑सूत॒ इति॑ सवि॒तृ - प्र॒सू॒तः॒ । \newline
42. ए॒वास्या᳚ स्यै॒वैवा स्यो॒र्द्ध्वा मू॒र्द्ध्वा म॑स्यै॒वैवा स्यो॒र्द्ध्वाम् । \newline
43. अ॒स्यो॒र्द्ध्वा मू॒र्द्ध्वा म॑स्या स्यो॒र्द्ध्वां ॅव॑रुणमे॒निं ॅव॑रुणमे॒नि मू॒र्द्ध्वा म॑स्या स्यो॒र्द्ध्वां ॅव॑रुणमे॒निम् । \newline
44. ऊ॒र्द्ध्वां ॅव॑रुणमे॒निं ॅव॑रुणमे॒नि मू॒र्द्ध्वा मू॒र्द्ध्वां ॅव॑रुणमे॒नि मुदुद् व॑रुणमे॒नि मू॒र्द्ध्वा मू॒र्द्ध्वां ॅव॑रुणमे॒नि मुत् । \newline
45. व॒रु॒ण॒मे॒नि मुदुद् व॑रुणमे॒निं ॅव॑रुणमे॒नि मुथ् सृ॑जति सृज॒त्युद् व॑रुणमे॒निं ॅव॑रुणमे॒नि मुथ् सृ॑जति । \newline
46. व॒रु॒ण॒मे॒निमिति॑ वरुण - मे॒निम् । \newline
47. उथ् सृ॑जति सृज॒त्युदुथ् सृ॑जति॒ द्वाभ्या॒म् द्वाभ्याꣳ॑ सृज॒ त्युदुथ् सृ॑जति॒ द्वाभ्या᳚म् । \newline
48. सृ॒ज॒ति॒ द्वाभ्या॒म् द्वाभ्याꣳ॑ सृजति सृजति॒ द्वाभ्या॒म् प्रति॑ष्ठित्यै॒ प्रति॑ष्ठित्यै॒ द्वाभ्याꣳ॑ सृजति सृजति॒ द्वाभ्या॒म् प्रति॑ष्ठित्यै । \newline
49. द्वाभ्या॒म् प्रति॑ष्ठित्यै॒ प्रति॑ष्ठित्यै॒ द्वाभ्या॒म् द्वाभ्या॒म् प्रति॑ष्ठित्यै॒ स स प्रति॑ष्ठित्यै॒ द्वाभ्या॒म् द्वाभ्या॒म् प्रति॑ष्ठित्यै॒ सः । \newline
50. प्रति॑ष्ठित्यै॒ स स प्रति॑ष्ठित्यै॒ प्रति॑ष्ठित्यै॒ स जा॒तो जा॒तः स प्रति॑ष्ठित्यै॒ प्रति॑ष्ठित्यै॒ स जा॒तः । \newline
51. प्रति॑ष्ठित्या॒ इति॒ प्रति॑ - स्थि॒त्यै॒ । \newline
52. स जा॒तो जा॒तः स स जा॒तो गर्भो॒ गर्भो॑ जा॒तः स स जा॒तो गर्भः॑ । \newline
53. जा॒तो गर्भो॒ गर्भो॑ जा॒तो जा॒तो गर्भो॑ अस्यसि॒ गर्भो॑ जा॒तो जा॒तो गर्भो॑ असि । \newline
54. गर्भो॑ अस्यसि॒ गर्भो॒ गर्भो॑ असि॒ रोद॑स्यो॒ रोद॑स्यो रसि॒ गर्भो॒ गर्भो॑ असि॒ रोद॑स्योः । \newline
55. अ॒सि॒ रोद॑स्यो॒ रोद॑स्यो रस्यसि॒ रोद॑स्यो॒ रितीति॒ रोद॑स्यो रस्यसि॒ रोद॑स्यो॒ रिति॑ । \newline
\pagebreak
\markright{ TS 5.1.5.4  \hfill https://www.vedavms.in \hfill}

\section{ TS 5.1.5.4 }

\textbf{TS 5.1.5.4 } \newline
\textbf{Samhita Paata} \newline

रोद॑स्यो॒रित्या॑हे॒मे वै रोद॑सी॒ तयो॑रे॒ष गर्भो॒ यद॒ग्नि-स्तस्मा॑-दे॒वमा॒हाग्ने॒ चारु॒र्विभृ॑त॒ ओष॑धी॒ष्वित्या॑ह य॒दा ह्ये॑तं ॅवि॒भर॒न्त्यथ॒ चारु॑तरो॒ भव॑ति॒ प्र मा॒तृभ्यो॒ अधि॒ कनि॑क्रदद्-गा॒ इत्या॒हौष॑धयो॒ वा अ॑स्य मा॒तर॒स्ताभ्य॑ ए॒वैनं॒ प्रच्या॑वयति स्थि॒रो भ॑व वी॒ड्व॑ङ्ग॒ इति॑ गर्द॒भ आ सा॑दयति॒ - [  ] \newline

\textbf{Pada Paata} \newline

रोद॑स्योः । इति॑ । आ॒ह॒ । इ॒मे इति॑ । वै । रोद॑सी॒ इति॑ । तयोः᳚ । ए॒षः । गर्भः॑ । यत् । अ॒ग्निः । तस्मा᳚त् । ए॒वम् । आ॒ह॒ । अग्ने᳚ । चारुः॑ । विभृ॑त॒ इति॒ वि - भृ॒तः॒ । ओष॑धीषु । इति॑ । आ॒ह॒ । य॒दा । हि । ए॒तम् । वि॒भर॒न्तीति॑ वि - भर॑न्ति । अथ॑ । चारु॑तर॒ इति॒ चारु॑ - त॒रः॒ । भव॑ति । प्रेति॑ । मा॒तृभ्य॒ इति॑ मा॒तृ - भ्यः॒ । अधीति॑ । कनि॑क्रदत् । गाः॒ । इति॑ । आ॒ह॒ । ओष॑धयः । वै । अ॒स्य॒ । मा॒तरः॑ । ताभ्यः॑ । ए॒व । ए॒न॒म् । प्रेति॑ । च्या॒व॒य॒ति॒ । स्थि॒रः । भ॒व॒ । वी॒ड्व॑ङ्ग॒ इति॑ वी॒डु - अ॒ङ्गः॒ । इति॑ । ग॒र्द॒भे । एति॑ । सा॒द॒य॒ति॒ ।  \newline


\textbf{Krama Paata} \newline

रोद॑स्यो॒रिति॑ । इत्या॑ह । आ॒हे॒मे । इ॒मे वै । इ॒मे इती॒मे । वै रोद॑सी । रोद॑सी॒ तयोः᳚ । रोद॑सी॒ इति॒ रोद॑सी । तयो॑रे॒षः । ए॒ष गर्भः॑ । गर्भो॒ यत् । यद॒ग्निः । अ॒ग्निस्तस्मा᳚त् । तस्मा॑दे॒वम् । ए॒वमा॑ह । आ॒हाग्ने᳚ । अग्ने॒ चारुः॑ । चारु॒र् विभृ॑तः । विभृ॑त॒ ओष॑धीषु । विभृ॑त॒ इति॒ वि - भृ॒तः॒ । ओष॑धी॒ष्विति॑ । इत्या॑ह । आ॒ह॒ य॒दा । य॒दा हि । ह्ये॑तम् । ए॒तम् ॅवि॒भर॑न्ति । वि॒भर॒न्त्यथ॑ । वि॒भर॒न्तीति॑वि - भर॑न्ति । अथ॒ चारु॑तरः । चारु॑तरो॒ भव॑ति । चारु॑तर॒ इति॒ चारु॑ - त॒रः॒ । भव॑ति॒ प्र । प्र मा॒तृभ्यः॑ । मा॒तृभ्यो॒ अधि॑ । मा॒तृभ्य॒ इति॑ मा॒तृ - भ्यः॒ । अधि॒ कनि॑क्रदत् । कनि॑क्रदद् गाः । गा॒ इति॑ । इत्या॑ह । आ॒हौष॑धयः । ओष॑धयो॒ वै । वा अ॑स्य । अ॒स्य॒ मा॒तरः॑ । मा॒तर॒स्ताभ्यः॑ । ताभ्य॑ ए॒व । ए॒वैन᳚म् । ए॒न॒म् प्र । प्र च्या॑वयति । च्या॒व॒य॒ति॒ स्थि॒रः । स्थि॒रो भ॑व । भ॒व॒ वी॒ड्व॑ङ्गः । वी॒ड्व॑ङ्ग॒ इति॑ । वी॒ड्व॑ङ्ग॒ इति॑ वी॒डु - अ॒ङ्गः॒ । इति॑ गर्द॒भे । ग॒र्द॒भ आ । आ सा॑दयति । सा॒द॒य॒ति॒ सम् \newline

\textbf{Jatai Paata} \newline

1. रोद॑स्यो॒ रितीति॒ रोद॑स्यो॒ रोद॑स्यो॒ रिति॑ । \newline
2. इत्या॑हा॒हे तीत्या॑ह । \newline
3. आ॒हे॒मे इ॒मे आ॑हा हे॒मे । \newline
4. इ॒मे वै वा इ॒मे इ॒मे वै । \newline
5. इ॒मे इती॒मे । \newline
6. वै रोद॑सी॒ रोद॑सी॒ वै वै रोद॑सी । \newline
7. रोद॑सी॒ तयो॒ स्तयो॒ रोद॑सी॒ रोद॑सी॒ तयोः᳚ । \newline
8. रोद॑सी॒ इति॒ रोद॑सी । \newline
9. तयो॑ रे॒ष ए॒ष तयो॒ स्तयो॑ रे॒षः । \newline
10. ए॒ष गर्भो॒ गर्भ॑ ए॒ष ए॒ष गर्भः॑ । \newline
11. गर्भो॒ यद् यद् गर्भो॒ गर्भो॒ यत् । \newline
12. यद॒ग्नि र॒ग्निर् यद् यद॒ग्निः । \newline
13. अ॒ग्नि स्तस्मा॒त् तस्मा॑ द॒ग्नि र॒ग्नि स्तस्मा᳚त् । \newline
14. तस्मा॑ दे॒व मे॒वम् तस्मा॒त् तस्मा॑ दे॒वम् । \newline
15. ए॒व मा॑हाहै॒व मे॒व मा॑ह । \newline
16. आ॒हाग्ने ऽग्न॑ आहा॒हाग्ने᳚ । \newline
17. अग्ने॒ चारु॒ श्चारु॒ रग्ने ऽग्ने॒ चारुः॑ । \newline
18. चारु॒र् विभृ॑तो॒ विभृ॑त॒ श्चारु॒ श्चारु॒र् विभृ॑तः । \newline
19. विभृ॑त॒ ओष॑धी॒ ष्वोष॑धीषु॒ विभृ॑तो॒ विभृ॑त॒ ओष॑धीषु । \newline
20. विभृ॑त॒ इति॒ वि - भृ॒तः॒ । \newline
21. ओष॑धी॒ ष्विती त्योष॑धी॒ ष्वोष॑धी॒ ष्विति॑ । \newline
22. इत्या॑हा॒हे तीत्या॑ह । \newline
23. आ॒ह॒ य॒दा य॒दा ऽऽहा॑ह य॒दा । \newline
24. य॒दा हि हि य॒दा य॒दा हि । \newline
25. ह्ये॑त मे॒तꣳ हि ह्ये॑तम् । \newline
26. ए॒तं ॅवि॒भर॑न्ति वि॒भर॑न्त्ये॒त मे॒तं ॅवि॒भर॑न्ति । \newline
27. वि॒भर॒ न्त्यथाथ॑ वि॒भर॑न्ति वि॒भर॒ न्त्यथ॑ । \newline
28. वि॒भर॒न्तीति॑ वि - भर॑न्ति । \newline
29. अथ॒ चारु॑तर॒ श्चारु॑त॒रो ऽथाथ॒ चारु॑तरः । \newline
30. चारु॑तरो॒ भव॑ति॒ भव॑ति॒ चारु॑तर॒ श्चारु॑तरो॒ भव॑ति । \newline
31. चारु॑तर॒ इति॒ चारु॑ - त॒रः॒ । \newline
32. भव॑ति॒ प्र प्र भव॑ति॒ भव॑ति॒ प्र । \newline
33. प्र मा॒तृभ्यो॑ मा॒तृभ्यः॒ प्र प्र मा॒तृभ्यः॑ । \newline
34. मा॒तृभ्यो॒ अध्यधि॑ मा॒तृभ्यो॑ मा॒तृभ्यो॒ अधि॑ । \newline
35. मा॒तृभ्य॒ इति॑ मा॒तृ - भ्यः॒ । \newline
36. अधि॒ कनि॑क्रद॒त् कनि॑क्र द॒दध्यधि॒ कनि॑क्रदत् । \newline
37. कनि॑क्रदद् गा गाः॒ कनि॑क्रद॒त् कनि॑क्रदद् गाः । \newline
38. गा॒ इतीति॑ गा गा॒ इति॑ । \newline
39. इत्या॑हा॒हे तीत्या॑ह । \newline
40. आ॒हौष॑धय॒ ओष॑धय आहा॒ हौष॑धयः । \newline
41. ओष॑धयो॒ वै वा ओष॑धय॒ ओष॑धयो॒ वै । \newline
42. वा अ॑स्यास्य॒ वै वा अ॑स्य । \newline
43. अ॒स्य॒ मा॒तरो॑ मा॒तरो᳚ ऽस्यास्य मा॒तरः॑ । \newline
44. मा॒तर॒ स्ताभ्य॒ स्ताभ्यो॑ मा॒तरो॑ मा॒तर॒ स्ताभ्यः॑ । \newline
45. ताभ्य॑ ए॒वैव ताभ्य॒ स्ताभ्य॑ ए॒व । \newline
46. ए॒वैन॑ मेन मे॒वैवैन᳚म् । \newline
47. ए॒न॒म् प्र प्रैन॑ मेन॒म् प्र । \newline
48. प्र च्या॑वयति च्यावयति॒ प्र प्र च्या॑वयति । \newline
49. च्या॒व॒य॒ति॒ स्थि॒रः स्थि॒र श्च्या॑वयति च्यावयति स्थि॒रः । \newline
50. स्थि॒रो भ॑व भव स्थि॒रः स्थि॒रो भ॑व । \newline
51. भ॒व॒ वी॒ड्व॑ङ्गो वी॒ड्व॑ङ्गो भव भव वी॒ड्व॑ङ्गः । \newline
52. वी॒ड्व॑ङ्ग॒ इतीति॑ वी॒ड्व॑ङ्गो वी॒ड्व॑ङ्ग॒ इति॑ । \newline
53. वी॒ड्व॑ङ्ग॒ इति॑ वी॒डु - अ॒ङ्गः॒ । \newline
54. इति॑ गर्द॒भे ग॑र्द॒भ इतीति॑ गर्द॒भे । \newline
55. ग॒र्द॒भ आ ग॑र्द॒भे ग॑र्द॒भ आ । \newline
56. आ सा॑दयति सादय॒त्या सा॑दयति । \newline
57. सा॒द॒य॒ति॒ सꣳ सꣳ सा॑दयति सादयति॒ सम् । \newline

\textbf{Ghana Paata } \newline

1. रोद॑स्यो॒ रितीति॒ रोद॑स्यो॒ रोद॑स्यो॒ रित्या॑ हा॒हेति॒ रोद॑स्यो॒ रोद॑स्यो॒ रित्या॑ह । \newline
2. इत्या॑हा॒हे तीत्या॑ हे॒मे इ॒मे आ॒हे तीत्या॑ हे॒मे । \newline
3. आ॒हे॒मे इ॒मे आ॑हा हे॒मे वै वा इ॒मे आ॑हा हे॒मे वै । \newline
4. इ॒मे वै वा इ॒मे इ॒मे वै रोद॑सी॒ रोद॑सी॒ वा इ॒मे इ॒मे वै रोद॑सी । \newline
5. इ॒मे इती॒मे । \newline
6. वै रोद॑सी॒ रोद॑सी॒ वै वै रोद॑सी॒ तयो॒ स्तयो॒ रोद॑सी॒ वै वै रोद॑सी॒ तयोः᳚ । \newline
7. रोद॑सी॒ तयो॒ स्तयो॒ रोद॑सी॒ रोद॑सी॒ तयो॑ रे॒ष ए॒ष तयो॒ रोद॑सी॒ रोद॑सी॒ तयो॑ रे॒षः । \newline
8. रोद॑सी॒ इति॒ रोद॑सी । \newline
9. तयो॑ रे॒ष ए॒ष तयो॒ स्तयो॑ रे॒ष गर्भो॒ गर्भ॑ ए॒ष तयो॒ स्तयो॑ रे॒ष गर्भः॑ । \newline
10. ए॒ष गर्भो॒ गर्भ॑ ए॒ष ए॒ष गर्भो॒ यद् यद् गर्भ॑ ए॒ष ए॒ष गर्भो॒ यत् । \newline
11. गर्भो॒ यद् यद् गर्भो॒ गर्भो॒ यद॒ग्नि र॒ग्निर् यद् गर्भो॒ गर्भो॒ यद॒ग्निः । \newline
12. यद॒ग्नि र॒ग्निर् यद् यद॒ग्नि स्तस्मा॒त् तस्मा॑ द॒ग्निर् यद् यद॒ग्नि स्तस्मा᳚त् । \newline
13. अ॒ग्नि स्तस्मा॒त् तस्मा॑ द॒ग्नि र॒ग्नि स्तस्मा॑ दे॒व मे॒वम् तस्मा॑ द॒ग्नि र॒ग्नि स्तस्मा॑ दे॒वम् । \newline
14. तस्मा॑ दे॒व मे॒वम् तस्मा॒त् तस्मा॑ दे॒व मा॑हा है॒वम् तस्मा॒त् तस्मा॑ दे॒व मा॑ह । \newline
15. ए॒व मा॑हा है॒व मे॒व मा॒हाग्ने ऽग्न॑ आहै॒व मे॒व मा॒हाग्ने᳚ । \newline
16. आ॒हाग्ने ऽग्न॑ आहा॒हाग्ने॒ चारु॒ श्चारु॒ रग्न॑ आहा॒हाग्ने॒ चारुः॑ । \newline
17. अग्ने॒ चारु॒ श्चारु॒ रग्ने ऽग्ने॒ चारु॒र् विभृ॑तो॒ विभृ॑त॒ श्चारु॒ रग्ने ऽग्ने॒ चारु॒र् विभृ॑तः । \newline
18. चारु॒र् विभृ॑तो॒ विभृ॑त॒ श्चारु॒ श्चारु॒र् विभृ॑त॒ ओष॑धी॒ ष्वोष॑धीषु॒ विभृ॑त॒ श्चारु॒ श्चारु॒र् विभृ॑त॒ ओष॑धीषु । \newline
19. विभृ॑त॒ ओष॑धी॒ ष्वोष॑धीषु॒ विभृ॑तो॒ विभृ॑त॒ ओष॑धी॒ ष्विती त्योष॑धीषु॒ विभृ॑तो॒ विभृ॑त॒ ओष॑धी॒ष्विति॑ । \newline
20. विभृ॑त॒ इति॒ वि - भृ॒तः॒ । \newline
21. ओष॑धी॒ ष्विती त्योष॑धी॒ ष्वोष॑धी॒ ष्वित्या॑हा॒हे त्योष॑धी॒ ष्वोष॑धी॒ ष्वित्या॑ह । \newline
22. इत्या॑हा॒हे तीत्या॑ह य॒दा य॒दा ऽऽहे तीत्या॑ह य॒दा । \newline
23. आ॒ह॒ य॒दा य॒दा ऽऽहा॑ह य॒दा हि हि य॒दा ऽऽहा॑ह य॒दा हि । \newline
24. य॒दा हि हि य॒दा य॒दा ह्ये॑त मे॒तꣳ हि य॒दा य॒दा ह्ये॑तम् । \newline
25. ह्ये॑त मे॒तꣳ हि ह्ये॑तं ॅवि॒भर॑न्ति वि॒भर॑ न्त्ये॒तꣳ हि ह्ये॑तं ॅवि॒भर॑न्ति । \newline
26. ए॒तं ॅवि॒भर॑न्ति वि॒भर॑ न्त्ये॒त मे॒तं ॅवि॒भर॒ न्त्यथाथ॑ वि॒भर॑ न्त्ये॒त मे॒तं ॅवि॒भर॒ न्त्यथ॑ । \newline
27. वि॒भर॒ न्त्यथाथ॑ वि॒भर॑न्ति वि॒भर॒ न्त्यथ॒ चारु॑तर॒ श्चारु॑त॒रो ऽथ॑ वि॒भर॑न्ति वि॒भर ॒न्त्यथ॒ चारु॑तरः । \newline
28. वि॒भर॒न्तीति॑ वि - भर॑न्ति । \newline
29. अथ॒ चारु॑तर॒ श्चारु॑त॒रो ऽथाथ॒ चारु॑तरो॒ भव॑ति॒ भव॑ति॒ चारु॑त॒रो ऽथाथ॒ चारु॑तरो॒ भव॑ति । \newline
30. चारु॑तरो॒ भव॑ति॒ भव॑ति॒ चारु॑तर॒ श्चारु॑तरो॒ भव॑ति॒ प्र प्र भव॑ति॒ चारु॑तर॒ श्चारु॑तरो॒ भव॑ति॒ प्र । \newline
31. चारु॑तर॒ इति॒ चारु॑ - त॒रः॒ । \newline
32. भव॑ति॒ प्र प्र भव॑ति॒ भव॑ति॒ प्र मा॒तृभ्यो॑ मा॒तृभ्यः॒ प्र भव॑ति॒ भव॑ति॒ प्र मा॒तृभ्यः॑ । \newline
33. प्र मा॒तृभ्यो॑ मा॒तृभ्यः॒ प्र प्र मा॒तृभ्यो॒ अध्यधि॑ मा॒तृभ्यः॒ प्र प्र मा॒तृभ्यो॒ अधि॑ । \newline
34. मा॒तृभ्यो॒ अध्यधि॑ मा॒तृभ्यो॑ मा॒तृभ्यो॒ अधि॒ कनि॑क्रद॒त् कनि॑क्र द॒दधि॑ मा॒तृभ्यो॑ मा॒तृभ्यो॒ अधि॒ कनि॑क्रदत् । \newline
35. मा॒तृभ्य॒ इति॑ मा॒तृ - भ्यः॒ । \newline
36. अधि॒ कनि॑क्रद॒त् कनि॑क्रद॒ दध्यधि॒ कनि॑क्रदद् गा गाः॒ कनि॑क्रद॒ दध्यधि॒ कनि॑क्रदद् गाः । \newline
37. कनि॑क्रदद् गा गाः॒ कनि॑क्रद॒त् कनि॑क्रदद् गा॒ इतीति॑ गाः॒ कनि॑क्रद॒त् कनि॑क्रदद् गा॒ इति॑ । \newline
38. गा॒ इतीति॑ गा गा॒ इत्या॑ हा॒हेति॑ गा गा॒ इत्या॑ह । \newline
39. इत्या॑ हा॒हेती त्या॒हौष॑धय॒ ओष॑धय आ॒हेती त्या॒हौष॑धयः । \newline
40. आ॒हौष॑धय॒ ओष॑धय आहा॒ हौष॑धयो॒ वै वा ओष॑धय आहा॒ हौष॑धयो॒ वै । \newline
41. ओष॑धयो॒ वै वा ओष॑धय॒ ओष॑धयो॒ वा अ॑स्यास्य॒ वा ओष॑धय॒ ओष॑धयो॒ वा अ॑स्य । \newline
42. वा अ॑स्यास्य॒ वै वा अ॑स्य मा॒तरो॑ मा॒तरो᳚ ऽस्य॒ वै वा अ॑स्य मा॒तरः॑ । \newline
43. अ॒स्य॒ मा॒तरो॑ मा॒तरो᳚ ऽस्यास्य मा॒तर॒ स्ताभ्य॒ स्ताभ्यो॑ मा॒तरो᳚ ऽस्यास्य मा॒तर॒ स्ताभ्यः॑ । \newline
44. मा॒तर॒ स्ताभ्य॒ स्ताभ्यो॑ मा॒तरो॑ मा॒तर॒ स्ताभ्य॑ ए॒वैव ताभ्यो॑ मा॒तरो॑ मा॒तर॒ स्ताभ्य॑ ए॒व । \newline
45. ताभ्य॑ ए॒वैव ताभ्य॒ स्ताभ्य॑ ए॒वैन॑ मेन मे॒व ताभ्य॒ स्ताभ्य॑ ए॒वैन᳚म् । \newline
46. ए॒वैन॑ मेन मे॒वैवैन॒म् प्र प्रैन॑ मे॒वैवैन॒म् प्र । \newline
47. ए॒न॒म् प्र प्रैन॑ मेन॒म् प्र च्या॑वयति च्यावयति॒ प्रैन॑ मेन॒म् प्र च्या॑वयति । \newline
48. प्र च्या॑वयति च्यावयति॒ प्र प्र च्या॑वयति स्थि॒रः स्थि॒र श्च्या॑वयति॒ प्र प्र च्या॑वयति स्थि॒रः । \newline
49. च्या॒व॒य॒ति॒ स्थि॒रः स्थि॒र श्च्या॑वयति च्यावयति स्थि॒रो भ॑व भव स्थि॒र श्च्या॑वयति च्यावयति स्थि॒रो भ॑व । \newline
50. स्थि॒रो भ॑व भव स्थि॒रः स्थि॒रो भ॑व वी॒ड्व॑ङ्गो वी॒ड्व॑ङ्गो भव स्थि॒रः स्थि॒रो भ॑व वी॒ड्व॑ङ्गः । \newline
51. भ॒व॒ वी॒ड्व॑ङ्गो वी॒ड्व॑ङ्गो भव भव वी॒ड्व॑ङ्ग॒ इतीति॑ वी॒ड्व॑ङ्गो भव भव वी॒ड्व॑ङ्ग॒ इति॑ । \newline
52. वी॒ड्व॑ङ्ग॒ इतीति॑ वी॒ड्व॑ङ्गो वी॒ड्व॑ङ्ग॒ इति॑ गर्द॒भे ग॑र्द॒भ इति॑ वी॒ड्व॑ङ्गो वी॒ड्व॑ङ्ग॒ इति॑ गर्द॒भे । \newline
53. वी॒ड्व॑ङ्ग॒ इति॑ वी॒डु - अ॒ङ्गः॒ । \newline
54. इति॑ गर्द॒भे ग॑र्द॒भ इतीति॑ गर्द॒भ आ ग॑र्द॒भ इतीति॑ गर्द॒भ आ । \newline
55. ग॒र्द॒भ आ ग॑र्द॒भे ग॑र्द॒भ आ सा॑दयति सादय॒त्या ग॑र्द॒भे ग॑र्द॒भ आ सा॑दयति । \newline
56. आ सा॑दयति सादय॒त्या सा॑दयति॒ सꣳ सꣳ सा॑दय॒त्या सा॑दयति॒ सम् । \newline
57. सा॒द॒य॒ति॒ सꣳ सꣳ सा॑दयति सादयति॒ सम् न॑ह्यति नह्यति॒ सꣳ सा॑दयति सादयति॒ सम् न॑ह्यति । \newline
\pagebreak
\markright{ TS 5.1.5.5  \hfill https://www.vedavms.in \hfill}

\section{ TS 5.1.5.5 }

\textbf{TS 5.1.5.5 } \newline
\textbf{Samhita Paata} \newline

सं न॑ह्यत्ये॒वैन॑मे॒तया᳚ स्थे॒म्ने ग॑र्द॒भेन॒ संभ॑रति॒ तस्मा᳚द् गर्द॒भः प॑शू॒नां भा॑रभा॒रित॑मो गर्द॒भेन॒ सं भ॑रति॒ तस्मा᳚द् गर्द॒भो-ऽप्य॑नाले॒शे-ऽत्य॒न्यान् प॒शून् मे᳚द्य॒त्यन्नꣳ॒॒ ह्ये॑नेना॒ऽर्कꣳ स॒भंर॑न्ति गर्द॒भेन॒ संभ॑रति॒ तस्मा᳚द् गर्द॒भो द्वि॒रेताः॒ सन् कनि॑ष्ठं पशू॒नां प्रजा॑यते॒ऽग्निर्.ह्य॑स्य॒ योनिं॑ नि॒र्दह॑ति प्र॒जासु॒ वा ए॒ष ए॒तर्.ह्यारू॑ढः॒ - [  ] \newline

\textbf{Pada Paata} \newline

समिति॑ । न॒ह्य॒ति॒ । ए॒व । ए॒न॒म् । ए॒तया᳚ । स्थे॒म्ने । ग॒र्द॒भेन॑ । समिति॑ । भ॒र॒ति॒ । तस्मा᳚त् । ग॒र्द॒भः । प॒शू॒नाम् । भा॒र॒भा॒रित॑म॒ इति॑ भारभा॒रि - त॒मः॒ । ग॒र्द॒भेन॑ । समिति॑ । भ॒र॒ति॒ । तस्मा᳚त् । ग॒र्द॒भः । अपीति॑ । अ॒ना॒ले॒श इत्य॑ना - ले॒शे । अतीति॑ । अ॒न्यान् । प॒शून् । मे॒द्य॒ति॒ । अन्न᳚म् । हि । ए॒ने॒न॒ । अ॒र्कम् । स॒भंर॒न्तीति॑ सं - भर॑न्ति । ग॒र्द॒भेन॑ । समिति॑ । भ॒र॒ति॒ । तस्मा᳚त् । ग॒र्द॒भः । द्वि॒रेता॒ इति॑ द्वि-रेताः᳚ । सन्न् । कनि॑ष्ठम् । प॒शू॒नाम् । प्रेति॑ । जा॒य॒ते॒ । अ॒ग्निः । हि । अ॒स्य॒ । योनि᳚म् । नि॒र्दह॒तीति॑ निः - दह॑ति । प्र॒जास्विति॑ प्र - जासु॑ । वै । ए॒षः । ए॒तर्.हि॑ । आरू॑ढ॒ इत्या - रू॒ढः॒ ।  \newline


\textbf{Krama Paata} \newline

सम् न॑ह्यति । न॒ह्य॒त्ये॒व । ए॒वैन᳚म् । ए॒न॒मे॒तया᳚ । ए॒तया᳚ स्थे॒म्ने । स्थे॒म्ने ग॑र्द॒भेन॑ । ग॒र्द॒भेन॒ सम् । सम् भ॑रति । भ॒र॒ति॒ तस्मा᳚त् । तस्मा᳚द् गर्द॒भः । ग॒र्द॒भः प॑शू॒नाम् । प॒शू॒नाम् भा॑रभा॒रित॑मः । भा॒र॒भा॒रित॑मो गर्द॒भेन॑ । भा॒र॒भा॒रित॑म॒ इति॑ भारभा॒रि - त॒मः॒ । ग॒र्द॒भेन॒ सम् । सम् भ॑रति । भ॒र॒ति॒ तस्मा᳚त् । तस्मा᳚द् गर्द॒भः । ग॒र्द॒भोऽपि॑ । अप्य॑नाले॒शे । अ॒ना॒ले॒शेऽति॑ । अ॒ना॒ले॒श इत्य॑ना - ले॒शे । अत्य॒न्यान् । अ॒न्यान् प॒शून् । प॒शून् मे᳚द्यति । मे॒द्य॒त्यन्न᳚म् । अन्नꣳ॒॒ हि । ह्ये॑नेन । ए॒ने॒ना॒र्कम् । अ॒र्कꣳ स॒म्भर॑न्ति । स॒म्भर॑न्ति गर्द॒भेन॑ । स॒म्भर॒न्तीति॑ सम् - भर॑न्ति । ग॒र्द॒भेन॒ सम् । सम् भ॑रति । भ॒र॒ति॒ तस्मा᳚त् । तस्मा᳚द् गर्द॒भः । ग॒र्द॒भो द्वि॒रेताः᳚ । द्वि॒रेताः॒ सन्न् । द्वि॒रेता॒ इति॑ द्वि - रेताः᳚ । सन् कनि॑ष्ठम् । कनि॑ष्ठम् पशू॒नाम् । प॒शू॒नाम् प्र । प्र जा॑यते । जा॒य॒ते॒ऽग्निः । अ॒ग्निर् हि । ह्य॑स्य । अ॒स्य॒ योनि᳚म् । योनि॑म् नि॒र्दह॑ति । नि॒र्दह॑ति प्र॒जासु॑ । नि॒र्दह॒तीति॑ निः - दह॑ति । प्र॒जासु॒ वै । प्र॒जास्विति॑ प्र - जासु॑ । वा ए॒षः । ए॒ष ए॒तर्.हि॑ । ए॒तर्.ह्यारू॑ढः । आरू॑ढः॒ सः । आरू॑ढ॒ इत्या - रू॒ढः॒ \newline

\textbf{Jatai Paata} \newline

1. सम् न॑ह्यति नह्यति॒ सꣳ सम् न॑ह्यति । \newline
2. न॒ह्य॒ त्ये॒वैव न॑ह्यति नह्य त्ये॒व । \newline
3. ए॒वैन॑ मेन मे॒वैवैन᳚म् । \newline
4. ए॒न॒ मे॒त यै॒त यै॑न मेन मे॒तया᳚ । \newline
5. ए॒तया᳚ स्थे॒म्ने स्थे॒म्न ए॒त यै॒तया᳚ स्थे॒म्ने । \newline
6. स्थे॒म्ने ग॑र्द॒भेन॑ गर्द॒भेन॑ स्थे॒म्ने स्थे॒म्ने ग॑र्द॒भेन॑ । \newline
7. ग॒र्द॒भेन॒ सꣳ सम् ग॑र्द॒भेन॑ गर्द॒भेन॒ सम् । \newline
8. सम् भ॑रति भरति॒ सꣳ सम् भ॑रति । \newline
9. भ॒र॒ति॒ तस्मा॒त् तस्मा᳚द् भरति भरति॒ तस्मा᳚त् । \newline
10. तस्मा᳚द् गर्द॒भो ग॑र्द॒भ स्तस्मा॒त् तस्मा᳚द् गर्द॒भः । \newline
11. ग॒र्द॒भः प॑शू॒नाम् प॑शू॒नाम् ग॑र्द॒भो ग॑र्द॒भः प॑शू॒नाम् । \newline
12. प॒शू॒नाम् भा॑रभा॒रित॑मो भारभा॒रित॑मः पशू॒नाम् प॑शू॒नाम् भा॑रभा॒रित॑मः । \newline
13. भा॒र॒भा॒रित॑मो गर्द॒भेन॑ गर्द॒भेन॑ भारभा॒रित॑मो भारभा॒रित॑मो गर्द॒भेन॑ । \newline
14. भा॒र॒भा॒रित॑म॒ इति॑ भारभा॒रि - त॒मः॒ । \newline
15. ग॒र्द॒भेन॒ सꣳ सम् ग॑र्द॒भेन॑ गर्द॒भेन॒ सम् । \newline
16. सम् भ॑रति भरति॒ सꣳ सम् भ॑रति । \newline
17. भ॒र॒ति॒ तस्मा॒त् तस्मा᳚द् भरति भरति॒ तस्मा᳚त् । \newline
18. तस्मा᳚द् गर्द॒भो ग॑र्द॒भ स्तस्मा॒त् तस्मा᳚द् गर्द॒भः । \newline
19. ग॒र्द॒भो ऽप्यपि॑ गर्द॒भो ग॑र्द॒भो ऽपि॑ । \newline
20. अप्य॑नाले॒शे॑ ऽनाले॒शे ऽप्य प्य॑नाले॒शे । \newline
21. अ॒ना॒ले॒शे ऽत्य त्य॑नाले॒शे॑ ऽनाले॒शे ऽति॑ । \newline
22. अ॒ना॒ले॒श इत्य॑ना - ले॒शे । \newline
23. अत्य॒न्या न॒न्या नत्य त्य॒न्यान् । \newline
24. अ॒न्यान् प॒शून् प॒शू न॒न्या न॒न्यान् प॒शून् । \newline
25. प॒शून् मे᳚द्यति मेद्यति प॒शून् प॒शून् मे᳚द्यति । \newline
26. मे॒द्य॒ त्यन्न॒ मन्न॑म् मेद्यति मेद्य॒ त्यन्न᳚म् । \newline
27. अन्नꣳ॒॒ हि ह्यन्न॒ मन्नꣳ॒॒ हि । \newline
28. ह्ये॑ने नैनेन॒ हि ह्ये॑नेन । \newline
29. ए॒ने॒ना॒र्क म॒र्क मे॑ने नैनेना॒र्कम् । \newline
30. अ॒र्कꣳ सं॒भर॑न्ति सं॒भर॑ न्त्य॒र्क म॒र्कꣳ सं॒भर॑न्ति । \newline
31. सं॒भर॑न्ति गर्द॒भेन॑ गर्द॒भेन॑ सं॒भर॑न्ति सं॒भर॑न्ति गर्द॒भेन॑ । \newline
32. सं॒भर॒न्तीति॑ सं - भर॑न्ति । \newline
33. ग॒र्द॒भेन॒ सꣳ सम् ग॑र्द॒भेन॑ गर्द॒भेन॒ सम् । \newline
34. सम् भ॑रति भरति॒ सꣳ सम् भ॑रति । \newline
35. भ॒र॒ति॒ तस्मा॒त् तस्मा᳚द् भरति भरति॒ तस्मा᳚त् । \newline
36. तस्मा᳚द् गर्द॒भो ग॑र्द॒भ स्तस्मा॒त् तस्मा᳚द् गर्द॒भः । \newline
37. ग॒र्द॒भो द्वि॒रेता᳚ द्वि॒रेता॑ गर्द॒भो ग॑र्द॒भो द्वि॒रेताः᳚ । \newline
38. द्वि॒रेताः॒ सन् थ्सन् द्वि॒रेता᳚ द्वि॒रेताः॒ सन्न् । \newline
39. द्वि॒रेता॒ इति॑ द्वि - रेताः᳚ । \newline
40. सन् कनि॑ष्ठ॒म् कनि॑ष्ठꣳ॒॒ सन् थ्सन् कनि॑ष्ठम् । \newline
41. कनि॑ष्ठम् पशू॒नाम् प॑शू॒नाम् कनि॑ष्ठ॒म् कनि॑ष्ठम् पशू॒नाम् । \newline
42. प॒शू॒नाम् प्र प्र प॑शू॒नाम् प॑शू॒नाम् प्र । \newline
43. प्र जा॑यते जायते॒ प्र प्र जा॑यते । \newline
44. जा॒य॒ते॒ ऽग्नि र॒ग्निर् जा॑यते जायते॒ ऽग्निः । \newline
45. अ॒ग्निर्. हि ह्य॑ग्नि र॒ग्निर्. हि । \newline
46. ह्य॑स्यास्य॒ हि ह्य॑स्य । \newline
47. अ॒स्य॒ योनिं॒ ॅयोनि॑ मस्यास्य॒ योनि᳚म् । \newline
48. योनि॑म् नि॒र्दह॑ति नि॒र्दह॑ति॒ योनिं॒ ॅयोनि॑म् नि॒र्दह॑ति । \newline
49. नि॒र्दह॑ति प्र॒जासु॑ प्र॒जासु॑ नि॒र्दह॑ति नि॒र्दह॑ति प्र॒जासु॑ । \newline
50. नि॒र्दह॒तीति॑ निः - दह॑ति । \newline
51. प्र॒जासु॒ वै वै प्र॒जासु॑ प्र॒जासु॒ वै । \newline
52. प्र॒जास्विति॑ प्र - जासु॑ । \newline
53. वा ए॒ष ए॒ष वै वा ए॒षः । \newline
54. ए॒ष ए॒तर् ह्ये॒तर् ह्ये॒ष ए॒ष ए॒तर्.हि॑ । \newline
55. ए॒तर् ह्यारू॑ढ॒ आरू॑ढ ए॒तर् ह्ये॒तर् ह्यारू॑ढः । \newline
56. आरू॑ढः॒ स स आरू॑ढ॒ आरू॑ढः॒ सः । \newline
57. आरू॑ढ॒ इत्या - रू॒ढः॒ । \newline

\textbf{Ghana Paata } \newline

1. सम् न॑ह्यति नह्यति॒ सꣳ सम् न॑ह्य त्ये॒वैव न॑ह्यति॒ सꣳ सम् न॑ह्यत्ये॒व । \newline
2. न॒ह्य॒ त्ये॒वैव न॑ह्यति नह्य त्ये॒वैन॑ मेन मे॒व न॑ह्यति नह्य त्ये॒वैन᳚म् । \newline
3. ए॒वैन॑ मेन मे॒वैवैन॑ मे॒त यै॒त यै॑न मे॒वैवैन॑ मे॒तया᳚ । \newline
4. ए॒न॒ मे॒त यै॒त यै॑न मेन मे॒तया᳚ स्थे॒म्ने स्थे॒म्न ए॒तयै॑न मेन मे॒तया᳚ स्थे॒म्ने । \newline
5. ए॒तया᳚ स्थे॒म्ने स्थे॒म्न ए॒त यै॒तया᳚ स्थे॒म्ने ग॑र्द॒भेन॑ गर्द॒भेन॑ स्थे॒म्न ए॒त यै॒तया᳚ स्थे॒म्ने ग॑र्द॒भेन॑ । \newline
6. स्थे॒म्ने ग॑र्द॒भेन॑ गर्द॒भेन॑ स्थे॒म्ने स्थे॒म्ने ग॑र्द॒भेन॒ सꣳ सम् ग॑र्द॒भेन॑ स्थे॒म्ने स्थे॒म्ने ग॑र्द॒भेन॒ सम् । \newline
7. ग॒र्द॒भेन॒ सꣳ सम् ग॑र्द॒भेन॑ गर्द॒भेन॒ सम् भ॑रति भरति॒ सम् ग॑र्द॒भेन॑ गर्द॒भेन॒ सम् भ॑रति । \newline
8. सम् भ॑रति भरति॒ सꣳ सम् भ॑रति॒ तस्मा॒त् तस्मा᳚द् भरति॒ सꣳ सम् भ॑रति॒ तस्मा᳚त् । \newline
9. भ॒र॒ति॒ तस्मा॒त् तस्मा᳚द् भरति भरति॒ तस्मा᳚द् गर्द॒भो ग॑र्द॒भ स्तस्मा᳚द् भरति भरति॒ तस्मा᳚द् गर्द॒भः । \newline
10. तस्मा᳚द् गर्द॒भो ग॑र्द॒भ स्तस्मा॒त् तस्मा᳚द् गर्द॒भः प॑शू॒नाम् प॑शू॒नाम् ग॑र्द॒भ स्तस्मा॒त् तस्मा᳚द् गर्द॒भः प॑शू॒नाम् । \newline
11. ग॒र्द॒भः प॑शू॒नाम् प॑शू॒नाम् ग॑र्द॒भो ग॑र्द॒भः प॑शू॒नाम् भा॑रभा॒रित॑मो भारभा॒रित॑मः पशू॒नाम् ग॑र्द॒भो ग॑र्द॒भः प॑शू॒नाम् भा॑रभा॒रित॑मः । \newline
12. प॒शू॒नाम् भा॑रभा॒रित॑मो भारभा॒रित॑मः पशू॒नाम् प॑शू॒नाम् भा॑रभा॒रित॑मो गर्द॒भेन॑ गर्द॒भेन॑ भारभा॒रित॑मः पशू॒नाम् प॑शू॒नाम् भा॑रभा॒रित॑मो गर्द॒भेन॑ । \newline
13. भा॒र॒भा॒रित॑मो गर्द॒भेन॑ गर्द॒भेन॑ भारभा॒रित॑मो भारभा॒रित॑मो गर्द॒भेन॒ सꣳ सम् ग॑र्द॒भेन॑ भारभा॒रित॑मो भारभा॒रित॑मो गर्द॒भेन॒ सम् । \newline
14. भा॒र॒भा॒रित॑म॒ इति॑ भारभा॒रि - त॒मः॒ । \newline
15. ग॒र्द॒भेन॒ सꣳ सम् ग॑र्द॒भेन॑ गर्द॒भेन॒ सम् भ॑रति भरति॒ सम् ग॑र्द॒भेन॑ गर्द॒भेन॒ सम् भ॑रति । \newline
16. सम् भ॑रति भरति॒ सꣳ सम् भ॑रति॒ तस्मा॒त् तस्मा᳚द् भरति॒ सꣳ सम् भ॑रति॒ तस्मा᳚त् । \newline
17. भ॒र॒ति॒ तस्मा॒त् तस्मा᳚द् भरति भरति॒ तस्मा᳚द् गर्द॒भो ग॑र्द॒भ स्तस्मा᳚द् भरति भरति॒ तस्मा᳚द् गर्द॒भः । \newline
18. तस्मा᳚द् गर्द॒भो ग॑र्द॒भ स्तस्मा॒त् तस्मा᳚द् गर्द॒भो ऽप्यपि॑ गर्द॒भ स्तस्मा॒त् तस्मा᳚द् गर्द॒भो ऽपि॑ । \newline
19. ग॒र्द॒भो ऽप्यपि॑ गर्द॒भो ग॑र्द॒भो ऽप्य॑नाले॒शे॑ ऽनाले॒शे ऽपि॑ गर्द॒भो ग॑र्द॒भो ऽप्य॑नाले॒शे । \newline
20. अप्य॑नाले॒शे॑ ऽनाले॒शे ऽप्य प्य॑नाले॒शे ऽत्य त्य॑नाले॒शे ऽप्य प्य॑नाले॒शे ऽति॑ । \newline
21. अ॒ना॒ले॒शे ऽत्य त्य॑नाले॒शे॑ ऽनाले॒शे ऽत्य॒ न्या न॒न्या नत्य॑नाले॒शे॑ ऽनाले॒शे ऽत्य॒न्यान् । \newline
22. अ॒ना॒ले॒श इत्य॑ना - ले॒शे । \newline
23. अत्य॒न्या न॒न्या नत्य त्य॒न्यान् प॒शून् प॒शू न॒न्या नत्य त्य॒न्यान् प॒शून् । \newline
24. अ॒न्यान् प॒शून् प॒शू न॒न्या न॒न्यान् प॒शून् मे᳚द्यति मेद्यति प॒शू न॒न्या न॒न्यान् प॒शून् मे᳚द्यति । \newline
25. प॒शून् मे᳚द्यति मेद्यति प॒शून् प॒शून् मे᳚द्य॒ त्यन्न॒ मन्न॑म् मेद्यति प॒शून् प॒शून् मे᳚द्य॒ त्यन्न᳚म् । \newline
26. मे॒द्य॒त्यन्न॒ मन्न॑म् मेद्यति मेद्य॒ त्यन्नꣳ॒॒ हि ह्यन्न॑म् मेद्यति मेद्य॒ त्यन्नꣳ॒॒ हि । \newline
27. अन्नꣳ॒॒ हि ह्यन्न॒ मन्नꣳ॒॒ ह्ये॑ने नैनेन॒ ह्यन्न॒ मन्नꣳ॒॒ ह्ये॑नेन । \newline
28. ह्ये॑ने नैनेन॒ हि ह्ये॑नेना॒र्क म॒र्क मे॑नेन॒ हि ह्ये॑नेना॒र्कम् । \newline
29. ए॒ने॒ना॒र्क म॒र्क मे॑नेनैनेना॒ र्कꣳ सं॒भर॑न्ति सं॒भर॑ न्त्य॒र्क मे॑नेनैनेना॒ र्कꣳ सं॒भर॑न्ति । \newline
30. अ॒र्कꣳ सं॒भर॑न्ति सं॒भर॑ न्त्य॒र्क म॒र्कꣳ सं॒भर॑न्ति गर्द॒भेन॑ गर्द॒भेन॑ सं॒भर॑ न्त्य॒र्क म॒र्कꣳ सं॒भर॑न्ति गर्द॒भेन॑ । \newline
31. सं॒भर॑न्ति गर्द॒भेन॑ गर्द॒भेन॑ सं॒भर॑न्ति सं॒भर॑न्ति गर्द॒भेन॒ सꣳ सम् ग॑र्द॒भेन॑ सं॒भर॑न्ति सं॒भर॑न्ति गर्द॒भेन॒ सम् । \newline
32. सं॒भर॒न्तीति॑ सं - भर॑न्ति । \newline
33. ग॒र्द॒भेन॒ सꣳ सम् ग॑र्द॒भेन॑ गर्द॒भेन॒ सम् भ॑रति भरति॒ सम् ग॑र्द॒भेन॑ गर्द॒भेन॒ सम् भ॑रति । \newline
34. सम् भ॑रति भरति॒ सꣳ सम् भ॑रति॒ तस्मा॒त् तस्मा᳚द् भरति॒ सꣳ सम् भ॑रति॒ तस्मा᳚त् । \newline
35. भ॒र॒ति॒ तस्मा॒त् तस्मा᳚द् भरति भरति॒ तस्मा᳚द् गर्द॒भो ग॑र्द॒भ स्तस्मा᳚द् भरति भरति॒ तस्मा᳚द् गर्द॒भः । \newline
36. तस्मा᳚द् गर्द॒भो ग॑र्द॒भ स्तस्मा॒त् तस्मा᳚द् गर्द॒भो द्वि॒रेता᳚ द्वि॒रेता॑ गर्द॒भ स्तस्मा॒त् तस्मा᳚द् गर्द॒भो द्वि॒रेताः᳚ । \newline
37. ग॒र्द॒भो द्वि॒रेता᳚ द्वि॒रेता॑ गर्द॒भो ग॑र्द॒भो द्वि॒रेताः॒ सन् थ्सन् द्वि॒रेता॑ गर्द॒भो ग॑र्द॒भो द्वि॒रेताः॒ सन्न् । \newline
38. द्वि॒रेताः॒ सन् थ्सन् द्वि॒रेता᳚ द्वि॒रेताः॒ सन् कनि॑ष्ठ॒म् कनि॑ष्ठꣳ॒॒ सन् द्वि॒रेता᳚ द्वि॒रेताः॒ सन् कनि॑ष्ठम् । \newline
39. द्वि॒रेता॒ इति॑ द्वि - रेताः᳚ । \newline
40. सन् कनि॑ष्ठ॒म् कनि॑ष्ठꣳ॒॒ सन् थ्सन् कनि॑ष्ठम् पशू॒नाम् प॑शू॒नाम् कनि॑ष्ठꣳ॒॒ सन् थ्सन् कनि॑ष्ठम् पशू॒नाम् । \newline
41. कनि॑ष्ठम् पशू॒नाम् प॑शू॒नाम् कनि॑ष्ठ॒म् कनि॑ष्ठम् पशू॒नाम् प्र प्र प॑शू॒नाम् कनि॑ष्ठ॒म् कनि॑ष्ठम् पशू॒नाम् प्र । \newline
42. प॒शू॒नाम् प्र प्र प॑शू॒नाम् प॑शू॒नाम् प्र जा॑यते जायते॒ प्र प॑शू॒नाम् प॑शू॒नाम् प्र जा॑यते । \newline
43. प्र जा॑यते जायते॒ प्र प्र जा॑यते॒ ऽग्नि र॒ग्निर् जा॑यते॒ प्र प्र जा॑यते॒ ऽग्निः । \newline
44. जा॒य॒ते॒ ऽग्नि र॒ग्निर् जा॑यते जायते॒ ऽग्निर्. हि ह्य॑ग्निर् जा॑यते जायते॒ ऽग्निर्. हि । \newline
45. अ॒ग्निर्. हि ह्य॑ग्नि र॒ग्निर् ह्य॑स्यास्य॒ ह्य॑ग्नि र॒ग्निर् ह्य॑स्य । \newline
46. ह्य॑स्यास्य॒ हि ह्य॑स्य॒ योनिं॒ ॅयोनि॑ मस्य॒ हि ह्य॑स्य॒ योनि᳚म् । \newline
47. अ॒स्य॒ योनिं॒ ॅयोनि॑ मस्यास्य॒ योनि॑म् नि॒र्दह॑ति नि॒र्दह॑ति॒ योनि॑ मस्यास्य॒ योनि॑म् नि॒र्दह॑ति । \newline
48. योनि॑म् नि॒र्दह॑ति नि॒र्दह॑ति॒ योनिं॒ ॅयोनि॑म् नि॒र्दह॑ति प्र॒जासु॑ प्र॒जासु॑ नि॒र्दह॑ति॒ योनिं॒ ॅयोनि॑म् नि॒र्दह॑ति प्र॒जासु॑ । \newline
49. नि॒र्दह॑ति प्र॒जासु॑ प्र॒जासु॑ नि॒र्दह॑ति नि॒र्दह॑ति प्र॒जासु॒ वै वै प्र॒जासु॑ नि॒र्दह॑ति नि॒र्दह॑ति प्र॒जासु॒ वै । \newline
50. नि॒र्दह॒तीति॑ निः - दह॑ति । \newline
51. प्र॒जासु॒ वै वै प्र॒जासु॑ प्र॒जासु॒ वा ए॒ष ए॒ष वै प्र॒जासु॑ प्र॒जासु॒ वा ए॒षः । \newline
52. प्र॒जास्विति॑ प्र - जासु॑ । \newline
53. वा ए॒ष ए॒ष वै वा ए॒ष ए॒तर्. ह्ये॒तर्. ह्ये॒ष वै वा ए॒ष ए॒तर्.हि॑ । \newline
54. ए॒ष ए॒तर्. ह्ये॒तर्. ह्ये॒ष ए॒ष ए॒तर्. ह्यारू॑ढ॒ आरू॑ढ ए॒तर्. ह्ये॒ष ए॒ष ए॒तर्. ह्यारू॑ढः । \newline
55. ए॒तर्. ह्यारू॑ढ॒ आरू॑ढ ए॒तर्. ह्ये॒तर्. ह्यारू॑ढः॒ स स आरू॑ढ ए॒तर्. ह्ये॒तर्. ह्यारू॑ढः॒ सः । \newline
56. आरू॑ढः॒ स स आरू॑ढ॒ आरू॑ढः॒ स ई᳚श्व॒र ई᳚श्व॒रः स आरू॑ढ॒ आरू॑ढः॒ स ई᳚श्व॒रः । \newline
57. आरू॑ढ॒ इत्या - रू॒ढः॒ । \newline
\pagebreak
\markright{ TS 5.1.5.6  \hfill https://www.vedavms.in \hfill}

\section{ TS 5.1.5.6 }

\textbf{TS 5.1.5.6 } \newline
\textbf{Samhita Paata} \newline

स ई᳚श्व॒रः प्र॒जाः शु॒चा प्र॒दहः॑ शि॒वो भ॑व प्र॒जाभ्य॒ इत्या॑ह प्र॒जाभ्य॑ ए॒वैनꣳ॑ शमयति॒ मानु॑षीभ्य॒स्त्वम॑ङ्गिर॒ इत्या॑ह मान॒व्यो॑ हि प्र॒जा मा द्यावा॑पृथि॒वी अ॒भि शू॑शुचो॒ माऽन्तरि॑क्षं॒ मा वन॒स्पती॒नित्या॑है॒भ्य ए॒वैनं॑ ॅलो॒केभ्यः॑ शमयति॒ प्रैतु॑ वा॒जी कनि॑क्रद॒दित्या॑ह वा॒जी ह्ये॑ष नान॑द॒द्-रास॑भः॒ पत्वेत्या॑ - [  ] \newline

\textbf{Pada Paata} \newline

सः । ई॒श्व॒रः । प्र॒जा इति॑ प्र - जाः । शु॒चा । प्र॒दह॒ इति॑ प्र-दहः॑ । शि॒वः । भ॒व॒ । प्र॒जाभ्य॒ इति॑ प्र - जाभ्यः॑ । इति॑ । आ॒ह॒ । प्र॒जाभ्य॒ इति॑ प्र - जाभ्यः॑ । ए॒व । ए॒न॒म् । श॒म॒य॒ति॒ । मानु॑षीभ्यः । त्वम् । अ॒ङ्गि॒रः॒ । इति॑ । आ॒ह॒ । मा॒न॒व्यः॑ । हि । प्र॒जा इति॑ प्र - जाः । मा । द्यावा॑पृथि॒वी इति॒ द्यावा᳚ - पृ॒थि॒वी । अ॒भीति॑ । शू॒शु॒चः॒ । मा । अ॒न्तरि॑क्षम् । मा । वन॒स्पतीन्॑ । इति॑ । आ॒ह॒ । ए॒भ्यः । ए॒व । ए॒न॒म् । लो॒केभ्यः॑ । श॒म॒य॒ति॒ । प्रेति॑ । ए॒तु॒ । वा॒जी । कनि॑क्रदत् । इति॑ । आ॒ह॒ । वा॒जी । हि । ए॒षः । नान॑दत् । रास॑भः । पत्वा᳚ । इति॑ ।  \newline


\textbf{Krama Paata} \newline

स ई᳚श्व॒रः । ई॒श्व॒रः प्र॒जाः । प्र॒जाः शु॒चा । प्र॒जा इति॑ प्र - जाः । शु॒चा प्र॒दहः॑ । प्र॒दहः॑ शि॒वः । प्र॒दह॒ इति॑ प्र - दहः॑ । शि॒वो भ॑व । भ॒व॒ प्र॒जाभ्यः॑ । प्र॒जाभ्य॒ इति॑ । प्र॒जाभ्य॒ इति॑ प्र - जाभ्यः॑ । इत्या॑ह । आ॒ह॒ प्र॒जाभ्यः॑ । प्र॒जाभ्य॑ ए॒व । प्र॒जाभ्य॒ इति॑ प्र - जाभ्यः॑ । ए॒वैन᳚म् । ए॒नꣳ॒॒ श॒म॒य॒ति॒ । श॒म॒य॒ति॒ मानु॑षीभ्यः । मानु॑षीभ्य॒स्त्वम् । त्वम॑ङ्गिरः । अ॒ङ्गि॒र॒ इति॑ । इत्या॑ह । आ॒ह॒ मा॒न॒व्यः॑ । मा॒न॒व्यो॑ हि । हि प्र॒जाः । प्र॒जा मा । प्र॒जा इति॑ प्र - जाः । मा द्यावा॑पृथि॒वी । द्यावा॑पृथि॒वी अ॒भि । द्यावा॑पृथि॒वी इति॒ द्यावा᳚ - पृ॒थि॒वी । अ॒भि शू॑शुचः । शू॒शु॒चो॒ मा । माऽन्तरि॑क्षम् । अ॒न्तरि॑क्ष॒म् मा । मा वन॒स्पतीन्॑ । वन॒स्पती॒निति॑ । इत्या॑ह । आ॒है॒भ्यः । ए॒भ्य ए॒व । ए॒वैन᳚म् । ए॒न॒म् ॅलो॒केभ्यः॑ । लो॒केभ्यः॑ शमयति । श॒म॒य॒ति॒ प्र । प्रैतु॑ । ए॒तु॒ वा॒जी । वा॒जी कनि॑क्रदत् । कनि॑क्रद॒दिति॑ । इत्या॑ह । आ॒ह॒ वा॒जी । वा॒जी हि । ह्ये॑षः । ए॒ष नान॑दत् । नान॑द॒द् रास॑भः । रास॑भः॒ पत्वा᳚ । पत्वेति॑ । इत्या॑ह \newline

\textbf{Jatai Paata} \newline

1. स ई᳚श्व॒र ई᳚श्व॒रः स स ई᳚श्व॒रः । \newline
2. ई॒श्व॒रः प्र॒जाः प्र॒जा ई᳚श्व॒र ई᳚श्व॒रः प्र॒जाः । \newline
3. प्र॒जाः शु॒चा शु॒चा प्र॒जाः प्र॒जाः शु॒चा । \newline
4. प्र॒जा इति॑ प्र - जाः । \newline
5. शु॒चा प्र॒दहः॑ प्र॒दहः॑ शु॒चा शु॒चा प्र॒दहः॑ । \newline
6. प्र॒दहः॑ शि॒वः शि॒वः प्र॒दहः॑ प्र॒दहः॑ शि॒वः । \newline
7. प्र॒दह॒ इति॑ प्र - दहः॑ । \newline
8. शि॒वो भ॑व भव शि॒वः शि॒वो भ॑व । \newline
9. भ॒व॒ प्र॒जाभ्यः॑ प्र॒जाभ्यो॑ भव भव प्र॒जाभ्यः॑ । \newline
10. प्र॒जाभ्य॒ इतीति॑ प्र॒जाभ्यः॑ प्र॒जाभ्य॒ इति॑ । \newline
11. प्र॒जाभ्य॒ इति॑ प्र - जाभ्यः॑ । \newline
12. इत्या॑हा॒हे तीत्या॑ह । \newline
13. आ॒ह॒ प्र॒जाभ्यः॑ प्र॒जाभ्य॑ आहाह प्र॒जाभ्यः॑ । \newline
14. प्र॒जाभ्य॑ ए॒वैव प्र॒जाभ्यः॑ प्र॒जाभ्य॑ ए॒व । \newline
15. प्र॒जाभ्य॒ इति॑ प्र - जाभ्यः॑ । \newline
16. ए॒वैन॑ मेन मे॒वैवैन᳚म् । \newline
17. ए॒नꣳ॒॒ श॒म॒य॒ति॒ श॒म॒य॒ त्ये॒न॒ मे॒नꣳ॒॒ श॒म॒य॒ति॒ । \newline
18. श॒म॒य॒ति॒ मानु॑षीभ्यो॒ मानु॑षीभ्यः शमयति शमयति॒ मानु॑षीभ्यः । \newline
19. मानु॑षीभ्य॒ स्त्वम् त्वम् मानु॑षीभ्यो॒ मानु॑षीभ्य॒ स्त्वम् । \newline
20. त्व म॑ङ्गिरो अङ्गि र॒स्त्वम् त्व म॑ङ्गिरः । \newline
21. अ॒ङ्गि॒र॒ इती त्य॑ङ्गिरो ऽङ्गिर॒ इति॑ । \newline
22. इत्या॑हा॒हे तीत्या॑ह । \newline
23. आ॒ह॒ मा॒न॒व्यो॑ मान॒व्य॑ आहाह मान॒व्यः॑ । \newline
24. मा॒न॒व्यो॑ हि हि मा॑न॒व्यो॑ मान॒व्यो॑ हि । \newline
25. हि प्र॒जाः प्र॒जा हि हि प्र॒जाः । \newline
26. प्र॒जा मा मा प्र॒जाः प्र॒जा मा । \newline
27. प्र॒जा इति॑ प्र - जाः । \newline
28. मा द्यावा॑पृथि॒वी द्यावा॑पृथि॒वी मा मा द्यावा॑पृथि॒वी । \newline
29. द्यावा॑पृथि॒वी अ॒भ्य॑भि द्यावा॑पृथि॒वी द्यावा॑पृथि॒वी अ॒भि । \newline
30. द्यावा॑पृथि॒वी इति॒ द्यावा᳚ - पृ॒थि॒वी । \newline
31. अ॒भि शू॑शुचः शूशुचो अ॒भ्य॑भि शू॑शुचः । \newline
32. शू॒शु॒चो॒ मा मा शू॑शुचः शूशुचो॒ मा । \newline
33. मा ऽन्तरि॑क्ष म॒न्तरि॑क्ष॒म् मा मा ऽन्तरि॑क्षम् । \newline
34. अ॒न्तरि॑क्ष॒म् मा मा ऽन्तरि॑क्ष म॒न्तरि॑क्ष॒म् मा । \newline
35. मा वन॒स्पती॒न्॒. वन॒स्पती॒न् मा मा वन॒स्पतीन्॑ । \newline
36. वन॒स्पती॒ नितीति॒ वन॒स्पती॒न्॒. वन॒स्पती॒ निति॑ । \newline
37. इत्या॑हा॒हे तीत्या॑ह । \newline
38. आ॒है॒भ्य ए॒भ्य आ॑हा है॒भ्यः । \newline
39. ए॒भ्य ए॒वैवैभ्य ए॒भ्य ए॒व । \newline
40. ए॒वैन॑ मेन मे॒वैवैन᳚म् । \newline
41. ए॒न॒म् ॅलो॒केभ्यो॑ लो॒केभ्य॑ एन मेनम् ॅलो॒केभ्यः॑ । \newline
42. लो॒केभ्यः॑ शमयति शमयति लो॒केभ्यो॑ लो॒केभ्यः॑ शमयति । \newline
43. श॒म॒य॒ति॒ प्र प्र श॑मयति शमयति॒ प्र । \newline
44. प्रैत्वे॑तु॒ प्र प्रैतु॑ । \newline
45. ए॒तु॒ वा॒जी वा॒ज्ये᳚त्वेतु वा॒जी । \newline
46. वा॒जी कनि॑क्रद॒त् कनि॑क्रदद् वा॒जी वा॒जी कनि॑क्रदत् । \newline
47. कनि॑क्रद॒ दितीति॒ कनि॑क्रद॒त् कनि॑क्रद॒ दिति॑ । \newline
48. इत्या॑हा॒हे तीत्या॑ह । \newline
49. आ॒ह॒ वा॒जी वा॒ज्या॑हाह वा॒जी । \newline
50. वा॒जी हि हि वा॒जी वा॒जी हि । \newline
51. ह्ये॑ष ए॒ष हि ह्ये॑षः । \newline
52. ए॒ष नान॑द॒न् नान॑द दे॒ष ए॒ष नान॑दत् । \newline
53. नान॑द॒द् रास॑भो॒ रास॑भो॒ नान॑द॒न् नान॑द॒द् रास॑भः । \newline
54. रास॑भः॒ पत्वा॒ पत्वा॒ रास॑भो॒ रास॑भः॒ पत्वा᳚ । \newline
55. पत्वेतीति॒ पत्वा॒ पत्वेति॑ । \newline
56. इत्या॑हा॒हे तीत्या॑ह । \newline

\textbf{Ghana Paata } \newline

1. स ई᳚श्व॒र ई᳚श्व॒रः स स ई᳚श्व॒रः प्र॒जाः प्र॒जा ई᳚श्व॒रः स स ई᳚श्व॒रः प्र॒जाः । \newline
2. ई॒श्व॒रः प्र॒जाः प्र॒जा ई᳚श्व॒र ई᳚श्व॒रः प्र॒जाः शु॒चा शु॒चा प्र॒जा ई᳚श्व॒र ई᳚श्व॒रः प्र॒जाः शु॒चा । \newline
3. प्र॒जाः शु॒चा शु॒चा प्र॒जाः प्र॒जाः शु॒चा प्र॒दहः॑ प्र॒दहः॑ शु॒चा प्र॒जाः प्र॒जाः शु॒चा प्र॒दहः॑ । \newline
4. प्र॒जा इति॑ प्र - जाः । \newline
5. शु॒चा प्र॒दहः॑ प्र॒दहः॑ शु॒चा शु॒चा प्र॒दहः॑ शि॒वः शि॒वः प्र॒दहः॑ शु॒चा शु॒चा प्र॒दहः॑ शि॒वः । \newline
6. प्र॒दहः॑ शि॒वः शि॒वः प्र॒दहः॑ प्र॒दहः॑ शि॒वो भ॑व भव शि॒वः प्र॒दहः॑ प्र॒दहः॑ शि॒वो भ॑व । \newline
7. प्र॒दह॒ इति॑ प्र - दहः॑ । \newline
8. शि॒वो भ॑व भव शि॒वः शि॒वो भ॑व प्र॒जाभ्यः॑ प्र॒जाभ्यो॑ भव शि॒वः शि॒वो भ॑व प्र॒जाभ्यः॑ । \newline
9. भ॒व॒ प्र॒जाभ्यः॑ प्र॒जाभ्यो॑ भव भव प्र॒जाभ्य॒ इतीति॑ प्र॒जाभ्यो॑ भव भव प्र॒जाभ्य॒ इति॑ । \newline
10. प्र॒जाभ्य॒ इतीति॑ प्र॒जाभ्यः॑ प्र॒जाभ्य॒ इत्या॑हा॒हेति॑ प्र॒जाभ्यः॑ प्र॒जाभ्य॒ इत्या॑ह । \newline
11. प्र॒जाभ्य॒ इति॑ प्र - जाभ्यः॑ । \newline
12. इत्या॑हा॒हेती त्या॑ह प्र॒जाभ्यः॑ प्र॒जाभ्य॑ आ॒हे तीत्या॑ह प्र॒जाभ्यः॑ । \newline
13. आ॒ह॒ प्र॒जाभ्यः॑ प्र॒जाभ्य॑ आहाह प्र॒जाभ्य॑ ए॒वैव प्र॒जाभ्य॑ आहाह प्र॒जाभ्य॑ ए॒व । \newline
14. प्र॒जाभ्य॑ ए॒वैव प्र॒जाभ्यः॑ प्र॒जाभ्य॑ ए॒वैन॑ मेन मे॒व प्र॒जाभ्यः॑ प्र॒जाभ्य॑ ए॒वैन᳚म् । \newline
15. प्र॒जाभ्य॒ इति॑ प्र - जाभ्यः॑ । \newline
16. ए॒वैन॑ मेन मे॒वैवैनꣳ॑ शमयति शमय त्येन मे॒वैवैनꣳ॑ शमयति । \newline
17. ए॒नꣳ॒॒ श॒म॒य॒ति॒ श॒म॒य॒ त्ये॒न॒ मे॒नꣳ॒॒ श॒म॒य॒ति॒ मानु॑षीभ्यो॒ मानु॑षीभ्यः शमय त्येन मेनꣳ शमयति॒ मानु॑षीभ्यः । \newline
18. श॒म॒य॒ति॒ मानु॑षीभ्यो॒ मानु॑षीभ्यः शमयति शमयति॒ मानु॑षीभ्य॒ स्त्वम् त्वम् मानु॑षीभ्यः शमयति शमयति॒ मानु॑षीभ्य॒ स्त्वम् । \newline
19. मानु॑षीभ्य॒ स्त्वम् त्वम् मानु॑षीभ्यो॒ मानु॑षीभ्य॒ स्त्व म॑ङ्गिरो अङ्गिर॒ स्त्वम् मानु॑षीभ्यो॒ मानु॑षीभ्य॒ स्त्व म॑ङ्गिरः । \newline
20. त्व म॑ङ्गिरो अङ्गिर॒ स्त्वम् त्व म॑ङ्गिर॒ इतीत्य॑ङ्गिर॒ स्त्वम् त्व म॑ङ्गिर॒ इति॑ । \newline
21. अ॒ङ्गि॒र॒ इतीत्य॑ङ्गिरो ऽङ्गिर॒ इत्या॑हा॒हे त्य॑ङ्गिरो ऽङ्गिर॒ इत्या॑ह । \newline
22. इत्या॑हा॒हे तीत्या॑ह मान॒व्यो॑ मान॒व्य॑ आ॒हे तीत्या॑ह मान॒व्यः॑ । \newline
23. आ॒ह॒ मा॒न॒व्यो॑ मान॒व्य॑ आहाह मान॒व्यो॑ हि हि मा॑न॒व्य॑ आहाह मान॒व्यो॑ हि । \newline
24. मा॒न॒व्यो॑ हि हि मा॑न॒व्यो॑ मान॒व्यो॑ हि प्र॒जाः प्र॒जा हि मा॑न॒व्यो॑ मान॒व्यो॑ हि प्र॒जाः । \newline
25. हि प्र॒जाः प्र॒जा हि हि प्र॒जा मा मा प्र॒जा हि हि प्र॒जा मा । \newline
26. प्र॒जा मा मा प्र॒जाः प्र॒जा मा द्यावा॑पृथि॒वी द्यावा॑पृथि॒वी मा प्र॒जाः प्र॒जा मा द्यावा॑पृथि॒वी । \newline
27. प्र॒जा इति॑ प्र - जाः । \newline
28. मा द्यावा॑पृथि॒वी द्यावा॑पृथि॒वी मा मा द्यावा॑पृथि॒वी अ॒भ्य॑भि द्यावा॑पृथि॒वी मा मा द्यावा॑पृथि॒वी अ॒भि । \newline
29. द्यावा॑पृथि॒वी अ॒भ्य॑भि द्यावा॑पृथि॒वी द्यावा॑पृथि॒वी अ॒भि शू॑शुचः शूशुचो अ॒भि द्यावा॑पृथि॒वी द्यावा॑पृथि॒वी अ॒भि शू॑शुचः । \newline
30. द्यावा॑पृथि॒वी इति॒ द्यावा᳚ - पृ॒थि॒वी । \newline
31. अ॒भि शू॑शुचः शूशुचो अ॒भ्य॑भि शू॑शुचो॒ मा मा शू॑शुचो अ॒भ्य॑भि शू॑शुचो॒ मा । \newline
32. शू॒शु॒चो॒ मा मा शू॑शुचः शूशुचो॒ मा ऽन्तरि॑क्ष म॒न्तरि॑क्ष॒म् मा शू॑शुचः शूशुचो॒ मा ऽन्तरि॑क्षम् । \newline
33. मा ऽन्तरि॑क्ष म॒न्तरि॑क्ष॒म् मा मा ऽन्तरि॑क्ष॒म् मा मा ऽन्तरि॑क्ष॒म् मा मा ऽन्तरि॑क्ष॒म् मा । \newline
34. अ॒न्तरि॑क्ष॒म् मा मा ऽन्तरि॑क्ष म॒न्तरि॑क्ष॒म् मा वन॒स्पती॒न्॒. वन॒स्पती॒न् मा ऽन्तरि॑क्ष म॒न्तरि॑क्ष॒म् मा वन॒स्पतीन्॑ । \newline
35. मा वन॒स्पती॒न्॒. वन॒स्पती॒न् मा मा वन॒स्पती॒ नितीति॒ वन॒स्पती॒न् मा मा वन॒स्पती॒ निति॑ । \newline
36. वन॒स्पती॒ नितीति॒ वन॒स्पती॒न्॒. वन॒स्पती॒ नित्या॑हा॒हेति॒ वन॒स्पती॒न्॒. वन॒स्पती॒ नित्या॑ह । \newline
37. इत्या॑हा॒हे तीत्या॑ है॒भ्य ए॒भ्य आ॒हे तीत्या॑ है॒भ्यः । \newline
38. आ॒है॒भ्य ए॒भ्य आ॑हा है॒भ्य ए॒वैवैभ्य आ॑हा है॒भ्य ए॒व । \newline
39. ए॒भ्य ए॒वै वैभ्य ए॒भ्य ए॒वैन॑ मेन मे॒वैभ्य ए॒भ्य ए॒वैन᳚म् । \newline
40. ए॒वैन॑ मेन मे॒वै वैन॑म् ॅलो॒केभ्यो॑ लो॒केभ्य॑ एन मे॒वै वैन॑म् ॅलो॒केभ्यः॑ । \newline
41. ए॒न॒म् ॅलो॒केभ्यो॑ लो॒केभ्य॑ एन मेनम् ॅलो॒केभ्यः॑ शमयति शमयति लो॒केभ्य॑ एन मेनम् ॅलो॒केभ्यः॑ शमयति । \newline
42. लो॒केभ्यः॑ शमयति शमयति लो॒केभ्यो॑ लो॒केभ्यः॑ शमयति॒ प्र प्र श॑मयति लो॒केभ्यो॑ लो॒केभ्यः॑ शमयति॒ प्र । \newline
43. श॒म॒य॒ति॒ प्र प्र श॑मयति शमयति॒ प्रैत्वे॑तु॒ प्र श॑मयति शमयति॒ प्रैतु॑ । \newline
44. प्रैत्वे॑तु॒ प्र प्रैतु॑ वा॒जी वा॒ज्ये॑तु॒ प्र प्रैतु॑ वा॒जी । \newline
45. ए॒तु॒ वा॒जी वा॒ज्ये᳚त्वेतु वा॒जी कनि॑क्रद॒त् कनि॑क्रदद् वा॒ज्ये᳚ त्वेतु वा॒जी कनि॑क्रदत् । \newline
46. वा॒जी कनि॑क्रद॒त् कनि॑क्रदद् वा॒जी वा॒जी कनि॑क्र द॒दितीति॒ कनि॑क्रदद् वा॒जी वा॒जी कनि॑क्र द॒दिति॑ । \newline
47. कनि॑क्र द॒दितीति॒ कनि॑क्रद॒त् कनि॑क्रद॒ दित्या॑हा॒हेति॒ कनि॑क्रद॒त् कनि॑क्र द॒दित्या॑ह । \newline
48. इत्या॑हा॒हे तीत्या॑ह वा॒जी वा॒ज्या॑हे तीत्या॑ह वा॒जी । \newline
49. आ॒ह॒ वा॒जी वा॒ज्या॑हाह वा॒जी हि हि वा॒ज्या॑हाह वा॒जी हि । \newline
50. वा॒जी हि हि वा॒जी वा॒जी ह्ये॑ष ए॒ष हि वा॒जी वा॒जी ह्ये॑षः । \newline
51. ह्ये॑ष ए॒ष हि ह्ये॑ष नान॑द॒न् नान॑ ददे॒ष हि ह्ये॑ष नान॑दत् । \newline
52. ए॒ष नान॑द॒न् नान॑ ददे॒ष ए॒ष नान॑द॒द् रास॑भो॒ रास॑भो॒ नान॑ ददे॒ष ए॒ष नान॑द॒द् रास॑भः । \newline
53. नान॑द॒द् रास॑भो॒ रास॑भो॒ नान॑द॒न् नान॑द॒द् रास॑भः॒ पत्वा॒ पत्वा॒ रास॑भो॒ नान॑द॒न् नान॑द॒द् रास॑भः॒ पत्वा᳚ । \newline
54. रास॑भः॒ पत्वा॒ पत्वा॒ रास॑भो॒ रास॑भः॒ पत्वेतीति॒ पत्वा॒ रास॑भो॒ रास॑भः॒ पत्वेति॑ । \newline
55. पत्वेतीति॒ पत्वा॒ पत्वे त्या॑हा॒हेति॒ पत्वा॒ पत्वे त्या॑ह । \newline
56. इत्या॑हा॒हे तीत्या॑ह॒ रास॑भो॒ रास॑भ आ॒हे तीत्या॑ह॒ रास॑भः । \newline
\pagebreak
\markright{ TS 5.1.5.7  \hfill https://www.vedavms.in \hfill}

\section{ TS 5.1.5.7 }

\textbf{TS 5.1.5.7 } \newline
\textbf{Samhita Paata} \newline

-ह॒ रास॑भ॒ इति॒ ह्ये॑तमृष॒योऽव॑द॒न् भर॑न्न॒ग्निं पु॑री॒ष्य॑मित्या॑हा॒ऽग्निꣳ ह्ये॑ष भर॑ति॒ मा पा॒द्यायु॑षः पु॒रेत्या॒हाऽऽ*यु॑रे॒वाऽस्मि॑न् दधाति॒ तस्मा᳚द् गर्द॒भः सर्व॒मायु॑रेति॒ तस्मा᳚द् गर्द॒भे पु॒राऽऽयु॑षः॒ प्रमी॑ते बिभ्यति॒ वृषा॒ऽग्निं ॅवृष॑णं॒ भर॒न्नित्या॑ह॒ वृषा॒ ह्ये॑ष वृषा॒ऽग्निर॒पां गर्भꣳ॑ - [  ] \newline

\textbf{Pada Paata} \newline

आ॒ह॒ । रास॑भः । इति॑ । हि । ए॒तम् । ऋष॑यः । अव॑दन्न् । भरन्न्॑ । अ॒ग्निम् । पु॒री॒ष्य᳚म् । इति॑ । आ॒ह॒ । अ॒ग्निम् । हि । ए॒षः । भर॑ति । मा । पा॒दि॒ । आयु॑षः । पु॒रा । इति॑ । आ॒ह॒ । आयुः॑ । ए॒व । अ॒स्मि॒न्न् । द॒धा॒ति॒ । तस्मा᳚त् । ग॒र्द॒भः । सर्व᳚म् । आयुः॑ । ए॒ति॒ । तस्मा᳚त् । ग॒र्द॒भे । पु॒रा । आयु॑षः । प्रमी॑त॒ इति॒ प्र-मी॒ते॒ । बि॒भ्य॒ति॒ । वृषा᳚ । अ॒ग्निम् । वृष॑णम् । भरन्न्॑ । इति॑ । आ॒ह॒ । वृषा᳚ । हि । ए॒षः । वृषा᳚ । अ॒ग्निः । अ॒पाम् । गर्भ᳚म् ।  \newline


\textbf{Krama Paata} \newline

आ॒ह॒ रास॑भः । रास॑भ॒ इति॑ । इति॒ हि । ह्ये॑तम् । ए॒तमृष॑यः । ऋष॒योऽव॑दन्न् । अव॑द॒न् भरन्न्॑ । भर॑न्न॒ग्निम् । अ॒ग्निम् पु॑री॒ष्य᳚म् । पु॒री॒ष्य॑मिति॑ । इत्या॑ह । आ॒हा॒ग्निम् । अ॒ग्निꣳ हि । ह्ये॑षः । ए॒ष भर॑ति । भर॑ति॒ मा । मा पा॑दि । पा॒द्यायु॑षः । आयु॑षः पु॒रा । पु॒रेति॑ । इत्या॑ह । आ॒हायुः॑ । आयु॑रे॒व । ए॒वास्मिन्न्॑ । अ॒स्मि॒न् द॒धा॒ति॒ । द॒धा॒ति॒ तस्मा᳚त् । तस्मा᳚द् गर्द॒भः । ग॒र्द॒भः सर्व᳚म् । सर्व॒मायुः॑ । आयु॑रेति । ए॒ति॒ तस्मा᳚त् । तस्मा᳚द् गर्द॒भे । ग॒र्द॒भे पु॒रा । पु॒राऽऽयु॑षः । आयु॑षः॒ प्रमी॑ते । प्रमी॑ते बिभ्यति । प्रमी॑त॒ इति॒ प्र - मी॒ते॒ । बि॒भ्य॒ति॒ वृषा᳚ । वृषा॒ऽग्निम् । अ॒ग्निम् ॅवृष॑णम् । वृष॑ण॒म् भरन्न्॑ । भर॒न्निति॑ । इत्या॑ह । आ॒ह॒ वृषा᳚ । वृषा॒ हि । ह्ये॑षः । ए॒ष वृषा᳚ । वृषा॒ऽग्निः । अ॒ग्निर॒पाम् । अ॒पाम् गर्भ᳚म् । गर्भꣳ॑ समु॒द्रिय᳚म् \newline

\textbf{Jatai Paata} \newline

1. आ॒ह॒ रास॑भो॒ रास॑भ आहाह॒ रास॑भः । \newline
2. रास॑भ॒ इतीति॒ रास॑भो॒ रास॑भ॒ इति॑ । \newline
3. इति॒ हि हीतीति॒ हि । \newline
4. ह्ये॑त मे॒तꣳ हि ह्ये॑तम् । \newline
5. ए॒त मृष॑य॒ ऋष॑य ए॒त मे॒त मृष॑यः । \newline
6. ऋष॒यो ऽव॑द॒न् नव॑द॒न् नृष॑य॒ ऋष॒यो ऽव॑दन्न् । \newline
7. अव॑द॒न् भर॒न् भर॒न् नव॑द॒न् नव॑द॒न् भरन्न्॑ । \newline
8. भर॑न् न॒ग्नि म॒ग्निम् भर॒न् भर॑न् न॒ग्निम् । \newline
9. अ॒ग्निम् पु॑री॒ष्य॑म् पुरी॒ष्य॑ म॒ग्नि म॒ग्निम् पु॑री॒ष्य᳚म् । \newline
10. पु॒री॒ष्य॑ मितीति॑ पुरी॒ष्य॑म् पुरी॒ष्य॑ मिति॑ । \newline
11. इत्या॑हा॒हे तीत्या॑ह । \newline
12. आ॒हा॒ग्नि म॒ग्नि मा॑हा हा॒ग्निम् । \newline
13. अ॒ग्निꣳ हि ह्य॑ग्नि म॒ग्निꣳ हि । \newline
14. ह्ये॑ष ए॒ष हि ह्ये॑षः । \newline
15. ए॒ष भर॑ति॒ भर॑ त्ये॒ष ए॒ष भर॑ति । \newline
16. भर॑ति॒ मा मा भर॑ति॒ भर॑ति॒ मा । \newline
17. मा पा॑दि पादि॒ मा मा पा॑दि । \newline
18. पा॒द्यायु॑ष॒ आयु॑षः पादि पा॒द्यायु॑षः । \newline
19. आयु॑षः पु॒रा पु॒रा ऽऽयु॑ष॒ आयु॑षः पु॒रा । \newline
20. पु॒रेतीति॑ पु॒रा पु॒रेति॑ । \newline
21. इत्या॑हा॒हे तीत्या॑ह । \newline
22. आ॒हायु॒ रायु॑ राहा॒ हायुः॑ । \newline
23. आयु॑ रे॒वैवायु॒ रायु॑ रे॒व । \newline
24. ए॒वास्मि॑न् नस्मिन् ने॒वैवास्मिन्न्॑ । \newline
25. अ॒स्मि॒न् द॒धा॒ति॒ द॒धा॒ त्य॒स्मि॒न् न॒स्मि॒न् द॒धा॒ति॒ । \newline
26. द॒धा॒ति॒ तस्मा॒त् तस्मा᳚द् दधाति दधाति॒ तस्मा᳚त् । \newline
27. तस्मा᳚द् गर्द॒भो ग॑र्द॒भ स्तस्मा॒त् तस्मा᳚द् गर्द॒भः । \newline
28. ग॒र्द॒भः सर्वꣳ॒॒ सर्व॑म् गर्द॒भो ग॑र्द॒भः सर्व᳚म् । \newline
29. सर्व॒ मायु॒रायुः॒ सर्वꣳ॒॒ सर्व॒ मायुः॑ । \newline
30. आयु॑ रेत्ये॒त्यायु॒ रायु॑रेति । \newline
31. ए॒ति॒ तस्मा॒त् तस्मा॑ देत्येति॒ तस्मा᳚त् । \newline
32. तस्मा᳚द् गर्द॒भे ग॑र्द॒भे तस्मा॒त् तस्मा᳚द् गर्द॒भे । \newline
33. ग॒र्द॒भे पु॒रा पु॒रा ग॑र्द॒भे ग॑र्द॒भे पु॒रा । \newline
34. पु॒रा ऽऽयु॑ष॒ आयु॑षः पु॒रा पु॒रा ऽऽयु॑षः । \newline
35. आयु॑षः॒ प्रमी॑ते॒ प्रमी॑त॒ आयु॑ष॒ आयु॑षः॒ प्रमी॑ते । \newline
36. प्रमी॑ते बिभ्यति बिभ्यति॒ प्रमी॑ते॒ प्रमी॑ते बिभ्यति । \newline
37. प्रमी॑त॒ इति॒ प्र - मी॒ते॒ । \newline
38. बि॒भ्य॒ति॒ वृषा॒ वृषा॑ बिभ्यति बिभ्यति॒ वृषा᳚ । \newline
39. वृषा॒ ऽग्नि म॒ग्निं ॅवृषा॒ वृषा॒ ऽग्निम् । \newline
40. अ॒ग्निं ॅवृष॑णं॒ ॅवृष॑ण म॒ग्नि म॒ग्निं ॅवृष॑णम् । \newline
41. वृष॑ण॒म् भर॒न् भर॒न् वृष॑णं॒ ॅवृष॑ण॒म् भरन्न्॑ । \newline
42. भर॒न् नितीति॒ भर॒न् भर॒न् निति॑ । \newline
43. इत्या॑हा॒हे तीत्या॑ह । \newline
44. आ॒ह॒ वृषा॒ वृषा॑ ऽऽहाह॒ वृषा᳚ । \newline
45. वृषा॒ हि हि वृषा॒ वृषा॒ हि । \newline
46. ह्ये॑ष ए॒ष हि ह्ये॑षः । \newline
47. ए॒ष वृषा॒ वृषै॒ष ए॒ष वृषा᳚ । \newline
48. वृषा॒ ऽग्नि र॒ग्निर् वृषा॒ वृषा॒ ऽग्निः । \newline
49. अ॒ग्निर॒पा म॒पा म॒ग्नि र॒ग्नि र॒पाम् । \newline
50. अ॒पाम् गर्भ॒म् गर्भ॑ म॒पा म॒पाम् गर्भ᳚म् । \newline
51. गर्भꣳ॑ समु॒द्रियꣳ॑ समु॒द्रिय॒म् गर्भ॒म् गर्भꣳ॑ समु॒द्रिय᳚म् । \newline

\textbf{Ghana Paata } \newline

1. आ॒ह॒ रास॑भो॒ रास॑भ आहाह॒ रास॑भ॒ इतीति॒ रास॑भ आहाह॒ रास॑भ॒ इति॑ । \newline
2. रास॑भ॒ इतीति॒ रास॑भो॒ रास॑भ॒ इति॒ हि हीति॒ रास॑भो॒ रास॑भ॒ इति॒ हि । \newline
3. इति॒ हि हीतीति॒ ह्ये॑त मे॒तꣳ हीतीति॒ ह्ये॑तम् । \newline
4. ह्ये॑त मे॒तꣳ हि ह्ये॑त मृष॑य॒ ऋष॑य ए॒तꣳ हि ह्ये॑त मृष॑यः । \newline
5. ए॒त मृष॑य॒ ऋष॑य ए॒त मे॒त मृष॒यो ऽव॑द॒न् नव॑द॒न् नृष॑य ए॒त मे॒त मृष॒यो ऽव॑दन्न् । \newline
6. ऋष॒यो ऽव॑द॒न् नव॑द॒न् नृष॑य॒ ऋष॒यो ऽव॑द॒न् भर॒न् भर॒न् नव॑द॒न् नृष॑य॒ ऋष॒यो ऽव॑द॒न् भरन्न्॑ । \newline
7. अव॑द॒न् भर॒न् भर॒न् नव॑द॒न् नव॑द॒न् भर॑न् न॒ग्नि म॒ग्निम् भर॒न् नव॑द॒न् नव॑द॒न् भर॑न् न॒ग्निम् । \newline
8. भर॑न् न॒ग्नि म॒ग्निम् भर॒न् भर॑न् न॒ग्निम् पु॑री॒ष्य॑म् पुरी॒ष्य॑ म॒ग्निम् भर॒न् भर॑न् न॒ग्निम् पु॑री॒ष्य᳚म् । \newline
9. अ॒ग्निम् पु॑री॒ष्य॑म् पुरी॒ष्य॑ म॒ग्नि म॒ग्निम् पु॑री॒ष्य॑ मितीति॑ पुरी॒ष्य॑ म॒ग्नि म॒ग्निम् पु॑री॒ष्य॑ मिति॑ । \newline
10. पु॒री॒ष्य॑ मितीति॑ पुरी॒ष्य॑म् पुरी॒ष्य॑ मित्या॑हा॒हेति॑ पुरी॒ष्य॑म् पुरी॒ष्य॑ मित्या॑ह । \newline
11. इत्या॑हा॒हे तीत्या॑हा॒ग्नि म॒ग्नि मा॒हे तीत्या॑हा॒ग्निम् । \newline
12. आ॒हा॒ग्नि म॒ग्नि मा॑हाहा॒ग्निꣳ हि ह्य॑ग्नि मा॑हाहा॒ग्निꣳ हि । \newline
13. अ॒ग्निꣳ हि ह्य॑ग्नि म॒ग्निꣳ ह्ये॑ष ए॒ष ह्य॑ग्नि म॒ग्निꣳ ह्ये॑षः । \newline
14. ह्ये॑ष ए॒ष हि ह्ये॑ष भर॑ति॒ भर॑त्ये॒ष हि ह्ये॑ष भर॑ति । \newline
15. ए॒ष भर॑ति॒ भर॑ त्ये॒ष ए॒ष भर॑ति॒ मा मा भर॑ त्ये॒ष ए॒ष भर॑ति॒ मा । \newline
16. भर॑ति॒ मा मा भर॑ति॒ भर॑ति॒ मा पा॑दि पादि॒ मा भर॑ति॒ भर॑ति॒ मा पा॑दि । \newline
17. मा पा॑दि पादि॒ मा मा पा॒द्यायु॑ष॒ आयु॑षः पादि॒ मा मा पा॒द्यायु॑षः । \newline
18. पा॒द्यायु॑ष॒ आयु॑षः पादि पा॒द्यायु॑षः पु॒रा पु॒रा ऽऽयु॑षः पादि पा॒द्यायु॑षः पु॒रा । \newline
19. आयु॑षः पु॒रा पु॒रा ऽऽयु॑ष॒ आयु॑षः पु॒रेतीति॑ पु॒रा ऽऽयु॑ष॒ आयु॑षः पु॒रेति॑ । \newline
20. पु॒रेतीति॑ पु॒रा पु॒रे त्या॑हा॒हेति॑ पु॒रा पु॒रेत्या॑ह । \newline
21. इत्या॑हा॒हे तीत्या॒हायु॒ रायु॑ रा॒हे तीत्या॒हायुः॑ । \newline
22. आ॒हायु॒ रायु॑ राहा॒हायु॑ रे॒वैवायु॑ राहा॒हायु॑ रे॒व । \newline
23. आयु॑ रे॒वैवायु॒ रायु॑ रे॒वास्मि॑न् नस्मिन् ने॒वायु॒ रायु॑ रे॒वास्मिन्न्॑ । \newline
24. ए॒वास्मि॑न् नस्मिन् ने॒वैवास्मि॑न् दधाति दधा त्यस्मिन् ने॒वैवास्मि॑न् दधाति । \newline
25. अ॒स्मि॒न् द॒धा॒ति॒ द॒धा॒ त्य॒स्मि॒न् न॒स्मि॒न् द॒धा॒ति॒ तस्मा॒त् तस्मा᳚द् दधा त्यस्मिन् नस्मिन् दधाति॒ तस्मा᳚त् । \newline
26. द॒धा॒ति॒ तस्मा॒त् तस्मा᳚द् दधाति दधाति॒ तस्मा᳚द् गर्द॒भो ग॑र्द॒भ स्तस्मा᳚द् दधाति दधाति॒ तस्मा᳚द् गर्द॒भः । \newline
27. तस्मा᳚द् गर्द॒भो ग॑र्द॒भ स्तस्मा॒त् तस्मा᳚द् गर्द॒भः सर्वꣳ॒॒ सर्व॑म् गर्द॒भ स्तस्मा॒त् तस्मा᳚द् गर्द॒भः सर्व᳚म् । \newline
28. ग॒र्द॒भः सर्वꣳ॒॒ सर्व॑म् गर्द॒भो ग॑र्द॒भः सर्व॒ मायु॒ रायुः॒ सर्व॑म् गर्द॒भो ग॑र्द॒भः सर्व॒ मायुः॑ । \newline
29. सर्व॒ मायु॒ रायुः॒ सर्वꣳ॒॒ सर्व॒ मायु॑ रेत्ये॒ त्यायुः॒ सर्वꣳ॒॒ सर्व॒ मायु॑रेति । \newline
30. आयु॑ रेत्ये॒ त्यायु॒ रायु॑ रेति॒ तस्मा॒त् तस्मा॑ दे॒त्यायु॒ रायु॑ रेति॒ तस्मा᳚त् । \newline
31. ए॒ति॒ तस्मा॒त् तस्मा॑ देत्येति॒ तस्मा᳚द् गर्द॒भे ग॑र्द॒भे तस्मा॑ देत्येति॒ तस्मा᳚द् गर्द॒भे । \newline
32. तस्मा᳚द् गर्द॒भे ग॑र्द॒भे तस्मा॒त् तस्मा᳚द् गर्द॒भे पु॒रा पु॒रा ग॑र्द॒भे तस्मा॒त् तस्मा᳚द् गर्द॒भे पु॒रा । \newline
33. ग॒र्द॒भे पु॒रा पु॒रा ग॑र्द॒भे ग॑र्द॒भे पु॒रा ऽऽयु॑ष॒ आयु॑षः पु॒रा ग॑र्द॒भे ग॑र्द॒भे पु॒रा ऽऽयु॑षः । \newline
34. पु॒रा ऽऽयु॑ष॒ आयु॑षः पु॒रा पु॒रा ऽऽयु॑षः॒ प्रमी॑ते॒ प्रमी॑त॒ आयु॑षः पु॒रा पु॒रा ऽऽयु॑षः॒ प्रमी॑ते । \newline
35. आयु॑षः॒ प्रमी॑ते॒ प्रमी॑त॒ आयु॑ष॒ आयु॑षः॒ प्रमी॑ते बिभ्यति बिभ्यति॒ प्रमी॑त॒ आयु॑ष॒ आयु॑षः॒ प्रमी॑ते बिभ्यति । \newline
36. प्रमी॑ते बिभ्यति बिभ्यति॒ प्रमी॑ते॒ प्रमी॑ते बिभ्यति॒ वृषा॒ वृषा॑ बिभ्यति॒ प्रमी॑ते॒ प्रमी॑ते बिभ्यति॒ वृषा᳚ । \newline
37. प्रमी॑त॒ इति॒ प्र - मी॒ते॒ । \newline
38. बि॒भ्य॒ति॒ वृषा॒ वृषा॑ बिभ्यति बिभ्यति॒ वृषा॒ ऽग्नि म॒ग्निं ॅवृषा॑ बिभ्यति बिभ्यति॒ वृषा॒ ऽग्निम् । \newline
39. वृषा॒ ऽग्नि म॒ग्निं ॅवृषा॒ वृषा॒ ऽग्निं ॅवृष॑णं॒ ॅवृष॑ण म॒ग्निं ॅवृषा॒ वृषा॒ ऽग्निं ॅवृष॑णम् । \newline
40. अ॒ग्निं ॅवृष॑णं॒ ॅवृष॑ण म॒ग्नि म॒ग्निं ॅवृष॑ण॒म् भर॒न् भर॒न् वृष॑ण म॒ग्नि म॒ग्निं ॅवृष॑ण॒म् भरन्न्॑ । \newline
41. वृष॑ण॒म् भर॒न् भर॒न् वृष॑णं॒ ॅवृष॑ण॒म् भर॒न् नितीति॒ भर॒न् वृष॑णं॒ ॅवृष॑ण॒म् भर॒न् निति॑ । \newline
42. भर॒न् नितीति॒ भर॒न् भर॒न् नित्या॑ हा॒हेति॒ भर॒न् भर॒न् नित्या॑ह । \newline
43. इत्या॑हा॒हे तीत्या॑ह॒ वृषा॒ वृषा॒ ऽऽहे तीत्या॑ह॒ वृषा᳚ । \newline
44. आ॒ह॒ वृषा॒ वृषा॑ ऽऽहाह॒ वृषा॒ हि हि वृषा॑ ऽऽहाह॒ वृषा॒ हि । \newline
45. वृषा॒ हि हि वृषा॒ वृषा॒ ह्ये॑ष ए॒ष हि वृषा॒ वृषा॒ ह्ये॑षः । \newline
46. ह्ये॑ष ए॒ष हि ह्ये॑ष वृषा॒ वृषै॒ष हि ह्ये॑ष वृषा᳚ । \newline
47. ए॒ष वृषा॒ वृषै॒ष ए॒ष वृषा॒ ऽग्नि र॒ग्निर् वृषै॒ष ए॒ष वृषा॒ ऽग्निः । \newline
48. वृषा॒ ऽग्नि र॒ग्निर् वृषा॒ वृषा॒ ऽग्नि र॒पा म॒पा म॒ग्निर् वृषा॒ वृषा॒ ऽग्नि र॒पाम् । \newline
49. अ॒ग्निर॒पा म॒पा म॒ग्नि र॒ग्नि र॒पाम् गर्भ॒म् गर्भ॑ म॒पा म॒ग्नि र॒ग्नि र॒पाम् गर्भ᳚म् । \newline
50. अ॒पाम् गर्भ॒म् गर्भ॑ म॒पा म॒पाम् गर्भꣳ॑ समु॒द्रियꣳ॑ समु॒द्रिय॒म् गर्भ॑ म॒पा म॒पाम् गर्भꣳ॑ समु॒द्रिय᳚म् । \newline
51. गर्भꣳ॑ समु॒द्रियꣳ॑ समु॒द्रिय॒म् गर्भ॒म् गर्भꣳ॑ समु॒द्रिय॒ मितीति॑ समु॒द्रिय॒म् गर्भ॒म् गर्भꣳ॑ समु॒द्रिय॒ मिति॑ । \newline
\pagebreak
\markright{ TS 5.1.5.8  \hfill https://www.vedavms.in \hfill}

\section{ TS 5.1.5.8 }

\textbf{TS 5.1.5.8 } \newline
\textbf{Samhita Paata} \newline

समु॒द्रिय॒-मित्या॑हा॒ऽपाꣳ ह्ये॑ष गर्भो॒ यद॒ग्निरग्न॒ आ या॑हि वी॒तय॒ इति॒ वा इ॒मौ लो॒कौ व्यै॑ता॒मग्न॒ आ या॑हि वी॒तय॒ इति॒ यदाहा॒ ऽनयो᳚र्लो॒कयो॒-र्वीत्यै॒ प्रच्यु॑तो॒ वा ए॒ष आ॒यत॑ना॒दग॑तः प्रति॒ष्ठाꣳ स ए॒तर्.ह्य॑द्ध्व॒र्युं च॒ यज॑मानं च द्ध्यायत्यृ॒तꣳ स॒त्यमित्या॑हे॒यं ॅवा ऋ॒तम॒सौ - [  ] \newline

\textbf{Pada Paata} \newline

स॒मु॒द्रिय᳚म् । इति॑ । आ॒ह॒ । अ॒पाम् । हि । ए॒षः । गर्भः॑ । यत् । अ॒ग्निः । अग्ने᳚ । एति॑ । या॒हि॒ । वी॒तये᳚ । इति॑ । वै । इ॒मौ । लो॒कौ । वीति॑ । ऐ॒ता॒म् । अग्ने᳚ । एति॑ । या॒हि॒ । वी॒तये᳚ । इति॑ । यत् । आह॑ । अ॒नयोः᳚ । लो॒कयोः᳚ । वीत्या॒ इति॒ वि - इ॒त्यै॒ । प्रच्यु॑त॒ इति॒ प्र - च्यु॒तः॒ । वै । ए॒षः । आ॒यत॑ना॒दित्या᳚ - यत॑नात् । अग॑तः । प्र॒ति॒ष्ठामिति॑ प्रति - स्थाम् । सः । ए॒तर्.हि॑ । अ॒द्ध्व॒र्युम् । च॒ । यज॑मानम् । च॒ । ध्या॒य॒ति॒ । ऋ॒तम् । स॒त्यम् । इति॑ । आ॒ह॒ । इ॒यम् । वै । ऋ॒तम् । अ॒सौ ।  \newline


\textbf{Krama Paata} \newline

स॒मु॒द्रिय॒मिति॑ । इत्या॑ह । आ॒हा॒पाम् । अ॒पाꣳ हि । ह्ये॑षः । ए॒ष गर्भः॑ । गर्भो॒ यत् । यद॒ग्निः । अ॒ग्निरग्ने᳚ । अग्न॒ आ । आ या॑हि । या॒हि॒ वी॒तये᳚ । वी॒तय॒ इति॑ । इति॒ वै । वा इ॒मौ । इ॒मौ लो॒कौ । लो॒कौ वि । व्यै॑ताम् । ऐ॒ता॒मग्ने᳚ । अग्न॒ आ । आ या॑हि । या॒हि॒ वी॒तये᳚ । वी॒तय॒ इति॑ । इति॒ यत् । यदाह॑ । आहा॒नयोः᳚ । अ॒नयो᳚र् लो॒कयोः᳚ । लो॒कयो॒र् वीत्यै᳚ । वीत्यै॒ प्रच्यु॑तः । वीत्या॒ इति॒ वि - इ॒त्यै॒ । प्रच्यु॑तो॒ वै । प्रच्यु॑त॒ इति॒ प्र - च्यु॒तः॒ । वा ए॒षः । ए॒ष आ॒यत॑नात् । आ॒यत॑ना॒दग॑तः । आ॒यत॑ना॒दित्या᳚ - यत॑नात् । अग॑तः प्रति॒ष्ठाम् । प्र॒ति॒ष्ठाꣳ सः । प्र॒ति॒ष्ठामिति॑ प्रति - स्थाम् । स ए॒तर्.हि॑ । ए॒तर्ह्य॑द्ध्व॒र्युम् । अ॒द्ध्व॒र्युम् च॑ । च॒ यज॑मानम् । यज॑मानम् च । च॒ ध्या॒य॒ति॒ । ध्या॒य॒त्यृ॒तम् । ऋ॒तꣳ स॒त्यम् । स॒त्यमिति॑ । इत्या॑ह । आ॒हे॒यम् । इ॒यम् ॅवै । वा ऋ॒तम् । ऋ॒तम॒सौ । अ॒सौ स॒त्यम् \newline

\textbf{Jatai Paata} \newline

1. स॒मु॒द्रिय॒ मितीति॑ समु॒द्रियꣳ॑ समु॒द्रिय॒ मिति॑ । \newline
2. इत्या॑हा॒हे तीत्या॑ह । \newline
3. आ॒हा॒पा म॒पा मा॑हाहा॒पाम् । \newline
4. अ॒पाꣳ हि ह्य॑पा म॒पाꣳ हि । \newline
5. ह्ये॑ष ए॒ष हि ह्ये॑षः । \newline
6. ए॒ष गर्भो॒ गर्भ॑ ए॒ष ए॒ष गर्भः॑ । \newline
7. गर्भो॒ यद् यद् गर्भो॒ गर्भो॒ यत् । \newline
8. यद॒ग्नि र॒ग्निर् यद् यद॒ग्निः । \newline
9. अ॒ग्नि रग्ने ऽग्ने॒ ऽग्नि र॒ग्नि रग्ने᳚ । \newline
10. अग्न॒ आ ऽग्ने ऽग्न॒ आ । \newline
11. आ या॑हि या॒ह्या या॑हि । \newline
12. या॒हि॒ वी॒तये॑ वी॒तये॑ याहि याहि वी॒तये᳚ । \newline
13. वी॒तय॒ इतीति॑ वी॒तये॑ वी॒तय॒ इति॑ । \newline
14. इति॒ वै वा इतीति॒ वै । \newline
15. वा इ॒मा वि॒मौ वै वा इ॒मौ । \newline
16. इ॒मौ लो॒कौ लो॒का वि॒मा वि॒मौ लो॒कौ । \newline
17. लो॒कौ वि वि लो॒कौ लो॒कौ वि । \newline
18. व्यै॑ता मैतां॒ ॅवि व्यै॑ताम् । \newline
19. ऐ॒ता॒ मग्ने ऽग्न॑ ऐता मैता॒ मग्ने᳚ । \newline
20. अग्न॒ आ ऽग्ने ऽग्न॒ आ । \newline
21. आ या॑हि या॒ह्या या॑हि । \newline
22. या॒हि॒ वी॒तये॑ वी॒तये॑ याहि याहि वी॒तये᳚ । \newline
23. वी॒तय॒ इतीति॑ वी॒तये॑ वी॒तय॒ इति॑ । \newline
24. इति॒ यद् यदितीति॒ यत् । \newline
25. यदाहाह॒ यद् यदाह॑ । \newline
26. आहा॒ नयो॑ र॒नयो॒ राहाहा॒ नयोः᳚ । \newline
27. अ॒नयो᳚र् लो॒कयो᳚र् लो॒कयो॑ र॒नयो॑ र॒नयो᳚र् लो॒कयोः᳚ । \newline
28. लो॒कयो॒र् वीत्यै॒ वीत्यै॑ लो॒कयो᳚र् लो॒कयो॒र् वीत्यै᳚ । \newline
29. वीत्यै॒ प्रच्यु॑तः॒ प्रच्यु॑तो॒ वीत्यै॒ वीत्यै॒ प्रच्यु॑तः । \newline
30. वीत्या॒ इति॒ वि - इ॒त्यै॒ । \newline
31. प्रच्यु॑तो॒ वै वै प्रच्यु॑तः॒ प्रच्यु॑तो॒ वै । \newline
32. प्रच्यु॑त॒ इति॒ प्र - च्यु॒तः॒ । \newline
33. वा ए॒ष ए॒ष वै वा ए॒षः । \newline
34. ए॒ष आ॒यत॑ना दा॒यत॑ना दे॒ष ए॒ष आ॒यत॑नात् । \newline
35. आ॒यत॑ना॒ दग॒तो ऽग॑त आ॒यत॑ना दा॒यत॑ना॒ दग॑तः । \newline
36. आ॒यत॑ना॒दित्या᳚ - यत॑नात् । \newline
37. अग॑तः प्रति॒ष्ठाम् प्र॑ति॒ष्ठा मग॒तो ऽग॑तः प्रति॒ष्ठाम् । \newline
38. प्र॒ति॒ष्ठाꣳ स स प्र॑ति॒ष्ठाम् प्र॑ति॒ष्ठाꣳ सः । \newline
39. प्र॒ति॒ष्ठामिति॑ प्रति - स्थाम् । \newline
40. स ए॒तर् ह्ये॒तर्.हि॒ स स ए॒तर्.हि॑ । \newline
41. ए॒तर् ह्य॑द्ध्व॒र्यु म॑द्ध्व॒र्यु मे॒तर् ह्ये॒तर् ह्य॑द्ध्व॒र्युम् । \newline
42. अ॒द्ध्व॒र्युम् च॑ चाद्ध्व॒र्यु म॑द्ध्व॒र्युम् च॑ । \newline
43. च॒ यज॑मानं॒ ॅयज॑मानम् च च॒ यज॑मानम् । \newline
44. यज॑मानम् च च॒ यज॑मानं॒ ॅयज॑मानम् च । \newline
45. च॒ ध्या॒य॒ति॒ ध्या॒य॒ति॒ च॒ च॒ ध्या॒य॒ति॒ । \newline
46. ध्या॒य॒ त्यृ॒त मृ॒तम् ध्या॑यति ध्याय त्यृ॒तम् । \newline
47. ऋ॒तꣳ स॒त्यꣳ स॒त्य मृ॒त मृ॒तꣳ स॒त्यम् । \newline
48. स॒त्य मितीति॑ स॒त्यꣳ स॒त्य मिति॑ । \newline
49. इत्या॑हा॒हे तीत्या॑ह । \newline
50. आ॒हे॒य मि॒य मा॑हा हे॒यम् । \newline
51. इ॒यं ॅवै वा इ॒य मि॒यं ॅवै । \newline
52. वा ऋ॒त मृ॒तं ॅवै वा ऋ॒तम् । \newline
53. ऋ॒त म॒सा व॒सा वृ॒त मृ॒त म॒सौ । \newline
54. अ॒सौ स॒त्यꣳ स॒त्य म॒सा व॒सौ स॒त्यम् । \newline

\textbf{Ghana Paata } \newline

1. स॒मु॒द्रिय॒ मितीति॑ समु॒द्रियꣳ॑ समु॒द्रिय॒ मित्या॑हा॒हेति॑ समु॒द्रियꣳ॑ समु॒द्रिय॒ मित्या॑ह । \newline
2. इत्या॑हा॒हे तीत्या॑हा॒पा म॒पा मा॒हे तीत्या॑हा॒पाम् । \newline
3. आ॒हा॒पा म॒पा मा॑हा हा॒पाꣳ हि ह्य॑पा मा॑हा हा॒पाꣳ हि । \newline
4. अ॒पाꣳ हि ह्य॑पा म॒पाꣳ ह्ये॑ष ए॒ष ह्य॑पा म॒पाꣳ ह्ये॑षः । \newline
5. ह्ये॑ष ए॒ष हि ह्ये॑ष गर्भो॒ गर्भ॑ ए॒ष हि ह्ये॑ष गर्भः॑ । \newline
6. ए॒ष गर्भो॒ गर्भ॑ ए॒ष ए॒ष गर्भो॒ यद् यद् गर्भ॑ ए॒ष ए॒ष गर्भो॒ यत् । \newline
7. गर्भो॒ यद् यद् गर्भो॒ गर्भो॒ यद॒ग्नि र॒ग्निर् यद् गर्भो॒ गर्भो॒ यद॒ग्निः । \newline
8. यद॒ग्नि र॒ग्निर् यद् यद॒ग्नि रग्ने ऽग्ने॒ ऽग्निर् यद् यद॒ग्नि रग्ने᳚ । \newline
9. अ॒ग्नि रग्ने ऽग्ने॒ ऽग्नि र॒ग्नि रग्न॒ आ ऽग्ने॒ ऽग्नि र॒ग्नि रग्न॒ आ । \newline
10. अग्न॒ आ ऽग्ने ऽग्न॒ आ या॑हि या॒ह्या ऽग्ने ऽग्न॒ आ या॑हि । \newline
11. आ या॑हि या॒ह्या या॑हि वी॒तये॑ वी॒तये॑ या॒ह्या या॑हि वी॒तये᳚ । \newline
12. या॒हि॒ वी॒तये॑ वी॒तये॑ याहि याहि वी॒तय॒ इतीति॑ वी॒तये॑ याहि याहि वी॒तय॒ इति॑ । \newline
13. वी॒तय॒ इतीति॑ वी॒तये॑ वी॒तय॒ इति॒ वै वा इति॑ वी॒तये॑ वी॒तय॒ इति॒ वै । \newline
14. इति॒ वै वा इतीति॒ वा इ॒मा वि॒मौ वा इतीति॒ वा इ॒मौ । \newline
15. वा इ॒मा वि॒मौ वै वा इ॒मौ लो॒कौ लो॒का वि॒मौ वै वा इ॒मौ लो॒कौ । \newline
16. इ॒मौ लो॒कौ लो॒का वि॒मा वि॒मौ लो॒कौ वि वि लो॒का वि॒मा वि॒मौ लो॒कौ वि । \newline
17. लो॒कौ वि वि लो॒कौ लो॒कौ व्यै॑ता मैतां॒ ॅवि लो॒कौ लो॒कौ व्यै॑ताम् । \newline
18. व्यै॑ता मैतां॒ ॅवि व्यै॑ता॒ मग्ने ऽग्न॑ ऐतां॒ ॅवि व्यै॑ता॒ मग्ने᳚ । \newline
19. ऐ॒ता॒ मग्ने ऽग्न॑ ऐता मैता॒ मग्न॒ आ ऽग्न॑ ऐता मैता॒ मग्न॒ आ । \newline
20. अग्न॒ आ ऽग्ने ऽग्न॒ आ या॑हि या॒ह्या ऽग्ने ऽग्न॒ आ या॑हि । \newline
21. आ या॑हि या॒ह्या या॑हि वी॒तये॑ वी॒तये॑ या॒ह्या या॑हि वी॒तये᳚ । \newline
22. या॒हि॒ वी॒तये॑ वी॒तये॑ याहि याहि वी॒तय॒ इतीति॑ वी॒तये॑ याहि याहि वी॒तय॒ इति॑ । \newline
23. वी॒तय॒ इतीति॑ वी॒तये॑ वी॒तय॒ इति॒ यद् यदिति॑ वी॒तये॑ वी॒तय॒ इति॒ यत् । \newline
24. इति॒ यद् यदितीति॒ यदाहाह॒ यदितीति॒ यदाह॑ । \newline
25. यदाहाह॒ यद् यदाहा॒ नयो॑ र॒नयो॒ राह॒ यद् यदाहा॒ नयोः᳚ । \newline
26. आहा॒ नयो॑ र॒नयो॒ राहाहा॒ नयो᳚र् लो॒कयो᳚र् लो॒कयो॑ र॒नयो॒ राहाहा॒ नयो᳚र् लो॒कयोः᳚ । \newline
27. अ॒नयो᳚र् लो॒कयो᳚र् लो॒कयो॑ र॒नयो॑ र॒नयो᳚र् लो॒कयो॒र् वीत्यै॒ वीत्यै॑ लो॒कयो॑ र॒नयो॑ र॒नयो᳚र् लो॒कयो॒र् वीत्यै᳚ । \newline
28. लो॒कयो॒र् वीत्यै॒ वीत्यै॑ लो॒कयो᳚र् लो॒कयो॒र् वीत्यै॒ प्रच्यु॑तः॒ प्रच्यु॑तो॒ वीत्यै॑ लो॒कयो᳚र् लो॒कयो॒र् वीत्यै॒ प्रच्यु॑तः । \newline
29. वीत्यै॒ प्रच्यु॑तः॒ प्रच्यु॑तो॒ वीत्यै॒ वीत्यै॒ प्रच्यु॑तो॒ वै वै प्रच्यु॑तो॒ वीत्यै॒ वीत्यै॒ प्रच्यु॑तो॒ वै । \newline
30. वीत्या॒ इति॒ वि - इ॒त्यै॒ । \newline
31. प्रच्यु॑तो॒ वै वै प्रच्यु॑तः॒ प्रच्यु॑तो॒ वा ए॒ष ए॒ष वै प्रच्यु॑तः॒ प्रच्यु॑तो॒ वा ए॒षः । \newline
32. प्रच्यु॑त॒ इति॒ प्र - च्यु॒तः॒ । \newline
33. वा ए॒ष ए॒ष वै वा ए॒ष आ॒यत॑ना दा॒यत॑ना दे॒ष वै वा ए॒ष आ॒यत॑नात् । \newline
34. ए॒ष आ॒यत॑ना दा॒यत॑ना दे॒ष ए॒ष आ॒यत॑ना॒ दग॒तो ऽग॑त आ॒यत॑ना दे॒ष ए॒ष आ॒यत॑ना॒ दग॑तः । \newline
35. आ॒यत॑ना॒ दग॒तो ऽग॑त आ॒यत॑ना दा॒यत॑ना॒ दग॑तः प्रति॒ष्ठाम् प्र॑ति॒ष्ठा मग॑त आ॒यत॑ना दा॒यत॑ना॒ दग॑तः प्रति॒ष्ठाम् । \newline
36. आ॒यत॑ना॒दित्या᳚ - यत॑नात् । \newline
37. अग॑तः प्रति॒ष्ठाम् प्र॑ति॒ष्ठा मग॒तो ऽग॑तः प्रति॒ष्ठाꣳ स स प्र॑ति॒ष्ठा मग॒तो ऽग॑तः प्रति॒ष्ठाꣳ सः । \newline
38. प्र॒ति॒ष्ठाꣳ स स प्र॑ति॒ष्ठाम् प्र॑ति॒ष्ठाꣳ स ए॒तर्. ह्ये॒तर्.हि॒ स प्र॑ति॒ष्ठाम् प्र॑ति॒ष्ठाꣳ स ए॒तर्.हि॑ । \newline
39. प्र॒ति॒ष्ठामिति॑ प्रति - स्थाम् । \newline
40. स ए॒तर्. ह्ये॒तर्.हि॒ स स ए॒तर्. ह्य॑द्ध्व॒र्यु म॑द्ध्व॒र्यु मे॒तर्.हि॒ स स ए॒तर्. ह्य॑द्ध्व॒र्युम् । \newline
41. ए॒तर्. ह्य॑द्ध्व॒र्यु म॑द्ध्व॒र्यु मे॒तर्. ह्ये॒तर्. ह्य॑द्ध्व॒र्युम् च॑ चाद्ध्व॒र्यु मे॒तर्. ह्ये॒तर्. ह्य॑द्ध्व॒र्युम् च॑ । \newline
42. अ॒द्ध्व॒र्युम् च॑ चाद्ध्व॒र्यु म॑द्ध्व॒र्युम् च॒ यज॑मानं॒ ॅयज॑मानम् चाद्ध्व॒र्यु म॑द्ध्व॒र्युम् च॒ यज॑मानम् । \newline
43. च॒ यज॑मानं॒ ॅयज॑मानम् च च॒ यज॑मानम् च च॒ यज॑मानम् च च॒ यज॑मानम् च । \newline
44. यज॑मानम् च च॒ यज॑मानं॒ ॅयज॑मानम् च ध्यायति ध्यायति च॒ यज॑मानं॒ ॅयज॑मानम् च ध्यायति । \newline
45. च॒ ध्या॒य॒ति॒ ध्या॒य॒ति॒ च॒ च॒ ध्या॒य॒ त्यृ॒त मृ॒तम् ध्या॑यति च च ध्याय त्यृ॒तम् । \newline
46. ध्या॒य॒ त्यृ॒त मृ॒तम् ध्या॑यति ध्याय त्यृ॒तꣳ स॒त्यꣳ स॒त्य मृ॒तम् ध्या॑यति ध्याय त्यृ॒तꣳ स॒त्यम् । \newline
47. ऋ॒तꣳ स॒त्यꣳ स॒त्य मृ॒त मृ॒तꣳ स॒त्य मितीति॑ स॒त्य मृ॒त मृ॒तꣳ स॒त्य मिति॑ । \newline
48. स॒त्य मितीति॑ स॒त्यꣳ स॒त्य मित्या॑ हा॒हेति॑ स॒त्यꣳ स॒त्य मित्या॑ह । \newline
49. इत्या॑हा॒हे तीत्या॑हे॒ य मि॒य मा॒हे तीत्या॑हे॒ यम् । \newline
50. आ॒हे॒ य मि॒य मा॑हाहे॒ यं ॅवै वा इ॒य मा॑हाहे॒ यं ॅवै । \newline
51. इ॒यं ॅवै वा इ॒य मि॒यं ॅवा ऋ॒त मृ॒तं ॅवा इ॒य मि॒यं ॅवा ऋ॒तम् । \newline
52. वा ऋ॒त मृ॒तं ॅवै वा ऋ॒त म॒सा व॒सा वृ॒तं ॅवै वा ऋ॒त म॒सौ । \newline
53. ऋ॒त म॒सा व॒सा वृ॒त मृ॒त म॒सौ स॒त्यꣳ स॒त्य म॒सा वृ॒त मृ॒त म॒सौ स॒त्यम् । \newline
54. अ॒सौ स॒त्यꣳ स॒त्य म॒सा व॒सौ स॒त्य म॒नयो॑ र॒नयोः᳚ स॒त्य म॒सा व॒सौ स॒त्य म॒नयोः᳚ । \newline
\pagebreak
\markright{ TS 5.1.5.9  \hfill https://www.vedavms.in \hfill}

\section{ TS 5.1.5.9 }

\textbf{TS 5.1.5.9 } \newline
\textbf{Samhita Paata} \newline

स॒त्यम॒नयो॑रे॒वैनं॒ प्रति॑ ष्ठापयति॒ नाऽऽ*र्ति॒मार्च्छ॑त्यद्ध्व॒र्युर्न यज॑मानो॒ वरु॑णो॒ वा ए॒ष यज॑मानम॒भ्यैति॒ यद॒ग्निरुप॑नद्ध॒ ओष॑धयः॒ प्रति॑ गृह्णीता॒ग्निमे॒त-मित्या॑ह॒ शान्त्यै॒ व्यस्य॒न् विश्वा॒ अम॑ती॒ररा॑ती॒-रित्या॑ह॒ रक्ष॑सा॒मप॑हत्यै नि॒षीद॑न् नो॒ अप॑ दुर्म॒तिꣳ ह॑न॒दित्या॑ह॒ प्रति॑ष्ठित्या॒ ओष॑धयः॒ प्रति॑मोदद्ध्व - [  ] \newline

\textbf{Pada Paata} \newline

स॒त्यम् । अ॒नयोः᳚ । ए॒व । ए॒न॒म् । प्रतीति॑ । स्था॒प॒य॒ति॒ । न । आर्ति᳚म् । एति॑ । ऋ॒च्छ॒ति॒ । अ॒द्ध्व॒र्युः । न । यज॑मानः । वरु॑णः । वै । ए॒षः । यज॑मानम् । अ॒भि । एति॑ । ए॒ति॒ । यत् । अ॒ग्निः । उप॑नद्ध॒ इत्युप॑ - न॒द्धः॒ । ओष॑धयः । प्रतीति॑ । गृ॒ह्णी॒त॒ । अ॒ग्निम् । ए॒तम् । इति॑ । आ॒ह॒ । शान्त्यै᳚ । व्यस्य॒न्निति॑ वि - अस्यन्न्॑ । विश्वाः᳚ । अम॑तीः । अरा॑तीः । इति॑ । आ॒ह॒ । रक्ष॑साम् । अप॑हत्या॒ इत्यप॑ - ह॒त्यै॒ । नि॒षीद॒न्निति॑ नि-सीदन्न्॑ । नः॒ । अपेति॑ । दु॒र्म॒तिमिति॑ दुः - म॒तिम् । ह॒न॒त् । इति॑ । आ॒ह॒ । प्रति॑ष्ठित्या॒ इति॒ प्रति॑ - स्थि॒त्यै॒ । ओष॑धयः । प्रतीति॑ । मो॒द॒द्ध्व॒म् ।  \newline


\textbf{Krama Paata} \newline

स॒त्यम॒नयोः᳚ । अ॒नयो॑रे॒व । ए॒वैन᳚म् । ए॒न॒म् प्रति॑ । प्रति॑ ष्ठापयति । स्था॒प॒य॒ति॒ न । नार्ति᳚म् । आर्ति॒मा । आर्च्छ॑ति । ऋ॒च्छ॒त्य॒द्ध्व॒र्युः । अ॒द्ध्व॒र्युर् न । न यज॑मानः । यज॑मानो॒ वरु॑णः । वरु॑णो॒ वै । वा ए॒षः । ए॒ष यज॑मानम् । यज॑मानम॒भि । अ॒भ्या । ऐति॑ । ए॒ति॒ यत् । यद॒ग्निः । अ॒ग्निरुप॑नद्धः । उप॑नद्ध॒ ओष॑धयः । उप॑नद्ध॒ इत्युप॑ - न॒द्धः॒ । ओष॑धयः॒ प्रति॑ । प्रति॑ गृह्णीत । गृ॒ह्णी॒ता॒ग्निम् । अ॒ग्निमे॒तम् । ए॒तमिति॑ । इत्या॑ह । आ॒ह॒ शान्त्यै᳚ । शान्त्यै॒ व्यस्यन्न्॑ । व्यस्य॒न् विश्वाः᳚ । व्यस्य॒न्निति॑ वि - अस्यन्न्॑ । विश्वा॒ अम॑तीः । अम॑ती॒ररा॑तीः । अरा॑ती॒रिति॑ । इत्या॑ह । आ॒ह॒ रक्ष॑साम् । रक्ष॑सा॒मप॑हत्यै । अप॑हत्यै नि॒षीदन्न्॑ । अप॑हत्या॒ इत्य॑प - ह॒त्यै॒ । नि॒षीद॑न् नः । नि॒षीद॒न्निति॑ नि - सीदन्न्॑ । नो॒ अप॑ । अप॑ दुर्म॒तिम् । दु॒र्म॒तिꣳ ह॑नत् । दु॒र्म॒तिमिति॑ दुः - म॒तिम् । ह॒न॒दिति॑ । इत्या॑ह । आ॒ह॒ प्रति॑ष्ठित्यै । प्रति॑ष्ठित्या॒ ओष॑धयः । प्रति॑ष्ठित्या॒ इति॒ प्रति॑ - स्थि॒त्यै॒ । ओष॑धयः॒ प्रति॑ । प्रति॑ मोदद्ध्वम् ( ) । मो॒द॒द्ध्व॒मे॒न॒म् \newline

\textbf{Jatai Paata} \newline

1. स॒त्य म॒नयो॑ र॒नयोः᳚ स॒त्यꣳ स॒त्य म॒नयोः᳚ । \newline
2. अ॒नयो॑ रे॒वैवा नयो॑ र॒नयो॑ रे॒व । \newline
3. ए॒वैन॑ मेन मे॒वैवैन᳚म् । \newline
4. ए॒न॒म् प्रति॒ प्रत्ये॑न मेन॒म् प्रति॑ । \newline
5. प्रति॑ ष्ठापयति स्थापयति॒ प्रति॒ प्रति॑ ष्ठापयति । \newline
6. स्था॒प॒य॒ति॒ न न स्था॑पयति स्थापयति॒ न । \newline
7. नार्ति॒ मार्ति॒म् न नार्ति᳚म् । \newline
8. आर्ति॒ मा ऽऽर्ति॒ मार्ति॒ मा । \newline
9. आर्च्छ॑ त्यृच्छ त्यार्च्छति । \newline
10. ऋ॒च्छ॒ त्य॒द्ध्व॒र्यु र॑द्ध्व॒र्युर्. ऋ॑च्छ त्यृच्छ त्यद्ध्व॒र्युः । \newline
11. अ॒द्ध्व॒र्युर् न नाद्ध्व॒र्यु र॑द्ध्व॒र्युर् न । \newline
12. न यज॑मानो॒ यज॑मानो॒ न न यज॑मानः । \newline
13. यज॑मानो॒ वरु॑णो॒ वरु॑णो॒ यज॑मानो॒ यज॑मानो॒ वरु॑णः । \newline
14. वरु॑णो॒ वै वै वरु॑णो॒ वरु॑णो॒ वै । \newline
15. वा ए॒ष ए॒ष वै वा ए॒षः । \newline
16. ए॒ष यज॑मानं॒ ॅयज॑मान मे॒ष ए॒ष यज॑मानम् । \newline
17. यज॑मान म॒भ्य॑भि यज॑मानं॒ ॅयज॑मान म॒भि । \newline
18. अ॒भ्या ऽभ्य॑भ्या । \newline
19. ऐत्ये॒त्यैति॑ । \newline
20. ए॒ति॒ यद् यदे᳚त्येति॒ यत् । \newline
21. यद॒ग्नि र॒ग्निर् यद् यद॒ग्निः । \newline
22. अ॒ग्नि रुप॑नद्ध॒ उप॑नद्धो॒ ऽग्नि र॒ग्नि रुप॑नद्धः । \newline
23. उप॑नद्ध॒ ओष॑धय॒ ओष॑धय॒ उप॑नद्ध॒ उप॑नद्ध॒ ओष॑धयः । \newline
24. उप॑नद्ध॒ इत्युप॑ - न॒द्धः॒ । \newline
25. ओष॑धयः॒ प्रति॒ प्रत्योष॑धय॒ ओष॑धयः॒ प्रति॑ । \newline
26. प्रति॑ गृह्णीत गृह्णीत॒ प्रति॒ प्रति॑ गृह्णीत । \newline
27. गृ॒ह्णी॒ता॒ग्नि म॒ग्निम् गृ॑ह्णीत गृह्णीता॒ग्निम् । \newline
28. अ॒ग्नि मे॒त मे॒त म॒ग्नि म॒ग्नि मे॒तम् । \newline
29. ए॒त मिती त्ये॒त मे॒त मिति॑ । \newline
30. इत्या॑हा॒हे तीत्या॑ह । \newline
31. आ॒ह॒ शान्त्यै॒ शान्त्या॑ आहाह॒ शान्त्यै᳚ । \newline
32. शान्त्यै॒ व्यस्य॒न् व्यस्य॒ञ् छान्त्यै॒ शान्त्यै॒ व्यस्यन्न्॑ । \newline
33. व्यस्य॒न्॒. विश्वा॒ विश्वा॒ व्यस्य॒न् व्यस्य॒न्॒. विश्वाः᳚ । \newline
34. व्यस्य॒न्निति॑ वि - अस्यन्न्॑ । \newline
35. विश्वा॒ अम॑ती॒ रम॑ती॒र् विश्वा॒ विश्वा॒ अम॑तीः । \newline
36. अम॑ती॒ ररा॑ती॒ ररा॑ती॒ रम॑ती॒ रम॑ती॒ ररा॑तीः । \newline
37. अरा॑ती॒ रिती त्यरा॑ती॒ ररा॑ती॒ रिति॑ । \newline
38. इत्या॑हा॒हे तीत्या॑ह । \newline
39. आ॒ह॒ रक्ष॑साꣳ॒॒ रक्ष॑सा माहाह॒ रक्ष॑साम् । \newline
40. रक्ष॑सा॒ मप॑हत्या॒ अप॑हत्यै॒ रक्ष॑साꣳ॒॒ रक्ष॑सा॒ मप॑हत्यै । \newline
41. अप॑हत्यै नि॒षीद॑न् नि॒षीद॒न् नप॑हत्या॒ अप॑हत्यै नि॒षीदन्न्॑ । \newline
42. अप॑हत्या॒ इत्यप॑ - ह॒त्यै॒ । \newline
43. नि॒षीद॑न् नो नो नि॒षीद॑न् नि॒षीद॑न् नः । \newline
44. नि॒षीद॒न्निति॑ नि - सीदन्न्॑ । \newline
45. नो॒ अपाप॑ नो नो॒ अप॑ । \newline
46. अप॑ दुर्म॒तिम् दु॑र्म॒ति मपाप॑ दुर्म॒तिम् । \newline
47. दु॒र्म॒तिꣳ ह॑नद्धनद् दुर्म॒तिम् दु॑र्म॒तिꣳ ह॑नत् । \newline
48. दु॒र्म॒तिमिति॑ दुः - म॒तिम् । \newline
49. ह॒न॒ दितीति॑ हन द्धन॒ दिति॑ । \newline
50. इत्या॑हा॒हे तीत्या॑ह । \newline
51. आ॒ह॒ प्रति॑ष्ठित्यै॒ प्रति॑ष्ठित्या आहाह॒ प्रति॑ष्ठित्यै । \newline
52. प्रति॑ष्ठित्या॒ ओष॑धय॒ ओष॑धयः॒ प्रति॑ष्ठित्यै॒ प्रति॑ष्ठित्या॒ ओष॑धयः । \newline
53. प्रति॑ष्ठित्या॒ इति॒ प्रति॑ - स्थि॒त्यै॒ । \newline
54. ओष॑धयः॒ प्रति॒ प्रत्योष॑धय॒ ओष॑धयः॒ प्रति॑ । \newline
55. प्रति॑ मोदद्ध्वम् मोदद्ध्व॒म् प्रति॒ प्रति॑ मोदद्ध्वम् । \newline
56. मो॒द॒द्ध्व॒ मे॒न॒ मे॒न॒म् मो॒द॒द्ध्व॒म् मो॒द॒द्ध्व॒ मे॒न॒म् । \newline

\textbf{Ghana Paata } \newline

1. स॒त्य म॒नयो॑ र॒नयोः᳚ स॒त्यꣳ स॒त्य म॒नयो॑ रे॒वैवानयोः᳚ स॒त्यꣳ स॒त्य म॒नयो॑ रे॒व । \newline
2. अ॒नयो॑ रे॒वैवानयो॑ र॒नयो॑ रे॒वैन॑ मेन मे॒वानयो॑ र॒नयो॑ रे॒वैन᳚म् । \newline
3. ए॒वैन॑ मेन मे॒वैवैन॒म् प्रति॒ प्रत्ये॑न मे॒वैवैन॒म् प्रति॑ । \newline
4. ए॒न॒म् प्रति॒ प्रत्ये॑न मेन॒म् प्रति॑ ष्ठापयति स्थापयति॒ प्रत्ये॑न मेन॒म् प्रति॑ ष्ठापयति । \newline
5. प्रति॑ ष्ठापयति स्थापयति॒ प्रति॒ प्रति॑ ष्ठापयति॒ न न स्था॑पयति॒ प्रति॒ प्रति॑ ष्ठापयति॒ न । \newline
6. स्था॒प॒य॒ति॒ न न स्था॑पयति स्थापयति॒ नार्ति॒ मार्ति॒म् न स्था॑पयति स्थापयति॒ नार्ति᳚म् । \newline
7. नार्ति॒ मार्ति॒म् न नार्ति॒ मा ऽऽर्ति॒म् न नार्ति॒ मा । \newline
8. आर्ति॒ मा ऽऽर्ति॒ मार्ति॒ मार्च्छ॑ त्यृच्छ॒त्या ऽऽर्ति॒ मार्ति॒ मार्च्छ॑ति । \newline
9. आर्च्छ॑ त्यृच्छ त्यार्च्छ् अत्यद्ध्व॒र्यु र॑द्ध्व॒र्युर्. ऋ॑च्छ त्यार्च्छ त्यद्ध्व॒र्युः । \newline
10. ऋ॒च्छ॒ त्य॒द्ध्व॒र्यु र॑द्ध्व॒र्युर्. ऋ॑च्छ त्यृच्छ त्यद्ध्व॒र्युर् न नाद्ध्व॒र्युर्. ऋ॑च्छ त्यृच्छ त्यद्ध्व॒र्युर् न । \newline
11. अ॒द्ध्व॒र्युर् न नाद्ध्व॒र्यु र॑द्ध्व॒र्युर् न यज॑मानो॒ यज॑मानो॒ नाद्ध्व॒र्यु र॑द्ध्व॒र्युर् न यज॑मानः । \newline
12. न यज॑मानो॒ यज॑मानो॒ न न यज॑मानो॒ वरु॑णो॒ वरु॑णो॒ यज॑मानो॒ न न यज॑मानो॒ वरु॑णः । \newline
13. यज॑मानो॒ वरु॑णो॒ वरु॑णो॒ यज॑मानो॒ यज॑मानो॒ वरु॑णो॒ वै वै वरु॑णो॒ यज॑मानो॒ यज॑मानो॒ वरु॑णो॒ वै । \newline
14. वरु॑णो॒ वै वै वरु॑णो॒ वरु॑णो॒ वा ए॒ष ए॒ष वै वरु॑णो॒ वरु॑णो॒ वा ए॒षः । \newline
15. वा ए॒ष ए॒ष वै वा ए॒ष यज॑मानं॒ ॅयज॑मान मे॒ष वै वा ए॒ष यज॑मानम् । \newline
16. ए॒ष यज॑मानं॒ ॅयज॑मान मे॒ष ए॒ष यज॑मान म॒भ्य॑भि यज॑मान मे॒ष ए॒ष यज॑मान म॒भि । \newline
17. यज॑मान म॒भ्य॑भि यज॑मानं॒ ॅयज॑मान म॒भ्या ऽभि यज॑मानं॒ ॅयज॑मान म॒भ्या । \newline
18. अ॒भ्या ऽभ्य॑भ्यै त्ये॒त्या ऽभ्य॑भ्यैति॑ । \newline
19. ऐत्ये॒ त्यैति॒ यद् यदे॒ त्यैति॒ यत् । \newline
20. ए॒ति॒ यद् यदे᳚ त्येति॒ यद॒ग्नि र॒ग्निर् यदे᳚ त्येति॒ यद॒ग्निः । \newline
21. यद॒ग्नि र॒ग्निर् यद् यद॒ग्नि रुप॑नद्ध॒ उप॑नद्धो॒ ऽग्निर् यद् यद॒ग्नि रुप॑नद्धः । \newline
22. अ॒ग्नि रुप॑नद्ध॒ उप॑नद्धो॒ ऽग्नि र॒ग्नि रुप॑नद्ध॒ ओष॑धय॒ ओष॑धय॒ उप॑नद्धो॒ ऽग्नि र॒ग्नि रुप॑नद्ध॒ ओष॑धयः । \newline
23. उप॑नद्ध॒ ओष॑धय॒ ओष॑धय॒ उप॑नद्ध॒ उप॑नद्ध॒ ओष॑धयः॒ प्रति॒ प्रत्योष॑धय॒ उप॑नद्ध॒ उप॑नद्ध॒ ओष॑धयः॒ प्रति॑ । \newline
24. उप॑नद्ध॒ इत्युप॑ - न॒द्धः॒ । \newline
25. ओष॑धयः॒ प्रति॒ प्रत्योष॑धय॒ ओष॑धयः॒ प्रति॑ गृह्णीत गृह्णीत॒ प्रत्योष॑धय॒ ओष॑धयः॒ प्रति॑ गृह्णीत । \newline
26. प्रति॑ गृह्णीत गृह्णीत॒ प्रति॒ प्रति॑ गृह्णीता॒ग्नि म॒ग्निम् गृ॑ह्णीत॒ प्रति॒ प्रति॑ गृह्णीता॒ग्निम् । \newline
27. गृ॒ह्णी॒ता॒ग्नि म॒ग्निम् गृ॑ह्णीत गृह्णीता॒ग्नि मे॒त मे॒त म॒ग्निम् गृ॑ह्णीत गृह्णीता॒ग्नि मे॒तम् । \newline
28. अ॒ग्नि मे॒त मे॒त म॒ग्नि म॒ग्नि मे॒त मिती त्ये॒त म॒ग्नि म॒ग्नि मे॒त मिति॑ । \newline
29. ए॒त मिती त्ये॒त मे॒त मित्या॑हा॒हे त्ये॒त मे॒त मित्या॑ह । \newline
30. इत्या॑हा॒हे तीत्या॑ह॒ शान्त्यै॒ शान्त्या॑ आ॒हे तीत्या॑ह॒ शान्त्यै᳚ । \newline
31. आ॒ह॒ शान्त्यै॒ शान्त्या॑ आहाह॒ शान्त्यै॒ व्यस्य॒न् व्यस्य॒ञ् छान्त्या॑ आहाह॒ शान्त्यै॒ व्यस्यन्न्॑ । \newline
32. शान्त्यै॒ व्यस्य॒न् व्यस्य॒ञ् छान्त्यै॒ शान्त्यै॒ व्यस्य॒न्॒. विश्वा॒ विश्वा॒ व्यस्य॒ञ् छान्त्यै॒ शान्त्यै॒ व्यस्य॒न्॒. विश्वाः᳚ । \newline
33. व्यस्य॒न्॒. विश्वा॒ विश्वा॒ व्यस्य॒न् व्यस्य॒न्॒. विश्वा॒ अम॑ती॒ रम॑ती॒र् विश्वा॒ व्यस्य॒न् व्यस्य॒न्॒. विश्वा॒ अम॑तीः । \newline
34. व्यस्य॒न्निति॑ वि - अस्यन्न्॑ । \newline
35. विश्वा॒ अम॑ती॒ रम॑ती॒र् विश्वा॒ विश्वा॒ अम॑ती॒ ररा॑ती॒ ररा॑ती॒ रम॑ती॒र् विश्वा॒ विश्वा॒ अम॑ती॒ ररा॑तीः । \newline
36. अम॑ती॒ ररा॑ती॒ ररा॑ती॒ रम॑ती॒ रम॑ती॒ ररा॑ती॒रिती त्यरा॑ती॒ रम॑ती॒ रम॑ती॒ ररा॑ती॒रिति॑ । \newline
37. अरा॑ती॒रिती त्यरा॑ती॒ ररा॑ती॒रि त्या॑हा॒हे त्यरा॑ती॒ ररा॑ती॒ रित्या॑ह । \newline
38. इत्या॑हा॒हे तीत्या॑ह॒ रक्ष॑साꣳ॒॒ रक्ष॑सा मा॒हे तीत्या॑ह॒ रक्ष॑साम् । \newline
39. आ॒ह॒ रक्ष॑साꣳ॒॒ रक्ष॑सा माहाह॒ रक्ष॑सा॒ मप॑हत्या॒ अप॑हत्यै॒ रक्ष॑सा माहाह॒ रक्ष॑सा॒ मप॑हत्यै । \newline
40. रक्ष॑सा॒ मप॑हत्या॒ अप॑हत्यै॒ रक्ष॑साꣳ॒॒ रक्ष॑सा॒ मप॑हत्यै नि॒षीद॑न् नि॒षीद॒न् नप॑हत्यै॒ रक्ष॑साꣳ॒॒ रक्ष॑सा॒ मप॑हत्यै नि॒षीदन्न्॑ । \newline
41. अप॑हत्यै नि॒षीद॑न् नि॒षीद॒न् नप॑हत्या॒ अप॑हत्यै नि॒षीद॑न् नो नो नि॒षीद॒न् नप॑हत्या॒ अप॑हत्यै नि॒षीद॑न् नः । \newline
42. अप॑हत्या॒ इत्यप॑ - ह॒त्यै॒ । \newline
43. नि॒षीद॑न् नो नो नि॒षीद॑न् नि॒षीद॑न् नो॒ अपाप॑ नो नि॒षीद॑न् नि॒षीद॑न् नो॒ अप॑ । \newline
44. नि॒षीद॒न्निति॑ नि - सीदन्न्॑ । \newline
45. नो॒ अपाप॑ नो नो॒ अप॑ दुर्म॒तिम् दु॑र्म॒ति मप॑ नो नो॒ अप॑ दुर्म॒तिम् । \newline
46. अप॑ दुर्म॒तिम् दु॑र्म॒ति मपाप॑ दुर्म॒तिꣳ ह॑नद्धनद् दुर्म॒ति मपाप॑ दुर्म॒तिꣳ ह॑नत् । \newline
47. दु॒र्म॒तिꣳ ह॑नद्धनद् दुर्म॒तिम् दु॑र्म॒तिꣳ ह॑न॒दितीति॑ हनद् दुर्म॒तिम् दु॑र्म॒तिꣳ ह॑न॒दिति॑ । \newline
48. दु॒र्म॒तिमिति॑ दुः - म॒तिम् । \newline
49. ह॒न॒ दितीति॑ हन द्धन॒ दित्या॑हा॒हेति॑ हन द्धन॒ दित्या॑ह । \newline
50. इत्या॑हा॒हे तीत्या॑ह॒ प्रति॑ष्ठित्यै॒ प्रति॑ष्ठित्या आ॒हे तीत्या॑ह॒ प्रति॑ष्ठित्यै । \newline
51. आ॒ह॒ प्रति॑ष्ठित्यै॒ प्रति॑ष्ठित्या आहाह॒ प्रति॑ष्ठित्या॒ ओष॑धय॒ ओष॑धयः॒ प्रति॑ष्ठित्या आहाह॒ प्रति॑ष्ठित्या॒ ओष॑धयः । \newline
52. प्रति॑ष्ठित्या॒ ओष॑धय॒ ओष॑धयः॒ प्रति॑ष्ठित्यै॒ प्रति॑ष्ठित्या॒ ओष॑धयः॒ प्रति॒ प्रत्योष॑धयः॒ प्रति॑ष्ठित्यै॒ प्रति॑ष्ठित्या॒ ओष॑धयः॒ प्रति॑ । \newline
53. प्रति॑ष्ठित्या॒ इति॒ प्रति॑ - स्थि॒त्यै॒ । \newline
54. ओष॑धयः॒ प्रति॒ प्रत्योष॑धय॒ ओष॑धयः॒ प्रति॑ मोदद्ध्वम् मोदद्ध्व॒म् प्रत्योष॑धय॒ ओष॑धयः॒ प्रति॑ मोदद्ध्वम् । \newline
55. प्रति॑ मोदद्ध्वम् मोदद्ध्व॒म् प्रति॒ प्रति॑ मोदद्ध्व मेन मेनम् मोदद्ध्व॒म् प्रति॒ प्रति॑ मोदद्ध्व मेनम् । \newline
56. मो॒द॒द्ध्व॒ मे॒न॒ मे॒न॒म् मो॒द॒द्ध्व॒म् मो॒द॒द्ध्व॒ मे॒न॒ मिती त्ये॑नम् मोदद्ध्वम् मोदद्ध्व मेन॒ मिति॑ । \newline
\pagebreak
\markright{ TS 5.1.5.10  \hfill https://www.vedavms.in \hfill}

\section{ TS 5.1.5.10 }

\textbf{TS 5.1.5.10 } \newline
\textbf{Samhita Paata} \newline

मेन॒मित्या॒हौष॑धयो॒ वा अ॒ग्नेर्भा॑ग॒धेयं॒ ताभि॑रे॒वैनꣳ॒॒ सम॑र्द्धयति॒ पुष्पा॑वतीः सुपिप्प॒ला इत्या॑ह॒ तस्मा॒दोष॑धयः॒ फलं॑ गृह्णन्त्य॒ यं ॅवो॒ गर्भ॑ ऋ॒त्वियः॑ प्र॒त्नꣳ स॒धस्थ॒मा-ऽस॑द॒दित्या॑ह॒ याभ्य॑ ए॒वैनं॑ प्रच्या॒वय॑ति॒ तास्वे॒वैनं॒ प्रति॑ष्ठापयति॒ द्वाभ्या॑मु॒पाव॑हरति॒ प्रति॑ष्ठित्यै ॥ \newline

\textbf{Pada Paata} \newline

ए॒न॒म् । इति॑ । आ॒ह॒ । ओष॑धयः । वै । अ॒ग्नेः । भा॒ग॒धेय॒मिति॑ भाग - धेय᳚म् । ताभिः॑ । ए॒व । ए॒न॒म् । समिति॑ । अ॒द्‌र्ध॒य॒ति॒ । पुष्पा॑वती॒रिति॒ पुष्प॑ - व॒तीः॒ । सु॒पि॒प्प॒ला इति॑ सु-पि॒प्प॒लाः । इति॑ । आ॒ह॒ । तस्मा᳚त् । ओष॑धयः । फल᳚म् । गृ॒ह्ण॒न्ति॒ । अ॒यम् । वः॒ । गर्भः॑ । ऋ॒त्वियः॑ । प्र॒त्नम् । स॒धस्थ॒मिति॑ स॒ध - स्थ॒म् । एति॑ । अ॒स॒द॒त् । इति॑ । आ॒ह॒ । याभ्यः॑ । ए॒व । ए॒न॒म् । प्र॒च्या॒वय॒तीति॑ प्र - च्या॒वय॑ति । तासु॑ । ए॒व । ए॒न॒म् । प्रतीति॑ । स्था॒प॒य॒ति॒ । द्वाभ्या᳚म् । उ॒पाव॑हर॒तीत्यु॑प - अव॑हरति । प्रति॑ष्ठित्या॒ इति॒ प्रति॑ - स्थि॒त्यै॒ ॥  \newline


\textbf{Krama Paata} \newline

ए॒न॒मिति॑ । इत्या॑ह । आ॒हौष॑धयः । ओष॑धयो॒ वै । वा अ॒ग्नेः । अ॒ग्नेर् भा॑ग॒धेय᳚म् । भा॒ग॒धेय॒म् ताभिः॑ । भा॒ग॒धेय॒मिति॑ भाग - धेय᳚म् । ताभि॑रे॒व । ए॒वैन᳚म् । ए॒नꣳ॒॒ सम् । सम॑र्द्धयति । अ॒र्द्ध॒य॒ति॒ पुष्पा॑वतीः । पुष्पा॑वतीः सुपिप्प॒लाः । पुष्पा॑वती॒रिति॒ पुष्प॑ - व॒तीः॒ । सु॒पि॒प्प॒ला इति॑ । सु॒पि॒प्प॒ला इति॑ सु - पि॒प्प॒लाः । इत्या॑ह । आ॒ह॒ तस्मा᳚त् । तस्मा॒दोष॑धयः । ओष॑धयः॒ फल᳚म् । फल॑म् गृह्णन्ति । गृ॒ह्ण॒न्त्य॒यम् । अ॒यम् ॅवः॑ । वो॒ गर्भः॑ । गर्भ॑ ऋ॒त्वियः॑ । ऋ॒त्वियः॑ प्र॒त्नम् । प्र॒त्नꣳ स॒धस्थ᳚म् । स॒धस्थ॒मा । स॒धस्थ॒मिति॑ स॒ध - स्थ॒म् । आऽस॑दत् । अ॒स॒द॒दिति॑ । इत्या॑ह । आ॒ह॒ याभ्यः॑ । याभ्य॑ ए॒व । ए॒वैन᳚म् । ए॒न॒म् प्र॒च्या॒वय॑ति । प्र॒च्या॒वय॑ति॒ तासु॑ । प्र॒च्या॒वय॒तीति॑ प्र - च्या॒वय॑ति । तास्वे॒व । ए॒वैन᳚म् । ए॒न॒म् प्रति॑ । प्रति॑ ष्ठापयति । स्था॒प॒य॒ति॒ द्वाभ्या᳚म् । द्वाभ्या॑मु॒पाव॑हरति । उ॒पाव॑हरति॒ प्रति॑ष्ठित्यै । उ॒पाव॑हर॒तीत्यु॑प - अव॑हरति । प्रति॑ष्ठित्या॒ इति॒ प्रति॑ - स्थि॒त्यै॒ । \newline

\textbf{Jatai Paata} \newline

1. ए॒न॒ मितीत्ये॑न मेन॒ मिति॑ । \newline
2. इत्या॑हा॒हे तीत्या॑ह । \newline
3. आ॒हौष॑धय॒ ओष॑धय आहा॒ हौष॑धयः । \newline
4. ओष॑धयो॒ वै वा ओष॑धय॒ ओष॑धयो॒ वै । \newline
5. वा अ॒ग्ने र॒ग्नेर् वै वा अ॒ग्नेः । \newline
6. अ॒ग्नेर् भा॑ग॒धेय॑म् भाग॒धेय॑ म॒ग्ने र॒ग्नेर् भा॑ग॒धेय᳚म् । \newline
7. भा॒ग॒धेय॒म् ताभि॒ स्ताभि॑र् भाग॒धेय॑म् भाग॒धेय॒म् ताभिः॑ । \newline
8. भा॒ग॒धेय॒मिति॑ भाग - धेय᳚म् । \newline
9. ताभि॑ रे॒वैव ताभि॒ स्ताभि॑ रे॒व । \newline
10. ए॒वैन॑ मेन मे॒वैवैन᳚म् । \newline
11. ए॒नꣳ॒॒ सꣳ स मे॑न मेनꣳ॒॒ सम् । \newline
12. स म॑र्द्धय त्यर्द्धयति॒ सꣳ स म॑र्द्धयति । \newline
13. अ॒र्द्ध॒य॒ति॒ पुष्पा॑वतीः॒ पुष्पा॑वती रर्द्धय त्यर्द्धयति॒ पुष्पा॑वतीः । \newline
14. पुष्पा॑वतीः सुपिप्प॒लाः सु॑पिप्प॒लाः पुष्पा॑वतीः॒ पुष्पा॑वतीः सुपिप्प॒लाः । \newline
15. पुष्पा॑वती॒रिति॒ पुष्प॑ - व॒तीः॒ । \newline
16. सु॒पि॒प्प॒ला इतीति॑ सुपिप्प॒लाः सु॑पिप्प॒ला इति॑ । \newline
17. सु॒पि॒प्प॒ला इति॑ सु - पि॒प्प॒लाः । \newline
18. इत्या॑हा॒हे तीत्या॑ह । \newline
19. आ॒ह॒ तस्मा॒त् तस्मा॑ दाहाह॒ तस्मा᳚त् । \newline
20. तस्मा॒ दोष॑धय॒ ओष॑धय॒ स्तस्मा॒त् तस्मा॒ दोष॑धयः । \newline
21. ओष॑धयः॒ फल॒म् फल॒ मोष॑धय॒ ओष॑धयः॒ फल᳚म् । \newline
22. फल॑म् गृह्णन्ति गृह्णन्ति॒ फल॒म् फल॑म् गृह्णन्ति । \newline
23. गृ॒ह्ण॒ न्त्य॒य म॒यम् गृ॑ह्णन्ति गृह्ण न्त्य॒यम् । \newline
24. अ॒यं ॅवो॑ वो॒ ऽय म॒यं ॅवः॑ । \newline
25. वो॒ गर्भो॒ गर्भो॑ वो वो॒ गर्भः॑ । \newline
26. गर्भ॑ ऋ॒त्विय॑ ऋ॒त्वियो॒ गर्भो॒ गर्भ॑ ऋ॒त्वियः॑ । \newline
27. ऋ॒त्वियः॑ प्र॒त्नम् प्र॒त्न मृ॒त्विय॑ ऋ॒त्वियः॑ प्र॒त्नम् । \newline
28. प्र॒त्नꣳ स॒धस्थꣳ॑ स॒धस्थ॑म् प्र॒त्नम् प्र॒त्नꣳ स॒धस्थ᳚म् । \newline
29. स॒धस्थ॒ मा स॒धस्थꣳ॑ स॒धस्थ॒ मा । \newline
30. स॒धस्थ॒मिति॑ स॒ध - स्थ॒म् । \newline
31. आ ऽस॑द दसद॒दा ऽस॑दत् । \newline
32. अ॒स॒द॒ दिती त्य॑सद दसद॒ दिति॑ । \newline
33. इत्या॑हा॒हे तीत्या॑ह । \newline
34. आ॒ह॒ याभ्यो॒ याभ्य॑ आहाह॒ याभ्यः॑ । \newline
35. याभ्य॑ ए॒वैव याभ्यो॒ याभ्य॑ ए॒व । \newline
36. ए॒वैन॑ मेन मे॒वैवैन᳚म् । \newline
37. ए॒न॒म् प्र॒च्या॒वय॑ति प्रच्या॒वय॑ त्येन मेनम् प्रच्या॒वय॑ति । \newline
38. प्र॒च्या॒वय॑ति॒ तासु॒ तासु॑ प्रच्या॒वय॑ति प्रच्या॒वय॑ति॒ तासु॑ । \newline
39. प्र॒च्या॒वय॒तीति॑ प्र - च्या॒वय॑ति । \newline
40. तास्वे॒वैव तासु॒ तास्वे॒व । \newline
41. ए॒वैन॑ मेन मे॒वैवैन᳚म् । \newline
42. ए॒न॒म् प्रति॒ प्रत्ये॑न मेन॒म् प्रति॑ । \newline
43. प्रति॑ ष्ठापयति स्थापयति॒ प्रति॒ प्रति॑ ष्ठापयति । \newline
44. स्था॒प॒य॒ति॒ द्वाभ्या॒म् द्वाभ्याꣳ॑ स्थापयति स्थापयति॒ द्वाभ्या᳚म् । \newline
45. द्वाभ्या॑ मु॒पाव॑हर त्यु॒पाव॑हरति॒ द्वाभ्या॒म् द्वाभ्या॑ मु॒पाव॑हरति । \newline
46. उ॒पाव॑हरति॒ प्रति॑ष्ठित्यै॒ प्रति॑ष्ठित्या उ॒पाव॑हर त्यु॒पाव॑हरति॒ प्रति॑ष्ठित्यै । \newline
47. उ॒पाव॑हर॒तीत्यु॑प - अव॑हरति । \newline
48. प्रति॑ष्ठित्या॒ इति॒ प्रति॑ - स्थि॒त्यै॒ । \newline

\textbf{Ghana Paata } \newline

1. ए॒न॒ मितीत्ये॑न मेन॒ मित्या॑हा॒हे त्ये॑न मेन॒ मित्या॑ह । \newline
2. इत्या॑हा॒हे तीत्या॒हौष॑धय॒ ओष॑धय आ॒हे तीत्या॒हौष॑धयः । \newline
3. आ॒हौष॑धय॒ ओष॑धय आहा॒ हौष॑धयो॒ वै वा ओष॑धय आहा॒हौ ष॑धयो॒ वै । \newline
4. ओष॑धयो॒ वै वा ओष॑धय॒ ओष॑धयो॒ वा अ॒ग्ने र॒ग्नेर् वा ओष॑धय॒ ओष॑धयो॒ वा अ॒ग्नेः । \newline
5. वा अ॒ग्ने र॒ग्नेर् वै वा अ॒ग्नेर् भा॑ग॒धेय॑म् भाग॒धेय॑ म॒ग्नेर् वै वा अ॒ग्नेर् भा॑ग॒धेय᳚म् । \newline
6. अ॒ग्नेर् भा॑ग॒धेय॑म् भाग॒धेय॑ म॒ग्ने र॒ग्नेर् भा॑ग॒धेय॒म् ताभि॒ स्ताभि॑र् भाग॒धेय॑ म॒ग्ने र॒ग्नेर् भा॑ग॒धेय॒म् ताभिः॑ । \newline
7. भा॒ग॒धेय॒म् ताभि॒ स्ताभि॑र् भाग॒धेय॑म् भाग॒धेय॒म् ताभि॑ रे॒वैव ताभि॑र् भाग॒धेय॑म् भाग॒धेय॒म् ताभि॑रे॒व । \newline
8. भा॒ग॒धेय॒मिति॑ भाग - धेय᳚म् । \newline
9. ताभि॑ रे॒वैव ताभि॒ स्ताभि॑ रे॒वैन॑ मेन मे॒व ताभि॒ स्ताभि॑ रे॒वैन᳚म् । \newline
10. ए॒वैन॑ मेन मे॒वैवैनꣳ॒॒ सꣳ स मे॑न मे॒वैवैनꣳ॒॒ सम् । \newline
11. ए॒नꣳ॒॒ सꣳ स मे॑न मेनꣳ॒॒ स म॑र्द्धय त्यर्द्धयति॒ स मे॑न मेनꣳ॒॒ स म॑र्द्धयति । \newline
12. स म॑र्द्धय त्यर्द्धयति॒ सꣳ स म॑र्द्धयति॒ पुष्पा॑वतीः॒ पुष्पा॑वती रर्द्धयति॒ सꣳ स म॑र्द्धयति॒ पुष्पा॑वतीः । \newline
13. अ॒र्द्ध॒य॒ति॒ पुष्पा॑वतीः॒ पुष्पा॑वती रर्द्धय त्यर्द्धयति॒ पुष्पा॑वतीः सुपिप्प॒लाः सु॑पिप्प॒लाः पुष्पा॑वती रर्द्धय त्यर्द्धयति॒ पुष्पा॑वतीः सुपिप्प॒लाः । \newline
14. पुष्पा॑वतीः सुपिप्प॒लाः सु॑पिप्प॒लाः पुष्पा॑वतीः॒ पुष्पा॑वतीः सुपिप्प॒ला इतीति॑ सुपिप्प॒लाः पुष्पा॑वतीः॒ पुष्पा॑वतीः सुपिप्प॒ला इति॑ । \newline
15. पुष्पा॑वती॒रिति॒ पुष्प॑ - व॒तीः॒ । \newline
16. सु॒पि॒प्प॒ला इतीति॑ सुपिप्प॒लाः सु॑पिप्प॒ला इत्या॑हा॒हेति॑ सुपिप्प॒लाः सु॑पिप्प॒ला इत्या॑ह । \newline
17. सु॒पि॒प्प॒ला इति॑ सु - पि॒प्प॒लाः । \newline
18. इत्या॑हा॒हे तीत्या॑ह॒ तस्मा॒त् तस्मा॑ दा॒हे तीत्या॑ह॒ तस्मा᳚त् । \newline
19. आ॒ह॒ तस्मा॒त् तस्मा॑ दाहाह॒ तस्मा॒ दोष॑धय॒ ओष॑धय॒ स्तस्मा॑ दाहाह॒ तस्मा॒ दोष॑धयः । \newline
20. तस्मा॒ दोष॑धय॒ ओष॑धय॒ स्तस्मा॒त् तस्मा॒ दोष॑धयः॒ फल॒म् फल॒ मोष॑धय॒ स्तस्मा॒त् तस्मा॒ दोष॑धयः॒ फल᳚म् । \newline
21. ओष॑धयः॒ फल॒म् फल॒ मोष॑धय॒ ओष॑धयः॒ फल॑म् गृह्णन्ति गृह्णन्ति॒ फल॒ मोष॑धय॒ ओष॑धयः॒ फल॑म् गृह्णन्ति । \newline
22. फल॑म् गृह्णन्ति गृह्णन्ति॒ फल॒म् फल॑म् गृह्णन्त्य॒य म॒यम् गृ॑ह्णन्ति॒ फल॒म् फल॑म् गृह्णन्त्य॒यम् । \newline
23. गृ॒ह्ण॒न्त्य॒य म॒यम् गृ॑ह्णन्ति गृह्णन्त्य॒यं ॅवो॑ वो॒ ऽयम् गृ॑ह्णन्ति गृह्णन्त्य॒यं ॅवः॑ । \newline
24. अ॒यं ॅवो॑ वो॒ ऽय म॒यं ॅवो॒ गर्भो॒ गर्भो॑ वो॒ ऽय म॒यं ॅवो॒ गर्भः॑ । \newline
25. वो॒ गर्भो॒ गर्भो॑ वो वो॒ गर्भ॑ ऋ॒त्विय॑ ऋ॒त्वियो॒ गर्भो॑ वो वो॒ गर्भ॑ ऋ॒त्वियः॑ । \newline
26. गर्भ॑ ऋ॒त्विय॑ ऋ॒त्वियो॒ गर्भो॒ गर्भ॑ ऋ॒त्वियः॑ प्र॒त्नम् प्र॒त्न मृ॒त्वियो॒ गर्भो॒ गर्भ॑ ऋ॒त्वियः॑ प्र॒त्नम् । \newline
27. ऋ॒त्वियः॑ प्र॒त्नम् प्र॒त्न मृ॒त्विय॑ ऋ॒त्वियः॑ प्र॒त्नꣳ स॒धस्थꣳ॑ स॒धस्थ॑म् प्र॒त्न मृ॒त्विय॑ ऋ॒त्वियः॑ प्र॒त्नꣳ स॒धस्थ᳚म् । \newline
28. प्र॒त्नꣳ स॒धस्थꣳ॑ स॒धस्थ॑म् प्र॒त्नम् प्र॒त्नꣳ स॒धस्थ॒ मा स॒धस्थ॑म् प्र॒त्नम् प्र॒त्नꣳ स॒धस्थ॒ मा । \newline
29. स॒धस्थ॒ मा स॒धस्थꣳ॑ स॒धस्थ॒ मा ऽस॑द दसद॒दा स॒धस्थꣳ॑ स॒धस्थ॒ मा ऽस॑दत् । \newline
30. स॒धस्थ॒मिति॑ स॒ध - स्थ॒म् । \newline
31. आ ऽस॑द दसद॒दा ऽस॑द॒दिती त्य॑सद॒दा ऽस॑द॒दिति॑ । \newline
32. अ॒स॒द॒दिती त्य॑सद दस द॒दित्या॑हा॒हे त्य॑सद दस द॒दित्या॑ह । \newline
33. इत्या॑हा॒हे तीत्या॑ह॒ याभ्यो॒ याभ्य॑ आ॒हे तीत्या॑ह॒ याभ्यः॑ । \newline
34. आ॒ह॒ याभ्यो॒ याभ्य॑ आहाह॒ याभ्य॑ ए॒वैव याभ्य॑ आहाह॒ याभ्य॑ ए॒व । \newline
35. याभ्य॑ ए॒वैव याभ्यो॒ याभ्य॑ ए॒वैन॑ मेन मे॒व याभ्यो॒ याभ्य॑ ए॒वैन᳚म् । \newline
36. ए॒वैन॑ मेन मे॒वैवैन॑म् प्रच्या॒वय॑ति प्रच्या॒वय॑ त्येन मे॒वैवैन॑म् प्रच्या॒वय॑ति । \newline
37. ए॒न॒म् प्र॒च्या॒वय॑ति प्रच्या॒वय॑ त्येन मेनम् प्रच्या॒वय॑ति॒ तासु॒ तासु॑ प्रच्या॒वय॑ त्येन मेनम् प्रच्या॒वय॑ति॒ तासु॑ । \newline
38. प्र॒च्या॒वय॑ति॒ तासु॒ तासु॑ प्रच्या॒वय॑ति प्रच्या॒वय॑ति॒ तास्वे॒वैव तासु॑ प्रच्या॒वय॑ति प्रच्या॒वय॑ति॒ तास्वे॒व । \newline
39. प्र॒च्या॒वय॒तीति॑ प्र - च्या॒वय॑ति । \newline
40. तास्वे॒वैव तासु॒ तास्वे॒वैन॑ मेन मे॒व तासु॒ तास्वे॒वैन᳚म् । \newline
41. ए॒वैन॑ मेन मे॒वैवैन॒म् प्रति॒ प्रत्ये॑न मे॒वैवैन॒म् प्रति॑ । \newline
42. ए॒न॒म् प्रति॒ प्रत्ये॑न मेन॒म् प्रति॑ ष्ठापयति स्थापयति॒ प्रत्ये॑न मेन॒म् प्रति॑ ष्ठापयति । \newline
43. प्रति॑ ष्ठापयति स्थापयति॒ प्रति॒ प्रति॑ ष्ठापयति॒ द्वाभ्या॒म् द्वाभ्याꣳ॑ स्थापयति॒ प्रति॒ प्रति॑ ष्ठापयति॒ द्वाभ्या᳚म् । \newline
44. स्था॒प॒य॒ति॒ द्वाभ्या॒म् द्वाभ्याꣳ॑ स्थापयति स्थापयति॒ द्वाभ्या॑ मु॒पाव॑हर त्यु॒पाव॑हरति॒ द्वाभ्याꣳ॑ स्थापयति स्थापयति॒ द्वाभ्या॑ मु॒पाव॑हरति । \newline
45. द्वाभ्या॑ मु॒पाव॑हर त्यु॒पाव॑हरति॒ द्वाभ्या॒म् द्वाभ्या॑ मु॒पाव॑हरति॒ प्रति॑ष्ठित्यै॒ प्रति॑ष्ठित्या उ॒पाव॑हरति॒ द्वाभ्या॒म् द्वाभ्या॑ मु॒पाव॑हरति॒ प्रति॑ष्ठित्यै । \newline
46. उ॒पाव॑हरति॒ प्रति॑ष्ठित्यै॒ प्रति॑ष्ठित्या उ॒पाव॑हर त्यु॒पाव॑हरति॒ प्रति॑ष्ठित्यै । \newline
47. उ॒पाव॑हर॒तीत्यु॑प - अव॑हरति । \newline
48. प्रति॑ष्ठित्या॒ इति॒ प्रति॑ - स्थि॒त्यै॒ । \newline
\pagebreak
\markright{ TS 5.1.6.1  \hfill https://www.vedavms.in \hfill}

\section{ TS 5.1.6.1 }

\textbf{TS 5.1.6.1 } \newline
\textbf{Samhita Paata} \newline

वा॒रु॒णो वा अ॒ग्निरुप॑नद्धो॒ वि पाज॒सेति॒ विस्रꣳ॑सयति सवि॒तृप्र॑सूत ए॒वास्य॒ विषू॑चीं ॅवरुणमे॒निं ॅविसृ॑जत्य॒प उप॑ सृज॒त्यापो॒ वै शा॒न्ताः शा॒न्ताभि॑रे॒वास्य॒ शुचꣳ॑ शमयति ति॒सृभि॒रुप॑ सृजति त्रि॒वृद्वा अ॒ग्निर्यावा॑ने॒वा-ग्निस्तस्य॒ शुचꣳ॑ शमयति मि॒त्रः सꣳ॒॒सृज्य॑ पृथि॒वीमित्या॑ह मि॒त्रो वै शि॒वो दे॒वानां॒ तेनै॒वै - [  ] \newline

\textbf{Pada Paata} \newline

वा॒रु॒णः । वै । अ॒ग्निः । उप॑नद्ध॒ इत्युप॑ - न॒द्धः॒ । वीति॑ । पाज॑सा । इति॑ । वीति॑ । स्रꣳ॒॒स॒य॒ति॒ । स॒वि॒तृप्र॑सूत॒ इति॑ सवि॒तृ - प्र॒सू॒तः॒ । ए॒व । अ॒स्य॒ । विषू॑चीम् । व॒रु॒ण॒मे॒निमिति॑ वरुण - मे॒निम् । वीति॑ । सृ॒ज॒ति॒ । अ॒पः । उपेति॑ । सृ॒ज॒ति॒ । आपः॑ । वै । शा॒न्ताः । शा॒न्ताभिः॑ । ए॒व । अ॒स्य॒ । शुच᳚म् । श॒म॒य॒ति॒ । ति॒सृभि॒रिति॑ ति॒सृ - भिः॒ । उपेति॑ । सृ॒ज॒ति॒ । त्रि॒वृदिति॑ त्रि -वृत् । वै । अ॒ग्निः । यावान्॑ । ए॒व । अ॒ग्निः । तस्य॑ । शुच᳚म् । श॒म॒य॒ति॒ । मि॒त्रः । सꣳ॒॒सृज्येति॑ सं - सृज्य॑ । पृ॒थि॒वीम् । इति॑ । आ॒ह॒ । मि॒त्रः । वै । शि॒वः । दे॒वाना᳚म् । तेन॑ । ए॒व ।  \newline


\textbf{Krama Paata} \newline

वा॒रु॒णो वै । वा अ॒ग्निः । अ॒ग्निरुप॑नद्धः । उप॑नद्धो॒ वि । उप॑नद्ध॒ इत्युप॑ - न॒द्धः॒ । वि पाज॑सा । पाज॒सेति॑ । इति॒ वि । वि स्रꣳ॑सयति । स्रꣳ॒॒स॒य॒ति॒ स॒वि॒तृप्र॑सूतः । स॒वि॒तृप्र॑सूत ए॒व । स॒वि॒तृप्र॑सूत॒ इति॑ सवि॒तृ - प्र॒सू॒तः॒ । ए॒वास्य॑ । अ॒स्य॒ विषू॑चीम् । विषू॑चीम् ॅवरुणमे॒निम् । व॒रु॒ण॒मे॒निम् ॅवि । व॒रु॒ण॒मे॒निमिति॑ वरुण - मे॒निम् । वि सृ॑जति । सृ॒ज॒त्य॒पः । अ॒प उप॑ । उप॑ सृजति । सृ॒ज॒त्यापः॑ । आपो॒ वै । वै शा॒न्ताः । शा॒न्ताः शा॒न्ताभिः॑ । शा॒न्ताभि॑रे॒व । ए॒वास्य॑ । अ॒स्य॒ शुच᳚म् । शुचꣳ॑ शमयति । श॒म॒य॒ति॒ ति॒सृभिः॑ । ति॒सृभि॒रुप॑ । ति॒सृभि॒रिति॑ ति॒सृ - भिः॒ । उप॑ सृजति । सृ॒ज॒ति॒ त्रि॒वृत् । त्रि॒वृद् वै । त्रि॒वृदिति॑ त्रि - वृत् । वा अ॒ग्निः । अ॒ग्निर् यावान्॑ । यावा॑ने॒व । ए॒वाग्निः । अ॒ग्निस्तस्य॑ । तस्य॒ शुच᳚म् । शुचꣳ॑ शमयति । श॒म॒य॒ति॒ मि॒त्रः । मि॒त्रः सꣳ॒॒सृज्य॑ । सꣳ॒॒सृज्य॑ पृथि॒वीम् । सꣳ॒॒सृज्येति॑ सम् - सृज्य॑ । पृ॒थि॒वीमिति॑ । इत्या॑ह । आ॒ह॒ मि॒त्रः । मि॒त्रो वै । वै शि॒वः । शि॒वो दे॒वाना᳚म् । दे॒वाना॒म् तेन॑ । तेनै॒व । ए॒वैन᳚म् \newline

\textbf{Jatai Paata} \newline

1. वा॒रु॒णो वै वै वा॑रु॒णो वा॑रु॒णो वै । \newline
2. वा अ॒ग्नि र॒ग्निर् वै वा अ॒ग्निः । \newline
3. अ॒ग्नि रुप॑नद्ध॒ उप॑नद्धो॒ ऽग्नि र॒ग्नि रुप॑नद्धः । \newline
4. उप॑नद्धो॒ वि व्युप॑नद्ध॒ उप॑नद्धो॒ वि । \newline
5. उप॑नद्ध॒ इत्युप॑ - न॒द्धः॒ । \newline
6. वि पाज॑सा॒ पाज॑सा॒ वि वि पाज॑सा । \newline
7. पाज॒सेतीति॒ पाज॑सा॒ पाज॒सेति॑ । \newline
8. इति॒ वि वीतीति॒ वि । \newline
9. वि स्रꣳ॑सयति स्रꣳसयति॒ वि वि स्रꣳ॑सयति । \newline
10. स्रꣳ॒॒स॒य॒ति॒ स॒वि॒तृप्र॑सूतः सवि॒तृप्र॑सूतः स्रꣳसयति स्रꣳसयति सवि॒तृप्र॑सूतः । \newline
11. स॒वि॒तृप्र॑सूत ए॒वैव स॑वि॒तृप्र॑सूतः सवि॒तृप्र॑सूत ए॒व । \newline
12. स॒वि॒तृप्र॑सूत॒ इति॑ सवि॒तृ - प्र॒सू॒तः॒ । \newline
13. ए॒वा स्या᳚ स्यै॒वै वास्य॑ । \newline
14. अ॒स्य॒ विषू॑चीं॒ ॅविषू॑ची मस्यास्य॒ विषू॑चीम् । \newline
15. विषू॑चीं ॅवरुणमे॒निं ॅव॑रुणमे॒निं ॅविषू॑चीं॒ ॅविषू॑चीं ॅवरुणमे॒निम् । \newline
16. व॒रु॒ण॒मे॒निं ॅवि वि व॑रुणमे॒निं ॅव॑रुणमे॒निं ॅवि । \newline
17. व॒रु॒ण॒मे॒निमिति॑ वरुण - मे॒निम् । \newline
18. वि सृ॑जति सृजति॒ वि वि सृ॑जति । \newline
19. सृ॒ज॒ त्य॒पो॑ ऽपः सृ॑जति सृज त्य॒पः । \newline
20. अ॒प उपोपा॒पो॑ ऽप उप॑ । \newline
21. उप॑ सृजति सृज॒ त्युपोप॑ सृजति । \newline
22. सृ॒ज॒ त्याप॒ आपः॑ सृजति सृज॒ त्यापः॑ । \newline
23. आपो॒ वै वा आप॒ आपो॒ वै । \newline
24. वै शा॒न्ताः शा॒न्ता वै वै शा॒न्ताः । \newline
25. शा॒न्ताः शा॒न्ताभिः॑ शा॒न्ताभिः॑ शा॒न्ताः शा॒न्ताः शा॒न्ताभिः॑ । \newline
26. शा॒न्ताभि॑ रे॒वैव शा॒न्ताभिः॑ शा॒न्ताभि॑ रे॒व । \newline
27. ए॒वास्या᳚ स्यै॒वै वास्य॑ । \newline
28. अ॒स्य॒ शुचꣳ॒॒ शुच॑ मस्या स्य॒ शुच᳚म् । \newline
29. शुचꣳ॑ शमयति शमयति॒ शुचꣳ॒॒ शुचꣳ॑ शमयति । \newline
30. श॒म॒य॒ति॒ ति॒सृभि॑ स्ति॒सृभिः॑ शमयति शमयति ति॒सृभिः॑ । \newline
31. ति॒सृभि॒ रुपोप॑ ति॒सृभि॑ स्ति॒सृभि॒ रुप॑ । \newline
32. ति॒सृभि॒रिति॑ ति॒सृ - भिः॒ । \newline
33. उप॑ सृजति सृज॒ त्युपोप॑ सृजति । \newline
34. सृ॒ज॒ति॒ त्रि॒वृत् त्रि॒वृथ् सृ॑जति सृजति त्रि॒वृत् । \newline
35. त्रि॒वृद् वै वै त्रि॒वृत् त्रि॒वृद् वै । \newline
36. त्रि॒वृदिति॑ त्रि - वृत् । \newline
37. वा अ॒ग्नि र॒ग्निर् वै वा अ॒ग्निः । \newline
38. अ॒ग्निर् यावा॒न्॒. यावा॑ न॒ग्नि र॒ग्निर् यावान्॑ । \newline
39. यावा॑ ने॒वैव यावा॒न्॒. यावा॑ ने॒व । \newline
40. ए॒वाग्नि र॒ग्नि रे॒वैवाग्निः । \newline
41. अ॒ग्नि स्तस्य॒ तस्या॒ग्नि र॒ग्नि स्तस्य॑ । \newline
42. तस्य॒ शुचꣳ॒॒ शुच॒म् तस्य॒ तस्य॒ शुच᳚म् । \newline
43. शुचꣳ॑ शमयति शमयति॒ शुचꣳ॒॒ शुचꣳ॑ शमयति । \newline
44. श॒म॒य॒ति॒ मि॒त्रो मि॒त्रः श॑मयति शमयति मि॒त्रः । \newline
45. मि॒त्रः सꣳ॒॒सृज्य॑ सꣳ॒॒सृज्य॑ मि॒त्रो मि॒त्रः सꣳ॒॒सृज्य॑ । \newline
46. सꣳ॒॒सृज्य॑ पृथि॒वीम् पृ॑थि॒वीꣳ सꣳ॒॒सृज्य॑ सꣳ॒॒सृज्य॑ पृथि॒वीम् । \newline
47. सꣳ॒॒सृज्येति॑ सं - सृज्य॑ । \newline
48. पृ॒थि॒वी मितीति॑ पृथि॒वीम् पृ॑थि॒वी मिति॑ । \newline
49. इत्या॑हा॒हे तीत्या॑ह । \newline
50. आ॒ह॒ मि॒त्रो मि॒त्र आ॑हाह मि॒त्रः । \newline
51. मि॒त्रो वै वै मि॒त्रो मि॒त्रो वै । \newline
52. वै शि॒वः शि॒वो वै वै शि॒वः । \newline
53. शि॒वो दे॒वाना᳚म् दे॒वानाꣳ॑ शि॒वः शि॒वो दे॒वाना᳚म् । \newline
54. दे॒वाना॒म् तेन॒ तेन॑ दे॒वाना᳚म् दे॒वाना॒म् तेन॑ । \newline
55. तेनै॒वैव तेन॒ तेनै॒व । \newline
56. ए॒वैन॑ मेन मे॒वैवैन᳚म् । \newline

\textbf{Ghana Paata } \newline

1. वा॒रु॒णो वै वै वा॑रु॒णो वा॑रु॒णो वा अ॒ग्नि र॒ग्निर् वै वा॑रु॒णो वा॑रु॒णो वा अ॒ग्निः । \newline
2. वा अ॒ग्नि र॒ग्निर् वै वा अ॒ग्नि रुप॑नद्ध॒ उप॑नद्धो॒ ऽग्निर् वै वा अ॒ग्नि रुप॑नद्धः । \newline
3. अ॒ग्नि रुप॑नद्ध॒ उप॑नद्धो॒ ऽग्नि र॒ग्नि रुप॑नद्धो॒ वि व्युप॑नद्धो॒ ऽग्नि र॒ग्नि रुप॑नद्धो॒ वि । \newline
4. उप॑नद्धो॒ वि व्युप॑नद्ध॒ उप॑नद्धो॒ वि पाज॑सा॒ पाज॑सा॒ व्युप॑नद्ध॒ उप॑नद्धो॒ वि पाज॑सा । \newline
5. उप॑नद्ध॒ इत्युप॑ - न॒द्धः॒ । \newline
6. वि पाज॑सा॒ पाज॑सा॒ वि वि पाज॒सेतीति॒ पाज॑सा॒ वि वि पाज॒सेति॑ । \newline
7. पाज॒सेतीति॒ पाज॑सा॒ पाज॒सेति॒ वि वीति॒ पाज॑सा॒ पाज॒सेति॒ वि । \newline
8. इति॒ वि वीतीति॒ वि स्रꣳ॑सयति स्रꣳसयति॒ वीतीति॒ वि स्रꣳ॑सयति । \newline
9. वि स्रꣳ॑सयति स्रꣳसयति॒ वि वि स्रꣳ॑सयति सवि॒तृप्र॑सूतः सवि॒तृप्र॑सूतः स्रꣳसयति॒ वि वि स्रꣳ॑सयति सवि॒तृप्र॑सूतः । \newline
10. स्रꣳ॒॒स॒य॒ति॒ स॒वि॒तृप्र॑सूतः सवि॒तृप्र॑सूतः स्रꣳसयति स्रꣳसयति सवि॒तृप्र॑सूत ए॒वैव स॑वि॒तृप्र॑सूतः स्रꣳसयति स्रꣳसयति सवि॒तृप्र॑सूत ए॒व । \newline
11. स॒वि॒तृप्र॑सूत ए॒वैव स॑वि॒तृप्र॑सूतः सवि॒तृप्र॑सूत ए॒वास्या᳚स्यै॒व स॑वि॒तृप्र॑सूतः सवि॒तृप्र॑सूत ए॒वास्य॑ । \newline
12. स॒वि॒तृप्र॑सूत॒ इति॑ सवि॒तृ - प्र॒सू॒तः॒ । \newline
13. ए॒वास्या᳚ स्यै॒वैवास्य॒ विषू॑चीं॒ ॅविषू॑ची मस्यै॒वैवास्य॒ विषू॑चीम् । \newline
14. अ॒स्य॒ विषू॑चीं॒ ॅविषू॑ची मस्यास्य॒ विषू॑चीं ॅवरुणमे॒निं ॅव॑रुणमे॒निं ॅविषू॑ची मस्यास्य॒ विषू॑चीं ॅवरुणमे॒निम् । \newline
15. विषू॑चीं ॅवरुणमे॒निं ॅव॑रुणमे॒निं ॅविषू॑चीं॒ ॅविषू॑चीं ॅवरुणमे॒निं ॅवि वि व॑रुणमे॒निं ॅविषू॑चीं॒ ॅविषू॑चीं ॅवरुणमे॒निं ॅवि । \newline
16. व॒रु॒ण॒मे॒निं ॅवि वि व॑रुणमे॒निं ॅव॑रुणमे॒निं ॅवि सृ॑जति सृजति॒ वि व॑रुणमे॒निं ॅव॑रुणमे॒निं ॅवि सृ॑जति । \newline
17. व॒रु॒ण॒मे॒निमिति॑ वरुण - मे॒निम् । \newline
18. वि सृ॑जति सृजति॒ वि वि सृ॑ज त्य॒पो॑ ऽपः सृ॑जति॒ वि वि सृ॑ज त्य॒पः । \newline
19. सृ॒ज॒ त्य॒पो॑ ऽपः सृ॑जति सृज त्य॒प उपोपा॒पः सृ॑जति सृज त्य॒प उप॑ । \newline
20. अ॒प उपोपा॒पो॑ ऽप उप॑ सृजति सृज॒ त्युपा॒पो॑ ऽप उप॑ सृजति । \newline
21. उप॑ सृजति सृज॒ त्युपोप॑ सृज॒ त्याप॒ आपः॑ सृज॒ त्युपोप॑ सृज॒ त्यापः॑ । \newline
22. सृ॒ज॒ त्याप॒ आपः॑ सृजति सृज॒ त्यापो॒ वै वा आपः॑ सृजति सृज॒ त्यापो॒ वै । \newline
23. आपो॒ वै वा आप॒ आपो॒ वै शा॒न्ताः शा॒न्ता वा आप॒ आपो॒ वै शा॒न्ताः । \newline
24. वै शा॒न्ताः शा॒न्ता वै वै शा॒न्ताः शा॒न्ताभिः॑ शा॒न्ताभिः॑ शा॒न्ता वै वै शा॒न्ताः शा॒न्ताभिः॑ । \newline
25. शा॒न्ताः शा॒न्ताभिः॑ शा॒न्ताभिः॑ शा॒न्ताः शा॒न्ताः शा॒न्ताभि॑ रे॒वैव शा॒न्ताभिः॑ शा॒न्ताः शा॒न्ताः शा॒न्ताभि॑ रे॒व । \newline
26. शा॒न्ताभि॑ रे॒वैव शा॒न्ताभिः॑ शा॒न्ताभि॑ रे॒वास्या᳚स्यै॒व शा॒न्ताभिः॑ शा॒न्ताभि॑ रे॒वास्य॑ । \newline
27. ए॒वास्या᳚ स्यै॒वैवास्य॒ शुचꣳ॒॒ शुच॑ मस्यै॒वैवास्य॒ शुच᳚म् । \newline
28. अ॒स्य॒ शुचꣳ॒॒ शुच॑ मस्यास्य॒ शुचꣳ॑ शमयति शमयति॒ शुच॑ मस्यास्य॒ शुचꣳ॑ शमयति । \newline
29. शुचꣳ॑ शमयति शमयति॒ शुचꣳ॒॒ शुचꣳ॑ शमयति ति॒सृभि॑ स्ति॒सृभिः॑ शमयति॒ शुचꣳ॒॒ शुचꣳ॑ शमयति ति॒सृभिः॑ । \newline
30. श॒म॒य॒ति॒ ति॒सृभि॑ स्ति॒सृभिः॑ शमयति शमयति ति॒सृभि॒ रुपोप॑ ति॒सृभिः॑ शमयति शमयति ति॒सृभि॒ रुप॑ । \newline
31. ति॒सृभि॒ रुपोप॑ ति॒सृभि॑ स्ति॒सृभि॒ रुप॑ सृजति सृज॒ त्युप॑ ति॒सृभि॑ स्ति॒सृभि॒ रुप॑ सृजति । \newline
32. ति॒सृभि॒रिति॑ ति॒सृ - भिः॒ । \newline
33. उप॑ सृजति सृज॒ त्युपोप॑ सृजति त्रि॒वृत् त्रि॒वृथ् सृ॑ज॒ त्युपोप॑ सृजति त्रि॒वृत् । \newline
34. सृ॒ज॒ति॒ त्रि॒वृत् त्रि॒वृथ् सृ॑जति सृजति त्रि॒वृद् वै वै त्रि॒वृथ् सृ॑जति सृजति त्रि॒वृद् वै । \newline
35. त्रि॒वृद् वै वै त्रि॒वृत् त्रि॒वृद् वा अ॒ग्नि र॒ग्निर् वै त्रि॒वृत् त्रि॒वृद् वा अ॒ग्निः । \newline
36. त्रि॒वृदिति॑ त्रि - वृत् । \newline
37. वा अ॒ग्नि र॒ग्निर् वै वा अ॒ग्निर् यावा॒न्॒. यावा॑ न॒ग्निर् वै वा अ॒ग्निर् यावान्॑ । \newline
38. अ॒ग्निर् यावा॒न्॒. यावा॑ न॒ग्नि र॒ग्निर् यावा॑ ने॒वैव यावा॑ न॒ग्नि र॒ग्निर् यावा॑ ने॒व । \newline
39. यावा॑ ने॒वैव यावा॒न्॒. यावा॑ ने॒वाग्नि र॒ग्नि रे॒व यावा॒न्॒. यावा॑ ने॒वाग्निः । \newline
40. ए॒वाग्नि र॒ग्नि रे॒वैवाग्नि स्तस्य॒ तस्या॒ग्नि रे॒वैवाग्नि स्तस्य॑ । \newline
41. अ॒ग्नि स्तस्य॒ तस्या॒ग्नि र॒ग्नि स्तस्य॒ शुचꣳ॒॒ शुच॒म् तस्या॒ग्नि र॒ग्नि स्तस्य॒ शुच᳚म् । \newline
42. तस्य॒ शुचꣳ॒॒ शुच॒म् तस्य॒ तस्य॒ शुचꣳ॑ शमयति शमयति॒ शुच॒म् तस्य॒ तस्य॒ शुचꣳ॑ शमयति । \newline
43. शुचꣳ॑ शमयति शमयति॒ शुचꣳ॒॒ शुचꣳ॑ शमयति मि॒त्रो मि॒त्रः श॑मयति॒ शुचꣳ॒॒ शुचꣳ॑ शमयति मि॒त्रः । \newline
44. श॒म॒य॒ति॒ मि॒त्रो मि॒त्रः श॑मयति शमयति मि॒त्रः सꣳ॒॒सृज्य॑ सꣳ॒॒सृज्य॑ मि॒त्रः श॑मयति शमयति मि॒त्रः सꣳ॒॒सृज्य॑ । \newline
45. मि॒त्रः सꣳ॒॒सृज्य॑ सꣳ॒॒सृज्य॑ मि॒त्रो मि॒त्रः सꣳ॒॒सृज्य॑ पृथि॒वीम् पृ॑थि॒वीꣳ सꣳ॒॒सृज्य॑ मि॒त्रो मि॒त्रः सꣳ॒॒सृज्य॑ पृथि॒वीम् । \newline
46. सꣳ॒॒सृज्य॑ पृथि॒वीम् पृ॑थि॒वीꣳ सꣳ॒॒सृज्य॑ सꣳ॒॒सृज्य॑ पृथि॒वी मितीति॑ पृथि॒वीꣳ सꣳ॒॒सृज्य॑ सꣳ॒॒सृज्य॑ पृथि॒वी मिति॑ । \newline
47. सꣳ॒॒सृज्येति॑ सं - सृज्य॑ । \newline
48. पृ॒थि॒वी मितीति॑ पृथि॒वीम् पृ॑थि॒वी मित्या॑हा॒हेति॑ पृथि॒वीम् पृ॑थि॒वी मित्या॑ह । \newline
49. इत्या॑हा॒हे तीत्या॑ह मि॒त्रो मि॒त्र आ॒हे तीत्या॑ह मि॒त्रः । \newline
50. आ॒ह॒ मि॒त्रो मि॒त्र आ॑हाह मि॒त्रो वै वै मि॒त्र आ॑हाह मि॒त्रो वै । \newline
51. मि॒त्रो वै वै मि॒त्रो मि॒त्रो वै शि॒वः शि॒वो वै मि॒त्रो मि॒त्रो वै शि॒वः । \newline
52. वै शि॒वः शि॒वो वै वै शि॒वो दे॒वाना᳚म् दे॒वानाꣳ॑ शि॒वो वै वै शि॒वो दे॒वाना᳚म् । \newline
53. शि॒वो दे॒वाना᳚म् दे॒वानाꣳ॑ शि॒वः शि॒वो दे॒वाना॒म् तेन॒ तेन॑ दे॒वानाꣳ॑ शि॒वः शि॒वो दे॒वाना॒म् तेन॑ । \newline
54. दे॒वाना॒म् तेन॒ तेन॑ दे॒वाना᳚म् दे॒वाना॒म् तेनै॒वैव तेन॑ दे॒वाना᳚म् दे॒वाना॒म् तेनै॒व । \newline
55. तेनै॒वैव तेन॒ तेनै॒वैन॑ मेन मे॒व तेन॒ तेनै॒वैन᳚म् । \newline
56. ए॒वैन॑ मेन मे॒वैवैनꣳ॒॒ सꣳ स मे॑न मे॒वैवैनꣳ॒॒ सम् । \newline
\pagebreak
\markright{ TS 5.1.6.2  \hfill https://www.vedavms.in \hfill}

\section{ TS 5.1.6.2 }

\textbf{TS 5.1.6.2 } \newline
\textbf{Samhita Paata} \newline

-नꣳ॒॒ सꣳ सृ॑जति॒ शान्त्यै॒ यद्ग्रा॒म्याणां॒ पात्रा॑णां क॒पालैः᳚ सꣳसृ॒जेद्-ग्रा॒म्याणि॒ पात्रा॑णि शु॒चाऽर्प॑येदर्मकपा॒लैः सꣳ सृ॑जत्ये॒तानि॒ वा अ॑नुपजीवनी॒यानि॒ तान्ये॒व शु॒चाऽर्प॑यति॒ शर्क॑राभिः॒ सꣳ सृ॑जति॒ धृत्या॒ अथो॑ श॒त्वांया॑ जलो॒मैः सꣳ सृ॑जत्ये॒षा वा अ॒ग्नेः प्रि॒या त॒नूर्यद॒जा प्रि॒ययै॒वैनं॑ त॒नुवा॒ सꣳ सृ॑ज॒त्यथो॒ तेज॑सा कृष्णाजि॒नस्य॒ लोम॑भिः॒ सꣳ - [  ] \newline

\textbf{Pada Paata} \newline

ए॒न॒म् । समिति॑ । सृ॒ज॒ति॒ । शान्त्यै᳚ । यत् । ग्रा॒म्याणा᳚म् । पात्रा॑णाम् । क॒पालैः᳚ । सꣳ॒॒सृ॒जेदिति॑ सं - सृ॒जेत् । ग्रा॒म्याणि॑ । पात्रा॑णि । शु॒चा । अ॒र्प॒ये॒त् । अ॒र्म॒क॒पा॒लैरित्य॑र्म-क॒पा॒लैः । समिति॑ । सृ॒ज॒ति॒ । ए॒तानि॑ । वै । अ॒नु॒प॒जी॒व॒नी॒यानीत्य॑नुप - जी॒व॒नी॒यानि॑ । तानि॑ । ए॒व । शु॒चा । अ॒र्प॒य॒ति॒ । शर्क॑राभिः । समिति॑ । सृ॒ज॒ति॒ । धृत्यै᳚ । अथो॒ इति॑ । श॒न्त्वायेति॑ शं - त्वाय॑ । अ॒ज॒लो॒मैरित्य॑ज-लो॒मैः । समिति॑ । सृ॒ज॒ति॒ । ए॒षा । वै । अ॒ग्नेः । प्रि॒या । त॒नूः । यत् । अ॒जा । प्रि॒यया᳚ । ए॒व । ए॒न॒म् । त॒नुवा᳚ । समिति॑ । सृ॒ज॒ति॒ । अथो॒ इति॑ । तेज॑सा । कृ॒ष्णा॒जि॒नस्येति॑ कृष्ण - अ॒जि॒नस्य॑ । लोम॑भि॒रिति॒ लोम॑ - भिः॒ । समिति॑ ।  \newline


\textbf{Krama Paata} \newline

ए॒नꣳ॒॒ सम् । सꣳ सृ॑जति । सृ॒ज॒ति॒ शान्त्यै᳚ । शान्त्यै॒ यत् । यद् ग्रा॒म्याणा᳚म् । ग्रा॒म्याणा॒म् पात्रा॑णाम् । पात्रा॑णाम् क॒पालैः᳚ । क॒पालैः᳚ सꣳसृ॒जेत् । सꣳ॒॒सृ॒जेद् ग्रा॒म्याणि॑ । सꣳ॒॒सृ॒जेदिति॑ सम् - सृ॒जेत् । ग्रा॒म्याणि॒ पात्रा॑णि । पात्रा॑णि शु॒चा । शु॒चाऽर्प॑येत् । अ॒र्प॒ये॒द॒र्म॒क॒पा॒लैः । अ॒र्म॒क॒पा॒लैः सम् । अ॒र्म॒क॒पा॒लैरित्य॑र्म - क॒पा॒लैः । सꣳ सृ॑जति । सृ॒ज॒त्ये॒तानि॑ । ए॒तानि॒ वै । वा अ॑नुपजीवनी॒यानि॑ । अ॒नु॒प॒जी॒व॒नी॒यानि॒ तानि॑ । अ॒नु॒प॒जी॒व॒नी॒यानीत्य॑नुप - जी॒व॒नी॒यानि॑ । तान्ये॒व । ए॒व शु॒चा । शु॒चाऽर्प॑यति । अ॒र्प॒य॒ति॒ शर्क॑राभिः । शर्क॑राभिः॒ सम् । सꣳ सृ॑जति । सृ॒ज॒ति॒ धृत्यै᳚ । धृत्या॒ अथो᳚ । अथो॑ श॒न्त्वाय॑ । अथो॒ इत्यथो᳚ । श॒न्त्वाया॑जलो॒मैः । श॒न्त्वायेति॑ शम् - त्वाय॑ । अ॒ज॒लो॒मैः सम् । अ॒ज॒लो॒मैरित्य॑ज - लो॒मैः । सꣳ सृ॑जति । सृ॒ज॒त्ये॒षा । ए॒षा वै । वा अ॒ग्नेः । अ॒ग्नेः प्रि॒या । प्रि॒या त॒नूः । त॒नूर् यत् । यद॒जा । अ॒जा प्रि॒यया᳚ । प्रि॒ययै॒व । ए॒वैन᳚म् । ए॒न॒म् त॒नुवा᳚ । त॒नुवा॒ सम् । सꣳ सृ॑जति । सृ॒ज॒त्यथो᳚ । अथो॒ तेज॑सा । अथो॒ इत्यथो᳚ । तेज॑सा कृष्णाजि॒नस्य॑ । कृ॒ष्णा॒जि॒नस्य॒ लोम॑भिः । कृ॒ष्णा॒जि॒नस्येति॑ कृष्ण - अ॒जि॒नस्य॑ । लोम॑भिः॒ सम् । लोम॑भि॒रिति॒ लोम॑ - भिः॒ । सꣳ सृ॑जति \newline

\textbf{Jatai Paata} \newline

1. ए॒नꣳ॒॒ सꣳ स मे॑न मेनꣳ॒॒ सम् । \newline
2. सꣳ सृ॑जति सृजति॒ सꣳ सꣳ सृ॑जति । \newline
3. सृ॒ज॒ति॒ शान्त्यै॒ शान्त्यै॑ सृजति सृजति॒ शान्त्यै᳚ । \newline
4. शान्त्यै॒ यद् यच्छान्त्यै॒ शान्त्यै॒ यत् । \newline
5. यद् ग्रा॒म्याणा᳚म् ग्रा॒म्याणां॒ ॅयद् यद् ग्रा॒म्याणा᳚म् । \newline
6. ग्रा॒म्याणा॒म् पात्रा॑णा॒म् पात्रा॑णाम् ग्रा॒म्याणा᳚म् ग्रा॒म्याणा॒म् पात्रा॑णाम् । \newline
7. पात्रा॑णाम् क॒पालैः᳚ क॒पालैः॒ पात्रा॑णा॒म् पात्रा॑णाम् क॒पालैः᳚ । \newline
8. क॒पालैः᳚ सꣳसृ॒जेथ् सꣳ॑सृ॒जेत् क॒पालैः᳚ क॒पालैः᳚ सꣳसृ॒जेत् । \newline
9. सꣳ॒॒सृ॒जेद् ग्रा॒म्याणि॑ ग्रा॒म्याणि॑ सꣳसृ॒जेथ् सꣳ॑सृ॒जेद् ग्रा॒म्याणि॑ । \newline
10. सꣳ॒॒सृ॒जेदिति॑ सं - सृ॒जेत् । \newline
11. ग्रा॒म्याणि॒ पात्रा॑णि॒ पात्रा॑णि ग्रा॒म्याणि॑ ग्रा॒म्याणि॒ पात्रा॑णि । \newline
12. पात्रा॑णि शु॒चा शु॒चा पात्रा॑णि॒ पात्रा॑णि शु॒चा । \newline
13. शु॒चा ऽर्प॑ये दर्पये च्छु॒चा शु॒चा ऽर्प॑येत् । \newline
14. अ॒र्प॒ये॒ द॒र्म॒क॒पा॒लै र॑र्मकपा॒लै र॑र्पये दर्पये दर्मकपा॒लैः । \newline
15. अ॒र्म॒क॒पा॒लैः सꣳ स म॑र्मकपा॒लै र॑र्मकपा॒लैः सम् । \newline
16. अ॒र्म॒क॒पा॒लैरित्य॑र्म - क॒पा॒लैः । \newline
17. सꣳ सृ॑जति सृजति॒ सꣳ सꣳ सृ॑जति । \newline
18. सृ॒ज॒ त्ये॒ता न्ये॒तानि॑ सृजति सृज त्ये॒तानि॑ । \newline
19. ए॒तानि॒ वै वा ए॒ता न्ये॒तानि॒ वै । \newline
20. वा अ॑नुपजीवनी॒या न्य॑नुपजीवनी॒यानि॒ वै वा अ॑नुपजीवनी॒यानि॑ । \newline
21. अ॒नु॒प॒जी॒व॒नी॒यानि॒ तानि॒ तान्य॑नुपजीवनी॒या न्य॑नुपजीवनी॒यानि॒ तानि॑ । \newline
22. अ॒नु॒प॒जी॒व॒नी॒यानीत्य॑नुप - जी॒व॒नी॒यानि॑ । \newline
23. तान्ये॒वैव तानि॒ तान्ये॒व । \newline
24. ए॒व शु॒चा शु॒चैवैव शु॒चा । \newline
25. शु॒चा ऽर्प॑य त्यर्पयति शु॒चा शु॒चा ऽर्प॑यति । \newline
26. अ॒र्प॒य॒ति॒ शर्क॑राभिः॒ शर्क॑राभि रर्पय त्यर्पयति॒ शर्क॑राभिः । \newline
27. शर्क॑राभिः॒ सꣳ सꣳ शर्क॑राभिः॒ शर्क॑राभिः॒ सम् । \newline
28. सꣳ सृ॑जति सृजति॒ सꣳ सꣳ सृ॑जति । \newline
29. सृ॒ज॒ति॒ धृत्यै॒ धृत्यै॑ सृजति सृजति॒ धृत्यै᳚ । \newline
30. धृत्या॒ अथो॒ अथो॒ धृत्यै॒ धृत्या॒ अथो᳚ । \newline
31. अथो॑ श॒न्त्वाय॑ श॒न्त्वायाथो॒ अथो॑ श॒न्त्वाय॑ । \newline
32. अथो॒ इत्यथो᳚ । \newline
33. श॒न्त्वाया॑ जलो॒मै र॑जलो॒मैः श॒न्त्वाय॑ श॒न्त्वाया॑ जलो॒मैः । \newline
34. श॒न्त्वायेति॑ शं - त्वाय॑ । \newline
35. अ॒ज॒लो॒मैः सꣳ स म॑जलो॒मै र॑जलो॒मैः सम् । \newline
36. अ॒ज॒लो॒मैरित्य॑ज - लो॒मैः । \newline
37. सꣳ सृ॑जति सृजति॒ सꣳ सꣳ सृ॑जति । \newline
38. सृ॒ज॒ त्ये॒षैषा सृ॑जति सृज त्ये॒षा । \newline
39. ए॒षा वै वा ए॒षैषा वै । \newline
40. वा अ॒ग्ने र॒ग्नेर् वै वा अ॒ग्नेः । \newline
41. अ॒ग्नेः प्रि॒या प्रि॒या ऽग्ने र॒ग्नेः प्रि॒या । \newline
42. प्रि॒या त॒नू स्त॒नूः प्रि॒या प्रि॒या त॒नूः । \newline
43. त॒नूर् यद् यत् त॒नू स्त॒नूर् यत् । \newline
44. यद॒जा ऽजा यद् यद॒जा । \newline
45. अ॒जा प्रि॒यया᳚ प्रि॒यया॒ ऽजा ऽजा प्रि॒यया᳚ । \newline
46. प्रि॒य यै॒वैव प्रि॒यया᳚ प्रि॒य यै॒व । \newline
47. ए॒वैन॑ मेन मे॒वैवैन᳚म् । \newline
48. ए॒न॒म् त॒नुवा॑ त॒नुवै॑न मेनम् त॒नुवा᳚ । \newline
49. त॒नुवा॒ सꣳ सम् त॒नुवा॑ त॒नुवा॒ सम् । \newline
50. सꣳ सृ॑जति सृजति॒ सꣳ सꣳ सृ॑जति । \newline
51. सृ॒ज॒ त्यथो॒ अथो॑ सृजति सृज॒ त्यथो᳚ । \newline
52. अथो॒ तेज॑सा॒ तेज॒सा ऽथो॒ अथो॒ तेज॑सा । \newline
53. अथो॒ इत्यथो᳚ । \newline
54. तेज॑सा कृष्णाजि॒नस्य॑ कृष्णाजि॒नस्य॒ तेज॑सा॒ तेज॑सा कृष्णाजि॒नस्य॑ । \newline
55. कृ॒ष्णा॒जि॒नस्य॒ लोम॑भि॒र् लोम॑भिः कृष्णाजि॒नस्य॑ कृष्णाजि॒नस्य॒ लोम॑भिः । \newline
56. कृ॒ष्णा॒जि॒नस्येति॑ कृष्ण - अ॒जि॒नस्य॑ । \newline
57. लोम॑भिः॒ सꣳ सम् ॅलोम॑भि॒र् लोम॑भिः॒ सम् । \newline
58. लोम॑भि॒रिति॒ लोम॑ - भिः॒ । \newline
59. सꣳ सृ॑जति सृजति॒ सꣳ सꣳ सृ॑जति । \newline

\textbf{Ghana Paata } \newline

1. ए॒नꣳ॒॒ सꣳ स मे॑न मेनꣳ॒॒ सꣳ सृ॑जति सृजति॒ स मे॑न मेनꣳ॒॒ सꣳ सृ॑जति । \newline
2. सꣳ सृ॑जति सृजति॒ सꣳ सꣳ सृ॑जति॒ शान्त्यै॒ शान्त्यै॑ सृजति॒ सꣳ सꣳ सृ॑जति॒ शान्त्यै᳚ । \newline
3. सृ॒ज॒ति॒ शान्त्यै॒ शान्त्यै॑ सृजति सृजति॒ शान्त्यै॒ यद् यच्छान्त्यै॑ सृजति सृजति॒ शान्त्यै॒ यत् । \newline
4. शान्त्यै॒ यद् यच्छान्त्यै॒ शान्त्यै॒ यद् ग्रा॒म्याणा᳚म् ग्रा॒म्याणां॒ ॅयच्छान्त्यै॒ शान्त्यै॒ यद् ग्रा॒म्याणा᳚म् । \newline
5. यद् ग्रा॒म्याणा᳚म् ग्रा॒म्याणां॒ ॅयद् यद् ग्रा॒म्याणा॒म् पात्रा॑णा॒म् पात्रा॑णाम् ग्रा॒म्याणां॒ ॅयद् यद् ग्रा॒म्याणा॒म् पात्रा॑णाम् । \newline
6. ग्रा॒म्याणा॒म् पात्रा॑णा॒म् पात्रा॑णाम् ग्रा॒म्याणा᳚म् ग्रा॒म्याणा॒म् पात्रा॑णाम् क॒पालैः᳚ क॒पालैः॒ पात्रा॑णाम् ग्रा॒म्याणा᳚म् ग्रा॒म्याणा॒म् पात्रा॑णाम् क॒पालैः᳚ । \newline
7. पात्रा॑णाम् क॒पालैः᳚ क॒पालैः॒ पात्रा॑णा॒म् पात्रा॑णाम् क॒पालैः᳚ सꣳसृ॒जेथ् सꣳ॑सृ॒जेत् क॒पालैः॒ पात्रा॑णा॒म् पात्रा॑णाम् क॒पालैः᳚ सꣳसृ॒जेत् । \newline
8. क॒पालैः᳚ सꣳसृ॒जेथ् सꣳ॑सृ॒जेत् क॒पालैः᳚ क॒पालैः᳚ सꣳसृ॒जेद् ग्रा॒म्याणि॑ ग्रा॒म्याणि॑ सꣳसृ॒जेत् क॒पालैः᳚ क॒पालैः᳚ सꣳसृ॒जेद् ग्रा॒म्याणि॑ । \newline
9. सꣳ॒॒सृ॒जेद् ग्रा॒म्याणि॑ ग्रा॒म्याणि॑ सꣳसृ॒जेथ् सꣳ॑सृ॒जेद् ग्रा॒म्याणि॒ पात्रा॑णि॒ पात्रा॑णि ग्रा॒म्याणि॑ सꣳसृ॒जेथ् सꣳ॑सृ॒जेद् ग्रा॒म्याणि॒ पात्रा॑णि । \newline
10. सꣳ॒॒सृ॒जेदिति॑ सं - सृ॒जेत् । \newline
11. ग्रा॒म्याणि॒ पात्रा॑णि॒ पात्रा॑णि ग्रा॒म्याणि॑ ग्रा॒म्याणि॒ पात्रा॑णि शु॒चा शु॒चा पात्रा॑णि ग्रा॒म्याणि॑ ग्रा॒म्याणि॒ पात्रा॑णि शु॒चा । \newline
12. पात्रा॑णि शु॒चा शु॒चा पात्रा॑णि॒ पात्रा॑णि शु॒चा ऽर्प॑ये दर्पये च्छु॒चा पात्रा॑णि॒ पात्रा॑णि शु॒चा ऽर्प॑येत् । \newline
13. शु॒चा ऽर्प॑ये दर्पये च्छु॒चा शु॒चा ऽर्प॑ये दर्मकपा॒लै र॑र्मकपा॒लै र॑र्पये च्छु॒चा शु॒चा ऽर्प॑येदर्मकपा॒लैः । \newline
14. अ॒र्प॒ये॒द॒र्म॒क॒पा॒लै र॑र्मकपा॒लै र॑र्पये दर्पये दर्मकपा॒लैः सꣳ स म॑र्मकपा॒लै र॑र्पये दर्पये दर्मकपा॒लैः सम् । \newline
15. अ॒र्म॒क॒पा॒लैः सꣳ स म॑र्मकपा॒लै र॑र्मकपा॒लैः सꣳ सृ॑जति सृजति॒ स म॑र्मकपा॒लै र॑र्मकपा॒लैः सꣳ सृ॑जति । \newline
16. अ॒र्म॒क॒पा॒लैरित्य॑र्म - क॒पा॒लैः । \newline
17. सꣳ सृ॑जति सृजति॒ सꣳ सꣳ सृ॑ज त्ये॒ता न्ये॒तानि॑ सृजति॒ सꣳ सꣳ सृ॑ज त्ये॒तानि॑ । \newline
18. सृ॒ज॒ त्ये॒ता न्ये॒तानि॑ सृजति सृज त्ये॒तानि॒ वै वा ए॒तानि॑ सृजति सृज त्ये॒तानि॒ वै । \newline
19. ए॒तानि॒ वै वा ए॒ता न्ये॒तानि॒ वा अ॑नुपजीवनी॒या न्य॑नुपजीवनी॒यानि॒ वा ए॒तान्ये॒तानि॒ वा अ॑नुपजीवनी॒यानि॑ । \newline
20. वा अ॑नुपजीवनी॒या न्य॑नुपजीवनी॒यानि॒ वै वा अ॑नुपजीवनी॒यानि॒ तानि॒ तान्य॑नुपजीवनी॒यानि॒ वै वा अ॑नुपजीवनी॒यानि॒ तानि॑ । \newline
21. अ॒नु॒प॒जी॒व॒नी॒यानि॒ तानि॒ तान्य॑नुपजीवनी॒या न्य॑नुपजीवनी॒यानि॒ तान्ये॒वैव तान्य॑नुपजीवनी॒या न्य॑नुपजीवनी॒यानि॒ तान्ये॒व । \newline
22. अ॒नु॒प॒जी॒व॒नी॒यानीत्य॑नुप - जी॒व॒नी॒यानि॑ । \newline
23. तान्ये॒वैव तानि॒ तान्ये॒व शु॒चा शु॒चैव तानि॒ तान्ये॒व शु॒चा । \newline
24. ए॒व शु॒चा शु॒चैवैव शु॒चा ऽर्प॑य त्यर्पयति शु॒चैवैव शु॒चा ऽर्प॑यति । \newline
25. शु॒चा ऽर्प॑य त्यर्पयति शु॒चा शु॒चा ऽर्प॑यति॒ शर्क॑राभिः॒ शर्क॑राभि रर्पयति शु॒चा शु॒चा ऽर्प॑यति॒ शर्क॑राभिः । \newline
26. अ॒र्प॒य॒ति॒ शर्क॑राभिः॒ शर्क॑राभि रर्पय त्यर्पयति॒ शर्क॑राभिः॒ सꣳ सꣳ शर्क॑राभि रर्पय त्यर्पयति॒ शर्क॑राभिः॒ सम् । \newline
27. शर्क॑राभिः॒ सꣳ सꣳ शर्क॑राभिः॒ शर्क॑राभिः॒ सꣳ सृ॑जति सृजति॒ सꣳ शर्क॑राभिः॒ शर्क॑राभिः॒ सꣳ सृ॑जति । \newline
28. सꣳ सृ॑जति सृजति॒ सꣳ सꣳ सृ॑जति॒ धृत्यै॒ धृत्यै॑ सृजति॒ सꣳ सꣳ सृ॑जति॒ धृत्यै᳚ । \newline
29. सृ॒ज॒ति॒ धृत्यै॒ धृत्यै॑ सृजति सृजति॒ धृत्या॒ अथो॒ अथो॒ धृत्यै॑ सृजति सृजति॒ धृत्या॒ अथो᳚ । \newline
30. धृत्या॒ अथो॒ अथो॒ धृत्यै॒ धृत्या॒ अथो॑ श॒न्त्वाय॑ श॒न्त्वायाथो॒ धृत्यै॒ धृत्या॒ अथो॑ श॒न्त्वाय॑ । \newline
31. अथो॑ श॒न्त्वाय॑ श॒न्त्वायाथो॒ अथो॑ श॒न्त्वाया॑ जलो॒मै र॑जलो॒मैः श॒न्त्वायाथो॒ अथो॑ श॒न्त्वाया॑ जलो॒मैः । \newline
32. अथो॒ इत्यथो᳚ । \newline
33. श॒न्त्वाया॑ जलो॒मै र॑जलो॒मैः श॒न्त्वाय॑ श॒न्त्वाया॑ जलो॒मैः सꣳ स म॑जलो॒मैः श॒न्त्वाय॑ श॒न्त्वाया॑ जलो॒मैः सम् । \newline
34. श॒न्त्वायेति॑ शं - त्वाय॑ । \newline
35. अ॒ज॒लो॒मैः सꣳ स म॑जलो॒मै र॑जलो॒मैः सꣳ सृ॑जति सृजति॒ स म॑जलो॒मै र॑जलो॒मैः सꣳ सृ॑जति । \newline
36. अ॒ज॒लो॒मैरित्य॑ज - लो॒मैः । \newline
37. सꣳ सृ॑जति सृजति॒ सꣳ सꣳ सृ॑ज त्ये॒षैषा सृ॑जति॒ सꣳ सꣳ सृ॑ज त्ये॒षा । \newline
38. सृ॒ज॒ त्ये॒षैषा सृ॑जति सृज त्ये॒षा वै वा ए॒षा सृ॑जति सृज त्ये॒षा वै । \newline
39. ए॒षा वै वा ए॒षैषा वा अ॒ग्ने र॒ग्नेर् वा ए॒षैषा वा अ॒ग्नेः । \newline
40. वा अ॒ग्ने र॒ग्नेर् वै वा अ॒ग्नेः प्रि॒या प्रि॒या ऽग्नेर् वै वा अ॒ग्नेः प्रि॒या । \newline
41. अ॒ग्नेः प्रि॒या प्रि॒या ऽग्ने र॒ग्नेः प्रि॒या त॒नू स्त॒नूः प्रि॒या ऽग्ने र॒ग्नेः प्रि॒या त॒नूः । \newline
42. प्रि॒या त॒नू स्त॒नूः प्रि॒या प्रि॒या त॒नूर् यद् यत् त॒नूः प्रि॒या प्रि॒या त॒नूर् यत् । \newline
43. त॒नूर् यद् यत् त॒नू स्त॒नूर् यद॒जा ऽजा यत् त॒नू स्त॒नूर् यद॒जा । \newline
44. यद॒जा ऽजा यद् यद॒जा प्रि॒यया᳚ प्रि॒यया॒ ऽजा यद् यद॒जा प्रि॒यया᳚ । \newline
45. अ॒जा प्रि॒यया᳚ प्रि॒यया॒ ऽजा ऽजा प्रि॒ययै॒वैव प्रि॒यया॒ ऽजा ऽजा प्रि॒ययै॒व । \newline
46. प्रि॒ययै॒वैव प्रि॒यया᳚ प्रि॒ययै॒वैन॑ मेन मे॒व प्रि॒यया᳚ प्रि॒ययै॒वैन᳚म् । \newline
47. ए॒वैन॑ मेन मे॒वैवैन॑म् त॒नुवा॑ त॒नुवै॑न मे॒वैवैन॑म् त॒नुवा᳚ । \newline
48. ए॒न॒म् त॒नुवा॑ त॒नुवै॑न मेनम् त॒नुवा॒ सꣳ सम् त॒नुवै॑न मेनम् त॒नुवा॒ सम् । \newline
49. त॒नुवा॒ सꣳ सम् त॒नुवा॑ त॒नुवा॒ सꣳ सृ॑जति सृजति॒ सम् त॒नुवा॑ त॒नुवा॒ सꣳ सृ॑जति । \newline
50. सꣳ सृ॑जति सृजति॒ सꣳ सꣳ सृ॑ज॒ त्यथो॒ अथो॑ सृजति॒ सꣳ सꣳ सृ॑ज॒ त्यथो᳚ । \newline
51. सृ॒ज॒ त्यथो॒ अथो॑ सृजति सृज॒ त्यथो॒ तेज॑सा॒ तेज॒सा ऽथो॑ सृजति सृज॒ त्यथो॒ तेज॑सा । \newline
52. अथो॒ तेज॑सा॒ तेज॒सा ऽथो॒ अथो॒ तेज॑सा कृष्णाजि॒नस्य॑ कृष्णाजि॒नस्य॒ तेज॒सा ऽथो॒ अथो॒ तेज॑सा कृष्णाजि॒नस्य॑ । \newline
53. अथो॒ इत्यथो᳚ । \newline
54. तेज॑सा कृष्णाजि॒नस्य॑ कृष्णाजि॒नस्य॒ तेज॑सा॒ तेज॑सा कृष्णाजि॒नस्य॒ लोम॑भि॒र् लोम॑भिः कृष्णाजि॒नस्य॒ तेज॑सा॒ तेज॑सा कृष्णाजि॒नस्य॒ लोम॑भिः । \newline
55. कृ॒ष्णा॒जि॒नस्य॒ लोम॑भि॒र् लोम॑भिः कृष्णाजि॒नस्य॑ कृष्णाजि॒नस्य॒ लोम॑भिः॒ सꣳ सम् ॅलोम॑भिः कृष्णाजि॒नस्य॑ कृष्णाजि॒नस्य॒ लोम॑भिः॒ सम् । \newline
56. कृ॒ष्णा॒जि॒नस्येति॑ कृष्ण - अ॒जि॒नस्य॑ । \newline
57. लोम॑भिः॒ सꣳ सम् ॅलोम॑भि॒र् लोम॑भिः॒ सꣳ सृ॑जति सृजति॒ सम् ॅलोम॑भि॒र् लोम॑भिः॒ सꣳ सृ॑जति । \newline
58. लोम॑भि॒रिति॒ लोम॑ - भिः॒ । \newline
59. सꣳ सृ॑जति सृजति॒ सꣳ सꣳ सृ॑जति य॒ज्ञो य॒ज्ञ्ः सृ॑जति॒ सꣳ सꣳ सृ॑जति य॒ज्ञ्ः । \newline
\pagebreak
\markright{ TS 5.1.6.3  \hfill https://www.vedavms.in \hfill}

\section{ TS 5.1.6.3 }

\textbf{TS 5.1.6.3 } \newline
\textbf{Samhita Paata} \newline

सृ॑जति य॒ज्ञो वै कृ॑ष्णाजि॒नं ॅय॒ज्ञेनै॒व य॒ज्ञ्ꣳ सꣳ सृ॑जति रु॒द्राः स॒भृंत्य॑ पृथि॒वीमित्या॑है॒ता वा ए॒तं दे॒वता॒ अग्रे॒ सम॑भर॒न् ताभि॑रे॒वैनꣳ॒॒ संभ॑रति म॒खस्य॒ शिरो॒ऽसीत्या॑ह य॒ज्ञो वै म॒खस्तस्यै॒त-च्छिरो॒ यदु॒खा तस्मा॑दे॒वमा॑ह य॒ज्ञ्स्य॑ प॒दे स्थ॒ इत्या॑ह य॒ज्ञ्स्य॒ ह्ये॑ते - [  ] \newline

\textbf{Pada Paata} \newline

सृ॒ज॒ति॒ । य॒ज्ञ्ः । वै । कृ॒ष्णा॒जि॒नमिति॑ कृष्ण - अ॒जि॒नम् । य॒ज्ञेन॑ । ए॒व । य॒ज्ञ्म् । समिति॑ । सृ॒ज॒ति॒ । रु॒द्राः । स॒भृंत्येति॑ सं - भृत्य॑ । पृ॒थि॒वीम् । इति॑ । आ॒ह॒ । ए॒ताः । वै । ए॒तम् । दे॒वताः᳚ । अग्रे᳚ । समिति॑ । अ॒भ॒र॒न्न् । ताभिः॑ । ए॒व । ए॒न॒म् । समिति॑ । भ॒र॒ति॒ । म॒खस्य॑ । शिरः॑ । अ॒सि॒ । इति॑ । आ॒ह॒ । य॒ज्ञ्ः । वै । म॒खः । तस्य॑ । ए॒तत् । शिरः॑ । यत् । उ॒खा । तस्मा᳚त् । ए॒वम् । आ॒ह॒ । य॒ज्ञ्स्य॑ । प॒दे इति॑ । स्थः॒ । इति॑ । आ॒ह॒ । य॒ज्ञ्स्य॑ । हि । ए॒ते इति॑ ।  \newline


\textbf{Krama Paata} \newline

सृ॒ज॒ति॒ य॒ज्ञ्ः । य॒ज्ञो वै । वै कृ॑ष्णाजि॒नम् । कृ॒ष्णा॒जि॒नम् ॅय॒ज्ञेन॑ । कृ॒ष्णा॒जि॒नमिति॑ कृष्ण - अ॒जि॒नम् । य॒ज्ञेनै॒व । ए॒व य॒ज्ञ्म् । य॒ज्ञ्ꣳ सम् । सꣳ सृ॑जति । सृ॒ज॒ति॒ रु॒द्राः । रु॒द्राः स॒म्भृत्य॑ । स॒म्भृत्य॑ पृथि॒वीम् । स॒म्भृत्येति॑ सम् - भृत्य॑ । पृ॒थि॒वीमिति॑ । इत्या॑ह । आ॒है॒ताः । ए॒ता वै । वा ए॒तम् । ए॒तम् दे॒वताः᳚ । दे॒वता॒ अग्रे᳚ । अग्रे॒ सम् । सम॑भरन्न् । अ॒भ॒र॒न् ताभिः॑ । ताभि॑रे॒व । ए॒वैन᳚म् । ए॒नꣳ॒॒ सम् । सम् भ॑रति । भ॒र॒ति॒ म॒खस्य॑ । म॒खस्य॒ शिरः॑ । शिरो॑ऽसि । अ॒सीति॑ । इत्या॑ह । आ॒ह॒ य॒ज्ञ्ः । य॒ज्ञो वै । वै म॒खः । म॒खस्तस्य॑ । तस्यै॒तत् । ए॒तच्छिरः॑ । शिरो॒ यत् । यदु॒खा । उ॒खा तस्मा᳚त् । तस्मा॑दे॒वम् । ए॒वमा॑ह । आ॒ह॒ य॒ज्ञ्स्य॑ । य॒ज्ञ्स्य॑ प॒दे । प॒दे स्थः॑ । प॒दे इति॑ प॒दे । स्थ॒ इति॑ । इत्या॑ह । आ॒ह॒ य॒ज्ञ्स्य॑ । य॒ज्ञ्स्य॒ हि । ह्ये॑ते । ए॒ते प॒दे । ए॒ते इत्ये॒ते \newline

\textbf{Jatai Paata} \newline

1. सृ॒ज॒ति॒ य॒ज्ञो य॒ज्ञ्ः सृ॑जति सृजति य॒ज्ञ्ः । \newline
2. य॒ज्ञो वै वै य॒ज्ञो य॒ज्ञो वै । \newline
3. वै कृ॑ष्णाजि॒नम् कृ॑ष्णाजि॒नं ॅवै वै कृ॑ष्णाजि॒नम् । \newline
4. कृ॒ष्णा॒जि॒नं ॅय॒ज्ञेन॑ य॒ज्ञेन॑ कृष्णाजि॒नम् कृ॑ष्णाजि॒नं ॅय॒ज्ञेन॑ । \newline
5. कृ॒ष्णा॒जि॒नमिति॑ कृष्ण - अ॒जि॒नम् । \newline
6. य॒ज्ञेनै॒वैव य॒ज्ञेन॑ य॒ज्ञेनै॒व । \newline
7. ए॒व य॒ज्ञ्ं ॅय॒ज्ञ् मे॒वैव य॒ज्ञ्म् । \newline
8. य॒ज्ञ्ꣳ सꣳ सं ॅय॒ज्ञ्ं ॅय॒ज्ञ्ꣳ सम् । \newline
9. सꣳ सृ॑जति सृजति॒ सꣳ सꣳ सृ॑जति । \newline
10. सृ॒ज॒ति॒ रु॒द्रा रु॒द्राः सृ॑जति सृजति रु॒द्राः । \newline
11. रु॒द्राः सं॒भृत्य॑ सं॒भृत्य॑ रु॒द्रा रु॒द्राः सं॒भृत्य॑ । \newline
12. सं॒भृत्य॑ पृथि॒वीम् पृ॑थि॒वीꣳ सं॒भृत्य॑ सं॒भृत्य॑ पृथि॒वीम् । \newline
13. सं॒भृत्येति॑ सं - भृत्य॑ । \newline
14. पृ॒थि॒वी मितीति॑ पृथि॒वीम् पृ॑थि॒वी मिति॑ । \newline
15. इत्या॑हा॒हे तीत्या॑ह । \newline
16. आ॒है॒ता ए॒ता आ॑हा है॒ताः । \newline
17. ए॒ता वै वा ए॒ता ए॒ता वै । \newline
18. वा ए॒त मे॒तं ॅवै वा ए॒तम् । \newline
19. ए॒तम् दे॒वता॑ दे॒वता॑ ए॒त मे॒तम् दे॒वताः᳚ । \newline
20. दे॒वता॒ अग्रे ऽग्रे॑ दे॒वता॑ दे॒वता॒ अग्रे᳚ । \newline
21. अग्रे॒ सꣳ स मग्रे ऽग्रे॒ सम् । \newline
22. स म॑भरन् नभर॒न् थ्सꣳ स म॑भरन्न् । \newline
23. अ॒भ॒र॒न् ताभि॒ स्ताभि॑ रभरन् नभर॒न् ताभिः॑ । \newline
24. ताभि॑ रे॒वैव ताभि॒ स्ताभि॑ रे॒व । \newline
25. ए॒वैन॑ मेन मे॒वैवैन᳚म् । \newline
26. ए॒नꣳ॒॒ सꣳ स मे॑न मेनꣳ॒॒ सम् । \newline
27. सम् भ॑रति भरति॒ सꣳ सम् भ॑रति । \newline
28. भ॒र॒ति॒ म॒खस्य॑ म॒खस्य॑ भरति भरति म॒खस्य॑ । \newline
29. म॒खस्य॒ शिरः॒ शिरो॑ म॒खस्य॑ म॒खस्य॒ शिरः॑ । \newline
30. शिरो᳚ ऽस्यसि॒ शिरः॒ शिरो॑ ऽसि । \newline
31. अ॒सीती त्य॑स्य॒सीति॑ । \newline
32. इत्या॑हा॒हे तीत्या॑ह । \newline
33. आ॒ह॒ य॒ज्ञो य॒ज्ञ् आ॑हाह य॒ज्ञ्ः । \newline
34. य॒ज्ञो वै वै य॒ज्ञो य॒ज्ञो वै । \newline
35. वै म॒खो म॒खो वै वै म॒खः । \newline
36. म॒ख स्तस्य॒ तस्य॑ म॒खो म॒ख स्तस्य॑ । \newline
37. तस्यै॒त दे॒तत् तस्य॒ तस्यै॒तत् । \newline
38. ए॒त च्छिरः॒ शिर॑ ए॒त दे॒त च्छिरः॑ । \newline
39. शिरो॒ यद् यच्छिरः॒ शिरो॒ यत् । \newline
40. यदु॒खोखा यद् यदु॒खा । \newline
41. उ॒खा तस्मा॒त् तस्मा॑ दु॒खोखा तस्मा᳚त् । \newline
42. तस्मा॑ दे॒व मे॒वम् तस्मा॒त् तस्मा॑ दे॒वम् । \newline
43. ए॒व मा॑हाहै॒व मे॒व मा॑ह । \newline
44. आ॒ह॒ य॒ज्ञ्स्य॑ य॒ज्ञ् स्या॑हाह य॒ज्ञ्स्य॑ । \newline
45. य॒ज्ञ्स्य॑ प॒दे प॒दे य॒ज्ञ्स्य॑ य॒ज्ञ्स्य॑ प॒दे । \newline
46. प॒दे स्थः॑ स्थः प॒दे प॒दे स्थः॑ । \newline
47. प॒दे इति॑ प॒दे । \newline
48. स्थ॒ इतीति॑ स्थः स्थ॒ इति॑ । \newline
49. इत्या॑हा॒हे तीत्या॑ह । \newline
50. आ॒ह॒ य॒ज्ञ्स्य॑ य॒ज्ञ् स्या॑हाह य॒ज्ञ्स्य॑ । \newline
51. य॒ज्ञ्स्य॒ हि हि य॒ज्ञ्स्य॑ य॒ज्ञ्स्य॒ हि । \newline
52. ह्ये॑ते ए॒ते हि ह्ये॑ते । \newline
53. ए॒ते प॒दे प॒दे ए॒ते ए॒ते प॒दे । \newline
54. ए॒ते इत्ये॒ते । \newline

\textbf{Ghana Paata } \newline

1. सृ॒ज॒ति॒ य॒ज्ञो य॒ज्ञ्ः सृ॑जति सृजति य॒ज्ञो वै वै य॒ज्ञ्ः सृ॑जति सृजति य॒ज्ञो वै । \newline
2. य॒ज्ञो वै वै य॒ज्ञो य॒ज्ञो वै कृ॑ष्णाजि॒नम् कृ॑ष्णाजि॒नं ॅवै य॒ज्ञो य॒ज्ञो वै कृ॑ष्णाजि॒नम् । \newline
3. वै कृ॑ष्णाजि॒नम् कृ॑ष्णाजि॒नं ॅवै वै कृ॑ष्णाजि॒नं ॅय॒ज्ञेन॑ य॒ज्ञेन॑ कृष्णाजि॒नं ॅवै वै कृ॑ष्णाजि॒नं ॅय॒ज्ञेन॑ । \newline
4. कृ॒ष्णा॒जि॒नं ॅय॒ज्ञेन॑ य॒ज्ञेन॑ कृष्णाजि॒नम् कृ॑ष्णाजि॒नं ॅय॒ज्ञेनै॒वैव य॒ज्ञेन॑ कृष्णाजि॒नम् कृ॑ष्णाजि॒नं ॅय॒ज्ञेनै॒व । \newline
5. कृ॒ष्णा॒जि॒नमिति॑ कृष्ण - अ॒जि॒नम् । \newline
6. य॒ज्ञेनै॒वैव य॒ज्ञेन॑ य॒ज्ञेनै॒व य॒ज्ञ्ं ॅय॒ज्ञ् मे॒व य॒ज्ञेन॑ य॒ज्ञेनै॒व य॒ज्ञ्म् । \newline
7. ए॒व य॒ज्ञ्ं ॅय॒ज्ञ् मे॒वैव य॒ज्ञ्ꣳ सꣳ सं ॅय॒ज्ञ् मे॒वैव य॒ज्ञ्ꣳ सम् । \newline
8. य॒ज्ञ्ꣳ सꣳ सं ॅय॒ज्ञ्ं ॅय॒ज्ञ्ꣳ सꣳ सृ॑जति सृजति॒ सं ॅय॒ज्ञ्ं ॅय॒ज्ञ्ꣳ सꣳ सृ॑जति । \newline
9. सꣳ सृ॑जति सृजति॒ सꣳ सꣳ सृ॑जति रु॒द्रा रु॒द्राः सृ॑जति॒ सꣳ सꣳ सृ॑जति रु॒द्राः । \newline
10. सृ॒ज॒ति॒ रु॒द्रा रु॒द्राः सृ॑जति सृजति रु॒द्राः सं॒भृत्य॑ सं॒भृत्य॑ रु॒द्राः सृ॑जति सृजति रु॒द्राः सं॒भृत्य॑ । \newline
11. रु॒द्राः सं॒भृत्य॑ सं॒भृत्य॑ रु॒द्रा रु॒द्राः सं॒भृत्य॑ पृथि॒वीम् पृ॑थि॒वीꣳ सं॒भृत्य॑ रु॒द्रा रु॒द्राः सं॒भृत्य॑ पृथि॒वीम् । \newline
12. सं॒भृत्य॑ पृथि॒वीम् पृ॑थि॒वीꣳ सं॒भृत्य॑ सं॒भृत्य॑ पृथि॒वी मितीति॑ पृथि॒वीꣳ सं॒भृत्य॑ सं॒भृत्य॑ पृथि॒वी मिति॑ । \newline
13. सं॒भृत्येति॑ सं - भृत्य॑ । \newline
14. पृ॒थि॒वी मितीति॑ पृथि॒वीम् पृ॑थि॒वी मित्या॑हा॒हेति॑ पृथि॒वीम् पृ॑थि॒वी मित्या॑ह । \newline
15. इत्या॑हा॒हे तीत्या॑ है॒ता ए॒ता आ॒हे तीत्या॑ है॒ताः । \newline
16. आ॒है॒ता ए॒ता आ॑हाहै॒ता वै वा ए॒ता आ॑हाहै॒ता वै । \newline
17. ए॒ता वै वा ए॒ता ए॒ता वा ए॒त मे॒तं ॅवा ए॒ता ए॒ता वा ए॒तम् । \newline
18. वा ए॒त मे॒तं ॅवै वा ए॒तम् दे॒वता॑ दे॒वता॑ ए॒तं ॅवै वा ए॒तम् दे॒वताः᳚ । \newline
19. ए॒तम् दे॒वता॑ दे॒वता॑ ए॒त मे॒तम् दे॒वता॒ अग्रे ऽग्रे॑ दे॒वता॑ ए॒त मे॒तम् दे॒वता॒ अग्रे᳚ । \newline
20. दे॒वता॒ अग्रे ऽग्रे॑ दे॒वता॑ दे॒वता॒ अग्रे॒ सꣳ स मग्रे॑ दे॒वता॑ दे॒वता॒ अग्रे॒ सम् । \newline
21. अग्रे॒ सꣳ स मग्रे ऽग्रे॒ स म॑भरन् नभर॒न् थ्स मग्रे ऽग्रे॒ स म॑भरन्न् । \newline
22. स म॑भरन् नभर॒न् थ्सꣳ स म॑भर॒न् ताभि॒ स्ताभि॑ रभर॒न् थ्सꣳ स म॑भर॒न् ताभिः॑ । \newline
23. अ॒भ॒र॒न् ताभि॒ स्ताभि॑ रभरन् नभर॒न् ताभि॑ रे॒वैव ताभि॑ रभरन् नभर॒न् ताभि॑ रे॒व । \newline
24. ताभि॑ रे॒वैव ताभि॒ स्ताभि॑ रे॒वैन॑ मेन मे॒व ताभि॒ स्ताभि॑ रे॒वैन᳚म् । \newline
25. ए॒वैन॑ मेन मे॒वैवैनꣳ॒॒ सꣳ स मे॑न मे॒वैवैनꣳ॒॒ सम् । \newline
26. ए॒नꣳ॒॒ सꣳ स मे॑न मेनꣳ॒॒ सम् भ॑रति भरति॒ स मे॑न मेनꣳ॒॒ सम् भ॑रति । \newline
27. सम् भ॑रति भरति॒ सꣳ सम् भ॑रति म॒खस्य॑ म॒खस्य॑ भरति॒ सꣳ सम् भ॑रति म॒खस्य॑ । \newline
28. भ॒र॒ति॒ म॒खस्य॑ म॒खस्य॑ भरति भरति म॒खस्य॒ शिरः॒ शिरो॑ म॒खस्य॑ भरति भरति म॒खस्य॒ शिरः॑ । \newline
29. म॒खस्य॒ शिरः॒ शिरो॑ म॒खस्य॑ म॒खस्य॒ शिरो᳚ ऽस्यसि॒ शिरो॑ म॒खस्य॑ म॒खस्य॒ शिरो॑ ऽसि । \newline
30. शिरो᳚ ऽस्यसि॒ शिरः॒ शिरो॒ ऽसीती त्य॑सि॒ शिरः॒ शिरो॒ ऽसीति॑ । \newline
31. अ॒सीती त्य॑स्य॒सी त्या॑हा॒हे त्य॑स्य॒सीत्या॑ह । \newline
32. इत्या॑हा॒हे तीत्या॑ह य॒ज्ञो य॒ज्ञ् आ॒हे तीत्या॑ह य॒ज्ञ्ः । \newline
33. आ॒ह॒ य॒ज्ञो य॒ज्ञ् आ॑हाह य॒ज्ञो वै वै य॒ज्ञ् आ॑हाह य॒ज्ञो वै । \newline
34. य॒ज्ञो वै वै य॒ज्ञो य॒ज्ञो वै म॒खो म॒खो वै य॒ज्ञो य॒ज्ञो वै म॒खः । \newline
35. वै म॒खो म॒खो वै वै म॒ख स्तस्य॒ तस्य॑ म॒खो वै वै म॒ख स्तस्य॑ । \newline
36. म॒ख स्तस्य॒ तस्य॑ म॒खो म॒ख स्तस्यै॒त दे॒तत् तस्य॑ म॒खो म॒ख स्तस्यै॒तत् । \newline
37. तस्यै॒त दे॒तत् तस्य॒ तस्यै॒त च्छिरः॒ शिर॑ ए॒तत् तस्य॒ तस्यै॒त च्छिरः॑ । \newline
38. ए॒त च्छिरः॒ शिर॑ ए॒त दे॒त च्छिरो॒ यद् यच्छिर॑ ए॒त दे॒त च्छिरो॒ यत् । \newline
39. शिरो॒ यद् यच्छिरः॒ शिरो॒ यदु॒खोखा यच्छिरः॒ शिरो॒ यदु॒खा । \newline
40. यदु॒खोखा यद् यदु॒खा तस्मा॒त् तस्मा॑ दु॒खा यद् यदु॒खा तस्मा᳚त् । \newline
41. उ॒खा तस्मा॒त् तस्मा॑ दु॒खोखा तस्मा॑ दे॒व मे॒वम् तस्मा॑ दु॒खोखा तस्मा॑ दे॒वम् । \newline
42. तस्मा॑ दे॒व मे॒वम् तस्मा॒त् तस्मा॑ दे॒व मा॑हाहै॒वम् तस्मा॒त् तस्मा॑ दे॒व मा॑ह । \newline
43. ए॒व मा॑हाहै॒व मे॒व मा॑ह य॒ज्ञ्स्य॑ य॒ज्ञ्स्या॑है॒व मे॒व मा॑ह य॒ज्ञ्स्य॑ । \newline
44. आ॒ह॒ य॒ज्ञ्स्य॑ य॒ज्ञ्स्या॑हाह य॒ज्ञ्स्य॑ प॒दे प॒दे य॒ज्ञ्स्या॑हाह य॒ज्ञ्स्य॑ प॒दे । \newline
45. य॒ज्ञ्स्य॑ प॒दे प॒दे य॒ज्ञ्स्य॑ य॒ज्ञ्स्य॑ प॒दे स्थः॑ स्थः प॒दे य॒ज्ञ्स्य॑ य॒ज्ञ्स्य॑ प॒दे स्थः॑ । \newline
46. प॒दे स्थः॑ स्थः प॒दे प॒दे स्थ॒ इतीति॑ स्थः प॒दे प॒दे स्थ॒ इति॑ । \newline
47. प॒दे इति॑ प॒दे । \newline
48. स्थ॒ इतीति॑ स्थः स्थ॒ इत्या॑ हा॒हेति॑ स्थः स्थ॒ इत्या॑ह । \newline
49. इत्या॑हा॒हे तीत्या॑ह य॒ज्ञ्स्य॑ य॒ज्ञ्स्या॒हे तीत्या॑ह य॒ज्ञ्स्य॑ । \newline
50. आ॒ह॒ य॒ज्ञ्स्य॑ य॒ज्ञ्स्या॑हाह य॒ज्ञ्स्य॒ हि हि य॒ज्ञ्स्या॑हाह य॒ज्ञ्स्य॒ हि । \newline
51. य॒ज्ञ्स्य॒ हि हि य॒ज्ञ्स्य॑ य॒ज्ञ्स्य॒ ह्ये॑ते ए॒ते हि य॒ज्ञ्स्य॑ य॒ज्ञ्स्य॒ ह्ये॑ते । \newline
52. ह्ये॑ते ए॒ते हि ह्ये॑ते प॒दे प॒दे ए॒ते हि ह्ये॑ते प॒दे । \newline
53. ए॒ते प॒दे प॒दे ए॒ते ए॒ते प॒दे अथो॒ अथो॑ प॒दे ए॒ते ए॒ते प॒दे अथो᳚ । \newline
54. ए॒ते इत्ये॒ते । \newline
\pagebreak
\markright{ TS 5.1.6.4  \hfill https://www.vedavms.in \hfill}

\section{ TS 5.1.6.4 }

\textbf{TS 5.1.6.4 } \newline
\textbf{Samhita Paata} \newline

प॒दे अथो॒ प्रति॑ष्ठित्यै॒ प्रान्याभि॒-र्यच्छ॒त्यन्व॒न्यै-र्म॑न्त्रयते मिथुन॒त्वाय॒ त्र्यु॑द्धिं करोति॒ त्रय॑ इ॒मे लो॒का ए॒षां ॅलो॒काना॒माप्त्यै॒ छन्दो॑भिः करोति वी॒र्यं॑ ॅवै छन्दाꣳ॑सि वी॒र्ये॑णै॒वैनां᳚ करोति॒ यजु॑षा॒ बिलं॑ करोति॒ व्यावृ॑त्त्या॒ इय॑तीं करोति प्र॒जाप॑तिना यज्ञ्मु॒खेन॒ संमि॑तां द्विस्त॒नां क॑रोति॒ यावा॑पृथि॒व्योर्दोहा॑य॒ चतुः॑स्तनां करोति पशू॒नां दोहा॑या॒ष्टास्त॑नां करोति॒ छन्द॑सां॒ दोहा॑य॒ नवा᳚श्रि-मभि॒चर॑तः ( ) कुर्यात् त्रि॒वृत॑मे॒व वज्रꣳ॑ स॒भृंत्य॒ भ्रातृ॑व्याय॒ प्रह॑रति॒ स्तृत्यै॑ कृ॒त्वाय॒ सा म॒हीमु॒खामिति॒ नि द॑धाति दे॒वता᳚स्वे॒वैनां॒ प्रति॑ष्ठापयति ॥ \newline

\textbf{Pada Paata} \newline

प॒दे इति॑ । अथो॒ इति॑ । प्रति॑ष्ठित्या॒ इति॒ प्रति॑ - स्थि॒त्यै॒ । प्रेति॑ । अ॒न्याभिः॑ । यच्छ॑ति । अन्विति॑ । अ॒न्यैः । म॒न्त्र॒य॒ते॒ । मि॒थु॒न॒त्वायेति॑ मिथुन - त्वाय॑ । त्र्यु॑द्धि॒मिति॒ त्रि - उ॒द्धि॒म् । क॒रो॒ति॒ । त्रयः॑ । इ॒मे । लो॒काः । ए॒षाम् । लो॒काना᳚म् । आप्त्यै᳚ । छन्दो॑भि॒रिति॒ छन्दः॑-भिः॒ । क॒रो॒ति॒ । वी॒र्य᳚म् । वै । छन्दाꣳ॑सि । वी॒र्ये॑ण । ए॒व । ए॒ना॒म् । क॒रो॒ति॒ । यजु॑षा । बिल᳚म् । क॒रो॒ति॒ । व्यावृ॑त्त्या॒ इति॑ वि - आवृ॑त्त्यै । इय॑तीम् । क॒रो॒ति॒ । प्र॒जाप॑ति॒नेति॑ प्र॒जा - प॒ति॒ना॒ । य॒ज्ञ्॒मु॒खेनेति॑ यज्ञ्-मु॒खेन॑ । संमि॑ता॒मिति॒ सं - मि॒ता॒म् । द्वि॒स्त॒नामिति॑ द्वि - स्त॒नाम् । क॒रो॒ति॒ । द्यावा॑पृथि॒व्योरिति॒ द्यावा᳚ - पृ॒थि॒व्योः । दोहा॑य । चतुः॑स्तना॒मिति॒ चतुः॑ - स्त॒ना॒म् । क॒रो॒ति॒ । प॒शू॒नाम् । दोहा॑य । अ॒ष्टास्त॑ना॒मित्य॒ष्टा - स्त॒ना॒म् । क॒रो॒ति॒ । छन्द॑साम् । दोहा॑य । नवा᳚श्रि॒मिति॒ नव॑ - अ॒श्रि॒म् । अ॒भि॒चर॑त॒ इत्य॑भि - चर॑तः ( ) । कु॒र्या॒त् । त्रि॒वृत॒मिति॑ त्रि - वृत᳚म् । ए॒व । वज्र᳚म् । स॒भृंत्येति॑ सं - भृत्य॑ । भ्रातृ॑व्याय । प्रेति॑ । ह॒र॒ति॒ । स्तृत्यै᳚ । कृ॒त्वाय॑ । सा । म॒हीम् । उ॒खाम् । इति॑ । नीति॑ । द॒धा॒ति॒ । दे॒वता॑सु । ए॒व । ए॒ना॒म् । प्रतीति॑ । स्था॒प॒य॒ति॒ ॥  \newline


\textbf{Krama Paata} \newline

प॒दे अथो᳚ । प॒दे इति॑ प॒दे । अथो॒ प्रति॑ष्ठित्यै । अथो॒ इत्यथो᳚ । प्रति॑ष्ठित्यै॒ प्र । प्रति॑ष्ठित्या॒ इति॒ प्रति॑ - स्थि॒त्यै॒ । प्रान्याभिः॑ । अ॒न्याभि॒र् यच्छ॑ति । यच्छ॒त्यनु॑ । अन्व॒न्यैः । अ॒न्व॒न्यैर् म॑न्त्रयते । म॒न्त्र॒य॒ते॒ मि॒थु॒न॒त्वाय॑ । मि॒थु॒न॒त्वाय॒ त्र्यु॑द्धिम् । मि॒थु॒न॒त्वायेति॑ मिथुन - त्वाय॑ । त्र्यु॑द्धिम् करोति । त्र्यु॑द्धि॒मिति॒ त्रि - उ॒द्धि॒म् । क॒रो॒ति॒ त्रयः॑ । त्रय॑ इ॒मे । इ॒मे लो॒काः । लो॒का ए॒षाम् । ए॒षाम् ॅलो॒काना᳚म् । लो॒काना॒माप्त्यै᳚ । आप्त्यै॒ छन्दो॑भिः । छन्दो॑भिः करोति । छन्दो॑भि॒रिति॒ छन्द॑ - भिः॒ । क॒रो॒ति॒ वी॒र्य᳚म् । वी॒र्य॑म् ॅवै । वै छन्दाꣳ॑सि । छन्दाꣳ॑सि वी॒र्ये॑ण । वी॒र्ये॑णै॒व । ए॒वैना᳚म् । ए॒ना॒म् क॒रो॒ति॒ । क॒रो॒ति॒ यजु॑षा । यजु॑षा॒ बिल᳚म् । बिल॑म् करोति । क॒रो॒ति॒ व्यावृ॑त्त्यै । व्यावृ॑त्त्या॒ इय॑तीम् । व्यावृ॑त्त्या॒ इति॑ वि - आवृ॑त्त्यै । इय॑तीम् करोति । क॒रो॒ति॒ प्र॒जाप॑तिना । प्र॒जाप॑तिना यज्ञ्मु॒खेन॑ । प्र॒जाप॑ति॒नेति॑ प्र॒जा - प॒ति॒ना॒ । य॒ज्ञ्॒मु॒खेन॒ सम्मि॑ताम् । य॒ज्ञ्॒मु॒खेनेति॑ यज्ञ् - मु॒खेन॑ । सम्मि॑ताम् द्विस्त॒नाम् । सम्मि॑ता॒मिति॒ सम् - मि॒ता॒म् । द्वि॒स्त॒नाम् क॑रोति । द्वि॒स्त॒नामिति॑ द्वि - स्त॒नाम् । क॒रो॒ति॒ द्यावा॑पृथि॒व्योः । द्यावा॑पृथि॒व्योर् दोहा॑य । द्यावा॑पृथि॒व्योरिति॒ द्यावा᳚ - पृ॒थि॒व्योः । दोहा॑य॒ चतु॑स्स्तनाम् । चतु॑स्स्तनाम् करोति । चतु॑स्स्तना॒मिति॒ चतुः॑ - स्त॒ना॒म् । क॒रो॒ति॒ प॒शू॒नाम् । प॒शू॒नाम् दोहा॑य । दोहा॑या॒ष्टास्त॑नाम् । अ॒ष्टास्त॑नाम् करोति । अ॒ष्टास्त॑ना॒मित्य॒ष्टा - स्त॒ना॒म् । क॒रो॒ति॒ छन्द॑साम् । छन्द॑सा॒म् दोहा॑य । दोहा॑य॒ नवा᳚श्रिम् । नवा᳚श्रिमभि॒चर॑तः ( ) । नवा᳚श्रि॒मिति॒ नव॑ - अ॒श्रि॒म् । अ॒भि॒चर॑तः कुर्यात् । अ॒भि॒चर॑त॒ इत्य॑भि - चर॑तः । कु॒र्या॒त् त्रि॒वृत᳚म् । त्रि॒वृत॑मे॒व । त्रि॒वृत॒मिति॑ त्रि - वृत᳚म् । ए॒व वज्र᳚म् । वज्रꣳ॑ स॒म्भृत्य॑ । स॒म्भृत्य॒ भ्रातृ॑व्याय । स॒म्भृत्येति॑ सम् - भृत्य॑ । भ्रातृ॑व्याय॒ प्र । प्र ह॑रति । ह॒र॒ति॒ स्तृत्यै᳚ । स्तृत्यै॑ कृ॒त्वाय॑ । कृ॒त्वाय॒ सा । सा म॒हीम् । म॒हीमु॒खाम् । उ॒खामिति॑ । इति॒ नि । नि द॑धाति । द॒धा॒ति॒ दे॒वता॑सु । दे॒वता᳚स्वे॒व । ए॒वैना᳚म् । ए॒ना॒म् प्रति॑ । प्रति॑ ष्ठापयति । स्था॒प॒य॒तीति॑ स्थापयति । \newline

\textbf{Jatai Paata} \newline

1. प॒दे अथो॒ अथो॑ प॒दे प॒दे अथो᳚ । \newline
2. प॒दे इति॑ प॒दे । \newline
3. अथो॒ प्रति॑ष्ठित्यै॒ प्रति॑ष्ठित्या॒ अथो॒ अथो॒ प्रति॑ष्ठित्यै । \newline
4. अथो॒ इत्यथो᳚ । \newline
5. प्रति॑ष्ठित्यै॒ प्र प्र प्रति॑ष्ठित्यै॒ प्रति॑ष्ठित्यै॒ प्र । \newline
6. प्रति॑ष्ठित्या॒ इति॒ प्रति॑ - स्थि॒त्यै॒ । \newline
7. प्रान्याभि॑ र॒न्याभिः॒ प्र प्रान्याभिः॑ । \newline
8. अ॒न्याभि॒र् यच्छ॑ति॒ यच्छ॑ त्य॒न्याभि॑ र॒न्याभि॒र् यच्छ॑ति । \newline
9. यच्छ॒ त्यन्वनु॒ यच्छ॑ति॒ यच्छ॒ त्यनु॑ । \newline
10. अन्व॒न्यै र॒न्यै रन्वन् व॒न्यैः । \newline
11. अ॒न्यैर् म॑न्त्रयते मन्त्रयते॒ ऽन्यै र॒न्यैर् म॑न्त्रयते । \newline
12. म॒न्त्र॒य॒ते॒ मि॒थु॒न॒त्वाय॑ मिथुन॒त्वाय॑ मन्त्रयते मन्त्रयते मिथुन॒त्वाय॑ । \newline
13. मि॒थु॒न॒त्वाय॒ त्र्यु॑द्धि॒म् त्र्यु॑द्धिम् मिथुन॒त्वाय॑ मिथुन॒त्वाय॒ त्र्यु॑द्धिम् । \newline
14. मि॒थु॒न॒त्वायेति॑ मिथुन - त्वाय॑ । \newline
15. त्र्यु॑द्धिम् करोति करोति॒ त्र्यु॑द्धि॒म् त्र्यु॑द्धिम् करोति । \newline
16. त्र्यु॑द्धि॒मिति॒ त्रि - उ॒द्धि॒म् । \newline
17. क॒रो॒ति॒ त्रय॒ स्त्रयः॑ करोति करोति॒ त्रयः॑ । \newline
18. त्रय॑ इ॒म इ॒मे त्रय॒ स्त्रय॑ इ॒मे । \newline
19. इ॒मे लो॒का लो॒का इ॒म इ॒मे लो॒काः । \newline
20. लो॒का ए॒षा मे॒षाम् ॅलो॒का लो॒का ए॒षाम् । \newline
21. ए॒षाम् ॅलो॒काना᳚म् ॅलो॒काना॑ मे॒षा मे॒षाम् ॅलो॒काना᳚म् । \newline
22. लो॒काना॒ माप्त्या॒ आप्त्यै॑ लो॒काना᳚म् ॅलो॒काना॒ माप्त्यै᳚ । \newline
23. आप्त्यै॒ छन्दो॑भि॒ श्छन्दो॑भि॒ राप्त्या॒ आप्त्यै॒ छन्दो॑भिः । \newline
24. छन्दो॑भिः करोति करोति॒ छन्दो॑भि॒ श्छन्दो॑भिः करोति । \newline
25. छन्दो॑भि॒रिति॒ छन्दः॑ - भिः॒ । \newline
26. क॒रो॒ति॒ वी॒र्यं॑ ॅवी॒र्य॑म् करोति करोति वी॒र्य᳚म् । \newline
27. वी॒र्यं॑ ॅवै वै वी॒र्यं॑ ॅवी॒र्यं॑ ॅवै । \newline
28. वै छन्दाꣳ॑सि॒ छन्दाꣳ॑सि॒ वै वै छन्दाꣳ॑सि । \newline
29. छन्दाꣳ॑सि वी॒र्ये॑ण वी॒र्ये॑ण॒ छन्दाꣳ॑सि॒ छन्दाꣳ॑सि वी॒र्ये॑ण । \newline
30. वी॒र्ये॑ णै॒वैव वी॒र्ये॑ण वी॒र्ये॑ णै॒व । \newline
31. ए॒वैना॑ मेना मे॒वैवैना᳚म् । \newline
32. ए॒ना॒म् क॒रो॒ति॒ क॒रो॒ त्ये॒ना॒ मे॒ना॒म् क॒रो॒ति॒ । \newline
33. क॒रो॒ति॒ यजु॑षा॒ यजु॑षा करोति करोति॒ यजु॑षा । \newline
34. यजु॑षा॒ बिल॒म् बिलं॒ ॅयजु॑षा॒ यजु॑षा॒ बिल᳚म् । \newline
35. बिल॑म् करोति करोति॒ बिल॒म् बिल॑म् करोति । \newline
36. क॒रो॒ति॒ व्यावृ॑त्त्यै॒ व्यावृ॑त्त्यै करोति करोति॒ व्यावृ॑त्त्यै । \newline
37. व्यावृ॑त्त्या॒ इय॑ती॒ मिय॑तीं॒ ॅव्यावृ॑त्त्यै॒ व्यावृ॑त्त्या॒ इय॑तीम् । \newline
38. व्यावृ॑त्त्या॒ इति॑ वि - आवृ॑त्त्यै । \newline
39. इय॑तीम् करोति करो॒तीय॑ती॒ मिय॑तीम् करोति । \newline
40. क॒रो॒ति॒ प्र॒जाप॑तिना प्र॒जाप॑तिना करोति करोति प्र॒जाप॑तिना । \newline
41. प्र॒जाप॑तिना यज्ञ्मु॒खेन॑ यज्ञ्मु॒खेन॑ प्र॒जाप॑तिना प्र॒जाप॑तिना यज्ञ्मु॒खेन॑ । \newline
42. प्र॒जाप॑ति॒नेति॑ प्र॒जा - प॒ति॒ना॒ । \newline
43. य॒ज्ञ्॒मु॒खेन॒ सम्मि॑ताꣳ॒॒ सम्मि॑तां ॅयज्ञ्मु॒खेन॑ यज्ञ्मु॒खेन॒ सम्मि॑ताम् । \newline
44. य॒ज्ञ्॒मु॒खेनेति॑ यज्ञ् - मु॒खेन॑ । \newline
45. सम्मि॑ताम् द्विस्त॒नाम् द्वि॑स्त॒नाꣳ सम्मि॑ताꣳ॒॒ सम्मि॑ताम् द्विस्त॒नाम् । \newline
46. सम्मि॑ता॒मिति॒ सं - मि॒ता॒म् । \newline
47. द्वि॒स्त॒नाम् क॑रोति करोति द्विस्त॒नाम् द्वि॑स्त॒नाम् क॑रोति । \newline
48. द्वि॒स्त॒नामिति॑ द्वि - स्त॒नाम् । \newline
49. क॒रो॒ति॒ द्यावा॑पृथि॒व्योर् द्यावा॑पृथि॒व्योः क॑रोति करोति॒ द्यावा॑पृथि॒व्योः । \newline
50. द्यावा॑पृथि॒व्योर् दोहा॑य॒ दोहा॑य॒ द्यावा॑पृथि॒व्योर् द्यावा॑पृथि॒व्योर् दोहा॑य । \newline
51. द्यावा॑पृथि॒व्योरिति॒ द्यावा᳚ - पृ॒थि॒व्योः । \newline
52. दोहा॑य॒ चतुः॑स्तना॒म् चतुः॑स्तना॒म् दोहा॑य॒ दोहा॑य॒ चतुः॑स्तनाम् । \newline
53. चतुः॑स्तनाम् करोति करोति॒ चतुः॑स्तना॒म् चतुः॑स्तनाम् करोति । \newline
54. चतुः॑स्तना॒मिति॒ चतुः॑ - स्त॒ना॒म् । \newline
55. क॒रो॒ति॒ प॒शू॒नाम् प॑शू॒नाम् क॑रोति करोति पशू॒नाम् । \newline
56. प॒शू॒नाम् दोहा॑य॒ दोहा॑य पशू॒नाम् प॑शू॒नाम् दोहा॑य । \newline
57. दोहा॑या॒ ष्टास्त॑ना म॒ष्टास्त॑ना॒म् दोहा॑य॒ दोहा॑या॒ ष्टास्त॑नाम् । \newline
58. अ॒ष्टास्त॑नाम् करोति करो त्य॒ष्टास्त॑ना म॒ष्टास्त॑नाम् करोति । \newline
59. अ॒ष्टास्त॑ना॒मित्य॒ष्टा - स्त॒ना॒म् । \newline
60. क॒रो॒ति॒ छन्द॑सा॒म् छन्द॑साम् करोति करोति॒ छन्द॑साम् । \newline
61. छन्द॑सा॒म् दोहा॑य॒ दोहा॑य॒ छन्द॑सा॒म् छन्द॑सा॒म् दोहा॑य । \newline
62. दोहा॑य॒ नवा᳚श्रि॒म् नवा᳚श्रि॒म् दोहा॑य॒ दोहा॑य॒ नवा᳚श्रिम् । \newline
63. नवा᳚श्रि मभि॒चर॑तो ऽभि॒चर॑तो॒ नवा᳚श्रि॒म् नवा᳚श्रि मभि॒चर॑तः । \newline
64. नवा᳚श्रि॒मिति॒ नव॑ - अ॒श्रि॒म् । \newline
65. अ॒भि॒चर॑तः कुर्यात् कुर्या दभि॒चर॑तो ऽभि॒चर॑तः कुर्यात् । \newline
66. अ॒भि॒चर॑त॒ इत्य॑भि - चर॑तः । \newline
67. कु॒र्या॒त् त्रि॒वृत॑म् त्रि॒वृत॑म् कुर्यात् कुर्यात् त्रि॒वृत᳚म् । \newline
68. त्रि॒वृत॑ मे॒वैव त्रि॒वृत॑म् त्रि॒वृत॑ मे॒व । \newline
69. त्रि॒वृत॒मिति॑ त्रि - वृत᳚म् । \newline
70. ए॒व वज्रं॒ ॅवज्र॑ मे॒वैव वज्र᳚म् । \newline
71. वज्रꣳ॑ सं॒भृत्य॑ सं॒भृत्य॒ वज्रं॒ ॅवज्रꣳ॑ सं॒भृत्य॑ । \newline
72. सं॒भृत्य॒ भ्रातृ॑व्याय॒ भ्रातृ॑व्याय सं॒भृत्य॑ सं॒भृत्य॒ भ्रातृ॑व्याय । \newline
73. सं॒भृत्येति॑ सं - भृत्य॑ । \newline
74. भ्रातृ॑व्याय॒ प्र प्र भ्रातृ॑व्याय॒ भ्रातृ॑व्याय॒ प्र । \newline
75. प्र ह॑रति हरति॒ प्र प्र ह॑रति । \newline
76. ह॒र॒ति॒ स्तृत्यै॒ स्तृत्यै॑ हरति हरति॒ स्तृत्यै᳚ । \newline
77. स्तृत्यै॑ कृ॒त्वाय॑ कृ॒त्वाय॒ स्तृत्यै॒ स्तृत्यै॑ कृ॒त्वाय॑ । \newline
78. कृ॒त्वाय॒ सा सा कृ॒त्वाय॑ कृ॒त्वाय॒ सा । \newline
79. सा म॒हीम् म॒हीꣳ सा सा म॒हीम् । \newline
80. म॒ही मु॒खा मु॒खाम् म॒हीम् म॒ही मु॒खाम् । \newline
81. उ॒खा मितीत्यु॒खा मु॒खा मिति॑ । \newline
82. इति॒ नि नीतीति॒ नि । \newline
83. नि द॑धाति दधाति॒ नि नि द॑धाति । \newline
84. द॒धा॒ति॒ दे॒वता॑सु दे॒वता॑सु दधाति दधाति दे॒वता॑सु । \newline
85. दे॒वता᳚ स्वे॒वैव दे॒वता॑सु दे॒वता᳚ स्वे॒व । \newline
86. ए॒वैना॑ मेना मे॒वैवैना᳚म् । \newline
87. ए॒ना॒म् प्रति॒ प्रत्ये॑ना मेना॒म् प्रति॑ । \newline
88. प्रति॑ ष्ठापयति स्थापयति॒ प्रति॒ प्रति॑ ष्ठापयति । \newline
89. स्था॒प॒य॒तीति॑ स्थापयति । \newline

\textbf{Ghana Paata } \newline

1. प॒दे अथो॒ अथो॑ प॒दे प॒दे अथो॒ प्रति॑ष्ठित्यै॒ प्रति॑ष्ठित्या॒ अथो॑ प॒दे प॒दे अथो॒ प्रति॑ष्ठित्यै । \newline
2. प॒दे इति॑ प॒दे । \newline
3. अथो॒ प्रति॑ष्ठित्यै॒ प्रति॑ष्ठित्या॒ अथो॒ अथो॒ प्रति॑ष्ठित्यै॒ प्र प्र प्रति॑ष्ठित्या॒ अथो॒ अथो॒ प्रति॑ष्ठित्यै॒ प्र । \newline
4. अथो॒ इत्यथो᳚ । \newline
5. प्रति॑ष्ठित्यै॒ प्र प्र प्रति॑ष्ठित्यै॒ प्रति॑ष्ठित्यै॒ प्रान्याभि॑ र॒न्याभिः॒ प्र प्रति॑ष्ठित्यै॒ प्रति॑ष्ठित्यै॒ प्रान्याभिः॑ । \newline
6. प्रति॑ष्ठित्या॒ इति॒ प्रति॑ - स्थि॒त्यै॒ । \newline
7. प्रान्याभि॑ र॒न्याभिः॒ प्र प्रान्याभि॒र् यच्छ॑ति॒ यच्छ॑ त्य॒न्याभिः॒ प्र प्रान्याभि॒र् यच्छ॑ति । \newline
8. अ॒न्याभि॒र् यच्छ॑ति॒ यच्छ॑ त्य॒न्याभि॑ र॒न्याभि॒र् यच्छ॒ त्यन्वनु॒ यच्छ॑ त्य॒न्याभि॑ र॒न्याभि॒र् यच्छ॒ त्यनु॑ । \newline
9. यच्छ॒ त्यन्वनु॒ यच्छ॑ति॒ यच्छ॒ त्यन्व॒न्यै र॒न्यै रनु॒ यच्छ॑ति॒ यच्छ॒ त्यन्व॒न्यैः । \newline
10. अन्व॒न्यै र॒न्यै रन्वन् व॒न्यैर् म॑न्त्रयते मन्त्रयते॒ ऽन्यै रन्वन् व॒न्यैर् म॑न्त्रयते । \newline
11. अ॒न्यैर् म॑न्त्रयते मन्त्रयते॒ ऽन्यै र॒न्यैर् म॑न्त्रयते मिथुन॒त्वाय॑ मिथुन॒त्वाय॑ मन्त्रयते॒ ऽन्यै र॒न्यैर् म॑न्त्रयते मिथुन॒त्वाय॑ । \newline
12. म॒न्त्र॒य॒ते॒ मि॒थु॒न॒त्वाय॑ मिथुन॒त्वाय॑ मन्त्रयते मन्त्रयते मिथुन॒त्वाय॒ त्र्यु॑द्धि॒म् त्र्यु॑द्धिम् मिथुन॒त्वाय॑ मन्त्रयते मन्त्रयते मिथुन॒त्वाय॒ त्र्यु॑द्धिम् । \newline
13. मि॒थु॒न॒त्वाय॒ त्र्यु॑द्धि॒म् त्र्यु॑द्धिम् मिथुन॒त्वाय॑ मिथुन॒त्वाय॒ त्र्यु॑द्धिम् करोति करोति॒ त्र्यु॑द्धिम् मिथुन॒त्वाय॑ मिथुन॒त्वाय॒ त्र्यु॑द्धिम् करोति । \newline
14. मि॒थु॒न॒त्वायेति॑ मिथुन - त्वाय॑ । \newline
15. त्र्यु॑द्धिम् करोति करोति॒ त्र्यु॑द्धि॒म् त्र्यु॑द्धिम् करोति॒ त्रय॒ स्त्रयः॑ करोति॒ त्र्यु॑द्धि॒म् त्र्यु॑द्धिम् करोति॒ त्रयः॑ । \newline
16. त्र्यु॑द्धि॒मिति॒ त्रि - उ॒द्धि॒म् । \newline
17. क॒रो॒ति॒ त्रय॒ स्त्रयः॑ करोति करोति॒ त्रय॑ इ॒म इ॒मे त्रयः॑ करोति करोति॒ त्रय॑ इ॒मे । \newline
18. त्रय॑ इ॒म इ॒मे त्रय॒ स्त्रय॑ इ॒मे लो॒का लो॒का इ॒मे त्रय॒ स्त्रय॑ इ॒मे लो॒काः । \newline
19. इ॒मे लो॒का लो॒का इ॒म इ॒मे लो॒का ए॒षा मे॒षाम् ॅलो॒का इ॒म इ॒मे लो॒का ए॒षाम् । \newline
20. लो॒का ए॒षा मे॒षाम् ॅलो॒का लो॒का ए॒षाम् ॅलो॒काना᳚म् ॅलो॒काना॑ मे॒षाम् ॅलो॒का लो॒का ए॒षाम् ॅलो॒काना᳚म् । \newline
21. ए॒षाम् ॅलो॒काना᳚म् ॅलो॒काना॑ मे॒षा मे॒षाम् ॅलो॒काना॒ माप्त्या॒ आप्त्यै॑ लो॒काना॑ मे॒षा मे॒षाम् ॅलो॒काना॒ माप्त्यै᳚ । \newline
22. लो॒काना॒ माप्त्या॒ आप्त्यै॑ लो॒काना᳚म् ॅलो॒काना॒ माप्त्यै॒ छन्दो॑भि॒ श्छन्दो॑भि॒ राप्त्यै॑ लो॒काना᳚म् ॅलो॒काना॒ माप्त्यै॒ छन्दो॑भिः । \newline
23. आप्त्यै॒ छन्दो॑भि॒ श्छन्दो॑भि॒ राप्त्या॒ आप्त्यै॒ छन्दो॑भिः करोति करोति॒ छन्दो॑भि॒ राप्त्या॒ आप्त्यै॒ छन्दो॑भिः करोति । \newline
24. छन्दो॑भिः करोति करोति॒ छन्दो॑भि॒ श्छन्दो॑भिः करोति वी॒र्यं॑ ॅवी॒र्य॑म् करोति॒ छन्दो॑भि॒ श्छन्दो॑भिः करोति वी॒र्य᳚म् । \newline
25. छन्दो॑भि॒रिति॒ छन्दः॑ - भिः॒ । \newline
26. क॒रो॒ति॒ वी॒र्यं॑ ॅवी॒र्य॑म् करोति करोति वी॒र्यं॑ ॅवै वै वी॒र्य॑म् करोति करोति वी॒र्यं॑ ॅवै । \newline
27. वी॒र्यं॑ ॅवै वै वी॒र्यं॑ ॅवी॒र्यं॑ ॅवै छन्दाꣳ॑सि॒ छन्दाꣳ॑सि॒ वै वी॒र्यं॑ ॅवी॒र्यं॑ ॅवै छन्दाꣳ॑सि । \newline
28. वै छन्दाꣳ॑सि॒ छन्दाꣳ॑सि॒ वै वै छन्दाꣳ॑सि वी॒र्ये॑ण वी॒र्ये॑ण॒ छन्दाꣳ॑सि॒ वै वै छन्दाꣳ॑सि वी॒र्ये॑ण । \newline
29. छन्दाꣳ॑सि वी॒र्ये॑ण वी॒र्ये॑ण॒ छन्दाꣳ॑सि॒ छन्दाꣳ॑सि वी॒र्ये॑णै॒वैव वी॒र्ये॑ण॒ छन्दाꣳ॑सि॒ छन्दाꣳ॑सि वी॒र्ये॑णै॒व । \newline
30. वी॒र्ये॑ णै॒वैव वी॒र्ये॑ण वी॒र्ये॑ णै॒वैना॑ मेना मे॒व वी॒र्ये॑ण वी॒र्ये॑ णै॒वैना᳚म् । \newline
31. ए॒वैना॑ मेना मे॒वैवैना᳚म् करोति करो त्येना मे॒वैवैना᳚म् करोति । \newline
32. ए॒ना॒म् क॒रो॒ति॒ क॒रो॒ त्ये॒ना॒ मे॒ना॒म् क॒रो॒ति॒ यजु॑षा॒ यजु॑षा करो त्येना मेनाम् करोति॒ यजु॑षा । \newline
33. क॒रो॒ति॒ यजु॑षा॒ यजु॑षा करोति करोति॒ यजु॑षा॒ बिल॒म् बिलं॒ ॅयजु॑षा करोति करोति॒ यजु॑षा॒ बिल᳚म् । \newline
34. यजु॑षा॒ बिल॒म् बिलं॒ ॅयजु॑षा॒ यजु॑षा॒ बिल॑म् करोति करोति॒ बिलं॒ ॅयजु॑षा॒ यजु॑षा॒ बिल॑म् करोति । \newline
35. बिल॑म् करोति करोति॒ बिल॒म् बिल॑म् करोति॒ व्यावृ॑त्त्यै॒ व्यावृ॑त्त्यै करोति॒ बिल॒म् बिल॑म् करोति॒ व्यावृ॑त्त्यै । \newline
36. क॒रो॒ति॒ व्यावृ॑त्त्यै॒ व्यावृ॑त्त्यै करोति करोति॒ व्यावृ॑त्त्या॒ इय॑ती॒ मिय॑तीं॒ ॅव्यावृ॑त्त्यै करोति करोति॒ व्यावृ॑त्त्या॒ इय॑तीम् । \newline
37. व्यावृ॑त्त्या॒ इय॑ती॒ मिय॑तीं॒ ॅव्यावृ॑त्त्यै॒ व्यावृ॑त्त्या॒ इय॑तीम् करोति करो॒तीय॑तीं॒ ॅव्यावृ॑त्त्यै॒ व्यावृ॑त्त्या॒ इय॑तीम् करोति । \newline
38. व्यावृ॑त्त्या॒ इति॑ वि - आवृ॑त्त्यै । \newline
39. इय॑तीम् करोति करो॒तीय॑ती॒ मिय॑तीम् करोति प्र॒जाप॑तिना प्र॒जाप॑तिना करो॒तीय॑ती॒ मिय॑तीम् करोति प्र॒जाप॑तिना । \newline
40. क॒रो॒ति॒ प्र॒जाप॑तिना प्र॒जाप॑तिना करोति करोति प्र॒जाप॑तिना यज्ञ्मु॒खेन॑ यज्ञ्मु॒खेन॑ प्र॒जाप॑तिना करोति करोति प्र॒जाप॑तिना यज्ञ्मु॒खेन॑ । \newline
41. प्र॒जाप॑तिना यज्ञ्मु॒खेन॑ यज्ञ्मु॒खेन॑ प्र॒जाप॑तिना प्र॒जाप॑तिना यज्ञ्मु॒खेन॒ सम्मि॑ताꣳ॒॒ सम्मि॑तां ॅयज्ञ्मु॒खेन॑ प्र॒जाप॑तिना प्र॒जाप॑तिना यज्ञ्मु॒खेन॒ सम्मि॑ताम् । \newline
42. प्र॒जाप॑ति॒नेति॑ प्र॒जा - प॒ति॒ना॒ । \newline
43. य॒ज्ञ्॒मु॒खेन॒ सम्मि॑ताꣳ॒॒ सम्मि॑तां ॅयज्ञ्मु॒खेन॑ यज्ञ्मु॒खेन॒ सम्मि॑ताम् द्विस्त॒नाम् द्वि॑स्त॒नाꣳ सम्मि॑तां ॅयज्ञ्मु॒खेन॑ यज्ञ्मु॒खेन॒ सम्मि॑ताम् द्विस्त॒नाम् । \newline
44. य॒ज्ञ्॒मु॒खेनेति॑ यज्ञ् - मु॒खेन॑ । \newline
45. सम्मि॑ताम् द्विस्त॒नाम् द्वि॑स्त॒नाꣳ सम्मि॑ताꣳ॒॒ सम्मि॑ताम् द्विस्त॒नाम् क॑रोति करोति द्विस्त॒नाꣳ सम्मि॑ताꣳ॒॒ सम्मि॑ताम् द्विस्त॒नाम् क॑रोति । \newline
46. सम्मि॑ता॒मिति॒ सं - मि॒ता॒म् । \newline
47. द्वि॒स्त॒नाम् क॑रोति करोति द्विस्त॒नाम् द्वि॑स्त॒नाम् क॑रोति॒ द्यावा॑पृथि॒व्योर् द्यावा॑पृथि॒व्योः क॑रोति द्विस्त॒नाम् द्वि॑स्त॒नाम् क॑रोति॒ द्यावा॑पृथि॒व्योः । \newline
48. द्वि॒स्त॒नामिति॑ द्वि - स्त॒नाम् । \newline
49. क॒रो॒ति॒ द्यावा॑पृथि॒व्योर् द्यावा॑पृथि॒व्योः क॑रोति करोति॒ द्यावा॑पृथि॒व्योर् दोहा॑य॒ दोहा॑य॒ द्यावा॑पृथि॒व्योः क॑रोति करोति॒ द्यावा॑पृथि॒व्योर् दोहा॑य । \newline
50. द्यावा॑पृथि॒व्योर् दोहा॑य॒ दोहा॑य॒ द्यावा॑पृथि॒व्योर् द्यावा॑पृथि॒व्योर् दोहा॑य॒ चतुः॑स्तना॒म् चतुः॑स्तना॒म् दोहा॑य॒ द्यावा॑पृथि॒व्योर् द्यावा॑पृथि॒व्योर् दोहा॑य॒ चतुः॑स्तनाम् । \newline
51. द्यावा॑पृथि॒व्योरिति॒ द्यावा᳚ - पृ॒थि॒व्योः । \newline
52. दोहा॑य॒ चतुः॑स्तना॒म् चतुः॑स्तना॒म् दोहा॑य॒ दोहा॑य॒ चतुः॑स्तनाम् करोति करोति॒ चतुः॑स्तना॒म् दोहा॑य॒ दोहा॑य॒ चतुः॑स्तनाम् करोति । \newline
53. चतुः॑स्तनाम् करोति करोति॒ चतुः॑स्तना॒म् चतुः॑स्तनाम् करोति पशू॒नाम् प॑शू॒नाम् क॑रोति॒ चतुः॑स्तना॒म् चतुः॑स्तनाम् करोति पशू॒नाम् । \newline
54. चतुः॑स्तना॒मिति॒ चतुः॑ - स्त॒ना॒म् । \newline
55. क॒रो॒ति॒ प॒शू॒नाम् प॑शू॒नाम् क॑रोति करोति पशू॒नाम् दोहा॑य॒ दोहा॑य पशू॒नाम् क॑रोति करोति पशू॒नाम् दोहा॑य । \newline
56. प॒शू॒नाम् दोहा॑य॒ दोहा॑य पशू॒नाम् प॑शू॒नाम् दोहा॑या॒ ष्टास्त॑ना म॒ष्टास्त॑ना॒म् दोहा॑य पशू॒नाम् प॑शू॒नाम् दोहा॑या॒ ष्टास्त॑नाम् । \newline
57. दोहा॑या॒ ष्टास्त॑ना म॒ष्टास्त॑ना॒म् दोहा॑य॒ दोहा॑या॒ ष्टास्त॑नाम् करोति करो त्य॒ष्टास्त॑ना॒म् दोहा॑य॒ दोहा॑या॒ ष्टास्त॑नाम् करोति । \newline
58. अ॒ष्टास्त॑नाम् करोति करो त्य॒ष्टास्त॑ना म॒ष्टास्त॑नाम् करोति॒ छन्द॑सा॒म् छन्द॑साम् करो त्य॒ष्टास्त॑ना म॒ष्टास्त॑नाम् करोति॒ छन्द॑साम् । \newline
59. अ॒ष्टास्त॑ना॒मित्य॒ष्टा - स्त॒ना॒म् । \newline
60. क॒रो॒ति॒ छन्द॑सा॒म् छन्द॑साम् करोति करोति॒ छन्द॑सा॒म् दोहा॑य॒ दोहा॑य॒ छन्द॑साम् करोति करोति॒ छन्द॑सा॒म् दोहा॑य । \newline
61. छन्द॑सा॒म् दोहा॑य॒ दोहा॑य॒ छन्द॑सा॒म् छन्द॑सा॒म् दोहा॑य॒ नवा᳚श्रि॒म् नवा᳚श्रि॒म् दोहा॑य॒ छन्द॑सा॒म् छन्द॑सा॒म् दोहा॑य॒ नवा᳚श्रिम् । \newline
62. दोहा॑य॒ नवा᳚श्रि॒म् नवा᳚श्रि॒म् दोहा॑य॒ दोहा॑य॒ नवा᳚श्रि मभि॒चर॑तो ऽभि॒चर॑तो॒ नवा᳚श्रि॒म् दोहा॑य॒ दोहा॑य॒ नवा᳚श्रि मभि॒चर॑तः । \newline
63. नवा᳚श्रि मभि॒चर॑तो ऽभि॒चर॑तो॒ नवा᳚श्रि॒म् नवा᳚श्रि मभि॒चर॑तः कुर्यात् कुर्या दभि॒चर॑तो॒ नवा᳚श्रि॒म् नवा᳚श्रि मभि॒चर॑तः कुर्यात् । \newline
64. नवा᳚श्रि॒मिति॒ नव॑ - अ॒श्रि॒म् । \newline
65. अ॒भि॒चर॑तः कुर्यात् कुर्या दभि॒चर॑तो ऽभि॒चर॑तः कुर्यात् त्रि॒वृत॑म् त्रि॒वृत॑म् कुर्या दभि॒चर॑तो ऽभि॒चर॑तः कुर्यात् त्रि॒वृत᳚म् । \newline
66. अ॒भि॒चर॑त॒ इत्य॑भि - चर॑तः । \newline
67. कु॒र्या॒त् त्रि॒वृत॑म् त्रि॒वृत॑म् कुर्यात् कुर्यात् त्रि॒वृत॑ मे॒वैव त्रि॒वृत॑म् कुर्यात् कुर्यात् त्रि॒वृत॑ मे॒व । \newline
68. त्रि॒वृत॑ मे॒वैव त्रि॒वृत॑म् त्रि॒वृत॑ मे॒व वज्रं॒ ॅवज्र॑ मे॒व त्रि॒वृत॑म् त्रि॒वृत॑ मे॒व वज्र᳚म् । \newline
69. त्रि॒वृत॒मिति॑ त्रि - वृत᳚म् । \newline
70. ए॒व वज्रं॒ ॅवज्र॑ मे॒वैव वज्रꣳ॑ सं॒भृत्य॑ सं॒भृत्य॒ वज्र॑ मे॒वैव वज्रꣳ॑ सं॒भृत्य॑ । \newline
71. वज्रꣳ॑ सं॒भृत्य॑ सं॒भृत्य॒ वज्रं॒ ॅवज्रꣳ॑ सं॒भृत्य॒ भ्रातृ॑व्याय॒ भ्रातृ॑व्याय सं॒भृत्य॒ वज्रं॒ ॅवज्रꣳ॑ सं॒भृत्य॒ भ्रातृ॑व्याय । \newline
72. सं॒भृत्य॒ भ्रातृ॑व्याय॒ भ्रातृ॑व्याय सं॒भृत्य॑ सं॒भृत्य॒ भ्रातृ॑व्याय॒ प्र प्र भ्रातृ॑व्याय सं॒भृत्य॑ सं॒भृत्य॒ भ्रातृ॑व्याय॒ प्र । \newline
73. सं॒भृत्येति॑ सं - भृत्य॑ । \newline
74. भ्रातृ॑व्याय॒ प्र प्र भ्रातृ॑व्याय॒ भ्रातृ॑व्याय॒ प्र ह॑रति हरति॒ प्र भ्रातृ॑व्याय॒ भ्रातृ॑व्याय॒ प्र ह॑रति । \newline
75. प्र ह॑रति हरति॒ प्र प्र ह॑रति॒ स्तृत्यै॒ स्तृत्यै॑ हरति॒ प्र प्र ह॑रति॒ स्तृत्यै᳚ । \newline
76. ह॒र॒ति॒ स्तृत्यै॒ स्तृत्यै॑ हरति हरति॒ स्तृत्यै॑ कृ॒त्वाय॑ कृ॒त्वाय॒ स्तृत्यै॑ हरति हरति॒ स्तृत्यै॑ कृ॒त्वाय॑ । \newline
77. स्तृत्यै॑ कृ॒त्वाय॑ कृ॒त्वाय॒ स्तृत्यै॒ स्तृत्यै॑ कृ॒त्वाय॒ सा सा कृ॒त्वाय॒ स्तृत्यै॒ स्तृत्यै॑ कृ॒त्वाय॒ सा । \newline
78. कृ॒त्वाय॒ सा सा कृ॒त्वाय॑ कृ॒त्वाय॒ सा म॒हीम् म॒हीꣳ सा कृ॒त्वाय॑ कृ॒त्वाय॒ सा म॒हीम् । \newline
79. सा म॒हीम् म॒हीꣳ सा सा म॒ही मु॒खा मु॒खाम् म॒हीꣳ सा सा म॒ही मु॒खाम् । \newline
80. म॒ही मु॒खा मु॒खाम् म॒हीम् म॒ही मु॒खा मिती त्यु॒खाम् म॒हीम् म॒ही मु॒खा मिति॑ । \newline
81. उ॒खा मिती त्यु॒खा मु॒खा मिति॒ नि नीत्यु॒खा मु॒खा मिति॒ नि । \newline
82. इति॒ नि नीतीति॒ नि द॑धाति दधाति॒ नीतीति॒ नि द॑धाति । \newline
83. नि द॑धाति दधाति॒ नि नि द॑धाति दे॒वता॑सु दे॒वता॑सु दधाति॒ नि नि द॑धाति दे॒वता॑सु । \newline
84. द॒धा॒ति॒ दे॒वता॑सु दे॒वता॑सु दधाति दधाति दे॒वता᳚ स्वे॒वैव दे॒वता॑सु दधाति दधाति दे॒वता᳚स्वे॒व । \newline
85. दे॒वता᳚ स्वे॒वैव दे॒वता॑सु दे॒वता᳚ स्वे॒वैना॑ मेना मे॒व दे॒वता॑सु दे॒वता᳚ स्वे॒वैना᳚म् । \newline
86. ए॒वैना॑ मेना मे॒वैवैना॒म् प्रति॒ प्रत्ये॑ना मे॒वैवैना॒म् प्रति॑ । \newline
87. ए॒ना॒म् प्रति॒ प्रत्ये॑ना मेना॒म् प्रति॑ ष्ठापयति स्थापयति॒ प्रत्ये॑ना मेना॒म् प्रति॑ ष्ठापयति । \newline
88. प्रति॑ ष्ठापयति स्थापयति॒ प्रति॒ प्रति॑ ष्ठापयति । \newline
89. स्था॒प॒य॒तीति॑ स्थापयति । \newline
\pagebreak
\markright{ TS 5.1.7.1  \hfill https://www.vedavms.in \hfill}

\section{ TS 5.1.7.1 }

\textbf{TS 5.1.7.1 } \newline
\textbf{Samhita Paata} \newline

स॒प्तभि॑र्द्धूपयति स॒प्त वै शी॑र्.ष॒ण्याः᳚ प्रा॒णाः शिर॑ ए॒तद्-य॒ज्ञ्स्य॒ यदु॒खा शी॒र्.॒षन्ने॒व य॒ज्ञ्स्य॑ प्रा॒णान् द॑धाति॒ तस्मा᳚थ् स॒प्त शी॒र्.॒षन् प्रा॒णा अ॑श्वश॒केन॑ धूपयति प्राजाप॒त्यो वा अश्वः॑ सयोनि॒त्वाया-दि॑ति॒स्त्वेत्या॑हे॒यं ॅवा अदि॑ति॒रदि॑त्यै॒वादि॑त्यां खनत्य॒स्या अक्रू॑रंकाराय॒ न हि स्वः स्वꣳ हि॒नस्ति॑ दे॒वानां᳚ त्वा॒ पत्नी॒रित्या॑ह दे॒वानां॒ - [  ] \newline

\textbf{Pada Paata} \newline

स॒प्तभि॒रिति॑ स॒प्त - भिः॒ । धू॒प॒य॒ति॒ । स॒प्त । वै । शी॒र्॒.ष॒ण्याः᳚ । प्रा॒णा इति॑ प्र - अ॒नाः । शिरः॑ । ए॒तत् । य॒ज्ञ्स्य॑ । यत् । उ॒खा । शी॒र्॒.षन्न् । ए॒व । य॒ज्ञ्स्य॑ । प्रा॒णानिति॑ प्र - अ॒नान् । द॒धा॒ति॒ । तस्मा᳚त् । स॒प्त । शी॒र॒.षन्न् । प्रा॒णा इति॑ प्र - अ॒नाः । अ॒श्व॒श॒केनेत्य॑श्व-श॒केन॑ । धू॒प॒य॒ति॒ । प्रा॒जा॒प॒त्य इति॑ प्राजा-प॒त्यः । वै । अश्वः॑ । स॒यो॒नि॒त्वायेति॑ सयोनि-त्वाय॑ । अदि॑तिः । त्वा॒ । इति॑ । आ॒ह॒ । इ॒यम् । वै । अदि॑तिः । अदि॑त्या । ए॒व । अदि॑त्याम् । ख॒न॒ति॒ । अ॒स्याः । अक्रू॑रंकारा॒येत्यक्रू॑रं - का॒रा॒य॒ । न । हि । स्वः । स्वम् । हि॒नस्ति॑ । दे॒वाना᳚म् । त्वा॒ । पत्नीः᳚ । इति॑ । आ॒ह॒ । दे॒वाना᳚म् ।  \newline


\textbf{Krama Paata} \newline

स॒प्तभि॑र् धूपयति । स॒प्तभि॒रिति॑ स॒प्त - भिः॒ । धू॒प॒य॒ति॒ स॒प्त । स॒प्त वै । वै शी॑र्.ष॒ण्याः᳚ । शी॒र्॒.ष॒ण्याः᳚ प्रा॒णाः । प्रा॒णाः शिरः॑ । प्रा॒णा इति॑ प्र - अ॒नाः । शिर॑ ए॒तत् । ए॒तद् य॒ज्ञ्स्य॑ । य॒ज्ञ्स्य॒ यत् । यदु॒खा । उ॒खा शी॒र्॒.षन्न् । शी॒र्॒.षन्ने॒व । ए॒व य॒ज्ञ्स्य॑ । य॒ज्ञ्स्य॑ प्रा॒णान् । प्रा॒णान् द॑धाति । प्रा॒णानिति॑ प्र - अ॒नान् । द॒धा॒ति॒ तस्मा᳚त् । तस्मा᳚थ् स॒प्त । स॒प्त शी॒र्॒.षन्न् । शी॒र्॒.षन् प्रा॒णाः । प्रा॒णा अ॑श्वश॒केन॑ । प्रा॒णा इति॑ प्र - अ॒नाः । अ॒श्व॒श॒केन॑ धूपयति । अ॒श्व॒श॒केनेत्य॑श्व - श॒केन॑ । धू॒प॒य॒ति॒ प्रा॒जा॒प॒त्यः । प्रा॒जा॒प॒त्यो वै । प्रा॒जा॒प॒त्य इति॑ प्राजा - प॒त्यः । वा अश्वः॑ । अश्वः॑ सयोनि॒त्वाय॑ । स॒यो॒नि॒त्वायादि॑तिः । स॒यो॒नि॒त्वायेति॑ सयोनि - त्वाय॑ । अदि॑तिस्त्वा । त्वेति॑ । इत्या॑ह । आ॒हे॒यम् । इ॒यम् ॅवै । वा अदि॑तिः । अदि॑ति॒रदि॑त्या । अदि॑त्यै॒व । ए॒वादि॑त्याम् । अदि॑त्याम् खनति । ख॒न॒त्य॒स्याः । अ॒स्या अक्रू॑रङ्काराय । अक्रू॑रङ्काराय॒ न । अक्रू॑रङ्कारा॒येत्यक्रू॑रम् - का॒रा॒य॒ । न हि । हि स्वः । स्वः स्वम् । स्वꣳ हि॒नस्ति॑ । हि॒नस्ति॑ दे॒वाना᳚म् । दे॒वाना᳚म् त्वा । त्वा॒ पत्नीः᳚ । पत्नी॒रिति॑ । इत्या॑ह । आ॒ह॒ दे॒वाना᳚म् । दे॒वाना॒म् ॅवै \newline

\textbf{Jatai Paata} \newline

1. स॒प्तभि॑र् धूपयति धूपयति स॒प्तभिः॑ स॒प्तभि॑र् धूपयति । \newline
2. स॒प्तभि॒रिति॑ स॒प्त - भिः॒ । \newline
3. धू॒प॒य॒ति॒ स॒प्त स॒प्त धू॑पयति धूपयति स॒प्त । \newline
4. स॒प्त वै वै स॒प्त स॒प्त वै । \newline
5. वै शी॑र्.ष॒ण्याः᳚ शीर्.ष॒ण्या॑ वै वै शी॑र्.ष॒ण्याः᳚ । \newline
6. शी॒र्॒.ष॒ण्याः᳚ प्रा॒णाः प्रा॒णाः शी॑र्.ष॒ण्याः᳚ शीर्.ष॒ण्याः᳚ प्रा॒णाः । \newline
7. प्रा॒णाः शिरः॒ शिरः॑ प्रा॒णाः प्रा॒णाः शिरः॑ । \newline
8. प्रा॒णा इति॑ प्र - अ॒नाः । \newline
9. शिर॑ ए॒त दे॒त च्छिरः॒ शिर॑ ए॒तत् । \newline
10. ए॒तद् य॒ज्ञ्स्य॑ य॒ज्ञ् स्यै॒त दे॒तद् य॒ज्ञ्स्य॑ । \newline
11. य॒ज्ञ्स्य॒ यद् यद् य॒ज्ञ्स्य॑ य॒ज्ञ्स्य॒ यत् । \newline
12. यदु॒खोखा यद् यदु॒खा । \newline
13. उ॒खा शी॒र्॒.षञ् छी॒र्॒.षन् नु॒खोखा शी॒र्॒.षन्न् । \newline
14. शी॒र्॒.षन् ने॒वैव शी॒र्॒.षञ् छी॒र्॒.षन् ने॒व । \newline
15. ए॒व य॒ज्ञ्स्य॑ य॒ज्ञ् स्यै॒वैव य॒ज्ञ्स्य॑ । \newline
16. य॒ज्ञ्स्य॑ प्रा॒णान् प्रा॒णान्. य॒ज्ञ्स्य॑ य॒ज्ञ्स्य॑ प्रा॒णान् । \newline
17. प्रा॒णान् द॑धाति दधाति प्रा॒णान् प्रा॒णान् द॑धाति । \newline
18. प्रा॒णानिति॑ प्र - अ॒नान् । \newline
19. द॒धा॒ति॒ तस्मा॒त् तस्मा᳚द् दधाति दधाति॒ तस्मा᳚त् । \newline
20. तस्मा᳚थ् स॒प्त स॒प्त तस्मा॒त् तस्मा᳚थ् स॒प्त । \newline
21. स॒प्त शी॒र्॒.षञ् छी॒र्॒.षन् थ्स॒प्त स॒प्त शी॒र्॒.षन्न् । \newline
22. शी॒र्॒.षन् प्रा॒णाः प्रा॒णाः शी॒र्॒.षञ् छी॒र्॒.षन् प्रा॒णाः । \newline
23. प्रा॒णा अ॑श्वश॒केना᳚ श्वश॒केन॑ प्रा॒णाः प्रा॒णा अ॑श्वश॒केन॑ । \newline
24. प्रा॒णा इति॑ प्र - अ॒नाः । \newline
25. अ॒श्व॒श॒केन॑ धूपयति धूपय त्यश्वश॒केना᳚ श्वश॒केन॑ धूपयति । \newline
26. अ॒श्व॒श॒केनेत्य॑श्व - श॒केन॑ । \newline
27. धू॒प॒य॒ति॒ प्रा॒जा॒प॒त्यः प्रा॑जाप॒त्यो धू॑पयति धूपयति प्राजाप॒त्यः । \newline
28. प्रा॒जा॒प॒त्यो वै वै प्रा॑जाप॒त्यः प्रा॑जाप॒त्यो वै । \newline
29. प्रा॒जा॒प॒त्य इति॑ प्राजा - प॒त्यः । \newline
30. वा अश्वो ऽश्वो॒ वै वा अश्वः॑ । \newline
31. अश्वः॑ सयोनि॒त्वाय॑ सयोनि॒त्वाया श्वो ऽश्वः॑ सयोनि॒त्वाय॑ । \newline
32. स॒यो॒नि॒त्वाया दि॑ति॒ रदि॑तिः सयोनि॒त्वाय॑ सयोनि॒त्वाया दि॑तिः । \newline
33. स॒यो॒नि॒त्वायेति॑ सयोनि - त्वाय॑ । \newline
34. अदि॑ति स्त्वा॒ त्वा ऽदि॑ति॒ रदि॑ति स्त्वा । \newline
35. त्वेतीति॑ त्वा॒ त्वेति॑ । \newline
36. इत्या॑हा॒हे तीत्या॑ह । \newline
37. आ॒हे॒य मि॒य मा॑हाहे॒यम् । \newline
38. इ॒यं ॅवै वा इ॒य मि॒यं ॅवै । \newline
39. वा अदि॑ति॒ रदि॑ति॒र् वै वा अदि॑तिः । \newline
40. अदि॑ति॒ रदि॒त्या ऽदि॒त्या ऽदि॑ति॒ रदि॑ति॒ रदि॑त्या । \newline
41. अदि॑ त्यै॒वै वादि॒त्या ऽदि॑त्यै॒व । \newline
42. ए॒वादि॑त्या॒ मदि॑त्या मे॒वै वादि॑त्याम् । \newline
43. अदि॑त्याम् खनति खन॒ त्यदि॑त्या॒ मदि॑त्याम् खनति । \newline
44. ख॒न॒ त्य॒स्या अ॒स्याः ख॑नति खन त्य॒स्याः । \newline
45. अ॒स्या अक्रू॑रङ्कारा॒या क्रू॑रङ्काराया॒ स्या अ॒स्या अक्रू॑रङ्काराय । \newline
46. अक्रू॑रङ्काराय॒ न नाक्रू॑रङ्कारा॒या क्रू॑रङ्काराय॒ न । \newline
47. अक्रू॑रङ्कारा॒येत्यक्रू॑रं - का॒रा॒य॒ । \newline
48. न हि हि न न हि । \newline
49. हि स्वः स्वो हि हि स्वः । \newline
50. स्वः स्वꣳ स्वꣳ स्वः स्वः स्वम् । \newline
51. स्वꣳ हि॒नस्ति॑ हि॒नस्ति॒ स्वꣳ स्वꣳ हि॒नस्ति॑ । \newline
52. हि॒नस्ति॑ दे॒वाना᳚म् दे॒वानाꣳ॑ हि॒नस्ति॑ हि॒नस्ति॑ दे॒वाना᳚म् । \newline
53. दे॒वाना᳚म् त्वा त्वा दे॒वाना᳚म् दे॒वाना᳚म् त्वा । \newline
54. त्वा॒ पत्नीः॒ पत्नी᳚ स्त्वा त्वा॒ पत्नीः᳚ । \newline
55. पत्नी॒ रितीति॒ पत्नीः॒ पत्नी॒ रिति॑ । \newline
56. इत्या॑हा॒हे तीत्या॑ह । \newline
57. आ॒ह॒ दे॒वाना᳚म् दे॒वाना॑ माहाह दे॒वाना᳚म् । \newline
58. दे॒वानां॒ ॅवै वै दे॒वाना᳚म् दे॒वानां॒ ॅवै । \newline

\textbf{Ghana Paata } \newline

1. स॒प्तभि॑र् धूपयति धूपयति स॒प्तभिः॑ स॒प्तभि॑र् धूपयति स॒प्त स॒प्त धू॑पयति स॒प्तभिः॑ स॒प्तभि॑र् धूपयति स॒प्त । \newline
2. स॒प्तभि॒रिति॑ स॒प्त - भिः॒ । \newline
3. धू॒प॒य॒ति॒ स॒प्त स॒प्त धू॑पयति धूपयति स॒प्त वै वै स॒प्त धू॑पयति धूपयति स॒प्त वै । \newline
4. स॒प्त वै वै स॒प्त स॒प्त वै शी॑र्.ष॒ण्याः᳚ शीर्.ष॒ण्या॑ वै स॒प्त स॒प्त वै शी॑र्.ष॒ण्याः᳚ । \newline
5. वै शी॑र्.ष॒ण्याः᳚ शीर्.ष॒ण्या॑ वै वै शी॑र्.ष॒ण्याः᳚ प्रा॒णाः प्रा॒णाः शी॑र्.ष॒ण्या॑ वै वै शी॑र्.ष॒ण्याः᳚ प्रा॒णाः । \newline
6. शी॒र्॒.ष॒ण्याः᳚ प्रा॒णाः प्रा॒णाः शी॑र्.ष॒ण्याः᳚ शीर्.ष॒ण्याः᳚ प्रा॒णाः शिरः॒ शिरः॑ प्रा॒णाः शी॑र्.ष॒ण्याः᳚ शीर्.ष॒ण्याः᳚ प्रा॒णाः शिरः॑ । \newline
7. प्रा॒णाः शिरः॒ शिरः॑ प्रा॒णाः प्रा॒णाः शिर॑ ए॒त दे॒त च्छिरः॑ प्रा॒णाः प्रा॒णाः शिर॑ ए॒तत् । \newline
8. प्रा॒णा इति॑ प्र - अ॒नाः । \newline
9. शिर॑ ए॒त दे॒त च्छिरः॒ शिर॑ ए॒तद् य॒ज्ञ्स्य॑ य॒ज्ञ्स्यै॒त च्छिरः॒ शिर॑ ए॒तद् य॒ज्ञ्स्य॑ । \newline
10. ए॒तद् य॒ज्ञ्स्य॑ य॒ज्ञ् स्यै॒त दे॒तद् य॒ज्ञ्स्य॒ यद् यद् य॒ज्ञ् स्यै॒त दे॒तद् य॒ज्ञ्स्य॒ यत् । \newline
11. य॒ज्ञ्स्य॒ यद् यद् य॒ज्ञ्स्य॑ य॒ज्ञ्स्य॒ यदु॒खोखा यद् य॒ज्ञ्स्य॑ य॒ज्ञ्स्य॒ यदु॒खा । \newline
12. यदु॒खोखा यद् यदु॒खा शी॒र्॒.षञ् छी॒र्॒.षन् नु॒खा यद् यदु॒खा शी॒र्॒.षन्न् । \newline
13. उ॒खा शी॒र्॒.षञ् छी॒र्॒.षन् नु॒खोखा शी॒र्॒.षन् ने॒वैव शी॒र्॒.षन् नु॒खोखा शी॒र्॒.षन् ने॒व । \newline
14. शी॒र्॒.षन् ने॒वैव शी॒र्॒.षञ् छी॒र्॒.षन् ने॒व य॒ज्ञ्स्य॑ य॒ज्ञ्स्यै॒व शी॒र्॒.षञ् छी॒र्॒.षन् ने॒व य॒ज्ञ्स्य॑ । \newline
15. ए॒व य॒ज्ञ्स्य॑ य॒ज्ञ् स्यै॒वैव य॒ज्ञ्स्य॑ प्रा॒णान् प्रा॒णान्. य॒ज्ञ् स्यै॒वैव य॒ज्ञ्स्य॑ प्रा॒णान् । \newline
16. य॒ज्ञ्स्य॑ प्रा॒णान् प्रा॒णान्. य॒ज्ञ्स्य॑ य॒ज्ञ्स्य॑ प्रा॒णान् द॑धाति दधाति प्रा॒णान्. य॒ज्ञ्स्य॑ य॒ज्ञ्स्य॑ प्रा॒णान् द॑धाति । \newline
17. प्रा॒णान् द॑धाति दधाति प्रा॒णान् प्रा॒णान् द॑धाति॒ तस्मा॒त् तस्मा᳚द् दधाति प्रा॒णान् प्रा॒णान् द॑धाति॒ तस्मा᳚त् । \newline
18. प्रा॒णानिति॑ प्र - अ॒नान् । \newline
19. द॒धा॒ति॒ तस्मा॒त् तस्मा᳚द् दधाति दधाति॒ तस्मा᳚थ् स॒प्त स॒प्त तस्मा᳚द् दधाति दधाति॒ तस्मा᳚थ् स॒प्त । \newline
20. तस्मा᳚थ् स॒प्त स॒प्त तस्मा॒त् तस्मा᳚थ् स॒प्त शी॒र्॒.षञ् छी॒र्॒.षन् थ्स॒प्त तस्मा॒त् तस्मा᳚थ् स॒प्त शी॒र्॒.षन्न् । \newline
21. स॒प्त शी॒र्॒.षञ् छी॒र्॒.षन् थ्स॒प्त स॒प्त शी॒र्॒.षन् प्रा॒णाः प्रा॒णाः शी॒र्॒.षन् थ्स॒प्त स॒प्त शी॒र्॒.षन् प्रा॒णाः । \newline
22. शी॒र॒.षन् प्रा॒णाः प्रा॒णाः शी॒र्॒.षञ् छी॒र्॒.षन् प्रा॒णा अ॑श्वश॒केना᳚ श्वश॒केन॑ प्रा॒णाः शी॒र्॒.षञ् छी॒र्॒.षन् प्रा॒णा अ॑श्वश॒केन॑ । \newline
23. प्रा॒णा अ॑श्वश॒केना᳚ श्वश॒केन॑ प्रा॒णाः प्रा॒णा अ॑श्वश॒केन॑ धूपयति धूपय त्यश्वश॒केन॑ प्रा॒णाः प्रा॒णा अ॑श्वश॒केन॑ धूपयति । \newline
24. प्रा॒णा इति॑ प्र - अ॒नाः । \newline
25. अ॒श्व॒श॒केन॑ धूपयति धूपय त्यश्वश॒केना᳚ श्वश॒केन॑ धूपयति प्राजाप॒त्यः प्रा॑जाप॒त्यो धू॑पय त्यश्वश॒केना᳚ श्वश॒केन॑ धूपयति प्राजाप॒त्यः । \newline
26. अ॒श्व॒श॒केनेत्य॑श्व - श॒केन॑ । \newline
27. धू॒प॒य॒ति॒ प्रा॒जा॒प॒त्यः प्रा॑जाप॒त्यो धू॑पयति धूपयति प्राजाप॒त्यो वै वै प्रा॑जाप॒त्यो धू॑पयति धूपयति प्राजाप॒त्यो वै । \newline
28. प्रा॒जा॒प॒त्यो वै वै प्रा॑जाप॒त्यः प्रा॑जाप॒त्यो वा अश्वो ऽश्वो॒ वै प्रा॑जाप॒त्यः प्रा॑जाप॒त्यो वा अश्वः॑ । \newline
29. प्रा॒जा॒प॒त्य इति॑ प्राजा - प॒त्यः । \newline
30. वा अश्वो ऽश्वो॒ वै वा अश्वः॑ सयोनि॒त्वाय॑ सयोनि॒त्वाया श्वो॒ वै वा अश्वः॑ सयोनि॒त्वाय॑ । \newline
31. अश्वः॑ सयोनि॒त्वाय॑ सयोनि॒त्वाया श्वो ऽश्वः॑ सयोनि॒त्वाया दि॑ति॒ रदि॑तिः सयोनि॒त्वाया श्वो ऽश्वः॑ सयोनि॒त्वाया दि॑तिः । \newline
32. स॒यो॒नि॒त्वाया दि॑ति॒ रदि॑तिः सयोनि॒त्वाय॑ सयोनि॒त्वाया दि॑ति स्त्वा॒ त्वा ऽदि॑तिः सयोनि॒त्वाय॑ सयोनि॒त्वाया दि॑ति स्त्वा । \newline
33. स॒यो॒नि॒त्वायेति॑ सयोनि - त्वाय॑ । \newline
34. अदि॑ति स्त्वा॒ त्वा ऽदि॑ति॒ रदि॑ति॒ स्त्वेतीति॒ त्वा ऽदि॑ति॒ रदि॑ति॒ स्त्वेति॑ । \newline
35. त्वेतीति॑ त्वा॒ त्वेत्या॑ हा॒हेति॑ त्वा॒ त्वेत्या॑ह । \newline
36. इत्या॑हा॒हे तीत्या॑ हे॒य मि॒य मा॒हे तीत्या॑ हे॒यम् । \newline
37. आ॒हे॒ यमि॒य मा॑हा हे॒यं ॅवै वा इ॒य मा॑हा हे॒यं ॅवै । \newline
38. इ॒यं ॅवै वा इ॒य मि॒यं ॅवा अदि॑ति॒ रदि॑ति॒र् वा इ॒य मि॒यं ॅवा अदि॑तिः । \newline
39. वा अदि॑ति॒ रदि॑ति॒र् वै वा अदि॑ति॒ रदि॒त्या ऽदि॒त्या ऽदि॑ति॒र् वै वा अदि॑ति॒ रदि॑त्या । \newline
40. अदि॑ति॒ रदि॒त्या ऽदि॒त्या ऽदि॑ति॒ रदि॑ति॒ रदि॑त्यै॒वै वादि॒त्या ऽदि॑ति॒ रदि॑ति॒ रदि॑त्यै॒व । \newline
41. अदि॑त्यै॒ वैवादि॒त्या ऽदि॑त्यै॒ वादि॑त्या॒ मदि॑त्या मे॒वादि॒त्या ऽदि॑त्यै॒ वादि॑त्याम् । \newline
42. ए॒वादि॑त्या॒ मदि॑त्या मे॒वैवादि॑त्याम् खनति खन॒ त्यदि॑त्या मे॒वैवादि॑त्याम् खनति । \newline
43. अदि॑त्याम् खनति खन॒ त्यदि॑त्या॒ मदि॑त्याम् खनत्य॒स्या अ॒स्याः ख॑न॒ त्यदि॑त्या॒ मदि॑त्याम् खनत्य॒स्याः । \newline
44. ख॒न॒ त्य॒स्या अ॒स्याः ख॑नति खन त्य॒स्या अक्रू॑रङ्कारा॒या क्रू॑रङ्काराया॒ स्याः ख॑नति खनत्य॒स्या अक्रू॑रङ्काराय । \newline
45. अ॒स्या अक्रू॑रङ्कारा॒या क्रू॑रङ्काराया॒ स्या अ॒स्या अक्रू॑रङ्काराय॒ न नाक्रू॑रङ्काराया॒ स्या अ॒स्या अक्रू॑रङ्काराय॒ न । \newline
46. अक्रू॑रङ्काराय॒ न नाक्रू॑रङ्कारा॒या क्रू॑रङ्काराय॒ न हि हि नाक्रू॑रङ्कारा॒या क्रू॑रङ्काराय॒ न हि । \newline
47. अक्रू॑रङ्कारा॒येत्यक्रू॑रं - का॒रा॒य॒ । \newline
48. न हि हि न न हि स्वः स्वो हि न न हि स्वः । \newline
49. हि स्वः स्वो हि हि स्वः स्वꣳ स्वꣳ स्वो हि हि स्वः स्वम् । \newline
50. स्वः स्वꣳ स्वꣳ स्वः स्वः स्वꣳ हि॒नस्ति॑ हि॒नस्ति॒ स्वꣳ स्वः स्वः स्वꣳ हि॒नस्ति॑ । \newline
51. स्वꣳ हि॒नस्ति॑ हि॒नस्ति॒ स्वꣳ स्वꣳ हि॒नस्ति॑ दे॒वाना᳚म् दे॒वानाꣳ॑ हि॒नस्ति॒ स्वꣳ स्वꣳ हि॒नस्ति॑ दे॒वाना᳚म् । \newline
52. हि॒नस्ति॑ दे॒वाना᳚म् दे॒वानाꣳ॑ हि॒नस्ति॑ हि॒नस्ति॑ दे॒वाना᳚म् त्वा त्वा दे॒वानाꣳ॑ हि॒नस्ति॑ हि॒नस्ति॑ दे॒वाना᳚म् त्वा । \newline
53. दे॒वाना᳚म् त्वा त्वा दे॒वाना᳚म् दे॒वाना᳚म् त्वा॒ पत्नीः॒ पत्नी᳚ स्त्वा दे॒वाना᳚म् दे॒वाना᳚म् त्वा॒ पत्नीः᳚ । \newline
54. त्वा॒ पत्नीः॒ पत्नी᳚ स्त्वा त्वा॒ पत्नी॒ रितीति॒ पत्नी᳚ स्त्वा त्वा॒ पत्नी॒ रिति॑ । \newline
55. पत्नी॒ रितीति॒ पत्नीः॒ पत्नी॒ रित्या॑ हा॒हेति॒ पत्नीः॒ पत्नी॒ रित्या॑ह । \newline
56. इत्या॑हा॒हे तीत्या॑ह दे॒वाना᳚म् दे॒वाना॑ मा॒हे तीत्या॑ह दे॒वाना᳚म् । \newline
57. आ॒ह॒ दे॒वाना᳚म् दे॒वाना॑ माहाह दे॒वानां॒ ॅवै वै दे॒वाना॑ माहाह दे॒वानां॒ ॅवै । \newline
58. दे॒वानां॒ ॅवै वै दे॒वाना᳚म् दे॒वानां॒ ॅवा ए॒ता मे॒तां ॅवै दे॒वाना᳚म् दे॒वानां॒ ॅवा ए॒ताम् । \newline
\pagebreak
\markright{ TS 5.1.7.2  \hfill https://www.vedavms.in \hfill}

\section{ TS 5.1.7.2 }

\textbf{TS 5.1.7.2 } \newline
\textbf{Samhita Paata} \newline

ॅवा ए॒तां पत्न॒योऽग्रे॑ऽकुर्व॒न् ताभि॑रे॒वैनां᳚ दधाति धि॒षणा॒स्त्वेत्या॑ह वि॒द्या वै धि॒षणा॑ वि॒द्याभि॑रे॒वैना॑म॒भीन्धे॒ ग्नास्त्वेत्या॑ह॒ छन्दाꣳ॑सि॒ वै ग्ना श्छन्दो॑भिरे॒वैनाꣳ॑ श्रपयति॒ वरू᳚त्रय॒स्त्त्वेत्या॑ह॒ होत्रा॒ वै वरू᳚त्रयो॒ होत्रा॑भिरे॒वैनां᳚ पचति॒ जन॑य॒स्त्वेत्या॑ह दे॒वानां॒ ॅवै पत्नी॒ - [  ] \newline

\textbf{Pada Paata} \newline

वै । ए॒ताम् । पत्न॑यः । अग्रे᳚ । अ॒कु॒र्व॒न्न् । ताभिः॑ । ए॒व । ए॒ना॒म् । द॒धा॒ति॒ । धि॒षणाः᳚ । त्वा॒ । इति॑ । आ॒ह॒ । वि॒द्याः । वै । धि॒षणाः᳚ । वि॒द्याभिः॑ । ए॒व । ए॒ना॒म् । अ॒भीति॑ । इ॒न्धे॒ । ग्नाः । त्वा॒ । इति॑ । आ॒ह॒ । छन्दाꣳ॑सि । वै । ग्नाः । छन्दो॑भि॒रिति॒ छन्दः॑ - भिः॒ । ए॒व । ए॒ना॒म् । श्र॒प॒य॒ति॒ । वरू᳚त्रयः । त्वा॒ । इति॑ । आ॒ह॒ । होत्राः᳚ । वै । वरू᳚त्रयः । होत्रा॑भिः । ए॒व । ए॒ना॒म् । प॒च॒ति॒ । जन॑यः । त्वा॒ । इति॑ । आ॒ह॒ । दे॒वाना᳚म् । वै । पत्नीः᳚ ।  \newline


\textbf{Krama Paata} \newline

वा ए॒ताम् । ए॒ताम् पत्न॑यः । पत्न॒योऽग्रे᳚ । अग्रे॑ऽकुर्वन्न् । अ॒कु॒र्व॒न् ताभिः॑ । ताभि॑रे॒व । ए॒वैना᳚म् । ए॒ना॒म् द॒धा॒ति॒ । द॒धा॒ति॒ धि॒षणाः᳚ । धि॒षणा᳚स्त्वा । त्वेति॑ । इत्या॑ह । आ॒ह॒ वि॒द्याः । वि॒द्या वै । वै धि॒षणाः᳚ । धि॒षणा॑ वि॒द्याभिः॑ । वि॒द्याभि॑रे॒व । ए॒वैना᳚म् । ए॒ना॒म॒भि । अ॒भीन्धे᳚ । इ॒न्धे॒ ग्नाः । ग्नास्त्वा᳚ । त्वेति॑ । इत्या॑ह । आ॒ह॒ छन्दाꣳ॑सि । छन्दाꣳ॑सि॒ वै । वै ग्नाः । ग्नाश्छन्दो॑भिः । छन्दो॑भिरे॒व । छन्दो॑भि॒रिति॒ छन्दः॑ - भिः॒ । ए॒वैना᳚म् । ए॒नाꣳ॒॒ श्र॒प॒य॒ति॒ । श्र॒प॒य॒ति॒ वरू᳚त्रयः । वरू᳚त्रयस्त्वा । त्वेति॑ । इत्या॑ह । आ॒ह॒ होत्राः᳚ । होत्रा॒ वै । वै वरू᳚त्रयः । वरू᳚त्रयो॒ होत्रा॑भिः । होत्रा॑भिरे॒व । ए॒वैना᳚म् । ए॒ना॒म् प॒च॒ति॒ । प॒च॒ति॒ जन॑यः । जन॑यस्त्वा । त्वेति॑ । इत्या॑ह । आ॒ह॒ दे॒वाना᳚म् । दे॒वाना॒म् ॅवै । वै पत्नीः᳚ । पत्नी॒र् जन॑यः \newline

\textbf{Jatai Paata} \newline

1. वा ए॒ता मे॒तां ॅवै वा ए॒ताम् । \newline
2. ए॒ताम् पत्न॑यः॒ पत्न॑य ए॒ता मे॒ताम् पत्न॑यः । \newline
3. पत्न॒यो ऽग्रे ऽग्रे॒ पत्न॑यः॒ पत्न॒यो ऽग्रे᳚ । \newline
4. अग्रे॑ ऽकुर्वन् नकुर्व॒न् नग्रे ऽग्रे॑ ऽकुर्वन्न् । \newline
5. अ॒कु॒र्व॒न् ताभि॒ स्ताभि॑ रकुर्वन् नकुर्व॒न् ताभिः॑ । \newline
6. ताभि॑ रे॒वैव ताभि॒ स्ताभि॑ रे॒व । \newline
7. ए॒वैना॑ मेना मे॒वैवैना᳚म् । \newline
8. ए॒ना॒म् द॒धा॒ति॒ द॒धा॒ त्ये॒ना॒ मे॒ना॒म् द॒धा॒ति॒ । \newline
9. द॒धा॒ति॒ धि॒षणा॑ धि॒षणा॑ दधाति दधाति धि॒षणाः᳚ । \newline
10. धि॒षणा᳚ स्त्वा त्वा धि॒षणा॑ धि॒षणा᳚ स्त्वा । \newline
11. त्वेतीति॑ त्वा॒ त्वेति॑ । \newline
12. इत्या॑हा॒हे तीत्या॑ह । \newline
13. आ॒ह॒ वि॒द्या वि॒द्या आ॑हाह वि॒द्याः । \newline
14. वि॒द्या वै वै वि॒द्या वि॒द्या वै । \newline
15. वै धि॒षणा॑ धि॒षणा॒ वै वै धि॒षणाः᳚ । \newline
16. धि॒षणा॑ वि॒द्याभि॑र् वि॒द्याभि॑र् धि॒षणा॑ धि॒षणा॑ वि॒द्याभिः॑ । \newline
17. वि॒द्याभि॑ रे॒वैव वि॒द्याभि॑र् वि॒द्याभि॑ रे॒व । \newline
18. ए॒वैना॑ मेना मे॒वैवैना᳚म् । \newline
19. ए॒ना॒ म॒भ्या᳚(1॒)भ्ये॑ना मेना म॒भि । \newline
20. अ॒भीन्ध॑ इन्धे॒ ऽभ्य॑भीन्धे᳚ । \newline
21. इ॒न्धे॒ ग्ना ग्ना इ॑न्ध इन्धे॒ ग्नाः । \newline
22. ग्ना स्त्वा᳚ त्वा॒ ग्ना ग्ना स्त्वा᳚ । \newline
23. त्वेतीति॑ त्वा॒ त्वेति॑ । \newline
24. इत्या॑हा॒हे तीत्या॑ह । \newline
25. आ॒ह॒ छन्दाꣳ॑सि॒ छन्दाꣳ॑ स्याहाह॒ छन्दाꣳ॑सि । \newline
26. छन्दाꣳ॑सि॒ वै वै छन्दाꣳ॑सि॒ छन्दाꣳ॑सि॒ वै । \newline
27. वै ग्ना ग्ना वै वै ग्नाः । \newline
28. ग्ना श्छन्दो॑भि॒ श्छन्दो॑भि॒र् ग्ना ग्ना श्छन्दो॑भिः । \newline
29. छन्दो॑भि रे॒वैव छन्दो॑भि॒ श्छन्दो॑भि रे॒व । \newline
30. छन्दो॑भि॒रिति॒ छन्दः॑ - भिः॒ । \newline
31. ए॒वैना॑ मेना मे॒वैवैना᳚म् । \newline
32. ए॒नाꣳ॒॒ श्र॒प॒य॒ति॒ श्र॒प॒य॒ त्ये॒ना॒ मे॒नाꣳ॒॒ श्र॒प॒य॒ति॒ । \newline
33. श्र॒प॒य॒ति॒ वरू᳚त्रयो॒ वरू᳚त्रयः श्रपयति श्रपयति॒ वरू᳚त्रयः । \newline
34. वरू᳚त्रय स्त्वा त्वा॒ वरू᳚त्रयो॒ वरू᳚त्रय स्त्वा । \newline
35. त्वेतीति॑ त्वा॒ त्वेति॑ । \newline
36. इत्या॑हा॒हे तीत्या॑ह । \newline
37. आ॒ह॒ होत्रा॒ होत्रा॑ आहाह॒ होत्राः᳚ । \newline
38. होत्रा॒ वै वै होत्रा॒ होत्रा॒ वै । \newline
39. वै वरू᳚त्रयो॒ वरू᳚त्रयो॒ वै वै वरू᳚त्रयः । \newline
40. वरू᳚त्रयो॒ होत्रा॑भि॒र्॒. होत्रा॑भि॒र् वरू᳚त्रयो॒ वरू᳚त्रयो॒ होत्रा॑भिः । \newline
41. होत्रा॑भि रे॒वैव होत्रा॑भि॒र्॒. होत्रा॑भि रे॒व । \newline
42. ए॒वैना॑ मेना मे॒वैवैना᳚म् । \newline
43. ए॒ना॒म् प॒च॒ति॒ प॒च॒ त्ये॒ना॒ मे॒ना॒म् प॒च॒ति॒ । \newline
44. प॒च॒ति॒ जन॑यो॒ जन॑यः पचति पचति॒ जन॑यः । \newline
45. जन॑य स्त्वा त्वा॒ जन॑यो॒ जन॑य स्त्वा । \newline
46. त्वेतीति॑ त्वा॒ त्वेति॑ । \newline
47. इत्या॑हा॒हे तीत्या॑ह । \newline
48. आ॒ह॒ दे॒वाना᳚म् दे॒वाना॑ माहाह दे॒वाना᳚म् । \newline
49. दे॒वानां॒ ॅवै वै दे॒वाना᳚म् दे॒वानां॒ ॅवै । \newline
50. वै पत्नीः॒ पत्नी॒र् वै वै पत्नीः᳚ । \newline
51. पत्नी॒र् जन॑यो॒ जन॑यः॒ पत्नीः॒ पत्नी॒र् जन॑यः । \newline

\textbf{Ghana Paata } \newline

1. वा ए॒ता मे॒तां ॅवै वा ए॒ताम् पत्न॑यः॒ पत्न॑य ए॒तां ॅवै वा ए॒ताम् पत्न॑यः । \newline
2. ए॒ताम् पत्न॑यः॒ पत्न॑य ए॒ता मे॒ताम् पत्न॒यो ऽग्रे ऽग्रे॒ पत्न॑य ए॒ता मे॒ताम् पत्न॒यो ऽग्रे᳚ । \newline
3. पत्न॒यो ऽग्रे ऽग्रे॒ पत्न॑यः॒ पत्न॒यो ऽग्रे॑ ऽकुर्वन् नकुर्व॒न् नग्रे॒ पत्न॑यः॒ पत्न॒यो ऽग्रे॑ ऽकुर्वन्न् । \newline
4. अग्रे॑ ऽकुर्वन् नकुर्व॒न् नग्रे ऽग्रे॑ ऽकुर्व॒न् ताभि॒ स्ताभि॑ रकुर्व॒न् नग्रे ऽग्रे॑ ऽकुर्व॒न् ताभिः॑ । \newline
5. अ॒कु॒र्व॒न् ताभि॒ स्ताभि॑ रकुर्वन् नकुर्व॒न् ताभि॑ रे॒वैव ताभि॑ रकुर्वन् नकुर्व॒न् ताभि॑ रे॒व । \newline
6. ताभि॑ रे॒वैव ताभि॒ स्ताभि॑ रे॒वैना॑ मेना मे॒व ताभि॒ स्ताभि॑ रे॒वैना᳚म् । \newline
7. ए॒वैना॑ मेना मे॒वैवैना᳚म् दधाति दधा त्येना मे॒वैवैना᳚म् दधाति । \newline
8. ए॒ना॒म् द॒धा॒ति॒ द॒धा॒ त्ये॒ना॒ मे॒ना॒म् द॒धा॒ति॒ धि॒षणा॑ धि॒षणा॑ दधा त्येना मेनाम् दधाति धि॒षणाः᳚ । \newline
9. द॒धा॒ति॒ धि॒षणा॑ धि॒षणा॑ दधाति दधाति धि॒षणा᳚ स्त्वा त्वा धि॒षणा॑ दधाति दधाति धि॒षणा᳚ स्त्वा । \newline
10. धि॒षणा᳚ स्त्वा त्वा धि॒षणा॑ धि॒षणा॒ स्त्वेतीति॑ त्वा धि॒षणा॑ धि॒षणा॒ स्त्वेति॑ । \newline
11. त्वेतीति॑ त्वा॒ त्वेत्या॑हा॒हेति॑ त्वा॒ त्वेत्या॑ह । \newline
12. इत्या॑हा॒हे तीत्या॑ह वि॒द्या वि॒द्या आ॒हे तीत्या॑ह वि॒द्याः । \newline
13. आ॒ह॒ वि॒द्या वि॒द्या आ॑हाह वि॒द्या वै वै वि॒द्या आ॑हाह वि॒द्या वै । \newline
14. वि॒द्या वै वै वि॒द्या वि॒द्या वै धि॒षणा॑ धि॒षणा॒ वै वि॒द्या वि॒द्या वै धि॒षणाः᳚ । \newline
15. वै धि॒षणा॑ धि॒षणा॒ वै वै धि॒षणा॑ वि॒द्याभि॑र् वि॒द्याभि॑र् धि॒षणा॒ वै वै धि॒षणा॑ वि॒द्याभिः॑ । \newline
16. धि॒षणा॑ वि॒द्याभि॑र् वि॒द्याभि॑र् धि॒षणा॑ धि॒षणा॑ वि॒द्याभि॑ रे॒वैव वि॒द्याभि॑र् धि॒षणा॑ धि॒षणा॑ वि॒द्याभि॑ रे॒व । \newline
17. वि॒द्याभि॑ रे॒वैव वि॒द्याभि॑र् वि॒द्याभि॑ रे॒वैना॑ मेना मे॒व वि॒द्याभि॑र् वि॒द्याभि॑ रे॒वैना᳚म् । \newline
18. ए॒वैना॑ मेना मे॒वैवैना॑ म॒भ्या᳚(1॒)भ्ये॑ना मे॒वैवैना॑ म॒भि । \newline
19. ए॒ना॒ म॒भ्या᳚(1॒)भ्ये॑ना मेना म॒भीन्ध॑ इन्धे॒ ऽभ्ये॑ना मेना म॒भीन्धे᳚ । \newline
20. अ॒भीन्ध॑ इन्धे॒ ऽभ्य॑भीन्धे॒ ग्ना ग्ना इ॑न्धे॒ ऽभ्य॑भीन्धे॒ ग्नाः । \newline
21. इ॒न्धे॒ ग्ना ग्ना इ॑न्ध इन्धे॒ ग्ना स्त्वा᳚ त्वा॒ ग्ना इ॑न्ध इन्धे॒ ग्ना स्त्वा᳚ । \newline
22. ग्ना स्त्वा᳚ त्वा॒ ग्ना ग्ना स्त्वेतीति॑ त्वा॒ ग्ना ग्ना स्त्वेति॑ । \newline
23. त्वेतीति॑ त्वा॒ त्वेत्या॑ हा॒हेति॑ त्वा॒ त्वेत्या॑ह । \newline
24. इत्या॑हा॒हे तीत्या॑ह॒ छन्दाꣳ॑सि॒ छन्दाꣳ॑ स्या॒हे तीत्या॑ह॒ छन्दाꣳ॑सि । \newline
25. आ॒ह॒ छन्दाꣳ॑सि॒ छन्दाꣳ॑ स्याहाह॒ छन्दाꣳ॑सि॒ वै वै छन्दाꣳ॑ स्याहाह॒ छन्दाꣳ॑सि॒ वै । \newline
26. छन्दाꣳ॑सि॒ वै वै छन्दाꣳ॑सि॒ छन्दाꣳ॑सि॒ वै ग्ना ग्ना वै छन्दाꣳ॑सि॒ छन्दाꣳ॑सि॒ वै ग्नाः । \newline
27. वै ग्ना ग्ना वै वै ग्ना श्छन्दो॑भि॒ श्छन्दो॑भि॒र् ग्ना वै वै ग्ना श्छन्दो॑भिः । \newline
28. ग्ना श्छन्दो॑भि॒ श्छन्दो॑भि॒र् ग्ना ग्ना श्छन्दो॑भि रे॒वैव छन्दो॑भि॒र् ग्ना ग्ना श्छन्दो॑भि रे॒व । \newline
29. छन्दो॑भि रे॒वैव छन्दो॑भि॒ श्छन्दो॑भि रे॒वैना॑ मेना मे॒व छन्दो॑भि॒ श्छन्दो॑भि रे॒वैना᳚म् । \newline
30. छन्दो॑भि॒रिति॒ छन्दः॑ - भिः॒ । \newline
31. ए॒वैना॑ मेना मे॒वैवैनाꣳ॑ श्रपयति श्रपय त्येना मे॒वैवैनाꣳ॑ श्रपयति । \newline
32. ए॒नाꣳ॒॒ श्र॒प॒य॒ति॒ श्र॒प॒य॒ त्ये॒ना॒ मे॒नाꣳ॒॒ श्र॒प॒य॒ति॒ वरू᳚त्रयो॒ वरू᳚त्रयः श्रपय त्येना मेनाꣳ श्रपयति॒ वरू᳚त्रयः । \newline
33. श्र॒प॒य॒ति॒ वरू᳚त्रयो॒ वरू᳚त्रयः श्रपयति श्रपयति॒ वरू᳚त्रय स्त्वा त्वा॒ वरू᳚त्रयः श्रपयति श्रपयति॒ वरू᳚त्रय स्त्वा । \newline
34. वरू᳚त्रय स्त्वा त्वा॒ वरू᳚त्रयो॒ वरू᳚त्रय॒ स्त्वेतीति॑ त्वा॒ वरू᳚त्रयो॒ वरू᳚त्रय॒ स्त्वेति॑ । \newline
35. त्वेतीति॑ त्वा॒ त्वेत्या॑हा॒हेति॑ त्वा॒ त्वेत्या॑ह । \newline
36. इत्या॑हा॒हे तीत्या॑ह॒ होत्रा॒ होत्रा॑ आ॒हे तीत्या॑ह॒ होत्राः᳚ । \newline
37. आ॒ह॒ होत्रा॒ होत्रा॑ आहाह॒ होत्रा॒ वै वै होत्रा॑ आहाह॒ होत्रा॒ वै । \newline
38. होत्रा॒ वै वै होत्रा॒ होत्रा॒ वै वरू᳚त्रयो॒ वरू᳚त्रयो॒ वै होत्रा॒ होत्रा॒ वै वरू᳚त्रयः । \newline
39. वै वरू᳚त्रयो॒ वरू᳚त्रयो॒ वै वै वरू᳚त्रयो॒ होत्रा॑भि॒र्॒. होत्रा॑भि॒र् वरू᳚त्रयो॒ वै वै वरू᳚त्रयो॒ होत्रा॑भिः । \newline
40. वरू᳚त्रयो॒ होत्रा॑भि॒र्॒. होत्रा॑भि॒र् वरू᳚त्रयो॒ वरू᳚त्रयो॒ होत्रा॑भि रे॒वैव होत्रा॑भि॒र् वरू᳚त्रयो॒ वरू᳚त्रयो॒ होत्रा॑भि रे॒व । \newline
41. होत्रा॑भि रे॒वैव होत्रा॑भि॒र्॒. होत्रा॑भि रे॒वैना॑ मेना मे॒व होत्रा॑भि॒र्॒. होत्रा॑भि रे॒वैना᳚म् । \newline
42. ए॒वैना॑ मेना मे॒वैवैना᳚म् पचति पच त्येना मे॒वैवैना᳚म् पचति । \newline
43. ए॒ना॒म् प॒च॒ति॒ प॒च॒ त्ये॒ना॒ मे॒ना॒म् प॒च॒ति॒ जन॑यो॒ जन॑यः पच त्येना मेनाम् पचति॒ जन॑यः । \newline
44. प॒च॒ति॒ जन॑यो॒ जन॑यः पचति पचति॒ जन॑य स्त्वा त्वा॒ जन॑यः पचति पचति॒ जन॑य स्त्वा । \newline
45. जन॑य स्त्वा त्वा॒ जन॑यो॒ जन॑य॒ स्त्वेतीति॑ त्वा॒ जन॑यो॒ जन॑य॒ स्त्वेति॑ । \newline
46. त्वेतीति॑ त्वा॒ त्वेत्या॑ हा॒हेति॑ त्वा॒ त्वेत्या॑ह । \newline
47. इत्या॑हा॒हे तीत्या॑ह दे॒वाना᳚म् दे॒वाना॑ मा॒हे तीत्या॑ह दे॒वाना᳚म् । \newline
48. आ॒ह॒ दे॒वाना᳚म् दे॒वाना॑ माहाह दे॒वानां॒ ॅवै वै दे॒वाना॑ माहाह दे॒वानां॒ ॅवै । \newline
49. दे॒वानां॒ ॅवै वै दे॒वाना᳚म् दे॒वानां॒ ॅवै पत्नीः॒ पत्नी॒र् वै दे॒वाना᳚म् दे॒वानां॒ ॅवै पत्नीः᳚ । \newline
50. वै पत्नीः॒ पत्नी॒र् वै वै पत्नी॒र् जन॑यो॒ जन॑यः॒ पत्नी॒र् वै वै पत्नी॒र् जन॑यः । \newline
51. पत्नी॒र् जन॑यो॒ जन॑यः॒ पत्नीः॒ पत्नी॒र् जन॑य॒ स्ताभि॒ स्ताभि॒र् जन॑यः॒ पत्नीः॒ पत्नी॒र् जन॑य॒ स्ताभिः॑ । \newline
\pagebreak
\markright{ TS 5.1.7.3  \hfill https://www.vedavms.in \hfill}

\section{ TS 5.1.7.3 }

\textbf{TS 5.1.7.3 } \newline
\textbf{Samhita Paata} \newline

-र्जन॑य॒स्ताभि॑रे॒वैनां᳚ पचति ष॒ड्भिः प॑चति॒ षड्वा ऋ॒तव॑ ऋ॒तुभि॑रे॒वैनां᳚ पचति॒ द्विः पच॒न्त्वित्या॑ह॒ तस्मा॒द् द्विः सं॑ॅवथ्स॒रस्य॑ स॒स्यं प॑च्यते वारु॒ण्यु॑खाऽभीद्धा॑ मै॒त्रियोपै॑ति॒ शान्त्यै॑ दे॒वस्त्वा॑ सवि॒तोद्-व॑प॒त्वित्या॑ह सवि॒तृप्र॑सूत ए॒वैनां॒ ब्रह्म॑णा दे॒वता॑भि॒रुद्-व॑प॒त्यप॑द्यमाना पृथि॒व्याशा॒ दिश॒ आ पृ॒णे - [  ] \newline

\textbf{Pada Paata} \newline

जन॑यः । ताभिः॑ । ए॒व । ए॒ना॒म् । प॒च॒ति॒ । ष॒ड्भिरिति॑ षट् - भिः । प॒च॒ति॒ । षट् । वै । ऋ॒तवः॑ । ऋ॒तुभि॒रित्यृ॒तु - भिः॒ । ए॒व । ए॒ना॒म् । प॒च॒ति॒ । द्विः । पच॑न्तु । इति॑ । आ॒ह॒ । तस्मा᳚त् । द्विः । सं॒ॅव॒थ्स॒रस्येति॑ सं-व॒थ्स॒रस्य॑ । स॒स्यम् । प॒च्य॒ते॒ । वा॒रु॒णी । उ॒खा । अ॒भीद्धेत्य॒भि-इ॒द्धा॒ । मै॒त्रिया᳚ । उपेति॑ । ए॒ति॒ । शान्त्यै᳚ । दे॒वः । त्वा॒ । स॒वि॒ता । उदिति॑ । व॒प॒तु॒ । इति॑ । आ॒ह॒ । स॒वि॒तृप्र॑सूत॒ इति॑ सवि॒तृ - प्र॒सू॒तः॒ । ए॒व । ए॒ना॒म् । ब्रह्म॑णा । दे॒वता॑भिः । उदिति॑ । व॒प॒ति॒ । अप॑द्यमाना । पृ॒थि॒वि॒ । आशाः᳚ । दिशः॑ । एति॑ । पृ॒ण॒ ।  \newline


\textbf{Krama Paata} \newline

जन॑य॒स्ताभिः॑ । ताभि॑रे॒व । ए॒वैना᳚म् । ए॒ना॒म् प॒च॒ति॒ । प॒च॒ति॒ ष॒ड्भिः । ष॒ड्भिः प॑चति । ष॒ड्भिरिति॑ षट् - भिः । प॒च॒ति॒ षट् । षड् वै । वा ऋ॒तवः॑ । ऋ॒तव॑ ऋ॒तुभिः॑ । ऋ॒तुभि॑रे॒व । ऋ॒तुभि॒रित्यृ॒तु - भिः॒ । ए॒वैना᳚म् । ए॒ना॒म् प॒च॒ति॒ । प॒च॒ति॒ द्विः । द्विः पच॑न्तु । पच॒न्त्विति॑ । इत्या॑ह । आ॒ह॒ तस्मा᳚त् । तस्मा॒द् द्विः । द्विः स॑म्ॅवथ्स॒रस्य॑ । स॒म्ॅव॒थ्स॒रस्य॑ स॒स्यम् । स॒म्ॅव॒थ्स॒रस्येति॑ सम् - व॒थ्स॒रस्य॑ । स॒स्यम् प॑च्यते । प॒च्य॒ते॒ वा॒रु॒णी । वा॒रु॒ण्यु॑खा । उ॒खाऽभीद्धा᳚ । अ॒भीद्धा॑ मै॒त्रिया᳚ । अ॒भीद्धेत्य॒भि - इ॒द्धा॒ । मै॒त्रियोप॑ । उपै॑ति । ए॒ति॒ शान्त्यै᳚ । शान्त्यै॑ दे॒वः । दे॒वस्त्वा᳚ । त्वा॒ स॒वि॒ता । स॒वि॒तोत् । उद् व॑पतु । व॒प॒त्विति॑ । इत्या॑ह । आ॒ह॒ स॒वि॒तृप्र॑सूतः । स॒वि॒तृप्र॑सूत ए॒व । स॒वि॒तृप्र॑सूत॒ इति॑ सवि॒तृ - प्र॒सू॒तः॒ । ए॒वैना᳚म् । ए॒ना॒म् ब्रह्म॑णा । ब्रह्म॑णा दे॒वता॑भिः । दे॒वता॑भि॒रुत् । उद् व॑पति । व॒प॒त्यप॑द्यमाना । अप॑द्यमाना पृथि॒वी । पृ॒थि॒व्याशाः᳚ । आशा॒ दिशः॑ । दिश॒ आ । आ पृ॑ण । पृ॒णेति॑ \newline

\textbf{Jatai Paata} \newline

1. जन॑य॒ स्ताभि॒ स्ताभि॒र् जन॑यो॒ जन॑य॒ स्ताभिः॑ । \newline
2. ताभि॑ रे॒वैव ताभि॒ स्ताभि॑ रे॒व । \newline
3. ए॒वैना॑ मेना मे॒वैवैना᳚म् । \newline
4. ए॒ना॒म् प॒च॒ति॒ प॒च॒ त्ये॒ना॒ मे॒ना॒म् प॒च॒ति॒ । \newline
5. प॒च॒ति॒ ष॒ड्भि ष्ष॒ड्भिः प॑चति पचति ष॒ड्भिः । \newline
6. ष॒ड्भिः प॑चति पचति ष॒ड्भि ष्ष॒ड्भिः प॑चति । \newline
7. ष॒ड्भिरिति॑ षट् - भिः । \newline
8. प॒च॒ति॒ षट् थ्षट् प॑चति पचति॒ षट् । \newline
9. षड् वै वै षट् थ्षड् वै । \newline
10. वा ऋ॒तव॑ ऋ॒तवो॒ वै वा ऋ॒तवः॑ । \newline
11. ऋ॒तव॑ ऋ॒तुभिर्॑. ऋ॒तुभिर्॑. ऋ॒तव॑ ऋ॒तव॑ ऋ॒तुभिः॑ । \newline
12. ऋ॒तुभि॑ रे॒वैव र्तुभिर्॑. ऋ॒तुभि॑ रे॒व । \newline
13. ऋ॒तुभि॒रित्यृ॒तु - भिः॒ । \newline
14. ए॒वैना॑ मेना मे॒वैवैना᳚म् । \newline
15. ए॒ना॒म् प॒च॒ति॒ प॒च॒ त्ये॒ना॒ मे॒ना॒म् प॒च॒ति॒ । \newline
16. प॒च॒ति॒ द्विर् द्विः प॑चति पचति॒ द्विः । \newline
17. द्विः पच॑न्तु॒ पच॑न्तु॒ द्विर् द्विः पच॑न्तु । \newline
18. पच॒ न्त्वितीति॒ पच॑न्तु॒ पच॒ न्त्विति॑ । \newline
19. इत्या॑हा॒हे तीत्या॑ह । \newline
20. आ॒ह॒ तस्मा॒त् तस्मा॑ दाहाह॒ तस्मा᳚त् । \newline
21. तस्मा॒द् द्विर् द्वि स्तस्मा॒त् तस्मा॒द् द्विः । \newline
22. द्विः सं॑ॅवथ्स॒रस्य॑ संॅवथ्स॒रस्य॒ द्विर् द्विः सं॑ॅवथ्स॒रस्य॑ । \newline
23. सं॒ॅव॒थ्स॒रस्य॑ स॒स्यꣳ स॒स्यꣳ सं॑ॅवथ्स॒रस्य॑ संॅवथ्स॒रस्य॑ स॒स्यम् । \newline
24. सं॒ॅव॒थ्स॒रस्येति॑ सं - व॒थ्स॒रस्य॑ । \newline
25. स॒स्यम् प॑च्यते पच्यते स॒स्यꣳ स॒स्यम् प॑च्यते । \newline
26. प॒च्य॒ते॒ वा॒रु॒णी वा॑रु॒णी प॑च्यते पच्यते वारु॒णी । \newline
27. वा॒रु॒ण्यु॑खोखा वा॑रु॒णी वा॑रु॒ण्यु॑खा । \newline
28. उ॒खा ऽभीद्धा॒ ऽभीद्धो॒खोखा ऽभीद्धा᳚ । \newline
29. अ॒भीद्धा॑ मै॒त्रिया॑ मै॒त्रिया॒ ऽभीद्धा॒ ऽभीद्धा॑ मै॒त्रिया᳚ । \newline
30. अ॒भीद्धेत्य॒भि - इ॒द्धा॒ । \newline
31. मै॒त्रियोपोप॑ मै॒त्रिया॑ मै॒त्रियोप॑ । \newline
32. उपै᳚त्ये॒ त्युपोपै॑ति । \newline
33. ए॒ति॒ शान्त्यै॒ शान्त्या॑ एत्येति॒ शान्त्यै᳚ । \newline
34. शान्त्यै॑ दे॒वो दे॒वः शान्त्यै॒ शान्त्यै॑ दे॒वः । \newline
35. दे॒व स्त्वा᳚ त्वा दे॒वो दे॒व स्त्वा᳚ । \newline
36. त्वा॒ स॒वि॒ता स॑वि॒ता त्वा᳚ त्वा सवि॒ता । \newline
37. स॒वि॒तोदुथ् स॑वि॒ता स॑वि॒तोत् । \newline
38. उद् व॑पतु वप॒तूदुद् व॑पतु । \newline
39. व॒प॒ त्वितीति॑ वपतु वप॒ त्विति॑ । \newline
40. इत्या॑हा॒हे तीत्या॑ह । \newline
41. आ॒ह॒ स॒वि॒तृप्र॑सूतः सवि॒तृप्र॑सूत आहाह सवि॒तृप्र॑सूतः । \newline
42. स॒वि॒तृप्र॑सूत ए॒वैव स॑वि॒तृप्र॑सूतः सवि॒तृप्र॑सूत ए॒व । \newline
43. स॒वि॒तृप्र॑सूत॒ इति॑ सवि॒तृ - प्र॒सू॒तः॒ । \newline
44. ए॒वैना॑ मेना मे॒वैवैना᳚म् । \newline
45. ए॒ना॒म् ब्रह्म॑णा॒ ब्रह्म॑णैना मेना॒म् ब्रह्म॑णा । \newline
46. ब्रह्म॑णा दे॒वता॑भिर् दे॒वता॑भि॒र् ब्रह्म॑णा॒ ब्रह्म॑णा दे॒वता॑भिः । \newline
47. दे॒वता॑भि॒ रुदुद् दे॒वता॑भिर् दे॒वता॑भि॒ रुत् । \newline
48. उद् व॑पति वप॒ त्युदुद् व॑पति । \newline
49. व॒प॒ त्यप॑द्यमा॒ना ऽप॑द्यमाना वपति वप॒ त्यप॑द्यमाना । \newline
50. अप॑द्यमाना पृथिवि पृथि॒ व्यप॑द्यमा॒ना ऽप॑द्यमाना पृथिवि । \newline
51. पृ॒थि॒ व्याशा॒ आशाः᳚ पृथिवि पृथि॒ व्याशाः᳚ । \newline
52. आशा॒ दिशो॒ दिश॒ आशा॒ आशा॒ दिशः॑ । \newline
53. दिश॒ आ दिशो॒ दिश॒ आ । \newline
54. आ पृ॑ण पृ॒णा पृ॑ण । \newline
55. पृ॒णे तीति॑ पृण पृ॒णेति॑ । \newline

\textbf{Ghana Paata } \newline

1. जन॑य॒ स्ताभि॒ स्ताभि॒र् जन॑यो॒ जन॑य॒ स्ताभि॑ रे॒वैव ताभि॒र् जन॑यो॒ जन॑य॒ स्ताभि॑ रे॒व । \newline
2. ताभि॑ रे॒वैव ताभि॒ स्ताभि॑ रे॒वैना॑ मेना मे॒व ताभि॒ स्ताभि॑ रे॒वैना᳚म् । \newline
3. ए॒वैना॑ मेना मे॒वैवैना᳚म् पचति पच त्येना मे॒वैवैना᳚म् पचति । \newline
4. ए॒ना॒म् प॒च॒ति॒ प॒च॒ त्ये॒ना॒ मे॒ना॒म् प॒च॒ति॒ ष॒ड्भि ष्ष॒ड्भिः प॑चत्येना मेनाम् पचति ष॒ड्भिः । \newline
5. प॒च॒ति॒ ष॒ड्भि ष्ष॒ड्भिः प॑चति पचति ष॒ड्भिः प॑चति पचति ष॒ड्भिः प॑चति पचति ष॒ड्भिः प॑चति । \newline
6. ष॒ड्भिः प॑चति पचति ष॒ड्भि ष्ष॒ड्भिः प॑चति॒ षट् थ्षट् प॑चति ष॒ड्भि ष्ष॒ड्भिः प॑चति॒ षट् । \newline
7. ष॒ड्भिरिति॑ षट् - भिः । \newline
8. प॒च॒ति॒ षट् थ्षट् प॑चति पचति॒ षड् वै वै षट् प॑चति पचति॒ षड् वै । \newline
9. षड् वै वै षट् थ्षड् वा ऋ॒तव॑ ऋ॒तवो॒ वै षट् थ्षड् वा ऋ॒तवः॑ । \newline
10. वा ऋ॒तव॑ ऋ॒तवो॒ वै वा ऋ॒तव॑ ऋ॒तुभिर्॑. ऋ॒तुभिर्॑. ऋ॒तवो॒ वै वा ऋ॒तव॑ ऋ॒तुभिः॑ । \newline
11. ऋ॒तव॑ ऋ॒तुभिर्॑. ऋ॒तुभिर्॑. ऋ॒तव॑ ऋ॒तव॑ ऋ॒तुभि॑ रे॒वैव र्‌तुभिर्॑. ऋ॒तव॑ ऋ॒तव॑ ऋ॒तुभि॑ रे॒व । \newline
12. ऋ॒तुभि॑ रे॒वैव र्‌तुभिर्॑. ऋ॒तुभि॑ रे॒वैना॑ मेना मे॒व र्‌तुभिर्॑. ऋ॒तुभि॑ रे॒वैना᳚म् । \newline
13. ऋ॒तुभि॒रित्यृ॒तु - भिः॒ । \newline
14. ए॒वैना॑ मेना मे॒वैवैना᳚म् पचति पच त्येना मे॒वैवैना᳚म् पचति । \newline
15. ए॒ना॒म् प॒च॒ति॒ प॒च॒ त्ये॒ना॒ मे॒ना॒म् प॒च॒ति॒ द्विर् द्विः प॑च त्येना मेनाम् पचति॒ द्विः । \newline
16. प॒च॒ति॒ द्विर् द्विः प॑चति पचति॒ द्विः पच॑न्तु॒ पच॑न्तु॒ द्विः प॑चति पचति॒ द्विः पच॑न्तु । \newline
17. द्विः पच॑न्तु॒ पच॑न्तु॒ द्विर् द्विः पच॒ न्त्वितीति॒ पच॑न्तु॒ द्विर् द्विः पच॒ न्त्विति॑ । \newline
18. पच॒ न्त्वितीति॒ पच॑न्तु॒ पच॒ न्त्वित्या॑ हा॒हेति॒ पच॑न्तु॒ पच॒ न्त्वित्या॑ह । \newline
19. इत्या॑हा॒हे तीत्या॑ह॒ तस्मा॒त् तस्मा॑ दा॒हे तीत्या॑ह॒ तस्मा᳚त् । \newline
20. आ॒ह॒ तस्मा॒त् तस्मा॑ दाहाह॒ तस्मा॒द् द्विर् द्वि स्तस्मा॑ दाहाह॒ तस्मा॒द् द्विः । \newline
21. तस्मा॒द् द्विर् द्वि स्तस्मा॒त् तस्मा॒द् द्विः सं॑ॅवथ्स॒रस्य॑ संॅवथ्स॒रस्य॒ द्वि स्तस्मा॒त् तस्मा॒द् द्विः सं॑ॅवथ्स॒रस्य॑ । \newline
22. द्विः सं॑ॅवथ्स॒रस्य॑ संॅवथ्स॒रस्य॒ द्विर् द्विः सं॑ॅवथ्स॒रस्य॑ स॒स्यꣳ स॒स्यꣳ सं॑ॅवथ्स॒रस्य॒ द्विर् द्विः सं॑ॅवथ्स॒रस्य॑ स॒स्यम् । \newline
23. सं॒ॅव॒थ्स॒रस्य॑ स॒स्यꣳ स॒स्यꣳ सं॑ॅवथ्स॒रस्य॑ संॅवथ्स॒रस्य॑ स॒स्यम् प॑च्यते पच्यते स॒स्यꣳ सं॑ॅवथ्स॒रस्य॑ संॅवथ्स॒रस्य॑ स॒स्यम् प॑च्यते । \newline
24. सं॒ॅव॒थ्स॒रस्येति॑ सं - व॒थ्स॒रस्य॑ । \newline
25. स॒स्यम् प॑च्यते पच्यते स॒स्यꣳ स॒स्यम् प॑च्यते वारु॒णी वा॑रु॒णी प॑च्यते स॒स्यꣳ स॒स्यम् प॑च्यते वारु॒णी । \newline
26. प॒च्य॒ते॒ वा॒रु॒णी वा॑रु॒णी प॑च्यते पच्यते वारु॒ण्यु॑खोखा वा॑रु॒णी प॑च्यते पच्यते वारु॒ण्यु॑खा । \newline
27. वा॒रु॒ण्यु॑खोखा वा॑रु॒णी वा॑रु॒ण्यु॑खा ऽभीद्धा॒ ऽभीद्धो॒खा वा॑रु॒णी वा॑रु॒ण्यु॑खा ऽभीद्धा᳚ । \newline
28. उ॒खा ऽभीद्धा॒ ऽभीद्धो॒खोखा ऽभीद्धा॑ मै॒त्रिया॑ मै॒त्रिया॒ ऽभीद्धो॒खोखा ऽभीद्धा॑ मै॒त्रिया᳚ । \newline
29. अ॒भीद्धा॑ मै॒त्रिया॑ मै॒त्रिया॒ ऽभीद्धा॒ ऽभीद्धा॑ मै॒त्रियोपोप॑ मै॒त्रिया॒ ऽभीद्धा॒ ऽभीद्धा॑ मै॒त्रियोप॑ । \newline
30. अ॒भीद्धेत्य॒भि - इ॒द्धा॒ । \newline
31. मै॒त्रियोपोप॑ मै॒त्रिया॑ मै॒त्रियोपै᳚ त्ये॒त्युप॑ मै॒त्रिया॑ मै॒त्रियोपै॑ति । \newline
32. उपै᳚त्ये॒ त्युपोपै॑ति॒ शान्त्यै॒ शान्त्या॑ ए॒त्युपोपै॑ति॒ शान्त्यै᳚ । \newline
33. ए॒ति॒ शान्त्यै॒ शान्त्या॑ एत्येति॒ शान्त्यै॑ दे॒वो दे॒वः शान्त्या॑ एत्येति॒ शान्त्यै॑ दे॒वः । \newline
34. शान्त्यै॑ दे॒वो दे॒वः शान्त्यै॒ शान्त्यै॑ दे॒व स्त्वा᳚ त्वा दे॒वः शान्त्यै॒ शान्त्यै॑ दे॒व स्त्वा᳚ । \newline
35. दे॒व स्त्वा᳚ त्वा दे॒वो दे॒व स्त्वा॑ सवि॒ता स॑वि॒ता त्वा॑ दे॒वो दे॒व स्त्वा॑ सवि॒ता । \newline
36. त्वा॒ स॒वि॒ता स॑वि॒ता त्वा᳚ त्वा सवि॒तोदुथ् स॑वि॒ता त्वा᳚ त्वा सवि॒तोत् । \newline
37. स॒वि॒तोदुथ् स॑वि॒ता स॑वि॒तोद् व॑पतु वप॒तूथ् स॑वि॒ता स॑वि॒तोद् व॑पतु । \newline
38. उद् व॑पतु वप॒तू दुद् व॑प॒ त्वितीति॑ वप॒तूदुद् व॑प॒ त्विति॑ । \newline
39. व॒प॒त्वितीति॑ वपतु वप॒ त्वित्या॑हा॒हेति॑ वपतु वप॒ त्वित्या॑ह । \newline
40. इत्या॑हा॒हे तीत्या॑ह सवि॒तृप्र॑सूतः सवि॒तृप्र॑सूत आ॒हे तीत्या॑ह सवि॒तृप्र॑सूतः । \newline
41. आ॒ह॒ स॒वि॒तृप्र॑सूतः सवि॒तृप्र॑सूत आहाह सवि॒तृप्र॑सूत ए॒वैव स॑वि॒तृप्र॑सूत आहाह सवि॒तृप्र॑सूत ए॒व । \newline
42. स॒वि॒तृप्र॑सूत ए॒वैव स॑वि॒तृप्र॑सूतः सवि॒तृप्र॑सूत ए॒वैना॑ मेना मे॒व स॑वि॒तृप्र॑सूतः सवि॒तृप्र॑सूत ए॒वैना᳚म् । \newline
43. स॒वि॒तृप्र॑सूत॒ इति॑ सवि॒तृ - प्र॒सू॒तः॒ । \newline
44. ए॒वैना॑ मेना मे॒वैवैना॒म् ब्रह्म॑णा॒ ब्रह्म॑णैना मे॒वैवैना॒म् ब्रह्म॑णा । \newline
45. ए॒ना॒म् ब्रह्म॑णा॒ ब्रह्म॑णैना मेना॒म् ब्रह्म॑णा दे॒वता॑भिर् दे॒वता॑भि॒र् ब्रह्म॑णैना मेना॒म् ब्रह्म॑णा दे॒वता॑भिः । \newline
46. ब्रह्म॑णा दे॒वता॑भिर् दे॒वता॑भि॒र् ब्रह्म॑णा॒ ब्रह्म॑णा दे॒वता॑भि॒ रुदुद् दे॒वता॑भि॒र् ब्रह्म॑णा॒ ब्रह्म॑णा दे॒वता॑भि॒ रुत् । \newline
47. दे॒वता॑भि॒ रुदुद् दे॒वता॑भिर् दे॒वता॑भि॒ रुद् व॑पति वप॒ त्युद् दे॒वता॑भिर् दे॒वता॑भि॒ रुद् व॑पति । \newline
48. उद् व॑पति वप॒ त्युदुद् व॑प॒ त्यप॑द्यमा॒ना ऽप॑द्यमाना वप॒ त्युदुद् व॑प॒ त्यप॑द्यमाना । \newline
49. व॒प॒ त्यप॑द्यमा॒ना ऽप॑द्यमाना वपति वप॒ त्यप॑द्यमाना पृथिवि पृथि॒ व्यप॑द्यमाना वपति वप॒ त्यप॑द्यमाना पृथिवि । \newline
50. अप॑द्यमाना पृथिवि पृथि॒ व्यप॑द्यमा॒ना ऽप॑द्यमाना पृथि॒व्याशा॒ आशाः᳚ पृथि॒ व्यप॑द्यमा॒ना ऽप॑द्यमाना पृथि॒व्याशाः᳚ । \newline
51. पृ॒थि॒व्याशा॒ आशाः᳚ पृथिवि पृथि॒व्याशा॒ दिशो॒ दिश॒ आशाः᳚ पृथिवि पृथि॒व्याशा॒ दिशः॑ । \newline
52. आशा॒ दिशो॒ दिश॒ आशा॒ आशा॒ दिश॒ आ दिश॒ आशा॒ आशा॒ दिश॒ आ । \newline
53. दिश॒ आ दिशो॒ दिश॒ आ पृ॑ण पृ॒णा दिशो॒ दिश॒ आ पृ॑ण । \newline
54. आ पृ॑ण पृ॒णा पृ॒णे तीति॑ पृ॒णा पृ॒णेति॑ । \newline
55. पृ॒णे तीति॑ पृण पृ॒णे त्या॑हा॒हेति॑ पृण पृ॒णे त्या॑ह । \newline
\pagebreak
\markright{ TS 5.1.7.4  \hfill https://www.vedavms.in \hfill}

\section{ TS 5.1.7.4 }

\textbf{TS 5.1.7.4 } \newline
\textbf{Samhita Paata} \newline

-त्या॑ह॒ तस्मा॑द॒ग्निः सर्वा॒ दिशोऽनु॒ विभा॒त्युत्ति॑ष्ठ बृह॒ती भ॑वो॒र्द्ध्वा ति॑ष्ठ ध्रु॒वा त्वमित्या॑ह॒ प्रति॑ष्ठित्या असु॒र्यं॑ पात्र॒मना᳚च्छृण्ण॒मा-च्छृ॑णत्ति देव॒त्रा-ऽक॑रजक्षी॒रेणा-ऽऽ*च्छृ॑णत्ति पर॒मं ॅवा ए॒तत् पयो॒ यद॑जक्षी॒रं प॑र॒मेणै॒वैनां॒ पय॒साऽऽच्छृ॑णत्ति॒ यजु॑षा॒ व्यावृ॑त्त्यै॒ छन्दो॑भि॒रा च्छृ॑णत्ति॒ छन्दो॑भि॒र्वा ए॒षा ( ) क्रि॑यते॒ छन्दो॑भिरे॒व छन्दाꣳ॒॒स्या च्छृ॑णत्ति ॥ \newline

\textbf{Pada Paata} \newline

इति॑ । आ॒ह॒ । तस्मा᳚त् । अ॒ग्निः । सर्वाः᳚ । दिशः॑ । अनु॑ । वीति॑ । भा॒ति॒ । उदिति॑ । ति॒ष्ठ॒ । बृ॒ह॒ती । भ॒व॒ । ऊ॒र्ध्वा । ति॒ष्ठ॒ । ध्रु॒वा । त्वम् । इति॑ । आ॒ह॒ । प्रति॑ष्ठित्या॒ इति॒ प्रति॑ - स्थि॒त्यै॒ । अ॒सु॒र्य᳚म् । पात्र᳚म् । अना᳚च्छृण्ण॒मित्यना᳚ - छृ॒ण्ण॒म् । एति॑ । छृ॒ण॒त्ति॒ । दे॒व॒त्रेति॑ देव-त्रा । अ॒कः॒ । अ॒ज॒क्षी॒रेणेत्य॑ज - क्षी॒रेण॑ । एति॑ । छृ॒ण॒त्ति॒ । प॒र॒मम् । वै । ए॒तत् । पयः॑ । यत् । अ॒ज॒क्षी॒रमित्य॑ज - क्षी॒रम् । प॒र॒मेण॑ । ए॒व । ए॒ना॒म् । पय॑सा । एति॑ । छृ॒ण॒त्ति॒ । यजु॑षा । व्यावृ॑त्त्या॒ इति॑ वि - आवृ॑त्त्यै । छन्दो॑भि॒रिति॒ छन्दः॑ - भिः॒ । एति॑ । छृ॒ण॒त्ति॒ । छन्दो॑भि॒रिति॒ छन्दः॑ - भिः॒ । वै । ए॒षा ( ) । क्रि॒य॒ते॒ । छन्दो॑भि॒रिति॒ छन्दः॑ - भिः॒ । ए॒व । छन्दाꣳ॑सि । एति॑ । छृ॒ण॒त्ति॒ ॥  \newline


\textbf{Krama Paata} \newline

इत्या॑ह । आ॒ह॒ तस्मा᳚त् । तस्मा॑द॒ग्निः । अ॒ग्निः सर्वाः᳚ । सर्वा॒ दिशः॑ । दिशोऽनु॑ । अनु॒ वि । वि भा॑ति । भा॒त्युत् । उत्ति॑ष्ठ । ति॒ष्ठ॒ बृ॒ह॒ती । बृ॒ह॒ती भ॑व । भ॒वो॒र्द्ध्वा । ऊ॒र्द्ध्वा ति॑ष्ठ । ति॒ष्ठ॒ ध्रु॒वा । ध्रु॒वा त्वम् । त्वमिति॑ । इत्या॑ह । आ॒ह॒ प्रति॑ष्ठित्यै । प्रति॑ष्ठित्या असु॒र्य᳚म् । प्रति॑ष्ठित्या॒ इति॒ प्रति॑ - स्थि॒त्यै॒ । अ॒सु॒र्य॑म् पात्र᳚म् । पात्र॒मना᳚च्छृण्णम् । अना᳚च्छृण्ण॒मा । अना᳚च्छृण्ण॒मित्यना᳚ - छृ॒ण्ण॒म् । आच्छृ॑णत्ति । छृ॒ण॒त्ति॒ दे॒व॒त्रा । दे॒व॒त्राऽकः॑ । दे॒व॒त्रेति॑ देव - त्रा । अ॒क॒र॒ज॒क्षी॒रेण॑ । अ॒ज॒क्षी॒रेणा । अ॒ज॒क्षी॒रेणेत्य॑ज - क्षी॒रेण॑ । आच्छृ॑णत्ति । छृ॒ण॒त्ति॒ प॒र॒मम् । प॒र॒मम् ॅवै । वा ए॒तत् । ए॒तत् पयः॑ । पयो॒ यत् । यद॑जक्षी॒रम् । अ॒ज॒क्षी॒रम् प॑र॒मेण॑ । अ॒ज॒क्षी॒रमित्य॑ज - क्षी॒रम् । प॒र॒मेणै॒व । ए॒वैना᳚म् । ए॒ना॒म् पय॑सा । पय॒सा । आच्छृ॑णत्ति । छृ॒ण॒त्ति॒ यजु॑षा । यजु॑षा॒ व्यावृ॑त्त्यै । व्यावृ॑त्त्यै॒ छन्दो॑भिः । व्यावृ॑त्त्या॒ इति॑ वि - आवृ॑त्त्यै । छन्दो॑भि॒रा । छन्दो॑भि॒रिति॒ छन्दः॑ - भिः॒ । आच्छृ॑णत्ति । छृ॒ण॒त्ति॒ छन्दो॑भिः । छन्दो॑भि॒र् वै । छन्दो॑भि॒रिति॒ छन्दः॑ - भिः॒ । वा ए॒षा ( ) । ए॒षा क्रि॑यते । क्रि॒य॒ते॒ छन्दो॑भिः । छन्दो॑भिरे॒व । छन्दो॑भि॒रिति॒ छन्दः॑ - भिः॒ । ए॒व छन्दाꣳ॑सि । छन्दाꣳ॒॒स्या । आच्छृ॑णत्ति । छृ॒ण॒त्तीति॑ छृणत्ति । \newline

\textbf{Jatai Paata} \newline

1. इत्या॑हा॒हे तीत्या॑ह । \newline
2. आ॒ह॒ तस्मा॒त् तस्मा॑ दाहाह॒ तस्मा᳚त् । \newline
3. तस्मा॑ द॒ग्नि र॒ग्नि स्तस्मा॒त् तस्मा॑ द॒ग्निः । \newline
4. अ॒ग्निः सर्वाः॒ सर्वा॑ अ॒ग्नि र॒ग्निः सर्वाः᳚ । \newline
5. सर्वा॒ दिशो॒ दिशः॒ सर्वाः॒ सर्वा॒ दिशः॑ । \newline
6. दिशो ऽन्वनु॒ दिशो॒ दिशो ऽनु॑ । \newline
7. अनु॒ वि व्यन्वनु॒ वि । \newline
8. वि भा॑ति भाति॒ वि वि भा॑ति । \newline
9. भा॒त्युदुद् भा॑ति भा॒त्युत् । \newline
10. उत् ति॑ष्ठ ति॒ष्ठोदुत् ति॑ष्ठ । \newline
11. ति॒ष्ठ॒ बृ॒ह॒ती बृ॑ह॒ती ति॑ष्ठ तिष्ठ बृह॒ती । \newline
12. बृ॒ह॒ती भ॑व भव बृह॒ती बृ॑ह॒ती भ॑व । \newline
13. भ॒वो॒र्द्ध्वोर्द्ध्वा भ॑व भवो॒र्द्ध्वा । \newline
14. ऊ॒र्द्ध्वा ति॑ष्ठ तिष्ठो॒र्द्ध्वोर्द्ध्वा ति॑ष्ठ । \newline
15. ति॒ष्ठ॒ ध्रु॒वा ध्रु॒वा ति॑ष्ठ तिष्ठ ध्रु॒वा । \newline
16. ध्रु॒वा त्वम् त्वम् ध्रु॒वा ध्रु॒वा त्वम् । \newline
17. त्व मितीति॒ त्वम् त्व मिति॑ । \newline
18. इत्या॑हा॒हे तीत्या॑ह । \newline
19. आ॒ह॒ प्रति॑ष्ठित्यै॒ प्रति॑ष्ठित्या आहाह॒ प्रति॑ष्ठित्यै । \newline
20. प्रति॑ष्ठित्या असु॒र्य॑ मसु॒र्य॑म् प्रति॑ष्ठित्यै॒ प्रति॑ष्ठित्या असु॒र्य᳚म् । \newline
21. प्रति॑ष्ठित्या॒ इति॒ प्रति॑ - स्थि॒त्यै॒ । \newline
22. अ॒सु॒र्य॑म् पात्र॒म् पात्र॑ मसु॒र्य॑ मसु॒र्य॑म् पात्र᳚म् । \newline
23. पात्र॒ मना᳚च्छृण्ण॒ मना᳚च्छृण्ण॒म् पात्र॒म् पात्र॒ मना᳚च्छृण्णम् । \newline
24. अना᳚च्छृण्ण॒ मा ऽना᳚च्छृण्ण॒ मना᳚च्छृण्ण॒ मा । \newline
25. अना᳚च्छृण्ण॒मित्यना᳚ - छृ॒ण्ण॒म् । \newline
26. आ च्छृ॑णत्ति छृण॒त्तिया च्छृ॑णत्ति । \newline
27. छृ॒ण॒त्ति॒ दे॒व॒त्रा दे॑व॒त्रा छृ॑णत्ति छृणत्ति देव॒त्रा । \newline
28. दे॒व॒त्रा ऽक॑ रकर् देव॒त्रा दे॑व॒त्रा ऽकः॑ । \newline
29. दे॒व॒त्रेति॑ देव - त्रा । \newline
30. अ॒क॒ र॒ज॒क्षी॒रेणा॑ जक्षी॒रेणा॑क रक रजक्षी॒रेण॑ । \newline
31. अ॒ज॒क्षी॒रेणा ऽज॑क्षी॒रेणा॑ जक्षी॒रेणा । \newline
32. अ॒ज॒क्षी॒रेणेत्य॑ज - क्षी॒रेण॑ । \newline
33. आ च्छृ॑णत्ति छृण॒त्त्या च्छृ॑णत्ति । \newline
34. छृ॒ण॒त्ति॒ प॒र॒मम् प॑र॒मम् छृ॑णत्ति छृणत्ति पर॒मम् । \newline
35. प॒र॒मं ॅवै वै प॑र॒मम् प॑र॒मं ॅवै । \newline
36. वा ए॒त दे॒तद् वै वा ए॒तत् । \newline
37. ए॒तत् पयः॒ पय॑ ए॒त दे॒तत् पयः॑ । \newline
38. पयो॒ यद् यत् पयः॒ पयो॒ यत् । \newline
39. यद॑जक्षी॒र म॑जक्षी॒रं ॅयद् यद॑जक्षी॒रम् । \newline
40. अ॒ज॒क्षी॒रम् प॑र॒मेण॑ पर॒मेणा॑ जक्षी॒र म॑जक्षी॒रम् प॑र॒मेण॑ । \newline
41. अ॒ज॒क्षी॒रमित्य॑ज - क्षी॒रम् । \newline
42. प॒र॒मे णै॒वैव प॑र॒मेण॑ पर॒मे णै॒व । \newline
43. ए॒वैना॑ मेना मे॒वैवैना᳚म् । \newline
44. ए॒ना॒म् पय॑सा॒ पय॑सैना मेना॒म् पय॑सा । \newline
45. पय॒सा ऽऽपय॑सा॒ पय॒सा । \newline
46. आ च्छृ॑णत्ति छृण॒त्त्या च्छृ॑णत्ति । \newline
47. छृ॒ण॒त्ति॒ यजु॑षा॒ यजु॑षा छृणत्ति छृणत्ति॒ यजु॑षा । \newline
48. यजु॑षा॒ व्यावृ॑त्त्यै॒ व्यावृ॑त्त्यै॒ यजु॑षा॒ यजु॑षा॒ व्यावृ॑त्त्यै । \newline
49. व्यावृ॑त्त्यै॒ छन्दो॑भि॒ श्छन्दो॑भि॒र् व्यावृ॑त्त्यै॒ व्यावृ॑त्त्यै॒ छन्दो॑भिः । \newline
50. व्यावृ॑त्त्या॒ इति॑ वि - आवृ॑त्त्यै । \newline
51. छन्दो॑भि॒रा छन्दो॑भि॒ श्छन्दो॑भि॒रा । \newline
52. छन्दो॑भि॒रिति॒ छन्दः॑ - भिः॒ । \newline
53. आ च्छृ॑णत्ति छृण॒त्त्या च्छृ॑णत्ति । \newline
54. छृ॒ण॒त्ति॒ छन्दो॑भि॒ श्छन्दो॑भि श्छृणत्ति छृणत्ति॒ छन्दो॑भिः । \newline
55. छन्दो॑भि॒र् वै वै छन्दो॑भि॒ श्छन्दो॑भि॒र् वै । \newline
56. छन्दो॑भि॒रिति॒ छन्दः॑ - भिः॒ । \newline
57. वा ए॒षैषा वै वा ए॒षा । \newline
58. ए॒षा क्रि॑यते क्रियत ए॒षैषा क्रि॑यते । \newline
59. क्रि॒य॒ते॒ छन्दो॑भि॒ श्छन्दो॑भिः क्रियते क्रियते॒ छन्दो॑भिः । \newline
60. छन्दो॑भि रे॒वैव छन्दो॑भि॒ श्छन्दो॑भि रे॒व । \newline
61. छन्दो॑भि॒रिति॒ छन्दः॑ - भिः॒ । \newline
62. ए॒व छन्दाꣳ॑सि॒ छन्दाꣳ॑ स्ये॒वैव छन्दाꣳ॑सि । \newline
63. छन्दाꣳ॒॒स्या छन्दाꣳ॑सि॒ छन्दाꣳ॒॒स्या । \newline
64. आ च्छृ॑णत्ति छृण॒त्त्या च्छृ॑णत्ति । \newline
65. छृ॒ण॒त्तीति॑ छृणत्ति । \newline

\textbf{Ghana Paata } \newline

1. इत्या॑हा॒हे तीत्या॑ह॒ तस्मा॒त् तस्मा॑ दा॒हे तीत्या॑ह॒ तस्मा᳚त् । \newline
2. आ॒ह॒ तस्मा॒त् तस्मा॑ दाहाह॒ तस्मा॑ द॒ग्नि र॒ग्नि स्तस्मा॑ दाहाह॒ तस्मा॑ द॒ग्निः । \newline
3. तस्मा॑ द॒ग्नि र॒ग्नि स्तस्मा॒त् तस्मा॑ द॒ग्निः सर्वाः॒ सर्वा॑ अ॒ग्नि स्तस्मा॒त् तस्मा॑ द॒ग्निः सर्वाः᳚ । \newline
4. अ॒ग्निः सर्वाः॒ सर्वा॑ अ॒ग्नि र॒ग्निः सर्वा॒ दिशो॒ दिशः॒ सर्वा॑ अ॒ग्नि र॒ग्निः सर्वा॒ दिशः॑ । \newline
5. सर्वा॒ दिशो॒ दिशः॒ सर्वाः॒ सर्वा॒ दिशो ऽन्वनु॒ दिशः॒ सर्वाः॒ सर्वा॒ दिशो ऽनु॑ । \newline
6. दिशो ऽन्वनु॒ दिशो॒ दिशो ऽनु॒ वि व्यनु॒ दिशो॒ दिशो ऽनु॒ वि । \newline
7. अनु॒ वि व्यन्वनु॒ वि भा॑ति भाति॒ व्यन्वनु॒ वि भा॑ति । \newline
8. वि भा॑ति भाति॒ वि वि भा॒ त्युदुद् भा॑ति॒ वि वि भा॒त्युत् । \newline
9. भा॒ त्युदुद् भा॑ति भा॒त्युत् ति॑ष्ठ ति॒ष्ठोद् भा॑ति भा॒त्युत् ति॑ष्ठ । \newline
10. उत् ति॑ष्ठ ति॒ष्ठोदुत् ति॑ष्ठ बृह॒ती बृ॑ह॒ती ति॒ष्ठोदुत् ति॑ष्ठ बृह॒ती । \newline
11. ति॒ष्ठ॒ बृ॒ह॒ती बृ॑ह॒ती ति॑ष्ठ तिष्ठ बृह॒ती भ॑व भव बृह॒ती ति॑ष्ठ तिष्ठ बृह॒ती भ॑व । \newline
12. बृ॒ह॒ती भ॑व भव बृह॒ती बृ॑ह॒ती भ॑वो॒ र्द्ध्वोर्द्ध्वा भ॑व बृह॒ती बृ॑ह॒ती भ॑वो॒र्द्ध्वा । \newline
13. भ॒वो॒ र्द्ध्वोर्द्ध्वा भ॑व भवो॒र्द्ध्वा ति॑ष्ठ तिष्ठो॒र्द्ध्वा भ॑व भवो॒र्द्ध्वा ति॑ष्ठ । \newline
14. ऊ॒र्द्ध्वा ति॑ष्ठ तिष्ठो॒ र्द्ध्वोर्द्ध्वा ति॑ष्ठ ध्रु॒वा ध्रु॒वा ति॑ष्ठो॒ र्द्ध्वोर्द्ध्वा ति॑ष्ठ ध्रु॒वा । \newline
15. ति॒ष्ठ॒ ध्रु॒वा ध्रु॒वा ति॑ष्ठ तिष्ठ ध्रु॒वा त्वम् त्वम् ध्रु॒वा ति॑ष्ठ तिष्ठ ध्रु॒वा त्वम् । \newline
16. ध्रु॒वा त्वम् त्वम् ध्रु॒वा ध्रु॒वा त्व मितीति॒ त्वम् ध्रु॒वा ध्रु॒वा त्व मिति॑ । \newline
17. त्व मितीति॒ त्वम् त्व मित्या॑ हा॒हेति॒ त्वम् त्व मित्या॑ह । \newline
18. इत्या॑हा॒हे तीत्या॑ह॒ प्रति॑ष्ठित्यै॒ प्रति॑ष्ठित्या आ॒हे तीत्या॑ह॒ प्रति॑ष्ठित्यै । \newline
19. आ॒ह॒ प्रति॑ष्ठित्यै॒ प्रति॑ष्ठित्या आहाह॒ प्रति॑ष्ठित्या असु॒र्य॑ मसु॒र्य॑म् प्रति॑ष्ठित्या आहाह॒ प्रति॑ष्ठित्या असु॒र्य᳚म् । \newline
20. प्रति॑ष्ठित्या असु॒र्य॑ मसु॒र्य॑म् प्रति॑ष्ठित्यै॒ प्रति॑ष्ठित्या असु॒र्य॑म् पात्र॒म् पात्र॑ मसु॒र्य॑म् प्रति॑ष्ठित्यै॒ प्रति॑ष्ठित्या असु॒र्य॑म् पात्र᳚म् । \newline
21. प्रति॑ष्ठित्या॒ इति॒ प्रति॑ - स्थि॒त्यै॒ । \newline
22. अ॒सु॒र्य॑म् पात्र॒म् पात्र॑ मसु॒र्य॑ मसु॒र्य॑म् पात्र॒ मना᳚च्छृण्ण॒ मना᳚च्छृण्ण॒म् पात्र॑ मसु॒र्य॑ मसु॒र्य॑म् पात्र॒ मना᳚च्छृण्णम् । \newline
23. पात्र॒ मना᳚च्छृण्ण॒ मना᳚च्छृण्ण॒म् पात्र॒म् पात्र॒ मना᳚च्छृण्ण॒ मा ऽना᳚च्छृण्ण॒म् पात्र॒म् पात्र॒ मना᳚च्छृण्ण॒ मा । \newline
24. अना᳚च्छृण्ण॒ मा ऽना᳚च्छृण्ण॒ मना᳚च्छृण्ण॒ मा छृ॑णत्ति छृणत्ति॒दा ऽना᳚च्छृण्ण॒ मना᳚च्छृण्ण॒ मा छृ॑णत्ति । \newline
25. अना᳚च्छृण्ण॒मित्यना᳚ - छृ॒ण्ण॒म् । \newline
26. आ च्छृ॑णत्ति छृण॒त्त्या च्छृ॑णत्ति देव॒त्रा दे॑व॒त्रा छृ॑ण॒त्त्या च्छृ॑णत्ति देव॒त्रा । \newline
27. छृ॒ण॒त्ति॒ दे॒व॒त्रा दे॑व॒त्रा छृ॑णत्ति छृणत्ति देव॒त्रा ऽक॑ रकर् देव॒त्रा छृ॑णत्ति छृणत्ति देव॒त्रा ऽकः॑ । \newline
28. दे॒व॒त्रा ऽक॑ रकर् देव॒त्रा दे॑व॒त्रा ऽक॑ रजक्षी॒रेणा॑ जक्षी॒रेणा॑कर् देव॒त्रा दे॑व॒त्रा ऽक॑ रजक्षी॒रेण॑ । \newline
29. दे॒व॒त्रेति॑ देव - त्रा । \newline
30. अ॒क॒ र॒ज॒क्षी॒रेणा॑ जक्षी॒रेणा॑क रक रजक्षी॒रेणा ऽज॑क्षी॒रेणा॑क रक रजक्षी॒रेणा । \newline
31. अ॒ज॒क्षी॒रेणा ऽज॑क्षी॒रेणा॑ जक्षी॒रेणा च्छृ॑णत्ति छृण॒त्त्या ऽज॑क्षी॒रेणा॑ जक्षी॒रेणा च्छृ॑णत्ति । \newline
32. अ॒ज॒क्षी॒रेणेत्य॑ज - क्षी॒रेण॑ । \newline
33. आ छृ॑णत्ति छृण॒त्त्या छृ॑णत्ति पर॒मम् प॑र॒मम् छृ॑ण॒त्त्या छृ॑णत्ति पर॒मम् । \newline
34. छृ॒ण॒त्ति॒ प॒र॒मम् प॑र॒मम् छृ॑णत्ति छृणत्ति पर॒मं ॅवै वै प॑र॒मम् छृ॑णत्ति छृणत्ति पर॒मं ॅवै । \newline
35. प॒र॒मं ॅवै वै प॑र॒मम् प॑र॒मं ॅवा ए॒त दे॒तद् वै प॑र॒मम् प॑र॒मं ॅवा ए॒तत् । \newline
36. वा ए॒त दे॒तद् वै वा ए॒तत् पयः॒ पय॑ ए॒तद् वै वा ए॒तत् पयः॑ । \newline
37. ए॒तत् पयः॒ पय॑ ए॒त दे॒तत् पयो॒ यद् यत् पय॑ ए॒त दे॒तत् पयो॒ यत् । \newline
38. पयो॒ यद् यत् पयः॒ पयो॒ यद॑जक्षी॒र म॑जक्षी॒रं ॅयत् पयः॒ पयो॒ यद॑जक्षी॒रम् । \newline
39. यद॑जक्षी॒र म॑जक्षी॒रं ॅयद् यद॑जक्षी॒रम् प॑र॒मेण॑ पर॒मेणा॑ जक्षी॒रं ॅयद् यद॑जक्षी॒रम् प॑र॒मेण॑ । \newline
40. अ॒ज॒क्षी॒रम् प॑र॒मेण॑ पर॒मेणा॑ जक्षी॒र म॑जक्षी॒रम् प॑र॒मे णै॒वैव प॑र॒मेणा॑ जक्षी॒र म॑जक्षी॒रम् प॑र॒मेणै॒व । \newline
41. अ॒ज॒क्षी॒रमित्य॑ज - क्षी॒रम् । \newline
42. प॒र॒मे णै॒वैव प॑र॒मेण॑ पर॒मे णै॒वैना॑ मेना मे॒व प॑र॒मेण॑ पर॒मे णै॒वैना᳚म् । \newline
43. ए॒वैना॑ मेना मे॒वैवैना॒म् पय॑सा॒ पय॑सैना मे॒वैवैना॒म् पय॑सा । \newline
44. ए॒ना॒म् पय॑सा॒ पय॑सैना मेना॒म् पय॒सा ऽऽपय॑सैना मेना॒म् पय॒सा । \newline
45. पय॒सा ऽऽपय॑सा॒ पय॒साऽऽ च्छृ॑णत्ति छृण॒त्त्या पय॑सा॒ पय॒साऽऽ च्छृ॑णत्ति । \newline
46. आ च्छृ॑णत्ति छृण॒त्त्या छृ॑णत्ति॒ यजु॑षा॒ यजु॑षा छृण॒त्त्या च्छृ॑णत्ति॒ यजु॑षा । \newline
47. छृ॒ण॒त्ति॒ यजु॑षा॒ यजु॑षा छृणत्ति छृणत्ति॒ यजु॑षा॒ व्यावृ॑त्त्यै॒ व्यावृ॑त्त्यै॒ यजु॑षा छृणत्ति छृणत्ति॒ यजु॑षा॒ व्यावृ॑त्त्यै । \newline
48. यजु॑षा॒ व्यावृ॑त्त्यै॒ व्यावृ॑त्त्यै॒ यजु॑षा॒ यजु॑षा॒ व्यावृ॑त्त्यै॒ छन्दो॑भि॒ श्छन्दो॑भि॒र् व्यावृ॑त्त्यै॒ यजु॑षा॒ यजु॑षा॒ व्यावृ॑त्त्यै॒ छन्दो॑भिः । \newline
49. व्यावृ॑त्त्यै॒ छन्दो॑भि॒ श्छन्दो॑भि॒र् व्यावृ॑त्त्यै॒ व्यावृ॑त्त्यै॒ छन्दो॑भि॒रा छन्दो॑भि॒र् व्यावृ॑त्त्यै॒ व्यावृ॑त्त्यै॒ छन्दो॑भि॒रा । \newline
50. व्यावृ॑त्त्या॒ इति॑ वि - आवृ॑त्त्यै । \newline
51. छन्दो॑भि॒रा छन्दो॑भि॒ श्छन्दो॑भि॒रा च्छृ॑णत्ति छृण॒त्त्या च्छन्दो॑भि॒ श्छन्दो॑भि॒रा च्छृ॑णत्ति । \newline
52. छन्दो॑भि॒रिति॒ छन्दः॑ - भिः॒ । \newline
53. आ च्छृ॑णत्ति छृण॒त्त्या छृ॑णत्ति॒ छन्दो॑भि॒ श्छन्दो॑भि श्छृण॒त्त्या च्छृ॑णत्ति॒ छन्दो॑भिः । \newline
54. छृ॒ण॒त्ति॒ छन्दो॑भि॒ श्छन्दो॑भि श्छृणत्ति छृणत्ति॒ छन्दो॑भि॒र् वै वै छन्दो॑भि श्छृणत्ति छृणत्ति॒ छन्दो॑भि॒र् वै । \newline
55. छन्दो॑भि॒र् वै वै छन्दो॑भि॒ श्छन्दो॑भि॒र् वा ए॒षैषा वै छन्दो॑भि॒ श्छन्दो॑भि॒र् वा ए॒षा । \newline
56. छन्दो॑भि॒रिति॒ छन्दः॑ - भिः॒ । \newline
57. वा ए॒षैषा वै वा ए॒षा क्रि॑यते क्रियत ए॒षा वै वा ए॒षा क्रि॑यते । \newline
58. ए॒षा क्रि॑यते क्रियत ए॒षैषा क्रि॑यते॒ छन्दो॑भि॒ श्छन्दो॑भिः क्रियत ए॒षैषा क्रि॑यते॒ छन्दो॑भिः । \newline
59. क्रि॒य॒ते॒ छन्दो॑भि॒ श्छन्दो॑भिः क्रियते क्रियते॒ छन्दो॑भि रे॒वैव छन्दो॑भिः क्रियते क्रियते॒ छन्दो॑भि रे॒व । \newline
60. छन्दो॑भि रे॒वैव छन्दो॑भि॒ श्छन्दो॑भि रे॒व छन्दाꣳ॑सि॒ छन्दाꣳ॑ स्ये॒व छन्दो॑भि॒ श्छन्दो॑भि रे॒व छन्दाꣳ॑सि । \newline
61. छन्दो॑भि॒रिति॒ छन्दः॑ - भिः॒ । \newline
62. ए॒व छन्दाꣳ॑सि॒ छन्दाꣳ॑ स्ये॒वैव छन्दाꣳ॒॒स्या छन्दाꣳ॑ स्ये॒वैव छन्दाꣳ॒॒स्या । \newline
63. छन्दाꣳ॒॒स्या छन्दाꣳ॑सि॒ छन्दाꣳ॒॒स्या च्छृ॑णत्ति छृण॒त्त्या छन्दाꣳ॑सि॒ छन्दाꣳ॒॒स्या च्छृ॑णत्ति । \newline
64. आ च्छृ॑णत्ति छृण॒त्त्या च्छृ॑णत्ति । \newline
65. छृ॒ण॒त्तीति॑ छृणत्ति । \newline
\pagebreak
\markright{ TS 5.1.8.1  \hfill https://www.vedavms.in \hfill}

\section{ TS 5.1.8.1 }

\textbf{TS 5.1.8.1 } \newline
\textbf{Samhita Paata} \newline

एक॑विꣳशत्या॒ माषैः᳚ पुरुषशी॒र्॒.ष-मच्छै᳚त्यमे॒द्ध्या वै माषा॑ अमे॒द्ध्यं पु॑रुषशी॒र्॒.ष-म॑मे॒द्ध्यैरे॒वा-स्या॑-मे॒द्ध्यं नि॑रव॒दाय॒ मेद्ध्यं॑ कृ॒त्वा ऽऽह॑र॒त्येक॑विꣳशति-र्भवन्त्येकविꣳ॒॒शो वै पुरु॑षः॒ पुरु॑ष॒स्याऽऽ*प्त्यै॒ व्यृ॑द्धं॒ ॅवा ए॒तत् प्रा॒णैर॑मे॒द्ध्यं ॅयत् पु॑रुषशी॒र्॒.षꣳ स॑प्त॒धा वितृ॑ण्णां ॅवल्मीकव॒पां प्रति॒ नि द॑धाति स॒प्त वै शी॑र्.ष॒ण्याः᳚ प्रा॒णाः प्रा॒णैरे॒वैन॒थ्-सम॑र्द्धयति मेद्ध्य॒त्वाय॒ याव॑न्तो॒ - [  ] \newline

\textbf{Pada Paata} \newline

एक॑विꣳश॒त्येत्येक॑ - विꣳ॒॒श॒त्या॒ । माषैः᳚ । पु॒रु॒ष॒शी॒र॒.षमिति॑ पुरुष - शी॒र॒.षम् । अच्छ॑ । ए॒ति॒ । अ॒मे॒द्ध्याः । वै । माषाः᳚ । अ॒मे॒द्ध्यम् । पु॒रु॒ष॒शी॒र्॒.षमिति॑ पुरुष - शी॒र्॒.॒षम् । अ॒मे॒द्ध्यैः । ए॒व । अ॒स्य॒ । अ॒मे॒द्ध्यम् । नि॒र॒व॒दायेति॑ निः - अ॒व॒दाय॑ । मेद्ध्य᳚म् । कृ॒त्वा । एति॑ । ह॒र॒ति॒ । एक॑विꣳशति॒रित्येक॑ - विꣳ॒॒श॒तिः॒ । भ॒व॒न्ति॒ । ए॒क॒विꣳ॒॒श इत्ये॑क - विꣳ॒॒शः । वै । पुरु॑षः । पुरु॑षस्य । आप्त्यै᳚ । व्यृ॑द्ध॒मिति॒ वि - ऋ॒द्ध॒म् । वै । ए॒तत् । प्रा॒णैरिति॑ प्र - अ॒नैः । अ॒मे॒द्ध्यम् । यत् । पु॒रु॒ष॒शी॒र्॒.षमिति॑ पुरुष - शी॒र्॒.षम् । स॒प्त॒धेति॑ सप्त - धा । वितृ॑ण्णा॒मिति॒ वि - तृ॒ण्णा॒म् । व॒ल्मी॒क॒व॒पामिति॑ वल्मीक-व॒पाम् । प्रति॑ । नीति॑ । द॒धा॒ति॒ । स॒प्त । वै । शी॒र॒.ष॒ण्याः᳚ । प्रा॒णा इति॑ प्र - अ॒नाः । प्रा॒णैरिति॑ प्र - अ॒नैः । ए॒व । ए॒न॒त् । समिति॑ । अ॒द्‌र्ध॒य॒ति॒ । मे॒द्ध्य॒त्वायेति॑ मेद्ध्य - त्वाय॑ । याव॑न्तः ।  \newline


\textbf{Krama Paata} \newline

एक॑विꣳशत्या॒ माषैः᳚ । एक॑विꣳश॒त्येत्येक॑ - विꣳ॒॒श॒त्या॒ । माषैः᳚ पुरुषशी॒र्॒.षम् । पु॒रु॒ष॒शी॒र्॒.षमच्छ॑ । पु॒रु॒ष॒शी॒र्.॒षमिति॑ पुरुष - शी॒र्॒.षम् । अच्छै॑ति । ए॒त्य॒मे॒द्ध्याः । अ॒मे॒द्ध्या वै । वै माषाः᳚ । माषा॑ अमे॒द्ध्यम् । अ॒मे॒द्ध्यम् पु॑रुषशी॒र्॒.षम् । पु॒रु॒ष॒शी॒र्॒.षम॑मे॒द्ध्यैः । पु॒रु॒ष॒शी॒र्॒.षमिति॑ पुरुष - शी॒र्॒.षम् । अ॒मे॒द्ध्यैरे॒व । ए॒वास्य॑ । अ॒स्या॒मे॒द्ध्यम् । अ॒मे॒द्ध्यम् नि॑रव॒दाय॑ । नि॒र॒व॒दाय॒ मेद्ध्य᳚म् । नि॒र॒व॒दायेति॑ निः - अ॒व॒दाय॑ । मेद्ध्य॑म् कृ॒त्वा । कृ॒त्वा ऽऽ ह॑रति । आ ह॑रति । ह॒र॒त्येक॑विꣳशतिः । एक॑विꣳशतिर् भवन्ति । एक॑विꣳशति॒रित्येक॑ - विꣳ॒॒श॒तिः॒ । भ॒व॒न्त्ये॒क॒विꣳ॒॒शः । ए॒क॒विꣳ॒॒शो वै । ए॒क॒विꣳ॒॒श इत्ये॑क - विꣳ॒॒शः । वै पुरु॑षः । पुरु॑षः॒ पुरु॑षस्य । पुरु॑ष॒स्याप्त्यै᳚ । आप्त्यै॒ व्यृ॑द्धम् । व्यृ॑द्ध॒म् ॅवै । व्यृ॑द्ध॒मिति॒ वि - ऋ॒द्ध॒म् । वा ए॒तत् । ए॒तत् प्रा॒णैः । प्रा॒णैर॑मे॒द्ध्यम् । प्रा॒णैरिति॑ प्र - अ॒नैः । अ॒मे॒द्ध्यम् ॅयत् । यत् पु॑रुषशी॒र्॒.षम् । पु॒रु॒ष॒शी॒र्.॒षꣳ स॑प्त॒धा । पु॒रु॒ष॒शी॒र्.॒षमिति॑ पुरुष - शी॒र्.॒षम् । स॒प्त॒धा वितृ॑ण्णाम् । स॒प्त॒धेति॑ सप्त - धा । वितृ॑ण्णाम् ॅवल्मीकव॒पाम् । वितृ॑ण्णा॒मिति॒ वि - तृ॒ण्णा॒म् । व॒ल्मी॒क॒व॒पाम् प्रति॑ । व॒ल्मि॒क॒व॒पामिति॑ वल्मीक - व॒पाम् । प्रति॒ नि । नि द॑धाति । द॒धा॒ति॒ स॒प्त । स॒प्त वै । वै शी॑र्.ष॒ण्याः᳚ । शी॒र्॒.ष॒ण्याः᳚ प्रा॒णाः । प्रा॒णाः प्रा॒णैः । प्रा॒णा इति॑ प्र - अ॒नाः । प्रा॒णैरे॒व । प्रा॒णैरिति॑ प्र - अ॒नैः । ए॒वैन॑त् । ए॒न॒थ् सम् । सम॑र्द्धयति । अ॒र्द्ध॒य॒ति॒ मे॒द्ध्य॒त्वाय॑ । मे॒द्ध्य॒त्वाय॒ याव॑न्तः । मे॒द्ध्य॒त्वायेति॑ मेद्ध्य - त्वाय॑ । याव॑न्तो॒ वै \newline

\textbf{Jatai Paata} \newline

1. एक॑विꣳशत्या॒ माषै॒र् माषै॒रेक॑विꣳश॒ त्यैक॑विꣳशत्या॒ माषैः᳚ । \newline
2. एक॑विꣳश॒त्येत्येक॑ - विꣳ॒॒श॒त्या॒ । \newline
3. माषैः᳚ पुरुषशीर॒.षम् पु॑रुषशीर॒.षम् माषै॒र् माषैः᳚ पुरुषशीर॒.षम् । \newline
4. पु॒रु॒ष॒शी॒र॒.ष मच्छाच्छ॑ पुरुषशीर॒.षम् पु॑रुषशीर॒.ष मच्छ॑ । \newline
5. पु॒रु॒ष॒शी॒र॒.षमिति॑ पुरुष - शी॒र॒.षम् । \newline
6. अच्छै᳚ त्ये॒ त्यच्छा च्छै॑ति । \newline
7. ए॒त्य॒मे॒द्ध्या अ॑मे॒द्ध्या ए᳚त्ये त्यमे॒द्ध्याः । \newline
8. अ॒मे॒द्ध्या वै वा अ॑मे॒द्ध्या अ॑मे॒द्ध्या वै । \newline
9. वै माषा॒ माषा॒ वै वै माषाः᳚ । \newline
10. माषा॑ अमे॒द्ध्य म॑मे॒द्ध्यम् माषा॒ माषा॑ अमे॒द्ध्यम् । \newline
11. अ॒मे॒द्ध्यम् पु॑रुषशी॒र्॒.षम् पु॑रुषशी॒र्॒.ष म॑मे॒द्ध्य म॑मे॒द्ध्यम् पु॑रुषशी॒र्॒.षम् । \newline
12. पु॒रु॒ष॒शी॒र्॒.ष म॑मे॒द्ध्यै र॑मे॒द्ध्यैः पु॑रुषशी॒र्॒.षम् पु॑रुषशी॒र्॒.ष म॑मे॒द्ध्यैः । \newline
13. पु॒रु॒ष॒शी॒र्॒.षमिति॑ पुरुष - शी॒र्॒.॒षम् । \newline
14. अ॒मे॒द्ध्यै रे॒वैवा मे॒द्ध्यै र॑मे॒द्ध्यै रे॒व । \newline
15. ए॒वास्या᳚ स्यै॒वै वास्य॑ । \newline
16. अ॒स्या॒मे॒द्ध्य म॑मे॒द्ध्य म॑स्यास्या मे॒द्ध्यम् । \newline
17. अ॒मे॒द्ध्यम् नि॑रव॒दाय॑ निरव॒दाया॑ मे॒द्ध्य म॑मे॒द्ध्यम् नि॑रव॒दाय॑ । \newline
18. नि॒र॒व॒दाय॒ मेद्ध्य॒म् मेद्ध्य॑म् निरव॒दाय॑ निरव॒दाय॒ मेद्ध्य᳚म् । \newline
19. नि॒र॒व॒दायेति॑ निः - अ॒व॒दाय॑ । \newline
20. मेद्ध्य॑म् कृ॒त्वा कृ॒त्वा मेद्ध्य॒म् मेद्ध्य॑म् कृ॒त्वा । \newline
21. कृ॒त्वा ऽऽह॑रति हर॒त्या कृ॒त्वा कृ॒त्वा ऽऽह॑रति । \newline
22. आ ह॑रति हर॒त्या ह॑रति । \newline
23. ह॒र॒त्येक॑विꣳशति॒ रेक॑विꣳशतिर्. हरति हर॒त्येक॑विꣳशतिः । \newline
24. एक॑विꣳशतिर् भवन्ति भव॒न्त्येक॑विꣳशति॒ रेक॑विꣳशतिर् भवन्ति । \newline
25. एक॑विꣳशति॒रित्येक॑ - विꣳ॒॒श॒तिः॒ । \newline
26. भ॒व॒ न्त्ये॒क॒विꣳ॒॒श ए॑कविꣳ॒॒शो भ॑वन्ति भव न्त्येकविꣳ॒॒शः । \newline
27. ए॒क॒विꣳ॒॒शो वै वा ए॑कविꣳ॒॒श ए॑कविꣳ॒॒शो वै । \newline
28. ए॒क॒विꣳ॒॒श इत्ये॑क - विꣳ॒॒शः । \newline
29. वै पुरु॑षः॒ पुरु॑षो॒ वै वै पुरु॑षः । \newline
30. पुरु॑षः॒ पुरु॑षस्य॒ पुरु॑षस्य॒ पुरु॑षः॒ पुरु॑षः॒ पुरु॑षस्य । \newline
31. पुरु॑ष॒स्याप्त्या॒ आप्त्यै॒ पुरु॑षस्य॒ पुरु॑ष॒स्याप्त्यै᳚ । \newline
32. आप्त्यै॒ व्यृ॑द्धं॒ ॅव्यृ॑द्ध॒ माप्त्या॒ आप्त्यै॒ व्यृ॑द्धम् । \newline
33. व्यृ॑द्धं॒ ॅवै वै व्यृ॑द्धं॒ ॅव्यृ॑द्धं॒ ॅवै । \newline
34. व्यृ॑द्ध॒मिति॒ वि - ऋ॒द्ध॒म् । \newline
35. वा ए॒त दे॒तद् वै वा ए॒तत् । \newline
36. ए॒तत् प्रा॒णैः प्रा॒णै रे॒त दे॒तत् प्रा॒णैः । \newline
37. प्रा॒णै र॑मे॒द्ध्य म॑मे॒द्ध्यम् प्रा॒णैः प्रा॒णै र॑मे॒द्ध्यम् । \newline
38. प्रा॒णैरिति॑ प्र - अ॒नैः । \newline
39. अ॒मे॒द्ध्यं ॅयद् यद॑मे॒द्ध्य म॑मे॒द्ध्यं ॅयत् । \newline
40. यत् पु॑रुषशी॒र्॒.षम् पु॑रुषशी॒र्॒.षं ॅयद् यत् पु॑रुषशी॒र्॒.षम् । \newline
41. पु॒रु॒ष॒शी॒र्॒.षꣳ स॑प्त॒धा स॑प्त॒धा पु॑रुषशी॒र्॒.षम् पु॑रुषशी॒र्॒.षꣳ स॑प्त॒धा । \newline
42. पु॒रु॒ष॒शी॒र्॒.षमिति॑ पुरुष - शी॒र्॒.षम् । \newline
43. स॒प्त॒धा वितृ॑ण्णां॒ ॅवितृ॑ण्णाꣳ सप्त॒धा स॑प्त॒धा वितृ॑ण्णाम् । \newline
44. स॒प्त॒धेति॑ सप्त - धा । \newline
45. वितृ॑ण्णां ॅवल्मीकव॒पां ॅव॑ल्मीकव॒पां ॅवितृ॑ण्णां॒ ॅवितृ॑ण्णां ॅवल्मीकव॒पाम् । \newline
46. वितृ॑ण्णा॒मिति॒ वि - तृ॒ण्णा॒म् । \newline
47. व॒ल्मी॒क॒व॒पाम् प्रति॒ प्रति॑ वल्मीकव॒पां ॅव॑ल्मीकव॒पाम् प्रति॑ । \newline
48. व॒ल्मी॒क॒व॒पामिति॑ वल्मीक - व॒पाम् । \newline
49. प्रति॒ नि नि प्रति॒ प्रति॒ नि । \newline
50. नि द॑धाति दधाति॒ नि नि द॑धाति । \newline
51. द॒धा॒ति॒ स॒प्त स॒प्त द॑धाति दधाति स॒प्त । \newline
52. स॒प्त वै वै स॒प्त स॒प्त वै । \newline
53. वै शी॑र्.ष॒ण्याः᳚ शीर्.ष॒ण्या॑ वै वै शी॑र्.ष॒ण्याः᳚ । \newline
54. शी॒र्॒.ष॒ण्याः᳚ प्रा॒णः प्रा॒णः शी॑र.ष॒ण्याः᳚ शीर.ष॒ण्याः᳚ प्रा॒णः । \newline
55. प्रा॒णः प्रा॒णैः प्रा॒णैः प्रा॒णः प्रा॒णः प्रा॒णैः । \newline
56. प्रा॒णा इति॑ प्र - अ॒नाः । \newline
57. प्रा॒णै रे॒वैव प्रा॒णैः प्रा॒णै रे॒व । \newline
58. प्रा॒णैरिति॑ प्र - अ॒नैः । \newline
59. ए॒वैन॑ देन दे॒वै वैन॑त् । \newline
60. ए॒न॒थ् सꣳ स मे॑न देन॒थ् सम् । \newline
61. स म॑र्द्धय त्यर्द्धयति॒ सꣳ स म॑र्द्धयति । \newline
62. अ॒र्द्ध॒य॒ति॒ मे॒द्ध्य॒त्वाय॑ मेद्ध्य॒त्वाया᳚ र्द्धय त्यर्द्धयति मेद्ध्य॒त्वाय॑ । \newline
63. मे॒द्ध्य॒त्वाय॒ याव॑न्तो॒ याव॑न्तो मेद्ध्य॒त्वाय॑ मेद्ध्य॒त्वाय॒ याव॑न्तः । \newline
64. मे॒द्ध्य॒त्वायेति॑ मेद्ध्य - त्वाय॑ । \newline
65. याव॑न्तो॒ वै वै याव॑न्तो॒ याव॑न्तो॒ वै । \newline

\textbf{Ghana Paata } \newline

1. एक॑विꣳशत्या॒ माषै॒र् माषै॒ रेक॑विꣳश॒ त्यैक॑विꣳशत्या॒ माषैः᳚ पुरुषशी॒र्॒.षम् पु॑रुषशी॒र्॒.षम् माषै॒ रेक॑विꣳश॒ त्यैक॑विꣳशत्या॒ माषैः᳚ पुरुषशी॒र्॒.षम् । \newline
2. एक॑विꣳश॒त्येत्येक॑ - विꣳ॒॒श॒त्या॒ । \newline
3. माषैः᳚ पुरुषशी॒र्॒.षम् पु॑रुषशी॒र्॒.षम् माषै॒र् माषैः᳚ पुरुषशी॒र्॒.ष मच्छाच्छ॑ पुरुषशी॒र्॒.षम् माषै॒र् माषैः᳚ पुरुषशी॒र्॒.ष मच्छ॑ । \newline
4. पु॒रु॒ष॒शी॒र्॒.ष मच्छाच्छ॑ पुरुषशी॒र्॒.षम् पु॑रुषशी॒र्॒.ष मच्छै᳚ त्ये॒त्यच्छ॑ पुरुषशी॒र्॒.षम् पु॑रुषशी॒र्॒.ष मच्छै॑ति । \newline
5. पु॒रु॒ष॒शी॒र्॒.षमिति॑ पुरुष - शी॒र्॒.षम् । \newline
6. अच्छै᳚ त्ये॒त्यच्छाच् छै᳚त्यमे॒द्ध्या अ॑मे॒द्ध्या ए॒त्यच्छा च्छै᳚त्यमे॒द्ध्याः । \newline
7. ए॒त्य॒मे॒द्ध्या अ॑मे॒द्ध्या ए᳚त्ये त्यमे॒द्ध्या वै वा अ॑मे॒द्ध्या ए᳚त्ये त्यमे॒द्ध्या वै । \newline
8. अ॒मे॒द्ध्या वै वा अ॑मे॒द्ध्या अ॑मे॒द्ध्या वै माषा॒ माषा॒ वा अ॑मे॒द्ध्या अ॑मे॒द्ध्या वै माषाः᳚ । \newline
9. वै माषा॒ माषा॒ वै वै माषा॑ अमे॒द्ध्य म॑मे॒द्ध्यम् माषा॒ वै वै माषा॑ अमे॒द्ध्यम् । \newline
10. माषा॑ अमे॒द्ध्य म॑मे॒द्ध्यम् माषा॒ माषा॑ अमे॒द्ध्यम् पु॑रुषशी॒र्॒.षम् पु॑रुषशी॒र्॒.ष म॑मे॒द्ध्यम् माषा॒ माषा॑ अमे॒द्ध्यम् पु॑रुषशी॒र्॒.षम् । \newline
11. अ॒मे॒द्ध्यम् पु॑रुषशी॒र्॒.षम् पु॑रुषशी॒र्॒.ष म॑मे॒द्ध्य म॑मे॒द्ध्यम् पु॑रुषशी॒र्॒.ष म॑मे॒द्ध्यै र॑मे॒द्ध्यैः पु॑रुषशी॒र्॒.ष म॑मे॒द्ध्य म॑मे॒द्ध्यम् पु॑रुषशी॒र्॒.ष म॑मे॒द्ध्यैः । \newline
12. पु॒रु॒ष॒शी॒र्॒.ष म॑मे॒द्ध्यै र॑मे॒द्ध्यैः पु॑रुषशी॒र्॒.षम् पु॑रुषशी॒र्॒.ष म॑मे॒द्ध्यै रे॒वैवामे॒द्ध्यैः पु॑रुषशी॒र्॒.षम् पु॑रुषशी॒र्॒.ष म॑मे॒द्ध्यैरे॒व । \newline
13. पु॒रु॒ष॒शी॒र्॒.षमिति॑ पुरुष - शी॒र्॒.॒षम् । \newline
14. अ॒मे॒द्ध्यै रे॒वैवा मे॒द्ध्यै र॑मे॒द्ध्यै रे॒वास्या᳚ स्यै॒वा मे॒द्ध्यै र॑मे॒द्ध्यै रे॒वास्य॑ । \newline
15. ए॒वास्या᳚ स्यै॒वै वास्या॑ मे॒द्ध्य म॑मे॒द्ध्य म॑स्यै॒वैवास्या॑ मे॒द्ध्यम् । \newline
16. अ॒स्या॒ मे॒द्ध्य म॑मे॒द्ध्य म॑स्यास्या मे॒द्ध्यम् नि॑रव॒दाय॑ निरव॒दाया॑ मे॒द्ध्य म॑स्यास्या मे॒द्ध्यम् नि॑रव॒दाय॑ । \newline
17. अ॒मे॒द्ध्यम् नि॑रव॒दाय॑ निरव॒दाया॑ मे॒द्ध्य म॑मे॒द्ध्यम् नि॑रव॒दाय॒ मेद्ध्य॒म् मेद्ध्य॑म् निरव॒दाया॑ मे॒द्ध्य म॑मे॒द्ध्यम् नि॑रव॒दाय॒ मेद्ध्य᳚म् । \newline
18. नि॒र॒व॒दाय॒ मेद्ध्य॒म् मेद्ध्य॑म् निरव॒दाय॑ निरव॒दाय॒ मेद्ध्य॑म् कृ॒त्वा कृ॒त्वा मेद्ध्य॑म् निरव॒दाय॑ निरव॒दाय॒ मेद्ध्य॑म् कृ॒त्वा । \newline
19. नि॒र॒व॒दायेति॑ निः - अ॒व॒दाय॑ । \newline
20. मेद्ध्य॑म् कृ॒त्वा कृ॒त्वा मेद्ध्य॒म् मेद्ध्य॑म् कृ॒त्वा ऽऽह॑रति हर॒त्या कृ॒त्वा मेद्ध्य॒म् मेद्ध्य॑म् कृ॒त्वा ऽऽह॑रति । \newline
21. कृ॒त्वा ऽऽह॑रति हर॒त्या कृ॒त्वा कृ॒त्वा ऽऽह॑र॒ त्येक॑विꣳशति॒ रेक॑विꣳशतिर्. हर॒त्या कृ॒त्वा कृ॒त्वा ऽऽह॑र॒ त्येक॑विꣳशतिः । \newline
22. आ ह॑रति हर॒त्या ह॑र॒ त्येक॑विꣳशति॒ रेक॑विꣳशतिर्. हर॒त्या ह॑र॒ त्येक॑विꣳशतिः । \newline
23. ह॒र॒त्येक॑विꣳशति॒ रेक॑विꣳशतिर्. हरति हर॒त्येक॑विꣳशतिर् भवन्ति भव॒ न्त्येक॑विꣳशतिर्. हरति हर॒त्येक॑विꣳशतिर् भवन्ति । \newline
24. एक॑विꣳशतिर् भवन्ति भव॒ न्त्येक॑विꣳशति॒ रेक॑विꣳशतिर् भव न्त्येकविꣳ॒॒श ए॑कविꣳ॒॒शो भ॑व॒ न्त्येक॑विꣳशति॒ रेक॑विꣳशतिर् भव न्त्येकविꣳ॒॒शः । \newline
25. एक॑विꣳशति॒रित्येक॑ - विꣳ॒॒श॒तिः॒ । \newline
26. भ॒व॒ न्त्ये॒क॒विꣳ॒॒श ए॑कविꣳ॒॒शो भ॑वन्ति भव न्त्येकविꣳ॒॒शो वै वा ए॑कविꣳ॒॒शो भ॑वन्ति भव न्त्येकविꣳ॒॒शो वै । \newline
27. ए॒क॒विꣳ॒॒शो वै वा ए॑कविꣳ॒॒श ए॑कविꣳ॒॒शो वै पुरु॑षः॒ पुरु॑षो॒ वा ए॑कविꣳ॒॒श ए॑कविꣳ॒॒शो वै पुरु॑षः । \newline
28. ए॒क॒विꣳ॒॒श इत्ये॑क - विꣳ॒॒शः । \newline
29. वै पुरु॑षः॒ पुरु॑षो॒ वै वै पुरु॑षः॒ पुरु॑षस्य॒ पुरु॑षस्य॒ पुरु॑षो॒ वै वै पुरु॑षः॒ पुरु॑षस्य । \newline
30. पुरु॑षः॒ पुरु॑षस्य॒ पुरु॑षस्य॒ पुरु॑षः॒ पुरु॑षः॒ पुरु॑ष॒स्या प्त्या॒ आप्त्यै॒ पुरु॑षस्य॒ पुरु॑षः॒ पुरु॑षः॒ पुरु॑ष॒स्या प्त्यै᳚ । \newline
31. पुरु॑ष॒स्याप्त्या॒ आप्त्यै॒ पुरु॑षस्य॒ पुरु॑ष॒ स्याप्त्यै॒ व्यृ॑द्धं॒ ॅव्यृ॑द्ध॒ माप्त्यै॒ पुरु॑षस्य॒ पुरु॑ष॒स्याप्त्यै॒ व्यृ॑द्धम् । \newline
32. आप्त्यै॒ व्यृ॑द्धं॒ ॅव्यृ॑द्ध॒ माप्त्या॒ आप्त्यै॒ व्यृ॑द्धं॒ ॅवै वै व्यृ॑द्ध॒ माप्त्या॒ आप्त्यै॒ व्यृ॑द्धं॒ ॅवै । \newline
33. व्यृ॑द्धं॒ ॅवै वै व्यृ॑द्धं॒ ॅव्यृ॑द्धं॒ ॅवा ए॒त दे॒तद् वै व्यृ॑द्धं॒ ॅव्यृ॑द्धं॒ ॅवा ए॒तत् । \newline
34. व्यृ॑द्ध॒मिति॒ वि - ऋ॒द्ध॒म् । \newline
35. वा ए॒त दे॒तद् वै वा ए॒तत् प्रा॒णैः प्रा॒णै रे॒तद् वै वा ए॒तत् प्रा॒णैः । \newline
36. ए॒तत् प्रा॒णैः प्रा॒णै रे॒त दे॒तत् प्रा॒णै र॑मे॒द्ध्य म॑मे॒द्ध्यम् प्रा॒णै रे॒त दे॒तत् प्रा॒णै र॑मे॒द्ध्यम् । \newline
37. प्रा॒णै र॑मे॒द्ध्य म॑मे॒द्ध्यम् प्रा॒णैः प्रा॒णै र॑मे॒द्ध्यं ॅयद् यद॑मे॒द्ध्यम् प्रा॒णैः प्रा॒णै र॑मे॒द्ध्यं ॅयत् । \newline
38. प्रा॒णैरिति॑ प्र - अ॒नैः । \newline
39. अ॒मे॒द्ध्यं ॅयद् यद॑मे॒द्ध्य म॑मे॒द्ध्यं ॅयत् पु॑रुषशी॒र्॒.षम् पु॑रुषशी॒र्॒.षं ॅयद॑मे॒द्ध्य म॑मे॒द्ध्यं ॅयत् पु॑रुषशी॒र्॒.षम् । \newline
40. यत् पु॑रुषशी॒र्॒.षम् पु॑रुषशी॒र्॒.षं ॅयद् यत् पु॑रुषशी॒र्॒.षꣳ स॑प्त॒धा स॑प्त॒धा पु॑रुषशी॒र्॒.षं ॅयद् यत् पु॑रुषशी॒र्॒.षꣳ स॑प्त॒धा । \newline
41. पु॒रु॒ष॒शी॒र्॒.षꣳ स॑प्त॒धा स॑प्त॒धा पु॑रुषशी॒र्॒.षम् पु॑रुषशी॒र्॒.षꣳ स॑प्त॒धा वितृ॑ण्णां॒ ॅवितृ॑ण्णाꣳ सप्त॒धा पु॑रुषशीर॒.षम् पु॑रुषशी॒र्॒.षꣳ स॑प्त॒धा वितृ॑ण्णाम् । \newline
42. पु॒रु॒ष॒शी॒र्॒.षमिति॑ पुरुष - शी॒र्॒.षम् । \newline
43. स॒प्त॒धा वितृ॑ण्णां॒ ॅवितृ॑ण्णाꣳ सप्त॒धा स॑प्त॒धा वितृ॑ण्णां ॅवल्मीकव॒पां ॅव॑ल्मीकव॒पां ॅवितृ॑ण्णाꣳ सप्त॒धा स॑प्त॒धा वितृ॑ण्णां ॅवल्मीकव॒पाम् । \newline
44. स॒प्त॒धेति॑ सप्त - धा । \newline
45. वितृ॑ण्णां ॅवल्मीकव॒पां ॅव॑ल्मीकव॒पां ॅवितृ॑ण्णां॒ ॅवितृ॑ण्णां ॅवल्मीकव॒पाम् प्रति॒ प्रति॑ वल्मीकव॒पां ॅवितृ॑ण्णां॒ ॅवितृ॑ण्णां ॅवल्मीकव॒पाम् प्रति॑ । \newline
46. वितृ॑ण्णा॒मिति॒ वि - तृ॒ण्णा॒म् । \newline
47. व॒ल्मी॒क॒व॒पाम् प्रति॒ प्रति॑ वल्मीकव॒पां ॅव॑ल्मीकव॒पाम् प्रति॒ नि नि प्रति॑ वल्मीकव॒पां ॅव॑ल्मीकव॒पाम् प्रति॒ नि । \newline
48. व॒ल्मी॒क॒व॒पामिति॑ वल्मीक - व॒पाम् । \newline
49. प्रति॒ नि नि प्रति॒ प्रति॒ नि द॑धाति दधाति॒ नि प्रति॒ प्रति॒ नि द॑धाति । \newline
50. नि द॑धाति दधाति॒ नि नि द॑धाति स॒प्त स॒प्त द॑धाति॒ नि नि द॑धाति स॒प्त । \newline
51. द॒धा॒ति॒ स॒प्त स॒प्त द॑धाति दधाति स॒प्त वै वै स॒प्त द॑धाति दधाति स॒प्त वै । \newline
52. स॒प्त वै वै स॒प्त स॒प्त वै शी॑र्.ष॒ण्याः᳚ शीर्.ष॒ण्या॑ वै स॒प्त स॒प्त वै शी॑र्.ष॒ण्याः᳚ । \newline
53. वै शी॑र्.ष॒ण्याः᳚ शीर्.ष॒ण्या॑ वै वै शी॑र्.ष॒ण्याः᳚ प्रा॒णः प्रा॒णः शी॑र्.ष॒ण्या॑ वै वै शी॑र्.ष॒ण्याः᳚ प्रा॒णः । \newline
54. शी॒र्॒.ष॒ण्याः᳚ प्रा॒णः प्रा॒णः शी॑र्.ष॒ण्याः᳚ शीर्.ष॒ण्याः᳚ प्रा॒णः प्रा॒णैः प्रा॒णैः प्रा॒णः शी॑र्.ष॒ण्याः᳚ शीर्.ष॒ण्याः᳚ प्रा॒णः प्रा॒णैः । \newline
55. प्रा॒णः प्रा॒णैः प्रा॒णैः प्रा॒णः प्रा॒णः प्रा॒णै रे॒वैव प्रा॒णैः प्रा॒णः प्रा॒णः प्रा॒णै रे॒व । \newline
56. प्रा॒णा इति॑ प्र - अ॒नाः । \newline
57. प्रा॒णै रे॒वैव प्रा॒णैः प्रा॒णै रे॒वैन॑ देन दे॒व प्रा॒णैः प्रा॒णै रे॒वैन॑त् । \newline
58. प्रा॒णैरिति॑ प्र - अ॒नैः । \newline
59. ए॒वैन॑ देन दे॒वैवैन॒थ् सꣳ स मे॑न दे॒वैवैन॒थ् सम् । \newline
60. ए॒न॒थ् सꣳ स मे॑नदेन॒थ् स म॑र्द्धय त्यर्द्धयति॒ स मे॑नदेन॒थ् स म॑र्द्धयति । \newline
61. स म॑र्द्धय त्यर्द्धयति॒ सꣳ स म॑र्द्धयति मेद्ध्य॒त्वाय॑ मेद्ध्य॒त्वाया᳚ र्द्धयति॒ सꣳ स म॑र्द्धयति मेद्ध्य॒त्वाय॑ । \newline
62. अ॒र्द्ध॒य॒ति॒ मे॒द्ध्य॒त्वाय॑ मेद्ध्य॒त्वाया᳚ र्द्धयत्यर्द्धयति मेद्ध्य॒त्वाय॒ याव॑न्तो॒ याव॑न्तो मेद्ध्य॒त्वाया᳚ र्द्धय त्यर्द्धयति मेद्ध्य॒त्वाय॒ याव॑न्तः । \newline
63. मे॒द्ध्य॒त्वाय॒ याव॑न्तो॒ याव॑न्तो मेद्ध्य॒त्वाय॑ मेद्ध्य॒त्वाय॒ याव॑न्तो॒ वै वै याव॑न्तो मेद्ध्य॒त्वाय॑ मेद्ध्य॒त्वाय॒ याव॑न्तो॒ वै । \newline
64. मे॒द्ध्य॒त्वायेति॑ मेद्ध्य - त्वाय॑ । \newline
65. याव॑न्तो॒ वै वै याव॑न्तो॒ याव॑न्तो॒ वै मृ॒त्युब॑न्धवो मृ॒त्युब॑न्धवो॒ वै याव॑न्तो॒ याव॑न्तो॒ वै मृ॒त्युब॑न्धवः । \newline
\pagebreak
\markright{ TS 5.1.8.2  \hfill https://www.vedavms.in \hfill}

\section{ TS 5.1.8.2 }

\textbf{TS 5.1.8.2 } \newline
\textbf{Samhita Paata} \newline

वै मृ॒त्युब॑न्धव॒स्तेषां᳚ ॅय॒म आधि॑पत्यं॒ परी॑याय यमगा॒थाभिः॒ परि॑गायति य॒मादे॒वैन॑द्-वृङ्क्ते ति॒सृभिः॒ परि॑गायति॒ त्रय॑ इ॒मे लो॒का ए॒भ्य ए॒वैन॑ल्लो॒केभ्यो॑ वृङ्क्ते॒ तस्मा॒द्-गाय॑ते॒ न देयं॒ गाथा॒ हि तद्-वृ॒ङ्क्ते᳚ ऽग्निभ्यः॑ प॒शूना ल॑भते॒ कामा॒ वा अ॒ग्नयः॒ कामा॑ने॒वाव॑ रुन्धे॒ यत् प॒शून् नाऽऽ*लभे॒ताऽन॑वरुद्धा अस्य - [  ] \newline

\textbf{Pada Paata} \newline

वै । मृ॒त्युब॑न्धव॒ इति॑ मृ॒त्यु - ब॒न्ध॒वः॒ । तेषा᳚म् । य॒मः । आधि॑पत्य॒मित्याधि॑ - प॒त्य॒म् । परीति॑ । इ॒या॒य॒ । य॒म॒गा॒थाभि॒रिति॑ यम - गा॒थाभिः॑ । परीति॑ । गा॒य॒ति॒ । य॒मात् । ए॒व । ए॒न॒त् । वृ॒ङ्क्ते॒ । ति॒सृभि॒रिति॑ ति॒सृ - भिः॒ । परीति॑ । गा॒य॒ति॒ । त्रयः॑ । इ॒मे । लो॒काः । ए॒भ्यः । ए॒व । ए॒न॒त् । लो॒केभ्यः॑ । वृ॒ङ्क्ते॒ । तस्मा᳚त् । गाय॑ते । न । देय᳚म् । गाथा᳚ । हि । तत् । वृ॒ङ्क्ते॒ । अ॒ग्निभ्य॒ इत्य॒ग्नि-भ्यः॒ । प॒शून् । एति॑ । ल॒भ॒ते॒ । कामाः᳚ । वै । अ॒ग्नयः॑ । कामान्॑ । ए॒व । अवेति॑ । रु॒न्धे॒ । यत् । प॒शून् । न । आ॒लभे॒तेत्या᳚ - लभे॑त । अन॑वरुद्धा॒ इत्यन॑व - रु॒द्धाः॒ । अ॒स्य॒ ।  \newline


\textbf{Krama Paata} \newline

वै मृ॒त्युब॑न्धवः । मृ॒त्युब॑न्धव॒स्तेषा᳚म् । मृ॒त्युब॑न्धव॒ इति॑ मृ॒त्यु - ब॒न्ध॒वः॒ । तेषा᳚म् ॅय॒मः । य॒म आधि॑पत्यम् । आधि॑पत्य॒म् परि॑ । आधि॑पत्य॒मित्याधि॑ - प॒त्य॒म् । परी॑याय । इ॒या॒य॒ य॒म॒गा॒थाभिः॑ । य॒म॒गा॒थाभिः॒ परि॑ । य॒म॒गा॒थाभि॒रिति॑ यम - गा॒थाभिः॑ । परि॑ गायति । गा॒य॒ति॒ य॒मात् । य॒मादे॒व । ए॒वैन॑त् । ए॒न॒द् वृ॒ङ्क्ते॒ । वृ॒ङ्क्ते॒ ति॒सृभिः॑ । ति॒सृभिः॒ परि॑ । ति॒सृभि॒रिति॑ ति॒सृ - भिः॒ । परि॑ गायति । गा॒य॒ति॒ त्रयः॑ । त्रय॑ इ॒मे । इ॒मे लो॒काः । लो॒का ए॒भ्यः । ए॒भ्य ए॒व । ए॒वैन॑त् । ए॒न॒ल्लो॒केभ्यः॑ । लो॒केभ्यो॑ वृङ्क्ते । वृ॒ङ्क्ते॒ तस्मा᳚त् । तस्मा॒द् गाय॑ते । गाय॑ते॒ न । न देय᳚म् । देय॒म् गाथा᳚ । गाथा॒ हि । हि तत् । तद् वृ॒ङ्क्ते । वृ॒ङ्क्ते᳚ऽग्निभ्यः॑ । अ॒ग्निभ्यः॑ प॒शून् । अ॒ग्निभ्य॒ इत्य॒ग्नि - भ्यः॒ । प॒शूना । आ ल॑भते । ल॒भ॒ते॒ कामाः᳚ । कामा॒ वै । वा अ॒ग्नयः॑ । अ॒ग्नयः॒ कामान्॑ । कामा॑ने॒व । ए॒वाव॑ । अव॑ रुन्धे । रु॒न्धे॒ यत् । यत् प॒शून् । प॒शून् न । नालभे॑त । आ॒लभे॒तान॑वरुद्धाः । आ॒लभे॒तेत्या᳚ - लभे॑त । अन॑वरुद्धा अस्य । अन॑वरुद्धा॒ इत्यन॑व - रु॒द्धाः॒ । अ॒स्य॒ प॒शवः॑ \newline

\textbf{Jatai Paata} \newline

1. वै मृ॒त्युब॑न्धवो मृ॒त्युब॑न्धवो॒ वै वै मृ॒त्युब॑न्धवः । \newline
2. मृ॒त्युब॑न्धव॒ स्तेषा॒म् तेषा᳚म् मृ॒त्युब॑न्धवो मृ॒त्युब॑न्धव॒ स्तेषा᳚म् । \newline
3. मृ॒त्युब॑न्धव॒ इति॑ मृ॒त्यु - ब॒न्ध॒वः॒ । \newline
4. तेषां᳚ ॅय॒मो य॒म स्तेषा॒म् तेषां᳚ ॅय॒मः । \newline
5. य॒म आधि॑पत्य॒ माधि॑पत्यं ॅय॒मो य॒म आधि॑पत्यम् । \newline
6. आधि॑पत्य॒म् परि॒ पर्याधि॑पत्य॒ माधि॑पत्य॒म् परि॑ । \newline
7. आधि॑पत्य॒मित्याधि॑ - प॒त्य॒म् । \newline
8. परी॑याये याय॒ परि॒ परी॑याय । \newline
9. इ॒या॒य॒ य॒म॒गा॒थाभि॑र् यमगा॒थाभि॑ रियाये याय यमगा॒थाभिः॑ । \newline
10. य॒म॒गा॒थाभिः॒ परि॒ परि॑ यमगा॒थाभि॑र् यमगा॒थाभिः॒ परि॑ । \newline
11. य॒म॒गा॒थाभि॒रिति॑ यम - गा॒थाभिः॑ । \newline
12. परि॑ गायति गायति॒ परि॒ परि॑ गायति । \newline
13. गा॒य॒ति॒ य॒माद् य॒माद् गा॑यति गायति य॒मात् । \newline
14. य॒मा दे॒वैव य॒माद् य॒मा दे॒व । \newline
15. ए॒वैन॑ देन दे॒वैवैन॑त् । \newline
16. ए॒न॒द् वृ॒ङ्क्ते॒ वृ॒ङ्क्त॒ ए॒न॒ दे॒न॒द् वृ॒ङ्क्ते॒ । \newline
17. वृ॒ङ्क्ते॒ ति॒सृभि॑ स्ति॒सृभि॑र् वृङ्क्ते वृङ्क्ते ति॒सृभिः॑ । \newline
18. ति॒सृभिः॒ परि॒ परि॑ ति॒सृभि॑ स्ति॒सृभिः॒ परि॑ । \newline
19. ति॒सृभि॒रिति॑ ति॒सृ - भिः॒ । \newline
20. परि॑ गायति गायति॒ परि॒ परि॑ गायति । \newline
21. गा॒य॒ति॒ त्रय॒ स्त्रयो॑ गायति गायति॒ त्रयः॑ । \newline
22. त्रय॑ इ॒म इ॒मे त्रय॒ स्त्रय॑ इ॒मे । \newline
23. इ॒मे लो॒का लो॒का इ॒म इ॒मे लो॒काः । \newline
24. लो॒का ए॒भ्य ए॒भ्यो लो॒का लो॒का ए॒भ्यः । \newline
25. ए॒भ्य ए॒वैवैभ्य ए॒भ्य ए॒व । \newline
26. ए॒वैन॑ देन दे॒वै वैन॑त् । \newline
27. ए॒न॒ ल्लो॒केभ्यो॑ लो॒केभ्य॑ एन देन ल्लो॒केभ्यः॑ । \newline
28. लो॒केभ्यो॑ वृङ्क्ते वृङ्क्ते लो॒केभ्यो॑ लो॒केभ्यो॑ वृङ्क्ते । \newline
29. वृ॒ङ्क्ते॒ तस्मा॒त् तस्मा᳚द् वृङ्क्ते वृङ्क्ते॒ तस्मा᳚त् । \newline
30. तस्मा॒द् गाय॑ते॒ गाय॑ते॒ तस्मा॒त् तस्मा॒द् गाय॑ते । \newline
31. गाय॑ते॒ न न गाय॑ते॒ गाय॑ते॒ न । \newline
32. न देय॒म् देय॒म् न न देय᳚म् । \newline
33. देय॒म् गाथा॒ गाथा॒ देय॒म् देय॒म् गाथा᳚ । \newline
34. गाथा॒ हि हि गाथा॒ गाथा॒ हि । \newline
35. हि तत् तद्धि हि तत् । \newline
36. तद् वृ॒ङ्क्ते वृ॒ङ्क्ते तत् तद् वृ॒ङ्क्ते । \newline
37. वृ॒ङ्क्ते᳚ ऽग्निभ्यो॒ ऽग्निभ्यो॑ वृ॒ङ्क्ते वृ॒ङ्क्ते᳚ ऽग्निभ्यः॑ । \newline
38. अ॒ग्निभ्यः॑ प॒शून् प॒शू न॒ग्निभ्यो॒ ऽग्निभ्यः॑ प॒शून् । \newline
39. अ॒ग्निभ्य॒ इत्य॒ग्नि - भ्यः॒ । \newline
40. प॒शू ना प॒शून् प॒शू ना । \newline
41. आ ल॑भते लभत॒ आ ल॑भते । \newline
42. ल॒भ॒ते॒ कामाः॒ कामा॑ लभते लभते॒ कामाः᳚ । \newline
43. कामा॒ वै वै कामाः॒ कामा॒ वै । \newline
44. वा अ॒ग्नयो॒ ऽग्नयो॒ वै वा अ॒ग्नयः॑ । \newline
45. अ॒ग्नयः॒ कामा॒न् कामा॑ न॒ग्नयो॒ ऽग्नयः॒ कामान्॑ । \newline
46. कामा॑ ने॒वैव कामा॒न् कामा॑ ने॒व । \newline
47. ए॒वावा वै॒वै वाव॑ । \newline
48. अव॑ रुन्धे रु॒न्धे ऽवाव॑ रुन्धे । \newline
49. रु॒न्धे॒ यद् यद् रु॑न्धे रुन्धे॒ यत् । \newline
50. यत् प॒शून् प॒शून्. यद् यत् प॒शून् । \newline
51. प॒शून् न न प॒शून् प॒शून् न । \newline
52. नालभे॑ता॒ लभे॑त॒ न नालभे॑त । \newline
53. आ॒लभे॒ता न॑वरुद्धा॒ अन॑वरुद्धा आ॒लभे॑ता॒ लभे॒ता न॑वरुद्धाः । \newline
54. आ॒लभे॒तेत्या᳚ - लभे॑त । \newline
55. अन॑वरुद्धा अस्या॒स्या न॑वरुद्धा॒ अन॑वरुद्धा अस्य । \newline
56. अन॑वरुद्धा॒ इत्यन॑व - रु॒द्धाः॒ । \newline
57. अ॒स्य॒ प॒शवः॑ प॒शवो᳚ ऽस्यास्य प॒शवः॑ । \newline

\textbf{Ghana Paata } \newline

1. वै मृ॒त्युब॑न्धवो मृ॒त्युब॑न्धवो॒ वै वै मृ॒त्युब॑न्धव॒ स्तेषा॒म् तेषा᳚म् मृ॒त्युब॑न्धवो॒ वै वै मृ॒त्युब॑न्धव॒ स्तेषा᳚म् । \newline
2. मृ॒त्युब॑न्धव॒ स्तेषा॒म् तेषा᳚म् मृ॒त्युब॑न्धवो मृ॒त्युब॑न्धव॒ स्तेषां᳚ ॅय॒मो य॒म स्तेषा᳚म् मृ॒त्युब॑न्धवो मृ॒त्युब॑न्धव॒ स्तेषां᳚ ॅय॒मः । \newline
3. मृ॒त्युब॑न्धव॒ इति॑ मृ॒त्यु - ब॒न्ध॒वः॒ । \newline
4. तेषां᳚ ॅय॒मो य॒म स्तेषा॒म् तेषां᳚ ॅय॒म आधि॑पत्य॒ माधि॑पत्यं ॅय॒म स्तेषा॒म् तेषां᳚ ॅय॒म आधि॑पत्यम् । \newline
5. य॒म आधि॑पत्य॒ माधि॑पत्यं ॅय॒मो य॒म आधि॑पत्य॒म् परि॒ पर्याधि॑पत्यं ॅय॒मो य॒म आधि॑पत्य॒म् परि॑ । \newline
6. आधि॑पत्य॒म् परि॒ पर्याधि॑पत्य॒ माधि॑पत्य॒म् परी॑याये याय॒ पर्याधि॑पत्य॒ माधि॑पत्य॒म् परी॑याय । \newline
7. आधि॑पत्य॒मित्याधि॑ - प॒त्य॒म् । \newline
8. परी॑याये याय॒ परि॒ परी॑याय यमगा॒थाभि॑र् यमगा॒थाभि॑ रियाय॒ परि॒ परी॑याय यमगा॒थाभिः॑ । \newline
9. इ॒या॒य॒ य॒म॒गा॒थाभि॑र् यमगा॒थाभि॑ रियाये याय यमगा॒थाभिः॒ परि॒ परि॑ यमगा॒थाभि॑ रियाये याय यमगा॒थाभिः॒ परि॑ । \newline
10. य॒म॒गा॒थाभिः॒ परि॒ परि॑ यमगा॒थाभि॑र् यमगा॒थाभिः॒ परि॑ गायति गायति॒ परि॑ यमगा॒थाभि॑र् यमगा॒थाभिः॒ परि॑ गायति । \newline
11. य॒म॒गा॒थाभि॒रिति॑ यम - गा॒थाभिः॑ । \newline
12. परि॑ गायति गायति॒ परि॒ परि॑ गायति य॒माद् य॒माद् गा॑यति॒ परि॒ परि॑ गायति य॒मात् । \newline
13. गा॒य॒ति॒ य॒माद् य॒माद् गा॑यति गायति य॒मा दे॒वैव य॒माद् गा॑यति गायति य॒मा दे॒व । \newline
14. य॒मा दे॒वैव य॒माद् य॒मा दे॒वैन॑ देन दे॒व य॒माद् य॒मा दे॒वैन॑त् । \newline
15. ए॒वैन॑ देन दे॒वैवैन॑द् वृङ्क्ते वृङ्क्त एन दे॒वैवैन॑द् वृङ्क्ते । \newline
16. ए॒न॒द् वृ॒ङ्क्ते॒ वृ॒ङ्क्त॒ ए॒न॒ दे॒न॒द् वृ॒ङ्क्ते॒ ति॒सृभि॑ स्ति॒सृभि॑र् वृङ्क्त एन देनद् वृङ्क्ते ति॒सृभिः॑ । \newline
17. वृ॒ङ्क्ते॒ ति॒सृभि॑ स्ति॒सृभि॑र् वृङ्क्ते वृङ्क्ते ति॒सृभिः॒ परि॒ परि॑ ति॒सृभि॑र् वृङ्क्ते वृङ्क्ते ति॒सृभिः॒ परि॑ । \newline
18. ति॒सृभिः॒ परि॒ परि॑ ति॒सृभि॑ स्ति॒सृभिः॒ परि॑ गायति गायति॒ परि॑ ति॒सृभि॑ स्ति॒सृभिः॒ परि॑ गायति । \newline
19. ति॒सृभि॒रिति॑ ति॒सृ - भिः॒ । \newline
20. परि॑ गायति गायति॒ परि॒ परि॑ गायति॒ त्रय॒ स्त्रयो॑ गायति॒ परि॒ परि॑ गायति॒ त्रयः॑ । \newline
21. गा॒य॒ति॒ त्रय॒ स्त्रयो॑ गायति गायति॒ त्रय॑ इ॒म इ॒मे त्रयो॑ गायति गायति॒ त्रय॑ इ॒मे । \newline
22. त्रय॑ इ॒म इ॒मे त्रय॒ स्त्रय॑ इ॒मे लो॒का लो॒का इ॒मे त्रय॒ स्त्रय॑ इ॒मे लो॒काः । \newline
23. इ॒मे लो॒का लो॒का इ॒म इ॒मे लो॒का ए॒भ्य ए॒भ्यो लो॒का इ॒म इ॒मे लो॒का ए॒भ्यः । \newline
24. लो॒का ए॒भ्य ए॒भ्यो लो॒का लो॒का ए॒भ्य ए॒वैवैभ्यो लो॒का लो॒का ए॒भ्य ए॒व । \newline
25. ए॒भ्य ए॒वैवैभ्य ए॒भ्य ए॒वैन॑ देन दे॒वैभ्य ए॒भ्य ए॒वैन॑त् । \newline
26. ए॒वैन॑ देन दे॒वैवैन॑ ल्लो॒केभ्यो॑ लो॒केभ्य॑ एन दे॒वैवैन॑ ल्लो॒केभ्यः॑ । \newline
27. ए॒न॒ ल्लो॒केभ्यो॑ लो॒केभ्य॑ एन देन ल्लो॒केभ्यो॑ वृङ्क्ते वृङ्क्ते लो॒केभ्य॑ एनदेन ल्लो॒केभ्यो॑ वृङ्क्ते । \newline
28. लो॒केभ्यो॑ वृङ्क्ते वृङ्क्ते लो॒केभ्यो॑ लो॒केभ्यो॑ वृङ्क्ते॒ तस्मा॒त् तस्मा᳚द् वृङ्क्ते लो॒केभ्यो॑ लो॒केभ्यो॑ वृङ्क्ते॒ तस्मा᳚त् । \newline
29. वृ॒ङ्क्ते॒ तस्मा॒त् तस्मा᳚द् वृङ्क्ते वृङ्क्ते॒ तस्मा॒द् गाय॑ते॒ गाय॑ते॒ तस्मा᳚द् वृङ्क्ते वृङ्क्ते॒ तस्मा॒द् गाय॑ते । \newline
30. तस्मा॒द् गाय॑ते॒ गाय॑ते॒ तस्मा॒त् तस्मा॒द् गाय॑ते॒ न न गाय॑ते॒ तस्मा॒त् तस्मा॒द् गाय॑ते॒ न । \newline
31. गाय॑ते॒ न न गाय॑ते॒ गाय॑ते॒ न देय॒म् देय॒म् न गाय॑ते॒ गाय॑ते॒ न देय᳚म् । \newline
32. न देय॒म् देय॒म् न न देय॒म् गाथा॒ गाथा॒ देय॒म् न न देय॒म् गाथा᳚ । \newline
33. देय॒म् गाथा॒ गाथा॒ देय॒म् देय॒म् गाथा॒ हि हि गाथा॒ देय॒म् देय॒म् गाथा॒ हि । \newline
34. गाथा॒ हि हि गाथा॒ गाथा॒ हि तत् तद्धि गाथा॒ गाथा॒ हि तत् । \newline
35. हि तत् तद्धि हि तद् वृ॒ङ्क्ते वृ॒ङ्क्ते तद्धि हि तद् वृ॒ङ्क्ते । \newline
36. तद् वृ॒ङ्क्ते वृ॒ङ्क्ते तत् तद् वृ॒ङ्क्ते᳚ ऽग्निभ्यो॒ ऽग्निभ्यो॑ वृ॒ङ्क्ते तत् तद् वृ॒ङ्क्ते᳚ ऽग्निभ्यः॑ । \newline
37. वृ॒ङ्क्ते᳚ ऽग्निभ्यो॒ ऽग्निभ्यो॑ वृ॒ङ्क्ते वृ॒ङ्क्ते᳚ ऽग्निभ्यः॑ प॒शून् प॒शू न॒ग्निभ्यो॑ वृ॒ङ्क्ते वृ॒ङ्क्ते᳚ ऽग्निभ्यः॑ प॒शून् । \newline
38. अ॒ग्निभ्यः॑ प॒शून् प॒शू न॒ग्निभ्यो॒ ऽग्निभ्यः॑ प॒शू ना प॒शू न॒ग्निभ्यो॒ ऽग्निभ्यः॑ प॒शू ना । \newline
39. अ॒ग्निभ्य॒ इत्य॒ग्नि - भ्यः॒ । \newline
40. प॒शू ना प॒शून् प॒शू ना ल॑भते लभत॒ आ प॒शून् प॒शू ना ल॑भते । \newline
41. आ ल॑भते लभत॒ आ ल॑भते॒ कामाः॒ कामा॑ लभत॒ आ ल॑भते॒ कामाः᳚ । \newline
42. ल॒भ॒ते॒ कामाः॒ कामा॑ लभते लभते॒ कामा॒ वै वै कामा॑ लभते लभते॒ कामा॒ वै । \newline
43. कामा॒ वै वै कामाः॒ कामा॒ वा अ॒ग्नयो॒ ऽग्नयो॒ वै कामाः॒ कामा॒ वा अ॒ग्नयः॑ । \newline
44. वा अ॒ग्नयो॒ ऽग्नयो॒ वै वा अ॒ग्नयः॒ कामा॒न् कामा॑ न॒ग्नयो॒ वै वा अ॒ग्नयः॒ कामान्॑ । \newline
45. अ॒ग्नयः॒ कामा॒न् कामा॑ न॒ग्नयो॒ ऽग्नयः॒ कामा॑ ने॒वैव कामा॑ न॒ग्नयो॒ ऽग्नयः॒ कामा॑ ने॒व । \newline
46. कामा॑ ने॒वैव कामा॒न् कामा॑ ने॒वावा वै॒व कामा॒न् कामा॑ ने॒वाव॑ । \newline
47. ए॒वावा वै॒वै वाव॑ रुन्धे रु॒न्धे ऽवै॒वै वाव॑ रुन्धे । \newline
48. अव॑ रुन्धे रु॒न्धे ऽवाव॑ रुन्धे॒ यद् यद् रु॒न्धे ऽवाव॑ रुन्धे॒ यत् । \newline
49. रु॒न्धे॒ यद् यद् रु॑न्धे रुन्धे॒ यत् प॒शून् प॒शून्. यद् रु॑न्धे रुन्धे॒ यत् प॒शून् । \newline
50. यत् प॒शून् प॒शून्. यद् यत् प॒शून् न न प॒शून्. यद् यत् प॒शून् न । \newline
51. प॒शून् न न प॒शून् प॒शून् नालभे॑ता॒ लभे॑त॒ न प॒शून् प॒शून् नालभे॑त । \newline
52. नालभे॑ता॒ लभे॑त॒ न नालभे॒ता न॑वरुद्धा॒ अन॑वरुद्धा आ॒लभे॑त॒ न नालभे॒ता न॑वरुद्धाः । \newline
53. आ॒लभे॒ता न॑वरुद्धा॒ अन॑वरुद्धा आ॒लभे॑ता॒ लभे॒ता न॑वरुद्धा अस्या॒ स्यान॑वरुद्धा आ॒लभे॑ता॒ लभे॒ता न॑वरुद्धा अस्य । \newline
54. आ॒लभे॒तेत्या᳚ - लभे॑त । \newline
55. अन॑वरुद्धा अस्या॒ स्यान॑वरुद्धा॒ अन॑वरुद्धा अस्य प॒शवः॑ प॒शवो॒ ऽस्यान॑वरुद्धा॒ अन॑वरुद्धा अस्य प॒शवः॑ । \newline
56. अन॑वरुद्धा॒ इत्यन॑व - रु॒द्धाः॒ । \newline
57. अ॒स्य॒ प॒शवः॑ प॒शवो᳚ ऽस्यास्य प॒शवः॑ स्युः स्युः प॒शवो᳚ ऽस्यास्य प॒शवः॑ स्युः । \newline
\pagebreak
\markright{ TS 5.1.8.3  \hfill https://www.vedavms.in \hfill}

\section{ TS 5.1.8.3 }

\textbf{TS 5.1.8.3 } \newline
\textbf{Samhita Paata} \newline

प॒शवः॑ स्यु॒र्यत् पर्य॑ग्निकृतानुथ्-सृ॒जेद्-य॑ज्ञ्वेश॒सं कु॑र्या॒द्-यथ् सꣳ॑स्था॒पये᳚द्-या॒तया॑मानि शी॒र्॒.षाणि॑ स्यु॒र्यत् प॒शूना॒लभ॑ते॒ तेनै॒व प॒शूनव॑ रुन्धे॒ यत् पर्य॑ग्निकृतानुथ्-सृ॒जति॑ शी॒र्ष्णा-मया॑तयामत्वाय प्राजाप॒त्येन॒ सꣳ स्था॑पयति य॒ज्ञो वै प्र॒जाप॑तिर्य॒ज्ञ् ए॒व य॒ज्ञ्ं प्रति॑ष्ठापयति प्र॒जाप॑तिः प्र॒जा अ॑सृजत॒ स रि॑रिचा॒नो॑ऽमन्यत॒ स ए॒ता आ॒प्रीर॑पश्य॒त् ताभि॒र्वै स मु॑ख॒त - [  ] \newline

\textbf{Pada Paata} \newline

प॒शवः॑ । स्युः॒ । यत् । पर्य॑ग्निकृता॒निति॒ पर्य॑ग्नि - कृ॒ता॒न् । उ॒थ्सृ॒जेदित्यु॑त् - स॒जेत् । य॒ज्ञ्॒वे॒श॒समिति॑ यज्ञ्-वे॒श॒सम् । कु॒र्या॒त् । यत् । सꣳ॒॒स्था॒पये॒दिति॑ सं - स्था॒पये᳚त् । या॒तया॑मा॒नीति॑ या॒त - या॒मा॒नि॒ । शी॒र्॒.षाणि॑ । स्युः॒ । यत् । प॒शून् । आ॒लभ॑त॒ इत्या᳚ - लभ॑ते । तेन॑ । ए॒व । प॒शून् । अवेति॑ । रु॒न्धे॒ । यत् । पर्य॑ग्निकृता॒निति॒ पर्य॑ग्नि - कृ॒ता॒न् । उ॒थ्सृ॒जतीत्यु॑त्-सृ॒जति॑ । शी॒र्ष्णाम् । अया॑तयामत्वा॒येत्यया॑तयाम - त्वा॒य॒ । रा॒जा॒प॒त्येनेति॑ प्राजा - प॒त्येन॑ । समिति॑ । स्था॒प॒य॒ति॒ । य॒ज्ञ्ः । वै । प्र॒जाप॑ति॒रिति॑ प्र॒जा - प॒तिः॒ । य॒ज्ञे । ए॒व । य॒ज्ञ्म् । प्रतीति॑ । स्था॒प॒य॒ति॒ । प्र॒जाप॑ति॒रिति॑ प्र॒जा - प॒तिः॒ । प्र॒जा इति॑ प्र - जाः । अ॒सृ॒ज॒त॒ । सः । रि॒रि॒चा॒नः । अ॒म॒न्य॒त॒ । सः । ए॒ताः । आ॒प्रीरित्या᳚ - प्रीः । अ॒प॒श्य॒त् । ताभिः॑ । वै । सः । मु॒ख॒तः ।  \newline


\textbf{Krama Paata} \newline

प॒शवः॑ स्युः । स्यु॒र् यत् । यत् पर्य॑ग्निकृतान् । पर्य॑ग्निकृतानुथ्सृ॒जेत् । पर्य॑ग्निकृता॒निति॒ पर्य॑ग्नि - कृ॒ता॒न्॒ । उ॒थ्सृ॒जेद् य॑ज्ञ्वेश॒सम् । उ॒थ्सृ॒जेदित्यु॑त् - सृ॒जेत् । य॒ज्ञ्॒वे॒श॒सम् कु॑र्यात् । य॒ज्ञ्॒वे॒श॒समिति॑ यज्ञ् - वे॒श॒सम् । कु॒र्या॒द् यत् । यथ् सꣳ॑स्था॒पये᳚त् । सꣳ॒॒स्था॒पये᳚द् या॒तया॑मानि । सꣳ॒॒स्था॒पये॒दिति॑ सम् - स्था॒पये᳚त् । या॒तया॑मानि शी॒र्.॒षाणि॑ । या॒तया॑मा॒नीति॑ या॒त - या॒मा॒नि॒ । शी॒र्.॒षाणि॑ स्युः । स्यु॒र् यत् । यत् प॒शून् । प॒शूना॒लभ॑ते । आ॒लभ॑ते॒ तेन॑ । आ॒लभ॑त॒ इत्या᳚ - लभ॑ते । तेनै॒व । ए॒व प॒शून् । प॒शूनव॑ । अव॑ रुन्धे । रु॒न्धे॒ यत् । यत् पर्य॑ग्निकृतान् । पर्य॑ग्निकृता,नुथ्सृ॒जति॑ । पर्य॑ग्निकृता॒निति॒ पर्य॑ग्नि - कृ॒ता॒न्॒ । उ॒थ्सृ॒जति॑ शी॒र्ष्णाम् । उ॒थ्सृ॒जतीत्यु॑त् - सृ॒जति॑ । शी॒र्ष्णामया॑तयामत्वाय । अया॑तयामत्वाय प्राजाप॒त्येन॑ । अया॑तयामत्वा॒येत्यया॑तयाम - त्वा॒य॒ । प्रा॒जा॒प॒त्येन॒ सम् । प्रा॒जा॒प॒त्येनेति॑ प्राजा - प॒त्येन॑ । सꣳ स्था॑पयति । स्था॒प॒य॒ति॒ य॒ज्ञ्ः । य॒ज्ञो वै । वै प्र॒जाप॑तिः । प्र॒जाप॑तिर् य॒ज्ञे । प्र॒जाप॑ति॒रिति॑ प्र॒जा - प॒तिः॒ । य॒ज्ञ् ए॒व । ए॒व य॒ज्ञ्म् । य॒ज्ञ्म् प्रति॑ । प्रति॑ ष्ठापयति । स्था॒प॒य॒ति॒ प्र॒जाप॑तिः । प्र॒जाप॑तिः प्र॒जाः । प्र॒जाप॑ति॒रिति॑ प्र॒जा - प॒तिः॒ । प्र॒जा अ॑सृजत । प्र॒जा इति॑ प्र - जाः । अ॒सृ॒ज॒त॒ सः । स रि॑रिचा॒नः । रि॒रि॒चा॒नो॑ऽमन्यत । अ॒म॒न्य॒त॒ स । स ए॒ताः । ए॒ता आ॒प्रीः । आ॒प्रीर॑पश्यत् । आ॒प्रीरित्या᳚ - प्रीः । अ॒प॒श्य॒त् ताभिः॑ । ताभि॒र् वै । वै सः । स मु॑ख॒तः । मु॒ख॒त आ॒त्मान᳚म् \newline

\textbf{Jatai Paata} \newline

1. प॒शवः॑ स्युः स्युः प॒शवः॑ प॒शवः॑ स्युः । \newline
2. स्यु॒र् यद् यथ् स्युः॑ स्यु॒र् यत् । \newline
3. यत् पर्य॑ग्निकृता॒न् पर्य॑ग्निकृता॒न्॒. यद् यत् पर्य॑ग्निकृतान् । \newline
4. पर्य॑ग्निकृता नुथ्सृ॒जे दु॑थ्सृ॒जेत् पर्य॑ग्निकृता॒न् पर्य॑ग्निकृता नुथ्सृ॒जेत् । \newline
5. पर्य॑ग्निकृता॒निति॒ पर्य॑ग्नि - कृ॒ता॒न् । \newline
6. उ॒थ्सृ॒जेद् य॑ज्ञ्वेश॒सं ॅय॑ज्ञ्वेश॒स मु॑थ्सृ॒जे दु॑थ्सृ॒जेद् य॑ज्ञ्वेश॒सम् । \newline
7. उ॒थ्सृ॒जेदित्यु॑त् - स॒जेत् । \newline
8. य॒ज्ञ्॒वे॒श॒सम् कु॑र्यात् कुर्याद् यज्ञ्वेश॒सं ॅय॑ज्ञ्वेश॒सम् कु॑र्यात् । \newline
9. य॒ज्ञ्॒वे॒श॒समिति॑ यज्ञ् - वे॒श॒सम् । \newline
10. कु॒र्या॒द् यद् यत् कु॑र्यात् कुर्या॒द् यत् । \newline
11. यथ् सꣳ॑स्था॒पये᳚थ् सꣳस्था॒पये॒द् यद् यथ् सꣳ॑स्था॒पये᳚त् । \newline
12. सꣳ॒॒स्था॒पये᳚द् या॒तया॑मानि या॒तया॑मानि सꣳस्था॒पये᳚थ् सꣳस्था॒पये᳚द् या॒तया॑मानि । \newline
13. सꣳ॒॒स्था॒पये॒दिति॑ सं - स्था॒पये᳚त् । \newline
14. या॒तया॑मानि शी॒र्॒.षाणि॑ शी॒र्॒.षाणि॑ या॒तया॑मानि या॒तया॑मानि शी॒र्॒.षाणि॑ । \newline
15. या॒तया॑मा॒नीति॑ या॒त - या॒मा॒नि॒ । \newline
16. शी॒र्॒.षाणि॑ स्युः स्युः शी॒र्॒.षाणि॑ शी॒र्॒.षाणि॑ स्युः । \newline
17. स्यु॒र् यद् यथ् स्युः॑ स्यु॒र् यत् । \newline
18. यत् प॒शून् प॒शून्. यद् यत् प॒शून् । \newline
19. प॒शू ना॒लभ॑त आ॒लभ॑ते प॒शून् प॒शू ना॒लभ॑ते । \newline
20. आ॒लभ॑ते॒ तेन॒ तेना॒ लभ॑त आ॒लभ॑ते॒ तेन॑ । \newline
21. आ॒लभ॑त॒ इत्या᳚ - लभ॑ते । \newline
22. तेनै॒वैव तेन॒ तेनै॒व । \newline
23. ए॒व प॒शून् प॒शू ने॒वैव प॒शून् । \newline
24. प॒शू नवाव॑ प॒शून् प॒शू नव॑ । \newline
25. अव॑ रुन्धे रु॒न्धे ऽवाव॑ रुन्धे । \newline
26. रु॒न्धे॒ यद् यद् रु॑न्धे रुन्धे॒ यत् । \newline
27. यत् पर्य॑ग्निकृता॒न् पर्य॑ग्निकृता॒न्॒. यद् यत् पर्य॑ग्निकृतान् । \newline
28. पर्य॑ग्निकृता नुथ्सृ॒ज त्यु॑थ्सृ॒जति॒ पर्य॑ग्निकृता॒न् पर्य॑ग्निकृता नुथ्सृ॒जति॑ । \newline
29. पर्य॑ग्निकृता॒निति॒ पर्य॑ग्नि - कृ॒ता॒न् । \newline
30. उ॒थ्सृ॒जति॑ शी॒र्ष्णाꣳ शी॒र्ष्णा मु॑थ्सृ॒ज त्यु॑थ्सृ॒जति॑ शी॒र्ष्णाम् । \newline
31. उ॒थ्सृ॒जतीत्यु॑त् - सृ॒जति॑ । \newline
32. शी॒र्ष्णा मया॑तयामत्वा॒या या॑तयामत्वाय शी॒र्ष्णाꣳ शी॒र्ष्णा मया॑तयामत्वाय । \newline
33. अया॑तयामत्वाय प्राजाप॒त्येन॑ प्राजाप॒त्येना या॑तयामत्वा॒या या॑तयामत्वाय प्राजाप॒त्येन॑ । \newline
34. अया॑तयामत्वा॒येत्यया॑तयाम - त्वा॒य॒ । \newline
35. प्रा॒जा॒प॒त्येन॒ सꣳ सम् प्रा॑जाप॒त्येन॑ प्राजाप॒त्येन॒ सम् । \newline
36. प्रा॒जा॒प॒त्येनेति॑ प्राजा - प॒त्येन॑ । \newline
37. सꣳ स्था॑पयति स्थापयति॒ सꣳ सꣳ स्था॑पयति । \newline
38. स्था॒प॒य॒ति॒ य॒ज्ञो य॒ज्ञ्ः स्था॑पयति स्थापयति य॒ज्ञ्ः । \newline
39. य॒ज्ञो वै वै य॒ज्ञो य॒ज्ञो वै । \newline
40. वै प्र॒जाप॑तिः प्र॒जाप॑ति॒र् वै वै प्र॒जाप॑तिः । \newline
41. प्र॒जाप॑तिर् य॒ज्ञे य॒ज्ञे प्र॒जाप॑तिः प्र॒जाप॑तिर् य॒ज्ञे । \newline
42. प्र॒जाप॑ति॒रिति॑ प्र॒जा - प॒तिः॒ । \newline
43. य॒ज्ञ् ए॒वैव य॒ज्ञे य॒ज्ञ् ए॒व । \newline
44. ए॒व य॒ज्ञ्ं ॅय॒ज्ञ् मे॒वैव य॒ज्ञ्म् । \newline
45. य॒ज्ञ्म् प्रति॒ प्रति॑ य॒ज्ञ्ं ॅय॒ज्ञ्म् प्रति॑ । \newline
46. प्रति॑ ष्ठापयति स्थापयति॒ प्रति॒ प्रति॑ ष्ठापयति । \newline
47. स्था॒प॒य॒ति॒ प्र॒जाप॑तिः प्र॒जाप॑तिः स्थापयति स्थापयति प्र॒जाप॑तिः । \newline
48. प्र॒जाप॑तिः प्र॒जाः प्र॒जाः प्र॒जाप॑तिः प्र॒जाप॑तिः प्र॒जाः । \newline
49. प्र॒जाप॑ति॒रिति॑ प्र॒जा - प॒तिः॒ । \newline
50. प्र॒जा अ॑सृजता सृजत प्र॒जाः प्र॒जा अ॑सृजत । \newline
51. प्र॒जा इति॑ प्र - जाः । \newline
52. अ॒सृ॒ज॒त॒ स सो॑ ऽसृजता सृजत॒ सः । \newline
53. स रि॑रिचा॒नो रि॑रिचा॒नः स स रि॑रिचा॒नः । \newline
54. रि॒रि॒चा॒नो॑ ऽमन्यता मन्यत रिरिचा॒नो रि॑रिचा॒नो॑ ऽमन्यत । \newline
55. अ॒म॒न्य॒त॒ स सो॑ ऽमन्यता मन्यत॒ सः । \newline
56. स ए॒ता ए॒ताः स स ए॒ताः । \newline
57. ए॒ता आ॒प्री रा॒प्री रे॒ता ए॒ता आ॒प्रीः । \newline
58. आ॒प्री र॑पश्य दपश्य दा॒प्री रा॒प्री र॑पश्यत् । \newline
59. आ॒प्रीरित्या᳚ - प्रीः । \newline
60. अ॒प॒श्य॒त् ताभि॒ स्ताभि॑ रपश्य दपश्य॒त् ताभिः॑ । \newline
61. ताभि॒र् वै वै ताभि॒ स्ताभि॒र् वै । \newline
62. वै स स वै वै सः । \newline
63. स मु॑ख॒तो मु॑ख॒तः स स मु॑ख॒तः । \newline
64. मु॒ख॒त आ॒त्मान॑ मा॒त्मान॑म् मुख॒तो मु॑ख॒त आ॒त्मान᳚म् । \newline

\textbf{Ghana Paata } \newline

1. प॒शवः॑ स्युः स्युः प॒शवः॑ प॒शवः॑ स्यु॒र् यद् यथ् स्युः॑ प॒शवः॑ प॒शवः॑ स्यु॒र् यत् । \newline
2. स्यु॒र् यद् यथ् स्युः॑ स्यु॒र् यत् पर्य॑ग्निकृता॒न् पर्य॑ग्निकृता॒न्॒. यथ् स्युः॑ स्यु॒र् यत् पर्य॑ग्निकृतान् । \newline
3. यत् पर्य॑ग्निकृता॒न् पर्य॑ग्निकृता॒न्॒. यद् यत् पर्य॑ग्निकृता नुथ्सृ॒जे दु॑थ्सृ॒जेत् पर्य॑ग्निकृता॒न्॒. यद् यत् पर्य॑ग्निकृता नुथ्सृ॒जेत् । \newline
4. पर्य॑ग्निकृता नुथ्सृ॒जे दु॑थ्सृ॒जेत् पर्य॑ग्निकृता॒न् पर्य॑ग्निकृता नुथ्सृ॒जेद् य॑ज्ञ्वेश॒सं ॅय॑ज्ञ्वेश॒स मु॑थ्सृ॒जेत् पर्य॑ग्निकृता॒न् पर्य॑ग्निकृता नुथ्सृ॒जेद् य॑ज्ञ्वेश॒सम् । \newline
5. पर्य॑ग्निकृता॒निति॒ पर्य॑ग्नि - कृ॒ता॒न् । \newline
6. उ॒थ्सृ॒जेद् य॑ज्ञ्वेश॒सं ॅय॑ज्ञ्वेश॒स मु॑थ्सृ॒जे दु॑थ्सृ॒जेद् य॑ज्ञ्वेश॒सम् कु॑र्यात् कुर्याद् यज्ञ्वेश॒स मु॑थ्सृ॒जे दु॑थ्सृ॒जेद् य॑ज्ञ्वेश॒सम् कु॑र्यात् । \newline
7. उ॒थ्सृ॒जेदित्यु॑त् - स॒जेत् । \newline
8. य॒ज्ञ्॒वे॒श॒सम् कु॑र्यात् कुर्याद् यज्ञ्वेश॒सं ॅय॑ज्ञ्वेश॒सम् कु॑र्या॒द् यद् यत् कु॑र्याद् यज्ञ्वेश॒सं ॅय॑ज्ञ्वेश॒सम् कु॑र्या॒द् यत् । \newline
9. य॒ज्ञ्॒वे॒श॒समिति॑ यज्ञ् - वे॒श॒सम् । \newline
10. कु॒र्या॒द् यद् यत् कु॑र्यात् कुर्या॒द् यथ् सꣳ॑स्था॒पये᳚थ् सꣳस्था॒पये॒द् यत् कु॑र्यात् कुर्या॒द् यथ् सꣳ॑स्था॒पये᳚त् । \newline
11. यथ् सꣳ॑स्था॒पये᳚थ् सꣳस्था॒पये॒द् यद् यथ् सꣳ॑स्था॒पये᳚द् या॒तया॑मानि या॒तया॑मानि सꣳस्था॒पये॒द् यद् यथ् सꣳ॑स्था॒पये᳚द् या॒तया॑मानि । \newline
12. सꣳ॒॒स्था॒पये᳚द् या॒तया॑मानि या॒तया॑मानि सꣳस्था॒पये᳚थ् सꣳस्था॒पये᳚द् या॒तया॑मानि शी॒र्॒.षाणि॑ शी॒र्॒.षाणि॑ या॒तया॑मानि सꣳस्था॒पये᳚थ् सꣳस्था॒पये᳚द् या॒तया॑मानि शी॒र्॒.षाणि॑ । \newline
13. सꣳ॒॒स्था॒पये॒दिति॑ सं - स्था॒पये᳚त् । \newline
14. या॒तया॑मानि शी॒र्॒.षाणि॑ शी॒र्॒.षाणि॑ या॒तया॑मानि या॒तया॑मानि शी॒र्॒.षाणि॑ स्युः स्युः शी॒र्॒.षाणि॑ या॒तया॑मानि या॒तया॑मानि शी॒र्॒.षाणि॑ स्युः । \newline
15. या॒तया॑मा॒नीति॑ या॒त - या॒मा॒नि॒ । \newline
16. शी॒र्॒.षाणि॑ स्युः स्युः शी॒र्॒.षाणि॑ शी॒र्॒.षाणि॑ स्यु॒र् यद् यथ् स्युः॑ शी॒र्॒.षाणि॑ शी॒र्॒.षाणि॑ स्यु॒र् यत् । \newline
17. स्यु॒र् यद् यथ् स्युः॑ स्यु॒र् यत् प॒शून् प॒शून्. यथ् स्युः॑ स्यु॒र् यत् प॒शून् । \newline
18. यत् प॒शून् प॒शून्. यद् यत् प॒शू ना॒लभ॑त आ॒लभ॑ते प॒शून्. यद् यत् प॒शू ना॒लभ॑ते । \newline
19. प॒शू ना॒लभ॑त आ॒लभ॑ते प॒शून् प॒शू ना॒लभ॑ते॒ तेन॒ तेना॒लभ॑ते प॒शून् प॒शू ना॒लभ॑ते॒ तेन॑ । \newline
20. आ॒लभ॑ते॒ तेन॒ तेना॒लभ॑त आ॒लभ॑ते॒ तेनै॒वैव तेना॒लभ॑त आ॒लभ॑ते॒ तेनै॒व । \newline
21. आ॒लभ॑त॒ इत्या᳚ - लभ॑ते । \newline
22. तेनै॒वैव तेन॒ तेनै॒व प॒शून् प॒शू ने॒व तेन॒ तेनै॒व प॒शून् । \newline
23. ए॒व प॒शून् प॒शू ने॒वैव प॒शू नवाव॑ प॒शू ने॒वैव प॒शू नव॑ । \newline
24. प॒शू नवाव॑ प॒शून् प॒शू नव॑ रुन्धे रु॒न्धे ऽव॑ प॒शून् प॒शू नव॑ रुन्धे । \newline
25. अव॑ रुन्धे रु॒न्धे ऽवाव॑ रुन्धे॒ यद् यद् रु॒न्धे ऽवाव॑ रुन्धे॒ यत् । \newline
26. रु॒न्धे॒ यद् यद् रु॑न्धे रुन्धे॒ यत् पर्य॑ग्निकृता॒न् पर्य॑ग्निकृता॒न्॒. यद् रु॑न्धे रुन्धे॒ यत् पर्य॑ग्निकृतान् । \newline
27. यत् पर्य॑ग्निकृता॒न् पर्य॑ग्निकृता॒न्॒. यद् यत् पर्य॑ग्निकृता नुथ्सृ॒ज त्यु॑थ्सृ॒जति॒ पर्य॑ग्निकृता॒न्॒. यद् यत् पर्य॑ग्निकृता नुथ्सृ॒जति॑ । \newline
28. पर्य॑ग्निकृता नुथ्सृ॒ज त्यु॑थ्सृ॒जति॒ पर्य॑ग्निकृता॒न् पर्य॑ग्निकृता नुथ्सृ॒जति॑ शी॒र्ष्णाꣳ शी॒र्ष्णा मु॑थ्सृ॒जति॒ पर्य॑ग्निकृता॒न् पर्य॑ग्निकृता नुथ्सृ॒जति॑ शी॒र्ष्णाम् । \newline
29. पर्य॑ग्निकृता॒निति॒ पर्य॑ग्नि - कृ॒ता॒न् । \newline
30. उ॒थ्सृ॒जति॑ शी॒र्ष्णाꣳ शी॒र्ष्णा मु॑थ्सृ॒ज त्यु॑थ्सृ॒जति॑ शी॒र्ष्णा मया॑तयामत्वा॒या या॑तयामत्वाय शी॒र्ष्णा मु॑थ्सृ॒ज त्यु॑थ्सृ॒जति॑ शी॒र्ष्णा मया॑तयामत्वाय । \newline
31. उ॒थ्सृ॒जतीत्यु॑त् - सृ॒जति॑ । \newline
32. शी॒र्ष्णा मया॑तयामत्वा॒या या॑तयामत्वाय शी॒र्ष्णाꣳ शी॒र्ष्णा मया॑तयामत्वाय प्राजाप॒त्येन॑ प्राजाप॒त्येनाया॑ तयामत्वाय शी॒र्ष्णाꣳ शी॒र्ष्णा मया॑तयामत्वाय प्राजाप॒त्येन॑ । \newline
33. अया॑तयामत्वाय प्राजाप॒त्येन॑ प्राजाप॒ त्येनाया॑तयामत्वा॒ याया॑तयामत्वाय प्राजाप॒त्येन॒ सꣳ सम् प्रा॑जाप॒ त्येनाया॑तया मत्वा॒याया॑तयामत्वाय प्राजाप॒त्येन॒ सम् । \newline
34. अया॑तयामत्वा॒येत्यया॑तयाम - त्वा॒य॒ । \newline
35. प्रा॒जा॒प॒त्येन॒ सꣳ सम् प्रा॑जाप॒त्येन॑ प्राजाप॒त्येन॒ सꣳ स्था॑पयति स्थापयति॒ सम् प्रा॑जाप॒त्येन॑ प्राजाप॒त्येन॒ सꣳ स्था॑पयति । \newline
36. प्रा॒जा॒प॒त्येनेति॑ प्राजा - प॒त्येन॑ । \newline
37. सꣳ स्था॑पयति स्थापयति॒ सꣳ सꣳ स्था॑पयति य॒ज्ञो य॒ज्ञ्ः स्था॑पयति॒ सꣳ सꣳ स्था॑पयति य॒ज्ञ्ः । \newline
38. स्था॒प॒य॒ति॒ य॒ज्ञो य॒ज्ञ्ः स्था॑पयति स्थापयति य॒ज्ञो वै वै य॒ज्ञ्ः स्था॑पयति स्थापयति य॒ज्ञो वै । \newline
39. य॒ज्ञो वै वै य॒ज्ञो य॒ज्ञो वै प्र॒जाप॑तिः प्र॒जाप॑ति॒र् वै य॒ज्ञो य॒ज्ञो वै प्र॒जाप॑तिः । \newline
40. वै प्र॒जाप॑तिः प्र॒जाप॑ति॒र् वै वै प्र॒जाप॑तिर् य॒ज्ञे य॒ज्ञे प्र॒जाप॑ति॒र् वै वै प्र॒जाप॑तिर् य॒ज्ञे । \newline
41. प्र॒जाप॑तिर् य॒ज्ञे य॒ज्ञे प्र॒जाप॑तिः प्र॒जाप॑तिर् य॒ज्ञ् ए॒वैव य॒ज्ञे प्र॒जाप॑तिः प्र॒जाप॑तिर् य॒ज्ञ् ए॒व । \newline
42. प्र॒जाप॑ति॒रिति॑ प्र॒जा - प॒तिः॒ । \newline
43. य॒ज्ञ् ए॒वैव य॒ज्ञे य॒ज्ञ् ए॒व य॒ज्ञ्ं ॅय॒ज्ञ् मे॒व य॒ज्ञे य॒ज्ञ् ए॒व य॒ज्ञ्म् । \newline
44. ए॒व य॒ज्ञ्ं ॅय॒ज्ञ् मे॒वैव य॒ज्ञ्म् प्रति॒ प्रति॑ य॒ज्ञ् मे॒वैव य॒ज्ञ्म् प्रति॑ । \newline
45. य॒ज्ञ्म् प्रति॒ प्रति॑ य॒ज्ञ्ं ॅय॒ज्ञ्म् प्रति॑ ष्ठापयति स्थापयति॒ प्रति॑ य॒ज्ञ्ं ॅय॒ज्ञ्म् प्रति॑ ष्ठापयति । \newline
46. प्रति॑ ष्ठापयति स्थापयति॒ प्रति॒ प्रति॑ ष्ठापयति प्र॒जाप॑तिः प्र॒जाप॑तिः स्थापयति॒ प्रति॒ प्रति॑ ष्ठापयति प्र॒जाप॑तिः । \newline
47. स्था॒प॒य॒ति॒ प्र॒जाप॑तिः प्र॒जाप॑तिः स्थापयति स्थापयति प्र॒जाप॑तिः प्र॒जाः प्र॒जाः प्र॒जाप॑तिः स्थापयति स्थापयति प्र॒जाप॑तिः प्र॒जाः । \newline
48. प्र॒जाप॑तिः प्र॒जाः प्र॒जाः प्र॒जाप॑तिः प्र॒जाप॑तिः प्र॒जा अ॑सृजता सृजत प्र॒जाः प्र॒जाप॑तिः प्र॒जाप॑तिः प्र॒जा अ॑सृजत । \newline
49. प्र॒जाप॑ति॒रिति॑ प्र॒जा - प॒तिः॒ । \newline
50. प्र॒जा अ॑सृजता सृजत प्र॒जाः प्र॒जा अ॑सृजत॒ स सो॑ ऽसृजत प्र॒जाः प्र॒जा अ॑सृजत॒ सः । \newline
51. प्र॒जा इति॑ प्र - जाः । \newline
52. अ॒सृ॒ज॒त॒ स सो॑ ऽसृजता सृजत॒ स रि॑रिचा॒नो रि॑रिचा॒नः सो॑ ऽसृजता सृजत॒ स रि॑रिचा॒नः । \newline
53. स रि॑रिचा॒नो रि॑रिचा॒नः स स रि॑रिचा॒नो॑ ऽमन्यता मन्यत रिरिचा॒नः स स रि॑रिचा॒नो॑ ऽमन्यत । \newline
54. रि॒रि॒चा॒नो॑ ऽमन्यता मन्यत रिरिचा॒नो रि॑रिचा॒नो॑ ऽमन्यत॒ स सो॑ ऽमन्यत रिरिचा॒नो रि॑रिचा॒नो॑ ऽमन्यत॒ सः । \newline
55. अ॒म॒न्य॒त॒ स सो॑ ऽमन्यता मन्यत॒ स ए॒ता ए॒ताः सो॑ ऽमन्यता मन्यत॒ स ए॒ताः । \newline
56. स ए॒ता ए॒ताः स स ए॒ता आ॒प्री रा॒प्री रे॒ताः स स ए॒ता आ॒प्रीः । \newline
57. ए॒ता आ॒प्री रा॒प्री रे॒ता ए॒ता आ॒प्री र॑पश्य दपश्य दा॒प्री रे॒ता ए॒ता आ॒प्री र॑पश्यत् । \newline
58. आ॒प्री र॑पश्य दपश्य दा॒प्री रा॒प्री र॑पश्य॒त् ताभि॒ स्ताभि॑ रपश्य दा॒प्री रा॒प्री र॑पश्य॒त् ताभिः॑ । \newline
59. आ॒प्रीरित्या᳚ - प्रीः । \newline
60. अ॒प॒श्य॒त् ताभि॒ स्ताभि॑ रपश्य दपश्य॒त् ताभि॒र् वै वै ताभि॑ रपश्य दपश्य॒त् ताभि॒र् वै । \newline
61. ताभि॒र् वै वै ताभि॒ स्ताभि॒र् वै स स वै ताभि॒ स्ताभि॒र् वै सः । \newline
62. वै स स वै वै स मु॑ख॒तो मु॑ख॒तः स वै वै स मु॑ख॒तः । \newline
63. स मु॑ख॒तो मु॑ख॒तः स स मु॑ख॒त आ॒त्मान॑ मा॒त्मान॑म् मुख॒तः स स मु॑ख॒त आ॒त्मान᳚म् । \newline
64. मु॒ख॒त आ॒त्मान॑ मा॒त्मान॑म् मुख॒तो मु॑ख॒त आ॒त्मान॒ मा ऽऽत्मान॑म् मुख॒तो मु॑ख॒त आ॒त्मान॒ मा । \newline
\pagebreak
\markright{ TS 5.1.8.4  \hfill https://www.vedavms.in \hfill}

\section{ TS 5.1.8.4 }

\textbf{TS 5.1.8.4 } \newline
\textbf{Samhita Paata} \newline

आ॒त्मान॒मा ऽप्री॑णीत॒ यदे॒ता आ॒प्रियो॒ भव॑न्ति य॒ज्ञो वै प्र॒जाप॑ति-र्य॒ज्ञ्मे॒वैताभि॑र्मुख॒त आ प्री॑णा॒त्य-प॑रिमितछन्दसो भव॒न्त्यप॑रिमितः प्र॒जाप॑तिः प्र॒जाप॑ते॒राप्त्या॑ ऊनातिरि॒क्ता मि॑थु॒नाः प्रजा᳚त्यै लोम॒शं ॅवै नामै॒तच्छन्दः॑ प्र॒जाप॑तेः प॒शवो॑ लोम॒शाः प॒शूने॒वाऽव॑ रुन्धे॒ सर्वा॑णि॒ वा ए॒ता रू॒पाणि॒ सर्वा॑णि रू॒पाण्य॒ग्नौ चित्ये᳚ क्रियन्ते॒ तस्मा॑दे॒ता अ॒ग्नेश्चित्य॑स्य - [  ] \newline

\textbf{Pada Paata} \newline

आ॒त्मान᳚म् । एति॑ । अ॒प्री॒णी॒त॒ । यत् । ए॒ताः । आ॒प्रिय॒ इत्या᳚-प्रियः॑ । भव॑न्ति । य॒ज्ञ्ः । वै । प्र॒जाप॑ति॒रिति॑ प्र॒जा - प॒तिः॒ । य॒ज्ञ्म् । ए॒व । ए॒ताभिः॑ । मु॒ख॒तः । एति॑ । प्री॒णा॒ति॒ । अप॑रिमितछन्दस॒ इत्यप॑रिमित - छ॒न्द॒सः॒ । भ॒व॒न्ति॒ । अप॑रिमित॒ इत्यप॑रि - मि॒तः॒ । प्र॒जाप॑ति॒रिति॑ प्र॒जा - प॒तिः॒ । प्र॒जाप॑ते॒रिति॑ प्र॒जा - प॒तेः॒ । आप्त्यै᳚ । ऊ॒ना॒ति॒रि॒क्ता इत्यू॑न - अ॒ति॒रि॒क्ताः । मि॒थु॒नाः । प्रजा᳚त्या॒ इति॒ प्र - जा॒त्यै॒ । लो॒म॒शम् । वै । नाम॑ । ए॒तत् । छन्दः॑ । प्र॒जाप॑ते॒रिति॑ प्र॒जा - प॒तेः॒ । प॒शवः॑ । लो॒म॒शाः । प॒शून् । ए॒व । अवेति॑ । रु॒न्धे॒ । सर्वा॑णि । वै । ए॒ताः । रू॒पाणि॑ । सर्वा॑णि । रू॒पाणि॑ । अ॒ग्नौ । चित्ये᳚ । क्रि॒य॒न्ते॒ । तस्मा᳚त् । ए॒ताः । अ॒ग्नेः । चित्य॑स्य ।  \newline


\textbf{Krama Paata} \newline

आ॒त्मान॒मा । आऽप्री॑णीत । अ॒प्री॒णी॒त॒ यत् । यदे॒ताः । ए॒ता आ॒प्रियः॑ । आ॒प्रियो॒ भव॑न्ति । आ॒प्रिय॒ इत्या᳚ - प्रियः॑ । भव॑न्ति य॒ज्ञ्ः । य॒ज्ञो वै । वै प्र॒जाप॑तिः । प्र॒जाप॑तिर् य॒ज्ञ्म् । प्र॒जाप॑ति॒रिति॑ प्र॒जा - प॒तिः॒ । य॒ज्ञ्मे॒व । ए॒वैताभिः॑ । ए॒ताभि॑र् मुख॒तः । मु॒ख॒त आ । आ प्री॑णाति । प्री॒णा॒त्यप॑रिमितछन्दसः । अप॑रिमितछन्दसो भवन्ति । अप॑रिमितछन्दस॒ इत्यप॑रिमित - छ॒न्द॒सः॒ । भ॒व॒न्त्यप॑रिमितः । अप॑रिमितः प्र॒जाप॑तिः । अप॑रिमित॒ इत्यप॑रि - मि॒तः॒ । प्र॒जाप॑तिः प्र॒जाप॑तेः । प्र॒जाप॑ति॒रिति॑ प्र॒जा - प॒तिः॒ । प्र॒जाप॑ते॒राप्त्यै᳚ । प्र॒जाप॑ते॒रिति॑ प्र॒जा - प॒तेः॒ । आप्त्या॑ ऊनातिरि॒क्ताः । ऊ॒ना॒ति॒रि॒क्ता मि॑थु॒नाः । ऊ॒ना॒ति॒रि॒क्ता इत्यू॑न - अ॒ति॒रि॒क्ताः । मि॒थु॒नाः प्रजा᳚त्यै । प्रजा᳚त्यै लोम॒शम् । प्रजा᳚त्या॒ इति॒ प्र - जा॒त्यै॒ । लो॒म॒शम् ॅवै । वै नाम॑ । नामै॒तत् । ए॒तच् छन्दः॑ । छन्दः॑ प्र॒जाप॑तेः । प्र॒जाप॑तेः प॒शवः॑ । प्र॒जाप॑ते॒रिति॑ प्र॒जा - प॒तेः॒ । प॒शवो॑ लोम॒शाः । लो॒म॒शाः प॒शून् । प॒शूने॒व । ए॒वाव॑ । अव॑ रुन्धे । रु॒न्धे॒ सर्वा॑णि । सर्वा॑णि॒ वै । वा ए॒ताः । ए॒ता रू॒पाणि॑ । रू॒पाणि॒ सर्वा॑णि । सर्वा॑णि रू॒पाणि॑ । रू॒पाण्य॒ग्नौ । अ॒ग्नौ चित्ये᳚ । चित्ये᳚ क्रियन्ते । क्रि॒य॒न्ते॒ तस्मा᳚त् । तस्मा॑दे॒ताः । ए॒ता अ॒ग्नेः । अ॒ग्नेश्चित्य॑स्य । चित्य॑स्य भवन्ति \newline

\textbf{Jatai Paata} \newline

1. आ॒त्मान॒ मा ऽऽत्मान॑ मा॒त्मान॒ मा । \newline
2. आ ऽप्री॑णीता प्रीणी॒ता ऽप्री॑णीत । \newline
3. अ॒प्री॒णी॒त॒ यद् यद॑प्रीणीता प्रीणीत॒ यत् । \newline
4. यदे॒ता ए॒ता यद् यदे॒ताः । \newline
5. ए॒ता आ॒प्रिय॑ आ॒प्रिय॑ ए॒ता ए॒ता आ॒प्रियः॑ । \newline
6. आ॒प्रियो॒ भव॑न्ति॒ भव॑ न्त्या॒प्रिय॑ आ॒प्रियो॒ भव॑न्ति । \newline
7. आ॒प्रिय॒ इत्या᳚ - प्रियः॑ । \newline
8. भव॑न्ति य॒ज्ञो य॒ज्ञो भव॑न्ति॒ भव॑न्ति य॒ज्ञ्ः । \newline
9. य॒ज्ञो वै वै य॒ज्ञो य॒ज्ञो वै । \newline
10. वै प्र॒जाप॑तिः प्र॒जाप॑ति॒र् वै वै प्र॒जाप॑तिः । \newline
11. प्र॒जाप॑तिर् य॒ज्ञ्ं ॅय॒ज्ञ्म् प्र॒जाप॑तिः प्र॒जाप॑तिर् य॒ज्ञ्म् । \newline
12. प्र॒जाप॑ति॒रिति॑ प्र॒जा - प॒तिः॒ । \newline
13. य॒ज्ञ् मे॒वैव य॒ज्ञ्ं ॅय॒ज्ञ् मे॒व । \newline
14. ए॒वैताभि॑ रे॒ताभि॑ रे॒वैवैताभिः॑ । \newline
15. ए॒ताभि॑र् मुख॒तो मु॑ख॒त ए॒ताभि॑ रे॒ताभि॑र् मुख॒तः । \newline
16. मु॒ख॒त आ मु॑ख॒तो मु॑ख॒त आ । \newline
17. आ प्री॑णाति प्रीणा॒त्या प्री॑णाति । \newline
18. प्री॒णा॒ त्यप॑रिमितछन्द॒सो ऽप॑रिमितछन्दसः प्रीणाति प्रीणा॒ त्यप॑रिमितछन्दसः । \newline
19. अप॑रिमितछन्दसो भवन्ति भव॒ न्त्यप॑रिमितछन्द॒सो ऽप॑रिमितछन्दसो भवन्ति । \newline
20. अप॑रिमितछन्दस॒ इत्यप॑रिमित - छ॒न्द॒सः॒ । \newline
21. भ॒व॒ न्त्यप॑रिमि॒तो ऽप॑रिमितो भवन्ति भव॒ न्त्यप॑रिमितः । \newline
22. अप॑रिमितः प्र॒जाप॑तिः प्र॒जाप॑ति॒ रप॑रिमि॒तो ऽप॑रिमितः प्र॒जाप॑तिः । \newline
23. अप॑रिमित॒ इत्यप॑रि - मि॒तः॒ । \newline
24. प्र॒जाप॑तिः प्र॒जाप॑तेः प्र॒जाप॑तेः प्र॒जाप॑तिः प्र॒जाप॑तिः प्र॒जाप॑तेः । \newline
25. प्र॒जाप॑ति॒रिति॑ प्र॒जा - प॒तिः॒ । \newline
26. प्र॒जाप॑ते॒ राप्त्या॒ आप्त्यै᳚ प्र॒जाप॑तेः प्र॒जाप॑ते॒ राप्त्यै᳚ । \newline
27. प्र॒जाप॑ते॒रिति॑ प्र॒जा - प॒तेः॒ । \newline
28. आप्त्या॑ ऊनातिरि॒क्ता ऊ॑नातिरि॒क्ता आप्त्या॒ आप्त्या॑ ऊनातिरि॒क्ताः । \newline
29. ऊ॒ना॒ति॒रि॒क्ता मि॑थु॒ना मि॑थु॒ना ऊ॑नातिरि॒क्ता ऊ॑नातिरि॒क्ता मि॑थु॒नाः । \newline
30. ऊ॒ना॒ति॒रि॒क्ता इत्यू॑न - अ॒ति॒रि॒क्ताः । \newline
31. मि॒थु॒नाः प्रजा᳚त्यै॒ प्रजा᳚त्यै मिथु॒ना मि॑थु॒नाः प्रजा᳚त्यै । \newline
32. प्रजा᳚त्यै लोम॒शम् ॅलो॑म॒शम् प्रजा᳚त्यै॒ प्रजा᳚त्यै लोम॒शम् । \newline
33. प्रजा᳚त्या॒ इति॒ प्र - जा॒त्यै॒ । \newline
34. लो॒म॒शं ॅवै वै लो॑म॒शम् ॅलो॑म॒शं ॅवै । \newline
35. वै नाम॒ नाम॒ वै वै नाम॑ । \newline
36. नामै॒त दे॒तन् नाम॒ नामै॒तत् । \newline
37. ए॒तच् छन्द॒ श्छन्द॑ ए॒त दे॒तच् छन्दः॑ । \newline
38. छन्दः॑ प्र॒जाप॑तेः प्र॒जाप॑ते॒ श्छन्द॒ श्छन्दः॑ प्र॒जाप॑तेः । \newline
39. प्र॒जाप॑तेः प॒शवः॑ प॒शवः॑ प्र॒जाप॑तेः प्र॒जाप॑तेः प॒शवः॑ । \newline
40. प्र॒जाप॑ते॒रिति॑ प्र॒जा - प॒तेः॒ । \newline
41. प॒शवो॑ लोम॒शा लो॑म॒शाः प॒शवः॑ प॒शवो॑ लोम॒शाः । \newline
42. लो॒म॒शाः प॒शून् प॒शून् ॅलो॑म॒शा लो॑म॒शाः प॒शून् । \newline
43. प॒शू ने॒वैव प॒शून् प॒शू ने॒व । \newline
44. ए॒वावा वै॒वै वाव॑ । \newline
45. अव॑ रुन्धे रु॒न्धे ऽवाव॑ रुन्धे । \newline
46. रु॒न्धे॒ सर्वा॑णि॒ सर्वा॑णि रुन्धे रुन्धे॒ सर्वा॑णि । \newline
47. सर्वा॑णि॒ वै वै सर्वा॑णि॒ सर्वा॑णि॒ वै । \newline
48. वा ए॒ता ए॒ता वै वा ए॒ताः । \newline
49. ए॒ता रू॒पाणि॑ रू॒पा ण्ये॒ता ए॒ता रू॒पाणि॑ । \newline
50. रू॒पाणि॒ सर्वा॑णि॒ सर्वा॑णि रू॒पाणि॑ रू॒पाणि॒ सर्वा॑णि । \newline
51. सर्वा॑णि रू॒पाणि॑ रू॒पाणि॒ सर्वा॑णि॒ सर्वा॑णि रू॒पाणि॑ । \newline
52. रू॒पा ण्य॒ग्ना व॒ग्नौ रू॒पाणि॑ रू॒पा ण्य॒ग्नौ । \newline
53. अ॒ग्नौ चित्ये॒ चित्ये॒ ऽग्ना व॒ग्नौ चित्ये᳚ । \newline
54. चित्ये᳚ क्रियन्ते क्रियन्ते॒ चित्ये॒ चित्ये᳚ क्रियन्ते । \newline
55. क्रि॒य॒न्ते॒ तस्मा॒त् तस्मा᳚त् क्रियन्ते क्रियन्ते॒ तस्मा᳚त् । \newline
56. तस्मा॑ दे॒ता ए॒ता स्तस्मा॒त् तस्मा॑ दे॒ताः । \newline
57. ए॒ता अ॒ग्ने र॒ग्ने रे॒ता ए॒ता अ॒ग्नेः । \newline
58. अ॒ग्नेश्चित्य॑स्य॒ चित्य॑स्या॒ ग्ने र॒ग्ने श्चित्य॑स्य । \newline
59. चित्य॑स्य भवन्ति भवन्ति॒ चित्य॑स्य॒ चित्य॑स्य भवन्ति । \newline

\textbf{Ghana Paata } \newline

1. आ॒त्मान॒ मा ऽऽत्मान॑ मा॒त्मान॒ मा ऽप्री॑णीता प्रीणी॒ता ऽऽत्मान॑ मा॒त्मान॒ मा ऽप्री॑णीत । \newline
2. आ ऽप्री॑णीता प्रीणी॒ता ऽप्री॑णीत॒ यद् यद॑प्रीणी॒ता ऽप्री॑णीत॒ यत् । \newline
3. अ॒प्री॒णी॒त॒ यद् यद॑प्रीणीता प्रीणीत॒ यदे॒ता ए॒ता यद॑प्रीणीता प्रीणीत॒ यदे॒ताः । \newline
4. यदे॒ता ए॒ता यद् यदे॒ता आ॒प्रिय॑ आ॒प्रिय॑ ए॒ता यद् यदे॒ता आ॒प्रियः॑ । \newline
5. ए॒ता आ॒प्रिय॑ आ॒प्रिय॑ ए॒ता ए॒ता आ॒प्रियो॒ भव॑न्ति॒ भव॑ न्त्या॒प्रिय॑ ए॒ता ए॒ता आ॒प्रियो॒ भव॑न्ति । \newline
6. आ॒प्रियो॒ भव॑न्ति॒ भव॑ न्त्या॒प्रिय॑ आ॒प्रियो॒ भव॑न्ति य॒ज्ञो य॒ज्ञो भव॑ न्त्या॒प्रिय॑ आ॒प्रियो॒ भव॑न्ति य॒ज्ञ्ः । \newline
7. आ॒प्रिय॒ इत्या᳚ - प्रियः॑ । \newline
8. भव॑न्ति य॒ज्ञो य॒ज्ञो भव॑न्ति॒ भव॑न्ति य॒ज्ञो वै वै य॒ज्ञो भव॑न्ति॒ भव॑न्ति य॒ज्ञो वै । \newline
9. य॒ज्ञो वै वै य॒ज्ञो य॒ज्ञो वै प्र॒जाप॑तिः प्र॒जाप॑ति॒र् वै य॒ज्ञो य॒ज्ञो वै प्र॒जाप॑तिः । \newline
10. वै प्र॒जाप॑तिः प्र॒जाप॑ति॒र् वै वै प्र॒जाप॑तिर् य॒ज्ञ्ं ॅय॒ज्ञ्म् प्र॒जाप॑ति॒र् वै वै प्र॒जाप॑तिर् य॒ज्ञ्म् । \newline
11. प्र॒जाप॑तिर् य॒ज्ञ्ं ॅय॒ज्ञ्म् प्र॒जाप॑तिः प्र॒जाप॑तिर् य॒ज्ञ् मे॒वैव य॒ज्ञ्म् प्र॒जाप॑तिः प्र॒जाप॑तिर् य॒ज्ञ् मे॒व । \newline
12. प्र॒जाप॑ति॒रिति॑ प्र॒जा - प॒तिः॒ । \newline
13. य॒ज्ञ् मे॒वैव य॒ज्ञ्ं ॅय॒ज्ञ् मे॒वैताभि॑ रे॒ताभि॑ रे॒व य॒ज्ञ्ं ॅय॒ज्ञ् मे॒वैताभिः॑ । \newline
14. ए॒वैताभि॑ रे॒ताभि॑ रे॒वैवैताभि॑र् मुख॒तो मु॑ख॒त ए॒ताभि॑ रे॒वैवैताभि॑र् मुख॒तः । \newline
15. ए॒ताभि॑र् मुख॒तो मु॑ख॒त ए॒ताभि॑ रे॒ताभि॑र् मुख॒त आ मु॑ख॒त ए॒ताभि॑ रे॒ताभि॑र् मुख॒त आ । \newline
16. मु॒ख॒त आ मु॑ख॒तो मु॑ख॒त आ प्री॑णाति प्रीणा॒त्या मु॑ख॒तो मु॑ख॒त आ प्री॑णाति । \newline
17. आ प्री॑णाति प्रीणा॒त्या प्री॑णा॒ त्यप॑रिमितछन्द॒सो ऽप॑रिमितछन्दसः प्रीणा॒त्या प्री॑णा॒ त्यप॑रिमितछन्दसः । \newline
18. प्री॒णा॒ त्यप॑रिमितछन्द॒सो ऽप॑रिमितछन्दसः प्रीणाति प्रीणा॒ त्यप॑रिमितछन्दसो भवन्ति भव॒ न्त्यप॑रिमितछन्दसः प्रीणाति प्रीणा॒ त्यप॑रिमितछन्दसो भवन्ति । \newline
19. अप॑रिमितछन्दसो भवन्ति भव॒ न्त्यप॑रिमितछन्द॒सो ऽप॑रिमितछन्दसो भव॒ न्त्यप॑रिमि॒तो ऽप॑रिमितो भव॒ न्त्यप॑रिमितछन्द॒सो ऽप॑रिमितछन्दसो भव॒ न्त्यप॑रिमितः । \newline
20. अप॑रिमितछन्दस॒ इत्यप॑रिमित - छ॒न्द॒सः॒ । \newline
21. भ॒व॒ न्त्यप॑रिमि॒तो ऽप॑रिमितो भवन्ति भव॒ न्त्यप॑रिमितः प्र॒जाप॑तिः प्र॒जाप॑ति॒ रप॑रिमितो भवन्ति भव॒ न्त्यप॑रिमितः प्र॒जाप॑तिः । \newline
22. अप॑रिमितः प्र॒जाप॑तिः प्र॒जाप॑ति॒ रप॑रिमि॒तो ऽप॑रिमितः प्र॒जाप॑तिः प्र॒जाप॑तेः प्र॒जाप॑तेः प्र॒जाप॑ति॒ रप॑रिमि॒तो ऽप॑रिमितः प्र॒जाप॑तिः प्र॒जाप॑तेः । \newline
23. अप॑रिमित॒ इत्यप॑रि - मि॒तः॒ । \newline
24. प्र॒जाप॑तिः प्र॒जाप॑तेः प्र॒जाप॑तेः प्र॒जाप॑तिः प्र॒जाप॑तिः प्र॒जाप॑ते॒ राप्त्या॒ आप्त्यै᳚ प्र॒जाप॑तेः प्र॒जाप॑तिः प्र॒जाप॑तिः प्र॒जाप॑ते॒ राप्त्यै᳚ । \newline
25. प्र॒जाप॑ति॒रिति॑ प्र॒जा - प॒तिः॒ । \newline
26. प्र॒जाप॑ते॒ राप्त्या॒ आप्त्यै᳚ प्र॒जाप॑तेः प्र॒जाप॑ते॒ राप्त्या॑ ऊनातिरि॒क्ता ऊ॑नातिरि॒क्ता आप्त्यै᳚ प्र॒जाप॑तेः प्र॒जाप॑ते॒ राप्त्या॑ ऊनातिरि॒क्ताः । \newline
27. प्र॒जाप॑ते॒रिति॑ प्र॒जा - प॒तेः॒ । \newline
28. आप्त्या॑ ऊनातिरि॒क्ता ऊ॑नातिरि॒क्ता आप्त्या॒ आप्त्या॑ ऊनातिरि॒क्ता मि॑थु॒ना मि॑थु॒ना ऊ॑नातिरि॒क्ता आप्त्या॒ आप्त्या॑ ऊनातिरि॒क्ता मि॑थु॒नाः । \newline
29. ऊ॒ना॒ति॒रि॒क्ता मि॑थु॒ना मि॑थु॒ना ऊ॑नातिरि॒क्ता ऊ॑नातिरि॒क्ता मि॑थु॒नाः प्रजा᳚त्यै॒ प्रजा᳚त्यै मिथु॒ना ऊ॑नातिरि॒क्ता ऊ॑नातिरि॒क्ता मि॑थु॒नाः प्रजा᳚त्यै । \newline
30. ऊ॒ना॒ति॒रि॒क्ता इत्यू॑न - अ॒ति॒रि॒क्ताः । \newline
31. मि॒थु॒नाः प्रजा᳚त्यै॒ प्रजा᳚त्यै मिथु॒ना मि॑थु॒नाः प्रजा᳚त्यै लोम॒शम् ॅलो॑म॒शम् प्रजा᳚त्यै मिथु॒ना मि॑थु॒नाः प्रजा᳚त्यै लोम॒शम् । \newline
32. प्रजा᳚त्यै लोम॒शम् ॅलो॑म॒शम् प्रजा᳚त्यै॒ प्रजा᳚त्यै लोम॒शं ॅवै वै लो॑म॒शम् प्रजा᳚त्यै॒ प्रजा᳚त्यै लोम॒शं ॅवै । \newline
33. प्रजा᳚त्या॒ इति॒ प्र - जा॒त्यै॒ । \newline
34. लो॒म॒शं ॅवै वै लो॑म॒शम् ॅलो॑म॒शं ॅवै नाम॒ नाम॒ वै लो॑म॒शम् ॅलो॑म॒शं ॅवै नाम॑ । \newline
35. वै नाम॒ नाम॒ वै वै नामै॒त दे॒तन् नाम॒ वै वै नामै॒तत् । \newline
36. नामै॒त दे॒तन् नाम॒ नामै॒तच् छन्द॒ श्छन्द॑ ए॒तन् नाम॒ नामै॒तच् छन्दः॑ । \newline
37. ए॒तच् छन्द॒ श्छन्द॑ ए॒त दे॒तच् छन्दः॑ प्र॒जाप॑तेः प्र॒जाप॑ते॒ श्छन्द॑ ए॒त दे॒तच् छन्दः॑ प्र॒जाप॑तेः । \newline
38. छन्दः॑ प्र॒जाप॑तेः प्र॒जाप॑ते॒ श्छन्द॒ श्छन्दः॑ प्र॒जाप॑तेः प॒शवः॑ प॒शवः॑ प्र॒जाप॑ते॒ श्छन्द॒ श्छन्दः॑ प्र॒जाप॑तेः प॒शवः॑ । \newline
39. प्र॒जाप॑तेः प॒शवः॑ प॒शवः॑ प्र॒जाप॑तेः प्र॒जाप॑तेः प॒शवो॑ लोम॒शा लो॑म॒शाः प॒शवः॑ प्र॒जाप॑तेः प्र॒जाप॑तेः प॒शवो॑ लोम॒शाः । \newline
40. प्र॒जाप॑ते॒रिति॑ प्र॒जा - प॒तेः॒ । \newline
41. प॒शवो॑ लोम॒शा लो॑म॒शाः प॒शवः॑ प॒शवो॑ लोम॒शाः प॒शून् प॒शून् ॅलो॑म॒शाः प॒शवः॑ प॒शवो॑ लोम॒शाः प॒शून् । \newline
42. लो॒म॒शाः प॒शून् प॒शून् ॅलो॑म॒शा लो॑म॒शाः प॒शू ने॒वैव प॒शून् ॅलो॑म॒शा लो॑म॒शाः प॒शू ने॒व । \newline
43. प॒शू ने॒वैव प॒शून् प॒शू ने॒वावा वै॒व प॒शून् प॒शू ने॒वाव॑ । \newline
44. ए॒वावा वै॒वै वाव॑ रुन्धे रु॒न्धे ऽवै॒वै वाव॑ रुन्धे । \newline
45. अव॑ रुन्धे रु॒न्धे ऽवाव॑ रुन्धे॒ सर्वा॑णि॒ सर्वा॑णि रु॒न्धे ऽवाव॑ रुन्धे॒ सर्वा॑णि । \newline
46. रु॒न्धे॒ सर्वा॑णि॒ सर्वा॑णि रुन्धे रुन्धे॒ सर्वा॑णि॒ वै वै सर्वा॑णि रुन्धे रुन्धे॒ सर्वा॑णि॒ वै । \newline
47. सर्वा॑णि॒ वै वै सर्वा॑णि॒ सर्वा॑णि॒ वा ए॒ता ए॒ता वै सर्वा॑णि॒ सर्वा॑णि॒ वा ए॒ताः । \newline
48. वा ए॒ता ए॒ता वै वा ए॒ता रू॒पाणि॑ रू॒पा ण्ये॒ता वै वा ए॒ता रू॒पाणि॑ । \newline
49. ए॒ता रू॒पाणि॑ रू॒पा ण्ये॒ता ए॒ता रू॒पाणि॒ सर्वा॑णि॒ सर्वा॑णि रू॒पा ण्ये॒ता ए॒ता रू॒पाणि॒ सर्वा॑णि । \newline
50. रू॒पाणि॒ सर्वा॑णि॒ सर्वा॑णि रू॒पाणि॑ रू॒पाणि॒ सर्वा॑णि रू॒पाणि॑ रू॒पाणि॒ सर्वा॑णि रू॒पाणि॑ रू॒पाणि॒ सर्वा॑णि रू॒पाणि॑ । \newline
51. सर्वा॑णि रू॒पाणि॑ रू॒पाणि॒ सर्वा॑णि॒ सर्वा॑णि रू॒पा ण्य॒ग्ना व॒ग्नौ रू॒पाणि॒ सर्वा॑णि॒ सर्वा॑णि रू॒पा ण्य॒ग्नौ । \newline
52. रू॒पा ण्य॒ग्ना व॒ग्नौ रू॒पाणि॑ रू॒पा ण्य॒ग्नौ चित्ये॒ चित्ये॒ ऽग्नौ रू॒पाणि॑ रू॒पा ण्य॒ग्नौ चित्ये᳚ । \newline
53. अ॒ग्नौ चित्ये॒ चित्ये॒ ऽग्ना व॒ग्नौ चित्ये᳚ क्रियन्ते क्रियन्ते॒ चित्ये॒ ऽग्ना व॒ग्नौ चित्ये᳚ क्रियन्ते । \newline
54. चित्ये᳚ क्रियन्ते क्रियन्ते॒ चित्ये॒ चित्ये᳚ क्रियन्ते॒ तस्मा॒त् तस्मा᳚त् क्रियन्ते॒ चित्ये॒ चित्ये᳚ क्रियन्ते॒ तस्मा᳚त् । \newline
55. क्रि॒य॒न्ते॒ तस्मा॒त् तस्मा᳚त् क्रियन्ते क्रियन्ते॒ तस्मा॑ दे॒ता ए॒ता स्तस्मा᳚त् क्रियन्ते क्रियन्ते॒ तस्मा॑ दे॒ताः । \newline
56. तस्मा॑ दे॒ता ए॒ता स्तस्मा॒त् तस्मा॑ दे॒ता अ॒ग्ने र॒ग्ने रे॒ता स्तस्मा॒त् तस्मा॑ दे॒ता अ॒ग्नेः । \newline
57. ए॒ता अ॒ग्ने र॒ग्ने रे॒ता ए॒ता अ॒ग्ने श्चित्य॑स्य॒ चित्य॑स्या॒ ग्ने रे॒ता ए॒ता अ॒ग्ने श्चित्य॑स्य । \newline
58. अ॒ग्ने श्चित्य॑स्य॒ चित्य॑स्या॒ ग्ने र॒ग्ने श्चित्य॑स्य भवन्ति भवन्ति॒ चित्य॑स्या॒ग्ने र॒ग्ने श्चित्य॑स्य भवन्ति । \newline
59. चित्य॑स्य भवन्ति भवन्ति॒ चित्य॑स्य॒ चित्य॑स्य भव॒ न्त्येक॑विꣳशति॒ मेक॑विꣳशतिम् भवन्ति॒ चित्य॑स्य॒ चित्य॑स्य भव॒ न्त्येक॑विꣳशतिम् । \newline
\pagebreak
\markright{ TS 5.1.8.5  \hfill https://www.vedavms.in \hfill}

\section{ TS 5.1.8.5 }

\textbf{TS 5.1.8.5 } \newline
\textbf{Samhita Paata} \newline

भव॒न्त्ये क॑विꣳ शतिꣳ सामिधे॒नीरन्वा॑ह॒ रुग्वा ए॑कविꣳ॒॒शो रुच॑मे॒व ग॑च्छ॒त्यथो᳚ प्रति॒ष्ठामे॒व प्र॑ति॒ष्ठा ह्ये॑कविꣳ॒॒श-श्चतु॑र्विꣳशति॒मन्वा॑ह॒ चतु॑र्विꣳशतिरर्द्धमा॒साः सं॑ॅवथ्स॒रः सं॑ॅवथ्स॒रो᳚ऽग्निर्वै᳚श्वान॒रः सा॒क्षादे॒व वै᳚श्वान॒रमव॑ रुन्धे॒ परा॑ची॒रन्वा॑ह॒ परा॑ङिव॒ हि सु॑व॒र्गो लो॒कः समा᳚स्त्वाऽग्न ऋ॒तवो॑ वर्द्धय॒न्त्वित्या॑ह॒ समा॑भिरे॒वाऽग्निं ॅव॑र्द्धय - [  ] \newline

\textbf{Pada Paata} \newline

भ॒व॒न्ति॒ । एक॑विꣳशति॒मित्येक॑ - विꣳ॒॒श॒ति॒म् । सा॒मि॒धे॒नीरिति॑ सां - इ॒धे॒नीः । अन्विति॑ । आ॒ह॒ । रुक् । वै । ए॒क॒विꣳ॒॒श इत्ये॑क - विꣳ॒॒शः । रुच᳚म् । ए॒व । ग॒च्छ॒ति॒ । अथो॒ इति॑ । प्र॒ति॒ष्ठामिति॑ प्रति - स्थाम् । ए॒व । प्र॒ति॒ष्ठेति॑ प्रति - स्था । हि । ए॒क॒विꣳ॒॒श इत्ये॑क - विꣳ॒॒शः । चतु॑र्विꣳशति॒मिति॒ चतुः॑ - विꣳ॒॒श॒ति॒म् । अन्विति॑ । आ॒ह॒ । चतु॑र्विꣳशति॒रिति॒ चतुः॑ - विꣳ॒॒श॒तिः॒ । अ॒द्‌र्ध॒मा॒सा इत्य॑द्‌र्ध - मा॒साः । सं॒ॅव॒थ्स॒र इति॑ सं - व॒थ्स॒रः । सं॒ॅव॒थ्स॒र इति॑ सं - व॒थ्स॒रः । अ॒ग्निः । वै॒श्वा॒न॒रः । सा॒क्षादिति॑ स - अ॒क्षात् । ए॒व । वै॒श्वा॒न॒रम् । अवेति॑ । रु॒न्धे॒ । परा॑चीः । अन्विति॑ । आ॒ह॒ । पराङ्॑ । इ॒व॒ । हि । सु॒व॒र्ग इति॑ सुवः - गः । लो॒कः । समाः᳚ । त्वा॒ । अ॒ग्ने॒ । ऋ॒तवः॑ । व॒द्‌र्ध॒य॒न्तु॒ । इति॑ । आ॒ह॒ । समा॑भिः । ए॒व । अ॒ग्निम् । व॒द्‌र्ध॒य॒ति॒ ।  \newline


\textbf{Krama Paata} \newline

भ॒व॒न्त्येक॑विꣳशतिम् । एक॑विꣳशतिꣳ सामिधे॒नीः । एक॑विꣳशति॒मित्येक॑ - विꣳ॒॒श॒ति॒म् । सा॒मि॒धे॒नीरनु॑ । सा॒मि॒धे॒नीरिति॑ साम् - इ॒धे॒नीः । अन्वा॑ह । आ॒ह॒ रुक् । रुग् वै । वा ए॑कविꣳ॒॒शः । ए॒क॒विꣳ॒॒शो रुच᳚म् । ए॒क॒विꣳ॒॒श इत्ये॑क - विꣳ॒॒शः । रुच॑मे॒व । ए॒व ग॑च्छति । ग॒च्छ॒त्यथो᳚ । अथो᳚ प्रति॒ष्ठाम् । अथो॒ इत्यथो᳚ । प्र॒ति॒ष्ठामे॒व । प्र॒ति॒ष्ठामिति॑ प्रति - स्थाम् । ए॒व प्र॑ति॒ष्ठा । प्र॒ति॒ष्ठा हि । प्र॒ति॒ष्ठेति॑ प्रति - स्था । ह्ये॑कविꣳ॒॒शः । ए॒क॒विꣳ॒॒श,श्चतु॑र्विꣳशतिम् । ए॒क॒विꣳ॒॒श इत्ये॑क - विꣳ॒॒शः । चतु॑र्विꣳशति॒मनु॑ । चतु॑र्विꣳशति॒मिति॒ चतुः॑ - विꣳ॒॒श॒ति॒म् । अन्वा॑ह । आ॒ह॒ चतु॑र्विꣳशतिः । चतु॑र्विꣳशतिरर्द्धमा॒साः । चतु॑र्विꣳशति॒रिति॒ चतुः॑ - विꣳ॒॒श॒तिः॒ । अ॒र्द्ध॒मा॒साः स॑म्ॅवथ्स॒रः । अ॒र्द्ध॒मा॒सा इत्य॑र्द्ध - मा॒साः । स॒म्ॅव॒थ्स॒रः स॑म्ॅवथ्स॒रः । स॒म्ॅव॒थ्स॒र इति॑ सम् - व॒थ्स॒रः । स॒म्ॅव॒थ्स॒रो᳚ऽग्निः । स॒म्ॅव॒थ्स॒र इति॑ सम् - व॒थ्स॒रः । अ॒ग्निर् वै᳚श्वान॒रः । वै॒श्वा॒न॒रः सा॒क्षात् । सा॒क्षादे॒व । सा॒क्षादिति॑ स - अ॒क्षात् । ए॒व वै᳚श्वान॒रम् । वै॒श्वा॒न॒रमव॑ । अव॑ रुन्धे । रु॒न्धे॒ परा॑चीः । परा॑ची॒रनु॑ । अन्वा॑ह । आ॒ह॒ पराङ्॑ । परा॑ङिव । इ॒व॒ हि । हि सु॑व॒र्गः । सु॒व॒र्गो लो॒कः । सु॒व॒र्ग इति॑ सुवः - गः । लो॒कः समाः᳚ । समा᳚स्त्वा । त्वा॒ऽग्ने॒ । अ॒ग्न॒ ऋ॒तवः॑ । ऋ॒तवो॑ वर्द्धयन्तु । व॒र्द्ध॒य॒न्त्विति॑ । इत्या॑ह । आ॒ह॒ समा॑भिः । समा॑भिरे॒व । ए॒वाग्निम् । अ॒ग्निम् ॅव॑र्द्धयति । व॒र्द्ध॒य॒त्यृ॒तुभिः॑ \newline

\textbf{Jatai Paata} \newline

1. भ॒व॒ न्त्येक॑विꣳशति॒ मेक॑विꣳशतिम् भवन्ति भव॒ न्त्येक॑विꣳशतिम् । \newline
2. एक॑विꣳशतिꣳ सामिधे॒नीः सा॑मिधे॒नी रेक॑विꣳशति॒ मेक॑विꣳशतिꣳ सामिधे॒नीः । \newline
3. एक॑विꣳशति॒मित्येक॑ - विꣳ॒॒श॒ति॒म् । \newline
4. सा॒मि॒धे॒नी रन्वनु॑ सामिधे॒नीः सा॑मिधे॒नी रनु॑ । \newline
5. सा॒मि॒धे॒नीरिति॑ सां - इ॒धे॒नीः । \newline
6. अन्वा॑ हा॒हा न्वन् वा॑ह । \newline
7. आ॒ह॒ रुग् रुगा॑हाह॒ रुक् । \newline
8. रुग् वै वै रुग् रुग् वै । \newline
9. वा ए॑कविꣳ॒॒श ए॑कविꣳ॒॒शो वै वा ए॑कविꣳ॒॒शः । \newline
10. ए॒क॒विꣳ॒॒शो रुचꣳ॒॒ रुच॑ मेकविꣳ॒॒श ए॑कविꣳ॒॒शो रुच᳚म् । \newline
11. ए॒क॒विꣳ॒॒श इत्ये॑क - विꣳ॒॒शः । \newline
12. रुच॑ मे॒वैव रुचꣳ॒॒ रुच॑ मे॒व । \newline
13. ए॒व ग॑च्छति गच्छ त्ये॒वैव ग॑च्छति । \newline
14. ग॒च्छ॒ त्यथो॒ अथो॑ गच्छति गच्छ॒ त्यथो᳚ । \newline
15. अथो᳚ प्रति॒ष्ठाम् प्र॑ति॒ष्ठा मथो॒ अथो᳚ प्रति॒ष्ठाम् । \newline
16. अथो॒ इत्यथो᳚ । \newline
17. प्र॒ति॒ष्ठा मे॒वैव प्र॑ति॒ष्ठाम् प्र॑ति॒ष्ठा मे॒व । \newline
18. प्र॒ति॒ष्ठामिति॑ प्रति - स्थाम् । \newline
19. ए॒व प्र॑ति॒ष्ठा प्र॑ति॒ ष्ठैवैव प्र॑ति॒ष्ठा । \newline
20. प्र॒ति॒ष्ठा हि हि प्र॑ति॒ष्ठा प्र॑ति॒ष्ठा हि । \newline
21. प्र॒ति॒ष्ठेति॑ प्रति - स्था । \newline
22. ह्ये॑कविꣳ॒॒श ए॑कविꣳ॒॒शो हि ह्ये॑कविꣳ॒॒शः । \newline
23. ए॒क॒विꣳ॒॒श श्चतु॑र्विꣳशति॒म् चतु॑र्विꣳशति मेकविꣳ॒॒श ए॑कविꣳ॒॒श श्चतु॑र्विꣳशतिम् । \newline
24. ए॒क॒विꣳ॒॒श इत्ये॑क - विꣳ॒॒शः । \newline
25. चतु॑र्विꣳशति॒ मन्वनु॒ चतु॑र्विꣳशति॒म् चतु॑र्विꣳशति॒ मनु॑ । \newline
26. चतु॑र्विꣳशति॒मिति॒ चतुः॑ - विꣳ॒॒श॒ति॒म् । \newline
27. अन्वा॑ हा॒हा न्वन् वा॑ह । \newline
28. आ॒ह॒ चतु॑र्विꣳशति॒ श्चतु॑र्विꣳशतिराहाह॒ चतु॑र्विꣳशतिः । \newline
29. चतु॑र्विꣳशति रर्द्धमा॒सा अ॑र्द्धमा॒सा श्चतु॑र्विꣳशति॒ श्चतु॑र्विꣳशति रर्द्धमा॒साः । \newline
30. चतु॑र्विꣳशति॒रिति॒ चतुः॑ - विꣳ॒॒श॒तिः॒ । \newline
31. अ॒र्द्ध॒मा॒साः सं॑ॅवथ्स॒रः सं॑ॅवथ्स॒रो᳚ ऽर्द्धमा॒सा अ॑र्द्धमा॒साः सं॑ॅवथ्स॒रः । \newline
32. अ॒र्द्ध॒मा॒सा इत्य॑र्द्ध - मा॒साः । \newline
33. सं॒ॅव॒थ्स॒रः सं॑ॅवथ्स॒रः । \newline
34. सं॒ॅव॒थ्स॒र इति॑ सं - व॒थ्स॒रः । \newline
35. सं॒ॅव॒थ्स॒रो᳚ ऽग्नि र॒ग्निः सं॑ॅवथ्स॒रः सं॑ॅवथ्स॒रो᳚ ऽग्निः । \newline
36. सं॒ॅव॒थ्स॒र इति॑ सं - व॒थ्स॒रः । \newline
37. अ॒ग्निर् वै᳚श्वान॒रो वै᳚श्वान॒रो᳚ ऽग्निर॒ग्निर् वै᳚श्वान॒रः । \newline
38. वै॒श्वा॒न॒रः सा॒क्षाथ् सा॒क्षाद् वै᳚श्वान॒रो वै᳚श्वान॒रः सा॒क्षात् । \newline
39. सा॒क्षा दे॒वैव सा॒क्षाथ् सा॒क्षा दे॒व । \newline
40. सा॒क्षादिति॑ स - अ॒क्षात् । \newline
41. ए॒व वै᳚श्वान॒रं ॅवै᳚श्वान॒र मे॒वैव वै᳚श्वान॒रम् । \newline
42. वै॒श्वा॒न॒र मवाव॑ वैश्वान॒रं ॅवै᳚श्वान॒र मव॑ । \newline
43. अव॑ रुन्धे रु॒न्धे ऽवाव॑ रुन्धे । \newline
44. रु॒न्धे॒ परा॑चीः॒ परा॑ची रुन्धे रुन्धे॒ परा॑चीः । \newline
45. परा॑ची॒ रन्वनु॒ परा॑चीः॒ परा॑ची॒ रनु॑ । \newline
46. अन्वा॑ हा॒हा न्वन् वा॑ह । \newline
47. आ॒ह॒ परा॒ङ् परा॑ ङाहाह॒ पराङ्॑ । \newline
48. परा॑ ङिवेव॒ परा॒ङ् परा॑ ङिव । \newline
49. इ॒व॒ हि हीवे॑व॒ हि । \newline
50. हि सु॑व॒र्गः सु॑व॒र्गो हि हि सु॑व॒र्गः । \newline
51. सु॒व॒र्गो लो॒को लो॒कः सु॑व॒र्गः सु॑व॒र्गो लो॒कः । \newline
52. सु॒व॒र्ग इति॑ सुवः - गः । \newline
53. लो॒कः समाः॒ समा॑ लो॒को लो॒कः समाः᳚ । \newline
54. समा᳚ स्त्वा त्वा॒ समाः॒ समा᳚ स्त्वा । \newline
55. त्वा॒ ऽग्ने॒ अ॒ग्ने॒ त्वा॒ त्वा॒ ऽग्ने॒ । \newline
56. अ॒ग्न॒ ऋ॒तव॑ ऋ॒तवो॑ अग्ने अग्न ऋ॒तवः॑ । \newline
57. ऋ॒तवो॑ वर्द्धयन्तु वर्द्धय न्त्वृ॒तव॑ ऋ॒तवो॑ वर्द्धयन्तु । \newline
58. व॒र्द्ध॒य॒ न्त्वितीति॑ वर्द्धयन्तु वर्द्धय॒ न्त्विति॑ । \newline
59. इत्या॑हा॒हे तीत्या॑ह । \newline
60. आ॒ह॒ समा॑भिः॒ समा॑भि राहाह॒ समा॑भिः । \newline
61. समा॑भि रे॒वैव समा॑भिः॒ समा॑भि रे॒व । \newline
62. ए॒वाग्नि म॒ग्नि मे॒वैवाग्निम् । \newline
63. अ॒ग्निं ॅव॑र्द्धयति वर्द्धय त्य॒ग्नि म॒ग्निं ॅव॑र्द्धयति । \newline
64. व॒र्द्ध॒य॒ त्यृ॒तुभिर्॑. ऋ॒तुभि॑र् वर्द्धयति वर्द्धय त्यृ॒तुभिः॑ । \newline

\textbf{Ghana Paata } \newline

1. भ॒व॒ न्त्येक॑विꣳशति॒ मेक॑विꣳशतिम् भवन्ति भव॒ न्त्येक॑विꣳशतिꣳ सामिधे॒नीः सा॑मिधे॒नी रेक॑विꣳशतिम् भवन्ति भव॒ न्त्येक॑विꣳशतिꣳ सामिधे॒नीः । \newline
2. एक॑विꣳशतिꣳ सामिधे॒नीः सा॑मिधे॒नी रेक॑विꣳशति॒ मेक॑विꣳशतिꣳ सामिधे॒नी रन्वनु॑ सामिधे॒नी रेक॑विꣳशति॒ मेक॑विꣳशतिꣳ सामिधे॒नी रनु॑ । \newline
3. एक॑विꣳशति॒मित्येक॑ - विꣳ॒॒श॒ति॒म् । \newline
4. सा॒मि॒धे॒नी रन्वनु॑ सामिधे॒नीः सा॑मिधे॒नी रन्वा॑ हा॒हानु॑ सामिधे॒नीः सा॑मिधे॒नी रन्वा॑ह । \newline
5. सा॒मि॒धे॒नीरिति॑ सां - इ॒धे॒नीः । \newline
6. अन्वा॑हा॒हान् वन् वा॑ह॒ रुग् रुगा॒ हान् वन् वा॑ह॒ रुक् । \newline
7. आ॒ह॒ रुग् रुगा॑हाह॒ रुग् वै वै रुगा॑हाह॒ रुग् वै । \newline
8. रुग् वै वै रुग् रुग् वा ए॑कविꣳ॒॒श ए॑कविꣳ॒॒शो वै रुग् रुग् वा ए॑कविꣳ॒॒शः । \newline
9. वा ए॑कविꣳ॒॒श ए॑कविꣳ॒॒शो वै वा ए॑कविꣳ॒॒शो रुचꣳ॒॒ रुच॑ मेकविꣳ॒॒शो वै वा ए॑कविꣳ॒॒शो रुच᳚म् । \newline
10. ए॒क॒विꣳ॒॒शो रुचꣳ॒॒ रुच॑ मेकविꣳ॒॒श ए॑कविꣳ॒॒शो रुच॑ मे॒वैव रुच॑ मेकविꣳ॒॒श ए॑कविꣳ॒॒शो रुच॑ मे॒व । \newline
11. ए॒क॒विꣳ॒॒श इत्ये॑क - विꣳ॒॒शः । \newline
12. रुच॑ मे॒वैव रुचꣳ॒॒ रुच॑ मे॒व ग॑च्छति गच्छ त्ये॒व रुचꣳ॒॒ रुच॑ मे॒व ग॑च्छति । \newline
13. ए॒व ग॑च्छति गच्छ त्ये॒वैव ग॑च्छ॒ त्यथो॒ अथो॑ गच्छ त्ये॒वैव ग॑च्छ॒ त्यथो᳚ । \newline
14. ग॒च्छ॒ त्यथो॒ अथो॑ गच्छति गच्छ॒ त्यथो᳚ प्रति॒ष्ठाम् प्र॑ति॒ष्ठा मथो॑ गच्छति गच्छ॒ त्यथो᳚ प्रति॒ष्ठाम् । \newline
15. अथो᳚ प्रति॒ष्ठाम् प्र॑ति॒ष्ठा मथो॒ अथो᳚ प्रति॒ष्ठा मे॒वैव प्र॑ति॒ष्ठा मथो॒ अथो᳚ प्रति॒ष्ठा मे॒व । \newline
16. अथो॒ इत्यथो᳚ । \newline
17. प्र॒ति॒ष्ठा मे॒वैव प्र॑ति॒ष्ठाम् प्र॑ति॒ष्ठा मे॒व प्र॑ति॒ष्ठा प्र॑ति॒ष्ठैव प्र॑ति॒ष्ठाम् प्र॑ति॒ष्ठा मे॒व प्र॑ति॒ष्ठा । \newline
18. प्र॒ति॒ष्ठामिति॑ प्रति - स्थाम् । \newline
19. ए॒व प्र॑ति॒ष्ठा प्र॑ति॒ष्ठैवैव प्र॑ति॒ष्ठा हि हि प्र॑ति॒ष्ठैवैव प्र॑ति॒ष्ठा हि । \newline
20. प्र॒ति॒ष्ठा हि हि प्र॑ति॒ष्ठा प्र॑ति॒ष्ठा ह्ये॑कविꣳ॒॒श ए॑कविꣳ॒॒शो हि प्र॑ति॒ष्ठा प्र॑ति॒ष्ठा ह्ये॑कविꣳ॒॒शः । \newline
21. प्र॒ति॒ष्ठेति॑ प्रति - स्था । \newline
22. ह्ये॑कविꣳ॒॒श ए॑कविꣳ॒॒शो हि ह्ये॑कविꣳ॒॒श श्चतु॑र्विꣳशति॒म् चतु॑र्विꣳशति मेकविꣳ॒॒शो हि ह्ये॑कविꣳ॒॒श श्चतु॑र्विꣳशतिम् । \newline
23. ए॒क॒विꣳ॒॒श श्चतु॑र्विꣳशति॒म् चतु॑र्विꣳशति मेकविꣳ॒॒श ए॑कविꣳ॒॒श श्चतु॑र्विꣳशति॒ मन्वनु॒ चतु॑र्विꣳशति मेकविꣳ॒॒श ए॑कविꣳ॒॒श श्चतु॑र्विꣳशति॒ मनु॑ । \newline
24. ए॒क॒विꣳ॒॒श इत्ये॑क - विꣳ॒॒शः । \newline
25. चतु॑र्विꣳशति॒ मन्वनु॒ चतु॑र्विꣳशति॒म् चतु॑र्विꣳशति॒ मन्वा॑ हा॒हानु॒ चतु॑र्विꣳशति॒म् चतु॑र्विꣳशति॒ मन्वा॑ह । \newline
26. चतु॑र्विꣳशति॒मिति॒ चतुः॑ - विꣳ॒॒श॒ति॒म् । \newline
27. अन्वा॑ हा॒हान् वन्वा॑ह॒ चतु॑र्विꣳशति॒ श्चतु॑र्विꣳशति रा॒हान् वन्वा॑ह॒ चतु॑र्विꣳशतिः । \newline
28. आ॒ह॒ चतु॑र्विꣳशति॒ श्चतु॑र्विꣳशति राहाह॒ चतु॑र्विꣳशति रर्द्धमा॒सा अ॑र्द्धमा॒सा श्चतु॑र्विꣳशतिराहाह॒ चतु॑र्विꣳशति रर्द्धमा॒साः । \newline
29. चतु॑र्विꣳशति रर्द्धमा॒सा अ॑र्द्धमा॒सा श्चतु॑र्विꣳशति॒ श्चतु॑र्विꣳशति रर्द्धमा॒साः सं॑ॅवथ्स॒रः सं॑ॅवथ्स॒रो᳚ ऽर्द्धमा॒सा श्चतु॑र्विꣳशति॒ श्चतु॑र्विꣳशति रर्द्धमा॒साः सं॑ॅवथ्स॒रः । \newline
30. चतु॑र्विꣳशति॒रिति॒ चतुः॑ - विꣳ॒॒श॒तिः॒ । \newline
31. अ॒र्द्ध॒मा॒साः सं॑ॅवथ्स॒रः सं॑ॅवथ्स॒रो᳚ ऽर्द्धमा॒सा अ॑र्द्धमा॒साः सं॑ॅवथ्स॒रः । \newline
32. अ॒र्द्ध॒मा॒सा इत्य॑र्द्ध - मा॒साः । \newline
33. सं॒ॅव॒थ्स॒रः सं॑ॅवथ्स॒रः । \newline
34. सं॒ॅव॒थ्स॒र इति॑ सं - व॒थ्स॒रः । \newline
35. सं॒ॅव॒थ्स॒रो᳚ ऽग्नि र॒ग्निः सं॑ॅवथ्स॒रः सं॑ॅवथ्स॒रो᳚ ऽग्निर् वै᳚श्वान॒रो वै᳚श्वान॒रो᳚ ऽग्निः सं॑ॅवथ्स॒रः सं॑ॅवथ्स॒रो᳚ ऽग्निर् वै᳚श्वान॒रः । \newline
36. सं॒ॅव॒थ्स॒र इति॑ सं - व॒थ्स॒रः । \newline
37. अ॒ग्निर् वै᳚श्वान॒रो वै᳚श्वान॒रो᳚ ऽग्नि र॒ग्निर् वै᳚श्वान॒रः सा॒क्षाथ् सा॒क्षाद् वै᳚श्वान॒रो᳚ ऽग्नि र॒ग्निर् वै᳚श्वान॒रः सा॒क्षात् । \newline
38. वै॒श्वा॒न॒रः सा॒क्षाथ् सा॒क्षाद् वै᳚श्वान॒रो वै᳚श्वान॒रः सा॒क्षा दे॒वैव सा॒क्षाद् वै᳚श्वान॒रो वै᳚श्वान॒रः सा॒क्षा दे॒व । \newline
39. सा॒क्षा दे॒वैव सा॒क्षाथ् सा॒क्षा दे॒व वै᳚श्वान॒रं ॅवै᳚श्वान॒र मे॒व सा॒क्षाथ् सा॒क्षा दे॒व वै᳚श्वान॒रम् । \newline
40. सा॒क्षादिति॑ स - अ॒क्षात् । \newline
41. ए॒व वै᳚श्वान॒रं ॅवै᳚श्वान॒र मे॒वैव वै᳚श्वान॒र मवाव॑ वैश्वान॒र मे॒वैव वै᳚श्वान॒र मव॑ । \newline
42. वै॒श्वा॒न॒र मवाव॑ वैश्वान॒रं ॅवै᳚श्वान॒र मव॑ रुन्धे रु॒न्धे ऽव॑ वैश्वान॒रं ॅवै᳚श्वान॒र मव॑ रुन्धे । \newline
43. अव॑ रुन्धे रु॒न्धे ऽवाव॑ रुन्धे॒ परा॑चीः॒ परा॑ची रु॒न्धे ऽवाव॑ रुन्धे॒ परा॑चीः । \newline
44. रु॒न्धे॒ परा॑चीः॒ परा॑ची रुन्धे रुन्धे॒ परा॑ची॒ रन्वनु॒ परा॑ची रुन्धे रुन्धे॒ परा॑ची॒ रनु॑ । \newline
45. परा॑ची॒ रन्वनु॒ परा॑चीः॒ परा॑ची॒ रन्वा॑ हा॒हानु॒ परा॑चीः॒ परा॑ची॒ रन्वा॑ह । \newline
46. अन्वा॑ हा॒हा न्वन्वा॑ह॒ परा॒ङ् परा॑ङा॒हान् वन्वा॑ह॒ पराङ्॑ । \newline
47. आ॒ह॒ परा॒ङ् परा॑ ङाहाह॒ परा॑ङिवेव॒ परा॑ ङाहाह॒ परा॑ङिव । \newline
48. परा॑ङिवेव॒ परा॒ङ् परा॑ङिव॒ हि हीव॒ परा॒ङ् परा॑ङिव॒ हि । \newline
49. इ॒व॒ हि हीवे॑व॒ हि सु॑व॒र्गः सु॑व॒र्गो हीवे॑व॒ हि सु॑व॒र्गः । \newline
50. हि सु॑व॒र्गः सु॑व॒र्गो हि हि सु॑व॒र्गो लो॒को लो॒कः सु॑व॒र्गो हि हि सु॑व॒र्गो लो॒कः । \newline
51. सु॒व॒र्गो लो॒को लो॒कः सु॑व॒र्गः सु॑व॒र्गो लो॒कः समाः॒ समा॑ लो॒कः सु॑व॒र्गः सु॑व॒र्गो लो॒कः समाः᳚ । \newline
52. सु॒व॒र्ग इति॑ सुवः - गः । \newline
53. लो॒कः समाः॒ समा॑ लो॒को लो॒कः समा᳚ स्त्वा त्वा॒ समा॑ लो॒को लो॒कः समा᳚ स्त्वा । \newline
54. समा᳚ स्त्वा त्वा॒ समाः॒ समा᳚ स्त्वा ऽग्ने अग्ने त्वा॒ समाः॒ समा᳚ स्त्वा ऽग्ने । \newline
55. त्वा॒ ऽग्ने॒ अ॒ग्ने॒ त्वा॒ त्वा॒ ऽग्न॒ ऋ॒तव॑ ऋ॒तवो॑ अग्ने त्वा त्वा ऽग्न ऋ॒तवः॑ । \newline
56. अ॒ग्न॒ ऋ॒तव॑ ऋ॒तवो॑ अग्ने अग्न ऋ॒तवो॑ वर्द्धयन्तु वर्द्धय न्त्वृ॒तवो॑ अग्ने अग्न ऋ॒तवो॑ वर्द्धयन्तु । \newline
57. ऋ॒तवो॑ वर्द्धयन्तु वर्द्धय न्त्वृ॒तव॑ ऋ॒तवो॑ वर्द्धय॒ न्त्वितीति॑ वर्द्धय न्त्वृ॒तव॑ ऋ॒तवो॑ वर्द्धय॒ न्त्विति॑ । \newline
58. व॒र्द्ध॒य॒ न्त्वितीति॑ वर्द्धयन्तु वर्द्धय॒ न्त्वित्या॑हा॒हेति॑ वर्द्धयन्तु वर्द्धय॒ न्त्वित्या॑ह । \newline
59. इत्या॑हा॒हे तीत्या॑ह॒ समा॑भिः॒ समा॑भि रा॒हे तीत्या॑ह॒ समा॑भिः । \newline
60. आ॒ह॒ समा॑भिः॒ समा॑भि राहाह॒ समा॑भि रे॒वैव समा॑भि राहाह॒ समा॑भि रे॒व । \newline
61. समा॑भि रे॒वैव समा॑भिः॒ समा॑भि रे॒वाग्नि म॒ग्नि मे॒व समा॑भिः॒ समा॑भि रे॒वाग्निम् । \newline
62. ए॒वाग्नि म॒ग्नि मे॒वैवाग्निं ॅव॑र्द्धयति वर्द्धय त्य॒ग्नि मे॒वैवाग्निं ॅव॑र्द्धयति । \newline
63. अ॒ग्निं ॅव॑र्द्धयति वर्द्धय त्य॒ग्नि म॒ग्निं ॅव॑र्द्धय त्यृ॒तुभिर्॑. ऋ॒तुभि॑र् वर्द्धय त्य॒ग्नि म॒ग्निं ॅव॑र्द्धय त्यृ॒तुभिः॑ । \newline
64. व॒र्द्ध॒य॒ त्यृ॒तुभिर्॑. ऋ॒तुभि॑र् वर्द्धयति वर्द्धय त्यृ॒तुभिः॑ संॅवथ्स॒रꣳ सं॑ॅवथ्स॒र मृ॒तुभि॑र् वर्द्धयति वर्द्धय त्यृ॒तुभिः॑ संॅवथ्स॒रम् । \newline
\pagebreak
\markright{ TS 5.1.8.6  \hfill https://www.vedavms.in \hfill}

\section{ TS 5.1.8.6 }

\textbf{TS 5.1.8.6 } \newline
\textbf{Samhita Paata} \newline

त्यृ॒तुभिः॑ संॅवथ्स॒रं ॅविश्वा॒ आ भा॑हि प्र॒दिशः॑ पृथि॒व्या इत्या॑ह॒ तस्मा॑द॒ग्निः सर्वा॒ दिशोऽनु॒ विभा॑ति॒ प्रत्यौ॑हताम॒श्विना॑ मृ॒त्युम॑स्मा॒दित्या॑ह मृ॒त्युमे॒वाऽस्मा॒दप॑ नुद॒त्युद्व॒यं तम॑स॒स्परीत्या॑ह पा॒प्मा वै तमः॑ पा॒प्मान॑मे॒वास्मा॒दप॑ ह॒न्त्यग॑न्म॒ ज्योति॑रुत्त॒म-मित्या॑हा॒ऽसौ वा आ॑दि॒त्यो ( ) ज्योति॑रुत्त॒म-मा॑दि॒त्यस्यै॒व सायु॑ज्यं गच्छति॒ न सं॑ॅवथ्स॒रस्ति॑ष्ठति॒ नास्य॒ श्रीस्ति॑ष्ठति॒ यस्यै॒ताः क्रि॒यन्ते॒ ज्योति॑ष्मती-मुत्त॒मामन्वा॑ह॒ ज्योति॑रे॒वास्मा॑ उ॒परि॑ष्टाद् दधाति सुव॒र्गस्य॑ लो॒कस्यानु॑ख्यात्यै ॥ \newline

\textbf{Pada Paata} \newline

ऋ॒तुभि॒रित्यृ॒तु - भिः॒ । सं॒ॅव॒थ्स॒रमिति॑ सं - व॒थ्स॒रम् । विश्वाः᳚ । एति॑ । भा॒हि॒ । प्र॒दिश॒ इति॑ प्र - दिशः॑ । पृ॒थि॒व्याः । इति॑ । आ॒ह॒ । तस्मा᳚त् । अ॒ग्निः । सर्वाः᳚ । दिशः॑ । अनु॑ । वीति॑ । भा॒ति॒ । प्रतीति॑ । औ॒ह॒ता॒म् । अ॒श्विना᳚ । मृ॒त्युम् । अ॒स्मा॒त् । इति॑ । आ॒ह॒ । मृ॒त्युम् । ए॒व । अ॒स्मा॒त् । अपेति॑ । नु॒द॒ति॒ । उदिति॑ । व॒यम् । तम॑सः । परीति॑ । इति॑ । आ॒ह॒ । पा॒प्मा । वै । तमः॑ । पा॒प्मान᳚म् । ए॒व । अ॒स्मा॒त् । अपेति॑ । ह॒न्ति॒ । अग॑न्म । ज्योतिः॑ । उ॒त्त॒ममित्यु॑त् - त॒मम् । इति॑ । आ॒ह॒ । अ॒सौ । वै । आ॒दि॒त्यः ( ) । ज्योतिः॑ । उ॒त्त॒ममित्यु॑त्-त॒मम् । आ॒दि॒त्यस्य॑ । ए॒व । सायु॑ज्यम् । ग॒च्छ॒ति॒ । न । सं॒ॅव॒थ्स॒र इति॑ सं - व॒थ्स॒रः । ति॒ष्ठ॒ति॒ । न । अ॒स्य॒ । श्रीः । ति॒ष्ठ॒ति॒ । यस्य॑ । ए॒ताः । क्रि॒यन्ते᳚ । ज्योति॑ष्मतीम् । उ॒त्त॒मामित्यु॑त् - त॒माम् । अन्विति॑ । आ॒ह॒ । ज्योतिः॑ । ए॒व । अ॒स्मै॒ । उ॒परि॑ष्टात् । द॒धा॒ति॒ । सु॒व॒र्गस्येति॑ सुवः - गस्य॑ । लो॒कस्य॑ । अनु॑ख्यात्या॒ इत्यनु॑ - ख्या॒त्यै॒ ॥  \newline


\textbf{Krama Paata} \newline

ऋ॒तुभिः॑ सम्ॅवथ्स॒रम् । ऋ॒तुभि॒रित्यृ॒तु - भिः॒ । स॒म्ॅव॒थ्स॒रम् ॅविश्वाः᳚ । स॒म्ॅव॒थ्स॒रमिति॑ सम् - व॒थ्स॒रम् । विश्वा॒ आ । आ भा॑हि । भा॒हि॒ प्र॒दिशः॑ । प्र॒दिशः॑ पृथि॒व्याः । प्र॒दिश॒ इति॑ प्र - दिशः॑ । पृ॒थि॒व्या इति॑ । इत्या॑ह । आ॒ह॒ तस्मा᳚त् । तस्मा॑द॒ग्निः । अ॒ग्निः सर्वाः᳚ । सर्वा॒ दिशः॑ । दिशोऽनु॑ । अनु॒ वि । वि भा॑ति । भा॒ति॒ प्रति॑ । प्रत्यौ॑हताम् । औ॒ह॒ता॒,म॒श्विना᳚ । अ॒श्विना॑ मृ॒त्युम् । मृ॒त्युम॑स्मात् । अ॒स्मा॒दिति॑ । इत्या॑ह । आ॒ह॒ मृ॒त्युम् । मृ॒त्युमे॒व । ए॒वास्मा᳚त् । अ॒स्मा॒दप॑ । अप॑ नुदति । नु॒द॒त्युत् । उद् व॒यम् । व॒यम् तम॑सः । तम॑स॒स्परि॑ । परीति॑ । इत्या॑ह । आ॒ह॒ पा॒प्मा । पा॒प्मा वै । वै तमः॑ । तमः॑ पा॒प्मान᳚म् । पा॒प्मान॑मे॒व । ए॒वास्मा᳚त् । अ॒स्मा॒दप॑ । अप॑ हन्ति । ह॒न्त्यग॑न्म । अग॑न्म॒ ज्योतिः॑ । ज्योति॑रुत्त॒मम् । उ॒त्त॒ममिति॑ । उ॒त्त॒ममित्यु॑त् - त॒मम् । इत्या॑ह । आ॒हा॒सौ । अ॒सौ वै । वा आ॑दि॒त्यः ( ) । आ॒दि॒त्यो ज्योतिः॑ । ज्योति॑रुत्त॒मम् । उ॒त्त॒ममा॑दि॒त्यस्य॑ । उ॒त्त॒ममित्यु॑त् - त॒मम् । आ॒दि॒त्यस्यै॒व । ए॒व सायु॑ज्यम् । सायु॑ज्यम् गच्छति । ग॒च्छ॒ति॒ न । न स॑म्ॅवथ्स॒रः । स॒म्ॅव॒थ्स॒रस्ति॑ष्ठति । स॒म्ॅव॒थ्स॒र इति॑ सम् - व॒थ्स॒रः । ति॒ष्ठ॒ति॒ न । नास्य॑ । अ॒स्य॒ श्रीः । श्रीस्ति॑ष्ठति । ति॒ष्ठ॒ति॒ यस्य॑ । यस्यै॒ताः । ए॒ताः क्रि॒यन्ते᳚ । क्रि॒यन्ते॒ ज्योति॑ष्मतीम् । ज्योति॑ष्मतीमुत्त॒माम् । उ॒त्त॒मामनु॑ । उ॒त्त॒मामित्यु॑त् - त॒माम् । अन्वा॑ह । आ॒ह॒ ज्योतिः॑ । ज्योति॑रे॒व । ए॒वास्मै᳚ । अ॒स्मा॒ उ॒परि॑ष्टात् । उ॒परि॑ष्टाद् दधाति । द॒धा॒ति॒ सु॒व॒र्गस्य॑ । सु॒व॒र्गस्य॑ लो॒कस्य॑ । सु॒व॒र्गस्येति॑ सुवः - गस्य॑ । लो॒कस्यानु॑ख्यात्यै । अनु॑ख्यात्या॒ इत्यनु॑ - ख्या॒त्यै॒ । \newline

\textbf{Jatai Paata} \newline

1. ऋ॒तुभिः॑ संॅवथ्स॒रꣳ सं॑ॅवथ्स॒र मृ॒तुभिर्॑. ऋ॒तुभिः॑ संॅवथ्स॒रम् । \newline
2. ऋ॒तुभि॒रित्यृ॒तु - भिः॒ । \newline
3. सं॒ॅव॒थ्स॒रं ॅविश्वा॒ विश्वाः᳚ संॅवथ्स॒रꣳ सं॑ॅवथ्स॒रं ॅविश्वाः᳚ । \newline
4. सं॒ॅव॒थ्स॒रमिति॑ सं - व॒थ्स॒रम् । \newline
5. विश्वा॒ आ विश्वा॒ विश्वा॒ आ । \newline
6. आ भा॑हि भा॒ह्या भा॑हि । \newline
7. भा॒हि॒ प्र॒दिशः॑ प्र॒दिशो॑ भाहि भाहि प्र॒दिशः॑ । \newline
8. प्र॒दिशः॑ पृथि॒व्याः पृ॑थि॒व्याः प्र॒दिशः॑ प्र॒दिशः॑ पृथि॒व्याः । \newline
9. प्र॒दिश॒ इति॑ प्र - दिशः॑ । \newline
10. पृ॒थि॒व्या इतीति॑ पृथि॒व्याः पृ॑थि॒व्या इति॑ । \newline
11. इत्या॑हा॒हे तीत्या॑ह । \newline
12. आ॒ह॒ तस्मा॒त् तस्मा॑ दाहाह॒ तस्मा᳚त् । \newline
13. तस्मा॑ द॒ग्नि र॒ग्नि स्तस्मा॒त् तस्मा॑ द॒ग्निः । \newline
14. अ॒ग्निः सर्वाः॒ सर्वा॑ अ॒ग्नि र॒ग्निः सर्वाः᳚ । \newline
15. सर्वा॒ दिशो॒ दिशः॒ सर्वाः॒ सर्वा॒ दिशः॑ । \newline
16. दिशो ऽन्वनु॒ दिशो॒ दिशो ऽनु॑ । \newline
17. अनु॒ वि व्यन्वनु॒ वि । \newline
18. वि भा॑ति भाति॒ वि वि भा॑ति । \newline
19. भा॒ति॒ प्रति॒ प्रति॑ भाति भाति॒ प्रति॑ । \newline
20. प्रत्यौ॑हता मौहता॒म् प्रति॒ प्रत्यौ॑हताम् । \newline
21. औ॒ह॒ता॒ म॒श्विना॒ ऽश्विनौ॑हता मौहता म॒श्विना᳚ । \newline
22. अ॒श्विना॑ मृ॒त्युम् मृ॒त्यु म॒श्विना॒ ऽश्विना॑ मृ॒त्युम् । \newline
23. मृ॒त्यु म॑स्मा दस्मान् मृ॒त्युम् मृ॒त्यु म॑स्मात् । \newline
24. अ॒स्मा॒ दितीत्य॑स्मा दस्मा॒ दिति॑ । \newline
25. इत्या॑हा॒हे तीत्या॑ह । \newline
26. आ॒ह॒ मृ॒त्युम् मृ॒त्यु मा॑हाह मृ॒त्युम् । \newline
27. मृ॒त्यु मे॒वैव मृ॒त्युम् मृ॒त्यु मे॒व । \newline
28. ए॒वास्मा॑ दस्मा दे॒वैवास्मा᳚त् । \newline
29. अ॒स्मा॒ दपापा᳚ स्मा दस्मा॒ दप॑ । \newline
30. अप॑ नुदति नुद॒ त्यपाप॑ नुदति । \newline
31. नु॒द॒ त्युदुन् नु॑दति नुद॒ त्युत् । \newline
32. उद् व॒यं ॅव॒य मुदुद् व॒यम् । \newline
33. व॒यम् तम॑स॒ स्तम॑सो व॒यं ॅव॒यम् तम॑सः । \newline
34. तम॑स॒ स्परि॒ परि॒ तम॑स॒ स्तम॑स॒ स्परि॑ । \newline
35. परीतीति॒ परि॒ परीति॑ । \newline
36. इत्या॑हा॒हे तीत्या॑ह । \newline
37. आ॒ह॒ पा॒प्मा पा॒प्मा ऽऽहा॑ह पा॒प्मा । \newline
38. पा॒प्मा वै वै पा॒प्मा पा॒प्मा वै । \newline
39. वै तम॒ स्तमो॒ वै वै तमः॑ । \newline
40. तमः॑ पा॒प्मान॑म् पा॒प्मान॒म् तम॒ स्तमः॑ पा॒प्मान᳚म् । \newline
41. पा॒प्मान॑ मे॒वैव पा॒प्मान॑म् पा॒प्मान॑ मे॒व । \newline
42. ए॒वास्मा॑ दस्मा दे॒वैवास्मा᳚त् । \newline
43. अ॒स्मा॒ दपापा᳚ स्मा दस्मा॒ दप॑ । \newline
44. अप॑ हन्ति ह॒न्त्यपाप॑ हन्ति । \newline
45. ह॒न्त्यग॒न्मा ग॑न्म हन्ति ह॒न्त्य ग॑न्म । \newline
46. अग॑न्म॒ ज्योति॒र् ज्योति॒ रग॒न्मा ग॑न्म॒ ज्योतिः॑ । \newline
47. ज्योति॑ रुत्त॒म मु॑त्त॒मम् ज्योति॒र् ज्योति॑ रुत्त॒मम् । \newline
48. उ॒त्त॒म मितीत्यु॑त्त॒म मु॑त्त॒म मिति॑ । \newline
49. उ॒त्त॒ममित्यु॑त् - त॒मम् । \newline
50. इत्या॑हा॒हे तीत्या॑ह । \newline
51. आ॒हा॒सा व॒सा वा॑हा हा॒सौ । \newline
52. अ॒सौ वै वा अ॒सा व॒सौ वै । \newline
53. वा आ॑दि॒त्य आ॑दि॒त्यो वै वा आ॑दि॒त्यः । \newline
54. आ॒दि॒त्यो ज्योति॒र् ज्योति॑ रादि॒त्य आ॑दि॒त्यो ज्योतिः॑ । \newline
55. ज्योति॑ रुत्त॒म मु॑त्त॒मम् ज्योति॒र् ज्योति॑ रुत्त॒मम् । \newline
56. उ॒त्त॒म मा॑दि॒त्यस्या॑ दि॒त्यस्यो᳚त्त॒म मु॑त्त॒म मा॑दि॒त्यस्य॑ । \newline
57. उ॒त्त॒ममित्यु॑त् - त॒मम् । \newline
58. आ॒दि॒त्य स्यै॒वैवा दि॒त्यस्या॑ दि॒त्य स्यै॒व । \newline
59. ए॒व सायु॑ज्यꣳ॒॒ सायु॑ज्य मे॒वैव सायु॑ज्यम् । \newline
60. सायु॑ज्यम् गच्छति गच्छति॒ सायु॑ज्यꣳ॒॒ सायु॑ज्यम् गच्छति । \newline
61. ग॒च्छ॒ति॒ न न ग॑च्छति गच्छति॒ न । \newline
62. न सं॑ॅवथ्स॒रः सं॑ॅवथ्स॒रो न न सं॑ॅवथ्स॒रः । \newline
63. सं॒ॅव॒थ्स॒र स्ति॑ष्ठति तिष्ठति संॅवथ्स॒रः सं॑ॅवथ्स॒र स्ति॑ष्ठति । \newline
64. सं॒ॅव॒थ्स॒र इति॑ सं - व॒थ्स॒रः । \newline
65. ति॒ष्ठ॒ति॒ न न ति॑ष्ठति तिष्ठति॒ न । \newline
66. नास्या᳚स्य॒ न नास्य॑ । \newline
67. अ॒स्य॒ श्रीः श्री र॑स्यास्य॒ श्रीः । \newline
68. श्री स्ति॑ष्ठति तिष्ठति॒ श्रीः श्री स्ति॑ष्ठति । \newline
69. ति॒ष्ठ॒ति॒ यस्य॒ यस्य॑ तिष्ठति तिष्ठति॒ यस्य॑ । \newline
70. यस्यै॒ता ए॒ता यस्य॒ यस्यै॒ताः । \newline
71. ए॒ताः क्रि॒यन्ते᳚ क्रि॒यन्त॑ ए॒ता ए॒ताः क्रि॒यन्ते᳚ । \newline
72. क्रि॒यन्ते॒ ज्योति॑ष्मती॒म् ज्योति॑ष्मतीम् क्रि॒यन्ते᳚ क्रि॒यन्ते॒ ज्योति॑ष्मतीम् । \newline
73. ज्योति॑ष्मती मुत्त॒मा मु॑त्त॒माम् ज्योति॑ष्मती॒म् ज्योति॑ष्मती मुत्त॒माम् । \newline
74. उ॒त्त॒मा मन्वनू᳚त्त॒मा मु॑त्त॒मा मनु॑ । \newline
75. उ॒त्त॒मामित्यु॑त् - त॒माम् । \newline
76. अन्वा॑ हा॒हा न्वन् वा॑ह । \newline
77. आ॒ह॒ ज्योति॒र् ज्योति॑ राहाह॒ ज्योतिः॑ । \newline
78. ज्योति॑ रे॒वैव ज्योति॒र् ज्योति॑ रे॒व । \newline
79. ए॒वास्मा॑ अस्मा ए॒वैवास्मै᳚ । \newline
80. अ॒स्मा॒ उ॒परि॑ष्टा दु॒परि॑ष्टा दस्मा अस्मा उ॒परि॑ष्टात् । \newline
81. उ॒परि॑ष्टाद् दधाति दधा त्यु॒परि॑ष्टा दु॒परि॑ष्टाद् दधाति । \newline
82. द॒धा॒ति॒ सु॒व॒र्गस्य॑ सुव॒र्गस्य॑ दधाति दधाति सुव॒र्गस्य॑ । \newline
83. सु॒व॒र्गस्य॑ लो॒कस्य॑ लो॒कस्य॑ सुव॒र्गस्य॑ सुव॒र्गस्य॑ लो॒कस्य॑ । \newline
84. सु॒व॒र्गस्येति॑ सुवः - गस्य॑ । \newline
85. लो॒कस्या नु॑ख्यात्या॒ अनु॑ख्यात्यै लो॒कस्य॑ लो॒कस्या नु॑ख्यात्यै । \newline
86. अनु॑ख्यात्या॒ इत्यनु॑ - ख्या॒त्यै॒ । \newline

\textbf{Ghana Paata } \newline

1. ऋ॒तुभिः॑ संॅवथ्स॒रꣳ सं॑ॅवथ्स॒र मृ॒तुभिर्॑. ऋ॒तुभिः॑ संॅवथ्स॒रं ॅविश्वा॒ विश्वाः᳚ संॅवथ्स॒र मृ॒तुभिर्॑. ऋ॒तुभिः॑ संॅवथ्स॒रं ॅविश्वाः᳚ । \newline
2. ऋ॒तुभि॒रित्यृ॒तु - भिः॒ । \newline
3. सं॒ॅव॒थ्स॒रं ॅविश्वा॒ विश्वाः᳚ संॅवथ्स॒रꣳ सं॑ॅवथ्स॒रं ॅविश्वा॒ आ विश्वाः᳚ संॅवथ्स॒रꣳ सं॑ॅवथ्स॒रं ॅविश्वा॒ आ । \newline
4. सं॒ॅव॒थ्स॒रमिति॑ सं - व॒थ्स॒रम् । \newline
5. विश्वा॒ आ विश्वा॒ विश्वा॒ आ भा॑हि भा॒ह्या विश्वा॒ विश्वा॒ आ भा॑हि । \newline
6. आ भा॑हि भा॒ह्या भा॑हि प्र॒दिशः॑ प्र॒दिशो॑ भा॒ह्या भा॑हि प्र॒दिशः॑ । \newline
7. भा॒हि॒ प्र॒दिशः॑ प्र॒दिशो॑ भाहि भाहि प्र॒दिशः॑ पृथि॒व्याः पृ॑थि॒व्याः प्र॒दिशो॑ भाहि भाहि प्र॒दिशः॑ पृथि॒व्याः । \newline
8. प्र॒दिशः॑ पृथि॒व्याः पृ॑थि॒व्याः प्र॒दिशः॑ प्र॒दिशः॑ पृथि॒व्या इतीति॑ पृथि॒व्याः प्र॒दिशः॑ प्र॒दिशः॑ पृथि॒व्या इति॑ । \newline
9. प्र॒दिश॒ इति॑ प्र - दिशः॑ । \newline
10. पृ॒थि॒व्या इतीति॑ पृथि॒व्याः पृ॑थि॒व्या इत्या॑हा॒हेति॑ पृथि॒व्याः पृ॑थि॒व्या इत्या॑ह । \newline
11. इत्या॑हा॒हे तीत्या॑ह॒ तस्मा॒त् तस्मा॑ दा॒हे तीत्या॑ह॒ तस्मा᳚त् । \newline
12. आ॒ह॒ तस्मा॒त् तस्मा॑ दाहाह॒ तस्मा॑ द॒ग्नि र॒ग्नि स्तस्मा॑ दाहाह॒ तस्मा॑ द॒ग्निः । \newline
13. तस्मा॑ द॒ग्नि र॒ग्नि स्तस्मा॒त् तस्मा॑ द॒ग्निः सर्वाः॒ सर्वा॑ अ॒ग्नि स्तस्मा॒त् तस्मा॑ द॒ग्निः सर्वाः᳚ । \newline
14. अ॒ग्निः सर्वाः॒ सर्वा॑ अ॒ग्नि र॒ग्निः सर्वा॒ दिशो॒ दिशः॒ सर्वा॑ अ॒ग्नि र॒ग्निः सर्वा॒ दिशः॑ । \newline
15. सर्वा॒ दिशो॒ दिशः॒ सर्वाः॒ सर्वा॒ दिशो ऽन्वनु॒ दिशः॒ सर्वाः॒ सर्वा॒ दिशो ऽनु॑ । \newline
16. दिशो ऽन्वनु॒ दिशो॒ दिशो ऽनु॒ वि व्यनु॒ दिशो॒ दिशो ऽनु॒ वि । \newline
17. अनु॒ वि व्यन्वनु॒ वि भा॑ति भाति॒ व्यन्वनु॒ वि भा॑ति । \newline
18. वि भा॑ति भाति॒ वि वि भा॑ति॒ प्रति॒ प्रति॑ भाति॒ वि वि भा॑ति॒ प्रति॑ । \newline
19. भा॒ति॒ प्रति॒ प्रति॑ भाति भाति॒ प्रत्यौ॑हता मौहता॒म् प्रति॑ भाति भाति॒ प्रत्यौ॑हताम् । \newline
20. प्रत्यौ॑हता मौहता॒म् प्रति॒ प्रत्यौ॑हता म॒श्विना॒ ऽश्विनौ॑हता॒म् प्रति॒ प्रत्यौ॑हता म॒श्विना᳚ । \newline
21. औ॒ह॒ता॒ म॒श्विना॒ ऽश्विनौ॑हता मौहता म॒श्विना॑ मृ॒त्युम् मृ॒त्यु म॒श्विनौ॑हता मौहता म॒श्विना॑ मृ॒त्युम् । \newline
22. अ॒श्विना॑ मृ॒त्युम् मृ॒त्यु म॒श्विना॒ ऽश्विना॑ मृ॒त्यु म॑स्मा दस्मान् मृ॒त्यु म॒श्विना॒ ऽश्विना॑ मृ॒त्यु म॑स्मात् । \newline
23. मृ॒त्यु म॑स्मा दस्मान् मृ॒त्युम् मृ॒त्यु म॑स्मा॒ दितीत्य॑स्मान् मृ॒त्युम् मृ॒त्यु म॑स्मा॒ दिति॑ । \newline
24. अ॒स्मा॒दिती त्य॑स्मा दस्मा॒ दित्या॑हा॒हे त्य॑स्मा दस्मा॒ दित्या॑ह । \newline
25. इत्या॑हा॒हे तीत्या॑ह मृ॒त्युम् मृ॒त्यु मा॒हे तीत्या॑ह मृ॒त्युम् । \newline
26. आ॒ह॒ मृ॒त्युम् मृ॒त्यु मा॑हाह मृ॒त्यु मे॒वैव मृ॒त्यु मा॑हाह मृ॒त्यु मे॒व । \newline
27. मृ॒त्यु मे॒वैव मृ॒त्युम् मृ॒त्यु मे॒वास्मा॑ दस्मा दे॒व मृ॒त्युम् मृ॒त्यु मे॒वास्मा᳚त् । \newline
28. ए॒वास्मा॑ दस्मा दे॒वैवास्मा॒ दपापा᳚स्मा दे॒वैवा स्मा॒दप॑ । \newline
29. अ॒स्मा॒ दपापा᳚ स्मा दस्मा॒ दप॑ नुदति नुद॒ त्यपा᳚स्मा दस्मा॒ दप॑ नुदति । \newline
30. अप॑ नुदति नुद॒ त्यपाप॑ नुद॒ त्युदुन् नु॑द॒ त्यपाप॑ नुद॒त्युत् । \newline
31. नु॒द॒ त्युदुन् नु॑दति नुद॒ त्युद् व॒यं ॅव॒य मुन् नु॑दति नुद॒ त्युद् व॒यम् । \newline
32. उद् व॒यं ॅव॒य मुदुद् व॒यम् तम॑स॒ स्तम॑सो व॒य मुदुद् व॒यम् तम॑सः । \newline
33. व॒यम् तम॑स॒ स्तम॑सो व॒यं ॅव॒यम् तम॑स॒ स्परि॒ परि॒ तम॑सो व॒यं ॅव॒यम् तम॑स॒ स्परि॑ । \newline
34. तम॑स॒ स्परि॒ परि॒ तम॑स॒ स्तम॑स॒ स्परीतीति॒ परि॒ तम॑स॒ स्तम॑स॒ स्परीति॑ । \newline
35. परीतीति॒ परि॒ परी त्या॑हा॒हेति॒ परि॒ परीत्या॑ह । \newline
36. इत्या॑हा॒हे तीत्या॑ह पा॒प्मा पा॒प्मा ऽऽहेतीत्या॑ह पा॒प्मा । \newline
37. आ॒ह॒ पा॒प्मा पा॒प्मा ऽऽहा॑ह पा॒प्मा वै वै पा॒प्मा ऽऽहा॑ह पा॒प्मा वै । \newline
38. पा॒प्मा वै वै पा॒प्मा पा॒प्मा वै तम॒ स्तमो॒ वै पा॒प्मा पा॒प्मा वै तमः॑ । \newline
39. वै तम॒ स्तमो॒ वै वै तमः॑ पा॒प्मान॑म् पा॒प्मान॒म् तमो॒ वै वै तमः॑ पा॒प्मान᳚म् । \newline
40. तमः॑ पा॒प्मान॑म् पा॒प्मान॒म् तम॒ स्तमः॑ पा॒प्मान॑ मे॒वैव पा॒प्मान॒म् तम॒ स्तमः॑ पा॒प्मान॑ मे॒व । \newline
41. पा॒प्मान॑ मे॒वैव पा॒प्मान॑म् पा॒प्मान॑ मे॒वास्मा॑ दस्मा दे॒व पा॒प्मान॑म् पा॒प्मान॑ मे॒वास्मा᳚त् । \newline
42. ए॒वास्मा॑ दस्मा दे॒वैवास्मा॒ दपापा᳚स्मा दे॒वैवास्मा॒ दप॑ । \newline
43. अ॒स्मा॒ दपापा᳚स्मा दस्मा॒ दप॑ हन्ति ह॒न्त्यपा᳚स्मा दस्मा॒ दप॑ हन्ति । \newline
44. अप॑ हन्ति ह॒न्त्यपाप॑ ह॒न्त्यग॒न्मा ग॑न्म ह॒न्त्यपाप॑ ह॒न्त्यग॑न्म । \newline
45. ह॒न्त्यग॒न्मा ग॑न्म हन्ति ह॒न्त्यग॑न्म॒ ज्योति॒र् ज्योति॒ रग॑न्म हन्ति ह॒न्त्यग॑न्म॒ ज्योतिः॑ । \newline
46. अग॑न्म॒ ज्योति॒र् ज्योति॒ रग॒न्मा ग॑न्म॒ ज्योति॑ रुत्त॒म मु॑त्त॒मम् ज्योति॒ रग॒न्मा ग॑न्म॒ ज्योति॑ रुत्त॒मम् । \newline
47. ज्योति॑ रुत्त॒म मु॑त्त॒मम् ज्योति॒र् ज्योति॑ रुत्त॒म मिती त्यु॑त्त॒मम् ज्योति॒र् ज्योति॑ रुत्त॒म मिति॑ । \newline
48. उ॒त्त॒म मितीत्यु॑त्त॒म मु॑त्त॒म मित्या॑हा॒हे त्यु॑त्त॒म मु॑त्त॒म मित्या॑ह । \newline
49. उ॒त्त॒ममित्यु॑त् - त॒मम् । \newline
50. इत्या॑हा॒हे तीत्या॑हा॒सा व॒सा वा॒हे तीत्या॑हा॒सौ । \newline
51. आ॒हा॒सा व॒सा वा॑हाहा॒सौ वै वा अ॒सा वा॑हाहा॒सौ वै । \newline
52. अ॒सौ वै वा अ॒सा व॒सौ वा आ॑दि॒त्य आ॑दि॒त्यो वा अ॒सा व॒सौ वा आ॑दि॒त्यः । \newline
53. वा आ॑दि॒त्य आ॑दि॒त्यो वै वा आ॑दि॒त्यो ज्योति॒र् ज्योति॑ रादि॒त्यो वै वा आ॑दि॒त्यो ज्योतिः॑ । \newline
54. आ॒दि॒त्यो ज्योति॒र् ज्योति॑ रादि॒त्य आ॑दि॒त्यो ज्योति॑ रुत्त॒म मु॑त्त॒मम् ज्योति॑ रादि॒त्य आ॑दि॒त्यो ज्योति॑ रुत्त॒मम् । \newline
55. ज्योति॑ रुत्त॒म मु॑त्त॒मम् ज्योति॒र् ज्योति॑ रुत्त॒म मा॑दि॒त्यस्या॑ दि॒त्यस्यो᳚त्त॒मम् ज्योति॒र् ज्योति॑ रुत्त॒म मा॑दि॒त्यस्य॑ । \newline
56. उ॒त्त॒म मा॑दि॒त्यस्या॑ दि॒त्यस्यो᳚त्त॒म मु॑त्त॒म मा॑दि॒त्यस्यै॒वैवा दि॒त्यस्यो᳚त्त॒म मु॑त्त॒म मा॑दि॒त्यस्यै॒व । \newline
57. उ॒त्त॒ममित्यु॑त् - त॒मम् । \newline
58. आ॒दि॒त्य स्यै॒वैवादि॒त्यस्या॑ दि॒त्यस्यै॒व सायु॑ज्यꣳ॒॒ सायु॑ज्य मे॒वादि॒त्यस्या॑ दि॒त्यस्यै॒व सायु॑ज्यम् । \newline
59. ए॒व सायु॑ज्यꣳ॒॒ सायु॑ज्य मे॒वैव सायु॑ज्यम् गच्छति गच्छति॒ सायु॑ज्य मे॒वैव सायु॑ज्यम् गच्छति । \newline
60. सायु॑ज्यम् गच्छति गच्छति॒ सायु॑ज्यꣳ॒॒ सायु॑ज्यम् गच्छति॒ न न ग॑च्छति॒ सायु॑ज्यꣳ॒॒ सायु॑ज्यम् गच्छति॒ न । \newline
61. ग॒च्छ॒ति॒ न न ग॑च्छति गच्छति॒ न सं॑ॅवथ्स॒रः सं॑ॅवथ्स॒रो न ग॑च्छति गच्छति॒ न सं॑ॅवथ्स॒रः । \newline
62. न सं॑ॅवथ्स॒रः सं॑ॅवथ्स॒रो न न सं॑ॅवथ्स॒र स्ति॑ष्ठति तिष्ठति संॅवथ्स॒रो न न सं॑ॅवथ्स॒र स्ति॑ष्ठति । \newline
63. सं॒ॅव॒थ्स॒र स्ति॑ष्ठति तिष्ठति संॅवथ्स॒रः सं॑ॅवथ्स॒र स्ति॑ष्ठति॒ न न ति॑ष्ठति संॅवथ्स॒रः सं॑ॅवथ्स॒र स्ति॑ष्ठति॒ न । \newline
64. सं॒ॅव॒थ्स॒र इति॑ सं - व॒थ्स॒रः । \newline
65. ति॒ष्ठ॒ति॒ न न ति॑ष्ठति तिष्ठति॒ नास्या᳚स्य॒ न ति॑ष्ठति तिष्ठति॒ नास्य॑ । \newline
66. नास्या᳚स्य॒ न नास्य॒ श्रीः श्रीर॑स्य॒ न नास्य॒ श्रीः । \newline
67. अ॒स्य॒ श्रीः श्री र॑स्यास्य॒ श्री स्ति॑ष्ठति तिष्ठति॒ श्री र॑स्यास्य॒ श्री स्ति॑ष्ठति । \newline
68. श्री स्ति॑ष्ठति तिष्ठति॒ श्रीः श्री स्ति॑ष्ठति॒ यस्य॒ यस्य॑ तिष्ठति॒ श्रीः श्री स्ति॑ष्ठति॒ यस्य॑ । \newline
69. ति॒ष्ठ॒ति॒ यस्य॒ यस्य॑ तिष्ठति तिष्ठति॒ यस्यै॒ता ए॒ता यस्य॑ तिष्ठति तिष्ठति॒ यस्यै॒ताः । \newline
70. यस्यै॒ता ए॒ता यस्य॒ यस्यै॒ताः क्रि॒यन्ते᳚ क्रि॒यन्त॑ ए॒ता यस्य॒ यस्यै॒ताः क्रि॒यन्ते᳚ । \newline
71. ए॒ताः क्रि॒यन्ते᳚ क्रि॒यन्त॑ ए॒ता ए॒ताः क्रि॒यन्ते॒ ज्योति॑ष्मती॒म् ज्योति॑ष्मतीम् क्रि॒यन्त॑ ए॒ता ए॒ताः क्रि॒यन्ते॒ ज्योति॑ष्मतीम् । \newline
72. क्रि॒यन्ते॒ ज्योति॑ष्मती॒म् ज्योति॑ष्मतीम् क्रि॒यन्ते᳚ क्रि॒यन्ते॒ ज्योति॑ष्मती मुत्त॒मा मु॑त्त॒माम् ज्योति॑ष्मतीम् क्रि॒यन्ते᳚ क्रि॒यन्ते॒ ज्योति॑ष्मती मुत्त॒माम् । \newline
73. ज्योति॑ष्मती मुत्त॒मा मु॑त्त॒माम् ज्योति॑ष्मती॒म् ज्योति॑ष्मती मुत्त॒मा मन्वनू᳚त्त॒माम् ज्योति॑ष्मती॒म् ज्योति॑ष्मती मुत्त॒मा मनु॑ । \newline
74. उ॒त्त॒मा मन्व नू᳚त्त॒मा मु॑त्त॒मा मन्वा॑हा॒हा नू᳚त्त॒मा मु॑त्त॒मा मन्वा॑ह । \newline
75. उ॒त्त॒मामित्यु॑त् - त॒माम् । \newline
76. अन्वा॑हा॒हान्वन् वा॑ह॒ ज्योति॒र् ज्योति॑ रा॒हान् वन्वा॑ह॒ ज्योतिः॑ । \newline
77. आ॒ह॒ ज्योति॒र् ज्योति॑ राहाह॒ ज्योति॑ रे॒वैव ज्योति॑ राहाह॒ ज्योति॑ रे॒व । \newline
78. ज्योति॑ रे॒वैव ज्योति॒र् ज्योति॑ रे॒वास्मा॑ अस्मा ए॒व ज्योति॒र् ज्योति॑ रे॒वास्मै᳚ । \newline
79. ए॒वास्मा॑ अस्मा ए॒वैवास्मा॑ उ॒परि॑ष्टा दु॒परि॑ष्टा दस्मा ए॒वैवास्मा॑ उ॒परि॑ष्टात् । \newline
80. अ॒स्मा॒ उ॒परि॑ष्टा दु॒परि॑ष्टा दस्मा अस्मा उ॒परि॑ष्टाद् दधाति दधा त्यु॒परि॑ष्टा दस्मा अस्मा उ॒परि॑ष्टाद् दधाति । \newline
81. उ॒परि॑ष्टाद् दधाति दधा त्यु॒परि॑ष्टा दु॒परि॑ष्टाद् दधाति सुव॒र्गस्य॑ सुव॒र्गस्य॑ दधा त्यु॒परि॑ष्टा दु॒परि॑ष्टाद् दधाति सुव॒र्गस्य॑ । \newline
82. द॒धा॒ति॒ सु॒व॒र्गस्य॑ सुव॒र्गस्य॑ दधाति दधाति सुव॒र्गस्य॑ लो॒कस्य॑ लो॒कस्य॑ सुव॒र्गस्य॑ दधाति दधाति सुव॒र्गस्य॑ लो॒कस्य॑ । \newline
83. सु॒व॒र्गस्य॑ लो॒कस्य॑ लो॒कस्य॑ सुव॒र्गस्य॑ सुव॒र्गस्य॑ लो॒कस्या नु॑ख्यात्या॒ अनु॑ख्यात्यै लो॒कस्य॑ सुव॒र्गस्य॑ सुव॒र्गस्य॑ लो॒कस्या नु॑ख्यात्यै । \newline
84. सु॒व॒र्गस्येति॑ सुवः - गस्य॑ । \newline
85. लो॒कस्या नु॑ख्यात्या॒ अनु॑ख्यात्यै लो॒कस्य॑ लो॒कस्या नु॑ख्यात्यै । \newline
86. अनु॑ख्यात्या॒ इत्यनु॑ - ख्या॒त्यै॒ । \newline
\pagebreak
\markright{ TS 5.1.9.1  \hfill https://www.vedavms.in \hfill}

\section{ TS 5.1.9.1 }

\textbf{TS 5.1.9.1 } \newline
\textbf{Samhita Paata} \newline

ष॒ड्भिर्दी᳚क्षयति॒ षड्वा ऋ॒तव॑ ऋ॒तुभि॑रे॒वैनं॑ दीक्षयति स॒प्तभि॑र्दीक्षयति स॒प्त छन्दाꣳ॑सि॒ छन्दो॑भिरे॒वैनं॑ दीक्षयति॒ विश्वे॑ दे॒वस्य॑ ने॒तुरित्य॑-नु॒ष्टुभो᳚त्त॒मया॑ जुहोति॒ वाग्वा अ॑नु॒ष्टुप् तस्मा᳚त् प्रा॒णानां॒ ॅवागु॑त्त॒मै- क॑स्माद॒क्षरा॒दना᳚प्तं प्रथ॒मं प॒दं तस्मा॒द्-यद्-वा॒चोऽना᳚प्तं॒ तन्म॑नु॒ष्या॑ उप॑ जीवन्ति पू॒र्णया॑ जुहोति पू॒र्ण इ॑व॒ हि प्र॒जाप॑तिः - [  ] \newline

\textbf{Pada Paata} \newline

ष॒ड्भिरिति॑ षट् - भिः । दी॒क्ष॒य॒ति॒ । षट् । वै । ऋ॒तवः॑ । ऋ॒तुभि॒रित्यृ॒तु-भिः॒ । ए॒व । ए॒न॒म् । दी॒क्ष॒य॒ति॒ । स॒प्तभि॒रिति॑ स॒प्त - भिः॒ । दी॒क्ष॒य॒ति॒ । स॒प्त । छन्दाꣳ॑सि । छन्दो॑भि॒रिति॒ छन्दः॑ - भिः॒ । ए॒व । ए॒न॒म् । दी॒क्ष॒य॒ति॒ । विश्वे᳚ । दे॒वस्य॑ । ने॒तुः । इति॑ । अ॒नु॒ष्टुभेत्य॑नु - स्तुभा᳚ । उ॒त्त॒मयेत्यु॑त् - त॒मया᳚ । जु॒हो॒ति॒ । वाक् । वै । अ॒नु॒ष्टुबित्य॑नु - स्तुप् । तस्मा᳚त् । प्रा॒णाना॒मिति॑ प्र - अ॒नाना᳚म् । वाक् । उ॒त्त॒मेत्यु॑त् - त॒मा । एक॑स्मात् । अ॒क्षरा᳚त् । अना᳚प्तम् । प्र॒थ॒मम् । प॒दम् । तस्मा᳚त् । यत् । वा॒चः । अना᳚प्तम् । तत् । म॒नु॒ष्याः᳚ । उपेति॑ । जी॒व॒न्ति॒ । पू॒र्णया᳚ । जु॒हो॒ति॒ । पू॒र्णः । इ॒व॒ । हि । प्र॒जाप॑ति॒रिति॑ प्र॒जा - प॒तिः॒ ।  \newline


\textbf{Krama Paata} \newline

ष॒ड्भिर् दी᳚क्षयति । ष॒ड्भिरिति॑ षट् - भिः । दी॒क्ष॒य॒ति॒ षट् । षड् वै । वा ऋ॒तवः॑ । ऋ॒तव॑ ऋ॒तुभिः॑ । ऋ॒तुभि॑रे॒व । ऋ॒तुभि॒रित्यृ॒तु - भिः॒ । ए॒वैन᳚म् । ए॒न॒म् दी॒क्ष॒य॒ति॒ । दी॒क्ष॒य॒ति॒ स॒प्तभिः॑ । स॒प्तभि॑र् दीक्षयति । स॒प्तभि॒रिति॑ स॒प्त - भिः॒ । दी॒क्ष॒य॒ति॒ स॒प्त । स॒प्त छन्दाꣳ॑सि । छन्दाꣳ॑सि॒ छन्दो॑भिः । छन्दो॑भिरे॒व । छन्दो॑भि॒रिति॒ छन्दः॑ - भिः॒ । ए॒वैन᳚म् । ए॒न॒म् दी॒क्ष॒य॒ति॒ । दी॒क्ष॒य॒ति॒ विश्वे᳚ । विश्वे॑ दे॒वस्य॑ । दे॒वस्य॑ ने॒तुः । ने॒तुरिति॑ । इत्य॑नु॒ष्टुभा᳚ । अ॒नु॒ष्टुभो᳚त्त॒मया᳚ । अ॒नु॒ष्टुभेत्य॑नु - स्तुभा᳚ । उ॒त्त॒मया॑ जुहोति । उ॒त्त॒मयेत्यु॑त् - त॒मया᳚ । जु॒हो॒ति॒ वाक् । वाग्वै । वा अ॑नु॒ष्टुप् । अ॒नु॒ष्टुप् तस्मा᳚त् । अ॒नु॒ष्टुबित्य॑नु - स्तुप् । तस्मा᳚त् प्रा॒णाना᳚म् । प्रा॒णाना॒म् ॅवाक् । प्रा॒णाना॒मिति॑ प्र - अ॒नाना᳚म् । वागु॑त्त॒मा । उ॒त्त॒मैक॑स्मात् । उ॒त्त॒मेत्यु॑त् - त॒मा । एक॑स्माद॒क्षरा᳚त् । अ॒क्षरा॒दना᳚प्तम् । अना᳚प्तम् प्रथ॒मम् । प्र॒थ॒मम् प॒दम् । प॒दम् तस्मा᳚त् । तस्मा॒द् यत् । यद् वा॒चः । वा॒चोऽना᳚प्तम् । अना᳚प्त॒म् तत् । तन् म॑नु॒ष्याः᳚ । म॒नु॒ष्या॑ उप॑ । उप॑ जीवन्ति । जी॒व॒न्ति॒ पू॒र्णया᳚ । पू॒र्णया॑ जुहोति । जु॒हो॒ति॒ पू॒र्णः । पू॒र्ण इ॑व । इ॒व॒ हि । हि प्र॒जाप॑तिः । प्र॒जाप॑तिः प्र॒जाप॑तेः । प्र॒जाप॑ति॒रिति॑ प्र॒जा - प॒तिः॒ \newline

\textbf{Jatai Paata} \newline

1. ष॒ड्भिर् दी᳚क्षयति दीक्षयति ष॒ड्भि ष्ष॒ड्भिर् दी᳚क्षयति । \newline
2. ष॒ड्भिरिति॑ षट् - भिः । \newline
3. दी॒क्ष॒य॒ति॒ षट् थ् षड् दी᳚क्षयति दीक्षयति॒ षट् । \newline
4. षड् वै वै षट् थ्षड् वै । \newline
5. वा ऋ॒तव॑ ऋ॒तवो॒ वै वा ऋ॒तवः॑ । \newline
6. ऋ॒तव॑ ऋ॒तुभिर्॑. ऋ॒तुभिर्॑. ऋ॒तव॑ ऋ॒तव॑ ऋ॒तुभिः॑ । \newline
7. ऋ॒तुभि॑ रे॒वैव र्‌तुभिर्॑. ऋ॒तुभि॑ रे॒व । \newline
8. ऋ॒तुभि॒रित्यृ॒तु - भिः॒ । \newline
9. ए॒वैन॑ मेन मे॒वैवैन᳚म् । \newline
10. ए॒न॒म् दी॒क्ष॒य॒ति॒ दी॒क्ष॒य॒ त्ये॒न॒ मे॒न॒म् दी॒क्ष॒य॒ति॒ । \newline
11. दी॒क्ष॒य॒ति॒ स॒प्तभिः॑ स॒प्तभि॑र् दीक्षयति दीक्षयति स॒प्तभिः॑ । \newline
12. स॒प्तभि॑र् दीक्षयति दीक्षयति स॒प्तभिः॑ स॒प्तभि॑र् दीक्षयति । \newline
13. स॒प्तभि॒रिति॑ स॒प्त - भिः॒ । \newline
14. दी॒क्ष॒य॒ति॒ स॒प्त स॒प्त दी᳚क्षयति दीक्षयति स॒प्त । \newline
15. स॒प्त छन्दाꣳ॑सि॒ छन्दाꣳ॑सि स॒प्त स॒प्त छन्दाꣳ॑सि । \newline
16. छन्दाꣳ॑सि॒ छन्दो॑भि॒ श्छन्दो॑भि॒ श्छन्दाꣳ॑सि॒ छन्दाꣳ॑सि॒ छन्दो॑भिः । \newline
17. छन्दो॑भि रे॒वैव छन्दो॑भि॒ श्छन्दो॑भि रे॒व । \newline
18. छन्दो॑भि॒रिति॒ छन्दः॑ - भिः॒ । \newline
19. ए॒वैन॑ मेन मे॒वैवैन᳚म् । \newline
20. ए॒न॒म् दी॒क्ष॒य॒ति॒ दी॒क्ष॒य॒ त्ये॒न॒ मे॒न॒म् दी॒क्ष॒य॒ति॒ । \newline
21. दी॒क्ष॒य॒ति॒ विश्वे॒ विश्वे॑ दीक्षयति दीक्षयति॒ विश्वे᳚ । \newline
22. विश्वे॑ दे॒वस्य॑ दे॒वस्य॒ विश्वे॒ विश्वे॑ दे॒वस्य॑ । \newline
23. दे॒वस्य॑ ने॒तुर् ने॒तुर् दे॒वस्य॑ दे॒वस्य॑ ने॒तुः । \newline
24. ने॒तु रितीति॑ ने॒तुर् ने॒तु रिति॑ । \newline
25. इत्य॑नु॒ष्टुभा॑ ऽनु॒ष्टुभेती त्य॑नु॒ष्टुभा᳚ । \newline
26. अ॒नु॒ष्टुभो᳚ त्त॒मयो᳚ त्त॒मया॑ ऽनु॒ष्टुभा॑ ऽनु॒ष्टुभो᳚ त्त॒मया᳚ । \newline
27. अ॒नु॒ष्टुभेत्य॑नु - स्तुभा᳚ । \newline
28. उ॒त्त॒मया॑ जुहोति जुहो त्युत्त॒मयो᳚ त्त॒मया॑ जुहोति । \newline
29. उ॒त्त॒मयेत्यु॑त् - त॒मया᳚ । \newline
30. जु॒हो॒ति॒ वाग् वाग् जु॑होति जुहोति॒ वाक् । \newline
31. वाग् वै वै वाग् वाग् वै । \newline
32. वा अ॑नु॒ष्टु ब॑नु॒ष्टुब् वै वा अ॑नु॒ष्टुप् । \newline
33. अ॒नु॒ष्टुप् तस्मा॒त् तस्मा॑ दनु॒ष्टु ब॑नु॒ष्टुप् तस्मा᳚त् । \newline
34. अ॒नु॒ष्टुबित्य॑नु - स्तुप् । \newline
35. तस्मा᳚त् प्रा॒णाना᳚म् प्रा॒णाना॒म् तस्मा॒त् तस्मा᳚त् प्रा॒णाना᳚म् । \newline
36. प्रा॒णानां॒ ॅवाग् वाक् प्रा॒णाना᳚म् प्रा॒णानां॒ ॅवाक् । \newline
37. प्रा॒णाना॒मिति॑ प्र - अ॒नाना᳚म् । \newline
38. वागु॑त्त॒मोत्त॒मा वाग् वागु॑त्त॒मा । \newline
39. उ॒त्त॒मैक॑स्मा॒ देक॑स्मा दुत्त॒मोत्त॒ मैक॑स्मात् । \newline
40. उ॒त्त॒मेत्यु॑त् - त॒मा । \newline
41. एक॑स्मा द॒क्षरा॑ द॒क्षरा॒ देक॑स्मा॒ देक॑स्मा द॒क्षरा᳚त् । \newline
42. अ॒क्षरा॒ दना᳚प्त॒ मना᳚प्त म॒क्षरा॑ द॒क्षरा॒ दना᳚प्तम् । \newline
43. अना᳚प्तम् प्रथ॒मम् प्र॑थ॒म मना᳚प्त॒ मना᳚प्तम् प्रथ॒मम् । \newline
44. प्र॒थ॒मम् प॒दम् प॒दम् प्र॑थ॒मम् प्र॑थ॒मम् प॒दम् । \newline
45. प॒दम् तस्मा॒त् तस्मा᳚त् प॒दम् प॒दम् तस्मा᳚त् । \newline
46. तस्मा॒द् यद् यत् तस्मा॒त् तस्मा॒द् यत् । \newline
47. यद् वा॒चो वा॒चो यद् यद् वा॒चः । \newline
48. वा॒चो ऽना᳚प्त॒ मना᳚प्तं ॅवा॒चो वा॒चो ऽना᳚प्तम् । \newline
49. अना᳚प्त॒म् तत् तदना᳚प्त॒ मना᳚प्त॒म् तत् । \newline
50. तन् म॑नु॒ष्या॑ मनु॒ष्या᳚ स्तत् तन् म॑नु॒ष्याः᳚ । \newline
51. म॒नु॒ष्या॑ उपोप॑ मनु॒ष्या॑ मनु॒ष्या॑ उप॑ । \newline
52. उप॑ जीवन्ति जीव॒ न्त्युपोप॑ जीवन्ति । \newline
53. जी॒व॒न्ति॒ पू॒र्णया॑ पू॒र्णया॑ जीवन्ति जीवन्ति पू॒र्णया᳚ । \newline
54. पू॒र्णया॑ जुहोति जुहोति पू॒र्णया॑ पू॒र्णया॑ जुहोति । \newline
55. जु॒हो॒ति॒ पू॒र्णः पू॒र्णो जु॑होति जुहोति पू॒र्णः । \newline
56. पू॒र्ण इ॑वेव पू॒र्णः पू॒र्ण इ॑व । \newline
57. इ॒व॒ हि हीवे॑व॒ हि । \newline
58. हि प्र॒जाप॑तिः प्र॒जाप॑ति॒र्॒. हि हि प्र॒जाप॑तिः । \newline
59. प्र॒जाप॑तिः प्र॒जाप॑तेः प्र॒जाप॑तेः प्र॒जाप॑तिः प्र॒जाप॑तिः प्र॒जाप॑तेः । \newline
60. प्र॒जाप॑ति॒रिति॑ प्र॒जा - प॒तिः॒ । \newline

\textbf{Ghana Paata } \newline

1. ष॒ड्भिर् दी᳚क्षयति दीक्षयति ष॒ड्भि ष्ष॒ड्भिर् दी᳚क्षयति॒ षट् थ्षड् दी᳚क्षयति ष॒ड्भि ष्ष॒ड्भिर् दी᳚क्षयति॒ षट् । \newline
2. ष॒ड्भिरिति॑ षट् - भिः । \newline
3. दी॒क्ष॒य॒ति॒ षट् थ्षड् दी᳚क्षयति दीक्षयति॒ षड् वै वै षड् दी᳚क्षयति दीक्षयति॒ षड् वै । \newline
4. षड् वै वै षट् थ्षड् वा ऋ॒तव॑ ऋ॒तवो॒ वै षट् थ्षड् वा ऋ॒तवः॑ । \newline
5. वा ऋ॒तव॑ ऋ॒तवो॒ वै वा ऋ॒तव॑ ऋ॒तुभिर्॑. ऋ॒तुभिर्॑. ऋ॒तवो॒ वै वा ऋ॒तव॑ ऋ॒तुभिः॑ । \newline
6. ऋ॒तव॑ ऋ॒तुभिर्॑. ऋ॒तुभिर्॑. ऋ॒तव॑ ऋ॒तव॑ ऋ॒तुभि॑ रे॒वैव र्‌तुभिर्॑. ऋ॒तव॑ ऋ॒तव॑ ऋ॒तुभि॑ रे॒व । \newline
7. ऋ॒तुभि॑ रे॒वैव र्‌तुभिर्॑. ऋ॒तुभि॑ रे॒वैन॑ मेन मे॒व र्‌तुभिर्॑. ऋ॒तुभि॑ रे॒वैन᳚म् । \newline
8. ऋ॒तुभि॒रित्यृ॒तु - भिः॒ । \newline
9. ए॒वैन॑ मेन मे॒वैवैन॑म् दीक्षयति दीक्षय त्येन मे॒वैवैन॑म् दीक्षयति । \newline
10. ए॒न॒म् दी॒क्ष॒य॒ति॒ दी॒क्ष॒य॒ त्ये॒न॒ मे॒न॒म् दी॒क्ष॒य॒ति॒ स॒प्तभिः॑ स॒प्तभि॑र् दीक्षयत्येन मेनम् दीक्षयति स॒प्तभिः॑ । \newline
11. दी॒क्ष॒य॒ति॒ स॒प्तभिः॑ स॒प्तभि॑र् दीक्षयति दीक्षयति स॒प्तभि॑र् दीक्षयति दीक्षयति स॒प्तभि॑र् दीक्षयति दीक्षयति स॒प्तभि॑र् दीक्षयति । \newline
12. स॒प्तभि॑र् दीक्षयति दीक्षयति स॒प्तभिः॑ स॒प्तभि॑र् दीक्षयति स॒प्त स॒प्त दी᳚क्षयति स॒प्तभिः॑ स॒प्तभि॑र् दीक्षयति स॒प्त । \newline
13. स॒प्तभि॒रिति॑ स॒प्त - भिः॒ । \newline
14. दी॒क्ष॒य॒ति॒ स॒प्त स॒प्त दी᳚क्षयति दीक्षयति स॒प्त छन्दाꣳ॑सि॒ छन्दाꣳ॑सि स॒प्त दी᳚क्षयति दीक्षयति स॒प्त छन्दाꣳ॑सि । \newline
15. स॒प्त छन्दाꣳ॑सि॒ छन्दाꣳ॑सि स॒प्त स॒प्त छन्दाꣳ॑सि॒ छन्दो॑भि॒ श्छन्दो॑भि॒ श्छन्दाꣳ॑सि स॒प्त स॒प्त छन्दाꣳ॑सि॒ छन्दो॑भिः । \newline
16. छन्दाꣳ॑सि॒ छन्दो॑भि॒ श्छन्दो॑भि॒ श्छन्दाꣳ॑सि॒ छन्दाꣳ॑सि॒ छन्दो॑भि रे॒वैव छन्दो॑भि॒ श्छन्दाꣳ॑सि॒ छन्दाꣳ॑सि॒ छन्दो॑भिरे॒व । \newline
17. छन्दो॑भि रे॒वैव छन्दो॑भि॒ श्छन्दो॑भि रे॒वैन॑ मेन मे॒व छन्दो॑भि॒ श्छन्दो॑भि रे॒वैन᳚म् । \newline
18. छन्दो॑भि॒रिति॒ छन्दः॑ - भिः॒ । \newline
19. ए॒वैन॑ मेन मे॒वैवैन॑म् दीक्षयति दीक्षय त्येन मे॒वैवैन॑म् दीक्षयति । \newline
20. ए॒न॒म् दी॒क्ष॒य॒ति॒ दी॒क्ष॒य॒ त्ये॒न॒ मे॒न॒म् दी॒क्ष॒य॒ति॒ विश्वे॒ विश्वे॑ दीक्षय त्येन मेनम् दीक्षयति॒ विश्वे᳚ । \newline
21. दी॒क्ष॒य॒ति॒ विश्वे॒ विश्वे॑ दीक्षयति दीक्षयति॒ विश्वे॑ दे॒वस्य॑ दे॒वस्य॒ विश्वे॑ दीक्षयति दीक्षयति॒ विश्वे॑ दे॒वस्य॑ । \newline
22. विश्वे॑ दे॒वस्य॑ दे॒वस्य॒ विश्वे॒ विश्वे॑ दे॒वस्य॑ ने॒तुर् ने॒तुर् दे॒वस्य॒ विश्वे॒ विश्वे॑ दे॒वस्य॑ ने॒तुः । \newline
23. दे॒वस्य॑ ने॒तुर् ने॒तुर् दे॒वस्य॑ दे॒वस्य॑ ने॒तु रितीति॑ ने॒तुर् दे॒वस्य॑ दे॒वस्य॑ ने॒तु रिति॑ । \newline
24. ने॒तु रितीति॑ ने॒तुर् ने॒तु रित्य॑नु॒ष्टुभा॑ ऽनु॒ष्टुभेति॑ ने॒तुर् ने॒तु रित्य॑नु॒ष्टुभा᳚ । \newline
25. इत्य॑नु॒ष्टुभा॑ ऽनु॒ष्टुभेती त्य॑नु॒ष्टुभो᳚ त्त॒मयो᳚त्त॒मया॑ ऽनु॒ष्टुभेती त्य॑नु॒ष्टुभो᳚त्त॒मया᳚ । \newline
26. अ॒नु॒ष्टुभो᳚ त्त॒मयो᳚त्त॒मया॑ ऽनु॒ष्टुभा॑ ऽनु॒ष्टुभो᳚ त्त॒मया॑ जुहोति जुहो त्युत्त॒मया॑ ऽनु॒ष्टुभा॑ ऽनु॒ष्टुभो᳚ त्त॒मया॑ जुहोति । \newline
27. अ॒नु॒ष्टुभेत्य॑नु - स्तुभा᳚ । \newline
28. उ॒त्त॒मया॑ जुहोति जुहो त्युत्त॒मयो᳚ त्त॒मया॑ जुहोति॒ वाग् वाग् जु॑हो त्युत्त॒मयो᳚ त्त॒मया॑ जुहोति॒ वाक् । \newline
29. उ॒त्त॒मयेत्यु॑त् - त॒मया᳚ । \newline
30. जु॒हो॒ति॒ वाग् वाग् जु॑होति जुहोति॒ वाग् वै वै वाग् जु॑होति जुहोति॒ वाग् वै । \newline
31. वाग् वै वै वाग् वाग् वा अ॑नु॒ष्टु ब॑नु॒ष्टुब् वै वाग् वाग् वा अ॑नु॒ष्टुप् । \newline
32. वा अ॑नु॒ष्टु ब॑नु॒ष्टुब् वै वा अ॑नु॒ष्टुप् तस्मा॒त् तस्मा॑ दनु॒ष्टुब् वै वा अ॑नु॒ष्टुप् तस्मा᳚त् । \newline
33. अ॒नु॒ष्टुप् तस्मा॒त् तस्मा॑ दनु॒ष्टु ब॑नु॒ष्टुप् तस्मा᳚त् प्रा॒णाना᳚म् प्रा॒णाना॒म् तस्मा॑ दनु॒ष्टु ब॑नु॒ष्टुप् तस्मा᳚त् प्रा॒णाना᳚म् । \newline
34. अ॒नु॒ष्टुबित्य॑नु - स्तुप् । \newline
35. तस्मा᳚त् प्रा॒णाना᳚म् प्रा॒णाना॒म् तस्मा॒त् तस्मा᳚त् प्रा॒णानां॒ ॅवाग् वाक् प्रा॒णाना॒म् तस्मा॒त् तस्मा᳚त् प्रा॒णानां॒ ॅवाक् । \newline
36. प्रा॒णानां॒ ॅवाग् वाक् प्रा॒णाना᳚म् प्रा॒णानां॒ ॅवागु॑त्त॒मोत्त॒मा वाक् प्रा॒णाना᳚म् प्रा॒णानां॒ ॅवागु॑त्त॒मा । \newline
37. प्रा॒णाना॒मिति॑ प्र - अ॒नाना᳚म् । \newline
38. वागु॑त्त॒ मोत्त॒मा वाग् वागु॑त्त॒ मैक॑स्मा॒ देक॑स्मा दुत्त॒मा वाग् वागु॑त्त॒ मैक॑स्मात् । \newline
39. उ॒त्त॒ मैक॑स्मा॒ देक॑स्मा दुत्त॒मोत्त॒ मैक॑स्मा द॒क्षरा॑ द॒क्षरा॒ देक॑स्मा दुत्त॒मोत्त॒ मैक॑स्मा द॒क्षरा᳚त् । \newline
40. उ॒त्त॒मेत्यु॑त् - त॒मा । \newline
41. एक॑स्मा द॒क्षरा॑ द॒क्षरा॒ देक॑स्मा॒ देक॑स्मा द॒क्षरा॒ दना᳚प्त॒ मना᳚प्त म॒क्षरा॒ देक॑स्मा॒ देक॑स्मा द॒क्षरा॒ दना᳚प्तम् । \newline
42. अ॒क्षरा॒ दना᳚प्त॒ मना᳚प्त म॒क्षरा॑ द॒क्षरा॒ दना᳚प्तम् प्रथ॒मम् प्र॑थ॒म मना᳚प्त म॒क्षरा॑ द॒क्षरा॒ दना᳚प्तम् प्रथ॒मम् । \newline
43. अना᳚प्तम् प्रथ॒मम् प्र॑थ॒म मना᳚प्त॒ मना᳚प्तम् प्रथ॒मम् प॒दम् प॒दम् प्र॑थ॒म मना᳚प्त॒ मना᳚प्तम् प्रथ॒मम् प॒दम् । \newline
44. प्र॒थ॒मम् प॒दम् प॒दम् प्र॑थ॒मम् प्र॑थ॒मम् प॒दम् तस्मा॒त् तस्मा᳚त् प॒दम् प्र॑थ॒मम् प्र॑थ॒मम् प॒दम् तस्मा᳚त् । \newline
45. प॒दम् तस्मा॒त् तस्मा᳚त् प॒दम् प॒दम् तस्मा॒द् यद् यत् तस्मा᳚त् प॒दम् प॒दम् तस्मा॒द् यत् । \newline
46. तस्मा॒द् यद् यत् तस्मा॒त् तस्मा॒द् यद् वा॒चो वा॒चो यत् तस्मा॒त् तस्मा॒द् यद् वा॒चः । \newline
47. यद् वा॒चो वा॒चो यद् यद् वा॒चो ऽना᳚प्त॒ मना᳚प्तं ॅवा॒चो यद् यद् वा॒चो ऽना᳚प्तम् । \newline
48. वा॒चो ऽना᳚प्त॒ मना᳚प्तं ॅवा॒चो वा॒चो ऽना᳚प्त॒म् तत् तदना᳚प्तं ॅवा॒चो वा॒चो ऽना᳚प्त॒म् तत् । \newline
49. अना᳚प्त॒म् तत् तदना᳚प्त॒ मना᳚प्त॒म् तन् म॑नु॒ष्या॑ मनु॒ष्या᳚ स्तदना᳚प्त॒ मना᳚प्त॒म् तन् म॑नु॒ष्याः᳚ । \newline
50. तन् म॑नु॒ष्या॑ मनु॒ष्या᳚ स्तत् तन् म॑नु॒ष्या॑ उपोप॑ मनु॒ष्या᳚ स्तत् तन् म॑नु॒ष्या॑ उप॑ । \newline
51. म॒नु॒ष्या॑ उपोप॑ मनु॒ष्या॑ मनु॒ष्या॑ उप॑ जीवन्ति जीव॒न्त्युप॑ मनु॒ष्या॑ मनु॒ष्या॑ उप॑ जीवन्ति । \newline
52. उप॑ जीवन्ति जीव॒ न्त्युपोप॑ जीवन्ति पू॒र्णया॑ पू॒र्णया॑ जीव॒ न्त्युपोप॑ जीवन्ति पू॒र्णया᳚ । \newline
53. जी॒व॒न्ति॒ पू॒र्णया॑ पू॒र्णया॑ जीवन्ति जीवन्ति पू॒र्णया॑ जुहोति जुहोति पू॒र्णया॑ जीवन्ति जीवन्ति पू॒र्णया॑ जुहोति । \newline
54. पू॒र्णया॑ जुहोति जुहोति पू॒र्णया॑ पू॒र्णया॑ जुहोति पू॒र्णः पू॒र्णो जु॑होति पू॒र्णया॑ पू॒र्णया॑ जुहोति पू॒र्णः । \newline
55. जु॒हो॒ति॒ पू॒र्णः पू॒र्णो जु॑होति जुहोति पू॒र्ण इ॑वेव पू॒र्णो जु॑होति जुहोति पू॒र्ण इ॑व । \newline
56. पू॒र्ण इ॑वेव पू॒र्णः पू॒र्ण इ॑व॒ हि हीव॑ पू॒र्णः पू॒र्ण इ॑व॒ हि । \newline
57. इ॒व॒ हि हीवे॑व॒ हि प्र॒जाप॑तिः प्र॒जाप॑ति॒र्॒. हीवे॑व॒ हि प्र॒जाप॑तिः । \newline
58. हि प्र॒जाप॑तिः प्र॒जाप॑ति॒र्॒. हि हि प्र॒जाप॑तिः प्र॒जाप॑तेः प्र॒जाप॑तेः प्र॒जाप॑ति॒र्॒. हि हि प्र॒जाप॑तिः प्र॒जाप॑तेः । \newline
59. प्र॒जाप॑तिः प्र॒जाप॑तेः प्र॒जाप॑तेः प्र॒जाप॑तिः प्र॒जाप॑तिः प्र॒जाप॑ते॒ राप्त्या॒ आप्त्यै᳚ प्र॒जाप॑तेः प्र॒जाप॑तिः प्र॒जाप॑तिः प्र॒जाप॑ते॒ राप्त्यै᳚ । \newline
60. प्र॒जाप॑ति॒रिति॑ प्र॒जा - प॒तिः॒ । \newline
\pagebreak
\markright{ TS 5.1.9.2  \hfill https://www.vedavms.in \hfill}

\section{ TS 5.1.9.2 }

\textbf{TS 5.1.9.2 } \newline
\textbf{Samhita Paata} \newline

प्र॒जाप॑ते॒रापत्यै॒ न्यू॑नया जुहोति॒ न्यू॑ना॒द्धि प्र॒जाप॑तिः प्र॒जा असृ॑जत प्र॒जानाꣳ॒॒ सृष्ट्यै॒ यद॒र्चिषि॑ प्रवृ॒ञ्ज्याद्-भू॒तमव॑ रुन्धीत॒ यदङ्गा॑रेषु भवि॒ष्यदङ्गा॑रेषु॒ प्रवृ॑णक्ति भवि॒ष्य दे॒वाव॑ रुन्धे भवि॒ष्यद्धि भूयो॑ भू॒ताद्-द्वाभ्यां॒ प्रवृ॑णक्ति द्वि॒पाद्-यज॑मानः॒ प्रति॑ष्ठित्यै॒ ब्रह्म॑णा॒ वा ए॒षा यजु॑षा॒ संभृ॑ता॒ यदु॒खा सा यद्भिद्ये॒ताऽऽ*र्ति॒मार्च्छे॒ - [  ] \newline

\textbf{Pada Paata} \newline

प्र॒जाप॑ते॒रिति॑ प्र॒जा - प॒तेः॒ । आप्त्यै᳚ । न्यू॑न॒येति॒ नि - ऊ॒न॒या॒ । जु॒हो॒ति॒ । न्यू॑ना॒दिति॒ नि - ऊ॒ना॒त् । हि । प्र॒जाप॑ति॒रिति॑ प्र॒जा-प॒तिः॒ । प्र॒जा इति॑ प्र - जाः । असृ॑जत । प्र॒जाना॒मिति॑ प्र - जाना᳚म् । सृष्ट्यै᳚ । यत् । अ॒र्चिषि॑ । प्र॒वृ॒ञ्ज्यादिति॑ प्र - वृ॒ञ्ज्यात् । भू॒तम् । अवेति॑ । रु॒न्धी॒त॒ । यत् । अङ्गा॑रेषु । भ॒वि॒ष्यत् । अङ्गा॑रेषु । प्रेति॑ । वृ॒ण॒क्ति॒ । भ॒वि॒ष्यत् । ए॒व । अवेति॑ । रु॒न्धे॒ । भ॒वि॒ष्यत् । हि । भूयः॑ । भू॒तात् । द्वाभ्या᳚म् । प्रेति॑ । वृ॒ण॒क्ति॒ । द्वि॒पादिति॑ द्वि-पात् । यज॑मानः । प्रति॑ष्ठित्या॒ इति॒ प्रति॑ - स्थि॒त्यै॒ । ब्रह्म॑णा । वै । ए॒षा । यजु॑षा । संभृ॒तेति॒ सं - भृ॒ता॒ । यत् । उ॒खा । सा । यत् । भिद्ये॑त । आर्ति᳚म् । एति॑ । ऋ॒च्छे॒त् ।  \newline


\textbf{Krama Paata} \newline

प्र॒जाप॑ते॒राप्त्यै᳚ । प्र॒जाप॑ते॒रिति॑ प्र॒जा - प॒तेः॒ । आप्त्यै॒ न्यू॑नया । न्यू॑नया जुहोति । न्यू॑न॒येति॒ नि - ऊ॒न॒या॒ । जु॒हो॒ति॒ न्यू॑नात् । न्यू॑ना॒द्धि । न्यू॑ना॒दिति॒ नि - ऊ॒ना॒त्॒ । हि प्र॒जाप॑तिः । प्र॒जाप॑तिः प्र॒जाः । प्र॒जाप॑ति॒रिति॑ प्र॒जा - प॒तिः॒ । प्र॒जा असृ॑जत । प्र॒जा इति॑ प्र - जाः । असृ॑जत प्र॒जाना᳚म् । प्र॒जानाꣳ॒॒ सृष्ट्यै᳚ । प्र॒जाना॒मिति॑ प्र - जाना᳚म् । सृष्ट्यै॒ यत् । यद॒र्चिषि॑ । अ॒र्चिषि॑ प्रवृ॒ञ्ज्यात् । प्र॒वृ॒ञ्ज्याद् भू॒तम् । प्र॒वृ॒ञ्ज्यादिति॑ प्र - वृ॒ञ्ज्यात् । भू॒तमव॑ । अव॑ रुन्धीत । रु॒न्धी॒त॒ यत् । यदङ्गा॑रेषु । अङ्गा॑रेषु भवि॒ष्यत् । भ॒वि॒ष्यदङ्गा॑रेषु । अङ्गा॑रेषु॒ प्र । प्र वृ॑णक्ति । वृ॒ण॒क्ति॒ भ॒वि॒ष्यत् । भ॒वि॒ष्यदे॒व । ए॒वाव॑ । अव॑ रुन्धे । रु॒न्धे॒ भ॒वि॒ष्यत् । भ॒वि॒ष्यद्धि । हि भूयः॑ । भूयो॑ भू॒तात् । भू॒ताद् द्वाभ्या᳚म् । द्वाभ्या॒म् प्र । प्र वृ॑णक्ति । वृ॒ण॒क्ति॒ द्वि॒पात् । द्वि॒पाद् यज॑मानः । द्वि॒पादिति॑ द्वि - पात् । यज॑मानः॒ प्रति॑ष्ठित्यै । प्रति॑ष्ठत्यै॒ ब्रह्म॑णा । प्रति॑ष्ठित्या॒ इति॒ प्रति॑ - स्थि॒त्यै॒ । ब्रह्म॑णा॒ वै । वा ए॒षा । ए॒षा यजु॑षा । यजु॑षा॒ सम्भृ॑ता । सम्भृ॑ता॒ यत् । सम्भृ॒तेति॒ सम् - भृ॒ता॒ । यदु॒खा । उ॒खा सा । सा यत् । यद् भिद्ये॑त । भिद्ये॒तार्ति᳚म् । आर्ति॒मा । आर्.च्छे᳚त् । ऋ॒च्छे॒द् यज॑मानः \newline

\textbf{Jatai Paata} \newline

1. प्र॒जाप॑ते॒ राप्त्या॒ आप्त्यै᳚ प्र॒जाप॑तेः प्र॒जाप॑ते॒ राप्त्यै᳚ । \newline
2. प्र॒जाप॑ते॒रिति॑ प्र॒जा - प॒तेः॒ । \newline
3. आप्त्यै॒ न्यू॑न॒या न्यू॑न॒या ऽऽप्त्या॒ आप्त्यै॒ न्यू॑न॒या । \newline
4. न्यू॑न॒या जु॑होति जुहोति॒ न्यू॑न॒या न्यू॑न॒या जु॑होति । \newline
5. न्यू॑न॒येति॒ नि - ऊ॒न॒या॒ । \newline
6. जु॒हो॒ति॒ न्यू॑ना॒न् न्यू॑नाज् जुहोति जुहोति॒ न्यू॑नात् । \newline
7. न्यू॑ना॒द्धि हि न्यू॑ना॒न् न्यू॑ना॒द्धि । \newline
8. न्यू॑ना॒दिति॒ नि - ऊ॒ना॒त् । \newline
9. हि प्र॒जाप॑तिः प्र॒जाप॑ति॒र्॒. हि हि प्र॒जाप॑तिः । \newline
10. प्र॒जाप॑तिः प्र॒जाः प्र॒जाः प्र॒जाप॑तिः प्र॒जाप॑तिः प्र॒जाः । \newline
11. प्र॒जाप॑ति॒रिति॑ प्र॒जा - प॒तिः॒ । \newline
12. प्र॒जा असृ॑ज॒ता सृ॑जत प्र॒जाः प्र॒जा असृ॑जत । \newline
13. प्र॒जा इति॑ प्र - जाः । \newline
14. असृ॑जत प्र॒जाना᳚म् प्र॒जाना॒ मसृ॑ज॒ता सृ॑जत प्र॒जाना᳚म् । \newline
15. प्र॒जानाꣳ॒॒ सृष्ट्यै॒ सृष्ट्यै᳚ प्र॒जाना᳚म् प्र॒जानाꣳ॒॒ सृष्ट्यै᳚ । \newline
16. प्र॒जाना॒मिति॑ प्र - जाना᳚म् । \newline
17. सृष्ट्यै॒ यद् यथ् सृष्ट्यै॒ सृष्ट्यै॒ यत् । \newline
18. यद॒र्चि ष्य॒र्चिषि॒ यद् यद॒र्चिषि॑ । \newline
19. अ॒र्चिषि॑ प्रवृ॒ञ्ज्यात् प्र॑वृ॒ञ्ज्या द॒र्चि ष्य॒र्चिषि॑ प्रवृ॒ञ्ज्यात् । \newline
20. प्र॒वृ॒ञ्ज्याद् भू॒तम् भू॒तम् प्र॑वृ॒ञ्ज्यात् प्र॑वृ॒ञ्ज्याद् भू॒तम् । \newline
21. प्र॒वृ॒ञ्ज्यादिति॑ प्र - वृ॒ञ्ज्यात् । \newline
22. भू॒त मवाव॑ भू॒तम् भू॒त मव॑ । \newline
23. अव॑ रुन्धीत रुन्धी॒ता वाव॑ रुन्धीत । \newline
24. रु॒न्धी॒त॒ यद् यद् रु॑न्धीत रुन्धीत॒ यत् । \newline
25. यदङ्गा॑रे॒ ष्वङ्गा॑रेषु॒ यद् यदङ्गा॑रेषु । \newline
26. अङ्गा॑रेषु भवि॒ष्यद् भ॑वि॒ष्य दङ्गा॑रे॒ ष्वङ्गा॑रेषु भवि॒ष्यत् । \newline
27. भ॒वि॒ष्य दङ्गा॑रे॒ ष्वङ्गा॑रेषु भवि॒ष्यद् भ॑वि॒ष्य दङ्गा॑रेषु । \newline
28. अङ्गा॑रेषु॒ प्र प्राङ्गा॑रे॒ ष्वङ्गा॑रेषु॒ प्र । \newline
29. प्र वृ॑णक्ति वृणक्ति॒ प्र प्र वृ॑णक्ति । \newline
30. वृ॒ण॒क्ति॒ भ॒वि॒ष्यद् भ॑वि॒ष्यद् वृ॑णक्ति वृणक्ति भवि॒ष्यत् । \newline
31. भ॒वि॒ष्य दे॒वैव भ॑वि॒ष्यद् भ॑वि॒ष्य दे॒व । \newline
32. ए॒वावा वै॒वै वाव॑ । \newline
33. अव॑ रुन्धे रु॒न्धे ऽवाव॑ रुन्धे । \newline
34. रु॒न्धे॒ भ॒वि॒ष्यद् भ॑वि॒ष्यद् रु॑न्धे रुन्धे भवि॒ष्यत् । \newline
35. भ॒वि॒ष्यद्धि हि भ॑वि॒ष्यद् भ॑वि॒ष्यद्धि । \newline
36. हि भूयो॒ भूयो॒ हि हि भूयः॑ । \newline
37. भूयो॑ भू॒ताद् भू॒ताद् भूयो॒ भूयो॑ भू॒तात् । \newline
38. भू॒ताद् द्वाभ्या॒म् द्वाभ्या᳚म् भू॒ताद् भू॒ताद् द्वाभ्या᳚म् । \newline
39. द्वाभ्या॒म् प्र प्र द्वाभ्या॒म् द्वाभ्या॒म् प्र । \newline
40. प्र वृ॑णक्ति वृणक्ति॒ प्र प्र वृ॑णक्ति । \newline
41. वृ॒ण॒क्ति॒ द्वि॒पाद् द्वि॒पाद् वृ॑णक्ति वृणक्ति द्वि॒पात् । \newline
42. द्वि॒पाद् यज॑मानो॒ यज॑मानो द्वि॒पाद् द्वि॒पाद् यज॑मानः । \newline
43. द्वि॒पादिति॑ द्वि - पात् । \newline
44. यज॑मानः॒ प्रति॑ष्ठित्यै॒ प्रति॑ष्ठित्यै॒ यज॑मानो॒ यज॑मानः॒ प्रति॑ष्ठित्यै । \newline
45. प्रति॑ष्ठित्यै॒ ब्रह्म॑णा॒ ब्रह्म॑णा॒ प्रति॑ष्ठित्यै॒ प्रति॑ष्ठित्यै॒ ब्रह्म॑णा । \newline
46. प्रति॑ष्ठित्या॒ इति॒ प्रति॑ - स्थि॒त्यै॒ । \newline
47. ब्रह्म॑णा॒ वै वै ब्रह्म॑णा॒ ब्रह्म॑णा॒ वै । \newline
48. वा ए॒षैषा वै वा ए॒षा । \newline
49. ए॒षा यजु॑षा॒ यजु॑ षै॒षैषा यजु॑षा । \newline
50. यजु॑षा॒ संभृ॑ता॒ संभृ॑ता॒ यजु॑षा॒ यजु॑षा॒ संभृ॑ता । \newline
51. संभृ॑ता॒ यद् यथ् संभृ॑ता॒ संभृ॑ता॒ यत् । \newline
52. संभृ॒तेति॒ सं - भृ॒ता॒ । \newline
53. यदु॒खोखा यद् यदु॒खा । \newline
54. उ॒खा सा सोखोखा सा । \newline
55. सा यद् यथ् सा सा यत् । \newline
56. यद् भिद्ये॑त॒ भिद्ये॑त॒ यद् यद् भिद्ये॑त । \newline
57. भिद्ये॒तार्ति॒ मार्ति॒म् भिद्ये॑त॒ भिद्ये॒तार्ति᳚म् । \newline
58. आर्ति॒ मा ऽऽर्ति॒ मार्ति॒ मा । \newline
59. आर्च्छे॑ दृच्छे॒ दार्च्छे᳚त् । \newline
60. ऋ॒च्छे॒द् यज॑मानो॒ यज॑मान ऋच्छे दृच्छे॒द् यज॑मानः । \newline

\textbf{Ghana Paata } \newline

1. प्र॒जाप॑ते॒ राप्त्या॒ आप्त्यै᳚ प्र॒जाप॑तेः प्र॒जाप॑ते॒ राप्त्यै॒ न्यू॑न॒या न्यू॑न॒या ऽऽप्त्यै᳚ प्र॒जाप॑तेः प्र॒जाप॑ते॒ राप्त्यै॒ न्यू॑न॒या । \newline
2. प्र॒जाप॑ते॒रिति॑ प्र॒जा - प॒तेः॒ । \newline
3. आप्त्यै॒ न्यू॑न॒या न्यू॑न॒या ऽऽप्त्या॒ आप्त्यै॒ न्यू॑न॒या जु॑होति जुहोति॒ न्यू॑न॒या ऽऽप्त्या॒ आप्त्यै॒ न्यू॑न॒या जु॑होति । \newline
4. न्यू॑न॒या जु॑होति जुहोति॒ न्यू॑न॒या न्यू॑न॒या जु॑होति॒ न्यू॑ना॒न् न्यू॑नाज् जुहोति॒ न्यू॑न॒या न्यू॑न॒या जु॑होति॒ न्यू॑नात् । \newline
5. न्यू॑न॒येति॒ नि - ऊ॒न॒या॒ । \newline
6. जु॒हो॒ति॒ न्यू॑ना॒न् न्यू॑नाज् जुहोति जुहोति॒ न्यू॑ना॒द्धि हि न्यू॑नाज् जुहोति जुहोति॒ न्यू॑ना॒द्धि । \newline
7. न्यू॑ना॒द्धि हि न्यू॑ना॒न् न्यू॑ना॒द्धि प्र॒जाप॑तिः प्र॒जाप॑ति॒र्॒. हि न्यू॑ना॒न् न्यू॑ना॒द्धि प्र॒जाप॑तिः । \newline
8. न्यू॑ना॒दिति॒ नि - ऊ॒ना॒त् । \newline
9. हि प्र॒जाप॑तिः प्र॒जाप॑ति॒र्॒. हि हि प्र॒जाप॑तिः प्र॒जाः प्र॒जाः प्र॒जाप॑ति॒र्॒. हि हि प्र॒जाप॑तिः प्र॒जाः । \newline
10. प्र॒जाप॑तिः प्र॒जाः प्र॒जाः प्र॒जाप॑तिः प्र॒जाप॑तिः प्र॒जा असृ॑ज॒ता सृ॑जत प्र॒जाः प्र॒जाप॑तिः प्र॒जाप॑तिः प्र॒जा असृ॑जत । \newline
11. प्र॒जाप॑ति॒रिति॑ प्र॒जा - प॒तिः॒ । \newline
12. प्र॒जा असृ॑ज॒ता सृ॑जत प्र॒जाः प्र॒जा असृ॑जत प्र॒जाना᳚म् प्र॒जाना॒ मसृ॑जत प्र॒जाः प्र॒जा असृ॑जत प्र॒जाना᳚म् । \newline
13. प्र॒जा इति॑ प्र - जाः । \newline
14. असृ॑जत प्र॒जाना᳚म् प्र॒जाना॒ मसृ॑ज॒ता सृ॑जत प्र॒जानाꣳ॒॒ सृष्ट्यै॒ सृष्ट्यै᳚ प्र॒जाना॒ मसृ॑ज॒ता सृ॑जत प्र॒जानाꣳ॒॒ सृष्ट्यै᳚ । \newline
15. प्र॒जानाꣳ॒॒ सृष्ट्यै॒ सृष्ट्यै᳚ प्र॒जाना᳚म् प्र॒जानाꣳ॒॒ सृष्ट्यै॒ यद् यथ् सृष्ट्यै᳚ प्र॒जाना᳚म् प्र॒जानाꣳ॒॒ सृष्ट्यै॒ यत् । \newline
16. प्र॒जाना॒मिति॑ प्र - जाना᳚म् । \newline
17. सृष्ट्यै॒ यद् यथ् सृष्ट्यै॒ सृष्ट्यै॒ यद॒र्चि ष्य॒र्चिषि॒ यथ् सृष्ट्यै॒ सृष्ट्यै॒ यद॒र्चिषि॑ । \newline
18. यद॒र्चि ष्य॒र्चिषि॒ यद् यद॒र्चिषि॑ प्रवृ॒ञ्ज्यात् प्र॑वृ॒ञ्ज्या द॒र्चिषि॒ यद् यद॒र्चिषि॑ प्रवृ॒ञ्ज्यात् । \newline
19. अ॒र्चिषि॑ प्रवृ॒ञ्ज्यात् प्र॑वृ॒ञ्ज्या द॒र्चि ष्य॒र्चिषि॑ प्रवृ॒ञ्ज्याद् भू॒तम् भू॒तम् प्र॑वृ॒ञ्ज्या द॒र्चि ष्य॒र्चिषि॑ प्रवृ॒ञ्ज्याद् भू॒तम् । \newline
20. प्र॒वृ॒ञ्ज्याद् भू॒तम् भू॒तम् प्र॑वृ॒ञ्ज्यात् प्र॑वृ॒ञ्ज्याद् भू॒त मवाव॑ भू॒तम् प्र॑वृ॒ञ्ज्यात् प्र॑वृ॒ञ्ज्याद् भू॒त मव॑ । \newline
21. प्र॒वृ॒ञ्ज्यादिति॑ प्र - वृ॒ञ्ज्यात् । \newline
22. भू॒त मवाव॑ भू॒तम् भू॒त मव॑ रुन्धीत रुन्धी॒ताव॑ भू॒तम् भू॒त मव॑ रुन्धीत । \newline
23. अव॑ रुन्धीत रुन्धी॒तावाव॑ रुन्धीत॒ यद् यद् रु॑न्धी॒तावाव॑ रुन्धीत॒ यत् । \newline
24. रु॒न्धी॒त॒ यद् यद् रु॑न्धीत रुन्धीत॒ यदङ्गा॑रे॒ ष्वङ्गा॑रेषु॒ यद् रु॑न्धीत रुन्धीत॒ यदङ्गा॑रेषु । \newline
25. यदङ्गा॑रे॒ ष्वङ्गा॑रेषु॒ यद् यदङ्गा॑रेषु भवि॒ष्यद् भ॑वि॒ष्य दङ्गा॑रेषु॒ यद् यदङ्गा॑रेषु भवि॒ष्यत् । \newline
26. अङ्गा॑रेषु भवि॒ष्यद् भ॑वि॒ष्य दङ्गा॑रे॒ ष्वङ्गा॑रेषु भवि॒ष्य दङ्गा॑रे॒ ष्वङ्गा॑रेषु भवि॒ष्य दङ्गा॑रे॒ ष्वङ्गा॑रेषु भवि॒ष्य दङ्गा॑रेषु । \newline
27. भ॒वि॒ष्य दङ्गा॑रे॒ ष्वङ्गा॑रेषु भवि॒ष्यद् भ॑वि॒ष्य दङ्गा॑रेषु॒ प्र प्राङ्गा॑रेषु भवि॒ष्यद् भ॑वि॒ष्य दङ्गा॑रेषु॒ प्र । \newline
28. अङ्गा॑रेषु॒ प्र प्राङ्गा॑रे॒ ष्वङ्गा॑रेषु॒ प्र वृ॑णक्ति वृणक्ति॒ प्राङ्गा॑रे॒ ष्वङ्गा॑रेषु॒ प्र वृ॑णक्ति । \newline
29. प्र वृ॑णक्ति वृणक्ति॒ प्र प्र वृ॑णक्ति भवि॒ष्यद् भ॑वि॒ष्यद् वृ॑णक्ति॒ प्र प्र वृ॑णक्ति भवि॒ष्यत् । \newline
30. वृ॒ण॒क्ति॒ भ॒वि॒ष्यद् भ॑वि॒ष्यद् वृ॑णक्ति वृणक्ति भवि॒ष्य दे॒वैव भ॑वि॒ष्यद् वृ॑णक्ति वृणक्ति भवि॒ष्यदे॒व । \newline
31. भ॒वि॒ष्य दे॒वैव भ॑वि॒ष्यद् भ॑वि॒ष्य दे॒वावा वै॒व भ॑वि॒ष्यद् भ॑वि॒ष्य दे॒वाव॑ । \newline
32. ए॒वावा वै॒वै वाव॑ रुन्धे रु॒न्धे ऽवै॒वै वाव॑ रुन्धे । \newline
33. अव॑ रुन्धे रु॒न्धे ऽवाव॑ रुन्धे भवि॒ष्यद् भ॑वि॒ष्यद् रु॒न्धे ऽवाव॑ रुन्धे भवि॒ष्यत् । \newline
34. रु॒न्धे॒ भ॒वि॒ष्यद् भ॑वि॒ष्यद् रु॑न्धे रुन्धे भवि॒ष्यद्धि हि भ॑वि॒ष्यद् रु॑न्धे रुन्धे भवि॒ष्यद्धि । \newline
35. भ॒वि॒ष्यद्धि हि भ॑वि॒ष्यद् भ॑वि॒ष्यद्धि भूयो॒ भूयो॒ हि भ॑वि॒ष्यद् भ॑वि॒ष्यद्धि भूयः॑ । \newline
36. हि भूयो॒ भूयो॒ हि हि भूयो॑ भू॒ताद् भू॒ताद् भूयो॒ हि हि भूयो॑ भू॒तात् । \newline
37. भूयो॑ भू॒ताद् भू॒ताद् भूयो॒ भूयो॑ भू॒ताद् द्वाभ्या॒म् द्वाभ्या᳚म् भू॒ताद् भूयो॒ भूयो॑ भू॒ताद् द्वाभ्या᳚म् । \newline
38. भू॒ताद् द्वाभ्या॒म् द्वाभ्या᳚म् भू॒ताद् भू॒ताद् द्वाभ्या॒म् प्र प्र द्वाभ्या᳚म् भू॒ताद् भू॒ताद् द्वाभ्या॒म् प्र । \newline
39. द्वाभ्या॒म् प्र प्र द्वाभ्या॒म् द्वाभ्या॒म् प्र वृ॑णक्ति वृणक्ति॒ प्र द्वाभ्या॒म् द्वाभ्या॒म् प्र वृ॑णक्ति । \newline
40. प्र वृ॑णक्ति वृणक्ति॒ प्र प्र वृ॑णक्ति द्वि॒पाद् द्वि॒पाद् वृ॑णक्ति॒ प्र प्र वृ॑णक्ति द्वि॒पात् । \newline
41. वृ॒ण॒क्ति॒ द्वि॒पाद् द्वि॒पाद् वृ॑णक्ति वृणक्ति द्वि॒पाद् यज॑मानो॒ यज॑मानो द्वि॒पाद् वृ॑णक्ति वृणक्ति द्वि॒पाद् यज॑मानः । \newline
42. द्वि॒पाद् यज॑मानो॒ यज॑मानो द्वि॒पाद् द्वि॒पाद् यज॑मानः॒ प्रति॑ष्ठित्यै॒ प्रति॑ष्ठित्यै॒ यज॑मानो द्वि॒पाद् द्वि॒पाद् यज॑मानः॒ प्रति॑ष्ठित्यै । \newline
43. द्वि॒पादिति॑ द्वि - पात् । \newline
44. यज॑मानः॒ प्रति॑ष्ठित्यै॒ प्रति॑ष्ठित्यै॒ यज॑मानो॒ यज॑मानः॒ प्रति॑ष्ठित्यै॒ ब्रह्म॑णा॒ ब्रह्म॑णा॒ प्रति॑ष्ठित्यै॒ यज॑मानो॒ यज॑मानः॒ प्रति॑ष्ठित्यै॒ ब्रह्म॑णा । \newline
45. प्रति॑ष्ठित्यै॒ ब्रह्म॑णा॒ ब्रह्म॑णा॒ प्रति॑ष्ठित्यै॒ प्रति॑ष्ठित्यै॒ ब्रह्म॑णा॒ वै वै ब्रह्म॑णा॒ प्रति॑ष्ठित्यै॒ प्रति॑ष्ठित्यै॒ ब्रह्म॑णा॒ वै । \newline
46. प्रति॑ष्ठित्या॒ इति॒ प्रति॑ - स्थि॒त्यै॒ । \newline
47. ब्रह्म॑णा॒ वै वै ब्रह्म॑णा॒ ब्रह्म॑णा॒ वा ए॒षैषा वै ब्रह्म॑णा॒ ब्रह्म॑णा॒ वा ए॒षा । \newline
48. वा ए॒षैषा वै वा ए॒षा यजु॑षा॒ यजु॑षै॒षा वै वा ए॒षा यजु॑षा । \newline
49. ए॒षा यजु॑षा॒ यजु॑षै॒षैषा यजु॑षा॒ संभृ॑ता॒ संभृ॑ता॒ यजु॑षै॒षैषा यजु॑षा॒ संभृ॑ता । \newline
50. यजु॑षा॒ संभृ॑ता॒ संभृ॑ता॒ यजु॑षा॒ यजु॑षा॒ संभृ॑ता॒ यद् यथ् संभृ॑ता॒ यजु॑षा॒ यजु॑षा॒ संभृ॑ता॒ यत् । \newline
51. संभृ॑ता॒ यद् यथ् संभृ॑ता॒ संभृ॑ता॒ यदु॒खोखा यथ् संभृ॑ता॒ संभृ॑ता॒ यदु॒खा । \newline
52. संभृ॒तेति॒ सं - भृ॒ता॒ । \newline
53. यदु॒खोखा यद् यदु॒खा सा सोखा यद् यदु॒खा सा । \newline
54. उ॒खा सा सोखोखा सा यद् यथ् सोखोखा सा यत् । \newline
55. सा यद् यथ् सा सा यद् भिद्ये॑त॒ भिद्ये॑त॒ यथ् सा सा यद् भिद्ये॑त । \newline
56. यद् भिद्ये॑त॒ भिद्ये॑त॒ यद् यद् भिद्ये॒तार्ति॒ मार्ति॒म् भिद्ये॑त॒ यद् यद् भिद्ये॒तार्ति᳚म् । \newline
57. भिद्ये॒तार्ति॒ मार्ति॒म् भिद्ये॑त॒ भिद्ये॒तार्ति॒ मा ऽऽर्ति॒म् भिद्ये॑त॒ भिद्ये॒तार्ति॒ मा । \newline
58. आर्ति॒ मा ऽऽर्ति॒ मार्ति॒ मार्च्छे॑ दृच्छे॒ दाऽऽर्ति॒ मार्ति॒ मार्च्छे᳚त् । \newline
59. आर्च्छे॑ दृच्छे॒ दार्च्छे॒द् यज॑मानो॒ यज॑मान ऋच्छे॒ दार्च्छे॒द् यज॑मानः । \newline
60. ऋ॒च्छे॒द् यज॑मानो॒ यज॑मान ऋच्छे दृच्छे॒द् यज॑मानो ह॒न्येत॑ ह॒न्येत॒ यज॑मान ऋच्छे दृच्छे॒द् यज॑मानो ह॒न्येत॑ । \newline
\pagebreak
\markright{ TS 5.1.9.3  \hfill https://www.vedavms.in \hfill}

\section{ TS 5.1.9.3 }

\textbf{TS 5.1.9.3 } \newline
\textbf{Samhita Paata} \newline

-द्यज॑मानो ह॒न्येता᳚ऽस्य य॒ज्ञो मित्रै॒तामु॒खां त॒पेत्या॑ह॒ ब्रह्म॒ वै मि॒त्रो ब्रह्म॑न्ने॒वैनां॒ प्रति॑ष्ठापयति॒ नाऽऽर्ति॒मार्च्छ॑ति॒ यज॑मानो॒ नास्य॑ य॒ज्ञो ह॑न्यते॒ यदि॒ भिद्ये॑त॒ तैरे॒व क॒पालैः॒ सꣳ सृ॑जे॒थ् सैव ततः॒ प्राय॑श्चित्ति॒र्यो ग॒तश्रीः॒ स्यान्म॑थि॒त्वा तस्याव॑ दद्ध्याद्-भू॒तो वा ए॒ष स स्वां - [  ] \newline

\textbf{Pada Paata} \newline

यज॑मानः । ह॒न्येत॑ । अ॒स्य॒ । य॒ज्ञ्ः । मित्र॑ । ए॒ताम् । उ॒खाम् । त॒प॒ । इति॑ । आ॒ह॒ । ब्रह्म॑ । वै । मि॒त्रः । ब्रह्मन्न्॑ । ए॒व । ए॒ना॒म् । प्रतीति॑ । स्था॒प॒य॒ति॒ । न । आर्ति᳚म् । एति॑ । ऋ॒च्छ॒ति॒ । यज॑मानः । न । अ॒स्य॒ । य॒ज्ञ्ः । ह॒न्य॒ते॒ । यदि॑ । भिद्ये॑त । तैः । ए॒व । क॒पालैः᳚ । समिति॑ । सृ॒जे॒त् । सा । ए॒व । ततः॑ । प्राय॑श्चित्तिः । यः । ग॒तश्री॒रिति॑ ग॒त - श्रीः॒ । स्यात् । म॒थि॒त्वा । तस्य॑ । अवेति॑ । द॒द्ध्या॒त् । भू॒तः । वै । ए॒षः । सः । स्वाम् ।  \newline


\textbf{Krama Paata} \newline

यज॑मानो ह॒न्येत॑ । ह॒न्येता᳚स्य । अ॒स्य॒ य॒ज्ञ्ः । य॒ज्ञो मित्र॑ । मित्रै॒ताम् । ए॒तामु॒खाम् । उ॒खाम् त॑प । त॒पेति॑ । इत्या॑ह । आ॒ह॒ ब्रह्म॑ । ब्रह्म॒ वै । वै मि॒त्रः । मि॒त्रो ब्रह्मन्न्॑ । ब्रह्म॑न्नै॒व । ए॒वैना᳚म् । ए॒ना॒म् प्रति॑ । प्रति॑ ष्ठापयति । स्था॒प॒य॒ति॒ न । नार्ति᳚म् । आर्ति॒मा । आर्च्छ॑ति । ऋ॒च्छ॒ति॒ यज॑मानः । यज॑मानो॒ न । नास्य॑ । अ॒स्य॒ य॒ज्ञ्ः । य॒ज्ञो ह॑न्यते । ह॒न्य॒ते॒ यदि॑ । यदि॒ भिद्ये॑त । भिद्ये॑त॒ तैः । तैरे॒व । ए॒व क॒पालैः᳚ । क॒पालैः॒ सम् । सꣳ सृ॑जेत् । सृ॒जे॒थ् सा । सैव । ए॒व ततः॑ । ततः॒ प्राय॑श्चितिः । प्राय॑श्चिति॒र् यः । यो ग॒तश्रीः᳚ । ग॒तश्रीः॒ स्यात् । ग॒तश्री॒रिति॑ ग॒त - श्रीः॒ । स्यान् म॑थि॒त्वा । म॒थि॒त्वा तस्य॑ । तस्याव॑ । अव॑ दद्ध्यात् । द॒द्ध्या॒द् भू॒तः । भू॒तो वै । वा ए॒षः । ए॒ष सः । स स्वाम् । स्वाम् दे॒वता᳚म् \newline

\textbf{Jatai Paata} \newline

1. यज॑मानो ह॒न्येत॑ ह॒न्येत॒ यज॑मानो॒ यज॑मानो ह॒न्येत॑ । \newline
2. ह॒न्येता᳚ स्यास्य ह॒न्येत॑ ह॒न्येता᳚स्य । \newline
3. अ॒स्य॒ य॒ज्ञो य॒ज्ञो᳚ ऽस्यास्य य॒ज्ञ्ः । \newline
4. य॒ज्ञो मित्र॒ मित्र॑ य॒ज्ञो य॒ज्ञो मित्र॑ । \newline
5. मित्रै॒ता मे॒ताम् मित्र॒ मित्रै॒ताम् । \newline
6. ए॒ता मु॒खा मु॒खा मे॒ता मे॒ता मु॒खाम् । \newline
7. उ॒खाम् त॑प तपो॒खा मु॒खाम् त॑प । \newline
8. त॒पे तीति॑ तप त॒पेति॑ । \newline
9. इत्या॑हा॒हे तीत्या॑ह । \newline
10. आ॒ह॒ ब्रह्म॒ ब्रह्मा॑हाह॒ ब्रह्म॑ । \newline
11. ब्रह्म॒ वै वै ब्रह्म॒ ब्रह्म॒ वै । \newline
12. वै मि॒त्रो मि॒त्रो वै वै मि॒त्रः । \newline
13. मि॒त्रो ब्रह्म॒न् ब्रह्म॑न् मि॒त्रो मि॒त्रो ब्रह्मन्न्॑ । \newline
14. ब्रह्म॑न् ने॒वैव ब्रह्म॒न् ब्रह्म॑न् ने॒व । \newline
15. ए॒वैना॑ मेना मे॒वैवैना᳚म् । \newline
16. ए॒ना॒म् प्रति॒ प्रत्ये॑ना मेना॒म् प्रति॑ । \newline
17. प्रति॑ ष्ठापयति स्थापयति॒ प्रति॒ प्रति॑ ष्ठापयति । \newline
18. स्था॒प॒य॒ति॒ न न स्था॑पयति स्थापयति॒ न । \newline
19. नार्ति॒ मार्ति॒म् न नार्ति᳚म् । \newline
20. आर्ति॒ मा ऽऽर्ति॒ मार्ति॒ मा । \newline
21. आर्च्छ॑ त्यृच्छ त्यार्च्छति । \newline
22. ऋ॒च्छ॒ति॒ यज॑मानो॒ यज॑मान ऋच्छ त्यृच्छति॒ यज॑मानः । \newline
23. यज॑मानो॒ न न यज॑मानो॒ यज॑मानो॒ न । \newline
24. नास्या᳚स्य॒ न नास्य॑ । \newline
25. अ॒स्य॒ य॒ज्ञो य॒ज्ञो᳚ ऽस्यास्य य॒ज्ञ्ः । \newline
26. य॒ज्ञो ह॑न्यते हन्यते य॒ज्ञो य॒ज्ञो ह॑न्यते । \newline
27. ह॒न्य॒ते॒ यदि॒ यदि॑ हन्यते हन्यते॒ यदि॑ । \newline
28. यदि॒ भिद्ये॑त॒ भिद्ये॑त॒ यदि॒ यदि॒ भिद्ये॑त । \newline
29. भिद्ये॑त॒ तै स्तैर् भिद्ये॑त॒ भिद्ये॑त॒ तैः । \newline
30. तै रे॒वैव तै स्तै रे॒व । \newline
31. ए॒व क॒पालैः᳚ क॒पालै॑ रे॒वैव क॒पालैः᳚ । \newline
32. क॒पालैः॒ सꣳ सम् क॒पालैः᳚ क॒पालैः॒ सम् । \newline
33. सꣳ सृ॑जेथ् सृजे॒थ् सꣳ सꣳ सृ॑जेत् । \newline
34. सृ॒जे॒थ् सा सा सृ॑जेथ् सृजे॒थ् सा । \newline
35. सैवैव सा सैव । \newline
36. ए॒व तत॒ स्तत॑ ए॒वैव ततः॑ । \newline
37. ततः॒ प्राय॑श्चित्तिः॒ प्राय॑श्चित्ति॒ स्तत॒ स्ततः॒ प्राय॑श्चित्तिः । \newline
38. प्राय॑श्चित्ति॒र् यो यः प्राय॑श्चित्तिः॒ प्राय॑श्चित्ति॒र् यः । \newline
39. यो ग॒तश्री᳚र् ग॒तश्री॒र् यो यो ग॒तश्रीः᳚ । \newline
40. ग॒तश्रीः॒ स्याथ् स्याद् ग॒तश्री᳚र् ग॒तश्रीः॒ स्यात् । \newline
41. ग॒तश्री॒रिति॑ ग॒त - श्रीः॒ । \newline
42. स्यान् म॑थि॒त्वा म॑थि॒त्वा स्याथ् स्यान् म॑थि॒त्वा । \newline
43. म॒थि॒त्वा तस्य॒ तस्य॑ मथि॒त्वा म॑थि॒त्वा तस्य॑ । \newline
44. तस्या वाव॒ तस्य॒ तस्याव॑ । \newline
45. अव॑ दद्ध्याद् दद्ध्या॒ दवाव॑ दद्ध्यात् । \newline
46. द॒द्ध्या॒द् भू॒तो भू॒तो द॑द्ध्याद् दद्ध्याद् भू॒तः । \newline
47. भू॒तो वै वै भू॒तो भू॒तो वै । \newline
48. वा ए॒ष ए॒ष वै वा ए॒षः । \newline
49. ए॒ष स स ए॒ष ए॒ष सः । \newline
50. स स्वाꣳ स्वाꣳ स स स्वाम् । \newline
51. स्वाम् दे॒वता᳚म् दे॒वताꣳ॒॒ स्वाꣳ स्वाम् दे॒वता᳚म् । \newline

\textbf{Ghana Paata } \newline

1. यज॑मानो ह॒न्येत॑ ह॒न्येत॒ यज॑मानो॒ यज॑मानो ह॒न्येता᳚ स्यास्य ह॒न्येत॒ यज॑मानो॒ यज॑मानो ह॒न्येता᳚स्य । \newline
2. ह॒न्येता᳚ स्यास्य ह॒न्येत॑ ह॒न्येता᳚स्य य॒ज्ञो य॒ज्ञो᳚ ऽस्य ह॒न्येत॑ ह॒न्येता᳚स्य य॒ज्ञ्ः । \newline
3. अ॒स्य॒ य॒ज्ञो य॒ज्ञो᳚ ऽस्यास्य य॒ज्ञो मित्र॒ मित्र॑ य॒ज्ञो᳚ ऽस्यास्य य॒ज्ञो मित्र॑ । \newline
4. य॒ज्ञो मित्र॒ मित्र॑ य॒ज्ञो य॒ज्ञो मित्रै॒ता मे॒ताम् मित्र॑ य॒ज्ञो य॒ज्ञो मित्रै॒ताम् । \newline
5. मित्रै॒ता मे॒ताम् मित्र॒ मित्रै॒ता मु॒खा मु॒खा मे॒ताम् मित्र॒ मित्रै॒ता मु॒खाम् । \newline
6. ए॒ता मु॒खा मु॒खा मे॒ता मे॒ता मु॒खाम् त॑प तपो॒खा मे॒ता मे॒ता मु॒खाम् त॑प । \newline
7. उ॒खाम् त॑प तपो॒खा मु॒खाम् त॒पे तीति॑ तपो॒खा मु॒खाम् त॒पेति॑ । \newline
8. त॒पे तीति॑ तप त॒पे त्या॑हा॒हेति॑ तप त॒पे त्या॑ह । \newline
9. इत्या॑हा॒हे तीत्या॑ह॒ ब्रह्म॒ ब्रह्मा॒हे तीत्या॑ह॒ ब्रह्म॑ । \newline
10. आ॒ह॒ ब्रह्म॒ ब्रह्मा॑हाह॒ ब्रह्म॒ वै वै ब्रह्मा॑हाह॒ ब्रह्म॒ वै । \newline
11. ब्रह्म॒ वै वै ब्रह्म॒ ब्रह्म॒ वै मि॒त्रो मि॒त्रो वै ब्रह्म॒ ब्रह्म॒ वै मि॒त्रः । \newline
12. वै मि॒त्रो मि॒त्रो वै वै मि॒त्रो ब्रह्म॒न् ब्रह्म॑न् मि॒त्रो वै वै मि॒त्रो ब्रह्मन्न्॑ । \newline
13. मि॒त्रो ब्रह्म॒न् ब्रह्म॑न् मि॒त्रो मि॒त्रो ब्रह्म॑न् ने॒वैव ब्रह्म॑न् मि॒त्रो मि॒त्रो ब्रह्म॑न् ने॒व । \newline
14. ब्रह्म॑न् ने॒वैव ब्रह्म॒न् ब्रह्म॑न् ने॒वैना॑ मेना मे॒व ब्रह्म॒न् ब्रह्म॑न् ने॒वैना᳚म् । \newline
15. ए॒वैना॑ मेना मे॒वैवैना॒म् प्रति॒ प्रत्ये॑ना मे॒वैवैना॒म् प्रति॑ । \newline
16. ए॒ना॒म् प्रति॒ प्रत्ये॑ना मेना॒म् प्रति॑ ष्ठापयति स्थापयति॒ प्रत्ये॑ना मेना॒म् प्रति॑ ष्ठापयति । \newline
17. प्रति॑ ष्ठापयति स्थापयति॒ प्रति॒ प्रति॑ ष्ठापयति॒ न न स्था॑पयति॒ प्रति॒ प्रति॑ ष्ठापयति॒ न । \newline
18. स्था॒प॒य॒ति॒ न न स्था॑पयति स्थापयति॒ नार्ति॒ मार्ति॒म् न स्था॑पयति स्थापयति॒ नार्ति᳚म् । \newline
19. नार्ति॒ मार्ति॒म् न नार्ति॒ मा ऽऽर्ति॒म् न नार्ति॒ मा । \newline
20. आर्ति॒ मा ऽऽर्ति॒ मार्ति॒ मार्च्छ॑ त्यृच्छ॒त्या ऽऽर्ति॒ मार्ति॒ मार्च्छ॑ति । \newline
21. आर्च्छ॑ त्यृच्छ त्यार्च्छति॒ यज॑मानो॒ यज॑मान ऋच्छ त्यार्च्छति॒ यज॑मानः । \newline
22. ऋ॒च्छ॒ति॒ यज॑मानो॒ यज॑मान ऋच्छ त्यृच्छति॒ यज॑मानो॒ न न यज॑मान ऋच्छ त्यृच्छति॒ यज॑मानो॒ न । \newline
23. यज॑मानो॒ न न यज॑मानो॒ यज॑मानो॒ नास्या᳚स्य॒ न यज॑मानो॒ यज॑मानो॒ नास्य॑ । \newline
24. नास्या᳚स्य॒ न नास्य॑ य॒ज्ञो य॒ज्ञो᳚ ऽस्य॒ न नास्य॑ य॒ज्ञ्ः । \newline
25. अ॒स्य॒ य॒ज्ञो य॒ज्ञो᳚ ऽस्यास्य य॒ज्ञो ह॑न्यते हन्यते य॒ज्ञो᳚ ऽस्यास्य य॒ज्ञो ह॑न्यते । \newline
26. य॒ज्ञो ह॑न्यते हन्यते य॒ज्ञो य॒ज्ञो ह॑न्यते॒ यदि॒ यदि॑ हन्यते य॒ज्ञो य॒ज्ञो ह॑न्यते॒ यदि॑ । \newline
27. ह॒न्य॒ते॒ यदि॒ यदि॑ हन्यते हन्यते॒ यदि॒ भिद्ये॑त॒ भिद्ये॑त॒ यदि॑ हन्यते हन्यते॒ यदि॒ भिद्ये॑त । \newline
28. यदि॒ भिद्ये॑त॒ भिद्ये॑त॒ यदि॒ यदि॒ भिद्ये॑त॒ तै स्तैर् भिद्ये॑त॒ यदि॒ यदि॒ भिद्ये॑त॒ तैः । \newline
29. भिद्ये॑त॒ तै स्तैर् भिद्ये॑त॒ भिद्ये॑त॒ तै रे॒वैव तैर् भिद्ये॑त॒ भिद्ये॑त॒ तै रे॒व । \newline
30. तै रे॒वैव तै स्तै रे॒व क॒पालैः᳚ क॒पालै॑ रे॒व तै स्तै रे॒व क॒पालैः᳚ । \newline
31. ए॒व क॒पालैः᳚ क॒पालै॑ रे॒वैव क॒पालैः॒ सꣳ सम् क॒पालै॑ रे॒वैव क॒पालैः॒ सम् । \newline
32. क॒पालैः॒ सꣳ सम् क॒पालैः᳚ क॒पालैः॒ सꣳ सृ॑जेथ् सृजे॒थ् सम् क॒पालैः᳚ क॒पालैः॒ सꣳ सृ॑जेत् । \newline
33. सꣳ सृ॑जेथ् सृजे॒थ् सꣳ सꣳ सृ॑जे॒थ् सा सा सृ॑जे॒थ् सꣳ सꣳ सृ॑जे॒थ् सा । \newline
34. सृ॒जे॒थ् सा सा सृ॑जेथ् सृजे॒थ् सैवैव सा सृ॑जेथ् सृजे॒थ् सैव । \newline
35. सैवैव सा सैव तत॒ स्तत॑ ए॒व सा सैव ततः॑ । \newline
36. ए॒व तत॒ स्तत॑ ए॒वैव ततः॒ प्राय॑श्चित्तिः॒ प्राय॑श्चित्ति॒ स्तत॑ ए॒वैव ततः॒ प्राय॑श्चित्तिः । \newline
37. ततः॒ प्राय॑श्चित्तिः॒ प्राय॑श्चित्ति॒ स्तत॒ स्ततः॒ प्राय॑श्चित्ति॒र् यो यः प्राय॑श्चित्ति॒ स्तत॒ स्ततः॒ प्राय॑श्चित्ति॒र् यः । \newline
38. प्राय॑श्चित्ति॒र् यो यः प्राय॑श्चित्तिः॒ प्राय॑श्चित्ति॒र् यो ग॒तश्री᳚र् ग॒तश्री॒र् यः प्राय॑श्चित्तिः॒ प्राय॑श्चित्ति॒र् यो ग॒तश्रीः᳚ । \newline
39. यो ग॒तश्री᳚र् ग॒तश्री॒र् यो यो ग॒तश्रीः॒ स्याथ् स्याद् ग॒तश्री॒र् यो यो ग॒तश्रीः॒ स्यात् । \newline
40. ग॒तश्रीः॒ स्याथ् स्याद् ग॒तश्री᳚र् ग॒तश्रीः॒ स्यान् म॑थि॒त्वा म॑थि॒त्वा स्याद् ग॒तश्री᳚र् ग॒तश्रीः॒ स्यान् म॑थि॒त्वा । \newline
41. ग॒तश्री॒रिति॑ ग॒त - श्रीः॒ । \newline
42. स्यान् म॑थि॒त्वा म॑थि॒त्वा स्याथ् स्यान् म॑थि॒त्वा तस्य॒ तस्य॑ मथि॒त्वा स्याथ् स्यान् म॑थि॒त्वा तस्य॑ । \newline
43. म॒थि॒त्वा तस्य॒ तस्य॑ मथि॒त्वा म॑थि॒त्वा तस्या वाव॒ तस्य॑ मथि॒त्वा म॑थि॒त्वा तस्याव॑ । \newline
44. तस्यावाव॒ तस्य॒ तस्याव॑ दद्ध्याद् दद्ध्या॒दव॒ तस्य॒ तस्याव॑ दद्ध्यात् । \newline
45. अव॑ दद्ध्याद् दद्ध्या॒द् अवाव॑ दद्ध्याद् भू॒तो भू॒तो द॑द्ध्या॒ दवाव॑ दद्ध्याद् भू॒तः । \newline
46. द॒द्ध्या॒द् भू॒तो भू॒तो द॑द्ध्याद् दद्ध्याद् भू॒तो वै वै भू॒तो द॑द्ध्याद् दद्ध्याद् भू॒तो वै । \newline
47. भू॒तो वै वै भू॒तो भू॒तो वा ए॒ष ए॒ष वै भू॒तो भू॒तो वा ए॒षः । \newline
48. वा ए॒ष ए॒ष वै वा ए॒ष स स ए॒ष वै वा ए॒ष सः । \newline
49. ए॒ष स स ए॒ष ए॒ष स स्वाꣳ स्वाꣳ स ए॒ष ए॒ष स स्वाम् । \newline
50. स स्वाꣳ स्वाꣳ स स स्वाम् दे॒वता᳚म् दे॒वताꣳ॒॒ स्वाꣳ स स स्वाम् दे॒वता᳚म् । \newline
51. स्वाम् दे॒वता᳚म् दे॒वताꣳ॒॒ स्वाꣳ स्वाम् दे॒वता॒ मुपोप॑ दे॒वताꣳ॒॒ स्वाꣳ स्वाम् दे॒वता॒ मुप॑ । \newline
\pagebreak
\markright{ TS 5.1.9.4  \hfill https://www.vedavms.in \hfill}

\section{ TS 5.1.9.4 }

\textbf{TS 5.1.9.4 } \newline
\textbf{Samhita Paata} \newline

दे॒वता॒मुपै॑ति॒ यो भूति॑कामः॒ स्याद्य उ॒खायै॑ स॒भंवे॒थ् स ए॒व तस्य॑ स्या॒दतो॒ ह्ये॑ष स॒भंव॑त्ये॒ष वै स्व॑य॒भूंर्नाम॒ भव॑त्ये॒व यं का॒मये॑त॒ भ्रातृ॑व्यमस्मै जनयेय॒मित्य॒-न्यत॒स्तस्या॒-ऽऽहृत्याऽव॑ दद्ध्याथ् सा॒क्षादे॒वास्मै॒ भ्रातृ॑व्यं जनयत्यम्ब॒रीषा॒दन्न॑ काम॒स्याव॑ दद्ध्यादंब॒रीषे॒ वा अन्नं॑ भ्रियते॒ सयो᳚न्ये॒वान्न॒ - [  ] \newline

\textbf{Pada Paata} \newline

दे॒वता᳚म् । उपेति॑ । ए॒ति॒ । यः । भूति॑काम॒ इति॒ भूति॑ - का॒मः॒ । स्यात् । यः । उ॒खायै᳚ । स॒भंवे॒दिति॑ सं - भवे᳚त् । सः । ए॒व । तस्य॑ । स्या॒त् । अतः॑ । हि । ए॒षः । स॒भंव॒तीति॑ सं-भव॑ति । ए॒षः । वै । स्व॒य॒भूंरिति॑ स्वयं-भूः । नाम॑ । भव॑ति । ए॒व । यम् । का॒मये॑त । भ्रातृ॑व्यम् । अ॒स्मै॒ । ज॒न॒ये॒य॒म् । इति॑ । अ॒न्यतः॑ । तस्य॑ । आ॒हृत्येत्या᳚ - हृत्य॑ । अवेति॑ । द॒द्ध्या॒त् । सा॒क्षादिति॑ स - अ॒क्षात् । ए॒व । अ॒स्मै॒ । भ्रातृ॑व्यम् । ज॒न॒य॒ति॒ । अ॒बं॒रीषा᳚त् । अन्न॑काम॒स्येत्यन्न॑ - का॒म॒स्य॒ । अवेति॑ । द॒द्ध्या॒त् । अ॒बं॒रीषे᳚ । वै । अन्न᳚म् । भ्रि॒य॒ते॒ । सयो॒नीति॒ स - यो॒नि॒ । ए॒व । अन्न᳚म् ।  \newline


\textbf{Krama Paata} \newline

दे॒वता॒मुप॑ । उपै॑ति । ए॒ति॒ यः । यो भूति॑कामः । भूति॑कामः॒ स्यात् । भूति॑काम॒ इति॒ भूति॑ - का॒मः॒ । स्याद् यः । य उ॒खायै᳚ । उ॒खायै॑ स॒म्भवे᳚त् । स॒म्भवे॒थ् सः । स॒म्भवे॒दिति॑ सम् - भवे᳚त् । स ए॒व । ए॒व तस्य॑ । तस्य॑ स्यात् । स्या॒दतः॑ । अतो॒ हि । ह्ये॑षः । ए॒ष स॒म्भव॑ति । स॒म्भव॑त्ये॒षः । स॒म्भव॒तीति॑ सम् - भव॑ति । ए॒ष वै । वै स्व॑य॒म्भूः । स्व॒य॒म्भूर् नाम॑ । स्व॒य॒म्भूरिति॑ स्वयम् - भूः । नाम॒ भव॑ति । भव॑त्ये॒व । ए॒व यम् । यम् का॒मये॑त । का॒मये॑त॒ भ्रातृ॑व्यम् । भ्रातृ॑व्यमस्मै । अ॒स्मै॒ ज॒न॒ये॒य॒म् । ज॒न॒ये॒य॒मिति॑ । इत्य॒न्यतः॑ । अ॒न्यत॒स्तस्य॑ । तस्या॒हृत्य॑ । आ॒हृत्याव॑ । आ॒हृत्येत्या᳚ - हृत्य॑ । अव॑ दद्ध्यात् । द॒द्ध्या॒थ् सा॒क्षात् । सा॒क्षादे॒व । सा॒क्षादिति॑ स - अ॒क्षात् । ए॒वास्मै᳚ । अ॒स्मै॒ भ्रातृ॑व्यम् । भ्रातृ॑व्यम् जनयति । ज॒न॒य॒त्य॒म्ब॒रीषा᳚त् । अ॒म्ब॒रीषा॒दन्न॑कामस्य । अन्न॑काम॒स्याव॑ । अन्न॑काम॒स्येत्यन्न॑ - का॒म॒स्य॒ । अव॑ दद्ध्यात् । द॒द्ध्या॒द॒म्ब॒रीषे᳚ । अ॒म्ब॒रीषे॒ वै । वा अन्न᳚म् । अन्न॑म् भ्रियते । भ्रि॒य॒ते॒ सयो॑नि । सयो᳚न्ये॒व । सयो॒नीति॒ स - यो॒नि॒ । ए॒वान्न᳚म् । अन्न॒मव॑ \newline

\textbf{Jatai Paata} \newline

1. दे॒वता॒ मुपोप॑ दे॒वता᳚म् दे॒वता॒ मुप॑ । \newline
2. उपै᳚ त्ये॒ त्युपोपै॑ति । \newline
3. ए॒ति॒ यो य ए᳚त्येति॒ यः । \newline
4. यो भूति॑कामो॒ भूति॑कामो॒ यो यो भूति॑कामः । \newline
5. भूति॑कामः॒ स्याथ् स्याद् भूति॑कामो॒ भूति॑कामः॒ स्यात् । \newline
6. भूति॑काम॒ इति॒ भूति॑ - का॒मः॒ । \newline
7. स्याद् यो यः स्याथ् स्याद् यः । \newline
8. य उ॒खाया॑ उ॒खायै॒ यो य उ॒खायै᳚ । \newline
9. उ॒खायै॑ सं॒भवे᳚थ् सं॒भवे॑ दु॒खाया॑ उ॒खायै॑ सं॒भवे᳚त् । \newline
10. सं॒भवे॒थ् स स सं॒भवे᳚थ् सं॒भवे॒थ् सः । \newline
11. सं॒भवे॒दिति॑ सं - भवे᳚त् । \newline
12. स ए॒वैव स स ए॒व । \newline
13. ए॒व तस्य॒ तस्यै॒वैव तस्य॑ । \newline
14. तस्य॑ स्याथ् स्या॒त् तस्य॒ तस्य॑ स्यात् । \newline
15. स्या॒ दतो ऽतः॑ स्याथ् स्या॒ दतः॑ । \newline
16. अतो॒ हि ह्यतो ऽतो॒ हि । \newline
17. ह्ये॑ष ए॒ष हि ह्ये॑षः । \newline
18. ए॒ष सं॒भव॑ति सं॒भव॑ त्ये॒ष ए॒ष सं॒भव॑ति । \newline
19. सं॒भव॑ त्ये॒ष ए॒ष सं॒भव॑ति सं॒भव॑ त्ये॒षः । \newline
20. सं॒भव॒तीति॑ सं - भव॑ति । \newline
21. ए॒ष वै वा ए॒ष ए॒ष वै । \newline
22. वै स्व॑यं॒भूः स्व॑यं॒भूर् वै वै स्व॑यं॒भूः । \newline
23. स्व॒यं॒भूर् नाम॒ नाम॑ स्वयं॒भूः स्व॑यं॒भूर् नाम॑ । \newline
24. स्व॒यं॒भूरिति॑ स्वयं - भूः । \newline
25. नाम॒ भव॑ति॒ भव॑ति॒ नाम॒ नाम॒ भव॑ति । \newline
26. भव॑ त्ये॒वैव भव॑ति॒ भव॑ त्ये॒व । \newline
27. ए॒व यं ॅय मे॒वैव यम् । \newline
28. यम् का॒मये॑त का॒मये॑त॒ यं ॅयम् का॒मये॑त । \newline
29. का॒मये॑त॒ भ्रातृ॑व्य॒म् भ्रातृ॑व्यम् का॒मये॑त का॒मये॑त॒ भ्रातृ॑व्यम् । \newline
30. भ्रातृ॑व्य मस्मा अस्मै॒ भ्रातृ॑व्य॒म् भ्रातृ॑व्य मस्मै । \newline
31. अ॒स्मै॒ ज॒न॒ये॒य॒म् ज॒न॒ये॒य॒ म॒स्मा॒ अ॒स्मै॒ ज॒न॒ये॒य॒म् । \newline
32. ज॒न॒ये॒य॒ मितीति॑ जनयेयम् जनयेय॒ मिति॑ । \newline
33. इत्य॒न्यतो॒ ऽन्यत॒ इती त्य॒न्यतः॑ । \newline
34. अ॒न्यत॒ स्तस्य॒ तस्या॒ न्यतो॒ ऽन्यत॒ स्तस्य॑ । \newline
35. तस्या॒ हृत्या॒ हृत्य॒ तस्य॒ तस्या॒ हृत्य॑ । \newline
36. आ॒हृत्या वावा॒ हृत्या॒ हृत्याव॑ । \newline
37. आ॒हृत्येत्या᳚ - हृत्य॑ । \newline
38. अव॑ दद्ध्याद् दद्ध्या॒ दवाव॑ दद्ध्यात् । \newline
39. द॒द्ध्या॒थ् सा॒क्षाथ् सा॒क्षाद् द॑द्ध्याद् दद्ध्याथ् सा॒क्षात् । \newline
40. सा॒क्षा दे॒वैव सा॒क्षाथ् सा॒क्षा दे॒व । \newline
41. सा॒क्षादिति॑ स - अ॒क्षात् । \newline
42. ए॒वास्मा॑ अस्मा ए॒वैवास्मै᳚ । \newline
43. अ॒स्मै॒ भ्रातृ॑व्य॒म् भ्रातृ॑व्य मस्मा अस्मै॒ भ्रातृ॑व्यम् । \newline
44. भ्रातृ॑व्यम् जनयति जनयति॒ भ्रातृ॑व्य॒म् भ्रातृ॑व्यम् जनयति । \newline
45. ज॒न॒य॒ त्यं॒ब॒रीषा॑ दंब॒रीषा᳚ज् जनयति जनय त्यंब॒रीषा᳚त् । \newline
46. अं॒ब॒रीषा॒ दन्न॑काम॒स्या न्न॑कामस्यांब॒रीषा॑ दंब॒रीषा॒ दन्न॑कामस्य । \newline
47. अन्न॑काम॒स्या वावा न्न॑काम॒स्या न्न॑काम॒स्याव॑ । \newline
48. अन्न॑काम॒स्येत्यन्न॑ - का॒म॒स्य॒ । \newline
49. अव॑ दद्ध्याद् दद्ध्या॒ दवाव॑ दद्ध्यात् । \newline
50. द॒द्ध्या॒ दं॒ब॒रीषें᳚ ब॒रीषे॑ दद्ध्याद् दद्ध्या दंब॒रीषे᳚ । \newline
51. अं॒ब॒रीषे॒ वै वा अं॑ब॒रीषें᳚ ब॒रीषे॒ वै । \newline
52. वा अन्न॒ मन्नं॒ ॅवै वा अन्न᳚म् । \newline
53. अन्न॑म् भ्रियते भ्रिय॒ते ऽन्न॒ मन्न॑म् भ्रियते । \newline
54. भ्रि॒य॒ते॒ सयो॑नि॒ सयो॑नि भ्रियते भ्रियते॒ सयो॑नि । \newline
55. सयो᳚ न्ये॒वैव सयो॑नि॒ सयो᳚ न्ये॒व । \newline
56. सयो॒नीति॒ स - यो॒नि॒ । \newline
57. ए॒वान्न॒ मन्न॑ मे॒वै वान्न᳚म् । \newline
58. अन्न॒ मवा वान्न॒ मन्न॒ मव॑ । \newline

\textbf{Ghana Paata } \newline

1. दे॒वता॒ मुपोप॑ दे॒वता᳚म् दे॒वता॒ मुपै᳚ त्ये॒त्युप॑ दे॒वता᳚म् दे॒वता॒ मुपै॑ति । \newline
2. उपै᳚त्ये॒ त्युपोपै॑ति॒ यो य ए॒त्युपोपै॑ति॒ यः । \newline
3. ए॒ति॒ यो य ए᳚त्येति॒ यो भूति॑कामो॒ भूति॑कामो॒ य ए᳚त्येति॒ यो भूति॑कामः । \newline
4. यो भूति॑कामो॒ भूति॑कामो॒ यो यो भूति॑कामः॒ स्याथ् स्याद् भूति॑कामो॒ यो यो भूति॑कामः॒ स्यात् । \newline
5. भूति॑कामः॒ स्याथ् स्याद् भूति॑कामो॒ भूति॑कामः॒ स्याद् यो यः स्याद् भूति॑कामो॒ भूति॑कामः॒ स्याद् यः । \newline
6. भूति॑काम॒ इति॒ भूति॑ - का॒मः॒ । \newline
7. स्याद् यो यः स्याथ् स्याद् य उ॒खाया॑ उ॒खायै॒ यः स्याथ् स्याद् य उ॒खायै᳚ । \newline
8. य उ॒खाया॑ उ॒खायै॒ यो य उ॒खायै॑ सं॒भवे᳚थ् सं॒भवे॑ दु॒खायै॒ यो य उ॒खायै॑ सं॒भवे᳚त् । \newline
9. उ॒खायै॑ सं॒भवे᳚थ् सं॒भवे॑ दु॒खाया॑ उ॒खायै॑ सं॒भवे॒थ् स स सं॒भवे॑ दु॒खाया॑ उ॒खायै॑ सं॒भवे॒थ् सः । \newline
10. सं॒भवे॒थ् स स सं॒भवे᳚थ् सं॒भवे॒थ् स ए॒वैव स सं॒भवे᳚थ् सं॒भवे॒थ् स ए॒व । \newline
11. सं॒भवे॒दिति॑ सं - भवे᳚त् । \newline
12. स ए॒वैव स स ए॒व तस्य॒ तस्यै॒व स स ए॒व तस्य॑ । \newline
13. ए॒व तस्य॒ तस्यै॒वैव तस्य॑ स्याथ् स्या॒त् तस्यै॒वैव तस्य॑ स्यात् । \newline
14. तस्य॑ स्याथ् स्या॒त् तस्य॒ तस्य॑ स्या॒दतो ऽतः॑ स्या॒त् तस्य॒ तस्य॑ स्या॒दतः॑ । \newline
15. स्या॒दतो ऽतः॑ स्याथ् स्या॒दतो॒ हि ह्यतः॑ स्याथ् स्या॒दतो॒ हि । \newline
16. अतो॒ हि ह्यतो ऽतो॒ ह्ये॑ष ए॒ष ह्यतो ऽतो॒ ह्ये॑षः । \newline
17. ह्ये॑ष ए॒ष हि ह्ये॑ष सं॒भव॑ति सं॒भव॑ त्ये॒ष हि ह्ये॑ष सं॒भव॑ति । \newline
18. ए॒ष सं॒भव॑ति सं॒भव॑ त्ये॒ष ए॒ष सं॒भव॑ त्ये॒ष ए॒ष सं॒भव॑ त्ये॒ष ए॒ष सं॒भव॑ त्ये॒षः । \newline
19. सं॒भव॑ त्ये॒ष ए॒ष सं॒भव॑ति सं॒भव॑ त्ये॒ष वै वा ए॒ष सं॒भव॑ति सं॒भव॑ त्ये॒ष वै । \newline
20. सं॒भव॒तीति॑ सं - भव॑ति । \newline
21. ए॒ष वै वा ए॒ष ए॒ष वै स्व॑यं॒भूः स्व॑यं॒भूर् वा ए॒ष ए॒ष वै स्व॑यं॒भूः । \newline
22. वै स्व॑यं॒भूः स्व॑यं॒भूर् वै वै स्व॑यं॒भूर् नाम॒ नाम॑ स्वयं॒भूर् वै वै स्व॑यं॒भूर् नाम॑ । \newline
23. स्व॒यं॒भूर् नाम॒ नाम॑ स्वयं॒भूः स्व॑यं॒भूर् नाम॒ भव॑ति॒ भव॑ति॒ नाम॑ स्वयं॒भूः स्व॑यं॒भूर् नाम॒ भव॑ति । \newline
24. स्व॒यं॒भूरिति॑ स्वयं - भूः । \newline
25. नाम॒ भव॑ति॒ भव॑ति॒ नाम॒ नाम॒ भव॑ त्ये॒वैव भव॑ति॒ नाम॒ नाम॒ भव॑ त्ये॒व । \newline
26. भव॑ त्ये॒वैव भव॑ति॒ भव॑ त्ये॒व यं ॅय मे॒व भव॑ति॒ भव॑ त्ये॒व यम् । \newline
27. ए॒व यं ॅय मे॒वैव यम् का॒मये॑त का॒मये॑त॒ य मे॒वैव यम् का॒मये॑त । \newline
28. यम् का॒मये॑त का॒मये॑त॒ यं ॅयम् का॒मये॑त॒ भ्रातृ॑व्य॒म् भ्रातृ॑व्यम् का॒मये॑त॒ यं ॅयम् का॒मये॑त॒ भ्रातृ॑व्यम् । \newline
29. का॒मये॑त॒ भ्रातृ॑व्य॒म् भ्रातृ॑व्यम् का॒मये॑त का॒मये॑त॒ भ्रातृ॑व्य मस्मा अस्मै॒ भ्रातृ॑व्यम् का॒मये॑त का॒मये॑त॒ भ्रातृ॑व्य मस्मै । \newline
30. भ्रातृ॑व्य मस्मा अस्मै॒ भ्रातृ॑व्य॒म् भ्रातृ॑व्य मस्मै जनयेयम् जनयेय मस्मै॒ भ्रातृ॑व्य॒म् भ्रातृ॑व्य मस्मै जनयेयम् । \newline
31. अ॒स्मै॒ ज॒न॒ये॒य॒म् ज॒न॒ये॒य॒ म॒स्मा॒ अ॒स्मै॒ ज॒न॒ये॒य॒ मितीति॑ जनयेय मस्मा अस्मै जनयेय॒ मिति॑ । \newline
32. ज॒न॒ये॒य॒ मितीति॑ जनयेयम् जनयेय॒ मित्य॒न्यतो॒ ऽन्यत॒ इति॑ जनयेयम् जनयेय॒ मित्य॒न्यतः॑ । \newline
33. इत्य॒न्यतो॒ ऽन्यत॒ इती त्य॒न्यत॒ स्तस्य॒ तस्या॒न्यत॒ इती त्य॒न्यत॒ स्तस्य॑ । \newline
34. अ॒न्यत॒ स्तस्य॒ तस्या॒ न्यतो॒ ऽन्यत॒ स्तस्या॒ हृत्या॒ हृत्य॒ तस्या॒ न्यतो॒ ऽन्यत॒ स्तस्या॒ हृत्य॑ । \newline
35. तस्या॒ हृत्या॒ हृत्य॒ तस्य॒ तस्या॒ हृत्या वावा॒ हृत्य॒ तस्य॒ तस्या॒ हृत्याव॑ । \newline
36. आ॒हृत्या वावा॒ हृत्या॒ हृत्याव॑ दद्ध्याद् दद्ध्या॒ दवा॒हृत्या॒ हृत्याव॑ दद्ध्यात् । \newline
37. आ॒हृत्येत्या᳚ - हृत्य॑ । \newline
38. अव॑ दद्ध्याद् दद्ध्या॒ दवाव॑ दद्ध्याथ् सा॒क्षाथ् सा॒क्षाद् द॑द्ध्या॒ दवाव॑ दद्ध्याथ् सा॒क्षात् । \newline
39. द॒द्ध्या॒थ् सा॒क्षाथ् सा॒क्षाद् द॑द्ध्याद् दद्ध्याथ् सा॒क्षा दे॒वैव सा॒क्षाद् द॑द्ध्याद् दद्ध्याथ् सा॒क्षा दे॒व । \newline
40. सा॒क्षा दे॒वैव सा॒क्षाथ् सा॒क्षा दे॒वास्मा॑ अस्मा ए॒व सा॒क्षाथ् सा॒क्षा दे॒वास्मै᳚ । \newline
41. सा॒क्षादिति॑ स - अ॒क्षात् । \newline
42. ए॒वास्मा॑ अस्मा ए॒वैवास्मै॒ भ्रातृ॑व्य॒म् भ्रातृ॑व्य मस्मा ए॒वैवास्मै॒ भ्रातृ॑व्यम् । \newline
43. अ॒स्मै॒ भ्रातृ॑व्य॒म् भ्रातृ॑व्य मस्मा अस्मै॒ भ्रातृ॑व्यम् जनयति जनयति॒ भ्रातृ॑व्य मस्मा अस्मै॒ भ्रातृ॑व्यम् जनयति । \newline
44. भ्रातृ॑व्यम् जनयति जनयति॒ भ्रातृ॑व्य॒म् भ्रातृ॑व्यम् जनय त्यंब॒रीषा॑ दंब॒रीषा᳚ज् जनयति॒ भ्रातृ॑व्य॒म् भ्रातृ॑व्यम् जनय त्यंब॒रीषा᳚त् । \newline
45. ज॒न॒य॒ त्यं॒ब॒रीषा॑ दंब॒रीषा᳚ज् जनयति जनय त्यंब॒रीषा॒ दन्न॑काम॒स्या न्न॑काम स्यांब॒रीषा᳚ज् जनयति जनय त्यंब॒रीषा॒ दन्न॑कामस्य । \newline
46. अं॒ब॒रीषा॒ दन्न॑काम॒स्या न्न॑काम स्यांब॒रीषा॑ दंब॒रीषा॒ दन्न॑काम॒स्या वावा न्न॑कामस्यां ब॒रीषा॑ दंब॒रीषा॒ दन्न॑काम॒स्याव॑ । \newline
47. अन्न॑काम॒स्या वावा न्न॑काम॒स्या न्न॑काम॒स्याव॑ दद्ध्याद् दद्ध्या॒ दवा न्न॑काम॒स्या न्न॑काम॒स्याव॑ दद्ध्यात् । \newline
48. अन्न॑काम॒स्येत्यन्न॑ - का॒म॒स्य॒ । \newline
49. अव॑ दद्ध्याद् दद्ध्या॒ दवाव॑ दद्ध्या दंब॒रीषें᳚ ब॒रीषे॑ दद्ध्या॒ दवाव॑ दद्ध्या दंब॒रीषे᳚ । \newline
50. द॒द्ध्या॒ दं॒ब॒रीषें᳚ ब॒रीषे॑ दद्ध्याद् दद्ध्या दंब॒रीषे॒ वै वा अं॑ब॒रीषे॑ दद्ध्याद् दद्ध्या दंब॒रीषे॒ वै । \newline
51. अं॒ब॒रीषे॒ वै वा अं॑ब॒रीषें᳚ ब॒रीषे॒ वा अन्न॒ मन्नं॒ ॅवा अं॑ब॒रीषें᳚ ब॒रीषे॒ वा अन्न᳚म् । \newline
52. वा अन्न॒ मन्नं॒ ॅवै वा अन्न॑म् भ्रियते भ्रिय॒ते ऽन्नं॒ ॅवै वा अन्न॑म् भ्रियते । \newline
53. अन्न॑म् भ्रियते भ्रिय॒ते ऽन्न॒ मन्न॑म् भ्रियते॒ सयो॑नि॒ सयो॑नि भ्रिय॒ते ऽन्न॒ मन्न॑म् भ्रियते॒ सयो॑नि । \newline
54. भ्रि॒य॒ते॒ सयो॑नि॒ सयो॑नि भ्रियते भ्रियते॒ सयो᳚ न्ये॒वैव सयो॑नि भ्रियते भ्रियते॒ सयो᳚ न्ये॒व । \newline
55. सयो᳚ न्ये॒वैव सयो॑नि॒ सयो᳚ न्ये॒वान्न॒ मन्न॑ मे॒व सयो॑नि॒ सयो᳚ न्ये॒वान्न᳚म् । \newline
56. सयो॒नीति॒ स - यो॒नि॒ । \newline
57. ए॒वान्न॒ मन्न॑ मे॒वैवान्न॒ मवावान्न॑ मे॒वैवान्न॒ मव॑ । \newline
58. अन्न॒ मवावान्न॒ मन्न॒ मव॑ रुन्धे रु॒न्धे ऽवान्न॒ मन्न॒ मव॑ रुन्धे । \newline
\pagebreak
\markright{ TS 5.1.9.5  \hfill https://www.vedavms.in \hfill}

\section{ TS 5.1.9.5 }

\textbf{TS 5.1.9.5 } \newline
\textbf{Samhita Paata} \newline

-मव॑ रुन्धे॒ मुञ्जा॒नव॑ दधा॒त्यूर्ग्वै मुञ्जा॒ ऊर्ज॑मे॒वास्मा॒ अपि॑ दधात्य॒ग्निर्दे॒वेभ्यो॒ निला॑यत॒ स क्रु॑मु॒कं प्राऽ*वि॑शत् क्रुमु॒कमव॑ दधाति॒ यदे॒वास्य॒ तत्र॒ न्य॑क्तं॒ तदे॒वाव॑ रुन्ध॒ आज्ये॑न॒ सं ॅयौ᳚त्ये॒तद्वा अ॒ग्नेः प्रि॒यं धाम॒ यदाज्यं॑ प्रि॒येणै॒वैनं॒ धाम्ना॒ सम॑र्द्धय॒त्यथो॒ तेज॑सा॒ - [  ] \newline

\textbf{Pada Paata} \newline

अवेति॑ । रु॒न्धे॒ । मुञ्जान्॑ । अवेति॑ । द॒धा॒ति॒ । ऊर्क् । वै । मुञ्जाः᳚ । ऊर्ज᳚म् । ए॒व । अ॒स्मै॒ । अपीति॑ । द॒धा॒ति॒ । अ॒ग्निः । दे॒वेभ्यः॑ । निला॑यत । सः । क्रु॒मु॒कम् । प्रेति॑ । अ॒वि॒श॒त् । क्रु॒मु॒कम् । अवेति॑ । द॒धा॒ति॒ । यत् । ए॒व । अ॒स्य॒ । तत्र॑ । न्य॑क्त॒मिति॒ नि - अ॒क्त॒म् । तत् । ए॒व । अवेति॑ । रु॒न्धे॒ । आज्ये॑न । समिति॑ । यौ॒ति॒ । ए॒तत् । वै । अ॒ग्नेः । प्रि॒यम् । धाम॑ । यत् । आज्य᳚म् । प्रि॒येण॑ । ए॒व । ए॒न॒म् । धाम्ना᳚ । समिति॑ । अ॒द्‌र्ध॒य॒ति॒ । अथो॒ इति॑ । तेज॑सा ।  \newline


\textbf{Krama Paata} \newline

अव॑ रुन्धे । रु॒न्धे॒ मुञ्जान्॑ । मुञ्जा॒नव॑ । अव॑ दधाति । द॒धा॒त्यूर्क् । ऊर्ग् वै । वै मुञ्जाः᳚ । मुञ्जा॒ ऊर्ज᳚म् । ऊर्ज॑मे॒व । ए॒वास्मै᳚ । अ॒स्मा॒ अपि॑ । अपि॑ दधाति । द॒धा॒त्य॒ग्निः । अ॒ग्निर् दे॒वेभ्यः॑ । दे॒वेभ्यो॒ निला॑यत । निला॑यत॒ सः । स क्रु॑मु॒कम् । क्रु॒मु॒कम् प्र । प्रावि॑शत् । अ॒वि॒श॒त् क्रु॒मु॒कम् । क्रु॒मु॒कमव॑ । अव॑ दधाति । द॒धा॒ति॒ यत् । यदे॒व । ए॒वास्य॑ । अ॒स्य॒ तत्र॑ । तत्र॒ न्य॑क्तम् । न्य॑क्त॒म् तत् । न्य॑क्त॒मिति॒ नि - अ॒क्त॒म् । तदे॒व । ए॒वाव॑ । अव॑ रुन्धे । रु॒न्ध॒ आज्ये॑न । आज्ये॑न॒ सम् । सम् ॅयौ॑ति । यौ॒त्ये॒तत् । ए॒तद् वै । वा अ॒ग्नेः । अ॒ग्नेः प्रि॒यम् । प्रि॒यम् धाम॑ । धाम॒ यत् । यदाज्य᳚म् । आज्य॑म् प्रि॒येण॑ । प्रि॒येणै॒व । ए॒वैन᳚म् । ए॒न॒म् धाम्ना᳚ । धाम्ना॒ सम् । सम॑र्द्धयति । अ॒र्द्ध॒य॒त्यथो᳚ । अथो॒ तेज॑सा ( ) । अथो॒ इत्यथो᳚ । तेज॑सा॒ वैक॑ङ्कतीम् \newline

\textbf{Jatai Paata} \newline

1. अव॑ रुन्धे रु॒न्धे ऽवाव॑ रुन्धे । \newline
2. रु॒न्धे॒ मुञ्जा॒न् मुञ्जा᳚न् रुन्धे रुन्धे॒ मुञ्जान्॑ । \newline
3. मुञ्जा॒ नवाव॒ मुञ्जा॒न् मुञ्जा॒ नव॑ । \newline
4. अव॑ दधाति दधा॒ त्यवाव॑ दधाति । \newline
5. द॒धा॒ त्यूर्गूर्ग् द॑धाति दधा॒ त्यूर्क् । \newline
6. ऊर्ग् वै वा ऊर्गूर्ग् वै । \newline
7. वै मुञ्जा॒ मुञ्जा॒ वै वै मुञ्जाः᳚ । \newline
8. मुञ्जा॒ ऊर्ज॒ मूर्ज॒म् मुञ्जा॒ मुञ्जा॒ ऊर्ज᳚म् । \newline
9. ऊर्ज॑ मे॒वैवोर्ज॒ मूर्ज॑ मे॒व । \newline
10. ए॒वास्मा॑ अस्मा ए॒वैवास्मै᳚ । \newline
11. अ॒स्मा॒ अप्य प्य॑स्मा अस्मा॒ अपि॑ । \newline
12. अपि॑ दधाति दधा॒ त्यप्यपि॑ दधाति । \newline
13. द॒धा॒ त्य॒ग्नि र॒ग्निर् द॑धाति दधा त्य॒ग्निः । \newline
14. अ॒ग्निर् दे॒वेभ्यो॑ दे॒वेभ्यो॒ ऽग्नि र॒ग्निर् दे॒वेभ्यः॑ । \newline
15. दे॒वेभ्यो॒ निला॑यत॒ निला॑यत दे॒वेभ्यो॑ दे॒वेभ्यो॒ निला॑यत । \newline
16. निला॑यत॒ स स निला॑यत॒ निला॑यत॒ सः । \newline
17. स क्रु॑मु॒कम् क्रु॑मु॒कꣳ स स क्रु॑मु॒कम् । \newline
18. क्रु॒मु॒कम् प्र प्र क्रु॑मु॒कम् क्रु॑मु॒कम् प्र । \newline
19. प्रावि॑श दविश॒त् प्र प्रावि॑शत् । \newline
20. अ॒वि॒श॒त् क्रु॒मु॒कम् क्रु॑मु॒क म॑विश दविशत् क्रुमु॒कम् । \newline
21. क्रु॒मु॒क मवाव॑ क्रुमु॒कम् क्रु॑मु॒क मव॑ । \newline
22. अव॑ दधाति दधा॒ त्यवाव॑ दधाति । \newline
23. द॒धा॒ति॒ यद् यद् द॑धाति दधाति॒ यत् । \newline
24. यदे॒वैव यद् यदे॒व । \newline
25. ए॒वास्या᳚ स्यै॒वैवास्य॑ । \newline
26. अ॒स्य॒ तत्र॒ तत्रा᳚ स्यास्य॒ तत्र॑ । \newline
27. तत्र॒ न्य॑क्त॒म् न्य॑क्त॒म् तत्र॒ तत्र॒ न्य॑क्तम् । \newline
28. न्य॑क्त॒म् तत् तन् न्य॑क्त॒म् न्य॑क्त॒म् तत् । \newline
29. न्य॑क्त॒मिति॒ नि - अ॒क्त॒म् । \newline
30. तदे॒वैव तत् तदे॒व । \newline
31. ए॒वावा वै॒वै वाव॑ । \newline
32. अव॑ रुन्धे रु॒न्धे ऽवाव॑ रुन्धे । \newline
33. रु॒न्ध॒ आज्ये॒ना ज्ये॑न रुन्धे रुन्ध॒ आज्ये॑न । \newline
34. आज्ये॑न॒ सꣳ स माज्ये॒ना ज्ये॑न॒ सम् । \newline
35. सं ॅयौ॑ति यौति॒ सꣳ सं ॅयौ॑ति । \newline
36. यौ॒त्ये॒त दे॒तद् यौ॑ति यौत्ये॒तत् । \newline
37. ए॒तद् वै वा ए॒त दे॒तद् वै । \newline
38. वा अ॒ग्ने र॒ग्नेर् वै वा अ॒ग्नेः । \newline
39. अ॒ग्नेः प्रि॒यम् प्रि॒य म॒ग्ने र॒ग्नेः प्रि॒यम् । \newline
40. प्रि॒यम् धाम॒ धाम॑ प्रि॒यम् प्रि॒यम् धाम॑ । \newline
41. धाम॒ यद् यद् धाम॒ धाम॒ यत् । \newline
42. यदाज्य॒ माज्यं॒ ॅयद् यदाज्य᳚म् । \newline
43. आज्य॑म् प्रि॒येण॑ प्रि॒येणाज्य॒ माज्य॑म् प्रि॒येण॑ । \newline
44. प्रि॒ये णै॒वैव प्रि॒येण॑ प्रि॒ये णै॒व । \newline
45. ए॒वैन॑ मेन मे॒वैवैन᳚म् । \newline
46. ए॒न॒म् धाम्ना॒ धाम्नै॑न मेन॒म् धाम्ना᳚ । \newline
47. धाम्ना॒ सꣳ सम् धाम्ना॒ धाम्ना॒ सम् । \newline
48. स म॑र्द्धय त्यर्द्धयति॒ सꣳ स म॑र्द्धयति । \newline
49. अ॒र्द्ध॒य॒ त्यथो॒ अथो॑ अर्द्धय त्यर्द्धय॒ त्यथो᳚ । \newline
50. अथो॒ तेज॑सा॒ तेज॒सा॒ ऽथो॒ अथो॒ तेज॑सा । \newline
51. अथो॒ इत्यथो᳚ । \newline
52. तेज॑सा॒ वैक॑ङ्कतीं॒ ॅवैक॑ङ्कती॒म् तेज॑सा॒ तेज॑सा॒ वैक॑ङ्कतीम् । \newline

\textbf{Ghana Paata } \newline

1. अव॑ रुन्धे रु॒न्धे ऽवाव॑ रुन्धे॒ मुञ्जा॒न् मुञ्जा᳚न् रु॒न्धे ऽवाव॑ रुन्धे॒ मुञ्जान्॑ । \newline
2. रु॒न्धे॒ मुञ्जा॒न् मुञ्जा᳚न् रुन्धे रुन्धे॒ मुञ्जा॒ नवाव॒ मुञ्जा᳚न् रुन्धे रुन्धे॒ मुञ्जा॒ नव॑ । \newline
3. मुञ्जा॒ नवाव॒ मुञ्जा॒न् मुञ्जा॒ नव॑ दधाति दधा॒ त्यव॒ मुञ्जा॒न् मुञ्जा॒ नव॑ दधाति । \newline
4. अव॑ दधाति दधा॒ त्यवाव॑ दधा॒ त्यूर्गूर्ग् द॑धा॒ त्यवाव॑ दधा॒ त्यूर्क् । \newline
5. द॒धा॒ त्यूर्गूर्ग् द॑धाति दधा॒ त्यूर्ग् वै वा ऊर्ग् द॑धाति दधा॒ त्यूर्ग् वै । \newline
6. ऊर्ग् वै वा ऊर्गूर्ग् वै मुञ्जा॒ मुञ्जा॒ वा ऊर्गूर्ग् वै मुञ्जाः᳚ । \newline
7. वै मुञ्जा॒ मुञ्जा॒ वै वै मुञ्जा॒ ऊर्ज॒ मूर्ज॒म् मुञ्जा॒ वै वै मुञ्जा॒ ऊर्ज᳚म् । \newline
8. मुञ्जा॒ ऊर्ज॒ मूर्ज॒म् मुञ्जा॒ मुञ्जा॒ ऊर्ज॑ मे॒वैवोर्ज॒म् मुञ्जा॒ मुञ्जा॒ ऊर्ज॑ मे॒व । \newline
9. ऊर्ज॑ मे॒वैवोर्ज॒ मूर्ज॑ मे॒वास्मा॑ अस्मा ए॒वोर्ज॒ मूर्ज॑ मे॒वास्मै᳚ । \newline
10. ए॒वास्मा॑ अस्मा ए॒वैवास्मा॒ अप्य प्य॑स्मा ए॒वैवास्मा॒ अपि॑ । \newline
11. अ॒स्मा॒ अप्य प्य॑स्मा अस्मा॒ अपि॑ दधाति दधा॒ त्यप्य॑स्मा अस्मा॒ अपि॑ दधाति । \newline
12. अपि॑ दधाति दधा॒ त्यप्यपि॑ दधा त्य॒ग्नि र॒ग्निर् द॑धा॒ त्यप्यपि॑ दधा त्य॒ग्निः । \newline
13. द॒धा॒ त्य॒ग्नि र॒ग्निर् द॑धाति दधा त्य॒ग्निर् दे॒वेभ्यो॑ दे॒वेभ्यो॒ ऽग्निर् द॑धाति दधा त्य॒ग्निर् दे॒वेभ्यः॑ । \newline
14. अ॒ग्निर् दे॒वेभ्यो॑ दे॒वेभ्यो॒ ऽग्नि र॒ग्निर् दे॒वेभ्यो॒ निला॑यत॒ निला॑यत दे॒वेभ्यो॒ ऽग्नि र॒ग्निर् दे॒वेभ्यो॒ निला॑यत । \newline
15. दे॒वेभ्यो॒ निला॑यत॒ निला॑यत दे॒वेभ्यो॑ दे॒वेभ्यो॒ निला॑यत॒ स स निला॑यत दे॒वेभ्यो॑ दे॒वेभ्यो॒ निला॑यत॒ सः । \newline
16. निला॑यत॒ स स निला॑यत॒ निला॑यत॒ स क्रु॑मु॒कम् क्रु॑मु॒कꣳ स निला॑यत॒ निला॑यत॒ स क्रु॑मु॒कम् । \newline
17. स क्रु॑मु॒कम् क्रु॑मु॒कꣳ स स क्रु॑मु॒कम् प्र प्र क्रु॑मु॒कꣳ स स क्रु॑मु॒कम् प्र । \newline
18. क्रु॒मु॒कम् प्र प्र क्रु॑मु॒कम् क्रु॑मु॒कम् प्रावि॑श दविश॒त् प्र क्रु॑मु॒कम् क्रु॑मु॒कम् प्रावि॑शत् । \newline
19. प्रावि॑श दविश॒त् प्र प्रावि॑शत् क्रुमु॒कम् क्रु॑मु॒क म॑विश॒त् प्र प्रावि॑शत् क्रुमु॒कम् । \newline
20. अ॒वि॒श॒त् क्रु॒मु॒कम् क्रु॑मु॒क म॑विश दविशत् क्रुमु॒क मवाव॑ क्रुमु॒क म॑विश दविशत् क्रुमु॒क मव॑ । \newline
21. क्रु॒मु॒क मवाव॑ क्रुमु॒कम् क्रु॑मु॒क मव॑ दधाति दधा॒ त्यव॑ क्रुमु॒कम् क्रु॑मु॒क मव॑ दधाति । \newline
22. अव॑ दधाति दधा॒ त्यवाव॑ दधाति॒ यद् यद् द॑धा॒ त्यवाव॑ दधाति॒ यत् । \newline
23. द॒धा॒ति॒ यद् यद् द॑धाति दधाति॒ यदे॒वैव यद् द॑धाति दधाति॒ यदे॒व । \newline
24. यदे॒वैव यद् यदे॒वास्या᳚ स्यै॒व यद् यदे॒वास्य॑ । \newline
25. ए॒वास्या᳚ स्यै॒वैवास्य॒ तत्र॒ तत्रा᳚ स्यै॒वैवास्य॒ तत्र॑ । \newline
26. अ॒स्य॒ तत्र॒ तत्रा᳚ स्यास्य॒ तत्र॒ न्य॑क्त॒म् न्य॑क्त॒म् तत्रा᳚ स्यास्य॒ तत्र॒ न्य॑क्तम् । \newline
27. तत्र॒ न्य॑क्त॒म् न्य॑क्त॒म् तत्र॒ तत्र॒ न्य॑क्त॒म् तत् तन् न्य॑क्त॒म् तत्र॒ तत्र॒ न्य॑क्त॒म् तत् । \newline
28. न्य॑क्त॒म् तत् तन् न्य॑क्त॒म् न्य॑क्त॒म् तदे॒वैव तन् न्य॑क्त॒म् न्य॑क्त॒म् तदे॒व । \newline
29. न्य॑क्त॒मिति॒ नि - अ॒क्त॒म् । \newline
30. तदे॒वैव तत् तदे॒ वावा वै॒व तत् तदे॒वाव॑ । \newline
31. ए॒वावा वै॒वै वाव॑ रुन्धे रु॒न्धे ऽवै॒वै वाव॑ रुन्धे । \newline
32. अव॑ रुन्धे रु॒न्धे ऽवाव॑ रुन्ध॒ आज्ये॒ नाज्ये॑न रु॒न्धे ऽवाव॑ रुन्ध॒ आज्ये॑न । \newline
33. रु॒न्ध॒ आज्ये॒ नाज्ये॑न रुन्धे रुन्ध॒ आज्ये॑न॒ सꣳ स माज्ये॑न रुन्धे रुन्ध॒ आज्ये॑न॒ सम् । \newline
34. आज्ये॑न॒ सꣳ स माज्ये॒ नाज्ये॑न॒ सं ॅयौ॑ति यौति॒ स माज्ये॒ नाज्ये॑न॒ सं ॅयौ॑ति । \newline
35. सं ॅयौ॑ति यौति॒ सꣳ सं ॅयौ᳚त्ये॒त दे॒तद् यौ॑ति॒ सꣳ सं ॅयौ᳚त्ये॒तत् । \newline
36. यौ॒त्ये॒त दे॒तद् यौ॑ति यौत्ये॒तद् वै वा ए॒तद् यौ॑ति यौत्ये॒तद् वै । \newline
37. ए॒तद् वै वा ए॒त दे॒तद् वा अ॒ग्ने र॒ग्नेर् वा ए॒त दे॒तद् वा अ॒ग्नेः । \newline
38. वा अ॒ग्ने र॒ग्नेर् वै वा अ॒ग्नेः प्रि॒यम् प्रि॒य म॒ग्नेर् वै वा अ॒ग्नेः प्रि॒यम् । \newline
39. अ॒ग्नेः प्रि॒यम् प्रि॒य म॒ग्ने र॒ग्नेः प्रि॒यम् धाम॒ धाम॑ प्रि॒य म॒ग्ने र॒ग्नेः प्रि॒यम् धाम॑ । \newline
40. प्रि॒यम् धाम॒ धाम॑ प्रि॒यम् प्रि॒यम् धाम॒ यद् यद् धाम॑ प्रि॒यम् प्रि॒यम् धाम॒ यत् । \newline
41. धाम॒ यद् यद् धाम॒ धाम॒ यदाज्य॒ माज्यं॒ ॅयद् धाम॒ धाम॒ यदाज्य᳚म् । \newline
42. यदाज्य॒ माज्यं॒ ॅयद् यदाज्य॑म् प्रि॒येण॑ प्रि॒येणाज्यं॒ ॅयद् यदाज्य॑म् प्रि॒येण॑ । \newline
43. आज्य॑म् प्रि॒येण॑ प्रि॒येणाज्य॒ माज्य॑म् प्रि॒येणै॒वैव प्रि॒येणाज्य॒ माज्य॑म् प्रि॒येणै॒व । \newline
44. प्रि॒येणै॒वैव प्रि॒येण॑ प्रि॒येणै॒वैन॑ मेन मे॒व प्रि॒येण॑ प्रि॒येणै॒वैन᳚म् । \newline
45. ए॒वैन॑ मेन मे॒वैवैन॒म् धाम्ना॒ धाम्नै॑न मे॒वैवैन॒म् धाम्ना᳚ । \newline
46. ए॒न॒म् धाम्ना॒ धाम्नै॑न मेन॒म् धाम्ना॒ सꣳ सम् धाम्नै॑न मेन॒म् धाम्ना॒ सम् । \newline
47. धाम्ना॒ सꣳ सम् धाम्ना॒ धाम्ना॒ स म॑र्द्धय त्यर्द्धयति॒ सम् धाम्ना॒ धाम्ना॒ स म॑र्द्धयति । \newline
48. स म॑र्द्धय त्यर्द्धयति॒ सꣳ स म॑र्द्धय॒ त्यथो॒ अथो॑ अर्द्धयति॒ सꣳ स म॑र्द्धय॒ त्यथो᳚ । \newline
49. अ॒र्द्ध॒य॒ त्यथो॒ अथो॑ अर्द्धय त्यर्द्धय॒ त्यथो॒ तेज॑सा॒ तेज॒सा ऽथो॑ अर्द्धय त्यर्द्धय॒ त्यथो॒ तेज॑सा । \newline
50. अथो॒ तेज॑सा॒ तेज॒सा ऽथो॒ अथो॒ तेज॑सा॒ वैक॑ङ्कतीं॒ ॅवैक॑ङ्कती॒म् तेज॒सा ऽथो॒ अथो॒ तेज॑सा॒ वैक॑ङ्कतीम् । \newline
51. अथो॒ इत्यथो᳚ । \newline
52. तेज॑सा॒ वैक॑ङ्कतीं॒ ॅवैक॑ङ्कती॒म् तेज॑सा॒ तेज॑सा॒ वैक॑ङ्कती॒ मा वैक॑ङ्कती॒म् तेज॑सा॒ तेज॑सा॒ वैक॑ङ्कती॒ मा । \newline
\pagebreak
\markright{ TS 5.1.9.6  \hfill https://www.vedavms.in \hfill}

\section{ TS 5.1.9.6 }

\textbf{TS 5.1.9.6 } \newline
\textbf{Samhita Paata} \newline

वै क॑कंती॒मा द॑धाति॒ भा ए॒वाव॑ रुन्धे शमी॒मयी॒मा द॑धाति॒ शान्त्यै॒ सीद॒ त्वं मा॒तुर॒स्या उ॒पस्थ॒ इति॑ ति॒सृभि॑र्जा॒तमुप॑ तिष्ठते॒ त्रय॑ इ॒मे लो॒का ए॒ष्वे॑व लो॒केष्वा॒विदं॑ गच्छ॒त्यथो᳚ प्रा॒णाने॒वाऽऽत्मन् ध॑त्ते ॥ \newline

\textbf{Pada Paata} \newline

वैक॑ङ्कतीम् । एति॑ । द॒धा॒ति॒ । भाः । ए॒व । अवेति॑ । रु॒न्धे॒ । श॒मी॒मयी॒मिति॑ शमी - मयी᳚म् । एति॑ । द॒धा॒ति॒ । शान्त्यै᳚ । सीद॑ । त्वम् । मा॒तुः । अ॒स्याः । उ॒पस्थ॒ इत्यु॒प - स्थे॒ । इति॑ । ति॒सृभि॒रिति॑ ति॒सृ - भिः॒ । जा॒तम् । उपेति॑ । ति॒ष्ठ॒ते॒ । त्रयः॑ । इ॒मे । लो॒काः । ए॒षु । ए॒व । लो॒केषु॑ । आ॒विद॒मित्या᳚ - विद᳚म् । ग॒च्छ॒ति॒ । अथो॒ इति॑ । प्रा॒णानिति॑ प्र - अ॒नान् । ए॒व । आ॒त्मन्न् । ध॒त्ते॒ ॥  \newline


\textbf{Krama Paata} \newline

वैक॑ङ्कती॒मा । आ द॑धाति । द॒धा॒ति॒ भाः । भा ए॒व । ए॒वाव॑ । अव॑ रुन्धे । रु॒न्धे॒ श॒मी॒मयी᳚म् । 
श॒मी॒मयी॒मा । श॒मी॒मयी॒मिति॑ शमी - मयी᳚म् । आ द॑धाति । द॒धा॒ति॒ शान्त्यै᳚ । शान्त्यै॒ सीद॑ । सीद॒ त्वम् । त्वम् मा॒तुः । मा॒तुर॒स्याः । अ॒स्या उ॒पस्थे᳚ । उ॒पस्थ॒ इति॑ । उ॒पस्थ॒ इत्यु॒प - स्थे॒ । इति॑ ति॒सृभिः॑ । ति॒सृभि॑र् जा॒तम् । ति॒सृभि॒रिति॑ ति॒सृ - भिः॒ । जा॒तमुप॑ । उप॑ तिष्ठते । ति॒ष्ठ॒ते॒ त्रयः॑ । त्रय॑ इ॒मे । इ॒मे लो॒काः । लो॒का ए॒षु । ए॒ष्वे॑व । ए॒व लो॒केषु॑ । लो॒केष्वा॒विद᳚म् । आ॒विद॑म् गच्छति । आ॒विद॒मित्या᳚ - विद᳚म् । ग॒च्छ॒त्यथो᳚ । अथो᳚ प्रा॒णान् । अथो॒ इत्यथो᳚ । प्रा॒णाने॒व । प्रा॒णानिति॑ प्र - अ॒नान् । ए॒वात्मन्न् । आ॒त्मन् ध॑त्ते । ध॒त्त॒ इति॑ धत्ते । \newline

\textbf{Jatai Paata} \newline

1. वैक॑ङ्कती॒ मा वैक॑ङ्कतीं॒ ॅवैक॑ङ्कती॒ मा । \newline
2. आ द॑धाति दधा॒त्या द॑धाति । \newline
3. द॒धा॒ति॒ भा भा द॑धाति दधाति॒ भाः । \newline
4. भा ए॒वैव भा भा ए॒व । \newline
5. ए॒वावा वै॒वै वाव॑ । \newline
6. अव॑ रुन्धे रु॒न्धे ऽवाव॑ रुन्धे । \newline
7. रु॒न्धे॒ श॒मी॒मयीꣳ॑ शमी॒मयीꣳ॑ रुन्धे रुन्धे शमी॒मयी᳚म् । \newline
8. श॒मी॒मयी॒ मा श॑मी॒मयीꣳ॑ शमी॒मयी॒ मा । \newline
9. श॒मी॒मयी॒मिति॑ शमी - मयी᳚म् । \newline
10. आ द॑धाति दधा॒त्या द॑धाति । \newline
11. द॒धा॒ति॒ शान्त्यै॒ शान्त्यै॑ दधाति दधाति॒ शान्त्यै᳚ । \newline
12. शान्त्यै॒ सीद॒ सीद॒ शान्त्यै॒ शान्त्यै॒ सीद॑ । \newline
13. सीद॒ त्वम् त्वꣳ सीद॒ सीद॒ त्वम् । \newline
14. त्वम् मा॒तुर् मा॒तु स्त्वम् त्वम् मा॒तुः । \newline
15. मा॒तु र॒स्या अ॒स्या मा॒तुर् मा॒तु र॒स्याः । \newline
16. अ॒स्या उ॒पस्थ॑ उ॒पस्थे॑ अ॒स्या अ॒स्या उ॒पस्थे᳚ । \newline
17. उ॒पस्थ॒ इती त्यु॒पस्थ॑ उ॒पस्थ॒ इति॑ । \newline
18. उ॒पस्थ॒ इत्यु॒प - स्थे॒ । \newline
19. इति॑ ति॒सृभि॑ स्ति॒सृभि॒ रितीति॑ ति॒सृभिः॑ । \newline
20. ति॒सृभि॑र् जा॒तम् जा॒तम् ति॒सृभि॑ स्ति॒सृभि॑र् जा॒तम् । \newline
21. ति॒सृभि॒रिति॑ ति॒सृ - भिः॒ । \newline
22. जा॒त मुपोप॑ जा॒तम् जा॒त मुप॑ । \newline
23. उप॑ तिष्ठते तिष्ठत॒ उपोप॑ तिष्ठते । \newline
24. ति॒ष्ठ॒ते॒ त्रय॒ स्त्रय॑ स्तिष्ठते तिष्ठते॒ त्रयः॑ । \newline
25. त्रय॑ इ॒म इ॒मे त्रय॒ स्त्रय॑ इ॒मे । \newline
26. इ॒मे लो॒का लो॒का इ॒म इ॒मे लो॒काः । \newline
27. लो॒का ए॒ष्वे॑षु लो॒का लो॒का ए॒षु । \newline
28. ए॒ष्वे॑ वैवै ष्वे᳚(1॒)ष्वे॑व । \newline
29. ए॒व लो॒केषु॑ लो॒के ष्वे॒वैव लो॒केषु॑ । \newline
30. लो॒के ष्वा॒विद॑ मा॒विद॑म् ॅलो॒केषु॑ लो॒के ष्वा॒विद᳚म् । \newline
31. आ॒विद॑म् गच्छति गच्छ त्या॒विद॑ मा॒विद॑म् गच्छति । \newline
32. आ॒विद॒मित्या᳚ - विद᳚म् । \newline
33. ग॒च्छ॒ त्यथो॒ अथो॑ गच्छति गच्छ॒ त्यथो᳚ । \newline
34. अथो᳚ प्रा॒णान् प्रा॒णा नथो॒ अथो᳚ प्रा॒णान् । \newline
35. अथो॒ इत्यथो᳚ । \newline
36. प्रा॒णा ने॒वैव प्रा॒णान् प्रा॒णा ने॒व । \newline
37. प्रा॒णानिति॑ प्र - अ॒नान् । \newline
38. ए॒वात्मन् ना॒त्मन् ने॒वैवात्मन्न् । \newline
39. आ॒त्मन् ध॑त्ते धत्त आ॒त्मन् ना॒त्मन् ध॑त्ते । \newline
40. ध॒त्त॒ इति॑ धत्ते । \newline

\textbf{Ghana Paata } \newline

1. वैक॑ङ्कती॒ मा वैक॑ङ्कतीं॒ ॅवैक॑ङ्कती॒ मा द॑धाति दधा॒त्या वैक॑ङ्कतीं॒ ॅवैक॑ङ्कती॒ मा द॑धाति । \newline
2. आ द॑धाति दधा॒त्या द॑धाति॒ भा भा द॑धा॒त्या द॑धाति॒ भाः । \newline
3. द॒धा॒ति॒ भा भा द॑धाति दधाति॒ भा ए॒वैव भा द॑धाति दधाति॒ भा ए॒व । \newline
4. भा ए॒वैव भा भा ए॒वावावै॒व भा भा ए॒वाव॑ । \newline
5. ए॒वावा वै॒वै वाव॑ रुन्धे रु॒न्धे ऽवै॒वै वाव॑ रुन्धे । \newline
6. अव॑ रुन्धे रु॒न्धे ऽवाव॑ रुन्धे शमी॒मयीꣳ॑ शमी॒मयीꣳ॑ रु॒न्धे ऽवाव॑ रुन्धे शमी॒मयी᳚म् । \newline
7. रु॒न्धे॒ श॒मी॒मयीꣳ॑ शमी॒मयीꣳ॑ रुन्धे रुन्धे शमी॒मयी॒ मा श॑मी॒मयीꣳ॑ रुन्धे रुन्धे शमी॒मयी॒ मा । \newline
8. श॒मी॒मयी॒ मा श॑मी॒मयीꣳ॑ शमी॒मयी॒ मा द॑धाति दधा॒त्या श॑मी॒मयीꣳ॑ शमी॒मयी॒ मा द॑धाति । \newline
9. श॒मी॒मयी॒मिति॑ शमी - मयी᳚म् । \newline
10. आ द॑धाति दधा॒त्या द॑धाति॒ शान्त्यै॒ शान्त्यै॑ दधा॒त्या द॑धाति॒ शान्त्यै᳚ । \newline
11. द॒धा॒ति॒ शान्त्यै॒ शान्त्यै॑ दधाति दधाति॒ शान्त्यै॒ सीद॒ सीद॒ शान्त्यै॑ दधाति दधाति॒ शान्त्यै॒ सीद॑ । \newline
12. शान्त्यै॒ सीद॒ सीद॒ शान्त्यै॒ शान्त्यै॒ सीद॒ त्वम् त्वꣳ सीद॒ शान्त्यै॒ शान्त्यै॒ सीद॒ त्वम् । \newline
13. सीद॒ त्वम् त्वꣳ सीद॒ सीद॒ त्वम् मा॒तुर् मा॒तु स्त्वꣳ सीद॒ सीद॒ त्वम् मा॒तुः । \newline
14. त्वम् मा॒तुर् मा॒तु स्त्वम् त्वम् मा॒तु र॒स्या अ॒स्या मा॒तु स्त्वम् त्वम् मा॒तु र॒स्याः । \newline
15. मा॒तु र॒स्या अ॒स्या मा॒तुर् मा॒तु र॒स्या उ॒पस्थ॑ उ॒पस्थे॑ अ॒स्या मा॒तुर् मा॒तु र॒स्या उ॒पस्थे᳚ । \newline
16. अ॒स्या उ॒पस्थ॑ उ॒पस्थे॑ अ॒स्या अ॒स्या उ॒पस्थ॒ इती त्यु॒पस्थे॑ अ॒स्या अ॒स्या उ॒पस्थ॒ इति॑ । \newline
17. उ॒पस्थ॒ इती त्यु॒पस्थ॑ उ॒पस्थ॒ इति॑ ति॒सृभि॑ स्ति॒सृभि॒ रित्यु॒पस्थ॑ उ॒पस्थ॒ इति॑ ति॒सृभिः॑ । \newline
18. उ॒पस्थ॒ इत्यु॒प - स्थे॒ । \newline
19. इति॑ ति॒सृभि॑ स्ति॒सृभि॒ रितीति॑ ति॒सृभि॑र् जा॒तम् जा॒तम् ति॒सृभि॒ रितीति॑ ति॒सृभि॑र् जा॒तम् । \newline
20. ति॒सृभि॑र् जा॒तम् जा॒तम् ति॒सृभि॑ स्ति॒सृभि॑र् जा॒त मुपोप॑ जा॒तम् ति॒सृभि॑ स्ति॒सृभि॑र् जा॒त मुप॑ । \newline
21. ति॒सृभि॒रिति॑ ति॒सृ - भिः॒ । \newline
22. जा॒त मुपोप॑ जा॒तम् जा॒त मुप॑ तिष्ठते तिष्ठत॒ उप॑ जा॒तम् जा॒त मुप॑ तिष्ठते । \newline
23. उप॑ तिष्ठते तिष्ठत॒ उपोप॑ तिष्ठते॒ त्रय॒ स्त्रय॑ स्तिष्ठत॒ उपोप॑ तिष्ठते॒ त्रयः॑ । \newline
24. ति॒ष्ठ॒ते॒ त्रय॒ स्त्रय॑ स्तिष्ठते तिष्ठते॒ त्रय॑ इ॒म इ॒मे त्रय॑ स्तिष्ठते तिष्ठते॒ त्रय॑ इ॒मे । \newline
25. त्रय॑ इ॒म इ॒मे त्रय॒ स्त्रय॑ इ॒मे लो॒का लो॒का इ॒मे त्रय॒ स्त्रय॑ इ॒मे लो॒काः । \newline
26. इ॒मे लो॒का लो॒का इ॒म इ॒मे लो॒का ए॒ष्वे॑षु लो॒का इ॒म इ॒मे लो॒का ए॒षु । \newline
27. लो॒का ए॒ष्वे॑षु लो॒का लो॒का ए॒ष्वे॑ वैवैषु लो॒का लो॒का ए॒ष्वे॑व । \newline
28. ए॒ष्वे॑ वैवैष्वे᳚(1॒)ष्वे॑व लो॒केषु॑ लो॒के ष्वे॒वैष्वे᳚(1॒)ष्वे॑व लो॒केषु॑ । \newline
29. ए॒व लो॒केषु॑ लो॒के ष्वे॒वैव लो॒के ष्वा॒विद॑ मा॒विद॑म् ॅलो॒के ष्वे॒वैव लो॒के ष्वा॒विद᳚म् । \newline
30. लो॒के ष्वा॒विद॑ मा॒विद॑म् ॅलो॒केषु॑ लो॒के ष्वा॒विद॑म् गच्छति गच्छ त्या॒विद॑म् ॅलो॒केषु॑ लो॒के ष्वा॒विद॑म् गच्छति । \newline
31. आ॒विद॑म् गच्छति गच्छ त्या॒विद॑ मा॒विद॑म् गच्छ॒त्यथो॒ अथो॑ गच्छ त्या॒विद॑ मा॒विद॑म् गच्छ॒ त्यथो᳚ । \newline
32. आ॒विद॒मित्या᳚ - विद᳚म् । \newline
33. ग॒च्छ॒ त्यथो॒ अथो॑ गच्छति गच्छ॒ त्यथो᳚ प्रा॒णान् प्रा॒णा नथो॑ गच्छति गच्छ॒ त्यथो᳚ प्रा॒णान् । \newline
34. अथो᳚ प्रा॒णान् प्रा॒णा नथो॒ अथो᳚ प्रा॒णा ने॒वैव प्रा॒णा नथो॒ अथो᳚ प्रा॒णा ने॒व । \newline
35. अथो॒ इत्यथो᳚ । \newline
36. प्रा॒णा ने॒वैव प्रा॒णान् प्रा॒णा ने॒वात्मन् ना॒त्मन् ने॒व प्रा॒णान् प्रा॒णा ने॒वात्मन्न् । \newline
37. प्रा॒णानिति॑ प्र - अ॒नान् । \newline
38. ए॒वात्मन् ना॒त्मन् ने॒वैवात्मन् ध॑त्ते धत्त आ॒त्मन् ने॒वैवात्मन् ध॑त्ते । \newline
39. आ॒त्मन् ध॑त्ते धत्त आ॒त्मन् ना॒त्मन् ध॑त्ते । \newline
40. ध॒त्त॒ इति॑ धत्ते । \newline
\pagebreak
\markright{ TS 5.1.10.1  \hfill https://www.vedavms.in \hfill}

\section{ TS 5.1.10.1 }

\textbf{TS 5.1.10.1 } \newline
\textbf{Samhita Paata} \newline

न ह॑ स्म॒ वै पु॒राऽग्निरप॑रशुवृक्णं दहति॒ तद॑स्मै प्रयो॒ग ए॒वर्.षि॑रस्वदय॒द्-यद॑ग्ने॒ यानि॒ कानि॒ चेति॑ स॒मिध॒मा द॑धा॒त्यप॑रशुवृक्ण-मे॒वास्मै᳚ स्वदयति॒ सर्व॑मस्मै स्वदते॒ य ए॒वं ॅवेदौदु॑म्बरी॒मा द॑धा॒त्यूर्ग्वा उ॑दु॒म्बर॒ ऊर्ज॑मे॒वास्मा॒ अपि॑ दधाति प्र॒जाप॑तिर॒ग्नि-म॑सृजत॒ तꣳ सृ॒ष्टꣳ रक्षाꣳ॑स्य - [  ] \newline

\textbf{Pada Paata} \newline

न । ह॒ । स्म॒ । वै । पु॒रा । अ॒ग्निः । अप॑रशुवृक्ण॒मित्यप॑रशु-वृ॒क्ण॒म् । द॒ह॒ति॒ । तत् । अ॒स्मै॒ । प्र॒यो॒ग इति॑ प्र - यो॒गः । ए॒व । ऋषिः॑ । अ॒स्व॒दय॒त् । यत् । अ॒ग्ने॒ । यानि॑ । कानि॑ । च॒ । इति॑ । स॒मिध॒मिति॑ सं - इध᳚म् । एति॑ । द॒धा॒ति॒ । अप॑रशुवृक्ण॒मित्यप॑रशु-वृ॒क्ण॒म् । ए॒व । अ॒स्मै॒ । स्व॒द॒य॒ति॒ । सर्व᳚म् । अ॒स्मै॒ । स्व॒द॒ते॒ । यः । ए॒वम् । वेद॑ । औदु॑बंरीम् । एति॑ । द॒धा॒ति॒ । ऊर्क् । वै । उ॒दु॒बंरः॑ । ऊर्ज᳚म् । ए॒व । अ॒स्मै॒ । अपीति॑ । द॒धा॒ति॒ । प्र॒जाप॑ति॒रिति॑ प्र॒जा - प॒तिः॒ । अ॒ग्निम् । अ॒सृ॒ज॒त॒ । तम् । सृ॒ष्टम् । रक्षाꣳ॑सि ।  \newline


\textbf{Krama Paata} \newline

न ह॑ । ह॒ स्म॒ । स्म॒ वै । वै पु॒रा । पु॒राऽग्निः । अ॒ग्निरप॑रशुवृक्णम् । अप॑रशुवृक्णम् दहति । अप॑रशुवृक्ण॒मित्यप॑रशु - वृ॒क्ण॒म् । द॒ह॒ति॒ तत् । तद॑स्मै । अ॒स्मै॒ प्र॒यो॒गः । प्र॒यो॒ग ए॒व । प्र॒यो॒ग इति॑ प्र - यो॒गः । ए॒वर्षिः॑ । ऋषि॑रस्वदयत् । अ॒स्व॒द॒य॒द् यत् । यद॑ग्ने । अ॒ग्ने॒ यानि॑ । यानि॒ कानि॑ । कानि॑ च । चेति॑ । इति॑ स॒मिध᳚म् । स॒मिध॒मा । स॒मिध॒मिति॑ सम् - इध᳚म् । आ द॑धाति । द॒धा॒त्यप॑रशुवृक्णम् । अप॑रशुवृक्णमे॒व । अप॑रशुवृक्ण॒मित्यप॑रशु - वृ॒क्ण॒म् । ए॒वास्मै᳚ । अ॒स्मै॒ स्व॒द॒य॒ति॒ । स्व॒द॒य॒ति॒ सर्व᳚म् । सर्व॑मस्मै । अ॒स्मै॒ स्व॒द॒ते॒ । स्व॒द॒ते॒ यः । य ए॒वम् । ए॒वम् ॅवेद॑ । वेदौदु॑म्बरीम् । औदु॑म्बरी॒मा । आ द॑धाति । द॒धा॒त्यूर्क् । ऊर्ग् वै । वा उ॑दु॒म्बरः॑ । उ॒दु॒म्बर॒ ऊर्ज᳚म् । ऊर्ज॑मे॒व । ए॒वास्मै᳚ । अ॒स्मा॒ अपि॑ । अपि॑ दधाति । द॒धा॒ति॒ प्र॒जाप॑तिः । प्र॒जाप॑तिर॒ग्निम् । प्र॒जाप॑ति॒रिति॑ प्र॒जा - प॒तिः॒ । अ॒ग्निम॑सृजत । अ॒सृ॒ज॒त॒ तम् । तꣳ सृ॒ष्टम् । सृ॒ष्टꣳ रक्षाꣳ॑सि । रक्षाꣳ॑स्य,जिघाꣳसन्न् \newline

\textbf{Jatai Paata} \newline

1. न ह॑ ह॒ न न ह॑ । \newline
2. ह॒ स्म॒ स्म॒ ह॒ ह॒ स्म॒ । \newline
3. स्म॒ वै वै स्म॑ स्म॒ वै । \newline
4. वै पु॒रा पु॒रा वै वै पु॒रा । \newline
5. पु॒रा ऽग्नि र॒ग्निः पु॒रा पु॒रा ऽग्निः । \newline
6. अ॒ग्नि रप॑रशुवृक्ण॒ मप॑रशुवृक्ण म॒ग्नि र॒ग्नि रप॑रशुवृक्णम् । \newline
7. अप॑रशुवृक्णम् दहति दह॒ त्यप॑रशुवृक्ण॒ मप॑रशुवृक्णम् दहति । \newline
8. अप॑रशुवृक्ण॒मित्यप॑रशु - वृ॒क्ण॒म् । \newline
9. द॒ह॒ति॒ तत् तद् द॑हति दहति॒ तत् । \newline
10. तद॑स्मा अस्मै॒ तत् तद॑स्मै । \newline
11. अ॒स्मै॒ प्र॒यो॒गः प्र॑यो॒गो᳚ ऽस्मा अस्मै प्रयो॒गः । \newline
12. प्र॒यो॒ग ए॒वैव प्र॑यो॒गः प्र॑यो॒ग ए॒व । \newline
13. प्र॒यो॒ग इति॑ प्र - यो॒गः । \newline
14. ए॒व र्.षि॒र्॒. ऋषि॑ रे॒वैव र्.षिः॑ । \newline
15. ऋषि॑ रस्व॒दय॒ दस्व॒दय॒ दृषि॒र्॒. ऋषि॑ रस्व॒दय॒त् । \newline
16. अ॒स्व॒दय॒द् यद् यद॑स्व॒दय॒ दस्व॒दय॒द् यत् । \newline
17. यद॑ग्ने अग्ने॒ यद् यद॑ग्ने । \newline
18. अ॒ग्ने॒ यानि॒ यान्य॑ग्ने अग्ने॒ यानि॑ । \newline
19. यानि॒ कानि॒ कानि॒ यानि॒ यानि॒ कानि॑ । \newline
20. कानि॑ च च॒ कानि॒ कानि॑ च । \newline
21. चेतीति॑ च॒ चेति॑ । \newline
22. इति॑ स॒मिधꣳ॑ स॒मिध॒ मितीति॑ स॒मिध᳚म् । \newline
23. स॒मिध॒ मा स॒मिधꣳ॑ स॒मिध॒ मा । \newline
24. स॒मिध॒मिति॑ सं - इध᳚म् । \newline
25. आ द॑धाति दधा॒त्या द॑धाति । \newline
26. द॒धा॒ त्यप॑रशुवृक्ण॒ मप॑रशुवृक्णम् दधाति दधा॒ त्यप॑रशुवृक्णम् । \newline
27. अप॑रशुवृक्ण मे॒वैवा प॑रशुवृक्ण॒ मप॑रशुवृक्ण मे॒व । \newline
28. अप॑रशुवृक्ण॒मित्यप॑रशु - वृ॒क्ण॒म् । \newline
29. ए॒वास्मा॑ अस्मा ए॒वैवास्मै᳚ । \newline
30. अ॒स्मै॒ स्व॒द॒य॒ति॒ स्व॒द॒य॒ त्य॒स्मा॒ अ॒स्मै॒ स्व॒द॒य॒ति॒ । \newline
31. स्व॒द॒य॒ति॒ सर्वꣳ॒॒ सर्वꣳ॑ स्वदयति स्वदयति॒ सर्व᳚म् । \newline
32. सर्व॑ मस्मा अस्मै॒ सर्वꣳ॒॒ सर्व॑ मस्मै । \newline
33. अ॒स्मै॒ स्व॒द॒ते॒ स्व॒द॒ते॒ ऽस्मा॒ अ॒स्मै॒ स्व॒द॒ते॒ । \newline
34. स्व॒द॒ते॒ यो यः स्व॑दते स्वदते॒ यः । \newline
35. य ए॒व मे॒वं ॅयो य ए॒वम् । \newline
36. ए॒वं ॅवेद॒ वेदै॒व मे॒वं ॅवेद॑ । \newline
37. वेदौदुं॑बरी॒ मौदुं॑बरीं॒ ॅवेद॒ वेदौदुं॑बरीम् । \newline
38. औदुं॑बरी॒ मौदुं॑बरी॒ मौदुं॑बरी॒ मा । \newline
39. आ द॑धाति दधा॒त्या द॑धाति । \newline
40. द॒धा॒ त्यूर्गूर्ग् द॑धाति दधा॒ त्यूर्क् । \newline
41. ऊर्ग् वै वा ऊर्गूर्ग् वै । \newline
42. वा उ॑दुं॒बर॑ उदुं॒बरो॒ वै वा उ॑दुं॒बरः॑ । \newline
43. उ॒दुं॒बर॒ ऊर्ज॒ मूर्ज॑ मुदुं॒बर॑ उदुं॒बर॒ ऊर्ज᳚म् । \newline
44. ऊर्ज॑ मे॒वैवोर्ज॒ मूर्ज॑ मे॒व । \newline
45. ए॒वास्मा॑ अस्मा ए॒वैवास्मै᳚ । \newline
46. अ॒स्मा॒ अप्य प्य॑स्मा अस्मा॒ अपि॑ । \newline
47. अपि॑ दधाति दधा॒ त्यप्यपि॑ दधाति । \newline
48. द॒धा॒ति॒ प्र॒जाप॑तिः प्र॒जाप॑तिर् दधाति दधाति प्र॒जाप॑तिः । \newline
49. प्र॒जाप॑ति र॒ग्नि म॒ग्निम् प्र॒जाप॑तिः प्र॒जाप॑ति र॒ग्निम् । \newline
50. प्र॒जाप॑ति॒रिति॑ प्र॒जा - प॒तिः॒ । \newline
51. अ॒ग्नि म॑सृजता सृजता॒ग्नि म॒ग्नि म॑सृजत । \newline
52. अ॒सृ॒ज॒त॒ तम् त म॑सृजता सृजत॒ तम् । \newline
53. तꣳ सृ॒ष्टꣳ सृ॒ष्टम् तम् तꣳ सृ॒ष्टम् । \newline
54. सृ॒ष्टꣳ रक्षाꣳ॑सि॒ रक्षाꣳ॑सि सृ॒ष्टꣳ सृ॒ष्टꣳ रक्षाꣳ॑सि । \newline
55. रक्षाꣳ॑ स्यजिघाꣳसन् नजिघाꣳस॒न् रक्षाꣳ॑सि॒ रक्षाꣳ॑ स्यजिघाꣳसन्न् । \newline

\textbf{Ghana Paata } \newline

1. न ह॑ ह॒ न न ह॑ स्म स्म ह॒ न न ह॑ स्म । \newline
2. ह॒ स्म॒ स्म॒ ह॒ ह॒ स्म॒ वै वै स्म॑ ह ह स्म॒ वै । \newline
3. स्म॒ वै वै स्म॑ स्म॒ वै पु॒रा पु॒रा वै स्म॑ स्म॒ वै पु॒रा । \newline
4. वै पु॒रा पु॒रा वै वै पु॒रा ऽग्नि र॒ग्निः पु॒रा वै वै पु॒रा ऽग्निः । \newline
5. पु॒रा ऽग्नि र॒ग्निः पु॒रा पु॒रा ऽग्नि रप॑रशुवृक्ण॒ मप॑रशुवृक्ण म॒ग्निः पु॒रा पु॒रा ऽग्नि रप॑रशुवृक्णम् । \newline
6. अ॒ग्नि रप॑रशुवृक्ण॒ मप॑रशुवृक्ण म॒ग्नि र॒ग्नि रप॑रशुवृक्णम् दहति दह॒ त्यप॑रशुवृक्ण म॒ग्नि र॒ग्नि रप॑रशुवृक्णम् दहति । \newline
7. अप॑रशुवृक्णम् दहति दह॒ त्यप॑रशुवृक्ण॒ मप॑रशुवृक्णम् दहति॒ तत् तद् द॑ह॒ त्यप॑रशुवृक्ण॒ मप॑रशुवृक्णम् दहति॒ तत् । \newline
8. अप॑रशुवृक्ण॒मित्यप॑रशु - वृ॒क्ण॒म् । \newline
9. द॒ह॒ति॒ तत् तद् द॑हति दहति॒ तद॑स्मा अस्मै॒ तद् द॑हति दहति॒ तद॑स्मै । \newline
10. तद॑स्मा अस्मै॒ तत् तद॑स्मै प्रयो॒गः प्र॑यो॒गो᳚ ऽस्मै॒ तत् तद॑स्मै प्रयो॒गः । \newline
11. अ॒स्मै॒ प्र॒यो॒गः प्र॑यो॒गो᳚ ऽस्मा अस्मै प्रयो॒ग ए॒वैव प्र॑यो॒गो᳚ ऽस्मा अस्मै प्रयो॒ग ए॒व । \newline
12. प्र॒यो॒ग ए॒वैव प्र॑यो॒गः प्र॑यो॒ग ए॒व र्.षि॒र्॒. ऋषि॑ रे॒व प्र॑यो॒गः प्र॑यो॒ग ए॒व र्.षिः॑ । \newline
13. प्र॒यो॒ग इति॑ प्र - यो॒गः । \newline
14. ए॒व र्.षि॒र्॒. ऋषि॑ रे॒वैव र्.षि॑ रस्व॒दय॒ दस्व॒दय॒ दृषि॑ रे॒वैव र्.षि॑ रस्व॒दय॒त् । \newline
15. ऋषि॑ रस्व॒दय॒ दस्व॒दय॒ दृषि॒र्॒. ऋषि॑ रस्व॒दय॒द् यद् यद॑स्व॒दय॒ दृषि॒र्॒. ऋषि॑ रस्व॒दय॒द् यत् । \newline
16. अ॒स्व॒दय॒द् यद् यद॑स्व॒दय॒ दस्व॒दय॒द् यद॑ग्ने अग्ने॒ यद॑स्व॒दय॒ दस्व॒दय॒द् यद॑ग्ने । \newline
17. यद॑ग्ने अग्ने॒ यद् यद॑ग्ने॒ यानि॒ यान्य॑ग्ने॒ यद् यद॑ग्ने॒ यानि॑ । \newline
18. अ॒ग्ने॒ यानि॒ यान्य॑ग्ने अग्ने॒ यानि॒ कानि॒ कानि॒ यान्य॑ग्ने अग्ने॒ यानि॒ कानि॑ । \newline
19. यानि॒ कानि॒ कानि॒ यानि॒ यानि॒ कानि॑ च च॒ कानि॒ यानि॒ यानि॒ कानि॑ च । \newline
20. कानि॑ च च॒ कानि॒ कानि॒ चेतीति॑ च॒ कानि॒ कानि॒ चेति॑ । \newline
21. चेतीति॑ च॒ चेति॑ स॒मिधꣳ॑ स॒मिध॒ मिति॑ च॒ चेति॑ स॒मिध᳚म् । \newline
22. इति॑ स॒मिधꣳ॑ स॒मिध॒ मितीति॑ स॒मिध॒ मा स॒मिध॒ मितीति॑ स॒मिध॒ मा । \newline
23. स॒मिध॒ मा स॒मिधꣳ॑ स॒मिध॒ मा द॑धाति दधा॒त्या स॒मिधꣳ॑ स॒मिध॒ मा द॑धाति । \newline
24. स॒मिध॒मिति॑ सं - इध᳚म् । \newline
25. आ द॑धाति दधा॒त्या द॑धा॒ त्यप॑रशुवृक्ण॒ मप॑रशुवृक्णम् दधा॒त्या द॑धा॒ त्यप॑रशुवृक्णम् । \newline
26. द॒धा॒ त्यप॑रशुवृक्ण॒ मप॑रशुवृक्णम् दधाति दधा॒ त्यप॑रशुवृक्ण मे॒वैवा प॑रशुवृक्णम् दधाति दधा॒ त्यप॑रशुवृक्ण मे॒व । \newline
27. अप॑रशुवृक्ण मे॒वैवा प॑रशुवृक्ण॒ मप॑रशुवृक्ण मे॒वास्मा॑ अस्मा ए॒वा प॑रशुवृक्ण॒ मप॑रशुवृक्ण मे॒वास्मै᳚ । \newline
28. अप॑रशुवृक्ण॒मित्यप॑रशु - वृ॒क्ण॒म् । \newline
29. ए॒वास्मा॑ अस्मा ए॒वैवास्मै᳚ स्वदयति स्वदय त्यस्मा ए॒वैवास्मै᳚ स्वदयति । \newline
30. अ॒स्मै॒ स्व॒द॒य॒ति॒ स्व॒द॒य॒ त्य॒स्मा॒ अ॒स्मै॒ स्व॒द॒य॒ति॒ सर्वꣳ॒॒ सर्वꣳ॑ स्वदय त्यस्मा अस्मै स्वदयति॒ सर्व᳚म् । \newline
31. स्व॒द॒य॒ति॒ सर्वꣳ॒॒ सर्वꣳ॑ स्वदयति स्वदयति॒ सर्व॑ मस्मा अस्मै॒ सर्वꣳ॑ स्वदयति स्वदयति॒ सर्व॑ मस्मै । \newline
32. सर्व॑ मस्मा अस्मै॒ सर्वꣳ॒॒ सर्व॑ मस्मै स्वदते स्वदते ऽस्मै॒ सर्वꣳ॒॒ सर्व॑ मस्मै स्वदते । \newline
33. अ॒स्मै॒ स्व॒द॒ते॒ स्व॒द॒ते॒ ऽस्मा॒ अ॒स्मै॒ स्व॒द॒ते॒ यो यः स्व॑दते ऽस्मा अस्मै स्वदते॒ यः । \newline
34. स्व॒द॒ते॒ यो यः स्व॑दते स्वदते॒ य ए॒व मे॒वं ॅयः स्व॑दते स्वदते॒ य ए॒वम् । \newline
35. य ए॒व मे॒वं ॅयो य ए॒वं ॅवेद॒ वेदै॒वं ॅयो य ए॒वं ॅवेद॑ । \newline
36. ए॒वं ॅवेद॒ वेदै॒व मे॒वं ॅवेदौदुं॑बरी॒ मौदुं॑बरीं॒ ॅवेदै॒व मे॒वं ॅवेदौदुं॑बरीम् । \newline
37. वेदौदुं॑बरी॒ मौदुं॑बरीं॒ ॅवेद॒ वेदौदुं॑बरी॒ मौदुं॑बरीं॒ ॅवेद॒ वेदौदुं॑बरी॒ मा । \newline
38. औदुं॑बरी॒ मौदुं॑बरी॒ मौदुं॑बरी॒ मा द॑धाति दधा॒ त्यौदुं॑बरी॒ मौदुं॑बरी॒ मा द॑धाति । \newline
39. आ द॑धाति दधा॒त्या द॑धा॒ त्यूर्गूर्ग् द॑धा॒त्या द॑धा॒त्यूर्क् । \newline
40. द॒धा॒ त्यूर्गूर्ग् द॑धाति दधा॒त्यूर्ग् वै वा ऊर्ग् द॑धाति दधा॒ त्यूर्ग् वै । \newline
41. ऊर्ग् वै वा ऊर्गूर्ग् वा उ॑दुं॒बर॑ उदुं॒बरो॒ वा ऊर्गूर्ग् वा उ॑दुं॒बरः॑ । \newline
42. वा उ॑दुं॒बर॑ उदुं॒बरो॒ वै वा उ॑दुं॒बर॒ ऊर्ज॒ मूर्ज॑ मुदुं॒बरो॒ वै वा उ॑दुं॒बर॒ ऊर्ज᳚म् । \newline
43. उ॒दुं॒बर॒ ऊर्ज॒ मूर्ज॑ मुदुं॒बर॑ उदुं॒बर॒ ऊर्ज॑ मे॒वैवोर्ज॑ मुदुं॒बर॑ उदुं॒बर॒ ऊर्ज॑ मे॒व । \newline
44. ऊर्ज॑ मे॒वैवोर्ज॒ मूर्ज॑ मे॒वास्मा॑ अस्मा ए॒वोर्ज॒ मूर्ज॑ मे॒वास्मै᳚ । \newline
45. ए॒वास्मा॑ अस्मा ए॒वैवास्मा॒ अप्य प्य॑स्मा ए॒वैवास्मा॒ अपि॑ । \newline
46. अ॒स्मा॒ अप्यप्य॑स्मा अस्मा॒ अपि॑ दधाति दधा॒ त्यप्य॑स्मा अस्मा॒ अपि॑ दधाति । \newline
47. अपि॑ दधाति दधा॒ त्यप्यपि॑ दधाति प्र॒जाप॑तिः प्र॒जाप॑तिर् दधा॒ त्यप्यपि॑ दधाति प्र॒जाप॑तिः । \newline
48. द॒धा॒ति॒ प्र॒जाप॑तिः प्र॒जाप॑तिर् दधाति दधाति प्र॒जाप॑ति र॒ग्नि म॒ग्निम् प्र॒जाप॑तिर् दधाति दधाति प्र॒जाप॑ति र॒ग्निम् । \newline
49. प्र॒जाप॑ति र॒ग्नि म॒ग्निम् प्र॒जाप॑तिः प्र॒जाप॑ति र॒ग्नि म॑सृजता सृजता॒ग्निम् प्र॒जाप॑तिः प्र॒जाप॑ति र॒ग्नि म॑सृजत । \newline
50. प्र॒जाप॑ति॒रिति॑ प्र॒जा - प॒तिः॒ । \newline
51. अ॒ग्नि म॑सृजता सृजता॒ग्नि म॒ग्नि म॑सृजत॒ तम् त म॑सृजता॒ग्नि म॒ग्नि म॑सृजत॒ तम् । \newline
52. अ॒सृ॒ज॒त॒ तम् त म॑सृजता सृजत॒ तꣳ सृ॒ष्टꣳ सृ॒ष्टम् त म॑सृजता सृजत॒ तꣳ सृ॒ष्टम् । \newline
53. तꣳ सृ॒ष्टꣳ सृ॒ष्टम् तम् तꣳ सृ॒ष्टꣳ रक्षाꣳ॑सि॒ रक्षाꣳ॑सि सृ॒ष्टम् तम् तꣳ सृ॒ष्टꣳ रक्षाꣳ॑सि । \newline
54. सृ॒ष्टꣳ रक्षाꣳ॑सि॒ रक्षाꣳ॑सि सृ॒ष्टꣳ सृ॒ष्टꣳ रक्षाꣳ॑ स्यजिघाꣳसन् नजिघाꣳस॒न् रक्षाꣳ॑सि सृ॒ष्टꣳ सृ॒ष्टꣳ रक्षाꣳ॑ स्यजिघाꣳसन्न् । \newline
55. रक्षाꣳ॑ स्यजिघाꣳसन् नजिघाꣳस॒न् रक्षाꣳ॑सि॒ रक्षाꣳ॑ स्यजिघाꣳस॒न् थ्स सो॑ ऽजिघाꣳस॒न् रक्षाꣳ॑सि॒ रक्षाꣳ॑ स्यजिघाꣳस॒न् थ्सः । \newline
\pagebreak
\markright{ TS 5.1.10.2  \hfill https://www.vedavms.in \hfill}

\section{ TS 5.1.10.2 }

\textbf{TS 5.1.10.2 } \newline
\textbf{Samhita Paata} \newline

-जिघाꣳस॒न्थ्स ए॒तद्-रा᳚क्षो॒घ्नम॑पश्य॒त् तेन॒ वै सरक्षाꣳ॒॒स्यपा॑ऽहत॒ यद्-रा᳚क्षो॒घ्नं भव॑त्य॒ग्नेरे॒व तेन॑ जा॒ताद्-रक्षाꣳ॒॒स्यप॑ ह॒न्त्याश्व॑त्थी॒मा द॑धात्यश्व॒त्थो वै वन॒स्पती॑नाꣳ सपत्नसा॒हो विजि॑त्यै॒ वैक॑ङ्कती॒मा द॑धाति॒ भा ए॒वाव॑ रुन्धे शमी॒मयी॒मा द॑धाति॒ शान्त्यै॒ सꣳशि॑तं मे॒ ब्रह्मोदे॑षां बा॒हू अ॑तिर॒मित्यु॑त्त॒मे औदु॑म्बरी - [  ] \newline

\textbf{Pada Paata} \newline

अ॒जि॒घाꣳ॒॒स॒न्न् । सः । ए॒तत् । रा॒क्षो॒घ्नमिति॑ राक्षः-घ्नम् । अ॒प॒श्य॒त् । तेन॑ । वै । सः । रक्षाꣳ॑सि । अपेति॑ । अ॒ह॒त॒ । यत् । रा॒क्षो॒घ्नमिति॑ राक्षः - घ्नम् । भव॑ति । अ॒ग्नेः । ए॒व । तेन॑ । जा॒तात् । रक्षाꣳ॑सि । अपेति॑ । ह॒न्ति॒ । आश्व॑त्थीम् । एति॑ । द॒धा॒ति॒ । अ॒श्व॒त्थः । वै । वन॒स्पती॑नाम् । स॒प॒त्न॒सा॒ह इति॑ सपत्न - सा॒हः । विजि॑त्या॒ इति॒ वि - जि॒त्यै॒ । वैक॑ङ्कतीम् । एति॑ । द॒धा॒ति॒ । भाः । ए॒व । अवेति॑ । रु॒न्धे॒ । श॒मी॒मयी॒मिति॑ शमी - मयी᳚म् । एति॑ । द॒धा॒ति॒ । शान्त्यै᳚ । सꣳशि॑त॒मिति॒ सं-शि॒त॒म् । मे॒ । ब्रह्म॑ । उदिति॑ । ए॒षा॒म् । बा॒हू इति॑ । अ॒ति॒र॒म् । इति॑ । उ॒त्त॒मे इत्यु॑त् - त॒मे । औदु॑बंरी॒ इति॑ ।  \newline


\textbf{Krama Paata} \newline

अ॒जि॒घाꣳ॒॒स॒न्थ् सः । स ए॒तत् । ए॒तद् रा᳚क्षो॒घ्नम् । रा॒क्षो॒घ्नम॑पश्यत् । रा॒क्षो॒घ्नमिति॑ राक्षः - घ्नम् । 
अ॒प॒श्य॒त् तेन॑ । तेन॒ वै । वै सः । स रक्षाꣳ॑सि । रक्षाꣳ॒॒स्यप॑ । अपा॑हत । अ॒ह॒त॒ यत् । यद् रा᳚क्षो॒घ्नम् । रा॒क्षो॒घ्नम् भव॑ति । रा॒क्षो॒घ्नमिति॑ राक्षः - घ्नम् । भव॑त्य॒ग्नेः । अ॒ग्नेरे॒व । ए॒व तेन॑ । तेन॑ जा॒तात् । जा॒ताद् रक्षाꣳ॑सि । रक्षाꣳ॒॒स्यप॑ । अप॑ हन्ति । ह॒न्त्याश्व॑त्थीम् । आश्व॑त्थी॒मा । आ द॑धाति । द॒धा॒त्य॒श्व॒त्थः । अ॒श्व॒त्थो वै । वै वन॒स्पती॑नाम् । वन॒स्पती॑नाꣳ सपत्नसा॒हः । स॒प॒त्न॒सा॒हो विजि॑त्यै । स॒प॒त्न॒सा॒ह इति॑ सपत्न - सा॒हः । विजि॑त्यै॒ वैक॑ङ्कतीम् । विजि॑त्या॒ इति॒ वि - जि॒त्यै॒ । वैक॑ङ्कती॒मा । आ द॑धाति । द॒धा॒ति॒ भाः । भा ए॒व । ए॒वाव॑ । अव॑ रुन्धे । रु॒न्धे॒ श॒मी॒मयी᳚म् । श॒मी॒मयी॒मा । श॒मी॒मयी॒मिति॑ शमी - मयी᳚म् । आ द॑धाति । द॒धा॒ति॒ शान्त्यै᳚ । शान्त्यै॒ सꣳशि॑तम् । सꣳशि॑तम् मे । सꣳशि॑त॒मिति॒ सम् - शि॒त॒म् । मे॒ ब्रह्म॑ । ब्रह्मोत् । उदे॑षाम् । ए॒षा॒म् बा॒हू । बा॒हू अ॑तिरम् । बा॒हू इति॑ बा॒हू । अ॒ति॒र॒मिति॑ । इत्यु॑त्त॒मे । उ॒त्त॒मे औदु॑म्बरी । उ॒त्त॒मे इत्यु॑त् - त॒मे । औदु॑म्बरी वाचयति । औदु॑म्बरी॒ इत्यौदु॑म्बरी \newline

\textbf{Jatai Paata} \newline

1. अ॒जि॒घाꣳ॒॒स॒न् थ्स सो॑ ऽजिघाꣳसन् नजिघाꣳस॒न् थ्सः । \newline
2. स ए॒त दे॒तथ् स स ए॒तत् । \newline
3. ए॒तद् रा᳚क्षो॒घ्नꣳ रा᳚क्षो॒घ्न मे॒त दे॒तद् रा᳚क्षो॒घ्नम् । \newline
4. रा॒क्षो॒घ्न म॑पश्य दपश्यद् राक्षो॒घ्नꣳ रा᳚क्षो॒घ्न म॑पश्यत् । \newline
5. रा॒क्षो॒घ्नमिति॑ राक्षः - घ्नम् । \newline
6. अ॒प॒श्य॒त् तेन॒ तेना॑ पश्य दपश्य॒त् तेन॑ । \newline
7. तेन॒ वै वै तेन॒ तेन॒ वै । \newline
8. वै स स वै वै सः । \newline
9. स रक्षाꣳ॑सि॒ रक्षाꣳ॑सि॒ स स रक्षाꣳ॑सि । \newline
10. रक्षाꣳ॒॒ स्यपाप॒ रक्षाꣳ॑सि॒ रक्षाꣳ॒॒ स्यप॑ । \newline
11. अपा॑हता ह॒ता पापा॑ हत । \newline
12. अ॒ह॒त॒ यद् यद॑हता हत॒ यत् । \newline
13. यद् रा᳚क्षो॒घ्नꣳ रा᳚क्षो॒घ्नं ॅयद् यद् रा᳚क्षो॒घ्नम् । \newline
14. रा॒क्षो॒घ्नम् भव॑ति॒ भव॑ति राक्षो॒घ्नꣳ रा᳚क्षो॒घ्नम् भव॑ति । \newline
15. रा॒क्षो॒घ्नमिति॑ राक्षः - घ्नम् । \newline
16. भव॑ त्य॒ग्ने र॒ग्नेर् भव॑ति॒ भव॑ त्य॒ग्नेः । \newline
17. अ॒ग्ने रे॒वैवाग्ने र॒ग्ने रे॒व । \newline
18. ए॒व तेन॒ तेनै॒वैव तेन॑ । \newline
19. तेन॑ जा॒ताज् जा॒तात् तेन॒ तेन॑ जा॒तात् । \newline
20. जा॒ताद् रक्षाꣳ॑सि॒ रक्षाꣳ॑सि जा॒ताज् जा॒ताद् रक्षाꣳ॑सि । \newline
21. रक्षाꣳ॒॒ स्यपाप॒ रक्षाꣳ॑सि॒ रक्षाꣳ॒॒ स्यप॑ । \newline
22. अप॑ हन्ति ह॒न्त्यपाप॑ हन्ति । \newline
23. ह॒न्त्याश्व॑त्थी॒ माश्व॑त्थीꣳ हन्ति ह॒न्त्याश्व॑त्थीम् । \newline
24. आश्व॑त्थी॒ मा ऽऽश्व॑त्थी॒ माश्व॑त्थी॒ मा । \newline
25. आ द॑धाति दधा॒त्या द॑धाति । \newline
26. द॒धा॒ त्य॒श्व॒त्थो᳚ ऽश्व॒त्थो द॑धाति दधा त्यश्व॒त्थः । \newline
27. अ॒श्व॒त्थो वै वा अ॑श्व॒त्थो᳚ ऽश्व॒त्थो वै । \newline
28. वै वन॒स्पती॑नां॒ ॅवन॒स्पती॑नां॒ ॅवै वै वन॒स्पती॑नाम् । \newline
29. वन॒स्पती॑नाꣳ सपत्नसा॒हः स॑पत्नसा॒हो वन॒स्पती॑नां॒ ॅवन॒स्पती॑नाꣳ सपत्नसा॒हः । \newline
30. स॒प॒त्न॒सा॒हो विजि॑त्यै॒ विजि॑त्यै सपत्नसा॒हः स॑पत्नसा॒हो विजि॑त्यै । \newline
31. स॒प॒त्न॒सा॒ह इति॑ सपत्न - सा॒हः । \newline
32. विजि॑त्यै॒ वैक॑ङ्कतीं॒ ॅवैक॑ङ्कतीं॒ ॅविजि॑त्यै॒ विजि॑त्यै॒ वैक॑ङ्कतीम् । \newline
33. विजि॑त्या॒ इति॒ वि - जि॒त्यै॒ । \newline
34. वैक॑ङ्कती॒ मा वैक॑ङ्कतीं॒ ॅवैक॑ङ्कती॒ मा । \newline
35. आ द॑धाति दधा॒त्या द॑धाति । \newline
36. द॒धा॒ति॒ भा भा द॑धाति दधाति॒ भाः । \newline
37. भा ए॒वैव भा भा ए॒व । \newline
38. ए॒वावा वै॒वै वाव॑ । \newline
39. अव॑ रुन्धे रु॒न्धे ऽवाव॑ रुन्धे । \newline
40. रु॒न्धे॒ श॒मी॒मयीꣳ॑ शमी॒मयीꣳ॑ रुन्धे रुन्धे शमी॒मयी᳚म् । \newline
41. श॒मी॒मयी॒ मा श॑मी॒मयीꣳ॑ शमी॒मयी॒ मा । \newline
42. श॒मी॒मयी॒मिति॑ शमी - मयी᳚म् । \newline
43. आ द॑धाति दधा॒त्या द॑धाति । \newline
44. द॒धा॒ति॒ शान्त्यै॒ शान्त्यै॑ दधाति दधाति॒ शान्त्यै᳚ । \newline
45. शान्त्यै॒ सꣳशि॑तꣳ॒॒ सꣳशि॑तꣳ॒॒ शान्त्यै॒ शान्त्यै॒ सꣳशि॑तम् । \newline
46. सꣳशि॑तम् मे मे॒ सꣳशि॑तꣳ॒॒ सꣳशि॑तम् मे । \newline
47. सꣳशि॑त॒मिति॒ सं - शि॒त॒म् । \newline
48. मे॒ ब्रह्म॒ ब्रह्म॑ मे मे॒ ब्रह्म॑ । \newline
49. ब्रह्मोदुद् ब्रह्म॒ ब्रह्मोत् । \newline
50. उदे॑षा मेषा॒ मुदुदे॑षाम् । \newline
51. ए॒षा॒म् बा॒हू बा॒हू ए॑षा मेषाम् बा॒हू । \newline
52. बा॒हू अ॑तिर मतिरम् बा॒हू बा॒हू अ॑तिरम् । \newline
53. बा॒हू इति॑ बा॒हू । \newline
54. अ॒ति॒र॒ मिती त्य॑तिर मतिर॒ मिति॑ । \newline
55. इत्यु॑त्त॒मे उ॑त्त॒मे इती त्यु॑त्त॒मे । \newline
56. उ॒त्त॒मे औदुं॑बरी॒ औदुं॑बरी उत्त॒मे उ॑त्त॒मे औदुं॑बरी । \newline
57. उ॒त्त॒मे इत्यु॑त् - त॒मे । \newline
58. औदुं॑बरी वाचयति वाचय॒ त्यौदुं॑बरी॒ औदुं॑बरी वाचयति । \newline
59. औदुं॑बरी॒ इत्यौदुं॑बरी । \newline

\textbf{Ghana Paata } \newline

1. अ॒जि॒घाꣳ॒॒स॒न् थ्स सो॑ ऽजिघाꣳसन् नजिघाꣳस॒न् थ्स ए॒त दे॒तथ् सो॑ ऽजिघाꣳसन् नजिघाꣳस॒न् थ्स ए॒तत् । \newline
2. स ए॒त दे॒तथ् स स ए॒तद् रा᳚क्षो॒घ्नꣳ रा᳚क्षो॒घ्न मे॒तथ् स स ए॒तद् रा᳚क्षो॒घ्नम् । \newline
3. ए॒तद् रा᳚क्षो॒घ्नꣳ रा᳚क्षो॒घ्न मे॒त दे॒तद् रा᳚क्षो॒घ्न म॑पश्य दपश्यद् राक्षो॒घ्न मे॒त दे॒तद् रा᳚क्षो॒घ्न म॑पश्यत् । \newline
4. रा॒क्षो॒घ्न म॑पश्य दपश्यद् राक्षो॒घ्नꣳ रा᳚क्षो॒घ्न म॑पश्य॒त् तेन॒ तेना॑ पश्यद् राक्षो॒घ्नꣳ रा᳚क्षो॒घ्न म॑पश्य॒त् तेन॑ । \newline
5. रा॒क्षो॒घ्नमिति॑ राक्षः - घ्नम् । \newline
6. अ॒प॒श्य॒त् तेन॒ तेना॑पश्य दपश्य॒त् तेन॒ वै वै तेना॑पश्य दपश्य॒त् तेन॒ वै । \newline
7. तेन॒ वै वै तेन॒ तेन॒ वै स स वै तेन॒ तेन॒ वै सः । \newline
8. वै स स वै वै स रक्षाꣳ॑सि॒ रक्षाꣳ॑सि॒ स वै वै स रक्षाꣳ॑सि । \newline
9. स रक्षाꣳ॑सि॒ रक्षाꣳ॑सि॒ स स रक्षाꣳ॒॒ स्यपाप॒ रक्षाꣳ॑सि॒ स स रक्षाꣳ॒॒ स्यप॑ । \newline
10. रक्षाꣳ॒॒ स्यपाप॒ रक्षाꣳ॑सि॒ रक्षाꣳ॒॒ स्यपा॑हता ह॒ताप॒ रक्षाꣳ॑सि॒ रक्षाꣳ॒॒ स्यपा॑हत । \newline
11. अपा॑हता ह॒तापापा॑ हत॒ यद् यद॑ह॒ता पापा॑ हत॒ यत् । \newline
12. अ॒ह॒त॒ यद् यद॑हता हत॒ यद् रा᳚क्षो॒घ्नꣳ रा᳚क्षो॒घ्नं ॅयद॑हता हत॒ यद् रा᳚क्षो॒घ्नम् । \newline
13. यद् रा᳚क्षो॒घ्नꣳ रा᳚क्षो॒घ्नं ॅयद् यद् रा᳚क्षो॒घ्नम् भव॑ति॒ भव॑ति राक्षो॒घ्नं ॅयद् यद् रा᳚क्षो॒घ्नम् भव॑ति । \newline
14. रा॒क्षो॒घ्नम् भव॑ति॒ भव॑ति राक्षो॒घ्नꣳ रा᳚क्षो॒घ्नम् भव॑ त्य॒ग्ने र॒ग्नेर् भव॑ति राक्षो॒घ्नꣳ रा᳚क्षो॒घ्नम् भव॑ त्य॒ग्नेः । \newline
15. रा॒क्षो॒घ्नमिति॑ राक्षः - घ्नम् । \newline
16. भव॑ त्य॒ग्ने र॒ग्नेर् भव॑ति॒ भव॑ त्य॒ग्ने रे॒वैवाग्नेर् भव॑ति॒ भव॑ त्य॒ग्ने रे॒व । \newline
17. अ॒ग्ने रे॒वैवाग्ने र॒ग्ने रे॒व तेन॒ तेनै॒वाग्ने र॒ग्ने रे॒व तेन॑ । \newline
18. ए॒व तेन॒ तेनै॒वैव तेन॑ जा॒ताज् जा॒तात् तेनै॒वैव तेन॑ जा॒तात् । \newline
19. तेन॑ जा॒ताज् जा॒तात् तेन॒ तेन॑ जा॒ताद् रक्षाꣳ॑सि॒ रक्षाꣳ॑सि जा॒तात् तेन॒ तेन॑ जा॒ताद् रक्षाꣳ॑सि । \newline
20. जा॒ताद् रक्षाꣳ॑सि॒ रक्षाꣳ॑सि जा॒ताज् जा॒ताद् रक्षाꣳ॒॒ स्यपाप॒ रक्षाꣳ॑सि जा॒ताज् जा॒ताद् रक्षाꣳ॒॒ स्यप॑ । \newline
21. रक्षाꣳ॒॒ स्यपाप॒ रक्षाꣳ॑सि॒ रक्षाꣳ॒॒ स्यप॑ हन्ति ह॒न्त्यप॒ रक्षाꣳ॑सि॒ रक्षाꣳ॒॒ स्यप॑ हन्ति । \newline
22. अप॑ हन्ति ह॒न्त्यपाप॑ ह॒न्त्या श्व॑त्थी॒ माश्व॑त्थीꣳ ह॒न्त्यपाप॑ ह॒न्त्या श्व॑त्थीम् । \newline
23. ह॒न्त्या श्व॑त्थी॒ माश्व॑त्थीꣳ हन्ति ह॒न्त्या श्व॑त्थी॒ मा ऽऽश्व॑त्थीꣳ हन्ति ह॒न्त्या श्व॑त्थी॒ मा । \newline
24. आश्व॑त्थी॒ मा ऽऽश्व॑त्थी॒ माश्व॑त्थी॒ मा द॑धाति दधा॒त्या ऽऽश्व॑त्थी॒ माश्व॑त्थी॒ मा द॑धाति । \newline
25. आ द॑धाति दधा॒त्या द॑धा त्यश्व॒त्थो᳚ ऽश्व॒त्थो द॑धा॒त्या द॑धा त्यश्व॒त्थः । \newline
26. द॒धा॒ त्य॒श्व॒त्थो᳚ ऽश्व॒त्थो द॑धाति दधा त्यश्व॒त्थो वै वा अ॑श्व॒त्थो द॑धाति दधा त्यश्व॒त्थो वै । \newline
27. अ॒श्व॒त्थो वै वा अ॑श्व॒त्थो᳚ ऽश्व॒त्थो वै वन॒स्पती॑नां॒ ॅवन॒स्पती॑नां॒ ॅवा अ॑श्व॒त्थो᳚ ऽश्व॒त्थो वै वन॒स्पती॑नाम् । \newline
28. वै वन॒स्पती॑नां॒ ॅवन॒स्पती॑नां॒ ॅवै वै वन॒स्पती॑नाꣳ सपत्नसा॒हः स॑पत्नसा॒हो वन॒स्पती॑नां॒ ॅवै वै वन॒स्पती॑नाꣳ सपत्नसा॒हः । \newline
29. वन॒स्पती॑नाꣳ सपत्नसा॒हः स॑पत्नसा॒हो वन॒स्पती॑नां॒ ॅवन॒स्पती॑नाꣳ सपत्नसा॒हो विजि॑त्यै॒ विजि॑त्यै सपत्नसा॒हो वन॒स्पती॑नां॒ ॅवन॒स्पती॑नाꣳ सपत्नसा॒हो विजि॑त्यै । \newline
30. स॒प॒त्न॒सा॒हो विजि॑त्यै॒ विजि॑त्यै सपत्नसा॒हः स॑पत्नसा॒हो विजि॑त्यै॒ वैक॑ङ्कतीं॒ ॅवैक॑ङ्कतीं॒ ॅविजि॑त्यै सपत्नसा॒हः स॑पत्नसा॒हो विजि॑त्यै॒ वैक॑ङ्कतीम् । \newline
31. स॒प॒त्न॒सा॒ह इति॑ सपत्न - सा॒हः । \newline
32. विजि॑त्यै॒ वैक॑ङ्कतीं॒ ॅवैक॑ङ्कतीं॒ ॅविजि॑त्यै॒ विजि॑त्यै॒ वैक॑ङ्कती॒ मा वैक॑ङ्कतीं॒ ॅविजि॑त्यै॒ विजि॑त्यै॒ वैक॑ङ्कती॒ मा । \newline
33. विजि॑त्या॒ इति॒ वि - जि॒त्यै॒ । \newline
34. वैक॑ङ्कती॒ मा वैक॑ङ्कतीं॒ ॅवैक॑ङ्कती॒ मा द॑धाति दधा॒त्या वैक॑ङ्कतीं॒ ॅवैक॑ङ्कती॒ मा द॑धाति । \newline
35. आ द॑धाति दधा॒त्या द॑धाति॒ भा भा द॑धा॒त्या द॑धाति॒ भाः । \newline
36. द॒धा॒ति॒ भा भा द॑धाति दधाति॒ भा ए॒वैव भा द॑धाति दधाति॒ भा ए॒व । \newline
37. भा ए॒वैव भा भा ए॒वावा वै॒व भा भा ए॒वाव॑ । \newline
38. ए॒वावा वै॒वै वाव॑ रुन्धे रु॒न्धे ऽवै॒वै वाव॑ रुन्धे । \newline
39. अव॑ रुन्धे रु॒न्धे ऽवाव॑ रुन्धे शमी॒मयीꣳ॑ शमी॒मयीꣳ॑ रु॒न्धे ऽवाव॑ रुन्धे शमी॒मयी᳚म् । \newline
40. रु॒न्धे॒ श॒मी॒मयीꣳ॑ शमी॒मयीꣳ॑ रुन्धे रुन्धे शमी॒मयी॒ मा श॑मी॒मयीꣳ॑ रुन्धे रुन्धे शमी॒मयी॒ मा । \newline
41. श॒मी॒मयी॒ मा श॑मी॒मयीꣳ॑ शमी॒मयी॒ मा द॑धाति दधा॒त्या श॑मी॒मयीꣳ॑ शमी॒मयी॒ मा द॑धाति । \newline
42. श॒मी॒मयी॒मिति॑ शमी - मयी᳚म् । \newline
43. आ द॑धाति दधा॒त्या द॑धाति॒ शान्त्यै॒ शान्त्यै॑ दधा॒त्या द॑धाति॒ शान्त्यै᳚ । \newline
44. द॒धा॒ति॒ शान्त्यै॒ शान्त्यै॑ दधाति दधाति॒ शान्त्यै॒ सꣳशि॑तꣳ॒॒ सꣳशि॑तꣳ॒॒ शान्त्यै॑ दधाति दधाति॒ शान्त्यै॒ सꣳशि॑तम् । \newline
45. शान्त्यै॒ सꣳशि॑तꣳ॒॒ सꣳशि॑तꣳ॒॒ शान्त्यै॒ शान्त्यै॒ सꣳशि॑तम् मे मे॒ सꣳशि॑तꣳ॒॒ शान्त्यै॒ शान्त्यै॒ सꣳशि॑तम् मे । \newline
46. सꣳशि॑तम् मे मे॒ सꣳशि॑तꣳ॒॒ सꣳशि॑तम् मे॒ ब्रह्म॒ ब्रह्म॑ मे॒ सꣳशि॑तꣳ॒॒ सꣳशि॑तम् मे॒ ब्रह्म॑ । \newline
47. सꣳशि॑त॒मिति॒ सं - शि॒त॒म् । \newline
48. मे॒ ब्रह्म॒ ब्रह्म॑ मे मे॒ ब्रह्मोदुद् ब्रह्म॑ मे मे॒ ब्रह्मोत् । \newline
49. ब्रह्मोदुद् ब्रह्म॒ ब्रह्मोदे॑षा मेषा॒ मुद् ब्रह्म॒ ब्रह्मोदे॑षाम् । \newline
50. उदे॑षा मेषा॒ मुदु दे॑षाम् बा॒हू बा॒हू ए॑षा॒ मुदु दे॑षाम् बा॒हू । \newline
51. ए॒षा॒म् बा॒हू बा॒हू ए॑षा मेषाम् बा॒हू अ॑तिर मतिरम् बा॒हू ए॑षा मेषाम् बा॒हू अ॑तिरम् । \newline
52. बा॒हू अ॑तिर मतिरम् बा॒हू बा॒हू अ॑तिर॒ मिती त्य॑तिरम् बा॒हू बा॒हू अ॑तिर॒ मिति॑ । \newline
53. बा॒हू इति॑ बा॒हू । \newline
54. अ॒ति॒र॒ मिती त्य॑तिर मतिर॒ मित्यु॑त्त॒मे उ॑त्त॒मे इत्य॑तिर मतिर॒ मित्यु॑त्त॒मे । \newline
55. इत्यु॑त्त॒मे उ॑त्त॒मे इती त्यु॑त्त॒मे औदुं॑बरी॒ औदुं॑बरी उत्त॒मे इती त्यु॑त्त॒मे औदुं॑बरी । \newline
56. उ॒त्त॒मे औदुं॑बरी॒ औदुं॑बरी उत्त॒मे उ॑त्त॒मे औदुं॑बरी वाचयति वाचय॒ त्यौदुं॑बरी उत्त॒मे उ॑त्त॒मे औदुं॑बरी वाचयति । \newline
57. उ॒त्त॒मे इत्यु॑त् - त॒मे । \newline
58. औदुं॑बरी वाचयति वाचय॒ त्यौदुं॑बरी॒ औदुं॑बरी वाचयति॒ ब्रह्म॑णा॒ ब्रह्म॑णा वाचय॒ त्यौदुं॑बरी॒ औदुं॑बरी वाचयति॒ ब्रह्म॑णा । \newline
59. औदुं॑बरी॒ इत्यौदुं॑बरी । \newline
\pagebreak
\markright{ TS 5.1.10.3  \hfill https://www.vedavms.in \hfill}

\section{ TS 5.1.10.3 }

\textbf{TS 5.1.10.3 } \newline
\textbf{Samhita Paata} \newline

वाचयति॒ ब्रह्म॑णै॒व क्ष॒त्रꣳ सꣳ श्य॑ति क्ष॒त्रेण॒ ब्रह्म॒ तस्मा᳚द् ब्राह्म॒णो रा॑ज॒न्य॑वा॒नत्य॒न्यं ब्रा᳚ह्म॒णं तस्मा᳚द्-राज॒न्यो᳚ ब्राह्म॒णवा॒नत्य॒न्यꣳ रा॑ज॒न्यं॑ मृ॒त्युर्वा ए॒ष यद॒ग्निर॒मृतꣳ॒॒ हिर॑ण्यꣳ रु॒क्ममन्त॑रं॒ प्रति॑मुञ्चते॒ ऽमृत॑मे॒व मृ॒त्योर॒न्तर्द्ध॑त्त॒ एक॑विꣳशतिनिर्बाधो भव॒त्येक॑विꣳशति॒र्वै दे॑वलो॒का द्वाद॑श॒ मासाः॒ पञ्च॒र्तव॒स्त्रय॑ इ॒मे लो॒का अ॒सावा॑दि॒त्य - [  ] \newline

\textbf{Pada Paata} \newline

वा॒च॒य॒ति॒ । ब्रह्म॑णा । ए॒व । क्ष॒त्रम् । समिति॑ । श्य॒ति॒ । क्ष॒त्रेण॑ । ब्रह्म॑ । तस्मा᳚त् । ब्रा॒ह्म॒णः । रा॒ज॒न्य॑वा॒निति॑ राज॒न्य॑ - वा॒न् । अतीति॑ । अ॒न्यम् । ब्रा॒ह्म॒णम् । तस्मा᳚त् । रा॒ज॒न्यः॑ । ब्रा॒ह्म॒णवा॒निति॑ ब्राह्म॒ण - वा॒न् । अतीति॑ । अ॒न्यम् । रा॒ज॒न्य᳚म् । मृ॒त्युः । वै । ए॒षः । यत् । अ॒ग्निः । अ॒मृत᳚म् । हिर॑ण्यम् । रु॒क्मम् । अन्त॑रम् । प्रतीति॑ । मु॒ञ्च॒ते॒ । अ॒मृत᳚म् । ए॒व । मृ॒त्योः । अ॒न्तः । ध॒त्ते॒ । एक॑विꣳशतिनिर्बाध॒ इत्येक॑विꣳशति - नि॒र्बा॒धः॒ । भ॒व॒ति॒ । एक॑विꣳशति॒रित्येक॑-विꣳ॒॒श॒तिः॒ । वै । दे॒व॒लो॒का इति॑ देव-लो॒काः । द्वाद॑श । मासाः᳚ । पञ्च॑ । ऋ॒तवः॑ । त्रयः॑ । इ॒मे । लो॒काः । अ॒सौ । आ॒दि॒त्यः ।  \newline


\textbf{Krama Paata} \newline

वा॒च॒य॒ति॒ ब्रह्म॑णा । ब्रह्म॑णै॒व । ए॒व क्ष॒त्रम् । क्ष॒त्रꣳ सम् । सꣳश्य॑ति । श्य॒ति॒ क्ष॒त्रेण॑ । क्ष॒त्रेण॒ ब्रह्म॑ । ब्रह्म॒ तस्मा᳚त् । तस्मा᳚द् ब्राह्म॒णः । ब्रा॒ह्म॒णो रा॑ज॒न्य॑वान् । रा॒ज॒न्य॑वा॒नति॑ । रा॒ज॒न्य॑वा॒निति॑ राज॒न्य॑ - वा॒न्॒ । अत्य॒न्यम् । अ॒न्यम् ब्रा᳚ह्म॒णम् । ब्रा॒ह्म॒णम् तस्मा᳚त् । तस्मा᳚द् राज॒न्यः॑ । रा॒ज॒न्यो᳚ ब्राह्म॒णवान्॑ । ब्रा॒ह्म॒णवा॒नति॑ । ब्रा॒ह्म॒णवा॒निति॑ ब्राह्म॒ण - वा॒न्॒ । अत्य॒न्यम् । अ॒न्यꣳ रा॑ज॒न्य᳚म् । रा॒ज॒न्य॑म् मृ॒त्युः । मृ॒त्युर् वै । वा ए॒षः । ए॒ष यत् । यद॒ग्निः । अ॒ग्निर॒मृत᳚म् । अ॒मृतꣳ॒॒ हिर॑ण्यम् । हिर॑ण्यꣳ रु॒क्मम् । रु॒क्ममन्त॑रम् । अन्त॑र॒म् प्रति॑ । प्रति॑ मुञ्चते । मु॒ञ्च॒ते॒ऽमृत᳚म् । अ॒मृत॑मे॒व । ए॒व मृ॒त्योः । मृ॒त्योर॒न्तः । अ॒न्तर् ध॑त्ते । ध॒त्त॒ एक॑विꣳशतिनिर्बाधः । एक॑विꣳशतिनिर्बाधो भवति । एक॑विꣳशतिनिर्बाध॒ इत्येक॑विꣳशति - नि॒र्बा॒धः॒ । भ॒व॒त्येक॑विꣳशतिः । एक॑विꣳशति॒र् वै । एक॑विꣳशति॒रित्येक॑ - विꣳ॒॒श॒तिः॒ । वै दे॑वलो॒काः । दे॒व॒लो॒का द्वाद॑श । दे॒व॒लो॒का इति॑ देव - लो॒काः । द्वाद॑श॒ मासाः᳚ । मासाः॒ पञ्च॑ । पञ्च॒र्तवः॑ । ऋ॒तव॒स्त्रयः॑ । त्रय॑ इ॒मे । इ॒मे लो॒काः । लो॒का अ॒सौ । अ॒सावा॑दि॒त्यः । आ॒दि॒त्य ए॑कविꣳ॒॒शः \newline

\textbf{Jatai Paata} \newline

1. वा॒च॒य॒ति॒ ब्रह्म॑णा॒ ब्रह्म॑णा वाचयति वाचयति॒ ब्रह्म॑णा । \newline
2. ब्रह्म॑णै॒वैव ब्रह्म॑णा॒ ब्रह्म॑णै॒व । \newline
3. ए॒व क्ष॒त्रम् क्ष॒त्र मे॒वैव क्ष॒त्रम् । \newline
4. क्ष॒त्रꣳ सꣳ सम् क्ष॒त्रम् क्ष॒त्रꣳ सम् । \newline
5. सꣳ श्य॑ति श्यति॒ सꣳ सꣳ श्य॑ति । \newline
6. श्य॒ति॒ क्ष॒त्रेण॑ क्ष॒त्रेण॑ श्यति श्यति क्ष॒त्रेण॑ । \newline
7. क्ष॒त्रेण॒ ब्रह्म॒ ब्रह्म॑ क्ष॒त्रेण॑ क्ष॒त्रेण॒ ब्रह्म॑ । \newline
8. ब्रह्म॒ तस्मा॒त् तस्मा॒द् ब्रह्म॒ ब्रह्म॒ तस्मा᳚त् । \newline
9. तस्मा᳚द् ब्राह्म॒णो ब्रा᳚ह्म॒ण स्तस्मा॒त् तस्मा᳚द् ब्राह्म॒णः । \newline
10. ब्रा॒ह्म॒णो रा॑ज॒न्य॑वान् राज॒न्य॑वान् ब्राह्म॒णो ब्रा᳚ह्म॒णो रा॑ज॒न्य॑वान् । \newline
11. रा॒ज॒न्य॑वा॒ नत्यति॑ राज॒न्य॑वान् राज॒न्य॑वा॒ नति॑ । \newline
12. रा॒ज॒न्य॑ वा॒निति॑ राज॒न्य॑ - वा॒न् । \newline
13. अत्य॒न्य म॒न्य मत्य त्य॒न्यम् । \newline
14. अ॒न्यम् ब्रा᳚ह्म॒णम् ब्रा᳚ह्म॒ण म॒न्य म॒न्यम् ब्रा᳚ह्म॒णम् । \newline
15. ब्रा॒ह्म॒णम् तस्मा॒त् तस्मा᳚द् ब्राह्म॒णम् ब्रा᳚ह्म॒णम् तस्मा᳚त् । \newline
16. तस्मा᳚द् राज॒न्यो॑ राज॒न्य॑ स्तस्मा॒त् तस्मा᳚द् राज॒न्यः॑ । \newline
17. रा॒ज॒न्यो᳚ ब्राह्म॒णवा᳚न् ब्राह्म॒णवा᳚न् राज॒न्यो॑ राज॒न्यो᳚ ब्राह्म॒णवान्॑ । \newline
18. ब्रा॒ह्म॒णवा॒ नत्यति॑ ब्राह्म॒णवा᳚न् ब्राह्म॒णवा॒ नति॑ । \newline
19. ब्रा॒ह्म॒णवा॒निति॑ ब्राह्म॒ण - वा॒न् । \newline
20. अत्य॒न्य म॒न्य मत्य त्य॒न्यम् । \newline
21. अ॒न्यꣳ रा॑ज॒न्यꣳ॑ राज॒न्य॑ म॒न्य म॒न्यꣳ रा॑ज॒न्य᳚म् । \newline
22. रा॒ज॒न्य॑म् मृ॒त्युर् मृ॒त्यू रा॑ज॒न्यꣳ॑ राज॒न्य॑म् मृ॒त्युः । \newline
23. मृ॒त्युर् वै वै मृ॒त्युर् मृ॒त्युर् वै । \newline
24. वा ए॒ष ए॒ष वै वा ए॒षः । \newline
25. ए॒ष यद् यदे॒ष ए॒ष यत् । \newline
26. यद॒ग्नि र॒ग्निर् यद् यद॒ग्निः । \newline
27. अ॒ग्नि र॒मृत॑ म॒मृत॑ म॒ग्नि र॒ग्नि र॒मृत᳚म् । \newline
28. अ॒मृतꣳ॒॒ हिर॑ण्यꣳ॒॒ हिर॑ण्य म॒मृत॑ म॒मृतꣳ॒॒ हिर॑ण्यम् । \newline
29. हिर॑ण्यꣳ रु॒क्मꣳ रु॒क्मꣳ हिर॑ण्यꣳ॒॒ हिर॑ण्यꣳ रु॒क्मम् । \newline
30. रु॒क्म मन्त॑र॒ मन्त॑रꣳ रु॒क्मꣳ रु॒क्म मन्त॑रम् । \newline
31. अन्त॑र॒म् प्रति॒ प्रत्यन्त॑र॒ मन्त॑र॒म् प्रति॑ । \newline
32. प्रति॑ मुञ्चते मुञ्चते॒ प्रति॒ प्रति॑ मुञ्चते । \newline
33. मु॒ञ्च॒ते॒ ऽमृत॑ म॒मृत॑म् मुञ्चते मुञ्चते॒ ऽमृत᳚म् । \newline
34. अ॒मृत॑ मे॒वैवामृत॑ म॒मृत॑ मे॒व । \newline
35. ए॒व मृ॒त्योर् मृ॒त्यो रे॒वैव मृ॒त्योः । \newline
36. मृ॒त्यो र॒न्त र॒न्तर् मृ॒त्योर् मृ॒त्यो र॒न्तः । \newline
37. अ॒न्तर् ध॑त्ते धत्ते॒ ऽन्त र॒न्तर् ध॑त्ते । \newline
38. ध॒त्त॒ एक॑विꣳशतिनिर्बाध॒ एक॑विꣳशतिनिर्बाधो धत्ते धत्त॒ एक॑विꣳशतिनिर्बाधः । \newline
39. एक॑विꣳशतिनिर्बाधो भवति भव॒ त्येक॑विꣳशतिनिर्बाध॒ एक॑विꣳशतिनिर्बाधो भवति । \newline
40. एक॑विꣳशतिनिर्बाध॒ इत्येक॑विꣳशति - नि॒र्बा॒धः॒ । \newline
41. भ॒व॒त्येक॑विꣳशति॒ रेक॑विꣳशतिर् भवति भव॒त्येक॑विꣳशतिः । \newline
42. एक॑विꣳशति॒र् वै वा एक॑विꣳशति॒ रेक॑विꣳशति॒र् वै । \newline
43. एक॑विꣳशति॒रित्येक॑ - विꣳ॒॒श॒तिः॒ । \newline
44. वै दे॑वलो॒का दे॑वलो॒का वै वै दे॑वलो॒काः । \newline
45. दे॒व॒लो॒का द्वाद॑श॒ द्वाद॑श देवलो॒का दे॑वलो॒का द्वाद॑श । \newline
46. दे॒व॒लो॒का इति॑ देव - लो॒काः । \newline
47. द्वाद॑श॒ मासा॒ मासा॒ द्वाद॑श॒ द्वाद॑श॒ मासाः᳚ । \newline
48. मासाः॒ पञ्च॒ पञ्च॒ मासा॒ मासाः॒ पञ्च॑ । \newline
49. पञ्च॒ र्‌तव॑ ऋ॒तवः॒ पञ्च॒ पञ्च॒ र्‌तवः॑ । \newline
50. ऋ॒तव॒ स्त्रय॒ स्त्रय॑ ऋ॒तव॑ ऋ॒तव॒ स्त्रयः॑ । \newline
51. त्रय॑ इ॒म इ॒मे त्रय॒ स्त्रय॑ इ॒मे । \newline
52. इ॒मे लो॒का लो॒का इ॒म इ॒मे लो॒काः । \newline
53. लो॒का अ॒सा व॒सौ लो॒का लो॒का अ॒सौ । \newline
54. अ॒सा वा॑दि॒त्य आ॑दि॒त्यो॑ ऽसा व॒सा वा॑दि॒त्यः । \newline
55. आ॒दि॒त्य ए॑कविꣳ॒॒श ए॑कविꣳ॒॒श आ॑दि॒त्य आ॑दि॒त्य ए॑कविꣳ॒॒शः । \newline

\textbf{Ghana Paata } \newline

1. वा॒च॒य॒ति॒ ब्रह्म॑णा॒ ब्रह्म॑णा वाचयति वाचयति॒ ब्रह्म॑णै॒वैव ब्रह्म॑णा वाचयति वाचयति॒ ब्रह्म॑णै॒व । \newline
2. ब्रह्म॑णै॒वैव ब्रह्म॑णा॒ ब्रह्म॑णै॒व क्ष॒त्रम् क्ष॒त्र मे॒व ब्रह्म॑णा॒ ब्रह्म॑णै॒व क्ष॒त्रम् । \newline
3. ए॒व क्ष॒त्रम् क्ष॒त्र मे॒वैव क्ष॒त्रꣳ सꣳ सम् क्ष॒त्र मे॒वैव क्ष॒त्रꣳ सम् । \newline
4. क्ष॒त्रꣳ सꣳ सम् क्ष॒त्रम् क्ष॒त्रꣳ सꣳ श्य॑ति श्यति॒ सम् क्ष॒त्रम् क्ष॒त्रꣳ सꣳ श्य॑ति । \newline
5. सꣳ श्य॑ति श्यति॒ सꣳ सꣳ श्य॑ति क्ष॒त्रेण॑ क्ष॒त्रेण॑ श्यति॒ सꣳ सꣳ श्य॑ति क्ष॒त्रेण॑ । \newline
6. श्य॒ति॒ क्ष॒त्रेण॑ क्ष॒त्रेण॑ श्यति श्यति क्ष॒त्रेण॒ ब्रह्म॒ ब्रह्म॑ क्ष॒त्रेण॑ श्यति श्यति क्ष॒त्रेण॒ ब्रह्म॑ । \newline
7. क्ष॒त्रेण॒ ब्रह्म॒ ब्रह्म॑ क्ष॒त्रेण॑ क्ष॒त्रेण॒ ब्रह्म॒ तस्मा॒त् तस्मा॒द् ब्रह्म॑ क्ष॒त्रेण॑ क्ष॒त्रेण॒ ब्रह्म॒ तस्मा᳚त् । \newline
8. ब्रह्म॒ तस्मा॒त् तस्मा॒द् ब्रह्म॒ ब्रह्म॒ तस्मा᳚द् ब्राह्म॒णो ब्रा᳚ह्म॒ण स्तस्मा॒द् ब्रह्म॒ ब्रह्म॒ तस्मा᳚द् ब्राह्म॒णः । \newline
9. तस्मा᳚द् ब्राह्म॒णो ब्रा᳚ह्म॒ण स्तस्मा॒त् तस्मा᳚द् ब्राह्म॒णो रा॑ज॒न्य॑वान् राज॒न्य॑वान् ब्राह्म॒ण स्तस्मा॒त् तस्मा᳚द् ब्राह्म॒णो रा॑ज॒न्य॑वान् । \newline
10. ब्रा॒ह्म॒णो रा॑ज॒न्य॑वान् राज॒न्य॑वान् ब्राह्म॒णो ब्रा᳚ह्म॒णो रा॑ज॒न्य॑वा॒ नत्यति॑ राज॒न्य॑वान् ब्राह्म॒णो ब्रा᳚ह्म॒णो रा॑ज॒न्य॑वा॒ नति॑ । \newline
11. रा॒ज॒न्य॑वा॒ नत्यति॑ राज॒न्य॑वान् राज॒न्य॑वा॒ नत्य॒न्य म॒न्य मति॑ राज॒न्य॑वान् राज॒न्य॑वा॒ नत्य॒न्यम् । \newline
12. रा॒ज॒न्य॑वा॒निति॑ राज॒न्य॑ - वा॒न् । \newline
13. अत्य॒न्य म॒न्य मत्य त्य॒न्यम् ब्रा᳚ह्म॒णम् ब्रा᳚ह्म॒ण म॒न्य मत्य त्य॒न्यम् ब्रा᳚ह्म॒णम् । \newline
14. अ॒न्यम् ब्रा᳚ह्म॒णम् ब्रा᳚ह्म॒ण म॒न्य म॒न्यम् ब्रा᳚ह्म॒णम् तस्मा॒त् तस्मा᳚द् ब्राह्म॒ण म॒न्य म॒न्यम् ब्रा᳚ह्म॒णम् तस्मा᳚त् । \newline
15. ब्रा॒ह्म॒णम् तस्मा॒त् तस्मा᳚द् ब्राह्म॒णम् ब्रा᳚ह्म॒णम् तस्मा᳚द् राज॒न्यो॑ राज॒न्य॑ स्तस्मा᳚द् ब्राह्म॒णम् ब्रा᳚ह्म॒णम् तस्मा᳚द् राज॒न्यः॑ । \newline
16. तस्मा᳚द् राज॒न्यो॑ राज॒न्य॑ स्तस्मा॒त् तस्मा᳚द् राज॒न्यो᳚ ब्राह्म॒णवा᳚न् ब्राह्म॒णवा᳚न् राज॒न्य॑ स्तस्मा॒त् तस्मा᳚द् राज॒न्यो᳚ ब्राह्म॒णवान्॑ । \newline
17. रा॒ज॒न्यो᳚ ब्राह्म॒णवा᳚न् ब्राह्म॒णवा᳚न् राज॒न्यो॑ राज॒न्यो᳚ ब्राह्म॒णवा॒ नत्यति॑ ब्राह्म॒णवा᳚न् राज॒न्यो॑ राज॒न्यो᳚ ब्राह्म॒णवा॒ नति॑ । \newline
18. ब्रा॒ह्म॒णवा॒ नत्यति॑ ब्राह्म॒णवा᳚न् ब्राह्म॒णवा॒ नत्य॒न्य म॒न्य मति॑ ब्राह्म॒णवा᳚न् ब्राह्म॒णवा॒ नत्य॒न्यम् । \newline
19. ब्रा॒ह्म॒णवा॒निति॑ ब्राह्म॒ण - वा॒न् । \newline
20. अत्य॒न्य म॒न्य मत्य त्य॒न्यꣳ रा॑ज॒न्यꣳ॑ राज॒न्य॑ म॒न्य मत्य त्य॒न्यꣳ रा॑ज॒न्य᳚म् । \newline
21. अ॒न्यꣳ रा॑ज॒न्यꣳ॑ राज॒न्य॑ म॒न्य म॒न्यꣳ रा॑ज॒न्य॑म् मृ॒त्युर् मृ॒त्यू रा॑ज॒न्य॑ म॒न्य म॒न्यꣳ रा॑ज॒न्य॑म् मृ॒त्युः । \newline
22. रा॒ज॒न्य॑म् मृ॒त्युर् मृ॒त्यू रा॑ज॒न्यꣳ॑ राज॒न्य॑म् मृ॒त्युर् वै वै मृ॒त्यू रा॑ज॒न्यꣳ॑ राज॒न्य॑म् मृ॒त्युर् वै । \newline
23. मृ॒त्युर् वै वै मृ॒त्युर् मृ॒त्युर् वा ए॒ष ए॒ष वै मृ॒त्युर् मृ॒त्युर् वा ए॒षः । \newline
24. वा ए॒ष ए॒ष वै वा ए॒ष यद् यदे॒ष वै वा ए॒ष यत् । \newline
25. ए॒ष यद् यदे॒ष ए॒ष यद॒ग्नि र॒ग्निर् यदे॒ष ए॒ष यद॒ग्निः । \newline
26. यद॒ग्नि र॒ग्निर् यद् यद॒ग्नि र॒मृत॑ म॒मृत॑ म॒ग्निर् यद् यद॒ग्नि र॒मृत᳚म् । \newline
27. अ॒ग्नि र॒मृत॑ म॒मृत॑ म॒ग्नि र॒ग्नि र॒मृतꣳ॒॒ हिर॑ण्यꣳ॒॒ हिर॑ण्य म॒मृत॑ म॒ग्नि र॒ग्नि र॒मृतꣳ॒॒ हिर॑ण्यम् । \newline
28. अ॒मृतꣳ॒॒ हिर॑ण्यꣳ॒॒ हिर॑ण्य म॒मृत॑ म॒मृतꣳ॒॒ हिर॑ण्यꣳ रु॒क्मꣳ रु॒क्मꣳ हिर॑ण्य म॒मृत॑ म॒मृतꣳ॒॒ हिर॑ण्यꣳ रु॒क्मम् । \newline
29. हिर॑ण्यꣳ रु॒क्मꣳ रु॒क्मꣳ हिर॑ण्यꣳ॒॒ हिर॑ण्यꣳ रु॒क्म मन्त॑र॒ मन्त॑रꣳ रु॒क्मꣳ हिर॑ण्यꣳ॒॒ हिर॑ण्यꣳ रु॒क्म मन्त॑रम् । \newline
30. रु॒क्म मन्त॑र॒ मन्त॑रꣳ रु॒क्मꣳ रु॒क्म मन्त॑र॒म् प्रति॒ प्रत्यन्त॑रꣳ रु॒क्मꣳ रु॒क्म मन्त॑र॒म् प्रति॑ । \newline
31. अन्त॑र॒म् प्रति॒ प्रत्यन्त॑र॒ मन्त॑र॒म् प्रति॑ मुञ्चते मुञ्चते॒ प्रत्यन्त॑र॒ मन्त॑र॒म् प्रति॑ मुञ्चते । \newline
32. प्रति॑ मुञ्चते मुञ्चते॒ प्रति॒ प्रति॑ मुञ्चते॒ ऽमृत॑ म॒मृत॑म् मुञ्चते॒ प्रति॒ प्रति॑ मुञ्चते॒ ऽमृत᳚म् । \newline
33. मु॒ञ्च॒ते॒ ऽमृत॑ म॒मृत॑म् मुञ्चते मुञ्चते॒ ऽमृत॑ मे॒वैवामृत॑म् मुञ्चते मुञ्चते॒ ऽमृत॑ मे॒व । \newline
34. अ॒मृत॑ मे॒वैवामृत॑ म॒मृत॑ मे॒व मृ॒त्योर् मृ॒त्यो रे॒वामृत॑ म॒मृत॑ मे॒व मृ॒त्योः । \newline
35. ए॒व मृ॒त्योर् मृ॒त्यो रे॒वैव मृ॒त्यो र॒न्त र॒न्तर् मृ॒त्यो रे॒वैव मृ॒त्यो र॒न्तः । \newline
36. मृ॒त्यो र॒न्त र॒न्तर् मृ॒त्योर् मृ॒त्यो र॒न्तर् ध॑त्ते धत्ते॒ ऽन्तर् मृ॒त्योर् मृ॒त्यो र॒न्तर् ध॑त्ते । \newline
37. अ॒न्तर् ध॑त्ते धत्ते॒ ऽन्त र॒न्तर् ध॑त्त॒ एक॑विꣳशतिनिर्बाध॒ एक॑विꣳशतिनिर्बाधो धत्ते॒ ऽन्त र॒न्तर् ध॑त्त॒ एक॑विꣳशतिनिर्बाधः । \newline
38. ध॒त्त॒ एक॑विꣳशतिनिर्बाध॒ एक॑विꣳशतिनिर्बाधो धत्ते धत्त॒ एक॑विꣳशतिनिर्बाधो भवति भव॒ त्येक॑विꣳशतिनिर्बाधो धत्ते धत्त॒ एक॑विꣳशतिनिर्बाधो भवति । \newline
39. एक॑विꣳशतिनिर्बाधो भवति भव॒ त्येक॑विꣳशतिनिर्बाध॒ एक॑विꣳशतिनिर्बाधो भव॒ त्येक॑विꣳशति॒रेक॑विꣳशतिर् भव॒ त्येक॑विꣳशतिनिर्बाध॒ एक॑विꣳशतिनिर्बाधो भव॒ त्येक॑विꣳशतिः । \newline
40. एक॑विꣳशतिनिर्बाध॒ इत्येक॑विꣳशति - नि॒र्बा॒धः॒ । \newline
41. भ॒व॒त्येक॑विꣳशति॒ रेक॑विꣳशतिर् भवति भव॒ त्येक॑विꣳशति॒र् वै वा एक॑विꣳशतिर् भवति भव॒ त्येक॑विꣳशति॒र् वै । \newline
42. एक॑विꣳशति॒र् वै वा एक॑विꣳशति॒ रेक॑विꣳशति॒र् वै दे॑वलो॒का दे॑वलो॒का वा एक॑विꣳशति॒ रेक॑विꣳशति॒र् वै दे॑वलो॒काः । \newline
43. एक॑विꣳशति॒रित्येक॑ - विꣳ॒॒श॒तिः॒ । \newline
44. वै दे॑वलो॒का दे॑वलो॒का वै वै दे॑वलो॒का द्वाद॑श॒ द्वाद॑श देवलो॒का वै वै दे॑वलो॒का द्वाद॑श । \newline
45. दे॒व॒लो॒का द्वाद॑श॒ द्वाद॑श देवलो॒का दे॑वलो॒का द्वाद॑श॒ मासा॒ मासा॒ द्वाद॑श देवलो॒का दे॑वलो॒का द्वाद॑श॒ मासाः᳚ । \newline
46. दे॒व॒लो॒का इति॑ देव - लो॒काः । \newline
47. द्वाद॑श॒ मासा॒ मासा॒ द्वाद॑श॒ द्वाद॑श॒ मासाः॒ पञ्च॒ पञ्च॒ मासा॒ द्वाद॑श॒ द्वाद॑श॒ मासाः॒ पञ्च॑ । \newline
48. मासाः॒ पञ्च॒ पञ्च॒ मासा॒ मासाः॒ पञ्च॒ र्‌तव॑ ऋ॒तवः॒ पञ्च॒ मासा॒ मासाः॒ पञ्च॒ र्‌तवः॑ । \newline
49. पञ्च॒ र्‌तव॑ ऋ॒तवः॒ पञ्च॒ पञ्च॒ र्‌तव॒ स्त्रय॒ स्त्रय॑ ऋ॒तवः॒ पञ्च॒ पञ्च॒ र्‌तव॒ स्त्रयः॑ । \newline
50. ऋ॒तव॒ स्त्रय॒ स्त्रय॑ ऋ॒तव॑ ऋ॒तव॒ स्त्रय॑ इ॒म इ॒मे त्रय॑ ऋ॒तव॑ ऋ॒तव॒ स्त्रय॑ इ॒मे । \newline
51. त्रय॑ इ॒म इ॒मे त्रय॒ स्त्रय॑ इ॒मे लो॒का लो॒का इ॒मे त्रय॒ स्त्रय॑ इ॒मे लो॒काः । \newline
52. इ॒मे लो॒का लो॒का इ॒म इ॒मे लो॒का अ॒सा व॒सौ लो॒का इ॒म इ॒मे लो॒का अ॒सौ । \newline
53. लो॒का अ॒सा व॒सौ लो॒का लो॒का अ॒सा वा॑दि॒त्य आ॑दि॒त्यो॑ ऽसौ लो॒का लो॒का अ॒सा वा॑दि॒त्यः । \newline
54. अ॒सा वा॑दि॒त्य आ॑दि॒त्यो॑ ऽसा व॒सा वा॑दि॒त्य ए॑कविꣳ॒॒श ए॑कविꣳ॒॒श आ॑दि॒त्यो॑ ऽसा व॒सा वा॑दि॒त्य ए॑कविꣳ॒॒शः । \newline
55. आ॒दि॒त्य ए॑कविꣳ॒॒श ए॑कविꣳ॒॒श आ॑दि॒त्य आ॑दि॒त्य ए॑कविꣳ॒॒श ए॒ताव॑न्त ए॒ताव॑न्त एकविꣳ॒॒श आ॑दि॒त्य आ॑दि॒त्य ए॑कविꣳ॒॒श ए॒ताव॑न्तः । \newline
\pagebreak
\markright{ TS 5.1.10.4  \hfill https://www.vedavms.in \hfill}

\section{ TS 5.1.10.4 }

\textbf{TS 5.1.10.4 } \newline
\textbf{Samhita Paata} \newline

ए॑कविꣳ॒॒श ए॒ताव॑न्तो॒ वै दे॑वलो॒कास्तेभ्य॑ ए॒व भ्रातृ॑व्यम॒न्तरे॑ति निर्बा॒धैर्वै दे॒वा असु॑रान् निर्बा॒धे॑ऽकुर्वत॒ तन्नि॑र्बा॒धानां᳚ निर्बाध॒त्वं नि॑र्बा॒धी भ॑वति॒ भ्रातृ॑व्याने॒व नि॑र्बा॒धे कु॑रुते सावित्रि॒या प्रति॑मुञ्चते॒ प्रसू᳚त्यै॒ नक्तो॒षासेत्युत्त॑रया ऽहोरा॒त्राभ्या॑मे॒वैन॒-मुद्य॑च्छते दे॒वा अ॒ग्निं धा॑रयन् द्रविणो॒दा इत्या॑ह प्रा॒णा वै दे॒वा द्र॑विणो॒दा अ॑होरा॒त्राभ्या॑मे॒वैन॑मु॒द्यत्य॑ - [  ] \newline

\textbf{Pada Paata} \newline

ए॒क॒विꣳ॒॒श इत्ये॑क - विꣳ॒॒शः । ए॒ताव॑न्तः । वै । दे॒व॒लो॒का इति॑ देव - लो॒काः । तेभ्यः॑ । ए॒व । भ्रातृ॑व्यम् । अ॒न्तः । ए॒ति॒ । नि॒र्बा॒धैरिति॑ निः - बा॒धैः । वै । दे॒वाः । असु॑रान् । नि॒र्बा॒ध इति॑ निः - बा॒धे । अ॒कु॒र्व॒त॒ । तत् । नि॒र्बा॒धाना॒मिति॑ निः - बा॒धाना᳚म् । नि॒र्बा॒ध॒त्वमिति॑ निर्बाध - त्वम् । नि॒र्बा॒धीति॑ निः - बा॒धी । भ॒व॒ति॒ । भ्रातृ॑व्यान् । ए॒व । नि॒र्बा॒ध इति॑ निः - बा॒धे । कु॒रु॒ते॒ । सा॒वि॒त्रि॒या । प्रतीति॑ । मु॒ञ्च॒ते॒ । प्रसू᳚त्या॒ इति॒ प्र - सू॒त्यै॒ । नक्तो॒षासा᳚ । इति॑ । उत्त॑र॒येत्युत् - त॒र॒या॒ । अ॒हो॒रा॒त्राभ्या॒मित्य॑हः - रा॒त्राभ्या᳚म् । ए॒व । ए॒न॒म् । उदिति॑ । य॒च्छ॒ते॒ । दे॒वाः । अ॒ग्निम् । धा॒र॒य॒न्न् । द्र॒वि॒णो॒दा इति॑ द्रविणः - दाः । इति॑ । आ॒ह॒ । प्रा॒णा इति॑ प्र - अ॒नाः । वै । दे॒वाः । द्र॒वि॒णो॒दा इति॑ द्रविणः - दाः । अ॒हो॒रा॒त्राभ्या॒मित्य॑हः - रा॒त्राभ्या᳚म् । ए॒व । ए॒न॒म् । उ॒द्यत्येतु॑त्-यत्य॑ ।  \newline


\textbf{Krama Paata} \newline

ए॒क॒विꣳ॒॒श ए॒ताव॑न्तः । ए॒क॒विꣳ॒॒श इत्ये॑क - विꣳ॒॒शः । ए॒ताव॑न्तो॒ वै । वै दे॑वलो॒काः । दे॒व॒लो॒कास्तेभ्यः॑ । दे॒व॒लो॒का इति॑ देव - लो॒काः । तेभ्य॑ ए॒व । ए॒व भ्रातृ॑व्यम् । भ्रातृ॑व्यम॒न्तः । अ॒न्तरे॑ति । ए॒ति॒ नि॒र्बा॒धैः । नि॒र्बा॒धैर् वै । नि॒र्बा॒धैरिति॑ निः - बा॒धैः । वै दे॒वाः । दे॒वा असु॑रान् । असु॑रान् निर्बा॒धे । नि॒र्बा॒धे॑ऽकुर्वत । नि॒र्बा॒ध इति॑ निः - बा॒धे । अ॒कु॒र्व॒त॒ तत् । तन् नि॑र्बा॒धाना᳚म् । नि॒र्बा॒धाना᳚म् निर्बाध॒त्वम् । नि॒र्बा॒धाना॒मिति॑ निः - बा॒धाना᳚म् । नि॒र्बा॒ध॒त्वम् नि॑र्बा॒धी । नि॒र्बा॒ध॒त्वमिति॑ निर्बाध - त्वम् । नि॒र्बा॒धी भ॑वति । नि॒र्बा॒धीति॑ निः - बा॒धी । भ॒व॒ति॒ भ्रातृ॑व्यान् । भ्रातृ॑व्याने॒व । ए॒व नि॑र्बा॒धे । नि॒र्बा॒धे कु॑रुते । नि॒र्बा॒ध इति॑ निः - बा॒धे । कु॒रु॒ते॒ सा॒वि॒त्रि॒या । सा॒वि॒त्रि॒या प्रति॑ । प्रति॑ मुञ्चते । मु॒ञ्च॒ते॒ प्रसू᳚त्यै । प्रसू᳚त्यै॒ नक्तो॒षासा᳚ । प्रसू᳚त्या॒ इति॒ प्र - सू॒त्यै॒ । नक्तो॒षासेति॑ । इत्युत्त॑रया । उत्त॑रयाऽहोरा॒त्राभ्या᳚म् । उत्त॑र॒येत्युत् - त॒र॒या॒ । अ॒हो॒रा॒त्राभ्या॑मे॒व । अ॒हो॒रा॒त्राभ्या॒मित्य॑हः - रा॒त्राभ्या᳚म् । ए॒वैन᳚म् । ए॒न॒मुत् । उद् य॑च्छते । य॒च्छ॒ते॒ दे॒वाः । दे॒वा अ॒ग्निम् । अ॒ग्निम् धा॑रयन्न् । धा॒र॒य॒न् द्र॒वि॒णो॒दाः । द्र॒वि॒णो॒दा इति॑ । द्र॒वि॒णो॒दा इति॑ द्रविणः - दाः । इत्या॑ह । आ॒ह॒ प्रा॒णाः । प्रा॒णा वै । प्रा॒णा इति॑ प्र - अ॒नाः । वै दे॒वाः । दे॒वा द्र॑विणो॒दाः । द्र॒वि॒णो॒दा अ॑होरा॒त्राभ्या᳚म् । द्र॒वि॒णो॒दा इति॑ द्रविणः - दाः । अ॒हो॒रा॒त्राभ्या॑मे॒व । अ॒हो॒रा॒त्राभ्या॒मित्य॑हः - रा॒त्राभ्या᳚म् । ए॒वैन᳚म् । ए॒न॒मु॒द्यत्य॑ । उ॒द्यत्य॑ प्रा॒णैः । उ॒द्यत्येत्यु॑त् - यत्य॑ \newline

\textbf{Jatai Paata} \newline

1. ए॒क॒विꣳ॒॒श ए॒ताव॑न्त ए॒ताव॑न्त एकविꣳ॒॒श ए॑कविꣳ॒॒श ए॒ताव॑न्तः । \newline
2. ए॒क॒विꣳ॒॒श इत्ये॑क - विꣳ॒॒शः । \newline
3. ए॒ताव॑न्तो॒ वै वा ए॒ताव॑न्त ए॒ताव॑न्तो॒ वै । \newline
4. वै दे॑वलो॒का दे॑वलो॒का वै वै दे॑वलो॒काः । \newline
5. दे॒व॒लो॒का स्तेभ्य॒ स्तेभ्यो॑ देवलो॒का दे॑वलो॒का स्तेभ्यः॑ । \newline
6. दे॒व॒लो॒का इति॑ देव - लो॒काः । \newline
7. तेभ्य॑ ए॒वैव तेभ्य॒ स्तेभ्य॑ ए॒व । \newline
8. ए॒व भ्रातृ॑व्य॒म् भ्रातृ॑व्य मे॒वैव भ्रातृ॑व्यम् । \newline
9. भ्रातृ॑व्य म॒न्त र॒न्तर् भ्रातृ॑व्य॒म् भ्रातृ॑व्य म॒न्तः । \newline
10. अ॒न्त रे᳚त्ये त्य॒न्त र॒न्त रे॑ति । \newline
11. ए॒ति॒ नि॒र्बा॒धैर् नि॑र्बा॒धै रे᳚त्येति निर्बा॒धैः । \newline
12. नि॒र्बा॒धैर् वै वै नि॑र्बा॒धैर् नि॑र्बा॒धैर् वै । \newline
13. नि॒र्बा॒धैरिति॑ निः - बा॒धैः । \newline
14. वै दे॒वा दे॒वा वै वै दे॒वाः । \newline
15. दे॒वा असु॑रा॒ नसु॑रान् दे॒वा दे॒वा असु॑रान् । \newline
16. असु॑रान् निर्बा॒धे नि॑र्बा॒धे ऽसु॑रा॒ नसु॑रान् निर्बा॒धे । \newline
17. नि॒र्बा॒धे॑ ऽकुर्वता कुर्वत निर्बा॒धे नि॑र्बा॒धे॑ ऽकुर्वत । \newline
18. नि॒र्बा॒ध इति॑ निः - बा॒धे । \newline
19. अ॒कु॒र्व॒त॒ तत् तद॑कुर्वता कुर्वत॒ तत् । \newline
20. तन् नि॑र्बा॒धाना᳚म् निर्बा॒धाना॒म् तत् तन् नि॑र्बा॒धाना᳚म् । \newline
21. नि॒र्बा॒धाना᳚म् निर्बाध॒त्वम् नि॑र्बाध॒त्वम् नि॑र्बा॒धाना᳚म् निर्बा॒धाना᳚म् निर्बाध॒त्वम् । \newline
22. नि॒र्बा॒धाना॒मिति॑ निः - बा॒धाना᳚म् । \newline
23. नि॒र्बा॒ध॒त्वम् नि॑र्बा॒धी नि॑र्बा॒धी नि॑र्बाध॒त्वम् नि॑र्बाध॒त्वम् नि॑र्बा॒धी । \newline
24. नि॒र्बा॒ध॒त्वमिति॑ निर्बाध - त्वम् । \newline
25. नि॒र्बा॒धी भ॑वति भवति निर्बा॒धी नि॑र्बा॒धी भ॑वति । \newline
26. नि॒र्बा॒धीति॑ निः - बा॒धी । \newline
27. भ॒व॒ति॒ भ्रातृ॑व्या॒न् भ्रातृ॑व्यान् भवति भवति॒ भ्रातृ॑व्यान् । \newline
28. भ्रातृ॑व्या ने॒वैव भ्रातृ॑व्या॒न् भ्रातृ॑व्या ने॒व । \newline
29. ए॒व नि॑र्बा॒धे नि॑र्बा॒ध ए॒वैव नि॑र्बा॒धे । \newline
30. नि॒र्बा॒धे कु॑रुते कुरुते निर्बा॒धे नि॑र्बा॒धे कु॑रुते । \newline
31. नि॒र्बा॒ध इति॑ निः - बा॒धे । \newline
32. कु॒रु॒ते॒ सा॒वि॒त्रि॒या सा॑वित्रि॒या कु॑रुते कुरुते सावित्रि॒या । \newline
33. सा॒वि॒त्रि॒या प्रति॒ प्रति॑ सावित्रि॒या सा॑वित्रि॒या प्रति॑ । \newline
34. प्रति॑ मुञ्चते मुञ्चते॒ प्रति॒ प्रति॑ मुञ्चते । \newline
35. मु॒ञ्च॒ते॒ प्रसू᳚त्यै॒ प्रसू᳚त्यै मुञ्चते मुञ्चते॒ प्रसू᳚त्यै । \newline
36. प्रसू᳚त्यै॒ नक्तो॒षासा॒ नक्तो॒षासा॒ प्रसू᳚त्यै॒ प्रसू᳚त्यै॒ नक्तो॒षासा᳚ । \newline
37. प्रसू᳚त्या॒ इति॒ प्र - सू॒त्यै॒ । \newline
38. नक्तो॒षासेतीति॒ नक्तो॒षासा॒ नक्तो॒षासेति॑ । \newline
39. इत्युत्त॑र॒ योत्त॑र॒येती त्युत्त॑रया । \newline
40. उत्त॑रया ऽहोरा॒त्राभ्या॑ महोरा॒त्राभ्या॒ मुत्त॑र॒ योत्त॑रया ऽहोरा॒त्राभ्या᳚म् । \newline
41. उत्त॑र॒येत्युत् - त॒र॒या॒ । \newline
42. अ॒हो॒रा॒त्राभ्या॑ मे॒वैवा हो॑रा॒त्राभ्या॑ महोरा॒त्राभ्या॑ मे॒व । \newline
43. अ॒हो॒रा॒त्राभ्या॒मित्य॑हः - रा॒त्राभ्या᳚म् । \newline
44. ए॒वैन॑ मेन मे॒वैवैन᳚म् । \newline
45. ए॒न॒ मुदु दे॑न मेन॒ मुत् । \newline
46. उद् य॑च्छते यच्छत॒ उदुद् य॑च्छते । \newline
47. य॒च्छ॒ते॒ दे॒वा दे॒वा य॑च्छते यच्छते दे॒वाः । \newline
48. दे॒वा अ॒ग्नि म॒ग्निम् दे॒वा दे॒वा अ॒ग्निम् । \newline
49. अ॒ग्निम् धा॑रयन् धारयन् न॒ग्नि म॒ग्निम् धा॑रयन्न् । \newline
50. धा॒र॒य॒न् द्र॒वि॒णो॒दा द्र॑विणो॒दा धा॑रयन् धारयन् द्रविणो॒दाः । \newline
51. द्र॒वि॒णो॒दा इतीति॑ द्रविणो॒दा द्र॑विणो॒दा इति॑ । \newline
52. द्र॒वि॒णो॒दा इति॑ द्रविणः - दाः । \newline
53. इत्या॑हा॒हे तीत्या॑ह । \newline
54. आ॒ह॒ प्रा॒णाः प्रा॒णा आ॑हाह प्रा॒णाः । \newline
55. प्रा॒णा वै वै प्रा॒णाः प्रा॒णा वै । \newline
56. प्रा॒णा इति॑ प्र - अ॒नाः । \newline
57. वै दे॒वा दे॒वा वै वै दे॒वाः । \newline
58. दे॒वा द्र॑विणो॒दा द्र॑विणो॒दा दे॒वा दे॒वा द्र॑विणो॒दाः । \newline
59. द्र॒वि॒णो॒दा अ॑होरा॒त्राभ्या॑ महोरा॒त्राभ्या᳚म् द्रविणो॒दा द्र॑विणो॒दा अ॑होरा॒त्राभ्या᳚म् । \newline
60. द्र॒वि॒णो॒दा इति॑ द्रविणः - दाः । \newline
61. अ॒हो॒रा॒त्राभ्या॑ मे॒वैवा हो॑रा॒त्राभ्या॑ महोरा॒त्राभ्या॑ मे॒व । \newline
62. अ॒हो॒रा॒त्राभ्या॒मित्य॑हः - रा॒त्राभ्या᳚म् । \newline
63. ए॒वैन॑ मेन मे॒वैवैन᳚म् । \newline
64. ए॒न॒ मु॒द्यत्यो॒द्य त्यै॑न मेन मु॒द्यत्य॑ । \newline
65. उ॒द्यत्य॑ प्रा॒णैः प्रा॒णै रु॒द्य त्यो॒द्यत्य॑ प्रा॒णैः । \newline
66. उ॒द्यत्येतु॑त् - यत्य॑ । \newline

\textbf{Ghana Paata } \newline

1. ए॒क॒विꣳ॒॒श ए॒ताव॑न्त ए॒ताव॑न्त एकविꣳ॒॒श ए॑कविꣳ॒॒श ए॒ताव॑न्तो॒ वै वा ए॒ताव॑न्त एकविꣳ॒॒श ए॑कविꣳ॒॒श ए॒ताव॑न्तो॒ वै । \newline
2. ए॒क॒विꣳ॒॒श इत्ये॑क - विꣳ॒॒शः । \newline
3. ए॒ताव॑न्तो॒ वै वा ए॒ताव॑न्त ए॒ताव॑न्तो॒ वै दे॑वलो॒का दे॑वलो॒का वा ए॒ताव॑न्त ए॒ताव॑न्तो॒ वै दे॑वलो॒काः । \newline
4. वै दे॑वलो॒का दे॑वलो॒का वै वै दे॑वलो॒का स्तेभ्य॒ स्तेभ्यो॑ देवलो॒का वै वै दे॑वलो॒का स्तेभ्यः॑ । \newline
5. दे॒व॒लो॒का स्तेभ्य॒ स्तेभ्यो॑ देवलो॒का दे॑वलो॒का स्तेभ्य॑ ए॒वैव तेभ्यो॑ देवलो॒का दे॑वलो॒का स्तेभ्य॑ ए॒व । \newline
6. दे॒व॒लो॒का इति॑ देव - लो॒काः । \newline
7. तेभ्य॑ ए॒वैव तेभ्य॒ स्तेभ्य॑ ए॒व भ्रातृ॑व्य॒म् भ्रातृ॑व्य मे॒व तेभ्य॒ स्तेभ्य॑ ए॒व भ्रातृ॑व्यम् । \newline
8. ए॒व भ्रातृ॑व्य॒म् भ्रातृ॑व्य मे॒वैव भ्रातृ॑व्य म॒न्त र॒न्तर् भ्रातृ॑व्य मे॒वैव भ्रातृ॑व्य म॒न्तः । \newline
9. भ्रातृ॑व्य म॒न्त र॒न्तर् भ्रातृ॑व्य॒म् भ्रातृ॑व्य म॒न्त रे᳚त्ये त्य॒न्तर् भ्रातृ॑व्य॒म् भ्रातृ॑व्य म॒न्त रे॑ति । \newline
10. अ॒न्त रे᳚त्ये त्य॒न्त र॒न्त रे॑ति निर्बा॒धैर् नि॑र्बा॒धै रे᳚त्य॒न्त र॒न्त रे॑ति निर्बा॒धैः । \newline
11. ए॒ति॒ नि॒र्बा॒धैर् नि॑र्बा॒धै रे᳚त्येति निर्बा॒धैर् वै वै नि॑र्बा॒धै रे᳚त्येति निर्बा॒धैर् वै । \newline
12. नि॒र्बा॒धैर् वै वै नि॑र्बा॒धैर् नि॑र्बा॒धैर् वै दे॒वा दे॒वा वै नि॑र्बा॒धैर् नि॑र्बा॒धैर् वै दे॒वाः । \newline
13. नि॒र्बा॒धैरिति॑ निः - बा॒धैः । \newline
14. वै दे॒वा दे॒वा वै वै दे॒वा असु॑रा॒ नसु॑रान् दे॒वा वै वै दे॒वा असु॑रान् । \newline
15. दे॒वा असु॑रा॒ नसु॑रान् दे॒वा दे॒वा असु॑रान् निर्बा॒धे नि॑र्बा॒धे ऽसु॑रान् दे॒वा दे॒वा असु॑रान् निर्बा॒धे । \newline
16. असु॑रान् निर्बा॒धे नि॑र्बा॒धे ऽसु॑रा॒ नसु॑रान् निर्बा॒धे॑ ऽकुर्वता कुर्वत निर्बा॒धे ऽसु॑रा॒ नसु॑रान् निर्बा॒धे॑ ऽकुर्वत । \newline
17. नि॒र्बा॒धे॑ ऽकुर्वता कुर्वत निर्बा॒धे नि॑र्बा॒धे॑ ऽकुर्वत॒ तत् तद॑कुर्वत निर्बा॒धे नि॑र्बा॒धे॑ ऽकुर्वत॒ तत् । \newline
18. नि॒र्बा॒ध इति॑ निः - बा॒धे । \newline
19. अ॒कु॒र्व॒त॒ तत् तद॑कुर्वता कुर्वत॒ तन् नि॑र्बा॒धाना᳚म् निर्बा॒धाना॒म् तद॑कुर्वता कुर्वत॒ तन् नि॑र्बा॒धाना᳚म् । \newline
20. तन् नि॑र्बा॒धाना᳚म् निर्बा॒धाना॒म् तत् तन् नि॑र्बा॒धाना᳚म् निर्बाध॒त्वम् नि॑र्बाध॒त्वम् नि॑र्बा॒धाना॒म् तत् तन् नि॑र्बा॒धाना᳚म् निर्बाध॒त्वम् । \newline
21. नि॒र्बा॒धाना᳚म् निर्बाध॒त्वम् नि॑र्बाध॒त्वम् नि॑र्बा॒धाना᳚म् निर्बा॒धाना᳚म् निर्बाध॒त्वम् नि॑र्बा॒धी नि॑र्बा॒धी नि॑र्बाध॒त्वम् नि॑र्बा॒धाना᳚म् निर्बा॒धाना᳚म् निर्बाध॒त्वम् नि॑र्बा॒धी । \newline
22. नि॒र्बा॒धाना॒मिति॑ निः - बा॒धाना᳚म् । \newline
23. नि॒र्बा॒ध॒त्वम् नि॑र्बा॒धी नि॑र्बा॒धी नि॑र्बाध॒त्वम् नि॑र्बाध॒त्वम् नि॑र्बा॒धी भ॑वति भवति निर्बा॒धी नि॑र्बाध॒त्वम् नि॑र्बाध॒त्वम् नि॑र्बा॒धी भ॑वति । \newline
24. नि॒र्बा॒ध॒त्वमिति॑ निर्बाध - त्वम् । \newline
25. नि॒र्बा॒धी भ॑वति भवति निर्बा॒धी नि॑र्बा॒धी भ॑वति॒ भ्रातृ॑व्या॒न् भ्रातृ॑व्यान् भवति निर्बा॒धी नि॑र्बा॒धी भ॑वति॒ भ्रातृ॑व्यान् । \newline
26. नि॒र्बा॒धीति॑ निः - बा॒धी । \newline
27. भ॒व॒ति॒ भ्रातृ॑व्या॒न् भ्रातृ॑व्यान् भवति भवति॒ भ्रातृ॑व्या ने॒वैव भ्रातृ॑व्यान् भवति भवति॒ भ्रातृ॑व्या ने॒व । \newline
28. भ्रातृ॑व्या ने॒वैव भ्रातृ॑व्या॒न् भ्रातृ॑व्या ने॒व नि॑र्बा॒धे नि॑र्बा॒ध ए॒व भ्रातृ॑व्या॒न् भ्रातृ॑व्या ने॒व नि॑र्बा॒धे । \newline
29. ए॒व नि॑र्बा॒धे नि॑र्बा॒ध ए॒वैव नि॑र्बा॒धे कु॑रुते कुरुते निर्बा॒ध ए॒वैव नि॑र्बा॒धे कु॑रुते । \newline
30. नि॒र्बा॒धे कु॑रुते कुरुते निर्बा॒धे नि॑र्बा॒धे कु॑रुते सावित्रि॒या सा॑वित्रि॒या कु॑रुते निर्बा॒धे नि॑र्बा॒धे कु॑रुते सावित्रि॒या । \newline
31. नि॒र्बा॒ध इति॑ निः - बा॒धे । \newline
32. कु॒रु॒ते॒ सा॒वि॒त्रि॒या सा॑वित्रि॒या कु॑रुते कुरुते सावित्रि॒या प्रति॒ प्रति॑ सावित्रि॒या कु॑रुते कुरुते सावित्रि॒या प्रति॑ । \newline
33. सा॒वि॒त्रि॒या प्रति॒ प्रति॑ सावित्रि॒या सा॑वित्रि॒या प्रति॑ मुञ्चते मुञ्चते॒ प्रति॑ सावित्रि॒या सा॑वित्रि॒या प्रति॑ मुञ्चते । \newline
34. प्रति॑ मुञ्चते मुञ्चते॒ प्रति॒ प्रति॑ मुञ्चते॒ प्रसू᳚त्यै॒ प्रसू᳚त्यै मुञ्चते॒ प्रति॒ प्रति॑ मुञ्चते॒ प्रसू᳚त्यै । \newline
35. मु॒ञ्च॒ते॒ प्रसू᳚त्यै॒ प्रसू᳚त्यै मुञ्चते मुञ्चते॒ प्रसू᳚त्यै॒ नक्तो॒षासा॒ नक्तो॒षासा॒ प्रसू᳚त्यै मुञ्चते मुञ्चते॒ प्रसू᳚त्यै॒ नक्तो॒षासा᳚ । \newline
36. प्रसू᳚त्यै॒ नक्तो॒षासा॒ नक्तो॒षासा॒ प्रसू᳚त्यै॒ प्रसू᳚त्यै॒ नक्तो॒षासेतीति॒ नक्तो॒षासा॒ प्रसू᳚त्यै॒ प्रसू᳚त्यै॒ नक्तो॒षासेति॑ । \newline
37. प्रसू᳚त्या॒ इति॒ प्र - सू॒त्यै॒ । \newline
38. नक्तो॒षासेतीति॒ नक्तो॒षासा॒ नक्तो॒षासे त्युत्त॑र॒यो त्त॑र॒येति॒ नक्तो॒षासा॒ नक्तो॒षासे त्युत्त॑रया । \newline
39. इत्युत्त॑र॒यो त्त॑र॒येती त्युत्त॑रया ऽहोरा॒त्राभ्या॑ महोरा॒त्राभ्या॒ मुत्त॑र॒येती त्युत्त॑रया ऽहोरा॒त्राभ्या᳚म् । \newline
40. उत्त॑रया ऽहोरा॒त्राभ्या॑ महोरा॒त्राभ्या॒ मुत्त॑र॒ योत्त॑रया ऽहोरा॒त्राभ्या॑ मे॒वैवा हो॑रा॒त्राभ्या॒ मुत्त॑र॒ योत्त॑रया ऽहोरा॒त्राभ्या॑ मे॒व । \newline
41. उत्त॑र॒येत्युत् - त॒र॒या॒ । \newline
42. अ॒हो॒रा॒त्राभ्या॑ मे॒वैवाहो॑रा॒त्राभ्या॑ महोरा॒त्राभ्या॑ मे॒वैन॑ मेन मे॒वाहो॑रा॒त्राभ्या॑ महोरा॒त्राभ्या॑ मे॒वैन᳚म् । \newline
43. अ॒हो॒रा॒त्राभ्या॒मित्य॑हः - रा॒त्राभ्या᳚म् । \newline
44. ए॒वैन॑ मेन मे॒वैवैन॒ मुदु दे॑न मे॒वैवैन॒ मुत् । \newline
45. ए॒न॒ मुदुदे॑न मेन॒ मुद् य॑च्छते यच्छत॒ उदे॑न मेन॒ मुद् य॑च्छते । \newline
46. उद् य॑च्छते यच्छत॒ उदुद् य॑च्छते दे॒वा दे॒वा य॑च्छत॒ उदुद् य॑च्छते दे॒वाः । \newline
47. य॒च्छ॒ते॒ दे॒वा दे॒वा य॑च्छते यच्छते दे॒वा अ॒ग्नि म॒ग्निम् दे॒वा य॑च्छते यच्छते दे॒वा अ॒ग्निम् । \newline
48. दे॒वा अ॒ग्नि म॒ग्निम् दे॒वा दे॒वा अ॒ग्निम् धा॑रयन् धारयन् न॒ग्निम् दे॒वा दे॒वा अ॒ग्निम् धा॑रयन्न् । \newline
49. अ॒ग्निम् धा॑रयन् धारयन् न॒ग्नि म॒ग्निम् धा॑रयन् द्रविणो॒दा द्र॑विणो॒दा धा॑रयन् न॒ग्नि म॒ग्निम् धा॑रयन् द्रविणो॒दाः । \newline
50. धा॒र॒य॒न् द्र॒वि॒णो॒दा द्र॑विणो॒दा धा॑रयन् धारयन् द्रविणो॒दा इतीति॑ द्रविणो॒दा धा॑रयन् धारयन् द्रविणो॒दा इति॑ । \newline
51. द्र॒वि॒णो॒दा इतीति॑ द्रविणो॒दा द्र॑विणो॒दा इत्या॑हा॒हेति॑ द्रविणो॒दा द्र॑विणो॒दा इत्या॑ह । \newline
52. द्र॒वि॒णो॒दा इति॑ द्रविणः - दाः । \newline
53. इत्या॑हा॒हे तीत्या॑ह प्रा॒णाः प्रा॒णा आ॒हे तीत्या॑ह प्रा॒णाः । \newline
54. आ॒ह॒ प्रा॒णाः प्रा॒णा आ॑हाह प्रा॒णा वै वै प्रा॒णा आ॑हाह प्राणा वै । \newline
55. प्रा॒णा वै वै प्रा॒णाः प्रा॒णा वै दे॒वा दे॒वा वै प्रा॒णाः प्रा॒णा वै दे॒वाः । \newline
56. प्रा॒णा इति॑ प्र - अ॒नाः । \newline
57. वै दे॒वा दे॒वा वै वै दे॒वा द्र॑विणो॒दा द्र॑विणो॒दा दे॒वा वै वै दे॒वा द्र॑विणो॒दाः । \newline
58. दे॒वा द्र॑विणो॒दा द्र॑विणो॒दा दे॒वा दे॒वा द्र॑विणो॒दा अ॑होरा॒त्राभ्या॑ महोरा॒त्राभ्या᳚म् द्रविणो॒दा दे॒वा दे॒वा द्र॑विणो॒दा अ॑होरा॒त्राभ्या᳚म् । \newline
59. द्र॒वि॒णो॒दा अ॑होरा॒त्राभ्या॑ महोरा॒त्राभ्या᳚म् द्रविणो॒दा द्र॑विणो॒दा अ॑होरा॒त्राभ्या॑ मे॒वैवा हो॑रा॒त्राभ्या᳚म् द्रविणो॒दा द्र॑विणो॒दा अ॑होरा॒त्राभ्या॑ मे॒व । \newline
60. द्र॒वि॒णो॒दा इति॑ द्रविणः - दाः । \newline
61. अ॒हो॒रा॒त्राभ्या॑ मे॒वैवाहो॑रा॒त्राभ्या॑ महोरा॒त्राभ्या॑ मे॒वैन॑ मेन मे॒वाहो॑रा॒त्राभ्या॑ महोरा॒त्राभ्या॑ मे॒वैन᳚म् । \newline
62. अ॒हो॒रा॒त्राभ्या॒मित्य॑हः - रा॒त्राभ्या᳚म् । \newline
63. ए॒वैन॑ मेन मे॒वैवैन॑ मु॒द्यत्यो॒ द्यत्यै॑न मे॒वैवैन॑ मु॒द्यत्य॑ । \newline
64. ए॒न॒ मु॒द्य त्यो॒द्यत्यै॑न मेन मु॒द्यत्य॑ प्रा॒णैः प्रा॒णै रु॒द्यत्यै॑न मेन मु॒द्यत्य॑ प्रा॒णैः । \newline
65. उ॒द्यत्य॑ प्रा॒णैः प्रा॒णै रु॒द्य त्यो॒द्यत्य॑ प्रा॒णैर् दा॑धार दाधार प्रा॒णै रु॒द्य त्यो॒द्यत्य॑ प्रा॒णैर् दा॑धार । \newline
66. उ॒द्यत्येतु॑त् - यत्य॑ । \newline
\pagebreak
\markright{ TS 5.1.10.5  \hfill https://www.vedavms.in \hfill}

\section{ TS 5.1.10.5 }

\textbf{TS 5.1.10.5 } \newline
\textbf{Samhita Paata} \newline

प्रा॒णैर्दा॑धा॒रा ऽऽसी॑नः॒ प्रति॑मुञ्चते॒ तस्मा॒दासी॑नाः प्र॒जाः प्रजा॑यन्ते कृष्णाजि॒नमुत्त॑रं॒ तेजो॒ वै हिर॑ण्यं॒ ब्रह्म॑ कृष्णाजि॒नं तेज॑सा चै॒वैनं॒ ब्रह्म॑णा चोभ॒यतः॒ परि॑गृह्णाति॒ षडु॑द्यामꣳ शि॒क्यं॑ भवति॒ षड्वा ऋ॒तव॑ ऋ॒तुभि॑रे॒वैन॒-मुद्य॑च्छते॒ यद् द्वाद॑शोद्यामꣳ संॅवथ्स॒रेणै॒व मौ॒ञ्जं भ॑व॒त्यूर्ग्वै मुञ्जा॑ ऊ॒र्जैवैनꣳ॒॒ स ( ) म॑र्द्धयति सुप॒र्णो॑ऽसि ग॒रुत्मा॒नित्यवे᳚क्षते रू॒पमे॒वास्यै॒तन्म॑हि॒मानं॒ ॅव्याच॑ष्टे॒ दिवं॑ गच्छ॒ सुवः॑ प॒तेत्या॑ह सुव॒र्गमे॒वैनं॑ ॅलो॒कं ग॑मयति ॥ \newline

\textbf{Pada Paata} \newline

प्रा॒णैरिति॑ प्र - अ॒नैः । दा॒धा॒र॒ । आसी॑नः । प्रतीति॑ । मु॒ञ्च॒ते॒ । तस्मा᳚त् । आसी॑नाः । प्र॒जा इति॑ प्र - जाः । प्रेति॑ । जा॒य॒न्ते॒ । कृ॒ष्णा॒जि॒नमिति॑ कृष्ण - अ॒जि॒नम् । उत्त॑र॒मित्युत् - त॒र॒म् । तेजः॑ । वै । हिर॑ण्यम् । ब्रह्म॑ । कृ॒ष्णा॒जि॒नमिति॑ कृष्ण - अ॒जि॒नम् । तेज॑सा । च॒ । ए॒व । ए॒न॒म् । ब्रह्म॑णा । च॒ । उ॒भ॒यतः॑ । परीति॑ । गृ॒ह्णा॒ति॒ । षडु॑द्याम॒मिति॒ षट् - उ॒द्या॒म॒म् । शि॒क्य᳚म् । भ॒व॒ति॒ । षट् । वै । ऋ॒तवः॑ । ऋ॒तुभि॒रित्यृ॒तु - भिः॒ । ए॒व । ए॒न॒म् । उदिति॑ । य॒च्छ॒ते॒ । यत् । द्वाद॑शोद्याम॒मिति॒ द्वाद॑श - उ॒द्या॒म॒म् । सं॒ॅव॒थ्स॒रेणेति॑ सं - व॒थ्स॒रेण॑ । ए॒व । मौ॒ञ्जम् । भ॒व॒ति॒ । ऊर्क् । वै । मुञ्जाः᳚ । ऊ॒र्जा । ए॒व । ए॒न॒म् । समिति॑ ( ) । अ॒द्‌र्ध॒य॒ति॒ । सु॒प॒र्ण इति॑ सु - प॒र्णः । अ॒सि॒ । ग॒रुत्मान्॑ । इति॑ । अवेति॑ । ई॒क्ष॒ते॒ । रू॒पम् । ए॒व । अ॒स्य॒ । ए॒तत् । म॒हि॒मान᳚म् । व्याच॑ष्ट॒ इति॑ वि - आच॑ष्टे । दिव᳚म् । ग॒च्छ॒ । सुवः॑ । प॒त॒ । इति॑ । आ॒ह॒ । सु॒व॒र्गमिति॑ सुवः - गम् । ए॒व । ए॒न॒म् । लो॒कम् । ग॒म॒य॒ति॒ ॥  \newline


\textbf{Krama Paata} \newline

प्रा॒णैर् दा॑धार । प्रा॒णैरिति॑ प्र - अ॒नैः । दा॒धा॒रासी॑नः । आसी॑नः॒ प्रति॑ । प्रति॑ मुञ्चते । मु॒ञ्च॒ते॒ तस्मा᳚त् । तस्मा॒दासी॑नाः । आसी॑नाः प्र॒जाः । प्र॒जाः प्र । प्र॒जा इति॑ प्र - जाः । प्र जा॑यन्ते । जा॒य॒न्ते॒ कृ॒ष्णा॒जि॒नम् । कृ॒ष्णा॒जि॒नमुत्त॑रम् । कृ॒ष्णा॒जि॒नमिति॑ कृष्ण - अ॒जि॒नम् । उत्त॑र॒म् तेजः॑ । उत्त॑र॒मित्युत् - त॒र॒म् । तेजो॒ वै । वै हिर॑ण्यम् । हिर॑ण्य॒म् ब्रह्म॑ । ब्रह्म॑ कृष्णाजि॒नम् । कृ॒ष्णा॒जि॒नम् तेज॑सा । कृ॒ष्णा॒जि॒नमिति॑ कृष्ण - अ॒जि॒नम् । तेज॑सा च । चै॒व । ए॒वैन᳚म् । ए॒न॒म् ब्रह्म॑णा । ब्रह्म॑णा च । चो॒भ॒यतः॑ । उ॒भ॒यतः॒ परि॑ । परि॑ गृह्णाति । गृ॒ह्णा॒ति॒ षडु॑द्यामम् । षडु॑द्यामꣳ शि॒क्य᳚म् । षडु॑द्याम॒मिति॒ षट् - उ॒द्या॒म॒म् । शि॒क्य॑म् भवति । भ॒व॒ति॒ षट् । षड् वै । वा ऋ॒तवः॑ । ऋ॒तव॑ ऋ॒तुभिः॑ । ऋ॒तुभि॑रे॒व । ऋ॒तुभि॒रित्यृ॒तु - भिः॒ । ए॒वैन᳚म् । ए॒न॒मुत् । उद् य॑च्छते । य॒च्छ॒ते॒ यत् । यद् द्वाद॑शोद्यामम् । द्वाद॑शोद्यामꣳ सम्ॅवथ्स॒रेण॑ । द्वाद॑शोद्याम॒मिति॒ द्वाद॑श - उ॒द्या॒म॒म् । स॒म्ॅव॒थ्स॒रेणै॒व । स॒म्ॅव॒थ्स॒रेणेति॑ सम् - व॒थ्स॒रेण॑ । ए॒व मौ॒ञ्जम् । मौ॒ञ्जम् भ॑वति । भ॒व॒त्यूर्क् । ऊर्ग् वै । वै मुञ्जाः᳚ । मुञ्जा॑ ऊ॒र्जा । ऊ॒र्जैव । ए॒वैन᳚म् । ए॒नꣳ॒॒ सम् ( ) । सम॑र्द्धयति । अ॒र्द्ध॒य॒ति॒ सु॒प॒र्णः । सु॒प॒र्णो॑ऽसि । सु॒प॒र्ण इति॑ सु - प॒र्णः । अ॒सि॒ ग॒रुत्मान्॑ । ग॒रुत्मा॒निति॑ । इत्यव॑ । अवे᳚क्षते । ई॒क्ष॒ते॒ रू॒पम् । रू॒पमे॒व । ए॒वास्य॑ । अ॒स्यै॒तत् । ए॒तन् म॑हि॒मान᳚म् । म॒हि॒मान॒म् ॅव्याच॑ष्टे । व्याच॑ष्टे॒ दिव᳚म् । व्याच॑ष्ट॒ इति॑ वि - आच॑ष्टे । दिव॑म् गच्छ । ग॒च्छ॒ सुवः॑ । सुवः॑ पत । प॒तेति॑ । इत्या॑ह । आ॒ह॒ सु॒व॒र्गम् । सु॒व॒र्गमे॒व । सु॒व॒र्गमिति॑ सुवः - गम् । ए॒वैन᳚म् । ए॒न॒म् ॅलो॒कम् । लो॒कम् ग॑मयति । ग॒म॒य॒तीति॑ गमयति । \newline

\textbf{Jatai Paata} \newline

1. प्रा॒णैर् दा॑धार दाधार प्रा॒णैः प्रा॒णैर् दा॑धार । \newline
2. प्रा॒णैरिति॑ प्र - अ॒नैः । \newline
3. दा॒धा॒ रासी॑न॒ आसी॑नो दाधार दाधा॒ रासी॑नः । \newline
4. आसी॑नः॒ प्रति॒ प्रत्यासी॑न॒ आसी॑नः॒ प्रति॑ । \newline
5. प्रति॑ मुञ्चते मुञ्चते॒ प्रति॒ प्रति॑ मुञ्चते । \newline
6. मु॒ञ्च॒ते॒ तस्मा॒त् तस्मा᳚न् मुञ्चते मुञ्चते॒ तस्मा᳚त् । \newline
7. तस्मा॒ दासी॑ना॒ आसी॑ना॒ स्तस्मा॒त् तस्मा॒ दासी॑नाः । \newline
8. आसी॑नाः प्र॒जाः प्र॒जा आसी॑ना॒ आसी॑नाः प्र॒जाः । \newline
9. प्र॒जाः प्र प्र प्र॒जाः प्र॒जाः प्र । \newline
10. प्र॒जा इति॑ प्र - जाः । \newline
11. प्र जा॑यन्ते जायन्ते॒ प्र प्र जा॑यन्ते । \newline
12. जा॒य॒न्ते॒ कृ॒ष्णा॒जि॒नम् कृ॑ष्णाजि॒नम् जा॑यन्ते जायन्ते कृष्णाजि॒नम् । \newline
13. कृ॒ष्णा॒जि॒न मुत्त॑र॒ मुत्त॑रम् कृष्णाजि॒नम् कृ॑ष्णाजि॒न मुत्त॑रम् । \newline
14. कृ॒ष्णा॒जि॒नमिति॑ कृष्ण - अ॒जि॒नम् । \newline
15. उत्त॑र॒म् तेज॒ स्तेज॒ उत्त॑र॒ मुत्त॑र॒म् तेजः॑ । \newline
16. उत्त॑र॒मित्युत् - त॒र॒म् । \newline
17. तेजो॒ वै वै तेज॒ स्तेजो॒ वै । \newline
18. वै हिर॑ण्यꣳ॒॒ हिर॑ण्यं॒ ॅवै वै हिर॑ण्यम् । \newline
19. हिर॑ण्य॒म् ब्रह्म॒ ब्रह्म॒ हिर॑ण्यꣳ॒॒ हिर॑ण्य॒म् ब्रह्म॑ । \newline
20. ब्रह्म॑ कृष्णाजि॒नम् कृ॑ष्णाजि॒नम् ब्रह्म॒ ब्रह्म॑ कृष्णाजि॒नम् । \newline
21. कृ॒ष्णा॒जि॒नम् तेज॑सा॒ तेज॑सा कृष्णाजि॒नम् कृ॑ष्णाजि॒नम् तेज॑सा । \newline
22. कृ॒ष्णा॒जि॒नमिति॑ कृष्ण - अ॒जि॒नम् । \newline
23. तेज॑सा च च॒ तेज॑सा॒ तेज॑सा च । \newline
24. चै॒वैव च॑ चै॒व । \newline
25. ए॒वैन॑ मेन मे॒वैवैन᳚म् । \newline
26. ए॒न॒म् ब्रह्म॑णा॒ ब्रह्म॑णैन मेन॒म् ब्रह्म॑णा । \newline
27. ब्रह्म॑णा च च॒ ब्रह्म॑णा॒ ब्रह्म॑णा च । \newline
28. चो॒भ॒यत॑ उभ॒यत॑श्च चोभ॒यतः॑ । \newline
29. उ॒भ॒यतः॒ परि॒ पर्यु॑भ॒यत॑ उभ॒यतः॒ परि॑ । \newline
30. परि॑ गृह्णाति गृह्णाति॒ परि॒ परि॑ गृह्णाति । \newline
31. गृ॒ह्णा॒ति॒ षडु॑द्यामꣳ॒॒ षडु॑द्यामम् गृह्णाति गृह्णाति॒ षडु॑द्यामम् । \newline
32. षडु॑द्यामꣳ शि॒क्यꣳ॑ शि॒क्यꣳ॑ षडु॑द्यामꣳ॒॒ षडु॑द्यामꣳ शि॒क्य᳚म् । \newline
33. षडु॑द्याम॒मिति॒ षट् - उ॒द्या॒म॒म् । \newline
34. शि॒क्य॑म् भवति भवति शि॒क्यꣳ॑ शि॒क्य॑म् भवति । \newline
35. भ॒व॒ति॒ षट् थ्षड् भ॑वति भवति॒ षट् । \newline
36. षड् वै वै षट् थ्षड् वै । \newline
37. वा ऋ॒तव॑ ऋ॒तवो॒ वै वा ऋ॒तवः॑ । \newline
38. ऋ॒तव॑ ऋ॒तुभिर्॑. ऋ॒तुभिर्॑. ऋ॒तव॑ ऋ॒तव॑ ऋ॒तुभिः॑ । \newline
39. ऋ॒तुभि॑ रे॒वैव र्‌तुभिर्॑. ऋ॒तुभि॑ रे॒व । \newline
40. ऋ॒तुभि॒रित्यृ॒तु - भिः॒ । \newline
41. ए॒वैन॑ मेन मे॒वैवैन᳚म् । \newline
42. ए॒न॒ मुदुदे॑न मेन॒ मुत् । \newline
43. उद् य॑च्छते यच्छत॒ उदुद् य॑च्छते । \newline
44. य॒च्छ॒ते॒ यद् यद् य॑च्छते यच्छते॒ यत् । \newline
45. यद् द्वाद॑शोद्याम॒म् द्वाद॑शोद्यामं॒ ॅयद् यद् द्वाद॑शोद्यामम् । \newline
46. द्वाद॑शोद्यामꣳ संॅवथ्स॒रेण॑ संॅवथ्स॒रेण॒ द्वाद॑शोद्याम॒म् द्वाद॑शोद्यामꣳ संॅवथ्स॒रेण॑ । \newline
47. द्वाद॑शोद्याम॒मिति॒ द्वाद॑श - उ॒द्या॒म॒म् । \newline
48. सं॒ॅव॒थ्स॒रे णै॒वैव सं॑ॅवथ्स॒रेण॑ संॅवथ्स॒रे णै॒व । \newline
49. सं॒ॅव॒थ्स॒रेणेति॑ सं - व॒थ्स॒रेण॑ । \newline
50. ए॒व मौ॒ञ्जम् मौ॒ञ्ज मे॒वैव मौ॒ञ्जम् । \newline
51. मौ॒ञ्जम् भ॑वति भवति मौ॒ञ्जम् मौ॒ञ्जम् भ॑वति । \newline
52. भ॒व॒ त्यूर्गूर्ग् भ॑वति भव॒ त्यूर्क् । \newline
53. ऊर्ग् वै वा ऊर्गूर्ग् वै । \newline
54. वै मुञ्जा॒ मुञ्जा॒ वै वै मुञ्जाः᳚ । \newline
55. मुञ्जा॑ ऊ॒र्जोर्जा मुञ्जा॒ मुञ्जा॑ ऊ॒र्जा । \newline
56. ऊ॒र्जै वैवोर्जोर् जैव । \newline
57. ए॒वैन॑ मेन मे॒वैवैन᳚म् । \newline
58. ए॒नꣳ॒॒ सꣳ स मे॑न मेनꣳ॒॒ सम् । \newline
59. स म॑र्द्धय त्यर्द्धयति॒ सꣳ स म॑र्द्धयति । \newline
60. अ॒र्द्ध॒य॒ति॒ सु॒प॒र्णः सु॑प॒र्णो᳚ ऽर्द्धय त्यर्द्धयति सुप॒र्णः । \newline
61. सु॒प॒र्णो᳚ ऽस्यसि सुप॒र्णः सु॑प॒र्णो॑ ऽसि । \newline
62. सु॒प॒र्ण इति॑ सु - प॒र्णः । \newline
63. अ॒सि॒ ग॒रुत्मा᳚न् ग॒रुत्मा॑ नस्यसि ग॒रुत्मान्॑ । \newline
64. ग॒रुत्मा॒ नितीति॑ ग॒रुत्मा᳚न् ग॒रुत्मा॒ निति॑ । \newline
65. इत्यवावे तीत्यव॑ । \newline
66. अवे᳚क्षत ईक्ष॒ते ऽवावे᳚क्षते । \newline
67. ई॒क्ष॒ते॒ रू॒पꣳ रू॒प मी᳚क्षत ईक्षते रू॒पम् । \newline
68. रू॒प मे॒वैव रू॒पꣳ रू॒प मे॒व । \newline
69. ए॒वास्या᳚ स्यै॒वैवास्य॑ । \newline
70. अ॒स्यै॒त दे॒त द॑स्या स्यै॒तत् । \newline
71. ए॒तन् म॑हि॒मान॑म् महि॒मान॑ मे॒त दे॒तन् म॑हि॒मान᳚म् । \newline
72. म॒हि॒मानं॒ ॅव्याच॑ष्टे॒ व्याच॑ष्टे महि॒मान॑म् महि॒मानं॒ ॅव्याच॑ष्टे । \newline
73. व्याच॑ष्टे॒ दिव॒म् दिवं॒ ॅव्याच॑ष्टे॒ व्याच॑ष्टे॒ दिव᳚म् । \newline
74. व्याच॑ष्ट॒ इति॑ वि - आच॑ष्टे । \newline
75. दिव॑म् गच्छ गच्छ॒ दिव॒म् दिव॑म् गच्छ । \newline
76. ग॒च्छ॒ सुवः॒ सुव॑र् गच्छ गच्छ॒ सुवः॑ । \newline
77. सुवः॑ पत पत॒ सुवः॒ सुवः॑ पत । \newline
78. प॒ते तीति॑ पत प॒तेति॑ । \newline
79. इत्या॑हा॒हे तीत्या॑ह । \newline
80. आ॒ह॒ सु॒व॒र्गꣳ सु॑व॒र्ग मा॑हाह सुव॒र्गम् । \newline
81. सु॒व॒र्ग मे॒वैव सु॑व॒र्गꣳ सु॑व॒र्ग मे॒व । \newline
82. सु॒व॒र्गमिति॑ सुवः - गम् । \newline
83. ए॒वैन॑ मेन मे॒वैवैन᳚म् । \newline
84. ए॒न॒म् ॅलो॒कम् ॅलो॒क मे॑न मेनम् ॅलो॒कम् । \newline
85. लो॒कम् ग॑मयति गमयति लो॒कम् ॅलो॒कम् ग॑मयति । \newline
86. ग॒म॒य॒तीति॑ गमयति । \newline

\textbf{Ghana Paata } \newline

1. प्रा॒णैर् दा॑धार दाधार प्रा॒णैः प्रा॒णैर् दा॑धा॒रा सी॑न॒ आसी॑नो दाधार प्रा॒णैः प्रा॒णैर् दा॑धा॒रा सी॑नः । \newline
2. प्रा॒णैरिति॑ प्र - अ॒नैः । \newline
3. दा॒धा॒रा सी॑न॒ आसी॑नो दाधार दाधा॒रा सी॑नः॒ प्रति॒ प्रत्यासी॑नो दाधार दाधा॒रा सी॑नः॒ प्रति॑ । \newline
4. आसी॑नः॒ प्रति॒ प्रत्यासी॑न॒ आसी॑नः॒ प्रति॑ मुञ्चते मुञ्चते॒ प्रत्यासी॑न॒ आसी॑नः॒ प्रति॑ मुञ्चते । \newline
5. प्रति॑ मुञ्चते मुञ्चते॒ प्रति॒ प्रति॑ मुञ्चते॒ तस्मा॒त् तस्मा᳚न् मुञ्चते॒ प्रति॒ प्रति॑ मुञ्चते॒ तस्मा᳚त् । \newline
6. मु॒ञ्च॒ते॒ तस्मा॒त् तस्मा᳚न् मुञ्चते मुञ्चते॒ तस्मा॒ दासी॑ना॒ आसी॑ना॒ स्तस्मा᳚न् मुञ्चते मुञ्चते॒ तस्मा॒ दासी॑नाः । \newline
7. तस्मा॒ दासी॑ना॒ आसी॑ना॒ स्तस्मा॒त् तस्मा॒ दासी॑नाः प्र॒जाः प्र॒जा आसी॑ना॒ स्तस्मा॒त् तस्मा॒ दासी॑नाः प्र॒जाः । \newline
8. आसी॑नाः प्र॒जाः प्र॒जा आसी॑ना॒ आसी॑नाः प्र॒जाः प्र प्र प्र॒जा आसी॑ना॒ आसी॑नाः प्र॒जाः प्र । \newline
9. प्र॒जाः प्र प्र प्र॒जाः प्र॒जाः प्र जा॑यन्ते जायन्ते॒ प्र प्र॒जाः प्र॒जाः प्र जा॑यन्ते । \newline
10. प्र॒जा इति॑ प्र - जाः । \newline
11. प्र जा॑यन्ते जायन्ते॒ प्र प्र जा॑यन्ते कृष्णाजि॒नम् कृ॑ष्णाजि॒नम् जा॑यन्ते॒ प्र प्र जा॑यन्ते कृष्णाजि॒नम् । \newline
12. जा॒य॒न्ते॒ कृ॒ष्णा॒जि॒नम् कृ॑ष्णाजि॒नम् जा॑यन्ते जायन्ते कृष्णाजि॒न मुत्त॑र॒ मुत्त॑रम् कृष्णाजि॒नम् जा॑यन्ते जायन्ते कृष्णाजि॒न मुत्त॑रम् । \newline
13. कृ॒ष्णा॒जि॒न मुत्त॑र॒ मुत्त॑रम् कृष्णाजि॒नम् कृ॑ष्णाजि॒न मुत्त॑र॒म् तेज॒ स्तेज॒ उत्त॑रम् कृष्णाजि॒नम् कृ॑ष्णाजि॒न मुत्त॑र॒म् तेजः॑ । \newline
14. कृ॒ष्णा॒जि॒नमिति॑ कृष्ण - अ॒जि॒नम् । \newline
15. उत्त॑र॒म् तेज॒ स्तेज॒ उत्त॑र॒ मुत्त॑र॒म् तेजो॒ वै वै तेज॒ उत्त॑र॒ मुत्त॑र॒म् तेजो॒ वै । \newline
16. उत्त॑र॒मित्युत् - त॒र॒म् । \newline
17. तेजो॒ वै वै तेज॒ स्तेजो॒ वै हिर॑ण्यꣳ॒॒ हिर॑ण्यं॒ ॅवै तेज॒ स्तेजो॒ वै हिर॑ण्यम् । \newline
18. वै हिर॑ण्यꣳ॒॒ हिर॑ण्यं॒ ॅवै वै हिर॑ण्य॒म् ब्रह्म॒ ब्रह्म॒ हिर॑ण्यं॒ ॅवै वै हिर॑ण्य॒म् ब्रह्म॑ । \newline
19. हिर॑ण्य॒म् ब्रह्म॒ ब्रह्म॒ हिर॑ण्यꣳ॒॒ हिर॑ण्य॒म् ब्रह्म॑ कृष्णाजि॒नम् कृ॑ष्णाजि॒नम् ब्रह्म॒ हिर॑ण्यꣳ॒॒ हिर॑ण्य॒म् ब्रह्म॑ कृष्णाजि॒नम् । \newline
20. ब्रह्म॑ कृष्णाजि॒नम् कृ॑ष्णाजि॒नम् ब्रह्म॒ ब्रह्म॑ कृष्णाजि॒नम् तेज॑सा॒ तेज॑सा कृष्णाजि॒नम् ब्रह्म॒ ब्रह्म॑ कृष्णाजि॒नम् तेज॑सा । \newline
21. कृ॒ष्णा॒जि॒नम् तेज॑सा॒ तेज॑सा कृष्णाजि॒नम् कृ॑ष्णाजि॒नम् तेज॑सा च च॒ तेज॑सा कृष्णाजि॒नम् कृ॑ष्णाजि॒नम् तेज॑सा च । \newline
22. कृ॒ष्णा॒जि॒नमिति॑ कृष्ण - अ॒जि॒नम् । \newline
23. तेज॑सा च च॒ तेज॑सा॒ तेज॑सा चै॒वैव च॒ तेज॑सा॒ तेज॑सा चै॒व । \newline
24. चै॒वैव च॑ चै॒वैन॑ मेन मे॒व च॑ चै॒वैन᳚म् । \newline
25. ए॒वैन॑ मेन मे॒वैवैन॒म् ब्रह्म॑णा॒ ब्रह्म॑णैन मे॒वैवैन॒म् ब्रह्म॑णा । \newline
26. ए॒न॒म् ब्रह्म॑णा॒ ब्रह्म॑णैन मेन॒म् ब्रह्म॑णा च च॒ ब्रह्म॑णैन मेन॒म् ब्रह्म॑णा च । \newline
27. ब्रह्म॑णा च च॒ ब्रह्म॑णा॒ ब्रह्म॑णा चोभ॒यत॑ उभ॒यत॑श्च॒ ब्रह्म॑णा॒ ब्रह्म॑णा चोभ॒यतः॑ । \newline
28. चो॒भ॒यत॑ उभ॒यत॑श्च चोभ॒यतः॒ परि॒ पर्यु॑भ॒यत॑श्च चोभ॒यतः॒ परि॑ । \newline
29. उ॒भ॒यतः॒ परि॒ पर्यु॑भ॒यत॑ उभ॒यतः॒ परि॑ गृह्णाति गृह्णाति॒ पर्यु॑भ॒यत॑ उभ॒यतः॒ परि॑ गृह्णाति । \newline
30. परि॑ गृह्णाति गृह्णाति॒ परि॒ परि॑ गृह्णाति॒ षडु॑द्यामꣳ॒॒ षडु॑द्यामम् गृह्णाति॒ परि॒ परि॑ गृह्णाति॒ षडु॑द्यामम् । \newline
31. गृ॒ह्णा॒ति॒ षडु॑द्यामꣳ॒॒ षडु॑द्यामम् गृह्णाति गृह्णाति॒ षडु॑द्यामꣳ शि॒क्यꣳ॑ शि॒क्यꣳ॑ षडु॑द्यामम् गृह्णाति गृह्णाति॒ षडु॑द्यामꣳ शि॒क्य᳚म् । \newline
32. षडु॑द्यामꣳ शि॒क्यꣳ॑ शि॒क्यꣳ॑ षडु॑द्यामꣳ॒॒ षडु॑द्यामꣳ शि॒क्य॑म् भवति भवति शि॒क्यꣳ॑ षडु॑द्यामꣳ॒॒ षडु॑द्यामꣳ शि॒क्य॑म् भवति । \newline
33. षडु॑द्याम॒मिति॒ षट् - उ॒द्या॒म॒म् । \newline
34. शि॒क्य॑म् भवति भवति शि॒क्यꣳ॑ शि॒क्य॑म् भवति॒ षट् थ्षड् भ॑वति शि॒क्यꣳ॑ शि॒क्य॑म् भवति॒ षट् । \newline
35. भ॒व॒ति॒ षट् थ्षड् भ॑वति भवति॒ षड् वै वै षड् भ॑वति भवति॒ षड् वै । \newline
36. षड् वै वै षट् थ्षड् वा ऋ॒तव॑ ऋ॒तवो॒ वै षट् थ्षड् वा ऋ॒तवः॑ । \newline
37. वा ऋ॒तव॑ ऋ॒तवो॒ वै वा ऋ॒तव॑ ऋ॒तुभिर्॑. ऋ॒तुभिर्॑. ऋ॒तवो॒ वै वा ऋ॒तव॑ ऋ॒तुभिः॑ । \newline
38. ऋ॒तव॑ ऋ॒तुभिर्॑. ऋ॒तुभिर्॑. ऋ॒तव॑ ऋ॒तव॑ ऋ॒तुभि॑ रे॒वैव र्‌तुभिर्॑. ऋ॒तव॑ ऋ॒तव॑ ऋ॒तुभि॑ रे॒व । \newline
39. ऋ॒तुभि॑ रे॒वैव र्‌तुभिर्॑. ऋ॒तुभि॑ रे॒वैन॑ मेन मे॒व र्‌तुभिर्॑. ऋ॒तुभि॑ रे॒वैन᳚म् । \newline
40. ऋ॒तुभि॒रित्यृ॒तु - भिः॒ । \newline
41. ए॒वैन॑ मेन मे॒वैवैन॒ मुदु दे॑न मे॒वैवैन॒ मुत् । \newline
42. ए॒न॒ मुदुदे॑न मेन॒ मुद् य॑च्छते यच्छत॒ उदे॑न मेन॒ मुद् य॑च्छते । \newline
43. उद् य॑च्छते यच्छत॒ उदुद् य॑च्छते॒ यद् यद् य॑च्छत॒ उदुद् य॑च्छते॒ यत् । \newline
44. य॒च्छ॒ते॒ यद् यद् य॑च्छते यच्छते॒ यद् द्वाद॑शोद्याम॒म् द्वाद॑शोद्यामं॒ ॅयद् य॑च्छते यच्छते॒ यद् द्वाद॑शोद्यामम् । \newline
45. यद् द्वाद॑शोद्याम॒म् द्वाद॑शोद्यामं॒ ॅयद् यद् द्वाद॑शोद्यामꣳ संॅवथ्स॒रेण॑ संॅवथ्स॒रेण॒ द्वाद॑शोद्यामं॒ ॅयद् यद् द्वाद॑शोद्यामꣳ संॅवथ्स॒रेण॑ । \newline
46. द्वाद॑शोद्यामꣳ संॅवथ्स॒रेण॑ संॅवथ्स॒रेण॒ द्वाद॑शोद्याम॒म् द्वाद॑शोद्यामꣳ संॅवथ्स॒रेणै॒वैव सं॑ॅवथ्स॒रेण॒ द्वाद॑शोद्याम॒म् द्वाद॑शोद्यामꣳ संॅवथ्स॒रेणै॒व । \newline
47. द्वाद॑शोद्याम॒मिति॒ द्वाद॑श - उ॒द्या॒म॒म् । \newline
48. सं॒ॅव॒थ्स॒रेणै॒वैव सं॑ॅवथ्स॒रेण॑ संॅवथ्स॒रेणै॒व मौ॒ञ्जम् मौ॒ञ्ज मे॒व सं॑ॅवथ्स॒रेण॑ संॅवथ्स॒रेणै॒व मौ॒ञ्जम् । \newline
49. सं॒ॅव॒थ्स॒रेणेति॑ सं - व॒थ्स॒रेण॑ । \newline
50. ए॒व मौ॒ञ्जम् मौ॒ञ्ज मे॒वैव मौ॒ञ्जम् भ॑वति भवति मौ॒ञ्ज मे॒वैव मौ॒ञ्जम् भ॑वति । \newline
51. मौ॒ञ्जम् भ॑वति भवति मौ॒ञ्जम् मौ॒ञ्जम् भ॑व॒ त्यूर्गूर्ग् भ॑वति मौ॒ञ्जम् मौ॒ञ्जम् भ॑व॒ त्यूर्क् । \newline
52. भ॒व॒ त्यूर्गूर्ग् भ॑वति भव॒ त्यूर्ग् वै वा ऊर्ग् भ॑वति भव॒ त्यूर्ग् वै । \newline
53. ऊर्ग् वै वा ऊर्गूर्ग् वै मुञ्जा॒ मुञ्जा॒ वा ऊर्गूर्ग् वै मुञ्जाः᳚ । \newline
54. वै मुञ्जा॒ मुञ्जा॒ वै वै मुञ्जा॑ ऊ॒र्जोर्जा मुञ्जा॒ वै वै मुञ्जा॑ ऊ॒र्जा । \newline
55. मुञ्जा॑ ऊ॒र्जोर्जा मुञ्जा॒ मुञ्जा॑ ऊ॒र्जै वैवोर्जा मुञ्जा॒ मुञ्जा॑ ऊ॒र्जैव । \newline
56. ऊ॒र्जै वैवोर्जो र्जैवैन॑ मेन मे॒वोर्जो र्जैवैन᳚म् । \newline
57. ए॒वैन॑ मेन मे॒वैवैनꣳ॒॒ सꣳ स मे॑न मे॒वैवैनꣳ॒॒ सम् । \newline
58. ए॒नꣳ॒॒ सꣳ स मे॑न मेनꣳ॒॒ स म॑र्द्धय त्यर्द्धयति॒ स मे॑न मेनꣳ॒॒ स म॑र्द्धयति । \newline
59. स म॑र्द्धय त्यर्द्धयति॒ सꣳ स म॑र्द्धयति सुप॒र्णः सु॑प॒र्णो᳚ ऽर्द्धयति॒ सꣳ स म॑र्द्धयति सुप॒र्णः । \newline
60. अ॒र्द्ध॒य॒ति॒ सु॒प॒र्णः सु॑प॒र्णो᳚ ऽर्द्धय त्यर्द्धयति सुप॒र्णो᳚ ऽस्यसि सुप॒र्णो᳚ ऽर्द्धय त्यर्द्धयति सुप॒र्णो॑ ऽसि । \newline
61. सु॒प॒र्णो᳚ ऽस्यसि सुप॒र्णः सु॑प॒र्णो॑ ऽसि ग॒रुत्मा᳚न् ग॒रुत्मा॑ नसि सुप॒र्णः सु॑प॒र्णो॑ ऽसि ग॒रुत्मान्॑ । \newline
62. सु॒प॒र्ण इति॑ सु - प॒र्णः । \newline
63. अ॒सि॒ ग॒रुत्मा᳚न् ग॒रुत्मा॑ नस्यसि ग॒रुत्मा॒ नितीति॑ ग॒रुत्मा॑ नस्यसि ग॒रुत्मा॒ निति॑ । \newline
64. ग॒रुत्मा॒ नितीति॑ ग॒रुत्मा᳚न् ग॒रुत्मा॒ नित्यवावेति॑ ग॒रुत्मा᳚न् ग॒रुत्मा॒ नित्यव॑ । \newline
65. इत्यवावे तीत्यवे᳚क्षत ईक्ष॒ते ऽवे तीत्यवे᳚क्षते । \newline
66. अवे᳚क्षत ईक्ष॒ते ऽवावे᳚क्षते रू॒पꣳ रू॒प मी᳚क्ष॒ते ऽवावे᳚क्षते रू॒पम् । \newline
67. ई॒क्ष॒ते॒ रू॒पꣳ रू॒प मी᳚क्षत ईक्षते रू॒प मे॒वैव रू॒प मी᳚क्षत ईक्षते रू॒प मे॒व । \newline
68. रू॒प मे॒वैव रू॒पꣳ रू॒प मे॒वास्या᳚ स्यै॒व रू॒पꣳ रू॒प मे॒वास्य॑ । \newline
69. ए॒वास्या᳚ स्यै॒वैवा स्यै॒त दे॒त द॑स्यै॒वैवा स्यै॒तत् । \newline
70. अ॒स्यै॒त दे॒त द॑स्या स्यै॒तन् म॑हि॒मान॑म् महि॒मान॑ मे॒त द॑स्या स्यै॒तन् म॑हि॒मान᳚म् । \newline
71. ए॒तन् म॑हि॒मान॑म् महि॒मान॑ मे॒त दे॒तन् म॑हि॒मानं॒ ॅव्याच॑ष्टे॒ व्याच॑ष्टे महि॒मान॑ मे॒त दे॒तन् म॑हि॒मानं॒ ॅव्याच॑ष्टे । \newline
72. म॒हि॒मानं॒ ॅव्याच॑ष्टे॒ व्याच॑ष्टे महि॒मान॑म् महि॒मानं॒ ॅव्याच॑ष्टे॒ दिव॒म् दिवं॒ ॅव्याच॑ष्टे महि॒मान॑म् महि॒मानं॒ ॅव्याच॑ष्टे॒ दिव᳚म् । \newline
73. व्याच॑ष्टे॒ दिव॒म् दिवं॒ ॅव्याच॑ष्टे॒ व्याच॑ष्टे॒ दिव॑म् गच्छ गच्छ॒ दिवं॒ ॅव्याच॑ष्टे॒ व्याच॑ष्टे॒ दिव॑म् गच्छ । \newline
74. व्याच॑ष्ट॒ इति॑ वि - आच॑ष्टे । \newline
75. दिव॑म् गच्छ गच्छ॒ दिव॒म् दिव॑म् गच्छ॒ सुवः॒ सुव॑र् गच्छ॒ दिव॒म् दिव॑म् गच्छ॒ सुवः॑ । \newline
76. ग॒च्छ॒ सुवः॒ सुव॑र् गच्छ गच्छ॒ सुवः॑ पत पत॒ सुव॑र् गच्छ गच्छ॒ सुवः॑ पत । \newline
77. सुवः॑ पत पत॒ सुवः॒ सुवः॑ प॒तेतीति॑ पत॒ सुवः॒ सुवः॑ प॒तेति॑ । \newline
78. प॒ते तीति॑ पत प॒ते त्या॑हा॒हेति॑ पत प॒ते त्या॑ह । \newline
79. इत्या॑हा॒हे तीत्या॑ह सुव॒र्गꣳ सु॑व॒र्ग मा॒हे तीत्या॑ह सुव॒र्गम् । \newline
80. आ॒ह॒ सु॒व॒र्गꣳ सु॑व॒र्ग मा॑हाह सुव॒र्ग मे॒वैव सु॑व॒र्ग मा॑हाह सुव॒र्ग मे॒व । \newline
81. सु॒व॒र्ग मे॒वैव सु॑व॒र्गꣳ सु॑व॒र्ग मे॒वैन॑ मेन मे॒व सु॑व॒र्गꣳ सु॑व॒र्ग मे॒वैन᳚म् । \newline
82. सु॒व॒र्गमिति॑ सुवः - गम् । \newline
83. ए॒वैन॑ मेन मे॒वैवैन॑म् ॅलो॒कम् ॅलो॒क मे॑न मे॒वैवैन॑म् ॅलो॒कम् । \newline
84. ए॒न॒म् ॅलो॒कम् ॅलो॒क मे॑न मेनम् ॅलो॒कम् ग॑मयति गमयति लो॒क मे॑न मेनम् ॅलो॒कम् ग॑मयति । \newline
85. लो॒कम् ग॑मयति गमयति लो॒कम् ॅलो॒कम् ग॑मयति । \newline
86. ग॒म॒य॒तीति॑ गमयति । \newline
\pagebreak
\markright{ TS 5.1.11.1  \hfill https://www.vedavms.in \hfill}

\section{ TS 5.1.11.1 }

\textbf{TS 5.1.11.1 } \newline
\textbf{Samhita Paata} \newline

समि॑द्धो अ॒ञ्जन् कृद॑रं मती॒नां घृ॒तम॑ग्ने॒ मधु॑म॒त् पिन्व॑मानः । वा॒जी वह॑न् वा॒जिनं॑ जातवेदो दे॒वानां᳚ ॅवक्षि प्रि॒यमा स॒धस्थं᳚ ॥घृ॒तेना॒ञ्जन्थ्सं प॒थो दे॑व॒याना᳚न् प्रजा॒नन् वा॒ज्यप्ये॑तु दे॒वान् । अनु॑ त्वा सप्ते प्र॒दिशः॑ सचन्ताꣳ स्व॒धाम॒स्मै यज॑मानाय धेहि ॥ ईड्य॒श्चासि॒ वन्द्य॑श्च वाजिन्ना॒शुश्चासि॒ मेद्ध्य॑श्च सप्ते । अ॒ग्निष्ट्वा॑ - [  ] \newline

\textbf{Pada Paata} \newline

समि॑द्ध इति॒ सं - इ॒द्धः॒ । अ॒ञ्जन्न् । कृद॑रम् । म॒ती॒नाम् । घृ॒तम् । अ॒ग्ने॒ । मधु॑म॒दिति॒ मधु॑ - म॒त् । पिन्व॑मानः ॥ वा॒जी । वहन्न्॑ । वा॒जिन᳚म् । जा॒त॒वे॒द॒ इति॑ जात - वे॒दः॒ । दे॒वाना᳚म् । व॒क्षि॒ । प्रि॒यम् । एति॑ । स॒धस्थ॒मिति॑ स॒ध - स्थ॒म् ॥ घृ॒तेन॑ । अ॒ञ्जन्न् । समिति॑ । प॒थः । दे॒व॒याना॒निति॑ देव - यानान्॑ । प्र॒जा॒नन्निति॑ प्र - जा॒नन्न् । वा॒जी । अपीति॑ । ए॒तु॒ । दे॒वान् ॥ अन्विति॑ । त्वा॒ । स॒प्ते॒ । प्र॒दिश॒ इति॑ प्र - दिशः॑ । स॒च॒न्ता॒म् । स्व॒धामिति॑ स्व - धाम् । अ॒स्मै । यज॑मानाय । धे॒हि॒ ॥ ईड्यः॑ । च॒ । असि॑ । वन्द्यः॑ । च॒ । वा॒जि॒न्न् । आ॒शुः । च॒ । असि॑ । मेद्ध्यः॑ । च॒ । स॒प्ते॒ ॥ अ॒ग्निः । त्वा॒ ।  \newline


\textbf{Krama Paata} \newline

समि॑द्धो अ॒ञ्जन्न् । समि॑द्ध॒ इति॒ सम् - इ॒द्धः॒ । अ॒ञ्जन् कृद॑रम् । कृद॑रम् मती॒नाम् । म॒ती॒नाम् घृ॒तम् । घृ॒तम॑ग्ने । अ॒ग्ने॒ मधु॑मत् । मधु॑म॒त् पिन्व॑मानः । मधु॑म॒दिति॒ मधु॑ - म॒त्॒ । पिन्व॑मान॒ इति॒ पिन्व॑मानः ॥ वा॒जी वहन्न्॑ । वह॑न् वा॒जिन᳚म् । वा॒जिन॑म् जातवेदः । जा॒त॒वे॒दो॒ दे॒वाना᳚म् । जा॒त॒वे॒द॒ इति॑ जात - वे॒दः॒ । दे॒वाना᳚म् ॅवक्षि । व॒क्षि॒ प्रि॒यम् । प्रि॒यमा । आ स॒धस्थ᳚म् । स॒धस्थ॒मिति॑ स॒ध - स्थ॒म् ॥ घृ॒तेना॒ऽञ्जन्न् । अ॒ञ्जन्थ् सम् । सम् प॒थः । प॒थो दे॑व॒यानान्॑ । दे॒व॒याना᳚न् प्रजा॒नन्न् । दे॒व॒याना॒निति॑ देव - यानान्॑ । प्र॒जा॒नन् वा॒जी । प्र॒जा॒नन्निति॑ प्र - जा॒नन्न् । वा॒ज्यपि॑ । अप्ये॑तु । ए॒तु॒ दे॒वान् । दे॒वानिति॑ दे॒वान् ॥ अनु॑ त्वा । त्वा॒ स॒प्ते॒ । स॒प्ते॒ प्र॒दिशः॑ । प्र॒दिशः॑ सचन्ताम् । प्र॒दिश॒ इति॑ प्र - दिशः॑ । स॒च॒न्ताꣳ॒॒ स्व॒धाम् । स्व॒धाम॒स्मै । स्व॒धामिति॑ स्व - धाम् । अ॒स्मै यज॑मानाय । यज॑मनाय धेहि । धे॒हीति॑ धेहि ॥ ईड्य॑श्च । चासि॑ । असि॒ वन्द्यः॑ । वन्द्य॑श्च । च॒ वा॒जि॒न्न्॒ । वा॒जि॒न्ना॒शुः । आ॒शुश्च॑ । चासि॑ । असि॒ मेद्ध्यः॑ । मेद्ध्य॑श्च । च॒ स॒प्ते॒ । स॒प्त॒ इति॑ सप्ते ॥ अ॒ग्निष्ट्वा᳚ । त्वा॒ दे॒वैः \newline

\textbf{Jatai Paata} \newline

1. समि॑द्धो अ॒ञ्जन् न॒ञ्जन् थ्समि॑द्धः॒ समि॑द्धो अ॒ञ्जन्न् । \newline
2. समि॑द्ध॒ इति॒ सं - इ॒द्धः॒ । \newline
3. अ॒ञ्जन् कृद॑र॒म् कृद॑र म॒ञ्जन् न॒ञ्जन् कृद॑रम् । \newline
4. कृद॑रम् मती॒नाम् म॑ती॒नाम् कृद॑र॒म् कृद॑रम् मती॒नाम् । \newline
5. म॒ती॒नाम् घृ॒तम् घृ॒तम् म॑ती॒नाम् म॑ती॒नाम् घृ॒तम् । \newline
6. घृ॒त म॑ग्ने ऽग्ने घृ॒तम् घृ॒त म॑ग्ने । \newline
7. अ॒ग्ने॒ मधु॑म॒न् मधु॑म दग्ने ऽग्ने॒ मधु॑मत् । \newline
8. मधु॑म॒त् पिन्व॑मानः॒ पिन्व॑मानो॒ मधु॑म॒न् मधु॑म॒त् पिन्व॑मानः । \newline
9. मधु॑म॒दिति॒ मधु॑ - म॒त् । \newline
10. पिन्व॑मान॒ इति॒ पिन्व॑मानः । \newline
11. वा॒जी वह॒न्॒. वह॑न्. वा॒जी वा॒जी वहन्न्॑ । \newline
12. वह॑न्. वा॒जिनं॑ ॅवा॒जिनं॒ ॅवह॒न्॒. वह॑न्. वा॒जिन᳚म् । \newline
13. वा॒जिन॑म् जातवेदो जातवेदो वा॒जिनं॑ ॅवा॒जिन॑म् जातवेदः । \newline
14. जा॒त॒वे॒दो॒ दे॒वाना᳚म् दे॒वाना᳚म् जातवेदो जातवेदो दे॒वाना᳚म् । \newline
15. जा॒त॒वे॒द॒ इति॑ जात - वे॒दः॒ । \newline
16. दे॒वानां᳚ ॅवक्षि वक्षि दे॒वाना᳚म् दे॒वानां᳚ ॅवक्षि । \newline
17. व॒क्षि॒ प्रि॒यम् प्रि॒यं ॅव॑क्षि वक्षि प्रि॒यम् । \newline
18. प्रि॒य मा प्रि॒यम् प्रि॒य मा । \newline
19. आ स॒धस्थꣳ॑ स॒धस्थ॒ मा स॒धस्थ᳚म् । \newline
20. स॒धस्थ॒मिति॑ स॒ध - स्थ॒म् । \newline
21. घृ॒तेना॒ञ्जन् न॒ञ्जन् घृ॒तेन॑ घृ॒तेना॒ञ्जन्न् । \newline
22. अ॒ञ्जन् थ्सꣳ स म॒ञ्जन् न॒ञ्जन् थ्सम् । \newline
23. सम् प॒थः प॒थः सꣳ सम् प॒थः । \newline
24. प॒थो दे॑व॒याना᳚न् देव॒याना᳚न् प॒थः प॒थो दे॑व॒यानान्॑ । \newline
25. दे॒व॒याना᳚न् प्रजा॒नन् प्र॑जा॒नन् दे॑व॒याना᳚न् देव॒याना᳚न् प्रजा॒नन्न् । \newline
26. दे॒व॒याना॒निति॑ देव - यानान्॑ । \newline
27. प्र॒जा॒नन्. वा॒जी वा॒जी प्र॑जा॒नन् प्र॑जा॒नन्. वा॒जी । \newline
28. प्र॒जा॒नन्निति॑ प्र - जा॒नन्न् । \newline
29. वा॒ज्यप्यपि॑ वा॒जी वा॒ज्यपि॑ । \newline
30. अप्ये᳚त्वे॒ त्वप्य प्ये॑तु । \newline
31. ए॒तु॒ दे॒वान् दे॒वा ने᳚त्वेतु दे॒वान् । \newline
32. दे॒वानिति॑ दे॒वान् । \newline
33. अनु॑ त्वा॒ त्वा ऽन्वनु॑ त्वा । \newline
34. त्वा॒ स॒प्ते॒ स॒प्ते॒ त्वा॒ त्वा॒ स॒प्ते॒ । \newline
35. स॒प्ते॒ प्र॒दिशः॑ प्र॒दिशः॑ सप्ते सप्ते प्र॒दिशः॑ । \newline
36. प्र॒दिशः॑ सचन्ताꣳ सचन्ताम् प्र॒दिशः॑ प्र॒दिशः॑ सचन्ताम् । \newline
37. प्र॒दिश॒ इति॑ प्र - दिशः॑ । \newline
38. स॒च॒न्ताꣳ॒॒ स्व॒धाꣳ स्व॒धाꣳ स॑चन्ताꣳ सचन्ताꣳ स्व॒धाम् । \newline
39. स्व॒धा म॒स्मा अ॒स्मै स्व॒धाꣳ स्व॒धा म॒स्मै । \newline
40. स्व॒धामिति॑ स्व - धाम् । \newline
41. अ॒स्मै यज॑मानाय॒ यज॑मानाया॒स्मा अ॒स्मै यज॑मानाय । \newline
42. यज॑मानाय धेहि धेहि॒ यज॑मानाय॒ यज॑मानाय धेहि । \newline
43. धे॒हीति॑ धेहि । \newline
44. ईड्य॑श्च॒ चेड्य॒ ईड्य॑श्च । \newline
45. चास्यसि॑ च॒ चासि॑ । \newline
46. असि॒ वन्द्यो॒ वन्द्यो ऽस्यसि॒ वन्द्यः॑ । \newline
47. वन्द्य॑श्च च॒ वन्द्यो॒ वन्द्य॑श्च । \newline
48. च॒ वा॒जि॒न्॒. वा॒जिꣳ॒॒ श्च॒च॒ वा॒जि॒न्न् । \newline
49. वा॒जि॒न् ना॒शु रा॒शुर् वा॑जिन्. वाजिन् ना॒शुः । \newline
50. आ॒शुश्च॑ चा॒शु रा॒शुश्च॑ । \newline
51. चास्यसि॑ च॒ चासि॑ । \newline
52. असि॒ मेद्ध्यो॒ मेद्ध्यो ऽस्यसि॒ मेद्ध्यः॑ । \newline
53. मेद्ध्य॑श्च च॒ मेद्ध्यो॒ मेद्ध्य॑श्च । \newline
54. च॒ स॒प्ते॒ स॒प्ते॒ च॒ च॒ स॒प्ते॒ । \newline
55. स॒प्त॒ इति॑ सप्ते । \newline
56. आ॒ग्नि ष्ट्वा᳚ त्वा॒ ऽग्नि र॒ग्नि ष्ट्वा᳚ । \newline
57. त्वा॒ दे॒वैर् दे॒वै स्त्वा᳚ त्वा दे॒वैः । \newline

\textbf{Ghana Paata } \newline

1. समि॑द्धो अ॒ञ्जन् न॒ञ्जन् थ्समि॑द्धः॒ समि॑द्धो अ॒ञ्जन् कृद॑र॒म् कृद॑र म॒ञ्जन् थ्समि॑द्धः॒ समि॑द्धो अ॒ञ्जन् कृद॑रम् । \newline
2. समि॑द्ध॒ इति॒ सं - इ॒द्धः॒ । \newline
3. अ॒ञ्जन् कृद॑र॒म् कृद॑र म॒ञ्जन् न॒ञ्जन् कृद॑रम् मती॒नाम् म॑ती॒नाम् कृद॑र म॒ञ्जन् न॒ञ्जन् कृद॑रम् मती॒नाम् । \newline
4. कृद॑रम् मती॒नाम् म॑ती॒नाम् कृद॑र॒म् कृद॑रम् मती॒नाम् घृ॒तम् घृ॒तम् म॑ती॒नाम् कृद॑र॒म् कृद॑रम् मती॒नाम् घृ॒तम् । \newline
5. म॒ती॒नाम् घृ॒तम् घृ॒तम् म॑ती॒नाम् म॑ती॒नाम् घृ॒त म॑ग्ने ऽग्ने घृ॒तम् म॑ती॒नाम् म॑ती॒नाम् घृ॒त म॑ग्ने । \newline
6. घृ॒त म॑ग्ने ऽग्ने घृ॒तम् घृ॒त म॑ग्ने॒ मधु॑म॒न् मधु॑मदग्ने घृ॒तम् घृ॒त म॑ग्ने॒ मधु॑मत् । \newline
7. अ॒ग्ने॒ मधु॑म॒न् मधु॑मदग्ने ऽग्ने॒ मधु॑म॒त् पिन्व॑मानः॒ पिन्व॑मानो॒ मधु॑मदग्ने ऽग्ने॒ मधु॑म॒त् पिन्व॑मानः । \newline
8. मधु॑म॒त् पिन्व॑मानः॒ पिन्व॑मानो॒ मधु॑म॒न् मधु॑म॒त् पिन्व॑मानः । \newline
9. मधु॑म॒दिति॒ मधु॑ - म॒त् । \newline
10. पिन्व॑मान॒ इति॒ पिन्व॑मानः । \newline
11. वा॒जी वह॒न्॒. वह॑न्. वा॒जी वा॒जी वह॑न्. वा॒जिनं॑ ॅवा॒जिनं॒ ॅवह॑न्. वा॒जी वा॒जी वह॑न्. वा॒जिन᳚म् । \newline
12. वह॑न्. वा॒जिनं॑ ॅवा॒जिनं॒ ॅवह॒न्॒. वह॑न्. वा॒जिन॑म् जातवेदो जातवेदो वा॒जिनं॒ ॅवह॒न्॒. वह॑न्. वा॒जिन॑म् जातवेदः । \newline
13. वा॒जिन॑म् जातवेदो जातवेदो वा॒जिनं॑ ॅवा॒जिन॑म् जातवेदो दे॒वाना᳚म् दे॒वाना᳚म् जातवेदो वा॒जिनं॑ ॅवा॒जिन॑म् जातवेदो दे॒वाना᳚म् । \newline
14. जा॒त॒वे॒दो॒ दे॒वाना᳚म् दे॒वाना᳚म् जातवेदो जातवेदो दे॒वानां᳚ ॅवक्षि वक्षि दे॒वाना᳚म् जातवेदो जातवेदो दे॒वानां᳚ ॅवक्षि । \newline
15. जा॒त॒वे॒द॒ इति॑ जात - वे॒दः॒ । \newline
16. दे॒वानां᳚ ॅवक्षि वक्षि दे॒वाना᳚म् दे॒वानां᳚ ॅवक्षि प्रि॒यम् प्रि॒यं ॅव॑क्षि दे॒वाना᳚म् दे॒वानां᳚ ॅवक्षि प्रि॒यम् । \newline
17. व॒क्षि॒ प्रि॒यम् प्रि॒यं ॅव॑क्षि वक्षि प्रि॒य मा प्रि॒यं ॅव॑क्षि वक्षि प्रि॒य मा । \newline
18. प्रि॒य मा प्रि॒यम् प्रि॒य मा स॒धस्थꣳ॑ स॒धस्थ॒ मा प्रि॒यम् प्रि॒य मा स॒धस्थ᳚म् । \newline
19. आ स॒धस्थꣳ॑ स॒धस्थ॒ मा स॒धस्थ᳚म् । \newline
20. स॒धस्थ॒मिति॑ स॒ध - स्थ॒म् । \newline
21. घृ॒तेना॒ञ्जन् न॒ञ्जन् घृ॒तेन॑ घृ॒तेना॒ञ्जन् थ्सꣳ स म॒ञ्जन् घृ॒तेन॑ घृ॒तेना॒ञ्जन् थ्सम् । \newline
22. अ॒ञ्जन् थ्सꣳ स म॒ञ्जन् न॒ञ्जन् थ्सम् प॒थः प॒थः स म॒ञ्जन् न॒ञ्जन् थ्सम् प॒थः । \newline
23. सम् प॒थः प॒थः सꣳ सम् प॒थो दे॑व॒याना᳚न् देव॒याना᳚न् प॒थः सꣳ सम् प॒थो दे॑व॒यानान्॑ । \newline
24. प॒थो दे॑व॒याना᳚न् देव॒याना᳚न् प॒थः प॒थो दे॑व॒याना᳚न् प्रजा॒नन् प्र॑जा॒नन् दे॑व॒याना᳚न् प॒थः प॒थो दे॑व॒याना᳚न् प्रजा॒नन्न् । \newline
25. दे॒व॒याना᳚न् प्रजा॒नन् प्र॑जा॒नन् दे॑व॒याना᳚न् देव॒याना᳚न् प्रजा॒नन्. वा॒जी वा॒जी प्र॑जा॒नन् दे॑व॒याना᳚न् देव॒याना᳚न् प्रजा॒नन्. वा॒जी । \newline
26. दे॒व॒याना॒निति॑ देव - यानान्॑ । \newline
27. प्र॒जा॒नन्. वा॒जी वा॒जी प्र॑जा॒नन् प्र॑जा॒नन्. वा॒ज्यप्यपि॑ वा॒जी प्र॑जा॒नन् प्र॑जा॒नन्. वा॒ज्यपि॑ । \newline
28. प्र॒जा॒नन्निति॑ प्र - जा॒नन्न् । \newline
29. वा॒ज्यप्यपि॑ वा॒जी वा॒ज्यप्ये᳚ त्वे॒त्वपि॑ वा॒जी वा॒ज्य प्ये॑तु । \newline
30. अप्ये᳚त्वे॒ त्वप्यप्ये॑तु दे॒वान् दे॒वा ने॒त्वप्यप्ये॑तु दे॒वान् । \newline
31. ए॒तु॒ दे॒वान् दे॒वा ने᳚त्वेतु दे॒वान् । \newline
32. दे॒वानिति॑ दे॒वान् । \newline
33. अनु॑ त्वा॒ त्वा ऽन्वनु॑ त्वा सप्ते सप्ते॒ त्वा ऽन्वनु॑ त्वा सप्ते । \newline
34. त्वा॒ स॒प्ते॒ स॒प्ते॒ त्वा॒ त्वा॒ स॒प्ते॒ प्र॒दिशः॑ प्र॒दिशः॑ सप्ते त्वा त्वा सप्ते प्र॒दिशः॑ । \newline
35. स॒प्ते॒ प्र॒दिशः॑ प्र॒दिशः॑ सप्ते सप्ते प्र॒दिशः॑ सचन्ताꣳ सचन्ताम् प्र॒दिशः॑ सप्ते सप्ते प्र॒दिशः॑ सचन्ताम् । \newline
36. प्र॒दिशः॑ सचन्ताꣳ सचन्ताम् प्र॒दिशः॑ प्र॒दिशः॑ सचन्ताꣳ स्व॒धाꣳ स्व॒धाꣳ स॑चन्ताम् प्र॒दिशः॑ प्र॒दिशः॑ सचन्ताꣳ स्व॒धाम् । \newline
37. प्र॒दिश॒ इति॑ प्र - दिशः॑ । \newline
38. स॒च॒न्ताꣳ॒॒ स्व॒धाꣳ स्व॒धाꣳ स॑चन्ताꣳ सचन्ताꣳ स्व॒धा म॒स्मा अ॒स्मै स्व॒धाꣳ स॑चन्ताꣳ सचन्ताꣳ स्व॒धा म॒स्मै । \newline
39. स्व॒धा म॒स्मा अ॒स्मै स्व॒धाꣳ स्व॒धा म॒स्मै यज॑मानाय॒ यज॑मानाया॒स्मै स्व॒धाꣳ स्व॒धा म॒स्मै यज॑मानाय । \newline
40. स्व॒धामिति॑ स्व - धाम् । \newline
41. अ॒स्मै यज॑मानाय॒ यज॑मानाया॒स्मा अ॒स्मै यज॑मानाय धेहि धेहि॒ यज॑मानाया॒स्मा अ॒स्मै यज॑मानाय धेहि । \newline
42. यज॑मानाय धेहि धेहि॒ यज॑मानाय॒ यज॑मानाय धेहि । \newline
43. धे॒हीति॑ धेहि । \newline
44. ईड्य॑श्च॒ चेड्य॒ ईड्य॒ श्चास्यसि॒ चेड्य॒ ईड्य॒ श्चासि॑ । \newline
45. चास्यसि॑ च॒ चासि॒ वन्द्यो॒ वन्द्यो ऽसि॑ च॒ चासि॒ वन्द्यः॑ । \newline
46. असि॒ वन्द्यो॒ वन्द्यो ऽस्यसि॒ वन्द्य॑श्च च॒ वन्द्यो ऽस्यसि॒ वन्द्य॑श्च । \newline
47. वन्द्य॑श्च च॒ वन्द्यो॒ वन्द्य॑श्च वाजिन्. वाजिꣳश्च॒ वन्द्यो॒ वन्द्य॑श्च वाजिन्न् । \newline
48. च॒ वा॒जि॒न्॒. वा॒जिꣳ॒॒ श्च॒ च॒ वा॒जि॒न् ना॒शु रा॒शुर् वा॑जिꣳ श्च च वाजिन् ना॒शुः । \newline
49. वा॒जि॒न् ना॒शु रा॒शुर् वा॑जिन्. वाजिन् ना॒शुश्च॑ चा॒शुर् वा॑जिन्. वाजिन् ना॒शुश्च॑ । \newline
50. आ॒शुश्च॑ चा॒शु रा॒शु श्चास्यसि॑ चा॒शु रा॒शु श्चासि॑ । \newline
51. चास्यसि॑ च॒ चासि॒ मेद्ध्यो॒ मेद्ध्यो ऽसि॑ च॒ चासि॒ मेद्ध्यः॑ । \newline
52. असि॒ मेद्ध्यो॒ मेद्ध्यो ऽस्यसि॒ मेद्ध्य॑श्च च॒ मेद्ध्यो ऽस्यसि॒ मेद्ध्य॑श्च । \newline
53. मेद्ध्य॑श्च च॒ मेद्ध्यो॒ मेद्ध्य॑श्च सप्ते सप्ते च॒ मेद्ध्यो॒ मेद्ध्य॑श्च सप्ते । \newline
54. च॒ स॒प्ते॒ स॒प्ते॒ च॒ च॒ स॒प्ते॒ । \newline
55. स॒प्त॒ इति॑ सप्ते । \newline
56. अ॒ग्नि ष्ट्वा᳚ त्वा॒ ऽग्नि र॒ग्नि ष्ट्वा॑ दे॒वैर् दे॒वै स्त्वा॒ ऽग्नि र॒ग्नि ष्ट्वा॑ दे॒वैः । \newline
57. त्वा॒ दे॒वैर् दे॒वै स्त्वा᳚ त्वा दे॒वैर् वसु॑भि॒र् वसु॑भिर् दे॒वै स्त्वा᳚ त्वा दे॒वैर् वसु॑भिः । \newline
\pagebreak
\markright{ TS 5.1.11.2  \hfill https://www.vedavms.in \hfill}

\section{ TS 5.1.11.2 }

\textbf{TS 5.1.11.2 } \newline
\textbf{Samhita Paata} \newline

दे॒वैर्वसु॑भिः स॒जोषाः᳚ प्री॒तं ॅवह्निं॑ ॅवहतु जा॒तवे॑दाः ॥ स्ती॒र्णम् ब॒र्॒.हिः सु॒ष्टरी॑मा जुषा॒णोरु पृ॒थु प्रथ॑मानं पृथि॒व्यां । दे॒वेभि॑र्यु॒क्तमदि॑तिः स॒जोषाः᳚ स्यो॒नं कृ॑ण्वा॒ना सु॑वि॒ते द॑धातु ॥ए॒ता उ॑ वः सु॒भगा॑ वि॒श्वरू॑पा॒ विपक्षो॑भिः॒ श्रय॑माणा॒ उदातैः᳚ ।ऋ॒ष्वाः स॒तीः क॒वषः॒ शुम्भ॑माना॒ द्वारो॑ दे॒वीः सु॑प्राय॒णा भ॑वन्तु ॥अ॒न्त॒रा मि॒त्रावरु॑णा॒ चर॑न्ती॒ मुखं॑ ॅय॒ज्ञाना॑म॒भि सं॑ॅविदा॒ने । उ॒षासा॑ वाꣳ - [  ] \newline

\textbf{Pada Paata} \newline

दे॒वैः । वसु॑भि॒रिति॒ वसु॑ - भिः॒ । स॒जोषा॒ इति॑ स - जोषाः᳚ । प्री॒तम् । वह्नि᳚म् । व॒ह॒तु॒ । जा॒तवे॑दा॒ इति॑ जा॒त - वे॒दाः॒ ॥ स्ती॒र्णम् । ब॒र्॒.हिः । सु॒ष्टरी॒मेति॑ सु - स्तरी॑म । जु॒षा॒णा । उ॒रु । पृ॒थु । प्रथ॑मानम् । पृ॒थि॒व्याम् ॥ दे॒वेभिः॑ । यु॒क्तम् । अदि॑तिः । स॒जोषा॒ इति॑ स - जोषाः᳚ । स्यो॒नम् । कृ॒ण्वा॒ना । सु॒वि॒ते । द॒धा॒तु॒ ॥ ए॒ताः । उ॒ । वः॒ । सु॒भगा॒ इति॑ सु - भगाः᳚ । वि॒श्वरू॑पा॒ इति॑ वि॒श्व - रू॒पाः॒ । वीति॑ । पक्षो॑भि॒रिति॒ पक्षः॑ - भिः॒ । श्रय॑माणाः । उदिति॑ । आतैः᳚ ॥ ऋ॒ष्वाः । स॒तीः । क॒वषः॑ । शुंभ॑मानाः । द्वारः॑ । दे॒वीः । सु॒प्रा॒य॒णा इति॑ सु - प्रा॒य॒णाः । भ॒व॒न्तु॒ ॥ अ॒न्त॒रा । मि॒त्रावरु॒णेति॑ मि॒त्रा - वरु॑णा । चर॑न्ती इति॑ । मुख᳚म् । य॒ज्ञाना᳚म् । अ॒भीति॑ । सं॒ॅवि॒दा॒ने इति॑ सं - वि॒दा॒ने ॥ उ॒षासा᳚ । वा॒म् ।  \newline


\textbf{Krama Paata} \newline

दे॒वैर् वसु॑भिः । वसु॑भिः स॒जोषाः᳚ । वसु॑भि॒रिति॒ वसु॑ - भिः॒ । स॒जोषाः᳚ प्री॒तम् । स॒जोषा॒ इति॑ स - जोषाः᳚ । प्री॒तम् ॅवह्नि᳚म् । वह्नि॑म् ॅवहतु । व॒ह॒तु॒ जा॒तवे॑दाः । जा॒तवे॑दा॒ इति॑ जा॒त - वे॒दाः॒ ॥ स्ती॒र्णम् ब॒र्॒.हिः । ब॒र्॒.हिः सु॒ष्टरी॑म । सु॒ष्टरी॑मा जुषा॒णा । सु॒ष्टरी॒मेति॑ सु - स्तरी॑म । जु॒षा॒णोरु । उ॒रु पृ॒थु । पृ॒थु प्रथ॑मानम् । प्रथ॑मानम् पृथि॒व्याम् । पृ॒थि॒व्यामिति॑ पृथि॒व्याम् ॥ दे॒वेभि॑र् यु॒क्तम् । यु॒क्तमदि॑तिः । अदि॑तिः स॒जोषाः᳚ । स॒जोषाः᳚ स्यो॒नम् । स॒जोषा॒ इति॑ स - जोषाः᳚ । स्यो॒नम् कृ॑ण्वा॒ना । कृ॒ण्वा॒ना सु॑वि॒ते । सु॒वि॒ते द॑धातु । द॒धा॒त्विति॑ दधातु ॥ ए॒ता उ॑ । उ॒ वः॒ । वः॒ सु॒भगाः᳚ । सु॒भगा॑ वि॒श्वरू॑पाः । सु॒भगा॒ इति॑ सु - भगाः᳚ । वि॒श्वरू॑पा॒ वि । वि॒श्वरू॑पा॒ इति॑ वि॒श्व - रू॒पाः॒ । वि पक्षो॑भिः । पक्षो॑भिः॒ श्रय॑माणाः । पक्षो॑भि॒रिति॒ पक्षः॑ - भिः॒ । श्रय॑माणा॒ उत् । उदातैः᳚ । आतै॒रित्यातैः᳚ ॥ ऋ॒ष्वाः स॒तीः । स॒तीः क॒वषः॑ । क॒वषः॒ शुम्भ॑मानाः । शुम्भ॑माना॒ द्वारः॑ । द्वारो॑ दे॒वीः । दे॒वीः सु॑प्राय॒णाः । सु॒प्रा॒य॒णा भ॑वन्तु । सु॒प्रा॒य॒णा इति॑ सु - प्रा॒य॒णाः । भ॒व॒न्त्विति॑ भवन्तु ॥ अ॒न्त॒रा मि॒त्रावरु॑णा । मि॒त्रावरु॑णा॒ चर॑न्ती । मि॒त्रावरु॒णेति॑ मि॒त्रा - वरु॑णा । चर॑न्ती॒ मुख᳚म् । चर॑न्ती॒ इति॒ चर॑न्ती । मुख॑म् ॅय॒ज्ञाना᳚म् । य॒ज्ञाना॑म॒भि । अ॒भि स॑म्ॅविदा॒ने । स॒म्ॅवि॒दा॒ने इति॑ सम् - वि॒दा॒ने ॥ उ॒षासा॑ वाम् । वाꣳ॒॒ सु॒हि॒र॒ण्ये \newline

\textbf{Jatai Paata} \newline

1. दे॒वैर् वसु॑भि॒र् वसु॑भिर् दे॒वैर् दे॒वैर् वसु॑भिः । \newline
2. वसु॑भिः स॒जोषाः᳚ स॒जोषा॒ वसु॑भि॒र् वसु॑भिः स॒जोषाः᳚ । \newline
3. वसु॑भि॒रिति॒ वसु॑ - भिः॒ । \newline
4. स॒जोषाः᳚ प्री॒तम् प्री॒तꣳ स॒जोषाः᳚ स॒जोषाः᳚ प्री॒तम् । \newline
5. स॒जोषा॒ इति॑ स - जोषाः᳚ । \newline
6. प्री॒तं ॅवह्निं॒ ॅवह्नि॑म् प्री॒तम् प्री॒तं ॅवह्नि᳚म् । \newline
7. वह्निं॑ ॅवहतु वहतु॒ वह्निं॒ ॅवह्निं॑ ॅवहतु । \newline
8. व॒ह॒तु॒ जा॒तवे॑दा जा॒तवे॑दा वहतु वहतु जा॒तवे॑दाः । \newline
9. जा॒तवे॑दा॒ इति॑ जा॒त - वे॒दाः॒ । \newline
10. स्ती॒र्णम् ब॒र्॒.हिर् ब॒र्॒.हिः स्ती॒र्णꣳ स्ती॒र्णम् ब॒र्॒.हिः । \newline
11. ब॒र्॒.हिः सु॒ष्टरी॑म सु॒ष्टरी॑म ब॒र्॒.हिर् ब॒र्॒.हिः सु॒ष्टरी॑म । \newline
12. सु॒ष्टरी॑मा जुषा॒णा जु॑षा॒णा सु॒ष्टरी॑म सु॒ष्टरी॑मा जुषा॒णा । \newline
13. सु॒ष्टरी॒मेति॑ सु - स्तरी॑म । \newline
14. जु॒षा॒णो रू॑रु जु॑षा॒णा जु॑षा॒णोरु । \newline
15. उ॒रु पृ॒थु पृ॒थू᳚(1॒)रू॑रु पृ॒थु । \newline
16. पृ॒थु प्रथ॑मान॒म् प्रथ॑मानम् पृ॒थु पृ॒थु प्रथ॑मानम् । \newline
17. प्रथ॑मानम् पृथि॒व्याम् पृ॑थि॒व्याम् प्रथ॑मान॒म् प्रथ॑मानम् पृथि॒व्याम् । \newline
18. पृ॒थि॒व्यामिति॑ पृथि॒व्याम् । \newline
19. दे॒वेभि॑र् यु॒क्तं ॅयु॒क्तम् दे॒वेभि॑र् दे॒वेभि॑र् यु॒क्तम् । \newline
20. यु॒क्त मदि॑ति॒ रदि॑तिर् यु॒क्तं ॅयु॒क्त मदि॑तिः । \newline
21. अदि॑तिः स॒जोषाः᳚ स॒जोषा॒ अदि॑ति॒ रदि॑तिः स॒जोषाः᳚ । \newline
22. स॒जोषाः᳚ स्यो॒नꣳ स्यो॒नꣳ स॒जोषाः᳚ स॒जोषाः᳚ स्यो॒नम् । \newline
23. स॒जोषा॒ इति॑ स - जोषाः᳚ । \newline
24. स्यो॒नम् कृ॑ण्वा॒ना कृ॑ण्वा॒ना स्यो॒नꣳ स्यो॒नम् कृ॑ण्वा॒ना । \newline
25. कृ॒ण्वा॒ना सु॑वि॒ते सु॑वि॒ते कृ॑ण्वा॒ना कृ॑ण्वा॒ना सु॑वि॒ते । \newline
26. सु॒वि॒ते द॑धातु दधातु सुवि॒ते सु॑वि॒ते द॑धातु । \newline
27. द॒धा॒त्विति॑ दधातु । \newline
28. ए॒ता उ॑ वु वे॒ता ए॒ता उ॑ । \newline
29. उ॒ वो॒ व॒ उ॒ वु॒ वः॒ । \newline
30. वः॒ सु॒भगाः᳚ सु॒भगा॑ वो वः सु॒भगाः᳚ । \newline
31. सु॒भगा॑ वि॒श्वरू॑पा वि॒श्वरू॑पाः सु॒भगाः᳚ सु॒भगा॑ वि॒श्वरू॑पाः । \newline
32. सु॒भगा॒ इति॑ सु - भगाः᳚ । \newline
33. वि॒श्वरू॑पा॒ वि वि वि॒श्वरू॑पा वि॒श्वरू॑पा॒ वि । \newline
34. वि॒श्वरू॑पा॒ इति॑ वि॒श्व - रू॒पाः॒ । \newline
35. वि पक्षो॑भिः॒ पक्षो॑भि॒र् वि वि पक्षो॑भिः । \newline
36. पक्षो॑भिः॒ श्रय॑माणाः॒ श्रय॑माणाः॒ पक्षो॑भिः॒ पक्षो॑भिः॒ श्रय॑माणाः । \newline
37. पक्षो॑भि॒रिति॒ पक्षः॑ - भिः॒ । \newline
38. श्रय॑माणा॒ उदुच्छ्रय॑माणाः॒ श्रय॑माणा॒ उत् । \newline
39. उदातै॒ रातै॒ रुदु दातैः᳚ । \newline
40. आतै॒रित्यातैः᳚ । \newline
41. ऋ॒ष्वाः स॒तीः स॒तीर्. ऋ॒ष्वा ऋ॒ष्वाः स॒तीः । \newline
42. स॒तीः क॒वषः॑ क॒वषः॑ स॒तीः स॒तीः क॒वषः॑ । \newline
43. क॒वषः॒ शुंभ॑मानाः॒ शुंभ॑मानाः क॒वषः॑ क॒वषः॒ शुंभ॑मानाः । \newline
44. शुंभ॑माना॒ द्वारो॒ द्वारः॒ शुंभ॑मानाः॒ शुंभ॑माना॒ द्वारः॑ । \newline
45. द्वारो॑ दे॒वीर् दे॒वीर् द्वारो॒ द्वारो॑ दे॒वीः । \newline
46. दे॒वीः सु॑प्राय॒णाः सु॑प्राय॒णा दे॒वीर् दे॒वीः सु॑प्राय॒णाः । \newline
47. सु॒प्रा॒य॒णा भ॑वन्तु भवन्तु सुप्राय॒णाः सु॑प्राय॒णा भ॑वन्तु । \newline
48. सु॒प्रा॒य॒णा इति॑ सु - प्रा॒य॒णाः । \newline
49. भ॒व॒न्त्विति॑ भवन्तु । \newline
50. अ॒न्त॒रा मि॒त्रावरु॑णा मि॒त्रावरु॑णा ऽन्त॒रा ऽन्त॒रा मि॒त्रावरु॑णा । \newline
51. मि॒त्रावरु॑णा॒ चर॑न्ती॒ चर॑न्ती मि॒त्रावरु॑णा मि॒त्रावरु॑णा॒ चर॑न्ती । \newline
52. मि॒त्रावरु॒णेति॑ मि॒त्रा - वरु॑णा । \newline
53. चर॑न्ती॒ मुख॒म् मुख॒म् चर॑न्ती॒ चर॑न्ती॒ मुख᳚म् । \newline
54. चर॑न्ती॒ इति॒ चर॑न्ती । \newline
55. मुखं॑ ॅय॒ज्ञानां᳚ ॅय॒ज्ञाना॒म् मुख॒म् मुखं॑ ॅय॒ज्ञाना᳚म् । \newline
56. य॒ज्ञाना॑ म॒भ्य॑भि य॒ज्ञानां᳚ ॅय॒ज्ञाना॑ म॒भि । \newline
57. अ॒भि सं॑ॅविदा॒ने सं॑ॅविदा॒ने अ॒भ्य॑भि सं॑ॅविदा॒ने । \newline
58. सं॒ॅवि॒दा॒ने इति॑ सं - वि॒दा॒ने । \newline
59. उ॒षासा॑ वां ॅवा मु॒षासो॒षासा॑ वाम् । \newline
60. वाꣳ॒॒ सु॒हि॒र॒ण्ये सु॑हिर॒ण्ये वां᳚ ॅवाꣳ सुहिर॒ण्ये । \newline

\textbf{Ghana Paata } \newline

1. दे॒वैर् वसु॑भि॒र् वसु॑भिर् दे॒वैर् दे॒वैर् वसु॑भिः स॒जोषाः᳚ स॒जोषा॒ वसु॑भिर् दे॒वैर् दे॒वैर् वसु॑भिः स॒जोषाः᳚ । \newline
2. वसु॑भिः स॒जोषाः᳚ स॒जोषा॒ वसु॑भि॒र् वसु॑भिः स॒जोषाः᳚ प्री॒तम् प्री॒तꣳ स॒जोषा॒ वसु॑भि॒र् वसु॑भिः स॒जोषाः᳚ प्री॒तम् । \newline
3. वसु॑भि॒रिति॒ वसु॑ - भिः॒ । \newline
4. स॒जोषाः᳚ प्री॒तम् प्री॒तꣳ स॒जोषाः᳚ स॒जोषाः᳚ प्री॒तं ॅवह्निं॒ ॅवह्नि॑म् प्री॒तꣳ स॒जोषाः᳚ स॒जोषाः᳚ प्री॒तं ॅवह्नि᳚म् । \newline
5. स॒जोषा॒ इति॑ स - जोषाः᳚ । \newline
6. प्री॒तं ॅवह्निं॒ ॅवह्नि॑म् प्री॒तम् प्री॒तं ॅवह्निं॑ ॅवहतु वहतु॒ वह्नि॑म् प्री॒तम् प्री॒तं ॅवह्निं॑ ॅवहतु । \newline
7. वह्निं॑ ॅवहतु वहतु॒ वह्निं॒ ॅवह्निं॑ ॅवहतु जा॒तवे॑दा जा॒तवे॑दा वहतु॒ वह्निं॒ ॅवह्निं॑ ॅवहतु जा॒तवे॑दाः । \newline
8. व॒ह॒तु॒ जा॒तवे॑दा जा॒तवे॑दा वहतु वहतु जा॒तवे॑दाः । \newline
9. जा॒तवे॑दा॒ इति॑ जा॒त - वे॒दाः॒ । \newline
10. स्ती॒र्णम् ब॒र्॒.हिर् ब॒र्॒.हिः स्ती॒र्णꣳ स्ती॒र्णम् ब॒र्॒.हिः सु॒ष्टरी॑म सु॒ष्टरी॑म ब॒र्॒.हिः स्ती॒र्णꣳ स्ती॒र्णम् ब॒र्॒.हिः सु॒ष्टरी॑म । \newline
11. ब॒र्॒.हिः सु॒ष्टरी॑म सु॒ष्टरी॑म ब॒र्॒.हिर् ब॒र्॒.हिः सु॒ष्टरी॑मा जुषा॒णा जु॑षा॒णा सु॒ष्टरी॑म ब॒र्॒.हिर् ब॒र्॒.हिः सु॒ष्टरी॑मा जुषा॒णा । \newline
12. सु॒ष्टरी॑मा जुषा॒णा जु॑षा॒णा सु॒ष्टरी॑म सु॒ष्टरी॑मा जुषा॒ णोरू॑रु जु॑षा॒णा सु॒ष्टरी॑म सु॒ष्टरी॑मा जुषा॒णोरु । \newline
13. सु॒ष्टरी॒मेति॑ सु - स्तरी॑म । \newline
14. जु॒षा॒णोरू॑रु जु॑षा॒णा जु॑षा॒णोरु पृ॒थु पृ॒थू॑रु जु॑षा॒णा जु॑षा॒णोरु पृ॒थु । \newline
15. उ॒रु पृ॒थु पृ॒थू᳚(1॒)रू॑रु पृ॒थु प्रथ॑मान॒म् प्रथ॑मानम् पृ॒थू᳚(1॒)रू॑रु पृ॒थु प्रथ॑मानम् । \newline
16. पृ॒थु प्रथ॑मान॒म् प्रथ॑मानम् पृ॒थु पृ॒थु प्रथ॑मानम् पृथि॒व्याम् पृ॑थि॒व्याम् प्रथ॑मानम् पृ॒थु पृ॒थु प्रथ॑मानम् पृथि॒व्याम् । \newline
17. प्रथ॑मानम् पृथि॒व्याम् पृ॑थि॒व्याम् प्रथ॑मान॒म् प्रथ॑मानम् पृथि॒व्याम् । \newline
18. पृ॒थि॒व्यामिति॑ पृथि॒व्याम् । \newline
19. दे॒वेभि॑र् यु॒क्तं ॅयु॒क्तम् दे॒वेभि॑र् दे॒वेभि॑र् यु॒क्त मदि॑ति॒ रदि॑तिर् यु॒क्तम् दे॒वेभि॑र् दे॒वेभि॑र् यु॒क्त मदि॑तिः । \newline
20. यु॒क्त मदि॑ति॒ रदि॑तिर् यु॒क्तं ॅयु॒क्त मदि॑तिः स॒जोषाः᳚ स॒जोषा॒ अदि॑तिर् यु॒क्तं ॅयु॒क्त मदि॑तिः स॒जोषाः᳚ । \newline
21. अदि॑तिः स॒जोषाः᳚ स॒जोषा॒ अदि॑ति॒ रदि॑तिः स॒जोषाः᳚ स्यो॒नꣳ स्यो॒नꣳ स॒जोषा॒ अदि॑ति॒ रदि॑तिः स॒जोषाः᳚ स्यो॒नम् । \newline
22. स॒जोषाः᳚ स्यो॒नꣳ स्यो॒नꣳ स॒जोषाः᳚ स॒जोषाः᳚ स्यो॒नम् कृ॑ण्वा॒ना कृ॑ण्वा॒ना स्यो॒नꣳ स॒जोषाः᳚ स॒जोषाः᳚ स्यो॒नम् कृ॑ण्वा॒ना । \newline
23. स॒जोषा॒ इति॑ स - जोषाः᳚ । \newline
24. स्यो॒नम् कृ॑ण्वा॒ना कृ॑ण्वा॒ना स्यो॒नꣳ स्यो॒नम् कृ॑ण्वा॒ना सु॑वि॒ते सु॑वि॒ते कृ॑ण्वा॒ना स्यो॒नꣳ स्यो॒नम् कृ॑ण्वा॒ना सु॑वि॒ते । \newline
25. कृ॒ण्वा॒ना सु॑वि॒ते सु॑वि॒ते कृ॑ण्वा॒ना कृ॑ण्वा॒ना सु॑वि॒ते द॑धातु दधातु सुवि॒ते कृ॑ण्वा॒ना कृ॑ण्वा॒ना सु॑वि॒ते द॑धातु । \newline
26. सु॒वि॒ते द॑धातु दधातु सुवि॒ते सु॑वि॒ते द॑धातु । \newline
27. द॒धा॒त्विति॑ दधातु । \newline
28. ए॒ता उ॑ वु वे॒ता ए॒ता उ॑ वो व उ वे॒ता ए॒ता उ॑ वः । \newline
29. उ॒ वो॒ व॒ उ॒ वु॒ वः॒ सु॒भगाः᳚ सु॒भगा॑ व उ वु वः सु॒भगाः᳚ । \newline
30. वः॒ सु॒भगाः᳚ सु॒भगा॑ वो वः सु॒भगा॑ वि॒श्वरू॑पा वि॒श्वरू॑पाः सु॒भगा॑ वो वः सु॒भगा॑ वि॒श्वरू॑पाः । \newline
31. सु॒भगा॑ वि॒श्वरू॑पा वि॒श्वरू॑पाः सु॒भगाः᳚ सु॒भगा॑ वि॒श्वरू॑पा॒ वि वि वि॒श्वरू॑पाः सु॒भगाः᳚ सु॒भगा॑ वि॒श्वरू॑पा॒ वि । \newline
32. सु॒भगा॒ इति॑ सु - भगाः᳚ । \newline
33. वि॒श्वरू॑पा॒ वि वि वि॒श्वरू॑पा वि॒श्वरू॑पा॒ वि पक्षो॑भिः॒ पक्षो॑भि॒र् वि वि॒श्वरू॑पा वि॒श्वरू॑पा॒ वि पक्षो॑भिः । \newline
34. वि॒श्वरू॑पा॒ इति॑ वि॒श्व - रू॒पाः॒ । \newline
35. वि पक्षो॑भिः॒ पक्षो॑भि॒र् वि वि पक्षो॑भिः॒ श्रय॑माणाः॒ श्रय॑माणाः॒ पक्षो॑भि॒र् वि वि पक्षो॑भिः॒ श्रय॑माणाः । \newline
36. पक्षो॑भिः॒ श्रय॑माणाः॒ श्रय॑माणाः॒ पक्षो॑भिः॒ पक्षो॑भिः॒ श्रय॑माणा॒ उदु च्छ्रय॑माणाः॒ पक्षो॑भिः॒ पक्षो॑भिः॒ श्रय॑माणा॒ उत् । \newline
37. पक्षो॑भि॒रिति॒ पक्षः॑ - भिः॒ । \newline
38. श्रय॑माणा॒ उदु च्छ्रय॑माणाः॒ श्रय॑माणा॒ उदातै॒ रातै॒ रुच्छ्रय॑माणाः॒ श्रय॑माणा॒ उदातैः᳚ । \newline
39. उदातै॒ रातै॒ रुदु दातैः᳚ । \newline
40. आतै॒रित्यातैः᳚ । \newline
41. ऋ॒ष्वाः स॒तीः स॒तीर्. ऋ॒ष्वा ऋ॒ष्वाः स॒तीः क॒वषः॑ क॒वषः॑ स॒तीर्. ऋ॒ष्वा ऋ॒ष्वाः स॒तीः क॒वषः॑ । \newline
42. स॒तीः क॒वषः॑ क॒वषः॑ स॒तीः स॒तीः क॒वषः॒ शुंभ॑मानाः॒ शुंभ॑मानाः क॒वषः॑ स॒तीः स॒तीः क॒वषः॒ शुंभ॑मानाः । \newline
43. क॒वषः॒ शुंभ॑मानाः॒ शुंभ॑मानाः क॒वषः॑ क॒वषः॒ शुंभ॑माना॒ द्वारो॒ द्वारः॒ शुंभ॑मानाः क॒वषः॑ क॒वषः॒ शुंभ॑माना॒ द्वारः॑ । \newline
44. शुंभ॑माना॒ द्वारो॒ द्वारः॒ शुंभ॑मानाः॒ शुंभ॑माना॒ द्वारो॑ दे॒वीर् दे॒वीर् द्वारः॒ शुंभ॑मानाः॒ शुंभ॑माना॒ द्वारो॑ दे॒वीः । \newline
45. द्वारो॑ दे॒वीर् दे॒वीर् द्वारो॒ द्वारो॑ दे॒वीः सु॑प्राय॒णाः सु॑प्राय॒णा दे॒वीर् द्वारो॒ द्वारो॑ दे॒वीः सु॑प्राय॒णाः । \newline
46. दे॒वीः सु॑प्राय॒णाः सु॑प्राय॒णा दे॒वीर् दे॒वीः सु॑प्राय॒णा भ॑वन्तु भवन्तु सुप्राय॒णा दे॒वीर् दे॒वीः सु॑प्राय॒णा भ॑वन्तु । \newline
47. सु॒प्रा॒य॒णा भ॑वन्तु भवन्तु सुप्राय॒णाः सु॑प्राय॒णा भ॑वन्तु । \newline
48. सु॒प्रा॒य॒णा इति॑ सु - प्रा॒य॒णाः । \newline
49. भ॒व॒न्त्विति॑ भवन्तु । \newline
50. अ॒न्त॒रा मि॒त्रावरु॑णा मि॒त्रावरु॑णा ऽन्त॒रा ऽन्त॒रा मि॒त्रावरु॑णा॒ चर॑न्ती॒ चर॑न्ती मि॒त्रावरु॑णा ऽन्त॒रा ऽन्त॒रा मि॒त्रावरु॑णा॒ चर॑न्ती । \newline
51. मि॒त्रावरु॑णा॒ चर॑न्ती॒ चर॑न्ती मि॒त्रावरु॑णा मि॒त्रावरु॑णा॒ चर॑न्ती॒ मुख॒म् मुख॒म् चर॑न्ती मि॒त्रावरु॑णा मि॒त्रावरु॑णा॒ चर॑न्ती॒ मुख᳚म् । \newline
52. मि॒त्रावरु॒णेति॑ मि॒त्रा - वरु॑णा । \newline
53. चर॑न्ती॒ मुख॒म् मुख॒म् चर॑न्ती॒ चर॑न्ती॒ मुखं॑ ॅय॒ज्ञानां᳚ ॅय॒ज्ञाना॒म् मुख॒म् चर॑न्ती॒ चर॑न्ती॒ मुखं॑ ॅय॒ज्ञाना᳚म् । \newline
54. चर॑न्ती॒ इति॒ चर॑न्ती । \newline
55. मुखं॑ ॅय॒ज्ञानां᳚ ॅय॒ज्ञाना॒म् मुख॒म् मुखं॑ ॅय॒ज्ञाना॑ म॒भ्य॑भि य॒ज्ञाना॒म् मुख॒म् मुखं॑ ॅय॒ज्ञाना॑ म॒भि । \newline
56. य॒ज्ञाना॑ म॒भ्य॑भि य॒ज्ञानां᳚ ॅय॒ज्ञाना॑ म॒भि सं॑ॅविदा॒ने सं॑ॅविदा॒ने अ॒भि य॒ज्ञानां᳚ ॅय॒ज्ञाना॑ म॒भि सं॑ॅविदा॒ने । \newline
57. अ॒भि सं॑ॅविदा॒ने सं॑ॅविदा॒ने अ॒भ्य॑भि सं॑ॅविदा॒ने । \newline
58. सं॒ॅवि॒दा॒ने इति॑ सं - वि॒दा॒ने । \newline
59. उ॒षासा॑ वां ॅवा मु॒षासो॒षासा॑ वाꣳ सुहिर॒ण्ये सु॑हिर॒ण्ये वा॑ मु॒षासो॒षासा॑ वाꣳ सुहिर॒ण्ये । \newline
60. वाꣳ॒॒ सु॒हि॒र॒ण्ये सु॑हिर॒ण्ये वां᳚ ॅवाꣳ सुहिर॒ण्ये सु॑शि॒ल्पे सु॑शि॒ल्पे सु॑हिर॒ण्ये वां᳚ ॅवाꣳ सुहिर॒ण्ये सु॑शि॒ल्पे । \newline
\pagebreak
\markright{ TS 5.1.11.3  \hfill https://www.vedavms.in \hfill}

\section{ TS 5.1.11.3 }

\textbf{TS 5.1.11.3 } \newline
\textbf{Samhita Paata} \newline

सुहिर॒ण्ये सु॑शि॒ल्पे ऋ॒तस्य॒ योना॑वि॒ह सा॑दयामि ॥प्र॒थ॒मा वाꣳ॑ सर॒थिना॑ सु॒वर्णा॑ दे॒वौ पश्य॑न्तौ॒ भुव॑नानि॒ विश्वा᳚ । अपि॑प्रयं॒ चोद॑ना वां॒ मिमा॑ना॒ होता॑रा॒ ज्योतिः॑ प्र॒दिशा॑ दि॒शन्ता᳚ ॥आ॒दि॒त्यैर्नो॒ भार॑ती वष्टु य॒ज्ञ्ꣳ सर॑स्वती स॒ह रु॒द्रैर्न॑ आवीत् । इडोप॑हूता॒ वसु॑भिः स॒जोषा॑ य॒ज्ञ्ं नो॑ देवीर॒मृते॑षु धत्त ॥त्वष्टा॑ वी॒रं दे॒वका॑मं जजान॒ त्वष्टु॒रर्वा॑ जायत आ॒शुरश्वः॑ ( ) । \newline

\textbf{Pada Paata} \newline

सु॒हि॒र॒ण्ये इति॑ सु - हि॒र॒ण्ये । सु॒शि॒ल्पे इति॑ सु-शि॒ल्पे । ऋ॒तस्य॑ । योनौ᳚ । इ॒ह । सा॒द॒या॒मि॒ ॥ प्र॒थ॒मा । वा॒म् । स॒र॒थिनेति॑ स - र॒थिना᳚ । सु॒वर्णेति॑ सु - वर्णा᳚ । दे॒वौ । पश्य॑न्तौ । भुव॑नानि । विश्वा᳚ ॥ अपि॑प्रयम् । चोद॑ना । वा॒म् । मिमा॑ना । होता॑रा । ज्योतिः॑ । प्र॒दिशेति॑ प्र - दिशा᳚ । दि॒शन्ता᳚ ॥ आ॒दि॒त्यैः । नः॒ । भार॑ती । व॒ष्टु॒ । य॒ज्ञ्म् । सर॑स्वती । स॒ह । रु॒द्रैः । नः॒ । आ॒वी॒त् ॥ इडा᳚ । उप॑हू॒तेत्युप॑-हू॒ता॒ । वसु॑भि॒रिति॒ वसु॑ - भिः॒ । स॒जोषा॒ इति॑ स - जोषाः᳚ । य॒ज्ञ्म् । नः॒ । दे॒वीः॒ । अ॒मृते॑षु । ध॒त्त॒ ॥ त्वष्टा᳚ । वी॒रम् । दे॒वका॑म॒मिति॑ दे॒व-का॒म॒म् । ज॒जा॒न॒ । त्वष्टुः॑ । अर्वा᳚ । जा॒य॒ते॒ । आ॒शुः । अश्वः॑ ॥  \newline


\textbf{Krama Paata} \newline

सु॒हि॒र॒ण्ये सु॑शि॒ल्पे । सु॒हि॒र॒ण्ये इति॑ सु - हि॒र॒ण्ये । सु॒शि॒ल्पे ऋ॒तस्य॑ । सु॒शि॒ल्पे इति॑ सु - शि॒ल्पे । ऋ॒तस्य॒ योनौ᳚ । योना॑वि॒ह । इ॒ह सा॑दयामि । सा॒द॒या॒मीति॑ सादयामि ॥ प्र॒थ॒मा वा᳚म् । वाꣳ॒॒ स॒र॒थिना᳚ । स॒र॒थिना॑ सु॒वर्णा᳚ । स॒र॒थिनेति॑ स - र॒थिना᳚ । सु॒वर्णा॑ दे॒वौ । सु॒वर्णेति॑ सु - वर्णा᳚ । दे॒वौ पश्य॑न्तौ । पश्य॑न्तौ॒ भुव॑नानि । भुव॑नानि॒ विश्वा᳚ । विश्वेति॒ विश्वा᳚ ॥ अपि॑प्रय॒म् चोद॑ना । चोद॑ना वाम् । वा॒म् मिमा॑ना । मिमा॑ना॒ होता॑रा । होता॑रा॒ ज्योतिः॑ । ज्योतिः॑ प्र॒दिशा᳚ । प्र॒दिशा॑ दि॒शन्ता᳚ । प्र॒दिशेति॑ प्र - दिशा᳚ । दि॒शन्तेति॑ दि॒शन्ता᳚ ॥ आ॒दि॒त्यैर् नः॑ । नो॒ भार॑ती । भार॑ती वष्टु । व॒ष्टु॒ य॒ज्ञ्म् । य॒ज्ञ्ꣳ सर॑स्वती । सर॑स्वती स॒ह । स॒ह रु॒द्रैः । रु॒द्रैर् नः॑ । न॒ आ॒वी॒त्॒ । आ॒वी॒दित्या॑वीत् ॥ इडोप॑हूता । उप॑हूता॒ वसु॑भिः । उप॑हू॒तेत्युप॑ - हू॒ता॒ । वसु॑भिः स॒जोषाः᳚ । वसु॑भि॒रिति॒ वसु॑ - भिः॒ । स॒जोषा॑ य॒ज्ञ्म् । स॒जोषा॒ इति॑ स - जोषाः᳚ । य॒ज्ञ्म् नः॑ । नो॒ दे॒वीः॒ । दे॒वी॒र॒मृते॑षु । अ॒मृते॑षु धत्त । ध॒त्तेति॑ धत्त ॥ त्वष्टा॑ वी॒रम् । वी॒रम् दे॒वका॑मम् । दे॒वका॑मम् जजान । दे॒वका॑म॒मिति॑ दे॒व - का॒म॒म् । ज॒जा॒न॒ त्वष्टुः॑ । 
त्वष्टु॒रर्वा᳚ । अर्वा॑ जायते । जा॒य॒त॒ आ॒शुः । आ॒शुरश्वः॑ ( ) । 
अश्व॒ इत्यश्वः॑ । \newline

\textbf{Jatai Paata} \newline

1. सु॒हि॒र॒ण्ये सु॑शि॒ल्पे सु॑शि॒ल्पे सु॑हिर॒ण्ये सु॑हिर॒ण्ये सु॑शि॒ल्पे । \newline
2. सु॒हि॒र॒ण्ये इति॑ सु - हि॒र॒ण्ये । \newline
3. सु॒शि॒ल्पे ऋ॒तस्य॒ र्‌तस्य॑ सुशि॒ल्पे सु॑शि॒ल्पे ऋ॒तस्य॑ । \newline
4. सु॒शि॒ल्पे इति॑ सु - शि॒ल्पे । \newline
5. ऋ॒तस्य॒ योनौ॒ योना॑ वृ॒तस्य॒ र्‌तस्य॒ योनौ᳚ । \newline
6. योना॑ वि॒हेह योनौ॒ योना॑ वि॒ह । \newline
7. इ॒ह सा॑दयामि सादयामी॒हेह सा॑दयामि । \newline
8. सा॒द॒या॒मीति॑ सादयामि । \newline
9. प्र॒थ॒मा वां᳚ ॅवाम् प्रथ॒मा प्र॑थ॒मा वा᳚म् । \newline
10. वाꣳ॒॒ स॒र॒थिना॑ सर॒थिना॑ वां ॅवाꣳ सर॒थिना᳚ । \newline
11. स॒र॒थिना॑ सु॒वर्णा॑ सु॒वर्णा॑ सर॒थिना॑ सर॒थिना॑ सु॒वर्णा᳚ । \newline
12. स॒र॒थिनेति॑ स - र॒थिना᳚ । \newline
13. सु॒वर्णा॑ दे॒वौ दे॒वौ सु॒वर्णा॑ सु॒वर्णा॑ दे॒वौ । \newline
14. सु॒वर्णेति॑ सु - वर्णा᳚ । \newline
15. दे॒वौ पश्य॑न्तौ॒ पश्य॑न्तौ दे॒वौ दे॒वौ पश्य॑न्तौ । \newline
16. पश्य॑न्तौ॒ भुव॑नानि॒ भुव॑नानि॒ पश्य॑न्तौ॒ पश्य॑न्तौ॒ भुव॑नानि । \newline
17. भुव॑नानि॒ विश्वा॒ विश्वा॒ भुव॑नानि॒ भुव॑नानि॒ विश्वा᳚ । \newline
18. विश्वेति॒ विश्वा᳚ । \newline
19. अपि॑प्रय॒म् चोद॑ना॒ चोद॒ना ऽपि॑प्रय॒ मपि॑प्रय॒म् चोद॑ना । \newline
20. चोद॑ना वां ॅवा॒म् चोद॑ना॒ चोद॑ना वाम् । \newline
21. वा॒म् मिमा॑ना॒ मिमा॑ना वां ॅवा॒म् मिमा॑ना । \newline
22. मिमा॑ना॒ होता॑रा॒ होता॑रा॒ मिमा॑ना॒ मिमा॑ना॒ होता॑रा । \newline
23. होता॑रा॒ ज्योति॒र् ज्योति॒र्॒. होता॑रा॒ होता॑रा॒ ज्योतिः॑ । \newline
24. ज्योतिः॑ प्र॒दिशा᳚ प्र॒दिशा॒ ज्योति॒र् ज्योतिः॑ प्र॒दिशा᳚ । \newline
25. प्र॒दिशा॑ दि॒शन्ता॑ दि॒शन्ता᳚ प्र॒दिशा᳚ प्र॒दिशा॑ दि॒शन्ता᳚ । \newline
26. प्र॒दिशेति॑ प्र - दिशा᳚ । \newline
27. दि॒शन्तेति॑ दि॒शन्ता᳚ । \newline
28. आ॒दि॒त्यैर् नो॑ न आदि॒त्यै रा॑दि॒त्यैर् नः॑ । \newline
29. नो॒ भार॑ती॒ भार॑ती नो नो॒ भार॑ती । \newline
30. भार॑ती वष्टु वष्टु॒ भार॑ती॒ भार॑ती वष्टु । \newline
31. व॒ष्टु॒ य॒ज्ञ्ं ॅय॒ज्ञ्ं ॅव॑ष्टु वष्टु य॒ज्ञ्म् । \newline
32. य॒ज्ञ्ꣳ सर॑स्वती॒ सर॑स्वती य॒ज्ञ्ं ॅय॒ज्ञ्ꣳ सर॑स्वती । \newline
33. सर॑स्वती स॒ह स॒ह सर॑स्वती॒ सर॑स्वती स॒ह । \newline
34. स॒ह रु॒द्रै रु॒द्रैः स॒ह स॒ह रु॒द्रैः । \newline
35. रु॒द्रैर् नो॑ नो रु॒द्रै रु॒द्रैर् नः॑ । \newline
36. न॒ आ॒वी॒ दा॒वी॒न् नो॒ न॒ आ॒वी॒त् । \newline
37. आ॒वी॒दित्या॑वीत् । \newline
38. इडोप॑हू॒तो प॑हू॒ते डेडोप॑हूता । \newline
39. उप॑हूता॒ वसु॑भि॒र् वसु॑भि॒ रुप॑हू॒तो प॑हूता॒ वसु॑भिः । \newline
40. उप॑हू॒तेत्युप॑ - हू॒ता॒ । \newline
41. वसु॑भिः स॒जोषाः᳚ स॒जोषा॒ वसु॑भि॒र् वसु॑भिः स॒जोषाः᳚ । \newline
42. वसु॑भि॒रिति॒ वसु॑ - भिः॒ । \newline
43. स॒जोषा॑ य॒ज्ञ्ं ॅय॒ज्ञ्ꣳ स॒जोषाः᳚ स॒जोषा॑ य॒ज्ञ्म् । \newline
44. स॒जोषा॒ इति॑ स - जोषाः᳚ । \newline
45. य॒ज्ञ्म् नो॑ नो य॒ज्ञ्ं ॅय॒ज्ञ्म् नः॑ । \newline
46. नो॒ दे॒वी॒र् दे॒वी॒र् नो॒ नो॒ दे॒वीः॒ । \newline
47. दे॒वी॒ र॒मृते᳚ ष्व॒मृते॑षु देवीर् देवी र॒मृते॑षु । \newline
48. अ॒मृते॑षु धत्त धत्ता॒मृते᳚ ष्व॒मृते॑षु धत्त । \newline
49. ध॒त्तेति॑ धत्त । \newline
50. त्वष्टा॑ वी॒रं ॅवी॒रम् त्वष्टा॒ त्वष्टा॑ वी॒रम् । \newline
51. वी॒रम् दे॒वका॑मम् दे॒वका॑मं ॅवी॒रं ॅवी॒रम् दे॒वका॑मम् । \newline
52. दे॒वका॑मम् जजान जजान दे॒वका॑मम् दे॒वका॑मम् जजान । \newline
53. दे॒वका॑म॒मिति॑ दे॒व - का॒म॒म् । \newline
54. ज॒जा॒न॒ त्वष्टु॒ स्त्वष्टु॑र् जजान जजान॒ त्वष्टुः॑ । \newline
55. त्वष्टु॒ रर्वा ऽर्वा॒ त्वष्टु॒ स्त्वष्टु॒ रर्वा᳚ । \newline
56. अर्वा॑ जायते जाय॒ते ऽर्वा ऽर्वा॑ जायते । \newline
57. जा॒य॒त॒ आ॒शु रा॒शुर् जा॑यते जायत आ॒शुः । \newline
58. आ॒शु रश्वो ऽश्व॑ आ॒शु रा॒शु रश्वः॑ । \newline
59. अश्व॒ इत्यश्वः॑ । \newline

\textbf{Ghana Paata } \newline

1. सु॒हि॒र॒ण्ये सु॑शि॒ल्पे सु॑शि॒ल्पे सु॑हिर॒ण्ये सु॑हिर॒ण्ये सु॑शि॒ल्पे ऋ॒तस्य॒ र्‌तस्य॑ सुशि॒ल्पे सु॑हिर॒ण्ये सु॑हिर॒ण्ये सु॑शि॒ल्पे ऋ॒तस्य॑ । \newline
2. सु॒हि॒र॒ण्ये इति॑ सु - हि॒र॒ण्ये । \newline
3. सु॒शि॒ल्पे ऋ॒तस्य॒ र्‌तस्य॑ सुशि॒ल्पे सु॑शि॒ल्पे ऋ॒तस्य॒ योनौ॒ योना॑ वृ॒तस्य॑ सुशि॒ल्पे सु॑शि॒ल्पे ऋ॒तस्य॒ योनौ᳚ । \newline
4. सु॒शि॒ल्पे इति॑ सु - शि॒ल्पे । \newline
5. ऋ॒तस्य॒ योनौ॒ योना॑ वृ॒तस्य॒ र्‌तस्य॒ योना॑ वि॒हेह योना॑ वृ॒तस्य॒ र्‌तस्य॒ योना॑ वि॒ह । \newline
6. योना॑ वि॒हेह योनौ॒ योना॑ वि॒ह सा॑दयामि सादयामी॒ह योनौ॒ योना॑ वि॒ह सा॑दयामि । \newline
7. इ॒ह सा॑दयामि सादयामी॒हेह सा॑दयामि । \newline
8. सा॒द॒या॒मीति॑ सादयामि । \newline
9. प्र॒थ॒मा वां᳚ ॅवाम् प्रथ॒मा प्र॑थ॒मा वाꣳ॑ सर॒थिना॑ सर॒थिना॑ वाम् प्रथ॒मा प्र॑थ॒मा वाꣳ॑ सर॒थिना᳚ । \newline
10. वाꣳ॒॒ स॒र॒थिना॑ सर॒थिना॑ वां ॅवाꣳ सर॒थिना॑ सु॒वर्णा॑ सु॒वर्णा॑ सर॒थिना॑ वां ॅवाꣳ सर॒थिना॑ सु॒वर्णा᳚ । \newline
11. स॒र॒थिना॑ सु॒वर्णा॑ सु॒वर्णा॑ सर॒थिना॑ सर॒थिना॑ सु॒वर्णा॑ दे॒वौ दे॒वौ सु॒वर्णा॑ सर॒थिना॑ सर॒थिना॑ सु॒वर्णा॑ दे॒वौ । \newline
12. स॒र॒थिनेति॑ स - र॒थिना᳚ । \newline
13. सु॒वर्णा॑ दे॒वौ दे॒वौ सु॒वर्णा॑ सु॒वर्णा॑ दे॒वौ पश्य॑न्तौ॒ पश्य॑न्तौ दे॒वौ सु॒वर्णा॑ सु॒वर्णा॑ दे॒वौ पश्य॑न्तौ । \newline
14. सु॒वर्णेति॑ सु - वर्णा᳚ । \newline
15. दे॒वौ पश्य॑न्तौ॒ पश्य॑न्तौ दे॒वौ दे॒वौ पश्य॑न्तौ॒ भुव॑नानि॒ भुव॑नानि॒ पश्य॑न्तौ दे॒वौ दे॒वौ पश्य॑न्तौ॒ भुव॑नानि । \newline
16. पश्य॑न्तौ॒ भुव॑नानि॒ भुव॑नानि॒ पश्य॑न्तौ॒ पश्य॑न्तौ॒ भुव॑नानि॒ विश्वा॒ विश्वा॒ भुव॑नानि॒ पश्य॑न्तौ॒ पश्य॑न्तौ॒ भुव॑नानि॒ विश्वा᳚ । \newline
17. भुव॑नानि॒ विश्वा॒ विश्वा॒ भुव॑नानि॒ भुव॑नानि॒ विश्वा᳚ । \newline
18. विश्वेति॒ विश्वा᳚ । \newline
19. अपि॑प्रय॒म् चोद॑ना॒ चोद॒ना ऽपि॑प्रय॒ मपि॑प्रय॒म् चोद॑ना वां ॅवा॒म् चोद॒ना ऽपि॑प्रय॒ मपि॑प्रय॒म् चोद॑ना वाम् । \newline
20. चोद॑ना वां ॅवा॒म् चोद॑ना॒ चोद॑ना वा॒म् मिमा॑ना॒ मिमा॑ना वा॒म् चोद॑ना॒ चोद॑ना वा॒म् मिमा॑ना । \newline
21. वा॒म् मिमा॑ना॒ मिमा॑ना वां ॅवा॒म् मिमा॑ना॒ होता॑रा॒ होता॑रा॒ मिमा॑ना वां ॅवा॒म् मिमा॑ना॒ होता॑रा । \newline
22. मिमा॑ना॒ होता॑रा॒ होता॑रा॒ मिमा॑ना॒ मिमा॑ना॒ होता॑रा॒ ज्योति॒र् ज्योति॒र्॒. होता॑रा॒ मिमा॑ना॒ मिमा॑ना॒ होता॑रा॒ ज्योतिः॑ । \newline
23. होता॑रा॒ ज्योति॒र् ज्योति॒र्॒. होता॑रा॒ होता॑रा॒ ज्योतिः॑ प्र॒दिशा᳚ प्र॒दिशा॒ ज्योति॒र्॒. होता॑रा॒ होता॑रा॒ ज्योतिः॑ प्र॒दिशा᳚ । \newline
24. ज्योतिः॑ प्र॒दिशा᳚ प्र॒दिशा॒ ज्योति॒र् ज्योतिः॑ प्र॒दिशा॑ दि॒शन्ता॑ दि॒शन्ता᳚ प्र॒दिशा॒ ज्योति॒र् ज्योतिः॑ प्र॒दिशा॑ दि॒शन्ता᳚ । \newline
25. प्र॒दिशा॑ दि॒शन्ता॑ दि॒शन्ता᳚ प्र॒दिशा᳚ प्र॒दिशा॑ दि॒शन्ता᳚ । \newline
26. प्र॒दिशेति॑ प्र - दिशा᳚ । \newline
27. दि॒शन्तेति॑ दि॒शन्ता᳚ । \newline
28. आ॒दि॒त्यैर् नो॑ न आदि॒त्यै रा॑दि॒त्यैर् नो॒ भार॑ती॒ भार॑ती न आदि॒त्यै रा॑दि॒त्यैर् नो॒ भार॑ती । \newline
29. नो॒ भार॑ती॒ भार॑ती नो नो॒ भार॑ती वष्टु वष्टु॒ भार॑ती नो नो॒ भार॑ती वष्टु । \newline
30. भार॑ती वष्टु वष्टु॒ भार॑ती॒ भार॑ती वष्टु य॒ज्ञ्ं ॅय॒ज्ञ्ं ॅव॑ष्टु॒ भार॑ती॒ भार॑ती वष्टु य॒ज्ञ्म् । \newline
31. व॒ष्टु॒ य॒ज्ञ्ं ॅय॒ज्ञ्ं ॅव॑ष्टु वष्टु य॒ज्ञ्ꣳ सर॑स्वती॒ सर॑स्वती य॒ज्ञ्ं ॅव॑ष्टु वष्टु य॒ज्ञ्ꣳ सर॑स्वती । \newline
32. य॒ज्ञ्ꣳ सर॑स्वती॒ सर॑स्वती य॒ज्ञ्ं ॅय॒ज्ञ्ꣳ सर॑स्वती स॒ह स॒ह सर॑स्वती य॒ज्ञ्ं ॅय॒ज्ञ्ꣳ सर॑स्वती स॒ह । \newline
33. सर॑स्वती स॒ह स॒ह सर॑स्वती॒ सर॑स्वती स॒ह रु॒द्रै रु॒द्रैः स॒ह सर॑स्वती॒ सर॑स्वती स॒ह रु॒द्रैः । \newline
34. स॒ह रु॒द्रै रु॒द्रैः स॒ह स॒ह रु॒द्रैर् नो॑ नो रु॒द्रैः स॒ह स॒ह रु॒द्रैर् नः॑ । \newline
35. रु॒द्रैर् नो॑ नो रु॒द्रै रु॒द्रैर् न॑ आवी दावीन् नो रु॒द्रै रु॒द्रैर् न॑ आवीत् । \newline
36. न॒ आ॒वी॒ दा॒वी॒न् नो॒ न॒ आ॒वी॒त् । \newline
37. आ॒वी॒दित्या॑वीत् । \newline
38. इडोप॑हू॒तो प॑हू॒ते डे डोप॑हूता॒ वसु॑भि॒र् वसु॑भि॒ रुप॑हू॒ते डे डोप॑हूता॒ वसु॑भिः । \newline
39. उप॑हूता॒ वसु॑भि॒र् वसु॑भि॒ रुप॑हू॒तो प॑हूता॒ वसु॑भिः स॒जोषाः᳚ स॒जोषा॒ वसु॑भि॒ रुप॑हू॒तो प॑हूता॒ वसु॑भिः स॒जोषाः᳚ । \newline
40. उप॑हू॒तेत्युप॑ - हू॒ता॒ । \newline
41. वसु॑भिः स॒जोषाः᳚ स॒जोषा॒ वसु॑भि॒र् वसु॑भिः स॒जोषा॑ य॒ज्ञ्ं ॅय॒ज्ञ्ꣳ स॒जोषा॒ वसु॑भि॒र् वसु॑भिः स॒जोषा॑ य॒ज्ञ्म् । \newline
42. वसु॑भि॒रिति॒ वसु॑ - भिः॒ । \newline
43. स॒जोषा॑ य॒ज्ञ्ं ॅय॒ज्ञ्ꣳ स॒जोषाः᳚ स॒जोषा॑ य॒ज्ञ्म् नो॑ नो य॒ज्ञ्ꣳ स॒जोषाः᳚ स॒जोषा॑ य॒ज्ञ्म् नः॑ । \newline
44. स॒जोषा॒ इति॑ स - जोषाः᳚ । \newline
45. य॒ज्ञ्म् नो॑ नो य॒ज्ञ्ं ॅय॒ज्ञ्म् नो॑ देवीर् देवीर् नो य॒ज्ञ्ं ॅय॒ज्ञ्म् नो॑ देवीः । \newline
46. नो॒ दे॒वी॒र् दे॒वी॒र् नो॒ नो॒ दे॒वी॒ र॒मृते᳚ ष्व॒मृते॑षु देवीर् नो नो देवी र॒मृते॑षु । \newline
47. दे॒वी॒ र॒मृते᳚ ष्व॒मृते॑षु देवीर् देवी र॒मृते॑षु धत्त धत्ता॒मृते॑षु देवीर् देवी र॒मृते॑षु धत्त । \newline
48. अ॒मृते॑षु धत्त धत्ता॒मृते᳚ ष्व॒मृते॑षु धत्त । \newline
49. ध॒त्तेति॑ धत्त । \newline
50. त्वष्टा॑ वी॒रं ॅवी॒रम् त्वष्टा॒ त्वष्टा॑ वी॒रम् दे॒वका॑मम् दे॒वका॑मं ॅवी॒रम् त्वष्टा॒ त्वष्टा॑ वी॒रम् दे॒वका॑मम् । \newline
51. वी॒रम् दे॒वका॑मम् दे॒वका॑मं ॅवी॒रं ॅवी॒रम् दे॒वका॑मम् जजान जजान दे॒वका॑मं ॅवी॒रं ॅवी॒रम् दे॒वका॑मम् जजान । \newline
52. दे॒वका॑मम् जजान जजान दे॒वका॑मम् दे॒वका॑मम् जजान॒ त्वष्टु॒ स्त्वष्टु॑र् जजान दे॒वका॑मम् दे॒वका॑मम् जजान॒ त्वष्टुः॑ । \newline
53. दे॒वका॑म॒मिति॑ दे॒व - का॒म॒म् । \newline
54. ज॒जा॒न॒ त्वष्टु॒ स्त्वष्टु॑र् जजान जजान॒ त्वष्टु॒रर्वा ऽर्वा॒ त्वष्टु॑र् जजान जजान॒ त्वष्टु॒रर्वा᳚ । \newline
55. त्वष्टु॒रर्वा ऽर्वा॒ त्वष्टु॒ स्त्वष्टु॒रर्वा॑ जायते जाय॒ते ऽर्वा॒ त्वष्टु॒ स्त्वष्टु॒रर्वा॑ जायते । \newline
56. अर्वा॑ जायते जाय॒ते ऽर्वा ऽर्वा॑ जायत आ॒शु रा॒शुर् जा॑य॒ते ऽर्वा ऽर्वा॑ जायत आ॒शुः । \newline
57. जा॒य॒त॒ आ॒शु रा॒शुर् जा॑यते जायत आ॒शु रश्वो ऽश्व॑ आ॒शुर् जा॑यते जायत आ॒शु रश्वः॑ । \newline
58. आ॒शु रश्वो ऽश्व॑ आ॒शु रा॒शु रश्वः॑ । \newline
59. अश्व॒ इत्यश्वः॑ । \newline
\pagebreak
\markright{ TS 5.1.11.4  \hfill https://www.vedavms.in \hfill}

\section{ TS 5.1.11.4 }

\textbf{TS 5.1.11.4 } \newline
\textbf{Samhita Paata} \newline

त्वष्टे॒दं ॅविश्वं॒ भुव॑नं जजान ब॒होः क॒र्तार॑मि॒ह य॑क्षि होतः ॥अश्वो॑ घृ॒तेन॒ त्मन्या॒ सम॑क्त॒ उप॑ दे॒वाꣳ ऋ॑तु॒शः पाथ॑ एतु । वन॒स्पति॑-र्देवलो॒कं प्र॑जा॒नन्न॒ग्निना॑ ह॒व्या स्व॑दि॒तानि॑ वक्षत् ॥प्र॒जाप॑ते॒स्तप॑सा वावृधा॒नः स॒द्यो जा॒तो द॑धिषे य॒ज्ञ्म॑ग्ने । स्वाहा॑कृतेन ह॒विषा॑ पुरोगा या॒हि सा॒द्ध्या ह॒विर॑दन्तु दे॒वाः ॥ \newline

\textbf{Pada Paata} \newline

त्वष्टा᳚ । इ॒दम् । विश्व᳚म् । भुव॑नम् । ज॒जा॒न॒ । ब॒होः । क॒र्तार᳚म् । इ॒ह । य॒क्षि॒ । हो॒तः॒ ॥ अश्वः॑ । घृ॒तेन॑ । त्मन्या᳚ । सम॑क्त॒ इति॒ सं - अ॒क्तः॒ । उपेति॑ । दे॒वान् । ऋ॒तु॒श इत्यृ॑तु - शः । पाथः॑ । ए॒तु॒ ॥ वन॒स्पतिः॑ । दे॒व॒लो॒कमिति॑ देव - लो॒कम् । प्र॒जा॒नन्निति॑ प्र-जा॒नन्न् । अ॒ग्निना᳚ । ह॒व्या । स्व॒दि॒तानि॑ । व॒क्ष॒त् ॥ प्र॒जाप॑ते॒रिति॑ प्र॒जा - प॒तेः॒ । तप॑सा । वा॒वृ॒धा॒नः । स॒द्यः । जा॒तः । द॒धि॒षे॒ । य॒ज्ञ्म् । अ॒ग्ने॒ ॥ स्वाहा॑कृते॒नेति॒ स्वाहा᳚ - कृ॒ते॒न॒ । ह॒विषा᳚ । पु॒रो॒गा॒ इति॑ पुरः - गाः॒ । या॒हि । सा॒द्ध्या । ह॒विः । अ॒द॒न्तु॒ । दे॒वाः ॥  \newline


\textbf{Krama Paata} \newline

त्वष्टे॒दम् । इ॒दम् ॅविश्व᳚म् । विश्व॒म् भुव॑नम् । भुव॑नम् जजान । ज॒जा॒न॒ ब॒होः । ब॒होः क॒र्तार᳚म् । क॒र्तार॑मि॒ह । इ॒ह य॑क्षि । य॒क्षि॒ हो॒तः॒ । हो॒त॒रिति॑ होतः ॥ अश्वो॑ घृ॒तेन॑ । घृ॒तेन॒ त्मन्या᳚ । त्मन्या॒ सम॑क्तः । सम॑क्त॒ उप॑ । सम॑क्त॒ इति॒ सम् - अ॒क्तः॒ । उप॑ दे॒वान् । दे॒वाꣳ ऋ॑तु॒शः । ऋ॒तु॒शः पाथः॑ । ऋ॒तु॒श इत्यृ॑तु - शः । पाथ॑ एतु । ए॒त्वित्ये॑तु ॥ वन॒स्पति॑र् देवलो॒कम् । दे॒व॒लो॒कम् प्र॑जा॒नन्न् । दे॒व॒लो॒कमिति॑ देव - लो॒कम् । प्र॒जा॒नन्न॒ग्निना᳚ । प्र॒जा॒नन्निति॑ प्र - जा॒नन्न् । अ॒ग्निना॑ ह॒व्या । ह॒व्या स्व॑दि॒तानि॑ । स्व॒दि॒तानि॑ वक्षत् । व॒क्ष॒दिति॑ वक्षत् ॥ प्र॒जाप॑ते॒स्तप॑सा । प्र॒जाप॑ते॒रिति॑ प्र॒जा - प॒तेः॒ । तप॑सा वावृधा॒नः । वा॒वृ॒धा॒नः स॒द्यः । स॒द्यो जा॒तः । जा॒तो द॑धिषे । द॒धि॒षे॒ य॒ज्ञ्म् । य॒ज्ञ्म॑ग्ने । अ॒ग्न॒ इत्य॑ग्ने ॥ स्वाहा॑कृतेन ह॒विषा᳚ । स्वाहा॑कृते॒नेति॒ स्वाहा᳚ - कृ॒ते॒न॒ । ह॒विषा॑ पुरोगाः । पु॒रो॒गा॒ या॒हि । पु॒रो॒गा॒ इति॑ पुरः - गाः॒ । या॒हि सा॒द्ध्या । सा॒द्ध्या ह॒विः । ह॒विर॑दन्तु । अ॒द॒न्तु॒ दे॒वाः । दे॒वा इति॑ दे॒वाः । \newline

\textbf{Jatai Paata} \newline

1. त्वष्टे॒द मि॒दम् त्वष्टा॒ त्वष्टे॒दम् । \newline
2. इ॒दं ॅविश्वं॒ ॅविश्व॑ मि॒द मि॒दं ॅविश्व᳚म् । \newline
3. विश्व॒म् भुव॑न॒म् भुव॑नं॒ ॅविश्वं॒ ॅविश्व॒म् भुव॑नम् । \newline
4. भुव॑नम् जजान जजान॒ भुव॑न॒म् भुव॑नम् जजान । \newline
5. ज॒जा॒न॒ ब॒होर् ब॒होर् ज॑जान जजान ब॒होः । \newline
6. ब॒होः क॒र्तार॑म् क॒र्तार॑म् ब॒होर् ब॒होः क॒र्तार᳚म् । \newline
7. क॒र्तार॑ मि॒हेह क॒र्तार॑म् क॒र्तार॑ मि॒ह । \newline
8. इ॒ह य॑क्षि यक्षी॒हेह य॑क्षि । \newline
9. य॒क्षि॒ हो॒त॒र्॒. हो॒त॒र् य॒क्षि॒ य॒क्षि॒ हो॒तः॒ । \newline
10. हो॒त॒रिति॑ होतः । \newline
11. अश्वो॑ घृ॒तेन॑ घृ॒तेनाश्वो ऽश्वो॑ घृ॒तेन॑ । \newline
12. घृ॒तेन॒ त्मन्या॒ त्मन्या॑ घृ॒तेन॑ घृ॒तेन॒ त्मन्या᳚ । \newline
13. त्मन्या॒ सम॑क्तः॒ सम॑क्त॒ स्‌त्मन्या॒ त्मन्या॒ सम॑क्तः । \newline
14. सम॑क्त॒ उपोप॒ सम॑क्तः॒ सम॑क्त॒ उप॑ । \newline
15. सम॑क्त॒ इति॒ सं - अ॒क्तः॒ । \newline
16. उप॑ दे॒वान् दे॒वाꣳ उपोप॑ दे॒वान् । \newline
17. दे॒वाꣳ ऋ॑तु॒श ऋ॑तु॒शो दे॒वान् दे॒वाꣳ ऋ॑तु॒शः । \newline
18. ऋ॒तु॒शः पाथः॒ पाथ॑ ऋतु॒श ऋ॑तु॒शः पाथः॑ । \newline
19. ऋ॒तु॒श इत्यृ॑तु - शः । \newline
20. पाथ॑ एत्वेतु॒ पाथः॒ पाथ॑ एतु । \newline
21. ए॒त्वित्ये॑तु । \newline
22. वन॒स्पति॑र् देवलो॒कम् दे॑वलो॒कं ॅवन॒स्पति॒र् वन॒स्पति॑र् देवलो॒कम् । \newline
23. दे॒व॒लो॒कम् प्र॑जा॒नन् प्र॑जा॒नन् दे॑वलो॒कम् दे॑वलो॒कम् प्र॑जा॒नन्न् । \newline
24. दे॒व॒लो॒कमिति॑ देव - लो॒कम् । \newline
25. प्र॒जा॒नन् न॒ग्निना॒ ऽग्निना᳚ प्रजा॒नन् प्र॑जा॒नन् न॒ग्निना᳚ । \newline
26. प्र॒जा॒नन्निति॑ प्र - जा॒नन्न् । \newline
27. अ॒ग्निना॑ ह॒व्या ह॒व्या ऽग्निना॒ ऽग्निना॑ ह॒व्या । \newline
28. ह॒व्या स्व॑दि॒तानि॑ स्वदि॒तानि॑ ह॒व्या ह॒व्या स्व॑दि॒तानि॑ । \newline
29. स्व॒दि॒तानि॑ वक्षद् वक्षथ् स्वदि॒तानि॑ स्वदि॒तानि॑ वक्षत् । \newline
30. व॒क्ष॒दिति॑ वक्षत् । \newline
31. प्र॒जाप॑ते॒ स्तप॑सा॒ तप॑सा प्र॒जाप॑तेः प्र॒जाप॑ते॒ स्तप॑सा । \newline
32. प्र॒जाप॑ते॒रिति॑ प्र॒जा - प॒तेः॒ । \newline
33. तप॑सा वावृधा॒नो वा॑वृधा॒न स्तप॑सा॒ तप॑सा वावृधा॒नः । \newline
34. वा॒वृ॒धा॒नः स॒द्यः स॒द्यो वा॑वृधा॒नो वा॑वृधा॒नः स॒द्यः । \newline
35. स॒द्यो जा॒तो जा॒तः स॒द्यः स॒द्यो जा॒तः । \newline
36. जा॒तो द॑धिषे दधिषे जा॒तो जा॒तो द॑धिषे । \newline
37. द॒धि॒षे॒ य॒ज्ञ्ं ॅय॒ज्ञ्म् द॑धिषे दधिषे य॒ज्ञ्म् । \newline
38. य॒ज्ञ् म॑ग्ने ऽग्ने य॒ज्ञ्ं ॅय॒ज्ञ् म॑ग्ने । \newline
39. अ॒ग्न॒ इत्य॑ग्ने । \newline
40. स्वाहा॑कृतेन ह॒विषा॑ ह॒विषा॒ स्वाहा॑कृतेन॒ स्वाहा॑कृतेन ह॒विषा᳚ । \newline
41. स्वाहा॑कृते॒नेति॒ स्वाहा᳚ - कृ॒ते॒न॒ । \newline
42. ह॒विषा॑ पुरोगाः पुरोगा ह॒विषा॑ ह॒विषा॑ पुरोगाः । \newline
43. पु॒रो॒गा॒ या॒हि या॒हि पु॑रोगाः पुरोगा या॒हि । \newline
44. पु॒रो॒गा॒ इति॑ पुरः - गाः॒ । \newline
45. या॒हि सा॒द्ध्या सा॒द्ध्या या॒हि या॒हि सा॒द्ध्या । \newline
46. सा॒द्ध्या ह॒विर्. ह॒विः सा॒द्ध्या सा॒द्ध्या ह॒विः । \newline
47. ह॒वि र॑द न्त्वदन्तु ह॒विर्. ह॒वि र॑दन्तु । \newline
48. अ॒द॒न्तु॒ दे॒वा दे॒वा अ॑द न्त्वदन्तु दे॒वाः । \newline
49. दे॒वा इति॑ दे॒वाः । \newline

\textbf{Ghana Paata } \newline

1. त्वष्टे॒द मि॒दम् त्वष्टा॒ त्वष्टे॒दं ॅविश्वं॒ ॅविश्व॑ मि॒दम् त्वष्टा॒ त्वष्टे॒दं ॅविश्व᳚म् । \newline
2. इ॒दं ॅविश्वं॒ ॅविश्व॑ मि॒द मि॒दं ॅविश्व॒म् भुव॑न॒म् भुव॑नं॒ ॅविश्व॑ मि॒द मि॒दं ॅविश्व॒म् भुव॑नम् । \newline
3. विश्व॒म् भुव॑न॒म् भुव॑नं॒ ॅविश्वं॒ ॅविश्व॒म् भुव॑नम् जजान जजान॒ भुव॑नं॒ ॅविश्वं॒ ॅविश्व॒म् भुव॑नम् जजान । \newline
4. भुव॑नम् जजान जजान॒ भुव॑न॒म् भुव॑नम् जजान ब॒होर् ब॒होर् ज॑जान॒ भुव॑न॒म् भुव॑नम् जजान ब॒होः । \newline
5. ज॒जा॒न॒ ब॒होर् ब॒होर् ज॑जान जजान ब॒होः क॒र्तार॑म् क॒र्तार॑म् ब॒होर् ज॑जान जजान ब॒होः क॒र्तार᳚म् । \newline
6. ब॒होः क॒र्तार॑म् क॒र्तार॑म् ब॒होर् ब॒होः क॒र्तार॑ मि॒हेह क॒र्तार॑म् ब॒होर् ब॒होः क॒र्तार॑ मि॒ह । \newline
7. क॒र्तार॑ मि॒हेह क॒र्तार॑म् क॒र्तार॑ मि॒ह य॑क्षि यक्षी॒ह क॒र्तार॑म् क॒र्तार॑ मि॒ह य॑क्षि । \newline
8. इ॒ह य॑क्षि यक्षी॒हेह य॑क्षि होतर्. होतर् यक्षी॒हेह य॑क्षि होतः । \newline
9. य॒क्षि॒ हो॒त॒र्॒. हो॒त॒र् य॒क्षि॒ य॒क्षि॒ हो॒तः॒ । \newline
10. हो॒त॒रिति॑ होतः । \newline
11. अश्वो॑ घृ॒तेन॑ घृ॒तेनाश्वो ऽश्वो॑ घृ॒तेन॒ त्मन्या॒ त्मन्या॑ घृ॒तेनाश्वो ऽश्वो॑ घृ॒तेन॒ त्मन्या᳚ । \newline
12. घृ॒तेन॒ त्मन्या॒ त्मन्या॑ घृ॒तेन॑ घृ॒तेन॒ त्मन्या॒ सम॑क्तः॒ सम॑क्त॒ स्त्मन्या॑ घृ॒तेन॑ घृ॒तेन॒ त्मन्या॒ सम॑क्तः । \newline
13. त्मन्या॒ सम॑क्तः॒ सम॑क्त॒ स्त्मन्या॒ त्मन्या॒ सम॑क्त॒ उपोप॒ सम॑क्त॒ स्त्मन्या॒ त्मन्या॒ सम॑क्त॒ उप॑ । \newline
14. सम॑क्त॒ उपोप॒ सम॑क्तः॒ सम॑क्त॒ उप॑ दे॒वान् दे॒वाꣳ उप॒ सम॑क्तः॒ सम॑क्त॒ उप॑ दे॒वान् । \newline
15. सम॑क्त॒ इति॒ सं - अ॒क्तः॒ । \newline
16. उप॑ दे॒वान् दे॒वाꣳ उपोप॑ दे॒वाꣳ ऋ॑तु॒श ऋ॑तु॒शो दे॒वाꣳ उपोप॑ दे॒वाꣳ ऋ॑तु॒शः । \newline
17. दे॒वाꣳ ऋ॑तु॒श ऋ॑तु॒शो दे॒वान् दे॒वाꣳ ऋ॑तु॒शः पाथः॒ पाथ॑ ऋतु॒शो दे॒वान् दे॒वाꣳ ऋ॑तु॒शः पाथः॑ । \newline
18. ऋ॒तु॒शः पाथः॒ पाथ॑ ऋतु॒श ऋ॑तु॒शः पाथ॑ एत्वेतु॒ पाथ॑ ऋतु॒श ऋ॑तु॒शः पाथ॑ एतु । \newline
19. ऋ॒तु॒श इत्यृ॑तु - शः । \newline
20. पाथ॑ एत्वेतु॒ पाथः॒ पाथ॑ एतु । \newline
21. ए॒त्वित्ये॑तु । \newline
22. वन॒स्पति॑र् देवलो॒कम् दे॑वलो॒कं ॅवन॒स्पति॒र् वन॒स्पति॑र् देवलो॒कम् प्र॑जा॒नन् प्र॑जा॒नन् दे॑वलो॒कं ॅवन॒स्पति॒र् वन॒स्पति॑र् देवलो॒कम् प्र॑जा॒नन्न् । \newline
23. दे॒व॒लो॒कम् प्र॑जा॒नन् प्र॑जा॒नन् दे॑वलो॒कम् दे॑वलो॒कम् प्र॑जा॒नन् न॒ग्निना॒ ऽग्निना᳚ प्रजा॒नन् दे॑वलो॒कम् दे॑वलो॒कम् प्र॑जा॒नन् न॒ग्निना᳚ । \newline
24. दे॒व॒लो॒कमिति॑ देव - लो॒कम् । \newline
25. प्र॒जा॒नन् न॒ग्निना॒ ऽग्निना᳚ प्रजा॒नन् प्र॑जा॒नन् न॒ग्निना॑ ह॒व्या ह॒व्या ऽग्निना᳚ प्रजा॒नन् प्र॑जा॒नन् न॒ग्निना॑ ह॒व्या । \newline
26. प्र॒जा॒नन्निति॑ प्र - जा॒नन्न् । \newline
27. अ॒ग्निना॑ ह॒व्या ह॒व्या ऽग्निना॒ ऽग्निना॑ ह॒व्या स्व॑दि॒तानि॑ स्वदि॒तानि॑ ह॒व्या ऽग्निना॒ ऽग्निना॑ ह॒व्या स्व॑दि॒तानि॑ । \newline
28. ह॒व्या स्व॑दि॒तानि॑ स्वदि॒तानि॑ ह॒व्या ह॒व्या स्व॑दि॒तानि॑ वक्षद् वक्षथ् स्वदि॒तानि॑ ह॒व्या ह॒व्या स्व॑दि॒तानि॑ वक्षत् । \newline
29. स्व॒दि॒तानि॑ वक्षद् वक्षथ् स्वदि॒तानि॑ स्वदि॒तानि॑ वक्षत् । \newline
30. व॒क्ष॒दिति॑ वक्षत् । \newline
31. प्र॒जाप॑ते॒ स्तप॑सा॒ तप॑सा प्र॒जाप॑तेः प्र॒जाप॑ते॒ स्तप॑सा वावृधा॒नो वा॑वृधा॒न स्तप॑सा प्र॒जाप॑तेः प्र॒जाप॑ते॒ स्तप॑सा वावृधा॒नः । \newline
32. प्र॒जाप॑ते॒रिति॑ प्र॒जा - प॒तेः॒ । \newline
33. तप॑सा वावृधा॒नो वा॑वृधा॒न स्तप॑सा॒ तप॑सा वावृधा॒नः स॒द्यः स॒द्यो वा॑वृधा॒न स्तप॑सा॒ तप॑सा वावृधा॒नः स॒द्यः । \newline
34. वा॒वृ॒धा॒नः स॒द्यः स॒द्यो वा॑वृधा॒नो वा॑वृधा॒नः स॒द्यो जा॒तो जा॒तः स॒द्यो वा॑वृधा॒नो वा॑वृधा॒नः स॒द्यो जा॒तः । \newline
35. स॒द्यो जा॒तो जा॒तः स॒द्यः स॒द्यो जा॒तो द॑धिषे दधिषे जा॒तः स॒द्यः स॒द्यो जा॒तो द॑धिषे । \newline
36. जा॒तो द॑धिषे दधिषे जा॒तो जा॒तो द॑धिषे य॒ज्ञ्ं ॅय॒ज्ञ्म् द॑धिषे जा॒तो जा॒तो द॑धिषे य॒ज्ञ्म् । \newline
37. द॒धि॒षे॒ य॒ज्ञ्ं ॅय॒ज्ञ्म् द॑धिषे दधिषे य॒ज्ञ् म॑ग्ने ऽग्ने य॒ज्ञ्म् द॑धिषे दधिषे य॒ज्ञ् म॑ग्ने । \newline
38. य॒ज्ञ् म॑ग्ने ऽग्ने य॒ज्ञ्ं ॅय॒ज्ञ् म॑ग्ने । \newline
39. अ॒ग्न॒ इत्य॑ग्ने । \newline
40. स्वाहा॑कृतेन ह॒विषा॑ ह॒विषा॒ स्वाहा॑कृतेन॒ स्वाहा॑कृतेन ह॒विषा॑ पुरोगाः पुरोगा ह॒विषा॒ स्वाहा॑कृतेन॒ स्वाहा॑कृतेन ह॒विषा॑ पुरोगाः । \newline
41. स्वाहा॑कृते॒नेति॒ स्वाहा᳚ - कृ॒ते॒न॒ । \newline
42. ह॒विषा॑ पुरोगाः पुरोगा ह॒विषा॑ ह॒विषा॑ पुरोगा या॒हि या॒हि पु॑रोगा ह॒विषा॑ ह॒विषा॑ पुरोगा या॒हि । \newline
43. पु॒रो॒गा॒ या॒हि या॒हि पु॑रोगाः पुरोगा या॒हि सा॒द्ध्या सा॒द्ध्या या॒हि पु॑रोगाः पुरोगा या॒हि सा॒द्ध्या । \newline
44. पु॒रो॒गा॒ इति॑ पुरः - गाः॒ । \newline
45. या॒हि सा॒द्ध्या सा॒द्ध्या या॒हि या॒हि सा॒द्ध्या ह॒विर्. ह॒विः सा॒द्ध्या या॒हि या॒हि सा॒द्ध्या ह॒विः । \newline
46. सा॒द्ध्या ह॒विर्. ह॒विः सा॒द्ध्या सा॒द्ध्या ह॒वि र॑दन् त्वदन्तु ह॒विः सा॒द्ध्या सा॒द्ध्या ह॒वि र॑दन्तु । \newline
47. ह॒वि र॑दन् त्वदन्तु ह॒विर्. ह॒वि र॑दन्तु दे॒वा दे॒वा अ॑दन्तु ह॒विर्. ह॒वि र॑दन्तु दे॒वाः । \newline
48. अ॒द॒न्तु॒ दे॒वा दे॒वा अ॑दन् त्वदन्तु दे॒वाः । \newline
49. दे॒वा इति॑ दे॒वाः । \newline
\pagebreak


\end{document}