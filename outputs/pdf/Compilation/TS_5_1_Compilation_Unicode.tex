\documentclass[17pt]{extarticle}
\usepackage{babel}
\usepackage{fontspec}
\usepackage{polyglossia}
\usepackage{extsizes}



\setmainlanguage{sanskrit}
\setotherlanguages{english} %% or other languages
\setlength{\parindent}{0pt}
\pagestyle{myheadings}
\newfontfamily\devanagarifont[Script=Devanagari]{AdishilaVedic}


\newcommand{\VAR}[1]{}
\newcommand{\BLOCK}[1]{}




\begin{document}
\begin{titlepage}
    \begin{center}
 
\begin{sanskrit}
    { \Huge
    कृष्ण यजुर्वेदीय तैत्तिरीय संहिता,पद,जटा,घन पाठः 
    }
    \\
    \vspace{2.5cm}
    \mbox{ \Huge
    5.1      पञ्चमकाण्डे प्रथमः प्रश्नः - उख्याग्निकथनं   }
\end{sanskrit}
\end{center}

\end{titlepage}
\tableofcontents
\pagebreak

\markright{ TS 5.1.1.1  \hfill https://www.vedavms.in \hfill}
\addcontentsline{toc}{section}{ TS 5.1.1.1 }
\section*{ TS 5.1.1.1 }

\textbf{TS 5.1.1.1 } \newline
\textbf{Samhita Paata} \newline

सा॒वि॒त्राणि॑ जुहोति॒ प्रसू᳚त्यै चतुर्गृही॒तेन॑ जुहोति॒ चतु॑ष्पादः प॒शवः॑ प॒शूने॒वाऽव॑ रुन्धे॒ चत॑स्रो॒ दिशो॑ दि॒क्ष्वे॑व प्रति॑ तिष्ठति॒ छन्दाꣳ॑सि दे॒वेभ्यो ऽपा᳚ऽक्राम॒न् न वो॑ भा॒गानि॑ ह॒व्यं ॅव॑क्ष्याम॒ इति॒ तेभ्य॑ ए॒तच्च॑तु-र्गृही॒तम॑-धारयन् पुरोऽनु वा॒क्या॑यै या॒ज्या॑यै दे॒वता॑यै वषट्का॒राय॒ यच्च॑तुर्गृही॒तं जु॒होति॒ छन्दाꣳ॑स्ये॒व तत् प्री॑णाति॒ तान्य॑स्य प्री॒तानि॑ दे॒वेभ्यो॑ ह॒व्यं ॅव॑हन्ति॒ यं का॒मये॑त॒ - [  ] \newline

\textbf{Pada Paata} \newline

सा॒वि॒त्राणि॑ । जु॒हो॒ति॒ । प्रसू᳚त्या॒ इति॒ प्र - सू॒त्यै॒ । च॒तु॒र्गृ॒ही॒तेनेति॑ चतुः - गृ॒ही॒तेन॑ । जु॒हो॒ति॒ । चतु॑ष्पाद॒ इति॒ चतुः॑-पा॒दः॒ । प॒शवः॑ । प॒शून् । ए॒व । अवेति॑ । रु॒न्धे॒ । चत॑स्रः । दिशः॑ । दि॒क्षु । ए॒व । प्रतीति॑ । ति॒ष्ठ॒ति॒ । छन्दाꣳ॑सि । दे॒वेभ्यः॑ । अपेति॑ । अ॒क्रा॒म॒न्न् । न । वः॒ । अ॒भा॒गानि॑ । ह॒व्यम् । व॒क्ष्या॒मः॒ । इति॑ । तेभ्यः॑ । ए॒तत् । च॒तु॒र्गृ॒ही॒तमिति॑ चतुः - गृ॒ही॒तम् । अ॒धा॒र॒य॒न्न् । पु॒रो॒नु॒वा॒क्या॑या॒ इति॑ पुरः - अ॒नु॒वा॒क्या॑यै । या॒ज्या॑यै । दे॒वता॑यै । व॒ष॒ट्का॒रायेति॑ वषट् - का॒राय॑ । यत् । च॒तु॒र्गृ॒ही॒तमिति॑ चतुः - गृ॒ही॒तम् । जु॒होति॑ । छन्दाꣳ॑सि । ए॒व । तत् । प्री॒णा॒ति॒ । तामि॑ । अ॒स्य॒ । प्री॒तानि॑ । दे॒वेभ्यः॑ । ह॒व्यम् । व॒ह॒न्ति॒ । यम् । का॒मये॑त ।  \newline




\markright{ TS 5.1.1.2  \hfill https://www.vedavms.in \hfill}
\addcontentsline{toc}{section}{ TS 5.1.1.2 }
\section*{ TS 5.1.1.2 }

\textbf{TS 5.1.1.2 } \newline
\textbf{Samhita Paata} \newline

पापी॑यान्थ् स्या॒दित्येकै॑कं॒ तस्य॑ जुहुया॒-दाहु॑तीभिरे॒वैन॒मप॑ गृह्णाति॒ पापी॑यान् भवति॒ यं का॒मये॑त॒ वसी॑यान्थ् स्या॒दिति॒ सर्वा॑णि॒ तस्या॑ऽनु॒द्रुत्य॑ जुहुया॒दाहु॑त्यै॒वैन॑म॒भि क्र॑मयति॒ वसी॑यान् भव॒त्यथो॑ य॒ज्ञ्स्यै॒वैषा-ऽभिक्रा᳚न्ति॒रेति॒ वा ए॒ष य॑ज्ञ्मु॒खा-दृद्ध्या॒ यो᳚ऽग्नेर्दे॒वता॑या॒ एत्य॒ष्टावे॒तानि॑ सावि॒त्राणि॑ भवन्त्य॒ष्टाक्ष॑रा गाय॒त्री गा॑य॒त्रो᳚ - [  ] \newline

\textbf{Pada Paata} \newline

पापी॑यान् । स्या॒त् । इति॑ । एकै॑क॒मित्येक᳚म् - ए॒क॒म् । तस्य॑ । जु॒हु॒या॒त् । आहु॑तीभि॒रित्याहु॑ति - भिः॒ । ए॒व । ए॒न॒म् । अवेति॑ । गृ॒ह्णा॒ति॒ । पापी॑यान् । भ॒व॒ति॒ । यम् । का॒मये॑त । वसी॑यान् । स्या॒त् । इति॑ । सर्वा॑णि । तस्य॑ । अ॒नु॒द्रुत्येत्य॑नु - द्रुत्य॑ । जु॒हु॒या॒त् । आहु॒त्येत्या - हु॒त्या॒ । ए॒व । ए॒न॒म् । अ॒भीति॑ । क्र॒म॒य॒ति॒ । वसी॑यान् । भ॒व॒ति॒ । अथो॒ इति॑ । य॒ज्ञ्स्य॑ । ए॒व । ए॒षा । अ॒भिक्रा᳚न्ति॒रित्य॒भि -  क्रा॒न्तिः॒ । एति॑ । वै । ए॒षः । य॒ज्ञ्॒मु॒खादिति॑ यज्ञ्-मु॒खात् । ऋद्ध्याः᳚ । यः । अ॒ग्नेः । दे॒वता॑याः । एति॑ । अ॒ष्टौ । ए॒तानि॑ । सा॒वि॒त्राणि॑ । भ॒व॒न्ति॒ । अ॒ष्टाक्ष॒रेत्य॒ष्टा - अ॒क्ष॒रा॒ । गा॒य॒त्री । गा॒य॒त्रः ।  \newline




\markright{ TS 5.1.1.3  \hfill https://www.vedavms.in \hfill}
\addcontentsline{toc}{section}{ TS 5.1.1.3 }
\section*{ TS 5.1.1.3 }

\textbf{TS 5.1.1.3 } \newline
\textbf{Samhita Paata} \newline

ऽग्निस्तेनै॒व य॑ज्ञ्मु॒खादृद्ध्या॑ अ॒ग्नेर्दे॒वता॑यै॒ नैत्य॒ष्टौ सा॑वि॒त्राणि॑ भव॒न्त्याहु॑तिर्नव॒मी त्रि॒वृत॑मे॒व य॑ज्ञ्मु॒खे विया॑तयति॒ यदि॑ का॒मये॑त॒ छन्दाꣳ॑सि यज्ञ्यश॒सेना᳚ ऽर्पयेय॒मित्यृच॑मन्त॒मां कु॑र्या॒च्छन्दाꣳ॑स्ये॒व य॑ज्ञ्यश॒सेना᳚ ऽर्पयति॒ यदि॑ का॒मये॑त॒ यज॑मानं ॅयज्ञ्यश॒सेना᳚-ऽर्पयेय॒मिति॒ यजु॑रन्त॒मं कु॑र्या॒द्-यज॑मानमे॒व य॑ज्ञ्यश॒सेना᳚-ऽर्पयत्यृ॒चा स्तोमꣳ॒॒ सम॑र्द्ध॒येत्या॑ - [  ] \newline

\textbf{Pada Paata} \newline

अ॒ग्निः । तेन॑ । ए॒व । य॒ज्ञ्॒मु॒खादिति॑ यज्ञ् - मु॒खात् । ऋद्ध्याः᳚ । अ॒ग्नेः । दे॒वता॑यै । न । ए॒ति॒ । अ॒ष्टौ । सा॒वि॒त्राणि॑ । भ॒व॒न्ति॒ । आहु॑ति॒रित्या - हू॒तिः॒ । न॒व॒मी । त्रि॒वृत॒मिति॑ त्रि - वृत᳚म् । ए॒व । य॒ज्ञ्॒मु॒ख इति॑ यज्ञ् - मु॒खे । वीति॑ । या॒त॒य॒ति॒ । यदि॑ । का॒मये॑त । छन्दाꣳ॑सि । य॒ज्ञ्॒य॒श॒सेनेति॑ यज्ञ् - य॒श॒सेन॑ । अ॒र्प॒ये॒य॒म् । इति॑ । ऋच᳚म् । अ॒न्त॒माम् । कु॒र्या॒त् । छन्दाꣳ॑सि । ए॒व । य॒ज्ञ्॒य॒श॒सेनेति॑ यज्ञ् - य॒श॒सेन॑ । अ॒र्प॒य॒ति॒ । यदि॑ । का॒मये॑त । यज॑मानम् । य॒ज्ञ्॒य॒श॒सेनेति॑ यज्ञ् - य॒श॒सेन॑ । अ॒र्प॒ये॒य॒म् । इति॑ । यजुः॑ । अ॒न्त॒मम् । कु॒र्या॒त् । यज॑मानम् । ए॒व । य॒ज्ञ्॒य॒श॒सेनेति॑ यज्ञ् - य॒श॒सेन॑ । अ॒र्प॒य॒ति॒ । ऋ॒चा । स्तोम᳚म् । समिति॑ । अ॒द्‌र्ध॒य॒ । इति॑ ।  \newline




\markright{ TS 5.1.1.4  \hfill https://www.vedavms.in \hfill}
\addcontentsline{toc}{section}{ TS 5.1.1.4 }
\section*{ TS 5.1.1.4 }

\textbf{TS 5.1.1.4 } \newline
\textbf{Samhita Paata} \newline

-ह॒ समृ॑द्ध्यै च॒तुर्भि॒रभ्रि॒मा द॑त्ते च॒त्वारि॒ छन्दाꣳ॑सि॒ छन्दो॑भिरे॒व दे॒वस्य॑ त्वा सवि॒तुः प्र॑स॒व इत्या॑ह॒ प्रसू᳚त्या अ॒ग्निर्दे॒वेभ्यो॒ निला॑यत॒ स वेणुं॒ प्राऽवि॑श॒थ् स ए॒तामू॒तिमनु॒ सम॑चर॒द्-यद्-वेणोः᳚ सुषि॒रꣳ सु॑षि॒रा-ऽभ्रि॑र्भवति सयोनि॒त्वाय॒ स यत्र॑य॒त्राऽव॑स॒त् तत् कृ॒ष्णम॑भवत् कल्मा॒षी भ॑वति रू॒पस॑मृद्ध्या उभयतः॒ क्ष्णूर्भ॑वती॒तश्चा॒ऽ-( ) -मुत॑श्चा॒ऽर्कस्या-व॑रुद्ध्यै व्याममा॒त्री भ॑वत्ये॒ताव॒द्वै पुरु॑षे वी॒र्यं॑ ॅवी॒र्य॑सम्मि॒ता ऽप॑रिमिता भव॒त्य-प॑रिमित॒स्याऽ व॑रुद्ध्यै॒ यो वन॒स्पती॑नां फल॒ग्रहिः॒ स ए॑षां ॅवी॒र्या॑वान् फल॒ग्रहि॒र्वेणु॑-र्वैण॒वी भ॑वति वी॒र्य॑स्या व॑रुद्ध्यै ॥ \newline

\textbf{Pada Paata} \newline

आ॒ह॒ । समृ॑द्ध्या॒ इति॒ सं - ऋ॒द्ध्यै॒ । च॒तुभि॒रिति॑ च॒तुः - भिः॒ । अभ्रि᳚म् । एति॑ । द॒त्ते॒ । च॒त्वारि॑ । छन्दाꣳ॑सि । छन्दो॑भि॒रिति॒ छन्दः॑ - भिः॒ । ए॒व । दे॒वस्य॑ । त्वा॒ । स॒वि॒तुः । प्र॒स॒व इति॑ प्र - स॒वे । इति॑ । आ॒ह॒ । प्रसू᳚त्या॒ इति॒ प्र - सू॒त्यै॒ । अ॒ग्निः । दे॒वेभ्यः॑ । निला॑यत । सः । वेणु᳚म् । प्रेति॑ । अ॒वि॒श॒त् । सः । ए॒ताम् । ऊ॒तिम् । अनु॑ । समिति॑ । अ॒च॒र॒त् । यत् । वेणोः᳚ । सु॒षि॒रम् । सु॒षि॒रा । अभ्रिः॑ । भ॒व॒ति॒ । स॒यो॒नि॒त्वायेति॑ सयोनि - त्वाय॑ । सः । यत्र॑य॒त्रेति॒ यत्र॑ - य॒त्र॒ । अव॑सत् । तत् । कृ॒ष्णम् । अ॒भ॒व॒त् । क॒ल्मा॒षी । भ॒व॒ति॒ । रू॒पस॑मृद्ध्या॒ इति॑ रू॒प - स॒मृ॒द्ध्यै॒ । उ॒भ॒य॒तः॒ क्ष्णूरित्यु॑भयतः-क्ष्णूः । भ॒व॒ति॒ । इ॒तः । च॒ ( ) । अ॒मुतः॑ । च॒ । अ॒र्कस्य॑ । अव॑रुद्ध्या॒ इत्यव॑ - रु॒द्ध्यै॒ । व्या॒म॒मा॒त्रीति॑ व्याम - मा॒त्री । भ॒व॒ति॒ । ए॒ताव॑त् । वै । पुरु॑षे । वी॒र्य᳚म् । वी॒र्य॑सम्मि॒तेति॑ वी॒र्य॑ - स॒म्मि॒ता॒ । अप॑रिमि॒तेत्यप॑रि-मि॒ता॒ । भ॒व॒ति॒ । अप॑रिमित॒स्येत्यप॑रि - मि॒त॒स्य॒ । अव॑रुद्ध्या॒ इत्यव॑ - रु॒द्ध्यै॒ । यः । वन॒स्पती॑नाम् । फ॒ल॒ग्रहि॒रिति॑ फल - ग्रहिः॑ । सः । ए॒षा॒म् । वी॒र्या॑वा॒निति॑ वी॒र्य॑ - वा॒न् । फ॒ल॒ग्रहि॒रिति॑ फल - ग्रहिः॑ । वेणुः॑ । वै॒ण॒वी । भ॒व॒ति॒ । वी॒र्य॑स्य । अव॑रुद्ध्या॒ इत्यव॑ - रु॒द्ध्यै॒ ॥  \newline




\markright{ TS 5.1.2.1  \hfill https://www.vedavms.in \hfill}
\addcontentsline{toc}{section}{ TS 5.1.2.1 }
\section*{ TS 5.1.2.1 }

\textbf{TS 5.1.2.1 } \newline
\textbf{Samhita Paata} \newline

व्यृ॑द्धं॒ ॅवा ए॒तद्-य॒ज्ञ्स्य॒ यद॑य॒जुष्के॑ण क्रि॒यत॑ इ॒माम॑गृभ्णन् रश॒ना-मृ॒तस्येत्य॑श्वाभि॒धानी॒मा द॑त्ते॒ यजु॑ष्कृत्यै य॒ज्ञ्स्य॒ समृ॑द्ध्यै॒ प्रतू᳚र्तं ॅवाजि॒न्ना द्र॒वेत्यश्व॑म॒भि द॑धाति रू॒पमे॒वास्यै॒तन् म॑हि॒मानं॒ ॅव्याच॑ष्टे यु॒ञ्जाथाꣳ॒॒ रास॑भं ॅयु॒वमिति॑ गर्द॒भमसे॑त्ये॒व ग॑र्द॒भं प्रति॑ ष्ठापयति॒ तस्मा॒दश्वा᳚द्-गर्द॒भोऽस॑त्तरो॒ योगे॑योगे त॒वस्त॑र॒मित्या॑ह॒ - [  ] \newline

\textbf{Pada Paata} \newline

व्यृ॑द्ध॒मिति॒ वि - ऋ॒द्ध॒म् । वै । ए॒तत् । य॒ज्ञ्स्य॑ । यत् । अ॒य॒जुष्के॒णेत्य॑य॒जुः-के॒न॒ । क्रि॒यते᳚ । इ॒माम् । अ॒गृ॒भ्ण॒न्न् । र॒श॒नाम् । ऋ॒तस्य॑ । इति॑ । अ॒श्वा॒भि॒धानी॒मित्य॑श्व - अ॒भि॒धानी᳚म् । एति॑ । द॒त्ते॒ । यजु॑ष्कृत्या॒ इति॒ यजुः॑-कृ॒त्यै॒ । य॒ज्ञ्स्य॑ । समृ॑द्ध्या॒ इति॒ सं-ऋ॒द्ध्यै॒ । प्रतू᳚र्त॒मिति॒ प्र - तू॒र्त॒म् । वा॒जि॒न्न् । एति॑ । द्र॒व॒ । इति॑ । अश्व᳚म् । अ॒भीति॑ । द॒धा॒ति॒ । रू॒पम् । ए॒व । अ॒स्य॒ । ए॒तत् । म॒हि॒मान᳚म् । व्याच॑ष्ट॒ इति॑ वि - आच॑ष्टे । यु॒ञ्जाथा᳚म् । रास॑भम् । यु॒वम् । इति॑ । ग॒र्द॒भम् । अस॑ति । ए॒व । ग॒र्द॒भम् । प्रतीति॑ । स्था॒प॒य॒ति॒ । तस्मा᳚त् । अश्वा᳚त् । ग॒र्द॒भः । अस॑त्तर॒ इत्यस॑त् - त॒रः॒ । योगे॑योग॒ इति॒ योगे᳚ - यो॒गे॒ । त॒वस्त॑र॒मिति॑ त॒वः - त॒र॒म् । इति॑ । आ॒ह॒ ।  \newline




\markright{ TS 5.1.2.2  \hfill https://www.vedavms.in \hfill}
\addcontentsline{toc}{section}{ TS 5.1.2.2 }
\section*{ TS 5.1.2.2 }

\textbf{TS 5.1.2.2 } \newline
\textbf{Samhita Paata} \newline

योगे॑योग ए॒वैनं॑ ॅयुङ्क्ते॒ वाजे॑वाजे हवामह॒ इत्या॒हान्नं॒ ॅवै वाजो-ऽन्न॑मे॒वाव॑ रुन्धे॒ सखा॑य॒ इन्द्र॑मू॒तय॒ इत्या॑हेन्द्रि॒यमे॒वाव॑ रुन्धे॒ ऽग्निर्दे॒वेभ्यो॒ निला॑यत॒ तं प्र॒जाप॑ति॒रन्व॑विन्दत् प्राजाप॒त्योऽश्वो ऽश्वे॑न॒ सं भ॑र॒त्यनु॑वित्त्यै पापवस्य॒सं ॅवा ए॒तत् क्रि॑यते॒ यच्छ्रेय॑सा च॒ पापी॑यसा च समा॒नं कर्म॑ कु॒र्वन्ति॒ पापी॑या॒न्॒. - [  ] \newline

\textbf{Pada Paata} \newline

योगे॑योग॒ इति॒ योगे᳚ - यो॒गे॒ । ए॒व । ए॒न॒म् । यु॒ङ्क्ते॒ । वाजे॑वाज॒ इति॒ वाजे᳚- वा॒जे॒ । ह॒वा॒म॒हे॒ । इति॑ । आ॒ह॒ । अन्न᳚म् । वै । वाजः॑ । अन्न᳚म् । ए॒व । अवेति॑ । रु॒न्धे॒ । सखा॑यः । इन्द्र᳚म् । ऊ॒तये᳚ । इति॑ । आ॒ह॒ । इ॒न्द्रि॒यम् । ए॒व । अवेति॑ । रु॒न्धे॒ । अ॒ग्निः । दे॒वेभ्यः॑ । निला॑यत । तम् । प्र॒जाप॑ति॒रिति॑ प्र॒जा-प॒तिः॒ । अन्विति॑ । अ॒वि॒न्द॒त् । प्रा॒जा॒प॒त्य इति॑ प्राजा - प॒त्यः । अश्वः॑ । अश्वे॑न । समिति॑ । भ॒र॒ति॒ । अनु॑वित्त्या॒ इत्यनु॑ - वि॒त्त्यै॒ । पा॒प॒व॒स्य॒समिति॑ पाप - व॒स्य॒सम् । वै । ए॒तत् । क्रि॒य॒ते॒ । यत् । श्रेय॑सा । च॒ । पापी॑यसा । च॒ । स॒मा॒नम् । कर्म॑ । कु॒र्वन्ति॑ । पापी॑यान् ।  \newline




\markright{ TS 5.1.2.3  \hfill https://www.vedavms.in \hfill}
\addcontentsline{toc}{section}{ TS 5.1.2.3 }
\section*{ TS 5.1.2.3 }

\textbf{TS 5.1.2.3 } \newline
\textbf{Samhita Paata} \newline

ह्यश्वा᳚द्-गर्द॒भोऽश्वं॒ पूर्वं॑ नयन्ति पापवस्य॒-सस्य॒ व्यावृ॑त्त्यै॒ तस्मा॒च्छ्रेयाꣳ॑सं॒ पापी॑यान् प॒श्चादन्वे॑ति ब॒हुर्वै भव॑तो॒ भ्रातृ॑व्यो॒ भव॑तीव॒ खलु॒ वा ए॒ष यो᳚ऽग्निं चि॑नु॒ते व॒ज्र्यश्वः॑ प्र॒तूर्व॒न्नेह्य॑व॒-क्राम॒न्न-श॑स्ती॒रित्या॑ह॒ वज्रे॑णै॒व पा॒प्मानं॒ भ्रातृ॑व्य॒मव॑ क्रामति रु॒द्रस्य॒ गाण॑पत्या॒दित्या॑ह रौ॒द्रा वै प॒शवो॑ रु॒द्रादे॒व - [  ] \newline

\textbf{Pada Paata} \newline

हि । अश्वा᳚त् । ग॒र्द॒भः । अश्व᳚म् । पूर्व᳚म् । न॒य॒न्ति॒ । पा॒प॒व॒स्य॒सस्येति॑ पाप - व॒स्य॒सस्य॑ । व्यावृ॑त्त्या॒ इति॑ वि - आवृ॑त्त्यै । तस्मा᳚त् । श्रेयाꣳ॑सम् । पापी॑यान् । प॒श्चात् । अन्विति॑ । ए॒ति॒ । ब॒हुः । वै । भव॑तः । भ्रातृ॑व्यः । भव॑ति । इ॒व॒ । खलु॑ । वै । ए॒षः । यः । अ॒ग्निम् । चि॒नु॒ते । व॒ज्री । अश्वः॑ । प्र॒तूर्व॒न्निति॑ प्र - तूर्वन्न्॑ । एति॑ । इ॒हि॒ । अ॒व॒क्राम॒न्नित्य॑व - क्रामन्न्॑ । अश॑स्तीः । इति॑ । आ॒ह॒ । वज्रे॑ण । ए॒व । पा॒प्मान᳚म् । भ्रातृ॑व्यम् । अवेति॑ । क्रा॒म॒ति॒ । रु॒द्रस्य॑ । गाण॑पत्या॒दिति॒ गाण॑ - प॒त्या॒त् । इति॑ । आ॒ह॒ । रौ॒द्राः । वै । प॒शवः॑ । रु॒द्रात् । ए॒व ।  \newline




\markright{ TS 5.1.2.4  \hfill https://www.vedavms.in \hfill}
\addcontentsline{toc}{section}{ TS 5.1.2.4 }
\section*{ TS 5.1.2.4 }

\textbf{TS 5.1.2.4 } \newline
\textbf{Samhita Paata} \newline

प॒शून् नि॒र्याच्या॒ऽऽ*त्मने॒ कर्म॑ कुरुते पू॒ष्णा स॒युजा॑ स॒हेत्या॑ह पू॒षा वा अद्ध्व॑नाꣳ सन्ने॒ता सम॑ष्ट्यै॒ पुरी॑षायतनो॒ वा ए॒ष यद॒ग्निरङ्गि॑रसो॒ वा ए॒तमग्रे॑ दे॒वता॑नाꣳ॒॒ सम॑भरन् पृथि॒व्याः स॒धस्था॑द॒ग्निं पु॑री॒ष्य॑-मङ्गिर॒स्व-दच्छे॒हीत्या॑ह॒ साय॑तनमे॒वैनं॑ दे॒वता॑भिः॒ सं भ॑रत्य॒ग्निं पु॑री॒ष्य॑-मङ्गिर॒स्व- दच्छे॑म॒ इत्या॑ह॒ येन॑ - [  ] \newline

\textbf{Pada Paata} \newline

प॒शून् । नि॒र्याच्येति॑ निः - याच्य॑ । आ॒त्मने᳚ । कर्म॑ । कु॒रु॒ते॒ । पू॒ष्णा । स॒युजेति॑ स-युजा᳚ । स॒ह । इति॑ । आ॒ह॒ । पू॒षा । वै । अद्ध्व॑नाम् । स॒नें॒तेति॑ सं - ने॒ता । सम॑ष्ट्या॒ इति॒ सं - अ॒ष्ट्यै॒ । पुरी॑षायतन॒ इति॒ पुरी॑ष - आ॒य॒त॒नः॒ । वै । ए॒षः । यत् । अ॒ग्निः । अङ्गि॑रसः । वै । ए॒तम् । अग्रे᳚ । दे॒वता॑नाम् । समिति॑ । अ॒भ॒र॒न्न् । पृ॒थि॒व्याः । स॒धस्था॒दिति॑ स॒ध - स्था॒त् । अ॒ग्निम् । पु॒री॒ष्य᳚म् । अ॒ङ्गि॒र॒स्वत् । अच्छ॑ । इ॒हि॒ । इति॑ । आ॒ह॒ । साय॑तन॒मिति॒ स - आ॒य॒त॒न॒म् । ए॒व । ए॒न॒म् । दे॒वता॑भिः । समिति॑ । भ॒र॒ति॒ । अ॒ग्निम् । पु॒री॒ष्य᳚म् । अ॒ङ्गि॒र॒स्वत् । अच्छ॑ । इ॒मः॒ । इति॑ । आ॒ह॒ । येन॑ ।  \newline




\markright{ TS 5.1.2.5  \hfill https://www.vedavms.in \hfill}
\addcontentsline{toc}{section}{ TS 5.1.2.5 }
\section*{ TS 5.1.2.5 }

\textbf{TS 5.1.2.5 } \newline
\textbf{Samhita Paata} \newline

स॒ङ्गच्छ॑ते॒ वाज॑मे॒वास्य॑ वृङ्क्ते प्र॒जाप॑तये प्रति॒प्रोच्या॒ग्निः स॒म्भृत्य॒ इत्या॑हुरि॒यं ॅवै प्र॒जाप॑ति॒स्तस्या॑ ए॒तच्छ्रोत्रं॒ ॅयद्व॒ल्मीको॒ऽग्निं पु॑री॒ष्य॑-मङ्गिर॒स्वद्-भ॑रिष्याम॒ इति॑ वल्मीकव॒पामुप॑ तिष्ठते सा॒क्षादे॒व प्र॒जाप॑तये प्रति॒प्रोच्या॒ऽग्निꣳ सं भ॑रत्य॒ग्निं पु॑री॒ष्य॑-मङ्गिर॒स्वद्-भ॑राम॒ इत्या॑ह॒ येन॑ स॒गंच्छ॑ते॒ वाज॑मे॒वास्य॑ वृ॒ङ्क्ते ऽन्व॒ग्निरु॒षसा॒मग्र॑ - [  ] \newline

\textbf{Pada Paata} \newline

स॒गंच्छ॑त॒ इति॑ सं - गच्छ॑ते । वाज᳚म् । ए॒व । अ॒स्य॒ । वृ॒ङ्क्ते॒ । प्र॒जाप॑तय॒ इति॑ प्र॒जा - प॒त॒ये॒ । प्र॒ति॒प्रोच्येति॑ प्रति - प्रोच्य॑ । अ॒ग्निः । स॒भृंत्य॒ इति॑ सं - भृत्यः॑ । इति॑ । आ॒हुः॒ । इ॒यम् । वै । प्र॒जाप॑ति॒रिति॑ प्र॒जा-प॒तिः॒ । तस्याः᳚ । ए॒तत् । श्रोत्र᳚म् । यत् । व॒ल्मीकः॑ । अ॒ग्निम् । पु॒री॒ष्य᳚म् । अ॒ङ्गि॒र॒स्वत् । भ॒रि॒ष्या॒मः॒ । इति॑ । व॒ल्मी॒क॒व॒पामिति॑ वल्मीक - व॒पाम् । उपेति॑ । ति॒ष्ठ॒ते॒ । सा॒क्षादिति॑ स-अ॒क्षात् । ए॒व । प्र॒जाप॑तय॒ इति॑ प्र॒जा - प॒त॒ये॒ । प्र॒ति॒प्रोच्येति॑ प्रति - प्रोच्य॑ । अ॒ग्निम् । समिति॑ । भ॒र॒ति॒ । अ॒ग्निम् । पु॒री॒ष्य᳚म् । अ॒ङ्गि॒र॒स्वत् । भ॒रा॒मः॒ । इति॑ । आ॒ह॒ । येन॑ । स॒गंच्छ॑त॒ इति॑ सं-गच्छ॑ते । वाज᳚म् । ए॒व । अ॒स्य॒ । वृ॒ङ्क्ते॒ । अन्विति॑ । अ॒ग्निः । उ॒षसा᳚म् । अग्र᳚म् ।  \newline




\markright{ TS 5.1.2.6  \hfill https://www.vedavms.in \hfill}
\addcontentsline{toc}{section}{ TS 5.1.2.6 }
\section*{ TS 5.1.2.6 }

\textbf{TS 5.1.2.6 } \newline
\textbf{Samhita Paata} \newline

मख्य॒दित्या॒हा-नु॑ख्यात्या आ॒गत्य॑ वा॒ज्यद्ध्व॑न आ॒क्रम्य॑ वाजिन् पृथि॒वीमित्या॑हे॒च्छत्ये॒वैनं॒ पूर्व॑या वि॒न्दत्युत्त॑रया॒ द्वाभ्या॒मा क्र॑मयति॒ प्रति॑ष्ठित्या॒ अनु॑रूपाभ्यां॒ तस्मा॒दनु॑रूपाः प॒शवः॒ प्रजा॑यन्ते॒ द्यौस्ते॑ पृ॒ष्ठं पृ॑थि॒वी स॒धस्थ॒मित्या॑है॒भ्यो वा ए॒तं ॅलो॒केभ्यः॑ प्र॒जाप॑तिः॒ समै॑रयद् रू॒पमे॒वास्यै॒-तन्म॑हि॒मानं॒ ॅव्याच॑ष्टे व॒ज्री वा ( ) ए॒ष यदश्वो॑ द॒द्-भिर॒न्यतो॑दद्भ्यो॒ भूयां॒ ॅलोम॑भिरुभ॒याद॑द्भ्यो॒ यं द्वि॒ष्यात् तम॑धस्प॒दं ध्या॑ये॒द्-वज्रे॑णै॒वैनꣳ॑ स्तृणुते ॥ \newline

\textbf{Pada Paata} \newline

अ॒ख्य॒त् । इति॑ । आ॒ह॒ । अनु॑ख्यात्या॒ इत्यनु॑ - ख्या॒त्यै॒ । आ॒गत्येत्या᳚ - गत्य॑ । वा॒जी । अद्ध्व॑नः । आ॒क्रम्येत्या᳚ - क्रम्य॑ । वा॒जि॒न्न् । पृ॒थि॒वीम् । इति॑ । आ॒ह॒ । इ॒च्छति॑ । ए॒व । ए॒न॒म् । पूर्व॑या । वि॒न्दति॑ । उत्त॑र॒येत्युत् - त॒र॒या॒ । द्वाभ्या᳚म् । एति॑ । क्र॒म॒य॒ति॒ । प्रति॑ष्ठित्या॒ इति॒ प्रति॑ - स्थि॒त्यै॒ । अनु॑रूपाभ्या॒मित्यनु॑ - रू॒पा॒भ्या॒म् । तस्मा᳚त् । अनु॑रूपा॒ इत्यनु॑-रू॒पाः॒ । प॒शवः॑ । प्रेति॑ । जा॒य॒न्ते॒ । द्यौः । ते॒ । पृ॒ष्ठम् । पृ॒थि॒वी । स॒धस्थ॒मिति॑ स॒ध - स्थ॒म् । इति॑ । आ॒ह॒ । ए॒भ्यः । वै । ए॒तम् । लो॒केभ्यः॑ । प्र॒जाप॑ति॒रिति॑ प्र॒जा - प॒तिः॒ । समिति॑ । ऐ॒र॒य॒त् । रू॒पम् । ए॒व । अ॒स्य॒ । ए॒तत् । म॒हि॒मान᳚म् । व्याच॑ष्ट॒ इति॑ वि - आच॑ष्टे । व॒ज्री । वै ( ) । ए॒षः । यत् । अश्वः॑ । द॒द्भिरिति॑ दत् - भिः । अ॒न्यतो॑दद्भ्य॒ इत्य॒न्यतो॑दत्-भ्यः॒ । भूयान्॑ । लोम॑भि॒रिति॒ लोम॑-भिः॒ । उ॒भ॒याद॑द्भ्य॒ इत्यु॑भ॒याद॑त् - भ्यः॒ । यम् । द्वि॒ष्यात् । तम् । अ॒ध॒स्प॒दमित्य॑धः - प॒दम् । ध्या॒ये॒त् । वज्रे॑ण । ए॒व । ए॒न॒म् । स्तृ॒णु॒ते॒ ॥  \newline




\markright{ TS 5.1.3.1  \hfill https://www.vedavms.in \hfill}
\addcontentsline{toc}{section}{ TS 5.1.3.1 }
\section*{ TS 5.1.3.1 }

\textbf{TS 5.1.3.1 } \newline
\textbf{Samhita Paata} \newline

उत्क्रा॒मो-द॑क्रमी॒दिति॒ द्वाभ्या॒मुत्क्र॑मयति॒ प्रति॑ष्ठित्या॒ अनु॑रूपाभ्यां॒ तस्मा॒दनु॑रूपाः प॒शवः॒ प्रजा॑यन्ते॒ ऽप उप॑ सृजति॒ यत्र॒ वा आप॑ उप॒ गच्छ॑न्ति॒ तदोष॑धयः॒ प्रति॑ तिष्ठ॒न्त्योष॑धीः प्रति॒तिष्ठ॑न्तीः प॒शवोऽनु॒ प्रति॑ तिष्ठन्ति प॒शून्. य॒ज्ञो य॒ज्ञ्ं ॅयज॑मानो॒ यज॑मानं प्र॒जास्तस्मा॑द॒प उप॑ सृजति॒ प्रति॑ष्ठित्यै॒ यद॑द्ध्व॒र्यु-र॑न॒ग्नावाहु॑तिं जुहु॒याद॒न्धो᳚ ऽद्ध्व॒र्युः - [  ] \newline

\textbf{Pada Paata} \newline

उदिति॑ । क्रा॒म॒ । उदिति॑ । अ॒क्र॒मी॒त् । इति॑ । द्वाभ्या᳚म् । उदिति॑ । क्र॒म॒य॒ति॒ । प्रति॑ष्ठित्या॒ इति॒ प्रति॑ - स्थि॒त्यै॒ । अनु॑रूपाभ्या॒मित्यनु॑-रू॒पा॒भ्या॒म् । तस्मा᳚त् । अनु॑रूपा॒ इत्यनु॑-रू॒पाः॒ । प॒शवः॑ । प्रेति॑ । जा॒य॒न्ते॒ । अ॒पः । उपेति॑ । सृ॒ज॒ति॒ । यत्र॑ । वै । आपः॑ । उ॒प॒गच्छ॒न्तीत्यु॑प - गच्छ॑न्ति । तत् । ओष॑धयः । प्रतीति॑ । ति॒ष्ठ॒न्ति॒ । ओष॑धीः । प्र॒ति॒तिष्ठ॑न्ती॒रिति॑ प्रति - तिष्ठ॑न्तीः । प॒शवः॑ । अनु॑ । प्रतीति॑ । ति॒ष्ठ॒न्ति॒ । प॒शून् । य॒ज्ञ्ः । य॒ज्ञ्म् । यज॑मानः । यज॑मानम् । प्र॒जा इति॑ प्र-जाः । तस्मा᳚त् । अ॒पः । उपेति॑ । सृ॒ज॒ति॒ । प्रति॑ष्ठित्या॒ इति॒ प्रति॑ - स्थि॒त्यै॒ । यत् । अ॒द्ध्व॒र्युः । अ॒न॒ग्नौ । आहु॑ति॒मित्या - हु॒ति॒म् । जु॒हु॒यात् । अ॒न्धः । अ॒द्ध्व॒र्युः ।  \newline




\markright{ TS 5.1.3.2  \hfill https://www.vedavms.in \hfill}
\addcontentsline{toc}{section}{ TS 5.1.3.2 }
\section*{ TS 5.1.3.2 }

\textbf{TS 5.1.3.2 } \newline
\textbf{Samhita Paata} \newline

स्या॒द्-रक्षाꣳ॑सि य॒ज्ञ्ꣳ ह॑न्यु॒र्॒.हिर॑ण्यमु॒पास्य॑ जुहोत्यग्नि॒वत्ये॒व जु॑होति॒ नान्धो᳚-ऽद्ध्व॒र्युर्भव॑ति॒ न य॒ज्ञ्ꣳ रक्षाꣳ॑सि घ्नन्ति॒ जिघ॑र्म्य॒ग्निं मन॑सा घृ॒तेनेत्या॑ह॒ मन॑सा॒ हि पुरु॑षो य॒ज्ञ्म॑भि॒गच्छ॑ति प्रति॒क्ष्यन्तं॒ भुव॑नानि॒ विश्वेत्या॑ह॒ सर्वꣳ॒॒ ह्ये॑ष प्र॒त्यङ् क्षेति॑ पृ॒थुं ति॑र॒श्चा वय॑सा बृ॒हन्त॒मित्या॒हाऽल्पो॒ ह्ये॑ष जा॒तो म॒हान् - [  ] \newline

\textbf{Pada Paata} \newline

स्या॒त् । रक्षाꣳ॑सि । य॒ज्ञ्म् । ह॒न्युः॒ । हिर॑ण्यम् । उ॒पास्येत्यु॑प - अस्य॑ । जु॒हो॒ति॒ । अ॒ग्नि॒वतीत्य॑ग्नि - वति॑ । ए॒व । जु॒हो॒ति॒ । न । अ॒न्धः । अ॒द्ध्व॒र्युः । भव॑ति । न । य॒ज्ञ्म् । रक्षाꣳ॑सि । घ्न॒न्ति॒ । जिघ॑र्मि । अ॒ग्निम् । मन॑सा । घृ॒तेन॑ । इति॑ । आ॒ह॒ । मन॑सा । हि । पुरु॑षः । य॒ज्ञ्म् । अ॒भि॒गच्छ॒तीत्य॑भि - गच्छ॑ति । प्र॒ति॒क्ष्यन्त॒मिति॑ प्रति - क्ष्यन्त᳚म् । भुव॑नानि । विश्वा᳚ । इति॑ । आ॒ह॒ । सर्व᳚म् । हि । ए॒षः । प्र॒त्यङ् । क्षेति॑ । पृ॒थुम् । ति॒र॒श्चा । वय॑सा । बृ॒हन्त᳚म् । इति॑ । आ॒ह॒ । अल्पः॑ । हि । ए॒षः । जा॒तः । म॒हान् ।  \newline




\markright{ TS 5.1.3.3  \hfill https://www.vedavms.in \hfill}
\addcontentsline{toc}{section}{ TS 5.1.3.3 }
\section*{ TS 5.1.3.3 }

\textbf{TS 5.1.3.3 } \newline
\textbf{Samhita Paata} \newline

भव॑ति॒ व्यचि॑ष्ठ॒मन्नꣳ॑ रभ॒सं ॅविदा॑न॒मित्या॒हा ऽन्न॑मे॒वाऽस्मै᳚ स्वदयति॒ सर्व॑मस्मै स्वदते॒ य ए॒वं ॅवेदा ऽऽ*त्वा॑ जिघर्मि॒ वच॑सा घृ॒तेनेत्या॑ह॒ तस्मा॒द्-यत् पुरु॑षो॒ मन॑सा-ऽभि॒गच्छ॑ति॒ तद्-वा॒चा व॑दत्य र॒क्षसेत्या॑ह॒ रक्ष॑सा॒मप॑हत्यै॒ मर्य॑श्रीः स्पृह॒यद्-व॑र्णो अ॒ग्निरित्या॒हा-प॑चितिमे॒वा-ऽस्मि॑न् दधा॒त्य-प॑चितिमान् भवति॒ य ए॒वं - [  ] \newline

\textbf{Pada Paata} \newline

भव॑ति । व्यचि॑ष्ठम् । अन्न᳚म् । र॒भ॒सम् । विदा॑नम् । इति॑ । आ॒ह॒ । अन्न᳚म् । ए॒व । अ॒स्मै॒ । स्व॒द॒य॒ति॒ । सर्व᳚म् । अ॒स्मै॒ । स्व॒द॒ते॒ । यः । ए॒वम् । वेद॑ । एति॑ । त्वा॒ । जि॒घ॒र्मि॒ । वच॑सा । घृ॒तेन॑ । इति॑ । आ॒ह॒ । तस्मा᳚त् । यत् । पुरु॑षः । मन॑सा । अ॒भि॒गच्छ॒तीत्य॑भि - गच्छ॑ति । तत् । वा॒चा । व॒द॒ति॒ । अ॒र॒क्षसा᳚ । इति॑ । आ॒ह॒ । रक्ष॑साम् । अप॑हत्या॒ इत्यप॑ - ह॒त्यै॒ । मर्य॑श्री॒रिति॒ मर्य॑ - श्रीः॒ । स्पृ॒ह॒यद्व॑र्ण॒ इति॑ स्पृह॒यत्-व॒र्णः॒ । अ॒ग्निः । इति॑ । आ॒ह॒ । अप॑चिति॒मित्यप॑-चि॒ति॒म् । ए॒व । अ॒स्मि॒न्न् । द॒धा॒ति॒ । अप॑चितिमा॒नित्यप॑चिति - मा॒न् । भ॒व॒ति॒ । यः । ए॒वम् ।  \newline




\markright{ TS 5.1.3.4  \hfill https://www.vedavms.in \hfill}
\addcontentsline{toc}{section}{ TS 5.1.3.4 }
\section*{ TS 5.1.3.4 }

\textbf{TS 5.1.3.4 } \newline
\textbf{Samhita Paata} \newline

ॅवेद॒ मन॑सा॒ त्वै तामाप्तु॑मर्.हति॒ याम॑द्ध्व॒र्युर॑-न॒ग्नावाहु॑तिं जु॒होति॒ मन॑स्वतीभ्यां जुहो॒त्याहु॑त्यो॒राप्त्यै॒ द्वाभ्यां॒ प्रति॑ष्ठित्यै यज्ञ्मु॒खे य॑ज्ञ्मुखे॒ वै क्रि॒यमा॑णे य॒ज्ञ्ꣳ रक्षाꣳ॑सि जिघाꣳ सन्त्ये॒तर्.हि॒ खलु॒ वा ए॒तद्-य॑ज्ञ्मु॒खं ॅयर्.ह्ये॑न॒दाहु॑तिरश्नु॒ते परि॑ लिखति॒ रक्ष॑सा॒मप॑हत्यै ति॒सृभिः॒ परि॑ लिखति त्रि॒वृद्वा अ॒ग्निर्यावा॑ने॒वा-ऽग्निस्तस्मा॒द्-रक्षाꣳ॒॒स्यप॑ हन्ति - [  ] \newline

\textbf{Pada Paata} \newline

वेद॑ । मन॑सा । तु । वै । ताम् । आप्तु᳚म् । अ॒र्॒.ह॒ति॒ । याम् । अ॒द्ध्व॒र्युः । अ॒न॒ग्नौ । आहु॑ति॒मित्या - हु॒ति॒म् । जु॒होति॑ । मन॑स्वतीभ्याम् । जु॒हो॒ति॒ । आहु॑त्यो॒रित्या - हू॒त्योः॒ । आप्त्यै᳚ । द्वाभ्या᳚म् । प्रति॑ष्ठित्या॒ इति॒ प्रति॑ - स्थि॒त्यै॒ । य॒ज्ञ्॒मु॒खे य॑ज्ञ्मुख॒ इति॑ यज्ञ्मु॒खे - य॒ज्ञ्॒मु॒खे॒ । वै । क्रि॒यमा॑णे । य॒ज्ञ्म् । रक्षाꣳ॑सि । जि॒घाꣳ॒॒स॒न्ति॒ । ए॒तर्.हि॑ । खलु॑ । वै । ए॒तत् । य॒ज्ञ्॒मु॒खमिति॑ यज्ञ् - मु॒खम् । यर्.हि॑ । ए॒न॒त् । आहु॑ति॒रित्या - हु॒तिः॒ । अ॒श्नु॒ते । परीति॑ । लि॒ख॒ति॒ । रक्ष॑साम् । अप॑हत्या॒ इत्यप॑ - ह॒त्यै॒ । ति॒सृभि॒रिति॑ ति॒सृ - भिः॒ । परीति॑ । लि॒ख॒ति॒ । त्रि॒वृदिति॑ त्रि - वृत् । वै । अ॒ग्निः । यावान्॑ । ए॒व । अ॒ग्निः । तस्मा᳚त् । रक्षाꣳ॑सि । अपेति॑ । ह॒न्ति॒ ।  \newline




\markright{ TS 5.1.3.5  \hfill https://www.vedavms.in \hfill}
\addcontentsline{toc}{section}{ TS 5.1.3.5 }
\section*{ TS 5.1.3.5 }

\textbf{TS 5.1.3.5 } \newline
\textbf{Samhita Paata} \newline

गायत्रि॒या परि॑ लिखति॒ तेजो॒ वै गा॑य॒त्री तेज॑सै॒वैनं॒ परि॑गृह्णाति त्रि॒ष्टुभा॒ परि॑ लिखतीन्द्रि॒यं ॅवै त्रि॒ष्टुगि॑न्द्रि॒येणै॒वैनं॒ परि॑ गृह्णात्यनु॒ष्टुभा॒ परि॑ लिखत्यनु॒ष्टुफ् सर्वा॑णि॒ छन्दाꣳ॑सि परि॒भूः पर्या᳚प्त्यै मद्ध्य॒तो॑ऽनु॒ष्टुभा॒ वाग्वा अ॑नु॒ष्टुप् तस्मा᳚न् मद्ध्य॒तो वा॒चा व॑दामो गायत्रि॒या प्र॑थ॒मया॒ परि॑ लिख॒त्यथा॑-ऽनु॒ष्टुभाऽथ॑ त्रि॒ष्टुभा॒ तेजो॒ वै गा॑य॒त्री ( ) य॒ज्ञो॑ ऽनु॒ष्टुगि॑न्द्रि॒यं त्रि॒ष्टुप् तेज॑सा चै॒वेन्द्रि॒येण॑ चोभ॒यतो॑ य॒ज्ञ्ं परि॑ गृह्णाति ॥ \newline

\textbf{Pada Paata} \newline

गा॒य॒त्रि॒या । परीति॑ । लि॒ख॒ति॒ । तेजः॑ । वै । गा॒य॒त्री । तेज॑सा । ए॒व । ए॒न॒म् । परीति॑ । गृ॒ह्णा॒ति॒ । त्रि॒ष्टुभा᳚ । परीति॑ । लि॒ख॒ति॒ । इ॒न्द्रि॒यम् । वै । त्रि॒ष्टुक् । इ॒न्द्रि॒येण॑ । ए॒व । ए॒न॒म् । परीति॑ । गृ॒ह्णा॒ति॒ । अ॒नु॒ष्टुभेत्य॑नु - स्तुभा᳚ । परीति॑ । लि॒ख॒ति॒ । अ॒नु॒ष्टुबित्य॑नु - स्तुप् । सर्वा॑णि । छन्दाꣳ॑सि । प॒रि॒भूरिति॑ परि - भूः । पर्या᳚प्त्या॒ इति॒ परि॑ - आ॒प्त्यै॒ । म॒द्ध्य॒तः । अ॒नु॒ष्टुभेत्य॑नु - स्तुभा᳚ । वाक् । वै । अ॒नु॒ष्टुबित्य॑नु - स्तुप् । तस्मा᳚त् । म॒द्ध्य॒तः । वा॒चा । व॒दा॒मः॒ । गा॒य॒त्रि॒या । प्र॒थ॒मया᳚ । परीति॑ । लि॒ख॒ति॒ । अथ॑ । अ॒नु॒ष्टुभेत्य॑नु - स्तुभा᳚ । अथ॑ । त्रि॒ष्टुभा᳚ । तेजः॑ । वै । गा॒य॒त्री ( ) । य॒ज्ञ्ः । अ॒नु॒ष्टुगित्य॑नु - स्तुक् । इ॒न्द्रि॒यम् । त्रि॒ष्टुप् । तेज॑सा । च॒ । ए॒व । इ॒न्द्रि॒येण॑ । च॒ । उ॒भ॒यतः॑ । य॒ज्ञ्म् । परीति॑ । गृ॒ह्णा॒ति॒ ॥  \newline




\markright{ TS 5.1.4.1  \hfill https://www.vedavms.in \hfill}
\addcontentsline{toc}{section}{ TS 5.1.4.1 }
\section*{ TS 5.1.4.1 }

\textbf{TS 5.1.4.1 } \newline
\textbf{Samhita Paata} \newline

दे॒वस्य॑ त्वा सवि॒तुः प्र॑स॒व इति॑ खनति॒ प्रसू᳚त्या॒ अथो॑ धू॒ममे॒वैतेन॑ जनयति॒ ज्योति॑ष्मन्तं त्वाऽग्ने सु॒प्रती॑क॒मित्या॑ह॒ ज्योति॑रे॒वैतेन॑ जनयति॒ सो᳚ऽग्निर्जा॒तः प्र॒जाः शु॒चाऽऽर्प॑य॒त् तं दे॒वा अ॑र्द्ध॒र्चेना॑-शमयच्छि॒वं प्र॒जाभ्योऽहिꣳ॑ सन्त॒मित्या॑ह प्र॒जाभ्य॑ ए॒वैनꣳ॑ शमयति॒ द्वाभ्यां᳚ खनति॒ प्रति॑ष्ठित्या अ॒पां पृ॒ष्ठम॒सीति॑ पुष्करप॒र्णमा - [  ] \newline

\textbf{Pada Paata} \newline

दे॒वस्य॑ । त्वा॒ । स॒वि॒तुः । प्र॒स॒व इति॑ प्र - स॒वे । इति॑ । ख॒न॒ति॒ । प्रसू᳚त्या॒ इति॒ प्र - सू॒त्यै॒ । अथो॒ इति॑ । धू॒मम् । ए॒व । ए॒तेन॑ । ज॒न॒य॒ति॒ । ज्योति॑ष्मन्तम् । त्वा॒ । अ॒ग्ने॒ । सु॒प्रती॑क॒मिति॑ सु - प्रती॑कम् । इति॑ । आ॒ह॒ । ज्योतिः॑ । ए॒व । ए॒तेन॑ । ज॒न॒य॒ति॒ । सः । अ॒ग्निः । जा॒तः । प्र॒जा इति॑ प्र - जाः । शु॒चा । आ॒र्प॒य॒त् । तम् । दे॒वाः । अ॒द्‌र्ध॒र्चेनेत्य॒॑द्‌र्ध - ऋ॒चेन॑ । अ॒श॒म॒य॒न्न् । शि॒वम् । प्र॒जाभ्य॒ इति॑ प्र - जाभ्यः॑ । अहिꣳ॑सन्तम् । इति॑ । आ॒ह॒ । प्र॒जाभ्य॒ इति॑ प्र - जाभ्यः॑ । ए॒व । ए॒न॒म् । श॒म॒य॒ति॒ । द्वाभ्या᳚म् । ख॒न॒ति॒ । प्रति॑ष्ठित्या॒ इति॒ प्रति॑ - स्थि॒त्यै॒ । अ॒पाम् । पृ॒ष्ठम् । अ॒सि॒ । इति॑ । पु॒ष्क॒र॒प॒र्णमिति॑ पुष्कर - प॒र्णम् । एति॑ ।  \newline




\markright{ TS 5.1.4.2  \hfill https://www.vedavms.in \hfill}
\addcontentsline{toc}{section}{ TS 5.1.4.2 }
\section*{ TS 5.1.4.2 }

\textbf{TS 5.1.4.2 } \newline
\textbf{Samhita Paata} \newline

ह॑रत्य॒पां ॅवा ए॒तत् पृ॒ष्ठं ॅयत् पु॑ष्करप॒र्णꣳ रू॒पेणै॒वैन॒दा ह॑रति पुष्करप॒र्णेन॒ सं भ॑रति॒ योनि॒र्वा अ॒ग्नेः पु॑ष्करप॒र्णꣳ सयो॑निमे॒वाग्निꣳ संभ॑रति कृष्णाजि॒नेन॒ संभ॑रति य॒ज्ञो वै कृ॑ष्णाजि॒नं ॅय॒ज्ञेनै॒व य॒ज्ञ्ꣳ संभ॑रति॒ यद् ग्रा॒म्याणां᳚ पशू॒नां चर्म॑णा स॒भंरे᳚द् ग्रा॒म्यान् प॒शूञ्छु॒चाऽर्प॑येत् कृष्णाजि॒नेन॒ संभ॑रत्यार॒ण्याने॒व प॒शून् - [  ] \newline

\textbf{Pada Paata} \newline

ह॒र॒ति॒ । अ॒पाम् । वै । ए॒तत् । पृ॒ष्ठम् । यत् । पु॒ष्क॒र॒प॒र्णमिति॑ पुष्कर - प॒र्णम् । रू॒पेण॑ । ए॒व । ए॒न॒त् । एति॑ । ह॒र॒ति॒ । पु॒ष्क॒र॒प॒र्णेनेति॑ पुष्कर - प॒र्णेन॑ । समिति॑ । भ॒र॒ति॒ । योनिः॑ । वै । अ॒ग्नेः । पु॒ष्क॒र॒प॒र्णमिति॑ पुष्कर - प॒र्णम् । सयो॑नि॒मिति॒ स-यो॒नि॒म् । ए॒व । अ॒ग्निम् । समिति॑ । भ॒र॒ति॒ । कृ॒ष्णा॒जि॒नेनेति॑ कृष्ण-अ॒जि॒नेन॑ । समिति॑ । भ॒र॒ति॒ । य॒ज्ञ्ः । वै । कृ॒ष्णा॒जि॒नमिति॑ कृष्ण - अ॒जि॒नम् । य॒ज्ञेन॑ । ए॒व । य॒ज्ञ्म् । समिति॑ । भ॒र॒ति॒ । यत् । ग्रा॒म्याणा᳚म् । प॒शू॒नाम् । चर्म॑णा । स॒भंरे॒दिति॑ सं - भरे᳚त् । ग्रा॒म्यान् । प॒शून् । शु॒चा । अ॒र्प॒ये॒त् । कृ॒ष्णा॒जि॒नेनेति॑ कृष्ण - अ॒जि॒नेन॑ । समिति॑ । भ॒र॒ति॒ । आ॒र॒ण्यान् । ए॒व । प॒शून् ।  \newline




\markright{ TS 5.1.4.3  \hfill https://www.vedavms.in \hfill}
\addcontentsline{toc}{section}{ TS 5.1.4.3 }
\section*{ TS 5.1.4.3 }

\textbf{TS 5.1.4.3 } \newline
\textbf{Samhita Paata} \newline

शु॒चाऽर्प॑यति॒ तस्मा᳚थ् स॒माव॑त् पशू॒नां प्र॒जाय॑मानाना-मार॒ण्याः प॒शवः॒ कनी॑याꣳसः शु॒चा ह्यृ॑ता लो॑म॒तः संभ॑र॒त्यतो॒ ह्य॑स्य॒ मेद्ध्यं॑ कृष्णाजि॒नं च॑ पुष्करप॒र्णं च॒ सꣳ स्तृ॑णाती॒यं ॅवै कृ॑ष्णाजि॒नम॒सौ पु॑ष्करप॒र्ण-मा॒भ्या-मे॒वैन॑-मुभ॒यतः॒ परि॑गृह्णात्य॒-ग्निर्दे॒वेभ्यो॒ निला॑यत॒ तमथ॒र्वा-ऽन्व॑पश्य॒दथ॑र्वा त्वा प्रथ॒मो निर॑मन्थदग्न॒ इत्या॑ - [  ] \newline

\textbf{Pada Paata} \newline

शु॒चा । अ॒र्प॒य॒ति॒ । तस्मा᳚त् । स॒माव॑त् । प॒शू॒नाम् । प्र॒जाय॑मानाना॒मिति॑ प्र-जाय॑मानानाम् । आ॒र॒ण्याः । प॒शवः॑ । कनी॑याꣳसः । शु॒चा । हि । ऋ॒ताः । लो॒म॒तः । समिति॑ । भ॒र॒ति॒ । अतः॑ । हि । अ॒स्य॒ । मेद्ध्य᳚म् । कृ॒ष्णा॒जि॒नमिति॑ कृष्ण - अ॒जि॒नम् । च॒ । पु॒ष्क॒र॒प॒र्णमिति॑ पुष्कर - प॒र्णम् । च॒ । समिति॑ । स्तृ॒णा॒ति॒ । इ॒यम् । वै । कृ॒ष्णा॒जि॒नमिति॑ कृष्ण - अ॒जि॒नम् । अ॒सौ । पु॒ष्क॒र॒प॒र्णमिति॑ पुष्कर - प॒र्णम् । आ॒भ्याम् । ए॒व । ए॒न॒म् । उ॒भ॒यतः॑ । परीति॑ । गृ॒ह्णा॒ति॒ । अ॒ग्निः । दे॒वेभ्यः॑ । निला॑यत । तम् । अथ॑र्वा । अन्विति॑ । अ॒प॒श्य॒त् । अथ॑र्वा । त्वा॒ । प्र॒थ॒मः । निरिति॑ । अ॒म॒न्थ॒त् । अ॒ग्ने॒ । इति॑ ।  \newline




\markright{ TS 5.1.4.4  \hfill https://www.vedavms.in \hfill}
\addcontentsline{toc}{section}{ TS 5.1.4.4 }
\section*{ TS 5.1.4.4 }

\textbf{TS 5.1.4.4 } \newline
\textbf{Samhita Paata} \newline

-ह॒ य ए॒वैन॑म॒न्वप॑श्य॒त् तेनै॒वैनꣳ॒॒ संभ॑रति॒ त्वाम॑ग्ने॒ पुष्क॑रा॒दधीत्या॑ह पुष्करप॒र्णे ह्ये॑न॒मुप॑श्रित॒-मवि॑न्द॒त् तमु॑ त्वा द॒द्ध्यङ्ङृषि॒रित्या॑ह द॒द्ध्यङ्वा आ॑थर्व॒ण-स्ते॑ज॒स्व्या॑सी॒त् तेज॑ ए॒वास्मि॑न् दधाति॒ तमु॑ त्वा पा॒थ्यो वृषेत्या॑ह॒ पूर्व॑मे॒वोदि॒त-मुत्त॑रेणा॒भि गृ॑णाति - [  ] \newline

\textbf{Pada Paata} \newline

आ॒ह॒ । यः । ए॒व । ए॒न॒म् । अ॒न्वप॑श्य॒दित्य॑नु-अप॑श्यत् । तेन॑ । ए॒व । ए॒न॒म् । समिति॑ । भ॒र॒ति॒ । त्वाम् । अ॒ग्ने॒ । पुष्क॑रात् । अधीति॑ । इति॑ । आ॒ह॒ । पु॒ष्क॒रप॒र्ण इति॑ पुष्कर - प॒र्णे । हि । ए॒न॒म् । उप॑श्रित॒मित्युप॑ - श्रि॒त॒म् । अवि॑न्दत् । तम् । उ॒ । त्वा॒ । द॒द्ध्यङ् । ऋषिः॑ । इति॑ । आ॒ह॒ । द॒द्ध्यङ् । वै । आ॒थ॒र्व॒णः । ते॒ज॒स्वी । आ॒सी॒त् । तेजः॑ । ए॒व । अ॒स्मि॒न्न् । द॒धा॒ति॒ । तम् । उ॒ । त्वा॒ । पा॒थ्यः । वृषा᳚ । इति॑ । आ॒ह॒ । पूर्व᳚म् । ए॒व । उ॒दि॒तम् । उत्त॑रे॒णेत्युत् - त॒रे॒ण॒ । अ॒भीति॑ । गृ॒णा॒ति॒ ।  \newline




\markright{ TS 5.1.4.5  \hfill https://www.vedavms.in \hfill}
\addcontentsline{toc}{section}{ TS 5.1.4.5 }
\section*{ TS 5.1.4.5 }

\textbf{TS 5.1.4.5 } \newline

\textbf{Pada Paata} \newline

च॒त॒सृभि॒रिति॑ चत॒सृ-भिः॒ । समिति॑ । भ॒र॒ति॒ । च॒त्वारि॑ । छन्दाꣳ॑सि । छन्दो॑भि॒रिति॒ छन्दः॑-भिः॒ । ए॒व । गा॒य॒त्रीभिः॑ । ब्रा॒ह्म॒णस्य॑ । गा॒य॒त्रः । हि । ब्रा॒ह्म॒णः । त्रि॒ष्टुग्भि॒रिति॑ त्रि॒ष्टुक् - भिः॒ । रा॒ज॒न्य॑स्य । त्रैष्टु॑भः । हि । रा॒ज॒न्यः॑ । यम् । का॒मये॑त । वसी॑यान् । स्या॒त् । इति॑ । उ॒भयी॑भिः । तस्य॑ । समिति॑ । भ॒रे॒त् । तेजः॑ । च॒ । ए॒व । अ॒स्मै॒ । इ॒न्द्रि॒यम् । च॒ । स॒मीची॒ इति॑ । द॒धा॒ति॒ । अ॒ष्टा॒भिः । समिति॑ । भ॒र॒ति॒ । अ॒ष्टाक्ष॒रेत्य॒ष्टा- अ॒क्ष॒रा॒ । गा॒य॒त्री । गा॒य॒त्रः । अ॒ग्निः । यावान्॑ । ए॒व । अ॒ग्निः । तम् । समिति॑ । भ॒र॒ति॒ । सीद॑ । हो॒तः॒ । इति॑ ( ) । आ॒ह॒ । दे॒वताः᳚ । ए॒व । अ॒स्मै॒ । समिति॑ । सा॒द॒य॒ति॒ । नीति॑ । होता᳚ । इति॑ । म॒नु॒ष्यान्॑ । समिति॑ । सी॒द॒स्व॒ । इति॑ । वयाꣳ॑सि । जनि॑ष्व । हि । जेन्यः॑ । अग्रे᳚ । अह्ना᳚म् । इति॑ । आ॒ह॒ । दे॒व॒म॒नु॒ष्यानिति॑ देव - म॒नु॒ष्यान् । ए॒व । अ॒स्मै॒ । सꣳस॑न्ना॒निति॒ सं - स॒न्ना॒न् । प्रेति॑ । ज॒न॒य॒ति॒ ॥  \newline




\markright{ TS 5.1.5.1  \hfill https://www.vedavms.in \hfill}
\addcontentsline{toc}{section}{ TS 5.1.5.1 }
\section*{ TS 5.1.5.1 }

\textbf{TS 5.1.5.1 } \newline
\textbf{Samhita Paata} \newline

क्रू॒रमि॑व॒ वा अ॑स्या ए॒तत् क॑रोति॒ यत् खन॑त्य॒प उप॑ सृज॒त्यापो॒ वै शा॒न्ताः शा॒न्ताभि॑रे॒वाऽस्यै॒ शुचꣳ॑ शमयति॒ सं ते॑ वा॒युर्मा॑त॒रिश्वा॑ दधा॒त्वित्या॑ह प्रा॒णो वै वा॒युः प्रा॒णेनै॒वास्यै᳚ प्रा॒णꣳ सं द॑धाति॒ सं ते॑ वा॒युरित्या॑ह॒ तस्मा᳚द्-वा॒युप्र॑च्युता दि॒वो वृष्टि॑रीर्ते॒ तस्मै॑ च देवि॒ वष॑डस्तु॒ - [  ] \newline

\textbf{Pada Paata} \newline

क्रू॒रम् । इ॒व॒ । वै । अ॒स्याः॒ । ए॒तत् । क॒रो॒ति॒ । यत् । खन॑ति । अ॒पः । उपेति॑ । सृ॒ज॒ति॒ । आपः॑ । वै । शा॒न्ताः । शा॒न्ताभिः॑ । ए॒व । अ॒स्यै॒ । शुच᳚म् । श॒म॒य॒ति॒ । समिति॑ । ते॒ । वा॒युः । मा॒त॒रिश्वा᳚ । द॒धा॒तु॒ । इति॑ । आ॒ह॒ । प्रा॒ण इति॑ प्र - अ॒नः । वै । वा॒युः । प्रा॒णेनेति॑ प्र - अ॒नेन॑ । ए॒व । अ॒स्यै॒ । प्रा॒णमिति॑ प्र - अ॒नम् । समिति॑ । द॒धा॒ति॒ । समिति॑ । ते॒ । वा॒युः । इति॑ । आ॒ह॒ । तस्मा᳚त् । वा॒युप्र॑च्यु॒तेति॑ वा॒यु - प्र॒च्यु॒ता॒ । दि॒वः । वृष्टिः॑ । ई॒र्ते॒ । तस्मै᳚ । च॒ । दे॒वि॒ । वष॑ट् । अ॒स्तु॒ ।  \newline




\markright{ TS 5.1.5.2  \hfill https://www.vedavms.in \hfill}
\addcontentsline{toc}{section}{ TS 5.1.5.2 }
\section*{ TS 5.1.5.2 }

\textbf{TS 5.1.5.2 } \newline
\textbf{Samhita Paata} \newline

तुभ्य॒मित्या॑ह॒ षड्वा ऋ॒तव॑ ऋ॒तुष्वे॒व वृष्टिं॑ दधाति॒ तस्मा॒थ् सर्वा॑नृ॒तून्. व॑र्.षति॒ यद्-व॑षट्कु॒र्याद्-या॒तया॑माऽस्य वषट्का॒रः स्या॒द्यन्न व॑षट्कु॒र्याद् रक्षाꣳ॑सि य॒ज्ञ्ꣳ ह॑न्यु॒र्वडित्या॑ह प॒रोक्ष॑मे॒व वष॑ट् करोति॒ नास्य॑ या॒तया॑मा वषट्का॒रो भव॑ति॒ न य॒ज्ञ्ꣳ रक्षाꣳ॑सि घ्नन्ति॒ सुजा॑तो॒ ज्योति॑षा स॒हेत्य॑नु॒ष्टुभोप॑ नह्यत्यनु॒ष्टु - [  ] \newline

\textbf{Pada Paata} \newline

तुभ्य᳚म् । इति॑ । आ॒ह॒ । षट् । वै । ऋ॒तवः॑ । ऋ॒तुषु॑ । ए॒व । वृष्टि᳚म् । द॒धा॒ति॒ । तस्मा᳚त् । सर्वान्॑ । ऋ॒तून् । व॒र्॒.ष॒ति॒ । यत् । व॒ष॒ट्कु॒र्यादिति॑ वषट् - कु॒र्यात् । या॒तया॒मेति॑ या॒त - या॒मा॒ । अ॒स्य॒ । व॒ष॒ट्का॒र इति॑ वषट् - का॒रः । स्या॒त् । यत् । न । व॒ष॒ट्कु॒र्यादिति॑ वषट् - कु॒र्यात् । रक्षाꣳ॑सि । य॒ज्ञ्म् । ह॒न्युः॒ । वट् । इति॑ । आ॒ह॒ । प॒रोक्ष॒मिति॑ परः - अक्ष᳚म् । ए॒व । वष॑ट् । क॒रो॒ति॒ । न । अ॒स्य॒ । या॒तया॒मेति॑ या॒त - या॒मा॒ । व॒ष॒ट्का॒र इति॑ वषट् - का॒रः । भव॑ति । न । य॒ज्ञ्म् । रक्षाꣳ॑सि । घ्न॒न्ति॒ । सुजा॑त॒ इति॒ सु - जा॒तः॒ । ज्योति॑षा । स॒ह । इति॑ । अ॒नु॒ष्टुभेत्य॑नु - स्तुभा᳚ । उपेति॑ । न॒ह्य॒ति॒ । अ॒नु॒ष्टुबित्य॑नु - स्तुप् ।  \newline




\markright{ TS 5.1.5.3  \hfill https://www.vedavms.in \hfill}
\addcontentsline{toc}{section}{ TS 5.1.5.3 }
\section*{ TS 5.1.5.3 }

\textbf{TS 5.1.5.3 } \newline
\textbf{Samhita Paata} \newline

-फ्सर्वा॑णि॒ छन्दाꣳ॑सि॒ छन्दाꣳ॑सि॒ खलु॒ वा अ॒ग्नेः प्रि॒या त॒नूः प्रि॒ययै॒वैनं॑ त॒नुवा॒ परि॑ दधाति॒ वेदु॑को॒ वासो॑ भवति॒य ए॒वं ॅवेद॑ वारु॒णो वा अ॒ग्निरुप॑नद्ध॒ उदु॑ तिष्ठ स्वद्ध्वरो॒र्द्ध्व ऊ॒ षुण॑ ऊ॒तय॒ इति॑ सावि॒त्रीभ्या॒मुत् ति॑ष्ठति सवि॒तृप्र॑सूत ए॒वास्यो॒र्द्ध्वां ॅव॑रुणमे॒निमुथ् सृ॑जति॒ द्वाभ्यां॒ प्रति॑ष्ठित्यै॒ स जा॒तो गर्भो॑ असि॒ - [  ] \newline

\textbf{Pada Paata} \newline

सर्वा॑णि । छन्दाꣳ॑सि । छन्दाꣳ॑सि । खलु॑ । वै । अ॒ग्नेः । प्रि॒या । त॒नूः । प्रि॒यया᳚ । ए॒व । ए॒न॒म् । त॒नुवा᳚ । परीति॑ । द॒धा॒ति॒ । वेदु॑कः । वासः॑ । भ॒व॒ति॒ । यः । ए॒वम् । वेद॑ । वा॒रु॒णः । वै । अ॒ग्निः । उप॑नद्ध॒ इत्युप॑ - न॒द्धः॒ । उदिति॑ । उ॒ । ति॒ष्ठ॒ । स्व॒द्ध्व॒रेति॑ सु - अ॒द्ध्व॒र॒ । ऊ॒द्‌र्ध्वः । उ॒ । स्विति॑ । नः॒ । ऊ॒तये᳚ । इति॑ । सा॒वि॒त्रीभ्या᳚म् । उदिति॑ । ति॒ष्ठ॒ति॒ । स॒वि॒तृप्र॑सूत॒ इति॑ सवि॒तृ - प्र॒सू॒तः॒ । ए॒व । अ॒स्य॒ । ऊ॒द्‌र्ध्वाम् । व॒रु॒ण॒मे॒निमिति॑ वरुण - मे॒निम् । उदिति॑ । सृ॒ज॒ति॒ । द्वाभ्या᳚म् । प्रति॑ष्ठित्या॒ इति॒ प्रति॑ - स्थि॒त्यै॒ । सः । जा॒तः । गर्भः॑ । अ॒सि॒ ।  \newline




\markright{ TS 5.1.5.4  \hfill https://www.vedavms.in \hfill}
\addcontentsline{toc}{section}{ TS 5.1.5.4 }
\section*{ TS 5.1.5.4 }

\textbf{TS 5.1.5.4 } \newline
\textbf{Samhita Paata} \newline

रोद॑स्यो॒रित्या॑हे॒मे वै रोद॑सी॒ तयो॑रे॒ष गर्भो॒ यद॒ग्नि-स्तस्मा॑-दे॒वमा॒हाग्ने॒ चारु॒र्विभृ॑त॒ ओष॑धी॒ष्वित्या॑ह य॒दा ह्ये॑तं ॅवि॒भर॒न्त्यथ॒ चारु॑तरो॒ भव॑ति॒ प्र मा॒तृभ्यो॒ अधि॒ कनि॑क्रदद्-गा॒ इत्या॒हौष॑धयो॒ वा अ॑स्य मा॒तर॒स्ताभ्य॑ ए॒वैनं॒ प्रच्या॑वयति स्थि॒रो भ॑व वी॒ड्व॑ङ्ग॒ इति॑ गर्द॒भ आ सा॑दयति॒ - [  ] \newline

\textbf{Pada Paata} \newline

रोद॑स्योः । इति॑ । आ॒ह॒ । इ॒मे इति॑ । वै । रोद॑सी॒ इति॑ । तयोः᳚ । ए॒षः । गर्भः॑ । यत् । अ॒ग्निः । तस्मा᳚त् । ए॒वम् । आ॒ह॒ । अग्ने᳚ । चारुः॑ । विभृ॑त॒ इति॒ वि - भृ॒तः॒ । ओष॑धीषु । इति॑ । आ॒ह॒ । य॒दा । हि । ए॒तम् । वि॒भर॒न्तीति॑ वि - भर॑न्ति । अथ॑ । चारु॑तर॒ इति॒ चारु॑ - त॒रः॒ । भव॑ति । प्रेति॑ । मा॒तृभ्य॒ इति॑ मा॒तृ - भ्यः॒ । अधीति॑ । कनि॑क्रदत् । गाः॒ । इति॑ । आ॒ह॒ । ओष॑धयः । वै । अ॒स्य॒ । मा॒तरः॑ । ताभ्यः॑ । ए॒व । ए॒न॒म् । प्रेति॑ । च्या॒व॒य॒ति॒ । स्थि॒रः । भ॒व॒ । वी॒ड्व॑ङ्ग॒ इति॑ वी॒डु - अ॒ङ्गः॒ । इति॑ । ग॒र्द॒भे । एति॑ । सा॒द॒य॒ति॒ ।  \newline




\markright{ TS 5.1.5.5  \hfill https://www.vedavms.in \hfill}
\addcontentsline{toc}{section}{ TS 5.1.5.5 }
\section*{ TS 5.1.5.5 }

\textbf{TS 5.1.5.5 } \newline
\textbf{Samhita Paata} \newline

सं न॑ह्यत्ये॒वैन॑मे॒तया᳚ स्थे॒म्ने ग॑र्द॒भेन॒ संभ॑रति॒ तस्मा᳚द् गर्द॒भः प॑शू॒नां भा॑रभा॒रित॑मो गर्द॒भेन॒ सं भ॑रति॒ तस्मा᳚द् गर्द॒भो-ऽप्य॑नाले॒शे-ऽत्य॒न्यान् प॒शून् मे᳚द्य॒त्यन्नꣳ॒॒ ह्ये॑नेना॒ऽर्कꣳ स॒भंर॑न्ति गर्द॒भेन॒ संभ॑रति॒ तस्मा᳚द् गर्द॒भो द्वि॒रेताः॒ सन् कनि॑ष्ठं पशू॒नां प्रजा॑यते॒ऽग्निर्.ह्य॑स्य॒ योनिं॑ नि॒र्दह॑ति प्र॒जासु॒ वा ए॒ष ए॒तर्.ह्यारू॑ढः॒ - [  ] \newline

\textbf{Pada Paata} \newline

समिति॑ । न॒ह्य॒ति॒ । ए॒व । ए॒न॒म् । ए॒तया᳚ । स्थे॒म्ने । ग॒र्द॒भेन॑ । समिति॑ । भ॒र॒ति॒ । तस्मा᳚त् । ग॒र्द॒भः । प॒शू॒नाम् । भा॒र॒भा॒रित॑म॒ इति॑ भारभा॒रि - त॒मः॒ । ग॒र्द॒भेन॑ । समिति॑ । भ॒र॒ति॒ । तस्मा᳚त् । ग॒र्द॒भः । अपीति॑ । अ॒ना॒ले॒श इत्य॑ना - ले॒शे । अतीति॑ । अ॒न्यान् । प॒शून् । मे॒द्य॒ति॒ । अन्न᳚म् । हि । ए॒ने॒न॒ । अ॒र्कम् । स॒भंर॒न्तीति॑ सं - भर॑न्ति । ग॒र्द॒भेन॑ । समिति॑ । भ॒र॒ति॒ । तस्मा᳚त् । ग॒र्द॒भः । द्वि॒रेता॒ इति॑ द्वि-रेताः᳚ । सन्न् । कनि॑ष्ठम् । प॒शू॒नाम् । प्रेति॑ । जा॒य॒ते॒ । अ॒ग्निः । हि । अ॒स्य॒ । योनि᳚म् । नि॒र्दह॒तीति॑ निः - दह॑ति । प्र॒जास्विति॑ प्र - जासु॑ । वै । ए॒षः । ए॒तर्.हि॑ । आरू॑ढ॒ इत्या - रू॒ढः॒ ।  \newline




\markright{ TS 5.1.5.6  \hfill https://www.vedavms.in \hfill}
\addcontentsline{toc}{section}{ TS 5.1.5.6 }
\section*{ TS 5.1.5.6 }

\textbf{TS 5.1.5.6 } \newline
\textbf{Samhita Paata} \newline

स ई᳚श्व॒रः प्र॒जाः शु॒चा प्र॒दहः॑ शि॒वो भ॑व प्र॒जाभ्य॒ इत्या॑ह प्र॒जाभ्य॑ ए॒वैनꣳ॑ शमयति॒ मानु॑षीभ्य॒स्त्वम॑ङ्गिर॒ इत्या॑ह मान॒व्यो॑ हि प्र॒जा मा द्यावा॑पृथि॒वी अ॒भि शू॑शुचो॒ माऽन्तरि॑क्षं॒ मा वन॒स्पती॒नित्या॑है॒भ्य ए॒वैनं॑ ॅलो॒केभ्यः॑ शमयति॒ प्रैतु॑ वा॒जी कनि॑क्रद॒दित्या॑ह वा॒जी ह्ये॑ष नान॑द॒द्-रास॑भः॒ पत्वेत्या॑ - [  ] \newline

\textbf{Pada Paata} \newline

सः । ई॒श्व॒रः । प्र॒जा इति॑ प्र - जाः । शु॒चा । प्र॒दह॒ इति॑ प्र-दहः॑ । शि॒वः । भ॒व॒ । प्र॒जाभ्य॒ इति॑ प्र - जाभ्यः॑ । इति॑ । आ॒ह॒ । प्र॒जाभ्य॒ इति॑ प्र - जाभ्यः॑ । ए॒व । ए॒न॒म् । श॒म॒य॒ति॒ । मानु॑षीभ्यः । त्वम् । अ॒ङ्गि॒रः॒ । इति॑ । आ॒ह॒ । मा॒न॒व्यः॑ । हि । प्र॒जा इति॑ प्र - जाः । मा । द्यावा॑पृथि॒वी इति॒ द्यावा᳚ - पृ॒थि॒वी । अ॒भीति॑ । शू॒शु॒चः॒ । मा । अ॒न्तरि॑क्षम् । मा । वन॒स्पतीन्॑ । इति॑ । आ॒ह॒ । ए॒भ्यः । ए॒व । ए॒न॒म् । लो॒केभ्यः॑ । श॒म॒य॒ति॒ । प्रेति॑ । ए॒तु॒ । वा॒जी । कनि॑क्रदत् । इति॑ । आ॒ह॒ । वा॒जी । हि । ए॒षः । नान॑दत् । रास॑भः । पत्वा᳚ । इति॑ ।  \newline




\markright{ TS 5.1.5.7  \hfill https://www.vedavms.in \hfill}
\addcontentsline{toc}{section}{ TS 5.1.5.7 }
\section*{ TS 5.1.5.7 }

\textbf{TS 5.1.5.7 } \newline
\textbf{Samhita Paata} \newline

-ह॒ रास॑भ॒ इति॒ ह्ये॑तमृष॒योऽव॑द॒न् भर॑न्न॒ग्निं पु॑री॒ष्य॑मित्या॑हा॒ऽग्निꣳ ह्ये॑ष भर॑ति॒ मा पा॒द्यायु॑षः पु॒रेत्या॒हाऽऽ*यु॑रे॒वाऽस्मि॑न् दधाति॒ तस्मा᳚द् गर्द॒भः सर्व॒मायु॑रेति॒ तस्मा᳚द् गर्द॒भे पु॒राऽऽयु॑षः॒ प्रमी॑ते बिभ्यति॒ वृषा॒ऽग्निं ॅवृष॑णं॒ भर॒न्नित्या॑ह॒ वृषा॒ ह्ये॑ष वृषा॒ऽग्निर॒पां गर्भꣳ॑ - [  ] \newline

\textbf{Pada Paata} \newline

आ॒ह॒ । रास॑भः । इति॑ । हि । ए॒तम् । ऋष॑यः । अव॑दन्न् । भरन्न्॑ । अ॒ग्निम् । पु॒री॒ष्य᳚म् । इति॑ । आ॒ह॒ । अ॒ग्निम् । हि । ए॒षः । भर॑ति । मा । पा॒दि॒ । आयु॑षः । पु॒रा । इति॑ । आ॒ह॒ । आयुः॑ । ए॒व । अ॒स्मि॒न्न् । द॒धा॒ति॒ । तस्मा᳚त् । ग॒र्द॒भः । सर्व᳚म् । आयुः॑ । ए॒ति॒ । तस्मा᳚त् । ग॒र्द॒भे । पु॒रा । आयु॑षः । प्रमी॑त॒ इति॒ प्र-मी॒ते॒ । बि॒भ्य॒ति॒ । वृषा᳚ । अ॒ग्निम् । वृष॑णम् । भरन्न्॑ । इति॑ । आ॒ह॒ । वृषा᳚ । हि । ए॒षः । वृषा᳚ । अ॒ग्निः । अ॒पाम् । गर्भ᳚म् ।  \newline




\markright{ TS 5.1.5.8  \hfill https://www.vedavms.in \hfill}
\addcontentsline{toc}{section}{ TS 5.1.5.8 }
\section*{ TS 5.1.5.8 }

\textbf{TS 5.1.5.8 } \newline
\textbf{Samhita Paata} \newline

समु॒द्रिय॒-मित्या॑हा॒ऽपाꣳ ह्ये॑ष गर्भो॒ यद॒ग्निरग्न॒ आ या॑हि वी॒तय॒ इति॒ वा इ॒मौ लो॒कौ व्यै॑ता॒मग्न॒ आ या॑हि वी॒तय॒ इति॒ यदाहा॒ ऽनयो᳚र्लो॒कयो॒-र्वीत्यै॒ प्रच्यु॑तो॒ वा ए॒ष आ॒यत॑ना॒दग॑तः प्रति॒ष्ठाꣳ स ए॒तर्.ह्य॑द्ध्व॒र्युं च॒ यज॑मानं च द्ध्यायत्यृ॒तꣳ स॒त्यमित्या॑हे॒यं ॅवा ऋ॒तम॒सौ - [  ] \newline

\textbf{Pada Paata} \newline

स॒मु॒द्रिय᳚म् । इति॑ । आ॒ह॒ । अ॒पाम् । हि । ए॒षः । गर्भः॑ । यत् । अ॒ग्निः । अग्ने᳚ । एति॑ । या॒हि॒ । वी॒तये᳚ । इति॑ । वै । इ॒मौ । लो॒कौ । वीति॑ । ऐ॒ता॒म् । अग्ने᳚ । एति॑ । या॒हि॒ । वी॒तये᳚ । इति॑ । यत् । आह॑ । अ॒नयोः᳚ । लो॒कयोः᳚ । वीत्या॒ इति॒ वि - इ॒त्यै॒ । प्रच्यु॑त॒ इति॒ प्र - च्यु॒तः॒ । वै । ए॒षः । आ॒यत॑ना॒दित्या᳚ - यत॑नात् । अग॑तः । प्र॒ति॒ष्ठामिति॑ प्रति - स्थाम् । सः । ए॒तर्.हि॑ । अ॒द्ध्व॒र्युम् । च॒ । यज॑मानम् । च॒ । ध्या॒य॒ति॒ । ऋ॒तम् । स॒त्यम् । इति॑ । आ॒ह॒ । इ॒यम् । वै । ऋ॒तम् । अ॒सौ ।  \newline




\markright{ TS 5.1.5.9  \hfill https://www.vedavms.in \hfill}
\addcontentsline{toc}{section}{ TS 5.1.5.9 }
\section*{ TS 5.1.5.9 }

\textbf{TS 5.1.5.9 } \newline
\textbf{Samhita Paata} \newline

स॒त्यम॒नयो॑रे॒वैनं॒ प्रति॑ ष्ठापयति॒ नाऽऽ*र्ति॒मार्च्छ॑त्यद्ध्व॒र्युर्न यज॑मानो॒ वरु॑णो॒ वा ए॒ष यज॑मानम॒भ्यैति॒ यद॒ग्निरुप॑नद्ध॒ ओष॑धयः॒ प्रति॑ गृह्णीता॒ग्निमे॒त-मित्या॑ह॒ शान्त्यै॒ व्यस्य॒न् विश्वा॒ अम॑ती॒ररा॑ती॒-रित्या॑ह॒ रक्ष॑सा॒मप॑हत्यै नि॒षीद॑न् नो॒ अप॑ दुर्म॒तिꣳ ह॑न॒दित्या॑ह॒ प्रति॑ष्ठित्या॒ ओष॑धयः॒ प्रति॑मोदद्ध्व - [  ] \newline

\textbf{Pada Paata} \newline

स॒त्यम् । अ॒नयोः᳚ । ए॒व । ए॒न॒म् । प्रतीति॑ । स्था॒प॒य॒ति॒ । न । आर्ति᳚म् । एति॑ । ऋ॒च्छ॒ति॒ । अ॒द्ध्व॒र्युः । न । यज॑मानः । वरु॑णः । वै । ए॒षः । यज॑मानम् । अ॒भि । एति॑ । ए॒ति॒ । यत् । अ॒ग्निः । उप॑नद्ध॒ इत्युप॑ - न॒द्धः॒ । ओष॑धयः । प्रतीति॑ । गृ॒ह्णी॒त॒ । अ॒ग्निम् । ए॒तम् । इति॑ । आ॒ह॒ । शान्त्यै᳚ । व्यस्य॒न्निति॑ वि - अस्यन्न्॑ । विश्वाः᳚ । अम॑तीः । अरा॑तीः । इति॑ । आ॒ह॒ । रक्ष॑साम् । अप॑हत्या॒ इत्यप॑ - ह॒त्यै॒ । नि॒षीद॒न्निति॑ नि-सीदन्न्॑ । नः॒ । अपेति॑ । दु॒र्म॒तिमिति॑ दुः - म॒तिम् । ह॒न॒त् । इति॑ । आ॒ह॒ । प्रति॑ष्ठित्या॒ इति॒ प्रति॑ - स्थि॒त्यै॒ । ओष॑धयः । प्रतीति॑ । मो॒द॒द्ध्व॒म् ।  \newline




\markright{ TS 5.1.5.10  \hfill https://www.vedavms.in \hfill}
\addcontentsline{toc}{section}{ TS 5.1.5.10 }
\section*{ TS 5.1.5.10 }

\textbf{TS 5.1.5.10 } \newline
\textbf{Samhita Paata} \newline

मेन॒मित्या॒हौष॑धयो॒ वा अ॒ग्नेर्भा॑ग॒धेयं॒ ताभि॑रे॒वैनꣳ॒॒ सम॑र्द्धयति॒ पुष्पा॑वतीः सुपिप्प॒ला इत्या॑ह॒ तस्मा॒दोष॑धयः॒ फलं॑ गृह्णन्त्य॒ यं ॅवो॒ गर्भ॑ ऋ॒त्वियः॑ प्र॒त्नꣳ स॒धस्थ॒मा-ऽस॑द॒दित्या॑ह॒ याभ्य॑ ए॒वैनं॑ प्रच्या॒वय॑ति॒ तास्वे॒वैनं॒ प्रति॑ष्ठापयति॒ द्वाभ्या॑मु॒पाव॑हरति॒ प्रति॑ष्ठित्यै ॥ \newline

\textbf{Pada Paata} \newline

ए॒न॒म् । इति॑ । आ॒ह॒ । ओष॑धयः । वै । अ॒ग्नेः । भा॒ग॒धेय॒मिति॑ भाग - धेय᳚म् । ताभिः॑ । ए॒व । ए॒न॒म् । समिति॑ । अ॒द्‌र्ध॒य॒ति॒ । पुष्पा॑वती॒रिति॒ पुष्प॑ - व॒तीः॒ । सु॒पि॒प्प॒ला इति॑ सु-पि॒प्प॒लाः । इति॑ । आ॒ह॒ । तस्मा᳚त् । ओष॑धयः । फल᳚म् । गृ॒ह्ण॒न्ति॒ । अ॒यम् । वः॒ । गर्भः॑ । ऋ॒त्वियः॑ । प्र॒त्नम् । स॒धस्थ॒मिति॑ स॒ध - स्थ॒म् । एति॑ । अ॒स॒द॒त् । इति॑ । आ॒ह॒ । याभ्यः॑ । ए॒व । ए॒न॒म् । प्र॒च्या॒वय॒तीति॑ प्र - च्या॒वय॑ति । तासु॑ । ए॒व । ए॒न॒म् । प्रतीति॑ । स्था॒प॒य॒ति॒ । द्वाभ्या᳚म् । उ॒पाव॑हर॒तीत्यु॑प - अव॑हरति । प्रति॑ष्ठित्या॒ इति॒ प्रति॑ - स्थि॒त्यै॒ ॥  \newline




\markright{ TS 5.1.6.1  \hfill https://www.vedavms.in \hfill}
\addcontentsline{toc}{section}{ TS 5.1.6.1 }
\section*{ TS 5.1.6.1 }

\textbf{TS 5.1.6.1 } \newline
\textbf{Samhita Paata} \newline

वा॒रु॒णो वा अ॒ग्निरुप॑नद्धो॒ वि पाज॒सेति॒ विस्रꣳ॑सयति सवि॒तृप्र॑सूत ए॒वास्य॒ विषू॑चीं ॅवरुणमे॒निं ॅविसृ॑जत्य॒प उप॑ सृज॒त्यापो॒ वै शा॒न्ताः शा॒न्ताभि॑रे॒वास्य॒ शुचꣳ॑ शमयति ति॒सृभि॒रुप॑ सृजति त्रि॒वृद्वा अ॒ग्निर्यावा॑ने॒वा-ग्निस्तस्य॒ शुचꣳ॑ शमयति मि॒त्रः सꣳ॒॒सृज्य॑ पृथि॒वीमित्या॑ह मि॒त्रो वै शि॒वो दे॒वानां॒ तेनै॒वै - [  ] \newline

\textbf{Pada Paata} \newline

वा॒रु॒णः । वै । अ॒ग्निः । उप॑नद्ध॒ इत्युप॑ - न॒द्धः॒ । वीति॑ । पाज॑सा । इति॑ । वीति॑ । स्रꣳ॒॒स॒य॒ति॒ । स॒वि॒तृप्र॑सूत॒ इति॑ सवि॒तृ - प्र॒सू॒तः॒ । ए॒व । अ॒स्य॒ । विषू॑चीम् । व॒रु॒ण॒मे॒निमिति॑ वरुण - मे॒निम् । वीति॑ । सृ॒ज॒ति॒ । अ॒पः । उपेति॑ । सृ॒ज॒ति॒ । आपः॑ । वै । शा॒न्ताः । शा॒न्ताभिः॑ । ए॒व । अ॒स्य॒ । शुच᳚म् । श॒म॒य॒ति॒ । ति॒सृभि॒रिति॑ ति॒सृ - भिः॒ । उपेति॑ । सृ॒ज॒ति॒ । त्रि॒वृदिति॑ त्रि -वृत् । वै । अ॒ग्निः । यावान्॑ । ए॒व । अ॒ग्निः । तस्य॑ । शुच᳚म् । श॒म॒य॒ति॒ । मि॒त्रः । सꣳ॒॒सृज्येति॑ सं - सृज्य॑ । पृ॒थि॒वीम् । इति॑ । आ॒ह॒ । मि॒त्रः । वै । शि॒वः । दे॒वाना᳚म् । तेन॑ । ए॒व ।  \newline




\markright{ TS 5.1.6.2  \hfill https://www.vedavms.in \hfill}
\addcontentsline{toc}{section}{ TS 5.1.6.2 }
\section*{ TS 5.1.6.2 }

\textbf{TS 5.1.6.2 } \newline
\textbf{Samhita Paata} \newline

-नꣳ॒॒ सꣳ सृ॑जति॒ शान्त्यै॒ यद्ग्रा॒म्याणां॒ पात्रा॑णां क॒पालैः᳚ सꣳसृ॒जेद्-ग्रा॒म्याणि॒ पात्रा॑णि शु॒चाऽर्प॑येदर्मकपा॒लैः सꣳ सृ॑जत्ये॒तानि॒ वा अ॑नुपजीवनी॒यानि॒ तान्ये॒व शु॒चाऽर्प॑यति॒ शर्क॑राभिः॒ सꣳ सृ॑जति॒ धृत्या॒ अथो॑ श॒त्वांया॑ जलो॒मैः सꣳ सृ॑जत्ये॒षा वा अ॒ग्नेः प्रि॒या त॒नूर्यद॒जा प्रि॒ययै॒वैनं॑ त॒नुवा॒ सꣳ सृ॑ज॒त्यथो॒ तेज॑सा कृष्णाजि॒नस्य॒ लोम॑भिः॒ सꣳ - [  ] \newline

\textbf{Pada Paata} \newline

ए॒न॒म् । समिति॑ । सृ॒ज॒ति॒ । शान्त्यै᳚ । यत् । ग्रा॒म्याणा᳚म् । पात्रा॑णाम् । क॒पालैः᳚ । सꣳ॒॒सृ॒जेदिति॑ सं - सृ॒जेत् । ग्रा॒म्याणि॑ । पात्रा॑णि । शु॒चा । अ॒र्प॒ये॒त् । अ॒र्म॒क॒पा॒लैरित्य॑र्म-क॒पा॒लैः । समिति॑ । सृ॒ज॒ति॒ । ए॒तानि॑ । वै । अ॒नु॒प॒जी॒व॒नी॒यानीत्य॑नुप - जी॒व॒नी॒यानि॑ । तानि॑ । ए॒व । शु॒चा । अ॒र्प॒य॒ति॒ । शर्क॑राभिः । समिति॑ । सृ॒ज॒ति॒ । धृत्यै᳚ । अथो॒ इति॑ । श॒न्त्वायेति॑ शं - त्वाय॑ । अ॒ज॒लो॒मैरित्य॑ज-लो॒मैः । समिति॑ । सृ॒ज॒ति॒ । ए॒षा । वै । अ॒ग्नेः । प्रि॒या । त॒नूः । यत् । अ॒जा । प्रि॒यया᳚ । ए॒व । ए॒न॒म् । त॒नुवा᳚ । समिति॑ । सृ॒ज॒ति॒ । अथो॒ इति॑ । तेज॑सा । कृ॒ष्णा॒जि॒नस्येति॑ कृष्ण - अ॒जि॒नस्य॑ । लोम॑भि॒रिति॒ लोम॑ - भिः॒ । समिति॑ ।  \newline




\markright{ TS 5.1.6.3  \hfill https://www.vedavms.in \hfill}
\addcontentsline{toc}{section}{ TS 5.1.6.3 }
\section*{ TS 5.1.6.3 }

\textbf{TS 5.1.6.3 } \newline
\textbf{Samhita Paata} \newline

सृ॑जति य॒ज्ञो वै कृ॑ष्णाजि॒नं ॅय॒ज्ञेनै॒व य॒ज्ञ्ꣳ सꣳ सृ॑जति रु॒द्राः स॒भृंत्य॑ पृथि॒वीमित्या॑है॒ता वा ए॒तं दे॒वता॒ अग्रे॒ सम॑भर॒न् ताभि॑रे॒वैनꣳ॒॒ संभ॑रति म॒खस्य॒ शिरो॒ऽसीत्या॑ह य॒ज्ञो वै म॒खस्तस्यै॒त-च्छिरो॒ यदु॒खा तस्मा॑दे॒वमा॑ह य॒ज्ञ्स्य॑ प॒दे स्थ॒ इत्या॑ह य॒ज्ञ्स्य॒ ह्ये॑ते - [  ] \newline

\textbf{Pada Paata} \newline

सृ॒ज॒ति॒ । य॒ज्ञ्ः । वै । कृ॒ष्णा॒जि॒नमिति॑ कृष्ण - अ॒जि॒नम् । य॒ज्ञेन॑ । ए॒व । य॒ज्ञ्म् । समिति॑ । सृ॒ज॒ति॒ । रु॒द्राः । स॒भृंत्येति॑ सं - भृत्य॑ । पृ॒थि॒वीम् । इति॑ । आ॒ह॒ । ए॒ताः । वै । ए॒तम् । दे॒वताः᳚ । अग्रे᳚ । समिति॑ । अ॒भ॒र॒न्न् । ताभिः॑ । ए॒व । ए॒न॒म् । समिति॑ । भ॒र॒ति॒ । म॒खस्य॑ । शिरः॑ । अ॒सि॒ । इति॑ । आ॒ह॒ । य॒ज्ञ्ः । वै । म॒खः । तस्य॑ । ए॒तत् । शिरः॑ । यत् । उ॒खा । तस्मा᳚त् । ए॒वम् । आ॒ह॒ । य॒ज्ञ्स्य॑ । प॒दे इति॑ । स्थः॒ । इति॑ । आ॒ह॒ । य॒ज्ञ्स्य॑ । हि । ए॒ते इति॑ ।  \newline




\markright{ TS 5.1.6.4  \hfill https://www.vedavms.in \hfill}
\addcontentsline{toc}{section}{ TS 5.1.6.4 }
\section*{ TS 5.1.6.4 }

\textbf{TS 5.1.6.4 } \newline
\textbf{Samhita Paata} \newline

प॒दे अथो॒ प्रति॑ष्ठित्यै॒ प्रान्याभि॒-र्यच्छ॒त्यन्व॒न्यै-र्म॑न्त्रयते मिथुन॒त्वाय॒ त्र्यु॑द्धिं करोति॒ त्रय॑ इ॒मे लो॒का ए॒षां ॅलो॒काना॒माप्त्यै॒ छन्दो॑भिः करोति वी॒र्यं॑ ॅवै छन्दाꣳ॑सि वी॒र्ये॑णै॒वैनां᳚ करोति॒ यजु॑षा॒ बिलं॑ करोति॒ व्यावृ॑त्त्या॒ इय॑तीं करोति प्र॒जाप॑तिना यज्ञ्मु॒खेन॒ संमि॑तां द्विस्त॒नां क॑रोति॒ यावा॑पृथि॒व्योर्दोहा॑य॒ चतुः॑स्तनां करोति पशू॒नां दोहा॑या॒ष्टास्त॑नां करोति॒ छन्द॑सां॒ दोहा॑य॒ नवा᳚श्रि-मभि॒चर॑तः ( ) कुर्यात् त्रि॒वृत॑मे॒व वज्रꣳ॑ स॒भृंत्य॒ भ्रातृ॑व्याय॒ प्रह॑रति॒ स्तृत्यै॑ कृ॒त्वाय॒ सा म॒हीमु॒खामिति॒ नि द॑धाति दे॒वता᳚स्वे॒वैनां॒ प्रति॑ष्ठापयति ॥ \newline

\textbf{Pada Paata} \newline

प॒दे इति॑ । अथो॒ इति॑ । प्रति॑ष्ठित्या॒ इति॒ प्रति॑ - स्थि॒त्यै॒ । प्रेति॑ । अ॒न्याभिः॑ । यच्छ॑ति । अन्विति॑ । अ॒न्यैः । म॒न्त्र॒य॒ते॒ । मि॒थु॒न॒त्वायेति॑ मिथुन - त्वाय॑ । त्र्यु॑द्धि॒मिति॒ त्रि - उ॒द्धि॒म् । क॒रो॒ति॒ । त्रयः॑ । इ॒मे । लो॒काः । ए॒षाम् । लो॒काना᳚म् । आप्त्यै᳚ । छन्दो॑भि॒रिति॒ छन्दः॑-भिः॒ । क॒रो॒ति॒ । वी॒र्य᳚म् । वै । छन्दाꣳ॑सि । वी॒र्ये॑ण । ए॒व । ए॒ना॒म् । क॒रो॒ति॒ । यजु॑षा । बिल᳚म् । क॒रो॒ति॒ । व्यावृ॑त्त्या॒ इति॑ वि - आवृ॑त्त्यै । इय॑तीम् । क॒रो॒ति॒ । प्र॒जाप॑ति॒नेति॑ प्र॒जा - प॒ति॒ना॒ । य॒ज्ञ्॒मु॒खेनेति॑ यज्ञ्-मु॒खेन॑ । संमि॑ता॒मिति॒ सं - मि॒ता॒म् । द्वि॒स्त॒नामिति॑ द्वि - स्त॒नाम् । क॒रो॒ति॒ । द्यावा॑पृथि॒व्योरिति॒ द्यावा᳚ - पृ॒थि॒व्योः । दोहा॑य । चतुः॑स्तना॒मिति॒ चतुः॑ - स्त॒ना॒म् । क॒रो॒ति॒ । प॒शू॒नाम् । दोहा॑य । अ॒ष्टास्त॑ना॒मित्य॒ष्टा - स्त॒ना॒म् । क॒रो॒ति॒ । छन्द॑साम् । दोहा॑य । नवा᳚श्रि॒मिति॒ नव॑ - अ॒श्रि॒म् । अ॒भि॒चर॑त॒ इत्य॑भि - चर॑तः ( ) । कु॒र्या॒त् । त्रि॒वृत॒मिति॑ त्रि - वृत᳚म् । ए॒व । वज्र᳚म् । स॒भृंत्येति॑ सं - भृत्य॑ । भ्रातृ॑व्याय । प्रेति॑ । ह॒र॒ति॒ । स्तृत्यै᳚ । कृ॒त्वाय॑ । सा । म॒हीम् । उ॒खाम् । इति॑ । नीति॑ । द॒धा॒ति॒ । दे॒वता॑सु । ए॒व । ए॒ना॒म् । प्रतीति॑ । स्था॒प॒य॒ति॒ ॥  \newline




\markright{ TS 5.1.7.1  \hfill https://www.vedavms.in \hfill}
\addcontentsline{toc}{section}{ TS 5.1.7.1 }
\section*{ TS 5.1.7.1 }

\textbf{TS 5.1.7.1 } \newline
\textbf{Samhita Paata} \newline

स॒प्तभि॑र्द्धूपयति स॒प्त वै शी॑र्.ष॒ण्याः᳚ प्रा॒णाः शिर॑ ए॒तद्-य॒ज्ञ्स्य॒ यदु॒खा शी॒र्.॒षन्ने॒व य॒ज्ञ्स्य॑ प्रा॒णान् द॑धाति॒ तस्मा᳚थ् स॒प्त शी॒र्.॒षन् प्रा॒णा अ॑श्वश॒केन॑ धूपयति प्राजाप॒त्यो वा अश्वः॑ सयोनि॒त्वाया-दि॑ति॒स्त्वेत्या॑हे॒यं ॅवा अदि॑ति॒रदि॑त्यै॒वादि॑त्यां खनत्य॒स्या अक्रू॑रंकाराय॒ न हि स्वः स्वꣳ हि॒नस्ति॑ दे॒वानां᳚ त्वा॒ पत्नी॒रित्या॑ह दे॒वानां॒ - [  ] \newline

\textbf{Pada Paata} \newline

स॒प्तभि॒रिति॑ स॒प्त - भिः॒ । धू॒प॒य॒ति॒ । स॒प्त । वै । शी॒र्॒.ष॒ण्याः᳚ । प्रा॒णा इति॑ प्र - अ॒नाः । शिरः॑ । ए॒तत् । य॒ज्ञ्स्य॑ । यत् । उ॒खा । शी॒र्॒.षन्न् । ए॒व । य॒ज्ञ्स्य॑ । प्रा॒णानिति॑ प्र - अ॒नान् । द॒धा॒ति॒ । तस्मा᳚त् । स॒प्त । शी॒र॒.षन्न् । प्रा॒णा इति॑ प्र - अ॒नाः । अ॒श्व॒श॒केनेत्य॑श्व-श॒केन॑ । धू॒प॒य॒ति॒ । प्रा॒जा॒प॒त्य इति॑ प्राजा-प॒त्यः । वै । अश्वः॑ । स॒यो॒नि॒त्वायेति॑ सयोनि-त्वाय॑ । अदि॑तिः । त्वा॒ । इति॑ । आ॒ह॒ । इ॒यम् । वै । अदि॑तिः । अदि॑त्या । ए॒व । अदि॑त्याम् । ख॒न॒ति॒ । अ॒स्याः । अक्रू॑रंकारा॒येत्यक्रू॑रं - का॒रा॒य॒ । न । हि । स्वः । स्वम् । हि॒नस्ति॑ । दे॒वाना᳚म् । त्वा॒ । पत्नीः᳚ । इति॑ । आ॒ह॒ । दे॒वाना᳚म् ।  \newline




\markright{ TS 5.1.7.2  \hfill https://www.vedavms.in \hfill}
\addcontentsline{toc}{section}{ TS 5.1.7.2 }
\section*{ TS 5.1.7.2 }

\textbf{TS 5.1.7.2 } \newline
\textbf{Samhita Paata} \newline

ॅवा ए॒तां पत्न॒योऽग्रे॑ऽकुर्व॒न् ताभि॑रे॒वैनां᳚ दधाति धि॒षणा॒स्त्वेत्या॑ह वि॒द्या वै धि॒षणा॑ वि॒द्याभि॑रे॒वैना॑म॒भीन्धे॒ ग्नास्त्वेत्या॑ह॒ छन्दाꣳ॑सि॒ वै ग्ना श्छन्दो॑भिरे॒वैनाꣳ॑ श्रपयति॒ वरू᳚त्रय॒स्त्त्वेत्या॑ह॒ होत्रा॒ वै वरू᳚त्रयो॒ होत्रा॑भिरे॒वैनां᳚ पचति॒ जन॑य॒स्त्वेत्या॑ह दे॒वानां॒ ॅवै पत्नी॒ - [  ] \newline

\textbf{Pada Paata} \newline

वै । ए॒ताम् । पत्न॑यः । अग्रे᳚ । अ॒कु॒र्व॒न्न् । ताभिः॑ । ए॒व । ए॒ना॒म् । द॒धा॒ति॒ । धि॒षणाः᳚ । त्वा॒ । इति॑ । आ॒ह॒ । वि॒द्याः । वै । धि॒षणाः᳚ । वि॒द्याभिः॑ । ए॒व । ए॒ना॒म् । अ॒भीति॑ । इ॒न्धे॒ । ग्नाः । त्वा॒ । इति॑ । आ॒ह॒ । छन्दाꣳ॑सि । वै । ग्नाः । छन्दो॑भि॒रिति॒ छन्दः॑ - भिः॒ । ए॒व । ए॒ना॒म् । श्र॒प॒य॒ति॒ । वरू᳚त्रयः । त्वा॒ । इति॑ । आ॒ह॒ । होत्राः᳚ । वै । वरू᳚त्रयः । होत्रा॑भिः । ए॒व । ए॒ना॒म् । प॒च॒ति॒ । जन॑यः । त्वा॒ । इति॑ । आ॒ह॒ । दे॒वाना᳚म् । वै । पत्नीः᳚ ।  \newline




\markright{ TS 5.1.7.3  \hfill https://www.vedavms.in \hfill}
\addcontentsline{toc}{section}{ TS 5.1.7.3 }
\section*{ TS 5.1.7.3 }

\textbf{TS 5.1.7.3 } \newline
\textbf{Samhita Paata} \newline

-र्जन॑य॒स्ताभि॑रे॒वैनां᳚ पचति ष॒ड्भिः प॑चति॒ षड्वा ऋ॒तव॑ ऋ॒तुभि॑रे॒वैनां᳚ पचति॒ द्विः पच॒न्त्वित्या॑ह॒ तस्मा॒द् द्विः सं॑ॅवथ्स॒रस्य॑ स॒स्यं प॑च्यते वारु॒ण्यु॑खाऽभीद्धा॑ मै॒त्रियोपै॑ति॒ शान्त्यै॑ दे॒वस्त्वा॑ सवि॒तोद्-व॑प॒त्वित्या॑ह सवि॒तृप्र॑सूत ए॒वैनां॒ ब्रह्म॑णा दे॒वता॑भि॒रुद्-व॑प॒त्यप॑द्यमाना पृथि॒व्याशा॒ दिश॒ आ पृ॒णे - [  ] \newline

\textbf{Pada Paata} \newline

जन॑यः । ताभिः॑ । ए॒व । ए॒ना॒म् । प॒च॒ति॒ । ष॒ड्भिरिति॑ षट् - भिः । प॒च॒ति॒ । षट् । वै । ऋ॒तवः॑ । ऋ॒तुभि॒रित्यृ॒तु - भिः॒ । ए॒व । ए॒ना॒म् । प॒च॒ति॒ । द्विः । पच॑न्तु । इति॑ । आ॒ह॒ । तस्मा᳚त् । द्विः । सं॒ॅव॒थ्स॒रस्येति॑ सं-व॒थ्स॒रस्य॑ । स॒स्यम् । प॒च्य॒ते॒ । वा॒रु॒णी । उ॒खा । अ॒भीद्धेत्य॒भि-इ॒द्धा॒ । मै॒त्रिया᳚ । उपेति॑ । ए॒ति॒ । शान्त्यै᳚ । दे॒वः । त्वा॒ । स॒वि॒ता । उदिति॑ । व॒प॒तु॒ । इति॑ । आ॒ह॒ । स॒वि॒तृप्र॑सूत॒ इति॑ सवि॒तृ - प्र॒सू॒तः॒ । ए॒व । ए॒ना॒म् । ब्रह्म॑णा । दे॒वता॑भिः । उदिति॑ । व॒प॒ति॒ । अप॑द्यमाना । पृ॒थि॒वि॒ । आशाः᳚ । दिशः॑ । एति॑ । पृ॒ण॒ ।  \newline




\markright{ TS 5.1.7.4  \hfill https://www.vedavms.in \hfill}
\addcontentsline{toc}{section}{ TS 5.1.7.4 }
\section*{ TS 5.1.7.4 }

\textbf{TS 5.1.7.4 } \newline
\textbf{Samhita Paata} \newline

-त्या॑ह॒ तस्मा॑द॒ग्निः सर्वा॒ दिशोऽनु॒ विभा॒त्युत्ति॑ष्ठ बृह॒ती भ॑वो॒र्द्ध्वा ति॑ष्ठ ध्रु॒वा त्वमित्या॑ह॒ प्रति॑ष्ठित्या असु॒र्यं॑ पात्र॒मना᳚च्छृण्ण॒मा-च्छृ॑णत्ति देव॒त्रा-ऽक॑रजक्षी॒रेणा-ऽऽ*च्छृ॑णत्ति पर॒मं ॅवा ए॒तत् पयो॒ यद॑जक्षी॒रं प॑र॒मेणै॒वैनां॒ पय॒साऽऽच्छृ॑णत्ति॒ यजु॑षा॒ व्यावृ॑त्त्यै॒ छन्दो॑भि॒रा च्छृ॑णत्ति॒ छन्दो॑भि॒र्वा ए॒षा ( ) क्रि॑यते॒ छन्दो॑भिरे॒व छन्दाꣳ॒॒स्या च्छृ॑णत्ति ॥ \newline

\textbf{Pada Paata} \newline

इति॑ । आ॒ह॒ । तस्मा᳚त् । अ॒ग्निः । सर्वाः᳚ । दिशः॑ । अनु॑ । वीति॑ । भा॒ति॒ । उदिति॑ । ति॒ष्ठ॒ । बृ॒ह॒ती । भ॒व॒ । ऊ॒र्ध्वा । ति॒ष्ठ॒ । ध्रु॒वा । त्वम् । इति॑ । आ॒ह॒ । प्रति॑ष्ठित्या॒ इति॒ प्रति॑ - स्थि॒त्यै॒ । अ॒सु॒र्य᳚म् । पात्र᳚म् । अना᳚च्छृण्ण॒मित्यना᳚ - छृ॒ण्ण॒म् । एति॑ । छृ॒ण॒त्ति॒ । दे॒व॒त्रेति॑ देव-त्रा । अ॒कः॒ । अ॒ज॒क्षी॒रेणेत्य॑ज - क्षी॒रेण॑ । एति॑ । छृ॒ण॒त्ति॒ । प॒र॒मम् । वै । ए॒तत् । पयः॑ । यत् । अ॒ज॒क्षी॒रमित्य॑ज - क्षी॒रम् । प॒र॒मेण॑ । ए॒व । ए॒ना॒म् । पय॑सा । एति॑ । छृ॒ण॒त्ति॒ । यजु॑षा । व्यावृ॑त्त्या॒ इति॑ वि - आवृ॑त्त्यै । छन्दो॑भि॒रिति॒ छन्दः॑ - भिः॒ । एति॑ । छृ॒ण॒त्ति॒ । छन्दो॑भि॒रिति॒ छन्दः॑ - भिः॒ । वै । ए॒षा ( ) । क्रि॒य॒ते॒ । छन्दो॑भि॒रिति॒ छन्दः॑ - भिः॒ । ए॒व । छन्दाꣳ॑सि । एति॑ । छृ॒ण॒त्ति॒ ॥  \newline




\markright{ TS 5.1.8.1  \hfill https://www.vedavms.in \hfill}
\addcontentsline{toc}{section}{ TS 5.1.8.1 }
\section*{ TS 5.1.8.1 }

\textbf{TS 5.1.8.1 } \newline
\textbf{Samhita Paata} \newline

एक॑विꣳशत्या॒ माषैः᳚ पुरुषशी॒र्॒.ष-मच्छै᳚त्यमे॒द्ध्या वै माषा॑ अमे॒द्ध्यं पु॑रुषशी॒र्॒.ष-म॑मे॒द्ध्यैरे॒वा-स्या॑-मे॒द्ध्यं नि॑रव॒दाय॒ मेद्ध्यं॑ कृ॒त्वा ऽऽह॑र॒त्येक॑विꣳशति-र्भवन्त्येकविꣳ॒॒शो वै पुरु॑षः॒ पुरु॑ष॒स्याऽऽ*प्त्यै॒ व्यृ॑द्धं॒ ॅवा ए॒तत् प्रा॒णैर॑मे॒द्ध्यं ॅयत् पु॑रुषशी॒र्॒.षꣳ स॑प्त॒धा वितृ॑ण्णां ॅवल्मीकव॒पां प्रति॒ नि द॑धाति स॒प्त वै शी॑र्.ष॒ण्याः᳚ प्रा॒णाः प्रा॒णैरे॒वैन॒थ्-सम॑र्द्धयति मेद्ध्य॒त्वाय॒ याव॑न्तो॒ - [  ] \newline

\textbf{Pada Paata} \newline

एक॑विꣳश॒त्येत्येक॑ - विꣳ॒॒श॒त्या॒ । माषैः᳚ । पु॒रु॒ष॒शी॒र॒.षमिति॑ पुरुष - शी॒र॒.षम् । अच्छ॑ । ए॒ति॒ । अ॒मे॒द्ध्याः । वै । माषाः᳚ । अ॒मे॒द्ध्यम् । पु॒रु॒ष॒शी॒र्॒.षमिति॑ पुरुष - शी॒र्॒.॒षम् । अ॒मे॒द्ध्यैः । ए॒व । अ॒स्य॒ । अ॒मे॒द्ध्यम् । नि॒र॒व॒दायेति॑ निः - अ॒व॒दाय॑ । मेद्ध्य᳚म् । कृ॒त्वा । एति॑ । ह॒र॒ति॒ । एक॑विꣳशति॒रित्येक॑ - विꣳ॒॒श॒तिः॒ । भ॒व॒न्ति॒ । ए॒क॒विꣳ॒॒श इत्ये॑क - विꣳ॒॒शः । वै । पुरु॑षः । पुरु॑षस्य । आप्त्यै᳚ । व्यृ॑द्ध॒मिति॒ वि - ऋ॒द्ध॒म् । वै । ए॒तत् । प्रा॒णैरिति॑ प्र - अ॒नैः । अ॒मे॒द्ध्यम् । यत् । पु॒रु॒ष॒शी॒र्॒.षमिति॑ पुरुष - शी॒र्॒.षम् । स॒प्त॒धेति॑ सप्त - धा । वितृ॑ण्णा॒मिति॒ वि - तृ॒ण्णा॒म् । व॒ल्मी॒क॒व॒पामिति॑ वल्मीक-व॒पाम् । प्रति॑ । नीति॑ । द॒धा॒ति॒ । स॒प्त । वै । शी॒र॒.ष॒ण्याः᳚ । प्रा॒णा इति॑ प्र - अ॒नाः । प्रा॒णैरिति॑ प्र - अ॒नैः । ए॒व । ए॒न॒त् । समिति॑ । अ॒द्‌र्ध॒य॒ति॒ । मे॒द्ध्य॒त्वायेति॑ मेद्ध्य - त्वाय॑ । याव॑न्तः ।  \newline




\markright{ TS 5.1.8.2  \hfill https://www.vedavms.in \hfill}
\addcontentsline{toc}{section}{ TS 5.1.8.2 }
\section*{ TS 5.1.8.2 }

\textbf{TS 5.1.8.2 } \newline
\textbf{Samhita Paata} \newline

वै मृ॒त्युब॑न्धव॒स्तेषां᳚ ॅय॒म आधि॑पत्यं॒ परी॑याय यमगा॒थाभिः॒ परि॑गायति य॒मादे॒वैन॑द्-वृङ्क्ते ति॒सृभिः॒ परि॑गायति॒ त्रय॑ इ॒मे लो॒का ए॒भ्य ए॒वैन॑ल्लो॒केभ्यो॑ वृङ्क्ते॒ तस्मा॒द्-गाय॑ते॒ न देयं॒ गाथा॒ हि तद्-वृ॒ङ्क्ते᳚ ऽग्निभ्यः॑ प॒शूना ल॑भते॒ कामा॒ वा अ॒ग्नयः॒ कामा॑ने॒वाव॑ रुन्धे॒ यत् प॒शून् नाऽऽ*लभे॒ताऽन॑वरुद्धा अस्य - [  ] \newline

\textbf{Pada Paata} \newline

वै । मृ॒त्युब॑न्धव॒ इति॑ मृ॒त्यु - ब॒न्ध॒वः॒ । तेषा᳚म् । य॒मः । आधि॑पत्य॒मित्याधि॑ - प॒त्य॒म् । परीति॑ । इ॒या॒य॒ । य॒म॒गा॒थाभि॒रिति॑ यम - गा॒थाभिः॑ । परीति॑ । गा॒य॒ति॒ । य॒मात् । ए॒व । ए॒न॒त् । वृ॒ङ्क्ते॒ । ति॒सृभि॒रिति॑ ति॒सृ - भिः॒ । परीति॑ । गा॒य॒ति॒ । त्रयः॑ । इ॒मे । लो॒काः । ए॒भ्यः । ए॒व । ए॒न॒त् । लो॒केभ्यः॑ । वृ॒ङ्क्ते॒ । तस्मा᳚त् । गाय॑ते । न । देय᳚म् । गाथा᳚ । हि । तत् । वृ॒ङ्क्ते॒ । अ॒ग्निभ्य॒ इत्य॒ग्नि-भ्यः॒ । प॒शून् । एति॑ । ल॒भ॒ते॒ । कामाः᳚ । वै । अ॒ग्नयः॑ । कामान्॑ । ए॒व । अवेति॑ । रु॒न्धे॒ । यत् । प॒शून् । न । आ॒लभे॒तेत्या᳚ - लभे॑त । अन॑वरुद्धा॒ इत्यन॑व - रु॒द्धाः॒ । अ॒स्य॒ ।  \newline




\markright{ TS 5.1.8.3  \hfill https://www.vedavms.in \hfill}
\addcontentsline{toc}{section}{ TS 5.1.8.3 }
\section*{ TS 5.1.8.3 }

\textbf{TS 5.1.8.3 } \newline
\textbf{Samhita Paata} \newline

प॒शवः॑ स्यु॒र्यत् पर्य॑ग्निकृतानुथ्-सृ॒जेद्-य॑ज्ञ्वेश॒सं कु॑र्या॒द्-यथ् सꣳ॑स्था॒पये᳚द्-या॒तया॑मानि शी॒र्॒.षाणि॑ स्यु॒र्यत् प॒शूना॒लभ॑ते॒ तेनै॒व प॒शूनव॑ रुन्धे॒ यत् पर्य॑ग्निकृतानुथ्-सृ॒जति॑ शी॒र्ष्णा-मया॑तयामत्वाय प्राजाप॒त्येन॒ सꣳ स्था॑पयति य॒ज्ञो वै प्र॒जाप॑तिर्य॒ज्ञ् ए॒व य॒ज्ञ्ं प्रति॑ष्ठापयति प्र॒जाप॑तिः प्र॒जा अ॑सृजत॒ स रि॑रिचा॒नो॑ऽमन्यत॒ स ए॒ता आ॒प्रीर॑पश्य॒त् ताभि॒र्वै स मु॑ख॒त - [  ] \newline

\textbf{Pada Paata} \newline

प॒शवः॑ । स्युः॒ । यत् । पर्य॑ग्निकृता॒निति॒ पर्य॑ग्नि - कृ॒ता॒न् । उ॒थ्सृ॒जेदित्यु॑त् - स॒जेत् । य॒ज्ञ्॒वे॒श॒समिति॑ यज्ञ्-वे॒श॒सम् । कु॒र्या॒त् । यत् । सꣳ॒॒स्था॒पये॒दिति॑ सं - स्था॒पये᳚त् । या॒तया॑मा॒नीति॑ या॒त - या॒मा॒नि॒ । शी॒र्॒.षाणि॑ । स्युः॒ । यत् । प॒शून् । आ॒लभ॑त॒ इत्या᳚ - लभ॑ते । तेन॑ । ए॒व । प॒शून् । अवेति॑ । रु॒न्धे॒ । यत् । पर्य॑ग्निकृता॒निति॒ पर्य॑ग्नि - कृ॒ता॒न् । उ॒थ्सृ॒जतीत्यु॑त्-सृ॒जति॑ । शी॒र्ष्णाम् । अया॑तयामत्वा॒येत्यया॑तयाम - त्वा॒य॒ । रा॒जा॒प॒त्येनेति॑ प्राजा - प॒त्येन॑ । समिति॑ । स्था॒प॒य॒ति॒ । य॒ज्ञ्ः । वै । प्र॒जाप॑ति॒रिति॑ प्र॒जा - प॒तिः॒ । य॒ज्ञे । ए॒व । य॒ज्ञ्म् । प्रतीति॑ । स्था॒प॒य॒ति॒ । प्र॒जाप॑ति॒रिति॑ प्र॒जा - प॒तिः॒ । प्र॒जा इति॑ प्र - जाः । अ॒सृ॒ज॒त॒ । सः । रि॒रि॒चा॒नः । अ॒म॒न्य॒त॒ । सः । ए॒ताः । आ॒प्रीरित्या᳚ - प्रीः । अ॒प॒श्य॒त् । ताभिः॑ । वै । सः । मु॒ख॒तः ।  \newline




\markright{ TS 5.1.8.4  \hfill https://www.vedavms.in \hfill}
\addcontentsline{toc}{section}{ TS 5.1.8.4 }
\section*{ TS 5.1.8.4 }

\textbf{TS 5.1.8.4 } \newline
\textbf{Samhita Paata} \newline

आ॒त्मान॒मा ऽप्री॑णीत॒ यदे॒ता आ॒प्रियो॒ भव॑न्ति य॒ज्ञो वै प्र॒जाप॑ति-र्य॒ज्ञ्मे॒वैताभि॑र्मुख॒त आ प्री॑णा॒त्य-प॑रिमितछन्दसो भव॒न्त्यप॑रिमितः प्र॒जाप॑तिः प्र॒जाप॑ते॒राप्त्या॑ ऊनातिरि॒क्ता मि॑थु॒नाः प्रजा᳚त्यै लोम॒शं ॅवै नामै॒तच्छन्दः॑ प्र॒जाप॑तेः प॒शवो॑ लोम॒शाः प॒शूने॒वाऽव॑ रुन्धे॒ सर्वा॑णि॒ वा ए॒ता रू॒पाणि॒ सर्वा॑णि रू॒पाण्य॒ग्नौ चित्ये᳚ क्रियन्ते॒ तस्मा॑दे॒ता अ॒ग्नेश्चित्य॑स्य - [  ] \newline

\textbf{Pada Paata} \newline

आ॒त्मान᳚म् । एति॑ । अ॒प्री॒णी॒त॒ । यत् । ए॒ताः । आ॒प्रिय॒ इत्या᳚-प्रियः॑ । भव॑न्ति । य॒ज्ञ्ः । वै । प्र॒जाप॑ति॒रिति॑ प्र॒जा - प॒तिः॒ । य॒ज्ञ्म् । ए॒व । ए॒ताभिः॑ । मु॒ख॒तः । एति॑ । प्री॒णा॒ति॒ । अप॑रिमितछन्दस॒ इत्यप॑रिमित - छ॒न्द॒सः॒ । भ॒व॒न्ति॒ । अप॑रिमित॒ इत्यप॑रि - मि॒तः॒ । प्र॒जाप॑ति॒रिति॑ प्र॒जा - प॒तिः॒ । प्र॒जाप॑ते॒रिति॑ प्र॒जा - प॒तेः॒ । आप्त्यै᳚ । ऊ॒ना॒ति॒रि॒क्ता इत्यू॑न - अ॒ति॒रि॒क्ताः । मि॒थु॒नाः । प्रजा᳚त्या॒ इति॒ प्र - जा॒त्यै॒ । लो॒म॒शम् । वै । नाम॑ । ए॒तत् । छन्दः॑ । प्र॒जाप॑ते॒रिति॑ प्र॒जा - प॒तेः॒ । प॒शवः॑ । लो॒म॒शाः । प॒शून् । ए॒व । अवेति॑ । रु॒न्धे॒ । सर्वा॑णि । वै । ए॒ताः । रू॒पाणि॑ । सर्वा॑णि । रू॒पाणि॑ । अ॒ग्नौ । चित्ये᳚ । क्रि॒य॒न्ते॒ । तस्मा᳚त् । ए॒ताः । अ॒ग्नेः । चित्य॑स्य ।  \newline




\markright{ TS 5.1.8.5  \hfill https://www.vedavms.in \hfill}
\addcontentsline{toc}{section}{ TS 5.1.8.5 }
\section*{ TS 5.1.8.5 }

\textbf{TS 5.1.8.5 } \newline
\textbf{Samhita Paata} \newline

भव॒न्त्ये क॑विꣳ शतिꣳ सामिधे॒नीरन्वा॑ह॒ रुग्वा ए॑कविꣳ॒॒शो रुच॑मे॒व ग॑च्छ॒त्यथो᳚ प्रति॒ष्ठामे॒व प्र॑ति॒ष्ठा ह्ये॑कविꣳ॒॒श-श्चतु॑र्विꣳशति॒मन्वा॑ह॒ चतु॑र्विꣳशतिरर्द्धमा॒साः सं॑ॅवथ्स॒रः सं॑ॅवथ्स॒रो᳚ऽग्निर्वै᳚श्वान॒रः सा॒क्षादे॒व वै᳚श्वान॒रमव॑ रुन्धे॒ परा॑ची॒रन्वा॑ह॒ परा॑ङिव॒ हि सु॑व॒र्गो लो॒कः समा᳚स्त्वाऽग्न ऋ॒तवो॑ वर्द्धय॒न्त्वित्या॑ह॒ समा॑भिरे॒वाऽग्निं ॅव॑र्द्धय - [  ] \newline

\textbf{Pada Paata} \newline

भ॒व॒न्ति॒ । एक॑विꣳशति॒मित्येक॑ - विꣳ॒॒श॒ति॒म् । सा॒मि॒धे॒नीरिति॑ सां - इ॒धे॒नीः । अन्विति॑ । आ॒ह॒ । रुक् । वै । ए॒क॒विꣳ॒॒श इत्ये॑क - विꣳ॒॒शः । रुच᳚म् । ए॒व । ग॒च्छ॒ति॒ । अथो॒ इति॑ । प्र॒ति॒ष्ठामिति॑ प्रति - स्थाम् । ए॒व । प्र॒ति॒ष्ठेति॑ प्रति - स्था । हि । ए॒क॒विꣳ॒॒श इत्ये॑क - विꣳ॒॒शः । चतु॑र्विꣳशति॒मिति॒ चतुः॑ - विꣳ॒॒श॒ति॒म् । अन्विति॑ । आ॒ह॒ । चतु॑र्विꣳशति॒रिति॒ चतुः॑ - विꣳ॒॒श॒तिः॒ । अ॒द्‌र्ध॒मा॒सा इत्य॑द्‌र्ध - मा॒साः । सं॒ॅव॒थ्स॒र इति॑ सं - व॒थ्स॒रः । सं॒ॅव॒थ्स॒र इति॑ सं - व॒थ्स॒रः । अ॒ग्निः । वै॒श्वा॒न॒रः । सा॒क्षादिति॑ स - अ॒क्षात् । ए॒व । वै॒श्वा॒न॒रम् । अवेति॑ । रु॒न्धे॒ । परा॑चीः । अन्विति॑ । आ॒ह॒ । पराङ्॑ । इ॒व॒ । हि । सु॒व॒र्ग इति॑ सुवः - गः । लो॒कः । समाः᳚ । त्वा॒ । अ॒ग्ने॒ । ऋ॒तवः॑ । व॒द्‌र्ध॒य॒न्तु॒ । इति॑ । आ॒ह॒ । समा॑भिः । ए॒व । अ॒ग्निम् । व॒द्‌र्ध॒य॒ति॒ ।  \newline




\markright{ TS 5.1.8.6  \hfill https://www.vedavms.in \hfill}
\addcontentsline{toc}{section}{ TS 5.1.8.6 }
\section*{ TS 5.1.8.6 }

\textbf{TS 5.1.8.6 } \newline
\textbf{Samhita Paata} \newline

त्यृ॒तुभिः॑ संॅवथ्स॒रं ॅविश्वा॒ आ भा॑हि प्र॒दिशः॑ पृथि॒व्या इत्या॑ह॒ तस्मा॑द॒ग्निः सर्वा॒ दिशोऽनु॒ विभा॑ति॒ प्रत्यौ॑हताम॒श्विना॑ मृ॒त्युम॑स्मा॒दित्या॑ह मृ॒त्युमे॒वाऽस्मा॒दप॑ नुद॒त्युद्व॒यं तम॑स॒स्परीत्या॑ह पा॒प्मा वै तमः॑ पा॒प्मान॑मे॒वास्मा॒दप॑ ह॒न्त्यग॑न्म॒ ज्योति॑रुत्त॒म-मित्या॑हा॒ऽसौ वा आ॑दि॒त्यो ( ) ज्योति॑रुत्त॒म-मा॑दि॒त्यस्यै॒व सायु॑ज्यं गच्छति॒ न सं॑ॅवथ्स॒रस्ति॑ष्ठति॒ नास्य॒ श्रीस्ति॑ष्ठति॒ यस्यै॒ताः क्रि॒यन्ते॒ ज्योति॑ष्मती-मुत्त॒मामन्वा॑ह॒ ज्योति॑रे॒वास्मा॑ उ॒परि॑ष्टाद् दधाति सुव॒र्गस्य॑ लो॒कस्यानु॑ख्यात्यै ॥ \newline

\textbf{Pada Paata} \newline

ऋ॒तुभि॒रित्यृ॒तु - भिः॒ । सं॒ॅव॒थ्स॒रमिति॑ सं - व॒थ्स॒रम् । विश्वाः᳚ । एति॑ । भा॒हि॒ । प्र॒दिश॒ इति॑ प्र - दिशः॑ । पृ॒थि॒व्याः । इति॑ । आ॒ह॒ । तस्मा᳚त् । अ॒ग्निः । सर्वाः᳚ । दिशः॑ । अनु॑ । वीति॑ । भा॒ति॒ । प्रतीति॑ । औ॒ह॒ता॒म् । अ॒श्विना᳚ । मृ॒त्युम् । अ॒स्मा॒त् । इति॑ । आ॒ह॒ । मृ॒त्युम् । ए॒व । अ॒स्मा॒त् । अपेति॑ । नु॒द॒ति॒ । उदिति॑ । व॒यम् । तम॑सः । परीति॑ । इति॑ । आ॒ह॒ । पा॒प्मा । वै । तमः॑ । पा॒प्मान᳚म् । ए॒व । अ॒स्मा॒त् । अपेति॑ । ह॒न्ति॒ । अग॑न्म । ज्योतिः॑ । उ॒त्त॒ममित्यु॑त् - त॒मम् । इति॑ । आ॒ह॒ । अ॒सौ । वै । आ॒दि॒त्यः ( ) । ज्योतिः॑ । उ॒त्त॒ममित्यु॑त्-त॒मम् । आ॒दि॒त्यस्य॑ । ए॒व । सायु॑ज्यम् । ग॒च्छ॒ति॒ । न । सं॒ॅव॒थ्स॒र इति॑ सं - व॒थ्स॒रः । ति॒ष्ठ॒ति॒ । न । अ॒स्य॒ । श्रीः । ति॒ष्ठ॒ति॒ । यस्य॑ । ए॒ताः । क्रि॒यन्ते᳚ । ज्योति॑ष्मतीम् । उ॒त्त॒मामित्यु॑त् - त॒माम् । अन्विति॑ । आ॒ह॒ । ज्योतिः॑ । ए॒व । अ॒स्मै॒ । उ॒परि॑ष्टात् । द॒धा॒ति॒ । सु॒व॒र्गस्येति॑ सुवः - गस्य॑ । लो॒कस्य॑ । अनु॑ख्यात्या॒ इत्यनु॑ - ख्या॒त्यै॒ ॥  \newline




\markright{ TS 5.1.9.1  \hfill https://www.vedavms.in \hfill}
\addcontentsline{toc}{section}{ TS 5.1.9.1 }
\section*{ TS 5.1.9.1 }

\textbf{TS 5.1.9.1 } \newline
\textbf{Samhita Paata} \newline

ष॒ड्भिर्दी᳚क्षयति॒ षड्वा ऋ॒तव॑ ऋ॒तुभि॑रे॒वैनं॑ दीक्षयति स॒प्तभि॑र्दीक्षयति स॒प्त छन्दाꣳ॑सि॒ छन्दो॑भिरे॒वैनं॑ दीक्षयति॒ विश्वे॑ दे॒वस्य॑ ने॒तुरित्य॑-नु॒ष्टुभो᳚त्त॒मया॑ जुहोति॒ वाग्वा अ॑नु॒ष्टुप् तस्मा᳚त् प्रा॒णानां॒ ॅवागु॑त्त॒मै- क॑स्माद॒क्षरा॒दना᳚प्तं प्रथ॒मं प॒दं तस्मा॒द्-यद्-वा॒चोऽना᳚प्तं॒ तन्म॑नु॒ष्या॑ उप॑ जीवन्ति पू॒र्णया॑ जुहोति पू॒र्ण इ॑व॒ हि प्र॒जाप॑तिः - [  ] \newline

\textbf{Pada Paata} \newline

ष॒ड्भिरिति॑ षट् - भिः । दी॒क्ष॒य॒ति॒ । षट् । वै । ऋ॒तवः॑ । ऋ॒तुभि॒रित्यृ॒तु-भिः॒ । ए॒व । ए॒न॒म् । दी॒क्ष॒य॒ति॒ । स॒प्तभि॒रिति॑ स॒प्त - भिः॒ । दी॒क्ष॒य॒ति॒ । स॒प्त । छन्दाꣳ॑सि । छन्दो॑भि॒रिति॒ छन्दः॑ - भिः॒ । ए॒व । ए॒न॒म् । दी॒क्ष॒य॒ति॒ । विश्वे᳚ । दे॒वस्य॑ । ने॒तुः । इति॑ । अ॒नु॒ष्टुभेत्य॑नु - स्तुभा᳚ । उ॒त्त॒मयेत्यु॑त् - त॒मया᳚ । जु॒हो॒ति॒ । वाक् । वै । अ॒नु॒ष्टुबित्य॑नु - स्तुप् । तस्मा᳚त् । प्रा॒णाना॒मिति॑ प्र - अ॒नाना᳚म् । वाक् । उ॒त्त॒मेत्यु॑त् - त॒मा । एक॑स्मात् । अ॒क्षरा᳚त् । अना᳚प्तम् । प्र॒थ॒मम् । प॒दम् । तस्मा᳚त् । यत् । वा॒चः । अना᳚प्तम् । तत् । म॒नु॒ष्याः᳚ । उपेति॑ । जी॒व॒न्ति॒ । पू॒र्णया᳚ । जु॒हो॒ति॒ । पू॒र्णः । इ॒व॒ । हि । प्र॒जाप॑ति॒रिति॑ प्र॒जा - प॒तिः॒ ।  \newline




\markright{ TS 5.1.9.2  \hfill https://www.vedavms.in \hfill}
\addcontentsline{toc}{section}{ TS 5.1.9.2 }
\section*{ TS 5.1.9.2 }

\textbf{TS 5.1.9.2 } \newline
\textbf{Samhita Paata} \newline

प्र॒जाप॑ते॒रापत्यै॒ न्यू॑नया जुहोति॒ न्यू॑ना॒द्धि प्र॒जाप॑तिः प्र॒जा असृ॑जत प्र॒जानाꣳ॒॒ सृष्ट्यै॒ यद॒र्चिषि॑ प्रवृ॒ञ्ज्याद्-भू॒तमव॑ रुन्धीत॒ यदङ्गा॑रेषु भवि॒ष्यदङ्गा॑रेषु॒ प्रवृ॑णक्ति भवि॒ष्य दे॒वाव॑ रुन्धे भवि॒ष्यद्धि भूयो॑ भू॒ताद्-द्वाभ्यां॒ प्रवृ॑णक्ति द्वि॒पाद्-यज॑मानः॒ प्रति॑ष्ठित्यै॒ ब्रह्म॑णा॒ वा ए॒षा यजु॑षा॒ संभृ॑ता॒ यदु॒खा सा यद्भिद्ये॒ताऽऽ*र्ति॒मार्च्छे॒ - [  ] \newline

\textbf{Pada Paata} \newline

प्र॒जाप॑ते॒रिति॑ प्र॒जा - प॒तेः॒ । आप्त्यै᳚ । न्यू॑न॒येति॒ नि - ऊ॒न॒या॒ । जु॒हो॒ति॒ । न्यू॑ना॒दिति॒ नि - ऊ॒ना॒त् । हि । प्र॒जाप॑ति॒रिति॑ प्र॒जा-प॒तिः॒ । प्र॒जा इति॑ प्र - जाः । असृ॑जत । प्र॒जाना॒मिति॑ प्र - जाना᳚म् । सृष्ट्यै᳚ । यत् । अ॒र्चिषि॑ । प्र॒वृ॒ञ्ज्यादिति॑ प्र - वृ॒ञ्ज्यात् । भू॒तम् । अवेति॑ । रु॒न्धी॒त॒ । यत् । अङ्गा॑रेषु । भ॒वि॒ष्यत् । अङ्गा॑रेषु । प्रेति॑ । वृ॒ण॒क्ति॒ । भ॒वि॒ष्यत् । ए॒व । अवेति॑ । रु॒न्धे॒ । भ॒वि॒ष्यत् । हि । भूयः॑ । भू॒तात् । द्वाभ्या᳚म् । प्रेति॑ । वृ॒ण॒क्ति॒ । द्वि॒पादिति॑ द्वि-पात् । यज॑मानः । प्रति॑ष्ठित्या॒ इति॒ प्रति॑ - स्थि॒त्यै॒ । ब्रह्म॑णा । वै । ए॒षा । यजु॑षा । संभृ॒तेति॒ सं - भृ॒ता॒ । यत् । उ॒खा । सा । यत् । भिद्ये॑त । आर्ति᳚म् । एति॑ । ऋ॒च्छे॒त् ।  \newline




\markright{ TS 5.1.9.3  \hfill https://www.vedavms.in \hfill}
\addcontentsline{toc}{section}{ TS 5.1.9.3 }
\section*{ TS 5.1.9.3 }

\textbf{TS 5.1.9.3 } \newline
\textbf{Samhita Paata} \newline

-द्यज॑मानो ह॒न्येता᳚ऽस्य य॒ज्ञो मित्रै॒तामु॒खां त॒पेत्या॑ह॒ ब्रह्म॒ वै मि॒त्रो ब्रह्म॑न्ने॒वैनां॒ प्रति॑ष्ठापयति॒ नाऽऽर्ति॒मार्च्छ॑ति॒ यज॑मानो॒ नास्य॑ य॒ज्ञो ह॑न्यते॒ यदि॒ भिद्ये॑त॒ तैरे॒व क॒पालैः॒ सꣳ सृ॑जे॒थ् सैव ततः॒ प्राय॑श्चित्ति॒र्यो ग॒तश्रीः॒ स्यान्म॑थि॒त्वा तस्याव॑ दद्ध्याद्-भू॒तो वा ए॒ष स स्वां - [  ] \newline

\textbf{Pada Paata} \newline

यज॑मानः । ह॒न्येत॑ । अ॒स्य॒ । य॒ज्ञ्ः । मित्र॑ । ए॒ताम् । उ॒खाम् । त॒प॒ । इति॑ । आ॒ह॒ । ब्रह्म॑ । वै । मि॒त्रः । ब्रह्मन्न्॑ । ए॒व । ए॒ना॒म् । प्रतीति॑ । स्था॒प॒य॒ति॒ । न । आर्ति᳚म् । एति॑ । ऋ॒च्छ॒ति॒ । यज॑मानः । न । अ॒स्य॒ । य॒ज्ञ्ः । ह॒न्य॒ते॒ । यदि॑ । भिद्ये॑त । तैः । ए॒व । क॒पालैः᳚ । समिति॑ । सृ॒जे॒त् । सा । ए॒व । ततः॑ । प्राय॑श्चित्तिः । यः । ग॒तश्री॒रिति॑ ग॒त - श्रीः॒ । स्यात् । म॒थि॒त्वा । तस्य॑ । अवेति॑ । द॒द्ध्या॒त् । भू॒तः । वै । ए॒षः । सः । स्वाम् ।  \newline




\markright{ TS 5.1.9.4  \hfill https://www.vedavms.in \hfill}
\addcontentsline{toc}{section}{ TS 5.1.9.4 }
\section*{ TS 5.1.9.4 }

\textbf{TS 5.1.9.4 } \newline
\textbf{Samhita Paata} \newline

दे॒वता॒मुपै॑ति॒ यो भूति॑कामः॒ स्याद्य उ॒खायै॑ स॒भंवे॒थ् स ए॒व तस्य॑ स्या॒दतो॒ ह्ये॑ष स॒भंव॑त्ये॒ष वै स्व॑य॒भूंर्नाम॒ भव॑त्ये॒व यं का॒मये॑त॒ भ्रातृ॑व्यमस्मै जनयेय॒मित्य॒-न्यत॒स्तस्या॒-ऽऽहृत्याऽव॑ दद्ध्याथ् सा॒क्षादे॒वास्मै॒ भ्रातृ॑व्यं जनयत्यम्ब॒रीषा॒दन्न॑ काम॒स्याव॑ दद्ध्यादंब॒रीषे॒ वा अन्नं॑ भ्रियते॒ सयो᳚न्ये॒वान्न॒ - [  ] \newline

\textbf{Pada Paata} \newline

दे॒वता᳚म् । उपेति॑ । ए॒ति॒ । यः । भूति॑काम॒ इति॒ भूति॑ - का॒मः॒ । स्यात् । यः । उ॒खायै᳚ । स॒भंवे॒दिति॑ सं - भवे᳚त् । सः । ए॒व । तस्य॑ । स्या॒त् । अतः॑ । हि । ए॒षः । स॒भंव॒तीति॑ सं-भव॑ति । ए॒षः । वै । स्व॒य॒भूंरिति॑ स्वयं-भूः । नाम॑ । भव॑ति । ए॒व । यम् । का॒मये॑त । भ्रातृ॑व्यम् । अ॒स्मै॒ । ज॒न॒ये॒य॒म् । इति॑ । अ॒न्यतः॑ । तस्य॑ । आ॒हृत्येत्या᳚ - हृत्य॑ । अवेति॑ । द॒द्ध्या॒त् । सा॒क्षादिति॑ स - अ॒क्षात् । ए॒व । अ॒स्मै॒ । भ्रातृ॑व्यम् । ज॒न॒य॒ति॒ । अ॒बं॒रीषा᳚त् । अन्न॑काम॒स्येत्यन्न॑ - का॒म॒स्य॒ । अवेति॑ । द॒द्ध्या॒त् । अ॒बं॒रीषे᳚ । वै । अन्न᳚म् । भ्रि॒य॒ते॒ । सयो॒नीति॒ स - यो॒नि॒ । ए॒व । अन्न᳚म् ।  \newline




\markright{ TS 5.1.9.5  \hfill https://www.vedavms.in \hfill}
\addcontentsline{toc}{section}{ TS 5.1.9.5 }
\section*{ TS 5.1.9.5 }

\textbf{TS 5.1.9.5 } \newline
\textbf{Samhita Paata} \newline

-मव॑ रुन्धे॒ मुञ्जा॒नव॑ दधा॒त्यूर्ग्वै मुञ्जा॒ ऊर्ज॑मे॒वास्मा॒ अपि॑ दधात्य॒ग्निर्दे॒वेभ्यो॒ निला॑यत॒ स क्रु॑मु॒कं प्राऽ*वि॑शत् क्रुमु॒कमव॑ दधाति॒ यदे॒वास्य॒ तत्र॒ न्य॑क्तं॒ तदे॒वाव॑ रुन्ध॒ आज्ये॑न॒ सं ॅयौ᳚त्ये॒तद्वा अ॒ग्नेः प्रि॒यं धाम॒ यदाज्यं॑ प्रि॒येणै॒वैनं॒ धाम्ना॒ सम॑र्द्धय॒त्यथो॒ तेज॑सा॒ - [  ] \newline

\textbf{Pada Paata} \newline

अवेति॑ । रु॒न्धे॒ । मुञ्जान्॑ । अवेति॑ । द॒धा॒ति॒ । ऊर्क् । वै । मुञ्जाः᳚ । ऊर्ज᳚म् । ए॒व । अ॒स्मै॒ । अपीति॑ । द॒धा॒ति॒ । अ॒ग्निः । दे॒वेभ्यः॑ । निला॑यत । सः । क्रु॒मु॒कम् । प्रेति॑ । अ॒वि॒श॒त् । क्रु॒मु॒कम् । अवेति॑ । द॒धा॒ति॒ । यत् । ए॒व । अ॒स्य॒ । तत्र॑ । न्य॑क्त॒मिति॒ नि - अ॒क्त॒म् । तत् । ए॒व । अवेति॑ । रु॒न्धे॒ । आज्ये॑न । समिति॑ । यौ॒ति॒ । ए॒तत् । वै । अ॒ग्नेः । प्रि॒यम् । धाम॑ । यत् । आज्य᳚म् । प्रि॒येण॑ । ए॒व । ए॒न॒म् । धाम्ना᳚ । समिति॑ । अ॒द्‌र्ध॒य॒ति॒ । अथो॒ इति॑ । तेज॑सा ।  \newline




\markright{ TS 5.1.9.6  \hfill https://www.vedavms.in \hfill}
\addcontentsline{toc}{section}{ TS 5.1.9.6 }
\section*{ TS 5.1.9.6 }

\textbf{TS 5.1.9.6 } \newline
\textbf{Samhita Paata} \newline

वै क॑कंती॒मा द॑धाति॒ भा ए॒वाव॑ रुन्धे शमी॒मयी॒मा द॑धाति॒ शान्त्यै॒ सीद॒ त्वं मा॒तुर॒स्या उ॒पस्थ॒ इति॑ ति॒सृभि॑र्जा॒तमुप॑ तिष्ठते॒ त्रय॑ इ॒मे लो॒का ए॒ष्वे॑व लो॒केष्वा॒विदं॑ गच्छ॒त्यथो᳚ प्रा॒णाने॒वाऽऽत्मन् ध॑त्ते ॥ \newline

\textbf{Pada Paata} \newline

वैक॑ङ्कतीम् । एति॑ । द॒धा॒ति॒ । भाः । ए॒व । अवेति॑ । रु॒न्धे॒ । श॒मी॒मयी॒मिति॑ शमी - मयी᳚म् । एति॑ । द॒धा॒ति॒ । शान्त्यै᳚ । सीद॑ । त्वम् । मा॒तुः । अ॒स्याः । उ॒पस्थ॒ इत्यु॒प - स्थे॒ । इति॑ । ति॒सृभि॒रिति॑ ति॒सृ - भिः॒ । जा॒तम् । उपेति॑ । ति॒ष्ठ॒ते॒ । त्रयः॑ । इ॒मे । लो॒काः । ए॒षु । ए॒व । लो॒केषु॑ । आ॒विद॒मित्या᳚ - विद᳚म् । ग॒च्छ॒ति॒ । अथो॒ इति॑ । प्रा॒णानिति॑ प्र - अ॒नान् । ए॒व । आ॒त्मन्न् । ध॒त्ते॒ ॥  \newline




\markright{ TS 5.1.10.1  \hfill https://www.vedavms.in \hfill}
\addcontentsline{toc}{section}{ TS 5.1.10.1 }
\section*{ TS 5.1.10.1 }

\textbf{TS 5.1.10.1 } \newline
\textbf{Samhita Paata} \newline

न ह॑ स्म॒ वै पु॒राऽग्निरप॑रशुवृक्णं दहति॒ तद॑स्मै प्रयो॒ग ए॒वर्.षि॑रस्वदय॒द्-यद॑ग्ने॒ यानि॒ कानि॒ चेति॑ स॒मिध॒मा द॑धा॒त्यप॑रशुवृक्ण-मे॒वास्मै᳚ स्वदयति॒ सर्व॑मस्मै स्वदते॒ य ए॒वं ॅवेदौदु॑म्बरी॒मा द॑धा॒त्यूर्ग्वा उ॑दु॒म्बर॒ ऊर्ज॑मे॒वास्मा॒ अपि॑ दधाति प्र॒जाप॑तिर॒ग्नि-म॑सृजत॒ तꣳ सृ॒ष्टꣳ रक्षाꣳ॑स्य - [  ] \newline

\textbf{Pada Paata} \newline

न । ह॒ । स्म॒ । वै । पु॒रा । अ॒ग्निः । अप॑रशुवृक्ण॒मित्यप॑रशु-वृ॒क्ण॒म् । द॒ह॒ति॒ । तत् । अ॒स्मै॒ । प्र॒यो॒ग इति॑ प्र - यो॒गः । ए॒व । ऋषिः॑ । अ॒स्व॒दय॒त् । यत् । अ॒ग्ने॒ । यानि॑ । कानि॑ । च॒ । इति॑ । स॒मिध॒मिति॑ सं - इध᳚म् । एति॑ । द॒धा॒ति॒ । अप॑रशुवृक्ण॒मित्यप॑रशु-वृ॒क्ण॒म् । ए॒व । अ॒स्मै॒ । स्व॒द॒य॒ति॒ । सर्व᳚म् । अ॒स्मै॒ । स्व॒द॒ते॒ । यः । ए॒वम् । वेद॑ । औदु॑बंरीम् । एति॑ । द॒धा॒ति॒ । ऊर्क् । वै । उ॒दु॒बंरः॑ । ऊर्ज᳚म् । ए॒व । अ॒स्मै॒ । अपीति॑ । द॒धा॒ति॒ । प्र॒जाप॑ति॒रिति॑ प्र॒जा - प॒तिः॒ । अ॒ग्निम् । अ॒सृ॒ज॒त॒ । तम् । सृ॒ष्टम् । रक्षाꣳ॑सि ।  \newline




\markright{ TS 5.1.10.2  \hfill https://www.vedavms.in \hfill}
\addcontentsline{toc}{section}{ TS 5.1.10.2 }
\section*{ TS 5.1.10.2 }

\textbf{TS 5.1.10.2 } \newline
\textbf{Samhita Paata} \newline

-जिघाꣳस॒न्थ्स ए॒तद्-रा᳚क्षो॒घ्नम॑पश्य॒त् तेन॒ वै सरक्षाꣳ॒॒स्यपा॑ऽहत॒ यद्-रा᳚क्षो॒घ्नं भव॑त्य॒ग्नेरे॒व तेन॑ जा॒ताद्-रक्षाꣳ॒॒स्यप॑ ह॒न्त्याश्व॑त्थी॒मा द॑धात्यश्व॒त्थो वै वन॒स्पती॑नाꣳ सपत्नसा॒हो विजि॑त्यै॒ वैक॑ङ्कती॒मा द॑धाति॒ भा ए॒वाव॑ रुन्धे शमी॒मयी॒मा द॑धाति॒ शान्त्यै॒ सꣳशि॑तं मे॒ ब्रह्मोदे॑षां बा॒हू अ॑तिर॒मित्यु॑त्त॒मे औदु॑म्बरी - [  ] \newline

\textbf{Pada Paata} \newline

अ॒जि॒घाꣳ॒॒स॒न्न् । सः । ए॒तत् । रा॒क्षो॒घ्नमिति॑ राक्षः-घ्नम् । अ॒प॒श्य॒त् । तेन॑ । वै । सः । रक्षाꣳ॑सि । अपेति॑ । अ॒ह॒त॒ । यत् । रा॒क्षो॒घ्नमिति॑ राक्षः - घ्नम् । भव॑ति । अ॒ग्नेः । ए॒व । तेन॑ । जा॒तात् । रक्षाꣳ॑सि । अपेति॑ । ह॒न्ति॒ । आश्व॑त्थीम् । एति॑ । द॒धा॒ति॒ । अ॒श्व॒त्थः । वै । वन॒स्पती॑नाम् । स॒प॒त्न॒सा॒ह इति॑ सपत्न - सा॒हः । विजि॑त्या॒ इति॒ वि - जि॒त्यै॒ । वैक॑ङ्कतीम् । एति॑ । द॒धा॒ति॒ । भाः । ए॒व । अवेति॑ । रु॒न्धे॒ । श॒मी॒मयी॒मिति॑ शमी - मयी᳚म् । एति॑ । द॒धा॒ति॒ । शान्त्यै᳚ । सꣳशि॑त॒मिति॒ सं-शि॒त॒म् । मे॒ । ब्रह्म॑ । उदिति॑ । ए॒षा॒म् । बा॒हू इति॑ । अ॒ति॒र॒म् । इति॑ । उ॒त्त॒मे इत्यु॑त् - त॒मे । औदु॑बंरी॒ इति॑ ।  \newline




\markright{ TS 5.1.10.3  \hfill https://www.vedavms.in \hfill}
\addcontentsline{toc}{section}{ TS 5.1.10.3 }
\section*{ TS 5.1.10.3 }

\textbf{TS 5.1.10.3 } \newline
\textbf{Samhita Paata} \newline

वाचयति॒ ब्रह्म॑णै॒व क्ष॒त्रꣳ सꣳ श्य॑ति क्ष॒त्रेण॒ ब्रह्म॒ तस्मा᳚द् ब्राह्म॒णो रा॑ज॒न्य॑वा॒नत्य॒न्यं ब्रा᳚ह्म॒णं तस्मा᳚द्-राज॒न्यो᳚ ब्राह्म॒णवा॒नत्य॒न्यꣳ रा॑ज॒न्यं॑ मृ॒त्युर्वा ए॒ष यद॒ग्निर॒मृतꣳ॒॒ हिर॑ण्यꣳ रु॒क्ममन्त॑रं॒ प्रति॑मुञ्चते॒ ऽमृत॑मे॒व मृ॒त्योर॒न्तर्द्ध॑त्त॒ एक॑विꣳशतिनिर्बाधो भव॒त्येक॑विꣳशति॒र्वै दे॑वलो॒का द्वाद॑श॒ मासाः॒ पञ्च॒र्तव॒स्त्रय॑ इ॒मे लो॒का अ॒सावा॑दि॒त्य - [  ] \newline

\textbf{Pada Paata} \newline

वा॒च॒य॒ति॒ । ब्रह्म॑णा । ए॒व । क्ष॒त्रम् । समिति॑ । श्य॒ति॒ । क्ष॒त्रेण॑ । ब्रह्म॑ । तस्मा᳚त् । ब्रा॒ह्म॒णः । रा॒ज॒न्य॑वा॒निति॑ राज॒न्य॑ - वा॒न् । अतीति॑ । अ॒न्यम् । ब्रा॒ह्म॒णम् । तस्मा᳚त् । रा॒ज॒न्यः॑ । ब्रा॒ह्म॒णवा॒निति॑ ब्राह्म॒ण - वा॒न् । अतीति॑ । अ॒न्यम् । रा॒ज॒न्य᳚म् । मृ॒त्युः । वै । ए॒षः । यत् । अ॒ग्निः । अ॒मृत᳚म् । हिर॑ण्यम् । रु॒क्मम् । अन्त॑रम् । प्रतीति॑ । मु॒ञ्च॒ते॒ । अ॒मृत᳚म् । ए॒व । मृ॒त्योः । अ॒न्तः । ध॒त्ते॒ । एक॑विꣳशतिनिर्बाध॒ इत्येक॑विꣳशति - नि॒र्बा॒धः॒ । भ॒व॒ति॒ । एक॑विꣳशति॒रित्येक॑-विꣳ॒॒श॒तिः॒ । वै । दे॒व॒लो॒का इति॑ देव-लो॒काः । द्वाद॑श । मासाः᳚ । पञ्च॑ । ऋ॒तवः॑ । त्रयः॑ । इ॒मे । लो॒काः । अ॒सौ । आ॒दि॒त्यः ।  \newline




\markright{ TS 5.1.10.4  \hfill https://www.vedavms.in \hfill}
\addcontentsline{toc}{section}{ TS 5.1.10.4 }
\section*{ TS 5.1.10.4 }

\textbf{TS 5.1.10.4 } \newline
\textbf{Samhita Paata} \newline

ए॑कविꣳ॒॒श ए॒ताव॑न्तो॒ वै दे॑वलो॒कास्तेभ्य॑ ए॒व भ्रातृ॑व्यम॒न्तरे॑ति निर्बा॒धैर्वै दे॒वा असु॑रान् निर्बा॒धे॑ऽकुर्वत॒ तन्नि॑र्बा॒धानां᳚ निर्बाध॒त्वं नि॑र्बा॒धी भ॑वति॒ भ्रातृ॑व्याने॒व नि॑र्बा॒धे कु॑रुते सावित्रि॒या प्रति॑मुञ्चते॒ प्रसू᳚त्यै॒ नक्तो॒षासेत्युत्त॑रया ऽहोरा॒त्राभ्या॑मे॒वैन॒-मुद्य॑च्छते दे॒वा अ॒ग्निं धा॑रयन् द्रविणो॒दा इत्या॑ह प्रा॒णा वै दे॒वा द्र॑विणो॒दा अ॑होरा॒त्राभ्या॑मे॒वैन॑मु॒द्यत्य॑ - [  ] \newline

\textbf{Pada Paata} \newline

ए॒क॒विꣳ॒॒श इत्ये॑क - विꣳ॒॒शः । ए॒ताव॑न्तः । वै । दे॒व॒लो॒का इति॑ देव - लो॒काः । तेभ्यः॑ । ए॒व । भ्रातृ॑व्यम् । अ॒न्तः । ए॒ति॒ । नि॒र्बा॒धैरिति॑ निः - बा॒धैः । वै । दे॒वाः । असु॑रान् । नि॒र्बा॒ध इति॑ निः - बा॒धे । अ॒कु॒र्व॒त॒ । तत् । नि॒र्बा॒धाना॒मिति॑ निः - बा॒धाना᳚म् । नि॒र्बा॒ध॒त्वमिति॑ निर्बाध - त्वम् । नि॒र्बा॒धीति॑ निः - बा॒धी । भ॒व॒ति॒ । भ्रातृ॑व्यान् । ए॒व । नि॒र्बा॒ध इति॑ निः - बा॒धे । कु॒रु॒ते॒ । सा॒वि॒त्रि॒या । प्रतीति॑ । मु॒ञ्च॒ते॒ । प्रसू᳚त्या॒ इति॒ प्र - सू॒त्यै॒ । नक्तो॒षासा᳚ । इति॑ । उत्त॑र॒येत्युत् - त॒र॒या॒ । अ॒हो॒रा॒त्राभ्या॒मित्य॑हः - रा॒त्राभ्या᳚म् । ए॒व । ए॒न॒म् । उदिति॑ । य॒च्छ॒ते॒ । दे॒वाः । अ॒ग्निम् । धा॒र॒य॒न्न् । द्र॒वि॒णो॒दा इति॑ द्रविणः - दाः । इति॑ । आ॒ह॒ । प्रा॒णा इति॑ प्र - अ॒नाः । वै । दे॒वाः । द्र॒वि॒णो॒दा इति॑ द्रविणः - दाः । अ॒हो॒रा॒त्राभ्या॒मित्य॑हः - रा॒त्राभ्या᳚म् । ए॒व । ए॒न॒म् । उ॒द्यत्येतु॑त्-यत्य॑ ।  \newline




\markright{ TS 5.1.10.5  \hfill https://www.vedavms.in \hfill}
\addcontentsline{toc}{section}{ TS 5.1.10.5 }
\section*{ TS 5.1.10.5 }

\textbf{TS 5.1.10.5 } \newline
\textbf{Samhita Paata} \newline

प्रा॒णैर्दा॑धा॒रा ऽऽसी॑नः॒ प्रति॑मुञ्चते॒ तस्मा॒दासी॑नाः प्र॒जाः प्रजा॑यन्ते कृष्णाजि॒नमुत्त॑रं॒ तेजो॒ वै हिर॑ण्यं॒ ब्रह्म॑ कृष्णाजि॒नं तेज॑सा चै॒वैनं॒ ब्रह्म॑णा चोभ॒यतः॒ परि॑गृह्णाति॒ षडु॑द्यामꣳ शि॒क्यं॑ भवति॒ षड्वा ऋ॒तव॑ ऋ॒तुभि॑रे॒वैन॒-मुद्य॑च्छते॒ यद् द्वाद॑शोद्यामꣳ संॅवथ्स॒रेणै॒व मौ॒ञ्जं भ॑व॒त्यूर्ग्वै मुञ्जा॑ ऊ॒र्जैवैनꣳ॒॒ स ( ) म॑र्द्धयति सुप॒र्णो॑ऽसि ग॒रुत्मा॒नित्यवे᳚क्षते रू॒पमे॒वास्यै॒तन्म॑हि॒मानं॒ ॅव्याच॑ष्टे॒ दिवं॑ गच्छ॒ सुवः॑ प॒तेत्या॑ह सुव॒र्गमे॒वैनं॑ ॅलो॒कं ग॑मयति ॥ \newline

\textbf{Pada Paata} \newline

प्रा॒णैरिति॑ प्र - अ॒नैः । दा॒धा॒र॒ । आसी॑नः । प्रतीति॑ । मु॒ञ्च॒ते॒ । तस्मा᳚त् । आसी॑नाः । प्र॒जा इति॑ प्र - जाः । प्रेति॑ । जा॒य॒न्ते॒ । कृ॒ष्णा॒जि॒नमिति॑ कृष्ण - अ॒जि॒नम् । उत्त॑र॒मित्युत् - त॒र॒म् । तेजः॑ । वै । हिर॑ण्यम् । ब्रह्म॑ । कृ॒ष्णा॒जि॒नमिति॑ कृष्ण - अ॒जि॒नम् । तेज॑सा । च॒ । ए॒व । ए॒न॒म् । ब्रह्म॑णा । च॒ । उ॒भ॒यतः॑ । परीति॑ । गृ॒ह्णा॒ति॒ । षडु॑द्याम॒मिति॒ षट् - उ॒द्या॒म॒म् । शि॒क्य᳚म् । भ॒व॒ति॒ । षट् । वै । ऋ॒तवः॑ । ऋ॒तुभि॒रित्यृ॒तु - भिः॒ । ए॒व । ए॒न॒म् । उदिति॑ । य॒च्छ॒ते॒ । यत् । द्वाद॑शोद्याम॒मिति॒ द्वाद॑श - उ॒द्या॒म॒म् । सं॒ॅव॒थ्स॒रेणेति॑ सं - व॒थ्स॒रेण॑ । ए॒व । मौ॒ञ्जम् । भ॒व॒ति॒ । ऊर्क् । वै । मुञ्जाः᳚ । ऊ॒र्जा । ए॒व । ए॒न॒म् । समिति॑ ( ) । अ॒द्‌र्ध॒य॒ति॒ । सु॒प॒र्ण इति॑ सु - प॒र्णः । अ॒सि॒ । ग॒रुत्मान्॑ । इति॑ । अवेति॑ । ई॒क्ष॒ते॒ । रू॒पम् । ए॒व । अ॒स्य॒ । ए॒तत् । म॒हि॒मान᳚म् । व्याच॑ष्ट॒ इति॑ वि - आच॑ष्टे । दिव᳚म् । ग॒च्छ॒ । सुवः॑ । प॒त॒ । इति॑ । आ॒ह॒ । सु॒व॒र्गमिति॑ सुवः - गम् । ए॒व । ए॒न॒म् । लो॒कम् । ग॒म॒य॒ति॒ ॥  \newline




\markright{ TS 5.1.11.1  \hfill https://www.vedavms.in \hfill}
\addcontentsline{toc}{section}{ TS 5.1.11.1 }
\section*{ TS 5.1.11.1 }

\textbf{TS 5.1.11.1 } \newline
\textbf{Samhita Paata} \newline

समि॑द्धो अ॒ञ्जन् कृद॑रं मती॒नां घृ॒तम॑ग्ने॒ मधु॑म॒त् पिन्व॑मानः । वा॒जी वह॑न् वा॒जिनं॑ जातवेदो दे॒वानां᳚ ॅवक्षि प्रि॒यमा स॒धस्थं᳚ ॥घृ॒तेना॒ञ्जन्थ्सं प॒थो दे॑व॒याना᳚न् प्रजा॒नन् वा॒ज्यप्ये॑तु दे॒वान् । अनु॑ त्वा सप्ते प्र॒दिशः॑ सचन्ताꣳ स्व॒धाम॒स्मै यज॑मानाय धेहि ॥ ईड्य॒श्चासि॒ वन्द्य॑श्च वाजिन्ना॒शुश्चासि॒ मेद्ध्य॑श्च सप्ते । अ॒ग्निष्ट्वा॑ - [  ] \newline

\textbf{Pada Paata} \newline

समि॑द्ध इति॒ सं - इ॒द्धः॒ । अ॒ञ्जन्न् । कृद॑रम् । म॒ती॒नाम् । घृ॒तम् । अ॒ग्ने॒ । मधु॑म॒दिति॒ मधु॑ - म॒त् । पिन्व॑मानः ॥ वा॒जी । वहन्न्॑ । वा॒जिन᳚म् । जा॒त॒वे॒द॒ इति॑ जात - वे॒दः॒ । दे॒वाना᳚म् । व॒क्षि॒ । प्रि॒यम् । एति॑ । स॒धस्थ॒मिति॑ स॒ध - स्थ॒म् ॥ घृ॒तेन॑ । अ॒ञ्जन्न् । समिति॑ । प॒थः । दे॒व॒याना॒निति॑ देव - यानान्॑ । प्र॒जा॒नन्निति॑ प्र - जा॒नन्न् । वा॒जी । अपीति॑ । ए॒तु॒ । दे॒वान् ॥ अन्विति॑ । त्वा॒ । स॒प्ते॒ । प्र॒दिश॒ इति॑ प्र - दिशः॑ । स॒च॒न्ता॒म् । स्व॒धामिति॑ स्व - धाम् । अ॒स्मै । यज॑मानाय । धे॒हि॒ ॥ ईड्यः॑ । च॒ । असि॑ । वन्द्यः॑ । च॒ । वा॒जि॒न्न् । आ॒शुः । च॒ । असि॑ । मेद्ध्यः॑ । च॒ । स॒प्ते॒ ॥ अ॒ग्निः । त्वा॒ ।  \newline




\markright{ TS 5.1.11.2  \hfill https://www.vedavms.in \hfill}
\addcontentsline{toc}{section}{ TS 5.1.11.2 }
\section*{ TS 5.1.11.2 }

\textbf{TS 5.1.11.2 } \newline
\textbf{Samhita Paata} \newline

दे॒वैर्वसु॑भिः स॒जोषाः᳚ प्री॒तं ॅवह्निं॑ ॅवहतु जा॒तवे॑दाः ॥ स्ती॒र्णम् ब॒र्॒.हिः सु॒ष्टरी॑मा जुषा॒णोरु पृ॒थु प्रथ॑मानं पृथि॒व्यां । दे॒वेभि॑र्यु॒क्तमदि॑तिः स॒जोषाः᳚ स्यो॒नं कृ॑ण्वा॒ना सु॑वि॒ते द॑धातु ॥ए॒ता उ॑ वः सु॒भगा॑ वि॒श्वरू॑पा॒ विपक्षो॑भिः॒ श्रय॑माणा॒ उदातैः᳚ ।ऋ॒ष्वाः स॒तीः क॒वषः॒ शुम्भ॑माना॒ द्वारो॑ दे॒वीः सु॑प्राय॒णा भ॑वन्तु ॥अ॒न्त॒रा मि॒त्रावरु॑णा॒ चर॑न्ती॒ मुखं॑ ॅय॒ज्ञाना॑म॒भि सं॑ॅविदा॒ने । उ॒षासा॑ वाꣳ - [  ] \newline

\textbf{Pada Paata} \newline

दे॒वैः । वसु॑भि॒रिति॒ वसु॑ - भिः॒ । स॒जोषा॒ इति॑ स - जोषाः᳚ । प्री॒तम् । वह्नि᳚म् । व॒ह॒तु॒ । जा॒तवे॑दा॒ इति॑ जा॒त - वे॒दाः॒ ॥ स्ती॒र्णम् । ब॒र्॒.हिः । सु॒ष्टरी॒मेति॑ सु - स्तरी॑म । जु॒षा॒णा । उ॒रु । पृ॒थु । प्रथ॑मानम् । पृ॒थि॒व्याम् ॥ दे॒वेभिः॑ । यु॒क्तम् । अदि॑तिः । स॒जोषा॒ इति॑ स - जोषाः᳚ । स्यो॒नम् । कृ॒ण्वा॒ना । सु॒वि॒ते । द॒धा॒तु॒ ॥ ए॒ताः । उ॒ । वः॒ । सु॒भगा॒ इति॑ सु - भगाः᳚ । वि॒श्वरू॑पा॒ इति॑ वि॒श्व - रू॒पाः॒ । वीति॑ । पक्षो॑भि॒रिति॒ पक्षः॑ - भिः॒ । श्रय॑माणाः । उदिति॑ । आतैः᳚ ॥ ऋ॒ष्वाः । स॒तीः । क॒वषः॑ । शुंभ॑मानाः । द्वारः॑ । दे॒वीः । सु॒प्रा॒य॒णा इति॑ सु - प्रा॒य॒णाः । भ॒व॒न्तु॒ ॥ अ॒न्त॒रा । मि॒त्रावरु॒णेति॑ मि॒त्रा - वरु॑णा । चर॑न्ती इति॑ । मुख᳚म् । य॒ज्ञाना᳚म् । अ॒भीति॑ । सं॒ॅवि॒दा॒ने इति॑ सं - वि॒दा॒ने ॥ उ॒षासा᳚ । वा॒म् ।  \newline




\markright{ TS 5.1.11.3  \hfill https://www.vedavms.in \hfill}
\addcontentsline{toc}{section}{ TS 5.1.11.3 }
\section*{ TS 5.1.11.3 }

\textbf{TS 5.1.11.3 } \newline
\textbf{Samhita Paata} \newline

सुहिर॒ण्ये सु॑शि॒ल्पे ऋ॒तस्य॒ योना॑वि॒ह सा॑दयामि ॥प्र॒थ॒मा वाꣳ॑ सर॒थिना॑ सु॒वर्णा॑ दे॒वौ पश्य॑न्तौ॒ भुव॑नानि॒ विश्वा᳚ । अपि॑प्रयं॒ चोद॑ना वां॒ मिमा॑ना॒ होता॑रा॒ ज्योतिः॑ प्र॒दिशा॑ दि॒शन्ता᳚ ॥आ॒दि॒त्यैर्नो॒ भार॑ती वष्टु य॒ज्ञ्ꣳ सर॑स्वती स॒ह रु॒द्रैर्न॑ आवीत् । इडोप॑हूता॒ वसु॑भिः स॒जोषा॑ य॒ज्ञ्ं नो॑ देवीर॒मृते॑षु धत्त ॥त्वष्टा॑ वी॒रं दे॒वका॑मं जजान॒ त्वष्टु॒रर्वा॑ जायत आ॒शुरश्वः॑ ( ) । \newline

\textbf{Pada Paata} \newline

सु॒हि॒र॒ण्ये इति॑ सु - हि॒र॒ण्ये । सु॒शि॒ल्पे इति॑ सु-शि॒ल्पे । ऋ॒तस्य॑ । योनौ᳚ । इ॒ह । सा॒द॒या॒मि॒ ॥ प्र॒थ॒मा । वा॒म् । स॒र॒थिनेति॑ स - र॒थिना᳚ । सु॒वर्णेति॑ सु - वर्णा᳚ । दे॒वौ । पश्य॑न्तौ । भुव॑नानि । विश्वा᳚ ॥ अपि॑प्रयम् । चोद॑ना । वा॒म् । मिमा॑ना । होता॑रा । ज्योतिः॑ । प्र॒दिशेति॑ प्र - दिशा᳚ । दि॒शन्ता᳚ ॥ आ॒दि॒त्यैः । नः॒ । भार॑ती । व॒ष्टु॒ । य॒ज्ञ्म् । सर॑स्वती । स॒ह । रु॒द्रैः । नः॒ । आ॒वी॒त् ॥ इडा᳚ । उप॑हू॒तेत्युप॑-हू॒ता॒ । वसु॑भि॒रिति॒ वसु॑ - भिः॒ । स॒जोषा॒ इति॑ स - जोषाः᳚ । य॒ज्ञ्म् । नः॒ । दे॒वीः॒ । अ॒मृते॑षु । ध॒त्त॒ ॥ त्वष्टा᳚ । वी॒रम् । दे॒वका॑म॒मिति॑ दे॒व-का॒म॒म् । ज॒जा॒न॒ । त्वष्टुः॑ । अर्वा᳚ । जा॒य॒ते॒ । आ॒शुः । अश्वः॑ ॥  \newline




\markright{ TS 5.1.11.4  \hfill https://www.vedavms.in \hfill}
\addcontentsline{toc}{section}{ TS 5.1.11.4 }
\section*{ TS 5.1.11.4 }

\textbf{TS 5.1.11.4 } \newline
\textbf{Samhita Paata} \newline

त्वष्टे॒दं ॅविश्वं॒ भुव॑नं जजान ब॒होः क॒र्तार॑मि॒ह य॑क्षि होतः ॥अश्वो॑ घृ॒तेन॒ त्मन्या॒ सम॑क्त॒ उप॑ दे॒वाꣳ ऋ॑तु॒शः पाथ॑ एतु । वन॒स्पति॑-र्देवलो॒कं प्र॑जा॒नन्न॒ग्निना॑ ह॒व्या स्व॑दि॒तानि॑ वक्षत् ॥प्र॒जाप॑ते॒स्तप॑सा वावृधा॒नः स॒द्यो जा॒तो द॑धिषे य॒ज्ञ्म॑ग्ने । स्वाहा॑कृतेन ह॒विषा॑ पुरोगा या॒हि सा॒द्ध्या ह॒विर॑दन्तु दे॒वाः ॥ \newline

\textbf{Pada Paata} \newline

त्वष्टा᳚ । इ॒दम् । विश्व᳚म् । भुव॑नम् । ज॒जा॒न॒ । ब॒होः । क॒र्तार᳚म् । इ॒ह । य॒क्षि॒ । हो॒तः॒ ॥ अश्वः॑ । घृ॒तेन॑ । त्मन्या᳚ । सम॑क्त॒ इति॒ सं - अ॒क्तः॒ । उपेति॑ । दे॒वान् । ऋ॒तु॒श इत्यृ॑तु - शः । पाथः॑ । ए॒तु॒ ॥ वन॒स्पतिः॑ । दे॒व॒लो॒कमिति॑ देव - लो॒कम् । प्र॒जा॒नन्निति॑ प्र-जा॒नन्न् । अ॒ग्निना᳚ । ह॒व्या । स्व॒दि॒तानि॑ । व॒क्ष॒त् ॥ प्र॒जाप॑ते॒रिति॑ प्र॒जा - प॒तेः॒ । तप॑सा । वा॒वृ॒धा॒नः । स॒द्यः । जा॒तः । द॒धि॒षे॒ । य॒ज्ञ्म् । अ॒ग्ने॒ ॥ स्वाहा॑कृते॒नेति॒ स्वाहा᳚ - कृ॒ते॒न॒ । ह॒विषा᳚ । पु॒रो॒गा॒ इति॑ पुरः - गाः॒ । या॒हि । सा॒द्ध्या । ह॒विः । अ॒द॒न्तु॒ । दे॒वाः ॥  \newline






\end{document}