\documentclass[17pt]{extarticle}
\usepackage{babel}
\usepackage{fontspec}
\usepackage{polyglossia}
\usepackage{extsizes}

\usepackage{color}   %May be necessary if you want to color links
\usepackage{hyperref}
\hypersetup{
    colorlinks=true, %set true if you want colored links
    linktoc=all,     %set to all if you want both sections and subsections linked
    linkcolor=black,  %choose some color if you want links to stand out
}

\setmainlanguage{sanskrit}
\setotherlanguages{english} %% or other languages
\setlength{\parindent}{0pt}
\pagestyle{myheadings}
\newfontfamily\devanagarifont[Script=Devanagari]{AdishilaVedic}
\renewcommand{\theHsection}{\thepart.section.\thesection}

\newcommand{\VAR}[1]{}
\newcommand{\BLOCK}[1]{}




\begin{document}
\begin{titlepage}
    \begin{center}
 
\begin{sanskrit}
    { \Large
    कृष्ण यजुर्वेदीय तैत्तिरीय संहिता,पद,जटा,घन पाठः 
    }
    \\
    \vspace{2.5cm}
    \mbox{ \Large
    1.2     प्रथमकाण्डे द्वितीयः प्रश्नः - (अग्निष्टोमे क्रयः)   }
\end{sanskrit}
\end{center}

\end{titlepage}
\tableofcontents
\phantomsection
\pagebreak

\markright{ TS 1.2.1.1  \hfill https://www.vedavms.in \hfill}

\section{ TS 1.2.1.1 }

\textbf{TS 1.2.1.1 } \newline
\textbf{Samhita Paata} \newline

आप॑ उन्दन्तु जी॒वसे॑ दीर्घायु॒त्वाय॒ वर्च॑स॒ ओष॑धे॒ त्राय॑स्वैनꣳ॒॒ स्वधि॑ते॒ मैनꣳ॑ हिꣳसीर् देव॒श्रूरे॒तानि॒ प्र व॑पे स्व॒स्त्युत्त॑राण्यशी॒यापो॑ अ॒स्मान् मा॒तरः॑ शुन्धन्तु घृ॒तेन॑ नो घृत॒पुवः॑ पुनन्तु॒ विश्व॑म॒स्मत् प्र व॑हन्तु रि॒प्रमुदा᳚भ्यः॒ शुचि॒रा पू॒त ए॑मि॒ सोम॑स्य त॒नूर॑सि त॒नुवं॑ मे पाहि मही॒नां पयो॑ऽसि वर्चो॒धा अ॑सि॒ वर्चो॒ - [ ] \newline

\textbf{Pada Paata} \newline

आपः॑ । उ॒न्द॒न्तु॒ । जी॒वसे᳚ । दी॒र्घा॒यु॒त्वायेति॑ दीर्घायु - त्वाय॑ । वर्च॑से । ओष॑धे । त्राय॑स्व । ए॒न॒म् । स्वधि॑त॒ इति॒ स्व-धि॒ते॒ । मा । ए॒न॒म् । हिꣳ॒॒सीः॒ । दे॒व॒श्रूरिति॑ देव - श्रूः । ए॒तानि॑ । प्रेति॑ । व॒पे॒ । स्व॒स्ति । उत्त॑रा॒णीत्युत् - त॒रा॒णि॒ । अ॒शी॒य॒ । आपः॑ । अ॒स्मान् । मा॒तरः॑ । शु॒न्ध॒न्तु॒ । घृ॒तेन॑ । नः॒ । घृ॒त॒पुव॒ इति॑ घृत-पुवः॑ । पु॒न॒न्तु॒ । विश्व᳚म् । अ॒स्मत् । प्रेति॑ । व॒ह॒न्तु॒ । रि॒प्रम् । उदिति॑ । आ॒भ्यः॒ । शुचिः॑ । एति॑ । पू॒तः । ए॒मि॒ । सोम॑स्य । त॒नूः । अ॒सि॒ । त॒नुव᳚म् । मे॒ । पा॒हि॒ । म॒ही॒नाम् । पयः॑ । अ॒सि॒ । व॒र्चो॒धा इति वर्चः - धाः । अ॒सि॒ । वर्चः॑ ।  \newline


\textbf{Krama Paata} \newline

आप॑ उन्दन्तु । उ॒न्द॒न्तु॒ जी॒वसे᳚ । जी॒वसे॑ दीर्घायु॒त्वाय॑ । दी॒र्घा॒यु॒त्वाय॒ वर्च॑से । दी॒र्घा॒यु॒त्वायेति॑ दीर्घायु - त्वाय॑ । वर्च॑स॒ ओष॑धे । ओष॑धे॒ त्राय॑स्व । त्राय॑स्वैनम् । ए॒नꣳ॒॒ स्वधि॑ते । स्वधि॑ते॒ मा । स्वधि॑त॒ इति॒ स्व - धि॒ते॒ । मैन᳚म् । ए॒नꣳ॒॒ हिꣳ॒॒सीः॒ । हिꣳ॒॒सी॒र् दे॒व॒श्रूः । दे॒व॒श्रूरे॒तानि॑ । दे॒व॒श्रूरिति॑ देव - श्रूः । ए॒तानि॒ प्र । प्र व॑पे । व॒पे॒ स्व॒स्ति । स्व॒स्त्युत्त॑राणि । उत्त॑राण्यशीय । उत्त॑रा॒णीत्युत् - त॒रा॒णि॒ । अ॒शी॒यापः॑ । आपो॑ अ॒स्मान् । अ॒स्मान् मा॒तरः॑ । मा॒तरः॑ शुन्धन्तु । शु॒न्ध॒न्तु॒ घृ॒तेन॑ । घृ॒तेन॑ नः । नो॒ घृ॒त॒पुवः॑ । घृ॒त॒पुवः॑ पुनन्तु । घृ॒त॒पुव॒ इति॑ घृत - पुवः॑ । पु॒न॒न्तु॒ विश्व᳚म् । विश्व॑म॒स्मत् । अ॒स्मत् प्र । प्र व॑हन्तु । व॒ह॒न्तु॒ रि॒प्रम् । रि॒प्रमुत् । उदा᳚भ्यः । आ॒भ्यः॒ शुचिः॑ । शुचि॒रा । आ पू॒तः । पू॒त ए॑मि । ए॒मि॒ सोम॑स्य । सोम॑स्य त॒नूः । त॒नूर॑सि । अ॒सि॒ त॒नुव᳚म् । त॒नुव॑म् मे । मे॒ पा॒हि॒ । पा॒हि॒ म॒ही॒नाम् । म॒ही॒नाम् पयः॑ । पयो॑ऽसि । अ॒सि॒ व॒र्चो॒धाः । व॒र्चो॒धा अ॑सि । व॒र्चो॒धा इति॑ वर्चः - धाः । अ॒सि॒ वर्चः॑ । वर्चो॒ मयि॑ \newline

\textbf{Jatai Paata} \newline

1. आप॑ उन्दन्तून्द॒न्त्वाप॒ आप॑ उन्दन्तु । \newline
2. उ॒न्द॒न्तु॒ जी॒वसे॑ जी॒वस॑ उन्दन्तू न्दन्तु जी॒वसे᳚ । \newline
3. जी॒वसे॑ दीर्घायु॒त्वाय॑ दीर्घायु॒त्वाय॑ जी॒वसे॑ जी॒वसे॑ दीर्घायु॒त्वाय॑ । \newline
4. दी॒र्घा॒यु॒त्वाय॒ वर्च॑से॒ वर्च॑से दीर्घायु॒त्वाय॑ दीर्घायु॒त्वाय॒ वर्च॑से । \newline
5. दी॒र्घा॒यु॒त्वायेति॑ दीर्घायु - त्वाय॑ । \newline
6. वर्च॑स॒ ओष॑ध॒ ओष॑धे॒ वर्च॑से॒ वर्च॑स॒ ओष॑धे । \newline
7. ओष॑धे॒ त्राय॑स्व॒ त्राय॒स्वौष॑ध॒ ओष॑धे॒ त्राय॑स्व । \newline
8. त्राय॑स्वैन मेन॒म् त्राय॑स्व॒ त्राय॑स्वैनम् । \newline
9. ए॒न॒(ग्ग्॒) स्वधि॑ते॒ स्वधि॑त एन मेन॒(ग्ग्॒) स्वधि॑ते । \newline
10. स्वधि॑ते॒ मा मा स्वधि॑ते॒ स्वधि॑ते॒ मा । \newline
11. स्वधि॑त॒ इति॒ स्व - धि॒ते॒ । \newline
12. मैन॑ मेन॒म् मा मैन᳚म् । \newline
13. ए॒न॒(ग्म्॒) हि॒(ग्म्॒)सी॒र्॒. हि॒(ग्म्॒)सी॒ रे॒न॒ मे॒न॒(ग्म्॒) हि॒(ग्म्॒)सीः॒ । \newline
14. हि॒(ग्म्॒)सी॒र् दे॒व॒श्रूर् दे॑व॒श्रूर्. हि(ग्म्॑)सीर्. हिꣳसीर् देव॒श्रूः । \newline
15. दे॒व॒श्रू रे॒तान्ये॒तानि॑ देव॒श्रूर् दे॑व॒श्रू रे॒तानि॑ । \newline
16. दे॒व॒श्रूरिति॑ देव - श्रूः । \newline
17. ए॒तानि॒ प्र प्रैतान्ये॒तानि॒ प्र । \newline
18. प्र व॑पे वपे॒ प्र प्र व॑पे । \newline
19. व॒पे॒ स्व॒स्ति स्व॒स्ति व॑पे वपे स्व॒स्ति । \newline
20. स्व॒ स्त्युत्त॑रा॒ ण्युत्त॑राणि स्व॒स्ति स्व॒ स्त्युत्त॑राणि । \newline
21. उत्त॑रा ण्यशीयाशी॒ योत्त॑रा॒ ण्युत्त॑रा ण्यशीय । \newline
22. उत्त॑रा॒णीत्युत् - त॒रा॒णि॒ । \newline
23. अ॒शी॒याप॒ आपो॑ ऽशीयाशी॒यापः॑ । \newline
24. आपो॑ अ॒स्मा न॒स्मा नाप॒ आपो॑ अ॒स्मान् । \newline
25. अ॒स्मान् मा॒तरो॑ मा॒तरो॒ ऽस्मा न॒स्मान् मा॒तरः॑ । \newline
26. मा॒तरः॑ शुन्धन्तु शुन्धन्तु मा॒तरो॑ मा॒तरः॑ शुन्धन्तु । \newline
27. शु॒न्ध॒न्तु॒ घृ॒तेन॑ घृ॒तेन॑ शुन्धन्तु शुन्धन्तु घृ॒तेन॑ । \newline
28. घृ॒तेन॑ नो नो घृ॒तेन॑ घृ॒तेन॑ नः । \newline
29. नो॒ घृ॒त॒पुवो॑ घृत॒पुवो॑ नो नो घृत॒पुवः॑ । \newline
30. घृ॒त॒पुवः॑ पुनन्तु पुनन्तु घृत॒पुवो॑ घृत॒पुवः॑ पुनन्तु । \newline
31. घृ॒त॒पुव॒ इति॑ घृत - पुवः॑ । \newline
32. पु॒न॒न्तु॒ विश्वं॒ ॅविश्व॑म् पुनन्तु पुनन्तु॒ विश्व᳚म् । \newline
33. विश्व॑ म॒स्म द॒स्मद् विश्वं॒ ॅविश्व॑ म॒स्मत् । \newline
34. अ॒स्मत् प्र प्रास्म द॒स्मत् प्र । \newline
35. प्र व॑हन्तु वहन्तु॒ प्र प्र व॑हन्तु । \newline
36. व॒ह॒न्तु॒ रि॒प्रꣳ रि॒प्रं ॅव॑हन्तु वहन्तु रि॒प्रम् । \newline
37. रि॒प्र मुदुद् रि॒प्रꣳ रि॒प्र मुत् । \newline
38. उदा᳚भ्य आभ्य॒ उदु दा᳚भ्यः । \newline
39. आ॒भ्यः॒ शुचिः॒ शुचि॑ राभ्य आभ्यः॒ शुचिः॑ । \newline
40. शुचि॒रा शुचिः॒ शुचि॒रा । \newline
41. आ पू॒तः पू॒त आ पू॒तः । \newline
42. पू॒त ए᳚म्येमि पू॒तः पू॒त ए॑मि । \newline
43. ए॒मि॒ सोम॑स्य॒ सोम॑स्यैम्येमि॒ सोम॑स्य । \newline
44. सोम॑स्य त॒नू स्त॒नूः सोम॑स्य॒ सोम॑स्य त॒नूः । \newline
45. त॒नू र॑स्यसि त॒नू स्त॒नू र॑सि । \newline
46. अ॒सि॒ त॒नुव॑म् त॒नुव॑ मस्यसि त॒नुव᳚म् । \newline
47. त॒नुव॑म् मे मे त॒नुव॑म् त॒नुव॑म् मे । \newline
48. मे॒ पा॒हि॒ पा॒हि॒ मे॒ मे॒ पा॒हि॒ । \newline
49. पा॒हि॒ म॒ही॒नाम् म॑ही॒नाम् पा॑हि पाहि मही॒नाम् । \newline
50. म॒ही॒नाम् पयः॒ पयो॑ मही॒नाम् म॑ही॒नाम् पयः॑ । \newline
51. पयो᳚ ऽस्यसि॒ पयः॒ पयो॑ ऽसि । \newline
52. अ॒सि॒ व॒र्चो॒धा व॑र्चो॒धा अ॑स्यसि वर्चो॒धाः । \newline
53. व॒र्चो॒धा अ॑स्यसि वर्चो॒धा व॑र्चो॒धा अ॑सि । \newline
54. व॒र्चो॒धा इति॑ वर्चः - धाः । \newline
55. अ॒सि॒ वर्चो॒ वर्चो᳚ ऽस्यसि॒ वर्चः॑ । \newline
56. वर्चो॒ मयि॒ मयि॒ वर्चो॒ वर्चो॒ मयि॑ । \newline

\textbf{Ghana Paata } \newline

1. आप॑ उन्दन्तून्द॒न्त्वाप॒ आप॑ उन्दन्तु जी॒वसे॑ जी॒वस॑ उन्द॒न्त्वाप॒ आप॑ उन्दन्तु जी॒वसे᳚ । \newline
2. उ॒न्द॒न्तु॒ जी॒वसे॑ जी॒वस॑ उन्दन्तून्दन्तु जी॒वसे॑ दीर्घायु॒त्वाय॑ दीर्घायु॒त्वाय॑ जी॒वस॑ उन्दन्तून्दन्तु जी॒वसे॑ दीर्घायु॒त्वाय॑ । \newline
3. जी॒वसे॑ दीर्घायु॒त्वाय॑ दीर्घायु॒त्वाय॑ जी॒वसे॑ जी॒वसे॑ दीर्घायु॒त्वाय॒ वर्च॑से॒ वर्च॑से दीर्घायु॒त्वाय॑ जी॒वसे॑ जी॒वसे॑ दीर्घायु॒त्वाय॒ वर्च॑से । \newline
4. दी॒र्घा॒यु॒त्वाय॒ वर्च॑से॒ वर्च॑से दीर्घायु॒त्वाय॑ दीर्घायु॒त्वाय॒ वर्च॑स॒ ओष॑ध॒ ओष॑धे॒ वर्च॑से दीर्घायु॒त्वाय॑ दीर्घायु॒त्वाय॒ वर्च॑स॒ ओष॑धे । \newline
5. दी॒र्घा॒यु॒त्वायेति॑ दीर्घायु - त्वाय॑ । \newline
6. वर्च॑स॒ ओष॑ध॒ ओष॑धे॒ वर्च॑से॒ वर्च॑स॒ ओष॑धे॒ त्राय॑स्व॒ त्राय॒स्वौष॑धे॒ वर्च॑से॒ वर्च॑स॒ ओष॑धे॒ त्राय॑स्व । \newline
7. ओष॑धे॒ त्राय॑स्व॒ त्राय॒स्वौष॑ध॒ ओष॑धे॒ त्राय॑स्वैन मेन॒म् त्राय॒स्वौष॑ध॒ ओष॑धे॒ त्राय॑स्वैनम् । \newline
8. त्राय॑स्वैन मेन॒म् त्राय॑स्व॒ त्राय॑स्वैन॒(ग्ग्॒) स्वधि॑ते॒ स्वधि॑त एन॒म् त्राय॑स्व॒ त्राय॑स्वैन॒(ग्ग्॒) स्वधि॑ते । \newline
9. ए॒न॒(ग्ग्॒) स्वधि॑ते॒ स्वधि॑त एन मेन॒(ग्ग्॒) स्वधि॑ते॒ मा मा स्वधि॑त एन मेन॒(ग्ग्॒) स्वधि॑ते॒ मा । \newline
10. स्वधि॑ते॒ मा मा स्वधि॑ते॒ स्वधि॑ते॒ मैन॑ मेन॒म् मा स्वधि॑ते॒ स्वधि॑ते॒ मैन᳚म् । \newline
11. स्वधि॑त॒ इति॒ स्व - धि॒ते॒ । \newline
12. मैन॑ मेन॒म् मा मैन(ग्म्॑) हिꣳसीर्. हिꣳसीरेन॒म् मा मैन(ग्म्॑) हिꣳसीः । \newline
13. ए॒न॒(ग्म्॒) हि॒(ग्म्॒)सी॒र्॒. हि॒(ग्म्॒)सी॒ रे॒न॒ मे॒न॒(ग्म्॒) हि॒(ग्म्॒)सी॒र् दे॒व॒श्रूर् दे॑व॒श्रूर्. हि(ग्म्॑)सी रेन मेनꣳ हिꣳसीर् देव॒श्रूः । \newline
14. हि॒(ग्म्॒)सी॒र् दे॒व॒श्रूर् दे॑व॒श्रूर्. हि(ग्म्॑)सीर्. हिꣳसीर् देव॒श्रूरे॒तान्ये॒तानि॑ देव॒श्रूर्. हि(ग्म्॑)सीर्. हिꣳसीर् देव॒श्रूरे॒तानि॑ । \newline
15. दे॒व॒श्रूरे॒तान्ये॒तानि॑ देव॒श्रूर् दे॑व॒श्रूरे॒तानि॒ प्र प्रैतानि॑ देव॒श्रूर् दे॑व॒श्रूरे॒तानि॒ प्र । \newline
16. दे॒व॒श्रूरिति॑ देव - श्रूः । \newline
17. ए॒तानि॒ प्र प्रैतान्ये॒तानि॒ प्र व॑पे वपे॒ प्रैतान्ये॒तानि॒ प्र व॑पे । \newline
18. प्र व॑पे वपे॒ प्र प्र व॑पे स्व॒स्ति स्व॒स्ति व॑पे॒ प्र प्र व॑पे स्व॒स्ति । \newline
19. व॒पे॒ स्व॒स्ति स्व॒स्ति व॑पे वपे स्व॒स्त्युत्त॑रा॒ण्युत्त॑राणि स्व॒स्ति व॑पे वपे स्व॒स्त्युत्त॑राणि । \newline
20. स्व॒स्त्युत्त॑ रा॒ण्युत्त॑राणि स्व॒स्ति स्व॒स्त्युत्त॑ राण्यशीयाशी॒ योत्त॑राणि स्व॒स्ति स्व॒स्त्युत्त॑ राण्यशीय । \newline
21. उत्त॑ राण्यशीयाशी॒योत्त॑ रा॒ण्युत्त॑राण्यशी॒याप॒ आपो॑ ऽशी॒योत्त॑ रा॒ण्युत्त॑राण्यशी॒यापः॑ । \newline
22. उत्त॑रा॒णीत्युत् - त॒रा॒णि॒ । \newline
23. अ॒शी॒याप॒ आपो॑ ऽशीयाशी॒यापो॑ अ॒स्मा न॒स्मान् आपो॑ ऽशीयाशी॒यापो॑ अ॒स्मान् । \newline
24. आपो॑ अ॒स्मा न॒स्मान् आप॒ आपो॑ अ॒स्मान् मा॒तरो॑ मा॒तरो॒ ऽस्मा नाप॒ आपो॑ अ॒स्मान् मा॒तरः॑ । \newline
25. अ॒स्मान् मा॒तरो॑ मा॒तरो॒ ऽस्मा न॒स्मान् मा॒तरः॑ शुन्धन्तु शुन्धन्तु मा॒तरो॒ ऽस्मा न॒स्मान् मा॒तरः॑ शुन्धन्तु । \newline
26. मा॒तरः॑ शुन्धन्तु शुन्धन्तु मा॒तरो॑ मा॒तरः॑ शुन्धन्तु घृ॒तेन॑ घृ॒तेन॑ शुन्धन्तु मा॒तरो॑ मा॒तरः॑ शुन्धन्तु घृ॒तेन॑ । \newline
27. शु॒न्ध॒न्तु॒ घृ॒तेन॑ घृ॒तेन॑ शुन्धन्तु शुन्धन्तु घृ॒तेन॑ नो नो घृ॒तेन॑ शुन्धन्तु शुन्धन्तु घृ॒तेन॑ नः । \newline
28. घृ॒तेन॑ नो नो घृ॒तेन॑ घृ॒तेन॑ नो घृत॒पुवो॑ घृत॒पुवो॑ नो घृ॒तेन॑ घृ॒तेन॑ नो घृत॒पुवः॑ । \newline
29. नो॒ घृ॒त॒पुवो॑ घृत॒पुवो॑ नो नो घृत॒पुवः॑ पुनन्तु पुनन्तु घृत॒पुवो॑ नो नो घृत॒पुवः॑ पुनन्तु । \newline
30. घृ॒त॒पुवः॑ पुनन्तु पुनन्तु घृत॒पुवो॑ घृत॒पुवः॑ पुनन्तु॒ विश्वं॒ ॅविश्व॑म् पुनन्तु घृत॒पुवो॑ घृत॒पुवः॑ पुनन्तु॒ विश्व᳚म् । \newline
31. घृ॒त॒पुव॒ इति॑ घृत - पुवः॑ । \newline
32. पु॒न॒न्तु॒ विश्वं॒ ॅविश्व॑म् पुनन्तु पुनन्तु॒ विश्व॑ म॒स्मद॒स्मद् विश्व॑म् पुनन्तु पुनन्तु॒ विश्व॑ म॒स्मत् । \newline
33. विश्व॑ म॒स्मद॒स्मद् विश्वं॒ ॅविश्व॑ म॒स्मत् प्र प्रास्मद् विश्वं॒ ॅविश्व॑ म॒स्मत् प्र । \newline
34. अ॒स्मत् प्र प्रास्म द॒स्मत् प्र व॑हन्तु वहन्तु॒ प्रास्म द॒स्मत् प्र व॑हन्तु । \newline
35. प्र व॑हन्तु वहन्तु॒ प्र प्र व॑हन्तु रि॒प्रꣳ रि॒प्रं ॅव॑हन्तु॒ प्र प्र व॑हन्तु रि॒प्रम् । \newline
36. व॒ह॒न्तु॒ रि॒प्रꣳ रि॒प्रं ॅव॑हन्तु वहन्तु रि॒प्र मुदुद् रि॒प्रं ॅव॑हन्तु वहन्तु रि॒प्र मुत् । \newline
37. रि॒प्र मुदुद् रि॒प्रꣳ रि॒प्र मुदा᳚भ्य आभ्य॒ उद् रि॒प्रꣳ रि॒प्र मुदा᳚भ्यः । \newline
38. उदा᳚भ्य आभ्य॒ उदुदा᳚भ्यः॒ शुचिः॒ शुचि॑राभ्य॒ उदुदा᳚भ्यः॒ शुचिः॑ । \newline
39. आ॒भ्यः॒ शुचिः॒ शुचि॑राभ्य आभ्यः॒ शुचि॒रा शुचि॑राभ्य आभ्यः॒ शुचि॒रा । \newline
40. शुचि॒रा शुचिः॒ शुचि॒रा पू॒तः पू॒त आ शुचिः॒ शुचि॒रा पू॒तः । \newline
41. आ पू॒तः पू॒त आ पू॒त ए᳚म्येमि पू॒त आ पू॒त ए॑मि । \newline
42. पू॒त ए᳚म्येमि पू॒तः पू॒त ए॑मि॒ सोम॑स्य॒ सोम॑स्यैमि पू॒तः पू॒त ए॑मि॒ सोम॑स्य । \newline
43. ए॒मि॒ सोम॑स्य॒ सोम॑स्यैम्येमि॒ सोम॑स्य त॒नूस्त॒नूः सोम॑स्यैम्येमि॒ सोम॑स्य त॒नूः । \newline
44. सोम॑स्य त॒नूस्त॒नूः सोम॑स्य॒ सोम॑स्य त॒नूर॑स्यसि त॒नूः सोम॑स्य॒ सोम॑स्य त॒नूर॑सि । \newline
45. त॒नू र॑स्यसि त॒नू स्त॒नूर॑सि त॒नुव॑म् त॒नुव॑ मसि त॒नू स्त॒नूर॑सि त॒नुव᳚म् । \newline
46. अ॒सि॒ त॒नुव॑म् त॒नुव॑ मस्यसि त॒नुव॑म् मे मे त॒नुव॑ मस्यसि त॒नुव॑म् मे । \newline
47. त॒नुव॑म् मे मे त॒नुव॑म् त॒नुव॑म् मे पाहि पाहि मे त॒नुव॑म् त॒नुव॑म् मे पाहि । \newline
48. मे॒ पा॒हि॒ पा॒हि॒ मे॒ मे॒ पा॒हि॒ म॒ही॒नाम् म॑ही॒नाम् पा॑हि मे मे पाहि मही॒नाम् । \newline
49. पा॒हि॒ म॒ही॒नाम् म॑ही॒नाम् पा॑हि पाहि मही॒नाम् पयः॒ पयो॑ मही॒नाम् पा॑हि पाहि मही॒नाम् पयः॑ । \newline
50. म॒ही॒नाम् पयः॒ पयो॑ मही॒नाम् म॑ही॒नाम् पयो᳚ ऽस्यसि॒ पयो॑ मही॒नाम् म॑ही॒नाम् पयो॑ ऽसि । \newline
51. पयो᳚ ऽस्यसि॒ पयः॒ पयो॑ ऽसि वर्चो॒धा व॑र्चो॒धा अ॑सि॒ पयः॒ पयो॑ ऽसि वर्चो॒धाः । \newline
52. अ॒सि॒ व॒र्चो॒धा व॑र्चो॒धा अ॑स्यसि वर्चो॒धा अ॑स्यसि वर्चो॒धा अ॑स्यसि वर्चो॒धा अ॑सि । \newline
53. व॒र्चो॒धा अ॑स्यसि वर्चो॒धा व॑र्चो॒धा अ॑सि॒ वर्चो॒ वर्चो॑ ऽसि वर्चो॒धा व॑र्चो॒धा अ॑सि॒ वर्चः॑ । \newline
54. व॒र्चो॒धा इति॑ वर्चः - धाः । \newline
55. अ॒सि॒ वर्चो॒ वर्चो᳚ ऽस्यसि॒ वर्चो॒ मयि॒ मयि॒ वर्चो᳚ ऽस्यसि॒ वर्चो॒ मयि॑ । \newline
56. वर्चो॒ मयि॒ मयि॒ वर्चो॒ वर्चो॒ मयि॑ धेहि धेहि॒ मयि॒ वर्चो॒ वर्चो॒ मयि॑ धेहि । \newline
\pagebreak
\markright{ TS 1.2.1.2  \hfill https://www.vedavms.in \hfill}

\section{ TS 1.2.1.2 }

\textbf{TS 1.2.1.2 } \newline
\textbf{Samhita Paata} \newline

मयि॑ धेहि वृ॒त्रस्य॑ क॒नीनि॑काऽसि चक्षु॒ष्पा अ॑सि॒ चक्षु॑र्मे पाहि चि॒त्पति॑स्त्वा पुनातु वा॒क्पति॑स्त्वा पुनातु दे॒वस्त्वा॑ सवि॒ता पु॑ना॒त्वच्छि॑द्रेण प॒वित्रे॑ण॒ वसोः॒ सूर्य॑स्य र॒श्मिभि॒स्तस्य॑ ते पवित्रपते प॒वित्रे॑ण॒ यस्मै॒ कं पु॒ने तच्छ॑केय॒मा वो॑ देवास ईमहे॒ सत्य॑धर्माणो अद्ध्व॒रे यद्वो॑ देवास आगु॒रे यज्ञि॑यासो॒ हवा॑मह॒ इन्द्रा᳚ग्नी॒ द्यावा॑पृथिवी॒ आप॑ ओषधी॒ ( ) स्त्वं दी॒क्षाणा॒-मधि॑पतिरसी॒ह मा॒ सन्तं॑ पाहि ॥ \newline

\textbf{Pada Paata} \newline

मयि॑ । धे॒हि॒ । वृ॒त्रस्य॑ । क॒नीनि॑का । अ॒सि॒ । च॒क्षु॒ष्पा इति॑ चक्षुः - पाः । अ॒सि॒ । चक्षुः॑ । मे॒ । पा॒हि॒ । चि॒त्पति॒रिति॑ चित् - पतिः॑ । त्वा॒ । पु॒ना॒तु॒ । वा॒क्पति॒रिति॑ वाक् - पतिः॑ । त्वा॒ । पु॒ना॒तु॒ । दे॒वः । त्वा॒ । स॒वि॒ता । पु॒ना॒तु॒ । अच्छि॑द्रेण । प॒वित्रे॑ण । वसोः᳚ । सूर्य॑स्य । र॒श्मिभि॒रिति॑ र॒श्मि - भिः॒ । तस्य॑ । ते॒ । प॒वि॒त्र॒प॒त॒ इति॑ पवित्र -प॒ते॒ । प॒वित्रे॑ण । यस्मै᳚ । कम् । पु॒ने । तत् । श॒के॒य॒म् । एति॑ । वः॒ । दे॒वा॒सः॒ । ई॒म॒हे॒ । सत्य॑धर्माण॒ इति॒ सत्य॑ - ध॒र्मा॒णः॒ । अ॒द्ध्व॒रे । यत् । वः॒ । दे॒वा॒सः॒ । आ॒गु॒र इत्या᳚ - गु॒रे । यज्ञि॑यासः । हवा॑महे । इन्द्रा᳚ग्नी॒ इतीन्द्र॑ - अ॒ग्नी॒ । द्यावा॑ पृथिवी॒ इति॒ द्यावा᳚ - पृ॒थि॒वी॒ । आपः॑ । ओ॒ष॒धीः॒ ( ) । त्वम् । दी॒क्षाणा᳚म् । अधि॑पति॒रित्यधि॑ - प॒तिः॒ । अ॒सि॒ । इ॒ह । मा॒ । सन्त᳚म् । पा॒हि॒ ॥  \newline


\textbf{Krama Paata} \newline

मयि॑ धेहि । धे॒हि॒ वृ॒त्रस्य॑ । वृ॒त्रस्य॑ क॒नीनि॑का । क॒नीनि॑काऽसि । अ॒सि॒ च॒क्षु॒ष्पाः । च॒क्षु॒ष्पा अ॑सि । च॒क्षु॒ष्पा इति॑ चक्षुः - पाः । अ॒सि॒ चक्षुः॑ । चक्षु॑र् मे । मे॒ पा॒हि॒ । पा॒हि॒ चि॒त्पतिः॑ । चि॒त्पति॑स्त्वा । चि॒त्पति॒रिति॑ चित् - पतिः॑ । त्वा॒ पु॒ना॒तु॒ । पु॒ना॒तु॒ वा॒क्पतिः॑ । वा॒क्पति॑स्त्वा । वा॒क्पति॒रिति॑ वाक् - पतिः॑ । त्वा॒ पु॒ना॒तु॒ । पु॒ना॒तु॒ दे॒वः । दे॒वस्त्वा᳚ । त्वा॒ स॒वि॒ता । स॒वि॒ता पु॑नातु । पु॒ना॒त्वच्छि॑द्रेण । अच्छि॑द्रेण प॒वित्रे॑ण । प॒वित्रे॑ण॒ वसोः᳚ । वसोः॒ सूर्य॑स्य । सूर्य॑स्य र॒श्मिभिः॑ । र॒श्मिभि॒स्तस्य॑ । र॒श्मिभि॒रिति॑ र॒श्मि - भिः॒ । तस्य॑ ते । ते॒ प॒वि॒त्र॒प॒ते॒ । प॒वि॒त्र॒प॒ते॒ प॒वित्रे॑ण । प॒वि॒त्र॒प॒त॒ इति॑ पवित्र - प॒ते॒ । प॒वित्रे॑ण॒ यस्मै᳚ । यस्मै॒ कम् । कम् पु॒ने । पु॒ने तत् । तच्छ॑केयम् । श॒के॒य॒मा । आ वः॑ । वो॒ दे॒वा॒सः॒ । दे॒वा॒स॒ ई॒म॒हे॒ । ई॒म॒हे॒ सत्य॑धर्माणः । सत्य॑धर्माणो अद्ध्व॒रे । सत्य॑धर्माण॒ इति॒ सत्य॑ - ध॒र्मा॒णः॒ । अ॒द्ध्व॒रे यत् । यद्वः॑ । वो॒ दे॒वा॒सः॒ । दे॒वा॒स॒ आ॒गु॒रे । आ॒गु॒रे यज्ञि॑यासः । आ॒गु॒र इत्या᳚ - गु॒रे । यज्ञि॑यासो॒ हवा॑महे । हवा॑मह॒ इन्द्रा᳚ग्नी । इन्द्रा᳚ग्नी॒ द्यावा॑पृथिवी । इन्द्रा᳚ग्नी॒ इतीन्द्र॑ - अ॒ग्नी॒ । द्यावा॑पृथिवी॒ आपः॑ । द्यावा॑पृथिवी॒ इति॒ द्यावा᳚ - पृ॒थि॒वी॒ । 
आप॑ ओषधीः ( ) । ओ॒ष॒धी॒स्त्वम् । त्वम् दी॒क्षाणा᳚म् । दी॒क्षाणा॒मधि॑पतिः । अधि॑पतिरसि । अधि॑पति॒रित्यधि॑ - प॒तिः॒ । 
अ॒सी॒ह । इ॒ह मा᳚ । मा॒ सन्त᳚म् । सन्त॑म् पाहि । पा॒हीति॑ पाहि । \newline

\textbf{Jatai Paata} \newline

1. मयि॑ धेहि धेहि॒ मयि॒ मयि॑ धेहि । \newline
2. धे॒हि॒ वृ॒त्रस्य॑ वृ॒त्रस्य॑ धेहि धेहि वृ॒त्रस्य॑ । \newline
3. वृ॒त्रस्य॑ क॒नीनि॑का क॒नीनि॑का वृ॒त्रस्य॑ वृ॒त्रस्य॑ क॒नीनि॑का । \newline
4. क॒नीनि॑का ऽस्यसि क॒नीनि॑का क॒नीनि॑का ऽसि । \newline
5. अ॒सि॒ च॒क्षु॒ष्पा श्च॑क्षु॒ष्पा अ॑स्यसि चक्षु॒ष्पाः । \newline
6. च॒क्षु॒ष्पा अ॑स्यसि चक्षु॒ष्पा श्च॑क्षु॒ष्पा अ॑सि । \newline
7. च॒क्षु॒ष्पा इति॑ चक्षुः - पाः । \newline
8. अ॒सि॒ चक्षु॒ श्चक्षु॑ रस्यसि॒ चक्षुः॑ । \newline
9. चक्षु॑र् मे मे॒ चक्षु॒ श्चक्षु॑र् मे । \newline
10. मे॒ पा॒हि॒ पा॒हि॒ मे॒ मे॒ पा॒हि॒ । \newline
11. पा॒हि॒ चि॒त्पति॑ श्चि॒त्पतिः॑ पाहि पाहि चि॒त्पतिः॑ । \newline
12. चि॒त्पति॑ स्त्वा त्वा चि॒त्पति॑ श्चि॒त्पति॑ स्त्वा । \newline
13. चि॒त्पति॒रिति॑ चित् - पतिः॑ । \newline
14. त्वा॒ पु॒ना॒तु॒ पु॒ना॒तु॒ त्वा॒ त्वा॒ पु॒ना॒तु॒ । \newline
15. पु॒ना॒तु॒ वा॒क्पति॑र् वा॒क्पतिः॑ पुनातु पुनातु वा॒क्पतिः॑ । \newline
16. वा॒क्पति॑ स्त्वा त्वा वा॒क्पति॑र् वा॒क्पति॑ स्त्वा । \newline
17. वा॒क्पति॒रिति॑ वाक् - पतिः॑ । \newline
18. त्वा॒ पु॒ना॒तु॒ पु॒ना॒तु॒ त्वा॒ त्वा॒ पु॒ना॒तु॒ । \newline
19. पु॒ना॒तु॒ दे॒वो दे॒वः पु॑नातु पुनातु दे॒वः । \newline
20. दे॒व स्त्वा᳚ त्वा दे॒वो दे॒व स्त्वा᳚ । \newline
21. त्वा॒ स॒वि॒ता स॑वि॒ता त्वा᳚ त्वा सवि॒ता । \newline
22. स॒वि॒ता पु॑नातु पुनातु सवि॒ता स॑वि॒ता पु॑नातु । \newline
23. पु॒ना॒ त्वच्छि॑द्रे॒ णाच्छि॑द्रेण पुनातु पुना॒ त्वच्छि॑द्रेण । \newline
24. अच्छि॑द्रेण प॒वित्रे॑ण प॒वित्रे॒ णाच्छि॑द्रे॒ णाच्छि॑द्रेण प॒वित्रे॑ण । \newline
25. प॒वित्रे॑ण॒ वसो॒र् वसोः᳚ प॒वित्रे॑ण प॒वित्रे॑ण॒ वसोः᳚ । \newline
26. वसोः॒ सूर्य॑स्य॒ सूर्य॑स्य॒ वसो॒र् वसोः॒ सूर्य॑स्य । \newline
27. सूर्य॑स्य र॒श्मिभी॑ र॒श्मिभिः॒ सूर्य॑स्य॒ सूर्य॑स्य र॒श्मिभिः॑ । \newline
28. र॒श्मिभि॒स्तस्य॒ तस्य॑ र॒श्मिभी॑ र॒श्मिभि॒स्तस्य॑ । \newline
29. र॒श्मिभि॒रिति॑ र॒श्मि - भिः॒ । \newline
30. तस्य॑ ते ते॒ तस्य॒ तस्य॑ ते । \newline
31. ते॒ प॒वि॒त्र॒प॒ते॒ प॒वि॒त्र॒प॒ते॒ ते॒ ते॒ प॒वि॒त्र॒प॒ते॒ । \newline
32. प॒वि॒त्र॒प॒ते॒ प॒वित्रे॑ण प॒वित्रे॑ण पवित्रपते पवित्रपते प॒वित्रे॑ण । \newline
33. प॒वि॒त्र॒प॒त॒ इति॑ पवित्र - प॒ते॒ । \newline
34. प॒वित्रे॑ण॒ यस्मै॒ यस्मै॑ प॒वित्रे॑ण प॒वित्रे॑ण॒ यस्मै᳚ । \newline
35. यस्मै॒ कम् कं ॅयस्मै॒ यस्मै॒ कम् । \newline
36. कम् पु॒ने पु॒ने कम् कम् पु॒ने । \newline
37. पु॒ने तत् तत् पु॒ने पु॒ने तत् । \newline
38. तच्छ॑केयꣳ शकेय॒म् तत् तच्छ॑केयम् । \newline
39. श॒के॒य॒ मा श॑केयꣳ शकेय॒ मा । \newline
40. आ वो॑ व॒ आ वः॑ । \newline
41. वो॒ दे॒वा॒सो॒ दे॒वा॒सो॒वो॒ वो॒ दे॒वा॒सः॒ । \newline
42. दे॒वा॒स॒ ई॒म॒ह॒ ई॒म॒हे॒ दे॒वा॒सो॒ दे॒वा॒स॒ ई॒म॒हे॒ । \newline
43. ई॒म॒हे॒ सत्य॑धर्माणः॒ सत्य॑धर्माण ईमह ईमहे॒ सत्य॑धर्माणः । \newline
44. सत्य॑धर्माणो अद्ध्व॒रे अ॑द्ध्व॒रे सत्य॑धर्माणः॒ सत्य॑धर्माणो अद्ध्व॒रे । \newline
45. सत्य॑धर्माण॒ इति॒ सत्य॑ - ध॒र्मा॒णः॒ । \newline
46. अ॒द्ध्व॒रे यद् यद॑द्ध्व॒रे अ॑द्ध्व॒रे यत् । \newline
47. यद् वो॑ वो॒ यद् यद् वः॑ । \newline
48. वो॒ दे॒वा॒सो॒ दे॒वा॒सो॒ वो॒ वो॒ दे॒वा॒सः॒ । \newline
49. दे॒वा॒स॒ आ॒गु॒र आ॑गु॒रे दे॑वासो देवास आगु॒रे । \newline
50. आ॒गु॒रे यज्ञि॑यासो॒ यज्ञि॑यास आगु॒र आ॑गु॒रे यज्ञि॑यासः । \newline
51. आ॒गु॒र इत्या᳚ - गु॒रे । \newline
52. यज्ञि॑यासो॒ हवा॑महे॒ हवा॑महे॒ यज्ञि॑यासो॒ यज्ञि॑यासो॒ हवा॑महे । \newline
53. हवा॑मह॒ इन्द्रा᳚ग्नी॒ इन्द्रा᳚ग्नी॒ हवा॑महे॒ हवा॑मह॒ इन्द्रा᳚ग्नी । \newline
54. इन्द्रा᳚ग्नी॒ द्यावा॑पृथिवी॒ द्यावा॑पृथिवी॒ इन्द्रा᳚ग्नी॒ इन्द्रा᳚ग्नी॒ द्यावा॑पृथिवी । \newline
55. इन्द्रा᳚ग्नी॒ इतीन्द्र॑ - अ॒ग्नी॒ । \newline
56. द्यावा॑पृथिवी॒ आप॒ आपो॒ द्यावा॑पृथिवी॒ द्यावा॑पृथिवी॒ आपः॑ । \newline
57. द्यावा॑पृथिवी॒ इति॒ द्यावा᳚ - पृ॒थि॒वी॒ । \newline
58. आप॑ ओषधी रोषधी॒ राप॒ आप॑ ओषधीः । \newline
59. ओ॒ष॒धी॒ स्त्वम् त्व मो॑षधी रोषधी॒ स्त्वम् । \newline
60. त्वम् दी॒क्षाणा᳚म् दी॒क्षाणा॒म् त्वम् त्वम् दी॒क्षाणा᳚म् । \newline
61. दी॒क्षाणा॒ मधि॑पति॒रधि॑पतिर् दी॒क्षाणा᳚म् दी॒क्षाणा॒ मधि॑पतिः । \newline
62. अधि॑पति रस्य॒ स्यधि॑पति॒ रधि॑पति रसि । \newline
63. अधि॑पति॒रित्यधि॑ - प॒तिः॒ । \newline
64. अ॒सी॒हे हास्य॑सी॒ह । \newline
65. इ॒ह मा॑ मे॒हे ह मा᳚ । \newline
66. मा॒ सन्त॒(ग्म्॒) सन्त॑म् मा मा॒ सन्त᳚म् । \newline
67. सन्त॑म् पाहि पाहि॒ सन्त॒(ग्म्॒) सन्त॑म् पाहि । \newline
68. पा॒हीति॑ पाहि । \newline

\textbf{Ghana Paata } \newline

1. मयि॑ धेहि धेहि॒ मयि॒ मयि॑ धेहि वृ॒त्रस्य॑ वृ॒त्रस्य॑ धेहि॒ मयि॒ मयि॑ धेहि वृ॒त्रस्य॑ । \newline
2. धे॒हि॒ वृ॒त्रस्य॑ वृ॒त्रस्य॑ धेहि धेहि वृ॒त्रस्य॑ क॒नीनि॑का क॒नीनि॑का वृ॒त्रस्य॑ धेहि धेहि वृ॒त्रस्य॑ क॒नीनि॑का । \newline
3. वृ॒त्रस्य॑ क॒नीनि॑का क॒नीनि॑का वृ॒त्रस्य॑ वृ॒त्रस्य॑ क॒नीनि॑का ऽस्यसि क॒नीनि॑का वृ॒त्रस्य॑ वृ॒त्रस्य॑ क॒नीनि॑का ऽसि । \newline
4. क॒नीनि॑का ऽस्यसि क॒नीनि॑का क॒नीनि॑का ऽसि चक्षु॒ष्पाश्च॑क्षु॒ष्पा अ॑सि क॒नीनि॑का क॒नीनि॑का ऽसि चक्षु॒ष्पाः । \newline
5. अ॒सि॒ च॒क्षु॒ष्पाश्च॑क्षु॒ष्पा अ॑स्यसि चक्षु॒ष्पा अ॑स्यसि चक्षु॒ष्पा अ॑स्यसि चक्षु॒ष्पा अ॑सि । \newline
6. च॒क्षु॒ष्पा अ॑स्यसि चक्षु॒ष्पाश्च॑क्षु॒ष्पा अ॑सि॒ चक्षु॒श्चक्षु॑रसि चक्षु॒ष्पाश्च॑क्षु॒ष्पा अ॑सि॒ चक्षुः॑ । \newline
7. च॒क्षु॒ष्पा इति॑ चक्षुः - पाः । \newline
8. अ॒सि॒ चक्षु॒श्चक्षु॑ रस्यसि॒ चक्षु॑र् मे मे॒ चक्षु॑ रस्यसि॒ चक्षु॑र् मे । \newline
9. चक्षु॑र् मे मे॒ चक्षु॒श्चक्षु॑र् मे पाहि पाहि मे॒ चक्षु॒श्चक्षु॑र् मे पाहि । \newline
10. मे॒ पा॒हि॒ पा॒हि॒ मे॒ मे॒ पा॒हि॒ चि॒त्पति॑ श्चि॒त्पतिः॑ पाहि मे मे पाहि चि॒त्पतिः॑ । \newline
11. पा॒हि॒ चि॒त्पति॑ श्चि॒त्पतिः॑ पाहि पाहि चि॒त्पति॑ स्त्वा त्वा चि॒त्पतिः॑ पाहि पाहि चि॒त्पति॑ स्त्वा । \newline
12. चि॒त्पति॑स्त्वा त्वा चि॒त्पति॑ श्चि॒त्पति॑ स्त्वा पुनातु पुनातु त्वा चि॒त्पति॑ श्चि॒त्पति॑ स्त्वा पुनातु । \newline
13. चि॒त्पति॒रिति॑ चित् - पतिः॑ । \newline
14. त्वा॒ पु॒ना॒तु॒ पु॒ना॒तु॒ त्वा॒ त्वा॒ पु॒ना॒तु॒ वा॒क्पति॑र् वा॒क्पतिः॑ पुनातु त्वा त्वा पुनातु वा॒क्पतिः॑ । \newline
15. पु॒ना॒तु॒ वा॒क्पति॑र् वा॒क्पतिः॑ पुनातु पुनातु वा॒क्पति॑स्त्वा त्वा वा॒क्पतिः॑ पुनातु पुनातु वा॒क्पति॑स्त्वा । \newline
16. वा॒क्पति॑स्त्वा त्वा वा॒क्पति॑र् वा॒क्पति॑स्त्वा पुनातु पुनातु त्वा वा॒क्पति॑र् वा॒क्पति॑स्त्वा पुनातु । \newline
17. वा॒क्पति॒रिति॑ वाक् - पतिः॑ । \newline
18. त्वा॒ पु॒ना॒तु॒ पु॒ना॒तु॒ त्वा॒ त्वा॒ पु॒ना॒तु॒ दे॒वो दे॒वः पु॑नातु त्वा त्वा पुनातु दे॒वः । \newline
19. पु॒ना॒तु॒ दे॒वो दे॒वः पु॑नातु पुनातु दे॒वस्त्वा᳚ त्वा दे॒वः पु॑नातु पुनातु दे॒वस्त्वा᳚ । \newline
20. दे॒वस्त्वा᳚ त्वा दे॒वो दे॒वस्त्वा॑ सवि॒ता स॑वि॒ता त्वा॑ दे॒वो दे॒वस्त्वा॑ सवि॒ता । \newline
21. त्वा॒ स॒वि॒ता स॑वि॒ता त्वा᳚ त्वा सवि॒ता पु॑नातु पुनातु सवि॒ता त्वा᳚ त्वा सवि॒ता पु॑नातु । \newline
22. स॒वि॒ता पु॑नातु पुनातु सवि॒ता स॑वि॒ता पु॑ना॒त्वच्छि॑द्रे॒णाच्छि॑द्रेण पुनातु सवि॒ता स॑वि॒ता पु॑ना॒त्वच्छि॑द्रेण । \newline
23. पु॒ना॒त्वच्छि॑द्रे॒णाच्छि॑द्रेण पुनातु पुना॒त्वच्छि॑द्रेण प॒वित्रे॑ण प॒वित्रे॒णाच्छि॑द्रेण पुनातु पुना॒त्वच्छि॑द्रेण प॒वित्रे॑ण । \newline
24. अच्छि॑द्रेण प॒वित्रे॑ण प॒वित्रे॒णाच्छि॑द्रे॒णाच्छि॑द्रेण प॒वित्रे॑ण॒ वसो॒र् वसोः᳚ प॒वित्रे॒णाच्छि॑द्रे॒णाच्छि॑द्रेण प॒वित्रे॑ण॒ वसोः᳚ । \newline
25. प॒वित्रे॑ण॒ वसो॒र् वसोः᳚ प॒वित्रे॑ण प॒वित्रे॑ण॒ वसोः॒ सूर्य॑स्य॒ सूर्य॑स्य॒ वसोः᳚ प॒वित्रे॑ण प॒वित्रे॑ण॒ वसोः॒ सूर्य॑स्य । \newline
26. वसोः॒ सूर्य॑स्य॒ सूर्य॑स्य॒ वसो॒र् वसोः॒ सूर्य॑स्य र॒श्मिभी॑ र॒श्मिभिः॒ सूर्य॑स्य॒ वसो॒र् वसोः॒ सूर्य॑स्य र॒श्मिभिः॑ । \newline
27. सूर्य॑स्य र॒श्मिभी॑ र॒श्मिभिः॒ सूर्य॑स्य॒ सूर्य॑स्य र॒श्मिभि॒ स्तस्य॒ तस्य॑ र॒श्मिभिः॒ सूर्य॑स्य॒ सूर्य॑स्य र॒श्मिभि॒ स्तस्य॑ । \newline
28. र॒श्मिभि॒ स्तस्य॒ तस्य॑ र॒श्मिभी॑ र॒श्मिभि॒ स्तस्य॑ ते ते॒ तस्य॑ र॒श्मिभी॑ र॒श्मिभि॒ स्तस्य॑ ते । \newline
29. र॒श्मिभि॒रिति॑ र॒श्मि - भिः॒ । \newline
30. तस्य॑ ते ते॒ तस्य॒ तस्य॑ ते पवित्रपते पवित्रपते ते॒ तस्य॒ तस्य॑ ते पवित्रपते । \newline
31. ते॒ प॒वि॒त्र॒प॒ते॒ प॒वि॒त्र॒प॒ते॒ ते॒ ते॒ प॒वि॒त्र॒प॒ते॒ प॒वित्रे॑ण प॒वित्रे॑ण पवित्रपते ते ते पवित्रपते प॒वित्रे॑ण । \newline
32. प॒वि॒त्र॒प॒ते॒ प॒वित्रे॑ण प॒वित्रे॑ण पवित्रपते पवित्रपते प॒वित्रे॑ण॒ यस्मै॒ यस्मै॑ प॒वित्रे॑ण पवित्रपते पवित्रपते प॒वित्रे॑ण॒ यस्मै᳚ । \newline
33. प॒वि॒त्र॒प॒त॒ इति॑ पवित्र - प॒ते॒ । \newline
34. प॒वित्रे॑ण॒ यस्मै॒ यस्मै॑ प॒वित्रे॑ण प॒वित्रे॑ण॒ यस्मै॒ कम् कं ॅयस्मै॑ प॒वित्रे॑ण प॒वित्रे॑ण॒ यस्मै॒ कम् । \newline
35. यस्मै॒ कम् कं ॅयस्मै॒ यस्मै॒ कम् पु॒ने पु॒ने कं ॅयस्मै॒ यस्मै॒ कम् पु॒ने । \newline
36. कम् पु॒ने पु॒ने कम् कम् पु॒ने तत् तत् पु॒ने कम् कम् पु॒ने तत् । \newline
37. पु॒ने तत् तत् पु॒ने पु॒ने तच्छ॑केयꣳ शकेय॒म् तत् पु॒ने पु॒ने तच्छ॑केयम् । \newline
38. तच्छ॑केयꣳ शकेय॒म् तत् तच्छ॑केय॒ मा श॑केय॒म् तत् तच्छ॑केय॒ मा । \newline
39. श॒के॒य॒ मा श॑केयꣳ शकेय॒ मा वो॑ व॒ आ श॑केयꣳ शकेय॒ मा वः॑ । \newline
40. आ वो॑ व॒ आ वो॑ देवासो देवासोव॒ आ वो॑ देवासः । \newline
41. वो॒ दे॒वा॒सो॒ दे॒वा॒सो॒वो॒ वो॒ दे॒वा॒स॒ ई॒म॒ह॒ ई॒म॒हे॒ दे॒वा॒सो॒वो॒ वो॒ दे॒वा॒स॒ ई॒म॒हे॒ । \newline
42. दे॒वा॒स॒ ई॒म॒ह॒ ई॒म॒हे॒ दे॒वा॒सो॒ दे॒वा॒स॒ ई॒म॒हे॒ सत्य॑धर्माणः॒ सत्य॑धर्माण ईमहे देवासो देवास ईमहे॒ सत्य॑धर्माणः । \newline
43. ई॒म॒हे॒ सत्य॑धर्माणः॒ सत्य॑धर्माण ईमह ईमहे॒ सत्य॑धर्माणो अद्ध्व॒रे अ॑द्ध्व॒रे सत्य॑धर्माण ईमह ईमहे॒ सत्य॑धर्माणो अद्ध्व॒रे । \newline
44. सत्य॑धर्माणो अद्ध्व॒रे अ॑द्ध्व॒रे सत्य॑धर्माणः॒ सत्य॑धर्माणो अद्ध्व॒रे यद् यद॑द्ध्व॒रे सत्य॑धर्माणः॒ सत्य॑धर्माणो अद्ध्व॒रे यत् । \newline
45. सत्य॑धर्माण॒ इति॒ सत्य॑ - ध॒र्मा॒णः॒ । \newline
46. अ॒द्ध्व॒रे यद् यद॑द्ध्व॒रे अ॑द्ध्व॒रे यद् वो॑ वो॒ यद॑द्ध्व॒रे अ॑द्ध्व॒रे यद् वः॑ । \newline
47. यद् वो॑ वो॒ यद् यद् वो॑ देवासो देवासो वो॒ यद् यद् वो॑ देवासः । \newline
48. वो॒ दे॒वा॒सो॒ दे॒वा॒सो॒वो॒ वो॒ दे॒वा॒स॒ आ॒गु॒र आ॑गु॒रे दे॑वासोवो वो देवास आगु॒रे । \newline
49. दे॒वा॒स॒ आ॒गु॒र आ॑गु॒रे दे॑वासो देवास आगु॒रे यज्ञि॑यासो॒ यज्ञि॑यास आगु॒रे दे॑वासो देवास आगु॒रे यज्ञि॑यासः । \newline
50. आ॒गु॒रे यज्ञि॑यासो॒ यज्ञि॑यास आगु॒र आ॑गु॒रे यज्ञि॑यासो॒ हवा॑महे॒ हवा॑महे॒ यज्ञि॑यास आगु॒र आ॑गु॒रे यज्ञि॑यासो॒ हवा॑महे । \newline
51. आ॒गु॒र इत्या᳚ - गु॒रे । \newline
52. यज्ञि॑यासो॒ हवा॑महे॒ हवा॑महे॒ यज्ञि॑यासो॒ यज्ञि॑यासो॒ हवा॑मह॒ इन्द्रा᳚ग्नी॒ इन्द्रा᳚ग्नी॒ हवा॑महे॒ यज्ञि॑यासो॒ यज्ञि॑यासो॒ हवा॑मह॒ इन्द्रा᳚ग्नी । \newline
53. हवा॑मह॒ इन्द्रा᳚ग्नी॒ इन्द्रा᳚ग्नी॒ हवा॑महे॒ हवा॑मह॒ इन्द्रा᳚ग्नी॒ द्यावा॑पृथिवी॒ द्यावा॑पृथिवी॒ इन्द्रा᳚ग्नी॒ हवा॑महे॒ हवा॑मह॒ इन्द्रा᳚ग्नी॒ द्यावा॑पृथिवी । \newline
54. इन्द्रा᳚ग्नी॒ द्यावा॑पृथिवी॒ द्यावा॑पृथिवी॒ इन्द्रा᳚ग्नी॒ इन्द्रा᳚ग्नी॒ द्यावा॑पृथिवी॒ आप॒ आपो॒ द्यावा॑पृथिवी॒ इन्द्रा᳚ग्नी॒ इन्द्रा᳚ग्नी॒ द्यावा॑पृथिवी॒ आपः॑ । \newline
55. इन्द्रा᳚ग्नी॒ इतीन्द्र॑ - अ॒ग्नी॒ । \newline
56. द्यावा॑पृथिवी॒ आप॒ आपो॒ द्यावा॑पृथिवी॒ द्यावा॑पृथिवी॒ आप॑ ओषधीरोषधी॒रापो॒ द्यावा॑पृथिवी॒ द्यावा॑पृथिवी॒ आप॑ ओषधीः । \newline
57. द्यावा॑पृथिवी॒ इति॒ द्यावा᳚ - पृ॒थि॒वी॒ । \newline
58. आप॑ ओषधीरोषधी॒राप॒ आप॑ ओषधी॒स्त्वम् त्व मो॑षधी॒राप॒ आप॑ ओषधी॒स्त्वम् । \newline
59. ओ॒ष॒धी॒स्त्वम् त्व मो॑षधीरोषधी॒स्त्वम् दी॒क्षाणा᳚म् दी॒क्षाणा॒म् त्व मो॑षधीरोषधी॒स्त्वम् दी॒क्षाणा᳚म् । \newline
60. त्वम् दी॒क्षाणा᳚म् दी॒क्षाणा॒म् त्वम् त्वम् दी॒क्षाणा॒ मधि॑पति॒रधि॑पतिर् दी॒क्षाणा॒म् त्वम् त्वम् दी॒क्षाणा॒ मधि॑पतिः । \newline
61. दी॒क्षाणा॒ मधि॑पति॒ रधि॑पतिर् दी॒क्षाणा᳚म् दी॒क्षाणा॒ मधि॑पति रस्य॒स्यधि॑पतिर् दी॒क्षाणा᳚म् दी॒क्षाणा॒ मधि॑पतिरसि । \newline
62. अधि॑पति रस्य॒स्यधि॑पति॒ रधि॑पतिरसी॒हे हास्यधि॑पति॒ रधि॑पति रसी॒ह । \newline
63. अधि॑पति॒रित्यधि॑ - प॒तिः॒ । \newline
64. अ॒सी॒हे हास्य॑सी॒ह मा॑ मे॒हास्य॑सी॒ह मा᳚ । \newline
65. इ॒ह मा॑ मे॒हे ह मा॒ सन्त॒(ग्म्॒) सन्त॑म् मे॒हे ह मा॒ सन्त᳚म् । \newline
66. मा॒ सन्त॒(ग्म्॒) सन्त॑म् मा मा॒ सन्त॑म् पाहि पाहि॒ सन्त॑म् मा मा॒ सन्त॑म् पाहि । \newline
67. सन्त॑म् पाहि पाहि॒ सन्त॒(ग्म्॒) सन्त॑म् पाहि । \newline
68. पा॒हीति॑ पाहि । \newline
\pagebreak
\markright{ TS 1.2.2.1  \hfill https://www.vedavms.in \hfill}

\section{ TS 1.2.2.1 }

\textbf{TS 1.2.2.1 } \newline
\textbf{Samhita Paata} \newline

आकू᳚त्यै प्र॒युजे॒ऽग्नये॒ स्वाहा॑ मे॒धायै॒ मन॑से॒ ऽग्नये॒ स्वाहा॑ दी॒क्षायै॒ तप॑से॒ऽग्नये॒ स्वाहा॒ सर॑स्वत्यै पू॒ष्णे᳚ऽग्नये॒ स्वाहाऽऽपो॑ देवीर् बृहतीर् विश्वशंभुवो॒ द्यावा॑पृथि॒वी उ॒र्व॑न्तरि॑क्षं॒ बृह॒स्पति॑र्नो ह॒विषा॑ वृधातु॒ स्वाहा॒ विश्वे॑ दे॒वस्य॑ ने॒तुर्मर्तो॑ वृणीत स॒ख्यं ॅविश्वे॑ रा॒य इ॑षुद्ध्यसि द्यु॒म्नं ॅवृ॑णीत पु॒ष्यसे॒ स्वाह॑र्.ख्सा॒मयोः॒ शिल्पे᳚ स्थ॒स्ते वा॒मा र॑भे॒ ते मा॑-[ ] \newline

\textbf{Pada Paata} \newline

आकू᳚त्या॒ इत्या - कू॒त्यै॒ । प्र॒युज॒ इति॑ प्र - युजे᳚ । अ॒ग्नये᳚ । स्वाहा᳚ । मे॒धायै᳚ । मन॑से । अ॒ग्नये᳚ । स्वाहा᳚ । दी॒क्षायै᳚ । तप॑से । अ॒ग्नये᳚ । स्वाहा᳚ । सर॑स्वत्यै । पू॒ष्णे । अ॒ग्नये᳚ । स्वाहा᳚ । आपः॑ । दे॒वीः॒ । बृ॒ह॒तीः॒ । वि॒श्व॒शं॒भु॒व॒ इति॑ विश्व - शं॒भु॒वः॒ । द्यावा॑पृथि॒वी इति॒ द्यावा᳚ - पृ॒थि॒वी । उ॒रु । अ॒न्तरि॑क्षम् । बृह॒स्पतिः॑ । नः॒ । ह॒विषा᳚ । वृ॒धा॒तु॒ । स्वाहा᳚ । विश्वे᳚ । दे॒वस्य॑ । ने॒तुः । मर्तः॑ । वृ॒णी॒त॒ । स॒ख्यम् । विश्वे᳚ । रा॒यः । इ॒षु॒द्ध्य॒सि॒ । द्यु॒म्नम् । वृ॒णी॒त॒ । पु॒ष्यसे᳚ । स्वाहा᳚ । ऋ॒ख्सा॒मयो॒रित्यृ॑क् -सा॒मयोः᳚ । शिल्पे॒ इति॑ । स्थः॒ । ते इति॑ । वा॒म् । एति॑ । र॒भे॒ । ते इति॑ । मा॒ ।  \newline


\textbf{Krama Paata} \newline

आकू᳚त्यै प्र॒युजे᳚ । आकू᳚त्या॒ इत्या - कू॒त्यै॒ । प्र॒युजे॒ ऽग्नये᳚ । 
प्र॒युज॒ इति॑ प्र - युजे᳚ । अ॒ग्नये॒ स्वाहा᳚ । स्वाहा॑ मे॒धायै᳚ । मे॒धायै॒ मन॑से । मन॑से॒ ऽग्नये᳚ । अ॒ग्नये॒ स्वाहा᳚ । स्वाहा॑ दी॒क्षायै᳚ । दी॒क्षायै॒ तप॑से । तप॑से॒ ऽग्नये᳚ । अ॒ग्नये॒ स्वाहा᳚ । स्वाहा॒ सर॑स्वत्यै । सर॑स्वत्यै पू॒ष्णे । पू॒ष्णे᳚ ऽग्नये᳚ । 
अ॒ग्नये॒ स्वाहा᳚ । स्वाहा ऽऽपः॑ । आपो॑ देवीः । दे॒वी॒र् बृ॒ह॒तीः॒ । बृ॒ह॒ती॒र् वि॒श्व॒श॒म्भु॒वः॒ । वि॒श्व॒श॒म्भु॒वो॒ द्यावा॑पृथि॒वी । वि॒श्व॒श॒म्भु॒व॒ इति॑ विश्व - श॒म्भु॒वः॒ । द्यावा॑पृथि॒वी उ॒रु । द्यावा॑पृथि॒वी इति॒ द्यावा᳚ - पृ॒थि॒वी । उ॒र् व॑न्तरि॑क्षम् । अ॒न्तरि॑क्ष॒म् बृह॒स्पतिः॑ । बृह॒स्पति॑र् नः । नो॒ ह॒विषा᳚ । ह॒विषा॑ वृधातु । वृ॒धा॒तु॒ स्वाहा᳚ । स्वाहा॒ विश्वे᳚ । विश्वे॑ दे॒वस्य॑ । दे॒वस्य॑ ने॒तुः । ने॒तुर् मर्तः॑ । मर्तो॑ वृणीत । वृ॒णी॒त॒ स॒ख्यम् । स॒ख्यं ॅविश्वे᳚ । विश्वे॑ रा॒यः । रा॒य इ॑षुद्ध्यसि । इ॒षु॒द्ध्य॒सि॒ द्यु॒म्नम् । द्यु॒म्नं ॅवृ॑णीत । वृ॒णी॒त॒ पु॒ष्यसे᳚ । पु॒ष्यसे॒ स्वाहा᳚ । स्वाह॑र्ख्सा॒मयोः᳚ । ऋ॒ख्सा॒मयोः॒ शिल्पे᳚ । ऋ॒ख्सा॒मयो॒रित्यृ॑क् - सा॒मयोः᳚ । शिल्पे᳚ स्थः । शिल्पे॒ इति॒ शिल्पे᳚ । स्थ॒स्ते । ते वा᳚म् । ते इति॒ ते । वा॒मा । आ र॑भे । र॒भे॒ ते । ते मा᳚ । ते इति॒ ते । मा॒ पा॒त॒म् \newline

\textbf{Jatai Paata} \newline

1. आकू᳚त्यै प्र॒युजे᳚ प्र॒युज॒ आकू᳚त्या॒ आकू᳚त्यै प्र॒युजे᳚ । \newline
2. आकू᳚त्या॒ इत्या - कू॒त्यै॒ । \newline
3. प्र॒युजे॒ ऽग्नये॒ ऽग्नये᳚ प्र॒युजे᳚ प्र॒युजे॒ ऽग्नये᳚ । \newline
4. प्र॒युज॒ इति॑ प्र - युजे᳚ । \newline
5. अ॒ग्नये॒ स्वाहा॒ स्वाहा॒ ऽग्नये॒ ऽग्नये॒ स्वाहा᳚ । \newline
6. स्वाहा॑ मे॒धायै॑ मे॒धायै॒ स्वाहा॒ स्वाहा॑ मे॒धायै᳚ । \newline
7. मे॒धायै॒ मन॑से॒ मन॑से मे॒धायै॑ मे॒धायै॒ मन॑से । \newline
8. मन॑से॒ ऽग्नये॒ ऽग्नये॒ मन॑से॒ मन॑से॒ ऽग्नये᳚ । \newline
9. अ॒ग्नये॒ स्वाहा॒ स्वाहा॒ ऽग्नये॒ ऽग्नये॒ स्वाहा᳚ । \newline
10. स्वाहा॑ दी॒क्षायै॑ दी॒क्षायै॒ स्वाहा॒ स्वाहा॑ दी॒क्षायै᳚ । \newline
11. दी॒क्षायै॒ तप॑से॒ तप॑से दी॒क्षायै॑ दी॒क्षायै॒ तप॑से । \newline
12. तप॑से॒ ऽग्नये॒ ऽग्नये॒ तप॑से॒ तप॑से॒ ऽग्नये᳚ । \newline
13. अ॒ग्नये॒ स्वाहा॒ स्वाहा॒ ऽग्नये॒ ऽग्नये॒ स्वाहा᳚ । \newline
14. स्वाहा॒ सर॑स्वत्यै॒ सर॑स्वत्यै॒ स्वाहा॒ स्वाहा॒ सर॑स्वत्यै । \newline
15. सर॑स्वत्यै पू॒ष्णे पू॒ष्णे सर॑स्वत्यै॒ सर॑स्वत्यै पू॒ष्णे । \newline
16. पू॒ष्णे᳚ ऽग्नये॒ ऽग्नये॑ पू॒ष्णे पू॒ष्णे᳚ ऽग्नये᳚ । \newline
17. अ॒ग्नये॒ स्वाहा॒ स्वाहा॒ ऽग्नये॒ ऽग्नये॒ स्वाहा᳚ । \newline
18. स्वाहा ऽऽप॒ आपः॒ स्वाहा॒ स्वाहा ऽऽपः॑ । \newline
19. आपो॑ देवीर् देवी॒राप॒ आपो॑ देवीः । \newline
20. दे॒वी॒र् बृ॒ह॒ती॒र् बृ॒ह॒ती॒र् दे॒वी॒र् दे॒वी॒र् बृ॒ह॒तीः॒ । \newline
21. बृ॒ह॒ती॒र् वि॒श्व॒शं॒भु॒वो॒ वि॒श्व॒शं॒भु॒वो॒ बृ॒ह॒ती॒र् बृ॒ह॒ती॒र् वि॒श्व॒शं॒भु॒वः॒ । \newline
22. वि॒श्व॒शं॒भु॒वो॒ द्यावा॑पृथि॒वी द्यावा॑पृथि॒वी वि॑श्वशंभुवो विश्वशंभुवो॒ द्यावा॑पृथि॒वी । \newline
23. वि॒श्व॒शं॒भु॒व॒ इति॑ विश्व - शं॒भु॒वः॒ । \newline
24. द्यावा॑पृथि॒वी उ॒रू॑रु द्यावा॑पृथि॒वी द्यावा॑पृथि॒वी उ॒रु । \newline
25. द्यावा॑पृथि॒वी इति॒ द्यावा᳚ - पृ॒थि॒वी । \newline
26. उ॒र्व॑न्तरि॑क्ष म॒न्तरि॑क्ष मु॒रू᳚(1॒)र्व॑न्तरि॑क्षम् । \newline
27. अ॒न्तरि॑क्ष॒म् बृह॒स्पति॒र् बृह॒स्पति॑र॒न्तरि॑क्ष म॒न्तरि॑क्ष॒म् बृह॒स्पतिः॑ । \newline
28. बृह॒स्पति॑र् नो नो॒ बृह॒स्पति॒र् बृह॒स्पति॑र् नः । \newline
29. नो॒ ह॒विषा॑ ह॒विषा॑ नो नो ह॒विषा᳚ । \newline
30. ह॒विषा॑ वृधातु वृधातु ह॒विषा॑ ह॒विषा॑ वृधातु । \newline
31. वृ॒धा॒तु॒ स्वाहा॒ स्वाहा॑ वृधातु वृधातु॒ स्वाहा᳚ । \newline
32. स्वाहा॒ विश्वे॒ विश्वे॒ स्वाहा॒ स्वाहा॒ विश्वे᳚ । \newline
33. विश्वे॑ दे॒वस्य॑ दे॒वस्य॒ विश्वे॒ विश्वे॑ दे॒वस्य॑ । \newline
34. दे॒वस्य॑ ने॒तुर् ने॒तुर् दे॒वस्य॑ दे॒वस्य॑ ने॒तुः । \newline
35. ने॒तुर् मर्तो॒ मर्तो॑ ने॒तुर् ने॒तुर् मर्तः॑ । \newline
36. मर्तो॑ वृणीत वृणीत॒ मर्तो॒ मर्तो॑ वृणीत । \newline
37. वृ॒णी॒त॒ स॒ख्यꣳ स॒ख्यं ॅवृ॑णीत वृणीत स॒ख्यम् । \newline
38. स॒ख्यं ॅविश्वे॒ विश्वे॑ स॒ख्यꣳ स॒ख्यं ॅविश्वे᳚ । \newline
39. विश्वे॑ रा॒यो रा॒यो विश्वे॒ विश्वे॑ रा॒यः । \newline
40. रा॒य इ॑षुद्ध्य सीषुद्ध्यसि रा॒यो रा॒य इ॑षुद्ध्यसि । \newline
41. इ॒षु॒द्ध्य॒सि॒ द्यु॒म्नम् द्यु॒म्न मि॑षुद्ध्य सीषुद्ध्यसि द्यु॒म्नम् । \newline
42. द्यु॒म्नं ॅवृ॑णीत वृणीत द्यु॒म्नम् द्यु॒म्नं ॅवृ॑णीत । \newline
43. वृ॒णी॒त॒ पु॒ष्यसे॑ पु॒ष्यसे॑ वृणीत वृणीत पु॒ष्यसे᳚ । \newline
44. पु॒ष्यसे॒ स्वाहा॒ स्वाहा॑ पु॒ष्यसे॑ पु॒ष्यसे॒ स्वाहा᳚ । \newline
45. स्वाह॑र्ख्सा॒मयोर्॑. ऋख्सा॒मयोः॒ स्वाहा॒ स्वाह॑र्ख्सा॒मयोः᳚ । \newline
46. ऋ॒ख्सा॒मयोः॒ शिल्पे॒ शिल्पे॑ ऋख्सा॒मयोर्॑. ऋख्सा॒मयोः॒ शिल्पे᳚ । \newline
47. ऋ॒ख्सा॒मयो॒रित्यृ॑क् - सा॒मयोः᳚ । \newline
48. शिल्पे᳚ स्थः स्थः॒ शिल्पे॒ शिल्पे᳚ स्थः । \newline
49. शिल्पे॒ इति॒ शिल्पे᳚ । \newline
50. स्थ॒ स्ते ते स्थः॑ स्थ॒ स्ते । \newline
51. ते वां᳚ ॅवा॒म् ते ते वा᳚म् । \newline
52. ते इति॒ ते । \newline
53. वा॒ मा वां᳚ ॅवा॒ मा । \newline
54. आ र॑भे रभ॒ आ र॑भे । \newline
55. र॒भे॒ ते ते र॑भे रभे॒ ते । \newline
56. ते मा॑ मा॒ ते ते मा᳚ । \newline
57. ते इति॒ ते । \newline
58. मा॒ पा॒त॒म् पा॒त॒म् मा॒ मा॒ पा॒त॒म् । \newline

\textbf{Ghana Paata } \newline

1. आकू᳚त्यै प्र॒युजे᳚ प्र॒युज॒ आकू᳚त्या॒ आकू᳚त्यै प्र॒युजे॒ ऽग्नये॒ ऽग्नये᳚ प्र॒युज॒ आकू᳚त्या॒ आकू᳚त्यै प्र॒युजे॒ ऽग्नये᳚ । \newline
2. आकू᳚त्या॒ इत्या - कू॒त्यै॒ । \newline
3. प्र॒युजे॒ ऽग्नये॒ ऽग्नये᳚ प्र॒युजे᳚ प्र॒युजे॒ ऽग्नये॒ स्वाहा॒ स्वाहा॒ ऽग्नये᳚ प्र॒युजे᳚ प्र॒युजे॒ ऽग्नये॒ स्वाहा᳚ । \newline
4. प्र॒युज॒ इति॑ प्र - युजे᳚ । \newline
5. अ॒ग्नये॒ स्वाहा॒ स्वाहा॒ ऽग्नये॒ ऽग्नये॒ स्वाहा॑ मे॒धायै॑ मे॒धायै॒ स्वाहा॒ ऽग्नये॒ ऽग्नये॒ स्वाहा॑ मे॒धायै᳚ । \newline
6. स्वाहा॑ मे॒धायै॑ मे॒धायै॒ स्वाहा॒ स्वाहा॑ मे॒धायै॒ मन॑से॒ मन॑से मे॒धायै॒ स्वाहा॒ स्वाहा॑ मे॒धायै॒ मन॑से । \newline
7. मे॒धायै॒ मन॑से॒ मन॑से मे॒धायै॑ मे॒धायै॒ मन॑से॒ ऽग्नये॒ ऽग्नये॒ मन॑से मे॒धायै॑ मे॒धायै॒ मन॑से॒ ऽग्नये᳚ । \newline
8. मन॑से॒ ऽग्नये॒ ऽग्नये॒ मन॑से॒ मन॑से॒ ऽग्नये॒ स्वाहा॒ स्वाहा॒ ऽग्नये॒ मन॑से॒ मन॑से॒ ऽग्नये॒ स्वाहा᳚ । \newline
9. अ॒ग्नये॒ स्वाहा॒ स्वाहा॒ ऽग्नये॒ ऽग्नये॒ स्वाहा॑ दी॒क्षायै॑ दी॒क्षायै॒ स्वाहा॒ ऽग्नये॒ ऽग्नये॒ स्वाहा॑ दी॒क्षायै᳚ । \newline
10. स्वाहा॑ दी॒क्षायै॑ दी॒क्षायै॒ स्वाहा॒ स्वाहा॑ दी॒क्षायै॒ तप॑से॒ तप॑से दी॒क्षायै॒ स्वाहा॒ स्वाहा॑ दी॒क्षायै॒ तप॑से । \newline
11. दी॒क्षायै॒ तप॑से॒ तप॑से दी॒क्षायै॑ दी॒क्षायै॒ तप॑से॒ ऽग्नये॒ ऽग्नये॒ तप॑से दी॒क्षायै॑ दी॒क्षायै॒ तप॑से॒ ऽग्नये᳚ । \newline
12. तप॑से॒ ऽग्नये॒ ऽग्नये॒ तप॑से॒ तप॑से॒ ऽग्नये॒ स्वाहा॒ स्वाहा॒ ऽग्नये॒ तप॑से॒ तप॑से॒ ऽग्नये॒ स्वाहा᳚ । \newline
13. अ॒ग्नये॒ स्वाहा॒ स्वाहा॒ ऽग्नये॒ ऽग्नये॒ स्वाहा॒ सर॑स्वत्यै॒ सर॑स्वत्यै॒ स्वाहा॒ ऽग्नये॒ ऽग्नये॒ स्वाहा॒ सर॑स्वत्यै । \newline
14. स्वाहा॒ सर॑स्वत्यै॒ सर॑स्वत्यै॒ स्वाहा॒ स्वाहा॒ सर॑स्वत्यै पू॒ष्णे पू॒ष्णे सर॑स्वत्यै॒ स्वाहा॒ स्वाहा॒ सर॑स्वत्यै पू॒ष्णे । \newline
15. सर॑स्वत्यै पू॒ष्णे पू॒ष्णे सर॑स्वत्यै॒ सर॑स्वत्यै पू॒ष्णे᳚ ऽग्नये॒ ऽग्नये॑ पू॒ष्णे सर॑स्वत्यै॒ सर॑स्वत्यै पू॒ष्णे᳚ ऽग्नये᳚ । \newline
16. पू॒ष्णे᳚ ऽग्नये॒ ऽग्नये॑ पू॒ष्णे पू॒ष्णे᳚ ऽग्नये॒ स्वाहा॒ स्वाहा॒ ऽग्नये॑ पू॒ष्णे पू॒ष्णे᳚ ऽग्नये॒ स्वाहा᳚ । \newline
17. अ॒ग्नये॒ स्वाहा॒ स्वाहा॒ ऽग्नये॒ ऽग्नये॒ स्वाहा ऽऽप॒ आपः॒ स्वाहा॒ ऽग्नये॒ ऽग्नये॒ स्वाहा ऽऽपः॑ । \newline
18. स्वाहा ऽऽप॒ आपः॒ स्वाहा॒ स्वाहा ऽऽपो॑ देवीर् देवी॒रापः॒ स्वाहा॒ स्वाहा ऽऽपो॑ देवीः । \newline
19. आपो॑ देवीर् देवी॒राप॒ आपो॑ देवीर् बृहतीर् बृहतीर् देवी॒राप॒ आपो॑ देवीर् बृहतीः । \newline
20. दे॒वी॒र् बृ॒ह॒ती॒र् बृ॒ह॒ती॒र् दे॒वी॒र् दे॒वी॒र् बृ॒ह॒ती॒र् वि॒श्व॒शं॒भु॒वो॒ वि॒श्व॒शं॒भु॒वो॒ बृ॒ह॒ती॒र् दे॒वी॒र् दे॒वी॒र् बृ॒ह॒ती॒र् वि॒श्व॒शं॒भु॒वः॒ । \newline
21. बृ॒ह॒ती॒र् वि॒श्व॒शं॒भु॒वो॒ वि॒श्व॒शं॒भु॒वो॒ बृ॒ह॒ती॒र् बृ॒ह॒ती॒र् वि॒श्व॒शं॒भु॒वो॒ द्यावा॑पृथि॒वी द्यावा॑पृथि॒वी वि॑श्वशंभुवो बृहतीर् बृहतीर् विश्वशंभुवो॒ द्यावा॑पृथि॒वी । \newline
22. वि॒श्व॒शं॒भु॒वो॒ द्यावा॑पृथि॒वी द्यावा॑पृथि॒वी वि॑श्वशंभुवो विश्वशंभुवो॒ द्यावा॑पृथि॒वी उ॒रू॑रु द्यावा॑पृथि॒वी वि॑श्वशंभुवो विश्वशंभुवो॒ द्यावा॑पृथि॒वी उ॒रु । \newline
23. वि॒श्व॒शं॒भु॒व॒ इति॑ विश्व - शं॒भु॒वः॒ । \newline
24. द्यावा॑पृथि॒वी उ॒रू॑रु द्यावा॑पृथि॒वी द्यावा॑पृथि॒वी उ॒र्व॑न्तरि॑क्ष म॒न्तरि॑क्ष मु॒रु द्यावा॑पृथि॒वी द्यावा॑पृथि॒वी उ॒र्व॑न्तरि॑क्षम् । \newline
25. द्यावा॑पृथि॒वी इति॒ द्यावा᳚ - पृ॒थि॒वी । \newline
26. उ॒र्व॑न्तरि॑क्ष म॒न्तरि॑क्ष मु॒रू᳚(1॒)र्व॑न्तरि॑क्ष॒म् बृह॒स्पति॒र् बृह॒स्पति॑र॒न्तरि॑क्ष मु॒रू᳚(1॒)र्व॑न्तरि॑क्ष॒म् बृह॒स्पतिः॑ । \newline
27. अ॒न्तरि॑क्ष॒म् बृह॒स्पति॒र् बृह॒स्पति॑ र॒न्तरि॑क्ष म॒न्तरि॑क्ष॒म् बृह॒स्पति॑र् नो नो॒ बृह॒स्पति॑ र॒न्तरि॑क्ष म॒न्तरि॑क्ष॒म् बृह॒स्पति॑र् नः । \newline
28. बृह॒स्पति॑र् नो नो॒ बृह॒स्पति॒र् बृह॒स्पति॑र् नो ह॒विषा॑ ह॒विषा॑ नो॒ बृह॒स्पति॒र् बृह॒स्पति॑र् नो ह॒विषा᳚ । \newline
29. नो॒ ह॒विषा॑ ह॒विषा॑ नो नो ह॒विषा॑ वृधातु वृधातु ह॒विषा॑ नो नो ह॒विषा॑ वृधातु । \newline
30. ह॒विषा॑ वृधातु वृधातु ह॒विषा॑ ह॒विषा॑ वृधातु॒ स्वाहा॒ स्वाहा॑ वृधातु ह॒विषा॑ ह॒विषा॑ वृधातु॒ स्वाहा᳚ । \newline
31. वृ॒धा॒तु॒ स्वाहा॒ स्वाहा॑ वृधातु वृधातु॒ स्वाहा॒ विश्वे॒ विश्वे॒ स्वाहा॑ वृधातु वृधातु॒ स्वाहा॒ विश्वे᳚ । \newline
32. स्वाहा॒ विश्वे॒ विश्वे॒ स्वाहा॒ स्वाहा॒ विश्वे॑ दे॒वस्य॑ दे॒वस्य॒ विश्वे॒ स्वाहा॒ स्वाहा॒ विश्वे॑ दे॒वस्य॑ । \newline
33. विश्वे॑ दे॒वस्य॑ दे॒वस्य॒ विश्वे॒ विश्वे॑ दे॒वस्य॑ ने॒तुर् ने॒तुर् दे॒वस्य॒ विश्वे॒ विश्वे॑ दे॒वस्य॑ ने॒तुः । \newline
34. दे॒वस्य॑ ने॒तुर् ने॒तुर् दे॒वस्य॑ दे॒वस्य॑ ने॒तुर् मर्तो॒ मर्तो॑ ने॒तुर् दे॒वस्य॑ दे॒वस्य॑ ने॒तुर् मर्तः॑ । \newline
35. ने॒तुर् मर्तो॒ मर्तो॑ ने॒तुर् ने॒तुर् मर्तो॑ वृणीत वृणीत॒ मर्तो॑ ने॒तुर् ने॒तुर् मर्तो॑ वृणीत । \newline
36. मर्तो॑ वृणीत वृणीत॒ मर्तो॒ मर्तो॑ वृणीत स॒ख्यꣳ स॒ख्यं ॅवृ॑णीत॒ मर्तो॒ मर्तो॑ वृणीत स॒ख्यम् । \newline
37. वृ॒णी॒त॒ स॒ख्यꣳ स॒ख्यं ॅवृ॑णीत वृणीत स॒ख्यं ॅविश्वे॒ विश्वे॑ स॒ख्यं ॅवृ॑णीत वृणीत स॒ख्यं ॅविश्वे᳚ । \newline
38. स॒ख्यं ॅविश्वे॒ विश्वे॑ स॒ख्यꣳ स॒ख्यं ॅविश्वे॑ रा॒यो रा॒यो विश्वे॑ स॒ख्यꣳ स॒ख्यं ॅविश्वे॑ रा॒यः । \newline
39. विश्वे॑ रा॒यो रा॒यो विश्वे॒ विश्वे॑ रा॒य इ॑षुद्ध्यसीषुद्ध्यसि रा॒यो विश्वे॒ विश्वे॑ रा॒य इ॑षुद्ध्यसि । \newline
40. रा॒य इ॑षुद्ध्यसीषुद्ध्यसि रा॒यो रा॒य इ॑षुद्ध्यसि द्यु॒म्नम् द्यु॒म्न मि॑षुद्ध्यसि रा॒यो रा॒य इ॑षुद्ध्यसि द्यु॒म्नम् । \newline
41. इ॒षु॒द्ध्य॒सि॒ द्यु॒म्नम् द्यु॒म्न मि॑षुद्ध्यसीषुद्ध्यसि द्यु॒म्नं ॅवृ॑णीत वृणीत द्यु॒म्न मि॑षुद्ध्यसीषुद्ध्यसि द्यु॒म्नं ॅवृ॑णीत । \newline
42. द्यु॒म्नं ॅवृ॑णीत वृणीत द्यु॒म्नम् द्यु॒म्नं ॅवृ॑णीत पु॒ष्यसे॑ पु॒ष्यसे॑ वृणीत द्यु॒म्नम् द्यु॒म्नं ॅवृ॑णीत पु॒ष्यसे᳚ । \newline
43. वृ॒णी॒त॒ पु॒ष्यसे॑ पु॒ष्यसे॑ वृणीत वृणीत पु॒ष्यसे॒ स्वाहा॒ स्वाहा॑ पु॒ष्यसे॑ वृणीत वृणीत पु॒ष्यसे॒ स्वाहा᳚ । \newline
44. पु॒ष्यसे॒ स्वाहा॒ स्वाहा॑ पु॒ष्यसे॑ पु॒ष्यसे॒ स्वाह॑र्ख्सा॒मयोर्॑. ऋख्सा॒मयोः॒ स्वाहा॑ पु॒ष्यसे॑ पु॒ष्यसे॒ स्वाह॑र्ख्सा॒मयोः᳚ । \newline
45. स्वाह॑र्ख्सा॒मयोर्॑. ऋख्सा॒मयोः॒ स्वाहा॒ स्वाह॑र्ख्सा॒मयोः॒ शिल्पे॒ शिल्पे॑ ऋख्सा॒मयोः॒ स्वाहा॒ स्वाह॑र्ख्सा॒मयोः॒ शिल्पे᳚ । \newline
46. ऋ॒ख्सा॒मयोः॒ शिल्पे॒ शिल्पे॑ ऋख्सा॒मयोर्॑. ऋख्सा॒मयोः॒ शिल्पे᳚ स्थः स्थः॒ शिल्पे॑ ऋख्सा॒मयोर्॑. ऋख्सा॒मयोः॒ शिल्पे᳚ स्थः । \newline
47. ऋ॒ख्सा॒मयो॒रित्यृ॑क् - सा॒मयोः᳚ । \newline
48. शिल्पे᳚ स्थः स्थः॒ शिल्पे॒ शिल्पे᳚ स्थ॒ स्ते ते स्थः॒ शिल्पे॒ शिल्पे᳚ स्थ॒स्ते । \newline
49. शिल्पे॒ इति॒ शिल्पे᳚ । \newline
50. स्थ॒स्ते ते स्थः॑ स्थ॒स्ते वां᳚ ॅवा॒म् ते स्थः॑ स्थ॒स्ते वा᳚म् । \newline
51. ते वां᳚ ॅवा॒म् ते ते वा॒ मा वा॒म् ते ते वा॒ मा । \newline
52. ते इति॒ ते । \newline
53. वा॒ मा वां᳚ ॅवा॒ मा र॑भे रभ॒ आ वां᳚ ॅवा॒ मा र॑भे । \newline
54. आ र॑भे रभ॒ आ र॑भे॒ ते ते र॑भ॒ आ र॑भे॒ ते । \newline
55. र॒भे॒ ते ते र॑भे रभे॒ ते मा॑ मा॒ ते र॑भे रभे॒ ते मा᳚ । \newline
56. ते मा॑ मा॒ ते ते मा॑ पातम् पातम् मा॒ ते ते मा॑ पातम् । \newline
57. ते इति॒ ते । \newline
58. मा॒ पा॒त॒म् पा॒त॒म् मा॒ मा॒ पा॒त॒ मा पा॑तम् मा मा पात॒ मा । \newline
\pagebreak
\markright{ TS 1.2.2.2  \hfill https://www.vedavms.in \hfill}

\section{ TS 1.2.2.2 }

\textbf{TS 1.2.2.2 } \newline
\textbf{Samhita Paata} \newline

पात॒मास्य य॒ज्ञ्स्यो॒दृच॑ इ॒मां धियꣳ॒॒ शिक्ष॑माणस्य देव॒ क्रतुं॒ दक्षं॑ ॅवरुण॒ सꣳ शि॑शाधि॒ ययाऽति॒ विश्वा॑ दुरि॒ता तरे॑म सु॒तर्मा॑ण॒मधि॒ नावꣳ॑ रुहे॒मोर्ग॑स्याङ्गिर॒स्यूर्ण॑म्रदा॒ ऊर्जं॑ मे यच्छ पा॒हि मा॒ मा मा॑ हिꣳसी॒र् विष्णोः॒ शर्मा॑सि॒ शर्म॒ यज॑मानस्य॒ शर्म॑ मे यच्छ॒ नक्ष॑त्राणां माऽतीका॒शात् पा॒हीन्द्र॑स्य॒ योनि॑रसि॒ - [ ] \newline

\textbf{Pada Paata} \newline

पा॒त॒म् । एति॑ । अ॒स्य । य॒ज्ञ्स्य॑ । उ॒दृच॒ इत्यु॑त् - ऋचः॑ । इ॒माम् । धिय᳚म् । शिक्ष॑माणस्य । दे॒व॒ । क्रतु᳚म् । दक्ष᳚म् । व॒रु॒ण॒ । समिति॑ । शि॒शा॒धि॒ । यया᳚ । अतीति॑ । विश्वा᳚ । दु॒रि॒तेति॑ दुः - इ॒ता । तरे॑म । सु॒तर्मा॑ण॒मिति॑ सु -तर्मा॑णम् । अधीति॑ । नाव᳚म् । रु॒हे॒म॒ । ऊर्क् । अ॒सि॒ । आ॒ङ्गि॒र॒सी । ऊर्ण॑म्रदा॒ इत्यूर्ण॑ - म्र॒दाः॒ । ऊर्ज᳚म् । मे॒ । य॒च्छ॒ । पा॒हि । मा॒  । मा । मा॒ । हिꣳ॒॒सीः॒ । विष्णोः᳚ । शर्म॑ । अ॒सि॒ । शर्म॑ । यज॑मानस्य । शर्म॑ । मे॒ । य॒च्छ॒ । नक्ष॑त्राणाम् । मा॒ । अ॒ती॒का॒शात् । पा॒हि॒ । इन्द्र॑स्य । योनिः॑ । अ॒सि॒ ।  \newline


\textbf{Krama Paata} \newline

पा॒त॒मा । आ ऽस्य । अ॒स्य य॒ज्ञ्स्य॑ । य॒ज्ञ्स्यो॒दृचः॑ । उ॒दृच॑ इ॒माम् । उ॒दृच॒ इत्यु॑त् - ऋचः॑ । इ॒माम् धिय᳚म् । धियꣳ॒॒ शिक्ष॑माणस्य । शिक्ष॑माणस्य देव । दे॒व॒ क्रतु᳚म् । क्रतु॒म् दक्ष᳚म् । दक्षं॑ ॅवरुण । व॒रु॒ण॒ सम् । सꣳ शि॑शाधि । शि॒शा॒धि॒ यया᳚ । ययाऽति॑ । अति॒ विश्वा᳚ । विश्वा॑ दुरि॒ता । दु॒रि॒ता तरे॑म । दु॒रि॒तेति॑ दुः - इ॒ता । तरे॑म सु॒तर्मा॑णम् । सु॒तर्मा॑ण॒मधि॑ । सु॒तर्मा॑ण॒मिति॑ सु - तर्मा॑णम् । अधि॒ नाव᳚म् । नावꣳ॑ रुहेम । रु॒हे॒मोर्क् । ऊर्ग॑सि । अ॒स्या॒ङ्गि॒र॒सी । आ॒ङ् गि॒र॒स्यूर्ण॑म्रदाः । ऊर्ण॑म्रदा॒ ऊर्ज᳚म् । ऊर्ण॑म्रदा॒ इत्यूर्ण॑ - म्र॒दाः॒ । ऊर्ज॑म् मे । मे॒ य॒च्छ॒ । य॒च्छ॒ पा॒हि । पा॒हि मा᳚ । मा॒ मा । मा मा᳚ । मा॒ हिꣳ॒॒सीः॒ । हिꣳ॒॒सी॒र् विष्णोः᳚ । विष्णोः॒ शर्म॑ । शर्मा॑सि । अ॒सि॒ शर्म॑ । शर्म॒ यज॑मानस्य । यज॑मानस्य॒ शर्म॑ । शर्म॑ मे । मे॒ य॒च्छ॒ । य॒च्छ॒ नक्ष॑त्राणाम् । नक्ष॑त्राणाम् मा । मा॒ ऽती॒का॒शात् । अ॒ती॒का॒शात् पा॑हि । पा॒हीन्द्र॑स्य । इन्द्र॑स्य॒ योनिः॑ । योनि॑रसि ( ) । अ॒सि॒ मा \newline

\textbf{Jatai Paata} \newline

1. पा॒त॒ मा पा॑तम् पात॒ मा । \newline
2. आ ऽस्यास्या ऽस्य । \newline
3. अ॒स्य य॒ज्ञ्स्य॑ य॒ज्ञ्स्या॒ स्यास्य य॒ज्ञ्स्य॑ । \newline
4. य॒ज्ञ्स्यो॒दृच॑ उ॒दृचो॑ य॒ज्ञ्स्य॑ य॒ज्ञ्स्यो॒दृचः॑ । \newline
5. उ॒दृच॑ इ॒मा मि॒मा मु॒दृच॑ उ॒दृच॑ इ॒माम् । \newline
6. उ॒दृच॒ इत्यु॑त् - ऋचः॑ । \newline
7. इ॒माम् धिय॒म् धिय॑ मि॒मा मि॒माम् धिय᳚म् । \newline
8. धिय॒(ग्म्॒) शिक्ष॑माणस्य॒ शिक्ष॑माणस्य॒ धिय॒म् धिय॒(ग्म्॒) शिक्ष॑माणस्य । \newline
9. शिक्ष॑माणस्य देव देव॒ शिक्ष॑माणस्य॒ शिक्ष॑माणस्य देव । \newline
10. दे॒व॒ क्रतु॒म् क्रतु॑म् देव देव॒ क्रतु᳚म् । \newline
11. क्रतु॒म् दक्ष॒म् दक्ष॒म् क्रतु॒म् क्रतु॒म् दक्ष᳚म् । \newline
12. दक्षं॑ ॅवरुण वरुण॒ दक्ष॒म् दक्षं॑ ॅवरुण । \newline
13. व॒रु॒ण॒ सꣳ सं ॅव॑रुण वरुण॒ सम् । \newline
14. सꣳ शि॑शाधि शिशाधि॒ सꣳ सꣳ शि॑शाधि । \newline
15. शि॒शा॒धि॒ यया॒ यया॑ शिशाधि शिशाधि॒ यया᳚ । \newline
16. यया ऽत्यति॒ यया॒ यया ऽति॑ । \newline
17. अति॒ विश्वा॒ विश्वा ऽत्यति॒ विश्वा᳚ । \newline
18. विश्वा॑ दुरि॒ता दु॑रि॒ता विश्वा॒ विश्वा॑ दुरि॒ता । \newline
19. दु॒रि॒ता तरे॑म॒ तरे॑म दुरि॒ता दु॑रि॒ता तरे॑म । \newline
20. दु॒रि॒तेति॑ दुः - इ॒ता । \newline
21. तरे॑म सु॒तर्मा॑णꣳ सु॒तर्मा॑ण॒म् तरे॑म॒ तरे॑म सु॒तर्मा॑णम् । \newline
22. सु॒तर्मा॑ण॒ मध्यधि॑ सु॒तर्मा॑णꣳ सु॒तर्मा॑ण॒ मधि॑ । \newline
23. सु॒तर्मा॑ण॒मिति॑ सु - तर्मा॑णम् । \newline
24. अधि॒ नाव॒न्नाव॒ मध्यधि॒ नाव᳚म् । \newline
25. नाव(ग्म्॑) रुहेम रुहेम॒ नाव॒म् नाव(ग्म्॑) रुहेम । \newline
26. रु॒हे॒मोर्गूर्ग् रु॑हेम रुहे॒मोर्क् । \newline
27. ऊर्ग॑स्य॒ स्यूर् गूर्ग॑सि । \newline
28. अ॒स्या॒ङ्गि॒ र॒स्या᳚ङ्गि र॒स्य॑स्य स्याङ्गि र॒सी । \newline
29. आ॒ङ्गि॒ र॒स्यूर्ण॑म्रदा॒ ऊर्ण॑म्रदा आङ्गि र॒स्या᳚ङ्गि र॒स्यूर्ण॑म्रदाः । \newline
30. ऊर्ण॑म्रदा॒ ऊर्ज॒ मूर्ज॒ मूर्ण॑म्रदा॒ ऊर्ण॑म्रदा॒ ऊर्ज᳚म् । \newline
31. ऊर्ण॑म्रदा॒ इत्यूर्ण॑ - म्र॒दाः॒ । \newline
32. ऊर्ज॑म् मे म॒ ऊर्ज॒ मूर्ज॑म् मे । \newline
33. मे॒ य॒च्छ॒ य॒च्छ॒ मे॒ मे॒ य॒च्छ॒ । \newline
34. य॒च्छ॒ पा॒हि पा॒हि य॑च्छ यच्छ पा॒हि । \newline
35. पा॒हि मा॑ मा पा॒हि पा॒हि मा᳚ । \newline
36. मा॒ मा मा मा॑ मा॒ मा । \newline
37. मा मा॑ मा॒ मा मा मा᳚ । \newline
38. मा॒ हि॒(ग्म्॒)सी॒र्॒. हि॒(ग्म्॒)सी॒र् मा॒ मा॒ हि॒(ग्म्॒)सीः॒ । \newline
39. हि॒(ग्म्॒)सी॒र् विष्णो॒र् विष्णोर्॑. हिꣳसीर्. हिꣳसी॒र् विष्णोः᳚ । \newline
40. विष्णोः॒ शर्म॒ शर्म॒ विष्णो॒र् विष्णोः॒ शर्म॑ । \newline
41. शर्मा᳚ स्यसि॒ शर्म॒ शर्मा॑सि । \newline
42. अ॒सि॒ शर्म॒ शर्मा᳚ स्यसि॒ शर्म॑ । \newline
43. शर्म॒ यज॑मानस्य॒ यज॑मानस्य॒ शर्म॒ शर्म॒ यज॑मानस्य । \newline
44. यज॑मानस्य॒ शर्म॒ शर्म॒ यज॑मानस्य॒ यज॑मानस्य॒ शर्म॑ । \newline
45. शर्म॑ मे मे॒ शर्म॒ शर्म॑ मे । \newline
46. मे॒ य॒च्छ॒ य॒च्छ॒ मे॒ मे॒ य॒च्छ॒ । \newline
47. य॒च्छ॒ नक्ष॑त्राणा॒म् नक्ष॑त्राणां ॅयच्छ यच्छ॒ नक्ष॑त्राणाम् । \newline
48. नक्ष॑त्राणाम् मा मा॒ नक्ष॑त्राणा॒म् नक्ष॑त्राणाम् मा । \newline
49. मा॒ ऽती॒का॒शा द॑तीका॒शान् मा॑ मा ऽतीका॒शात् । \newline
50. अ॒ती॒का॒शात् पा॑हि पाह्यतीका॒शा द॑तीका॒शात् पा॑हि । \newline
51. पा॒हीन्द्र॒स्ये न्द्र॑स्य पाहि पा॒हीन्द्र॑स्य । \newline
52. इन्द्र॑स्य॒ योनि॒र् योनि॒ रिन्द्र॒स्ये न्द्र॑स्य॒ योनिः॑ । \newline
53. योनि॑ रस्यसि॒ योनि॒र् योनि॑ रसि । \newline
54. अ॒सि॒ मा मा ऽस्य॑सि॒ मा । \newline

\textbf{Ghana Paata } \newline

1. पा॒त॒ मा पा॑तम् पात॒ मा ऽस्यास्या पा॑तम् पात॒ मा ऽस्य । \newline
2. आ ऽस्यास्या ऽस्य य॒ज्ञ्स्य॑ य॒ज्ञ्स्या॒स्या ऽस्य य॒ज्ञ्स्य॑ । \newline
3. अ॒स्य य॒ज्ञ्स्य॑ य॒ज्ञ्स्या॒स्यास्य य॒ज्ञ्स्यो॒दृच॑ उ॒दृचो॑ य॒ज्ञ्स्या॒स्यास्य य॒ज्ञ्स्यो॒दृचः॑ । \newline
4. य॒ज्ञ्स्यो॒दृच॑ उ॒दृचो॑ य॒ज्ञ्स्य॑ य॒ज्ञ्स्यो॒दृच॑ इ॒मा मि॒मा मु॒दृचो॑ य॒ज्ञ्स्य॑ य॒ज्ञ्स्यो॒दृच॑ इ॒माम् । \newline
5. उ॒दृच॑ इ॒मा मि॒मा मु॒दृच॑ उ॒दृच॑ इ॒माम् धिय॒म् धिय॑ मि॒मा मु॒दृच॑ उ॒दृच॑ इ॒माम् धिय᳚म् । \newline
6. उ॒दृच॒ इत्यु॑त् - ऋचः॑ । \newline
7. इ॒माम् धिय॒म् धिय॑ मि॒मा मि॒माम् धिय॒(ग्म्॒) शिक्ष॑माणस्य॒ शिक्ष॑माणस्य॒ धिय॑ मि॒मा मि॒माम् 
धिय॒(ग्म्॒) शिक्ष॑माणस्य । \newline
8. धिय॒(ग्म्॒) शिक्ष॑माणस्य॒ शिक्ष॑माणस्य॒ धिय॒म् धिय॒(ग्म्॒) शिक्ष॑माणस्य देव देव॒ शिक्ष॑माणस्य॒ धिय॒म् धिय॒(ग्म्॒) शिक्ष॑माणस्य देव । \newline
9. शिक्ष॑माणस्य देव देव॒ शिक्ष॑माणस्य॒ शिक्ष॑माणस्य देव॒ क्रतु॒म् क्रतु॑म् देव॒ शिक्ष॑माणस्य॒ शिक्ष॑माणस्य देव॒ क्रतु᳚म् । \newline
10. दे॒व॒ क्रतु॒म् क्रतु॑म् देव देव॒ क्रतु॒म् दक्ष॒म् दक्ष॒म् क्रतु॑म् देव देव॒ क्रतु॒म् दक्ष᳚म् । \newline
11. क्रतु॒म् दक्ष॒म् दक्ष॒म् क्रतु॒म् क्रतु॒म् दक्षं॑ ॅवरुण वरुण॒ दक्ष॒म् क्रतु॒म् क्रतु॒म् दक्षं॑ ॅवरुण । \newline
12. दक्षं॑ ॅवरुण वरुण॒ दक्ष॒म् दक्षं॑ ॅवरुण॒ सꣳ सं ॅव॑रुण॒ दक्ष॒म् दक्षं॑ ॅवरुण॒ सम् । \newline
13. व॒रु॒ण॒ सꣳ सं ॅव॑रुण वरुण॒ सꣳ शि॑शाधि शिशाधि॒ सं ॅव॑रुण वरुण॒ सꣳ शि॑शाधि । \newline
14. सꣳ शि॑शाधि शिशाधि॒ सꣳ सꣳ शि॑शाधि॒ यया॒ यया॑ शिशाधि॒ सꣳ सꣳ शि॑शाधि॒ यया᳚ । \newline
15. शि॒शा॒धि॒ यया॒ यया॑ शिशाधि शिशाधि॒ यया ऽत्यति॒ यया॑ शिशाधि शिशाधि॒ यया ऽति॑ । \newline
16. यया ऽत्यति॒ यया॒ यया ऽति॒ विश्वा॒ विश्वा ऽति॒ यया॒ यया ऽति॒ विश्वा᳚ । \newline
17. अति॒ विश्वा॒ विश्वा ऽत्यति॒ विश्वा॑ दुरि॒ता दु॑रि॒ता विश्वा ऽत्यति॒ विश्वा॑ दुरि॒ता । \newline
18. विश्वा॑ दुरि॒ता दु॑रि॒ता विश्वा॒ विश्वा॑ दुरि॒ता तरे॑म॒ तरे॑म दुरि॒ता विश्वा॒ विश्वा॑ दुरि॒ता तरे॑म । \newline
19. दु॒रि॒ता तरे॑म॒ तरे॑म दुरि॒ता दु॑रि॒ता तरे॑म सु॒तर्मा॑णꣳ सु॒तर्मा॑ण॒म् तरे॑म दुरि॒ता दु॑रि॒ता तरे॑म सु॒तर्मा॑णम् । \newline
20. दु॒रि॒तेति॑ दुः - इ॒ता । \newline
21. तरे॑म सु॒तर्मा॑णꣳ सु॒तर्मा॑ण॒म् तरे॑म॒ तरे॑म सु॒तर्मा॑ण॒ मध्यधि॑ सु॒तर्मा॑ण॒म् तरे॑म॒ तरे॑म सु॒तर्मा॑ण॒ मधि॑ । \newline
22. सु॒तर्मा॑ण॒ मध्यधि॑ सु॒तर्मा॑णꣳ सु॒तर्मा॑ण॒ मधि॒ नाव॒न्नाव॒ मधि॑ सु॒तर्मा॑णꣳ सु॒तर्मा॑ण॒ मधि॒ नाव᳚म् । \newline
23. सु॒तर्मा॑ण॒मिति॑ सु - तर्मा॑णम् । \newline
24. अधि॒ नाव॒न्नाव॒ मध्यधि॒ नाव(ग्म्॑) रुहेम रुहेम॒ नाव॒ मध्यधि॒ नाव(ग्म्॑) रुहेम । \newline
25. नाव(ग्म्॑) रुहेम रुहेम॒ नाव॒न्नाव(ग्म्॑) रुहे॒मोर्गूर्ग् रु॑हेम॒ नाव॒न्नाव(ग्म्॑) रुहे॒मोर्क् । \newline
26. रु॒हे॒मोर्गूर्ग् रु॑हेम रुहे॒मोर्ग॑स्य॒स्यूर्ग् रु॑हेम रुहे॒मोर्ग॑सि । \newline
27. ऊर्ग॑ स्य॒ स्यूर्गूर्ग॑स्या ङ्गिर॒स्या᳚ ङ्गिर॒स्य॑ स्यूर्गूर्ग॑ स्याङ्गिर॒सी । \newline
28. अ॒स्या॒ ङ्गि॒र॒स्या᳚ ङ्गिर॒स्य॑स्यस्या ङ्गिर॒स्यूर्ण॑म्रदा॒ ऊर्ण॑म्रदा आङ्गिर॒स्य॑स्यस्या ङ्गिर॒स्यूर्ण॑म्रदाः । \newline
29. आ॒ङ्गि॒र॒स्यूर्ण॑म्रदा॒ ऊर्ण॑म्रदा आङ्गिर॒स्या᳚ ङ्गिर॒स्यूर्ण॑म्रदा॒ ऊर्ज॒ मूर्ज॒ मूर्ण॑म्रदा आङ्गिर॒स्या᳚ङ्गिर॒स्यूर्ण॑म्रदा॒ ऊर्ज᳚म् । \newline
30. ऊर्ण॑म्रदा॒ ऊर्ज॒ मूर्ज॒ मूर्ण॑म्रदा॒ ऊर्ण॑म्रदा॒ ऊर्ज॑म् मे म॒ ऊर्ज॒ मूर्ण॑म्रदा॒ ऊर्ण॑म्रदा॒ ऊर्ज॑म् मे । \newline
31. ऊर्ण॑म्रदा॒ इत्यूर्ण॑ - म्र॒दाः॒ । \newline
32. ऊर्ज॑म् मे म॒ ऊर्ज॒ मूर्ज॑म् मे यच्छ यच्छ म॒ ऊर्ज॒ मूर्ज॑म् मे यच्छ । \newline
33. मे॒ य॒च्छ॒ य॒च्छ॒ मे॒ मे॒ य॒च्छ॒ पा॒हि पा॒हि य॑च्छ मे मे यच्छ पा॒हि । \newline
34. य॒च्छ॒ पा॒हि पा॒हि य॑च्छ यच्छ पा॒हि मा॑ मा पा॒हि य॑च्छ यच्छ पा॒हि मा᳚ । \newline
35. पा॒हि मा॑ मा पा॒हि पा॒हि मा॒ मा मा मा॑ पा॒हि पा॒हि मा॒ मा । \newline
36. मा॒ मा मा मा॑ मा॒ मा मा॑ मा॒ मा मा॑ मा॒ मा मा᳚ । \newline
37. मा मा॑ मा॒ मा मा मा॑ हिꣳसीर्. हिꣳसीर् मा॒ मा मा मा॑ हिꣳसीः । \newline
38. मा॒ हि॒(ग्म्॒)सी॒र्॒. हि॒(ग्म्॒)सी॒र् मा॒ मा॒ हि॒(ग्म्॒)सी॒र् विष्णो॒र् विष्णोर्॑. हिꣳसीर् मा मा हिꣳसी॒र् विष्णोः᳚ । \newline
39. हि॒(ग्म्॒)सी॒र् विष्णो॒र् विष्णोर्॑. हिꣳसीर्. हिꣳसी॒र् विष्णोः॒ शर्म॒ शर्म॒ विष्णोर्॑. हिꣳसीर्. हिꣳसी॒र् विष्णोः॒ शर्म॑ । \newline
40. विष्णोः॒ शर्म॒ शर्म॒ विष्णो॒र् विष्णोः॒ शर्मा᳚स्यसि॒ शर्म॒ विष्णो॒र् विष्णोः॒ शर्मा॑सि । \newline
41. शर्मा᳚स्यसि॒ शर्म॒ शर्मा॑सि॒ शर्म॒ शर्मा॑सि॒ शर्म॒ शर्मा॑सि॒ शर्म॑ । \newline
42. अ॒सि॒ शर्म॒ शर्मा᳚स्यसि॒ शर्म॒ यज॑मानस्य॒ यज॑मानस्य॒ शर्मा᳚स्यसि॒ शर्म॒ यज॑मानस्य । \newline
43. शर्म॒ यज॑मानस्य॒ यज॑मानस्य॒ शर्म॒ शर्म॒ यज॑मानस्य॒ शर्म॒ शर्म॒ यज॑मानस्य॒ शर्म॒ शर्म॒ यज॑मानस्य॒ शर्म॑ । \newline
44. यज॑मानस्य॒ शर्म॒ शर्म॒ यज॑मानस्य॒ यज॑मानस्य॒ शर्म॑ मे मे॒ शर्म॒ यज॑मानस्य॒ यज॑मानस्य॒ शर्म॑ मे । \newline
45. शर्म॑ मे मे॒ शर्म॒ शर्म॑ मे यच्छ यच्छ मे॒ शर्म॒ शर्म॑ मे यच्छ । \newline
46. मे॒ य॒च्छ॒ य॒च्छ॒ मे॒ मे॒ य॒च्छ॒ नक्ष॑त्राणा॒ न्नक्ष॑त्राणां ॅयच्छ मे मे यच्छ॒ नक्ष॑त्राणाम् । \newline
47. य॒च्छ॒ नक्ष॑त्राणा॒ न्नक्ष॑त्राणां ॅयच्छ यच्छ॒ नक्ष॑त्राणाम् मा मा॒ नक्ष॑त्राणां ॅयच्छ यच्छ॒ नक्ष॑त्राणाम् मा । \newline
48. नक्ष॑त्राणाम् मा मा॒ नक्ष॑त्राणा॒ न्नक्ष॑त्राणाम् मा ऽतीका॒शा द॑तीका॒शान् मा॒ नक्ष॑त्राणा॒ न्नक्ष॑त्राणाम् मा ऽतीका॒शात् । \newline
49. मा॒ ऽती॒का॒शा द॑तीका॒शान् मा॑ मा ऽतीका॒शात् पा॑हि पाह्यतीका॒शान् मा॑ मा ऽतीका॒शात् पा॑हि । \newline
50. अ॒ती॒का॒शात् पा॑हि पाह्यतीका॒शा द॑तीका॒शात् पा॒हीन्द्र॒स्ये न्द्र॑स्य पाह्यतीका॒शा द॑तीका॒शात् पा॒हीन्द्र॑स्य । \newline
51. पा॒हीन्द्र॒स्ये न्द्र॑स्य पाहि पा॒हीन्द्र॑स्य॒ योनि॒र् योनि॒रिन्द्र॑स्य पाहि पा॒हीन्द्र॑स्य॒ योनिः॑ । \newline
52. इन्द्र॑स्य॒ योनि॒र् योनि॒रिन्द्र॒स्ये न्द्र॑स्य॒ योनि॑रस्यसि॒ योनि॒रिन्द्र॒स्ये न्द्र॑स्य॒ योनि॑रसि । \newline
53. योनि॑रस्यसि॒ योनि॒र् योनि॑रसि॒ मा मा ऽसि॒ योनि॒र् योनि॑रसि॒ मा । \newline
54. अ॒सि॒ मा मा ऽस्य॑सि॒ मा मा॑ मा॒ मा ऽस्य॑सि॒ मा मा᳚ । \newline
\pagebreak
\markright{ TS 1.2.2.3  \hfill https://www.vedavms.in \hfill}

\section{ TS 1.2.2.3 }

\textbf{TS 1.2.2.3 } \newline
\textbf{Samhita Paata} \newline

मा मा॑ हिꣳसीः कृ॒ष्यै त्वा॑ सुस॒स्यायै॑ सुपिप्प॒लाभ्य॒-स्त्वौष॑धीभ्यः सूप॒स्था दे॒वो वन॒स्पति॑रू॒र्द्ध्वो मा॑ पा॒ह्योदृचः॒ स्वाहा॑ य॒ज्ञ्ं मन॑सा॒ स्वाहा॒ द्यावा॑पृथि॒वीभ्याꣳ॒॒ स्वाहो॒रो-र॒न्तरि॑क्षा॒थ् स्वाहा॑ य॒ज्ञ्ं ॅवाता॒दा र॑भे ॥ \newline

\textbf{Pada Paata} \newline

मा । मा॒ । हिꣳ॒॒सीः॒ । कृ॒ष्यै । त्वा॒ । सु॒स॒स्याया॒ इति॑ सु - स॒स्यायै᳚ । सु॒पि॒प्प॒लाभ्य॒ इति॑ सु - पि॒प्प॒लाभ्यः॑ । त्वा॒ । ओष॑धीभ्य॒ इत्योष॑धि - भ्यः॒ । सू॒प॒स्था इति॑ सु - उ॒प॒स्थाः । दे॒वः । वन॒स्पतिः॑ । ऊ॒र्द्ध्वः । मा॒ । पा॒हि॒ । एति॑ । उ॒दृच॒ इत्यु॑त् - ऋचः॑ । स्वाहा᳚ । य॒ज्ञ्म् । मन॑सा । स्वाहा᳚ । द्यावा॑पृथि॒वीभ्या॒मिति॒ द्यावा᳚ - पृ॒थि॒वीभ्या᳚म् । स्वाहा᳚ । उ॒रोः । अ॒न्तरि॑क्षात् । स्वाहा᳚ । य॒ज्ञ्म् । वाता᳚त् । एति॑ । र॒भे॒ ॥  \newline


\textbf{Krama Paata} \newline

मा मा᳚ । मा॒ हिꣳ॒॒सीः॒ । हिꣳ॒॒सीः॒ कृ॒ष्यै । कृ॒ष्यै त्वा᳚ । त्वा॒ सु॒स॒स्यायै᳚ । सु॒स॒स्यायै॑ सुपिप्प॒लाभ्यः॑ । सु॒स॒स्याया॒ इति॑ सु - स॒स्यायै᳚ । सु॒पि॒प्प॒लाभ्य॑स्त्वा । सु॒पि॒प्प॒लाभ्य॒ इति॑ सु - पि॒प्प॒लाभ्यः॑ । त्वौष॑धीभ्यः । ओष॑धीभ्यः सूप॒स्थाः । ओष॑धीभ्य॒ इत्योष॑धि - भ्यः॒ । सू॒प॒स्था दे॒वः । सू॒प॒स्था इति॑ सु - उ॒प॒स्थाः । दे॒वो वन॒स्पतिः॑ । वन॒स्पति॑रू॒र्द्ध्वः । 
ऊ॒र्द्ध्वो मा᳚ । मा॒ पा॒हि॒ । पा॒ह्या । ओदृचः॑ । उ॒दृचः॒ स्वाहा᳚ । उ॒दृच॒ इत्यु॑त् - ऋचः॑ । स्वाहा॑ य॒ज्ञ्म् । य॒ज्ञ्म् मन॑सा । मन॑सा॒ स्वाहा᳚ । स्वाहा॒ द्यावा॑पृथि॒वीभ्या᳚म् । द्यावा॑पृथि॒वीभ्याꣳ॒॒ 
स्वाहा᳚ । द्यावा॑पृथि॒वीभ्या॒मिति॒ द्यावा᳚ - पृ॒थि॒वीभ्या᳚म् । स्वाहो॒रोः । उ॒रोर॒न्तरि॑क्षात् । अ॒न्तरि॑क्षा॒थ् स्वाहा᳚ । स्वाहा॑ य॒ज्ञ्म् । य॒ज्ञ्ं ॅवाता᳚त् । वाता॒दा । आ र॑भे । र॒भ॒ इति॑ रभे । \newline

\textbf{Jatai Paata} \newline

1. मा मा॑ मा॒ मा मा मा᳚ । \newline
2. मा॒ हि॒(ग्म्॒)सी॒र्॒. हि॒(ग्म्॒)सी॒र् मा॒ मा॒ हि॒(ग्म्॒)सीः॒ । \newline
3. हि॒(ग्म्॒)सीः॒ कृ॒ष्यै कृ॒ष्यै हि(ग्म्॑)सीर्. हिꣳसीः कृ॒ष्यै । \newline
4. कृ॒ष्यै त्वा᳚ त्वा कृ॒ष्यै कृ॒ष्यै त्वा᳚ । \newline
5. त्वा॒ सु॒स॒स्यायै॑ सुस॒स्यायै᳚ त्वा त्वा सुस॒स्यायै᳚ । \newline
6. सु॒स॒स्यायै॑ सुपिप्प॒लाभ्यः॑ सुपिप्प॒लाभ्यः॑ सुस॒स्यायै॑ सुस॒स्यायै॑ सुपिप्प॒लाभ्यः॑ । \newline
7. सु॒स॒स्याया॒ इति॑ सु - स॒स्यायै᳚ । \newline
8. सु॒पि॒प्प॒लाभ्य॑ स्त्वा त्वा सुपिप्प॒लाभ्यः॑ सुपिप्प॒लाभ्य॑ स्त्वा । \newline
9. सु॒पि॒प्प॒लाभ्य॒ इति॑ सु - पि॒प्प॒लाभ्यः॑ । \newline
10. त्वौष॑धीभ्य॒ ओष॑धीभ्य स्त्वा॒ त्वौष॑धीभ्यः । \newline
11. ओष॑धीभ्यः सूप॒स्थाः सू॑प॒स्था ओष॑धीभ्य॒ ओष॑धीभ्यः सूप॒स्थाः । \newline
12. ओष॑धीभ्य॒ इत्योष॑धि - भ्यः॒ । \newline
13. सू॒प॒स्था दे॒वो दे॒वः सू॑प॒स्थाः सू॑प॒स्था दे॒वः । \newline
14. सू॒प॒स्था इति॑ सु - उ॒प॒स्थाः । \newline
15. दे॒वो वन॒स्पति॒र् वन॒स्पति॑र् दे॒वो दे॒वो वन॒स्पतिः॑ । \newline
16. वन॒स्पति॑ रू॒र्द्ध्व ऊ॒र्द्ध्वो वन॒स्पति॒र् वन॒स्पति॑ रू॒र्द्ध्वः । \newline
17. ऊ॒र्द्ध्वो मा॑ मो॒र्द्ध्व ऊ॒र्द्ध्वो मा᳚ । \newline
18. मा॒ पा॒हि॒ पा॒हि॒ मा॒ मा॒ पा॒हि॒ । \newline
19. पा॒ह्या पा॑हि पा॒ह्या । \newline
20. ओदृच॑ उ॒दृच॒ ओदृचः॑ । \newline
21. उ॒दृचः॒ स्वाहा॒ स्वाहो॒दृच॑ उ॒दृचः॒ स्वाहा᳚ । \newline
22. उ॒दृच॒ इत्यु॑त् - ऋचः॑ । \newline
23. स्वाहा॑ य॒ज्ञ्ं ॅय॒ज्ञ्ꣳ स्वाहा॒ स्वाहा॑ य॒ज्ञ्म् । \newline
24. य॒ज्ञ्म् मन॑सा॒ मन॑सा य॒ज्ञ्ं ॅय॒ज्ञ्म् मन॑सा । \newline
25. मन॑सा॒ स्वाहा॒ स्वाहा॒ मन॑सा॒ मन॑सा॒ स्वाहा᳚ । \newline
26. स्वाहा॒ द्यावा॑पृथि॒वीभ्या॒म् द्यावा॑पृथि॒वीभ्या॒(ग्ग्॒) स्वाहा॒ स्वाहा॒ द्यावा॑पृथि॒वीभ्या᳚म् । \newline
27. द्यावा॑पृथि॒वीभ्या॒(ग्ग्॒) स्वाहा॒ स्वाहा॒ द्यावा॑पृथि॒वीभ्या॒म् द्यावा॑पृथि॒वीभ्या॒(ग्ग्॒) स्वाहा᳚ । \newline
28. द्यावा॑पृथि॒वीभ्या॒मिति॒ द्यावा᳚ - पृ॒थि॒वीभ्या᳚म् । \newline
29. स्वाहो॒ रोरु॒रोः स्वाहा॒ स्वाहो॒रोः । \newline
30. उ॒रो र॒न्तरि॑क्षा द॒न्तरि॑क्षा दु॒रो रु॒रो र॒न्तरि॑क्षात् । \newline
31. अ॒न्तरि॑क्षा॒थ् स्वाहा॒ स्वाहा॒ ऽन्तरि॑क्षा द॒न्तरि॑क्षा॒थ् स्वाहा᳚ । \newline
32. स्वाहा॑ य॒ज्ञ्ं ॅय॒ज्ञ्ꣳ स्वाहा॒ स्वाहा॑ य॒ज्ञ्म् । \newline
33. य॒ज्ञ्ं ॅवाता॒द् वाता᳚द् य॒ज्ञ्ं ॅय॒ज्ञ्ं ॅवाता᳚त् । \newline
34. वाता॒दा वाता॒द् वाता॒दा । \newline
35. आ र॑भे रभ॒ आ र॑भे । \newline
36. र॒भ॒ इति॑ रभे । \newline

\textbf{Ghana Paata } \newline

1. मा मा॑ मा॒ मा मा मा॑ हिꣳसीर्. हिꣳसीर् मा॒ मा मा मा॑ हिꣳसीः । \newline
2. मा॒ हि॒(ग्म्॒)सी॒र्॒. हि॒(ग्म्॒)सी॒र् मा॒ मा॒ हि॒(ग्म्॒)सीः॒ कृ॒ष्यै कृ॒ष्यै हि(ग्म्॑)सीर् मा मा हिꣳसीः कृ॒ष्यै । \newline
3. हि॒(ग्म्॒)सीः॒ कृ॒ष्यै कृ॒ष्यै हि(ग्म्॑)सीर्. हिꣳसीः कृ॒ष्यै त्वा᳚ त्वा कृ॒ष्यै हि(ग्म्॑)सीर्. हिꣳसीः कृ॒ष्यै त्वा᳚ । \newline
4. कृ॒ष्यै त्वा᳚ त्वा कृ॒ष्यै कृ॒ष्यै त्वा॑ सुस॒स्यायै॑ सुस॒स्यायै᳚ त्वा कृ॒ष्यै कृ॒ष्यै त्वा॑ सुस॒स्यायै᳚ । \newline
5. त्वा॒ सु॒स॒स्यायै॑ सुस॒स्यायै᳚ त्वा त्वा सुस॒स्यायै॑ सुपिप्प॒लाभ्यः॑ सुपिप्प॒लाभ्यः॑ सुस॒स्यायै᳚ त्वा त्वा सुस॒स्यायै॑ सुपिप्प॒लाभ्यः॑ । \newline
6. सु॒स॒स्यायै॑ सुपिप्प॒लाभ्यः॑ सुपिप्प॒लाभ्यः॑ सुस॒स्यायै॑ सुस॒स्यायै॑ सुपिप्प॒लाभ्य॑स्त्वा त्वा सुपिप्प॒लाभ्यः॑ सुस॒स्यायै॑ सुस॒स्यायै॑ सुपिप्प॒लाभ्य॑स्त्वा । \newline
7. सु॒स॒स्याया॒ इति॑ सु - स॒स्यायै᳚ । \newline
8. सु॒पि॒प्प॒लाभ्य॑ स्त्वा त्वा सुपिप्प॒लाभ्यः॑ सुपिप्प॒लाभ्य॒ स्त्वौष॑धीभ्य॒ ओष॑धीभ्यस्त्वा सुपिप्प॒लाभ्यः॑ सुपिप्प॒लाभ्य॒ स्त्वौष॑धीभ्यः । \newline
9. सु॒पि॒प्प॒लाभ्य॒ इति॑ सु - पि॒प्प॒लाभ्यः॑ । \newline
10. त्वौष॑धीभ्य॒ ओष॑धीभ्यस्त्वा॒ त्वौष॑धीभ्यः सूप॒स्थाः सू॑प॒स्था ओष॑धीभ्यस्त्वा॒ त्वौष॑धीभ्यः सूप॒स्थाः । \newline
11. ओष॑धीभ्यः सूप॒स्थाः सू॑प॒स्था ओष॑धीभ्य॒ ओष॑धीभ्यः सूप॒स्था दे॒वो दे॒वः सू॑प॒स्था ओष॑धीभ्य॒ ओष॑धीभ्यः सूप॒स्था दे॒वः । \newline
12. ओष॑धीभ्य॒ इत्योष॑धि - भ्यः॒ । \newline
13. सू॒प॒स्था दे॒वो दे॒वः सू॑प॒स्थाः सू॑प॒स्था दे॒वो वन॒स्पति॒र् वन॒स्पति॑र् दे॒वः सू॑प॒स्थाः सू॑प॒स्था दे॒वो वन॒स्पतिः॑ । \newline
14. सू॒प॒स्था इति॑ सु - उ॒प॒स्थाः । \newline
15. दे॒वो वन॒स्पति॒र् वन॒स्पति॑र् दे॒वो दे॒वो वन॒स्पति॑ रू॒र्द्ध्व ऊ॒र्द्ध्वो वन॒स्पति॑र् दे॒वो दे॒वो वन॒स्पति॑ रू॒र्द्ध्वः । \newline
16. वन॒स्पति॑ रू॒र्द्ध्व ऊ॒र्द्ध्वो वन॒स्पति॒र् वन॒स्पति॑ रू॒र्द्ध्वो मा॑ मो॒र्द्ध्वो वन॒स्पति॒र् वन॒स्पति॑ रू॒र्द्ध्वो मा᳚ । \newline
17. ऊ॒र्द्ध्वो मा॑ मो॒र्द्ध्व ऊ॒र्द्ध्वो मा॑ पाहि पाहि मो॒र्द्ध्व ऊ॒र्द्ध्वो मा॑ पाहि । \newline
18. मा॒ पा॒हि॒ पा॒हि॒ मा॒ मा॒ पा॒ह्या पा॑हि मा मा पा॒ह्या । \newline
19. पा॒ह्या पा॑हि पा॒ह्योदृच॑ उ॒दृच॒ आ पा॑हि पा॒ह्योदृचः॑ । \newline
20. ओदृच॑ उ॒दृच॒ ओदृचः॒ स्वाहा॒ स्वाहो॒दृच॒ ओदृचः॒ स्वाहा᳚ । \newline
21. उ॒दृचः॒ स्वाहा॒ स्वाहो॒दृच॑ उ॒दृचः॒ स्वाहा॑ य॒ज्ञ्ं ॅय॒ज्ञ्ꣳ स्वाहो॒दृच॑ उ॒दृचः॒ स्वाहा॑ य॒ज्ञ्म् । \newline
22. उ॒दृच॒ इत्यु॑त् - ऋचः॑ । \newline
23. स्वाहा॑ य॒ज्ञ्ं ॅय॒ज्ञ्ꣳ स्वाहा॒ स्वाहा॑ य॒ज्ञ्म् मन॑सा॒ मन॑सा य॒ज्ञ्ꣳ स्वाहा॒ स्वाहा॑ य॒ज्ञ्म् मन॑सा । \newline
24. य॒ज्ञ्म् मन॑सा॒ मन॑सा य॒ज्ञ्ं ॅय॒ज्ञ्म् मन॑सा॒ स्वाहा॒ स्वाहा॒ मन॑सा य॒ज्ञ्ं ॅय॒ज्ञ्म् मन॑सा॒ स्वाहा᳚ । \newline
25. मन॑सा॒ स्वाहा॒ स्वाहा॒ मन॑सा॒ मन॑सा॒ स्वाहा॒ द्यावा॑पृथि॒वीभ्या॒म् द्यावा॑पृथि॒वीभ्या॒(ग्ग्॒) स्वाहा॒ मन॑सा॒ मन॑सा॒ स्वाहा॒ द्यावा॑पृथि॒वीभ्या᳚म् । \newline
26. स्वाहा॒ द्यावा॑पृथि॒वीभ्या॒म् द्यावा॑पृथि॒वीभ्या॒(ग्ग्॒) स्वाहा॒ स्वाहा॒ द्यावा॑पृथि॒वीभ्या॒(ग्ग्॒) स्वाहा॒ स्वाहा॒ द्यावा॑पृथि॒वीभ्या॒(ग्ग्॒) स्वाहा॒ स्वाहा॒ द्यावा॑पृथि॒वीभ्या॒(ग्ग्॒) स्वाहा᳚ । \newline
27. द्यावा॑पृथि॒वीभ्या॒(ग्ग्॒) स्वाहा॒ स्वाहा॒ द्यावा॑पृथि॒वीभ्या॒म् द्यावा॑पृथि॒वीभ्या॒(ग्ग्॒) स्वाहो॒ रोरु॒रोः स्वाहा॒ द्यावा॑पृथि॒वीभ्या॒म् द्यावा॑पृथि॒वीभ्या॒(ग्ग्॒) स्वाहो॒रोः । \newline
28. द्यावा॑पृथि॒वीभ्या॒मिति॒ द्यावा᳚ - पृ॒थि॒वीभ्या᳚म् । \newline
29. स्वाहो॒रोरु॒रोः स्वाहा॒ स्वाहो॒रो र॒न्तरि॑क्षा द॒न्तरि॑क्षादु॒रोः स्वाहा॒ स्वाहो॒रो र॒न्तरि॑क्षात् । \newline
30. उ॒रो र॒न्तरि॑क्षा द॒न्तरि॑क्षा दु॒रोरु॒रो र॒न्तरि॑क्षा॒थ् स्वाहा॒ स्वाहा॒ ऽन्तरि॑क्षा दु॒रोरु॒रो र॒न्तरि॑क्षा॒थ् स्वाहा᳚ । \newline
31. अ॒न्तरि॑क्षा॒थ् स्वाहा॒ स्वाहा॒ ऽन्तरि॑क्षा द॒न्तरि॑क्षा॒थ् स्वाहा॑ य॒ज्ञ्ं ॅय॒ज्ञ्ꣳ स्वाहा॒ ऽन्तरि॑क्षा द॒न्तरि॑क्षा॒थ् स्वाहा॑ य॒ज्ञ्म् । \newline
32. स्वाहा॑ य॒ज्ञ्ं ॅय॒ज्ञ्ꣳ स्वाहा॒ स्वाहा॑ य॒ज्ञ्ं ॅवाता॒द् वाता᳚द् य॒ज्ञ्ꣳ स्वाहा॒ स्वाहा॑ य॒ज्ञ्ं ॅवाता᳚त् । \newline
33. य॒ज्ञ्ं ॅवाता॒द् वाता᳚द् य॒ज्ञ्ं ॅय॒ज्ञ्ं ॅवाता॒दा वाता᳚द् य॒ज्ञ्ं ॅय॒ज्ञ्ं ॅवाता॒दा । \newline
34. वाता॒दा वाता॒द् वाता॒दा र॑भे रभ॒ आ वाता॒द् वाता॒दा र॑भे । \newline
35. आ र॑भे रभ॒ आ र॑भे । \newline
36. र॒भ॒ इति॑ रभे । \newline
\pagebreak
\markright{ TS 1.2.3.1  \hfill https://www.vedavms.in \hfill}

\section{ TS 1.2.3.1 }

\textbf{TS 1.2.3.1 } \newline
\textbf{Samhita Paata} \newline

दैवीं॒ धियं॑ मनामहे सुमृडी॒का-म॒भिष्ट॑ये वर्चो॒धां ॅय॒ज्ञ्वा॑हसꣳ सुपा॒रा नो॑ अस॒द् वशे᳚ । ये दे॒वा मनो॑जाता मनो॒युजः॑ सु॒दक्षा॒ दक्ष॑पितार॒स्ते नः॑ पान्तु॒ ते नो॑ऽवन्तु॒ तेभ्यो॒ नम॒स्तेभ्यः॒ स्वाहाऽग्ने॒ त्वꣳ सु जा॑गृहि व॒यꣳ सु म॑न्दिषीमहि गोपा॒य नः॑ स्व॒स्तये᳚ प्र॒बुधे॑ नः॒ पुन॑र्ददः । त्वम॑ग्ने व्रत॒पा अ॑सि दे॒व आ मर्त्ये॒ष्वा । त्वं - [ ] \newline

\textbf{Pada Paata} \newline

दैवी᳚म् । धिय᳚म् । म॒ना॒म॒हे॒ । सु॒मृ॒डी॒कामिति॑ सु - मृ॒डी॒काम् । अ॒भिष्ट॑ये । व॒र्चो॒धामिति॑ वर्चः - धाम् । य॒ज्ञ्वा॑हस॒मिति॑ य॒ज्ञ् - वा॒ह॒स॒म् । सु॒पा॒रेति॑ सु - पा॒रा । नः॒ । अ॒स॒त् । वशे᳚ ॥ ये । दे॒वाः । मनो॑जाता॒ इति॒ मनः॑ - जा॒ताः॒ । म॒नो॒युज॒ इति॑ मनः - युजः॑ । सु॒दक्षा॒ इति॑ सु - दक्षाः᳚ । दक्ष॑पितार॒ इति॒ दक्ष॑- पि॒ता॒रः॒ । ते । नः॒ । पा॒न्तु॒ । ते । नः॒ । अ॒व॒न्तु॒ । तेभ्यः॑ । नमः॑ । तेभ्यः॑ । स्वाहा᳚ । अग्ने᳚ । त्वम् । स्विति॑ । जा॒गृ॒हि॒ । व॒यम् । स्विति॑ । म॒न्दि॒षी॒म॒हि॒ । गो॒पा॒य । नः॒ । स्व॒स्तये᳚ । प्र॒बुध॒ इति॑ प्र - बुधे᳚ । नः॒ । पुनः॑ । द॒दः॒ ॥ त्वम् । अ॒ग्ने॒ । व्र॒त॒पा इति॑ व्रत - पाः । अ॒सि॒ । दे॒वः । एति॑ । मर्त्ये॑षु । आ ॥ त्वम् ।  \newline


\textbf{Krama Paata} \newline

दैवी॒म् धिय᳚म् । धिय॑म् मनामहे । म॒ना॒म॒हे॒ सु॒मृ॒डी॒काम् । सु॒मृ॒डी॒काम॒भिष्ट॑ये । सु॒मृ॒डी॒कामिति॑ सु - मृ॒डी॒काम् । अ॒भिष्ट॑ये वर्चो॒धाम् । व॒र्चो॒धां ॅय॒ज्ञ्वा॑हसम् । व॒र्चो॒धामिति॑ वर्चः - धाम् । य॒ज्ञ्वा॑हसꣳ सुपा॒रा । य॒ज्ञ्वा॑हस॒मिति॑ य॒ज्ञ् - वा॒ह॒स॒म् । सु॒पा॒रा नः॑ । सु॒पा॒रेति॑ सु - पा॒रा । नो॒ अ॒स॒त्॒ । अ॒स॒द्वशे᳚ । वश॒ इति॒ वशे᳚ ॥ ये दे॒वाः । दे॒वा मनो॑जाताः । मनो॑जाता मनो॒युजः॑ । मनो॑जाता॒ इति॒ मनः॑ - जा॒ताः॒ । म॒नो॒युजः॑ सु॒दक्षाः᳚ । म॒नो॒युज॒ इति॑ मनः - युजः॑ । सु॒दक्षा॒ दक्ष॑पितारः । सु॒दक्षा॒ इति॑ सु - दक्षाः᳚ । दक्ष॑पितार॒स्ते । दक्ष॑पितार॒ इति॒ दक्ष॑ - पि॒ता॒रः॒ । ते नः॑ । नः॒ पा॒न्तु॒ । पा॒न्तु॒ ते । ते नः॑ । नो॒ऽव॒न्तु॒ । अ॒व॒न्तु॒ तेभ्यः॑ । तेभ्यो॒ नमः॑ । नम॒स्तेभ्यः॑ । तेभ्यः॒ स्वाहा᳚ । स्वाहाऽग्ने᳚ । अग्ने॒ त्वम् । त्वꣳ सु । सु जा॑गृहि । जा॒गृ॒हि॒ व॒यम् । व॒यꣳ सु । सु म॑न्दिषीमहि । म॒न्दि॒षी॒म॒हि॒ गो॒पा॒य । गो॒पा॒य नः॑ । नः॒ स्व॒स्तये᳚ । स्व॒स्तये᳚ प्र॒बुधे᳚ । प्र॒बुधे॑ नः । प्र॒बुध॒ इति॑ प्र - बुधे᳚ । नः॒ पुनः॑ । पुन॑र् ददः । द॒द॒ इति॑ ददः ॥ त्वम॑ग्ने । अ॒ग्ने॒ व्र॒त॒पाः । व्र॒त॒पा अ॑सि । व्र॒त॒पा इति॑ व्रत - पाः । अ॒सि॒ दे॒वः । दे॒व आ । आ मर्त्ये॑षु । मर्त्ये॒ष्वा । एत्या ॥ त्वं ॅय॒ज्ञेषु॑ \newline

\textbf{Jatai Paata} \newline

1. दैवी॒म् धिय॒म् धिय॒म् दैवी॒म् दैवी॒म् धिय᳚म् । \newline
2. धिय॑म् मनामहे मनामहे॒ धिय॒म् धिय॑म् मनामहे । \newline
3. म॒ना॒म॒हे॒ सु॒मृ॒डी॒काꣳ सु॑मृडी॒काम् म॑नामहे मनामहे सुमृडी॒काम् । \newline
4. सु॒मृ॒डी॒का म॒भिष्ट॑ये॒ ऽभिष्ट॑ये सुमृडी॒काꣳ सु॑मृडी॒का म॒भिष्ट॑ये । \newline
5. सु॒मृ॒डी॒कामिति॑ सु - मृ॒डी॒काम् । \newline
6. अ॒भिष्ट॑ये वर्चो॒धां ॅव॑र्चो॒धा म॒भिष्ट॑ये॒ ऽभिष्ट॑ये वर्चो॒धाम् । \newline
7. व॒र्चो॒धां ॅय॒ज्ञ्वा॑हसं ॅय॒ज्ञ्वा॑हसं ॅवर्चो॒धां ॅव॑र्चो॒धां ॅय॒ज्ञ्वा॑हसम् । \newline
8. व॒र्चो॒धामिति॑ वर्चः - धाम् । \newline
9. य॒ज्ञ्वा॑हसꣳ सुपा॒रा सु॑पा॒रा य॒ज्ञ्वा॑हसं ॅय॒ज्ञ्वा॑हसꣳ सुपा॒रा । \newline
10. य॒ज्ञ्वा॑हस॒मिति॑ य॒ज्ञ् - वा॒ह॒स॒म् । \newline
11. सु॒पा॒रा नो॑ नः सुपा॒रा सु॑पा॒रा नः॑ । \newline
12. सु॒पा॒रेति॑ सु - पा॒रा । \newline
13. नो॒ अ॒स॒द॒स॒न् नो॒ नो॒ अ॒स॒त् । \newline
14. अ॒स॒द् वशे॒ वशे॑ ऽस दस॒द् वशे᳚ । \newline
15. वश॒ इति॒ वशे᳚ । \newline
16. ये दे॒वा दे॒वा ये ये दे॒वाः । \newline
17. दे॒वा मनो॑जाता॒ मनो॑जाता दे॒वा दे॒वा मनो॑जाताः । \newline
18. मनो॑जाता मनो॒युजो॑ मनो॒युजो॒ मनो॑जाता॒ मनो॑जाता मनो॒युजः॑ । \newline
19. मनो॑जाता॒ इति॒ मनः॑ - जा॒ताः॒ । \newline
20. म॒नो॒युजः॑ सु॒दक्षाः᳚ सु॒दक्षा॑ मनो॒युजो॑ मनो॒युजः॑ सु॒दक्षाः᳚ । \newline
21. म॒नो॒युज॒ इति॑ मनः - युजः॑ । \newline
22. सु॒दक्षा॒ दक्ष॑पितारो॒ दक्ष॑पितारः सु॒दक्षाः᳚ सु॒दक्षा॒ दक्ष॑पितारः । \newline
23. सु॒दक्षा॒ इति॑ सु - दक्षाः᳚ । \newline
24. दक्ष॑पितार॒ स्ते ते दक्ष॑पितारो॒ दक्ष॑पितार॒ स्ते । \newline
25. दक्ष॑पितार॒ इति॒ दक्ष॑ - पि॒ता॒रः॒ । \newline
26. ते नो॑ न॒ स्ते ते नः॑ । \newline
27. नः॒ पा॒न्तु॒ पा॒न्तु॒ नो॒ नः॒ पा॒न्तु॒ । \newline
28. पा॒न्तु॒ ते ते पा᳚न्तु पान्तु॒ ते । \newline
29. ते नो॑ न॒ स्ते ते नः॑ । \newline
30. नो॒ ऽव॒ न्त्व॒व॒न्तु॒ नो॒ नो॒ ऽव॒न्तु॒ । \newline
31. अ॒व॒न्तु॒ तेभ्य॒ स्तेभ्यो॑ ऽव न्त्ववन्तु॒ तेभ्यः॑ । \newline
32. तेभ्यो॒ नमो॒ नम॒ स्तेभ्य॒ स्तेभ्यो॒ नमः॑ । \newline
33. नम॒ स्तेभ्य॒ स्तेभ्यो॒ नमो॒ नम॒ स्तेभ्यः॑ । \newline
34. तेभ्यः॒ स्वाहा॒ स्वाहा॒ तेभ्य॒ स्तेभ्यः॒ स्वाहा᳚ । \newline
35. स्वाहा ऽग्ने॒ अग्ने॒ स्वाहा॒ स्वाहा ऽग्ने᳚ । \newline
36. अग्ने॒ त्वम् त्व मग्ने ऽग्ने॒ त्वम् । \newline
37. त्वꣳ सु सु त्वम् त्वꣳ सु । \newline
38. सु जा॑गृहि जागृहि॒ सु सु जा॑गृहि । \newline
39. जा॒गृ॒हि॒ व॒यं ॅव॒यम् जा॑गृहि जागृहि व॒यम् । \newline
40. व॒यꣳ सु सु व॒यं ॅव॒यꣳ सु । \newline
41. सु म॑न्दिषीमहि मन्दिषीमहि॒ सु सु म॑न्दिषीमहि । \newline
42. म॒न्दि॒षी॒म॒हि॒ गो॒पा॒य गो॑पा॒य म॑न्दिषीमहि मन्दिषीमहि गोपा॒य । \newline
43. गो॒पा॒य नो॑ नो गोपा॒य गो॑पा॒य नः॑ । \newline
44. नः॒ स्व॒स्तये᳚ स्व॒स्तये॑ नो नः स्व॒स्तये᳚ । \newline
45. स्व॒स्तये᳚ प्र॒बुधे᳚ प्र॒बुधे᳚ स्व॒स्तये᳚ स्व॒स्तये᳚ प्र॒बुधे᳚ । \newline
46. प्र॒बुधे॑ नो नः प्र॒बुधे᳚ प्र॒बुधे॑ नः । \newline
47. प्र॒बुध॒ इति॑ प्र - बुधे᳚ । \newline
48. नः॒ पुनः॒ पुन॑र् नो नः॒ पुनः॑ । \newline
49. पुन॑र् ददो ददः॒ पुनः॒ पुन॑र् ददः । \newline
50. द॒द॒ इति॑ ददः । \newline
51. त्व म॑ग्ने अग्ने॒ त्वम् त्व म॑ग्ने । \newline
52. अ॒ग्ने॒ व्र॒त॒पा व्र॑त॒पा अ॑ग्ने अग्ने व्रत॒पाः । \newline
53. व्र॒त॒पा अ॑स्यसि व्रत॒पा व्र॑त॒पा अ॑सि । \newline
54. व्र॒त॒पा इति॑ व्रत - पाः । \newline
55. अ॒सि॒ दे॒वो दे॒वो᳚ ऽस्यसि दे॒वः । \newline
56. दे॒व आ दे॒वो दे॒व आ । \newline
57. आ मर्त्ये॑षु॒ मर्त्ये॒ष्वा मर्त्ये॑षु । \newline
58. मर्त्ये॒ष्वा मर्त्ये॑षु॒ मर्त्ये॒ष्वा । \newline
59. एत्या । \newline
60. त्वं ॅय॒ज्ञेषु॑ य॒ज्ञेषु॒ त्वम् त्वं ॅय॒ज्ञेषु॑ । \newline

\textbf{Ghana Paata } \newline

1. दैवी॒म् धिय॒म् धिय॒म् दैवी॒म् दैवी॒म् धिय॑म् मनामहे मनामहे॒ धिय॒म् दैवी॒म् दैवी॒म् धिय॑म् मनामहे । \newline
2. धिय॑म् मनामहे मनामहे॒ धिय॒म् धिय॑म् मनामहे सुमृडी॒काꣳ सु॑मृडी॒काम् म॑नामहे॒ धिय॒म् धिय॑म् मनामहे सुमृडी॒काम् । \newline
3. म॒ना॒म॒हे॒ सु॒मृ॒डी॒काꣳ सु॑मृडी॒काम् म॑नामहे मनामहे सुमृडी॒का म॒भिष्ट॑ये॒ ऽभिष्ट॑ये सुमृडी॒काम् म॑नामहे मनामहे सुमृडी॒का म॒भिष्ट॑ये । \newline
4. सु॒मृ॒डी॒का म॒भिष्ट॑ये॒ ऽभिष्ट॑ये सुमृडी॒काꣳ सु॑मृडी॒का म॒भिष्ट॑ये वर्चो॒धां ॅव॑र्चो॒धा म॒भिष्ट॑ये सुमृडी॒काꣳ सु॑मृडी॒का म॒भिष्ट॑ये वर्चो॒धाम् । \newline
5. सु॒मृ॒डी॒कामिति॑ सु - मृ॒डी॒काम् । \newline
6. अ॒भिष्ट॑ये वर्चो॒धां ॅव॑र्चो॒धा म॒भिष्ट॑ये॒ ऽभिष्ट॑ये वर्चो॒धां ॅय॒ज्ञ्वा॑हसं ॅय॒ज्ञ्वा॑हसं ॅवर्चो॒धा म॒भिष्ट॑ये॒ ऽभिष्ट॑ये वर्चो॒धां ॅय॒ज्ञ्वा॑हसम् । \newline
7. व॒र्चो॒धां ॅय॒ज्ञ्वा॑हसं ॅय॒ज्ञ्वा॑हसं ॅवर्चो॒धां ॅव॑र्चो॒धां ॅय॒ज्ञ्वा॑हसꣳ सुपा॒रा सु॑पा॒रा य॒ज्ञ्वा॑हसं ॅवर्चो॒धां ॅव॑र्चो॒धां ॅय॒ज्ञ्वा॑हसꣳ सुपा॒रा । \newline
8. व॒र्चो॒धामिति॑ वर्चः - धाम् । \newline
9. य॒ज्ञ्वा॑हसꣳ सुपा॒रा सु॑पा॒रा य॒ज्ञ्वा॑हसं ॅय॒ज्ञ्वा॑हसꣳ सुपा॒रा नो॑ नः सुपा॒रा य॒ज्ञ्वा॑हसं ॅय॒ज्ञ्वा॑हसꣳ सुपा॒रा नः॑ । \newline
10. य॒ज्ञ्वा॑हस॒मिति॑ य॒ज्ञ् - वा॒ह॒स॒म् । \newline
11. सु॒पा॒रा नो॑ नः सुपा॒रा सु॑पा॒रा नो॑ असदसन् नः सुपा॒रा सु॑पा॒रा नो॑ असत् । \newline
12. सु॒पा॒रेति॑ सु - पा॒रा । \newline
13. नो॒ अ॒स॒द॒स॒न् नो॒ नो॒ अ॒स॒द् वशे॒ वशे॑ ऽसन् नो नो अस॒द् वशे᳚ । \newline
14. अ॒स॒द् वशे॒ वशे॑ ऽसदस॒द् वशे᳚ । \newline
15. वश॒ इति॒ वशे᳚ । \newline
16. ये दे॒वा दे॒वा ये ये दे॒वा मनो॑जाता॒ मनो॑जाता दे॒वा ये ये दे॒वा मनो॑जाताः । \newline
17. दे॒वा मनो॑जाता॒ मनो॑जाता दे॒वा दे॒वा मनो॑जाता मनो॒युजो॑ मनो॒युजो॒ मनो॑जाता दे॒वा दे॒वा मनो॑जाता मनो॒युजः॑ । \newline
18. मनो॑जाता मनो॒युजो॑ मनो॒युजो॒ मनो॑जाता॒ मनो॑जाता मनो॒युजः॑ सु॒दक्षाः᳚ सु॒दक्षा॑ मनो॒युजो॒ मनो॑जाता॒ मनो॑जाता मनो॒युजः॑ सु॒दक्षाः᳚ । \newline
19. मनो॑जाता॒ इति॒ मनः॑ - जा॒ताः॒ । \newline
20. म॒नो॒युजः॑ सु॒दक्षाः᳚ सु॒दक्षा॑ मनो॒युजो॑ मनो॒युजः॑ सु॒दक्षा॒ दक्ष॑पितारो॒ दक्ष॑पितारः सु॒दक्षा॑ मनो॒युजो॑ मनो॒युजः॑ सु॒दक्षा॒ दक्ष॑पितारः । \newline
21. म॒नो॒युज॒ इति॑ मनः - युजः॑ । \newline
22. सु॒दक्षा॒ दक्ष॑पितारो॒ दक्ष॑पितारः सु॒दक्षाः᳚ सु॒दक्षा॒ दक्ष॑पितार॒ स्ते ते दक्ष॑पितारः सु॒दक्षाः᳚ सु॒दक्षा॒ दक्ष॑पितार॒ स्ते । \newline
23. सु॒दक्षा॒ इति॑ सु - दक्षाः᳚ । \newline
24. दक्ष॑पितार॒ स्ते ते दक्ष॑पितारो॒ दक्ष॑पितार॒ स्ते नो॑ न॒स्ते दक्ष॑पितारो॒ दक्ष॑पितार॒ स्ते नः॑ । \newline
25. दक्ष॑पितार॒ इति॒ दक्ष॑ - पि॒ता॒रः॒ । \newline
26. ते नो॑ न॒स्ते ते नः॑ पान्तु पान्तु न॒स्ते ते नः॑ पान्तु । \newline
27. नः॒ पा॒न्तु॒ पा॒न्तु॒ नो॒ नः॒ पा॒न्तु॒ ते ते पा᳚न्तु नो नः पान्तु॒ ते । \newline
28. पा॒न्तु॒ ते ते पा᳚न्तु पान्तु॒ ते नो॑ न॒स्ते पा᳚न्तु पान्तु॒ ते नः॑ । \newline
29. ते नो॑ न॒स्ते ते नो॑ ऽवन्त्ववन्तु न॒स्ते ते नो॑ ऽवन्तु । \newline
30. नो॒ ऽव॒न्त्व॒व॒न्तु॒ नो॒ नो॒ ऽव॒न्तु॒ तेभ्य॒स्तेभ्यो॑ ऽवन्तु नो नो ऽवन्तु॒ तेभ्यः॑ । \newline
31. अ॒व॒न्तु॒ तेभ्य॒स्तेभ्यो॑ ऽवन्त्ववन्तु॒ तेभ्यो॒ नमो॒ नम॒स्तेभ्यो॑ ऽवन्त्ववन्तु॒ तेभ्यो॒ नमः॑ । \newline
32. तेभ्यो॒ नमो॒ नम॒ स्तेभ्य॒ स्तेभ्यो॒ नम॒ स्तेभ्य॒ स्तेभ्यो॒ नम॒ स्तेभ्य॒ स्तेभ्यो॒ नम॒ स्तेभ्यः॑ । \newline
33. नम॒ स्तेभ्य॒ स्तेभ्यो॒ नमो॒ नम॒स्तेभ्यः॒ स्वाहा॒ स्वाहा॒ तेभ्यो॒ नमो॒ नम॒स्तेभ्यः॒ स्वाहा᳚ । \newline
34. तेभ्यः॒ स्वाहा॒ स्वाहा॒ तेभ्य॒स्तेभ्यः॒ स्वाहा ऽग्ने ऽग्ने॒ स्वाहा॒ तेभ्य॒स्तेभ्यः॒ स्वाहा ऽग्ने᳚ । \newline
35. स्वाहा ऽग्ने ऽग्ने॒ स्वाहा॒ स्वाहा ऽग्ने॒ त्वम् त्व मग्ने॒ स्वाहा॒ स्वाहा ऽग्ने॒ त्वम् । \newline
36. अग्ने॒ त्वम् त्व मग्ने ऽग्ने॒ त्वꣳ सु सु त्व मग्ने ऽग्ने॒ त्वꣳ सु । \newline
37. त्वꣳ सु सु त्वम् त्वꣳ सु जा॑गृहि जागृहि॒ सु त्वम् त्वꣳ सु जा॑गृहि । \newline
38. सु जा॑गृहि जागृहि॒ सु सु जा॑गृहि व॒यं ॅव॒यम् जा॑गृहि॒ सु सु जा॑गृहि व॒यम् । \newline
39. जा॒गृ॒हि॒ व॒यं ॅव॒यम् जा॑गृहि जागृहि व॒यꣳ सु सु व॒यम् जा॑गृहि जागृहि व॒यꣳ सु । \newline
40. व॒यꣳ सु सु व॒यं ॅव॒यꣳ सु म॑न्दिषीमहि मन्दिषीमहि॒ सु व॒यं ॅव॒यꣳ सु म॑न्दिषीमहि । \newline
41. सु म॑न्दिषीमहि मन्दिषीमहि॒ सु सु म॑न्दिषीमहि गोपा॒य गो॑पा॒य म॑न्दिषीमहि॒ सु सु म॑न्दिषीमहि गोपा॒य । \newline
42. म॒न्दि॒षी॒म॒हि॒ गो॒पा॒य गो॑पा॒य म॑न्दिषीमहि मन्दिषीमहि गोपा॒य नो॑ नो गोपा॒य म॑न्दिषीमहि मन्दिषीमहि गोपा॒य नः॑ । \newline
43. गो॒पा॒य नो॑ नो गोपा॒य गो॑पा॒य नः॑ स्व॒स्तये᳚ स्व॒स्तये॑ नो गोपा॒य गो॑पा॒य नः॑ स्व॒स्तये᳚ । \newline
44. नः॒ स्व॒स्तये᳚ स्व॒स्तये॑ नो नः स्व॒स्तये᳚ प्र॒बुधे᳚ प्र॒बुधे᳚ स्व॒स्तये॑ नो नः स्व॒स्तये᳚ प्र॒बुधे᳚ । \newline
45. स्व॒स्तये᳚ प्र॒बुधे᳚ प्र॒बुधे᳚ स्व॒स्तये᳚ स्व॒स्तये᳚ प्र॒बुधे॑ नो नः प्र॒बुधे᳚ स्व॒स्तये᳚ स्व॒स्तये᳚ प्र॒बुधे॑ नः । \newline
46. प्र॒बुधे॑ नो नः प्र॒बुधे᳚ प्र॒बुधे॑ नः॒ पुनः॒ पुन॑र् नः प्र॒बुधे᳚ प्र॒बुधे॑ नः॒ पुनः॑ । \newline
47. प्र॒बुध॒ इति॑ प्र - बुधे᳚ । \newline
48. नः॒ पुनः॒ पुन॑र् नो नः॒ पुन॑र् ददो ददः॒ पुन॑र् नो नः॒ पुन॑र् ददः । \newline
49. पुन॑र् ददो ददः॒ पुनः॒ पुन॑र् ददः । \newline
50. द॒द॒ इति॑ ददः । \newline
51. त्व म॑ग्ने अग्ने॒ त्वम् त्व म॑ग्ने व्रत॒पा व्र॑त॒पा अ॑ग्ने॒ त्वम् त्व म॑ग्ने व्रत॒पाः । \newline
52. अ॒ग्ने॒ व्र॒त॒पा व्र॑त॒पा अ॑ग्ने अग्ने व्रत॒पा अ॑स्यसि व्रत॒पा अ॑ग्ने अग्ने व्रत॒पा अ॑सि । \newline
53. व्र॒त॒पा अ॑स्यसि व्रत॒पा व्र॑त॒पा अ॑सि दे॒वो दे॒वो॑ ऽसि व्रत॒पा व्र॑त॒पा अ॑सि दे॒वः । \newline
54. व्र॒त॒पा इति॑ व्रत - पाः । \newline
55. अ॒सि॒ दे॒वो दे॒वो᳚ ऽस्यसि दे॒व आ दे॒वो᳚ ऽस्यसि दे॒व आ । \newline
56. दे॒व आ दे॒वो दे॒व आ मर्त्ये॑षु॒ मर्त्ये॒ष्वा दे॒वो दे॒व आ मर्त्ये॑षु । \newline
57. आ मर्त्ये॑षु॒ मर्त्ये॒ष्वा मर्त्ये॒ष्वा मर्त्ये॒ष्वा मर्त्ये॒ष्वा । \newline
58. मर्त्ये॒ष्वा मर्त्ये॑षु॒ मर्त्ये॒ष्वा । \newline
59. एत्या । \newline
60. त्वं ॅय॒ज्ञेषु॑ य॒ज्ञेषु॒ त्वम् त्वं ॅय॒ज्ञेष्वीड्य॒ ईड्यो॑ य॒ज्ञेषु॒ त्वम् त्वं ॅय॒ज्ञेष्वीड्यः॑ । \newline
\pagebreak
\markright{ TS 1.2.3.2  \hfill https://www.vedavms.in \hfill}

\section{ TS 1.2.3.2 }

\textbf{TS 1.2.3.2 } \newline
\textbf{Samhita Paata} \newline

ॅय॒ज्ञेष्वीड्यः॑ ॥ विश्वे॑ दे॒वा अ॒भि मामाऽव॑वृत्रन् पू॒षा स॒न्या सोमो॒ राध॑सा दे॒वः स॑वि॒ता वसो᳚र्वसु॒दावा॒ रास्वेय॑थ् सो॒मा ऽऽ*भूयो॑ भर॒ मा पृ॒णन् पू॒र्त्या वि रा॑धि॒ माऽहमायु॑षा च॒न्द्रम॑सि॒ मम॒ भोगा॑य भव॒ वस्त्र॑मसि॒ मम॒ भोगा॑य भवो॒स्राऽसि॒ मम॒ भोगा॑य भव॒ हयो॑ऽसि॒ मम॒ भोगा॑य भव॒ - [ ] \newline

\textbf{Pada Paata} \newline

य॒ज्ञेषु॑ । ईड्‍यः॑ ॥ विश्वे᳚ । दे॒वाः । अ॒भीति॑ । माम् । एति॑ । अ॒व॒वृ॒त्र॒न्न् । पू॒षा । स॒न्या । सोमः॑ । राध॑सा । दे॒वः । स॒वि॒ता । वसोः᳚ । व॒सु॒दावेति॑ वसु - दावा᳚ । रास्व॑ । इय॑त् । सो॒म॒ । एति॑ । भूयः॑ । भ॒र॒ । मा । पृ॒णन्न् । पू॒र्त्या । वीति॑ । रा॒धि॒ । मा । अ॒हम् । आयु॑षा । च॒न्द्रम् । अ॒सि॒ । मम॑ । भोगा॑य । भ॒व॒ । वस्त्र᳚म् । अ॒सि॒ । मम॑ । भोगा॑य । भ॒व॒ । उ॒स्रा । अ॒सि॒ । मम॑ । भोगा॑य । भ॒व॒ । हयः॑ । अ॒सि॒ । मम॑ । भोगा॑य । भ॒व॒ ।  \newline


\textbf{Krama Paata} \newline

य॒ज्ञेष्वीड्यः॑ । ईड्य॒ इतीड्यः॑ ॥ विश्वे॑ दे॒वाः । दे॒वा अ॒भि । अ॒भि माम् । मामा । आऽव॑वृत्रन्न् । अ॒व॒वृ॒त्र॒न् पू॒षा । पू॒षा स॒न्या । स॒न्या सोमः॑ । सोमो॒ राध॑सा । राध॑सा दे॒वः । दे॒वः स॑वि॒ता । स॒वि॒ता वसोः᳚ । वसो᳚र् वसु॒दावा᳚ । व॒सु॒दावा॒ रास्व॑ । व॒सु॒दावेति॑ वसु - दावा᳚ । रास्वेय॑त् । इय॑थ् सो॒म । सो॒मा । आ भूयः॑ । भूयो॑ भर । भ॒र॒ मा । मा पृ॒णन्न् । पृ॒णन् पू॒र्त्या । पू॒र्त्या वि । वि रा॑धि । रा॒धि॒ मा । माऽहम् । अ॒हमायु॑षा । आयु॑षा च॒न्द्रम् । च॒न्द्रम॑सि । अ॒सि॒ मम॑ । मम॒ भोगा॑य । भोगा॑य भव । भ॒व॒ वस्त्र᳚म् । वस्त्र॑मसि । अ॒सि॒ मम॑ । मम॒ भोगा॑य । भोगा॑य भव । भ॒वो॒स्रा । उ॒स्राऽसि॑ । अ॒सि॒ मम॑ । मम॒ भोगा॑य । भोगा॑य भव । भ॒व॒ हयः॑ । हयो॑ऽसि । अ॒सि॒ मम॑ । मम॒ भोगा॑य । भोगा॑य भव । भ॒व॒ छागः॑ \newline

\textbf{Jatai Paata} \newline

1. य॒ज्ञे ष्वीड्य॒ ईड्यो॑ य॒ज्ञेषु॑ य॒ज्ञे ष्वीड्यः॑ । \newline
2. ईड्य॒ इतीड्यः॑ । \newline
3. विश्वे॑ दे॒वा दे॒वा विश्वे॒ विश्वे॑ दे॒वाः । \newline
4. दे॒वा अ॒भ्य॑भि दे॒वा दे॒वा अ॒भि । \newline
5. अ॒भि माम् मा म॒भ्य॑भि माम् । \newline
6. मा मा माम् मा मा । \newline
7. आ ऽव॑वृत्रन् नववृत्र॒न् ना ऽव॑वृत्रन्न् । \newline
8. अ॒व॒वृ॒त्र॒न् पू॒षा पू॒षा ऽव॑वृत्रन् नववृत्रन् पू॒षा । \newline
9. पू॒षा स॒न्या स॒न्या पू॒षा पू॒षा स॒न्या । \newline
10. स॒न्या सोमः॒ सोमः॑ स॒न्या स॒न्या सोमः॑ । \newline
11. सोमो॒ राध॑सा॒ राध॑सा॒ सोमः॒ सोमो॒ राध॑सा । \newline
12. राध॑सा दे॒वो दे॒वो राध॑सा॒ राध॑सा दे॒वः । \newline
13. दे॒वः स॑वि॒ता स॑वि॒ता दे॒वो दे॒वः स॑वि॒ता । \newline
14. स॒वि॒ता वसो॒र् वसोः᳚ सवि॒ता स॑वि॒ता वसोः᳚ । \newline
15. वसो᳚र् वसु॒दावा॑ वसु॒दावा॒ वसो॒र् वसो᳚र् वसु॒दावा᳚ । \newline
16. व॒सु॒दावा॒ रास्व॒ रास्व॑ वसु॒दावा॑ वसु॒दावा॒ रास्व॑ । \newline
17. व॒सु॒दावेति॑ वसु - दावा᳚ । \newline
18. रास्वे य॒दिय॒द् रास्व॒ रास्वे य॑त् । \newline
19. इय॑थ् सोम सो॒मे य॒दिय॑थ् सोम । \newline
20. सो॒मा सो॑म सो॒मा । \newline
21. आ भूयो॒ भूय॒ आ भूयः॑ । \newline
22. भूयो॑ भर भर॒ भूयो॒ भूयो॑ भर । \newline
23. भ॒र॒ मा मा भ॑र भर॒ मा । \newline
24. मा पृ॒णन् पृ॒णन् मा मा पृ॒णन्न् । \newline
25. पृ॒णन् पू॒र्त्या पू॒र्त्या पृ॒णन् पृ॒णन् पू॒र्त्या । \newline
26. पू॒र्त्या वि वि पू॒र्त्या पू॒र्त्या वि । \newline
27. वि रा॑धि राधि॒ वि वि रा॑धि । \newline
28. रा॒धि॒ मा मा रा॑धि राधि॒ मा । \newline
29. मा ऽह म॒हम् मा मा ऽहम् । \newline
30. अ॒ह मायु॒षा ऽऽयु॑षा॒ ऽह म॒ह मायु॑षा । \newline
31. आयु॑षा च॒न्द्रम् च॒न्द्र मायु॒षा ऽऽयु॑षा च॒न्द्रम् । \newline
32. च॒न्द्र म॑स्यसि च॒न्द्रम् च॒न्द्र म॑सि । \newline
33. अ॒सि॒ मम॒ ममा᳚ स्यसि॒ मम॑ । \newline
34. मम॒ भोगा॑य॒ भोगा॑य॒ मम॒ मम॒ भोगा॑य । \newline
35. भोगा॑य भव भव॒ भोगा॑य॒ भोगा॑य भव । \newline
36. भ॒व॒ वस्त्रं॒ ॅवस्त्र॑म् भव भव॒ वस्त्र᳚म् । \newline
37. वस्त्र॑ मस्यसि॒ वस्त्रं॒ ॅवस्त्र॑ मसि । \newline
38. अ॒सि॒ मम॒ ममा᳚ स्यसि॒ मम॑ । \newline
39. मम॒ भोगा॑य॒ भोगा॑य॒ मम॒ मम॒ भोगा॑य । \newline
40. भोगा॑य भव भव॒ भोगा॑य॒ भोगा॑य भव । \newline
41. भ॒वो॒ स्रोस्रा भ॑व भवो॒स्रा । \newline
42. उ॒स्रा ऽस्य॑ स्यु॒स्रोस्रा ऽसि॑ । \newline
43. अ॒सि॒ मम॒ ममा᳚ स्यसि॒ मम॑ । \newline
44. मम॒ भोगा॑य॒ भोगा॑य॒ मम॒ मम॒ भोगा॑य । \newline
45. भोगा॑य भव भव॒ भोगा॑य॒ भोगा॑य भव । \newline
46. भ॒व॒ हयो॒ हयो॑ भव भव॒ हयः॑ । \newline
47. हयो᳚ ऽस्यसि॒ हयो॒ हयो॑ ऽसि । \newline
48. अ॒सि॒ मम॒ ममा᳚ स्यसि॒ मम॑ । \newline
49. मम॒ भोगा॑य॒ भोगा॑य॒ मम॒ मम॒ भोगा॑य । \newline
50. भोगा॑य भव भव॒ भोगा॑य॒ भोगा॑य भव । \newline
51. भ॒व॒ छाग॒ श्छागो॑ भव भव॒ छागः॑ । \newline

\textbf{Ghana Paata } \newline

1. य॒ज्ञेष्वीड्य॒ ईड्यो॑ य॒ज्ञेषु॑ य॒ज्ञेष्वीड्यः॑ । \newline
2. ईड्य॒ इतीड्यः॑ । \newline
3. विश्वे॑ दे॒वा दे॒वा विश्वे॒ विश्वे॑ दे॒वा अ॒भ्य॑भि दे॒वा विश्वे॒ विश्वे॑ दे॒वा अ॒भि । \newline
4. दे॒वा अ॒भ्य॑भि दे॒वा दे॒वा अ॒भि माम् मा म॒भि दे॒वा दे॒वा अ॒भि माम् । \newline
5. अ॒भि माम् मा म॒भ्य॑भि मा मा मा म॒भ्य॑भि मा मा । \newline
6. मा मा माम् मा मा ऽव॑वृत्रन् नववृत्र॒न् ना माम् मा मा ऽव॑वृत्रन्न् । \newline
7. आ ऽव॑वृत्रन् नववृत्र॒न् ना ऽव॑वृत्रन् पू॒षा पू॒षा ऽव॑वृत्र॒न् ना ऽव॑वृत्रन् पू॒षा । \newline
8. अ॒व॒वृ॒त्र॒न् पू॒षा पू॒षा ऽव॑वृत्रन् नववृत्रन् पू॒षा स॒न्या स॒न्या पू॒षा ऽव॑वृत्रन् नववृत्रन् पू॒षा स॒न्या । \newline
9. पू॒षा स॒न्या स॒न्या पू॒षा पू॒षा स॒न्या सोमः॒ सोमः॑ स॒न्या पू॒षा पू॒षा स॒न्या सोमः॑ । \newline
10. स॒न्या सोमः॒ सोमः॑ स॒न्या स॒न्या सोमो॒ राध॑सा॒ राध॑सा॒ सोमः॑ स॒न्या स॒न्या सोमो॒ राध॑सा । \newline
11. सोमो॒ राध॑सा॒ राध॑सा॒ सोमः॒ सोमो॒ राध॑सा दे॒वो दे॒वो राध॑सा॒ सोमः॒ सोमो॒ राध॑सा दे॒वः । \newline
12. राध॑सा दे॒वो दे॒वो राध॑सा॒ राध॑सा दे॒वः स॑वि॒ता स॑वि॒ता दे॒वो राध॑सा॒ राध॑सा दे॒वः स॑वि॒ता । \newline
13. दे॒वः स॑वि॒ता स॑वि॒ता दे॒वो दे॒वः स॑वि॒ता वसो॒र् वसोः᳚ सवि॒ता दे॒वो दे॒वः स॑वि॒ता वसोः᳚ । \newline
14. स॒वि॒ता वसो॒र् वसोः᳚ सवि॒ता स॑वि॒ता वसो᳚र् वसु॒दावा॑ वसु॒दावा॒ वसोः᳚ सवि॒ता स॑वि॒ता वसो᳚र् वसु॒दावा᳚ । \newline
15. वसो᳚र् वसु॒दावा॑ वसु॒दावा॒ वसो॒र् वसो᳚र् वसु॒दावा॒ रास्व॒ रास्व॑ वसु॒दावा॒ वसो॒र् वसो᳚र् वसु॒दावा॒ रास्व॑ । \newline
16. व॒सु॒दावा॒ रास्व॒ रास्व॑ वसु॒दावा॑ वसु॒दावा॒ रास्वे य॒दिय॒द् रास्व॑ वसु॒दावा॑ वसु॒दावा॒ रास्वे य॑त् । \newline
17. व॒सु॒दावेति॑ वसु - दावा᳚ । \newline
18. रास्वे य॒दिय॒द् रास्व॒ रास्वे य॑थ् सोम सो॒मे य॒द् रास्व॒ रास्वे य॑थ् सोम । \newline
19. इय॑थ् सोम सो॒मे य॒दिय॑थ् सो॒मा सो॒मे य॒दिय॑थ् सो॒मा । \newline
20. सो॒मा सो॑म सो॒मा भूयो॒ भूय॒ आ सो॑म सो॒मा भूयः॑ । \newline
21. आ भूयो॒ भूय॒ आ भूयो॑ भर भर॒ भूय॒ आ भूयो॑ भर । \newline
22. भूयो॑ भर भर॒ भूयो॒ भूयो॑ भर॒ मा मा भ॑र॒ भूयो॒ भूयो॑ भर॒ मा । \newline
23. भ॒र॒ मा मा भ॑र भर॒ मा पृ॒णन् पृ॒णन् मा भ॑र भर॒ मा पृ॒णन्न् । \newline
24. मा पृ॒णन् पृ॒णन् मा मा पृ॒णन् पू॒र्त्या पू॒र्त्या पृ॒णन् मा मा पृ॒णन् पू॒र्त्या । \newline
25. पृ॒णन् पू॒र्त्या पू॒र्त्या पृ॒णन् पृ॒णन् पू॒र्त्या वि वि पू॒र्त्या पृ॒णन् पृ॒णन् पू॒र्त्या वि । \newline
26. पू॒र्त्या वि वि पू॒र्त्या पू॒र्त्या वि रा॑धि राधि॒ वि पू॒र्त्या पू॒र्त्या वि रा॑धि । \newline
27. वि रा॑धि राधि॒ वि वि रा॑धि॒ मा मा रा॑धि॒ वि वि रा॑धि॒ मा । \newline
28. रा॒धि॒ मा मा रा॑धि राधि॒ मा ऽह म॒हम् मा रा॑धि राधि॒ मा ऽहम् । \newline
29. मा ऽह म॒हम् मा मा ऽह मायु॒षा ऽऽयु॑षा॒ ऽहम् मा मा ऽह मायु॑षा । \newline
30. अ॒ह मायु॒षा ऽऽयु॑षा॒ ऽह म॒ह मायु॑षा च॒न्द्रम् च॒न्द्र मायु॑षा॒ ऽह म॒ह मायु॑षा च॒न्द्रम् । \newline
31. आयु॑षा च॒न्द्रम् च॒न्द्र मायु॒षा ऽऽयु॑षा च॒न्द्र म॑स्यसि च॒न्द्र मायु॒षा ऽऽयु॑षा च॒न्द्र म॑सि । \newline
32. च॒न्द्र म॑स्यसि च॒न्द्रम् च॒न्द्र म॑सि॒ मम॒ ममा॑सि च॒न्द्रम् च॒न्द्र म॑सि॒ मम॑ । \newline
33. अ॒सि॒ मम॒ ममा᳚स्यसि॒ मम॒ भोगा॑य॒ भोगा॑य॒ ममा᳚स्यसि॒ मम॒ भोगा॑य । \newline
34. मम॒ भोगा॑य॒ भोगा॑य॒ मम॒ मम॒ भोगा॑य भव भव॒ भोगा॑य॒ मम॒ मम॒ भोगा॑य भव । \newline
35. भोगा॑य भव भव॒ भोगा॑य॒ भोगा॑य भव॒ वस्त्रं॒ ॅवस्त्र॑म् भव॒ भोगा॑य॒ भोगा॑य भव॒ वस्त्र᳚म् । \newline
36. भ॒व॒ वस्त्रं॒ ॅवस्त्र॑म् भव भव॒ वस्त्र॑ मस्यसि॒ वस्त्र॑म् भव भव॒ वस्त्र॑ मसि । \newline
37. वस्त्र॑ मस्यसि॒ वस्त्रं॒ ॅवस्त्र॑ मसि॒ मम॒ ममा॑सि॒ वस्त्रं॒ ॅवस्त्र॑ मसि॒ मम॑ । \newline
38. अ॒सि॒ मम॒ ममा᳚स्यसि॒ मम॒ भोगा॑य॒ भोगा॑य॒ ममा᳚स्यसि॒ मम॒ भोगा॑य । \newline
39. मम॒ भोगा॑य॒ भोगा॑य॒ मम॒ मम॒ भोगा॑य भव भव॒ भोगा॑य॒ मम॒ मम॒ भोगा॑य भव । \newline
40. भोगा॑य भव भव॒ भोगा॑य॒ भोगा॑य भवो॒स्रोस्रा भ॑व॒ भोगा॑य॒ भोगा॑य भवो॒स्रा । \newline
41. भ॒वो॒स्रोस्रा भ॑व भवो॒स्रा ऽस्य॑स्यु॒स्रा भ॑व भवो॒स्रा ऽसि॑ । \newline
42. उ॒स्रा ऽस्य॑स्यु॒स्रोस्रा ऽसि॒ मम॒ ममा᳚स्यु॒स्रोस्रा ऽसि॒ मम॑ । \newline
43. अ॒सि॒ मम॒ ममा᳚स्यसि॒ मम॒ भोगा॑य॒ भोगा॑य॒ ममा᳚स्यसि॒ मम॒ भोगा॑य । \newline
44. मम॒ भोगा॑य॒ भोगा॑य॒ मम॒ मम॒ भोगा॑य भव भव॒ भोगा॑य॒ मम॒ मम॒ भोगा॑य भव । \newline
45. भोगा॑य भव भव॒ भोगा॑य॒ भोगा॑य भव॒ हयो॒ हयो॑ भव॒ भोगा॑य॒ भोगा॑य भव॒ हयः॑ । \newline
46. भ॒व॒ हयो॒ हयो॑ भव भव॒ हयो᳚ ऽस्यसि॒ हयो॑ भव भव॒ हयो॑ ऽसि । \newline
47. हयो᳚ ऽस्यसि॒ हयो॒ हयो॑ ऽसि॒ मम॒ ममा॑सि॒ हयो॒ हयो॑ ऽसि॒ मम॑ । \newline
48. अ॒सि॒ मम॒ ममा᳚स्यसि॒ मम॒ भोगा॑य॒ भोगा॑य॒ ममा᳚स्यसि॒ मम॒ भोगा॑य । \newline
49. मम॒ भोगा॑य॒ भोगा॑य॒ मम॒ मम॒ भोगा॑य भव भव॒ भोगा॑य॒ मम॒ मम॒ भोगा॑य भव । \newline
50. भोगा॑य भव भव॒ भोगा॑य॒ भोगा॑य भव॒ छाग॒ श्छागो॑ भव॒ भोगा॑य॒ भोगा॑य भव॒ छागः॑ । \newline
51. भ॒व॒ छाग॒ श्छागो॑ भव भव॒ छागो᳚ ऽस्यसि॒ छागो॑ भव भव॒ छागो॑ ऽसि । \newline
\pagebreak
\markright{ TS 1.2.3.3  \hfill https://www.vedavms.in \hfill}

\section{ TS 1.2.3.3 }

\textbf{TS 1.2.3.3 } \newline
\textbf{Samhita Paata} \newline

छागो॑ऽसि॒ मम॒ भोगा॑य भव मे॒षो॑ऽसि॒ मम॒ भोगा॑य भव वा॒यवे᳚ त्वा॒ वरु॑णाय त्वा॒ निर्.ऋ॑त्यै त्वा रु॒द्राय॑ त्वा॒ देवी॑रापो अपां नपा॒द्य ऊ॒र्मिर्. ह॑वि॒ष्य॑ इन्द्रि॒यावा᳚न् म॒दिन्त॑म॒स्तं ॅवो॒ माऽव॑ क्रमिष॒मच्छि॑न्नं॒ तन्तुं॑ पृथि॒व्या अनु॑ गेषं भ॒द्राद॒भि श्रेयः॒ प्रेहि॒ बृह॒स्पतिः॑ पुरए॒ता ते॑ अ॒स्त्वथे॒मव॑ स्य॒ ( ) वर॒ आ पृ॑थि॒व्या आ॒रे शत्रू᳚न् कृणुहि॒ सर्व॑वीर॒ एदम॑गन्म देव॒यज॑नं पृथि॒व्या विश्वे॑ दे॒वा यदजु॑षन्त॒ पूर्व॑ ऋख्सा॒माभ्यां॒ ॅयजु॑षा स॒न्तर॑न्तो रा॒यस्पोषे॑ण॒ समि॒षा म॑देम ॥ \newline

\textbf{Pada Paata} \newline

छागः॑ । अ॒सि॒ । मम॑ । भोगा॑य । भ॒व॒ । मे॒षः । अ॒सि॒ । मम॑ । भोगा॑य । भ॒व॒ । वा॒यवे᳚ । त्वा॒ । वरु॑णाय । त्वा॒ । निर्.ऋ॑त्या॒ इति॒ निः - ऋ॒त्यै॒ । त्वा॒ । रु॒द्राय॑ । त्वा॒ । देवीः᳚ । आ॒पः॒ । अ॒पा॒म् । न॒पा॒त् । यः । ऊ॒र्मिः । ह॒वि॒ष्यः॑ । इ॒न्द्रि॒यावा॒निती᳚न्द्रि॒य - वा॒न् । म॒दिन्त॑मः । तम् । वः॒ । मा । अवेति॑ । क्र॒मि॒ष॒म् । अच्छि॑न्नम् । तन्तु᳚म् । पृ॒थि॒व्याः । अन्विति॑ । गे॒ष॒म् । भ॒द्रात् । अ॒भीति॑ । श्रेयः॑ । प्रेति॑ । इ॒हि॒ । बृह॒स्पतिः॑ । पु॒र॒ ए॒तेति॑ पुरः - ए॒ता । ते॒ । अ॒स्तु॒ । अथ॑ । ई॒म् । अवेति॑ । स्य॒ ( ) । वरे᳚ । एति॑ । पृ॒थि॒व्याः । आ॒रे । शत्रून्॑ । कृ॒णु॒हि॒ । सर्व॑वीर॒ इति॒ सर्व॑ - वी॒रः॒ । एति॑ । इ॒दम् । अ॒ग॒न्म॒ । दे॒व॒यज॑न॒मिति॑ देव - यज॑नम् । पृ॒थि॒व्याः । विश्वे᳚ । दे॒वाः । यत् । अजु॑षन्त । पूर्वे᳚ । ऋ॒ख्सा॒माभ्या॒मित्यृ॑ख्सा॒म -भ्या॒म् । यजु॑षा । सं॒तर॑न्त॒ इति॑ सं - तर॑न्तः । रा॒यः । पोषे॑ण । समिति॑ । इ॒षा । म॒दे॒म॒ ॥  \newline


\textbf{Krama Paata} \newline

छागो॑ऽसि । अ॒सि॒ मम॑ । मम॒ भोगा॑य । भोगा॑य भव । भ॒व॒ मे॒षः । मे॒षो॑ऽसि । अ॒सि॒ मम॑ । मम॒ भोगा॑य । भोगा॑य भव । भ॒व॒ वा॒यवे᳚ । वा॒यवे᳚ त्वा । त्वा॒ वरु॑णाय । वरु॑णाय त्वा । त्वा॒ निर्.ऋ॑त्यै । निर्.ऋ॑त्यै त्वा । निर्.ऋ॑त्या॒ इति॒ निः - ऋ॒त्यै॒ । त्वा॒ रु॒द्राय॑ । रु॒द्राय॑ त्वा । त्वा॒ देवीः᳚ । देवी॑रापः । आ॒पो॒ अ॒पा॒म् । अ॒पा॒न्न॒पा॒त्॒ । न॒पा॒द्यः । य ऊ॒र्मिः । ऊ॒र्मिर्. ह॑वि॒ष्यः॑ । ह॒वि॒ष्य॑ इन्द्रि॒यावान्॑ । इ॒न्द्रि॒यावा᳚न् म॒दिन्त॑मः । इ॒न्द्रि॒यावा॒निती᳚न्द्रि॒य - वा॒न्॒ । म॒दिन्त॑म॒स्तम् । तं ॅवः॑ । वो॒ मा । माऽव॑ । अव॑ क्रमिषम् । क्र॒मि॒ष॒मच्छि॑न्नम् । अच्छि॑न्न॒म् तन्तु᳚म् । तन्तु॑म् पृथि॒व्याः । पृ॒थि॒व्या अनु॑ । अनु॑ गेषम् । गे॒ष॒म् भ॒द्रात् । भ॒द्राद॒भि । अ॒भि श्रेयः॑ । श्रेयः॒ प्र । प्रेहि॑ । इ॒हि॒ बृह॒स्पतिः॑ । बृह॒स्पतिः॑ पुरए॒ता । पु॒र॒ए॒ता ते᳚ । पु॒र॒ए॒तेति॑ पुरः - ए॒ता । ते॒ अ॒स्तु॒ । अ॒स्त्वथ॑ । अथे᳚म् । ई॒मव॑ । अव॑ स्य ( ) । स्य॒ वरे᳚ । वर॒ आ । आ पृ॑थि॒व्याः । पृ॒थि॒व्या आ॒रे । आ॒रे शत्रून्॑ । शत्रू᳚न् कृणुहि । कृ॒णु॒हि॒ सर्व॑वीरः । सर्व॑वीर॒ आ । सर्व॑वीर॒ इति॒ सर्व॑ - वी॒रः॒ । एदम् । इ॒दम॑गन्म । अ॒ग॒न्म॒ दे॒व॒यज॑नम् । दे॒व॒यज॑नम् पृथि॒व्याः । दे॒व॒यज॑न॒मिति॑ देव - यज॑नम् । पृ॒थि॒व्या विश्वे᳚ । विश्वे॑ दे॒वाः । दे॒वा यत् । यदजु॑षन्त । अजु॑षन्त॒ पूर्वे᳚ । पूर्व॑ ऋख्सा॒माभ्या᳚म् । ऋ॒ख्सा॒माभ्यां॒ ॅयजु॑षा । ऋ॒ख्सा॒माभ्या॒मित्यृ॑ख्सा॒म - भ्या॒म् । यजु॑षा स॒न्तर॑न्तः । स॒न्तर॑न्तो रा॒यः । स॒न्तर॑न्त॒ इति॑ सम् - तर॑न्तः । रा॒यस्पोषे॑ण । पोषे॑ण॒ सम् । समि॒षा । इ॒षा म॑देम । म॒दे॒मेति॑ मदेम । \newline

\textbf{Jatai Paata} \newline

1. छागो᳚ ऽस्यसि॒ छाग॒ श्छागो॑ ऽसि । \newline
2. अ॒सि॒ मम॒ ममा᳚ स्यसि॒ मम॑ । \newline
3. मम॒ भोगा॑य॒ भोगा॑य॒ मम॒ मम॒ भोगा॑य । \newline
4. भोगा॑य भव भव॒ भोगा॑य॒ भोगा॑य भव । \newline
5. भ॒व॒ मे॒षो मे॒षो भ॑व भव मे॒षः । \newline
6. मे॒षो᳚ ऽस्यसि मे॒षो मे॒षो॑ ऽसि । \newline
7. अ॒सि॒ मम॒ ममा᳚ स्यसि॒ मम॑ । \newline
8. मम॒ भोगा॑य॒ भोगा॑य॒ मम॒ मम॒ भोगा॑य । \newline
9. भोगा॑य भव भव॒ भोगा॑य॒ भोगा॑य भव । \newline
10. भ॒व॒ वा॒यवे॑ वा॒यवे॑ भव भव वा॒यवे᳚ । \newline
11. वा॒यवे᳚ त्वा त्वा वा॒यवे॑ वा॒यवे᳚ त्वा । \newline
12. त्वा॒ वरु॑णाय॒ वरु॑णाय त्वा त्वा॒ वरु॑णाय । \newline
13. वरु॑णाय त्वा त्वा॒ वरु॑णाय॒ वरु॑णाय त्वा । \newline
14. त्वा॒ निर्.ऋ॑त्यै॒ निर्.ऋ॑त्यै त्वा त्वा॒ निर्.ऋ॑त्यै । \newline
15. निर्.ऋ॑त्यै त्वा त्वा॒ निर्.ऋ॑त्यै॒ निर्.ऋ॑त्यै त्वा । \newline
16. निर्.ऋ॑त्या॒ इति॒ निः - ऋ॒त्यै॒ । \newline
17. त्वा॒ रु॒द्राय॑ रु॒द्राय॑ त्वा त्वा रु॒द्राय॑ । \newline
18. रु॒द्राय॑ त्वा त्वा रु॒द्राय॑ रु॒द्राय॑ त्वा । \newline
19. त्वा॒ देवी॒र् देवी᳚ स्त्वा त्वा॒ देवीः᳚ । \newline
20. देवी॑ राप आपो॒ देवी॒र् देवी॑ रापः । \newline
21. आ॒पो॒ अ॒पा॒ म॒पा॒ मा॒प॒ आ॒पो॒ अ॒पा॒म् । \newline
22. अ॒पा॒म् न॒पा॒न् न॒पा॒द॒पा॒ म॒पा॒म् न॒पा॒त् । \newline
23. न॒पा॒द् यो यो न॑पान् नपा॒द् यः । \newline
24. य ऊ॒र्मि रू॒र्मिर् यो य ऊ॒र्मिः । \newline
25. ऊ॒र्मिर्. ह॑वि॒ष्यो॑ हवि॒ष्य॑ ऊ॒र्मि रू॒र्मिर्. ह॑वि॒ष्यः॑ । \newline
26. ह॒वि॒ष्य॑ इन्द्रि॒यावा॑ निन्द्रि॒यावान्॑. हवि॒ष्यो॑ हवि॒ष्य॑ इन्द्रि॒यावान्॑ । \newline
27. इ॒न्द्रि॒यावा᳚न् म॒दिन्त॑मो म॒दिन्त॑म इन्द्रि॒यावा॑ निन्द्रि॒यावा᳚न् म॒दिन्त॑मः । \newline
28. इ॒न्द्रि॒यावा॒निती᳚न्द्रि॒य - वा॒न् । \newline
29. म॒दिन्त॑म॒ स्तम् तम् म॒दिन्त॑मो म॒दिन्त॑म॒ स्तम् । \newline
30. तं ॅवो॑ व॒ स्तम् तं ॅवः॑ । \newline
31. वो॒ मा मा वो॑ वो॒ मा । \newline
32. मा ऽवाव॒ मा मा ऽव॑ । \newline
33. अव॑ क्रमिषम् क्रमिष॒ मवाव॑ क्रमिषम् । \newline
34. क्र॒मि॒ष॒ मच्छि॑न्न॒ मच्छि॑न्नम् क्रमिषम् क्रमिष॒ मच्छि॑न्नम् । \newline
35. अच्छि॑न्न॒म् तन्तु॒म् तन्तु॒ मच्छि॑न्न॒ मच्छि॑न्न॒म् तन्तु᳚म् । \newline
36. तन्तु॑म् पृथि॒व्याः पृ॑थि॒व्या स्तन्तु॒म् तन्तु॑म् पृथि॒व्याः । \newline
37. पृ॒थि॒व्या अन्वनु॑ पृथि॒व्याः पृ॑थि॒व्या अनु॑ । \newline
38. अनु॑ गेषम् गेष॒ मन्वनु॑ गेषम् । \newline
39. गे॒ष॒म् भ॒द्राद् भ॒द्राद् गे॑षम् गेषम् भ॒द्रात् । \newline
40. भ॒द्रा द॒भ्य॑भि भ॒द्राद् भ॒द्रा द॒भि । \newline
41. अ॒भि श्रेयः॒ श्रेयो॒ ऽभ्य॑भि श्रेयः॑ । \newline
42. श्रेयः॒ प्र प्र श्रेयः॒ श्रेयः॒ प्र । \newline
43. प्रे ही॑हि॒ प्र प्रे हि॑ । \newline
44. इ॒हि॒ बृह॒स्पति॒र् बृह॒स्पति॑ रिहीहि॒ बृह॒स्पतिः॑ । \newline
45. बृह॒स्पतिः॑ पुरए॒ता पु॑रए॒ता बृह॒स्पति॒र् बृह॒स्पतिः॑ पुरए॒ता । \newline
46. पु॒र॒ए॒ता ते॑ ते पुरए॒ता पु॑रए॒ता ते᳚ । \newline
47. पु॒र॒ए॒तेति॑ पुरः - ए॒ता । \newline
48. ते॒ अ॒स्त्व॒स्तु॒ ते॒ ते॒ अ॒स्तु॒ । \newline
49. अ॒स्त्व थाथा᳚ स्त्व॒ स्त्वथ॑ । \newline
50. अथे॑ मी॒ मथाथे᳚म् । \newline
51. ई॒ मवावे॑ मी॒ मव॑ । \newline
52. अव॑ स्य॒ स्यावाव॑ स्य । \newline
53. स्य॒ वरे॒ वरे᳚ स्य स्य॒ वरे᳚ । \newline
54. वर॒ आ वरे॒ वर॒ आ । \newline
55. आ पृ॑थि॒व्याः पृ॑थि॒व्या आ पृ॑थि॒व्याः । \newline
56. पृ॒थि॒व्या आ॒र आ॒रे पृ॑थि॒व्याः पृ॑थि॒व्या आ॒रे । \newline
57. आ॒रे शत्रू॒ञ् छत्रू॑ ना॒र आ॒रे शत्रून्॑ । \newline
58. शत्रू᳚न् कृणुहि कृणुहि॒ शत्रू॒ञ् छत्रू᳚न् कृणुहि । \newline
59. कृ॒णु॒हि॒ सर्व॑वीरः॒ सर्व॑वीरः कृणुहि कृणुहि॒ सर्व॑वीरः । \newline
60. सर्व॑वीर॒ आ सर्व॑वीरः॒ सर्व॑वीर॒ आ । \newline
61. सर्व॑वीर॒ इति॒ सर्व॑ - वी॒रः॒ । \newline
62. एद मि॒द मेदम् । \newline
63. इ॒द म॑गन्मागन्मे॒ द मि॒द म॑गन्म । \newline
64. अ॒ग॒न्म॒ दे॒व॒यज॑नम् देव॒यज॑न मगन्मागन्म देव॒यज॑नम् । \newline
65. दे॒व॒यज॑नम् पृथि॒व्याः पृ॑थि॒व्या दे॑व॒यज॑नम् देव॒यज॑नम् पृथि॒व्याः । \newline
66. दे॒व॒यज॑न॒मिति॑ देव - यज॑नम् । \newline
67. पृ॒थि॒व्या विश्वे॒ विश्वे॑ पृथि॒व्याः पृ॑थि॒व्या विश्वे᳚ । \newline
68. विश्वे॑ दे॒वा दे॒वा विश्वे॒ विश्वे॑ दे॒वाः । \newline
69. दे॒वा यद् यद् दे॒वा दे॒वा यत् । \newline
70. यदजु॑ष॒ न्ताजु॑षन्त॒ यद् यदजु॑षन्त । \newline
71. अजु॑षन्त॒ पूर्वे॒ पूर्वे ऽजु॑ष॒ न्ताजु॑षन्त॒ पूर्वे᳚ । \newline
72. पूर्व॑ ऋख्सा॒माभ्या॑ मृख्सा॒माभ्या॒म् पूर्वे॒ पूर्व॑ ऋख्सा॒माभ्या᳚म् । \newline
73. ऋ॒ख्सा॒माभ्यां॒ ॅयजु॑षा॒ यजु॑षर्ख्सा॒माभ्या॑ मृख्सा॒माभ्यां॒ ॅयजु॑षा । \newline
74. ऋ॒ख्सा॒माभ्या॒मित्यृ॑ख्सा॒म - भ्या॒म् । \newline
75. यजु॑षा स॒न्तर॑न्तः स॒न्तर॑न्तो॒ यजु॑षा॒ यजु॑षा स॒न्तर॑न्तः । \newline
76. स॒न्तर॑न्तो रा॒यो रा॒यः स॒न्तर॑न्तः स॒न्तर॑न्तो रा॒यः । \newline
77. स॒न्तर॑न्त॒ इति॑ सं - तर॑न्तः । \newline
78. रा॒य स्पोषे॑ण॒ पोषे॑ण रा॒यो रा॒य स्पोषे॑ण । \newline
79. पोषे॑ण॒ सꣳ सम् पोषे॑ण॒ पोषे॑ण॒ सम् । \newline
80. स मि॒षेषा सꣳ स मि॒षा । \newline
81. इ॒षा म॑देम मदेमे॒ षेषा म॑देम । \newline
82. म॒दे॒मेति॑ मदेम । \newline

\textbf{Ghana Paata } \newline

1. छागो᳚ ऽस्यसि॒ छाग॒ श्छागो॑ ऽसि॒ मम॒ ममा॑सि॒ छाग॒ श्छागो॑ ऽसि॒ मम॑ । \newline
2. अ॒सि॒ मम॒ ममा᳚स्यसि॒ मम॒ भोगा॑य॒ भोगा॑य॒ ममा᳚स्यसि॒ मम॒ भोगा॑य । \newline
3. मम॒ भोगा॑य॒ भोगा॑य॒ मम॒ मम॒ भोगा॑य भव भव॒ भोगा॑य॒ मम॒ मम॒ भोगा॑य भव । \newline
4. भोगा॑य भव भव॒ भोगा॑य॒ भोगा॑य भव मे॒षो मे॒षो भ॑व॒ भोगा॑य॒ भोगा॑य भव मे॒षः । \newline
5. भ॒व॒ मे॒षो मे॒षो भ॑व भव मे॒षो᳚ ऽस्यसि मे॒षो भ॑व भव मे॒षो॑ ऽसि । \newline
6. मे॒षो᳚ ऽस्यसि मे॒षो मे॒षो॑ ऽसि॒ मम॒ ममा॑सि मे॒षो मे॒षो॑ ऽसि॒ मम॑ । \newline
7. अ॒सि॒ मम॒ ममा᳚स्यसि॒ मम॒ भोगा॑य॒ भोगा॑य॒ ममा᳚स्यसि॒ मम॒ भोगा॑य । \newline
8. मम॒ भोगा॑य॒ भोगा॑य॒ मम॒ मम॒ भोगा॑य भव भव॒ भोगा॑य॒ मम॒ मम॒ भोगा॑य भव । \newline
9. भोगा॑य भव भव॒ भोगा॑य॒ भोगा॑य भव वा॒यवे॑ वा॒यवे॑ भव॒ भोगा॑य॒ भोगा॑य भव वा॒यवे᳚ । \newline
10. भ॒व॒ वा॒यवे॑ वा॒यवे॑ भव भव वा॒यवे᳚ त्वा त्वा वा॒यवे॑ भव भव वा॒यवे᳚ त्वा । \newline
11. वा॒यवे᳚ त्वा त्वा वा॒यवे॑ वा॒यवे᳚ त्वा॒ वरु॑णाय॒ वरु॑णाय त्वा वा॒यवे॑ वा॒यवे᳚ त्वा॒ वरु॑णाय । \newline
12. त्वा॒ वरु॑णाय॒ वरु॑णाय त्वा त्वा॒ वरु॑णाय त्वा त्वा॒ वरु॑णाय त्वा त्वा॒ वरु॑णाय त्वा । \newline
13. वरु॑णाय त्वा त्वा॒ वरु॑णाय॒ वरु॑णाय त्वा॒ निर्.ऋ॑त्यै॒ निर्.ऋ॑त्यै त्वा॒ वरु॑णाय॒ वरु॑णाय त्वा॒ निर्.ऋ॑त्यै । \newline
14. त्वा॒ निर्.ऋ॑त्यै॒ निर्.ऋ॑त्यै त्वा त्वा॒ निर्.ऋ॑त्यै त्वा त्वा॒ निर्.ऋ॑त्यै त्वा त्वा॒ निर्.ऋ॑त्यै त्वा । \newline
15. निर्.ऋ॑त्यै त्वा त्वा॒ निर्.ऋ॑त्यै॒ निर्.ऋ॑त्यै त्वा रु॒द्राय॑ रु॒द्राय॑ त्वा॒ निर्.ऋ॑त्यै॒ निर्.ऋ॑त्यै त्वा रु॒द्राय॑ । \newline
16. निर्.ऋ॑त्या॒ इति॒ निः - ऋ॒त्यै॒ । \newline
17. त्वा॒ रु॒द्राय॑ रु॒द्राय॑ त्वा त्वा रु॒द्राय॑ त्वा त्वा रु॒द्राय॑ त्वा त्वा रु॒द्राय॑ त्वा । \newline
18. रु॒द्राय॑ त्वा त्वा रु॒द्राय॑ रु॒द्राय॑ त्वा॒ देवी॒र् देवी᳚स्त्वा रु॒द्राय॑ रु॒द्राय॑ त्वा॒ देवीः᳚ । \newline
19. त्वा॒ देवी॒र् देवी᳚स्त्वा त्वा॒ देवी॑राप आपो॒ देवी᳚स्त्वा त्वा॒ देवी॑रापः । \newline
20. देवी॑राप आपो॒ देवी॒र् देवी॑रापो अपा मपा मापो॒ देवी॒र् देवी॑रापो अपाम् । \newline
21. आ॒पो॒ अ॒पा॒ म॒पा॒ मा॒प॒ आ॒पो॒ अ॒पा॒न्न॒पा॒न् न॒पा॒द॒पा॒ मा॒प॒ आ॒पो॒ अ॒पा॒न्न॒पा॒त् । \newline
22. अ॒पा॒न्न॒पा॒न् न॒पा॒द॒पा॒ म॒पा॒न्न॒पा॒द् यो यो न॑पादपा मपान्नपा॒द् यः । \newline
23. न॒पा॒द् यो यो न॑पान् नपा॒द् य ऊ॒र्मि रू॒र्मिर् यो न॑पान् नपा॒द् य ऊ॒र्मिः । \newline
24. य ऊ॒र्मि रू॒र्मिर् यो य ऊ॒र्मिर्. ह॑वि॒ष्यो॑ हवि॒ष्य॑ ऊ॒र्मिर् यो य ऊ॒र्मिर्. ह॑वि॒ष्यः॑ । \newline
25. ऊ॒र्मिर्. ह॑वि॒ष्यो॑ हवि॒ष्य॑ ऊ॒र्मिरू॒र्मिर्. ह॑वि॒ष्य॑ इन्द्रि॒यावा॑ निन्द्रि॒यावा॑न्. हवि॒ष्य॑ ऊ॒र्मि रू॒र्मिर्. ह॑वि॒ष्य॑ इन्द्रि॒यावान्॑ । \newline
26. ह॒वि॒ष्य॑ इन्द्रि॒यावा॑ निन्द्रि॒यावा॑न्. हवि॒ष्यो॑ हवि॒ष्य॑ इन्द्रि॒यावा᳚न् म॒दिन्त॑मो म॒दिन्त॑म इन्द्रि॒यावा॑न्. हवि॒ष्यो॑ हवि॒ष्य॑ इन्द्रि॒यावा᳚न् म॒दिन्त॑मः । \newline
27. इ॒न्द्रि॒यावा᳚न् म॒दिन्त॑मो म॒दिन्त॑म इन्द्रि॒यावा॑ निन्द्रि॒यावा᳚न् म॒दिन्त॑म॒स्तम् तम् म॒दिन्त॑म इन्द्रि॒यावा॑ निन्द्रि॒यावा᳚न् म॒दिन्त॑म॒स्तम् । \newline
28. इ॒न्द्रि॒यावा॒निती᳚न्द्रि॒य - वा॒न् । \newline
29. म॒दिन्त॑म॒स्तम् तम् म॒दिन्त॑मो म॒दिन्त॑म॒स्तं ॅवो॑ व॒स्तम् म॒दिन्त॑मो म॒दिन्त॑म॒स्तं ॅवः॑ । \newline
30. तं ॅवो॑ व॒स्तम् तं ॅवो॒ मा मा व॒स्तम् तं ॅवो॒ मा । \newline
31. वो॒ मा मा वो॑ वो॒ मा ऽवाव॒ मा वो॑ वो॒ मा ऽव॑ । \newline
32. मा ऽवाव॒ मा मा ऽव॑ क्रमिषम् क्रमिष॒ मव॒ मा मा ऽव॑ क्रमिषम् । \newline
33. अव॑ क्रमिषम् क्रमिष॒ मवाव॑ क्रमिष॒ मच्छि॑न्न॒ मच्छि॑न्नम् क्रमिष॒ मवाव॑ क्रमिष॒ मच्छि॑न्नम् । \newline
34. क्र॒मि॒ष॒ मच्छि॑न्न॒ मच्छि॑न्नम् क्रमिषम् क्रमिष॒ मच्छि॑न्न॒म् तन्तु॒म् तन्तु॒ मच्छि॑न्नम् क्रमिषम् क्रमिष॒ मच्छि॑न्न॒म् तन्तु᳚म् । \newline
35. अच्छि॑न्न॒म् तन्तु॒म् तन्तु॒ मच्छि॑न्न॒ मच्छि॑न्न॒म् तन्तु॑म् पृथि॒व्याः पृ॑थि॒व्या स्तन्तु॒ मच्छि॑न्न॒ मच्छि॑न्न॒म् तन्तु॑म् पृथि॒व्याः । \newline
36. तन्तु॑म् पृथि॒व्याः पृ॑थि॒व्या स्तन्तु॒म् तन्तु॑म् पृथि॒व्या अन्वनु॑ पृथि॒व्या स्तन्तु॒म् तन्तु॑म् पृथि॒व्या अनु॑ । \newline
37. पृ॒थि॒व्या अन्वनु॑ पृथि॒व्याः पृ॑थि॒व्या अनु॑ गेषम् गेष॒ मनु॑ पृथि॒व्याः पृ॑थि॒व्या अनु॑ गेषम् । \newline
38. अनु॑ गेषम् गेष॒ मन्वनु॑ गेषम् भ॒द्राद् भ॒द्राद् गे॑ष॒ मन्वनु॑ गेषम् भ॒द्रात् । \newline
39. गे॒ष॒म् भ॒द्राद् भ॒द्राद् गे॑षम् गेषम् भ॒द्रा द॒भ्य॑भि भ॒द्राद् गे॑षम् गेषम् भ॒द्रा द॒भि । \newline
40. भ॒द्रा द॒भ्य॑भि भ॒द्राद् भ॒द्रा द॒भि श्रेयः॒ श्रेयो॒ ऽभि भ॒द्राद् भ॒द्रा द॒भि श्रेयः॑ । \newline
41. अ॒भि श्रेयः॒ श्रेयो॒ ऽभ्य॑भि श्रेयः॒ प्र प्र श्रेयो॒ ऽभ्य॑भि श्रेयः॒ प्र । \newline
42. श्रेयः॒ प्र प्र श्रेयः॒ श्रेयः॒ प्रे ही॑हि॒ प्र श्रेयः॒ श्रेयः॒ प्रे हि॑ । \newline
43. प्रे ही॑हि॒ प्र प्रे हि॒ बृह॒स्पति॒र् बृह॒स्पति॑रिहि॒ प्र प्रे हि॒ बृह॒स्पतिः॑ । \newline
44. इ॒हि॒ बृह॒स्पति॒र् बृह॒स्पति॑रिहीहि॒ बृह॒स्पतिः॑ पुरए॒ता पु॑रए॒ता बृह॒स्पति॑रिहीहि॒ बृह॒स्पतिः॑ पुरए॒ता । \newline
45. बृह॒स्पतिः॑ पुरए॒ता पु॑रए॒ता बृह॒स्पति॒र् बृह॒स्पतिः॑ पुरए॒ता ते॑ ते पुरए॒ता बृह॒स्पति॒र् बृह॒स्पतिः॑ पुरए॒ता ते᳚ । \newline
46. पु॒र॒ए॒ता ते॑ ते पुरए॒ता पु॑रए॒ता ते॑ अस्त्वस्तु ते पुरए॒ता पु॑रए॒ता ते॑ अस्तु । \newline
47. पु॒र॒ए॒तेति॑ पुरः - ए॒ता । \newline
48. ते॒ अ॒स्त्व॒स्तु॒ ते॒ ते॒ अ॒स्त्वथाथा᳚स्तु ते ते अ॒स्त्वथ॑ । \newline
49. अ॒स्त्वथाथा᳚ स्त्व॒स्त्वथे॑ मी॒ मथा᳚ स्त्व॒ स्त्वथे᳚म् । \newline
50. अथे॑ मी॒ मथाथे॒ मवावे॒ मथाथे॒ मव॑ । \newline
51. ई॒ मवावे॑ मी॒ मव॑ स्य॒ स्यावे॑ मी॒ मव॑ स्य । \newline
52. अव॑ स्य॒ स्यावाव॑ स्य॒ वरे॒ वरे॒ स्यावाव॑ स्य॒ वरे᳚ । \newline
53. स्य॒ वरे॒ वरे᳚ स्य स्य॒ वर॒ आ वरे᳚ स्य स्य॒ वर॒ आ । \newline
54. वर॒ आ वरे॒ वर॒ आ पृ॑थि॒व्याः पृ॑थि॒व्या आ वरे॒ वर॒ आ पृ॑थि॒व्याः । \newline
55. आ पृ॑थि॒व्याः पृ॑थि॒व्या आ पृ॑थि॒व्या आ॒र आ॒रे पृ॑थि॒व्या आ पृ॑थि॒व्या आ॒रे । \newline
56. पृ॒थि॒व्या आ॒र आ॒रे पृ॑थि॒व्याः पृ॑थि॒व्या आ॒रे शत्रू॒ञ् छत्रू॑ ना॒रे पृ॑थि॒व्याः पृ॑थि॒व्या आ॒रे शत्रून्॑ । \newline
57. आ॒रे शत्रू॒ञ् छत्रू॑ ना॒र आ॒रे शत्रू᳚न् कृणुहि कृणुहि॒ शत्रू॑ ना॒र आ॒रे शत्रू᳚न् कृणुहि । \newline
58. शत्रू᳚न् कृणुहि कृणुहि॒ शत्रू॒ञ् छत्रू᳚न् कृणुहि॒ सर्व॑वीरः॒ सर्व॑वीरः कृणुहि॒ शत्रू॒ञ् छत्रू᳚न् कृणुहि॒ सर्व॑वीरः । \newline
59. कृ॒णु॒हि॒ सर्व॑वीरः॒ सर्व॑वीरः कृणुहि कृणुहि॒ सर्व॑वीर॒ आ सर्व॑वीरः कृणुहि कृणुहि॒ सर्व॑वीर॒ आ । \newline
60. सर्व॑वीर॒ आ सर्व॑वीरः॒ सर्व॑वीर॒ एद मि॒द मा सर्व॑वीरः॒ सर्व॑वीर॒ एदम् । \newline
61. सर्व॑वीर॒ इति॒ सर्व॑ - वी॒रः॒ । \newline
62. एद मि॒दम् एद म॑गन्मागन्मे॒ द मेद म॑गन्म । \newline
63. इ॒द म॑गन्मागन्मे॒ द मि॒द म॑गन्म देव॒यज॑नम् देव॒यज॑न मगन्मे॒ द मि॒द म॑गन्म देव॒यज॑नम् । \newline
64. अ॒ग॒न्म॒ दे॒व॒यज॑नम् देव॒यज॑न मगन्मागन्म देव॒यज॑नम् पृथि॒व्याः पृ॑थि॒व्या दे॑व॒यज॑न मगन्मागन्म देव॒यज॑नम् पृथि॒व्याः । \newline
65. दे॒व॒यज॑नम् पृथि॒व्याः पृ॑थि॒व्या दे॑व॒यज॑नम् देव॒यज॑नम् पृथि॒व्या विश्वे॒ विश्वे॑ पृथि॒व्या दे॑व॒यज॑नम् देव॒यज॑नम् पृथि॒व्या विश्वे᳚ । \newline
66. दे॒व॒यज॑न॒मिति॑ देव - यज॑नम् । \newline
67. पृ॒थि॒व्या विश्वे॒ विश्वे॑ पृथि॒व्याः पृ॑थि॒व्या विश्वे॑ दे॒वा दे॒वा विश्वे॑ पृथि॒व्याः पृ॑थि॒व्या विश्वे॑ दे॒वाः । \newline
68. विश्वे॑ दे॒वा दे॒वा विश्वे॒ विश्वे॑ दे॒वा यद् यद् दे॒वा विश्वे॒ विश्वे॑ दे॒वा यत् । \newline
69. दे॒वा यद् यद् दे॒वा दे॒वा यदजु॑ष॒न्ताजु॑षन्त॒ यद् दे॒वा दे॒वा यदजु॑षन्त । \newline
70. यदजु॑ष॒न्ताजु॑षन्त॒ यद् यदजु॑षन्त॒ पूर्वे॒ पूर्वे ऽजु॑षन्त॒ यद् यदजु॑षन्त॒ पूर्वे᳚ । \newline
71. अजु॑षन्त॒ पूर्वे॒ पूर्वे ऽजु॑ष॒न्ताजु॑षन्त॒ पूर्व॑ ऋख्सा॒माभ्या॑ मृख्सा॒माभ्या॒म् पूर्वे ऽजु॑ष॒न्ताजु॑षन्त॒ पूर्व॑ ऋख्सा॒माभ्या᳚म् । \newline
72. पूर्व॑ ऋख्सा॒माभ्या॑ मृख्सा॒माभ्या॒म् पूर्वे॒ पूर्व॑ ऋख्सा॒माभ्यां॒ ॅयजु॑षा॒ यजु॑षर्ख्सा॒माभ्या॒म् पूर्वे॒ पूर्व॑ ऋख्सा॒माभ्यां॒ ॅयजु॑षा । \newline
73. ऋ॒ख्सा॒माभ्यां॒ ॅयजु॑षा॒ यजु॑षर्ख्सा॒माभ्या॑ मृख्सा॒माभ्यां॒ ॅयजु॑षा स॒न्तर॑न्तः स॒न्तर॑न्तो॒ यजु॑षर्ख्सा॒माभ्या॑ मृख्सा॒माभ्यां॒ ॅयजु॑षा स॒न्तर॑न्तः । \newline
74. ऋ॒ख्सा॒माभ्या॒मित्यृ॑ख्सा॒म - भ्या॒म् । \newline
75. यजु॑षा स॒न्तर॑न्तः स॒न्तर॑न्तो॒ यजु॑षा॒ यजु॑षा स॒न्तर॑न्तो रा॒यो रा॒यः स॒न्तर॑न्तो॒ यजु॑षा॒ यजु॑षा स॒न्तर॑न्तो रा॒यः । \newline
76. स॒न्तर॑न्तो रा॒यो रा॒यः स॒न्तर॑न्तः स॒न्तर॑न्तो रा॒य स्पोषे॑ण॒ पोषे॑ण रा॒यः स॒न्तर॑न्तः स॒न्तर॑न्तो 
रा॒य स्पोषे॑ण । \newline
77. स॒न्तर॑न्त॒ इति॑ सं - तर॑न्तः । \newline
78. रा॒य स्पोषे॑ण॒ पोषे॑ण रा॒यो रा॒य स्पोषे॑ण॒ सꣳ सम् पोषे॑ण रा॒यो रा॒य स्पोषे॑ण॒ सम् । \newline
79. पोषे॑ण॒ सꣳ सम् पोषे॑ण॒ पोषे॑ण॒ स मि॒षेषा सम् पोषे॑ण॒ पोषे॑ण॒ स मि॒षा । \newline
80. स मि॒षेषा सꣳ स मि॒षा म॑देम मदेमे॒ षा सꣳ स मि॒षा म॑देम । \newline
81. इ॒षा म॑देम मदेमे॒ षेषा म॑देम । \newline
82. म॒दे॒मेति॑ मदेम । \newline
\pagebreak
\markright{ TS 1.2.4.1  \hfill https://www.vedavms.in \hfill}

\section{ TS 1.2.4.1 }

\textbf{TS 1.2.4.1 } \newline
\textbf{Samhita Paata} \newline

इ॒यं ते॑ शुक्र त॒नूरि॒दं ॅवर्च॒स्तया॒ सं भ॑व॒ भ्राजं॑ गच्छ॒ जूर॑सि धृ॒ता मन॑सा॒ जुष्टा॒ विष्ण॑वे॒ तस्या᳚स्ते स॒त्यस॑वसः प्रस॒वे वा॒चो य॒न्त्रम॑शीय॒ स्वाहा॑ शु॒क्रम॑स्य॒मृत॑मसि वैश्वदे॒वꣳ ह॒विः सूर्य॑स्य॒ चक्षु॒रा -ऽरु॑हम॒ग्ने र॒क्ष्णः क॒नीनि॑कां॒ ॅयदेत॑शेभि॒रीय॑से॒ भ्राज॑मानो विप॒श्चिता॒ चिद॑सि म॒नाऽसि॒ धीर॑सि॒ दक्षि॑णा-[ ] \newline

\textbf{Pada Paata} \newline

इ॒यम् । ते॒ । शु॒क्र॒ । त॒नूः । इ॒दम् । वर्चः॑ । तया᳚ । समिति॑ । भ॒व॒ । भ्राज᳚म् । ग॒च्छ॒ । जूः । अ॒सि॒ । धृ॒ता । मन॑सा । जुष्टा᳚ । विष्ण॑वे । तस्याः᳚ । ते॒ । स॒त्यस॑वस॒ इति॑ स॒त्य - स॒व॒सः॒ । प्र॒स॒वे इति॑ प्र -स॒वे । वा॒चः । य॒न्त्रम् । अ॒शी॒य॒ । स्वाहा᳚ । शु॒क्रम् । अ॒सि॒ । अ॒मृत᳚म् । अ॒सि॒ । वै॒श्व॒दे॒वमिति॑ वैश्व -दे॒वम् । ह॒विः । सूर्य॑स्य । चक्षुः॑ । एति॑ । अ॒रु॒ह॒म् । अ॒ग्नेः । अ॒क्ष्णः । क॒नीनि॑काम् । यत् । एत॑शेभिः । ईय॑से । भ्राज॑मानः । वि॒प॒श्चिता᳚ । चित् । अ॒सि॒ । म॒ना । अ॒सि॒ । धीः । अ॒सि॒ । दक्षि॑णा ।  \newline


\textbf{Krama Paata} \newline

इ॒यम् ते᳚ । ते॒ शु॒क्र॒ । शु॒क्र॒ त॒नूः । त॒नूरि॒दम् । इ॒दं ॅवर्चः॑ । वर्च॒स्तया᳚ । तया॒ सम् । सम् भ॑व । भ॒व॒ भ्राज᳚म् । भ्राज॑म् गच्छ । ग॒च्छ॒ जूः । जूर॑सि । अ॒सि॒ धृ॒ता । धृ॒ता मन॑सा । मन॑सा॒ जुष्टा᳚ । जुष्टा॒ विष्ण॑वे । विष्ण॑वे॒ तस्याः᳚ । तस्या᳚स्ते । ते॒ स॒त्यस॑वसः । स॒त्यस॑वसः प्रस॒वे । स॒त्यस॑वस॒ इति॑ स॒त्य - स॒व॒सः॒ । प्र॒स॒वे वा॒चः । प्र॒स॒व इति॑ प्र - स॒वे । वा॒चो य॒न्त्रम् । य॒न्त्रम॑शीय । अ॒शी॒य॒ स्वाहा᳚ । स्वाहा॑ शु॒क्रम् । शु॒क्रम॑सि । अ॒स्य॒मृत᳚म् । अ॒मृत॑मसि । अ॒सि॒ वै॒श्व॒दे॒वम् । वै॒श्व॒दे॒वꣳ ह॒विः । वै॒श्व॒दे॒वमिति॑ वैश्व - दे॒वम् । ह॒विः सूर्य॑स्य । सूर्य॑स्य॒ चक्षुः॑ । चक्षु॒रा । आ ऽरु॑हम् । अ॒रु॒ह॒म॒ग्नेः । अ॒ग्नेर॒क्ष्णः । अ॒क्ष्णः क॒नीनि॑काम् । क॒नीनि॑कां॒ ॅयत् । यदेत॑शेभिः । एत॑शेभि॒रीय॑से । ईय॑से॒ भ्राज॑मानः । भ्राज॑मानो विप॒श्चिता᳚ । वि॒प॒श्चिता॒ चित् । चिद॑सि । अ॒सि॒ म॒ना । म॒नाऽसि॑ । अ॒सि॒ धीः । धीर॑सि । अ॒सि॒ दक्षि॑णा । दक्षि॑णाऽसि \newline

\textbf{Jatai Paata} \newline

1. इ॒यम् ते॑ त इ॒य मि॒यम् ते᳚ । \newline
2. ते॒ शु॒क्र॒ शु॒क्र॒ ते॒ ते॒ शु॒क्र॒ । \newline
3. शु॒क्र॒ त॒नू स्त॒नूः शु॑क्र शुक्र त॒नूः । \newline
4. त॒नू रि॒द मि॒दम् त॒नू स्त॒नू रि॒दम् । \newline
5. इ॒दं ॅवर्चो॒ वर्च॑ इ॒द मि॒दं ॅवर्चः॑ । \newline
6. वर्च॒ स्तया॒ तया॒ वर्चो॒ वर्च॒ स्तया᳚ । \newline
7. तया॒ सꣳ सम् तया॒ तया॒ सम् । \newline
8. सम् भ॑व भव॒ सꣳ सम् भ॑व । \newline
9. भ॒व॒ भ्राज॒म् भ्राज॑म् भव भव॒ भ्राज᳚म् । \newline
10. भ्राज॑म् गच्छ गच्छ॒ भ्राज॒म् भ्राज॑म् गच्छ । \newline
11. ग॒च्छ॒ जूर् जूर् ग॑च्छ गच्छ॒ जूः । \newline
12. जू र॑स्यसि॒ जूर् जूर॑सि । \newline
13. अ॒सि॒ धृ॒ता धृ॒ता ऽस्य॑सि धृ॒ता । \newline
14. धृ॒ता मन॑सा॒ मन॑सा धृ॒ता धृ॒ता मन॑सा । \newline
15. मन॑सा॒ जुष्टा॒ जुष्टा॒ मन॑सा॒ मन॑सा॒ जुष्टा᳚ । \newline
16. जुष्टा॒ विष्ण॑वे॒ विष्ण॑वे॒ जुष्टा॒ जुष्टा॒ विष्ण॑वे । \newline
17. विष्ण॑वे॒ तस्या॒ स्तस्या॒ विष्ण॑वे॒ विष्ण॑वे॒ तस्याः᳚ । \newline
18. तस्या᳚ स्ते ते॒ तस्या॒ स्तस्या᳚ स्ते । \newline
19. ते॒ स॒त्यस॑वसः स॒त्यस॑वस स्ते ते स॒त्यस॑वसः । \newline
20. स॒त्यस॑वसः प्रस॒वे प्र॑स॒वे स॒त्यस॑वसः स॒त्यस॑वसः प्रस॒वे । \newline
21. स॒त्यस॑वस॒ इति॑ स॒त्य - स॒व॒सः॒ । \newline
22. प्र॒स॒वे वा॒चो वा॒चः प्र॑स॒वे प्र॑स॒वे वा॒चः । \newline
23. प्र॒स॒व इति॑ प्र - स॒वे । \newline
24. वा॒चो य॒न्त्रं ॅय॒न्त्रं ॅवा॒चो वा॒चो य॒न्त्रम् । \newline
25. य॒न्त्र म॑शीयाशीय य॒न्त्रं ॅय॒न्त्र म॑शीय । \newline
26. अ॒शी॒य॒ स्वाहा॒ स्वाहा॑ ऽशीयाशीय॒ स्वाहा᳚ । \newline
27. स्वाहा॑ शु॒क्रꣳ शु॒क्रꣳ स्वाहा॒ स्वाहा॑ शु॒क्रम् । \newline
28. शु॒क्र म॑स्यसि शु॒क्रꣳ शु॒क्र म॑सि । \newline
29. अ॒स्य॒मृत॑ म॒मृत॑ मस्य स्य॒मृत᳚म् । \newline
30. अ॒मृत॑ मस्य स्य॒मृत॑ म॒मृत॑ मसि । \newline
31. अ॒सि॒ वै॒श्व॒दे॒वं ॅवै᳚श्वदे॒व म॑स्यसि वैश्वदे॒वम् । \newline
32. वै॒श्व॒दे॒वꣳ ह॒विर्. ह॒विर् वै᳚श्वदे॒वं ॅवै᳚श्वदे॒वꣳ ह॒विः । \newline
33. वै॒श्व॒दे॒वमिति॑ वैश्व - दे॒वम् । \newline
34. ह॒विः सूर्य॑स्य॒ सूर्य॑स्य ह॒विर्. ह॒विः सूर्य॑स्य । \newline
35. सूर्य॑स्य॒ चक्षु॒ श्चक्षुः॒ सूर्य॑स्य॒ सूर्य॑स्य॒ चक्षुः॑ । \newline
36. चक्षु॒रा चक्षु॒ श्चक्षु॒रा । \newline
37. आ ऽरु॑ह मरुह॒ मा ऽरु॑हम् । \newline
38. अ॒रु॒ह॒ म॒ग्ने र॒ग्ने र॑रुह मरुह म॒ग्नेः । \newline
39. अ॒ग्ने र॒क्ष्णो᳚ (1॒ ओ) ऽक्ष्णो᳚ ऽग्ने र॒ग्ने र॒क्ष्णः । \newline
40. अ॒क्ष्णः क॒नीनि॑काम् क॒नीनि॑का म॒क्ष्णो᳚ ऽक्ष्णः क॒नीनि॑काम् । \newline
41. क॒नीनि॑कां॒ ॅयद् यत् क॒नीनि॑काम् क॒नीनि॑कां॒ ॅयत् । \newline
42. यदेत॑शेभि॒ रेत॑शेभि॒र् यद् यदेत॑शेभिः । \newline
43. एत॑शेभि॒ रीय॑स॒ ईय॑स॒ एत॑शेभि॒ रेत॑शेभि॒ रीय॑से । \newline
44. ईय॑से॒ भ्राज॑मानो॒ भ्राज॑मान॒ ईय॑स॒ ईय॑से॒ भ्राज॑मानः । \newline
45. भ्राज॑मानो विप॒श्चिता॑ विप॒श्चिता॒ भ्राज॑मानो॒ भ्राज॑मानो विप॒श्चिता᳚ । \newline
46. वि॒प॒श्चिता॒ चिच् चिद् वि॑प॒श्चिता॑ विप॒श्चिता॒ चित् । \newline
47. चिद॑ स्यसि॒ चिच् चिद॑सि । \newline
48. अ॒सि॒ म॒ना म॒ना ऽस्य॑सि म॒ना । \newline
49. म॒ना ऽस्य॑सि म॒ना म॒ना ऽसि॑ । \newline
50. अ॒सि॒ धीर् धी र॑स्यसि॒ धीः । \newline
51. धी र॑स्यसि॒ धीर् धीर॑सि । \newline
52. अ॒सि॒ दक्षि॑णा॒ दक्षि॑णा ऽस्यसि॒ दक्षि॑णा । \newline
53. दक्षि॑णा ऽस्यसि॒ दक्षि॑णा॒ दक्षि॑णा ऽसि । \newline

\textbf{Ghana Paata } \newline

1. इ॒यम् ते॑ त इ॒य मि॒यम् ते॑ शुक्र शुक्र त इ॒य मि॒यम् ते॑ शुक्र । \newline
2. ते॒ शु॒क्र॒ शु॒क्र॒ ते॒ ते॒ शु॒क्र॒ त॒नू स्त॒नूः शु॑क्र ते ते शुक्र त॒नूः । \newline
3. शु॒क्र॒ त॒नू स्त॒नूः शु॑क्र शुक्र त॒नूरि॒द मि॒दम् त॒नूः शु॑क्र शुक्र त॒नूरि॒दम् । \newline
4. त॒नू रि॒द मि॒दम् त॒नू स्त॒नू रि॒दं ॅवर्चो॒ वर्च॑ इ॒दम् त॒नू स्त॒नू रि॒दं ॅवर्चः॑ । \newline
5. इ॒दं ॅवर्चो॒ वर्च॑ इ॒द मि॒दं ॅवर्च॒स्तया॒ तया॒ वर्च॑ इ॒द मि॒दं ॅवर्च॒स्तया᳚ । \newline
6. वर्च॒स्तया॒ तया॒ वर्चो॒ वर्च॒स्तया॒ सꣳ सम् तया॒ वर्चो॒ वर्च॒स्तया॒ सम् । \newline
7. तया॒ सꣳ सम् तया॒ तया॒ सम् भ॑व भव॒ सम् तया॒ तया॒ सम् भ॑व । \newline
8. सम् भ॑व भव॒ सꣳ सम् भ॑व॒ भ्राज॒म् भ्राज॑म् भव॒ सꣳ सम् भ॑व॒ भ्राज᳚म् । \newline
9. भ॒व॒ भ्राज॒म् भ्राज॑म् भव भव॒ भ्राज॑म् गच्छ गच्छ॒ भ्राज॑म् भव भव॒ भ्राज॑म् गच्छ । \newline
10. भ्राज॑म् गच्छ गच्छ॒ भ्राज॒म् भ्राज॑म् गच्छ॒ जूर् जूर् ग॑च्छ॒ भ्राज॒म् भ्राज॑म् गच्छ॒ जूः । \newline
11. ग॒च्छ॒ जूर् जूर् ग॑च्छ गच्छ॒ जूर॑ स्यसि॒ जूर् ग॑च्छ गच्छ॒ जू र॑सि । \newline
12. जूर॑ स्यसि॒ जूर् जूर॑सि धृ॒ता धृ॒ता ऽसि॒ जूर् जूर॑सि धृ॒ता । \newline
13. अ॒सि॒ धृ॒ता धृ॒ता ऽस्य॑सि धृ॒ता मन॑सा॒ मन॑सा धृ॒ता ऽस्य॑सि धृ॒ता मन॑सा । \newline
14. धृ॒ता मन॑सा॒ मन॑सा धृ॒ता धृ॒ता मन॑सा॒ जुष्टा॒ जुष्टा॒ मन॑सा धृ॒ता धृ॒ता मन॑सा॒ जुष्टा᳚ । \newline
15. मन॑सा॒ जुष्टा॒ जुष्टा॒ मन॑सा॒ मन॑सा॒ जुष्टा॒ विष्ण॑वे॒ विष्ण॑वे॒ जुष्टा॒ मन॑सा॒ मन॑सा॒ जुष्टा॒ विष्ण॑वे । \newline
16. जुष्टा॒ विष्ण॑वे॒ विष्ण॑वे॒ जुष्टा॒ जुष्टा॒ विष्ण॑वे॒ तस्या॒ स्तस्या॒ विष्ण॑वे॒ जुष्टा॒ जुष्टा॒ विष्ण॑वे॒ तस्याः᳚ । \newline
17. विष्ण॑वे॒ तस्या॒ स्तस्या॒ विष्ण॑वे॒ विष्ण॑वे॒ तस्या᳚स्ते ते॒ तस्या॒ विष्ण॑वे॒ विष्ण॑वे॒ तस्या᳚स्ते । \newline
18. तस्या᳚स्ते ते॒ तस्या॒स्तस्या᳚स्ते स॒त्यस॑वसः स॒त्यस॑वसस्ते॒ तस्या॒स्तस्या᳚स्ते स॒त्यस॑वसः । \newline
19. ते॒ स॒त्यस॑वसः स॒त्यस॑वसस्ते ते स॒त्यस॑वसः प्रस॒वे प्र॑स॒वे स॒त्यस॑वसस्ते ते स॒त्यस॑वसः प्रस॒वे । \newline
20. स॒त्यस॑वसः प्रस॒वे प्र॑स॒वे स॒त्यस॑वसः स॒त्यस॑वसः प्रस॒वे वा॒चो वा॒चः प्र॑स॒वे स॒त्यस॑वसः स॒त्यस॑वसः प्रस॒वे वा॒चः । \newline
21. स॒त्यस॑वस॒ इति॑ स॒त्य - स॒व॒सः॒ । \newline
22. प्र॒स॒वे वा॒चो वा॒चः प्र॑स॒वे प्र॑स॒वे वा॒चो य॒न्त्रं ॅय॒न्त्रं ॅवा॒चः प्र॑स॒वे प्र॑स॒वे वा॒चो य॒न्त्रम् । \newline
23. प्र॒स॒व इति॑ प्र - स॒वे । \newline
24. वा॒चो य॒न्त्रं ॅय॒न्त्रं ॅवा॒चो वा॒चो य॒न्त्र म॑शीयाशीय य॒न्त्रं ॅवा॒चो वा॒चो य॒न्त्र म॑शीय । \newline
25. य॒न्त्र म॑शीयाशीय य॒न्त्रं ॅय॒न्त्र म॑शीय॒ स्वाहा॒ स्वाहा॑ ऽशीय य॒न्त्रं ॅय॒न्त्र म॑शीय॒ स्वाहा᳚ । \newline
26. अ॒शी॒य॒ स्वाहा॒ स्वाहा॑ ऽशीयाशीय॒ स्वाहा॑ शु॒क्रꣳ शु॒क्रꣳ स्वाहा॑ ऽशीयाशीय॒ स्वाहा॑ शु॒क्रम् । \newline
27. स्वाहा॑ शु॒क्रꣳ शु॒क्रꣳ स्वाहा॒ स्वाहा॑ शु॒क्र म॑स्यसि शु॒क्रꣳ स्वाहा॒ स्वाहा॑ शु॒क्र म॑सि । \newline
28. शु॒क्र म॑स्यसि शु॒क्रꣳ शु॒क्र म॑स्य॒मृत॑ म॒मृत॑ मसि शु॒क्रꣳ शु॒क्र म॑स्य॒मृत᳚म् । \newline
29. अ॒स्य॒मृत॑ म॒मृत॑ मस्यस्य॒मृत॑ मस्यस्य॒मृत॑ मस्यस्य॒मृत॑ मसि । \newline
30. अ॒मृत॑ मस्यस्य॒मृत॑ म॒मृत॑ मसि वैश्वदे॒वं ॅवै᳚श्वदे॒व म॑स्य॒मृत॑ म॒मृत॑ मसि वैश्वदे॒वम् । \newline
31. अ॒सि॒ वै॒श्व॒दे॒वं ॅवै᳚श्वदे॒व म॑स्यसि वैश्वदे॒वꣳ ह॒विर्. ह॒विर् वै᳚श्वदे॒व म॑स्यसि वैश्वदे॒वꣳ ह॒विः । \newline
32. वै॒श्व॒दे॒वꣳ ह॒विर्. ह॒विर् वै᳚श्वदे॒वं ॅवै᳚श्वदे॒वꣳ ह॒विः सूर्य॑स्य॒ सूर्य॑स्य ह॒विर् वै᳚श्वदे॒वं ॅवै᳚श्वदे॒वꣳ ह॒विः सूर्य॑स्य । \newline
33. वै॒श्व॒दे॒वमिति॑ वैश्व - दे॒वम् । \newline
34. ह॒विः सूर्य॑स्य॒ सूर्य॑स्य ह॒विर्. ह॒विः सूर्य॑स्य॒ चक्षु॒श्चक्षुः॒ सूर्य॑स्य ह॒विर्. ह॒विः सूर्य॑स्य॒ चक्षुः॑ । \newline
35. सूर्य॑स्य॒ चक्षु॒श्चक्षुः॒ सूर्य॑स्य॒ सूर्य॑स्य॒ चक्षु॒रा चक्षुः॒ सूर्य॑स्य॒ सूर्य॑स्य॒ चक्षु॒रा । \newline
36. चक्षु॒रा चक्षु॒श्चक्षु॒रा ऽरु॑ह मरुह॒ मा चक्षु॒श्चक्षु॒रा ऽरु॑हम् । \newline
37. आ ऽरु॑ह मरुह॒ मा ऽरु॑ह म॒ग्ने र॒ग्ने र॑रुह॒ मा ऽरु॑ह म॒ग्नेः । \newline
38. अ॒रु॒ह॒ म॒ग्ने र॒ग्ने र॑रुह मरुह म॒ग्ने र॒क्ष्णो᳚ (1॒) ऽक्ष्णो᳚ ऽग्नेर॑रुह मरुह म॒ग्ने र॒क्ष्णः । \newline
39. अ॒ग्ने र॒क्ष्णो᳚ (1) ऽक्ष्णो᳚ ऽग्ने र॒ग्ने र॒क्ष्णः क॒नीनि॑काम् क॒नीनि॑का म॒क्ष्णो᳚ ऽग्ने र॒ग्ने र॒क्ष्णः क॒नीनि॑काम् । \newline
40. अ॒क्ष्णः क॒नीनि॑काम् क॒नीनि॑का म॒क्ष्णो᳚ ऽक्ष्णः क॒नीनि॑कां॒ ॅयद् यत् क॒नीनि॑का म॒क्ष्णो᳚ ऽक्ष्णः क॒नीनि॑कां॒ ॅयत् । \newline
41. क॒नीनि॑कां॒ ॅयद् यत् क॒नीनि॑काम् क॒नीनि॑कां॒ ॅयदेत॑शेभि॒ रेत॑शेभि॒र् यत् क॒नीनि॑काम् क॒नीनि॑कां॒ ॅयदेत॑शेभिः । \newline
42. यदेत॑शेभि॒ रेत॑शेभि॒र् यद् यदेत॑शेभि॒ रीय॑स॒ ईय॑स॒ एत॑शेभि॒र् यद् यदेत॑शेभि॒ रीय॑से । \newline
43. एत॑शेभि॒ रीय॑स॒ ईय॑स॒ एत॑शेभि॒ रेत॑शेभि॒ रीय॑से॒ भ्राज॑मानो॒ भ्राज॑मान॒ ईय॑स॒ एत॑शेभि॒ रेत॑शेभि॒ रीय॑से॒ भ्राज॑मानः । \newline
44. ईय॑से॒ भ्राज॑मानो॒ भ्राज॑मान॒ ईय॑स॒ ईय॑से॒ भ्राज॑मानो विप॒श्चिता॑ विप॒श्चिता॒ भ्राज॑मान॒ ईय॑स॒ ईय॑से॒ भ्राज॑मानो विप॒श्चिता᳚ । \newline
45. भ्राज॑मानो विप॒श्चिता॑ विप॒श्चिता॒ भ्राज॑मानो॒ भ्राज॑मानो विप॒श्चिता॒ चिच् चिद् वि॑प॒श्चिता॒ भ्राज॑मानो॒ भ्राज॑मानो विप॒श्चिता॒ चित् । \newline
46. वि॒प॒श्चिता॒ चिच् चिद् वि॑प॒श्चिता॑ विप॒श्चिता॒ चिद॑स्यसि॒ चिद् वि॑प॒श्चिता॑ विप॒श्चिता॒ चिद॑सि । \newline
47. चिद॑स्यसि॒ चिच् चिद॑सि म॒ना म॒ना ऽसि॒ चिच् चिद॑सि म॒ना । \newline
48. अ॒सि॒ म॒ना म॒ना ऽस्य॑सि म॒ना ऽस्य॑सि म॒ना ऽस्य॑सि म॒ना ऽसि॑ । \newline
49. म॒ना ऽस्य॑सि म॒ना म॒ना ऽसि॒ धीर् धीर॑सि म॒ना म॒ना ऽसि॒ धीः । \newline
50. अ॒सि॒ धीर् धीर॑स्यसि॒ धीर॑स्यसि॒ धीर॑स्यसि॒ धीर॑सि । \newline
51. धीर॑स्यसि॒ धीर् धीर॑सि॒ दक्षि॑णा॒ दक्षि॑णा ऽसि॒ धीर् धीर॑सि॒ दक्षि॑णा । \newline
52. अ॒सि॒ दक्षि॑णा॒ दक्षि॑णा ऽस्यसि॒ दक्षि॑णा ऽस्यसि॒ दक्षि॑णा ऽस्यसि॒ दक्षि॑णा ऽसि । \newline
53. दक्षि॑णा ऽस्यसि॒ दक्षि॑णा॒ दक्षि॑णा ऽसि य॒ज्ञिया॑ य॒ज्ञिया॑ ऽसि॒ दक्षि॑णा॒ दक्षि॑णा ऽसि य॒ज्ञिया᳚ । \newline
\pagebreak
\markright{ TS 1.2.4.2  \hfill https://www.vedavms.in \hfill}

\section{ TS 1.2.4.2 }

\textbf{TS 1.2.4.2 } \newline
\textbf{Samhita Paata} \newline

ऽसि य॒ज्ञिया॑ऽसि क्ष॒त्रिया॒ ऽस्यदि॑ति-रस्युभ॒यत॑॑श्शीर्ष्णी॒ सा नः॒ सुप्रा॑ची॒ सुप्र॑तीची॒ सं भ॑व मि॒त्रस्त्वा॑ प॒दि ब॑द्ध्नातु पू॒षाऽद्ध्व॑नः पा॒त्विन्द्रा॒या-द्ध्य॑क्षा॒यानु॑ त्वा मा॒ता म॑न्यता॒मनु॑ पि॒ताऽनु॒ भ्राता॒ सग॒र्भ्योऽनु॒ सखा॒ सयू᳚थ्यः॒ सा दे॑वि दे॒वमच्छे॒हीन्द्रा॑य॒ सोमꣳ॑  रु॒द्रस्त्वा ऽऽ*व॑र्तयतु मि॒त्रस्य॑ प॒था स्व॒स्ति सोम॑सखा॒ ( ) पुन॒रेहि॑ स॒ह र॒य्या ॥ \newline

\textbf{Pada Paata} \newline

अ॒सि॒ । य॒ज्ञिया᳚ । अ॒सि॒ । क्ष॒त्रिया᳚ । अ॒सि॒ । अदि॑तिः । अ॒सि॒ । उ॒भ॒यतः॑ शी॒र्ष्णीत्यु॑भ॒यतः॑ - शी॒र्ष्णी॒ । सा । नः॒ । सुप्रा॒चीति॒ सु -प्रा॒ची॒ । सुप्र॑ती॒चीति॒ सु - प्र॒ती॒ची॒ । समिति॑ । भ॒व॒ । मि॒त्रः । त्वा॒ । प॒दि । ब॒द्ध्ना॒तु॒ । पू॒षा । अद्ध्व॑नः । पा॒तु॒ । इन्द्रा॑य । अद्ध्य॑क्षा॒येत्यधि॑ - अ॒क्षा॒य॒ । अन्विति॑ । त्वा॒ । मा॒ता । म॒न्य॒ता॒म् । अन्विति॑ । पि॒ता । अन्विति॑ । भ्राता᳚ । सग॑र्भ्य॒ इति॒ स - ग॒र्भ्यः॒ । अन्विति॑ । सखा᳚ । सयू॑थ्य॒ इति॒ स - यू॒थ्यः॒ । सा । दे॒वि॒ । दे॒वम् । अच्छ॑ । इ॒हि॒ । इन्द्रा॑य । सोम᳚म् । रु॒द्रः । त्वा॒ । एति॑ । व॒र्त॒य॒तु॒ । मि॒त्रस्य॑ । प॒था । स्व॒स्ति । सोम॑स॒खेति॒ सोम॑ -स॒खा॒ ( ) । पुनः॑ । एति॑ । इ॒हि॒ । स॒ह । र॒य्या ॥  \newline


\textbf{Krama Paata} \newline

अ॒सि॒ य॒ज्ञिया᳚ । य॒ज्ञिया॑ऽसि । अ॒सि॒ क्ष॒त्रिया᳚ । क्ष॒त्रिया॑ऽसि । अ॒स्यदि॑तिः । अदि॑तिरसि । अ॒स्यु॒भ॒यत॑श्शीर्ष्णी । उ॒भ॒यत॑श्शीर्ष्णी॒ सा । उ॒भ॒यत॑श्शी॒र्ष्णीत्यु॑भ॒यतः॑ - शी॒र्ष्णी॒ । सा नः॑ । नः॒ सुप्रा॑ची । सुप्रा॑ची॒ सुप्र॑तीची । सुप्रा॒चीति॒ सु - प्रा॒ची॒ । सुप्र॑तीची॒ सम् । सुप्र॑ती॒चीति॒ सु - प्र॒ती॒ची॒ । सम् भ॑व । भ॒व॒ मि॒त्रः । मि॒त्रस्त्वा᳚ । त्वा॒ प॒दि । प॒दि ब॑द्ध्नातु । ब॒द्ध्ना॒तु॒ पू॒षा । पू॒षा ऽद्ध्व॑नः । अद्ध्व॑नः पातु । पा॒त्विन्द्रा॑य । इन्द्रा॒याद्ध्य॑क्षाय । अद्ध्य॑क्षा॒यानु॑ । अद्ध्य॑क्षा॒येत्यधि॑ - अ॒क्षा॒य॒ । अनु॑ त्वा । त्वा॒ मा॒ता । मा॒ता म॑न्यताम् । म॒न्य॒ता॒मनु॑ । अनु॑ पि॒ता । पि॒ताऽनु॑ । अनु॒ भ्राता᳚ । भ्राता॒ सग॑र्भ्यः । सग॒र्भ्योऽनु॑ । सग॑र्भ्य॒ इति॒ स - ग॒र्भ्यः॒ । अनु॒ सखा᳚ । सखा॒ सयू᳚थ्यः । सयू᳚थ्यः॒ सा । सयू᳚थ्य॒ इति॒ स - यू॒थ्यः॒ । सा दे॑वि । दे॒वि॒ दे॒वम् । दे॒वमच्छ॑ । अच्छे॑हि । इ॒हीन्द्रा॑य । इन्द्रा॑य॒ सोम᳚म् । सोमꣳ॑ रु॒द्रः । रु॒द्रस्त्वा᳚ । त्वा । आ व॑र्तयतु । व॒र्त॒य॒तु॒ मि॒त्रस्य॑ । मि॒त्रस्य॑ प॒था । प॒था स्व॒स्ति । स्व॒स्ति सोम॑सखा ( ) । सोम॑सखा॒ पुनः॑ । सोम॑स॒खेति॒ सोम॑ - स॒खा॒ । पुन॒रा । एहि॑ । इ॒हि॒ स॒ह । स॒ह र॒य्या । र॒य्येति॑ र॒य्या । \newline

\textbf{Jatai Paata} \newline

1. अ॒सि॒ य॒ज्ञिया॑ य॒ज्ञिया᳚ ऽस्यसि य॒ज्ञिया᳚ । \newline
2. य॒ज्ञिया᳚ ऽस्यसि य॒ज्ञिया॑ य॒ज्ञिया॑ ऽसि । \newline
3. अ॒सि॒ क्ष॒त्रिया᳚ क्ष॒त्रिया᳚ ऽस्यसि क्ष॒त्रिया᳚ । \newline
4. क्ष॒त्रिया᳚ ऽस्यसि क्ष॒त्रिया᳚ क्ष॒त्रिया॑ ऽसि । \newline
5. अ॒स्य दि॑ति॒ रदि॑ति रस्य॒स्य दि॑तिः । \newline
6. अदि॑ति रस्य॒स्य दि॑ति॒ रदि॑ति रसि । \newline
7. अ॒स्यु॒भ॒यत॑श्शी॒र्ष्ण् यु॑भ॒यत॑श्शी॒र्ष्ण्य॑ स्य स्युभ॒यत॑श्शी॒र्ष्णी । \newline
8. उ॒भ॒यत॑श्शी॒र्ष्णी सा सोभ॒यत॑श्शी॒र्ष्ण् यु॑भ॒यत॑श्शी॒र्ष्णी सा । \newline
9. उ॒भ॒यत॑श्शी॒र्ष्णीत्यु॑भ॒यतः॑ - शी॒र्ष्णी॒ । \newline
10. सा नो॑ नः॒ सा सा नः॑ । \newline
11. नः॒ सुप्रा॑ची॒ सुप्रा॑ची नो नः॒ सुप्रा॑ची । \newline
12. सुप्रा॑ची॒ सुप्र॑तीची॒ सुप्र॑तीची॒ सुप्रा॑ची॒ सुप्रा॑ची॒ सुप्र॑तीची । \newline
13. सुप्रा॒चीति॒ सु - प्रा॒ची॒ । \newline
14. सुप्र॑तीची॒ सꣳ सꣳ सुप्र॑तीची॒ सुप्र॑तीची॒ सम् । \newline
15. सुप्र॑ती॒चीति॒ सु - प्र॒ती॒ची॒ । \newline
16. सम् भ॑व भव॒ सꣳ सम् भ॑व । \newline
17. भ॒व॒ मि॒त्रो मि॒त्रो भ॑व भव मि॒त्रः । \newline
18. मि॒त्र स्त्वा᳚ त्वा मि॒त्रो मि॒त्र स्त्वा᳚ । \newline
19. त्वा॒ प॒दि प॒दि त्वा᳚ त्वा प॒दि । \newline
20. प॒दि ब॑द्ध्नातु बद्ध्नातु प॒दि प॒दि ब॑द्ध्नातु । \newline
21. ब॒द्ध्ना॒तु॒ पू॒षा पू॒षा ब॑द्ध्नातु बद्ध्नातु पू॒षा । \newline
22. पू॒षा ऽद्ध्व॒नो ऽद्ध्व॑नः पू॒षा पू॒षा ऽद्ध्व॑नः । \newline
23. अद्ध्व॑नः पातु पा॒त्वद्ध्व॒नो ऽद्ध्व॑नः पातु । \newline
24. पा॒त्विन्द्रा॒ये न्द्रा॑य पातु पा॒त्विन्द्रा॑य । \newline
25. इन्द्रा॒ याद्ध्य॑क्षा॒ याद्ध्य॑क्षा॒ये न्द्रा॒ये न्द्रा॒ याद्ध्य॑क्षाय । \newline
26. अद्ध्य॑क्षा॒या न्वन्वद्ध्य॑क्षा॒ याद्ध्य॑क्षा॒या नु॑ । \newline
27. अद्ध्य॑क्षा॒येत्यधि॑ - अ॒क्षा॒य॒ । \newline
28. अनु॑ त्वा॒ त्वा ऽन्वनु॑ त्वा । \newline
29. त्वा॒ मा॒ता मा॒ता त्वा᳚ त्वा मा॒ता । \newline
30. मा॒ता म॑न्यताम् मन्यताम् मा॒ता मा॒ता म॑न्यताम् । \newline
31. म॒न्य॒ता॒ मन्वनु॑ मन्यताम् मन्यता॒ मनु॑ । \newline
32. अनु॑ पि॒ता पि॒ता ऽन्वनु॑ पि॒ता । \newline
33. पि॒ता ऽन्वनु॑ पि॒ता पि॒ता ऽनु॑ । \newline
34. अनु॒ भ्राता॒ भ्राता ऽन्वनु॒ भ्राता᳚ । \newline
35. भ्राता॒ सग॑र्भ्यः॒ सग॑र्भ्यो॒ भ्राता॒ भ्राता॒ सग॑र्भ्यः । \newline
36. सग॒र्भ्यो ऽन्वनु॒ सग॑र्भ्यः॒ सग॒र्भ्यो ऽनु॑ । \newline
37. सग॑र्भ्य॒ इति॒ स - ग॒र्भ्यः॒ । \newline
38. अनु॒ सखा॒ सखा ऽन्वनु॒ सखा᳚ । \newline
39. सखा॒ सयू᳚थ्यः॒ सयू᳚थ्यः॒ सखा॒ सखा॒ सयू᳚थ्यः । \newline
40. सयू᳚थ्यः॒ सा सा सयू᳚थ्यः॒ सयू᳚थ्यः॒ सा । \newline
41. सयू᳚थ्य॒ इति॒ स - यू॒थ्यः॒ । \newline
42. सा दे॑वि देवि॒ सा सा दे॑वि । \newline
43. दे॒वि॒ दे॒वम् दे॒वम् दे॑वि देवि दे॒वम् । \newline
44. दे॒व मच्छाच्छ॑ दे॒वम् दे॒व मच्छ॑ । \newline
45. अच्छे॑ ही॒ह्य च्छाच्छे॑ हि । \newline
46. इ॒ही न्द्रा॒ये न्द्रा॑ये ही॒ही न्द्रा॑य । \newline
47. इन्द्रा॑य॒ सोम॒(ग्म्॒) सोम॒ मिन्द्रा॒ये न्द्रा॑य॒ सोम᳚म् । \newline
48. सोम(ग्म्॑) रु॒द्रो रु॒द्रः सोम॒(ग्म्॒) सोम(ग्म्॑) रु॒द्रः । \newline
49. रु॒द्र स्त्वा᳚ त्वा रु॒द्रो रु॒द्र स्त्वा᳚ । \newline
50. त्वा ऽऽत्वा॒ त्वा । \newline
51. आ व॑र्तयतु वर्तय॒त्वा व॑र्तयतु । \newline
52. व॒र्त॒य॒तु॒ मि॒त्रस्य॑ मि॒त्रस्य॑ वर्तयतु वर्तयतु मि॒त्रस्य॑ । \newline
53. मि॒त्रस्य॑ प॒था प॒था मि॒त्रस्य॑ मि॒त्रस्य॑ प॒था । \newline
54. प॒था स्व॒स्ति स्व॒स्ति प॒था प॒था स्व॒स्ति । \newline
55. स्व॒स्ति सोम॑सखा॒ सोम॑सखा स्व॒स्ति स्व॒स्ति सोम॑सखा । \newline
56. सोम॑सखा॒ पुनः॒ पुनः॒ सोम॑सखा॒ सोम॑सखा॒ पुनः॑ । \newline
57. सोम॑स॒खेति॒ सोम॑ - स॒खा॒ । \newline
58. पुन॒ रा पुनः॒ पुन॒ रा । \newline
59. एही॒ह्येहि॑ । \newline
60. इ॒हि॒ स॒ह स॒हे ही॑हि स॒ह । \newline
61. स॒ह र॒य्या र॒य्या स॒ह स॒ह र॒य्या । \newline
62. र॒य्येति॑ र॒य्या । \newline

\textbf{Ghana Paata } \newline

1. अ॒सि॒ य॒ज्ञिया॑ य॒ज्ञिया᳚ ऽस्यसि य॒ज्ञिया᳚ ऽस्यसि य॒ज्ञिया᳚ ऽस्यसि य॒ज्ञिया॑ ऽसि । \newline
2. य॒ज्ञिया᳚ ऽस्यसि य॒ज्ञिया॑ य॒ज्ञिया॑ ऽसि क्ष॒त्रिया᳚ क्ष॒त्रिया॑ ऽसि य॒ज्ञिया॑ य॒ज्ञिया॑ ऽसि क्ष॒त्रिया᳚ । \newline
3. अ॒सि॒ क्ष॒त्रिया᳚ क्ष॒त्रिया᳚ ऽस्यसि क्ष॒त्रिया᳚ ऽस्यसि क्ष॒त्रिया᳚ ऽस्यसि क्ष॒त्रिया॑ ऽसि । \newline
4. क्ष॒त्रिया᳚ ऽस्यसि क्ष॒त्रिया᳚ क्ष॒त्रिया॒ ऽस्यदि॑ति॒ रदि॑तिरसि क्ष॒त्रिया᳚ क्ष॒त्रिया॒ ऽस्यदि॑तिः । \newline
5. अ॒स्यदि॑ति॒ रदि॑ति रस्य॒स्य दि॑ति रस्य॒स्य दि॑ति रस्य॒स्य दि॑ति रसि । \newline
6. अदि॑ति रस्य॒स्य दि॑ति॒ रदि॑ति रस्युभ॒यत॑श्शी॒र्ष्ण् यु॑भ॒यत॑श्शी॒र्ष्ण्य॑स्य दि॑ति॒ रदि॑ति रस्युभ॒यत॑श्शी॒र्ष्णी । \newline
7. अ॒स्यु॒भ॒यत॑श्शी॒र्ष्ण् यु॑भ॒यत॑श्शी॒र्ष्ण् य॑स्य स्युभ॒यत॑श्शी॒र्ष्णी सा सोभ॒यत॑श्शी॒र्ष्ण् य॑स्यस्युभ॒यत॑श्शी॒र्ष्णी सा । \newline
8. उ॒भ॒यत॑श्शी॒र्ष्णी सा सोभ॒यत॑श्शी॒र्ष्ण् यु॑भ॒यत॑श्शी॒र्ष्णी सा नो॑ नः॒ सोभ॒यत॑श्शी॒र्ष्ण् यु॑भ॒यत॑श्शी॒र्ष्णी सा नः॑ । \newline
9. उ॒भ॒यत॑श्शी॒र्ष्णीत्यु॑भ॒यतः॑ - शी॒र्ष्णी॒ । \newline
10. सा नो॑ नः॒ सा सा नः॒ सुप्रा॑ची॒ सुप्रा॑ची नः॒ सा सा नः॒ सुप्रा॑ची । \newline
11. नः॒ सुप्रा॑ची॒ सुप्रा॑ची नो नः॒ सुप्रा॑ची॒ सुप्र॑तीची॒ सुप्र॑तीची॒ सुप्रा॑ची नो नः॒ सुप्रा॑ची॒ सुप्र॑तीची । \newline
12. सुप्रा॑ची॒ सुप्र॑तीची॒ सुप्र॑तीची॒ सुप्रा॑ची॒ सुप्रा॑ची॒ सुप्र॑तीची॒ सꣳ सꣳ सुप्र॑तीची॒ सुप्रा॑ची॒ सुप्रा॑ची॒ सुप्र॑तीची॒ सम् । \newline
13. सुप्रा॒चीति॒ सु - प्रा॒ची॒ । \newline
14. सुप्र॑तीची॒ सꣳ सꣳ सुप्र॑तीची॒ सुप्र॑तीची॒ सम् भ॑व भव॒ सꣳ सुप्र॑तीची॒ सुप्र॑तीची॒ सम् भ॑व । \newline
15. सुप्र॑ती॒चीति॒ सु - प्र॒ती॒ची॒ । \newline
16. सम् भ॑व भव॒ सꣳ सम् भ॑व मि॒त्रो मि॒त्रो भ॑व॒ सꣳ सम् भ॑व मि॒त्रः । \newline
17. भ॒व॒ मि॒त्रो मि॒त्रो भ॑व भव मि॒त्रस्त्वा᳚ त्वा मि॒त्रो भ॑व भव मि॒त्रस्त्वा᳚ । \newline
18. मि॒त्रस्त्वा᳚ त्वा मि॒त्रो मि॒त्रस्त्वा॑ प॒दि प॒दि त्वा॑ मि॒त्रो मि॒त्रस्त्वा॑ प॒दि । \newline
19. त्वा॒ प॒दि प॒दि त्वा᳚ त्वा प॒दि ब॑द्ध्नातु बद्ध्नातु प॒दि त्वा᳚ त्वा प॒दि ब॑द्ध्नातु । \newline
20. प॒दि ब॑द्ध्नातु बद्ध्नातु प॒दि प॒दि ब॑द्ध्नातु पू॒षा पू॒षा ब॑द्ध्नातु प॒दि प॒दि ब॑द्ध्नातु पू॒षा । \newline
21. ब॒द्ध्ना॒तु॒ पू॒षा पू॒षा ब॑द्ध्नातु बद्ध्नातु पू॒षा ऽद्ध्व॒नो ऽद्ध्व॑नः पू॒षा ब॑द्ध्नातु बद्ध्नातु पू॒षा ऽद्ध्व॑नः । \newline
22. पू॒षा ऽद्ध्व॒नो ऽद्ध्व॑नः पू॒षा पू॒षा ऽद्ध्व॑नः पातु पा॒त्वद्ध्व॑नः पू॒षा पू॒षा ऽद्ध्व॑नः पातु । \newline
23. अद्ध्व॑नः पातु पा॒त्वद्ध्व॒नो ऽद्ध्व॑नः पा॒त्विन्द्रा॒ये न्द्रा॑य पा॒त्वद्ध्व॒नो ऽद्ध्व॑नः पा॒त्विन्द्रा॑य । \newline
24. पा॒त्विन्द्रा॒ये न्द्रा॑य पातु पा॒त्विन्द्रा॒या द्ध्य॑क्षा॒याद्ध्य॑क्षा॒ये न्द्रा॑य पातु पा॒त्विन्द्रा॒या द्ध्य॑क्षाय । \newline
25. इन्द्रा॒याद्ध्य॑क्षा॒या द्ध्य॑क्षा॒ये न्द्रा॒ये न्द्रा॒या द्ध्य॑क्षा॒यान्वन्वद्ध्य॑क्षा॒ये न्द्रा॒ये न्द्रा॒या द्ध्य॑क्षा॒यानु॑ । \newline
26. अद्ध्य॑क्षा॒ यान्वन्वद्ध्य॑क्षा॒ याद्ध्य॑क्षा॒यानु॑ त्वा॒ त्वा ऽन्वद्ध्य॑क्षा॒ याद्ध्य॑क्षा॒यानु॑ त्वा । \newline
27. अद्ध्य॑क्षा॒येत्यधि॑ - अ॒क्षा॒य॒ । \newline
28. अनु॑ त्वा॒ त्वा ऽन्वनु॑ त्वा मा॒ता मा॒ता त्वा ऽन्वनु॑ त्वा मा॒ता । \newline
29. त्वा॒ मा॒ता मा॒ता त्वा᳚ त्वा मा॒ता म॑न्यताम् मन्यताम् मा॒ता त्वा᳚ त्वा मा॒ता म॑न्यताम् । \newline
30. मा॒ता म॑न्यताम् मन्यताम् मा॒ता मा॒ता म॑न्यता॒ मन्वनु॑ मन्यताम् मा॒ता मा॒ता म॑न्यता॒ मनु॑ । \newline
31. म॒न्य॒ता॒ मन्वनु॑ मन्यताम् मन्यता॒ मनु॑ पि॒ता पि॒ता ऽनु॑ मन्यताम् मन्यता॒ मनु॑ पि॒ता । \newline
32. अनु॑ पि॒ता पि॒ता ऽन्वनु॑ पि॒ता ऽन्वनु॑ पि॒ता ऽन्वनु॑ पि॒ता ऽनु॑ । \newline
33. पि॒ता ऽन्वनु॑ पि॒ता पि॒ता ऽनु॒ भ्राता॒ भ्राता ऽनु॑ पि॒ता पि॒ता ऽनु॒ भ्राता᳚ । \newline
34. अनु॒ भ्राता॒ भ्राता ऽन्वनु॒ भ्राता॒ सग॑र्भ्यः॒ सग॑र्भ्यो॒ भ्राता ऽन्वनु॒ भ्राता॒ सग॑र्भ्यः । \newline
35. भ्राता॒ सग॑र्भ्यः॒ सग॑र्भ्यो॒ भ्राता॒ भ्राता॒ सग॒र्भ्यो ऽन्वनु॒ सग॑र्भ्यो॒ भ्राता॒ भ्राता॒ सग॒र्भ्यो ऽनु॑ । \newline
36. सग॒र्भ्यो ऽन्वनु॒ सग॑र्भ्यः॒ सग॒र्भ्यो ऽनु॒ सखा॒ सखा ऽनु॒ सग॑र्भ्यः॒ सग॒र्भ्यो ऽनु॒ सखा᳚ । \newline
37. सग॑र्भ्य॒ इति॒ स - ग॒र्भ्यः॒ । \newline
38. अनु॒ सखा॒ सखा ऽन्वनु॒ सखा॒ सयू᳚थ्यः॒ सयू᳚थ्यः॒ सखा ऽन्वनु॒ सखा॒ सयू᳚थ्यः । \newline
39. सखा॒ सयू᳚थ्यः॒ सयू᳚थ्यः॒ सखा॒ सखा॒ सयू᳚थ्यः॒ सा सा सयू᳚थ्यः॒ सखा॒ सखा॒ सयू᳚थ्यः॒ सा । \newline
40. सयू᳚थ्यः॒ सा सा सयू᳚थ्यः॒ सयू᳚थ्यः॒ सा दे॑वि देवि॒ सा सयू᳚थ्यः॒ सयू᳚थ्यः॒ सा दे॑वि । \newline
41. सयू᳚थ्य॒ इति॒ स - यू॒थ्यः॒ । \newline
42. सा दे॑वि देवि॒ सा सा दे॑वि दे॒वम् दे॒वम् दे॑वि॒ सा सा दे॑वि दे॒वम् । \newline
43. दे॒वि॒ दे॒वम् दे॒वम् दे॑वि देवि दे॒व मच्छाच्छ॑ दे॒वम् दे॑वि देवि दे॒व मच्छ॑ । \newline
44. दे॒व मच्छाच्छ॑ दे॒वम् दे॒व मच्छे॑ ही॒ह्यच्छ॑ दे॒वम् दे॒व मच्छे॑ हि । \newline
45. अच्छे॑ ही॒ह्यच्छाच्छे॒ हीन्द्रा॒ये न्द्रा॑ये॒ ह्यच्छाच्छे॒ हीन्द्रा॑य । \newline
46. इ॒हीन्द्रा॒ये न्द्रा॑ये ही॒हीन्द्रा॑य॒ सोम॒(ग्म्॒) सोम॒ मिन्द्रा॑ये ही॒हीन्द्रा॑य॒ सोम᳚म् । \newline
47. इन्द्रा॑य॒ सोम॒(ग्म्॒) सोम॒ मिन्द्रा॒ये न्द्रा॑य॒ सोम(ग्म्॑) रु॒द्रो रु॒द्रः सोम॒ मिन्द्रा॒ये न्द्रा॑य॒ सोम(ग्म्॑) रु॒द्रः । \newline
48. सोम(ग्म्॑) रु॒द्रो रु॒द्रः सोम॒(ग्म्॒) सोम(ग्म्॑) रु॒द्रस्त्वा᳚ त्वा रु॒द्रः सोम॒(ग्म्॒) सोम(ग्म्॑) रु॒द्रस्त्वा᳚ । \newline
49. रु॒द्रस्त्वा᳚ त्वा रु॒द्रो रु॒द्रस्त्वा ऽऽत्वा॑ रु॒द्रो रु॒द्रस्त्वा । \newline
50. त्वा ऽऽत्वा॒ त्वा ऽऽव॑र्तयतु वर्तय॒त्वा त्वा॒ त्वा ऽऽव॑र्तयतु । \newline
51. आ व॑र्तयतु वर्तय॒त्वा व॑र्तयतु मि॒त्रस्य॑ मि॒त्रस्य॑ वर्तय॒त्वा व॑र्तयतु मि॒त्रस्य॑ । \newline
52. व॒र्त॒य॒तु॒ मि॒त्रस्य॑ मि॒त्रस्य॑ वर्तयतु वर्तयतु मि॒त्रस्य॑ प॒था प॒था मि॒त्रस्य॑ वर्तयतु वर्तयतु मि॒त्रस्य॑ प॒था । \newline
53. मि॒त्रस्य॑ प॒था प॒था मि॒त्रस्य॑ मि॒त्रस्य॑ प॒था स्व॒स्ति स्व॒स्ति प॒था मि॒त्रस्य॑ मि॒त्रस्य॑ प॒था स्व॒स्ति । \newline
54. प॒था स्व॒स्ति स्व॒स्ति प॒था प॒था स्व॒स्ति सोम॑सखा॒ सोम॑सखा स्व॒स्ति प॒था प॒था स्व॒स्ति सोम॑सखा । \newline
55. स्व॒स्ति सोम॑सखा॒ सोम॑सखा स्व॒स्ति स्व॒स्ति सोम॑सखा॒ पुनः॒ पुनः॒ सोम॑सखा स्व॒स्ति स्व॒स्ति सोम॑सखा॒ पुनः॑ । \newline
56. सोम॑सखा॒ पुनः॒ पुनः॒ सोम॑सखा॒ सोम॑सखा॒ पुन॒ रा पुनः॒ सोम॑सखा॒ सोम॑सखा॒ पुन॒ रा । \newline
57. सोम॑स॒खेति॒ सोम॑ - स॒खा॒ । \newline
58. पुन॒ रा पुनः॒ पुन॒ रेही॒ह्या पुनः॒ पुन॒ रेहि॑ । \newline
59. एही॒ह्येहि॑ स॒ह स॒हे ह्येहि॑ स॒ह । \newline
60. इ॒हि॒ स॒ह स॒हे ही॑हि स॒ह र॒य्या र॒य्या स॒हे ही॑हि स॒ह र॒य्या । \newline
61. स॒ह र॒य्या र॒य्या स॒ह स॒ह र॒य्या । \newline
62. र॒य्येति॑ र॒य्या । \newline
\pagebreak
\markright{ TS 1.2.5.1  \hfill https://www.vedavms.in \hfill}

\section{ TS 1.2.5.1 }

\textbf{TS 1.2.5.1 } \newline
\textbf{Samhita Paata} \newline

वस्व्य॑सि रु॒द्राऽस्यदि॑ति-रस्यादि॒त्याऽसि॑ शु॒क्राऽसि॑ च॒न्द्राऽसि॒ बृह॒स्पति॑स्त्वा सु॒म्ने र॑ण्वतु रु॒द्रो वसु॑भि॒रा चि॑केतु पृथि॒व्यास्त्वा॑ मू॒र्द्धन्ना जि॑घर्मि देव॒यज॑न॒ इडा॑याः प॒दे घृ॒तव॑ति॒ स्वाहा॒ परि॑लिखितꣳ॒॒ रक्षः॒ परि॑लिखिता॒ अरा॑तय इ॒दम॒हꣳ रक्ष॑सो ग्री॒वा अपि॑ कृन्तामि॒ यो᳚ऽस्मान् द्वेष्टि॒ यं च॑ व॒यं द्वि॒ष्म इ॒दम॑स्य ग्री॒वा - [ ] \newline

\textbf{Pada Paata} \newline

वस्वी᳚ । अ॒सि॒ । रु॒द्रा । अ॒सि॒ । अदि॑तिः । अ॒सि॒ । आ॒दि॒त्या । अ॒सि॒ । शु॒क्रा । अ॒सि॒ । च॒न्द्रा । अ॒सि॒ । बृह॒स्पतिः॑ । त्वा॒ । सु॒म्ने । र॒ण्व॒तु॒ । रु॒द्रः । वसु॑भि॒रिति॒ वसु॑ -भिः॒ । एति॑ । चि॒के॒तु॒ । पृ॒थि॒व्याः । त्वा॒ । मू॒र्द्धन्न् । एति॑ । जि॒घ॒र्मि॒ । दे॒व॒यज॑न॒ इति॑ देव - यज॑ने । इडा॑याः । प॒दे । घृ॒तव॒तीति॑ घृ॒त - व॒ति॒ । स्वाहा᳚ । परि॑लिखित॒मिति॒ परि॑ - लि॒खि॒त॒म् । रक्षः॑ । परि॑लिखिता॒ इति॒ परि॑ - लि॒खि॒ताः॒ । अरा॑तयः । इ॒दम् । अ॒हम् । रक्ष॑सः । ग्री॒वाः । अपीति॑ । कृ॒न्ता॒मि॒ । यः । अ॒स्मान् । द्वेष्टि॑ । यम् । च॒ । व॒यम् । द्वि॒ष्मः । इ॒दम् । अ॒स्य॒ । ग्री॒वाः ।  \newline


\textbf{Krama Paata} \newline

वस्व्य॑सि । अ॒सि॒ रु॒द्रा । रु॒द्राऽसि॑ । अ॒स्यदि॑तिः । अदि॑तिरसि । अ॒स्या॒दि॒त्या । आ॒दि॒त्याऽसि॑ । अ॒सि॒ शु॒क्रा । शु॒क्राऽसि॑ । अ॒सि॒ च॒न्द्रा । च॒न्द्राऽसि॑ । अ॒सि॒ बृह॒स्पतिः॑ । बृह॒स्पति॑स्त्वा । त्वा॒ सु॒म्ने । सु॒म्ने र॑ण्वतु । र॒ण्व॒तु॒ रु॒द्रः । रु॒द्रो वसु॑भिः । वसु॑भि॒रा । वसु॑भि॒रिति॒ वसु॑ - भिः॒ । आ चि॑केतु । चि॒के॒तु॒ पृ॒थि॒व्याः । पृ॒थि॒व्यास्त्वा᳚ । त्वा॒ मू॒र्द्धन्न् । मू॒र्द्धन्ना । आ जि॑घर्मि । जि॒घ॒र्मि॒ दे॒व॒यज॑ने । दे॒व॒यज॑न॒ इडा॑याः । दे॒व॒यज॑न॒ इति॑ देव - यज॑ने । इडा॑याः प॒दे । प॒दे घृ॒तव॑ति । घृ॒तव॑ति॒ स्वाहा᳚ । घृ॒तव॒तीति॑ घृ॒त - व॒ति॒ । स्वाहा॒ परि॑लिखितम् । परि॑लिखितꣳ॒॒ रक्षः॑ । परि॑लिखित॒मिति॒ परि॑ - लि॒खि॒त॒म् । रक्षः॒ परि॑लिखिताः । परि॑लिखिता॒ अरा॑तयः । परि॑लिखिता॒ इति॒ परि॑ - लि॒खि॒ताः॒ । अरा॑तय इ॒दम् । इ॒दम॒हम् । अ॒हꣳ रक्ष॑सः । रक्ष॑सो ग्री॒वाः । ग्री॒वा अपि॑ । अपि॑ कृन्तामि । कृ॒न्ता॒मि॒ यः । यो᳚ऽस्मान् । अ॒स्मान् द्वेष्टि॑ । द्वेष्टि॒ यम् । यम् च॑ । च॒ व॒यम् । व॒यम् द्वि॒ष्मः । द्वि॒ष्म इ॒दम् । इ॒दम॑स्य । अ॒स्य॒ ग्री॒वाः ( ) । ग्री॒वा अपि॑ \newline

\textbf{Jatai Paata} \newline

1. वस्व्य॑स्यसि॒ वस्वी॒ वस्व्य॑सि । \newline
2. अ॒सि॒ रु॒द्रा रु॒द्रा ऽस्य॑सि रु॒द्रा । \newline
3. रु॒द्रा ऽस्य॑सि रु॒द्रा रु॒द्रा ऽसि॑ । \newline
4. अ॒स्य दि॑ति॒ रदि॑ति रस्य॒ स्यदि॑तिः । \newline
5. अदि॑ति रस्य॒स्य दि॑ति॒ रदि॑ति रसि । \newline
6. अ॒स्या॒दि॒त्या ऽऽदि॒त्या ऽस्य॑स्यादि॒त्या । \newline
7. आ॒दि॒त्या ऽस्य॑स्यादि॒त्या ऽऽदि॒त्या ऽसि॑ । \newline
8. अ॒सि॒ शु॒क्रा शु॒क्रा ऽस्य॑सि शु॒क्रा । \newline
9. शु॒क्रा ऽस्य॑सि शु॒क्रा शु॒क्रा ऽसि॑ । \newline
10. अ॒सि॒ च॒न्द्रा च॒न्द्रा ऽस्य॑सि च॒न्द्रा । \newline
11. च॒न्द्रा ऽस्य॑सि च॒न्द्रा च॒न्द्रा ऽसि॑ । \newline
12. अ॒सि॒ बृह॒स्पति॒र् बृह॒स्पति॑ रस्यसि॒ बृह॒स्पतिः॑ । \newline
13. बृह॒स्पति॑ स्त्वा त्वा॒ बृह॒स्पति॒र् बृह॒स्पति॑ स्त्वा । \newline
14. त्वा॒ सु॒म्ने सु॒म्ने त्वा᳚ त्वा सु॒म्ने । \newline
15. सु॒म्ने र॑ण्वतु रण्वतु सु॒म्ने सु॒म्ने र॑ण्वतु । \newline
16. र॒ण्व॒तु॒ रु॒द्रो रु॒द्रो र॑ण्वतु रण्वतु रु॒द्रः । \newline
17. रु॒द्रो वसु॑भि॒र् वसु॑भी रु॒द्रो रु॒द्रो वसु॑भिः । \newline
18. वसु॑भि॒रा वसु॑भि॒र् वसु॑भि॒रा । \newline
19. वसु॑भि॒रिति॒ वसु॑ - भिः॒ । \newline
20. आ चि॑केतु चिके॒त्वा चि॑केतु । \newline
21. चि॒के॒तु॒ पृ॒थि॒व्याः पृ॑थि॒व्या श्चि॑केतु चिकेतु पृथि॒व्याः । \newline
22. पृ॒थि॒व्या स्त्वा᳚ त्वा पृथि॒व्याः पृ॑थि॒व्या स्त्वा᳚ । \newline
23. त्वा॒ मू॒र्द्धन् मू॒र्द्धन् त्वा᳚ त्वा मू॒र्द्धन्न् । \newline
24. मू॒र्द्धन् ना मू॒र्द्धन् मू॒र्द्धन् ना । \newline
25. आ जि॑घर्मि जिघ॒र्म्या जि॑घर्मि । \newline
26. जि॒घ॒र्मि॒ दे॒व॒यज॑ने देव॒यज॑ने जिघर्मि जिघर्मि देव॒यज॑ने । \newline
27. दे॒व॒यज॑न॒ इडा॑या॒ इडा॑या देव॒यज॑ने देव॒यज॑न॒ इडा॑याः । \newline
28. दे॒व॒यज॑न॒ इति॑ देव - यज॑ने । \newline
29. इडा॑याः प॒दे प॒द इडा॑या॒ इडा॑याः प॒दे । \newline
30. प॒दे घृ॒तव॑ति घृ॒तव॑ति प॒दे प॒दे घृ॒तव॑ति । \newline
31. घृ॒तव॑ति॒ स्वाहा॒ स्वाहा॑ घृ॒तव॑ति घृ॒तव॑ति॒ स्वाहा᳚ । \newline
32. घृ॒तव॒तीति॑ घृ॒त - व॒ति॒ । \newline
33. स्वाहा॒ परि॑लिखित॒म् परि॑लिखित॒(ग्ग्॒) स्वाहा॒ स्वाहा॒ परि॑लिखितम् । \newline
34. परि॑लिखित॒(ग्म्॒) रक्षो॒ रक्षः॒ परि॑लिखित॒म् परि॑लिखित॒(ग्म्॒) रक्षः॑ । \newline
35. परि॑लिखित॒मिति॒ परि॑ - लि॒खि॒त॒म् । \newline
36. रक्षः॒ परि॑लिखिताः॒ परि॑लिखिता॒ रक्षो॒ रक्षः॒ परि॑लिखिताः । \newline
37. परि॑लिखिता॒ अरा॑त॒यो ऽरा॑तयः॒ परि॑लिखिताः॒ परि॑लिखिता॒ अरा॑तयः । \newline
38. परि॑लिखिता॒ इति॒ परि॑ - लि॒खि॒ताः॒ । \newline
39. अरा॑तय इ॒द मि॒द मरा॑त॒यो ऽरा॑तय इ॒दम् । \newline
40. इ॒द म॒ह म॒ह मि॒द मि॒द म॒हम् । \newline
41. अ॒हꣳ रक्ष॑सो॒ रक्ष॑सो॒ ऽह म॒हꣳ रक्ष॑सः । \newline
42. रक्ष॑सो ग्री॒वा ग्री॒वा रक्ष॑सो॒ रक्ष॑सो ग्री॒वाः । \newline
43. ग्री॒वा अप्यपि॑ ग्री॒वा ग्री॒वा अपि॑ । \newline
44. अपि॑ कृन्तामि कृन्ता॒म्यप्यपि॑ कृन्तामि । \newline
45. कृ॒न्ता॒मि॒ यो यः कृ॑न्तामि कृन्तामि॒ यः । \newline
46. यो᳚ ऽस्मा न॒स्मान्. यो यो᳚ ऽस्मान् । \newline
47. अ॒स्मान् द्वेष्टि॒ द्वेष्ट्य॒स्मा न॒स्मान् द्वेष्टि॑ । \newline
48. द्वेष्टि॒ यं ॅयम् द्वेष्टि॒ द्वेष्टि॒ यम् । \newline
49. यम् च॑ च॒ यं ॅयम् च॑ । \newline
50. च॒ व॒यं ॅव॒यम् च॑ च व॒यम् । \newline
51. व॒यम् द्वि॒ष्मो द्वि॒ष्मो व॒यं ॅव॒यम् द्वि॒ष्मः । \newline
52. द्वि॒ष्म इ॒द मि॒दम् द्वि॒ष्मो द्वि॒ष्म इ॒दम् । \newline
53. इ॒द म॑स्यास्ये॒ द मि॒द म॑स्य । \newline
54. अ॒स्य॒ ग्री॒वा ग्री॒वा अ॑स्यास्य ग्री॒वाः । \newline
55. ग्री॒वा अप्यपि॑ ग्री॒वा ग्री॒वा अपि॑ । \newline

\textbf{Ghana Paata } \newline

1. वस्व्य॑स्यसि॒ वस्वी॒ वस्व्य॑सि रु॒द्रा रु॒द्रा ऽसि॒ वस्वी॒ वस्व्य॑सि रु॒द्रा । \newline
2. अ॒सि॒ रु॒द्रा रु॒द्रा ऽस्य॑सि रु॒द्रा ऽस्य॑सि रु॒द्रा ऽस्य॑सि रु॒द्रा ऽसि॑ । \newline
3. रु॒द्रा ऽस्य॑सि रु॒द्रा रु॒द्रा ऽस्यदि॑ति॒ रदि॑तिरसि रु॒द्रा रु॒द्रा ऽस्यदि॑तिः । \newline
4. अ॒स्य दि॑ति॒ रदि॑ति रस्य॒स्य दि॑ति रस्य॒स्य दि॑ति रस्य॒स्य दि॑ति रसि । \newline
5. अदि॑ति रस्य॒स्य दि॑ति॒ रदि॑तिरस्यादि॒त्या ऽऽदि॒त्या ऽस्यदि॑ति॒ रदि॑ति रस्यादि॒त्या । \newline
6. अ॒स्या॒दि॒त्या ऽऽदि॒त्या ऽस्य॑स्यादि॒त्या ऽस्य॑स्यादि॒त्या ऽस्य॑स्यादि॒त्या ऽसि॑ । \newline
7. आ॒दि॒त्या ऽस्य॑स्यादि॒त्या ऽऽदि॒त्या ऽसि॑ शु॒क्रा शु॒क्रा ऽस्या॑दि॒त्या ऽऽदि॒त्या ऽसि॑ शु॒क्रा । \newline
8. अ॒सि॒ शु॒क्रा शु॒क्रा ऽस्य॑सि शु॒क्रा ऽस्य॑सि शु॒क्रा ऽस्य॑सि शु॒क्रा ऽसि॑ । \newline
9. शु॒क्रा ऽस्य॑सि शु॒क्रा शु॒क्रा ऽसि॑ च॒न्द्रा च॒न्द्रा ऽसि॑ शु॒क्रा शु॒क्रा ऽसि॑ च॒न्द्रा । \newline
10. अ॒सि॒ च॒न्द्रा च॒न्द्रा ऽस्य॑सि च॒न्द्रा ऽस्य॑सि च॒न्द्रा ऽस्य॑सि च॒न्द्रा ऽसि॑ । \newline
11. च॒न्द्रा ऽस्य॑सि च॒न्द्रा च॒न्द्रा ऽसि॒ बृह॒स्पति॒र् बृह॒स्पति॑रसि च॒न्द्रा च॒न्द्रा ऽसि॒ बृह॒स्पतिः॑ । \newline
12. अ॒सि॒ बृह॒स्पति॒र् बृह॒स्पति॑रस्यसि॒ बृह॒स्पति॑स्त्वा त्वा॒ बृह॒स्पति॑रस्यसि॒ बृह॒स्पति॑स्त्वा । \newline
13. बृह॒स्पति॑स्त्वा त्वा॒ बृह॒स्पति॒र् बृह॒स्पति॑स्त्वा सु॒म्ने सु॒म्ने त्वा॒ बृह॒स्पति॒र् बृह॒स्पति॑स्त्वा सु॒म्ने । \newline
14. त्वा॒ सु॒म्ने सु॒म्ने त्वा᳚ त्वा सु॒म्ने र॑ण्वतु रण्वतु सु॒म्ने त्वा᳚ त्वा सु॒म्ने र॑ण्वतु । \newline
15. सु॒म्ने र॑ण्वतु रण्वतु सु॒म्ने सु॒म्ने र॑ण्वतु रु॒द्रो रु॒द्रो र॑ण्वतु सु॒म्ने सु॒म्ने र॑ण्वतु रु॒द्रः । \newline
16. र॒ण्व॒तु॒ रु॒द्रो रु॒द्रो र॑ण्वतु रण्वतु रु॒द्रो वसु॑भि॒र् वसु॑भी रु॒द्रो र॑ण्वतु रण्वतु रु॒द्रो वसु॑भिः । \newline
17. रु॒द्रो वसु॑भि॒र् वसु॑भी रु॒द्रो रु॒द्रो वसु॑भि॒रा वसु॑भी रु॒द्रो रु॒द्रो वसु॑भि॒रा । \newline
18. वसु॑भि॒रा वसु॑भि॒र् वसु॑भि॒रा चि॑केतु चिके॒त्वा वसु॑भि॒र् वसु॑भि॒रा चि॑केतु । \newline
19. वसु॑भि॒रिति॒ वसु॑ - भिः॒ । \newline
20. आ चि॑केतु चिके॒त्वा चि॑केतु पृथि॒व्याः पृ॑थि॒व्याश्चि॑के॒त्वा चि॑केतु पृथि॒व्याः । \newline
21. चि॒के॒तु॒ पृ॒थि॒व्याः पृ॑थि॒व्याश्चि॑केतु चिकेतु पृथि॒व्यास्त्वा᳚ त्वा पृथि॒व्याश्चि॑केतु चिकेतु पृथि॒व्यास्त्वा᳚ । \newline
22. पृ॒थि॒व्यास्त्वा᳚ त्वा पृथि॒व्याः पृ॑थि॒व्यास्त्वा॑ मू॒र्द्धन् मू॒र्द्धन् त्वा॑ पृथि॒व्याः पृ॑थि॒व्यास्त्वा॑ मू॒र्द्धन्न् । \newline
23. त्वा॒ मू॒र्द्धन् मू॒र्द्धन् त्वा᳚ त्वा मू॒र्द्धन् ना मू॒र्द्धन् त्वा᳚ त्वा मू॒र्द्धन् ना । \newline
24. मू॒र्द्धन् ना मू॒र्द्धन् मू॒र्द्धन् ना जि॑घर्मि जिघ॒र्म्या मू॒र्द्धन् मू॒र्द्धन् ना जि॑घर्मि । \newline
25. आ जि॑घर्मि जिघ॒र्म्या जि॑घर्मि देव॒यज॑ने देव॒यज॑ने जिघ॒र्म्या जि॑घर्मि देव॒यज॑ने । \newline
26. जि॒घ॒र्मि॒ दे॒व॒यज॑ने देव॒यज॑ने जिघर्मि जिघर्मि देव॒यज॑न॒ इडा॑या॒ इडा॑या देव॒यज॑ने जिघर्मि जिघर्मि देव॒यज॑न॒ इडा॑याः । \newline
27. दे॒व॒यज॑न॒ इडा॑या॒ इडा॑या देव॒यज॑ने देव॒यज॑न॒ इडा॑याः प॒दे प॒द इडा॑या देव॒यज॑ने देव॒यज॑न॒ इडा॑याः प॒दे । \newline
28. दे॒व॒यज॑न॒ इति॑ देव - यज॑ने । \newline
29. इडा॑याः प॒दे प॒द इडा॑या॒ इडा॑याः प॒दे घृ॒तव॑ति घृ॒तव॑ति प॒द इडा॑या॒ इडा॑याः प॒दे घृ॒तव॑ति । \newline
30. प॒दे घृ॒तव॑ति घृ॒तव॑ति प॒दे प॒दे घृ॒तव॑ति॒ स्वाहा॒ स्वाहा॑ घृ॒तव॑ति प॒दे प॒दे घृ॒तव॑ति॒ स्वाहा᳚ । \newline
31. घृ॒तव॑ति॒ स्वाहा॒ स्वाहा॑ घृ॒तव॑ति घृ॒तव॑ति॒ स्वाहा॒ परि॑लिखित॒म् परि॑लिखित॒(ग्ग्॒) स्वाहा॑ घृ॒तव॑ति घृ॒तव॑ति॒ स्वाहा॒ परि॑लिखितम् । \newline
32. घृ॒तव॒तीति॑ घृ॒त - व॒ति॒ । \newline
33. स्वाहा॒ परि॑लिखित॒म् परि॑लिखित॒(ग्ग्॒) स्वाहा॒ स्वाहा॒ परि॑लिखित॒(ग्म्॒) रक्षो॒ रक्षः॒ परि॑लिखित॒(ग्ग्॒) स्वाहा॒ स्वाहा॒ परि॑लिखित॒(ग्म्॒) रक्षः॑ । \newline
34. परि॑लिखित॒(ग्म्॒) रक्षो॒ रक्षः॒ परि॑लिखित॒म् परि॑लिखित॒(ग्म्॒) रक्षः॒ परि॑लिखिताः॒ परि॑लिखिता॒ रक्षः॒ परि॑लिखित॒म् परि॑लिखित॒(ग्म्॒) रक्षः॒ परि॑लिखिताः । \newline
35. परि॑लिखित॒मिति॒ परि॑ - लि॒खि॒त॒म् । \newline
36. रक्षः॒ परि॑लिखिताः॒ परि॑लिखिता॒ रक्षो॒ रक्षः॒ परि॑लिखिता॒ अरा॑त॒यो ऽरा॑तयः॒ परि॑लिखिता॒ रक्षो॒ रक्षः॒ परि॑लिखिता॒ अरा॑तयः । \newline
37. परि॑लिखिता॒ अरा॑त॒यो ऽरा॑तयः॒ परि॑लिखिताः॒ परि॑लिखिता॒ अरा॑तय इ॒द मि॒द मरा॑तयः॒ परि॑लिखिताः॒ परि॑लिखिता॒ अरा॑तय इ॒दम् । \newline
38. परि॑लिखिता॒ इति॒ परि॑ - लि॒खि॒ताः॒ । \newline
39. अरा॑तय इ॒द मि॒द मरा॑त॒यो ऽरा॑तय इ॒द म॒ह म॒ह मि॒द मरा॑त॒यो ऽरा॑तय इ॒द म॒हम् । \newline
40. इ॒द म॒ह म॒ह मि॒द मि॒द म॒हꣳ रक्ष॑सो॒ रक्ष॑सो॒ ऽह मि॒द मि॒द म॒हꣳ रक्ष॑सः । \newline
41. अ॒हꣳ रक्ष॑सो॒ रक्ष॑सो॒ ऽह म॒हꣳ रक्ष॑सो ग्री॒वा ग्री॒वा रक्ष॑सो॒ ऽह म॒हꣳ रक्ष॑सो ग्री॒वाः । \newline
42. रक्ष॑सो ग्री॒वा ग्री॒वा रक्ष॑सो॒ रक्ष॑सो ग्री॒वा अप्यपि॑ ग्री॒वा रक्ष॑सो॒ रक्ष॑सो ग्री॒वा अपि॑ । \newline
43. ग्री॒वा अप्यपि॑ ग्री॒वा ग्री॒वा अपि॑ कृन्तामि कृन्ता॒म्यपि॑ ग्री॒वा ग्री॒वा अपि॑ कृन्तामि । \newline
44. अपि॑ कृन्तामि कृन्ता॒म्यप्यपि॑ कृन्तामि॒ यो यः कृ॑न्ता॒म्यप्यपि॑ कृन्तामि॒ यः । \newline
45. कृ॒न्ता॒मि॒ यो यः कृ॑न्तामि कृन्तामि॒ यो᳚ ऽस्मा न॒स्मान्. यः कृ॑न्तामि कृन्तामि॒ यो᳚ ऽस्मान् । \newline
46. यो᳚ ऽस्मा न॒स्मान्. यो यो᳚ ऽस्मान् द्वेष्टि॒ द्वेष्ट्य॒स्मान्. यो यो᳚ ऽस्मान् द्वेष्टि॑ । \newline
47. अ॒स्मान् द्वेष्टि॒ द्वेष्ट्य॒स्मा न॒स्मान् द्वेष्टि॒ यं ॅयम् द्वेष्ट्य॒स्मा न॒स्मान् द्वेष्टि॒ यम् । \newline
48. द्वेष्टि॒ यं ॅयम् द्वेष्टि॒ द्वेष्टि॒ यम् च॑ च॒ यम् द्वेष्टि॒ द्वेष्टि॒ यम् च॑ । \newline
49. यम् च॑ च॒ यं ॅयम् च॑ व॒यं ॅव॒यम् च॒ यं ॅयम् च॑ व॒यम् । \newline
50. च॒ व॒यं ॅव॒यम् च॑ च व॒यम् द्वि॒ष्मो द्वि॒ष्मो व॒यम् च॑ च व॒यम् द्वि॒ष्मः । \newline
51. व॒यम् द्वि॒ष्मो द्वि॒ष्मो व॒यं ॅव॒यम् द्वि॒ष्म इ॒द मि॒दम् द्वि॒ष्मो व॒यं ॅव॒यम् द्वि॒ष्म इ॒दम् । \newline
52. द्वि॒ष्म इ॒द मि॒दम् द्वि॒ष्मो द्वि॒ष्म इ॒द म॑स्यास्ये॒ दम् द्वि॒ष्मो द्वि॒ष्म इ॒द म॑स्य । \newline
53. इ॒द म॑स्यास्ये॒ द मि॒द म॑स्य ग्री॒वा ग्री॒वा अ॑स्ये॒ द मि॒द म॑स्य ग्री॒वाः । \newline
54. अ॒स्य॒ ग्री॒वा ग्री॒वा अ॑स्यास्य ग्री॒वा अप्यपि॑ ग्री॒वा अ॑स्यास्य ग्री॒वा अपि॑ । \newline
55. ग्री॒वा अप्यपि॑ ग्री॒वा ग्री॒वा अपि॑ कृन्तामि कृन्ता॒म्यपि॑ ग्री॒वा ग्री॒वा अपि॑ कृन्तामि । \newline
\pagebreak
\markright{ TS 1.2.5.2  \hfill https://www.vedavms.in \hfill}

\section{ TS 1.2.5.2 }

\textbf{TS 1.2.5.2 } \newline
\textbf{Samhita Paata} \newline

अपि॑ कृन्ताम्य॒स्मे राय॒स्त्वे राय॒स्तोते॒ रायः॒ सं दे॑वि दे॒व्योर्वश्या॑ पश्यस्व॒ त्वष्टी॑मती ते सपेय सु॒रेता॒ रेतो॒ दधा॑ना वी॒रं ॅवि॑देय॒ तव॑ स॒न्दृशि॒ माऽहꣳरा॒यस्पोषे॑ण॒ वि यो॑षं ॥ \newline

\textbf{Pada Paata} \newline

अपीति॑ । कृ॒न्ता॒मि॒ । अ॒स्मे इति॑ । रायः॑ । त्वे इति॑ । रायः॑ । तोते᳚ । रायः॑ । समिति॑ । दे॒वि॒ । दे॒व्या । उ॒र्वश्या᳚ । प॒श्य॒स्व॒ । त्वष्टी॑मती । ते॒ । स॒पे॒य॒ । सु॒रेता॒ इति॑ सु - रेताः᳚ । रेतः॑ । दधा॑ना । वी॒रम् । वि॒दे॒य॒ । तव॑ । सं॒दृशीति॑ सं - दृशि॑ । मा॒ । अ॒हम् । रा॒यः । पोषे॑ण । वीति॑ । यो॒ष॒म् ॥  \newline


\textbf{Krama Paata} \newline

अपि॑ कृन्तामि । कृ॒न्ता॒म्य॒स्मे । अ॒स्मे रायः॑ । अ॒स्मे इत्य॒स्मे । राय॒स्त्वे । त्वे रायः॑ । त्वे इति॒ त्वे । राय॒स्तोते᳚ । तोते॒ रायः॑ । रायः॒ सम् । सम् दे॑वि । दे॒वि॒ दे॒व्या । दे॒व्योर्वश्या᳚ । उ॒र्वश्या॑ पश्यस्व । प॒श्य॒स्व॒ त्वष्टी॑मती । त्वष्टी॑मती ते । ते॒ स॒पे॒य॒ । स॒पे॒य॒ सु॒रेताः᳚ । सु॒रेता॒ रेतः॑ । सु॒रेता॒ इति॑ सु - रेताः᳚ । रेतो॒ दधा॑ना । दधा॑ना वी॒रम् । वी॒रं ॅवि॑देय । वि॒दे॒य॒ तव॑ । तव॑ स॒न्दृशि॑ । स॒न्दृशि॒ मा । स॒न्दृशीति॑ सं - दृशि॑ । माऽहम् । अ॒हꣳ रा॒यः । रा॒यस्पोषे॑ण । पोषे॑ण॒ वि । वि यो॑षम् । यो॒ष॒मिति॑ योषम् । \newline

\textbf{Jatai Paata} \newline

1. अपि॑ कृन्तामि कृन्ता॒ म्यप्यपि॑ कृन्तामि । \newline
2. कृ॒न्ता॒ म्य॒स्मे अ॒स्मे कृ॑न्तामि कृन्ता म्य॒स्मे । \newline
3. अ॒स्मे रायो॒ रायो॒ ऽस्मे अ॒स्मे रायः॑ । \newline
4. अ॒स्मे इत्य॒स्मे । \newline
5. राय॒ स्त्वे त्वे रायो॒ राय॒ स्त्वे । \newline
6. त्वे रायो॒ राय॒ स्त्वे त्वे रायः॑ । \newline
7. त्वे इति॒ त्वे । \newline
8. राय॒स्तोते॒ तोते॒ रायो॒ राय॒स्तोते᳚ । \newline
9. तोते॒ रायो॒ राय॒स्तोते॒ तोते॒ रायः॑ । \newline
10. रायः॒ सꣳ सꣳ रायो॒ रायः॒ सम् । \newline
11. सम् दे॑वि देवि॒ सꣳ सम् दे॑वि । \newline
12. दे॒वि॒ दे॒व्या दे॒व्या दे॑वि देवि दे॒व्या । \newline
13. दे॒व्योर्वश्यो॒र्वश्या॑ दे॒व्या दे॒व्योर्वश्या᳚ । \newline
14. उ॒र्वश्या॑ पश्यस्व पश्यस्वो॒र्वश्यो॒र्वश्या॑ पश्यस्व । \newline
15. प॒श्य॒स्व॒ त्वष्टी॑मती॒ त्वष्टी॑मती पश्यस्व पश्यस्व॒ त्वष्टी॑मती । \newline
16. त्वष्टी॑मती ते ते॒ त्वष्टी॑मती॒ त्वष्टी॑मती ते । \newline
17. ते॒ स॒पे॒य॒ स॒पे॒य॒ ते॒ ते॒ स॒पे॒य॒ । \newline
18. स॒पे॒य॒ सु॒रेताः᳚ सु॒रेताः᳚ सपेय सपेय सु॒रेताः᳚ । \newline
19. सु॒रेता॒ रेतो॒ रेतः॑ सु॒रेताः᳚ सु॒रेता॒ रेतः॑ । \newline
20. सु॒रेता॒ इति॑ सु - रेताः᳚ । \newline
21. रेतो॒ दधा॑ना॒ दधा॑ना॒ रेतो॒ रेतो॒ दधा॑ना । \newline
22. दधा॑ना वी॒रं ॅवी॒रम् दधा॑ना॒ दधा॑ना वी॒रम् । \newline
23. वी॒रं ॅवि॑देय विदेय वी॒रं ॅवी॒रं ॅवि॑देय । \newline
24. वि॒दे॒य॒ तव॒ तव॑ विदेय विदेय॒ तव॑ । \newline
25. तव॑ स॒न्दृशि॑ स॒न्दृशि॒ तव॒ तव॑ स॒न्दृशि॑ । \newline
26. स॒न्दृशि॑ मा मा स॒न्दृशि॑ स॒न्दृशि॑ मा । \newline
27. स॒न्दृशीति॑ सं - दृशि॑ । \newline
28. मा॒ ऽह म॒हम् मा॑ मा॒ ऽहम् । \newline
29. अ॒हꣳ रा॒यो रा॒यो॑ ऽह म॒हꣳ रा॒यः । \newline
30. रा॒य स्पोषे॑ण॒ पोषे॑ण रा॒यो रा॒य स्पोषे॑ण । \newline
31. पोषे॑ण॒ वि वि पोषे॑ण॒ पोषे॑ण॒ वि । \newline
32. वि यो॑षं ॅयोषं॒ ॅवि वि यो॑षम् । \newline
33. यो॒ष॒मिति॑ योषम् । \newline

\textbf{Ghana Paata } \newline

1. अपि॑ कृन्तामि कृन्ता॒म्यप्यपि॑ कृन्ताम्य॒स्मे अ॒स्मे कृ॑न्ता॒म्यप्यपि॑ कृन्ताम्य॒स्मे । \newline
2. कृ॒न्ता॒म्य॒स्मे अ॒स्मे कृ॑न्तामि कृन्ताम्य॒स्मे रायो॒ रायो॒ ऽस्मे कृ॑न्तामि कृन्ताम्य॒स्मे रायः॑ । \newline
3. अ॒स्मे रायो॒ रायो॒ ऽस्मे अ॒स्मे राय॒स्त्वे त्वे रायो॒ ऽस्मे अ॒स्मे राय॒स्त्वे । \newline
4. अ॒स्मे इत्य॒स्मे । \newline
5. राय॒स्त्वे त्वे रायो॒ राय॒स्त्वे रायो॒ राय॒स्त्वे रायो॒ राय॒स्त्वे रायः॑ । \newline
6. त्वे रायो॒ राय॒स्त्वे त्वे राय॒स्तोते॒ तोते॒ राय॒स्त्वे त्वे राय॒स्तोते᳚ । \newline
7. त्वे इति॒ त्वे । \newline
8. राय॒स्तोते॒ तोते॒ रायो॒ राय॒स्तोते॒ रायो॒ राय॒स्तोते॒ रायो॒ राय॒स्तोते॒ रायः॑ । \newline
9. तोते॒ रायो॒ राय॒स्तोते॒ तोते॒ रायः॒ सꣳ सꣳ राय॒स्तोते॒ तोते॒ रायः॒ सम् । \newline
10. रायः॒ सꣳ सꣳ रायो॒ रायः॒ सम् दे॑वि देवि॒ सꣳ रायो॒ रायः॒ सम् दे॑वि । \newline
11. सम् दे॑वि देवि॒ सꣳ सम् दे॑वि दे॒व्या दे॒व्या दे॑वि॒ सꣳ सम् दे॑वि दे॒व्या । \newline
12. दे॒वि॒ दे॒व्या दे॒व्या दे॑वि देवि दे॒व्योर्वश्यो॒र्वश्या॑ दे॒व्या दे॑वि देवि दे॒व्योर्वश्या᳚ । \newline
13. दे॒व्योर्वश्यो॒र्वश्या॑ दे॒व्या दे॒व्योर्वश्या॑ पश्यस्व पश्यस्वो॒र्वश्या॑ दे॒व्या दे॒व्योर्वश्या॑ पश्यस्व । \newline
14. उ॒र्वश्या॑ पश्यस्व पश्य स्वो॒र्व श्यो॒र्वश्या॑ पश्यस्व॒ त्वष्टी॑मती॒ त्वष्टी॑मती पश्य स्वो॒र्व श्यो॒र्वश्या॑ पश्यस्व॒ त्वष्टी॑मती । \newline
15. प॒श्य॒स्व॒ त्वष्टी॑मती॒ त्वष्टी॑मती पश्यस्व पश्यस्व॒ त्वष्टी॑मती ते ते॒ त्वष्टी॑मती पश्यस्व पश्यस्व॒ त्वष्टी॑मती ते । \newline
16. त्वष्टी॑मती ते ते॒ त्वष्टी॑मती॒ त्वष्टी॑मती ते सपेय सपेय ते॒ त्वष्टी॑मती॒ त्वष्टी॑मती ते सपेय । \newline
17. ते॒ स॒पे॒य॒ स॒पे॒य॒ ते॒ ते॒ स॒पे॒य॒ सु॒रेताः᳚ सु॒रेताः᳚ सपेय ते ते सपेय सु॒रेताः᳚ । \newline
18. स॒पे॒य॒ सु॒रेताः᳚ सु॒रेताः᳚ सपेय सपेय सु॒रेता॒ रेतो॒ रेतः॑ सु॒रेताः᳚ सपेय सपेय सु॒रेता॒ रेतः॑ । \newline
19. सु॒रेता॒ रेतो॒ रेतः॑ सु॒रेताः᳚ सु॒रेता॒ रेतो॒ दधा॑ना॒ दधा॑ना॒ रेतः॑ सु॒रेताः᳚ सु॒रेता॒ रेतो॒ दधा॑ना । \newline
20. सु॒रेता॒ इति॑ सु - रेताः᳚ । \newline
21. रेतो॒ दधा॑ना॒ दधा॑ना॒ रेतो॒ रेतो॒ दधा॑ना वी॒रं ॅवी॒रम् दधा॑ना॒ रेतो॒ रेतो॒ दधा॑ना वी॒रम् । \newline
22. दधा॑ना वी॒रं ॅवी॒रम् दधा॑ना॒ दधा॑ना वी॒रं ॅवि॑देय विदेय वी॒रम् दधा॑ना॒ दधा॑ना वी॒रं ॅवि॑देय । \newline
23. वी॒रं ॅवि॑देय विदेय वी॒रं ॅवी॒रं ॅवि॑देय॒ तव॒ तव॑ विदेय वी॒रं ॅवी॒रं ॅवि॑देय॒ तव॑ । \newline
24. वि॒दे॒य॒ तव॒ तव॑ विदेय विदेय॒ तव॑ स॒न्दृशि॑ स॒न्दृशि॒ तव॑ विदेय विदेय॒ तव॑ स॒न्दृशि॑ । \newline
25. तव॑ स॒न्दृशि॑ स॒न्दृशि॒ तव॒ तव॑ स॒न्दृशि॑ मा मा स॒न्दृशि॒ तव॒ तव॑ स॒न्दृशि॑ मा । \newline
26. स॒न्दृशि॑ मा मा स॒न्दृशि॑ स॒न्दृशि॑ मा॒ ऽह म॒हम् मा॑ स॒न्दृशि॑ स॒न्दृशि॑ मा॒ ऽहम् । \newline
27. स॒न्दृशीति॑ सं - दृशि॑ । \newline
28. मा॒ ऽह म॒हम् मा॑ मा॒ ऽहꣳ रा॒यो रा॒यो॑ ऽहम् मा॑ मा॒ ऽहꣳ रा॒यः । \newline
29. अ॒हꣳ रा॒यो रा॒यो॑ ऽह म॒हꣳ रा॒य स्पोषे॑ण॒ पोषे॑ण रा॒यो॑ ऽह म॒हꣳ रा॒य स्पोषे॑ण । \newline
30. रा॒य स्पोषे॑ण॒ पोषे॑ण रा॒यो रा॒य स्पोषे॑ण॒ वि वि पोषे॑ण रा॒यो रा॒य स्पोषे॑ण॒ वि । \newline
31. पोषे॑ण॒ वि वि पोषे॑ण॒ पोषे॑ण॒ वि यो॑षं ॅयोषं॒ ॅवि पोषे॑ण॒ पोषे॑ण॒ वि यो॑षम् । \newline
32. वि यो॑षं ॅयोषं॒ ॅवि वि यो॑षम् । \newline
33. यो॒ष॒मिति॑ योषम् । \newline
\pagebreak
\markright{ TS 1.2.6.1  \hfill https://www.vedavms.in \hfill}

\section{ TS 1.2.6.1 }

\textbf{TS 1.2.6.1 } \newline
\textbf{Samhita Paata} \newline

अꣳ॒॒शुना॑ ते अꣳ॒॒शुः पृ॑च्यतां॒ परु॑षा॒ परु॑र् ग॒न्धस्ते॒ काम॑मवतु॒ मदा॑य॒ रसो॒ अच्यु॑तो॒ ऽमात्यो॑ऽसि शु॒क्रस्ते॒ ग्रहो॒ऽभि त्यं दे॒वꣳ स॑वि॒तार॑मू॒ण्योः᳚ क॒विक्र॑तु॒मर्चा॑मि स॒त्यस॑वसꣳ रत्न॒धाम॒भि प्रि॒यम म॒तिमू॒र्द्ध्वा यस्या॒मति॒र्भा अदि॑द्युत॒थ् सवी॑मनि॒ हिर॑ण्यपाणिरमिमीत सु॒क्रतुः॑ कृ॒पा सुवः॑ । प्र॒जाभ्य॑स्त्वा प्रा॒णाय॑ त्वा व्या॒नाय॑ त्वा प्र॒जास्त्वमनु॒ ( ) प्राणि॑हि प्र॒जास्त्वामनु॒ प्राण॑न्तु ॥ \newline

\textbf{Pada Paata} \newline

अꣳ॒॒शुना᳚ । ते॒ । अꣳ॒॒शुः । पृ॒च्य॒ता॒म् । परु॑षा । परुः॑ । ग॒न्धः । ते॒ । काम᳚म् । अ॒व॒तु॒ । मदा॑य । रसः॑ । अच्यु॑तः । अ॒मात्यः॑ । अ॒सि॒ । शु॒क्रः । ते॒ । ग्रहः॑ । अ॒भीति॑ । त्यम् । दे॒वम् । स॒वि॒तार᳚म् । ऊ॒ण्योः᳚ । क॒विक्र॑तु॒मिति॑ क॒वि - क्र॒तु॒म् । अर्चा॑मि । स॒त्यस॑वस॒मिति॑ स॒त्य -स॒व॒स॒म् । र॒त्न॒धामिति॑ रत्न - धाम् । अ॒भीति॑ । प्रि॒यम् । म॒तिम् । ऊ॒र्द्ध्वा । यस्य॑ । अ॒मतिः॑ । भाः । अदि॑द्युतत् । सवी॑मनि । हिर॑ण्यपाणि॒रिति॒ हिर॑ण्य - पा॒णिः॒ । अ॒मि॒मी॒त॒ । सु॒क्रतु॒रिति॑ सु - क्रतुः॑ । कृ॒पा । सुवः॑ ॥ प्र॒जाभ्य॒ इति॑ प्र - जाभ्यः॑ । त्वा॒ । प्रा॒णायेति॑ प्र - अ॒नाय॑ । त्वा॒ । व्या॒नायेति॑ वि -अ॒नाय॑ । त्वा॒ । प्र॒जा इति॑ प्र - जाः । त्वम् । अनु॑ ( ) । प्रेति॑ । अ॒नि॒हि॒ । प्र॒जा इति॑ प्र - जाः । त्वाम् । अनु॑ । प्रेति॑ । अ॒न॒न्तु॒ ॥  \newline


\textbf{Krama Paata} \newline

अꣳ॒॒शुना॑ ते । ते॒ अꣳ॒॒शुः । अꣳ॒॒शुः पृ॑च्यताम् । पृ॒च्य॒ता॒म् परु॑षा । परु॑षा॒ परुः॑ । परु॑र् ग॒न्धः । ग॒न्धस्ते᳚ । ते॒ काम᳚म् । काम॑मवतु । अ॒व॒तु॒ मदा॑य । मदा॑य॒ रसः॑ । रसो॒ अच्यु॑तः । अच्यु॑तो॒ऽमात्यः॑ । अ॒मात्यो॑ऽसि । अ॒सि॒ शु॒क्रः । शु॒क्रस्ते᳚ । ते॒ ग्रहः॑ । ग्रहो॒ऽभि । अ॒भि त्यम् । त्यम् दे॒वम् । दे॒वꣳ स॑वि॒तार᳚म् । स॒वि॒तार॑मू॒ण्योः᳚ । ऊ॒ण्योः᳚ क॒विक्र॑तुम् । क॒विक्र॑तु॒मर्चा॑मि । क॒विक्र॑तु॒मिति॑ क॒वि - क्र॒तु॒म् । अर्चा॑मि स॒त्यस॑वसम् । स॒त्यस॑वसꣳ रत्न॒धाम् । स॒त्यस॑वस॒मिति॑ स॒त्य - स॒व॒स॒म् । र॒त्न॒धाम॒भि । र॒त्न॒धामिति॑ रत्न - धाम् । अ॒भि प्रि॒यम् । प्रि॒यम् म॒तिम् । म॒तिमू॒र्द्ध्वा । ऊ॒र्द्ध्वा यस्य॑ । यस्या॒मतिः॑ । अ॒मति॒र् भाः । भा अदि॑द्युतत् । अदि॑द्युत॒थ् सवी॑मनि । सवी॑मनि॒ हिर॑ण्यपाणिः । हिर॑ण्यपाणिरमिमीत । हिर॑ण्यपाणि॒रिति॒ हिर॑ण्य - पा॒णिः॒ । अ॒मि॒मी॒त॒ सु॒क्रतुः॑ । सु॒क्रतुः॑ कृ॒पा । सु॒क्रतु॒रिति॑ सु - क्रतुः॑ । कृ॒पा सुवः॑ । सुव॒रिति॒ सुवः॑ ॥ प्र॒जाभ्य॑स्त्वा । प्र॒जाभ्य॒ इति॑ प्र - जाभ्यः॑ । त्वा॒ प्रा॒णाय॑ । प्रा॒णाय॑ त्वा । प्रा॒णायेति॑ प्र - अ॒नाय॑ । त्वा॒ व्या॒नाय॑ । व्या॒नाय॑ त्वा । व्या॒नायेति॑ वि - अ॒नाय॑ । त्वा॒ प्र॒जाः । प्र॒जास्त्वम् । प्र॒जा इति॑ प्र - जाः । त्वमनु॑ ( ) । अनु॒ प्र । प्राणि॑हि । अ॒नि॒हि॒ प्र॒जाः । प्र॒जास्त्वाम् । प्र॒जा इति॑ प्र - जाः । त्वामनु॑  । अनु॒ प्र । प्राण॑न्तु । अ॒न॒न्त्वित्य॑नन्तु । \newline

\textbf{Jatai Paata} \newline

1. अ॒(ग्म्॒)शुना॑ ते ते॒ ऽ(ग्म्॒)शुना॒ ऽ(ग्म्॒)शुना॑ ते । \newline
2. ते॒ अ॒(ग्म्॒)शु र॒(ग्म्॒)शु स्ते॑ ते अ॒(ग्म्॒)शुः । \newline
3. अ॒(ग्म्॒)शुः पृ॑च्यताम् पृच्यता म॒(ग्म्॒)शु र॒(ग्म्॒)शुः पृ॑च्यताम् । \newline
4. पृ॒च्य॒ता॒म् परु॑षा॒ परु॑षा पृच्यताम् पृच्यता॒म् परु॑षा । \newline
5. परु॑षा॒ परुः॒ परुः॒ परु॑षा॒ परु॑षा॒ परुः॑ । \newline
6. परु॑र् ग॒न्धो ग॒न्धः परुः॒ परु॑र् ग॒न्धः । \newline
7. ग॒न्ध स्ते॑ ते ग॒न्धो ग॒न्ध स्ते᳚ । \newline
8. ते॒ काम॒म् काम॑म् ते ते॒ काम᳚म् । \newline
9. काम॑ मवत्ववतु॒ काम॒म् काम॑ मवतु । \newline
10. अ॒व॒तु॒ मदा॑य॒ मदा॑ यावत्ववतु॒ मदा॑य । \newline
11. मदा॑य॒ रसो॒ रसो॒ मदा॑य॒ मदा॑य॒ रसः॑ । \newline
12. रसो॒ अच्यु॑तो॒ अच्यु॑तो॒ रसो॒ रसो॒ अच्यु॑तः । \newline
13. अच्यु॑तो॒ ऽमात्यो॒ ऽमात्यो॒ अच्यु॑तो॒ अच्यु॑तो॒ ऽमात्यः॑ । \newline
14. अ॒मात्यो᳚ ऽस्यस्य॒मात्यो॒ ऽमात्यो॑ ऽसि । \newline
15. अ॒सि॒ शु॒क्रः शु॒क्रो᳚ ऽस्यसि शु॒क्रः । \newline
16. शु॒क्र स्ते॑ ते शु॒क्रः शु॒क्र स्ते᳚ । \newline
17. ते॒ ग्रहो॒ ग्रह॑स्ते ते॒ ग्रहः॑ । \newline
18. ग्रहो॒ ऽभ्य॑भि ग्रहो॒ ग्रहो॒ ऽभि । \newline
19. अ॒भि त्यम् त्य म॒भ्य॑भि त्यम् । \newline
20. त्यम् दे॒वम् दे॒वम् त्यम् त्यम् दे॒वम् । \newline
21. दे॒वꣳ स॑वि॒तार(ग्म्॑) सवि॒तार॑म् दे॒वम् दे॒वꣳ स॑वि॒तार᳚म् । \newline
22. स॒वि॒तार॑ मू॒ण्यो॑ रू॒ण्योः᳚ सवि॒तार(ग्म्॑) सवि॒तार॑ मू॒ण्योः᳚ । \newline
23. ऊ॒ण्योः᳚ क॒विक्र॑तुम् क॒विक्र॑तु मू॒ण्यो॑ रू॒ण्योः᳚ क॒विक्र॑तुम् । \newline
24. क॒विक्र॑तु॒ मर्चा॒ म्यर्चा॑मि क॒विक्र॑तुम् क॒विक्र॑तु॒ मर्चा॑मि । \newline
25. क॒विक्र॑तु॒मिति॑ क॒वि - क्र॒तु॒म् । \newline
26. अर्चा॑मि स॒त्यस॑वसꣳ स॒त्यस॑वस॒ मर्चा॒म्यर्चा॑मि स॒त्यस॑वसम् । \newline
27. स॒त्यस॑वसꣳ रत्न॒धाꣳ र॑त्न॒धाꣳ स॒त्यस॑वसꣳ स॒त्यस॑वसꣳ रत्न॒धाम् । \newline
28. स॒त्यस॑वस॒मिति॑ स॒त्य - स॒व॒स॒म् । \newline
29. र॒त्न॒धा म॒भ्य॑भि र॑त्न॒धाꣳ र॑त्न॒धा म॒भि । \newline
30. र॒त्न॒धामिति॑ रत्न - धाम् । \newline
31. अ॒भि प्रि॒यम् प्रि॒य म॒भ्य॑भि प्रि॒यम् । \newline
32. प्रि॒यम् म॒तिम् म॒तिम् प्रि॒यम् प्रि॒यम् म॒तिम् । \newline
33. म॒ति मू॒र्द्ध्वोर्द्ध्वा म॒तिम् म॒ति मू॒र्द्ध्वा । \newline
34. ऊ॒र्द्ध्वा यस्य॒ यस्यो॒र्द्ध्वोर्द्ध्वा यस्य॑ । \newline
35. यस्या॒मति॑ र॒मति॒र् यस्य॒ यस्या॒मतिः॑ । \newline
36. अ॒मति॒र् भा भा अ॒मति॑ र॒मति॒र् भाः । \newline
37. भा अदि॑द्युत॒ ददि॑द्युत॒द् भा भा अदि॑द्युतत् । \newline
38. अदि॑द्युत॒थ् सवी॑मनि॒ सवी॑म॒न्यदि॑द्युत॒ ददि॑द्युत॒थ् सवी॑मनि । \newline
39. सवी॑मनि॒ हिर॑ण्यपाणि॒र्॒. हिर॑ण्यपाणिः॒ सवी॑मनि॒ सवी॑मनि॒ हिर॑ण्यपाणिः । \newline
40. हिर॑ण्यपाणि रमिमीतामिमीत॒ हिर॑ण्यपाणि॒र्॒. हिर॑ण्यपाणि रमिमीत । \newline
41. हिर॑ण्यपाणि॒रिति॒ हिर॑ण्य - पा॒णिः॒ । \newline
42. अ॒मि॒मी॒त॒ सु॒क्रतुः॑ सु॒क्रतु॑ रमिमीतामिमीत सु॒क्रतुः॑ । \newline
43. सु॒क्रतुः॑ कृ॒पा कृ॒पा सु॒क्रतुः॑ सु॒क्रतुः॑ कृ॒पा । \newline
44. सु॒क्रतु॒रिति॑ सु - क्रतुः॑ । \newline
45. कृ॒पा सुवः॒ सुवः॑ कृ॒पा कृ॒पा सुवः॑ । \newline
46. सुव॒रिति॒ सुवः॑ । \newline
47. प्र॒जाभ्य॑ स्त्वा त्वा प्र॒जाभ्यः॑ प्र॒जाभ्य॑ स्त्वा । \newline
48. प्र॒जाभ्य॒ इति॑ प्र - जाभ्यः॑ । \newline
49. त्वा॒ प्रा॒णाय॑ प्रा॒णाय॑ त्वा त्वा प्रा॒णाय॑ । \newline
50. प्रा॒णाय॑ त्वा त्वा प्रा॒णाय॑ प्रा॒णाय॑ त्वा । \newline
51. प्रा॒णायेति॑ प्र - अ॒नाय॑ । \newline
52. त्वा॒ व्या॒नाय॑ व्या॒नाय॑ त्वा त्वा व्या॒नाय॑ । \newline
53. व्या॒नाय॑ त्वा त्वा व्या॒नाय॑ व्या॒नाय॑ त्वा । \newline
54. व्या॒नायेति॑ वि - अ॒नाय॑ । \newline
55. त्वा॒ प्र॒जाः प्र॒जा स्त्वा᳚ त्वा प्र॒जाः । \newline
56. प्र॒जा स्त्वम् त्वम् प्र॒जाः प्र॒जा स्त्वम् । \newline
57. प्र॒जा इति॑ प्र - जाः । \newline
58. त्व मन्वनु॒ त्वम् त्व मनु॑ । \newline
59. अनु॒ प्र प्राण्वनु॒ प्र । \newline
60. प्राणि॑ह्यनिहि॒ प्र प्राणि॑हि । \newline
61. अ॒नि॒हि॒ प्र॒जाः प्र॒जा अ॑निह्यनिहि प्र॒जाः । \newline
62. प्र॒जा स्त्वाम् त्वाम् प्र॒जाः प्र॒जा स्त्वाम् । \newline
63. प्र॒जा इति॑ प्र - जाः । \newline
64. त्वा मन्वनु॒ त्वाम् त्वा मनु॑ । \newline
65. अनु॒ प्र प्राण्वनु॒ प्र । \newline
66. प्राण॑न्त्वनन्तु॒ प्र प्राण॑न्तु । \newline
67. अ॒न॒न्त्वित्य॑नन्तु । \newline

\textbf{Ghana Paata } \newline

1. अ॒(ग्म्॒)शुना॑ ते ते॒ ऽ(ग्म्॒)शुना॒ ऽ(ग्म्॒)शुना॑ ते अ॒(ग्म्॒)शु र॒(ग्म्॒)शुस्ते॒ ऽ(ग्म्॒)शुना॒ ऽ(ग्म्॒)शुना॑ ते अ॒(ग्म्॒)शुः । \newline
2. ते॒ अ॒(ग्म्॒)शु र॒(ग्म्॒)शुस्ते॑ ते अ॒(ग्म्॒)शुः पृ॑च्यताम् पृच्यता म॒(ग्म्॒)शुस्ते॑ ते अ॒(ग्म्॒)शुः पृ॑च्यताम् । \newline
3. अ॒(ग्म्॒)शुः पृ॑च्यताम् पृच्यता मꣳ॒॒शु र॒(ग्म्॒)शुः पृ॑च्यता॒म् परु॑षा॒ परु॑षा पृच्यता मꣳ॒॒शुर॒(ग्म्॒)शुः पृ॑च्यता॒म् परु॑षा । \newline
4. पृ॒च्य॒ता॒म् परु॑षा॒ परु॑षा पृच्यताम् पृच्यता॒म् परु॑षा॒ परुः॒ परुः॒ परु॑षा पृच्यताम् पृच्यता॒म् परु॑षा॒ परुः॑ । \newline
5. परु॑षा॒ परुः॒ परुः॒ परु॑षा॒ परु॑षा॒ परु॑र् ग॒न्धो ग॒न्धः परुः॒ परु॑षा॒ परु॑षा॒ परु॑र् ग॒न्धः । \newline
6. परु॑र् ग॒न्धो ग॒न्धः परुः॒ परु॑र् ग॒न्धस्ते॑ ते ग॒न्धः परुः॒ परु॑र् ग॒न्धस्ते᳚ । \newline
7. ग॒न्धस्ते॑ ते ग॒न्धो ग॒न्धस्ते॒ काम॒म् काम॑म् ते ग॒न्धो ग॒न्धस्ते॒ काम᳚म् । \newline
8. ते॒ काम॒म् काम॑म् ते ते॒ काम॑ मवत्ववतु॒ काम॑म् ते ते॒ काम॑ मवतु । \newline
9. काम॑ मवत्ववतु॒ काम॒म् काम॑ मवतु॒ मदा॑य॒ मदा॑यावतु॒ काम॒म् काम॑ मवतु॒ मदा॑य । \newline
10. अ॒व॒तु॒ मदा॑य॒ मदा॑यावत्ववतु॒ मदा॑य॒ रसो॒ रसो॒ मदा॑यावत्ववतु॒ मदा॑य॒ रसः॑ । \newline
11. मदा॑य॒ रसो॒ रसो॒ मदा॑य॒ मदा॑य॒ रसो॒ अच्यु॑तो॒ अच्यु॑तो॒ रसो॒ मदा॑य॒ मदा॑य॒ रसो॒ अच्यु॑तः । \newline
12. रसो॒ अच्यु॑तो॒ अच्यु॑तो॒ रसो॒ रसो॒ अच्यु॑तो॒ ऽमात्यो॒ ऽमात्यो॒ अच्यु॑तो॒ रसो॒ रसो॒ अच्यु॑तो॒ ऽमात्यः॑ । \newline
13. अच्यु॑तो॒ ऽमात्यो॒ ऽमात्यो॒ अच्यु॑तो॒ अच्यु॑तो॒ ऽमात्यो᳚ ऽस्यस्य॒मात्यो॒ अच्यु॑तो॒ अच्यु॑तो॒ ऽमात्यो॑ ऽसि । \newline
14. अ॒मात्यो᳚ ऽस्यस्य॒मात्यो॒ ऽमात्यो॑ ऽसि शु॒क्रः शु॒क्रो᳚ ऽस्य॒मात्यो॒ ऽमात्यो॑ ऽसि शु॒क्रः । \newline
15. अ॒सि॒ शु॒क्रः शु॒क्रो᳚ ऽस्यसि शु॒क्रस्ते॑ ते शु॒क्रो᳚ ऽस्यसि शु॒क्रस्ते᳚ । \newline
16. शु॒क्रस्ते॑ ते शु॒क्रः शु॒क्रस्ते॒ ग्रहो॒ ग्रह॑स्ते शु॒क्रः शु॒क्रस्ते॒ ग्रहः॑ । \newline
17. ते॒ ग्रहो॒ ग्रह॑स्ते ते॒ ग्रहो॒ ऽभ्य॑भि ग्रह॑स्ते ते॒ ग्रहो॒ ऽभि । \newline
18. ग्रहो॒ ऽभ्य॑भि ग्रहो॒ ग्रहो॒ ऽभि त्यम् त्य म॒भि ग्रहो॒ ग्रहो॒ ऽभि त्यम् । \newline
19. अ॒भि त्यम् त्य म॒भ्य॑भि त्यम् दे॒वम् दे॒वम् त्य म॒भ्य॑भि त्यम् दे॒वम् । \newline
20. त्यम् दे॒वम् दे॒वम् त्यम् त्यम् दे॒वꣳ स॑वि॒तार(ग्म्॑) सवि॒तार॑म् दे॒वम् त्यम् त्यम् दे॒वꣳ स॑वि॒तार᳚म् । \newline
21. दे॒वꣳ स॑वि॒तार(ग्म्॑) सवि॒तार॑म् दे॒वम् दे॒वꣳ स॑वि॒तार॑ मू॒ण्यो॑ रू॒ण्योः᳚ सवि॒तार॑म् दे॒वम् दे॒वꣳ स॑वि॒तार॑ मू॒ण्योः᳚ । \newline
22. स॒वि॒तार॑ मू॒ण्यो॑ रू॒ण्योः᳚ सवि॒तार(ग्म्॑) सवि॒तार॑ मू॒ण्योः᳚ क॒विक्र॑तुम् क॒विक्र॑तु मू॒ण्योः᳚ सवि॒तार(ग्म्॑) सवि॒तार॑ मू॒ण्योः᳚ क॒विक्र॑तुम् । \newline
23. ऊ॒ण्योः᳚ क॒विक्र॑तुम् क॒विक्र॑तु मू॒ण्यो॑ रू॒ण्योः᳚ क॒विक्र॑तु॒ मर्चा॒म्यर्चा॑मि क॒विक्र॑तु मू॒ण्यो॑ रू॒ण्योः᳚ क॒विक्र॑तु॒ मर्चा॑मि । \newline
24. क॒विक्र॑तु॒ मर्चा॒म्यर्चा॑मि क॒विक्र॑तुम् क॒विक्र॑तु॒ मर्चा॑मि स॒त्यस॑वसꣳ स॒त्यस॑वस॒ मर्चा॑मि क॒विक्र॑तुम् क॒विक्र॑तु॒ मर्चा॑मि स॒त्यस॑वसम् । \newline
25. क॒विक्र॑तु॒मिति॑ क॒वि - क्र॒तु॒म् । \newline
26. अर्चा॑मि स॒त्यस॑वसꣳ स॒त्यस॑वस॒ मर्चा॒म्यर्चा॑मि स॒त्यस॑वसꣳ रत्न॒धाꣳ र॑त्न॒धाꣳ स॒त्यस॑वस॒ मर्चा॒म्यर्चा॑मि स॒त्यस॑वसꣳ रत्न॒धाम् । \newline
27. स॒त्यस॑वसꣳ रत्न॒धाꣳ र॑त्न॒धाꣳ स॒त्यस॑वसꣳ स॒त्यस॑वसꣳ रत्न॒धा म॒भ्य॑भि र॑त्न॒धाꣳ स॒त्यस॑वसꣳ स॒त्यस॑वसꣳ रत्न॒धा म॒भि । \newline
28. स॒त्यस॑वस॒मिति॑ स॒त्य - स॒व॒स॒म् । \newline
29. र॒त्न॒धा म॒भ्य॑भि र॑त्न॒धाꣳ र॑त्न॒धा म॒भि प्रि॒यम् प्रि॒य म॒भि र॑त्न॒धाꣳ र॑त्न॒धा म॒भि प्रि॒यम् । \newline
30. र॒त्न॒धामिति॑ रत्न - धाम् । \newline
31. अ॒भि प्रि॒यम् प्रि॒य म॒भ्य॑भि प्रि॒यम् म॒तिम् म॒तिम् प्रि॒य म॒भ्य॑भि प्रि॒यम् म॒तिम् । \newline
32. प्रि॒यम् म॒तिम् म॒तिम् प्रि॒यम् प्रि॒यम् म॒ति मू॒र्द्ध्वोर्द्ध्वा म॒तिम् प्रि॒यम् प्रि॒यम् म॒ति मू॒र्द्ध्वा । \newline
33. म॒ति मू॒र्द्ध्वोर्द्ध्वा म॒तिम् म॒ति मू॒र्द्ध्वा यस्य॒ यस्यो॒र्द्ध्वा म॒तिम् म॒ति मू॒र्द्ध्वा यस्य॑ । \newline
34. ऊ॒र्द्ध्वा यस्य॒ यस्यो॒र्द्ध्वोर्द्ध्वा यस्या॒मति॑र॒मति॒र् यस्यो॒र्द्ध्वोर्द्ध्वा यस्या॒मतिः॑ । \newline
35. यस्या॒ मति॑ र॒मति॒र् यस्य॒ यस्या॒मति॒र् भा भा अ॒मति॒र् यस्य॒ यस्या॒मति॒र् भाः । \newline
36. अ॒मति॒र् भा भा अ॒मति॑ र॒मति॒र् भा अदि॑द्युत॒ ददि॑द्युत॒द् भा अ॒मति॑ र॒मति॒र् भा अदि॑द्युतत् । \newline
37. भा अदि॑द्युत॒ ददि॑द्युत॒द् भा भा अदि॑द्युत॒थ् सवी॑मनि॒ सवी॑म॒न्यदि॑द्युत॒द् भा भा अदि॑द्युत॒थ् सवी॑मनि । \newline
38. अदि॑द्युत॒थ् सवी॑मनि॒ सवी॑म॒न्यदि॑द्युत॒ ददि॑द्युत॒थ् सवी॑मनि॒ हिर॑ण्यपाणि॒र्॒. हिर॑ण्यपाणिः॒ सवी॑म॒न्यदि॑द्युत॒ ददि॑द्युत॒थ् सवी॑मनि॒ हिर॑ण्यपाणिः । \newline
39. सवी॑मनि॒ हिर॑ण्यपाणि॒र्॒. हिर॑ण्यपाणिः॒ सवी॑मनि॒ सवी॑मनि॒ हिर॑ण्यपाणि रमिमीतामिमीत॒ हिर॑ण्यपाणिः॒ सवी॑मनि॒ सवी॑मनि॒ हिर॑ण्यपाणि रमिमीत । \newline
40. हिर॑ण्यपाणि रमिमीतामिमीत॒ हिर॑ण्यपाणि॒र्॒. हिर॑ण्यपाणि रमिमीत सु॒क्रतुः॑ सु॒क्रतु॑ रमिमीत॒ हिर॑ण्यपाणि॒र्॒. हिर॑ण्यपाणि रमिमीत सु॒क्रतुः॑ । \newline
41. हिर॑ण्यपाणि॒रिति॒ हिर॑ण्य - पा॒णिः॒ । \newline
42. अ॒मि॒मी॒त॒ सु॒क्रतुः॑ सु॒क्रतु॑ रमिमीतामिमीत सु॒क्रतुः॑ कृ॒पा कृ॒पा सु॒क्रतु॑ रमिमीतामिमीत सु॒क्रतुः॑ कृ॒पा । \newline
43. सु॒क्रतुः॑ कृ॒पा कृ॒पा सु॒क्रतुः॑ सु॒क्रतुः॑ कृ॒पा सुवः॒ सुवः॑ कृ॒पा सु॒क्रतुः॑ सु॒क्रतुः॑ कृ॒पा सुवः॑ । \newline
44. सु॒क्रतु॒रिति॑ सु - क्रतुः॑ । \newline
45. कृ॒पा सुवः॒ सुवः॑ कृ॒पा कृ॒पा सुवः॑ । \newline
46. सुव॒रिति॒ सुवः॑ । \newline
47. प्र॒जाभ्य॑स्त्वा त्वा प्र॒जाभ्यः॑ प्र॒जाभ्य॑स्त्वा प्रा॒णाय॑ प्रा॒णाय॑ त्वा प्र॒जाभ्यः॑ प्र॒जाभ्य॑स्त्वा प्रा॒णाय॑ । \newline
48. प्र॒जाभ्य॒ इति॑ प्र - जाभ्यः॑ । \newline
49. त्वा॒ प्रा॒णाय॑ प्रा॒णाय॑ त्वा त्वा प्रा॒णाय॑ त्वा त्वा प्रा॒णाय॑ त्वा त्वा प्रा॒णाय॑ त्वा । \newline
50. प्रा॒णाय॑ त्वा त्वा प्रा॒णाय॑ प्रा॒णाय॑ त्वा व्या॒नाय॑ व्या॒नाय॑ त्वा प्रा॒णाय॑ प्रा॒णाय॑ त्वा व्या॒नाय॑ । \newline
51. प्रा॒णायेति॑ प्र - अ॒नाय॑ । \newline
52. त्वा॒ व्या॒नाय॑ व्या॒नाय॑ त्वा त्वा व्या॒नाय॑ त्वा त्वा व्या॒नाय॑ त्वा त्वा व्या॒नाय॑ त्वा । \newline
53. व्या॒नाय॑ त्वा त्वा व्या॒नाय॑ व्या॒नाय॑ त्वा प्र॒जाः प्र॒जास्त्वा᳚ व्या॒नाय॑ व्या॒नाय॑ त्वा प्र॒जाः । \newline
54. व्या॒नायेति॑ वि - अ॒नाय॑ । \newline
55. त्वा॒ प्र॒जाः प्र॒जास्त्वा᳚ त्वा प्र॒जास्त्वम् त्वम् प्र॒जास्त्वा᳚ त्वा प्र॒जास्त्वम् । \newline
56. प्र॒जास्त्वम् त्वम् प्र॒जाः प्र॒जास्त्व मन्वनु॒ त्वम् प्र॒जाः प्र॒जास्त्व मनु॑ । \newline
57. प्र॒जा इति॑ प्र - जाः । \newline
58. त्व मन्वनु॒ त्वम् त्व मनु॒ प्र प्राणु॒ त्वम् त्व मनु॒ प्र । \newline
59. अनु॒ प्र प्राण्वनु॒ प्राणि॑ह्यनिहि॒ प्राण्वनु॒ प्राणि॑हि । \newline
60. प्राणि॑ह्यनिहि॒ प्र प्राणि॑हि प्र॒जाः प्र॒जा अ॑निहि॒ प्र प्राणि॑हि प्र॒जाः । \newline
61. अ॒नि॒हि॒ प्र॒जाः प्र॒जा अ॑निह्यनिहि प्र॒जास्त्वाम् त्वाम् प्र॒जा अ॑निह्यनिहि प्र॒जास्त्वाम् । \newline
62. प्र॒जास्त्वाम् त्वाम् प्र॒जाः प्र॒जास्त्वा मन्वनु॒ त्वाम् प्र॒जाः प्र॒जास्त्वा मनु॑ । \newline
63. प्र॒जा इति॑ प्र - जाः । \newline
64. त्वा मन्वनु॒ त्वाम् त्वा मनु॒ प्र प्राणु॒ त्वाम् त्वा मनु॒ प्र । \newline
65. अनु॒ प्र प्राण्वनु॒ प्राण॑न्त्वनन्तु॒ प्राण्वनु॒ प्राण॑न्तु । \newline
66. प्राण॑न्त्वनन्तु॒ प्र प्राण॑न्तु । \newline
67. अ॒न॒न्त्वित्य॑नन्तु । \newline
\pagebreak
\markright{ TS 1.2.7.1  \hfill https://www.vedavms.in \hfill}

\section{ TS 1.2.7.1 }

\textbf{TS 1.2.7.1 } \newline
\textbf{Samhita Paata} \newline

सोमं॑ ते क्रीणा॒म्यूर्ज॑स्वन्तं॒ पय॑स्वन्तं ॅवी॒र्या॑वन्तमभि-माति॒षाहꣳ॑ शु॒क्रं ते॑ शु॒क्रेण॑ क्रीणामि च॒न्द्रं च॒न्द्रेणा॒मृत॑म॒मृते॑न स॒म्यत्ते॒ गोर॒स्मे च॒न्द्राणि॒ तप॑सस्त॒नूर॑सि प्र॒जाप॑ते॒र् वर्ण॒स्तस्या᳚स्ते सहस्रपो॒षं पुष्य॑न्त्याश्चर॒मेण॑ प॒शुना᳚ क्रीणाम्य॒स्मे ते॒ बन्धु॒र्मयि॑ ते॒ रायः॑ श्रयन्ताम॒स्मे ज्योतिः॑ सोमविक्र॒यिणि॒ तमो॑ मि॒त्रो न॒ एहि॒ सुमि॑त्रधा॒ इन्द्र॑स्यो॒रु ( ) मा वि॑श॒ दक्षि॑ण-मु॒शन्नु॒शन्तꣳ॑ स्यो॒नः स्यो॒नꣳ स्वान॒ भ्राजाङ्घा॑रे॒ बंभा॑रे॒ हस्त॒ सुह॑स्त॒ कृशा॑नवे॒ते वः॑ सोम॒ क्रय॑णा॒स्तान् र॑क्षद्ध्वं॒ मा वो॑ दभन् ॥ \newline

\textbf{Pada Paata} \newline

सोम᳚म् । ते॒ । क्री॒णा॒मि॒ । ऊर्ज॑स्वन्तम् । पय॑स्वन्तम् । वी॒र्या॑वन्त॒मिति॑ वी॒र्य॑ - व॒न्त॒म् । अ॒भि॒मा॒ति॒षाह॒मित्य॑भिमाति - साह᳚म् । शु॒क्रम् । ते॒ । शु॒क्रेण॑ । क्री॒णा॒मि॒ । च॒न्द्रम् । च॒न्द्रेण॑ । अ॒मृत᳚म् । अ॒मृते॑न । स॒म्यत् । ते॒ । गोः । अ॒स्मे इति॑ । च॒न्द्राणि॑ । तप॑सः । त॒नूः । अ॒सि॒ । प्र॒जाप॑ते॒रिति॑ प्र॒जा - प॒तेः॒ । वर्णः॑ । तस्याः᳚ । ते॒ । स॒ह॒स्र॒पो॒षमिति॑ सहस्र -पो॒षम् । पुष्य॑न्त्याः । च॒र॒मेण॑ । प॒शुना᳚ । क्री॒णा॒मि॒ । अ॒स्मे इति॑ । ते॒ । बन्धुः॑ । मयि॑ । ते॒ । रायः॑ । श्र॒य॒न्ता॒म् । अ॒स्मे इति॑ । ज्योतिः॑ । सो॒म॒वि॒क्र॒यिणीति॑ सोम - वि॒क्र॒यिणि॑ । तमः॑ । मि॒त्रः । नः॒ । एति॑ । इ॒हि॒ । सुमि॑त्रधा॒ इति॒ सुमि॑त्र - धाः॒ । इन्द्र॑स्य । ऊ॒रुम् ( ) । एति॑ । वि॒श॒ । दक्षि॑णम् । उ॒शन्न् । उ॒शन्त᳚म् । स्यो॒नः । स्यो॒नम् । स्वान॑ । भ्राज॑ । अङ्घा॑रे । बंभा॑रे । हस्त॑ । सुह॒स्तेति॒ सु - ह॒स्त॒ । कृशा॑न॒विति॒ कृश॑ - अ॒नो॒ । ए॒ते । वः॒ । सो॒म॒क्रय॑णा॒ इति॑ सोम - क्रय॑णाः । तान् । र॒क्ष॒द्ध्व॒म् । मा । वः॒ । द॒भ॒न्न् ॥  \newline


\textbf{Krama Paata} \newline

सोम॑म् ते । ते॒ क्री॒णा॒मि॒ । क्री॒णा॒म्यूर्ज॑स्वन्तम् । ऊर्ज॑स्वन्त॒म् पय॑स्वन्तम् । पय॑स्वन्तं ॅवी॒र्या॑वन्तम् । वी॒र्या॑वन्तमभिमाति॒षाह᳚म् । वी॒र्या॑वन्त॒मिति॑ वी॒र्य॑ - व॒न्त॒म् । अ॒भि॒मा॒ति॒षाहꣳ॑ शु॒क्रम् । अ॒भि॒मा॒ति॒षाह॒मित्य॑भिमाति - साह᳚म् । शु॒क्रम् ते᳚ । ते॒ शु॒क्रेण॑ । शु॒क्रेण॑ क्रीणामि । क्री॒णा॒मि॒ च॒न्द्रम् । च॒न्द्रम् च॒न्द्रेण॑ । च॒न्द्रेणा॒मृत᳚म् । अ॒मृत॑म॒मृते॑न । अ॒मृते॑न स॒म्ॅयत् । स॒म्ॅयत् ते᳚ । ते॒ गोः । गोर॒स्मे । अ॒स्मे च॒न्द्राणि॑ । अ॒स्मे इत्य॒स्मे । च॒न्द्राणि॒ तप॑सः । तप॑सस्त॒नूः । त॒नूर॑सि । अ॒सि॒ प्र॒जाप॑तेः । प्र॒जाप॑ते॒र् वर्णः॑ । प्र॒जाप॑ते॒रिति॑ प्र॒जा - प॒तेः॒ । वर्ण॒स्तस्याः᳚ । तस्या᳚स्ते । ते॒ स॒ह॒स्र॒पो॒षम् । स॒ह॒स्र॒पो॒षम् पुष्य॑न्त्याः । स॒ह॒स्र॒पो॒षमिति॑ सहस्र - पो॒षम् । पुष्य॑न्त्याश्चर॒मेण॑ । च॒र॒मेण॑ प॒शुना᳚ । प॒शुना᳚ क्रीणामि । क्री॒णा॒म्य॒स्मे । अ॒स्मे ते᳚ । अ॒स्मे इत्य॒स्मे । ते॒ बन्धुः॑ । बन्धु॒र् मयि॑ । मयि॑ ते । ते॒ रायः॑ । रायः॑ श्रयन्ताम् । श्र॒य॒न्ता॒म॒स्मे । अ॒स्मे ज्योतिः॑ । अ॒स्मे इत्य॒स्मे । ज्योतिः॑ सोमविक्र॒यिणि॑ । सो॒म॒वि॒क्र॒यिणि॒ तमः॑ । सो॒म॒वि॒क्र॒यिणीति॑ सोम - वि॒क्र॒यिणि॑ । तमो॑ मि॒त्रः । मि॒त्रो नः॑ । न॒ आ । एहि॑ । इ॒हि॒ सुमि॑त्रधाः । सुमि॑त्रधा॒ इन्द्र॑स्य । सुमि॑त्रधा॒ इति॒ सुमि॑त्र - धाः॒ । इन्द्र॑स्यो॒रुम् ( ) । ऊ॒रुमा । आ वि॑श । वि॒श॒ दक्षि॑णम् । दक्षि॑णमु॒शन्न् । उ॒शन्नु॒शन्त᳚म् । उ॒शन्तꣳ॑ स्यो॒नः । स्यो॒नः स्यो॒नम् । स्यो॒नꣳ स्वान॑ । स्वान॒ भ्राज॑ । भ्राजाङ्घा॑रे । अङ्घा॑रे॒ बम्भा॑रे । बम्भा॑रे॒ हस्त॑ । हस्त॒ सुह॑स्त । सुह॑स्त॒ कृशा॑नो । सुह॒स्तेति॒ सु - ह॒स्त॒ । कृशा॑नवे॒ते । कृशा॑न॒विति॒ कृश॑ - अ॒नो॒ । ए॒ते वः॑ । वः॒ सो॒म॒क्रय॑णाः । सो॒म॒क्रय॑णा॒स्तान् । सो॒म॒क्रय॑णा॒ इति॑ सोम - क्रय॑णाः । तान् र॑क्षद्ध्वम् । र॒क्ष॒द्ध्व॒म् मा । मा वः॑ । वो॒ द॒भ॒न्न्॒ । द॒भ॒न्निति॑ दभन्न् । \newline

\textbf{Jatai Paata} \newline

1. सोम॑म् ते ते॒ सोम॒(ग्म्॒) सोम॑म् ते । \newline
2. ते॒ क्री॒णा॒मि॒ क्री॒णा॒मि॒ ते॒ ते॒ क्री॒णा॒मि॒ । \newline
3. क्री॒णा॒ म्यूर्ज॑स्वन्त॒ मूर्ज॑स्वन्तम् क्रीणामि क्रीणा॒ म्यूर्ज॑स्वन्तम् । \newline
4. ऊर्ज॑स्वन्त॒म् पय॑स्वन्त॒म् पय॑स्वन्त॒ मूर्ज॑स्वन्त॒ मूर्ज॑स्वन्त॒म् पय॑स्वन्तम् । \newline
5. पय॑स्वन्तं ॅवी॒र्या॑वन्तं ॅवी॒र्या॑वन्त॒म् पय॑स्वन्त॒म् पय॑स्वन्तं ॅवी॒र्या॑वन्तम् । \newline
6. वी॒र्या॑वन्त मभिमाति॒षाह॑ मभिमाति॒षाहं॑ ॅवी॒र्या॑वन्तं ॅवी॒र्या॑वन्त मभिमाति॒षाह᳚म् । \newline
7. वी॒र्या॑वन्त॒मिति॑ वी॒र्य॑ - व॒न्त॒म् । \newline
8. अ॒भि॒मा॒ति॒षाह(ग्म्॑) शु॒क्रꣳ शु॒क्र म॑भिमाति॒षाह॑ मभिमाति॒षाह(ग्म्॑) शु॒क्रम् । \newline
9. अ॒भि॒मा॒ति॒षाह॒मित्य॑भिमाति - साह᳚म् । \newline
10. शु॒क्रम् ते॑ ते शु॒क्रꣳ शु॒क्रम् ते᳚ । \newline
11. ते॒ शु॒क्रेण॑ शु॒क्रेण॑ ते ते शु॒क्रेण॑ । \newline
12. शु॒क्रेण॑ क्रीणामि क्रीणामि शु॒क्रेण॑ शु॒क्रेण॑ क्रीणामि । \newline
13. क्री॒णा॒मि॒ च॒न्द्रम् च॒न्द्रम् क्री॑णामि क्रीणामि च॒न्द्रम् । \newline
14. च॒न्द्रम् च॒न्द्रेण॑ च॒न्द्रेण॑ च॒न्द्रम् च॒न्द्रम् च॒न्द्रेण॑ । \newline
15. च॒न्द्रेणा॒मृत॑ म॒मृत॑म् च॒न्द्रेण॑ च॒न्द्रेणा॒मृत᳚म् । \newline
16. अ॒मृत॑ म॒मृते॑ना॒मृते॑ना॒मृत॑ म॒मृत॑ म॒मृते॑न । \newline
17. अ॒मृते॑न स॒म्यथ् स॒म्य द॒मृते॑ना॒मृते॑न स॒म्यत् । \newline
18. स॒म्यत् ते॑ ते स॒म्यथ् स॒म्यत् ते᳚ । \newline
19. ते॒ गोर् गोस्ते॑ ते॒ गोः । \newline
20. गोर॒स्मे अ॒स्मे गोर् गोर॒स्मे । \newline
21. अ॒स्मे च॒न्द्राणि॑ च॒न्द्राण्य॒स्मे अ॒स्मे च॒न्द्राणि॑ । \newline
22. अ॒स्मे इत्य॒स्मे । \newline
23. च॒न्द्राणि॒ तप॑स॒ स्तप॑स श्च॒न्द्राणि॑ च॒न्द्राणि॒ तप॑सः । \newline
24. तप॑स स्त॒नू स्त॒नू स्तप॑स॒ स्तप॑स स्त॒नूः । \newline
25. त॒नू र॑स्यसि त॒नू स्त॒नू र॑सि । \newline
26. अ॒सि॒ प्र॒जाप॑तेः प्र॒जाप॑ते रस्यसि प्र॒जाप॑तेः । \newline
27. प्र॒जाप॑ते॒र् वर्णो॒ वर्णः॑ प्र॒जाप॑तेः प्र॒जाप॑ते॒र् वर्णः॑ । \newline
28. प्र॒जाप॑ते॒रिति॑ प्र॒जा - प॒तेः॒ । \newline
29. वर्ण॒ स्तस्या॒ स्तस्या॒ वर्णो॒ वर्ण॒ स्तस्याः᳚ । \newline
30. तस्या᳚ स्ते ते॒ तस्या॒ स्त स्या᳚स्ते । \newline
31. ते॒ स॒ह॒स्र॒पो॒षꣳ स॑हस्रपो॒षम् ते॑ ते सहस्रपो॒षम् । \newline
32. स॒ह॒स्र॒पो॒षम् पुष्य॑न्त्याः॒ पुष्य॑न्त्याः सहस्रपो॒षꣳ स॑हस्रपो॒षम् पुष्य॑न्त्याः । \newline
33. स॒ह॒स्र॒पो॒षमिति॑ सहस्र - पो॒षम् । \newline
34. पुष्य॑न्त्याश्चर॒मेण॑ चर॒मेण॒ पुष्य॑न्त्याः॒ पुष्य॑न्त्याश्चर॒मेण॑ । \newline
35. च॒र॒मेण॑ प॒शुना॑ प॒शुना॑ चर॒मेण॑ चर॒मेण॑ प॒शुना᳚ । \newline
36. प॒शुना᳚ क्रीणामि क्रीणामि प॒शुना॑ प॒शुना᳚ क्रीणामि । \newline
37. क्री॒णा॒म्य॒स्मे अ॒स्मे क्री॑णामि क्रीणाम्य॒स्मे । \newline
38. अ॒स्मे ते॑ ते॒ ऽस्मे अ॒स्मे ते᳚ । \newline
39. अ॒स्मे इत्य॒स्मे । \newline
40. ते॒ बन्धु॒र् बन्धु॑ स्ते ते॒ बन्धुः॑ । \newline
41. बन्धु॒र् मयि॒ मयि॒ बन्धु॒र् बन्धु॒र् मयि॑ । \newline
42. मयि॑ ते ते॒ मयि॒ मयि॑ ते । \newline
43. ते॒ रायो॒ राय॑ स्ते ते॒ रायः॑ । \newline
44. रायः॑ श्रयन्ताꣳ श्रयन्ता॒(ग्म्॒) रायो॒ रायः॑ श्रयन्ताम् । \newline
45. श्र॒य॒न्ता॒ म॒स्मे अ॒स्मे श्र॑यन्ताꣳ श्रयन्ता म॒स्मे । \newline
46. अ॒स्मे ज्योति॒र् ज्योति॑ र॒स्मे अ॒स्मे ज्योतिः॑ । \newline
47. अ॒स्मे इत्य॒स्मे । \newline
48. ज्योतिः॑ सोमविक्र॒यिणि॑ सोमविक्र॒यिणि॒ ज्योति॒र् ज्योतिः॑ सोमविक्र॒यिणि॑ । \newline
49. सो॒म॒वि॒क्र॒यिणि॒ तम॒ स्तमः॑ सोमविक्र॒यिणि॑ सोमविक्र॒यिणि॒ तमः॑ । \newline
50. सो॒म॒वि॒क्र॒यिणीति॑ सोम - वि॒क्र॒यिणि॑ । \newline
51. तमो॑ मि॒त्रो मि॒त्र स्तम॒ स्तमो॑ मि॒त्रः । \newline
52. मि॒त्रो नो॑ नो मि॒त्रो मि॒त्रो नः॑ । \newline
53. न॒ आ नो॑ न॒ आ । \newline
54. एही॒ह्येहि॑ । \newline
55. इ॒हि॒ सुमि॑त्रधाः॒ सुमि॑त्रधा इहीहि॒ सुमि॑त्रधाः । \newline
56. सुमि॑त्रधा॒ इन्द्र॒स्ये न्द्र॑स्य॒ सुमि॑त्रधाः॒ सुमि॑त्रधा॒ इन्द्र॑स्य । \newline
57. सुमि॑त्रधा॒ इति॒ सुमि॑त्र - धाः॒ । \newline
58. इन्द्र॑स्यो॒रु मू॒रु मिन्द्र॒स्ये न्द्र॑स्यो॒रुम् । \newline
59. ऊ॒रु मोरु मू॒रु मा । \newline
60. आ वि॑श वि॒शा वि॑श । \newline
61. वि॒श॒ दक्षि॑ण॒म् दक्षि॑णं ॅविश विश॒ दक्षि॑णम् । \newline
62. दक्षि॑ण मु॒शन् नु॒शन् दक्षि॑ण॒म् दक्षि॑ण मु॒शन्न् । \newline
63. उ॒शन् नु॒शन्त॑ मु॒शन्त॑ मु॒शन् नु॒शन् नु॒शन्त᳚म् । \newline
64. उ॒शन्त(ग्ग्॑) स्यो॒नः स्यो॒न उ॒शन्त॑ मु॒शन्त(ग्ग्॑) स्यो॒नः । \newline
65. स्यो॒नः स्यो॒नꣳ स्यो॒नꣳ स्यो॒नः स्यो॒नः स्यो॒नम् । \newline
66. स्यो॒नꣳ स्वान॒ स्वान॑ स्यो॒नꣳ स्यो॒नꣳ स्वान॑ । \newline
67. स्वान॒ भ्राज॒ भ्राज॒ स्वान॒ स्वान॒ भ्राज॑ । \newline
68. भ्राजाङ्घा॒रे ऽङ्घा॑रे॒ भ्राज॒ भ्राजाङ्घा॑रे । \newline
69. अङ्घा॑रे॒ बंभा॑रे॒ बंभा॒रे ऽङ्घा॒रे ऽङ्घा॑रे॒ बंभा॑रे । \newline
70. बंभा॑रे॒ हस्त॒ हस्त॒ बंभा॑रे॒ बंभा॑रे॒ हस्त॑ । \newline
71. हस्त॒ सुह॑स्त॒ सुह॑स्त॒ हस्त॒ हस्त॒ सुह॑स्त । \newline
72. सुह॑स्त॒ कृशा॑नो॒ कृशा॑नो॒ सुह॑स्त॒ सुह॑स्त॒ कृशा॑नो । \newline
73. सुह॒स्तेति॒ सु - ह॒स्त॒ । \newline
74. कृशा॑नवे॒त ए॒ते कृशा॑नो॒ कृशा॑नवे॒ते । \newline
75. कृशा॑न॒विति॒ कृश॑ - अ॒नो॒ । \newline
76. ए॒ते वो॑ व ए॒त ए॒ते वः॑ । \newline
77. वः॒ सो॒म॒क्रय॑णाः सोम॒क्रय॑णा वो वः सोम॒क्रय॑णाः । \newline
78. सो॒म॒क्रय॑णा॒ स्ताꣳ स्तान् थ्सो॑म॒क्रय॑णाः सोम॒क्रय॑णा॒ स्तान् । \newline
79. सो॒म॒क्रय॑णा॒ इति॑ सोम - क्रय॑णाः । \newline
80. तान् र॑क्षद्ध्वꣳ रक्षद्ध्व॒म् ताꣳ स्तान् र॑क्षद्ध्वम् । \newline
81. र॒क्ष॒द्ध्व॒म् मा मा र॑क्षद्ध्वꣳ रक्षद्ध्व॒म् मा । \newline
82. मा वो॑ वो॒ मा मा वः॑ । \newline
83. वो॒ द॒भ॒न् द॒भ॒न्॒. वो॒ वो॒ द॒भ॒न्न् । \newline
84. द॒भ॒न्निति॑ दभन्न् । \newline

\textbf{Ghana Paata } \newline

1. सोम॑म् ते ते॒ सोम॒(ग्म्॒) सोम॑म् ते क्रीणामि क्रीणामि ते॒ सोम॒(ग्म्॒) सोम॑म् ते क्रीणामि । \newline
2. ते॒ क्री॒णा॒मि॒ क्री॒णा॒मि॒ ते॒ ते॒ क्री॒णा॒म्यूर्ज॑स्वन्त॒ मूर्ज॑स्वन्तम् क्रीणामि ते ते क्रीणा॒म्यूर्ज॑स्वन्तम् । \newline
3. क्री॒णा॒म्यूर्ज॑स्वन्त॒ मूर्ज॑स्वन्तम् क्रीणामि क्रीणा॒म्यूर्ज॑स्वन्त॒म् पय॑स्वन्त॒म् पय॑स्वन्त॒ मूर्ज॑स्वन्तम् क्रीणामि क्रीणा॒म्यूर्ज॑स्वन्त॒म् पय॑स्वन्तम् । \newline
4. ऊर्ज॑स्वन्त॒म् पय॑स्वन्त॒म् पय॑स्वन्त॒ मूर्ज॑स्वन्त॒ मूर्ज॑स्वन्त॒म् पय॑स्वन्तं ॅवी॒र्या॑वन्तं ॅवी॒र्या॑वन्त॒म् पय॑स्वन्त॒ मूर्ज॑स्वन्त॒ मूर्ज॑स्वन्त॒म् पय॑स्वन्तं ॅवी॒र्या॑वन्तम् । \newline
5. पय॑स्वन्तं ॅवी॒र्या॑वन्तं ॅवी॒र्या॑वन्त॒म् पय॑स्वन्त॒म् पय॑स्वन्तं ॅवी॒र्या॑वन्त मभिमाति॒षाह॑ मभिमाति॒षाहं॑ ॅवी॒र्या॑वन्त॒म् पय॑स्वन्त॒म् पय॑स्वन्तं ॅवी॒र्या॑वन्त मभिमाति॒षाह᳚म् । \newline
6. वी॒र्या॑वन्त मभिमाति॒षाह॑ मभिमाति॒षाहं॑ ॅवी॒र्या॑वन्तं ॅवी॒र्या॑वन्त मभिमाति॒षाह(ग्म्॑) शु॒क्रꣳ शु॒क्र म॑भिमाति॒षाहं॑ ॅवी॒र्या॑वन्तं ॅवी॒र्या॑वन्त मभिमाति॒षाह(ग्म्॑) शु॒क्रम् । \newline
7. वी॒र्या॑वन्त॒मिति॑ वी॒र्य॑ - व॒न्त॒म् । \newline
8. अ॒भि॒मा॒ति॒षाह(ग्म्॑) शु॒क्रꣳ शु॒क्र म॑भिमाति॒षाह॑ मभिमाति॒षाह(ग्म्॑) शु॒क्रम् ते॑ ते शु॒क्र म॑भिमाति॒षाह॑ मभिमाति॒षाह(ग्म्॑) शु॒क्रम् ते᳚ । \newline
9. अ॒भि॒मा॒ति॒षाह॒मित्य॑भिमाति - साह᳚म् । \newline
10. शु॒क्रम् ते॑ ते शु॒क्रꣳ शु॒क्रम् ते॑ शु॒क्रेण॑ शु॒क्रेण॑ ते शु॒क्रꣳ शु॒क्रम् ते॑ शु॒क्रेण॑ । \newline
11. ते॒ शु॒क्रेण॑ शु॒क्रेण॑ ते ते शु॒क्रेण॑ क्रीणामि क्रीणामि शु॒क्रेण॑ ते ते शु॒क्रेण॑ क्रीणामि । \newline
12. शु॒क्रेण॑ क्रीणामि क्रीणामि शु॒क्रेण॑ शु॒क्रेण॑ क्रीणामि च॒न्द्रम् च॒न्द्रम् क्री॑णामि शु॒क्रेण॑ शु॒क्रेण॑ क्रीणामि च॒न्द्रम् । \newline
13. क्री॒णा॒मि॒ च॒न्द्रम् च॒न्द्रम् क्री॑णामि क्रीणामि च॒न्द्रम् च॒न्द्रेण॑ च॒न्द्रेण॑ च॒न्द्रम् क्री॑णामि क्रीणामि च॒न्द्रम् च॒न्द्रेण॑ । \newline
14. च॒न्द्रम् च॒न्द्रेण॑ च॒न्द्रेण॑ च॒न्द्रम् च॒न्द्रम् च॒न्द्रेणा॒मृत॑ म॒मृत॑म् च॒न्द्रेण॑ च॒न्द्रम् च॒न्द्रम् च॒न्द्रेणा॒मृत᳚म् । \newline
15. च॒न्द्रेणा॒मृत॑ म॒मृत॑म् च॒न्द्रेण॑ च॒न्द्रेणा॒मृत॑ म॒मृते॑ना॒मृते॑ना॒मृत॑म् च॒न्द्रेण॑ च॒न्द्रेणा॒मृत॑ म॒मृते॑न । \newline
16. अ॒मृत॑ म॒मृते॑ना॒मृते॑ना॒मृत॑ म॒मृत॑ म॒मृते॑न स॒म्यथ् स॒म्यद॒मृते॑ना॒मृत॑ म॒मृत॑ म॒मृते॑न स॒म्यत् । \newline
17. अ॒मृते॑न स॒म्यथ् स॒म्यद॒मृते॑ना॒मृते॑न स॒म्यत् ते॑ ते स॒म्यद॒मृते॑ना॒मृते॑न स॒म्यत् ते᳚ । \newline
18. स॒म्यत् ते॑ ते स॒म्यथ् स॒म्यत् ते॒ गोर् गोस्ते॑ स॒म्यथ् स॒म्यत् ते॒ गोः । \newline
19. ते॒ गोर् गोस्ते॑ ते॒ गोर॒स्मे अ॒स्मे गोस्ते॑ ते॒ गोर॒स्मे । \newline
20. गोर॒स्मे अ॒स्मे गोर् गोर॒स्मे च॒न्द्राणि॑ च॒न्द्राण्य॒स्मे गोर् गोर॒स्मे च॒न्द्राणि॑ । \newline
21. अ॒स्मे च॒न्द्राणि॑ च॒न्द्राण्य॒स्मे अ॒स्मे च॒न्द्राणि॒ तप॑स॒स्तप॑सश्च॒न्द्राण्य॒स्मे अ॒स्मे च॒न्द्राणि॒ तप॑सः । \newline
22. अ॒स्मे इत्य॒स्मे । \newline
23. च॒न्द्राणि॒ तप॑स॒ स्तप॑स श्च॒न्द्राणि॑ च॒न्द्राणि॒ तप॑स स्त॒नू स्त॒नू स्तप॑स श्च॒न्द्राणि॑ च॒न्द्राणि॒ तप॑सस्त॒नूः । \newline
24. तप॑स स्त॒नू स्त॒नू स्तप॑स॒ स्तप॑स स्त॒नू र॑स्यसि त॒नू स्तप॑स॒ स्तप॑स स्त॒नूर॑सि । \newline
25. त॒नूर॑स्यसि त॒नूस्त॒नूर॑सि प्र॒जाप॑तेः प्र॒जाप॑तेरसि त॒नूस्त॒नूर॑सि प्र॒जाप॑तेः । \newline
26. अ॒सि॒ प्र॒जाप॑तेः प्र॒जाप॑तेरस्यसि प्र॒जाप॑ते॒र् वर्णो॒ वर्णः॑ प्र॒जाप॑तेरस्यसि प्र॒जाप॑ते॒र् वर्णः॑ । \newline
27. प्र॒जाप॑ते॒र् वर्णो॒ वर्णः॑ प्र॒जाप॑तेः प्र॒जाप॑ते॒र् वर्ण॒स्तस्या॒स्तस्या॒ वर्णः॑ प्र॒जाप॑तेः प्र॒जाप॑ते॒र् वर्ण॒स्तस्याः᳚ । \newline
28. प्र॒जाप॑ते॒रिति॑ प्र॒जा - प॒तेः॒ । \newline
29. वर्ण॒स्तस्या॒स्तस्या॒ वर्णो॒ वर्ण॒स्तस्या᳚स्ते ते॒ तस्या॒ वर्णो॒ वर्ण॒स्तस्या᳚स्ते । \newline
30. तस्या᳚स्ते ते॒ तस्या॒स्तस्या᳚स्ते सहस्रपो॒षꣳ स॑हस्रपो॒षम् ते॒ तस्या॒स्तस्या᳚स्ते सहस्रपो॒षम् । \newline
31. ते॒ स॒ह॒स्र॒पो॒षꣳ स॑हस्रपो॒षम् ते॑ ते सहस्रपो॒षम् पुष्य॑न्त्याः॒ पुष्य॑न्त्याः सहस्रपो॒षम् ते॑ ते सहस्रपो॒षम् पुष्य॑न्त्याः । \newline
32. स॒ह॒स्र॒पो॒षम् पुष्य॑न्त्याः॒ पुष्य॑न्त्याः सहस्रपो॒षꣳ स॑हस्रपो॒षम् पुष्य॑न्त्याश्चर॒मेण॑ चर॒मेण॒ पुष्य॑न्त्याः सहस्रपो॒षꣳ स॑हस्रपो॒षम् पुष्य॑न्त्याश्चर॒मेण॑ । \newline
33. स॒ह॒स्र॒पो॒षमिति॑ सहस्र - पो॒षम् । \newline
34. पुष्य॑न्त्याश्चर॒मेण॑ चर॒मेण॒ पुष्य॑न्त्याः॒ पुष्य॑न्त्याश्चर॒मेण॑ प॒शुना॑ प॒शुना॑ चर॒मेण॒ पुष्य॑न्त्याः॒ पुष्य॑न्त्याश्चर॒मेण॑ प॒शुना᳚ । \newline
35. च॒र॒मेण॑ प॒शुना॑ प॒शुना॑ चर॒मेण॑ चर॒मेण॑ प॒शुना᳚ क्रीणामि क्रीणामि प॒शुना॑ चर॒मेण॑ चर॒मेण॑ प॒शुना᳚ क्रीणामि । \newline
36. प॒शुना᳚ क्रीणामि क्रीणामि प॒शुना॑ प॒शुना᳚ क्रीणाम्य॒स्मे अ॒स्मे क्री॑णामि प॒शुना॑ प॒शुना᳚ क्रीणाम्य॒स्मे । \newline
37. क्री॒णा॒म्य॒स्मे अ॒स्मे क्री॑णामि क्रीणाम्य॒स्मे ते॑ ते॒ ऽस्मे क्री॑णामि क्रीणाम्य॒स्मे ते᳚ । \newline
38. अ॒स्मे ते॑ ते॒ ऽस्मे अ॒स्मे ते॒ बन्धु॒र् बन्धु॑स्ते अ॒स्मे अ॒स्मे ते॒ बन्धुः॑ । \newline
39. अ॒स्मे इत्य॒स्मे । \newline
40. ते॒ बन्धु॒र् बन्धु॑स्ते ते॒ बन्धु॒र् मयि॒ मयि॒ बन्धु॑स्ते ते॒ बन्धु॒र् मयि॑ । \newline
41. बन्धु॒र् मयि॒ मयि॒ बन्धु॒र् बन्धु॒र् मयि॑ ते ते॒ मयि॒ बन्धु॒र् बन्धु॒र् मयि॑ ते । \newline
42. मयि॑ ते ते॒ मयि॒ मयि॑ ते॒ रायो॒ राय॑स्ते॒ मयि॒ मयि॑ ते॒ रायः॑ । \newline
43. ते॒ रायो॒ राय॑स्ते ते॒ रायः॑ श्रयन्ताꣳ श्रयन्ता॒(ग्म्॒) राय॑स्ते ते॒ रायः॑ श्रयन्ताम् । \newline
44. रायः॑ श्रयन्ताꣳ श्रयन्ता॒(ग्म्॒) रायो॒ रायः॑ श्रयन्ता म॒स्मे अ॒स्मे श्र॑यन्ता॒(ग्म्॒) रायो॒ रायः॑ श्रयन्ता म॒स्मे । \newline
45. श्र॒य॒न्ता॒ म॒स्मे अ॒स्मे श्र॑यन्ताꣳ श्रयन्ता म॒स्मे ज्योति॒र् ज्योति॑ र॒स्मे श्र॑यन्ताꣳ श्रयन्ता म॒स्मे ज्योतिः॑ । \newline
46. अ॒स्मे ज्योति॒र् ज्योति॑ र॒स्मे अ॒स्मे ज्योतिः॑ सोमविक्र॒यिणि॑ सोमविक्र॒यिणि॒ ज्योति॑र॒स्मे अ॒स्मे ज्योतिः॑ सोमविक्र॒यिणि॑ । \newline
47. अ॒स्मे इत्य॒स्मे । \newline
48. ज्योतिः॑ सोमविक्र॒यिणि॑ सोमविक्र॒यिणि॒ ज्योति॒र् ज्योतिः॑ सोमविक्र॒यिणि॒ तम॒स्तमः॑ सोमविक्र॒यिणि॒ ज्योति॒र् ज्योतिः॑ सोमविक्र॒यिणि॒ तमः॑ । \newline
49. सो॒म॒वि॒क्र॒यिणि॒ तम॒स्तमः॑ सोमविक्र॒यिणि॑ सोमविक्र॒यिणि॒ तमो॑ मि॒त्रो मि॒त्रस्तमः॑ सोमविक्र॒यिणि॑ सोमविक्र॒यिणि॒ तमो॑ मि॒त्रः । \newline
50. सो॒म॒वि॒क्र॒यिणीति॑ सोम - वि॒क्र॒यिणि॑ । \newline
51. तमो॑ मि॒त्रो मि॒त्र स्तम॒ स्तमो॑ मि॒त्रो नो॑ नो मि॒त्रस्तम॒स्तमो॑ मि॒त्रो नः॑ । \newline
52. मि॒त्रो नो॑ नो मि॒त्रो मि॒त्रो न॒ आ नो॑ मि॒त्रो मि॒त्रो न॒ आ । \newline
53. न॒ आ नो॑ न॒ एही॒ह्या नो॑ न॒ एहि॑ । \newline
54. एही॒ह्येहि॒ सुमि॑त्रधाः॒ सुमि॑त्रधा इ॒ह्येहि॒ सुमि॑त्रधाः । \newline
55. इ॒हि॒ सुमि॑त्रधाः॒ सुमि॑त्रधा इहीहि॒ सुमि॑त्रधा॒ इन्द्र॒स्ये न्द्र॑स्य॒ सुमि॑त्रधा इहीहि॒ सुमि॑त्रधा॒ इन्द्र॑स्य । \newline
56. सुमि॑त्रधा॒ इन्द्र॒स्ये न्द्र॑स्य॒ सुमि॑त्रधाः॒ सुमि॑त्रधा॒ इन्द्र॑स्यो॒रु मू॒रु मिन्द्र॑स्य॒ सुमि॑त्रधाः॒ सुमि॑त्रधा॒ इन्द्र॑स्यो॒रुम् । \newline
57. सुमि॑त्रधा॒ इति॒ सुमि॑त्र - धाः॒ । \newline
58. इन्द्र॑स्यो॒रु मू॒रु मिन्द्र॒स्ये न्द्र॑स्यो॒रु मोरु मिन्द्र॒स्ये न्द्र॑स्यो॒रु मा । \newline
59. ऊ॒रु मोरु मू॒रु मा वि॑श वि॒शोरु मू॒रु मा वि॑श । \newline
60. आ वि॑श वि॒शा वि॑श॒ दक्षि॑ण॒म् दक्षि॑णं ॅवि॒शा वि॑श॒ दक्षि॑णम् । \newline
61. वि॒श॒ दक्षि॑ण॒म् दक्षि॑णं ॅविश विश॒ दक्षि॑ण मु॒शन् नु॒शन् दक्षि॑णं ॅविश विश॒ दक्षि॑ण मु॒शन्न् । \newline
62. दक्षि॑ण मु॒शन् नु॒शन् दक्षि॑ण॒म् दक्षि॑ण मु॒शन् नु॒शन्त॑ मु॒शन्त॑ मु॒शन् दक्षि॑ण॒म् दक्षि॑ण 
मु॒शन् नु॒शन्त᳚म् । \newline
63. उ॒शन् नु॒शन्त॑ मु॒शन्त॑ मु॒शन् नु॒शन् नु॒शन्त(ग्ग्॑) स्यो॒नः स्यो॒न उ॒शन्त॑ मु॒शन् नु॒शन् नु॒शन्त(ग्ग्॑) स्यो॒नः । \newline
64. उ॒शन्त(ग्ग्॑) स्यो॒नः स्यो॒न उ॒शन्त॑ मु॒शन्त(ग्ग्॑) स्यो॒नः स्यो॒नꣳ स्यो॒नꣳ स्यो॒न उ॒शन्त॑ मु॒शन्त(ग्ग्॑) स्यो॒नः स्यो॒नम् । \newline
65. स्यो॒नः स्यो॒नꣳ स्यो॒नꣳ स्यो॒नः स्यो॒नः स्यो॒नꣳ स्वान॒ स्वान॑ स्यो॒नꣳ स्यो॒नः स्यो॒नः स्यो॒नꣳ स्वान॑ । \newline
66. स्यो॒नꣳ स्वान॒ स्वान॑ स्यो॒नꣳ स्यो॒नꣳ स्वान॒ भ्राज॒ भ्राज॒ स्वान॑ स्यो॒नꣳ स्यो॒नꣳ स्वान॒ भ्राज॑ । \newline
67. स्वान॒ भ्राज॒ भ्राज॒ स्वान॒ स्वान॒ भ्राजाङ्घा॒रे ऽङ्घा॑रे॒ भ्राज॒ स्वान॒ स्वान॒ भ्राजाङ्घा॑रे । \newline
68. भ्राजाङ्घा॒रे ऽङ्घा॑रे॒ भ्राज॒ भ्राजाङ्घा॑रे॒ बंभा॑रे॒ बंभा॒रे ऽङ्घा॑रे॒ भ्राज॒ भ्राजाङ्घा॑रे॒ बंभा॑रे । \newline
69. अङ्घा॑रे॒ बंभा॑रे॒ बंभा॒रे ऽङ्घा॒रे ऽङ्घा॑रे॒ बंभा॑रे॒ हस्त॒ हस्त॒ बंभा॒रे ऽङ्घा॒रे ऽङ्घा॑रे॒ बंभा॑रे॒ हस्त॑ । \newline
70. बंभा॑रे॒ हस्त॒ हस्त॒ बंभा॑रे॒ बंभा॑रे॒ हस्त॒ सुह॑स्त॒ सुह॑स्त॒ हस्त॒ बंभा॑रे॒ बंभा॑रे॒ हस्त॒ सुह॑स्त । \newline
71. हस्त॒ सुह॑स्त॒ सुह॑स्त॒ हस्त॒ हस्त॒ सुह॑स्त॒ कृशा॑नो॒ कृशा॑नो॒ सुह॑स्त॒ हस्त॒ हस्त॒ सुह॑स्त॒ कृशा॑नो । \newline
72. सुह॑स्त॒ कृशा॑नो॒ कृशा॑नो॒ सुह॑स्त॒ सुह॑स्त॒ कृशा॑नवे॒त ए॒ते कृशा॑नो॒ सुह॑स्त॒ सुह॑स्त॒ कृशा॑नवे॒ते । \newline
73. सुह॒स्तेति॒ सु - ह॒स्त॒ । \newline
74. कृशा॑नवे॒त ए॒ते कृशा॑नो॒ कृशा॑नवे॒ते वो॑ व ए॒ते कृशा॑नो॒ कृशा॑नवे॒ते वः॑ । \newline
75. कृशा॑न॒विति॒ कृश॑ - अ॒नो॒ । \newline
76. ए॒ते वो॑ व ए॒त ए॒ते वः॑ सोम॒क्रय॑णाः सोम॒क्रय॑णा व ए॒त ए॒ते वः॑ सोम॒क्रय॑णाः । \newline
77. वः॒ सो॒म॒क्रय॑णाः सोम॒क्रय॑णा वो वः सोम॒क्रय॑णा॒ स्ताꣳ स्तान् थ्सो॑म॒क्रय॑णा वो वः सोम॒क्रय॑णा॒स्तान् । \newline
78. सो॒म॒क्रय॑णा॒स्ताꣳ स्तान् थ्सो॑म॒क्रय॑णाः सोम॒क्रय॑णा॒स्तान् र॑क्षद्ध्वꣳ रक्षद्ध्व॒म् तान् थ्सो॑म॒क्रय॑णाः सोम॒क्रय॑णा॒स्तान् र॑क्षद्ध्वम् । \newline
79. सो॒म॒क्रय॑णा॒ इति॑ सोम - क्रय॑णाः । \newline
80. तान् र॑क्षद्ध्वꣳ रक्षद्ध्व॒म् ताꣳ स्तान् र॑क्षद्ध्व॒म् मा मा र॑क्षद्ध्व॒म् ताꣳ स्तान् र॑क्षद्ध्व॒म् मा । \newline
81. र॒क्ष॒द्ध्व॒म् मा मा र॑क्षद्ध्वꣳ रक्षद्ध्व॒म् मा वो॑ वो॒ मा र॑क्षद्ध्वꣳ रक्षद्ध्व॒म् मा वः॑ । \newline
82. मा वो॑ वो॒ मा मा वो॑ दभन् दभन्. वो॒ मा मा वो॑ दभन्न् । \newline
83. वो॒ द॒भ॒न् द॒भ॒न्॒. वो॒ वो॒ द॒भ॒न्न् । \newline
84. द॒भ॒न्निति॑ दभन्न् । \newline
\pagebreak
\markright{ TS 1.2.8.1  \hfill https://www.vedavms.in \hfill}

\section{ TS 1.2.8.1 }

\textbf{TS 1.2.8.1 } \newline
\textbf{Samhita Paata} \newline

उदायु॑षा स्वा॒युषोदोष॑धीनाꣳ॒॒ रसे॒नोत् प॒र्जन्य॑स्य॒ शुष्मे॒णोद॑स्थाम॒मृताꣳ॒॒ अनु॑ । उ॒र्व॑न्तरि॑क्ष॒मन्वि॒ह्यदि॑त्याः॒ सदो॒ऽस्यदि॑त्याः॒ सद॒ आ सी॒दास्त॑भ्ना॒द्-द्यामृ॑ष॒भो अ॒न्तरि॑क्ष॒ममि॑मीत वरि॒माणं॑ पृथि॒व्या आऽसी॑द॒द् विश्वा॒ भुव॑नानि स॒म्राड् विश्वेत्तानि॒ वरु॑णस्य व्र॒तानि॒ वने॑षु॒ व्य॑न्तरि॑क्षं ततान॒ वाज॒मर्व॑थ्सु॒ पयो॑ अघ्नि॒यासु॑ हृ॒थ्सु-[ ] \newline

\textbf{Pada Paata} \newline

उदिति॑ । आयु॑षा । स्वा॒युषेति॑ सु - आ॒युषा᳚ । उदिति॑ । ओष॑धीनाम् । रसे॑न । उदिति॑ । प॒र्जन्य॑स्य । शुष्मे॑ण । उदिति॑ । अ॒स्था॒म् । अ॒मृतान्॑ । अनु॑ ॥ उ॒रु । अ॒न्तरि॑क्षम् । अन्विति॑ । इ॒हि॒ । अदि॑त्याः । सदः॑ । अ॒सि॒ । अदि॑त्याः । सदः॑ । एति॑ । सी॒द॒ । अस्त॑भ्नात् । द्याम् । ऋ॒ष॒भः । अ॒न्तरि॑क्षम् । अमि॑मीत । व॒रि॒माण᳚म् । पृ॒थि॒व्याः । एति॑ । अ॒सी॒द॒त् । विश्वा᳚ । भुव॑नानि । सं॒राडिति सं - राट् । विश्वा᳚ । इत् । तानि॑ । वरु॑णस्य । व्र॒तानि॑ । वने॑षु । वीति॑ । अ॒न्तरि॑क्षम् । त॒ता॒न॒ । वाज᳚म् । अर्व॒थ्स्वित्यर्व॑त् - सु॒ । पयः॑ । अ॒घ्नि॒यासु॑ । हृ॒थ्स्विति॑ हृत् - सु ।  \newline


\textbf{Krama Paata} \newline

उदायु॑षा । आयु॑षा स्वा॒युषा᳚ । स्वा॒युषोत् । स्वा॒युषेति॑ सु - आ॒युषा᳚ । उदोष॑धीनाम् । ओष॑धीनाꣳ॒॒ रसे॑न । रसे॒नोत् । उत् प॒र्जन्य॑स्य । प॒र्जन्य॑स्य॒ शुष्मे॑ण । शुष्मे॒णोत् । उद॑स्थाम् । अ॒स्था॒म॒मृतान्॑ । अ॒मृताꣳ॒॒ अनु॑ । अन्वित्यनु॑ ॥ उ॒र्व॑न्तरि॑क्षम् । अ॒न्तरि॑क्ष॒मनु॑ । अन्वि॑हि । इ॒ह्यदि॑त्याः । अदि॑त्याः॒ सदः॑ । सदो॑ऽसि । अ॒स्यदि॑त्याः । अदि॑त्याः॒ सदः॑ । सद॒ आ । आ सी॑द । सी॒दास्त॑भ्नात् । अस्त॑भ्ना॒द् द्याम् । द्यामृ॑ष॒भः । ऋ॒ष॒भो अ॒न्तरि॑क्षम् । अ॒न्तरि॑क्ष॒ममि॑मीत । अमि॑मीत वरि॒माण᳚म् । व॒रि॒माण॑म् पृथि॒व्याः । पृ॒थि॒व्या आ । आ ऽसी॑दत् । अ॒सी॒द॒द् विश्वा᳚ । विश्वा॒ भुव॑नानि । भुव॑नानि स॒म्राट् । 
स॒म्राड् विश्वा᳚ । स॒म्राडिति॑ सं - राट् । विश्वेत् । इत् तानि॑ । तानि॒ वरु॑णस्य । वरु॑णस्य व्र॒तानि॑ । व्र॒तानि॒ वने॑षु । वने॑षु॒ वि । व्य॑न्तरि॑क्षम् । अ॒न्तरि॑क्षम् ततान । त॒ता॒न॒ वाज᳚म् । वाज॒मर्व॑थ्सु । अर्व॑थ्सु॒ पयः॑ । अर्व॒थ्स्वित्यर्व॑त् - सु॒ । पयो॑ अघ्नि॒यासु॑ । अ॒घ्नि॒यासु॑ हृ॒थ्सु ( ) । हृ॒थ्सु क्रतु᳚म् । हृ॒थ्स्विति॑ हृ॒त् - सु \newline

\textbf{Jatai Paata} \newline

1. उदायु॒षा ऽऽयु॒षोदुदायु॑षा । \newline
2. आयु॑षा स्वा॒युषा᳚ स्वा॒युषा ऽऽयु॒षा ऽऽयु॑षा स्वा॒युषा᳚ । \newline
3. स्वा॒युषोदुथ् स्वा॒युषा᳚ स्वा॒युषोत् । \newline
4. स्वा॒युषेति॑ सु - आ॒युषा᳚ । \newline
5. उदोष॑धीना॒ मोष॑धीना॒ मुदुदोष॑धीनाम् । \newline
6. ओष॑धीना॒(ग्म्॒) रसे॑न॒ रसे॒नौष॑धीना॒ मोष॑धीना॒(ग्म्॒) रसे॑न । \newline
7. रसे॒नोदुद् रसे॑न॒ रसे॒नोत् । \newline
8. उत् प॒र्जन्य॑स्य प॒र्जन्य॒स्योदुत् प॒र्जन्य॑स्य । \newline
9. प॒र्जन्य॑स्य॒ शुष्मे॑ण॒ शुष्मे॑ण प॒र्जन्य॑स्य प॒र्जन्य॑स्य॒ शुष्मे॑ण । \newline
10. शुष्मे॒णो दुच्छुष्मे॑ण॒ शुष्मे॒णोत् । \newline
11. उद॑स्था मस्था॒ मुदु द॑स्थाम् । \newline
12. अ॒स्था॒ म॒मृता(ग्म्॑) अ॒मृता(ग्म्॑) अस्था मस्था म॒मृतान्॑ । \newline
13. अ॒मृता॒(ग्म्॒) अन्वन्व॒मृता(ग्म्॑) अ॒मृता॒(ग्म्॒) अनु॑ । \newline
14. अन्वित्यनु॑ । \newline
15. उ॒र्व॑न्तरि॑क्ष म॒न्तरि॑क्ष मु॒रू᳚(1॒)र्व॑न्तरि॑क्षम् । \newline
16. अ॒न्तरि॑क्ष॒ मन्वन्व॒न्तरि॑क्ष म॒न्तरि॑क्ष॒ मनु॑ । \newline
17. अन्वि॑ही॒ह्यन्वन्वि॑हि । \newline
18. इ॒ह्यदि॑त्या॒ अदि॑त्या इही॒ह्यदि॑त्याः । \newline
19. अदि॑त्याः॒ सदः॒ सदो ऽदि॑त्या॒ अदि॑त्याः॒ सदः॑ । \newline
20. सदो᳚ ऽस्यसि॒ सदः॒ सदो॑ ऽसि । \newline
21. अ॒स्यदि॑त्या॒ अदि॑त्या अस्य॒स्यदि॑त्याः । \newline
22. अदि॑त्याः॒ सदः॒ सदो ऽदि॑त्या॒ अदि॑त्याः॒ सदः॑ । \newline
23. सद॒ आ सदः॒ सद॒ आ । \newline
24. आ सी॑द सी॒दा सी॑द । \newline
25. सी॒दास्त॑भ्ना॒ दस्त॑भ्नाथ् सीद सी॒दास्त॑भ्नात् । \newline
26. अस्त॑भ्ना॒द् द्याम् द्या मस्त॑भ्ना॒ दस्त॑भ्ना॒द् द्याम् । \newline
27. द्या मृ॑ष॒भ ऋ॑ष॒भो द्याम् द्या मृ॑ष॒भः । \newline
28. ऋ॒ष॒भो अ॒न्तरि॑क्ष म॒न्तरि॑क्ष मृष॒भ ऋ॑ष॒भो अ॒न्तरि॑क्षम् । \newline
29. अ॒न्तरि॑क्ष॒ ममि॑मी॒ता मि॑मीता॒ न्तरि॑क्ष म॒न्तरि॑क्ष॒ ममि॑मीत । \newline
30. अमि॑मीत वरि॒माणं॑ ॅवरि॒माण॒ ममि॑मी॒तामि॑मीत वरि॒माण᳚म् । \newline
31. व॒रि॒माण॑म् पृथि॒व्याः पृ॑थि॒व्या व॑रि॒माणं॑ ॅवरि॒माण॑म् पृथि॒व्याः । \newline
32. पृ॒थि॒व्या आ पृ॑थि॒व्याः पृ॑थि॒व्या आ । \newline
33. आ ऽसी॑ददसीद॒दा ऽसी॑दत् । \newline
34. अ॒सी॒द॒द् विश्वा॒ विश्वा॑ ऽसीद दसीद॒द् विश्वा᳚ । \newline
35. विश्वा॒ भुव॑नानि॒ भुव॑नानि॒ विश्वा॒ विश्वा॒ भुव॑नानि । \newline
36. भुव॑नानि स॒म्राट्थ् स॒म्राड् भुव॑नानि॒ भुव॑नानि स॒म्राट् । \newline
37. स॒म्राड् विश्वा॒ विश्वा॑ स॒म्राट्थ् स॒म्राड् विश्वा᳚ । \newline
38. स॒म्राडिति॑ सं - राट् । \newline
39. विश्वेदिद् विश्वा॒ विश्वेत् । \newline
40. इत् तानि॒ तानीदित् तानि॑ । \newline
41. तानि॒ वरु॑णस्य॒ वरु॑णस्य॒ तानि॒ तानि॒ वरु॑णस्य । \newline
42. वरु॑णस्य व्र॒तानि॑ व्र॒तानि॒ वरु॑णस्य॒ वरु॑णस्य व्र॒तानि॑ । \newline
43. व्र॒तानि॒ वने॑षु॒ वने॑षु व्र॒तानि॑ व्र॒तानि॒ वने॑षु । \newline
44. वने॑षु॒ वि वि वने॑षु॒ वने॑षु॒ वि । \newline
45. व्य॑न्तरि॑क्ष म॒न्तरि॑क्षं॒ ॅवि व्य॑न्तरि॑क्षम् । \newline
46. अ॒न्तरि॑क्षम् ततान तताना॒न्तरि॑क्ष म॒न्तरि॑क्षम् ततान । \newline
47. त॒ता॒न॒ वाजं॒ ॅवाज॑म् ततान ततान॒ वाज᳚म् । \newline
48. वाज॒ मर्व॒थ्स्वर्व॑थ्सु॒ वाजं॒ ॅवाज॒ मर्व॑थ्सु । \newline
49. अर्व॑थ्सु॒ पयः॒ पयो ऽर्व॒थ्स्वर्व॑थ्सु॒ पयः॑ । \newline
50. अर्व॒थ्स्वित्यर्व॑त् - सु॒ । \newline
51. पयो॑ अघ्नि॒या स्व॑घ्नि॒यासु॒ पयः॒ पयो॑ अघ्नि॒यासु॑ । \newline
52. अ॒घ्नि॒यासु॑ हृ॒थ्सु हृ॒थ्स्व॑घ्नि॒या स्व॑घ्नि॒यासु॑ हृ॒थ्सु । \newline
53. हृ॒थ्सु क्रतु॒म् क्रतु(ग्म्॑) हृ॒थ्सु हृ॒थ्सु क्रतु᳚म् । \newline
54. हृ॒थ्स्विति॑ हृत् - सु । \newline

\textbf{Ghana Paata } \newline

1. उदायु॒षा ऽऽयु॒षोदुदायु॑षा स्वा॒युषा᳚ स्वा॒युषा ऽऽयु॒षोदुदायु॑षा स्वा॒युषा᳚ । \newline
2. आयु॑षा स्वा॒युषा᳚ स्वा॒युषा ऽऽयु॒षा ऽऽयु॑षा स्वा॒युषोदुथ् स्वा॒युषा ऽऽयु॒षा ऽऽयु॑षा स्वा॒युषोत् । \newline
3. स्वा॒युषोदुथ् स्वा॒युषा᳚ स्वा॒युषोदोष॑धीना॒ मोष॑धीना॒ मुथ् स्वा॒युषा᳚ स्वा॒युषोदोष॑धीनाम् । \newline
4. स्वा॒युषेति॑ सु - आ॒युषा᳚ । \newline
5. उदोष॑धीना॒ मोष॑धीना॒ मुदुदोष॑धीना॒(ग्म्॒) रसे॑न॒ रसे॒नौष॑धीना॒ मुदुदोष॑धीना॒(ग्म्॒) रसे॑न । \newline
6. ओष॑धीना॒(ग्म्॒) रसे॑न॒ रसे॒नौष॑धीना॒ मोष॑धीना॒(ग्म्॒) रसे॒नोदुद् रसे॒नौष॑धीना॒ मोष॑धीना॒(ग्म्॒) रसे॒नोत् । \newline
7. रसे॒नोदुद् रसे॑न॒ रसे॒नोत् प॒र्जन्य॑स्य प॒र्जन्य॒स्योद् रसे॑न॒ रसे॒नोत् प॒र्जन्य॑स्य । \newline
8. उत् प॒र्जन्य॑स्य प॒र्जन्य॒स्योदुत् प॒र्जन्य॑स्य॒ शुष्मे॑ण॒ शुष्मे॑ण प॒र्जन्य॒स्योदुत् प॒र्जन्य॑स्य॒ शुष्मे॑ण । \newline
9. प॒र्जन्य॑स्य॒ शुष्मे॑ण॒ शुष्मे॑ण प॒र्जन्य॑स्य प॒र्जन्य॑स्य॒ शुष्मे॒णोदु च्छुष्मे॑ण प॒र्जन्य॑स्य प॒र्जन्य॑स्य॒ शुष्मे॒णोत् । \newline
10. शुष्मे॒णोदु च्छुष्मे॑ण॒ शुष्मे॒णोद॑स्था मस्था॒ मुच्छुष्मे॑ण॒ शुष्मे॒णोद॑स्थाम् । \newline
11. उद॑स्था मस्था॒ मुदुद॑स्था म॒मृता(ग्म्॑) अ॒मृता(ग्म्॑) अस्था॒ मुदुद॑स्था म॒मृतान्॑ । \newline
12. अ॒स्था॒ म॒मृता(ग्म्॑) अ॒मृता(ग्म्॑) अस्था मस्था म॒मृता॒(ग्म्॒) अन्वन्व॒मृता(ग्म्॑) अस्था मस्था म॒मृता॒(ग्म्॒) अनु॑ । \newline
13. अ॒मृता॒(ग्म्॒) अन्वन्व॒मृता(ग्म्॑) अ॒मृता॒(ग्म्॒) अनु॑ । \newline
14. अन्वित्यनु॑ । \newline
15. उ॒र्व॑न्तरि॑क्ष म॒न्तरि॑क्ष मु॒रू᳚(1॒)र्व॑न्तरि॑क्ष॒ मन्वन्व॒न्तरि॑क्ष मु॒रू᳚(1॒)र्व॑न्तरि॑क्ष॒ मनु॑ । \newline
16. अ॒न्तरि॑क्ष॒ मन्वन्व॒न्तरि॑क्ष म॒न्तरि॑क्ष॒ मन्वि॑ही॒ह्यन्व॒न्तरि॑क्ष म॒न्तरि॑क्ष॒ मन्वि॑हि । \newline
17. अन्वि॑ही॒ ह्यन्वन्वि॒ ह्यदि॑त्या॒ अदि॑त्या इ॒ह्यन्वन्वि॒ ह्यदि॑त्याः । \newline
18. इ॒ह्यदि॑त्या॒ अदि॑त्या इही॒ह्यदि॑त्याः॒ सदः॒ सदो ऽदि॑त्या इही॒ह्यदि॑त्याः॒ सदः॑ । \newline
19. अदि॑त्याः॒ सदः॒ सदो ऽदि॑त्या॒ अदि॑त्याः॒ सदो᳚ ऽस्यसि॒ सदो ऽदि॑त्या॒ अदि॑त्याः॒ सदो॑ ऽसि । \newline
20. सदो᳚ ऽस्यसि॒ सदः॒ सदो॒ ऽस्यदि॑त्या॒ अदि॑त्या असि॒ सदः॒ सदो॒ ऽस्यदि॑त्याः । \newline
21. अ॒स्यदि॑त्या॒ अदि॑त्या अस्य॒स्यदि॑त्याः॒ सदः॒ सदो ऽदि॑त्या अस्य॒स्यदि॑त्याः॒ सदः॑ । \newline
22. अदि॑त्याः॒ सदः॒ सदो ऽदि॑त्या॒ अदि॑त्याः॒ सद॒ आ सदो ऽदि॑त्या॒ अदि॑त्याः॒ सद॒ आ । \newline
23. सद॒ आ सदः॒ सद॒ आ सी॑द सी॒दा सदः॒ सद॒ आ सी॑द । \newline
24. आ सी॑द सी॒दा सी॒दास्त॑भ्ना॒ दस्त॑भ्नाथ् सी॒दा सी॒दास्त॑भ्नात् । \newline
25. सी॒दास्त॑भ्ना॒ दस्त॑भ्नाथ् सीद सी॒दास्त॑भ्ना॒द् द्याम् द्या मस्त॑भ्नाथ् सीद सी॒दास्त॑भ्ना॒द् द्याम् । \newline
26. अस्त॑भ्ना॒द् द्याम् द्या मस्त॑भ्ना॒ दस्त॑भ्ना॒द् द्या मृ॑ष॒भ ऋ॑ष॒भो द्या मस्त॑भ्ना॒ दस्त॑भ्ना॒द् द्या मृ॑ष॒भः । \newline
27. द्या मृ॑ष॒भ ऋ॑ष॒भो द्याम् द्या मृ॑ष॒भो अ॒न्तरि॑क्ष म॒न्तरि॑क्ष मृष॒भो द्याम् द्या मृ॑ष॒भो अ॒न्तरि॑क्षम् । \newline
28. ऋ॒ष॒भो अ॒न्तरि॑क्ष म॒न्तरि॑क्ष मृष॒भ ऋ॑ष॒भो अ॒न्तरि॑क्ष॒ ममि॑मी॒तामि॑मीता॒न्तरि॑क्ष मृष॒भ ऋ॑ष॒भो अ॒न्तरि॑क्ष॒ ममि॑मीत । \newline
29. अ॒न्तरि॑क्ष॒ ममि॑मी॒तामि॑मीता॒न्तरि॑क्ष म॒न्तरि॑क्ष॒ ममि॑मीत वरि॒माणं॑ ॅवरि॒माण॒ ममि॑मीता॒न्तरि॑क्ष म॒न्तरि॑क्ष॒ ममि॑मीत वरि॒माण᳚म् । \newline
30. अमि॑मीत वरि॒माणं॑ ॅवरि॒माण॒ ममि॑मी॒तामि॑मीत वरि॒माण॑म् पृथि॒व्याः पृ॑थि॒व्या व॑रि॒माण॒ ममि॑मी॒तामि॑मीत वरि॒माण॑म् पृथि॒व्याः । \newline
31. व॒रि॒माण॑म् पृथि॒व्याः पृ॑थि॒व्या व॑रि॒माणं॑ ॅवरि॒माण॑म् पृथि॒व्या आ पृ॑थि॒व्या व॑रि॒माणं॑ ॅवरि॒माण॑म् पृथि॒व्या आ । \newline
32. पृ॒थि॒व्या आ पृ॑थि॒व्याः पृ॑थि॒व्या आ ऽसी॑द दसीद॒ दा पृ॑थि॒व्याः पृ॑थि॒व्या आ ऽसी॑दत् । \newline
33. आ ऽसी॑द दसीद॒ दा ऽसी॑द॒द् विश्वा॒ विश्वा॑ ऽसीद॒ दा ऽसी॑द॒द् विश्वा᳚ । \newline
34. अ॒सी॒द॒द् विश्वा॒ विश्वा॑ ऽसीद दसीद॒द् विश्वा॒ भुव॑नानि॒ भुव॑नानि॒ विश्वा॑ ऽसीद दसीद॒द् विश्वा॒ भुव॑नानि । \newline
35. विश्वा॒ भुव॑नानि॒ भुव॑नानि॒ विश्वा॒ विश्वा॒ भुव॑नानि स॒म्राट् थ्स॒म्राड् भुव॑नानि॒ विश्वा॒ विश्वा॒ भुव॑नानि स॒म्राट् । \newline
36. भुव॑नानि स॒म्राट् थ्स॒म्राड् भुव॑नानि॒ भुव॑नानि स॒म्राड् विश्वा॒ विश्वा॑ स॒म्राड् भुव॑नानि॒ भुव॑नानि स॒म्राड् विश्वा᳚ । \newline
37. स॒म्राड् विश्वा॒ विश्वा॑ स॒म्राट् थ्स॒म्राड् विश्वेदिद् विश्वा॑ स॒म्राट् थ्स॒म्राड् विश्वेत् । \newline
38. स॒म्राडिति॑ सं - राट् । \newline
39. विश्वेदिद् विश्वा॒ विश्वेत् तानि॒ तानीद् विश्वा॒ विश्वेत् तानि॑ । \newline
40. इत् तानि॒ तानीदित् तानि॒ वरु॑णस्य॒ वरु॑णस्य॒ तानीदित् तानि॒ वरु॑णस्य । \newline
41. तानि॒ वरु॑णस्य॒ वरु॑णस्य॒ तानि॒ तानि॒ वरु॑णस्य व्र॒तानि॑ व्र॒तानि॒ वरु॑णस्य॒ तानि॒ तानि॒ वरु॑णस्य व्र॒तानि॑ । \newline
42. वरु॑णस्य व्र॒तानि॑ व्र॒तानि॒ वरु॑णस्य॒ वरु॑णस्य व्र॒तानि॒ वने॑षु॒ वने॑षु व्र॒तानि॒ वरु॑णस्य॒ वरु॑णस्य व्र॒तानि॒ वने॑षु । \newline
43. व्र॒तानि॒ वने॑षु॒ वने॑षु व्र॒तानि॑ व्र॒तानि॒ वने॑षु॒ वि वि वने॑षु व्र॒तानि॑ व्र॒तानि॒ वने॑षु॒ वि । \newline
44. वने॑षु॒ वि वि वने॑षु॒ वने॑षु व्य॑न्तरि॑क्ष म॒न्तरि॑क्षं॒ ॅवि वने॑षु॒ वने॑षु व्य॑न्तरि॑क्षम् । \newline
45. व्य॑न्तरि॑क्ष म॒न्तरि॑क्षं॒ ॅवि व्य॑न्तरि॑क्षम् ततान तताना॒न्तरि॑क्षं॒ ॅवि व्य॑न्तरि॑क्षम् ततान । \newline
46. अ॒न्तरि॑क्षम् ततान तताना॒न्तरि॑क्ष म॒न्तरि॑क्षम् ततान॒ वाजं॒ ॅवाज॑म् तताना॒न्तरि॑क्ष म॒न्तरि॑क्षम् ततान॒ वाज᳚म् । \newline
47. त॒ता॒न॒ वाजं॒ ॅवाज॑म् ततान ततान॒ वाज॒ मर्व॒थ्स्वर्व॑थ्सु॒ वाज॑म् ततान ततान॒ वाज॒ मर्व॑थ्सु । \newline
48. वाज॒ मर्व॒थ्स्वर्व॑थ्सु॒ वाजं॒ ॅवाज॒ मर्व॑थ्सु॒ पयः॒ पयो ऽर्व॑थ्सु॒ वाजं॒ ॅवाज॒ मर्व॑थ्सु॒ पयः॑ । \newline
49. अर्व॑थ्सु॒ पयः॒ पयो ऽर्व॒थ्स्वर्व॑थ्सु॒ पयो॑ अघ्नि॒यास्व॑घ्नि॒यासु॒ पयो ऽर्व॒थ्स्वर्व॑थ्सु॒ पयो॑ अघ्नि॒यासु॑ । \newline
50. अर्व॒थ्स्वित्यर्व॑त् - सु॒ । \newline
51. पयो॑ अघ्नि॒यास्व॑घ्नि॒यासु॒ पयः॒ पयो॑ अघ्नि॒यासु॑ हृ॒थ्सु हृ॒थ्स्व॑घ्नि॒यासु॒ पयः॒ पयो॑ अघ्नि॒यासु॑ हृ॒थ्सु । \newline
52. अ॒घ्नि॒यासु॑ हृ॒थ्सु हृ॒थ्स्व॑घ्नि॒या स्व॑घ्नि॒यासु॑ हृ॒थ्सु क्रतु॒म् क्रतु(ग्म्॑) हृ॒थ्स्व॑घ्नि॒या स्व॑घ्नि॒यासु॑ हृ॒थ्सु क्रतु᳚म् । \newline
53. हृ॒थ्सु क्रतु॒म् क्रतु(ग्म्॑) हृ॒थ्सु हृ॒थ्सु क्रतुं॒ ॅवरु॑णो॒ वरु॑णः॒ क्रतु(ग्म्॑) हृ॒थ्सु हृ॒थ्सु क्रतुं॒ ॅवरु॑णः । \newline
54. हृ॒थ्स्विति॑ हृत् - सु । \newline
\pagebreak
\markright{ TS 1.2.8.2  \hfill https://www.vedavms.in \hfill}

\section{ TS 1.2.8.2 }

\textbf{TS 1.2.8.2 } \newline
\textbf{Samhita Paata} \newline

क्रतुं॒ ॅवरु॑णो वि॒क्ष्व॑ग्निं दि॒वि सूर्य॑मदधा॒थ् सोम॒मद्रा॒वुदु॒त्यं जा॒तवे॑दसं दे॒वं ॅव॑हन्ति के॒तवः॑ । दृ॒शे विश्वा॑य॒ सूर्यं᳚ ॥ उस्रा॒वेतं॑ धूर्.षाहावन॒श्रू अवी॑रहणौ ब्रह्म॒चोद॑नौ॒ वरु॑णस्य॒ स्कंभ॑नमसि॒ वरु॑णस्य स्कंभ॒सर्ज॑नमसि॒ प्रत्य॑स्तो॒ वरु॑णस्य॒ पाशः॑ ॥ \newline

\textbf{Pada Paata} \newline

क्रतु᳚म् । वरु॑णः । वि॒क्षु । अ॒ग्निम् । दि॒वि । सूर्य᳚म् । अ॒द॒धा॒त् । सोम᳚म् । अद्रौ᳚ । उदिति॑ । उ॒ । त्यम् । जा॒तवे॑दस॒मिति॑ जा॒त - वे॒द॒स॒म् । दे॒वम् । व॒ह॒न्ति॒ । के॒तवः॑ ॥ दृ॒शे । विश्वा॑य । सूर्य᳚म् ॥ उस्रौ᳚ । एति॑ । इ॒त॒म् । धू॒र्.॒षा॒हा॒विति॑ धूः - सा॒हौ॒ । अ॒न॒श्रू इति॑ । अवी॑रहणा॒वित्यवी॑र - ह॒नौ॒ । ब्र॒ह्म॒चोद॑ना॒विति॑ ब्रह्म - चोद॑नौ । वरु॑णस्य । स्कंभ॑नम् । अ॒सि॒ । वरु॑णस्य । स्कं॒भ॒सर्ज॑न॒मिति॑ स्कंभ -सर्ज॑नम् । अ॒सि॒ । प्रत्य॑स्त॒ इति॒ प्रति॑ - अ॒स्तः॒ । वरु॑णस्य । पाशः॑ ॥  \newline


\textbf{Krama Paata} \newline

क्रतुं॒ ॅवरु॑णः । वरु॑णो वि॒क्षु । वि॒क्ष्व॑ग्निम् । अ॒ग्निम् दि॒वि । दि॒वि सूर्य᳚म् । सूर्य॑मदधात् । अ॒द॒धा॒थ् सोम᳚म् । सोम॒मद्रौ᳚ । अद्रा॒वुत् । उदु॑ । उ॒ त्यम् । त्य॒म् जा॒तवे॑दसम् । जा॒तवे॑दसम् दे॒वम् । जा॒तवे॑दस॒मिति॑ जा॒त - वे॒द॒स॒म् । दे॒वं ॅव॑हन्ति । व॒ह॒न्ति॒ के॒तवः॑ । के॒तव॒ इति॑ के॒तवः॑ ॥ दृ॒शे विश्वा॑य । विश्वा॑य॒ सूर्य᳚म् । सूर्य॒मिति॒ सूर्य᳚म् ॥ उस्रा॒वा । एत᳚म् । इ॒त॒म् धू॒र्॒.षा॒हौ॒ । धू॒र्॒.षा॒हा॒व॒न॒श्रू । धू॒र्॒.षा॒हा॒विति॑ धूः - सा॒हौ॒ । अ॒न॒श्रू अवी॑रहणौ । अ॒न॒श्रू इत्य॑न॒श्रू । अवी॑रहणौ ब्रह्म॒चोद॑नौ । अवी॑रहणा॒वित्यवी॑र - ह॒नौ॒ । ब्र॒ह्म॒चोद॑नौ॒ वरु॑णस्य । ब्र॒ह्म॒चोद॑ना॒विति॑ ब्रह्म - चोद॑नौ । वरु॑णस्य॒ स्कम्भ॑नम् । स्कम्भ॑नमसि । अ॒सि॒ वरु॑णस्य । वरु॑णस्य स्कम्भ॒सर्ज॑नम् । स्क॒म्भ॒सर्ज॑नमसि । स्क॒म्भ॒सर्ज॑न॒मिति॑ स्कम्भ - सर्ज॑नम् । अ॒सि॒ प्रत्य॑स्तः । प्रत्य॑स्तो॒ वरु॑णस्य । प्रत्य॑स्त॒ इति॒ प्रति॑ - अ॒स्तः॒ । वरु॑णस्य॒ पाशः॑ । पाश॒ इति॒ पाशः॑ । \newline

\textbf{Jatai Paata} \newline

1. क्रतुं॒ ॅवरु॑णो॒ वरु॑णः॒ क्रतु॒म् क्रतुं॒ ॅवरु॑णः । \newline
2. वरु॑णो वि॒क्षु वि॒क्षु वरु॑णो॒ वरु॑णो वि॒क्षु । \newline
3. वि॒क्ष्व॑ग्नि म॒ग्निं ॅवि॒क्षु वि॒क्ष्व॑ग्निम् । \newline
4. अ॒ग्निम् दि॒वि दि॒व्य॑ग्नि म॒ग्निम् दि॒वि । \newline
5. दि॒वि सूर्य॒(ग्म्॒) सूर्य॑म् दि॒वि दि॒वि सूर्य᳚म् । \newline
6. सूर्य॑ मदधा ददधा॒थ् सूर्य॒(ग्म्॒) सूर्य॑ मदधात् । \newline
7. अ॒द॒धा॒थ् सोम॒(ग्म्॒) सोम॑ मदधा ददधा॒थ् सोम᳚म् । \newline
8. सोम॒ मद्रा॒ वद्रौ॒ सोम॒(ग्म्॒) सोम॒ मद्रौ᳚ । \newline
9. अद्रा॒ वुदु दद्रा॒ वद्रा॒ वुत् । \newline
10. उदु॑ वु॒ वुदुदु॑ । \newline
11. उ॒ त्यम् त्य मु॑ वु॒ त्यम् । \newline
12. त्यम् जा॒तवे॑दसम् जा॒तवे॑दस॒म् त्यम् त्यम् जा॒तवे॑दसम् । \newline
13. जा॒तवे॑दसम् दे॒वम् दे॒वम् जा॒तवे॑दसम् जा॒तवे॑दसम् दे॒वम् । \newline
14. जा॒तवे॑दस॒मिति॑ जा॒त - वे॒द॒स॒म् । \newline
15. दे॒वं ॅव॑हन्ति वहन्ति दे॒वम् दे॒वं ॅव॑हन्ति । \newline
16. व॒ह॒न्ति॒ के॒तवः॑ के॒तवो॑ वहन्ति वहन्ति के॒तवः॑ । \newline
17. के॒तव॒ इति॑ के॒तवः॑ । \newline
18. दृ॒शे विश्वा॑य॒ विश्वा॑य दृ॒शे दृ॒शे विश्वा॑य । \newline
19. विश्वा॑य॒ सूर्य॒(ग्म्॒) सूर्यं॒ ॅविश्वा॑य॒ विश्वा॑य॒ सूर्य᳚म् । \newline
20. सूर्य॒मिति॒ सूर्य᳚म् । \newline
21. उस्रा॒ वोस्रा॒ वुस्रा॒ वा । \newline
22. एत॑ मित॒ मेत᳚म् । \newline
23. इ॒त॒म् धू॒र्॒.षा॒हौ॒ धू॒र्॒.षा॒हा॒ वि॒त॒ मि॒त॒म् धू॒र्॒.षा॒हौ॒ । \newline
24. धू॒र्॒.षा॒हा॒ व॒न॒श्रू अ॑न॒श्रू धू॑र्.षाहौ धूर्.षाहा वन॒श्रू । \newline
25. धू॒र्॒.षा॒हा॒विति॑ धूः - सा॒हौ॒ । \newline
26. अ॒न॒श्रू अवी॑रहणा॒ ववी॑रहणा वन॒श्रू अ॑न॒श्रू अवी॑रहणौ । \newline
27. अ॒न॒श्रू इत्य॑न॒श्रू । \newline
28. अवी॑रहणौ ब्रह्म॒चोद॑नौ ब्रह्म॒चोद॑ना॒ ववी॑रहणा॒ ववी॑रहणौ ब्रह्म॒चोद॑नौ । \newline
29. अवी॑रहणा॒वित्यवी॑र - ह॒नौ॒ । \newline
30. ब्र॒ह्म॒चोद॑नौ॒ वरु॑णस्य॒ वरु॑णस्य ब्रह्म॒चोद॑नौ ब्रह्म॒चोद॑नौ॒ वरु॑णस्य । \newline
31. ब्र॒ह्म॒चोद॑ना॒विति॑ ब्रह्म - चोद॑नौ । \newline
32. वरु॑णस्य॒ स्कंभ॑न॒(ग्ग्॒) स्कंभ॑नं॒ ॅवरु॑णस्य॒ वरु॑णस्य॒ स्कंभ॑नम् । \newline
33. स्कंभ॑न मस्यसि॒ स्कंभ॑न॒(ग्ग्॒) स्कंभ॑न मसि । \newline
34. अ॒सि॒ वरु॑णस्य॒ वरु॑णस्यास्यसि॒ वरु॑णस्य । \newline
35. वरु॑णस्य स्कंभ॒सर्ज॑नꣳ स्कंभ॒सर्ज॑नं॒ ॅवरु॑णस्य॒ वरु॑णस्य स्कंभ॒सर्ज॑नम् । \newline
36. स्कं॒भ॒सर्ज॑न मस्यसि स्कंभ॒सर्ज॑नꣳ स्कंभ॒सर्ज॑न मसि । \newline
37. स्कं॒भ॒सर्ज॑न॒मिति॑ स्कंभ - सर्ज॑नम् । \newline
38. अ॒सि॒ प्रत्य॑स्तः॒ प्रत्य॑स्तो ऽस्यसि॒ प्रत्य॑स्तः । \newline
39. प्रत्य॑स्तो॒ वरु॑णस्य॒ वरु॑णस्य॒ प्रत्य॑स्तः॒ प्रत्य॑स्तो॒ वरु॑णस्य । \newline
40. प्रत्य॑स्त॒ इति॒ प्रति॑ - अ॒स्तः॒ । \newline
41. वरु॑णस्य॒ पाशः॒ पाशो॒ वरु॑णस्य॒ वरु॑णस्य॒ पाशः॑ । \newline
42. पाश॒ इति॒ पाशः॑ । \newline

\textbf{Ghana Paata } \newline

1. क्रतुं॒ ॅवरु॑णो॒ वरु॑णः॒ क्रतु॒म् क्रतुं॒ ॅवरु॑णो वि॒क्षु वि॒क्षु वरु॑णः॒ क्रतु॒म् क्रतुं॒ ॅवरु॑णो वि॒क्षु । \newline
2. वरु॑णो वि॒क्षु वि॒क्षु वरु॑णो॒ वरु॑णो वि॒क्ष्व॑ग्नि म॒ग्निं ॅवि॒क्षु वरु॑णो॒ वरु॑णो वि॒क्ष्व॑ग्निम् । \newline
3. वि॒क्ष्व॑ग्नि म॒ग्निं ॅवि॒क्षु वि॒क्ष्व॑ग्निम् दि॒वि दि॒व्य॑ग्निं ॅवि॒क्षु वि॒क्ष्व॑ग्निम् दि॒वि । \newline
4. अ॒ग्निम् दि॒वि दि॒व्य॑ग्नि म॒ग्निम् दि॒वि सूर्य॒(ग्म्॒) सूर्य॑म् दि॒व्य॑ग्नि म॒ग्निम् दि॒वि सूर्य᳚म् । \newline
5. दि॒वि सूर्य॒(ग्म्॒) सूर्य॑म् दि॒वि दि॒वि सूर्य॑ मदधा ददधा॒थ् सूर्य॑म् दि॒वि दि॒वि सूर्य॑ मदधात् । \newline
6. सूर्य॑ मदधाद दधा॒थ् सूर्य॒(ग्म्॒) सूर्य॑ मदधा॒थ् सोम॒(ग्म्॒) सोम॑ मदधा॒थ् सूर्य॒(ग्म्॒) सूर्य॑ मदधा॒थ् सोम᳚म् । \newline
7. अ॒द॒धा॒थ् सोम॒(ग्म्॒) सोम॑ मदधाद दधा॒थ् सोम॒ मद्रा॒ वद्रौ॒ सोम॑ मदधाद दधा॒थ् सोम॒ मद्रौ᳚ । \newline
8. सोम॒ मद्रा॒ वद्रौ॒ सोम॒(ग्म्॒) सोम॒ मद्रा॒ वुदुदद्रौ॒ सोम॒(ग्म्॒) सोम॒ मद्रा॒ वुत् । \newline
9. अद्रा॒ वुदुदद्रा॒ वद्रा॒ वुदु॑ वु॒ वुदद्रा॒ वद्रा॒ वुदु॑ । \newline
10. उदु॑ वु॒ वुदुदु॒ त्यम् त्य मु॒ वुदुदु॒ त्यम् । \newline
11. उ॒ त्यम् त्य मु॑ वु॒ त्यम् जा॒तवे॑दसम् जा॒तवे॑दस॒म् त्य मु॑ वु॒ त्यम् जा॒तवे॑दसम् । \newline
12. त्यम् जा॒तवे॑दसम् जा॒तवे॑दस॒म् त्यम् त्यम् जा॒तवे॑दसम् दे॒वम् दे॒वम् जा॒तवे॑दस॒म् त्यम् त्यम् जा॒तवे॑दसम् दे॒वम् । \newline
13. जा॒तवे॑दसम् दे॒वम् दे॒वम् जा॒तवे॑दसम् जा॒तवे॑दसम् दे॒वं ॅव॑हन्ति वहन्ति दे॒वम् जा॒तवे॑दसम् जा॒तवे॑दसम् दे॒वं ॅव॑हन्ति । \newline
14. जा॒तवे॑दस॒मिति॑ जा॒त - वे॒द॒स॒म् । \newline
15. दे॒वं ॅव॑हन्ति वहन्ति दे॒वम् दे॒वं ॅव॑हन्ति के॒तवः॑ के॒तवो॑ वहन्ति दे॒वम् दे॒वं ॅव॑हन्ति के॒तवः॑ । \newline
16. व॒ह॒न्ति॒ के॒तवः॑ के॒तवो॑ वहन्ति वहन्ति के॒तवः॑ । \newline
17. के॒तव॒ इति॑ के॒तवः॑ । \newline
18. दृ॒शे विश्वा॑य॒ विश्वा॑य दृ॒शे दृ॒शे विश्वा॑य॒ सूर्य॒(ग्म्॒) सूर्यं॒ ॅविश्वा॑य दृ॒शे दृ॒शे विश्वा॑य॒ सूर्य᳚म् । \newline
19. विश्वा॑य॒ सूर्य॒(ग्म्॒) सूर्यं॒ ॅविश्वा॑य॒ विश्वा॑य॒ सूर्य᳚म् । \newline
20. सूर्य॒मिति॒ सूर्य᳚म् । \newline
21. उस्रा॒ वोस्रा॒ वुस्रा॒ वेत॑ मित॒ मोस्रा॒ वुस्रा॒ वेत᳚म् । \newline
22. एत॑ मित॒म् एत॑म् धूर्.षाहौ धूर्.षाहा वित॒ मेत॑म् धूर्.षाहौ । \newline
23. इ॒त॒म् धू॒र्॒.षा॒हौ॒ धू॒र्॒.षा॒हा॒ वि॒त॒ मि॒त॒म् धू॒र्॒.षा॒हा॒ व॒न॒श्रू अ॑न॒श्रू धू॑र्.षाहा वित मितम् धूर्.षाहा वन॒श्रू । \newline
24. धू॒र्॒.षा॒हा॒ व॒न॒श्रू अ॑न॒श्रू धू॑र्.षाहौ धूर्.षाहा वन॒श्रू अवी॑रहणा॒ ववी॑रहणा वन॒श्रू धू॑र्.षाहौ धूर्.षाहा वन॒श्रू अवी॑रहणौ । \newline
25. धू॒र्॒.षा॒हा॒विति॑ धूः - सा॒हौ॒ । \newline
26. अ॒न॒श्रू अवी॑रहणा॒ ववी॑रहणा वन॒श्रू अ॑न॒श्रू अवी॑रहणौ ब्रह्म॒चोद॑नौ ब्रह्म॒चोद॑ना॒ ववी॑रहणा वन॒श्रू अ॑न॒श्रू अवी॑रहणौ ब्रह्म॒चोद॑नौ । \newline
27. अ॒न॒श्रू इत्य॑न॒श्रू । \newline
28. अवी॑रहणौ ब्रह्म॒चोद॑नौ ब्रह्म॒चोद॑ना॒ ववी॑रहणा॒ ववी॑रहणौ ब्रह्म॒चोद॑नौ॒ वरु॑णस्य॒ वरु॑णस्य ब्रह्म॒चोद॑ना॒ ववी॑रहणा॒ ववी॑रहणौ ब्रह्म॒चोद॑नौ॒ वरु॑णस्य । \newline
29. अवी॑रहणा॒वित्यवी॑र - ह॒नौ॒ । \newline
30. ब्र॒ह्म॒चोद॑नौ॒ वरु॑णस्य॒ वरु॑णस्य ब्रह्म॒चोद॑नौ ब्रह्म॒चोद॑नौ॒ वरु॑णस्य॒ स्कंभ॑न॒(ग्ग्॒) स्कंभ॑नं॒ ॅवरु॑णस्य ब्रह्म॒चोद॑नौ ब्रह्म॒चोद॑नौ॒ वरु॑णस्य॒ स्कंभ॑नम् । \newline
31. ब्र॒ह्म॒चोद॑ना॒विति॑ ब्रह्म - चोद॑नौ । \newline
32. वरु॑णस्य॒ स्कंभ॑न॒(ग्ग्॒) स्कंभ॑नं॒ ॅवरु॑णस्य॒ वरु॑णस्य॒ स्कंभ॑न मस्यसि॒ स्कंभ॑नं॒ ॅवरु॑णस्य॒ वरु॑णस्य॒ स्कंभ॑न मसि । \newline
33. स्कंभ॑न मस्यसि॒ स्कंभ॑न॒(ग्ग्॒) स्कंभ॑न मसि॒ वरु॑णस्य॒ वरु॑णस्यासि॒ स्कंभ॑न॒(ग्ग्॒) स्कंभ॑न मसि॒ वरु॑णस्य । \newline
34. अ॒सि॒ वरु॑णस्य॒ वरु॑णस्यास्यसि॒ वरु॑णस्य स्कंभ॒सर्ज॑नꣳ स्कंभ॒सर्ज॑नं॒ ॅवरु॑णस्यास्यसि॒ वरु॑णस्य स्कंभ॒सर्ज॑नम् । \newline
35. वरु॑णस्य स्कंभ॒सर्ज॑नꣳ स्कंभ॒सर्ज॑नं॒ ॅवरु॑णस्य॒ वरु॑णस्य स्कंभ॒सर्ज॑न मस्यसि स्कंभ॒सर्ज॑नं॒ ॅवरु॑णस्य॒ वरु॑णस्य स्कंभ॒सर्ज॑न मसि । \newline
36. स्कं॒भ॒सर्ज॑न मस्यसि स्कंभ॒सर्ज॑नꣳ स्कंभ॒सर्ज॑न मसि॒ प्रत्य॑स्तः॒ प्रत्य॑स्तो ऽसि स्कंभ॒सर्ज॑नꣳ स्कंभ॒सर्ज॑न मसि॒ प्रत्य॑स्तः । \newline
37. स्कं॒भ॒सर्ज॑न॒मिति॑ स्कंभ - सर्ज॑नम् । \newline
38. अ॒सि॒ प्रत्य॑स्तः॒ प्रत्य॑स्तो ऽस्यसि॒ प्रत्य॑स्तो॒ वरु॑णस्य॒ वरु॑णस्य॒ प्रत्य॑स्तो ऽस्यसि॒ प्रत्य॑स्तो॒ वरु॑णस्य । \newline
39. प्रत्य॑स्तो॒ वरु॑णस्य॒ वरु॑णस्य॒ प्रत्य॑स्तः॒ प्रत्य॑स्तो॒ वरु॑णस्य॒ पाशः॒ पाशो॒ वरु॑णस्य॒ प्रत्य॑स्तः॒ प्रत्य॑स्तो॒ वरु॑णस्य॒ पाशः॑ । \newline
40. प्रत्य॑स्त॒ इति॒ प्रति॑ - अ॒स्तः॒ । \newline
41. वरु॑णस्य॒ पाशः॒ पाशो॒ वरु॑णस्य॒ वरु॑णस्य॒ पाशः॑ । \newline
42. पाश॒ इति॒ पाशः॑ । \newline
\pagebreak
\markright{ TS 1.2.9.1  \hfill https://www.vedavms.in \hfill}

\section{ TS 1.2.9.1 }

\textbf{TS 1.2.9.1 } \newline
\textbf{Samhita Paata} \newline

प्र च्य॑वस्व भुवस्पते॒ विश्वा᳚न्य॒भि धामा॑नि॒ मा त्वा॑ परिप॒री वि॑द॒न्मा त्वा॑ परिप॒न्थिनो॑ विद॒न्मा त्वा॒ वृका॑ अघा॒यवो॒ मा ग॑न्ध॒र्वो वि॒श्वाव॑सु॒रा द॑घच्छ्ये॒नो भू॒त्वा परा॑ पत॒ यज॑मानस्य नो गृ॒हे दे॒वैः सꣳ॑स्कृ॒तं ॅयज॑मानस्य स्व॒स्त्यय॑न्य॒स्यपि॒ पन्था॑मगस्महि स्वस्ति॒गा-म॑ने॒हसं॒ ॅयेन॒ विश्वाः॒ परि॒ द्विषो॑ वृ॒णक्ति॑ वि॒न्दते॒ वसु॒ नमो॑ मि॒त्रस्य॒ ( ) वरु॑णस्य॒ चक्ष॑से म॒हो दे॒वाय॒ तदृ॒तꣳ स॑पर्यत दूरे॒दृशे॑ दे॒वजा॑ताय के॒तवे॑ दि॒वस्पु॒त्राय॒ सूर्या॑य शꣳसत॒ वरु॑णस्य॒ स्कंभ॑नमसि॒ वरु॑णस्य स्कंभ॒सर्ज॑न-म॒स्युन्मु॑क्तो॒ वरु॑णस्य॒ पाशः॑ ॥ \newline

\textbf{Pada Paata} \newline

प्रेति॑ । च्य॒व॒स्व॒ । भु॒वः॒ । प॒ते॒ । विश्वा॑नि । अ॒भीति॑ । धामा॑नि । मा । त्वा॒ । प॒रि॒प॒रीति॑ परि - प॒री । वि॒द॒त् । मा । त्वा॒ । प॒रि॒प॒न्थिन॒ इति॑ परि - प॒न्थिनः॑ । वि॒द॒न्न् । मा । त्वा॒ । वृकाः᳚ । अ॒घा॒यव॒ इत्य॑घ - यवः॑ । मा । ग॒न्ध॒र्वः । वि॒श्वाव॑सु॒रिति॑ वि॒श्व - व॒सुः॒ । एति॑ । द॒घ॒त् । श्ये॒नः । भू॒त्वा । परेति॑ । प॒त॒ । यज॑मानस्य । नः॒ । गृ॒हे । दे॒वैः । सꣳ॒॒स्कृ॒तम् । यज॑मानस्य । स्व॒स्त्यय॒नीति॑ स्वस्ति - अय॑नी । अ॒सि॒ । अपीति॑ । पन्था᳚म् । अ॒ग॒स्म॒हि॒ । स्व॒स्ति॒गामिति॑ स्वस्ति - गाम् । अ॒ने॒हस᳚म् । येन॑ । विश्वाः᳚ । परीति॑ । द्विषः॑ । वृ॒णक्ति॑ । वि॒न्दते᳚ । वसु॑ । नमः॑ । मि॒त्रस्य॑ ( ) । वरु॑णस्य । चक्ष॑से । म॒हः । दे॒वाय॑ । तत् । ऋ॒तम् । स॒प॒र्य॒त॒ । दू॒रे॒दृश॒ इति॑ दूरे - दृशे᳚ । दे॒वजा॑ता॒येति॑ दे॒व - जा॒ता॒य॒ । के॒तवे᳚ । दि॒वः । पु॒त्राय॑ । सूर्या॑य । शꣳ॒॒स॒त॒ । वरु॑णस्य । स्कंभ॑नम् । अ॒सि॒ । वरु॑णस्य । स्कं॒भ॒सर्ज॑न॒मिति॑ स्कंभ - सर्ज॑नम् । अ॒सि॒ । उन्मु॑क्त॒ इत्युत् - मु॒क्तः॒ । वरु॑णस्य । पाशः॑ ॥  \newline


\textbf{Krama Paata} \newline

प्र च्य॑वस्व । च्य॒व॒स्व॒ भु॒वः॒ । भु॒व॒स्प॒ते॒ । प॒ते॒ विश्वा॑नि । विश्वा᳚न्य॒भि । अ॒भि धामा॑नि । धामा॑नि॒ मा । मा त्वा᳚ । त्वा॒ प॒रि॒प॒री । प॒रि॒प॒री वि॑दत् । प॒रि॒प॒रीति॑ परि - प॒री । वि॒द॒न्मा । मा त्वा᳚ । त्वा॒ प॒रि॒प॒न्थिनः॑ । प॒रि॒प॒न्थिनो॑ विदन्न् । प॒रि॒प॒न्थिन॒ इति॑ परि - प॒न्थिनः॑ । वि॒द॒न्मा । मा त्वा᳚ । त्वा॒ वृकाः᳚ । वृका॑ अघा॒यवः॑ । अ॒घा॒यवो॒ मा । अ॒घा॒यव॒ इत्य॑घ - यवः॑ । मा ग॑न्ध॒र्वः । ग॒न्ध॒र्वो वि॒श्वाव॑सुः । वि॒श्वाव॑सु॒रा । वि॒श्वाव॑सु॒रिति॑ वि॒श्व - व॒सुः॒ । आ द॑घत् । द॒घ॒च्छ्ये॒नः । श्ये॒नो भू॒त्वा । भू॒त्वा परा᳚ । परा॑ पत । प॒त॒ यज॑मानस्य । यज॑मानस्य नः । नो॒ गृ॒हे । गृ॒हे दे॒वैः । दे॒वैः सꣳ॑स्कृ॒तम् । सꣳ॒॒स्कृ॒तं ॅयज॑मानस्य । यज॑मानस्य स्व॒स्त्यय॑नी । स्व॒स्त्यय॑न्यसि । स्व॒स्त्यय॒नीति॑ स्वस्ति - अय॑नी । अ॒स्यपि॑ । अपि॒ पन्था᳚म् । पन्था॑मगस्महि । अ॒ग॒स्म॒हि॒ स्व॒स्ति॒गाम् । स्व॒स्ति॒गाम॑ने॒हस᳚म् । स्व॒स्ति॒गामिति॑ स्वस्ति - गाम् । अ॒ने॒हसं॒ ॅयेन॑ । येन॒ विश्वाः᳚ । विश्वाः॒ परि॑ । परि॒ द्विषः॑ । द्विषो॑ वृ॒णक्ति॑ । वृ॒णक्ति॑ वि॒न्दते᳚ । वि॒न्दते॒ वसु॑ । वसु॒ नमः॑ । नमो॑ मि॒त्रस्य॑ ( ) । मि॒त्रस्य॒ वरु॑णस्य । वरु॑णस्य॒ चक्ष॑से । चक्ष॑से म॒हः । म॒हो दे॒वाय॑ । दे॒वाय॒ तत् । तदृ॒तम् । ऋ॒तꣳ स॑पर्यत । स॒प॒र्य॒त॒ दू॒रे॒दृशे᳚ । दू॒रे॒दृशे॑ दे॒वजा॑ताय । दू॒रे॒दृश॒ इति॑ दूरे - दृशे᳚ । दे॒वजा॑ताय के॒तवे᳚ । दे॒वजा॑ता॒येति॑ दे॒व - जा॒ता॒य॒ । के॒तवे॑ दि॒वः । दि॒वस्पु॒त्राय॑ । पु॒त्राय॒ सूर्या॑य । सूर्या॑य शꣳसत । शꣳ॒॒स॒त॒ वरु॑णस्य । वरु॑णस्य॒ स्कम्भ॑नम् । स्कम्भ॑नमसि । अ॒सि॒ वरु॑णस्य । वरु॑णस्य स्कम्भ॒सर्ज॑नम् । 
स्क॒म्भ॒सर्ज॑नमसि । स्क॒म्भ॒सर्ज॑न॒मिति॑ स्कम्भ - सर्ज॑नम् । अ॒स्युन्मु॑क्तः । उन्मु॑क्तो॒ वरु॑णस्य । उन्मु॑क्त॒ इत्युत् - मु॒क्तः॒ । वरु॑णस्य॒ पाशः॑ । पाश॒ इति॒ पाशः॑ । \newline

\textbf{Jatai Paata} \newline

1. प्र च्य॑वस्व च्यवस्व॒ प्र प्र च्य॑वस्व । \newline
2. च्य॒व॒स्व॒ भु॒वो॒ भु॒व॒श्च्य॒व॒स्व॒ च्य॒व॒स्व॒ भु॒वः॒ । \newline
3. भु॒व॒ स्प॒ते॒ प॒ते॒ भु॒वो॒ भु॒व॒ स्प॒ते॒ । \newline
4. प॒ते॒ विश्वा॑नि॒ विश्वा॑नि पते पते॒ विश्वा॑नि । \newline
5. विश्वा᳚ न्य॒भ्य॑भि विश्वा॑नि॒ विश्वा᳚न्य॒भि । \newline
6. अ॒भि धामा॑नि॒ धामा᳚ न्य॒भ्य॑भि धामा॑नि । \newline
7. धामा॑नि॒ मा मा धामा॑नि॒ धामा॑नि॒ मा । \newline
8. मा त्वा᳚ त्वा॒ मा मा त्वा᳚ । \newline
9. त्वा॒ प॒रि॒प॒री प॑रिप॒री त्वा᳚ त्वा परिप॒री । \newline
10. प॒रि॒प॒री वि॑दद् विदत् परिप॒री प॑रिप॒री वि॑दत् । \newline
11. प॒रि॒प॒रीति॑ परि - प॒री । \newline
12. वि॒द॒न् मा मा वि॑दद् विद॒न् मा । \newline
13. मा त्वा᳚ त्वा॒ मा मा त्वा᳚ । \newline
14. त्वा॒ प॒रि॒प॒न्थिनः॑ परिप॒न्थिन॑ स्त्वा त्वा परिप॒न्थिनः॑ । \newline
15. प॒रि॒प॒न्थिनो॑ विदन्. विदन् परिप॒न्थिनः॑ परिप॒न्थिनो॑ विदन्न् । \newline
16. प॒रि॒प॒न्थिन॒ इति॑ परि - प॒न्थिनः॑ । \newline
17. वि॒द॒न् मा मा वि॑दन्. विद॒न् मा । \newline
18. मा त्वा᳚ त्वा॒ मा मा त्वा᳚ । \newline
19. त्वा॒ वृका॒ वृका᳚ स्त्वा त्वा॒ वृकाः᳚ । \newline
20. वृका॑ अघा॒यवो॑ ऽघा॒यवो॒ वृका॒ वृका॑ अघा॒यवः॑ । \newline
21. अ॒घा॒यवो॒ मा मा ऽघा॒यवो॑ ऽघा॒यवो॒ मा । \newline
22. अ॒घा॒यव॒ इत्य॑घ - यवः॑ । \newline
23. मा ग॑न्ध॒र्वो ग॑न्ध॒र्वो मा मा ग॑न्ध॒र्वः । \newline
24. ग॒न्ध॒र्वो वि॒श्वाव॑सुर् वि॒श्वाव॑सुर् गन्ध॒र्वो ग॑न्ध॒र्वो वि॒श्वाव॑सुः । \newline
25. वि॒श्वाव॑सु॒रा वि॒श्वाव॑सुर् वि॒श्वाव॑सु॒रा । \newline
26. वि॒श्वाव॑सु॒रिति॑ वि॒श्व - व॒सुः॒ । \newline
27. आ द॑घद् दघ॒दा द॑घत् । \newline
28. द॒घ॒ च्छ्‌ये॒नः श्ये॒नो द॑घद् दघ च्छ्‌ये॒नः । \newline
29. श्ये॒नो भू॒त्वा भू॒त्वा श्ये॒नः श्ये॒नो भू॒त्वा । \newline
30. भू॒त्वा परा॒ परा॑ भू॒त्वा भू॒त्वा परा᳚ । \newline
31. परा॑ पत पत॒ परा॒ परा॑ पत । \newline
32. प॒त॒ यज॑मानस्य॒ यज॑मानस्य पत पत॒ यज॑मानस्य । \newline
33. यज॑मानस्य नो नो॒ यज॑मानस्य॒ यज॑मानस्य नः । \newline
34. नो॒ गृ॒हे गृ॒हे नो॑ नो गृ॒हे । \newline
35. गृ॒हे दे॒वैर् दे॒वैर् गृ॒हे गृ॒हे दे॒वैः । \newline
36. दे॒वैः स(ग्ग्॑)स्कृ॒तꣳ स(ग्ग्॑)स्कृ॒तम् दे॒वैर् दे॒वैः स(ग्ग्॑)स्कृ॒तम् । \newline
37. स॒(ग्ग्॒)स्कृ॒तं ॅयज॑मानस्य॒ यज॑मानस्य सꣳस्कृ॒तꣳ स(ग्ग्॑)स्कृ॒तं ॅयज॑मानस्य । \newline
38. यज॑मानस्य स्व॒स्त्य॑नी स्व॒स्त्य॑नी॒ यज॑मानस्य॒ यज॑मानस्य स्व॒स्त्य॑नी । \newline
39. स्व॒स्त्यय॑ न्यस्यसि स्व॒स्त्यय॑नी स्व॒स्त्यय॑ न्यसि । \newline
40. स्व॒स्त्यय॒नीति॑ स्वस्ति - अय॑नी । \newline
41. अ॒स्य प्य प्य॑ स्य॒ स्यपि॑ । \newline
42. अपि॒ पन्था॒म् पन्था॒ मप्यपि॒ पन्था᳚म् । \newline
43. पन्था॑ मगस्मह्यगस्महि॒ पन्था॒म् पन्था॑ मगस्महि । \newline
44. अ॒ग॒स्म॒हि॒ स्व॒स्ति॒गाꣳ स्व॑स्ति॒गा म॑गस्मह्यगस्महि स्वस्ति॒गाम् । \newline
45. स्व॒स्ति॒गा म॑ने॒हस॑ मने॒हस(ग्ग्॑) स्वस्ति॒गाꣳ स्व॑स्ति॒गा म॑ने॒हस᳚म् । \newline
46. स्व॒स्ति॒गामिति॑ स्वस्ति - गाम् । \newline
47. अ॒ने॒हसं॒ ॅयेन॒ येना॑ने॒हस॑ मने॒हसं॒ ॅयेन॑ । \newline
48. येन॒ विश्वा॒ विश्वा॒ येन॒ येन॒ विश्वाः᳚ । \newline
49. विश्वाः॒ परि॒ परि॒ विश्वा॒ विश्वाः॒ परि॑ । \newline
50. परि॒ द्विषो॒ द्विषः॒ परि॒ परि॒ द्विषः॑ । \newline
51. द्विषो॑ वृ॒णक्ति॑ वृ॒णक्ति॒ द्विषो॒ द्विषो॑ वृ॒णक्ति॑ । \newline
52. वृ॒णक्ति॑ वि॒न्दते॑ वि॒न्दते॑ वृ॒णक्ति॑ वृ॒णक्ति॑ वि॒न्दते᳚ । \newline
53. वि॒न्दते॒ वसु॒ वसु॑ वि॒न्दते॑ वि॒न्दते॒ वसु॑ । \newline
54. वसु॒ नमो॒ नमो॒ वसु॒ वसु॒ नमः॑ । \newline
55. नमो॑ मि॒त्रस्य॑ मि॒त्रस्य॒ नमो॒ नमो॑ मि॒त्रस्य॑ । \newline
56. मि॒त्रस्य॒ वरु॑णस्य॒ वरु॑णस्य मि॒त्रस्य॑ मि॒त्रस्य॒ वरु॑णस्य । \newline
57. वरु॑णस्य॒ चक्ष॑से॒ चक्ष॑से॒ वरु॑णस्य॒ वरु॑णस्य॒ चक्ष॑से । \newline
58. चक्ष॑से म॒हो म॒ह श्चक्ष॑से॒ चक्ष॑से म॒हः । \newline
59. म॒हो दे॒वाय॑ दे॒वाय॑ म॒हो म॒हो दे॒वाय॑ । \newline
60. दे॒वाय॒ तत् तद् दे॒वाय॑ दे॒वाय॒ तत् । \newline
61. तदृ॒त मृ॒तम् तत् तदृ॒तम् । \newline
62. ऋ॒तꣳ स॑पर्यत सपर्यत॒ र्त मृ॒तꣳ स॑पर्यत । \newline
63. स॒प॒र्य॒त॒ दू॒रे॒दृशे॑ दूरे॒दृशे॑ सपर्यत सपर्यत दूरे॒दृशे᳚ । \newline
64. दू॒रे॒दृशे॑ दे॒वजा॑ताय दे॒वजा॑ताय दूरे॒दृशे॑ दूरे॒दृशे॑ दे॒वजा॑ताय । \newline
65. दू॒रे॒दृश॒ इति॑ दूरे - दृशे᳚ । \newline
66. दे॒वजा॑ताय के॒तवे॑ के॒तवे॑ दे॒वजा॑ताय दे॒वजा॑ताय के॒तवे᳚ । \newline
67. दे॒वजा॑ता॒येति॑ दे॒व - जा॒ता॒य॒ । \newline
68. के॒तवे॑ दि॒वो दि॒वः के॒तवे॑ के॒तवे॑ दि॒वः । \newline
69. दि॒व स्पु॒त्राय॑ पु॒त्राय॑ दि॒वो दि॒व स्पु॒त्राय॑ । \newline
70. पु॒त्राय॒ सूर्या॑य॒ सूर्या॑य पु॒त्राय॑ पु॒त्राय॒ सूर्या॑य । \newline
71. सूर्या॑य शꣳसत शꣳसत॒ सूर्या॑य॒ सूर्या॑य शꣳसत । \newline
72. श॒(ग्म्॒)स॒त॒ वरु॑णस्य॒ वरु॑णस्य शꣳसत शꣳसत॒ वरु॑णस्य । \newline
73. वरु॑णस्य॒ स्कंभ॑न॒(ग्ग्॒) स्कंभ॑नं॒ ॅवरु॑णस्य॒ वरु॑णस्य॒ स्कंभ॑नम् । \newline
74. स्कंभ॑न मस्यसि॒ स्कंभ॑न॒(ग्ग्॒) स्कंभ॑न मसि । \newline
75. अ॒सि॒ वरु॑णस्य॒ वरु॑णस्यास्यसि॒ वरु॑णस्य । \newline
76. वरु॑णस्य स्कंभ॒सर्ज॑नꣳ स्कंभ॒सर्ज॑नं॒ ॅवरु॑णस्य॒ वरु॑णस्य स्कंभ॒सर्ज॑नम् । \newline
77. स्कं॒भ॒सर्ज॑न मस्यसि स्कंभ॒सर्ज॑नꣳ स्कंभ॒सर्ज॑न मसि । \newline
78. स्कं॒भ॒सर्ज॑न॒मिति॑ स्कंभ - सर्ज॑नम् । \newline
79. अ॒स्युन्मु॑क्त॒ उन्मु॑क्तो ऽस्य॒ स्युन्मु॑क्तः । \newline
80. उन्मु॑क्तो॒ वरु॑णस्य॒ वरु॑ण॒ स्योन्मु॑क्त॒ उन्मु॑क्तो॒ वरु॑णस्य । \newline
81. उन्मु॑क्त॒ इत्युत् - मु॒क्तः॒ । \newline
82. वरु॑णस्य॒ पाशः॒ पाशो॒ वरु॑णस्य॒ वरु॑णस्य॒ पाशः॑ । \newline
83. पाश॒ इति॒ पाशः॑ । \newline

\textbf{Ghana Paata } \newline

1. प्र च्य॑वस्व च्यवस्व॒ प्र प्र च्य॑वस्व भुवो भुव श्च्यवस्व॒ प्र प्र च्य॑वस्व भुवः । \newline
2. च्य॒व॒स्व॒ भु॒वो॒ भु॒व॒ श्च्य॒व॒स्व॒ च्य॒व॒स्व॒ भु॒व॒ स्प॒ते॒ प॒ते॒ भु॒व॒ श्च्य॒व॒स्व॒ च्य॒व॒स्व॒ भु॒व॒ स्प॒ते॒ । \newline
3. भु॒व॒ स्प॒ते॒ प॒ते॒ भु॒वो॒ भु॒व॒ स्प॒ते॒ विश्वा॑नि॒ विश्वा॑नि पते भुवो भुव स्पते॒ विश्वा॑नि । \newline
4. प॒ते॒ विश्वा॑नि॒ विश्वा॑नि पते पते॒ विश्वा᳚न्य॒भ्य॑भि विश्वा॑नि पते पते॒ विश्वा᳚न्य॒भि । \newline
5. विश्वा᳚न्य॒भ्य॑भि विश्वा॑नि॒ विश्वा᳚न्य॒भि धामा॑नि॒ धामा᳚न्य॒भि विश्वा॑नि॒ विश्वा᳚न्य॒भि धामा॑नि । \newline
6. अ॒भि धामा॑नि॒ धामा᳚न्य॒भ्य॑भि धामा॑नि॒ मा मा धामा᳚न्य॒भ्य॑भि धामा॑नि॒ मा । \newline
7. धामा॑नि॒ मा मा धामा॑नि॒ धामा॑नि॒ मा त्वा᳚ त्वा॒ मा धामा॑नि॒ धामा॑नि॒ मा त्वा᳚ । \newline
8. मा त्वा᳚ त्वा॒ मा मा त्वा॑ परिप॒री प॑रिप॒री त्वा॒ मा मा त्वा॑ परिप॒री । \newline
9. त्वा॒ प॒रि॒प॒री प॑रिप॒री त्वा᳚ त्वा परिप॒री वि॑दद् विदत् परिप॒री त्वा᳚ त्वा परिप॒री वि॑दत् । \newline
10. प॒रि॒प॒री वि॑दद् विदत् परिप॒री प॑रिप॒री वि॑द॒न् मा मा वि॑दत् परिप॒री प॑रिप॒री वि॑द॒न् मा । \newline
11. प॒रि॒प॒रीति॑ परि - प॒री । \newline
12. वि॒द॒न् मा मा वि॑दद् विद॒न् मा त्वा᳚ त्वा॒ मा वि॑दद् विद॒न् मा त्वा᳚ । \newline
13. मा त्वा᳚ त्वा॒ मा मा त्वा॑ परिप॒न्थिनः॑ परिप॒न्थिन॑स्त्वा॒ मा मा त्वा॑ परिप॒न्थिनः॑ । \newline
14. त्वा॒ प॒रि॒प॒न्थिनः॑ परिप॒न्थिन॑स्त्वा त्वा परिप॒न्थिनो॑ विदन्. विदन् परिप॒न्थिन॑स्त्वा त्वा परिप॒न्थिनो॑ विदन्न् । \newline
15. प॒रि॒प॒न्थिनो॑ विदन्. विदन् परिप॒न्थिनः॑ परिप॒न्थिनो॑ विद॒न् मा मा वि॑दन् परिप॒न्थिनः॑ परिप॒न्थिनो॑ विद॒न् मा । \newline
16. प॒रि॒प॒न्थिन॒ इति॑ परि - प॒न्थिनः॑ । \newline
17. वि॒द॒न् मा मा वि॑दन्. विद॒न् मा त्वा᳚ त्वा॒ मा वि॑दन्. विद॒न् मा त्वा᳚ । \newline
18. मा त्वा᳚ त्वा॒ मा मा त्वा॒ वृका॒ वृका᳚स्त्वा॒ मा मा त्वा॒ वृकाः᳚ । \newline
19. त्वा॒ वृका॒ वृका᳚स्त्वा त्वा॒ वृका॑ अघा॒यवो॑ ऽघा॒यवो॒ वृका᳚स्त्वा त्वा॒ वृका॑ अघा॒यवः॑ । \newline
20. वृका॑ अघा॒यवो॑ ऽघा॒यवो॒ वृका॒ वृका॑ अघा॒यवो॒ मा मा ऽघा॒यवो॒ वृका॒ वृका॑ अघा॒यवो॒ मा । \newline
21. अ॒घा॒यवो॒ मा मा ऽघा॒यवो॑ ऽघा॒यवो॒ मा ग॑न्ध॒र्वो ग॑न्ध॒र्वो मा ऽघा॒यवो॑ ऽघा॒यवो॒ मा ग॑न्ध॒र्वः । \newline
22. अ॒घा॒यव॒ इत्य॑घ - यवः॑ । \newline
23. मा ग॑न्ध॒र्वो ग॑न्ध॒र्वो मा मा ग॑न्ध॒र्वो वि॒श्वाव॑सुर् वि॒श्वाव॑सुर् गन्ध॒र्वो मा मा ग॑न्ध॒र्वो वि॒श्वाव॑सुः । \newline
24. ग॒न्ध॒र्वो वि॒श्वाव॑सुर् वि॒श्वाव॑सुर् गन्ध॒र्वो ग॑न्ध॒र्वो वि॒श्वाव॑सु॒रा वि॒श्वाव॑सुर् गन्ध॒र्वो ग॑न्ध॒र्वो वि॒श्वाव॑सु॒रा । \newline
25. वि॒श्वाव॑सु॒रा वि॒श्वाव॑सुर् वि॒श्वाव॑सु॒रा द॑घद् दघ॒दा वि॒श्वाव॑सुर् वि॒श्वाव॑सु॒रा द॑घत् । \newline
26. वि॒श्वाव॑सु॒रिति॑ वि॒श्व - व॒सुः॒ । \newline
27. आ द॑घद् दघ॒दा द॑घच्छ्‌ये॒नः श्ये॒नो द॑घ॒दा द॑घच्छ्‌ये॒नः । \newline
28. द॒घ॒च्छ्‌ये॒नः श्ये॒नो द॑घद् दघच्छ्‌ये॒नो भू॒त्वा भू॒त्वा श्ये॒नो द॑घद् दघच्छ्‌ये॒नो भू॒त्वा । \newline
29. श्ये॒नो भू॒त्वा भू॒त्वा श्ये॒नः श्ये॒नो भू॒त्वा परा॒ परा॑ भू॒त्वा श्ये॒नः श्ये॒नो भू॒त्वा परा᳚ । \newline
30. भू॒त्वा परा॒ परा॑ भू॒त्वा भू॒त्वा परा॑ पत पत॒ परा॑ भू॒त्वा भू॒त्वा परा॑ पत । \newline
31. परा॑ पत पत॒ परा॒ परा॑ पत॒ यज॑मानस्य॒ यज॑मानस्य पत॒ परा॒ परा॑ पत॒ यज॑मानस्य । \newline
32. प॒त॒ यज॑मानस्य॒ यज॑मानस्य पत पत॒ यज॑मानस्य नो नो॒ यज॑मानस्य पत पत॒ यज॑मानस्य नः । \newline
33. यज॑मानस्य नो नो॒ यज॑मानस्य॒ यज॑मानस्य नो गृ॒हे गृ॒हे नो॒ यज॑मानस्य॒ यज॑मानस्य नो गृ॒हे । \newline
34. नो॒ गृ॒हे गृ॒हे नो॑ नो गृ॒हे दे॒वैर् दे॒वैर् गृ॒हे नो॑ नो गृ॒हे दे॒वैः । \newline
35. गृ॒हे दे॒वैर् दे॒वैर् गृ॒हे गृ॒हे दे॒वैः स(ग्ग्॑)स्कृ॒तꣳ स(ग्ग्॑)स्कृ॒तम् दे॒वैर् गृ॒हे गृ॒हे दे॒वैः स(ग्ग्॑)स्कृ॒तम् । \newline
36. दे॒वैः स(ग्ग्॑)स्कृ॒तꣳ स(ग्ग्॑)स्कृ॒तम् दे॒वैर् दे॒वैः स(ग्ग्॑)स्कृ॒तं ॅयज॑मानस्य॒ यज॑मानस्य सꣳस्कृ॒तम् दे॒वैर् दे॒वैः स(ग्ग्॑)स्कृ॒तं ॅयज॑मानस्य । \newline
37. स॒(ग्ग्॒)स्कृ॒तं ॅयज॑मानस्य॒ यज॑मानस्य सꣳस्कृ॒तꣳ स(ग्ग्॑)स्कृ॒तं ॅयज॑मानस्य स्व॒स्त्यय॑नी स्व॒स्त्यय॑नी॒ यज॑मानस्य सꣳस्कृ॒तꣳ स(ग्ग्॑)स्कृ॒तं ॅयज॑मानस्य स्व॒स्त्यय॑नी । \newline
38. यज॑मानस्य स्व॒स्त्यय॑नी स्व॒स्त्यय॑नी॒ यज॑मानस्य॒ यज॑मानस्य स्व॒स्त्यय॑ न्यस्यसि स्व॒स्त्यय॑नी॒ यज॑मानस्य॒ यज॑मानस्य स्व॒स्त्यय॑ न्यसि । \newline
39. स्व॒स्त्यय॑ न्यस्यसि स्व॒स्त्यय॑नी स्व॒स्त्यय॑ न्य॒स्यप्यप्य॑सि स्व॒स्त्यय॑नी स्व॒स्त्यय॑ न्य॒स्यपि॑ । \newline
40. स्व॒स्त्यय॒नीति॑ स्वस्ति - अय॑नी । \newline
41. अ॒स्यप्य प्य॑स्य॒स्यपि॒ पन्था॒म् पन्था॒ मप्य॑स्य॒स्यपि॒ पन्था᳚म् । \newline
42. अपि॒ पन्था॒म् पन्था॒ मप्यपि॒ पन्था॑ मगस्मह्यगस्महि॒ पन्था॒ मप्यपि॒ पन्था॑ मगस्महि । \newline
43. पन्था॑ मगस्मह्यगस्महि॒ पन्था॒म् पन्था॑ मगस्महि स्वस्ति॒गाꣳ स्व॑स्ति॒गा म॑गस्महि॒ पन्था॒म् पन्था॑ मगस्महि स्वस्ति॒गाम् । \newline
44. अ॒ग॒स्म॒हि॒ स्व॒स्ति॒गाꣳ स्व॑स्ति॒गा म॑गस्मह्यगस्महि स्वस्ति॒गा म॑ने॒हस॑ मने॒हस(ग्ग्॑) स्वस्ति॒गा म॑गस्मह्यगस्महि स्वस्ति॒गा म॑ने॒हस᳚म् । \newline
45. स्व॒स्ति॒गा म॑ने॒हस॑ मने॒हस(ग्ग्॑) स्वस्ति॒गाꣳ स्व॑स्ति॒गा म॑ने॒हसं॒ ॅयेन॒ येना॑ने॒हस(ग्ग्॑) स्वस्ति॒गाꣳ स्व॑स्ति॒गा म॑ने॒हसं॒ ॅयेन॑ । \newline
46. स्व॒स्ति॒गामिति॑ स्वस्ति - गाम् । \newline
47. अ॒ने॒हसं॒ ॅयेन॒ येना॑ने॒हस॑ मने॒हसं॒ ॅयेन॒ विश्वा॒ विश्वा॒ येना॑ने॒हस॑ मने॒हसं॒ ॅयेन॒ विश्वाः᳚ । \newline
48. येन॒ विश्वा॒ विश्वा॒ येन॒ येन॒ विश्वाः॒ परि॒ परि॒ विश्वा॒ येन॒ येन॒ विश्वाः॒ परि॑ । \newline
49. विश्वाः॒ परि॒ परि॒ विश्वा॒ विश्वाः॒ परि॒ द्विषो॒ द्विषः॒ परि॒ विश्वा॒ विश्वाः॒ परि॒ द्विषः॑ । \newline
50. परि॒ द्विषो॒ द्विषः॒ परि॒ परि॒ द्विषो॑ वृ॒णक्ति॑ वृ॒णक्ति॒ द्विषः॒ परि॒ परि॒ द्विषो॑ वृ॒णक्ति॑ । \newline
51. द्विषो॑ वृ॒णक्ति॑ वृ॒णक्ति॒ द्विषो॒ द्विषो॑ वृ॒णक्ति॑ वि॒न्दते॑ वि॒न्दते॑ वृ॒णक्ति॒ द्विषो॒ द्विषो॑ वृ॒णक्ति॑ वि॒न्दते᳚ । \newline
52. वृ॒णक्ति॑ वि॒न्दते॑ वि॒न्दते॑ वृ॒णक्ति॑ वृ॒णक्ति॑ वि॒न्दते॒ वसु॒ वसु॑ वि॒न्दते॑ वृ॒णक्ति॑ वृ॒णक्ति॑ वि॒न्दते॒ वसु॑ । \newline
53. वि॒न्दते॒ वसु॒ वसु॑ वि॒न्दते॑ वि॒न्दते॒ वसु॒ नमो॒ नमो॒ वसु॑ वि॒न्दते॑ वि॒न्दते॒ वसु॒ नमः॑ । \newline
54. वसु॒ नमो॒ नमो॒ वसु॒ वसु॒ नमो॑ मि॒त्रस्य॑ मि॒त्रस्य॒ नमो॒ वसु॒ वसु॒ नमो॑ मि॒त्रस्य॑ । \newline
55. नमो॑ मि॒त्रस्य॑ मि॒त्रस्य॒ नमो॒ नमो॑ मि॒त्रस्य॒ वरु॑णस्य॒ वरु॑णस्य मि॒त्रस्य॒ नमो॒ नमो॑ मि॒त्रस्य॒ वरु॑णस्य । \newline
56. मि॒त्रस्य॒ वरु॑णस्य॒ वरु॑णस्य मि॒त्रस्य॑ मि॒त्रस्य॒ वरु॑णस्य॒ चक्ष॑से॒ चक्ष॑से॒ वरु॑णस्य मि॒त्रस्य॑ मि॒त्रस्य॒ वरु॑णस्य॒ चक्ष॑से । \newline
57. वरु॑णस्य॒ चक्ष॑से॒ चक्ष॑से॒ वरु॑णस्य॒ वरु॑णस्य॒ चक्ष॑से म॒हो म॒हश्चक्ष॑से॒ वरु॑णस्य॒ वरु॑णस्य॒ चक्ष॑से म॒हः । \newline
58. चक्ष॑से म॒हो म॒हश्चक्ष॑से॒ चक्ष॑से म॒हो दे॒वाय॑ दे॒वाय॑ म॒हश्चक्ष॑से॒ चक्ष॑से म॒हो दे॒वाय॑ । \newline
59. म॒हो दे॒वाय॑ दे॒वाय॑ म॒हो म॒हो दे॒वाय॒ तत् तद् दे॒वाय॑ म॒हो म॒हो दे॒वाय॒ तत् । \newline
60. दे॒वाय॒ तत् तद् दे॒वाय॑ दे॒वाय॒ तदृ॒त मृ॒तम् तद् दे॒वाय॑ दे॒वाय॒ तदृ॒तम् । \newline
61. तदृ॒त मृ॒तम् तत् तदृ॒तꣳ स॑पर्यत सपर्यत॒ र्तम् तत् तदृ॒तꣳ स॑पर्यत । \newline
62. ऋ॒तꣳ स॑पर्यत सपर्यत॒ र्त मृ॒तꣳ स॑पर्यत दूरे॒दृशे॑ दूरे॒दृशे॑ सपर्यत॒ र्त मृ॒तꣳ स॑पर्यत दूरे॒दृशे᳚ । \newline
63. स॒प॒र्य॒त॒ दू॒रे॒दृशे॑ दूरे॒दृशे॑ सपर्यत सपर्यत दूरे॒दृशे॑ दे॒वजा॑ताय दे॒वजा॑ताय दूरे॒दृशे॑ सपर्यत सपर्यत दूरे॒दृशे॑ दे॒वजा॑ताय । \newline
64. दू॒रे॒दृशे॑ दे॒वजा॑ताय दे॒वजा॑ताय दूरे॒दृशे॑ दूरे॒दृशे॑ दे॒वजा॑ताय के॒तवे॑ के॒तवे॑ दे॒वजा॑ताय दूरे॒दृशे॑ दूरे॒दृशे॑ दे॒वजा॑ताय के॒तवे᳚ । \newline
65. दू॒रे॒दृश॒ इति॑ दूरे - दृशे᳚ । \newline
66. दे॒वजा॑ताय के॒तवे॑ के॒तवे॑ दे॒वजा॑ताय दे॒वजा॑ताय के॒तवे॑ दि॒वो दि॒वः के॒तवे॑ दे॒वजा॑ताय दे॒वजा॑ताय के॒तवे॑ दि॒वः । \newline
67. दे॒वजा॑ता॒येति॑ दे॒व - जा॒ता॒य॒ । \newline
68. के॒तवे॑ दि॒वो दि॒वः के॒तवे॑ के॒तवे॑ दि॒व स्पु॒त्राय॑ पु॒त्राय॑ दि॒वः के॒तवे॑ के॒तवे॑ दि॒व स्पु॒त्राय॑ । \newline
69. दि॒व स्पु॒त्राय॑ पु॒त्राय॑ दि॒वो दि॒व स्पु॒त्राय॒ सूर्या॑य॒ सूर्या॑य पु॒त्राय॑ दि॒वो दि॒व स्पु॒त्राय॒ सूर्या॑य । \newline
70. पु॒त्राय॒ सूर्या॑य॒ सूर्या॑य पु॒त्राय॑ पु॒त्राय॒ सूर्या॑य शꣳसत शꣳसत॒ सूर्या॑य पु॒त्राय॑ पु॒त्राय॒ सूर्या॑य शꣳसत । \newline
71. सूर्या॑य शꣳसत शꣳसत॒ सूर्या॑य॒ सूर्या॑य शꣳसत॒ वरु॑णस्य॒ वरु॑णस्य शꣳसत॒ सूर्या॑य॒ सूर्या॑य शꣳसत॒ वरु॑णस्य । \newline
72. श॒(ग्म्॒)स॒त॒ वरु॑णस्य॒ वरु॑णस्य शꣳसत शꣳसत॒ वरु॑णस्य॒ स्कंभ॑न॒(ग्ग्॒) स्कंभ॑नं॒ ॅवरु॑णस्य शꣳसत शꣳसत॒ वरु॑णस्य॒ स्कंभ॑नम् । \newline
73. वरु॑णस्य॒ स्कंभ॑न॒(ग्ग्॒) स्कंभ॑नं॒ ॅवरु॑णस्य॒ वरु॑णस्य॒ स्कंभ॑न मस्यसि॒ स्कंभ॑नं॒ ॅवरु॑णस्य॒ वरु॑णस्य॒ स्कंभ॑न मसि । \newline
74. स्कंभ॑न मस्यसि॒ स्कंभ॑न॒(ग्ग्॒) स्कंभ॑न मसि॒ वरु॑णस्य॒ वरु॑णस्यासि॒ स्कंभ॑न॒(ग्ग्॒) स्कंभ॑न मसि॒ वरु॑णस्य । \newline
75. अ॒सि॒ वरु॑णस्य॒ वरु॑णस्यास्यसि॒ वरु॑णस्य स्कंभ॒सर्ज॑नꣳ स्कंभ॒सर्ज॑नं॒ ॅवरु॑णस्यास्यसि॒ वरु॑णस्य स्कंभ॒सर्ज॑नम् । \newline
76. वरु॑णस्य स्कंभ॒सर्ज॑नꣳ स्कंभ॒सर्ज॑नं॒ ॅवरु॑णस्य॒ वरु॑णस्य स्कंभ॒सर्ज॑न मस्यसि स्कंभ॒सर्ज॑नं॒ ॅवरु॑णस्य॒ वरु॑णस्य स्कंभ॒सर्ज॑न मसि । \newline
77. स्कं॒भ॒सर्ज॑न मस्यसि स्कंभ॒सर्ज॑नꣳ स्कंभ॒सर्ज॑न म॒स्युन्मु॑क्त॒ उन्मु॑क्तो ऽसि स्कंभ॒सर्ज॑नꣳ स्कंभ॒सर्ज॑न म॒स्युन्मु॑क्तः । \newline
78. स्कं॒भ॒सर्ज॑न॒मिति॑ स्कंभ - सर्ज॑नम् । \newline
79. अ॒स्युन्मु॑क्त॒ उन्मु॑क्तो ऽस्य॒स्युन्मु॑क्तो॒ वरु॑णस्य॒ वरु॑ण॒स्योन्मु॑क्तो ऽस्य॒स्युन्मु॑क्तो॒ वरु॑णस्य । \newline
80. उन्मु॑क्तो॒ वरु॑णस्य॒ वरु॑ण॒स्योन्मु॑क्त॒ उन्मु॑क्तो॒ वरु॑णस्य॒ पाशः॒ पाशो॒ वरु॑ण॒स्योन्मु॑क्त॒ उन्मु॑क्तो॒ वरु॑णस्य॒ पाशः॑ । \newline
81. उन्मु॑क्त॒ इत्युत् - मु॒क्तः॒ । \newline
82. वरु॑णस्य॒ पाशः॒ पाशो॒ वरु॑णस्य॒ वरु॑णस्य॒ पाशः॑ । \newline
83. पाश॒ इति॒ पाशः॑ । \newline
\pagebreak
\markright{ TS 1.2.10.1  \hfill https://www.vedavms.in \hfill}

\section{ TS 1.2.10.1 }

\textbf{TS 1.2.10.1 } \newline
\textbf{Samhita Paata} \newline

अ॒ग्ने-रा॑ति॒थ्यम॑सि॒ विष्ण॑वे त्वा॒ सोम॑स्याऽऽ*ति॒थ्यम॑सि॒ विष्ण॑वे॒ त्वा-ऽति॑थेराति॒थ्यम॑सि॒ विष्ण॑वे त्वा॒ऽग्नये᳚ त्वा रायस्पोष॒दाव्.न्ने॒ विष्ण॑वे त्वा श्ये॒नाय॑ त्वा सोम॒भृते॒ विष्ण॑वे त्वा॒ या ते॒ धामा॑नि ह॒विषा॒ यज॑न्ति॒ ता ते॒ विश्वा॑ परि॒भूर॑स्तु य॒ज्ञ्ं ग॑य॒स्फानः॑ प्र॒तर॑णः सु॒वीरोऽवी॑रहा॒ प्रच॑रा सोम॒ दुर्या॒नदि॑त्याः॒ सदो॒ऽस्यदि॑त्याः॒ सद॒ आ-[ ] \newline

\textbf{Pada Paata} \newline

अ॒ग्नेः । आ॒ति॒थ्यम् । अ॒सि॒ । विष्ण॑वे । त्वा॒ । सोम॑स्य । आ॒ति॒थ्यम् । अ॒सि॒ । विष्ण॑वे । त्वा॒ । अति॑थेः । आ॒ति॒थ्यम् । अ॒सि॒ । विष्ण॑वे । त्वा॒ । अ॒ग्नये᳚ । त्वा॒ । रा॒य॒स्पो॒ष॒दाव्‌न्न॒ इति॑ रायस्पोष - दाव्‌न्ने᳚ । विष्ण॑वे । त्वा॒ । श्ये॒नाय॑ । त्वा॒ । सो॒म॒भृत॒ इति॑ सोम - भृते᳚ । विष्ण॑वे । त्वा॒ । या । ते॒ । धामा॑नि । ह॒विषा᳚ । यज॑न्ति । ता । ते॒ । विश्वा᳚ । प॒रि॒भूरिति॑ परि - भूः । अ॒स्तु॒ । य॒ज्ञ्म् । ग॒य॒स्फान॒ इति॑ गय - स्फानः॑ । प्र॒तर॑ण॒ इति॑ प्र - तर॑णः । सु॒वीर॒ इति॑ सु - वीरः॑ । अवी॑र॒हेत्यवी॑र - हा॒ । प्रेति॑ । च॒र ॒। सो॒म॒ । दुर्यान्॑ । अदि॑त्याः । सदः॑ । अ॒सि॒ । अदि॑त्याः । सदः॑ । एति॑ ।  \newline


\textbf{Krama Paata} \newline

अ॒ग्नेरा॑ति॒थ्यम् । आ॒ति॒थ्यम॑सि । अ॒सि॒ विष्ण॑वे । विष्ण॑वे त्वा । त्वा॒ सोम॑स्य । सोम॑स्याति॒थ्यम् । आ॒ति॒थ्यम॑सि । अ॒सि॒ विष्ण॑वे । विष्ण॑वे त्वा । त्वाऽति॑थेः । अति॑थेराति॒थ्यम् । आ॒ति॒थ्यम॑सि । अ॒सि॒ विष्ण॑वे । विष्ण॑वे त्वा । त्वा॒ऽग्नये᳚ । अ॒ग्नये᳚ त्वा । त्वा॒ रा॒य॒स्पो॒ष॒दाव्.न्ने᳚ । रा॒य॒स्पो॒ष॒दाव्.न्ने॒ विष्ण॑वे । रा॒य॒स्पो॒ष॒दाव्.न्न॒ इति॑ रायस्पोष - दाव्.न्ने᳚ । विष्ण॑वे त्वा । त्वा॒ श्ये॒नाय॑ । श्ये॒नाय॑ त्वा । त्वा॒ सो॒म॒भृते᳚ । सो॒म॒भृते॒ विष्ण॑वे । सो॒म॒भृत॒ इति॑ सोम - भृते᳚ । विष्ण॑वे त्वा । त्वा॒ या । या ते᳚ । ते॒ धामा॑नि । धामा॑नि ह॒विषा᳚ । ह॒विषा॒ यज॑न्ति । यज॑न्ति॒ ता । ता ते᳚ । ते॒ विश्वा᳚ । विश्वा॑ परि॒भूः । प॒रि॒भूर॑स्तु । प॒रि॒भूरिति॑ परि - भूः । अ॒स्तु॒ य॒ज्ञ्म् । य॒ज्ञ्ं ग॑य॒स्फानः॑ । ग॒य॒स्फानः॑ प्र॒तर॑णः । ग॒य॒स्फान॒ इति॑ गय - स्फानः॑ । प्र॒तर॑णः सु॒वीरः॑ । प्र॒तर॑ण॒ इति॑ प्र - तर॑णः । सु॒वीरोऽवी॑रहा । सु॒वीर॒ इति॑ सु - वीरः॑ । अवी॑रहा॒ प्र । अवी॑र॒हेत्यवी॑र - हा॒ । प्र च॑र । च॒रा॒ सो॒म॒ । सो॒म॒ दुर्यान्॑ । दुर्या॒नदि॑त्याः । अदि॑त्याः॒ सदः॑ । सदो॑ऽसि । अ॒स्यदि॑त्याः । अदि॑त्याः॒ सदः॑ । सद॒ आ । आ सी॑द \newline

\textbf{Jatai Paata} \newline

1. अ॒ग्ने रा॑ति॒थ्य मा॑ति॒थ्य म॒ग्ने र॒ग्ने रा॑ति॒थ्यम् । \newline
2. आ॒ति॒थ्य म॑स्यस्याति॒थ्य मा॑ति॒थ्य म॑सि । \newline
3. अ॒सि॒ विष्ण॑वे॒ विष्ण॑वे ऽस्यसि॒ विष्ण॑वे । \newline
4. विष्ण॑वे त्वा त्वा॒ विष्ण॑वे॒ विष्ण॑वे त्वा । \newline
5. त्वा॒ सोम॑स्य॒ सोम॑स्य त्वा त्वा॒ सोम॑स्य । \newline
6. सोम॑स्याति॒थ्य मा॑ति॒थ्यꣳ सोम॑स्य॒ सोम॑स्याति॒थ्यम् । \newline
7. आ॒ति॒थ्य म॑स्यस्याति॒थ्य मा॑ति॒थ्य म॑सि । \newline
8. अ॒सि॒ विष्ण॑वे॒ विष्ण॑वे ऽस्यसि॒ विष्ण॑वे । \newline
9. विष्ण॑वे त्वा त्वा॒ विष्ण॑वे॒ विष्ण॑वे त्वा । \newline
10. त्वा ऽति॑थे॒ रति॑थे स्त्वा॒ त्वा ऽति॑थेः । \newline
11. अति॑थे राति॒थ्य मा॑ति॒थ्य मति॑थे॒ रति॑थे राति॒थ्यम् । \newline
12. आ॒ति॒थ्य म॑स्यस्याति॒थ्य मा॑ति॒थ्य म॑सि । \newline
13. अ॒सि॒ विष्ण॑वे॒ विष्ण॑वे ऽस्यसि॒ विष्ण॑वे । \newline
14. विष्ण॑वे त्वा त्वा॒ विष्ण॑वे॒ विष्ण॑वे त्वा । \newline
15. त्वा॒ ऽग्नये॒ ऽग्नये᳚ त्वा त्वा॒ ऽग्नये᳚ । \newline
16. अ॒ग्नये᳚ त्वा त्वा॒ ऽग्नये॒ ऽग्नये᳚ त्वा । \newline
17. त्वा॒ रा॒य॒स्पो॒ष॒दाव्.न्ने॑ रायस्पोष॒दाव्.न्ने᳚ त्वा त्वा रायस्पोष॒दाव्.न्ने᳚ । \newline
18. रा॒य॒स्पो॒ष॒दाव्.न्ने॒ विष्ण॑वे॒ विष्ण॑वे रायस्पोष॒दाव्.न्ने॑ रायस्पोष॒दाव्.न्ने॒ विष्ण॑वे । \newline
19. रा॒य॒स्पो॒ष॒दाव्.न्न॒ इति॑ रायस्पोष - दाव्.न्ने᳚ । \newline
20. विष्ण॑वे त्वा त्वा॒ विष्ण॑वे॒ विष्ण॑वे त्वा । \newline
21. त्वा॒ श्ये॒नाय॑ श्ये॒नाय॑ त्वा त्वा श्ये॒नाय॑ । \newline
22. श्ये॒नाय॑ त्वा त्वा श्ये॒नाय॑ श्ये॒नाय॑ त्वा । \newline
23. त्वा॒ सो॒म॒भृते॑ सोम॒भृते᳚ त्वा त्वा सोम॒भृते᳚ । \newline
24. सो॒म॒भृते॒ विष्ण॑वे॒ विष्ण॑वे सोम॒भृते॑ सोम॒भृते॒ विष्ण॑वे । \newline
25. सो॒म॒भृत॒ इति॑ सोम - भृते᳚ । \newline
26. विष्ण॑वे त्वा त्वा॒ विष्ण॑वे॒ विष्ण॑वे त्वा । \newline
27. त्वा॒ या या त्वा᳚ त्वा॒ या । \newline
28. या ते॑ ते॒ या या ते᳚ । \newline
29. ते॒ धामा॑नि॒ धामा॑नि ते ते॒ धामा॑नि । \newline
30. धामा॑नि ह॒विषा॑ ह॒विषा॒ धामा॑नि॒ धामा॑नि ह॒विषा᳚ । \newline
31. ह॒विषा॒ यज॑न्ति॒ यज॑न्ति ह॒विषा॑ ह॒विषा॒ यज॑न्ति । \newline
32. यज॑न्ति॒ ता ता यज॑न्ति॒ यज॑न्ति॒ ता । \newline
33. ता ते॑ ते॒ ता ता ते᳚ । \newline
34. ते॒ विश्वा॒ विश्वा॑ ते ते॒ विश्वा᳚ । \newline
35. विश्वा॑ परि॒भूः प॑रि॒भूर् विश्वा॒ विश्वा॑ परि॒भूः । \newline
36. प॒रि॒भू र॑स्त्वस्तु परि॒भूः प॑रि॒भू र॑स्तु । \newline
37. प॒रि॒भूरिति॑ परि - भूः । \newline
38. अ॒स्तु॒ य॒ज्ञ्ं ॅय॒ज्ञ् म॑स्त्वस्तु य॒ज्ञ्म् । \newline
39. य॒ज्ञ्म् ग॑य॒स्फानो॑ गय॒स्फानो॑ य॒ज्ञ्ं ॅय॒ज्ञ्म् ग॑य॒स्फानः॑ । \newline
40. ग॒य॒स्फानः॑ प्र॒तर॑णः प्र॒तर॑णो गय॒स्फानो॑ गय॒स्फानः॑ प्र॒तर॑णः । \newline
41. ग॒य॒स्फान॒ इति॑ गय - स्फानः॑ । \newline
42. प्र॒तर॑णः सु॒वीरः॑ सु॒वीरः॑ प्र॒तर॑णः प्र॒तर॑णः सु॒वीरः॑ । \newline
43. प्र॒तर॑ण॒ इति॑ प्र - तर॑णः । \newline
44. सु॒वीरो ऽवी॑र॒हा ऽवी॑रहा सु॒वीरः॑ सु॒वीरो ऽवी॑रहा । \newline
45. सु॒वीर॒ इति॑ सु - वीरः॑ । \newline
46. अवी॑रहा॒ प्र प्रावी॑र॒हा ऽवी॑रहा॒ प्र । \newline
47. अवी॑र॒हेत्यवी॑र - हा॒ । \newline
48. प्र च॑र चर॒ प्र प्र च॑र । \newline
49. च॒रा॒ सो॒म॒ सो॒म॒ च॒र॒ च॒रा॒ सो॒म॒ । \newline
50. सो॒म॒ दुर्या॒न् दुर्या᳚न् थ्सोम सोम॒ दुर्यान्॑ । \newline
51. दुर्या॒ नदि॑त्या॒ अदि॑त्या॒ दुर्या॒न् दुर्या॒ नदि॑त्याः । \newline
52. अदि॑त्याः॒ सदः॒ सदो ऽदि॑त्या॒ अदि॑त्याः॒ सदः॑ । \newline
53. सदो᳚ ऽस्यसि॒ सदः॒ सदो॑ ऽसि । \newline
54. अ॒स्यदि॑त्या॒ अदि॑त्या अस्य॒ स्यदि॑त्याः । \newline
55. अदि॑त्याः॒ सदः॒ सदो ऽदि॑त्या॒ अदि॑त्याः॒ सदः॑ । \newline
56. सद॒ आ सदः॒ सद॒ आ । \newline
57. आ सी॑द सी॒दा सी॑द । \newline

\textbf{Ghana Paata } \newline

1. अ॒ग्नेरा॑ति॒थ्य मा॑ति॒थ्य म॒ग्ने र॒ग्ने रा॑ति॒थ्य म॑स्यस्याति॒थ्य म॒ग्ने र॒ग्ने रा॑ति॒थ्य म॑सि । \newline
2. आ॒ति॒थ्य म॑स्यस्याति॒थ्य मा॑ति॒थ्य म॑सि॒ विष्ण॑वे॒ विष्ण॑वे ऽस्याति॒थ्य मा॑ति॒थ्य म॑सि॒ विष्ण॑वे । \newline
3. अ॒सि॒ विष्ण॑वे॒ विष्ण॑वे ऽस्यसि॒ विष्ण॑वे त्वा त्वा॒ विष्ण॑वे ऽस्यसि॒ विष्ण॑वे त्वा । \newline
4. विष्ण॑वे त्वा त्वा॒ विष्ण॑वे॒ विष्ण॑वे त्वा॒ सोम॑स्य॒ सोम॑स्य त्वा॒ विष्ण॑वे॒ विष्ण॑वे त्वा॒ सोम॑स्य । \newline
5. त्वा॒ सोम॑स्य॒ सोम॑स्य त्वा त्वा॒ सोम॑स्याति॒थ्य मा॑ति॒थ्यꣳ सोम॑स्य त्वा त्वा॒ सोम॑स्याति॒थ्यम् । \newline
6. सोम॑स्याति॒थ्य मा॑ति॒थ्यꣳ सोम॑स्य॒ सोम॑स्याति॒थ्य म॑स्यस्याति॒थ्यꣳ सोम॑स्य॒ सोम॑स्याति॒थ्य म॑सि । \newline
7. आ॒ति॒थ्य म॑स्यस्याति॒थ्य मा॑ति॒थ्य म॑सि॒ विष्ण॑वे॒ विष्ण॑वे ऽस्याति॒थ्य मा॑ति॒थ्य म॑सि॒ विष्ण॑वे । \newline
8. अ॒सि॒ विष्ण॑वे॒ विष्ण॑वे ऽस्यसि॒ विष्ण॑वे त्वा त्वा॒ विष्ण॑वे ऽस्यसि॒ विष्ण॑वे त्वा । \newline
9. विष्ण॑वे त्वा त्वा॒ विष्ण॑वे॒ विष्ण॑वे॒ त्वा ऽति॑थे॒ रति॑थेस्त्वा॒ विष्ण॑वे॒ विष्ण॑वे॒ त्वा ऽति॑थेः । \newline
10. त्वा ऽति॑थे॒ रति॑थेस्त्वा॒ त्वा ऽति॑थेराति॒थ्य मा॑ति॒थ्य मति॑थेस्त्वा॒ त्वा ऽति॑थेराति॒थ्यम् । \newline
11. अति॑थेराति॒थ्य मा॑ति॒थ्य मति॑थे॒ रति॑थेराति॒थ्य म॑स्यस्याति॒थ्य मति॑थे॒ रति॑थेराति॒थ्य म॑सि । \newline
12. आ॒ति॒थ्य म॑स्यस्याति॒थ्य मा॑ति॒थ्य म॑सि॒ विष्ण॑वे॒ विष्ण॑वे ऽस्याति॒थ्य मा॑ति॒थ्य म॑सि॒ विष्ण॑वे । \newline
13. अ॒सि॒ विष्ण॑वे॒ विष्ण॑वे ऽस्यसि॒ विष्ण॑वे त्वा त्वा॒ विष्ण॑वे ऽस्यसि॒ विष्ण॑वे त्वा । \newline
14. विष्ण॑वे त्वा त्वा॒ विष्ण॑वे॒ विष्ण॑वे त्वा॒ ऽग्नये॒ ऽग्नये᳚ त्वा॒ विष्ण॑वे॒ विष्ण॑वे त्वा॒ ऽग्नये᳚ । \newline
15. त्वा॒ ऽग्नये॒ ऽग्नये᳚ त्वा त्वा॒ ऽग्नये᳚ त्वा त्वा॒ ऽग्नये᳚ त्वा त्वा॒ ऽग्नये᳚ त्वा । \newline
16. अ॒ग्नये᳚ त्वा त्वा॒ ऽग्नये॒ ऽग्नये᳚ त्वा रायस्पोष॒दाव्.न्ने॑ रायस्पोष॒दाव्.न्ने᳚ त्वा॒ ऽग्नये॒ ऽग्नये᳚ त्वा रायस्पोष॒दाव्.न्ने᳚ । \newline
17. त्वा॒ रा॒य॒स्पो॒ष॒दाव्.न्ने॑ रायस्पोष॒दाव्.न्ने᳚ त्वा त्वा रायस्पोष॒दाव्.न्ने॒ विष्ण॑वे॒ विष्ण॑वे रायस्पोष॒दाव्.न्ने᳚ त्वा त्वा रायस्पोष॒दाव्.न्ने॒ विष्ण॑वे । \newline
18. रा॒य॒स्पो॒ष॒दाव्.न्ने॒ विष्ण॑वे॒ विष्ण॑वे रायस्पोष॒दाव्.न्ने॑ रायस्पोष॒दाव्.न्ने॒ विष्ण॑वे त्वा त्वा॒ विष्ण॑वे रायस्पोष॒दाव्.न्ने॑ रायस्पोष॒दाव्.न्ने॒ विष्ण॑वे त्वा । \newline
19. रा॒य॒स्पो॒ष॒दाव्.न्न॒ इति॑ रायस्पोष - दाव्.न्ने᳚ । \newline
20. विष्ण॑वे त्वा त्वा॒ विष्ण॑वे॒ विष्ण॑वे त्वा श्ये॒नाय॑ श्ये॒नाय॑ त्वा॒ विष्ण॑वे॒ विष्ण॑वे त्वा श्ये॒नाय॑ । \newline
21. त्वा॒ श्ये॒नाय॑ श्ये॒नाय॑ त्वा त्वा श्ये॒नाय॑ त्वा त्वा श्ये॒नाय॑ त्वा त्वा श्ये॒नाय॑ त्वा । \newline
22. श्ये॒नाय॑ त्वा त्वा श्ये॒नाय॑ श्ये॒नाय॑ त्वा सोम॒भृते॑ सोम॒भृते᳚ त्वा श्ये॒नाय॑ श्ये॒नाय॑ त्वा सोम॒भृते᳚ । \newline
23. त्वा॒ सो॒म॒भृते॑ सोम॒भृते᳚ त्वा त्वा सोम॒भृते॒ विष्ण॑वे॒ विष्ण॑वे सोम॒भृते᳚ त्वा त्वा सोम॒भृते॒ विष्ण॑वे । \newline
24. सो॒म॒भृते॒ विष्ण॑वे॒ विष्ण॑वे सोम॒भृते॑ सोम॒भृते॒ विष्ण॑वे त्वा त्वा॒ विष्ण॑वे सोम॒भृते॑ सोम॒भृते॒ विष्ण॑वे त्वा । \newline
25. सो॒म॒भृत॒ इति॑ सोम - भृते᳚ । \newline
26. विष्ण॑वे त्वा त्वा॒ विष्ण॑वे॒ विष्ण॑वे त्वा॒ या या त्वा॒ विष्ण॑वे॒ विष्ण॑वे त्वा॒ या । \newline
27. त्वा॒ या या त्वा᳚ त्वा॒ या ते॑ ते॒ या त्वा᳚ त्वा॒ या ते᳚ । \newline
28. या ते॑ ते॒ या या ते॒ धामा॑नि॒ धामा॑नि ते॒ या या ते॒ धामा॑नि । \newline
29. ते॒ धामा॑नि॒ धामा॑नि ते ते॒ धामा॑नि ह॒विषा॑ ह॒विषा॒ धामा॑नि ते ते॒ धामा॑नि ह॒विषा᳚ । \newline
30. धामा॑नि ह॒विषा॑ ह॒विषा॒ धामा॑नि॒ धामा॑नि ह॒विषा॒ यज॑न्ति॒ यज॑न्ति ह॒विषा॒ धामा॑नि॒ धामा॑नि ह॒विषा॒ यज॑न्ति । \newline
31. ह॒विषा॒ यज॑न्ति॒ यज॑न्ति ह॒विषा॑ ह॒विषा॒ यज॑न्ति॒ ता ता यज॑न्ति ह॒विषा॑ ह॒विषा॒ यज॑न्ति॒ ता । \newline
32. यज॑न्ति॒ ता ता यज॑न्ति॒ यज॑न्ति॒ ता ते॑ ते॒ ता यज॑न्ति॒ यज॑न्ति॒ ता ते᳚ । \newline
33. ता ते॑ ते॒ ता ता ते॒ विश्वा॒ विश्वा॑ ते॒ ता ता ते॒ विश्वा᳚ । \newline
34. ते॒ विश्वा॒ विश्वा॑ ते ते॒ विश्वा॑ परि॒भूः प॑रि॒भूर् विश्वा॑ ते ते॒ विश्वा॑ परि॒भूः । \newline
35. विश्वा॑ परि॒भूः प॑रि॒भूर् विश्वा॒ विश्वा॑ परि॒भूर॑स्त्वस्तु परि॒भूर् विश्वा॒ विश्वा॑ परि॒भूर॑स्तु । \newline
36. प॒रि॒भूर॑स्त्वस्तु परि॒भूः प॑रि॒भूर॑स्तु य॒ज्ञ्ं ॅय॒ज्ञ् म॑स्तु परि॒भूः प॑रि॒भूर॑स्तु य॒ज्ञ्म् । \newline
37. प॒रि॒भूरिति॑ परि - भूः । \newline
38. अ॒स्तु॒ य॒ज्ञ्ं ॅय॒ज्ञ् म॑स्त्वस्तु य॒ज्ञ्म् ग॑य॒स्फानो॑ गय॒स्फानो॑ य॒ज्ञ् म॑स्त्वस्तु य॒ज्ञ्म् ग॑य॒स्फानः॑ । \newline
39. य॒ज्ञ्म् ग॑य॒स्फानो॑ गय॒स्फानो॑ य॒ज्ञ्ं ॅय॒ज्ञ्म् ग॑य॒स्फानः॑ प्र॒तर॑णः प्र॒तर॑णो गय॒स्फानो॑ य॒ज्ञ्ं ॅय॒ज्ञ्म् ग॑य॒स्फानः॑ प्र॒तर॑णः । \newline
40. ग॒य॒स्फानः॑ प्र॒तर॑णः प्र॒तर॑णो गय॒स्फानो॑ गय॒स्फानः॑ प्र॒तर॑णः सु॒वीरः॑ सु॒वीरः॑ प्र॒तर॑णो गय॒स्फानो॑ गय॒स्फानः॑ प्र॒तर॑णः सु॒वीरः॑ । \newline
41. ग॒य॒स्फान॒ इति॑ गय - स्फानः॑ । \newline
42. प्र॒तर॑णः सु॒वीरः॑ सु॒वीरः॑ प्र॒तर॑णः प्र॒तर॑णः सु॒वीरो ऽवी॑र॒हा ऽवी॑रहा सु॒वीरः॑ प्र॒तर॑णः प्र॒तर॑णः सु॒वीरो ऽवी॑रहा । \newline
43. प्र॒तर॑ण॒ इति॑ प्र - तर॑णः । \newline
44. सु॒वीरो ऽवी॑र॒हा ऽवी॑रहा सु॒वीरः॑ सु॒वीरो ऽवी॑रहा॒ प्र प्रावी॑रहा सु॒वीरः॑ सु॒वीरो ऽवी॑रहा॒ प्र । \newline
45. सु॒वीर॒ इति॑ सु - वीरः॑ । \newline
46. अवी॑रहा॒ प्र प्रावी॑र॒हा ऽवी॑रहा॒ प्र च॑र चर॒ प्रावी॑र॒हा ऽवी॑रहा॒ प्र च॑र । \newline
47. अवी॑र॒हेत्यवी॑र - हा॒ । \newline
48. प्र च॑र चर॒ प्र प्र च॑रा सोम सोम चर॒ प्र प्र च॑रा सोम । \newline
49. च॒रा॒ सो॒म॒ सो॒म॒ च॒र॒ च॒रा॒ सो॒म॒ दुर्या॒न् दुर्या᳚न् थ्सोम चर चरा सोम॒ दुर्यान्॑ । \newline
50. सो॒म॒ दुर्या॒न् दुर्या᳚न् थ्सोम सोम॒ दुर्या॒ नदि॑त्या॒ अदि॑त्या॒ दुर्या᳚न् थ्सोम सोम॒ दुर्या॒ नदि॑त्याः । \newline
51. दुर्या॒ नदि॑त्या॒ अदि॑त्या॒ दुर्या॒न् दुर्या॒ नदि॑त्याः॒ सदः॒ सदो ऽदि॑त्या॒ दुर्या॒न् दुर्या॒ नदि॑त्याः॒ सदः॑ । \newline
52. अदि॑त्याः॒ सदः॒ सदो ऽदि॑त्या॒ अदि॑त्याः॒ सदो᳚ ऽस्यसि॒ सदो ऽदि॑त्या॒ अदि॑त्याः॒ सदो॑ ऽसि । \newline
53. सदो᳚ ऽस्यसि॒ सदः॒ सदो॒ ऽस्यदि॑त्या॒ अदि॑त्या असि॒ सदः॒ सदो॒ ऽस्यदि॑त्याः । \newline
54. अ॒स्यदि॑त्या॒ अदि॑त्या अस्य॒स्यदि॑त्याः॒ सदः॒ सदो ऽदि॑त्या अस्य॒स्यदि॑त्याः॒ सदः॑ । \newline
55. अदि॑त्याः॒ सदः॒ सदो ऽदि॑त्या॒ अदि॑त्याः॒ सद॒ आ सदो ऽदि॑त्या॒ अदि॑त्याः॒ सद॒ आ । \newline
56. सद॒ आ सदः॒ सद॒ आ सी॑द सी॒दा सदः॒ सद॒ आ सी॑द । \newline
57. आ सी॑द सी॒दा सी॑द॒ वरु॑णो॒ वरु॑णः सी॒दा सी॑द॒ वरु॑णः । \newline
\pagebreak
\markright{ TS 1.2.10.2  \hfill https://www.vedavms.in \hfill}

\section{ TS 1.2.10.2 }

\textbf{TS 1.2.10.2 } \newline
\textbf{Samhita Paata} \newline

सी॑द॒ वरु॑णोऽसि धृ॒तव्र॑तो वारु॒णम॑सि शं॒ॅयोर् दे॒वानाꣳ॑ स॒ख्यान्मा दे॒वाना॑-म॒पस॑-श्छिथ्स्म॒ह्याप॑तये त्वा गृह्णामि॒ परि॑पतये त्वा गृह्णामि॒ तनू॒नप्त्रे᳚ त्वा गृह्णामि शाक्व॒राय॑ त्वा गृह्णामि॒ शक्म॒न्नोजि॑ष्ठाय त्वा गृह्णा॒म्य-ना॑धृष्टमस्य-नाधृ॒ष्यं दे॒वाना॒मोजो॑- ऽभिषस्ति॒पा अ॑नभिशस्ते॒ऽन्यमनु॑ मे दी॒क्षां दी॒क्षाप॑तिर् मन्यता॒मनु॒ तप॒स्तप॑स्पति॒रञ्ज॑सा स॒त्यमुप॑ गेषꣳ सुवि॒ते मा॑ ( ) धाः ॥ \newline

\textbf{Pada Paata} \newline

सी॒द॒ । वरु॑णः । अ॒सि॒ । धृ॒तव्र॑त॒ इति॑ धृ॒त - व्र॒तः॒ । वा॒रु॒णम् । अ॒सि॒ । शं॒योरिति॑ शं - योः । दे॒वाना᳚म् । स॒ख्यात् । मा । दे॒वाना᳚म् । अ॒पसः॑ । छि॒थ्स्म॒हि॒ । आप॑तय॒ इत्या - प॒त॒ये॒ । त्वा॒ ।  गृ॒ह्णा॒मि॒ । परि॑पतय॒ इति॒ परि॑ - प॒त॒ये॒ । त्वा॒ । गृ॒ह्णा॒मि॒ । तनू॒नप्त्र॒ इति॒ तनू᳚ - नप्त्रे᳚ । त्वा॒ । गृ॒ह्णा॒मि॒ । शा॒क्व॒राय॑ । त्वा॒  । गृ॒ह्णा॒मि॒  । शक्मन्न्॑ । ओजि॑ष्ठाय । त्वा॒ । गृ॒ह्णा॒मि॒ । अना॑धृष्ट॒मित्यना᳚ - धृ॒ष्ट॒म् । अ॒सि॒ । अ॒ना॒धृ॒ष्यमित्य॑ना - धृ॒ष्यम् । दे॒वाना᳚म् । ओजः॑ । अ॒भि॒श॒स्ति॒पा इत्य॑भिशस्ति - पाः । अ॒न॒भि॒श॒स्ते॒न्यमित्य॑नभि - श॒स्ते॒न्यम् । अन्विति॑ । मे॒ । दी॒क्षाम् । दी॒क्षाप॑ति॒रिति॑ दी॒क्षा - प॒तिः॒ । म॒न्य॒ता॒म् । अन्विति॑ । तपः॑ । तप॑स्पति॒रिति॒ तपः॑ - प॒तिः॒ । अञ्ज॑सा । स॒त्यम् । उपेति॑ । गे॒ष॒म् । सु॒वि॒ते । मा॒ ( ) । धाः॒ ॥  \newline


\textbf{Krama Paata} \newline

सी॒द॒ वरु॑णः । वरु॑णोऽसि । अ॒सि॒ धृ॒तव्र॑तः । धृ॒तव्र॑तो वारु॒णम् । धृ॒तव्र॑त॒ इति॑ धृ॒त - व्र॒तः॒ । वा॒रु॒णम॑सि । अ॒सि॒ श॒म्ॅयोः । श॒म्ॅयोर् दे॒वाना᳚म् । श॒म्ॅयोरिति॑ शम् - योः । दे॒वानाꣳ॑ स॒ख्यात् । स॒ख्यान्मा । मा दे॒वाना᳚म् । दे॒वाना॑म॒पसः॑ । अ॒पसः॑ छिथ्स्महि । छि॒थ्स्म॒ह्याप॑तये । आप॑तये त्वा । आप॑तय॒ इत्या - प॒त॒ये॒ । त्वा॒ गृ॒ह्णा॒मि॒ । गृ॒ह्णा॒मि॒ परि॑पतये । परि॑पतये त्वा । परि॑पतय॒ इति॒ परि॑ - प॒त॒ये॒ । त्वा॒ गृ॒ह्णा॒मि॒ । गृ॒ह्णा॒मि॒ तनू॒नप्त्रे᳚ । तनू॒नप्त्रे᳚ त्वा । तनू॒नप्त्र॒ इति॒ तनू᳚ - नप्त्रे᳚ । त्वा॒ गृ॒ह्णा॒मि॒ । गृ॒ह्णा॒मि॒ शा॒क्व॒राय॑ । शा॒क्व॒राय॑ त्वा । त्वा॒ गृ॒ह्णा॒मि॒ । गृ॒ह्णा॒मि॒ शक्मन्न्॑ । शक्म॒न्नोजि॑ष्ठाय । ओजि॑ष्ठाय त्वा । त्वा॒ गृ॒ह्णा॒मि॒ । गृ॒ह्णा॒म्यना॑धृष्टम् । अना॑धृष्टमसि । अना॑धृष्ट॒मित्यना᳚ - धृ॒ष्ट॒म् । अ॒स्य॒ना॒धृ॒ष्यम् । अ॒ना॒धृ॒ष्यम् दे॒वाना᳚म् । अ॒ना॒धृ॒ष्यमित्य॑ना - धृ॒ष्यम् । दे॒वाना॒मोजः॑ । ओजो॑ऽभिशस्ति॒पाः । अ॒भि॒श॒स्ति॒पा अ॑नभिशस्ते॒न्यम् । अ॒भि॒श॒स्ति॒पा इत्य॑भिशस्ति - पाः । अ॒न॒भि॒श॒स्ते॒न्यमनु॑ । अ॒न॒भि॒श॒स्ते॒न्यमित्य॑नभि - श॒स्ते॒न्यम् । अनु॑ मे । मे॒ दी॒क्षाम् । दी॒क्षाम् दी॒क्षाप॑तिः । दी॒क्षाप॑तिर् मन्यताम् । दी॒क्षाप॑ति॒रिति॑ दी॒क्षा - प॒तिः॒ । म॒न्य॒तामनु॑ । अनु॒ तपः॑ । तप॒स्तप॑स्पतिः । तप॑स्पति॒रञ्ज॑सा । तप॑स्पति॒रिति॒ तपः॑ - प॒तिः॒ । अञ्ज॑सा स॒त्यम् । स॒त्यमुप॑ । उप॑ गेषम् । गे॒षꣳ॒॒ सु॒वि॒ते । सु॒वि॒ते मा᳚ ( ) । मा॒ धाः॒ । धा॒ इति॑ धाः । \newline

\textbf{Jatai Paata} \newline

1. सी॒द॒ वरु॑णो॒ वरु॑णः सीद सीद॒ वरु॑णः । \newline
2. वरु॑णो ऽस्यसि॒ वरु॑णो॒ वरु॑णो ऽसि । \newline
3. अ॒सि॒ धृ॒तव्र॑तो धृ॒तव्र॑तो ऽस्यसि धृ॒तव्र॑तः । \newline
4. धृ॒तव्र॑तो वारु॒णं ॅवा॑रु॒णम् धृ॒तव्र॑तो धृ॒तव्र॑तो वारु॒णम् । \newline
5. धृ॒तव्र॑त॒ इति॑ धृ॒त - व्र॒तः॒ । \newline
6. वा॒रु॒ण म॑स्यसि वारु॒णं ॅवा॑रु॒ण म॑सि । \newline
7. अ॒सि॒ शं॒ॅयोः शं॒ॅयो र॑स्यसि शं॒ॅयोः । \newline
8. शं॒ॅयोर् दे॒वाना᳚म् दे॒वाना(ग्म्॑) शं॒ॅयोः शं॒ॅयोर् दे॒वाना᳚म् । \newline
9. शं॒ॅयोरिति॑ शं - योः । \newline
10. दे॒वाना(ग्म्॑) स॒ख्याथ् स॒ख्याद् दे॒वाना᳚म् दे॒वाना(ग्म्॑) स॒ख्यात् । \newline
11. स॒ख्यान् मा मा स॒ख्याथ् स॒ख्यान् मा । \newline
12. मा दे॒वाना᳚म् दे॒वाना॒म् मा मा दे॒वाना᳚म् । \newline
13. दे॒वाना॑ म॒पसो॒ ऽपसो॑ दे॒वाना᳚म् दे॒वाना॑ म॒पसः॑ । \newline
14. अ॒पस॑ श्छिथ्स्महि छिथ्स्मह्य॒पसो॒ ऽपस॑ श्छिथ्स्महि । \newline
15. छि॒थ्स्म॒ह्याप॑तय॒ आप॑तये छिथ्स्महि छिथ्स्म॒ह्याप॑तये । \newline
16. आप॑तये त्वा॒ त्वा ऽऽप॑तय॒ आप॑तये त्वा । \newline
17. आप॑तय॒ इत्या - प॒त॒ये॒ । \newline
18. त्वा॒ गृ॒ह्णा॒मि॒ गृ॒ह्णा॒मि॒ त्वा॒ त्वा॒ गृ॒ह्णा॒मि॒ । \newline
19. गृ॒ह्णा॒मि॒ परि॑पतये॒ परि॑पतये गृह्णामि गृह्णामि॒ परि॑पतये । \newline
20. परि॑पतये त्वा त्वा॒ परि॑पतये॒ परि॑पतये त्वा । \newline
21. परि॑पतय॒ इति॒ परि॑ - प॒त॒ये॒ । \newline
22. त्वा॒ गृ॒ह्णा॒मि॒ गृ॒ह्णा॒मि॒ त्वा॒ त्वा॒ गृ॒ह्णा॒मि॒ । \newline
23. गृ॒ह्णा॒मि॒ तनू॒नप्त्रे॒ तनू॒नप्त्रे॑ गृह्णामि गृह्णामि॒ तनू॒नप्त्रे᳚ । \newline
24. तनू॒नप्त्रे᳚ त्वा त्वा॒ तनू॒नप्त्रे॒ तनू॒नप्त्रे᳚ त्वा । \newline
25. तनू॒नप्त्र॒ इति॒ तनू᳚ - नप्त्रे᳚ । \newline
26. त्वा॒ गृ॒ह्णा॒मि॒ गृ॒ह्णा॒मि॒ त्वा॒ त्वा॒ गृ॒ह्णा॒मि॒ । \newline
27. गृ॒ह्णा॒मि॒ शा॒क्व॒राय॑ शाक्व॒राय॑ गृह्णामि गृह्णामि शाक्व॒राय॑ । \newline
28. शा॒क्व॒राय॑ त्वा त्वा शाक्व॒राय॑ शाक्व॒राय॑ त्वा । \newline
29. त्वा॒ गृ॒ह्णा॒मि॒ गृ॒ह्णा॒मि॒ त्वा॒ त्वा॒ गृ॒ह्णा॒मि॒ । \newline
30. गृ॒ह्णा॒मि॒ शक्म॒ञ् छक्म॑न् गृह्णामि गृह्णामि॒ शक्मन्न्॑ । \newline
31. शक्म॒न् नोजि॑ष्ठा॒यौजि॑ष्ठाय॒ शक्म॒ञ् छक्म॒न् नोजि॑ष्ठाय । \newline
32. ओजि॑ष्ठाय त्वा॒ त्वौजि॑ष्ठा॒ यौजि॑ष्ठाय त्वा । \newline
33. त्वा॒ गृ॒ह्णा॒मि॒ गृ॒ह्णा॒मि॒ त्वा॒ त्वा॒ गृ॒ह्णा॒मि॒ । \newline
34. गृ॒ह्णा॒म्यना॑धृष्ट॒ मना॑धृष्टम् गृह्णामि गृह्णा॒म्यना॑धृष्टम् । \newline
35. अना॑धृष्ट मस्य॒स्यना॑धृष्ट॒ मना॑धृष्ट मसि । \newline
36. अना॑धृष्ट॒मित्यना᳚ - धृ॒ष्ट॒म् । \newline
37. अ॒स्य॒ना॒धृ॒ष्य म॑नाधृ॒ष्य म॑स्यस्यनाधृ॒ष्यम् । \newline
38. अ॒ना॒धृ॒ष्यम् दे॒वाना᳚म् दे॒वाना॑ मनाधृ॒ष्य म॑नाधृ॒ष्यम् दे॒वाना᳚म् । \newline
39. अ॒ना॒धृ॒ष्यमित्य॑ना - धृ॒ष्यम् । \newline
40. दे॒वाना॒ मोज॒ ओजो॑ दे॒वाना᳚म् दे॒वाना॒ मोजः॑ । \newline
41. ओजो॑ ऽभिशस्ति॒पा अ॑भिशस्ति॒पा ओज॒ ओजो॑ ऽभिशस्ति॒पाः । \newline
42. अ॒भि॒श॒स्ति॒पा अ॑नभिशस्ते॒न्य म॑नभिशस्ते॒न्य म॑भिशस्ति॒पा अ॑भिशस्ति॒पा अ॑नभिशस्ते॒न्यम् । \newline
43. अ॒भि॒श॒स्ति॒पा इत्य॑भिशस्ति - पाः । \newline
44. अ॒न॒भि॒श॒स्ते॒न्य मन्वन्व॑नभिशस्ते॒न्य म॑नभिशस्ते॒न्य मनु॑ । \newline
45. अ॒न॒भि॒श॒स्ते॒न्यमित्य॑नभि - श॒स्ते॒न्यम् । \newline
46. अनु॑ मे॒ मे ऽन्वनु॑ मे । \newline
47. मे॒ दी॒क्षाम् दी॒क्षाम् मे॑ मे दी॒क्षाम् । \newline
48. दी॒क्षाम् दी॒क्षाप॑तिर् दी॒क्षाप॑तिर् दी॒क्षाम् दी॒क्षाम् दी॒क्षाप॑तिः । \newline
49. दी॒क्षाप॑तिर् मन्यताम् मन्यताम् दी॒क्षाप॑तिर् दी॒क्षाप॑तिर् मन्यताम् । \newline
50. दी॒क्षाप॑ति॒रिति॑ दी॒क्षा - प॒तिः॒ । \newline
51. म॒न्य॒ता॒ मन्वनु॑ मन्यताम् मन्यता॒ मनु॑ । \newline
52. अनु॒ तप॒स्तपो ऽन्वनु॒ तपः॑ । \newline
53. तप॒ स्तप॑ स्पति॒ स्तप॑ स्पति॒ स्तप॒ स्तप॒ स्तप॑ स्पतिः । \newline
54. तप॑स्पति॒ रञ्ज॒सा ऽञ्ज॑सा॒ तप॑स्पति॒ स्तप॑स्पति॒ रञ्ज॑सा । \newline
55. तप॑स्पति॒रिति॒ तपः॑ - प॒तिः॒ । \newline
56. अञ्ज॑सा स॒त्यꣳ स॒त्य मञ्ज॒सा ऽञ्ज॑सा स॒त्यम् । \newline
57. स॒त्य मुपोप॑ स॒त्यꣳ स॒त्य मुप॑ । \newline
58. उप॑ गेषम् गेष॒ मुपोप॑ गेषम् । \newline
59. गे॒ष॒(ग्म्॒) सु॒वि॒ते सु॑वि॒ते गे॑षम् गेषꣳ सुवि॒ते । \newline
60. सु॒वि॒ते मा॑ मा सुवि॒ते सु॑वि॒ते मा᳚ । \newline
61. मा॒ धा॒ धा॒ मा॒ मा॒ धाः॒ । \newline
62. धा॒ इति॑ धाः । \newline

\textbf{Ghana Paata } \newline

1. सी॒द॒ वरु॑णो॒ वरु॑णः सीद सीद॒ वरु॑णो ऽस्यसि॒ वरु॑णः सीद सीद॒ वरु॑णो ऽसि । \newline
2. वरु॑णो ऽस्यसि॒ वरु॑णो॒ वरु॑णो ऽसि धृ॒तव्र॑तो धृ॒तव्र॑तो ऽसि॒ वरु॑णो॒ वरु॑णो ऽसि धृ॒तव्र॑तः । \newline
3. अ॒सि॒ धृ॒तव्र॑तो धृ॒तव्र॑तो ऽस्यसि धृ॒तव्र॑तो वारु॒णं ॅवा॑रु॒णम् धृ॒तव्र॑तो ऽस्यसि धृ॒तव्र॑तो वारु॒णम् । \newline
4. धृ॒तव्र॑तो वारु॒णं ॅवा॑रु॒णम् धृ॒तव्र॑तो धृ॒तव्र॑तो वारु॒ण म॑स्यसि वारु॒णम् धृ॒तव्र॑तो धृ॒तव्र॑तो वारु॒ण म॑सि । \newline
5. धृ॒तव्र॑त॒ इति॑ धृ॒त - व्र॒तः॒ । \newline
6. वा॒रु॒ण म॑स्यसि वारु॒णं ॅवा॑रु॒ण म॑सि शं॒ॅयोः शं॒ॅयोर॑सि वारु॒णं ॅवा॑रु॒ण म॑सि शं॒ॅयोः । \newline
7. अ॒सि॒ शं॒ॅयोः शं॒ॅयोर॑स्यसि शं॒ॅयोर् दे॒वाना᳚म् दे॒वाना(ग्म्॑) शं॒ॅयोर॑स्यसि शं॒ॅयोर् दे॒वाना᳚म् । \newline
8. शं॒ॅयोर् दे॒वाना᳚म् दे॒वाना(ग्म्॑) शं॒ॅयोः शं॒ॅयोर् दे॒वाना(ग्म्॑) स॒ख्याथ् स॒ख्याद् दे॒वाना(ग्म्॑) शं॒ॅयोः शं॒ॅयोर् दे॒वाना(ग्म्॑) स॒ख्यात् । \newline
9. शं॒ॅयोरिति॑ शं - योः । \newline
10. दे॒वाना(ग्म्॑) स॒ख्याथ् स॒ख्याद् दे॒वाना᳚म् दे॒वाना(ग्म्॑) स॒ख्यान् मा मा स॒ख्याद् दे॒वाना᳚म् दे॒वाना(ग्म्॑) स॒ख्यान् मा । \newline
11. स॒ख्यान् मा मा स॒ख्याथ् स॒ख्यान् मा दे॒वाना᳚म् दे॒वाना॒म् मा स॒ख्याथ् स॒ख्यान् मा दे॒वाना᳚म् । \newline
12. मा दे॒वाना᳚म् दे॒वाना॒म् मा मा दे॒वाना॑ म॒पसो॒ ऽपसो॑ दे॒वाना॒म् मा मा दे॒वाना॑ म॒पसः॑ । \newline
13. दे॒वाना॑ म॒पसो॒ ऽपसो॑ दे॒वाना᳚म् दे॒वाना॑ म॒पस॑ श्छिथ्स्महि छिथ्स्मह्य॒पसो॑ दे॒वाना᳚म् दे॒वाना॑ म॒पस॑ श्छिथ्स्महि । \newline
14. अ॒पस॑ श्छिथ्स्महि छिथ्स्मह्य॒पसो॒ ऽपस॑ श्छिथ्स्म॒ह्याप॑तय॒ आप॑तये छिथ्स्मह्य॒पसो॒ ऽपस॑ श्छिथ्स्म॒ह्याप॑तये । \newline
15. छि॒थ्स्म॒ह्याप॑तय॒ आप॑तये छिथ्स्महि छिथ्स्म॒ह्याप॑तये त्वा॒ त्वा ऽऽप॑तये छिथ्स्महि छिथ्स्म॒ह्याप॑तये त्वा । \newline
16. आप॑तये त्वा॒ त्वा ऽऽप॑तय॒ आप॑तये त्वा गृह्णामि गृह्णामि॒ त्वा ऽऽप॑तय॒ आप॑तये त्वा गृह्णामि । \newline
17. आप॑तय॒ इत्या - प॒त॒ये॒ । \newline
18. त्वा॒ गृ॒ह्णा॒मि॒ गृ॒ह्णा॒मि॒ त्वा॒ त्वा॒ गृ॒ह्णा॒मि॒ परि॑पतये॒ परि॑पतये गृह्णामि त्वा त्वा गृह्णामि॒ परि॑पतये । \newline
19. गृ॒ह्णा॒मि॒ परि॑पतये॒ परि॑पतये गृह्णामि गृह्णामि॒ परि॑पतये त्वा त्वा॒ परि॑पतये गृह्णामि गृह्णामि॒ परि॑पतये त्वा । \newline
20. परि॑पतये त्वा त्वा॒ परि॑पतये॒ परि॑पतये त्वा गृह्णामि गृह्णामि त्वा॒ परि॑पतये॒ परि॑पतये त्वा गृह्णामि । \newline
21. परि॑पतय॒ इति॒ परि॑ - प॒त॒ये॒ । \newline
22. त्वा॒ गृ॒ह्णा॒मि॒ गृ॒ह्णा॒मि॒ त्वा॒ त्वा॒ गृ॒ह्णा॒मि॒ तनू॒नप्त्रे॒ तनू॒नप्त्रे॑ गृह्णामि त्वा त्वा गृह्णामि॒ तनू॒नप्त्रे᳚ । \newline
23. गृ॒ह्णा॒मि॒ तनू॒नप्त्रे॒ तनू॒नप्त्रे॑ गृह्णामि गृह्णामि॒ तनू॒नप्त्रे᳚ त्वा त्वा॒ तनू॒नप्त्रे॑ गृह्णामि गृह्णामि॒ तनू॒नप्त्रे᳚ त्वा । \newline
24. तनू॒नप्त्रे᳚ त्वा त्वा॒ तनू॒नप्त्रे॒ तनू॒नप्त्रे᳚ त्वा गृह्णामि गृह्णामि त्वा॒ तनू॒नप्त्रे॒ तनू॒नप्त्रे᳚ त्वा गृह्णामि । \newline
25. तनू॒नप्त्र॒ इति॒ तनू᳚ - नप्त्रे᳚ । \newline
26. त्वा॒ गृ॒ह्णा॒मि॒ गृ॒ह्णा॒मि॒ त्वा॒ त्वा॒ गृ॒ह्णा॒मि॒ शा॒क्व॒राय॑ शाक्व॒राय॑ गृह्णामि त्वा त्वा गृह्णामि शाक्व॒राय॑ । \newline
27. गृ॒ह्णा॒मि॒ शा॒क्व॒राय॑ शाक्व॒राय॑ गृह्णामि गृह्णामि शाक्व॒राय॑ त्वा त्वा शाक्व॒राय॑ गृह्णामि गृह्णामि शाक्व॒राय॑ त्वा । \newline
28. शा॒क्व॒राय॑ त्वा त्वा शाक्व॒राय॑ शाक्व॒राय॑ त्वा गृह्णामि गृह्णामि त्वा शाक्व॒राय॑ शाक्व॒राय॑ त्वा गृह्णामि । \newline
29. त्वा॒ गृ॒ह्णा॒मि॒ गृ॒ह्णा॒मि॒ त्वा॒ त्वा॒ गृ॒ह्णा॒मि॒ शक्म॒ञ् छक्म॑न् गृह्णामि त्वा त्वा गृह्णामि॒ शक्मन्न्॑ । \newline
30. गृ॒ह्णा॒मि॒ शक्म॒ञ् छक्म॑न् गृह्णामि गृह्णामि॒ शक्म॒न् नोजि॑ष्ठा॒यौजि॑ष्ठाय॒ शक्म॑न् गृह्णामि गृह्णामि॒ शक्म॒न् नोजि॑ष्ठाय । \newline
31. शक्म॒न् नोजि॑ष्ठा॒यौजि॑ष्ठाय॒ शक्म॒ञ् छक्म॒न् नोजि॑ष्ठाय त्वा॒ त्वौजि॑ष्ठाय॒ शक्म॒ञ् छक्म॒न् नोजि॑ष्ठाय त्वा । \newline
32. ओजि॑ष्ठाय त्वा॒ त्वौजि॑ष्ठा॒यौजि॑ष्ठाय त्वा गृह्णामि गृह्णामि॒ त्वौजि॑ष्ठा॒यौजि॑ष्ठाय त्वा गृह्णामि । \newline
33. त्वा॒ गृ॒ह्णा॒मि॒ गृ॒ह्णा॒मि॒ त्वा॒ त्वा॒ गृ॒ह्णा॒म्यना॑धृष्ट॒ मना॑धृष्टम् गृह्णामि त्वा त्वा गृह्णा॒म्यना॑धृष्टम् । \newline
34. गृ॒ह्णा॒म्यना॑धृष्ट॒ मना॑धृष्टम् गृह्णामि गृह्णा॒म्यना॑धृष्ट मस्य॒स्यना॑धृष्टम् गृह्णामि गृह्णा॒म्यना॑धृष्ट मसि । \newline
35. अना॑धृष्ट मस्य॒स्यना॑धृष्ट॒ मना॑धृष्ट मस्यनाधृ॒ष्य म॑नाधृ॒ष्य म॒स्यना॑धृष्ट॒ मना॑धृष्ट मस्यनाधृ॒ष्यम् । \newline
36. अना॑धृष्ट॒मित्यना᳚ - धृ॒ष्ट॒म् । \newline
37. अ॒स्य॒ना॒धृ॒ष्य म॑नाधृ॒ष्य म॑स्यस्यनाधृ॒ष्यम् दे॒वाना᳚म् दे॒वाना॑ मनाधृ॒ष्य म॑स्यस्यनाधृ॒ष्यम् दे॒वाना᳚म् । \newline
38. अ॒ना॒धृ॒ष्यम् दे॒वाना᳚म् दे॒वाना॑ मनाधृ॒ष्य म॑नाधृ॒ष्यम् दे॒वाना॒ मोज॒ ओजो॑ दे॒वाना॑ मनाधृ॒ष्य म॑नाधृ॒ष्यम् दे॒वाना॒ मोजः॑ । \newline
39. अ॒ना॒धृ॒ष्यमित्य॑ना - धृ॒ष्यम् । \newline
40. दे॒वाना॒ मोज॒ ओजो॑ दे॒वाना᳚म् दे॒वाना॒ मोजो॑ ऽभिशस्ति॒पा अ॑भिशस्ति॒पा ओजो॑ दे॒वाना᳚म् दे॒वाना॒ मोजो॑ ऽभिशस्ति॒पाः । \newline
41. ओजो॑ ऽभिशस्ति॒पा अ॑भिशस्ति॒पा ओज॒ ओजो॑ ऽभिशस्ति॒पा अ॑नभिशस्ते॒न्य म॑नभिशस्ते॒न्य म॑भिशस्ति॒पा ओज॒ ओजो॑ ऽभिशस्ति॒पा अ॑नभिशस्ते॒न्यम् । \newline
42. अ॒भि॒श॒स्ति॒पा अ॑नभिशस्ते॒न्य म॑नभिशस्ते॒न्य म॑भिशस्ति॒पा अ॑भिशस्ति॒पा अ॑नभिशस्ते॒न्य मन्वन्व॑नभिशस्ते॒न्य म॑भिशस्ति॒पा अ॑भिशस्ति॒पा अ॑नभिशस्ते॒न्य मनु॑ । \newline
43. अ॒भि॒श॒स्ति॒पा इत्य॑भिशस्ति - पाः । \newline
44. अ॒न॒भि॒श॒स्ते॒न्य मन्वन्व॑नभिशस्ते॒न्य म॑नभिशस्ते॒न्य मनु॑ मे॒ मे ऽन्व॑नभिशस्ते॒न्य म॑नभिशस्ते॒न्य मनु॑ मे । \newline
45. अ॒न॒भि॒श॒स्ते॒न्यमित्य॑नभि - श॒स्ते॒न्यम् । \newline
46. अनु॑ मे॒ मे ऽन्वनु॑ मे दी॒क्षाम् दी॒क्षाम् मे ऽन्वनु॑ मे दी॒क्षाम् । \newline
47. मे॒ दी॒क्षाम् दी॒क्षाम् मे॑ मे दी॒क्षाम् दी॒क्षाप॑तिर् दी॒क्षाप॑तिर् दी॒क्षाम् मे॑ मे दी॒क्षाम् दी॒क्षाप॑तिः । \newline
48. दी॒क्षाम् दी॒क्षाप॑तिर् दी॒क्षाप॑तिर् दी॒क्षाम् दी॒क्षाम् दी॒क्षाप॑तिर् मन्यताम् मन्यताम् दी॒क्षाप॑तिर् दी॒क्षाम् दी॒क्षाम् दी॒क्षाप॑तिर् मन्यताम् । \newline
49. दी॒क्षाप॑तिर् मन्यताम् मन्यताम् दी॒क्षाप॑तिर् दी॒क्षाप॑तिर् मन्यता॒ मन्वनु॑ मन्यताम् दी॒क्षाप॑तिर् दी॒क्षाप॑तिर् मन्यता॒ मनु॑ । \newline
50. दी॒क्षाप॑ति॒रिति॑ दी॒क्षा - प॒तिः॒ । \newline
51. म॒न्य॒ता॒ मन्वनु॑ मन्यताम् मन्यता॒ मनु॒ तप॒स्तपो ऽनु॑ मन्यताम् मन्यता॒ मनु॒ तपः॑ । \newline
52. अनु॒ तप॒ स्तपो ऽन्वनु॒ तप॒ स्तप॑स्पति॒ स्तप॑स्पति॒ स्तपो ऽन्वनु॒ तप॒स्तप॑स्पतिः । \newline
53. तप॒ स्तप॑स्पति॒ स्तप॑स्पति॒ स्तप॒ स्तप॒ स्तप॑स्पति॒ रञ्ज॒सा ऽञ्ज॑सा॒ तप॑स्पति॒ स्तप॒ स्तप॒ स्तप॑स्पति॒ रञ्ज॑सा । \newline
54. तप॑स्पति॒ रञ्ज॒सा ऽञ्ज॑सा॒ तप॑स्पति॒ स्तप॑स्पति॒ रञ्ज॑सा स॒त्यꣳ स॒त्य मञ्ज॑सा॒ तप॑स्पति॒ स्तप॑स्पति॒ रञ्ज॑सा स॒त्यम् । \newline
55. तप॑स्पति॒रिति॒ तपः॑ - प॒तिः॒ । \newline
56. अञ्ज॑सा स॒त्यꣳ स॒त्य मञ्ज॒सा ऽञ्ज॑सा स॒त्य मुपोप॑ स॒त्य मञ्ज॒सा ऽञ्ज॑सा स॒त्य मुप॑ । \newline
57. स॒त्य मुपोप॑ स॒त्यꣳ स॒त्य मुप॑ गेषम् गेष॒ मुप॑ स॒त्यꣳ स॒त्य मुप॑ गेषम् । \newline
58. उप॑ गेषम् गेष॒ मुपोप॑ गेषꣳ सुवि॒ते सु॑वि॒ते गे॑ष॒ मुपोप॑ गेषꣳ सुवि॒ते । \newline
59. गे॒ष॒(ग्म्॒) सु॒वि॒ते सु॑वि॒ते गे॑षम् गेषꣳ सुवि॒ते मा॑ मा सुवि॒ते गे॑षम् गेषꣳ सुवि॒ते मा᳚ । \newline
60. सु॒वि॒ते मा॑ मा सुवि॒ते सु॑वि॒ते मा॑ धा धा मा सुवि॒ते सु॑वि॒ते मा॑ धाः । \newline
61. मा॒ धा॒ धा॒ मा॒ मा॒ धाः॒ । \newline
62. धा॒ इति॑ धाः । \newline
\pagebreak
\markright{ TS 1.2.11.1  \hfill https://www.vedavms.in \hfill}

\section{ TS 1.2.11.1 }

\textbf{TS 1.2.11.1 } \newline
\textbf{Samhita Paata} \newline

अꣳ॒॒शुरꣳ॑शुस्ते देव सो॒माऽऽ*प्या॑यता॒-मिन्द्रा॑यैकधन॒विद॒ आ तुभ्य॒मिन्द्रः॑ प्यायता॒मा त्वमिन्द्रा॑य प्याय॒स्वाऽऽ*प्या॑यय॒ सखी᳚न्थ् स॒न्या मे॒धया᳚ स्व॒स्ति ते॑ देव सोम सु॒त्याम॑शी॒येष्टा॒ रायः॒ प्रेषे भगा॑य॒र्तमृ॑तवा॒दिभ्यो॒ नमो॑ दि॒वे नमः॑ पृथि॒व्या अग्ने᳚ व्रतपते॒ त्वं ॅव्र॒तानां᳚ ॅव्र॒तप॑तिरसि॒ या मम॑ त॒नूरे॒षा सा त्वयि॒ -[ ] \newline

\textbf{Pada Paata} \newline

अꣳ॒॒शुरꣳ॑शु॒रित्यꣳ॒॒शुः - अꣳ॒॒शुः॒ । ते॒ । दे॒व॒ । सो॒म॒ । एति॑ । प्या॒य॒ता॒म् । इन्द्रा॑य । ए॒क॒ध॒न॒विद॒ इत्ये॑कधन - विदे᳚ । एति॑ । तुभ्य᳚म् । इन्द्रः॑ । प्या॒य॒ता॒म् । एति॑ । त्वम् । इन्द्रा॑य । प्या॒य॒स्व॒ । एति॑ । प्या॒य॒य॒ । सखीन्॑ । स॒न्या । मे॒धया᳚ । स्व॒स्ति । ते॒ । दे॒व॒ । सो॒म॒ । सु॒त्याम् । अ॒शी॒य॒ । एष्टः॑ । रायः॑ । प्रेति॑ । इ॒षे । भगा॑य । ऋ॒तम् । ऋ॒त॒वा॒दिभ्य॒ इत्यृ॑तवा॒दि - भ्यः॒ । नमः॑ । दि॒वे । नमः॑ । पृ॒थि॒व्यै । अग्ने᳚ । व्र॒त॒प॒त॒ इति॑ व्रत - प॒ते॒ । त्वम् । व्र॒ताना᳚म् । व्र॒तप॑ति॒रिति॑ व्र॒त - प॒तिः॒ । अ॒सि॒ । या । मम॑ । त॒नूः । ए॒षा । सा । त्वयि॑ ।  \newline


\textbf{Krama Paata} \newline

अꣳ॒॒शुरꣳ॑शुस्ते । अꣳ॒॒शुरꣳ॑शु॒रित्यꣳ॒॒शुः - अꣳ॒॒शुः॒ । ते॒ दे॒व॒ । दे॒व॒ सो॒म॒ । सो॒मा । आ प्या॑यताम् । प्या॒य॒ता॒मिन्द्रा॑य । इन्द्रा॑यैकधन॒विदे᳚ । ए॒क॒ध॒न॒विद॒ आ । ए॒क॒ध॒न॒विद॒ इत्ये॑कधन - विदे᳚ । आ तुभ्य᳚म् । तुभ्य॒मिन्द्रः॑ । इन्द्रः॑ प्यायताम् । प्या॒य॒ता॒ मा । आ त्वम् । त्वमिन्द्रा॑य । इन्द्रा॑य प्यायस्व । प्या॒य॒स्वा । आ प्या॑यय । प्या॒य॒य॒ सखीन्॑ । सखी᳚न्थ् स॒न्या । स॒न्या मे॒धया᳚ । मे॒धया᳚ स्व॒स्ति । स्व॒स्ति ते᳚ । ते॒ दे॒व॒ । दे॒व॒ सो॒म॒ । सो॒म॒ सु॒त्याम् । सु॒त्याम॑शीय । अ॒शी॒येष्टः॑ । एष्टा॒ रायः॑ । रायः॒ प्र । प्रेषे । इ॒षे भगा॑य । भगा॑य॒र्तम् । ऋ॒तमृ॑तवा॒दिभ्यः॑ । ऋ॒त॒वा॒दिभ्यो॒ नमः॑ । ऋ॒त॒वा॒दिभ्य॒ इत्यृ॑तवा॒दि - भ्यः॒ । नमो॑ दि॒वे । दि॒वे नमः॑ । नमः॑ पृथि॒व्यै । पृ॒थि॒व्या अग्ने᳚ । अग्ने᳚ व्रतपते । व्र॒त॒प॒ते॒ त्वम् । व्र॒त॒प॒त॒ इति॑ व्रत - प॒ते॒ । त्वं ॅव्र॒ताना᳚म् । व्र॒तानां᳚ ॅव्र॒तप॑तिः । व्र॒तप॑तिरसि । व्र॒तप॑ति॒रिति॑ व्र॒त - प॒तिः॒ । अ॒सि॒ या । या मम॑ । मम॑ त॒नूः । त॒नूरे॒षा । ए॒षा सा । सा त्वयि॑ ( ) । त्वयि॒ या \newline

\textbf{Jatai Paata} \newline

1. अ॒(ग्म्॒)शु र(ग्म्॑)शुस्ते ते अ॒(ग्म्॒)शु र(ग्म्॑)शु र॒(ग्म्॒)शु र(ग्म्॑)शु स्ते । \newline
2. अ॒(ग्म्॒)शुर(ग्म्॑)शु॒रित्य॒(ग्म्॒)शुः - अ॒(ग्म्॒)शुः॒ । \newline
3. ते॒ दे॒व॒ दे॒व॒ ते॒ ते॒ दे॒व॒ । \newline
4. दे॒व॒ सो॒म॒ सो॒म॒ दे॒व॒ दे॒व॒ सो॒म॒ । \newline
5. सो॒मा सो॑म सो॒मा । \newline
6. आ प्या॑यताम् प्यायता॒ मा प्या॑यताम् । \newline
7. प्या॒य॒ता॒ मिन्द्रा॒ये न्द्रा॑य प्यायताम् प्यायता॒ मिन्द्रा॑य । \newline
8. इन्द्रा॑यैकधन॒विद॑ एकधन॒विद॒ इन्द्रा॒ये न्द्रा॑यैकधन॒विदे᳚ । \newline
9. ए॒क॒ध॒न॒विद॒ ऐक॑धन॒विद॑ एकधन॒विद॒ आ । \newline
10. ए॒क॒ध॒न॒विद॒ इत्ये॑कधन - विदे᳚ । \newline
11. आ तुभ्य॒म् तुभ्य॒ मा तुभ्य᳚म् । \newline
12. तुभ्य॒ मिन्द्र॒ इन्द्र॒स्तुभ्य॒म् तुभ्य॒ मिन्द्रः॑ । \newline
13. इन्द्रः॑ प्यायताम् प्यायता॒ मिन्द्र॒ इन्द्रः॑ प्यायताम् । \newline
14. प्या॒य॒ता॒ मा प्या॑यताम् प्यायता॒ मा । \newline
15. आ त्वम् त्व मा त्वम् । \newline
16. त्व मिन्द्रा॒ये न्द्रा॑य॒ त्वम् त्व मिन्द्रा॑य । \newline
17. इन्द्रा॑य प्यायस्व प्याय॒स्वे न्द्रा॒ये न्द्रा॑य प्यायस्व । \newline
18. प्या॒य॒स्वा प्या॑यस्व प्याय॒स्वा । \newline
19. आ प्या॑यय प्याय॒या प्या॑यय । \newline
20. प्या॒य॒य॒ सखी॒न् थ्सखी᳚न् प्यायय प्यायय॒ सखीन्॑ । \newline
21. सखी᳚न् थ्स॒न्या स॒न्या सखी॒न् थ्सखी᳚न् थ्स॒न्या । \newline
22. स॒न्या मे॒धया॑ मे॒धया॑ स॒न्या स॒न्या मे॒धया᳚ । \newline
23. मे॒धया᳚ स्व॒स्ति स्व॒स्ति मे॒धया॑ मे॒धया᳚ स्व॒स्ति । \newline
24. स्व॒स्ति ते॑ ते स्व॒स्ति स्व॒स्ति ते᳚ । \newline
25. ते॒ दे॒व॒ दे॒व॒ ते॒ ते॒ दे॒व॒ । \newline
26. दे॒व॒ सो॒म॒ सो॒म॒ दे॒व॒ दे॒व॒ सो॒म॒ । \newline
27. सो॒म॒ सु॒त्याꣳ सु॒त्याꣳ सो॑म सोम सु॒त्याम् । \newline
28. सु॒त्या म॑शीयाशीय सु॒त्याꣳ सु॒त्या म॑शीय । \newline
29. अ॒शी॒येष्ट॒ रेष्ट॑ रशीयाशी॒येष्टः॑ । \newline
30. एष्टा॒ रायो॒ राय॒ एष्ट॒ रेष्टा॒ रायः॑ । \newline
31. रायः॒ प्र प्र रायो॒ रायः॒ प्र । \newline
32. प्रे ष इ॒षे प्र प्रे षे । \newline
33. इ॒षे भगा॑य॒ भगा॑ये॒ ष इ॒षे भगा॑य । \newline
34. भगा॑य॒ र्त मृ॒तम् भगा॑य॒ भगा॑य॒ र्तम् । \newline
35. ऋ॒त मृ॑तवा॒दिभ्य॑ ऋतवा॒दिभ्य॑ ऋ॒त मृ॒त मृ॑तवा॒दिभ्यः॑ । \newline
36. ऋ॒त॒वा॒दिभ्यो॒ नमो॒ नम॑ ऋतवा॒दिभ्य॑ ऋतवा॒दिभ्यो॒ नमः॑ । \newline
37. ऋ॒त॒वा॒दिभ्य॒ इत्यृ॑तवा॒दि - भ्यः॒ । \newline
38. नमो॑ दि॒वे दि॒वे नमो॒ नमो॑ दि॒वे । \newline
39. दि॒वे नमो॒ नमो॑ दि॒वे दि॒वे नमः॑ । \newline
40. नमः॑ पृथि॒व्यै पृ॑थि॒व्यै नमो॒ नमः॑ पृथि॒व्यै । \newline
41. पृ॒थि॒व्या अग्ने ऽग्ने॑ पृथि॒व्यै पृ॑थि॒व्या अग्ने᳚ । \newline
42. अग्ने᳚ व्रतपते व्रतप॒ते ऽग्ने ऽग्ने᳚ व्रतपते । \newline
43. व्र॒त॒प॒ते॒ त्वम् त्वं ॅव्र॑तपते व्रतपते॒ त्वम् । \newline
44. व्र॒त॒प॒त॒ इति॑ व्रत - प॒ते॒ । \newline
45. त्वं ॅव्र॒तानां᳚ ॅव्र॒ताना॒म् त्वम् त्वं ॅव्र॒ताना᳚म् । \newline
46. व्र॒तानां᳚ ॅव्र॒तप॑तिर् व्र॒तप॑तिर् व्र॒तानां᳚ ॅव्र॒तानां᳚ ॅव्र॒तप॑तिः । \newline
47. व्र॒तप॑तिरस्यसि व्र॒तप॑तिर् व्र॒तप॑तिरसि । \newline
48. व्र॒तप॑ति॒रिति॑ व्र॒त - प॒तिः॒ । \newline
49. अ॒सि॒ या या ऽस्य॑सि॒ या । \newline
50. या मम॒ मम॒ या या मम॑ । \newline
51. मम॑ त॒नू स्त॒नूर् मम॒ मम॑ त॒नूः । \newline
52. त॒नू रे॒षैषा त॒नू स्त॒नू रे॒षा । \newline
53. ए॒षा सा सैषैषा सा । \newline
54. सा त्वयि॒ त्वयि॒ सा सा त्वयि॑ । \newline
55. त्वयि॒ या या त्वयि॒ त्वयि॒ या । \newline

\textbf{Ghana Paata } \newline

1. अ॒(ग्म्॒)शुर(ग्म्॑)शु स्ते ते अ॒(ग्म्॒)शुर(ग्म्॑)शु र॒(ग्म्॒)शुर(ग्म्॑)शु स्ते देव देव ते अ॒(ग्म्॒)शुर(ग्म्॑)शु र॒(ग्म्॒)शुर(ग्म्॑)शु स्ते देव । \newline
2. अ॒(ग्म्॒)शुर(ग्म्॑)शु॒रित्य॒(ग्म्॒)शुः - अ॒(ग्म्॒)शुः॒ । \newline
3. ते॒ दे॒व॒ दे॒व॒ ते॒ ते॒ दे॒व॒ सो॒म॒ सो॒म॒ दे॒व॒ ते॒ ते॒ दे॒व॒ सो॒म॒ । \newline
4. दे॒व॒ सो॒म॒ सो॒म॒ दे॒व॒ दे॒व॒ सो॒मा सो॑म देव देव सो॒मा । \newline
5. सो॒मा सो॑म सो॒मा प्या॑यताम् प्यायता॒ मा सो॑म सो॒मा प्या॑यताम् । \newline
6. आ प्या॑यताम् प्यायता॒ मा प्या॑यता॒ मिन्द्रा॒ये न्द्रा॑य प्यायता॒ मा प्या॑यता॒ मिन्द्रा॑य । \newline
7. प्या॒य॒ता॒ मिन्द्रा॒ये न्द्रा॑य प्यायताम् प्यायता॒ मिन्द्रा॑यैकधन॒विद॑ एकधन॒विद॒ इन्द्रा॑य प्यायताम् प्यायता॒ मिन्द्रा॑यैकधन॒विदे᳚ । \newline
8. इन्द्रा॑यैकधन॒विद॑ एकधन॒विद॒ इन्द्रा॒ये न्द्रा॑यैकधन॒विद॒ ऐक॑धन॒विद॒ इन्द्रा॒ये न्द्रा॑यैकधन॒विद॒ आ । \newline
9. ए॒क॒ध॒न॒विद॒ ऐक॑धन॒विद॑ एकधन॒विद॒ आ तुभ्य॒म् तुभ्य॒ मैक॑धन॒विद॑ एकधन॒विद॒ आ तुभ्य᳚म् । \newline
10. ए॒क॒ध॒न॒विद॒ इत्ये॑कधन - विदे᳚ । \newline
11. आ तुभ्य॒म् तुभ्य॒ मा तुभ्य॒ मिन्द्र॒ इन्द्र॒स्तुभ्य॒ मा तुभ्य॒ मिन्द्रः॑ । \newline
12. तुभ्य॒ मिन्द्र॒ इन्द्र॒स्तुभ्य॒म् तुभ्य॒ मिन्द्रः॑ प्यायताम् प्यायता॒ मिन्द्र॒स्तुभ्य॒म् तुभ्य॒ मिन्द्रः॑ प्यायताम् । \newline
13. इन्द्रः॑ प्यायताम् प्यायता॒ मिन्द्र॒ इन्द्रः॑ प्यायता॒ मा प्या॑यता॒ मिन्द्र॒ इन्द्रः॑ प्यायता॒ मा । \newline
14. प्या॒य॒ता॒ मा प्या॑यताम् प्यायता॒ मा त्वम् त्व मा प्या॑यताम् प्यायता॒ मा त्वम् । \newline
15. आ त्वम् त्व मा त्व मिन्द्रा॒ये न्द्रा॑य॒ त्व मा त्व मिन्द्रा॑य । \newline
16. त्व मिन्द्रा॒ये न्द्रा॑य॒ त्वम् त्व मिन्द्रा॑य प्यायस्व प्याय॒स्वे न्द्रा॑य॒ त्वम् त्व मिन्द्रा॑य प्यायस्व । \newline
17. इन्द्रा॑य प्यायस्व प्याय॒स्वे न्द्रा॒ये न्द्रा॑य प्याय॒स्वा प्या॑य॒स्वे न्द्रा॒ये न्द्रा॑य प्याय॒स्वा । \newline
18. प्या॒य॒स्वा प्या॑यस्व प्याय॒स्वा प्या॑यय प्याय॒या प्या॑यस्व प्याय॒स्वा प्या॑यय । \newline
19. आ प्या॑यय प्याय॒या प्या॑यय॒ सखी॒न् थ्सखी᳚न् प्याय॒या प्या॑यय॒ सखीन्॑ । \newline
20. प्या॒य॒य॒ सखी॒न् थ्सखी᳚न् प्यायय प्यायय॒ सखी᳚न् थ्स॒न्या स॒न्या सखी᳚न् प्यायय प्यायय॒ सखी᳚न् थ्स॒न्या । \newline
21. सखी᳚न् थ्स॒न्या स॒न्या सखी॒न् थ्सखी᳚न् थ्स॒न्या मे॒धया॑ मे॒धया॑ स॒न्या सखी॒न् थ्सखी᳚न् थ्स॒न्या मे॒धया᳚ । \newline
22. स॒न्या मे॒धया॑ मे॒धया॑ स॒न्या स॒न्या मे॒धया᳚ स्व॒स्ति स्व॒स्ति मे॒धया॑ स॒न्या स॒न्या मे॒धया᳚ स्व॒स्ति । \newline
23. मे॒धया᳚ स्व॒स्ति स्व॒स्ति मे॒धया॑ मे॒धया᳚ स्व॒स्ति ते॑ ते स्व॒स्ति मे॒धया॑ मे॒धया᳚ स्व॒स्ति ते᳚ । \newline
24. स्व॒स्ति ते॑ ते स्व॒स्ति स्व॒स्ति ते॑ देव देव ते स्व॒स्ति स्व॒स्ति ते॑ देव । \newline
25. ते॒ दे॒व॒ दे॒व॒ ते॒ ते॒ दे॒व॒ सो॒म॒ सो॒म॒ दे॒व॒ ते॒ ते॒ दे॒व॒ सो॒म॒ । \newline
26. दे॒व॒ सो॒म॒ सो॒म॒ दे॒व॒ दे॒व॒ सो॒म॒ सु॒त्याꣳ सु॒त्याꣳ सो॑म देव देव सोम सु॒त्याम् । \newline
27. सो॒म॒ सु॒त्याꣳ सु॒त्याꣳ सो॑म सोम सु॒त्या म॑शीयाशीय सु॒त्याꣳ सो॑म सोम सु॒त्या म॑शीय । \newline
28. सु॒त्या म॑शीयाशीय सु॒त्याꣳ सु॒त्या म॑शी॒येष्ट॒ रेष्ट॑ रशीय सु॒त्याꣳ सु॒त्या म॑शी॒येष्टः॑ । \newline
29. अ॒शी॒येष्ट॒ रेष्ट॑ रशीयाशी॒ येष्टा॒ रायो॒ राय॒ एष्ट॑ रशीयाशी॒ येष्टा॒ रायः॑ । \newline
30. एष्टा॒ रायो॒ राय॒ एष्ट॒ रेष्टा॒ रायः॒ प्र प्र राय॒ एष्ट॒ रेष्टा॒ रायः॒ प्र । \newline
31. रायः॒ प्र प्र रायो॒ रायः॒ प्रे ष इ॒षे प्र रायो॒ रायः॒ प्रे षे । \newline
32. प्रे ष इ॒षे प्र प्रे षे भगा॑य॒ भगा॑ये॒ षे प्र प्रे षे भगा॑य । \newline
33. इ॒षे भगा॑य॒ भगा॑ये॒ ष इ॒षे भगा॑य॒र्त मृ॒तम् भगा॑ये॒ ष इ॒षे भगा॑य॒र्तम् । \newline
34. भगा॑य॒र्त मृ॒तम् भगा॑य॒ भगा॑य॒ र्त मृ॑तवा॒दिभ्य॑ ऋतवा॒दिभ्य॑ ऋ॒तम् भगा॑य॒ भगा॑य॒र्त मृ॑तवा॒दिभ्यः॑ । \newline
35. ऋ॒त मृ॑तवा॒दिभ्य॑ ऋतवा॒दिभ्य॑ ऋ॒त मृ॒त मृ॑तवा॒दिभ्यो॒ नमो॒ नम॑ ऋतवा॒दिभ्य॑ ऋ॒त मृ॒त मृ॑तवा॒दिभ्यो॒ नमः॑ । \newline
36. ऋ॒त॒वा॒दिभ्यो॒ नमो॒ नम॑ ऋतवा॒दिभ्य॑ ऋतवा॒दिभ्यो॒ नमो॑ दि॒वे दि॒वे नम॑ ऋतवा॒दिभ्य॑ ऋतवा॒दिभ्यो॒ नमो॑ दि॒वे । \newline
37. ऋ॒त॒वा॒दिभ्य॒ इत्यृ॑तवा॒दि - भ्यः॒ । \newline
38. नमो॑ दि॒वे दि॒वे नमो॒ नमो॑ दि॒वे नमो॒ नमो॑ दि॒वे नमो॒ नमो॑ दि॒वे नमः॑ । \newline
39. दि॒वे नमो॒ नमो॑ दि॒वे दि॒वे नमः॑ पृथि॒व्यै पृ॑थि॒व्यै नमो॑ दि॒वे दि॒वे नमः॑ पृथि॒व्यै । \newline
40. नमः॑ पृथि॒व्यै पृ॑थि॒व्यै नमो॒ नमः॑ पृथि॒व्या अग्ने ऽग्ने॑ पृथि॒व्यै नमो॒ नमः॑ पृथि॒व्या अग्ने᳚ । \newline
41. पृ॒थि॒व्या अग्ने ऽग्ने॑ पृथि॒व्यै पृ॑थि॒व्या अग्ने᳚ व्रतपते व्रतप॒ते ऽग्ने॑ पृथि॒व्यै पृ॑थि॒व्या अग्ने᳚ व्रतपते । \newline
42. अग्ने᳚ व्रतपते व्रतप॒ते ऽग्ने ऽग्ने᳚ व्रतपते॒ त्वम् त्वं ॅव्र॑तप॒ते ऽग्ने ऽग्ने᳚ व्रतपते॒ त्वम् । \newline
43. व्र॒त॒प॒ते॒ त्वम् त्वं ॅव्र॑तपते व्रतपते॒ त्वं ॅव्र॒तानां᳚ ॅव्र॒ताना॒म् त्वं ॅव्र॑तपते व्रतपते॒ त्वं ॅव्र॒ताना᳚म् । \newline
44. व्र॒त॒प॒त॒ इति॑ व्रत - प॒ते॒ । \newline
45. त्वं ॅव्र॒तानां᳚ ॅव्र॒ताना॒म् त्वम् त्वं ॅव्र॒तानां᳚ ॅव्र॒तप॑तिर् व्र॒तप॑तिर् व्र॒ताना॒म् त्वम् त्वं ॅव्र॒तानां᳚ ॅव्र॒तप॑तिः । \newline
46. व्र॒तानां᳚ ॅव्र॒तप॑तिर् व्र॒तप॑तिर् व्र॒तानां᳚ ॅव्र॒तानां᳚ ॅव्र॒तप॑ति रस्यसि व्र॒तप॑तिर् व्र॒तानां᳚ ॅव्र॒तानां᳚ ॅव्र॒तप॑तिरसि । \newline
47. व्र॒तप॑ति रस्यसि व्र॒तप॑तिर् व्र॒तप॑तिरसि॒ या या ऽसि॑ व्र॒तप॑तिर् व्र॒तप॑तिरसि॒ या । \newline
48. व्र॒तप॑ति॒रिति॑ व्र॒त - प॒तिः॒ । \newline
49. अ॒सि॒ या या ऽस्य॑सि॒ या मम॒ मम॒ या ऽस्य॑सि॒ या मम॑ । \newline
50. या मम॒ मम॒ या या मम॑ त॒नू स्त॒नूर् मम॒ या या मम॑ त॒नूः । \newline
51. मम॑ त॒नू स्त॒नूर् मम॒ मम॑ त॒नूरे॒षैषा त॒नूर् मम॒ मम॑ त॒नूरे॒षा । \newline
52. त॒नूरे॒षैषा त॒नूस्त॒नूरे॒षा सा सैषा त॒नूस्त॒नूरे॒षा सा । \newline
53. ए॒षा सा सैषैषा सा त्वयि॒ त्वयि॒ सैषैषा सा त्वयि॑ । \newline
54. सा त्वयि॒ त्वयि॒ सा सा त्वयि॒ या या त्वयि॒ सा सा त्वयि॒ या । \newline
55. त्वयि॒ या या त्वयि॒ त्वयि॒ या तव॒ तव॒ या त्वयि॒ त्वयि॒ या तव॑ । \newline
\pagebreak
\markright{ TS 1.2.11.2  \hfill https://www.vedavms.in \hfill}

\section{ TS 1.2.11.2 }

\textbf{TS 1.2.11.2 } \newline
\textbf{Samhita Paata} \newline

या तव॑ त॒नूरि॒यꣳ सा मयि॑ स॒ह नौ᳚ व्रतपते व्र॒तिनो᳚र् व्र॒तानि॒ या ते॑ अग्ने॒ रुद्रि॑या त॒नूस्तया॑ नः पाहि॒ तस्या᳚स्ते॒ स्वाहा॒ या ते॑ अग्नेऽयाश॒या र॑जाश॒या ह॑राश॒या त॒नूर्वर्.षि॑ष्ठा गह्वरे॒ष्ठोऽग्रं ॅवचो॒ अपा॑वधीं त्वे॒षं ॅवचो॒ अपा॑वधीꣳ॒॒ स्वाहा᳚ ॥ \newline

\textbf{Pada Paata} \newline

या । तव॑ । त॒नूः । इ॒यम् । सा । मयि॑ । स॒ह । नौ॒ । व्र॒त॒प॒त॒ इति॑ व्रत -प॒ते॒ । व्र॒तिनोः᳚ । व्र॒तानि॑ । या । ते॒ । अ॒ग्ने॒ । रुद्रि॑या । त॒नूः । तया᳚ । नः॒ । पा॒हि॒ । तस्याः᳚ । ते॒ । स्वाहा᳚ । या । ते॒ । अ॒ग्ने॒ । अ॒या॒श॒येत्य॑या - श॒या । र॒जा॒श॒येति॑ रजा - श॒या । ह॒रा॒श॒येति॑ हरा - श॒या । त॒नूः । वर्.षि॑ष्ठा । ग॒ह्व॒रे॒ष्ठेति॑ गह्वरे - स्था । उ॒ग्रम् । वचः॑ । अपेति॑ । अ॒व॒धी॒म् । त्वे॒षम् । वचः॑ । अपेति॑ । अ॒व॒धी॒म् । स्वाहा᳚ ॥  \newline


\textbf{Krama Paata} \newline

या तव॑ । तव॑ त॒नूः । त॒नूरि॒यम् । इ॒यꣳ सा । सा मयि॑ । मयि॑ स॒ह । स॒ह नौ᳚ । नौ॒ व्र॒त॒प॒ते॒ । व्र॒त॒प॒ते॒ व्र॒तिनोः᳚ । व्र॒त॒प॒त॒ इति॑ व्रत - प॒ते॒ । व्र॒तिनो᳚र् व्र॒तानि॑ । व्र॒तानि॒ या । या ते᳚ । ते॒ अ॒ग्ने॒ । अ॒ग्ने॒ रुद्रि॑या । रुद्रि॑या त॒नूः । त॒नूस्तया᳚ । तया॑ नः । नः॒ पा॒हि॒ । पा॒हि॒ तस्याः᳚ । तस्या᳚स्ते । ते॒ स्वाहा᳚ । स्वाहा॒ या । या ते᳚ । ते॒ अ॒ग्ने॒ । अ॒ग्ने॒ऽया॒श॒या । अ॒या॒श॒या र॑जाश॒या । अ॒या॒श॒येत्य॑या - श॒या । र॒जा॒श॒या ह॑राश॒या । र॒जा॒श॒येति॑ रजा - श॒या । ह॒रा॒श॒या त॒नूः । ह॒रा॒श॒येति॑ हरा - श॒या । त॒नूर् वर्.षि॑ष्ठा । वर्.षि॑ष्ठा गह्वरे॒ष्ठा । ग॒ह्व॒रे॒ष्ठोग्रम् । ग॒ह्व॒रे॒ष्ठेति॑ गह्वरे - स्था । उ॒ग्रं ॅवचः॑ । वचो॒ अप॑ । अपा॑वधीम् । अ॒व॒धी॒म् त्वे॒षम् । त्वे॒षं ॅवचः॑ । वचो॒ अप॑ । अपा॑वधीम् । अ॒व॒धीꣳ॒॒ स्वाहा᳚ । स्वाहेति॒ स्वाहा᳚ । \newline

\textbf{Jatai Paata} \newline

1. या तव॒ तव॒ या या तव॑ । \newline
2. तव॑ त॒नू स्त॒नू स्तव॒ तव॑ त॒नूः । \newline
3. त॒नूरि॒य मि॒यम् त॒नू स्त॒नू रि॒यम् । \newline
4. इ॒यꣳ सा सेय मि॒यꣳ सा । \newline
5. सा मयि॒ मयि॒ सा सा मयि॑ । \newline
6. मयि॑ स॒ह स॒ह मयि॒ मयि॑ स॒ह । \newline
7. स॒ह नौ॑ नौ स॒ह स॒ह नौ᳚ । \newline
8. नौ॒ व्र॒त॒प॒ते॒ व्र॒त॒प॒ते॒ नौ॒ नौ॒ व्र॒त॒प॒ते॒ । \newline
9. व्र॒त॒प॒ते॒ व्र॒तिनो᳚र् व्र॒तिनो᳚र् व्रतपते व्रतपते व्र॒तिनोः᳚ । \newline
10. व्र॒त॒प॒त॒ इति॑ व्रत - प॒ते॒ । \newline
11. व्र॒तिनो᳚र् व्र॒तानि॑ व्र॒तानि॑ व्र॒तिनो᳚र् व्र॒तिनो᳚र् व्र॒तानि॑ । \newline
12. व्र॒तानि॒ या या व्र॒तानि॑ व्र॒तानि॒ या । \newline
13. या ते॑ ते॒ या या ते᳚ । \newline
14. ते॒ अ॒ग्ने॒ ऽग्ने॒ ते॒ ते॒ अ॒ग्ने॒ । \newline
15. अ॒ग्ने॒ रुद्रि॑या॒ रुद्रि॑या ऽग्ने ऽग्ने॒ रुद्रि॑या । \newline
16. रुद्रि॑या त॒नू स्त॒नू रुद्रि॑या॒ रुद्रि॑या त॒नूः । \newline
17. त॒नू स्तया॒ तया॑ त॒नू स्त॒नू स्तया᳚ । \newline
18. तया॑ नो न॒ स्तया॒ तया॑ नः । \newline
19. नः॒ पा॒हि॒ पा॒हि॒ नो॒ नः॒ पा॒हि॒ । \newline
20. पा॒हि॒ तस्या॒ स्तस्याः᳚ पाहि पाहि॒ तस्याः᳚ । \newline
21. तस्या᳚ स्ते ते॒ तस्या॒ स्तस्या᳚ स्ते । \newline
22. ते॒ स्वाहा॒ स्वाहा॑ ते ते॒ स्वाहा᳚ । \newline
23. स्वाहा॒ या या स्वाहा॒ स्वाहा॒ या । \newline
24. या ते॑ ते॒ या या ते᳚ । \newline
25. ते॒ अ॒ग्ने॒ ऽग्ने॒ ते॒ ते॒ अ॒ग्ने॒ । \newline
26. अ॒ग्ने॒ ऽया॒श॒या ऽया॑श॒या ऽग्ने᳚ ऽग्ने ऽयाश॒या । \newline
27. अ॒या॒श॒या र॑जाश॒या र॑जाश॒या ऽया॑श॒या ऽया॑श॒या र॑जाश॒या । \newline
28. अ॒या॒श॒येत्य॑या - श॒या । \newline
29. र॒जा॒श॒या ह॑राश॒या ह॑राश॒या र॑जाश॒या र॑जाश॒या ह॑राश॒या । \newline
30. र॒जा॒श॒येति॑ रजा - श॒या । \newline
31. ह॒रा॒श॒या त॒नू स्त॒नूर्. ह॑राश॒या ह॑राश॒या त॒नूः । \newline
32. ह॒रा॒श॒येति॑ हरा - श॒या । \newline
33. त॒नूर् वर्.षि॑ष्ठा॒ वर्.षि॑ष्ठा त॒नू स्त॒नूर् वर्.षि॑ष्ठा । \newline
34. वर्.षि॑ष्ठा गह्वरे॒ष्ठा ग॑ह्वरे॒ष्ठा वर्.षि॑ष्ठा॒ वर्.षि॑ष्ठा गह्वरे॒ष्ठा । \newline
35. ग॒ह्व॒रे॒ष्ठोग्र मु॒ग्रम् ग॑ह्वरे॒ष्ठा ग॑ह्वरे॒ष्ठोग्रम् । \newline
36. ग॒ह्व॒रे॒ष्ठेति॑ गह्वरे - स्था । \newline
37. उ॒ग्रं ॅवचो॒ वच॑ उ॒ग्र मु॒ग्रं ॅवचः॑ । \newline
38. वचो॒ अपाप॒ वचो॒ वचो॒ अप॑ । \newline
39. अपा॑वधी मवधी॒ मपापा॑वधीम् । \newline
40. अ॒व॒धी॒म् त्वे॒षम् त्वे॒ष म॑वधी मवधीम् त्वे॒षम् । \newline
41. त्वे॒षं ॅवचो॒ वच॑स्त्वे॒षम् त्वे॒षं ॅवचः॑ । \newline
42. वचो॒ अपाप॒ वचो॒ वचो॒ अप॑ । \newline
43. अपा॑वधी मवधी॒ मपापा॑वधीम् । \newline
44. अ॒व॒धी॒(ग्ग्॒) स्वाहा॒ स्वाहा॑ ऽवधी मवधी॒(ग्ग्॒) स्वाहा᳚ । \newline
45. स्वाहेति॒ स्वाहा᳚ । \newline

\textbf{Ghana Paata } \newline

1. या तव॒ तव॒ या या तव॑ त॒नू स्त॒नू स्तव॒ या या तव॑ त॒नूः । \newline
2. तव॑ त॒नू स्त॒नू स्तव॒ तव॑ त॒नूरि॒य मि॒यम् त॒नूस्तव॒ तव॑ त॒नूरि॒यम् । \newline
3. त॒नूरि॒य मि॒यम् त॒नू स्त॒नू रि॒यꣳ सा सेयम् त॒नू स्त॒नू रि॒यꣳ सा । \newline
4. इ॒यꣳ सा सेय मि॒यꣳ सा मयि॒ मयि॒ सेय मि॒यꣳ सा मयि॑ । \newline
5. सा मयि॒ मयि॒ सा सा मयि॑ स॒ह स॒ह मयि॒ सा सा मयि॑ स॒ह । \newline
6. मयि॑ स॒ह स॒ह मयि॒ मयि॑ स॒ह नौ॑ नौ स॒ह मयि॒ मयि॑ स॒ह नौ᳚ । \newline
7. स॒ह नौ॑ नौ स॒ह स॒ह नौ᳚ व्रतपते व्रतपते नौ स॒ह स॒ह नौ᳚ व्रतपते । \newline
8. नौ॒ व्र॒त॒प॒ते॒ व्र॒त॒प॒ते॒ नौ॒ नौ॒ व्र॒त॒प॒ते॒ व्र॒तिनो᳚र् व्र॒तिनो᳚र् व्रतपते नौ नौ व्रतपते व्र॒तिनोः᳚ । \newline
9. व्र॒त॒प॒ते॒ व्र॒तिनो᳚र् व्र॒तिनो᳚र् व्रतपते व्रतपते व्र॒तिनो᳚र् व्र॒तानि॑ व्र॒तानि॑ व्र॒तिनो᳚र् व्रतपते व्रतपते व्र॒तिनो᳚र् व्र॒तानि॑ । \newline
10. व्र॒त॒प॒त॒ इति॑ व्रत - प॒ते॒ । \newline
11. व्र॒तिनो᳚र् व्र॒तानि॑ व्र॒तानि॑ व्र॒तिनो᳚र् व्र॒तिनो᳚र् व्र॒तानि॒ या या व्र॒तानि॑ व्र॒तिनो᳚र् व्र॒तिनो᳚र् व्र॒तानि॒ या । \newline
12. व्र॒तानि॒ या या व्र॒तानि॑ व्र॒तानि॒ या ते॑ ते॒ या व्र॒तानि॑ व्र॒तानि॒ या ते᳚ । \newline
13. या ते॑ ते॒ या या ते॑ अग्ने ऽग्ने ते॒ या या ते॑ अग्ने । \newline
14. ते॒ अ॒ग्ने॒ ऽग्ने॒ ते॒ ते॒ अ॒ग्ने॒ रुद्रि॑या॒ रुद्रि॑या ऽग्ने ते ते अग्ने॒ रुद्रि॑या । \newline
15. अ॒ग्ने॒ रुद्रि॑या॒ रुद्रि॑या ऽग्ने ऽग्ने॒ रुद्रि॑या त॒नू स्त॒नू रुद्रि॑या ऽग्ने ऽग्ने॒ रुद्रि॑या त॒नूः । \newline
16. रुद्रि॑या त॒नू स्त॒नू रुद्रि॑या॒ रुद्रि॑या त॒नू स्तया॒ तया॑ त॒नू रुद्रि॑या॒ रुद्रि॑या त॒नूस्तया᳚ । \newline
17. त॒नूस्तया॒ तया॑ त॒नू स्त॒नू स्तया॑ नो न॒स्तया॑ त॒नू स्त॒नू स्तया॑ नः । \newline
18. तया॑ नो न॒स्तया॒ तया॑ नः पाहि पाहि न॒स्तया॒ तया॑ नः पाहि । \newline
19. नः॒ पा॒हि॒ पा॒हि॒ नो॒ नः॒ पा॒हि॒ तस्या॒ स्तस्याः᳚ पाहि नो नः पाहि॒ तस्याः᳚ । \newline
20. पा॒हि॒ तस्या॒ स्तस्याः᳚ पाहि पाहि॒ तस्या᳚ स्ते ते॒ तस्याः᳚ पाहि पाहि॒ तस्या᳚ स्ते । \newline
21. तस्या᳚स्ते ते॒ तस्या॒स्तस्या᳚स्ते॒ स्वाहा॒ स्वाहा॑ ते॒ तस्या॒स्तस्या᳚स्ते॒ स्वाहा᳚ । \newline
22. ते॒ स्वाहा॒ स्वाहा॑ ते ते॒ स्वाहा॒ या या स्वाहा॑ ते ते॒ स्वाहा॒ या । \newline
23. स्वाहा॒ या या स्वाहा॒ स्वाहा॒ या ते॑ ते॒ या स्वाहा॒ स्वाहा॒ या ते᳚ । \newline
24. या ते॑ ते॒ या या ते॑ अग्ने ऽग्ने ते॒ या या ते॑ अग्ने । \newline
25. ते॒ अ॒ग्ने॒ ऽग्ने॒ ते॒ ते॒ अ॒ग्ने॒ ऽया॒श॒या ऽया॑श॒या ऽग्ने॑ ते ते अग्ने ऽयाश॒या । \newline
26. अ॒ग्ने॒ ऽया॒श॒या ऽया॑श॒या ऽग्ने᳚ ऽग्ने ऽयाश॒या र॑जाश॒या र॑जाश॒या ऽया॑श॒या ऽग्ने᳚ ऽग्ने ऽयाश॒या र॑जाश॒या । \newline
27. अ॒या॒श॒या र॑जाश॒या र॑जाश॒या ऽया॑श॒या ऽया॑श॒या र॑जाश॒या ह॑राश॒या ह॑राश॒या र॑जाश॒या ऽया॑श॒या ऽया॑श॒या र॑जाश॒या ह॑राश॒या । \newline
28. अ॒या॒श॒येत्य॑या - श॒या । \newline
29. र॒जा॒श॒या ह॑राश॒या ह॑राश॒या र॑जाश॒या र॑जाश॒या ह॑राश॒या त॒नूस्त॒नूर्. ह॑राश॒या र॑जाश॒या र॑जाश॒या ह॑राश॒या त॒नूः । \newline
30. र॒जा॒श॒येति॑ रजा - श॒या । \newline
31. ह॒रा॒श॒या त॒नूस्त॒नूर्. ह॑राश॒या ह॑राश॒या त॒नूर् वर्.षि॑ष्ठा॒ वर्.षि॑ष्ठा त॒नूर्. ह॑राश॒या ह॑राश॒या त॒नूर् वर्.षि॑ष्ठा । \newline
32. ह॒रा॒श॒येति॑ हरा - श॒या । \newline
33. त॒नूर् वर्.षि॑ष्ठा॒ वर्.षि॑ष्ठा त॒नूस्त॒नूर् वर्.षि॑ष्ठा गह्वरे॒ष्ठा ग॑ह्वरे॒ष्ठा वर्.षि॑ष्ठा त॒नूस्त॒नूर् वर्.षि॑ष्ठा गह्वरे॒ष्ठा । \newline
34. वर्.षि॑ष्ठा गह्वरे॒ष्ठा ग॑ह्वरे॒ष्ठा वर्.षि॑ष्ठा॒ वर्.षि॑ष्ठा गह्वरे॒ष्ठोग्र मु॒ग्रम् ग॑ह्वरे॒ष्ठा वर्.षि॑ष्ठा॒ वर्.षि॑ष्ठा गह्वरे॒ष्ठोग्रम् । \newline
35. ग॒ह्व॒रे॒ष्ठोग्र मु॒ग्रम् ग॑ह्वरे॒ष्ठा ग॑ह्वरे॒ष्ठोग्रं ॅवचो॒ वच॑ उ॒ग्रम् ग॑ह्वरे॒ष्ठा ग॑ह्वरे॒ष्ठोग्रं ॅवचः॑ । \newline
36. ग॒ह्व॒रे॒ष्ठेति॑ गह्वरे - स्था । \newline
37. उ॒ग्रं ॅवचो॒ वच॑ उ॒ग्र मु॒ग्रं ॅवचो॒ अपाप॒ वच॑ उ॒ग्र मु॒ग्रं ॅवचो॒ अप॑ । \newline
38. वचो॒ अपाप॒ वचो॒ वचो॒ अपा॑वधी मवधी॒ मप॒ वचो॒ वचो॒ अपा॑वधीम् । \newline
39. अपा॑वधी मवधी॒ मपापा॑वधीम् त्वे॒षम् त्वे॒ष म॑वधी॒ मपापा॑वधीम् त्वे॒षम् । \newline
40. अ॒व॒धी॒म् त्वे॒षम् त्वे॒ष म॑वधी मवधीम् त्वे॒षं ॅवचो॒ वच॑स्त्वे॒ष म॑वधी मवधीम् त्वे॒षं ॅवचः॑ । \newline
41. त्वे॒षं ॅवचो॒ वच॑स्त्वे॒षम् त्वे॒षं ॅवचो॒ अपाप॒ वच॑स्त्वे॒षम् त्वे॒षं ॅवचो॒ अप॑ । \newline
42. वचो॒ अपाप॒ वचो॒ वचो॒ अपा॑वधी मवधी॒ मप॒ वचो॒ वचो॒ अपा॑वधीम् । \newline
43. अपा॑वधी मवधी॒ मपापा॑वधी॒(ग्ग्॒) स्वाहा॒ स्वाहा॑ ऽवधी॒ मपापा॑वधी॒(ग्ग्॒) स्वाहा᳚ । \newline
44. अ॒व॒धी॒(ग्ग्॒) स्वाहा॒ स्वाहा॑ ऽवधी मवधी॒(ग्ग्॒) स्वाहा᳚ । \newline
45. स्वाहेति॒ स्वाहा᳚ । \newline
\pagebreak
\markright{ TS 1.2.12.1  \hfill https://www.vedavms.in \hfill}

\section{ TS 1.2.12.1 }

\textbf{TS 1.2.12.1 } \newline
\textbf{Samhita Paata} \newline

वि॒त्ताय॑नी मेऽसि ति॒क्ताय॑नी मे॒ऽस्यव॑तान्मा नाथि॒तमव॑तान्मा व्यथि॒तं ॅवि॒देर॒ग्निर्नभो॒ नामाग्ने॑ अङ्गिरो॒ यो᳚ऽस्यां पृ॑थि॒व्यामस्यायु॑षा॒ नाम्नेहि॒ यत्तेऽना॑धृष्टं॒ नाम॑ य॒ज्ञियं॒ तेन॒ त्वाऽऽ*द॒धेऽग्ने॑ अङ्गिरो॒ यो द्वि॒तीय॑स्यां तृ॒तीय॑स्यां पृथि॒व्या-मस्यायु॑षा॒ नाम्नेहि॒ यत्तेऽना॑धृष्टं॒ नाम॑-[ ] \newline

\textbf{Pada Paata} \newline

वि॒त्ताय॒नीति॑ वित्त - अय॑नी । मे॒ । अ॒सि॒ । ति॒क्ताय॒नीति॑ तिक्त - अय॑नी । मे॒ । अ॒सि॒ । अव॑तात् । मा॒ । ना॒थि॒तम् । अव॑तात् । मा॒ । व्य॒थि॒तम् । वि॒देः । अ॒ग्निः । नभः॑ । नाम॑ । अग्ने᳚ । अ॒ङ्गि॒रः॒ । यः । अ॒स्याम् । पृ॒थि॒व्याम् । असि॑ । आयु॑षा । नाम्ना᳚ । एति॑ । इ॒हि॒ । यत् । ते॒ । अना॑धृष्ट॒मित्यना᳚ - धृ॒ष्ट॒म् । नाम॑ । य॒ज्ञिय᳚म् । तेन॑ । त्वा॒ । एति॑ । द॒धे॒ । अग्ने᳚ । अ॒ङ्गि॒रः॒ । यः । द्वि॒तीय॑स्याम् । तृ॒तीय॑स्याम् । पृ॒थि॒व्याम् । असि॑ । आयु॑षा । नाम्ना᳚ । एति॑ । इ॒हि॒ । यत् । ते॒ । अना॑धृष्ट॒मित्यना᳚ - धृ॒ष्ट॒म् । नाम॑ ।  \newline


\textbf{Krama Paata} \newline

वि॒त्ताय॑नी मे । वि॒त्ताय॒नीति॑ वित्त - अय॑नी । मे॒ऽसि॒ । अ॒सि॒ ति॒क्ताय॑नी । ति॒क्ताय॑नी मे । ति॒क्ताय॒नीति॑ तिक्त - अय॑नी । मे॒ऽसि॒ । अ॒स्यव॑तात् । अव॑तान्मा । मा॒ ना॒थि॒तम् । ना॒थि॒तमव॑तात् । अव॑तान्मा । मा॒ व्य॒थि॒तम् । व्य॒थि॒तं ॅवि॒देः । वि॒देर॒ग्निः । अ॒ग्निर् नभः॑ । नभो॒ नाम॑ । नामाग्ने᳚ । अग्ने॑ अङ्गिरः । अ॒ङ्गि॒रो॒ यः । यो᳚ऽस्याम् । अ॒स्याम् पृ॑थि॒व्याम् । पृ॒थि॒व्यामसि॑ । अस्यायु॑षा । आयु॑षा॒ नाम्ना᳚ । नाम्ना । एहि॑ । इ॒हि॒ यत् । यत्ते᳚ । तेऽना॑धृष्टम् । अना॑धृष्ट॒म् नाम॑ । अना॑धृष्ट॒मित्यना᳚ - धृ॒ष्ट॒म् । नाम॑ य॒ज्ञिय᳚म् । य॒ज्ञिय॒म् तेन॑ । तेन॑ त्वा । त्वा । आ द॑धे । द॒धेऽग्ने᳚ । अग्ने॑ अङ्गिरः । अ॒ङ्गि॒रो॒ यः । यो द्वि॒तीय॑स्याम् । द्वि॒तीय॑स्याम् तृ॒तीय॑स्याम् । तृ॒तीय॑स्याम् पृथि॒व्याम् । पृ॒थि॒व्यामसि॑ । अस्यायु॑षा । आयु॑षा॒ नाम्ना᳚ । नाम्ना । एहि॑ । इ॒हि॒ यत् । यत्ते᳚ । तेऽना॑धृष्टम् । अना॑धृष्ट॒म् नाम॑ । अना॑धृष्ट॒मित्यना᳚ - धृ॒ष्ट॒म् । नाम॑ य॒ज्ञिय᳚म् \newline

\textbf{Jatai Paata} \newline

1. वि॒त्ताय॑नी मे मे वि॒त्ताय॑नी वि॒त्ताय॑नी मे । \newline
2. वि॒त्ताय॒नीति॑ वित्त - अय॑नी । \newline
3. मे॒ ऽस्य॒सि॒ मे॒ मे॒ ऽसि॒ । \newline
4. अ॒सि॒ ति॒क्ताय॑नी ति॒क्ताय॑ न्यस्यसि ति॒क्ताय॑नी । \newline
5. ति॒क्ताय॑नी मे मे ति॒क्ताय॑नी ति॒क्ताय॑नी मे । \newline
6. ति॒क्ताय॒नीति॑ तिक्त - अय॑नी । \newline
7. मे॒ ऽस्य॒सि॒ मे॒ मे॒ ऽसि॒ । \newline
8. अ॒स्यव॑ता॒ दव॑ता दस्य॒ स्यव॑तात् । \newline
9. अव॑तान् मा॒ मा ऽव॑ता॒ दव॑तान् मा । \newline
10. मा॒ ना॒थि॒तम् ना॑थि॒तम् मा॑ मा नाथि॒तम् । \newline
11. ना॒थि॒त मव॑ता॒ दव॑तान् नाथि॒तम् ना॑थि॒त मव॑तात् । \newline
12. अव॑तान् मा॒ मा ऽव॑ता॒ दव॑तान् मा । \newline
13. मा॒ व्य॒थि॒तं ॅव्य॑थि॒तम् मा॑ मा व्यथि॒तम् । \newline
14. व्य॒थि॒तं ॅवि॒देर् वि॒देर् व्य॑थि॒तं ॅव्य॑थि॒तं ॅवि॒देः । \newline
15. वि॒दे र॒ग्नि र॒ग्निर् वि॒देर् वि॒दे र॒ग्निः । \newline
16. अ॒ग्निर् नभो॒ नभो॒ ऽग्नि र॒ग्निर् नभः॑ । \newline
17. नभो॒ नाम॒ नाम॒ नभो॒ नभो॒ नाम॑ । \newline
18. नामाग्ने ऽग्ने॒ नाम॒ नामाग्ने᳚ । \newline
19. अग्ने॑ अङ्गिरो अङ्गि॒रो ऽग्ने ऽग्ने॑ अङ्गिरः । \newline
20. अ॒ङ्गि॒रो॒ यो यो अ॑ङ्गिरो अङ्गिरो॒ यः । \newline
21. यो᳚ ऽस्या म॒स्यां ॅयो यो᳚ ऽस्याम् । \newline
22. अ॒स्याम् पृ॑थि॒व्याम् पृ॑थि॒व्या म॒स्या म॒स्याम् पृ॑थि॒व्याम् । \newline
23. पृ॒थि॒व्या मस्यसि॑ पृथि॒व्याम् पृ॑थि॒व्या मसि॑ । \newline
24. अस्यायु॒षा ऽऽयु॒षा ऽस्यस्यायु॑षा । \newline
25. आयु॑षा॒ नाम्ना॒ नाम्ना ऽऽयु॒षा ऽऽयु॑षा॒ नाम्ना᳚ । \newline
26. नाम्ना नाम्ना॒ नाम्ना । \newline
27. एही॒ह्येहि॑ । \newline
28. इ॒हि॒ यद् यदि॑हीहि॒ यत् । \newline
29. यत् ते॑ ते॒ यद् यत् ते᳚ । \newline
30. ते ऽना॑धृष्ट॒ मना॑धृष्टम् ते॒ ते ऽना॑धृष्टम् । \newline
31. अना॑धृष्ट॒म् नाम॒ नामाना॑धृष्ट॒ मना॑धृष्ट॒म् नाम॑ । \newline
32. अना॑धृष्ट॒मित्यना᳚ - धृ॒ष्ट॒म् । \newline
33. नाम॑ य॒ज्ञियं॑ ॅय॒ज्ञिय॒न्नाम॒ नाम॑ य॒ज्ञिय᳚म् । \newline
34. य॒ज्ञिय॒म् तेन॒ तेन॑ य॒ज्ञियं॑ ॅय॒ज्ञिय॒म् तेन॑ । \newline
35. तेन॑ त्वा त्वा॒ तेन॒ तेन॑ त्वा । \newline
36. त्वा ऽऽत्वा॒ त्वा । \newline
37. आ द॑धे दध॒ आ द॑धे । \newline
38. द॒धे ऽग्ने ऽग्ने॑ दधे द॒धे ऽग्ने᳚ । \newline
39. अग्ने॑ अङ्गिरो अङ्गि॒रो ऽग्ने ऽग्ने॑ अङ्गिरः । \newline
40. अ॒ङ्गि॒रो॒ यो यो अ॑ङ्गिरो अङ्गिरो॒ यः । \newline
41. यो द्वि॒तीय॑स्याम् द्वि॒तीय॑स्यां॒ ॅयो यो द्वि॒तीय॑स्याम् । \newline
42. द्वि॒तीय॑स्याम् तृ॒तीय॑स्याम् तृ॒तीय॑स्याम् द्वि॒तीय॑स्याम् द्वी॒तीय॑स्याम् तृ॒तीय॑स्याम् । \newline
43. तृ॒तीय॑स्याम् पृथि॒व्याम् पृ॑थि॒व्याम् तृ॒तीय॑स्याम् तृ॒तीय॑स्याम् पृथि॒व्याम् । \newline
44. पृ॒थि॒व्या मस्यसि॑ पृथि॒व्याम् पृ॑थि॒व्या मसि॑ । \newline
45. अस्यायु॒षा ऽऽयु॒षा ऽस्यस्यायु॑षा । \newline
46. आयु॑षा॒ नाम्ना॒ नाम्ना ऽऽयु॒षा ऽऽयु॑षा॒ नाम्ना᳚ । \newline
47. नाम्ना नाम्ना॒ नाम्ना । \newline
48. एही॒ह्येहि॑ । \newline
49. इ॒हि॒ यद् यदि॑हीहि॒ यत् । \newline
50. यत् ते॑ ते॒ यद् यत् ते᳚ । \newline
51. ते ऽना॑धृष्ट॒ मना॑धृष्टम् ते॒ ते ऽना॑धृष्टम् । \newline
52. अना॑धृष्ट॒म् नाम॒ नामाना॑धृष्ट॒ मना॑धृष्ट॒म् नाम॑ । \newline
53. अना॑धृष्ट॒मित्यना᳚ - धृ॒ष्ट॒म् । \newline
54. नाम॑ य॒ज्ञियं॑ ॅय॒ज्ञिय॒म् नाम॒ नाम॑ य॒ज्ञिय᳚म् । \newline

\textbf{Ghana Paata } \newline

1. वि॒त्ताय॑नी मे मे वि॒त्ताय॑नी वि॒त्ताय॑नी मे ऽस्यसि मे वि॒त्ताय॑नी वि॒त्ताय॑नी मे ऽसि । \newline
2. वि॒त्ताय॒नीति॑ वित्त - अय॑नी । \newline
3. मे॒ ऽस्य॒सि॒ मे॒ मे॒ ऽसि॒ ति॒क्ताय॑नी ति॒क्ताय॑न्यसि मे मे ऽसि ति॒क्ताय॑नी । \newline
4. अ॒सि॒ ति॒क्ताय॑नी ति॒क्ताय॑न्यस्यसि ति॒क्ताय॑नी मे मे ति॒क्ताय॑न्यस्यसि ति॒क्ताय॑नी मे । \newline
5. ति॒क्ताय॑नी मे मे ति॒क्ताय॑नी ति॒क्ताय॑नी मे ऽस्यसि मे ति॒क्ताय॑नी ति॒क्ताय॑नी मे ऽसि । \newline
6. ति॒क्ताय॒नीति॑ तिक्त - अय॑नी । \newline
7. मे॒ ऽस्य॒सि॒ मे॒ मे॒ ऽस्यव॑ता॒ दव॑ता दसि मे मे॒ ऽस्यव॑तात् । \newline
8. अ॒स्यव॑ता॒ दव॑ता दस्य॒स्यव॑तान् मा॒ मा ऽव॑ता दस्य॒स्यव॑तान् मा । \newline
9. अव॑तान् मा॒ मा ऽव॑ता॒दव॑तान् मा नाथि॒तन्ना॑थि॒तम् मा ऽव॑ता॒दव॑तान् मा नाथि॒तम् । \newline
10. मा॒ ना॒थि॒तन्ना॑थि॒तम् मा॑ मा नाथि॒त मव॑ता॒दव॑तान् नाथि॒तम् मा॑ मा नाथि॒त मव॑तात् । \newline
11. ना॒थि॒त मव॑ता॒दव॑तान् नाथि॒तन्ना॑थि॒त मव॑तान् मा॒ मा ऽव॑तान् नाथि॒तन्ना॑थि॒त मव॑तान् मा । \newline
12. अव॑तान् मा॒ मा ऽव॑ता॒दव॑तान् मा व्यथि॒तं ॅव्य॑थि॒तम् मा ऽव॑ता॒दव॑तान् मा व्यथि॒तम् । \newline
13. मा॒ व्य॒थि॒तं ॅव्य॑थि॒तम् मा॑ मा व्यथि॒तं ॅवि॒देर् वि॒देर् व्य॑थि॒तम् मा॑ मा व्यथि॒तं ॅवि॒देः । \newline
14. व्य॒थि॒तं ॅवि॒देर् वि॒देर् व्य॑थि॒तं ॅव्य॑थि॒तं ॅवि॒दे र॒ग्नि र॒ग्निर् वि॒देर् व्य॑थि॒तं ॅव्य॑थि॒तं ॅवि॒देर॒ग्निः । \newline
15. वि॒दे र॒ग्नि र॒ग्निर् वि॒देर् वि॒दे र॒ग्निर् नभो॒ नभो॒ ऽग्निर् वि॒देर् वि॒दे र॒ग्निर् नभः॑ । \newline
16. अ॒ग्निर् नभो॒ नभो॒ ऽग्नि र॒ग्निर् नभो॒ नाम॒ नाम॒ नभो॒ ऽग्नि र॒ग्निर् नभो॒ नाम॑ । \newline
17. नभो॒ नाम॒ नाम॒ नभो॒ नभो॒ नामाग्ने ऽग्ने॒ नाम॒ नभो॒ नभो॒ नामाग्ने᳚ । \newline
18. नामाग्ने ऽग्ने॒ नाम॒ नामाग्ने॑ अङ्गिरो अङ्गि॒रो ऽग्ने॒ नाम॒ नामाग्ने॑ अङ्गिरः । \newline
19. अग्ने॑ अङ्गिरो अङ्गि॒रो ऽग्ने ऽग्ने॑ अङ्गिरो॒ यो यो अ॑ङ्गि॒रो ऽग्ने ऽग्ने॑ अङ्गिरो॒ यः । \newline
20. अ॒ङ्गि॒रो॒ यो यो अ॑ङ्गिरो अङ्गिरो॒ यो᳚ ऽस्या म॒स्यां ॅयो अ॑ङ्गिरो अङ्गिरो॒ यो᳚ ऽस्याम् । \newline
21. यो᳚ ऽस्या म॒स्यां ॅयो यो᳚ ऽस्याम् पृ॑थि॒व्याम् पृ॑थि॒व्या म॒स्यां ॅयो यो᳚ ऽस्याम् पृ॑थि॒व्याम् । \newline
22. अ॒स्याम् पृ॑थि॒व्याम् पृ॑थि॒व्या म॒स्या म॒स्याम् पृ॑थि॒व्या मस्यसि॑ पृथि॒व्या म॒स्या म॒स्याम् पृ॑थि॒व्या मसि॑ । \newline
23. पृ॒थि॒व्या मस्यसि॑ पृथि॒व्याम् पृ॑थि॒व्या मस्यायु॒षा ऽऽयु॒षा ऽसि॑ पृथि॒व्याम् पृ॑थि॒व्या मस्यायु॑षा । \newline
24. अस्यायु॒षा ऽऽयु॒षा ऽस्यस्यायु॑षा॒ नाम्ना॒ नाम्ना ऽऽयु॒षा ऽस्यस्यायु॑षा॒ नाम्ना᳚ । \newline
25. आयु॑षा॒ नाम्ना॒ नाम्ना ऽऽयु॒षा ऽऽयु॑षा॒ नाम्ना ऽऽनाम्ना ऽऽयु॒षा ऽऽयु॑षा॒ नाम्ना । \newline
26. नाम्ना नाम्ना॒ नाम्नेही॒ह्या नाम्ना॒ नाम्नेहि॑ । \newline
27. एही॒ह्येहि॒ यद् यदि॒ह्येहि॒ यत् । \newline
28. इ॒हि॒ यद् यदि॑हीहि॒ यत् ते॑ ते॒ यदि॑हीहि॒ यत् ते᳚ । \newline
29. यत् ते॑ ते॒ यद् यत् ते ऽना॑धृष्ट॒ मना॑धृष्टम् ते॒ यद् यत् ते ऽना॑धृष्टम् । \newline
30. ते ऽना॑धृष्ट॒ मना॑धृष्टम् ते॒ ते ऽना॑धृष्ट॒न्नाम॒ नामाना॑धृष्टम् ते॒ ते ऽना॑धृष्ट॒न्नाम॑ । \newline
31. अना॑धृष्ट॒न्नाम॒ नामाना॑धृष्ट॒ मना॑धृष्ट॒न्नाम॑ य॒ज्ञियं॑ ॅय॒ज्ञिय॒न्नामाना॑धृष्ट॒ मना॑धृष्ट॒न्नाम॑ य॒ज्ञिय᳚म् । \newline
32. अना॑धृष्ट॒मित्यना᳚ - धृ॒ष्ट॒म् । \newline
33. नाम॑ य॒ज्ञियं॑ ॅय॒ज्ञिय॒न्नाम॒ नाम॑ य॒ज्ञिय॒म् तेन॒ तेन॑ य॒ज्ञिय॒न्नाम॒ नाम॑ य॒ज्ञिय॒म् तेन॑ । \newline
34. य॒ज्ञिय॒म् तेन॒ तेन॑ य॒ज्ञियं॑ ॅय॒ज्ञिय॒म् तेन॑ त्वा त्वा॒ तेन॑ य॒ज्ञियं॑ ॅय॒ज्ञिय॒म् तेन॑ त्वा । \newline
35. तेन॑ त्वा त्वा॒ तेन॒ तेन॒ त्वा ऽऽत्वा॒ तेन॒ तेन॒ त्वा । \newline
36. त्वा ऽऽत्वा॒ त्वा ऽऽद॑धे दध॒ आ त्वा॒ त्वा ऽऽद॑धे । \newline
37. आ द॑धे दध॒ आ द॒धे ऽग्ने ऽग्ने॑ दध॒ आ द॒धे ऽग्ने᳚ । \newline
38. द॒धे ऽग्ने ऽग्ने॑ दधे द॒धे ऽग्ने॑ अङ्गिरो अङ्गि॒रो ऽग्ने॑ दधे द॒धे ऽग्ने॑ अङ्गिरः । \newline
39. अग्ने॑ अङ्गिरो अङ्गि॒रो ऽग्ने ऽग्ने॑ अङ्गिरो॒ यो यो अ॑ङ्गि॒रो ऽग्ने ऽग्ने॑ अङ्गिरो॒ यः । \newline
40. अ॒ङ्गि॒रो॒ यो यो अ॑ङ्गिरो अङ्गिरो॒ यो द्वि॒तीय॑स्याम् द्वि॒तीय॑स्यां॒ ॅयो अ॑ङ्गिरो अङ्गिरो॒ यो द्वि॒तीय॑स्याम् । \newline
41. यो द्वि॒तीय॑स्याम् द्वि॒तीय॑स्यां॒ ॅयो यो द्वि॒तीय॑स्याम् तृ॒तीय॑स्याम् तृ॒तीय॑स्याम् द्वि॒तीय॑स्यां॒ ॅयो यो द्वि॒तीय॑स्याम् तृ॒तीय॑स्याम् । \newline
42. द्वि॒तीय॑स्याम् तृ॒तीय॑स्याम् तृ॒तीय॑स्याम् द्वी॒तीय॑स्याम् द्वि॒तीय॑स्याम् तृ॒तीय॑स्याम् पृथि॒व्याम् पृ॑थि॒व्याम् तृ॒तीय॑स्याम् द्वि॒तीय॑स्याम् द्वि॒तीय॑स्याम् तृ॒तीय॑स्याम् पृथि॒व्याम् । \newline
43. तृ॒तीय॑स्याम् पृथि॒व्याम् पृ॑थि॒व्याम् तृ॒तीय॑स्याम् तृ॒तीय॑स्याम् पृथि॒व्या मस्यसि॑ पृथि॒व्याम् तृ॒तीय॑स्याम् तृ॒तीय॑स्याम् पृथि॒व्या मसि॑ । \newline
44. पृ॒थि॒व्या मस्यसि॑ पृथि॒व्याम् पृ॑थि॒व्या मस्यायु॒षा ऽऽयु॒षा ऽसि॑ पृथि॒व्याम् पृ॑थि॒व्या मस्यायु॑षा । \newline
45. अस्यायु॒षा ऽऽयु॒षा ऽस्यस्यायु॑षा॒ नाम्ना॒ नाम्ना ऽऽयु॒षा ऽस्यस्यायु॑षा॒ नाम्ना᳚ । \newline
46. आयु॑षा॒ नाम्ना॒ नाम्ना ऽऽयु॒षा ऽऽयु॑षा॒ नाम्ना नाम्ना ऽऽयु॒षा ऽऽयु॑षा॒ नाम्ना । \newline
47. नाम्ना नाम्ना॒ नाम्नेही॒ह्या नाम्ना॒ नाम्नेहि॑ । \newline
48. एही॒ह्येहि॒ यद् यदि॒ह्येहि॒ यत् । \newline
49. इ॒हि॒ यद् यदि॑हीहि॒ यत् ते॑ ते॒ यदि॑हीहि॒ यत् ते᳚ । \newline
50. यत् ते॑ ते॒ यद् यत् ते ऽना॑धृष्ट॒ मना॑धृष्टम् ते॒ यद् यत् ते ऽना॑धृष्टम् । \newline
51. ते ऽना॑धृष्ट॒ मना॑धृष्टम् ते॒ ते ऽना॑धृष्ट॒न्नाम॒ नामाना॑धृष्टम् ते॒ ते ऽना॑धृष्ट॒न्नाम॑ । \newline
52. अना॑धृष्ट॒न्नाम॒ नामाना॑धृष्ट॒ मना॑धृष्ट॒न्नाम॑ य॒ज्ञियं॑ ॅय॒ज्ञिय॒न्नामाना॑धृष्ट॒ मना॑धृष्ट॒न्नाम॑ य॒ज्ञिय᳚म् । \newline
53. अना॑धृष्ट॒मित्यना᳚ - धृ॒ष्ट॒म् । \newline
54. नाम॑ य॒ज्ञियं॑ ॅय॒ज्ञिय॒न्नाम॒ नाम॑ य॒ज्ञिय॒म् तेन॒ तेन॑ य॒ज्ञिय॒न्नाम॒ नाम॑ य॒ज्ञिय॒म् तेन॑ । \newline
\pagebreak
\markright{ TS 1.2.12.2  \hfill https://www.vedavms.in \hfill}

\section{ TS 1.2.12.2 }

\textbf{TS 1.2.12.2 } \newline
\textbf{Samhita Paata} \newline

य॒ज्ञियं॒ तेन॒ त्वाऽऽ*द॑धे सिꣳ॒॒हीर॑सि महि॒षीर॑स्यु॒रु प्र॑थस्वो॒रु ते॑ य॒ज्ञ्प॑तिः प्रथतां ध्रु॒वाऽसि॑ दे॒वेभ्यः॑ शुन्धस्व दे॒वेभ्यः॑ शुंभस्वेन्द्रघो॒षस्त्वा॒ वसु॑भिः पु॒रस्ता᳚त् पातु॒ मनो॑जवास्त्वा पि॒तृभि॑र् दक्षिण॒तः पा॑तु॒ प्रचे॑तास्त्वा रु॒द्रैः प॒श्चात् पा॑तु वि॒श्वक॑र्मा त्वाऽऽदि॒त्यैरु॑त्तर॒तः पा॑तु॒ सिꣳ॒॒हीर॑सि सपत्नसा॒ही स्वाहा॑ सिꣳ॒॒हीर॑सि सुप्रजा॒वनिः॒ स्वाहा॑ सिꣳ॒॒ही-[ ] \newline

\textbf{Pada Paata} \newline

य॒ज्ञिय᳚म् । तेन॑ । त्वा॒ । एति॑ । द॒धे॒ । सिꣳ॒॒हीः । अ॒सि॒ । म॒हि॒षीः । अ॒सि॒ । उ॒रु । प्र॒थ॒स्व॒ । उ॒रु । ते॒ । य॒ज्ञ्प॑ति॒रिति॑ य॒ज्ञ् - प॒तिः॒ । प्र॒थ॒ता॒म् । ध्रु॒वा । अ॒सि॒ । दे॒वेभ्यः॑ । शु॒न्ध॒स्व॒ । दे॒वेभ्यः॑ । शुं॒भ॒स्व॒ । इ॒न्द्र॒घो॒ष इती᳚न्द्र - घो॒षः । त्वा॒ । वसु॑भि॒रिति॒ वसु॑ - भिः॒ । पु॒रस्ता᳚त् । पा॒तु॒ । मनो॑जवा॒ इति॒ मनः॑ - ज॒वाः॒ । त्वा॒ । पि॒तृभि॒रिति॑ पि॒तृ - भिः॒ । द॒क्षि॒ण॒तः । पा॒तु॒ । प्रचे॑ता॒ इति॒ प्र - चे॒ताः॒ । त्वा॒ । रु॒द्रैः । प॒श्चात् । पा॒तु॒ । वि॒श्वक॒र्मेति॑ वि॒श्व - क॒र्मा॒ । त्वा॒ । आ॒दि॒त्यैः । उ॒त्त॒र॒त इत्यु॑त् - त॒र॒तः । पा॒तु॒ । सिꣳ॒॒हीः । अ॒सि॒ । स॒प॒त्न॒सा॒हीति॑ सपत्न - सा॒ही । स्वाहा᳚ । सिꣳ॒॒हीः । अ॒सि॒ । सु॒प्र॒जा॒वनि॒रिति॑ सुप्रजा - वनिः॑ । स्वाहा᳚ । सिꣳ॒॒हीः ।  \newline


\textbf{Krama Paata} \newline

य॒ज्ञिय॒म् तेन॑ । तेन॑ त्वा । त्वा । आ द॑धे । द॒धे॒ सिꣳ॒॒हीः । सिꣳ॒॒हीर॑सि । अ॒सि॒ म॒हि॒षीः । म॒हि॒षीर॑सि । अ॒स्यु॒रु । उ॒रु प्र॑थस्व । प्र॒थ॒स्वो॒रु । उ॒रु ते᳚ । ते॒ य॒ज्ञ्प॑तिः । य॒ज्ञ्प॑तिः प्रथताम् । य॒ज्ञ्प॑ति॒रिति॑ य॒ज्ञ् - प॒तिः॒ । प्र॒थ॒ता॒म् ध्रु॒वा । ध्रु॒वाऽसि॑ । अ॒सि॒ दे॒वेभ्यः॑ । दे॒वेभ्यः॑ शुन्धस्व । शु॒न्ध॒स्व॒ दे॒वेभ्यः॑ । दे॒वेभ्यः॑ शुम्भस्व । शु॒म्भ॒स्वे॒न्द्र॒घो॒षः । इ॒न्द्र॒घो॒षस्त्वा᳚ । इ॒न्द्र॒घो॒ष इती᳚न्द्र - घो॒षः । त्वा॒ वसु॑भिः । वसु॑भिः पु॒रस्ता᳚त् । वसु॑भि॒रिति॒ वसु॑ - भिः॒ । पु॒रस्ता᳚त् पातु । पा॒तु॒ मनो॑जवाः । मनो॑जवास्त्वा । मनो॑जवा॒ इति॒ मनः॑ - ज॒वाः॒ । त्वा॒ पि॒तृभिः॑ । पि॒तृभि॑र् दक्षिण॒तः । पि॒तृभि॒रिति॑ पि॒तृ - भिः॒ । द॒क्षि॒ण॒तः पा॑तु । पा॒तु॒ प्रचे॑ताः । प्रचे॑तास्त्वा । प्रचे॑ता॒ इति॒ प्र - चे॒ताः॒ । त्वा॒ रु॒द्रैः । रु॒द्रैः प॒श्चात् । प॒श्चात् पा॑तु । पा॒तु॒ वि॒श्वक॑र्मा । वि॒श्वक॑र्मा त्वा । वि॒श्वक॒र्मेति॑ वि॒श्व - क॒र्मा॒ । त्वा॒ऽऽदि॒त्यैः । आ॒दि॒त्यैरु॑त्तर॒तः । उ॒त्त॒र॒तः पा॑तु । उ॒त्त॒र॒त इत्यु॑त् - त॒र॒तः । पा॒तु॒ सिꣳ॒॒हीः । सिꣳ॒॒हीर॑सि । अ॒सि॒ स॒प॒त्न॒सा॒ही । स॒प॒त्न॒सा॒ही स्वाहा᳚ । स॒प॒त्न॒सा॒हीति॑ सपत्न - सा॒ही । स्वाहा॑ सिꣳ॒॒हीः । सिꣳ॒॒हीर॑सि । अ॒सि॒ सु॒प्र॒जा॒वनिः॑ । सु॒प्र॒जा॒वनिः॒ स्वाहा᳚ । सु॒प्र॒जा॒वनि॒रिति॑ सुप्रजा - वनिः॑ । स्वाहा॑ सिꣳ॒॒हीः ( ) । सिꣳ॒॒हीर॑सि \newline

\textbf{Jatai Paata} \newline

1. य॒ज्ञिय॒म् तेन॒ तेन॑ य॒ज्ञियं॑ ॅय॒ज्ञिय॒म् तेन॑ । \newline
2. तेन॑ त्वा त्वा॒ तेन॒ तेन॑ त्वा । \newline
3. त्वा ऽऽत्वा॒ त्वा । \newline
4. आ द॑धे दध॒ आ द॑धे । \newline
5. द॒धे॒ सि॒(ग्म्॒)हीः सि॒(ग्म्॒)हीर् द॑धे दधे सि॒(ग्म्॒)हीः । \newline
6. सि॒(ग्म्॒)ही र॑स्यसि सि॒(ग्म्॒)हीः सि॒(ग्म्॒)ही र॑सि । \newline
7. अ॒सि॒ म॒हि॒षीर् म॑हि॒षी र॑स्यसि महि॒षीः । \newline
8. म॒हि॒षी र॑स्यसि महि॒षीर् म॑हि॒षी र॑सि । \newline
9. अ॒स्यु॒रू᳚(1॒)र्व॑ स्य स्यु॒रु । \newline
10. उ॒रु प्र॑थस्व प्रथस्वो॒रू॑रु प्र॑थस्व । \newline
11. प्र॒थ॒स्वो॒रू॑रु प्र॑थस्व प्रथस्वो॒रु । \newline
12. उ॒रु ते॑ त उ॒रू॑रु ते᳚ । \newline
13. ते॒ य॒ज्ञ्प॑तिर् य॒ज्ञ्प॑ति स्ते ते य॒ज्ञ्प॑तिः । \newline
14. य॒ज्ञ्प॑तिः प्रथताम् प्रथतां ॅय॒ज्ञ्प॑तिर् य॒ज्ञ्प॑तिः प्रथताम् । \newline
15. य॒ज्ञ्प॑ति॒रिति॑ य॒ज्ञ् - प॒तिः॒ । \newline
16. प्र॒थ॒ता॒म् ध्रु॒वा ध्रु॒वा प्र॑थताम् प्रथताम् ध्रु॒वा । \newline
17. ध्रु॒वा ऽस्य॑सि ध्रु॒वा ध्रु॒वा ऽसि॑ । \newline
18. अ॒सि॒ दे॒वेभ्यो॑ दे॒वेभ्यो᳚ ऽस्यसि दे॒वेभ्यः॑ । \newline
19. दे॒वेभ्यः॑ शुन्धस्व शुन्धस्व दे॒वेभ्यो॑ दे॒वेभ्यः॑ शुन्धस्व । \newline
20. शु॒न्ध॒स्व॒ दे॒वेभ्यो॑ दे॒वेभ्यः॑ शुन्धस्व शुन्धस्व दे॒वेभ्यः॑ । \newline
21. दे॒वेभ्यः॑ शुंभस्व शुंभस्व दे॒वेभ्यो॑ दे॒वेभ्यः॑ शुंभस्व । \newline
22. शुं॒भ॒स्वे॒ न्द्र॒घो॒ष इ॑न्द्रघो॒षः शुं॑भस्व शुंभस्वे न्द्रघो॒षः । \newline
23. इ॒न्द्र॒घो॒ष स्त्वा᳚ त्वेन्द्रघो॒ष इ॑न्द्रघो॒ष स्त्वा᳚ । \newline
24. इ॒न्द्र॒घो॒ष इती᳚न्द्र - घो॒षः । \newline
25. त्वा॒ वसु॑भि॒र् वसु॑भि स्त्वा त्वा॒ वसु॑भिः । \newline
26. वसु॑भिः पु॒रस्ता᳚त् पु॒रस्ता॒द् वसु॑भि॒र् वसु॑भिः पु॒रस्ता᳚त् । \newline
27. वसु॑भि॒रिति॒ वसु॑ - भिः॒ । \newline
28. पु॒रस्ता᳚त् पातु पातु पु॒रस्ता᳚त् पु॒रस्ता᳚त् पातु । \newline
29. पा॒तु॒ मनो॑जवा॒ मनो॑जवाः पातु पातु॒ मनो॑जवाः । \newline
30. मनो॑जवा स्त्वा त्वा॒ मनो॑जवा॒ मनो॑जवा स्त्वा । \newline
31. मनो॑जवा॒ इति॒ मनः॑ - ज॒वाः॒ । \newline
32. त्वा॒ पि॒तृभिः॑ पि॒तृभि॑ स्त्वा त्वा पि॒तृभिः॑ । \newline
33. पि॒तृभि॑र् दक्षिण॒तो द॑क्षिण॒तः पि॒तृभिः॑ पि॒तृभि॑र् दक्षिण॒तः । \newline
34. पि॒तृभि॒रिति॑ पि॒तृ - भिः॒ । \newline
35. द॒क्षि॒ण॒तः पा॑तु पातु दक्षिण॒तो द॑क्षिण॒तः पा॑तु । \newline
36. पा॒तु॒ प्रचे॑ताः॒ प्रचे॑ताः पातु पातु॒ प्रचे॑ताः । \newline
37. प्रचे॑ता स्त्वा त्वा॒ प्रचे॑ताः॒ प्रचे॑ता स्त्वा । \newline
38. प्रचे॑ता॒ इति॒ प्र - चे॒ताः॒ । \newline
39. त्वा॒ रु॒द्रै रु॒द्रै स्त्वा᳚ त्वा रु॒द्रैः । \newline
40. रु॒द्रैः प॒श्चात् प॒श्चाद् रु॒द्रै रु॒द्रैः प॒श्चात् । \newline
41. प॒श्चात् पा॑तु पातु प॒श्चात् प॒श्चात् पा॑तु । \newline
42. पा॒तु॒ वि॒श्वक॑र्मा वि॒श्वक॑र्मा पातु पातु वि॒श्वक॑र्मा । \newline
43. वि॒श्वक॑र्मा त्वा त्वा वि॒श्वक॑र्मा वि॒श्वक॑र्मा त्वा । \newline
44. वि॒श्वक॒र्मेति॑ वि॒श्व - क॒र्मा॒ । \newline
45. त्वा॒ ऽऽदि॒त्यै रा॑दि॒त्यै स्त्वा᳚ त्वा ऽऽदि॒त्यैः । \newline
46. आ॒दि॒त्यै रु॑त्तर॒त उ॑त्तर॒त आ॑दि॒त्यै रा॑दि॒त्यै रु॑त्तर॒तः । \newline
47. उ॒त्त॒र॒तः पा॑तु पातूत्तर॒त उ॑त्तर॒तः पा॑तु । \newline
48. उ॒त्त॒र॒त इत्यु॑त् - त॒र॒तः । \newline
49. पा॒तु॒ सि॒(ग्म्॒)हीः सि॒(ग्म्॒)हीः पा॑तु पातु सि॒(ग्म्॒)हीः । \newline
50. सि॒(ग्म्॒)ही र॑स्यसि सि॒(ग्म्॒)हीः सि॒(ग्म्॒)ही र॑सि । \newline
51. अ॒सि॒ स॒प॒त्न॒सा॒ही स॑पत्नसा॒ ह्य॑स्यसि सपत्नसा॒ही । \newline
52. स॒प॒त्न॒सा॒ही स्वाहा॒ स्वाहा॑ सपत्नसा॒ही स॑पत्नसा॒ही स्वाहा᳚ । \newline
53. स॒प॒त्न॒सा॒हीति॑ सपत्न - सा॒ही । \newline
54. स्वाहा॑ सि॒(ग्म्॒)हीः सि॒(ग्म्॒)हीः स्वाहा॒ स्वाहा॑ सि॒(ग्म्॒)हीः । \newline
55. सि॒(ग्म्॒)ही र॑स्यसि सि॒(ग्म्॒)हीः सि॒(ग्म्॒)ही र॑सि । \newline
56. अ॒सि॒ सु॒प्र॒जा॒वनिः॑ सुप्रजा॒वनि॑ रस्यसि सुप्रजा॒वनिः॑ । \newline
57. सु॒प्र॒जा॒वनिः॒ स्वाहा॒ स्वाहा॑ सुप्रजा॒वनिः॑ सुप्रजा॒वनिः॒ स्वाहा᳚ । \newline
58. सु॒प्र॒जा॒वनि॒रिति॑ सुप्रजा - वनिः॑ । \newline
59. स्वाहा॑ सि॒(ग्म्॒)हीः सि॒(ग्म्॒)हीः स्वाहा॒ स्वाहा॑ सि॒(ग्म्॒)हीः । \newline
60. सि॒(ग्म्॒)ही र॑स्यसि सि॒(ग्म्॒)हीः सि॒(ग्म्॒)ही र॑सि । \newline

\textbf{Ghana Paata } \newline

1. य॒ज्ञिय॒म् तेन॒ तेन॑ य॒ज्ञियं॑ ॅय॒ज्ञिय॒म् तेन॑ त्वा त्वा॒ तेन॑ य॒ज्ञियं॑ ॅय॒ज्ञिय॒म् तेन॑ त्वा । \newline
2. तेन॑ त्वा त्वा॒ तेन॒ तेन॒ त्वा ऽऽत्वा॒ तेन॒ तेन॒ त्वा । \newline
3. त्वा ऽऽत्वा॒ त्वा ऽऽद॑धे दध॒ आ त्वा॒ त्वा ऽऽद॑धे । \newline
4. आ द॑धे दध॒ आ द॑धे सि॒(ग्म्॒)हीः सि॒(ग्म्॒)हीर् द॑ध॒ आ द॑धे सि॒(ग्म्॒)हीः । \newline
5. द॒धे॒ सि॒(ग्म्॒)हीः सि॒(ग्म्॒)हीर् द॑धे दधे सि॒(ग्म्॒)हीर॑स्यसि सि॒(ग्म्॒)हीर् द॑धे दधे सि॒(ग्म्॒)हीर॑सि । \newline
6. सि॒(ग्म्॒)हीर॑स्यसि सि॒(ग्म्॒)हीः सि॒(ग्म्॒)हीर॑सि महि॒षीर् म॑हि॒षीर॑सि सि॒(ग्म्॒)हीः सि॒(ग्म्॒)हीर॑सि महि॒षीः । \newline
7. अ॒सि॒ म॒हि॒षीर् म॑हि॒षीर॑स्यसि महि॒षीर॑स्यसि महि॒षीर॑स्यसि महि॒षीर॑सि । \newline
8. म॒हि॒षीर॑स्यसि महि॒षीर् म॑हि॒षीर॑स्यु॒रू᳚(1॒)र्व॑सि महि॒षीर् म॑हि॒षीर॑स्यु॒रु । \newline
9. अ॒स्यु॒रू᳚(1॒)र्व॑स्यस्यु॒रु प्र॑थस्व प्रथस्वो॒र्व॑स्यस्यु॒रु प्र॑थस्व । \newline
10. उ॒रु प्र॑थस्व प्रथस्वो॒रू॑रु प्र॑थस्वो॒रू॑रु प्र॑थस्वो॒रू॑रु प्र॑थस्वो॒रु । \newline
11. प्र॒थ॒स्वो॒रू॑रु प्र॑थस्व प्रथस्वो॒रु ते॑ त उ॒रु प्र॑थस्व प्रथस्वो॒रु ते᳚ । \newline
12. उ॒रु ते॑ त उ॒रू॑रु ते॑ य॒ज्ञ्प॑तिर् य॒ज्ञ्प॑तिस्त उ॒रू॑रु ते॑ य॒ज्ञ्प॑तिः । \newline
13. ते॒ य॒ज्ञ्प॑तिर् य॒ज्ञ्प॑तिस्ते ते य॒ज्ञ्प॑तिः प्रथताम् प्रथतां ॅय॒ज्ञ्प॑तिस्ते ते य॒ज्ञ्प॑तिः प्रथताम् । \newline
14. य॒ज्ञ्प॑तिः प्रथताम् प्रथतां ॅय॒ज्ञ्प॑तिर् य॒ज्ञ्प॑तिः प्रथताम् ध्रु॒वा ध्रु॒वा प्र॑थतां ॅय॒ज्ञ्प॑तिर् य॒ज्ञ्प॑तिः प्रथताम् ध्रु॒वा । \newline
15. य॒ज्ञ्प॑ति॒रिति॑ य॒ज्ञ् - प॒तिः॒ । \newline
16. प्र॒थ॒ता॒म् ध्रु॒वा ध्रु॒वा प्र॑थताम् प्रथताम् ध्रु॒वा ऽस्य॑सि ध्रु॒वा प्र॑थताम् प्रथताम् ध्रु॒वा ऽसि॑ । \newline
17. ध्रु॒वा ऽस्य॑सि ध्रु॒वा ध्रु॒वा ऽसि॑ दे॒वेभ्यो॑ दे॒वेभ्यो॑ ऽसि ध्रु॒वा ध्रु॒वा ऽसि॑ दे॒वेभ्यः॑ । \newline
18. अ॒सि॒ दे॒वेभ्यो॑ दे॒वेभ्यो᳚ ऽस्यसि दे॒वेभ्यः॑ शुन्धस्व शुन्धस्व दे॒वेभ्यो᳚ ऽस्यसि दे॒वेभ्यः॑ शुन्धस्व । \newline
19. दे॒वेभ्यः॑ शुन्धस्व शुन्धस्व दे॒वेभ्यो॑ दे॒वेभ्यः॑ शुन्धस्व दे॒वेभ्यो॑ दे॒वेभ्यः॑ शुन्धस्व दे॒वेभ्यो॑ दे॒वेभ्यः॑ शुन्धस्व दे॒वेभ्यः॑ । \newline
20. शु॒न्ध॒स्व॒ दे॒वेभ्यो॑ दे॒वेभ्यः॑ शुन्धस्व शुन्धस्व दे॒वेभ्यः॑ शुंभस्व शुंभस्व दे॒वेभ्यः॑ शुन्धस्व शुन्धस्व दे॒वेभ्यः॑ शुंभस्व । \newline
21. दे॒वेभ्यः॑ शुंभस्व शुंभस्व दे॒वेभ्यो॑ दे॒वेभ्यः॑ शुंभस्वे न्द्रघो॒ष इ॑न्द्रघो॒षः शुं॑भस्व दे॒वेभ्यो॑ दे॒वेभ्यः॑ शुंभस्वे न्द्रघो॒षः । \newline
22. शुं॒भ॒स्वे॒ न्द्र॒घो॒ष इ॑न्द्रघो॒षः शुं॑भस्व शुंभस्वे न्द्रघो॒षस्त्वा᳚ त्वेन्द्रघो॒षः शुं॑भस्व शुंभस्वे न्द्रघो॒षस्त्वा᳚ । \newline
23. इ॒न्द्र॒घो॒षस्त्वा᳚ त्वेन्द्रघो॒ष इ॑न्द्रघो॒षस्त्वा॒ वसु॑भि॒र् वसु॑भिस्त्वेन्द्रघो॒ष इ॑न्द्रघो॒षस्त्वा॒ वसु॑भिः । \newline
24. इ॒न्द्र॒घो॒ष इती᳚न्द्र - घो॒षः । \newline
25. त्वा॒ वसु॑भि॒र् वसु॑भिस्त्वा त्वा॒ वसु॑भिः पु॒रस्ता᳚त् पु॒रस्ता॒द् वसु॑भिस्त्वा त्वा॒ वसु॑भिः पु॒रस्ता᳚त् । \newline
26. वसु॑भिः पु॒रस्ता᳚त् पु॒रस्ता॒द् वसु॑भि॒र् वसु॑भिः पु॒रस्ता᳚त् पातु पातु पु॒रस्ता॒द् वसु॑भि॒र् वसु॑भिः पु॒रस्ता᳚त् पातु । \newline
27. वसु॑भि॒रिति॒ वसु॑ - भिः॒ । \newline
28. पु॒रस्ता᳚त् पातु पातु पु॒रस्ता᳚त् पु॒रस्ता᳚त् पातु॒ मनो॑जवा॒ मनो॑जवाः पातु पु॒रस्ता᳚त् पु॒रस्ता᳚त् पातु॒ मनो॑जवाः । \newline
29. पा॒तु॒ मनो॑जवा॒ मनो॑जवाः पातु पातु॒ मनो॑जवास्त्वा त्वा॒ मनो॑जवाः पातु पातु॒ मनो॑जवास्त्वा । \newline
30. मनो॑जवास्त्वा त्वा॒ मनो॑जवा॒ मनो॑जवास्त्वा पि॒तृभिः॑ पि॒तृभि॑स्त्वा॒ मनो॑जवा॒ मनो॑जवास्त्वा पि॒तृभिः॑ । \newline
31. मनो॑जवा॒ इति॒ मनः॑ - ज॒वाः॒ । \newline
32. त्वा॒ पि॒तृभिः॑ पि॒तृभि॑स्त्वा त्वा पि॒तृभि॑र् दक्षिण॒तो द॑क्षिण॒तः पि॒तृभि॑स्त्वा त्वा पि॒तृभि॑र् दक्षिण॒तः । \newline
33. पि॒तृभि॑र् दक्षिण॒तो द॑क्षिण॒तः पि॒तृभिः॑ पि॒तृभि॑र् दक्षिण॒तः पा॑तु पातु दक्षिण॒तः पि॒तृभिः॑ पि॒तृभि॑र् दक्षिण॒तः पा॑तु । \newline
34. पि॒तृभि॒रिति॑ पि॒तृ - भिः॒ । \newline
35. द॒क्षि॒ण॒तः पा॑तु पातु दक्षिण॒तो द॑क्षिण॒तः पा॑तु॒ प्रचे॑ताः॒ प्रचे॑ताः पातु दक्षिण॒तो द॑क्षिण॒तः पा॑तु॒ प्रचे॑ताः । \newline
36. पा॒तु॒ प्रचे॑ताः॒ प्रचे॑ताः पातु पातु॒ प्रचे॑ता स्त्वा त्वा॒ प्रचे॑ताः पातु पातु॒ प्रचे॑ता स्त्वा । \newline
37. प्रचे॑ता स्त्वा त्वा॒ प्रचे॑ताः॒ प्रचे॑तास्त्वा रु॒द्रै रु॒द्रै स्त्वा॒ प्रचे॑ताः॒ प्रचे॑ता स्त्वा रु॒द्रैः । \newline
38. प्रचे॑ता॒ इति॒ प्र - चे॒ताः॒ । \newline
39. त्वा॒ रु॒द्रै रु॒द्रैस्त्वा᳚ त्वा रु॒द्रैः प॒श्चात् प॒श्चाद् रु॒द्रैस्त्वा᳚ त्वा रु॒द्रैः प॒श्चात् । \newline
40. रु॒द्रैः प॒श्चात् प॒श्चाद् रु॒द्रै रु॒द्रैः प॒श्चात् पा॑तु पातु प॒श्चाद् रु॒द्रै रु॒द्रैः प॒श्चात् पा॑तु । \newline
41. प॒श्चात् पा॑तु पातु प॒श्चात् प॒श्चात् पा॑तु वि॒श्वक॑र्मा वि॒श्वक॑र्मा पातु प॒श्चात् प॒श्चात् पा॑तु वि॒श्वक॑र्मा । \newline
42. पा॒तु॒ वि॒श्वक॑र्मा वि॒श्वक॑र्मा पातु पातु वि॒श्वक॑र्मा त्वा त्वा वि॒श्वक॑र्मा पातु पातु वि॒श्वक॑र्मा त्वा । \newline
43. वि॒श्वक॑र्मा त्वा त्वा वि॒श्वक॑र्मा वि॒श्वक॑र्मा त्वा ऽऽदि॒त्यैरा॑दि॒त्यैस्त्वा॑ वि॒श्वक॑र्मा वि॒श्वक॑र्मा त्वा ऽऽदि॒त्यैः । \newline
44. वि॒श्वक॒र्मेति॑ वि॒श्व - क॒र्मा॒ । \newline
45. त्वा॒ ऽऽदि॒त्यैरा॑दि॒त्यैस्त्वा᳚ त्वा ऽऽदि॒त्यैरु॑त्तर॒त उ॑त्तर॒त आ॑दि॒त्यैस्त्वा᳚ त्वा ऽऽदि॒त्यैरु॑त्तर॒तः । \newline
46. आ॒दि॒त्यै रु॑त्तर॒त उ॑त्तर॒त आ॑दि॒त्यै रा॑दि॒त्यै रु॑त्तर॒तः पा॑तु पातूत्तर॒त आ॑दि॒त्यै रा॑दि॒त्यै रु॑त्तर॒तः पा॑तु । \newline
47. उ॒त्त॒र॒तः पा॑तु पातूत्तर॒त उ॑त्तर॒तः पा॑तु सि॒(ग्म्॒)हीः सि॒(ग्म्॒)हीः पा॒तूत्तर॒त उ॑त्तर॒तः पा॑तु सि॒(ग्म्॒)हीः । \newline
48. उ॒त्त॒र॒त इत्यु॑त् - त॒र॒तः । \newline
49. पा॒तु॒ सि॒(ग्म्॒)हीः सि॒(ग्म्॒)हीः पा॑तु पातु सि॒(ग्म्॒)हीर॑स्यसि सि॒(ग्म्॒)हीः पा॑तु पातु सि॒(ग्म्॒)हीर॑सि । \newline
50. सि॒(ग्म्॒)हीर॑स्यसि सि॒(ग्म्॒)हीः सि॒(ग्म्॒)हीर॑सि सपत्नसा॒ही स॑पत्नसा॒ह्य॑सि सि॒(ग्म्॒)हीः सि॒(ग्म्॒)हीर॑सि सपत्नसा॒ही । \newline
51. अ॒सि॒ स॒प॒त्न॒सा॒ही स॑पत्नसा॒ह्य॑स्यसि सपत्नसा॒ही स्वाहा॒ स्वाहा॑ सपत्नसा॒ह्य॑स्यसि सपत्नसा॒ही स्वाहा᳚ । \newline
52. स॒प॒त्न॒सा॒ही स्वाहा॒ स्वाहा॑ सपत्नसा॒ही स॑पत्नसा॒ही स्वाहा॑ सि॒(ग्म्॒)हीः सि॒(ग्म्॒)हीः स्वाहा॑ सपत्नसा॒ही स॑पत्नसा॒ही स्वाहा॑ सि॒(ग्म्॒)हीः । \newline
53. स॒प॒त्न॒सा॒हीति॑ सपत्न - सा॒ही । \newline
54. स्वाहा॑ सि॒(ग्म्॒)हीः सि॒(ग्म्॒)हीः स्वाहा॒ स्वाहा॑ सि॒(ग्म्॒)हीर॑स्यसि सि॒(ग्म्॒)हीः स्वाहा॒ स्वाहा॑ सि॒(ग्म्॒)हीर॑सि । \newline
55. सि॒(ग्म्॒)हीर॑स्यसि सि॒(ग्म्॒)हीः सि॒(ग्म्॒)हीर॑सि सुप्रजा॒वनिः॑ सुप्रजा॒वनि॑रसि सि॒(ग्म्॒)हीः सि॒(ग्म्॒)हीर॑सि सुप्रजा॒वनिः॑ । \newline
56. अ॒सि॒ सु॒प्र॒जा॒वनिः॑ सुप्रजा॒वनि॑रस्यसि सुप्रजा॒वनिः॒ स्वाहा॒ स्वाहा॑ सुप्रजा॒वनि॑रस्यसि सुप्रजा॒वनिः॒ स्वाहा᳚ । \newline
57. सु॒प्र॒जा॒वनिः॒ स्वाहा॒ स्वाहा॑ सुप्रजा॒वनिः॑ सुप्रजा॒वनिः॒ स्वाहा॑ सि॒(ग्म्॒)हीः सि॒(ग्म्॒)हीः स्वाहा॑ सुप्रजा॒वनिः॑ सुप्रजा॒वनिः॒ स्वाहा॑ सि॒(ग्म्॒)हीः । \newline
58. सु॒प्र॒जा॒वनि॒रिति॑ सुप्रजा - वनिः॑ । \newline
59. स्वाहा॑ सि॒(ग्म्॒)हीः सि॒(ग्म्॒)हीः स्वाहा॒ स्वाहा॑ सि॒(ग्म्॒)हीर॑स्यसि सि॒(ग्म्॒)हीः स्वाहा॒ स्वाहा॑ सि॒(ग्म्॒)हीर॑सि । \newline
60. सि॒(ग्म्॒)हीर॑स्यसि सि॒(ग्म्॒)हीः सि॒(ग्म्॒)हीर॑सि रायस्पोष॒वनी॑ रायस्पोष॒वनि॑रसि सि॒(ग्म्॒)हीः सि॒(ग्म्॒)हीर॑सि रायस्पोष॒वनिः॑ । \newline
\pagebreak
\markright{ TS 1.2.12.3  \hfill https://www.vedavms.in \hfill}

\section{ TS 1.2.12.3 }

\textbf{TS 1.2.12.3 } \newline
\textbf{Samhita Paata} \newline

र॑सि रायस्पोष॒वनिः॒ स्वाहा॑ सिꣳ॒॒हीर॑स्यादित्य॒वनिः॒ स्वाहा॑ सिꣳ॒॒हीर॒स्या व॑ह दे॒वान्दे॑वय॒ते यज॑मानाय॒ स्वाहा॑ भू॒तेभ्य॑स्त्वा वि॒श्वायु॑रसि पृथि॒वीं दृꣳ॑ह ध्रुव॒क्षिद॑स्य॒न्तरि॑क्षं दृꣳहाच्युत॒क्षिद॑सि॒ दिवं॑ दृꣳहा॒ग्नेर् भस्मा᳚स्य॒ग्नेः पुरी॑षमसि ॥ \newline

\textbf{Pada Paata} \newline

अ॒सि॒ । रा॒य॒स्पो॒ष॒वनि॒रिति॑ रायस्पोष - वनिः॑ । स्वाहा᳚ । सिꣳ॒॒हीः । अ॒सि॒ । आ॒दि॒त्य॒वनि॒रित्या॑दित्य - वनिः॑ । स्वाहा᳚ । सिꣳ॒॒हीः । अ॒सि॒ । एति॑ । व॒ह॒ । दे॒वान् । दे॒व॒य॒त इति॑ देव - य॒ते । यज॑मानाय । स्वाहा᳚ । भू॒तेभ्यः॑ । त्वा॒ । वि॒श्वायु॒रिति॑ वि॒श्व - आ॒युः॒ । अ॒सि॒ । पृ॒थि॒वीम् । दृꣳ॒॒ह॒ । ध्रु॒व॒क्षिदिति॑ ध्रुव - क्षित् । अ॒सि॒ । अ॒न्तरि॑क्षम् । दृꣳ॒॒ह॒ । अ॒च्यु॒त॒क्षिदित्य॑च्युत -क्षित् । अ॒सि॒ । दिव᳚म् । दृꣳ॒॒ह॒ । अ॒ग्नेः । भस्म॑ । अ॒सि॒ । अ॒ग्नेः । पुरी॑षम् । अ॒सि॒ ॥  \newline


\textbf{Krama Paata} \newline

अ॒सि॒ रा॒य॒स्पो॒ष॒वनिः॑ । रा॒य॒स्पो॒ष॒वनिः॒ स्वाहा᳚ । रा॒य॒स्पो॒ष॒वनि॒रिति॑ रायस्पोष - वनिः॑ । स्वाहा॑ सिꣳ॒॒हीः । सिꣳ॒॒हीर॑सि । अ॒स्या॒दि॒त्य॒वनिः॑ । आ॒दि॒त्य॒वनिः॒ स्वाहा᳚ । आ॒दि॒त्य॒वनि॒रित्या॑दित्य - वनिः॑ । स्वाहा॑ सिꣳ॒॒हीः । सिꣳ॒॒हीर॑सि । अ॒स्या । आ व॑ह । व॒ह॒ दे॒वान् । दे॒वान् दे॑वय॒ते । दे॒व॒य॒ते यज॑मानाय । दे॒व॒य॒त इति॑ देव - य॒ते । यज॑मानाय॒ स्वाहा᳚ । स्वाहा॑ भू॒तेभ्यः॑ । भू॒तेभ्य॑स्त्वा । त्वा॒ वि॒श्वायुः॑ । वि॒श्वायु॑रसि । वि॒श्वायु॒रिति॑ वि॒श्व - आ॒युः॒ । अ॒सि॒ पृ॒थि॒वीम् । पृ॒थि॒वीम् दृꣳ॑ह । दृꣳ॒॒ह॒ ध्रु॒व॒क्षित् । ध्रु॒व॒क्षिद॑सि । धु॒व॒क्षिदिति॑ ध्रुव - क्षित् । अ॒स्य॒न्तरि॑क्षम् । अ॒न्तरि॑क्षम् दृꣳह । दृꣳ॒॒हा॒च्यु॒त॒क्षित् । अ॒च्यु॒त॒क्षिद॑सि । अ॒च्यु॒त॒क्षिदित्य॑च्युत - क्षित् । अ॒सि॒ दिव᳚म् । दिव॑म् दृꣳह । दृꣳ॒॒हा॒ग्नेः । अ॒ग्नेर् भस्म॑ । भस्मा॑सि । अ॒स्य॒ग्नेः । अ॒ग्नेः पुरी॑षम् । पुरी॑षमसि । अ॒सीत्य॑सि । \newline

\textbf{Jatai Paata} \newline

1. अ॒सि॒ रा॒य॒स्पो॒ष॒वनी॑ रायस्पोष॒वनि॑ रस्यसि रायस्पोष॒वनिः॑ । \newline
2. रा॒य॒स्पो॒ष॒वनिः॒ स्वाहा॒ स्वाहा॑ रायस्पोष॒वनी॑ रायस्पोष॒वनिः॒ स्वाहा᳚ । \newline
3. रा॒य॒स्पो॒ष॒वनि॒रिति॑ रायस्पोष - वनिः॑ । \newline
4. स्वाहा॑ सि॒(ग्म्॒)हीः सि॒(ग्म्॒)हीः स्वाहा॒ स्वाहा॑ सि॒(ग्म्॒)हीः । \newline
5. सि॒(ग्म्॒)ही र॑स्यसि सि॒(ग्म्॒)हीः सि॒(ग्म्॒)ही र॑सि । \newline
6. अ॒स्या॒दि॒त्य॒वनि॑ रादित्य॒वनि॑ रस्यस्यादित्य॒वनिः॑ । \newline
7. आ॒दि॒त्य॒वनिः॒ स्वाहा॒ स्वाहा॑ ऽऽदित्य॒वनि॑ रादित्य॒वनिः॒ स्वाहा᳚ । \newline
8. आ॒दि॒त्य॒वनि॒रित्या॑दित्य - वनिः॑ । \newline
9. स्वाहा॑ सि॒(ग्म्॒)हीः सि॒(ग्म्॒)हीः स्वाहा॒ स्वाहा॑ सि॒(ग्म्॒)हीः । \newline
10. सि॒(ग्म्॒)ही र॑स्यसि सि॒(ग्म्॒)हीः सि॒(ग्म्॒)ही र॑सि । \newline
11. अ॒स्या ऽस्य॒स्या । \newline
12. आ व॑ह व॒हा व॑ह । \newline
13. व॒ह॒ दे॒वान् दे॒वान्. व॑ह वह दे॒वान् । \newline
14. दे॒वान् दे॑वय॒ते दे॑वय॒ते दे॒वान् दे॒वान् दे॑वय॒ते । \newline
15. दे॒व॒य॒ते यज॑मानाय॒ यज॑मानाय देवय॒ते दे॑वय॒ते यज॑मानाय । \newline
16. दे॒व॒य॒त इति॑ देव - य॒ते । \newline
17. यज॑मानाय॒ स्वाहा॒ स्वाहा॒ यज॑मानाय॒ यज॑मानाय॒ स्वाहा᳚ । \newline
18. स्वाहा॑ भू॒तेभ्यो॑ भू॒तेभ्यः॒ स्वाहा॒ स्वाहा॑ भू॒तेभ्यः॑ । \newline
19. भू॒तेभ्य॑ स्त्वा त्वा भू॒तेभ्यो॑ भू॒तेभ्य॑ स्त्वा । \newline
20. त्वा॒ वि॒श्वायु॑र् वि॒श्वायु॑ष् ट्वा त्वा वि॒श्वायुः॑ । \newline
21. वि॒श्वायु॑ रस्यसि वि॒श्वायु॑र् वि॒श्वायु॑ रसि । \newline
22. वि॒श्वायु॒रिति॑ वि॒श्व - आ॒युः॒ । \newline
23. अ॒सि॒ पृ॒थि॒वीम् पृ॑थि॒वी म॑स्यसि पृथि॒वीम् । \newline
24. पृ॒थि॒वीम् दृ(ग्म्॑)ह दृꣳह पृथि॒वीम् पृ॑थि॒वीम् दृ(ग्म्॑)ह । \newline
25. दृ॒(ग्म्॒)ह॒ ध्रु॒व॒क्षिद् ध्रु॑व॒क्षिद् दृ(ग्म्॑)ह दृꣳह ध्रुव॒क्षित् । \newline
26. ध्रु॒व॒क्षि द॑स्यसि ध्रुव॒क्षिद् ध्रु॑व॒क्षि द॑सि । \newline
27. ध्रु॒व॒क्षिदिति॑ ध्रुव - क्षित् । \newline
28. अ॒स्य॒न्तरि॑क्ष म॒न्तरि॑क्ष मस्यस्य॒न्तरि॑क्षम् । \newline
29. अ॒न्तरि॑क्षम् दृꣳह दृꣳहा॒न्तरि॑क्ष म॒न्तरि॑क्षम् दृꣳह । \newline
30. दृ॒(ग्म्॒) हा॒च्यु॒त॒क्षि द॑च्युत॒क्षिद् दृ(ग्म्॑)ह दृꣳ हाच्युत॒क्षित् । \newline
31. अ॒च्यु॒त॒क्षि द॑स्यस्य च्युत॒क्षि द॑च्युत॒क्षि द॑सि । \newline
32. अ॒च्यु॒त॒क्षिदित्य॑च्युत - क्षित् । \newline
33. अ॒सि॒ दिव॒म् दिव॑ मस्यसि॒ दिव᳚म् । \newline
34. दिव॑म् दृꣳह दृꣳह॒ दिव॒म् दिव॑म् दृꣳह । \newline
35. दृ॒(ग्म्॒)हा॒ग्ने र॒ग्नेर् दृ(ग्म्॑)ह दृꣳहा॒ग्नेः । \newline
36. अ॒ग्नेर् भस्म॒ भस्मा॒ग्ने र॒ग्नेर् भस्म॑ । \newline
37. भस्मा᳚स्यसि॒ भस्म॒ भस्मा॑सि । \newline
38. अ॒स्य॒ग्ने र॒ग्ने र॑स्यस्य॒ग्नेः । \newline
39. अ॒ग्नेः पुरी॑ष॒म् पुरी॑ष म॒ग्ने र॒ग्नेः पुरी॑षम् । \newline
40. पुरी॑ष मस्यसि॒ पुरी॑ष॒म् पुरी॑ष मसि । \newline
41. अ॒सीत्य॑सि । \newline

\textbf{Ghana Paata } \newline

1. अ॒सि॒ रा॒य॒स्पो॒ष॒वनी॑ रायस्पोष॒वनि॑ रस्यसि रायस्पोष॒वनिः॒ स्वाहा॒ स्वाहा॑ रायस्पोष॒वनि॑ रस्यसि रायस्पोष॒वनिः॒ स्वाहा᳚ । \newline
2. रा॒य॒स्पो॒ष॒वनिः॒ स्वाहा॒ स्वाहा॑ रायस्पोष॒वनी॑ रायस्पोष॒वनिः॒ स्वाहा॑ सि॒(ग्म्॒)हीः सि॒(ग्म्॒)हीः स्वाहा॑ रायस्पोष॒वनी॑ रायस्पोष॒वनिः॒ स्वाहा॑ सि॒(ग्म्॒)हीः । \newline
3. रा॒य॒स्पो॒ष॒वनि॒रिति॑ रायस्पोष - वनिः॑ । \newline
4. स्वाहा॑ सि॒(ग्म्॒)हीः सि॒(ग्म्॒)हीः स्वाहा॒ स्वाहा॑ सि॒(ग्म्॒)ही र॑स्यसि सि॒(ग्म्॒)हीः स्वाहा॒ स्वाहा॑ सि॒(ग्म्॒)हीर॑सि । \newline
5. सि॒(ग्म्॒)हीर॑स्यसि सि॒(ग्म्॒)हीः सि॒(ग्म्॒)ही र॑स्यादित्य॒वनि॑ रादित्य॒वनि॑रसि सि॒(ग्म्॒)हीः सि॒(ग्म्॒)ही र॑स्यादित्य॒वनिः॑ । \newline
6. अ॒स्या॒दि॒त्य॒वनि॑ रादित्य॒वनि॑ रस्यस्यादित्य॒वनिः॒ स्वाहा॒ स्वाहा॑ ऽऽदित्य॒वनि॑ रस्यस्यादित्य॒वनिः॒ स्वाहा᳚ । \newline
7. आ॒दि॒त्य॒वनिः॒ स्वाहा॒ स्वाहा॑ ऽऽदित्य॒वनि॑ रादित्य॒वनिः॒ स्वाहा॑ सि॒(ग्म्॒)हीः सि॒(ग्म्॒)हीः स्वाहा॑ ऽऽदित्य॒वनि॑ रादित्य॒वनिः॒ स्वाहा॑ सि॒(ग्म्॒)हीः । \newline
8. आ॒दि॒त्य॒वनि॒रित्या॑दित्य - वनिः॑ । \newline
9. स्वाहा॑ सि॒(ग्म्॒)हीः सि॒(ग्म्॒)हीः स्वाहा॒ स्वाहा॑ सि॒(ग्म्॒)हीर॑स्यसि सि॒(ग्म्॒)हीः स्वाहा॒ स्वाहा॑ सि॒(ग्म्॒)हीर॑सि । \newline
10. सि॒(ग्म्॒)हीर॑स्यसि सि॒(ग्म्॒)हीः सि॒(ग्म्॒)हीर॒स्या ऽसि॑ सि॒(ग्म्॒)हीः सि॒(ग्म्॒)हीर॒स्या । \newline
11. अ॒स्या ऽस्य॒स्या व॑ह व॒हा ऽस्य॒स्या व॑ह । \newline
12. आ व॑ह व॒हा व॑ह दे॒वान् दे॒वान्. व॒हा व॑ह दे॒वान् । \newline
13. व॒ह॒ दे॒वान् दे॒वान्. व॑ह वह दे॒वान् दे॑वय॒ते दे॑वय॒ते दे॒वान्. व॑ह वह दे॒वान् दे॑वय॒ते । \newline
14. दे॒वान् दे॑वय॒ते दे॑वय॒ते दे॒वान् दे॒वान् दे॑वय॒ते यज॑मानाय॒ यज॑मानाय देवय॒ते दे॒वान् दे॒वान् दे॑वय॒ते यज॑मानाय । \newline
15. दे॒व॒य॒ते यज॑मानाय॒ यज॑मानाय देवय॒ते दे॑वय॒ते यज॑मानाय॒ स्वाहा॒ स्वाहा॒ यज॑मानाय देवय॒ते दे॑वय॒ते यज॑मानाय॒ स्वाहा᳚ । \newline
16. दे॒व॒य॒त इति॑ देव - य॒ते । \newline
17. यज॑मानाय॒ स्वाहा॒ स्वाहा॒ यज॑मानाय॒ यज॑मानाय॒ स्वाहा॑ भू॒तेभ्यो॑ भू॒तेभ्यः॒ स्वाहा॒ यज॑मानाय॒ यज॑मानाय॒ स्वाहा॑ भू॒तेभ्यः॑ । \newline
18. स्वाहा॑ भू॒तेभ्यो॑ भू॒तेभ्यः॒ स्वाहा॒ स्वाहा॑ भू॒तेभ्य॑स्त्वा त्वा भू॒तेभ्यः॒ स्वाहा॒ स्वाहा॑ भू॒तेभ्य॑स्त्वा । \newline
19. भू॒तेभ्य॑स्त्वा त्वा भू॒तेभ्यो॑ भू॒तेभ्य॑स्त्वा वि॒श्वायु॑र् वि॒श्वायु॑ष्ट्वा भू॒तेभ्यो॑ भू॒तेभ्य॑स्त्वा वि॒श्वायुः॑ । \newline
20. त्वा॒ वि॒श्वायु॑र् वि॒श्वायु॑ष्ट्वा त्वा वि॒श्वायु॑ रस्यसि वि॒श्वायु॑ष्ट्वा त्वा वि॒श्वायु॑रसि । \newline
21. वि॒श्वायु॑रस्यसि वि॒श्वायु॑र् वि॒श्वायु॑रसि पृथि॒वीम् पृ॑थि॒वी म॑सि वि॒श्वायु॑र् वि॒श्वायु॑रसि पृथि॒वीम् । \newline
22. वि॒श्वायु॒रिति॑ वि॒श्व - आ॒युः॒ । \newline
23. अ॒सि॒ पृ॒थि॒वीम् पृ॑थि॒वी म॑स्यसि पृथि॒वीम् दृ(ग्म्॑)ह दृꣳह पृथि॒वी म॑स्यसि पृथि॒वीम् दृ(ग्म्॑)ह । \newline
24. पृ॒थि॒वीम् दृ(ग्म्॑)ह दृꣳह पृथि॒वीम् पृ॑थि॒वीम् दृ(ग्म्॑)ह ध्रुव॒क्षिद् ध्रु॑व॒क्षिद् दृ(ग्म्॑)ह पृथि॒वीम् पृ॑थि॒वीम् दृ(ग्म्॑)ह ध्रुव॒क्षित् । \newline
25. दृ॒(ग्म्॒)ह॒ ध्रु॒व॒क्षिद् ध्रु॑व॒क्षिद् दृ(ग्म्॑)ह दृꣳह ध्रुव॒क्षिद॑स्यसि ध्रुव॒क्षिद् दृ(ग्म्॑)ह दृꣳह ध्रुव॒क्षिद॑सि । \newline
26. ध्रु॒व॒क्षि द॑स्यसि ध्रुव॒क्षिद् ध्रु॑व॒क्षि द॑स्य॒न्तरि॑क्ष म॒न्तरि॑क्ष मसि ध्रुव॒क्षिद् ध्रु॑व॒क्षि द॑स्य॒न्तरि॑क्षम् । \newline
27. ध्रु॒व॒क्षिदिति॑ ध्रुव - क्षित् । \newline
28. अ॒स्य॒न्तरि॑क्ष म॒न्तरि॑क्ष मस्यस्य॒न्तरि॑क्षम् दृꣳह दृꣳहा॒न्तरि॑क्ष मस्यस्य॒न्तरि॑क्षम् दृꣳह । \newline
29. अ॒न्तरि॑क्षम् दृꣳह दृꣳहा॒न्तरि॑क्ष म॒न्तरि॑क्षम् दृꣳहाच्युत॒क्षि द॑च्युत॒क्षिद् दृ(ग्म्॑)हा॒न्तरि॑क्ष म॒न्तरि॑क्षम् दृꣳहाच्युत॒क्षित् । \newline
30. दृ॒(ग्म्॒)हा॒च्यु॒त॒क्षि द॑च्युत॒क्षिद् दृ(ग्म्॑)ह दृꣳहाच्युत॒क्षि द॑स्यस्यच्युत॒क्षिद् दृ(ग्म्॑)ह दृꣳहाच्युत॒क्षिद॑सि । \newline
31. अ॒च्यु॒त॒क्षि द॑स्यस्यच्युत॒क्षि द॑च्युत॒क्षिद॑सि॒ दिव॒म् दिव॑ मस्यच्युत॒क्षि द॑च्युत॒क्षिद॑सि॒ दिव᳚म् । \newline
32. अ॒च्यु॒त॒क्षिदित्य॑च्युत - क्षित् । \newline
33. अ॒सि॒ दिव॒म् दिव॑ मस्यसि॒ दिव॑म् दृꣳह दृꣳह॒ दिव॑ मस्यसि॒ दिव॑म् दृꣳह । \newline
34. दिव॑म् दृꣳह दृꣳह॒ दिव॒म् दिव॑म् दृꣳहा॒ग्नेर॒ग्नेर् दृ(ग्म्॑)ह॒ दिव॒म् दिव॑म् दृꣳहा॒ग्नेः । \newline
35. दृ॒(ग्म्॒)हा॒ग्नेर॒ग्नेर् दृ(ग्म्॑)ह दृꣳहा॒ग्नेर् भस्म॒ भस्मा॒ग्नेर् दृ(ग्म्॑)ह दृꣳहा॒ग्नेर् भस्म॑ । \newline
36. अ॒ग्नेर् भस्म॒ भस्मा॒ग्ने र॒ग्नेर् भस्मा᳚स्यसि॒ भस्मा॒ग्ने र॒ग्नेर् भस्मा॑सि । \newline
37. भस्मा᳚स्यसि॒ भस्म॒ भस्मा᳚ स्य॒ग्ने र॒ग्नेर॑सि॒ भस्म॒ भस्मा᳚स्य॒ग्नेः । \newline
38. अ॒स्य॒ग्ने र॒ग्ने र॑स्यस्य॒ग्नेः पुरी॑ष॒म् पुरी॑ष म॒ग्ने र॑स्यस्य॒ग्नेः पुरी॑षम् । \newline
39. अ॒ग्नेः पुरी॑ष॒म् पुरी॑ष म॒ग्नेर॒ग्नेः पुरी॑ष मस्यसि॒ पुरी॑ष म॒ग्नेर॒ग्नेः पुरी॑ष मसि । \newline
40. पुरी॑ष मस्यसि॒ पुरी॑ष॒म् पुरी॑ष मसि । \newline
41. अ॒सीत्य॑सि । \newline
\pagebreak
\markright{ TS 1.2.13.1  \hfill https://www.vedavms.in \hfill}

\section{ TS 1.2.13.1 }

\textbf{TS 1.2.13.1 } \newline
\textbf{Samhita Paata} \newline

यु॒ञ्जते॒ मन॑ उ॒त यु॑ञ्जते॒ धियो॒ विप्रा॒ विप्र॑स्य बृह॒तो वि॑प॒श्चितः॑ । वि होत्रा॑ दधे वयुना॒विदेक॒ इन्म॒ही दे॒वस्य॑ सवि॒तुः परि॑ष्टुतिः ॥ सु॒वाग्दे॑व॒ दुर्याꣳ॒॒ आ व॑द देव॒श्रुतौ॑ दे॒वेष्वा घो॑षेथा॒मा नो॑ वी॒रो जा॑यतां कर्म॒ण्यो॑ यꣳ सर्वे॑ऽनु॒ जीवा॑म॒ यो ब॑हू॒नामस॑द्व॒शी ॥ इ॒दं ॅविष्णु॒र् विच॑क्रमे त्रे॒धा नि द॑धे प॒दं ॥ समू॑ढमस्य - [ ] \newline

\textbf{Pada Paata} \newline

यु॒ञ्जते᳚ । मनः॑ । उ॒त । यु॒ञ्ज॒ते॒ । धियः॑ । विप्राः᳚ । विप्र॑स्य । बृ॒ह॒तः । वि॒प॒श्चितः॑ ॥ वीति॑ । होत्राः᳚ । द॒धे॒ । व॒यु॒ना॒विदिति॑ वयुन - वित् । एकः॑ । इत् । म॒ही । दे॒वस्य॑ । स॒वि॒तुः । परि॑ष्टुति॒रिति॒ परि॑ - स्तु॒तिः॒ ॥ सु॒वागिति॑ सु- वाक् । दे॒व॒ । दुर्यान्॑ । एति॑ । व॒द॒ । दे॒व॒श्रुता॒विति॑ देव - श्रुतौ᳚ । दे॒वेषु॑ । एति॑ । घो॒षे॒था॒म् । एति॑ । नः॒ । वी॒रः । जा॒य॒ता॒म् । क॒र्म॒ण्यः॑ । यम् । सर्वे᳚ । अ॒नु॒जीवा॒मेत्य॑नु - जीवा॑म । यः । ब॒हू॒नाम् । अस॑त् । व॒शी ॥ इ॒दम् । विष्णुः॑ । वीति॑ । च॒क्र॒मे॒ । त्रे॒धा । नीति॑ । द॒धे॒ । प॒दम् ॥ समू॑ढ॒मिति॒ सम् - ऊ॒ढ॒म् । अ॒स्य॒ ।  \newline


\textbf{Krama Paata} \newline

यु॒ञ्जते॒ मनः॑ । मन॑ उ॒त । उ॒त यु॑ञ्जते । यु॒ञ्ज॒ते॒ धियः॑ । धियो॒ विप्राः᳚ । विप्रा॒ विप्र॑स्य । विप्र॑स्य बृह॒तः । बृ॒ह॒तो वि॑प॒श्चितः॑ । वि॒प॒श्चित॒ इति॑ विप॒श्चितः॑ ॥ वि होत्राः᳚ । होत्रा॑ दधे । द॒धे॒ व॒यु॒ना॒वित् । व॒यु॒ना॒विदेकः॑ । व॒यु॒ना॒विदिति॑ वयुन - वित् । एक॒ इत् । इन् म॒ही । म॒ही दे॒वस्य॑ । दे॒वस्य॑ सवि॒तुः । स॒वि॒तुः परि॑ष्टुतिः । परि॑ष्टुति॒रिति॒ परि॑ - स्तु॒तिः॒ ॥ सु॒वाग् दे॑व । सु॒वागिति॑ सु - वाक् । दे॒व॒ दुर्यान्॑ । दुर्याꣳ॒॒ आ । आ व॑द । व॒द॒ दे॒व॒श्रुतौ᳚ । दे॒व॒श्रुतौ॑ दे॒वेषु॑ । दे॒व॒श्रुता॒विति॑ देव - श्रुतौ᳚ । दे॒वेष्वा । आ घो॑षेथाम् । घो॒षे॒था॒मा । आ नः॑ । नो॒ वी॒रः । वी॒रो जा॑यताम् । जा॒य॒ता॒म् क॒र्म॒ण्यः॑ । क॒र्म॒ण्यो॑ यम् । यꣳ सर्वे᳚ । सर्वे॑ऽनु॒जीवा॑म । अ॒नु॒जीवा॑म॒ यः । अ॒नु॒जीवा॒मेत्य॑नु - जीवा॑म । यो ब॑हू॒नाम् । ब॒हू॒नामस॑त् । अस॑द् व॒शी । व॒शीति॑ व॒शी ॥ इ॒दं ॅविष्णुः॑ । विष्णु॒र् वि । वि च॑क्रमे । च॒क्र॒मे॒ त्रे॒धा । त्रे॒धा नि । नि द॑धे । द॒धे॒ प॒दम् । प॒दमिति॑ प॒दम् ॥ समू॑ढमस्य । समू॑ढ॒मिति॒ सं - ऊ॒ढ॒म् । अ॒स्य॒ पाꣳ॒॒सु॒रे \newline

\textbf{Jatai Paata} \newline

1. यु॒ञ्जते॒ मनो॒ मनो॑ यु॒ञ्जते॑ यु॒ञ्जते॒ मनः॑ । \newline
2. मन॑ उ॒तोत मनो॒ मन॑ उ॒त । \newline
3. उ॒त यु॑ञ्जते युञ्जत उ॒तोत यु॑ञ्जते । \newline
4. यु॒ञ्ज॒ते॒ धियो॒ धियो॑ युञ्जते युञ्जते॒ धियः॑ । \newline
5. धियो॒ विप्रा॒ विप्रा॒ धियो॒ धियो॒ विप्राः᳚ । \newline
6. विप्रा॒ विप्र॑स्य॒ विप्र॑स्य॒ विप्रा॒ विप्रा॒ विप्र॑स्य । \newline
7. विप्र॑स्य बृह॒तो बृ॑ह॒तो विप्र॑स्य॒ विप्र॑स्य बृह॒तः । \newline
8. बृ॒ह॒तो वि॑प॒श्चितो॑ विप॒श्चितो॑ बृह॒तो बृ॑ह॒तो वि॑प॒श्चितः॑ । \newline
9. वि॒प॒श्चित॒ इति॑ विप॒श्चितः॑ । \newline
10. वि होत्रा॒ होत्रा॒ वि वि होत्राः᳚ । \newline
11. होत्रा॑ दधे दधे॒ होत्रा॒ होत्रा॑ दधे । \newline
12. द॒धे॒ व॒यु॒ना॒विद् व॑युना॒विद् द॑धे दधे वयुना॒वित् । \newline
13. व॒यु॒ना॒विदेक॒ एको॑ वयुना॒विद् व॑युना॒विदेकः॑ । \newline
14. व॒यु॒ना॒विदिति॑ वयुन - वित् । \newline
15. एक॒ इदिदेक॒ एक॒ इत् । \newline
16. इन् म॒ही म॒हीदिन् म॒ही । \newline
17. म॒ही दे॒वस्य॑ दे॒वस्य॑ म॒ही म॒ही दे॒वस्य॑ । \newline
18. दे॒वस्य॑ सवि॒तुः स॑वि॒तुर् दे॒वस्य॑ दे॒वस्य॑ सवि॒तुः । \newline
19. स॒वि॒तुः परि॑ष्टुतिः॒ परि॑ष्टुतिः सवि॒तुः स॑वि॒तुः परि॑ष्टुतिः । \newline
20. परि॑ष्टुति॒रिति॒ परि॑ - स्तु॒तिः॒ । \newline
21. सु॒वाग् दे॑व देव सु॒वाख् सु॒वाग् दे॑व । \newline
22. सु॒वागिति॑ सु - वाक् । \newline
23. दे॒व॒ दुर्या॒न् दुर्या᳚न् देव देव॒ दुर्यान्॑ । \newline
24. दुर्या॒(ग्म्॒) आ दुर्या॒न् दुर्या॒(ग्म्॒) आ । \newline
25. आ व॑द व॒दा व॑द । \newline
26. व॒द॒ दे॒व॒श्रुतौ॑ देव॒श्रुतौ॑ वद वद देव॒श्रुतौ᳚ । \newline
27. दे॒व॒श्रुतौ॑ दे॒वेषु॑ दे॒वेषु॑ देव॒श्रुतौ॑ देव॒श्रुतौ॑ दे॒वेषु॑ । \newline
28. दे॒व॒श्रुता॒विति॑ देव - श्रुतौ᳚ । \newline
29. दे॒वेष्वा दे॒वेषु॑ दे॒वेष्वा । \newline
30. आ घो॑षेथाम् घोषेथा॒ मा घो॑षेथाम् । \newline
31. घो॒षे॒था॒ मा घो॑षेथाम् घोषेथा॒ मा । \newline
32. आ नो॑ न॒ आ नः॑ । \newline
33. नो॒ वी॒रो वी॒रो नो॑ नो वी॒रः । \newline
34. वी॒रो जा॑यताम् जायतां ॅवी॒रो वी॒रो जा॑यताम् । \newline
35. जा॒य॒ता॒म् क॒र्म॒ण्यः॑ कर्म॒ण्यो॑ जायताम् जायताम् कर्म॒ण्यः॑ । \newline
36. क॒र्म॒ण्यो॑ यं ॅयम् क॑र्म॒ण्यः॑ कर्म॒ण्यो॑ यम् । \newline
37. यꣳ सर्वे॒ सर्वे॒ यं ॅयꣳ सर्वे᳚ । \newline
38. सर्वे॑ ऽनु॒जीवा॑मा नु॒जीवा॑म॒ सर्वे॒ सर्वे॑ ऽनु॒जीवा॑म । \newline
39. अ॒नु॒जीवा॑म॒ यो यो॑ ऽनु॒जीवा॑मानु॒जीवा॑म॒ यः । \newline
40. अ॒नु॒जीवा॒मेत्य॑नु - जीवा॑म । \newline
41. यो ब॑हू॒नाम् ब॑हू॒नां ॅयो यो ब॑हू॒नाम् । \newline
42. ब॒हू॒ना मस॒दस॑द् बहू॒नाम् ब॑हू॒ना मस॑त् । \newline
43. अस॑द् व॒शी व॒श्यस॒दस॑द् व॒शी । \newline
44. व॒शीति॑ व॒शी । \newline
45. इ॒दं ॅविष्णु॒र् विष्णु॑ रि॒द मि॒दं ॅविष्णुः॑ । \newline
46. विष्णु॒र् वि वि विष्णु॒र् विष्णु॒र् वि । \newline
47. वि च॑क्रमे चक्रमे॒ वि वि च॑क्रमे । \newline
48. च॒क्र॒मे॒ त्रे॒धा त्रे॒धा च॑क्रमे चक्रमे त्रे॒धा । \newline
49. त्रे॒धा नि नि त्रे॒धा त्रे॒धा नि । \newline
50. नि द॑धे दधे॒ नि नि द॑धे । \newline
51. द॒धे॒ प॒दम् प॒दम् द॑धे दधे प॒दम् । \newline
52. प॒दमिति॑ प॒दम् । \newline
53. समू॑ढ मस्यास्य॒ समू॑ढ॒(ग्म्॒) समू॑ढ मस्य । \newline
54. समू॑ढ॒मिति॒ सम् - ऊ॒ढ॒म् । \newline
55. अ॒स्य॒ पा॒(ग्म्॒)सु॒रे पा(ग्म्॑)सु॒रे᳚ ऽस्यास्य पाꣳसु॒रे । \newline

\textbf{Ghana Paata } \newline

1. यु॒ञ्जते॒ मनो॒ मनो॑ यु॒ञ्जते॑ यु॒ञ्जते॒ मन॑ उ॒तोत मनो॑ यु॒ञ्जते॑ यु॒ञ्जते॒ मन॑ उ॒त । \newline
2. मन॑ उ॒तोत मनो॒ मन॑ उ॒त यु॑ञ्जते युञ्जत उ॒त मनो॒ मन॑ उ॒त यु॑ञ्जते । \newline
3. उ॒त यु॑ञ्जते युञ्जत उ॒तोत यु॑ञ्जते॒ धियो॒ धियो॑ युञ्जत उ॒तोत यु॑ञ्जते॒ धियः॑ । \newline
4. यु॒ञ्ज॒ते॒ धियो॒ धियो॑ युञ्जते युञ्जते॒ धियो॒ विप्रा॒ विप्रा॒ धियो॑ युञ्जते युञ्जते॒ धियो॒ विप्राः᳚ । \newline
5. धियो॒ विप्रा॒ विप्रा॒ धियो॒ धियो॒ विप्रा॒ विप्र॑स्य॒ विप्र॑स्य॒ विप्रा॒ धियो॒ धियो॒ विप्रा॒ विप्र॑स्य । \newline
6. विप्रा॒ विप्र॑स्य॒ विप्र॑स्य॒ विप्रा॒ विप्रा॒ विप्र॑स्य बृह॒तो बृ॑ह॒तो विप्र॑स्य॒ विप्रा॒ विप्रा॒ विप्र॑स्य बृह॒तः । \newline
7. विप्र॑स्य बृह॒तो बृ॑ह॒तो विप्र॑स्य॒ विप्र॑स्य बृह॒तो वि॑प॒श्चितो॑ विप॒श्चितो॑ बृह॒तो विप्र॑स्य॒ विप्र॑स्य बृह॒तो वि॑प॒श्चितः॑ । \newline
8. बृ॒ह॒तो वि॑प॒श्चितो॑ विप॒श्चितो॑ बृह॒तो बृ॑ह॒तो वि॑प॒श्चितः॑ । \newline
9. वि॒प॒श्चित॒ इति॑ विप॒श्चितः॑ । \newline
10. वि होत्रा॒ होत्रा॒ वि वि होत्रा॑ दधे दधे॒ होत्रा॒ वि वि होत्रा॑ दधे । \newline
11. होत्रा॑ दधे दधे॒ होत्रा॒ होत्रा॑ दधे वयुना॒विद् व॑युना॒विद् द॑धे॒ होत्रा॒ होत्रा॑ दधे वयुना॒वित् । \newline
12. द॒धे॒ व॒यु॒ना॒विद् व॑युना॒विद् द॑धे दधे वयुना॒विदेक॒ एको॑ वयुना॒विद् द॑धे दधे वयुना॒विदेकः॑ । \newline
13. व॒यु॒ना॒विदेक॒ एको॑ वयुना॒विद् व॑युना॒विदेक॒ इदिदेको॑ वयुना॒विद् व॑युना॒विदेक॒ इत् । \newline
14. व॒यु॒ना॒विदिति॑ वयुन - वित् । \newline
15. एक॒ इदिदेक॒ एक॒ इन् म॒ही म॒हीदेक॒ एक॒ इन् म॒ही । \newline
16. इन् म॒ही म॒हीदिन् म॒ही दे॒वस्य॑ दे॒वस्य॑ म॒हीदिन् म॒ही दे॒वस्य॑ । \newline
17. म॒ही दे॒वस्य॑ दे॒वस्य॑ म॒ही म॒ही दे॒वस्य॑ सवि॒तुः स॑वि॒तुर् दे॒वस्य॑ म॒ही म॒ही दे॒वस्य॑ सवि॒तुः । \newline
18. दे॒वस्य॑ सवि॒तुः स॑वि॒तुर् दे॒वस्य॑ दे॒वस्य॑ सवि॒तुः परि॑ष्टुतिः॒ परि॑ष्टुतिः सवि॒तुर् दे॒वस्य॑ दे॒वस्य॑ सवि॒तुः परि॑ष्टुतिः । \newline
19. स॒वि॒तुः परि॑ष्टुतिः॒ परि॑ष्टुतिः सवि॒तुः स॑वि॒तुः परि॑ष्टुतिः । \newline
20. परि॑ष्टुति॒रिति॒ परि॑ - स्तु॒तिः॒ । \newline
21. सु॒वाग् दे॑व देव सु॒वाख् सु॒वाग् दे॑व॒ दुर्या॒न् दुर्या᳚न् देव सु॒वाख् सु॒वाग् दे॑व॒ दुर्यान्॑ । \newline
22. सु॒वागिति॑ सु - वाक् । \newline
23. दे॒व॒ दुर्या॒न् दुर्या᳚न् देव देव॒ दुर्या॒(ग्म्॒) आ दुर्या᳚न् देव देव॒ दुर्या॒(ग्म्॒) आ । \newline
24. दुर्या॒(ग्म्॒) आ दुर्या॒न् दुर्या॒(ग्म्॒) आ व॑द व॒दा दुर्या॒न् दुर्या॒(ग्म्॒) आ व॑द । \newline
25. आ व॑द व॒दा व॑द देव॒श्रुतौ॑ देव॒श्रुतौ॑ व॒दा व॑द देव॒श्रुतौ᳚ । \newline
26. व॒द॒ दे॒व॒श्रुतौ॑ देव॒श्रुतौ॑ वद वद देव॒श्रुतौ॑ दे॒वेषु॑ दे॒वेषु॑ देव॒श्रुतौ॑ वद वद देव॒श्रुतौ॑ दे॒वेषु॑ । \newline
27. दे॒व॒श्रुतौ॑ दे॒वेषु॑ दे॒वेषु॑ देव॒श्रुतौ॑ देव॒श्रुतौ॑ दे॒वेष्वा दे॒वेषु॑ देव॒श्रुतौ॑ देव॒श्रुतौ॑ दे॒वेष्वा । \newline
28. दे॒व॒श्रुता॒विति॑ देव - श्रुतौ᳚ । \newline
29. दे॒वेष्वा दे॒वेषु॑ दे॒वेष्वा घो॑षेथाम् घोषेथा॒ मा दे॒वेषु॑ दे॒वेष्वा घो॑षेथाम् । \newline
30. आ घो॑षेथाम् घोषेथा॒ मा घो॑षेथा॒ मा घो॑षेथा॒ मा घो॑षेथा॒ मा । \newline
31. घो॒षे॒था॒ मा घो॑षेथाम् घोषेथा॒ मा नो॑ न॒ आ घो॑षेथाम् घोषेथा॒ मा नः॑ । \newline
32. आ नो॑ न॒ आ नो॑ वी॒रो वी॒रो न॒ आ नो॑ वी॒रः । \newline
33. नो॒ वी॒रो वी॒रो नो॑ नो वी॒रो जा॑यताम् जायतां ॅवी॒रो नो॑ नो वी॒रो जा॑यताम् । \newline
34. वी॒रो जा॑यताम् जायतां ॅवी॒रो वी॒रो जा॑यताम् कर्म॒ण्यः॑ कर्म॒ण्यो॑ जायतां ॅवी॒रो वी॒रो जा॑यताम् कर्म॒ण्यः॑ । \newline
35. जा॒य॒ता॒म् क॒र्म॒ण्यः॑ कर्म॒ण्यो॑ जायताम् जायताम् कर्म॒ण्यो॑ यं ॅयम् क॑र्म॒ण्यो॑ जायताम् जायताम् कर्म॒ण्यो॑ यम् । \newline
36. क॒र्म॒ण्यो॑ यं ॅयम् क॑र्म॒ण्यः॑ कर्म॒ण्यो॑ यꣳ सर्वे॒ सर्वे॒ यम् क॑र्म॒ण्यः॑ कर्म॒ण्यो॑ यꣳ सर्वे᳚ । \newline
37. यꣳ सर्वे॒ सर्वे॒ यं ॅयꣳ सर्वे॑ ऽनु॒जीवा॑मानु॒जीवा॑म॒ सर्वे॒ यं ॅयꣳ सर्वे॑ ऽनु॒जीवा॑म । \newline
38. सर्वे॑ ऽनु॒जीवा॑मानु॒जीवा॑म॒ सर्वे॒ सर्वे॑ ऽनु॒जीवा॑म॒ यो यो॑ ऽनु॒जीवा॑म॒ सर्वे॒ सर्वे॑ ऽनु॒जीवा॑म॒ यः । \newline
39. अ॒नु॒जीवा॑म॒ यो यो॑ ऽनु॒जीवा॑मानु॒जीवा॑म॒ यो ब॑हू॒नाम् ब॑हू॒नां ॅयो॑ ऽनु॒जीवा॑मानु॒जीवा॑म॒ यो ब॑हू॒नाम् । \newline
40. अ॒नु॒जीवा॒मेत्य॑नु - जीवा॑म । \newline
41. यो ब॑हू॒नाम् ब॑हू॒नां ॅयो यो ब॑हू॒ना मस॒दस॑द् बहू॒नां ॅयो यो ब॑हू॒ना मस॑त् । \newline
42. ब॒हू॒ना मस॒दस॑द् बहू॒नाम् ब॑हू॒ना मस॑द् व॒शी व॒श्यस॑द् बहू॒नाम् ब॑हू॒ना मस॑द् व॒शी । \newline
43. अस॑द् व॒शी व॒श्यस॒दस॑द् व॒शी । \newline
44. व॒शीति॑ व॒शी । \newline
45. इ॒दं ॅविष्णु॒र् विष्णु॑रि॒द मि॒दं ॅविष्णु॒र् वि वि विष्णु॑रि॒द मि॒दं ॅविष्णु॒र् वि । \newline
46. विष्णु॒र् वि वि विष्णु॒र् विष्णु॒र् वि च॑क्रमे चक्रमे॒ वि विष्णु॒र् विष्णु॒र् वि च॑क्रमे । \newline
47. वि च॑क्रमे चक्रमे॒ वि वि च॑क्रमे त्रे॒धा त्रे॒धा च॑क्रमे॒ वि वि च॑क्रमे त्रे॒धा । \newline
48. च॒क्र॒मे॒ त्रे॒धा त्रे॒धा च॑क्रमे चक्रमे त्रे॒धा नि नि त्रे॒धा च॑क्रमे चक्रमे त्रे॒धा नि । \newline
49. त्रे॒धा नि नि त्रे॒धा त्रे॒धा नि द॑धे दधे॒ नि त्रे॒धा त्रे॒धा नि द॑धे । \newline
50. नि द॑धे दधे॒ नि नि द॑धे प॒दम् प॒दम् द॑धे॒ नि नि द॑धे प॒दम् । \newline
51. द॒धे॒ प॒दम् प॒दम् द॑धे दधे प॒दम् । \newline
52. प॒दमिति॑ प॒दम् । \newline
53. समू॑ढ मस्यास्य॒ समू॑ढ॒(ग्म्॒) समू॑ढ मस्य पाꣳसु॒रे पा(ग्म्॑)सु॒रे᳚ ऽस्य॒ समू॑ढ॒(ग्म्॒) समू॑ढ मस्य पाꣳसु॒रे । \newline
54. समू॑ढ॒मिति॒ सम् - ऊ॒ढ॒म् । \newline
55. अ॒स्य॒ पा॒(ग्म्॒)सु॒रे पा(ग्म्॑)सु॒रे᳚ ऽस्यास्य पाꣳसु॒र इरा॑वती॒ इरा॑वती पाꣳसु॒रे᳚ ऽस्यास्य पाꣳसु॒र इरा॑वती । \newline
\pagebreak
\markright{ TS 1.2.13.2  \hfill https://www.vedavms.in \hfill}

\section{ TS 1.2.13.2 }

\textbf{TS 1.2.13.2 } \newline
\textbf{Samhita Paata} \newline

पाꣳसु॒॒र इरा॑वती धेनु॒मती॒ हि भू॒तꣳ सू॑यव॒सिनी॒ मन॑वे यश॒स्ये᳚ । व्य॑स्कभ्ना॒द्-रोद॑सी॒ विष्णु॑रे॒ते दा॒धार॑ पृथि॒वीम॒भितो॑ म॒यूखैः᳚ ॥ प्राची॒ प्रेत॑मद्ध्व॒रं क॒ल्पय॑न्ती ऊ॒र्द्ध्वं ॅय॒ज्ञ्ं न॑यतं॒ मा जी᳚ह्वरत॒मत्र॑ रमेथां॒ ॅवर्ष्म॑न् पृथि॒व्या दि॒वो वा॑ विष्णवु॒त वा॑ पृथि॒व्या म॒हो वा॑ विष्णवु॒त वा॒ऽन्तरि॑क्षा॒द्धस्तौ॑ पृणस्व ब॒हुभि॑र् वस॒व्यै॑रा प्र य॑च्छ॒ - [ ] \newline

\textbf{Pada Paata} \newline

पाꣳ॒॒सु॒रे । इरा॑वती॒ इतीरा᳚ - व॒ती॒ । धे॒नु॒मती॒ इति॑ धेनु - मती᳚ । हि । भू॒तम् । सू॒य॒व॒सिनी॒ इति॑ सु - य॒व॒सिनी᳚ । मन॑वे । य॒श॒स्ये॑ इति॑ ॥ वीति॑ । अ॒स्क॒भ्ना॒त् । रोद॑सी॒ इति॑ । विष्णुः॑ । ए॒ते इति॑ । दा॒धार॑ । पृ॒थि॒वीम् । अ॒भितः॑ । म॒यूखैः᳚ ॥ प्राची॒ इति॑ । प्रेति॑ । इ॒त॒म् । अ॒द्ध्व॒रम् । क॒ल्पय॑न्ती॒ इति॑ । ऊ॒र्द्ध्वम् । य॒ज्ञ्म् । न॒य॒त॒म् । मा । जी॒ह्व॒र॒त॒म् । अत्र॑ । र॒मे॒था॒म् । वर्.ष्मन्न्॑ । पृ॒थि॒व्याः । दि॒वः । वा॒ । वि॒ष्णो॒ । उ॒त । वा॒ । पृ॒थि॒व्याः । म॒हः । वा॒ । वि॒ष्णो॒ । उ॒त । वा॒ । अ॒न्तरि॑क्षात् । हस्तौ᳚ । पृ॒ण॒स्व॒ । ब॒हुभि॒रिति॑ ब॒हु - भिः॒ । व॒स॒व्यैः᳚ । आ* । प्रेति॑ । य॒च्छ॒ ।  \newline


\textbf{Krama Paata} \newline

पाꣳ॒॒सु॒र इरा॑वती । इरा॑वती धेनु॒मती᳚ । इरा॑वती॒ इतीरा᳚ - व॒ती॒ । धे॒नु॒मती॒ हि । धे॒नु॒मती॒ इति॑ धेनु - मती᳚ । हि भू॒तम् । भू॒तꣳ सू॑यव॒सिनी᳚ । सू॒य॒व॒सिनी॒ मन॑वे । सू॒य॒व॒सिनी॒ इति॑ सु - य॒व॒सिनी᳚ । मन॑वे यश॒स्ये᳚ । य॒श॒स्ये॑ इति॑ यश॒स्ये᳚ ॥ व्य॑स्कभ्नात् । अ॒स्क॒भ्ना॒द् रोद॑सी । रोद॑सी॒ विष्णुः॑ । रोद॑सी॒ इति॒ रोद॑सी । विष्णु॑रे॒ते । ए॒ते दा॒धार॑ । ए॒ते इत्ये॒ते । दा॒धार॑ पृथि॒वीम् । पृ॒थि॒वीम॒भितः॑ । अ॒भितो॑ म॒यूखैः᳚ । म॒यूखै॒रिति॑ म॒यूखैः᳚ ॥ प्राची॒ प्र । प्राची॒ इति॒ प्राची᳚ । प्रेत᳚म् । इ॒त॒म॒द्ध्व॒रम् । अ॒द्ध्व॒रम् क॒ल्पय॑न्ती । क॒ल्पय॑न्ती ऊ॒र्द्ध्वम् । क॒ल्पय॑न्ती॒ इति॑ क॒ल्पय॑न्ती । ऊ॒र्द्ध्वं ॅय॒ज्ञ्म् । य॒ज्ञ्ं न॑यतम् । न॒य॒त॒म् मा । मा जी᳚ह्वरतम् । जी॒ह्व॒र॒त॒मत्र॑ । अत्र॑ रमेथाम् । र॒मे॒थां॒ ॅवर्ष्मन्न्॑ । वर्ष्म॑न् पृथि॒व्याः । पृ॒थि॒व्या दि॒वः । दि॒वो वा᳚ । वा॒ वि॒ष्णो॒ । वि॒ष्ण॒वु॒त । उ॒त वा᳚ । वा॒ पृ॒थि॒व्याः । पृ॒थि॒व्या म॒हः । म॒हो वा᳚ । वा॒ वि॒ष्णो॒ । वि॒ष्ण॒वु॒त । उ॒त वा᳚ । वा॒ऽन्तरि॑क्षात् । अ॒न्तरि॑क्षा॒द्धस्तौ᳚ । हस्तौ॑ पृणस्व । पृ॒ण॒स्व॒ ब॒हुभिः॑ । ब॒हुभि॑र् वस॒व्यैः᳚ । ब॒हुभि॒रिति॑ ब॒हु - भिः॒ । व॒स॒वै॑रा । आ प्र । प्र य॑च्छ ( ) । य॒च्छ॒ दक्षि॑णात् \newline

\textbf{Jatai Paata} \newline

1. पा॒(ग्म्॒)सु॒र इरा॑वती॒ इरा॑वती पाꣳसु॒रे पा(ग्म्॑)सु॒र इरा॑वती । \newline
2. इरा॑वती धेनु॒मती॑ धेनु॒मती॒ इरा॑वती॒ इरा॑वती धेनु॒मती᳚ । \newline
3. इरा॑वती॒ इतीरा᳚ - व॒ती॒ । \newline
4. धे॒नु॒मती॒ हि हि धे॑नु॒मती॑ धेनु॒मती॒ हि । \newline
5. धे॒नु॒मती॒ इति॑ धेनु - मती᳚ । \newline
6. हि भू॒तम् भू॒तꣳ हि हि भू॒तम् । \newline
7. भू॒तꣳ सू॑यव॒सिनी॑ सूयव॒सिनी॑ भू॒तम् भू॒तꣳ सू॑यव॒सिनी᳚ । \newline
8. सू॒य॒व॒सिनी॒ मन॑वे॒ मन॑वे सूयव॒सिनी॑ सूयव॒सिनी॒ मन॑वे । \newline
9. सू॒य॒व॒सिनी॒ इति॑ सु - य॒व॒सिनी᳚ । \newline
10. मन॑वे यश॒स्ये॑ यश॒स्ये॑ मन॑वे॒ मन॑वे यश॒स्ये᳚ । \newline
11. य॒श॒स्ये॑ इति॑ यश॒स्ये᳚ । \newline
12. व्य॑स्कभ्ना दस्कभ्ना॒द् वि व्य॑स्कभ्नात् । \newline
13. अ॒स्क॒भ्ना॒द् रोद॑सी॒ रोद॑सी अस्कभ्ना दस्कभ्ना॒द् रोद॑सी । \newline
14. रोद॑सी॒ विष्णु॒र् विष्णू॒ रोद॑सी॒ रोद॑सी॒ विष्णुः॑ । \newline
15. रोद॑सी॒ इति॒ रोद॑सी । \newline
16. विष्णु॑ रे॒ते ए॒ते विष्णु॒र् विष्णु॑ रे॒ते । \newline
17. ए॒ते दा॒धार॑ दा॒धारै॒ते ए॒ते दा॒धार॑ । \newline
18. ए॒ते इत्ये॒ते । \newline
19. दा॒धार॑ पृथि॒वीम् पृ॑थि॒वीम् दा॒धार॑ दा॒धार॑ पृथि॒वीम् । \newline
20. पृ॒थि॒वी म॒भितो॒ ऽभितः॑ पृथि॒वीम् पृ॑थि॒वी म॒भितः॑ । \newline
21. अ॒भितो॑ म॒यूखै᳚र् म॒यूखै॑ र॒भितो॒ ऽभितो॑ म॒यूखैः᳚ । \newline
22. म॒यूखै॒रिति॑ म॒यूखैः᳚ । \newline
23. प्राची॒ प्र प्र प्राची॒ प्राची॒ प्र । \newline
24. प्राची॒ इति॒ प्राची᳚ । \newline
25. प्रे त॑ मित॒म् प्र प्रे त᳚म् । \newline
26. इ॒त॒ म॒द्ध्व॒र म॑द्ध्व॒र मि॑त मित मद्ध्व॒रम् । \newline
27. अ॒द्ध्व॒रम् क॒ल्पय॑न्ती क॒ल्पय॑न्ती अद्ध्व॒र म॑द्ध्व॒रम् क॒ल्पय॑न्ती । \newline
28. क॒ल्पय॑न्ती ऊ॒र्द्ध्व मू॒र्द्ध्वम् क॒ल्पय॑न्ती क॒ल्पय॑न्ती ऊ॒र्द्ध्वम् । \newline
29. क॒ल्पय॑न्ती॒ इति॑ क॒ल्पय॑न्ती । \newline
30. ऊ॒र्द्ध्वं ॅय॒ज्ञ्ं ॅय॒ज्ञ् मू॒र्द्ध्व मू॒र्द्ध्वं ॅय॒ज्ञ्म् । \newline
31. य॒ज्ञ्म् न॑यतम् नयतं ॅय॒ज्ञ्ं ॅय॒ज्ञ्म् न॑यतम् । \newline
32. न॒य॒त॒म् मा मा न॑यतम् नयत॒म् मा । \newline
33. मा जी᳚ह्वरतम् जीह्वरत॒म् मा मा जी᳚ह्वरतम् । \newline
34. जी॒ह्व॒र॒त॒ मत्रात्र॑ जीह्वरतम् जीह्वरत॒ मत्र॑ । \newline
35. अत्र॑ रमेथाꣳ रमेथा॒ मत्रात्र॑ रमेथाम् । \newline
36. र॒मे॒थां॒ ॅवर्ष्म॒न्॒. वर्ष्म॑न् रमेथाꣳ रमेथां॒ ॅवर्ष्मन्न्॑ । \newline
37. वर्ष्म॑न् पृथि॒व्याः पृ॑थि॒व्या वर्ष्म॒न्॒. वर्ष्म॑न् पृथि॒व्याः । \newline
38. पृ॒थि॒व्या दि॒वो दि॒वः पृ॑थि॒व्याः पृ॑थि॒व्या दि॒वः । \newline
39. दि॒वो वा॑ वा दि॒वो दि॒वो वा᳚ । \newline
40. वा॒ वि॒ष्णो॒ वि॒ष्णो॒ वा॒ वा॒ वि॒ष्णो॒ । \newline
41. वि॒ष्ण॒वु॒तोत वि॑ष्णो विष्णवु॒त । \newline
42. उ॒त वा॑ वो॒तोत वा᳚ । \newline
43. वा॒ पृ॒थि॒व्याः पृ॑थि॒व्या वा॑ वा पृथि॒व्याः । \newline
44. पृ॒थि॒व्या म॒हो म॒हः पृ॑थि॒व्याः पृ॑थि॒व्या म॒हः । \newline
45. म॒हो वा॑ वा म॒हो म॒हो वा᳚ । \newline
46. वा॒ वि॒ष्णो॒ वि॒ष्णो॒ वा॒ वा॒ वि॒ष्णो॒ । \newline
47. वि॒ष्ण॒वु॒तोत वि॑ष्णो विष्णवु॒त । \newline
48. उ॒त वा॑ वो॒तोत वा᳚ । \newline
49. वा॒ ऽन्तरि॑क्षा द॒न्तरि॑क्षाद् वा वा॒ ऽन्तरि॑क्षात् । \newline
50. अ॒न्तरि॑क्षा॒द्धस्तौ॒ हस्ता॑ व॒न्तरि॑क्षा द॒न्तरि॑क्षा॒द्धस्तौ᳚ । \newline
51. हस्तौ॑ पृणस्व पृणस्व॒ हस्तौ॒ हस्तौ॑ पृणस्व । \newline
52. पृ॒ण॒स्व॒ ब॒हुभि॑र् ब॒हुभिः॑ पृणस्व पृणस्व ब॒हुभिः॑ । \newline
53. ब॒हुभि॑र् वस॒व्यै᳚र् वस॒व्यै᳚र् ब॒हुभि॑र् ब॒हुभि॑र् वस॒व्यैः᳚ । \newline
54. ब॒हुभि॒रिति॑ ब॒हु - भिः॒ । \newline
55. व॒स॒व्यै॑रा व॑स॒व्यै᳚र् वस॒व्यै॑रा । \newline
56. आ प्र प्रा प्र । \newline
57. प्र य॑च्छ यच्छ॒ प्र प्र य॑च्छ । \newline
58. य॒च्छ॒ दक्षि॑णा॒द् दक्षि॑णाद् यच्छ यच्छ॒ दक्षि॑णात् । \newline

\textbf{Ghana Paata } \newline

1. पा॒(ग्म्॒)सु॒र इरा॑वती॒ इरा॑वती पाꣳसु॒रे पा(ग्म्॑)सु॒र इरा॑वती धेनु॒मती॑ धेनु॒मती॒ इरा॑वती पाꣳसु॒रे पा(ग्म्॑)सु॒र इरा॑वती धेनु॒मती᳚ । \newline
2. इरा॑वती धेनु॒मती॑ धेनु॒मती॒ इरा॑वती॒ इरा॑वती धेनु॒मती॒ हि हि धे॑नु॒मती॒ इरा॑वती॒ इरा॑वती धेनु॒मती॒ हि । \newline
3. इरा॑वती॒ इतीरा᳚ - व॒ती॒ । \newline
4. धे॒नु॒मती॒ हि हि धे॑नु॒मती॑ धेनु॒मती॒ हि भू॒तम् भू॒तꣳ हि धे॑नु॒मती॑ धेनु॒मती॒ हि भू॒तम् । \newline
5. धे॒नु॒मती॒ इति॑ धेनु - मती᳚ । \newline
6. हि भू॒तम् भू॒तꣳ हि हि भू॒तꣳ सू॑यव॒सिनी॑ सूयव॒सिनी॑ भू॒तꣳ हि हि भू॒तꣳ सू॑यव॒सिनी᳚ । \newline
7. भू॒तꣳ सू॑यव॒सिनी॑ सूयव॒सिनी॑ भू॒तम् भू॒तꣳ सू॑यव॒सिनी॒ मन॑वे॒ मन॑वे सूयव॒सिनी॑ भू॒तम् भू॒तꣳ सू॑यव॒सिनी॒ मन॑वे । \newline
8. सू॒य॒व॒सिनी॒ मन॑वे॒ मन॑वे सूयव॒सिनी॑ सूयव॒सिनी॒ मन॑वे यश॒स्ये॑ यश॒स्ये॑ मन॑वे सूयव॒सिनी॑ सूयव॒सिनी॒ मन॑वे यश॒स्ये᳚ । \newline
9. सू॒य॒व॒सिनी॒ इति॑ सु - य॒व॒सिनी᳚ । \newline
10. मन॑वे यश॒स्ये॑ यश॒स्ये॑ मन॑वे॒ मन॑वे यश॒स्ये᳚ । \newline
11. य॒श॒स्ये॑ इति॑ यश॒स्ये᳚ । \newline
12. व्य॑स्कभ्ना दस्कभ्ना॒द् वि व्य॑स्कभ्ना॒द् रोद॑सी॒ रोद॑सी अस्कभ्ना॒द् वि व्य॑स्कभ्ना॒द् रोद॑सी । \newline
13. अ॒स्क॒भ्ना॒द् रोद॑सी॒ रोद॑सी अस्कभ्ना दस्कभ्ना॒द् रोद॑सी॒ विष्णु॒र् विष्णू॒ रोद॑सी अस्कभ्ना दस्कभ्ना॒द् रोद॑सी॒ विष्णुः॑ । \newline
14. रोद॑सी॒ विष्णु॒र् विष्णू॒ रोद॑सी॒ रोद॑सी॒ विष्णु॑रे॒ते ए॒ते विष्णू॒ रोद॑सी॒ रोद॑सी॒ विष्णु॑रे॒ते । \newline
15. रोद॑सी॒ इति॒ रोद॑सी । \newline
16. विष्णु॑रे॒ते ए॒ते विष्णु॒र् विष्णु॑रे॒ते दा॒धार॑ दा॒धारै॒ते विष्णु॒र् विष्णु॑रे॒ते दा॒धार॑ । \newline
17. ए॒ते दा॒धार॑ दा॒धारै॒ते ए॒ते दा॒धार॑ पृथि॒वीम् पृ॑थि॒वीम् दा॒धारै॒ते ए॒ते दा॒धार॑ पृथि॒वीम् । \newline
18. ए॒ते इत्ये॒ते । \newline
19. दा॒धार॑ पृथि॒वीम् पृ॑थि॒वीम् दा॒धार॑ दा॒धार॑ पृथि॒वी म॒भितो॒ ऽभितः॑ पृथि॒वीम् दा॒धार॑ दा॒धार॑ पृथि॒वी म॒भितः॑ । \newline
20. पृ॒थि॒वी म॒भितो॒ ऽभितः॑ पृथि॒वीम् पृ॑थि॒वी म॒भितो॑ म॒यूखै᳚र् म॒यूखै॑ र॒भितः॑ पृथि॒वीम् पृ॑थि॒वी म॒भितो॑ म॒यूखैः᳚ । \newline
21. अ॒भितो॑ म॒यूखै᳚र् म॒यूखै॑ र॒भितो॒ ऽभितो॑ म॒यूखैः᳚ । \newline
22. म॒यूखै॒रिति॑ म॒यूखैः᳚ । \newline
23. प्राची॒ प्र प्र प्राची॒ प्राची॒ प्रे त॑ मित॒म् प्र प्राची॒ प्राची॒ प्रे त᳚म् । \newline
24. प्राची॒ इति॒ प्राची᳚ । \newline
25. प्रे त॑ मित॒म् प्र प्रे त॑ मद्ध्व॒र म॑द्ध्व॒र मि॑त॒म् प्र प्रे त॑ मद्ध्व॒रम् । \newline
26. इ॒त॒ म॒द्ध्व॒र म॑द्ध्व॒र मि॑त मित मद्ध्व॒रम् क॒ल्पय॑न्ती क॒ल्पय॑न्ती अद्ध्व॒र मि॑त मित मद्ध्व॒रम् क॒ल्पय॑न्ती । \newline
27. अ॒द्ध्व॒रम् क॒ल्पय॑न्ती क॒ल्पय॑न्ती अद्ध्व॒र म॑द्ध्व॒रम् क॒ल्पय॑न्ती ऊ॒र्द्ध्व मू॒र्द्ध्वम् क॒ल्पय॑न्ती अद्ध्व॒र म॑द्ध्व॒रम् क॒ल्पय॑न्ती ऊ॒र्द्ध्वम् । \newline
28. क॒ल्पय॑न्ती ऊ॒र्द्ध्व मू॒र्द्ध्वम् क॒ल्पय॑न्ती क॒ल्पय॑न्ती ऊ॒र्द्ध्वं ॅय॒ज्ञ्ं ॅय॒ज्ञ् मू॒र्द्ध्वम् क॒ल्पय॑न्ती क॒ल्पय॑न्ती ऊ॒र्द्ध्वं ॅय॒ज्ञ्म् । \newline
29. क॒ल्पय॑न्ती॒ इति॑ क॒ल्पय॑न्ती । \newline
30. ऊ॒र्द्ध्वं ॅय॒ज्ञ्ं ॅय॒ज्ञ् मू॒र्द्ध्व मू॒र्द्ध्वं ॅय॒ज्ञ्न्न॑यतन्नयतं ॅय॒ज्ञ् मू॒र्द्ध्व मू॒र्द्ध्वं ॅय॒ज्ञ्न्न॑यतम् । \newline
31. य॒ज्ञ्न्न॑यतन्नयतं ॅय॒ज्ञ्ं ॅय॒ज्ञ्न्न॑यत॒म् मा मा न॑यतं ॅय॒ज्ञ्ं ॅय॒ज्ञ्न्न॑यत॒म् मा । \newline
32. न॒य॒त॒म् मा मा न॑यतन्नयत॒म् मा जी᳚ह्वरतम् जीह्वरत॒म् मा न॑यतन्नयत॒म् मा जी᳚ह्वरतम् । \newline
33. मा जी᳚ह्वरतम् जीह्वरत॒म् मा मा जी᳚ह्वरत॒ मत्रात्र॑ जीह्वरत॒म् मा मा जी᳚ह्वरत॒ मत्र॑ । \newline
34. जी॒ह्व॒र॒त॒ मत्रात्र॑ जीह्वरतम् जीह्वरत॒ मत्र॑ रमेथाꣳ रमेथा॒ मत्र॑ जीह्वरतम् जीह्वरत॒ मत्र॑ रमेथाम् । \newline
35. अत्र॑ रमेथाꣳ रमेथा॒ मत्रात्र॑ रमेथां॒ ॅवर्ष्म॒न्॒. वर्ष्म॑न् रमेथा॒ मत्रात्र॑ रमेथां॒ ॅवर्ष्मन्न्॑ । \newline
36. र॒मे॒थां॒ ॅवर्ष्म॒न्॒. वर्ष्म॑न् रमेथाꣳ रमेथां॒ ॅवर्ष्म॑न् पृथि॒व्याः पृ॑थि॒व्या वर्ष्म॑न् रमेथाꣳ रमेथां॒ ॅवर्ष्म॑न् पृथि॒व्याः । \newline
37. वर्ष्म॑न् पृथि॒व्याः पृ॑थि॒व्या वर्ष्म॒न्॒. वर्ष्म॑न् पृथि॒व्या दि॒वो दि॒वः पृ॑थि॒व्या वर्ष्म॒न्॒. वर्ष्म॑न् पृथि॒व्या दि॒वः । \newline
38. पृ॒थि॒व्या दि॒वो दि॒वः पृ॑थि॒व्याः पृ॑थि॒व्या दि॒वो वा॑ वा दि॒वः पृ॑थि॒व्याः पृ॑थि॒व्या दि॒वो वा᳚ । \newline
39. दि॒वो वा॑ वा दि॒वो दि॒वो वा॑ विष्णो विष्णो वा दि॒वो दि॒वो वा॑ विष्णो । \newline
40. वा॒ वि॒ष्णो॒ वि॒ष्णो॒ वा॒ वा॒ वि॒ष्ण॒वु॒तोत वि॑ष्णो वा वा विष्णवु॒त । \newline
41. वि॒ष्ण॒वु॒तोत वि॑ष्णो विष्णवु॒त वा॑ वो॒त वि॑ष्णो विष्णवु॒त वा᳚ । \newline
42. उ॒त वा॑ वो॒तोत वा॑ पृथि॒व्याः पृ॑थि॒व्या वो॒तोत वा॑ पृथि॒व्याः । \newline
43. वा॒ पृ॒थि॒व्याः पृ॑थि॒व्या वा॑ वा पृथि॒व्या म॒हो म॒हः पृ॑थि॒व्या वा॑ वा पृथि॒व्या म॒हः । \newline
44. पृ॒थि॒व्या म॒हो म॒हः पृ॑थि॒व्याः पृ॑थि॒व्या म॒हो वा॑ वा म॒हः पृ॑थि॒व्याः पृ॑थि॒व्या म॒हो वा᳚ । \newline
45. म॒हो वा॑ वा म॒हो म॒हो वा॑ विष्णो विष्णो वा म॒हो म॒हो वा॑ विष्णो । \newline
46. वा॒ वि॒ष्णो॒ वि॒ष्णो॒ वा॒ वा॒ वि॒ष्ण॒वु॒तोत वि॑ष्णो वा वा विष्णवु॒त । \newline
47. वि॒ष्ण॒वु॒तोत वि॑ष्णो विष्णवु॒त वा॑ वो॒त वि॑ष्णो विष्णवु॒त वा᳚ । \newline
48. उ॒त वा॑ वो॒तोत वा॒ ऽन्तरि॑क्षा द॒न्तरि॑क्षाद् वो॒तोत वा॒ ऽन्तरि॑क्षात् । \newline
49. वा॒ ऽन्तरि॑क्षा द॒न्तरि॑क्षाद् वा वा॒ ऽन्तरि॑क्षा॒द्धस्तौ॒ हस्ता॑ व॒न्तरि॑क्षाद् वा वा॒ ऽन्तरि॑क्षा॒द्धस्तौ᳚ । \newline
50. अ॒न्तरि॑क्षा॒द्धस्तौ॒ हस्ता॑ व॒न्तरि॑क्षा द॒न्तरि॑क्षा॒द्धस्तौ॑ पृणस्व पृणस्व॒ हस्ता॑ व॒न्तरि॑क्षा द॒न्तरि॑क्षा॒द्धस्तौ॑ पृणस्व । \newline
51. हस्तौ॑ पृणस्व पृणस्व॒ हस्तौ॒ हस्तौ॑ पृणस्व ब॒हुभि॑र् ब॒हुभिः॑ पृणस्व॒ हस्तौ॒ हस्तौ॑ पृणस्व ब॒हुभिः॑ । \newline
52. पृ॒ण॒स्व॒ ब॒हुभि॑र् ब॒हुभिः॑ पृणस्व पृणस्व ब॒हुभि॑र् वस॒व्यै᳚र् वस॒व्यै᳚र् ब॒हुभिः॑ पृणस्व पृणस्व ब॒हुभि॑र् वस॒व्यैः᳚ । \newline
53. ब॒हुभि॑र् वस॒व्यै᳚र् वस॒व्यै᳚र् ब॒हुभि॑र् ब॒हुभि॑र् वस॒व्यै॑रा व॑स॒व्यै᳚र् ब॒हुभि॑र् ब॒हुभि॑र् वस॒व्यै॑रा । \newline
54. ब॒हुभि॒रिति॑ ब॒हु - भिः॒ । \newline
55. व॒स॒व्यै॑रा व॑स॒व्यै᳚र् वस॒व्यै॑रा प्र प्रा व॑स॒व्यै᳚र् वस॒व्यै॑रा प्र । \newline
56. आ प्र प्रा प्र य॑च्छ यच्छ॒ प्रा प्र य॑च्छ । \newline
57. प्र य॑च्छ यच्छ॒ प्र प्र य॑च्छ॒ दक्षि॑णा॒द् दक्षि॑णाद् यच्छ॒ प्र प्र य॑च्छ॒ दक्षि॑णात् । \newline
58. य॒च्छ॒ दक्षि॑णा॒द् दक्षि॑णाद् यच्छ यच्छ॒ दक्षि॑णा॒दा दक्षि॑णाद् यच्छ यच्छ॒ दक्षि॑णा॒दा । \newline
\pagebreak
\markright{ TS 1.2.13.3  \hfill https://www.vedavms.in \hfill}

\section{ TS 1.2.13.3 }

\textbf{TS 1.2.13.3 } \newline
\textbf{Samhita Paata} \newline

दक्षि॑णा॒दोत स॒व्यात् । विष्णो॒र्नुकं॑ ॅवी॒र्या॑णि॒ प्र वो॑चं॒ ॅयः पार्त्थि॑वानि विम॒मे रजाꣳ॑सि॒ यो अस्क॑भाय॒दुत्त॑रꣳ स॒धस्थं॑ ॅविचक्रमा॒ण स्त्रे॒धोरु॑गा॒यो विष्णो॑ र॒राट॑मसि॒ विष्णोः᳚ पृ॒ष्ठम॑सि॒ विष्णोः॒ श्ञप्त्रे᳚ स्थो॒ विष्णोः॒ स्यूर॑सि॒ विष्णो᳚र् ध्रु॒वम॑सि वैष्ण॒वम॑सि॒ विष्ण॑वे त्वा ॥ \newline

\textbf{Pada Paata} \newline

दक्षि॑णात् । एति॑ । उ॒त । स॒व्यात् ॥ विष्णोः᳚ । नुक᳚म् । वी॒र्या॑णि । प्रेति॑ । वो॒च॒म् । यः । पार्थि॑वानि । वि॒म॒म इति॑ वि - म॒मे । रजाꣳ॑सि । यः । अस्क॑भायत् । उत्त॑र॒मित्युत् - त॒र॒म् । स॒धस्थ॒मिति॑ स॒ध - स्थ॒म् । वि॒च॒क्र॒मा॒ण इति॑ वि - च॒क्र॒मा॒णः । त्रे॒धा । उ॒रु॒गा॒य इत्यु॑रु - गा॒यः । विष्णोः᳚ । र॒राट᳚म् । अ॒सि॒ । विष्णोः᳚ । पृ॒ष्ठम् । अ॒सि॒ । विष्णोः᳚ । श्नप्त्रे॒ इति॑ । स्थः॒ । विष्णोः᳚ । स्यूः । अ॒सि॒ । विष्णोः᳚ । ध्रु॒वम् । अ॒सि॒ । वै॒ष्ण॒वम् । अ॒सि॒ । विष्ण॑वे । त्वा॒ ॥  \newline


\textbf{Krama Paata} \newline

दक्षि॑णा॒दा । ओत । उ॒त स॒व्यात् । स॒व्यादिति॑ स॒व्यात् ॥ विष्णो॒र् नुक᳚म् । नुकं॑ ॅवी॒र्या॑णि । वी॒र्या॑णि॒ प्र । प्र वो॑चम् । वो॒चं॒ ॅयः । यः पार्त्थि॑वानि । पार्त्थि॑वानि विम॒मे । वि॒म॒मे रजाꣳ॑सि । वि॒म॒म इति॑ वि - म॒मे । रजाꣳ॑सि॒ यः । यो अस्क॑भायत् । अस्क॑भाय॒दुत्त॑रम् । उत्त॑रꣳ स॒धस्थ᳚म् । उत्त॑र॒मित्युत् - त॒र॒म् । स॒धस्थं॑ ॅविचक्रमा॒णः । स॒धस्थ॒मिति॑ स॒ध - स्थ॒म् । वि॒च॒क्र॒मा॒णस्त्रे॒धा । वि॒च॒क्र॒मा॒ण इति॑ वि - च॒क्र॒मा॒णः । त्रे॒धोरु॑गा॒यः । उ॒रु॒गा॒यो विष्णोः᳚ । उ॒रु॒गा॒य इत्यु॑रु - गा॒यः । विष्णो॑ र॒राट᳚म् । र॒राट॑मसि । अ॒सि॒ विष्णोः᳚ । विष्णोः᳚ पृ॒ष्ठम् । पृ॒ष्ठम॑सि । अ॒सि॒ विष्णोः᳚ । विष्णोः॒ श्ञप्त्रे᳚ । श्ञप्त्रे᳚ स्थः । श्ञप्त्रे॒ इति॒ श्ञप्त्रे᳚ । स्थो॒ विष्णोः᳚ । विष्णोः॒ स्यूः । स्यूर॑सि । अ॒सि॒ विष्णोः᳚ । विष्णो᳚ ध्रु॒वम् । ध्रु॒वम॑सि । अ॒सि॒ वै॒ष्ण॒वम् । वै॒ष्ण॒वम॑सि । अ॒सि॒ विष्ण॑वे । विष्ण॑वे त्वा । त्वेति॑ त्वा । \newline

\textbf{Jatai Paata} \newline

1. दक्षि॑णा॒दा दक्षि॑णा॒द् दक्षि॑णा॒दा । \newline
2. ओतोतोत । \newline
3. उ॒त स॒व्याथ् स॒व्यादु॒तोत स॒व्यात् । \newline
4. स॒व्यादिति॑ स॒व्यात् । \newline
5. विष्णो॒र् नुक॒म् नुकं॒ ॅविष्णो॒र् विष्णो॒र् नुक᳚म् । \newline
6. नुकं॑ ॅवी॒र्या॑णि वी॒र्या॑णि॒ नुक॒म् नुकं॑ ॅवी॒र्या॑णि । \newline
7. वी॒र्या॑णि॒ प्र प्र वी॒र्या॑णि वी॒र्या॑णि॒ प्र । \newline
8. प्र वो॑चं ॅवोच॒म् प्र प्र वो॑चम् । \newline
9. वो॒चं॒ ॅयो यो वो॑चं ॅवोचं॒ ॅयः । \newline
10. यः पार्थि॑वानि॒ पार्थि॑वानि॒ यो यः पार्थि॑वानि । \newline
11. पार्थि॑वानि विम॒मे वि॑म॒मे पार्थि॑वानि॒ पार्थि॑वानि विम॒मे । \newline
12. वि॒म॒मे रजा(ग्म्॑)सि॒ रजा(ग्म्॑)सि विम॒मे वि॑म॒मे रजा(ग्म्॑)सि । \newline
13. वि॒म॒म इति॑ वि - म॒मे । \newline
14. रजा(ग्म्॑)सि॒ यो यो रजा(ग्म्॑)सि॒ रजा(ग्म्॑)सि॒ यः । \newline
15. यो अस्क॑भाय॒ दस्क॑भाय॒द् यो यो अस्क॑भायत् । \newline
16. अस्क॑भाय॒ दुत्त॑र॒ मुत्त॑र॒ मस्क॑भाय॒ दस्क॑भाय॒ दुत्त॑रम् । \newline
17. उत्त॑रꣳ स॒धस्थ(ग्म्॑) स॒धस्थ॒ मुत्त॑र॒ मुत्त॑रꣳ स॒धस्थ᳚म् । \newline
18. उत्त॑र॒मित्युत् - त॒र॒म् । \newline
19. स॒धस्थं॑ ॅविचक्रमा॒णो वि॑चक्रमा॒णः स॒धस्थ(ग्म्॑) स॒धस्थं॑ ॅविचक्रमा॒णः । \newline
20. स॒धस्थ॒मिति॑ स॒ध - स्थ॒म् । \newline
21. वि॒च॒क्र॒मा॒ण स्त्रे॒धा त्रे॒धा वि॑चक्रमा॒णो वि॑चक्रमा॒ण स्त्रे॒धा । \newline
22. वि॒च॒क्र॒मा॒ण इति॑ वि - च॒क्र॒मा॒णः । \newline
23. त्रे॒धो रु॑गा॒य उ॑रुगा॒य स्त्रे॒धा त्रे॒धो रु॑गा॒यः । \newline
24. उ॒रु॒गा॒यो विष्णो॒र् विष्णो॑ रुरुगा॒य उ॑रुगा॒यो विष्णोः᳚ । \newline
25. उ॒रु॒गा॒य इत्यु॑रु - गा॒यः । \newline
26. विष्णो॑ र॒राट(ग्म्॑) र॒राटं॒ ॅविष्णो॒र् विष्णो॑ र॒राट᳚म् । \newline
27. र॒राट॑ मस्यसि र॒राट(ग्म्॑) र॒राट॑ मसि । \newline
28. अ॒सि॒ विष्णो॒र् विष्णो॑ रस्यसि॒ विष्णोः᳚ । \newline
29. विष्णोः᳚ पृ॒ष्ठम् पृ॒ष्ठं ॅविष्णो॒र् विष्णोः᳚ पृ॒ष्ठम् । \newline
30. पृ॒ष्ठ म॑स्यसि पृ॒ष्ठम् पृ॒ष्ठ म॑सि । \newline
31. अ॒सि॒ विष्णो॒र् विष्णो॑ रस्यसि॒ विष्णोः᳚ । \newline
32. विष्णोः॒ श्ञप्त्रे॒ श्ञप्त्रे॒ विष्णो॒र् विष्णोः॒ श्ञप्त्रे᳚ । \newline
33. श्ञप्त्रे᳚ स्थः स्थः॒ श्ञप्त्रे॒ श्ञप्त्रे᳚ स्थः । \newline
34. श्ञप्त्रे॒ इति॒ श्ञप्त्रे᳚ । \newline
35. स्थो॒ विष्णो॒र् विष्णोः᳚ स्थः स्थो॒ विष्णोः᳚ । \newline
36. विष्णोः॒ स्यूः स्यूर् विष्णो॒र् विष्णोः॒ स्यूः । \newline
37. स्यू र॑स्यसि॒ स्यूः स्यू र॑सि । \newline
38. अ॒सि॒ विष्णो॒र् विष्णो॑ रस्यसि॒ विष्णोः᳚ । \newline
39. विष्णो᳚र् ध्रु॒वम् ध्रु॒वं ॅविष्णो॒र् विष्णो᳚र् ध्रु॒वम् । \newline
40. ध्रु॒व म॑स्यसि ध्रु॒वम् ध्रु॒व म॑सि । \newline
41. अ॒सि॒ वै॒ष्ण॒वं ॅवै᳚ष्ण॒व म॑स्यसि वैष्ण॒वम् । \newline
42. वै॒ष्ण॒व म॑स्यसि वैष्ण॒वं ॅवै᳚ष्ण॒व म॑सि । \newline
43. अ॒सि॒ विष्ण॑वे॒ विष्ण॑वे ऽस्यसि॒ विष्ण॑वे । \newline
44. विष्ण॑वे त्वा त्वा॒ विष्ण॑वे॒ विष्ण॑वे त्वा । \newline
45. त्वेति॑ त्वा । \newline

\textbf{Ghana Paata } \newline

1. दक्षि॑णा॒दा दक्षि॑णा॒द् दक्षि॑णा॒दोतोता दक्षि॑णा॒द् दक्षि॑णा॒दोत । \newline
2. ओतोतोत स॒व्याथ् स॒व्यादु॒तोत स॒व्यात् । \newline
3. उ॒त स॒व्याथ् स॒व्यादु॒तोत स॒व्यात् । \newline
4. स॒व्यादिति॑ स॒व्यात् । \newline
5. विष्णो॒र् नुक॒न्नुकं॒ ॅविष्णो॒र् विष्णो॒र् नुकं॑ ॅवी॒र्या॑णि वी॒र्या॑णि॒ नुकं॒ ॅविष्णो॒र् विष्णो॒र् नुकं॑ ॅवी॒र्या॑णि । \newline
6. नुकं॑ ॅवी॒र्या॑णि वी॒र्या॑णि॒ नुक॒न्नुकं॑ ॅवी॒र्या॑णि॒ प्र प्र वी॒र्या॑णि॒ नुक॒न्नुकं॑ ॅवी॒र्या॑णि॒ प्र । \newline
7. वी॒र्या॑णि॒ प्र प्र वी॒र्या॑णि वी॒र्या॑णि॒ प्र वो॑चं ॅवोच॒म् प्र वी॒र्या॑णि वी॒र्या॑णि॒ प्र वो॑चम् । \newline
8. प्र वो॑चं ॅवोच॒म् प्र प्र वो॑चं॒ ॅयो यो वो॑च॒म् प्र प्र वो॑चं॒ ॅयः । \newline
9. वो॒चं॒ ॅयो यो वो॑चं ॅवोचं॒ ॅयः पार्थि॑वानि॒ पार्थि॑वानि॒ यो वो॑चं ॅवोचं॒ ॅयः पार्थि॑वानि । \newline
10. यः पार्थि॑वानि॒ पार्थि॑वानि॒ यो यः पार्थि॑वानि विम॒मे वि॑म॒मे पार्थि॑वानि॒ यो यः पार्थि॑वानि विम॒मे । \newline
11. पार्थि॑वानि विम॒मे वि॑म॒मे पार्थि॑वानि॒ पार्थि॑वानि विम॒मे रजा(ग्म्॑)सि॒ रजा(ग्म्॑)सि विम॒मे पार्थि॑वानि॒ पार्थि॑वानि विम॒मे रजा(ग्म्॑)सि । \newline
12. वि॒म॒मे रजा(ग्म्॑)सि॒ रजा(ग्म्॑)सि विम॒मे वि॑म॒मे रजा(ग्म्॑)सि॒ यो यो रजा(ग्म्॑)सि विम॒मे वि॑म॒मे रजा(ग्म्॑)सि॒ यः । \newline
13. वि॒म॒म इति॑ वि - म॒मे । \newline
14. रजा(ग्म्॑)सि॒ यो यो रजा(ग्म्॑)सि॒ रजा(ग्म्॑)सि॒ यो अस्क॑भाय॒ दस्क॑भाय॒द् यो रजा(ग्म्॑)सि॒ रजा(ग्म्॑)सि॒ यो अस्क॑भायत् । \newline
15. यो अस्क॑भाय॒ दस्क॑भाय॒द् यो यो अस्क॑भाय॒ दुत्त॑र॒ मुत्त॑र॒ मस्क॑भाय॒द् यो यो अस्क॑भाय॒ दुत्त॑रम् । \newline
16. अस्क॑भाय॒ दुत्त॑र॒ मुत्त॑र॒ मस्क॑भाय॒ दस्क॑भाय॒ दुत्त॑रꣳ स॒धस्थ(ग्म्॑) स॒धस्थ॒ मुत्त॑र॒ मस्क॑भाय॒ दस्क॑भाय॒ दुत्त॑रꣳ स॒धस्थ᳚म् । \newline
17. उत्त॑रꣳ स॒धस्थ(ग्म्॑) स॒धस्थ॒ मुत्त॑र॒ मुत्त॑रꣳ स॒धस्थं॑ ॅविचक्रमा॒णो वि॑चक्रमा॒णः स॒धस्थ॒ मुत्त॑र॒ मुत्त॑रꣳ स॒धस्थं॑ ॅविचक्रमा॒णः । \newline
18. उत्त॑र॒मित्युत् - त॒र॒म् । \newline
19. स॒धस्थं॑ ॅविचक्रमा॒णो वि॑चक्रमा॒णः स॒धस्थ(ग्म्॑) स॒धस्थं॑ ॅविचक्रमा॒ण स्त्रे॒धा त्रे॒धा वि॑चक्रमा॒णः स॒धस्थ(ग्म्॑) स॒धस्थं॑ ॅविचक्रमा॒ण स्त्रे॒धा । \newline
20. स॒धस्थ॒मिति॑ स॒ध - स्थ॒म् । \newline
21. वि॒च॒क्र॒मा॒ण स्त्रे॒धा त्रे॒धा वि॑चक्रमा॒णो वि॑चक्रमा॒ण स्त्रे॒धोरु॑गा॒य उ॑रुगा॒यस्त्रे॒धा वि॑चक्रमा॒णो वि॑चक्रमा॒ण स्त्रे॒धोरु॑गा॒यः । \newline
22. वि॒च॒क्र॒मा॒ण इति॑ वि - च॒क्र॒मा॒णः । \newline
23. त्रे॒धोरु॑गा॒य उ॑रुगा॒यस्त्रे॒धा त्रे॒धोरु॑गा॒यो विष्णो॒र् विष्णो॑ रुरुगा॒य स्त्रे॒धा त्रे॒धोरु॑गा॒यो विष्णोः᳚ । \newline
24. उ॒रु॒गा॒यो विष्णो॒र् विष्णो॑ रुरुगा॒य उ॑रुगा॒यो विष्णो॑ र॒राट(ग्म्॑) र॒राटं॒ ॅविष्णो॑ रुरुगा॒य उ॑रुगा॒यो विष्णो॑ र॒राट᳚म् । \newline
25. उ॒रु॒गा॒य इत्यु॑रु - गा॒यः । \newline
26. विष्णो॑ र॒राट(ग्म्॑) र॒राटं॒ ॅविष्णो॒र् विष्णो॑ र॒राट॑ मस्यसि र॒राटं॒ ॅविष्णो॒र् विष्णो॑ र॒राट॑ मसि । \newline
27. र॒राट॑ मस्यसि र॒राट(ग्म्॑) र॒राट॑ मसि॒ विष्णो॒र् विष्णो॑रसि र॒राट(ग्म्॑) र॒राट॑ मसि॒ विष्णोः᳚ । \newline
28. अ॒सि॒ विष्णो॒र् विष्णो॑रस्यसि॒ विष्णोः᳚ पृ॒ष्ठम् पृ॒ष्ठं ॅविष्णो॑रस्यसि॒ विष्णोः᳚ पृ॒ष्ठम् । \newline
29. विष्णोः᳚ पृ॒ष्ठम् पृ॒ष्ठं ॅविष्णो॒र् विष्णोः᳚ पृ॒ष्ठ म॑स्यसि पृ॒ष्ठं ॅविष्णो॒र् विष्णोः᳚ पृ॒ष्ठ म॑सि । \newline
30. पृ॒ष्ठ म॑स्यसि पृ॒ष्ठम् पृ॒ष्ठ म॑सि॒ विष्णो॒र् विष्णो॑ रसि पृ॒ष्ठम् पृ॒ष्ठ म॑सि॒ विष्णोः᳚ । \newline
31. अ॒सि॒ विष्णो॒र् विष्णो॑ रस्यसि॒ विष्णोः॒ श्ञप्त्रे॒ श्ञप्त्रे॒ विष्णो॑ रस्यसि॒ विष्णोः॒ श्ञप्त्रे᳚ । \newline
32. विष्णोः॒ श्ञप्त्रे॒ श्ञप्त्रे॒ विष्णो॒र् विष्णोः॒ श्ञप्त्रे᳚ स्थः स्थः॒ श्ञप्त्रे॒ विष्णो॒र् विष्णोः॒ श्ञप्त्रे᳚ स्थः । \newline
33. श्ञप्त्रे᳚ स्थः स्थः॒ श्ञप्त्रे॒ श्ञप्त्रे᳚ स्थो॒ विष्णो॒र् विष्णोः᳚ स्थः॒ श्ञप्त्रे॒ श्ञप्त्रे᳚ स्थो॒ विष्णोः᳚ । \newline
34. श्ञप्त्रे॒ इति॒ श्ञप्त्रे᳚ । \newline
35. स्थो॒ विष्णो॒र् विष्णोः᳚ स्थः स्थो॒ विष्णोः॒ स्यूः स्यूर् विष्णोः᳚ स्थः स्थो॒ विष्णोः॒ स्यूः । \newline
36. विष्णोः॒ स्यूः स्यूर् विष्णो॒र् विष्णोः॒ स्यूर॑स्यसि॒ स्यूर् विष्णो॒र् विष्णोः॒ स्यूर॑सि । \newline
37. स्यूर॑ स्यसि॒ स्यूः स्यूर॑सि॒ विष्णो॒र् विष्णो॑ रसि॒ स्यूः स्यूर॑सि॒ विष्णोः᳚ । \newline
38. अ॒सि॒ विष्णो॒र् विष्णो॑रस्यसि॒ विष्णो᳚र् ध्रु॒वम् ध्रु॒वं ॅविष्णो॑रस्यसि॒ विष्णो᳚र् ध्रु॒वम् । \newline
39. विष्णो᳚र् ध्रु॒वम् ध्रु॒वं ॅविष्णो॒र् विष्णो᳚र् ध्रु॒व म॑स्यसि ध्रु॒वं ॅविष्णो॒र् विष्णो᳚र् ध्रु॒व म॑सि । \newline
40. ध्रु॒व म॑स्यसि ध्रु॒वम् ध्रु॒व म॑सि वैष्ण॒वं ॅवै᳚ष्ण॒व म॑सि ध्रु॒वम् ध्रु॒व म॑सि वैष्ण॒वम् । \newline
41. अ॒सि॒ वै॒ष्ण॒वं ॅवै᳚ष्ण॒व म॑स्यसि वैष्ण॒व म॑स्यसि वैष्ण॒व म॑स्यसि वैष्ण॒व म॑सि । \newline
42. वै॒ष्ण॒व म॑स्यसि वैष्ण॒वं ॅवै᳚ष्ण॒व म॑सि॒ विष्ण॑वे॒ विष्ण॑वे ऽसि वैष्ण॒वं ॅवै᳚ष्ण॒व म॑सि॒ विष्ण॑वे । \newline
43. अ॒सि॒ विष्ण॑वे॒ विष्ण॑वे ऽस्यसि॒ विष्ण॑वे त्वा त्वा॒ विष्ण॑वे ऽस्यसि॒ विष्ण॑वे त्वा । \newline
44. विष्ण॑वे त्वा त्वा॒ विष्ण॑वे॒ विष्ण॑वे त्वा । \newline
45. त्वेति॑ त्वा । \newline
\pagebreak
\markright{ TS 1.2.14.1  \hfill https://www.vedavms.in \hfill}

\section{ TS 1.2.14.1 }

\textbf{TS 1.2.14.1 } \newline
\textbf{Samhita Paata} \newline

कृ॒णु॒ष्व पाजः॒ प्रसि॑ति॒न्न पृ॒थ्वीं ॅया॒हि राजे॒वाम॑वाꣳ॒॒ इभे॑न । तृ॒ष्वीमनु॒ प्रसि॑तिं-द्रूणा॒नो-ऽस्ता॑ऽसि॒ विद्ध्य॑ र॒क्षस॒ स्तपि॑ष्ठैः ॥ तव॑ भ्र॒मास॑ आशु॒या प॑न्त॒त्यनु॑ स्पृश-धृष॒ता शोशु॑चानः । तपूꣳ॑ष्यग्ने जु॒ह्वा॑ पत॒ङ्गानस॑न्दितो॒ विसृ॑ज॒ विष्व॑गु॒ल्काः ॥ प्रति॒स्पशो॒ विसृ॑ज॒-तूर्णि॑तमो॒ भवा॑ पा॒युर्वि॒शो अ॒स्या अद॑ब्धः । यो नो॑ दू॒रे अ॒घशꣳ॑सो॒ - [ ] \newline

\textbf{Pada Paata} \newline

कृ॒णु॒ष्व । पाजः॑ । प्रसि॑ति॒मिति॒ प्र - सि॒ति॒म् । न । पृ॒थ्वीम् । या॒हि । राजा᳚ । इ॒व॒ । अम॑वा॒नित्यम॑ - वा॒न् । इभे॑न ॥ तृ॒ष्वीम् । अन्विति॑ । प्रसि॑ति॒मिति॒ प्र - सि॒ति॒म् । द्रू॒णा॒नः । अस्ता᳚ । अ॒सि॒ । विद्ध्य॑ । र॒क्षसः॑ । तपि॑ष्ठैः ॥ तव॑ । भ्र॒मासः॑ । आ॒शु॒या । प॒त॒न्ति॒ । अन्विति॑ । स्पृ॒श॒ । धृ॒ष॒ता । शोशु॑चानः ॥ तपूꣳ॑षि । अ॒ग्ने॒ । जु॒ह्वा᳚ । प॒त॒ङ्गान् । अस॑न्दित॒ इत्यसं᳚ - दि॒तः॒ । वीति॑ । सृ॒ज॒ । विष्व॑क् । उ॒ल्काः ॥ प्रतीति॑ । स्पशः॑ । वीति॑ । सृ॒ज॒ । तूर्णि॑तम॒ इति॒ तूर्णि॑ - त॒मः॒ । भव॑ । पा॒युः । वि॒शः । अ॒स्याः । अद॑ब्दः ॥ यः । नः॒ । दू॒रे । अ॒घशꣳ॑स॒ इत्य॒घ - शꣳ॒॒सः॒ ।  \newline


\textbf{Krama Paata} \newline

कृ॒णु॒ष्व पाजः॑ । पाजः॒ प्रसि॑तिम् । प्रसि॑ति॒म् न । प्रसि॑ति॒मिति॒ प्र - सि॒ति॒म् । न पृ॒थ्वीम् । पृ॒थ्वीं ॅया॒हि । या॒हि राजा᳚ । राजे॑व । इ॒वाम॑वान् । अम॑वाꣳ॒॒ इभे॑न । अम॑वा॒नित्यम॑ - वा॒न्॒ । इभे॒नेतीभे॑न ॥ तृ॒ष्वीमनु॑ । अनु॒ प्रसि॑तिम् । प्रसि॑तिम् द्रूणा॒नः । प्रसि॑ति॒मिति॒ प्र - सि॒ति॒म् । द्रू॒णा॒नोऽस्ता᳚ । अस्ता॑ऽसि । अ॒सि॒ विद्ध्य॑ । विद्ध्य॑ र॒क्षसः॑ । र॒क्षस॒ स्तपि॑ष्ठैः । तपि॑ष्ठै॒रिति॒ तपि॑ष्ठैः ॥ तव॑ भ्र॒मासः॑ । भ्र॒मास॑ आशु॒या । आ॒शु॒या प॑तन्ति । प॒त॒न्त्यनु॑ । अनु॑ स्पृश । स्पृ॒श॒ धृ॒ष॒ता । धृ॒ष॒ता शोशु॑चानः । शोशु॑चान॒ इति॒ शोशु॑चानः ॥ तपूꣳ॑ष्यग्ने । अ॒ग्ने॒ जु॒ह्वा᳚ । जु॒ह्वा॑ पत॒ङ्गान् । प॒त॒ङ्गानस॑न्दितः । अस॑न्दितो॒ वि । अस॑न्दित॒ इत्यसं᳚ - दि॒तः॒ । वि सृ॑ज । सृ॒ज॒ विष्व॑क् । विष्व॑गु॒ल्काः । उ॒ल्का इत्यु॒ल्काः ॥ प्रति॒ स्पशः॑ । स्पशो॒ वि । वि सृ॑ज । सृ॒ज॒ तूर्णि॑तमः । तूर्णि॑तमो॒ भव॑ । तूर्णि॑तम॒ इति॒ तूर्णि॑ - त॒मः॒ । भवा॑ पा॒युः । पा॒युर् वि॒शः । वि॒शो अ॒स्याः । अ॒स्या अद॑ब्धः । अद॑ब्ध॒ इत्यद॑ब्धः ॥ यो नः॑ । नो॒ दू॒रे । दू॒रे अ॒घशꣳ॑सः । अ॒घशꣳ॑सो॒ यः । अ॒घशꣳ॑स॒ इत्य॒घ - शꣳ॒॒सः॒ \newline

\textbf{Jatai Paata} \newline

1. कृ॒णु॒ष्व पाजः॒ पाजः॑ कृणु॒ष्व कृ॑णु॒ष्व पाजः॑ । \newline
2. पाजः॒ प्रसि॑ति॒म् प्रसि॑ति॒म् पाजः॒ पाजः॒ प्रसि॑तिम् । \newline
3. प्रसि॑ति॒म् न न प्रसि॑ति॒म् प्रसि॑ति॒म् न । \newline
4. प्रसि॑ति॒मिति॒ प्र - सि॒ति॒म् । \newline
5. न पृ॒थ्वीम् पृ॒थ्वीम् न न पृ॒थ्वीम् । \newline
6. पृ॒थ्वीं ॅया॒हि या॒हि पृ॒थ्वीम् पृ॒थ्वीं ॅया॒हि । \newline
7. या॒हि राजा॒ राजा॑ या॒हि या॒हि राजा᳚ । \newline
8. राजे॑वे व॒ राजा॒ राजे॑व । \newline
9. इ॒वाम॑वा॒(ग्म्॒) अम॑वाꣳ इवे॒ वाम॑वान् । \newline
10. अम॑वा॒(ग्म्॒) इभे॒ने भे॒नाम॑वा॒(ग्म्॒) अम॑वा॒(ग्म्॒) इभे॑न । \newline
11. अम॑वा॒नित्यम॑ - वा॒न् । \newline
12. इभे॒नेतीभे॑न । \newline
13. तृ॒ष्वी मन्वनु॑ तृ॒ष्वीम् तृ॒ष्वी मनु॑ । \newline
14. अनु॒ प्रसि॑ति॒म् प्रसि॑ति॒ मन्वनु॒ प्रसि॑तिम् । \newline
15. प्रसि॑तिम् द्रूणा॒नो द्रू॑णा॒नः प्रसि॑ति॒म् प्रसि॑तिम् द्रूणा॒नः । \newline
16. प्रसि॑ति॒मिति॒ प्र - सि॒ति॒म् । \newline
17. द्रू॒णा॒नो ऽस्ता ऽस्ता᳚ द्रूणा॒नो द्रू॑णा॒नो ऽस्ता᳚ । \newline
18. अस्ता᳚ ऽस्य॒ स्यस्ता ऽस्ता॑ ऽसि । \newline
19. अ॒सि॒ विद्ध्य॒ विद्ध्या᳚ स्यसि॒ विद्ध्य॑ । \newline
20. विद्ध्य॑ र॒क्षसो॑ र॒क्षसो॒ विद्ध्य॒ विद्ध्य॑ र॒क्षसः॑ । \newline
21. र॒क्षस॒ स्तपि॑ष्ठै॒ स्तपि॑ष्ठै र॒क्षसो॑ र॒क्षस॒ स्तपि॑ष्ठैः । \newline
22. तपि॑ष्ठै॒रिति॒ तपि॑ष्ठैः । \newline
23. तव॑ भ्र॒मासो᳚ भ्र॒मास॒ स्तव॒ तव॑ भ्र॒मासः॑ । \newline
24. भ्र॒मास॑ आशु॒या ऽऽशु॒या भ्र॒मासो᳚ भ्र॒मास॑ आशु॒या । \newline
25. आ॒शु॒या प॑तन्ति पतन्त्याशु॒या ऽऽशु॒या प॑तन्ति । \newline
26. प॒त॒न्त्यन्वनु॑ पतन्ति पत॒न्त्यनु॑ । \newline
27. अनु॑ स्पृश स्पृ॒शान्वनु॑ स्पृश । \newline
28. स्पृ॒श॒ धृ॒ष॒ता धृ॑ष॒ता स्पृ॑श स्पृश धृष॒ता । \newline
29. धृ॒ष॒ता शोशु॑चानः॒ शोशु॑चानो धृष॒ता धृ॑ष॒ता शोशु॑चानः । \newline
30. शोशु॑चान॒ इति॒ शोशु॑चानः । \newline
31. तपू(ग्ग्॑)ष्यग्ने अग्ने॒ तपू(ग्म्॑)षि॒ तपू(ग्ग्॑)ष्यग्ने । \newline
32. अ॒ग्ने॒ जु॒ह्वा॑ जु॒ह्वा᳚ ऽग्ने अग्ने जु॒ह्वा᳚ । \newline
33. जु॒ह्वा॑ पत॒ङ्गान् प॑त॒ङ्गान् जु॒ह्वा॑ जु॒ह्वा॑ पत॒ङ्गान् । \newline
34. प॒त॒ङ्गा नस॑न्दितो॒ अस॑न्दितः पत॒ङ्गान् प॑त॒ङ्गा नस॑न्दितः । \newline
35. अस॑न्दितो॒ वि व्यस॑न्दितो॒ अस॑न्दितो॒ वि । \newline
36. अस॑न्दित॒ इत्यसं᳚ - दि॒तः॒ । \newline
37. वि सृ॑ज सृज॒ वि वि सृ॑ज । \newline
38. सृ॒ज॒ विष्व॒ग् विष्व॑ख् सृज सृज॒ विष्व॑क् । \newline
39. विष्व॑गु॒ल्का उ॒ल्का विष्व॒ग् विष्व॑गु॒ल्काः । \newline
40. उ॒ल्का इत्यु॒ल्काः । \newline
41. प्रति॒ स्पशः॒ स्पशः॒ प्रति॒ प्रति॒ स्पशः॑ । \newline
42. स्पशो॒ वि वि स्पशः॒ स्पशो॒ वि । \newline
43. वि सृ॑ज सृज॒ वि वि सृ॑ज । \newline
44. सृ॒ज॒ तूर्णि॑तम॒ स्तूर्णि॑तमः सृज सृज॒ तूर्णि॑तमः । \newline
45. तूर्णि॑तमो॒ भव॒ भव॒ तूर्णि॑तम॒ स्तूर्णि॑तमो॒ भव॑ । \newline
46. तूर्णि॑तम॒ इति॒ तूर्णि॑ - त॒मः॒ । \newline
47. भवा॑ पा॒युः पा॒युर् भव॒ भवा॑ पा॒युः । \newline
48. पा॒युर् वि॒शो वि॒शः पा॒युः पा॒युर् वि॒शः । \newline
49. वि॒शो अ॒स्या अ॒स्या वि॒शो वि॒शो अ॒स्याः । \newline
50. अ॒स्या अद॑ब्धो॒ अद॑ब्धो अ॒स्या अ॒स्या अद॑ब्धः । \newline
51. अद॑ब्ध॒ इत्यद॑ब्धः । \newline
52. यो नो॑ नो॒ यो यो नः॑ । \newline
53. नो॒ दू॒रे दू॒रे नो॑ नो दू॒रे । \newline
54. दू॒रे अ॒घश(ग्म्॑)सो अ॒घश(ग्म्॑)सो दू॒रे दू॒रे अ॒घश(ग्म्॑)सः । \newline
55. अ॒घश(ग्म्॑)सो॒ यो यो अ॒घश(ग्म्॑)सो अ॒घश(ग्म्॑)सो॒ यः । \newline
56. अ॒घश(ग्म्॑)स॒ इत्य॒घ - श॒(ग्म्॒)सः॒ । \newline

\textbf{Ghana Paata } \newline

1. कृ॒णु॒ष्व पाजः॒ पाजः॑ कृणु॒ष्व कृ॑णु॒ष्व पाजः॒ प्रसि॑ति॒म् प्रसि॑ति॒म् पाजः॑ कृणु॒ष्व कृ॑णु॒ष्व पाजः॒ प्रसि॑तिम् । \newline
2. पाजः॒ प्रसि॑ति॒म् प्रसि॑ति॒म् पाजः॒ पाजः॒ प्रसि॑ति॒न्न न प्रसि॑ति॒म् पाजः॒ पाजः॒ प्रसि॑ति॒न्न । \newline
3. प्रसि॑ति॒न्न न प्रसि॑ति॒म् प्रसि॑ति॒न्न पृ॒थ्वीम् पृ॒थ्वीन्न प्रसि॑ति॒म् प्रसि॑ति॒न्न पृ॒थ्वीम् । \newline
4. प्रसि॑ति॒मिति॒ प्र - सि॒ति॒म् । \newline
5. न पृ॒थ्वीम् पृ॒थ्वीन्न न पृ॒थ्वीं ॅया॒हि या॒हि पृ॒थ्वीन्न न पृ॒थ्वीं ॅया॒हि । \newline
6. पृ॒थ्वीं ॅया॒हि या॒हि पृ॒थ्वीम् पृ॒थ्वीं ॅया॒हि राजा॒ राजा॑ या॒हि पृ॒थ्वीम् पृ॒थ्वीं ॅया॒हि राजा᳚ । \newline
7. या॒हि राजा॒ राजा॑ या॒हि या॒हि राजे॑वे व॒ राजा॑ या॒हि या॒हि राजे॑व । \newline
8. राजे॑वे व॒ राजा॒ राजे॒वाम॑वा॒(ग्म्॒) अम॑वाꣳ इव॒ राजा॒ राजे॒वाम॑वान् । \newline
9. इ॒वाम॑वा॒(ग्म्॒) अम॑वाꣳ इवे॒ वाम॑वा॒(ग्म्॒) इभे॒ने भे॒नाम॑वाꣳ इवे॒ वाम॑वा॒(ग्म्॒) इभे॑न । \newline
10. अम॑वा॒(ग्म्॒) इभे॒ने भे॒नाम॑वा॒(ग्म्॒) अम॑वा॒(ग्म्॒) इभे॑न । \newline
11. अम॑वा॒नित्यम॑ - वा॒न् । \newline
12. इभे॒नेतीभे॑न । \newline
13. तृ॒ष्वी मन्वनु॑ तृ॒ष्वीम् तृ॒ष्वी मनु॒ प्रसि॑ति॒म् प्रसि॑ति॒ मनु॑ तृ॒ष्वीम् तृ॒ष्वी मनु॒ प्रसि॑तिम् । \newline
14. अनु॒ प्रसि॑ति॒म् प्रसि॑ति॒ मन्वनु॒ प्रसि॑तिम् द्रूणा॒नो द्रू॑णा॒नः प्रसि॑ति॒ मन्वनु॒ प्रसि॑तिम् द्रूणा॒नः । \newline
15. प्रसि॑तिम् द्रूणा॒नो द्रू॑णा॒नः प्रसि॑ति॒म् प्रसि॑तिम् द्रूणा॒नो ऽस्ता ऽस्ता᳚ द्रूणा॒नः प्रसि॑ति॒म् प्रसि॑तिम् द्रूणा॒नो ऽस्ता᳚ । \newline
16. प्रसि॑ति॒मिति॒ प्र - सि॒ति॒म् । \newline
17. द्रू॒णा॒नो ऽस्ता ऽस्ता᳚ द्रूणा॒नो द्रू॑णा॒नो ऽस्ता᳚ ऽस्य॒स्यस्ता᳚ द्रूणा॒नो द्रू॑णा॒नो ऽस्ता॑ ऽसि । \newline
18. अस्ता᳚ ऽस्य॒स्यस्ता ऽस्ता॑ ऽसि॒ विद्ध्य॒ विद्ध्या॒स्यस्ता ऽस्ता॑ ऽसि॒ विद्ध्य॑ । \newline
19. अ॒सि॒ विद्ध्य॒ विद्ध्या᳚स्यसि॒ विद्ध्य॑ र॒क्षसो॑ र॒क्षसो॒ विद्ध्या᳚स्यसि॒ विद्ध्य॑ र॒क्षसः॑ । \newline
20. विद्ध्य॑ र॒क्षसो॑ र॒क्षसो॒ विद्ध्य॒ विद्ध्य॑ र॒क्षस॒ स्तपि॑ष्ठै॒ स्तपि॑ष्ठै र॒क्षसो॒ विद्ध्य॒ विद्ध्य॑ र॒क्षस॒स्तपि॑ष्ठैः । \newline
21. र॒क्षस॒ स्तपि॑ष्ठै॒ स्तपि॑ष्ठै र॒क्षसो॑ र॒क्षस॒ स्तपि॑ष्ठैः । \newline
22. तपि॑ष्ठै॒रिति॒ तपि॑ष्ठैः । \newline
23. तव॑ भ्र॒मासो᳚ भ्र॒मास॒ स्तव॒ तव॑ भ्र॒मास॑ आशु॒या ऽऽशु॒या भ्र॒मास॒ स्तव॒ तव॑ भ्र॒मास॑ आशु॒या । \newline
24. भ्र॒मास॑ आशु॒या ऽऽशु॒या भ्र॒मासो᳚ भ्र॒मास॑ आशु॒या प॑तन्ति पतन्त्याशु॒या भ्र॒मासो᳚ भ्र॒मास॑ आशु॒या प॑तन्ति । \newline
25. आ॒शु॒या प॑तन्ति पतन्त्याशु॒या ऽऽशु॒या प॑त॒न्त्यन्वनु॑ पतन्त्याशु॒या ऽऽशु॒या प॑त॒न्त्यनु॑ । \newline
26. प॒त॒न्त्यन्वनु॑ पतन्ति पत॒न्त्यनु॑ स्पृश स्पृ॒शानु॑ पतन्ति पत॒न्त्यनु॑ स्पृश । \newline
27. अनु॑ स्पृश स्पृ॒शान्वनु॑ स्पृश धृष॒ता धृ॑ष॒ता स्पृ॒शान्वनु॑ स्पृश धृष॒ता । \newline
28. स्पृ॒श॒ धृ॒ष॒ता धृ॑ष॒ता स्पृ॑श स्पृश धृष॒ता शोशु॑चानः॒ शोशु॑चानो धृष॒ता स्पृ॑श स्पृश धृष॒ता शोशु॑चानः । \newline
29. धृ॒ष॒ता शोशु॑चानः॒ शोशु॑चानो धृष॒ता धृ॑ष॒ता शोशु॑चानः । \newline
30. शोशु॑चान॒ इति॒ शोशु॑चानः । \newline
31. तपू(ग्ग्॑)ष्यग्ने अग्ने॒ तपू(ग्म्॑)षि॒ तपू(ग्ग्॑)ष्यग्ने जु॒ह्वा॑ जु॒ह्वा᳚ ऽग्ने॒ तपू(ग्म्॑)षि॒ तपू(ग्ग्॑)ष्यग्ने जु॒ह्वा᳚ । \newline
32. अ॒ग्ने॒ जु॒ह्वा॑ जु॒ह्वा᳚ ऽग्ने अग्ने जु॒ह्वा॑ पत॒ङ्गान् प॑त॒ङ्गान् जु॒ह्वा᳚ ऽग्ने अग्ने जु॒ह्वा॑ पत॒ङ्गान् । \newline
33. जु॒ह्वा॑ पत॒ङ्गान् प॑त॒ङ्गान् जु॒ह्वा॑ जु॒ह्वा॑ पत॒ङ्गा नस॑न्दितो॒ अस॑न्दितः पत॒ङ्गान् जु॒ह्वा॑ जु॒ह्वा॑ पत॒ङ्गा नस॑न्दितः । \newline
34. प॒त॒ङ्गा नस॑न्दितो॒ अस॑न्दितः पत॒ङ्गान् प॑त॒ङ्गा नस॑न्दितो॒ वि व्यस॑न्दितः पत॒ङ्गान् प॑त॒ङ्गा नस॑न्दितो॒ वि । \newline
35. अस॑न्दितो॒ वि व्यस॑न्दितो॒ अस॑न्दितो॒ वि सृ॑ज सृज॒ व्यस॑न्दितो॒ अस॑न्दितो॒ वि सृ॑ज । \newline
36. अस॑न्दित॒ इत्यसं᳚ - दि॒तः॒ । \newline
37. वि सृ॑ज सृज॒ वि वि सृ॑ज॒ विष्व॒ग् विष्व॑ख् सृज॒ वि वि सृ॑ज॒ विष्व॑क् । \newline
38. सृ॒ज॒ विष्व॒ग् विष्व॑ख् सृज सृज॒ विष्व॑गु॒ल्का उ॒ल्का विष्व॑ख् सृज सृज॒ विष्व॑गु॒ल्काः । \newline
39. विष्व॑गु॒ल्का उ॒ल्का विष्व॒ग् विष्व॑गु॒ल्काः । \newline
40. उ॒ल्का इत्यु॒ल्काः । \newline
41. प्रति॒ स्पशः॒ स्पशः॒ प्रति॒ प्रति॒ स्पशो॒ वि वि स्पशः॒ प्रति॒ प्रति॒ स्पशो॒ वि । \newline
42. स्पशो॒ वि वि स्पशः॒ स्पशो॒ वि सृ॑ज सृज॒ वि स्पशः॒ स्पशो॒ वि सृ॑ज । \newline
43. वि सृ॑ज सृज॒ वि वि सृ॑ज॒ तूर्णि॑तम॒ स्तूर्णि॑तमः सृज॒ वि वि सृ॑ज॒ तूर्णि॑तमः । \newline
44. सृ॒ज॒ तूर्णि॑तम॒ स्तूर्णि॑तमः सृज सृज॒ तूर्णि॑तमो॒ भव॒ भव॒ तूर्णि॑तमः सृज सृज॒ तूर्णि॑तमो॒ भव॑ । \newline
45. तूर्णि॑तमो॒ भव॒ भव॒ तूर्णि॑तम॒ स्तूर्णि॑तमो॒ भवा॑ पा॒युः पा॒युर् भव॒ तूर्णि॑तम॒ स्तूर्णि॑तमो॒ भवा॑ पा॒युः । \newline
46. तूर्णि॑तम॒ इति॒ तूर्णि॑ - त॒मः॒ । \newline
47. भवा॑ पा॒युः पा॒युर् भव॒ भवा॑ पा॒युर् वि॒शो वि॒शः पा॒युर् भव॒ भवा॑ पा॒युर् वि॒शः । \newline
48. पा॒युर् वि॒शो वि॒शः पा॒युः पा॒युर् वि॒शो अ॒स्या अ॒स्या वि॒शः पा॒युः पा॒युर् वि॒शो अ॒स्याः । \newline
49. वि॒शो अ॒स्या अ॒स्या वि॒शो वि॒शो अ॒स्या अद॑ब्धो॒ अद॑ब्धो अ॒स्या वि॒शो वि॒शो अ॒स्या अद॑ब्धः । \newline
50. अ॒स्या अद॑ब्धो॒ अद॑ब्धो अ॒स्या अ॒स्या अद॑ब्धः । \newline
51. अद॑ब्ध॒ इत्यद॑ब्धः । \newline
52. यो नो॑ नो॒ यो यो नो॑ दू॒रे दू॒रे नो॒ यो यो नो॑ दू॒रे । \newline
53. नो॒ दू॒रे दू॒रे नो॑ नो दू॒रे अ॒घश(ग्म्॑)सो अ॒घश(ग्म्॑)सो दू॒रे नो॑ नो दू॒रे अ॒घश(ग्म्॑)सः । \newline
54. दू॒रे अ॒घश(ग्म्॑)सो अ॒घश(ग्म्॑)सो दू॒रे दू॒रे अ॒घश(ग्म्॑)सो॒ यो यो अ॒घश(ग्म्॑)सो दू॒रे दू॒रे अ॒घश(ग्म्॑)सो॒ यः । \newline
55. अ॒घश(ग्म्॑)सो॒ यो यो अ॒घश(ग्म्॑)सो अ॒घश(ग्म्॑)सो॒ यो अन्त्यन्ति॒ यो अ॒घश(ग्म्॑)सो अ॒घश(ग्म्॑)सो॒ यो अन्ति॑ । \newline
56. अ॒घश(ग्म्॑)स॒ इत्य॒घ - श॒(ग्म्॒)सः॒ । \newline
\pagebreak
\markright{ TS 1.2.14.2  \hfill https://www.vedavms.in \hfill}

\section{ TS 1.2.14.2 }

\textbf{TS 1.2.14.2 } \newline
\textbf{Samhita Paata} \newline

यो अन्त्यग्ने॒ माकि॑ष्टे॒ व्यथि॒रा द॑धर्.षीत् ॥ उद॑ग्ने तिष्ठ॒ प्रत्या *ऽऽत॑नुष्व॒ न्य॑मित्राꣳ॑ ओषतात् तिग्महेते । यो नो॒ अरा॑तिꣳ समिधान च॒क्रे नी॒चा तं ध॑क्ष्यत॒सं न शुष्कं᳚ ॥ ऊ॒र्द्ध्वो भ॑व॒ प्रति॑वि॒द्ध्या-ऽद्ध्य॒स्मदा॒विष्कृ॑णुष्व॒ दैव्या᳚न्यग्ने । अव॑स्थि॒रा त॑नुहि यातु॒जूनां᳚ जा॒मिमजा॑मिं॒ प्रमृ॑णीहि॒ शत्रून्॑ ॥ स ते॑ - [ ] \newline

\textbf{Pada Paata} \newline

यः । अन्ति॑ । अग्ने᳚ । माकिः॑ । ते॒ । व्यथिः॑ । एति॑ । द॒ध॒र्.षी॒त् ॥ उदिति॑ । अ॒ग्ने॒ । ति॒ष्ठ॒ । प्रईति॑ । एति॑ । त॒नु॒ष्व॒ । नीति॑ । अ॒मित्रान्॑ । ओ॒ष॒ता॒त् । ति॒ग्म॒हे॒त॒ इति॑ तिग्म - हे॒ते॒ ॥ यः । नः॒ । अरा॑तिम् । स॒मि॒धा॒नेति॑ सम् - इ॒धा॒न॒ । च॒क्रे । नी॒चा । तम् । ध॒क्षि॒ । अ॒त॒सम् । न । शुष्क᳚म् ॥ ऊ॒र्द्ध्वः । भ॒व॒ । प्रतीति॑ । वि॒द्ध्य॒ । अधीति॑ । अ॒स्मत् । आ॒विः । कृ॒णु॒ष्व॒ । दैव्या॑नी । अ॒ग्ने॒ ॥ अवेति॑ । स्थि॒रा । त॒नु॒हि॒ । या॒तु॒जूना᳚म् । जा॒मिम् । अजा॑मिम् । प्रेति॑ । मृ॒णी॒हि॒ । शत्रून्॑ ॥ सः । ते॒ ।  \newline


\textbf{Krama Paata} \newline

यो अन्ति॑ । अन्त्यग्ने᳚ । अग्ने॒ माकिः॑ । माकि॑ष्टे । ते॒ व्यथिः॑ । व्यथि॒रा । आ द॑धर्.षीत् । द॒ध॒र्॒.षी॒दिति॑ दधर्.षीत् ॥ उद॑ग्ने । अ॒ग्ने॒ ति॒ष्ठ॒ । ति॒ष्ठ॒ प्रति॑ । प्रत्या । आ त॑नुष्व । त॒नु॒ष्व॒ नि । न्य॑मित्रान्॑ । अ॒मित्राꣳ॑ ओषतात् । ओ॒ष॒ता॒त् ति॒ग्म॒हे॒ते॒ । ति॒ग्म॒हे॒त॒ इति॑ तिग्म - हे॒ते॒ ॥ यो नः॑ । नो॒ अरा॑तिम् । अरा॑तिꣳ समिधान । स॒मि॒धा॒न॒ च॒क्रे । स॒मि॒धा॒नेति॑ सं - इ॒धा॒न॒ । च॒क्रे नी॒चा । नी॒चा तम् । तम् ध॑क्षि । ध॒क्ष्य॒त॒सम् । अ॒त॒सम् न । न शुष्क᳚म् । शुष्क॒मिति॒ शुष्क᳚म् ॥ ऊ॒र्द्ध्वो भ॑व । भ॒व॒ प्रति॑ । प्रति॑ विद्ध्य । वि॒द्ध्याधि॑ । अद्ध्य॒स्मत् । अ॒स्मदा॒विः । आ॒विष्कृ॑णुष्व । कृ॒णु॒ष्व॒ दैव्या॑नि । दैव्या᳚न्यग्ने । अ॒ग्ने॒ इत्य॑ग्ने ॥ अव॑ स्थि॒रा । स्थि॒रा त॑नुहि । त॒नु॒हि॒ या॒तु॒जूना᳚म् । या॒तु॒जूना᳚म् जा॒मिम् । जा॒मिमजा॑मिम् । अजा॑मि॒म् प्र । प्र मृ॑णीहि । मृ॒णी॒हि॒ शत्रून्॑ । शत्रू॒निति॒ शत्रून्॑ ॥ 
स ते᳚ । ते॒ जा॒ना॒ति॒ \newline

\textbf{Jatai Paata} \newline

1. यो अन्त्यन्ति॒ यो यो अन्ति॑ । \newline
2. अन्त्यग्ने ऽग्ने॒ अन्त्यन्त्यग्ने᳚ । \newline
3. अग्ने॒ माकि॒र् माकि॒रग्ने ऽग्ने॒ माकिः॑ । \newline
4. माकि॑ष्टे ते॒ माकि॒र् माकि॑ष्टे । \newline
5. ते॒ व्यथि॒र् व्यथि॑ स्ते ते॒ व्यथिः॑ । \newline
6. व्यथि॒रा व्यथि॒र् व्यथि॒रा । \newline
7. आ द॑धर्.षीद् दधर्.षी॒दा द॑धर्.षीत् । \newline
8. द॒ध॒र्॒.षी॒दिति॑ दधर्.षीत् । \newline
9. उद॑ग्ने अग्न॒ उदुद॑ग्ने । \newline
10. अ॒ग्ने॒ ति॒ष्ठ॒ ति॒ष्ठा॒ग्ने॒ अ॒ग्ने॒ ति॒ष्ठ॒ । \newline
11. ति॒ष्ठ॒ प्रति॒ प्रति॑ तिष्ठ तिष्ठ॒ प्रति॑ । \newline
12. प्रत्या प्रति॒ प्रत्या । \newline
13. आ त॑नुष्व तनु॒ष्वा त॑नुष्व । \newline
14. त॒नु॒ष्व॒ नि नि त॑नुष्व तनुष्व॒ नि । \newline
15. न्य॑मित्रा(ग्म्॑) अ॒मित्रा॒न् नि न्य॑मित्रान्॑ । \newline
16. अ॒मित्रा(ग्म्॑) ओषता दोषता द॒मित्रा(ग्म्॑) अ॒मित्रा(ग्म्॑) ओषतात् । \newline
17. ओ॒ष॒ता॒त् ति॒ग्म॒हे॒ते॒ ति॒ग्म॒हे॒त॒ ओ॒ष॒ता॒ दो॒ष॒ता॒त् ति॒ग्म॒हे॒ते॒ । \newline
18. ति॒ग्म॒हे॒त॒ इति॑ तिग्म - हे॒ते॒ । \newline
19. यो नो॑ नो॒ यो यो नः॑ । \newline
20. नो॒ अरा॑ति॒ मरा॑तिन्नो नो॒ अरा॑तिम् । \newline
21. अरा॑तिꣳ समिधान समिधा॒ना रा॑ति॒ मरा॑तिꣳ समिधान । \newline
22. स॒मि॒धा॒न॒ च॒क्रे च॒क्रे स॑मिधान समिधान च॒क्रे । \newline
23. स॒मि॒धा॒नेति॑ सम् - इ॒धा॒न॒ । \newline
24. च॒क्रे नी॒चा नी॒चा च॒क्रे च॒क्रे नी॒चा । \newline
25. नी॒चा तम् तन्नी॒चा नी॒चा तम् । \newline
26. तम् ध॑क्षि धक्षि॒ तम् तम् ध॑क्षि । \newline
27. ध॒क्ष्य॒त॒स म॑त॒सम् ध॑क्षि धक्ष्यत॒सम् । \newline
28. अ॒त॒सम् न नात॒स म॑त॒सम् न । \newline
29. न शुष्क॒(ग्म्॒) शुष्क॒म् न न शुष्क᳚म् । \newline
30. शुष्क॒मिति॒ शुष्क᳚म् । \newline
31. ऊ॒र्द्ध्वो भ॑व भवो॒र्द्ध्व ऊ॒र्द्ध्वो भ॑व । \newline
32. भ॒व॒ प्रति॒ प्रति॑ भव भव॒ प्रति॑ । \newline
33. प्रति॑ विद्ध्य विद्ध्य॒ प्रति॒ प्रति॑ विद्ध्य । \newline
34. वि॒द्ध्याध्यधि॑ विद्ध्य वि॒द्ध्याधि॑ । \newline
35. अध्य॒स्म द॒स्म दध्य ध्य॒स्मत् । \newline
36. अ॒स्म दा॒वि रा॒वि र॒स्म द॒स्म दा॒विः । \newline
37. आ॒विष्कृ॑णुष्व कृणुष्वा॒वि रा॒विष्कृ॑णुष्व । \newline
38. कृ॒णु॒ष्व॒ दैव्या॑नि॒ दैव्या॑नि कृणुष्व कृणुष्व॒ दैव्या॑नि । \newline
39. दैव्या᳚न्यग्ने अग्ने॒ दैव्या॑नि॒ दैव्या᳚न्यग्ने । \newline
40. अ॒ग्ने॒ इत्य॑ग्ने । \newline
41. अव॑ स्थि॒रा स्थि॒रा ऽवाव॑ स्थि॒रा । \newline
42. स्थि॒रा त॑नुहि तनुहि स्थि॒रा स्थि॒रा त॑नुहि । \newline
43. त॒नु॒हि॒ या॒तु॒जूनां᳚ ॅयातु॒जूना᳚म् तनुहि तनुहि यातु॒जूना᳚म् । \newline
44. या॒तु॒जूना᳚म् जा॒मिम् जा॒मिं ॅया॑तु॒जूनां᳚ ॅयातु॒जूना᳚म् जा॒मिम् । \newline
45. जा॒मि मजा॑मि॒ मजा॑मिम् जा॒मिम् जा॒मि मजा॑मिम् । \newline
46. अजा॑मि॒म् प्र प्राजा॑मि॒ मजा॑मि॒म् प्र । \newline
47. प्र मृ॑णीहि मृणीहि॒ प्र प्र मृ॑णीहि । \newline
48. मृ॒णी॒हि॒ शत्रू॒ञ् छत्रू᳚न् मृणीहि मृणीहि॒ शत्रून्॑ । \newline
49. शत्रू॒निति॒ शत्रून्॑ । \newline
50. स ते॑ ते॒ स स ते᳚ । \newline
51. ते॒ जा॒ना॒ति॒ जा॒ना॒ति॒ ते॒ ते॒ जा॒ना॒ति॒ । \newline

\textbf{Ghana Paata } \newline

1. यो अन्त्यन्ति॒ यो यो अन्त्यग्ने ऽग्ने॒ अन्ति॒ यो यो अन्त्यग्ने᳚ । \newline
2. अन्त्यग्ने ऽग्ने॒ अन्त्यन्त्यग्ने॒ माकि॒र् माकि॒ रग्ने॒ अन्त्यन्त्यग्ने॒ माकिः॑ । \newline
3. अग्ने॒ माकि॒र् माकि॒रग्ने ऽग्ने॒ माकि॑ष् टे ते॒ माकि॒रग्ने ऽग्ने॒ माकि॑ष् टे । \newline
4. माकि॑ष्टे ते॒ माकि॒र् माकि॑ष्टे॒ व्यथि॒र् व्यथि॑स्ते॒ माकि॒र् माकि॑ष्टे॒ व्यथिः॑ । \newline
5. ते॒ व्यथि॒र् व्यथि॑स्ते ते॒ व्यथि॒रा व्यथि॑स्ते ते॒ व्यथि॒रा । \newline
6. व्यथि॒रा व्यथि॒र् व्यथि॒रा द॑धर्.षीद् दधर्.षी॒दा व्यथि॒र् व्यथि॒रा द॑धर्.षीत् । \newline
7. आ द॑धर्.षीद् दधर्.षी॒दा द॑धर्.षीत् । \newline
8. द॒ध॒र्॒.षी॒दिति॑ दधर्.षीत् । \newline
9. उद॑ग्ने अग्न॒ उदुद॑ग्ने तिष्ठ तिष्ठाग्न॒ उदुद॑ग्ने तिष्ठ । \newline
10. अ॒ग्ने॒ ति॒ष्ठ॒ ति॒ष्ठा॒ग्ने॒ अ॒ग्ने॒ ति॒ष्ठ॒ प्रति॒ प्रति॑ तिष्ठाग्ने अग्ने तिष्ठ॒ प्रति॑ । \newline
11. ति॒ष्ठ॒ प्रति॒ प्रति॑ तिष्ठ तिष्ठ॒ प्रत्या प्रति॑ तिष्ठ तिष्ठ॒ प्रत्या । \newline
12. प्रत्या प्रति॒ प्रत्या त॑नुष्व तनु॒ष्वा प्रति॒ प्रत्या त॑नुष्व । \newline
13. आ त॑नुष्व तनु॒ष्वा त॑नुष्व॒ नि नि त॑नु॒ष्वा त॑नुष्व॒ नि । \newline
14. त॒नु॒ष्व॒ नि नि त॑नुष्व तनुष्व॒ न्य॑मित्रा(ग्म्॑) अ॒मित्रा॒न् नि त॑नुष्व तनुष्व॒ न्य॑मित्रान्॑ । \newline
15. न्य॑मित्रा(ग्म्॑) अ॒मित्रा॒न् नि न्य॑मित्रा(ग्म्॑) ओषतादोषता द॒मित्रा॒न् नि न्य॑मित्रा(ग्म्॑) ओषतात् । \newline
16. अ॒मित्रा(ग्म्॑) ओषता दोषता द॒मित्रा(ग्म्॑) अ॒मित्रा(ग्म्॑) ओषतात् तिग्महेते तिग्महेत ओषता द॒मित्रा(ग्म्॑) अ॒मित्रा(ग्म्॑) ओषतात् तिग्महेते । \newline
17. ओ॒ष॒ता॒त् ति॒ग्म॒हे॒ते॒ ति॒ग्म॒हे॒त॒ ओ॒ष॒ता॒ दो॒ष॒ता॒त् ति॒ग्म॒हे॒ते॒ । \newline
18. ति॒ग्म॒हे॒त॒ इति॑ तिग्म - हे॒ते॒ । \newline
19. यो नो॑ नो॒ यो यो नो॒ अरा॑ति॒ मरा॑तिन्नो॒ यो यो नो॒ अरा॑तिम् । \newline
20. नो॒ अरा॑ति॒ मरा॑तिन्नो नो॒ अरा॑तिꣳ समिधान समिधा॒ना रा॑तिन्नो नो॒ अरा॑तिꣳ समिधान । \newline
21. अरा॑तिꣳ समिधान समिधा॒नारा॑ति॒ मरा॑तिꣳ समिधान च॒क्रे च॒क्रे स॑मिधा॒नारा॑ति॒ मरा॑तिꣳ समिधान च॒क्रे । \newline
22. स॒मि॒धा॒न॒ च॒क्रे च॒क्रे स॑मिधान समिधान च॒क्रे नी॒चा नी॒चा च॒क्रे स॑मिधान समिधान च॒क्रे नी॒चा । \newline
23. स॒मि॒धा॒नेति॑ सम् - इ॒धा॒न॒ । \newline
24. च॒क्रे नी॒चा नी॒चा च॒क्रे च॒क्रे नी॒चा तम् तन्नी॒चा च॒क्रे च॒क्रे नी॒चा तम् । \newline
25. नी॒चा तम् तन्नी॒चा नी॒चा तम् ध॑क्षि धक्षि॒ तन्नी॒चा नी॒चा तम् ध॑क्षि । \newline
26. तम् ध॑क्षि धक्षि॒ तम् तम् ध॑क्ष्यत॒स म॑त॒सम् ध॑क्षि॒ तम् तम् ध॑क्ष्यत॒सम् । \newline
27. ध॒क्ष्य॒त॒स म॑त॒सम् ध॑क्षि धक्ष्यत॒सन्न नात॒सम् ध॑क्षि धक्ष्यत॒सन्न । \newline
28. अ॒त॒सन्न नात॒स म॑त॒सन्न शुष्क॒(ग्म्॒) शुष्क॒न्नात॒स म॑त॒सन्न शुष्क᳚म् । \newline
29. न शुष्क॒(ग्म्॒) शुष्क॒न्न न शुष्क᳚म् । \newline
30. शुष्क॒मिति॒ शुष्क᳚म् । \newline
31. ऊ॒र्द्ध्वो भ॑व भवो॒र्द्ध्व ऊ॒र्द्ध्वो भ॑व॒ प्रति॒ प्रति॑ भवो॒र्द्ध्व ऊ॒र्द्ध्वो भ॑व॒ प्रति॑ । \newline
32. भ॒व॒ प्रति॒ प्रति॑ भव भव॒ प्रति॑ विद्ध्य विद्ध्य॒ प्रति॑ भव भव॒ प्रति॑ विद्ध्य । \newline
33. प्रति॑ विद्ध्य विद्ध्य॒ प्रति॒ प्रति॑ वि॒द्ध्याध्यधि॑ विद्ध्य॒ प्रति॒ प्रति॑ वि॒द्ध्याधि॑ । \newline
34. वि॒द्ध्या ध्यधि॑ विद्ध्य वि॒द्ध्या ध्य॒स्म द॒स्मदधि॑ विद्ध्य वि॒द्ध्या ध्य॒स्मत् । \newline
35. अध्य॒स्म द॒स्म दध्यध्य॒ स्मदा॒वि रा॒वि र॒स्म दध्यध्य॒ स्मदा॒विः । \newline
36. अ॒स्मदा॒वि रा॒वि र॒स्म द॒स्मदा॒विष् कृ॑णुष्व कृणुष्वा॒वि र॒स्म द॒स्मदा॒विष् कृ॑णुष्व । \newline
37. आ॒विष् कृ॑णुष्व कृणुष्वा॒वि रा॒विष् कृ॑णुष्व॒ दैव्या॑नि॒ दैव्या॑नि कृणुष्वा॒वि रा॒विष् कृ॑णुष्व॒ दैव्या॑नि । \newline
38. कृ॒णु॒ष्व॒ दैव्या॑नि॒ दैव्या॑नि कृणुष्व कृणुष्व॒ दैव्या᳚न्यग्ने अग्ने॒ दैव्या॑नि कृणुष्व कृणुष्व॒ दैव्या᳚न्यग्ने । \newline
39. दैव्या᳚न्यग्ने अग्ने॒ दैव्या॑नि॒ दैव्या᳚न्यग्ने । \newline
40. अ॒ग्ने॒ इत्य॑ग्ने । \newline
41. अव॑ स्थि॒रा स्थि॒रा ऽवाव॑ स्थि॒रा त॑नुहि तनुहि स्थि॒रा ऽवाव॑ स्थि॒रा त॑नुहि । \newline
42. स्थि॒रा त॑नुहि तनुहि स्थि॒रा स्थि॒रा त॑नुहि यातु॒जूनां᳚ ॅयातु॒जूना᳚म् तनुहि स्थि॒रा स्थि॒रा त॑नुहि यातु॒जूना᳚म् । \newline
43. त॒नु॒हि॒ या॒तु॒जूनां᳚ ॅयातु॒जूना᳚म् तनुहि तनुहि यातु॒जूना᳚म् जा॒मिम् जा॒मिं ॅया॑तु॒जूना᳚म् तनुहि तनुहि यातु॒जूना᳚म् जा॒मिम् । \newline
44. या॒तु॒जूना᳚म् जा॒मिम् जा॒मिं ॅया॑तु॒जूनां᳚ ॅयातु॒जूना᳚म् जा॒मि मजा॑मि॒ मजा॑मिम् जा॒मिं ॅया॑तु॒जूनां᳚ ॅयातु॒जूना᳚म् जा॒मि मजा॑मिम् । \newline
45. जा॒मि मजा॑मि॒ मजा॑मिम् जा॒मिम् जा॒मि मजा॑मि॒म् प्र प्राजा॑मिम् जा॒मिम् जा॒मि मजा॑मि॒म् प्र । \newline
46. अजा॑मि॒म् प्र प्राजा॑मि॒ मजा॑मि॒म् प्र मृ॑णीहि मृणीहि॒ प्राजा॑मि॒ मजा॑मि॒म् प्र मृ॑णीहि । \newline
47. प्र मृ॑णीहि मृणीहि॒ प्र प्र मृ॑णीहि॒ शत्रू॒ञ् छत्रू᳚न् मृणीहि॒ प्र प्र मृ॑णीहि॒ शत्रून्॑ । \newline
48. मृ॒णी॒हि॒ शत्रू॒ञ् छत्रू᳚न् मृणीहि मृणीहि॒ शत्रून्॑ । \newline
49. शत्रू॒निति॒ शत्रून्॑ । \newline
50. स ते॑ ते॒ स स ते॑ जानाति जानाति ते॒ स स ते॑ जानाति । \newline
51. ते॒ जा॒ना॒ति॒ जा॒ना॒ति॒ ते॒ ते॒ जा॒ना॒ति॒ सु॒म॒तिꣳ सु॑म॒तिम् जा॑नाति ते ते जानाति सुम॒तिम् । \newline
\pagebreak
\markright{ TS 1.2.14.3  \hfill https://www.vedavms.in \hfill}

\section{ TS 1.2.14.3 }

\textbf{TS 1.2.14.3 } \newline
\textbf{Samhita Paata} \newline

जानाति सुम॒तिं ॅय॑विष्ठ॒य ईव॑ते॒ ब्रह्म॑णे गा॒तुमैर॑त् । विश्वा᳚न्यस्मै सु॒दिना॑नि रा॒यो द्यु॒म्नान्य॒र्यो विदुरो॑ अ॒भि द्यौ᳚त् ॥ सेद॑ग्ने अस्तु सु॒भगः॑ सु॒दानु॒-र्यस्त्वा॒ नित्ये॑न ह॒विषा॒य उ॒क्थैः । पिप्री॑षति॒ स्व आयु॑षि दुरो॒णे विश्वेद॑स्मै सु॒दिना॒ साऽस॑दि॒ष्टिः ॥ अर्चा॑मि ते सुम॒तिं घोष्य॒र्वाख्-सन्ते॑ वा॒ वा ता॑ जरता - [ ] \newline

\textbf{Pada Paata} \newline

जा॒ना॒ति॒ । सु॒म॒तिमिति॑ सु - म॒तिम् । य॒वि॒ष्ठ॒ । यः । ईव॑ते । ब्रह्म॑णे । गा॒तुम् । ऐर॑त् ॥ विश्वा॑नि । अ॒स्मै॒ । सु॒दिना॒नीति॑ सु - दिना॑नि । रा॒यः । द्यु॒म्नानि॑ । अ॒र्यः । वीति॑ । दुरः॑ । अ॒भीति॑ । द्यौ॒त् ॥ सः । इत् । अ॒ग्ने॒ । अ॒स्तु॒ । सु॒भग॒ इति॑ सु - भगः॑ । सु॒दानु॒रिति॑ सु - दानुः॑ । यः । त्वा॒ । नित्ये॑न । ह॒विषा᳚ । यः । उ॒क्थैः ॥ पिप्री॑षति । स्वे । आयु॑षि । दु॒रो॒ण इति॑ दुः - ओ॒ने । विश्वा᳚ । इत् । अ॒स्मै॒ । सु॒दिनेति॑ सु- दिना᳚ । सा । अस॑त् । इ॒ष्टिः ॥ अर्चा॑मि । ते॒ । सु॒म॒तिमिति॑ सु - म॒तिम् । घोषि॑ । अ॒र्वाक् । समिति॑ । ते॒ । वा॒वाता᳚ । ज॒र॒तां॒ ।  \newline


\textbf{Krama Paata} \newline

जा॒ना॒ति॒ सु॒म॒तिम् । सु॒म॒तिं ॅय॑विष्ठ । सु॒म॒तिमिति॑ सु - म॒तिम् । य॒वि॒ष्ठ॒ यः । य ईव॑ते । ईव॑ते॒ ब्रह्म॑णे । ब्रह्म॑णे गा॒तुम् । गा॒तुमैर॑त् । ऐर॒दित्यैर॑त् ॥ विश्वा᳚न्यस्मै । अ॒स्मै॒ सु॒दिना॑नि । सु॒दिना॑नि रा॒यः । सु॒दिना॒नीति॑ सु - दिना॑नि । रा॒यो द्यु॒म्नानि॑ । द्यु॒म्नान्य॒र्यः । अ॒र्यो वि । वि दुरः॑ । दुरो॑ अ॒भि । अ॒भि द्यौ᳚त् । द्यौ॒दिति॑ द्यौत् ॥ सेत् । इद॑ग्ने । अ॒ग्ने॒ अ॒स्तु॒ । अ॒स्तु॒ सु॒भगः॑ । सु॒भगः॑ सु॒दानुः॑ । सु॒भग॒ इति॑ सु - भगः॑ । सु॒दानु॒र् यः । सु॒दानु॒रिति॑ सु - दानुः॑ । यस्त्वा᳚ । त्वा॒ नित्ये॑न । नित्ये॑न ह॒विषा᳚ । ह॒विषा॒ यः । य उ॒क्थैः । उ॒क्थैरित्यु॒क्थैः ॥ पिप्री॑षति॒ स्वे । स्व आयु॑षि । आयु॑षि दुरो॒णे । दु॒रो॒णे विश्वा᳚ । दु॒रो॒ण इति॑ दुः - ओ॒ने । विश्वेत् । इद॑स्मै । अ॒स्मै॒ सु॒दिना᳚ । सु॒दिना॒ सा । सु॒दिनेति॑ सु - दिना᳚ । साऽस॑त् । अस॑दि॒ष्टिः । इ॒ष्टिरिती॒ष्टिः ॥ अर्चा॑मि ते । ते॒ सु॒म॒तिम् । सु॒म॒तिम् घोषि॑ । सु॒म॒तिमिति॑ सु - म॒तिम् । घोष्य॒र्वाक् । अ॒र्वाख् सम् । सम् ते᳚ । ते॒ वा॒वाता᳚ । वा॒वाता॑ जरताम् । ज॒र॒ता॒मि॒यम् \newline

\textbf{Jatai Paata} \newline

1. जा॒ना॒ति॒ सु॒म॒तिꣳ सु॑म॒तिम् जा॑नाति जानाति सुम॒तिम् । \newline
2. सु॒म॒तिं ॅय॑विष्ठ यविष्ठ सुम॒तिꣳ सु॑म॒तिं ॅय॑विष्ठ । \newline
3. सु॒म॒तिमिति॑ सु - म॒तिम् । \newline
4. य॒वि॒ष्ठ॒ यो यो य॑विष्ठ यविष्ठ॒ यः । \newline
5. य ईव॑त॒ ईव॑ते॒ यो य ईव॑ते । \newline
6. ईव॑ते॒ ब्रह्म॑णे॒ ब्रह्म॑ण॒ ईव॑त॒ ईव॑ते॒ ब्रह्म॑णे । \newline
7. ब्रह्म॑णे गा॒तुम् गा॒तुम् ब्रह्म॑णे॒ ब्रह्म॑णे गा॒तुम् । \newline
8. गा॒तु मैर॒ दैर॑द् गा॒तुम् गा॒तु मैर॑त् । \newline
9. ऐर॒दित्यैर॑त् । \newline
10. विश्वा᳚न्यस्मा अस्मै॒ विश्वा॑नि॒ विश्वा᳚न्यस्मै । \newline
11. अ॒स्मै॒ सु॒दिना॑नि सु॒दिना᳚ न्यस्मा अस्मै सु॒दिना॑नि । \newline
12. सु॒दिना॑नि रा॒यो रा॒यः सु॒दिना॑नि सु॒दिना॑नि रा॒यः । \newline
13. सु॒दिना॒नीति॑ सु - दिना॑नि । \newline
14. रा॒यो द्यु॒म्नानि॑ द्यु॒म्नानि॑ रा॒यो रा॒यो द्यु॒म्नानि॑ । \newline
15. द्यु॒म्नान्य॒र्यो अ॒र्यो द्यु॒म्नानि॑ द्यु॒म्नान्य॒र्यः । \newline
16. अ॒र्यो वि व्य॑र्यो अ॒र्यो वि । \newline
17. वि दुरो॒ दुरो॒ वि वि दुरः॑ । \newline
18. दुरो॑ अ॒भ्य॑भि दुरो॒ दुरो॑ अ॒भि । \newline
19. अ॒भि द्यौ᳚द् द्यौ द॒भ्य॑भि द्यौ᳚त् । \newline
20. द्यौ॒दिति॑ द्यौत् । \newline
21. सेदिथ् स सेत् । \newline
22. इद॑ग्ने अग्न॒ इदिद॑ग्ने । \newline
23. अ॒ग्ने॒ अ॒स्त्व॒ स्त्व॒ग्ने॒ अ॒ग्ने॒ अ॒स्तु॒ । \newline
24. अ॒स्तु॒ सु॒भगः॑ सु॒भगो॑ अस्त्वस्तु सु॒भगः॑ । \newline
25. सु॒भगः॑ सु॒दानुः॑ सु॒दानुः॑ सु॒भगः॑ सु॒भगः॑ सु॒दानुः॑ । \newline
26. सु॒भग॒ इति॑ सु - भगः॑ । \newline
27. सु॒दानु॒र् यो यः सु॒दानुः॑ सु॒दानु॒र् यः । \newline
28. सु॒दानु॒रिति॑ सु - दानुः॑ । \newline
29. यस्त्वा᳚ त्वा॒ यो य स्त्वा᳚ । \newline
30. त्वा॒ नित्ये॑न॒ नित्ये॑न त्वा त्वा॒ नित्ये॑न । \newline
31. नित्ये॑न ह॒विषा॑ ह॒विषा॒ नित्ये॑न॒ नित्ये॑न ह॒विषा᳚ । \newline
32. ह॒विषा॒ यो यो ह॒विषा॑ ह॒विषा॒ यः । \newline
33. य उ॒क्थै रु॒क्थैर् यो य उ॒क्थैः । \newline
34. उ॒क्थैरित्यु॒क्थैः । \newline
35. पिप्री॑षति॒ स्वे स्वे पिप्री॑षति॒ पिप्री॑षति॒ स्वे । \newline
36. स्व आयु॒ष्यायु॑षि॒ स्वे स्व आयु॑षि । \newline
37. आयु॑षि दुरो॒णे दु॑रो॒ण आयु॒ष्यायु॑षि दुरो॒णे । \newline
38. दु॒रो॒णे विश्वा॒ विश्वा॑ दुरो॒णे दु॑रो॒णे विश्वा᳚ । \newline
39. दु॒रो॒ण इति॑ दुः - ओ॒ने । \newline
40. विश्वेदिद् विश्वा॒ विश्वेत् । \newline
41. इद॑स्मा अस्मा॒ इदि द॑स्मै । \newline
42. अ॒स्मै॒ सु॒दिना॑ सु॒दिना᳚ ऽस्मा अस्मै सु॒दिना᳚ । \newline
43. सु॒दिना॒ सा सा सु॒दिना॑ सु॒दिना॒ सा । \newline
44. सु॒दिनेति॑ सु - दिना᳚ । \newline
45. सा ऽस॒दस॒थ् सा सा ऽस॑त् । \newline
46. अस॑ दि॒ष्टि रि॒ष्टि रस॒ दस॑ दि॒ष्टिः । \newline
47. इ॒ष्टिरिती॒ष्टिः । \newline
48. अर्चा॑मि ते ते॒ अर्चा॒म्यर्चा॑मि ते । \newline
49. ते॒ सु॒म॒तिꣳ सु॑म॒तिम् ते॑ ते सुम॒तिम् । \newline
50. सु॒म॒तिम् घोषि॒ घोषि॑ सुम॒तिꣳ सु॑म॒तिम् घोषि॑ । \newline
51. सु॒म॒तिमिति॑ सु - म॒तिम् । \newline
52. घोष्य॒र्वा ग॒र्वाग् घोषि॒ घोष्य॒र्वाक् । \newline
53. अ॒र्वाख् सꣳ स म॒र्वा ग॒र्वाख् सम् । \newline
54. सम् ते॑ ते॒ सꣳ सम् ते᳚ । \newline
55. ते॒ वा॒वाता॑ वा॒वाता॑ ते ते वा॒वाता᳚ । \newline
56. वा॒वाता॑ जरतां जरतां वा॒वाता॑ वा॒वाता॑ जरतां । \newline
57. ज॒र॒ता॒ मि॒य मि॒यम् ज॑रतां जरता मि॒यम् । \newline

\textbf{Ghana Paata } \newline

1. जा॒ना॒ति॒ सु॒म॒तिꣳ सु॑म॒तिम् जा॑नाति जानाति सुम॒तिं ॅय॑विष्ठ यविष्ठ सुम॒तिम् जा॑नाति जानाति सुम॒तिं ॅय॑विष्ठ । \newline
2. सु॒म॒तिं ॅय॑विष्ठ यविष्ठ सुम॒तिꣳ सु॑म॒तिं ॅय॑विष्ठ॒ यो यो य॑विष्ठ सुम॒तिꣳ सु॑म॒तिं ॅय॑विष्ठ॒ यः । \newline
3. सु॒म॒तिमिति॑ सु - म॒तिम् । \newline
4. य॒वि॒ष्ठ॒ यो यो य॑विष्ठ यविष्ठ॒ य ईव॑त॒ ईव॑ते॒ यो य॑विष्ठ यविष्ठ॒ य ईव॑ते । \newline
5. य ईव॑त॒ ईव॑ते॒ यो य ईव॑ते॒ ब्रह्म॑णे॒ ब्रह्म॑ण॒ ईव॑ते॒ यो य ईव॑ते॒ ब्रह्म॑णे । \newline
6. ईव॑ते॒ ब्रह्म॑णे॒ ब्रह्म॑ण॒ ईव॑त॒ ईव॑ते॒ ब्रह्म॑णे गा॒तुम् गा॒तुम् ब्रह्म॑ण॒ ईव॑त॒ ईव॑ते॒ ब्रह्म॑णे गा॒तुम् । \newline
7. ब्रह्म॑णे गा॒तुम् गा॒तुम् ब्रह्म॑णे॒ ब्रह्म॑णे गा॒तु मैर॒दैर॑द् गा॒तुम् ब्रह्म॑णे॒ ब्रह्म॑णे गा॒तु मैर॑त् । \newline
8. गा॒तु मैर॒दैर॑द् गा॒तुम् गा॒तु मैर॑त् । \newline
9. ऐर॒दित्यैर॑त् । \newline
10. विश्वा᳚न्यस्मा अस्मै॒ विश्वा॑नि॒ विश्वा᳚न्यस्मै सु॒दिना॑नि सु॒दिना᳚न्यस्मै॒ विश्वा॑नि॒ विश्वा᳚न्यस्मै सु॒दिना॑नि । \newline
11. अ॒स्मै॒ सु॒दिना॑नि सु॒दिना᳚न्यस्मा अस्मै सु॒दिना॑नि रा॒यो रा॒यः सु॒दिना᳚न्यस्मा अस्मै सु॒दिना॑नि रा॒यः । \newline
12. सु॒दिना॑नि रा॒यो रा॒यः सु॒दिना॑नि सु॒दिना॑नि रा॒यो द्यु॒म्नानि॑ द्यु॒म्नानि॑ रा॒यः सु॒दिना॑नि सु॒दिना॑नि रा॒यो द्यु॒म्नानि॑ । \newline
13. सु॒दिना॒नीति॑ सु - दिना॑नि । \newline
14. रा॒यो द्यु॒म्नानि॑ द्यु॒म्नानि॑ रा॒यो रा॒यो द्यु॒म्नान्य॒र्यो अ॒र्यो द्यु॒म्नानि॑ रा॒यो रा॒यो द्यु॒म्नान्य॒र्यः । \newline
15. द्यु॒म्नान्य॒र्यो अ॒र्यो द्यु॒म्नानि॑ द्यु॒म्नान्य॒र्यो वि व्य॑र्यो द्यु॒म्नानि॑ द्यु॒म्नान्य॒र्यो वि । \newline
16. अ॒र्यो वि व्य॑र्यो अ॒र्यो वि दुरो॒ दुरो॒ व्य॑र्यो अ॒र्यो वि दुरः॑ । \newline
17. वि दुरो॒ दुरो॒ वि वि दुरो॑ अ॒भ्य॑भि दुरो॒ वि वि दुरो॑ अ॒भि । \newline
18. दुरो॑ अ॒भ्य॑भि दुरो॒ दुरो॑ अ॒भि द्यौ᳚द् द्यौद॒भि दुरो॒ दुरो॑ अ॒भि द्यौ᳚त् । \newline
19. अ॒भि द्यौ᳚द् द्यौद॒भ्य॑भि द्यौ᳚त् । \newline
20. द्यौ॒दिति॑ द्यौत् । \newline
21. सेदिथ् स सेद॑ग्ने अग्न॒ इथ् स सेद॑ग्ने । \newline
22. इद॑ग्ने अग्न॒ इदिद॑ग्ने अस्त्वस्त्वग्न॒ इदिद॑ग्ने अस्तु । \newline
23. अ॒ग्ने॒ अ॒स्त्व॒स्त्व॒ग्ने॒ अ॒ग्ने॒ अ॒स्तु॒ सु॒भगः॑ सु॒भगो॑ अस्त्वग्ने अग्ने अस्तु सु॒भगः॑ । \newline
24. अ॒स्तु॒ सु॒भगः॑ सु॒भगो॑ अस्त्वस्तु सु॒भगः॑ सु॒दानुः॑ सु॒दानुः॑ सु॒भगो॑ अस्त्वस्तु सु॒भगः॑ सु॒दानुः॑ । \newline
25. सु॒भगः॑ सु॒दानुः॑ सु॒दानुः॑ सु॒भगः॑ सु॒भगः॑ सु॒दानु॒र् यो यः सु॒दानुः॑ सु॒भगः॑ सु॒भगः॑ सु॒दानु॒र् यः । \newline
26. सु॒भग॒ इति॑ सु - भगः॑ । \newline
27. सु॒दानु॒र् यो यः सु॒दानुः॑ सु॒दानु॒र् यस्त्वा᳚ त्वा॒ यः सु॒दानुः॑ सु॒दानु॒र् यस्त्वा᳚ । \newline
28. सु॒दानु॒रिति॑ सु - दानुः॑ । \newline
29. यस्त्वा᳚ त्वा॒ यो यस्त्वा॒ नित्ये॑न॒ नित्ये॑न त्वा॒ यो यस्त्वा॒ नित्ये॑न । \newline
30. त्वा॒ नित्ये॑न॒ नित्ये॑न त्वा त्वा॒ नित्ये॑न ह॒विषा॑ ह॒विषा॒ नित्ये॑न त्वा त्वा॒ नित्ये॑न ह॒विषा᳚ । \newline
31. नित्ये॑न ह॒विषा॑ ह॒विषा॒ नित्ये॑न॒ नित्ये॑न ह॒विषा॒ यो यो ह॒विषा॒ नित्ये॑न॒ नित्ये॑न ह॒विषा॒ यः । \newline
32. ह॒विषा॒ यो यो ह॒विषा॑ ह॒विषा॒ य उ॒क्थै रु॒क्थैर् यो ह॒विषा॑ ह॒विषा॒ य उ॒क्थैः । \newline
33. य उ॒क्थै रु॒क्थैर् यो य उ॒क्थैः । \newline
34. उ॒क्थैरित्यु॒क्थैः । \newline
35. पिप्री॑षति॒ स्वे स्वे पिप्री॑षति॒ पिप्री॑षति॒ स्व आयु॒ष्यायु॑षि॒ स्वे पिप्री॑षति॒ पिप्री॑षति॒ स्व आयु॑षि । \newline
36. स्व आयु॒ष्यायु॑षि॒ स्वे स्व आयु॑षि दुरो॒णे दु॑रो॒ण आयु॑षि॒ स्वे स्व आयु॑षि दुरो॒णे । \newline
37. आयु॑षि दुरो॒णे दु॑रो॒ण आयु॒ष्यायु॑षि दुरो॒णे विश्वा॒ विश्वा॑ दुरो॒ण आयु॒ष्यायु॑षि दुरो॒णे विश्वा᳚ । \newline
38. दु॒रो॒णे विश्वा॒ विश्वा॑ दुरो॒णे दु॑रो॒णे विश्वेदिद् विश्वा॑ दुरो॒णे दु॑रो॒णे विश्वेत् । \newline
39. दु॒रो॒ण इति॑ दुः - ओ॒ने । \newline
40. विश्वेदिद् विश्वा॒ विश्वेद॑स्मा अस्मा॒ इद् विश्वा॒ विश्वेद॑स्मै । \newline
41. इद॑स्मा अस्मा॒ इदिद॑स्मै सु॒दिना॑ सु॒दिना᳚ ऽस्मा॒ इदिद॑स्मै सु॒दिना᳚ । \newline
42. अ॒स्मै॒ सु॒दिना॑ सु॒दिना᳚ ऽस्मा अस्मै सु॒दिना॒ सा सा सु॒दिना᳚ ऽस्मा अस्मै सु॒दिना॒ सा । \newline
43. सु॒दिना॒ सा सा सु॒दिना॑ सु॒दिना॒ सा ऽस॒दस॒थ् सा सु॒दिना॑ सु॒दिना॒ सा ऽस॑त् । \newline
44. सु॒दिनेति॑ सु - दिना᳚ । \newline
45. सा ऽस॒दस॒थ् सा सा ऽस॑दि॒ष्टि रि॒ष्टिरस॒थ् सा सा ऽस॑दि॒ष्टिः । \newline
46. अस॑ दि॒ष्टि रि॒ष्टि रस॒ दस॑ दि॒ष्टिः । \newline
47. इ॒ष्टिरिती॒ष्टिः । \newline
48. अर्चा॑मि ते ते॒ अर्चा॒म्यर्चा॑मि ते सुम॒तिꣳ सु॑म॒तिम् ते॒ अर्चा॒म्यर्चा॑मि ते सुम॒तिम् । \newline
49. ते॒ सु॒म॒तिꣳ सु॑म॒तिम् ते॑ ते सुम॒तिम् घोषि॒ घोषि॑ सुम॒तिम् ते॑ ते सुम॒तिम् घोषि॑ । \newline
50. सु॒म॒तिम् घोषि॒ घोषि॑ सुम॒तिꣳ सु॑म॒तिम् घोष्य॒र्वाग॒र्वाग् घोषि॑ सुम॒तिꣳ सु॑म॒तिम् घोष्य॒र्वाक् । \newline
51. सु॒म॒तिमिति॑ सु - म॒तिम् । \newline
52. घोष्य॒र्वाग॒र्वाग् घोषि॒ घोष्य॒र्वाख् सꣳ स म॒र्वाग् घोषि॒ घोष्य॒र्वाख् सम् । \newline
53. अ॒र्वाख् सꣳ स म॒र्वाग॒र्वाख् सम् ते॑ ते॒ स म॒र्वाग॒र्वाख् सम् ते᳚ । \newline
54. सम् ते॑ ते॒ सꣳ सम् ते॑ वा॒वाता॑ वा॒वाता॑ ते॒ सꣳ सम् ते॑ वा॒वाता᳚ । \newline
55. ते॒ वा॒वाता॑ वा॒वाता॑ ते ते वा॒वाता॑ जरतां जरतां वा॒वाता॑ ते ते वा॒वाता॑ जरतां । \newline
56. वा॒वाता॑ जरतां जरतां वा॒वाता॑ वा॒वाता॑ जरता मि॒य मि॒यम् ज॑रतां वा॒वाता॑ वा॒वाता॑ जरता मि॒यम् । \newline
57. ज॒र॒ता॒ मि॒य मि॒यम् ज॑रतां जरता मि॒यम् गीर् गीरि॒यम् ज॑रतां जरता मि॒यम् गीः । \newline
\pagebreak
\markright{ TS 1.2.14.4  \hfill https://www.vedavms.in \hfill}

\section{ TS 1.2.14.4 }

\textbf{TS 1.2.14.4 } \newline
\textbf{Samhita Paata} \newline

मि॒यंगीः । स्वश्वा᳚स्त्वा सु॒रथा॑ मर्जयेमा॒स्मे क्ष॒त्राणि॑ धारये॒रनु॒ द्यून् ॥ इ॒ह त्वा॒ भूर्या च॑रे॒ दुप॒त्मन् दोषा॑वस्तर् दीदि॒वाꣳस॒मनु॒ द्यून् । क्रीड॑न्तस्त्वा सु॒मन॑सः सपेमा॒भि द्यु॒म्ना त॑स्थि॒वाꣳसो॒ जना॑नां ॥ यस्त्वा॒-स्वश्वः॑ सुहिर॒ण्यो अ॑ग्न उप॒याति॒ वसु॑मता॒ रथे॑न । तस्य॑ त्रा॒ता-भ॑वसि॒ तस्य॒ सखा॒ यस्त॑ आति॒थ्यमा॑नु॒षग् जुजो॑षत् ॥ म॒हो रु॑जामि - [ ] \newline

\textbf{Pada Paata} \newline

इ॒यम् । गीः ॥ स्वश्वा॒ इति॑ सु - अश्वाः᳚ । त्वा॒ । सु॒रथा॒ इति॑ सु - रथाः᳚ । म॒र्ज॒ये॒म॒ । अ॒स्मे इति॑ । क्ष॒त्राणि॑ । धा॒र॒येः॒ । अन्विति॑ । द्यून् ॥ इ॒ह । त्वा॒ । भूरि॑ । एति॑ । च॒रे॒त् । उपेति॑ । त्‍मन्न् । दोषा॑वस्त॒रिति॒ दोषा᳚ - व॒स्तः॒ । दी॒दि॒वाꣳस᳚म् । अन्विति॑ । द्यून् ॥ क्रीड॑न्तः । त्वा॒ । सु॒मन॑स॒ इति॑ सु - मन॑सः । स॒पे॒म॒ । अ॒भीति॑ । द्यु॒म्ना । त॒स्थि॒वाꣳसः॑ । जना॑नाम् ॥ यः । त्वा॒ । स्वश्व॒ इति॑ सु - अश्वः॑ । सु॒हि॒र॒ण्य इति॑ सु - हि॒र॒ण्यः । अ॒ग्ने॒ । उ॒प॒यातीत्यु॑प - याति॑ । वसु॑म॒तेति॒ वसु॑ - म॒ता॒ । रथे॑न ॥ तस्य॑ । त्रा॒ता । भ॒व॒सि॒ । तस्य॑ । सखा᳚ । यः । ते॒ । आ॒ति॒थ्यम् । आ॒नु॒षक् । जुजो॑षत् ॥ म॒हः । रु॒जा॒मि॒ ।  \newline


\textbf{Krama Paata} \newline

इ॒यम् गीः । गीरिति॒ गीः ॥ स्वश्वा᳚स्ता । स्वश्वा॒ इति॑ सु - अश्वाः᳚ । त्वा॒ सु॒रथाः᳚ । सु॒रथा॑ मर्जयेम । सु॒रथा॒ इति॑ सु - रथाः᳚ । म॒र्ज॒ये॒मा॒स्मे । अ॒स्मे क्ष॒त्राणि॑ । अ॒स्मे इत्य॒स्मे । क्ष॒त्राणि॑ धारयेः । धा॒र॒ये॒रनु॑ । अनु॒ द्यून् । द्यूनिति॒ द्यून् ॥ इ॒ह त्वा᳚ । त्वा॒ भूरि॑ । भूर्या । आ च॑रेत् । च॒रे॒दुप॑ । उप॒ त्मन्न् । त्मन् दोषा॑वस्तः । दोषा॑वस्तर् दीदि॒वाꣳस᳚म् । दोषा॑वस्त॒रिति॒ दोषा᳚ - व॒स्तः॒ । दी॒दि॒वाꣳस॒मनु॑ । अनु॒ द्यून् । द्यूनिति॒ द्यून् ॥ क्रीड॑न्तस्त्वा । त्वा॒ सु॒मन॑सः । सु॒मन॑सः सपेम । सु॒मन॑स॒ इति॑ सु - मन॑सः । स॒पे॒मा॒भि । अ॒भि द्यु॒म्ना । द्यु॒म्ना त॑स्थि॒वाꣳसः॑ । त॒स्थि॒वाꣳसो॒ जना॑नाम् । जना॑ना॒मिति॒ जना॑नाम् ॥ यस्त्वा᳚ । त्वा॒ स्वश्वः॑ । स्वश्वः॑ सुहिर॒ण्यः । स्वश्व॒ इति॑ सु - अश्वः॑ । सु॒हि॒र॒ण्यो अ॑ग्ने । सु॒हि॒र॒ण्य इति॑ सु - हि॒र॒ण्यः । अ॒ग्न॒ उ॒प॒याति॑ । उ॒प॒याति॒ वसु॑मता । उ॒प॒यातीत्यु॑प - याति॑ । वसु॑मता॒ रथे॑न । वसु॑म॒तेति॒ वसु॑ - म॒ता॒ । रथे॒नेति॒ रथे॑न ॥ तस्य॑ त्रा॒ता । त्रा॒ता भ॑वसि । भ॒व॒सि॒ तस्य॑ । तस्य॒ सखा᳚ । सखा॒ यः । यस्ते᳚ । त॒ आ॒ति॒थ्यम् । आ॒ति॒थ्यमा॑नु॒षक् । आ॒नु॒षग् जुजो॑षत् । जुजो॑ष॒दिति॒ जुजो॑षत् ॥ म॒हो रु॑जामि । रु॒जा॒मि॒ ब॒न्धुता᳚ \newline

\textbf{Jatai Paata} \newline

1. इ॒यम् गीर् गीरि॒य मि॒यम् गीः । \newline
2. गीरिति॒ गीः । \newline
3. स्वश्वा᳚ स्त्वा त्वा॒ स्वश्वाः॒ स्वश्वा᳚ स्त्वा । \newline
4. स्वश्वा॒ इति॑ सु - अश्वाः᳚ । \newline
5. त्वा॒ सु॒रथाः᳚ सु॒रथा᳚ स्त्वा त्वा सु॒रथाः᳚ । \newline
6. सु॒रथा॑ मर्जयेम मर्जयेम सु॒रथाः᳚ सु॒रथा॑ मर्जयेम । \newline
7. सु॒रथा॒ इति॑ सु - रथाः᳚ । \newline
8. म॒र्ज॒ये॒मा॒स्मे अ॒स्मे म॑र्जयेम मर्जयेमा॒स्मे । \newline
9. अ॒स्मे क्ष॒त्राणि॑ क्ष॒त्राण्य॒स्मे अ॒स्मे क्ष॒त्राणि॑ । \newline
10. अ॒स्मे इत्य॒स्मे । \newline
11. क्ष॒त्राणि॑ धारयेर् धारयेः क्ष॒त्राणि॑ क्ष॒त्राणि॑ धारयेः । \newline
12. धा॒र॒ये॒रन्वनु॑ धारयेर् धारये॒रनु॑ । \newline
13. अनु॒ द्यून् द्यू नन्वनु॒ द्यून् । \newline
14. द्यूनिति॒ द्यून् । \newline
15. इ॒ह त्वा᳚ त्वे॒हे ह त्वा᳚ । \newline
16. त्वा॒ भूरि॒ भूरि॑ त्वा त्वा॒ भूरि॑ । \newline
17. भूर्या भूरि॒ भूर्या । \newline
18. आ च॑रेच् चरे॒दा च॑रेत् । \newline
19. च॒रे॒दुपोप॑ चरेच् चरे॒दुप॑ । \newline
20. उप॒ त्मन् त्मन् नुपोप॒ त्मन्न् । \newline
21. त्मन् दोषा॑वस्त॒र् दोषा॑वस्त॒ स्त्मन् त्मन् दोषा॑वस्तः । \newline
22. दोषा॑वस्तर् दीदि॒वाꣳस॑म् दीदि॒वाꣳस॒म् दोषा॑वस्त॒र् दोषा॑वस्तर् दीदि॒वाꣳस᳚म् । \newline
23. दोषा॑वस्त॒रिति॒ दोषा᳚ - व॒स्तः॒ । \newline
24. दी॒दि॒वाꣳस॒ मन्वनु॑ दीदि॒वाꣳस॑म् दीदि॒वाꣳस॒ मनु॑ । \newline
25. अनु॒ द्यून् द्यू नन्वनु॒ द्यून् । \newline
26. द्यूनिति॒ द्यून् । \newline
27. क्रीड॑न्त स्त्वा त्वा॒ क्रीड॑न्तः॒ क्रीड॑न्त स्त्वा । \newline
28. त्वा॒ सु॒मन॑सः सु॒मन॑स स्त्वा त्वा सु॒मन॑सः । \newline
29. सु॒मन॑सः सपेम सपेम सु॒मन॑सः सु॒मन॑सः सपेम । \newline
30. सु॒मन॑स॒ इति॑ सु - मन॑सः । \newline
31. स॒पे॒मा॒भ्य॑भि ष॑पेम सपेमा॒भि । \newline
32. अ॒भि द्यु॒म्ना द्यु॒म्ना ऽभ्य॑भि द्यु॒म्ना । \newline
33. द्यु॒म्ना त॑स्थि॒वाꣳस॑ स्तस्थि॒वाꣳसो᳚ द्यु॒म्ना द्यु॒म्ना त॑स्थि॒वाꣳसः॑ । \newline
34. त॒स्थि॒वाꣳसो॒ जना॑ना॒म् जना॑नाम् तस्थि॒वाꣳस॑ स्तस्थि॒वाꣳसो॒ जना॑नाम् । \newline
35. जना॑ना॒मिति॒ जना॑नाम् । \newline
36. य स्त्वा᳚ त्वा॒ यो य स्त्वा᳚ । \newline
37. त्वा॒ स्वश्वः॒ स्वश्व॑ स्त्वा त्वा॒ स्वश्वः॑ । \newline
38. स्वश्वः॑ सुहिर॒ण्यः सु॑हिर॒ण्यः स्वश्वः॒ स्वश्वः॑ सुहिर॒ण्यः । \newline
39. स्वश्व॒ इति॑ सु - अश्वः॑ । \newline
40. सु॒हि॒र॒ण्यो अ॑ग्ने अग्ने सुहिर॒ण्यः सु॑हिर॒ण्यो अ॑ग्ने । \newline
41. सु॒हि॒र॒ण्य इति॑ सु - हि॒र॒ण्यः । \newline
42. अ॒ग्न॒ उ॒प॒या त्यु॑प॒यात्य॑ग्ने अग्न उप॒याति॑ । \newline
43. उ॒प॒याति॒ वसु॑मता॒ वसु॑ मतोप॒या त्यु॑प॒याति॒ वसु॑मता । \newline
44. उ॒प॒यातीत्यु॑प - याति॑ । \newline
45. वसु॑मता॒ रथे॑न॒ रथे॑न॒ वसु॑मता॒ वसु॑मता॒ रथे॑न । \newline
46. वसु॑म॒तेति॒ वसु॑ - म॒ता॒ । \newline
47. रथे॒नेति॒ रथे॑न । \newline
48. तस्य॑ त्रा॒ता त्रा॒ता तस्य॒ तस्य॑ त्रा॒ता । \newline
49. त्रा॒ता भ॑वसि भवसि त्रा॒ता त्रा॒ता भ॑वसि । \newline
50. भ॒व॒सि॒ तस्य॒ तस्य॑ भवसि भवसि॒ तस्य॑ । \newline
51. तस्य॒ सखा॒ सखा॒ तस्य॒ तस्य॒ सखा᳚ । \newline
52. सखा॒ यो यः सखा॒ सखा॒ यः । \newline
53. य स्ते॑ ते॒ यो य स्ते᳚ । \newline
54. त॒ आ॒ति॒थ्य मा॑ति॒थ्यम् ते॑ त आति॒थ्यम् । \newline
55. आ॒ति॒थ्य मा॑नु॒ष गा॑नु॒षगा॑ति॒थ्य मा॑ति॒थ्य मा॑नु॒षक् । \newline
56. आ॒नु॒षग् जुजो॑ष॒ज् जुजो॑ष दानु॒ष गा॑नु॒षग् जुजो॑षत् । \newline
57. जुजो॑ष॒दिति॒ जुजो॑षत् । \newline
58. म॒हो रु॑जामि रुजामि म॒हो म॒हो रु॑जामि । \newline
59. रु॒जा॒मि॒ ब॒न्धुता॑ ब॒न्धुता॑ रुजामि रुजामि ब॒न्धुता᳚ । \newline

\textbf{Ghana Paata } \newline

1. इ॒यम् गीर् गीरि॒य मि॒यम् गीः । \newline
2. गीरिति॒ गीः । \newline
3. स्वश्वा᳚स्त्वा त्वा॒ स्वश्वाः॒ स्वश्वा᳚स्त्वा सु॒रथाः᳚ सु॒रथा᳚स्त्वा॒ स्वश्वाः॒ स्वश्वा᳚स्त्वा सु॒रथाः᳚ । \newline
4. स्वश्वा॒ इति॑ सु - अश्वाः᳚ । \newline
5. त्वा॒ सु॒रथाः᳚ सु॒रथा᳚स्त्वा त्वा सु॒रथा॑ मर्जयेम मर्जयेम सु॒रथा᳚स्त्वा त्वा सु॒रथा॑ मर्जयेम । \newline
6. सु॒रथा॑ मर्जयेम मर्जयेम सु॒रथाः᳚ सु॒रथा॑ मर्जयेमा॒स्मे अ॒स्मे म॑र्जयेम सु॒रथाः᳚ सु॒रथा॑ मर्जयेमा॒स्मे । \newline
7. सु॒रथा॒ इति॑ सु - रथाः᳚ । \newline
8. म॒र्ज॒ये॒मा॒स्मे अ॒स्मे म॑र्जयेम मर्जयेमा॒स्मे क्ष॒त्राणि॑ क्ष॒त्राण्य॒स्मे म॑र्जयेम मर्जयेमा॒स्मे क्ष॒त्राणि॑ । \newline
9. अ॒स्मे क्ष॒त्राणि॑ क्ष॒त्राण्य॒स्मे अ॒स्मे क्ष॒त्राणि॑ धारयेर् धारयेः क्ष॒त्राण्य॒स्मे अ॒स्मे क्ष॒त्राणि॑ धारयेः । \newline
10. अ॒स्मे इत्य॒स्मे । \newline
11. क्ष॒त्राणि॑ धारयेर् धारयेः क्ष॒त्राणि॑ क्ष॒त्राणि॑ धारये॒रन्वनु॑ धारयेः क्ष॒त्राणि॑ क्ष॒त्राणि॑ धारये॒रनु॑ । \newline
12. धा॒र॒ये॒रन्वनु॑ धारयेर् धारये॒रनु॒ द्यून् द्यू ननु॑ धारयेर् धारये॒रनु॒ द्यून् । \newline
13. अनु॒ द्यून् द्यू नन्वनु॒ द्यून् । \newline
14. द्यूनिति॒ द्यून् । \newline
15. इ॒ह त्वा᳚ त्वे॒हे ह त्वा॒ भूरि॒ भूरि॑ त्वे॒हे ह त्वा॒ भूरि॑ । \newline
16. त्वा॒ भूरि॒ भूरि॑ त्वा त्वा॒ भूर्या भूरि॑ त्वा त्वा॒ भूर्या । \newline
17. भूर्या भूरि॒ भूर्या च॑रेच् चरे॒दा भूरि॒ भूर्या च॑रेत् । \newline
18. आ च॑रेच् चरे॒दा च॑रे॒दुपोप॑ चरे॒दा च॑रे॒दुप॑ । \newline
19. च॒रे॒दुपोप॑ चरेच् चरे॒दुप॒ त्मन् त्मन् नुप॑ चरेच् चरे॒दुप॒ त्मन्न् । \newline
20. उप॒ त्मन् त्मन् नुपोप॒ त्मन् दोषा॑वस्त॒र् दोषा॑वस्त॒स्त्मन् नुपोप॒ त्मन् दोषा॑वस्तः । \newline
21. त्मन् दोषा॑वस्त॒र् दोषा॑वस्त॒स्त्मन् त्मन् दोषा॑वस्तर् दीदि॒वाꣳस॑म् दीदि॒वाꣳस॒म् दोषा॑वस्त॒स्त्मन् त्मन् दोषा॑वस्तर् दीदि॒वाꣳस᳚म् । \newline
22. दोषा॑वस्तर् दीदि॒वाꣳस॑म् दीदि॒वाꣳस॒म् दोषा॑वस्त॒र् दोषा॑वस्तर् दीदि॒वाꣳस॒ मन्वनु॑ दीदि॒वाꣳस॒म् दोषा॑वस्त॒र् दोषा॑वस्तर् दीदि॒वाꣳस॒ मनु॑ । \newline
23. दोषा॑वस्त॒रिति॒ दोषा᳚ - व॒स्तः॒ । \newline
24. दी॒दि॒वाꣳस॒ मन्वनु॑ दीदि॒वाꣳस॑म् दीदि॒वाꣳस॒ मनु॒ द्यून् द्यू ननु॑ दीदि॒वाꣳस॑म् दीदि॒वाꣳस॒ मनु॒ द्यून् । \newline
25. अनु॒ द्यून् द्यू नन्वनु॒ द्यून् । \newline
26. द्यूनिति॒ द्यून् । \newline
27. क्रीड॑न्तस्त्वा त्वा॒ क्रीड॑न्तः॒ क्रीड॑न्तस्त्वा सु॒मन॑सः सु॒मन॑सस्त्वा॒ क्रीड॑न्तः॒ क्रीड॑न्तस्त्वा सु॒मन॑सः । \newline
28. त्वा॒ सु॒मन॑सः सु॒मन॑सस्त्वा त्वा सु॒मन॑सः सपेम सपेम सु॒मन॑सस्त्वा त्वा सु॒मन॑सः सपेम । \newline
29. सु॒मन॑सः सपेम सपेम सु॒मन॑सः सु॒मन॑सः सपेमा॒भ्य॑भि ष॑पेम सु॒मन॑सः सु॒मन॑सः सपेमा॒भि । \newline
30. सु॒मन॑स॒ इति॑ सु - मन॑सः । \newline
31. स॒पे॒मा॒भ्य॑भि ष॑पेम सपेमा॒भि द्यु॒म्ना द्यु॒म्ना ऽभि ष॑पेम सपेमा॒भि द्यु॒म्ना । \newline
32. अ॒भि द्यु॒म्ना द्यु॒म्ना ऽभ्य॑भि द्यु॒म्ना त॑स्थि॒वाꣳस॑ स्तस्थि॒वाꣳसो᳚ द्यु॒म्ना ऽभ्य॑भि द्यु॒म्ना त॑स्थि॒वाꣳसः॑ । \newline
33. द्यु॒म्ना त॑स्थि॒वाꣳस॑ स्तस्थि॒वाꣳसो᳚ द्यु॒म्ना द्यु॒म्ना त॑स्थि॒वाꣳसो॒ जना॑ना॒म् जना॑नाम् तस्थि॒वाꣳसो᳚ द्यु॒म्ना द्यु॒म्ना त॑स्थि॒वाꣳसो॒ जना॑नाम् । \newline
34. त॒स्थि॒वाꣳसो॒ जना॑ना॒म् जना॑नाम् तस्थि॒वाꣳस॑ स्तस्थि॒वाꣳसो॒ जना॑नाम् । \newline
35. जना॑ना॒मिति॒ जना॑नाम् । \newline
36. यस्त्वा᳚ त्वा॒ यो यस्त्वा॒ स्वश्वः॒ स्वश्व॑स्त्वा॒ यो यस्त्वा॒ स्वश्वः॑ । \newline
37. त्वा॒ स्वश्वः॒ स्वश्व॑स्त्वा त्वा॒ स्वश्वः॑ सुहिर॒ण्यः सु॑हिर॒ण्यः स्वश्व॑स्त्वा त्वा॒ स्वश्वः॑ सुहिर॒ण्यः । \newline
38. स्वश्वः॑ सुहिर॒ण्यः सु॑हिर॒ण्यः स्वश्वः॒ स्वश्वः॑ सुहिर॒ण्यो अ॑ग्ने अग्ने सुहिर॒ण्यः स्वश्वः॒ स्वश्वः॑ सुहिर॒ण्यो अ॑ग्ने । \newline
39. स्वश्व॒ इति॑ सु - अश्वः॑ । \newline
40. सु॒हि॒र॒ण्यो अ॑ग्ने अग्ने सुहिर॒ण्यः सु॑हिर॒ण्यो अ॑ग्न उप॒यात्यु॑प॒यात्य॑ग्ने सुहिर॒ण्यः सु॑हिर॒ण्यो अ॑ग्न उप॒याति॑ । \newline
41. सु॒हि॒र॒ण्य इति॑ सु - हि॒र॒ण्यः । \newline
42. अ॒ग्न॒ उ॒प॒यात्यु॑प॒यात्य॑ग्ने अग्न उप॒याति॒ वसु॑मता॒ वसु॑मतोप॒यात्य॑ग्ने अग्न उप॒याति॒ वसु॑मता । \newline
43. उ॒प॒याति॒ वसु॑मता॒ वसु॑मतोप॒यात्यु॑प॒याति॒ वसु॑मता॒ रथे॑न॒ रथे॑न॒ वसु॑मतोप॒यात्यु॑प॒याति॒ वसु॑मता॒ रथे॑न । \newline
44. उ॒प॒यातीत्यु॑प - याति॑ । \newline
45. वसु॑मता॒ रथे॑न॒ रथे॑न॒ वसु॑मता॒ वसु॑मता॒ रथे॑न । \newline
46. वसु॑म॒तेति॒ वसु॑ - म॒ता॒ । \newline
47. रथे॒नेति॒ रथे॑न । \newline
48. तस्य॑ त्रा॒ता त्रा॒ता तस्य॒ तस्य॑ त्रा॒ता भ॑वसि भवसि त्रा॒ता तस्य॒ तस्य॑ त्रा॒ता भ॑वसि । \newline
49. त्रा॒ता भ॑वसि भवसि त्रा॒ता त्रा॒ता भ॑वसि॒ तस्य॒ तस्य॑ भवसि त्रा॒ता त्रा॒ता भ॑वसि॒ तस्य॑ । \newline
50. भ॒व॒सि॒ तस्य॒ तस्य॑ भवसि भवसि॒ तस्य॒ सखा॒ सखा॒ तस्य॑ भवसि भवसि॒ तस्य॒ सखा᳚ । \newline
51. तस्य॒ सखा॒ सखा॒ तस्य॒ तस्य॒ सखा॒ यो यः सखा॒ तस्य॒ तस्य॒ सखा॒ यः । \newline
52. सखा॒ यो यः सखा॒ सखा॒ यस्ते॑ ते॒ यः सखा॒ सखा॒ यस्ते᳚ । \newline
53. यस्ते॑ ते॒ यो यस्त॑ आति॒थ्य मा॑ति॒थ्यम् ते॒ यो यस्त॑ आति॒थ्यम् । \newline
54. त॒ आ॒ति॒थ्य मा॑ति॒थ्यम् ते॑ त आति॒थ्य मा॑नु॒षगा॑नु॒षगा॑ति॒थ्यम् ते॑ त आति॒थ्य मा॑नु॒षक् । \newline
55. आ॒ति॒थ्य मा॑नु॒षगा॑नु॒षगा॑ति॒थ्य मा॑ति॒थ्य मा॑नु॒षग् जुजो॑ष॒ज् जुजो॑षदानु॒षगा॑ति॒थ्य मा॑ति॒थ्य मा॑नु॒षग् जुजो॑षत् । \newline
56. आ॒नु॒षग् जुजो॑ष॒ज् जुजो॑षदानु॒षगा॑नु॒षग् जुजो॑षत् । \newline
57. जुजो॑ष॒दिति॒ जुजो॑षत् । \newline
58. म॒हो रु॑जामि रुजामि म॒हो म॒हो रु॑जामि ब॒न्धुता॑ ब॒न्धुता॑ रुजामि म॒हो म॒हो रु॑जामि ब॒न्धुता᳚ । \newline
59. रु॒जा॒मि॒ ब॒न्धुता॑ ब॒न्धुता॑ रुजामि रुजामि ब॒न्धुता॒ वचो॑भि॒र् वचो॑भिर् ब॒न्धुता॑ रुजामि रुजामि ब॒न्धुता॒ वचो॑भिः । \newline
\pagebreak
\markright{ TS 1.2.14.5  \hfill https://www.vedavms.in \hfill}

\section{ TS 1.2.14.5 }

\textbf{TS 1.2.14.5 } \newline
\textbf{Samhita Paata} \newline

ब॒न्धुता॒ वचो॑भि॒स्तन्मा॑ पि॒तुर्गोत॑मा॒द-न्वि॑याय । त्वन्नो॑ अ॒स्य वच॑स-श्चिकिद्धि॒ होत॑र्यविष्ठ सुक्रतो॒ दमू॑नाः ॥ अस्व॑प्नज स्त॒रण॑यः सु॒शेवा॒ अत॑न्द्रासोऽवृ॒का अश्र॑मिष्ठाः । ते पा॒यवः॑ स॒द्ध्रिय॑ञ्चो नि॒षद्याऽग्ने॒ तव॑नः पान्त्वमूर ॥ ये पा॒यवो॑ मामते॒यं ते॑ अग्ने॒ पश्य॑न्तो अ॒न्धं दु॑रि॒तादर॑क्षन् । र॒रक्ष॒तान्थ् सु॒कृतो॑ वि॒श्ववे॑दा॒ दिफ्स॑न्त॒ इद्रि॒पवो॒ ना ह॑ - [ ] \newline

\textbf{Pada Paata} \newline

ब॒न्धुता᳚ । वचो॑भि॒रिति॒ वचः॑ - भिः॒ । तत् । मा॒ । पि॒तुः । गोत॑मात् । अन्विति॑ । इ॒या॒य॒ ॥ त्वम् । नः॒ । अ॒स्य । वच॑सः । चि॒कि॒द्धि॒ । होतः॑ । य॒वि॒ष्ठ॒ । सु॒क्र॒तो॒ इति॑ सु - क्र॒तो॒ । दमू॑नाः ॥ अस्व॑प्नज॒ इत्यस्व॑प्न - जः॒ । त॒रण॑यः । सु॒शेवा॒ इति॑ सु-शेवाः᳚ । अत॑न्द्रासः । अ॒वृ॒काः । अश्र॑मिष्ठाः ॥ ते । पा॒यवः॑ । स॒द्ध्रिय॑ञ्चः । नि॒षद्येति॑ नि - सद्य॑ । अग्ने᳚ । तव॑ । नः॒ । पा॒न्तु॒ । अ॒मू॒र॒ ॥ ये । पा॒यवः॑ । मा॒म॒ते॒यम् । ते॒ । अ॒ग्ने॒ । पश्य॑न्तः । अ॒न्धम् । दु॒रि॒तादिति॑ दुः - इ॒तात् । अर॑क्षन्न् ॥ र॒रक्ष॑ । तान् । सु॒कृत॒ इति॑ सु - कृतः॑ । वि॒श्ववे॑दा॒ इति॑ वि॒श्व - वे॒दाः॒ । दिफ्स॑न्तः । इत् । रि॒पवः॑ । न । ह॒ ।  \newline


\textbf{Krama Paata} \newline

ब॒न्धुता॒ वचो॑भिः । वचो॑भि॒स्तत् । वचो॑भि॒रिति॒ वचः॑ - भिः॒ । तन्मा᳚ । मा॒ पि॒तुः । पि॒तुर् गोत॑मात् । गोत॑मा॒दनु॑ । अन्वि॑याय । इ॒या॒येती॑याय ॥ त्वन्नः॑ । नो॒ अ॒स्य । अ॒स्य वच॑सः । वच॑सश्चिकिद्धि । चि॒कि॒द्धि॒ होतः॑ । होत॑र् यविष्ठ । य॒वि॒ष्ठ॒ सु॒क्र॒तो॒ । सु॒क्र॒तो॒ दमू॑नाः । सु॒क्र॒तो॒ इति॑ सु - क्र॒तो॒ । दमू॑ना॒ इति॒ दमू॑नाः ॥ अस्व॑प्नज स्त॒रण॑यः । अस्व॑प्नज॒ इत्यस्व॑प्न - जः॒ । त॒रण॑यः सु॒शेवाः᳚ । सु॒शेवा॒ अत॑न्द्रासः । सु॒शेवा॒ इति॑ सु - शेवाः᳚ । अत॑न्द्रासोऽवृ॒काः । अ॒वृ॒का अश्र॑मिष्ठाः । अश्र॑मिष्ठा॒ इत्यश्र॑मिष्ठाः ॥ ते पा॒यवः॑ । पा॒यवः॑ स॒द्ध्रिय॑ञ्चः । स॒द्ध्रिय॑ञ्चो नि॒षद्य॑ । नि॒षद्याग्ने᳚ । नि॒षद्येति॑ नि - सद्य॑ । अग्ने॒ तव॑ । तव॑ नः । नः॒ पा॒न्तु॒ । पा॒न्त्व॒मू॒र॒ । अ॒मू॒रेत्य॑मूर ॥ ये पा॒यवः॑ । पा॒यवो॑ मामते॒यम् । मा॒म॒ते॒यम् ते᳚ । ते॒ अ॒ग्ने॒ । अ॒ग्ने॒ पश्य॑न्तः । पश्य॑न्तो अ॒न्धम् । अ॒न्धम् दु॑रि॒तात् । दु॒रि॒तादर॑क्षन्न् । दु॒रि॒तादिति॑ दुः - इ॒तात् । अर॑क्ष॒न्नित्यर॑क्षन्न् ॥ र॒रक्ष॒ तान् । तान्थ् सु॒कृतः॑ । सु॒कृतो॑ वि॒श्ववे॑दाः । सु॒कृत॒ इति॑ सु - कृतः॑ । वि॒श्ववे॑दा॒ दिफ्स॑न्तः । वि॒श्ववे॑दा॒ इति॑ वि॒श्व - वे॒दाः॒ । दिफ्स॑न्त॒ इत् । इद् रि॒पवः॑ । रि॒पवो॒ न । ना ह॑ । ह॒ दे॒भुः॒ \newline

\textbf{Jatai Paata} \newline

1. ब॒न्धुता॒ वचो॑भि॒र् वचो॑भिर् ब॒न्धुता॑ ब॒न्धुता॒ वचो॑भिः । \newline
2. वचो॑भि॒ स्तत् तद् वचो॑भि॒र् वचो॑भि॒ स्तत् । \newline
3. वचो॑भि॒रिति॒ वचः॑ - भिः॒ । \newline
4. तन् मा॑ मा॒ तत् तन् मा᳚ । \newline
5. मा॒ पि॒तुः पि॒तुर् मा॑ मा पि॒तुः । \newline
6. पि॒तुर् गोत॑मा॒द् गोत॑मात् पि॒तुः पि॒तुर् गोत॑मात् । \newline
7. गोत॑मा॒ दन्वनु॒ गोत॑मा॒द् गोत॑मा॒दनु॑ । \newline
8. अन्वि॑याये या॒या न्वन् वि॑याय । \newline
9. इ॒या॒येती॑याय । \newline
10. त्वम् नो॑ न॒ स्त्वम् त्वम् नः॑ । \newline
11. नो॒ अ॒स्यास्य नो॑ नो अ॒स्य । \newline
12. अ॒स्य वच॑सो॒ वच॑सो अ॒स्यास्य वच॑सः । \newline
13. वच॑स श्चिकिद्धि चिकिद्धि॒ वच॑सो॒ वच॑स श्चिकिद्धि । \newline
14. चि॒कि॒द्धि॒ होत॒र्॒. होत॑ श्चिकिद्धि चिकिद्धि॒ होतः॑ । \newline
15. होत॑र् यविष्ठ यविष्ठ॒ होत॒र्॒. होत॑र् यविष्ठ । \newline
16. य॒वि॒ष्ठ॒ सु॒क्र॒तो॒ सु॒क्र॒तो॒ य॒वि॒ष्ठ॒ य॒वि॒ष्ठ॒ सु॒क्र॒तो॒ । \newline
17. सु॒क्र॒तो॒ दमू॑ना॒ दमू॑नाः सुक्रतो सुक्रतो॒ दमू॑नाः । \newline
18. सु॒क्र॒तो॒ इति॑ सु - क्र॒तो॒ । \newline
19. दमू॑ना॒ इति॒ दमू॑नाः । \newline
20. अस्व॑प्नज स्त॒रण॑य स्त॒रण॑यो॒ अस्व॑प्नजो॒ अस्व॑प्नज स्त॒रण॑यः । \newline
21. अस्व॑प्नज॒ इत्यस्व॑प्न - जः॒ । \newline
22. त॒रण॑यः सु॒शेवाः᳚ सु॒शेवा᳚ स्त॒रण॑य स्त॒रण॑यः सु॒शेवाः᳚ । \newline
23. सु॒शेवा॒ अत॑न्द्रासो॒ अत॑न्द्रासः सु॒शेवाः᳚ सु॒शेवा॒ अत॑न्द्रासः । \newline
24. सु॒शेवा॒ इति॑ सु - शेवाः᳚ । \newline
25. अत॑न्द्रासो ऽवृ॒का अ॑वृ॒का अत॑न्द्रासो॒ अत॑न्द्रासो ऽवृ॒काः । \newline
26. अ॒वृ॒का अश्र॑मिष्ठा॒ अश्र॑मिष्ठा अवृ॒का अ॑वृ॒का अश्र॑मिष्ठाः । \newline
27. अश्र॑मिष्ठा॒ इत्यश्र॑मिष्ठाः । \newline
28. ते पा॒यवः॑ पा॒यव॒ स्ते ते पा॒यवः॑ । \newline
29. पा॒यवः॑ स॒द्ध्रिय॑ञ्चः स॒द्ध्रिय॑ञ्चः पा॒यवः॑ पा॒यवः॑ स॒द्ध्रिय॑ञ्चः । \newline
30. स॒द्ध्रिय॑ञ्चो नि॒षद्य॑ नि॒षद्य॑ स॒द्ध्रिय॑ञ्चः स॒द्ध्रिय॑ञ्चो नि॒षद्य॑ । \newline
31. नि॒षद्याग्ने ऽग्ने॑ नि॒षद्य॑ नि॒षद्याग्ने᳚ । \newline
32. नि॒षद्येति॑ नि - सद्य॑ । \newline
33. अग्ने॒ तव॒ तवाग्ने ऽग्ने॒ तव॑ । \newline
34. तव॑ नो न॒स्तव॒ तव॑ नः । \newline
35. नः॒ पा॒न्तु॒ पा॒न्तु॒ नो॒ नः॒ पा॒न्तु॒ । \newline
36. पा॒न्त्व॒मू॒रा॒मू॒र॒ पा॒न्तु॒ पा॒न्त्व॒मू॒र॒ । \newline
37. अ॒मू॒रेत्य॑मूर । \newline
38. ये पा॒यवः॑ पा॒यवो॒ ये ये पा॒यवः॑ । \newline
39. पा॒यवो॑ मामते॒यम् मा॑मते॒यम् पा॒यवः॑ पा॒यवो॑ मामते॒यम् । \newline
40. मा॒म॒ते॒यम् ते॑ ते मामते॒यम् मा॑मते॒यम् ते᳚ । \newline
41. ते॒ अ॒ग्ने॒ अ॒ग्ने॒ ते॒ ते॒ अ॒ग्ने॒ । \newline
42. अ॒ग्ने॒ पश्य॑न्तः॒ पश्य॑न्तो अग्ने अग्ने॒ पश्य॑न्तः । \newline
43. पश्य॑न्तो अ॒न्ध म॒न्धम् पश्य॑न्तः॒ पश्य॑न्तो अ॒न्धम् । \newline
44. अ॒न्धम् दु॑रि॒ताद् दु॑रि॒ता द॒न्ध म॒न्धम् दु॑रि॒तात् । \newline
45. दु॒रि॒तादर॑क्ष॒न् नर॑क्षन् दुरि॒ताद् दु॑रि॒ता दर॑क्षन्न् । \newline
46. दु॒रि॒तादिति॑ दुः - इ॒तात् । \newline
47. अर॑क्ष॒न्नित्यर॑क्षन्न् । \newline
48. र॒रक्ष॒ ताꣳ स्तान् र॒रक्ष॑ र॒रक्ष॒ तान् । \newline
49. तान् थ्सु॒कृतः॑ सु॒कृत॒ स्ताꣳ स्तान् थ्सु॒कृतः॑ । \newline
50. सु॒कृतो॑ वि॒श्ववे॑दा वि॒श्ववे॑दाः सु॒कृतः॑ सु॒कृतो॑ वि॒श्ववे॑दाः । \newline
51. सु॒कृत॒ इति॑ सु - कृतः॑ । \newline
52. वि॒श्ववे॑दा॒ दिफ्स॑न्तो॒ दिफ्स॑न्तो वि॒श्ववे॑दा वि॒श्ववे॑दा॒ दिफ्स॑न्तः । \newline
53. वि॒श्ववे॑दा॒ इति॑ वि॒श्व - वे॒दाः॒ । \newline
54. दिफ्स॑न्त॒ इदिद् दिफ्स॑न्तो॒ दिफ्स॑न्त॒ इत् । \newline
55. इद् रि॒पवो॑ रि॒पव॒ इदिद् रि॒पवः॑ । \newline
56. रि॒पवो॒ न न रि॒पवो॑ रि॒पवो॒ न । \newline
57. ना ह॑ ह॒ न ना ह॑ । \newline
58. ह॒ दे॒भु॒र् दे॒भु॒र्॒. ह॒ ह॒ दे॒भुः॒ । \newline

\textbf{Ghana Paata } \newline

1. ब॒न्धुता॒ वचो॑भि॒र् वचो॑भिर् ब॒न्धुता॑ ब॒न्धुता॒ वचो॑भि॒स्तत् तद् वचो॑भिर् ब॒न्धुता॑ ब॒न्धुता॒ वचो॑भि॒स्तत् । \newline
2. वचो॑भि॒स्तत् तद् वचो॑भि॒र् वचो॑भि॒स्तन् मा॑ मा॒ तद् वचो॑भि॒र् वचो॑भि॒स्तन् मा᳚ । \newline
3. वचो॑भि॒रिति॒ वचः॑ - भिः॒ । \newline
4. तन् मा॑ मा॒ तत् तन् मा॑ पि॒तुः पि॒तुर् मा॒ तत् तन् मा॑ पि॒तुः । \newline
5. मा॒ पि॒तुः पि॒तुर् मा॑ मा पि॒तुर् गोत॑मा॒द् गोत॑मात् पि॒तुर् मा॑ मा पि॒तुर् गोत॑मात् । \newline
6. पि॒तुर् गोत॑मा॒द् गोत॑मात् पि॒तुः पि॒तुर् गोत॑मा॒दन्वनु॒ गोत॑मात् पि॒तुः पि॒तुर् गोत॑मा॒दनु॑ । \newline
7. गोत॑मा॒दन्वनु॒ गोत॑मा॒द् गोत॑मा॒दन्वि॑याये या॒यानु॒ गोत॑मा॒द् गोत॑मा॒दन्वि॑याय । \newline
8. अन्वि॑याये या॒यान्वन्वि॑याय । \newline
9. इ॒या॒येती॑याय । \newline
10. त्वन्नो॑ न॒स्त्वम् त्वन्नो॑ अ॒स्यास्य न॒स्त्वम् त्वन्नो॑ अ॒स्य । \newline
11. नो॒ अ॒स्यास्य नो॑ नो अ॒स्य वच॑सो॒ वच॑सो अ॒स्य नो॑ नो अ॒स्य वच॑सः । \newline
12. अ॒स्य वच॑सो॒ वच॑सो अ॒स्यास्य वच॑सश्चिकिद्धि चिकिद्धि॒ वच॑सो अ॒स्यास्य वच॑सश्चिकिद्धि । \newline
13. वच॑सश्चिकिद्धि चिकिद्धि॒ वच॑सो॒ वच॑सश्चिकिद्धि॒ होत॒र्॒. होत॑श्चिकिद्धि॒ वच॑सो॒ वच॑स श्चिकिद्धि॒ होतः॑ । \newline
14. चि॒कि॒द्धि॒ होत॒र्॒. होत॑श्चिकिद्धि चिकिद्धि॒ होत॑र् यविष्ठ यविष्ठ॒ होत॑श्चिकिद्धि चिकिद्धि॒ होत॑र् यविष्ठ । \newline
15. होत॑र् यविष्ठ यविष्ठ॒ होत॒र्॒. होत॑र् यविष्ठ सुक्रतो सुक्रतो यविष्ठ॒ होत॒र्॒. होत॑र् यविष्ठ सुक्रतो । \newline
16. य॒वि॒ष्ठ॒ सु॒क्र॒तो॒ सु॒क्र॒तो॒ य॒वि॒ष्ठ॒ य॒वि॒ष्ठ॒ सु॒क्र॒तो॒ दमू॑ना॒ दमू॑नाः सुक्रतो यविष्ठ यविष्ठ सुक्रतो॒ दमू॑नाः । \newline
17. सु॒क्र॒तो॒ दमू॑ना॒ दमू॑नाः सुक्रतो सुक्रतो॒ दमू॑नाः । \newline
18. सु॒क्र॒तो॒ इति॑ सु - क्र॒तो॒ । \newline
19. दमू॑ना॒ इति॒ दमू॑नाः । \newline
20. अस्व॑प्नज स्त॒रण॑य स्त॒रण॑यो॒ अस्व॑प्नजो॒ अस्व॑प्नज स्त॒रण॑यः सु॒शेवाः᳚ सु॒शेवा᳚ स्त॒रण॑यो॒ अस्व॑प्नजो॒ अस्व॑प्नज स्त॒रण॑यः सु॒शेवाः᳚ । \newline
21. अस्व॑प्नज॒ इत्यस्व॑प्न - जः॒ । \newline
22. त॒रण॑यः सु॒शेवाः᳚ सु॒शेवा᳚ स्त॒रण॑य स्त॒रण॑यः सु॒शेवा॒ अत॑न्द्रासो॒ अत॑न्द्रासः सु॒शेवा᳚ स्त॒रण॑य स्त॒रण॑यः सु॒शेवा॒ अत॑न्द्रासः । \newline
23. सु॒शेवा॒ अत॑न्द्रासो॒ अत॑न्द्रासः सु॒शेवाः᳚ सु॒शेवा॒ अत॑न्द्रासो ऽवृ॒का अ॑वृ॒का अत॑न्द्रासः सु॒शेवाः᳚ सु॒शेवा॒ अत॑न्द्रासो ऽवृ॒काः । \newline
24. सु॒शेवा॒ इति॑ सु - शेवाः᳚ । \newline
25. अत॑न्द्रासो ऽवृ॒का अ॑वृ॒का अत॑न्द्रासो॒ अत॑न्द्रासो ऽवृ॒का अश्र॑मिष्ठा॒ अश्र॑मिष्ठा अवृ॒का अत॑न्द्रासो॒ अत॑न्द्रासो ऽवृ॒का अश्र॑मिष्ठाः । \newline
26. अ॒वृ॒का अश्र॑मिष्ठा॒ अश्र॑मिष्ठा अवृ॒का अ॑वृ॒का अश्र॑मिष्ठाः । \newline
27. अश्र॑मिष्ठा॒ इत्यश्र॑मिष्ठाः । \newline
28. ते पा॒यवः॑ पा॒यव॒स्ते ते पा॒यवः॑ स॒द्ध्रिय॑ञ्चः स॒द्ध्रिय॑ञ्चः पा॒यव॒स्ते ते पा॒यवः॑ स॒द्ध्रिय॑ञ्चः । \newline
29. पा॒यवः॑ स॒द्ध्रिय॑ञ्चः स॒द्ध्रिय॑ञ्चः पा॒यवः॑ पा॒यवः॑ स॒द्ध्रिय॑ञ्चो नि॒षद्य॑ नि॒षद्य॑ स॒द्ध्रिय॑ञ्चः पा॒यवः॑ पा॒यवः॑ स॒द्ध्रिय॑ञ्चो नि॒षद्य॑ । \newline
30. स॒द्ध्रिय॑ञ्चो नि॒षद्य॑ नि॒षद्य॑ स॒द्ध्रिय॑ञ्चः स॒द्ध्रिय॑ञ्चो नि॒षद्याग्ने ऽग्ने॑ नि॒षद्य॑ स॒द्ध्रिय॑ञ्चः स॒द्ध्रिय॑ञ्चो नि॒षद्याग्ने᳚ । \newline
31. नि॒षद्याग्ने ऽग्ने॑ नि॒षद्य॑ नि॒षद्याग्ने॒ तव॒ तवाग्ने॑ नि॒षद्य॑ नि॒षद्याग्ने॒ तव॑ । \newline
32. नि॒षद्येति॑ नि - सद्य॑ । \newline
33. अग्ने॒ तव॒ तवाग्ने ऽग्ने॒ तव॑ नो न॒स्तवाग्ने ऽग्ने॒ तव॑ नः । \newline
34. तव॑ नो न॒स्तव॒ तव॑ नः पान्तु पान्तु न॒स्तव॒ तव॑ नः पान्तु । \newline
35. नः॒ पा॒न्तु॒ पा॒न्तु॒ नो॒ नः॒ पा॒न्त्व॒मू॒रा॒मू॒र॒ पा॒न्तु॒ नो॒ नः॒ पा॒न्त्व॒मू॒र॒ । \newline
36. पा॒न्त्व॒मू॒रा॒मू॒र॒ पा॒न्तु॒ पा॒न्त्व॒मू॒र॒ । \newline
37. अ॒मू॒रेत्य॑मूर । \newline
38. ये पा॒यवः॑ पा॒यवो॒ ये ये पा॒यवो॑ मामते॒यम् मा॑मते॒यम् पा॒यवो॒ ये ये पा॒यवो॑ मामते॒यम् । \newline
39. पा॒यवो॑ मामते॒यम् मा॑मते॒यम् पा॒यवः॑ पा॒यवो॑ मामते॒यम् ते॑ ते मामते॒यम् पा॒यवः॑ पा॒यवो॑ मामते॒यम् ते᳚ । \newline
40. मा॒म॒ते॒यम् ते॑ ते मामते॒यम् मा॑मते॒यम् ते॑ अग्ने अग्ने ते मामते॒यम् मा॑मते॒यम् ते॑ अग्ने । \newline
41. ते॒ अ॒ग्ने॒ अ॒ग्ने॒ ते॒ ते॒ अ॒ग्ने॒ पश्य॑न्तः॒ पश्य॑न्तो अग्ने ते ते अग्ने॒ पश्य॑न्तः । \newline
42. अ॒ग्ने॒ पश्य॑न्तः॒ पश्य॑न्तो अग्ने अग्ने॒ पश्य॑न्तो अ॒न्ध म॒न्धम् पश्य॑न्तो अग्ने अग्ने॒ पश्य॑न्तो अ॒न्धम् । \newline
43. पश्य॑न्तो अ॒न्ध म॒न्धम् पश्य॑न्तः॒ पश्य॑न्तो अ॒न्धम् दु॑रि॒ताद् दु॑रि॒ताद॒न्धम् पश्य॑न्तः॒ पश्य॑न्तो अ॒न्धम् दु॑रि॒तात् । \newline
44. अ॒न्धम् दु॑रि॒ताद् दु॑रि॒ताद॒न्ध म॒न्धम् दु॑रि॒तादर॑क्ष॒न् नर॑क्षन् दुरि॒ताद॒न्ध म॒न्धम् दु॑रि॒तादर॑क्षन्न् । \newline
45. दु॒रि॒ता दर॑क्ष॒न् नर॑क्षन् दुरि॒ताद् दु॑रि॒ता दर॑क्षन्न् । \newline
46. दु॒रि॒तादिति॑ दुः - इ॒तात् । \newline
47. अर॑क्ष॒न्नित्यर॑क्षन्न् । \newline
48. र॒रक्ष॒ ताꣳ स्तान् र॒रक्ष॑ र॒रक्ष॒ तान् थ्सु॒कृतः॑ सु॒कृत॒स्तान् र॒रक्ष॑ र॒रक्ष॒ तान् थ्सु॒कृतः॑ । \newline
49. तान् थ्सु॒कृतः॑ सु॒कृत॒ स्ताꣳ स्तान् थ्सु॒कृतो॑ वि॒श्ववे॑दा वि॒श्ववे॑दाः सु॒कृत॒ स्ताꣳ स्तान् थ्सु॒कृतो॑ वि॒श्ववे॑दाः । \newline
50. सु॒कृतो॑ वि॒श्ववे॑दा वि॒श्ववे॑दाः सु॒कृतः॑ सु॒कृतो॑ वि॒श्ववे॑दा॒ दिफ्स॑न्तो॒ दिफ्स॑न्तो वि॒श्ववे॑दाः सु॒कृतः॑ सु॒कृतो॑ वि॒श्ववे॑दा॒ दिफ्स॑न्तः । \newline
51. सु॒कृत॒ इति॑ सु - कृतः॑ । \newline
52. वि॒श्ववे॑दा॒ दिफ्स॑न्तो॒ दिफ्स॑न्तो वि॒श्ववे॑दा वि॒श्ववे॑दा॒ दिफ्स॑न्त॒ इदिद् दिफ्स॑न्तो वि॒श्ववे॑दा वि॒श्ववे॑दा॒ दिफ्स॑न्त॒ इत् । \newline
53. वि॒श्ववे॑दा॒ इति॑ वि॒श्व - वे॒दाः॒ । \newline
54. दिफ्स॑न्त॒ इदिद् दिफ्स॑न्तो॒ दिफ्स॑न्त॒ इद् रि॒पवो॑ रि॒पव॒ इद् दिफ्स॑न्तो॒ दिफ्स॑न्त॒ इद् रि॒पवः॑ । \newline
55. इद् रि॒पवो॑ रि॒पव॒ इदिद् रि॒पवो॒ न न रि॒पव॒ इदिद् रि॒पवो॒ न । \newline
56. रि॒पवो॒ न न रि॒पवो॑ रि॒पवो॒ ना ह॑ ह॒ न रि॒पवो॑ रि॒पवो॒ ना ह॑ । \newline
57. ना ह॑ ह॒ न ना ह॑ देभुर् देभुर्. ह॒ न ना ह॑ देभुः । \newline
58. ह॒ दे॒भु॒र् दे॒भु॒र्॒. ह॒ ह॒ दे॒भुः॒ । \newline
\pagebreak
\markright{ TS 1.2.14.6  \hfill https://www.vedavms.in \hfill}

\section{ TS 1.2.14.6 }

\textbf{TS 1.2.14.6 } \newline
\textbf{Samhita Paata} \newline

देभुः ॥ त्वया॑ व॒यꣳ स॑ध॒न्य॑-स्त्वोता॒-स्तव॒ प्रणी᳚त्यश्याम॒ वाजान्॑ । उ॒भा शꣳसा॑ सूदय सत्यताते-ऽनुष्ठु॒या कृ॑णुह्यह्रयाण ॥ अ॒या ते॑ अग्ने स॒मिधा॑ विधेम॒ प्रति॒स्तोमꣳ॑ श॒स्यमा॑नं गृभाय । दहा॒शसो॑ र॒क्षसः॑ पा॒ह्य॑स्मान् द्रु॒हो नि॒दो मि॑त्रमहो अव॒द्यात् ॥ र॒क्षो॒हणं॑ ॅवा॒जिन॒माजि॑घर्मि मि॒त्रं प्रथि॑ष्ठ॒-मुप॑यामि॒ शर्म॑ । शिशा॑नो अ॒ग्निः क्रतु॑भिः॒ समि॑द्धः॒सनो॒ दिवा॒ - [ ] \newline

\textbf{Pada Paata} \newline

दे॒भुः॒ ॥ त्वया᳚ । व॒यम् । स॒ध॒न्य॑ इति॑ सध - न्यः॑ । त्वोताः᳚ । तव॑ । प्रणी॒तीति॒ प्र - नी॒ती॒ । अ॒श्या॒म॒ । वाजान्॑ ॥ उ॒भा । शꣳसा᳚ । सू॒द॒य॒ । स॒त्य॒ता॒त॒ इति॑ सत्य - ता॒ते॒ । अ॒नु॒ष्ठु॒या । कृ॒णु॒हि॒ । अ॒ह्र॒या॒ण॒ ॥ अ॒या । ते॒ । अ॒ग्ने॒ । स॒मिधेति॑ सं - इधा᳚ । वि॒धे॒म॒ । प्रतीति॑ । स्तोम᳚म् । श॒स्यमा॑नम् । गृ॒भा॒य॒ ॥ दह॑ । अ॒शसः॑ । र॒क्षसः॑ । पा॒हि । अ॒स्मान् । द्रु॒हः । नि॒दः । मि॒त्र॒म॒ह॒ इति॑ मित्र - म॒हः॒ । अ॒व॒द्यात् ॥ र॒क्षो॒हण॒मिति॑ रक्षः - हन᳚म् । वा॒जिन᳚म् । एति॑ । जि॒घ॒र्मि॒ । मि॒त्रम् । प्रथि॑ष्ठम् । उपेति॑ । या॒मि॒ । शर्म॑ ॥ शिशा॑नः । अ॒ग्निः । क्रतु॑भि॒रिति॒ क्रतु॑ - भिः॒ । समि॑द्ध॒ इति॒ सं - इ॒द्धः॒ । सः । नः॒ । दिवा᳚ ।  \newline


\textbf{Krama Paata} \newline

दे॒भु॒रिति॑ देभुः ॥ त्वया॑ व॒यम् । व॒यꣳ स॑ध॒न्यः॑ । स॒ध॒न्य॑स्त्वोताः᳚ । स॒ध॒न्य॑ इति॑ सध - न्यः॑ । त्वोता॒ स्तव॑ । तव॒ प्रणी॑ति । प्रणी᳚त्यश्याम । प्रणी॒तीति॒ प्र - नी॒ती॒ । अ॒श्या॒म॒ वाजान्॑ । वाजा॒निति॒ वाजान्॑ ॥ उ॒भा शꣳसा᳚ । शꣳसा॑ सूदय । सू॒द॒य॒ स॒त्य॒ता॒ते॒ । स॒त्य॒ता॒ते॒ ऽनु॒ष्ठु॒या । स॒त्य॒ता॒त॒ इति॑ सत्य - ता॒ते॒ । अ॒नु॒ष्ठु॒या कृ॑णुहि । कृ॒णु॒ह्य॒ह्र॒या॒ण॒ । अ॒ह्र॒या॒णेत्य॑ह्रयाण ॥ अ॒या ते᳚ । ते॒ अ॒ग्ने॒ । अ॒ग्ने॒ स॒मिधा᳚ । स॒मिधा॑ विधेम । स॒मिधेति॑ सं - इधा᳚ । वि॒धे॒म॒ प्रति॑ । प्रति॒ स्तोम᳚म् । स्तोमꣳ॑ श॒स्यमा॑नम् । श॒स्यमा॑नम् गृभाय । गृ॒भा॒येति॑ गृभाय ॥ दहा॒शसः॑ । अ॒शसो॑ र॒क्षसः॑ । र॒क्षसः॑ पा॒हि । पा॒ह्य॑स्मान् । अ॒स्मान् द्रु॒हः । द्रु॒हो नि॒दः । नि॒दो मि॑त्रमहः । मि॒त्र॒म॒हो॒ अ॒व॒द्यात् । मि॒त्र॒म॒ह॒ इति॑ मित्र - म॒हः॒ । अ॒व॒द्यादित्य॑व॒द्यात् ॥ र॒क्षो॒हणं॑ ॅवा॒जिन᳚म् । र॒क्षो॒हण॒मिति॑ रक्षः - हन᳚म् । वा॒जिन॒मा । आ जि॑घर्मि । जि॒घ॒र्मि॒ मि॒त्रम् । मि॒त्रम् प्रथि॑ष्ठम् । प्रथि॑ष्ठ॒मुप॑ । उप॑ यामि । या॒मि॒ शर्म॑ । शर्मेति॒ शर्म॑ ॥ शिशा॑नो अ॒ग्निः । अ॒ग्निः क्रतु॑भिः । क्रतु॑भिः॒ समि॑द्धः । क्रतु॑भि॒रिति॒ क्रतु॑ - भिः॒ । समि॑द्धः॒ सः । समि॑द्ध॒ इति॒ सम् - इ॒द्धः॒ । स नः॑ । नो॒ दिवा᳚ ( ) । दिवा॒ सः \newline

\textbf{Jatai Paata} \newline

1. दे॒भु॒रिति॑ देभुः । \newline
2. त्वया॑ व॒यं ॅव॒यम् त्वया॒ त्वया॑ व॒यम् । \newline
3. व॒यꣳ स॑ध॒न्यः॑ सध॒न्यो॑ व॒यं ॅव॒यꣳ स॑ध॒न्यः॑ । \newline
4. स॒ध॒न्य॑ स्त्वोता॒ स्त्वोताः᳚ सध॒न्यः॑ सध॒न्य॑ स्त्वोताः᳚ । \newline
5. स॒ध॒न्य॑ इति॑ सध - न्यः॑ । \newline
6. त्वोता॒ स्तव॒ तव॒ त्वोता॒ स्त्वोता॒ स्तव॑ । \newline
7. तव॒ प्रणी॑ती॒ प्रणी॑ती॒ तव॒ तव॒ प्रणी॑ती । \newline
8. प्रणी᳚ त्यश्यामा श्याम॒ प्रणी॑ती॒ प्रणी᳚त्यश्याम । \newline
9. प्रणी॒तीति॒ प्र - नी॒ती॒ । \newline
10. अ॒श्या॒म॒ वाजा॒न्॒. वाजा॑ नश्यामा श्याम॒ वाजान्॑ । \newline
11. वाजा॒निति॒ वाजान्॑ । \newline
12. उ॒भा शꣳसा॒ शꣳसो॒ भोभा शꣳसा᳚ । \newline
13. शꣳसा॑ सूदय सूदय॒ शꣳसा॒ शꣳसा॑ सूदय । \newline
14. सू॒द॒य॒ स॒त्य॒ता॒ते॒ स॒त्य॒ता॒ते॒ सू॒द॒य॒ सू॒द॒य॒ स॒त्य॒ता॒ते॒ । \newline
15. स॒त्य॒ता॒ते॒ ऽनु॒ष्ठु॒या ऽनु॑ष्ठु॒या स॑त्यताते सत्यताते ऽनुष्ठु॒या । \newline
16. स॒त्य॒ता॒त॒ इति॑ सत्य - ता॒ते॒ । \newline
17. अ॒नु॒ष्ठु॒या कृ॑णुहि कृणु ह्यनुष्ठु॒या ऽनु॑ष्ठु॒या कृ॑णुहि । \newline
18. कृ॒णु॒ह्य॒ ह्र॒या॒णा॒ ह्र॒या॒ण॒ कृ॒णु॒हि॒ कृ॒णु॒ह्य॒ ह्र॒या॒ण॒ । \newline
19. अ॒ह्र॒या॒णेत्य॑ह्रयाण । \newline
20. अ॒या ते॑ ते॒ ऽया ऽया ते᳚ । \newline
21. ते॒ अ॒ग्ने॒ अ॒ग्ने॒ ते॒ ते॒ अ॒ग्ने॒ । \newline
22. अ॒ग्ने॒ स॒मिधा॑ स॒मिधा᳚ ऽग्ने अग्ने स॒मिधा᳚ । \newline
23. स॒मिधा॑ विधेम विधेम स॒मिधा॑ स॒मिधा॑ विधेम । \newline
24. स॒मिधेति॑ सं - इधा᳚ । \newline
25. वि॒धे॒म॒ प्रति॒ प्रति॑ विधेम विधेम॒ प्रति॑ । \newline
26. प्रति॒ स्तोम॒(ग्ग्॒) स्तोम॒म् प्रति॒ प्रति॒ स्तोम᳚म् । \newline
27. स्तोम(ग्म्॑) श॒स्यमा॑नꣳ श॒स्यमा॑न॒(ग्ग्॒) स्तोम॒(ग्ग्॒) स्तोम(ग्म्॑) श॒स्यमा॑नम् । \newline
28. श॒स्यमा॑नम् गृभाय गृभाय श॒स्यमा॑नꣳ श॒स्यमा॑नम् गृभाय । \newline
29. गृ॒भा॒येति॑ गृभाय । \newline
30. दहा॒शसो॑ अ॒शसो॒ दह॒ दहा॒शसः॑ । \newline
31. अ॒शसो॑ र॒क्षसो॑ र॒क्षसो॑ अ॒शसो॑ अ॒शसो॑ र॒क्षसः॑ । \newline
32. र॒क्षसः॑ पा॒हि पा॒हि र॒क्षसो॑ र॒क्षसः॑ पा॒हि । \newline
33. पा॒ह्य॑स्मा न॒स्मान् पा॒हि पा॒ह्य॑स्मान् । \newline
34. अ॒स्मान् द्रु॒हो द्रु॒हो अ॒स्मा न॒स्मान् द्रु॒हः । \newline
35. द्रु॒हो नि॒दो नि॒दो द्रु॒हो द्रु॒हो नि॒दः । \newline
36. नि॒दो मि॑त्रमहो मित्रमहो नि॒दो नि॒दो मि॑त्रमहः । \newline
37. मि॒त्र॒म॒हो॒ अ॒व॒द्या द॑व॒द्यान् मि॑त्रमहो मित्रमहो अव॒द्यात् । \newline
38. मि॒त्र॒म॒ह॒ इति॑ मित्र - म॒हः॒ । \newline
39. अ॒व॒द्यादित्य॑व॒द्यात् । \newline
40. र॒क्षो॒हणं॑ ॅवा॒जिनं॑ ॅवा॒जिन(ग्म्॑) रक्षो॒हण(ग्म्॑) रक्षो॒हणं॑ ॅवा॒जिन᳚म् । \newline
41. र॒क्षो॒हण॒मिति॑ रक्षः - हन᳚म् । \newline
42. वा॒जिन॒ मा वा॒जिनं॑ ॅवा॒जिन॒ मा । \newline
43. आ जि॑घर्मि जिघ॒र्म्या जि॑घर्मि । \newline
44. जि॒घ॒र्मि॒ मि॒त्रम् मि॒त्रम् जि॑घर्मि जिघर्मि मि॒त्रम् । \newline
45. मि॒त्रम् प्रथि॑ष्ठ॒म् प्रथि॑ष्ठम् मि॒त्रम् मि॒त्रम् प्रथि॑ष्ठम् । \newline
46. प्रथि॑ष्ठ॒ मुपोप॒ प्रथि॑ष्ठ॒म् प्रथि॑ष्ठ॒ मुप॑ । \newline
47. उप॑ यामि या॒म्युपोप॑ यामि । \newline
48. या॒मि॒ शर्म॒ शर्म॑ यामि यामि॒ शर्म॑ । \newline
49. शर्मेति॒ शर्म॑ । \newline
50. शिशा॑नो अ॒ग्नि र॒ग्निः शिशा॑नः॒ शिशा॑नो अ॒ग्निः । \newline
51. अ॒ग्निः क्रतु॑भिः॒ क्रतु॑भि र॒ग्नि र॒ग्निः क्रतु॑भिः । \newline
52. क्रतु॑भिः॒ समि॑द्धः॒ समि॑द्धः॒ क्रतु॑भिः॒ क्रतु॑भिः॒ समि॑द्धः । \newline
53. क्रतु॑भि॒रिति॒ क्रतु॑ - भिः॒ । \newline
54. समि॑द्धः॒ स स समि॑द्धः॒ समि॑द्धः॒ सः । \newline
55. समि॑द्ध॒ इति॒ सं - इ॒द्धः॒ । \newline
56. स नो॑ नः॒ स स नः॑ । \newline
57. नो॒ दिवा॒ दिवा॑ नो नो॒ दिवा᳚ । \newline
58. दिवा॒ स स दिवा॒ दिवा॒ सः । \newline

\textbf{Ghana Paata } \newline

1. दे॒भु॒रिति॑ देभुः । \newline
2. त्वया॑ व॒यं ॅव॒यम् त्वया॒ त्वया॑ व॒यꣳ स॑ध॒न्यः॑ सध॒न्यो॑ व॒यम् त्वया॒ त्वया॑ व॒यꣳ स॑ध॒न्यः॑ । \newline
3. व॒यꣳ स॑ध॒न्यः॑ सध॒न्यो॑ व॒यं ॅव॒यꣳ स॑ध॒न्य॑ स्त्वोता॒ स्त्वोताः᳚ सध॒न्यो॑ व॒यं ॅव॒यꣳ स॑ध॒न्य॑ स्त्वोताः᳚ । \newline
4. स॒ध॒न्य॑ स्त्वोता॒ स्त्वोताः᳚ सध॒न्यः॑ सध॒न्य॑ स्त्वोता॒ स्तव॒ तव॒ त्वोताः᳚ सध॒न्यः॑ सध॒न्य॑ स्त्वोता॒ स्तव॑ । \newline
5. स॒ध॒न्य॑ इति॑ सध - न्यः॑ । \newline
6. त्वोता॒स्तव॒ तव॒ त्वोता॒ स्त्वोता॒ स्तव॒ प्रणी॑ती॒ प्रणी॑ती॒ तव॒ त्वोता॒ स्त्वोता॒ स्तव॒ प्रणी॑ती । \newline
7. तव॒ प्रणी॑ती॒ प्रणी॑ती॒ तव॒ तव॒ प्रणी᳚त्यश्यामाश्याम॒ प्रणी॑ती॒ तव॒ तव॒ प्रणी᳚त्यश्याम । \newline
8. प्रणी᳚त्यश्यामाश्याम॒ प्रणी॑ती॒ प्रणी᳚त्यश्याम॒ वाजा॒न्॒. वाजा॑ नश्याम॒ प्रणी॑ती॒ प्रणी᳚त्यश्याम॒ वाजान्॑ । \newline
9. प्रणी॒तीति॒ प्र - नी॒ती॒ । \newline
10. अ॒श्या॒म॒ वाजा॒न्॒. वाजा॑ नश्यामाश्याम॒ वाजान्॑ । \newline
11. वाजा॒निति॒ वाजान्॑ । \newline
12. उ॒भा शꣳसा॒ शꣳसो॒भोभा शꣳसा॑ सूदय सूदय॒ शꣳसो॒भोभा शꣳसा॑ सूदय । \newline
13. शꣳसा॑ सूदय सूदय॒ शꣳसा॒ शꣳसा॑ सूदय सत्यताते सत्यताते सूदय॒ शꣳसा॒ शꣳसा॑ सूदय सत्यताते । \newline
14. सू॒द॒य॒ स॒त्य॒ता॒ते॒ स॒त्य॒ता॒ते॒ सू॒द॒य॒ सू॒द॒य॒ स॒त्य॒ता॒ते॒ ऽनु॒ष्ठु॒या ऽनु॑ष्ठु॒या स॑त्यताते सूदय सूदय सत्यताते ऽनुष्ठु॒या । \newline
15. स॒त्य॒ता॒ते॒ ऽनु॒ष्ठु॒या ऽनु॑ष्ठु॒या स॑त्यताते सत्यताते ऽनुष्ठु॒या कृ॑णुहि कृणुह्यनुष्ठु॒या स॑त्यताते सत्यताते ऽनुष्ठु॒या कृ॑णुहि । \newline
16. स॒त्य॒ता॒त॒ इति॑ सत्य - ता॒ते॒ । \newline
17. अ॒नु॒ष्ठु॒या कृ॑णुहि कृणु ह्यनुष्ठु॒या ऽनु॑ष्ठु॒या कृ॑णु ह्यह्रयाणाह्रयाण कृणु ह्यनुष्ठु॒या ऽनु॑ष्ठु॒या कृ॑णु ह्यह्रयाण । \newline
18. कृ॒णु॒ ह्य॒ह्र॒या॒णा॒ह्र॒या॒ण॒ कृ॒णु॒हि॒ कृ॒णु॒ ह्य॒ह्र॒या॒ण॒ । \newline
19. अ॒ह्र॒या॒णेत्य॑ह्रयाण । \newline
20. अ॒या ते॑ ते॒ ऽया ऽया ते॑ अग्ने अग्ने ते॒ ऽया ऽया ते॑ अग्ने । \newline
21. ते॒ अ॒ग्ने॒ अ॒ग्ने॒ ते॒ ते॒ अ॒ग्ने॒ स॒मिधा॑ स॒मिधा᳚ ऽग्ने ते ते अग्ने स॒मिधा᳚ । \newline
22. अ॒ग्ने॒ स॒मिधा॑ स॒मिधा᳚ ऽग्ने अग्ने स॒मिधा॑ विधेम विधेम स॒मिधा᳚ ऽग्ने अग्ने स॒मिधा॑ विधेम । \newline
23. स॒मिधा॑ विधेम विधेम स॒मिधा॑ स॒मिधा॑ विधेम॒ प्रति॒ प्रति॑ विधेम स॒मिधा॑ स॒मिधा॑ विधेम॒ प्रति॑ । \newline
24. स॒मिधेति॑ सं - इधा᳚ । \newline
25. वि॒धे॒म॒ प्रति॒ प्रति॑ विधेम विधेम॒ प्रति॒ स्तोम॒(ग्ग्॒) स्तोम॒म् प्रति॑ विधेम विधेम॒ प्रति॒ स्तोम᳚म् । \newline
26. प्रति॒ स्तोम॒(ग्ग्॒) स्तोम॒म् प्रति॒ प्रति॒ स्तोम(ग्म्॑) श॒स्यमा॑नꣳ श॒स्यमा॑न॒(ग्ग्॒) स्तोम॒म् प्रति॒ प्रति॒ स्तोम(ग्म्॑) श॒स्यमा॑नम् । \newline
27. स्तोम(ग्म्॑) श॒स्यमा॑नꣳ श॒स्यमा॑न॒(ग्ग्॒) स्तोम॒(ग्ग्॒) स्तोम(ग्म्॑) श॒स्यमा॑नम् गृभाय गृभाय श॒स्यमा॑न॒(ग्ग्॒) स्तोम॒(ग्ग्॒) स्तोम(ग्म्॑) श॒स्यमा॑नम् गृभाय । \newline
28. श॒स्यमा॑नम् गृभाय गृभाय श॒स्यमा॑नꣳ श॒स्यमा॑नम् गृभाय । \newline
29. गृ॒भा॒येति॑ गृभाय । \newline
30. दहा॒शसो॑ अ॒शसो॒ दह॒ दहा॒शसो॑ र॒क्षसो॑ र॒क्षसो॑ अ॒शसो॒ दह॒ दहा॒शसो॑ र॒क्षसः॑ । \newline
31. अ॒शसो॑ र॒क्षसो॑ र॒क्षसो॑ अ॒शसो॑ अ॒शसो॑ र॒क्षसः॑ पा॒हि पा॒हि र॒क्षसो॑ अ॒शसो॑ अ॒शसो॑ र॒क्षसः॑ पा॒हि । \newline
32. र॒क्षसः॑ पा॒हि पा॒हि र॒क्षसो॑ र॒क्षसः॑ पा॒ह्य॑स्मा न॒स्मान् पा॒हि र॒क्षसो॑ र॒क्षसः॑ पा॒ह्य॑स्मान् । \newline
33. पा॒ह्य॑स्मा न॒स्मान् पा॒हि पा॒ह्य॑स्मान् द्रु॒हो द्रु॒हो अ॒स्मान् पा॒हि पा॒ह्य॑स्मान् द्रु॒हः । \newline
34. अ॒स्मान् द्रु॒हो द्रु॒हो अ॒स्मा न॒स्मान् द्रु॒हो नि॒दो नि॒दो द्रु॒हो अ॒स्मा न॒स्मान् द्रु॒हो नि॒दः । \newline
35. द्रु॒हो नि॒दो नि॒दो द्रु॒हो द्रु॒हो नि॒दो मि॑त्रमहो मित्रमहो नि॒दो द्रु॒हो द्रु॒हो नि॒दो मि॑त्रमहः । \newline
36. नि॒दो मि॑त्रमहो मित्रमहो नि॒दो नि॒दो मि॑त्रमहो अव॒द्या द॑व॒द्यान् मि॑त्रमहो नि॒दो नि॒दो मि॑त्रमहो अव॒द्यात् । \newline
37. मि॒त्र॒म॒हो॒ अ॒व॒द्या द॑व॒द्यान् मि॑त्रमहो मित्रमहो अव॒द्यात् । \newline
38. मि॒त्र॒म॒ह॒ इति॑ मित्र - म॒हः॒ । \newline
39. अ॒व॒द्यादित्य॑व॒द्यात् । \newline
40. र॒क्षो॒हणं॑ ॅवा॒जिनं॑ ॅवा॒जिन(ग्म्॑) रक्षो॒हण(ग्म्॑) रक्षो॒हणं॑ ॅवा॒जिन॒ मा वा॒जिन(ग्म्॑) रक्षो॒हण(ग्म्॑) रक्षो॒हणं॑ ॅवा॒जिन॒ मा । \newline
41. र॒क्षो॒हण॒मिति॑ रक्षः - हन᳚म् । \newline
42. वा॒जिन॒ मा वा॒जिनं॑ ॅवा॒जिन॒ मा जि॑घर्मि जिघ॒र्म्या वा॒जिनं॑ ॅवा॒जिन॒ मा जि॑घर्मि । \newline
43. आ जि॑घर्मि जिघ॒र्म्या जि॑घर्मि मि॒त्रम् मि॒त्रम् जि॑घ॒र्म्या जि॑घर्मि मि॒त्रम् । \newline
44. जि॒घ॒र्मि॒ मि॒त्रम् मि॒त्रम् जि॑घर्मि जिघर्मि मि॒त्रम् प्रथि॑ष्ठ॒म् प्रथि॑ष्ठम् मि॒त्रम् जि॑घर्मि जिघर्मि मि॒त्रम् प्रथि॑ष्ठम् । \newline
45. मि॒त्रम् प्रथि॑ष्ठ॒म् प्रथि॑ष्ठम् मि॒त्रम् मि॒त्रम् प्रथि॑ष्ठ॒ मुपोप॒ प्रथि॑ष्ठम् मि॒त्रम् मि॒त्रम् प्रथि॑ष्ठ॒ मुप॑ । \newline
46. प्रथि॑ष्ठ॒ मुपोप॒ प्रथि॑ष्ठ॒म् प्रथि॑ष्ठ॒ मुप॑ यामि या॒म्युप॒ प्रथि॑ष्ठ॒म् प्रथि॑ष्ठ॒ मुप॑ यामि । \newline
47. उप॑ यामि या॒म्युपोप॑ यामि॒ शर्म॒ शर्म॑ या॒म्युपोप॑ यामि॒ शर्म॑ । \newline
48. या॒मि॒ शर्म॒ शर्म॑ यामि यामि॒ शर्म॑ । \newline
49. शर्मेति॒ शर्म॑ । \newline
50. शिशा॑नो अ॒ग्निर॒ग्निः शिशा॑नः॒ शिशा॑नो अ॒ग्निः क्रतु॑भिः॒ क्रतु॑भिर॒ग्निः शिशा॑नः॒ शिशा॑नो अ॒ग्निः क्रतु॑भिः । \newline
51. अ॒ग्निः क्रतु॑भिः॒ क्रतु॑भि र॒ग्निर॒ग्निः क्रतु॑भिः॒ समि॑द्धः॒ समि॑द्धः॒ क्रतु॑भिर॒ग्निर॒ग्निः क्रतु॑भिः॒ समि॑द्धः । \newline
52. क्रतु॑भिः॒ समि॑द्धः॒ समि॑द्धः॒ क्रतु॑भिः॒ क्रतु॑भिः॒ समि॑द्धः॒ स स समि॑द्धः॒ क्रतु॑भिः॒ क्रतु॑भिः॒ समि॑द्धः॒ सः । \newline
53. क्रतु॑भि॒रिति॒ क्रतु॑ - भिः॒ । \newline
54. समि॑द्धः॒ स स समि॑द्धः॒ समि॑द्धः॒ स नो॑ नः॒ स समि॑द्धः॒ समि॑द्धः॒ स नः॑ । \newline
55. समि॑द्ध॒ इति॒ सं - इ॒द्धः॒ । \newline
56. स नो॑ नः॒ स स नो॒ दिवा॒ दिवा॑ नः॒ स स नो॒ दिवा᳚ । \newline
57. नो॒ दिवा॒ दिवा॑ नो नो॒ दिवा॒ स स दिवा॑ नो नो॒ दिवा॒ सः । \newline
58. दिवा॒ स स दिवा॒ दिवा॒ स रि॒षो रि॒षः स दिवा॒ दिवा॒ स रि॒षः । \newline
\pagebreak
\markright{ TS 1.2.14.7  \hfill https://www.vedavms.in \hfill}

\section{ TS 1.2.14.7 }

\textbf{TS 1.2.14.7 } \newline
\textbf{Samhita Paata} \newline

सरि॒षः पा॑तु॒नक्तं᳚ ॥ विज्योति॑षा बृह॒ता भा᳚त्य॒ग्नि-रा॒विर् विश्वा॑नि कृणुते महि॒त्वा । प्रादे॑वीर् मा॒याः स॑हते-दु॒रेवाः॒ शिशी॑ते॒ शृङ्गे॒ रक्ष॑से वि॒निक्षे᳚ ॥ उ॒त स्वा॒नासो॑ दि॒विष॑न्त्व॒ग्ने स्ति॒ग्मायु॑धा॒ रक्ष॑से॒ हन्त॒वा उ॑ । मदे॑ चिदस्य॒ प्ररु॑जन्ति॒ भामा॒ न व॑रन्ते परि॒बाधो॒ अदे॑वीः ॥ \newline

\textbf{Pada Paata} \newline

सः । रि॒षः । पा॒तु॒ । नक्त᳚म् ॥ वीति॑ । ज्योति॑षा । बृ॒ह॒ता । भा॒ति॒ । अ॒ग्निः । आ॒विः । विश्वा॑नि । कृ॒णु॒ते॒ । म॒हि॒त्वेति॑ महि - त्वा ॥ प्रेति॑ । अदे॑वीः । मा॒याः । स॒ह॒ते॒ । दु॒रेवा॒ इति॑ दुः - एवाः᳚ । शिशी॑ते । शृङ्गे॒ इति॑ । रक्ष॑से । वि॒निक्ष॒ इति॑ वि - निक्षे᳚ ॥ उ॒त । स्वा॒नासः॑ । दि॒वि । स॒न्तु॒ । अ॒ग्नेः । ति॒ग्मायु॑धा॒ इति॑ ति॒ग्म - आ॒यु॒धाः॒ । रक्ष॑से । हन्त॒वै । उ॒ ॥ मदे᳚ । चि॒त् । अ॒स्य॒ । प्रेति॑ । रु॒ज॒न्ति॒ । भामाः᳚ । न । व॒र॒न्ते॒ । प॒रि॒बाध॒ इति॑ परि - बाधः॑ । अदे॑वीः ॥  \newline


\textbf{Krama Paata} \newline

स रि॒षः । रि॒षः पा॑तु । पा॒तु॒ नक्त᳚म् । नक्त॒मिति॒ नक्त᳚म् ॥ वि ज्योति॑षा । ज्योति॑षा बृह॒ता । बृ॒ह॒ता भा॑ति । भा॒त्य॒ग्निः । अ॒ग्निरा॒विः । आ॒विर् विश्वा॑नि । विश्वा॑नि कृणुते । कृ॒णु॒ते॒ म॒हि॒त्वा । म॒हि॒त्वेति॑ महि - त्वा ॥ प्रादे॑वीः । अदे॑वीर् मा॒याः । मा॒याः स॑हते । स॒ह॒ते॒ दु॒रेवाः᳚ । दु॒रेवाः॒ शिशी॑ते । दु॒रेवा॒ इति॑ दुः - एवाः᳚ । शिशी॑ते॒ शृङ्गे᳚ । शृङ्गे॒ रक्ष॑से । शृङ्गे॒ इति॒ शृङ्गे᳚ । रक्ष॑से वि॒निक्षे᳚ । वि॒निक्ष॒ इति॑ वि - निक्षे᳚ ॥ उ॒त स्वा॒नासः॑ । स्वा॒नासो॑ दि॒वि । दि॒विष॑न्तु । स॒न्त्व॒ग्नेः । अ॒ग्ने स्ति॒ग्मायु॑धाः । ति॒ग्मायु॑धा॒ रक्ष॑से । ति॒ग्मायु॑धा॒ इति॑ ति॒ग्म - आ॒यु॒धाः॒ । रक्ष॑से॒ हन्त॒वै । हन्त॒वा उ॑ । उ॒वित्यु॑ ॥ मदे॑ चित् । चि॒द॒स्य॒ । अ॒स्य॒ प्र । प्र रु॑जन्ति । रु॒ज॒न्ति॒ भामाः᳚ । भामा॒ न । न व॑रन्ते । व॒र॒न्ते॒ प॒रि॒बाधः॑ । प॒रि॒बाधो॒ अदे॑वीः । प॒रि॒बाध॒ इति॑ परि - बाधः॑ । अदे॑वी॒रित्यदे॑वीः । \newline

\textbf{Jatai Paata} \newline

1. स रि॒षो रि॒षः स स रि॒षः । \newline
2. रि॒षः पा॑तु पातु रि॒षो रि॒षः पा॑तु । \newline
3. पा॒तु॒ नक्त॒न्नक्त॑म् पातु पातु॒ नक्त᳚म् । \newline
4. नक्त॒मिति॒ नक्त᳚म् । \newline
5. वि ज्योति॑षा॒ ज्योति॑षा॒ वि वि ज्योति॑षा । \newline
6. ज्योति॑षा बृह॒ता बृ॑ह॒ता ज्योति॑षा॒ ज्योति॑षा बृह॒ता । \newline
7. बृ॒ह॒ता भा॑ति भाति बृह॒ता बृ॑ह॒ता भा॑ति । \newline
8. भा॒त्य॒ग्नि र॒ग्निर् भा॑ति भात्य॒ग्निः । \newline
9. अ॒ग्नि रा॒वि रा॒वि र॒ग्नि र॒ग्नि रा॒विः । \newline
10. आ॒विर् विश्वा॑नि॒ विश्वा᳚न्या॒ विरा॒विर् विश्वा॑नि । \newline
11. विश्वा॑नि कृणुते कृणुते॒ विश्वा॑नि॒ विश्वा॑नि कृणुते । \newline
12. कृ॒णु॒ते॒ म॒हि॒त्वा म॑हि॒त्वा कृ॑णुते कृणुते महि॒त्वा । \newline
13. म॒हि॒त्वेति॑ महि - त्वा । \newline
14. प्रादे॑वी॒ रदे॑वीः॒ प्र प्रादे॑वीः । \newline
15. अदे॑वीर् मा॒या मा॒या अदे॑वी॒ रदे॑वीर् मा॒याः । \newline
16. मा॒याः स॑हते सहते मा॒या मा॒याः स॑हते । \newline
17. स॒ह॒ते॒ दु॒रेवा॑ दु॒रेवाः᳚ सहते सहते दु॒रेवाः᳚ । \newline
18. दु॒रेवाः॒ शिशी॑ते॒ शिशी॑ते दु॒रेवा॑ दु॒रेवाः॒ शिशी॑ते । \newline
19. दु॒रेवा॒ इति॑ दुः - एवाः᳚ । \newline
20. शिशी॑ते॒ शृङ्गे॒ शृङ्गे॒ शिशी॑ते॒ शिशी॑ते॒ शृङ्गे᳚ । \newline
21. शृङ्गे॒ रक्ष॑से॒ रक्ष॑से॒ शृङ्गे॒ शृङ्गे॒ रक्ष॑से । \newline
22. शृङ्गे॒ इति॒ शृङ्गे᳚ । \newline
23. रक्ष॑से वि॒निक्षे॑ वि॒निक्षे॒ रक्ष॑से॒ रक्ष॑से वि॒निक्षे᳚ । \newline
24. वि॒निक्ष॒ इति॑ वि - निक्षे᳚ । \newline
25. उ॒त स्वा॒नासः॑ स्वा॒नास॑ उ॒तोत स्वा॒नासः॑ । \newline
26. स्वा॒नासो॑ दि॒वि दि॒वि स्वा॒नासः॑ स्वा॒नासो॑ दि॒वि । \newline
27. दि॒वि ष॑न्तु सन्तु दि॒वि दि॒वि ष॑न्तु । \newline
28. स॒न्त्व॒ग्नेर॒ग्नेः स॑न्तु सन्त्व॒ग्नेः । \newline
29. अ॒ग्ने स्ति॒ग्मायु॑धा स्ति॒ग्मायु॑धा अ॒ग्ने र॒ग्ने स्ति॒ग्मायु॑धाः । \newline
30. ति॒ग्मायु॑धा॒ रक्ष॑से॒ रक्ष॑से ति॒ग्मायु॑धा स्ति॒ग्मायु॑धा॒ रक्ष॑से । \newline
31. ति॒ग्मायु॑धा॒ इति॑ ति॒ग्म - आ॒यु॒धाः॒ । \newline
32. रक्ष॑से॒ हन्त॒वै हन्त॒वै रक्ष॑से॒ रक्ष॑से॒ हन्त॒वै । \newline
33. हन्त॒वा उ॑ वु॒ हन्त॒वै हन्त॒वा उ॑ । \newline
34. उ॒वित्यु॑ । \newline
35. मदे॑ चिच् चि॒न् मदे॒ मदे॑ चित् । \newline
36. चि॒द॒ स्या॒स्य॒ चि॒च् चि॒द॒स्य॒ । \newline
37. अ॒स्य॒ प्र प्रास्या᳚स्य॒ प्र । \newline
38. प्र रु॑जन्ति रुजन्ति॒ प्र प्र रु॑जन्ति । \newline
39. रु॒ज॒न्ति॒ भामा॒ भामा॑ रुजन्ति रुजन्ति॒ भामाः᳚ । \newline
40. भामा॒ न न भामा॒ भामा॒ न । \newline
41. न व॑रन्ते वरन्ते॒ न न व॑रन्ते । \newline
42. व॒र॒न्ते॒ प॒रि॒बाधः॑ परि॒बाधो॑ वरन्ते वरन्ते परि॒बाधः॑ । \newline
43. प॒रि॒बाधो॒ अदे॑वी॒रदे॑वीः परि॒बाधः॑ परि॒बाधो॒ अदे॑वीः । \newline
44. प॒रि॒बाध॒ इति॑ परि - बाधः॑ । \newline
45. अदे॑वी॒रित्यदे॑वीः । \newline

\textbf{Ghana Paata } \newline

1. स रि॒षो रि॒षः स स रि॒षः पा॑तु पातु रि॒षः स स रि॒षः पा॑तु । \newline
2. रि॒षः पा॑तु पातु रि॒षो रि॒षः पा॑तु॒ नक्त॒न्नक्त॑म् पातु रि॒षो रि॒षः पा॑तु॒ नक्त᳚म् । \newline
3. पा॒तु॒ नक्त॒न्नक्त॑म् पातु पातु॒ नक्त᳚म् । \newline
4. नक्त॒मिति॒ नक्त᳚म् । \newline
5. वि ज्योति॑षा॒ ज्योति॑षा॒ वि वि ज्योति॑षा बृह॒ता बृ॑ह॒ता ज्योति॑षा॒ वि वि ज्योति॑षा बृह॒ता । \newline
6. ज्योति॑षा बृह॒ता बृ॑ह॒ता ज्योति॑षा॒ ज्योति॑षा बृह॒ता भा॑ति भाति बृह॒ता ज्योति॑षा॒ ज्योति॑षा बृह॒ता भा॑ति । \newline
7. बृ॒ह॒ता भा॑ति भाति बृह॒ता बृ॑ह॒ता भा᳚त्य॒ग्नि र॒ग्निर् भा॑ति बृह॒ता बृ॑ह॒ता भा᳚त्य॒ग्निः । \newline
8. भा॒त्य॒ग्नि र॒ग्निर् भा॑ति भात्य॒ग्नि रा॒वि रा॒वि र॒ग्निर् भा॑ति भात्य॒ग्निरा॒विः । \newline
9. अ॒ग्नि रा॒वि रा॒वि र॒ग्नि र॒ग्नि रा॒विर् विश्वा॑नि॒ विश्वा᳚न्या॒वि र॒ग्नि र॒ग्नि रा॒विर् विश्वा॑नि । \newline
10. आ॒विर् विश्वा॑नि॒ विश्वा᳚न्या॒विरा॒विर् विश्वा॑नि कृणुते कृणुते॒ विश्वा᳚न्या॒विरा॒विर् विश्वा॑नि कृणुते । \newline
11. विश्वा॑नि कृणुते कृणुते॒ विश्वा॑नि॒ विश्वा॑नि कृणुते महि॒त्वा म॑हि॒त्वा कृ॑णुते॒ विश्वा॑नि॒ विश्वा॑नि कृणुते महि॒त्वा । \newline
12. कृ॒णु॒ते॒ म॒हि॒त्वा म॑हि॒त्वा कृ॑णुते कृणुते महि॒त्वा । \newline
13. म॒हि॒त्वेति॑ महि - त्वा । \newline
14. प्रादे॑वी॒रदे॑वीः॒ प्र प्रादे॑वीर् मा॒या मा॒या अदे॑वीः॒ प्र प्रादे॑वीर् मा॒याः । \newline
15. अदे॑वीर् मा॒या मा॒या अदे॑वी॒रदे॑वीर् मा॒याः स॑हते सहते मा॒या अदे॑वी॒रदे॑वीर् मा॒याः स॑हते । \newline
16. मा॒याः स॑हते सहते मा॒या मा॒याः स॑हते दु॒रेवा॑ दु॒रेवाः᳚ सहते मा॒या मा॒याः स॑हते दु॒रेवाः᳚ । \newline
17. स॒ह॒ते॒ दु॒रेवा॑ दु॒रेवाः᳚ सहते सहते दु॒रेवाः॒ शिशी॑ते॒ शिशी॑ते दु॒रेवाः᳚ सहते सहते दु॒रेवाः॒ शिशी॑ते । \newline
18. दु॒रेवाः॒ शिशी॑ते॒ शिशी॑ते दु॒रेवा॑ दु॒रेवाः॒ शिशी॑ते॒ शृङ्गे॒ शृङ्गे॒ शिशी॑ते दु॒रेवा॑ दु॒रेवाः॒ शिशी॑ते॒ शृङ्गे᳚ । \newline
19. दु॒रेवा॒ इति॑ दुः - एवाः᳚ । \newline
20. शिशी॑ते॒ शृङ्गे॒ शृङ्गे॒ शिशी॑ते॒ शिशी॑ते॒ शृङ्गे॒ रक्ष॑से॒ रक्ष॑से॒ शृङ्गे॒ शिशी॑ते॒ शिशी॑ते॒ शृङ्गे॒ रक्ष॑से । \newline
21. शृङ्गे॒ रक्ष॑से॒ रक्ष॑से॒ शृङ्गे॒ शृङ्गे॒ रक्ष॑से वि॒निक्षे॑ वि॒निक्षे॒ रक्ष॑से॒ शृङ्गे॒ शृङ्गे॒ रक्ष॑से वि॒निक्षे᳚ । \newline
22. शृङ्गे॒ इति॒ शृङ्गे᳚ । \newline
23. रक्ष॑से वि॒निक्षे॑ वि॒निक्षे॒ रक्ष॑से॒ रक्ष॑से वि॒निक्षे᳚ । \newline
24. वि॒निक्ष॒ इति॑ वि - निक्षे᳚ । \newline
25. उ॒त स्वा॒नासः॑ स्वा॒नास॑ उ॒तोत स्वा॒नासो॑ दि॒वि दि॒वि स्वा॒नास॑ उ॒तोत स्वा॒नासो॑ दि॒वि । \newline
26. स्वा॒नासो॑ दि॒वि दि॒वि स्वा॒नासः॑ स्वा॒नासो॑ दि॒वि ष॑न्तु सन्तु दि॒वि स्वा॒नासः॑ स्वा॒नासो॑ दि॒वि ष॑न्तु । \newline
27. दि॒वि ष॑न्तु सन्तु दि॒वि दि॒वि ष॑न्त्व॒ग्नेर॒ग्नेः स॑न्तु दि॒वि दि॒वि ष॑न्त्व॒ग्नेः । \newline
28. स॒न्त्व॒ग्नेर॒ग्नेः स॑न्तु सन्त्व॒ग्ने स्ति॒ग्मायु॑धा स्ति॒ग्मायु॑धा अ॒ग्नेः स॑न्तु सन्त्व॒ग्ने स्ति॒ग्मायु॑धाः । \newline
29. अ॒ग्ने स्ति॒ग्मायु॑धा स्ति॒ग्मायु॑धा अ॒ग्नेर॒ग्ने स्ति॒ग्मायु॑धा॒ रक्ष॑से॒ रक्ष॑से ति॒ग्मायु॑धा अ॒ग्नेर॒ग्ने स्ति॒ग्मायु॑धा॒ रक्ष॑से । \newline
30. ति॒ग्मायु॑धा॒ रक्ष॑से॒ रक्ष॑से ति॒ग्मायु॑धा स्ति॒ग्मायु॑धा॒ रक्ष॑से॒ हन्त॒वै हन्त॒वै रक्ष॑से ति॒ग्मायु॑धा स्ति॒ग्मायु॑धा॒ रक्ष॑से॒ हन्त॒वै । \newline
31. ति॒ग्मायु॑धा॒ इति॑ ति॒ग्म - आ॒यु॒धाः॒ । \newline
32. रक्ष॑से॒ हन्त॒वै हन्त॒वै रक्ष॑से॒ रक्ष॑से॒ हन्त॒वा उ॑ वु॒ हन्त॒वै रक्ष॑से॒ रक्ष॑से॒ हन्त॒वा उ॑ । \newline
33. हन्त॒वा उ॑ वु॒ हन्त॒वै हन्त॒वा उ॑ । \newline
34. उ॒वित्यु॑ । \newline
35. मदे॑ चिच् चि॒न् मदे॒ मदे॑ चिदस्यास्य चि॒न् मदे॒ मदे॑ चिदस्य । \newline
36. चि॒द॒स्या॒स्य॒ चि॒च् चि॒द॒स्य॒ प्र प्रास्य॑ चिच् चिदस्य॒ प्र । \newline
37. अ॒स्य॒ प्र प्रास्या᳚स्य॒ प्र रु॑जन्ति रुजन्ति॒ प्रास्या᳚स्य॒ प्र रु॑जन्ति । \newline
38. प्र रु॑जन्ति रुजन्ति॒ प्र प्र रु॑जन्ति॒ भामा॒ भामा॑ रुजन्ति॒ प्र प्र रु॑जन्ति॒ भामाः᳚ । \newline
39. रु॒ज॒न्ति॒ भामा॒ भामा॑ रुजन्ति रुजन्ति॒ भामा॒ न न भामा॑ रुजन्ति रुजन्ति॒ भामा॒ न । \newline
40. भामा॒ न न भामा॒ भामा॒ न व॑रन्ते वरन्ते॒ न भामा॒ भामा॒ न व॑रन्ते । \newline
41. न व॑रन्ते वरन्ते॒ न न व॑रन्ते परि॒बाधः॑ परि॒बाधो॑ वरन्ते॒ न न व॑रन्ते परि॒बाधः॑ । \newline
42. व॒र॒न्ते॒ प॒रि॒बाधः॑ परि॒बाधो॑ वरन्ते वरन्ते परि॒बाधो॒ अदे॑वी॒रदे॑वीः परि॒बाधो॑ वरन्ते वरन्ते परि॒बाधो॒ अदे॑वीः । \newline
43. प॒रि॒बाधो॒ अदे॑वी॒रदे॑वीः परि॒बाधः॑ परि॒बाधो॒ अदे॑वीः । \newline
44. प॒रि॒बाध॒ इति॑ परि - बाधः॑ । \newline
45. अदे॑वी॒रित्यदे॑वीः । \newline
\pagebreak


\end{document}