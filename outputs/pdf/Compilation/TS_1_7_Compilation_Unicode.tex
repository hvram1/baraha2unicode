\documentclass[17pt]{extarticle}
\usepackage{babel}
\usepackage{fontspec}
\usepackage{polyglossia}
\usepackage{extsizes}



\setmainlanguage{sanskrit}
\setotherlanguages{english} %% or other languages
\setlength{\parindent}{0pt}
\pagestyle{myheadings}
\newfontfamily\devanagarifont[Script=Devanagari]{AdishilaVedic}


\newcommand{\VAR}[1]{}
\newcommand{\BLOCK}[1]{}




\begin{document}
\begin{titlepage}
    \begin{center}
 
\begin{sanskrit}
    { \Huge
    कृष्ण यजुर्वेदीय तैत्तिरीय संहिता,पद,जटा,घन पाठः 
    }
    \\
    \vspace{2.5cm}
    \mbox{ \Huge
    1.7     प्रथमकाण्डे सप्तमः प्रश्नः - (याजमान ब्राह्मणं)   }
\end{sanskrit}
\end{center}

\end{titlepage}
\tableofcontents
\pagebreak

\markright{ TS 1.7.1.1  \hfill https://www.vedavms.in \hfill}
\addcontentsline{toc}{section}{ TS 1.7.1.1 }
\section*{ TS 1.7.1.1 }

\textbf{TS 1.7.1.1 } \newline
\textbf{Samhita Paata} \newline

पा॒क॒य॒ज्ञ्ं ॅवा अन्वाहि॑ताग्नेः प॒शव॒ उप॑ तिष्ठन्त॒ इडा॒ खलु॒ वै पा॑कय॒ज्ञ्ः सैषाऽन्त॒रा प्र॑याजानूया॒जान्. यज॑मानस्य लो॒केऽव॑हिता॒ तामा᳚ह्रि॒यमा॑णाम॒भि म॑न्त्रयेत॒ सुरू॑पवर्.षवर्ण॒ एहीति॑ प॒शवो॒ वा इडा॑ प॒शूने॒वोप॑ ह्वयते य॒ज्ञ्ं ॅवै दे॒वा अदु॑ह्रन्. य॒ज्ञोऽसु॑राꣳ अदुह॒त् तेऽसु॑रा ॒ज्ञ्दु॑ग्धाः॒ परा॑ऽभव॒न्॒. यो वै य॒ज्ञ्स्य॒ दोहं॑ ॅवि॒द्वान् - [ ] \newline

\textbf{Pada Paata} \newline

पा॒क॒य॒ज्ञ्मिति॑ पाक - य॒ज्ञ्म् । वै । अन्विति॑ । आहि॑ताग्ने॒रित्याहि॑त - अ॒ग्नेः॒ । प॒शवः॑ । उपेति॑ । ति॒ष्ठ॒न्ते॒ । इडा᳚ । खलु॑ । वै । पा॒क॒य॒ज्ञ् इति॑ पाक - य॒ज्ञ्ः । सा । ए॒षा । अ॒न्त॒रा । प्र॒या॒जा॒नू॒या॒जानिति॑ प्रयाज - अ॒नू॒या॒जान् । यज॑मानस्य । लो॒के । अव॑हि॒तेयव॑ - हि॒ता॒ । ताम् । आ॒ह्रि॒यमा॑णा॒मित्या᳚ - ह्रि॒यमा॑णाम् । अ॒भीति॑ । म॒न्त्र॒ये॒त॒ । सुरू॑पवर्.षवर्ण॒ इति॒ सुरू॑प -व॒र्॒.ष॒व॒र्णे॒ । एति॑ । इ॒हि॒ । इति॑ । प॒शवः॑ । वै । इडा᳚ । प॒शून् । ए॒व । उपेति॑ । ह्व॒य॒ते॒ । य॒ज्ञ्म् । वै । दे॒वाः । अदु॑ह्रन्न् । य॒ज्ञ्ः । असु॑रान् । अ॒दु॒ह॒त् । ते । असु॑राः । य॒ज्ञ्दु॑ग्धा॒ इति॑ य॒ज्ञ् - दु॒ग्धाः॒ । परेति॑ । अ॒भ॒व॒न्न् । यः । वै । य॒ज्ञ्स्य॑ । दोह᳚म् । वि॒द्वान् ।  \newline


\textbf{Krama Paata} \newline

पा॒क॒य॒ज्ञ्ं ॅवै । पा॒क॒य॒ज्ञ्मिति॑ पाक - य॒ज्ञ्म् । वा अनु॑ । अन्वाहि॑ताग्नेः । आहि॑ताग्नेः प॒शवः॑ । आहि॑ताग्ने॒रित्याहि॑त - अ॒ग्नेः॒ । प॒शव॒ उप॑ । उप॑ तिष्ठन्ते । ति॒ष्ठ॒न्त॒ इडा᳚ । इडा॒ खलु॑ । खलु॒ वै । वै पा॑कय॒ज्ञ्ः । पा॒क॒य॒ज्ञ्ः सा । पा॒क॒य॒ज्ञ् इति॑ पाक - य॒ज्ञ्ः । सै॒षा । ए॒षाऽन्त॒रा । अ॒न्त॒रा प्र॑याजानूया॒जान् । प्र॒या॒जा॒नू॒या॒जान्. यज॑मानस्य । प्र॒या॒जा॒नू॒या॒जानिति॑ प्रयाज - अ॒नू॒या॒जान् । यज॑मानस्य लो॒के । लो॒केऽव॑हिता । अव॑हिता॒ ताम् । अव॑हि॒तेत्यव॑ - हि॒ता॒ । तामा᳚ह्रि॒यमा॑णाम् । आ॒ह्रि॒यमा॑णाम॒भि । आ॒ह्रि॒यमा॑णा॒मित्या᳚ - ह्रि॒यमा॑णाम् । अ॒भि म॑न्त्रयेत । म॒न्त्र॒ये॒त॒ सुरू॑पवर्.षवर्णे । सुरू॑पवर्.षवर्ण॒ आ । सुरू॑पवर्.षवर्ण॒ इति॒ सुरू॑प - व॒र्॒.ष॒व॒र्णे॒ । एहि॑ । इ॒हीति॑ । इति॑ प॒शवः॑ । प॒शवो॒ वै । वा इडा᳚ । इडा॑ प॒शून् । प॒शूने॒व । ए॒वोप॑ । उप॑ ह्वयते । ह्व॒य॒ते॒ य॒ज्ञ्म् । य॒ज्ञ्ं ॅवै । वै दे॒वाः । दे॒वा अदु॑ह्रन्न् । अदु॑ह्रन्. य॒ज्ञ्ः । य॒ज्ञो ऽसु॑रान् । असु॑राꣳ अदुहत् । अ॒दु॒ह॒त् ते । तेऽसु॑राः । असु॑रा य॒ज्ञ्दु॑ग्धाः । य॒ज्ञ्दु॑ग्धाः॒ परा᳚ । य॒ज्ञ्दु॑ग्धा॒ इति॑ य॒ज्ञ् - दु॒ग्धाः॒ । परा॑ऽभवन्न् । अ॒भ॒व॒न्. यः । यो वै । वै य॒ज्ञ्स्य॑ । य॒ज्ञ्स्य॒ दोह᳚म् । दोहं॑ ॅवि॒द्वान् । वि॒द्वान्. यज॑ते \newline

\textbf{Jatai Paata} \newline

1. पा॒क॒य॒ज्ञ्ं ॅवै वै पा॑कय॒ज्ञ्म् पा॑कय॒ज्ञ्ं ॅवै । \newline
2. पा॒क॒य॒ज्ञ्मिति॑ पाक - य॒ज्ञ्म् । \newline
3. वा अन्वनु॒ वै वा अनु॑ । \newline
4. अन्वाहि॑ताग्ने॒ राहि॑ताग्ने॒ रन्वन्वा हि॑ताग्नेः । \newline
5. आहि॑ताग्नेः प॒शवः॑ प॒शव॒ आहि॑ताग्ने॒ राहि॑ताग्नेः प॒शवः॑ । \newline
6. आहि॑ताग्ने॒रित्याहि॑त - अ॒ग्नेः॒ । \newline
7. प॒शव॒ उपोप॑ प॒शवः॑ प॒शव॒ उप॑ । \newline
8. उप॑ तिष्ठन्ते तिष्ठन्त॒ उपोप॑ तिष्ठन्ते । \newline
9. ति॒ष्ठ॒न्त॒ इडेडा॑ तिष्ठन्ते तिष्ठन्त॒ इडा᳚ । \newline
10. इडा॒ खलु॒ खल्विडेडा॒ खलु॑ । \newline
11. खलु॒ वै वै खलु॒ खलु॒ वै । \newline
12. वै पा॑कय॒ज्ञ्ः पा॑कय॒ज्ञो वै वै पा॑कय॒ज्ञ्ः । \newline
13. पा॒क॒य॒ज्ञ्ः सा सा पा॑कय॒ज्ञ्ः पा॑कय॒ज्ञ्ः सा । \newline
14. पा॒क॒य॒ज्ञ् इति॑ पाक - य॒ज्ञ्ः । \newline
15. सैषैषा सा सैषा । \newline
16. ए॒षा ऽन्त॒रा ऽन्त॒ रैषैषा ऽन्त॒रा । \newline
17. अ॒न्त॒रा प्र॑याजानूया॒जान् प्र॑याजानूया॒जा न॑न्त॒रा ऽन्त॒रा प्र॑याजानूया॒जान् । \newline
18. प्र॒या॒जा॒नू॒या॒जान्. यज॑मानस्य॒ यज॑मानस्य प्रयाजानूया॒जान् प्र॑याजानूया॒जान्. यज॑मानस्य । \newline
19. प्र॒या॒जा॒नू॒या॒जानिति॑ प्रयाज - अ॒नू॒या॒जान् । \newline
20. यज॑मानस्य लो॒के लो॒के यज॑मानस्य॒ यज॑मानस्य लो॒के । \newline
21. लो॒के ऽव॑हि॒ता ऽव॑हिता लो॒के लो॒के ऽव॑हिता । \newline
22. अव॑हिता॒ ताम् ता मव॑हि॒ता ऽव॑हिता॒ ताम् । \newline
23. अव॑हि॒तेत्यव॑ - हि॒ता॒ । \newline
24. ता मा᳚ह्रि॒यमा॑णा माह्रि॒यमा॑णा॒म् ताम् ता मा᳚ह्रि॒यमा॑णाम् । \newline
25. आ॒ह्रि॒यमा॑णा म॒भ्या᳚(1॒)भ्या᳚ह्रि॒यमा॑णा माह्रि॒यमा॑णा म॒भि । \newline
26. आ॒ह्रि॒यमा॑णा॒मित्या᳚ - ह्रि॒यमा॑णाम् । \newline
27. अ॒भि म॑न्त्रयेत मन्त्रयेता॒ भ्य॑भि म॑न्त्रयेत । \newline
28. म॒न्त्र॒ये॒त॒ सुरू॑पवर्.षवर्णे॒ सुरू॑पवर्.षवर्णे मन्त्रयेत मन्त्रयेत॒ सुरू॑पवर्.षवर्णे । \newline
29. सुरू॑पवर्.षवर्ण॒ आ सुरू॑पवर्.षवर्णे॒ सुरू॑पवर्.षवर्ण॒ आ । \newline
30. सुरू॑पवर्.षवर्ण॒ इति॒ सुरू॑प - व॒र्॒.ष॒व॒र्णे॒ । \newline
31. एही॒ह्येहि॑ । \newline
32. इ॒ही तीती॑ ही॒ही ति॑ । \newline
33. इति॑ प॒शवः॑ प॒शव॒ इतीति॑ प॒शवः॑ । \newline
34. प॒शवो॒ वै वै प॒शवः॑ प॒शवो॒ वै । \newline
35. वा इडेडा॒ वै वा इडा᳚ । \newline
36. इडा॑ प॒शून् प॒शू निडेडा॑ प॒शून् । \newline
37. प॒शू ने॒वैव प॒शून् प॒शू ने॒व । \newline
38. ए॒वो पोपै॒ वैवो प॑ । \newline
39. उप॑ ह्वयते ह्वयत॒ उपोप॑ ह्वयते । \newline
40. ह्व॒य॒ते॒ य॒ज्ञ्ं ॅय॒ज्ञ्ꣳ ह्व॑यते ह्वयते य॒ज्ञ्म् । \newline
41. य॒ज्ञ्ं ॅवै वै य॒ज्ञ्ं ॅय॒ज्ञ्ं ॅवै । \newline
42. वै दे॒वा दे॒वा वै वै दे॒वाः । \newline
43. दे॒वा अदु॑ह्र॒न् नदु॑ह्रन् दे॒वा दे॒वा अदु॑ह्रन्न् । \newline
44. अदु॑ह्रन्. य॒ज्ञो य॒ज्ञो ऽदु॑ह्र॒न् नदु॑ह्रन्. य॒ज्ञ्ः । \newline
45. य॒ज्ञो ऽसु॑राꣳ॒॒ असु॑रान्. य॒ज्ञो य॒ज्ञो ऽसु॑रान् । \newline
46. असु॑राꣳ अदुह ददुह॒ दसु॑राꣳ॒॒ असु॑राꣳ अदुहत् । \newline
47. अ॒दु॒ह॒त् ते ते॑ ऽदुह ददुह॒त् ते । \newline
48. ते ऽसु॑रा॒ असु॑रा॒ स्ते ते ऽसु॑राः । \newline
49. असु॑रा य॒ज्ञ्दु॑ग्धा य॒ज्ञ्दु॑ग्धा॒ असु॑रा॒ असु॑रा य॒ज्ञ्दु॑ग्धाः । \newline
50. य॒ज्ञ्दु॑ग्धाः॒ परा॒ परा॑ य॒ज्ञ्दु॑ग्धा य॒ज्ञ्दु॑ग्धाः॒ परा᳚ । \newline
51. य॒ज्ञ्दु॑ग्धा॒ इति॑ य॒ज्ञ् - दु॒ग्धाः॒ । \newline
52. परा॑ ऽभवन् नभव॒न् परा॒ परा॑ ऽभवन्न् । \newline
53. अ॒भ॒व॒न्॒. यो यो॑ ऽभवन् नभव॒न्॒. यः । \newline
54. यो वै वै यो यो वै । \newline
55. वै य॒ज्ञ्स्य॑ य॒ज्ञ्स्य॒ वै वै य॒ज्ञ्स्य॑ । \newline
56. य॒ज्ञ्स्य॒ दोह॒म् दोहं॑ ॅय॒ज्ञ्स्य॑ य॒ज्ञ्स्य॒ दोह᳚म् । \newline
57. दोहं॑ ॅवि॒द्वान्. वि॒द्वान् दोह॒म् दोहं॑ ॅवि॒द्वान् । \newline
58. वि॒द्वान्. यज॑ते॒ यज॑ते वि॒द्वान्. वि॒द्वान्. यज॑ते । \newline

\textbf{Ghana Paata } \newline

1. पा॒क॒य॒ज्ञ्ं ॅवै वै पा॑कय॒ज्ञ्म् पा॑कय॒ज्ञ्ं ॅवा अन्वनु॒ वै पा॑कय॒ज्ञ्म् पा॑कय॒ज्ञ्ं ॅवा अनु॑ । \newline
2. पा॒क॒य॒ज्ञ्मिति॑ पाक - य॒ज्ञ्म् । \newline
3. वा अन्वनु॒ वै वा अन्वाहि॑ताग्ने॒ राहि॑ताग्ने॒ रनु॒ वै वा अन्वाहि॑ताग्नेः । \newline
4. अन्वाहि॑ताग्ने॒ राहि॑ताग्ने॒ रन्वन्वाहि॑ताग्नेः प॒शवः॑ प॒शव॒ आहि॑ताग्ने॒ रन्वन्वाहि॑ताग्नेः प॒शवः॑ । \newline
5. आहि॑ताग्नेः प॒शवः॑ प॒शव॒ आहि॑ताग्ने॒ राहि॑ताग्नेः प॒शव॒ उपोप॑ प॒शव॒ आहि॑ताग्ने॒ राहि॑ताग्नेः प॒शव॒ उप॑ । \newline
6. आहि॑ताग्ने॒रित्याहि॑त - अ॒ग्नेः॒ । \newline
7. प॒शव॒ उपोप॑ प॒शवः॑ प॒शव॒ उप॑ तिष्ठन्ते तिष्ठन्त॒ उप॑ प॒शवः॑ प॒शव॒ उप॑ तिष्ठन्ते । \newline
8. उप॑ तिष्ठन्ते तिष्ठन्त॒ उपोप॑ तिष्ठन्त॒ इडेडा॑ तिष्ठन्त॒ उपोप॑ तिष्ठन्त॒ इडा᳚ । \newline
9. ति॒ष्ठ॒न्त॒ इडेडा॑ तिष्ठन्ते तिष्ठन्त॒ इडा॒ खलु॒ खल्विडा॑ तिष्ठन्ते तिष्ठन्त॒ इडा॒ खलु॑ । \newline
10. इडा॒ खलु॒ खल्वि डेडा॒ खलु॒ वै वै खल्वि डेडा॒ खलु॒ वै । \newline
11. खलु॒ वै वै खलु॒ खलु॒ वै पा॑कय॒ज्ञ्ः पा॑कय॒ज्ञो वै खलु॒ खलु॒ वै पा॑कय॒ज्ञ्ः । \newline
12. वै पा॑कय॒ज्ञ्ः पा॑कय॒ज्ञो वै वै पा॑कय॒ज्ञ्ः सा सा पा॑कय॒ज्ञो वै वै पा॑कय॒ज्ञ्ः सा । \newline
13. पा॒क॒य॒ज्ञ्ः सा सा पा॑कय॒ज्ञ्ः पा॑कय॒ज्ञ्ः सैषैषा सा पा॑कय॒ज्ञ्ः पा॑कय॒ज्ञ्ः सैषा । \newline
14. पा॒क॒य॒ज्ञ् इति॑ पाक - य॒ज्ञ्ः । \newline
15. सैषैषा सा सैषा ऽन्त॒रा ऽन्त॒रैषा सा सैषा ऽन्त॒रा । \newline
16. ए॒षा ऽन्त॒रा ऽन्त॒रैषैषा ऽन्त॒रा प्र॑याजानूया॒जान् प्र॑याजानूया॒जा न॑न्त॒रैषैषा ऽन्त॒रा प्र॑याजानूया॒जान् । \newline
17. अ॒न्त॒रा प्र॑याजानूया॒जान् प्र॑याजानूया॒जा न॑न्त॒रा ऽन्त॒रा प्र॑याजानूया॒जान्. यज॑मानस्य॒ यज॑मानस्य प्रयाजानूया॒जा न॑न्त॒रा ऽन्त॒रा प्र॑याजानूया॒जान्. यज॑मानस्य । \newline
18. प्र॒या॒जा॒नू॒या॒जान्. यज॑मानस्य॒ यज॑मानस्य प्रयाजानूया॒जान् प्र॑याजानूया॒जान्. यज॑मानस्य लो॒के लो॒के यज॑मानस्य प्रयाजानूया॒जान् प्र॑याजानूया॒जान्. यज॑मानस्य लो॒के । \newline
19. प्र॒या॒जा॒नू॒या॒जानिति॑ प्रयाज - अ॒नू॒या॒जान् । \newline
20. यज॑मानस्य लो॒के लो॒के यज॑मानस्य॒ यज॑मानस्य लो॒के ऽव॑हि॒ता ऽव॑हिता लो॒के यज॑मानस्य॒ यज॑मानस्य 
लो॒के ऽव॑हिता । \newline
21. लो॒के ऽव॑हि॒ता ऽव॑हिता लो॒के लो॒के ऽव॑हिता॒ ताम् ता मव॑हिता लो॒के लो॒के ऽव॑हिता॒ ताम् । \newline
22. अव॑हिता॒ ताम् ता मव॑हि॒ता ऽव॑हिता॒ ता मा᳚ह्रि॒यमा॑णा माह्रि॒यमा॑णा॒म् ता मव॑हि॒ता ऽव॑हिता॒ ता मा᳚ह्रि॒यमा॑णाम् । \newline
23. अव॑हि॒तेत्यव॑ - हि॒ता॒ । \newline
24. ता मा᳚ह्रि॒यमा॑णा माह्रि॒यमा॑णा॒म् ताम् ता मा᳚ह्रि॒यमा॑णा म॒भ्या᳚(1॒)भ्या᳚ह्रि॒यमा॑णा॒म् ताम् ता मा᳚ह्रि॒यमा॑णा म॒भि । \newline
25. आ॒ह्रि॒यमा॑णा म॒भ्या᳚(1॒)भ्या᳚ह्रि॒यमा॑णा माह्रि॒यमा॑णा म॒भि म॑न्त्रयेत मन्त्रयेता॒ भ्या᳚ह्रि॒यमा॑णा माह्रि॒यमा॑णा म॒भि म॑न्त्रयेत । \newline
26. आ॒ह्रि॒यमा॑णा॒मित्या᳚ - ह्रि॒यमा॑णाम् । \newline
27. अ॒भि म॑न्त्रयेत मन्त्रयेता॒ भ्य॑भि म॑न्त्रयेत॒ सुरू॑पवर्.षवर्णे॒ सुरू॑पवर्.षवर्णे मन्त्रयेता॒ भ्य॑भि म॑न्त्रयेत॒ सुरू॑पवर्.षवर्णे । \newline
28. म॒न्त्र॒ये॒त॒ सुरू॑पवर्.षवर्णे॒ सुरू॑पवर्.षवर्णे मन्त्रयेत मन्त्रयेत॒ सुरू॑पवर्.षवर्ण॒ आ सुरू॑पवर्.षवर्णे मन्त्रयेत मन्त्रयेत॒ सुरू॑पवर्.षवर्ण॒ आ । \newline
29. सुरू॑पवर्.षवर्ण॒ आ सुरू॑पवर्.षवर्णे॒ सुरू॑पवर्.षवर्ण॒ एही॒ह्या सुरू॑पवर्.षवर्णे॒ सुरू॑पवर्.षवर्ण॒ एहि॑ । \newline
30. सुरू॑पवर्.षवर्ण॒ इति॒ सुरू॑प - व॒र्॒.ष॒व॒र्णे॒ । \newline
31. एही॒ ह्येहीती ती॒ह्येहीति॑ । \newline
32. इ॒ही तीती॑ ही॒हीति॑ प॒शवः॑ प॒शव॒ इती॑ही॒ हीति॑ प॒शवः॑ । \newline
33. इति॑ प॒शवः॑ प॒शव॒ इतीति॑ प॒शवो॒ वै वै प॒शव॒ इतीति॑ प॒शवो॒ वै । \newline
34. प॒शवो॒ वै वै प॒शवः॑ प॒शवो॒ वा इडेडा॒ वै प॒शवः॑ प॒शवो॒ वा इडा᳚ । \newline
35. वा इडेडा॒ वै वा इडा॑ प॒शून् प॒शू निडा॒ वै वा इडा॑ प॒शून् । \newline
36. इडा॑ प॒शून् प॒शू निडेडा॑ प॒शू ने॒वैव प॒शू निडेडा॑ प॒शू ने॒व । \newline
37. प॒शू ने॒वैव प॒शून् प॒शू ने॒वोपोपै॒व प॒शून् प॒शू ने॒वोप॑ । \newline
38. ए॒वोपो पै॒वैवोप॑ ह्वयते ह्वयत॒ उपै॒वैवोप॑ ह्वयते । \newline
39. उप॑ ह्वयते ह्वयत॒ उपोप॑ ह्वयते य॒ज्ञ्ं ॅय॒ज्ञ्ꣳ ह्व॑यत॒ उपोप॑ ह्वयते य॒ज्ञ्म् । \newline
40. ह्व॒य॒ते॒ य॒ज्ञ्ं ॅय॒ज्ञ्ꣳ ह्व॑यते ह्वयते य॒ज्ञ्ं ॅवै वै य॒ज्ञ्ꣳ ह्व॑यते ह्वयते य॒ज्ञ्ं ॅवै । \newline
41. य॒ज्ञ्ं ॅवै वै य॒ज्ञ्ं ॅय॒ज्ञ्ं ॅवै दे॒वा दे॒वा वै य॒ज्ञ्ं ॅय॒ज्ञ्ं ॅवै दे॒वाः । \newline
42. वै दे॒वा दे॒वा वै वै दे॒वा अदु॑ह्र॒न् नदु॑ह्रन् दे॒वा वै वै दे॒वा अदु॑ह्रन्न् । \newline
43. दे॒वा अदु॑ह्र॒न् नदु॑ह्रन् दे॒वा दे॒वा अदु॑ह्रन्. य॒ज्ञो य॒ज्ञो ऽदु॑ह्रन् दे॒वा दे॒वा अदु॑ह्रन्. य॒ज्ञ्ः । \newline
44. अदु॑ह्रन्. य॒ज्ञो य॒ज्ञो ऽदु॑ह्र॒न् नदु॑ह्रन्. य॒ज्ञो ऽसु॑रा॒(ग्म्॒) असु॑रान्. य॒ज्ञो ऽदु॑ह्र॒न् नदु॑ह्रन्. य॒ज्ञो ऽसु॑रान् । \newline
45. य॒ज्ञो ऽसु॑रा॒(ग्म्॒) असु॑रान्. य॒ज्ञो य॒ज्ञो ऽसु॑राꣳ अदुह ददुह॒ दसु॑रान्. य॒ज्ञो य॒ज्ञो ऽसु॑राꣳ अदुहत् । \newline
46. असु॑राꣳ अदुह ददुह॒ दसु॑रा॒(ग्म्॒) असु॑राꣳ अदुह॒त् ते ते॑ ऽदुह॒दसु॑रा॒(ग्म्॒) असु॑राꣳ अदुह॒त् ते । \newline
47. अ॒दु॒ह॒त् ते ते॑ ऽदुह ददुह॒त् ते ऽसु॑रा॒ असु॑रा॒स्ते॑ ऽदुह ददुह॒त् ते ऽसु॑राः । \newline
48. ते ऽसु॑रा॒ असु॑रा॒स्ते ते ऽसु॑रा य॒ज्ञ्दु॑ग्धा य॒ज्ञ्दु॑ग्धा॒ असु॑रा॒स्ते ते ऽसु॑रा य॒ज्ञ्दु॑ग्धाः । \newline
49. असु॑रा य॒ज्ञ्दु॑ग्धा य॒ज्ञ्दु॑ग्धा॒ असु॑रा॒ असु॑रा य॒ज्ञ्दु॑ग्धाः॒ परा॒ परा॑ य॒ज्ञ्दु॑ग्धा॒ असु॑रा॒ असु॑रा य॒ज्ञ्दु॑ग्धाः॒ परा᳚ । \newline
50. य॒ज्ञ्दु॑ग्धाः॒ परा॒ परा॑ य॒ज्ञ्दु॑ग्धा य॒ज्ञ्दु॑ग्धाः॒ परा॑ ऽभवन् नभव॒न् परा॑ य॒ज्ञ्दु॑ग्धा य॒ज्ञ्दु॑ग्धाः॒ परा॑ ऽभवन्न् । \newline
51. य॒ज्ञ्दु॑ग्धा॒ इति॑ य॒ज्ञ् - दु॒ग्धाः॒ । \newline
52. परा॑ ऽभवन् नभव॒न् परा॒ परा॑ ऽभव॒न्॒. यो यो॑ ऽभव॒न् परा॒ परा॑ ऽभव॒न्॒. यः । \newline
53. अ॒भ॒व॒न्॒. यो यो॑ ऽभवन् नभव॒न्॒. यो वै वै यो॑ ऽभवन् नभव॒न्॒. यो वै । \newline
54. यो वै वै यो यो वै य॒ज्ञ्स्य॑ य॒ज्ञ्स्य॒ वै यो यो वै य॒ज्ञ्स्य॑ । \newline
55. वै य॒ज्ञ्स्य॑ य॒ज्ञ्स्य॒ वै वै य॒ज्ञ्स्य॒ दोह॒म् दोहं॑ ॅय॒ज्ञ्स्य॒ वै वै य॒ज्ञ्स्य॒ दोह᳚म् । \newline
56. य॒ज्ञ्स्य॒ दोह॒म् दोहं॑ ॅय॒ज्ञ्स्य॑ य॒ज्ञ्स्य॒ दोहं॑ ॅवि॒द्वान्. वि॒द्वान् दोहं॑ ॅय॒ज्ञ्स्य॑ य॒ज्ञ्स्य॒ दोहं॑ ॅवि॒द्वान् । \newline
57. दोहं॑ ॅवि॒द्वान्. वि॒द्वान् दोह॒म् दोहं॑ ॅवि॒द्वान्. यज॑ते॒ यज॑ते वि॒द्वान् दोह॒म् दोहं॑ ॅवि॒द्वान्. यज॑ते । \newline
58. वि॒द्वान्. यज॑ते॒ यज॑ते वि॒द्वान्. वि॒द्वान्. यज॒ते ऽप्यपि॒ यज॑ते वि॒द्वान्. वि॒द्वान्. यज॒ते ऽपि॑ । \newline
\pagebreak
\markright{ TS 1.7.1.2  \hfill https://www.vedavms.in \hfill}
\addcontentsline{toc}{section}{ TS 1.7.1.2 }
\section*{ TS 1.7.1.2 }

\textbf{TS 1.7.1.2 } \newline
\textbf{Samhita Paata} \newline

यज॒तेऽप्य॒न्यं ॅयज॑मानं दुहे॒ सा मे॑ स॒त्याऽऽशीर॒स्य य॒ज्ञ्स्य॑ भूया॒दित्या॑है॒ष वै य॒ज्ञ्स्य॒ दोह॒स्तेनै॒वैनं॑ दुहे॒ प्रत्ता॒ वै गौर्दु॑हे॒ प्रत्तेडा॒ यज॑मानाय दुह ए॒ते वा इडा॑यै॒ स्तना॒ इडोप॑हू॒तेति॑ वा॒युर्व॒थ्सो यर्.हि॒ होतेडा॑मुप॒ह्वये॑त॒ तर्.हि॒ यज॑मानो॒ होता॑र॒मीक्ष॑माणो वा॒युं मन॑सा ध्यायेन्-[ ] \newline

\textbf{Pada Paata} \newline

यज॑ते । अपीति॑ । अ॒न्यम् । यज॑मानम् । दु॒हे॒ । सा । मे॒ । स॒त्या । आ॒शीरित्या᳚-शीः । अ॒स्य । य॒ज्ञ्स्य॑ । भू॒या॒त् । इति॑ । आ॒ह॒ । ए॒षः । वै । य॒ज्ञ्स्य॑ । दोहः॑ । तेन॑ । ए॒व । ए॒न॒म् । दु॒हे॒ । प्रत्ता᳚ । वै । गौः । दु॒हे॒ । प्रत्ता᳚ । इडा᳚ । यज॑मानाय । दु॒हे॒ । ए॒ते । वै । इडा॑यै । स्तनाः᳚ । इडा᳚ । उप॑हू॒तेत्युप॑ - हू॒ता॒ । इति॑ । वा॒युः । व॒थ्सः । यर्.हि॑ । होता᳚ । इडा᳚म् । उ॒प॒ह्वय॒तेत्यु॑प - ह्वये॑त । तर्.हि॑ । यज॑मानः । होता॑रम् । ईक्ष॑माणः । वा॒युम् । मन॑सा । ध्या॒ये॒त् ।  \newline


\textbf{Krama Paata} \newline

यज॒तेऽपि॑ । अप्य॒न्यम् । अ॒न्यं ॅयज॑मानम् । यज॑मानम् दुहे । दु॒हे॒ सा । सा मे᳚ । मे॒ स॒त्या । स॒त्याऽऽशीः । आ॒शीर॒स्य । आ॒शीरित्या᳚ - शीः । अ॒स्य य॒ज्ञ्स्य॑ । य॒ज्ञ्स्य॑ भूयात् । भू॒या॒दिति॑ । इत्या॑ह । आ॒है॒षः । ए॒ष वै । वै य॒ज्ञ्स्य॑ । य॒ज्ञ्स्य॒ दोहः॑ । दोह॒स्तेन॑ । तेनै॒व । ए॒वैन᳚म् । ए॒न॒म् दु॒हे॒ । दु॒हे॒ प्रत्ता᳚ । प्रत्ता॒ वै । वै गौः । गौर्,दु॑हे । दु॒हे॒ प्रत्ता᳚ । प्रत्तेडा᳚ । इडा॒ यज॑मानाय । यज॑मानाय दुहे । दु॒ह॒ ए॒ते । ए॒ते वै । वा इडा॑यै । इडा॑यै॒ स्तनाः᳚ । स्तना॒ इडा᳚ । इडोप॑हू॒ता । उप॑हू॒तेति॑ । उप॑हू॒तेत्युप॑ - हू॒ता॒ । इति॑ वा॒युः । वा॒युर्,व॒थ्सः । व॒थ्सो यर्.हि॑ । यर्.हि॒ होता᳚ । होतेडा᳚म् । इडा॑मुप॒ह्वये॑त । उ॒प॒ह्वये॑त॒ तर्.हि॑ । उ॒प॒ह्वये॒तेत्यु॑प - ह्वये॑त । तर्.हि॒ यज॑मानः । यज॑मानो॒ होता॑रम् । होता॑र॒मीक्ष॑माणः । ईक्ष॑माणो वा॒युम् । वा॒युम् मन॑सा । मन॑सा ध्यायेत् । ध्या॒ये॒न्,मा॒त्रे \newline

\textbf{Jatai Paata} \newline

1. यज॒ते ऽप्यपि॒ यज॑ते॒ यज॒ते ऽपि॑ । \newline
2. अप्य॒न्य म॒न्य मप्य प्य॒न्यम् । \newline
3. अ॒न्यं ॅयज॑मानं॒ ॅयज॑मान म॒न्य म॒न्यं ॅयज॑मानम् । \newline
4. यज॑मानम् दुहे दुहे॒ यज॑मानं॒ ॅयज॑मानम् दुहे । \newline
5. दु॒हे॒ सा सा दु॑हे दुहे॒ सा । \newline
6. सा मे॑ मे॒ सा सा मे᳚ । \newline
7. मे॒ स॒त्या स॒त्या मे॑ मे स॒त्या । \newline
8. स॒त्या ऽऽशी रा॒शीः स॒त्या स॒त्या ऽऽशीः । \newline
9. आ॒शी र॒स्यास्या शी रा॒शी र॒स्य । \newline
10. आ॒शीरित्या᳚ - शीः । \newline
11. अ॒स्य य॒ज्ञ्स्य॑ य॒ज्ञ्स्या॒ स्यास्य य॒ज्ञ्स्य॑ । \newline
12. य॒ज्ञ्स्य॑ भूयाद् भूयाद् य॒ज्ञ्स्य॑ य॒ज्ञ्स्य॑ भूयात् । \newline
13. भू॒या॒ दितीति॑ भूयाद् भूया॒ दिति॑ । \newline
14. इत्या॑हा॒हे तीत्या॑ह । \newline
15. आ॒है॒ष ए॒ष आ॑हा है॒षः । \newline
16. ए॒ष वै वा ए॒ष ए॒ष वै । \newline
17. वै य॒ज्ञ्स्य॑ य॒ज्ञ्स्य॒ वै वै य॒ज्ञ्स्य॑ । \newline
18. य॒ज्ञ्स्य॒ दोहो॒ दोहो॑ य॒ज्ञ्स्य॑ य॒ज्ञ्स्य॒ दोहः॑ । \newline
19. दोह॒स्तेन॒ तेन॒ दोहो॒ दोह॒ स्तेन॑ । \newline
20. तेनै॒ वैव तेन॒ तेनै॒व । \newline
21. ए॒वैन॑ मेन मे॒वै वैन᳚म् । \newline
22. ए॒न॒म् दु॒हे॒ दु॒ह॒ ए॒न॒ मे॒न॒म् दु॒हे॒ । \newline
23. दु॒हे॒ प्रत्ता॒ प्रत्ता॑ दुहे दुहे॒ प्रत्ता᳚ । \newline
24. प्रत्ता॒ वै वै प्रत्ता॒ प्रत्ता॒ वै । \newline
25. वै गौर् गौर् वै वै गौः । \newline
26. गौर् दु॑हे दुहे॒ गौर् गौर् दु॑हे । \newline
27. दु॒हे॒ प्रत्ता॒ प्रत्ता॑ दुहे दुहे॒ प्रत्ता᳚ । \newline
28. प्रत्तेडेडा॒ प्रत्ता॒ प्रत्तेडा᳚ । \newline
29. इडा॒ यज॑मानाय॒ यज॑माना॒ये डेडा॒ यज॑मानाय । \newline
30. यज॑मानाय दुहे दुहे॒ यज॑मानाय॒ यज॑मानाय दुहे । \newline
31. दु॒ह॒ ए॒त ए॒ते दु॑हे दुह ए॒ते । \newline
32. ए॒ते वै वा ए॒त ए॒ते वै । \newline
33. वा इडा॑या॒ इडा॑यै॒ वै वा इडा॑यै । \newline
34. इडा॑यै॒ स्तनाः॒ स्तना॒ इडा॑या॒ इडा॑यै॒ स्तनाः᳚ । \newline
35. स्तना॒ इडेडा॒ स्तनाः॒ स्तना॒ इडा᳚ । \newline
36. इडो प॑हू॒तो प॑हू॒ तेडेडो प॑हूता । \newline
37. उप॑हू॒तेती त्युप॑हू॒तो प॑हू॒ तेति॑ । \newline
38. उप॑हू॒तेत्युप॑ - हू॒ता॒ । \newline
39. इति॑ वा॒युर् वा॒यु रितीति॑ वा॒युः । \newline
40. वा॒युर् व॒थ्सो व॒थ्सो वा॒युर् वा॒युर् व॒थ्सः । \newline
41. व॒थ्सो यर्.हि॒ यर्.हि॑ व॒थ्सो व॒थ्सो यर्.हि॑ । \newline
42. यर्.हि॒ होता॒ होता॒ यर्.हि॒ यर्.हि॒ होता᳚ । \newline
43. होतेडा॒ मिडाꣳ॒॒ होता॒ होतेडा᳚म् । \newline
44. इडा॑ मुप॒ह्वये॑तो प॒ह्वये॒ते डा॒ मिडा॑ मुप॒ह्वये॑त । \newline
45. उ॒प॒ह्वये॑त॒ तर्.हि॒ तर्ह्यु॑प॒ह्वये॑ तोप॒ह्वये॑त॒ तर्.हि॑ । \newline
46. उ॒प॒ह्वय॒तेत्यु॑प - ह्वये॑त । \newline
47. तर्.हि॒ यज॑मानो॒ यज॑मान॒ स्तर्.हि॒ तर्.हि॒ यज॑मानः । \newline
48. यज॑मानो॒ होता॑रꣳ॒॒ होता॑रं॒ ॅयज॑मानो॒ यज॑मानो॒ होता॑रम् । \newline
49. होता॑र॒ मीक्ष॑माण॒ ईक्ष॑माणो॒ होता॑रꣳ॒॒ होता॑र॒ मीक्ष॑माणः । \newline
50. ईक्ष॑माणो वा॒युं ॅवा॒यु मीक्ष॑माण॒ ईक्ष॑माणो वा॒युम् । \newline
51. वा॒युम् मन॑सा॒ मन॑सा वा॒युं ॅवा॒युम् मन॑सा । \newline
52. मन॑सा ध्यायेद् ध्याये॒न् मन॑सा॒ मन॑सा ध्यायेत् । \newline
53. ध्या॒ये॒न् मा॒त्रे मा॒त्रे ध्या॑येद् ध्यायेन् मा॒त्रे । \newline

\textbf{Ghana Paata } \newline

1. यज॒ते ऽप्यपि॒ यज॑ते॒ यज॒ते ऽप्य॒न्य म॒न्य मपि॒ यज॑ते॒ यज॒ते ऽप्य॒न्यम् । \newline
2. अप्य॒न्य म॒न्य मप्यप्य॒न्यं ॅयज॑मानं॒ ॅयज॑मान म॒न्य मप्यप्य॒न्यं ॅयज॑मानम् । \newline
3. अ॒न्यं ॅयज॑मानं॒ ॅयज॑मान म॒न्य म॒न्यं ॅयज॑मानम् दुहे दुहे॒ यज॑मान म॒न्य म॒न्यं ॅयज॑मानम् दुहे । \newline
4. यज॑मानम् दुहे दुहे॒ यज॑मानं॒ ॅयज॑मानम् दुहे॒ सा सा दु॑हे॒ यज॑मानं॒ ॅयज॑मानम् दुहे॒ सा । \newline
5. दु॒हे॒ सा सा दु॑हे दुहे॒ सा मे॑ मे॒ सा दु॑हे दुहे॒ सा मे᳚ । \newline
6. सा मे॑ मे॒ सा सा मे॑ स॒त्या स॒त्या मे॒ सा सा मे॑ स॒त्या । \newline
7. मे॒ स॒त्या स॒त्या मे॑ मे स॒त्या ऽऽशीरा॒शीः स॒त्या मे॑ मे स॒त्या ऽऽशीः । \newline
8. स॒त्या ऽऽशीरा॒शीः स॒त्या स॒त्या ऽऽशी र॒स्यास्याशीः स॒त्या स॒त्या ऽऽशीर॒स्य । \newline
9. आ॒शी र॒स्यास्याशी रा॒शीर॒स्य य॒ज्ञ्स्य॑ य॒ज्ञ्स्या॒स्याशी रा॒शीर॒स्य य॒ज्ञ्स्य॑ । \newline
10. आ॒शीरित्या᳚ - शीः । \newline
11. अ॒स्य य॒ज्ञ्स्य॑ य॒ज्ञ्स्या॒ स्यास्य य॒ज्ञ्स्य॑ भूयाद् भूयाद् य॒ज्ञ्स्या॒ स्यास्य य॒ज्ञ्स्य॑ भूयात् । \newline
12. य॒ज्ञ्स्य॑ भूयाद् भूयाद् य॒ज्ञ्स्य॑ य॒ज्ञ्स्य॑ भूया॒दितीति॑ भूयाद् य॒ज्ञ्स्य॑ य॒ज्ञ्स्य॑ भूया॒दिति॑ । \newline
13. भू॒या॒दितीति॑ भूयाद् भूया॒ दित्या॑हा॒हे ति॑ भूयाद् भूया॒ दित्या॑ह । \newline
14. इत्या॑हा॒हे तीत्या॑है॒ष ए॒ष आ॒हे तीत्या॑है॒षः । \newline
15. आ॒है॒ष ए॒ष आ॑हाहै॒ष वै वा ए॒ष आ॑हाहै॒ष वै । \newline
16. ए॒ष वै वा ए॒ष ए॒ष वै य॒ज्ञ्स्य॑ य॒ज्ञ्स्य॒ वा ए॒ष ए॒ष वै य॒ज्ञ्स्य॑ । \newline
17. वै य॒ज्ञ्स्य॑ य॒ज्ञ्स्य॒ वै वै य॒ज्ञ्स्य॒ दोहो॒ दोहो॑ य॒ज्ञ्स्य॒ वै वै य॒ज्ञ्स्य॒ दोहः॑ । \newline
18. य॒ज्ञ्स्य॒ दोहो॒ दोहो॑ य॒ज्ञ्स्य॑ य॒ज्ञ्स्य॒ दोह॒ स्तेन॒ तेन॒ दोहो॑ य॒ज्ञ्स्य॑ य॒ज्ञ्स्य॒ दोह॒ स्तेन॑ । \newline
19. दोह॒ स्तेन॒ तेन॒ दोहो॒ दोह॒ स्तेनै॒वैव तेन॒ दोहो॒ दोह॒ स्तेनै॒व । \newline
20. तेनै॒वैव तेन॒ तेनै॒वैन॑ मेन मे॒व तेन॒ तेनै॒वैन᳚म् । \newline
21. ए॒वैन॑ मेन मे॒वै वैन॑म् दुहे दुह एन मे॒वै वैन॑म् दुहे । \newline
22. ए॒न॒म् दु॒हे॒ दु॒ह॒ ए॒न॒ मे॒न॒म् दु॒हे॒ प्रत्ता॒ प्रत्ता॑ दुह एन मेनम् दुहे॒ प्रत्ता᳚ । \newline
23. दु॒हे॒ प्रत्ता॒ प्रत्ता॑ दुहे दुहे॒ प्रत्ता॒ वै वै प्रत्ता॑ दुहे दुहे॒ प्रत्ता॒ वै । \newline
24. प्रत्ता॒ वै वै प्रत्ता॒ प्रत्ता॒ वै गौर् गौर् वै प्रत्ता॒ प्रत्ता॒ वै गौः । \newline
25. वै गौर् गौर् वै वै गौर् दु॑हे दुहे॒ गौर् वै वै गौर् दु॑हे । \newline
26. गौर् दु॑हे दुहे॒ गौर् गौर् दु॑हे॒ प्रत्ता॒ प्रत्ता॑ दुहे॒ गौर् गौर् दु॑हे॒ प्रत्ता᳚ । \newline
27. दु॒हे॒ प्रत्ता॒ प्रत्ता॑ दुहे दुहे॒ प्रत्तेडेडा॒ प्रत्ता॑ दुहे दुहे॒ प्रत्तेडा᳚ । \newline
28. प्रत्तेडेडा॒ प्रत्ता॒ प्रत्तेडा॒ यज॑मानाय॒ यज॑माना॒ये डा॒ प्रत्ता॒ प्रत्तेडा॒ यज॑मानाय । \newline
29. इडा॒ यज॑मानाय॒ यज॑माना॒ये डेडा॒ यज॑मानाय दुहे दुहे॒ यज॑माना॒ये डेडा॒ यज॑मानाय दुहे । \newline
30. यज॑मानाय दुहे दुहे॒ यज॑मानाय॒ यज॑मानाय दुह ए॒त ए॒ते दु॑हे॒ यज॑मानाय॒ यज॑मानाय दुह ए॒ते । \newline
31. दु॒ह॒ ए॒त ए॒ते दु॑हे दुह ए॒ते वै वा ए॒ते दु॑हे दुह ए॒ते वै । \newline
32. ए॒ते वै वा ए॒त ए॒ते वा इडा॑या॒ इडा॑यै॒ वा ए॒त ए॒ते वा इडा॑यै । \newline
33. वा इडा॑या॒ इडा॑यै॒ वै वा इडा॑यै॒ स्तनाः॒ स्तना॒ इडा॑यै॒ वै वा इडा॑यै॒ स्तनाः᳚ । \newline
34. इडा॑यै॒ स्तनाः॒ स्तना॒ इडा॑या॒ इडा॑यै॒ स्तना॒ इडेडा॒ स्तना॒ इडा॑या॒ इडा॑यै॒ स्तना॒ इडा᳚ । \newline
35. स्तना॒ इडेडा॒ स्तनाः॒ स्तना॒ इडोप॑हू॒तो प॑हू॒तेडा॒ स्तनाः॒ स्तना॒ इडोप॑हूता । \newline
36. इडोप॑हू॒तो प॑हू॒ तेडेडोप॑हू॒तेती त्युप॑हू॒ तेडेडो प॑हू॒तेति॑ । \newline
37. उप॑हू॒तेती त्युप॑हू॒तो प॑हू॒तेति॑ वा॒युर् वा॒युरि त्युप॑हू॒तो प॑हू॒तेति॑ वा॒युः । \newline
38. उप॑हू॒तेत्युप॑ - हू॒ता॒ । \newline
39. इति॑ वा॒युर् वा॒यु रितीति॑ वा॒युर् व॒थ्सो व॒थ्सो वा॒युरितीति॑ वा॒युर् व॒थ्सः । \newline
40. वा॒युर् व॒थ्सो व॒थ्सो वा॒युर् वा॒युर् व॒थ्सो यर्.हि॒ यर्.हि॑ व॒थ्सो वा॒युर् वा॒युर् व॒थ्सो यर्.हि॑ । \newline
41. व॒थ्सो यर्.हि॒ यर्.हि॑ व॒थ्सो व॒थ्सो यर्.हि॒ होता॒ होता॒ यर्.हि॑ व॒थ्सो व॒थ्सो यर्.हि॒ होता᳚ । \newline
42. यर्.हि॒ होता॒ होता॒ यर्.हि॒ यर्.हि॒ होतेडा॒ मिडा॒(ग्म्॒) होता॒ यर्.हि॒ यर्.हि॒ होतेडा᳚म् । \newline
43. होतेडा॒ मिडा॒(ग्म्॒) होता॒ होतेडा॑ मुप॒ह्वये॑तो प॒ह्वये॒ते डा॒(ग्म्॒) होता॒ होतेडा॑ मुप॒ह्वये॑त । \newline
44. इडा॑ मुप॒ह्वये॑तो प॒ह्वये॒ते डा॒ मिडा॑ मुप॒ह्वये॑त॒ तर्.हि॒ तर्ह्यु॑ प॒ह्वये॒ते डा॒ मिडा॑ मुप॒ह्वये॑त॒ तर्.हि॑ । \newline
45. उ॒प॒ह्वये॑त॒ तर्.हि॒ तर्ह्यु॑ प॒ह्वये॑तो प॒ह्वये॑त॒ तर्.हि॒ यज॑मानो॒ यज॑मान॒ स्तर्ह्यु॑ प॒ह्वये॑तो प॒ह्वये॑त॒ तर्.हि॒ यज॑मानः । \newline
46. उ॒प॒ह्वय॒तेत्यु॑प - ह्वये॑त । \newline
47. तर्.हि॒ यज॑मानो॒ यज॑मान॒ स्तर्.हि॒ तर्.हि॒ यज॑मानो॒ होता॑र॒(ग्म्॒) होता॑रं॒ ॅयज॑मान॒ स्तर्.हि॒ तर्.हि॒ यज॑मानो॒ होता॑रम् । \newline
48. यज॑मानो॒ होता॑र॒(ग्म्॒) होता॑रं॒ ॅयज॑मानो॒ यज॑मानो॒ होता॑र॒ मीक्ष॑माण॒ ईक्ष॑माणो॒ होता॑रं॒ ॅयज॑मानो॒ यज॑मानो॒ होता॑र॒ मीक्ष॑माणः । \newline
49. होता॑र॒ मीक्ष॑माण॒ ईक्ष॑माणो॒ होता॑र॒(ग्म्॒) होता॑र॒ मीक्ष॑माणो वा॒युं ॅवा॒यु मीक्ष॑माणो॒ होता॑र॒(ग्म्॒) होता॑र॒ मीक्ष॑माणो वा॒युम् । \newline
50. ईक्ष॑माणो वा॒युं ॅवा॒यु मीक्ष॑माण॒ ईक्ष॑माणो वा॒युम् मन॑सा॒ मन॑सा वा॒यु मीक्ष॑माण॒ ईक्ष॑माणो वा॒युम् मन॑सा । \newline
51. वा॒युम् मन॑सा॒ मन॑सा वा॒युं ॅवा॒युम् मन॑सा ध्यायेद् ध्याये॒न् मन॑सा वा॒युं ॅवा॒युम् मन॑सा ध्यायेत् । \newline
52. मन॑सा ध्यायेद् ध्याये॒न् मन॑सा॒ मन॑सा ध्यायेन् मा॒त्रे मा॒त्रे ध्या॑ये॒न् मन॑सा॒ मन॑सा ध्यायेन् मा॒त्रे । \newline
53. ध्या॒ये॒न् मा॒त्रे मा॒त्रे ध्या॑येद् ध्यायेन् मा॒त्रे व॒थ्सं ॅव॒थ्सम् मा॒त्रे ध्या॑येद् ध्यायेन् मा॒त्रे व॒थ्सम् । \newline
\pagebreak
\markright{ TS 1.7.1.3  \hfill https://www.vedavms.in \hfill}
\addcontentsline{toc}{section}{ TS 1.7.1.3 }
\section*{ TS 1.7.1.3 }

\textbf{TS 1.7.1.3 } \newline
\textbf{Samhita Paata} \newline

मा॒त्रे व॒थ्स-मु॒पाव॑सृजति॒ सर्वे॑ण॒ वै य॒ज्ञेन॑ दे॒वाः सु॑व॒र्गं ॅलो॒कमा॑यन् पाकय॒ज्ञेन॒ मनु॑रश्राम्य॒थ्सेडा॒ मनु॑मु॒पाव॑र्तत॒ तां दे॑वासु॒रा व्य॑ह्वयन्त प्र॒तीचीं᳚ दे॒वाः परा॑ची॒मसु॑राः॒ सा दे॒वानु॒पाव॑र्तत प॒शवो॒ वै तद् दे॒वान॑वृणत प॒शवोऽसु॑रानजहु॒र्यं का॒मये॑ताप॒शुः स्या॒दिति॒ परा॑चीं॒ तस्येडा॒मुप॑ ह्वयेताप॒शुरे॒व भ॑वति॒ यं-[ ] \newline

\textbf{Pada Paata} \newline

मा॒त्रे । व॒थ्सम् । उ॒पाव॑सृज॒तीत्यु॑प - अव॑सृजति । सर्वे॑ण । वै । य॒ज्ञेन॑ । दे॒वाः । सु॒व॒र्गमिति॑ सुवः - गम् । लो॒कम् । आ॒य॒न्न् । पा॒क॒य॒ज्ञेनेति॑ पाक-य॒ज्ञेन॑ । मनुः॑ । अ॒श्रा॒म्य॒त् । सा । इडा᳚ । मनु᳚म् । उ॒पाव॑र्त॒तेत्यु॑प - आव॑र्तत । ताम् । दे॒वा॒सु॒रा इति॑ देव - अ॒सु॒राः । वीति॑ । अ॒ह्व॒य॒न्त॒ । प्र॒तीची᳚म् । दे॒वाः । परा॑चीम् । असु॑राः । सा । दे॒वान् । उ॒पाव॑र्त॒तेत्यु॑प - आव॑र्तत । प॒शवः॑ । वै । तत् । दे॒वान् । अ॒वृ॒ण॒त॒ । प॒शवः॑ । असु॑रान् । अ॒ज॒हुः॒ । यम् । का॒मये॑त । अ॒प॒शुः । स्या॒त् । इति॑ । परा॑चीम् । तस्य॑ । इडा᳚म् । उपेति॑ । ह्व॒ये॒त॒ । अ॒प॒शुः । ए॒व । भ॒व॒ति॒ । यम् ।  \newline


\textbf{Krama Paata} \newline

मा॒त्रे व॒थ्सम् । व॒थ्समु॒पाव॑सृजति । उ॒पाव॑सृजति॒ सर्वे॑ण । उ॒पाव॑सृज॒तीत्यु॑प - अव॑सृजति । सर्वे॑ण॒ वै । वै य॒ज्ञेन॑ । य॒ज्ञेन॑ दे॒वाः । दे॒वाः सु॑व॒र्गम् । सु॒व॒र्गं ॅलो॒कम् । सु॒व॒र्गमिति॑ सुवः - गम् । लो॒कमा॑यन्न् । आ॒य॒न् पा॒क॒य॒ज्ञेन॑ । पा॒क॒य॒ज्ञेन॒ मनुः॑ । पा॒क॒य॒ज्ञेनेति॑ पाक - य॒ज्ञेन॑ । मनु॑रश्राम्यत् । अ॒श्रा॒म्य॒थ् सा । सेडा᳚ । इडा॒ मनु᳚म् । मनु॑मु॒पाव॑र्तत । उ॒पाव॑र्तत॒ ताम् । उ॒पाव॑र्त॒तेत्यु॑प - आव॑र्तत । ताम्,दे॑वासु॒राः । दे॒वा॒सु॒रा वि । दे॒वा॒सु॒रा इति॑ देव - अ॒सु॒राः । व्य॑ह्वयन्त । अ॒ह्व॒य॒न्त॒ प्र॒तीची᳚म् । प्र॒तीची᳚म् दे॒वाः । दे॒वाः परा॑चीम् । परा॑ची॒मसु॑राः । असु॑राः॒ सा । सा दे॒वान् । दे॒वानु॒पाव॑र्तत । उ॒पाव॑र्तत प॒शवः॑ । उ॒पाव॑र्त॒तेत्यु॑प - आव॑र्तत । प॒शवो॒ वै । वै तत् । तद् दे॒वान् । दे॒वान॑वृणत । अ॒वृ॒ण॒त॒ प॒शवः॑ । प॒शवो ऽसु॑रान् । असु॑रानजहुः । अ॒ज॒हु॒र्,यम् । यम् का॒मये॑त । का॒मये॑ताप॒शुः । अ॒प॒शुः स्या᳚त् । स्या॒दिति॑ । इति॒ परा॑चीम् । परा॑ची॒म् तस्य॑ । तस्येडा᳚म् । इडा॒मुप॑ । उप॑ ह्वयेत । ह्व॒ये॒ता॒प॒शुः । अ॒प॒शुरे॒व । ए॒व भ॑वति । भ॒व॒ति॒ यम् । यम् का॒मये॑त \newline

\textbf{Jatai Paata} \newline

1. मा॒त्रे व॒थ्सं ॅव॒थ्सम् मा॒त्रे मा॒त्रे व॒थ्सम् । \newline
2. व॒थ्स मु॒पाव॑सृज त्यु॒पाव॑सृजति व॒थ्सं ॅव॒थ्स मु॒पाव॑सृजति । \newline
3. उ॒पाव॑सृजति॒ सर्वे॑ण॒ सर्वे॑ णो॒पाव॑सृज त्यु॒पाव॑सृजति॒ सर्वे॑ण । \newline
4. उ॒पाव॑सृज॒तीत्यु॑प - अव॑सृजति । \newline
5. सर्वे॑ण॒ वै वै सर्वे॑ण॒ सर्वे॑ण॒ वै । \newline
6. वै य॒ज्ञेन॑ य॒ज्ञेन॒ वै वै य॒ज्ञेन॑ । \newline
7. य॒ज्ञेन॑ दे॒वा दे॒वा य॒ज्ञेन॑ य॒ज्ञेन॑ दे॒वाः । \newline
8. दे॒वाः सु॑व॒र्गꣳ सु॑व॒र्गम् दे॒वा दे॒वाः सु॑व॒र्गम् । \newline
9. सु॒व॒र्गम् ॅलो॒कम् ॅलो॒कꣳ सु॑व॒र्गꣳ सु॑व॒र्गम् ॅलो॒कम् । \newline
10. सु॒व॒र्गमिति॑ सुवः - गम् । \newline
11. लो॒क मा॑यन् नायन् ॅलो॒कम् ॅलो॒क मा॑यन्न् । \newline
12. आ॒य॒न् पा॒क॒य॒ज्ञेन॑ पाकय॒ज्ञेना॑यन् नायन् पाकय॒ज्ञेन॑ । \newline
13. पा॒क॒य॒ज्ञेन॒ मनु॒र् मनुः॑ पाकय॒ज्ञेन॑ पाकय॒ज्ञेन॒ मनुः॑ । \newline
14. पा॒क॒य॒ज्ञेनेति॑ पाक - य॒ज्ञेन॑ । \newline
15. मनु॑ रश्राम्य दश्राम्य॒न् मनु॒र् मनु॑ रश्राम्यत् । \newline
16. अ॒श्रा॒म्य॒थ् सा सा ऽश्रा᳚म्य दश्राम्य॒थ् सा । \newline
17. सेडेडा॒ सा सेडा᳚ । \newline
18. इडा॒ मनु॒म् मनु॒ मिडेडा॒ मनु᳚म् । \newline
19. मनु॑ मु॒पाव॑र्ततो॒ पाव॑र्तत॒ मनु॒म् मनु॑ मु॒पाव॑र्तत । \newline
20. उ॒पाव॑र्तत॒ ताम् ता मु॒पाव॑र्त तो॒पाव॑र्तत॒ ताम् । \newline
21. उ॒पाव॑र्त॒तेत्यु॑प - आव॑र्तत । \newline
22. ताम् दे॑वासु॒रा दे॑वासु॒रा स्ताम् ताम् दे॑वासु॒राः । \newline
23. दे॒वा॒सु॒रा वि वि दे॑वासु॒रा दे॑वासु॒रा वि । \newline
24. दे॒वा॒सु॒रा इति॑ देव - अ॒सु॒राः । \newline
25. व्य॑ह्वयन्ता ह्वयन्त॒ वि व्य॑ह्वयन्त । \newline
26. अ॒ह्व॒य॒न्त॒ प्र॒तीची᳚म् प्र॒तीची॑ मह्वयन्ता ह्वयन्त प्र॒तीची᳚म् । \newline
27. प्र॒तीची᳚म् दे॒वा दे॒वाः प्र॒तीची᳚म् प्र॒तीची᳚म् दे॒वाः । \newline
28. दे॒वाः परा॑ची॒म् परा॑चीम् दे॒वा दे॒वाः परा॑चीम् । \newline
29. परा॑ची॒ मसु॑रा॒ असु॑राः॒ परा॑ची॒म् परा॑ची॒ मसु॑राः । \newline
30. असु॑राः॒ सा सा ऽसु॑रा॒ असु॑राः॒ सा । \newline
31. सा दे॒वान् दे॒वान् थ्सा सा दे॒वान् । \newline
32. दे॒वा नु॒पाव॑र्ततो॒ पाव॑र्तत दे॒वान् दे॒वा नु॒पाव॑र्तत । \newline
33. उ॒पाव॑र्तत प॒शवः॑ प॒शव॑ उ॒पाव॑र्ततो॒ पाव॑र्तत प॒शवः॑ । \newline
34. उ॒पाव॑र्त॒तेत्यु॑प - आव॑र्तत । \newline
35. प॒शवो॒ वै वै प॒शवः॑ प॒शवो॒ वै । \newline
36. वै तत् तद् वै वै तत् । \newline
37. तद् दे॒वान् दे॒वान् तत् तद् दे॒वान् । \newline
38. दे॒वा न॑वृणता वृणत दे॒वान् दे॒वा न॑वृणत । \newline
39. अ॒वृ॒ण॒त॒ प॒शवः॑ प॒शवो॑ ऽवृणता वृणत प॒शवः॑ । \newline
40. प॒शवो ऽसु॑रा॒ नसु॑रान् प॒शवः॑ प॒शवो ऽसु॑रान् । \newline
41. असु॑रा नजहु रजहु॒ रसु॑रा॒ नसु॑रा नजहुः । \newline
42. अ॒ज॒हु॒र् यं ॅय म॑जहु रजहु॒र् यम् । \newline
43. यम् का॒मये॑त का॒मये॑त॒ यं ॅयम् का॒मये॑त । \newline
44. का॒मये॑ता प॒शु र॑प॒शुः का॒मये॑त का॒मये॑ता प॒शुः । \newline
45. अ॒प॒शुः स्या᳚थ् स्या दप॒शु र॑प॒शुः स्या᳚त् । \newline
46. स्या॒दितीति॑ स्याथ् स्या॒दिति॑ । \newline
47. इति॒ परा॑ची॒म् परा॑ची॒ मितीति॒ परा॑चीम् । \newline
48. परा॑ची॒म् तस्य॒ तस्य॒ परा॑ची॒म् परा॑ची॒म् तस्य॑ । \newline
49. तस्ये डा॒ मिडा॒म् तस्य॒ तस्ये डा᳚म् । \newline
50. इडा॒ मुपोपे डा॒ मिडा॒ मुप॑ । \newline
51. उप॑ ह्वयेत ह्वये॒तो पोप॑ ह्वयेत । \newline
52. ह्व॒ये॒ता॒ प॒शु र॑प॒शुर् ह्व॑येत ह्वयेता प॒शुः । \newline
53. अ॒प॒शु रे॒वैवा प॒शु र॑प॒शु रे॒व । \newline
54. ए॒व भ॑वति भवत्ये॒वैव भ॑वति । \newline
55. भ॒व॒ति॒ यं ॅयम् भ॑वति भवति॒ यम् । \newline
56. यम् का॒मये॑त का॒मये॑त॒ यं ॅयम् का॒मये॑त । \newline

\textbf{Ghana Paata } \newline

1. मा॒त्रे व॒थ्सं ॅव॒थ्सम् मा॒त्रे मा॒त्रे व॒थ्स मु॒पाव॑सृज त्यु॒पाव॑सृजति व॒थ्सम् मा॒त्रे मा॒त्रे व॒थ्स मु॒पाव॑सृजति । \newline
2. व॒थ्स मु॒पाव॑सृज त्यु॒पाव॑सृजति व॒थ्सं ॅव॒थ्स मु॒पाव॑सृजति॒ सर्वे॑ण॒ सर्वे॑णो॒पाव॑सृजति व॒थ्सं ॅव॒थ्स मु॒पाव॑सृजति॒ सर्वे॑ण । \newline
3. उ॒पाव॑सृजति॒ सर्वे॑ण॒ सर्वे॑णो॒पा व॑सृज त्यु॒पाव॑सृजति॒ सर्वे॑ण॒ वै वै सर्वे॑णो॒पा व॑सृज त्यु॒पाव॑सृजति॒ सर्वे॑ण॒ वै । \newline
4. उ॒पाव॑सृज॒तीत्यु॑प - अव॑सृजति । \newline
5. सर्वे॑ण॒ वै वै सर्वे॑ण॒ सर्वे॑ण॒ वै य॒ज्ञेन॑ य॒ज्ञेन॒ वै सर्वे॑ण॒ सर्वे॑ण॒ वै य॒ज्ञेन॑ । \newline
6. वै य॒ज्ञेन॑ य॒ज्ञेन॒ वै वै य॒ज्ञेन॑ दे॒वा दे॒वा य॒ज्ञेन॒ वै वै य॒ज्ञेन॑ दे॒वाः । \newline
7. य॒ज्ञेन॑ दे॒वा दे॒वा य॒ज्ञेन॑ य॒ज्ञेन॑ दे॒वाः सु॑व॒र्गꣳ सु॑व॒र्गम् दे॒वा य॒ज्ञेन॑ य॒ज्ञेन॑ दे॒वाः सु॑व॒र्गम् । \newline
8. दे॒वाः सु॑व॒र्गꣳ सु॑व॒र्गम् दे॒वा दे॒वाः सु॑व॒र्गम् ॅलो॒कम् ॅलो॒कꣳ सु॑व॒र्गम् दे॒वा दे॒वाः सु॑व॒र्गम् ॅलो॒कम् । \newline
9. सु॒व॒र्गम् ॅलो॒कम् ॅलो॒कꣳ सु॑व॒र्गꣳ सु॑व॒र्गम् ॅलो॒क मा॑यन् नायन् ॅलो॒कꣳ सु॑व॒र्गꣳ सु॑व॒र्गम् ॅलो॒क मा॑यन्न् । \newline
10. सु॒व॒र्गमिति॑ सुवः - गम् । \newline
11. लो॒क मा॑यन् नायन् ॅलो॒कम् ॅलो॒क मा॑यन् पाकय॒ज्ञेन॑ पाकय॒ज्ञेना॑यन् ॅलो॒कम् ॅलो॒क मा॑यन् पाकय॒ज्ञेन॑ । \newline
12. आ॒य॒न् पा॒क॒य॒ज्ञेन॑ पाकय॒ज्ञेना॑यन् नायन् पाकय॒ज्ञेन॒ मनु॒र् मनुः॑ पाकय॒ज्ञेना॑यन् नायन् पाकय॒ज्ञेन॒ मनुः॑ । \newline
13. पा॒क॒य॒ज्ञेन॒ मनु॒र् मनुः॑ पाकय॒ज्ञेन॑ पाकय॒ज्ञेन॒ मनु॑ रश्राम्य दश्राम्य॒न् मनुः॑ पाकय॒ज्ञेन॑ पाकय॒ज्ञेन॒ मनु॑ रश्राम्यत् । \newline
14. पा॒क॒य॒ज्ञेनेति॑ पाक - य॒ज्ञेन॑ । \newline
15. मनु॑ रश्राम्य दश्राम्य॒न् मनु॒र् मनु॑ रश्राम्य॒थ् सा सा ऽश्रा᳚म्य॒न् मनु॒र् मनु॑ रश्राम्य॒थ् सा । \newline
16. अ॒श्रा॒म्य॒थ् सा सा ऽश्रा᳚म्य दश्राम्य॒थ् सेडेडा॒ सा ऽश्रा᳚म्य दश्राम्य॒थ् सेडा᳚ । \newline
17. सेडेडा॒ सा सेडा॒ मनु॒म् मनु॒ मिडा॒ सा सेडा॒ मनु᳚म् । \newline
18. इडा॒ मनु॒म् मनु॒ मिडेडा॒ मनु॑ मु॒पाव॑र्त तो॒पाव॑र्तत॒ मनु॒ मिडेडा॒ मनु॑ मु॒पाव॑र्तत । \newline
19. मनु॑ मु॒पाव॑र्त तो॒पाव॑र्तत॒ मनु॒म् मनु॑ मु॒पाव॑र्तत॒ ताम् ता मु॒पाव॑र्तत॒ मनु॒म् मनु॑ मु॒पाव॑र्तत॒ ताम् । \newline
20. उ॒पाव॑र्तत॒ ताम् ता मु॒पाव॑र्त तो॒पाव॑र्तत॒ ताम् दे॑वासु॒रा दे॑वासु॒रा स्ता मु॒पाव॑र्त तो॒पाव॑र्तत॒ ताम् दे॑वासु॒राः । \newline
21. उ॒पाव॑र्त॒तेत्यु॑प - आव॑र्तत । \newline
22. ताम् दे॑वासु॒रा दे॑वासु॒रा स्ताम् ताम् दे॑वासु॒रा वि वि दे॑वासु॒रा स्ताम् ताम् दे॑वासु॒रा वि । \newline
23. दे॒वा॒सु॒रा वि वि दे॑वासु॒रा दे॑वासु॒रा व्य॑ह्वयन्ता ह्वयन्त॒ वि दे॑वासु॒रा दे॑वासु॒रा व्य॑ह्वयन्त । \newline
24. दे॒वा॒सु॒रा इति॑ देव - अ॒सु॒राः । \newline
25. व्य॑ह्वयन्ता ह्वयन्त॒ वि व्य॑ह्वयन्त प्र॒तीची᳚म् प्र॒तीची॑ मह्वयन्त॒ वि व्य॑ह्वयन्त प्र॒तीची᳚म् । \newline
26. अ॒ह्व॒य॒न्त॒ प्र॒तीची᳚म् प्र॒तीची॑ मह्वयन्ता ह्वयन्त प्र॒तीची᳚म् दे॒वा दे॒वाः प्र॒तीची॑ मह्वयन्ता ह्वयन्त प्र॒तीची᳚म् दे॒वाः । \newline
27. प्र॒तीची᳚म् दे॒वा दे॒वाः प्र॒तीची᳚म् प्र॒तीची᳚म् दे॒वाः परा॑ची॒म् परा॑चीम् दे॒वाः प्र॒तीची᳚म् प्र॒तीची᳚म् दे॒वाः परा॑चीम् । \newline
28. दे॒वाः परा॑ची॒म् परा॑चीम् दे॒वा दे॒वाः परा॑ची॒ मसु॑रा॒ असु॑राः॒ परा॑चीम् दे॒वा दे॒वाः परा॑ची॒ मसु॑राः । \newline
29. परा॑ची॒ मसु॑रा॒ असु॑राः॒ परा॑ची॒म् परा॑ची॒ मसु॑राः॒ सा सा ऽसु॑राः॒ परा॑ची॒म् परा॑ची॒ मसु॑राः॒ सा । \newline
30. असु॑राः॒ सा सा ऽसु॑रा॒ असु॑राः॒ सा दे॒वान् दे॒वान् थ्सा ऽसु॑रा॒ असु॑राः॒ सा दे॒वान् । \newline
31. सा दे॒वान् दे॒वान् थ्सा सा दे॒वा नु॒पाव॑र्त तो॒पाव॑र्तत दे॒वान् थ्सा सा दे॒वा नु॒पाव॑र्तत । \newline
32. दे॒वा नु॒पाव॑र्त तो॒पाव॑र्तत दे॒वान् दे॒वा नु॒पाव॑र्तत प॒शवः॑ प॒शव॑ उ॒पाव॑र्तत दे॒वान् दे॒वा नु॒पाव॑र्तत प॒शवः॑ । \newline
33. उ॒पाव॑र्तत प॒शवः॑ प॒शव॑ उ॒पाव॑र्त तो॒पाव॑र्तत प॒शवो॒ वै वै प॒शव॑ उ॒पाव॑र्त तो॒पाव॑र्तत प॒शवो॒ वै । \newline
34. उ॒पाव॑र्त॒तेत्यु॑प - आव॑र्तत । \newline
35. प॒शवो॒ वै वै प॒शवः॑ प॒शवो॒ वै तत् तद् वै प॒शवः॑ प॒शवो॒ वै तत् । \newline
36. वै तत् तद् वै वै तद् दे॒वान् दे॒वान् तद् वै वै तद् दे॒वान् । \newline
37. तद् दे॒वान् दे॒वान् तत् तद् दे॒वा न॑वृणतावृणत दे॒वान् तत् तद् दे॒वा न॑वृणत । \newline
38. दे॒वा न॑वृणतावृणत दे॒वान् दे॒वा न॑वृणत प॒शवः॑ प॒शवो॑ ऽवृणत दे॒वान् दे॒वा न॑वृणत प॒शवः॑ । \newline
39. अ॒वृ॒ण॒त॒ प॒शवः॑ प॒शवो॑ ऽवृणतावृणत प॒शवो ऽसु॑रा॒ नसु॑रान् प॒शवो॑ ऽवृणतावृणत प॒शवो ऽसु॑रान् । \newline
40. प॒शवो ऽसु॑रा॒ नसु॑रान् प॒शवः॑ प॒शवो ऽसु॑रा नजहु रजहु॒ रसु॑रान् प॒शवः॑ प॒शवो ऽसु॑रा नजहुः । \newline
41. असु॑रा नजहु रजहु॒ रसु॑रा॒ नसु॑रा नजहु॒र् यं ॅय म॑जहु॒ रसु॑रा॒ नसु॑रा नजहु॒र् यम् । \newline
42. अ॒ज॒हु॒र् यं ॅय म॑जहु रजहु॒र् यम् का॒मये॑त का॒मये॑त॒ य म॑जहु रजहु॒र् यम् का॒मये॑त । \newline
43. यम् का॒मये॑त का॒मये॑त॒ यं ॅयम् का॒मये॑ताप॒शु र॑प॒शुः का॒मये॑त॒ यं ॅयम् का॒मये॑ताप॒शुः । \newline
44. का॒मये॑ताप॒शु र॑प॒शुः का॒मये॑त का॒मये॑ताप॒शुः स्या᳚थ् स्यादप॒शुः का॒मये॑त का॒मये॑ताप॒शुः स्या᳚त् । \newline
45. अ॒प॒शुः स्या᳚थ् स्या दप॒शु र॑प॒शुः स्या॒दितीति॑ स्या दप॒शु र॑प॒शुः स्या॒दिति॑ । \newline
46. स्या॒दितीति॑ स्याथ् स्या॒दिति॒ परा॑ची॒म् परा॑ची॒ मिति॑ स्याथ् स्या॒दिति॒ परा॑चीम् । \newline
47. इति॒ परा॑ची॒म् परा॑ची॒ मितीति॒ परा॑ची॒म् तस्य॒ तस्य॒ परा॑ची॒ मितीति॒ परा॑ची॒म् तस्य॑ । \newline
48. परा॑ची॒म् तस्य॒ तस्य॒ परा॑ची॒म् परा॑ची॒म् तस्ये डा॒ मिडा॒म् तस्य॒ परा॑ची॒म् परा॑ची॒म् तस्ये डा᳚म् । \newline
49. तस्ये डा॒ मिडा॒म् तस्य॒ तस्ये डा॒ मुपोपे डा॒म् तस्य॒ तस्ये डा॒ मुप॑ । \newline
50. इडा॒ मुपोपे डा॒ मिडा॒ मुप॑ ह्वयेत ह्वये॒तोपे डा॒ मिडा॒ मुप॑ ह्वयेत । \newline
51. उप॑ ह्वयेत ह्वये॒तोपोप॑ ह्वयेताप॒शु र॑प॒शुर् ह्व॑ये॒तोपोप॑ ह्वयेताप॒शुः । \newline
52. ह्व॒ये॒ता॒प॒शु र॑प॒शुर् ह्व॑येत ह्वयेताप॒शु रे॒वैवाप॒शुर् ह्व॑येत ह्वयेताप॒शुरे॒व । \newline
53. अ॒प॒शु रे॒वैवाप॒शु र॑प॒शुरे॒व भ॑वति भवत्ये॒वा प॒शुर॑प॒शु रे॒व भ॑वति । \newline
54. ए॒व भ॑वति भवत्ये॒वैव भ॑वति॒ यं ॅयम् भ॑वत्ये॒वैव भ॑वति॒ यम् । \newline
55. भ॒व॒ति॒ यं ॅयम् भ॑वति भवति॒ यम् का॒मये॑त का॒मये॑त॒ यम् भ॑वति भवति॒ यम् का॒मये॑त । \newline
56. यम् का॒मये॑त का॒मये॑त॒ यं ॅयम् का॒मये॑त पशु॒मान् प॑शु॒मान् का॒मये॑त॒ यं ॅयम् का॒मये॑त पशु॒मान् । \newline
\pagebreak
\markright{ TS 1.7.1.4  \hfill https://www.vedavms.in \hfill}
\addcontentsline{toc}{section}{ TS 1.7.1.4 }
\section*{ TS 1.7.1.4 }

\textbf{TS 1.7.1.4 } \newline
\textbf{Samhita Paata} \newline

का॒मये॑त पशु॒मान्थ् स्या॒दिति॑ प्र॒तीचीं॒ तस्येडा॒-मुप॑ ह्वयेत पशु॒माने॒व भ॑वति ब्रह्मवा॒दिनो॑ वदन्ति॒ स त्वा इडा॒मुप॑ ह्वयेत॒ य इडा॑- मुप॒हूया॒त्मान॒-मिडा॑या-मुप॒ह्वये॒तेति॒ सा नः॑ प्रि॒या सु॒प्रतू᳚र्तिर् म॒घोनीत्या॒हेडा॑-मे॒वोप॒हूया॒ऽऽत्मान॒ -मिडा॑या॒मुप॑ ह्वयते॒ व्य॑स्तमिव॒ वा ए॒तद्-य॒ज्ञ्स्य॒ यदिडा॑ सा॒मि प्रा॒श्ञन्ति॑ - [ ] \newline

\textbf{Pada Paata} \newline

का॒मये॑त । प॒शु॒मानिति॑ पशु - मान् । स्या॒त् । इति॑ । प्र॒तीची᳚म् । तस्य॑ । इडा᳚म् । उपेति॑ । ह्व॒ये॒त॒ । प॒शु॒मानिति॑ पशु - मान् । ए॒व । भ॒व॒ति॒ । ब्र॒ह्म॒वा॒दिन॒ इति॑ ब्रह्म - वा॒दिनः॑ । व॒द॒न्ति॒ । सः । तु । वै । इडा᳚म् । उपेति॑ । ह्व॒ये॒त॒ । यः । इडा᳚म् । उ॒प॒हूयेत्यु॑प - हूय॑ । आ॒त्मान᳚म् । इडा॑याम् । उ॒प॒ह्वये॒तेत्यु॑प - ह्वये॑त । इति॑ । सा । नः॒ । प्रि॒या । सु॒प्रतू᳚र्ति॒रिति॑ सु - प्रतू᳚र्तिः । म॒घोनी᳚ । इति॑ । आ॒ह॒ । इडा᳚म् । ए॒व । उ॒प॒हूयेत्यु॑प - हूय॑ । आ॒त्मान᳚म् । इडा॑याम् । उपेति॑ । ह्व॒य॒ते॒ । व्य॑स्त॒मिति॒ वि-अ॒स्त॒म् । इ॒व॒ । वै । ए॒तत् । य॒ज्ञ्स्य॑ । यत् । इडा᳚ । सा॒मि । प्रा॒श्नन्तीति॑ प्र - अ॒श्नन्ति॑ ।  \newline


\textbf{Krama Paata} \newline

का॒मये॑त पशु॒मान् । प॒शु॒मान्थ् स्या᳚त् । प॒शु॒मानिति॑ पशु - मान् । स्या॒दिति॑ । इति॑ प्र॒तीची᳚म् । प्र॒तीची॒म् तस्य॑ । तस्येडा᳚म् । इडा॒मुप॑ । उप॑ ह्वयेत । ह्व॒ये॒त॒ प॒शु॒मान् । प॒शु॒माने॒व । प॒शु॒मानिति॑ पशु - मान् । ए॒व भ॑वति । भ॒व॒ति॒ ब्र॒ह्म॒वा॒दिनः॑ । ब्र॒ह्म॒वा॒दिनो॑ वदन्ति । ब्र॒ह्म॒वा॒दिन॒ इति॑ ब्रह्म - वा॒दिनः॑ । व॒द॒न्ति॒ सः । स तु । त्वै । वा इडा᳚म् । इडा॒मुप॑ । उप॑ ह्वयेत । ह्व॒ये॒त॒ यः । य इडा᳚म् । इडा॑मुप॒हूय॑ । उ॒प॒हूया॒त्मान᳚म् । उ॒प॒हूयेत्यु॑प - हूय॑ । आ॒त्मान॒मिडा॑याम् । इडा॑यामुप॒ह्वये॑त । उ॒प॒ह्वये॒तेति॑ । उ॒प॒ह्वये॒तेत्यु॑प - ह्वये॑त । इति॒ सा । सा नः॑ । नः॒ प्रि॒या । प्रि॒या सु॒प्रतू᳚र्तिः । सु॒प्रतू᳚र्तिर्,म॒घोनी᳚ । सु॒प्रतू᳚र्ति॒रिति॑ सु - प्रतू᳚र्तिः । म॒घोनीति॑ । इत्या॑ह । आ॒हेडा᳚म् । इडा॑मे॒व । ए॒वोप॒हूय॑ । उ॒प॒हूया॒त्मान᳚म् । उ॒प॒हूयेत्यु॑प - हूय॑ । आ॒त्मान॒मिडा॑याम् । इडा॑या॒मुप॑ । उप॑ ह्वयते । ह्व॒य॒ते॒ व्य॑स्तम् । व्य॑स्तमिव । व्य॑स्त॒मिति॒ वि - अ॒स्त॒म् । इ॒व॒ वै । वा ए॒तत् । ए॒तद् य॒ज्ञ्स्य॑ । य॒ज्ञ्स्य॒ यत् । यदिडा᳚ । इडा॑ सा॒मि । सा॒मि प्रा॒श्ञन्ति॑ । प्रा॒श्ञन्ति॑ सा॒मि । प्रा॒श्ञन्तीति॑ प्र - अ॒श्ञन्ति॑ \newline

\textbf{Jatai Paata} \newline

1. का॒मये॑त पशु॒मान् प॑शु॒मान् का॒मये॑त का॒मये॑त पशु॒मान् । \newline
2. प॒शु॒मान् थ्स्या᳚थ् स्यात् पशु॒मान् प॑शु॒मान् थ्स्या᳚त् । \newline
3. प॒शु॒मानिति॑ पशु - मान् । \newline
4. स्या॒ दितीति॑ स्याथ् स्या॒ दिति॑ । \newline
5. इति॑ प्र॒तीची᳚म् प्र॒तीची॒ मितीति॑ प्र॒तीची᳚म् । \newline
6. प्र॒तीची॒म् तस्य॒ तस्य॑ प्र॒तीची᳚म् प्र॒तीची॒म् तस्य॑ । \newline
7. तस्ये डा॒ मिडा॒म् तस्य॒ तस्ये डा᳚म् । \newline
8. इडा॒ मुपोपे डा॒ मिडा॒ मुप॑ । \newline
9. उप॑ ह्वयेत ह्वये॒तो पोप॑ ह्वयेत । \newline
10. ह्व॒ये॒त॒ प॒शु॒मान् प॑शु॒मान् ह्व॑येत ह्वयेत पशु॒मान् । \newline
11. प॒शु॒मा ने॒वैव प॑शु॒मान् प॑शु॒मा ने॒व । \newline
12. प॒शु॒मानिति॑ पशु - मान् । \newline
13. ए॒व भ॑वति भव त्ये॒वैव भ॑वति । \newline
14. भ॒व॒ति॒ ब्र॒ह्म॒वा॒दिनो᳚ ब्रह्मवा॒दिनो॑ भवति भवति ब्रह्मवा॒दिनः॑ । \newline
15. ब्र॒ह्म॒वा॒दिनो॑ वदन्ति वदन्ति ब्रह्मवा॒दिनो᳚ ब्रह्मवा॒दिनो॑ वदन्ति । \newline
16. ब्र॒ह्म॒वा॒दिन॒ इति॑ ब्रह्म - वा॒दिनः॑ । \newline
17. व॒द॒न्ति॒ स स व॑दन्ति वदन्ति॒ सः । \newline
18. स तु तु स स तु । \newline
19. त्वै वै तुत् वै । \newline
20. वा इडा॒ मिडां॒ ॅवै वा इडा᳚म् । \newline
21. इडा॒ मुपोपे डा॒ मिडा॒ मुप॑ । \newline
22. उप॑ ह्वयेत ह्वये॒तो पोप॑ ह्वयेत । \newline
23. ह्व॒ये॒त॒ यो यो ह्व॑येत ह्वयेत॒ यः । \newline
24. य इडा॒ मिडां॒ ॅयो य इडा᳚म् । \newline
25. इडा॑ मुप॒हू यो॑प॒हूये डा॒ मिडा॑ मुप॒हूय॑ । \newline
26. उ॒प॒हूया॒ त्मान॑ मा॒त्मान॑ मुप॒हूयो॑ प॒हूया॒ त्मान᳚म् । \newline
27. उ॒प॒हूयेत्यु॑प - हूय॑ । \newline
28. आ॒त्मान॒ मिडा॑या॒ मिडा॑या मा॒त्मान॑ मा॒त्मान॒ मिडा॑याम् । \newline
29. इडा॑या मुप॒ह्वये॑तो प॒ह्वये॒ते डा॑या॒ मिडा॑या मुप॒ह्वये॑त । \newline
30. उ॒प॒ह्वये॒ते ती त्यु॑प॒ह्वये॑ तोप॒ह्वये॒ते ति॑ । \newline
31. उ॒प॒ह्वये॒तेत्यु॑प - ह्वये॑त । \newline
32. इति॒ सा सेतीति॒ सा । \newline
33. सा नो॑ नः॒ सा सा नः॑ । \newline
34. नः॒ प्रि॒या प्रि॒या नो॑ नः प्रि॒या । \newline
35. प्रि॒या सु॒प्रतू᳚र्तिः सु॒प्रतू᳚र्तिः प्रि॒या प्रि॒या सु॒प्रतू᳚र्तिः । \newline
36. सु॒प्रतू᳚र्तिर् म॒घोनी॑ म॒घोनी॑ सु॒प्रतू᳚र्तिः सु॒प्रतू᳚र्तिर् म॒घोनी᳚ । \newline
37. सु॒प्रतू᳚र्ति॒रिति॑ सु - प्रतू᳚र्तिः । \newline
38. म॒घोनी तीति॑ म॒घोनी॑ म॒घोनीति॑ । \newline
39. इत्या॑हा॒हे तीत्या॑ह । \newline
40. आ॒हे डा॒ मिडा॑ माहा॒हे डा᳚म् । \newline
41. इडा॑ मे॒वैवे डा॒ मिडा॑ मे॒व । \newline
42. ए॒वो प॒हूयो॑ प॒हू यै॒वै वोप॒हूय॑ । \newline
43. उ॒प॒हूया॒ त्मान॑ मा॒त्मान॑ मुप॒हूयो॑ प॒हूया॒ त्मान᳚म् । \newline
44. उ॒प॒हूयेत्यु॑प - हूय॑ । \newline
45. आ॒त्मान॒ मिडा॑या॒ मिडा॑या मा॒त्मान॑ मा॒त्मान॒ मिडा॑याम् । \newline
46. इडा॑या॒ मुपोपे डा॑या॒ मिडा॑या॒ मुप॑ । \newline
47. उप॑ ह्वयते ह्वयत॒ उपोप॑ ह्वयते । \newline
48. ह्व॒य॒ते॒ व्य॑स्तं॒ ॅव्य॑स्तꣳ ह्वयते ह्वयते॒ व्य॑स्तम् । \newline
49. व्य॑स्त मिवे व॒ व्य॑स्तं॒ ॅव्य॑स्त मिव । \newline
50. व्य॑स्त॒मिति॒ वि - अ॒स्त॒म् । \newline
51. इ॒व॒ वै वा इ॑वे व॒ वै । \newline
52. वा ए॒त दे॒तद् वै वा ए॒तत् । \newline
53. ए॒तद् य॒ज्ञ्स्य॑ य॒ज्ञ् स्यै॒त दे॒तद् य॒ज्ञ्स्य॑ । \newline
54. य॒ज्ञ्स्य॒ यद् यद् य॒ज्ञ्स्य॑ य॒ज्ञ्स्य॒ यत् । \newline
55. यदिडेडा॒ यद् यदिडा᳚ । \newline
56. इडा॑ सा॒मि सा॒मी डेडा॑ सा॒मि । \newline
57. सा॒मि प्रा॒श्ञन्ति॑ प्रा॒श्ञन्ति॑ सा॒मि सा॒मि प्रा॒श्ञन्ति॑ । \newline
58. प्रा॒श्ञन्ति॑ सा॒मि सा॒मि प्रा॒श्ञन्ति॑ प्रा॒श्ञन्ति॑ सा॒मि । \newline
59. प्रा॒श्ञन्तीति॑ प्र - अ॒श्ञन्ति॑ । \newline

\textbf{Ghana Paata } \newline

1. का॒मये॑त पशु॒मान् प॑शु॒मान् का॒मये॑त का॒मये॑त पशु॒मान् थ्स्या᳚थ् स्यात् पशु॒मान् का॒मये॑त का॒मये॑त पशु॒मान् थ्स्या᳚त् । \newline
2. प॒शु॒मान् थ्स्या᳚थ् स्यात् पशु॒मान् प॑शु॒मान् थ्स्या॒दितीति॑ स्यात् पशु॒मान् प॑शु॒मान् थ्स्या॒दिति॑ । \newline
3. प॒शु॒मानिति॑ पशु - मान् । \newline
4. स्या॒दितीति॑ स्याथ् स्या॒दिति॑ प्र॒तीची᳚म् प्र॒तीची॒ मिति॑ स्याथ् स्या॒दिति॑ प्र॒तीची᳚म् । \newline
5. इति॑ प्र॒तीची᳚म् प्र॒तीची॒ मितीति॑ प्र॒तीची॒म् तस्य॒ तस्य॑ प्र॒तीची॒ मितीति॑ प्र॒तीची॒म् तस्य॑ । \newline
6. प्र॒तीची॒म् तस्य॒ तस्य॑ प्र॒तीची᳚म् प्र॒तीची॒म् तस्ये डा॒ मिडा॒म् तस्य॑ प्र॒तीची᳚म् प्र॒तीची॒म् तस्ये डा᳚म् । \newline
7. तस्ये डा॒ मिडा॒म् तस्य॒ तस्ये डा॒ मुपोपे डा॒म् तस्य॒ तस्ये डा॒ मुप॑ । \newline
8. इडा॒ मुपोपे डा॒ मिडा॒ मुप॑ ह्वयेत ह्वये॒तोपे डा॒ मिडा॒ मुप॑ ह्वयेत । \newline
9. उप॑ ह्वयेत ह्वये॒तोपोप॑ ह्वयेत पशु॒मान् प॑शु॒मान् ह्व॑ये॒तोपोप॑ ह्वयेत पशु॒मान् । \newline
10. ह्व॒ये॒त॒ प॒शु॒मान् प॑शु॒मान् ह्व॑येत ह्वयेत पशु॒मा ने॒वैव प॑शु॒मान् ह्व॑येत ह्वयेत पशु॒मा ने॒व । \newline
11. प॒शु॒मा ने॒वैव प॑शु॒मान् प॑शु॒मा ने॒व भ॑वति भवत्ये॒व प॑शु॒मान् प॑शु॒मा ने॒व भ॑वति । \newline
12. प॒शु॒मानिति॑ पशु - मान् । \newline
13. ए॒व भ॑वति भवत्ये॒वैव भ॑वति ब्रह्मवा॒दिनो᳚ ब्रह्मवा॒दिनो॑ भवत्ये॒वैव भ॑वति ब्रह्मवा॒दिनः॑ । \newline
14. भ॒व॒ति॒ ब्र॒ह्म॒वा॒दिनो᳚ ब्रह्मवा॒दिनो॑ भवति भवति ब्रह्मवा॒दिनो॑ वदन्ति वदन्ति ब्रह्मवा॒दिनो॑ भवति भवति ब्रह्मवा॒दिनो॑ वदन्ति । \newline
15. ब्र॒ह्म॒वा॒दिनो॑ वदन्ति वदन्ति ब्रह्मवा॒दिनो᳚ ब्रह्मवा॒दिनो॑ वदन्ति॒ स स व॑दन्ति ब्रह्मवा॒दिनो᳚ ब्रह्मवा॒दिनो॑ वदन्ति॒ सः । \newline
16. ब्र॒ह्म॒वा॒दिन॒ इति॑ ब्रह्म - वा॒दिनः॑ । \newline
17. व॒द॒न्ति॒ स स व॑दन्ति वदन्ति॒ स तु तु स व॑दन्ति वदन्ति॒ स तु । \newline
18. स तु तु स सत् वै वै तु स सत् वै । \newline
19. त्वै वै तुत् वा इडा॒ मिडां॒ ॅवै तुत् वा इडा᳚म् । \newline
20. वा इडा॒ मिडां॒ ॅवै वा इडा॒ मुपोपे डां॒ ॅवै वा इडा॒ मुप॑ । \newline
21. इडा॒ मुपोपे डा॒ मिडा॒ मुप॑ ह्वयेत ह्वये॒तोपे डा॒ मिडा॒ मुप॑ ह्वयेत । \newline
22. उप॑ ह्वयेत ह्वये॒तोपोप॑ ह्वयेत॒ यो यो ह्व॑ये॒तोपोप॑ ह्वयेत॒ यः । \newline
23. ह्व॒ये॒त॒ यो यो ह्व॑येत ह्वयेत॒ य इडा॒ मिडां॒ ॅयो ह्व॑येत ह्वयेत॒ य इडा᳚म् । \newline
24. य इडा॒ मिडां॒ ॅयो य इडा॑ मुप॒हूयो॑ प॒हूये डां॒ ॅयो य इडा॑ मुप॒हूय॑ । \newline
25. इडा॑ मुप॒हूयो॑प॒हूये डा॒ मिडा॑ मुप॒हूया॒ त्मान॑ मा॒त्मान॑ मुप॒हूये डा॒ मिडा॑ मुप॒हूया॒त्मान᳚म् । \newline
26. उ॒प॒हूया॒त्मान॑ मा॒त्मान॑ मुप॒हूयो॑ प॒हूया॒त्मान॒ मिडा॑या॒ मिडा॑या मा॒त्मान॑ मुप॒हूयो॑ प॒हूया॒त्मान॒ मिडा॑याम् । \newline
27. उ॒प॒हूयेत्यु॑प - हूय॑ । \newline
28. आ॒त्मान॒ मिडा॑या॒ मिडा॑या मा॒त्मान॑ मा॒त्मान॒ मिडा॑या मुप॒ह्वये॑तो प॒ह्वये॒ते डा॑या मा॒त्मान॑ मा॒त्मान॒ मिडा॑या मुप॒ह्वये॑त । \newline
29. इडा॑या मुप॒ह्वये॑तो प॒ह्वये॒ते डा॑या॒ मिडा॑या मुप॒ह्वये॒ते तीत्यु॑प॒ह्वये॒ते डा॑या॒ मिडा॑या मुप॒ह्वये॒ते ति॑ । \newline
30. उ॒प॒ह्वये॒ते तीत्यु॑ प॒ह्वये॑तो प॒ह्वये॒ते ति॒ सा सेत्यु॑ प॒ह्वये॑तो प॒ह्वये॒ते ति॒ सा । \newline
31. उ॒प॒ह्वये॒तेत्यु॑प - ह्वये॑त । \newline
32. इति॒ सा सेतीति॒ सा नो॑ नः॒ सेतीति॒ सा नः॑ । \newline
33. सा नो॑ नः॒ सा सा नः॑ प्रि॒या प्रि॒या नः॒ सा सा नः॑ प्रि॒या । \newline
34. नः॒ प्रि॒या प्रि॒या नो॑ नः प्रि॒या सु॒प्रतू᳚र्तिः सु॒प्रतू᳚र्तिः प्रि॒या नो॑ नः प्रि॒या सु॒प्रतू᳚र्तिः । \newline
35. प्रि॒या सु॒प्रतू᳚र्तिः सु॒प्रतू᳚र्तिः प्रि॒या प्रि॒या सु॒प्रतू᳚र्तिर् म॒घोनी॑ म॒घोनी॑ सु॒प्रतू᳚र्तिः प्रि॒या प्रि॒या सु॒प्रतू᳚र्तिर् म॒घोनी᳚ । \newline
36. सु॒प्रतू᳚र्तिर् म॒घोनी॑ म॒घोनी॑ सु॒प्रतू᳚र्तिः सु॒प्रतू᳚र्तिर् म॒घोनी तीति॑ म॒घोनी॑ सु॒प्रतू᳚र्तिः सु॒प्रतू᳚र्तिर् म॒घोनीति॑ । \newline
37. सु॒प्रतू᳚र्ति॒रिति॑ सु - प्रतू᳚र्तिः । \newline
38. म॒घोनी तीति॑ म॒घोनी॑ म॒घोनीत्या॑हा॒हे ति॑ म॒घोनी॑ म॒घोनीत्या॑ह । \newline
39. इत्या॑हा॒हे तीत्या॒हे डा॒ मिडा॑ मा॒हे तीत्या॒हे डा᳚म् । \newline
40. आ॒हे डा॒ मिडा॑ माहा॒हे डा॑ मे॒वैवे डा॑ माहा॒हे डा॑ मे॒व । \newline
41. इडा॑ मे॒वैवे डा॒ मिडा॑ मे॒वोप॒हूयो॑ प॒हूयै॒वे डा॒ मिडा॑ मे॒वोप॒हूय॑ । \newline
42. ए॒वो प॒हूयो॑ प॒हू यै॒वैवो प॒हूया॒ त्मान॑ मा॒त्मान॑ मुप॒हू यै॒वैवो प॒हूया॒ त्मान᳚म् । \newline
43. उ॒प॒हूया॒त्मान॑ मा॒त्मान॑ मुप॒हूयो॑ प॒हूया॒त्मान॒ मिडा॑या॒ मिडा॑या मा॒त्मान॑ मुप॒हूयो॑ प॒हूया॒त्मान॒ मिडा॑याम् । \newline
44. उ॒प॒हूयेत्यु॑प - हूय॑ । \newline
45. आ॒त्मान॒ मिडा॑या॒ मिडा॑या मा॒त्मान॑ मा॒त्मान॒ मिडा॑या॒ मुपोपे डा॑या मा॒त्मान॑ मा॒त्मान॒ मिडा॑या॒ मुप॑ । \newline
46. इडा॑या॒ मुपोपे डा॑या॒ मिडा॑या॒ मुप॑ ह्वयते ह्वयत॒ उपे डा॑या॒ मिडा॑या॒ मुप॑ ह्वयते । \newline
47. उप॑ ह्वयते ह्वयत॒ उपोप॑ ह्वयते॒ व्य॑स्तं॒ ॅव्य॑स्तꣳ ह्वयत॒ उपोप॑ ह्वयते॒ व्य॑स्तम् । \newline
48. ह्व॒य॒ते॒ व्य॑स्तं॒ ॅव्य॑स्तꣳ ह्वयते ह्वयते॒ व्य॑स्त मिवे व॒ व्य॑स्तꣳ ह्वयते ह्वयते॒ व्य॑स्त मिव । \newline
49. व्य॑स्त मिवे व॒ व्य॑स्तं॒ ॅव्य॑स्त मिव॒ वै वा इ॑व॒ व्य॑स्तं॒ ॅव्य॑स्त मिव॒ वै । \newline
50. व्य॑स्त॒मिति॒ वि - अ॒स्त॒म् । \newline
51. इ॒व॒ वै वा इ॑वे व॒ वा ए॒त दे॒तद् वा इ॑वे व॒ वा ए॒तत् । \newline
52. वा ए॒त दे॒तद् वै वा ए॒तद् य॒ज्ञ्स्य॑ य॒ज्ञ् स्यै॒तद् वै वा ए॒तद् य॒ज्ञ्स्य॑ । \newline
53. ए॒तद् य॒ज्ञ्स्य॑ य॒ज्ञ् स्यै॒त दे॒तद् य॒ज्ञ्स्य॒ यद् यद् य॒ज्ञ् स्यै॒त दे॒तद् य॒ज्ञ्स्य॒ यत् । \newline
54. य॒ज्ञ्स्य॒ यद् यद् य॒ज्ञ्स्य॑ य॒ज्ञ्स्य॒ यदिडेडा॒ यद् य॒ज्ञ्स्य॑ य॒ज्ञ्स्य॒ यदिडा᳚ । \newline
55. यदिडेडा॒ यद् यदिडा॑ सा॒मि सा॒मीडा॒ यद् यदिडा॑ सा॒मि । \newline
56. इडा॑ सा॒मि सा॒मीडेडा॑ सा॒मि प्रा॒श्ञन्ति॑ प्रा॒श्ञन्ति॑ सा॒मीडेडा॑ सा॒मि प्रा॒श्ञन्ति॑ । \newline
57. सा॒मि प्रा॒श्ञन्ति॑ प्रा॒श्ञन्ति॑ सा॒मि सा॒मि प्रा॒श्ञन्ति॑ सा॒मि सा॒मि प्रा॒श्ञन्ति॑ सा॒मि सा॒मि प्रा॒श्ञन्ति॑ सा॒मि । \newline
58. प्रा॒श्ञन्ति॑ सा॒मि सा॒मि प्रा॒श्ञन्ति॑ प्रा॒श्ञन्ति॑ सा॒मि मा᳚र्जयन्ते मार्जयन्ते सा॒मि प्रा॒श्ञन्ति॑ प्रा॒श्ञन्ति॑ सा॒मि मा᳚र्जयन्ते । \newline
59. प्रा॒श्ञन्तीति॑ प्र - अ॒श्ञन्ति॑ । \newline
\pagebreak
\markright{ TS 1.7.1.5  \hfill https://www.vedavms.in \hfill}
\addcontentsline{toc}{section}{ TS 1.7.1.5 }
\section*{ TS 1.7.1.5 }

\textbf{TS 1.7.1.5 } \newline
\textbf{Samhita Paata} \newline

सा॒मि मा᳚र्जयन्त ए॒तत् प्रति॒ वा असु॑राणां ॅय॒ज्ञो व्य॑च्छिद्यत॒ ब्रह्म॑णा दे॒वाः सम॑दधु॒र् बृह॒स्पति॑ -स्तनुतामि॒मं न॒ इत्या॑ह॒ ब्रह्म॒ वै दे॒वानां॒ बृह॒स्पति॒र् ब्रह्म॑णै॒व य॒ज्ञ्ꣳ सं द॑धाति॒ विच्छि॑न्नं ॅय॒ज्ञ्ꣳ समि॒मं द॑धा॒त्वित्या॑ह॒ सन्त॑त्यै॒ विश्वे॑ दे॒वा इ॒ह मा॑दयन्ता॒मित्या॑ह स॒न्तत्यै॒व य॒ज्ञ्ं दे॒वेभ्योऽनु॑ दिशति॒ यां ॅवै - [ ] \newline

\textbf{Pada Paata} \newline

सा॒मि । मा॒र्ज॒य॒न्ते॒ । ए॒तत् । प्रतीति॑ । वै । असु॑राणाम् । य॒ज्ञ्ः । वीति॑ । अ॒च्छि॒द्य॒त॒ । ब्रह्म॑णा । दे॒वाः । समिति॑ । अ॒द॒धुः॒ । बृह॒स्पतिः॑ । त॒नु॒ता॒म् । इ॒मम् । नः॒ । इति॑ । आ॒ह॒ । ब्रह्म॑ । वै । दे॒वाना᳚म् । बृह॒स्पतिः॑ । ब्रह्म॑णा । ए॒व । य॒ज्ञ्म् । समिति॑ । द॒धा॒ति॒ । विच्छि॑न्न॒मिति॒ वि - छि॒न्न॒म् । य॒ज्ञ्म् । समिति॑ । इ॒मम् । द॒धा॒तु॒ । इति॑ । आ॒ह॒ । संत॑त्या॒ इति॒ सं - त॒त्यै॒ । विश्वे᳚ । दे॒वाः । इ॒ह । मा॒द॒य॒न्ता॒म् । इति॑ । आ॒ह॒ । स॒तंत्येति॑ सं - तत्य॑ । ए॒व । य॒ज्ञ्म् । दे॒वेभ्यः॑ । अन्विति॑ । दि॒श॒ति॒ । याम् । वै ।  \newline


\textbf{Krama Paata} \newline

सा॒मि मा᳚र्जयन्ते । मा॒र्ज॒य॒न्त॒ ए॒तत् । ए॒तत्,प्रति॑ । प्रति॒ वै । वा असु॑राणाम् । असु॑राणां ॅय॒ज्ञ्ः । य॒ज्ञो वि । व्य॑च्छिद्यत । अ॒च्छि॒द्य॒त॒ ब्रह्म॑णा । ब्रह्म॑णा दे॒वाः । दे॒वाः सम् । सम॑दधुः । अ॒द॒धु॒र्,बृह॒स्पतिः॑ । बृह॒स्पति॑ स्तनुताम् । त॒नु॒ता॒मि॒मम् । इ॒मम् नः॑ । न॒ इति॑ । इत्या॑ह । आ॒ह॒ ब्रह्म॑ । ब्रह्म॒ वै । वै दे॒वाना᳚म् । दे॒वाना॒म् बृह॒स्पतिः॑ । बृह॒स्पति॒र्,ब्रह्म॑णा । ब्रह्म॑णै॒व । ए॒व य॒ज्ञ्म् । य॒ज्ञ्ꣳ सम् । सम् द॑धाति । द॒धा॒ति॒ विच्छि॑न्नम् । विच्छि॑न्नं ॅय॒ज्ञ्म् । विच्छि॑न्न॒मिति॒ वि - छि॒न्न॒म् । य॒ज्ञ्ꣳ सम् । समि॒मम् । इ॒मम् द॑धातु । द॒धा॒त्विति॑ । इत्या॑ह । आ॒ह॒ सन्त॑त्यै । सन्त॑त्यै॒ विश्वे᳚ । सन्त॑त्या॒ इति॒ सम् - त॒त्यै॒ । विश्वे॑ दे॒वाः । दे॒वा इ॒ह । इ॒ह मा॑दयन्ताम् । मा॒द॒य॒न्ता॒मिति॑ । इत्या॑ह । आ॒ह॒ स॒न्तत्य॑ । स॒न्तत्यै॒व । स॒न्तत्येति॑ सं - तत्य॑ । ए॒व य॒ज्ञ्म् । य॒ज्ञ्म् दे॒वेभ्यः॑ । दे॒वेभ्योऽनु॑ । अनु॑ दिशति । दि॒श॒ति॒ याम् । यां ॅवै । वै य॒ज्ञे \newline

\textbf{Jatai Paata} \newline

1. सा॒मि मा᳚र्जयन्ते मार्जयन्ते सा॒मि सा॒मि मा᳚र्जयन्ते । \newline
2. मा॒र्ज॒य॒न्त॒ ए॒तदे॒तन् मा᳚र्जयन्ते मार्जयन्त ए॒तत् । \newline
3. ए॒तत् प्रति॒ प्रत्ये॒त दे॒तत् प्रति॑ । \newline
4. प्रति॒ वै वै प्रति॒ प्रति॒ वै । \newline
5. वा असु॑राणा॒ मसु॑राणां॒ ॅवै वा असु॑राणाम् । \newline
6. असु॑राणां ॅय॒ज्ञो य॒ज्ञो ऽसु॑राणा॒ मसु॑राणां ॅय॒ज्ञ्ः । \newline
7. य॒ज्ञो वि वि य॒ज्ञो य॒ज्ञो वि । \newline
8. व्य॑च्छिद्यता च्छिद्यत॒ वि व्य॑च्छिद्यत । \newline
9. अ॒च्छि॒द्य॒त॒ ब्रह्म॑णा॒ ब्रह्म॑णा ऽच्छिद्यता च्छिद्यत॒ ब्रह्म॑णा । \newline
10. ब्रह्म॑णा दे॒वा दे॒वा ब्रह्म॑णा॒ ब्रह्म॑णा दे॒वाः । \newline
11. दे॒वाः सꣳ सम् दे॒वा दे॒वाः सम् । \newline
12. स म॑दधु रदधुः॒ सꣳ स म॑दधुः । \newline
13. अ॒द॒धु॒र् बृह॒स्पति॒र् बृह॒स्पति॑ रदधु रदधु॒र् बृह॒स्पतिः॑ । \newline
14. बृह॒स्पति॑ स्तनुताम् तनुता॒म् बृह॒स्पति॒र् बृह॒स्पति॑ स्तनुताम् । \newline
15. त॒नु॒ता॒ मि॒म मि॒मम् त॑नुताम् तनुता मि॒मम् । \newline
16. इ॒मन्नो॑ न इ॒म मि॒मन्नः॑ । \newline
17. न॒ इतीति॑ नो न॒ इति॑ । \newline
18. इत्या॑हा॒हे तीत्या॑ह । \newline
19. आ॒ह॒ ब्रह्म॒ ब्रह्मा॑ हाह॒ ब्रह्म॑ । \newline
20. ब्रह्म॒ वै वै ब्रह्म॒ ब्रह्म॒ वै । \newline
21. वै दे॒वाना᳚म् दे॒वानां॒ ॅवै वै दे॒वाना᳚म् । \newline
22. दे॒वाना॒म् बृह॒स्पति॒र् बृह॒स्पति॑र् दे॒वाना᳚म् दे॒वाना॒म् बृह॒स्पतिः॑ । \newline
23. बृह॒स्पति॒र् ब्रह्म॑णा॒ ब्रह्म॑णा॒ बृह॒स्पति॒र् बृह॒स्पति॒र् ब्रह्म॑णा । \newline
24. ब्रह्म॑ णै॒वैव ब्रह्म॑णा॒ ब्रह्म॑ णै॒व । \newline
25. ए॒व य॒ज्ञ्ं ॅय॒ज्ञ् मे॒वैव य॒ज्ञ्म् । \newline
26. य॒ज्ञ्ꣳ सꣳ सं ॅय॒ज्ञ्ं ॅय॒ज्ञ्ꣳ सम् । \newline
27. सम् द॑धाति दधाति॒ सꣳ सम् द॑धाति । \newline
28. द॒धा॒ति॒ विच्छि॑न्नं॒ ॅविच्छि॑न्नम् दधाति दधाति॒ विच्छि॑न्नम् । \newline
29. विच्छि॑न्नं ॅय॒ज्ञ्ं ॅय॒ज्ञ्ं ॅविच्छि॑न्नं॒ ॅविच्छि॑न्नं ॅय॒ज्ञ्म् । \newline
30. विच्छि॑न्न॒मिति॒ वि - छि॒न्न॒म् । \newline
31. य॒ज्ञ्ꣳ सꣳ सं ॅय॒ज्ञ्ं ॅय॒ज्ञ्ꣳ सम् । \newline
32. स मि॒म मि॒मꣳ सꣳ स मि॒मम् । \newline
33. इ॒मम् द॑धातु दधा त्वि॒म मि॒मम् द॑धातु । \newline
34. द॒धा॒ त्वितीति॑ दधातु दधा॒ त्विति॑ । \newline
35. इत्या॑हा॒हे तीत्या॑ह । \newline
36. आ॒ह॒ सन्त॑त्यै॒ सन्त॑त्या आहाह॒ सन्त॑त्यै । \newline
37. सन्त॑त्यै॒ विश्वे॒ विश्वे॒ सन्त॑त्यै॒ सन्त॑त्यै॒ विश्वे᳚ । \newline
38. सन्त॑त्या॒ इति॒ सं - त॒त्यै॒ । \newline
39. विश्वे॑ दे॒वा दे॒वा विश्वे॒ विश्वे॑ दे॒वाः । \newline
40. दे॒वा इ॒हे ह दे॒वा दे॒वा इ॒ह । \newline
41. इ॒ह मा॑दयन्ताम् मादयन्ता मि॒हे ह मा॑दयन्ताम् । \newline
42. मा॒द॒य॒न्ता॒ मितीति॑ मादयन्ताम् मादयन्ता॒ मिति॑ । \newline
43. इत्या॑हा॒हे तीत्या॑ह । \newline
44. आ॒ह॒ स॒न्तत्य॑ स॒न्त त्या॑हाह स॒न्तत्य॑ । \newline
45. स॒न्त त्यै॒वैव स॒न्तत्य॑ स॒न्त त्यै॒व । \newline
46. स॒न्तत्येति॑ सं - तत्य॑ । \newline
47. ए॒व य॒ज्ञ्ं ॅय॒ज्ञ् मे॒वैव य॒ज्ञ्म् । \newline
48. य॒ज्ञ्म् दे॒वेभ्यो॑ दे॒वेभ्यो॑ य॒ज्ञ्ं ॅय॒ज्ञ्म् दे॒वेभ्यः॑ । \newline
49. दे॒वेभ्यो ऽन्वनु॑ दे॒वेभ्यो॑ दे॒वेभ्यो ऽनु॑ । \newline
50. अनु॑ दिशति दिश॒ त्यन्वनु॑ दिशति । \newline
51. दि॒श॒ति॒ यां ॅयाम् दि॑शति दिशति॒ याम् । \newline
52. यां ॅवै वै यां ॅयां ॅवै । \newline
53. वै य॒ज्ञे य॒ज्ञे वै वै य॒ज्ञे । \newline

\textbf{Ghana Paata } \newline

1. सा॒मि मा᳚र्जयन्ते मार्जयन्ते सा॒मि सा॒मि मा᳚र्जयन्त ए॒तदे॒तन् मा᳚र्जयन्ते सा॒मि सा॒मि मा᳚र्जयन्त ए॒तत् । \newline
2. मा॒र्ज॒य॒न्त॒ ए॒तदे॒तन् मा᳚र्जयन्ते मार्जयन्त ए॒तत् प्रति॒ प्रत्ये॒तन् मा᳚र्जयन्ते मार्जयन्त ए॒तत् प्रति॑ । \newline
3. ए॒तत् प्रति॒ प्रत्ये॒त दे॒तत् प्रति॒ वै वै प्रत्ये॒त दे॒तत् प्रति॒ वै । \newline
4. प्रति॒ वै वै प्रति॒ प्रति॒ वा असु॑राणा॒ मसु॑राणां॒ ॅवै प्रति॒ प्रति॒ वा असु॑राणाम् । \newline
5. वा असु॑राणा॒ मसु॑राणां॒ ॅवै वा असु॑राणां ॅय॒ज्ञो य॒ज्ञो ऽसु॑राणां॒ ॅवै वा असु॑राणां ॅय॒ज्ञ्ः । \newline
6. असु॑राणां ॅय॒ज्ञो य॒ज्ञो ऽसु॑राणा॒ मसु॑राणां ॅय॒ज्ञो वि वि य॒ज्ञो ऽसु॑राणा॒ मसु॑राणां ॅय॒ज्ञो वि । \newline
7. य॒ज्ञो वि वि य॒ज्ञो य॒ज्ञो व्य॑च्छिद्यता च्छिद्यत॒ वि य॒ज्ञो य॒ज्ञो व्य॑च्छिद्यत । \newline
8. व्य॑च्छिद्यता च्छिद्यत॒ वि व्य॑च्छिद्यत॒ ब्रह्म॑णा॒ ब्रह्म॑णा ऽच्छिद्यत॒ वि व्य॑च्छिद्यत॒ ब्रह्म॑णा । \newline
9. अ॒च्छि॒द्य॒त॒ ब्रह्म॑णा॒ ब्रह्म॑णा ऽच्छिद्यता च्छिद्यत॒ ब्रह्म॑णा दे॒वा दे॒वा ब्रह्म॑णा ऽच्छिद्यता च्छिद्यत॒ ब्रह्म॑णा दे॒वाः । \newline
10. ब्रह्म॑णा दे॒वा दे॒वा ब्रह्म॑णा॒ ब्रह्म॑णा दे॒वाः सꣳ सम् दे॒वा ब्रह्म॑णा॒ ब्रह्म॑णा दे॒वाः सम् । \newline
11. दे॒वाः सꣳ सम् दे॒वा दे॒वाः स म॑दधु रदधुः॒ सम् दे॒वा दे॒वाः स म॑दधुः । \newline
12. स म॑दधु रदधुः॒ सꣳ स म॑दधु॒र् बृह॒स्पति॒र् बृह॒स्पति॑ रदधुः॒ सꣳ स म॑दधु॒र् बृह॒स्पतिः॑ । \newline
13. अ॒द॒धु॒र् बृह॒स्पति॒र् बृह॒स्पति॑ रदधुर दधु॒र् बृह॒स्पति॑ स्तनुताम् तनुता॒म् बृह॒स्पति॑ रदधुर दधु॒र् बृह॒स्पति॑ स्तनुताम् । \newline
14. बृह॒स्पति॑ स्तनुताम् तनुता॒म् बृह॒स्पति॒र् बृह॒स्पति॑ स्तनुता मि॒म मि॒मम् त॑नुता॒म् बृह॒स्पति॒र् बृह॒स्पति॑ स्तनुता मि॒मम् । \newline
15. त॒नु॒ता॒ मि॒म मि॒मम् त॑नुताम् तनुता मि॒मम् नो॑ न इ॒मम् त॑नुताम् तनुता मि॒मम् नः॑ । \newline
16. इ॒मम् नो॑ न इ॒म मि॒मम् न॒ इतीति॑ न इ॒म मि॒मम् न॒ इति॑ । \newline
17. न॒ इतीति॑ नो न॒ इत्या॑हा॒हे ति॑ नो न॒ इत्या॑ह । \newline
18. इत्या॑हा॒हे तीत्या॑ह॒ ब्रह्म॒ ब्रह्मा॒हे तीत्या॑ह॒ ब्रह्म॑ । \newline
19. आ॒ह॒ ब्रह्म॒ ब्रह्मा॑हाह॒ ब्रह्म॒ वै वै ब्रह्मा॑हाह॒ ब्रह्म॒ वै । \newline
20. ब्रह्म॒ वै वै ब्रह्म॒ ब्रह्म॒ वै दे॒वाना᳚म् दे॒वानां॒ ॅवै ब्रह्म॒ ब्रह्म॒ वै दे॒वाना᳚म् । \newline
21. वै दे॒वाना᳚म् दे॒वानां॒ ॅवै वै दे॒वाना॒म् बृह॒स्पति॒र् बृह॒स्पति॑र् दे॒वानां॒ ॅवै वै दे॒वाना॒म् बृह॒स्पतिः॑ । \newline
22. दे॒वाना॒म् बृह॒स्पति॒र् बृह॒स्पति॑र् दे॒वाना᳚म् दे॒वाना॒म् बृह॒स्पति॒र् ब्रह्म॑णा॒ ब्रह्म॑णा॒ बृह॒स्पति॑र् दे॒वाना᳚म् दे॒वाना॒म् बृह॒स्पति॒र् ब्रह्म॑णा । \newline
23. बृह॒स्पति॒र् ब्रह्म॑णा॒ ब्रह्म॑णा॒ बृह॒स्पति॒र् बृह॒स्पति॒र् ब्रह्म॑णै॒वैव ब्रह्म॑णा॒ बृह॒स्पति॒र् बृह॒स्पति॒र् ब्रह्म॑णै॒व । \newline
24. ब्रह्म॑णै॒वैव ब्रह्म॑णा॒ ब्रह्म॑णै॒व य॒ज्ञ्ं ॅय॒ज्ञ् मे॒व ब्रह्म॑णा॒ ब्रह्म॑णै॒व य॒ज्ञ्म् । \newline
25. ए॒व य॒ज्ञ्ं ॅय॒ज्ञ् मे॒वैव य॒ज्ञ्ꣳ सꣳ सं ॅय॒ज्ञ् मे॒वैव य॒ज्ञ्ꣳ सम् । \newline
26. य॒ज्ञ्ꣳ सꣳ सं ॅय॒ज्ञ्ं ॅय॒ज्ञ्ꣳ सम् द॑धाति दधाति॒ सं ॅय॒ज्ञ्ं ॅय॒ज्ञ्ꣳ सम् द॑धाति । \newline
27. सम् द॑धाति दधाति॒ सꣳ सम् द॑धाति॒ विच्छि॑न्नं॒ ॅविच्छि॑न्नम् दधाति॒ सꣳ सम् द॑धाति॒ विच्छि॑न्नम् । \newline
28. द॒धा॒ति॒ विच्छि॑न्नं॒ ॅविच्छि॑न्नम् दधाति दधाति॒ विच्छि॑न्नं ॅय॒ज्ञ्ं ॅय॒ज्ञ्ं ॅविच्छि॑न्नम् दधाति दधाति॒ विच्छि॑न्नं ॅय॒ज्ञ्म् । \newline
29. विच्छि॑न्नं ॅय॒ज्ञ्ं ॅय॒ज्ञ्ं ॅविच्छि॑न्नं॒ ॅविच्छि॑न्नं ॅय॒ज्ञ्ꣳ सꣳ सं ॅय॒ज्ञ्ं ॅविच्छि॑न्नं॒ ॅविच्छि॑न्नं ॅय॒ज्ञ्ꣳ सम् । \newline
30. विच्छि॑न्न॒मिति॒ वि - छि॒न्न॒म् । \newline
31. य॒ज्ञ्ꣳ सꣳ सं ॅय॒ज्ञ्ं ॅय॒ज्ञ्ꣳ स मि॒म मि॒मꣳ सं ॅय॒ज्ञ्ं ॅय॒ज्ञ्ꣳ स मि॒मम् । \newline
32. स मि॒म मि॒मꣳ सꣳ स मि॒मम् द॑धातु दधात्वि॒मꣳ सꣳ स मि॒मम् द॑धातु । \newline
33. इ॒मम् द॑धातु दधात्वि॒म मि॒मम् द॑धा॒त्वितीति॑ दधात्वि॒म मि॒मम् द॑धा॒त्विति॑ । \newline
34. द॒धा॒त्वितीति॑ दधातु दधा॒ त्वित्या॑हा॒हे ति॑ दधातु दधा॒ त्वित्या॑ह । \newline
35. इत्या॑हा॒हे तीत्या॑ह॒ सन्त॑त्यै॒ सन्त॑त्या आ॒हे तीत्या॑ह॒ सन्त॑त्यै । \newline
36. आ॒ह॒ सन्त॑त्यै॒ सन्त॑त्या आहाह॒ सन्त॑त्यै॒ विश्वे॒ विश्वे॒ सन्त॑त्या आहाह॒ सन्त॑त्यै॒ विश्वे᳚ । \newline
37. सन्त॑त्यै॒ विश्वे॒ विश्वे॒ सन्त॑त्यै॒ सन्त॑त्यै॒ विश्वे॑ दे॒वा दे॒वा विश्वे॒ सन्त॑त्यै॒ सन्त॑त्यै॒ विश्वे॑ दे॒वाः । \newline
38. सन्त॑त्या॒ इति॒ सं - त॒त्यै॒ । \newline
39. विश्वे॑ दे॒वा दे॒वा विश्वे॒ विश्वे॑ दे॒वा इ॒हे ह दे॒वा विश्वे॒ विश्वे॑ दे॒वा इ॒ह । \newline
40. दे॒वा इ॒हे ह दे॒वा दे॒वा इ॒ह मा॑दयन्ताम् मादयन्ता मि॒ह दे॒वा दे॒वा इ॒ह मा॑दयन्ताम् । \newline
41. इ॒ह मा॑दयन्ताम् मादयन्ता मि॒हे ह मा॑दयन्ता॒ मितीति॑ मादयन्ता मि॒हे ह मा॑दयन्ता॒ मिति॑ । \newline
42. मा॒द॒य॒न्ता॒ मितीति॑ मादयन्ताम् मादयन्ता॒ मित्या॑हा॒हे ति॑ मादयन्ताम् मादयन्ता॒ मित्या॑ह । \newline
43. इत्या॑हा॒हे तीत्या॑ह स॒न्तत्य॑ स॒न्तत्या॒हे तीत्या॑ह स॒न्तत्य॑ । \newline
44. आ॒ह॒ स॒न्तत्य॑ स॒न्तत्या॑हाह स॒न्त त्यै॒वैव स॒न्तत्या॑हाह स॒न्त त्यै॒व । \newline
45. स॒न्तत्यै॒वैव स॒न्तत्य॑ स॒न्तत्यै॒व य॒ज्ञ्ं ॅय॒ज्ञ् मे॒व स॒न्तत्य॑ स॒न्तत्यै॒व य॒ज्ञ्म् । \newline
46. स॒न्तत्येति॑ सं - तत्य॑ । \newline
47. ए॒व य॒ज्ञ्ं ॅय॒ज्ञ् मे॒वैव य॒ज्ञ्म् दे॒वेभ्यो॑ दे॒वेभ्यो॑ य॒ज्ञ् मे॒वैव य॒ज्ञ्म् दे॒वेभ्यः॑ । \newline
48. य॒ज्ञ्म् दे॒वेभ्यो॑ दे॒वेभ्यो॑ य॒ज्ञ्ं ॅय॒ज्ञ्म् दे॒वेभ्यो ऽन्वनु॑ दे॒वेभ्यो॑ य॒ज्ञ्ं ॅय॒ज्ञ्म् दे॒वेभ्यो ऽनु॑ । \newline
49. दे॒वेभ्यो ऽन्वनु॑ दे॒वेभ्यो॑ दे॒वेभ्यो ऽनु॑ दिशति दिश॒त्यनु॑ दे॒वेभ्यो॑ दे॒वेभ्यो ऽनु॑ दिशति । \newline
50. अनु॑ दिशति दिश॒ त्यन्वनु॑ दिशति॒ यां ॅयाम् दि॑श॒ त्यन्वनु॑ दिशति॒ याम् । \newline
51. दि॒श॒ति॒ यां ॅयाम् दि॑शति दिशति॒ यां ॅवै वै याम् दि॑शति दिशति॒ यां ॅवै । \newline
52. यां ॅवै वै यां ॅयां ॅवै य॒ज्ञे य॒ज्ञे वै यां ॅयां ॅवै य॒ज्ञे । \newline
53. वै य॒ज्ञे य॒ज्ञे वै वै य॒ज्ञे दक्षि॑णा॒म् दक्षि॑णां ॅय॒ज्ञे वै वै य॒ज्ञे दक्षि॑णाम् । \newline
\pagebreak
\markright{ TS 1.7.1.6  \hfill https://www.vedavms.in \hfill}
\addcontentsline{toc}{section}{ TS 1.7.1.6 }
\section*{ TS 1.7.1.6 }

\textbf{TS 1.7.1.6 } \newline
\textbf{Samhita Paata} \newline

य॒ज्ञे दक्षि॑णां॒ ददा॑ति॒ ताम॑स्य प॒शवोऽनु॒ सं क्रा॑मन्ति॒ स ए॒ष ई॑जा॒नो॑ऽप॒शुर् भावु॑को॒ यज॑मानेन॒ खलु॒ वै तत्का॒र्य॑-मित्या॑हु॒र् यथा॑ देव॒त्रा द॒त्तं कु॑र्वी॒तात्मन् प॒शून् र॒मये॒तेति॒ ब्रद्ध्न॒ पिन्व॒स्वेत्या॑ह य॒ज्ञो वै ब्र॒द्ध्नो य॒ज्ञ्मे॒व तन्म॑हय॒त्यथो॑ देव॒त्रैव द॒त्तं कु॑रुत आ॒त्मन् प॒शून् र॑मयते॒ दद॑तो मे॒ ( ) मा क्षा॒यीत्या॒हाक्षि॑ति-मे॒वोपै॑ति कुर्व॒तो मे॒ मोप॑ दस॒दित्या॑ह भू॒मान॑मे॒वोपै॑ति ॥ \newline

\textbf{Pada Paata} \newline

य॒ज्ञे । दक्षि॑णाम् । ददा॑ति । ताम् । अ॒स्य॒ । प॒शवः॑ । अनु॑ । समिति॑ । क्रा॒म॒न्ति॒ । सः । ए॒षः । ई॒जा॒नः । अ॒प॒शुः । भावु॑कः । यज॑मानेन । खलु॑ । वै । तत् । का॒र्य᳚म् । इति॑ । आ॒हुः॒ । यथा᳚ । दे॒व॒त्रेति॑ देव - त्रा । द॒त्तम् । कु॒र्वी॒त । आ॒त्मन्न् । प॒शून् । र॒मये॑त । इति॑ । ब्रद्ध्न॑ । पिन्व॑स्व । इति॑ । आ॒ह॒ । य॒ज्ञ्ः । वै । ब्र॒द्ध्नः । य॒ज्ञ्म् । ए॒व । तत् । म॒ह॒य॒ति॒ । अथो॒ इति॑ । दे॒व॒त्रेति॑ देव - त्रा । ए॒व । द॒त्तम् । कु॒रु॒ते॒ । आ॒त्मन्न् । प॒शून् । र॒म॒य॒ते॒ । दद॑तः । मे॒ ( ) । मा । क्षा॒यि॒ । इति॑ । आ॒ह॒ । अक्षि॑तिम् । ए॒व । उपेति॑ । ए॒ति॒ । कु॒र्व॒तः । मे॒ । मा । उपेति॑ । द॒स॒त् । इति॑ । आ॒ह॒ । भू॒मान᳚म् । ए॒व । उपेति॑ । ए॒ति॒ ॥  \newline


\textbf{Krama Paata} \newline

य॒ज्ञे दक्षि॑णाम् । दक्षि॑णा॒म् ददा॑ति । ददा॑ति॒ ताम् । ताम॑स्य । अ॒स्य॒ प॒शवः॑ । प॒शवोऽनु॑ । अनु॒ सम् । सम् क्रा॑मन्ति । क्रा॒म॒न्ति॒ सः । स ए॒षः । ए॒ष ई॑जा॒नः । ई॒जा॒नो॑ऽप॒शुः । अ॒प॒शुर्,भावु॑कः । भावु॑को॒ यज॑मानेन । यज॑मानेन॒ खलु॑ । खलु॒ वै । वै तत् । तत् का॒र्य᳚म् । का॒र्य॑मिति॑ । इत्या॑हुः । आ॒हु॒र्,यथा᳚ । यथा॑ देव॒त्रा । दे॒व॒त्रा द॒त्तम् । दे॒व॒त्रेति॑ देव - त्रा । द॒त्तम् कु॑र्वी॒त । कु॒र्वी॒तात्मन्न् । आ॒त्मन् प॒शून् । प॒शून्,र॒मये॑त । र॒मये॒तेति॑ । इति॒ ब्रद्ध्न॑ । ब्रद्ध्न॒ पिन्व॑स्व । पिन्व॒स्वेति॑ । इत्या॑ह । आ॒ह॒ य॒ज्ञ्ः । य॒ज्ञो वै । वै ब्र॒द्ध्नः । ब्र॒द्ध्नो य॒ज्ञ्म् । य॒ज्ञ्मे॒व । ए॒व तत् । तन्म॑हयति । म॒ह॒य॒त्यथो᳚ । अथो॑ देव॒त्रा । अथो॒ इत्यथो᳚ । दे॒व॒त्रैव । दे॒व॒त्रेति॑ देव - त्रा । ए॒व द॒त्तम् । द॒त्तम् कु॑रुते । कु॒रु॒त॒ आ॒त्मन्न् । आ॒त्मन्,प॒शून् । प॒शून्,र॑मयते । र॒म॒य॒ते॒ दद॑तः । दद॑तो मे ( ) । मे॒ मा । मा क्षा॑यि । क्षा॒यीति॑ । इत्या॑ह । आ॒हाक्षि॑तिम् । अक्षि॑तिमे॒व । ए॒वोप॑ । उपै॑ति । ए॒ति॒ कु॒र्व॒तः । कु॒र्व॒तो मे᳚ । मे॒ मा । मोप॑ । उप॑ दसत् । द॒स॒दिति॑ । इत्या॑ह । आ॒ह॒ भू॒मान᳚म् । भू॒मान॑मे॒व । ए॒वोप॑ । उपै॑ति । ए॒तीत्ये॑ति । \newline

\textbf{Jatai Paata} \newline

1. य॒ज्ञे दक्षि॑णा॒म् दक्षि॑णां ॅय॒ज्ञे य॒ज्ञे दक्षि॑णाम् । \newline
2. दक्षि॑णा॒म् ददा॑ति॒ ददा॑ति॒ दक्षि॑णा॒म् दक्षि॑णा॒म् ददा॑ति । \newline
3. ददा॑ति॒ ताम् ताम् ददा॑ति॒ ददा॑ति॒ ताम् । \newline
4. ता म॑स्यास्य॒ ताम् ता म॑स्य । \newline
5. अ॒स्य॒ प॒शवः॑ प॒शवो᳚ ऽस्यास्य प॒शवः॑ । \newline
6. प॒शवो ऽन्वनु॑ प॒शवः॑ प॒शवो ऽनु॑ । \newline
7. अनु॒ सꣳ स मन्वनु॒ सम् । \newline
8. सम् क्रा॑मन्ति क्रामन्ति॒ सꣳ सम् क्रा॑मन्ति । \newline
9. क्रा॒म॒न्ति॒ स स क्रा॑मन्ति क्रामन्ति॒ सः । \newline
10. स ए॒ष ए॒ष स स ए॒षः । \newline
11. ए॒ष ई॑जा॒न ई॑जा॒न ए॒ष ए॒ष ई॑जा॒नः । \newline
12. ई॒जा॒नो॑ ऽप॒शु र॑प॒शु री॑जा॒न ई॑जा॒नो॑ ऽप॒शुः । \newline
13. अ॒प॒शुर् भावु॑को॒ भावु॑को ऽप॒शु र॑प॒शुर् भावु॑कः । \newline
14. भावु॑को॒ यज॑मानेन॒ यज॑मानेन॒ भावु॑को॒ भावु॑को॒ यज॑मानेन । \newline
15. यज॑मानेन॒ खलु॒ खलु॒ यज॑मानेन॒ यज॑मानेन॒ खलु॑ । \newline
16. खलु॒ वै वै खलु॒ खलु॒ वै । \newline
17. वै तत् तद् वै वै तत् । \newline
18. तत् का॒र्य॑म् का॒र्य॑म् तत् तत् का॒र्य᳚म् । \newline
19. का॒र्य॑ मितीति॑ का॒र्य॑म् का॒र्य॑ मिति॑ । \newline
20. इत्या॑हु राहु॒ रिती त्या॑हुः । \newline
21. आ॒हु॒र् यथा॒ यथा॑ ऽऽहु राहु॒र् यथा᳚ । \newline
22. यथा॑ देव॒त्रा दे॑व॒त्रा यथा॒ यथा॑ देव॒त्रा । \newline
23. दे॒व॒त्रा द॒त्तम् द॒त्तम् दे॑व॒त्रा दे॑व॒त्रा द॒त्तम् । \newline
24. दे॒व॒त्रेति॑ देव - त्रा । \newline
25. द॒त्तम् कु॑र्वी॒त कु॑र्वी॒त द॒त्तम् द॒त्तम् कु॑र्वी॒त । \newline
26. कु॒र्वी॒ तात्मन् ना॒त्मन् कु॑र्वी॒त कु॑र्वी॒ तात्मन्न् । \newline
27. आ॒त्मन् प॒शून् प॒शू ना॒त्मन् ना॒त्मन् प॒शून् । \newline
28. प॒शून् र॒मये॑त र॒मये॑त प॒शून् प॒शून् र॒मये॑त । \newline
29. र॒मये॒ते तीति॑ र॒मये॑त र॒मये॒ते ति॑ । \newline
30. इति॒ ब्रद्ध्न॒ ब्रद्ध्ने तीति॒ ब्रद्ध्न॑ । \newline
31. ब्रद्ध्न॒ पिन्व॑स्व॒ पिन्व॑स्व॒ ब्रद्ध्न॒ ब्रद्ध्न॒ पिन्व॑स्व । \newline
32. पिन्व॒स्वे तीति॒ पिन्व॑स्व॒ पिन्व॒स्वे ति॑ । \newline
33. इत्या॑हा॒हे तीत्या॑ह । \newline
34. आ॒ह॒ य॒ज्ञो य॒ज्ञ् आ॑हाह य॒ज्ञ्ः । \newline
35. य॒ज्ञो वै वै य॒ज्ञो य॒ज्ञो वै । \newline
36. वै ब्र॒द्ध्नो ब्र॒द्ध्नो वै वै ब्र॒द्ध्नः । \newline
37. ब्र॒द्ध्नो य॒ज्ञ्ं ॅय॒ज्ञ्म् ब्र॒द्ध्नो ब्र॒द्ध्नो य॒ज्ञ्म् । \newline
38. य॒ज्ञ् मे॒वैव य॒ज्ञ्ं ॅय॒ज्ञ् मे॒व । \newline
39. ए॒व तत् तदे॒वैव तत् । \newline
40. तन् म॑हयति महयति॒ तत् तन् म॑हयति । \newline
41. म॒ह॒य॒ त्यथो॒ अथो॑ महयति महय॒ त्यथो᳚ । \newline
42. अथो॑ देव॒त्रा दे॑व॒त्रा ऽथो॒ अथो॑ देव॒त्रा । \newline
43. अथो॒ इत्यथो᳚ । \newline
44. दे॒व॒त्रैवैव दे॑व॒त्रा दे॑व॒त्रैव । \newline
45. दे॒व॒त्रेति॑ देव - त्रा । \newline
46. ए॒व द॒त्तम् द॒त्त मे॒वैव द॒त्तम् । \newline
47. द॒त्तम् कु॑रुते कुरुते द॒त्तम् द॒त्तम् कु॑रुते । \newline
48. कु॒रु॒त॒ आ॒त्मन् ना॒त्मन् कु॑रुते कुरुत आ॒त्मन्न् । \newline
49. आ॒त्मन् प॒शून् प॒शू ना॒त्मन् ना॒त्मन् प॒शून् । \newline
50. प॒शून् र॑मयते रमयते प॒शून् प॒शून् र॑मयते । \newline
51. र॒म॒य॒ते॒ दद॑तो॒ दद॑तो रमयते रमयते॒ दद॑तः । \newline
52. दद॑तो मे मे॒ दद॑तो॒ दद॑तो मे । \newline
53. मे॒ मा मा मे॑ मे॒ मा । \newline
54. मा क्षा॑यि क्षायि॒ मा मा क्षा॑यि । \newline
55. क्षा॒यीतीति॑ क्षायि क्षा॒यीति॑ । \newline
56. इत्या॑हा॒हे तीत्या॑ह । \newline
57. आ॒हा क्षि॑ति॒ मक्षि॑ति माहा॒हा क्षि॑तिम् । \newline
58. अक्षि॑ति मे॒वै वाक्षि॑ति॒ मक्षि॑ति मे॒व । \newline
59. ए॒वो पोपै॒ वै वोप॑ । \newline
60. उपै᳚ त्ये॒ त्युपो पै॑ति । \newline
61. ए॒ति॒ कु॒र्व॒तः कु॑र्व॒त ए᳚त्येति कुर्व॒तः । \newline
62. कु॒र्व॒तो मे॑ मे कुर्व॒तः कु॑र्व॒तो मे᳚ । \newline
63. मे॒ मा मा मे॑ मे॒ मा । \newline
64. मोपोप॒ मा मोप॑ । \newline
65. उप॑ दसद् दस॒ दुपोप॑ दसत् । \newline
66. द॒स॒ दितीति॑ दसद् दस॒ दिति॑ । \newline
67. इत्या॑हा॒हे तीत्या॑ह । \newline
68. आ॒ह॒ भू॒मान॑म् भू॒मान॑ माहाह भू॒मान᳚म् । \newline
69. भू॒मान॑ मे॒वैव भू॒मान॑म् भू॒मान॑ मे॒व । \newline
70. ए॒वो पोपै॒ वै वोप॑ । \newline
71. उपै᳚ त्ये॒ त्युपो पै॑ति । \newline
72. ए॒तीत्ये॑ति । \newline

\textbf{Ghana Paata } \newline

1. य॒ज्ञे दक्षि॑णा॒म् दक्षि॑णां ॅय॒ज्ञे य॒ज्ञे दक्षि॑णा॒म् ददा॑ति॒ ददा॑ति॒ दक्षि॑णां ॅय॒ज्ञे य॒ज्ञे दक्षि॑णा॒म् ददा॑ति । \newline
2. दक्षि॑णा॒म् ददा॑ति॒ ददा॑ति॒ दक्षि॑णा॒म् दक्षि॑णा॒म् ददा॑ति॒ ताम् ताम् ददा॑ति॒ दक्षि॑णा॒म् दक्षि॑णा॒म् ददा॑ति॒ ताम् । \newline
3. ददा॑ति॒ ताम् ताम् ददा॑ति॒ ददा॑ति॒ ता म॑स्यास्य॒ ताम् ददा॑ति॒ ददा॑ति॒ ता म॑स्य । \newline
4. ता म॑स्यास्य॒ ताम् ता म॑स्य प॒शवः॑ प॒शवो᳚ ऽस्य॒ ताम् ता म॑स्य प॒शवः॑ । \newline
5. अ॒स्य॒ प॒शवः॑ प॒शवो᳚ ऽस्यास्य प॒शवो ऽन्वनु॑ प॒शवो᳚ ऽस्यास्य प॒शवो ऽनु॑ । \newline
6. प॒शवो ऽन्वनु॑ प॒शवः॑ प॒शवो ऽनु॒ सꣳ स मनु॑ प॒शवः॑ प॒शवो ऽनु॒ सम् । \newline
7. अनु॒ सꣳ स मन्वनु॒ सम् क्रा॑मन्ति क्रामन्ति॒ स मन्वनु॒ सम् क्रा॑मन्ति । \newline
8. सम् क्रा॑मन्ति क्रामन्ति॒ सꣳ सम् क्रा॑मन्ति॒ स स क्रा॑मन्ति॒ सꣳ सम् क्रा॑मन्ति॒ सः । \newline
9. क्रा॒म॒न्ति॒ स स क्रा॑मन्ति क्रामन्ति॒ स ए॒ष ए॒ष स क्रा॑मन्ति क्रामन्ति॒ स ए॒षः । \newline
10. स ए॒ष ए॒ष स स ए॒ष ई॑जा॒न ई॑जा॒न ए॒ष स स ए॒ष ई॑जा॒नः । \newline
11. ए॒ष ई॑जा॒न ई॑जा॒न ए॒ष ए॒ष ई॑जा॒नो॑ ऽप॒शु र॑प॒शु री॑जा॒न ए॒ष ए॒ष ई॑जा॒नो॑ ऽप॒शुः । \newline
12. ई॒जा॒नो॑ ऽप॒शु र॑प॒शु री॑जा॒न ई॑जा॒नो॑ ऽप॒शुर् भावु॑को॒ भावु॑को ऽप॒शु री॑जा॒न ई॑जा॒नो॑ ऽप॒शुर् भावु॑कः । \newline
13. अ॒प॒शुर् भावु॑को॒ भावु॑को ऽप॒शु र॑प॒शुर् भावु॑को॒ यज॑मानेन॒ यज॑मानेन॒ भावु॑को ऽप॒शु र॑प॒शुर् भावु॑को॒ यज॑मानेन । \newline
14. भावु॑को॒ यज॑मानेन॒ यज॑मानेन॒ भावु॑को॒ भावु॑को॒ यज॑मानेन॒ खलु॒ खलु॒ यज॑मानेन॒ भावु॑को॒ भावु॑को॒ यज॑मानेन॒ खलु॑ । \newline
15. यज॑मानेन॒ खलु॒ खलु॒ यज॑मानेन॒ यज॑मानेन॒ खलु॒ वै वै खलु॒ यज॑मानेन॒ यज॑मानेन॒ खलु॒ वै । \newline
16. खलु॒ वै वै खलु॒ खलु॒ वै तत् तद् वै खलु॒ खलु॒ वै तत् । \newline
17. वै तत् तद् वै वै तत् का॒र्य॑म् का॒र्य॑म् तद् वै वै तत् का॒र्य᳚म् । \newline
18. तत् का॒र्य॑म् का॒र्य॑म् तत् तत् का॒र्य॑ मितीति॑ का॒र्य॑म् तत् तत् का॒र्य॑ मिति॑ । \newline
19. का॒र्य॑ मितीति॑ का॒र्य॑म् का॒र्य॑ मित्या॑हु राहु॒रिति॑ का॒र्य॑म् का॒र्य॑ मित्या॑हुः । \newline
20. इत्या॑हु राहु॒ रितीत्या॑हु॒र् यथा॒ यथा॑ ऽऽहु॒ रिती त्या॑हु॒र् यथा᳚ । \newline
21. आ॒हु॒र् यथा॒ यथा॑ ऽऽहुराहु॒र् यथा॑ देव॒त्रा दे॑व॒त्रा यथा॑ ऽऽहुराहु॒र् यथा॑ देव॒त्रा । \newline
22. यथा॑ देव॒त्रा दे॑व॒त्रा यथा॒ यथा॑ देव॒त्रा द॒त्तम् द॒त्तम् दे॑व॒त्रा यथा॒ यथा॑ देव॒त्रा द॒त्तम् । \newline
23. दे॒व॒त्रा द॒त्तम् द॒त्तम् दे॑व॒त्रा दे॑व॒त्रा द॒त्तम् कु॑र्वी॒त कु॑र्वी॒त द॒त्तम् दे॑व॒त्रा दे॑व॒त्रा द॒त्तम् कु॑र्वी॒त । \newline
24. दे॒व॒त्रेति॑ देव - त्रा । \newline
25. द॒त्तम् कु॑र्वी॒त कु॑र्वी॒त द॒त्तम् द॒त्तम् कु॑र्वी॒तात्मन् ना॒त्मन् कु॑र्वी॒त द॒त्तम् द॒त्तम् कु॑र्वी॒तात्मन्न् । \newline
26. कु॒र्वी॒तात्मन् ना॒त्मन् कु॑र्वी॒त कु॑र्वी॒तात्मन् प॒शून् प॒शू ना॒त्मन् कु॑र्वी॒त कु॑र्वी॒तात्मन् प॒शून् । \newline
27. आ॒त्मन् प॒शून् प॒शू ना॒त्मन् ना॒त्मन् प॒शून् र॒मये॑त र॒मये॑त प॒शू ना॒त्मन् ना॒त्मन् प॒शून् र॒मये॑त । \newline
28. प॒शून् र॒मये॑त र॒मये॑त प॒शून् प॒शून् र॒मये॒ते तीति॑ र॒मये॑त प॒शून् प॒शून् र॒मये॒ते ति॑ । \newline
29. र॒मये॒ते तीति॑ र॒मये॑त र॒मये॒ते ति॒ ब्रद्ध्न॒ ब्रद्ध्ने ति॑ र॒मये॑त र॒मये॒ते ति॒ ब्रद्ध्न॑ । \newline
30. इति॒ ब्रद्ध्न॒ ब्रद्ध्ने तीति॒ ब्रद्ध्न॒ पिन्व॑स्व॒ पिन्व॑स्व॒ ब्रद्ध्ने तीति॒ ब्रद्ध्न॒ पिन्व॑स्व । \newline
31. ब्रद्ध्न॒ पिन्व॑स्व॒ पिन्व॑स्व॒ ब्रद्ध्न॒ ब्रद्ध्न॒ पिन्व॒स्वे तीति॒ पिन्व॑स्व॒ ब्रद्ध्न॒ ब्रद्ध्न॒ पिन्व॒स्वे ति॑ । \newline
32. पिन्व॒स्वे तीति॒ पिन्व॑स्व॒ पिन्व॒स्वे त्या॑हा॒हे ति॒ पिन्व॑स्व॒ पिन्व॒स्वे त्या॑ह । \newline
33. इत्या॑हा॒हे तीत्या॑ह य॒ज्ञो य॒ज्ञ् आ॒हे तीत्या॑ह य॒ज्ञ्ः । \newline
34. आ॒ह॒ य॒ज्ञो य॒ज्ञ् आ॑हाह य॒ज्ञो वै वै य॒ज्ञ् आ॑हाह य॒ज्ञो वै । \newline
35. य॒ज्ञो वै वै य॒ज्ञो य॒ज्ञो वै ब्र॒द्ध्नो ब्र॒द्ध्नो वै य॒ज्ञो य॒ज्ञो वै ब्र॒द्ध्नः । \newline
36. वै ब्र॒द्ध्नो ब्र॒द्ध्नो वै वै ब्र॒द्ध्नो य॒ज्ञ्ं ॅय॒ज्ञ्म् ब्र॒द्ध्नो वै वै ब्र॒द्ध्नो य॒ज्ञ्म् । \newline
37. ब्र॒द्ध्नो य॒ज्ञ्ं ॅय॒ज्ञ्म् ब्र॒द्ध्नो ब्र॒द्ध्नो य॒ज्ञ् मे॒वैव य॒ज्ञ्म् ब्र॒द्ध्नो ब्र॒द्ध्नो य॒ज्ञ् मे॒व । \newline
38. य॒ज्ञ् मे॒वैव य॒ज्ञ्ं ॅय॒ज्ञ् मे॒व तत् तदे॒व य॒ज्ञ्ं ॅय॒ज्ञ् मे॒व तत् । \newline
39. ए॒व तत् तदे॒वैव तन् म॑हयति महयति॒ तदे॒वैव तन् म॑हयति । \newline
40. तन् म॑हयति महयति॒ तत् तन् म॑हय॒त्यथो॒ अथो॑ महयति॒ तत् तन् म॑हय॒त्यथो᳚ । \newline
41. म॒ह॒य॒त्यथो॒ अथो॑ महयति महय॒त्यथो॑ देव॒त्रा दे॑व॒त्रा ऽथो॑ महयति महय॒त्यथो॑ देव॒त्रा । \newline
42. अथो॑ देव॒त्रा दे॑व॒त्रा ऽथो॒ अथो॑ देव॒त्रैवैव दे॑व॒त्रा ऽथो॒ अथो॑ देव॒त्रैव । \newline
43. अथो॒ इत्यथो᳚ । \newline
44. दे॒व॒त्रैवैव दे॑व॒त्रा दे॑व॒त्रैव द॒त्तम् द॒त्त मे॒व दे॑व॒त्रा दे॑व॒त्रैव द॒त्तम् । \newline
45. दे॒व॒त्रेति॑ देव - त्रा । \newline
46. ए॒व द॒त्तम् द॒त्त मे॒वैव द॒त्तम् कु॑रुते कुरुते द॒त्त मे॒वैव द॒त्तम् कु॑रुते । \newline
47. द॒त्तम् कु॑रुते कुरुते द॒त्तम् द॒त्तम् कु॑रुत आ॒त्मन् ना॒त्मन् कु॑रुते द॒त्तम् द॒त्तम् कु॑रुत आ॒त्मन्न् । \newline
48. कु॒रु॒त॒ आ॒त्मन् ना॒त्मन् कु॑रुते कुरुत आ॒त्मन् प॒शून् प॒शू ना॒त्मन् कु॑रुते कुरुत आ॒त्मन् प॒शून् । \newline
49. आ॒त्मन् प॒शून् प॒शू ना॒त्मन् ना॒त्मन् प॒शून् र॑मयते रमयते प॒शू ना॒त्मन् ना॒त्मन् प॒शून् र॑मयते । \newline
50. प॒शून् र॑मयते रमयते प॒शून् प॒शून् र॑मयते॒ दद॑तो॒ दद॑तो रमयते प॒शून् प॒शून् र॑मयते॒ दद॑तः । \newline
51. र॒म॒य॒ते॒ दद॑तो॒ दद॑तो रमयते रमयते॒ दद॑तो मे मे॒ दद॑तो रमयते रमयते॒ दद॑तो मे । \newline
52. दद॑तो मे मे॒ दद॑तो॒ दद॑तो मे॒ मा मा मे॒ दद॑तो॒ दद॑तो मे॒ मा । \newline
53. मे॒ मा मा मे॑ मे॒ मा क्षा॑यि क्षायि॒ मा मे॑ मे॒ मा क्षा॑यि । \newline
54. मा क्षा॑यि क्षायि॒ मा मा क्षा॒यीतीति॑ क्षायि॒ मा मा क्षा॒यीति॑ । \newline
55. क्षा॒यीतीति॑ क्षायि क्षा॒यी त्या॑हा॒हे ति॑ क्षायि क्षा॒यीत्या॑ह । \newline
56. इत्या॑हा॒हे तीत्या॒हा क्षि॑ति॒ मक्षि॑ति मा॒हे तीत्या॒हा क्षि॑तिम् । \newline
57. आ॒हाक्षि॑ति॒ मक्षि॑ति माहा॒हा क्षि॑ति मे॒वैवा क्षि॑ति माहा॒हा क्षि॑ति मे॒व । \newline
58. अक्षि॑ति मे॒वैवा क्षि॑ति॒ मक्षि॑ति मे॒वो पोपै॒वा क्षि॑ति॒ मक्षि॑ति मे॒वोप॑ । \newline
59. ए॒वोपोपै॒ वैवोपै᳚ त्ये॒त्यु पै॒वैवोपै॑ति । \newline
60. उपै᳚त्ये॒ त्युपोपै॑ति कुर्व॒तः कु॑र्व॒त ए॒त्युपोपै॑ति कुर्व॒तः । \newline
61. ए॒ति॒ कु॒र्व॒तः कु॑र्व॒त ए᳚त्येति कुर्व॒तो मे॑ मे कुर्व॒त ए᳚त्येति कुर्व॒तो मे᳚ । \newline
62. कु॒र्व॒तो मे॑ मे कुर्व॒तः कु॑र्व॒तो मे॒ मा मा मे॑ कुर्व॒तः कु॑र्व॒तो मे॒ मा । \newline
63. मे॒ मा मा मे॑ मे॒ मोपोप॒ मा मे॑ मे॒ मोप॑ । \newline
64. मोपोप॒ मा मोप॑ दसद् दस॒दुप॒ मा मोप॑ दसत् । \newline
65. उप॑ दसद् दस॒ दुपोप॑ दस॒ दितीति॑ दस॒ दुपोप॑ दस॒दिति॑ । \newline
66. द॒स॒ दितीति॑ दसद् दस॒ दित्या॑हा॒हे ति॑ दसद् दस॒ दित्या॑ह । \newline
67. इत्या॑हा॒हे तीत्या॑ह भू॒मान॑म् भू॒मान॑ मा॒हे तीत्या॑ह भू॒मान᳚म् । \newline
68. आ॒ह॒ भू॒मान॑म् भू॒मान॑ माहाह भू॒मान॑ मे॒वैव भू॒मान॑ माहाह भू॒मान॑ मे॒व । \newline
69. भू॒मान॑ मे॒वैव भू॒मान॑म् भू॒मान॑ मे॒वोपोपै॒व भू॒मान॑म् भू॒मान॑ मे॒वोप॑ । \newline
70. ए॒वो पोपै॒वै वोपै᳚त्ये॒ त्युपै॒वै वोपै॑ति । \newline
71. उपै᳚त्ये॒त्युपोपै॑ति । \newline
72. ए॒तीत्ये॑ति । \newline
\pagebreak
\markright{ TS 1.7.2.1  \hfill https://www.vedavms.in \hfill}
\addcontentsline{toc}{section}{ TS 1.7.2.1 }
\section*{ TS 1.7.2.1 }

\textbf{TS 1.7.2.1 } \newline
\textbf{Samhita Paata} \newline

सꣳश्र॑वा ह सौवर्चन॒सः तुमि॑ञ्ज॒मौपो॑दिति-मुवाच॒ यथ्स॒त्रिणाꣳ॒॒ होताऽभूः॒ कामिडा॒मुपा᳚ह्वथा॒ इति॒ तामुपा᳚ह्व॒ इति॑ होवाच॒ या प्रा॒णेन॑ दे॒वान् दा॒धार॑ व्या॒नेन॑ मनु॒ष्या॑नपा॒नेन॑ पि॒तृनिति॑ छि॒नत्ति॒ सा न छि॑न॒त्ती(3) इति॑ छि॒नत्तीति॑ होवाच॒ शरी॑रं॒ ॅवा अ॑स्यै॒ तदुपा᳚ह्वथा॒ इति॑ होवाच॒ गौर्वा - [ ] \newline

\textbf{Pada Paata} \newline

सꣳश्र॑वा॒ इति॒ सं - श्र॒वाः॒ । ह॒ । सौ॒व॒र्च॒न॒सः । तुमि॑ञ्जम् । औपो॑दिति॒मित्यौप॑ - उ॒दि॒ति॒म् । उ॒वा॒च॒ । यत् । स॒त्रिणा᳚म् । होता᳚ । अभूः᳚ । काम् । इडा᳚म् । उपेति॑ । अ॒ह्व॒थाः॒ । इति॑ । ताम् । उपेति॑ । अ॒ह्वे॒ । इति॑ । ह॒ । उ॒वा॒च॒ । या । प्रा॒णेनेति॑ प्र - अ॒नेन॑ । दे॒वान् । दा॒धार॑ । व्या॒नेनेति॑ वि - अ॒नेन॑ । म॒नु॒ष्यान्॑ । अ॒पा॒नेनेत्य॑प-अ॒नेन॑ । पि॒तॄन् । इति॑ । छि॒नत्ति॑ । सा । न । छि॒न॒त्ती(3) । इति॑ । छि॒नत्ति॑ । इति॑ । ह॒ । उ॒वा॒च॒ । शरी॑रम् । वै । अ॒स्यै॒ । तत् । उपेति॑ । अ॒ह्व॒थाः॒ । इति॑ । ह॒ । उ॒वा॒च॒ । गौः । वै ।  \newline


\textbf{Krama Paata} \newline

सꣳश्र॑वा ह । सꣳश्र॑वा॒ इति॒ सम् - श्र॒वाः॒ । ह॒ सौ॒व॒र्च॒न॒सः । सौ॒व॒र्च॒न॒स,स्तुमि॑ञ्जम् । तुमि॑ञ्ज॒मौपो॑दितिम् । औपो॑दितिमुवाच । औपो॑दिति॒मित्यौप॑ - उ॒दि॒ति॒म् । उ॒वा॒च॒ यत् । यथ् स॒त्रिणा᳚म् । स॒त्रिणाꣳ॒॒ होता᳚ । होताऽभूः᳚ । अभूः॒ काम् । कामिडा᳚म् । इडा॒मुप॑ । उपा᳚ह्वथाः । अ॒ह्व॒था॒ इति॑ । इति॒ ताम् । तामुप॑ । उपा᳚ह्वे । अह्व॒ इति॑ । इति॑ ह । हो॒वा॒च॒ । उ॒वा॒च॒ या । या प्रा॒णेन॑ । प्रा॒णेन॑ दे॒वान् । प्रा॒णेनेति॑ प्र - अ॒नेन॑ । दे॒वान्,दा॒धार॑ । दा॒धार॑ व्या॒नेन॑ । व्या॒नेन॑ मनु॒ष्यान्॑ । व्या॒नेनेति॑ वि - अ॒नेन॑ । म॒नु॒ष्या॑नपा॒नेन॑ । अ॒पा॒नेन॑ पि॒तॄन् । अ॒पा॒नेनेत्य॑प - अ॒नेन॑ । पि॒तॄनिति॑ । इति॑ छि॒नत्ति॑ । छि॒नत्ति॒ सा । सा न । न छि॑न॒त्ती(3) । छि॒न॒त्ती(3) इति॑ । इति॑ छि॒नत्ति॑ । छि॒नत्तीति॑ । इति॑ ह । हो॒वा॒च॒ । उ॒वा॒च॒ शरी॑रम् । शरी॑रं॒ ॅवै । वा अ॑स्यै । अ॒स्यै॒ तत् । तदुप॑ । उपा᳚ह्वथाः । अ॒ह्व॒था॒ इति॑ । इति॑ ह । हो॒वा॒च॒ । उ॒वा॒च॒ गौः । गौर् वै । वा अ॑स्यै \newline

\textbf{Jatai Paata} \newline

1. सꣳश्र॑वा ह ह॒ सꣳश्र॑वाः॒ सꣳश्र॑वा ह । \newline
2. सꣳश्र॑वा॒ इति॒ सं - श्र॒वाः॒ । \newline
3. ह॒ सौ॒व॒र्च॒न॒सः सौ॑वर्चन॒सो ह॑ ह सौवर्चन॒सः । \newline
4. सौ॒व॒र्च॒न॒स स्तुमि॑ञ्ज॒म् तुमि॑ञ्जꣳ सौवर्चन॒सः सौ॑वर्चन॒स स्तुमि॑ञ्जम् । \newline
5. तुमि॑ञ्ज॒ मौपो॑दिति॒ मौपो॑दिति॒म् तुमि॑ञ्ज॒म् तुमि॑ञ्ज॒ मौपो॑दितिम् । \newline
6. औपो॑दिति मुवाचो वा॒चौ पो॑दिति॒ मौपो॑दिति मुवाच । \newline
7. औपो॑दिति॒मित्यौप॑ - उ॒दि॒ति॒म् । \newline
8. उ॒वा॒च॒ यद् यदु॑वाचो वाच॒ यत् । \newline
9. यथ् स॒त्रिणाꣳ॑ स॒त्रिणां॒ ॅयद् यथ् स॒त्रिणा᳚म् । \newline
10. स॒त्रिणाꣳ॒॒ होता॒ होता॑ स॒त्रिणाꣳ॑ स॒त्रिणाꣳ॒॒ होता᳚ । \newline
11. होता ऽभू॒ रभू॒र्॒. होता॒ होता ऽभूः᳚ । \newline
12. अभूः॒ काम् का मभू॒ रभूः॒ काम् । \newline
13. का मिडा॒ मिडा॒म् काम् का मिडा᳚म् । \newline
14. इडा॒ मुपोपे डा॒ मिडा॒ मुप॑ । \newline
15. उपा᳚ह्वथा अह्वथा॒ उपोपा᳚ह्वथाः । \newline
16. अ॒ह्व॒था॒ इती त्य॑ह्वथा अह्वथा॒ इति॑ । \newline
17. इति॒ ताम् ता मितीति॒ ताम् । \newline
18. ता मुपोप॒ ताम् ता मुप॑ । \newline
19. उपा᳚ह्वे ऽह्व॒ उपोपा᳚ह्वे । \newline
20. अ॒ह्व॒ इती त्य॑ह्वे ऽह्व॒ इति॑ । \newline
21. इति॑ ह॒ हे तीति॑ ह । \newline
22. हो॒वा॒चो॒ वा॒च॒ ह॒ हो॒वा॒च॒ । \newline
23. उ॒वा॒च॒ या योवा॑चो वाच॒ या । \newline
24. या प्रा॒णेन॑ प्रा॒णेन॒ या या प्रा॒णेन॑ । \newline
25. प्रा॒णेन॑ दे॒वान् दे॒वान् प्रा॒णेन॑ प्रा॒णेन॑ दे॒वान् । \newline
26. प्रा॒णेनेति॑ प्र - अ॒नेन॑ । \newline
27. दे॒वान् दा॒धार॑ दा॒धार॑ दे॒वान् दे॒वान् दा॒धार॑ । \newline
28. दा॒धार॑ व्या॒नेन॑ व्या॒नेन॑ दा॒धार॑ दा॒धार॑ व्या॒नेन॑ । \newline
29. व्या॒नेन॑ मनु॒ष्या᳚न् मनु॒ष्या᳚न् व्या॒नेन॑ व्या॒नेन॑ मनु॒ष्यान्॑ । \newline
30. व्या॒नेनेति॑ वि - अ॒नेन॑ । \newline
31. म॒नु॒ष्या॑ नपा॒नेना॑ पा॒नेन॑ मनु॒ष्या᳚न् मनु॒ष्या॑ नपा॒नेन॑ । \newline
32. अ॒पा॒नेन॑ पि॒तॄन् पि॒तॄ न॑पा॒नेना॑ पा॒नेन॑ पि॒तॄन् । \newline
33. अ॒पा॒नेनेत्य॑प - अ॒नेन॑ । \newline
34. पि॒तॄ नितीति॑ पि॒तॄन् पि॒तॄ निति॑ । \newline
35. इति॑ छि॒नत्ति॑ छि॒नत्तीतीति॑ छि॒नत्ति॑ । \newline
36. छि॒नत्ति॒ सा सा छि॒नत्ति॑ छि॒नत्ति॒ सा । \newline
37. सा न न सा सा न । \newline
38. न छि॑न॒त्ती(3) छि॑न॒त्ती(3) न न छि॑न॒त्ती(3) । \newline
39. छि॒न॒त्ती(3) इतीति॑ छिन॒त्ती(3) छि॑न॒त्ती(3) इति॑ । \newline
40. इति॑ छि॒नत्ति॑ छि॒नत्ती तीति॑ छि॒नत्ति॑ । \newline
41. छि॒नत्ती तीति॑ छि॒नत्ति॑ छि॒नत्ती ति॑ । \newline
42. इति॑ ह॒ हे तीति॑ ह । \newline
43. हो॒वा॒चो॒ वा॒च॒ ह॒ हो॒वा॒च॒ । \newline
44. उ॒वा॒च॒ शरी॑रꣳ॒॒ शरी॑र मुवाचो वाच॒ शरी॑रम् । \newline
45. शरी॑रं॒ ॅवै वै शरी॑रꣳ॒॒ शरी॑रं॒ ॅवै । \newline
46. वा अ॑स्या अस्यै॒ वै वा अ॑स्यै । \newline
47. अ॒स्यै॒ तत् तद॑स्या अस्यै॒ तत् । \newline
48. तदुपोप॒ तत् तदुप॑ । \newline
49. उपा᳚ह्वथा अह्वथा॒ उपोपा᳚ह्वथाः । \newline
50. अ॒ह्व॒था॒ इती त्य॑ह्वथा अह्वथा॒ इति॑ । \newline
51. इति॑ ह॒ हे तीति॑ ह । \newline
52. हो॒वा॒चो॒ वा॒च॒ ह॒ हो॒वा॒च॒ । \newline
53. उ॒वा॒च॒ गौर् गौरु॑वाचो वाच॒ गौः । \newline
54. गौर् वै वै गौर् गौर् वै । \newline
55. वा अ॑स्या अस्यै॒ वै वा अ॑स्यै । \newline

\textbf{Ghana Paata } \newline

1. सꣳश्र॑वा ह ह॒ सꣳश्र॑वाः॒ सꣳश्र॑वा ह सौवर्चन॒सः सौ॑वर्चन॒सो ह॒ सꣳश्र॑वाः॒ सꣳश्र॑वा ह सौवर्चन॒सः । \newline
2. सꣳश्र॑वा॒ इति॒ सं - श्र॒वाः॒ । \newline
3. ह॒ सौ॒व॒र्च॒न॒सः सौ॑वर्चन॒सो ह॑ ह सौवर्चन॒स स्तुमि॑ञ्ज॒म् तुमि॑ञ्जꣳ सौवर्चन॒सो ह॑ ह सौवर्चन॒स स्तुमि॑ञ्जम् । \newline
4. सौ॒व॒र्च॒न॒स स्तुमि॑ञ्ज॒म् तुमि॑ञ्जꣳ सौवर्चन॒सः सौ॑वर्चन॒स स्तुमि॑ञ्ज॒ मौपो॑दिति॒ मौपो॑दिति॒म् तुमि॑ञ्जꣳ सौवर्चन॒सः सौ॑वर्चन॒स स्तुमि॑ञ्ज॒ मौपो॑दितिम् । \newline
5. तुमि॑ञ्ज॒ मौपो॑दिति॒ मौपो॑दिति॒म् तुमि॑ञ्ज॒म् तुमि॑ञ्ज॒ मौपो॑दिति मुवाचोवा॒ चौपो॑दिति॒म् तुमि॑ञ्ज॒म् तुमि॑ञ्ज॒ मौपो॑दिति मुवाच । \newline
6. औपो॑दिति मुवाचोवा॒ चौपो॑दिति॒ मौपो॑दिति मुवाच॒ यद् यदु॑वा॒ चौपो॑दिति॒ मौपो॑दिति मुवाच॒ यत् । \newline
7. औपो॑दिति॒मित्यौप॑ - उ॒दि॒ति॒म् । \newline
8. उ॒वा॒च॒ यद् यदु॑वाचो वाच॒ यथ् स॒त्रिणा(ग्म्॑) स॒त्रिणां॒ ॅयदु॑वाचो वाच॒ यथ् स॒त्रिणा᳚म् । \newline
9. यथ् स॒त्रिणा(ग्म्॑) स॒त्रिणां॒ ॅयद् यथ् स॒त्रिणा॒(ग्म्॒) होता॒ होता॑ स॒त्रिणां॒ ॅयद् यथ् स॒त्रिणा॒(ग्म्॒) होता᳚ । \newline
10. स॒त्रिणा॒(ग्म्॒) होता॒ होता॑ स॒त्रिणा(ग्म्॑) स॒त्रिणा॒(ग्म्॒) होता ऽभू॒रभू॒र्॒. होता॑ स॒त्रिणा(ग्म्॑) स॒त्रिणा॒(ग्म्॒) होता ऽभूः᳚ । \newline
11. होता ऽभू॒रभू॒र्॒. होता॒ होता ऽभूः॒ काम् का मभू॒र्॒. होता॒ होता ऽभूः॒ काम् । \newline
12. अभूः॒ काम् का मभू॒ रभूः॒ का मिडा॒ मिडा॒म् का मभू॒ रभूः॒ का मिडा᳚म् । \newline
13. का मिडा॒ मिडा॒म् काम् का मिडा॒ मुपोपे डा॒म् काम् का मिडा॒ मुप॑ । \newline
14. इडा॒ मुपोपे डा॒ मिडा॒ मुपा᳚ह्वथा अह्वथा॒ उपे डा॒ मिडा॒ मुपा᳚ह्वथाः । \newline
15. उपा᳚ह्वथा अह्वथा॒ उपोपा᳚ह्वथा॒ इती त्य॑ह्वथा॒ उपोपा᳚ह्वथा॒ इति॑ । \newline
16. अ॒ह्व॒था॒ इतीत्य॑ह्वथा अह्वथा॒ इति॒ ताम् ता मित्य॑ह्वथा अह्वथा॒ इति॒ ताम् । \newline
17. इति॒ ताम् ता मितीति॒ ता मुपोप॒ ता मितीति॒ ता मुप॑ । \newline
18. ता मुपोप॒ ताम् ता मुपा᳚ह्वे ऽह्व॒ उप॒ ताम् ता मुपा᳚ह्वे । \newline
19. उपा᳚ह्वे ऽह्व॒ उपोपा᳚ह्व॒ इतीत्य॑ह्व॒ उपोपा᳚ह्व॒ इति॑ । \newline
20. अ॒ह्व॒ इतीत्य॑ह्वे ऽह्व॒ इति॑ ह॒ हे त्य॑ह्वे ऽह्व॒ इति॑ ह । \newline
21. इति॑ ह॒ हे तीति॑ होवाचो वाच॒ हे तीति॑ होवाच । \newline
22. हो॒वा॒चो॒ वा॒च॒ ह॒ हो॒वा॒च॒ या योवा॑च ह होवाच॒ या । \newline
23. उ॒वा॒च॒ या योवा॑चो वाच॒ या प्रा॒णेन॑ प्रा॒णेन॒ योवा॑चो वाच॒ या प्रा॒णेन॑ । \newline
24. या प्रा॒णेन॑ प्रा॒णेन॒ या या प्रा॒णेन॑ दे॒वान् दे॒वान् प्रा॒णेन॒ या या प्रा॒णेन॑ दे॒वान् । \newline
25. प्रा॒णेन॑ दे॒वान् दे॒वान् प्रा॒णेन॑ प्रा॒णेन॑ दे॒वान् दा॒धार॑ दा॒धार॑ दे॒वान् प्रा॒णेन॑ प्रा॒णेन॑ दे॒वान् दा॒धार॑ । \newline
26. प्रा॒णेनेति॑ प्र - अ॒नेन॑ । \newline
27. दे॒वान् दा॒धार॑ दा॒धार॑ दे॒वान् दे॒वान् दा॒धार॑ व्या॒नेन॑ व्या॒नेन॑ दा॒धार॑ दे॒वान् दे॒वान् दा॒धार॑ व्या॒नेन॑ । \newline
28. दा॒धार॑ व्या॒नेन॑ व्या॒नेन॑ दा॒धार॑ दा॒धार॑ व्या॒नेन॑ मनु॒ष्या᳚न् मनु॒ष्या᳚न् व्या॒नेन॑ दा॒धार॑ दा॒धार॑ व्या॒नेन॑ मनु॒ष्यान्॑ । \newline
29. व्या॒नेन॑ मनु॒ष्या᳚न् मनु॒ष्या᳚न् व्या॒नेन॑ व्या॒नेन॑ मनु॒ष्या॑ नपा॒नेना॑ पा॒नेन॑ मनु॒ष्या᳚न् व्या॒नेन॑ व्या॒नेन॑ मनु॒ष्या॑ नपा॒नेन॑ । \newline
30. व्या॒नेनेति॑ वि - अ॒नेन॑ । \newline
31. म॒नु॒ष्या॑ नपा॒नेना॑ पा॒नेन॑ मनु॒ष्या᳚न् मनु॒ष्या॑ नपा॒नेन॑ पि॒तॄन् पि॒तॄ न॑पा॒नेन॑ मनु॒ष्या᳚न् मनु॒ष्या॑ नपा॒नेन॑ पि॒तॄन् । \newline
32. अ॒पा॒नेन॑ पि॒तॄन् पि॒तॄ न॑पा॒नेना॑ पा॒नेन॑ पि॒तॄ नितीति॑ पि॒तॄ न॑पा॒नेना॑ पा॒नेन॑ पि॒तॄ निति॑ । \newline
33. अ॒पा॒नेनेत्य॑प - अ॒नेन॑ । \newline
34. पि॒तॄ नितीति॑ पि॒तॄन् पि॒तॄ निति॑ छि॒नत्ति॑ छि॒नत्तीति॑ पि॒तॄन् पि॒तॄ निति॑ छि॒नत्ति॑ । \newline
35. इति॑ छि॒नत्ति॑ छि॒नत्ती तीति॑ छि॒नत्ति॒ सा सा छि॒नत्ती तीति॑ छि॒नत्ति॒ सा । \newline
36. छि॒नत्ति॒ सा सा छि॒नत्ति॑ छि॒नत्ति॒ सा न न सा छि॒नत्ति॑ छि॒नत्ति॒ सा न । \newline
37. सा न न सा सा न छि॑न॒त्ती(3) छि॑न॒त्ती(3) न सा सा न छि॑न॒त्ती(3) । \newline
38. न छि॑न॒त्ती(3) छि॑न॒त्ती(3) न न छि॑न॒त्ती(3) इतीति॑ छिन॒त्ती(3) न न छि॑न॒त्ती(3) इति॑ । \newline
39. छि॒न॒त्ती(3) इतीति॑ छिन॒त्ती(3) छि॑न॒त्ती(3) इति॑ छि॒नत्ति॑ छि॒नत्तीति॑ छिन॒त्ती(3) छि॑न॒त्ती(3) इति॑ छि॒नत्ति॑ । \newline
40. इति॑ छि॒नत्ति॑ छि॒नत्ती तीति॑ छि॒नत्ती तीति॑ छि॒नत्ती तीति॑ छि॒नत्तीति॑ । \newline
41. छि॒नत्ती तीति॑ छि॒नत्ति॑ छि॒नत्तीति॑ ह॒ हे ति॑ छि॒नत्ति॑ छि॒नत्तीति॑ ह । \newline
42. इति॑ ह॒ हे तीति॑ होवाचो वाच॒ हे तीति॑ होवाच । \newline
43. हो॒वा॒चो॒ वा॒च॒ ह॒ हो॒वा॒च॒ शरी॑र॒(ग्म्॒) शरी॑र मुवाच ह होवाच॒ शरी॑रम् । \newline
44. उ॒वा॒च॒ शरी॑र॒(ग्म्॒) शरी॑र मुवाचो वाच॒ शरी॑रं॒ ॅवै वै शरी॑र मुवाचो वाच॒ शरी॑रं॒ ॅवै । \newline
45. शरी॑रं॒ ॅवै वै शरी॑र॒(ग्म्॒) शरी॑रं॒ ॅवा अ॑स्या अस्यै॒ वै शरी॑र॒(ग्म्॒) शरी॑रं॒ ॅवा अ॑स्यै । \newline
46. वा अ॑स्या अस्यै॒ वै वा अ॑स्यै॒ तत् तद॑स्यै॒ वै वा अ॑स्यै॒ तत् । \newline
47. अ॒स्यै॒ तत् तद॑स्या अस्यै॒ तदुपोप॒ तद॑स्या अस्यै॒ तदुप॑ । \newline
48. तदुपोप॒ तत् तदुपा᳚ह्वथा अह्वथा॒ उप॒ तत् तदुपा᳚ह्वथाः । \newline
49. उपा᳚ह्वथा अह्वथा॒ उपोपा᳚ह्वथा॒ इतीत्य॑ह्वथा॒ उपोपा᳚ह्वथा॒ इति॑ । \newline
50. अ॒ह्व॒था॒ इतीत्य॑ह्वथा अह्वथा॒ इति॑ ह॒ हे त्य॑ह्वथा अह्वथा॒ इति॑ ह । \newline
51. इति॑ ह॒ हे तीति॑ होवाचो वाच॒ हे तीति॑ होवाच । \newline
52. हो॒वा॒चो॒ वा॒च॒ ह॒ हो॒वा॒च॒ गौर् गौ रु॑वाच ह होवाच॒ गौः । \newline
53. उ॒वा॒च॒ गौर् गौ रु॑वाचो वाच॒ गौर् वै वै गौ रु॑वाचो वाच॒ गौर् वै । \newline
54. गौर् वै वै गौर् गौर् वा अ॑स्या अस्यै॒ वै गौर् गौर् वा अ॑स्यै । \newline
55. वा अ॑स्या अस्यै॒ वै वा अ॑स्यै॒ शरी॑र॒(ग्म्॒) शरी॑र मस्यै॒ वै वा अ॑स्यै॒ शरी॑रम् । \newline
\pagebreak
\markright{ TS 1.7.2.2  \hfill https://www.vedavms.in \hfill}
\addcontentsline{toc}{section}{ TS 1.7.2.2 }
\section*{ TS 1.7.2.2 }

\textbf{TS 1.7.2.2 } \newline
\textbf{Samhita Paata} \newline

अ॑स्यै॒ शरी॑रं॒ गां ॅवाव तौ तत् पर्य॑वदतां॒ ॅया य॒ज्ञे दी॒यते॒ सा प्रा॒णेन॑ दे॒वान् दा॑धार॒ यया॑ मनु॒ष्या॑ जीव॑न्ति॒ सा व्या॒नेन॑ मनु॒ष्यान्॑ यां पि॒तृभ्यो॒ घ्नन्ति॒ साऽपा॒नेन॑ पि॒तॄन्. य ए॒वं ॅवेद॑ पशु॒मान् भ॑व॒त्यथ॒ वै तामुपा᳚ह्व॒ इति॑ होवाच॒ या प्र॒जाः प्र॒भव॑न्तीः॒ प्रत्या॒भव॒तीत्यन्नं॒ ॅवा अ॑स्यै॒ तदु - [ ] \newline

\textbf{Pada Paata} \newline

अ॒स्यै॒ । शरी॑रम् । गाम् । वाव । तौ । तत् । परीति॑ । अ॒व॒द॒ता॒म् । या । य॒ज्ञे । दी॒यते᳚ । सा । प्रा॒णेनेति॑ प्र - अ॒नेन॑ । दे॒वान् । दा॒धा॒र॒ । यया᳚ । म॒नु॒ष्याः᳚ । जीव॑न्ति । सा । व्या॒नेनेति॑ वि - अ॒नेन॑ । म॒नु॒ष्यान्॑ । याम् । पि॒तृभ्य॒ इति॑ पि॒तृ - भ्यः॒ । घ्नन्ति॑ । सा । अ॒पा॒नेनेत्य॑प - अ॒नेन॑ । पि॒तॄन् । यः । ए॒वम् । वेद॑ । प॒शु॒मानिति॑ पशु-मान् । भ॒व॒ति॒ । अथ॑ । वै । ताम् । उपेति॑ । अ॒ह्वे॒ । इति॑ । ह॒ । उ॒वा॒च॒ । या । प्र॒जा इति॑ प्र - जाः । प्र॒भव॑न्ती॒रिति॑ प्र - भव॑न्तीः । प्रतीति॑ । आ॒भव॒तीत्या᳚ - भव॑ति । इति॑ । अन्न᳚म् । वै । अ॒स्यै॒ । तत् ।  \newline


\textbf{Krama Paata} \newline

अ॒स्यै॒ शरी॑रम् । शरी॑र॒म् गाम् । गां ॅवाव । वाव तौ । तौ तत् । तत्,परि॑ । पर्य॑वदताम् । अ॒व॒द॒तां॒ ॅया । या य॒ज्ञे । य॒ज्ञे दी॒यते᳚ । दी॒यते॒ सा । सा प्रा॒णेन॑ । प्रा॒णेन॑ दे॒वान् । प्रा॒णेनेति॑ प्र - अ॒नेन॑ । दे॒वान्,दा॑धार । दा॒धा॒र॒ यया᳚ । यया॑ मनु॒ष्याः᳚ । म॒नु॒ष्या॑ जीव॑न्ति । जीव॑न्ति॒ सा । सा व्या॒नेन॑ । व्या॒नेन॑ मनु॒ष्यान्॑ । व्या॒नेनेति॑ वि - अ॒नेन॑ । म॒नु॒ष्यान्॑. याम् । याम् पि॒तृभ्यः॑ । पि॒तृभ्यो॒ घ्नन्ति॑ । पि॒तृभ्य॒ इति॑ पि॒तृ - भ्यः॒ । घ्नन्ति॒ सा । साऽपा॒नेन॑ । अ॒पा॒नेन॑ पि॒तॄन् । अ॒पा॒नेनेत्य॑प - अ॒नेन॑ । पि॒तॄन्. यः । य ए॒वम् । ए॒वं ॅवेद॑ । वेद॑ पशु॒मान् । प॒शु॒मान् भ॑वति । प॒शु॒मानिति॑ पशु - मान् । भ॒व॒त्यथ॑ । अथ॒ वै । वै ताम् । तामुप॑ । उपा᳚ह्वे । अ॒ह्व॒ इति॑ । इति॑ ह । हो॒वा॒च॒ । उ॒वा॒च॒ या । या प्र॒जाः । प्र॒जाः प्र॒भव॑न्तीः । प्र॒जा इति॑ प्र - जाः । प्र॒भव॑न्तीः॒ प्रति॑ । प्र॒भव॑न्ती॒रिति॑ प्र - भव॑न्तीः । प्रत्या॒भव॑ति । आ॒भव॒तीति॑ । आ॒भव॒तीत्या᳚ - भव॑ति । इत्यन्न᳚म् । अन्नं॒ ॅवै । वा अ॑स्यै । अ॒स्यै॒ तत् । तदुप॑ \newline

\textbf{Jatai Paata} \newline

1. अ॒स्यै॒ शरी॑रꣳ॒॒ शरी॑र मस्या अस्यै॒ शरी॑रम् । \newline
2. शरी॑र॒म् गाम् गाꣳ शरी॑रꣳ॒॒ शरी॑र॒म् गाम् । \newline
3. गां ॅवाव वाव गाम् गां ॅवाव । \newline
4. वाव तौ तौ वाव वाव तौ । \newline
5. तौ तत् तत् तौ तौ तत् । \newline
6. तत् परि॒ परि॒ तत् तत् परि॑ । \newline
7. पर्य॑वदता मवदता॒म् परि॒ पर्य॑वदताम् । \newline
8. अ॒व॒द॒तां॒ ॅया या ऽव॑दता मवदतां॒ ॅया । \newline
9. या य॒ज्ञे य॒ज्ञे या या य॒ज्ञे । \newline
10. य॒ज्ञे दी॒यते॑ दी॒यते॑ य॒ज्ञे य॒ज्ञे दी॒यते᳚ । \newline
11. दी॒यते॒ सा सा दी॒यते॑ दी॒यते॒ सा । \newline
12. सा प्रा॒णेन॑ प्रा॒णेन॒ सा सा प्रा॒णेन॑ । \newline
13. प्रा॒णेन॑ दे॒वान् दे॒वान् प्रा॒णेन॑ प्रा॒णेन॑ दे॒वान् । \newline
14. प्रा॒णेनेति॑ प्र - अ॒नेन॑ । \newline
15. दे॒वान् दा॑धार दाधार दे॒वान् दे॒वान् दा॑धार । \newline
16. दा॒धा॒र॒ यया॒ यया॑ दाधार दाधार॒ यया᳚ । \newline
17. यया॑ मनु॒ष्या॑ मनु॒ष्या॑ यया॒ यया॑ मनु॒ष्याः᳚ । \newline
18. म॒नु॒ष्या॑ जीव॑न्ति॒ जीव॑न्ति मनु॒ष्या॑ मनु॒ष्या॑ जीव॑न्ति । \newline
19. जीव॑न्ति॒ सा सा जीव॑न्ति॒ जीव॑न्ति॒ सा । \newline
20. सा व्या॒नेन॑ व्या॒नेन॒ सा सा व्या॒नेन॑ । \newline
21. व्या॒नेन॑ मनु॒ष्या᳚न् मनु॒ष्या᳚न् व्या॒नेन॑ व्या॒नेन॑ मनु॒ष्यान्॑ । \newline
22. व्या॒नेनेति॑ वि - अ॒नेन॑ । \newline
23. म॒नु॒ष्या॒न्॒. यां ॅयाम् म॑नु॒ष्या᳚न् मनु॒ष्या॒न्॒. याम् । \newline
24. याम् पि॒तृभ्यः॑ पि॒तृभ्यो॒ यां ॅयाम् पि॒तृभ्यः॑ । \newline
25. पि॒तृभ्यो॒ घ्नन्ति॒ घ्नन्ति॑ पि॒तृभ्यः॑ पि॒तृभ्यो॒ घ्नन्ति॑ । \newline
26. पि॒तृभ्य॒ इति॑ पि॒तृ - भ्यः॒ । \newline
27. घ्नन्ति॒ सा सा घ्नन्ति॒ घ्नन्ति॒ सा । \newline
28. सा ऽपा॒नेना॑ पा॒नेन॒ सा सा ऽपा॒नेन॑ । \newline
29. अ॒पा॒नेन॑ पि॒तॄन् पि॒तॄ न॑पा॒नेना॑ पा॒नेन॑ पि॒तॄन् । \newline
30. अ॒पा॒नेनेत्य॑प - अ॒नेन॑ । \newline
31. पि॒तॄन्. यो यः पि॒तॄन् पि॒तॄन्. यः । \newline
32. य ए॒व मे॒वं ॅयो य ए॒वम् । \newline
33. ए॒वं ॅवेद॒ वेदै॒व मे॒वं ॅवेद॑ । \newline
34. वेद॑ पशु॒मान् प॑शु॒मान्. वेद॒ वेद॑ पशु॒मान् । \newline
35. प॒शु॒मान् भ॑वति भवति पशु॒मान् प॑शु॒मान् भ॑वति । \newline
36. प॒शु॒मानिति॑ पशु - मान् । \newline
37. भ॒व॒ त्यथाथ॑ भवति भव॒ त्यथ॑ । \newline
38. अथ॒ वै वा अथाथ॒ वै । \newline
39. वै ताम् तां ॅवै वै ताम् । \newline
40. ता मुपोप॒ ताम् ता मुप॑ । \newline
41. उपा᳚ह्वे ऽह्व॒ उपोपा᳚ह्वे । \newline
42. अ॒ह्व॒ इती त्य॑ह्वे ऽह्व॒ इति॑ । \newline
43. इति॑ ह॒ हे तीति॑ ह । \newline
44. हो॒वा॒चो॒ वा॒च॒ ह॒ हो॒ वा॒च॒ । \newline
45. उ॒वा॒च॒ या योवा॑चो वाच॒ या । \newline
46. या प्र॒जाः प्र॒जा या या प्र॒जाः । \newline
47. प्र॒जाः प्र॒भव॑न्तीः प्र॒भव॑न्तीः प्र॒जाः प्र॒जाः प्र॒भव॑न्तीः । \newline
48. प्र॒जा इति॑ प्र - जाः । \newline
49. प्र॒भव॑न्तीः॒ प्रति॒ प्रति॑ प्र॒भव॑न्तीः प्र॒भव॑न्तीः॒ प्रति॑ । \newline
50. प्र॒भव॑न्ती॒रिति॑ प्र - भव॑न्तीः । \newline
51. प्रत्या॒भव॑ त्या॒भव॑ति॒ प्रति॒ प्रत्या॒भव॑ति । \newline
52. आ॒भव॒ती ती त्या॒भव॑ त्या॒भव॒ती ति॑ । \newline
53. आ॒भव॒तीत्या᳚ - भव॑ति । \newline
54. इत्यन्न॒ मन्न॒ मिती त्यन्न᳚म् । \newline
55. अन्नं॒ ॅवै वा अन्न॒ मन्नं॒ ॅवै । \newline
56. वा अ॑स्या अस्यै॒ वै वा अ॑स्यै । \newline
57. अ॒स्यै॒ तत् तद॑स्या अस्यै॒ तत् । \newline
58. तदुपोप॒ तत् तदुप॑ । \newline

\textbf{Ghana Paata } \newline

1. अ॒स्यै॒ शरी॑र॒(ग्म्॒) शरी॑र मस्या अस्यै॒ शरी॑र॒म् गाम् गाꣳ शरी॑र मस्या अस्यै॒ शरी॑र॒म् गाम् । \newline
2. शरी॑र॒म् गाम् गाꣳ शरी॑र॒(ग्म्॒) शरी॑र॒म् गां ॅवाव वाव गाꣳ शरी॑र॒(ग्म्॒) शरी॑र॒म् गां ॅवाव । \newline
3. गां ॅवाव वाव गाम् गां ॅवाव तौ तौ वाव गाम् गां ॅवाव तौ । \newline
4. वाव तौ तौ वाव वाव तौ तत् तत् तौ वाव वाव तौ तत् । \newline
5. तौ तत् तत् तौ तौ तत् परि॒ परि॒ तत् तौ तौ तत् परि॑ । \newline
6. तत् परि॒ परि॒ तत् तत् पर्य॑वदता मवदता॒म् परि॒ तत् तत् पर्य॑वदताम् । \newline
7. पर्य॑वदता मवदता॒म् परि॒ पर्य॑वदतां॒ ॅया या ऽव॑दता॒म् परि॒ पर्य॑वदतां॒ ॅया । \newline
8. अ॒व॒द॒तां॒ ॅया या ऽव॑दता मवदतां॒ ॅया य॒ज्ञे य॒ज्ञे या ऽव॑दता मवदतां॒ ॅया य॒ज्ञे । \newline
9. या य॒ज्ञे य॒ज्ञे या या य॒ज्ञे दी॒यते॑ दी॒यते॑ य॒ज्ञे या या य॒ज्ञे दी॒यते᳚ । \newline
10. य॒ज्ञे दी॒यते॑ दी॒यते॑ य॒ज्ञे य॒ज्ञे दी॒यते॒ सा सा दी॒यते॑ य॒ज्ञे य॒ज्ञे दी॒यते॒ सा । \newline
11. दी॒यते॒ सा सा दी॒यते॑ दी॒यते॒ सा प्रा॒णेन॑ प्रा॒णेन॒ सा दी॒यते॑ दी॒यते॒ सा प्रा॒णेन॑ । \newline
12. सा प्रा॒णेन॑ प्रा॒णेन॒ सा सा प्रा॒णेन॑ दे॒वान् दे॒वान् प्रा॒णेन॒ सा सा प्रा॒णेन॑ दे॒वान् । \newline
13. प्रा॒णेन॑ दे॒वान् दे॒वान् प्रा॒णेन॑ प्रा॒णेन॑ दे॒वान् दा॑धार दाधार दे॒वान् प्रा॒णेन॑ प्रा॒णेन॑ दे॒वान् दा॑धार । \newline
14. प्रा॒णेनेति॑ प्र - अ॒नेन॑ । \newline
15. दे॒वान् दा॑धार दाधार दे॒वान् दे॒वान् दा॑धार॒ यया॒ यया॑ दाधार दे॒वान् दे॒वान् दा॑धार॒ यया᳚ । \newline
16. दा॒धा॒र॒ यया॒ यया॑ दाधार दाधार॒ यया॑ मनु॒ष्या॑ मनु॒ष्या॑ यया॑ दाधार दाधार॒ यया॑ मनु॒ष्याः᳚ । \newline
17. यया॑ मनु॒ष्या॑ मनु॒ष्या॑ यया॒ यया॑ मनु॒ष्या॑ जीव॑न्ति॒ जीव॑न्ति मनु॒ष्या॑ यया॒ यया॑ मनु॒ष्या॑ जीव॑न्ति । \newline
18. म॒नु॒ष्या॑ जीव॑न्ति॒ जीव॑न्ति मनु॒ष्या॑ मनु॒ष्या॑ जीव॑न्ति॒ सा सा जीव॑न्ति मनु॒ष्या॑ मनु॒ष्या॑ जीव॑न्ति॒ सा । \newline
19. जीव॑न्ति॒ सा सा जीव॑न्ति॒ जीव॑न्ति॒ सा व्या॒नेन॑ व्या॒नेन॒ सा जीव॑न्ति॒ जीव॑न्ति॒ सा व्या॒नेन॑ । \newline
20. सा व्या॒नेन॑ व्या॒नेन॒ सा सा व्या॒नेन॑ मनु॒ष्या᳚न् मनु॒ष्या᳚न् व्या॒नेन॒ सा सा व्या॒नेन॑ मनु॒ष्यान्॑ । \newline
21. व्या॒नेन॑ मनु॒ष्या᳚न् मनु॒ष्या᳚न् व्या॒नेन॑ व्या॒नेन॑ मनु॒ष्या॒न्॒. यां ॅयाम् म॑नु॒ष्या᳚न् व्या॒नेन॑ व्या॒नेन॑ मनु॒ष्या॒न्॒. याम् । \newline
22. व्या॒नेनेति॑ वि - अ॒नेन॑ । \newline
23. म॒नु॒ष्या॒न्॒. यां ॅयाम् म॑नु॒ष्या᳚न् मनु॒ष्या॒न्॒. याम् पि॒तृभ्यः॑ पि॒तृभ्यो॒ याम् म॑नु॒ष्या᳚न् मनु॒ष्या॒न्॒. याम् पि॒तृभ्यः॑ । \newline
24. याम् पि॒तृभ्यः॑ पि॒तृभ्यो॒ यां ॅयाम् पि॒तृभ्यो॒ घ्नन्ति॒ घ्नन्ति॑ पि॒तृभ्यो॒ यां ॅयाम् पि॒तृभ्यो॒ घ्नन्ति॑ । \newline
25. पि॒तृभ्यो॒ घ्नन्ति॒ घ्नन्ति॑ पि॒तृभ्यः॑ पि॒तृभ्यो॒ घ्नन्ति॒ सा सा घ्नन्ति॑ पि॒तृभ्यः॑ पि॒तृभ्यो॒ घ्नन्ति॒ सा । \newline
26. पि॒तृभ्य॒ इति॑ पि॒तृ - भ्यः॒ । \newline
27. घ्नन्ति॒ सा सा घ्नन्ति॒ घ्नन्ति॒ सा ऽपा॒नेना॑ पा॒नेन॒ सा घ्नन्ति॒ घ्नन्ति॒ सा ऽपा॒नेन॑ । \newline
28. सा ऽपा॒नेना॑ पा॒नेन॒ सा सा ऽपा॒नेन॑ पि॒तॄन् पि॒तॄ न॑पा॒नेन॒ सा सा ऽपा॒नेन॑ पि॒तॄन् । \newline
29. अ॒पा॒नेन॑ पि॒तॄन् पि॒तॄ न॑पा॒नेना॑ पा॒नेन॑ पि॒तॄन्. यो यः पि॒तॄ न॑पा॒नेना॑ पा॒नेन॑ पि॒तॄन्. यः । \newline
30. अ॒पा॒नेनेत्य॑प - अ॒नेन॑ । \newline
31. पि॒तॄन्. यो यः पि॒तॄन् पि॒तॄन्. य ए॒व मे॒वं ॅयः पि॒तॄन् पि॒तॄन्. य ए॒वम् । \newline
32. य ए॒व मे॒वं ॅयो य ए॒वं ॅवेद॒ वेदै॒वं ॅयो य ए॒वं ॅवेद॑ । \newline
33. ए॒वं ॅवेद॒ वेदै॒व मे॒वं ॅवेद॑ पशु॒मान् प॑शु॒मान्. वेदै॒व मे॒वं ॅवेद॑ पशु॒मान् । \newline
34. वेद॑ पशु॒मान् प॑शु॒मान्. वेद॒ वेद॑ पशु॒मान् भ॑वति भवति पशु॒मान्. वेद॒ वेद॑ पशु॒मान् भ॑वति । \newline
35. प॒शु॒मान् भ॑वति भवति पशु॒मान् प॑शु॒मान् भ॑व॒त्यथाथ॑ भवति पशु॒मान् प॑शु॒मान् भ॑व॒त्यथ॑ । \newline
36. प॒शु॒मानिति॑ पशु - मान् । \newline
37. भ॒व॒त्यथाथ॑ भवति भव॒त्यथ॒ वै वा अथ॑ भवति भव॒त्यथ॒ वै । \newline
38. अथ॒ वै वा अथाथ॒ वै ताम् तां ॅवा अथाथ॒ वै ताम् । \newline
39. वै ताम् तां ॅवै वै ता मुपोप॒ तां ॅवै वै ता मुप॑ । \newline
40. ता मुपोप॒ ताम् ता मुपा᳚ह्वे ऽह्व॒ उप॒ ताम् ता मुपा᳚ह्वे । \newline
41. उपा᳚ह्वे ऽह्व॒ उपोपा᳚ह्व॒ इतीत्य॑ह्व॒ उपोपा᳚ह्व॒ इति॑ । \newline
42. अ॒ह्व॒ इतीत्य॑ह्वे ऽह्व॒ इति॑ ह॒ हे त्य॑ह्वे ऽह्व॒ इति॑ ह । \newline
43. इति॑ ह॒ हे तीति॑ होवाचो वाच॒ हे तीति॑ होवाच । \newline
44. हो॒वा॒चो॒ वा॒च॒ ह॒ हो॒वा॒च॒ या योवा॑च ह होवाच॒ या । \newline
45. उ॒वा॒च॒ या योवा॑चो वाच॒ या प्र॒जाः प्र॒जा योवा॑चो वाच॒ या प्र॒जाः । \newline
46. या प्र॒जाः प्र॒जा या या प्र॒जाः प्र॒भव॑न्तीः प्र॒भव॑न्तीः प्र॒जा या या प्र॒जाः प्र॒भव॑न्तीः । \newline
47. प्र॒जाः प्र॒भव॑न्तीः प्र॒भव॑न्तीः प्र॒जाः प्र॒जाः प्र॒भव॑न्तीः॒ प्रति॒ प्रति॑ प्र॒भव॑न्तीः प्र॒जाः प्र॒जाः प्र॒भव॑न्तीः॒ प्रति॑ । \newline
48. प्र॒जा इति॑ प्र - जाः । \newline
49. प्र॒भव॑न्तीः॒ प्रति॒ प्रति॑ प्र॒भव॑न्तीः प्र॒भव॑न्तीः॒ प्रत्या॒भव॑ त्या॒भव॑ति॒ प्रति॑ प्र॒भव॑न्तीः प्र॒भव॑न्तीः॒ प्रत्या॒भव॑ति । \newline
50. प्र॒भव॑न्ती॒रिति॑ प्र - भव॑न्तीः । \newline
51. प्रत्या॒भव॑ त्या॒भव॑ति॒ प्रति॒ प्रत्या॒भव॒ती तीत्या॒भव॑ति॒ प्रति॒ प्रत्या॒भव॒तीति॑ । \newline
52. आ॒भव॒तीती त्या॒भव॑ त्या॒भव॒तीत्यन्न॒ मन्न॒ मित्या॒भव॑ त्या॒भव॒तीत्यन्न᳚म् । \newline
53. आ॒भव॒तीत्या᳚ - भव॑ति । \newline
54. इत्यन्न॒ मन्न॒ मितीत्यन्नं॒ ॅवै वा अन्न॒ मितीत्यन्नं॒ ॅवै । \newline
55. अन्नं॒ ॅवै वा अन्न॒ मन्नं॒ ॅवा अ॑स्या अस्यै॒ वा अन्न॒ मन्नं॒ ॅवा अ॑स्यै । \newline
56. वा अ॑स्या अस्यै॒ वै वा अ॑स्यै॒ तत् तद॑स्यै॒ वै वा अ॑स्यै॒ तत् । \newline
57. अ॒स्यै॒ तत् तद॑स्या अस्यै॒ तदुपोप॒ तद॑स्या अस्यै॒ तदुप॑ । \newline
58. तदुपोप॒ तत् तदुपा᳚ह्वथा अह्वथा॒ उप॒ तत् तदुपा᳚ह्वथाः । \newline
\pagebreak
\markright{ TS 1.7.2.3  \hfill https://www.vedavms.in \hfill}
\addcontentsline{toc}{section}{ TS 1.7.2.3 }
\section*{ TS 1.7.2.3 }

\textbf{TS 1.7.2.3 } \newline
\textbf{Samhita Paata} \newline

पा᳚ह्वथा॒ इति॑ होवा॒चौष॑धयो॒ वा अ॑स्या॒ अन्न॒मोष॑धयो॒ वै प्र॒जाः प्र॒भव॑न्तीः॒ प्रत्या भ॑वन्ति॒ य ए॒वं ॅवेदा᳚न्ना॒दो भ॑व॒त्यथ॒ वै तामुपा᳚ह्व॒ इति॑ होवाच॒ या प्र॒जाः प॑रा॒भव॑न्ती-रनुगृ॒ह्णाति॒ प्रत्या॒भव॑न्तीर् गृ॒ह्णातीति॑ प्रति॒ष्ठां ॅवा अ॑स्यै॒ तदुपा᳚ह्वथा॒ इति॑ होवाचे॒यं ॅवा अ॑स्यै प्रति॒ष्ठे - [ ] \newline

\textbf{Pada Paata} \newline

उपेति॑ । अ॒ह्व॒थाः॒ । इति॑ । ह॒ । उ॒वा॒च॒ । ओष॑धयः । वै । अ॒स्याः॒ । अन्न᳚म् । ओष॑धयः । वै । प्र॒जा इति॑ प्र - जाः । प्र॒भव॑न्ती॒रिति॑ प्र - भव॑न्तीः । प्रति॑ । एति॑ । भ॒व॒न्ति॒ । यः । ए॒वम् । वेद॑ । अ॒न्ना॒द इत्य॑न्न - अ॒दः । भ॒व॒ति॒ । अथ॑ । वै । ताम् । उपेति॑ । अ॒ह्वे॒ । इति॑ । ह॒ । उ॒वा॒च॒ । या । प्र॒जा इति॑ प्र - जाः । प॒रा॒भव॑न्ती॒रिति॑ परा - भव॑न्तीः । अ॒नु॒गृ॒ह्णातीत्य॑नु - गृ॒ह्णाति॑ । प्रतीति॑ । आ॒भव॑न्ती॒रित्या᳚ - भव॑न्तीः । गृ॒ह्णाति॑ । इति॑ । प्र॒ति॒ष्ठामिति॑ प्रति - स्थाम् । वै । अ॒स्यै॒ । तत् । उपेति॑ । अ॒ह्व॒थाः॒ । इति॑ । ह॒ । उ॒वा॒च॒ । इ॒यम् । वै । अ॒स्यै॒ । प्र॒ति॒ष्ठेति॑ प्रति - स्था ।  \newline


\textbf{Krama Paata} \newline

उपा᳚ह्वथाः । अ॒ह्व॒था॒ इति॑ । इति॑ ह । हो॒वा॒च॒ । उ॒वा॒चौष॑धयः । ओष॑धयो॒ वै । वा अ॑स्याः । अ॒स्या॒ अन्न᳚म् । अन्न॒मोष॑धयः । ओष॑धयो॒ वै । वै प्र॒जाः । प्र॒जाः प्र॒भव॑न्तीः । प्र॒जा इति॑ प्र - जाः । प्र॒भव॑न्तीः॒ प्रति॑ । प्र॒भव॑न्ती॒रिति॑ प्र - भव॑न्तीः । प्रत्या । आ भ॑वन्ति । भ॒व॒न्ति॒ यः । य ए॒वम् । ए॒वं ॅवेद॑ । वेदा᳚न्ना॒दः । अ॒न्ना॒दो भ॑वति । अ॒न्ना॒द इत्य॑न्न - अ॒दः । भ॒व॒त्यथ॑ । अथ॒ वै । वै ताम् । तामुप॑ । उपा᳚ह्वे । अ॒ह्व॒ इति॑ । इति॑ ह । हो॒वा॒च॒ । उ॒वा॒च॒ या । या प्र॒जाः । प्र॒जाः प॑रा॒भव॑न्तीः । प्र॒जा इति॑ प्र - जाः । प॒रा॒भव॑न्तीरनुगृ॒ह्णाति॑ । प॒रा॒भव॑न्ती॒रिति॑ परा - भव॑न्तीः । अ॒नु॒गृ॒ह्णाति॒ प्रति॑ । अ॒नु॒गृ॒ह्णातीत्य॑नु - गृ॒ह्णाति॑ । प्रत्या॒भव॑न्तीः । आ॒भव॑न्तीर् गृ॒ह्णाति॑ । आ॒भव॑न्ती॒रित्या᳚ - भव॑न्तीः । गृ॒ह्णातीति॑ । इति॑ प्रति॒ष्ठाम् । प्र॒ति॒ष्ठां ॅवै । प्र॒ति॒ष्ठामिति॑ प्रति - स्थाम् । वा अ॑स्यै । अ॒स्यै॒ तत् । तदुप॑ । उपा᳚ह्वथाः । 
अ॒ह्व॒था॒ इति॑ ( ) । इति॑ ह । हो॒वा॒च॒ । उ॒वा॒चे॒यम् । इ॒यं ॅवै । 
वा अ॑स्यै । अ॒स्यै॒ प्र॒ति॒ष्ठा । प्र॒ति॒ष्ठेयम् । 
प्र॒ति॒ष्ठेति॑ प्रति - स्था \newline

\textbf{Jatai Paata} \newline

1. उपा᳚ह्वथा अह्वथा॒ उपोपा᳚ह्वथाः । \newline
2. अ॒ह्व॒था॒ इती त्य॑ह्वथा अह्वथा॒ इति॑ । \newline
3. इति॑ ह॒ हे तीति॑ ह । \newline
4. हो॒वा॒चो॒ वा॒च॒ ह॒ हो॒ वा॒च॒ । \newline
5. उ॒वा॒चौष॑धय॒ ओष॑धय उवाचो वा॒चौष॑धयः । \newline
6. ओष॑धयो॒ वै वा ओष॑धय॒ ओष॑धयो॒ वै । \newline
7. वा अ॑स्या अस्या॒ वै वा अ॑स्याः । \newline
8. अ॒स्या॒ अन्न॒ मन्न॑ मस्या अस्या॒ अन्न᳚म् । \newline
9. अन्न॒ मोष॑धय॒ ओष॑ध॒यो ऽन्न॒ मन्न॒ मोष॑धयः । \newline
10. ओष॑धयो॒ वै वा ओष॑धय॒ ओष॑धयो॒ वै । \newline
11. वै प्र॒जाः प्र॒जा वै वै प्र॒जाः । \newline
12. प्र॒जाः प्र॒भव॑न्तीः प्र॒भव॑न्तीः प्र॒जाः प्र॒जाः प्र॒भव॑न्तीः । \newline
13. प्र॒जा इति॑ प्र - जाः । \newline
14. प्र॒भव॑न्तीः॒ प्रति॒ प्रति॑ प्र॒भव॑न्तीः प्र॒भव॑न्तीः॒ प्रति॑ । \newline
15. प्र॒भव॑न्ती॒रिति॑ प्र - भव॑न्तीः । \newline
16. प्रत्या प्रति॒ प्रत्या । \newline
17. आ भ॑वन्ति भव॒न्त्या भ॑वन्ति । \newline
18. भ॒व॒न्ति॒ यो यो भ॑वन्ति भवन्ति॒ यः । \newline
19. य ए॒व मे॒वं ॅयो य ए॒वम् । \newline
20. ए॒वं ॅवेद॒ वेदै॒व मे॒वं ॅवेद॑ । \newline
21. वेदा᳚न्ना॒दो᳚ ऽन्ना॒दो वेद॒ वेदा᳚न्ना॒दः । \newline
22. अ॒न्ना॒दो भ॑वति भव त्यन्ना॒दो᳚ ऽन्ना॒दो भ॑वति । \newline
23. अ॒न्ना॒द इत्य॑न्न - अ॒दः । \newline
24. भ॒व॒ त्यथाथ॑ भवति भव॒ त्यथ॑ । \newline
25. अथ॒ वै वा अथाथ॒ वै । \newline
26. वै ताम् तां ॅवै वै ताम् । \newline
27. ता मुपोप॒ ताम् ता मुप॑ । \newline
28. उपा᳚ह्वे ऽह्व॒ उपोपा᳚ह्वे । \newline
29. अ॒ह्व॒ इती त्य॑ह्वे ऽह्व॒ इति॑ । \newline
30. इति॑ ह॒ हे तीति॑ ह । \newline
31. हो॒वा॒चो॒ वा॒च॒ ह॒ हो॒ वा॒च॒ । \newline
32. उ॒वा॒च॒ या योवा॑चो वाच॒ या । \newline
33. या प्र॒जाः प्र॒जा या या प्र॒जाः । \newline
34. प्र॒जाः प॑रा॒भव॑न्तीः परा॒भव॑न्तीः प्र॒जाः प्र॒जाः प॑रा॒भव॑न्तीः । \newline
35. प्र॒जा इति॑ प्र - जाः । \newline
36. प॒रा॒भव॑न्ती रनुगृ॒ह्णा त्य॑नुगृ॒ह्णाति॑ परा॒भव॑न्तीः परा॒भव॑न्ती रनुगृ॒ह्णाति॑ । \newline
37. प॒रा॒भव॑न्ती॒रिति॑ परा - भव॑न्तीः । \newline
38. अ॒नु॒गृ॒ह्णाति॒ प्रति॒ प्रत्य॑नुगृ॒ह्णा त्य॑नुगृ॒ह्णाति॒ प्रति॑ । \newline
39. अ॒नु॒गृ॒ह्णातीत्य॑नु - गृ॒ह्णाति॑ । \newline
40. प्रत्या॒भव॑न्ती रा॒भव॑न्तीः॒ प्रति॒ प्रत्या॒भव॑न्तीः । \newline
41. आ॒भव॑न्तीर् गृ॒ह्णाति॑ गृ॒ह्णा त्या॒भव॑न्ती रा॒भव॑न्तीर् गृ॒ह्णाति॑ । \newline
42. आ॒भव॑न्ती॒रित्या᳚ - भव॑न्तीः । \newline
43. गृ॒ह्णा तीतीति॑ गृ॒ह्णाति॑ गृ॒ह्णा तीति॑ । \newline
44. इति॑ प्रति॒ष्ठाम् प्र॑ति॒ष्ठा मितीति॑ प्रति॒ष्ठाम् । \newline
45. प्र॒ति॒ष्ठां ॅवै वै प्र॑ति॒ष्ठाम् प्र॑ति॒ष्ठां ॅवै । \newline
46. प्र॒ति॒ष्ठामिति॑ प्रति - स्थाम् । \newline
47. वा अ॑स्या अस्यै॒ वै वा अ॑स्यै । \newline
48. अ॒स्यै॒ तत् तद॑स्या अस्यै॒ तत् । \newline
49. तदुपोप॒ तत् तदुप॑ । \newline
50. उपा᳚ह्वथा अह्वथा॒ उपोपा᳚ह्वथाः । \newline
51. अ॒ह्व॒था॒ इती त्य॑ह्वथा अह्वथा॒ इति॑ । \newline
52. इति॑ ह॒ हे तीति॑ ह । \newline
53. हो॒वा॒चो॒ वा॒च॒ ह॒ हो॒वा॒च॒ । \newline
54. उ॒वा॒चे॒ य मि॒य मु॑वाचो वाचे॒ यम् । \newline
55. इ॒यं ॅवै वा इ॒य मि॒यं ॅवै । \newline
56. वा अ॑स्या अस्यै॒ वै वा अ॑स्यै । \newline
57. अ॒स्यै॒ प्र॒ति॒ष्ठा प्र॑ति॒ष्ठा ऽस्या॑ अस्यै प्रति॒ष्ठा । \newline
58. प्र॒ति॒ष्ठेय मि॒यम् प्र॑ति॒ष्ठा प्र॑ति॒ष्ठेयम् । \newline
59. प्र॒ति॒ष्ठेति॑ प्रति - स्था । \newline

\textbf{Ghana Paata } \newline

1. उपा᳚ह्वथा अह्वथा॒ उपोपा᳚ह्वथा॒ इतीत्य॑ह्वथा॒ उपोपा᳚ह्वथा॒ इति॑ । \newline
2. अ॒ह्व॒था॒ इतीत्य॑ह्वथा अह्वथा॒ इति॑ ह॒ हे त्य॑ह्वथा अह्वथा॒ इति॑ ह । \newline
3. इति॑ ह॒ हे तीति॑ होवाचो वाच॒ हे तीति॑ होवाच । \newline
4. हो॒वा॒चो॒ वा॒च॒ ह॒ हो॒वा॒चौष॑धय॒ ओष॑धय उवाच ह होवा॒चौष॑धयः । \newline
5. उ॒वा॒चौष॑धय॒ ओष॑धय उवाचो वा॒चौष॑धयो॒ वै वा ओष॑धय उवाचो वा॒चौष॑धयो॒ वै । \newline
6. ओष॑धयो॒ वै वा ओष॑धय॒ ओष॑धयो॒ वा अ॑स्या अस्या॒ वा ओष॑धय॒ ओष॑धयो॒ वा अ॑स्याः । \newline
7. वा अ॑स्या अस्या॒ वै वा अ॑स्या॒ अन्न॒ मन्न॑ मस्या॒ वै वा अ॑स्या॒ अन्न᳚म् । \newline
8. अ॒स्या॒ अन्न॒ मन्न॑ मस्या अस्या॒ अन्न॒ मोष॑धय॒ ओष॑ध॒यो ऽन्न॑ मस्या अस्या॒ अन्न॒ मोष॑धयः । \newline
9. अन्न॒ मोष॑धय॒ ओष॑ध॒यो ऽन्न॒ मन्न॒ मोष॑धयो॒ वै वा ओष॑ध॒यो ऽन्न॒ मन्न॒ मोष॑धयो॒ वै । \newline
10. ओष॑धयो॒ वै वा ओष॑धय॒ ओष॑धयो॒ वै प्र॒जाः प्र॒जा वा ओष॑धय॒ ओष॑धयो॒ वै प्र॒जाः । \newline
11. वै प्र॒जाः प्र॒जा वै वै प्र॒जाः प्र॒भव॑न्तीः प्र॒भव॑न्तीः प्र॒जा वै वै प्र॒जाः प्र॒भव॑न्तीः । \newline
12. प्र॒जाः प्र॒भव॑न्तीः प्र॒भव॑न्तीः प्र॒जाः प्र॒जाः प्र॒भव॑न्तीः॒ प्रति॒ प्रति॑ प्र॒भव॑न्तीः प्र॒जाः प्र॒जाः प्र॒भव॑न्तीः॒ प्रति॑ । \newline
13. प्र॒जा इति॑ प्र - जाः । \newline
14. प्र॒भव॑न्तीः॒ प्रति॒ प्रति॑ प्र॒भव॑न्तीः प्र॒भव॑न्तीः॒ प्रत्या प्रति॑ प्र॒भव॑न्तीः प्र॒भव॑न्तीः॒ प्रत्या । \newline
15. प्र॒भव॑न्ती॒रिति॑ प्र - भव॑न्तीः । \newline
16. प्रत्या प्रति॒ प्रत्या भ॑वन्ति भव॒न्त्या प्रति॒ प्रत्या भ॑वन्ति । \newline
17. आ भ॑वन्ति भव॒न्त्या भ॑वन्ति॒ यो यो भ॑व॒न्त्या भ॑वन्ति॒ यः । \newline
18. भ॒व॒न्ति॒ यो यो भ॑वन्ति भवन्ति॒ य ए॒व मे॒वं ॅयो भ॑वन्ति भवन्ति॒ य ए॒वम् । \newline
19. य ए॒व मे॒वं ॅयो य ए॒वं ॅवेद॒ वेदै॒वं ॅयो य ए॒वं ॅवेद॑ । \newline
20. ए॒वं ॅवेद॒ वेदै॒व मे॒वं ॅवेदा᳚न्ना॒दो᳚ ऽन्ना॒दो वेदै॒व मे॒वं ॅवेदा᳚न्ना॒दः । \newline
21. वेदा᳚न्ना॒दो᳚ ऽन्ना॒दो वेद॒ वेदा᳚न्ना॒दो भ॑वति भवत्यन्ना॒दो वेद॒ वेदा᳚न्ना॒दो भ॑वति । \newline
22. अ॒न्ना॒दो भ॑वति भवत्यन्ना॒दो᳚ ऽन्ना॒दो भ॑व॒त्यथाथ॑ भवत्यन्ना॒दो᳚ ऽन्ना॒दो भ॑व॒त्यथ॑ । \newline
23. अ॒न्ना॒द इत्य॑न्न - अ॒दः । \newline
24. भ॒व॒त्यथाथ॑ भवति भव॒त्यथ॒ वै वा अथ॑ भवति भव॒त्यथ॒ वै । \newline
25. अथ॒ वै वा अथाथ॒ वै ताम् तां ॅवा अथाथ॒ वै ताम् । \newline
26. वै ताम् तां ॅवै वै ता मुपोप॒ तां ॅवै वै ता मुप॑ । \newline
27. ता मुपोप॒ ताम् ता मुपा᳚ह्वे ऽह्व॒ उप॒ ताम् ता मुपा᳚ह्वे । \newline
28. उपा᳚ह्वे ऽह्व॒ उपोपा᳚ह्व॒ इतीत्य॑ह्व॒ उपोपा᳚ह्व॒ इति॑ । \newline
29. अ॒ह्व॒ इतीत्य॑ह्वे ऽह्व॒ इति॑ ह॒ हे त्य॑ह्वे ऽह्व॒ इति॑ ह । \newline
30. इति॑ ह॒ हे तीति॑ होवाचो वाच॒ हे तीति॑ होवाच । \newline
31. हो॒वा॒चो॒ आ॒च॒ ह॒ हो॒वा॒च॒ या योवा॑च ह होवाच॒ या । \newline
32. उ॒वा॒च॒ या योवा॑चो वाच॒ या प्र॒जाः प्र॒जा योवा॑चो वाच॒ या प्र॒जाः । \newline
33. या प्र॒जाः प्र॒जा या या प्र॒जाः प॑रा॒भव॑न्तीः परा॒भव॑न्तीः प्र॒जा या या प्र॒जाः प॑रा॒भव॑न्तीः । \newline
34. प्र॒जाः प॑रा॒भव॑न्तीः परा॒भव॑न्तीः प्र॒जाः प्र॒जाः प॑रा॒भव॑न्ती रनुगृ॒ह्णा त्य॑नुगृ॒ह्णाति॑ परा॒भव॑न्तीः प्र॒जाः प्र॒जाः प॑रा॒भव॑न्ती रनुगृ॒ह्णाति॑ । \newline
35. प्र॒जा इति॑ प्र - जाः । \newline
36. प॒रा॒भव॑न्ती रनुगृ॒ह्णा त्य॑नुगृ॒ह्णाति॑ परा॒भव॑न्तीः परा॒भव॑न्ती रनुगृ॒ह्णाति॒ प्रति॒ प्रत्य॑नुगृ॒ह्णाति॑ परा॒भव॑न्तीः परा॒भव॑न्ती रनुगृ॒ह्णाति॒ प्रति॑ । \newline
37. प॒रा॒भव॑न्ती॒रिति॑ परा - भव॑न्तीः । \newline
38. अ॒नु॒गृ॒ह्णाति॒ प्रति॒ प्रत्य॑नुगृ॒ह्णा त्य॑नुगृ॒ह्णाति॒ प्रत्या॒भव॑न्ती रा॒भव॑न्तीः॒ प्रत्य॑नुगृ॒ह्णा त्य॑नुगृ॒ह्णाति॒ प्रत्या॒भव॑न्तीः । \newline
39. अ॒नु॒गृ॒ह्णातीत्य॑नु - गृ॒ह्णाति॑ । \newline
40. प्रत्या॒भव॑न्ती रा॒भव॑न्तीः॒ प्रति॒ प्रत्या॒भव॑न्तीर् गृ॒ह्णाति॑ गृ॒ह्णात्या॒भव॑न्तीः॒ प्रति॒ प्रत्या॒भव॑न्तीर् गृ॒ह्णाति॑ । \newline
41. आ॒भव॑न्तीर् गृ॒ह्णाति॑ गृ॒ह्णात्या॒भव॑न्ती रा॒भव॑न्तीर् गृ॒ह्णातीतीति॑ गृ॒ह्णात्या॒भव॑न्ती रा॒भव॑न्तीर् गृ॒ह्णातीति॑ । \newline
42. आ॒भव॑न्ती॒रित्या᳚ - भव॑न्तीः । \newline
43. गृ॒ह्णातीतीति॑ गृ॒ह्णाति॑ गृ॒ह्णातीति॑ प्रति॒ष्ठाम् प्र॑ति॒ष्ठा मिति॑ गृ॒ह्णाति॑ गृ॒ह्णातीति॑ प्रति॒ष्ठाम् । \newline
44. इति॑ प्रति॒ष्ठाम् प्र॑ति॒ष्ठा मितीति॑ प्रति॒ष्ठां ॅवै वै प्र॑ति॒ष्ठा मितीति॑ प्रति॒ष्ठां ॅवै । \newline
45. प्र॒ति॒ष्ठां ॅवै वै प्र॑ति॒ष्ठाम् प्र॑ति॒ष्ठां ॅवा अ॑स्या अस्यै॒ वै प्र॑ति॒ष्ठाम् प्र॑ति॒ष्ठां ॅवा अ॑स्यै । \newline
46. प्र॒ति॒ष्ठामिति॑ प्रति - स्थाम् । \newline
47. वा अ॑स्या अस्यै॒ वै वा अ॑स्यै॒ तत् तद॑स्यै॒ वै वा अ॑स्यै॒ तत् । \newline
48. अ॒स्यै॒ तत् तद॑स्या अस्यै॒ तदुपोप॒ तद॑स्या अस्यै॒ तदुप॑ । \newline
49. तदुपोप॒ तत् तदुपा᳚ह्वथा अह्वथा॒ उप॒ तत् तदुपा᳚ह्वथाः । \newline
50. उपा᳚ह्वथा अह्वथा॒ उपोपा᳚ह्वथा॒ इतीत्य॑ह्वथा॒ उपोपा᳚ह्वथा॒ इति॑ । \newline
51. अ॒ह्व॒था॒ इतीत्य॑ह्वथा अह्वथा॒ इति॑ ह॒ हे त्य॑ह्वथा अह्वथा॒ इति॑ ह । \newline
52. इति॑ ह॒ हे तीति॑ होवाचो वाच॒ हे तीति॑ होवाच । \newline
53. हो॒वा॒चो॒ वा॒च॒ ह॒ हो॒वा॒चे॒ य मि॒य मु॑वाच ह होवाचे॒ यम् । \newline
54. उ॒वा॒चे॒ य मि॒य मु॑वाचो वाचे॒ यं ॅवै वा इ॒य मु॑वाचो वाचे॒ यं ॅवै । \newline
55. इ॒यं ॅवै वा इ॒य मि॒यं ॅवा अ॑स्या अस्यै॒ वा इ॒य मि॒यं ॅवा अ॑स्यै । \newline
56. वा अ॑स्या अस्यै॒ वै वा अ॑स्यै प्रति॒ष्ठा प्र॑ति॒ष्ठा ऽस्यै॒ वै वा अ॑स्यै प्रति॒ष्ठा । \newline
57. अ॒स्यै॒ प्र॒ति॒ष्ठा प्र॑ति॒ष्ठा ऽस्या॑ अस्यै प्रति॒ष्ठेय मि॒यम् प्र॑ति॒ष्ठा ऽस्या॑ अस्यै प्रति॒ष्ठेयम् । \newline
58. प्र॒ति॒ष्ठेय मि॒यम् प्र॑ति॒ष्ठा प्र॑ति॒ष्ठेयं ॅवै वा इ॒यम् प्र॑ति॒ष्ठा प्र॑ति॒ष्ठेयं ॅवै । \newline
59. प्र॒ति॒ष्ठेति॑ प्रति - स्था । \newline
\pagebreak
\markright{ TS 1.7.2.4  \hfill https://www.vedavms.in \hfill}
\addcontentsline{toc}{section}{ TS 1.7.2.4 }
\section*{ TS 1.7.2.4 }

\textbf{TS 1.7.2.4 } \newline
\textbf{Samhita Paata} \newline

यं ॅवै प्र॒जाः प॑रा॒भव॑न्ती॒रनु॑ गृह्णाति॒ प्रत्या॒भव॑न्तीर् गृह्णाति॒ य ए॒वं ॅवेद॒ प्रत्ये॒व ति॑ष्ठ॒त्यथ॒ वै तामुपा᳚ह्व॒ इति॑ होवाच॒ यस्यै॑ नि॒क्रम॑णे घृ॒तं प्र॒जाः स॒ञ्ज॑007आ;व॑न्तीः॒ पिब॒न्तीति॑ छि॒नत्ति॒ सा न छि॑न॒त्ती (3) इति॒ न छि॑न॒त्तीति॑ होवाच॒ प्र तु ज॑नय॒तीत्ये॒ष वा इडा॒मुपा᳚ह्वथा॒ इति॑ ( ) होवाच॒ वृष्टि॒र्॒.वा इडा॒ वृष्ट्यै॒ वै नि॒क्रम॑णे घृ॒तं प्र॒जाः स॒ञ्ज॑007आ;व॑न्तीः पिबन्ति॒ य ए॒वं ॅवेद॒ प्रैव जा॑यतेऽन्ना॒दो भ॑वति ॥ \newline

\textbf{Pada Paata} \newline

इ॒यम् । वै । प्र॒जा इति॑ प्र - जाः । प॒रा॒भव॑न्ती॒रिति॑ परा - भव॑न्तीः । अन्विति॑॑ । गृ॒ह्णा॒ति॒ । प्रतीति॑ । आ॒भव॑न्ती॒रित्या᳚ - भव॑न्तीः । गृ॒ह्णा॒ति॒ । यः । ए॒वम् । वेद॑ । प्रतीति॑ । ए॒व । ति॒ष्ठ॒ति॒ । अथ॑ । वै । ताम् । उपेति॑ । अ॒ह्वे॒ । इति॑ । ह॒ । उ॒वा॒च॒ । यस्यै᳚ । नि॒क्रम॑ण॒ इति॑ नि - क्रम॑णे । घृ॒तम् । प्र॒जा इति॑ प्र - जाः । स॒जींव॑न्ती॒रिति॑ सं - जीव॑न्तीः । पिब॑न्ति । इति॑ । छि॒नत्ति॑ । सा । न । छि॒न॒त्ती(3) । इति॑ । न । छि॒न॒त्ति॒ । इति॑ । ह॒ । उ॒वा॒च॒ । प्रेति॑ । तु । ज॒न॒य॒ति॒ । इति॑ । ए॒षः । वै । इडा᳚म् । उपेति॑ । अ॒ह्व॒थाः॒ । इति॑ ( ) । ह॒ । उ॒वा॒च॒ । वृष्टिः॑ । वै । इडा᳚ । वृष्‌ट्यै᳚ । वै । नि॒क्रम॑ण॒ इति॑ नि - क्रम॑णे । घृ॒तम् । प्र॒जा इति॑ प्र - जाः । स॒जींव॑न्ती॒रिति॑ सं - जीव॑न्तीः । पि॒ब॒न्ति॒ । यः । ए॒वम् । वेद॑ । प्रेति॑ । ए॒व । जा॒य॒ते॒ । अ॒न्ना॒द इत्य॑न्न - अ॒दः । भ॒व॒ति॒ ॥  \newline


\textbf{Krama Paata} \newline

इ॒यं ॅवै । वै प्र॒जाः । प्र॒जाः प॑रा॒भव॑न्तीः । प्र॒जा इति॑ प्र - जाः । प॒रा॒भव॑न्ती॒रनु॑ । प॒रा॒भव॑न्ती॒रिति॑ परा - भव॑न्तीः । अनु॑ गृह्णाति । गृ॒ह्णा॒ति॒ प्रति॑ । प्रत्या॒भव॑न्तीः । आ॒भव॑न्तीर्,गृह्णाति । आ॒भव॑न्ती॒रित्या᳚ - भव॑न्तीः । गृ॒ह्णा॒ति॒ यः । य ए॒वम् । ए॒वं ॅवेद॑ । वेद॒ प्रति॑ । प्रत्ये॒व । ए॒व ति॑ष्ठति । ति॒ष्ठ॒त्यथ॑ । अथ॒ वै । वै ताम् । तामुप॑ । उपा᳚ह्वे । अ॒ह्व॒ इति॑ । इति॑ ह । हो॒वा॒च॒ । उ॒वा॒च॒ यस्यै᳚ । यस्यै॑ नि॒क्रम॑णे । नि॒क्रम॑णे घृ॒तम् । नि॒क्रम॑ण॒ इति॑ नि - क्रम॑णे । घृ॒तम् प्र॒जाः । प्र॒जाः स॒ञ्जीव॑न्तीः । प्र॒जा इति॑ प्र - जाः । स॒ञ्जीव॑न्तीः॒ पिब॑न्ति । स॒ञ्जीव॑न्ती॒रिति॑ सम् - जीव॑न्तीः । पिब॒न्तीति॑ । इति॑ छि॒नत्ति॑ । छि॒नत्ति॒ सा । सा न । न छि॑न॒त्ती(3) । छि॒न॒त्ती(3) इति॑ । इति॒ न । न छि॑नत्ति । छि॒न॒त्तीति॑ । इति॑ ह । हो॒वा॒च॒ । उ॒वा॒च॒ प्र । प्र तु । तु ज॑नयति । ज॒न॒य॒तीति॑ । इत्ये॒षः । ए॒ष वै । वा इडा᳚म् । इडा॒मुप॑ । उपा᳚ह्वथाः । अ॒ह्व॒था॒ इति॑ ( ) । इति॑ ह । हो॒वा॒च॒ । उ॒वा॒च॒ वृष्टिः॑ । वृष्टि॒र् वै । वा इडा᳚ । इडा॒ वृष्ट्यै᳚ । वृष्ट्यै॒ वै । वै नि॒क्रम॑णे । नि॒क्रम॑णे घृ॒तम् । नि॒क्रम॑ण॒ इति॑ नि - क्रम॑णे । घृ॒तम् प्र॒जाः । प्र॒जाः स॒ञ्जीव॑न्तीः । प्र॒जा इति॑ प्र - जाः । स॒ञ्जीव॑न्तीः पिबन्ति । स॒ञ्जीव॑न्ती॒रिति॑ सम् - जीव॑न्तीः । पि॒ब॒न्ति॒ यः । य ए॒वम् । ए॒वं ॅवेद॑ । वेद॒ प्र । प्रैव । ए॒व जा॑यते । जा॒य॒ते॒ ऽन्ना॒दः । अ॒न्ना॒दो भ॑वति । अ॒न्ना॒द इत्य॑न्न - अ॒दः । भ॒व॒तीति॑ भवति । \newline

\textbf{Jatai Paata} \newline

1. इ॒यं ॅवै वा इ॒य मि॒यं ॅवै । \newline
2. वै प्र॒जाः प्र॒जा वै वै प्र॒जाः । \newline
3. प्र॒जाः प॑रा॒भव॑न्तीः परा॒भव॑न्तीः प्र॒जाः प्र॒जाः प॑रा॒भव॑न्तीः । \newline
4. प्र॒जा इति॑ प्र - जाः । \newline
5. प॒रा॒भव॑न्ती॒ रन्वनु॑ परा॒भव॑न्तीः परा॒भव॑न्ती॒ रनु॑ । \newline
6. प॒रा॒भव॑न्ती॒रिति॑ परा - भव॑न्तीः । \newline
7. अनु॑ गृह्णाति गृह्णा॒ त्यन्वनु॑ गृह्णाति । \newline
8. गृ॒ह्णा॒ति॒ प्रति॒ प्रति॑ गृह्णाति गृह्णाति॒ प्रति॑ । \newline
9. प्रत्या॒भव॑न्ती रा॒भव॑न्तीः॒ प्रति॒ प्रत्या॒भव॑न्तीः । \newline
10. आ॒भव॑न्तीर् गृह्णाति गृह्णा त्या॒भव॑न्ती रा॒भव॑न्तीर् गृह्णाति । \newline
11. आ॒भव॑न्ती॒रित्या᳚ - भव॑न्तीः । \newline
12. गृ॒ह्णा॒ति॒ यो यो गृ॑ह्णाति गृह्णाति॒ यः । \newline
13. य ए॒व मे॒वं ॅयो य ए॒वम् । \newline
14. ए॒वं ॅवेद॒ वेदै॒व मे॒वं ॅवेद॑ । \newline
15. वेद॒ प्रति॒ प्रति॒ वेद॒ वेद॒ प्रति॑ । \newline
16. प्रत्ये॒ वैव प्रति॒ प्रत्ये॒व । \newline
17. ए॒व ति॑ष्ठति तिष्ठ त्ये॒वैव ति॑ष्ठति । \newline
18. ति॒ष्ठ॒ त्यथाथ॑ तिष्ठति तिष्ठ॒ त्यथ॑ । \newline
19. अथ॒ वै वा अथाथ॒ वै । \newline
20. वै ताम् तां ॅवै वै ताम् । \newline
21. ता मुपोप॒ ताम् ता मुप॑ । \newline
22. उपा᳚ह्वे ऽह्व॒ उपोपा᳚ह्वे । \newline
23. अ॒ह्व॒ इती त्य॑ह्वे ऽह्व॒ इति॑ । \newline
24. इति॑ ह॒ हे तीति॑ ह । \newline
25. हो॒वा॒चो॒ वा॒च॒ ह॒ हो॒वा॒च॒ । \newline
26. उ॒वा॒च॒ यस्यै॒ यस्या॑ उवाचो वाच॒ यस्यै᳚ । \newline
27. यस्यै॑ नि॒क्रम॑णे नि॒क्रम॑णे॒ यस्यै॒ यस्यै॑ नि॒क्रम॑णे । \newline
28. नि॒क्रम॑णे घृ॒तम् घृ॒तन् नि॒क्रम॑णे नि॒क्रम॑णे घृ॒तम् । \newline
29. नि॒क्रम॑ण॒ इति॑ नि - क्रम॑णे । \newline
30. घृ॒तम् प्र॒जाः प्र॒जा घृ॒तम् घृ॒तम् प्र॒जाः । \newline
31. प्र॒जाः स॒ञ्जीव॑न्तीः स॒ञ्जीव॑न्तीः प्र॒जाः प्र॒जाः स॒ञ्जीव॑न्तीः । \newline
32. प्र॒जा इति॑ प्र - जाः । \newline
33. स॒ञ्जीव॑न्तीः॒ पिब॑न्ति॒ पिब॑न्ति स॒ञ्जीव॑न्तीः स॒ञ्जीव॑न्तीः॒ पिब॑न्ति । \newline
34. स॒ञ्जीव॑न्ती॒रिति॑ सं - जीव॑न्तीः । \newline
35. पिब॒न्ती तीति॒ पिब॑न्ति॒ पिब॒न्ती ति॑ । \newline
36. इति॑ छि॒नत्ति॑ छि॒नत्ती तीति॑ छि॒नत्ति॑ । \newline
37. छि॒नत्ति॒ सा सा छि॒नत्ति॑ छि॒नत्ति॒ सा । \newline
38. सा न न सा सा न । \newline
39. न छि॑न॒त्ती(3) छि॑न॒त्ती(3) न न छि॑न॒त्ती(3) । \newline
40. छि॒न॒त्ती(3) इतीति॑ छिन॒त्ती(3) छि॑न॒त्ती(3) इति॑ । \newline
41. इति॒ न ने तीति॒ न । \newline
42. न छि॑नत्ति छिनत्ति॒ न न छि॑नत्ति । \newline
43. छि॒न॒त्ती तीति॑ छिनत्ति छिन॒त्ती ति॑ । \newline
44. इति॑ ह॒ हे तीति॑ ह । \newline
45. हो॒वा॒चो॒ वा॒च॒ ह॒ हो॒वा॒च॒ । \newline
46. उ॒वा॒च॒ प्र प्रोवा॑चो वाच॒ प्र । \newline
47. प्र तु तु प्र प्र तु । \newline
48. तु ज॑नयति जनयति॒ तु तु ज॑नयति । \newline
49. ज॒न॒य॒ती तीति॑ जनयति जनय॒ती ति॑ । \newline
50. इत्ये॒ष ए॒ष इती त्ये॒षः । \newline
51. ए॒ष वै वा ए॒ष ए॒ष वै । \newline
52. वा इडा॒ मिडां॒ ॅवै वा इडा᳚म् । \newline
53. इडा॒ मुपोपे डा॒ मिडा॒ मुप॑ । \newline
54. उपा᳚ह्वथा अह्वथा॒ उपोपा᳚ह्वथाः । \newline
55. अ॒ह्व॒था॒ इती त्य॑ह्वथा अह्वथा॒ इति॑ । \newline
56. इति॑ ह॒ हे तीति॑ ह । \newline
57. हो॒वा॒चो॒ वा॒च॒ ह॒ हो॒वा॒च॒ । \newline
58. उ॒वा॒च॒ वृष्टि॒र् वृष्टि॑ रुवाचो वाच॒ वृष्टिः॑ । \newline
59. वृष्टि॒र् वै वै वृष्टि॒र् वृष्टि॒र् वै । \newline
60. वा इडेडा॒ वै वा इडा᳚ । \newline
61. इडा॒ वृष्ट्यै॒ वृष्ट्या॒ इडेडा॒ वृष्ट्यै᳚ । \newline
62. वृष्ट्यै॒ वै वै वृष्ट्यै॒ वृष्ट्यै॒ वै । \newline
63. वै नि॒क्रम॑णे नि॒क्रम॑णे॒ वै वै नि॒क्रम॑णे । \newline
64. नि॒क्रम॑णे घृ॒तम् घृ॒तन् नि॒क्रम॑णे नि॒क्रम॑णे घृ॒तम् । \newline
65. नि॒क्रम॑ण॒ इति॑ नि - क्रम॑णे । \newline
66. घृ॒तम् प्र॒जाः प्र॒जा घृ॒तम् घृ॒तम् प्र॒जाः । \newline
67. प्र॒जाः स॒ञ्जीव॑न्तीः स॒ञ्जीव॑न्तीः प्र॒जाः प्र॒जाः स॒ञ्जीव॑न्तीः । \newline
68. प्र॒जा इति॑ प्र - जाः । \newline
69. स॒ञ्जीव॑न्तीः पिबन्ति पिबन्ति स॒ञ्जीव॑न्तीः स॒ञ्जीव॑न्तीः पिबन्ति । \newline
70. स॒ञ्जीव॑न्ती॒रिति॑ सं - जीव॑न्तीः । \newline
71. पि॒ब॒न्ति॒ यो यः पि॑बन्ति पिबन्ति॒ यः । \newline
72. य ए॒व मे॒वं ॅयो य ए॒वम् । \newline
73. ए॒वं ॅवेद॒ वेदै॒व मे॒वं ॅवेद॑ । \newline
74. वेद॒ प्र प्र वेद॒ वेद॒ प्र । \newline
75. प्रैवैव प्र प्रैव । \newline
76. ए॒व जा॑यते जायत ए॒वैव जा॑यते । \newline
77. जा॒य॒ते॒ ऽन्ना॒दो᳚ ऽन्ना॒दो जा॑यते जायते ऽन्ना॒दः । \newline
78. अ॒न्ना॒दो भ॑वति भव त्यन्ना॒दो᳚ ऽन्ना॒दो भ॑वति । \newline
79. अ॒न्ना॒द इत्य॑न्न - अ॒दः । \newline
80. भ॒व॒तीति॑ भवति । \newline

\textbf{Ghana Paata } \newline

1. इ॒यं ॅवै वा इ॒य मि॒यं ॅवै प्र॒जाः प्र॒जा वा इ॒य मि॒यं ॅवै प्र॒जाः । \newline
2. वै प्र॒जाः प्र॒जा वै वै प्र॒जाः प॑रा॒भव॑न्तीः परा॒भव॑न्तीः प्र॒जा वै वै प्र॒जाः प॑रा॒भव॑न्तीः । \newline
3. प्र॒जाः प॑रा॒भव॑न्तीः परा॒भव॑न्तीः प्र॒जाः प्र॒जाः प॑रा॒भव॑न्ती॒ रन्वनु॑ परा॒भव॑न्तीः प्र॒जाः प्र॒जाः प॑रा॒भव॑न्ती॒ रनु॑ । \newline
4. प्र॒जा इति॑ प्र - जाः । \newline
5. प॒रा॒भव॑न्ती॒ रन्वनु॑ परा॒भव॑न्तीः परा॒भव॑न्ती॒ रनु॑ गृह्णाति गृह्णा॒त्यनु॑ परा॒भव॑न्तीः परा॒भव॑न्ती॒ रनु॑ गृह्णाति । \newline
6. प॒रा॒भव॑न्ती॒रिति॑ परा - भव॑न्तीः । \newline
7. अनु॑ गृह्णाति गृह्णा॒ त्यन्वनु॑ गृह्णाति॒ प्रति॒ प्रति॑ गृह्णा॒ त्यन्वनु॑ गृह्णाति॒ प्रति॑ । \newline
8. गृ॒ह्णा॒ति॒ प्रति॒ प्रति॑ गृह्णाति गृह्णाति॒ प्रत्या॒भव॑न्ती रा॒भव॑न्तीः॒ प्रति॑ गृह्णाति गृह्णाति॒ प्रत्या॒भव॑न्तीः । \newline
9. प्रत्या॒भव॑न्ती रा॒भव॑न्तीः॒ प्रति॒ प्रत्या॒भव॑न्तीर् गृह्णाति गृह्णात्या॒भव॑न्तीः॒ प्रति॒ प्रत्या॒भव॑न्तीर् गृह्णाति । \newline
10. आ॒भव॑न्तीर् गृह्णाति गृह्णा त्या॒भव॑न्ती रा॒भव॑न्तीर् गृह्णाति॒ यो यो गृ॑ह्णा त्या॒भव॑न्ती रा॒भव॑न्तीर् गृह्णाति॒ यः । \newline
11. आ॒भव॑न्ती॒रित्या᳚ - भव॑न्तीः । \newline
12. गृ॒ह्णा॒ति॒ यो यो गृ॑ह्णाति गृह्णाति॒ य ए॒व मे॒वं ॅयो गृ॑ह्णाति गृह्णाति॒ य ए॒वम् । \newline
13. य ए॒व मे॒वं ॅयो य ए॒वं ॅवेद॒ वेदै॒वं ॅयो य ए॒वं ॅवेद॑ । \newline
14. ए॒वं ॅवेद॒ वेदै॒व मे॒वं ॅवेद॒ प्रति॒ प्रति॒ वेदै॒व मे॒वं ॅवेद॒ प्रति॑ । \newline
15. वेद॒ प्रति॒ प्रति॒ वेद॒ वेद॒ प्रत्ये॒वैव प्रति॒ वेद॒ वेद॒ प्रत्ये॒व । \newline
16. प्रत्ये॒वैव प्रति॒ प्रत्ये॒व ति॑ष्ठति तिष्ठत्ये॒व प्रति॒ प्रत्ये॒व ति॑ष्ठति । \newline
17. ए॒व ति॑ष्ठति तिष्ठत्ये॒वैव ति॑ष्ठ॒त्यथाथ॑ तिष्ठत्ये॒वैव ति॑ष्ठ॒त्यथ॑ । \newline
18. ति॒ष्ठ॒त्यथाथ॑ तिष्ठति तिष्ठ॒त्यथ॒ वै वा अथ॑ तिष्ठति तिष्ठ॒त्यथ॒ वै । \newline
19. अथ॒ वै वा अथाथ॒ वै ताम् तां ॅवा अथाथ॒ वै ताम् । \newline
20. वै ताम् तां ॅवै वै ता मुपोप॒ तां ॅवै वै ता मुप॑ । \newline
21. ता मुपोप॒ ताम् ता मुपा᳚ह्वे ऽह्व॒ उप॒ ताम् ता मुपा᳚ह्वे । \newline
22. उपा᳚ह्वे ऽह्व॒ उपोपा᳚ह्व॒ इतीत्य॑ह्व॒ उपोपा᳚ह्व॒ इति॑ । \newline
23. अ॒ह्व॒ इतीत्य॑ह्वे ऽह्व॒ इति॑ ह॒ हे त्य॑ह्वे ऽह्व॒ इति॑ ह । \newline
24. इति॑ ह॒ हे तीति॑ होवाचो वाच॒ हे तीति॑ होवाच । \newline
25. हो॒वा॒चो॒ वा॒च॒ ह॒ हो॒वा॒च॒ यस्यै॒ यस्या॑ उवाच ह होवाच॒ यस्यै᳚ । \newline
26. उ॒वा॒च॒ यस्यै॒ यस्या॑ उवाचो वाच॒ यस्यै॑ नि॒क्रम॑णे नि॒क्रम॑णे॒ यस्या॑ उवाचो वाच॒ यस्यै॑ नि॒क्रम॑णे । \newline
27. यस्यै॑ नि॒क्रम॑णे नि॒क्रम॑णे॒ यस्यै॒ यस्यै॑ नि॒क्रम॑णे घृ॒तम् घृ॒तम् नि॒क्रम॑णे॒ यस्यै॒ यस्यै॑ नि॒क्रम॑णे घृ॒तम् । \newline
28. नि॒क्रम॑णे घृ॒तम् घृ॒तम् नि॒क्रम॑णे नि॒क्रम॑णे घृ॒तम् प्र॒जाः प्र॒जा घृ॒तम् नि॒क्रम॑णे नि॒क्रम॑णे घृ॒तम् प्र॒जाः । \newline
29. नि॒क्रम॑ण॒ इति॑ नि - क्रम॑णे । \newline
30. घृ॒तम् प्र॒जाः प्र॒जा घृ॒तम् घृ॒तम् प्र॒जाः स॒ञ्जीव॑न्तीः स॒ञ्जीव॑न्तीः प्र॒जा घृ॒तम् घृ॒तम् प्र॒जाः स॒ञ्जीव॑न्तीः । \newline
31. प्र॒जाः स॒ञ्जीव॑न्तीः स॒ञ्जीव॑न्तीः प्र॒जाः प्र॒जाः स॒ञ्जीव॑न्तीः॒ पिब॑न्ति॒ पिब॑न्ति स॒ञ्जीव॑न्तीः प्र॒जाः प्र॒जाः स॒ञ्जीव॑न्तीः॒ पिब॑न्ति । \newline
32. प्र॒जा इति॑ प्र - जाः । \newline
33. स॒ञ्जीव॑न्तीः॒ पिब॑न्ति॒ पिब॑न्ति स॒ञ्जीव॑न्तीः स॒ञ्जीव॑न्तीः॒ पिब॒न्तीतीति॒ पिब॑न्ति स॒ञ्जीव॑न्तीः स॒ञ्जीव॑न्तीः॒ पिब॒न्तीति॑ । \newline
34. स॒ञ्जीव॑न्ती॒रिति॑ सं - जीव॑न्तीः । \newline
35. पिब॒न्तीतीति॒ पिब॑न्ति॒ पिब॒न्तीति॑ छि॒नत्ति॑ छि॒नत्तीति॒ पिब॑न्ति॒ पिब॒न्तीति॑ छि॒नत्ति॑ । \newline
36. इति॑ छि॒नत्ति॑ छि॒नत्तीतीति॑ छि॒नत्ति॒ सा सा छि॒नत्तीतीति॑ छि॒नत्ति॒ सा । \newline
37. छि॒नत्ति॒ सा सा छि॒नत्ति॑ छि॒नत्ति॒ सा न न सा छि॒नत्ति॑ छि॒नत्ति॒ सा न । \newline
38. सा न न सा सा न छि॑न॒त्ती(3) छि॑न॒त्ती(3) न सा सा न छि॑न॒त्ती(3) । \newline
39. न छि॑न॒त्ती(3) छि॑न॒त्ती(3) न न छि॑न॒त्ती(3) इतीति॑ छिन॒त्ती(3) न न छि॑न॒त्ती(3) इति॑ । \newline
40. छि॒न॒त्ती(3) इतीति॑ छिन॒त्ती(3) छि॑न॒त्ती(3) इति॒ न ने ति॑ छिन॒त्ती(3) छि॑न॒त्ती(3) इति॒ न । \newline
41. इति॒ न ने तीति॒ न छि॑नत्ति छिनत्ति॒ ने तीति॒ न छि॑नत्ति । \newline
42. न छि॑नत्ति छिनत्ति॒ न न छि॑न॒त्ती तीति॑ छिनत्ति॒ न न छि॑न॒त्तीति॑ । \newline
43. छि॒न॒त्ती तीति॑ छिनत्ति छिन॒त्तीति॑ ह॒ हे ति॑ छिनत्ति छिन॒त्तीति॑ ह । \newline
44. इति॑ ह॒ हे तीति॑ होवाचो वाच॒ हे तीति॑ होवाच । \newline
45. हो॒वा॒चो॒ वा॒च॒ ह॒ हो॒वा॒च॒ प्र प्रोवा॑च ह होवाच॒ प्र । \newline
46. उ॒वा॒च॒ प्र प्रोवा॑चो वाच॒ प्र तु तु प्रोवा॑चो वाच॒ प्र तु । \newline
47. प्र तु तु प्र प्र तु ज॑नयति जनयति॒ तु प्र प्र तु ज॑नयति । \newline
48. तु ज॑नयति जनयति॒ तु तु ज॑नय॒ती तीति॑ जनयति॒ तु तु ज॑नय॒तीति॑ । \newline
49. ज॒न॒य॒ती तीति॑ जनयति जनय॒ती त्ये॒ष ए॒ष इति॑ जनयति जनय॒ती त्ये॒षः । \newline
50. इत्ये॒ष ए॒ष इतीत्ये॒ष वै वा ए॒ष इतीत्ये॒ष वै । \newline
51. ए॒ष वै वा ए॒ष ए॒ष वा इडा॒ मिडां॒ ॅवा ए॒ष ए॒ष वा इडा᳚म् । \newline
52. वा इडा॒ मिडां॒ ॅवै वा इडा॒ मुपोपे डां॒ ॅवै वा इडा॒ मुप॑ । \newline
53. इडा॒ मुपोपे डा॒ मिडा॒ मुपा᳚ह्वथा अह्वथा॒ उपे डा॒ मिडा॒ मुपा᳚ह्वथाः । \newline
54. उपा᳚ह्वथा अह्वथा॒ उपोपा᳚ह्वथा॒ इतीत्य॑ह्वथा॒ उपोपा᳚ह्वथा॒ इति॑ । \newline
55. अ॒ह्व॒था॒ इतीत्य॑ह्वथा अह्वथा॒ इति॑ ह॒ हे त्य॑ह्वथा अह्वथा॒ इति॑ ह । \newline
56. इति॑ ह॒ हे तीति॑ होवाचो वाच॒ हे तीति॑ होवाच । \newline
57. हो॒वा॒चो॒ वा॒च॒ ह॒ हो॒वा॒च॒ वृष्टि॒र् वृष्टि॑रुवाच ह होवाच॒ वृष्टिः॑ । \newline
58. उ॒वा॒च॒ वृष्टि॒र् वृष्टि॑ रुवाचो वाच॒ वृष्टि॒र् वै वै वृष्टि॑ रुवाचो वाच॒ वृष्टि॒र् वै । \newline
59. वृष्टि॒र् वै वै वृष्टि॒र् वृष्टि॒र् वा इडेडा॒ वै वृष्टि॒र् वृष्टि॒र् वा इडा᳚ । \newline
60. वा इडेडा॒ वै वा इडा॒ वृष्ट्यै॒ वृष्ट्या॒ इडा॒ वै वा इडा॒ वृष्ट्यै᳚ । \newline
61. इडा॒ वृष्ट्यै॒ वृष्ट्या॒ इडेडा॒ वृष्ट्यै॒ वै वै वृष्ट्या॒ इडेडा॒ वृष्ट्यै॒ वै । \newline
62. वृष्ट्यै॒ वै वै वृष्ट्यै॒ वृष्ट्यै॒ वै नि॒क्रम॑णे नि॒क्रम॑णे॒ वै वृष्ट्यै॒ वृष्ट्यै॒ वै नि॒क्रम॑णे । \newline
63. वै नि॒क्रम॑णे नि॒क्रम॑णे॒ वै वै नि॒क्रम॑णे घृ॒तम् घृ॒तम् नि॒क्रम॑णे॒ वै वै नि॒क्रम॑णे घृ॒तम् । \newline
64. नि॒क्रम॑णे घृ॒तम् घृ॒तम् नि॒क्रम॑णे नि॒क्रम॑णे घृ॒तम् प्र॒जाः प्र॒जा घृ॒तम् नि॒क्रम॑णे नि॒क्रम॑णे घृ॒तम् प्र॒जाः । \newline
65. नि॒क्रम॑ण॒ इति॑ नि - क्रम॑णे । \newline
66. घृ॒तम् प्र॒जाः प्र॒जा घृ॒तम् घृ॒तम् प्र॒जाः स॒ञ्जीव॑न्तीः स॒ञ्जीव॑न्तीः प्र॒जा घृ॒तम् घृ॒तम् प्र॒जाः स॒ञ्जीव॑न्तीः । \newline
67. प्र॒जाः स॒ञ्जीव॑न्तीः स॒ञ्जीव॑न्तीः प्र॒जाः प्र॒जाः स॒ञ्जीव॑न्तीः पिबन्ति पिबन्ति स॒ञ्जीव॑न्तीः प्र॒जाः प्र॒जाः स॒ञ्जीव॑न्तीः पिबन्ति । \newline
68. प्र॒जा इति॑ प्र - जाः । \newline
69. स॒ञ्जीव॑न्तीः पिबन्ति पिबन्ति स॒ञ्जीव॑न्तीः स॒ञ्जीव॑न्तीः पिबन्ति॒ यो यः पि॑बन्ति स॒ञ्जीव॑न्तीः स॒ञ्जीव॑न्तीः पिबन्ति॒ यः । \newline
70. स॒ञ्जीव॑न्ती॒रिति॑ सं - जीव॑न्तीः । \newline
71. पि॒ब॒न्ति॒ यो यः पि॑बन्ति पिबन्ति॒ य ए॒व मे॒वं ॅयः पि॑बन्ति पिबन्ति॒ य ए॒वम् । \newline
72. य ए॒व मे॒वं ॅयो य ए॒वं ॅवेद॒ वेदै॒वं ॅयो य ए॒वं ॅवेद॑ । \newline
73. ए॒वं ॅवेद॒ वेदै॒व मे॒वं ॅवेद॒ प्र प्र वेदै॒व मे॒वं ॅवेद॒ प्र । \newline
74. वेद॒ प्र प्र वेद॒ वेद॒ प्रैवैव प्र वेद॒ वेद॒ प्रैव । \newline
75. प्रैवैव प्र प्रैव जा॑यते जायत ए॒व प्र प्रैव जा॑यते । \newline
76. ए॒व जा॑यते जायत ए॒वैव जा॑यते ऽन्ना॒दो᳚ ऽन्ना॒दो जा॑यत ए॒वैव जा॑यते ऽन्ना॒दः । \newline
77. जा॒य॒ते॒ ऽन्ना॒दो᳚ ऽन्ना॒दो जा॑यते जायते ऽन्ना॒दो भ॑वति भवत्यन्ना॒दो जा॑यते जायते ऽन्ना॒दो भ॑वति । \newline
78. अ॒न्ना॒दो भ॑वति भवत्यन्ना॒दो᳚ ऽन्ना॒दो भ॑वति । \newline
79. अ॒न्ना॒द इत्य॑न्न - अ॒दः । \newline
80. भ॒व॒तीति॑ भवति । \newline
\pagebreak
\markright{ TS 1.7.3.1  \hfill https://www.vedavms.in \hfill}
\addcontentsline{toc}{section}{ TS 1.7.3.1 }
\section*{ TS 1.7.3.1 }

\textbf{TS 1.7.3.1 } \newline
\textbf{Samhita Paata} \newline

प॒रोक्षं॒ ॅवा अ॒न्ये दे॒वा इ॒ज्यन्ते᳚ प्र॒त्यक्ष॑म॒न्ये यद्-यज॑ते॒ य ए॒व दे॒वाः प॒रोक्ष॑मि॒ज्यन्ते॒ ताने॒व तद्-य॑जति॒ यद॑न्वाहा॒र्य॑-मा॒हर॑त्ये॒ते वै दे॒वाः प्र॒त्यक्षं॒ ॅयद् ब्रा᳚ह्म॒णास्ताने॒व तेन॑ प्रीणा॒त्यथो॒ दक्षि॑णै॒वास्यै॒षाऽथो॑ य॒ज्ञ्स्यै॒व छि॒द्रमपि॑ दधाति॒ यद्वै य॒ज्ञ्स्य॑ क्रू॒रं ॅयद्विलि॑ष्टं॒ तद॑न्वाहा॒र्ये॑णा॒ - [ ] \newline

\textbf{Pada Paata} \newline

प॒रोक्ष॒मिति॑ परः - अक्ष᳚म् । वै । अ॒न्ये । दे॒वाः । इ॒ज्यन्ते᳚ । प्र॒त्यक्ष॒मिति॑ प्रति - अक्ष᳚म् । अ॒न्ये । यत् । यज॑ते । ये । ए॒व । दे॒वाः । प॒रोक्ष॒मिति॑ परः - अक्ष᳚म् । इ॒ज्यन्ते᳚ । तान् । ए॒व । तत् । य॒ज॒ति॒ । यत् । अ॒न्वा॒हा॒र्य॑मित्य॑नु - आ॒हा॒र्य᳚म् । आ॒हर॒तीत्या᳚-हर॑ति । ए॒ते । वै । दे॒वाः । प्र॒त्यक्ष॒मिति॑ प्रति - अक्ष᳚म् । यत् । ब्रा॒ह्म॒णाः । तान् । ए॒व । तेन॑ । प्री॒णा॒ति॒ । अथो॒ इति॑ । दक्षि॑णा । ए॒व । अ॒स्य॒ । ए॒षा । अथो॒ इति॑ । य॒ज्ञ्स्य॑ । ए॒व । छि॒द्रम् । अपीति॑ । द॒धा॒ति॒ । यत् । वै । य॒ज्ञ्स्य॑ । क्रू॒रम् । यत् । विलि॑ष्ट॒मिति॒ वि - लि॒ष्ट॒म् । तत् । अ॒न्वा॒हा॒र्ये॑णेत्य॑नु - आ॒हा॒र्ये॑ण ।  \newline


\textbf{Krama Paata} \newline

प॒रोक्षं॒ ॅवै । प॒रोक्ष॒मिति॑ परः - अक्ष᳚म् । वा अ॒न्ये । अ॒न्ये दे॒वाः । दे॒वा इ॒ज्यन्ते᳚ । इ॒ज्यन्ते᳚ प्र॒त्यक्ष᳚म् । प्र॒त्यक्ष॑म॒न्ये । प्र॒त्यक्ष॒मिति॑ प्रति - अक्ष᳚म् । अ॒न्ये यत् । यद् यज॑ते । यज॑ते॒ ये । 
य ए॒व । ए॒व दे॒वाः । दे॒वाः प॒रोक्ष᳚म् । प॒रोक्ष॑मि॒ज्यन्ते᳚ । प॒रोक्ष॒मिति॑ परः - अक्ष᳚म् । इ॒ज्यन्ते॒ तान् । ताने॒व । ए॒व तत् । तद् य॑जति । य॒ज॒ति॒ यत् । यद॑न्वाहा॒र्य᳚म् । अ॒न्वा॒हा॒र्य॑मा॒हर॑ति । अ॒न्वा॒हा॒र्य॑मित्य॑नु - आ॒हा॒र्य᳚म् । आ॒हर॑त्ये॒ते । आ॒हर॒तीत्या᳚ - हर॑ति । ए॒ते वै । वै दे॒वाः । दे॒वाः प्र॒त्यक्ष᳚म् । प्र॒त्यक्षं॒ ॅयत् । प्र॒त्यक्ष॒मिति॑ प्रति - अक्ष᳚म् । यद् ब्रा᳚ह्म॒णाः । ब्रा॒ह्म॒णास्तान् । ताने॒व । ए॒व तेन॑ । तेन॑ प्रीणाति । प्री॒णा॒त्यथो᳚ । अथो॒ दक्षि॑णा । अथो॒ इत्यथो᳚ । दक्षि॑णै॒व । ए॒वास्य॑ । अ॒स्यै॒षा । ए॒षाऽथो᳚ । अथो॑ य॒ज्ञ्स्य॑ । अथो॒ इत्यथो᳚ । य॒ज्ञ्स्यै॒व । ए॒व छि॒द्रम् । छि॒द्रमपि॑ । अपि॑ दधाति । द॒धा॒ति॒ यत् । यद् वै । वै य॒ज्ञ्स्य॑ । य॒ज्ञ्स्य॑ क्रू॒रम् । क्रू॒रं ॅयत् । यद् विलि॑ष्टम् । विलि॑ष्ट॒म् तत् । विलि॑ष्ट॒मिति॒ वि - लि॒ष्ट॒म् । तद॑न्वाहा॒र्ये॑ण । अ॒न्वा॒हा॒र्ये॑णा॒न्वाह॑रति । अ॒न्वा॒हा॒र्ये॑णेत्य॑नु - आ॒हा॒र्ये॑ण \newline

\textbf{Jatai Paata} \newline

1. प॒रोक्षं॒ ॅवै वै प॒रोक्ष॑म् प॒रोक्षं॒ ॅवै । \newline
2. प॒रोक्ष॒मिति॑ परः - अक्ष᳚म् । \newline
3. वा अ॒न्ये᳚ ऽन्ये वै वा अ॒न्ये । \newline
4. अ॒न्ये दे॒वा दे॒वा अ॒न्ये᳚ ऽन्ये दे॒वाः । \newline
5. दे॒वा इ॒ज्यन्त॑ इ॒ज्यन्ते॑ दे॒वा दे॒वा इ॒ज्यन्ते᳚ । \newline
6. इ॒ज्यन्ते᳚ प्र॒त्यक्ष॑म् प्र॒त्यक्ष॑ मि॒ज्यन्त॑ इ॒ज्यन्ते᳚ प्र॒त्यक्ष᳚म् । \newline
7. प्र॒त्यक्ष॑ म॒न्ये᳚ ऽन्ये प्र॒त्यक्ष॑म् प्र॒त्यक्ष॑ म॒न्ये । \newline
8. प्र॒त्यक्ष॒मिति॑ प्रति - अक्ष᳚म् । \newline
9. अ॒न्ये यद् यद॒न्ये᳚ ऽन्ये यत् । \newline
10. यद् यज॑ते॒ यज॑ते॒ यद् यद् यज॑ते । \newline
11. यज॑ते॒ ये ये यज॑ते॒ यज॑ते॒ ये । \newline
12. य ए॒वैव ये य ए॒व । \newline
13. ए॒व दे॒वा दे॒वा ए॒वैव दे॒वाः । \newline
14. दे॒वाः प॒रोक्ष॑म् प॒रोक्ष॑म् दे॒वा दे॒वाः प॒रोक्ष᳚म् । \newline
15. प॒रोक्ष॑ मि॒ज्यन्त॑ इ॒ज्यन्ते॑ प॒रोक्ष॑म् प॒रोक्ष॑ मि॒ज्यन्ते᳚ । \newline
16. प॒रोक्ष॒मिति॑ परः - अक्ष᳚म् । \newline
17. इ॒ज्यन्ते॒ ताꣳ स्ता नि॒ज्यन्त॑ इ॒ज्यन्ते॒ तान् । \newline
18. ता ने॒वैव ताꣳ स्ता ने॒व । \newline
19. ए॒व तत् तदे॒वैव तत् । \newline
20. तद् य॑जति यजति॒ तत् तद् य॑जति । \newline
21. य॒ज॒ति॒ यद् यद् य॑जति यजति॒ यत् । \newline
22. यद॑न्वाहा॒र्य॑ मन्वाहा॒र्यं॑ ॅयद् यद॑न्वाहा॒र्य᳚म् । \newline
23. अ॒न्वा॒हा॒र्य॑ मा॒हर॑ त्या॒हर॑ त्यन्वाहा॒र्य॑ मन्वाहा॒र्य॑ मा॒हर॑ति । \newline
24. अ॒न्वा॒हा॒र्य॑मित्य॑नु - आ॒हा॒र्य᳚म् । \newline
25. आ॒हर॑ त्ये॒त ए॒त आ॒हर॑ त्या॒हर॑ त्ये॒ते । \newline
26. आ॒हर॒तीत्या᳚ - हर॑ति । \newline
27. ए॒ते वै वा ए॒त ए॒ते वै । \newline
28. वै दे॒वा दे॒वा वै वै दे॒वाः । \newline
29. दे॒वाः प्र॒त्यक्ष॑म् प्र॒त्यक्ष॑म् दे॒वा दे॒वाः प्र॒त्यक्ष᳚म् । \newline
30. प्र॒त्यक्षं॒ ॅयद् यत् प्र॒त्यक्ष॑म् प्र॒त्यक्षं॒ ॅयत् । \newline
31. प्र॒त्यक्ष॒मिति॑ प्रति - अक्ष᳚म् । \newline
32. यद् ब्रा᳚ह्म॒णा ब्रा᳚ह्म॒णा यद् यद् ब्रा᳚ह्म॒णाः । \newline
33. ब्रा॒ह्म॒णा स्ताꣳ स्तान् ब्रा᳚ह्म॒णा ब्रा᳚ह्म॒णा स्तान् । \newline
34. ता ने॒वैव ताꣳ स्ता ने॒व । \newline
35. ए॒व तेन॒ ते नै॒वैव तेन॑ । \newline
36. तेन॑ प्रीणाति प्रीणाति॒ तेन॒ तेन॑ प्रीणाति । \newline
37. प्री॒णा॒ त्यथो॒ अथो᳚ प्रीणाति प्रीणा॒ त्यथो᳚ । \newline
38. अथो॒ दक्षि॑णा॒ दक्षि॒णा ऽथो॒ अथो॒ दक्षि॑णा । \newline
39. अथो॒ इत्यथो᳚ । \newline
40. दक्षि॑ णै॒वैव दक्षि॑णा॒ दक्षि॑ णै॒व । \newline
41. ए॒वास्या᳚ स्यै॒वै वास्य॑ । \newline
42. अ॒स्यै॒ षैषा ऽस्या᳚ स्यै॒षा । \newline
43. ए॒षा ऽथो॒ अथो॑ ए॒षैषा ऽथो᳚ । \newline
44. अथो॑ य॒ज्ञ्स्य॑ य॒ज्ञ्स्याथो॒ अथो॑ य॒ज्ञ्स्य॑ । \newline
45. अथो॒ इत्यथो᳚ । \newline
46. य॒ज्ञ् स्यै॒वैव य॒ज्ञ्स्य॑ य॒ज्ञ् स्यै॒व । \newline
47. ए॒व छि॒द्रम् छि॒द्र मे॒वैव छि॒द्रम् । \newline
48. छि॒द्र मप्यपि॑ छि॒द्रम् छि॒द्र मपि॑ । \newline
49. अपि॑ दधाति दधा॒ त्यप्यपि॑ दधाति । \newline
50. द॒धा॒ति॒ यद् यद् द॑धाति दधाति॒ यत् । \newline
51. यद् वै वै यद् यद् वै । \newline
52. वै य॒ज्ञ्स्य॑ य॒ज्ञ्स्य॒ वै वै य॒ज्ञ्स्य॑ । \newline
53. य॒ज्ञ्स्य॑ क्रू॒रम् क्रू॒रं ॅय॒ज्ञ्स्य॑ य॒ज्ञ्स्य॑ क्रू॒रम् । \newline
54. क्रू॒रं ॅयद् यत् क्रू॒रम् क्रू॒रं ॅयत् । \newline
55. यद् विलि॑ष्टं॒ ॅविलि॑ष्टं॒ ॅयद् यद् विलि॑ष्टम् । \newline
56. विलि॑ष्ट॒म् तत् तद् विलि॑ष्टं॒ ॅविलि॑ष्ट॒म् तत् । \newline
57. विलि॑ष्ट॒मिति॒ वि - लि॒ष्ट॒म् । \newline
58. त द॑न्वाहा॒र्ये॑ णान्वाहा॒र्ये॑ण॒ तत् तद॑ न्वाहा॒र्ये॑ण । \newline
59. अ॒न्वा॒हा॒र्ये॑णा॒ न्वाह॑र त्य॒न्वाह॑र त्यन्वाहा॒र्ये॑ णान्वाहा॒र्ये॑ णा॒न्वाह॑रति । \newline
60. अ॒न्वा॒हा॒र्ये॑णेत्य॑नु - आ॒हा॒र्ये॑ण । \newline

\textbf{Ghana Paata } \newline

1. प॒रोक्षं॒ ॅवै वै प॒रोक्ष॑म् प॒रोक्षं॒ ॅवा अ॒न्ये᳚ ऽन्ये वै प॒रोक्ष॑म् प॒रोक्षं॒ ॅवा अ॒न्ये । \newline
2. प॒रोक्ष॒मिति॑ परः - अक्ष᳚म् । \newline
3. वा अ॒न्ये᳚ ऽन्ये वै वा अ॒न्ये दे॒वा दे॒वा अ॒न्ये वै वा अ॒न्ये दे॒वाः । \newline
4. अ॒न्ये दे॒वा दे॒वा अ॒न्ये᳚ ऽन्ये दे॒वा इ॒ज्यन्त॑ इ॒ज्यन्ते॑ दे॒वा अ॒न्ये᳚ ऽन्ये दे॒वा इ॒ज्यन्ते᳚ । \newline
5. दे॒वा इ॒ज्यन्त॑ इ॒ज्यन्ते॑ दे॒वा दे॒वा इ॒ज्यन्ते᳚ प्र॒त्यक्ष॑म् प्र॒त्यक्ष॑ मि॒ज्यन्ते॑ दे॒वा दे॒वा इ॒ज्यन्ते᳚ प्र॒त्यक्ष᳚म् । \newline
6. इ॒ज्यन्ते᳚ प्र॒त्यक्ष॑म् प्र॒त्यक्ष॑ मि॒ज्यन्त॑ इ॒ज्यन्ते᳚ प्र॒त्यक्ष॑ म॒न्ये᳚ ऽन्ये प्र॒त्यक्ष॑ मि॒ज्यन्त॑ इ॒ज्यन्ते᳚ प्र॒त्यक्ष॑ म॒न्ये । \newline
7. प्र॒त्यक्ष॑ म॒न्ये᳚ ऽन्ये प्र॒त्यक्ष॑म् प्र॒त्यक्ष॑ म॒न्ये यद् यद॒न्ये प्र॒त्यक्ष॑म् प्र॒त्यक्ष॑ म॒न्ये यत् । \newline
8. प्र॒त्यक्ष॒मिति॑ प्रति - अक्ष᳚म् । \newline
9. अ॒न्ये यद् यद॒न्ये᳚ ऽन्ये यद् यज॑ते॒ यज॑ते॒ यद॒न्ये᳚ ऽन्ये यद् यज॑ते । \newline
10. यद् यज॑ते॒ यज॑ते॒ यद् यद् यज॑ते॒ ये ये यज॑ते॒ यद् यद् यज॑ते॒ ये । \newline
11. यज॑ते॒ ये ये यज॑ते॒ यज॑ते॒ य ए॒वैव ये यज॑ते॒ यज॑ते॒ य ए॒व । \newline
12. य ए॒वैव ये य ए॒व दे॒वा दे॒वा ए॒व ये य ए॒व दे॒वाः । \newline
13. ए॒व दे॒वा दे॒वा ए॒वैव दे॒वाः प॒रोक्ष॑म् प॒रोक्ष॑म् दे॒वा ए॒वैव दे॒वाः प॒रोक्ष᳚म् । \newline
14. दे॒वाः प॒रोक्ष॑म् प॒रोक्ष॑म् दे॒वा दे॒वाः प॒रोक्ष॑ मि॒ज्यन्त॑ इ॒ज्यन्ते॑ प॒रोक्ष॑म् दे॒वा दे॒वाः प॒रोक्ष॑ मि॒ज्यन्ते᳚ । \newline
15. प॒रोक्ष॑ मि॒ज्यन्त॑ इ॒ज्यन्ते॑ प॒रोक्ष॑म् प॒रोक्ष॑ मि॒ज्यन्ते॒ ताꣳ स्ता नि॒ज्यन्ते॑ प॒रोक्ष॑म् प॒रोक्ष॑ मि॒ज्यन्ते॒ तान् । \newline
16. प॒रोक्ष॒मिति॑ परः - अक्ष᳚म् । \newline
17. इ॒ज्यन्ते॒ ताꣳ स्ता नि॒ज्यन्त॑ इ॒ज्यन्ते॒ ता ने॒वैव ता नि॒ज्यन्त॑ इ॒ज्यन्ते॒ ता ने॒व । \newline
18. ता ने॒वैव ताꣳ स्ता ने॒व तत् तदे॒व ताꣳ स्ता ने॒व तत् । \newline
19. ए॒व तत् तदे॒वैव तद् य॑जति यजति॒ तदे॒वैव तद् य॑जति । \newline
20. तद् य॑जति यजति॒ तत् तद् य॑जति॒ यद् यद् य॑जति॒ तत् तद् य॑जति॒ यत् । \newline
21. य॒ज॒ति॒ यद् यद् य॑जति यजति॒ यद॑न्वाहा॒र्य॑ मन्वाहा॒र्यं॑ ॅयद् य॑जति यजति॒ यद॑न्वाहा॒र्य᳚म् । \newline
22. यद॑न्वाहा॒र्य॑ मन्वाहा॒र्यं॑ ॅयद् यद॑न्वाहा॒र्य॑ मा॒हर॑ त्या॒हर॑ त्यन्वाहा॒र्यं॑ ॅयद् यद॑न्वाहा॒र्य॑ मा॒हर॑ति । \newline
23. अ॒न्वा॒हा॒र्य॑ मा॒हर॑ त्या॒हर॑ त्यन्वाहा॒र्य॑ मन्वाहा॒र्य॑ मा॒हर॑त्ये॒त ए॒त आ॒हर॑ त्यन्वाहा॒र्य॑ मन्वाहा॒र्य॑ मा॒हर॑त्ये॒ते । \newline
24. अ॒न्वा॒हा॒र्य॑मित्य॑नु - आ॒हा॒र्य᳚म् । \newline
25. आ॒हर॑ त्ये॒त ए॒त आ॒हर॑ त्या॒हर॑ त्ये॒ते वै वा ए॒त आ॒हर॑ त्या॒हर॑ त्ये॒ते वै । \newline
26. आ॒हर॒तीत्या᳚ - हर॑ति । \newline
27. ए॒ते वै वा ए॒त ए॒ते वै दे॒वा दे॒वा वा ए॒त ए॒ते वै दे॒वाः । \newline
28. वै दे॒वा दे॒वा वै वै दे॒वाः प्र॒त्यक्ष॑म् प्र॒त्यक्ष॑म् दे॒वा वै वै दे॒वाः प्र॒त्यक्ष᳚म् । \newline
29. दे॒वाः प्र॒त्यक्ष॑म् प्र॒त्यक्ष॑म् दे॒वा दे॒वाः प्र॒त्यक्षं॒ ॅयद् यत् प्र॒त्यक्ष॑म् दे॒वा दे॒वाः प्र॒त्यक्षं॒ ॅयत् । \newline
30. प्र॒त्यक्षं॒ ॅयद् यत् प्र॒त्यक्ष॑म् प्र॒त्यक्षं॒ ॅयद् ब्रा᳚ह्म॒णा ब्रा᳚ह्म॒णा यत् प्र॒त्यक्ष॑म् प्र॒त्यक्षं॒ ॅयद् ब्रा᳚ह्म॒णाः । \newline
31. प्र॒त्यक्ष॒मिति॑ प्रति - अक्ष᳚म् । \newline
32. यद् ब्रा᳚ह्म॒णा ब्रा᳚ह्म॒णा यद् यद् ब्रा᳚ह्म॒णा स्ताꣳ स्तान् ब्रा᳚ह्म॒णा यद् यद् ब्रा᳚ह्म॒णा स्तान् । \newline
33. ब्रा॒ह्म॒णा स्ताꣳ स्तान् ब्रा᳚ह्म॒णा ब्रा᳚ह्म॒णा स्ता ने॒वैव तान् ब्रा᳚ह्म॒णा ब्रा᳚ह्म॒णा स्ता ने॒व । \newline
34. ता ने॒वैव ताꣳ स्ता ने॒व तेन॒ तेनै॒व ताꣳ स्ता ने॒व तेन॑ । \newline
35. ए॒व तेन॒ तेनै॒वैव तेन॑ प्रीणाति प्रीणाति॒ तेनै॒वैव तेन॑ प्रीणाति । \newline
36. तेन॑ प्रीणाति प्रीणाति॒ तेन॒ तेन॑ प्रीणा॒त्यथो॒ अथो᳚ प्रीणाति॒ तेन॒ तेन॑ प्रीणा॒त्यथो᳚ । \newline
37. प्री॒णा॒त्यथो॒ अथो᳚ प्रीणाति प्रीणा॒त्यथो॒ दक्षि॑णा॒ दक्षि॒णा ऽथो᳚ प्रीणाति प्रीणा॒त्यथो॒ दक्षि॑णा । \newline
38. अथो॒ दक्षि॑णा॒ दक्षि॒णा ऽथो॒ अथो॒ दक्षि॑णै॒वैव दक्षि॒णा ऽथो॒ अथो॒ दक्षि॑णै॒व । \newline
39. अथो॒ इत्यथो᳚ । \newline
40. दक्षि॑णै॒वैव दक्षि॑णा॒ दक्षि॑णै॒ वास्या᳚स्यै॒व दक्षि॑णा॒ दक्षि॑णै॒वास्य॑ । \newline
41. ए॒वास्या᳚ स्यै॒ वैवा स्यै॒षैषा ऽस्यै॒ वैवा स्यै॒षा । \newline
42. अ॒स्यै॒ षैषा ऽस्या᳚स्यै॒षा ऽथो॒ अथो॑ ए॒षा ऽस्या᳚स्यै॒षा ऽथो᳚ । \newline
43. ए॒षा ऽथो॒ अथो॑ ए॒षैषा ऽथो॑ य॒ज्ञ्स्य॑ य॒ज्ञ्स्याथो॑ ए॒षैषा ऽथो॑ य॒ज्ञ्स्य॑ । \newline
44. अथो॑ य॒ज्ञ्स्य॑ य॒ज्ञ्स्याथो॒ अथो॑ य॒ज्ञ्स्यै॒वैव य॒ज्ञ्स्याथो॒ अथो॑ य॒ज्ञ्स्यै॒व । \newline
45. अथो॒ इत्यथो᳚ । \newline
46. य॒ज्ञ्स्यै॒वैव य॒ज्ञ्स्य॑ य॒ज्ञ्स्यै॒व छि॒द्रम् छि॒द्र मे॒व य॒ज्ञ्स्य॑ य॒ज्ञ्स्यै॒व छि॒द्रम् । \newline
47. ए॒व छि॒द्रम् छि॒द्र मे॒वैव छि॒द्र मप्यपि॑ छि॒द्र मे॒वैव छि॒द्र मपि॑ । \newline
48. छि॒द्र मप्यपि॑ छि॒द्रम् छि॒द्र मपि॑ दधाति दधा॒त्यपि॑ छि॒द्रम् छि॒द्र मपि॑ दधाति । \newline
49. अपि॑ दधाति दधा॒ त्यप्यपि॑ दधाति॒ यद् यद् द॑धा॒ त्यप्यपि॑ दधाति॒ यत् । \newline
50. द॒धा॒ति॒ यद् यद् द॑धाति दधाति॒ यद् वै वै यद् द॑धाति दधाति॒ यद् वै । \newline
51. यद् वै वै यद् यद् वै य॒ज्ञ्स्य॑ य॒ज्ञ्स्य॒ वै यद् यद् वै य॒ज्ञ्स्य॑ । \newline
52. वै य॒ज्ञ्स्य॑ य॒ज्ञ्स्य॒ वै वै य॒ज्ञ्स्य॑ क्रू॒रम् क्रू॒रं ॅय॒ज्ञ्स्य॒ वै वै य॒ज्ञ्स्य॑ क्रू॒रम् । \newline
53. य॒ज्ञ्स्य॑ क्रू॒रम् क्रू॒रं ॅय॒ज्ञ्स्य॑ य॒ज्ञ्स्य॑ क्रू॒रं ॅयद् यत् क्रू॒रं ॅय॒ज्ञ्स्य॑ य॒ज्ञ्स्य॑ क्रू॒रं ॅयत् । \newline
54. क्रू॒रं ॅयद् यत् क्रू॒रम् क्रू॒रं ॅयद् विलि॑ष्टं॒ ॅविलि॑ष्टं॒ ॅयत् क्रू॒रम् क्रू॒रं ॅयद् विलि॑ष्टम् । \newline
55. यद् विलि॑ष्टं॒ ॅविलि॑ष्टं॒ ॅयद् यद् विलि॑ष्ट॒म् तत् तद् विलि॑ष्टं॒ ॅयद् यद् विलि॑ष्ट॒म् तत् । \newline
56. विलि॑ष्ट॒म् तत् तद् विलि॑ष्टं॒ ॅविलि॑ष्ट॒म् तद॑न्वाहा॒र्ये॑णा न्वाहा॒र्ये॑ण॒ तद् विलि॑ष्टं॒ ॅविलि॑ष्ट॒म् तद॑न्वा हा॒र्ये॑ण । \newline
57. विलि॑ष्ट॒मिति॒ वि - लि॒ष्ट॒म् । \newline
58. तद॑न्वाहा॒र्ये॑णा न्वाहा॒र्ये॑ण॒ तत् तद॑न्वाहा॒र्ये॑णा॒ न्वाह॑र त्य॒न्वाह॑र त्यन्वाहा॒र्ये॑ण॒ तत् तद॑ न्वाहा॒र्ये॑णा॒ न्वाह॑रति । \newline
59. अ॒न्वा॒हा॒र्ये॑णा॒ न्वाह॑र त्य॒न्वाह॑र त्यन्वाहा॒र्ये॑णा न्वाहा॒र्ये॑णा॒ न्वाह॑रति॒ तत् तद॒न्वाह॑र त्यन्वाहा॒र्ये॑णा न्वाहा॒र्ये॑णा॒ न्वाह॑रति॒ तत् । \newline
60. अ॒न्वा॒हा॒र्ये॑णेत्य॑नु - आ॒हा॒र्ये॑ण । \newline
\pagebreak
\markright{ TS 1.7.3.2  \hfill https://www.vedavms.in \hfill}
\addcontentsline{toc}{section}{ TS 1.7.3.2 }
\section*{ TS 1.7.3.2 }

\textbf{TS 1.7.3.2 } \newline
\textbf{Samhita Paata} \newline

ऽन्वाह॑रति॒ तद॑न्वाहा॒र्य॑स्या-न्वाहार्य॒त्वं दे॑वदू॒ता वा ए॒ते यद्-ऋ॒त्विजो॒ यद॑न्वाहा॒र्य॑-मा॒हर॑ति देवदू॒ताने॒व प्री॑णाति प्र॒जाप॑तिर् दे॒वेभ्यो॑ य॒ज्ञान् व्यादि॑श॒थ् स रि॑रिचा॒नो॑ऽमन्यत॒ स ए॒तम॑न्वाहा॒र्य॑-मभ॑क्त-मपश्य॒त् तमा॒त्मन्न॑धत्त॒स वा ए॒ष प्रा॑जाप॒त्यो यद॑न्वाहा॒र्यो॑ यस्यै॒वं ॅवि॒दुषो᳚ऽन्वाहा॒र्य॑ आह्रि॒यते॑ सा॒क्षादे॒व प्र॒जाप॑ति-मृद्ध्नो॒त्यप॑रिमितोनि॒रुप्योऽप॑रिमितः प्र॒जाप॑तिः प्र॒जाप॑ते॒- [ ] \newline

\textbf{Pada Paata} \newline

अ॒न्वाह॑र॒तीत्य॑नु - आह॑रति । तत् । अ॒न्वा॒हा॒र्य॑स्येत्य॑नु-आ॒हा॒र्य॑स्य । अ॒न्वा॒हा॒र्य॒त्वमित्य॑न्वाहार्य - त्वम् । दे॒व॒दू॒ता इति॑ देव - दू॒ताः । वै । ए॒ते । यत् । ऋ॒त्विजः॑ । यत् । अ॒न्वा॒हा॒र्य॑मित्य॑नु - आ॒हा॒र्य᳚म् । आ॒हर॒तीत्या᳚ - हर॑ति । दे॒व॒दू॒तानिति॑ देव - दू॒तान् । ए॒व । प्री॒णा॒ति॒ । प्र॒जाप॑ति॒रिति॑ प्र॒जा - प॒तिः॒ । दे॒वेभ्यः॑ । य॒ज्ञान् । व्यादि॑श॒दिति॑ वि - आदि॑शत् । सः । रि॒रि॒चा॒नः । अ॒म॒न्य॒त॒ । सः । ए॒तम् । अ॒न्वा॒हा॒र्य॑मित्य॑नु - आ॒हा॒र्य᳚म् । अभ॑क्तम् । अ॒प॒श्य॒त् । तम् । आ॒त्मन्न् । अ॒ध॒त्त॒ । सः । वै । ए॒षः । प्रा॒जा॒प॒त्य इति॑ प्राजा-प॒त्यः । यत् । अ॒न्वा॒हा॒र्य॑ इत्य॑नु - आ॒हा॒र्यः॑ । यस्य॑ । ए॒वम् । वि॒दुषः॑ । अ॒न्वा॒हा॒र्य॑ इत्य॑नु - आ॒हा॒र्यः॑ । आ॒ह्रि॒यत॒ इत्या᳚ - ह्रि॒यते᳚ । सा॒क्षादिति॑ स - अ॒क्षात् । ए॒व । प्र॒जाप॑ति॒मिति॑ प्र॒जा - प॒ति॒म् । ऋ॒द्ध्नो॒ति॒ । अप॑रिमित॒ इत्यप॑रि - मि॒तः॒ । नि॒रुप्य॒ इति॑ निः-उप्यः॑ । अप॑रिमित॒ इत्यप॑रि - मि॒तः॒ । प्र॒जाप॑ति॒रिति॑ प्र॒जा - प॒तिः॒ । प्र॒जाप॑ते॒रिति॑ प्र॒जा - प॒तेः॒ ।  \newline


\textbf{Krama Paata} \newline

अ॒न्वाह॑रति॒ तत् । अ॒न्वाह॑र॒तीत्य॑नु - आह॑रति । तद॑न्वाहा॒र्य॑स्य । अ॒न्वा॒हा॒र्य॑स्यान्वाहार्य॒त्वम् । अ॒न्वा॒हा॒र्य॑स्येत्य॑नु - आ॒हा॒र्य॑स्य । अ॒न्वा॒हा॒र्य॒त्वम् दे॑वदू॒ताः । अ॒न्वा॒हा॒र्य॒त्वमित्य॑न्वाहार्य - त्वम् । दे॒व॒दू॒ता वै । दे॒व॒दू॒ता इति॑ देव - दू॒ताः । वा ए॒ते । ए॒ते यत् । यदृ॒त्विजः॑ । ऋ॒त्विजो॒ यत् । यद॑न्वाहा॒र्य᳚म् । अ॒न्वा॒हा॒र्य॑मा॒हर॑ति । अ॒न्वा॒हा॒र्य॑मित्य॑नु - आ॒हा॒र्य᳚म् । आ॒हर॑ति देवदू॒तान् । आ॒हर॒तीत्या᳚ - हर॑ति । दे॒व॒दू॒ताने॒व । दे॒व॒दू॒तानिति॑ देव - दू॒तान् । ए॒व प्री॑णाति । प्री॒णा॒ति॒ प्र॒जाप॑तिः । प्र॒जाप॑तिर्,दे॒वेभ्यः॑ । प्र॒जाप॑ति॒रिति॑ प्र॒जा - प॒तिः॒ । दे॒वेभ्यो॑ य॒ज्ञान् । य॒ज्ञान्,व्यादि॑शत् । व्यादि॑श॒थ् सः । व्यादि॑श॒दिति॑ वि - आदि॑शत् । स रि॑रिचा॒नः । रि॒रि॒चा॒नो॑ऽमन्यत । अ॒म॒न्य॒त॒ सः । स ए॒तम् । ए॒तम॑न्वाहा॒र्य᳚म् । अ॒न्वा॒हा॒र्य॑मभ॑क्तम् । अ॒न्वा॒हा॒र्य॑मित्य॑नु - आ॒हा॒र्य᳚म् । अभ॑क्तमपश्यत् । अ॒प॒श्य॒त् तम् । तमा॒त्मन्न् । आ॒त्मन्न॑धत्त । अ॒ध॒त्त॒ सः । स वै । वा ए॒षः । ए॒ष प्रा॑जाप॒त्यः । प्रा॒जा॒प॒त्यो यत् । प्रा॒जा॒प॒त्य इति॑ प्राजा - प॒त्यः । यद॑न्वाहा॒र्यः॑ । अ॒न्वा॒हा॒र्यो॑ यस्य॑ । अ॒न्वा॒हा॒र्य॑ इत्य॑नु - आ॒हा॒र्यः॑ । यस्यै॒वम् । ए॒वं ॅवि॒दुषः॑ । वि॒दुषो᳚ ऽन्वाहा॒र्यः॑ । अ॒न्वा॒हा॒र्य॑ आह्रि॒यते᳚ । अ॒न्वा॒हा॒र्य॑ इत्य॑नु - आ॒हा॒र्यः॑ । आ॒ह्रि॒यते॑ सा॒क्षात् । आ॒ह्रि॒यत॒ इत्या᳚ - ह्रि॒यते᳚ । सा॒क्षादे॒व । सा॒क्षादिति॑ स - अ॒क्षात् । ए॒व प्र॒जाप॑तिम् । प्र॒जाप॑तिमृद्ध्नोति । प्र॒जाप॑ति॒मिति॑ प्र॒जा - प॒ति॒म् । ऋ॒द्ध्नो॒त्यप॑रिमितः । अप॑रितमितो नि॒रुप्यः॑ । अप॑रिमित॒ इत्यप॑रि - मि॒तः॒ । नि॒रुप्यो ऽप॑रिमितः । नि॒रुप्य॒ इति॑ निः - उप्यः॑ । अप॑रिमितः प्र॒जाप॑तिः । अप॑रिमित॒ इत्यप॑रि - मि॒तः॒ । प्र॒जाप॑तिः प्र॒जाप॑तेः । प्र॒जाप॑ति॒रिति॑ प्र॒जा - प॒तिः॒ । प्र॒जाप॑ते॒राप्त्यै᳚ । प्र॒जाप॑ते॒रिति॑ प्र॒जा - प॒तेः॒ \newline

\textbf{Jatai Paata} \newline

1. अ॒न्वाह॑रति॒ तत् तद॒न्वाह॑र त्य॒न्वाह॑रति॒ तत् । \newline
2. अ॒न्वाह॑र॒तीत्य॑नु - आह॑रति । \newline
3. तद॑न्वाहा॒र्य॑स्या न्वाहा॒र्य॑स्य॒ तत् तद॑न्वाहा॒र्य॑स्य । \newline
4. अ॒न्वा॒हा॒र्य॑स्या न्वाहार्य॒त्व म॑न्वाहार्य॒त्व म॑न्वाहा॒र्य॑स्या न्वाहा॒र्य॑स्या न्वाहार्य॒त्वम् । \newline
5. अ॒न्वा॒हा॒र्य॑स्येत्य॑नु - आ॒हा॒र्य॑स्य । \newline
6. अ॒न्वा॒हा॒र्य॒त्वम् दे॑वदू॒ता दे॑वदू॒ता अ॑न्वाहार्य॒त्व म॑न्वाहार्य॒त्वम् दे॑वदू॒ताः । \newline
7. अ॒न्वा॒हा॒र्य॒त्वमित्य॑न्वाहार्य - त्वम् । \newline
8. दे॒व॒दू॒ता वै वै दे॑वदू॒ता दे॑वदू॒ता वै । \newline
9. दे॒व॒दू॒ता इति॑ देव - दू॒ताः । \newline
10. वा ए॒त ए॒ते वै वा ए॒ते । \newline
11. ए॒ते यद् यदे॒त ए॒ते यत् । \newline
12. यदृ॒त्विज॑ ऋ॒त्विजो॒ यद् यदृ॒त्विजः॑ । \newline
13. ऋ॒त्विजो॒ यद् यदृ॒त्विज॑ ऋ॒त्विजो॒ यत् । \newline
14. यद॑न्वाहा॒र्य॑ मन्वाहा॒र्यं॑ ॅयद् यद॑न्वाहा॒र्य᳚म् । \newline
15. अ॒न्वा॒हा॒र्य॑ मा॒हर॑ त्या॒हर॑ त्यन्वाहा॒र्य॑ मन्वाहा॒र्य॑ मा॒हर॑ति । \newline
16. अ॒न्वा॒हा॒र्य॑मित्य॑नु - आ॒हा॒र्य᳚म् । \newline
17. आ॒हर॑ति देवदू॒तान् दे॑वदू॒ता ना॒हर॑ त्या॒हर॑ति देवदू॒तान् । \newline
18. आ॒हर॒तीत्या᳚ - हर॑ति । \newline
19. दे॒व॒दू॒ता ने॒वैव दे॑वदू॒तान् दे॑वदू॒ता ने॒व । \newline
20. दे॒व॒दू॒तानिति॑ देव - दू॒तान् । \newline
21. ए॒व प्री॑णाति प्रीणा त्ये॒वैव प्री॑णाति । \newline
22. प्री॒णा॒ति॒ प्र॒जाप॑तिः प्र॒जाप॑तिः प्रीणाति प्रीणाति प्र॒जाप॑तिः । \newline
23. प्र॒जाप॑तिर् दे॒वेभ्यो॑ दे॒वेभ्यः॑ प्र॒जाप॑तिः प्र॒जाप॑तिर् दे॒वेभ्यः॑ । \newline
24. प्र॒जाप॑ति॒रिति॑ प्र॒जा - प॒तिः॒ । \newline
25. दे॒वेभ्यो॑ य॒ज्ञान्. य॒ज्ञान् दे॒वेभ्यो॑ दे॒वेभ्यो॑ य॒ज्ञान् । \newline
26. य॒ज्ञान् व्यादि॑श॒द् व्यादि॑शद् य॒ज्ञान्. य॒ज्ञान् व्यादि॑शत् । \newline
27. व्यादि॑श॒थ् स स व्यादि॑श॒द् व्यादि॑श॒थ् सः । \newline
28. व्यादि॑श॒दिति॑ वि - आदि॑शत् । \newline
29. स रि॑रिचा॒नो रि॑रिचा॒नः स स रि॑रिचा॒नः । \newline
30. रि॒रि॒चा॒नो॑ ऽमन्यता मन्यत रिरिचा॒नो रि॑रिचा॒नो॑ ऽमन्यत । \newline
31. अ॒म॒न्य॒त॒ स सो॑ ऽमन्यता मन्यत॒ सः । \newline
32. स ए॒त मे॒तꣳ स स ए॒तम् । \newline
33. ए॒त म॑न्वाहा॒र्य॑ मन्वाहा॒र्य॑ मे॒त मे॒त म॑न्वाहा॒र्य᳚म् । \newline
34. अ॒न्वा॒हा॒र्य॑ मभ॑क्त॒ मभ॑क्त मन्वाहा॒र्य॑ मन्वाहा॒र्य॑ मभ॑क्तम् । \newline
35. अ॒न्वा॒हा॒र्य॑मित्य॑नु - आ॒हा॒र्य᳚म् । \newline
36. अभ॑क्त मपश्य दपश्य॒ दभ॑क्त॒ मभ॑क्त मपश्यत् । \newline
37. अ॒प॒श्य॒त् तम् त म॑पश्य दपश्य॒त् तम् । \newline
38. त मा॒त्मन् ना॒त्मन् तम् त मा॒त्मन्न् । \newline
39. आ॒त्मन् न॑धत्ता धत्ता॒त्मन् ना॒त्मन् न॑धत्त । \newline
40. अ॒ध॒त्त॒ स सो॑ ऽधत्ता धत्त॒ सः । \newline
41. स वै वै स स वै । \newline
42. वा ए॒ष ए॒ष वै वा ए॒षः । \newline
43. ए॒ष प्रा॑जाप॒त्यः प्रा॑जाप॒त्य ए॒ष ए॒ष प्रा॑जाप॒त्यः । \newline
44. प्रा॒जा॒प॒त्यो यद् यत् प्रा॑जाप॒त्यः प्रा॑जाप॒त्यो यत् । \newline
45. प्रा॒जा॒प॒त्य इति॑ प्राजा - प॒त्यः । \newline
46. यद॑न्वाहा॒र्यो᳚ ऽन्वाहा॒र्यो॑ यद् यद॑न्वाहा॒र्यः॑ । \newline
47. अ॒न्वा॒हा॒र्यो॑ यस्य॒ यस्या᳚न्वाहा॒र्यो᳚ ऽन्वाहा॒र्यो॑ यस्य॑ । \newline
48. अ॒न्वा॒हा॒र्य॑ इत्य॑नु - आ॒हा॒र्यः॑ । \newline
49. यस्यै॒व मे॒वं ॅयस्य॒ यस्यै॒वम् । \newline
50. ए॒वं ॅवि॒दुषो॑ वि॒दुष॑ ए॒व मे॒वं ॅवि॒दुषः॑ । \newline
51. वि॒दुषो᳚ ऽन्वाहा॒र्यो᳚ ऽन्वाहा॒र्यो॑ वि॒दुषो॑ वि॒दुषो᳚ ऽन्वाहा॒र्यः॑ । \newline
52. अ॒न्वा॒हा॒र्य॑ आह्रि॒यत॑ आह्रि॒यते᳚ ऽन्वाहा॒र्यो᳚ ऽन्वाहा॒र्य॑ आह्रि॒यते᳚ । \newline
53. अ॒न्वा॒हा॒र्य॑ इत्य॑नु - आ॒हा॒र्यः॑ । \newline
54. आ॒ह्रि॒यते॑ सा॒क्षाथ् सा॒क्षा दा᳚ह्रि॒यत॑ आह्रि॒यते॑ सा॒क्षात् । \newline
55. आ॒ह्रि॒यत॒ इत्या᳚ - ह्रि॒यते᳚ । \newline
56. सा॒क्षा दे॒वैव सा॒क्षाथ् सा॒क्षा दे॒व । \newline
57. सा॒क्षादिति॑ स - अ॒क्षात् । \newline
58. ए॒व प्र॒जाप॑तिम् प्र॒जाप॑ति मे॒वैव प्र॒जाप॑तिम् । \newline
59. प्र॒जाप॑ति मृद्ध्नो त्यृद्ध्नोति प्र॒जाप॑तिम् प्र॒जाप॑ति मृद्ध्नोति । \newline
60. प्र॒जाप॑ति॒मिति॑ प्र॒जा - प॒ति॒म् । \newline
61. ऋ॒द्ध्नो॒ त्यप॑रिमि॒तो ऽप॑रिमित ऋद्ध्नो त्यृद्ध्नो॒ त्यप॑रिमितः । \newline
62. अप॑रिमितो नि॒रुप्यो॑ नि॒रुप्यो ऽप॑रिमि॒तो ऽप॑रिमितो नि॒रुप्यः॑ । \newline
63. अप॑रिमित॒ इत्यप॑रि - मि॒तः॒ । \newline
64. नि॒रुप्यो ऽप॑रिमि॒तो ऽप॑रिमितो नि॒रुप्यो॑ नि॒रुप्यो ऽप॑रिमितः । \newline
65. नि॒रुप्य॒ इति॑ निः - उप्यः॑ । \newline
66. अप॑रिमितः प्र॒जाप॑तिः प्र॒जाप॑ति॒ रप॑रिमि॒तो ऽप॑रिमितः प्र॒जाप॑तिः । \newline
67. अप॑रिमित॒ इत्यप॑रि - मि॒तः॒ । \newline
68. प्र॒जाप॑तिः प्र॒जाप॑तेः प्र॒जाप॑तेः प्र॒जाप॑तिः प्र॒जाप॑तिः प्र॒जाप॑तेः । \newline
69. प्र॒जाप॑ति॒रिति॑ प्र॒जा - प॒तिः॒ । \newline
70. प्र॒जाप॑ते॒ राप्त्या॒ आप्त्यै᳚ प्र॒जाप॑तेः प्र॒जाप॑ते॒ राप्त्यै᳚ । \newline
71. प्र॒जाप॑ते॒रिति॑ प्र॒जा - प॒तेः॒ । \newline

\textbf{Ghana Paata } \newline

1. अ॒न्वाह॑रति॒ तत् तद॒न्वाह॑र त्य॒न्वाह॑रति॒ तद॑न्वाहा॒र्य॑स्या न्वाहा॒र्य॑स्य॒ तद॒न्वाह॑र त्य॒न्वाह॑रति॒ तद॑न्वाहा॒र्य॑स्य । \newline
2. अ॒न्वाह॑र॒तीत्य॑नु - आह॑रति । \newline
3. तद॑न्वाहा॒र्य॑स्या न्वाहा॒र्य॑स्य॒ तत् तद॑न्वाहा॒र्य॑स्या न्वाहार्य॒त्व म॑न्वाहार्य॒त्व म॑न्वाहा॒र्य॑स्य॒ तत् तद॑न्वाहा॒र्य॑स्या न्वाहार्य॒त्वम् । \newline
4. अ॒न्वा॒हा॒र्य॑स्या न्वाहार्य॒त्व म॑न्वाहार्य॒त्व म॑न्वाहा॒र्य॑स्या न्वाहा॒र्य॑स्या न्वाहार्य॒त्वम् दे॑वदू॒ता दे॑वदू॒ता अ॑न्वाहार्य॒त्व म॑न्वाहा॒र्य॑स्या न्वाहा॒र्य॑स्या न्वाहार्य॒त्वम् दे॑वदू॒ताः । \newline
5. अ॒न्वा॒हा॒र्य॑स्येत्य॑नु - आ॒हा॒र्य॑स्य । \newline
6. अ॒न्वा॒हा॒र्य॒त्वम् दे॑वदू॒ता दे॑वदू॒ता अ॑न्वाहार्य॒त्व म॑न्वाहार्य॒त्वम् दे॑वदू॒ता वै वै दे॑वदू॒ता अ॑न्वाहार्य॒त्व म॑न्वाहार्य॒त्वम् दे॑वदू॒ता वै । \newline
7. अ॒न्वा॒हा॒र्य॒त्वमित्य॑न्वाहार्य - त्वम् । \newline
8. दे॒व॒दू॒ता वै वै दे॑वदू॒ता दे॑वदू॒ता वा ए॒त ए॒ते वै दे॑वदू॒ता दे॑वदू॒ता वा ए॒ते । \newline
9. दे॒व॒दू॒ता इति॑ देव - दू॒ताः । \newline
10. वा ए॒त ए॒ते वै वा ए॒ते यद् यदे॒ते वै वा ए॒ते यत् । \newline
11. ए॒ते यद् यदे॒त ए॒ते यदृ॒त्विज॑ ऋ॒त्विजो॒ यदे॒त ए॒ते यदृ॒त्विजः॑ । \newline
12. यदृ॒त्विज॑ ऋ॒त्विजो॒ यद् यदृ॒त्विजो॒ यद् यदृ॒त्विजो॒ यद् यदृ॒त्विजो॒ यत् । \newline
13. ऋ॒त्विजो॒ यद् यदृ॒त्विज॑ ऋ॒त्विजो॒ यद॑न्वाहा॒र्य॑ मन्वाहा॒र्यं॑ ॅयदृ॒त्विज॑ ऋ॒त्विजो॒ यद॑न्वाहा॒र्य᳚म् । \newline
14. यद॑न्वाहा॒र्य॑ मन्वाहा॒र्यं॑ ॅयद् यद॑न्वाहा॒र्य॑ मा॒हर॑ त्या॒हर॑ त्यन्वाहा॒र्यं॑ ॅयद् यद॑न्वाहा॒र्य॑ मा॒हर॑ति । \newline
15. अ॒न्वा॒हा॒र्य॑ मा॒हर॑ त्या॒हर॑ त्यन्वाहा॒र्य॑ मन्वाहा॒र्य॑ मा॒हर॑ति देवदू॒तान् दे॑वदू॒ता ना॒हर॑ त्यन्वाहा॒र्य॑ मन्वाहा॒र्य॑ मा॒हर॑ति देवदू॒तान् । \newline
16. अ॒न्वा॒हा॒र्य॑मित्य॑नु - आ॒हा॒र्य᳚म् । \newline
17. आ॒हर॑ति देवदू॒तान् दे॑वदू॒ता ना॒हर॑ त्या॒हर॑ति देवदू॒ता ने॒वैव दे॑वदू॒ता ना॒हर॑ त्या॒हर॑ति देवदू॒ता ने॒व । \newline
18. आ॒हर॒तीत्या᳚ - हर॑ति । \newline
19. दे॒व॒दू॒ता ने॒वैव दे॑वदू॒तान् दे॑वदू॒ता ने॒व प्री॑णाति प्रीणात्ये॒व दे॑वदू॒तान् दे॑वदू॒ता ने॒व प्री॑णाति । \newline
20. दे॒व॒दू॒तानिति॑ देव - दू॒तान् । \newline
21. ए॒व प्री॑णाति प्रीणात्ये॒वैव प्री॑णाति प्र॒जाप॑तिः प्र॒जाप॑तिः प्रीणात्ये॒वैव प्री॑णाति प्र॒जाप॑तिः । \newline
22. प्री॒णा॒ति॒ प्र॒जाप॑तिः प्र॒जाप॑तिः प्रीणाति प्रीणाति प्र॒जाप॑तिर् दे॒वेभ्यो॑ दे॒वेभ्यः॑ प्र॒जाप॑तिः प्रीणाति प्रीणाति प्र॒जाप॑तिर् दे॒वेभ्यः॑ । \newline
23. प्र॒जाप॑तिर् दे॒वेभ्यो॑ दे॒वेभ्यः॑ प्र॒जाप॑तिः प्र॒जाप॑तिर् दे॒वेभ्यो॑ य॒ज्ञान्. य॒ज्ञान् दे॒वेभ्यः॑ प्र॒जाप॑तिः प्र॒जाप॑तिर् दे॒वेभ्यो॑ य॒ज्ञान् । \newline
24. प्र॒जाप॑ति॒रिति॑ प्र॒जा - प॒तिः॒ । \newline
25. दे॒वेभ्यो॑ य॒ज्ञान्. य॒ज्ञान् दे॒वेभ्यो॑ दे॒वेभ्यो॑ य॒ज्ञान् व्यादि॑श॒द् व्यादि॑शद् य॒ज्ञान् दे॒वेभ्यो॑ दे॒वेभ्यो॑ 
य॒ज्ञान् व्यादि॑शत् । \newline
26. य॒ज्ञान् व्यादि॑श॒द् व्यादि॑शद् य॒ज्ञान्. य॒ज्ञान् व्यादि॑श॒थ् स स व्यादि॑शद् य॒ज्ञान्. य॒ज्ञान् व्यादि॑श॒थ् सः । \newline
27. व्यादि॑श॒थ् स स व्यादि॑श॒द् व्यादि॑श॒थ् स रि॑रिचा॒नो रि॑रिचा॒नः स व्यादि॑श॒द् व्यादि॑श॒थ् स रि॑रिचा॒नः । \newline
28. व्यादि॑श॒दिति॑ वि - आदि॑शत् । \newline
29. स रि॑रिचा॒नो रि॑रिचा॒नः स स रि॑रिचा॒नो॑ ऽमन्यतामन्यत रिरिचा॒नः स स रि॑रिचा॒नो॑ ऽमन्यत । \newline
30. रि॒रि॒चा॒नो॑ ऽमन्यतामन्यत रिरिचा॒नो रि॑रिचा॒नो॑ ऽमन्यत॒ स सो॑ ऽमन्यत रिरिचा॒नो रि॑रिचा॒नो॑ ऽमन्यत॒ सः । \newline
31. अ॒म॒न्य॒त॒ स सो॑ ऽमन्यतामन्यत॒ स ए॒त मे॒तꣳ सो॑ ऽमन्यतामन्यत॒ स ए॒तम् । \newline
32. स ए॒त मे॒तꣳ स स ए॒त म॑न्वाहा॒र्य॑ मन्वाहा॒र्य॑ मे॒तꣳ स स ए॒त म॑न्वाहा॒र्य᳚म् । \newline
33. ए॒त म॑न्वाहा॒र्य॑ मन्वाहा॒र्य॑ मे॒त मे॒त म॑न्वाहा॒र्य॑ मभ॑क्त॒ मभ॑क्त मन्वाहा॒र्य॑ मे॒त मे॒त म॑न्वाहा॒र्य॑ मभ॑क्तम् । \newline
34. अ॒न्वा॒हा॒र्य॑ मभ॑क्त॒ मभ॑क्त मन्वाहा॒र्य॑ मन्वाहा॒र्य॑ मभ॑क्त मपश्य दपश्य॒ दभ॑क्त मन्वाहा॒र्य॑ मन्वाहा॒र्य॑ मभ॑क्त मपश्यत् । \newline
35. अ॒न्वा॒हा॒र्य॑मित्य॑नु - आ॒हा॒र्य᳚म् । \newline
36. अभ॑क्त मपश्य दपश्य॒ दभ॑क्त॒ मभ॑क्त मपश्य॒त् तम् त म॑पश्य॒ दभ॑क्त॒ मभ॑क्त मपश्य॒त् तम् । \newline
37. अ॒प॒श्य॒त् तम् त म॑पश्य दपश्य॒त् त मा॒त्मन् ना॒त्मन् त म॑पश्य दपश्य॒त् त मा॒त्मन्न् । \newline
38. त मा॒त्मन् ना॒त्मन् तम् त मा॒त्मन् न॑धत्ताधत्ता॒त्मन् तम् त मा॒त्मन् न॑धत्त । \newline
39. आ॒त्मन् न॑धत्ताधत्ता॒त्मन् ना॒त्मन् न॑धत्त॒ स सो॑ ऽधत्ता॒त्मन् ना॒त्मन् न॑धत्त॒ सः । \newline
40. अ॒ध॒त्त॒ स सो॑ ऽधत्ताधत्त॒ स वै वै सो॑ ऽधत्ताधत्त॒ स वै । \newline
41. स वै वै स स वा ए॒ष ए॒ष वै स स वा ए॒षः । \newline
42. वा ए॒ष ए॒ष वै वा ए॒ष प्रा॑जाप॒त्यः प्रा॑जाप॒त्य ए॒ष वै वा ए॒ष प्रा॑जाप॒त्यः । \newline
43. ए॒ष प्रा॑जाप॒त्यः प्रा॑जाप॒त्य ए॒ष ए॒ष प्रा॑जाप॒त्यो यद् यत् प्रा॑जाप॒त्य ए॒ष ए॒ष प्रा॑जाप॒त्यो यत् । \newline
44. प्रा॒जा॒प॒त्यो यद् यत् प्रा॑जाप॒त्यः प्रा॑जाप॒त्यो यद॑न्वाहा॒र्यो᳚ ऽन्वाहा॒र्यो॑ यत् प्रा॑जाप॒त्यः प्रा॑जाप॒त्यो यद॑न्वाहा॒र्यः॑ । \newline
45. प्रा॒जा॒प॒त्य इति॑ प्राजा - प॒त्यः । \newline
46. यद॑न्वाहा॒र्यो᳚ ऽन्वाहा॒र्यो॑ यद् यद॑न्वाहा॒र्यो॑ यस्य॒ यस्या᳚न्वाहा॒र्यो॑ यद् यद॑न्वाहा॒र्यो॑ यस्य॑ । \newline
47. अ॒न्वा॒हा॒र्यो॑ यस्य॒ यस्या᳚न्वाहा॒र्यो᳚ ऽन्वाहा॒र्यो॑ यस्यै॒व मे॒वं ॅयस्या᳚न्वाहा॒र्यो᳚ ऽन्वाहा॒र्यो॑ यस्यै॒वम् । \newline
48. अ॒न्वा॒हा॒र्य॑ इत्य॑नु - आ॒हा॒र्यः॑ । \newline
49. यस्यै॒व मे॒वं ॅयस्य॒ यस्यै॒वं ॅवि॒दुषो॑ वि॒दुष॑ ए॒वं ॅयस्य॒ यस्यै॒वं ॅवि॒दुषः॑ । \newline
50. ए॒वं ॅवि॒दुषो॑ वि॒दुष॑ ए॒व मे॒वं ॅवि॒दुषो᳚ ऽन्वाहा॒र्यो᳚ ऽन्वाहा॒र्यो॑ वि॒दुष॑ ए॒व मे॒वं ॅवि॒दुषो᳚ ऽन्वाहा॒र्यः॑ । \newline
51. वि॒दुषो᳚ ऽन्वाहा॒र्यो᳚ ऽन्वाहा॒र्यो॑ वि॒दुषो॑ वि॒दुषो᳚ ऽन्वाहा॒र्य॑ आह्रि॒यत॑ आह्रि॒यते᳚ ऽन्वाहा॒र्यो॑ वि॒दुषो॑ वि॒दुषो᳚ ऽन्वाहा॒र्य॑ आह्रि॒यते᳚ । \newline
52. अ॒न्वा॒हा॒र्य॑ आह्रि॒यत॑ आह्रि॒यते᳚ ऽन्वाहा॒र्यो᳚ ऽन्वाहा॒र्य॑ आह्रि॒यते॑ सा॒क्षाथ् सा॒क्षा दा᳚ह्रि॒यते᳚ ऽन्वाहा॒र्यो᳚ ऽन्वाहा॒र्य॑ आह्रि॒यते॑ सा॒क्षात् । \newline
53. अ॒न्वा॒हा॒र्य॑ इत्य॑नु - आ॒हा॒र्यः॑ । \newline
54. आ॒ह्रि॒यते॑ सा॒क्षाथ् सा॒क्षा दा᳚ह्रि॒यत॑ आह्रि॒यते॑ सा॒क्षादे॒वैव सा॒क्षा दा᳚ह्रि॒यत॑ आह्रि॒यते॑ सा॒क्षादे॒व । \newline
55. आ॒ह्रि॒यत॒ इत्या᳚ - ह्रि॒यते᳚ । \newline
56. सा॒क्षादे॒वैव सा॒क्षाथ् सा॒क्षादे॒व प्र॒जाप॑तिम् प्र॒जाप॑ति मे॒व सा॒क्षाथ् सा॒क्षादे॒व प्र॒जाप॑तिम् । \newline
57. सा॒क्षादिति॑ स - अ॒क्षात् । \newline
58. ए॒व प्र॒जाप॑तिम् प्र॒जाप॑ति मे॒वैव प्र॒जाप॑ति मृद्ध्नो त्यृद्ध्नोति प्र॒जाप॑ति मे॒वैव प्र॒जाप॑ति मृद्ध्नोति । \newline
59. प्र॒जाप॑ति मृद्ध्नो त्यृद्ध्नोति प्र॒जाप॑तिम् प्र॒जाप॑ति मृद्ध्नो॒ त्यप॑रिमि॒तो ऽप॑रिमित ऋद्ध्नोति प्र॒जाप॑तिम् प्र॒जाप॑ति मृद्ध्नो॒ त्यप॑रिमितः । \newline
60. प्र॒जाप॑ति॒मिति॑ प्र॒जा - प॒ति॒म् । \newline
61. ऋ॒द्ध्नो॒ त्यप॑रिमि॒तो ऽप॑रिमित ऋद्ध्नो त्यृद्ध्नो॒ त्यप॑रिमितो नि॒रुप्यो॑ नि॒रुप्यो ऽप॑रिमित ऋद्ध्नो त्यृद्ध्नो॒ त्यप॑रिमितो नि॒रुप्यः॑ । \newline
62. अप॑रिमितो नि॒रुप्यो॑ नि॒रुप्यो ऽप॑रिमि॒तो ऽप॑रिमितो नि॒रुप्यो ऽप॑रिमि॒तो ऽप॑रिमितो नि॒रुप्यो ऽप॑रिमि॒तो ऽप॑रिमितो नि॒रुप्यो ऽप॑रिमितः । \newline
63. अप॑रिमित॒ इत्यप॑रि - मि॒तः॒ । \newline
64. नि॒रुप्यो ऽप॑रिमि॒तो ऽप॑रिमितो नि॒रुप्यो॑ नि॒रुप्यो ऽप॑रिमितः प्र॒जाप॑तिः प्र॒जाप॑ति॒ रप॑रिमितो नि॒रुप्यो॑ नि॒रुप्यो ऽप॑रिमितः प्र॒जाप॑तिः । \newline
65. नि॒रुप्य॒ इति॑ निः - उप्यः॑ । \newline
66. अप॑रिमितः प्र॒जाप॑तिः प्र॒जाप॑ति॒ रप॑रिमि॒तो ऽप॑रिमितः प्र॒जाप॑तिः प्र॒जाप॑तेः प्र॒जाप॑तेः प्र॒जाप॑ति॒ रप॑रिमि॒तो ऽप॑रिमितः प्र॒जाप॑तिः प्र॒जाप॑तेः । \newline
67. अप॑रिमित॒ इत्यप॑रि - मि॒तः॒ । \newline
68. प्र॒जाप॑तिः प्र॒जाप॑तेः प्र॒जाप॑तेः प्र॒जाप॑तिः प्र॒जाप॑तिः प्र॒जाप॑ते॒ राप्त्या॒ आप्त्यै᳚ प्र॒जाप॑तेः प्र॒जाप॑तिः प्र॒जाप॑तिः प्र॒जाप॑ते॒ राप्त्यै᳚ । \newline
69. प्र॒जाप॑ति॒रिति॑ प्र॒जा - प॒तिः॒ । \newline
70. प्र॒जाप॑ते॒ राप्त्या॒ आप्त्यै᳚ प्र॒जाप॑तेः प्र॒जाप॑ते॒ राप्त्यै॑ दे॒वा दे॒वा आप्त्यै᳚ प्र॒जाप॑तेः प्र॒जाप॑ते॒ राप्त्यै॑ दे॒वाः । \newline
71. प्र॒जाप॑ते॒रिति॑ प्र॒जा - प॒तेः॒ । \newline
\pagebreak
\markright{ TS 1.7.3.3  \hfill https://www.vedavms.in \hfill}
\addcontentsline{toc}{section}{ TS 1.7.3.3 }
\section*{ TS 1.7.3.3 }

\textbf{TS 1.7.3.3 } \newline
\textbf{Samhita Paata} \newline

राप्त्यै॑ दे॒वा वै यद्-य॒ज्ञेऽकु॑र्वत॒ तदसु॑रा अकुर्वत॒ ते दे॒वा ए॒तं प्रा॑जाप॒त्य-म॑न्वाहा॒र्य॑-मपश्य॒न् तम॒न्वाह॑रन्त॒ ततो॑ दे॒वा अभ॑व॒न् परासु॑रा॒ यस्यै॒वं ॅवि॒दुषो᳚ऽन्वाहा॒र्य॑ आह्रि॒यते॒ भव॑त्या॒त्मना॒ परा᳚स्य॒ भ्रातृ॑व्यो भवति य॒ज्ञेन॒ वा इ॒ष्टी प॒क्वेन॑ पू॒र्ती यस्यै॒वं ॅवि॒दुषो᳚ऽन्वाहा॒र्य॑ आह्रि॒यते॒ स त्वे॑वेष्टा॑पू॒र्ती प्र॒जाप॑तेर्भा॒गो॑ऽसी- [ ] \newline

\textbf{Pada Paata} \newline

आप्त्यै᳚ । दे॒वाः । वै । यत् । य॒ज्ञे । अकु॑र्वत । तत् । असु॑राः । अ॒कु॒र्व॒त॒ । ते । दे॒वाः । ए॒तम् । प्रा॒जा॒प॒त्यमिति॑ प्राजा - प॒त्यम् । अ॒न्वा॒हा॒र्य॑मित्य॑नु - आ॒हा॒र्य᳚म् । अ॒प॒श्य॒न्न् । तम् । अ॒न्वाह॑र॒न्तेत्य॑नु - आह॑रन्त । ततः॑ । दे॒वाः । अभ॑वन्न् । परेति॑ । असु॑राः । यस्य॑ । ए॒वम् । वि॒दुषः॑ । अ॒न्वा॒हा॒र्य॑ इत्य॑नु - आ॒हा॒र्यः॑ । आ॒ह्रि॒यत॒ इत्या᳚ - ह्रि॒यते᳚ । भव॑ति । आ॒त्मना᳚ । परेति॑ । अ॒स्य॒ । भ्रातृ॑व्यः । भ॒व॒ति॒ । य॒ज्ञेन॑ । वै । इ॒ष्टी । प॒क्वेन॑ । पू॒र्ती । यस्य॑ । ए॒वम् । वि॒दुषः॑ । अ॒न्वा॒हा॒र्य॑ इत्य॑नु - आ॒हा॒र्यः॑ । आ॒ह्रि॒यत॒ इत्या᳚ - ह्रि॒यते᳚ । सः । तु । ए॒व । इ॒ष्टा॒पू॒र्तीती᳚ष्ट - पू॒र्ती । प्र॒जाप॑ते॒रिति॑ प्र॒जा - प॒तेः॒ । भा॒गः । अ॒सि॒ ।  \newline


\textbf{Krama Paata} \newline

आप्त्यै॑ दे॒वाः । दे॒वा वै । वै यत् । यद् य॒ज्ञे । य॒ज्ञेऽकु॑र्वत । अकु॑र्वत॒ तत् । तदसु॑राः । असु॑रा अकुर्वत । अ॒कु॒र्व॒त॒ ते । ते दे॒वाः । दे॒वा ए॒तम् । ए॒तम् प्रा॑जाप॒त्यम् । प्रा॒जा॒प॒त्यम॑न्वाहा॒र्य᳚म् । प्रा॒जा॒प॒त्यमिति॑ प्राजा - प॒त्यम् । अ॒न्वा॒हा॒र्य॑मपश्यन्न् । अ॒न्वा॒हा॒र्य॑मित्य॑नु - आ॒हा॒र्य᳚म् । अ॒प॒श्य॒न् तम् । तम॒न्वाह॑रन्त । अ॒न्वाह॑रन्त॒ ततः । अ॒न्वाह॑र॒न्तेत्य॑नु - आह॑रन्त । ततो॑ दे॒वाः । दे॒वा अभ॑वन्न् । अभ॑व॒न्,परा᳚ । परा ऽसु॑राः । असु॑रा॒ यस्य॑ । यस्यै॒वम् । ए॒वं ॅवि॒दुषः॑ । वि॒दुषो᳚ ऽन्वाहा॒र्यः॑ । अ॒न्वा॒हा॒र्य॑ आह्रि॒यते᳚ । अ॒न्वा॒हा॒र्य॑ इत्य॑नु - आ॒हा॒र्यः॑ । आ॒ह्रि॒यते॒ भव॑ति । आ॒ह्रि॒यत॒ इत्या᳚ - ह्रि॒यते᳚ । भव॑त्या॒त्मना᳚ । आ॒त्मना॒ परा᳚ । परा᳚ऽस्य । अ॒स्य॒ भ्रातृ॑व्यः । भ्रातृ॑व्यो भवति । भ॒व॒ति॒ य॒ज्ञेन॑ । य॒ज्ञेन॒ वै । वा इ॒ष्टी । इ॒ष्टी प॒क्वेन॑ । प॒क्वेन॑ पू॒र्ती । पू॒र्ती यस्य॑ । यस्यै॒वम् । ए॒वं ॅवि॒दुषः॑ । वि॒दुषो᳚ ऽन्वाहा॒र्यः॑ । अ॒न्वा॒हा॒र्य॑ आह्रि॒यते᳚ । अ॒न्वा॒हा॒र्य॑ इत्य॑नु - आ॒हा॒र्यः॑ । आ॒ह्रि॒यते॒ सः । आ॒ह्रि॒यत॒ इत्या᳚ - ह्रि॒यते᳚ । स तु । त्वे॑व । ए॒वेष्टा॑पू॒र्ती । इ॒ष्टा॒पू॒र्ती प्र॒जाप॑तेः । इ॒ष्टा॒पू॒र्तीती᳚ष्ट - पू॒र्ती । प्र॒जाप॑तेर्,भा॒गः । प्र॒जाप॑ते॒रिति॑ प्र॒जा - प॒तेः॒ । भा॒गो॑ऽसि । अ॒सीति॑ \newline

\textbf{Jatai Paata} \newline

1. आप्त्यै॑ दे॒वा दे॒वा आप्त्या॒ आप्त्यै॑ दे॒वाः । \newline
2. दे॒वा वै वै दे॒वा दे॒वा वै । \newline
3. वै यद् यद् वै वै यत् । \newline
4. यद् य॒ज्ञे य॒ज्ञे यद् यद् य॒ज्ञे । \newline
5. य॒ज्ञे ऽकु॑र्व॒ता कु॑र्वत य॒ज्ञे य॒ज्ञे ऽकु॑र्वत । \newline
6. अकु॑र्वत॒ तत् तदकु॑र्व॒ता कु॑र्वत॒ तत् । \newline
7. तदसु॑रा॒ असु॑रा॒ स्तत् तदसु॑राः । \newline
8. असु॑रा अकुर्वता कुर्व॒ता सु॑रा॒ असु॑रा अकुर्वत । \newline
9. अ॒कु॒र्व॒त॒ ते ते॑ ऽकुर्वता कुर्वत॒ ते । \newline
10. ते दे॒वा दे॒वा स्ते ते दे॒वाः । \newline
11. दे॒वा ए॒त मे॒तम् दे॒वा दे॒वा ए॒तम् । \newline
12. ए॒तम् प्रा॑जाप॒त्यम् प्रा॑जाप॒त्य मे॒त मे॒तम् प्रा॑जाप॒त्यम् । \newline
13. प्रा॒जा॒प॒त्य म॑न्वाहा॒र्य॑ मन्वाहा॒र्य॑म् प्राजाप॒त्यम् प्रा॑जाप॒त्य म॑न्वाहा॒र्य᳚म् । \newline
14. प्रा॒जा॒प॒त्यमिति॑ प्राजा - प॒त्यम् । \newline
15. अ॒न्वा॒हा॒र्य॑ मपश्यन् नपश्यन् नन्वाहा॒र्य॑ मन्वाहा॒र्य॑ मपश्यन्न् । \newline
16. अ॒न्वा॒हा॒र्य॑मित्य॑नु - आ॒हा॒र्य᳚म् । \newline
17. अ॒प॒श्य॒न् तम् त म॑पश्यन् नपश्य॒न् तम् । \newline
18. त म॒न्वाह॑रन्ता॒ न्वाह॑रन्त॒ तम् त म॒न्वाह॑रन्त । \newline
19. अ॒न्वाह॑रन्त॒ तत॒स्ततो॒ ऽन्वाह॑रन्ता॒ न्वाह॑रन्त॒ ततः॑ । \newline
20. अ॒न्वाह॑र॒न्तेत्य॑नु - आह॑रन्त । \newline
21. ततो॑ दे॒वा दे॒वा स्तत॒ स्ततो॑ दे॒वाः । \newline
22. दे॒वा अभ॑व॒न् नभ॑वन् दे॒वा दे॒वा अभ॑वन्न् । \newline
23. अभ॑व॒न् परा॒ परा ऽभ॑व॒न् नभ॑व॒न् परा᳚ । \newline
24. परा ऽसु॑रा॒ असु॑राः॒ परा॒ परा ऽसु॑राः । \newline
25. असु॑रा॒ यस्य॒ यस्या सु॑रा॒ असु॑रा॒ यस्य॑ । \newline
26. यस्यै॒व मे॒वं ॅयस्य॒ यस्यै॒वम् । \newline
27. ए॒वं ॅवि॒दुषो॑ वि॒दुष॑ ए॒व मे॒वं ॅवि॒दुषः॑ । \newline
28. वि॒दुषो᳚ ऽन्वाहा॒र्यो᳚ ऽन्वाहा॒र्यो॑ वि॒दुषो॑ वि॒दुषो᳚ ऽन्वाहा॒र्यः॑ । \newline
29. अ॒न्वा॒हा॒र्य॑ आह्रि॒यत॑ आह्रि॒यते᳚ ऽन्वाहा॒र्यो᳚ ऽन्वाहा॒र्य॑ आह्रि॒यते᳚ । \newline
30. अ॒न्वा॒हा॒र्य॑ इत्य॑नु - आ॒हा॒र्यः॑ । \newline
31. आ॒ह्रि॒यते॒ भव॑ति॒ भव॑त्या ह्रि॒यत॑ आह्रि॒यते॒ भव॑ति । \newline
32. आ॒ह्रि॒यत॒ इत्या᳚ - ह्रि॒यते᳚ । \newline
33. भव॑त्या॒ त्मना॒ ऽऽत्मना॒ भव॑ति॒ भव॑त्या॒ त्मना᳚ । \newline
34. आ॒त्मना॒ परा॒ परा॒ ऽऽत्मना॒ ऽऽत्मना॒ परा᳚ । \newline
35. परा᳚ ऽस्यास्य॒ परा॒ परा᳚ ऽस्य । \newline
36. अ॒स्य॒ भ्रातृ॑व्यो॒ भ्रातृ॑व्यो ऽस्यास्य॒ भ्रातृ॑व्यः । \newline
37. भ्रातृ॑व्यो भवति भवति॒ भ्रातृ॑व्यो॒ भ्रातृ॑व्यो भवति । \newline
38. भ॒व॒ति॒ य॒ज्ञेन॑ य॒ज्ञेन॑ भवति भवति य॒ज्ञेन॑ । \newline
39. य॒ज्ञेन॒ वै वै य॒ज्ञेन॑ य॒ज्ञेन॒ वै । \newline
40. वा इ॒ष्टीष्टी वै वा इ॒ष्टी । \newline
41. इ॒ष्टी प॒क्वेन॑ प॒क्वेने॒ ष्टीष्टी प॒क्वेन॑ । \newline
42. प॒क्वेन॑ पू॒र्ती पू॒र्ती प॒क्वेन॑ प॒क्वेन॑ पू॒र्ती । \newline
43. पू॒र्ती यस्य॒ यस्य॑ पू॒र्ती पू॒र्ती यस्य॑ । \newline
44. यस्यै॒व मे॒वं ॅयस्य॒ यस्यै॒वम् । \newline
45. ए॒वं ॅवि॒दुषो॑ वि॒दुष॑ ए॒व मे॒वं ॅवि॒दुषः॑ । \newline
46. वि॒दुषो᳚ ऽन्वाहा॒र्यो᳚ ऽन्वाहा॒र्यो॑ वि॒दुषो॑ वि॒दुषो᳚ ऽन्वाहा॒र्यः॑ । \newline
47. अ॒न्वा॒हा॒र्य॑ आह्रि॒यत॑ आह्रि॒यते᳚ ऽन्वाहा॒र्यो᳚ ऽन्वाहा॒र्य॑ आह्रि॒यते᳚ । \newline
48. अ॒न्वा॒हा॒र्य॑ इत्य॑नु - आ॒हा॒र्यः॑ । \newline
49. आ॒ह्रि॒यते॒ स स आ᳚ह्रि॒यत॑ आह्रि॒यते॒ सः । \newline
50. आ॒ह्रि॒यत॒ इत्या᳚ - ह्रि॒यते᳚ । \newline
51. स तु तु स स तु । \newline
52. त्वे॑वैव तु त्वे॑व । \newline
53. ए॒वे ष्टा॑पू॒र्ती ष्टा॑पू॒र् त्ये॑वैवे ष्टा॑पू॒र्ती । \newline
54. इ॒ष्टा॒पू॒र्ती प्र॒जाप॑तेः प्र॒जाप॑ते रिष्टापू॒र्ती ष्टा॑पू॒र्ती प्र॒जाप॑तेः । \newline
55. इ॒ष्टा॒पू॒र्तीती᳚ष्ट - पू॒र्ती । \newline
56. प्र॒जाप॑तेर् भा॒गो भा॒गः प्र॒जाप॑तेः प्र॒जाप॑तेर् भा॒गः । \newline
57. प्र॒जाप॑ते॒रिति॑ प्र॒जा - प॒तेः॒ । \newline
58. भा॒गो᳚ ऽस्यसि भा॒गो भा॒गो॑ ऽसि । \newline
59. अ॒सी ती त्य॑स्य॒सी ति॑ । \newline

\textbf{Ghana Paata } \newline

1. आप्त्यै॑ दे॒वा दे॒वा आप्त्या॒ आप्त्यै॑ दे॒वा वै वै दे॒वा आप्त्या॒ आप्त्यै॑ दे॒वा वै । \newline
2. दे॒वा वै वै दे॒वा दे॒वा वै यद् यद् वै दे॒वा दे॒वा वै यत् । \newline
3. वै यद् यद् वै वै यद् य॒ज्ञे य॒ज्ञे यद् वै वै यद् य॒ज्ञे । \newline
4. यद् य॒ज्ञे य॒ज्ञे यद् यद् य॒ज्ञे ऽकु॑र्व॒ताकु॑र्वत य॒ज्ञे यद् यद् य॒ज्ञे ऽकु॑र्वत । \newline
5. य॒ज्ञे ऽकु॑र्व॒ताकु॑र्वत य॒ज्ञे य॒ज्ञे ऽकु॑र्वत॒ तत् तदकु॑र्वत य॒ज्ञे य॒ज्ञे ऽकु॑र्वत॒ तत् । \newline
6. अकु॑र्वत॒ तत् तदकु॑र्व॒ता कु॑र्वत॒ तदसु॑रा॒ असु॑रा॒ स्तदकु॑र्व॒ता कु॑र्वत॒ तदसु॑राः । \newline
7. तदसु॑रा॒ असु॑रा॒ स्तत् तदसु॑रा अकुर्वता कुर्व॒तासु॑रा॒ स्तत् तदसु॑रा अकुर्वत । \newline
8. असु॑रा अकुर्वता कुर्व॒तासु॑रा॒ असु॑रा अकुर्वत॒ ते ते॑ ऽकुर्व॒तासु॑रा॒ असु॑रा अकुर्वत॒ ते । \newline
9. अ॒कु॒र्व॒त॒ ते ते॑ ऽकुर्वताकुर्वत॒ ते दे॒वा दे॒वास्ते॑ ऽकुर्वताकुर्वत॒ ते दे॒वाः । \newline
10. ते दे॒वा दे॒वास्ते ते दे॒वा ए॒त मे॒तम् दे॒वास्ते ते दे॒वा ए॒तम् । \newline
11. दे॒वा ए॒त मे॒तम् दे॒वा दे॒वा ए॒तम् प्रा॑जाप॒त्यम् प्रा॑जाप॒त्य मे॒तम् दे॒वा दे॒वा ए॒तम् प्रा॑जाप॒त्यम् । \newline
12. ए॒तम् प्रा॑जाप॒त्यम् प्रा॑जाप॒त्य मे॒त मे॒तम् प्रा॑जाप॒त्य म॑न्वाहा॒र्य॑ मन्वाहा॒र्य॑म् प्राजाप॒त्य मे॒त मे॒तम् प्रा॑जाप॒त्य म॑न्वाहा॒र्य᳚म् । \newline
13. प्रा॒जा॒प॒त्य म॑न्वाहा॒र्य॑ मन्वाहा॒र्य॑म् प्राजाप॒त्यम् प्रा॑जाप॒त्य म॑न्वाहा॒र्य॑ मपश्यन् नपश्यन् नन्वाहा॒र्य॑म् प्राजाप॒त्यम् प्रा॑जाप॒त्य म॑न्वाहा॒र्य॑ मपश्यन्न् । \newline
14. प्रा॒जा॒प॒त्यमिति॑ प्राजा - प॒त्यम् । \newline
15. अ॒न्वा॒हा॒र्य॑ मपश्यन् नपश्यन् नन्वाहा॒र्य॑ मन्वाहा॒र्य॑ मपश्य॒न् तम् त म॑पश्यन् नन्वाहा॒र्य॑ मन्वाहा॒र्य॑ मपश्य॒न् तम् । \newline
16. अ॒न्वा॒हा॒र्य॑मित्य॑नु - आ॒हा॒र्य᳚म् । \newline
17. अ॒प॒श्य॒न् तम् त म॑पश्यन् नपश्य॒न् त म॒न्वाह॑रन्ता॒ न्वाह॑रन्त॒ त म॑पश्यन् नपश्य॒न् त म॒न्वाह॑रन्त । \newline
18. त म॒न्वाह॑रन्ता॒ न्वाह॑रन्त॒ तम् त म॒न्वाह॑रन्त॒ तत॒स्ततो॒ ऽन्वाह॑रन्त॒ तम् त म॒न्वाह॑रन्त॒ ततः॑ । \newline
19. अ॒न्वाह॑रन्त॒ तत॒ स्ततो॒ ऽन्वाह॑रन्ता॒ न्वाह॑रन्त॒ ततो॑ दे॒वा दे॒वा स्ततो॒ ऽन्वाह॑रन्ता॒ न्वाह॑रन्त॒ ततो॑ दे॒वाः । \newline
20. अ॒न्वाह॑र॒न्तेत्य॑नु - आह॑रन्त । \newline
21. ततो॑ दे॒वा दे॒वा स्तत॒ स्ततो॑ दे॒वा अभ॑व॒न् नभ॑वन् दे॒वा स्तत॒ स्ततो॑ दे॒वा अभ॑वन्न् । \newline
22. दे॒वा अभ॑व॒न् नभ॑वन् दे॒वा दे॒वा अभ॑व॒न् परा॒ परा ऽभ॑वन् दे॒वा दे॒वा अभ॑व॒न् परा᳚ । \newline
23. अभ॑व॒न् परा॒ परा ऽभ॑व॒न् नभ॑व॒न् परा ऽसु॑रा॒ असु॑राः॒ परा ऽभ॑व॒न् नभ॑व॒न् परा ऽसु॑राः । \newline
24. परा ऽसु॑रा॒ असु॑राः॒ परा॒ परा ऽसु॑रा॒ यस्य॒ यस्यासु॑राः॒ परा॒ परा ऽसु॑रा॒ यस्य॑ । \newline
25. असु॑रा॒ यस्य॒ यस्यासु॑रा॒ असु॑रा॒ यस्यै॒व मे॒वं ॅयस्यासु॑रा॒ असु॑रा॒ यस्यै॒वम् । \newline
26. यस्यै॒व मे॒वं ॅयस्य॒ यस्यै॒वं ॅवि॒दुषो॑ वि॒दुष॑ ए॒वं ॅयस्य॒ यस्यै॒वं ॅवि॒दुषः॑ । \newline
27. ए॒वं ॅवि॒दुषो॑ वि॒दुष॑ ए॒व मे॒वं ॅवि॒दुषो᳚ ऽन्वाहा॒र्यो᳚ ऽन्वाहा॒र्यो॑ वि॒दुष॑ ए॒व मे॒वं ॅवि॒दुषो᳚ ऽन्वाहा॒र्यः॑ । \newline
28. वि॒दुषो᳚ ऽन्वाहा॒र्यो᳚ ऽन्वाहा॒र्यो॑ वि॒दुषो॑ वि॒दुषो᳚ ऽन्वाहा॒र्य॑ आह्रि॒यत॑ आह्रि॒यते᳚ ऽन्वाहा॒र्यो॑ वि॒दुषो॑ वि॒दुषो᳚ ऽन्वाहा॒र्य॑ आह्रि॒यते᳚ । \newline
29. अ॒न्वा॒हा॒र्य॑ आह्रि॒यत॑ आह्रि॒यते᳚ ऽन्वाहा॒र्यो᳚ ऽन्वाहा॒र्य॑ आह्रि॒यते॒ भव॑ति॒ भव॑त्याह्रि॒यते᳚ ऽन्वाहा॒र्यो᳚ ऽन्वाहा॒र्य॑ आह्रि॒यते॒ भव॑ति । \newline
30. अ॒न्वा॒हा॒र्य॑ इत्य॑नु - आ॒हा॒र्यः॑ । \newline
31. आ॒ह्रि॒यते॒ भव॑ति॒ भव॑त्याह्रि॒यत॑ आह्रि॒यते॒ भव॑त्या॒त्मना॒ ऽऽत्मना॒ भव॑त्याह्रि॒यत॑ आह्रि॒यते॒ भव॑त्या॒त्मना᳚ । \newline
32. आ॒ह्रि॒यत॒ इत्या᳚ - ह्रि॒यते᳚ । \newline
33. भव॑त्या॒त्मना॒ ऽऽत्मना॒ भव॑ति॒ भव॑त्या॒त्मना॒ परा॒ परा॒ ऽऽत्मना॒ भव॑ति॒ भव॑त्या॒त्मना॒ परा᳚ । \newline
34. आ॒त्मना॒ परा॒ परा॒ ऽऽत्मना॒ ऽऽत्मना॒ परा᳚ ऽस्यास्य॒ परा॒ ऽऽत्मना॒ ऽऽत्मना॒ परा᳚ ऽस्य । \newline
35. परा᳚ ऽस्यास्य॒ परा॒ परा᳚ ऽस्य॒ भ्रातृ॑व्यो॒ भ्रातृ॑व्यो ऽस्य॒ परा॒ परा᳚ ऽस्य॒ भ्रातृ॑व्यः । \newline
36. अ॒स्य॒ भ्रातृ॑व्यो॒ भ्रातृ॑व्यो ऽस्यास्य॒ भ्रातृ॑व्यो भवति भवति॒ भ्रातृ॑व्यो ऽस्यास्य॒ भ्रातृ॑व्यो भवति । \newline
37. भ्रातृ॑व्यो भवति भवति॒ भ्रातृ॑व्यो॒ भ्रातृ॑व्यो भवति य॒ज्ञेन॑ य॒ज्ञेन॑ भवति॒ भ्रातृ॑व्यो॒ भ्रातृ॑व्यो भवति य॒ज्ञेन॑ । \newline
38. भ॒व॒ति॒ य॒ज्ञेन॑ य॒ज्ञेन॑ भवति भवति य॒ज्ञेन॒ वै वै य॒ज्ञेन॑ भवति भवति य॒ज्ञेन॒ वै । \newline
39. य॒ज्ञेन॒ वै वै य॒ज्ञेन॑ य॒ज्ञेन॒ वा इ॒ष्टीष्टी वै य॒ज्ञेन॑ य॒ज्ञेन॒ वा इ॒ष्टी । \newline
40. वा इ॒ष्टीष्टी वै वा इ॒ष्टी प॒क्वेन॑ प॒क्वेने॒ ष्टी वै वा इ॒ष्टी प॒क्वेन॑ । \newline
41. इ॒ष्टी प॒क्वेन॑ प॒क्वेने॒ ष्टीष्टी प॒क्वेन॑ पू॒र्ती पू॒र्ती प॒क्वेने॒ ष्टीष्टी प॒क्वेन॑ पू॒र्ती । \newline
42. प॒क्वेन॑ पू॒र्ती पू॒र्ती प॒क्वेन॑ प॒क्वेन॑ पू॒र्ती यस्य॒ यस्य॑ पू॒र्ती प॒क्वेन॑ प॒क्वेन॑ पू॒र्ती यस्य॑ । \newline
43. पू॒र्ती यस्य॒ यस्य॑ पू॒र्ती पू॒र्ती यस्यै॒व मे॒वं ॅयस्य॑ पू॒र्ती पू॒र्ती यस्यै॒वम् । \newline
44. यस्यै॒व मे॒वं ॅयस्य॒ यस्यै॒वं ॅवि॒दुषो॑ वि॒दुष॑ ए॒वं ॅयस्य॒ यस्यै॒वं ॅवि॒दुषः॑ । \newline
45. ए॒वं ॅवि॒दुषो॑ वि॒दुष॑ ए॒व मे॒वं ॅवि॒दुषो᳚ ऽन्वाहा॒र्यो᳚ ऽन्वाहा॒र्यो॑ वि॒दुष॑ ए॒व मे॒वं ॅवि॒दुषो᳚ ऽन्वाहा॒र्यः॑ । \newline
46. वि॒दुषो᳚ ऽन्वाहा॒र्यो᳚ ऽन्वाहा॒र्यो॑ वि॒दुषो॑ वि॒दुषो᳚ ऽन्वाहा॒र्य॑ आह्रि॒यत॑ आह्रि॒यते᳚ ऽन्वाहा॒र्यो॑ वि॒दुषो॑ वि॒दुषो᳚ ऽन्वाहा॒र्य॑ आह्रि॒यते᳚ । \newline
47. अ॒न्वा॒हा॒र्य॑ आह्रि॒यत॑ आह्रि॒यते᳚ ऽन्वाहा॒र्यो᳚ ऽन्वाहा॒र्य॑ आह्रि॒यते॒ स स आ᳚ह्रि॒यते᳚ ऽन्वाहा॒र्यो᳚ ऽन्वाहा॒र्य॑ आह्रि॒यते॒ सः । \newline
48. अ॒न्वा॒हा॒र्य॑ इत्य॑नु - आ॒हा॒र्यः॑ । \newline
49. आ॒ह्रि॒यते॒ स स आ᳚ह्रि॒यत॑ आह्रि॒यते॒ स तु तु स आ᳚ह्रि॒यत॑ आह्रि॒यते॒ स तु । \newline
50. आ॒ह्रि॒यत॒ इत्या᳚ - ह्रि॒यते᳚ । \newline
51. स तु तु स स त्वे॑वैव तु स स त्वे॑व । \newline
52. त्वे॑वैव तु त्वे॑वे ष्टा॑पू॒र्ती ष्टा॑पू॒र्त्ये॑व तु त्वे॑वे ष्टा॑पू॒र्ती । \newline
53. ए॒वे ष्टा॑पू॒र्ती ष्टा॑पू॒र्त्ये॑ वैवे ष्टा॑पू॒र्ती प्र॒जाप॑तेः प्र॒जाप॑ते रिष्टापू॒र् त्ये॑वैवे ष्टा॑पू॒र्ती प्र॒जाप॑तेः । \newline
54. इ॒ष्टा॒पू॒र्ती प्र॒जाप॑तेः प्र॒जाप॑ते रिष्टापू॒र्ती ष्टा॑पू॒र्ती प्र॒जाप॑तेर् भा॒गो भा॒गः प्र॒जाप॑ते रिष्टापू॒र्ती ष्टा॑पू॒र्ती प्र॒जाप॑तेर् भा॒गः । \newline
55. इ॒ष्टा॒पू॒र्तीती᳚ष्ट - पू॒र्ती । \newline
56. प्र॒जाप॑तेर् भा॒गो भा॒गः प्र॒जाप॑तेः प्र॒जाप॑तेर् भा॒गो᳚ ऽस्यसि भा॒गः प्र॒जाप॑तेः प्र॒जाप॑तेर् भा॒गो॑ ऽसि । \newline
57. प्र॒जाप॑ते॒रिति॑ प्र॒जा - प॒तेः॒ । \newline
58. भा॒गो᳚ ऽस्यसि भा॒गो भा॒गो॑ ऽसीतीत्य॑सि भा॒गो भा॒गो॑ ऽसीति॑ । \newline
59. अ॒सीती त्य॑स्य॒सी त्या॑हा॒हे त्य॑स्य॒सी त्या॑ह । \newline
\pagebreak
\markright{ TS 1.7.3.4  \hfill https://www.vedavms.in \hfill}
\addcontentsline{toc}{section}{ TS 1.7.3.4 }
\section*{ TS 1.7.3.4 }

\textbf{TS 1.7.3.4 } \newline
\textbf{Samhita Paata} \newline

त्या॑ह प्र॒जाप॑तिमे॒व भा॑ग॒धेये॑न॒ सम॑र्द्धय॒त्यूर्ज॑स्वा॒न् पय॑स्वा॒नित्या॒होर्ज॑-मे॒वास्मि॒न् पयो॑ दधाति प्राणापा॒नौ मे॑ पाहि समानव्या॒नौ मे॑ पा॒हीत्या॑हा॒ऽऽशिष॑मे॒वैतामा शा॒स्ते ऽक्षि॑तो॒ ऽस्यक्षि॑त्यै त्वा॒ मा मे᳚ क्षेष्ठा अ॒मुत्रा॒मुष्मि॑न् ॅलो॒क इत्या॑ह॒ क्षीय॑ते॒ वा अ॒मुष्मि॑न् ॅलो॒केऽन्न॑-मि॒तःप्र॑दानꣳ॒॒ ह्य॑मुष्मिन् ॅलो॒के ( ) प्र॒जा उ॑प॒जीव॑न्ति॒ यदे॒व-म॑भिमृ॒शत्यक्षि॑ति-मे॒वैन॑द् गमयति॒ नास्या॒मुष्मि॑न् ॅलो॒केऽन्नं॑ क्षीयते ॥ \newline

\textbf{Pada Paata} \newline

इति॑ । आ॒ह॒ । प्र॒जाप॑ति॒मिति॑ प्र॒जा - प॒ति॒म् । ए॒व । भा॒ग॒धेये॒नेति॑ भाग - धेये॑न । समिति॑ । अ॒द्‌र्ध॒य॒ति॒ । ऊर्ज॑स्वान् । पय॑स्वान् । इति॑ । आ॒ह॒ । ऊर्ज᳚म् । ए॒व । अ॒स्मि॒न्न् । पयः॑ । द॒धा॒ति॒ । प्रा॒णा॒पा॒नाविति॑ प्राण -अ॒पा॒नौ । मे॒ । पा॒हि॒ । स॒मा॒न॒व्या॒नाविति॑ समान - व्या॒नौ । मे॒ । पा॒हि॒ । इति॑ । आ॒ह॒ । आ॒शिष॒मित्या᳚ - शिष᳚म् । ए॒व । ए॒ताम् । एति॑ । शा॒स्ते॒ । अक्षि॑तः । अ॒सि॒ । अक्षि॑त्यै । त्वा॒ । मा । मे॒ । क्षे॒ष्ठाः॒ । अ॒मुत्र॑ । अ॒मुष्मिन्न्॑ । लो॒के । इति॑ । आ॒ह॒ । क्षीय॑ते । वै । अ॒मुष्मिन्न्॑ । लो॒के । अन्न᳚म् । इ॒तः प्र॑दान॒मिती॒तः - प्र॒दा॒न॒म् । हि । अ॒मुष्मिन्न्॑ । लो॒के ( ) । प्र॒जा इति॑ प्र - जाः । उ॒प॒जीव॒न्तीत्यु॑प - जीव॑न्ति । यत् । ए॒वम् । अ॒भि॒मृ॒शतीत्य॑भि - मृ॒शति॑ । अक्षि॑तिम् । ए॒व । ए॒न॒त् । ग॒म॒य॒ति॒ । न । अ॒स्य॒ । अ॒मुष्मिन्न्॑ । लो॒के । अन्न᳚म् । क्षी॒य॒ते॒ ॥  \newline


\textbf{Krama Paata} \newline

इत्या॑ह । आ॒ह॒ प्र॒जाप॑तिम् । प्र॒जाप॑तिमे॒व । प्र॒जाप॑ति॒मिति॑ प्र॒जा - प॒ति॒म् । ए॒व भा॑ग॒धेये॑न । भा॒ग॒धेये॑न॒ सम् । भा॒ग॒धेये॒नेति॑ भाग - धेये॑न । सम॑र्द्धयति । अ॒र्द्ध॒य॒त्यूर्ज॑स्वान् । ऊर्ज॑स्वा॒न् पय॑स्वान् । पय॑स्वा॒निति॑ । इत्या॑ह । आ॒होर्ज᳚म् । ऊर्ज॑मे॒व । ए॒वास्मिन्न्॑ । अ॒स्मि॒न्,पयः॑ । पयो॑ दधाति । द॒धा॒ति॒ प्रा॒णा॒पा॒नौ । प्रा॒णा॒पा॒नौ मे᳚ । प्रा॒णा॒पा॒नाविति॑ प्राण - अ॒पा॒नौ । मे॒ पा॒हि॒ । पा॒हि॒ स॒मा॒न॒व्या॒नौ । स॒मा॒न॒व्या॒नौ मे᳚ । स॒मा॒न॒व्या॒नाविति॑ समान - व्या॒नौ । मे॒ पा॒हि॒ । पा॒हीति॑ । इत्या॑ह । आ॒हा॒शिष᳚म् । आ॒शिष॑मे॒व । आ॒शिष॒मित्या᳚ - शिष᳚म् । ए॒वैताम् । ए॒तामा । आ शा᳚स्ते । शा॒स्तेऽक्षि॑तः । अक्षि॑तोऽसि । अ॒स्यक्षि॑त्यै । अक्षि॑त्यै त्वा । त्वा॒ मा । मा मे᳚ । मे॒ क्षे॒ष्ठाः॒ । क्षे॒ष्ठा॒ अ॒मुत्र॑ । अ॒मुत्रा॒मुष्मिन्न्॑ । अ॒मुष्मिन्॑ ॅलो॒के । लो॒क इति॑ । इत्या॑ह । आ॒ह॒ क्षीय॑ते । क्षीय॑ते॒ वै । वा अ॒मुष्मिन्न्॑ । अ॒मुष्मि॑न् ॅलो॒के । लो॒केऽन्न᳚म् । अन्न॑मि॒तःप्र॑दानम् । इ॒तःप्र॑दानꣳ॒॒ हि । इ॒तःप्र॑दान॒मिती॒तः - प्र॒दा॒न॒म् । ह्य॑मुष्मिन्न्॑ । अ॒मुष्मि॑न् ॅलो॒के ( ) । लो॒के प्र॒जाः । प्र॒जा उ॑प॒जीव॑न्ति । प्र॒जा इति॑ प्र - जाः । उ॒प॒जीव॑न्ति॒ यत् । उ॒प॒जीव॒न्तीत्यु॑प - जीव॑न्ति । यदे॒वम् । ए॒वम॑भिमृ॒शति॑ । अ॒भि॒मृ॒शत्यक्षि॑तिम् । अ॒भि॒मृ॒शतीत्य॑भि - मृ॒शति॑ । अक्षि॑तिमे॒व । ए॒वैन॑त् । ए॒न॒द् ग॒म॒य॒ति॒ । ग॒म॒य॒ति॒ न । नास्य॑ । अ॒स्या॒मुष्मिन्न्॑ । अ॒मुष्मि॑न् लो॒के । लो॒केऽन्न᳚म् । अन्न॑म् क्षीयते । क्षी॒य॒त॒ इति॑ क्षीयते । \newline

\textbf{Jatai Paata} \newline

1. इत्या॑हा॒हे तीत्या॑ह । \newline
2. आ॒ह॒ प्र॒जाप॑तिम् प्र॒जाप॑ति माहाह प्र॒जाप॑तिम् । \newline
3. प्र॒जाप॑ति मे॒वैव प्र॒जाप॑तिम् प्र॒जाप॑ति मे॒व । \newline
4. प्र॒जाप॑ति॒मिति॑ प्र॒जा - प॒ति॒म् । \newline
5. ए॒व भा॑ग॒धेये॑न भाग॒धेये॑ नै॒वैव भा॑ग॒धेये॑न । \newline
6. भा॒ग॒धेये॑न॒ सꣳ सम् भा॑ग॒धेये॑न भाग॒धेये॑न॒ सम् । \newline
7. भा॒ग॒धेये॒नेति॑ भाग - धेये॑न । \newline
8. स म॑र्द्धय त्यर्द्धयति॒ सꣳ स म॑र्द्धयति । \newline
9. अ॒र्द्ध॒य॒ त्यूर्ज॑स्वा॒ नूर्ज॑स्वा नर्द्धय त्यर्द्धय॒ त्यूर्ज॑स्वान् । \newline
10. ऊर्ज॑स्वा॒न् पय॑स्वा॒न् पय॑स्वा॒ नूर्ज॑स्वा॒ नूर्ज॑स्वा॒न् पय॑स्वान् । \newline
11. पय॑स्वा॒ नितीति॒ पय॑स्वा॒न् पय॑स्वा॒ निति॑ । \newline
12. इत्या॑हा॒हे तीत्या॑ह । \newline
13. आ॒होर्ज॒ मूर्ज॑ माहा॒ होर्ज᳚म् । \newline
14. ऊर्ज॑ मे॒वै वोर्ज॒ मूर्ज॑ मे॒व । \newline
15. ए॒वास्मि॑न् नस्मिन् ने॒वैवास्मिन्न्॑ । \newline
16. अ॒स्मि॒न् पयः॒ पयो᳚ ऽस्मिन् नस्मि॒न् पयः॑ । \newline
17. पयो॑ दधाति दधाति॒ पयः॒ पयो॑ दधाति । \newline
18. द॒धा॒ति॒ प्रा॒णा॒पा॒नौ प्रा॑णापा॒नौ द॑धाति दधाति प्राणापा॒नौ । \newline
19. प्रा॒णा॒पा॒नौ मे॑ मे प्राणापा॒नौ प्रा॑णापा॒नौ मे᳚ । \newline
20. प्रा॒णा॒पा॒नाविति॑ प्राण - अ॒पा॒नौ । \newline
21. मे॒ पा॒हि॒ पा॒हि॒ मे॒ मे॒ पा॒हि॒ । \newline
22. पा॒हि॒ स॒मा॒न॒व्या॒नौ स॑मानव्या॒नौ पा॑हि पाहि समानव्या॒नौ । \newline
23. स॒मा॒न॒व्या॒नौ मे॑ मे समानव्या॒नौ स॑मानव्या॒नौ मे᳚ । \newline
24. स॒मा॒न॒व्या॒नाविति॑ समान - व्या॒नौ । \newline
25. मे॒ पा॒हि॒ पा॒हि॒ मे॒ मे॒ पा॒हि॒ । \newline
26. पा॒हीतीति॑ पाहि पा॒हीति॑ । \newline
27. इत्या॑हा॒हे तीत्या॑ह । \newline
28. आ॒हा॒शिष॑ मा॒शिष॑ माहा हा॒शिष᳚म् । \newline
29. आ॒शिष॑ मे॒वै वाशिष॑ मा॒शिष॑ मे॒व । \newline
30. आ॒शिष॒मित्या᳚ - शिष᳚म् । \newline
31. ए॒वैता मे॒ता मे॒वै वैताम् । \newline
32. ए॒ता मैता मे॒ता मा । \newline
33. आ शा᳚स्ते शास्त॒ आ शा᳚स्ते । \newline
34. शा॒स्ते ऽक्षि॒तो ऽक्षि॑तः शास्ते शा॒स्ते ऽक्षि॑तः । \newline
35. अक्षि॑तो ऽस्य॒स्य क्षि॒तो ऽक्षि॑तो ऽसि । \newline
36. अ॒स्य क्षि॑त्या॒ अक्षि॑त्या अस्य॒स्य क्षि॑त्यै । \newline
37. अक्षि॑त्यै त्वा॒ त्वा ऽक्षि॑त्या॒ अक्षि॑त्यै त्वा । \newline
38. त्वा॒ मा मा त्वा᳚ त्वा॒ मा । \newline
39. मा मे॑ मे॒ मा मा मे᳚ । \newline
40. मे॒ क्षे॒ष्ठाः॒ क्षे॒ष्ठा॒ मे॒ मे॒ क्षे॒ष्ठाः॒ । \newline
41. क्षे॒ष्ठा॒ अ॒मुत्रा॒ मुत्र॑ क्षेष्ठाः क्षेष्ठा अ॒मुत्र॑ । \newline
42. अ॒मुत्रा॒ मुष्मि॑न् न॒मुष्मि॑न् न॒मुत्रा॒ मुत्रा॒ मुष्मिन्न्॑ । \newline
43. अ॒मुष्मि॑न् ॅलो॒के लो॒के॑ ऽमुष्मि॑न् न॒मुष्मि॑न् ॅलो॒के । \newline
44. लो॒क इतीति॑ लो॒के लो॒क इति॑ । \newline
45. इत्या॑हा॒हे तीत्या॑ह । \newline
46. आ॒ह॒ क्षीय॑ते॒ क्षीय॑त आहाह॒ क्षीय॑ते । \newline
47. क्षीय॑ते॒ वै वै क्षीय॑ते॒ क्षीय॑ते॒ वै । \newline
48. वा अ॒मुष्मि॑न् न॒मुष्मि॒न्॒. वै वा अ॒मुष्मिन्न्॑ । \newline
49. अ॒मुष्मि॑न् ॅलो॒के लो॒के॑ ऽमुष्मि॑न् न॒मुष्मि॑न् ॅलो॒के । \newline
50. लो॒के ऽन्न॒ मन्न॑म् ॅलो॒के लो॒के ऽन्न᳚म् । \newline
51. अन्न॑ मि॒तःप्र॑दान मि॒तःप्र॑दान॒ मन्न॒ मन्न॑ मि॒तःप्र॑दानम् । \newline
52. इ॒तःप्र॑दानꣳ॒॒ हि हीतःप्र॑दान मि॒तःप्र॑दानꣳ॒॒ हि । \newline
53. इ॒तःप्र॑दान॒मिती॒तः - प्र॒दा॒न॒म् । \newline
54. ह्य॑मुष्मि॑न् न॒मुष्मि॒न्॒. हि ह्य॑मुष्मिन्न्॑ । \newline
55. अ॒मुष्मि॑न् ॅलो॒के लो॒के॑ ऽमुष्मि॑न् न॒मुष्मि॑न् ॅलो॒के । \newline
56. लो॒के प्र॒जाः प्र॒जा लो॒के लो॒के प्र॒जाः । \newline
57. प्र॒जा उ॑प॒जीव॑ न्त्युप॒जीव॑न्ति प्र॒जाः प्र॒जा उ॑प॒जीव॑न्ति । \newline
58. प्र॒जा इति॑ प्र - जाः । \newline
59. उ॒प॒जीव॑न्ति॒ यद् यदु॑प॒जीव॑ न्त्युप॒जीव॑न्ति॒ यत् । \newline
60. उ॒प॒जीव॒न्तीत्यु॑प - जीव॑न्ति । \newline
61. यदे॒व मे॒वं ॅयद् यदे॒वम् । \newline
62. ए॒व म॑भिमृ॒श त्य॑भिमृ॒श त्ये॒व मे॒व म॑भिमृ॒शति॑ । \newline
63. अ॒भि॒मृ॒श त्यक्षि॑ति॒ मक्षि॑ति मभिमृ॒श त्य॑भिमृ॒श त्यक्षि॑तिम् । \newline
64. अ॒भि॒मृ॒शतीत्य॑भि - मृ॒शति॑ । \newline
65. अक्षि॑ति मे॒वै वाक्षि॑ति॒ मक्षि॑ति मे॒व । \newline
66. ए॒वैन॑ देन दे॒वै वैन॑त् । \newline
67. ए॒न॒द् ग॒म॒य॒ति॒ ग॒म॒य॒ त्ये॒न॒ दे॒न॒द् ग॒म॒य॒ति॒ । \newline
68. ग॒म॒य॒ति॒ न न ग॑मयति गमयति॒ न । \newline
69. नास्या᳚स्य॒ न नास्य॑ । \newline
70. अ॒स्या॒ मुष्मि॑न् न॒मुष्मि॑न् नस्यास्या॒ मुष्मिन्न्॑ । \newline
71. अ॒मुष्मि॑न् ॅलो॒के लो॒के॑ ऽमुष्मि॑न् न॒मुष्मि॑न् ॅलो॒के । \newline
72. लो॒के ऽन्न॒ मन्न॑म् ॅलो॒के लो॒के ऽन्न᳚म् । \newline
73. अन्न॑म् क्षीयते क्षीय॒ते ऽन्न॒ मन्न॑म् क्षीयते । \newline
74. क्षी॒य॒त॒ इति॑ क्षीयते । \newline

\textbf{Ghana Paata } \newline

1. इत्या॑हा॒हे तीत्या॑ह प्र॒जाप॑तिम् प्र॒जाप॑ति मा॒हे तीत्या॑ह प्र॒जाप॑तिम् । \newline
2. आ॒ह॒ प्र॒जाप॑तिम् प्र॒जाप॑ति माहाह प्र॒जाप॑ति मे॒वैव प्र॒जाप॑ति माहाह प्र॒जाप॑ति मे॒व । \newline
3. प्र॒जाप॑ति मे॒वैव प्र॒जाप॑तिम् प्र॒जाप॑ति मे॒व भा॑ग॒धेये॑न भाग॒धेये॑नै॒व प्र॒जाप॑तिम् प्र॒जाप॑ति मे॒व भा॑ग॒धेये॑न । \newline
4. प्र॒जाप॑ति॒मिति॑ प्र॒जा - प॒ति॒म् । \newline
5. ए॒व भा॑ग॒धेये॑न भाग॒धेये॑ नै॒वैव भा॑ग॒धेये॑न॒ सꣳ सम् भा॑ग॒धेये॑ नै॒वैव भा॑ग॒धेये॑न॒ सम् । \newline
6. भा॒ग॒धेये॑न॒ सꣳ सम् भा॑ग॒धेये॑न भाग॒धेये॑न॒ स म॑र्द्धय त्यर्द्धयति॒ सम् भा॑ग॒धेये॑न भाग॒धेये॑न॒ स म॑र्द्धयति । \newline
7. भा॒ग॒धेये॒नेति॑ भाग - धेये॑न । \newline
8. स म॑र्द्धय त्यर्द्धयति॒ सꣳ स म॑र्द्धय॒ त्यूर्ज॑स्वा॒ नूर्ज॑स्वा नर्द्धयति॒ सꣳ स म॑र्द्धय॒ त्यूर्ज॑स्वान् । \newline
9. अ॒र्द्ध॒य॒ त्यूर्ज॑स्वा॒ नूर्ज॑स्वा नर्द्धय त्यर्द्धय॒ त्यूर्ज॑स्वा॒न् पय॑स्वा॒न् पय॑स्वा॒ नूर्ज॑स्वा नर्द्धय त्यर्द्धय॒ त्यूर्ज॑स्वा॒न् पय॑स्वान् । \newline
10. ऊर्ज॑स्वा॒न् पय॑स्वा॒न् पय॑स्वा॒ नूर्ज॑स्वा॒ नूर्ज॑स्वा॒न् पय॑स्वा॒ नितीति॒ पय॑स्वा॒ नूर्ज॑स्वा॒ नूर्ज॑स्वा॒न् पय॑स्वा॒ निति॑ । \newline
11. पय॑स्वा॒ नितीति॒ पय॑स्वा॒न् पय॑स्वा॒ नित्या॑हा॒हे ति॒ पय॑स्वा॒न् पय॑स्वा॒ नित्या॑ह । \newline
12. इत्या॑हा॒हे तीत्या॒होर्ज॒ मूर्ज॑ मा॒हे तीत्या॒होर्ज᳚म् । \newline
13. आ॒होर्ज॒ मूर्ज॑ माहा॒होर्ज॑ मे॒वैवोर्ज॑ माहा॒होर्ज॑ मे॒व । \newline
14. ऊर्ज॑ मे॒वैवोर्ज॒ मूर्ज॑ मे॒वास्मि॑न् नस्मिन् ने॒वोर्ज॒ मूर्ज॑ मे॒वास्मिन्न्॑ । \newline
15. ए॒वास्मि॑न् नस्मिन् ने॒वैवास्मि॒न् पयः॒ पयो᳚ ऽस्मिन् ने॒वैवास्मि॒न् पयः॑ । \newline
16. अ॒स्मि॒न् पयः॒ पयो᳚ ऽस्मिन् नस्मि॒न् पयो॑ दधाति दधाति॒ पयो᳚ ऽस्मिन् नस्मि॒न् पयो॑ दधाति । \newline
17. पयो॑ दधाति दधाति॒ पयः॒ पयो॑ दधाति प्राणापा॒नौ प्रा॑णापा॒नौ द॑धाति॒ पयः॒ पयो॑ दधाति प्राणापा॒नौ । \newline
18. द॒धा॒ति॒ प्रा॒णा॒पा॒नौ प्रा॑णापा॒नौ द॑धाति दधाति प्राणापा॒नौ मे॑ मे प्राणापा॒नौ द॑धाति दधाति प्राणापा॒नौ मे᳚ । \newline
19. प्रा॒णा॒पा॒नौ मे॑ मे प्राणापा॒नौ प्रा॑णापा॒नौ मे॑ पाहि पाहि मे प्राणापा॒नौ प्रा॑णापा॒नौ मे॑ पाहि । \newline
20. प्रा॒णा॒पा॒नाविति॑ प्राण - अ॒पा॒नौ । \newline
21. मे॒ पा॒हि॒ पा॒हि॒ मे॒ मे॒ पा॒हि॒ स॒मा॒न॒व्या॒नौ स॑मानव्या॒नौ पा॑हि मे मे पाहि समानव्या॒नौ । \newline
22. पा॒हि॒ स॒मा॒न॒व्या॒नौ स॑मानव्या॒नौ पा॑हि पाहि समानव्या॒नौ मे॑ मे समानव्या॒नौ पा॑हि पाहि समानव्या॒नौ मे᳚ । \newline
23. स॒मा॒न॒व्या॒नौ मे॑ मे समानव्या॒नौ स॑मानव्या॒नौ मे॑ पाहि पाहि मे समानव्या॒नौ स॑मानव्या॒नौ मे॑ पाहि । \newline
24. स॒मा॒न॒व्या॒नाविति॑ समान - व्या॒नौ । \newline
25. मे॒ पा॒हि॒ पा॒हि॒ मे॒ मे॒ पा॒हीतीति॑ पाहि मे मे पा॒हीति॑ । \newline
26. पा॒ही तीति॑ पाहि पा॒ही त्या॑हा॒हे ति॑ पाहि पा॒ही त्या॑ह । \newline
27. इत्या॑हा॒हे ती त्या॑हा॒शिष॑ मा॒शिष॑ मा॒हे ती त्या॑हा॒शिष᳚म् । \newline
28. आ॒हा॒शिष॑ मा॒शिष॑ माहाहा॒शिष॑ मे॒वैवाशिष॑ माहाहा॒शिष॑ मे॒व । \newline
29. आ॒शिष॑ मे॒वैवाशिष॑ मा॒शिष॑ मे॒वैता मे॒ता मे॒वाशिष॑ मा॒शिष॑ मे॒वैताम् । \newline
30. आ॒शिष॒मित्या᳚ - शिष᳚म् । \newline
31. ए॒वैता मे॒ता मे॒वैवैता मैता मे॒वैवैता मा । \newline
32. ए॒ता मैता मे॒ता मा शा᳚स्ते शास्त॒ ऐता मे॒ता मा शा᳚स्ते । \newline
33. आ शा᳚स्ते शास्त॒ आ शा॒स्ते ऽक्षि॒तो ऽक्षि॑तः शास्त॒ आ शा॒स्ते ऽक्षि॑तः । \newline
34. शा॒स्ते ऽक्षि॒तो ऽक्षि॑तः शास्ते शा॒स्ते ऽक्षि॑तो ऽस्य॒ स्यक्षि॑तः शास्ते शा॒स्ते ऽक्षि॑तो ऽसि । \newline
35. अक्षि॑तो ऽस्य॒ स्यक्षि॒तो ऽक्षि॑तो॒ ऽस्यक्षि॑त्या॒ अक्षि॑त्या अ॒स्यक्षि॒तो ऽक्षि॑तो॒ ऽस्यक्षि॑त्यै । \newline
36. अ॒स्यक्षि॑त्या॒ अक्षि॑त्या अस्य॒ स्यक्षि॑त्यै त्वा॒ त्वा ऽक्षि॑त्या अस्य॒ स्यक्षि॑त्यै त्वा । \newline
37. अक्षि॑त्यै त्वा॒ त्वा ऽक्षि॑त्या॒ अक्षि॑त्यै त्वा॒ मा मा त्वा ऽक्षि॑त्या॒ अक्षि॑त्यै त्वा॒ मा । \newline
38. त्वा॒ मा मा त्वा᳚ त्वा॒ मा मे॑ मे॒ मा त्वा᳚ त्वा॒ मा मे᳚ । \newline
39. मा मे॑ मे॒ मा मा मे᳚ क्षेष्ठाः क्षेष्ठा मे॒ मा मा मे᳚ क्षेष्ठाः । \newline
40. मे॒ क्षे॒ष्ठाः॒ क्षे॒ष्ठा॒ मे॒ मे॒ क्षे॒ष्ठा॒ अ॒मुत्रा॒मुत्र॑ क्षेष्ठा मे मे क्षेष्ठा अ॒मुत्र॑ । \newline
41. क्षे॒ष्ठा॒ अ॒मुत्रा॒मुत्र॑ क्षेष्ठाः क्षेष्ठा अ॒मुत्रा॒ मुष्मि॑न् न॒मुष्मि॑न् न॒मुत्र॑ क्षेष्ठाः क्षेष्ठा अ॒मुत्रा॒ मुष्मिन्न्॑ । \newline
42. अ॒मुत्रा॒ मुष्मि॑न् न॒मुष्मि॑न् न॒मुत्रा॒मुत्रा॒ मुष्मि॑न् ॅलो॒के लो॒के॑ ऽमुष्मि॑न् न॒मुत्रा॒मुत्रा॒ मुष्मि॑न् ॅलो॒के । \newline
43. अ॒मुष्मि॑न् ॅलो॒के लो॒के॑ ऽमुष्मि॑न् न॒मुष्मि॑न् ॅलो॒क इतीति॑ लो॒के॑ ऽमुष्मि॑न् न॒मुष्मि॑न् ॅलो॒क इति॑ । \newline
44. लो॒क इतीति॑ लो॒के लो॒क इत्या॑हा॒हे ति॑ लो॒के लो॒क इत्या॑ह । \newline
45. इत्या॑हा॒हे तीत्या॑ह॒ क्षीय॑ते॒ क्षीय॑त आ॒हे तीत्या॑ह॒ क्षीय॑ते । \newline
46. आ॒ह॒ क्षीय॑ते॒ क्षीय॑त आहाह॒ क्षीय॑ते॒ वै वै क्षीय॑त आहाह॒ क्षीय॑ते॒ वै । \newline
47. क्षीय॑ते॒ वै वै क्षीय॑ते॒ क्षीय॑ते॒ वा अ॒मुष्मि॑न् न॒मुष्मि॒न्॒. वै क्षीय॑ते॒ क्षीय॑ते॒ वा अ॒मुष्मिन्न्॑ । \newline
48. वा अ॒मुष्मि॑न् न॒मुष्मि॒न्॒. वै वा अ॒मुष्मि॑न् ॅलो॒के लो॒के॑ ऽमुष्मि॒न्॒. वै वा अ॒मुष्मि॑न् ॅलो॒के । \newline
49. अ॒मुष्मि॑न् ॅलो॒के लो॒के॑ ऽमुष्मि॑न् न॒मुष्मि॑न् ॅलो॒के ऽन्न॒ मन्न॑म् ॅलो॒के॑ ऽमुष्मि॑न् न॒मुष्मि॑न् ॅलो॒के ऽन्न᳚म् । \newline
50. लो॒के ऽन्न॒ मन्न॑म् ॅलो॒के लो॒के ऽन्न॑ मि॒तःप्र॑दान मि॒तःप्र॑दान॒ मन्न॑म् ॅलो॒के लो॒के ऽन्न॑ मि॒तःप्र॑दानम् । \newline
51. अन्न॑ मि॒तःप्र॑दान मि॒तःप्र॑दान॒ मन्न॒ मन्न॑ मि॒तःप्र॑दान॒(ग्म्॒) हि हीतःप्र॑दान॒ मन्न॒ मन्न॑ मि॒तःप्र॑दान॒(ग्म्॒) हि । \newline
52. इ॒तःप्र॑दान॒(ग्म्॒) हि हीतःप्र॑दान मि॒तःप्र॑दानꣳ॒॒ ह्य॑मुष्मि॑न् न॒मुष्मि॒न्॒. हीतःप्र॑दान मि॒तःप्र॑दानꣳ॒॒ ह्य॑मुष्मिन्न्॑ । \newline
53. इ॒तःप्र॑दान॒मिती॒तः - प्र॒दा॒न॒म् । \newline
54. ह्य॑मुष्मि॑न् न॒मुष्मि॒न्॒. हि ह्य॑मुष्मि॑न् ॅलो॒के लो॒के॑ ऽमुष्मि॒न्॒. हि ह्य॑मुष्मि॑न् ॅलो॒के । \newline
55. अ॒मुष्मि॑न् ॅलो॒के लो॒के॑ ऽमुष्मि॑न् न॒मुष्मि॑न् ॅलो॒के प्र॒जाः प्र॒जा लो॒के॑ ऽमुष्मि॑न् न॒मुष्मि॑न् ॅलो॒के प्र॒जाः । \newline
56. लो॒के प्र॒जाः प्र॒जा लो॒के लो॒के प्र॒जा उ॑प॒जीव॑ न्त्युप॒जीव॑न्ति प्र॒जा लो॒के लो॒के प्र॒जा उ॑प॒जीव॑न्ति । \newline
57. प्र॒जा उ॑प॒जीव॑न् त्युप॒जीव॑न्ति प्र॒जाः प्र॒जा उ॑प॒जीव॑न्ति॒ यद् यदु॑प॒जीव॑न्ति प्र॒जाः प्र॒जा उ॑प॒जीव॑न्ति॒ यत् । \newline
58. प्र॒जा इति॑ प्र - जाः । \newline
59. उ॒प॒जीव॑न्ति॒ यद् यदु॑प॒जीव॑ न्त्युप॒जीव॑न्ति॒ यदे॒व मे॒वं ॅयदु॑प॒जीव॑ न्त्युप॒जीव॑न्ति॒ यदे॒वम् । \newline
60. उ॒प॒जीव॒न्तीत्यु॑प - जीव॑न्ति । \newline
61. यदे॒व मे॒वं ॅयद् यदे॒व म॑भिमृ॒श त्य॑भिमृ॒श त्ये॒वं ॅयद् यदे॒व म॑भिमृ॒शति॑ । \newline
62. ए॒व म॑भिमृ॒श त्य॑भिमृ॒श त्ये॒व मे॒व म॑भिमृ॒श त्यक्षि॑ति॒ मक्षि॑ति मभिमृ॒श त्ये॒व मे॒व म॑भिमृ॒श त्यक्षि॑तिम् । \newline
63. अ॒भि॒मृ॒श त्यक्षि॑ति॒ मक्षि॑ति मभिमृ॒श त्य॑भिमृ॒श त्यक्षि॑ति मे॒वैवाक्षि॑ति मभिमृ॒श त्य॑भिमृ॒श त्यक्षि॑ति मे॒व । \newline
64. अ॒भि॒मृ॒शतीत्य॑भि - मृ॒शति॑ । \newline
65. अक्षि॑ति मे॒वैवाक्षि॑ति॒ मक्षि॑ति मे॒वैन॑ देन दे॒वाक्षि॑ति॒ मक्षि॑ति मे॒वैन॑त् । \newline
66. ए॒वैन॑ देन दे॒वैवैन॑द् गमयति गमय त्येन दे॒वैवैन॑द् गमयति । \newline
67. ए॒न॒द् ग॒म॒य॒ति॒ ग॒म॒य॒ त्ये॒न॒ दे॒न॒द् ग॒म॒य॒ति॒ न न ग॑मय त्येन देनद् गमयति॒ न । \newline
68. ग॒म॒य॒ति॒ न न ग॑मयति गमयति॒ नास्या᳚स्य॒ न ग॑मयति गमयति॒ नास्य॑ । \newline
69. नास्या᳚स्य॒ न नास्या॒मुष्मि॑न् न॒मुष्मि॑न् नस्य॒ न नास्या॒मुष्मिन्न्॑ । \newline
70. अ॒स्या॒मुष्मि॑न् न॒मुष्मि॑न् नस्या स्या॒मुष्मि॑न् ॅलो॒के लो॒के॑ ऽमुष्मि॑न् नस्या स्या॒मुष्मि॑न् ॅलो॒के । \newline
71. अ॒मुष्मि॑न् ॅलो॒के लो॒के॑ ऽमुष्मि॑न् न॒मुष्मि॑न् ॅलो॒के ऽन्न॒ मन्न॑म् ॅलो॒के॑ ऽमुष्मि॑न् न॒मुष्मि॑न् ॅलो॒के ऽन्न᳚म् । \newline
72. लो॒के ऽन्न॒ मन्न॑म् ॅलो॒के लो॒के ऽन्न॑म् क्षीयते क्षीय॒ते ऽन्न॑म् ॅलो॒के लो॒के ऽन्न॑म् क्षीयते । \newline
73. अन्न॑म् क्षीयते क्षीय॒ते ऽन्न॒ मन्न॑म् क्षीयते । \newline
74. क्षी॒य॒त॒ इति॑ क्षीयते । \newline
\pagebreak
\markright{ TS 1.7.4.1  \hfill https://www.vedavms.in \hfill}
\addcontentsline{toc}{section}{ TS 1.7.4.1 }
\section*{ TS 1.7.4.1 }

\textbf{TS 1.7.4.1 } \newline
\textbf{Samhita Paata} \newline

ब॒र्॒.हिषो॒ऽहं दे॑वय॒ज्यया᳚ प्र॒जावा᳚न् भूयास॒मित्या॑ह ब॒र्॒.हिषा॒ वै प्र॒जाप॑तिः प्र॒जा अ॑सृजत॒ तेनै॒व प्र॒जाः सृ॑जते॒ नरा॒शꣳस॑स्या॒हं दे॑वय॒ज्यया॑ पशु॒मान् भू॑यास॒मित्या॑ह॒ नरा॒शꣳसे॑न॒ वै प्र॒जाप॑तिः प॒शून॑सृजत॒ तेनै॒व प॒शून्थ् सृ॑जते॒ऽग्नेः स्वि॑ष्ट॒कृतो॒ऽहं दे॑वय॒ज्ययाऽऽयु॑ष्मान्. य॒ज्ञेन॑ प्रति॒ष्ठां ग॑मेय॒मित्या॒हाऽऽयु॑रे॒वात्मन् ध॑त्ते॒ प्रति॑ य॒ज्ञेन॑ तिष्ठति दर्.शपूर्णमा॒सयो॒र् - [ ] \newline

\textbf{Pada Paata} \newline

ब॒र्॒.हिषः॑ । अ॒हम् । दे॒व॒य॒ज्ययेति॑ देव - य॒ज्यया᳚ । प्र॒जावा॒निति॑ प्र॒जा -वा॒न् । भू॒या॒स॒म् । इति॑ । आ॒ह॒ । ब॒र॒.हिषा᳚ । वै । प्र॒जाप॑ति॒रिति॑ प्र॒जा - प॒तिः॒ । प्र॒जा इति॑ प्र - जाः । अ॒सृ॒ज॒त॒ । तेन॑ । ए॒व । प्र॒जा इति॑ प्र - जाः । सृ॒ज॒ते॒ । नरा॒शꣳस॑स्य । अ॒हम् । दे॒व॒य॒ज्ययेति॑ देव - य॒ज्यया᳚ । प॒शु॒मानिति॑ पशु-मान् । भू॒या॒स॒म् । इति॑ । आ॒ह॒ । नरा॒शꣳसे॑न । वै । प्र॒जाप॑ति॒रिति॑ प्र॒जा - प॒तिः॒ । प॒शून् । अ॒सृ॒ज॒त॒ । तेन॑ । ए॒व । प॒शून् । सृ॒ज॒ते॒ । अ॒ग्नेः । स्वि॒ष्ट॒कृत॒ इति॑ स्विष्ट - कृतः॑ । अ॒हम् । दे॒व॒य॒ज्ययेति॑ देव - य॒ज्यया᳚ । आयु॑ष्मान् । य॒ज्ञेन॑ । प्र॒ति॒ष्ठामिति॑ प्रति-स्थाम् । ग॒मे॒य॒म् । इति॑ । आ॒ह॒ । आयुः॑ । ए॒व । आ॒त्मन्न् । ध॒त्ते॒ । प्रतीति॑ । य॒ज्ञेन॑ । ति॒ष्ठ॒ति॒ । द॒र॒.श॒पू॒र्ण॒मा॒सयो॒रिति॑ दर्.श - पू॒र्ण॒मा॒सयोः᳚ ।  \newline


\textbf{Krama Paata} \newline

ब॒र्.॒॒हिषो॒ऽहम् । अ॒हम् दे॑वय॒ज्यया᳚ । दे॒व॒य॒ज्यया᳚ प्र॒जावान्॑ । दे॒व॒य॒ज्ययेति॑ देव - य॒ज्यया᳚ । प्र॒जावा᳚न् भूयासम् । प्र॒जावा॒निति॑ प्र॒जा - वा॒न्॒ । भू॒या॒स॒मिति॑ । इत्या॑ह । आ॒ह॒ ब॒र्.॒॒॒हिषा᳚ । ब॒र्.॒॒॒हिषा॒ वै । वै प्र॒जाप॑तिः । प्र॒जाप॑तिः प्र॒जाः । प्र॒जाप॑ति॒रिति॑ प्र॒जा - प॒तिः॒ । प्र॒जा अ॑सृजत । प्र॒जा इति॑ प्र - जाः । अ॒सृ॒ज॒त॒ तेन॑ । तेनै॒व । ए॒व प्र॒जाः । प्र॒जाः सृ॑जते । प्र॒जा इति॑ प्र - जाः । सृ॒ज॒ते॒ नरा॒शꣳस॑स्य । नरा॒शꣳस॑स्या॒हम् । अ॒हम् दे॑वय॒ज्यया᳚ । दे॒व॒य॒ज्यया॑ पशु॒मान् । दे॒व॒य॒ज्ययेति॑ देव - य॒ज्यया᳚ । प॒शु॒मान् भू॑यासम् । प॒शु॒मानिति॑ पशु - मान् । भू॒या॒स॒मिति॑ । इत्या॑ह । आ॒ह॒ नरा॒शꣳसे॑न । नरा॒शꣳसे॑न॒ वै । वै प्र॒जाप॑तिः । प्र॒जाप॑तिः प॒शून् । प्र॒जाप॑ति॒रिति॑ प्र॒जा - प॒तिः॒ । प॒शून॑सृजत । अ॒सृ॒ज॒त॒ तेन॑ । तेनै॒व । ए॒व प॒शून् । प॒शून्थ् सृ॑जते । सृ॒ज॒ते॒ऽग्नेः । अ॒ग्ने स्वि॑ष्ट॒कृतः॑ । स्वि॒ष्ट॒कृतो॒ऽहम् । स्वि॒ष्ट॒कृत॒ इति॑ स्विष्ट - कृतः॑ । अ॒हम् दे॑वय॒ज्यया᳚ । दे॒व॒य॒ज्यया ऽऽयु॑ष्मान् । दे॒व॒य॒ज्ययेति॑ देव - य॒ज्यया᳚ । आयु॑ष्मान्. य॒ज्ञेन॑ । य॒ज्ञेन॑ प्रति॒ष्ठाम् । प्र॒ति॒ष्ठाम् ग॑मेयम् । प्र॒ति॒ष्ठामिति॑ प्रति - स्थाम् । ग॒मे॒य॒मिति॑ । इत्या॑ह । आ॒हायुः॑ । आयु॑रे॒व । ए॒वात्मन्न् । आ॒त्मन्,ध॑त्ते । ध॒त्ते॒ प्रति॑ । प्रति॑ य॒ज्ञेन॑ । य॒ज्ञेन॑ तिष्ठति । ति॒ष्ठ॒ति॒ द॒र्॒.श॒पू॒र्ण॒मा॒सयोः᳚ । द॒र्॒.श॒पू॒र्ण॒मा॒सयो॒र्,वै । द॒र्॒.श॒पू॒र्ण॒मा॒सयो॒रिति॑ दर्.श - पू॒र्ण॒मा॒सयोः᳚ \newline

\textbf{Jatai Paata} \newline

1. ब॒र्॒.हिषो॒ ऽह म॒हम् ब॒र्॒.हिषो॑ ब॒र्॒.हिषो॒ ऽहम् । \newline
2. अ॒हम् दे॑वय॒ज्यया॑ देवय॒ज्यया॒ ऽह म॒हम् दे॑वय॒ज्यया᳚ । \newline
3. दे॒व॒य॒ज्यया᳚ प्र॒जावा᳚न् प्र॒जावा᳚न् देवय॒ज्यया॑ देवय॒ज्यया᳚ प्र॒जावान्॑ । \newline
4. दे॒व॒य॒ज्ययेति॑ देव - य॒ज्यया᳚ । \newline
5. प्र॒जावा᳚न् भूयासम् भूयासम् प्र॒जावा᳚न् प्र॒जावा᳚न् भूयासम् । \newline
6. प्र॒जावा॒निति॑ प्र॒जा - वा॒न् । \newline
7. भू॒या॒स॒ मितीति॑ भूयासम् भूयास॒ मिति॑ । \newline
8. इत्या॑हा॒हे तीत्या॑ह । \newline
9. आ॒ह॒ ब॒र्॒.हिषा॑ ब॒र्॒.हिषा॑ ऽऽहाह ब॒र्॒.हिषा᳚ । \newline
10. ब॒र्॒.हिषा॒ वै वै ब॒र्॒.हिषा॑ ब॒र्॒.हिषा॒ वै । \newline
11. वै प्र॒जाप॑तिः प्र॒जाप॑ति॒र् वै वै प्र॒जाप॑तिः । \newline
12. प्र॒जाप॑तिः प्र॒जाः प्र॒जाः प्र॒जाप॑तिः प्र॒जाप॑तिः प्र॒जाः । \newline
13. प्र॒जाप॑ति॒रिति॑ प्र॒जा - प॒तिः॒ । \newline
14. प्र॒जा अ॑सृजता सृजत प्र॒जाः प्र॒जा अ॑सृजत । \newline
15. प्र॒जा इति॑ प्र - जाः । \newline
16. अ॒सृ॒ज॒त॒ तेन॒ तेना॑सृजता सृजत॒ तेन॑ । \newline
17. तेनै॒वैव तेन॒ तेनै॒व । \newline
18. ए॒व प्र॒जाः प्र॒जा ए॒वैव प्र॒जाः । \newline
19. प्र॒जाः सृ॑जते सृजते प्र॒जाः प्र॒जाः सृ॑जते । \newline
20. प्र॒जा इति॑ प्र - जाः । \newline
21. सृ॒ज॒ते॒ नरा॒शꣳस॑स्य॒ नरा॒शꣳस॑स्य सृजते सृजते॒ नरा॒शꣳस॑स्य । \newline
22. नरा॒शꣳस॑स्या॒ह म॒हन् नरा॒शꣳस॑स्य॒ नरा॒शꣳस॑स्या॒हम् । \newline
23. अ॒हम् दे॑वय॒ज्यया॑ देवय॒ज्यया॒ ऽह म॒हम् दे॑वय॒ज्यया᳚ । \newline
24. दे॒व॒य॒ज्यया॑ पशु॒मान् प॑शु॒मान् दे॑वय॒ज्यया॑ देवय॒ज्यया॑ पशु॒मान् । \newline
25. दे॒व॒य॒ज्ययेति॑ देव - य॒ज्यया᳚ । \newline
26. प॒शु॒मान् भू॑यासम् भूयासम् पशु॒मान् प॑शु॒मान् भू॑यासम् । \newline
27. प॒शु॒मानिति॑ पशु - मान् । \newline
28. भू॒या॒स॒ मितीति॑ भूयासम् भूयास॒ मिति॑ । \newline
29. इत्या॑हा॒हे तीत्या॑ह । \newline
30. आ॒ह॒ नरा॒शꣳसे॑न॒ नरा॒शꣳसे॑ नाहाह॒ नरा॒शꣳसे॑न । \newline
31. नरा॒शꣳसे॑न॒ वै वै नरा॒शꣳसे॑न॒ नरा॒शꣳसे॑न॒ वै । \newline
32. वै प्र॒जाप॑तिः प्र॒जाप॑ति॒र् वै वै प्र॒जाप॑तिः । \newline
33. प्र॒जाप॑तिः प॒शून् प॒शून् प्र॒जाप॑तिः प्र॒जाप॑तिः प॒शून् । \newline
34. प्र॒जाप॑ति॒रिति॑ प्र॒जा - प॒तिः॒ । \newline
35. प॒शू न॑सृजता सृजत प॒शून् प॒शू न॑सृजत । \newline
36. अ॒सृ॒ज॒त॒ तेन॒ तेना॑सृजता सृजत॒ तेन॑ । \newline
37. तेनै॒वैव तेन॒ तेनै॒व । \newline
38. ए॒व प॒शून् प॒शू ने॒वैव प॒शून् । \newline
39. प॒शून् थ्सृ॑जते सृजते प॒शून् प॒शून् थ्सृ॑जते । \newline
40. सृ॒ज॒ते॒ ऽग्ने र॒ग्नेः सृ॑जते सृजते॒ ऽग्नेः । \newline
41. अ॒ग्नेः स्वि॑ष्ट॒कृतः॑ स्विष्ट॒कृतो॒ ऽग्ने र॒ग्नेः स्वि॑ष्ट॒कृतः॑ । \newline
42. स्वि॒ष्ट॒कृतो॒ ऽह म॒हꣳ स्वि॑ष्ट॒कृतः॑ स्विष्ट॒कृतो॒ ऽहम् । \newline
43. स्वि॒ष्ट॒कृत॒ इति॑ स्विष्ट - कृतः॑ । \newline
44. अ॒हम् दे॑वय॒ज्यया॑ देवय॒ज्यया॒ ऽह म॒हम् दे॑वय॒ज्यया᳚ । \newline
45. दे॒व॒य॒ज्यया ऽऽयु॑ष्मा॒ नायु॑ष्मान् देवय॒ज्यया॑ देवय॒ज्यया ऽऽयु॑ष्मान् । \newline
46. दे॒व॒य॒ज्ययेति॑ देव - य॒ज्यया᳚ । \newline
47. आयु॑ष्मान्. य॒ज्ञेन॑ य॒ज्ञेनायु॑ष्मा॒ नायु॑ष्मान्. य॒ज्ञेन॑ । \newline
48. य॒ज्ञेन॑ प्रति॒ष्ठाम् प्र॑ति॒ष्ठां ॅय॒ज्ञेन॑ य॒ज्ञेन॑ प्रति॒ष्ठाम् । \newline
49. प्र॒ति॒ष्ठाम् ग॑मेयम् गमेयम् प्रति॒ष्ठाम् प्र॑ति॒ष्ठाम् ग॑मेयम् । \newline
50. प्र॒ति॒ष्ठामिति॑ प्रति - स्थाम् । \newline
51. ग॒मे॒य॒ मितीति॑ गमेयम् गमेय॒ मिति॑ । \newline
52. इत्या॑हा॒हे तीत्या॑ह । \newline
53. आ॒हायु॒ रायु॑ राहा॒ हायुः॑ । \newline
54. आयु॑ रे॒वै वायु॒ रायु॑ रे॒व । \newline
55. ए॒वात्मन् ना॒त्मन् ने॒वै वात्मन्न् । \newline
56. आ॒त्मन् ध॑त्ते धत्त आ॒त्मन् ना॒त्मन् ध॑त्ते । \newline
57. ध॒त्ते॒ प्रति॒ प्रति॑ धत्ते धत्ते॒ प्रति॑ । \newline
58. प्रति॑ य॒ज्ञेन॑ य॒ज्ञेन॒ प्रति॒ प्रति॑ य॒ज्ञेन॑ । \newline
59. य॒ज्ञेन॑ तिष्ठति तिष्ठति य॒ज्ञेन॑ य॒ज्ञेन॑ तिष्ठति । \newline
60. ति॒ष्ठ॒ति॒ द॒र्॒.श॒पू॒र्ण॒मा॒सयो᳚र् दर्.शपूर्णमा॒सयो᳚ स्तिष्ठति तिष्ठति दर्.शपूर्णमा॒सयोः᳚ । \newline
61. द॒र्॒.श॒पू॒र्ण॒मा॒सयो॒र् वै वै द॑र्.शपूर्णमा॒सयो᳚र् दर्.शपूर्णमा॒सयो॒र् वै । \newline
62. द॒र्॒.श॒पू॒र्ण॒मा॒सयो॒रिति॑ दर्.श - पू॒र्ण॒मा॒सयोः᳚ । \newline

\textbf{Ghana Paata } \newline

1. ब॒र्॒.हिषो॒ ऽह म॒हम् ब॒र्॒.हिषो॑ ब॒र्॒.हिषो॒ ऽहम् दे॑वय॒ज्यया॑ देवय॒ज्यया॒ ऽहम् ब॒र्॒.हिषो॑ ब॒र्॒.हिषो॒ ऽहम् दे॑वय॒ज्यया᳚ । \newline
2. अ॒हम् दे॑वय॒ज्यया॑ देवय॒ज्यया॒ ऽह म॒हम् दे॑वय॒ज्यया᳚ प्र॒जावा᳚न् प्र॒जावा᳚न् देवय॒ज्यया॒ ऽह म॒हम् दे॑वय॒ज्यया᳚ प्र॒जावान्॑ । \newline
3. दे॒व॒य॒ज्यया᳚ प्र॒जावा᳚न् प्र॒जावा᳚न् देवय॒ज्यया॑ देवय॒ज्यया᳚ प्र॒जावा᳚न् भूयासम् भूयासम् प्र॒जावा᳚न् देवय॒ज्यया॑ देवय॒ज्यया᳚ प्र॒जावा᳚न् भूयासम् । \newline
4. दे॒व॒य॒ज्ययेति॑ देव - य॒ज्यया᳚ । \newline
5. प्र॒जावा᳚न् भूयासम् भूयासम् प्र॒जावा᳚न् प्र॒जावा᳚न् भूयास॒ मितीति॑ भूयासम् प्र॒जावा᳚न् प्र॒जावा᳚न् भूयास॒ मिति॑ । \newline
6. प्र॒जावा॒निति॑ प्र॒जा - वा॒न् । \newline
7. भू॒या॒स॒ मितीति॑ भूयासम् भूयास॒ मित्या॑हा॒हे ति॑ भूयासम् भूयास॒ मित्या॑ह । \newline
8. इत्या॑हा॒हे तीत्या॑ह ब॒र्॒.हिषा॑ ब॒र्॒.हिषा॒ ऽऽहे तीत्या॑ह ब॒र्॒.हिषा᳚ । \newline
9. आ॒ह॒ ब॒र्॒,हिषा॑ ब॒र्॒.हिषा॑ ऽऽहाह ब॒र्॒.हिषा॒ वै वै ब॒र्.॒हिषा॑ ऽऽहाह ब॒र्॒.हिषा॒ वै । \newline
10. ब॒र्॒.हिषा॒ वै वै ब॒र्॒.हिषा॑ ब॒र्॒.हिषा॒ वै प्र॒जाप॑तिः प्र॒जाप॑ति॒र् वै ब॒र्॒.हिषा॑ ब॒र्॒.हिषा॒ वै प्र॒जाप॑तिः । \newline
11. वै प्र॒जाप॑तिः प्र॒जाप॑ति॒र् वै वै प्र॒जाप॑तिः प्र॒जाः प्र॒जाः प्र॒जाप॑ति॒र् वै वै प्र॒जाप॑तिः प्र॒जाः । \newline
12. प्र॒जाप॑तिः प्र॒जाः प्र॒जाः प्र॒जाप॑तिः प्र॒जाप॑तिः प्र॒जा अ॑सृजता सृजत प्र॒जाः प्र॒जाप॑तिः प्र॒जाप॑तिः प्र॒जा अ॑सृजत । \newline
13. प्र॒जाप॑ति॒रिति॑ प्र॒जा - प॒तिः॒ । \newline
14. प्र॒जा अ॑सृजता सृजत प्र॒जाः प्र॒जा अ॑सृजत॒ तेन॒ तेना॑सृजत प्र॒जाः प्र॒जा अ॑सृजत॒ तेन॑ । \newline
15. प्र॒जा इति॑ प्र - जाः । \newline
16. अ॒सृ॒ज॒त॒ तेन॒ तेना॑ सृजता सृजत॒ तेनै॒वैव तेना॑सृजता सृजत॒ तेनै॒व । \newline
17. तेनै॒वैव तेन॒ तेनै॒व प्र॒जाः प्र॒जा ए॒व तेन॒ तेनै॒व प्र॒जाः । \newline
18. ए॒व प्र॒जाः प्र॒जा ए॒वैव प्र॒जाः सृ॑जते सृजते प्र॒जा ए॒वैव प्र॒जाः सृ॑जते । \newline
19. प्र॒जाः सृ॑जते सृजते प्र॒जाः प्र॒जाः सृ॑जते॒ नरा॒शꣳस॑स्य॒ नरा॒शꣳस॑स्य सृजते प्र॒जाः प्र॒जाः सृ॑जते॒ नरा॒शꣳस॑स्य । \newline
20. प्र॒जा इति॑ प्र - जाः । \newline
21. सृ॒ज॒ते॒ नरा॒शꣳस॑स्य॒ नरा॒शꣳस॑स्य सृजते सृजते॒ नरा॒शꣳस॑स्या॒ह म॒हम् नरा॒शꣳस॑स्य सृजते सृजते॒ नरा॒शꣳस॑स्या॒हम् । \newline
22. नरा॒शꣳस॑स्या॒ह म॒हम् नरा॒शꣳस॑स्य॒ नरा॒शꣳस॑स्या॒हम् दे॑वय॒ज्यया॑ देवय॒ज्यया॒ ऽहम् नरा॒शꣳस॑स्य॒ नरा॒शꣳस॑स्या॒हम् दे॑वय॒ज्यया᳚ । \newline
23. अ॒हम् दे॑वय॒ज्यया॑ देवय॒ज्यया॒ ऽह म॒हम् दे॑वय॒ज्यया॑ पशु॒मान् प॑शु॒मान् दे॑वय॒ज्यया॒ ऽह म॒हम् दे॑वय॒ज्यया॑ पशु॒मान् । \newline
24. दे॒व॒य॒ज्यया॑ पशु॒मान् प॑शु॒मान् दे॑वय॒ज्यया॑ देवय॒ज्यया॑ पशु॒मान् भू॑यासम् भूयासम् पशु॒मान् दे॑वय॒ज्यया॑ देवय॒ज्यया॑ पशु॒मान् भू॑यासम् । \newline
25. दे॒व॒य॒ज्ययेति॑ देव - य॒ज्यया᳚ । \newline
26. प॒शु॒मान् भू॑यासम् भूयासम् पशु॒मान् प॑शु॒मान् भू॑यास॒ मितीति॑ भूयासम् पशु॒मान् प॑शु॒मान् भू॑यास॒ मिति॑ । \newline
27. प॒शु॒मानिति॑ पशु - मान् । \newline
28. भू॒या॒स॒ मितीति॑ भूयासम् भूयास॒ मित्या॑हा॒हे ति॑ भूयासम् भूयास॒ मित्या॑ह । \newline
29. इत्या॑हा॒हे तीत्या॑ह॒ नरा॒शꣳसे॑न॒ नरा॒शꣳसे॑ना॒हे तीत्या॑ह॒ नरा॒शꣳसे॑न । \newline
30. आ॒ह॒ नरा॒शꣳसे॑न॒ नरा॒शꣳसे॑नाहाह॒ नरा॒शꣳसे॑न॒ वै वै नरा॒शꣳसे॑नाहाह॒ नरा॒शꣳसे॑न॒ वै । \newline
31. नरा॒शꣳसे॑न॒ वै वै नरा॒शꣳसे॑न॒ नरा॒शꣳसे॑न॒ वै प्र॒जाप॑तिः प्र॒जाप॑ति॒र् वै नरा॒शꣳसे॑न॒ नरा॒शꣳसे॑न॒ वै प्र॒जाप॑तिः । \newline
32. वै प्र॒जाप॑तिः प्र॒जाप॑ति॒र् वै वै प्र॒जाप॑तिः प॒शून् प॒शून् प्र॒जाप॑ति॒र् वै वै प्र॒जाप॑तिः प॒शून् । \newline
33. प्र॒जाप॑तिः प॒शून् प॒शून् प्र॒जाप॑तिः प्र॒जाप॑तिः प॒शू न॑सृजता सृजत प॒शून् प्र॒जाप॑तिः प्र॒जाप॑तिः प॒शू न॑सृजत । \newline
34. प्र॒जाप॑ति॒रिति॑ प्र॒जा - प॒तिः॒ । \newline
35. प॒शू न॑सृजता सृजत प॒शून् प॒शू न॑सृजत॒ तेन॒ तेना॑सृजत प॒शून् प॒शू न॑सृजत॒ तेन॑ । \newline
36. अ॒सृ॒ज॒त॒ तेन॒ तेना॑सृजता सृजत॒ तेनै॒वैव तेना॑सृजता सृजत॒ तेनै॒व । \newline
37. तेनै॒वैव तेन॒ तेनै॒व प॒शून् प॒शू ने॒व तेन॒ तेनै॒व प॒शून् । \newline
38. ए॒व प॒शून् प॒शू ने॒वैव प॒शून् थ्सृ॑जते सृजते प॒शू ने॒वैव प॒शून् थ्सृ॑जते । \newline
39. प॒शून् थ्सृ॑जते सृजते प॒शून् प॒शून् थ्सृ॑जते॒ ऽग्नेर॒ग्नेः सृ॑जते प॒शून् प॒शून् थ्सृ॑जते॒ ऽग्नेः । \newline
40. सृ॒ज॒ते॒ ऽग्नेर॒ग्नेः सृ॑जते सृजते॒ ऽग्नेः स्वि॑ष्ट॒कृतः॑ स्विष्ट॒कृतो॒ ऽग्नेः सृ॑जते सृजते॒ ऽग्नेः स्वि॑ष्ट॒कृतः॑ । \newline
41. अ॒ग्नेः स्वि॑ष्ट॒कृतः॑ स्विष्ट॒कृतो॒ ऽग्नेर॒ग्नेः स्वि॑ष्ट॒कृतो॒ ऽह म॒हꣳ स्वि॑ष्ट॒कृतो॒ ऽग्नेर॒ग्नेः स्वि॑ष्ट॒कृतो॒ ऽहम् । \newline
42. स्वि॒ष्ट॒कृतो॒ ऽह म॒हꣳ स्वि॑ष्ट॒कृतः॑ स्विष्ट॒कृतो॒ ऽहम् दे॑वय॒ज्यया॑ देवय॒ज्यया॒ ऽहꣳ स्वि॑ष्ट॒कृतः॑ स्विष्ट॒कृतो॒ ऽहम् दे॑वय॒ज्यया᳚ । \newline
43. स्वि॒ष्ट॒कृत॒ इति॑ स्विष्ट - कृतः॑ । \newline
44. अ॒हम् दे॑वय॒ज्यया॑ देवय॒ज्यया॒ ऽह म॒हम् दे॑वय॒ज्यया ऽऽयु॑ष्मा॒ नायु॑ष्मान् देवय॒ज्यया॒ ऽह म॒हम् दे॑वय॒ज्यया ऽऽयु॑ष्मान् । \newline
45. दे॒व॒य॒ज्यया ऽऽयु॑ष्मा॒ नायु॑ष्मान् देवय॒ज्यया॑ देवय॒ज्यया ऽऽयु॑ष्मान्. य॒ज्ञेन॑ य॒ज्ञेनायु॑ष्मान् देवय॒ज्यया॑ देवय॒ज्यया ऽऽयु॑ष्मान्. य॒ज्ञेन॑ । \newline
46. दे॒व॒य॒ज्ययेति॑ देव - य॒ज्यया᳚ । \newline
47. आयु॑ष्मान्. य॒ज्ञेन॑ य॒ज्ञेनायु॑ष्मा॒ नायु॑ष्मान्. य॒ज्ञेन॑ प्रति॒ष्ठाम् प्र॑ति॒ष्ठां ॅय॒ज्ञेनायु॑ष्मा॒ नायु॑ष्मान्. य॒ज्ञेन॑ प्रति॒ष्ठाम् । \newline
48. य॒ज्ञेन॑ प्रति॒ष्ठाम् प्र॑ति॒ष्ठां ॅय॒ज्ञेन॑ य॒ज्ञेन॑ प्रति॒ष्ठाम् ग॑मेयम् गमेयम् प्रति॒ष्ठां ॅय॒ज्ञेन॑ य॒ज्ञेन॑ प्रति॒ष्ठाम् ग॑मेयम् । \newline
49. प्र॒ति॒ष्ठाम् ग॑मेयम् गमेयम् प्रति॒ष्ठाम् प्र॑ति॒ष्ठाम् ग॑मेय॒ मितीति॑ गमेयम् प्रति॒ष्ठाम् प्र॑ति॒ष्ठाम् ग॑मेय॒ मिति॑ । \newline
50. प्र॒ति॒ष्ठामिति॑ प्रति - स्थाम् । \newline
51. ग॒मे॒य॒ मितीति॑ गमेयम् गमेय॒ मित्या॑हा॒हे ति॑ गमेयम् गमेय॒ मित्या॑ह । \newline
52. इत्या॑हा॒हे ती त्या॒हायु॒ रायु॑ रा॒हे ती त्या॒हायुः॑ । \newline
53. आ॒हायु॒ रायु॑ राहा॒हायु॑ रे॒वैवायु॑ राहा॒हायु॑रे॒व । \newline
54. आयु॑ रे॒वैवायु॒ रायु॑ रे॒वात्मन् ना॒त्मन् ने॒वायु॒ रायु॑ रे॒वात्मन्न् । \newline
55. ए॒वात्मन् ना॒त्मन् ने॒वैवात्मन् ध॑त्ते धत्त आ॒त्मन् ने॒वैवात्मन् ध॑त्ते । \newline
56. आ॒त्मन् ध॑त्ते धत्त आ॒त्मन् ना॒त्मन् ध॑त्ते॒ प्रति॒ प्रति॑ धत्त आ॒त्मन् ना॒त्मन् ध॑त्ते॒ प्रति॑ । \newline
57. ध॒त्ते॒ प्रति॒ प्रति॑ धत्ते धत्ते॒ प्रति॑ य॒ज्ञेन॑ य॒ज्ञेन॒ प्रति॑ धत्ते धत्ते॒ प्रति॑ य॒ज्ञेन॑ । \newline
58. प्रति॑ य॒ज्ञेन॑ य॒ज्ञेन॒ प्रति॒ प्रति॑ य॒ज्ञेन॑ तिष्ठति तिष्ठति य॒ज्ञेन॒ प्रति॒ प्रति॑ य॒ज्ञेन॑ तिष्ठति । \newline
59. य॒ज्ञेन॑ तिष्ठति तिष्ठति य॒ज्ञेन॑ य॒ज्ञेन॑ तिष्ठति दर्.शपूर्णमा॒सयो᳚र् दर्.शपूर्णमा॒सयो᳚ स्तिष्ठति य॒ज्ञेन॑ य॒ज्ञेन॑ तिष्ठति दर्.शपूर्णमा॒सयोः᳚ । \newline
60. ति॒ष्ठ॒ति॒ द॒र्॒.श॒पू॒र्ण॒मा॒सयो᳚र् दर्.शपूर्णमा॒सयो᳚ स्तिष्ठति तिष्ठति दर्.शपूर्णमा॒सयो॒र् 
वै वै द॑र्.शपूर्णमा॒सयो᳚ स्तिष्ठति तिष्ठति दर्.शपूर्णमा॒सयो॒र् वै । \newline
61. द॒र्॒.श॒पू॒र्ण॒मा॒सयो॒र् वै वै द॑र्.शपूर्णमा॒सयो᳚र् दर्.शपूर्णमा॒सयो॒र् वै दे॒वा दे॒वा वै 
द॑र्.शपूर्णमा॒सयो᳚र् दर्.शपूर्णमा॒सयो॒र् वै दे॒वाः । \newline
62. द॒र्॒.श॒पू॒र्ण॒मा॒सयो॒रिति॑ दर्.श - पू॒र्ण॒मा॒सयोः᳚ । \newline
\pagebreak
\markright{ TS 1.7.4.2  \hfill https://www.vedavms.in \hfill}
\addcontentsline{toc}{section}{ TS 1.7.4.2 }
\section*{ TS 1.7.4.2 }

\textbf{TS 1.7.4.2 } \newline
\textbf{Samhita Paata} \newline

वै दे॒वा उज्जि॑ति॒-मनूद॑जयन् दर्.शपूर्णमा॒साभ्या॒- मसु॑रा॒नपा॑-नुदन्ता॒ग्ने-र॒हमुज्जि॑ति॒-मनूज्जे॑ष॒-मित्या॑ह दर्.शपूर्णमा॒सयो॑रे॒व दे॒वता॑नां॒ ॅयज॑मान॒ उज्जि॑ति॒मनूज्ज॑यति दर्.शपूर्णमा॒साभ्यां॒ भ्रातृ॑व्या॒नप॑ नुदते॒ वाज॑वतीभ्यां॒ ॅव्यू॑ह॒त्यन्नं॒ ॅवै वाजोऽन्न॑मे॒वाव॑ रुन्धे॒ द्वाभ्यां॒ प्रति॑ष्ठित्यै॒ यो वै य॒ज्ञ्स्य॒ द्वौ दोहौ॑ वि॒द्वान् यज॑त उभ॒यत॑ - [ ] \newline

\textbf{Pada Paata} \newline

वै । दे॒वाः । उज्जि॑ति॒मित्युत् - जि॒ति॒म् । अनु॑ । उदिति॑ । अ॒ज॒य॒न्न् । द॒र्॒.श॒पू॒र्ण॒मा॒साभ्या॒मिति॑ दर्.श-पू॒र्ण॒मा॒साभ्या᳚म् । असु॑रान् । अपेति॑ । अ॒नु॒द॒न्त॒ । अ॒ग्नेः । अ॒हम् । उज्जि॑ति॒मित्युत् - जि॒ति॒म् । अनु॑ । उदिति॑ । जे॒ष॒म् । इति॑ । आ॒ह॒ । द॒र्॒.श॒पू॒र्ण॒मा॒सयो॒रिति॑ दर्.श - पू॒र्ण॒मा॒सयोः᳚ । ए॒व । दे॒वता॑नाम् । यज॑मानः । उज्जि॑ति॒मित्युत् - जि॒ति॒म् । अनु॑ । उदिति॑ । ज॒य॒ति॒ । द॒र॒.श॒पू॒र्ण॒मा॒साभ्या॒मिति॑ दर्.श-पू॒र्ण॒मा॒साभ्या᳚म् । भ्रातृ॑व्यान् । अपेति॑ । नु॒द॒ते॒ । वाज॑वतीभ्या॒मिति॒ वाज॑ - व॒ती॒भ्या॒म् । वीति॑ । ऊ॒ह॒ति॒ । अन्न᳚म् । वै । वाजः॑ । अन्न᳚म् । ए॒व । अवेति॑ । रु॒न्धे॒ । द्वाभ्या᳚म् । प्रति॑ष्ठित्या॒ इति॒ प्रति॑ - स्थि॒त्यै॒ । यः । वै । य॒ज्ञ्स्य॑ । द्वौ । दोहौ᳚ । वि॒द्वान् । यज॑ते । उ॒भ॒यतः॑ ।  \newline


\textbf{Krama Paata} \newline

वै दे॒वाः । दे॒वा उज्जि॑तिम् । उज्जि॑ति॒मनु॑ । उज्जि॑ति॒मित्युत् - जि॒ति॒म् । अनूत् । उद॑जयन्न् । अ॒ज॒य॒न्,द॒र्॒.श॒पू॒र्ण॒मा॒साभ्या᳚म् । द॒र्॒.श॒पू॒र्ण॒मा॒साभ्या॒मसु॑रान् । द॒र्॒.श॒पू॒र्ण॒मा॒साभ्या॒मिति॑ दर्.श - पू॒र्ण॒मा॒साभ्या᳚म् । असु॑रा॒नप॑ । अपा॑नुदन्त । अ॒नु॒द॒न्ता॒ग्नेः । अ॒ग्नेर॒हम् । अ॒हमुज्जि॑तिम् । उज्जि॑ति॒मनु॑ । उज्जि॑ति॒मित्युत् - जि॒ति॒म् । अनूत् । उज्जे॑षम् । जे॒ष॒मिति॑ । इत्या॑ह । आ॒ह॒ द॒र्॒.श॒पू॒र्ण॒मा॒सयोः᳚ । द॒र्॒.श॒पू॒र्ण॒मा॒सयो॑रे॒व । द॒र्॒.श॒पू॒र्ण॒मा॒सयो॒रिति॑ दर्.श - पू॒र्ण॒मा॒सयोः᳚ । ए॒व दे॒वता॑नाम् । दे॒वता॑नां॒ ॅयज॑मानः । यज॑मान॒ उज्जि॑तिम् । उज्जि॑ति॒मनु॑ । उज्जि॑ति॒मित्युत् - जि॒ति॒म् । अनूत् । उज्ज॑यति । ज॒य॒ति॒ द॒र्॒.श॒पू॒र्ण॒मा॒साभ्या᳚म् । द॒र्॒.श॒पू॒र्ण॒मा॒साभ्या॒म् भ्रातृ॑व्यान् । द॒र्॒.श॒पू॒र्ण॒मा॒साभ्या॒मिति॑ दर्.श - पू॒र्ण॒मा॒साभ्या᳚म् । भ्रातृ॑व्या॒नप॑ । अप॑ नुदते । नु॒द॒ते॒ वाज॑वतीभ्याम् । वाज॑वतीभ्यां॒ ॅवि । वाज॑वतीभ्या॒मिति॒ वाज॑ - व॒ती॒भ्या॒म् । व्यू॑हति । ऊ॒ह॒त्यन्न᳚म् । अन्नं॒ ॅवै । वै वाजः॑ । वाजोऽन्न᳚म् । अन्न॑मे॒व । ए॒वाव॑ । अव॑ रुन्धे । रु॒न्धे॒ द्वाभ्या᳚म् । द्वाभ्या॒म् प्रति॑ष्ठित्यै । प्रति॑ष्ठित्यै॒ यः । प्रति॑ष्ठित्या॒ इति॒ प्रति॑ - स्थि॒त्यै॒ । यो वै । वै य॒ज्ञ्स्य॑ । य॒ज्ञ्स्य॒ द्वौ । द्वौ दोहौ᳚ । दोहौ॑ वि॒द्वान् । वि॒द्वान्. यज॑ते । यज॑त उभ॒यतः॑ । उ॒भ॒यत॑ ए॒व \newline

\textbf{Jatai Paata} \newline

1. वै दे॒वा दे॒वा वै वै दे॒वाः । \newline
2. दे॒वा उज्जि॑ति॒ मुज्जि॑तिम् दे॒वा दे॒वा उज्जि॑तिम् । \newline
3. उज्जि॑ति॒ मन्वनूज्जि॑ति॒ मुज्जि॑ति॒ मनु॑ । \newline
4. उज्जि॑ति॒मित्युत् - जि॒ति॒म् । \newline
5. अनू दु दन्वनूत् । \newline
6. उद॑जयन् नजय॒न् नुदु द॑जयन्न् । \newline
7. अ॒ज॒य॒न् द॒र्॒.श॒पू॒र्ण॒मा॒साभ्या᳚म् दर्.शपूर्णमा॒साभ्या॑ मजयन् नजयन् दर्.शपूर्णमा॒साभ्या᳚म् । \newline
8. द॒र्॒.श॒पू॒र्ण॒मा॒साभ्या॒ मसु॑रा॒ नसु॑रान् दर्.शपूर्णमा॒साभ्या᳚म् दर्.शपूर्णमा॒साभ्या॒ मसु॑रान् । \newline
9. द॒र्॒.श॒पू॒र्ण॒मा॒साभ्या॒मिति॑ दर्.श - पू॒र्ण॒मा॒साभ्या᳚म् । \newline
10. असु॑रा॒ नपापा सु॑रा॒ नसु॑रा॒ नप॑ । \newline
11. अपा॑नुदन्ता नुद॒न्ता पापा॑नुदन्त । \newline
12. अ॒नु॒द॒न्ता॒ ग्ने र॒ग्ने र॑नुदन्ता नुदन्ता॒ ग्नेः । \newline
13. अ॒ग्ने र॒ह म॒ह म॒ग्ने र॒ग्ने र॒हम् । \newline
14. अ॒ह मुज्जि॑ति॒ मुज्जि॑ति म॒ह म॒ह मुज्जि॑तिम् । \newline
15. उज्जि॑ति॒ मन्वनू ज्जि॑ति॒ मुज्जि॑ति॒ मनु॑ । \newline
16. उज्जि॑ति॒मित्युत् - जि॒ति॒म् । \newline
17. अनू दुदन्व नूत् । \newline
18. उज् जे॑षम् जेष॒ मुदुज् जे॑षम् । \newline
19. जे॒ष॒ मितीति॑ जेषम् जेष॒ मिति॑ । \newline
20. इत्या॑हा॒हे तीत्या॑ह । \newline
21. आ॒ह॒ द॒र्॒.श॒पू॒र्ण॒मा॒सयो᳚र् दर्.शपूर्णमा॒सयो॑राहाह दर्.शपूर्णमा॒सयोः᳚ । \newline
22. द॒र्॒.श॒पू॒र्ण॒मा॒सयो॑ रे॒वैव द॑र्.शपूर्णमा॒सयो᳚र् दर्.शपूर्णमा॒सयो॑ रे॒व । \newline
23. द॒र्॒.श॒पू॒र्ण॒मा॒सयो॒रिति॑ दर्.श - पू॒र्ण॒मा॒सयोः᳚ । \newline
24. ए॒व दे॒वता॑नाम् दे॒वता॑ना मे॒वैव दे॒वता॑नाम् । \newline
25. दे॒वता॑नां॒ ॅयज॑मानो॒ यज॑मानो दे॒वता॑नाम् दे॒वता॑नां॒ ॅयज॑मानः । \newline
26. यज॑मान॒ उज्जि॑ति॒ मुज्जि॑तिं॒ ॅयज॑मानो॒ यज॑मान॒ उज्जि॑तिम् । \newline
27. उज्जि॑ति॒ मन्वनूज्जि॑ति॒ मुज्जि॑ति॒ मनु॑ । \newline
28. उज्जि॑ति॒मित्युत् - जि॒ति॒म् । \newline
29. अनू दु दन्व नूत् । \newline
30. उज् ज॑यति जय॒ त्युदुज् ज॑यति । \newline
31. ज॒य॒ति॒ द॒र्॒.श॒पू॒र्ण॒मा॒साभ्या᳚म् दर्.शपूर्णमा॒साभ्या᳚म् जयति जयति दरशपूर्णमा॒साभ्या᳚म् । \newline
32. द॒र्॒.श॒पू॒र्ण॒मा॒साभ्या॒म् भ्रातृ॑व्या॒न् भ्रातृ॑व्यान् दर्.शपूर्णमा॒साभ्या᳚म् दर्.शपूर्णमा॒साभ्या॒म् भ्रातृ॑व्यान् । \newline
33. द॒र्॒.श॒पू॒र्ण॒मा॒साभ्या॒मिति॑ दर्.श - पू॒र्ण॒मा॒साभ्या᳚म् । \newline
34. भ्रातृ॑व्या॒ नपाप॒ भ्रातृ॑व्या॒न् भ्रातृ॑व्या॒ नप॑ । \newline
35. अप॑ नुदते नुद॒ते ऽपाप॑ नुदते । \newline
36. नु॒द॒ते॒ वाज॑वतीभ्यां॒ ॅवाज॑वतीभ्यान् नुदते नुदते॒ वाज॑वतीभ्याम् । \newline
37. वाज॑वतीभ्यां॒ ॅवि वि वाज॑वतीभ्यां॒ ॅवाज॑वतीभ्यां॒ ॅवि । \newline
38. वाज॑वतीभ्या॒मिति॒ वाज॑ - व॒ती॒भ्या॒म् । \newline
39. व्यू॑ह त्यूहति॒ वि व्यू॑हति । \newline
40. ऊ॒ह॒ त्यन्न॒ मन्न॑ मूह त्यूह॒ त्यन्न᳚म् । \newline
41. अन्नं॒ ॅवै वा अन्न॒ मन्नं॒ ॅवै । \newline
42. वै वाजो॒ वाजो॒ वै वै वाजः॑ । \newline
43. वाजो ऽन्न॒ मन्नं॒ ॅवाजो॒ वाजो ऽन्न᳚म् । \newline
44. अन्न॑ मे॒ वैवान्न॒ मन्न॑ मे॒व । \newline
45. ए॒वावा वै॒वै वाव॑ । \newline
46. अव॑ रुन्धे रु॒न्धे ऽवाव॑ रुन्धे । \newline
47. रु॒न्धे॒ द्वाभ्या॒म् द्वाभ्याꣳ॑ रुन्धे रुन्धे॒ द्वाभ्या᳚म् । \newline
48. द्वाभ्या॒म् प्रति॑ष्ठित्यै॒ प्रति॑ष्ठित्यै॒ द्वाभ्या॒म् द्वाभ्या॒म् प्रति॑ष्ठित्यै । \newline
49. प्रति॑ष्ठित्यै॒ यो यः प्रति॑ष्ठित्यै॒ प्रति॑ष्ठित्यै॒ यः । \newline
50. प्रति॑ष्ठित्या॒ इति॒ प्रति॑ - स्थि॒त्यै॒ । \newline
51. यो वै वै यो यो वै । \newline
52. वै य॒ज्ञ्स्य॑ य॒ज्ञ्स्य॒ वै वै य॒ज्ञ्स्य॑ । \newline
53. य॒ज्ञ्स्य॒ द्वौ द्वौ य॒ज्ञ्स्य॑ य॒ज्ञ्स्य॒ द्वौ । \newline
54. द्वौ दोहौ॒ दोहौ॒ द्वौ द्वौ दोहौ᳚ । \newline
55. दोहौ॑ वि॒द्वान्. वि॒द्वान् दोहौ॒ दोहौ॑ वि॒द्वान् । \newline
56. वि॒द्वान्. यज॑ते॒ यज॑ते वि॒द्वान्. वि॒द्वान्. यज॑ते । \newline
57. यज॑त उभ॒यत॑ उभ॒यतो॒ यज॑ते॒ यज॑त उभ॒यतः॑ । \newline
58. उ॒भ॒यत॑ ए॒वैवो भ॒यत॑ उभ॒यत॑ ए॒व । \newline

\textbf{Ghana Paata } \newline

1. वै दे॒वा दे॒वा वै वै दे॒वा उज्जि॑ति॒ मुज्जि॑तिम् दे॒वा वै वै दे॒वा उज्जि॑तिम् । \newline
2. दे॒वा उज्जि॑ति॒ मुज्जि॑तिम् दे॒वा दे॒वा उज्जि॑ति॒ मन्वनूज्जि॑तिम् दे॒वा दे॒वा उज्जि॑ति॒ मनु॑ । \newline
3. उज्जि॑ति॒ मन्व नूज्जि॑ति॒ मुज्जि॑ति॒ मनू दुदनूज्जि॑ति॒ मुज्जि॑ति॒ मनूत् । \newline
4. उज्जि॑ति॒मित्युत् - जि॒ति॒म् । \newline
5. अनू दुदन्वनू द॑जयन् नजय॒न् नुदन्वनू द॑जयन्न् । \newline
6. उद॑जयन् नजय॒न् नुदुद॑जयन् दर्.शपूर्णमा॒साभ्या᳚म् दर्.शपूर्णमा॒साभ्या॑ मजय॒न् नुदुद॑जयन् दर्.शपूर्णमा॒साभ्या᳚म् । \newline
7. अ॒ज॒य॒न् द॒र्॒.श॒पू॒र्ण॒मा॒साभ्या᳚म् दर्.शपूर्णमा॒साभ्या॑ मजयन् नजयन् दर्.शपूर्णमा॒साभ्या॒ मसु॑रा॒ नसु॑रान् दर्.शपूर्णमा॒साभ्या॑ मजयन् नजयन् दर्.शपूर्णमा॒साभ्या॒ मसु॑रान् । \newline
8. द॒र्॒.श॒पू॒र्ण॒मा॒साभ्या॒ मसु॑रा॒ नसु॑रान् दर्.शपूर्णमा॒साभ्या᳚म् दर्.शपूर्णमा॒साभ्या॒ मसु॑रा॒ नपापासु॑रान् दर्.शपूर्णमा॒साभ्या᳚म् दर्.शपूर्णमा॒साभ्या॒ मसु॑रा॒ नप॑ । \newline
9. द॒र्॒.श॒पू॒र्ण॒मा॒साभ्या॒मिति॑ दर्.श - पू॒र्ण॒मा॒साभ्या᳚म् । \newline
10. असु॑रा॒ नपापासु॑रा॒ नसु॑रा॒ नपा॑नुदन्ता नुद॒न्तापासु॑रा॒ नसु॑रा॒ नपा॑नुदन्त । \newline
11. अपा॑नुदन्ता नुद॒न्तापापा॑ नुदन्ता॒ ग्नेर॒ग्ने र॑नुद॒न्तापापा॑ नुदन्ता॒ग्नेः । \newline
12. अ॒नु॒द॒न्ता॒ ग्नेर॒ग्ने र॑नुदन्तानुदन्ता॒ ग्नेर॒ह म॒ह म॒ग्ने र॑नुदन्तानुदन्ता॒ ग्नेर॒हम् । \newline
13. अ॒ग्नेर॒ह म॒ह म॒ग्ने र॒ग्नेर॒ह मुज्जि॑ति॒ मुज्जि॑ति म॒ह म॒ग्ने र॒ग्नेर॒ह मुज्जि॑तिम् । \newline
14. अ॒ह मुज्जि॑ति॒ मुज्जि॑ति म॒ह म॒ह मुज्जि॑ति॒ मन्वनूज्जि॑ति म॒ह म॒ह मुज्जि॑ति॒ मनु॑ । \newline
15. उज्जि॑ति॒ मन्वनूज्जि॑ति॒ मुज्जि॑ति॒ मनू दुदनूज्जि॑ति॒ मुज्जि॑ति॒ मनूत् । \newline
16. उज्जि॑ति॒मित्युत् - जि॒ति॒म् । \newline
17. अनू दुदन्वनूज् जे॑षम् जेष॒ मुदन्वनूज् जे॑षम् । \newline
18. उज् जे॑षम् जेष॒ मुदुज् जे॑ष॒ मितीति॑ जेष॒ मुदुज् जे॑ष॒ मिति॑ । \newline
19. जे॒ष॒ मितीति॑ जेषम् जेष॒ मित्या॑हा॒हे ति॑ जेषम् जेष॒ मित्या॑ह । \newline
20. इत्या॑हा॒हे तीत्या॑ह दर्.शपूर्णमा॒सयो᳚र् दर्.शपूर्णमा॒सयो॑रा॒हे तीत्या॑ह दर्.शपूर्णमा॒सयोः᳚ । \newline
21. आ॒ह॒ द॒र्॒.श॒पू॒र्ण॒मा॒सयो᳚र् दर्.शपूर्णमा॒सयो॑ राहाह दर्.शपूर्णमा॒सयो॑रे॒वैव द॑र्.शपूर्णमा॒सयो॑राहाह दर्.शपूर्णमा॒सयो॑ रे॒व । \newline
22. द॒र्॒.श॒पू॒र्ण॒मा॒सयो॑ रे॒वैव द॑र्.शपूर्णमा॒सयो᳚र् दर्.शपूर्णमा॒सयो॑ रे॒व दे॒वता॑नाम् दे॒वता॑ना मे॒व द॑र्.शपूर्णमा॒सयो᳚र् दर्.शपूर्णमा॒सयो॑ रे॒व दे॒वता॑नाम् । \newline
23. द॒र्॒.श॒पू॒र्ण॒मा॒सयो॒रिति॑ दर्.श - पू॒र्ण॒मा॒सयोः᳚ । \newline
24. ए॒व दे॒वता॑नाम् दे॒वता॑ना मे॒वैव दे॒वता॑नां॒ ॅयज॑मानो॒ यज॑मानो दे॒वता॑ना मे॒वैव दे॒वता॑नां॒ ॅयज॑मानः । \newline
25. दे॒वता॑नां॒ ॅयज॑मानो॒ यज॑मानो दे॒वता॑नाम् दे॒वता॑नां॒ ॅयज॑मान॒ उज्जि॑ति॒ मुज्जि॑तिं॒ ॅयज॑मानो दे॒वता॑नाम् दे॒वता॑नां॒ ॅयज॑मान॒ उज्जि॑तिम् । \newline
26. यज॑मान॒ उज्जि॑ति॒ मुज्जि॑तिं॒ ॅयज॑मानो॒ यज॑मान॒ उज्जि॑ति॒ मन्व नूज्जि॑तिं॒ ॅयज॑मानो॒ यज॑मान॒ उज्जि॑ति॒ मनु॑ । \newline
27. उज्जि॑ति॒ मन्वनूज्जि॑ति॒ मुज्जि॑ति॒ मनू दुदनूज्जि॑ति॒ मुज्जि॑ति॒ मनूत् । \newline
28. उज्जि॑ति॒मित्युत् - जि॒ति॒म् । \newline
29. अनू दुदन्वनूज् ज॑यति जय॒ त्युदन्वनूज् ज॑यति । \newline
30. उज् ज॑यति जय॒त्युदुज् ज॑यति दर्.शपूर्णमा॒साभ्या᳚म् दर्.शपूर्णमा॒साभ्या᳚म् जय॒त्युदुज् ज॑यति दर्.शपूर्णमा॒साभ्या᳚म् । \newline
31. ज॒य॒ति॒ द॒र्॒.श॒पू॒र्ण॒मा॒साभ्या᳚म् दर्.शपूर्णमा॒साभ्या᳚म् जयति जयति दर्.शपूर्णमा॒साभ्या॒म् भ्रातृ॑व्या॒न् 
भ्रातृ॑व्यान् दर्.शपूर्णमा॒साभ्या᳚म् जयति जयति दर्.शपूर्णमा॒साभ्या॒म् भ्रातृ॑व्यान् । \newline
32. द॒र्॒श॒पू॒र्ण॒मा॒साभ्या॒म् भ्रातृ॑व्या॒न् भ्रातृ॑व्यान् दर्.शपूर्णमा॒साभ्या᳚म् दर्.शपूर्णमा॒साभ्या॒म् भ्रातृ॑व्या॒ नपाप॒ भ्रातृ॑व्यान् दर्.शपूर्णमा॒साभ्या᳚म् दर्.शपूर्णमा॒साभ्या॒म् भ्रातृ॑व्या॒ नप॑ । \newline
33. द॒र्॒.श॒पू॒र्ण॒मा॒साभ्या॒मिति॑ दर्.श - पू॒र्ण॒मा॒साभ्या᳚म् । \newline
34. भ्रातृ॑व्या॒ नपाप॒ भ्रातृ॑व्या॒न् भ्रातृ॑व्या॒ नप॑ नुदते नुद॒ते ऽप॒ भ्रातृ॑व्या॒न् भ्रातृ॑व्या॒ नप॑ नुदते । \newline
35. अप॑ नुदते नुद॒ते ऽपाप॑ नुदते॒ वाज॑वतीभ्यां॒ ॅवाज॑वतीभ्याम् नुद॒ते ऽपाप॑ नुदते॒ वाज॑वतीभ्याम् । \newline
36. नु॒द॒ते॒ वाज॑वतीभ्यां॒ ॅवाज॑वतीभ्याम् नुदते नुदते॒ वाज॑वतीभ्यां॒ ॅवि वि वाज॑वतीभ्याम् नुदते नुदते॒ वाज॑वतीभ्यां॒ ॅवि । \newline
37. वाज॑वतीभ्यां॒ ॅवि वि वाज॑वतीभ्यां॒ ॅवाज॑वतीभ्यां॒ ॅव्यू॑हत्यूहति॒ वि वाज॑वतीभ्यां॒ ॅवाज॑वतीभ्यां॒ ॅव्यू॑हति । \newline
38. वाज॑वतीभ्या॒मिति॒ वाज॑ - व॒ती॒भ्या॒म् । \newline
39. व्यू॑हत्यूहति॒ वि व्यू॑ह॒त्यन्न॒ मन्न॑ मूहति॒ वि व्यू॑ह॒त्यन्न᳚म् । \newline
40. ऊ॒ह॒त्यन्न॒ मन्न॑ मूहत्यूह॒त्यन्नं॒ ॅवै वा अन्न॑ मूहत्यूह॒त्यन्नं॒ ॅवै । \newline
41. अन्नं॒ ॅवै वा अन्न॒ मन्नं॒ ॅवै वाजो॒ वाजो॒ वा अन्न॒ मन्नं॒ ॅवै वाजः॑ । \newline
42. वै वाजो॒ वाजो॒ वै वै वाजो ऽन्न॒ मन्नं॒ ॅवाजो॒ वै वै वाजो ऽन्न᳚म् । \newline
43. वाजो ऽन्न॒ मन्नं॒ ॅवाजो॒ वाजो ऽन्न॑ मे॒वैवान्नं॒ ॅवाजो॒ वाजो ऽन्न॑ मे॒व । \newline
44. अन्न॑ मे॒वैवान्न॒ मन्न॑ मे॒वावा वै॒वान्न॒ मन्न॑ मे॒वाव॑ । \newline
45. ए॒वावा वै॒वैवाव॑ रुन्धे रु॒न्धे ऽवै॒वैवाव॑ रुन्धे । \newline
46. अव॑ रुन्धे रु॒न्धे ऽवाव॑ रुन्धे॒ द्वाभ्या॒म् द्वाभ्या(ग्म्॑) रु॒न्धे ऽवाव॑ रुन्धे॒ द्वाभ्या᳚म् । \newline
47. रु॒न्धे॒ द्वाभ्या॒म् द्वाभ्या(ग्म्॑) रुन्धे रुन्धे॒ द्वाभ्या॒म् प्रति॑ष्ठित्यै॒ प्रति॑ष्ठित्यै॒ द्वाभ्या(ग्म्॑) रुन्धे रुन्धे॒ द्वाभ्या॒म् प्रति॑ष्ठित्यै । \newline
48. द्वाभ्या॒म् प्रति॑ष्ठित्यै॒ प्रति॑ष्ठित्यै॒ द्वाभ्या॒म् द्वाभ्या॒म् प्रति॑ष्ठित्यै॒ यो यः प्रति॑ष्ठित्यै॒ द्वाभ्या॒म् द्वाभ्या॒म् प्रति॑ष्ठित्यै॒ यः । \newline
49. प्रति॑ष्ठित्यै॒ यो यः प्रति॑ष्ठित्यै॒ प्रति॑ष्ठित्यै॒ यो वै वै यः प्रति॑ष्ठित्यै॒ प्रति॑ष्ठित्यै॒ यो वै । \newline
50. प्रति॑ष्ठित्या॒ इति॒ प्रति॑ - स्थि॒त्यै॒ । \newline
51. यो वै वै यो यो वै य॒ज्ञ्स्य॑ य॒ज्ञ्स्य॒ वै यो यो वै य॒ज्ञ्स्य॑ । \newline
52. वै य॒ज्ञ्स्य॑ य॒ज्ञ्स्य॒ वै वै य॒ज्ञ्स्य॒ द्वौ द्वौ य॒ज्ञ्स्य॒ वै वै य॒ज्ञ्स्य॒ द्वौ । \newline
53. य॒ज्ञ्स्य॒ द्वौ द्वौ य॒ज्ञ्स्य॑ य॒ज्ञ्स्य॒ द्वौ दोहौ॒ दोहौ॒ द्वौ य॒ज्ञ्स्य॑ य॒ज्ञ्स्य॒ द्वौ दोहौ᳚ । \newline
54. द्वौ दोहौ॒ दोहौ॒ द्वौ द्वौ दोहौ॑ वि॒द्वान्. वि॒द्वान् दोहौ॒ द्वौ द्वौ दोहौ॑ वि॒द्वान् । \newline
55. दोहौ॑ वि॒द्वान्. वि॒द्वान् दोहौ॒ दोहौ॑ वि॒द्वान्. यज॑ते॒ यज॑ते वि॒द्वान् दोहौ॒ दोहौ॑ वि॒द्वान्. यज॑ते । \newline
56. वि॒द्वान्. यज॑ते॒ यज॑ते वि॒द्वान्. वि॒द्वान्. यज॑त उभ॒यत॑ उभ॒यतो॒ यज॑ते वि॒द्वान्. वि॒द्वान्. यज॑त उभ॒यतः॑ । \newline
57. यज॑त उभ॒यत॑ उभ॒यतो॒ यज॑ते॒ यज॑त उभ॒यत॑ ए॒वैवोभ॒यतो॒ यज॑ते॒ यज॑त उभ॒यत॑ ए॒व । \newline
58. उ॒भ॒यत॑ ए॒वैवोभ॒यत॑ उभ॒यत॑ ए॒व य॒ज्ञ्ं ॅय॒ज्ञ् मे॒वोभ॒यत॑ उभ॒यत॑ ए॒व य॒ज्ञ्म् । \newline
\pagebreak
\markright{ TS 1.7.4.3  \hfill https://www.vedavms.in \hfill}
\addcontentsline{toc}{section}{ TS 1.7.4.3 }
\section*{ TS 1.7.4.3 }

\textbf{TS 1.7.4.3 } \newline
\textbf{Samhita Paata} \newline

ए॒व य॒ज्ञ्ं दु॑हे पु॒रस्ता᳚च्चो॒परि॑ष्टाच्चै॒ष वा अ॒न्यो य॒ज्ञ्स्य॒ दोह॒ इडा॑याम॒न्यो यर्.हि॒ होता॒ यज॑मानस्य॒ नाम॑ गृह्णी॒यात् तर्.हि॑ ब्रूया॒देमा अ॑ग्मन्ना॒शिषो॒ दोह॑कामा॒ इति॒ सꣳस्तु॑ता ए॒व दे॒वता॑ दु॒हेऽथो॑ उभ॒यत॑ ए॒व य॒ज्ञ्ं दु॑हे पु॒रस्ता᳚च्चो॒परि॑ष्टाच्च॒ रोहि॑तेन त्वा॒ऽग्निर्दे॒वतां᳚ गमय॒त्वित्या॑है॒ते वै दे॑वा॒श्वा - [ ] \newline

\textbf{Pada Paata} \newline

ए॒व । य॒ज्ञ्म् । दु॒हे॒ । पु॒रस्ता᳚त् । च॒ । उ॒परि॑ष्टात् । च॒ । ए॒षः । वै । अ॒न्यः । य॒ज्ञ्स्य॑ । दोहः॑ । इडा॑याम् । अ॒न्यः । यर्.हि॑ । होता᳚ । यज॑मानस्य । नाम॑ । गृ॒ह्णी॒यात् । तर्.हि॑ । ब्रू॒या॒त् । एति॑ । इ॒माः । अ॒ग्म॒न्न् । आ॒शिष॒ इत्या᳚ - शिषः॑ । दोह॑कामा॒ इति॒ दोह॑ - का॒माः॒ । इति॑ । सꣳस्तु॑ता॒ इति॒ सं - स्तु॒ताः॒ । ए॒व । दे॒वताः᳚ । दु॒हे॒ । अथो॒ इति॑ । उ॒भ॒यतः॑ । ए॒व । य॒ज्ञ्म् । दु॒हे॒ । पु॒रस्ता᳚त् । च॒ । उ॒परि॑ष्ठत् । च॒ । रोहि॑तेन । त्वा॒ । अ॒ग्निः । दे॒वता᳚म् । ग॒म॒य॒तु॒ । इति॑ । आ॒ह॒ । ए॒ते । वै । दे॒वा॒श्वा इति॑ देव - अ॒श्वाः ।  \newline


\textbf{Krama Paata} \newline

ए॒व य॒ज्ञ्म् । य॒ज्ञ्म् दु॑हे । दु॒हे॒ पु॒रस्ता᳚त् । पु॒रस्ता᳚च्च । चो॒परि॑ष्टात् । उ॒परि॑ष्टाच्च । चै॒षः । ए॒ष वै । वा अ॒न्यः । अ॒न्यो य॒ज्ञ्स्य॑ । य॒ज्ञ्स्य॒ दोहः॑ । दोह॒ इडा॑याम् । इडा॑याम॒न्यः । अ॒न्यो यर्.हि॑ । यर्.हि॒ होता᳚ । होता॒ यज॑मानस्य । यज॑मानस्य॒ नाम॑ । नाम॑ गृह्णी॒यात् । गृ॒ह्णी॒यात्,तर्.हि॑ । तर्.हि॑ ब्रूयात् । ब्रू॒या॒दा । एमाः । इ॒मा अ॑ग्मन्न् । अ॒ग्म॒न्ना॒शिषः॑ । आ॒शिषो॒ दोह॑कामाः । आ॒शिष॒ इत्या᳚ - शिषः॑ । दोह॑कामा॒ इति॑ । दोह॑कामा॒ इति॒ दोह॑ - का॒माः॒ । इति॒ सꣳस्तु॑ताः । सꣳस्तु॑ता ए॒व । सꣳस्तु॑ता॒ इति॒ सं - स्तु॒ताः॒ । ए॒व दे॒वताः᳚ । दे॒वता॑ दु॒हे । दु॒हेऽथो᳚ । अथो॑ उभ॒यतः॑ । अथो॒ इत्यथो᳚ । उ॒भ॒यत॑ ए॒व । ए॒व य॒ज्ञ्म् । य॒ज्ञ्म् दु॑हे । दु॒हे॒ पु॒रस्ता᳚त् । पु॒रस्ता᳚च्च । चो॒परि॑ष्टात् । उ॒परि॑ष्टाच्च । च॒ रोहि॑तेन । रोहि॑तेन त्वा । त्वा॒ऽग्निः । अ॒ग्निर्,दे॒वता᳚म् । दे॒वता᳚म् गमयतु । ग॒म॒य॒त्विति॑ । इत्या॑ह । आ॒है॒ते । ए॒ते वै । वै दे॑वा॒श्वाः । दे॒वा॒श्वा यज॑मानः । दे॒वा॒श्वा इति॑ देव - अ॒श्वाः \newline

\textbf{Jatai Paata} \newline

1. ए॒व य॒ज्ञ्ं ॅय॒ज्ञ् मे॒वैव य॒ज्ञ्म् । \newline
2. य॒ज्ञ्म् दु॑हे दुहे य॒ज्ञ्ं ॅय॒ज्ञ्म् दु॑हे । \newline
3. दु॒हे॒ पु॒रस्ता᳚त् पु॒रस्ता᳚द् दुहे दुहे पु॒रस्ता᳚त् । \newline
4. पु॒रस्ता᳚च् च च पु॒रस्ता᳚त् पु॒रस्ता᳚च् च । \newline
5. चो॒परि॑ष्टा दु॒परि॑ष्टाच् च चो॒परि॑ष्टात् । \newline
6. उ॒परि॑ष्टाच् च चो॒परि॑ष्टा दु॒परि॑ष्टाच् च । \newline
7. चै॒ष ए॒ष च॑ चै॒षः । \newline
8. ए॒ष वै वा ए॒ष ए॒ष वै । \newline
9. वा अ॒न्यो᳚ ऽन्यो वै वा अ॒न्यः । \newline
10. अ॒न्यो य॒ज्ञ्स्य॑ य॒ज्ञ्स्या॒न्यो᳚ ऽन्यो य॒ज्ञ्स्य॑ । \newline
11. य॒ज्ञ्स्य॒ दोहो॒ दोहो॑ य॒ज्ञ्स्य॑ य॒ज्ञ्स्य॒ दोहः॑ । \newline
12. दोह॒ इडा॑या॒ मिडा॑या॒म् दोहो॒ दोह॒ इडा॑याम् । \newline
13. इडा॑या म॒न्यो᳚ ऽन्य इडा॑या॒ मिडा॑या म॒न्यः । \newline
14. अ॒न्यो यर्.हि॒ यर्ह्य॒न्यो᳚ ऽन्यो यर्.हि॑ । \newline
15. यर्.हि॒ होता॒ होता॒ यर्.हि॒ यर्.हि॒ होता᳚ । \newline
16. होता॒ यज॑मानस्य॒ यज॑मानस्य॒ होता॒ होता॒ यज॑मानस्य । \newline
17. यज॑मानस्य॒ नाम॒ नाम॒ यज॑मानस्य॒ यज॑मानस्य॒ नाम॑ । \newline
18. नाम॑ गृह्णी॒याद् गृ॑ह्णी॒यान् नाम॒ नाम॑ गृह्णी॒यात् । \newline
19. गृ॒ह्णी॒यात् तर्.हि॒ तर्.हि॑ गृह्णी॒याद् गृ॑ह्णी॒यात् तर्.हि॑ । \newline
20. तर्.हि॑ ब्रूयाद् ब्रूया॒त् तर्.हि॒ तर्.हि॑ ब्रूयात् । \newline
21. ब्रू॒या॒दा ब्रू॑याद् ब्रूया॒दा । \newline
22. एमा इ॒मा एमाः । \newline
23. इ॒मा अ॑ग्मन् नग्मन् नि॒मा इ॒मा अ॑ग्मन्न् । \newline
24. अ॒ग्म॒न् ना॒शिष॑ आ॒शिषो᳚ ऽग्मन् नग्मन् ना॒शिषः॑ । \newline
25. आ॒शिषो॒ दोह॑कामा॒ दोह॑कामा आ॒शिष॑ आ॒शिषो॒ दोह॑कामाः । \newline
26. आ॒शिष॒ इत्या᳚ - शिषः॑ । \newline
27. दोह॑कामा॒ इतीति॒ दोह॑कामा॒ दोह॑कामा॒ इति॑ । \newline
28. दोह॑कामा॒ इति॒ दोह॑ - का॒माः॒ । \newline
29. इति॒ सꣳस्तु॑ताः॒ सꣳस्तु॑ता॒ इतीति॒ सꣳस्तु॑ताः । \newline
30. सꣳस्तु॑ता ए॒वैव सꣳस्तु॑ताः॒ सꣳस्तु॑ता ए॒व । \newline
31. सꣳस्तु॑ता॒ इति॒ सं - स्तु॒ताः॒ । \newline
32. ए॒व दे॒वता॑ दे॒वता॑ ए॒वैव दे॒वताः᳚ । \newline
33. दे॒वता॑ दुहे दुहे दे॒वता॑ दे॒वता॑ दुहे । \newline
34. दु॒हे ऽथो॒ अथो॑ दुहे दु॒हे ऽथो᳚ । \newline
35. अथो॑ उभ॒यत॑ उभ॒यतो ऽथो॒ अथो॑ उभ॒यतः॑ । \newline
36. अथो॒ इत्यथो᳚ । \newline
37. उ॒भ॒यत॑ ए॒वैवो भ॒यत॑ उभ॒यत॑ ए॒व । \newline
38. ए॒व य॒ज्ञ्ं ॅय॒ज्ञ् मे॒वैव य॒ज्ञ्म् । \newline
39. य॒ज्ञ्म् दु॑हे दुहे य॒ज्ञ्ं ॅय॒ज्ञ्म् दु॑हे । \newline
40. दु॒हे॒ पु॒रस्ता᳚त् पु॒रस्ता᳚द् दुहे दुहे पु॒रस्ता᳚त् । \newline
41. पु॒रस्ता᳚च् च च पु॒रस्ता᳚त् पु॒रस्ता᳚च् च । \newline
42. चो॒परि॑ष्टा दु॒परि॑ष्टाच् च चो॒परि॑ष्टात् । \newline
43. उ॒परि॑ष्टाच् च चो॒परि॑ष्टा दु॒परि॑ष्टाच् च । \newline
44. च॒ रोहि॑तेन॒ रोहि॑तेन च च॒ रोहि॑तेन । \newline
45. रोहि॑तेन त्वा त्वा॒ रोहि॑तेन॒ रोहि॑तेन त्वा । \newline
46. त्वा॒ ऽग्नि र॒ग्नि स्त्वा᳚ त्वा॒ ऽग्निः । \newline
47. अ॒ग्निर् दे॒वता᳚म् दे॒वता॑ म॒ग्नि र॒ग्निर् दे॒वता᳚म् । \newline
48. दे॒वता᳚म् गमयतु गमयतु दे॒वता᳚म् दे॒वता᳚म् गमयतु । \newline
49. ग॒म॒य॒ त्वितीति॑ गमयतु गमय॒ त्विति॑ । \newline
50. इत्या॑हा॒हे तीत्या॑ह । \newline
51. आ॒है॒त ए॒त आ॑हा है॒ते । \newline
52. ए॒ते वै वा ए॒त ए॒ते वै । \newline
53. वै दे॑वा॒श्वा दे॑वा॒श्वा वै वै दे॑वा॒श्वाः । \newline
54. दे॒वा॒श्वा यज॑मानो॒ यज॑मानो देवा॒श्वा दे॑वा॒श्वा यज॑मानः । \newline
55. दे॒वा॒श्वा इति॑ देव - अ॒श्वाः । \newline

\textbf{Ghana Paata } \newline

1. ए॒व य॒ज्ञ्ं ॅय॒ज्ञ् मे॒वैव य॒ज्ञ्म् दु॑हे दुहे य॒ज्ञ् मे॒वैव य॒ज्ञ्म् दु॑हे । \newline
2. य॒ज्ञ्म् दु॑हे दुहे य॒ज्ञ्ं ॅय॒ज्ञ्म् दु॑हे पु॒रस्ता᳚त् पु॒रस्ता᳚द् दुहे य॒ज्ञ्ं ॅय॒ज्ञ्म् दु॑हे पु॒रस्ता᳚त् । \newline
3. दु॒हे॒ पु॒रस्ता᳚त् पु॒रस्ता᳚द् दुहे दुहे पु॒रस्ता᳚च् च च पु॒रस्ता᳚द् दुहे दुहे पु॒रस्ता᳚च् च । \newline
4. पु॒रस्ता᳚च् च च पु॒रस्ता᳚त् पु॒रस्ता᳚च् चो॒परि॑ष्टा दु॒परि॑ष्टाच् च पु॒रस्ता᳚त् पु॒रस्ता᳚च् चो॒परि॑ष्टात् । \newline
5. चो॒परि॑ष्टा दु॒परि॑ष्टाच् च चो॒परि॑ष्टाच् च चो॒परि॑ष्टाच् च चो॒परि॑ष्टाच् च । \newline
6. उ॒परि॑ष्टाच् च चो॒परि॑ष्टा दु॒परि॑ष्टाच् चै॒ष ए॒ष चो॒परि॑ष्टा दु॒परि॑ष्टाच् चै॒षः । \newline
7. चै॒ष ए॒ष च॑ चै॒ष वै वा ए॒ष च॑ चै॒ष वै । \newline
8. ए॒ष वै वा ए॒ष ए॒ष वा अ॒न्यो᳚ ऽन्यो वा ए॒ष ए॒ष वा अ॒न्यः । \newline
9. वा अ॒न्यो᳚ ऽन्यो वै वा अ॒न्यो य॒ज्ञ्स्य॑ य॒ज्ञ्स्या॒न्यो वै वा अ॒न्यो य॒ज्ञ्स्य॑ । \newline
10. अ॒न्यो य॒ज्ञ्स्य॑ य॒ज्ञ्स्या॒न्यो᳚ ऽन्यो य॒ज्ञ्स्य॒ दोहो॒ दोहो॑ य॒ज्ञ्स्या॒न्यो᳚ ऽन्यो य॒ज्ञ्स्य॒ दोहः॑ । \newline
11. य॒ज्ञ्स्य॒ दोहो॒ दोहो॑ य॒ज्ञ्स्य॑ य॒ज्ञ्स्य॒ दोह॒ इडा॑या॒ मिडा॑या॒म् दोहो॑ य॒ज्ञ्स्य॑ य॒ज्ञ्स्य॒ दोह॒ इडा॑याम् । \newline
12. दोह॒ इडा॑या॒ मिडा॑या॒म् दोहो॒ दोह॒ इडा॑या म॒न्यो᳚ ऽन्य इडा॑या॒म् दोहो॒ दोह॒ इडा॑या म॒न्यः । \newline
13. इडा॑या म॒न्यो᳚ ऽन्य इडा॑या॒ मिडा॑या म॒न्यो यर्.हि॒ यर्ह्य॒न्य इडा॑या॒ मिडा॑या म॒न्यो यर्.हि॑ । \newline
14. अ॒न्यो यर्.हि॒ यर्ह्य॒न्यो᳚ ऽन्यो यर्.हि॒ होता॒ होता॒ यर्ह्य॒न्यो᳚ ऽन्यो यर्.हि॒ होता᳚ । \newline
15. यर्.हि॒ होता॒ होता॒ यर्.हि॒ यर्.हि॒ होता॒ यज॑मानस्य॒ यज॑मानस्य॒ होता॒ यर्.हि॒ यर्.हि॒ होता॒ यज॑मानस्य । \newline
16. होता॒ यज॑मानस्य॒ यज॑मानस्य॒ होता॒ होता॒ यज॑मानस्य॒ नाम॒ नाम॒ यज॑मानस्य॒ होता॒ होता॒ यज॑मानस्य॒ नाम॑ । \newline
17. यज॑मानस्य॒ नाम॒ नाम॒ यज॑मानस्य॒ यज॑मानस्य॒ नाम॑ गृह्णी॒याद् गृ॑ह्णी॒यान् नाम॒ यज॑मानस्य॒ यज॑मानस्य॒ नाम॑ गृह्णी॒यात् । \newline
18. नाम॑ गृह्णी॒याद् गृ॑ह्णी॒यान् नाम॒ नाम॑ गृह्णी॒यात् तर्.हि॒ तर्.हि॑ गृह्णी॒यान् नाम॒ नाम॑ गृह्णी॒यात् तर्.हि॑ । \newline
19. गृ॒ह्णी॒यात् तर्.हि॒ तर्.हि॑ गृह्णी॒याद् गृ॑ह्णी॒यात् तर्.हि॑ ब्रूयाद् ब्रूया॒त् तर्.हि॑ गृह्णी॒याद् गृ॑ह्णी॒यात् तर्.हि॑ ब्रूयात् । \newline
20. तर्.हि॑ ब्रूयाद् ब्रूया॒त् तर्.हि॒ तर्.हि॑ ब्रूया॒दा ब्रू॑या॒त् तर्.हि॒ तर्.हि॑ ब्रूया॒दा । \newline
21. ब्रू॒या॒दा ब्रू॑याद् ब्रूया॒देमा इ॒मा आ ब्रू॑याद् ब्रूया॒देमाः । \newline
22. एमा इ॒मा एमा अ॑ग्मन् नग्मन् नि॒मा एमा अ॑ग्मन्न् । \newline
23. इ॒मा अ॑ग्मन् नग्मन् नि॒मा इ॒मा अ॑ग्मन् ना॒शिष॑ आ॒शिषो᳚ ऽग्मन् नि॒मा इ॒मा अ॑ग्मन् ना॒शिषः॑ । \newline
24. अ॒ग्म॒न् ना॒शिष॑ आ॒शिषो᳚ ऽग्मन् नग्मन् ना॒शिषो॒ दोह॑कामा॒ दोह॑कामा आ॒शिषो᳚ ऽग्मन् नग्मन् ना॒शिषो॒ दोह॑कामाः । \newline
25. आ॒शिषो॒ दोह॑कामा॒ दोह॑कामा आ॒शिष॑ आ॒शिषो॒ दोह॑कामा॒ इतीति॒ दोह॑कामा आ॒शिष॑ आ॒शिषो॒ दोह॑कामा॒ इति॑ । \newline
26. आ॒शिष॒ इत्या᳚ - शिषः॑ । \newline
27. दोह॑कामा॒ इतीति॒ दोह॑कामा॒ दोह॑कामा॒ इति॒ सꣳस्तु॑ताः॒ सꣳस्तु॑ता॒ इति॒ दोह॑कामा॒ दोह॑कामा॒ इति॒ सꣳस्तु॑ताः । \newline
28. दोह॑कामा॒ इति॒ दोह॑ - का॒माः॒ । \newline
29. इति॒ सꣳस्तु॑ताः॒ सꣳस्तु॑ता॒ इतीति॒ सꣳस्तु॑ता ए॒वैव सꣳस्तु॑ता॒ इतीति॒ सꣳस्तु॑ता ए॒व । \newline
30. सꣳस्तु॑ता ए॒वैव सꣳस्तु॑ताः॒ सꣳस्तु॑ता ए॒व दे॒वता॑ दे॒वता॑ ए॒व सꣳस्तु॑ताः॒ सꣳस्तु॑ता ए॒व दे॒वताः᳚ । \newline
31. सꣳस्तु॑ता॒ इति॒ सं - स्तु॒ताः॒ । \newline
32. ए॒व दे॒वता॑ दे॒वता॑ ए॒वैव दे॒वता॑ दुहे दुहे दे॒वता॑ ए॒वैव दे॒वता॑ दुहे । \newline
33. दे॒वता॑ दुहे दुहे दे॒वता॑ दे॒वता॑ दु॒हे ऽथो॒ अथो॑ दुहे दे॒वता॑ दे॒वता॑ दु॒हे ऽथो᳚ । \newline
34. दु॒हे ऽथो॒ अथो॑ दुहे दु॒हे ऽथो॑ उभ॒यत॑ उभ॒यतो ऽथो॑ दुहे दु॒हे ऽथो॑ उभ॒यतः॑ । \newline
35. अथो॑ उभ॒यत॑ उभ॒यतो ऽथो॒ अथो॑ उभ॒यत॑ ए॒वैवो भ॒यतो ऽथो॒ अथो॑ उभ॒यत॑ ए॒व । \newline
36. अथो॒ इत्यथो᳚ । \newline
37. उ॒भ॒यत॑ ए॒वैवो भ॒यत॑ उभ॒यत॑ ए॒व य॒ज्ञ्ं ॅय॒ज्ञ् मे॒वोभ॒यत॑ उभ॒यत॑ ए॒व य॒ज्ञ्म् । \newline
38. ए॒व य॒ज्ञ्ं ॅय॒ज्ञ् मे॒वैव य॒ज्ञ्म् दु॑हे दुहे य॒ज्ञ् मे॒वैव य॒ज्ञ्म् दु॑हे । \newline
39. य॒ज्ञ्म् दु॑हे दुहे य॒ज्ञ्ं ॅय॒ज्ञ्म् दु॑हे पु॒रस्ता᳚त् पु॒रस्ता᳚द् दुहे य॒ज्ञ्ं ॅय॒ज्ञ्म् दु॑हे पु॒रस्ता᳚त् । \newline
40. दु॒हे॒ पु॒रस्ता᳚त् पु॒रस्ता᳚द् दुहे दुहे पु॒रस्ता᳚च् च च पु॒रस्ता᳚द् दुहे दुहे पु॒रस्ता᳚च् च । \newline
41. पु॒रस्ता᳚च् च च पु॒रस्ता᳚त् पु॒रस्ता᳚च् चो॒परि॑ष्टा दु॒परि॑ष्टाच् च पु॒रस्ता᳚त् पु॒रस्ता᳚च् चो॒परि॑ष्टात् । \newline
42. चो॒परि॑ष्टा दु॒परि॑ष्टाच् च चो॒परि॑ष्टाच् च चो॒परि॑ष्टाच् च चो॒परि॑ष्टाच् च । \newline
43. उ॒परि॑ष्टाच् च चो॒परि॑ष्टा दु॒परि॑ष्टाच् च॒ रोहि॑तेन॒ रोहि॑तेन चो॒परि॑ष्टा दु॒परि॑ष्टाच् च॒ रोहि॑तेन । \newline
44. च॒ रोहि॑तेन॒ रोहि॑तेन च च॒ रोहि॑तेन त्वा त्वा॒ रोहि॑तेन च च॒ रोहि॑तेन त्वा । \newline
45. रोहि॑तेन त्वा त्वा॒ रोहि॑तेन॒ रोहि॑तेन त्वा॒ ऽग्नि र॒ग्नि स्त्वा॒ रोहि॑तेन॒ रोहि॑तेन त्वा॒ ऽग्निः । \newline
46. त्वा॒ ऽग्नि र॒ग्नि स्त्वा᳚ त्वा॒ ऽग्निर् दे॒वता᳚म् दे॒वता॑ म॒ग्नि स्त्वा᳚ त्वा॒ ऽग्निर् दे॒वता᳚म् । \newline
47. अ॒ग्निर् दे॒वता᳚म् दे॒वता॑ म॒ग्नि र॒ग्निर् दे॒वता᳚म् गमयतु गमयतु दे॒वता॑ म॒ग्नि र॒ग्निर् दे॒वता᳚म् गमयतु । \newline
48. दे॒वता᳚म् गमयतु गमयतु दे॒वता᳚म् दे॒वता᳚म् गमय॒त्वितीति॑ गमयतु दे॒वता᳚म् दे॒वता᳚म् गमय॒त्विति॑ । \newline
49. ग॒म॒य॒त्वि तीति॑ गमयतु गमय॒त्वि त्या॑हा॒हे ति॑ गमयतु गमय॒त्वि त्या॑ह । \newline
50. इत्या॑हा॒हे तीत्या॑है॒त ए॒त आ॒हे तीत्या॑है॒ते । \newline
51. आ॒है॒त ए॒त आ॑हाहै॒ते वै वा ए॒त आ॑हाहै॒ते वै । \newline
52. ए॒ते वै वा ए॒त ए॒ते वै दे॑वा॒श्वा दे॑वा॒श्वा वा ए॒त ए॒ते वै दे॑वा॒श्वाः । \newline
53. वै दे॑वा॒श्वा दे॑वा॒श्वा वै वै दे॑वा॒श्वा यज॑मानो॒ यज॑मानो देवा॒श्वा वै वै दे॑वा॒श्वा यज॑मानः । \newline
54. दे॒वा॒श्वा यज॑मानो॒ यज॑मानो देवा॒श्वा दे॑वा॒श्वा यज॑मानः प्रस्त॒रः प्र॑स्त॒रो यज॑मानो देवा॒श्वा दे॑वा॒श्वा यज॑मानः प्रस्त॒रः । \newline
55. दे॒वा॒श्वा इति॑ देव - अ॒श्वाः । \newline
\pagebreak
\markright{ TS 1.7.4.4  \hfill https://www.vedavms.in \hfill}
\addcontentsline{toc}{section}{ TS 1.7.4.4 }
\section*{ TS 1.7.4.4 }

\textbf{TS 1.7.4.4 } \newline
\textbf{Samhita Paata} \newline

यज॑मानः प्रस्त॒रो यदे॒तैः प्र॑स्त॒रं प्र॒हर॑ति देवा॒श्वैरे॒व यज॑मानꣳ सुव॒र्गं ॅलो॒कं ग॑मयति॒ वि ते॑ मुञ्चामि रश॒ना वि र॒श्मीनित्या॑है॒ष वा अ॒ग्नेर्वि॑मो॒कस्ते-नै॒वैनं॒ ॅविमु॑ञ्चति ॒विष्णोः᳚ शं॒ॅयोर॒हं दे॑वय॒ज्यया॑ य॒ज्ञेन॑ प्रति॒ष्ठां ग॑मेय॒मित्या॑ह य॒ज्ञो वै विष्णु॑र् य॒ज्ञ् ए॒वान्त॒तः प्रति॑ तिष्ठति॒ सोम॑स्या॒हं दे॑वय॒ज्यया॑ सु॒रेता॒ -[ ] \newline

\textbf{Pada Paata} \newline

यज॑मानः । प्र॒स्त॒र इति॑ प्र - स्त॒रः । यत् । ए॒तैः । प्र॒स्त॒रमिति॑ प्र - स्त॒रम् । प्र॒हर॒तीति॑ प्र - हर॑ति । दे॒वा॒श्वैरिति॑ देव - अ॒श्वैः । ए॒व । यज॑मानम् । सु॒व॒र्गमिति॑ सुवः - गम् । लो॒कम् । ग॒म॒य॒ति॒ । वीति॑ । ते॒ । मु॒ञ्चा॒मि॒ । र॒श॒नाः । वीति॑ । र॒श्मीन् । इति॑ । आ॒ह॒ । ए॒षः । वै । अ॒ग्नेः । वि॒मो॒क इति॑ वि - मो॒कः । तेन॑ । ए॒व । ए॒न॒म् । वीति॑ । मु॒ञ्च॒ति॒ । विष्णोः᳚ । शं॒ॅयोरिति॑ शं - योः । अ॒हम् । दे॒व॒य॒ज्ययेति॑ देव - य॒ज्यया᳚ । य॒ज्ञेन॑ । प्र॒ति॒ष्ठामिति॑ प्रति - स्थाम् । ग॒मे॒य॒म् । इति॑ । आ॒ह॒ । य॒ज्ञ्ः । वै । विष्णुः॑ । य॒ज्ञे । ए॒व । अ॒न्त॒तः । प्रतीति॑ । ति॒ष्ठ॒ति॒ । सोम॑स्य । अ॒हम् । दे॒व॒य॒ज्ययेति॑ देव - य॒ज्यया᳚ । सु॒रेता॒ इति॑ सु - रेताः᳚ ।  \newline


\textbf{Krama Paata} \newline

यज॑मानः प्रस्त॒रः । प्र॒स्त॒रो यत् । प्र॒स्त॒र इति॑ प्र - स्त॒रः । यदे॒तैः । ए॒तैः प्र॑स्त॒रम् । प्र॒स्त॒रम् प्र॒हर॑ति । प्र॒स्त॒रमिति॑ प्र - स्त॒रम् । प्र॒हर॑ति देवा॒श्वैः । प्र॒हर॒तीति॑ प्र - हर॑ति । दे॒वा॒श्वैरे॒व । दे॒वा॒श्वैरिति॑ देव - अ॒श्वैः । ए॒व यज॑मानम् । यज॑मानꣳ सुव॒र्गम् । सु॒व॒र्गं ॅलो॒कम् । सु॒व॒र्गमिति॑ सुवः - गम् । लो॒कम् ग॑मयति । ग॒म॒य॒ति॒ वि । वि ते᳚ । ते॒ मु॒ञ्चा॒मि॒ । मु॒ञ्चा॒मि॒ र॒श॒नाः । र॒श॒ना वि । वि र॒श्मीन् । र॒श्मीनिति॑ । इत्या॑ह । आ॒है॒षः । ए॒ष वै । वा अ॒ग्नेः । अ॒ग्नेर्,वि॑मो॒कः । वि॒मो॒कस्तेन॑ । वि॒मो॒क इति॑ वि - मो॒कः । तेनै॒व । ए॒वैन᳚म् । ए॒नं॒ ॅवि । वि मु॑ञ्चति । मु॒ञ्च॒ति॒ विष्णोः᳚ । विष्णोः᳚ श॒म्ॅयोः । श॒म्ॅयोर॒हम् । श॒म्ॅयोरिति॑ शं - योः । अ॒हम् दे॑वय॒ज्यया᳚ । दे॒व॒य॒ज्यया॑ य॒ज्ञेन॑ । दे॒व॒य॒ज्ययेति॑ देव - य॒ज्यया᳚ । य॒ज्ञेन॑ प्रति॒ष्ठाम् । प्र॒ति॒ष्ठाम् ग॑मेयम् । प्र॒ति॒ष्ठामिति॑ प्रति - स्थाम् । ग॒मे॒य॒मिति॑ । इत्या॑ह । आ॒ह॒ य॒ज्ञ्ः । य॒ज्ञो वै । वै विष्णुः॑ । विष्णु॑र्,य॒ज्ञे । य॒ज्ञ् ए॒व । ए॒वान्त॒तः । अ॒न्त॒तः प्रति॑ । प्रति॑ तिष्ठति । ति॒ष्ठ॒ति॒ सोम॑स्य । सोम॑स्या॒हम् । अ॒हम् दे॑वय॒ज्यया᳚ । दे॒व॒य॒ज्यया॑ सु॒रेताः᳚ । दे॒व॒य॒ज्ययेति॑ देव - य॒ज्यया᳚ । सु॒रेता॒ रेतः॑ । सु॒रेता॒ इति॑ सु - रेताः᳚ \newline

\textbf{Jatai Paata} \newline

1. यज॑मानः प्रस्त॒रः प्र॑स्त॒रो यज॑मानो॒ यज॑मानः प्रस्त॒रः । \newline
2. प्र॒स्त॒रो यद् यत् प्र॑स्त॒रः प्र॑स्त॒रो यत् । \newline
3. प्र॒स्त॒र इति॑ प्र - स्त॒रः । \newline
4. यदे॒तै रे॒तैर् यद् यदे॒तैः । \newline
5. ए॒तैः प्र॑स्त॒रम् प्र॑स्त॒र मे॒तैरे॒तैः प्र॑स्त॒रम् । \newline
6. प्र॒स्त॒रम् प्र॒हर॑ति प्र॒हर॑ति प्रस्त॒रम् प्र॑स्त॒रम् प्र॒हर॑ति । \newline
7. प्र॒स्त॒रमिति॑ प्र - स्त॒रम् । \newline
8. प्र॒हर॑ति देवा॒श्वैर् दे॑वा॒श्वैः प्र॒हर॑ति प्र॒हर॑ति देवा॒श्वैः । \newline
9. प्र॒हर॒तीति॑ प्र - हर॑ति । \newline
10. दे॒वा॒श्वै रे॒वैव दे॑वा॒श्वैर् दे॑वा॒श्वै रे॒व । \newline
11. दे॒वा॒श्वैरिति॑ देव - अ॒श्वैः । \newline
12. ए॒व यज॑मानं॒ ॅयज॑मान मे॒वैव यज॑मानम् । \newline
13. यज॑मानꣳ सुव॒र्गꣳ सु॑व॒र्गं ॅयज॑मानं॒ ॅयज॑मानꣳ सुव॒र्गम् । \newline
14. सु॒व॒र्गम् ॅलो॒कम् ॅलो॒कꣳ सु॑व॒र्गꣳ सु॑व॒र्गम् ॅलो॒कम् । \newline
15. सु॒व॒र्गमिति॑ सुवः - गम् । \newline
16. लो॒कम् ग॑मयति गमयति लो॒कम् ॅलो॒कम् ग॑मयति । \newline
17. ग॒म॒य॒ति॒ वि वि ग॑मयति गमयति॒ वि । \newline
18. वि ते॑ ते॒ वि वि ते᳚ । \newline
19. ते॒ मु॒ञ्चा॒मि॒ मु॒ञ्चा॒मि॒ ते॒ ते॒ मु॒ञ्चा॒मि॒ । \newline
20. मु॒ञ्चा॒मि॒ र॒श॒ना र॑श॒ना मु॑ञ्चामि मुञ्चामि रश॒नाः । \newline
21. र॒श॒ना वि वि र॑श॒ना र॑श॒ना वि । \newline
22. वि र॒श्मीन् र॒श्मीन्. वि वि र॒श्मीन् । \newline
23. र॒श्मी नितीति॑ र॒श्मीन् र॒श्मी निति॑ । \newline
24. इत्या॑ हा॒हे तीत्या॑ह । \newline
25. आ॒है॒ष ए॒ष आ॑हा है॒षः । \newline
26. ए॒ष वै वा ए॒ष ए॒ष वै । \newline
27. वा अ॒ग्ने र॒ग्नेर् वै वा अ॒ग्नेः । \newline
28. अ॒ग्नेर् वि॑मो॒को वि॑मो॒को᳚ ऽग्ने र॒ग्नेर् वि॑मो॒कः । \newline
29. वि॒मो॒क स्तेन॒ तेन॑ विमो॒को वि॑मो॒क स्तेन॑ । \newline
30. वि॒मो॒क इति॑ वि - मो॒कः । \newline
31. तेनै॒वैव तेन॒ तेनै॒व । \newline
32. ए॒वैन॑ मेन मे॒वै वैन᳚म् । \newline
33. ए॒नं॒ ॅवि व्ये॑न मेनं॒ ॅवि । \newline
34. वि मु॑ञ्चति मुञ्चति॒ वि वि मु॑ञ्चति । \newline
35. मु॒ञ्च॒ति॒ विष्णो॒र् विष्णो᳚र् मुञ्चति मुञ्चति॒ विष्णोः᳚ । \newline
36. विष्णोः᳚ शं॒ॅयोः शं॒ॅयोर् विष्णो॒र् विष्णोः᳚ शं॒ॅयोः । \newline
37. शं॒ॅयो र॒ह म॒हꣳ शं॒ॅयोः शं॒ॅयो र॒हम् । \newline
38. शं॒ॅयोरिति॑ शं - योः । \newline
39. अ॒हम् दे॑वय॒ज्यया॑ देवय॒ज्यया॒ ऽह म॒हम् दे॑वय॒ज्यया᳚ । \newline
40. दे॒व॒य॒ज्यया॑ य॒ज्ञेन॑ य॒ज्ञेन॑ देवय॒ज्यया॑ देवय॒ज्यया॑ य॒ज्ञेन॑ । \newline
41. दे॒व॒य॒ज्ययेति॑ देव - य॒ज्यया᳚ । \newline
42. य॒ज्ञेन॑ प्रति॒ष्ठाम् प्र॑ति॒ष्ठां ॅय॒ज्ञेन॑ य॒ज्ञेन॑ प्रति॒ष्ठाम् । \newline
43. प्र॒ति॒ष्ठाम् ग॑मेयम् गमेयम् प्रति॒ष्ठाम् प्र॑ति॒ष्ठाम् ग॑मेयम् । \newline
44. प्र॒ति॒ष्ठामिति॑ प्रति - स्थाम् । \newline
45. ग॒मे॒य॒ मितीति॑ गमेयम् गमेय॒ मिति॑ । \newline
46. इत्या॑हा॒हे तीत्या॑ह । \newline
47. आ॒ह॒ य॒ज्ञो य॒ज्ञ् आ॑हाह य॒ज्ञ्ः । \newline
48. य॒ज्ञो वै वै य॒ज्ञो य॒ज्ञो वै । \newline
49. वै विष्णु॒र् विष्णु॒र् वै वै विष्णुः॑ । \newline
50. विष्णु॑र् य॒ज्ञे य॒ज्ञे विष्णु॒र् विष्णु॑र् य॒ज्ञे । \newline
51. य॒ज्ञ् ए॒वैव य॒ज्ञे य॒ज्ञ् ए॒व । \newline
52. ए॒वान्त॒तो᳚ ऽन्त॒त ए॒वै वान्त॒तः । \newline
53. अ॒न्त॒तः प्रति॒ प्रत्य॑न्त॒तो᳚ ऽन्त॒तः प्रति॑ । \newline
54. प्रति॑ तिष्ठति तिष्ठति॒ प्रति॒ प्रति॑ तिष्ठति । \newline
55. ति॒ष्ठ॒ति॒ सोम॑स्य॒ सोम॑स्य तिष्ठति तिष्ठति॒ सोम॑स्य । \newline
56. सोम॑स्या॒ह म॒हꣳ सोम॑स्य॒ सोम॑स्या॒हम् । \newline
57. अ॒हम् दे॑वय॒ज्यया॑ देवय॒ज्यया॒ ऽह म॒हम् दे॑वय॒ज्यया᳚ । \newline
58. दे॒व॒य॒ज्यया॑ सु॒रेताः᳚ सु॒रेता॑ देवय॒ज्यया॑ देवय॒ज्यया॑ सु॒रेताः᳚ । \newline
59. दे॒व॒य॒ज्ययेति॑ देव - य॒ज्यया᳚ । \newline
60. सु॒रेता॒ रेतो॒ रेतः॑ सु॒रेताः᳚ सु॒रेता॒ रेतः॑ । \newline
61. सु॒रेता॒ इति॑ सु - रेताः᳚ । \newline

\textbf{Ghana Paata } \newline

1. यज॑मानः प्रस्त॒रः प्र॑स्त॒रो यज॑मानो॒ यज॑मानः प्रस्त॒रो यद् यत् प्र॑स्त॒रो यज॑मानो॒ यज॑मानः प्रस्त॒रो यत् । \newline
2. प्र॒स्त॒रो यद् यत् प्र॑स्त॒रः प्र॑स्त॒रो यदे॒तै रे॒तैर् यत् प्र॑स्त॒रः प्र॑स्त॒रो यदे॒तैः । \newline
3. प्र॒स्त॒र इति॑ प्र - स्त॒रः । \newline
4. यदे॒तै रे॒तैर् यद् यदे॒तैः प्र॑स्त॒रम् प्र॑स्त॒र मे॒तैर् यद् यदे॒तैः प्र॑स्त॒रम् । \newline
5. ए॒तैः प्र॑स्त॒रम् प्र॑स्त॒र मे॒तैरे॒तैः प्र॑स्त॒रम् प्र॒हर॑ति प्र॒हर॑ति प्रस्त॒र मे॒तै रे॒तैः प्र॑स्त॒रम् प्र॒हर॑ति । \newline
6. प्र॒स्त॒रम् प्र॒हर॑ति प्र॒हर॑ति प्रस्त॒रम् प्र॑स्त॒रम् प्र॒हर॑ति देवा॒श्वैर् दे॑वा॒श्वैः प्र॒हर॑ति प्रस्त॒रम् प्र॑स्त॒रम् प्र॒हर॑ति देवा॒श्वैः । \newline
7. प्र॒स्त॒रमिति॑ प्र - स्त॒रम् । \newline
8. प्र॒हर॑ति देवा॒श्वैर् दे॑वा॒श्वैः प्र॒हर॑ति प्र॒हर॑ति देवा॒श्वै रे॒वैव दे॑वा॒श्वैः प्र॒हर॑ति प्र॒हर॑ति देवा॒श्वै रे॒व । \newline
9. प्र॒हर॒तीति॑ प्र - हर॑ति । \newline
10. दे॒वा॒श्वै रे॒वैव दे॑वा॒श्वैर् दे॑वा॒श्वैरे॒व यज॑मानं॒ ॅयज॑मान मे॒व दे॑वा॒श्वैर् दे॑वा॒श्वै रे॒व यज॑मानम् । \newline
11. दे॒वा॒श्वैरिति॑ देव - अ॒श्वैः । \newline
12. ए॒व यज॑मानं॒ ॅयज॑मान मे॒वैव यज॑मानꣳ सुव॒र्गꣳ सु॑व॒र्गं ॅयज॑मान मे॒वैव यज॑मानꣳ सुव॒र्गम् । \newline
13. यज॑मानꣳ सुव॒र्गꣳ सु॑व॒र्गं ॅयज॑मानं॒ ॅयज॑मानꣳ सुव॒र्गम् ॅलो॒कम् ॅलो॒कꣳ सु॑व॒र्गं ॅयज॑मानं॒ ॅयज॑मानꣳ सुव॒र्गम् ॅलो॒कम् । \newline
14. सु॒व॒र्गम् ॅलो॒कम् ॅलो॒कꣳ सु॑व॒र्गꣳ सु॑व॒र्गम् ॅलो॒कम् ग॑मयति गमयति लो॒कꣳ सु॑व॒र्गꣳ सु॑व॒र्गम् ॅलो॒कम् ग॑मयति । \newline
15. सु॒व॒र्गमिति॑ सुवः - गम् । \newline
16. लो॒कम् ग॑मयति गमयति लो॒कम् ॅलो॒कम् ग॑मयति॒ वि वि ग॑मयति लो॒कम् ॅलो॒कम् ग॑मयति॒ वि । \newline
17. ग॒म॒य॒ति॒ वि वि ग॑मयति गमयति॒ वि ते॑ ते॒ वि ग॑मयति गमयति॒ वि ते᳚ । \newline
18. वि ते॑ ते॒ वि वि ते॑ मुञ्चामि मुञ्चामि ते॒ वि वि ते॑ मुञ्चामि । \newline
19. ते॒ मु॒ञ्चा॒मि॒ मु॒ञ्चा॒मि॒ ते॒ ते॒ मु॒ञ्चा॒मि॒ र॒श॒ना र॑श॒ना मु॑ञ्चामि ते ते मुञ्चामि रश॒नाः । \newline
20. मु॒ञ्चा॒मि॒ र॒श॒ना र॑श॒ना मु॑ञ्चामि मुञ्चामि रश॒ना वि वि र॑श॒ना मु॑ञ्चामि मुञ्चामि रश॒ना वि । \newline
21. र॒श॒ना वि वि र॑श॒ना र॑श॒ना वि र॒श्मीन् र॒श्मीन्. वि र॑श॒ना र॑श॒ना वि र॒श्मीन् । \newline
22. वि र॒श्मीन् र॒श्मीन्. वि वि र॒श्मी नितीति॑ र॒श्मीन्. वि वि र॒श्मी निति॑ । \newline
23. र॒श्मी नितीति॑ र॒श्मीन् र॒श्मी नित्या॑हा॒हे ति॑ र॒श्मीन् र॒श्मी नित्या॑ह । \newline
24. इत्या॑हा॒हे तीत्या॑है॒ष ए॒ष आ॒हे तीत्या॑है॒षः । \newline
25. आ॒है॒ष ए॒ष आ॑हा है॒ष वै वा ए॒ष आ॑हा है॒ष वै । \newline
26. ए॒ष वै वा ए॒ष ए॒ष वा अ॒ग्ने र॒ग्नेर् वा ए॒ष ए॒ष वा अ॒ग्नेः । \newline
27. वा अ॒ग्नेर॒ग्नेर् वै वा अ॒ग्नेर् वि॑मो॒को वि॑मो॒को᳚ ऽग्नेर् वै वा अ॒ग्नेर् वि॑मो॒कः । \newline
28. अ॒ग्नेर् वि॑मो॒को वि॑मो॒को᳚ ऽग्ने र॒ग्नेर् वि॑मो॒क स्तेन॒ तेन॑ विमो॒को᳚ ऽग्ने र॒ग्नेर् वि॑मो॒क स्तेन॑ । \newline
29. वि॒मो॒क स्तेन॒ तेन॑ विमो॒को वि॑मो॒क स्तेनै॒वैव तेन॑ विमो॒को वि॑मो॒क स्तेनै॒व । \newline
30. वि॒मो॒क इति॑ वि - मो॒कः । \newline
31. तेनै॒वैव तेन॒ तेनै॒वैन॑ मेन मे॒व तेन॒ तेनै॒वैन᳚म् । \newline
32. ए॒वैन॑ मेन मे॒वैवैनं॒ ॅवि व्ये॑न मे॒वैवैनं॒ ॅवि । \newline
33. ए॒नं॒ ॅवि व्ये॑न मेनं॒ ॅवि मु॑ञ्चति मुञ्चति॒ व्ये॑न मेनं॒ ॅवि मु॑ञ्चति । \newline
34. वि मु॑ञ्चति मुञ्चति॒ वि वि मु॑ञ्चति॒ विष्णो॒र् विष्णो᳚र् मुञ्चति॒ वि वि मु॑ञ्चति॒ विष्णोः᳚ । \newline
35. मु॒ञ्च॒ति॒ विष्णो॒र् विष्णो᳚र् मुञ्चति मुञ्चति॒ विष्णोः᳚ शं॒ॅयोः शं॒ॅयोर् विष्णो᳚र् मुञ्चति मुञ्चति॒ विष्णोः᳚ शं॒ॅयोः । \newline
36. विष्णोः᳚ शं॒ॅयोः शं॒ॅयोर् विष्णो॒र् विष्णोः᳚ शं॒ॅयोर॒ह म॒हꣳ शं॒ॅयोर् विष्णो॒र् विष्णोः᳚ शं॒ॅयोर॒हम् । \newline
37. शं॒ॅयोर॒ह म॒हꣳ शं॒ॅयोः शं॒ॅयोर॒हम् दे॑वय॒ज्यया॑ देवय॒ज्यया॒ ऽहꣳ शं॒ॅयोः शं॒ॅयोर॒हम् दे॑वय॒ज्यया᳚ । \newline
38. शं॒ॅयोरिति॑ शं - योः । \newline
39. अ॒हम् दे॑वय॒ज्यया॑ देवय॒ज्यया॒ ऽह म॒हम् दे॑वय॒ज्यया॑ य॒ज्ञेन॑ य॒ज्ञेन॑ देवय॒ज्यया॒ ऽह म॒हम् दे॑वय॒ज्यया॑ य॒ज्ञेन॑ । \newline
40. दे॒व॒य॒ज्यया॑ य॒ज्ञेन॑ य॒ज्ञेन॑ देवय॒ज्यया॑ देवय॒ज्यया॑ य॒ज्ञेन॑ प्रति॒ष्ठाम् प्र॑ति॒ष्ठां ॅय॒ज्ञेन॑ देवय॒ज्यया॑ देवय॒ज्यया॑ य॒ज्ञेन॑ प्रति॒ष्ठाम् । \newline
41. दे॒व॒य॒ज्ययेति॑ देव - य॒ज्यया᳚ । \newline
42. य॒ज्ञेन॑ प्रति॒ष्ठाम् प्र॑ति॒ष्ठां ॅय॒ज्ञेन॑ य॒ज्ञेन॑ प्रति॒ष्ठाम् ग॑मेयम् गमेयम् प्रति॒ष्ठां ॅय॒ज्ञेन॑ य॒ज्ञेन॑ प्रति॒ष्ठाम् ग॑मेयम् । \newline
43. प्र॒ति॒ष्ठाम् ग॑मेयम् गमेयम् प्रति॒ष्ठाम् प्र॑ति॒ष्ठाम् ग॑मेय॒ मितीति॑ गमेयम् प्रति॒ष्ठाम् प्र॑ति॒ष्ठाम् ग॑मेय॒ मिति॑ । \newline
44. प्र॒ति॒ष्ठामिति॑ प्रति - स्थाम् । \newline
45. ग॒मे॒य॒ मितीति॑ गमेयम् गमेय॒ मित्या॑हा॒हे ति॑ गमेयम् गमेय॒ मित्या॑ह । \newline
46. इत्या॑हा॒हे तीत्या॑ह य॒ज्ञो य॒ज्ञ् आ॒हे तीत्या॑ह य॒ज्ञ्ः । \newline
47. आ॒ह॒ य॒ज्ञो य॒ज्ञ् आ॑हाह य॒ज्ञो वै वै य॒ज्ञ् आ॑हाह य॒ज्ञो वै । \newline
48. य॒ज्ञो वै वै य॒ज्ञो य॒ज्ञो वै विष्णु॒र् विष्णु॒र् वै य॒ज्ञो य॒ज्ञो वै विष्णुः॑ । \newline
49. वै विष्णु॒र् विष्णु॒र् वै वै विष्णु॑र् य॒ज्ञे य॒ज्ञे विष्णु॒र् वै वै विष्णु॑र् य॒ज्ञे । \newline
50. विष्णु॑र् य॒ज्ञे य॒ज्ञे विष्णु॒र् विष्णु॑र् य॒ज्ञ् ए॒वैव य॒ज्ञे विष्णु॒र् विष्णु॑र् य॒ज्ञ् ए॒व । \newline
51. य॒ज्ञ् ए॒वैव य॒ज्ञे य॒ज्ञ् ए॒वान्त॒तो᳚ ऽन्त॒त ए॒व य॒ज्ञे य॒ज्ञ् ए॒वान्त॒तः । \newline
52. ए॒वान्त॒तो᳚ ऽन्त॒त ए॒वैवान्त॒तः प्रति॒ प्रत्य॑न्त॒त ए॒वैवान्त॒तः प्रति॑ । \newline
53. अ॒न्त॒तः प्रति॒ प्रत्य॑न्त॒तो᳚ ऽन्त॒तः प्रति॑ तिष्ठति तिष्ठति॒ प्रत्य॑न्त॒तो᳚ ऽन्त॒तः प्रति॑ तिष्ठति । \newline
54. प्रति॑ तिष्ठति तिष्ठति॒ प्रति॒ प्रति॑ तिष्ठति॒ सोम॑स्य॒ सोम॑स्य तिष्ठति॒ प्रति॒ प्रति॑ तिष्ठति॒ सोम॑स्य । \newline
55. ति॒ष्ठ॒ति॒ सोम॑स्य॒ सोम॑स्य तिष्ठति तिष्ठति॒ सोम॑स्या॒ह म॒हꣳ सोम॑स्य तिष्ठति तिष्ठति॒ सोम॑स्या॒हम् । \newline
56. सोम॑स्या॒ह म॒हꣳ सोम॑स्य॒ सोम॑स्या॒हम् दे॑वय॒ज्यया॑ देवय॒ज्यया॒ ऽहꣳ सोम॑स्य॒ सोम॑स्या॒हम् दे॑वय॒ज्यया᳚ । \newline
57. अ॒हम् दे॑वय॒ज्यया॑ देवय॒ज्यया॒ ऽह म॒हम् दे॑वय॒ज्यया॑ सु॒रेताः᳚ सु॒रेता॑ देवय॒ज्यया॒ ऽह म॒हम् दे॑वय॒ज्यया॑ सु॒रेताः᳚ । \newline
58. दे॒व॒य॒ज्यया॑ सु॒रेताः᳚ सु॒रेता॑ देवय॒ज्यया॑ देवय॒ज्यया॑ सु॒रेता॒ रेतो॒ रेतः॑ सु॒रेता॑ देवय॒ज्यया॑ देवय॒ज्यया॑ सु॒रेता॒ रेतः॑ । \newline
59. दे॒व॒य॒ज्ययेति॑ देव - य॒ज्यया᳚ । \newline
60. सु॒रेता॒ रेतो॒ रेतः॑ सु॒रेताः᳚ सु॒रेता॒ रेतो॑ धिषीय धिषीय॒ रेतः॑ सु॒रेताः᳚ सु॒रेता॒ रेतो॑ धिषीय । \newline
61. सु॒रेता॒ इति॑ सु - रेताः᳚ । \newline
\pagebreak
\markright{ TS 1.7.4.5  \hfill https://www.vedavms.in \hfill}
\addcontentsline{toc}{section}{ TS 1.7.4.5 }
\section*{ TS 1.7.4.5 }

\textbf{TS 1.7.4.5 } \newline
\textbf{Samhita Paata} \newline

रेतो॑ धिषी॒येत्या॑ह॒ सोमो॒ वै रे॑तो॒धास्तेनै॒व रेत॑ आ॒त्मन् ध॑त्ते॒ त्वष्टु॑र॒हं दे॑वय॒ज्यया॑ पशू॒नाꣳ रू॒पं पु॑षेय॒मित्या॑ह॒ त्वष्टा॒ वै प॑शू॒नां मि॑थु॒नानाꣳ॑ रूप॒कृत्तेनै॒व प॑शू॒नाꣳ रू॒पमा॒त्मन् ध॑त्ते दे॒वानां॒ पत्नी॑र॒ग्निर् गृ॒हप॑तिर् य॒ज्ञ्स्य॑ मिथु॒नं तयो॑र॒हं दे॑वय॒ज्यया॑ मिथु॒नेन॒ प्रभू॑यास॒-मित्या॑है॒तस्मा॒द् वै मि॑थु॒नात् प्र॒जाप॑तिर् मिथु॒नेन॒ - [ ] \newline

\textbf{Pada Paata} \newline

रेतः॑ । धि॒षी॒य॒ । इति॑ । आ॒ह॒ । सोमः॑ । वै । रे॒तो॒धा इति॑ रेतः-धाः । तेन॑ । ए॒व । रेतः॑ । आ॒त्मन्न् । ध॒त्ते॒ । त्वष्टुः॑ । अ॒हम् । दे॒व॒य॒ज्ययेति॑ देव - य॒ज्यया᳚ । प॒शू॒नाम् । रू॒पम् । पु॒षे॒य॒म् । इति॑ । आ॒ह॒ । त्वष्टा᳚ । वै । प॒शू॒नाम् । मि॒थु॒नाना᳚म् । रू॒प॒कृदिति॑ रूप - कृत् । तेन॑ । ए॒व । प॒शू॒नाम् । रू॒पम् । आ॒त्मन्न् । ध॒त्ते॒ । दे॒वाना᳚म् । पत्नीः᳚ । अ॒ग्निः । गृ॒हप॑ति॒रिति॑ गृ॒ह - प॒तिः॒ । य॒ज्ञ्स्य॑ । मि॒थु॒नम् । तयोः᳚ । अ॒हम् । दे॒व॒य॒ज्ययेति॑ देव - य॒ज्यया᳚ । मि॒थु॒नेन॑ । प्रेति॑ । भू॒या॒स॒म् । इति॑ । आ॒ह॒ । ए॒तस्मा᳚त् । वै । मि॒थु॒नात् । प्र॒जाप॑ति॒रिति॑ प्र॒जा - प॒तिः॒ । मि॒थु॒नेन॑ ।  \newline


\textbf{Krama Paata} \newline

रेतो॑ धिषीय । धि॒षी॒येति॑ । इत्या॑ह । आ॒ह॒ सोमः॑ । सोमो॒ वै । वै रे॑तो॒धाः । रे॒तो॒धास्तेन॑ । रे॒तो॒धा इति॑ रेतः - धाः । तेनै॒व । ए॒व रेतः॑ । रेत॑ आ॒त्मन्न् । आ॒त्मन्,ध॑त्ते । ध॒त्ते॒ त्वष्टुः॑ । त्वष्टु॑र॒हम् । अ॒हम् दे॑वय॒ज्यया᳚ । दे॒व॒य॒ज्यया॑ पशू॒नाम् । दे॒व॒य॒ज्ययेति॑ देव - य॒ज्यया᳚ । प॒शू॒नाꣳ रू॒पम् । रू॒पम् पु॑षेयम् । पु॒षे॒य॒मिति॑ । इत्या॑ह । आ॒ह॒ त्वष्टा᳚ । त्वष्टा॒ वै । वै प॑शू॒नाम् । प॒शू॒नाम् मि॑थु॒नाना᳚म् । मि॒थु॒नानाꣳ॑ रूप॒कृत् । रू॒प॒कृत्,तेन॑ । रू॒प॒कृदिति॑ रूप - कृत् । तेनै॒व । ए॒व प॑शू॒नाम् । प॒शू॒नाꣳ रू॒पम् । रू॒पमा॒त्मन्न् । आ॒त्मन्,ध॑त्ते । ध॒त्ते॒ दे॒वाना᳚म् । दे॒वाना॒म् पत्नीः᳚ । पत्नी॑र॒ग्निः । अ॒ग्निर्,गृ॒हप॑तिः । गृ॒हप॑तिर्,य॒ज्ञ्स्य॑ । गृ॒हप॑ति॒रिति॑ गृ॒ह - प॒तिः॒ । य॒ज्ञ्स्य॑ मिथु॒नम् । मि॒थु॒नम् तयोः᳚ । तयो॑र॒हम् । अ॒हम् दे॑वय॒ज्यया᳚ । दे॒व॒य॒ज्यया॑ मिथु॒नेन॑ । दे॒व॒य॒ज्ययेति॑ देव - य॒ज्यया᳚ । मि॒थु॒नेन॒ प्र । प्र भू॑यासम् । भू॒या॒स॒मिति॑ । इत्या॑ह । आ॒है॒तस्मा᳚त् । ए॒तस्मा॒द् वै । वै मि॑थु॒नात् । मि॒थु॒नात् प्र॒जाप॑तिः । प्र॒जाप॑तिर् मिथु॒नेन॑ । प्र॒जाप॑ति॒रिति॑ प्र॒जा - प॒तिः॒ । मि॒थु॒नेन॒ प्र \newline

\textbf{Jatai Paata} \newline

1. रेतो॑ धिषीय धिषीय॒ रेतो॒ रेतो॑ धिषीय । \newline
2. धि॒षी॒ये तीति॑ धिषीय धिषी॒ये ति॑ । \newline
3. इत्या॑हा॒हे तीत्या॑ह । \newline
4. आ॒ह॒ सोमः॒ सोम॑ आहाह॒ सोमः॑ । \newline
5. सोमो॒ वै वै सोमः॒ सोमो॒ वै । \newline
6. वै रे॑तो॒धा रे॑तो॒धा वै वै रे॑तो॒धाः । \newline
7. रे॒तो॒धा स्तेन॒ तेन॑ रेतो॒धा रे॑तो॒धा स्तेन॑ । \newline
8. रे॒तो॒धा इति॑ रेतः - धाः । \newline
9. तेनै॒वैव तेन॒ तेनै॒व । \newline
10. ए॒व रेतो॒ रेत॑ ए॒वैव रेतः॑ । \newline
11. रेत॑ आ॒त्मन् ना॒त्मन् रेतो॒ रेत॑ आ॒त्मन्न् । \newline
12. आ॒त्मन् ध॑त्ते धत्त आ॒त्मन् ना॒त्मन् ध॑त्ते । \newline
13. ध॒त्ते॒ त्वष्टु॒ स्त्वष्टु॑र् धत्ते धत्ते॒ त्वष्टुः॑ । \newline
14. त्वष्टु॑ र॒ह म॒हम् त्वष्टु॒ स्त्वष्टु॑ र॒हम् । \newline
15. अ॒हम् दे॑वय॒ज्यया॑ देवय॒ज्यया॒ ऽह म॒हम् दे॑वय॒ज्यया᳚ । \newline
16. दे॒व॒य॒ज्यया॑ पशू॒नाम् प॑शू॒नाम् दे॑वय॒ज्यया॑ देवय॒ज्यया॑ पशू॒नाम् । \newline
17. दे॒व॒य॒ज्ययेति॑ देव - य॒ज्यया᳚ । \newline
18. प॒शू॒नाꣳ रू॒पꣳ रू॒पम् प॑शू॒नाम् प॑शू॒नाꣳ रू॒पम् । \newline
19. रू॒पम् पु॑षेयम् पुषेयꣳ रू॒पꣳ रू॒पम् पु॑षेयम् । \newline
20. पु॒षे॒य॒ मितीति॑ पुषेयम् पुषेय॒ मिति॑ । \newline
21. इत्या॑ हा॒हे तीत्या॑ह । \newline
22. आ॒ह॒ त्वष्टा॒ त्वष्टा॑ ऽऽहाह॒ त्वष्टा᳚ । \newline
23. त्वष्टा॒ वै वै त्वष्टा॒ त्वष्टा॒ वै । \newline
24. वै प॑शू॒नाम् प॑शू॒नां ॅवै वै प॑शू॒नाम् । \newline
25. प॒शू॒नाम् मि॑थु॒नाना᳚म् मिथु॒नाना᳚म् पशू॒नाम् प॑शू॒नाम् मि॑थु॒नाना᳚म् । \newline
26. मि॒थु॒नानाꣳ॑ रूप॒कृद् रू॑प॒कृण् मि॑थु॒नाना᳚म् मिथु॒नानाꣳ॑ रूप॒कृत् । \newline
27. रू॒प॒कृत् तेन॒ तेन॑ रूप॒कृद् रू॑प॒कृत् तेन॑ । \newline
28. रू॒प॒कृदिति॑ रूप - कृत् । \newline
29. तेनै॒ वैव तेन॒ तेनै॒व । \newline
30. ए॒व प॑शू॒नाम् प॑शू॒ना मे॒वैव प॑शू॒नाम् । \newline
31. प॒शू॒नाꣳ रू॒पꣳ रू॒पम् प॑शू॒नाम् प॑शू॒नाꣳ रू॒पम् । \newline
32. रू॒प मा॒त्मन् ना॒त्मन् रू॒पꣳ रू॒प मा॒त्मन्न् । \newline
33. आ॒त्मन् ध॑त्ते धत्त आ॒त्मन् ना॒त्मन् ध॑त्ते । \newline
34. ध॒त्ते॒ दे॒वाना᳚म् दे॒वाना᳚म् धत्ते धत्ते दे॒वाना᳚म् । \newline
35. दे॒वाना॒म् पत्नीः॒ पत्नी᳚र् दे॒वाना᳚म् दे॒वाना॒म् पत्नीः᳚ । \newline
36. पत्नी॑ र॒ग्नि र॒ग्निः पत्नीः॒ पत्नी॑ र॒ग्निः । \newline
37. अ॒ग्निर् गृ॒हप॑तिर् गृ॒हप॑ति र॒ग्नि र॒ग्निर् गृ॒हप॑तिः । \newline
38. गृ॒हप॑तिर् य॒ज्ञ्स्य॑ य॒ज्ञ्स्य॑ गृ॒हप॑तिर् गृ॒हप॑तिर् य॒ज्ञ्स्य॑ । \newline
39. गृ॒हप॑ति॒रिति॑ गृ॒ह - प॒तिः॒ । \newline
40. य॒ज्ञ्स्य॑ मिथु॒नम् मि॑थु॒नं ॅय॒ज्ञ्स्य॑ य॒ज्ञ्स्य॑ मिथु॒नम् । \newline
41. मि॒थु॒नम् तयो॒ स्तयो᳚र् मिथु॒नम् मि॑थु॒नम् तयोः᳚ । \newline
42. तयो॑ र॒ह म॒हम् तयो॒ स्तयो॑ र॒हम् । \newline
43. अ॒हम् दे॑वय॒ज्यया॑ देवय॒ज्यया॒ ऽह म॒हम् दे॑वय॒ज्यया᳚ । \newline
44. दे॒व॒य॒ज्यया॑ मिथु॒नेन॑ मिथु॒नेन॑ देवय॒ज्यया॑ देवय॒ज्यया॑ मिथु॒नेन॑ । \newline
45. दे॒व॒य॒ज्ययेति॑ देव - य॒ज्यया᳚ । \newline
46. मि॒थु॒नेन॒ प्र प्र मि॑थु॒नेन॑ मिथु॒नेन॒ प्र । \newline
47. प्र भू॑यासम् भूयास॒म् प्र प्र भू॑यासम् । \newline
48. भू॒या॒स॒ मितीति॑ भूयासम् भूयास॒ मिति॑ । \newline
49. इत्या॑ हा॒हे तीत्या॑ह । \newline
50. आ॒है॒तस्मा॑ दे॒तस्मा॑ दाहा है॒तस्मा᳚त् । \newline
51. ए॒तस्मा॒द् वै वा ए॒तस्मा॑ दे॒तस्मा॒द् वै । \newline
52. वै मि॑थु॒नान् मि॑थु॒नाद् वै वै मि॑थु॒नात् । \newline
53. मि॒थु॒नात् प्र॒जाप॑तिः प्र॒जाप॑तिर् मिथु॒नान् मि॑थु॒नात् प्र॒जाप॑तिः । \newline
54. प्र॒जाप॑तिर् मिथु॒नेन॑ मिथु॒नेन॑ प्र॒जाप॑तिः प्र॒जाप॑तिर् मिथु॒नेन॑ । \newline
55. प्र॒जाप॑ति॒रिति॑ प्र॒जा - प॒तिः॒ । \newline
56. मि॒थु॒नेन॒ प्र प्र मि॑थु॒नेन॑ मिथु॒नेन॒ प्र । \newline

\textbf{Ghana Paata } \newline

1. रेतो॑ धिषीय धिषीय॒ रेतो॒ रेतो॑ धिषी॒ये तीति॑ धिषीय॒ रेतो॒ रेतो॑ धिषी॒ये ति॑ । \newline
2. धि॒षी॒ये तीति॑ धिषीय धिषी॒ये त्या॑हा॒हे ति॑ धिषीय धिषी॒ये त्या॑ह । \newline
3. इत्या॑हा॒हे तीत्या॑ह॒ सोमः॒ सोम॑ आ॒हे तीत्या॑ह॒ सोमः॑ । \newline
4. आ॒ह॒ सोमः॒ सोम॑ आहाह॒ सोमो॒ वै वै सोम॑ आहाह॒ सोमो॒ वै । \newline
5. सोमो॒ वै वै सोमः॒ सोमो॒ वै रे॑तो॒धा रे॑तो॒धा वै सोमः॒ सोमो॒ वै रे॑तो॒धाः । \newline
6. वै रे॑तो॒धा रे॑तो॒धा वै वै रे॑तो॒धा स्तेन॒ तेन॑ रेतो॒धा वै वै रे॑तो॒धा स्तेन॑ । \newline
7. रे॒तो॒धा स्तेन॒ तेन॑ रेतो॒धा रे॑तो॒धा स्तेनै॒वैव तेन॑ रेतो॒धा रे॑तो॒धा स्तेनै॒व । \newline
8. रे॒तो॒धा इति॑ रेतः - धाः । \newline
9. तेनै॒वैव तेन॒ तेनै॒व रेतो॒ रेत॑ ए॒व तेन॒ तेनै॒व रेतः॑ । \newline
10. ए॒व रेतो॒ रेत॑ ए॒वैव रेत॑ आ॒त्मन् ना॒त्मन् रेत॑ ए॒वैव रेत॑ आ॒त्मन्न् । \newline
11. रेत॑ आ॒त्मन् ना॒त्मन् रेतो॒ रेत॑ आ॒त्मन् ध॑त्ते धत्त आ॒त्मन् रेतो॒ रेत॑ आ॒त्मन् ध॑त्ते । \newline
12. आ॒त्मन् ध॑त्ते धत्त आ॒त्मन् ना॒त्मन् ध॑त्ते॒ त्वष्टु॒ स्त्वष्टु॑र् धत्त आ॒त्मन् ना॒त्मन् ध॑त्ते॒ त्वष्टुः॑ । \newline
13. ध॒त्ते॒ त्वष्टु॒ स्त्वष्टु॑र् धत्ते धत्ते॒ त्वष्टु॑र॒ह म॒हम् त्वष्टु॑र् धत्ते धत्ते॒ त्वष्टु॑ र॒हम् । \newline
14. त्वष्टु॑ र॒ह म॒हम् त्वष्टु॒ स्त्वष्टु॑ र॒हम् दे॑वय॒ज्यया॑ देवय॒ज्यया॒ ऽहम् त्वष्टु॒ स्त्वष्टु॑ र॒हम् दे॑वय॒ज्यया᳚ । \newline
15. अ॒हम् दे॑वय॒ज्यया॑ देवय॒ज्यया॒ ऽह म॒हम् दे॑वय॒ज्यया॑ पशू॒नाम् प॑शू॒नाम् दे॑वय॒ज्यया॒ ऽह म॒हम् दे॑वय॒ज्यया॑ पशू॒नाम् । \newline
16. दे॒व॒य॒ज्यया॑ पशू॒नाम् प॑शू॒नाम् दे॑वय॒ज्यया॑ देवय॒ज्यया॑ पशू॒नाꣳ रू॒पꣳ रू॒पम् प॑शू॒नाम् दे॑वय॒ज्यया॑ देवय॒ज्यया॑ पशू॒नाꣳ रू॒पम् । \newline
17. दे॒व॒य॒ज्ययेति॑ देव - य॒ज्यया᳚ । \newline
18. प॒शू॒नाꣳ रू॒पꣳ रू॒पम् प॑शू॒नाम् प॑शू॒नाꣳ रू॒पम् पु॑षेयम् पुषेयꣳ रू॒पम् प॑शू॒नाम् प॑शू॒नाꣳ रू॒पम् पु॑षेयम् । \newline
19. रू॒पम् पु॑षेयम् पुषेयꣳ रू॒पꣳ रू॒पम् पु॑षेय॒ मितीति॑ पुषेयꣳ रू॒पꣳ रू॒पम् पु॑षेय॒ मिति॑ । \newline
20. पु॒षे॒य॒ मितीति॑ पुषेयम् पुषेय॒ मित्या॑हा॒हे ति॑ पुषेयम् पुषेय॒ मित्या॑ह । \newline
21. इत्या॑हा॒हे तीत्या॑ह॒ त्वष्टा॒ त्वष्टा॒ ऽऽहे तीत्या॑ह॒ त्वष्टा᳚ । \newline
22. आ॒ह॒ त्वष्टा॒ त्वष्टा॑ ऽऽहाह॒ त्वष्टा॒ वै वै त्वष्टा॑ ऽऽहाह॒ त्वष्टा॒ वै । \newline
23. त्वष्टा॒ वै वै त्वष्टा॒ त्वष्टा॒ वै प॑शू॒नाम् प॑शू॒नां ॅवै त्वष्टा॒ त्वष्टा॒ वै प॑शू॒नाम् । \newline
24. वै प॑शू॒नाम् प॑शू॒नां ॅवै वै प॑शू॒नाम् मि॑थु॒नाना᳚म् मिथु॒नाना᳚म् पशू॒नां ॅवै वै प॑शू॒नाम् मि॑थु॒नाना᳚म् । \newline
25. प॒शू॒नाम् मि॑थु॒नाना᳚म् मिथु॒नाना᳚म् पशू॒नाम् प॑शू॒नाम् मि॑थु॒नाना(ग्म्॑) रूप॒कृद् रू॑प॒कृण् मि॑थु॒नाना᳚म् पशू॒नाम् प॑शू॒नाम् मि॑थु॒नाना(ग्म्॑) रूप॒कृत् । \newline
26. मि॒थु॒नाना(ग्म्॑) रूप॒कृद् रू॑प॒कृण् मि॑थु॒नाना᳚म् मिथु॒नाना(ग्म्॑) रूप॒कृत् तेन॒ तेन॑ रूप॒कृण् मि॑थु॒नाना᳚म् मिथु॒नाना(ग्म्॑) रूप॒कृत् तेन॑ । \newline
27. रू॒प॒कृत् तेन॒ तेन॑ रूप॒कृद् रू॑प॒कृत् तेनै॒वैव तेन॑ रूप॒कृद् रू॑प॒कृत् तेनै॒व । \newline
28. रू॒प॒कृदिति॑ रूप - कृत् । \newline
29. तेनै॒वैव तेन॒ तेनै॒व प॑शू॒नाम् प॑शू॒ना मे॒व तेन॒ तेनै॒व प॑शू॒नाम् । \newline
30. ए॒व प॑शू॒नाम् प॑शू॒ना मे॒वैव प॑शू॒नाꣳ रू॒पꣳ रू॒पम् प॑शू॒ना मे॒वैव प॑शू॒नाꣳ रू॒पम् । \newline
31. प॒शू॒नाꣳ रू॒पꣳ रू॒पम् प॑शू॒नाम् प॑शू॒नाꣳ रू॒प मा॒त्मन् ना॒त्मन् रू॒पम् प॑शू॒नाम् प॑शू॒नाꣳ रू॒प मा॒त्मन्न् । \newline
32. रू॒प मा॒त्मन् ना॒त्मन् रू॒पꣳ रू॒प मा॒त्मन् ध॑त्ते धत्त आ॒त्मन् रू॒पꣳ रू॒प मा॒त्मन् ध॑त्ते । \newline
33. आ॒त्मन् ध॑त्ते धत्त आ॒त्मन् ना॒त्मन् ध॑त्ते दे॒वाना᳚म् दे॒वाना᳚म् धत्त आ॒त्मन् ना॒त्मन् ध॑त्ते दे॒वाना᳚म् । \newline
34. ध॒त्ते॒ दे॒वाना᳚म् दे॒वाना᳚म् धत्ते धत्ते दे॒वाना॒म् पत्नीः॒ पत्नी᳚र् दे॒वाना᳚म् धत्ते धत्ते दे॒वाना॒म् पत्नीः᳚ । \newline
35. दे॒वाना॒म् पत्नीः॒ पत्नी᳚र् दे॒वाना᳚म् दे॒वाना॒म् पत्नी॑ र॒ग्नि र॒ग्निः पत्नी᳚र् दे॒वाना᳚म् दे॒वाना॒म् पत्नी॑ र॒ग्निः । \newline
36. पत्नी॑ र॒ग्नि र॒ग्निः पत्नीः॒ पत्नी॑ र॒ग्निर् गृ॒हप॑तिर् गृ॒हप॑ति र॒ग्निः पत्नीः॒ पत्नी॑ र॒ग्निर् गृ॒हप॑तिः । \newline
37. अ॒ग्निर् गृ॒हप॑तिर् गृ॒हप॑ति र॒ग्नि र॒ग्निर् गृ॒हप॑तिर् य॒ज्ञ्स्य॑ य॒ज्ञ्स्य॑ गृ॒हप॑ति र॒ग्नि र॒ग्निर् गृ॒हप॑तिर् य॒ज्ञ्स्य॑ । \newline
38. गृ॒हप॑तिर् य॒ज्ञ्स्य॑ य॒ज्ञ्स्य॑ गृ॒हप॑तिर् गृ॒हप॑तिर् य॒ज्ञ्स्य॑ मिथु॒नम् मि॑थु॒नं ॅय॒ज्ञ्स्य॑ गृ॒हप॑तिर् गृ॒हप॑तिर् य॒ज्ञ्स्य॑ मिथु॒नम् । \newline
39. गृ॒हप॑ति॒रिति॑ गृ॒ह - प॒तिः॒ । \newline
40. य॒ज्ञ्स्य॑ मिथु॒नम् मि॑थु॒नं ॅय॒ज्ञ्स्य॑ य॒ज्ञ्स्य॑ मिथु॒नम् तयो॒स्तयो᳚र् मिथु॒नं ॅय॒ज्ञ्स्य॑ य॒ज्ञ्स्य॑ मिथु॒नम् तयोः᳚ । \newline
41. मि॒थु॒नम् तयो॒ स्तयो᳚र् मिथु॒नम् मि॑थु॒नम् तयो॑र॒ह म॒हम् तयो᳚र् मिथु॒नम् मि॑थु॒नम् तयो॑र॒हम् । \newline
42. तयो॑र॒ह म॒हम् तयो॒ स्तयो॑ र॒हम् दे॑वय॒ज्यया॑ देवय॒ज्यया॒ ऽहम् तयो॒ स्तयो॑ र॒हम् दे॑वय॒ज्यया᳚ । \newline
43. अ॒हम् दे॑वय॒ज्यया॑ देवय॒ज्यया॒ ऽह म॒हम् दे॑वय॒ज्यया॑ मिथु॒नेन॑ मिथु॒नेन॑ देवय॒ज्यया॒ ऽह म॒हम् दे॑वय॒ज्यया॑ मिथु॒नेन॑ । \newline
44. दे॒व॒य॒ज्यया॑ मिथु॒नेन॑ मिथु॒नेन॑ देवय॒ज्यया॑ देवय॒ज्यया॑ मिथु॒नेन॒ प्र प्र मि॑थु॒नेन॑ देवय॒ज्यया॑ देवय॒ज्यया॑ मिथु॒नेन॒ प्र । \newline
45. दे॒व॒य॒ज्ययेति॑ देव - य॒ज्यया᳚ । \newline
46. मि॒थु॒नेन॒ प्र प्र मि॑थु॒नेन॑ मिथु॒नेन॒ प्र भू॑यासम् भूयास॒म् प्र मि॑थु॒नेन॑ मिथु॒नेन॒ प्र भू॑यासम् । \newline
47. प्र भू॑यासम् भूयास॒म् प्र प्र भू॑यास॒ मितीति॑ भूयास॒म् प्र प्र भू॑यास॒ मिति॑ । \newline
48. भू॒या॒स॒ मितीति॑ भूयासम् भूयास॒ मित्या॑हा॒हे ति॑ भूयासम् भूयास॒ मित्या॑ह । \newline
49. इत्या॑हा॒हे तीत्या॑ है॒तस्मा॑ दे॒तस्मा॑दा॒हे तीत्या॑ है॒तस्मा᳚त् । \newline
50. आ॒है॒ तस्मा॑द् ए॒तस्मा॑ दाहा है॒तस्मा॒द् वै वा ए॒तस्मा॑ दाहा है॒तस्मा॒द् वै । \newline
51. ए॒तस्मा॒द् वै वा ए॒तस्मा॑ दे॒तस्मा॒द् वै मि॑थु॒नान् मि॑थु॒नाद् वा ए॒तस्मा॑ दे॒तस्मा॒द् वै मि॑थु॒नात् । \newline
52. वै मि॑थु॒नान् मि॑थु॒नाद् वै वै मि॑थु॒नात् प्र॒जाप॑तिः प्र॒जाप॑तिर् मिथु॒नाद् वै वै मि॑थु॒नात् प्र॒जाप॑तिः । \newline
53. मि॒थु॒नात् प्र॒जाप॑तिः प्र॒जाप॑तिर् मिथु॒नान् मि॑थु॒नात् प्र॒जाप॑तिर् मिथु॒नेन॑ मिथु॒नेन॑ प्र॒जाप॑तिर् मिथु॒नान् मि॑थु॒नात् प्र॒जाप॑तिर् मिथु॒नेन॑ । \newline
54. प्र॒जाप॑तिर् मिथु॒नेन॑ मिथु॒नेन॑ प्र॒जाप॑तिः प्र॒जाप॑तिर् मिथु॒नेन॒ प्र प्र मि॑थु॒नेन॑ प्र॒जाप॑तिः प्र॒जाप॑तिर् मिथु॒नेन॒ प्र । \newline
55. प्र॒जाप॑ति॒रिति॑ प्र॒जा - प॒तिः॒ । \newline
56. मि॒थु॒नेन॒ प्र प्र मि॑थु॒नेन॑ मिथु॒नेन॒ प्राजा॑यता जायत॒ प्र मि॑थु॒नेन॑ मिथु॒नेन॒ प्राजा॑यत । \newline
\pagebreak
\markright{ TS 1.7.4.6  \hfill https://www.vedavms.in \hfill}
\addcontentsline{toc}{section}{ TS 1.7.4.6 }
\section*{ TS 1.7.4.6 }

\textbf{TS 1.7.4.6 } \newline
\textbf{Samhita Paata} \newline

प्राजा॑यत॒ तस्मा॑दे॒व यज॑मानो मिथु॒नेन॒ प्रजा॑यते वे॒दो॑ऽसि॒ वित्ति॑रसि वि॒देयेत्या॑ह वे॒देन॒ वै दे॒वा असु॑राणां ॅवि॒त्तं ॅवेद्य॑मविन्दन्त॒ तद्-वे॒दस्य॑ वेद॒त्वं ॅयद्य॒द् भ्रातृ॑व्यस्याभि॒द्ध्याये॒त् तस्य॒ नाम॑ गृह्णीया॒त् तदे॒वास्य॒ सर्वं॑ ॅवृङ्क्ते घृ॒तव॑न्तं कुला॒यिनꣳ॑ रा॒यस्पोषꣳ॑ सह॒स्रिणं॑ ॅवे॒दो द॑दातु वा॒जिन॒मित्या॑ह॒ प्रस॒हस्रं॑ प॒शूना᳚प्नो॒त्या ( ) स्य॑ प्र॒जायां᳚ ॅवा॒जी जा॑यते॒ य ए॒वं ॅवेद॑ ॥ \newline

\textbf{Pada Paata} \newline

प्रेति॑ । अ॒जा॒य॒त॒ । तस्मा᳚त् । ए॒व । यज॑मानः । मि॒थु॒नेन॑ । प्रेति॑ । जा॒य॒ते॒ । वे॒दः । अ॒सि॒ । वित्तिः॑ । अ॒सि॒ । वि॒देय॑ । इति॑ । आ॒ह॒ । वे॒देन॑ । वै । दे॒वाः । असु॑राणाम् । वि॒त्तम् । वेद्य᳚म् । अ॒वि॒न्द॒न्त॒ । तत् । वे॒दस्य॑ । वे॒द॒त्वमिति॑ वेद - त्वम् । यद्य॒दिति॒ यत् - य॒त् । भ्रातृ॑व्यस्य । अ॒भि॒द्ध्याये॒दित्य॑भि-ध्याये᳚त् । तस्य॑ । नाम॑ । गृ॒ह्णी॒या॒त् । तत् । ए॒व । अ॒स्य॒ । सर्व᳚म् । वृ॒ङ्क्ते॒ । घृ॒तव॑न्त॒मिति॑ घृ॒त - व॒न्त॒म् । कु॒ला॒यिन᳚म् । रा॒यः । पोष᳚म् । स॒ह॒स्रिण᳚म् । वे॒दः । द॒दा॒तु॒ । वा॒जिन᳚म् । इति॑ । आ॒ह॒ । प्रेति॑ । स॒हस्र᳚म् । प॒शून् । आ॒प्नो॒ति॒ ( ) । एति॑ । अ॒स्य॒ । प्र॒जाया॒मिति॑ प्र - जाया᳚म् । वा॒जी । जा॒य॒ते॒ । यः । ए॒वम् । वेद॑ ॥  \newline


\textbf{Krama Paata} \newline

प्राजा॑यत । अ॒जा॒य॒त॒ तस्मा᳚त् । तस्मा॑दे॒व । ए॒व यज॑मानः । यज॑मानो मिथु॒नेन॑ । मि॒थु॒नेन॒ प्र । प्र जा॑यते । जा॒य॒ते॒ वे॒दः । वे॒दो॑ऽसि । अ॒सि॒ वित्तिः॑ । वित्ति॑रसि । अ॒सि॒ वि॒देय॑ । वि॒देयेति॑ । इत्या॑ह । आ॒ह॒ वे॒देन॑ । वे॒देन॒ वै । वै दे॒वाः । दे॒वा असु॑राणाम् । असु॑राणां ॅवि॒त्तम् । वि॒त्तं ॅवेद्य᳚म् । वेद्य॑मविन्दन्त । अ॒वि॒न्द॒न्त॒ तत् । तद् वे॒दस्य॑ । वे॒दस्य॑ वेद॒त्वम् । वे॒द॒त्वं ॅयद्य॑त् । वे॒द॒त्वमिति॑ वेद - त्वम् । यद्य॒द् भ्रातृ॑व्यस्य । यद्य॒दिति॒ यत् - य॒त्॒ । भ्रातृ॑व्यस्याभि॒द्ध्याये᳚त् । अ॒भि॒द्ध्याये॒त्,तस्य॑ । अ॒भि॒द्ध्याये॒दित्य॑भि - ध्याये᳚त् । तस्य॒ नाम॑ । नाम॑ गृह्णीयात् । गृ॒ह्णी॒या॒त्,तत् । तदे॒व । ए॒वास्य॑ । अ॒स्य॒ सर्व᳚म् । सर्वं॑ ॅवृङ्ते । वृ॒ङ्ते॒ घृ॒तव॑न्तम् । घृ॒तव॑न्तम् कुला॒यिन᳚म् । घृ॒तव॑न्त॒मिति॑ घृ॒त - व॒न्त॒म् । कु॒ला॒यिनꣳ॑ रा॒यः । रा॒यस्पोष᳚म् । पोषꣳ॑ सह॒स्रिण᳚म् । स॒ह॒स्रिणं॑ ॅवे॒दः । वे॒दो द॑दातु । द॒दा॒तु॒ वा॒जिन᳚म् । वा॒जिन॒मिति॑ । इत्या॑ह । आ॒ह॒ प्र । प्र स॒हस्र᳚म् । स॒हस्र॑म् प॒शून् । प॒शूना᳚प्नोति ( ) । आ॒प्नो॒त्या । आऽस्य॑ । अ॒स्य॒ प्र॒जाया᳚म् । प्र॒जायां᳚ ॅवा॒जी । प्र॒जाया॒मिति॑ प्र - जाया᳚म् । वा॒जी जा॑यते । जा॒य॒ते॒ यः । य ए॒वम् । ए॒वं ॅवेद॑ । वेदेति॒ वेद॑ । \newline

\textbf{Jatai Paata} \newline

1. प्राजा॑यता जायत॒ प्र प्राजा॑यत । \newline
2. अ॒जा॒य॒त॒ तस्मा॒त् तस्मा॑ दजायता जायत॒ तस्मा᳚त् । \newline
3. तस्मा॑ दे॒वैव तस्मा॒त् तस्मा॑ दे॒व । \newline
4. ए॒व यज॑मानो॒ यज॑मान ए॒वैव यज॑मानः । \newline
5. यज॑मानो मिथु॒नेन॑ मिथु॒नेन॒ यज॑मानो॒ यज॑मानो मिथु॒नेन॑ । \newline
6. मि॒थु॒नेन॒ प्र प्र मि॑थु॒नेन॑ मिथु॒नेन॒ प्र । \newline
7. प्र जा॑यते जायते॒ प्र प्र जा॑यते । \newline
8. जा॒य॒ते॒ वे॒दो वे॒दो जा॑यते जायते वे॒दः । \newline
9. वे॒दो᳚ ऽस्यसि वे॒दो वे॒दो॑ ऽसि । \newline
10. अ॒सि॒ वित्ति॒र् वित्ति॑ रस्यसि॒ वित्तिः॑ । \newline
11. वित्ति॑ रस्यसि॒ वित्ति॒र् वित्ति॑ रसि । \newline
12. अ॒सि॒ वि॒देय॑ वि॒देया᳚स्यसि वि॒देय॑ । \newline
13. वि॒देये तीति॑ वि॒देय॑ वि॒देये ति॑ । \newline
14. इत्या॑ हा॒हे तीत्या॑ह । \newline
15. आ॒ह॒ वे॒देन॑ वे॒दे ना॑हाह वे॒देन॑ । \newline
16. वे॒देन॒ वै वै वे॒देन॑ वे॒देन॒ वै । \newline
17. वै दे॒वा दे॒वा वै वै दे॒वाः । \newline
18. दे॒वा असु॑राणा॒ मसु॑राणाम् दे॒वा दे॒वा असु॑राणाम् । \newline
19. असु॑राणां ॅवि॒त्तं ॅवि॒त्त मसु॑राणा॒ मसु॑राणां ॅवि॒त्तम् । \newline
20. वि॒त्तं ॅवेद्यं॒ ॅवेद्यं॑ ॅवि॒त्तं ॅवि॒त्तं ॅवेद्य᳚म् । \newline
21. वेद्य॑ मविन्दन्ता विन्दन्त॒ वेद्यं॒ ॅवेद्य॑ मविन्दन्त । \newline
22. अ॒वि॒न्द॒न्त॒ तत् तद॑विन्दन्ता विन्दन्त॒ तत् । \newline
23. तद् वे॒दस्य॑ वे॒दस्य॒ तत् तद् वे॒दस्य॑ । \newline
24. वे॒दस्य॑ वेद॒त्वं ॅवे॑द॒त्वं ॅवे॒दस्य॑ वे॒दस्य॑ वेद॒त्वम् । \newline
25. वे॒द॒त्वं ॅयद्य॒द् यद्य॑द् वेद॒त्वं ॅवे॑द॒त्वं ॅयद्य॑त् । \newline
26. वे॒द॒त्वमिति॑ वेद - त्वम् । \newline
27. यद्य॒द् भ्रातृ॑व्यस्य॒ भ्रातृ॑व्यस्य॒ यद्य॒द् यद्य॒द् भ्रातृ॑व्यस्य । \newline
28. यद्य॒दिति॒ यत् - य॒त् । \newline
29. भ्रातृ॑व्यस्या भि॒द्ध्याये॑ दभि॒द्ध्याये॒द् भ्रातृ॑व्यस्य॒ भ्रातृ॑व्यस्या भि॒द्ध्याये᳚त् । \newline
30. अ॒भि॒द्ध्याये॒त् तस्य॒ तस्या॑ भि॒द्ध्याये॑ दभि॒द्ध्याये॒त् तस्य॑ । \newline
31. अ॒भि॒द्ध्याये॒दित्य॑भि - ध्याये᳚त् । \newline
32. तस्य॒ नाम॒ नाम॒ तस्य॒ तस्य॒ नाम॑ । \newline
33. नाम॑ गृह्णीयाद् गृह्णीया॒न् नाम॒ नाम॑ गृह्णीयात् । \newline
34. गृ॒ह्णी॒या॒त् तत् तद् गृ॑ह्णीयाद् गृह्णीया॒त् तत् । \newline
35. तदे॒वैव तत् तदे॒व । \newline
36. ए॒वास्या᳚ स्यै॒वै वास्य॑ । \newline
37. अ॒स्य॒ सर्वꣳ॒॒ सर्व॑ मस्यास्य॒ सर्व᳚म् । \newline
38. सर्वं॑ ॅवृङ्क्ते वृङ्क्ते॒ सर्वꣳ॒॒ सर्वं॑ ॅवृङ्क्ते । \newline
39. वृ॒ङ्क्ते॒ घृ॒तव॑न्तम् घृ॒तव॑न्तं ॅवृङ्क्ते वृङ्क्ते घृ॒तव॑न्तम् । \newline
40. घृ॒तव॑न्तम् कुला॒यिन॑म् कुला॒यिन॑म् घृ॒तव॑न्तम् घृ॒तव॑न्तम् कुला॒यिन᳚म् । \newline
41. घृ॒तव॑न्त॒मिति॑ घृ॒त - व॒न्त॒म् । \newline
42. कु॒ला॒यिनꣳ॑ रा॒यो रा॒यः कु॑ला॒यिन॑म् कुला॒यिनꣳ॑ रा॒यः । \newline
43. रा॒य स्पोष॒म् पोषꣳ॑ रा॒यो रा॒य स्पोष᳚म् । \newline
44. पोषꣳ॑ सह॒स्रिणꣳ॑ सह॒स्रिण॒म् पोष॒म् पोषꣳ॑ सह॒स्रिण᳚म् । \newline
45. स॒ह॒स्रिणं॑ ॅवे॒दो वे॒दः स॑ह॒स्रिणꣳ॑ सह॒स्रिणं॑ ॅवे॒दः । \newline
46. वे॒दो द॑दातु ददातु वे॒दो वे॒दो द॑दातु । \newline
47. द॒दा॒तु॒ वा॒जिनं॑ ॅवा॒जिन॑म् ददातु ददातु वा॒जिन᳚म् । \newline
48. वा॒जिन॒ मितीति॑ वा॒जिनं॑ ॅवा॒जिन॒ मिति॑ । \newline
49. इत्या॑हा॒हे तीत्या॑ह । \newline
50. आ॒ह॒ प्र प्राहा॑ह॒ प्र । \newline
51. प्र स॒हस्रꣳ॑ स॒हस्र॒म् प्र प्र स॒हस्र᳚म् । \newline
52. स॒हस्र॑म् प॒शून् प॒शून् थ्स॒हस्रꣳ॑ स॒हस्र॑म् प॒शून् । \newline
53. प॒शू ना᳚प्नो त्याप्नोति प॒शून् प॒शू ना᳚प्नोति । \newline
54. आ॒प्नो॒त्या ऽऽप्नो᳚ त्याप्नो॒त्या । \newline
55. आ ऽस्या॒स्या ऽस्य॑ । \newline
56. अ॒स्य॒ प्र॒जाया᳚म् प्र॒जाया॑ मस्यास्य प्र॒जाया᳚म् । \newline
57. प्र॒जायां᳚ ॅवा॒जी वा॒जी प्र॒जाया᳚म् प्र॒जायां᳚ ॅवा॒जी । \newline
58. प्र॒जाया॒मिति॑ प्र - जाया᳚म् । \newline
59. वा॒जी जा॑यते जायते वा॒जी वा॒जी जा॑यते । \newline
60. जा॒य॒ते॒ यो यो जा॑यते जायते॒ यः । \newline
61. य ए॒व मे॒वं ॅयो य ए॒वम् । \newline
62. ए॒वं ॅवेद॒ वेदै॒व मे॒वं ॅवेद॑ । \newline
63. वेदेति॒ वेद॑ । \newline

\textbf{Ghana Paata } \newline

1. प्राजा॑यता जायत॒ प्र प्राजा॑यत॒ तस्मा॒त् तस्मा॑द जायत॒ प्र प्राजा॑यत॒ तस्मा᳚त् । \newline
2. अ॒जा॒य॒त॒ तस्मा॒त् तस्मा॑ दजायता जायत॒ तस्मा॑दे॒वैव तस्मा॑ दजायता जायत॒ तस्मा॑दे॒व । \newline
3. तस्मा॑दे॒वैव तस्मा॒त् तस्मा॑दे॒व यज॑मानो॒ यज॑मान ए॒व तस्मा॒त् तस्मा॑दे॒व यज॑मानः । \newline
4. ए॒व यज॑मानो॒ यज॑मान ए॒वैव यज॑मानो मिथु॒नेन॑ मिथु॒नेन॒ यज॑मान ए॒वैव यज॑मानो मिथु॒नेन॑ । \newline
5. यज॑मानो मिथु॒नेन॑ मिथु॒नेन॒ यज॑मानो॒ यज॑मानो मिथु॒नेन॒ प्र प्र मि॑थु॒नेन॒ यज॑मानो॒ यज॑मानो मिथु॒नेन॒ प्र । \newline
6. मि॒थु॒नेन॒ प्र प्र मि॑थु॒नेन॑ मिथु॒नेन॒ प्र जा॑यते जायते॒ प्र मि॑थु॒नेन॑ मिथु॒नेन॒ प्र जा॑यते । \newline
7. प्र जा॑यते जायते॒ प्र प्र जा॑यते वे॒दो वे॒दो जा॑यते॒ प्र प्र जा॑यते वे॒दः । \newline
8. जा॒य॒ते॒ वे॒दो वे॒दो जा॑यते जायते वे॒दो᳚ ऽस्यसि वे॒दो जा॑यते जायते वे॒दो॑ ऽसि । \newline
9. वे॒दो᳚ ऽस्यसि वे॒दो वे॒दो॑ ऽसि॒ वित्ति॒र् वित्ति॑रसि वे॒दो वे॒दो॑ ऽसि॒ वित्तिः॑ । \newline
10. अ॒सि॒ वित्ति॒र् वित्ति॑ रस्यसि॒ वित्ति॑ रस्यसि॒ वित्ति॑ रस्यसि॒ वित्ति॑ रसि । \newline
11. वित्ति॑ रस्यसि॒ वित्ति॒र् वित्ति॑ रसि वि॒देय॑ वि॒देया॑सि॒ वित्ति॒र् वित्ति॑ रसि वि॒देय॑ । \newline
12. अ॒सि॒ वि॒देय॑ वि॒देया᳚स्यसि वि॒देये तीति॑ वि॒देया᳚स्यसि वि॒देये ति॑ । \newline
13. वि॒देये तीति॑ वि॒देय॑ वि॒देये त्या॑हा॒हे ति॑ वि॒देय॑ वि॒देये त्या॑ह । \newline
14. इत्या॑हा॒हे तीत्या॑ह वे॒देन॑ वे॒देना॒हे तीत्या॑ह वे॒देन॑ । \newline
15. आ॒ह॒ वे॒देन॑ वे॒देना॑हाह वे॒देन॒ वै वै वे॒देना॑हाह वे॒देन॒ वै । \newline
16. वे॒देन॒ वै वै वे॒देन॑ वे॒देन॒ वै दे॒वा दे॒वा वै वे॒देन॑ वे॒देन॒ वै दे॒वाः । \newline
17. वै दे॒वा दे॒वा वै वै दे॒वा असु॑राणा॒ मसु॑राणाम् दे॒वा वै वै दे॒वा असु॑राणाम् । \newline
18. दे॒वा असु॑राणा॒ मसु॑राणाम् दे॒वा दे॒वा असु॑राणां ॅवि॒त्तं ॅवि॒त्त मसु॑राणाम् दे॒वा दे॒वा असु॑राणां ॅवि॒त्तम् । \newline
19. असु॑राणां ॅवि॒त्तं ॅवि॒त्त मसु॑राणा॒ मसु॑राणां ॅवि॒त्तं ॅवेद्यं॒ ॅवेद्यं॑ ॅवि॒त्त मसु॑राणा॒ मसु॑राणां ॅवि॒त्तं ॅवेद्य᳚म् । \newline
20. वि॒त्तं ॅवेद्यं॒ ॅवेद्यं॑ ॅवि॒त्तं ॅवि॒त्तं ॅवेद्य॑ मविन्दन्ता विन्दन्त॒ वेद्यं॑ ॅवि॒त्तं ॅवि॒त्तं ॅवेद्य॑ मविन्दन्त । \newline
21. वेद्य॑ मविन्दन्ता विन्दन्त॒ वेद्यं॒ ॅवेद्य॑ मविन्दन्त॒ तत् तद॑विन्दन्त॒ वेद्यं॒ ॅवेद्य॑ मविन्दन्त॒ तत् । \newline
22. अ॒वि॒न्द॒न्त॒ तत् तद॑विन्दन्ता विन्दन्त॒ तद् वे॒दस्य॑ वे॒दस्य॒ तद॑विन्दन्ता विन्दन्त॒ तद् वे॒दस्य॑ । \newline
23. तद् वे॒दस्य॑ वे॒दस्य॒ तत् तद् वे॒दस्य॑ वेद॒त्वं ॅवे॑द॒त्वं ॅवे॒दस्य॒ तत् तद् वे॒दस्य॑ वेद॒त्वम् । \newline
24. वे॒दस्य॑ वेद॒त्वं ॅवे॑द॒त्वं ॅवे॒दस्य॑ वे॒दस्य॑ वेद॒त्वं ॅयद्य॒द् यद्य॑द् वेद॒त्वं ॅवे॒दस्य॑ वे॒दस्य॑ वेद॒त्वं ॅयद्य॑त् । \newline
25. वे॒द॒त्वं ॅयद्य॒द् यद्य॑द् वेद॒त्वं ॅवे॑द॒त्वं ॅयद्य॒द् भ्रातृ॑व्यस्य॒ भ्रातृ॑व्यस्य॒ यद्य॑द् वेद॒त्वं ॅवे॑द॒त्वं ॅयद्य॒द् भ्रातृ॑व्यस्य । \newline
26. वे॒द॒त्वमिति॑ वेद - त्वम् । \newline
27. यद्य॒द् भ्रातृ॑व्यस्य॒ भ्रातृ॑व्यस्य॒ यद्य॒द् यद्य॒द् भ्रातृ॑व्यस्या भि॒द्ध्याये॑ दभि॒द्ध्याये॒द् भ्रातृ॑व्यस्य॒ यद्य॒द् यद्य॒द् भ्रातृ॑व्यस्या भि॒द्ध्याये᳚त् । \newline
28. यद्य॒दिति॒ यत् - य॒त् । \newline
29. भ्रातृ॑व्यस्या भि॒द्ध्याये॑ दभि॒द्ध्याये॒द् भ्रातृ॑व्यस्य॒ भ्रातृ॑व्यस्या भि॒द्ध्याये॒त् तस्य॒ तस्या॑ भि॒द्ध्याये॒द् भ्रातृ॑व्यस्य॒ भ्रातृ॑व्यस्या भि॒द्ध्याये॒त् तस्य॑ । \newline
30. अ॒भि॒द्ध्याये॒त् तस्य॒ तस्या॑ भि॒द्ध्याये॑ दभि॒द्ध्याये॒त् तस्य॒ नाम॒ नाम॒ तस्या॑ भि॒द्ध्याये॑ दभि॒द्ध्याये॒त् तस्य॒ नाम॑ । \newline
31. अ॒भि॒द्ध्याये॒दित्य॑भि - ध्याये᳚त् । \newline
32. तस्य॒ नाम॒ नाम॒ तस्य॒ तस्य॒ नाम॑ गृह्णीयाद् गृह्णीया॒न् नाम॒ तस्य॒ तस्य॒ नाम॑ गृह्णीयात् । \newline
33. नाम॑ गृह्णीयाद् गृह्णीया॒न् नाम॒ नाम॑ गृह्णीया॒त् तत् तद् गृ॑ह्णीया॒न् नाम॒ नाम॑ गृह्णीया॒त् तत् । \newline
34. गृ॒ह्णी॒या॒त् तत् तद् गृ॑ह्णीयाद् गृह्णीया॒त् तदे॒वैव तद् गृ॑ह्णीयाद् गृह्णीया॒त् तदे॒व । \newline
35. तदे॒वैव तत् तदे॒वास्या᳚स्यै॒व तत् तदे॒वास्य॑ । \newline
36. ए॒वास्या᳚ स्यै॒वैवास्य॒ सर्व॒(ग्म्॒) सर्व॑ मस्यै॒ वैवास्य॒ सर्व᳚म् । \newline
37. अ॒स्य॒ सर्व॒(ग्म्॒) सर्व॑ मस्यास्य॒ सर्वं॑ ॅवृङ्क्ते वृङ्क्ते॒ सर्व॑ मस्यास्य॒ सर्वं॑ ॅवृङ्क्ते । \newline
38. सर्वं॑ ॅवृङ्क्ते वृङ्क्ते॒ सर्व॒(ग्म्॒) सर्वं॑ ॅवृङ्क्ते घृ॒तव॑न्तम् घृ॒तव॑न्तं ॅवृङ्क्ते॒ सर्व॒(ग्म्॒) सर्वं॑ ॅवृङ्क्ते घृ॒तव॑न्तम् । \newline
39. वृ॒ङ्क्ते॒ घृ॒तव॑न्तम् घृ॒तव॑न्तं ॅवृङ्क्ते वृङ्क्ते घृ॒तव॑न्तम् कुला॒यिन॑म् कुला॒यिन॑म् घृ॒तव॑न्तं ॅवृङ्क्ते वृङ्क्ते घृ॒तव॑न्तम् कुला॒यिन᳚म् । \newline
40. घृ॒तव॑न्तम् कुला॒यिन॑म् कुला॒यिन॑म् घृ॒तव॑न्तम् घृ॒तव॑न्तम् कुला॒यिन(ग्म्॑) रा॒यो रा॒यः कु॑ला॒यिन॑म् घृ॒तव॑न्तम् घृ॒तव॑न्तम् कुला॒यिन(ग्म्॑) रा॒यः । \newline
41. घृ॒तव॑न्त॒मिति॑ घृ॒त - व॒न्त॒म् । \newline
42. कु॒ला॒यिन(ग्म्॑) रा॒यो रा॒यः कु॑ला॒यिन॑म् कुला॒यिन(ग्म्॑) रा॒य स्पोष॒म् पोष(ग्म्॑) रा॒यः कु॑ला॒यिन॑म् कुला॒यिन(ग्म्॑) रा॒य स्पोष᳚म् । \newline
43. रा॒य स्पोष॒म् पोष(ग्म्॑) रा॒यो रा॒य स्पोष(ग्म्॑) सह॒स्रिण(ग्म्॑) सह॒स्रिण॒म् पोष(ग्म्॑) रा॒यो रा॒य स्पोष(ग्म्॑) सह॒स्रिण᳚म् । \newline
44. पोष(ग्म्॑) सह॒स्रिण(ग्म्॑) सह॒स्रिण॒म् पोष॒म् पोष(ग्म्॑) सह॒स्रिणं॑ ॅवे॒दो वे॒दः स॑ह॒स्रिण॒म् पोष॒म् पोष(ग्म्॑) सह॒स्रिणं॑ ॅवे॒दः । \newline
45. स॒ह॒स्रिणं॑ ॅवे॒दो वे॒दः स॑ह॒स्रिण(ग्म्॑) सह॒स्रिणं॑ ॅवे॒दो द॑दातु ददातु वे॒दः स॑ह॒स्रिण(ग्म्॑) सह॒स्रिणं॑ ॅवे॒दो द॑दातु । \newline
46. वे॒दो द॑दातु ददातु वे॒दो वे॒दो द॑दातु वा॒जिनं॑ ॅवा॒जिन॑म् ददातु वे॒दो वे॒दो द॑दातु वा॒जिन᳚म् । \newline
47. द॒दा॒तु॒ वा॒जिनं॑ ॅवा॒जिन॑म् ददातु ददातु वा॒जिन॒ मितीति॑ वा॒जिन॑म् ददातु ददातु वा॒जिन॒ मिति॑ । \newline
48. वा॒जिन॒ मितीति॑ वा॒जिनं॑ ॅवा॒जिन॒ मित्या॑हा॒हे ति॑ वा॒जिनं॑ ॅवा॒जिन॒ मित्या॑ह । \newline
49. इत्या॑हा॒हे तीत्या॑ह॒ प्र प्राहे तीत्या॑ह॒ प्र । \newline
50. आ॒ह॒ प्र प्राहा॑ह॒ प्र स॒हस्र(ग्म्॑) स॒हस्र॒म् प्राहा॑ह॒ प्र स॒हस्र᳚म् । \newline
51. प्र स॒हस्र(ग्म्॑) स॒हस्र॒म् प्र प्र स॒हस्र॑म् प॒शून् प॒शून् थ्स॒हस्र॒म् प्र प्र स॒हस्र॑म् प॒शून् । \newline
52. स॒हस्र॑म् प॒शून् प॒शून् थ्स॒हस्र(ग्म्॑) स॒हस्र॑म् प॒शू ना᳚प्नो त्याप्नोति प॒शून् थ्स॒हस्र(ग्म्॑) स॒हस्र॑म् प॒शू ना᳚प्नोति । \newline
53. प॒शू ना᳚प्नो त्याप्नोति प॒शून् प॒शू ना᳚प्नो॒त्या ऽऽप्नो॑ति प॒शून् प॒शू ना᳚प्नो॒त्या । \newline
54. आ॒प्नो॒त्या ऽऽप्नो᳚त्या प्नो॒त्या ऽस्या॒स्या ऽऽप्नो᳚त्या प्नो॒त्या ऽस्य॑ । \newline
55. आ ऽस्या॒स्या ऽस्य॑ प्र॒जाया᳚म् प्र॒जाया॑ म॒स्या ऽस्य॑ प्र॒जाया᳚म् । \newline
56. अ॒स्य॒ प्र॒जाया᳚म् प्र॒जाया॑ मस्यास्य प्र॒जायां᳚ ॅवा॒जी वा॒जी प्र॒जाया॑ मस्यास्य प्र॒जायां᳚ ॅवा॒जी । \newline
57. प्र॒जायां᳚ ॅवा॒जी वा॒जी प्र॒जाया᳚म् प्र॒जायां᳚ ॅवा॒जी जा॑यते जायते वा॒जी प्र॒जाया᳚म् प्र॒जायां᳚ ॅवा॒जी जा॑यते । \newline
58. प्र॒जाया॒मिति॑ प्र - जाया᳚म् । \newline
59. वा॒जी जा॑यते जायते वा॒जी वा॒जी जा॑यते॒ यो यो जा॑यते वा॒जी वा॒जी जा॑यते॒ यः । \newline
60. जा॒य॒ते॒ यो यो जा॑यते जायते॒ य ए॒व मे॒वं ॅयो जा॑यते जायते॒ य ए॒वम् । \newline
61. य ए॒व मे॒वं ॅयो य ए॒वं ॅवेद॒ वेदै॒वं ॅयो य ए॒वं ॅवेद॑ । \newline
62. ए॒वं ॅवेद॒ वेदै॒व मे॒वं ॅवेद॑ । \newline
63. वेदेति॒ वेद॑ । \newline
\pagebreak
\markright{ TS 1.7.5.1  \hfill https://www.vedavms.in \hfill}
\addcontentsline{toc}{section}{ TS 1.7.5.1 }
\section*{ TS 1.7.5.1 }

\textbf{TS 1.7.5.1 } \newline
\textbf{Samhita Paata} \newline

ध्रु॒वां ॅवै रिच्य॑मानां ॅय॒ज्ञोऽनु॑ रिच्यते य॒ज्ञ्ं ॅयज॑मानो॒ यज॑मानं प्र॒जा ध्रु॒वामा॒प्याय॑मानां ॅय॒ज्ञोऽन्वा प्या॑यते य॒ज्ञ्ं ॅयज॑मानो॒ यज॑मानं प्र॒जा आ प्या॑यतां ध्रु॒वा घृ॒तेनेत्या॑ह ध्रु॒वामे॒वा ऽऽ प्या॑ययति॒ तामा॒प्याय॑मानां ॅय॒ज्ञोऽन्वा प्या॑यते य॒ज्ञ्ं ॅयज॑मानो॒ यज॑मानं प्र॒जाः प्र॒जाप॑तेर् वि॒भान्नाम॑ लो॒कस्तस्मिꣳ॑स्त्वा दधामि स॒ह यज॑माने॒नेत्या॑ - [ ] \newline

\textbf{Pada Paata} \newline

ध्रु॒वाम् । वै । रिच्य॑मानाम् । य॒ज्ञ्ः । अन्विति॑ । रि॒च्य॒ते॒ । य॒ज्ञ्म् । यज॑मानः । यज॑मानम् । प्र॒जा इति॑ प्र - जाः । ध्रु॒वाम् । आ॒प्याय॑माना॒मित्या᳚ - प्याय॑मानाम् । य॒ज्ञ्ः । अनु॑ । एति॑ । प्या॒य॒ते॒ । य॒ज्ञ्म् । यज॑मानः । यज॑मानम् । प्र॒जा इति॑ प्र - जाः । एति॑ । प्या॒य॒ता॒म् । ध्रु॒वा । घृ॒तेन॑ । इति॑ । आ॒ह॒ । ध्रु॒वाम् । ए॒व । एति॑ । प्या॒य॒य॒ति॒ । ताम् । आ॒प्याय॑माना॒मित्या᳚ - प्याय॑मानाम् । य॒ज्ञ्ः । अनु॑ । एति॑ । प्या॒य॒ते॒ । य॒ज्ञ्म् । यज॑मानः । यज॑मानम् । प्र॒जा इति॑ प्र - जाः । प्र॒जाप॑ते॒रिति॑ प्र॒जा - प॒तेः॒ । वि॒भानिति॑ वि - भान् । नाम॑ । लो॒कः । तस्मिन्न्॑ । त्वा॒ । द॒धा॒मि॒ । स॒ह । यज॑मानेन । इति॑ ।  \newline


\textbf{Krama Paata} \newline

ध्रु॒वां ॅवै । वै रिच्य॑मानाम् । रिच्य॑मानां ॅय॒ज्ञ्ः । य॒ज्ञोऽनु॑ । अनु॑ रिच्यते । रि॒च्य॒ते॒ य॒ज्ञ्म् । य॒ज्ञ्ं ॅयज॑मानः । यज॑मानो॒ यज॑मानम् । यज॑मानम् प्र॒जाः । प्र॒जा ध्रु॒वाम् । प्र॒जा इति॑ प्र - जाः । ध्रु॒वामा॒प्याय॑मानाम् । आ॒प्याय॑मानां ॅय॒ज्ञ्ः । आ॒प्याय॑माना॒मित्या᳚ - प्याय॑मानाम् । य॒ज्ञोऽनु॑ । अन्वा । आ प्या॑यते । प्या॒य॒ते॒ य॒ज्ञ्म् । य॒ज्ञ्ं ॅयज॑मानः । यज॑मानो॒ यज॑मानम् । यज॑मानम् प्र॒जाः । प्र॒जा आ । प्र॒जा इति॑ प्र - जाः । आ प्या॑यताम् । प्या॒य॒ता॒म् ध्रु॒वा । ध्रु॒वा घृ॒तेन॑ । घृ॒तेनेति॑ । इत्या॑ह । आ॒ह॒ ध्रु॒वाम् । ध्रु॒वामे॒व । ए॒वा । आ प्या॑ययति । प्या॒य॒य॒ति॒ ताम् । तामा॒प्याय॑मानाम् । आ॒प्याय॑मानां ॅय॒ज्ञ्ः । आ॒प्याय॑माना॒मित्या᳚ - प्याय॑मानाम् । य॒ज्ञोऽनु॑ । अन्वा । आ प्या॑यते । प्या॒य॒ते॒ य॒ज्ञ्म् । य॒ज्ञ्ं ॅयज॑मानः । यज॑मानो॒ यज॑मानम् । यज॑मानम् प्र॒जाः । प्र॒जाः प्र॒जाप॑तेः । प्र॒जा इति॑ प्र - जाः । प्र॒जाप॑तेर्,वि॒भान् । प्र॒जाप॑ते॒रिति॑ प्र॒जा - प॒तेः॒ । वि॒भान्नाम॑ । वि॒भानिति॑ वि - भान् । नाम॑ लो॒कः । लो॒कस्तस्मिन्न्॑ । तस्मिꣳ॑ त्वा । त्वा॒ द॒धा॒मि॒ । द॒धा॒मि॒ स॒ह । स॒ह यज॑मानेन । यज॑माने॒नेति॑ । इत्या॑ह \newline

\textbf{Jatai Paata} \newline

1. ध्रु॒वां ॅवै वै ध्रु॒वाम् ध्रु॒वां ॅवै । \newline
2. वै रिच्य॑मानाꣳ॒॒ रिच्य॑मानां॒ ॅवै वै रिच्य॑मानाम् । \newline
3. रिच्य॑मानां ॅय॒ज्ञो य॒ज्ञो रिच्य॑मानाꣳ॒॒ रिच्य॑मानां ॅय॒ज्ञ्ः । \newline
4. य॒ज्ञो ऽन्वनु॑ य॒ज्ञो य॒ज्ञो ऽनु॑ । \newline
5. अनु॑ रिच्यते रिच्य॒ते ऽन्वनु॑ रिच्यते । \newline
6. रि॒च्य॒ते॒ य॒ज्ञ्ं ॅय॒ज्ञ्ꣳ रि॑च्यते रिच्यते य॒ज्ञ्म् । \newline
7. य॒ज्ञ्ं ॅयज॑मानो॒ यज॑मानो य॒ज्ञ्ं ॅय॒ज्ञ्ं ॅयज॑मानः । \newline
8. यज॑मानो॒ यज॑मानं॒ ॅयज॑मानं॒ ॅयज॑मानो॒ यज॑मानो॒ यज॑मानम् । \newline
9. यज॑मानम् प्र॒जाः प्र॒जा यज॑मानं॒ ॅयज॑मानम् प्र॒जाः । \newline
10. प्र॒जा ध्रु॒वाम् ध्रु॒वाम् प्र॒जाः प्र॒जा ध्रु॒वाम् । \newline
11. प्र॒जा इति॑ प्र - जाः । \newline
12. ध्रु॒वा मा॒प्याय॑माना मा॒प्याय॑मानाम् ध्रु॒वाम् ध्रु॒वा मा॒प्याय॑मानाम् । \newline
13. आ॒प्याय॑मानां ॅय॒ज्ञो य॒ज्ञ् आ॒प्याय॑माना मा॒प्याय॑मानां ॅय॒ज्ञ्ः । \newline
14. आ॒प्याय॑माना॒मित्या᳚ - प्याय॑मानाम् । \newline
15. य॒ज्ञो ऽन्वनु॑ य॒ज्ञो य॒ज्ञो ऽनु॑ । \newline
16. अन्वा अन्वन्वा । \newline
17. आ प्या॑यते प्यायत॒ आ प्या॑यते । \newline
18. प्या॒य॒ते॒ य॒ज्ञ्ं ॅय॒ज्ञ्म् प्या॑यते प्यायते य॒ज्ञ्म् । \newline
19. य॒ज्ञ्ं ॅयज॑मानो॒ यज॑मानो य॒ज्ञ्ं ॅय॒ज्ञ्ं ॅयज॑मानः । \newline
20. यज॑मानो॒ यज॑मानं॒ ॅयज॑मानं॒ ॅयज॑मानो॒ यज॑मानो॒ यज॑मानम् । \newline
21. यज॑मानम् प्र॒जाः प्र॒जा यज॑मानं॒ ॅयज॑मानम् प्र॒जाः । \newline
22. प्र॒जा आ प्र॒जाः प्र॒जा आ । \newline
23. प्र॒जा इति॑ प्र - जाः । \newline
24. आ प्या॑यताम् प्यायता॒ मा प्या॑यताम् । \newline
25. प्या॒य॒ता॒म् ध्रु॒वा ध्रु॒वा प्या॑यताम् प्यायताम् ध्रु॒वा । \newline
26. ध्रु॒वा घृ॒तेन॑ घृ॒तेन॑ ध्रु॒वा ध्रु॒वा घृ॒तेन॑ । \newline
27. घृ॒तेने तीति॑ घृ॒तेन॑ घृ॒तेने ति॑ । \newline
28. इत्या॑ हा॒हे तीत्या॑ह । \newline
29. आ॒ह॒ ध्रु॒वाम् ध्रु॒वा मा॑हाह ध्रु॒वाम् । \newline
30. ध्रु॒वा मे॒वैव ध्रु॒वाम् ध्रु॒वा मे॒व । \newline
31. ए॒वैवैवा । \newline
32. आ प्या॑ययति प्यायय॒त्या प्या॑ययति । \newline
33. प्या॒य॒य॒ति॒ ताम् ताम् प्या॑ययति प्याययति॒ ताम् । \newline
34. ता मा॒प्याय॑माना मा॒प्याय॑माना॒म् ताम् ता मा॒प्याय॑मानाम् । \newline
35. आ॒प्याय॑मानां ॅय॒ज्ञो य॒ज्ञ् आ॒प्याय॑माना मा॒प्याय॑मानां ॅय॒ज्ञ्ः । \newline
36. आ॒प्याय॑माना॒मित्या᳚ - प्याय॑मानाम् । \newline
37. य॒ज्ञो ऽन्वनु॑ य॒ज्ञो य॒ज्ञो ऽनु॑ । \newline
38. अन्वा अन्वन्वा । \newline
39. आ प्या॑यते प्यायत॒ आ प्या॑यते । \newline
40. प्या॒य॒ते॒ य॒ज्ञ्ं ॅय॒ज्ञ्म् प्या॑यते प्यायते य॒ज्ञ्म् । \newline
41. य॒ज्ञ्ं ॅयज॑मानो॒ यज॑मानो य॒ज्ञ्ं ॅय॒ज्ञ्ं ॅयज॑मानः । \newline
42. यज॑मानो॒ यज॑मानं॒ ॅयज॑मानं॒ ॅयज॑मानो॒ यज॑मानो॒ यज॑मानम् । \newline
43. यज॑मानम् प्र॒जाः प्र॒जा यज॑मानं॒ ॅयज॑मानम् प्र॒जाः । \newline
44. प्र॒जाः प्र॒जाप॑तेः प्र॒जाप॑तेः प्र॒जाः प्र॒जाः प्र॒जाप॑तेः । \newline
45. प्र॒जा इति॑ प्र - जाः । \newline
46. प्र॒जाप॑तेर् वि॒भान्. वि॒भान् प्र॒जाप॑तेः प्र॒जाप॑तेर् वि॒भान् । \newline
47. प्र॒जाप॑ते॒रिति॑ प्र॒जा - प॒तेः॒ । \newline
48. वि॒भान् नाम॒ नाम॑ वि॒भान्. वि॒भान् नाम॑ । \newline
49. वि॒भानिति॑ वि - भान् । \newline
50. नाम॑ लो॒को लो॒को नाम॒ नाम॑ लो॒कः । \newline
51. लो॒क स्तस्मिꣳ॒॒ स्तस्मि॑न् ॅलो॒को लो॒क स्तस्मिन्न्॑ । \newline
52. तस्मिꣳ॑ स्त्वा त्वा॒ तस्मिꣳ॒॒ स्तस्मिꣳ॑ स्त्वा । \newline
53. त्वा॒ द॒धा॒मि॒ द॒धा॒मि॒ त्वा॒ त्वा॒ द॒धा॒मि॒ । \newline
54. द॒धा॒मि॒ स॒ह स॒ह द॑धामि दधामि स॒ह । \newline
55. स॒ह यज॑मानेन॒ यज॑मानेन स॒ह स॒ह यज॑मानेन । \newline
56. यज॑माने॒ने तीति॒ यज॑मानेन॒ यज॑माने॒ने ति॑ । \newline
57. इत्या॑हा॒हे तीत्या॑ह । \newline

\textbf{Ghana Paata } \newline

1. ध्रु॒वां ॅवै वै ध्रु॒वाम् ध्रु॒वां ॅवै रिच्य॑माना॒(ग्म्॒) रिच्य॑मानां॒ ॅवै ध्रु॒वाम् ध्रु॒वां ॅवै रिच्य॑मानाम् । \newline
2. वै रिच्य॑माना॒(ग्म्॒) रिच्य॑मानां॒ ॅवै वै रिच्य॑मानां ॅय॒ज्ञो य॒ज्ञो रिच्य॑मानां॒ ॅवै वै रिच्य॑मानां ॅय॒ज्ञ्ः । \newline
3. रिच्य॑मानां ॅय॒ज्ञो य॒ज्ञो रिच्य॑माना॒(ग्म्॒) रिच्य॑मानां ॅय॒ज्ञो ऽन्वनु॑ य॒ज्ञो रिच्य॑माना॒(ग्म्॒) रिच्य॑मानां ॅय॒ज्ञो ऽनु॑ । \newline
4. य॒ज्ञो ऽन्वनु॑ य॒ज्ञो य॒ज्ञो ऽनु॑ रिच्यते रिच्य॒ते ऽनु॑ य॒ज्ञो य॒ज्ञो ऽनु॑ रिच्यते । \newline
5. अनु॑ रिच्यते रिच्य॒ते ऽन्वनु॑ रिच्यते य॒ज्ञ्ं ॅय॒ज्ञ्ꣳ रि॑च्य॒ते ऽन्वनु॑ रिच्यते य॒ज्ञ्म् । \newline
6. रि॒च्य॒ते॒ य॒ज्ञ्ं ॅय॒ज्ञ्ꣳ रि॑च्यते रिच्यते य॒ज्ञ्ं ॅयज॑मानो॒ यज॑मानो य॒ज्ञ्ꣳ रि॑च्यते रिच्यते य॒ज्ञ्ं ॅयज॑मानः । \newline
7. य॒ज्ञ्ं ॅयज॑मानो॒ यज॑मानो य॒ज्ञ्ं ॅय॒ज्ञ्ं ॅयज॑मानो॒ यज॑मानं॒ ॅयज॑मानं॒ ॅयज॑मानो य॒ज्ञ्ं ॅय॒ज्ञ्ं ॅयज॑मानो॒ यज॑मानम् । \newline
8. यज॑मानो॒ यज॑मानं॒ ॅयज॑मानं॒ ॅयज॑मानो॒ यज॑मानो॒ यज॑मानम् प्र॒जाः प्र॒जा यज॑मानं॒ ॅयज॑मानो॒ यज॑मानो॒ यज॑मानम् प्र॒जाः । \newline
9. यज॑मानम् प्र॒जाः प्र॒जा यज॑मानं॒ ॅयज॑मानम् प्र॒जा ध्रु॒वाम् ध्रु॒वाम् प्र॒जा यज॑मानं॒ ॅयज॑मानम् प्र॒जा ध्रु॒वाम् । \newline
10. प्र॒जा ध्रु॒वाम् ध्रु॒वाम् प्र॒जाः प्र॒जा ध्रु॒वा मा॒प्याय॑माना मा॒प्याय॑मानाम् ध्रु॒वाम् प्र॒जाः प्र॒जा ध्रु॒वा मा॒प्याय॑मानाम् । \newline
11. प्र॒जा इति॑ प्र - जाः । \newline
12. ध्रु॒वा मा॒प्याय॑माना मा॒प्याय॑मानाम् ध्रु॒वाम् ध्रु॒वा मा॒प्याय॑मानां ॅय॒ज्ञो य॒ज्ञ् आ॒प्याय॑मानाम् ध्रु॒वाम् ध्रु॒वा मा॒प्याय॑मानां ॅय॒ज्ञ्ः । \newline
13. आ॒प्याय॑मानां ॅय॒ज्ञो य॒ज्ञ् आ॒प्याय॑माना मा॒प्याय॑मानां ॅय॒ज्ञो ऽन्वनु॑ य॒ज्ञ् आ॒प्याय॑माना मा॒प्याय॑मानां ॅय॒ज्ञो ऽनु॑ । \newline
14. आ॒प्याय॑माना॒मित्या᳚ - प्याय॑मानाम् । \newline
15. य॒ज्ञो ऽन्वनु॑ य॒ज्ञो य॒ज्ञो ऽन्वा ऽनु॑ य॒ज्ञो य॒ज्ञो ऽन्वा । \newline
16. अन्वा अन्वन्वा प्या॑यते प्यायत॒ आ अन्वन्वा प्या॑यते । \newline
17. आ प्या॑यते प्यायत॒ आ प्या॑यते य॒ज्ञ्ं ॅय॒ज्ञ्म् प्या॑यत॒ आ प्या॑यते य॒ज्ञ्म् । \newline
18. प्या॒य॒ते॒ य॒ज्ञ्ं ॅय॒ज्ञ्म् प्या॑यते प्यायते य॒ज्ञ्ं ॅयज॑मानो॒ यज॑मानो य॒ज्ञ्म् प्या॑यते प्यायते य॒ज्ञ्ं ॅयज॑मानः । \newline
19. य॒ज्ञ्ं ॅयज॑मानो॒ यज॑मानो य॒ज्ञ्ं ॅय॒ज्ञ्ं ॅयज॑मानो॒ यज॑मानं॒ ॅयज॑मानं॒ ॅयज॑मानो य॒ज्ञ्ं ॅय॒ज्ञ्ं ॅयज॑मानो॒ यज॑मानम् । \newline
20. यज॑मानो॒ यज॑मानं॒ ॅयज॑मानं॒ ॅयज॑मानो॒ यज॑मानो॒ यज॑मानम् प्र॒जाः प्र॒जा यज॑मानं॒ ॅयज॑मानो॒ यज॑मानो॒ यज॑मानम् प्र॒जाः । \newline
21. यज॑मानम् प्र॒जाः प्र॒जा यज॑मानं॒ ॅयज॑मानम् प्र॒जा आ प्र॒जा यज॑मानं॒ ॅयज॑मानम् प्र॒जा आ । \newline
22. प्र॒जा आ प्र॒जाः प्र॒जा आ प्या॑यताम् प्यायता॒ मा प्र॒जाः प्र॒जा आ प्या॑यताम् । \newline
23. प्र॒जा इति॑ प्र - जाः । \newline
24. आ प्या॑यताम् प्यायता॒ मा प्या॑यताम् ध्रु॒वा ध्रु॒वा प्या॑यता॒ मा प्या॑यताम् ध्रु॒वा । \newline
25. प्या॒य॒ता॒म् ध्रु॒वा ध्रु॒वा प्या॑यताम् प्यायताम् ध्रु॒वा घृ॒तेन॑ घृ॒तेन॑ ध्रु॒वा प्या॑यताम् प्यायताम् ध्रु॒वा घृ॒तेन॑ । \newline
26. ध्रु॒वा घृ॒तेन॑ घृ॒तेन॑ ध्रु॒वा ध्रु॒वा घृ॒तेने तीति॑ घृ॒तेन॑ ध्रु॒वा ध्रु॒वा घृ॒तेने ति॑ । \newline
27. घृ॒तेने तीति॑ घृ॒तेन॑ घृ॒तेने त्या॑हा॒हे ति॑ घृ॒तेन॑ घृ॒तेने त्या॑ह । \newline
28. इत्या॑हा॒हे तीत्या॑ह ध्रु॒वाम् ध्रु॒वा मा॒हे तीत्या॑ह ध्रु॒वाम् । \newline
29. आ॒ह॒ ध्रु॒वाम् ध्रु॒वा मा॑हाह ध्रु॒वा मे॒वैव ध्रु॒वा मा॑हाह ध्रु॒वा मे॒व । \newline
30. ध्रु॒वा मे॒वैव ध्रु॒वाम् ध्रु॒वा मे॒वैव ध्रु॒वाम् ध्रु॒वा मे॒वा । \newline
31. ए॒वैवैवा प्या॑ययति प्यायय॒ त्यैवैवा प्या॑ययति । \newline
32. आ प्या॑ययति प्यायय॒त्या प्या॑ययति॒ ताम् ताम् प्या॑यय॒त्या प्या॑ययति॒ ताम् । \newline
33. प्या॒य॒य॒ति॒ ताम् ताम् प्या॑ययति प्याययति॒ ता मा॒प्याय॑माना मा॒प्याय॑माना॒म् ताम् प्या॑ययति प्याययति॒ ता मा॒प्याय॑मानाम् । \newline
34. ता मा॒प्याय॑माना मा॒प्याय॑माना॒म् ताम् ता मा॒प्याय॑मानां ॅय॒ज्ञो य॒ज्ञ् आ॒प्याय॑माना॒म् ताम् ता मा॒प्याय॑मानां ॅय॒ज्ञ्ः । \newline
35. आ॒प्याय॑मानां ॅय॒ज्ञो य॒ज्ञ् आ॒प्याय॑माना मा॒प्याय॑मानां ॅय॒ज्ञो ऽन्वनु॑ य॒ज्ञ् आ॒प्याय॑माना मा॒प्याय॑मानां ॅय॒ज्ञो ऽनु॑ । \newline
36. आ॒प्याय॑माना॒मित्या᳚ - प्याय॑मानाम् । \newline
37. य॒ज्ञो ऽन्वनु॑ य॒ज्ञो य॒ज्ञो ऽन्वा ऽनु॑ य॒ज्ञो य॒ज्ञो ऽन्वा । \newline
38. अन्वा अन्वन्वा प्या॑यते प्यायत॒ आ अन्वन्वा प्या॑यते । \newline
39. आ प्या॑यते प्यायत॒ आ प्या॑यते य॒ज्ञ्ं ॅय॒ज्ञ्म् प्या॑यत॒ आ प्या॑यते य॒ज्ञ्म् । \newline
40. प्या॒य॒ते॒ य॒ज्ञ्ं ॅय॒ज्ञ्म् प्या॑यते प्यायते य॒ज्ञ्ं ॅयज॑मानो॒ यज॑मानो य॒ज्ञ्म् प्या॑यते प्यायते य॒ज्ञ्ं ॅयज॑मानः । \newline
41. य॒ज्ञ्ं ॅयज॑मानो॒ यज॑मानो य॒ज्ञ्ं ॅय॒ज्ञ्ं ॅयज॑मानो॒ यज॑मानं॒ ॅयज॑मानं॒ ॅयज॑मानो य॒ज्ञ्ं ॅय॒ज्ञ्ं ॅयज॑मानो॒ यज॑मानम् । \newline
42. यज॑मानो॒ यज॑मानं॒ ॅयज॑मानं॒ ॅयज॑मानो॒ यज॑मानो॒ यज॑मानम् प्र॒जाः प्र॒जा यज॑मानं॒ ॅयज॑मानो॒ यज॑मानो॒ यज॑मानम् प्र॒जाः । \newline
43. यज॑मानम् प्र॒जाः प्र॒जा यज॑मानं॒ ॅयज॑मानम् प्र॒जाः प्र॒जाप॑तेः प्र॒जाप॑तेः प्र॒जा यज॑मानं॒ ॅयज॑मानम् प्र॒जाः प्र॒जाप॑तेः । \newline
44. प्र॒जाः प्र॒जाप॑तेः प्र॒जाप॑तेः प्र॒जाः प्र॒जाः प्र॒जाप॑तेर् वि॒भान्. वि॒भान् प्र॒जाप॑तेः प्र॒जाः प्र॒जाः प्र॒जाप॑तेर् वि॒भान् । \newline
45. प्र॒जा इति॑ प्र - जाः । \newline
46. प्र॒जाप॑तेर् वि॒भान्. वि॒भान् प्र॒जाप॑तेः प्र॒जाप॑तेर् वि॒भान् नाम॒ नाम॑ वि॒भान् प्र॒जाप॑तेः प्र॒जाप॑तेर् वि॒भान् नाम॑ । \newline
47. प्र॒जाप॑ते॒रिति॑ प्र॒जा - प॒तेः॒ । \newline
48. वि॒भान् नाम॒ नाम॑ वि॒भान्. वि॒भान् नाम॑ लो॒को लो॒को नाम॑ वि॒भान्. वि॒भान् नाम॑ लो॒कः । \newline
49. वि॒भानिति॑ वि - भान् । \newline
50. नाम॑ लो॒को लो॒को नाम॒ नाम॑ लो॒क स्तस्मि॒(ग्ग्॒) स्तस्मि॑न् ॅलो॒को नाम॒ नाम॑ लो॒क स्तस्मिन्न्॑ । \newline
51. लो॒क स्तस्मि॒(ग्ग्॒) स्तस्मि॑न् ॅलो॒को लो॒क स्तस्मि(ग्ग्॑) स्त्वा त्वा॒ तस्मि॑न् ॅलो॒को लो॒क स्तस्मि(ग्ग्॑) स्त्वा । \newline
52. तस्मि(ग्ग्॑) स्त्वा त्वा॒ तस्मि॒(ग्ग्॒) स्तस्मि(ग्ग्॑) स्त्वा दधामि दधामि त्वा॒ तस्मि॒(ग्ग्॒) स्तस्मि(ग्ग्॑) स्त्वा दधामि । \newline
53. त्वा॒ द॒धा॒मि॒ द॒धा॒मि॒ त्वा॒ त्वा॒ द॒धा॒मि॒ स॒ह स॒ह द॑धामि त्वा त्वा दधामि स॒ह । \newline
54. द॒धा॒मि॒ स॒ह स॒ह द॑धामि दधामि स॒ह यज॑मानेन॒ यज॑मानेन स॒ह द॑धामि दधामि स॒ह यज॑मानेन । \newline
55. स॒ह यज॑मानेन॒ यज॑मानेन स॒ह स॒ह यज॑माने॒ने तीति॒ यज॑मानेन स॒ह स॒ह यज॑माने॒ने ति॑ । \newline
56. यज॑माने॒ने तीति॒ यज॑मानेन॒ यज॑माने॒ने त्या॑हा॒हे ति॒ यज॑मानेन॒ यज॑माने॒ने त्या॑ह । \newline
57. इत्या॑हा॒हे तीत्या॑हा॒य म॒य मा॒हे तीत्या॑हा॒यम् । \newline
\pagebreak
\markright{ TS 1.7.5.2  \hfill https://www.vedavms.in \hfill}
\addcontentsline{toc}{section}{ TS 1.7.5.2 }
\section*{ TS 1.7.5.2 }

\textbf{TS 1.7.5.2 } \newline
\textbf{Samhita Paata} \newline

हा॒यं ॅवै प्र॒जाप॑तेर् वि॒भान्नाम॑ लो॒कस्तस्मि॑-न्ने॒वैनं॑ दधाति स॒ह यज॑मानेन॒ रिच्य॑त इव॒ वा ए॒तद्-यद्-यज॑ते॒ यद्-य॑जमानभा॒गं प्रा॒श्ञात्या॒त्मान॑मे॒व प्री॑णात्ये॒तावा॒न्॒. वै य॒ज्ञो यावान्॑. यजमानभा॒गो य॒ज्ञो यज॑मानो॒ यद्-य॑जमानभा॒गं प्रा॒श्ञाति॑ य॒ज्ञ् ए॒व य॒ज्ञ्ं प्रति॑ ष्ठापयत्ये॒तद्वै सू॒यव॑सꣳ॒॒ सोद॑कं॒ यद्ब॒र्॒.हिश्चाऽऽप॑श्चै॒तद् - [ ] \newline

\textbf{Pada Paata} \newline

आ॒ह॒ । अ॒यम् । वै । प्र॒जाप॑ते॒रिति॑ प्र॒जा - प॒तेः॒ । वि॒भानिति॑ वि - भान् । नाम॑ । लो॒कः । तस्मिन्न्॑ । ए॒व । ए॒न॒म् । द॒धा॒ति॒ । स॒ह । यज॑मानेन । रिच्य॑ते । इ॒व॒ । वै । ए॒तत् । यत् । यज॑ते । यत् । य॒ज॒मा॒न॒भा॒गमिति॑ यजमान - भा॒गम् । प्रा॒श्नातीति॑ प्र - अ॒श्नाति॑ । आ॒त्मान᳚म् । ए॒व । प्री॒णा॒ति॒ । ए॒तावान्॑ । वै । य॒ज्ञ्ः । यावान्॑ । य॒ज॒मा॒न॒भा॒ग इति॑ यजमान - भा॒गः । य॒ज्ञ्ः । यज॑मानः । यत् । य॒ज॒मा॒न॒भा॒गमिति॑ यजमान - भा॒गम् । प्रा॒श्नातीति॑ प्र - अ॒श्नाति॑ । य॒ज्ञे । ए॒व । य॒ज्ञ्म् । प्रतीति॑ । स्था॒प॒य॒ति॒ । ए॒तत् । वै । सू॒यव॑स॒मिति॑ सु - यव॑सम् । सोद॑क॒मिति॒ स - उ॒द॒क॒म् । यत् । ब॒र्॒.हिः । च॒ । आपः॑ । च॒ । ए॒तत् ।  \newline


\textbf{Krama Paata} \newline

आ॒हा॒यम् । अ॒यं ॅवै । वै प्र॒जाप॑तेः । प्र॒जाप॑तेर्,वि॒भान् । प्र॒जाप॑ते॒रिति॑ प्र॒जा - प॒तेः॒ । वि॒भान्नाम॑ । वि॒भानिति॑ वि - भान् । नाम॑ लो॒कः । लो॒कस्तस्मिन्न्॑ । तस्मि॑न्ने॒व । ए॒वैन᳚म् । ए॒न॒म् द॒धा॒ति॒ । द॒धा॒ति॒ स॒ह । स॒ह यज॑मानेन । यज॑मानेन॒ रिच्य॑ते । रिच्य॑त इव । इ॒व॒ वै । वा ए॒तत् । ए॒तद् यत् । यद् यज॑ते । यज॑ते॒ यत् । 
यद् य॑जमानभा॒गम् । य॒ज॒मा॒न॒भा॒गम् प्रा॒श्ञाति॑ । य॒ज॒मा॒न॒भा॒गमिति॑ यजमान - भा॒गम् । प्रा॒श्ञात्या॒त्मान᳚म् । प्रा॒श्ञातीति॑ प्र - अ॒श्ञाति॑ । आ॒त्मान॑मे॒व । ए॒व प्री॑णाति । प्री॒णा॒त्ये॒तावान्॑ । ए॒तावा॒न्. वै । वै य॒ज्ञ्ः । य॒ज्ञो यावान्॑ । यावा॑न्. यजमानभा॒गः । य॒ज॒मा॒न॒भा॒गो य॒ज्ञ्ः । य॒ज॒मा॒न॒भा॒ग इति॑ यजमान - भा॒गः । य॒ज्ञो यज॑मानः । यज॑मानो॒ यत् । यद् य॑जमानभा॒गम् । य॒ज॒मा॒न॒भा॒गम् प्रा॒श्ञाति॑ । य॒ज॒मा॒न॒भा॒गमिति॑ यजमान - भा॒गम् । प्रा॒श्ञाति॑ य॒ज्ञे । प्रा॒श्ञातीति॑ प्र - अ॒श्ञाति॑ । य॒ज्ञ् ए॒व । ए॒व य॒ज्ञ्म् । य॒ज्ञ्म् प्रति॑ । प्रति॑ ष्ठापयति । स्था॒प॒य॒त्ये॒तत् । ए॒तद् वै । वै सू॒यव॑सम् । सू॒यव॑सꣳ॒॒ सोद॑कम् । सू॒यव॑स॒मिति॑ सु - यव॑सम् । सोद॑कं॒ ॅयत् । सोद॑क॒मिति॒ स - उ॒द॒क॒म् । यद् ब॒र्.॒॒॒हिः । ब॒र्.॒॒॒हिश्च॑ । चापः॑ । आप॑श्च । चै॒तत् । 
ए॒तद् यज॑मानस्य \newline

\textbf{Jatai Paata} \newline

1. आ॒हा॒य म॒य मा॑हाहा॒यम् । \newline
2. अ॒यं ॅवै वा अ॒य म॒यं ॅवै । \newline
3. वै प्र॒जाप॑तेः प्र॒जाप॑ते॒र् वै वै प्र॒जाप॑तेः । \newline
4. प्र॒जाप॑तेर् वि॒भान्. वि॒भान् प्र॒जाप॑तेः प्र॒जाप॑तेर् वि॒भान् । \newline
5. प्र॒जाप॑ते॒रिति॑ प्र॒जा - प॒तेः॒ । \newline
6. वि॒भान् नाम॒ नाम॑ वि॒भान्. वि॒भान् नाम॑ । \newline
7. वि॒भानिति॑ वि - भान् । \newline
8. नाम॑ लो॒को लो॒को नाम॒ नाम॑ लो॒कः । \newline
9. लो॒क स्तस्मिꣳ॒॒ स्तस्मि॑न् ॅलो॒को लो॒क स्तस्मिन्न्॑ । \newline
10. तस्मि॑न् ने॒वैव तस्मिꣳ॒॒ स्तस्मि॑न् ने॒व । \newline
11. ए॒वैन॑ मेन मे॒वैवैन᳚म् । \newline
12. ए॒न॒म् द॒धा॒ति॒ द॒धा॒त्ये॒न॒ मे॒न॒म् द॒धा॒ति॒ । \newline
13. द॒धा॒ति॒ स॒ह स॒ह द॑धाति दधाति स॒ह । \newline
14. स॒ह यज॑मानेन॒ यज॑मानेन स॒ह स॒ह यज॑मानेन । \newline
15. यज॑मानेन॒ रिच्य॑ते॒ रिच्य॑ते॒ यज॑मानेन॒ यज॑मानेन॒ रिच्य॑ते । \newline
16. रिच्य॑त इवे व॒ रिच्य॑ते॒ रिच्य॑त इव । \newline
17. इ॒व॒ वै वा इ॑वे व॒ वै । \newline
18. वा ए॒त दे॒तद् वै वा ए॒तत् । \newline
19. ए॒तद् यद् यदे॒त दे॒तद् यत् । \newline
20. यद् यज॑ते॒ यज॑ते॒ यद् यद् यज॑ते । \newline
21. यज॑ते॒ यद् यद् यज॑ते॒ यज॑ते॒ यत् । \newline
22. यद् य॑जमानभा॒गं ॅय॑जमानभा॒गं ॅयद् यद् य॑जमानभा॒गम् । \newline
23. य॒ज॒मा॒न॒भा॒गम् प्रा॒श्ञाति॑ प्रा॒श्ञाति॑ यजमानभा॒गं ॅय॑जमानभा॒गम् प्रा॒श्ञाति॑ । \newline
24. य॒ज॒मा॒न॒भा॒गमिति॑ यजमान - भा॒गम् । \newline
25. प्रा॒श्ञा त्या॒त्मान॑ मा॒त्मान॑म् प्रा॒श्ञाति॑ प्रा॒श्ञा त्या॒त्मान᳚म् । \newline
26. प्रा॒श्ञातीति॑ प्र - अ॒श्ञाति॑ । \newline
27. आ॒त्मान॑ मे॒वै वात्मान॑ मा॒त्मान॑ मे॒व । \newline
28. ए॒व प्री॑णाति प्रीणा त्ये॒वैव प्री॑णाति । \newline
29. प्री॒णा॒ त्ये॒तावा॑ ने॒तावा᳚न् प्रीणाति प्रीणा त्ये॒तावान्॑ । \newline
30. ए॒तावा॒न्॒. वै वा ए॒तावा॑ ने॒तावा॒न्॒. वै । \newline
31. वै य॒ज्ञो य॒ज्ञो वै वै य॒ज्ञ्ः । \newline
32. य॒ज्ञो यावा॒न्॒. यावान्॑. य॒ज्ञो य॒ज्ञो यावान्॑ । \newline
33. यावान्॑. यजमानभा॒गो य॑जमानभा॒गो यावा॒न्॒. यावान्॑. यजमानभा॒गः । \newline
34. य॒ज॒मा॒न॒भा॒गो य॒ज्ञो य॒ज्ञो य॑जमानभा॒गो य॑जमानभा॒गो य॒ज्ञ्ः । \newline
35. य॒ज॒मा॒न॒भा॒ग इति॑ यजमान - भा॒गः । \newline
36. य॒ज्ञो यज॑मानो॒ यज॑मानो य॒ज्ञो य॒ज्ञो यज॑मानः । \newline
37. यज॑मानो॒ यद् यद् यज॑मानो॒ यज॑मानो॒ यत् । \newline
38. यद् य॑जमानभा॒गं ॅय॑जमानभा॒गं ॅयद् यद् य॑जमानभा॒गम् । \newline
39. य॒ज॒मा॒न॒भा॒गम् प्रा॒श्ञाति॑ प्रा॒श्ञाति॑ यजमानभा॒गं ॅय॑जमानभा॒गम् प्रा॒श्ञाति॑ । \newline
40. य॒ज॒मा॒न॒भा॒गमिति॑ यजमान - भा॒गम् । \newline
41. प्रा॒श्ञाति॑ य॒ज्ञे य॒ज्ञे प्रा॒श्ञाति॑ प्रा॒श्ञाति॑ य॒ज्ञे । \newline
42. प्रा॒श्ञातीति॑ प्र - अ॒श्ञाति॑ । \newline
43. य॒ज्ञ् ए॒वैव य॒ज्ञे य॒ज्ञ् ए॒व । \newline
44. ए॒व य॒ज्ञ्ं ॅय॒ज्ञ् मे॒वैव य॒ज्ञ्म् । \newline
45. य॒ज्ञ्म् प्रति॒ प्रति॑ य॒ज्ञ्ं ॅय॒ज्ञ्म् प्रति॑ । \newline
46. प्रति॑ ष्ठापयति स्थापयति॒ प्रति॒ प्रति॑ ष्ठापयति । \newline
47. स्था॒प॒य॒ त्ये॒त दे॒तथ् स्था॑पयति स्थापय त्ये॒तत् । \newline
48. ए॒तद् वै वा ए॒त दे॒तद् वै । \newline
49. वै सू॒यव॑सꣳ सू॒यव॑सं॒ ॅवै वै सू॒यव॑सम् । \newline
50. सू॒यव॑सꣳ॒॒ सोद॑कꣳ॒॒ सोद॑कꣳ सू॒यव॑सꣳ सू॒यव॑सꣳ॒॒ सोद॑कम् । \newline
51. सू॒यव॑स॒मिति॑ सु - यव॑सम् । \newline
52. सोद॑कं॒ ॅयद् यथ् सोद॑कꣳ॒॒ सोद॑कं॒ ॅयत् । \newline
53. सोद॑क॒मिति॒ स - उ॒द॒क॒म् । \newline
54. यद् ब॒र्॒.हिर् ब॒र्॒.हिर् यद् यद् ब॒र्॒.हिः । \newline
55. ब॒र्॒.हिश्च॑ च ब॒र्॒.हिर् ब॒र्॒.हिश्च॑ । \newline
56. चाप॒ आप॑श्च॒ चापः॑ । \newline
57. आप॑श्च॒ चाप॒ आप॑श्च । \newline
58. चै॒त दे॒तच् च॑ चै॒तत् । \newline
59. ए॒तद् यज॑मानस्य॒ यज॑मान स्यै॒त दे॒तद् यज॑मानस्य । \newline

\textbf{Ghana Paata } \newline

1. आ॒हा॒य म॒य मा॑हाहा॒यं ॅवै वा अ॒य मा॑हाहा॒यं ॅवै । \newline
2. अ॒यं ॅवै वा अ॒य म॒यं ॅवै प्र॒जाप॑तेः प्र॒जाप॑ते॒र् वा अ॒य म॒यं ॅवै प्र॒जाप॑तेः । \newline
3. वै प्र॒जाप॑तेः प्र॒जाप॑ते॒र् वै वै प्र॒जाप॑तेर् वि॒भान्. वि॒भान् प्र॒जाप॑ते॒र् वै वै प्र॒जाप॑तेर् वि॒भान् । \newline
4. प्र॒जाप॑तेर् वि॒भान्. वि॒भान् प्र॒जाप॑तेः प्र॒जाप॑तेर् वि॒भान् नाम॒ नाम॑ वि॒भान् प्र॒जाप॑तेः प्र॒जाप॑तेर् वि॒भान् नाम॑ । \newline
5. प्र॒जाप॑ते॒रिति॑ प्र॒जा - प॒तेः॒ । \newline
6. वि॒भान् नाम॒ नाम॑ वि॒भान्. वि॒भान् नाम॑ लो॒को लो॒को नाम॑ वि॒भान्. वि॒भान् नाम॑ लो॒कः । \newline
7. वि॒भानिति॑ वि - भान् । \newline
8. नाम॑ लो॒को लो॒को नाम॒ नाम॑ लो॒क स्तस्मि॒(ग्ग्॒) स्तस्मि॑न् ॅलो॒को नाम॒ नाम॑ लो॒क स्तस्मिन्न्॑ । \newline
9. लो॒क स्तस्मि॒(ग्ग्॒) स्तस्मि॑न् ॅलो॒को लो॒क स्तस्मि॑न् ने॒वैव तस्मि॑न् ॅलो॒को लो॒क स्तस्मि॑न् ने॒व । \newline
10. तस्मि॑न् ने॒वैव तस्मि॒(ग्ग्॒) स्तस्मि॑न् ने॒वैन॑ मेन मे॒व तस्मि॒(ग्ग्॒) स्तस्मि॑न् ने॒वैन᳚म् । \newline
11. ए॒वैन॑ मेन मे॒वैवैन॑म् दधाति दधात्येन मे॒वै वैन॑म् दधाति । \newline
12. ए॒न॒म् द॒धा॒ति॒ द॒धा॒त्ये॒न॒ मे॒न॒म् द॒धा॒ति॒ स॒ह स॒ह द॑धात्येन मेनम् दधाति स॒ह । \newline
13. द॒धा॒ति॒ स॒ह स॒ह द॑धाति दधाति स॒ह यज॑मानेन॒ यज॑मानेन स॒ह द॑धाति दधाति स॒ह यज॑मानेन । \newline
14. स॒ह यज॑मानेन॒ यज॑मानेन स॒ह स॒ह यज॑मानेन॒ रिच्य॑ते॒ रिच्य॑ते॒ यज॑मानेन स॒ह स॒ह यज॑मानेन॒ रिच्य॑ते । \newline
15. यज॑मानेन॒ रिच्य॑ते॒ रिच्य॑ते॒ यज॑मानेन॒ यज॑मानेन॒ रिच्य॑त इवे व॒ रिच्य॑ते॒ यज॑मानेन॒ यज॑मानेन॒ रिच्य॑त इव । \newline
16. रिच्य॑त इवे व॒ रिच्य॑ते॒ रिच्य॑त इव॒ वै वा इ॑व॒ रिच्य॑ते॒ रिच्य॑त इव॒ वै । \newline
17. इ॒व॒ वै वा इ॑वे व॒ वा ए॒त दे॒तद् वा इ॑वे व॒ वा ए॒तत् । \newline
18. वा ए॒तदे॒तद् वै वा ए॒तद् यद् यदे॒तद् वै वा ए॒तद् यत् । \newline
19. ए॒तद् यद् यदे॒त दे॒तद् यद् यज॑ते॒ यज॑ते॒ यदे॒त दे॒तद् यद् यज॑ते । \newline
20. यद् यज॑ते॒ यज॑ते॒ यद् यद् यज॑ते॒ यद् यद् यज॑ते॒ यद् यद् यज॑ते॒ यत् । \newline
21. यज॑ते॒ यद् यद् यज॑ते॒ यज॑ते॒ यद् य॑जमानभा॒गं ॅय॑जमानभा॒गं ॅयद् यज॑ते॒ यज॑ते॒ यद् य॑जमानभा॒गम् । \newline
22. यद् य॑जमानभा॒गं ॅय॑जमानभा॒गं ॅयद् यद् य॑जमानभा॒गम् प्रा॒श्ञाति॑ प्रा॒श्ञाति॑ यजमानभा॒गं ॅयद् यद् य॑जमानभा॒गम् प्रा॒श्ञाति॑ । \newline
23. य॒ज॒मा॒न॒भा॒गम् प्रा॒श्ञाति॑ प्रा॒श्ञाति॑ यजमानभा॒गं ॅय॑जमानभा॒गम् प्रा॒श्ञा त्या॒त्मान॑ मा॒त्मान॑म् प्रा॒श्ञाति॑ यजमानभा॒गं ॅय॑जमानभा॒गम् प्रा॒श्ञा त्या॒त्मान᳚म् । \newline
24. य॒ज॒मा॒न॒भा॒गमिति॑ यजमान - भा॒गम् । \newline
25. प्रा॒श्ञा त्या॒त्मान॑ मा॒त्मान॑म् प्रा॒श्ञाति॑ प्रा॒श्ञा त्या॒त्मान॑ मे॒वै वात्मान॑म् प्रा॒श्ञाति॑ प्रा॒श्ञा त्या॒त्मान॑ मे॒व । \newline
26. प्रा॒श्ञातीति॑ प्र - अ॒श्ञाति॑ । \newline
27. आ॒त्मान॑ मे॒वै वात्मान॑ मा॒त्मान॑ मे॒व प्री॑णाति प्रीणा त्ये॒वात्मान॑ मा॒त्मान॑ मे॒व प्री॑णाति । \newline
28. ए॒व प्री॑णाति प्रीणा त्ये॒वैव प्री॑णा त्ये॒तावा॑ ने॒तावा᳚न् प्रीणा त्ये॒वैव प्री॑णा त्ये॒तावान्॑ । \newline
29. प्री॒णा॒ त्ये॒तावा॑ ने॒तावा᳚न् प्रीणाति प्रीणा त्ये॒तावा॒न्॒. वै वा ए॒तावा᳚न् प्रीणाति प्रीणा त्ये॒तावा॒न्॒. वै । \newline
30. ए॒तावा॒न्॒. वै वा ए॒तावा॑ ने॒तावा॒न्॒. वै य॒ज्ञो य॒ज्ञो वा ए॒तावा॑ ने॒तावा॒न्॒. वै य॒ज्ञ्ः । \newline
31. वै य॒ज्ञो य॒ज्ञो वै वै य॒ज्ञो यावा॒न्॒. यावान्॑. य॒ज्ञो वै वै य॒ज्ञो यावान्॑ । \newline
32. य॒ज्ञो यावा॒न्॒. यावान्॑. य॒ज्ञो य॒ज्ञो यावान्॑. यजमानभा॒गो य॑जमानभा॒गो यावान्॑. य॒ज्ञो य॒ज्ञो यावान्॑. यजमानभा॒गः । \newline
33. यावान्॑. यजमानभा॒गो य॑जमानभा॒गो यावा॒न्॒. यावान्॑. यजमानभा॒गो य॒ज्ञो य॒ज्ञो य॑जमानभा॒गो यावा॒न्॒. 
यावान्॑. यजमानभा॒गो य॒ज्ञ्ः । \newline
34. य॒ज॒मा॒न॒भा॒गो य॒ज्ञो य॒ज्ञो य॑जमानभा॒गो य॑जमानभा॒गो य॒ज्ञो यज॑मानो॒ यज॑मानो य॒ज्ञो य॑जमानभा॒गो य॑जमानभा॒गो य॒ज्ञो यज॑मानः । \newline
35. य॒ज॒मा॒न॒भा॒ग इति॑ यजमान - भा॒गः । \newline
36. य॒ज्ञो यज॑मानो॒ यज॑मानो य॒ज्ञो य॒ज्ञो यज॑मानो॒ यद् यद् यज॑मानो य॒ज्ञो य॒ज्ञो यज॑मानो॒ यत् । \newline
37. यज॑मानो॒ यद् यद् यज॑मानो॒ यज॑मानो॒ यद् य॑जमानभा॒गं ॅय॑जमानभा॒गं ॅयद् यज॑मानो॒ यज॑मानो॒ यद् य॑जमानभा॒गम् । \newline
38. यद् य॑जमानभा॒गं ॅय॑जमानभा॒गं ॅयद् यद् य॑जमानभा॒गम् प्रा॒श्ञाति॑ प्रा॒श्ञाति॑ यजमानभा॒गं ॅयद् यद् य॑जमानभा॒गम् प्रा॒श्ञाति॑ । \newline
39. य॒ज॒मा॒न॒भा॒गम् प्रा॒श्ञाति॑ प्रा॒श्ञाति॑ यजमानभा॒गं ॅय॑जमानभा॒गम् प्रा॒श्ञाति॑ य॒ज्ञे य॒ज्ञे प्रा॒श्ञाति॑ यजमानभा॒गं ॅय॑जमानभा॒गम् प्रा॒श्ञाति॑ य॒ज्ञे । \newline
40. य॒ज॒मा॒न॒भा॒गमिति॑ यजमान - भा॒गम् । \newline
41. प्रा॒श्ञाति॑ य॒ज्ञे य॒ज्ञे प्रा॒श्ञाति॑ प्रा॒श्ञाति॑ य॒ज्ञ् ए॒वैव य॒ज्ञे प्रा॒श्ञाति॑ प्रा॒श्ञाति॑ य॒ज्ञ् ए॒व । \newline
42. प्रा॒श्ञातीति॑ प्र - अ॒श्ञाति॑ । \newline
43. य॒ज्ञ् ए॒वैव य॒ज्ञे य॒ज्ञ् ए॒व य॒ज्ञ्ं ॅय॒ज्ञ् मे॒व य॒ज्ञे य॒ज्ञ् ए॒व य॒ज्ञ्म् । \newline
44. ए॒व य॒ज्ञ्ं ॅय॒ज्ञ् मे॒वैव य॒ज्ञ्म् प्रति॒ प्रति॑ य॒ज्ञ् मे॒वैव य॒ज्ञ्म् प्रति॑ । \newline
45. य॒ज्ञ्म् प्रति॒ प्रति॑ य॒ज्ञ्ं ॅय॒ज्ञ्म् प्रति॑ ष्ठापयति स्थापयति॒ प्रति॑ य॒ज्ञ्ं ॅय॒ज्ञ्म् प्रति॑ ष्ठापयति । \newline
46. प्रति॑ ष्ठापयति स्थापयति॒ प्रति॒ प्रति॑ ष्ठापयत्ये॒तदे॒तथ् स्था॑पयति॒ प्रति॒ प्रति॑ ष्ठापयत्ये॒तत् । \newline
47. स्था॒प॒य॒त्ये॒तदे॒तथ् स्था॑पयति स्थापयत्ये॒तद् वै वा ए॒तथ् स्था॑पयति स्थापयत्ये॒तद् वै । \newline
48. ए॒तद् वै वा ए॒तदे॒तद् वै सू॒यव॑सꣳ सू॒यव॑सं॒ ॅवा ए॒तदे॒तद् वै सू॒यव॑सम् । \newline
49. वै सू॒यव॑सꣳ सू॒यव॑सं॒ ॅवै वै सू॒यव॑स॒(ग्म्॒) सोद॑क॒(ग्म्॒) सोद॑कꣳ सू॒यव॑सं॒ ॅवै वै सू॒यव॑स॒(ग्म्॒) सोद॑कम् । \newline
50. सू॒यव॑स॒(ग्म्॒) सोद॑क॒(ग्म्॒) सोद॑कꣳ सू॒यव॑सꣳ सू॒यव॑स॒(ग्म्॒) सोद॑कं॒ ॅयद् यथ् सोद॑कꣳ सू॒यव॑सꣳ सू॒यव॑स॒(ग्म्॒) सोद॑कं॒ ॅयत् । \newline
51. सू॒यव॑स॒मिति॑ सु - यव॑सम् । \newline
52. सोद॑कं॒ ॅयद् यथ् सोद॑क॒(ग्म्॒) सोद॑कं॒ ॅयद् ब॒र्॒.हिर् ब॒र्॒.हिर् यथ् सोद॑क॒(ग्म्॒) सोद॑कं॒ ॅयद् ब॒र्॒.हिः । \newline
53. सोद॑क॒मिति॒ स - उ॒द॒क॒म् । \newline
54. यद् ब॒र्॒.हिर् ब॒र्॒.हिर् यद् यद् ब॒र्॒.हिश्च॑ च ब॒र्॒.हिर् यद् यद् ब॒र्॒.हिश्च॑ । \newline
55. ब॒र्॒.हिश्च॑ च ब॒र्॒.हिर् ब॒र्॒.हि श्चाप॒ आप॑श्च ब॒र्॒.हिर् ब॒र्॒.हि श्चापः॑ । \newline
56. चाप॒ आप॑श्च॒ चाप॑श्च॒ चाप॑श्च॒ चाप॑श्च । \newline
57. आप॑श्च॒ चाप॒ आप॑ श्चै॒त दे॒तच् चाप॒ आप॑ श्चै॒तत् । \newline
58. चै॒त दे॒तच् च॑ चै॒तद् यज॑मानस्य॒ यज॑मान स्यै॒तच् च॑ चै॒तद् यज॑मानस्य । \newline
59. ए॒तद् यज॑मानस्य॒ यज॑मान स्यै॒तदे॒तद् यज॑मान स्या॒यत॑न मा॒यत॑नं॒ ॅयज॑मान स्यै॒त दे॒तद् यज॑मान स्या॒यत॑नम् । \newline
\pagebreak
\markright{ TS 1.7.5.3  \hfill https://www.vedavms.in \hfill}
\addcontentsline{toc}{section}{ TS 1.7.5.3 }
\section*{ TS 1.7.5.3 }

\textbf{TS 1.7.5.3 } \newline
\textbf{Samhita Paata} \newline

यज॑मानस्या॒ऽऽयत॑नं॒ ॅयद्वेदि॒र्यत् पू᳚र्णपा॒त्र-म॑न्तर्वे॒दि नि॒नय॑ति॒ स्व ए॒वाऽऽय॑तने सू॒यव॑सꣳ॒॒ सोद॑कं कुरुते॒ सद॑सि॒ सन्मे॑ भूया॒ इत्या॒हाऽऽपो॒ वै य॒ज्ञ् आपो॒ऽमृतं॑ ॅय॒ज्ञ्मे॒वामृत॑-मा॒त्मन् ध॑त्ते॒ सर्वा॑णि॒ वै भू॒तानि॑ व्र॒त-मु॑प॒यन्त॒ -मनूप॑ यन्ति॒ प्राच्यां᳚ दि॒शि दे॒वा ऋ॒त्विजो॑ मार्जयन्ता॒-मित्या॑है॒ष वै द॑र्.शपूर्णमा॒सयो॑-रवभृ॒थो -[ ] \newline

\textbf{Pada Paata} \newline

यज॑मानस्य । आ॒यत॑न॒मित्या᳚ - यत॑नम् । यत् । वेदिः॑ । यत् । पू॒र्ण॒पा॒त्रमिति॑ पूर्ण - पा॒त्रम् । अ॒न्त॒र्वे॒दीत्य॑न्तः - वे॒दि । नि॒नय॒तीति॑ नि - नय॑ति । स्वे । ए॒व । आ॒यत॑न॒ इत्या᳚ - यत॑ने । सू॒यव॑स॒मिति॑ सु - यव॑सम् । सोद॑क॒मिति॒ स - उ॒द॒क॒म् । कु॒रु॒ते॒ । सत् । अ॒सि॒ । सत् । मे॒ । भू॒याः॒ । इति॑ । आ॒ह॒ । आपः॑ । वै । य॒ज्ञ्ः । आपः॑ । अ॒मृत᳚म् । य॒ज्ञ्म् । ए॒व । अ॒मृत᳚म् । आ॒त्मन्न् । ध॒त्ते॒ । सर्वा॑णि । वै । भू॒तानि॑ । व्र॒तम् । उ॒प॒यन्त॒मित्यु॑प - यन्त᳚म् । अनु॑ । उपेति॑ । य॒न्ति॒ । प्राच्या᳚म् । दि॒शि । दे॒वाः । ऋ॒त्विजः॑ । मा॒र्ज॒य॒न्ता॒म् । इति॑ । आ॒ह॒ । ए॒षः । वै । द॒र्॒.श॒पू॒र्ण॒मा॒सयो॒रिति॑ दर्.श - पू॒र्ण॒मा॒सयोः᳚ । अ॒व॒भृ॒थ इत्य॑व - भृ॒थः ।  \newline


\textbf{Krama Paata} \newline

यज॑मानस्या॒यत॑नम् । आ॒यत॑नं॒ ॅयत् । आ॒यत॑न॒मित्या᳚ - यत॑नम् । यद् वेदिः॑ । वेदि॒र्,यत् । यत्,पू᳚र्णपा॒त्रम् ॥ पू॒र्ण॒पा॒त्रम॑न्तर्वे॒दि । पू॒र्ण॒पा॒त्रमिति॑ पूर्ण - पा॒त्रम् । अ॒न्त॒र्वे॒दि नि॒नय॑ति । अ॒न्त॒र्वे॒दीत्य॑न्तः - वे॒दि । नि॒नय॑ति॒ स्वे । नि॒नय॒तीति॑ नि - नय॑ति । स्व ए॒व । ए॒वायत॑ने । आ॒यत॑ने सू॒यव॑सम् । आ॒यत॑न॒ इत्या᳚ - यत॑ने । सू॒यव॑सꣳ॒॒ सोद॑कम् । सू॒यव॑स॒मिति॑ सु - यव॑सम् । सोद॑कम् कुरुते । सोद॑क॒मिति॒ स - उ॒द॒क॒म् । कु॒रु॒ते॒ सत् । सद॑सि । अ॒सि॒ सत् । सन्मे᳚ । मे॒ भू॒याः॒ । भू॒या॒ इति॑ । इत्या॑ह । आ॒हापः॑ । आपो॒ वै । वै य॒ज्ञ्ः । य॒ज्ञ् आपः॑ । आपो॒ ऽमृत᳚म् । अ॒मृतं॑ ॅय॒ज्ञ्म् । य॒ज्ञ्मे॒व । ए॒वामृत᳚म् ॥ अ॒मृत॑मा॒त्मन् । आ॒त्मन्,ध॑त्ते । ध॒त्ते॒ सर्वा॑णि । सर्वा॑णि॒ वै । वै भू॒तानि॑ । भू॒तानि॑ व्र॒तम् । व्र॒तमु॑प॒यन्त᳚म् । उ॒प॒यन्त॒मनु॑ । उ॒प॒यन्त॒मित्यु॑प - यन्त᳚म् । अनूप॑ । उप॑ यन्ति । य॒न्ति॒ प्राच्या᳚म् । प्राच्या᳚म् दि॒शि । दि॒शि दे॒वाः । दे॒वा ऋ॒त्विजः॑ । ऋ॒त्विजो॑ मार्जयन्ताम् । मा॒र्ज॒य॒न्ता॒मिति॑ । इत्या॑ह । आ॒है॒षः । ए॒ष वै । वै द॑र्.शपूर्णमा॒सयोः᳚ । द॒र्.॒॒श॒पू॒र्ण॒मा॒सयो॑रवभृ॒थः । द॒र्.॒॒श॒पू॒र्ण॒मा॒सयो॒रिति॑ दर्.श - पू॒र्ण॒मा॒सयोः᳚ । अ॒व॒भृ॒थो यानि॑ । अ॒व॒भृ॒थ इत्य॑व - भृ॒थः \newline

\textbf{Jatai Paata} \newline

1. यज॑मान स्या॒यत॑न मा॒यत॑नं॒ ॅयज॑मानस्य॒ यज॑मान स्या॒यत॑नम् । \newline
2. आ॒यत॑नं॒ ॅयद् यदा॒यत॑न मा॒यत॑नं॒ ॅयत् । \newline
3. आ॒यत॑न॒मित्या᳚ - यत॑नम् । \newline
4. यद् वेदि॒र् वेदि॒र् यद् यद् वेदिः॑ । \newline
5. वेदि॒र् यद् यद् वेदि॒र् वेदि॒र् यत् । \newline
6. यत् पू᳚र्णपा॒त्रम् पू᳚र्णपा॒त्रं ॅयद् यत् पू᳚र्णपा॒त्रम् । \newline
7. पू॒र्ण॒पा॒त्र म॑न्तर्वे॒ द्य॑न्तर्वे॒दि पू᳚र्णपा॒त्रम् पू᳚र्णपा॒त्र म॑न्तर्वे॒दि । \newline
8. पू॒र्ण॒पा॒त्रमिति॑ पूर्ण - पा॒त्रम् । \newline
9. अ॒न्त॒र्वे॒दि नि॒नय॑ति नि॒नय॑ त्यन्तर्वे॒ द्य॑न्तर्वे॒दि नि॒नय॑ति । \newline
10. अ॒न्त॒र्वे॒दीत्य॑न्तः - वे॒दि । \newline
11. नि॒नय॑ति॒ स्वे स्वे नि॒नय॑ति नि॒नय॑ति॒ स्वे । \newline
12. नि॒नय॒तीति॑ नि - नय॑ति । \newline
13. स्व ए॒वैव स्वे स्व ए॒व । \newline
14. ए॒वायत॑न आ॒यत॑न ए॒वै वायत॑ने । \newline
15. आ॒यत॑ने सू॒यव॑सꣳ सू॒यव॑स मा॒यत॑न आ॒यत॑ने सू॒यव॑सम् । \newline
16. आ॒यत॑न॒ इत्या᳚ - यत॑ने । \newline
17. सू॒यव॑सꣳ॒॒ सोद॑कꣳ॒॒ सोद॑कꣳ सू॒यव॑सꣳ सू॒यव॑सꣳ॒॒ सोद॑कम् । \newline
18. सू॒यव॑स॒मिति॑ सु - यव॑सम् । \newline
19. सोद॑कम् कुरुते कुरुते॒ सोद॑कꣳ॒॒ सोद॑कम् कुरुते । \newline
20. सोद॑क॒मिति॒ स - उ॒द॒क॒म् । \newline
21. कु॒रु॒ते॒ सथ् सत् कु॑रुते कुरुते॒ सत् । \newline
22. सद॑स्यसि॒ सथ् सद॑सि । \newline
23. अ॒सि॒ सथ् सद॑स्यसि॒ सत् । \newline
24. सन् मे॑ मे॒ सथ् सन् मे᳚ । \newline
25. मे॒ भू॒या॒ भू॒या॒ मे॒ मे॒ भू॒याः॒ । \newline
26. भू॒या॒ इतीति॑ भूया भूया॒ इति॑ । \newline
27. इत्या॑ हा॒हे तीत्या॑ह । \newline
28. आ॒हाप॒ आप॑ आहा॒ हापः॑ । \newline
29. आपो॒ वै वा आप॒ आपो॒ वै । \newline
30. वै य॒ज्ञो य॒ज्ञो वै वै य॒ज्ञ्ः । \newline
31. य॒ज्ञ् आप॒ आपो॑ य॒ज्ञो य॒ज्ञ् आपः॑ । \newline
32. आपो॒ ऽमृत॑ म॒मृत॒ माप॒ आपो॒ ऽमृत᳚म् । \newline
33. अ॒मृतं॑ ॅय॒ज्ञ्ं ॅय॒ज्ञ् म॒मृत॑ म॒मृतं॑ ॅय॒ज्ञ्म् । \newline
34. य॒ज्ञ् मे॒वैव य॒ज्ञ्ं ॅय॒ज्ञ् मे॒व । \newline
35. ए॒वामृत॑ म॒मृत॑ मे॒वै वामृत᳚म् । \newline
36. अ॒मृत॑ मा॒त्मन् ना॒त्मन् न॒मृत॑ म॒मृत॑ मा॒त्मन्न् । \newline
37. आ॒त्मन् ध॑त्ते धत्त आ॒त्मन् ना॒त्मन् ध॑त्ते । \newline
38. ध॒त्ते॒ सर्वा॑णि॒ सर्वा॑णि धत्ते धत्ते॒ सर्वा॑णि । \newline
39. सर्वा॑णि॒ वै वै सर्वा॑णि॒ सर्वा॑णि॒ वै । \newline
40. वै भू॒तानि॑ भू॒तानि॒ वै वै भू॒तानि॑ । \newline
41. भू॒तानि॑ व्र॒तं ॅव्र॒तम् भू॒तानि॑ भू॒तानि॑ व्र॒तम् । \newline
42. व्र॒त मु॑प॒यन्त॑ मुप॒यन्तं॑ ॅव्र॒तं ॅव्र॒त मु॑प॒यन्त᳚म् । \newline
43. उ॒प॒यन्त॒ मन्वनू॑ प॒यन्त॑ मुप॒यन्त॒ मनु॑ । \newline
44. उ॒प॒यन्त॒मित्यु॑प - यन्त᳚म् । \newline
45. अनू पोपा न्वनूप॑ । \newline
46. उप॑ यन्ति य॒न्त्युपोप॑ यन्ति । \newline
47. य॒न्ति॒ प्राच्या॒म् प्राच्यां᳚ ॅयन्ति यन्ति॒ प्राच्या᳚म् । \newline
48. प्राच्या᳚म् दि॒शि दि॒शि प्राच्या॒म् प्राच्या᳚म् दि॒शि । \newline
49. दि॒शि दे॒वा दे॒वा दि॒शि दि॒शि दे॒वाः । \newline
50. दे॒वा ऋ॒त्विज॑ ऋ॒त्विजो॑ दे॒वा दे॒वा ऋ॒त्विजः॑ । \newline
51. ऋ॒त्विजो॑ मार्जयन्ताम् मार्जयन्ता मृ॒त्विज॑ ऋ॒त्विजो॑ मार्जयन्ताम् । \newline
52. मा॒र्ज॒य॒न्ता॒ मितीति॑ मार्जयन्ताम् मार्जयन्ता॒ मिति॑ । \newline
53. इत्या॑ हा॒हे तीत्या॑ह । \newline
54. आ॒है॒ष ए॒ष आ॑हा है॒षः । \newline
55. ए॒ष वै वा ए॒ष ए॒ष वै । \newline
56. वै द॑र्.शपूर्णमा॒सयो᳚र् दर्.शपूर्णमा॒सयो॒र् वै वै द॑र्.शपूर्णमा॒सयोः᳚ । \newline
57. द॒र्॒.श॒पू॒र्ण॒मा॒सयो॑ रवभृ॒थो॑ ऽवभृ॒थो द॑र्.शपूर्णमा॒सयो᳚र् दर्.शपूर्णमा॒सयो॑ रवभृ॒थः । \newline
58. द॒र्॒.श॒पू॒र्ण॒मा॒सयो॒रिति॑ दर्.श - पू॒र्ण॒मा॒सयोः᳚ । \newline
59. अ॒व॒भृ॒थो यानि॒ या न्य॑वभृ॒थो॑ ऽवभृ॒थो यानि॑ । \newline
60. अ॒व॒भृ॒थ इत्य॑व - भृ॒थः । \newline

\textbf{Ghana Paata } \newline

1. यज॑मान स्या॒यत॑न मा॒यत॑नं॒ ॅयज॑मानस्य॒ यज॑मान स्या॒यत॑नं॒ ॅयद् यदा॒यत॑नं॒ ॅयज॑मानस्य॒ यज॑मान स्या॒यत॑नं॒ ॅयत् । \newline
2. आ॒यत॑नं॒ ॅयद् यदा॒यत॑न मा॒यत॑नं॒ ॅयद् वेदि॒र् वेदि॒र् यदा॒यत॑न मा॒यत॑नं॒ ॅयद् वेदिः॑ । \newline
3. आ॒यत॑न॒मित्या᳚ - यत॑नम् । \newline
4. यद् वेदि॒र् वेदि॒र् यद् यद् वेदि॒र् यद् यद् वेदि॒र् यद् यद् वेदि॒र् यत् । \newline
5. वेदि॒र् यद् यद् वेदि॒र् वेदि॒र् यत् पू᳚र्णपा॒त्रम् पू᳚र्णपा॒त्रं ॅयद् वेदि॒र् वेदि॒र् यत् पू᳚र्णपा॒त्रम् । \newline
6. यत् पू᳚र्णपा॒त्रम् पू᳚र्णपा॒त्रं ॅयद् यत् पू᳚र्णपा॒त्र म॑न्तर्वे॒द्य॑न्तर्वे॒दि पू᳚र्णपा॒त्रं ॅयद् यत् पू᳚र्णपा॒त्र म॑न्तर्वे॒दि । \newline
7. पू॒र्ण॒पा॒त्र म॑न्तर्वे॒द्य॑न्तर्वे॒दि पू᳚र्णपा॒त्रम् पू᳚र्णपा॒त्र म॑न्तर्वे॒दि नि॒नय॑ति नि॒नय॑ त्यन्तर्वे॒दि पू᳚र्णपा॒त्रम् पू᳚र्णपा॒त्र म॑न्तर्वे॒दि नि॒नय॑ति । \newline
8. पू॒र्ण॒पा॒त्रमिति॑ पूर्ण - पा॒त्रम् । \newline
9. अ॒न्त॒र्वे॒दि नि॒नय॑ति नि॒नय॑ त्यन्तर्वे॒द्य॑न्तर्वे॒दि नि॒नय॑ति॒ स्वे स्वे नि॒नय॑ त्यन्तर्वे॒द्य॑न्तर्वे॒दि नि॒नय॑ति॒ स्वे । \newline
10. अ॒न्त॒र्वे॒दीत्य॑न्तः - वे॒दि । \newline
11. नि॒नय॑ति॒ स्वे स्वे नि॒नय॑ति नि॒नय॑ति॒ स्व ए॒वैव स्वे नि॒नय॑ति नि॒नय॑ति॒ स्व ए॒व । \newline
12. नि॒नय॒तीति॑ नि - नय॑ति । \newline
13. स्व ए॒वैव स्वे स्व ए॒वायत॑न आ॒यत॑न ए॒व स्वे स्व ए॒वायत॑ने । \newline
14. ए॒वायत॑न आ॒यत॑न ए॒वैवायत॑ने सू॒यव॑सꣳ सू॒यव॑स मा॒यत॑न ए॒वैवायत॑ने सू॒यव॑सम् । \newline
15. आ॒यत॑ने सू॒यव॑सꣳ सू॒यव॑स मा॒यत॑न आ॒यत॑ने सू॒यव॑स॒(ग्म्॒) सोद॑क॒(ग्म्॒) सोद॑कꣳ सू॒यव॑स मा॒यत॑न आ॒यत॑ने सू॒यव॑स॒(ग्म्॒) सोद॑कम् । \newline
16. आ॒यत॑न॒ इत्या᳚ - यत॑ने । \newline
17. सू॒यव॑स॒(ग्म्॒) सोद॑क॒(ग्म्॒) सोद॑कꣳ सू॒यव॑सꣳ सू॒यव॑स॒(ग्म्॒) सोद॑कम् कुरुते कुरुते॒ सोद॑कꣳ सू॒यव॑सꣳ सू॒यव॑स॒(ग्म्॒) सोद॑कम् कुरुते । \newline
18. सू॒यव॑स॒मिति॑ सु - यव॑सम् । \newline
19. सोद॑कम् कुरुते कुरुते॒ सोद॑क॒(ग्म्॒) सोद॑कम् कुरुते॒ सथ् सत् कु॑रुते॒ सोद॑क॒(ग्म्॒) सोद॑कम् कुरुते॒ सत् । \newline
20. सोद॑क॒मिति॒ स - उ॒द॒क॒म् । \newline
21. कु॒रु॒ते॒ सथ् सत् कु॑रुते कुरुते॒ सद॑स्यसि॒ सत् कु॑रुते कुरुते॒ सद॑सि । \newline
22. सद॑स्यसि॒ सथ् सद॑सि॒ सथ् सद॑सि॒ सथ् सद॑सि॒ सत् । \newline
23. अ॒सि॒ सथ् सद॑स्यसि॒ सन् मे॑ मे॒ सद॑स्यसि॒ सन् मे᳚ । \newline
24. सन् मे॑ मे॒ सथ् सन् मे॑ भूया भूया मे॒ सथ् सन् मे॑ भूयाः । \newline
25. मे॒ भू॒या॒ भू॒या॒ मे॒ मे॒ भू॒या॒ इतीति॑ भूया मे मे भूया॒ इति॑ । \newline
26. भू॒या॒ इतीति॑ भूया भूया॒ इत्या॑हा॒हे ति॑ भूया भूया॒ इत्या॑ह । \newline
27. इत्या॑हा॒हे तीत्या॒हाप॒ आप॑ आ॒हे तीत्या॒हापः॑ । \newline
28. आ॒हाप॒ आप॑ आहा॒हापो॒ वै वा आप॑ आहा॒हापो॒ वै । \newline
29. आपो॒ वै वा आप॒ आपो॒ वै य॒ज्ञो य॒ज्ञो वा आप॒ आपो॒ वै य॒ज्ञ्ः । \newline
30. वै य॒ज्ञो य॒ज्ञो वै वै य॒ज्ञ् आप॒ आपो॑ य॒ज्ञो वै वै य॒ज्ञ् आपः॑ । \newline
31. य॒ज्ञ् आप॒ आपो॑ य॒ज्ञो य॒ज्ञ् आपो॒ ऽमृत॑ म॒मृत॒ मापो॑ य॒ज्ञो य॒ज्ञ् आपो॒ ऽमृत᳚म् । \newline
32. आपो॒ ऽमृत॑ म॒मृत॒ माप॒ आपो॒ ऽमृतं॑ ॅय॒ज्ञ्ं ॅय॒ज्ञ् म॒मृत॒ माप॒ आपो॒ ऽमृतं॑ ॅय॒ज्ञ्म् । \newline
33. अ॒मृतं॑ ॅय॒ज्ञ्ं ॅय॒ज्ञ् म॒मृत॑ म॒मृतं॑ ॅय॒ज्ञ् मे॒वैव य॒ज्ञ् म॒मृत॑ म॒मृतं॑ ॅय॒ज्ञ् मे॒व । \newline
34. य॒ज्ञ् मे॒वैव य॒ज्ञ्ं ॅय॒ज्ञ् मे॒वामृत॑ म॒मृत॑ मे॒व य॒ज्ञ्ं ॅय॒ज्ञ् मे॒वामृत᳚म् । \newline
35. ए॒वामृत॑ म॒मृत॑ मे॒वैवामृत॑ मा॒त्मन् ना॒त्मन् न॒मृत॑ मे॒वैवामृत॑ मा॒त्मन्न् । \newline
36. अ॒मृत॑ मा॒त्मन् ना॒त्मन् न॒मृत॑ म॒मृत॑ मा॒त्मन् ध॑त्ते धत्त आ॒त्मन् न॒मृत॑ म॒मृत॑ मा॒त्मन् ध॑त्ते । \newline
37. आ॒त्मन् ध॑त्ते धत्त आ॒त्मन् ना॒त्मन् ध॑त्ते॒ सर्वा॑णि॒ सर्वा॑णि धत्त आ॒त्मन् ना॒त्मन् ध॑त्ते॒ सर्वा॑णि । \newline
38. ध॒त्ते॒ सर्वा॑णि॒ सर्वा॑णि धत्ते धत्ते॒ सर्वा॑णि॒ वै वै सर्वा॑णि धत्ते धत्ते॒ सर्वा॑णि॒ वै । \newline
39. सर्वा॑णि॒ वै वै सर्वा॑णि॒ सर्वा॑णि॒ वै भू॒तानि॑ भू॒तानि॒ वै सर्वा॑णि॒ सर्वा॑णि॒ वै भू॒तानि॑ । \newline
40. वै भू॒तानि॑ भू॒तानि॒ वै वै भू॒तानि॑ व्र॒तं ॅव्र॒तम् भू॒तानि॒ वै वै भू॒तानि॑ व्र॒तम् । \newline
41. भू॒तानि॑ व्र॒तं ॅव्र॒तम् भू॒तानि॑ भू॒तानि॑ व्र॒त मु॑प॒यन्त॑ मुप॒यन्तं॑ ॅव्र॒तम् भू॒तानि॑ भू॒तानि॑ व्र॒त मु॑प॒यन्त᳚म् । \newline
42. व्र॒त मु॑प॒यन्त॑ मुप॒यन्तं॑ ॅव्र॒तं ॅव्र॒त मु॑प॒यन्त॒ मन्व नू॑प॒यन्तं॑ ॅव्र॒तं ॅव्र॒त मु॑प॒यन्त॒ मनु॑ । \newline
43. उ॒प॒यन्त॒ मन्व नू॑प॒यन्त॑ मुप॒यन्त॒ मनू पोपा नू॑प॒यन्त॑ मुप॒यन्त॒ मनूप॑ । \newline
44. उ॒प॒यन्त॒मित्यु॑प - यन्त᳚म् । \newline
45. अनूपोपान् वनूप॑ यन्ति य॒न्त्युपा न्वनूप॑ यन्ति । \newline
46. उप॑ यन्ति य॒न्त्युपोप॑ यन्ति॒ प्राच्या॒म् प्राच्यां᳚ ॅय॒न्त्युपोप॑ यन्ति॒ प्राच्या᳚म् । \newline
47. य॒न्ति॒ प्राच्या॒म् प्राच्यां᳚ ॅयन्ति यन्ति॒ प्राच्या᳚म् दि॒शि दि॒शि प्राच्यां᳚ ॅयन्ति यन्ति॒ प्राच्या᳚म् दि॒शि । \newline
48. प्राच्या᳚म् दि॒शि दि॒शि प्राच्या॒म् प्राच्या᳚म् दि॒शि दे॒वा दे॒वा दि॒शि प्राच्या॒म् प्राच्या᳚म् दि॒शि दे॒वाः । \newline
49. दि॒शि दे॒वा दे॒वा दि॒शि दि॒शि दे॒वा ऋ॒त्विज॑ ऋ॒त्विजो॑ दे॒वा दि॒शि दि॒शि दे॒वा ऋ॒त्विजः॑ । \newline
50. दे॒वा ऋ॒त्विज॑ ऋ॒त्विजो॑ दे॒वा दे॒वा ऋ॒त्विजो॑ मार्जयन्ताम् मार्जयन्ता मृ॒त्विजो॑ दे॒वा दे॒वा ऋ॒त्विजो॑ मार्जयन्ताम् । \newline
51. ऋ॒त्विजो॑ मार्जयन्ताम् मार्जयन्ता मृ॒त्विज॑ ऋ॒त्विजो॑ मार्जयन्ता॒ मितीति॑ मार्जयन्ता मृ॒त्विज॑ ऋ॒त्विजो॑ मार्जयन्ता॒ मिति॑ । \newline
52. मा॒र्ज॒य॒न्ता॒ मितीति॑ मार्जयन्ताम् मार्जयन्ता॒ मित्या॑हा॒हे ति॑ मार्जयन्ताम् मार्जयन्ता॒ मित्या॑ह । \newline
53. इत्या॑हा॒हे तीत्या॑ है॒ष ए॒ष आ॒हे तीत्या॑ है॒षः । \newline
54. आ॒है॒ष ए॒ष आ॑हाहै॒ष वै वा ए॒ष आ॑हा है॒ष वै । \newline
55. ए॒ष वै वा ए॒ष ए॒ष वै द॑र्.शपूर्णमा॒सयो᳚र् दर्.शपूर्णमा॒सयो॒र् वा ए॒ष ए॒ष वै द॑र्.शपूर्णमा॒सयोः᳚ । \newline
56. वै द॑र्.शपूर्णमा॒सयो᳚र् दर्.शपूर्णमा॒सयो॒र् वै वै द॑र्.शपूर्णमा॒सयो॑ रवभृ॒थो॑ ऽवभृ॒थो द॑र्.शपूर्णमा॒सयो॒र् वै वै द॑र्.शपूर्णमा॒सयो॑ रवभृ॒थः । \newline
57. द॒र्॒.श॒पू॒र्ण॒मा॒सयो॑ रवभृ॒थो॑ ऽवभृ॒थो द॑र्.शपूर्णमा॒सयो᳚र् दर्.शपूर्णमा॒सयो॑ रवभृ॒थो यानि॒ या न्य॑वभृ॒थो द॑र्.शपूर्णमा॒सयो᳚र् दर्.शपूर्णमा॒सयो॑ रवभृ॒थो यानि॑ । \newline
58. द॒र्॒.श॒पू॒र्ण॒मा॒सयो॒रिति॑ दर्.श - पू॒र्ण॒मा॒सयोः᳚ । \newline
59. अ॒व॒भृ॒थो यानि॒ यान्य॑वभृ॒थो॑ ऽवभृ॒थो यान्ये॒ वैव यान्य॑वभृ॒थो॑ ऽवभृ॒थो यान्ये॒व । \newline
60. अ॒व॒भृ॒थ इत्य॑व - भृ॒थः । \newline
\pagebreak
\markright{ TS 1.7.5.4  \hfill https://www.vedavms.in \hfill}
\addcontentsline{toc}{section}{ TS 1.7.5.4 }
\section*{ TS 1.7.5.4 }

\textbf{TS 1.7.5.4 } \newline
\textbf{Samhita Paata} \newline

यान्ये॒वैनं॑ भू॒तानि॑ व्र॒तमु॑प॒यन्त॑-मनूप॒यन्ति॒ तैरे॒व स॒हाव॑भृ॒थमवै॑ति॒ विष्णु॑मुखा॒ वै दे॒वा श्छन्दो॑भिरि॒मान् ॅलो॒का-न॑नपज॒य्यम॒भ्य॑जय॒न्॒. यद्-वि॑ष्णुक्र॒मान् क्रम॑ते॒ विष्णु॑रे॒व भू॒त्वा यज॑मान॒श्छन्दो॑भिरि॒मान् ॅलो॒का-न॑नपज॒य्यम॒भि ज॑यति॒ विष्णोः॒ क्रमो᳚ऽस्यभिमाति॒हेत्या॑ह गाय॒त्री वै पृ॑थि॒वी त्रैष्टु॑भम॒न्तरि॑क्षं॒ जाग॑ती॒ द्यौरानु॑ष्टुभी॒र् दिश॒ श्छन्दो॑भिरे॒वेमान् ॅलो॒कान्. य॑थापू॒र्वम॒भि ज॑यति ॥ \newline

\textbf{Pada Paata} \newline

यानि॑ । ए॒व । ए॒न॒म् । भू॒तानि॑ । व्र॒तम् । उ॒प॒यन्त॒मित्यु॑प - यन्त᳚म् । अ॒नू॒प॒यन्तीत्य॑नु - उ॒प॒यन्ति॑ । तैः । ए॒व । स॒ह । अ॒व॒भृ॒थमित्य॑व - भृ॒थम् । अवेति॑ । ए॒ति॒ । विष्णु॑मुखा॒ इति॒ विष्णु॑ - मु॒खाः॒ । वै । दे॒वाः । छन्दो॑भि॒रिति॒ छन्दः॑ - भिः॒ । इ॒मान् । लो॒कान् । अ॒न॒प॒ज॒य्यमित्य॑नप-ज॒य्यम् । अ॒भीति॑ । अ॒ज॒य॒न्न् । यत् । वि॒ष्णु॒क्र॒मानिति॑ विष्णु - क्र॒मान् । क्रम॑ते । विष्णुः॑ । ए॒व । भू॒त्वा । यज॑मानः । छन्दो॑भि॒रिति॒ छन्दः॑ - भिः॒ । इ॒मान् । लो॒कान् । अ॒न॒प॒ज॒य्यमित्य॑नप - ज॒य्यम् । अ॒भीति॑ । ज॒य॒ति॒ । विष्णोः᳚ । क्रमः॑ । अ॒सि॒ । अ॒भि॒मा॒ति॒हेत्य॑भिमाति - हा । इति॑ । आ॒ह॒ । गा॒य॒त्री । वै । पृ॒थि॒वी । त्रैष्टु॑भम् । अ॒न्तरि॑क्षम् । जाग॑ती । द्यौः । आनु॑ष्टुभी॒रित्यानु॑ - स्थु॒भीः॒ । दिशः॑ ( ) । छन्दो॑भि॒रिति॒ छन्दः॑ - भिः॒ । ए॒व । इ॒मान् । लो॒कान् । य॒था॒पू॒र्वमिति॑ यथा - पू॒र्वम् । अ॒भीति॑ । ज॒य॒ति॒ ॥  \newline


\textbf{Krama Paata} \newline

यान्ये॒व । ए॒वैन᳚म् । ए॒न॒म् भू॒तानि॑ । भू॒तानि॑ व्र॒तम् । व्र॒तमु॑प॒यन्त᳚म् । उ॒प॒यन्त॑मनूप॒यन्ति॑ । उ॒प॒यन्त॒मित्यु॑प - यन्त᳚म् । अ॒नू॒प॒यन्ति॒ तैः । अ॒नू॒प॒यन्तीत्य॑नु - उ॒प॒यन्ति॑ । तैरे॒व । ए॒व स॒ह । स॒हाव॑भृ॒थम् । अ॒व॒भृ॒थमव॑ । अ॒व॒भृ॒थमित्य॑व - भृ॒थम् । अवै॑ति । ए॒ति॒ विष्णु॑मुखाः । विष्णु॑मुखा॒ वै । विष्णु॑मुखा॒ इति॒ विष्णु॑ - मु॒खाः॒ । वै दे॒वाः । दे॒वाः छन्दो॑भिः । छन्दो॑भिरि॒मान् । छन्दो॑भि॒रिति॒ छन्दः॑ - भिः॒ । इ॒मान् ॅलो॒कान् । लो॒कान॑नपज॒य्यम् । अ॒न॒प॒ज॒य्यम॒भि । अ॒न॒प॒ज॒य्यमित्य॑नप - ज॒य्यम् । अ॒भ्य॑जयन्न् । अ॒ज॒य॒न्. यत् । यद् वि॑ष्णुक्र॒मान् । वि॒ष्णु॒क्र॒मान्,क्रम॑ते । वि॒ष्णु॒क्र॒मानिति॑ विष्णु - क्र॒मान् । क्रम॑ते॒ विष्णुः॑ । विष्णु॑रे॒व । ए॒व भू॒त्वा । भू॒त्वा यज॑मानः । यज॑मानः॒ छन्दो॑भिः । छन्दो॑भिरि॒मान् । छन्दो॑भि॒रिति॒ छन्दः॑ - भिः॒ । इ॒मान् ॅलो॒कान् । लो॒कान॑नपज॒य्यम् । अ॒न॒प॒ज॒य्यम॒भि । अ॒न॒प॒ज॒य्यमित्य॑नप - ज॒य्यम् । अ॒भि ज॑यति । ज॒य॒ति॒ विष्णोः᳚ । विष्णोः॒ क्रमः॑ । क्रमो॑ऽसि । अ॒स्य॒भि॒मा॒ति॒हा । अ॒भि॒मा॒ति॒हेति॑ । अ॒भि॒मा॒ति॒हेत्य॑भिमाति - हा । इत्या॑ह । आ॒ह॒ गा॒य॒त्री । गा॒य॒त्री वै । वै पृ॑थि॒वी । पृ॒थि॒वी त्रैष्टु॑भम् । त्रैष्टु॑भम॒न्तरि॑क्षम् । अ॒न्तरि॑क्ष॒म् जाग॑ती । जाग॑ती॒ द्यौः । द्यौरानु॑ष्टुभीः । आनु॑ष्टुभी॒र्,दिशः॑ ( ) । आनु॑ष्टुभी॒रित्यानु॑ - स्तु॒भीः॒ । दिशः॒ छन्दो॑भिः । छन्दो॑भिरे॒व । छन्दो॑भि॒रिति॒ छन्दः॑ - भिः॒ । ए॒वेमान् । इ॒मान् ॅलो॒कान् । लो॒कान्. य॑थापू॒र्वम् । य॒था॒पू॒र्वम॒भि । य॒था॒पू॒र्वमिति॑ यथा - पू॒र्वम् । अ॒भि ज॑यति । ज॒य॒तीति॑ जयति । \newline

\textbf{Jatai Paata} \newline

1. या न्ये॒वैव यानि॒ यान्ये॒व । \newline
2. ए॒वैन॑ मेन मे॒वैवैन᳚म् । \newline
3. ए॒न॒म् भू॒तानि॑ भू॒तान्ये॑न मेनम् भू॒तानि॑ । \newline
4. भू॒तानि॑ व्र॒तं ॅव्र॒तम् भू॒तानि॑ भू॒तानि॑ व्र॒तम् । \newline
5. व्र॒त मु॑प॒यन्त॑ मुप॒यन्तं॑ ॅव्र॒तं ॅव्र॒त मु॑प॒यन्त᳚म् । \newline
6. उ॒प॒यन्त॑ मनूप॒यन् त्य॑नूप॒य न्त्यु॑प॒यन्त॑ मुप॒यन्त॑ मनूप॒यन्ति॑ । \newline
7. उ॒प॒यन्त॒मित्यु॑प - यन्त᳚म् । \newline
8. अ॒नू॒प॒यन्ति॒ तै स्तै र॑नूप॒य न्त्य॑नूप॒यन्ति॒ तैः । \newline
9. अ॒नू॒प॒यन्तीत्य॑नु - उ॒प॒यन्ति॑ । \newline
10. तै रे॒वैव तै स्तै रे॒व । \newline
11. ए॒व स॒ह स॒है वैव स॒ह । \newline
12. स॒हाव॑भृ॒थ म॑वभृ॒थꣳ स॒ह स॒हाव॑भृ॒थम् । \newline
13. अ॒व॒भृ॒थ मवा वा॑वभृ॒थ म॑वभृ॒थ मव॑ । \newline
14. अ॒व॒भृ॒थमित्य॑व - भृ॒थम् । \newline
15. अवै᳚ त्ये॒ त्यवावै॑ति । \newline
16. ए॒ति॒ विष्णु॑मुखा॒ विष्णु॑मुखा एत्येति॒ विष्णु॑मुखाः । \newline
17. विष्णु॑मुखा॒ वै वै विष्णु॑मुखा॒ विष्णु॑मुखा॒ वै । \newline
18. विष्णु॑मुखा॒ इति॒ विष्णु॑ - मु॒खाः॒ । \newline
19. वै दे॒वा दे॒वा वै वै दे॒वाः । \newline
20. दे॒वा श्छन्दो॑भि॒ श्छन्दो॑भिर् दे॒वा दे॒वा श्छन्दो॑भिः । \newline
21. छन्दो॑भि रि॒मा नि॒मान् छन्दो॑भि॒ श्छन्दो॑भि रि॒मान् । \newline
22. छन्दो॑भि॒रिति॒ छन्दः॑ - भिः॒ । \newline
23. इ॒मान् ॅलो॒कान् ॅलो॒का नि॒मा नि॒मान् ॅलो॒कान् । \newline
24. लो॒का न॑नपज॒य्य म॑नपज॒य्यम् ॅलो॒कान् ॅलो॒का न॑नपज॒य्यम् । \newline
25. अ॒न॒प॒ज॒य्य म॒भ्या᳚(1॒)भ्य॑नपज॒य्य म॑नपज॒य्य म॒भि । \newline
26. अ॒न॒प॒ज॒य्यमित्य॑नप - ज॒य्यम् । \newline
27. अ॒भ्य॑जयन् नजयन् न॒भ्या᳚(1॒)भ्य॑जयन्न् । \newline
28. अ॒ज॒य॒न्॒. यद् यद॑जयन् नजय॒न्॒. यत् । \newline
29. यद् वि॑ष्णुक्र॒मान्. वि॑ष्णुक्र॒मान्. यद् यद् वि॑ष्णुक्र॒मान् । \newline
30. वि॒ष्णु॒क्र॒मान् क्रम॑ते॒ क्रम॑ते विष्णुक्र॒मान्. वि॑ष्णुक्र॒मान् क्रम॑ते । \newline
31. वि॒ष्णु॒क्र॒मानिति॑ विष्णु - क्र॒मान् । \newline
32. क्रम॑ते॒ विष्णु॒र् विष्णुः॒ क्रम॑ते॒ क्रम॑ते॒ विष्णुः॑ । \newline
33. विष्णु॑ रे॒वैव विष्णु॒र् विष्णु॑ रे॒व । \newline
34. ए॒व भू॒त्वा भू॒त्वै वैव भू॒त्वा । \newline
35. भू॒त्वा यज॑मानो॒ यज॑मानो भू॒त्वा भू॒त्वा यज॑मानः । \newline
36. यज॑मान॒ श्छन्दो॑भि॒ श्छन्दो॑भि॒र् यज॑मानो॒ यज॑मान॒ श्छन्दो॑भिः । \newline
37. छन्दो॑भि रि॒मा नि॒मान् छन्दो॑भि॒ श्छन्दो॑भि रि॒मान् । \newline
38. छन्दो॑भि॒रिति॒ छन्दः॑ - भिः॒ । \newline
39. इ॒मान् ॅलो॒कान् ॅलो॒का नि॒मा नि॒मान् ॅलो॒कान् । \newline
40. लो॒का न॑नपज॒य्य म॑नपज॒य्यम् ॅलो॒कान् ॅलो॒का न॑नपज॒य्यम् । \newline
41. अ॒न॒प॒ज॒य्य म॒भ्या᳚(1॒)भ्य॑नपज॒य्य म॑नपज॒य्य म॒भि । \newline
42. अ॒न॒प॒ज॒य्यमित्य॑नप - ज॒य्यम् । \newline
43. अ॒भि ज॑यति जय त्य॒भ्य॑भि ज॑यति । \newline
44. ज॒य॒ति॒ विष्णो॒र् विष्णो᳚र् जयति जयति॒ विष्णोः᳚ । \newline
45. विष्णोः॒ क्रमः॒ क्रमो॒ विष्णो॒र् विष्णोः॒ क्रमः॑ । \newline
46. क्रमो᳚ ऽस्यसि॒ क्रमः॒ क्रमो॑ ऽसि । \newline
47. अ॒स्य॒ भि॒मा॒ति॒हा ऽभि॑ माति॒हा ऽस्य॑स्य भिमाति॒हा । \newline
48. अ॒भि॒मा॒ति॒हेती त्य॑भिमाति॒हा ऽभि॑माति॒हेति॑ । \newline
49. अ॒भि॒मा॒ति॒हेत्य॑भिमाति - हा । \newline
50. इत्या॑ हा॒हे तीत्या॑ह । \newline
51. आ॒ह॒ गा॒य॒त्री गा॑य॒ त्र्या॑हाह गाय॒त्री । \newline
52. गा॒य॒त्री वै वै गा॑य॒त्री गा॑य॒त्री वै । \newline
53. वै पृ॑थि॒वी पृ॑थि॒वी वै वै पृ॑थि॒वी । \newline
54. पृ॒थि॒वी त्रैष्टु॑भ॒म् त्रैष्टु॑भम् पृथि॒वी पृ॑थि॒वी त्रैष्टु॑भम् । \newline
55. त्रैष्टु॑भ म॒न्तरि॑क्ष म॒न्तरि॑क्ष॒म् त्रैष्टु॑भ॒म् त्रैष्टु॑भ म॒न्तरि॑क्षम् । \newline
56. अ॒न्तरि॑क्ष॒म् जाग॑ती॒ जाग॑त्य॒न्तरि॑क्ष म॒न्तरि॑क्ष॒म् जाग॑ती । \newline
57. जाग॑ती॒ द्यौर् द्यौर् जाग॑ती॒ जाग॑ती॒ द्यौः । \newline
58. द्यौ रानु॑ष्टुभी॒ रानु॑ष्टुभी॒र् द्यौर् द्यौ रानु॑ष्टुभीः । \newline
59. आनु॑ष्टुभी॒र् दिशो॒ दिश॒ आनु॑ष्टुभी॒ रानु॑ष्टुभी॒र् दिशः॑ । \newline
60. आनु॑ष्टुभी॒रित्यानु॑ - स्तु॒भीः॒ । \newline
61. दिश॒ श्छन्दो॑भि॒ श्छन्दो॑भि॒र् दिशो॒ दिश॒ श्छन्दो॑भिः । \newline
62. छन्दो॑भि रे॒वैव छन्दो॑भि॒ श्छन्दो॑भि रे॒व । \newline
63. छन्दो॑भि॒रिति॒ छन्दः॑ - भिः॒ । \newline
64. ए॒वे मा नि॒मा ने॒वैवे मान् । \newline
65. इ॒मान् ॅलो॒कान् ॅलो॒का नि॒मा नि॒मान् ॅलो॒कान् । \newline
66. लो॒कान्. य॑थापू॒र्वं ॅय॑थापू॒र्वम् ॅलो॒कान् ॅलो॒कान्. य॑थापू॒र्वम् । \newline
67. य॒था॒पू॒र्व म॒भ्य॑भि य॑थापू॒र्वं ॅय॑थापू॒र्व म॒भि । \newline
68. य॒था॒पू॒र्वमिति॑ यथा - पू॒र्वम् । \newline
69. अ॒भि ज॑यति जय त्य॒भ्य॑भि ज॑यति । \newline
70. ज॒य॒तीति॑ जयति । \newline

\textbf{Ghana Paata } \newline

1. यान्ये॒वैव यानि॒ यान्ये॒ वैन॑ मेन मे॒व यानि॒ यान्ये॒ वैन᳚म् । \newline
2. ए॒वैन॑ मेन मे॒वै वैन॑म् भू॒तानि॑ भू॒तान्ये॑न मे॒वै वैन॑म् भू॒तानि॑ । \newline
3. ए॒न॒म् भू॒तानि॑ भू॒तान्ये॑न मेनम् भू॒तानि॑ व्र॒तं ॅव्र॒तम् भू॒तान्ये॑न मेनम् भू॒तानि॑ व्र॒तम् । \newline
4. भू॒तानि॑ व्र॒तं ॅव्र॒तम् भू॒तानि॑ भू॒तानि॑ व्र॒त मु॑प॒यन्त॑ मुप॒यन्तं॑ ॅव्र॒तम् भू॒तानि॑ भू॒तानि॑ व्र॒त मु॑प॒यन्त᳚म् । \newline
5. व्र॒त मु॑प॒यन्त॑ मुप॒यन्तं॑ ॅव्र॒तं ॅव्र॒त मु॑प॒यन्त॑ मनूप॒यन् त्य॑नूप॒य न्त्यु॑प॒यन्तं॑ ॅव्र॒तं ॅव्र॒त मु॑प॒यन्त॑ मनूप॒यन्ति॑ । \newline
6. उ॒प॒यन्त॑ मनूप॒यन्त्य॑ नूप॒यन् त्यु॑प॒यन्त॑ मुप॒यन्त॑ मनूप॒यन्ति॒ तै स्तै र॑नूप॒यन् त्यु॑प॒यन्त॑ मुप॒यन्त॑ मनूप॒यन्ति॒ तैः । \newline
7. उ॒प॒यन्त॒मित्यु॑प - यन्त᳚म् । \newline
8. अ॒नू॒प॒यन्ति॒ तै स्तै र॑नूप॒य न्त्य॑नूप॒यन्ति॒ तै रे॒वैव तै र॑नूप॒य न्त्य॑नूप॒यन्ति॒ तै रे॒व । \newline
9. अ॒नू॒प॒यन्तीत्य॑नु - उ॒प॒यन्ति॑ । \newline
10. तै रे॒वैव तै स्तै रे॒व स॒ह स॒हैव तै स्तै रे॒व स॒ह । \newline
11. ए॒व स॒ह स॒हैवैव स॒हाव॑भृ॒थ म॑वभृ॒थꣳ स॒हैवैव स॒हाव॑भृ॒थम् । \newline
12. स॒हाव॑भृ॒थ म॑वभृ॒थꣳ स॒ह स॒हाव॑भृ॒थ मवावा॑व भृ॒थꣳ स॒ह स॒हाव॑ भृ॒थ मव॑ । \newline
13. अ॒व॒भृ॒थ मवावा॑व भृ॒थ म॑वभृ॒थ मवै᳚ त्ये॒त्यवा॑व भृ॒थ म॑व भृ॒थ मवै॑ति । \newline
14. अ॒व॒भृ॒थमित्य॑व - भृ॒थम् । \newline
15. अवै᳚ त्ये॒ त्यवा वै॑ति॒ विष्णु॑मुखा॒ विष्णु॑मुखा ए॒त्यवा वै॑ति॒ विष्णु॑मुखाः । \newline
16. ए॒ति॒ विष्णु॑मुखा॒ विष्णु॑मुखा एत्येति॒ विष्णु॑मुखा॒ वै वै विष्णु॑मुखा एत्येति॒ विष्णु॑मुखा॒ वै । \newline
17. विष्णु॑मुखा॒ वै वै विष्णु॑मुखा॒ विष्णु॑मुखा॒ वै दे॒वा दे॒वा वै विष्णु॑मुखा॒ विष्णु॑मुखा॒ वै दे॒वाः । \newline
18. विष्णु॑मुखा॒ इति॒ विष्णु॑ - मु॒खाः॒ । \newline
19. वै दे॒वा दे॒वा वै वै दे॒वा श्छन्दो॑भि॒ श्छन्दो॑भिर् दे॒वा वै वै दे॒वा श्छन्दो॑भिः । \newline
20. दे॒वा श्छन्दो॑भि॒ श्छन्दो॑भिर् दे॒वा दे॒वा श्छन्दो॑भि रि॒मा नि॒मान् छन्दो॑भिर् दे॒वा दे॒वा श्छन्दो॑भि रि॒मान् । \newline
21. छन्दो॑भि रि॒मा नि॒मान् छन्दो॑भि॒ श्छन्दो॑भि रि॒मान् ॅलो॒कान् ॅलो॒का नि॒मान् छन्दो॑भि॒ श्छन्दो॑भि रि॒मान् ॅलो॒कान् । \newline
22. छन्दो॑भि॒रिति॒ छन्दः॑ - भिः॒ । \newline
23. इ॒मान् ॅलो॒कान् ॅलो॒का नि॒मा नि॒मान् ॅलो॒का न॑नपज॒य्य म॑नपज॒य्यम् ॅलो॒का नि॒मा नि॒मान् ॅलो॒का न॑नपज॒य्यम् । \newline
24. लो॒का न॑नपज॒य्य म॑नपज॒य्यम् ॅलो॒कान् ॅलो॒का न॑नपज॒य्य म॒भ्या᳚(1॒)भ्य॑ नपज॒य्यम् ॅलो॒कान् ॅलो॒का न॑नपज॒य्य म॒भि । \newline
25. अ॒न॒प॒ज॒य्य म॒भ्या᳚(1॒)भ्य॑ नपज॒य्य म॑नपज॒य्य म॒भ्य॑जयन् नजयन् न॒भ्य॑नपज॒य्य म॑नपज॒य्य म॒भ्य॑जयन्न् । \newline
26. अ॒न॒प॒ज॒य्यमित्य॑नप - ज॒य्यम् । \newline
27. अ॒भ्य॑जयन् नजयन् न॒भ्या᳚(1॒)भ्य॑जय॒न्॒. यद् यद॑जयन् न॒भ्या᳚(1॒)भ्य॑जय॒न्॒. यत् । \newline
28. अ॒ज॒य॒न्॒. यद् यद॑जयन् नजय॒न्॒. यद् वि॑ष्णुक्र॒मान्. वि॑ष्णुक्र॒मान्. यद॑जयन् नजय॒न्॒. यद् वि॑ष्णुक्र॒मान् । \newline
29. यद् वि॑ष्णुक्र॒मान्. वि॑ष्णुक्र॒मान्. यद् यद् वि॑ष्णुक्र॒मान् क्रम॑ते॒ क्रम॑ते विष्णुक्र॒मान्. यद् यद् 
वि॑ष्णुक्र॒मान् क्रम॑ते । \newline
30. वि॒ष्णु॒क्र॒मान् क्रम॑ते॒ क्रम॑ते विष्णुक्र॒मान्. वि॑ष्णुक्र॒मान् क्रम॑ते॒ विष्णु॒र् विष्णुः॒ क्रम॑ते विष्णुक्र॒मान्. वि॑ष्णुक्र॒मान् क्रम॑ते॒ विष्णुः॑ । \newline
31. वि॒ष्णु॒क्र॒मानिति॑ विष्णु - क्र॒मान् । \newline
32. क्रम॑ते॒ विष्णु॒र् विष्णुः॒ क्रम॑ते॒ क्रम॑ते॒ विष्णु॑ रे॒वैव विष्णुः॒ क्रम॑ते॒ क्रम॑ते॒ विष्णु॑ रे॒व । \newline
33. विष्णु॑ रे॒वैव विष्णु॒र् विष्णु॑ रे॒व भू॒त्वा भू॒त्वैव विष्णु॒र् विष्णु॑ रे॒व भू॒त्वा । \newline
34. ए॒व भू॒त्वा भू॒ त्वै वैव भू॒त्वा यज॑मानो॒ यज॑मानो भू॒ त्वै वैव भू॒त्वा यज॑मानः । \newline
35. भू॒त्वा यज॑मानो॒ यज॑मानो भू॒त्वा भू॒त्वा यज॑मान॒ श्छन्दो॑भि॒ श्छन्दो॑भि॒र् यज॑मानो भू॒त्वा भू॒त्वा यज॑मान॒ श्छन्दो॑भिः । \newline
36. यज॑मान॒ श्छन्दो॑भि॒ श्छन्दो॑भि॒र् यज॑मानो॒ यज॑मान॒ श्छन्दो॑भि रि॒मा नि॒मान् छन्दो॑भि॒र् यज॑मानो॒ यज॑मान॒ श्छन्दो॑भि रि॒मान् । \newline
37. छन्दो॑भि रि॒मा नि॒मान् छन्दो॑भि॒ श्छन्दो॑भि रि॒मान् ॅलो॒कान् ॅलो॒का नि॒मान् छन्दो॑भि॒ श्छन्दो॑भि रि॒मान् ॅलो॒कान् । \newline
38. छन्दो॑भि॒रिति॒ छन्दः॑ - भिः॒ । \newline
39. इ॒मान् ॅलो॒कान् ॅलो॒का नि॒मा नि॒मान् ॅलो॒का न॑नपज॒य्य म॑नपज॒य्यम् ॅलो॒का नि॒मा नि॒मान् ॅलो॒का न॑नपज॒य्यम् । \newline
40. लो॒का न॑नपज॒य्य म॑नपज॒य्यम् ॅलो॒कान् ॅलो॒का न॑नपज॒य्य म॒भ्या᳚(1॒)भ्य॑ नपज॒य्यम् ॅलो॒कान् ॅलो॒का न॑नपज॒य्य म॒भि । \newline
41. अ॒न॒प॒ज॒य्य म॒भ्या᳚(1॒)भ्य॑ नपज॒य्य म॑नपज॒य्य म॒भि ज॑यति जयत्य॒भ्य॑ नपज॒य्य म॑नपज॒य्य म॒भि ज॑यति । \newline
42. अ॒न॒प॒ज॒य्यमित्य॑नप - ज॒य्यम् । \newline
43. अ॒भि ज॑यति जयत्य॒भ्य॑भि ज॑यति॒ विष्णो॒र् विष्णो᳚र् जयत्य॒भ्य॑भि ज॑यति॒ विष्णोः᳚ । \newline
44. ज॒य॒ति॒ विष्णो॒र् विष्णो᳚र् जयति जयति॒ विष्णोः॒ क्रमः॒ क्रमो॒ विष्णो᳚र् जयति जयति॒ विष्णोः॒ क्रमः॑ । \newline
45. विष्णोः॒ क्रमः॒ क्रमो॒ विष्णो॒र् विष्णोः॒ क्रमो᳚ ऽस्यसि॒ क्रमो॒ विष्णो॒र् विष्णोः॒ क्रमो॑ ऽसि । \newline
46. क्रमो᳚ ऽस्यसि॒ क्रमः॒ क्रमो᳚ ऽस्यभिमाति॒हा ऽभि॑माति॒हा ऽसि॒ क्रमः॒ क्रमो᳚ ऽस्यभिमाति॒हा । \newline
47. अ॒स्य॒भि॒मा॒ति॒हा ऽभि॑माति॒हा ऽस्य॑ स्यभिमाति॒ हेती त्य॑भिमाति॒हा ऽस्य॑ स्यभिमाति॒ हेति॑ । \newline
48. अ॒भि॒मा॒ति॒ हेतीत्य॑भिमाति॒हा ऽभि॑माति॒ हेत्या॑हा॒हे त्य॑भिमाति॒हा ऽभि॑माति॒ हेत्या॑ह । \newline
49. अ॒भि॒मा॒ति॒हेत्य॑भिमाति - हा । \newline
50. इत्या॑हा॒हे तीत्या॑ह गाय॒त्री गा॑य॒त्र्या॑हे तीत्या॑ह गाय॒त्री । \newline
51. आ॒ह॒ गा॒य॒त्री गा॑य॒त्र्या॑हाह गाय॒त्री वै वै गा॑य॒त्र्या॑हाह गाय॒त्री वै । \newline
52. गा॒य॒त्री वै वै गा॑य॒त्री गा॑य॒त्री वै पृ॑थि॒वी पृ॑थि॒वी वै गा॑य॒त्री गा॑य॒त्री वै पृ॑थि॒वी । \newline
53. वै पृ॑थि॒वी पृ॑थि॒वी वै वै पृ॑थि॒वी त्रैष्टु॑भ॒म् त्रैष्टु॑भम् पृथि॒वी वै वै पृ॑थि॒वी त्रैष्टु॑भम् । \newline
54. पृ॒थि॒वी त्रैष्टु॑भ॒म् त्रैष्टु॑भम् पृथि॒वी पृ॑थि॒वी त्रैष्टु॑भ म॒न्तरि॑क्ष म॒न्तरि॑क्ष॒म् त्रैष्टु॑भम् पृथि॒वी पृ॑थि॒वी त्रैष्टु॑भ म॒न्तरि॑क्षम् । \newline
55. त्रैष्टु॑भ म॒न्तरि॑क्ष म॒न्तरि॑क्ष॒म् त्रैष्टु॑भ॒म् त्रैष्टु॑भ म॒न्तरि॑क्ष॒म् जाग॑ती॒ जाग॑ त्य॒न्तरि॑क्ष॒म् त्रैष्टु॑भ॒म् त्रैष्टु॑भ म॒न्तरि॑क्ष॒म् जाग॑ती । \newline
56. अ॒न्तरि॑क्ष॒म् जाग॑ती॒ जाग॑ त्य॒न्तरि॑क्ष म॒न्तरि॑क्ष॒म् जाग॑ती॒ द्यौर् द्यौर् जाग॑ त्य॒न्तरि॑क्ष म॒न्तरि॑क्ष॒म् जाग॑ती॒ द्यौः । \newline
57. जाग॑ती॒ द्यौर् द्यौर् जाग॑ती॒ जाग॑ती॒ द्यौ रानु॑ष्टुभी॒ रानु॑ष्टुभी॒र् द्यौर् जाग॑ती॒ जाग॑ती॒ द्यौ रानु॑ष्टुभीः । \newline
58. द्यौ रानु॑ष्टुभी॒ रानु॑ष्टुभी॒र् द्यौर् द्यौ रानु॑ष्टुभी॒र् दिशो॒ दिश॒ आनु॑ष्टुभी॒र् द्यौर् द्यौ रानु॑ष्टुभी॒र् दिशः॑ । \newline
59. आनु॑ष्टुभी॒र् दिशो॒ दिश॒ आनु॑ष्टुभी॒ रानु॑ष्टुभी॒र् दिश॒ श्छन्दो॑भि॒ श्छन्दो॑भि॒र् दिश॒ आनु॑ष्टुभी॒ रानु॑ष्टुभी॒र् दिश॒ श्छन्दो॑भिः । \newline
60. आनु॑ष्टुभी॒रित्यानु॑ - स्तु॒भीः॒ । \newline
61. दिश॒ श्छन्दो॑भि॒ श्छन्दो॑भि॒र् दिशो॒ दिश॒ श्छन्दो॑भि रे॒वैव छन्दो॑भि॒र् दिशो॒ दिश॒ श्छन्दो॑भि रे॒व । \newline
62. छन्दो॑भि रे॒वैव छन्दो॑भि॒ श्छन्दो॑भि रे॒वे मा नि॒मा ने॒व छन्दो॑भि॒ श्छन्दो॑भि रे॒वे मान् । \newline
63. छन्दो॑भि॒रिति॒ छन्दः॑ - भिः॒ । \newline
64. ए॒वे मा नि॒मा ने॒वैवे मान् ॅलो॒कान् ॅलो॒का नि॒मा ने॒वैवे मान् ॅलो॒कान् । \newline
65. इ॒मान् ॅलो॒कान् ॅलो॒का नि॒मा नि॒मान् ॅलो॒कान्. य॑थापू॒र्वं ॅय॑थापू॒र्वम् ॅलो॒का नि॒मा नि॒मान् ॅलो॒कान्. य॑थापू॒र्वम् । \newline
66. लो॒कान्. य॑थापू॒र्वं ॅय॑थापू॒र्वम् ॅलो॒कान् ॅलो॒कान्. य॑थापू॒र्व म॒भ्य॑भि य॑थापू॒र्वम् ॅलो॒कान् 
ॅलो॒कान्. य॑थापू॒र्व म॒भि । \newline
67. य॒था॒पू॒र्व म॒भ्य॑भि य॑थापू॒र्वं ॅय॑थापू॒र्व म॒भि ज॑यति जयत्य॒भि य॑थापू॒र्वं ॅय॑थापू॒र्व म॒भि ज॑यति । \newline
68. य॒था॒पू॒र्वमिति॑ यथा - पू॒र्वम् । \newline
69. अ॒भि ज॑यति जय त्य॒भ्य॑भि ज॑यति । \newline
70. ज॒य॒तीति॑ जयति । \newline
\pagebreak
\markright{ TS 1.7.6.1  \hfill https://www.vedavms.in \hfill}
\addcontentsline{toc}{section}{ TS 1.7.6.1 }
\section*{ TS 1.7.6.1 }

\textbf{TS 1.7.6.1 } \newline
\textbf{Samhita Paata} \newline

अग॑न्म॒ सुवः॒ सुव॑रग॒न्मेत्या॑ह सुव॒र्गमे॒व लो॒कमे॑ति स॒न्दृश॑स्ते॒ मा छि॑थ्सि॒ यत्ते॒ तप॒स्तस्मै॑ ते॒ मा ऽऽ वृ॒क्षीत्या॑ह यथाय॒जु-रे॒वैतथ् सु॒भूर॑सि॒ श्रेष्ठो॑ रश्मी॒नामा॑यु॒र्द्धा अ॒स्यायु॑र्मे धे॒हीत्या॑हा॒ऽऽशिष॑मे॒वैतामा शा᳚स्ते॒ प्र वा ए॒षो᳚ऽस्मान् ॅलो॒काच्च्य॑वते॒ यो - [ ] \newline

\textbf{Pada Paata} \newline

अग॑न्म । सुवः॑ । सुवः॑ । अ॒ग॒न्म॒ । इति॑ । आ॒ह॒ । सु॒व॒र्गमिति॑ सुवः - गम् । ए॒व । लो॒कम् । ए॒ति॒ । स॒दृंश॒ इति॑ सं -दृशः॑ । ते॒ । मा । छि॒थ्सि॒ । यत् । ते॒ । तपः॑ । तस्मै᳚ । ते॒ । मा । एति॑ । वृ॒क्षि॒ । इति॑ । आ॒ह॒ । य॒था॒य॒जुरिति॑ यथा - य॒जुः । ए॒व । ए॒तत् । सु॒भूरिति॑ सु- भूः । अ॒सि॒ । श्रेष्ठः॑ । र॒श्मी॒नाम् । आ॒यु॒द्‌र्धा इत्या॑युः - धाः । अ॒सि॒ । आयुः॑ । मे॒ । धे॒हि॒ । इति॑ । आ॒ह॒ । आ॒शिष॒मित्या᳚-शिष᳚म् । ए॒व । ए॒ताम् । एति॑ । शा॒स्ते॒ । प्रेति॑ । वै । ए॒षः । अ॒स्मात् । लो॒कात् । च्य॒व॒ते॒ । यः ।  \newline


\textbf{Krama Paata} \newline

अग॑न्म॒ सुवः॑ । सुवः॒ सुवः॑ । सुव॑रगन्म । अ॒ग॒न्मेति॑ । इत्या॑ह । आ॒ह॒ सु॒व॒र्गम् । सु॒व॒र्गमे॒व । सु॒व॒र्गमिति॑ सुवः - गम् । ए॒व लो॒कम् । लो॒कमे॑ति । ए॒ति॒ स॒न्दृशः॑ । स॒न्दृश॑स्ते । स॒न्दृश॒ इति॑ सम् - दृशः॑ । ते॒ मा । मा छि॑थ्सि । छि॒थ्सि॒ यत् । यत् ते᳚ । ते॒ तपः॑ । तप॒स्तस्मै᳚ । तस्मै॑ ते । ते॒ मा । मा ऽऽ वृ॑क्षि । आ वृ॑क्षि । वृ॒क्षीति॑ । इत्या॑ह । आ॒ह॒ य॒था॒य॒जुः । य॒था॒य॒जुरे॒व । य॒था॒य॒जुरिति॑ यथा - य॒जुः । ए॒वैतत् । ए॒तथ् सु॒भूः । सु॒भूर॑सि । सु॒भूरिति॑ सु - भूः । अ॒सि॒ श्रेष्ठः॑ । श्रेष्ठो॑ रश्मी॒नाम् । र॒श्मी॒नामा॑यु॒र्द्धाः । आ॒यु॒र्द्धा अ॑सि । आ॒यु॒र्द्धा इत्या॑युः - धाः । अ॒स्यायुः॑ । आयु॑र्,मे । मे॒ धे॒हि॒ । धे॒हीति॑ । इत्या॑ह । आ॒हा॒शिष᳚म् । आ॒शिष॑मे॒व । आ॒शिष॒मित्या᳚ - शिष᳚म् । ए॒वैताम् । ए॒तामा । आ शा᳚स्ते । शा॒स्ते॒ प्र । प्र वै । वा ए॒षः । ए॒षो᳚ऽस्मात् । अ॒स्माल्लो॒कात् । लो॒काच्च्य॑वते । च्य॒व॒ते॒ यः । यो वि॑ष्णुक्र॒मान् \newline

\textbf{Jatai Paata} \newline

1. अग॑न्म॒ सुवः॒ सुव॒ रग॒न्मा ग॑न्म॒ सुवः॑ । \newline
2. सुवः॒ सुवः॑ । \newline
3. सुव॑ रगन्मा गन्म॒ सुवः॒ सुव॑ रगन्म । \newline
4. अ॒ग॒न्मे ती त्य॑गन्मा ग॒न्मे ति॑ । \newline
5. इत्या॑ हा॒हे तीत्या॑ह । \newline
6. आ॒ह॒ सु॒व॒र्गꣳ सु॑व॒र्ग मा॑हाह सुव॒र्गम् । \newline
7. सु॒व॒र्ग मे॒वैव सु॑व॒र्गꣳ सु॑व॒र्ग मे॒व । \newline
8. सु॒व॒र्गमिति॑ सुवः - गम् । \newline
9. ए॒व लो॒कम् ॅलो॒क मे॒वैव लो॒कम् । \newline
10. लो॒क मे᳚त्येति लो॒कम् ॅलो॒क मे॑ति । \newline
11. ए॒ति॒ स॒न्दृशः॑ स॒न्दृश॑ एत्येति स॒न्दृशः॑ । \newline
12. स॒न्दृश॑ स्ते ते स॒न्दृशः॑ स॒न्दृश॑ स्ते । \newline
13. स॒न्दृश॒ इति॑ सं - दृशः॑ । \newline
14. ते॒ मा मा ते॑ ते॒ मा । \newline
15. मा छि॑थ्सि छिथ्सि॒ मा मा छि॑थ्सि । \newline
16. छि॒थ्सि॒ यद् यच् छि॑थ्सि छिथ्सि॒ यत् । \newline
17. यत् ते॑ ते॒ यद् यत् ते᳚ । \newline
18. ते॒ तप॒ स्तप॑ स्ते ते॒ तपः॑ । \newline
19. तप॒ स्तस्मै॒ तस्मै॒ तप॒ स्तप॒ स्तस्मै᳚ । \newline
20. तस्मै॑ ते ते॒ तस्मै॒ तस्मै॑ ते । \newline
21. ते॒ मा मा ते॑ ते॒ मा । \newline
22. मा ऽऽवृ॑क्षि वृ॒क्ष्या मा मा ऽऽवृ॑क्षि । \newline
23. आ वृ॑क्षि वृ॒क्ष्या वृ॑क्षि । \newline
24. वृ॒क्षीतीति॑ वृक्षि वृ॒क्षीति॑ । \newline
25. इत्या॑हा॒हे तीत्या॑ह । \newline
26. आ॒ह॒ य॒था॒य॒जुर् य॑थाय॒जु रा॑हाह यथाय॒जुः । \newline
27. य॒था॒य॒जु रे॒वैव य॑थाय॒जुर् य॑थाय॒जु रे॒व । \newline
28. य॒था॒य॒जुरिति॑ यथा - य॒जुः । \newline
29. ए॒वै त दे॒त दे॒वै वैतत् । \newline
30. ए॒तथ् सु॒भूः सु॒भू रे॒त दे॒तथ् सु॒भूः । \newline
31. सु॒भू र॑स्यसि सु॒भूः सु॒भू र॑सि । \newline
32. सु॒भूरिति॑ सु - भूः । \newline
33. अ॒सि॒ श्रेष्ठः॒ श्रेष्ठो᳚ ऽस्यसि॒ श्रेष्ठः॑ । \newline
34. श्रेष्ठो॑ रश्मी॒नाꣳ र॑श्मी॒नाꣳ श्रेष्ठः॒ श्रेष्ठो॑ रश्मी॒नाम् । \newline
35. र॒श्मी॒ना मा॑यु॒र्द्धा आ॑यु॒र्द्धा र॑श्मी॒नाꣳ र॑श्मी॒ना मा॑यु॒र्द्धाः । \newline
36. आ॒यु॒र्द्धा अ॑स्य स्यायु॒र्द्धा आ॑यु॒र्द्धा अ॑सि । \newline
37. आ॒यु॒र्द्धा इत्या॑युः - धाः । \newline
38. अ॒स्यायु॒ रायु॑ रस्य॒स्यायुः॑ । \newline
39. आयु॑र् मे म॒ आयु॒ रायु॑र् मे । \newline
40. मे॒ धे॒हि॒ धे॒हि॒ मे॒ मे॒ धे॒हि॒ । \newline
41. धे॒हीतीति॑ धेहि धे॒हीति॑ । \newline
42. इत्या॑ हा॒हे तीत्या॑ह । \newline
43. आ॒हा॒शिष॑ मा॒शिष॑ माहा हा॒शिष᳚म् । \newline
44. आ॒शिष॑ मे॒वै वाशिष॑ मा॒शिष॑ मे॒व । \newline
45. आ॒शिष॒मित्या᳚ - शिष᳚म् । \newline
46. ए॒वैता मे॒ता मे॒वै वैताम् । \newline
47. ए॒ता मैता मे॒ता मा । \newline
48. आ शा᳚स्ते शास्त॒ आ शा᳚स्ते । \newline
49. शा॒स्ते॒ प्र प्र शा᳚स्ते शास्ते॒ प्र । \newline
50. प्र वै वै प्र प्र वै । \newline
51. वा ए॒ष ए॒ष वै वा ए॒षः । \newline
52. ए॒षो᳚ ऽस्मा द॒स्मा दे॒ष ए॒षो᳚ ऽस्मात् । \newline
53. अ॒स्मा ल्लो॒का ल्लो॒का द॒स्मा द॒स्मा ल्लो॒कात् । \newline
54. लो॒काच् च्य॑वते च्यवते लो॒का ल्लो॒काच् च्य॑वते । \newline
55. च्य॒व॒ते॒ यो यश्च्य॑वते च्यवते॒ यः । \newline
56. यो वि॑ष्णुक्र॒मान्. वि॑ष्णुक्र॒मान्. यो यो वि॑ष्णुक्र॒मान् । \newline

\textbf{Ghana Paata } \newline

1. अग॑न्म॒ सुवः॒ सुव॒ रग॒न्मा ग॑न्म॒ सुवः॑ । \newline
2. सुवः॒ सुवः॑ । \newline
3. सुव॑ रगन्मा गन्म॒ सुवः॒ सुव॑ रग॒न्मे तीत्य॑ गन्म॒ सुवः॒ सुव॑ रग॒न्मे ति॑ । \newline
4. अ॒ग॒न्मे तीत् य॑गन्मा ग॒न्मे त्या॑हा॒हे त्य॑गन्मा ग॒न्मे त्या॑ह । \newline
5. इत्या॑हा॒हे तीत्या॑ह सुव॒र्गꣳ सु॑व॒र्ग मा॒हे तीत्या॑ह सुव॒र्गम् । \newline
6. आ॒ह॒ सु॒व॒र्गꣳ सु॑व॒र्ग मा॑हाह सुव॒र्ग मे॒वैव सु॑व॒र्ग मा॑हाह सुव॒र्ग मे॒व । \newline
7. सु॒व॒र्ग मे॒वैव सु॑व॒र्गꣳ सु॑व॒र्ग मे॒व लो॒कम् ॅलो॒क मे॒व सु॑व॒र्गꣳ सु॑व॒र्ग मे॒व लो॒कम् । \newline
8. सु॒व॒र्गमिति॑ सुवः - गम् । \newline
9. ए॒व लो॒कम् ॅलो॒क मे॒वैव लो॒क मे᳚त्येति लो॒क मे॒वैव लो॒क मे॑ति । \newline
10. लो॒क मे᳚त्येति लो॒कम् ॅलो॒क मे॑ति स॒न्दृशः॑ स॒न्दृश॑ एति लो॒कम् ॅलो॒क मे॑ति स॒न्दृशः॑ । \newline
11. ए॒ति॒ स॒न्दृशः॑ स॒न्दृश॑ एत्येति स॒न्दृश॑ स्ते ते स॒न्दृश॑ एत्येति स॒न्दृश॑ स्ते । \newline
12. स॒न्दृश॑ स्ते ते स॒न्दृशः॑ स॒न्दृश॑ स्ते॒ मा मा ते॑ स॒न्दृशः॑ स॒न्दृश॑ स्ते॒ मा । \newline
13. स॒न्दृश॒ इति॑ सं - दृशः॑ । \newline
14. ते॒ मा मा ते॑ ते॒ मा छि॑थ्सि छिथ्सि॒ मा ते॑ ते॒ मा छि॑थ्सि । \newline
15. मा छि॑थ्सि छिथ्सि॒ मा मा छि॑थ्सि॒ यद् यच् छि॑थ्सि॒ मा मा छि॑थ्सि॒ यत् । \newline
16. छि॒थ्सि॒ यद् यच् छि॑थ्सि छिथ्सि॒ यत् ते॑ ते॒ यच् छि॑थ्सि छिथ्सि॒ यत् ते᳚ । \newline
17. यत् ते॑ ते॒ यद् यत् ते॒ तप॒ स्तप॑ स्ते॒ यद् यत् ते॒ तपः॑ । \newline
18. ते॒ तप॒ स्तप॑ स्ते ते॒ तप॒ स्तस्मै॒ तस्मै॒ तप॑स्ते ते॒ तप॒ स्तस्मै᳚ । \newline
19. तप॒ स्तस्मै॒ तस्मै॒ तप॒ स्तप॒ स्तस्मै॑ ते ते॒ तस्मै॒ तप॒ स्तप॒ स्तस्मै॑ ते । \newline
20. तस्मै॑ ते ते॒ तस्मै॒ तस्मै॑ ते॒ मा मा ते॒ तस्मै॒ तस्मै॑ ते॒ मा । \newline
21. ते॒ मा मा ते॑ ते॒ मा ऽऽवृ॑क्षि वृ॒क्ष्या मा ते॑ ते॒ मा ऽऽवृ॑क्षि । \newline
22. मा ऽऽवृ॑क्षि वृ॒क्ष्या मा मा ऽऽवृ॒क्षीतीति॑ वृ॒क्ष्या मा मा ऽऽवृ॒क्षीति॑ । \newline
23. आ वृ॑क्षि वृ॒क्ष्या वृ॒क्षीतीति॑ वृ॒क्ष्या वृ॒क्षीति॑ । \newline
24. वृ॒क्षी तीति॑ वृक्षि वृ॒क्षी त्या॑हा॒हे ति॑ वृक्षि वृ॒क्षी त्या॑ह । \newline
25. इत्या॑हा॒हे तीत्या॑ह यथाय॒जुर् य॑थाय॒जुरा॒हे तीत्या॑ह यथाय॒जुः । \newline
26. आ॒ह॒ य॒था॒य॒जुर् य॑थाय॒जु रा॑हाह यथाय॒जु रे॒वैव य॑थाय॒जु रा॑हाह यथाय॒जु रे॒व । \newline
27. य॒था॒य॒जु रे॒वैव य॑थाय॒जुर् य॑थाय॒जु रे॒वैत दे॒त दे॒व य॑थाय॒जुर् य॑थाय॒जु रे॒वैतत् । \newline
28. य॒था॒य॒जुरिति॑ यथा - य॒जुः । \newline
29. ए॒वैत दे॒त दे॒वैवैतथ् सु॒भूः सु॒भू रे॒त दे॒वैवैतथ् सु॒भूः । \newline
30. ए॒तथ् सु॒भूः सु॒भू रे॒त दे॒तथ् सु॒भू र॑स्यसि सु॒भू रे॒त दे॒तथ् सु॒भू र॑सि । \newline
31. सु॒भू र॑स्यसि सु॒भूः सु॒भू र॑सि॒ श्रेष्ठः॒ श्रेष्ठो॑ ऽसि सु॒भूः सु॒भू र॑सि॒ श्रेष्ठः॑ । \newline
32. सु॒भूरिति॑ सु - भूः । \newline
33. अ॒सि॒ श्रेष्ठः॒ श्रेष्ठो᳚ ऽस्यसि॒ श्रेष्ठो॑ रश्मी॒नाꣳ र॑श्मी॒नाꣳ श्रेष्ठो᳚ ऽस्यसि॒ श्रेष्ठो॑ रश्मी॒नाम् । \newline
34. श्रेष्ठो॑ रश्मी॒नाꣳ र॑श्मी॒नाꣳ श्रेष्ठः॒ श्रेष्ठो॑ रश्मी॒ना मा॑यु॒र्द्धा आ॑यु॒र्द्धा र॑श्मी॒नाꣳ श्रेष्ठः॒ श्रेष्ठो॑ रश्मी॒ना मा॑यु॒र्द्धाः । \newline
35. र॒श्मी॒ना मा॑यु॒र्द्धा आ॑यु॒र्द्धा र॑श्मी॒नाꣳ र॑श्मी॒ना मा॑यु॒र्द्धा अ॑स्य स्यायु॒र्द्धा र॑श्मी॒नाꣳ र॑श्मी॒ना मा॑यु॒र्द्धा अ॑सि । \newline
36. आ॒यु॒र्द्धा अ॑स्य स्यायु॒र्द्धा आ॑यु॒र्द्धा अ॒स्यायु॒ रायु॑ रस्यायु॒र्द्धा आ॑यु॒र्द्धा अ॒स्यायुः॑ । \newline
37. आ॒यु॒र्द्धा इत्या॑युः - धाः । \newline
38. अ॒स्यायु॒ रायु॑ रस्य॒ स्यायु॑र् मे म॒ आयु॑ रस्य॒स्यायु॑र् मे । \newline
39. आयु॑र् मे म॒ आयु॒ रायु॑र् मे धेहि धेहि म॒ आयु॒ रायु॑र् मे धेहि । \newline
40. मे॒ धे॒हि॒ धे॒हि॒ मे॒ मे॒ धे॒हीतीति॑ धेहि मे मे धे॒हीति॑ । \newline
41. धे॒हीतीति॑ धेहि धे॒ही त्या॑हा॒हे ति॑ धेहि धे॒हीत्या॑ह । \newline
42. इत्या॑हा॒हे ती त्या॑हा॒शिष॑ मा॒शिष॑ मा॒हे ती त्या॑हा॒शिष᳚म् । \newline
43. आ॒हा॒शिष॑ मा॒शिष॑ माहा हा॒शिष॑ मे॒वै वाशिष॑ माहा हा॒शिष॑ मे॒व । \newline
44. आ॒शिष॑ मे॒वै वाशिष॑ मा॒शिष॑ मे॒वैता मे॒ता मे॒वाशिष॑ मा॒शिष॑ मे॒वैताम् । \newline
45. आ॒शिष॒मित्या᳚ - शिष᳚म् । \newline
46. ए॒वैता मे॒ता मे॒ वैवैता मैता मे॒ वैवैता मा । \newline
47. ए॒ता मैता मे॒ता मा शा᳚स्ते शास्त॒ ऐता मे॒ता मा शा᳚स्ते । \newline
48. आ शा᳚स्ते शास्त॒ आ शा᳚स्ते॒ प्र प्र शा᳚स्त॒ आ शा᳚स्ते॒ प्र । \newline
49. शा॒स्ते॒ प्र प्र शा᳚स्ते शास्ते॒ प्र वै वै प्र शा᳚स्ते शास्ते॒ प्र वै । \newline
50. प्र वै वै प्र प्र वा ए॒ष ए॒ष वै प्र प्र वा ए॒षः । \newline
51. वा ए॒ष ए॒ष वै वा ए॒षो᳚ऽस्मा द॒स्मा दे॒ष वै वा ए॒षो᳚ऽस्मात् । \newline
52. ए॒षो᳚ऽस्मा द॒स्मा दे॒ष ए॒षो᳚ऽस्मा ल्लो॒का ल्लो॒का द॒स्मा दे॒ष ए॒षो᳚ऽस्मा ल्लो॒कात् । \newline
53. अ॒स्मा ल्लो॒का ल्लो॒का द॒स्मा द॒स्मा ल्लो॒काच् च्य॑वते च्यवते लो॒का द॒स्मा द॒स्मा ल्लो॒काच् च्य॑वते । \newline
54. लो॒काच् च्य॑वते च्यवते लो॒का ल्लो॒काच् च्य॑वते॒ यो यश्च्य॑वते लो॒का ल्लो॒काच् च्य॑वते॒ यः । \newline
55. च्य॒व॒ते॒ यो यश्च्य॑वते च्यवते॒ यो वि॑ष्णुक्र॒मान्. वि॑ष्णुक्र॒मान्. यश्च्य॑वते च्यवते॒ यो वि॑ष्णुक्र॒मान् । \newline
56. यो वि॑ष्णुक्र॒मान्. वि॑ष्णुक्र॒मान्. यो यो वि॑ष्णुक्र॒मान् क्रम॑ते॒ क्रम॑ते विष्णुक्र॒मान्. यो यो वि॑ष्णुक्र॒मान् क्रम॑ते । \newline
\pagebreak
\markright{ TS 1.7.6.2  \hfill https://www.vedavms.in \hfill}
\addcontentsline{toc}{section}{ TS 1.7.6.2 }
\section*{ TS 1.7.6.2 }

\textbf{TS 1.7.6.2 } \newline
\textbf{Samhita Paata} \newline

वि॑ष्णुक्र॒मान् क्रम॑ते सुव॒र्गाय॒ हि लो॒काय॑ विष्णुक्र॒माः क्र॒म्यन्ते᳚ ब्रह्मवा॒दिनो॑ वदन्ति॒ स त्वै वि॑ष्णुक्र॒मान् क्र॑मेत॒ य इ॒मान् ॅलो॒कान् भ्रातृ॑व्यस्य सं॒ॅविद्य॒ पुन॑रि॒मं ॅलो॒कं प्र॑त्यव॒रोहे॒दित्ये॒ष वा अ॒स्य लो॒कस्य॑ प्रत्यवरो॒हो यदाहे॒दम॒हम॒मुं भ्रातृ॑व्यमा॒भ्यो दि॒ग्भ्यो᳚ऽस्यै दि॒व इती॒माने॒व लो॒कान् भ्रातृ॑व्यस्य सं॒ॅविद्य॒ पुन॑रि॒मं ॅलो॒कं प्र॒त्यव॑रोहति॒ सं - [ ] \newline

\textbf{Pada Paata} \newline

वि॒ष्णु॒क्र॒मानिति॑ विष्णु - क्र॒मान् । क्रम॑ते । सु॒व॒र्गायेति॑ सुवः-गाय॑ । हि । लो॒काय॑ । वि॒ष्णु॒क्र॒मा इति॑ विष्णु - क्र॒माः । क्र॒म्यन्ते᳚ । ब्र॒ह्म॒वा॒दिन॒ इति॑ ब्रह्म - वा॒दिनः॑ । व॒द॒न्ति॒ । सः । तु । वै । वि॒ष्णु॒क्र॒मानिति॑ विष्णु - क्र॒मान् । क्र॒मे॒त॒ । यः । इ॒मान् । लो॒कान् । भ्रातृ॑व्यस्य । स॒म्ॅविद्येति॑ सं - विद्य॑ । पुनः॑ । इ॒मम् । लो॒कम् । प्र॒त्य॒व॒रोहे॒दिति॑ प्रति - अ॒व॒रोहे᳚त् । इति॑ । ए॒षः । वै । अ॒स्य । लो॒कस्य॑ । प्र॒त्य॒व॒रो॒ह इति॑ प्रति - अ॒व॒रो॒हः । यत् । आह॑ । इ॒दम् । अ॒हम् । अ॒मुम् । भ्रातृ॑व्यम् । आ॒भ्यः । दि॒ग्भ्य इति॑ दिक् - भ्यः । अ॒स्यै । दि॒वः । इति॑ । इ॒मान् । ए॒व । लो॒कान् । भ्रातृ॑व्यस्य । स॒म्ॅविद्येति॑ सं - विद्य॑ । पुनः॑ । इ॒मम् । लो॒कम् । प्र॒त्यव॑रोह॒तीति॑ प्रति - अव॑रोहति । समिति॑ ।  \newline


\textbf{Krama Paata} \newline

वि॒ष्णु॒क्र॒मान्,क्रम॑ते । वि॒ष्णु॒क्र॒मानिति॑ विष्णु - क्र॒मान् । क्रम॑ते सुव॒र्गाय॑ । सु॒व॒र्गाय॒ हि । सु॒व॒र्गायेति॑ सुवः - गाय॑ । हि लो॒काय॑ । लो॒काय॑ विष्णुक्र॒माः । वि॒ष्णु॒क्र॒माः क्र॒म्यन्ते᳚ । वि॒ष्णु॒क्र॒मा इति॑ विष्णु - क्र॒माः । क्र॒म्यन्ते᳚ ब्रह्मवा॒दिनः॑ । ब्र॒ह्म॒वा॒दिनो॑ वदन्ति । ब्र॒ह्म॒वा॒दिन॒ इति॑ ब्रह्म - वा॒दिनः॑ । व॒द॒न्ति॒ सः । स तु । त्वै । वै वि॑ष्णुक्र॒मान् । वि॒ष्णु॒क्र॒मान्,क्र॑मेत । वि॒ष्णु॒क्र॒मानिति॑ विष्णु - क्र॒मान् । क्र॒मे॒त॒ यः । य इ॒मान् । इ॒मान् ॅलो॒कान् । लो॒कान् भ्रातृ॑व्यस्य । भ्रातृ॑व्यस्य स॒म्ॅविद्य॑ । स॒म्ॅविद्य॒ पुनः॑ । स॒म्ॅविद्येति॑ सं - विद्य॑ । पुन॑रि॒मम् । इ॒मं ॅलो॒कम् । लो॒कम् प्र॑त्यव॒रोहे᳚त् । प्र॒त्य॒व॒रोहे॒दिति॑ । प्र॒त्य॒व॒रोहे॒दिति॑ प्रति - अ॒व॒रोहे᳚त् । इत्ये॒षः । ए॒ष वै । वा अ॒स्य । अ॒स्य लो॒कस्य॑ । लो॒कस्य॑ प्रत्यवरो॒हः । प्र॒त्य॒व॒रो॒हो यत् । प्र॒त्य॒व॒रो॒ह इति॑ प्रति - अ॒व॒रो॒हः । यदाह॑ । आहे॒दम् । इ॒दम॒हम् । अ॒हम॒मुम् । अ॒मुम् भ्रातृ॑व्यम् । भ्रातृ॑व्यमा॒भ्यः । आ॒भ्यो दि॒ग्भ्यः । दि॒ग्भ्यो᳚ऽस्यै । दि॒ग्भ्य इति॑ दिक् - भ्यः । अ॒स्यै दि॒वः । दि॒व इति॑ । इती॒मान् । इ॒माने॒व । ए॒व लो॒कान् । लो॒कान् भ्रातृ॑व्यस्य । भ्रातृ॑व्यस्य स॒म्ॅविद्य॑ । स॒म्ॅविद्य॒ पुनः॑ । स॒म्ॅविद्येति॑ सम् - विद्य॑ । पुन॑रि॒मम् । इ॒मम् ॅलो॒कम् । लो॒कम् प्र॒त्यव॑रोहति । प्र॒त्यव॑रोहति॒ सम् । प्र॒त्यव॑रोह॒तीति॑ प्रति - अव॑रोहति । सम् ज्योति॑षा \newline

\textbf{Jatai Paata} \newline

1. वि॒ष्णु॒क्र॒मान् क्रम॑ते॒ क्रम॑ते विष्णुक्र॒मान्. वि॑ष्णुक्र॒मान् क्रम॑ते । \newline
2. वि॒ष्णु॒क्र॒मानिति॑ विष्णु - क्र॒मान् । \newline
3. क्रम॑ते सुव॒र्गाय॑ सुव॒र्गाय॒ क्रम॑ते॒ क्रम॑ते सुव॒र्गाय॑ । \newline
4. सु॒व॒र्गाय॒ हि हि सु॑व॒र्गाय॑ सुव॒र्गाय॒ हि । \newline
5. सु॒व॒र्गायेति॑ सुवः - गाय॑ । \newline
6. हि लो॒काय॑ लो॒काय॒ हि हि लो॒काय॑ । \newline
7. लो॒काय॑ विष्णुक्र॒मा वि॑ष्णुक्र॒मा लो॒काय॑ लो॒काय॑ विष्णुक्र॒माः । \newline
8. वि॒ष्णु॒क्र॒माः क्र॒म्यन्ते᳚ क्र॒म्यन्ते॑ विष्णुक्र॒मा वि॑ष्णुक्र॒माः क्र॒म्यन्ते᳚ । \newline
9. वि॒ष्णु॒क्र॒मा इति॑ विष्णु - क्र॒माः । \newline
10. क्र॒म्यन्ते᳚ ब्रह्मवा॒दिनो᳚ ब्रह्मवा॒दिनः॑ क्र॒म्यन्ते᳚ क्र॒म्यन्ते᳚ ब्रह्मवा॒दिनः॑ । \newline
11. ब्र॒ह्म॒वा॒दिनो॑ वदन्ति वदन्ति ब्रह्मवा॒दिनो᳚ ब्रह्मवा॒दिनो॑ वदन्ति । \newline
12. ब्र॒ह्म॒वा॒दिन॒ इति॑ ब्रह्म - वा॒दिनः॑ । \newline
13. व॒द॒न्ति॒ स स व॑दन्ति वदन्ति॒ सः । \newline
14. स तु तु स स तु । \newline
15. त्वै वै तु त्वै । \newline
16. वै वि॑ष्णुक्र॒मान्. वि॑ष्णुक्र॒मान्. वै वै वि॑ष्णुक्र॒मान् । \newline
17. वि॒ष्णु॒क्र॒मान् क्र॑मेत क्रमेत विष्णुक्र॒मान्. वि॑ष्णुक्र॒मान् क्र॑मेत । \newline
18. वि॒ष्णु॒क्र॒मानिति॑ विष्णु - क्र॒मान् । \newline
19. क्र॒मे॒त॒ यो यः क्र॑मेत क्रमेत॒ यः । \newline
20. य इ॒मा नि॒मान्. यो य इ॒मान् । \newline
21. इ॒मान् ॅलो॒कान् ॅलो॒का नि॒मा नि॒मान् ॅलो॒कान् । \newline
22. लो॒कान् भ्रातृ॑व्यस्य॒ भ्रातृ॑व्यस्य लो॒कान् ॅलो॒कान् भ्रातृ॑व्यस्य । \newline
23. भ्रातृ॑व्यस्य स॒म्ॅविद्य॑ स॒म्ॅविद्य॒ भ्रातृ॑व्यस्य॒ भ्रातृ॑व्यस्य स॒म्ॅविद्य॑ । \newline
24. स॒म्ॅविद्य॒ पुनः॒ पुनः॑ स॒म्ॅविद्य॑ स॒म्ॅविद्य॒ पुनः॑ । \newline
25. स॒म्ॅविद्येति॑ सं - विद्य॑ । \newline
26. पुन॑ रि॒म मि॒मम् पुनः॒ पुन॑ रि॒मम् । \newline
27. इ॒मम् ॅलो॒कम् ॅलो॒क मि॒म मि॒मम् ॅलो॒कम् । \newline
28. लो॒कम् प्र॑त्यव॒रोहे᳚त् प्रत्यव॒रोहे᳚ ल्लो॒कम् ॅलो॒कम् प्र॑त्यव॒रोहे᳚त् । \newline
29. प्र॒त्य॒व॒रोहे॒ दितीति॑ प्रत्यव॒रोहे᳚त् प्रत्यव॒रोहे॒ दिति॑ । \newline
30. प्र॒त्य॒व॒रोहे॒दिति॑ प्रति - अ॒व॒रोहे᳚त् । \newline
31. इत्ये॒ष ए॒ष इती त्ये॒षः । \newline
32. ए॒ष वै वा ए॒ष ए॒ष वै । \newline
33. वा अ॒स्यास्य वै वा अ॒स्य । \newline
34. अ॒स्य लो॒कस्य॑ लो॒कस्या॒ स्यास्य लो॒कस्य॑ । \newline
35. लो॒कस्य॑ प्रत्यवरो॒हः प्र॑त्यवरो॒हो लो॒कस्य॑ लो॒कस्य॑ प्रत्यवरो॒हः । \newline
36. प्र॒त्य॒व॒रो॒हो यद् यत् प्र॑त्यवरो॒हः प्र॑त्यवरो॒हो यत् । \newline
37. प्र॒त्य॒व॒रो॒ह इति॑ प्रति - अ॒व॒रो॒हः । \newline
38. यदा हाह॒ यद् यदाह॑ । \newline
39. आहे॒ द मि॒द माहाहे॒ दम् । \newline
40. इ॒द म॒ह म॒ह मि॒द मि॒द म॒हम् । \newline
41. अ॒ह म॒मु म॒मु म॒ह म॒ह म॒मुम् । \newline
42. अ॒मुम् भ्रातृ॑व्य॒म् भ्रातृ॑व्य म॒मु म॒मुम् भ्रातृ॑व्यम् । \newline
43. भ्रातृ॑व्य मा॒भ्य आ॒भ्यो भ्रातृ॑व्य॒म् भ्रातृ॑व्य मा॒भ्यः । \newline
44. आ॒भ्यो दि॒ग्भ्यो दि॒ग्भ्य आ॒भ्य आ॒भ्यो दि॒ग्भ्यः । \newline
45. दि॒ग्भ्यो᳚ ऽस्या अ॒स्यै दि॒ग्भ्यो दि॒ग्भ्यो᳚ ऽस्यै । \newline
46. दि॒ग्भ्य इति॑ दिक् - भ्यः । \newline
47. अ॒स्यै दि॒वो दि॒वो᳚ ऽस्या अ॒स्यै दि॒वः । \newline
48. दि॒व इतीति॑ दि॒वो दि॒व इति॑ । \newline
49. इती॒मा नि॒मा निती ती॒मान् । \newline
50. इ॒मा ने॒वैवे मा नि॒मा ने॒व । \newline
51. ए॒व लो॒कान् ॅलो॒का ने॒वैव लो॒कान् । \newline
52. लो॒कान् भ्रातृ॑व्यस्य॒ भ्रातृ॑व्यस्य लो॒कान् ॅलो॒कान् भ्रातृ॑व्यस्य । \newline
53. भ्रातृ॑व्यस्य स॒म्ॅविद्य॑ स॒म्ॅविद्य॒ भ्रातृ॑व्यस्य॒ भ्रातृ॑व्यस्य स॒म्ॅविद्य॑ । \newline
54. स॒म्ॅविद्य॒ पुनः॒ पुनः॑ स॒म्ॅविद्य॑ स॒म्ॅविद्य॒ पुनः॑ । \newline
55. स॒म्ॅविद्येति॑ सं - विद्य॑ । \newline
56. पुन॑ रि॒म मि॒मम् पुनः॒ पुन॑ रि॒मम् । \newline
57. इ॒मम् ॅलो॒कम् ॅलो॒क मि॒म मि॒मम् ॅलो॒कम् । \newline
58. लो॒कम् प्र॒त्यव॑रोहति प्र॒त्यव॑रोहति लो॒कम् ॅलो॒कम् प्र॒त्यव॑रोहति । \newline
59. प्र॒त्यव॑रोहति॒ सꣳ सम् प्र॒त्यव॑रोहति प्र॒त्यव॑रोहति॒ सम् । \newline
60. प्र॒त्यव॑रोह॒तीति॑ प्रति - अव॑रोहति । \newline
61. सम् ज्योति॑षा॒ ज्योति॑षा॒ सꣳ सम् ज्योति॑षा । \newline

\textbf{Ghana Paata } \newline

1. वि॒ष्णु॒क्र॒मान् क्रम॑ते॒ क्रम॑ते विष्णुक्र॒मान्. वि॑ष्णुक्र॒मान् क्रम॑ते सुव॒र्गाय॑ सुव॒र्गाय॒ क्रम॑ते विष्णुक्र॒मान्. वि॑ष्णुक्र॒मान् क्रम॑ते सुव॒र्गाय॑ । \newline
2. वि॒ष्णु॒क्र॒मानिति॑ विष्णु - क्र॒मान् । \newline
3. क्रम॑ते सुव॒र्गाय॑ सुव॒र्गाय॒ क्रम॑ते॒ क्रम॑ते सुव॒र्गाय॒ हि हि सु॑व॒र्गाय॒ क्रम॑ते॒ क्रम॑ते सुव॒र्गाय॒ हि । \newline
4. सु॒व॒र्गाय॒ हि हि सु॑व॒र्गाय॑ सुव॒र्गाय॒ हि लो॒काय॑ लो॒काय॒ हि सु॑व॒र्गाय॑ सुव॒र्गाय॒ हि लो॒काय॑ । \newline
5. सु॒व॒र्गायेति॑ सुवः - गाय॑ । \newline
6. हि लो॒काय॑ लो॒काय॒ हि हि लो॒काय॑ विष्णुक्र॒मा वि॑ष्णुक्र॒मा लो॒काय॒ हि हि लो॒काय॑ विष्णुक्र॒माः । \newline
7. लो॒काय॑ विष्णुक्र॒मा वि॑ष्णुक्र॒मा लो॒काय॑ लो॒काय॑ विष्णुक्र॒माः क्र॒म्यन्ते᳚ क्र॒म्यन्ते॑ विष्णुक्र॒मा लो॒काय॑ लो॒काय॑ विष्णुक्र॒माः क्र॒म्यन्ते᳚ । \newline
8. वि॒ष्णु॒क्र॒माः क्र॒म्यन्ते᳚ क्र॒म्यन्ते॑ विष्णुक्र॒मा वि॑ष्णुक्र॒माः क्र॒म्यन्ते᳚ ब्रह्मवा॒दिनो᳚ ब्रह्मवा॒दिनः॑ क्र॒म्यन्ते॑ विष्णुक्र॒मा वि॑ष्णुक्र॒माः क्र॒म्यन्ते᳚ ब्रह्मवा॒दिनः॑ । \newline
9. वि॒ष्णु॒क्र॒मा इति॑ विष्णु - क्र॒माः । \newline
10. क्र॒म्यन्ते᳚ ब्रह्मवा॒दिनो᳚ ब्रह्मवा॒दिनः॑ क्र॒म्यन्ते᳚ क्र॒म्यन्ते᳚ ब्रह्मवा॒दिनो॑ वदन्ति वदन्ति ब्रह्मवा॒दिनः॑ क्र॒म्यन्ते᳚ क्र॒म्यन्ते᳚ ब्रह्मवा॒दिनो॑ वदन्ति । \newline
11. ब्र॒ह्म॒वा॒दिनो॑ वदन्ति वदन्ति ब्रह्मवा॒दिनो᳚ ब्रह्मवा॒दिनो॑ वदन्ति॒ स स व॑दन्ति ब्रह्मवा॒दिनो᳚ ब्रह्मवा॒दिनो॑ वदन्ति॒ सः । \newline
12. ब्र॒ह्म॒वा॒दिन॒ इति॑ ब्रह्म - वा॒दिनः॑ । \newline
13. व॒द॒न्ति॒ स स व॑दन्ति वदन्ति॒ स तु तु स व॑दन्ति वदन्ति॒ स तु । \newline
14. स तु तु स स त्वै वै तु स स त्वै । \newline
15. त्वै वै तुत् वै वि॑ष्णुक्र॒मान्. वि॑ष्णुक्र॒मान्. वै तुत् वै वि॑ष्णुक्र॒मान् । \newline
16. वै वि॑ष्णुक्र॒मान्. वि॑ष्णुक्र॒मान्. वै वै वि॑ष्णुक्र॒मान् क्र॑मेत क्रमेत विष्णुक्र॒मान्. वै वै वि॑ष्णुक्र॒मान् क्र॑मेत । \newline
17. वि॒ष्णु॒क्र॒मान् क्र॑मेत क्रमेत विष्णुक्र॒मान्. वि॑ष्णुक्र॒मान् क्र॑मेत॒ यो यः क्र॑मेत विष्णुक्र॒मान्. वि॑ष्णुक्र॒मान् क्र॑मेत॒ यः । \newline
18. वि॒ष्णु॒क्र॒मानिति॑ विष्णु - क्र॒मान् । \newline
19. क्र॒मे॒त॒ यो यः क्र॑मेत क्रमेत॒ य इ॒मा नि॒मान्. यः क्र॑मेत क्रमेत॒ य इ॒मान् । \newline
20. य इ॒मा नि॒मान्. यो य इ॒मान् ॅलो॒कान् ॅलो॒का नि॒मान्. यो य इ॒मान् ॅलो॒कान् । \newline
21. इ॒मान् ॅलो॒कान् ॅलो॒का नि॒मा नि॒मान् ॅलो॒कान् भ्रातृ॑व्यस्य॒ भ्रातृ॑व्यस्य लो॒का नि॒मा नि॒मान् ॅलो॒कान् भ्रातृ॑व्यस्य । \newline
22. लो॒कान् भ्रातृ॑व्यस्य॒ भ्रातृ॑व्यस्य लो॒कान् ॅलो॒कान् भ्रातृ॑व्यस्य स॒म्ॅविद्य॑ स॒म्ॅविद्य॒ भ्रातृ॑व्यस्य लो॒कान् ॅलो॒कान् भ्रातृ॑व्यस्य स॒म्ॅविद्य॑ । \newline
23. भ्रातृ॑व्यस्य स॒म्ॅविद्य॑ स॒म्ॅविद्य॒ भ्रातृ॑व्यस्य॒ भ्रातृ॑व्यस्य स॒म्ॅविद्य॒ पुनः॒ पुनः॑ स॒म्ॅविद्य॒ भ्रातृ॑व्यस्य॒ भ्रातृ॑व्यस्य स॒म्ॅविद्य॒ पुनः॑ । \newline
24. स॒म्ॅविद्य॒ पुनः॒ पुनः॑ स॒म्ॅविद्य॑ स॒म्ॅविद्य॒ पुन॑ रि॒म मि॒मम् पुनः॑ स॒म्ॅविद्य॑ स॒म्ॅविद्य॒ पुन॑ रि॒मम् । \newline
25. स॒म्ॅविद्येति॑ सं - विद्य॑ । \newline
26. पुन॑ रि॒म मि॒मम् पुनः॒ पुन॑ रि॒मम् ॅलो॒कम् ॅलो॒क मि॒मम् पुनः॒ पुन॑ रि॒मम् ॅलो॒कम् । \newline
27. इ॒मम् ॅलो॒कम् ॅलो॒क मि॒म मि॒मम् ॅलो॒कम् प्र॑त्यव॒रोहे᳚त् प्रत्यव॒रोहे᳚ ल्लो॒क मि॒म मि॒मम् ॅलो॒कम् प्र॑त्यव॒रोहे᳚त् । \newline
28. लो॒कम् प्र॑त्यव॒रोहे᳚त् प्रत्यव॒रोहे᳚ ल्लो॒कम् ॅलो॒कम् प्र॑त्यव॒रोहे॒ दितीति॑ प्रत्यव॒रोहे᳚ ल्लो॒कम् ॅलो॒कम् प्र॑त्यव॒रोहे॒ दिति॑ । \newline
29. प्र॒त्य॒व॒रोहे॒ दितीति॑ प्रत्यव॒रोहे᳚त् प्रत्यव॒रोहे॒ दित्ये॒ष ए॒ष इति॑ प्रत्यव॒रोहे᳚त् प्रत्यव॒रोहे॒ दित्ये॒षः । \newline
30. प्र॒त्य॒व॒रोहे॒दिति॑ प्रति - अ॒व॒रोहे᳚त् । \newline
31. इत्ये॒ष ए॒ष इतीत्ये॒ष वै वा ए॒ष इतीत्ये॒ष वै । \newline
32. ए॒ष वै वा ए॒ष ए॒ष वा अ॒स्यास्य वा ए॒ष ए॒ष वा अ॒स्य । \newline
33. वा अ॒स्यास्य वै वा अ॒स्य लो॒कस्य॑ लो॒क स्या॒स्य वै वा अ॒स्य लो॒कस्य॑ । \newline
34. अ॒स्य लो॒कस्य॑ लो॒क स्या॒स्या स्य लो॒कस्य॑ प्रत्यवरो॒हः प्र॑त्यवरो॒हो लो॒क स्या॒स्या स्य लो॒कस्य॑ प्रत्यवरो॒हः । \newline
35. लो॒कस्य॑ प्रत्यवरो॒हः प्र॑त्यवरो॒हो लो॒कस्य॑ लो॒कस्य॑ प्रत्यवरो॒हो यद् यत् प्र॑त्यवरो॒हो लो॒कस्य॑ लो॒कस्य॑ प्रत्यवरो॒हो यत् । \newline
36. प्र॒त्य॒व॒रो॒हो यद् यत् प्र॑त्यवरो॒हः प्र॑त्यवरो॒हो यदाहाह॒ यत् प्र॑त्यवरो॒हः प्र॑त्यवरो॒हो यदाह॑ । \newline
37. प्र॒त्य॒व॒रो॒ह इति॑ प्रति - अ॒व॒रो॒हः । \newline
38. यदाहाह॒ यद् यदाहे॒ द मि॒द माह॒ यद् यदाहे॒ दम् । \newline
39. आहे॒ द मि॒द माहाहे॒ द म॒ह म॒ह मि॒द माहाहे॒ द म॒हम् । \newline
40. इ॒द म॒ह म॒ह मि॒द मि॒द म॒ह म॒मु म॒मु म॒ह मि॒द मि॒द म॒ह म॒मुम् । \newline
41. अ॒ह म॒मु म॒मु म॒ह म॒ह म॒मुम् भ्रातृ॑व्य॒म् भ्रातृ॑व्य म॒मु म॒ह म॒ह म॒मुम् भ्रातृ॑व्यम् । \newline
42. अ॒मुम् भ्रातृ॑व्य॒म् भ्रातृ॑व्य म॒मु म॒मुम् भ्रातृ॑व्य मा॒भ्य आ॒भ्यो भ्रातृ॑व्य म॒मु म॒मुम् भ्रातृ॑व्य मा॒भ्यः । \newline
43. भ्रातृ॑व्य मा॒भ्य आ॒भ्यो भ्रातृ॑व्य॒म् भ्रातृ॑व्य मा॒भ्यो दि॒ग्भ्यो दि॒ग्भ्य आ॒भ्यो भ्रातृ॑व्य॒म् भ्रातृ॑व्य मा॒भ्यो दि॒ग्भ्यः । \newline
44. आ॒भ्यो दि॒ग्भ्यो दि॒ग्भ्य आ॒भ्य आ॒भ्यो दि॒ग्भ्यो᳚ ऽस्या अ॒स्यै दि॒ग्भ्य आ॒भ्य आ॒भ्यो दि॒ग्भ्यो᳚ ऽस्यै । \newline
45. दि॒ग्भ्यो᳚ ऽस्या अ॒स्यै दि॒ग्भ्यो दि॒ग्भ्यो᳚ ऽस्यै दि॒वो दि॒वो᳚ ऽस्यै दि॒ग्भ्यो दि॒ग्भ्यो᳚ ऽस्यै दि॒वः । \newline
46. दि॒ग्भ्य इति॑ दिक् - भ्यः । \newline
47. अ॒स्यै दि॒वो दि॒वो᳚ ऽस्या अ॒स्यै दि॒व इतीति॑ दि॒वो᳚ ऽस्या अ॒स्यै दि॒व इति॑ । \newline
48. दि॒व इतीति॑ दि॒वो दि॒व इती॒मा नि॒मा निति॑ दि॒वो दि॒व इती॒मान् । \newline
49. इती॒मा नि॒मा नि तीती॒मा ने॒वैवे मा नि तीती॒मा ने॒व । \newline
50. इ॒मा ने॒वैवे मा नि॒मा ने॒व लो॒कान् ॅलो॒का ने॒वे मा नि॒मा ने॒व लो॒कान् । \newline
51. ए॒व लो॒कान् ॅलो॒का ने॒वैव लो॒कान् भ्रातृ॑व्यस्य॒ भ्रातृ॑व्यस्य लो॒का ने॒वैव लो॒कान् भ्रातृ॑व्यस्य । \newline
52. लो॒कान् भ्रातृ॑व्यस्य॒ भ्रातृ॑व्यस्य लो॒कान् ॅलो॒कान् भ्रातृ॑व्यस्य स॒म्ॅविद्य॑ स॒म्ॅविद्य॒ भ्रातृ॑व्यस्य लो॒कान् ॅलो॒कान् भ्रातृ॑व्यस्य स॒म्ॅविद्य॑ । \newline
53. भ्रातृ॑व्यस्य स॒म्ॅविद्य॑ स॒म्ॅविद्य॒ भ्रातृ॑व्यस्य॒ भ्रातृ॑व्यस्य स॒म्ॅविद्य॒ पुनः॒ पुनः॑ स॒म्ॅविद्य॒ भ्रातृ॑व्यस्य॒ भ्रातृ॑व्यस्य स॒म्ॅविद्य॒ पुनः॑ । \newline
54. स॒म्ॅविद्य॒ पुनः॒ पुनः॑ स॒म्ॅविद्य॑ स॒म्ॅविद्य॒ पुन॑ रि॒म मि॒मम् पुनः॑ स॒म्ॅविद्य॑ स॒म्ॅविद्य॒ पुन॑ रि॒मम् । \newline
55. स॒म्ॅविद्येति॑ सं - विद्य॑ । \newline
56. पुन॑ रि॒म मि॒मम् पुनः॒ पुन॑ रि॒मम् ॅलो॒कम् ॅलो॒क मि॒मम् पुनः॒ पुन॑ रि॒मम् ॅलो॒कम् । \newline
57. इ॒मम् ॅलो॒कम् ॅलो॒क मि॒म मि॒मम् ॅलो॒कम् प्र॒त्यव॑रोहति प्र॒त्यव॑रोहति लो॒क मि॒म मि॒मम् ॅलो॒कम् प्र॒त्यव॑रोहति । \newline
58. लो॒कम् प्र॒त्यव॑रोहति प्र॒त्यव॑रोहति लो॒कम् ॅलो॒कम् प्र॒त्यव॑रोहति॒ सꣳ सम् प्र॒त्यव॑रोहति लो॒कम् ॅलो॒कम् प्र॒त्यव॑रोहति॒ सम् । \newline
59. प्र॒त्यव॑रोहति॒ सꣳ सम् प्र॒त्यव॑रोहति प्र॒त्यव॑रोहति॒ सम् ज्योति॑षा॒ ज्योति॑षा॒ सम् प्र॒त्यव॑रोहति प्र॒त्यव॑रोहति॒ सम् ज्योति॑षा । \newline
60. प्र॒त्यव॑रोह॒तीति॑ प्रति - अव॑रोहति । \newline
61. सम् ज्योति॑षा॒ ज्योति॑षा॒ सꣳ सम् ज्योति॑षा ऽभूव मभूव॒म् ज्योति॑षा॒ सꣳ सम् ज्योति॑षा ऽभूवम् । \newline
\pagebreak
\markright{ TS 1.7.6.3  \hfill https://www.vedavms.in \hfill}
\addcontentsline{toc}{section}{ TS 1.7.6.3 }
\section*{ TS 1.7.6.3 }

\textbf{TS 1.7.6.3 } \newline
\textbf{Samhita Paata} \newline

ज्योति॑षाऽभूव॒मित्या॑हा॒स्मिन्ने॒व लो॒के प्रति॑ तिष्ठत्यै॒न्द्री-मा॒वृत॑-म॒न्वाव॑र्त॒ इत्या॑हा॒सौ वा आ॑दि॒त्य इन्द्र॒स्तस्यै॒वाऽऽवृत॒मनु॑ प॒र्याव॑र्तते दक्षि॒णा प॒र्याव॑र्तते॒ स्वमे॒व वी॒र्य॑मनु॑ प॒र्याव॑र्तते॒ तस्मा॒द् दक्षि॒णोऽर्द्ध॑ आ॒त्मनो॑ वी॒र्या॑वत्त॒रोऽथो॑ आदि॒त्यस्यै॒वाऽऽवृत॒मनु॑ प॒र्याव॑र्तते॒ सम॒हं प्र॒जया॒ सं मया᳚ प्र॒जेत्या॑हा॒ऽऽशिष॑ - [ ] \newline

\textbf{Pada Paata} \newline

ज्योति॑षा । अ॒भू॒व॒म् । इति॑ । आ॒ह॒ । अ॒स्मिन्न् । ए॒व । लो॒के । प्रतीति॑ । ति॒ष्ठ॒ति॒ । ऐ॒न्द्रीम् । आ॒वृत॒मित्या᳚ - वृत᳚म् । अ॒न्वाव॑र्त॒ इत्य॑नु - आव॑र्ते । इति॑ । आ॒ह॒ । अ॒सौ । वै । आ॒दि॒त्यः । इन्द्रः॑ । तस्य॑ । ए॒व । आ॒वृत॒मित्या᳚ - वृत᳚म् । अन्विति॑ । प॒र्याव॑र्तत॒ इति॑ परि - आव॑र्तते । द॒क्षि॒णा । प॒र्याव॑र्तत॒ इति॑ परि - आव॑र्तते । स्वम् । ए॒व । वी॒र्य᳚म् । अन्विति॑ । प॒र्याव॑र्तत॒ इति॑ परि - आव॑र्तते । तस्मा᳚त् । दक्षि॑णः । अद्‌र्धः॑ । आ॒त्मनः॑ । वी॒र्या॑वत्तर॒ इति॑ वी॒र्या॑वत् - त॒रः॒ । अथो॒ इति॑ । आ॒दि॒त्यस्य॑ । ए॒व । आ॒वृत॒मित्या᳚ - वृत᳚म् । अन्विति॑ । प॒र्याव॑र्तत॒ इति॑ परि - आव॑र्तते । समिति॑ । अ॒हम् । प्र॒जयेति॑ प्र - जया᳚ । समिति॑ । मया᳚ । प्र॒जेति॑ प्र - जा । इति॑ । आ॒ह॒ । आ॒शिष॒मित्या᳚ - शिष᳚म् ।  \newline


\textbf{Krama Paata} \newline

ज्योति॑षाऽभूवम् । अ॒भू॒व॒मिति॑ । इत्या॑ह । आ॒हा॒स्मिन्न् । अ॒स्मिन्ने॒व । ए॒व लो॒के । लो॒के प्रति॑ । प्रति॑ तिष्ठति । ति॒ष्ठ॒त्यै॒न्द्रीम् । ऐ॒न्द्रीमा॒वृत᳚म् । आ॒वृत॑म॒न्वाव॑र्ते । आ॒वृत॒मित्या᳚ - वृत᳚म् । अ॒न्वाव॑र्त॒ इति॑ । अ॒न्वाव॑र्त॒ इत्य॑नु - आव॑र्ते । इत्या॑ह । आ॒हा॒सौ । अ॒सौ वै । वा आ॑दि॒त्यः । आ॒दि॒त्य इन्द्रः॑ । इन्द्र॒स्तस्य॑ । तस्यै॒व । ए॒वावृत᳚म् । आ॒वृत॒मनु॑ । आ॒वृत॒मित्या᳚ - वृत᳚म् । अनु॑ प॒र्याव॑र्तते । प॒र्याव॑र्तते दक्षि॒णा । प॒र्याव॑र्तत॒ इति॑ परि - आव॑र्तते । द॒क्षि॒णा प॒र्याव॑र्तते । प॒र्याव॑र्तते॒ स्वम् । प॒र्याव॑र्तत॒ इति॑ परि - आव॑र्तते । स्वमे॒व । ए॒व वी॒र्य᳚म् । वी॒र्य॑मनु॑ । अनु॑ प॒र्याव॑र्तते । प॒र्याव॑र्तते॒ तस्मा᳚त् । प॒र्याव॑र्तत॒ इति॑ परि - आव॑र्तते । तस्मा॒द् दक्षि॑णः । दक्षि॒णोऽर्द्धः॑ । अर्द्ध॑ आ॒त्मनः॑ । आ॒त्मनो॑ वी॒र्या॑वत्तरः । वी॒र्या॑वत्त॒रोऽथो᳚ । वी॒र्या॑वत्तर॒ इति॑ वी॒र्या॑वत् - त॒रः॒ । अथो॑ आदि॒त्यस्य॑ । अथो॒ इत्यथो᳚ । आ॒दि॒त्यस्यै॒व । ए॒वावृत᳚म् । आ॒वृत॒मनु॑ । आ॒वृत॒मित्या᳚ - वृत᳚म् । अनु॑ प॒र्याव॑र्तते । प॒र्याव॑र्तते॒ सम् । प॒र्याव॑र्तत॒ इति॑ परि - आव॑र्तते । सम॒हम् । अ॒हम् प्र॒जया᳚ । प्र॒जया॒ सम् । प्र॒जयेति॑ प्र - जया᳚ । सम् मया᳚ । मया᳚ प्र॒जा । प्र॒जेति॑ । प्र॒जेति॑ प्र - जा । इत्या॑ह । आ॒हा॒शिष᳚म् । आ॒शिष॑मे॒व । आ॒शिष॒मित्या᳚ - शिष᳚म् \newline

\textbf{Jatai Paata} \newline

1. ज्योति॑षा ऽभूव मभूव॒म् ज्योति॑षा॒ ज्योति॑षा ऽभूवम् । \newline
2. अ॒भू॒व॒ मिती त्य॑भूव मभूव॒ मिति॑ । \newline
3. इत्या॑ हा॒हे तीत्या॑ह । \newline
4. आ॒हा॒स्मिन् न॒स्मिन् ना॑ हाहा॒स्मिन्न् । \newline
5. अ॒स्मिन् ने॒वै वास्मिन् न॒स्मिन् ने॒व । \newline
6. ए॒व लो॒के लो॒क ए॒वैव लो॒के । \newline
7. लो॒के प्रति॒ प्रति॑ लो॒के लो॒के प्रति॑ । \newline
8. प्रति॑ तिष्ठति तिष्ठति॒ प्रति॒ प्रति॑ तिष्ठति । \newline
9. ति॒ष्ठ॒त्यै॒न्द्री मै॒न्द्रीम् ति॑ष्ठति तिष्ठत्यै॒न्द्रीम् । \newline
10. ऐ॒न्द्री मा॒वृत॑ मा॒वृत॑ मै॒न्द्री मै॒न्द्री मा॒वृत᳚म् । \newline
11. आ॒वृत॑ म॒न्वाव॑र्ते॒ ऽन्वाव॑र्त आ॒वृत॑ मा॒वृत॑ म॒न्वाव॑र्ते । \newline
12. आ॒वृत॒मित्या᳚ - वृत᳚म् । \newline
13. अ॒न्वाव॑र्त॒ इती त्य॒न्वाव॑र्ते॒ ऽन्वाव॑र्त॒ इति॑ । \newline
14. अ॒न्वाव॑र्त॒ इत्य॑नु - आव॑र्ते । \newline
15. इत्या॑ हा॒हे तीत्या॑ह । \newline
16. आ॒हा॒सा व॒सा वा॑हा हा॒सौ । \newline
17. अ॒सौ वै वा अ॒सा व॒सौ वै । \newline
18. वा आ॑दि॒त्य आ॑दि॒त्यो वै वा आ॑दि॒त्यः । \newline
19. आ॒दि॒त्य इन्द्र॒ इन्द्र॑ आदि॒त्य आ॑दि॒त्य इन्द्रः॑ । \newline
20. इन्द्र॒ स्तस्य॒ तस्ये न्द्र॒ इन्द्र॒ स्तस्य॑ । \newline
21. तस्यै॒ वैव तस्य॒ तस्यै॒व । \newline
22. ए॒वावृत॑ मा॒वृत॑ मे॒वै वावृत᳚म् । \newline
23. आ॒वृत॒ मन्वन्वा॒वृत॑ मा॒वृत॒ मनु॑ । \newline
24. आ॒वृत॒मित्या᳚ - वृत᳚म् । \newline
25. अनु॑ प॒र्याव॑र्तते प॒र्याव॑र्त॒ते ऽन्वनु॑ प॒र्याव॑र्तते । \newline
26. प॒र्याव॑र्तते दक्षि॒णा द॑क्षि॒णा प॒र्याव॑र्तते प॒र्याव॑र्तते दक्षि॒णा । \newline
27. प॒र्याव॑र्तत॒ इति॑ परि - आव॑र्तते । \newline
28. द॒क्षि॒णा प॒र्याव॑र्तते प॒र्याव॑र्तते दक्षि॒णा द॑क्षि॒णा प॒र्याव॑र्तते । \newline
29. प॒र्याव॑र्तते॒ स्वꣳ स्वम् प॒र्याव॑र्तते प॒र्याव॑र्तते॒ स्वम् । \newline
30. प॒र्याव॑र्तत॒ इति॑ परि - आव॑र्तते । \newline
31. स्व मे॒वैव स्वꣳ स्व मे॒व । \newline
32. ए॒व वी॒र्यं॑ ॅवी॒र्य॑ मे॒वैव वी॒र्य᳚म् । \newline
33. वी॒र्य॑ मन्वनु॑ वी॒र्यं॑ ॅवी॒र्य॑ मनु॑ । \newline
34. अनु॑ प॒र्याव॑र्तते प॒र्याव॑र्त॒ते ऽन्वनु॑ प॒र्याव॑र्तते । \newline
35. प॒र्याव॑र्तते॒ तस्मा॒त् तस्मा᳚त् प॒र्याव॑र्तते प॒र्याव॑र्तते॒ तस्मा᳚त् । \newline
36. प॒र्याव॑र्तत॒ इति॑ परि - आव॑र्तते । \newline
37. तस्मा॒द् दक्षि॑णो॒ दक्षि॑ण॒ स्तस्मा॒त् तस्मा॒द् दक्षि॑णः । \newline
38. दक्षि॒णो ऽर्द्धो ऽर्द्धो॒ दक्षि॑णो॒ दक्षि॒णो ऽर्द्धः॑ । \newline
39. अर्द्ध॑ आ॒त्मन॑ आ॒त्मनो ऽर्द्धो ऽर्द्ध॑ आ॒त्मनः॑ । \newline
40. आ॒त्मनो॑ वी॒र्या॑वत्तरो वी॒र्या॑वत्तर आ॒त्मन॑ आ॒त्मनो॑ वी॒र्या॑वत्तरः । \newline
41. वी॒र्या॑वत्त॒रो ऽथो॒ अथो॑ वी॒र्या॑वत्तरो वी॒र्या॑वत्त॒रो ऽथो᳚ । \newline
42. वी॒र्या॑वत्तर॒ इति॑ वी॒र्या॑वत् - त॒रः॒ । \newline
43. अथो॑ आदि॒त्यस्या॑ दि॒त्यस्याथो॒ अथो॑ आदि॒त्यस्य॑ । \newline
44. अथो॒ इत्यथो᳚ । \newline
45. आ॒दि॒त्य स्यै॒वै वादि॒त्य स्या॑दि॒त्य स्यै॒व । \newline
46. ए॒वावृत॑ मा॒वृत॑ मे॒वै वावृत᳚म् । \newline
47. आ॒वृत॒ मन्वन्वा॒वृत॑ मा॒वृत॒ मनु॑ । \newline
48. आ॒वृत॒मित्या᳚ - वृत᳚म् । \newline
49. अनु॑ प॒र्याव॑र्तते प॒र्याव॑र्त॒ते ऽन्वनु॑ प॒र्याव॑र्तते । \newline
50. प॒र्याव॑र्तते॒ सꣳ सम् प॒र्याव॑र्तते प॒र्याव॑र्तते॒ सम् । \newline
51. प॒र्याव॑र्तत॒ इति॑ परि - आव॑र्तते । \newline
52. स म॒ह म॒हꣳ सꣳ स म॒हम् । \newline
53. अ॒हम् प्र॒जया᳚ प्र॒जया॒ ऽह म॒हम् प्र॒जया᳚ । \newline
54. प्र॒जया॒ सꣳ सम् प्र॒जया᳚ प्र॒जया॒ सम् । \newline
55. प्र॒जयेति॑ प्र - जया᳚ । \newline
56. सम् मया॒ मया॒ सꣳ सम् मया᳚ । \newline
57. मया᳚ प्र॒जा प्र॒जा मया॒ मया᳚ प्र॒जा । \newline
58. प्र॒जेतीति॑ प्र॒जा प्र॒जेति॑ । \newline
59. प्र॒जेति॑ प्र - जा । \newline
60. इत्या॑ हा॒हे तीत्या॑ह । \newline
61. आ॒हा॒शिष॑ मा॒शिष॑ माहा हा॒शिष᳚म् । \newline
62. आ॒शिष॑ मे॒वै वाशिष॑ मा॒शिष॑ मे॒व । \newline
63. आ॒शिष॒मित्या᳚ - शिष᳚म् । \newline

\textbf{Ghana Paata } \newline

1. ज्योति॑षा ऽभूव मभूव॒म् ज्योति॑षा॒ ज्योति॑षा ऽभूव॒ मिती त्य॑भूव॒म् ज्योति॑षा॒ ज्योति॑षा ऽभूव॒ मिति॑ । \newline
2. अ॒भू॒व॒ मिती त्य॑भूव मभूव॒ मित्या॑हा॒हे त्य॑भूव मभूव॒ मित्या॑ह । \newline
3. इत्या॑हा॒हे ती त्या॑हा॒स्मिन् न॒स्मिन् ना॒हे ती त्या॑हा॒स्मिन्न् । \newline
4. आ॒हा॒स्मिन् न॒स्मिन् ना॑हाहा॒ स्मिन् ने॒वै वास्मिन् ना॑हा हा॒स्मिन् ने॒व । \newline
5. अ॒स्मिन् ने॒वै वास्मिन् न॒स्मिन् ने॒व लो॒के लो॒क ए॒वास्मिन् न॒स्मिन् ने॒व लो॒के । \newline
6. ए॒व लो॒के लो॒क ए॒वैव लो॒के प्रति॒ प्रति॑ लो॒क ए॒वैव लो॒के प्रति॑ । \newline
7. लो॒के प्रति॒ प्रति॑ लो॒के लो॒के प्रति॑ तिष्ठति तिष्ठति॒ प्रति॑ लो॒के लो॒के प्रति॑ तिष्ठति । \newline
8. प्रति॑ तिष्ठति तिष्ठति॒ प्रति॒ प्रति॑ तिष्ठ त्यै॒न्द्री मै॒न्द्रीम् ति॑ष्ठति॒ प्रति॒ प्रति॑ तिष्ठ त्यै॒न्द्रीम् । \newline
9. ति॒ष्ठ॒ त्यै॒न्द्री मै॒न्द्रीम् ति॑ष्ठति तिष्ठ त्यै॒न्द्री मा॒वृत॑ मा॒वृत॑ मै॒न्द्रीम् ति॑ष्ठति तिष्ठ त्यै॒न्द्री मा॒वृत᳚म् । \newline
10. ऐ॒न्द्री मा॒वृत॑ मा॒वृत॑ मै॒न्द्री मै॒न्द्री मा॒वृत॑ म॒न्वाव॑र्ते॒ ऽन्वाव॑र्त आ॒वृत॑ मै॒न्द्री मै॒न्द्री मा॒वृत॑ म॒न्वाव॑र्ते । \newline
11. आ॒वृत॑ म॒न्वाव॑र्ते॒ ऽन्वाव॑र्त आ॒वृत॑ मा॒वृत॑ म॒न्वाव॑र्त॒ इती त्य॒न्वाव॑र्त आ॒वृत॑ मा॒वृत॑ म॒न्वाव॑र्त॒ इति॑ । \newline
12. आ॒वृत॒मित्या᳚ - वृत᳚म् । \newline
13. अ॒न्वाव॑र्त॒ इती त्य॒न्वाव॑र्ते॒ ऽन्वाव॑र्त॒ इत्या॑हा॒हे त्य॒न्वाव॑र्ते॒ ऽन्वाव॑र्त॒ इत्या॑ह । \newline
14. अ॒न्वाव॑र्त॒ इत्य॑नु - आव॑र्ते । \newline
15. इत्या॑हा॒हे ती त्या॑हा॒सा व॒सा वा॒हे ती त्या॑हा॒सौ । \newline
16. आ॒हा॒सा व॒सा वा॑हा हा॒सौ वै वा अ॒सा वा॑हा हा॒सौ वै । \newline
17. अ॒सौ वै वा अ॒सा व॒सौ वा आ॑दि॒त्य आ॑दि॒त्यो वा अ॒सा व॒सौ वा आ॑दि॒त्यः । \newline
18. वा आ॑दि॒त्य आ॑दि॒त्यो वै वा आ॑दि॒त्य इन्द्र॒ इन्द्र॑ आदि॒त्यो वै वा आ॑दि॒त्य इन्द्रः॑ । \newline
19. आ॒दि॒त्य इन्द्र॒ इन्द्र॑ आदि॒त्य आ॑दि॒त्य इन्द्र॒ स्तस्य॒ तस्ये न्द्र॑ आदि॒त्य आ॑दि॒त्य इन्द्र॒ स्तस्य॑ । \newline
20. इन्द्र॒ स्तस्य॒ तस्ये न्द्र॒ इन्द्र॒ स्तस्यै॒ वैव तस्ये न्द्र॒ इन्द्र॒ स्तस्यै॒व । \newline
21. तस्यै॒ वैव तस्य॒ तस्यै॒ वावृत॑ मा॒वृत॑ मे॒व तस्य॒ तस्यै॒वा वृत᳚म् । \newline
22. ए॒वावृत॑ मा॒वृत॑ मे॒वै वावृत॒ मन्वन्वा॒वृत॑ मे॒वै वावृत॒ मनु॑ । \newline
23. आ॒वृत॒ मन्वन्वा॒वृत॑ मा॒वृत॒ मनु॑ प॒र्याव॑र्तते प॒र्याव॑र्त॒ते ऽन्वा॒वृत॑ मा॒वृत॒ मनु॑ प॒र्याव॑र्तते । \newline
24. आ॒वृत॒मित्या᳚ - वृत᳚म् । \newline
25. अनु॑ प॒र्याव॑र्तते प॒र्याव॑र्त॒ते ऽन्वनु॑ प॒र्याव॑र्तते दक्षि॒णा द॑क्षि॒णा प॒र्याव॑र्त॒ते ऽन्वनु॑ प॒र्याव॑र्तते दक्षि॒णा । \newline
26. प॒र्याव॑र्तते दक्षि॒णा द॑क्षि॒णा प॒र्याव॑र्तते प॒र्याव॑र्तते दक्षि॒णा प॒र्याव॑र्तते प॒र्याव॑र्तते दक्षि॒णा प॒र्याव॑र्तते प॒र्याव॑र्तते दक्षि॒णा प॒र्याव॑र्तते । \newline
27. प॒र्याव॑र्तत॒ इति॑ परि - आव॑र्तते । \newline
28. द॒क्षि॒णा प॒र्याव॑र्तते प॒र्याव॑र्तते दक्षि॒णा द॑क्षि॒णा प॒र्याव॑र्तते॒ स्वꣳ स्वम् प॒र्याव॑र्तते दक्षि॒णा द॑क्षि॒णा प॒र्याव॑र्तते॒ स्वम् । \newline
29. प॒र्याव॑र्तते॒ स्वꣳ स्वम् प॒र्याव॑र्तते प॒र्याव॑र्तते॒ स्व मे॒वैव स्वम् प॒र्याव॑र्तते प॒र्याव॑र्तते॒ स्व मे॒व । \newline
30. प॒र्याव॑र्तत॒ इति॑ परि - आव॑र्तते । \newline
31. स्व मे॒वैव स्वꣳ स्व मे॒व वी॒र्यं॑ ॅवी॒र्य॑ मे॒व स्वꣳ स्व मे॒व वी॒र्य᳚म् । \newline
32. ए॒व वी॒र्यं॑ ॅवी॒र्य॑ मे॒वैव वी॒र्य॑ मन्वनु॑ वी॒र्य॑ मे॒वैव वी॒र्य॑ मनु॑ । \newline
33. वी॒र्य॑ मन्वनु॑ वी॒र्यं॑ ॅवी॒र्य॑ मनु॑ प॒र्याव॑र्तते प॒र्याव॑र्त॒ते ऽनु॑ वी॒र्यं॑ ॅवी॒र्य॑ मनु॑ प॒र्याव॑र्तते । \newline
34. अनु॑ प॒र्याव॑र्तते प॒र्याव॑र्त॒ते ऽन्वनु॑ प॒र्याव॑र्तते॒ तस्मा॒त् तस्मा᳚त् प॒र्याव॑र्त॒ते ऽन्वनु॑ प॒र्याव॑र्तते॒ तस्मा᳚त् । \newline
35. प॒र्याव॑र्तते॒ तस्मा॒त् तस्मा᳚त् प॒र्याव॑र्तते प॒र्याव॑र्तते॒ तस्मा॒द् दक्षि॑णो॒ दक्षि॑ण॒ स्तस्मा᳚त् प॒र्याव॑र्तते प॒र्याव॑र्तते॒ तस्मा॒द् दक्षि॑णः । \newline
36. प॒र्याव॑र्तत॒ इति॑ परि - आव॑र्तते । \newline
37. तस्मा॒द् दक्षि॑णो॒ दक्षि॑ण॒ स्तस्मा॒त् तस्मा॒द् दक्षि॒णो ऽर्द्धो ऽर्द्धो॒ दक्षि॑ण॒ स्तस्मा॒त् तस्मा॒द् दक्षि॒णो ऽर्द्धः॑ । \newline
38. दक्षि॒णो ऽर्द्धो ऽर्द्धो॒ दक्षि॑णो॒ दक्षि॒णो ऽर्द्ध॑ आ॒त्मन॑ आ॒त्मनो ऽर्द्धो॒ दक्षि॑णो॒ दक्षि॒णो ऽर्द्ध॑ आ॒त्मनः॑ । \newline
39. अर्द्ध॑ आ॒त्मन॑ आ॒त्मनो ऽर्द्धो ऽर्द्ध॑ आ॒त्मनो॑ वी॒र्या॑वत्तरो वी॒र्या॑वत्तर आ॒त्मनो ऽर्द्धो ऽर्द्ध॑ आ॒त्मनो॑ वी॒र्या॑वत्तरः । \newline
40. आ॒त्मनो॑ वी॒र्या॑वत्तरो वी॒र्या॑वत्तर आ॒त्मन॑ आ॒त्मनो॑ वी॒र्या॑वत्त॒रो ऽथो॒ अथो॑ वी॒र्या॑वत्तर आ॒त्मन॑ आ॒त्मनो॑ वी॒र्या॑वत्त॒रो ऽथो᳚ । \newline
41. वी॒र्या॑वत्त॒रो ऽथो॒ अथो॑ वी॒र्या॑वत्तरो वी॒र्या॑वत्त॒रो ऽथो॑ आदि॒त्यस्या॑ दि॒त्यस्याथो॑ वी॒र्या॑वत्तरो वी॒र्या॑वत्त॒रो ऽथो॑ आदि॒त्यस्य॑ । \newline
42. वी॒र्या॑वत्तर॒ इति॑ वी॒र्या॑वत् - त॒रः॒ । \newline
43. अथो॑ आदि॒त्यस्या॑ दि॒त्यस्याथो॒ अथो॑ आदि॒त्य स्यै॒ वैवादि॒ त्यस्याथो॒ अथो॑ आदि॒त्य स्यै॒व । \newline
44. अथो॒ इत्यथो᳚ । \newline
45. आ॒दि॒त्य स्यै॒वै वादि॒त्य स्या॑दि॒त्यस्यै॒ वावृत॑ मा॒वृत॑ मे॒वादि॒त्य स्या॑दि॒त्यस्यै॒ वावृत᳚म् । \newline
46. ए॒वावृत॑ मा॒वृत॑ मे॒वै वावृत॒ मन्वन्वा॒वृत॑ मे॒वै वावृत॒ मनु॑ । \newline
47. आ॒वृत॒ मन्वन्वा॒वृत॑ मा॒वृत॒ मनु॑ प॒र्याव॑र्तते प॒र्याव॑र्त॒ते ऽन्वा॒वृत॑ मा॒वृत॒ मनु॑ प॒र्याव॑र्तते । \newline
48. आ॒वृत॒मित्या᳚ - वृत᳚म् । \newline
49. अनु॑ प॒र्याव॑र्तते प॒र्याव॑र्त॒ते ऽन्वनु॑ प॒र्याव॑र्तते॒ सꣳ सम् प॒र्याव॑र्त॒ते ऽन्वनु॑ प॒र्याव॑र्तते॒ सम् । \newline
50. प॒र्याव॑र्तते॒ सꣳ सम् प॒र्याव॑र्तते प॒र्याव॑र्तते॒ स म॒ह म॒हꣳ सम् प॒र्याव॑र्तते प॒र्याव॑र्तते॒ स म॒हम् । \newline
51. प॒र्याव॑र्तत॒ इति॑ परि - आव॑र्तते । \newline
52. स म॒ह म॒हꣳ सꣳ स म॒हम् प्र॒जया᳚ प्र॒जया॒ ऽहꣳ सꣳ स म॒हम् प्र॒जया᳚ । \newline
53. अ॒हम् प्र॒जया᳚ प्र॒जया॒ ऽह म॒हम् प्र॒जया॒ सꣳ सम् प्र॒जया॒ ऽह म॒हम् प्र॒जया॒ सम् । \newline
54. प्र॒जया॒ सꣳ सम् प्र॒जया᳚ प्र॒जया॒ सम् मया॒ मया॒ सम् प्र॒जया᳚ प्र॒जया॒ सम् मया᳚ । \newline
55. प्र॒जयेति॑ प्र - जया᳚ । \newline
56. सम् मया॒ मया॒ सꣳ सम् मया᳚ प्र॒जा प्र॒जा मया॒ सꣳ सम् मया᳚ प्र॒जा । \newline
57. मया᳚ प्र॒जा प्र॒जा मया॒ मया᳚ प्र॒जेतीति॑ प्र॒जा मया॒ मया᳚ प्र॒जेति॑ । \newline
58. प्र॒जेतीति॑ प्र॒जा प्र॒जे त्या॑हा॒हे ति॑ प्र॒जा प्र॒जेत्या॑ह । \newline
59. प्र॒जेति॑ प्र - जा । \newline
60. इत्या॑हा॒हे तीत्या॑हा॒ शिष॑ मा॒शिष॑ मा॒हे तीत्या॑ हा॒शिष᳚म् । \newline
61. आ॒हा॒शिष॑ मा॒शिष॑ माहाहा॒ शिष॑ मे॒वै वाशिष॑ माहा हा॒शिष॑ मे॒व । \newline
62. आ॒शिष॑ मे॒वै वाशिष॑ मा॒शिष॑ मे॒वैता मे॒ता मे॒वाशिष॑ मा॒शिष॑ मे॒वैताम् । \newline
63. आ॒शिष॒मित्या᳚ - शिष᳚म् । \newline
\pagebreak
\markright{ TS 1.7.6.4  \hfill https://www.vedavms.in \hfill}
\addcontentsline{toc}{section}{ TS 1.7.6.4 }
\section*{ TS 1.7.6.4 }

\textbf{TS 1.7.6.4 } \newline
\textbf{Samhita Paata} \newline

मे॒वैतामा शा᳚स्ते॒ समि॑द्धो अग्ने मे दीदिहि समे॒द्धा ते॑ अग्ने दीद्यास॒मित्या॑ह यथाय॒जु-रे॒वैतद्वसु॑मान्. य॒ज्ञो वसी॑यान् भूयास॒-मित्या॑हा॒-ऽऽशिष॑मे॒वेतामा शा᳚स्ते ब॒हु वै गार्.ह॑पत्य॒स्यान्ते॑ मि॒श्रमि॑व चर्यत आग्निपावमा॒नीभ्यां॒ गार्.ह॑पत्य॒मुप॑ तिष्ठते पु॒नात्ये॒वाग्निं पु॑नी॒त आ॒त्मानं॒ द्वाभ्यां॒ प्रति॑ष्ठित्या॒ अग्ने॑ गृहपत॒ इत्या॑ह - [ ] \newline

\textbf{Pada Paata} \newline

ए॒व । ए॒ताम् । एति॑ । शा॒स्ते॒ । समि॑द्ध॒ इति॒ सं - इ॒द्धः॒ । अ॒ग्ने॒ । मे॒ । दी॒दि॒हि॒ । स॒मे॒द्धेति॑ सं - ए॒द्धा । ते॒ । अ॒ग्ने॒ । दी॒द्या॒स॒म् । इति॑ । आ॒ह॒ । य॒था॒य॒जुरिति॑ यथा - य॒जुः । ए॒व । ए॒तत् । वसु॑मा॒निति॒ वसु॑ - मा॒न् । य॒ज्ञ्ः । वसी॑यान् । भू॒या॒स॒म् । इति॑ । आ॒ह॒ । आ॒शिष॒मित्या᳚-शिष᳚म् । ए॒व । ए॒ताम् । एति॑ । शा॒स्ते॒ । ब॒हु । वै । गार्.ह॑पत्य॒स्येति॒ गार्.ह॑ - प॒त्य॒स्य॒ । अन्ते᳚ । मि॒श्रम् । इ॒व॒ । च॒र्य॒ते॒ । आ॒ग्नि॒पा॒व॒मा॒नीभ्या॒मित्या᳚ग्नि - पा॒व॒मा॒नीभ्या᳚म् । गार्.ह॑पत्य॒मिति॒ गार्.ह॑ - प॒त्य॒म् । उपेति॑ । ति॒ष्ठ॒ते॒ । पु॒नाति॑ । ए॒व । अ॒ग्निम् । पु॒नी॒ते । आ॒त्मान᳚म् । द्वाभ्या᳚म् । प्रति॑ष्ठित्या॒ इति॒ प्रति॑ - स्थि॒त्यै॒ । अग्ने᳚ । गृ॒ह॒प॒त॒ इति॑ गृह - प॒ते॒ । इति॑ । आ॒ह॒ ।  \newline


\textbf{Krama Paata} \newline

ए॒वैताम् । ए॒तामा । आ शा᳚स्ते । शा॒स्ते॒ समि॑द्धः । समि॑द्धो अग्ने । समि॑द्ध॒ इति॒ सम् - इ॒द्धः॒ । अ॒ग्ने॒ मे॒ । मे॒ दी॒दि॒हि॒ । दी॒दि॒हि॒ स॒मे॒द्धा । स॒मे॒द्धा ते᳚ । स॒मे॒द्धेति॑ सम् - ए॒द्धा । ते॒ अ॒ग्ने॒ । अ॒ग्ने॒ दी॒द्या॒स॒म् । दी॒द्या॒स॒मिति॑ । इत्या॑ह । आ॒ह॒ य॒था॒य॒जुः । य॒था॒य॒जुरे॒व । य॒था॒य॒जुरिति॑ यथा - य॒जुः । ए॒वैतत् । ए॒तद् वसु॑मान् । वसु॑मान्. य॒ज्ञ्ः । वसु॑मा॒निति॒ वसु॑ - मा॒न्॒ । य॒ज्ञो वसी॑यान् । वसी॑यान्,भूयासम् । भू॒या॒स॒मिति॑ । इत्या॑ह । आ॒हा॒शिष᳚म् । आ॒शिष॑मे॒व । आ॒शिष॒मित्या᳚ - शिष᳚म् । ए॒वैताम् । ए॒तामा । आ शा᳚स्ते । शा॒स्ते॒ ब॒हु । ब॒हु वै । वै गार्.ह॑पत्यस्य । गार्.ह॑पत्य॒स्यान्ते᳚ । गार्.ह॑पत्य॒स्येति॒ गार्.ह॑ - प॒त्य॒स्य॒ । अन्ते॑ मि॒श्रम् । मि॒श्रमि॑व । इ॒व॒ च॒र्य॒ते॒ । च॒र्य॒त॒ आ॒ग्नि॒पा॒व॒मा॒नीभ्या᳚म् । आ॒ग्नि॒पा॒व॒मा॒नीभ्या॒म् गार्.ह॑पत्यम् । आ॒ग्नि॒पा॒व॒मा॒नीभ्या॒मित्या᳚ग्नि - पा॒व॒मा॒नीभ्या᳚म् । गार्.ह॑पत्य॒मुप॑ । गार्.ह॑पत्य॒मिति॒ गार्.ह॑ - प॒त्य॒म् । उप॑ तिष्ठते । ति॒ष्ठ॒ते॒ पु॒नाति॑ । पु॒नात्ये॒व । ए॒वाग्निम् । अ॒ग्निम् पु॑नी॒ते । पु॒नी॒त आ॒त्मान᳚म् । आ॒त्मान॒म् द्वाभ्या᳚म् । द्वाभ्या॒म् प्रति॑ष्ठित्यै । प्रति॑ष्ठित्या॒ अग्ने᳚ । प्रति॑ष्ठित्या॒ इति॒ प्रति॑ - स्थि॒त्यै॒ । अग्ने॑ गृहपते । गृ॒ह॒प॒त॒ इति॑ । गृ॒ह॒प॒त॒ इति॑ गृह - प॒ते॒ । इत्या॑ह । आ॒ह॒ य॒था॒य॒जुः \newline

\textbf{Jatai Paata} \newline

1. ए॒वैता मे॒ता मे॒वै वैताम् । \newline
2. ए॒ता मैता मे॒ता मा । \newline
3. आ शा᳚स्ते शास्त॒ आ शा᳚स्ते । \newline
4. शा॒स्ते॒ समि॑द्धः॒ समि॑द्धः शास्ते शास्ते॒ समि॑द्धः । \newline
5. समि॑द्धो अग्ने ऽग्ने॒ समि॑द्धः॒ समि॑द्धो अग्ने । \newline
6. समि॑द्ध॒ इति॒ सं - इ॒द्धः॒ । \newline
7. अ॒ग्ने॒ मे॒ मे॒ अ॒ग्ने॒ ऽग्ने॒ मे॒ । \newline
8. मे॒ दी॒दि॒हि॒ दी॒दि॒हि॒ मे॒ मे॒ दी॒दि॒हि॒ । \newline
9. दी॒दि॒हि॒ स॒मे॒द्धा स॑मे॒द्धा दी॑दिहि दीदिहि समे॒द्धा । \newline
10. स॒मे॒द्धा ते॑ ते समे॒द्धा स॑मे॒द्धा ते᳚ । \newline
11. स॒मे॒द्धेति॑ सं - ए॒द्धा । \newline
12. ते॒ अ॒ग्ने॒ ऽग्ने॒ ते॒ ते॒ अ॒ग्ने॒ । \newline
13. अ॒ग्ने॒ दी॒द्या॒स॒म् दी॒द्या॒स॒ म॒ग्ने॒ ऽग्ने॒ दी॒द्या॒स॒म् । \newline
14. दी॒द्या॒स॒ मितीति॑ दीद्यासम् दीद्यास॒ मिति॑ । \newline
15. इत्या॑ हा॒हे तीत्या॑ह । \newline
16. आ॒ह॒ य॒था॒य॒जुर् य॑थाय॒जु रा॑हाह यथाय॒जुः । \newline
17. य॒था॒य॒जु रे॒वैव य॑थाय॒जुर् य॑थाय॒जु रे॒व । \newline
18. य॒था॒य॒जुरिति॑ यथा - य॒जुः । \newline
19. ए॒वैत दे॒त दे॒वै वैतत् । \newline
20. ए॒तद् वसु॑मा॒न्॒. वसु॑मा ने॒तदे॒तद् वसु॑मान् । \newline
21. वसु॑मान्. य॒ज्ञो य॒ज्ञो वसु॑मा॒न्॒. वसु॑मान्. य॒ज्ञ्ः । \newline
22. वसु॑मा॒निति॒ वसु॑ - मा॒न् । \newline
23. य॒ज्ञो वसी॑या॒न्॒. वसी॑यान्. य॒ज्ञो य॒ज्ञो वसी॑यान् । \newline
24. वसी॑यान् भूयासम् भूयासं॒ ॅवसी॑या॒न्॒. वसी॑यान् भूयासम् । \newline
25. भू॒या॒स॒ मितीति॑ भूयासम् भूयास॒ मिति॑ । \newline
26. इत्या॑ हा॒हे तीत्या॑ह । \newline
27. आ॒हा॒शिष॑ मा॒शिष॑ माहा हा॒शिष᳚म् । \newline
28. आ॒शिष॑ मे॒वैवाशिष॑ मा॒शिष॑ मे॒व । \newline
29. आ॒शिष॒मित्या᳚ - शिष᳚म् । \newline
30. ए॒वैता मे॒ता मे॒वै वैताम् । \newline
31. ए॒ता मैता मे॒ता मा । \newline
32. आ शा᳚स्ते शास्त॒ आ शा᳚स्ते । \newline
33. शा॒स्ते॒ ब॒हु ब॒हु शा᳚स्ते शास्ते ब॒हु । \newline
34. ब॒हु वै वै ब॒हु ब॒हु वै । \newline
35. वै गार्.ह॑पत्यस्य॒ गार्.ह॑पत्यस्य॒ वै वै गार्.ह॑पत्यस्य । \newline
36. गार्.ह॑पत्य॒स्यान्ते ऽन्ते॒ गार्.ह॑पत्यस्य॒ गार्.ह॑पत्य॒स्यान्ते᳚ । \newline
37. गार्.ह॑पत्य॒स्येति॒ गार्.ह॑ - प॒त्य॒स्य॒ । \newline
38. अन्ते॑ मि॒श्रम् मि॒श्र मन्ते ऽन्ते॑ मि॒श्रम् । \newline
39. मि॒श्र मि॑वे व मि॒श्रम् मि॒श्र मि॑व । \newline
40. इ॒व॒ च॒र्य॒ते॒ च॒र्य॒त॒ इ॒वे॒ व॒ च॒र्य॒ते॒ । \newline
41. च॒र्य॒त॒ आ॒ग्नि॒पा॒व॒मा॒नीभ्या॑ माग्निपावमा॒नीभ्या᳚म् चर्यते चर्यत आग्निपावमा॒नीभ्या᳚म् । \newline
42. आ॒ग्नि॒पा॒व॒मा॒नीभ्या॒म् गार्.ह॑पत्य॒म् गार्.ह॑पत्य माग्निपावमा॒नीभ्या॑ माग्निपावमा॒नीभ्या॒म् गार्.ह॑पत्यम् । \newline
43. आ॒ग्नि॒पा॒व॒मा॒नीभ्या॒मित्या᳚ग्नि - पा॒व॒मा॒नीभ्या᳚म् । \newline
44. गार्.ह॑पत्य॒ मुपोप॒ गार्.ह॑पत्य॒म् गार्.ह॑पत्य॒ मुप॑ । \newline
45. गार्.ह॑पत्य॒मिति॒ गार्.ह॑ - प॒त्य॒म् । \newline
46. उप॑ तिष्ठते तिष्ठत॒ उपोप॑ तिष्ठते । \newline
47. ति॒ष्ठ॒ते॒ पु॒नाति॑ पु॒नाति॑ तिष्ठते तिष्ठते पु॒नाति॑ । \newline
48. पु॒ना त्ये॒वैव पु॒नाति॑ पु॒ना त्ये॒व । \newline
49. ए॒वाग्नि म॒ग्नि मे॒वै वाग्निम् । \newline
50. अ॒ग्निम् पु॑नी॒ते पु॑नी॒ते᳚ ऽग्नि म॒ग्निम् पु॑नी॒ते । \newline
51. पु॒नी॒त आ॒त्मान॑ मा॒त्मान॑म् पुनी॒ते पु॑नी॒त आ॒त्मान᳚म् । \newline
52. आ॒त्मान॒म् द्वाभ्या॒म् द्वाभ्या॑ मा॒त्मान॑ मा॒त्मान॒म् द्वाभ्या᳚म् । \newline
53. द्वाभ्या॒म् प्रति॑ष्ठित्यै॒ प्रति॑ष्ठित्यै॒ द्वाभ्या॒म् द्वाभ्या॒म् प्रति॑ष्ठित्यै । \newline
54. प्रति॑ष्ठित्या॒ अग्ने ऽग्ने॒ प्रति॑ष्ठित्यै॒ प्रति॑ष्ठित्या॒ अग्ने᳚ । \newline
55. प्रति॑ष्ठित्या॒ इति॒ प्रति॑ - स्थि॒त्यै॒ । \newline
56. अग्ने॑ गृहपते गृहप॒ते ऽग्ने ऽग्ने॑ गृहपते । \newline
57. गृ॒ह॒प॒त॒ इतीति॑ गृहपते गृहपत॒ इति॑ । \newline
58. गृ॒ह॒प॒त॒ इति॑ गृह - प॒ते॒ । \newline
59. इत्या॑ हा॒हे तीत्या॑ह । \newline
60. आ॒ह॒ य॒था॒य॒जुर् य॑थाय॒जु रा॑हाह यथाय॒जुः । \newline

\textbf{Ghana Paata } \newline

1. ए॒वैता मे॒ता मे॒वै वैता मैता मे॒वै वैता मा । \newline
2. ए॒ता मैता मे॒ता मा शा᳚स्ते शास्त॒ ऐता मे॒ता मा शा᳚स्ते । \newline
3. आ शा᳚स्ते शास्त॒ आ शा᳚स्ते॒ समि॑द्धः॒ समि॑द्धः शास्त॒ आ शा᳚स्ते॒ समि॑द्धः । \newline
4. शा॒स्ते॒ समि॑द्धः॒ समि॑द्धः शास्ते शास्ते॒ समि॑द्धो अग्ने ऽग्ने॒ समि॑द्धः शास्ते शास्ते॒ समि॑द्धो अग्ने । \newline
5. समि॑द्धो अग्ने ऽग्ने॒ समि॑द्धः॒ समि॑द्धो अग्ने मे मे अग्ने॒ समि॑द्धः॒ समि॑द्धो अग्ने मे । \newline
6. समि॑द्ध॒ इति॒ सं - इ॒द्धः॒ । \newline
7. अ॒ग्ने॒ मे॒ मे॒ अ॒ग्ने॒ ऽग्ने॒ मे॒ दी॒दि॒हि॒ दी॒दि॒हि॒ मे॒ अ॒ग्ने॒ ऽग्ने॒ मे॒ दी॒दि॒हि॒ । \newline
8. मे॒ दी॒दि॒हि॒ दी॒दि॒हि॒ मे॒ मे॒ दी॒दि॒हि॒ स॒मे॒द्धा स॑मे॒द्धा दी॑दिहि मे मे दीदिहि समे॒द्धा । \newline
9. दी॒दि॒हि॒ स॒मे॒द्धा स॑मे॒द्धा दी॑दिहि दीदिहि समे॒द्धा ते॑ ते समे॒द्धा दी॑दिहि दीदिहि समे॒द्धा ते᳚ । \newline
10. स॒मे॒द्धा ते॑ ते समे॒द्धा स॑मे॒द्धा ते॑ अग्ने ऽग्ने ते समे॒द्धा स॑मे॒द्धा ते॑ अग्ने । \newline
11. स॒मे॒द्धेति॑ सं - ए॒द्धा । \newline
12. ते॒ अ॒ग्ने॒ ऽग्ने॒ ते॒ ते॒ अ॒ग्ने॒ दी॒द्या॒स॒म् दी॒द्या॒स॒ म॒ग्ने॒ ते॒ ते॒ अ॒ग्ने॒ दी॒द्या॒स॒म् । \newline
13. अ॒ग्ने॒ दी॒द्या॒स॒म् दी॒द्या॒स॒ म॒ग्ने॒ ऽग्ने॒ दी॒द्या॒स॒ मितीति॑ दीद्यास मग्ने ऽग्ने दीद्यास॒ मिति॑ । \newline
14. दी॒द्या॒स॒ मितीति॑ दीद्यासम् दीद्यास॒ मित्या॑हा॒हे ति॑ दीद्यासम् दीद्यास॒ मित्या॑ह । \newline
15. इत्या॑हा॒हे तीत्या॑ह यथाय॒जुर् य॑थाय॒जु रा॒हे तीत्या॑ह यथाय॒जुः । \newline
16. आ॒ह॒ य॒था॒य॒जुर् य॑थाय॒जु रा॑हाह यथाय॒जु रे॒वैव य॑थाय॒जु रा॑हाह यथाय॒जु रे॒व । \newline
17. य॒था॒य॒जु रे॒वैव य॑थाय॒जुर् य॑थाय॒जु रे॒वैत दे॒त दे॒व य॑थाय॒जुर् य॑थाय॒जु रे॒वैतत् । \newline
18. य॒था॒य॒जुरिति॑ यथा - य॒जुः । \newline
19. ए॒वैत दे॒त दे॒वैवैतद् वसु॑मा॒न्॒. वसु॑मा ने॒त दे॒वैवैतद् वसु॑मान् । \newline
20. ए॒तद् वसु॑मा॒न्॒. वसु॑मा ने॒तदे॒तद् वसु॑मान्. य॒ज्ञो य॒ज्ञो वसु॑मा ने॒तदे॒तद् वसु॑मान्. य॒ज्ञ्ः । \newline
21. वसु॑मान्. य॒ज्ञो य॒ज्ञो वसु॑मा॒न्॒. वसु॑मान्. य॒ज्ञो वसी॑या॒न्॒. वसी॑यान्. य॒ज्ञो वसु॑मा॒न्॒. वसु॑मान्. य॒ज्ञो वसी॑यान् । \newline
22. वसु॑मा॒निति॒ वसु॑ - मा॒न् । \newline
23. य॒ज्ञो वसी॑या॒न्॒. वसी॑यान्. य॒ज्ञो य॒ज्ञो वसी॑यान् भूयासम् भूयासं॒ ॅवसी॑यान्. य॒ज्ञो य॒ज्ञो वसी॑यान् भूयासम् । \newline
24. वसी॑यान् भूयासम् भूयासं॒ ॅवसी॑या॒न्॒. वसी॑यान् भूयास॒ मितीति॑ भूयासं॒ ॅवसी॑या॒न्॒. वसी॑यान् भूयास॒ मिति॑ । \newline
25. भू॒या॒स॒ मितीति॑ भूयासम् भूयास॒ मित्या॑हा॒हे ति॑ भूयासम् भूयास॒ मित्या॑ह । \newline
26. इत्या॑हा॒हे तीत्या॑हा॒ शिष॑ मा॒शिष॑ मा॒हे तीत्या॑ हा॒शिष᳚म् । \newline
27. आ॒हा॒शिष॑ मा॒शिष॑ माहा हा॒शिष॑ मे॒वै वाशिष॑ माहा हा॒शिष॑ मे॒व । \newline
28. आ॒शिष॑ मे॒वै वाशिष॑ मा॒शिष॑ मे॒वैता मे॒ता मे॒ वाशिष॑ मा॒शिष॑ मे॒वैताम् । \newline
29. आ॒शिष॒मित्या᳚ - शिष᳚म् । \newline
30. ए॒वैता मे॒ता मे॒वै वैता मैता मे॒वै वैता मा । \newline
31. ए॒ता मैता मे॒ता मा शा᳚स्ते शास्त॒ ऐता मे॒ता मा शा᳚स्ते । \newline
32. आ शा᳚स्ते शास्त॒ आ शा᳚स्ते ब॒हु ब॒हु शा᳚स्त॒ आ शा᳚स्ते ब॒हु । \newline
33. शा॒स्ते॒ ब॒हु ब॒हु शा᳚स्ते शास्ते ब॒हु वै वै ब॒हु शा᳚स्ते शास्ते ब॒हु वै । \newline
34. ब॒हु वै वै ब॒हु ब॒हु वै गार्.ह॑पत्यस्य॒ गार्.ह॑पत्यस्य॒ वै ब॒हु ब॒हु वै गार्.ह॑पत्यस्य । \newline
35. वै गार्.ह॑पत्यस्य॒ गार्.ह॑पत्यस्य॒ वै वै गार्.ह॑पत्य॒स्यान्ते ऽन्ते॒ गार्.ह॑पत्यस्य॒ वै वै गार्.ह॑पत्य॒स्यान्ते᳚ । \newline
36. गार्.ह॑पत्य॒स्यान्ते ऽन्ते॒ गार्.ह॑पत्यस्य॒ गार्.ह॑पत्य॒स्यान्ते॑ मि॒श्रम् मि॒श्र मन्ते॒ गार्.ह॑पत्यस्य॒ गार्.ह॑पत्य॒स्यान्ते॑ मि॒श्रम् । \newline
37. गार्.ह॑पत्य॒स्येति॒ गार्.ह॑ - प॒त्य॒स्य॒ । \newline
38. अन्ते॑ मि॒श्रम् मि॒श्र मन्ते ऽन्ते॑ मि॒श्र मि॑वे व मि॒श्र मन्ते ऽन्ते॑ मि॒श्र मि॑व । \newline
39. मि॒श्र मि॑वे व मि॒श्रम् मि॒श्र मि॑व चर्यते चर्यत इव मि॒श्रम् मि॒श्र मि॑व चर्यते । \newline
40. इ॒व॒ च॒र्य॒ते॒ च॒र्य॒त॒ इ॒वे॒ व॒ च॒र्य॒त॒ आ॒ग्नि॒पा॒व॒मा॒नीभ्या॑ माग्निपावमा॒नीभ्या᳚म् चर्यत इवे व चर्यत आग्निपावमा॒नीभ्या᳚म् । \newline
41. च॒र्य॒त॒ आ॒ग्नि॒पा॒व॒मा॒नीभ्या॑ माग्निपावमा॒नीभ्या᳚म् चर्यते चर्यत आग्निपावमा॒नीभ्या॒म् गार्.ह॑पत्य॒म् गार्.ह॑पत्य माग्निपावमा॒नीभ्या᳚म् चर्यते चर्यत आग्निपावमा॒नीभ्या॒म् गार्.ह॑पत्यम् । \newline
42. आ॒ग्नि॒पा॒व॒मा॒नीभ्या॒म् गार्.ह॑पत्य॒म् गार्.ह॑पत्य माग्निपावमा॒नीभ्या॑ माग्निपावमा॒नीभ्या॒म् गार्.ह॑पत्य॒ मुपोप॒ गार्.ह॑पत्य माग्निपावमा॒नीभ्या॑ माग्निपावमा॒नीभ्या॒म् गार्.ह॑पत्य॒ मुप॑ । \newline
43. आ॒ग्नि॒पा॒व॒मा॒नीभ्या॒मित्या᳚ग्नि - पा॒व॒मा॒नीभ्या᳚म् । \newline
44. गार्.ह॑पत्य॒ मुपोप॒ गार्.ह॑पत्य॒म् गार्.ह॑पत्य॒ मुप॑ तिष्ठते तिष्ठत॒ उप॒ गार्.ह॑पत्य॒म् गार्.ह॑पत्य॒ मुप॑ तिष्ठते । \newline
45. गार्.ह॑पत्य॒मिति॒ गार्.ह॑ - प॒त्य॒म् । \newline
46. उप॑ तिष्ठते तिष्ठत॒ उपोप॑ तिष्ठते पु॒नाति॑ पु॒नाति॑ तिष्ठत॒ उपोप॑ तिष्ठते पु॒नाति॑ । \newline
47. ति॒ष्ठ॒ते॒ पु॒नाति॑ पु॒नाति॑ तिष्ठते तिष्ठते पु॒नात्ये॒वैव पु॒नाति॑ तिष्ठते तिष्ठते पु॒नात्ये॒व । \newline
48. पु॒नात्ये॒वैव पु॒नाति॑ पु॒नात्ये॒ वाग्नि म॒ग्नि मे॒व पु॒नाति॑ पु॒ना त्ये॒वाग्निम् । \newline
49. ए॒वाग्नि म॒ग्नि मे॒वै वाग्निम् पु॑नी॒ते पु॑नी॒ते᳚ ऽग्नि मे॒वै वाग्निम् पु॑नी॒ते । \newline
50. अ॒ग्निम् पु॑नी॒ते पु॑नी॒ते᳚ ऽग्नि म॒ग्निम् पु॑नी॒त आ॒त्मान॑ मा॒त्मान॑म् पुनी॒ते᳚ ऽग्नि म॒ग्निम् पु॑नी॒त आ॒त्मान᳚म् । \newline
51. पु॒नी॒त आ॒त्मान॑ मा॒त्मान॑म् पुनी॒ते पु॑नी॒त आ॒त्मान॒म् द्वाभ्या॒म् द्वाभ्या॑ मा॒त्मान॑म् पुनी॒ते पु॑नी॒त आ॒त्मान॒म् द्वाभ्या᳚म् । \newline
52. आ॒त्मान॒म् द्वाभ्या॒म् द्वाभ्या॑ मा॒त्मान॑ मा॒त्मान॒म् द्वाभ्या॒म् प्रति॑ष्ठित्यै॒ प्रति॑ष्ठित्यै॒ द्वाभ्या॑ मा॒त्मान॑ मा॒त्मान॒म् द्वाभ्या॒म् प्रति॑ष्ठित्यै । \newline
53. द्वाभ्या॒म् प्रति॑ष्ठित्यै॒ प्रति॑ष्ठित्यै॒ द्वाभ्या॒म् द्वाभ्या॒म् प्रति॑ष्ठित्या॒ अग्ने ऽग्ने॒ प्रति॑ष्ठित्यै॒ द्वाभ्या॒म् द्वाभ्या॒म् प्रति॑ष्ठित्या॒ अग्ने᳚ । \newline
54. प्रति॑ष्ठित्या॒ अग्ने ऽग्ने॒ प्रति॑ष्ठित्यै॒ प्रति॑ष्ठित्या॒ अग्ने॑ गृहपते गृहप॒ते ऽग्ने॒ प्रति॑ष्ठित्यै॒ प्रति॑ष्ठित्या॒ अग्ने॑ गृहपते । \newline
55. प्रति॑ष्ठित्या॒ इति॒ प्रति॑ - स्थि॒त्यै॒ । \newline
56. अग्ने॑ गृहपते गृहप॒ते ऽग्ने ऽग्ने॑ गृहपत॒ इतीति॑ गृहप॒ते ऽग्ने ऽग्ने॑ गृहपत॒ इति॑ । \newline
57. गृ॒ह॒प॒त॒ इतीति॑ गृहपते गृहपत॒ इत्या॑हा॒हे ति॑ गृहपते गृहपत॒ इत्या॑ह । \newline
58. गृ॒ह॒प॒त॒ इति॑ गृह - प॒ते॒ । \newline
59. इत्या॑हा॒हे तीत्या॑ह यथाय॒जुर् य॑थाय॒जु रा॒हे तीत्या॑ह यथाय॒जुः । \newline
60. आ॒ह॒ य॒था॒य॒जुर् य॑थाय॒जु रा॑हाह यथाय॒जु रे॒वैव य॑थाय॒जु रा॑हाह यथाय॒जु रे॒व । \newline
\pagebreak
\markright{ TS 1.7.6.5  \hfill https://www.vedavms.in \hfill}
\addcontentsline{toc}{section}{ TS 1.7.6.5 }
\section*{ TS 1.7.6.5 }

\textbf{TS 1.7.6.5 } \newline
\textbf{Samhita Paata} \newline

यथाय॒जुरे॒वैतच्छ॒तꣳ हिमा॒ इत्या॑ह श॒तं त्वा॑ हेम॒न्तानि॑न्धिषी॒येति॒ वावैतदा॑ह पु॒त्रस्य॒ नाम॑ गृह्णात्यन्ना॒दमे॒वैनं॑ करोति॒ तामा॒शिष॒मा शा॑से॒ तन्त॑वे॒ ज्योति॑ष्मती॒मिति॑ ब्रूया॒द्-यस्य॑ पु॒त्रोऽजा॑तः॒ स्यात् ते॑ज॒स्व्ये॑वास्य॑ ब्रह्मवर्च॒सी पु॒त्रो जा॑यते॒ तामा॒शिष॒मा शा॑से॒ऽमुष्मै॒ ज्योति॑ष्मती॒मिति॑ ब्रूया॒द्-यस्य॑ पु॒त्रो - [ ] \newline

\textbf{Pada Paata} \newline

य॒था॒य॒जुरिति॑ यथा - य॒जुः । ए॒व । ए॒तत् । श॒तम् । हिमाः᳚ । इति॑ । आ॒ह॒ । श॒तम् । त्वा॒ । हे॒म॒न्तान् । इ॒न्धि॒षी॒य॒ । इति॑ । वाव । ए॒तत् । आ॒ह॒ । पु॒त्रस्य॑ । नाम॑ । गृ॒ह्णा॒ति॒ । अ॒न्ना॒दमित्य॑न्न - अ॒दम् । ए॒व । ए॒न॒म् । क॒रो॒ति॒ । ताम् । आ॒शिष॒मित्या᳚ - शिष᳚म् । एति॑ । शा॒से॒ । तन्त॑वे । ज्योति॑ष्मतीम् । इति॑ । ब्रू॒या॒त् । यस्य॑ । पु॒त्रः । अजा॑तः । स्यात् । ते॒ज॒स्वी । ए॒व । अ॒स्य॒ । ब्र॒ह्म॒व॒र्च॒सीति॑ ब्रह्म - व॒र्च॒सी । पु॒त्रः । जा॒य॒ते॒ । ताम् । आ॒शिष॒मित्या᳚ - शिष᳚म् । एति॑ । शा॒से॒ । अ॒मुष्मै᳚ । ज्योति॑ष्मतीम् । इति॑ । ब्रू॒या॒त् । यस्य॑ । पु॒त्रः ।  \newline


\textbf{Krama Paata} \newline

य॒था॒य॒जुरे॒व । य॒था॒य॒जुरिति॑ यथा - य॒जुः । ए॒वैतत् । ए॒तच्छ॒तम् । श॒तꣳ हिमाः᳚ । हिमा॒ इति॑ । इत्या॑ह । आ॒ह॒ श॒तम् । श॒तम् त्वा᳚ । त्वा॒ हे॒म॒न्तान् । हे॒म॒न्तानि॑न्धिषीय । इ॒न्धि॒षी॒येति॑ । इति॒ वाव । वावैतत् । ए॒तदा॑ह । आ॒ह॒ पु॒त्रस्य॑ । पु॒त्रस्य॒ नाम॑ । नाम॑ गृह्णाति । गृ॒ह्णा॒त्य॒न्ना॒दम् । अ॒न्ना॒दमे॒व । अ॒न्ना॒दमित्य॑न्न - अ॒दम् । ए॒वैन᳚म् । ए॒न॒म् क॒रो॒ति॒ । क॒रो॒ति॒ ताम् । तामा॒शिष᳚म् । आ॒शिष॒मा । आ॒शिष॒मित्या᳚ - शिष᳚म् । आ शा॑से । शा॒से॒ तन्त॑वे । तन्त॑वे॒ ज्योति॑ष्मतीम् । ज्योति॑ष्मती॒मिति॑ । इति॑ ब्रूयात् । ब्रू॒या॒द् यस्य॑ । यस्य॑ पु॒त्रः । पु॒त्रोऽजा॑तः । अजा॑तः॒ स्यात् । स्यात् ते॑ज॒स्वी । ते॒ज॒स्व्ये॑व । ए॒वास्य॑ । अ॒स्य॒ ब्र॒ह्म॒व॒र्च॒सी । ब्र॒ह्म॒व॒र्च॒सी पु॒त्रः । ब्र॒ह्म॒व॒र्च॒सीति॑ ब्रह्म - व॒र्च॒सी । पु॒त्रो जा॑यते । जा॒य॒ते॒ ताम् । तामा॒शिष᳚म् । आ॒शिष॒ मा । आ॒शिष॒मित्या᳚ - शिष᳚म् । आ शा॑से । शा॒से॒ऽमुष्मै᳚ । अ॒मुष्मै॒ ज्योति॑ष्मतीम् । ज्योति॑ष्मती॒मिति॑ । इति॑ ब्रूयात् । ब्रू॒या॒द् यस्य॑ । यस्य॑ पु॒त्रः । पु॒त्रो जा॒तः \newline

\textbf{Jatai Paata} \newline

1. य॒था॒य॒जु रे॒वैव य॑थाय॒जुर् य॑थाय॒जु रे॒व । \newline
2. य॒था॒य॒जुरिति॑ यथा - य॒जुः । \newline
3. ए॒वैत दे॒त दे॒वै वैतत् । \newline
4. ए॒तच् छ॒तꣳ श॒त मे॒त दे॒तच् छ॒तम् । \newline
5. श॒तꣳ हिमा॒ हिमाः᳚ श॒तꣳ श॒तꣳ हिमाः᳚ । \newline
6. हिमा॒ इतीति॒ हिमा॒ हिमा॒ इति॑ । \newline
7. इत्या॑ हा॒हे तीत्या॑ह । \newline
8. आ॒ह॒ श॒तꣳ श॒त मा॑हाह श॒तम् । \newline
9. श॒तम् त्वा᳚ त्वा श॒तꣳ श॒तम् त्वा᳚ । \newline
10. त्वा॒ हे॒म॒न्तान्. हे॑म॒न्तान् त्वा᳚ त्वा हेम॒न्तान् । \newline
11. हे॒म॒न्ता नि॑न्धिषीये न्धिषीय हेम॒न्तान्. हे॑म॒न्ता नि॑न्धिषीय । \newline
12. इ॒न्धि॒षी॒ये तीती᳚न्धिषीये न्धिषी॒ये ति॑ । \newline
13. इति॒ वाव वावे तीति॒ वाव । \newline
14. वावैत दे॒तद् वाव वावैतत् । \newline
15. ए॒त दा॑हाहै॒त दे॒त दा॑ह । \newline
16. आ॒ह॒ पु॒त्रस्य॑ पु॒त्रस्या॑ हाह पु॒त्रस्य॑ । \newline
17. पु॒त्रस्य॒ नाम॒ नाम॑ पु॒त्रस्य॑ पु॒त्रस्य॒ नाम॑ । \newline
18. नाम॑ गृह्णाति गृह्णाति॒ नाम॒ नाम॑ गृह्णाति । \newline
19. गृ॒ह्णा॒त्य॒न्ना॒द म॑न्ना॒दम् गृ॑ह्णाति गृह्णात्यन्ना॒दम् । \newline
20. अ॒न्ना॒द मे॒वैवान्ना॒द म॑न्ना॒द मे॒व । \newline
21. अ॒न्ना॒दमित्य॑न्न - अ॒दम् । \newline
22. ए॒वैन॑ मेन मे॒वै वैन᳚म् । \newline
23. ए॒न॒म् क॒रो॒ति॒ क॒रो॒ त्ये॒न॒ मे॒न॒म् क॒रो॒ति॒ । \newline
24. क॒रो॒ति॒ ताम् ताम् क॑रोति करोति॒ ताम् । \newline
25. ता मा॒शिष॑ मा॒शिष॒म् ताम् ता मा॒शिष᳚म् । \newline
26. आ॒शिष॒ मा ऽऽशिष॑ मा॒शिष॒ मा । \newline
27. आ॒शिष॒मित्या᳚ - शिष᳚म् । \newline
28. आ शा॑से शास॒ आ शा॑से । \newline
29. शा॒से॒ तन्त॑वे॒ तन्त॑वे शासे शासे॒ तन्त॑वे । \newline
30. तन्त॑वे॒ ज्योति॑ष्मती॒म् ज्योति॑ष्मती॒म् तन्त॑वे॒ तन्त॑वे॒ ज्योति॑ष्मतीम् । \newline
31. ज्योति॑ष्मती॒ मितीति॒ ज्योति॑ष्मती॒म् ज्योति॑ष्मती॒ मिति॑ । \newline
32. इति॑ ब्रूयाद् ब्रूया॒ दितीति॑ ब्रूयात् । \newline
33. ब्रू॒या॒द् यस्य॒ यस्य॑ ब्रूयाद् ब्रूया॒द् यस्य॑ । \newline
34. यस्य॑ पु॒त्रः पु॒त्रो यस्य॒ यस्य॑ पु॒त्रः । \newline
35. पु॒त्रो ऽजा॒तो ऽजा॑तः पु॒त्रः पु॒त्रो ऽजा॑तः । \newline
36. अजा॑तः॒ स्याथ् स्या दजा॒तो ऽजा॑तः॒ स्यात् । \newline
37. स्यात् ते॑ज॒स्वी ते॑ज॒स्वी स्याथ् स्यात् ते॑ज॒स्वी । \newline
38. ते॒ज॒ स्व्ये॑वैव ते॑ज॒स्वी ते॑ज॒ स्व्ये॑व । \newline
39. ए॒वास्या᳚ स्यै॒वै वास्य॑ । \newline
40. अ॒स्य॒ ब्र॒ह्म॒व॒र्च॒सी ब्र॑ह्मवर्च॒स्य॑ स्यास्य ब्रह्मवर्च॒सी । \newline
41. ब्र॒ह्म॒व॒र्च॒सी पु॒त्रः पु॒त्रो ब्र॑ह्मवर्च॒सी ब्र॑ह्मवर्च॒सी पु॒त्रः । \newline
42. ब्र॒ह्म॒व॒र्च॒सीति॑ ब्रह्म - व॒र्च॒सी । \newline
43. पु॒त्रो जा॑यते जायते पु॒त्रः पु॒त्रो जा॑यते । \newline
44. जा॒य॒ते॒ ताम् ताम् जा॑यते जायते॒ ताम् । \newline
45. ता मा॒शिष॑ मा॒शिष॒म् ताम् ता मा॒शिष᳚म् । \newline
46. आ॒शिष॒ मा ऽऽशिष॑ मा॒शिष॒ मा । \newline
47. आ॒शिष॒मित्या᳚ - शिष᳚म् । \newline
48. आ शा॑से शास॒ आ शा॑से । \newline
49. शा॒से॒ ऽमुष्मा॑ अ॒मुष्मै॑ शासे शासे॒ ऽमुष्मै᳚ । \newline
50. अ॒मुष्मै॒ ज्योति॑ष्मती॒म् ज्योति॑ष्मती म॒मुष्मा॑ अ॒मुष्मै॒ ज्योति॑ष्मतीम् । \newline
51. ज्योति॑ष्मती॒ मितीति॒ ज्योति॑ष्मती॒म् ज्योति॑ष्मती॒ मिति॑ । \newline
52. इति॑ ब्रूयाद् ब्रूया॒ दितीति॑ ब्रूयात् । \newline
53. ब्रू॒या॒द् यस्य॒ यस्य॑ ब्रूयाद् ब्रूया॒द् यस्य॑ । \newline
54. यस्य॑ पु॒त्रः पु॒त्रो यस्य॒ यस्य॑ पु॒त्रः । \newline
55. पु॒त्रो जा॒तो जा॒तः पु॒त्रः पु॒त्रो जा॒तः । \newline

\textbf{Ghana Paata } \newline

1. य॒था॒य॒जु रे॒वैव य॑थाय॒जुर् य॑थाय॒जु रे॒वैत दे॒त दे॒व य॑थाय॒जुर् य॑थाय॒जु रे॒वैतत् । \newline
2. य॒था॒य॒जुरिति॑ यथा - य॒जुः । \newline
3. ए॒वैत दे॒त दे॒वैवैत च्छ॒तꣳ श॒त मे॒त दे॒वैवैत च्छ॒तम् । \newline
4. ए॒त च्छ॒तꣳ श॒त मे॒त दे॒त च्छ॒तꣳ हिमा॒ हिमाः᳚ श॒त मे॒त दे॒त च्छ॒तꣳ हिमाः᳚ । \newline
5. श॒तꣳ हिमा॒ हिमाः᳚ श॒तꣳ श॒तꣳ हिमा॒ इतीति॒ हिमाः᳚ श॒तꣳ श॒तꣳ हिमा॒ इति॑ । \newline
6. हिमा॒ इतीति॒ हिमा॒ हिमा॒ इत्या॑हा॒हे ति॒ हिमा॒ हिमा॒ इत्या॑ह । \newline
7. इत्या॑हा॒हे तीत्या॑ह श॒तꣳ श॒त मा॒हे तीत्या॑ह श॒तम् । \newline
8. आ॒ह॒ श॒तꣳ श॒त मा॑हाह श॒तम् त्वा᳚ त्वा श॒त मा॑हाह श॒तम् त्वा᳚ । \newline
9. श॒तम् त्वा᳚ त्वा श॒तꣳ श॒तम् त्वा॑ हेम॒न्तान्. हे॑म॒न्तान् त्वा॑ श॒तꣳ श॒तम् त्वा॑ हेम॒न्तान् । \newline
10. त्वा॒ हे॒म॒न्तान्. हे॑म॒न्तान् त्वा᳚ त्वा हेम॒न्ता नि॑न्धिषीये न्धिषीय हेम॒न्तान् त्वा᳚ त्वा हेम॒न्ता नि॑न्धिषीय । \newline
11. हे॒म॒न्ता नि॑न्धिषीये न्धिषीय हेम॒न्तान्. हे॑म॒न्ता नि॑न्धिषी॒ये तीती᳚न्धिषीय हेम॒न्तान्. हे॑म॒न्ता नि॑न्धिषी॒ये ति॑ । \newline
12. इ॒न्धि॒षी॒ये तीती᳚न्धिषीये न्धिषी॒ये ति॒ वाव वावे ती᳚न्धिषीये न्धिषी॒ये ति॒ वाव । \newline
13. इति॒ वाव वावे तीति॒ वावैत दे॒तद् वावे तीति॒ वावैतत् । \newline
14. वावैत दे॒तद् वाव वावैत दा॑हा है॒तद् वाव वावैत दा॑ह । \newline
15. ए॒तदा॑हा है॒त दे॒त दा॑ह पु॒त्रस्य॑ पु॒त्रस्या॑ है॒त दे॒त दा॑ह पु॒त्रस्य॑ । \newline
16. आ॒ह॒ पु॒त्रस्य॑ पु॒त्रस्या॑हाह पु॒त्रस्य॒ नाम॒ नाम॑ पु॒त्रस्या॑हाह पु॒त्रस्य॒ नाम॑ । \newline
17. पु॒त्रस्य॒ नाम॒ नाम॑ पु॒त्रस्य॑ पु॒त्रस्य॒ नाम॑ गृह्णाति गृह्णाति॒ नाम॑ पु॒त्रस्य॑ पु॒त्रस्य॒ नाम॑ गृह्णाति । \newline
18. नाम॑ गृह्णाति गृह्णाति॒ नाम॒ नाम॑ गृह्णात्यन्ना॒द म॑न्ना॒दम् गृ॑ह्णाति॒ नाम॒ नाम॑ गृह्णात्यन्ना॒दम् । \newline
19. गृ॒ह्णा॒त्य॒न्ना॒द म॑न्ना॒दम् गृ॑ह्णाति गृह्णात्यन्ना॒द मे॒वैवान्ना॒दम् गृ॑ह्णाति गृह्णात्यन्ना॒द मे॒व । \newline
20. अ॒न्ना॒द मे॒वैवान्ना॒द म॑न्ना॒द मे॒वैन॑ मेन मे॒वान्ना॒द म॑न्ना॒द मे॒वैन᳚म् । \newline
21. अ॒न्ना॒दमित्य॑न्न - अ॒दम् । \newline
22. ए॒वैन॑ मेन मे॒वै वैन॑म् करोति करोत्येन मे॒वै वैन॑म् करोति । \newline
23. ए॒न॒म् क॒रो॒ति॒ क॒रो॒ त्ये॒न॒ मे॒न॒म् क॒रो॒ति॒ ताम् ताम् क॑रो त्येन मेनम् करोति॒ ताम् । \newline
24. क॒रो॒ति॒ ताम् ताम् क॑रोति करोति॒ ता मा॒शिष॑ मा॒शिष॒म् ताम् क॑रोति करोति॒ ता मा॒शिष᳚म् । \newline
25. ता मा॒शिष॑ मा॒शिष॒म् ताम् ता मा॒शिष॒ मा ऽऽशिष॒म् ताम् ता मा॒शिष॒ मा । \newline
26. आ॒शिष॒ मा ऽऽशिष॑ मा॒शिष॒ मा शा॑से शास॒ आ ऽऽशिष॑ मा॒शिष॒ मा शा॑से । \newline
27. आ॒शिष॒मित्या᳚ - शिष᳚म् । \newline
28. आ शा॑से शास॒ आ शा॑से॒ तन्त॑वे॒ तन्त॑वे शास॒ आ शा॑से॒ तन्त॑वे । \newline
29. शा॒से॒ तन्त॑वे॒ तन्त॑वे शासे शासे॒ तन्त॑वे॒ ज्योति॑ष्मती॒म् ज्योति॑ष्मती॒म् तन्त॑वे शासे शासे॒ तन्त॑वे॒ ज्योति॑ष्मतीम् । \newline
30. तन्त॑वे॒ ज्योति॑ष्मती॒म् ज्योति॑ष्मती॒म् तन्त॑वे॒ तन्त॑वे॒ ज्योति॑ष्मती॒ मितीति॒ ज्योति॑ष्मती॒म् तन्त॑वे॒ तन्त॑वे॒ ज्योति॑ष्मती॒ मिति॑ । \newline
31. ज्योति॑ष्मती॒ मितीति॒ ज्योति॑ष्मती॒म् ज्योति॑ष्मती॒ मिति॑ ब्रूयाद् ब्रूया॒दिति॒ ज्योति॑ष्मती॒म् ज्योति॑ष्मती॒ मिति॑ ब्रूयात् । \newline
32. इति॑ ब्रूयाद् ब्रूया॒ दितीति॑ ब्रूया॒द् यस्य॒ यस्य॑ ब्रूया॒ दितीति॑ ब्रूया॒द् यस्य॑ । \newline
33. ब्रू॒या॒द् यस्य॒ यस्य॑ ब्रूयाद् ब्रूया॒द् यस्य॑ पु॒त्रः पु॒त्रो यस्य॑ ब्रूयाद् ब्रूया॒द् यस्य॑ पु॒त्रः । \newline
34. यस्य॑ पु॒त्रः पु॒त्रो यस्य॒ यस्य॑ पु॒त्रो ऽजा॒तो ऽजा॑तः पु॒त्रो यस्य॒ यस्य॑ पु॒त्रो ऽजा॑तः । \newline
35. पु॒त्रो ऽजा॒तो ऽजा॑तः पु॒त्रः पु॒त्रो ऽजा॑तः॒ स्याथ् स्या दजा॑तः पु॒त्रः पु॒त्रो ऽजा॑तः॒ स्यात् । \newline
36. अजा॑तः॒ स्याथ् स्या दजा॒तो ऽजा॑तः॒ स्यात् ते॑ज॒स्वी ते॑ज॒स्वी स्या दजा॒तो ऽजा॑तः॒ स्यात् ते॑ज॒स्वी । \newline
37. स्यात् ते॑ज॒स्वी ते॑ज॒स्वी स्याथ् स्यात् ते॑ज॒ स्व्ये॑वैव ते॑ज॒स्वी स्याथ् स्यात् ते॑ज॒ स्व्ये॑व । \newline
38. ते॒ज॒ स्व्ये॑वैव ते॑ज॒स्वी ते॑ज॒ स्व्ये॑वास्या᳚ स्यै॒व ते॑ज॒स्वी ते॑ज॒ स्व्ये॑वास्य॑ । \newline
39. ए॒वास्या᳚ स्यै॒वै वास्य॑ ब्रह्मवर्च॒सी ब्र॑ह्मवर्च॒स्य॑ स्यै॒वै वास्य॑ ब्रह्मवर्च॒सी । \newline
40. अ॒स्य॒ ब्र॒ह्म॒व॒र्च॒सी ब्र॑ह्मवर्च॒स्य॑स्यास्य ब्रह्मवर्च॒सी पु॒त्रः पु॒त्रो ब्र॑ह्मवर्च॒स्य॑ स्यास्य ब्रह्मवर्च॒सी पु॒त्रः । \newline
41. ब्र॒ह्म॒व॒र्च॒सी पु॒त्रः पु॒त्रो ब्र॑ह्मवर्च॒सी ब्र॑ह्मवर्च॒सी पु॒त्रो जा॑यते जायते पु॒त्रो ब्र॑ह्मवर्च॒सी ब्र॑ह्मवर्च॒सी पु॒त्रो जा॑यते । \newline
42. ब्र॒ह्म॒व॒र्च॒सीति॑ ब्रह्म - व॒र्च॒सी । \newline
43. पु॒त्रो जा॑यते जायते पु॒त्रः पु॒त्रो जा॑यते॒ ताम् ताम् जा॑यते पु॒त्रः पु॒त्रो जा॑यते॒ ताम् । \newline
44. जा॒य॒ते॒ ताम् ताम् जा॑यते जायते॒ ता मा॒शिष॑ मा॒शिष॒म् ताम् जा॑यते जायते॒ ता मा॒शिष᳚म् । \newline
45. ता मा॒शिष॑ मा॒शिष॒म् ताम् ता मा॒शिष॒ मा ऽऽशिष॒म् ताम् ता मा॒शिष॒ मा । \newline
46. आ॒शिष॒ मा ऽऽशिष॑ मा॒शिष॒ मा शा॑से शास॒ आ ऽऽशिष॑ मा॒शिष॒ मा शा॑से । \newline
47. आ॒शिष॒मित्या᳚ - शिष᳚म् । \newline
48. आ शा॑से शास॒ आ शा॑से॒ ऽमुष्मा॑ अ॒मुष्मै॑ शास॒ आ शा॑से॒ ऽमुष्मै᳚ । \newline
49. शा॒से॒ ऽमुष्मा॑ अ॒मुष्मै॑ शासे शासे॒ ऽमुष्मै॒ ज्योति॑ष्मती॒म् ज्योति॑ष्मती म॒मुष्मै॑ शासे शासे॒ ऽमुष्मै॒ ज्योति॑ष्मतीम् । \newline
50. अ॒मुष्मै॒ ज्योति॑ष्मती॒म् ज्योति॑ष्मती म॒मुष्मा॑ अ॒मुष्मै॒ ज्योति॑ष्मती॒ मितीति॒ ज्योति॑ष्मती म॒मुष्मा॑ अ॒मुष्मै॒ ज्योति॑ष्मती॒ मिति॑ । \newline
51. ज्योति॑ष्मती॒ मितीति॒ ज्योति॑ष्मती॒म् ज्योति॑ष्मती॒ मिति॑ ब्रूयाद् ब्रूया॒ दिति॒ ज्योति॑ष्मती॒म् ज्योति॑ष्मती॒ मिति॑ ब्रूयात् । \newline
52. इति॑ ब्रूयाद् ब्रूया॒ दितीति॑ ब्रूया॒द् यस्य॒ यस्य॑ ब्रूया॒ दितीति॑ ब्रूया॒द् यस्य॑ । \newline
53. ब्रू॒या॒द् यस्य॒ यस्य॑ ब्रूयाद् ब्रूया॒द् यस्य॑ पु॒त्रः पु॒त्रो यस्य॑ ब्रूयाद् ब्रूया॒द् यस्य॑ पु॒त्रः । \newline
54. यस्य॑ पु॒त्रः पु॒त्रो यस्य॒ यस्य॑ पु॒त्रो जा॒तो जा॒तः पु॒त्रो यस्य॒ यस्य॑ पु॒त्रो जा॒तः । \newline
55. पु॒त्रो जा॒तो जा॒तः पु॒त्रः पु॒त्रो जा॒तः स्याथ् स्याज् जा॒तः पु॒त्रः पु॒त्रो जा॒तः स्यात् । \newline
\pagebreak
\markright{ TS 1.7.6.6  \hfill https://www.vedavms.in \hfill}
\addcontentsline{toc}{section}{ TS 1.7.6.6 }
\section*{ TS 1.7.6.6 }

\textbf{TS 1.7.6.6 } \newline
\textbf{Samhita Paata} \newline

जा॒तः स्यात्तेज॑ ए॒वास्मि॑न् ब्रह्मवर्च॒सं द॑धाति॒ यो वै य॒ज्ञ्ं प्र॒युज्य॒ न वि॑मु॒ञ्चत्य॑प्रतिष्ठा॒नो वै स भ॑वति॒ कस्त्वा॑ युनक्ति॒ स त्वा॒ वि मु॑ञ्च॒त्वित्या॑ह प्र॒जाप॑ति॒र् वै कः प्र॒जाप॑तिनै॒वैनं॑ ॅयु॒नक्ति॑ प्र॒जाप॑तिना॒ वि मु॑ञ्चति॒ प्रति॑ष्ठित्या ईश्व॒रं ॅवै व्र॒तमवि॑सृष्टं प्र॒दहोऽग्ने᳚ व्रतपते व्र॒तम॑चारिष॒मित्या॑ह व्र॒तमे॒व-[ ] \newline

\textbf{Pada Paata} \newline

जा॒तः । स्यात् । तेजः॑ । ए॒व । अ॒स्मि॒न्न् । ब्र॒ह्म॒व॒र्च॒समिति॑ ब्रह्म - व॒र्च॒सम् । द॒धा॒ति॒ । यः । वै । य॒ज्ञ्म् । प्र॒युज्येति॑ प्र-युज्य॑ । न । वि॒मु॒ञ्चतीति॑ वि - मु॒ञ्चति॑ । अ॒प्र॒ति॒ष्ठा॒न इत्य॑प्रति - स्था॒नः । वै । सः । भ॒व॒ति॒ । कः । त्वा॒ । यु॒न॒क्ति॒ । सः । त्वा॒ । वीति॑ । मु॒ञ्च॒तु॒ । इति॑ । आ॒ह॒ । प्र॒जाप॑ति॒रिति॑ प्र॒जा - प॒तिः॒ । वै । कः । प्र॒जाप॑ति॒नेति॑ प्र॒जा - प॒ति॒ना॒ । ए॒व । ए॒न॒म् । यु॒नक्ति॑ । प्र॒जाप॑ति॒नेति॑ प्र॒जा - प॒ति॒ना॒ । वीति॑ । मु॒ञ्च॒ति॒ । प्रति॑ष्ठित्या॒ इति॒ प्रति॑ - स्थि॒त्यै॒ । ई॒श्व॒रम् । वै । व्र॒तम् । अवि॑सृष्ट॒मित्यवि॑ - सृ॒ष्ट॒म् । प्र॒दह॒ इति॑ प्र - दहः॑ । अग्ने᳚ । व्र॒त॒प॒त॒ इति॑ व्रत - प॒ते॒ । व्र॒तम् । अ॒चा॒रि॒ष॒म् । इति॑ । आ॒ह॒ । व्र॒तम् । ए॒व ।  \newline


\textbf{Krama Paata} \newline

जा॒तः स्यात् । स्यात् तेजः॑ । तेज॑ ए॒व । ए॒वास्मिन्न्॑ । अ॒स्मि॒न्,ब्र॒ह्म॒व॒र्च॒सम् । ब्र॒ह्म॒व॒र्च॒सम् द॑धाति । ब्र॒ह्म॒व॒र्च॒समिति॑ ब्रह्म - व॒र्च॒सम् । द॒धा॒ति॒ यः । यो वै । वै य॒ज्ञ्म् । य॒ज्ञ्म् प्र॒युज्य॑ । प्र॒युज्य॒ न । प्र॒युज्येति॑ प्र - युज्य॑ । न वि॑मु॒ञ्चति॑ । वि॒मु॒ञ्चत्य॑प्रतिष्ठा॒नः । वि॒मु॒ञ्चतीति॑ वि - मु॒ञ्चति॑ । अ॒प्र॒ति॒ष्ठा॒नो वै । अ॒प्र॒ति॒ष्ठा॒न इत्य॑प्रति - स्था॒नः । वै सः । स भ॑वति । भ॒व॒ति॒ कः । कस्त्वा᳚ । त्वा॒ यु॒न॒क्ति॒ । यु॒न॒क्ति॒ सः । स त्वा᳚ । त्वा॒ वि । वि मु॑ञ्चतु । मु॒ञ्च॒त्विति॑ । इत्या॑ह । आ॒ह॒ प्र॒जाप॑तिः । प्र॒जाप॑ति॒र् वै । प्र॒जाप॑ति॒रिति॑ प्र॒जा - प॒तिः॒ । वै कः । कः प्र॒जाप॑तिना । प्र॒जाप॑तिनै॒व । प्र॒जाप॑ति॒नेति॑ प्र॒जा - प॒ति॒ना॒ । ए॒वैन᳚म् । ए॒नं॒ ॅयु॒नक्ति॑ । यु॒नक्ति॑ प्र॒जाप॑तिना । प्र॒जाप॑तिना॒ वि । प्र॒जाप॑ति॒नेति॑ प्र॒जा - प॒ति॒ना॒ । वि मु॑ञ्चति । मु॒ञ्च॒ति॒ प्रति॑ष्ठित्यै । प्रति॑ष्ठित्या ईश्व॒रम् । प्रति॑ष्ठित्या॒ इति॒ प्रति॑ - स्थि॒त्यै॒ । ई॒श्व॒रं ॅवै । वै व्र॒तम् । व्र॒तमवि॑सृष्टम् । अवि॑सृष्टम् प्र॒दहः॑ । अवि॑सृष्ट॒मित्यवि॑ - सृ॒ष्ट॒म् । प्र॒दहोऽग्ने᳚ । प्र॒दह॒ इति॑ प्र - दहः॑ । अग्ने᳚ व्रतपते । व्र॒त॒प॒ते॒ व्र॒तम् । व्र॒त॒प॒त॒ इति॑ व्रत - प॒ते॒ । व्र॒तम॑चारिषम् । अ॒चा॒रि॒ष॒मिति॑ । इत्या॑ह । आ॒ह॒ व्र॒तम् । व्र॒तमे॒व । ए॒व वि \newline

\textbf{Jatai Paata} \newline

1. जा॒तः स्याथ् स्याज् जा॒तो जा॒तः स्यात् । \newline
2. स्यात् तेज॒ स्तेजः॒ स्याथ् स्यात् तेजः॑ । \newline
3. तेज॑ ए॒वैव तेज॒ स्तेज॑ ए॒व । \newline
4. ए॒वास्मि॑न् नस्मिन् ने॒वै वास्मिन्न्॑ । \newline
5. अ॒स्मि॒न् ब्र॒ह्म॒व॒र्च॒सम् ब्र॑ह्मवर्च॒स म॑स्मिन् नस्मिन् ब्रह्मवर्च॒सम् । \newline
6. ब्र॒ह्म॒व॒र्च॒सम् द॑धाति दधाति ब्रह्मवर्च॒सम् ब्र॑ह्मवर्च॒सम् द॑धाति । \newline
7. ब्र॒ह्म॒व॒र्च॒समिति॑ ब्रह्म - व॒र्च॒सम् । \newline
8. द॒धा॒ति॒ यो यो द॑धाति दधाति॒ यः । \newline
9. यो वै वै यो यो वै । \newline
10. वै य॒ज्ञ्ं ॅय॒ज्ञ्ं ॅवै वै य॒ज्ञ्म् । \newline
11. य॒ज्ञ्म् प्र॒युज्य॑ प्र॒युज्य॑ य॒ज्ञ्ं ॅय॒ज्ञ्म् प्र॒युज्य॑ । \newline
12. प्र॒युज्य॒ न न प्र॒युज्य॑ प्र॒युज्य॒ न । \newline
13. प्र॒युज्येति॑ प्र - युज्य॑ । \newline
14. न वि॑मु॒ञ्चति॑ विमु॒ञ्चति॒ न न वि॑मु॒ञ्चति॑ । \newline
15. वि॒मु॒ञ्च त्य॑प्रतिष्ठा॒नो᳚ ऽप्रतिष्ठा॒नो वि॑मु॒ञ्चति॑ विमु॒ञ्च त्य॑प्रतिष्ठा॒नः । \newline
16. वि॒मु॒ञ्चतीति॑ वि - मु॒ञ्चति॑ । \newline
17. अ॒प्र॒ति॒ष्ठा॒नो वै वा अ॑प्रतिष्ठा॒नो᳚ ऽप्रतिष्ठा॒नो वै । \newline
18. अ॒प्र॒ति॒ष्ठा॒न इत्य॑प्रति - स्था॒नः । \newline
19. वै स स वै वै सः । \newline
20. स भ॑वति भवति॒ स स भ॑वति । \newline
21. भ॒व॒ति॒ कः को भ॑वति भवति॒ कः । \newline
22. क स्त्वा᳚ त्वा॒ कः क स्त्वा᳚ । \newline
23. त्वा॒ यु॒न॒क्ति॒ यु॒न॒क्ति॒ त्वा॒ त्वा॒ यु॒न॒क्ति॒ । \newline
24. यु॒न॒क्ति॒ स स यु॑नक्ति युनक्ति॒ सः । \newline
25. स त्वा᳚ त्वा॒ स स त्वा᳚ । \newline
26. त्वा॒ वि वि त्वा᳚ त्वा॒ वि । \newline
27. वि मु॑ञ्चतु मुञ्चतु॒ वि वि मु॑ञ्चतु । \newline
28. मु॒ञ्च॒ त्वितीति॑ मुञ्चतु मुञ्च॒ त्विति॑ । \newline
29. इत्या॑ हा॒हे तीत्या॑ह । \newline
30. आ॒ह॒ प्र॒जाप॑तिः प्र॒जाप॑ति राहाह प्र॒जाप॑तिः । \newline
31. प्र॒जाप॑ति॒र् वै वै प्र॒जाप॑तिः प्र॒जाप॑ति॒र् वै । \newline
32. प्र॒जाप॑ति॒रिति॑ प्र॒जा - प॒तिः॒ । \newline
33. वै कः को वै वै कः । \newline
34. कः प्र॒जाप॑तिना प्र॒जाप॑तिना॒ कः कः प्र॒जाप॑तिना । \newline
35. प्र॒जाप॑ति नै॒वैव प्र॒जाप॑तिना प्र॒जाप॑ति नै॒व । \newline
36. प्र॒जाप॑ति॒नेति॑ प्र॒जा - प॒ति॒ना॒ । \newline
37. ए॒वैन॑ मेन मे॒वैवैन᳚म् । \newline
38. ए॒नं॒ ॅयु॒नक्ति॑ यु॒नक्त्ये॑न मेनं ॅयु॒नक्ति॑ । \newline
39. यु॒नक्ति॑ प्र॒जाप॑तिना प्र॒जाप॑तिना यु॒नक्ति॑ यु॒नक्ति॑ प्र॒जाप॑तिना । \newline
40. प्र॒जाप॑तिना॒ वि वि प्र॒जाप॑तिना प्र॒जाप॑तिना॒ वि । \newline
41. प्र॒जाप॑ति॒नेति॑ प्र॒जा - प॒ति॒ना॒ । \newline
42. वि मु॑ञ्चति मुञ्चति॒ वि वि मु॑ञ्चति । \newline
43. मु॒ञ्च॒ति॒ प्रति॑ष्ठित्यै॒ प्रति॑ष्ठित्यै मुञ्चति मुञ्चति॒ प्रति॑ष्ठित्यै । \newline
44. प्रति॑ष्ठित्या ईश्व॒र मी᳚श्व॒रम् प्रति॑ष्ठित्यै॒ प्रति॑ष्ठित्या ईश्व॒रम् । \newline
45. प्रति॑ष्ठित्या॒ इति॒ प्रति॑ - स्थि॒त्यै॒ । \newline
46. ई॒श्व॒रं ॅवै वा ई᳚श्व॒र मी᳚श्व॒रं ॅवै । \newline
47. वै व्र॒तं ॅव्र॒तं ॅवै वै व्र॒तम् । \newline
48. व्र॒त मवि॑सृष्ट॒ मवि॑सृष्टं ॅव्र॒तं ॅव्र॒त मवि॑सृष्टम् । \newline
49. अवि॑सृष्टम् प्र॒दहः॑ प्र॒दहो ऽवि॑सृष्ट॒ मवि॑सृष्टम् प्र॒दहः॑ । \newline
50. अवि॑सृष्ट॒मित्यवि॑ - सृ॒ष्ट॒म् । \newline
51. प्र॒दहो ऽग्ने ऽग्ने᳚ प्र॒दहः॑ प्र॒दहो ऽग्ने᳚ । \newline
52. प्र॒दह॒ इति॑ प्र - दहः॑ । \newline
53. अग्ने᳚ व्रतपते व्रतप॒ते ऽग्ने ऽग्ने᳚ व्रतपते । \newline
54. व्र॒त॒प॒ते॒ व्र॒तं ॅव्र॒तं ॅव्र॑तपते व्रतपते व्र॒तम् । \newline
55. व्र॒त॒प॒त॒ इति॑ व्रत - प॒ते॒ । \newline
56. व्र॒त म॑चारिष मचारिषं ॅव्र॒तं ॅव्र॒त म॑चारिषम् । \newline
57. अ॒चा॒रि॒ष॒ मिती त्य॑चारिष मचारिष॒ मिति॑ । \newline
58. इत्या॑ हा॒हे तीत्या॑ह । \newline
59. आ॒ह॒ व्र॒तं ॅव्र॒त मा॑हाह व्र॒तम् । \newline
60. व्र॒त मे॒वैव व्र॒तं ॅव्र॒त मे॒व । \newline
61. ए॒व वि व्ये॑वैव वि । \newline

\textbf{Ghana Paata } \newline

1. जा॒तः स्याथ् स्याज् जा॒तो जा॒तः स्यात् तेज॒ स्तेजः॒ स्याज् जा॒तो जा॒तः स्यात् तेजः॑ । \newline
2. स्यात् तेज॒ स्तेजः॒ स्याथ् स्यात् तेज॑ ए॒वैव तेजः॒ स्याथ् स्यात् तेज॑ ए॒व । \newline
3. तेज॑ ए॒वैव तेज॒ स्तेज॑ ए॒वास्मि॑न् नस्मिन् ने॒व तेज॒ स्तेज॑ ए॒वास्मिन्न्॑ । \newline
4. ए॒वास्मि॑न् नस्मिन् ने॒वैवास्मि॑न् ब्रह्मवर्च॒सम् ब्र॑ह्मवर्च॒स म॑स्मिन् ने॒वैवास्मि॑न् ब्रह्मवर्च॒सम् । \newline
5. अ॒स्मि॒न् ब्र॒ह्म॒व॒र्च॒सम् ब्र॑ह्मवर्च॒स म॑स्मिन् नस्मिन् ब्रह्मवर्च॒सम् द॑धाति दधाति ब्रह्मवर्च॒स म॑स्मिन् नस्मिन् ब्रह्मवर्च॒सम् द॑धाति । \newline
6. ब्र॒ह्म॒व॒र्च॒सम् द॑धाति दधाति ब्रह्मवर्च॒सम् ब्र॑ह्मवर्च॒सम् द॑धाति॒ यो यो द॑धाति ब्रह्मवर्च॒सम् ब्र॑ह्मवर्च॒सम् द॑धाति॒ यः । \newline
7. ब्र॒ह्म॒व॒र्च॒समिति॑ ब्रह्म - व॒र्च॒सम् । \newline
8. द॒धा॒ति॒ यो यो द॑धाति दधाति॒ यो वै वै यो द॑धाति दधाति॒ यो वै । \newline
9. यो वै वै यो यो वै य॒ज्ञ्ं ॅय॒ज्ञ्ं ॅवै यो यो वै य॒ज्ञ्म् । \newline
10. वै य॒ज्ञ्ं ॅय॒ज्ञ्ं ॅवै वै य॒ज्ञ्म् प्र॒युज्य॑ प्र॒युज्य॑ य॒ज्ञ्ं ॅवै वै य॒ज्ञ्म् प्र॒युज्य॑ । \newline
11. य॒ज्ञ्म् प्र॒युज्य॑ प्र॒युज्य॑ य॒ज्ञ्ं ॅय॒ज्ञ्म् प्र॒युज्य॒ न न प्र॒युज्य॑ य॒ज्ञ्ं ॅय॒ज्ञ्म् प्र॒युज्य॒ न । \newline
12. प्र॒युज्य॒ न न प्र॒युज्य॑ प्र॒युज्य॒ न वि॑मु॒ञ्चति॑ विमु॒ञ्चति॒ न प्र॒युज्य॑ प्र॒युज्य॒ न वि॑मु॒ञ्चति॑ । \newline
13. प्र॒युज्येति॑ प्र - युज्य॑ । \newline
14. न वि॑मु॒ञ्चति॑ विमु॒ञ्चति॒ न न वि॑मु॒ञ्च त्य॑प्रतिष्ठा॒नो᳚ ऽप्रतिष्ठा॒नो वि॑मु॒ञ्चति॒ न न वि॑मु॒ञ्च त्य॑प्रतिष्ठा॒नः । \newline
15. वि॒मु॒ञ्च त्य॑प्रतिष्ठा॒नो᳚ ऽप्रतिष्ठा॒नो वि॑मु॒ञ्चति॑ विमु॒ञ्च त्य॑प्रतिष्ठा॒नो वै वा अ॑प्रतिष्ठा॒नो वि॑मु॒ञ्चति॑ विमु॒ञ्च त्य॑प्रतिष्ठा॒नो वै । \newline
16. वि॒मु॒ञ्चतीति॑ वि - मु॒ञ्चति॑ । \newline
17. अ॒प्र॒ति॒ष्ठा॒नो वै वा अ॑प्रतिष्ठा॒नो᳚ ऽप्रतिष्ठा॒नो वै स स वा अ॑प्रतिष्ठा॒नो᳚ ऽप्रतिष्ठा॒नो वै सः । \newline
18. अ॒प्र॒ति॒ष्ठा॒न इत्य॑प्रति - स्था॒नः । \newline
19. वै स स वै वै स भ॑वति भवति॒ स वै वै स भ॑वति । \newline
20. स भ॑वति भवति॒ स स भ॑वति॒ कः को भ॑वति॒ स स भ॑वति॒ कः । \newline
21. भ॒व॒ति॒ कः को भ॑वति भवति॒ कस्त्वा᳚ त्वा॒ को भ॑वति भवति॒ कस्त्वा᳚ । \newline
22. कस्त्वा᳚ त्वा॒ कः कस्त्वा॑ युनक्ति युनक्ति त्वा॒ कः कस्त्वा॑ युनक्ति । \newline
23. त्वा॒ यु॒न॒क्ति॒ यु॒न॒क्ति॒ त्वा॒ त्वा॒ यु॒न॒क्ति॒ स स यु॑नक्ति त्वा त्वा युनक्ति॒ सः । \newline
24. यु॒न॒क्ति॒ स स यु॑नक्ति युनक्ति॒ स त्वा᳚ त्वा॒ स यु॑नक्ति युनक्ति॒ स त्वा᳚ । \newline
25. स त्वा᳚ त्वा॒ स स त्वा॒ वि वि त्वा॒ स स त्वा॒ वि । \newline
26. त्वा॒ वि वि त्वा᳚ त्वा॒ वि मु॑ञ्चतु मुञ्चतु॒ वि त्वा᳚ त्वा॒ वि मु॑ञ्चतु । \newline
27. वि मु॑ञ्चतु मुञ्चतु॒ वि वि मु॑ञ्च॒ त्वितीति॑ मुञ्चतु॒ वि वि मु॑ञ्च॒त्विति॑ । \newline
28. मु॒ञ्च॒ त्वितीति॑ मुञ्चतु मुञ्च॒ त्वित्या॑हा॒हे ति॑ मुञ्चतु मुञ्च॒ त्वित्या॑ह । \newline
29. इत्या॑हा॒हे तीत्या॑ह प्र॒जाप॑तिः प्र॒जाप॑ति रा॒हे तीत्या॑ह प्र॒जाप॑तिः । \newline
30. आ॒ह॒ प्र॒जाप॑तिः प्र॒जाप॑ति राहाह प्र॒जाप॑ति॒र् वै वै प्र॒जाप॑ति राहाह प्र॒जाप॑ति॒र् वै । \newline
31. प्र॒जाप॑ति॒र् वै वै प्र॒जाप॑तिः प्र॒जाप॑ति॒र् वै कः को वै प्र॒जाप॑तिः प्र॒जाप॑ति॒र् वै कः । \newline
32. प्र॒जाप॑ति॒रिति॑ प्र॒जा - प॒तिः॒ । \newline
33. वै कः को वै वै कः प्र॒जाप॑तिना प्र॒जाप॑तिना॒ को वै वै कः प्र॒जाप॑तिना । \newline
34. कः प्र॒जाप॑तिना प्र॒जाप॑तिना॒ कः कः प्र॒जाप॑तिनै॒वैव प्र॒जाप॑तिना॒ कः कः प्र॒जाप॑तिनै॒व । \newline
35. प्र॒जाप॑ति नै॒वैव प्र॒जाप॑तिना प्र॒जाप॑ति नै॒वैन॑ मेन मे॒व प्र॒जाप॑तिना प्र॒जाप॑ति नै॒वैन᳚म् । \newline
36. प्र॒जाप॑ति॒नेति॑ प्र॒जा - प॒ति॒ना॒ । \newline
37. ए॒वैन॑ मेन मे॒वैवैनं॑ ॅयु॒नक्ति॑ यु॒नक्त्ये॑न मे॒वैवैनं॑ ॅयु॒नक्ति॑ । \newline
38. ए॒नं॒ ॅयु॒नक्ति॑ यु॒नक्त्ये॑न मेनं ॅयु॒नक्ति॑ प्र॒जाप॑तिना प्र॒जाप॑तिना यु॒नक्त्ये॑न मेनं ॅयु॒नक्ति॑ प्र॒जाप॑तिना । \newline
39. यु॒नक्ति॑ प्र॒जाप॑तिना प्र॒जाप॑तिना यु॒नक्ति॑ यु॒नक्ति॑ प्र॒जाप॑तिना॒ वि वि प्र॒जाप॑तिना यु॒नक्ति॑ यु॒नक्ति॑ प्र॒जाप॑तिना॒ वि । \newline
40. प्र॒जाप॑तिना॒ वि वि प्र॒जाप॑तिना प्र॒जाप॑तिना॒ वि मु॑ञ्चति मुञ्चति॒ वि प्र॒जाप॑तिना प्र॒जाप॑तिना॒ वि मु॑ञ्चति । \newline
41. प्र॒जाप॑ति॒नेति॑ प्र॒जा - प॒ति॒ना॒ । \newline
42. वि मु॑ञ्चति मुञ्चति॒ वि वि मु॑ञ्चति॒ प्रति॑ष्ठित्यै॒ प्रति॑ष्ठित्यै मुञ्चति॒ वि वि मु॑ञ्चति॒ प्रति॑ष्ठित्यै । \newline
43. मु॒ञ्च॒ति॒ प्रति॑ष्ठित्यै॒ प्रति॑ष्ठित्यै मुञ्चति मुञ्चति॒ प्रति॑ष्ठित्या ईश्व॒र मी᳚श्व॒रम् प्रति॑ष्ठित्यै मुञ्चति मुञ्चति॒ प्रति॑ष्ठित्या ईश्व॒रम् । \newline
44. प्रति॑ष्ठित्या ईश्व॒र मी᳚श्व॒रम् प्रति॑ष्ठित्यै॒ प्रति॑ष्ठित्या ईश्व॒रं ॅवै वा ई᳚श्व॒रम् प्रति॑ष्ठित्यै॒ प्रति॑ष्ठित्या ईश्व॒रं ॅवै । \newline
45. प्रति॑ष्ठित्या॒ इति॒ प्रति॑ - स्थि॒त्यै॒ । \newline
46. ई॒श्व॒रं ॅवै वा ई᳚श्व॒र मी᳚श्व॒रं ॅवै व्र॒तं ॅव्र॒तं ॅवा ई᳚श्व॒र मी᳚श्व॒रं ॅवै व्र॒तम् । \newline
47. वै व्र॒तं ॅव्र॒तं ॅवै वै व्र॒त मवि॑सृष्ट॒ मवि॑सृष्टं ॅव्र॒तं ॅवै वै व्र॒त मवि॑सृष्टम् । \newline
48. व्र॒त मवि॑सृष्ट॒ मवि॑सृष्टं ॅव्र॒तं ॅव्र॒त मवि॑सृष्टम् प्र॒दहः॑ प्र॒दहो ऽवि॑सृष्टं ॅव्र॒तं ॅव्र॒त मवि॑सृष्टम् प्र॒दहः॑ । \newline
49. अवि॑सृष्टम् प्र॒दहः॑ प्र॒दहो ऽवि॑सृष्ट॒ मवि॑सृष्टम् प्र॒दहो ऽग्ने ऽग्ने᳚ प्र॒दहो ऽवि॑सृष्ट॒ मवि॑सृष्टम् प्र॒दहो ऽग्ने᳚ । \newline
50. अवि॑सृष्ट॒मित्यवि॑ - सृ॒ष्ट॒म् । \newline
51. प्र॒दहो ऽग्ने ऽग्ने᳚ प्र॒दहः॑ प्र॒दहो ऽग्ने᳚ व्रतपते व्रतप॒ते ऽग्ने᳚ प्र॒दहः॑ प्र॒दहो ऽग्ने᳚ व्रतपते । \newline
52. प्र॒दह॒ इति॑ प्र - दहः॑ । \newline
53. अग्ने᳚ व्रतपते व्रतप॒ते ऽग्ने ऽग्ने᳚ व्रतपते व्र॒तं ॅव्र॒तं ॅव्र॑तप॒ते ऽग्ने ऽग्ने᳚ व्रतपते व्र॒तम् । \newline
54. व्र॒त॒प॒ते॒ व्र॒तं ॅव्र॒तं ॅव्र॑तपते व्रतपते व्र॒त म॑चारिष मचारिषं ॅव्र॒तं ॅव्र॑तपते व्रतपते व्र॒त म॑चारिषम् । \newline
55. व्र॒त॒प॒त॒ इति॑ व्रत - प॒ते॒ । \newline
56. व्र॒त म॑चारिष मचारिषं ॅव्र॒तं ॅव्र॒त म॑चारिष॒ मिती त्य॑चारिषं ॅव्र॒तं ॅव्र॒त म॑चारिष॒ मिति॑ । \newline
57. अ॒चा॒रि॒ष॒ मिती त्य॑चारिष मचारिष॒ मित्या॑हा॒हे त्य॑चारिष मचारिष॒ मित्या॑ह । \newline
58. इत्या॑हा॒हे तीत्या॑ह व्र॒तं ॅव्र॒त मा॒हे तीत्या॑ह व्र॒तम् । \newline
59. आ॒ह॒ व्र॒तं ॅव्र॒त मा॑हाह व्र॒त मे॒वैव व्र॒त मा॑हाह व्र॒त मे॒व । \newline
60. व्र॒त मे॒वैव व्र॒तं ॅव्र॒त मे॒व वि व्ये॑व व्र॒तं ॅव्र॒त मे॒व वि । \newline
61. ए॒व वि व्ये॑वैव वि सृ॑जते सृजते॒ व्ये॑वैव वि सृ॑जते । \newline
\pagebreak
\markright{ TS 1.7.6.7  \hfill https://www.vedavms.in \hfill}
\addcontentsline{toc}{section}{ TS 1.7.6.7 }
\section*{ TS 1.7.6.7 }

\textbf{TS 1.7.6.7 } \newline
\textbf{Samhita Paata} \newline

वि सृ॑जते॒ शान्त्या॒ अप्र॑दाहाय॒ परा॒ङ॒. वाव य॒ज्ञ् ए॑ति॒ न नि व॑र्तते॒ पुन॒र्यो वै य॒ज्ञ्स्य॑ पुनरालं॒भं ॅवि॒द्वान्. यज॑ते॒ तम॒भि नि व॑र्तते य॒ज्ञो ब॑भूव॒ स आ ब॑भू॒वेत्या॑है॒ष वै य॒ज्ञ्स्य॑ पुनरालं॒भ-स्तेनै॒वैनं॒ पुन॒रा ल॑भ॒तेऽन॑वरुद्धा॒ वा ए॒तस्य॑ वि॒राड् य आहि॑ताग्निः॒ सन्न॑स॒भः प॒शवः॒ खलु॒ वै ( ) ब्रा᳚ह्म॒णस्य॑ स॒भेष्ट्वा प्राङु॒त्क्रम्य॑ ब्रूया॒द् गोमाꣳ॑ अ॒ग्नेऽवि॑माꣳ अ॒श्वी य॒ज्ञ् इत्यव॑ स॒भाꣳ रु॒न्धे प्र स॒हस्रं॑ प॒शूना᳚प्नो॒त्याऽस्य॑ प्र॒जायां᳚ ॅवा॒जी जा॑यते ॥(यः-स-मा॒सिषं॑-गृहपत॒-इत्या॑हा॒-मुष्मै॒ ज्योति॑ष्मती॒मिति॑ ब्रूया॒द् यस्य॑पु॒त्रो-व्र॒तमे॒व-खलु॒ वै- \newline

\textbf{Pada Paata} \newline

वीति॑ । सृ॒ज॒ते॒ । शान्त्यै᳚ । अप्र॑दाहा॒येत्यप्र॑ - दा॒हा॒य॒ । पराङ्॑ । वाव । य॒ज्ञ्ः । ए॒ति॒ । न । नीति॑ । व॒र्त॒ते॒ । पुनः॑ । यः । वै । य॒ज्ञ्स्य॑ । पु॒न॒रा॒ल॒भंमिति॑ पुनः - आ॒ल॒भंम् । वि॒द्वान् । यज॑ते । तम् । अ॒भि । नीति॑ । व॒र्त॒ते॒ । य॒ज्ञ्ः । ब॒भू॒व॒ । सः । एति॑ । ब॒भू॒व॒ । इति॑ । आ॒ह॒ । ए॒षः । वै । य॒ज्ञ्स्य॑ । पु॒न॒रा॒ल॒भं इति॑ पुनः-आ॒ल॒भंः । तेन॑ । ए॒व । ए॒न॒म् । पुनः॑ । एति॑ । ल॒भ॒ते॒ । अन॑वरु॒द्धेत्यन॑व - रु॒द्धा॒ । वै । ए॒तस्य॑ । वि॒राडिति॑ वि - राट् । यः । आहि॑ताग्नि॒रित्याहि॑त-अ॒ग्निः॒ । सन्न् । अ॒स॒भः । प॒शवः॑ । खलु॑ । वै ( ) । ब्रा॒ह्म॒णस्य॑ । स॒भा । इ॒ष्ट्वा । प्राङ् । उ॒त्क्रम्येत्यु॑त् - क्रम्य॑ । ब्रू॒या॒त् । गोमा॒निति॒ गो - मा॒न् । अ॒ग्ने॒ । अवि॑मा॒नित्यवि॑ - मा॒न् । अ॒श्वी । य॒ज्ञ्ः । इति॑ । अवेति॑ । स॒भाम् । रु॒न्धे । प्रेति॑ । स॒हस्र᳚म् । प॒शून् । आ॒प्नो॒ति॒ । एति॑ । अ॒स्य॒ । प्र॒जाया॒मिति॑ प्र - जाया᳚म् । वा॒जी । जा॒य॒ते॒ ॥  \newline


\textbf{Krama Paata} \newline

वि सृ॑जते । सृ॒ज॒ते॒ शान्त्यै᳚ । शान्त्या॒ अप्र॑दाहाय । अप्र॑दाहाय॒ पराङ्॑ । अप्र॑दाहा॒येत्यप्र॑ - दा॒हा॒य॒ । परा॒ङ्॒ वाव । वाव य॒ज्ञ्ः । य॒ज्ञ् ए॑ति । ए॒ति॒ न । न नि । नि व॑र्तते । व॒र्त॒ते॒ पुनः॑ । पुन॒र्,यः । यो वै । वै य॒ज्ञ्स्य॑ । य॒ज्ञ्स्य॑ पुनराल॒म्भम् । पु॒न॒रा॒ल॒म्भं ॅवि॒द्वान् । पु॒न॒रा॒ल॒म्भमिति॑ पुनः - आ॒ल॒म्भम् । वि॒द्वान्. यज॑ते । यज॑ते॒ तम् । तम॒भि । अ॒भि नि । नि व॑र्तते । व॒र्त॒ते॒ य॒ज्ञ्ः । य॒ज्ञो ब॑भूव । ब॒भू॒व॒ सः । स आ । आ ब॑भूव । ब॒भू॒वेति॑ । इत्या॑ह । आ॒है॒षः । ए॒ष वै । वै य॒ज्ञ्स्य॑ । य॒ज्ञ्स्य॑ पुनराल॒म्भः । पु॒न॒रा॒ल॒म्भस्तेन॑ । पु॒न॒रा॒ल॒म्भ इति॑ पुनः - आ॒ल॒म्भः । तेनै॒व । ए॒वैन᳚म् । ए॒न॒म् पुनः॑ । पुन॒रा । आ ल॑भते । ल॒भ॒तेऽन॑वरुद्धाः । अन॑वरुद्धा॒ वै । अन॑वरु॒द्धेत्यन॑व - रु॒द्धा॒ । वा ए॒तस्य॑ । ए॒तस्य॑ वि॒राट् । वि॒राड् यः । वि॒राडिति॑ वि - राट् । य आहि॑ताग्निः । आहि॑ताग्निः॒ सन्न् । आहि॑ताग्नि॒रित्या॑हित - अ॒ग्निः॒ । सन्न॑स॒भः । अ॒स॒भः प॒शवः॑ । प॒शवः॒ खलु॑ । खलु॒ वै ( ) । वै ब्रा᳚ह्म॒णस्य॑ । ब्रा॒ह्म॒णस्य॑ स॒भा । स॒भेष्ट्वा । इ॒ष्ट्वा प्राङ् । प्राङु॒त्क्रम्य॑ । उ॒त्क्रम्य॑ ब्रूयात् । उ॒त्क्रम्येत्यु॑त् - क्रम्य॑ । ब्रू॒या॒द् गोमान्॑ । गोमाꣳ॑ अग्ने । गोमा॒निति॒ गो - मा॒न्॒ । अ॒ग्नेऽवि॑मान् । अवि॑माꣳ अ॒श्वी । अवि॑मा॒नित्यवि॑ - मा॒न्॒ । अ॒श्वी य॒ज्ञ्ः । य॒ज्ञ् इति॑ । इत्यव॑ । अव॑ स॒भाम् । स॒भाꣳ रु॒न्धे । रु॒न्धे प्र । प्र स॒हस्र᳚म् । स॒हस्र॑म् प॒शून् । प॒शूना᳚प्नोति । आ॒प्नो॒त्या । आऽस्य॑ । अ॒स्य॒ प्र॒जाया᳚म् । प्र॒जायां᳚ ॅवा॒जी । प्र॒जाया॒मिति॑ प्र - जाया᳚म् । वा॒जी जा॑यते । जा॒य॒त॒ इति॑ जायते । \newline

\textbf{Jatai Paata} \newline

1. वि सृ॑जते सृजते॒ वि वि सृ॑जते । \newline
2. सृ॒ज॒ते॒ शान्त्यै॒ शान्त्यै॑ सृजते सृजते॒ शान्त्यै᳚ । \newline
3. शान्त्या॒ अप्र॑दाहा॒या प्र॑दाहाय॒ शान्त्यै॒ शान्त्या॒ अप्र॑दाहाय । \newline
4. अप्र॑दाहाय॒ परा॒ङ् परा॒ ङप्र॑दाहा॒या प्र॑दाहाय॒ पराङ्॑ । \newline
5. अप्र॑दाहा॒येत्यप्र॑ - दा॒हा॒य॒ । \newline
6. परा॒ङ्. वाव वाव परा॒ङ् परा॒ङ्. वाव । \newline
7. वाव य॒ज्ञो य॒ज्ञो वाव वाव य॒ज्ञ्ः । \newline
8. य॒ज्ञ् ए᳚त्येति य॒ज्ञो य॒ज्ञ् ए॑ति । \newline
9. ए॒ति॒ न नैत्ये॑ति॒ न । \newline
10. न नि नि न न नि । \newline
11. नि व॑र्तते वर्तते॒ नि नि व॑र्तते । \newline
12. व॒र्त॒ते॒ पुनः॒ पुन॑र् वर्तते वर्तते॒ पुनः॑ । \newline
13. पुन॒र् यो यः पुनः॒ पुन॒र् यः । \newline
14. यो वै वै यो यो वै । \newline
15. वै य॒ज्ञ्स्य॑ य॒ज्ञ्स्य॒ वै वै य॒ज्ञ्स्य॑ । \newline
16. य॒ज्ञ्स्य॑ पुनरालं॒भम् पु॑नरालं॒भं ॅय॒ज्ञ्स्य॑ य॒ज्ञ्स्य॑ पुनरालं॒भम् । \newline
17. पु॒न॒रा॒लं॒भं ॅवि॒द्वान्. वि॒द्वान् पु॑नरालं॒भम् पु॑नरालं॒भं ॅवि॒द्वान् । \newline
18. पु॒न॒रा॒लं॒भमिति॑ पुनः - आ॒लं॒भम् । \newline
19. वि॒द्वान्. यज॑ते॒ यज॑ते वि॒द्वान्. वि॒द्वान्. यज॑ते । \newline
20. यज॑ते॒ तम् तं ॅयज॑ते॒ यज॑ते॒ तम् । \newline
21. त म॒भ्य॑भि तम् त म॒भि । \newline
22. अ॒भि नि न्या᳚(1॒)भ्य॑भि नि । \newline
23. नि व॑र्तते वर्तते॒ नि नि व॑र्तते । \newline
24. व॒र्त॒ते॒ य॒ज्ञो य॒ज्ञो व॑र्तते वर्तते य॒ज्ञ्ः । \newline
25. य॒ज्ञो ब॑भूव बभूव य॒ज्ञो य॒ज्ञो ब॑भूव । \newline
26. ब॒भू॒व॒ स स ब॑भूव बभूव॒ सः । \newline
27. स आ स स आ । \newline
28. आ ब॑भूव बभू॒वा ब॑भूव । \newline
29. ब॒भू॒वे तीति॑ बभूव बभू॒वे ति॑ । \newline
30. इत्या॑ हा॒हे तीत्या॑ह । \newline
31. आ॒है॒ष ए॒ष आ॑हा है॒षः । \newline
32. ए॒ष वै वा ए॒ष ए॒ष वै । \newline
33. वै य॒ज्ञ्स्य॑ य॒ज्ञ्स्य॒ वै वै य॒ज्ञ्स्य॑ । \newline
34. य॒ज्ञ्स्य॑ पुनरालं॒भः पु॑नरालं॒भो य॒ज्ञ्स्य॑ य॒ज्ञ्स्य॑ पुनरालं॒भः । \newline
35. पु॒न॒रा॒लं॒भ स्तेन॒ तेन॑ पुनरालं॒भः पु॑नरालं॒भ स्तेन॑ । \newline
36. पु॒न॒रा॒लं॒भ इति॑ पुनः - आ॒लं॒भः । \newline
37. तेनै॒ वैव तेन॒ तेनै॒व । \newline
38. ए॒वैन॑ मेन मे॒वै वैन᳚म् । \newline
39. ए॒न॒म् पुनः॒ पुन॑ रेन मेन॒म् पुनः॑ । \newline
40. पुन॒ रा पुनः॒ पुन॒ रा । \newline
41. आ ल॑भते लभत॒ आ ल॑भते । \newline
42. ल॒भ॒ते ऽन॑वरु॒द्धा ऽन॑वरुद्धा लभते लभ॒ते ऽन॑वरुद्धा । \newline
43. अन॑वरुद्धा॒ वै वा अन॑वरु॒द्धा ऽन॑वरुद्धा॒ वै । \newline
44. अन॑वरु॒द्धेत्यन॑व - रु॒द्धा॒ । \newline
45. वा ए॒त स्यै॒तस्य॒ वै वा ए॒तस्य॑ । \newline
46. ए॒तस्य॑ वि॒राड् वि॒रा डे॒त स्यै॒तस्य॑ वि॒राट् । \newline
47. वि॒राड् यो यो वि॒राड् वि॒राड् यः । \newline
48. वि॒राडिति॑ वि - राट् । \newline
49. य आहि॑ताग्नि॒ राहि॑ताग्नि॒र् यो य आहि॑ताग्निः । \newline
50. आहि॑ताग्निः॒ सन् थ्सन् नाहि॑ताग्नि॒ राहि॑ताग्निः॒ सन्न् । \newline
51. आहि॑ताग्नि॒रित्याहि॑त - अ॒ग्निः॒ । \newline
52. सन् न॑स॒भो॑ ऽस॒भः सन् थ्सन् न॑स॒भः । \newline
53. अ॒स॒भः प॒शवः॑ प॒शवो॑ ऽस॒भो॑ ऽस॒भः प॒शवः॑ । \newline
54. प॒शवः॒ खलु॒ खलु॑ प॒शवः॑ प॒शवः॒ खलु॑ । \newline
55. खलु॒ वै वै खलु॒ खलु॒ वै । \newline
56. वै ब्रा᳚ह्म॒णस्य॑ ब्राह्म॒णस्य॒ वै वै ब्रा᳚ह्म॒णस्य॑ । \newline
57. ब्रा॒ह्म॒णस्य॑ स॒भा स॒भा ब्रा᳚ह्म॒णस्य॑ ब्राह्म॒णस्य॑ स॒भा । \newline
58. स॒भेष्ट्वे ष्ट्वा स॒भा स॒भेष्ट्वा । \newline
59. इ॒ष्ट्वा प्राङ् प्रा ङि॒ष्ट्वे ष्ट्वा प्राङ् । \newline
60. प्रा ङु॒त्क्रम्यो॒ त्क्रम्य॒ प्राङ् प्रा ङु॒त्क्रम्य॑ । \newline
61. उ॒त्क्रम्य॑ ब्रूयाद् ब्रूया दु॒त्क्रम्यो॒ त्क्रम्य॑ ब्रूयात् । \newline
62. उ॒त्क्रम्येत्यु॑त् - क्रम्य॑ । \newline
63. ब्रू॒या॒द् गोमा॒न् गोमा᳚न् ब्रूयाद् ब्रूया॒द् गोमान्॑ । \newline
64. गोमाꣳ॑ अग्ने ऽग्ने॒ गोमा॒न् गोमाꣳ॑ अग्ने । \newline
65. गोमा॒निति॒ गो - मा॒न् । \newline
66. अ॒ग्ने ऽवि॑माꣳ॒॒ अवि॑माꣳ अग्ने॒ ऽग्ने ऽवि॑मान् । \newline
67. अवि॑माꣳ अ॒श्व्य॑ श्व्यवि॑माꣳ॒॒ अवि॑माꣳ अ॒श्वी । \newline
68. अवि॑मा॒नित्यवि॑ - मा॒न् । \newline
69. अ॒श्वी य॒ज्ञो य॒ज्ञो᳚(1॒) ऽश्व्य॑श्वी य॒ज्ञ्ः । \newline
70. य॒ज्ञ् इतीति॑ य॒ज्ञो य॒ज्ञ् इति॑ । \newline
71. इत्यवावे तीत्यव॑ । \newline
72. अव॑ स॒भाꣳ स॒भा मवाव॑ स॒भाम् । \newline
73. स॒भाꣳ रु॒न्धे रु॒न्धे स॒भाꣳ स॒भाꣳ रु॒न्धे । \newline
74. रु॒न्धे प्र प्र रु॒न्धे रु॒न्धे प्र । \newline
75. प्र स॒हस्रꣳ॑ स॒हस्र॒म् प्र प्र स॒हस्र᳚म् । \newline
76. स॒हस्र॑म् प॒शून् प॒शून् थ्स॒हस्रꣳ॑ स॒हस्र॑म् प॒शून् । \newline
77. प॒शू ना᳚प्नो त्याप्नोति प॒शून् प॒शू ना᳚प्नोति । \newline
78. आ॒प्नो॒त्या ऽऽप्नो᳚ त्याप्नो॒त्या । \newline
79. आ ऽस्या॒स्या ऽस्य॑ । \newline
80. अ॒स्य॒ प्र॒जाया᳚म् प्र॒जाया॑ मस्यास्य प्र॒जाया᳚म् । \newline
81. प्र॒जायां᳚ ॅवा॒जी वा॒जी प्र॒जाया᳚म् प्र॒जायां᳚ ॅवा॒जी । \newline
82. प्र॒जाया॒मिति॑ प्र - जाया᳚म् । \newline
83. वा॒जी जा॑यते जायते वा॒जी वा॒जी जा॑यते । \newline
84. जा॒य॒त॒ इति॑ जायते । \newline

\textbf{Ghana Paata } \newline

1. वि सृ॑जते सृजते॒ वि वि सृ॑जते॒ शान्त्यै॒ शान्त्यै॑ सृजते॒ वि वि सृ॑जते॒ शान्त्यै᳚ । \newline
2. सृ॒ज॒ते॒ शान्त्यै॒ शान्त्यै॑ सृजते सृजते॒ शान्त्या॒ अप्र॑दाहा॒या प्र॑दाहाय॒ शान्त्यै॑ सृजते सृजते॒ शान्त्या॒ अप्र॑दाहाय । \newline
3. शान्त्या॒ अप्र॑दाहा॒या प्र॑दाहाय॒ शान्त्यै॒ शान्त्या॒ अप्र॑दाहाय॒ परा॒ङ् परा॒ङप्र॑दाहाय॒ शान्त्यै॒ शान्त्या॒ अप्र॑दाहाय॒ पराङ्॑ । \newline
4. अप्र॑दाहाय॒ परा॒ङ् परा॒ ङप्र॑दाहा॒या प्र॑दाहाय॒ परा॒ङ्. वाव वाव परा॒ ङप्र॑दाहा॒या प्र॑दाहाय॒ परा॒ङ्. वाव । \newline
5. अप्र॑दाहा॒येत्यप्र॑ - दा॒हा॒य॒ । \newline
6. परा॒ङ्. वाव वाव परा॒ङ् परा॒ङ्. वाव य॒ज्ञो य॒ज्ञो वाव परा॒ङ् परा॒ङ्. वाव य॒ज्ञ्ः । \newline
7. वाव य॒ज्ञो य॒ज्ञो वाव वाव य॒ज्ञ् ए᳚त्येति य॒ज्ञो वाव वाव य॒ज्ञ् ए॑ति । \newline
8. य॒ज्ञ् ए᳚त्येति य॒ज्ञो य॒ज्ञ् ए॑ति॒ न नैति॑ य॒ज्ञो य॒ज्ञ् ए॑ति॒ न । \newline
9. ए॒ति॒ न नैत्ये॑ति॒ न नि नि नैत्ये॑ति॒ न नि । \newline
10. न नि नि न न नि व॑र्तते वर्तते॒ नि न न नि व॑र्तते । \newline
11. नि व॑र्तते वर्तते॒ नि नि व॑र्तते॒ पुनः॒ पुन॑र् वर्तते॒ नि नि व॑र्तते॒ पुनः॑ । \newline
12. व॒र्त॒ते॒ पुनः॒ पुन॑र् वर्तते वर्तते॒ पुन॒र् यो यः पुन॑र् वर्तते वर्तते॒ पुन॒र् यः । \newline
13. पुन॒र् यो यः पुनः॒ पुन॒र् यो वै वै यः पुनः॒ पुन॒र् यो वै । \newline
14. यो वै वै यो यो वै य॒ज्ञ्स्य॑ य॒ज्ञ्स्य॒ वै यो यो वै य॒ज्ञ्स्य॑ । \newline
15. वै य॒ज्ञ्स्य॑ य॒ज्ञ्स्य॒ वै वै य॒ज्ञ्स्य॑ पुनरालं॒भम् पु॑नरालं॒भं ॅय॒ज्ञ्स्य॒ वै वै य॒ज्ञ्स्य॑ पुनरालं॒भम् । \newline
16. य॒ज्ञ्स्य॑ पुनरालं॒भम् पु॑नरालं॒भं ॅय॒ज्ञ्स्य॑ य॒ज्ञ्स्य॑ पुनरालं॒भं ॅवि॒द्वान्. वि॒द्वान् पु॑नरालं॒भं ॅय॒ज्ञ्स्य॑ य॒ज्ञ्स्य॑ पुनरालं॒भं ॅवि॒द्वान् । \newline
17. पु॒न॒रा॒लं॒भं ॅवि॒द्वान्. वि॒द्वान् पु॑नरालं॒भम् पु॑नरालं॒भं ॅवि॒द्वान्. यज॑ते॒ यज॑ते वि॒द्वान् पु॑नरालं॒भम् पु॑नरालं॒भं ॅवि॒द्वान्. यज॑ते । \newline
18. पु॒न॒रा॒लं॒भमिति॑ पुनः - आ॒लं॒भम् । \newline
19. वि॒द्वान्. यज॑ते॒ यज॑ते वि॒द्वान्. वि॒द्वान्. यज॑ते॒ तम् तं ॅयज॑ते वि॒द्वान्. वि॒द्वान्. यज॑ते॒ तम् । \newline
20. यज॑ते॒ तम् तं ॅयज॑ते॒ यज॑ते॒ त म॒भ्य॑भि तं ॅयज॑ते॒ यज॑ते॒ त म॒भि । \newline
21. त म॒भ्य॑भि तम् त म॒भि नि न्य॑भि तम् त म॒भि नि । \newline
22. अ॒भि नि न्या᳚(1॒)भ्य॑भि नि व॑र्तते वर्तते॒ न्या᳚(1॒)भ्य॑भि नि व॑र्तते । \newline
23. नि व॑र्तते वर्तते॒ नि नि व॑र्तते य॒ज्ञो य॒ज्ञो व॑र्तते॒ नि नि व॑र्तते य॒ज्ञ्ः । \newline
24. व॒र्त॒ते॒ य॒ज्ञो य॒ज्ञो व॑र्तते वर्तते य॒ज्ञो ब॑भूव बभूव य॒ज्ञो व॑र्तते वर्तते य॒ज्ञो ब॑भूव । \newline
25. य॒ज्ञो ब॑भूव बभूव य॒ज्ञो य॒ज्ञो ब॑भूव॒ स स ब॑भूव य॒ज्ञो य॒ज्ञो ब॑भूव॒ सः । \newline
26. ब॒भू॒व॒ स स ब॑भूव बभूव॒ स आ स ब॑भूव बभूव॒ स आ । \newline
27. स आ स स आ ब॑भूव बभू॒वा स स आ ब॑भूव । \newline
28. आ ब॑भूव बभू॒वा ब॑भू॒वे तीति॑ बभू॒वा ब॑भू॒वे ति॑ । \newline
29. ब॒भू॒वे तीति॑ बभूव बभू॒वे त्या॑हा॒हे ति॑ बभूव बभू॒वे त्या॑ह । \newline
30. इत्या॑हा॒हे तीत्या॑है॒ष ए॒ष आ॒हे तीत्या॑है॒षः । \newline
31. आ॒है॒ष ए॒ष आ॑हाहै॒ष वै वा ए॒ष आ॑हाहै॒ष वै । \newline
32. ए॒ष वै वा ए॒ष ए॒ष वै य॒ज्ञ्स्य॑ य॒ज्ञ्स्य॒ वा ए॒ष ए॒ष वै य॒ज्ञ्स्य॑ । \newline
33. वै य॒ज्ञ्स्य॑ य॒ज्ञ्स्य॒ वै वै य॒ज्ञ्स्य॑ पुनरालं॒भः पु॑नरालं॒भो य॒ज्ञ्स्य॒ वै वै य॒ज्ञ्स्य॑ पुनरालं॒भः । \newline
34. य॒ज्ञ्स्य॑ पुनरालं॒भः पु॑नरालं॒भो य॒ज्ञ्स्य॑ य॒ज्ञ्स्य॑ पुनरालं॒भ स्तेन॒ तेन॑ पुनरालं॒भो य॒ज्ञ्स्य॑ य॒ज्ञ्स्य॑ पुनरालं॒भ स्तेन॑ । \newline
35. पु॒न॒रा॒लं॒भ स्तेन॒ तेन॑ पुनरालं॒भः पु॑नरालं॒भ स्ते नै॒वैव तेन॑ पुनरालं॒भः पु॑नरालं॒भ स्तेनै॒व । \newline
36. पु॒न॒रा॒लं॒भ इति॑ पुनः - आ॒लं॒भः । \newline
37. तेनै॒वैव तेन॒ तेनै॒वैन॑ मेन मे॒व तेन॒ तेनै॒वैन᳚म् । \newline
38. ए॒वैन॑ मेन मे॒वै वैन॒म् पुनः॒ पुन॑ रेन मे॒वै वैन॒म् पुनः॑ । \newline
39. ए॒न॒म् पुनः॒ पुन॑ रेन मेन॒म् पुन॒ रा पुन॑ रेन मेन॒म् पुन॒ रा । \newline
40. पुन॒ रा पुनः॒ पुन॒ रा ल॑भते लभत॒ आ पुनः॒ पुन॒ रा ल॑भते । \newline
41. आ ल॑भते लभत॒ आ ल॑भ॒ते ऽन॑वरु॒द्धा ऽन॑वरुद्धा लभत॒ आ ल॑भ॒ते ऽन॑वरुद्धा । \newline
42. ल॒भ॒ते ऽन॑वरु॒द्धा ऽन॑वरुद्धा लभते लभ॒ते ऽन॑वरुद्धा॒ वै वा अन॑वरुद्धा लभते लभ॒ते ऽन॑वरुद्धा॒ वै । \newline
43. अन॑वरुद्धा॒ वै वा अन॑वरु॒द्धा ऽन॑वरुद्धा॒ वा ए॒त स्यै॒तस्य॒ वा अन॑वरु॒द्धा ऽन॑वरुद्धा॒ वा ए॒तस्य॑ । \newline
44. अन॑वरु॒द्धेत्यन॑व - रु॒द्धा॒ । \newline
45. वा ए॒त स्यै॒तस्य॒ वै वा ए॒तस्य॑ वि॒राड् वि॒रा डे॒तस्य॒ वै वा ए॒तस्य॑ वि॒राट् । \newline
46. ए॒तस्य॑ वि॒राड् वि॒रा डे॒त स्यै॒तस्य॑ वि॒राड् यो यो वि॒रा डे॒त स्यै॒तस्य॑ वि॒राड् यः । \newline
47. वि॒राड् यो यो वि॒राड् वि॒राड् य आहि॑ताग्नि॒ राहि॑ताग्नि॒र् यो वि॒राड् वि॒राड् य आहि॑ताग्निः । \newline
48. वि॒राडिति॑ वि - राट् । \newline
49. य आहि॑ताग्नि॒ राहि॑ताग्नि॒र् यो य आहि॑ताग्निः॒ सन् थ्सन् नाहि॑ताग्नि॒र् यो य आहि॑ताग्निः॒ सन्न् । \newline
50. आहि॑ताग्निः॒ सन् थ्सन् नाहि॑ताग्नि॒ राहि॑ताग्निः॒ सन् न॑स॒भो॑ ऽस॒भः सन् नाहि॑ताग्नि॒ राहि॑ताग्निः॒ सन् न॑स॒भः । \newline
51. आहि॑ताग्नि॒रित्याहि॑त - अ॒ग्निः॒ । \newline
52. सन् न॑स॒भो॑ ऽस॒भः सन् थ्सन् न॑स॒भः प॒शवः॑ प॒शवो॑ ऽस॒भः सन् थ्सन् न॑स॒भः प॒शवः॑ । \newline
53. अ॒स॒भः प॒शवः॑ प॒शवो॑ ऽस॒भो॑ ऽस॒भः प॒शवः॒ खलु॒ खलु॑ प॒शवो॑ ऽस॒भो॑ ऽस॒भः प॒शवः॒ खलु॑ । \newline
54. प॒शवः॒ खलु॒ खलु॑ प॒शवः॑ प॒शवः॒ खलु॒ वै वै खलु॑ प॒शवः॑ प॒शवः॒ खलु॒ वै । \newline
55. खलु॒ वै वै खलु॒ खलु॒ वै ब्रा᳚ह्म॒णस्य॑ ब्राह्म॒णस्य॒ वै खलु॒ खलु॒ वै ब्रा᳚ह्म॒णस्य॑ । \newline
56. वै ब्रा᳚ह्म॒णस्य॑ ब्राह्म॒णस्य॒ वै वै ब्रा᳚ह्म॒णस्य॑ स॒भा स॒भा ब्रा᳚ह्म॒णस्य॒ वै वै ब्रा᳚ह्म॒णस्य॑ स॒भा । \newline
57. ब्रा॒ह्म॒णस्य॑ स॒भा स॒भा ब्रा᳚ह्म॒णस्य॑ ब्राह्म॒णस्य॑ स॒भे ष्ट्वे ष्ट्वा स॒भा ब्रा᳚ह्म॒णस्य॑ ब्राह्म॒णस्य॑ स॒भे ष्ट्वा । \newline
58. स॒भे ष्ट्वे ष्ट्वा स॒भा स॒भेष्ट्वा प्राङ् प्रा ङि॒ष्ट्वा स॒भा स॒भेष्ट्वा प्राङ् । \newline
59. इ॒ष्ट्वा प्राङ् प्रा ङि॒ष्ट्वे ष्ट्वा प्रा ङु॒त्क्रम्यो॒त् क्रम्य॒ प्रा ङि॒ष्ट्वे ष्ट्वा प्रा ङु॒त्क्रम्य॑ । \newline
60. प्रा ङु॒त्क्रम्यो॒त् क्रम्य॒ प्राङ् प्रा ङु॒त्क्रम्य॑ ब्रूयाद् ब्रूया दु॒त्क्रम्य॒ प्राङ् प्रा ङु॒त्क्रम्य॑ ब्रूयात् । \newline
61. उ॒त्क्रम्य॑ ब्रूयाद् ब्रूया दु॒त्क्रम्यो॒त् क्रम्य॑ ब्रूया॒द् गोमा॒न् गोमा᳚न् ब्रूया दु॒त्क्रम्यो॒त् क्रम्य॑ ब्रूया॒द् गोमान्॑ । \newline
62. उ॒त्क्रम्येत्यु॑त् - क्रम्य॑ । \newline
63. ब्रू॒या॒द् गोमा॒न् गोमा᳚न् ब्रूयाद् ब्रूया॒द् गोमा(ग्म्॑) अग्ने ऽग्ने॒ गोमा᳚न् ब्रूयाद् ब्रूया॒द् गोमा(ग्म्॑) अग्ने । \newline
64. गोमा(ग्म्॑) अग्ने ऽग्ने॒ गोमा॒न् गोमा(ग्म्॑) अ॒ग्ने ऽवि॑मा॒(ग्म्॒) अवि॑माꣳ अग्ने॒ गोमा॒न् गोमा(ग्म्॑) अ॒ग्ने ऽवि॑मान् । \newline
65. गोमा॒निति॒ गो - मा॒न् । \newline
66. अ॒ग्ने ऽवि॑मा॒(ग्म्॒) अवि॑माꣳ अग्ने॒ ऽग्ने ऽवि॑माꣳ अ॒श्व्य॑ श्व्यवि॑माꣳ अग्ने॒ ऽग्ने ऽवि॑माꣳ अ॒श्वी । \newline
67. अवि॑माꣳ अ॒श्व्य॑ श्व्यवि॑मा॒(ग्म्॒) अवि॑माꣳ अ॒श्वी य॒ज्ञो य॒ज्ञो᳚ ऽश्व्यवि॑मा॒(ग्म्॒) अवि॑माꣳ अ॒श्वी य॒ज्ञ्ः । \newline
68. अवि॑मा॒नित्यवि॑ - मा॒न् । \newline
69. अ॒श्वी य॒ज्ञो य॒ज्ञो᳚(1॒) ऽश्व्य॑श्वी य॒ज्ञ् इतीति॑ य॒ज्ञो᳚(1॒) ऽश्व्य॑श्वी य॒ज्ञ् इति॑ । \newline
70. य॒ज्ञ् इतीति॑ य॒ज्ञो य॒ज्ञ् इत्यवावे ति॑ य॒ज्ञो य॒ज्ञ् इत्यव॑ । \newline
71. इत्यवावे तीत्यव॑ स॒भाꣳ स॒भा मवे तीत्यव॑ स॒भाम् । \newline
72. अव॑ स॒भाꣳ स॒भा मवाव॑ स॒भाꣳ रु॒न्धे रु॒न्धे स॒भा मवाव॑ स॒भाꣳ रु॒न्धे । \newline
73. स॒भाꣳ रु॒न्धे रु॒न्धे स॒भाꣳ स॒भाꣳ रु॒न्धे प्र प्र रु॒न्धे स॒भाꣳ स॒भाꣳ रु॒न्धे प्र । \newline
74. रु॒न्धे प्र प्र रु॒न्धे रु॒न्धे प्र स॒हस्र(ग्म्॑) स॒हस्र॒म् प्र रु॒न्धे रु॒न्धे प्र स॒हस्र᳚म् । \newline
75. प्र स॒हस्र(ग्म्॑) स॒हस्र॒म् प्र प्र स॒हस्र॑म् प॒शून् प॒शून् थ्स॒हस्र॒म् प्र प्र स॒हस्र॑म् प॒शून् । \newline
76. स॒हस्र॑म् प॒शून् प॒शून् थ्स॒हस्र(ग्म्॑) स॒हस्र॑म् प॒शू ना᳚प्नो त्याप्नोति प॒शून् थ्स॒हस्र(ग्म्॑) स॒हस्र॑म् प॒शू ना᳚प्नोति । \newline
77. प॒शू ना᳚प्नो त्याप्नोति प॒शून् प॒शू ना᳚प्नो॒त्या ऽऽप्नो॑ति प॒शून् प॒शू ना᳚प्नो॒त्या । \newline
78. आ॒प्नो॒त्या ऽऽप्नो᳚ त्याप्नो॒त्या ऽस्या॒स्या ऽऽप्नो᳚ त्याप्नो॒त्या ऽस्य॑ । \newline
79. आ ऽस्या॒स्या ऽस्य॑ प्र॒जाया᳚म् प्र॒जाया॑ म॒स्या ऽस्य॑ प्र॒जाया᳚म् । \newline
80. अ॒स्य॒ प्र॒जाया᳚म् प्र॒जाया॑ मस्यास्य प्र॒जायां᳚ ॅवा॒जी वा॒जी प्र॒जाया॑ मस्यास्य प्र॒जायां᳚ ॅवा॒जी । \newline
81. प्र॒जायां᳚ ॅवा॒जी वा॒जी प्र॒जाया᳚म् प्र॒जायां᳚ ॅवा॒जी जा॑यते जायते वा॒जी प्र॒जाया᳚म् प्र॒जायां᳚ ॅवा॒जी जा॑यते । \newline
82. प्र॒जाया॒मिति॑ प्र - जाया᳚म् । \newline
83. वा॒जी जा॑यते जायते वा॒जी वा॒जी जा॑यते । \newline
84. जा॒य॒त॒ इति॑ जायते । \newline
\pagebreak
\markright{ TS 1.7.7.1  \hfill https://www.vedavms.in \hfill}
\addcontentsline{toc}{section}{ TS 1.7.7.1 }
\section*{ TS 1.7.7.1 }

\textbf{TS 1.7.7.1 } \newline
\textbf{Samhita Paata} \newline

देव॑ सवितः॒ प्र सु॑व य॒ज्ञ्ं प्र सु॑व य॒ज्ञ्प॑तिं॒ भगा॑य दि॒व्यो ग॑न्ध॒र्वः । के॒त॒पूः केतं॑ नः पुनातु वा॒चस्पति॒र् वाच॑म॒द्य स्व॑दाति नः ॥ इन्द्र॑स्य॒ वज्रो॑ऽसि॒ वार्त्र॑घ्न॒स्त्वया॒ऽयं ॅवृ॒त्रं ॅव॑द्ध्यात् ॥ वाज॑स्य॒ नु प्र॑स॒वे मा॒तरं॑ म॒हीमदि॑तिं॒ नाम॒ वच॑सा करामहे । यस्या॑मि॒दं ॅविश्वं॒ भुव॑न-मावि॒वेश॒ तस्यां᳚ नो दे॒वः स॑वि॒ता धर्म॑ साविषत् ॥ अ॒फ्स्व॑-[ ] \newline

\textbf{Pada Paata} \newline

देव॑ । स॒वि॒तः॒ । प्रेति॑ । सु॒व॒ । य॒ज्ञ्म् । प्रेति॑ । सु॒व॒ । य॒ज्ञ्प॑ति॒मिति॑ य॒ज्ञ् - प॒ति॒म् । भगा॑य । दि॒व्यः । ग॒न्ध॒र्वः ॥ के॒त॒पूरिति॑ केत - पूः । केत᳚म् । नः॒ । पु॒ना॒तु॒ । वा॒चः । पतिः॑ । वाच᳚म् । अ॒द्य । स्व॒दा॒ति॒ । नः॒ ॥ इन्द्र॑स्य । वज्रः॑ । अ॒सि॒ । वार्त्र॑घ्न॒ इति॒ वार्त्र॑ - घ्नः॒ । त्वया᳚ । अ॒यम् । वृ॒त्रम् । व॒द्ध्या॒त् ॥ वाज॑स्य । नु । प्र॒स॒व इति॑ प्र - स॒वे । मा॒तर᳚म् । म॒हीम् । अदि॑तिम् । नाम॑ । वच॑सा । क॒रा॒म॒हे॒ ॥ यस्या᳚म् । इ॒दम् । विश्व᳚म् । भुव॑नम् । आ॒वि॒वेशेत्या᳚ - वि॒वेश॑ । तस्या᳚म् । नः॒ । दे॒वः । स॒वि॒ता । धर्म॑ । सा॒वि॒ष॒त् ॥ अ॒फ्स्वित्य॑प् - सु ।  \newline


\textbf{Krama Paata} \newline

देव॑ सवितः । स॒वि॒तः॒ प्र । प्र सु॑व । सु॒व॒ य॒ज्ञ्म् । य॒ज्ञ्म् प्र । प्र सु॑व । सु॒व॒ य॒ज्ञ्प॑तिम् । य॒ज्ञ्प॑ति॒म् भगा॑य । य॒ज्ञ्प॑ति॒मिति॑ य॒ज्ञ् - प॒ति॒म् । भगा॑य दि॒व्यः । दि॒व्यो ग॑न्ध॒र्वः । ग॒न्ध॒र्व इति॑ गन्ध॒र्वः ॥ के॒त॒पूः केत᳚म् । के॒त॒पूरिति॑ केत - पूः । केत॑म् नः । नः॒ पु॒ना॒तु॒ । पु॒ना॒तु॒ वा॒चः । वा॒चस्पतिः॑ । पति॒र्,वाच᳚म् । वाच॑म॒द्य । अ॒द्य स्व॑दाति । स्व॒दा॒ति॒ नः॒ । न॒ इति॑ नः ॥ इन्द्र॑स्य॒ वज्रः॑ । वज्रो॑ऽसि । अ॒सि॒ वार्त्र॑घ्नः । वार्त्र॑घ्न॒ स्त्वया᳚ । वार्त्र॑घ्न॒ इति॒ वार्त्र॑ - घ्नः॒ । त्वया॒ऽयम् । अ॒यं ॅवृ॒त्रम् । वृ॒त्रं ॅव॑द्ध्यात् । व॒द्ध्या॒दिति॑ वद्ध्यात् ॥ वाज॑स्य॒ नु । नु प्र॑स॒वे । प्र॒स॒वे मा॒तर᳚म् । प्र॒स॒व इति॑ प्र - स॒वे । मा॒तर॑म् म॒हीम् । म॒हीमदि॑तिम् । अदि॑ति॒म् नाम॑ । नाम॒ वच॑सा । वच॑सा करामहे । क॒रा॒म॒ह॒ इति॑ करामहे ॥ यस्या॑मि॒दम् । इ॒दं ॅविश्व᳚म् । विश्व॒म् भुव॑नम् । भुव॑नमावि॒वेश॑ । आ॒वि॒वेश॒ तस्या᳚म् । आ॒वि॒वेशेत्या᳚ - वि॒वेश॑ । तस्या᳚म् नः । नो॒ दे॒वः । दे॒वः स॑वि॒ता । स॒वि॒ता धर्म॑ । धर्म॑ साविषत् । सा॒वि॒ष॒दिति॑ साविषत् ॥ अ॒फ्स्व॑न्तः । अ॒फ्स्वित्य॑प् - सु \newline

\textbf{Jatai Paata} \newline

1. देव॑ सवितः सवित॒र् देव॒ देव॑ सवितः । \newline
2. स॒वि॒तः॒ प्र प्र स॑वितः सवितः॒ प्र । \newline
3. प्र सु॑व सुव॒ प्र प्र सु॑व । \newline
4. सु॒व॒ य॒ज्ञ्ं ॅय॒ज्ञ्ꣳ सु॑व सुव य॒ज्ञ्म् । \newline
5. य॒ज्ञ्म् प्र प्र य॒ज्ञ्ं ॅय॒ज्ञ्म् प्र । \newline
6. प्र सु॑व सुव॒ प्र प्र सु॑व । \newline
7. सु॒व॒ य॒ज्ञ्प॑तिं ॅय॒ज्ञ्प॑तिꣳ सुव सुव य॒ज्ञ्प॑तिम् । \newline
8. य॒ज्ञ्प॑ति॒म् भगा॑य॒ भगा॑य य॒ज्ञ्प॑तिं ॅय॒ज्ञ्प॑ति॒म् भगा॑य । \newline
9. य॒ज्ञ्प॑ति॒मिति॑ य॒ज्ञ् - प॒ति॒म् । \newline
10. भगा॑य दि॒व्यो दि॒व्यो भगा॑य॒ भगा॑य दि॒व्यः । \newline
11. दि॒व्यो ग॑न्ध॒र्वो ग॑न्ध॒र्वो दि॒व्यो दि॒व्यो ग॑न्ध॒र्वः । \newline
12. ग॒न्ध॒र्व इति॑ गन्ध॒र्वः । \newline
13. के॒त॒पूः केत॒म् केत॑म् केत॒पूः के॑त॒पूः केत᳚म् । \newline
14. के॒त॒पूरिति॑ केत - पूः । \newline
15. केत॑न्नो नः॒ केत॒म् केत॑न्नः । \newline
16. नः॒ पु॒ना॒तु॒ पु॒ना॒तु॒ नो॒ नः॒ पु॒ना॒तु॒ । \newline
17. पु॒ना॒तु॒ वा॒चो वा॒चः पु॑नातु पुनातु वा॒चः । \newline
18. वा॒च स्पति॒ष् पति॑र् वा॒चो वा॒च स्पतिः॑ । \newline
19. पति॒र् वाचं॒ ॅवाच॒म् पति॒ष् पति॒र् वाच᳚म् । \newline
20. वाच॑ म॒द्याद्य वाचं॒ ॅवाच॑ म॒द्य । \newline
21. अ॒द्य स्व॑दाति स्वदा त्य॒द्याद्य स्व॑दाति । \newline
22. स्व॒दा॒ति॒ नो॒ नः॒ स्व॒दा॒ति॒ स्व॒दा॒ति॒ नः॒ । \newline
23. न॒ इति॑ नः । \newline
24. इन्द्र॑स्य॒ वज्रो॒ वज्र॒ इन्द्र॒स्ये न्द्र॑स्य॒ वज्रः॑ । \newline
25. वज्रो᳚ ऽस्यसि॒ वज्रो॒ वज्रो॑ ऽसि । \newline
26. अ॒सि॒ वार्त्र॑घ्नो॒ वार्त्र॑घ्नो ऽस्यसि॒ वार्त्र॑घ्नः । \newline
27. वार्त्र॑घ्न॒ स्त्वया॒ त्वया॒ वार्त्र॑घ्नो॒ वार्त्र॑घ्न॒ स्त्वया᳚ । \newline
28. वार्त्र॑घ्न॒ इति॒ वार्त्र॑ - घ्नः॒ । \newline
29. त्वया॒ ऽय म॒यम् त्वया॒ त्वया॒ ऽयम् । \newline
30. अ॒यं ॅवृ॒त्रं ॅवृ॒त्र म॒य म॒यं ॅवृ॒त्रम् । \newline
31. वृ॒त्रं ॅव॑द्ध्याद् वद्ध्याद् वृ॒त्रं ॅवृ॒त्रं ॅव॑द्ध्यात् । \newline
32. व॒द्ध्या॒दिति॑ वद्ध्यात् । \newline
33. वाज॑स्य॒ नु नु वाज॑स्य॒ वाज॑स्य॒ नु । \newline
34. नु प्र॑स॒वे प्र॑स॒वे नु नु प्र॑स॒वे । \newline
35. प्र॒स॒वे मा॒तर॑म् मा॒तर॑म् प्रस॒वे प्र॑स॒वे मा॒तर᳚म् । \newline
36. प्र॒स॒व इति॑ प्र - स॒वे । \newline
37. मा॒तर॑म् म॒हीम् म॒हीम् मा॒तर॑म् मा॒तर॑म् म॒हीम् । \newline
38. म॒ही मदि॑ति॒ मदि॑तिम् म॒हीम् म॒ही मदि॑तिम् । \newline
39. अदि॑ति॒न् नाम॒ नामादि॑ति॒ मदि॑ति॒न् नाम॑ । \newline
40. नाम॒ वच॑सा॒ वच॑सा॒ नाम॒ नाम॒ वच॑सा । \newline
41. वच॑सा करामहे करामहे॒ वच॑सा॒ वच॑सा करामहे । \newline
42. क॒रा॒म॒ह॒ इति॑ करामहे । \newline
43. यस्या॑ मि॒द मि॒दं ॅयस्यां॒ ॅयस्या॑ मि॒दम् । \newline
44. इ॒दं ॅविश्वं॒ ॅविश्व॑ मि॒द मि॒दं ॅविश्व᳚म् । \newline
45. विश्व॒म् भुव॑न॒म् भुव॑नं॒ ॅविश्वं॒ ॅविश्व॒म् भुव॑नम् । \newline
46. भुव॑न मावि॒वेशा॑ वि॒वेश॒ भुव॑न॒म् भुव॑न मावि॒वेश॑ । \newline
47. आ॒वि॒वेश॒ तस्या॒म् तस्या॑ मावि॒वेशा॑ वि॒वेश॒ तस्या᳚म् । \newline
48. आ॒वि॒वेशेत्या᳚ - वि॒वेश॑ । \newline
49. तस्या᳚न् नो न॒स्तस्या॒म् तस्या᳚न् नः । \newline
50. नो॒ दे॒वो दे॒वो नो॑ नो दे॒वः । \newline
51. दे॒वः स॑वि॒ता स॑वि॒ता दे॒वो दे॒वः स॑वि॒ता । \newline
52. स॒वि॒ता धर्म॒ धर्म॑ सवि॒ता स॑वि॒ता धर्म॑ । \newline
53. धर्म॑ साविषथ् साविष॒द् धर्म॒ धर्म॑ साविषत् । \newline
54. सा॒वि॒ष॒दिति॑ साविषत् । \newline
55. अ॒फ्स्व॑न्त र॒न्त र॒फ्स्वा᳚(1॒)फ्स्व॑न्तः । \newline
56. अ॒फ्स्वित्य॑प् - सु । \newline

\textbf{Ghana Paata } \newline

1. देव॑ सवितः सवित॒र् देव॒ देव॑ सवितः॒ प्र प्र स॑वित॒र् देव॒ देव॑ सवितः॒ प्र । \newline
2. स॒वि॒तः॒ प्र प्र स॑वितः सवितः॒ प्र सु॑व सुव॒ प्र स॑वितः सवितः॒ प्र सु॑व । \newline
3. प्र सु॑व सुव॒ प्र प्र सु॑व य॒ज्ञ्ं ॅय॒ज्ञ्ꣳ सु॑व॒ प्र प्र सु॑व य॒ज्ञ्म् । \newline
4. सु॒व॒ य॒ज्ञ्ं ॅय॒ज्ञ्ꣳ सु॑व सुव य॒ज्ञ्म् प्र प्र य॒ज्ञ्ꣳ सु॑व सुव य॒ज्ञ्म् प्र । \newline
5. य॒ज्ञ्म् प्र प्र य॒ज्ञ्ं ॅय॒ज्ञ्म् प्र सु॑व सुव॒ प्र य॒ज्ञ्ं ॅय॒ज्ञ्म् प्र सु॑व । \newline
6. प्र सु॑व सुव॒ प्र प्र सु॑व य॒ज्ञ्प॑तिं ॅय॒ज्ञ्प॑तिꣳ सुव॒ प्र प्र सु॑व य॒ज्ञ्प॑तिम् । \newline
7. सु॒व॒ य॒ज्ञ्प॑तिं ॅय॒ज्ञ्प॑तिꣳ सुव सुव य॒ज्ञ्प॑ति॒म् भगा॑य॒ भगा॑य य॒ज्ञ्प॑तिꣳ सुव सुव य॒ज्ञ्प॑ति॒म् भगा॑य । \newline
8. य॒ज्ञ्प॑ति॒म् भगा॑य॒ भगा॑य य॒ज्ञ्प॑तिं ॅय॒ज्ञ्प॑ति॒म् भगा॑य दि॒व्यो दि॒व्यो भगा॑य य॒ज्ञ्प॑तिं ॅय॒ज्ञ्प॑ति॒म् भगा॑य दि॒व्यः । \newline
9. य॒ज्ञ्प॑ति॒मिति॑ य॒ज्ञ् - प॒ति॒म् । \newline
10. भगा॑य दि॒व्यो दि॒व्यो भगा॑य॒ भगा॑य दि॒व्यो ग॑न्ध॒र्वो ग॑न्ध॒र्वो दि॒व्यो भगा॑य॒ भगा॑य दि॒व्यो ग॑न्ध॒र्वः । \newline
11. दि॒व्यो ग॑न्ध॒र्वो ग॑न्ध॒र्वो दि॒व्यो दि॒व्यो ग॑न्ध॒र्वः । \newline
12. ग॒न्ध॒र्व इति॑ गन्ध॒र्वः । \newline
13. के॒त॒पूः केत॒म् केत॑म् केत॒पूः के॑त॒पूः केत॑म् नो नः॒ केत॑म् केत॒पूः के॑त॒पूः केत॑म् नः । \newline
14. के॒त॒पूरिति॑ केत - पूः । \newline
15. केत॑म् नो नः॒ केत॒म् केत॑म् नः पुनातु पुनातु नः॒ केत॒म् केत॑म् नः पुनातु । \newline
16. नः॒ पु॒ना॒तु॒ पु॒ना॒तु॒ नो॒ नः॒ पु॒ना॒तु॒ वा॒चो वा॒चः पु॑नातु नो नः पुनातु वा॒चः । \newline
17. पु॒ना॒तु॒ वा॒चो वा॒चः पु॑नातु पुनातु वा॒च स्पति॒ष् पति॑र् वा॒चः पु॑नातु पुनातु वा॒च स्पतिः॑ । \newline
18. वा॒च स्पति॒ष् पति॑र् वा॒चो वा॒च स्पति॒र् वाचं॒ ॅवाच॒म् पति॑र् वा॒चो वा॒च स्पति॒र् वाच᳚म् । \newline
19. पति॒र् वाचं॒ ॅवाच॒म् पति॒ष् पति॒र् वाच॑ म॒द्याद्य वाच॒म् पति॒ष् पति॒र् वाच॑ म॒द्य । \newline
20. वाच॑ म॒द्याद्य वाचं॒ ॅवाच॑ म॒द्य स्व॑दाति स्वदात्य॒द्य वाचं॒ ॅवाच॑ म॒द्य स्व॑दाति । \newline
21. अ॒द्य स्व॑दाति स्वदा त्य॒द्याद्य स्व॑दाति नो नः स्वदा त्य॒द्याद्य स्व॑दाति नः । \newline
22. स्व॒दा॒ति॒ नो॒ नः॒ स्व॒दा॒ति॒ स्व॒दा॒ति॒ नः॒ । \newline
23. न॒ इति॑ नः । \newline
24. इन्द्र॑स्य॒ वज्रो॒ वज्र॒ इन्द्र॒स्ये न्द्र॑स्य॒ वज्रो᳚ ऽस्यसि॒ वज्र॒ इन्द्र॒स्ये न्द्र॑स्य॒ वज्रो॑ ऽसि । \newline
25. वज्रो᳚ ऽस्यसि॒ वज्रो॒ वज्रो॑ ऽसि॒ वार्त्र॑घ्नो॒ वार्त्र॑घ्नो ऽसि॒ वज्रो॒ वज्रो॑ ऽसि॒ वार्त्र॑घ्नः । \newline
26. अ॒सि॒ वार्त्र॑घ्नो॒ वार्त्र॑घ्नो ऽस्यसि॒ वार्त्र॑घ्न॒ स्त्वया॒ त्वया॒ वार्त्र॑घ्नो ऽस्यसि॒ वार्त्र॑घ्न॒ स्त्वया᳚ । \newline
27. वार्त्र॑घ्न॒ स्त्वया॒ त्वया॒ वार्त्र॑घ्नो॒ वार्त्र॑घ्न॒ स्त्वया॒ ऽय म॒यम् त्वया॒ वार्त्र॑घ्नो॒ वार्त्र॑घ्न॒ स्त्वया॒ ऽयम् । \newline
28. वार्त्र॑घ्न॒ इति॒ वार्त्र॑ - घ्नः॒ । \newline
29. त्वया॒ ऽय म॒यम् त्वया॒ त्वया॒ ऽयं ॅवृ॒त्रं ॅवृ॒त्र म॒यम् त्वया॒ त्वया॒ ऽयं ॅवृ॒त्रम् । \newline
30. अ॒यं ॅवृ॒त्रं ॅवृ॒त्र म॒य म॒यं ॅवृ॒त्रं ॅव॑द्ध्याद् वद्ध्याद् वृ॒त्र म॒य म॒यं ॅवृ॒त्रं ॅव॑द्ध्यात् । \newline
31. वृ॒त्रं ॅव॑द्ध्याद् वद्ध्याद् वृ॒त्रं ॅवृ॒त्रं ॅव॑द्ध्यात् । \newline
32. व॒द्ध्या॒दिति॑ वद्ध्यात् । \newline
33. वाज॑स्य॒ नु नु वाज॑स्य॒ वाज॑स्य॒ नु प्र॑स॒वे प्र॑स॒वे नु वाज॑स्य॒ वाज॑स्य॒ नु प्र॑स॒वे । \newline
34. नु प्र॑स॒वे प्र॑स॒वे नु नु प्र॑स॒वे मा॒तर॑म् मा॒तर॑म् प्रस॒वे नु नु प्र॑स॒वे मा॒तर᳚म् । \newline
35. प्र॒स॒वे मा॒तर॑म् मा॒तर॑म् प्रस॒वे प्र॑स॒वे मा॒तर॑म् म॒हीम् म॒हीम् मा॒तर॑म् प्रस॒वे प्र॑स॒वे मा॒तर॑म् म॒हीम् । \newline
36. प्र॒स॒व इति॑ प्र - स॒वे । \newline
37. मा॒तर॑म् म॒हीम् म॒हीम् मा॒तर॑म् मा॒तर॑म् म॒ही मदि॑ति॒ मदि॑तिम् म॒हीम् मा॒तर॑म् मा॒तर॑म् म॒ही मदि॑तिम् । \newline
38. म॒ही मदि॑ति॒ मदि॑तिम् म॒हीम् म॒ही मदि॑ति॒म् नाम॒ नामादि॑तिम् म॒हीम् म॒ही मदि॑ति॒म् नाम॑ । \newline
39. अदि॑ति॒म् नाम॒ नामादि॑ति॒ मदि॑ति॒म् नाम॒ वच॑सा॒ वच॑सा॒ नामादि॑ति॒ मदि॑ति॒म् नाम॒ वच॑सा । \newline
40. नाम॒ वच॑सा॒ वच॑सा॒ नाम॒ नाम॒ वच॑सा करामहे करामहे॒ वच॑सा॒ नाम॒ नाम॒ वच॑सा करामहे । \newline
41. वच॑सा करामहे करामहे॒ वच॑सा॒ वच॑सा करामहे । \newline
42. क॒रा॒म॒ह॒ इति॑ करामहे । \newline
43. यस्या॑ मि॒द मि॒दं ॅयस्यां॒ ॅयस्या॑ मि॒दं ॅविश्वं॒ ॅविश्व॑ मि॒दं ॅयस्यां॒ ॅयस्या॑ मि॒दं ॅविश्व᳚म् । \newline
44. इ॒दं ॅविश्वं॒ ॅविश्व॑ मि॒द मि॒दं ॅविश्व॒म् भुव॑न॒म् भुव॑नं॒ ॅविश्व॑ मि॒द मि॒दं ॅविश्व॒म् भुव॑नम् । \newline
45. विश्व॒म् भुव॑न॒म् भुव॑नं॒ ॅविश्वं॒ ॅविश्व॒म् भुव॑न मावि॒वेशा॑ वि॒वेश॒ भुव॑नं॒ ॅविश्वं॒ ॅविश्व॒म् भुव॑न मावि॒वेश॑ । \newline
46. भुव॑न मावि॒वेशा॑ वि॒वेश॒ भुव॑न॒म् भुव॑न मावि॒वेश॒ तस्या॒म् तस्या॑ मावि॒वेश॒ भुव॑न॒म् भुव॑न मावि॒वेश॒ तस्या᳚म् । \newline
47. आ॒वि॒वेश॒ तस्या॒म् तस्या॑ मावि॒वेशा॑ वि॒वेश॒ तस्या᳚न्नो न॒स्तस्या॑ मावि॒वेशा॑ वि॒वेश॒ तस्या᳚न्नः । \newline
48. आ॒वि॒वेशेत्या᳚ - वि॒वेश॑ । \newline
49. तस्या᳚म् नो न॒ स्तस्या॒म् तस्या᳚म् नो दे॒वो दे॒वो न॒ स्तस्या॒म् तस्या᳚म् नो दे॒वः । \newline
50. नो॒ दे॒वो दे॒वो नो॑ नो दे॒वः स॑वि॒ता स॑वि॒ता दे॒वो नो॑ नो दे॒वः स॑वि॒ता । \newline
51. दे॒वः स॑वि॒ता स॑वि॒ता दे॒वो दे॒वः स॑वि॒ता धर्म॒ धर्म॑ सवि॒ता दे॒वो दे॒वः स॑वि॒ता धर्म॑ । \newline
52. स॒वि॒ता धर्म॒ धर्म॑ सवि॒ता स॑वि॒ता धर्म॑ साविषथ् साविष॒द् धर्म॑ सवि॒ता स॑वि॒ता धर्म॑ साविषत् । \newline
53. धर्म॑ साविषथ् साविष॒द् धर्म॒ धर्म॑ साविषत् । \newline
54. सा॒वि॒ष॒दिति॑ साविषत् । \newline
55. अ॒फ्स्व॑न्त र॒न्त र॒फ्स्वा᳚(1॒)फ्स्व॑न्त र॒मृत॑ म॒मृत॑ म॒न्त र॒फ्स्वा᳚(1॒)फ्स्व॑न्त र॒मृत᳚म् । \newline
56. अ॒फ्स्वित्य॑प् - सु । \newline
\pagebreak
\markright{ TS 1.7.7.2  \hfill https://www.vedavms.in \hfill}
\addcontentsline{toc}{section}{ TS 1.7.7.2 }
\section*{ TS 1.7.7.2 }

\textbf{TS 1.7.7.2 } \newline
\textbf{Samhita Paata} \newline

न्तर॒मृत॑म॒फ्सु भे॑ष॒जम॒पामु॒त प्रश॑स्ति॒ष्वश्वा॑ भवथ वाजिनः ॥ वा॒युर् वा᳚ त्वा॒ मनु॑र् वा त्वा गन्ध॒र्वाः स॒प्तविꣳ॑शतिः । ते अग्रे॒ अश्व॑मायुञ्ज॒न्ते अ॑स्मिञ्ज॒वमाऽद॑धुः ॥ अपां᳚ नपादाशुहेम॒न्॒. य ऊ॒र्मिः क॒कुद्मा॒न् प्रतू᳚र्तिर् वाज॒सात॑म॒स्तेना॒यं ॅवाजꣳ॑ सेत् ॥ विष्णोः॒ क्रमो॑ऽसि॒ विष्णोः᳚ क्रा॒न्तम॑सि॒ विष्णो॒र् विक्रा᳚न्तमस्य॒ङ्कौ न्य॒ङ्का ( ) व॒भितो॒ रथं॒ ॅयौ ध्वा॒न्तं ॅवा॑ता॒ग्रमनु॑ स॒ञ्चर॑न्तौ दू॒रेहे॑ति-रिन्द्रि॒यावा᳚न् पत॒त्री ते नो॒ऽग्नयः॒ पप्र॑यः पारयन्तु ॥ \newline

\textbf{Pada Paata} \newline

अ॒न्तः । अ॒मृत᳚म् । अ॒फ्स्वित्य॑प् - सु । भे॒ष॒जम् । अ॒पाम् । उ॒त । प्रश॑स्ति॒ष्विति॒ प्र - श॒स्ति॒षु॒ । अश्वाः᳚ । भ॒व॒थ॒ । वा॒जि॒नः॒ ॥ वा॒युः । वा॒ । त्वा॒ । मनुः॑ । वा॒ । त्वा॒ । ग॒न्ध॒र्वाः । स॒प्तविꣳ॑शति॒रिति॑ स॒प्त - विꣳ॒॒श॒तिः॒ ॥ ते । अग्रे᳚ । अश्व᳚म् । आ॒यु॒ञ्ज॒न्न् । ते । अ॒स्मि॒न्न् । ज॒वम् । एति॑ । अ॒द॒धुः॒ ॥ अपा᳚म् । न॒पा॒त् । आ॒शु॒हे॒म॒न्नित्या॑शु - हे॒म॒न् । यः । ऊ॒र्मिः । क॒कुद्मा॒निति॑ क॒कुत् - मा॒न् । प्रतू᳚र्ति॒रिति॒ प्र - तू॒र्तिः॒ । वा॒ज॒सात॑म॒ इति॑ वाज - सात॑मः । तेन॑ । अ॒यम् । वाज᳚म् । से॒त् ॥ विष्णोः᳚ । क्रमः॑ । अ॒सि॒ । विष्णोः᳚ । क्रा॒न्तम् । अ॒सि॒ । विष्णोः᳚ । विक्रा᳚न्त॒मिति॒ वि - क्रा॒न्त॒म् । अ॒सि॒ । अ॒ङ्कौ । न्य॒ङ्काविति॑ नि - अ॒ङ्कौ ( ) । अ॒भितः॑ । रथ᳚म् । यौ । ध्वा॒न्तम् । वा॒ता॒ग्रमिति॑ वात - अ॒ग्रम् । अन्विति॑ । स॒चंर॑न्ता॒विति॑ सं - चर॑न्तौ । दू॒रेहे॑ति॒रिति॑ दू॒रे - हे॒तिः॒ । इ॒न्द्रि॒यावा॒निती᳚न्द्रि॒य - वा॒न् । प॒त॒त्री । ते । नः॒ । अ॒ग्नयः॑ । पप्र॑यः । पा॒र॒य॒न्तु॒ ॥  \newline


\textbf{Krama Paata} \newline

अ॒न्तर॒मृत᳚म् । अ॒मृत॑म॒फ्सु । अ॒फ्सु भे॑ष॒जम् । अ॒फ्स्वित्य॑प् - सु । भे॒ष॒जम॒पाम् । अ॒पामु॒त । उ॒त प्रश॑स्तिषु । प्रश॑स्ति॒ष्वश्वाः᳚ । प्रश॑स्ति॒ष्विति॒ प्र - श॒स्ति॒षु॒ । अश्वा॑ भवथ । भ॒व॒थ॒ वा॒जि॒नः॒ । वा॒जि॒न॒ इति॑ वाजिनः ॥ वा॒युर् वा᳚ । वा॒ त्वा॒ । त्वा॒ मनुः॑ । मनु॑र् वा । वा॒ त्वा॒ । त्वा॒ ग॒न्ध॒र्वाः । ग॒न्ध॒र्वाः स॒प्तविꣳ॑शतिः । स॒प्तविꣳ॑शति॒रिति॑ स॒प्त - विꣳ॒॒श॒तिः॒ । ते अग्रे᳚ । अग्रे॒ अश्व᳚म् । अश्व॑मायुञ्जन्न् । आ॒यु॒ञ्ज॒न् ते । ते अ॑स्मिन्न् । अ॒स्मि॒ञ्ज॒वम् । ज॒वमा । आऽद॑धुः । अ॒द॒धु॒रित्य॑दधुः ॥ अपा᳚म् नपात् । न॒पा॒दा॒शु॒हे॒म॒न्न्॒ । आ॒शु॒हे॒म॒न्.॒ यः । आ॒शु॒हे॒म॒न्नित्या॑शु - हे॒म॒न्न्॒ । य ऊ॒र्मिः । ऊ॒र्मिः क॒कुद्मान्॑ । क॒कुद्मा॒न्,प्रतू᳚र्तिः । क॒कुद्मा॒निति॑ क॒कुत् - मा॒न्॒ । प्रतू᳚र्तिर्,
वाज॒सात॑मः । प्रतू᳚र्ति॒रिति॒ प्र - तू॒र्तिः॒ । वा॒ज॒सात॑म॒स्तेन॑ । वा॒ज॒सात॑म॒ इति॑ वाज - सात॑मः । तेना॒यम् । अ॒यं ॅवाज᳚म् । वाजꣳ॑ सेत् । से॒दिति॑ सेत् ॥ विष्णोः॒ क्रमः॑ । क्रमो॑ऽसि । अ॒सि॒ विष्णोः᳚ । विष्णोः᳚ क्रा॒न्तम् । क्रा॒न्तम॑सि । अ॒सि॒ विष्णोः᳚ । विष्णो॒र् विक्रा᳚न्तम् । विक्रा᳚न्तमसि । विक्रा᳚न्त॒मिति॒ वि - क्रा॒न्त॒म् । अ॒स्य॒ङ्कौ ( ) । अ॒ङ्कौ न्य॒ङ्कौ । न्य॒ङ्काव॒भितः॑ । न्य॒ङ्काविति॑ नि - अ॒ङ्कौ । अ॒भितो॒ रथ᳚म् । रथं॒ ॅयौ । यौ ध्वा॒न्तम् । ध्वा॒न्तं ॅवा॑ता॒ग्रम् । वा॒ता॒ग्रमनु॑ । वा॒ता॒ग्रमिति॑ वात - अ॒ग्रम् । अनु॑ स॒ञ्चर॑न्तौ । स॒ञ्चर॑न्तौ दू॒रेहे॑तिः । स॒ञ्चर॑न्ता॒विति॑ सम् - चर॑न्तौ । दू॒रेहे॑तिरिन्द्रि॒यावान्॑ । दू॒रेहे॑ति॒रिति॑ दू॒रे - हे॒तिः॒ । इ॒न्द्रि॒यावा᳚न्,पत॒त्री । इ॒न्द्रि॒यावा॒निती᳚न्द्रि॒य - वा॒न्॒ । प॒त॒त्री ते । ते नः॑ । नो॒ऽग्नयः॑ । अ॒ग्नयः॒ पप्र॑यः । पप्र॑यः पारयन्तु । पा॒र॒य॒न्त्विति॑ पारयन्तु । \newline

\textbf{Jatai Paata} \newline

1. अ॒न्त र॒मृत॑ म॒मृत॑ म॒न्त र॒न्त र॒मृत᳚म् । \newline
2. अ॒मृत॑ म॒फ्स्वा᳚(1॒)फ्स्व॑मृत॑ म॒मृत॑ म॒फ्सु । \newline
3. अ॒फ्सु भे॑ष॒जम् भे॑ष॒ज म॒फ्स्व॑फ्सु भे॑ष॒जम् । \newline
4. अ॒फ्स्वित्य॑प् - सु । \newline
5. भे॒ष॒ज म॒पा म॒पाम् भे॑ष॒जम् भे॑ष॒ज म॒पाम् । \newline
6. अ॒पा मु॒तोतापा म॒पा मु॒त । \newline
7. उ॒त प्रश॑स्तिषु॒ प्रश॑स्तिषू॒तोत प्रश॑स्तिषु । \newline
8. प्रश॑स्ति॒ ष्वश्वा॒ अश्वाः॒ प्रश॑स्तिषु॒ प्रश॑स्ति॒ ष्वश्वाः᳚ । \newline
9. प्रश॑स्ति॒ष्विति॒ प्र - श॒स्ति॒षु॒ । \newline
10. अश्वा॑ भवथ भव॒थाश्वा॒ अश्वा॑ भवथ । \newline
11. भ॒व॒थ॒ वा॒जि॒नो॒ वा॒जि॒नो॒ भ॒व॒थ॒ भ॒व॒थ॒ वा॒जि॒नः॒ । \newline
12. वा॒जि॒न॒ इति॑ वाजिनः । \newline
13. वा॒युर् वा॑ वा वा॒युर् वा॒युर् वा᳚ । \newline
14. वा॒ त्वा॒ त्वा॒ वा॒ वा॒ त्वा॒ । \newline
15. त्वा॒ मनु॒र् मनु॑ स्त्वा त्वा॒ मनुः॑ । \newline
16. मनु॑र् वा वा॒ मनु॒र् मनु॑र् वा । \newline
17. वा॒ त्वा॒ त्वा॒ वा॒ वा॒ त्वा॒ । \newline
18. त्वा॒ ग॒न्ध॒र्वा ग॑न्ध॒र्वा स्त्वा᳚ त्वा गन्ध॒र्वाः । \newline
19. ग॒न्ध॒र्वाः स॒प्तविꣳ॑शतिः स॒प्तविꣳ॑शतिर् गन्ध॒र्वा ग॑न्ध॒र्वाः स॒प्तविꣳ॑शतिः । \newline
20. स॒प्तविꣳ॑शति॒रिति॑ स॒प्त - विꣳ॒॒श॒तिः॒ । \newline
21. ते अग्रे॒ अग्रे॒ ते ते अग्रे᳚ । \newline
22. अग्रे॒ अश्व॒ मश्व॒ मग्रे॒ अग्रे॒ अश्व᳚म् । \newline
23. अश्व॑ मायुञ्जन् नायुञ्ज॒न् नश्व॒ मश्व॑ मायुञ्जन्न् । \newline
24. आ॒यु॒ञ्ज॒न् ते त आ॑युञ्जन् नायुञ्ज॒न् ते । \newline
25. ते अ॑स्मिन् नस्मि॒न् ते ते अ॑स्मिन्न् । \newline
26. अ॒स्मि॒न् ज॒वम् ज॒व म॑स्मिन् नस्मिन् ज॒वम् । \newline
27. ज॒व मा ज॒वम् ज॒व मा । \newline
28. आ ऽद॑धु रदधु॒रा ऽद॑धुः । \newline
29. अ॒द॒धु॒ रित्य॑दधुः । \newline
30. अपा᳚न् नपान् नपा॒दपा॒ मपा᳚न् नपात् । \newline
31. न॒पा॒ दा॒शु॒हे॒म॒न् ना॒शु॒हे॒म॒न् न॒पा॒न् न॒पा॒ दा॒शु॒हे॒म॒न्न् । \newline
32. आ॒शु॒हे॒म॒न्॒. यो य आ॑शुहेमन् नाशुहेम॒न्॒. यः । \newline
33. आ॒शु॒हे॒म॒न्नित्या॑शु - हे॒म॒न्न् । \newline
34. य ऊ॒र्मि रू॒र्मिर् यो य ऊ॒र्मिः । \newline
35. ऊ॒र्मिः क॒कुद्मा᳚न् क॒कुद्मा॑ नू॒र्मि रू॒र्मिः क॒कुद्मान्॑ । \newline
36. क॒कुद्मा॒न् प्रतू᳚र्तिः॒ प्रतू᳚र्तिः क॒कुद्मा᳚न् क॒कुद्मा॒न् प्रतू᳚र्तिः । \newline
37. क॒कुद्मा॒निति॑ क॒कुत् - मा॒न् । \newline
38. प्रतू᳚र्तिर् वाज॒सात॑मो वाज॒सात॑मः॒ प्रतू᳚र्तिः॒ प्रतू᳚र्तिर् वाज॒सात॑मः । \newline
39. प्रतू᳚र्ति॒रिति॒ प्र - तू॒र्तिः॒ । \newline
40. वा॒ज॒सात॑म॒ स्तेन॒ तेन॑ वाज॒सात॑मो वाज॒सात॑म॒ स्तेन॑ । \newline
41. वा॒ज॒सात॑म॒ इति॑ वाज - सात॑मः । \newline
42. तेना॒य म॒यम् तेन॒ तेना॒यम् । \newline
43. अ॒यं ॅवाजं॒ ॅवाज॑ म॒य म॒यं ॅवाज᳚म् । \newline
44. वाजꣳ॑ सेथ् से॒द् वाजं॒ ॅवाजꣳ॑ सेत् । \newline
45. से॒दिति॑ सेत् । \newline
46. विष्णोः॒ क्रमः॒ क्रमो॒ विष्णो॒र् विष्णोः॒ क्रमः॑ । \newline
47. क्रमो᳚ ऽस्यसि॒ क्रमः॒ क्रमो॑ ऽसि । \newline
48. अ॒सि॒ विष्णो॒र् विष्णो॑ रस्यसि॒ विष्णोः᳚ । \newline
49. विष्णोः᳚ क्रा॒न्तम् क्रा॒न्तं ॅविष्णो॒र् विष्णोः᳚ क्रा॒न्तम् । \newline
50. क्रा॒न्त म॑स्यसि क्रा॒न्तम् क्रा॒न्त म॑सि । \newline
51. अ॒सि॒ विष्णो॒र् विष्णो॑ रस्यसि॒ विष्णोः᳚ । \newline
52. विष्णो॒र् विक्रा᳚न्तं॒ ॅविक्रा᳚न्तं॒ ॅविष्णो॒र् विष्णो॒र् विक्रा᳚न्तम् । \newline
53. विक्रा᳚न्त मस्यसि॒ विक्रा᳚न्तं॒ ॅविक्रा᳚न्त मसि । \newline
54. विक्रा᳚न्त॒मिति॒ वि - क्रा॒न्त॒म् । \newline
55. अ॒स्य॒ङ्का व॒ङ्का व॑स्य स्य॒ङ्कौ । \newline
56. अ॒ङ्कौ न्य॒ङ्कौ न्य॒ङ्का व॒ङ्का व॒ङ्कौ न्य॒ङ्कौ । \newline
57. न्य॒ङ्का व॒भितो॑ अ॒भितो᳚ न्य॒ङ्कौ न्य॒ङ्का व॒भितः॑ । \newline
58. न्य॒ङ्काविति॑ नि - अ॒ङ्कौ । \newline
59. अ॒भितो॒ रथꣳ॒॒ रथ॑ म॒भितो॑ अ॒भितो॒ रथ᳚म् । \newline
60. रथं॒ ॅयौ यौ रथꣳ॒॒ रथं॒ ॅयौ । \newline
61. यौ ध्वा॒न्तम् ध्वा॒न्तं ॅयौ यौ ध्वा॒न्तम् । \newline
62. ध्वा॒न्तं ॅवा॑ता॒ग्रं ॅवा॑ता॒ग्रम् ध्वा॒न्तम् ध्वा॒न्तं ॅवा॑ता॒ग्रम् । \newline
63. वा॒ता॒ग्र मन्वनु॑ वाता॒ग्रं ॅवा॑ता॒ग्र मनु॑ । \newline
64. वा॒ता॒ग्रमिति॑ वात - अ॒ग्रम् । \newline
65. अनु॑ स॒ञ्चर॑न्तौ स॒ञ्चर॑न्ता॒ वन्वनु॑ स॒ञ्चर॑न्तौ । \newline
66. स॒ञ्चर॑न्तौ दू॒रेहे॑तिर् दू॒रेहे॑तिः स॒ञ्चर॑न्तौ स॒ञ्चर॑न्तौ दू॒रेहे॑तिः । \newline
67. स॒ञ्चर॑न्ता॒विति॑ सं - चर॑न्तौ । \newline
68. दू॒रेहे॑ति रिन्द्रि॒यावा॑ निन्द्रि॒यावा᳚न् दू॒रेहे॑तिर् दू॒रेहे॑ति रिन्द्रि॒यावान्॑ । \newline
69. दू॒रेहे॑ति॒रिति॑ दू॒रे - हे॒तिः॒ । \newline
70. इ॒न्द्रि॒यावा᳚न् पत॒त्री प॑त॒त्री न्द्रि॒यावा॑ निन्द्रि॒यावा᳚न् पत॒त्री । \newline
71. इ॒न्द्रि॒यावा॒निती᳚न्द्रि॒य - वा॒न् । \newline
72. प॒त॒त्री ते ते प॑त॒त्री प॑त॒त्री ते । \newline
73. ते नो॑ न॒स्ते ते नः॑ । \newline
74. नो॒ ऽग्नयो॑ अ॒ग्नयो॑ नो नो॒ ऽग्नयः॑ । \newline
75. अ॒ग्नयः॒ पप्र॑यः॒ पप्र॑यो अ॒ग्नयो॒ ऽग्नयः॒ पप्र॑यः । \newline
76. पप्र॑यः पारयन्तु पारयन्तु॒ पप्र॑यः॒ पप्र॑यः पारयन्तु । \newline
77. पा॒र॒य॒न्त्विति॑ पारयन्तु । \newline

\textbf{Ghana Paata } \newline

1. अ॒न्त र॒मृत॑ म॒मृत॑ म॒न्त र॒न्त र॒मृत॑ म॒फ्स्वा᳚(1॒)फ्स्व॑मृत॑ म॒न्त र॒न्त र॒मृत॑ म॒फ्सु । \newline
2. अ॒मृत॑ म॒फ्स्वा᳚(1॒)फ्स्व॑मृत॑ म॒मृत॑ म॒फ्सु भे॑ष॒जम् भे॑ष॒ज म॒फ्स्व॑मृत॑ म॒मृत॑ म॒फ्सु भे॑ष॒जम् । \newline
3. अ॒फ्सु भे॑ष॒जम् भे॑ष॒ज म॒फ्स्व॑फ्सु भे॑ष॒ज म॒पा म॒पाम् भे॑ष॒ज म॒फ्स्व॑फ्सु भे॑ष॒ज म॒पाम् । \newline
4. अ॒फ्स्वित्य॑प् - सु । \newline
5. भे॒ष॒ज म॒पा म॒पाम् भे॑ष॒जम् भे॑ष॒ज म॒पा मु॒तोतापाम् भे॑ष॒जम् भे॑ष॒ज म॒पा मु॒त । \newline
6. अ॒पा मु॒तोतापा म॒पा मु॒त प्रश॑स्तिषु॒ प्रश॑स्तिषू॒तापा म॒पा मु॒त प्रश॑स्तिषु । \newline
7. उ॒त प्रश॑स्तिषु॒ प्रश॑स्तिषू॒तोत प्रश॑स्ति॒ ष्वश्वा॒ अश्वाः॒ प्रश॑स्तिषू॒तोत प्रश॑स्ति॒ ष्वश्वाः᳚ । \newline
8. प्रश॑स्ति॒ ष्वश्वा॒ अश्वाः॒ प्रश॑स्तिषु॒ प्रश॑स्ति॒ ष्वश्वा॑ भवथ भव॒थाश्वाः॒ प्रश॑स्तिषु॒ प्रश॑स्ति॒ ष्वश्वा॑ भवथ । \newline
9. प्रश॑स्ति॒ष्विति॒ प्र - श॒स्ति॒षु॒ । \newline
10. अश्वा॑ भवथ भव॒थाश्वा॒ अश्वा॑ भवथ वाजिनो वाजिनो भव॒थाश्वा॒ अश्वा॑ भवथ वाजिनः । \newline
11. भ॒व॒थ॒ वा॒जि॒नो॒ वा॒जि॒नो॒ भ॒व॒थ॒ भ॒व॒थ॒ वा॒जि॒नः॒ । \newline
12. वा॒जि॒न॒ इति॑ वाजिनः । \newline
13. वा॒युर् वा॑ वा वा॒युर् वा॒युर् वा᳚ त्वा त्वा वा वा॒युर् वा॒युर् वा᳚ त्वा । \newline
14. वा॒ त्वा॒ त्वा॒ वा॒ वा॒ त्वा॒ मनु॒र् मनु॑ स्त्वा वा वा त्वा॒ मनुः॑ । \newline
15. त्वा॒ मनु॒र् मनु॑ स्त्वा त्वा॒ मनु॑र् वा वा॒ मनु॑ स्त्वा त्वा॒ मनु॑र् वा । \newline
16. मनु॑र् वा वा॒ मनु॒र् मनु॑र् वा त्वा त्वा वा॒ मनु॒र् मनु॑र् वा त्वा । \newline
17. वा॒ त्वा॒ त्वा॒ वा॒ वा॒ त्वा॒ ग॒न्ध॒र्वा ग॑न्ध॒र्वा स्त्वा॑ वा वा त्वा गन्ध॒र्वाः । \newline
18. त्वा॒ ग॒न्ध॒र्वा ग॑न्ध॒र्वा स्त्वा᳚ त्वा गन्ध॒र्वाः स॒प्तवि(ग्म्॑)शतिः स॒प्तवि(ग्म्॑)शतिर् गन्ध॒र्वा स्त्वा᳚ त्वा गन्ध॒र्वाः स॒प्तवि(ग्म्॑)शतिः । \newline
19. ग॒न्ध॒र्वाः स॒प्तवि(ग्म्॑)शतिः स॒प्तवि(ग्म्॑)शतिर् गन्ध॒र्वा ग॑न्ध॒र्वाः स॒प्तवि(ग्म्॑)शतिः । \newline
20. स॒प्तवि(ग्म्॑)शति॒रिति॑ स॒प्त - वि॒(ग्म्॒)श॒तिः॒ । \newline
21. ते अग्रे॒ अग्रे॒ ते ते अग्रे॒ अश्व॒ मश्व॒ मग्रे॒ ते ते अग्रे॒ अश्व᳚म् । \newline
22. अग्रे॒ अश्व॒ मश्व॒ मग्रे॒ अग्रे॒ अश्व॑ मायुञ्जन् नायुञ्ज॒न् नश्व॒ मग्रे॒ अग्रे॒ अश्व॑ मायुञ्जन्न् । \newline
23. अश्व॑ मायुञ्जन् नायुञ्ज॒न् नश्व॒ मश्व॑ मायुञ्ज॒न् ते त आ॑युञ्ज॒न् नश्व॒ मश्व॑ मायुञ्ज॒न् ते । \newline
24. आ॒यु॒ञ्ज॒न् ते त आ॑युञ्जन् नायुञ्ज॒न् ते अ॑स्मिन् नस्मि॒न् त आ॑युञ्जन् नायुञ्ज॒न् ते अ॑स्मिन्न् । \newline
25. ते अ॑स्मिन् नस्मि॒न् ते ते अ॑स्मिन् ज॒वम् ज॒व म॑स्मि॒न् ते ते अ॑स्मिन् ज॒वम् । \newline
26. अ॒स्मि॒न् ज॒वम् ज॒व म॑स्मिन् नस्मिन् ज॒व मा ज॒व म॑स्मिन् नस्मिन् ज॒व मा । \newline
27. ज॒व मा ज॒वम् ज॒व मा ऽद॑धु रदधु॒रा ज॒वम् ज॒व मा ऽद॑धुः । \newline
28. आ ऽद॑धु रदधु॒रा ऽद॑धुः । \newline
29. अ॒द॒धु॒रित्य॑दधुः । \newline
30. अपा᳚म् नपान् नपा॒दपा॒ मपा᳚ नपा दाशुहेमन् नाशुहेमन् नपा॒दपा॒ मपा᳚म् नपा दाशुहेमन्न् । \newline
31. न॒पा॒ दा॒शु॒हे॒म॒न् ना॒शु॒हे॒म॒न् न॒पा॒न् न॒पा॒ दा॒शु॒हे॒म॒न्॒. यो य आ॑शुहेमन् नपान् नपा दाशुहेम॒न्॒. यः । \newline
32. आ॒शु॒हे॒म॒न्॒. यो य आ॑शुहेमन् नाशुहेम॒न्॒. य ऊ॒र्मि रू॒र्मिर् य आ॑शुहेमन् नाशुहेम॒न्॒. य ऊ॒र्मिः । \newline
33. आ॒शु॒हे॒म॒न्नित्या॑शु - हे॒म॒न्न् । \newline
34. य ऊ॒र्मि रू॒र्मिर् यो य ऊ॒र्मिः क॒कुद्मा᳚न् क॒कुद्मा॑ नू॒र्मिर् यो य ऊ॒र्मिः क॒कुद्मान्॑ । \newline
35. ऊ॒र्मिः क॒कुद्मा᳚न् क॒कुद्मा॑ नू॒र्मिरू॒र्मिः क॒कुद्मा॒न् प्रतू᳚र्तिः॒ प्रतू᳚र्तिः क॒कुद्मा॑ नू॒र्मि रू॒र्मिः क॒कुद्मा॒न् प्रतू᳚र्तिः । \newline
36. क॒कुद्मा॒न् प्रतू᳚र्तिः॒ प्रतू᳚र्तिः क॒कुद्मा᳚न् क॒कुद्मा॒न् प्रतू᳚र्तिर् वाज॒सात॑मो वाज॒सात॑मः॒ प्रतू᳚र्तिः क॒कुद्मा᳚न् क॒कुद्मा॒न् प्रतू᳚र्तिर् वाज॒सात॑मः । \newline
37. क॒कुद्मा॒निति॑ क॒कुत् - मा॒न् । \newline
38. प्रतू᳚र्तिर् वाज॒सात॑मो वाज॒सात॑मः॒ प्रतू᳚र्तिः॒ प्रतू᳚र्तिर् वाज॒सात॑म॒ स्तेन॒ तेन॑ वाज॒सात॑मः॒ प्रतू᳚र्तिः॒ प्रतू᳚र्तिर् वाज॒सात॑म॒ स्तेन॑ । \newline
39. प्रतू᳚र्ति॒रिति॒ प्र - तू॒र्तिः॒ । \newline
40. वा॒ज॒सात॑म॒ स्तेन॒ तेन॑ वाज॒सात॑मो वाज॒सात॑म॒ स्तेना॒य म॒यम् तेन॑ वाज॒सात॑मो वाज॒सात॑म॒ स्तेना॒यम् । \newline
41. वा॒ज॒सात॑म॒ इति॑ वाज - सात॑मः । \newline
42. तेना॒य म॒यम् तेन॒ तेना॒यं ॅवाजं॒ ॅवाज॑ म॒यम् तेन॒ तेना॒यं ॅवाज᳚म् । \newline
43. अ॒यं ॅवाजं॒ ॅवाज॑ म॒य म॒यं ॅवाज(ग्म्॑) सेथ् से॒द् वाज॑ म॒य म॒यं ॅवाज(ग्म्॑) सेत् । \newline
44. वाज(ग्म्॑) सेथ् से॒द् वाजं॒ ॅवाज(ग्म्॑) सेत् । \newline
45. से॒दिति॑ सेत् । \newline
46. विष्णोः॒ क्रमः॒ क्रमो॒ विष्णो॒र् विष्णोः॒ क्रमो᳚ ऽस्यसि॒ क्रमो॒ विष्णो॒र् विष्णोः॒ क्रमो॑ ऽसि । \newline
47. क्रमो᳚ ऽस्यसि॒ क्रमः॒ क्रमो॑ ऽसि॒ विष्णो॒र् विष्णो॑ रसि॒ क्रमः॒ क्रमो॑ ऽसि॒ विष्णोः᳚ । \newline
48. अ॒सि॒ विष्णो॒र् विष्णो॑ रस्यसि॒ विष्णोः᳚ क्रा॒न्तम् क्रा॒न्तं ॅविष्णो॑ रस्यसि॒ विष्णोः᳚ क्रा॒न्तम् । \newline
49. विष्णोः᳚ क्रा॒न्तम् क्रा॒न्तं ॅविष्णो॒र् विष्णोः᳚ क्रा॒न्त म॑स्यसि क्रा॒न्तं ॅविष्णो॒र् विष्णोः᳚ क्रा॒न्त म॑सि । \newline
50. क्रा॒न्त म॑स्यसि क्रा॒न्तम् क्रा॒न्त म॑सि॒ विष्णो॒र् विष्णो॑ रसि क्रा॒न्तम् क्रा॒न्त म॑सि॒ विष्णोः᳚ । \newline
51. अ॒सि॒ विष्णो॒र् विष्णो॑ रस्यसि॒ विष्णो॒र् विक्रा᳚न्तं॒ ॅविक्रा᳚न्तं॒ ॅविष्णो॑ रस्यसि॒ विष्णो॒र् विक्रा᳚न्तम् । \newline
52. विष्णो॒र् विक्रा᳚न्तं॒ ॅविक्रा᳚न्तं॒ ॅविष्णो॒र् विष्णो॒र् विक्रा᳚न्त मस्यसि॒ विक्रा᳚न्तं॒ ॅविष्णो॒र् विष्णो॒र् विक्रा᳚न्त मसि । \newline
53. विक्रा᳚न्त मस्यसि॒ विक्रा᳚न्तं॒ ॅविक्रा᳚न्त मस्य॒ङ्का व॒ङ्का व॑सि॒ विक्रा᳚न्तं॒ ॅविक्रा᳚न्त मस्य॒ङ्कौ । \newline
54. विक्रा᳚न्त॒मिति॒ वि - क्रा॒न्त॒म् । \newline
55. अ॒स्य॒ङ्का व॒ङ्का व॑स्यस्य॒ङ्कौ न्य॒ङ्कौ न्य॒ङ्का व॒ङ्का व॑स्यस्य॒ङ्कौ न्य॒ङ्कौ । \newline
56. अ॒ङ्कौ न्य॒ङ्कौ न्य॒ङ्का व॒ङ्का व॒ङ्कौ न्य॒ङ्का व॒भितो॑ अ॒भितो᳚ न्य॒ङ्का व॒ङ्का व॒ङ्कौ न्य॒ङ्का व॒भितः॑ । \newline
57. न्य॒ङ्का व॒भितो॑ अ॒भितो᳚ न्य॒ङ्कौ न्य॒ङ्का व॒भितो॒ रथ॒(ग्म्॒) रथ॑ म॒भितो᳚ न्य॒ङ्कौ न्य॒ङ्का व॒भितो॒ रथ᳚म् । \newline
58. न्य॒ङ्काविति॑ नि - अ॒ङ्कौ । \newline
59. अ॒भितो॒ रथ॒(ग्म्॒) रथ॑ म॒भितो॑ अ॒भितो॒ रथं॒ ॅयौ यौ रथ॑ म॒भितो॑ अ॒भितो॒ रथं॒ ॅयौ । \newline
60. रथं॒ ॅयौ यौ रथ॒(ग्म्॒) रथं॒ ॅयौ ध्वा॒न्तम् ध्वा॒न्तं ॅयौ रथ॒(ग्म्॒) रथं॒ ॅयौ ध्वा॒न्तम् । \newline
61. यौ ध्वा॒न्तम् ध्वा॒न्तं ॅयौ यौ ध्वा॒न्तं ॅवा॑ता॒ग्रं ॅवा॑ता॒ग्रम् ध्वा॒न्तं ॅयौ यौ ध्वा॒न्तं ॅवा॑ता॒ग्रम् । \newline
62. ध्वा॒न्तं ॅवा॑ता॒ग्रं ॅवा॑ता॒ग्रम् ध्वा॒न्तम् ध्वा॒न्तं ॅवा॑ता॒ग्र मन्वनु॑ वाता॒ग्रम् ध्वा॒न्तम् ध्वा॒न्तं ॅवा॑ता॒ग्र मनु॑ । \newline
63. वा॒ता॒ग्र मन्वनु॑ वाता॒ग्रं ॅवा॑ता॒ग्र मनु॑ स॒ञ्चर॑न्तौ स॒ञ्चर॑न्ता॒ वनु॑ वाता॒ग्रं ॅवा॑ता॒ग्र मनु॑ स॒ञ्चर॑न्तौ । \newline
64. वा॒ता॒ग्रमिति॑ वात - अ॒ग्रम् । \newline
65. अनु॑ स॒ञ्चर॑न्तौ स॒ञ्चर॑न्ता॒ वन्वनु॑ स॒ञ्चर॑न्तौ दू॒रेहे॑तिर् दू॒रेहे॑तिः स॒ञ्चर॑न्ता॒ वन्वनु॑ स॒ञ्चर॑न्तौ दू॒रेहे॑तिः । \newline
66. स॒ञ्चर॑न्तौ दू॒रेहे॑तिर् दू॒रेहे॑तिः स॒ञ्चर॑न्तौ स॒ञ्चर॑न्तौ दू॒रेहे॑ति रिन्द्रि॒यावा॑ निन्द्रि॒यावा᳚न् दू॒रेहे॑तिः स॒ञ्चर॑न्तौ स॒ञ्चर॑न्तौ दू॒रेहे॑ति रिन्द्रि॒यावान्॑ । \newline
67. स॒ञ्चर॑न्ता॒विति॑ सं - चर॑न्तौ । \newline
68. दू॒रेहे॑ति रिन्द्रि॒यावा॑ निन्द्रि॒यावा᳚न् दू॒रेहे॑तिर् दू॒रेहे॑ति रिन्द्रि॒यावा᳚न् पत॒त्री प॑त॒ त्रीन्द्रि॒यावा᳚न् दू॒रेहे॑तिर् दू॒रेहे॑ति रिन्द्रि॒यावा᳚न् पत॒त्री । \newline
69. दू॒रेहे॑ति॒रिति॑ दू॒रे - हे॒तिः॒ । \newline
70. इ॒न्द्रि॒यावा᳚न् पत॒त्री प॑त॒ त्रीन्द्रि॒यावा॑ निन्द्रि॒यावा᳚न् पत॒त्री ते ते प॑त॒ त्रीन्द्रि॒यावा॑ निन्द्रि॒यावा᳚न् पत॒त्री ते । \newline
71. इ॒न्द्रि॒यावा॒निती᳚न्द्रि॒य - वा॒न् । \newline
72. प॒त॒त्री ते ते प॑त॒त्री प॑त॒त्री ते नो॑ न॒स्ते प॑त॒त्री प॑त॒त्री ते नः॑ । \newline
73. ते नो॑ न॒स्ते ते नो॒ ऽग्नयो॑ अ॒ग्नयो॑ न॒स्ते ते नो॒ ऽग्नयः॑ । \newline
74. नो॒ ऽग्नयो॑ अ॒ग्नयो॑ नो नो॒ ऽग्नयः॒ पप्र॑यः॒ पप्र॑यो अ॒ग्नयो॑ नो नो॒ ऽग्नयः॒ पप्र॑यः । \newline
75. अ॒ग्नयः॒ पप्र॑यः॒ पप्र॑यो अ॒ग्नयो॑ अ॒ग्नयः॒ पप्र॑यः पारयन्तु पारयन्तु॒ पप्र॑यो अ॒ग्नयो॑ अ॒ग्नयः॒ पप्र॑यः पारयन्तु । \newline
76. पप्र॑यः पारयन्तु पारयन्तु॒ पप्र॑यः॒ पप्र॑यः पारयन्तु । \newline
77. पा॒र॒य॒न्त्विति॑ पारयन्तु । \newline
\pagebreak
\markright{ TS 1.7.8.1  \hfill https://www.vedavms.in \hfill}
\addcontentsline{toc}{section}{ TS 1.7.8.1 }
\section*{ TS 1.7.8.1 }

\textbf{TS 1.7.8.1 } \newline
\textbf{Samhita Paata} \newline

दे॒वस्या॒हꣳ स॑वि॒तुः प्र॑स॒वे बृह॒स्पति॑ना वाज॒जिता॒ वाजं॑ जेषं दे॒वस्या॒हꣳ स॑वि॒तुः प्र॑स॒वे बृह॒स्पति॑ना वाज॒जिता॒ वर्.षि॑ष्ठं॒ नाकꣳ॑ रुहेय॒मिन्द्रा॑य॒ वाचं॑ ॅवद॒तेन्द्रं॒ वाजं॑ जापय॒तेन्द्रो॒ वाज॑मजयित् ॥ अश्वा॑जनि वाजिनि॒ वाजे॑षु वाजिनीव॒त्यश्वा᳚न्थ् स॒मथ्सु॑ वाजय ॥ अर्वा॑ऽसि॒ सप्ति॑रसि वा॒ज्य॑सि॒ वाजि॑नो॒ वाजं॑ धावत म॒रुतां᳚ प्रस॒वे ज॑यत॒ वि योज॑ना मिमीद्ध्व॒मद्ध्व॑नः स्कभ्नीत॒-[ ] \newline

\textbf{Pada Paata} \newline

दे॒वस्य॑ । अ॒हम् । स॒वि॒तुः । प्र॒स॒व इति॑ प्र - स॒वे । बृह॒स्पति॑ना । वा॒ज॒जितेति॑ वाज - जिता᳚ । वाज᳚म् । जे॒ष॒म् । दे॒वस्य॑ । अ॒हम् । स॒वि॒तुः । प्र॒स॒व इति॑ प्र - स॒वे । बृह॒स्पति॑ना । वा॒ज॒जितेति॑ वाज - जिता᳚ । वर्.षि॑ष्ठम् । नाक᳚म् । रु॒हे॒य॒म् । इन्द्रा॑य । वाच᳚म् । व॒द॒त॒ । इन्द्र᳚म् । वाज᳚म् । जा॒प॒य॒त॒ । इन्द्रः॑ । वाज᳚म् । अ॒ज॒यि॒त् ॥ अश्वा॑ज॒नीत्यश्व॑ - अ॒ज॒नि॒ । वा॒जि॒नि॒ । वाजे॑षु । वा॒जि॒नी॒व॒तीति॑ वाजिनी - व॒ति॒ । अश्वान्॑ । स॒मथ्स्विति॑ स॒मत्-स॒ । वा॒ज॒य॒ ॥ अर्वा᳚ । अ॒सि॒ । सप्तिः॑ । अ॒सि॒ । वा॒जी । अ॒सि॒ । वाजि॑नः । वाज᳚म् । धा॒व॒त॒ । म॒रुता᳚म् । प्र॒स॒व इति॑ प्र - स॒वे । ज॒य॒त॒ । वीति॑ । योज॑ना । मि॒मी॒द्ध्व॒म् । अद्ध्व॑नः । स्क॒भ्नी॒त॒ ।  \newline


\textbf{Krama Paata} \newline

दे॒वस्या॒हम् । अ॒हꣳ स॑वि॒तुः । स॒वि॒तुः प्र॑स॒वे । प्र॒स॒वे बृह॒स्पति॑ना । प्र॒स॒व इति॑ प्र - स॒वे । बृह॒स्पति॑ना वाज॒जिता᳚ । वा॒ज॒जिता॒ वाज᳚म् । वा॒ज॒जितेति॑ वाज - जिता᳚ । वाज॑म् जेषम् । जे॒ष॒म् दे॒वस्य॑ । दे॒वस्या॒हम् । अ॒हꣳ स॑वि॒तुः । स॒वि॒तुः प्र॑स॒वे । प्र॒स॒वे बृह॒स्पति॑ना । प्र॒स॒व इति॑ प्र - स॒वे । बृह॒स्पति॑ना वाज॒जिता᳚ । वा॒ज॒जिता॒ वर्.षि॑ष्ठम् । वा॒ज॒जितेति॑ वाज - जिता᳚ । वर्.षि॑ष्ठ॒म् नाक᳚म् । नाकꣳ॑ रुहेयम् । रु॒हे॒य॒मिन्द्रा॑य । इन्द्रा॑य॒ वाच᳚म् । वाचं॑ ॅवदत । व॒द॒तेन्द्र᳚म् । इन्द्रं॒ ॅवाज᳚म् । वाज॑म् जापयत । जा॒प॒य॒तेन्द्रः॑ । इन्द्रो॒ वाज᳚म् । वाज॑मजयित् । अ॒ज॒य॒दित्य॑जयित् ॥ अश्वा॑जनि वाजिनि । अश्वा॑ज॒नीत्यश्व॑ - अ॒ज॒नि॒ । वा॒जि॒नि॒ वाजे॑षु । वाजे॑षु वाजिनीवति । वा॒जि॒नी॒व॒त्यश्वान्॑ । वा॒जि॒नी॒व॒तीति॑ वाजिनी - व॒ति॒ । अश्वा᳚न्थ् स॒मथ्सु॑ । स॒मथ्सु॑ वाजय । स॒मथ्स्विति॑ स॒मत् - सु॒ । वा॒ज॒येति॑ वाजय ॥ अर्वा॑ऽसि । अ॒सि॒ सप्तिः॑ । सप्ति॑रसि । अ॒सि॒ वा॒जी । वा॒ज्य॑सि । अ॒सि॒ वाजि॑नः । वाजि॑नो॒ वाज᳚म् । वाज॑म् धावत । धा॒व॒त॒ म॒रुता᳚म् । म॒रुता᳚म् प्रस॒वे । प्र॒स॒वे ज॑यत । प्र॒स॒व इति॑ प्र - स॒वे । ज॒य॒त॒ वि । वि योज॑ना । योज॑ना मिमीद्ध्वम् । मि॒मी॒द्ध्व॒मद्ध्व॑नः । अद्ध्व॑नः स्कभ्नीत । स्क॒भ्नी॒त॒ काष्ठा᳚म् \newline

\textbf{Jatai Paata} \newline

1. दे॒वस्या॒ह म॒हम् दे॒वस्य॑ दे॒वस्या॒हम् । \newline
2. अ॒हꣳ स॑वि॒तुः स॑वि॒तु र॒ह म॒हꣳ स॑वि॒तुः । \newline
3. स॒वि॒तुः प्र॑स॒वे प्र॑स॒वे स॑वि॒तुः स॑वि॒तुः प्र॑स॒वे । \newline
4. प्र॒स॒वे बृह॒स्पति॑ना॒ बृह॒स्पति॑ना प्रस॒वे प्र॑स॒वे बृह॒स्पति॑ना । \newline
5. प्र॒स॒व इति॑ प्र - स॒वे । \newline
6. बृह॒स्पति॑ना वाज॒जिता॑ वाज॒जिता॒ बृह॒स्पति॑ना॒ बृह॒स्पति॑ना वाज॒जिता᳚ । \newline
7. वा॒ज॒जिता॒ वाजं॒ ॅवाजं॑ ॅवाज॒जिता॑ वाज॒जिता॒ वाज᳚म् । \newline
8. वा॒ज॒जितेति॑ वाज - जिता᳚ । \newline
9. वाज॑म् जेषम् जेषं॒ ॅवाजं॒ ॅवाज॑म् जेषम् । \newline
10. जे॒ष॒म् दे॒वस्य॑ दे॒वस्य॑ जेषम् जेषम् दे॒वस्य॑ । \newline
11. दे॒वस्या॒ह म॒हम् दे॒वस्य॑ दे॒वस्या॒हम् । \newline
12. अ॒हꣳ स॑वि॒तुः स॑वि॒तु र॒ह म॒हꣳ स॑वि॒तुः । \newline
13. स॒वि॒तुः प्र॑स॒वे प्र॑स॒वे स॑वि॒तुः स॑वि॒तुः प्र॑स॒वे । \newline
14. प्र॒स॒वे बृह॒स्पति॑ना॒ बृह॒स्पति॑ना प्रस॒वे प्र॑स॒वे बृह॒स्पति॑ना । \newline
15. प्र॒स॒व इति॑ प्र - स॒वे । \newline
16. बृह॒स्पति॑ना वाज॒जिता॑ वाज॒जिता॒ बृह॒स्पति॑ना॒ बृह॒स्पति॑ना वाज॒जिता᳚ । \newline
17. वा॒ज॒जिता॒ वर्.षि॑ष्ठं॒ ॅवर्.षि॑ष्ठं ॅवाज॒जिता॑ वाज॒जिता॒ वर्.षि॑ष्ठम् । \newline
18. वा॒ज॒जितेति॑ वाज - जिता᳚ । \newline
19. वर्.षि॑ष्ठ॒न् नाक॒न् नाकं॒ ॅवर्.षि॑ष्ठं॒ ॅवर्.षि॑ष्ठ॒न् नाक᳚म् । \newline
20. नाकꣳ॑ रुहेयꣳ रुहेय॒म् नाक॒म् नाकꣳ॑ रुहेयम् । \newline
21. रु॒हे॒य॒ मिन्द्रा॒ये न्द्रा॑य रुहेयꣳ रुहेय॒ मिन्द्रा॑य । \newline
22. इन्द्रा॑य॒ वाचं॒ ॅवाच॒ मिन्द्रा॒ये न्द्रा॑य॒ वाच᳚म् । \newline
23. वाचं॑ ॅवदत वदत॒ वाचं॒ ॅवाचं॑ ॅवदत । \newline
24. व॒द॒ते न्द्र॒ मिन्द्रं॑ ॅवदत वद॒ते न्द्र᳚म् । \newline
25. इन्द्रं॒ ॅवाजं॒ ॅवाज॒ मिन्द्र॒ मिन्द्रं॒ ॅवाज᳚म् । \newline
26. वाज॑म् जापयत जापयत॒ वाजं॒ ॅवाज॑म् जापयत । \newline
27. जा॒प॒य॒ते न्द्र॒ इन्द्रो॑ जापयत जापय॒ते न्द्रः॑ । \newline
28. इन्द्रो॒ वाजं॒ ॅवाज॒ मिन्द्र॒ इन्द्रो॒ वाज᳚म् । \newline
29. वाज॑ मजयि दजयि॒द् वाजं॒ ॅवाज॑ मजयित् । \newline
30. अ॒ज॒य॒दित्य॑जयित् । \newline
31. अश्वा॑जनि वाजिनि वाजि॒ न्यश्वा॑ज॒ न्यश्वा॑जनि वाजिनि । \newline
32. अश्वा॑ज॒नीत्यश्व॑ - अ॒ज॒नि॒ । \newline
33. वा॒जि॒नि॒ वाजे॑षु॒ वाजे॑षु वाजिनि वाजिनि॒ वाजे॑षु । \newline
34. वाजे॑षु वाजिनी॒वति॑ वाजिनी॒वति॒ वाजे॑षु॒ वाजे॑षु वाजिनी॒वति॑ । \newline
35. वा॒जि॒नी॒व त्यश्वा॒ नश्वान्॑. वाजिनी॒वति॑ वाजिनी॒व त्यश्वान्॑ । \newline
36. वा॒जि॒नी॒व॒तीति॑ वाजिनी - व॒ति॒ । \newline
37. अश्वा᳚न् थ्स॒मथ्सु॑ स॒मथ्स्वश्वा॒ नश्वा᳚न् थ्स॒मथ्सु॑ । \newline
38. स॒मथ्सु॑ वाजय वाजय स॒मथ्सु॑ स॒मथ्सु॑ वाजय । \newline
39. स॒मथ्स्विति॑ स॒मत् - सु॒ । \newline
40. वा॒ज॒येति॑ वाजय । \newline
41. अर्वा᳚ ऽस्य॒स्यर्वा ऽर्वा॑ ऽसि । \newline
42. अ॒सि॒ सप्तिः॒ सप्ति॑ रस्यसि॒ सप्तिः॑ । \newline
43. सप्ति॑ रस्यसि॒ सप्तिः॒ सप्ति॑ रसि । \newline
44. अ॒सि॒ वा॒जी वा॒ज्य॑ स्यसि वा॒जी । \newline
45. वा॒ज्य॑ स्यसि वा॒जी वा॒ज्य॑सि । \newline
46. अ॒सि॒ वाजि॑नो॒ वाजि॑नो ऽस्यसि॒ वाजि॑नः । \newline
47. वाजि॑नो॒ वाजं॒ ॅवाजं॒ ॅवाजि॑नो॒ वाजि॑नो॒ वाज᳚म् । \newline
48. वाज॑म् धावत धावत॒ वाजं॒ ॅवाज॑म् धावत । \newline
49. धा॒व॒त॒ म॒रुता᳚म् म॒रुता᳚म् धावत धावत म॒रुता᳚म् । \newline
50. म॒रुता᳚म् प्रस॒वे प्र॑स॒वे म॒रुता᳚म् म॒रुता᳚म् प्रस॒वे । \newline
51. प्र॒स॒वे ज॑यत जयत प्रस॒वे प्र॑स॒वे ज॑यत । \newline
52. प्र॒स॒व इति॑ प्र - स॒वे । \newline
53. ज॒य॒त॒ वि वि ज॑यत जयत॒ वि । \newline
54. वि योज॑ना॒ योज॑ना॒ वि वि योज॑ना । \newline
55. योज॑ना मिमीद्ध्वम् मिमीद्ध्वं॒ ॅयोज॑ना॒ योज॑ना मिमीद्ध्वम् । \newline
56. मि॒मी॒द्ध्व॒ मद्ध्व॑नो॒ अद्ध्व॑नो मिमीद्ध्वम् मिमीद्ध्व॒ मद्ध्व॑नः । \newline
57. अद्ध्व॑नः स्कभ्नीत स्कभ्नी॒ता द्ध्व॑नो॒ अद्ध्व॑नः स्कभ्नीत । \newline
58. स्क॒भ्नी॒त॒ काष्ठा॒म् काष्ठाꣳ॑ स्कभ्नीत स्कभ्नीत॒ काष्ठा᳚म् । \newline

\textbf{Ghana Paata } \newline

1. दे॒वस्या॒ह म॒हम् दे॒वस्य॑ दे॒वस्या॒हꣳ स॑वि॒तुः स॑वि॒तु र॒हम् दे॒वस्य॑ दे॒वस्या॒हꣳ स॑वि॒तुः । \newline
2. अ॒हꣳ स॑वि॒तुः स॑वि॒तु र॒ह म॒हꣳ स॑वि॒तुः प्र॑स॒वे प्र॑स॒वे स॑वि॒तु र॒ह म॒हꣳ स॑वि॒तुः प्र॑स॒वे । \newline
3. स॒वि॒तुः प्र॑स॒वे प्र॑स॒वे स॑वि॒तुः स॑वि॒तुः प्र॑स॒वे बृह॒स्पति॑ना॒ बृह॒स्पति॑ना प्रस॒वे स॑वि॒तुः स॑वि॒तुः प्र॑स॒वे बृह॒स्पति॑ना । \newline
4. प्र॒स॒वे बृह॒स्पति॑ना॒ बृह॒स्पति॑ना प्रस॒वे प्र॑स॒वे बृह॒स्पति॑ना वाज॒जिता॑ वाज॒जिता॒ बृह॒स्पति॑ना प्रस॒वे प्र॑स॒वे बृह॒स्पति॑ना वाज॒जिता᳚ । \newline
5. प्र॒स॒व इति॑ प्र - स॒वे । \newline
6. बृह॒स्पति॑ना वाज॒जिता॑ वाज॒जिता॒ बृह॒स्पति॑ना॒ बृह॒स्पति॑ना वाज॒जिता॒ वाजं॒ ॅवाजं॑ ॅवाज॒जिता॒ बृह॒स्पति॑ना॒ बृह॒स्पति॑ना वाज॒जिता॒ वाज᳚म् । \newline
7. वा॒ज॒जिता॒ वाजं॒ ॅवाजं॑ ॅवाज॒जिता॑ वाज॒जिता॒ वाज॑म् जेषम् जेषं॒ ॅवाजं॑ ॅवाज॒जिता॑ वाज॒जिता॒ वाज॑म् जेषम् । \newline
8. वा॒ज॒जितेति॑ वाज - जिता᳚ । \newline
9. वाज॑म् जेषम् जेषं॒ ॅवाजं॒ ॅवाज॑म् जेषम् दे॒वस्य॑ दे॒वस्य॑ जेषं॒ ॅवाजं॒ ॅवाज॑म् जेषम् दे॒वस्य॑ । \newline
10. जे॒ष॒म् दे॒वस्य॑ दे॒वस्य॑ जेषम् जेषम् दे॒वस्या॒ह म॒हम् दे॒वस्य॑ जेषम् जेषम् दे॒वस्या॒हम् । \newline
11. दे॒वस्या॒ह म॒हम् दे॒वस्य॑ दे॒वस्या॒हꣳ स॑वि॒तुः स॑वि॒तु र॒हम् दे॒वस्य॑ दे॒वस्या॒हꣳ स॑वि॒तुः । \newline
12. अ॒हꣳ स॑वि॒तुः स॑वि॒तुर॒ह म॒हꣳ स॑वि॒तुः प्र॑स॒वे प्र॑स॒वे स॑वि॒तु र॒ह म॒हꣳ स॑वि॒तुः प्र॑स॒वे । \newline
13. स॒वि॒तुः प्र॑स॒वे प्र॑स॒वे स॑वि॒तुः स॑वि॒तुः प्र॑स॒वे बृह॒स्पति॑ना॒ बृह॒स्पति॑ना प्रस॒वे स॑वि॒तुः स॑वि॒तुः प्र॑स॒वे बृह॒स्पति॑ना । \newline
14. प्र॒स॒वे बृह॒स्पति॑ना॒ बृह॒स्पति॑ना प्रस॒वे प्र॑स॒वे बृह॒स्पति॑ना वाज॒जिता॑ वाज॒जिता॒ बृह॒स्पति॑ना प्रस॒वे प्र॑स॒वे बृह॒स्पति॑ना वाज॒जिता᳚ । \newline
15. प्र॒स॒व इति॑ प्र - स॒वे । \newline
16. बृह॒स्पति॑ना वाज॒जिता॑ वाज॒जिता॒ बृह॒स्पति॑ना॒ बृह॒स्पति॑ना वाज॒जिता॒ वर्.षि॑ष्ठं॒ ॅवर्.षि॑ष्ठं ॅवाज॒जिता॒ बृह॒स्पति॑ना॒ बृह॒स्पति॑ना वाज॒जिता॒ वर्.षि॑ष्ठम् । \newline
17. वा॒ज॒जिता॒ वर्.षि॑ष्ठं॒ ॅवर्.षि॑ष्ठं ॅवाज॒जिता॑ वाज॒जिता॒ वर्.षि॑ष्ठ॒म् नाक॒म् नाकं॒ ॅवर्.षि॑ष्ठं ॅवाज॒जिता॑ वाज॒जिता॒ वर्.षि॑ष्ठ॒म् नाक᳚म् । \newline
18. वा॒ज॒जितेति॑ वाज - जिता᳚ । \newline
19. वर्.षि॑ष्ठ॒म् नाक॒म् नाकं॒ ॅवर्.षि॑ष्ठं॒ ॅवर्.षि॑ष्ठ॒म् नाक(ग्म्॑) रुहेयꣳ रुहेय॒म् नाकं॒ ॅवर्.षि॑ष्ठं॒ ॅवर्.षि॑ष्ठ॒म् नाक(ग्म्॑) रुहेयम् । \newline
20. नाक(ग्म्॑) रुहेयꣳ रुहेय॒म् नाक॒म् नाक(ग्म्॑) रुहेय॒ मिन्द्रा॒ये न्द्रा॑य रुहेय॒म् नाक॒म् नाक(ग्म्॑) रुहेय॒ मिन्द्रा॑य । \newline
21. रु॒हे॒य॒ मिन्द्रा॒ये न्द्रा॑य रुहेयꣳ रुहेय॒ मिन्द्रा॑य॒ वाचं॒ ॅवाच॒ मिन्द्रा॑य रुहेयꣳ रुहेय॒ मिन्द्रा॑य॒ वाच᳚म् । \newline
22. इन्द्रा॑य॒ वाचं॒ ॅवाच॒ मिन्द्रा॒ये न्द्रा॑य॒ वाचं॑ ॅवदत वदत॒ वाच॒ मिन्द्रा॒ये न्द्रा॑य॒ वाचं॑ ॅवदत । \newline
23. वाचं॑ ॅवदत वदत॒ वाचं॒ ॅवाचं॑ ॅवद॒ते न्द्र॒ मिन्द्रं॑ ॅवदत॒ वाचं॒ ॅवाचं॑ ॅवद॒ते न्द्र᳚म् । \newline
24. व॒द॒ते न्द्र॒ मिन्द्रं॑ ॅवदत वद॒ते न्द्रं॒ ॅवाजं॒ ॅवाज॒ मिन्द्रं॑ ॅवदत वद॒ते न्द्रं॒ ॅवाज᳚म् । \newline
25. इन्द्रं॒ ॅवाजं॒ ॅवाज॒ मिन्द्र॒ मिन्द्रं॒ ॅवाज॑म् जापयत जापयत॒ वाज॒ मिन्द्र॒ मिन्द्रं॒ ॅवाज॑म् जापयत । \newline
26. वाज॑म् जापयत जापयत॒ वाजं॒ ॅवाज॑म् जापय॒ते न्द्र॒ इन्द्रो॑ जापयत॒ वाजं॒ ॅवाज॑म् जापय॒ते न्द्रः॑ । \newline
27. जा॒प॒य॒ते न्द्र॒ इन्द्रो॑ जापयत जापय॒ते न्द्रो॒ वाजं॒ ॅवाज॒ मिन्द्रो॑ जापयत जापय॒ते न्द्रो॒ वाज᳚म् । \newline
28. इन्द्रो॒ वाजं॒ ॅवाज॒ मिन्द्र॒ इन्द्रो॒ वाज॑ मजयि दजयि॒द् वाज॒ मिन्द्र॒ इन्द्रो॒ वाज॑ मजयित् । \newline
29. वाज॑ मजयि दजयि॒द् वाजं॒ ॅवाज॑ मजयित् । \newline
30. अ॒ज॒य॒दित्य॑जयित् । \newline
31. अश्वा॑जनि वाजिनि वाजि॒ न्यश्वा॑ज॒ न्यश्वा॑जनि वाजिनि॒ वाजे॑षु॒ वाजे॑षु वाजि॒ न्यश्वा॑ज॒ न्यश्वा॑जनि वाजिनि॒ वाजे॑षु । \newline
32. अश्वा॑ज॒नीत्यश्व॑ - अ॒ज॒नि॒ । \newline
33. वा॒जि॒नि॒ वाजे॑षु॒ वाजे॑षु वाजिनि वाजिनि॒ वाजे॑षु वाजिनी॒वति॑ वाजिनी॒वति॒ वाजे॑षु वाजिनि वाजिनि॒ वाजे॑षु वाजिनी॒वति॑ । \newline
34. वाजे॑षु वाजिनी॒वति॑ वाजिनी॒वति॒ वाजे॑षु॒ वाजे॑षु वाजिनी॒व त्यश्वा॒ नश्वान्॑. वाजिनी॒वति॒ वाजे॑षु॒ वाजे॑षु वाजिनी॒व त्यश्वान्॑ । \newline
35. वा॒जि॒नी॒व त्यश्वा॒ नश्वान्॑. वाजिनी॒वति॑ वाजिनी॒व त्यश्वा᳚न् थ्स॒मथ्सु॑ स॒मथ्स्वश्वान्॑. वाजिनी॒वति॑ वाजिनी॒व त्यश्वा᳚न् थ्स॒मथ्सु॑ । \newline
36. वा॒जि॒नी॒व॒तीति॑ वाजिनी - व॒ति॒ । \newline
37. अश्वा᳚न् थ्स॒मथ्सु॑ स॒मथ्स्वश्वा॒ नश्वा᳚न् थ्स॒मथ्सु॑ वाजय वाजय स॒मथ्स्वश्वा॒ नश्वा᳚न् थ्स॒मथ्सु॑ वाजय । \newline
38. स॒मथ्सु॑ वाजय वाजय स॒मथ्सु॑ स॒मथ्सु॑ वाजय । \newline
39. स॒मथ्स्विति॑ स॒मत् - सु॒ । \newline
40. वा॒ज॒येति॑ वाजय । \newline
41. अर्वा᳚ ऽस्य॒स्यर्वा ऽर्वा॑ ऽसि॒ सप्तिः॒ सप्ति॑ र॒स्यर्वा ऽर्वा॑ ऽसि॒ सप्तिः॑ । \newline
42. अ॒सि॒ सप्तिः॒ सप्ति॑ रस्यसि॒ सप्ति॑रस्यसि॒ सप्ति॑ रस्यसि॒ सप्ति॑ रसि । \newline
43. सप्ति॑ रस्यसि॒ सप्तिः॒ सप्ति॑ रसि वा॒जी वा॒ज्य॑सि॒ सप्तिः॒ सप्ति॑ रसि वा॒जी । \newline
44. अ॒सि॒ वा॒जी वा॒ज्य॑स्यसि वा॒ज्य॑स्यसि वा॒ज्य॑स्यसि वा॒ज्य॑सि । \newline
45. वा॒ज्य॑स्यसि वा॒जी वा॒ज्य॑सि॒ वाजि॑नो॒ वाजि॑नो ऽसि वा॒जी वा॒ज्य॑सि॒ वाजि॑नः । \newline
46. अ॒सि॒ वाजि॑नो॒ वाजि॑नो ऽस्यसि॒ वाजि॑नो॒ वाजं॒ ॅवाजं॒ ॅवाजि॑नो ऽस्यसि॒ वाजि॑नो॒ वाज᳚म् । \newline
47. वाजि॑नो॒ वाजं॒ ॅवाजं॒ ॅवाजि॑नो॒ वाजि॑नो॒ वाज॑म् धावत धावत॒ वाजं॒ ॅवाजि॑नो॒ वाजि॑नो॒ वाज॑म् धावत । \newline
48. वाज॑म् धावत धावत॒ वाजं॒ ॅवाज॑म् धावत म॒रुता᳚म् म॒रुता᳚म् धावत॒ वाजं॒ ॅवाज॑म् धावत म॒रुता᳚म् । \newline
49. धा॒व॒त॒ म॒रुता᳚म् म॒रुता᳚म् धावत धावत म॒रुता᳚म् प्रस॒वे प्र॑स॒वे म॒रुता᳚म् धावत धावत म॒रुता᳚म् प्रस॒वे । \newline
50. म॒रुता᳚म् प्रस॒वे प्र॑स॒वे म॒रुता᳚म् म॒रुता᳚म् प्रस॒वे ज॑यत जयत प्रस॒वे म॒रुता᳚म् म॒रुता᳚म् प्रस॒वे ज॑यत । \newline
51. प्र॒स॒वे ज॑यत जयत प्रस॒वे प्र॑स॒वे ज॑यत॒ वि वि ज॑यत प्रस॒वे प्र॑स॒वे ज॑यत॒ वि । \newline
52. प्र॒स॒व इति॑ प्र - स॒वे । \newline
53. ज॒य॒त॒ वि वि ज॑यत जयत॒ वि योज॑ना॒ योज॑ना॒ वि ज॑यत जयत॒ वि योज॑ना । \newline
54. वि योज॑ना॒ योज॑ना॒ वि वि योज॑ना मिमीद्ध्वम् मिमीद्ध्वं॒ ॅयोज॑ना॒ वि वि योज॑ना मिमीद्ध्वम् । \newline
55. योज॑ना मिमीद्ध्वम् मिमीद्ध्वं॒ ॅयोज॑ना॒ योज॑ना मिमीद्ध्व॒ मद्ध्व॑नो॒ अद्ध्व॑नो मिमीद्ध्वं॒ ॅयोज॑ना॒ योज॑ना मिमीद्ध्व॒ मद्ध्व॑नः । \newline
56. मि॒मी॒द्ध्व॒ मद्ध्व॑नो॒ अद्ध्व॑नो मिमीद्ध्वम् मिमीद्ध्व॒ मद्ध्व॑नः स्कभ्नीत स्कभ्नी॒ताद्ध्व॑नो मिमीद्ध्वम् मिमीद्ध्व॒ मद्ध्व॑नः स्कभ्नीत । \newline
57. अद्ध्व॑नः स्कभ्नीत स्कभ्नी॒ ताद्ध्व॑नो॒ अद्ध्व॑नः स्कभ्नीत॒ काष्ठा॒म् काष्ठा(ग्ग्॑) स्कभ्नी॒ ताद्ध्व॑नो॒ अद्ध्व॑नः स्कभ्नीत॒ काष्ठा᳚म् । \newline
58. स्क॒भ्नी॒त॒ काष्ठा॒म् काष्ठा(ग्ग्॑) स्कभ्नीत स्कभ्नीत॒ काष्ठा᳚म् गच्छत गच्छत॒ काष्ठा(ग्ग्॑) स्कभ्नीत स्कभ्नीत॒ काष्ठा᳚म् गच्छत । \newline
\pagebreak
\markright{ TS 1.7.8.2  \hfill https://www.vedavms.in \hfill}
\addcontentsline{toc}{section}{ TS 1.7.8.2 }
\section*{ TS 1.7.8.2 }

\textbf{TS 1.7.8.2 } \newline
\textbf{Samhita Paata} \newline

काष्ठां᳚ गच्छत॒ वाजे॑वाजेऽवत वाजिनो नो॒ धने॑षु विप्रा अमृता ऋतज्ञाः ॥ अ॒स्य मद्ध्वः॑ पिबत मा॒दय॑द्ध्वं तृ॒प्ता या॑त प॒थिभि॑र् देव॒यानैः᳚ ॥ ते नो॒ अर्व॑न्तो हवन॒श्रुतो॒ हवं॒ ॅविश्वे॑ शृण्वन्तु वा॒जिनः॑ ॥ मि॒तद्र॑वः सहस्र॒सा मे॒धसा॑ता सनि॒ष्यवः॑ । म॒हो ये रत्नꣳ॑ समि॒थेषु॑ जभ्रि॒रे शन्नो॑ भवन्तु वा॒जिनो॒ हवे॑षु ॥दे॒वता॑ता मि॒तद्र॑वः स्व॒र्काः । जं॒भय॒न्तोऽहिं॒ ॅवृकꣳ॒॒ रक्षाꣳ॑सि॒ सने᳚म्य॒स्मद्यु॑यव॒ - [ ] \newline

\textbf{Pada Paata} \newline

काष्ठा᳚म् । ग॒च्छ॒त॒ । वाजे॑वाज॒ इति॒ वाजे᳚-वा॒जे॒ । अ॒व॒त॒ । वा॒जि॒नः॒ । नः॒ । धने॑षु । वि॒प्राः॒ । अ॒मृ॒ताः॒ । ऋ॒त॒ज्ञा॒ इत्यृ॑त - ज्ञाः॒ ॥ अ॒स्य । मद्ध्वः॑ । पि॒ब॒त॒ । मा॒दय॑द्ध्वम् । तृ॒प्ताः । या॒त॒ । प॒थिभि॒रिति॑ प॒थि - भिः॒ । दे॒व॒यान॒रिति॑ देव - यानैः᳚ ॥ ते । नः॒ । अर्व॑न्तः । ह॒व॒न॒श्रुत॒ इति॑ हवन - श्रुतः॑ । हव᳚म् । विश्वे᳚ । शृ॒ण्व॒न्तु॒ । वा॒जिनः॑ ॥ मि॒तद्र॑व॒ इति॑ मि॒त - द्र॒वः॒ । स॒ह॒स्र॒सा इति॑ सहस्र-साः । मे॒धसा॒तेति॑ मे॒ध-सा॒ता॒ । स॒नि॒ष्यवः॑ ॥ म॒हः । ये । रत्न᳚म् । स॒मि॒थेष्विति॑ सं - इ॒थेषु॑ । ज॒भ्रि॒रे । शम् । नः॒ । भ॒व॒न्तु॒ । वा॒जिनः॑ । हवे॑षु ॥ दे॒वता॒तेति॑ दे॒व - ता॒ता॒ । मि॒तद्र॑व॒ इति॑ मि॒त - द्र॒वः॒ । स्व॒र्का इति॑ सू - अ॒र्काः ॥ ज॒भंय॑न्तः । अहि᳚म् । वृक᳚म् । रक्षाꣳ॑सि । सने॑मि । अ॒स्मत् । यु॒य॒व॒न्न् ।  \newline


\textbf{Krama Paata} \newline

काष्ठा᳚म् गच्छत । ग॒च्छ॒त॒ वाजे॑वाजे । वाजे॑वाजेऽवत । वाजे॑वाज॒ इति॒ वाजे᳚ - वा॒जे॒ । अ॒व॒त॒ वा॒जि॒नः॒ । वा॒जि॒नो॒ नः॒ । नो॒ धने॑षु । धने॑षु विप्राः । वि॒प्रा॒ अ॒मृ॒ताः॒ । अ॒मृ॒ता॒ ऋ॒त॒ज्ञाः॒ । ऋ॒त॒ज्ञा॒ इत्यृ॑त - ज्ञाः॒ ॥ अ॒स्य मद्ध्वः॑ । मद्ध्वः॑ पिबत । पि॒ब॒त॒ मा॒दय॑द्ध्वम् । मा॒दय॑द्ध्वम् तृ॒प्ताः । तृ॒प्ता या॑त । या॒त॒ प॒थिभिः॑ । प॒थिभि॑र् देव॒यानैः᳚ । प॒थिभि॒रिति॑ प॒थि - भिः॒ । दे॒व॒यानै॒रिति॑ देव - यानैः᳚ ॥ ते नः॑ । नो॒ अर्व॑न्तः । अर्व॑न्तो हवन॒श्रुतः॑ । ह॒व॒न॒श्रुतो॒ हव᳚म् । ह॒व॒न॒श्रुत॒ इति॑ हवन - श्रुतः॑ । हव॒म् ॅविश्वे᳚ । विश्वे॑ शृण्वन्तु । शृ॒ण्व॒न्तु॒ वा॒जिनः॑ । वा॒जिन॒ इति॑ वा॒जिनः॑ ॥ मि॒तद्र॑वः सहस्र॒साः । मि॒तद्र॑व॒ इति॑ मि॒त - द्र॒वः॒ । स॒ह॒स्र॒सा मे॒धसा॑ता । स॒ह॒स्र॒सा इति॑ सहस्र - साः । मे॒धसा॑ता सनि॒ष्यवः॑ । मे॒धसा॒तेति॑ मे॒ध - सा॒ता॒ । स॒नि॒ष्यव॒ इति॑ सनि॒ष्यवः॑ ॥ म॒हो ये । ये रत्न᳚म् । रत्नꣳ॑ समि॒थेषु॑ । स॒मि॒थेषु॑ जभ्रि॒रे । स॒मि॒थेष्विति॑ सम् - इ॒थेषु॑ । ज॒भ्रि॒रे शम् । शम् नः॑ । नो॒ भ॒व॒न्तु॒ । भ॒व॒न्तु॒ वा॒जिनः॑ । वा॒जिनो॒ हवे॑षु । हवे॒ष्विति॒ हवे॑षु ॥ दे॒वता॑ता मि॒तद्र॑वः । दे॒वता॒तेति॑ दे॒व - ता॒ता॒ । मि॒तद्र॑वः स्व॒र्काः । मि॒तद्र॑व॒ इति॑ मि॒त - द्र॒वः॒ । स्व॒र्का इति॑ सु - अ॒र्काः ॥ ज॒म्भय॒न्तोऽहि᳚म् । अहिं॒ ॅवृक᳚म् । वृकꣳ॒॒ रक्षाꣳ॑सि । रक्षाꣳ॑सि॒ सने॑मि । सने᳚म्य॒स्मत् । अ॒स्मद् यु॑यवन्न् । यु॒य॒व॒न्नमी॑वाः \newline

\textbf{Jatai Paata} \newline

1. काष्ठा᳚म् गच्छत गच्छत॒ काष्ठा॒म् काष्ठा᳚म् गच्छत । \newline
2. ग॒च्छ॒त॒ वाजे॑वाजे॒ वाजे॑वाजे गच्छत गच्छत॒ वाजे॑वाजे । \newline
3. वाजे॑वाजे ऽवतावत॒ वाजे॑वाजे॒ वाजे॑वाजे ऽवत । \newline
4. वाजे॑वाज॒ इति॒ वाजे᳚ - वा॒जे॒ । \newline
5. अ॒व॒त॒ वा॒जि॒नो॒ वा॒जि॒नो॒ ऽव॒ता॒व॒त॒ वा॒जि॒नः॒ । \newline
6. वा॒जि॒नो॒ नो॒ नो॒ वा॒जि॒नो॒ वा॒जि॒नो॒ नः॒ । \newline
7. नो॒ धने॑षु॒ धने॑षु नो नो॒ धने॑षु । \newline
8. धने॑षु विप्रा विप्रा॒ धने॑षु॒ धने॑षु विप्राः । \newline
9. वि॒प्रा॒ अ॒मृ॒ता॒ अ॒मृ॒ता॒ वि॒प्रा॒ वि॒प्रा॒ अ॒मृ॒ताः॒ । \newline
10. अ॒मृ॒ता॒ ऋ॒त॒ज्ञा॒ ऋ॒त॒ज्ञा॒ अ॒मृ॒ता॒ अ॒मृ॒ता॒ ऋ॒त॒ज्ञाः॒ । \newline
11. ऋ॒त॒ज्ञा॒ इत्यृ॑त - ज्ञाः॒ । \newline
12. अ॒स्य मद्ध्वो॒ मद्ध्वो॑ अ॒स्यास्य मद्ध्वः॑ । \newline
13. मद्ध्वः॑ पिबत पिबत॒ मद्ध्वो॒ मद्ध्वः॑ पिबत । \newline
14. पि॒ब॒त॒ मा॒दय॑द्ध्वम् मा॒दय॑द्ध्वम् पिबत पिबत मा॒दय॑द्ध्वम् । \newline
15. मा॒दय॑द्ध्वम् तृ॒प्तास्तृ॒प्ता मा॒दय॑द्ध्वम् मा॒दय॑द्ध्वम् तृ॒प्ताः । \newline
16. तृ॒प्ता या॑त यात तृ॒प्ता स्तृ॒प्ता या॑त । \newline
17. या॒त॒ प॒थिभिः॑ प॒थिभि॑र् यात यात प॒थिभिः॑ । \newline
18. प॒थिभि॑र् देव॒यानै᳚र् देव॒यानैः᳚ प॒थिभिः॑ प॒थिभि॑र् देव॒यानैः᳚ । \newline
19. प॒थिभि॒रिति॑ प॒थि - भिः॒ । \newline
20. दे॒व॒यानै॒रिति॑ देव - यानैः᳚ । \newline
21. ते नो॑ न॒ स्ते ते नः॑ । \newline
22. नो॒ अर्व॑न्तो॒ अर्व॑न्तो नो नो॒ अर्व॑न्तः । \newline
23. अर्व॑न्तो हवन॒श्रुतो॑ हवन॒श्रुतो॒ अर्व॑न्तो॒ अर्व॑न्तो हवन॒श्रुतः॑ । \newline
24. ह॒व॒न॒श्रुतो॒ हवꣳ॒॒ हवꣳ॑ हवन॒श्रुतो॑ हवन॒श्रुतो॒ हव᳚म् । \newline
25. ह॒व॒न॒श्रुत॒ इति॑ हवन - श्रुतः॑ । \newline
26. हवं॒ ॅविश्वे॒ विश्वे॒ हवꣳ॒॒ हवं॒ ॅविश्वे᳚ । \newline
27. विश्वे॑ शृण्वन्तु शृण्वन्तु॒ विश्वे॒ विश्वे॑ शृण्वन्तु । \newline
28. शृ॒ण्व॒न्तु॒ वा॒जिनो॑ वा॒जिनः॑ शृण्वन्तु शृण्वन्तु वा॒जिनः॑ । \newline
29. वा॒जिन॒ इति॑ वा॒जिनः॑ । \newline
30. मि॒तद्र॑वः सहस्र॒साः स॑हस्र॒सा मि॒तद्र॑वो मि॒तद्र॑वः सहस्र॒साः । \newline
31. मि॒तद्र॑व॒ इति॑ मि॒त - द्र॒वः॒ । \newline
32. स॒ह॒स्र॒सा मे॒धसा॑ता मे॒धसा॑ता सहस्र॒साः स॑हस्र॒सा मे॒धसा॑ता । \newline
33. स॒ह॒स्र॒सा इति॑ सहस्र - साः । \newline
34. मे॒धसा॑ता सनि॒ष्यवः॑ सनि॒ष्यवो॑ मे॒धसा॑ता मे॒धसा॑ता सनि॒ष्यवः॑ । \newline
35. मे॒धसा॒तेति॑ मे॒ध - सा॒ता॒ । \newline
36. स॒नि॒ष्यव॒ इति॑ सनि॒ष्यवः॑ । \newline
37. म॒हो ये ये म॒हो म॒हो ये । \newline
38. ये रत्नꣳ॒॒ रत्नं॒ ॅये ये रत्न᳚म् । \newline
39. रत्नꣳ॑ समि॒थेषु॑ समि॒थेषु॒ रत्नꣳ॒॒ रत्नꣳ॑ समि॒थेषु॑ । \newline
40. स॒मि॒थेषु॑ जभ्रि॒रे ज॑भ्रि॒रे स॑मि॒थेषु॑ समि॒थेषु॑ जभ्रि॒रे । \newline
41. स॒मि॒थेष्विति॑ सं - इ॒थेषु॑ । \newline
42. ज॒भ्रि॒रे शꣳ शम् ज॑भ्रि॒रे ज॑भ्रि॒रे शम् । \newline
43. शन्नो॑ नः॒ शꣳ शन्नः॑ । \newline
44. नो॒ भ॒व॒न्तु॒ भ॒व॒न्तु॒ नो॒ नो॒ भ॒व॒न्तु॒ । \newline
45. भ॒व॒न्तु॒ वा॒जिनो॑ वा॒जिनो॑ भवन्तु भवन्तु वा॒जिनः॑ । \newline
46. वा॒जिनो॒ हवे॑षु॒ हवे॑षु वा॒जिनो॑ वा॒जिनो॒ हवे॑षु । \newline
47. हवे॒ष्विति॒ हवे॑षु । \newline
48. दे॒वता॑ता मि॒तद्र॑वो मि॒तद्र॑वो दे॒वता॑ता दे॒वता॑ता मि॒तद्र॑वः । \newline
49. दे॒वता॒तेति॑ दे॒व - ता॒ता॒ । \newline
50. मि॒तद्र॑वः स्व॒र्काः स्व॒र्का मि॒तद्र॑वो मि॒तद्र॑वः स्व॒र्काः । \newline
51. मि॒तद्र॑व॒ इति॑ मि॒त - द्र॒वः॒ । \newline
52. स्व॒र्का इति॑ सु - अ॒र्काः । \newline
53. जं॒भय॒न्तो ऽहि॒ महि॑म् जं॒भय॑न्तो जं॒भय॒न्तो ऽहि᳚म् । \newline
54. अहिं॒ ॅवृकं॒ ॅवृक॒ महि॒ महिं॒ ॅवृक᳚म् । \newline
55. वृकꣳ॒॒ रक्षाꣳ॑सि॒ रक्षाꣳ॑सि॒ वृकं॒ ॅवृकꣳ॒॒ रक्षाꣳ॑सि । \newline
56. रक्षाꣳ॑सि॒ सने॑मि॒ सने॑मि॒ रक्षाꣳ॑सि॒ रक्षाꣳ॑सि॒ सने॑मि । \newline
57. सने᳚म्य॒स्म द॒स्मथ् सने॑मि॒ सने᳚म्य॒स्मत् । \newline
58. अ॒स्मद् यु॑यवन् युयवन् न॒स्म द॒स्मद् यु॑यवन्न् । \newline
59. यु॒य॒व॒न् नमी॑वा॒ अमी॑वा युयवन्. युयव॒न् नमी॑वाः । \newline

\textbf{Ghana Paata } \newline

1. काष्ठा᳚म् गच्छत गच्छत॒ काष्ठा॒म् काष्ठा᳚म् गच्छत॒ वाजे॑वाजे॒ वाजे॑वाजे गच्छत॒ काष्ठा॒म् काष्ठा᳚म् गच्छत॒ वाजे॑वाजे । \newline
2. ग॒च्छ॒त॒ वाजे॑वाजे॒ वाजे॑वाजे गच्छत गच्छत॒ वाजे॑वाजे ऽवतावत॒ वाजे॑वाजे गच्छत गच्छत॒ वाजे॑वाजे ऽवत । \newline
3. वाजे॑वाजे ऽवतावत॒ वाजे॑वाजे॒ वाजे॑वाजे ऽवत वाजिनो वाजिनो ऽवत॒ वाजे॑वाजे॒ वाजे॑वाजे ऽवत वाजिनः । \newline
4. वाजे॑वाज॒ इति॒ वाजे᳚ - वा॒जे॒ । \newline
5. अ॒व॒त॒ वा॒जि॒नो॒ वा॒जि॒नो॒ ऽव॒ता॒व॒त॒ वा॒जि॒नो॒ नो॒ नो॒ वा॒जि॒नो॒ ऽव॒ता॒व॒त॒ वा॒जि॒नो॒ नः॒ । \newline
6. वा॒जि॒नो॒ नो॒ नो॒ वा॒जि॒नो॒ वा॒जि॒नो॒ नो॒ धने॑षु॒ धने॑षु नो वाजिनो वाजिनो नो॒ धने॑षु । \newline
7. नो॒ धने॑षु॒ धने॑षु नो नो॒ धने॑षु विप्रा विप्रा॒ धने॑षु नो नो॒ धने॑षु विप्राः । \newline
8. धने॑षु विप्रा विप्रा॒ धने॑षु॒ धने॑षु विप्रा अमृता अमृता विप्रा॒ धने॑षु॒ धने॑षु विप्रा अमृताः । \newline
9. वि॒प्रा॒ अ॒मृ॒ता॒ अ॒मृ॒ता॒ वि॒प्रा॒ वि॒प्रा॒ अ॒मृ॒ता॒ ऋ॒त॒ज्ञा॒ ऋ॒त॒ज्ञा॒ अ॒मृ॒ता॒ वि॒प्रा॒ वि॒प्रा॒ अ॒मृ॒ता॒ ऋ॒त॒ज्ञाः॒ । \newline
10. अ॒मृ॒ता॒ ऋ॒त॒ज्ञा॒ ऋ॒त॒ज्ञा॒ अ॒मृ॒ता॒ अ॒मृता॒ ऋ॒त॒ज्ञाः॒ । \newline
11. ऋ॒त॒ज्ञा॒ इत्यृ॑त - ज्ञाः॒ । \newline
12. अ॒स्य मद्ध्वो॒ मद्ध्वो॑ अ॒स्यास्य मद्ध्वः॑ पिबत पिबत॒ मद्ध्वो॑ अ॒स्यास्य मद्ध्वः॑ पिबत । \newline
13. मद्ध्वः॑ पिबत पिबत॒ मद्ध्वो॒ मद्ध्वः॑ पिबत मा॒दय॑द्ध्वम् मा॒दय॑द्ध्वम् पिबत॒ मद्ध्वो॒ मद्ध्वः॑ पिबत मा॒दय॑द्ध्वम् । \newline
14. पि॒ब॒त॒ मा॒दय॑द्ध्वम् मा॒दय॑द्ध्वम् पिबत पिबत मा॒दय॑द्ध्वम् तृ॒प्तास्तृ॒प्ता मा॒दय॑द्ध्वम् पिबत पिबत मा॒दय॑द्ध्वम् तृ॒प्ताः । \newline
15. मा॒दय॑द्ध्वम् तृ॒प्तास्तृ॒प्ता मा॒दय॑द्ध्वम् मा॒दय॑द्ध्वम् तृ॒प्ता या॑त यात तृ॒प्ता मा॒दय॑द्ध्वम् मा॒दय॑द्ध्वम् तृ॒प्ता या॑त । \newline
16. तृ॒प्ता या॑त यात तृ॒प्तास्तृ॒प्ता या॑त प॒थिभिः॑ प॒थिभि॑र् यात तृ॒प्तास्तृ॒प्ता या॑त प॒थिभिः॑ । \newline
17. या॒त॒ प॒थिभिः॑ प॒थिभि॑र् यात यात प॒थिभि॑र् देव॒यानै᳚र् देव॒यानैः᳚ प॒थिभि॑र् यात यात प॒थिभि॑र् देव॒यानैः᳚ । \newline
18. प॒थिभि॑र् देव॒यानै᳚र् देव॒यानैः᳚ प॒थिभिः॑ प॒थिभि॑र् देव॒यानैः᳚ । \newline
19. प॒थिभि॒रिति॑ प॒थि - भिः॒ । \newline
20. दे॒व॒यानै॒रिति॑ देव - यानैः᳚ । \newline
21. ते नो॑ न॒स्ते ते नो॒ अर्व॑न्तो॒ अर्व॑न्तो न॒स्ते ते नो॒ अर्व॑न्तः । \newline
22. नो॒ अर्व॑न्तो॒ अर्व॑न्तो नो नो॒ अर्व॑न्तो हवन॒श्रुतो॑ हवन॒श्रुतो॒ अर्व॑न्तो नो नो॒ अर्व॑न्तो हवन॒श्रुतः॑ । \newline
23. अर्व॑न्तो हवन॒श्रुतो॑ हवन॒श्रुतो॒ अर्व॑न्तो॒ अर्व॑न्तो हवन॒श्रुतो॒ हव॒(ग्म्॒) हव(ग्म्॑) हवन॒श्रुतो॒ अर्व॑न्तो॒ अर्व॑न्तो हवन॒श्रुतो॒ हव᳚म् । \newline
24. ह॒व॒न॒श्रुतो॒ हव॒(ग्म्॒) हव(ग्म्॑) हवन॒श्रुतो॑ हवन॒श्रुतो॒ हवं॒ ॅविश्वे॒ विश्वे॒ हव(ग्म्॑) हवन॒श्रुतो॑ हवन॒श्रुतो॒ हवं॒ ॅविश्वे᳚ । \newline
25. ह॒व॒न॒श्रुत॒ इति॑ हवन - श्रुतः॑ । \newline
26. हवं॒ ॅविश्वे॒ विश्वे॒ हव॒(ग्म्॒) हवं॒ ॅविश्वे॑ शृण्वन्तु शृण्वन्तु॒ विश्वे॒ हव॒(ग्म्॒) हवं॒ ॅविश्वे॑ शृण्वन्तु । \newline
27. विश्वे॑ शृण्वन्तु शृण्वन्तु॒ विश्वे॒ विश्वे॑ शृण्वन्तु वा॒जिनो॑ वा॒जिनः॑ शृण्वन्तु॒ विश्वे॒ विश्वे॑ शृण्वन्तु वा॒जिनः॑ । \newline
28. शृ॒ण्व॒न्तु॒ वा॒जिनो॑ वा॒जिनः॑ शृण्वन्तु शृण्वन्तु वा॒जिनः॑ । \newline
29. वा॒जिन॒ इति॑ वा॒जिनः॑ । \newline
30. मि॒तद्र॑वः सहस्र॒साः स॑हस्र॒सा मि॒तद्र॑वो मि॒तद्र॑वः सहस्र॒सा मे॒धसा॑ता मे॒धसा॑ता सहस्र॒सा मि॒तद्र॑वो मि॒तद्र॑वः सहस्र॒सा मे॒धसा॑ता । \newline
31. मि॒तद्र॑व॒ इति॑ मि॒त - द्र॒वः॒ । \newline
32. स॒ह॒स्र॒सा मे॒धसा॑ता मे॒धसा॑ता सहस्र॒साः स॑हस्र॒सा मे॒धसा॑ता सनि॒ष्यवः॑ सनि॒ष्यवो॑ मे॒धसा॑ता सहस्र॒साः स॑हस्र॒सा मे॒धसा॑ता सनि॒ष्यवः॑ । \newline
33. स॒ह॒स्र॒सा इति॑ सहस्र - साः । \newline
34. मे॒धसा॑ता सनि॒ष्यवः॑ सनि॒ष्यवो॑ मे॒धसा॑ता मे॒धसा॑ता सनि॒ष्यवः॑ । \newline
35. मे॒धसा॒तेति॑ मे॒ध - सा॒ता॒ । \newline
36. स॒नि॒ष्यव॒ इति॑ सनि॒ष्यवः॑ । \newline
37. म॒हो ये ये म॒हो म॒हो ये रत्न॒(ग्म्॒) रत्नं॒ ॅये म॒हो म॒हो ये रत्न᳚म् । \newline
38. ये रत्न॒(ग्म्॒) रत्नं॒ ॅये ये रत्न(ग्म्॑) समि॒थेषु॑ समि॒थेषु॒ रत्नं॒ ॅये ये रत्न(ग्म्॑) समि॒थेषु॑ । \newline
39. रत्न(ग्म्॑) समि॒थेषु॑ समि॒थेषु॒ रत्न॒(ग्म्॒) रत्न(ग्म्॑) समि॒थेषु॑ जभ्रि॒रे ज॑भ्रि॒रे स॑मि॒थेषु॒ रत्न॒(ग्म्॒) रत्न(ग्म्॑) समि॒थेषु॑ जभ्रि॒रे । \newline
40. स॒मि॒थेषु॑ जभ्रि॒रे ज॑भ्रि॒रे स॑मि॒थेषु॑ समि॒थेषु॑ जभ्रि॒रे शꣳ शम् ज॑भ्रि॒रे स॑मि॒थेषु॑ समि॒थेषु॑ जभ्रि॒रे शम् । \newline
41. स॒मि॒थेष्विति॑ सं - इ॒थेषु॑ । \newline
42. ज॒भ्रि॒रे शꣳ शम् ज॑भ्रि॒रे ज॑भ्रि॒रे शन्नो॑ नः॒ शम् ज॑भ्रि॒रे ज॑भ्रि॒रे शन्नः॑ । \newline
43. शन्नो॑ नः॒ शꣳ शन्नो॑ भवन्तु भवन्तु नः॒ शꣳ शन्नो॑ भवन्तु । \newline
44. नो॒ भ॒व॒न्तु॒ भ॒व॒न्तु॒ नो॒ नो॒ भ॒व॒न्तु॒ वा॒जिनो॑ वा॒जिनो॑ भवन्तु नो नो भवन्तु वा॒जिनः॑ । \newline
45. भ॒व॒न्तु॒ वा॒जिनो॑ वा॒जिनो॑ भवन्तु भवन्तु वा॒जिनो॒ हवे॑षु॒ हवे॑षु वा॒जिनो॑ भवन्तु भवन्तु वा॒जिनो॒ हवे॑षु । \newline
46. वा॒जिनो॒ हवे॑षु॒ हवे॑षु वा॒जिनो॑ वा॒जिनो॒ हवे॑षु । \newline
47. हवे॒ष्विति॒ हवे॑षु । \newline
48. दे॒वता॑ता मि॒तद्र॑वो मि॒तद्र॑वो दे॒वता॑ता दे॒वता॑ता मि॒तद्र॑वः स्व॒र्काः स्व॒र्का मि॒तद्र॑वो दे॒वता॑ता दे॒वता॑ता मि॒तद्र॑वः स्व॒र्काः । \newline
49. दे॒वता॒तेति॑ दे॒व - ता॒ता॒ । \newline
50. मि॒तद्र॑वः स्व॒र्काः स्व॒र्का मि॒तद्र॑वो मि॒तद्र॑वः स्व॒र्काः । \newline
51. मि॒तद्र॑व॒ इति॑ मि॒त - द्र॒वः॒ । \newline
52. स्व॒र्का इति॑ सु - अ॒र्काः । \newline
53. जं॒भय॒न्तो ऽहि॒ महि॑म् जं॒भय॑न्तो जं॒भय॒न्तो ऽहिं॒ ॅवृकं॒ ॅवृक॒ महि॑म् जं॒भय॑न्तो जं॒भय॒न्तो ऽहिं॒ ॅवृक᳚म् । \newline
54. अहिं॒ ॅवृकं॒ ॅवृक॒ महि॒ महिं॒ ॅवृक॒(ग्म्॒) रक्षा(ग्म्॑)सि॒ रक्षा(ग्म्॑)सि॒ वृक॒ महि॒ महिं॒ ॅवृक॒(ग्म्॒) रक्षा(ग्म्॑)सि । \newline
55. वृक॒(ग्म्॒) रक्षा(ग्म्॑)सि॒ रक्षा(ग्म्॑)सि॒ वृकं॒ ॅवृक॒(ग्म्॒) रक्षा(ग्म्॑)सि॒ सने॑मि॒ सने॑मि॒ रक्षा(ग्म्॑)सि॒ वृकं॒ ॅवृक॒(ग्म्॒) रक्षा(ग्म्॑)सि॒ सने॑मि । \newline
56. रक्षा(ग्म्॑)सि॒ सने॑मि॒ सने॑मि॒ रक्षा(ग्म्॑)सि॒ रक्षा(ग्म्॑)सि॒ सने᳚ म्य॒स्म द॒स्मथ् सने॑मि॒ रक्षा(ग्म्॑)सि॒ रक्षा(ग्म्॑)सि॒ सने᳚म्य॒स्मत् । \newline
57. सने᳚ म्य॒स्म द॒स्मथ् सने॑मि॒ सने᳚म्य॒स्मद् यु॑यवन्. युयवन् न॒स्मथ् सने॑मि॒ सने᳚म्य॒स्मद् यु॑यवन्न् । \newline
58. अ॒स्मद् यु॑यवन्. युयवन् न॒स्म द॒स्मद् यु॑यव॒न् नमी॑वा॒ अमी॑वा युयवन् न॒स्म द॒स्मद् यु॑यव॒न् नमी॑वाः । \newline
59. यु॒य॒व॒न् नमी॑वा॒ अमी॑वा युयवन्. युयव॒न् नमी॑वाः । \newline
\pagebreak
\markright{ TS 1.7.8.3  \hfill https://www.vedavms.in \hfill}
\addcontentsline{toc}{section}{ TS 1.7.8.3 }
\section*{ TS 1.7.8.3 }

\textbf{TS 1.7.8.3 } \newline
\textbf{Samhita Paata} \newline

न्नमी॑वाः ॥ ए॒ष स्य वा॒जी क्षि॑प॒णिं तु॑रण्यति ग्री॒वायां᳚ ब॒द्धो अ॑पिक॒क्ष आ॒सनि॑ । क्रतुं॑ दधि॒क्रा अनु॑ स॒न्तवी᳚त्वत् प॒थामङ्काꣳ॒॒स्यन्वा॒पनी॑फणत् ॥उ॒त स्मा᳚स्य॒ द्रव॑त-स्तुरण्य॒तः प॒र्णं न वे-रनु॑ वाति प्रग॒र्द्धिनः॑ । श्ये॒नस्ये॑व॒ ध्रज॑तो अङ्क॒सं परि॑ दधि॒क्राव्.ण्णः॑ स॒होर्जा तरि॑त्रतः ॥ आ मा॒ वाज॑स्य प्रस॒वो ज॑गम्या॒दा द्यावा॑पृथि॒वी वि॒श्वशं॑भू । आ मा॑ गन्तां पि॒तरा॑ - [ ] \newline

\textbf{Pada Paata} \newline

अमी॑वाः ॥ ए॒षः । स्यः । वा॒जी । क्षि॒प॒णिम् । तु॒र॒ण्य॒ति॒ । ग्री॒वाया᳚म् । ब॒द्धः । अ॒पि॒क॒क्ष इत्य॑पि - क॒क्षे । आ॒सनि॑ ॥ क्रतु᳚म् । द॒धि॒क्रा इति॑ दधि - क्राः । अन्विति॑ । स॒न्तवी᳚त्व॒दिति॑ सं - तवी᳚त्वत् । प॒थाम् । अङ्काꣳ॑सि । अन्विति॑ । आ॒पनी॑फण॒दित्या᳚ - पनी॑फणत् ॥ उ॒त । स्म॒ । अ॒स्य॒ । द्रव॑तः । तु॒र॒ण्य॒तः । प॒र्णम् । न । वेः । अन्विति॑ । वा॒ति॒ । प्र॒ग॒द्‌र्धिन॒ इति॑ प्र - ग॒द्‌र्धिनः॑ ॥ श्ये॒नस्य॑ । इ॒व॒ । ध्रज॑तः । अ॒ङ्क॒सम् । परीति॑ । द॒धि॒क्राव्.ण्ण॒ इति॑ दधि-क्राव्.ण्णः॑ । स॒ह । ऊ॒र्जा । तरि॑त्रतः ॥ एति॑ । मा॒ । वाज॑स्य । प्र॒स॒व इति॑ प्र - स॒वः । ज॒ग॒म्या॒त् । एति॑ । द्यावा॑पृथि॒वी इति॒ द्यावा᳚ - पृ॒थि॒वी । वि॒श्वश॑भूं॒ इति॑ वि॒श्व - श॒भूं॒ ॥ एति॑ । मा॒ । ग॒न्ता॒म् । पि॒तरा᳚ ।  \newline


\textbf{Krama Paata} \newline

अमी॑वा॒ इत्यमी॑वाः ॥ ए॒ष स्यः । स्य वा॒जी । वा॒जी क्षि॑प॒णिम् । क्षि॒प॒णिम् तु॑रण्यति । तु॒र॒ण्य॒ति॒ ग्री॒वाया᳚म् । ग्री॒वाया᳚म् ब॒द्धः । ब॒द्धो अ॑पिक॒क्षे । अ॒पि॒क॒क्ष आ॒सनि॑ । अ॒पि॒क॒क्ष इत्य॑पि - क॒क्षे । आ॒सनीत्या॒सनि॑ ॥ क्रतु॑म् दधि॒क्राः । द॒धि॒क्रा अनु॑ । द॒धि॒क्रा इति॑ दधि - क्राः । अनु॑ स॒न्तवी᳚त्वत् । स॒न्तवी᳚त्वत् प॒थाम् । स॒न्तवी᳚त्व॒दिति॑ सम् - तवी᳚त्वत् । प॒थामङ्काꣳ॑सि । अङ्काꣳ॒॒स्यनु॑ । अन्वा॒पनी॑फणत् । आ॒पनी॑फण॒दित्या᳚ - पनी॑फणत् ॥ उ॒त स्म॑ । स्मा॒स्य॒ । अ॒स्य॒ द्रव॑तः । द्रव॑तस्तुरण्य॒तः । तु॒र॒ण्य॒तः प॒र्णम् । प॒र्णम् न । न वेः । वेरनु॑ । अनु॑ वाति । वा॒ति॒ प्र॒ग॒र्द्धिनः॑ । प्र॒ग॒र्द्धिन॒ इति॑ प्र - ग॒र्द्धिनः॑ ॥ श्ये॒नस्ये॑व । इ॒व॒ ध्रज॑तः । ध्रज॑तो अङ्क॒सम् । अ॒ङ्क॒सम् परि॑ । परि॑ दधि॒क्राव्.ण्णः॑ । द॒धि॒क्राव्.ण्णः॑ स॒ह । द॒धि॒क्राव्.ण्ण॒ इति॑ दधि - क्राव्.ण्णः॑ । स॒होर्जा । ऊ॒र्जा तरि॑त्रतः । तरि॑त्रत॒ इति॒ तरि॑त्रतः ॥ आ मा᳚ । मा॒ वाज॑स्य । वाज॑स्य प्रस॒वः । प्र॒स॒वो ज॑गम्यात् । प्र॒स॒व इति॑ प्र - स॒वः । ज॒ग॒म्या॒दा । आ द्यावा॑पृथि॒वी । द्यावा॑पृथि॒वी वि॒श्वश॑म्भू । द्यावा॑पृथि॒वी इति॒ द्यावा᳚ - पृ॒थि॒वी । वि॒श्वश॑म्भू॒ इति॑ वि॒श्व - श॒म्भू॒ ॥ आ मा᳚ । मा॒ ग॒न्ता॒म् । ग॒न्ता॒म् पि॒तरा᳚ । पि॒तरा॑ मा॒तरा᳚ \newline

\textbf{Jatai Paata} \newline

1. अमी॑वा॒ इत्यमी॑वाः । \newline
2. ए॒ष स्य स्य ए॒ष ए॒ष स्यः । \newline
3. स्य वा॒जी वा॒जी स्य स्य वा॒जी । \newline
4. वा॒जी क्षि॑प॒णिम् क्षि॑प॒णिं ॅवा॒जी वा॒जी क्षि॑प॒णिम् । \newline
5. क्षि॒प॒णिम् तु॑रण्यति तुरण्यति क्षिप॒णिम् क्षि॑प॒णिम् तु॑रण्यति । \newline
6. तु॒र॒ण्य॒ति॒ ग्री॒वाया᳚म् ग्री॒वाया᳚म् तुरण्यति तुरण्यति ग्री॒वाया᳚म् । \newline
7. ग्री॒वाया᳚म् ब॒द्धो ब॒द्धो ग्री॒वाया᳚म् ग्री॒वाया᳚म् ब॒द्धः । \newline
8. ब॒द्धो अ॑पिक॒क्षे अ॑पिक॒क्षे ब॒द्धो ब॒द्धो अ॑पिक॒क्षे । \newline
9. अ॒पि॒क॒क्ष आ॒सन्या॒ सन्य॑पिक॒क्षे अ॑पिक॒क्ष आ॒सनि॑ । \newline
10. अ॒पि॒क॒क्ष इत्य॑पि - क॒क्षे । \newline
11. आ॒सनीत्या॒सनि॑ । \newline
12. क्रतु॑म् दधि॒क्रा द॑धि॒क्राः क्रतु॒म् क्रतु॑म् दधि॒क्राः । \newline
13. द॒धि॒क्रा अन्वनु॑ दधि॒क्रा द॑धि॒क्रा अनु॑ । \newline
14. द॒धि॒क्रा इति॑ दधि - क्राः । \newline
15. अनु॑ स॒न्तवी᳚त्वथ् स॒न्तवी᳚त्व॒ दन्वनु॑ स॒न्तवी᳚त्वत् । \newline
16. स॒न्तवी᳚त्वत् प॒थाम् प॒थाꣳ स॒न्तवी᳚त्वथ् स॒न्तवी᳚त्वत् प॒थाम् । \newline
17. स॒न्तवी᳚त्व॒दिति॑ सं - तवी᳚त्वत् । \newline
18. प॒था मङ्काꣳ॒॒ स्यङ्काꣳ॑सि प॒थाम् प॒था मङ्काꣳ॑सि । \newline
19. अङ्काꣳ॒॒ स्यन्वन्वङ्काꣳ॒॒ स्यङ्काꣳ॒॒ स्यनु॑ । \newline
20. अन्वा॒पनी॑फण दा॒पनी॑फण॒ दन्वन्वा॒ पनी॑फणत् । \newline
21. आ॒पनी॑फण॒दित्या᳚ - पनी॑फणत् । \newline
22. उ॒त स्म॑ स्मो॒तोत स्म॑ । \newline
23. स्मा॒स्या॒स्य॒ स्म॒ स्मा॒स्य॒ । \newline
24. अ॒स्य॒ द्रव॑तो॒ द्रव॑तो अस्यास्य॒ द्रव॑तः । \newline
25. द्रव॑त स्तुरण्य॒त स्तु॑रण्य॒तो द्रव॑तो॒ द्रव॑त स्तुरण्य॒तः । \newline
26. तु॒र॒ण्य॒तः प॒र्णम् प॒र्णम् तु॑रण्य॒त स्तु॑रण्य॒तः प॒र्णम् । \newline
27. प॒र्णन्न न प॒र्णम् प॒र्णन्न । \newline
28. न वेर् वेर् न न वेः । \newline
29. वेरन्वनु॒ वेर् वेरनु॑ । \newline
30. अनु॑ वाति वा॒त्यन्वनु॑ वाति । \newline
31. वा॒ति॒ प्र॒ग॒र्द्धिनः॑ प्रग॒र्द्धिनो॑ वाति वाति प्रग॒र्द्धिनः॑ । \newline
32. प्र॒ग॒र्द्धिन॒ इति॑ प्र - ग॒र्द्धिनः॑ । \newline
33. श्ये॒नस्ये॑ वे व श्ये॒नस्य॑ श्ये॒नस्ये॑ व । \newline
34. इ॒व॒ ध्रज॑तो॒ ध्रज॑त इवे व॒ ध्रज॑तः । \newline
35. ध्रज॑तो अङ्क॒स म॑ङ्क॒सम् ध्रज॑तो॒ ध्रज॑तो अङ्क॒सम् । \newline
36. अ॒ङ्क॒सम् परि॒ पर्य॑ङ्क॒स म॑ङ्क॒सम् परि॑ । \newline
37. परि॑ दधि॒क्राव्.ण्णो॑ दधि॒क्राव्.ण्णः॒ परि॒ परि॑ दधि॒क्राव्.ण्णः॑ । \newline
38. द॒धि॒क्राव्.ण्णः॑ स॒ह स॒ह द॑धि॒क्राव्.ण्णो॑ दधि॒क्राव्.ण्णः॑ स॒ह । \newline
39. द॒धि॒क्राव्.ण्ण॒ इति॑ दधि - क्राव्.ण्णः॑ । \newline
40. स॒होर्जोर्जा स॒ह स॒होर्जा । \newline
41. ऊ॒र्जा तरि॑त्रत॒ स्तरि॑त्रत ऊ॒र्जोर्जा तरि॑त्रतः । \newline
42. तरि॑त्रत॒ इति॒ तरि॑त्रतः । \newline
43. आ मा॒ मा ऽऽमा᳚ । \newline
44. मा॒ वाज॑स्य॒ वाज॑स्य मा मा॒ वाज॑स्य । \newline
45. वाज॑स्य प्रस॒वः प्र॑स॒वो वाज॑स्य॒ वाज॑स्य प्रस॒वः । \newline
46. प्र॒स॒वो ज॑गम्याज् जगम्यात् प्रस॒वः प्र॑स॒वो ज॑गम्यात् । \newline
47. प्र॒स॒व इति॑ प्र - स॒वः । \newline
48. ज॒ग॒म्या॒दा ज॑गम्याज् जगम्या॒दा । \newline
49. आ द्यावा॑पृथि॒वी द्यावा॑पृथि॒वी आ द्यावा॑पृथि॒वी । \newline
50. द्यावा॑पृथि॒वी वि॒श्वशं॑भू वि॒श्वशं॑भू॒ द्यावा॑पृथि॒वी द्यावा॑पृथि॒वी वि॒श्वशं॑भू । \newline
51. द्यावा॑पृथि॒वी इति॒ द्यावा᳚ - पृ॒थि॒वी । \newline
52. वि॒श्वशं॑भू॒ इति॑ वि॒श्व - शं॒भू॒ । \newline
53. आ मा॒ मा ऽऽमा᳚ । \newline
54. मा॒ ग॒न्ता॒म् ग॒न्ता॒म् मा॒ मा॒ ग॒न्ता॒म् । \newline
55. ग॒न्ता॒म् पि॒तरा॑ पि॒तरा॑ गन्ताम् गन्ताम् पि॒तरा᳚ । \newline
56. पि॒तरा॑ मा॒तरा॑ मा॒तरा॑ पि॒तरा॑ पि॒तरा॑ मा॒तरा᳚ । \newline

\textbf{Ghana Paata } \newline

1. अमी॑वा॒ इत्यमी॑वाः । \newline
2. ए॒ष स्य स्य ए॒ष ए॒ष स्य वा॒जी वा॒जी स्य ए॒ष ए॒ष स्य वा॒जी । \newline
3. स्य वा॒जी वा॒जी स्य स्य वा॒जी क्षि॑प॒णिम् क्षि॑प॒णिं ॅवा॒जी स्य स्य वा॒जी क्षि॑प॒णिम् । \newline
4. वा॒जी क्षि॑प॒णिम् क्षि॑प॒णिं ॅवा॒जी वा॒जी क्षि॑प॒णिम् तु॑रण्यति तुरण्यति क्षिप॒णिं ॅवा॒जी वा॒जी क्षि॑प॒णिम् तु॑रण्यति । \newline
5. क्षि॒प॒णिम् तु॑रण्यति तुरण्यति क्षिप॒णिम् क्षि॑प॒णिम् तु॑रण्यति ग्री॒वाया᳚म् ग्री॒वाया᳚म् तुरण्यति क्षिप॒णिम् क्षि॑प॒णिम् तु॑रण्यति ग्री॒वाया᳚म् । \newline
6. तु॒र॒ण्य॒ति॒ ग्री॒वाया᳚म् ग्री॒वाया᳚म् तुरण्यति तुरण्यति ग्री॒वाया᳚म् ब॒द्धो ब॒द्धो ग्री॒वाया᳚म् तुरण्यति तुरण्यति ग्री॒वाया᳚म् ब॒द्धः । \newline
7. ग्री॒वाया᳚म् ब॒द्धो ब॒द्धो ग्री॒वाया᳚म् ग्री॒वाया᳚म् ब॒द्धो अ॑पिक॒क्षे अ॑पिक॒क्षे ब॒द्धो ग्री॒वाया᳚म् ग्री॒वाया᳚म् ब॒द्धो अ॑पिक॒क्षे । \newline
8. ब॒द्धो अ॑पिक॒क्षे अ॑पिक॒क्षे ब॒द्धो ब॒द्धो अ॑पिक॒क्ष आ॒स न्या॒सन्य॑पिक॒क्षे ब॒द्धो ब॒द्धो अ॑पिक॒क्ष आ॒सनि॑ । \newline
9. अ॒पि॒क॒क्ष आ॒सन्या॒स न्य॑पिक॒क्षे अ॑पिक॒क्ष आ॒सनि॑ । \newline
10. अ॒पि॒क॒क्ष इत्य॑पि - क॒क्षे । \newline
11. आ॒सनीत्या॒सनि॑ । \newline
12. क्रतु॑म् दधि॒क्रा द॑धि॒क्राः क्रतु॒म् क्रतु॑म् दधि॒क्रा अन्वनु॑ दधि॒क्राः क्रतु॒म् क्रतु॑म् दधि॒क्रा अनु॑ । \newline
13. द॒धि॒क्रा अन्वनु॑ दधि॒क्रा द॑धि॒क्रा अनु॑ स॒न्तवी᳚त्वथ् स॒न्तवी᳚त्व॒दनु॑ दधि॒क्रा द॑धि॒क्रा अनु॑ स॒न्तवी᳚त्वत् । \newline
14. द॒धि॒क्रा इति॑ दधि - क्राः । \newline
15. अनु॑ स॒न्तवी᳚त्वथ् स॒न्तवी᳚त्व॒दन्वनु॑ स॒न्तवी᳚त्वत् प॒थाम् प॒थाꣳ स॒न्तवी᳚त्व॒दन्वनु॑ स॒न्तवी᳚त्वत् प॒थाम् । \newline
16. स॒न्तवी᳚त्वत् प॒थाम् प॒थाꣳ स॒न्तवी᳚त्वथ् स॒न्तवी᳚त्वत् प॒था मङ्का॒(ग्ग्॒) स्यङ्का(ग्म्॑)सि प॒थाꣳ स॒न्तवी᳚त्वथ् स॒न्तवी᳚त्वत् प॒था मङ्का(ग्म्॑)सि । \newline
17. स॒न्तवी᳚त्व॒दिति॑ सं - तवी᳚त्वत् । \newline
18. प॒था मङ्का॒(ग्ग्॒) स्यङ्का(ग्म्॑)सि प॒थाम् प॒था मङ्का॒(ग्ग्॒) स्यन्वन्वङ्का(ग्म्॑)सि प॒थाम् प॒था मङ्का॒(ग्ग्॒) स्यनु॑ । \newline
19. अङ्का॒(ग्ग्॒) स्य न्व न्वङ्का॒(ग्ग्॒) स्यङ्का॒(ग्ग्॒) स्यन्वा॒पनी॑फण दा॒पनी॑फण॒ दन्वङ्का॒(ग्ग्॒) स्यङ्का॒(ग्ग्॒)स्यन्वा॒पनी॑फणत् । \newline
20. अन्वा॒पनी॑फण दा॒पनी॑फण॒ दन्वन् वा॒पनी॑फणत् । \newline
21. आ॒पनी॑फण॒दित्या᳚ - पनी॑फणत् । \newline
22. उ॒त स्म॑ स्मो॒तोत स्मा᳚स्यास्य स्मो॒तोत स्मा᳚स्य । \newline
23. स्मा॒स्या॒स्य॒ स्म॒ स्मा॒स्य॒ द्रव॑तो॒ द्रव॑तो अस्य स्म स्मास्य॒ द्रव॑तः । \newline
24. अ॒स्य॒ द्रव॑तो॒ द्रव॑तो अस्यास्य॒ द्रव॑त स्तुरण्य॒त स्तु॑रण्य॒तो द्रव॑तो अस्यास्य॒ द्रव॑त स्तुरण्य॒तः । \newline
25. द्रव॑त स्तुरण्य॒त स्तु॑रण्य॒तो द्रव॑तो॒ द्रव॑त स्तुरण्य॒तः प॒र्णम् प॒र्णम् तु॑रण्य॒तो द्रव॑तो॒ द्रव॑त स्तुरण्य॒तः प॒र्णम् । \newline
26. तु॒र॒ण्य॒तः प॒र्णम् प॒र्णम् तु॑रण्य॒त स्तु॑रण्य॒तः प॒र्णन्न न प॒र्णम् तु॑रण्य॒त स्तु॑रण्य॒तः प॒र्णन्न । \newline
27. प॒र्णन्न न प॒र्णम् प॒र्णन्न वेर् वेर् न प॒र्णम् प॒र्णन्न वेः । \newline
28. न वेर् वेर् न न वे रन्वनु॒ वेर् न न वे रनु॑ । \newline
29. वे रन्वनु॒ वेर् वेरनु॑ वाति वा॒त्यनु॒ वेर् वेरनु॑ वाति । \newline
30. अनु॑ वाति वा॒त्यन्वनु॑ वाति प्रग॒र्द्धिनः॑ प्रग॒र्द्धिनो॑ वा॒त्यन्वनु॑ वाति प्रग॒र्द्धिनः॑ । \newline
31. वा॒ति॒ प्र॒ग॒र्द्धिनः॑ प्रग॒र्द्धिनो॑ वाति वाति प्रग॒र्द्धिनः॑ । \newline
32. प्र॒ग॒र्द्धिन॒ इति॑ प्र - ग॒र्द्धिनः॑ । \newline
33. श्ये॒नस्ये॑ वे व श्ये॒नस्य॑ श्ये॒नस्ये॑ व॒ ध्रज॑तो॒ ध्रज॑त इव श्ये॒नस्य॑ श्ये॒नस्ये॑ व॒ ध्रज॑तः । \newline
34. इ॒व॒ ध्रज॑तो॒ ध्रज॑त इवे व॒ ध्रज॑तो अङ्क॒स म॑ङ्क॒सम् ध्रज॑त इवे व॒ ध्रज॑तो अङ्क॒सम् । \newline
35. ध्रज॑तो अङ्क॒स म॑ङ्क॒सम् ध्रज॑तो॒ ध्रज॑तो अङ्क॒सम् परि॒ पर्य॑ङ्क॒सम् ध्रज॑तो॒ ध्रज॑तो अङ्क॒सम् परि॑ । \newline
36. अ॒ङ्क॒सम् परि॒ पर्य॑ङ्क॒स म॑ङ्क॒सम् परि॑ दधि॒क्राव्.ण्णो॑ दधि॒क्राव्.ण्णः॒ पर्य॑ङ्क॒स म॑ङ्क॒सम् परि॑ दधि॒क्राव्.ण्णः॑ । \newline
37. परि॑ दधि॒क्राव्.ण्णो॑ दधि॒क्राव्.ण्णः॒ परि॒ परि॑ दधि॒क्राव्.ण्णः॑ स॒ह स॒ह द॑धि॒क्राव्.ण्णः॒ परि॒ परि॑ दधि॒क्राव्.ण्णः॑ स॒ह । \newline
38. द॒धि॒क्राव्.ण्णः॑ स॒ह स॒ह द॑धि॒क्राव्.ण्णो॑ दधि॒क्राव्ण्णः॑ स॒होर्जोर्जा स॒ह द॑धि॒क्राव्.ण्णो॑ दधि॒क्राव्.ण्णः॑ स॒होर्जा । \newline
39. द॒धि॒क्राव्.ण्ण॒ इति॑ दधि - क्राव्.ण्णः॑ । \newline
40. स॒होर्जोर्जा स॒ह स॒होर्जा तरि॑त्रत॒ स्तरि॑त्रत ऊ॒र्जा स॒ह स॒होर्जा तरि॑त्रतः । \newline
41. ऊ॒र्जा तरि॑त्रत॒ स्तरि॑त्रत ऊ॒र्जोर्जा तरि॑त्रतः । \newline
42. तरि॑त्रत॒ इति॒ तरि॑त्रतः । \newline
43. आ मा॒ मा ऽऽमा॒ वाज॑स्य॒ वाज॑स्य॒ मा ऽऽमा॒ वाज॑स्य । \newline
44. मा॒ वाज॑स्य॒ वाज॑स्य मा मा॒ वाज॑स्य प्रस॒वः प्र॑स॒वो वाज॑स्य मा मा॒ वाज॑स्य प्रस॒वः । \newline
45. वाज॑स्य प्रस॒वः प्र॑स॒वो वाज॑स्य॒ वाज॑स्य प्रस॒वो ज॑गम्याज् जगम्यात् प्रस॒वो वाज॑स्य॒ वाज॑स्य प्रस॒वो ज॑गम्यात् । \newline
46. प्र॒स॒वो ज॑गम्याज् जगम्यात् प्रस॒वः प्र॑स॒वो ज॑गम्या॒दा ज॑गम्यात् प्रस॒वः प्र॑स॒वो ज॑गम्या॒दा । \newline
47. प्र॒स॒व इति॑ प्र - स॒वः । \newline
48. ज॒ग॒म्या॒दा ज॑गम्याज् जगम्या॒दा द्यावा॑पृथि॒वी द्यावा॑पृथि॒वी आ ज॑गम्याज् जगम्या॒दा द्यावा॑पृथि॒वी । \newline
49. आ द्यावा॑पृथि॒वी द्यावा॑पृथि॒वी आ द्यावा॑पृथि॒वी वि॒श्वशं॑भू वि॒श्वशं॑भू॒ द्यावा॑पृथि॒वी आ द्यावा॑पृथि॒वी वि॒श्वशं॑भू । \newline
50. द्यावा॑पृथि॒वी वि॒श्वशं॑भू वि॒श्वशं॑भू॒ द्यावा॑पृथि॒वी द्यावा॑पृथि॒वी वि॒श्वशं॑भू । \newline
51. द्यावा॑पृथि॒वी इति॒ द्यावा᳚ - पृ॒थि॒वी । \newline
52. वि॒श्वशं॑भू॒ इति॑ वि॒श्व - शं॒भू॒ । \newline
53. आ मा॒ मा ऽऽमा॑ गन्ताम् गन्ता॒म् मा ऽऽमा॑ गन्ताम् । \newline
54. मा॒ ग॒न्ता॒म् ग॒न्ता॒म् मा॒ मा॒ ग॒न्ता॒म् पि॒तरा॑ पि॒तरा॑ गन्ताम् मा मा गन्ताम् पि॒तरा᳚ । \newline
55. ग॒न्ता॒म् पि॒तरा॑ पि॒तरा॑ गन्ताम् गन्ताम् पि॒तरा॑ मा॒तरा॑ मा॒तरा॑ पि॒तरा॑ गन्ताम् गन्ताम् पि॒तरा॑ मा॒तरा᳚ । \newline
56. पि॒तरा॑ मा॒तरा॑ मा॒तरा॑ पि॒तरा॑ पि॒तरा॑ मा॒तरा॑ च च मा॒तरा॑ पि॒तरा॑ पि॒तरा॑ मा॒तरा॑ च । \newline
\pagebreak
\markright{ TS 1.7.8.4  \hfill https://www.vedavms.in \hfill}
\addcontentsline{toc}{section}{ TS 1.7.8.4 }
\section*{ TS 1.7.8.4 }

\textbf{TS 1.7.8.4 } \newline
\textbf{Samhita Paata} \newline

मा॒तरा॒ चाऽऽ मा॒ सोमो॑ अमृत॒त्वाय॑ गम्यात् ॥ वाजि॑नो वाजजितो॒ वाजꣳ॑ सरि॒ष्यन्तो॒ वाजं॑ जे॒ष्यन्तो॒ बृह॒स्पते᳚र् भा॒गमव॑ जिघ्रत॒ वाजि॑नो वाजजितो॒ वाजꣳ॑ ससृ॒वाꣳसो॒ वाजं॑ जिगि॒वाꣳसो॒ बृह॒स्पते᳚र् भा॒गे नि मृ॑ढ्वमि॒यं ॅवः॒ सा स॒त्या स॒न्धाऽभू॒द्यामिन्द्रे॑ण स॒मध॑द्ध्व॒-मजी॑जिपत वनस्पतय॒ इन्द्रं॒ ॅवाजं॒ ॅवि मु॑च्यद्ध्वं ॥ \newline

\textbf{Pada Paata} \newline

मा॒तरा᳚ । च॒ । एति॑ । मा॒ । सोमः॑ । अ॒मृ॒त॒त्वायेत्य॑मृत - त्वाय॑ । ग॒म्या॒त् ॥ वाजि॑नः । वा॒ज॒जि॒त॒ इति॑ वाज - जि॒तः॒ । वाज᳚म् । स॒रि॒ष्यन्तः॑ । वाज᳚म् । जे॒ष्यन्तः॑ । बृह॒स्पतेः᳚ । भा॒गम् । अवेति॑ । जि॒घ्र॒त॒ । वाजि॑नः । वा॒ज॒जि॒त॒ इति॑ वाज - जि॒तः॒ । वाज᳚म् । स॒सृ॒वाꣳसः॑ । वाज᳚म् । जि॒गि॒वाꣳसः॑ । बृह॒स्पतेः᳚ । भा॒गे । नीति॑ । मृ॒ढ्व॒म् । इ॒यम् । वः॒ । सा । स॒त्या । स॒धेंति॑ सं - धा । अ॒भू॒त् । याम् । इन्द्रे॑ण । स॒मध॑द्ध्व॒मिति॑ सं - अध॑द्ध्वम् । अजी॑जिपत । व॒न॒स्प॒त॒यः॒ । इन्द्र᳚म् । वाज᳚म् । वीति॑ । मु॒च्य॒द्ध्व॒म् ॥  \newline


\textbf{Krama Paata} \newline

मा॒तरा॑ च । चा । आ मा᳚ । मा॒ सोमः॑ । सोमो॑ अमृत॒त्वाय॑ । अ॒मृ॒त॒त्वाय॑ गम्यात् । अ॒मृ॒त॒त्वायेत्य॑मृत - त्वाय॑ । ग॒म्या॒दिति॑ गम्यात् ॥ वाजि॑नो वाजजितः । वा॒ज॒जि॒तो॒ वाज᳚म् । वा॒ज॒जि॒त॒ इति॑ वाज - जि॒तः॒ । वाजꣳ॑ सरि॒ष्यन्तः॑ । स॒रि॒ष्यन्तो॒ वाज᳚म् । वाज॑म् जे॒ष्यन्तः॑ । जे॒ष्यन्तो॒ बृह॒स्पतेः᳚ । बृह॒स्पते᳚र् भा॒गम् । भा॒गमव॑ । अव॑ जिघ्रत । जि॒घ्र॒त॒ वाजि॑नः । वाजि॑नो वाजजितः । वा॒ज॒जि॒तो॒ वाज᳚म् । वा॒ज॒जि॒त॒ इति॑ वाज - जि॒तः॒ । वाजꣳ॑ ससृ॒वाꣳसः॑ । स॒सृ॒वाꣳसो॒ वाज᳚म् । वाज॑म् जिगि॒वाꣳसः॑ । जि॒गि॒वाꣳसो॒ बृह॒स्पतेः᳚ । बृह॒स्पते᳚र् भा॒गे । भा॒गे नि । नि मृ॑ढ्वम् । मृ॒ढ्व॒मि॒यम् । इ॒यं ॅवः॑ । वः॒ सा । सा स॒त्या । स॒त्या स॒न्धा । स॒न्धाऽभू᳚त् । स॒न्धेति॑ सम् - धा । अ॒भू॒द् याम् । यामिन्द्रे॑ण । इन्द्रे॑ण स॒मध॑द्ध्वम् । स॒मध॑द्ध्व॒मजी॑जिपत । स॒मध॑द्ध्व॒मिति॑ सम् - अद॑द्ध्वम् । अजी॑जिपत वनस्पतयः । व॒न॒स्प॒त॒य॒ इन्द्र᳚म् । इन्द्रं॒ ॅवाज᳚म् । वाजं॒ ॅवि । वि मु॑च्यद्ध्वम् । मु॒च्य॒द्ध्व॒मिति॑ मुच्यद्ध्वम् । \newline

\textbf{Jatai Paata} \newline

1. मा॒तरा॑ च च मा॒तरा॑ मा॒तरा॑ च । \newline
2. चा च॒ चा । \newline
3. आ मा॒ मा ऽऽमा᳚ । \newline
4. मा॒ सोमः॒ सोमो॑ मा मा॒ सोमः॑ । \newline
5. सोमो॑ अमृत॒त्वाया॑ मृत॒त्वाय॒ सोमः॒ सोमो॑ अमृत॒त्वाय॑ । \newline
6. अ॒मृ॒त॒त्वाय॑ गम्याद् गम्या दमृत॒त्वाया॑ मृत॒त्वाय॑ गम्यात् । \newline
7. अ॒मृ॒त॒त्वायेत्य॑मृत - त्वाय॑ । \newline
8. ग॒म्या॒दिति॑ गम्यात् । \newline
9. वाजि॑नो वाजजितो वाजजितो॒ वाजि॑नो॒ वाजि॑नो वाजजितः । \newline
10. वा॒ज॒जि॒तो॒ वाजं॒ ॅवाजं॑ ॅवाजजितो वाजजितो॒ वाज᳚म् । \newline
11. वा॒ज॒जि॒त॒ इति॑ वाज - जि॒तः॒ । \newline
12. वाजꣳ॑ सरि॒ष्यन्तः॑ सरि॒ष्यन्तो॒ वाजं॒ ॅवाजꣳ॑ सरि॒ष्यन्तः॑ । \newline
13. स॒रि॒ष्यन्तो॒ वाजं॒ ॅवाजꣳ॑ सरि॒ष्यन्तः॑ सरि॒ष्यन्तो॒ वाज᳚म् । \newline
14. वाज॑म् जे॒ष्यन्तो॑ जे॒ष्यन्तो॒ वाजं॒ ॅवाज॑म् जे॒ष्यन्तः॑ । \newline
15. जे॒ष्यन्तो॒ बृह॒स्पते॒र् बृह॒स्पते᳚र् जे॒ष्यन्तो॑ जे॒ष्यन्तो॒ बृह॒स्पतेः᳚ । \newline
16. बृह॒स्पते᳚र् भा॒गम् भा॒गम् बृह॒स्पते॒र् बृह॒स्पते᳚र् भा॒गम् । \newline
17. भा॒ग मवाव॑ भा॒गम् भा॒ग मव॑ । \newline
18. अव॑ जिघ्रत जिघ्र॒तावाव॑ जिघ्रत । \newline
19. जि॒घ्र॒त॒ वाजि॑नो॒ वाजि॑नो जिघ्रत जिघ्रत॒ वाजि॑नः । \newline
20. वाजि॑नो वाजजितो वाजजितो॒ वाजि॑नो॒ वाजि॑नो वाजजितः । \newline
21. वा॒ज॒जि॒तो॒ वाजं॒ ॅवाजं॑ ॅवाजजितो वाजजितो॒ वाज᳚म् । \newline
22. वा॒ज॒जि॒त॒ इति॑ वाज - जि॒तः॒ । \newline
23. वाजꣳ॑ ससृ॒ वाꣳसः॑ ससृ॒वाꣳसो॒ वाजं॒  ॅवाजꣳ॑ ससृ॒वाꣳसः॑ । \newline
24. स॒सृ॒वाꣳसो॒ वाजं॒ ॅवाजꣳ॑ ससृ॒वाꣳसः॑ ससृ॒वाꣳसो॒वाज᳚म् । \newline
25. वाज॑म् जिगि॒वाꣳसो॑ जिगि॒वाꣳसो॒ वाजं॒ ॅवाज॑म् जिगि॒वाꣳसः॑ । \newline
26. जि॒गि॒वाꣳसो॒ बृह॒स्पते॒र् बृह॒स्पते᳚र् जिगि॒वाꣳसो॑ जिगि॒वाꣳसो॒ बृह॒स्पतेः᳚ । \newline
27. बृह॒स्पते᳚र् भा॒गे भा॒गे बृह॒स्पते॒र् बृह॒स्पते᳚र् भा॒गे । \newline
28. भा॒गे नि नि भा॒गे भा॒गे नि । \newline
29. नि मृ॑ढ्वम् मृढ्व॒न् नि नि मृ॑ढ्वम् । \newline
30. मृ॒ढ्व॒ मि॒य मि॒यम् मृ॑ढ्वम् मृढ्व मि॒यम् । \newline
31. इ॒यं ॅवो॑ व इ॒य मि॒यं ॅवः॑ । \newline
32. वः॒ सा सा वो॑ वः॒ सा । \newline
33. सा स॒त्या स॒त्या सा सा स॒त्या । \newline
34. स॒त्या स॒न्धा स॒न्धा स॒त्या स॒त्या स॒न्धा । \newline
35. स॒न्धा ऽभू॑दभूथ् स॒न्धा स॒न्धा ऽभू᳚त् । \newline
36. स॒न्धेति॑ सं - धा । \newline
37. अ॒भू॒द् यां ॅया म॑भू दभू॒द् याम् । \newline
38. या मिन्द्रे॒णे न्द्रे॑ण॒ यां ॅया मिन्द्रे॑ण । \newline
39. इन्द्रे॑ण स॒मध॑द्ध्वꣳ स॒मध॑द्ध्व॒ मिन्द्रे॒णे न्द्रे॑ण स॒मध॑द्ध्वम् । \newline
40. स॒मध॑द्ध्व॒ मजी॑जिप॒ता जी॑जिपत स॒मध॑द्ध्वꣳ स॒मध॑द्ध्व॒ मजी॑जिपत । \newline
41. स॒मध॑द्ध्व॒मिति॑ सं - अध॑द्ध्वम् । \newline
42. अजी॑जिपत वनस्पतयो वनस्पत॒यो ऽजी॑जिप॒ता जी॑जिपत वनस्पतयः । \newline
43. व॒न॒स्प॒त॒य॒ इन्द्र॒ मिन्द्रं॑ ॅवनस्पतयो वनस्पतय॒ इन्द्र᳚म् । \newline
44. इन्द्रं॒ ॅवाजं॒ ॅवाज॒ मिन्द्र॒ मिन्द्रं॒ ॅवाज᳚म् । \newline
45. वाजं॒ ॅवि वि वाजं॒ ॅवाजं॒ ॅवि । \newline
46. वि मु॑च्यद्ध्वम् मुच्यद्ध्वं॒ ॅवि वि मु॑च्यद्ध्वम् । \newline
47. मु॒च्य॒द्ध्व॒मिति॑ मुच्यद्ध्वम् । \newline

\textbf{Ghana Paata } \newline

1. मा॒तरा॑ च च मा॒तरा॑ मा॒तरा॒ चा च॑ मा॒तरा॑ मा॒तरा॒ चा । \newline
2. चा च॒ चा मा॒ मा ऽऽच॒ चा मा᳚ । \newline
3. आ मा॒ मा ऽऽमा॒ सोमः॒ सोमो॒ मा ऽऽमा॒ सोमः॑ । \newline
4. मा॒ सोमः॒ सोमो॑ मा मा॒ सोमो॑ अमृत॒त्वाया॑ मृत॒त्वाय॒ सोमो॑ मा मा॒ सोमो॑ अमृत॒त्वाय॑ । \newline
5. सोमो॑ अमृत॒त्वाया॑ मृत॒त्वाय॒ सोमः॒ सोमो॑ अमृत॒त्वाय॑ गम्याद् गम्या दमृत॒त्वाय॒ सोमः॒ सोमो॑ अमृत॒त्वाय॑ गम्यात् । \newline
6. अ॒मृ॒त॒त्वाय॑ गम्याद् गम्या दमृत॒त्वाया॑ मृत॒त्वाय॑ गम्यात् । \newline
7. अ॒मृ॒त॒त्वायेत्य॑मृत - त्वाय॑ । \newline
8. ग॒म्या॒दिति॑ गम्यात् । \newline
9. वाजि॑नो वाजजितो वाजजितो॒ वाजि॑नो॒ वाजि॑नो वाजजितो॒ वाजं॒ ॅवाजं॑ ॅवाजजितो॒ वाजि॑नो॒ वाजि॑नो वाजजितो॒ वाज᳚म् । \newline
10. वा॒ज॒जि॒तो॒ वाजं॒ ॅवाजं॑ ॅवाजजितो वाजजितो॒ वाज(ग्म्॑) सरि॒ष्यन्तः॑ सरि॒ष्यन्तो॒ वाजं॑ ॅवाजजितो वाजजितो॒ वाज(ग्म्॑) सरि॒ष्यन्तः॑ । \newline
11. वा॒ज॒जि॒त॒ इति॑ वाज - जि॒तः॒ । \newline
12. वाज(ग्म्॑) सरि॒ष्यन्तः॑ सरि॒ष्यन्तो॒ वाजं॒ ॅवाज(ग्म्॑) सरि॒ष्यन्तो॒ वाजं॒ ॅवाज(ग्म्॑) सरि॒ष्यन्तो॒ वाजं॒ ॅवाज(ग्म्॑) सरि॒ष्यन्तो॒ वाज᳚म् । \newline
13. स॒रि॒ष्यन्तो॒ वाजं॒ ॅवाज(ग्म्॑) सरि॒ष्यन्तः॑ सरि॒ष्यन्तो॒ वाज॑म् जे॒ष्यन्तो॑ जे॒ष्यन्तो॒ वाज(ग्म्॑) सरि॒ष्यन्तः॑ सरि॒ष्यन्तो॒ वाज॑म् जे॒ष्यन्तः॑ । \newline
14. वाज॑म् जे॒ष्यन्तो॑ जे॒ष्यन्तो॒ वाजं॒ ॅवाज॑म् जे॒ष्यन्तो॒ बृह॒स्पते॒र् बृह॒स्पते᳚र् जे॒ष्यन्तो॒ वाजं॒ ॅवाज॑म् जे॒ष्यन्तो॒ बृह॒स्पतेः᳚ । \newline
15. जे॒ष्यन्तो॒ बृह॒स्पते॒र् बृह॒स्पते᳚र् जे॒ष्यन्तो॑ जे॒ष्यन्तो॒ बृह॒स्पते᳚र् भा॒गम् भा॒गम् बृह॒स्पते᳚र् जे॒ष्यन्तो॑ जे॒ष्यन्तो॒ बृह॒स्पते᳚र् भा॒गम् । \newline
16. बृह॒स्पते᳚र् भा॒गम् भा॒गम् बृह॒स्पते॒र् बृह॒स्पते᳚र् भा॒ग मवाव॑ भा॒गम् बृह॒स्पते॒र् बृह॒स्पते᳚र् भा॒ग मव॑ । \newline
17. भा॒ग मवाव॑ भा॒गम् भा॒ग मव॑ जिघ्रत जिघ्र॒ताव॑ भा॒गम् भा॒ग मव॑ जिघ्रत । \newline
18. अव॑ जिघ्रत जिघ्र॒ तावाव॑ जिघ्रत॒ वाजि॑नो॒ वाजि॑नो जिघ्र॒ तावाव॑ जिघ्रत॒ वाजि॑नः । \newline
19. जि॒घ्र॒त॒ वाजि॑नो॒ वाजि॑नो जिघ्रत जिघ्रत॒ वाजि॑नो वाजजितो वाजजितो॒ वाजि॑नो जिघ्रत जिघ्रत॒ वाजि॑नो वाजजितः । \newline
20. वाजि॑नो वाजजितो वाजजितो॒ वाजि॑नो॒ वाजि॑नो वाजजितो॒ वाजं॒ ॅवाजं॑ ॅवाजजितो॒ वाजि॑नो॒ वाजि॑नो वाजजितो॒ वाज᳚म् । \newline
21. वा॒ज॒जि॒तो॒ वाजं॒ ॅवाजं॑ ॅवाजजितो वाजजितो॒ वाज(ग्म्॑) ससृ॒वाꣳसः॑ ससृ॒वाꣳसो॒ वाजं॑ ॅवाजजितो वाजजितो॒ वाज(ग्म्॑) ससृ॒वाꣳसः॑ । \newline
22. वा॒ज॒जि॒त॒ इति॑ वाज - जि॒तः॒ । \newline
23. वाज(ग्म्॑) ससृ॒वाꣳसः॑ ससृ॒वाꣳसो॒ वाजं॒ ॅवाज(ग्म्॑) ससृ॒वाꣳसो॒ वाजं॒ ॅवाज(ग्म्॑) ससृ॒वाꣳसो॒ वाजं॒ ॅवाज(ग्म्॑) ससृ॒वाꣳसो॒ वाज᳚म् । \newline
24. स॒सृ॒वाꣳसो॒ वाजं॒ ॅवाज(ग्म्॑) ससृ॒वाꣳसः॑ ससृ॒वाꣳसो॒ वाज॑म् जिगि॒वाꣳसो॑ जिगि॒वाꣳसो॒ वाज(ग्म्॑) ससृ॒वाꣳसः॑ ससृ॒वाꣳसो॒ वाज॑म् जिगि॒वाꣳसः॑ । \newline
25. वाज॑म् जिगि॒वाꣳसो॑ जिगि॒वाꣳसो॒ वाजं॒ ॅवाज॑म् जिगि॒वाꣳसो॒ बृह॒स्पते॒र् बृह॒स्पते᳚र् जिगि॒वाꣳसो॒ वाजं॒ ॅवाज॑म् जिगि॒वाꣳसो॒ बृह॒स्पतेः᳚ । \newline
26. जि॒गि॒वाꣳसो॒ बृह॒स्पते॒र् बृह॒स्पते᳚र् जिगि॒वाꣳसो॑ जिगि॒वाꣳसो॒ बृह॒स्पते᳚र् भा॒गे भा॒गे बृह॒स्पते᳚र् जिगि॒वाꣳसो॑ जिगि॒वाꣳसो॒ बृह॒स्पते᳚र् भा॒गे । \newline
27. बृह॒स्पते᳚र् भा॒गे भा॒गे बृह॒स्पते॒र् बृह॒स्पते᳚र् भा॒गे नि नि भा॒गे बृह॒स्पते॒र् बृह॒स्पते᳚र् भा॒गे नि । \newline
28. भा॒गे नि नि भा॒गे भा॒गे नि मृ॑ढ्वम् मृढ्व॒न्नि भा॒गे भा॒गे नि मृ॑ढ्वम् । \newline
29. नि मृ॑ढ्वम् मृढ्व॒न्नि नि मृ॑ढ्व मि॒य मि॒यम् मृ॑ढ्व॒न्नि नि मृ॑ढ्व मि॒यम् । \newline
30. मृ॒ढ्व॒ मि॒य मि॒यम् मृ॑ढ्वम् मृढ्व मि॒यं ॅवो॑ व इ॒यम् मृ॑ढ्वम् मृढ्व मि॒यं ॅवः॑ । \newline
31. इ॒यं ॅवो॑ व इ॒य मि॒यं ॅवः॒ सा सा व॑ इ॒य मि॒यं ॅवः॒ सा । \newline
32. वः॒ सा सा वो॑ वः॒ सा स॒त्या स॒त्या सा वो॑ वः॒ सा स॒त्या । \newline
33. सा स॒त्या स॒त्या सा सा स॒त्या स॒न्धा स॒न्धा स॒त्या सा सा स॒त्या स॒न्धा । \newline
34. स॒त्या स॒न्धा स॒न्धा स॒त्या स॒त्या स॒न्धा ऽभू॑दभूथ् स॒न्धा स॒त्या स॒त्या स॒न्धा ऽभू᳚त् । \newline
35. स॒न्धा ऽभू॑दभूथ् स॒न्धा स॒न्धा ऽभू॒द् यां ॅया म॑भूथ् स॒न्धा स॒न्धा ऽभू॒द् याम् । \newline
36. स॒न्धेति॑ सं - धा । \newline
37. अ॒भू॒द् यां ॅया म॑भू दभू॒द् या मिन्द्रे॒णे न्द्रे॑ण॒ या म॑भू दभू॒द् या मिन्द्रे॑ण । \newline
38. या मिन्द्रे॒णे न्द्रे॑ण॒ यां ॅया मिन्द्रे॑ण स॒मध॑द्ध्वꣳ स॒मध॑द्ध्व॒ मिन्द्रे॑ण॒ यां ॅया मिन्द्रे॑ण स॒मध॑द्ध्वम् । \newline
39. इन्द्रे॑ण स॒मध॑द्ध्वꣳ स॒मध॑द्ध्व॒ मिन्द्रे॒णे न्द्रे॑ण स॒मध॑द्ध्व॒ मजी॑जिप॒ता जी॑जिपत स॒मध॑द्ध्व॒ मिन्द्रे॒णे न्द्रे॑ण स॒मध॑द्ध्व॒ मजी॑जिपत । \newline
40. स॒मध॑द्ध्व॒ मजी॑जिप॒ता जी॑जिपत स॒मध॑द्ध्वꣳ स॒मध॑द्ध्व॒ मजी॑जिपत वनस्पतयो वनस्पत॒यो ऽजी॑जिपत स॒मध॑द्ध्वꣳ स॒मध॑द्ध्व॒ मजी॑जिपत वनस्पतयः । \newline
41. स॒मध॑द्ध्व॒मिति॑ सं - अध॑द्ध्वम् । \newline
42. अजी॑जिपत वनस्पतयो वनस्पत॒यो ऽजी॑जिप॒ता जी॑जिपत वनस्पतय॒ इन्द्र॒ मिन्द्रं॑ ॅवनस्पत॒यो ऽजी॑जिप॒ता जी॑जिपत वनस्पतय॒ इन्द्र᳚म् । \newline
43. व॒न॒स्प॒त॒य॒ इन्द्र॒ मिन्द्रं॑ ॅवनस्पतयो वनस्पतय॒ इन्द्रं॒ ॅवाजं॒ ॅवाज॒ मिन्द्रं॑ ॅवनस्पतयो वनस्पतय॒ इन्द्रं॒ ॅवाज᳚म् । \newline
44. इन्द्रं॒ ॅवाजं॒ ॅवाज॒ मिन्द्र॒ मिन्द्रं॒ ॅवाजं॒ ॅवि वि वाज॒ मिन्द्र॒ मिन्द्रं॒ ॅवाजं॒ ॅवि । \newline
45. वाजं॒ ॅवि वि वाजं॒ ॅवाजं॒ ॅवि मु॑च्यद्ध्वम् मुच्यद्ध्वं॒ ॅवि वाजं॒ ॅवाजं॒ ॅवि मु॑च्यद्ध्वम् । \newline
46. वि मु॑च्यद्ध्वम् मुच्यद्ध्वं॒ ॅवि वि मु॑च्यद्ध्वम् । \newline
47. मु॒च्य॒द्ध्व॒मिति॑ मुच्यद्ध्वम् । \newline
\pagebreak
\markright{ TS 1.7.9.1  \hfill https://www.vedavms.in \hfill}
\addcontentsline{toc}{section}{ TS 1.7.9.1 }
\section*{ TS 1.7.9.1 }

\textbf{TS 1.7.9.1 } \newline
\textbf{Samhita Paata} \newline

क्ष॒त्रस्योलꣳ॑मसि क्ष॒त्रस्य॒ योनि॑रसि॒ जाय॒ एहि॒ सुवो॒ रोहा॑व॒ रोहा॑व॒ हि सुव॑र॒हं ना॑वु॒भयोः॒ सुवो॑ रोक्ष्यामि॒ वाज॑श्च प्रस॒वश्चा॑पि॒जश्च॒ क्रतु॑श्च॒ सुव॑श्च मू॒र्द्धा च॒ व्यश्ञ्नि॑यश्चाऽऽन्त्याय॒न श्चान्त्य॑श्च भौव॒नश्च॒ भुव॑न॒श्चाधि॑पतिश्च । आयु॑र् य॒ज्ञेन॑ कल्पतां प्रा॒णो य॒ज्ञेन॑ कल्पतामपा॒नो - [ ] \newline

\textbf{Pada Paata} \newline

क्ष॒त्रस्य॑ । उल्ब᳚म् । अ॒सि॒ । क्ष॒त्रस्य॑ । योनिः॑ । अ॒सि॒ । जाये᳚ । एति॑ । इ॒हि॒ । सुवः॑ । रोहा॑व । रोहा॑व । हि । सुवः॑ । अ॒हम् । नौ॒ । उ॒भयोः᳚ । सुवः॑ । रो॒क्ष्या॒मि॒ । वाजः॑ । च॒ । प्र॒स॒व इति॑ प्र - स॒वः । च॒ । अ॒पि॒ज इत्य॑पि- जः । च॒ । क्रतुः॑ । च॒ । सुवः॑ । च॒ । मू॒द्‌र्धा । च॒ । व्यश्नि॑य॒ इति॑ वि - अश्नि॑यः । च॒ । आ॒न्त्या॒य॒नः । च॒ । अन्त्यः॑ । च॒ । भौ॒व॒नः । च॒ । भुव॑नः । च॒ । अधि॑पति॒रित्यधि॑ - प॒तिः॒ । च॒ ॥ आयुः॑ । य॒ज्ञेन॑ । क॒ल्प॒ता॒म् । प्रा॒ण इति॑ प्र - अ॒नः । य॒ज्ञेन॑ । क॒ल्प॒ता॒म् । अ॒पा॒न इत्य॑प - अ॒नः ।  \newline


\textbf{Krama Paata} \newline

क्ष॒त्रस्योल्ब᳚म् । उल्ब॑मसि । अ॒सि॒ क्ष॒त्रस्य॑ । क्ष॒त्रस्य॒ योनिः॑ । योनि॑रसि । अ॒सि॒ जाये᳚ । जाय॒ आ । एहि॑ । इ॒हि॒ सुवः॑ । सुवो॒ रोहा॑व । रोहा॑व॒ रोहा॑व । रोहा॑व॒ हि । हि सुवः॑ । सुव॑र॒हम् । अ॒हम् नौ᳚ । ना॒वु॒भयोः᳚ । उ॒भयोः॒ सुवः॑ । सुवो॑ रोक्ष्यामि । रो॒क्ष्या॒मि॒ वाजः॑ । वाज॑श्च । च॒ प्र॒स॒वः । प्र॒स॒वश्च॑ । प्र॒स॒व इति॑ प्र - स॒वः । चा॒पि॒जः । अ॒पि॒जश्च॑ । अ॒पि॒ज इत्य॑पि - जः । च॒ क्रतुः॑ । क्रतु॑श्च । च॒ सुवः॑ । सुव॑श्च । च॒ मू॒र्द्धा । मू॒र्द्धा च॑ । च॒ व्यश्ञि॑यः । व्यश्ञि॑यश्च । व्यश्ञि॑य॒ इति॑ वि - अश्ञि॑यः । चा॒न्त्या॒य॒नः । आ॒न्त्या॒य॒नश्च॑ । चान्त्यः॑ । अन्त्य॑श्च । च॒ भौ॒व॒नः । भौ॒व॒नश्च॑ । च॒ भुव॑नः । भुव॑नश्च । चाधि॑पतिः । अधि॑पतिश्च । अधि॑पति॒रित्यधि॑ - प॒तिः॒ । चेति॑ च ॥ आयु॑र्. य॒ज्ञेन॑ । य॒ज्ञेन॑ कल्पताम् । क॒ल्प॒ता॒म् प्रा॒णः । प्रा॒णो य॒ज्ञेन॑ । प्रा॒ण इति॑ प्र - अ॒नः । य॒ज्ञेन॑ कल्पताम् । क॒ल्प॒ता॒म॒पा॒नः । अ॒पा॒नो य॒ज्ञेन॑ । अ॒पा॒न इत्य॑प - अ॒नः \newline

\textbf{Jatai Paata} \newline

1. क्ष॒त्रस्योल्ब॒ मुल्ब॑म् क्ष॒त्रस्य॑ क्ष॒त्रस्योल्ब᳚म् । \newline
2. उल्ब॑ मस्य॒ स्युल्ब॒ मुल्ब॑ मसि । \newline
3. अ॒सि॒ क्ष॒त्रस्य॑ क्ष॒त्र स्या᳚स्यसि क्ष॒त्रस्य॑ । \newline
4. क्ष॒त्रस्य॒ योनि॒र् योनिः॑ क्ष॒त्रस्य॑ क्ष॒त्रस्य॒ योनिः॑ । \newline
5. योनि॑ रस्यसि॒ योनि॒र् योनि॑ रसि । \newline
6. अ॒सि॒ जाये॒ जाये᳚ ऽस्यसि॒ जाये᳚ । \newline
7. जाय॒ आ जाये॒ जाय॒ आ । \newline
8. एही॒ह्येहि॑ । \newline
9. इ॒हि॒ सुवः॒ सुव॑ रिहीहि॒ सुवः॑ । \newline
10. सुवो॒ रोहा॑व॒ रोहा॑व॒ सुवः॒ सुवो॒ रोहा॑व । \newline
11. रोहा॑व॒ रोहा॑व । \newline
12. रोहा॑व॒ हि हि रोहा॑व॒ रोहा॑व॒ हि । \newline
13. हि सुवः॒ सुव॒र्॒. हि हि सुवः॑ । \newline
14. सुव॑ र॒ह म॒हꣳ सुवः॒ सुव॑ र॒हम् । \newline
15. अ॒हन्नौ॑ ना व॒ह म॒हन्नौ᳚ । \newline
16. ना॒ वु॒भयो॑ रु॒भयो᳚र् नौ ना वु॒भयोः᳚ । \newline
17. उ॒भयोः॒ सुवः॒ सुव॑ रु॒भयो॑ रु॒भयोः॒ सुवः॑ । \newline
18. सुवो॑ रोक्ष्यामि रोक्ष्यामि॒ सुवः॒ सुवो॑ रोक्ष्यामि । \newline
19. रो॒क्ष्या॒मि॒ वाजो॒ वाजो॑ रोक्ष्यामि रोक्ष्यामि॒ वाजः॑ । \newline
20. वाज॑श्च च॒ वाजो॒ वाज॑श्च । \newline
21. च॒ प्र॒स॒वः प्र॑स॒वश्च॑ च प्रस॒वः । \newline
22. प्र॒स॒वश्च॑ च प्रस॒वः प्र॑स॒वश्च॑ । \newline
23. प्र॒स॒व इति॑ प्र - स॒वः । \newline
24. चा॒पि॒जो अ॑पि॒जश्च॑ चापि॒जः । \newline
25. अ॒पि॒जश्च॑ चापि॒जो अ॑पि॒जश्च॑ । \newline
26. अ॒पि॒ज इत्य॑पि - जः । \newline
27. च॒ क्रतुः॒ क्रतु॑श्च च॒ क्रतुः॑ । \newline
28. क्रतु॑श्च च॒ क्रतुः॒ क्रतु॑श्च । \newline
29. च॒ सुवः॒ सुव॑श्च च॒ सुवः॑ । \newline
30. सुव॑श्च च॒ सुवः॒ सुव॑श्च । \newline
31. च॒ मू॒र्द्धा मू॒र्द्धा च॑ च मू॒र्द्धा । \newline
32. मू॒र्द्धा च॑ च मू॒र्द्धा मू॒र्द्धा च॑ । \newline
33. च॒ व्यश्ञि॑यो॒ व्यश्ञि॑यश्च च॒ व्यश्ञि॑यः । \newline
34. व्यश्ञि॑यश्च च॒ व्यश्ञि॑यो॒ व्यश्ञि॑यश्च । \newline
35. व्यश्ञि॑य॒ इति॑ वि - अश्ञि॑यः । \newline
36. चा॒न्त्या॒य॒न आ᳚न्त्याय॒नश्च॑ चान्त्याय॒नः । \newline
37. आ॒न्त्या॒य॒नश्च॑ चान्त्याय॒न आ᳚न्त्याय॒नश्च॑ । \newline
38. चान्त्यो॒ अन्त्य॑श्च॒ चान्त्यः॑ । \newline
39. अन्त्य॑श्च॒ चान्त्यो॒ अन्त्य॑श्च । \newline
40. च॒ भौ॒व॒नो भौ॑व॒नश्च॑ च भौव॒नः । \newline
41. भौ॒व॒नश्च॑ च भौव॒नो भौ॑व॒नश्च॑ । \newline
42. च॒ भुव॑नो॒ भुव॑नश्च च॒ भुव॑नः । \newline
43. भुव॑नश्च च॒ भुव॑नो॒ भुव॑नश्च । \newline
44. चाधि॑पति॒ रधि॑पतिश्च॒ चाधि॑पतिः । \newline
45. अधि॑पतिश्च॒ चाधि॑पति॒ रधि॑पतिश्च । \newline
46. अधि॑पति॒रित्यधि॑ - प॒तिः॒ । \newline
47. चेति॑ च । \newline
48. आयु॑र् य॒ज्ञेन॑ य॒ज्ञेनायु॒ रायु॑र् य॒ज्ञेन॑ । \newline
49. य॒ज्ञेन॑ कल्पताम् कल्पतां ॅय॒ज्ञेन॑ य॒ज्ञेन॑ कल्पताम् । \newline
50. क॒ल्प॒ता॒म् प्रा॒णः प्रा॒णः क॑ल्पताम् कल्पताम् प्रा॒णः । \newline
51. प्रा॒णो य॒ज्ञेन॑ य॒ज्ञेन॑ प्रा॒णः प्रा॒णो य॒ज्ञेन॑ । \newline
52. प्रा॒ण इति॑ प्र - अ॒नः । \newline
53. य॒ज्ञेन॑ कल्पताम् कल्पतां ॅय॒ज्ञेन॑ य॒ज्ञेन॑ कल्पताम् । \newline
54. क॒ल्प॒ता॒ म॒पा॒नो अ॑पा॒नः क॑ल्पताम् कल्पता मपा॒नः । \newline
55. अ॒पा॒नो य॒ज्ञेन॑ य॒ज्ञे ना॑पा॒नो अ॑पा॒नो य॒ज्ञेन॑ । \newline
56. अ॒पा॒न इत्य॑प - अ॒नः । \newline

\textbf{Ghana Paata } \newline

1. क्ष॒त्रस्योल्ब॒ मुल्ब॑म् क्ष॒त्रस्य॑ क्ष॒त्रस्योल्ब॑ मस्य॒स्युल्ब॑म् क्ष॒त्रस्य॑ क्ष॒त्रस्योल्ब॑ मसि । \newline
2. उल्ब॑ मस्य॒ स्युल्ब॒ मुल्ब॑ मसि क्ष॒त्रस्य॑ क्ष॒त्रस्या॒ स्युल्ब॒ मुल्ब॑ मसि क्ष॒त्रस्य॑ । \newline
3. अ॒सि॒ क्ष॒त्रस्य॑ क्ष॒त्रस्या᳚स्यसि क्ष॒त्रस्य॒ योनि॒र् योनिः॑ क्ष॒त्रस्या᳚स्यसि क्ष॒त्रस्य॒ योनिः॑ । \newline
4. क्ष॒त्रस्य॒ योनि॒र् योनिः॑ क्ष॒त्रस्य॑ क्ष॒त्रस्य॒ योनि॑ रस्यसि॒ योनिः॑ क्ष॒त्रस्य॑ क्ष॒त्रस्य॒ योनि॑ रसि । \newline
5. योनि॑ रस्यसि॒ योनि॒र् योनि॑ रसि॒ जाये॒ जाये॑ ऽसि॒ योनि॒र् योनि॑ रसि॒ जाये᳚ । \newline
6. अ॒सि॒ जाये॒ जाये᳚ ऽस्यसि॒ जाय॒ आ जाये᳚ ऽस्यसि॒ जाय॒ आ । \newline
7. जाय॒ आ जाये॒ जाय॒ एही॒ह्या जाये॒ जाय॒ एहि॑ । \newline
8. एही॒ह्येहि॒ सुवः॒ सुव॑ रि॒ह्येहि॒ सुवः॑ । \newline
9. इ॒हि॒ सुवः॒ सुव॑ रिहीहि॒ सुवो॒ रोहा॑व॒ रोहा॑व॒ सुव॑ रिहीहि॒ सुवो॒ रोहा॑व । \newline
10. सुवो॒ रोहा॑व॒ रोहा॑व॒ सुवः॒ सुवो॒ रोहा॑व । \newline
11. रोहा॑व॒ रोहा॑व । \newline
12. रोहा॑व॒ हि हि रोहा॑व॒ रोहा॑व॒ हि सुवः॒ सुव॒र्॒. हि रोहा॑व॒ रोहा॑व॒ हि सुवः॑ । \newline
13. हि सुवः॒ सुव॒र्॒. हि हि सुव॑ र॒ह म॒हꣳ सुव॒र्॒. हि हि सुव॑ र॒हम् । \newline
14. सुव॑ र॒ह म॒हꣳ सुवः॒ सुव॑ र॒हम् नौ॑ ना व॒हꣳ सुवः॒ सुव॑ र॒हम् नौ᳚ । \newline
15. अ॒हन्नौ॑ ना व॒ह म॒हम् ना॑ वु॒भयो॑ रु॒भयो᳚र् ना व॒ह म॒हम् ना॑ वु॒भयोः᳚ । \newline
16. ना॒ वु॒भयो॑ रु॒भयो᳚र् नौ ना वु॒भयोः॒ सुवः॒ सुव॑ रु॒भयो᳚र् नौ ना वु॒भयोः॒ सुवः॑ । \newline
17. उ॒भयोः॒ सुवः॒ सुव॑ रु॒भयो॑ रु॒भयोः॒ सुवो॑ रोक्ष्यामि रोक्ष्यामि॒ सुव॑ रु॒भयो॑ रु॒भयोः॒ सुवो॑ रोक्ष्यामि । \newline
18. सुवो॑ रोक्ष्यामि रोक्ष्यामि॒ सुवः॒ सुवो॑ रोक्ष्यामि॒ वाजो॒ वाजो॑ रोक्ष्यामि॒ सुवः॒ सुवो॑ रोक्ष्यामि॒ वाजः॑ । \newline
19. रो॒क्ष्या॒मि॒ वाजो॒ वाजो॑ रोक्ष्यामि रोक्ष्यामि॒ वाज॑श्च च॒ वाजो॑ रोक्ष्यामि रोक्ष्यामि॒ वाज॑श्च । \newline
20. वाज॑श्च च॒ वाजो॒ वाज॑श्च प्रस॒वः प्र॑स॒वश्च॒ वाजो॒ वाज॑श्च प्रस॒वः । \newline
21. च॒ प्र॒स॒वः प्र॑स॒वश्च॑ च प्रस॒वश्च॑ च प्रस॒वश्च॑ च प्रस॒वश्च॑ । \newline
22. प्र॒स॒वश्च॑ च प्रस॒वः प्र॑स॒व श्चा॑पि॒जो अ॑पि॒जश्च॑ प्रस॒वः प्र॑स॒व श्चा॑पि॒जः । \newline
23. प्र॒स॒व इति॑ प्र - स॒वः । \newline
24. चा॒पि॒जो अ॑पि॒जश्च॑ चापि॒जश्च॑ चापि॒जश्च॑ चापि॒जश्च॑ । \newline
25. अ॒पि॒जश्च॑ चापि॒जो अ॑पि॒जश्च॒ क्रतुः॒ क्रतु॑ श्चापि॒जो अ॑पि॒जश्च॒ क्रतुः॑ । \newline
26. अ॒पि॒ज इत्य॑पि - जः । \newline
27. च॒ क्रतुः॒ क्रतु॑श्च च॒ क्रतु॑श्च च॒ क्रतु॑श्च च॒ क्रतु॑श्च । \newline
28. क्रतु॑श्च च॒ क्रतुः॒ क्रतु॑श्च॒ सुवः॒ सुव॑श्च॒ क्रतुः॒ क्रतु॑श्च॒ सुवः॑ । \newline
29. च॒ सुवः॒ सुव॑श्च च॒ सुव॑श्च च॒ सुव॑श्च च॒ सुव॑श्च । \newline
30. सुव॑श्च च॒ सुवः॒ सुव॑श्च मू॒र्द्धा मू॒र्द्धा च॒ सुवः॒ सुव॑श्च मू॒र्द्धा । \newline
31. च॒ मू॒र्द्धा मू॒र्द्धा च॑ च मू॒र्द्धा च॑ च मू॒र्द्धा च॑ च मू॒र्द्धा च॑ । \newline
32. मू॒र्द्धा च॑ च मू॒र्द्धा मू॒र्द्धा च॒ व्यश्ञि॑यो॒ व्यश्ञि॑यश्च मू॒र्द्धा मू॒र्द्धा च॒ व्यश्ञि॑यः । \newline
33. च॒ व्यश्ञि॑यो॒ व्यश्ञि॑यश्च च॒ व्यश्ञि॑यश्च च॒ व्यश्ञि॑यश्च च॒ व्यश्ञि॑यश्च । \newline
34. व्यश्ञि॑यश्च च॒ व्यश्ञि॑यो॒ व्यश्ञि॑य श्चान्त्याय॒न आ᳚न्त्याय॒नश्च॒ व्यश्ञि॑यो॒ व्यश्ञि॑य श्चान्त्याय॒नः । \newline
35. व्यश्ञि॑य॒ इति॑ वि - अश्ञि॑यः । \newline
36. चा॒न्त्या॒य॒न आ᳚न्त्याय॒नश्च॑ चान्त्याय॒नश्च॑ चान्त्याय॒नश्च॑ चान्त्याय॒नश्च॑ । \newline
37. आ॒न्त्या॒य॒नश्च॑ चान्त्याय॒न आ᳚न्त्याय॒न श्चान्त्यो॒ अन्त्य॑ श्चान्त्याय॒न आ᳚न्त्याय॒न श्चान्त्यः॑ । \newline
38. चान्त्यो॒ अन्त्य॑श्च॒ चान्त्य॑श्च॒ चान्त्य॑श्च॒ चान्त्य॑श्च । \newline
39. अन्त्य॑श्च॒ चान्त्यो॒ अन्त्य॑श्च भौव॒नो भौ॑व॒न श्चान्त्यो॒ अन्त्य॑श्च भौव॒नः । \newline
40. च॒ भौ॒व॒नो भौ॑व॒नश्च॑ च भौव॒नश्च॑ च भौव॒नश्च॑ च भौव॒नश्च॑ । \newline
41. भौ॒व॒नश्च॑ च भौव॒नो भौ॑व॒नश्च॒ भुव॑नो॒ भुव॑नश्च भौव॒नो भौ॑व॒नश्च॒ भुव॑नः । \newline
42. च॒ भुव॑नो॒ भुव॑नश्च च॒ भुव॑नश्च च॒ भुव॑नश्च च॒ भुव॑नश्च । \newline
43. भुव॑नश्च च॒ भुव॑नो॒ भुव॑न॒ श्चाधि॑पति॒ रधि॑पतिश्च॒ भुव॑नो॒ भुव॑न॒ श्चाधि॑पतिः । \newline
44. चाधि॑पति॒ रधि॑पतिश्च॒ चाधि॑पतिश्च॒ चाधि॑पतिश्च॒ चाधि॑पतिश्च । \newline
45. अधि॑पतिश्च॒ चाधि॑पति॒ रधि॑पतिश्च । \newline
46. अधि॑पति॒रित्यधि॑ - प॒तिः॒ । \newline
47. चेति॑ च । \newline
48. आयु॑र् य॒ज्ञेन॑ य॒ज्ञेनायु॒ रायु॑र् य॒ज्ञेन॑ कल्पताम् कल्पतां ॅय॒ज्ञेनायु॒ रायु॑र् य॒ज्ञेन॑ कल्पताम् । \newline
49. य॒ज्ञेन॑ कल्पताम् कल्पतां ॅय॒ज्ञेन॑ य॒ज्ञेन॑ कल्पताम् प्रा॒णः प्रा॒णः क॑ल्पतां ॅय॒ज्ञेन॑ य॒ज्ञेन॑ कल्पताम् प्रा॒णः । \newline
50. क॒ल्प॒ता॒म् प्रा॒णः प्रा॒णः क॑ल्पताम् कल्पताम् प्रा॒णो य॒ज्ञेन॑ य॒ज्ञेन॑ प्रा॒णः क॑ल्पताम् कल्पताम् प्रा॒णो य॒ज्ञेन॑ । \newline
51. प्रा॒णो य॒ज्ञेन॑ य॒ज्ञेन॑ प्रा॒णः प्रा॒णो य॒ज्ञेन॑ कल्पताम् कल्पतां ॅय॒ज्ञेन॑ प्रा॒णः प्रा॒णो य॒ज्ञेन॑ कल्पताम् । \newline
52. प्रा॒ण इति॑ प्र - अ॒नः । \newline
53. य॒ज्ञेन॑ कल्पताम् कल्पतां ॅय॒ज्ञेन॑ य॒ज्ञेन॑ कल्पता मपा॒नो अ॑पा॒नः क॑ल्पतां ॅय॒ज्ञेन॑ य॒ज्ञेन॑ कल्पता मपा॒नः । \newline
54. क॒ल्प॒ता॒ म॒पा॒नो अ॑पा॒नः क॑ल्पताम् कल्पता मपा॒नो य॒ज्ञेन॑ य॒ज्ञेना॑पा॒नः क॑ल्पताम् कल्पता मपा॒नो य॒ज्ञेन॑ । \newline
55. अ॒पा॒नो य॒ज्ञेन॑ य॒ज्ञेना॑पा॒नो अ॑पा॒नो य॒ज्ञेन॑ कल्पताम् कल्पतां ॅय॒ज्ञेना॑पा॒नो अ॑पा॒नो य॒ज्ञेन॑ कल्पताम् । \newline
56. अ॒पा॒न इत्य॑प - अ॒नः । \newline
\pagebreak
\markright{ TS 1.7.9.2  \hfill https://www.vedavms.in \hfill}
\addcontentsline{toc}{section}{ TS 1.7.9.2 }
\section*{ TS 1.7.9.2 }

\textbf{TS 1.7.9.2 } \newline
\textbf{Samhita Paata} \newline

य॒ज्ञेन॑ कल्पतां ॅव्या॒नो य॒ज्ञेन॑ कल्पतां॒ चक्षु॑र् य॒ज्ञेन॑ कल्पताꣳ॒॒ श्रोत्रं॑ ॅय॒ज्ञेन॑ कल्पतां॒ मनो॑ य॒ज्ञेन॑ कल्पतां॒ ॅवाग् य॒ज्ञेन॑ कल्पता-मा॒त्मा य॒ज्ञेन॑ कल्पतां ॅय॒ज्ञो य॒ज्ञेन॑ कल्पताꣳ॒॒ सुव॑र् दे॒वाꣳ अ॑गन्मा॒मृता॑ अभूम प्र॒जाप॑तेः प्र॒जा अ॑भूम॒ सम॒हं प्र॒जया॒ सं मया᳚ प्र॒जा सम॒हꣳ रा॒यस्पोषे॑ण॒ सं मया॑ रा॒यस्पोषोऽन्ना॑य त्वा॒ऽन्नाद्या॑य त्वा॒ वाजा॑य ( ) त्वा वाजजि॒त्यायै᳚ त्वा॒ ऽमृत॑मसि॒ पुष्टि॑रसि प्र॒जन॑नमसि ॥ \newline

\textbf{Pada Paata} \newline

य॒ज्ञेन॑ । क॒ल्प॒ता॒म् । व्या॒न इति॑ वि - अ॒नः । य॒ज्ञेन॑ । क॒ल्प॒ता॒म् । चक्षुः॑ । य॒ज्ञेन॑ । क॒ल्प॒ता॒म् । श्रोत्र᳚म् । य॒ज्ञेन॑ । क॒ल्प॒ता॒म् । मनः॑ । य॒ज्ञेन॑ । क॒ल्प॒ता॒म् । वाक् । य॒ज्ञेन॑ । क॒ल्प॒ता॒म् । आ॒त्मा । य॒ज्ञेन॑ । क॒ल्प॒ता॒म् । य॒ज्ञ्ः । य॒ज्ञेन॑ । क॒ल्प॒ता॒म् । सुवः॑ । दे॒वान् । अ॒ग॒न्म॒ । अ॒मृता᳚ । अ॒भू॒म॒ । प्र॒जाप॑ते॒रिति॑ प्र॒जा-प॒तेः॒ । प्र॒जा इति॑ प्र - जाः । अ॒भू॒म॒ । समिति॑ । अ॒हम् । प्र॒जयेति॑ प्र - जया᳚ । समिति॑ । मया᳚ । प्र॒जेति॑ प्र-जा । समिति॑ । अ॒हम् । रा॒यः । पोषे॑ण । समिति॑ । मया᳚ । रा॒यः । पोषः॑ । अन्ना॑य । त्वा॒ । अ॒न्नाद्या॒येत्य॑न्न - अद्या॑य । त्वा॒ । वाजा॑य ( ) । त्वा॒ । वा॒ज॒जि॒त्याया॒ इति॑ वाज - जि॒त्यायै᳚ । त्वा॒ । अ॒मृत᳚म् । अ॒सि॒ । पुष्टिः॑ । अ॒सि॒ । प्र॒जन॑न॒मिति॑ प्र - जन॑नम् । अ॒सि॒ ॥  \newline


\textbf{Krama Paata} \newline

य॒ज्ञेन॑ कल्पताम् । क॒ल्प॒तां॒ ॅव्या॒नः । व्या॒नो य॒ज्ञेन॑ । व्या॒न इति॑ वि - अ॒नः । य॒ज्ञेन॑ कल्पताम् । क॒ल्प॒ता॒म् चक्षुः॑ । चक्षु॑र्,य॒ज्ञेन॑ । य॒ज्ञेन॑ कल्पताम् । क॒ल्प॒ताꣳ॒॒ श्रोत्र᳚म् । श्रोत्रं॑ ॅय॒ज्ञेन॑ । य॒ज्ञेन॑ कल्पताम् । क॒ल्प॒ता॒म् मनः॑ । मनो॑ य॒ज्ञेन॑ । य॒ज्ञेन॑ कल्पताम् । क॒ल्प॒तां॒ ॅवाक् । वा॒ग् य॒ज्ञेन॑ । य॒ज्ञेन॑ कल्पताम् । क॒ल्प॒ता॒मा॒त्मा । आ॒त्मा य॒ज्ञेन॑ । य॒ज्ञेन॑ कल्पताम् । क॒ल्प॒तां॒ ॅय॒ज्ञ्ः । य॒ज्ञो य॒ज्ञेन॑ । य॒ज्ञेन॑ कल्पताम् । क॒ल्प॒ताꣳ॒॒ सुवः॑ । सुव॑र्,दे॒वान् । दे॒वाꣳ अ॑गन्म । अ॒ग॒न्मा॒मृताः᳚ । अ॒मृता॑ अभूम । अ॒भू॒म॒ प्र॒जाप॑तेः । प्र॒जाप॑तेः प्र॒जाः । प्र॒जाप॑ते॒रिति॑ प्र॒जा - प॒तेः॒ । प्र॒जा अ॑भूम । प्र॒जा इति॑ प्र - जाः । अ॒भू॒म॒ सम् । सम॒हम् । अ॒हम् प्र॒जया᳚ । प्र॒जया॒ सम् । प्र॒जयेति॑ प्र - जया᳚ । सम् मया᳚ । मया᳚ प्र॒जा । प्र॒जा सम् । प्र॒जेति॑ प्र - जा । सम॒हम् । अ॒हꣳ रा॒यः । रा॒यस्पोषे॑ण । पोषे॑ण॒ सम् । सम् मया᳚ । मया॑ रा॒यः । रा॒यस्पोषः॑ । पोषोऽन्ना॑य । अन्ना॑य त्वा । त्वा॒ऽन्नाद्या॑य । अ॒न्नाद्या॑य त्वा । अ॒न्नाद्या॒येत्य॑न्न - अद्या॑य । त्वा॒ वाजा॑य ( ) । वाजा॑य त्वा । त्वा॒ वा॒ज॒जि॒त्यायै᳚ । वा॒ज॒जि॒त्यायै᳚ त्वा । वा॒ज॒जि॒त्याया॒ इति॑ वाज - जि॒त्यायै᳚ । त्वा॒ऽमृत᳚म् । अ॒मृत॑मसि । अ॒सि॒ पुष्टिः॑ । पुष्टि॑रसि । अ॒सि॒ प्र॒जन॑नम् । प्र॒जन॑नमसि । प्र॒जन॑न॒मिति॑ प्र - जन॑नम् । अ॒सीत्य॑सि । \newline

\textbf{Jatai Paata} \newline

1. य॒ज्ञेन॑ कल्पताम् कल्पतां ॅय॒ज्ञेन॑ य॒ज्ञेन॑ कल्पताम् । \newline
2. क॒ल्प॒तां॒ ॅव्या॒नो व्या॒नः क॑ल्पताम् कल्पतां ॅव्या॒नः । \newline
3. व्या॒नो य॒ज्ञेन॑ य॒ज्ञेन॑ व्या॒नो व्या॒नो य॒ज्ञेन॑ । \newline
4. व्या॒न इति॑ वि - अ॒नः । \newline
5. य॒ज्ञेन॑ कल्पताम् कल्पतां ॅय॒ज्ञेन॑ य॒ज्ञेन॑ कल्पताम् । \newline
6. क॒ल्प॒ता॒म् चक्षु॒ श्चक्षुः॑ कल्पताम् कल्पता॒म् चक्षुः॑ । \newline
7. चक्षु॑र् य॒ज्ञेन॑ य॒ज्ञेन॒ चक्षु॒ श्चक्षु॑र् य॒ज्ञेन॑ । \newline
8. य॒ज्ञेन॑ कल्पताम् कल्पतां ॅय॒ज्ञेन॑ य॒ज्ञेन॑ कल्पताम् । \newline
9. क॒ल्प॒ताꣳ॒॒ श्रोत्रꣳ॒॒ श्रोत्र॑म् कल्पताम् कल्पताꣳ॒॒ श्रोत्र᳚म् । \newline
10. श्रोत्रं॑ ॅय॒ज्ञेन॑ य॒ज्ञेन॒ श्रोत्रꣳ॒॒ श्रोत्रं॑ ॅय॒ज्ञेन॑ । \newline
11. य॒ज्ञेन॑ कल्पताम् कल्पतां ॅय॒ज्ञेन॑ य॒ज्ञेन॑ कल्पताम् । \newline
12. क॒ल्प॒ता॒म् मनो॒ मनः॑ कल्पताम् कल्पता॒म् मनः॑ । \newline
13. मनो॑ य॒ज्ञेन॑ य॒ज्ञेन॒ मनो॒ मनो॑ य॒ज्ञेन॑ । \newline
14. य॒ज्ञेन॑ कल्पताम् कल्पतां ॅय॒ज्ञेन॑ य॒ज्ञेन॑ कल्पताम् । \newline
15. क॒ल्प॒तां॒ ॅवाग् वाक् क॑ल्पताम् कल्पतां॒ ॅवाक् । \newline
16. वाग् य॒ज्ञेन॑ य॒ज्ञेन॒ वाग् वाग् य॒ज्ञेन॑ । \newline
17. य॒ज्ञेन॑ कल्पताम् कल्पतां ॅय॒ज्ञेन॑ य॒ज्ञेन॑ कल्पताम् । \newline
18. क॒ल्प॒ता॒ मा॒त्मा ऽऽत्मा क॑ल्पताम् कल्पता मा॒त्मा । \newline
19. आ॒त्मा य॒ज्ञेन॑ य॒ज्ञेना॒त्मा ऽऽत्मा य॒ज्ञेन॑ । \newline
20. य॒ज्ञेन॑ कल्पताम् कल्पतां ॅय॒ज्ञेन॑ य॒ज्ञेन॑ कल्पताम् । \newline
21. क॒ल्प॒तां॒ ॅय॒ज्ञो य॒ज्ञ्ः क॑ल्पताम् कल्पतां ॅय॒ज्ञ्ः । \newline
22. य॒ज्ञो य॒ज्ञेन॑ य॒ज्ञेन॑ य॒ज्ञो य॒ज्ञो य॒ज्ञेन॑ । \newline
23. य॒ज्ञेन॑ कल्पताम् कल्पतां ॅय॒ज्ञेन॑ य॒ज्ञेन॑ कल्पताम् । \newline
24. क॒ल्प॒ताꣳ॒॒ सुवः॒ सुवः॑ कल्पताम् कल्पताꣳ॒॒ सुवः॑ । \newline
25. सुव॑र् दे॒वान् दे॒वान् थ्सुवः॒ सुव॑र् दे॒वान् । \newline
26. दे॒वाꣳ अ॑गन्मागन्म दे॒वान् दे॒वाꣳ अ॑गन्म । \newline
27. अ॒ग॒न्मा॒ मृता॑ अ॒मृता॑ अगन्मा गन्मा॒ मृताः᳚ । \newline
28. अ॒मृता॑ अभूमा भूमा॒ मृता॑ अ॒मृता॑ अभूम । \newline
29. अ॒भू॒म॒ प्र॒जाप॑तेः प्र॒जाप॑ते रभूमा भूम प्र॒जाप॑तेः । \newline
30. प्र॒जाप॑तेः प्र॒जाः प्र॒जाः प्र॒जाप॑तेः प्र॒जाप॑तेः प्र॒जाः । \newline
31. प्र॒जाप॑ते॒रिति॑ प्र॒जा - प॒तेः॒ । \newline
32. प्र॒जा अ॑भूमा भूम प्र॒जाः प्र॒जा अ॑भूम । \newline
33. प्र॒जा इति॑ प्र - जाः । \newline
34. अ॒भू॒म॒ सꣳ स म॑भूमा भूम॒ सम् । \newline
35. स म॒ह म॒हꣳ सꣳ स म॒हम् । \newline
36. अ॒हम् प्र॒जया᳚ प्र॒जया॒ ऽह म॒हम् प्र॒जया᳚ । \newline
37. प्र॒जया॒ सꣳ सम् प्र॒जया᳚ प्र॒जया॒ सम् । \newline
38. प्र॒जयेति॑ प्र - जया᳚ । \newline
39. सम् मया॒ मया॒ सꣳ सम् मया᳚ । \newline
40. मया᳚ प्र॒जा प्र॒जा मया॒ मया᳚ प्र॒जा । \newline
41. प्र॒जा सꣳ सम् प्र॒जा प्र॒जा सम् । \newline
42. प्र॒जेति॑ प्र - जा । \newline
43. स म॒ह म॒हꣳ सꣳ स म॒हम् । \newline
44. अ॒हꣳ रा॒यो रा॒यो॑ ऽह म॒हꣳ रा॒यः । \newline
45. रा॒य स्पोषे॑ण॒ पोषे॑ण रा॒यो रा॒य स्पोषे॑ण । \newline
46. पोषे॑ण॒ सꣳ सम् पोषे॑ण॒ पोषे॑ण॒ सम् । \newline
47. सम् मया॒ मया॒ सꣳ सम् मया᳚ । \newline
48. मया॑ रा॒यो रा॒यो मया॒ मया॑ रा॒यः । \newline
49. रा॒य स्पोषः॒ पोषो॑ रा॒यो रा॒य स्पोषः॑ । \newline
50. पोषो ऽन्ना॒ यान्ना॑य॒ पोषः॒ पोषो ऽन्ना॑य । \newline
51. अन्ना॑य त्वा॒ त्वा ऽन्ना॒ यान्ना॑य त्वा । \newline
52. त्वा॒ ऽन्नाद्या॑ या॒न्नाद्या॑य त्वा त्वा॒ ऽन्नाद्या॑य । \newline
53. अ॒न्नाद्या॑य त्वा त्वा॒ ऽन्नाद्या॑ या॒न्नाद्या॑य त्वा । \newline
54. अ॒न्नाद्या॒येत्य॑न्न - अद्या॑य । \newline
55. त्वा॒ वाजा॑य॒ वाजा॑य त्वा त्वा॒ वाजा॑य । \newline
56. वाजा॑य त्वा त्वा॒ वाजा॑य॒ वाजा॑य त्वा । \newline
57. त्वा॒ वा॒ज॒जि॒त्यायै॑ वाजजि॒त्यायै᳚ त्वा त्वा वाजजि॒त्यायै᳚ । \newline
58. वा॒ज॒जि॒त्यायै᳚ त्वा त्वा वाजजि॒त्यायै॑ वाजजि॒त्यायै᳚ त्वा । \newline
59. वा॒ज॒जि॒त्याया॒ इति॑ वाज - जि॒त्यायै᳚ । \newline
60. त्वा॒ ऽमृत॑ म॒मृत॑म् त्वा त्वा॒ ऽमृत᳚म् । \newline
61. अ॒मृत॑ मस्य स्य॒मृत॑ म॒मृत॑ मसि । \newline
62. अ॒सि॒ पुष्टिः॒ पुष्टि॑ रस्यसि॒ पुष्टिः॑ । \newline
63. पुष्टि॑ रस्यसि॒ पुष्टिः॒ पुष्टि॑ रसि । \newline
64. अ॒सि॒ प्र॒जन॑नम् प्र॒जन॑न मस्यसि प्र॒जन॑नम् । \newline
65. प्र॒जन॑न मस्यसि प्र॒जन॑नम् प्र॒जन॑न मसि । \newline
66. प्र॒जन॑न॒मिति॑ प्र - जन॑नम् । \newline
67. अ॒सीत्य॑सि । \newline

\textbf{Ghana Paata } \newline

1. य॒ज्ञेन॑ कल्पताम् कल्पतां ॅय॒ज्ञेन॑ य॒ज्ञेन॑ कल्पतां ॅव्या॒नो व्या॒नः क॑ल्पतां ॅय॒ज्ञेन॑ य॒ज्ञेन॑ कल्पतां ॅव्या॒नः । \newline
2. क॒ल्प॒तां॒ ॅव्या॒नो व्या॒नः क॑ल्पताम् कल्पतां ॅव्या॒नो य॒ज्ञेन॑ य॒ज्ञेन॑ व्या॒नः क॑ल्पताम् कल्पतां ॅव्या॒नो य॒ज्ञेन॑ । \newline
3. व्या॒नो य॒ज्ञेन॑ य॒ज्ञेन॑ व्या॒नो व्या॒नो य॒ज्ञेन॑ कल्पताम् कल्पतां ॅय॒ज्ञेन॑ व्या॒नो व्या॒नो य॒ज्ञेन॑ कल्पताम् । \newline
4. व्या॒न इति॑ वि - अ॒नः । \newline
5. य॒ज्ञेन॑ कल्पताम् कल्पतां ॅय॒ज्ञेन॑ य॒ज्ञेन॑ कल्पता॒म् चक्षु॒ श्चक्षुः॑ कल्पतां ॅय॒ज्ञेन॑ य॒ज्ञेन॑ कल्पता॒म् चक्षुः॑ । \newline
6. क॒ल्प॒ता॒म् चक्षु॒ श्चक्षुः॑ कल्पताम् कल्पता॒म् चक्षु॑र् य॒ज्ञेन॑ य॒ज्ञेन॒ चक्षुः॑ कल्पताम् कल्पता॒म् चक्षु॑र् य॒ज्ञेन॑ । \newline
7. चक्षु॑र् य॒ज्ञेन॑ य॒ज्ञेन॒ चक्षु॒ श्चक्षु॑र् य॒ज्ञेन॑ कल्पताम् कल्पतां ॅय॒ज्ञेन॒ चक्षु॒ श्चक्षु॑र् य॒ज्ञेन॑ कल्पताम् । \newline
8. य॒ज्ञेन॑ कल्पताम् कल्पतां ॅय॒ज्ञेन॑ य॒ज्ञेन॑ कल्पता॒(ग्ग्॒) श्रोत्र॒(ग्ग्॒) श्रोत्र॑म् कल्पतां ॅय॒ज्ञेन॑ य॒ज्ञेन॑ कल्पता॒(ग्ग्॒) श्रोत्र᳚म् । \newline
9. क॒ल्प॒ता॒(ग्ग्॒) श्रोत्र॒(ग्ग्॒) श्रोत्र॑म् कल्पताम् कल्पता॒(ग्ग्॒) श्रोत्रं॑ ॅय॒ज्ञेन॑ य॒ज्ञेन॒ श्रोत्र॑म् कल्पताम् कल्पता॒(ग्ग्॒) श्रोत्रं॑ ॅय॒ज्ञेन॑ । \newline
10. श्रोत्रं॑ ॅय॒ज्ञेन॑ य॒ज्ञेन॒ श्रोत्र॒(ग्ग्॒) श्रोत्रं॑ ॅय॒ज्ञेन॑ कल्पताम् कल्पतां ॅय॒ज्ञेन॒ श्रोत्र॒(ग्ग्॒) श्रोत्रं॑ ॅय॒ज्ञेन॑ कल्पताम् । \newline
11. य॒ज्ञेन॑ कल्पताम् कल्पतां ॅय॒ज्ञेन॑ य॒ज्ञेन॑ कल्पता॒म् मनो॒ मनः॑ कल्पतां ॅय॒ज्ञेन॑ य॒ज्ञेन॑ कल्पता॒म् मनः॑ । \newline
12. क॒ल्प॒ता॒म् मनो॒ मनः॑ कल्पताम् कल्पता॒म् मनो॑ य॒ज्ञेन॑ य॒ज्ञेन॒ मनः॑ कल्पताम् कल्पता॒म् मनो॑ य॒ज्ञेन॑ । \newline
13. मनो॑ य॒ज्ञेन॑ य॒ज्ञेन॒ मनो॒ मनो॑ य॒ज्ञेन॑ कल्पताम् कल्पतां ॅय॒ज्ञेन॒ मनो॒ मनो॑ य॒ज्ञेन॑ कल्पताम् । \newline
14. य॒ज्ञेन॑ कल्पताम् कल्पतां ॅय॒ज्ञेन॑ य॒ज्ञेन॑ कल्पतां॒ ॅवाग् वाक् क॑ल्पतां ॅय॒ज्ञेन॑ य॒ज्ञेन॑ कल्पतां॒ ॅवाक् । \newline
15. क॒ल्प॒तां॒ ॅवाग् वाक् क॑ल्पताम् कल्पतां॒ ॅवाग् य॒ज्ञेन॑ य॒ज्ञेन॒ वाक् क॑ल्पताम् कल्पतां॒ ॅवाग् य॒ज्ञेन॑ । \newline
16. वाग् य॒ज्ञेन॑ य॒ज्ञेन॒ वाग् वाग् य॒ज्ञेन॑ कल्पताम् कल्पतां ॅय॒ज्ञेन॒ वाग् वाग् य॒ज्ञेन॑ कल्पताम् । \newline
17. य॒ज्ञेन॑ कल्पताम् कल्पतां ॅय॒ज्ञेन॑ य॒ज्ञेन॑ कल्पता मा॒त्मा ऽऽत्मा क॑ल्पतां ॅय॒ज्ञेन॑ य॒ज्ञेन॑ कल्पता मा॒त्मा । \newline
18. क॒ल्प॒ता॒ मा॒त्मा ऽऽत्मा क॑ल्पताम् कल्पता मा॒त्मा य॒ज्ञेन॑ य॒ज्ञेना॒त्मा क॑ल्पताम् कल्पता मा॒त्मा य॒ज्ञेन॑ । \newline
19. आ॒त्मा य॒ज्ञेन॑ य॒ज्ञेना॒त्मा ऽऽत्मा य॒ज्ञेन॑ कल्पताम् कल्पतां ॅय॒ज्ञेना॒त्मा ऽऽत्मा य॒ज्ञेन॑ कल्पताम् । \newline
20. य॒ज्ञेन॑ कल्पताम् कल्पतां ॅय॒ज्ञेन॑ य॒ज्ञेन॑ कल्पतां ॅय॒ज्ञो य॒ज्ञ्ः क॑ल्पतां ॅय॒ज्ञेन॑ य॒ज्ञेन॑ कल्पतां ॅय॒ज्ञ्ः । \newline
21. क॒ल्प॒तां॒ ॅय॒ज्ञो य॒ज्ञ्ः क॑ल्पताम् कल्पतां ॅय॒ज्ञो य॒ज्ञेन॑ य॒ज्ञेन॑ य॒ज्ञ्ः क॑ल्पताम् कल्पतां ॅय॒ज्ञो य॒ज्ञेन॑ । \newline
22. य॒ज्ञो य॒ज्ञेन॑ य॒ज्ञेन॑ य॒ज्ञो य॒ज्ञो य॒ज्ञेन॑ कल्पताम् कल्पतां ॅय॒ज्ञेन॑ य॒ज्ञो य॒ज्ञो य॒ज्ञेन॑ कल्पताम् । \newline
23. य॒ज्ञेन॑ कल्पताम् कल्पतां ॅय॒ज्ञेन॑ य॒ज्ञेन॑ कल्पता॒(ग्म्॒) सुवः॒ सुवः॑ कल्पतां ॅय॒ज्ञेन॑ य॒ज्ञेन॑ कल्पता॒(ग्म्॒) सुवः॑ । \newline
24. क॒ल्प॒ता॒(ग्म्॒) सुवः॒ सुवः॑ कल्पताम् कल्पता॒(ग्म्॒) सुव॑र् दे॒वान् दे॒वान् थ्सुवः॑ कल्पताम् कल्पता॒(ग्म्॒) सुव॑र् दे॒वान् । \newline
25. सुव॑र् दे॒वान् दे॒वान् थ्सुवः॒ सुव॑र् दे॒वाꣳ अ॑गन्मा गन्म दे॒वान् थ्सुवः॒ सुव॑र् दे॒वाꣳ अ॑गन्म । \newline
26. दे॒वाꣳ अ॑गन्मा गन्म दे॒वान् दे॒वाꣳ अ॑गन्मा॒मृता॑ अ॒मृता॑ अगन्म दे॒वान् दे॒वाꣳ अ॑गन्मा॒मृताः᳚ । \newline
27. अ॒ग॒न्मा॒ मृता॑ अ॒मृता॑ अगन्मा गन्मा॒ मृता॑ अभूमा भूमा॒मृता॑ अगन्मा गन्मा॒ मृता॑ अभूम । \newline
28. अ॒मृता॑ अभूमा भूमा॒ मृता॑ अ॒मृता॑ अभूम प्र॒जाप॑तेः प्र॒जाप॑ते रभूमा॒मृता॑ अ॒मृता॑ अभूम प्र॒जाप॑तेः । \newline
29. अ॒भू॒म॒ प्र॒जाप॑तेः प्र॒जाप॑ते रभूमाभूम प्र॒जाप॑तेः प्र॒जाः प्र॒जाः प्र॒जाप॑ते रभूमाभूम प्र॒जाप॑तेः प्र॒जाः । \newline
30. प्र॒जाप॑तेः प्र॒जाः प्र॒जाः प्र॒जाप॑तेः प्र॒जाप॑तेः प्र॒जा अ॑भूमा भूम प्र॒जाः प्र॒जाप॑तेः प्र॒जाप॑तेः प्र॒जा अ॑भूम । \newline
31. प्र॒जाप॑ते॒रिति॑ प्र॒जा - प॒तेः॒ । \newline
32. प्र॒जा अ॑भूमाभूम प्र॒जाः प्र॒जा अ॑भूम॒ सꣳ स म॑भूम प्र॒जाः प्र॒जा अ॑भूम॒ सम् । \newline
33. प्र॒जा इति॑ प्र - जाः । \newline
34. अ॒भू॒म॒ सꣳ स म॑भूमा भूम॒ स म॒ह म॒हꣳ स म॑भूमा भूम॒ स म॒हम् । \newline
35. स म॒ह म॒हꣳ सꣳ स म॒हम् प्र॒जया᳚ प्र॒जया॒ ऽहꣳ सꣳ स म॒हम् प्र॒जया᳚ । \newline
36. अ॒हम् प्र॒जया᳚ प्र॒जया॒ ऽह म॒हम् प्र॒जया॒ सꣳ सम् प्र॒जया॒ ऽह म॒हम् प्र॒जया॒ सम् । \newline
37. प्र॒जया॒ सꣳ सम् प्र॒जया᳚ प्र॒जया॒ सम् मया॒ मया॒ सम् प्र॒जया᳚ प्र॒जया॒ सम् मया᳚ । \newline
38. प्र॒जयेति॑ प्र - जया᳚ । \newline
39. सम् मया॒ मया॒ सꣳ सम् मया᳚ प्र॒जा प्र॒जा मया॒ सꣳ सम् मया᳚ प्र॒जा । \newline
40. मया᳚ प्र॒जा प्र॒जा मया॒ मया᳚ प्र॒जा सꣳ सम् प्र॒जा मया॒ मया᳚ प्र॒जा सम् । \newline
41. प्र॒जा सꣳ सम् प्र॒जा प्र॒जा स म॒ह म॒हꣳ सम् प्र॒जा प्र॒जा स म॒हम् । \newline
42. प्र॒जेति॑ प्र - जा । \newline
43. स म॒ह म॒हꣳ सꣳ स म॒हꣳ रा॒यो रा॒यो॑ ऽहꣳ सꣳ स म॒हꣳ रा॒यः । \newline
44. अ॒हꣳ रा॒यो रा॒यो॑ ऽह म॒हꣳ रा॒य स्पोषे॑ण॒ पोषे॑ण रा॒यो॑ ऽह म॒हꣳ रा॒य स्पोषे॑ण । \newline
45. रा॒य स्पोषे॑ण॒ पोषे॑ण रा॒यो रा॒य स्पोषे॑ण॒ सꣳ सम् पोषे॑ण रा॒यो रा॒य स्पोषे॑ण॒ सम् । \newline
46. पोषे॑ण॒ सꣳ सम् पोषे॑ण॒ पोषे॑ण॒ सम् मया॒ मया॒ सम् पोषे॑ण॒ पोषे॑ण॒ सम् मया᳚ । \newline
47. सम् मया॒ मया॒ सꣳ सम् मया॑ रा॒यो रा॒यो मया॒ सꣳ सम् मया॑ रा॒यः । \newline
48. मया॑ रा॒यो रा॒यो मया॒ मया॑ रा॒य स्पोषः॒ पोषो॑ रा॒यो मया॒ मया॑ रा॒य स्पोषः॑ । \newline
49. रा॒य स्पोषः॒ पोषो॑ रा॒यो रा॒य स्पोषो ऽन्ना॒यान्ना॑य॒ पोषो॑ रा॒यो रा॒य स्पोषो ऽन्ना॑य । \newline
50. पोषो ऽन्ना॒यान्ना॑य॒ पोषः॒ पोषो ऽन्ना॑य त्वा॒ त्वा ऽन्ना॑य॒ पोषः॒ पोषो ऽन्ना॑य त्वा । \newline
51. अन्ना॑य त्वा॒ त्वा ऽन्ना॒या न्ना॑य त्वा॒ ऽन्नाद्या॑या॒ न्नाद्या॑य॒ त्वा ऽन्ना॒या न्ना॑य त्वा॒ ऽन्नाद्या॑य । \newline
52. त्वा॒ ऽन्नाद्या॑या॒ न्नाद्या॑य त्वा त्वा॒ ऽन्नाद्या॑य त्वा त्वा॒ ऽन्नाद्या॑य त्वा त्वा॒ ऽन्नाद्या॑य त्वा । \newline
53. अ॒न्नाद्या॑य त्वा त्वा॒ ऽन्नाद्या॑या॒ न्नाद्या॑य त्वा॒ वाजा॑य॒ वाजा॑य त्वा॒ ऽन्नाद्या॑या॒ न्नाद्या॑य त्वा॒ वाजा॑य । \newline
54. अ॒न्नाद्या॒येत्य॑न्न - अद्या॑य । \newline
55. त्वा॒ वाजा॑य॒ वाजा॑य त्वा त्वा॒ वाजा॑य त्वा त्वा॒ वाजा॑य त्वा त्वा॒ वाजा॑य त्वा । \newline
56. वाजा॑य त्वा त्वा॒ वाजा॑य॒ वाजा॑य त्वा वाजजि॒त्यायै॑ वाजजि॒त्यायै᳚ त्वा॒ वाजा॑य॒ वाजा॑य त्वा वाजजि॒त्यायै᳚ । \newline
57. त्वा॒ वा॒ज॒जि॒त्यायै॑ वाजजि॒त्यायै᳚ त्वा त्वा वाजजि॒त्यायै᳚ त्वा त्वा वाजजि॒त्यायै᳚ त्वा त्वा वाजजि॒त्यायै᳚ त्वा । \newline
58. वा॒ज॒जि॒त्यायै᳚ त्वा त्वा वाजजि॒त्यायै॑ वाजजि॒त्यायै᳚ त्वा॒ ऽमृत॑ म॒मृत॑म् त्वा वाजजि॒त्यायै॑ वाजजि॒त्यायै᳚ त्वा॒ ऽमृत᳚म् । \newline
59. वा॒ज॒जि॒त्याया॒ इति॑ वाज - जि॒त्यायै᳚ । \newline
60. त्वा॒ ऽमृत॑ म॒मृत॑म् त्वा त्वा॒ ऽमृत॑ मस्य स्य॒मृत॑म् त्वा त्वा॒ ऽमृत॑ मसि । \newline
61. अ॒मृत॑ मस्यस्य॒मृत॑ म॒मृत॑ मसि॒ पुष्टिः॒ पुष्टि॑ रस्य॒मृत॑ म॒मृत॑ मसि॒ पुष्टिः॑ । \newline
62. अ॒सि॒ पुष्टिः॒ पुष्टि॑ रस्यसि॒ पुष्टि॑ रस्यसि॒ पुष्टि॑ रस्यसि॒ पुष्टि॑ रसि । \newline
63. पुष्टि॑ रस्यसि॒ पुष्टिः॒ पुष्टि॑रसि प्र॒जन॑नम् प्र॒जन॑न मसि॒ पुष्टिः॒ पुष्टि॑रसि प्र॒जन॑नम् । \newline
64. अ॒सि॒ प्र॒जन॑नम् प्र॒जन॑न मस्यसि प्र॒जन॑न मस्यसि प्र॒जन॑न मस्यसि प्र॒जन॑न मसि । \newline
65. प्र॒जन॑न मस्यसि प्र॒जन॑नम् प्र॒जन॑न मसि । \newline
66. प्र॒जन॑न॒मिति॑ प्र - जन॑नम् । \newline
67. अ॒सीत्य॑सि । \newline
\pagebreak
\markright{ TS 1.7.10.1  \hfill https://www.vedavms.in \hfill}
\addcontentsline{toc}{section}{ TS 1.7.10.1 }
\section*{ TS 1.7.10.1 }

\textbf{TS 1.7.10.1 } \newline
\textbf{Samhita Paata} \newline

वाज॑स्ये॒मं प्र॑स॒वः सु॑षुवे॒ अग्रे॒ सोमꣳ॒॒ राजा॑न॒मोष॑धीष्व॒फ्सु । ता अ॒स्मभ्यं॒ मधु॑मतीर् भवन्तु व॒यꣳ रा॒ष्ट्रे जा᳚ग्रियाम पु॒रोहि॑ताः । वाज॑स्ये॒दं प्र॑स॒व आ ब॑भूवे॒मा च॒ विश्वा॒ भुव॑नानि स॒र्वतः॑ । स वि॒राजं॒ पर्ये॑ति प्रजा॒नन् प्र॒जां पुष्टिं॑ ॅव॒र्द्धय॑मानो अ॒स्मे । वाज॑स्ये॒मां प्र॑स॒वः शि॑श्रिये॒ दिव॑मि॒मा च॒ विश्वा॒ भुव॑नानि स॒म्राट् । अदि॑थ्सन्तं दापयतु प्रजा॒नन् र॒यिं - [ ] \newline

\textbf{Pada Paata} \newline

वाज॑स्य । इ॒मम् । प्र॒स॒व इति॑ प्र - स॒वः । सु॒षु॒वे॒ । अग्रे᳚ । सोम᳚म् । राजा॑नम् । ओष॑धीषु । अ॒फ्स्वित्य॑प् - सु ॥ ताः । अ॒स्मभ्य॒मित्य॒स्म - भ्य॒म् । मधु॑मती॒रिति॒ मधु॑ - म॒तीः॒ । भ॒व॒न्तु॒ । व॒यम् । रा॒ष्ट्रे । जा॒ग्रि॒या॒म॒ । पु॒रोहि॑ता॒ इति॑ पु॒रः - हि॒ताः॒ ॥ वाज॑स्य । इ॒दम् । प्र॒स॒व इति॑ प्र - स॒वः । एति॑ । ब॒भू॒व॒ । इ॒मा । च॒ । विश्वा᳚ । भुव॑नानि । स॒र्वतः॑ ॥ सः । वि॒राज॒मिति॑ वि - राज᳚म् । परीति॑ । ए॒ति॒ । प्र॒जा॒नन्निति॑ प्र - जा॒नन्न् । प्र॒जामिति॑ प्र - जाम् । पुष्टि᳚म् । व॒द्‌र्धय॑मानः । अ॒स्मे इति॑ ॥ वाज॑स्य । इ॒माम् । प्र॒स॒व इति॑ प्र-स॒वः । शि॒श्रि॒ये॒ । दिव᳚म् । इ॒मा । च॒ । विश्वा᳚ । भुव॑नानि । स॒म्राडिति॑ सं - राट् ॥ अदि॑थ्सन्तम् । दा॒प॒य॒तु॒ । प्र॒जा॒नन्निति॑ प्र - जा॒नन्न् । र॒यिम् ।  \newline


\textbf{Krama Paata} \newline

वाज॑स्ये॒मम् । इ॒मम् प्र॑स॒वः । प्र॒स॒वः सु॑षुवे । प्र॒स॒व इति॑ प्र - स॒वः । सु॒षु॒वे॒ अग्रे᳚ । अग्रे॒ सोम᳚म् । सोमꣳ॒॒ राजा॑नम् । राजा॑न॒मोष॑धीषु । ओष॑धीष्व॒फ्सु । अ॒फ्स्वित्य॑प् - सु ॥ ता अ॒स्मभ्य᳚म् । अ॒स्मभ्य॒म् मधु॑मतीः । अ॒स्मभ्य॒मित्य॒स्म - भ्य॒म् । मधु॑मतीर् भवन्तु । मधु॑मती॒रिति॒ मधु॑ - म॒तीः॒ । भ॒व॒न्तु॒ व॒यम् । व॒यꣳ रा॒ष्ट्रे ॥ रा॒ष्ट्रे जा᳚ग्रियाम । जा॒ग्रि॒या॒म॒ पु॒रोहि॑ताः । पु॒रोहि॑ता॒ इति॑ पु॒रः - हि॒ताः॒ ॥ वाज॑स्ये॒दम् । इ॒दम् प्र॑स॒वः । प्र॒स॒व आ । प्र॒स॒व इति॑ प्र - स॒वः । आ ब॑भूव । ब॒भू॒वे॒मा । इ॒मा च॑ । च॒ विश्वा᳚ । विश्वा॒ भुव॑नानि । भुव॑नानि स॒र्वतः॑ । स॒र्वत॒ इति॑ स॒र्वतः॑ ॥ स वि॒राज᳚म् । वि॒राज॒म् परि॑ । वि॒राज॒मिति॑ वि - राज᳚म् । पर्ये॑ति । ए॒ति॒ प्र॒जा॒नन्न् । प्र॒जा॒नन् प्र॒जाम् । प्र॒जा॒नन्निति॑ प्र - जा॒नन्न् । प्र॒जाम् पुष्टि᳚म् । प्र॒जामिति॑ प्र - जाम् । पुष्टिं॑ ॅव॒र्द्धय॑मानः । व॒र्द्धय॑मानो अ॒स्मे । अ॒स्मे इत्य॒स्मे ॥ वाज॑स्ये॒माम् । इ॒माम् प्र॑स॒वः । प्र॒स॒वः शि॑श्रिये । प्र॒स॒व इति॑ प्र - स॒वः । शि॒श्रि॒ये॒ दिव᳚म् । दिव॑मि॒मा । इ॒मा च॑ । च॒ विश्वा᳚ । विश्वा॒ भुव॑नानि । भुव॑नानि स॒म्राट् । स॒म्राडिति॑ सम् - राट् ॥ अदि॑थ्सन्तम् दापयतु । दा॒प॒य॒तु॒ प्र॒जा॒नन्न् । प्र॒जा॒नन् र॒यिम् । प्र॒जा॒नन्निति॑ प्र - जा॒नन्न् । र॒यिम् च॑ \newline

\textbf{Jatai Paata} \newline

1. वाज॑स्ये॒ म मि॒मं ॅवाज॑स्य॒ वाज॑स्ये॒ मम् । \newline
2. इ॒मम् प्र॑स॒वः प्र॑स॒व इ॒म मि॒मम् प्र॑स॒वः । \newline
3. प्र॒स॒वः सु॑षुवे सुषुवे प्रस॒वः प्र॑स॒वः सु॑षुवे । \newline
4. प्र॒स॒व इति॑ प्र - स॒वः । \newline
5. सु॒षु॒वे॒ अग्रे॒ अग्रे॑ सुषुवे सुषुवे॒ अग्रे᳚ । \newline
6. अग्रे॒ सोमꣳ॒॒ सोम॒ मग्रे॒ अग्रे॒ सोम᳚म् । \newline
7. सोमꣳ॒॒ राजा॑नꣳ॒॒ राजा॑नꣳ॒॒ सोमꣳ॒॒ सोमꣳ॒॒ राजा॑नम् । \newline
8. राजा॑न॒ मोष॑धी॒ ष्वोष॑धीषु॒ राजा॑नꣳ॒॒ राजा॑न॒ मोष॑धीषु । \newline
9. ओष॑धी ष्व॒फ्स्व॑ फ्स्वोष॑धी॒ ष्वोष॑धी ष्व॒फ्सु । \newline
10. अ॒फ्स्वित्य॑प् - सु । \newline
11. ता अ॒स्मभ्य॑ म॒स्मभ्य॒म् तास्ता अ॒स्मभ्य᳚म् । \newline
12. अ॒स्मभ्य॒म् मधु॑मती॒र् मधु॑मती र॒स्मभ्य॑ म॒स्मभ्य॒म् मधु॑मतीः । \newline
13. अ॒स्मभ्य॒मित्य॒स्म - भ्य॒म् । \newline
14. मधु॑मतीर् भवन्तु भवन्तु॒ मधु॑मती॒र् मधु॑मतीर् भवन्तु । \newline
15. मधु॑मती॒रिति॒ मधु॑ - म॒तीः॒ । \newline
16. भ॒व॒न्तु॒ व॒यं ॅव॒यम् भ॑वन्तु भवन्तु व॒यम् । \newline
17. व॒यꣳ रा॒ष्ट्रे रा॒ष्ट्रे व॒यं ॅव॒यꣳ रा॒ष्ट्रे । \newline
18. रा॒ष्ट्रे जा᳚ग्रियाम जाग्रियाम रा॒ष्ट्रे रा॒ष्ट्रे जा᳚ग्रियाम । \newline
19. जा॒ग्रि॒या॒म॒ पु॒रोहि॑ताः पु॒रोहि॑ता जाग्रियाम जाग्रियाम पु॒रोहि॑ताः । \newline
20. पु॒रोहि॑ता॒ इति॑ पु॒रः - हि॒ताः॒ । \newline
21. वाज॑स्ये॒ द मि॒दं ॅवाज॑स्य॒ वाज॑स्ये॒ दम् । \newline
22. इ॒दम् प्र॑स॒वः प्र॑स॒व इ॒द मि॒दम् प्र॑स॒वः । \newline
23. प्र॒स॒व आ प्र॑स॒वः प्र॑स॒व आ । \newline
24. प्र॒स॒व इति॑ प्र - स॒वः । \newline
25. आ ब॑भूव बभू॒वा ब॑भूव । \newline
26. ब॒भू॒वे॒ मेमा ब॑भूव बभूवे॒ मा । \newline
27. इ॒मा च॑ चे॒ मेमा च॑ । \newline
28. च॒ विश्वा॒ विश्वा॑ च च॒ विश्वा᳚ । \newline
29. विश्वा॒ भुव॑नानि॒ भुव॑नानि॒ विश्वा॒ विश्वा॒ भुव॑नानि । \newline
30. भुव॑नानि स॒र्वतः॑ स॒र्वतो॒ भुव॑नानि॒ भुव॑नानि स॒र्वतः॑ । \newline
31. स॒र्वत॒ इति॑ स॒र्वतः॑ । \newline
32. स वि॒राजं॑ ॅवि॒राजꣳ॒॒ स स वि॒राज᳚म् । \newline
33. वि॒राज॒म् परि॒ परि॑ वि॒राजं॑ ॅवि॒राज॒म् परि॑ । \newline
34. वि॒राज॒मिति॑ वि - राज᳚म् । \newline
35. पर्ये᳚त्येति॒ परि॒ पर्ये॑ति । \newline
36. ए॒ति॒ प्र॒जा॒नन् प्र॑जा॒नन् ने᳚त्येति प्रजा॒नन्न् । \newline
37. प्र॒जा॒नन् प्र॒जाम् प्र॒जाम् प्र॑जा॒नन् प्र॑जा॒नन् प्र॒जाम् । \newline
38. प्र॒जा॒नन्निति॑ प्र - जा॒नन्न् । \newline
39. प्र॒जाम् पुष्टि॒म् पुष्टि॑म् प्र॒जाम् प्र॒जाम् पुष्टि᳚म् । \newline
40. प्र॒जामिति॑ प्र - जाम् । \newline
41. पुष्टिं॑ ॅव॒र्द्धय॑मानो व॒र्द्धय॑मानः॒ पुष्टि॒म् पुष्टिं॑ ॅव॒र्द्धय॑मानः । \newline
42. व॒र्द्धय॑मानो अ॒स्मे अ॒स्मे व॒र्द्धय॑मानो व॒र्द्धय॑मानो अ॒स्मे । \newline
43. अ॒स्मे इत्य॒स्मे । \newline
44. वाज॑स्ये॒ मा मि॒मां ॅवाज॑स्य॒ वाज॑स्ये॒ माम् । \newline
45. इ॒माम् प्र॑स॒वः प्र॑स॒व इ॒मा मि॒माम् प्र॑स॒वः । \newline
46. प्र॒स॒वः शि॑श्रिये शिश्रिये प्रस॒वः प्र॑स॒वः शि॑श्रिये । \newline
47. प्र॒स॒व इति॑ प्र - स॒वः । \newline
48. शि॒श्रि॒ये॒ दिव॒म् दिवꣳ॑ शिश्रिये शिश्रिये॒ दिव᳚म् । \newline
49. दिव॑ मि॒मेमा दिव॒म् दिव॑ मि॒मा । \newline
50. इ॒मा च॑ चे॒ मेमा च॑ । \newline
51. च॒ विश्वा॒ विश्वा॑ च च॒ विश्वा᳚ । \newline
52. विश्वा॒ भुव॑नानि॒ भुव॑नानि॒ विश्वा॒ विश्वा॒ भुव॑नानि । \newline
53. भुव॑नानि स॒म्राट् थ्स॒म्राड् भुव॑नानि॒ भुव॑नानि स॒म्राट् । \newline
54. स॒म्राडिति॑ सं - राट् । \newline
55. अदि॑थ्सन्तम् दापयतु दापय॒ त्वदि॑थ्सन्त॒ मदि॑थ्सन्तम् दापयतु । \newline
56. दा॒प॒य॒तु॒ प्र॒जा॒नन् प्र॑जा॒नन् दा॑पयतु दापयतु प्रजा॒नन्न् । \newline
57. प्र॒जा॒नन् र॒यिꣳ र॒यिम् प्र॑जा॒नन् प्र॑जा॒नन् र॒यिम् । \newline
58. प्र॒जा॒नन्निति॑ प्र - जा॒नन्न् । \newline
59. र॒यिम् च॑ च र॒यिꣳ र॒यिम् च॑ । \newline

\textbf{Ghana Paata } \newline

1. वाज॑स्ये॒ म मि॒मं ॅवाज॑स्य॒ वाज॑स्ये॒ मम् प्र॑स॒वः प्र॑स॒व इ॒मं ॅवाज॑स्य॒ वाज॑स्ये॒ मम् प्र॑स॒वः । \newline
2. इ॒मम् प्र॑स॒वः प्र॑स॒व इ॒म मि॒मम् प्र॑स॒वः सु॑षुवे सुषुवे प्रस॒व इ॒म मि॒मम् प्र॑स॒वः सु॑षुवे । \newline
3. प्र॒स॒वः सु॑षुवे सुषुवे प्रस॒वः प्र॑स॒वः सु॑षुवे॒ अग्रे॒ अग्रे॑ सुषुवे प्रस॒वः प्र॑स॒वः सु॑षुवे॒ अग्रे᳚ । \newline
4. प्र॒स॒व इति॑ प्र - स॒वः । \newline
5. सु॒षु॒वे॒ अग्रे॒ अग्रे॑ सुषुवे सुषुवे॒ अग्रे॒ सोम॒(ग्म्॒) सोम॒ मग्रे॑ सुषुवे सुषुवे॒ अग्रे॒ सोम᳚म् । \newline
6. अग्रे॒ सोम॒(ग्म्॒) सोम॒ मग्रे॒ अग्रे॒ सोम॒(ग्म्॒) राजा॑न॒(ग्म्॒) राजा॑न॒(ग्म्॒) सोम॒ मग्रे॒ अग्रे॒ सोम॒(ग्म्॒) राजा॑नम् । \newline
7. सोम॒(ग्म्॒) राजा॑न॒(ग्म्॒) राजा॑न॒(ग्म्॒) सोम॒(ग्म्॒) सोम॒(ग्म्॒) राजा॑न॒ मोष॑धी॒ ष्वोष॑धीषु॒ राजा॑न॒(ग्म्॒) सोम॒(ग्म्॒) सोम॒(ग्म्॒) राजा॑न॒ मोष॑धीषु । \newline
8. राजा॑न॒ मोष॑धी॒ ष्वोष॑धीषु॒ राजा॑न॒(ग्म्॒) राजा॑न॒ मोष॑धी ष्व॒फ्स्व॑ फ्स्वोष॑धीषु॒ राजा॑न॒(ग्म्॒) राजा॑न॒ मोष॑धी ष्व॒फ्सु । \newline
9. ओष॑धी ष्व॒फ्स्व॑ फ्स्वोष॑धी॒ ष्वोष॑धी ष्व॒फ्सु । \newline
10. अ॒फ्स्वित्य॑प् - सु । \newline
11. ता अ॒स्मभ्य॑ म॒स्मभ्य॒म् तास्ता अ॒स्मभ्य॒म् मधु॑मती॒र् मधु॑मती र॒स्मभ्य॒म् तास्ता अ॒स्मभ्य॒म् मधु॑मतीः । \newline
12. अ॒स्मभ्य॒म् मधु॑मती॒र् मधु॑मती र॒स्मभ्य॑ म॒स्मभ्य॒म् मधु॑मतीर् भवन्तु भवन्तु॒ मधु॑मती र॒स्मभ्य॑ म॒स्मभ्य॒म् मधु॑मतीर् भवन्तु । \newline
13. अ॒स्मभ्य॒मित्य॒स्म - भ्य॒म् । \newline
14. मधु॑मतीर् भवन्तु भवन्तु॒ मधु॑मती॒र् मधु॑मतीर् भवन्तु व॒यं ॅव॒यम् भ॑वन्तु॒ मधु॑मती॒र् मधु॑मतीर् भवन्तु व॒यम् । \newline
15. मधु॑मती॒रिति॒ मधु॑ - म॒तीः॒ । \newline
16. भ॒व॒न्तु॒ व॒यं ॅव॒यम् भ॑वन्तु भवन्तु व॒यꣳ रा॒ष्ट्रे रा॒ष्ट्रे व॒यम् भ॑वन्तु भवन्तु व॒यꣳ रा॒ष्ट्रे । \newline
17. व॒यꣳ रा॒ष्ट्रे रा॒ष्ट्रे व॒यं ॅव॒यꣳ रा॒ष्ट्रे जा᳚ग्रियाम जाग्रियाम रा॒ष्ट्रे व॒यं ॅव॒यꣳ रा॒ष्ट्रे जा᳚ग्रियाम । \newline
18. रा॒ष्ट्रे जा᳚ग्रियाम जाग्रियाम रा॒ष्ट्रे रा॒ष्ट्रे जा᳚ग्रियाम पु॒रोहि॑ताः पु॒रोहि॑ता जाग्रियाम रा॒ष्ट्रे रा॒ष्ट्रे जा᳚ग्रियाम पु॒रोहि॑ताः । \newline
19. जा॒ग्रि॒या॒म॒ पु॒रोहि॑ताः पु॒रोहि॑ता जाग्रियाम जाग्रियाम पु॒रोहि॑ताः । \newline
20. पु॒रोहि॑ता॒ इति॑ पु॒रः - हि॒ताः॒ । \newline
21. वाज॑स्ये॒ द मि॒दं ॅवाज॑स्य॒ वाज॑स्ये॒ दम् प्र॑स॒वः प्र॑स॒व इ॒दं ॅवाज॑स्य॒ वाज॑स्ये॒ दम् प्र॑स॒वः । \newline
22. इ॒दम् प्र॑स॒वः प्र॑स॒व इ॒द मि॒दम् प्र॑स॒व आ प्र॑स॒व इ॒द मि॒दम् प्र॑स॒व आ । \newline
23. प्र॒स॒व आ प्र॑स॒वः प्र॑स॒व आ ब॑भूव बभू॒वा प्र॑स॒वः प्र॑स॒व आ ब॑भूव । \newline
24. प्र॒स॒व इति॑ प्र - स॒वः । \newline
25. आ ब॑भूव बभू॒वा ब॑भूवे॒ मेमा ब॑भू॒वा ब॑भूवे॒ मा । \newline
26. ब॒भू॒वे॒ मेमा ब॑भूव बभूवे॒ मा च॑ चे॒ मा ब॑भूव बभूवे॒ मा च॑ । \newline
27. इ॒मा च॑ चे॒ मेमा च॒ विश्वा॒ विश्वा॑ चे॒ मेमा च॒ विश्वा᳚ । \newline
28. च॒ विश्वा॒ विश्वा॑ च च॒ विश्वा॒ भुव॑नानि॒ भुव॑नानि॒ विश्वा॑ च च॒ विश्वा॒ भुव॑नानि । \newline
29. विश्वा॒ भुव॑नानि॒ भुव॑नानि॒ विश्वा॒ विश्वा॒ भुव॑नानि स॒र्वतः॑ स॒र्वतो॒ भुव॑नानि॒ विश्वा॒ विश्वा॒ भुव॑नानि स॒र्वतः॑ । \newline
30. भुव॑नानि स॒र्वतः॑ स॒र्वतो॒ भुव॑नानि॒ भुव॑नानि स॒र्वतः॑ । \newline
31. स॒र्वत॒ इति॑ स॒र्वतः॑ । \newline
32. स वि॒राजं॑ ॅवि॒राज॒(ग्म्॒) स स वि॒राज॒म् परि॒ परि॑ वि॒राज॒(ग्म्॒) स स वि॒राज॒म् परि॑ । \newline
33. वि॒राज॒म् परि॒ परि॑ वि॒राजं॑ ॅवि॒राज॒म् पर्ये᳚त्येति॒ परि॑ वि॒राजं॑ ॅवि॒राज॒म् पर्ये॑ति । \newline
34. वि॒राज॒मिति॑ वि - राज᳚म् । \newline
35. पर्ये᳚त्येति॒ परि॒ पर्ये॑ति प्रजा॒नन् प्र॑जा॒नन् ने॑ति॒ परि॒ पर्ये॑ति प्रजा॒नन्न् । \newline
36. ए॒ति॒ प्र॒जा॒नन् प्र॑जा॒नन् ने᳚त्येति प्रजा॒नन् प्र॒जाम् प्र॒जाम् प्र॑जा॒नन् ने᳚त्येति प्रजा॒नन् प्र॒जाम् । \newline
37. प्र॒जा॒नन् प्र॒जाम् प्र॒जाम् प्र॑जा॒नन् प्र॑जा॒नन् प्र॒जाम् पुष्टि॒म् पुष्टि॑म् प्र॒जाम् प्र॑जा॒नन् प्र॑जा॒नन् प्र॒जाम् पुष्टि᳚म् । \newline
38. प्र॒जा॒नन्निति॑ प्र - जा॒नन्न् । \newline
39. प्र॒जाम् पुष्टि॒म् पुष्टि॑म् प्र॒जाम् प्र॒जाम् पुष्टिं॑ ॅव॒र्द्धय॑मानो व॒र्द्धय॑मानः॒ पुष्टि॑म् प्र॒जाम् प्र॒जाम् पुष्टिं॑ ॅव॒र्द्धय॑मानः । \newline
40. प्र॒जामिति॑ प्र - जाम् । \newline
41. पुष्टिं॑ ॅव॒र्द्धय॑मानो व॒र्द्धय॑मानः॒ पुष्टि॒म् पुष्टिं॑ ॅव॒र्द्धय॑मानो अ॒स्मे अ॒स्मे व॒र्द्धय॑मानः॒ पुष्टि॒म् पुष्टिं॑ ॅव॒र्द्धय॑मानो अ॒स्मे । \newline
42. व॒र्द्धय॑मानो अ॒स्मे अ॒स्मे व॒र्द्धय॑मानो व॒र्द्धय॑मानो अ॒स्मे । \newline
43. अ॒स्मे इत्य॒स्मे । \newline
44. वाज॑स्ये॒ मा मि॒मां ॅवाज॑स्य॒ वाज॑स्ये॒ माम् प्र॑स॒वः प्र॑स॒व इ॒मां ॅवाज॑स्य॒ वाज॑स्ये॒ माम् प्र॑स॒वः । \newline
45. इ॒माम् प्र॑स॒वः प्र॑स॒व इ॒मा मि॒माम् प्र॑स॒वः शि॑श्रिये शिश्रिये प्रस॒व इ॒मा मि॒माम् प्र॑स॒वः शि॑श्रिये । \newline
46. प्र॒स॒वः शि॑श्रिये शिश्रिये प्रस॒वः प्र॑स॒वः शि॑श्रिये॒ दिव॒म् दिव(ग्म्॑) शिश्रिये प्रस॒वः प्र॑स॒वः शि॑श्रिये॒ दिव᳚म् । \newline
47. प्र॒स॒व इति॑ प्र - स॒वः । \newline
48. शि॒श्रि॒ये॒ दिव॒म् दिव(ग्म्॑) शिश्रिये शिश्रिये॒ दिव॑ मि॒मेमा दिव(ग्म्॑) शिश्रिये शिश्रिये॒ दिव॑ मि॒मा । \newline
49. दिव॑ मि॒मेमा दिव॒म् दिव॑ मि॒मा च॑ चे॒ मा दिव॒म् दिव॑ मि॒मा च॑ । \newline
50. इ॒मा च॑ चे॒ मेमा च॒ विश्वा॒ विश्वा॑ चे॒ मेमा च॒ विश्वा᳚ । \newline
51. च॒ विश्वा॒ विश्वा॑ च च॒ विश्वा॒ भुव॑नानि॒ भुव॑नानि॒ विश्वा॑ च च॒ विश्वा॒ भुव॑नानि । \newline
52. विश्वा॒ भुव॑नानि॒ भुव॑नानि॒ विश्वा॒ विश्वा॒ भुव॑नानि स॒म्राट् थ्स॒म्राड् भुव॑नानि॒ विश्वा॒ विश्वा॒ भुव॑नानि स॒म्राट् । \newline
53. भुव॑नानि स॒म्राट् थ्स॒म्राड् भुव॑नानि॒ भुव॑नानि स॒म्राट् । \newline
54. स॒म्राडिति॑ सं - राट् । \newline
55. अदि॑थ्सन्तम् दापयतु दापय॒ त्वदि॑थ्सन्त॒ मदि॑थ्सन्तम् दापयतु प्रजा॒नन् प्र॑जा॒नन् दा॑पय॒ त्वदि॑थ्सन्त॒ मदि॑थ्सन्तम् दापयतु प्रजा॒नन्न् । \newline
56. दा॒प॒य॒तु॒ प्र॒जा॒नन् प्र॑जा॒नन् दा॑पयतु दापयतु प्रजा॒नन् र॒यिꣳ र॒यिम् प्र॑जा॒नन् दा॑पयतु दापयतु प्रजा॒नन् र॒यिम् । \newline
57. प्र॒जा॒नन् र॒यिꣳ र॒यिम् प्र॑जा॒नन् प्र॑जा॒नन् र॒यिम् च॑ च र॒यिम् प्र॑जा॒नन् प्र॑जा॒नन् र॒यिम् च॑ । \newline
58. प्र॒जा॒नन्निति॑ प्र - जा॒नन्न् । \newline
59. र॒यिम् च॑ च र॒यिꣳ र॒यिम् च॑ नो नश्च र॒यिꣳ र॒यिम् च॑ नः । \newline
\pagebreak
\markright{ TS 1.7.10.2  \hfill https://www.vedavms.in \hfill}
\addcontentsline{toc}{section}{ TS 1.7.10.2 }
\section*{ TS 1.7.10.2 }

\textbf{TS 1.7.10.2 } \newline
\textbf{Samhita Paata} \newline

च॑ नः॒ सर्व॑वीरां॒ नि य॑च्छतु ॥ अग्ने॒ अच्छा॑ वदे॒ह नः॒ प्रति॑ नः सु॒मना॑ भव । प्र णो॑ यच्छ भुवस्पते धन॒दा अ॑सि न॒स्त्वं ॥ प्र णो॑ यच्छत्वर्य॒मा प्र भगः॒ प्र बृह॒स्पतिः॑ । प्र दे॒वाः प्रोत सू॒नृता॒ प्र वाग् दे॒वी द॑दातु नः ॥ अ॒र्य॒मणं॒ बृह॒स्पति॒मिन्द्रं॒ दाना॑य चोदय । वाचं॒ ॅविष्णुꣳ॒॒ सर॑स्वतीꣳ सवि॒तारं॑ - [ ] \newline

\textbf{Pada Paata} \newline

च॒ । नः॒ । सर्व॑वीरा॒मिति॒ सर्व॑ - वी॒रा॒म् । नीति॑ । य॒च्छ॒तु॒ ॥ अग्ने᳚ । अच्छ॑ । व॒द॒ । इ॒ह । नः॒ । प्रतीति॑ । नः॒ । सु॒मना॒ इति॑ सु - मनाः᳚ । भ॒व॒ ॥ प्रेति॑ । नः॒ । य॒च्छ॒ । भु॒वः॒ । प॒ते॒ । ध॒न॒दा इति॑ धन - दाः । अ॒सि॒ । नः॒ । त्वम् ॥ प्रेति॑ । नः॒ । य॒च्छ॒तु॒ । अ॒र्य॒मा । प्रेति॑ । भगः॑ । प्रेति॑ । बृह॒स्पतिः॑ ॥ प्रेति॑ । दे॒वाः । प्रेति॑ । उ॒त । सू॒नृता᳚ । प्रेति॑ । वाक् । दे॒वी । द॒दा॒तु॒ । नः॒ ॥ अ॒र्य॒मण᳚म् । बृह॒स्पति᳚म् । इन्द्र᳚म् । दाना॑य । चो॒द॒य॒ ॥ वाच᳚म् । विष्णु᳚म् । सर॑स्वतीम् । स॒वि॒तार᳚म् ।  \newline


\textbf{Krama Paata} \newline

च॒ नः॒ । नः॒ सर्व॑वीराम् । सर्व॑वीरा॒म् नि । सर्व॑वीरा॒मिति॒ सर्व॑ - वी॒रा॒म् । नि य॑च्छतु । य॒च्छ॒त्विति॑ यच्छतु ॥ अग्ने॒ अच्छ॑ । अच्छा॑ वद । व॒दे॒ह । इ॒ह नः॑ । नः॒ प्रति॑ । प्रति॑ नः । नः॒ सु॒मनाः᳚ । सु॒मना॑ भव । सु॒मना॒ इति॑ सु - मनाः᳚ । भ॒वेति॑ भव ॥ प्र णः॑ । नो॒ य॒च्छ॒ ॥ य॒च्छ॒ भु॒वः॒ । भु॒व॒स्प॒ते॒ । प॒ते॒ ध॒न॒दाः । ध॒न॒दा अ॑सि । ध॒न॒दा इति॑ धन - दाः । अ॒सि॒ नः॒ । न॒स्त्वम् । त्वमिति॒ त्वम् ॥ प्र णः॑ । नो॒ य॒च्छ॒तु॒ । य॒च्छ॒त्व॒र्य॒मा । अ॒र्य॒मा प्र । प्र भगः॑ । भगः॒ प्र । प्र बृह॒स्पतिः॑ । बृह॒स्पति॒रिति॒ बृह॒स्पतिः॑ ॥ प्र दे॒वाः । दे॒वाः प्र । प्रोत । उ॒त सू॒नृता᳚ । सू॒नृता॒ प्र । प्र वाक् । वाग्दे॒वी । दे॒वी द॑दातु । द॒दा॒तु॒ नः॒ । न॒ इति॑ नः ॥ अ॒र्य॒मण॒म् बृह॒स्पति᳚म् । बृह॒स्पति॒मिन्द्र᳚म् । इन्द्र॒म् दाना॑य । दाना॑य चोदय । चो॒द॒ये॒ति॑ चोदय ॥ वाचं॒ ॅविष्णु᳚म् । विष्णुꣳ॒॒ सर॑स्वतीम् । सर॑स्वतीꣳ सवि॒तार᳚म् । स॒वि॒तार॑म् च \newline

\textbf{Jatai Paata} \newline

1. च॒ नो॒ न॒श्च॒ च॒ नः॒ । \newline
2. नः॒ सर्व॑वीराꣳ॒॒ सर्व॑वीरान् नो नः॒ सर्व॑वीराम् । \newline
3. सर्व॑वीरा॒म् नि नि सर्व॑वीराꣳ॒॒ सर्व॑वीरा॒म् नि । \newline
4. सर्व॑वीरा॒मिति॒ सर्व॑ - वी॒रा॒म् । \newline
5. नि य॑च्छतु यच्छतु॒ नि नि य॑च्छतु । \newline
6. य॒च्छ॒त्विति॑ यच्छतु । \newline
7. अग्ने॒ अच्छा च्छाग्ने ऽग्ने॒ अच्छ॑ । \newline
8. अच्छा॑ वद व॒दाच्छा च्छा॑ वद । \newline
9. व॒दे॒ हे ह व॑द वदे॒ ह । \newline
10. इ॒ह नो॑ न इ॒हे ह नः॑ । \newline
11. नः॒ प्रति॒ प्रति॑ नो नः॒ प्रति॑ । \newline
12. प्रति॑ नो नः॒ प्रति॒ प्रति॑ नः । \newline
13. नः॒ सु॒मनाः᳚ सु॒मना॑ नो नः सु॒मनाः᳚ । \newline
14. सु॒मना॑ भव भव सु॒मनाः᳚ सु॒मना॑ भव । \newline
15. सु॒मना॒ इति॑ सु - मनाः᳚ । \newline
16. भ॒वेति॑ भव । \newline
17. प्र णो॑ नः॒ प्र प्र णः॑ । \newline
18. नो॒ य॒च्छ॒ य॒च्छ॒ नो॒ नो॒ य॒च्छ॒ । \newline
19. य॒च्छ॒ भु॒वो॒ भु॒वो॒ य॒च्छ॒ य॒च्छ॒ भु॒वः॒ । \newline
20. भु॒व॒ स्प॒ते॒ प॒ते॒ भु॒वो॒ भु॒व॒ स्प॒ते॒ । \newline
21. प॒ते॒ ध॒न॒दा ध॑न॒दा स्प॑ते पते धन॒दाः । \newline
22. ध॒न॒दा अ॑स्यसि धन॒दा ध॑न॒दा अ॑सि । \newline
23. ध॒न॒दा इति॑ धन - दाः । \newline
24. अ॒सि॒ नो॒ नो॒ ऽस्य॒सि॒ नः॒ । \newline
25. न॒स्त्वम् त्वन्नो॑ न॒स्त्वम् । \newline
26. त्वमिति॒ त्वम् । \newline
27. प्र णो॑ नः॒ प्र प्र णः॑ । \newline
28. नो॒ य॒च्छ॒तु॒ य॒च्छ॒तु॒ नो॒ नो॒ य॒च्छ॒तु॒ । \newline
29. य॒च्छ॒त्व॒र्य॒मा ऽर्य॒मा य॑च्छतु यच्छत्वर्य॒मा । \newline
30. अ॒र्य॒मा प्र प्रार्य॒मा ऽर्य॒मा प्र । \newline
31. प्र भगो॒ भगः॒ प्र प्र भगः॑ । \newline
32. भगः॒ प्र प्र भगो॒ भगः॒ प्र । \newline
33. प्र बृह॒स्पति॒र् बृह॒स्पतिः॒ प्र प्र बृह॒स्पतिः॑ । \newline
34. बृह॒स्पति॒रिति॒ बृह॒स्पतिः॑ । \newline
35. प्र दे॒वा दे॒वाः प्र प्र दे॒वाः । \newline
36. दे॒वाः प्र प्र दे॒वा दे॒वाः प्र । \newline
37. प्रोतोत प्र प्रोत । \newline
38. उ॒त सू॒नृता॑ सू॒नृ तो॒तोत सू॒नृता᳚ । \newline
39. सू॒नृता॒ प्र प्र सू॒नृता॑ सू॒नृता॒ प्र । \newline
40. प्र वाग् वाक् प्र प्र वाक् । \newline
41. वाग् दे॒वी दे॒वी वाग् वाग् दे॒वी । \newline
42. दे॒वी द॑दातु ददातु दे॒वी दे॒वी द॑दातु । \newline
43. द॒दा॒तु॒ नो॒ नो॒ द॒दा॒तु॒ द॒दा॒तु॒ नः॒ । \newline
44. न॒ इति॑ नः । \newline
45. अ॒र्य॒मण॒म् बृह॒स्पति॒म् बृह॒स्पति॑ मर्य॒मण॑ मर्य॒मण॒म् बृह॒स्पति᳚म् । \newline
46. बृह॒स्पति॒ मिन्द्र॒ मिन्द्र॒म् बृह॒स्पति॒म् बृह॒स्पति॒ मिन्द्र᳚म् । \newline
47. इन्द्र॒म् दाना॑य॒ दाना॒ये न्द्र॒ मिन्द्र॒म् दाना॑य । \newline
48. दाना॑य चोदय चोदय॒ दाना॑य॒ दाना॑य चोदय । \newline
49. चो॒द॒येति॑ चोदय । \newline
50. वाचं॒ ॅविष्णुं॒ ॅविष्णुं॒ ॅवाचं॒ ॅवाचं॒ ॅविष्णु᳚म् । \newline
51. विष्णुꣳ॒॒ सर॑स्वतीꣳ॒॒ सर॑स्वतीं॒ ॅविष्णुं॒ ॅविष्णुꣳ॒॒ सर॑स्वतीम् । \newline
52. सर॑स्वतीꣳ सवि॒तारꣳ॑ सवि॒तारꣳ॒॒ सर॑स्वतीꣳ॒॒ सर॑स्वतीꣳ सवि॒तार᳚म् । \newline
53. स॒वि॒तार॑म् च च सवि॒तारꣳ॑ सवि॒तार॑म् च । \newline

\textbf{Ghana Paata } \newline

1. च॒ नो॒ न॒श्च॒ च॒ नः॒ सर्व॑वीरा॒(ग्म्॒) सर्व॑वीराम् नश्च च नः॒ सर्व॑वीराम् । \newline
2. नः॒ सर्व॑वीरा॒(ग्म्॒) सर्व॑वीराम् नो नः॒ सर्व॑वीरा॒म् नि नि सर्व॑वीराम् नो नः॒ सर्व॑वीरा॒म् नि । \newline
3. सर्व॑वीरा॒म् नि नि सर्व॑वीरा॒(ग्म्॒) सर्व॑वीरा॒म् नि य॑च्छतु यच्छतु॒ नि सर्व॑वीरा॒(ग्म्॒) सर्व॑वीरा॒म् नि य॑च्छतु । \newline
4. सर्व॑वीरा॒मिति॒ सर्व॑ - वी॒रा॒म् । \newline
5. नि य॑च्छतु यच्छतु॒ नि नि य॑च्छतु । \newline
6. य॒च्छ॒त्विति॑ यच्छतु । \newline
7. अग्ने॒ अच्छा च्छाग्ने ऽग्ने॒ अच्छा॑ वद व॒दा च्छाग्ने ऽग्ने॒ अच्छा॑ वद । \newline
8. अच्छा॑ वद व॒दाच्छा च्छा॑ वदे॒ हे ह व॒दा च्छा च्छा॑ वदे॒ ह । \newline
9. व॒दे॒ हे ह व॑द वदे॒ ह नो॑ न इ॒ह व॑द वदे॒ ह नः॑ । \newline
10. इ॒ह नो॑ न इ॒हे ह नः॒ प्रति॒ प्रति॑ न इ॒हे ह नः॒ प्रति॑ । \newline
11. नः॒ प्रति॒ प्रति॑ नो नः॒ प्रति॑ नो नः॒ प्रति॑ नो नः॒ प्रति॑ नः । \newline
12. प्रति॑ नो नः॒ प्रति॒ प्रति॑ नः सु॒मनाः᳚ सु॒मना॑ नः॒ प्रति॒ प्रति॑ नः सु॒मनाः᳚ । \newline
13. नः॒ सु॒मनाः᳚ सु॒मना॑ नो नः सु॒मना॑ भव भव सु॒मना॑ नो नः सु॒मना॑ भव । \newline
14. सु॒मना॑ भव भव सु॒मनाः᳚ सु॒मना॑ भव । \newline
15. सु॒मना॒ इति॑ सु - मनाः᳚ । \newline
16. भ॒वेति॑ भव । \newline
17. प्र णो॑ नः॒ प्र प्र णो॑ यच्छ यच्छ नः॒ प्र प्र णो॑ यच्छ । \newline
18. नो॒ य॒च्छ॒ य॒च्छ॒ नो॒ नो॒ य॒च्छ॒ भु॒वो॒ भु॒वो॒ य॒च्छ॒ नो॒ नो॒ य॒च्छ॒ भु॒वः॒ । \newline
19. य॒च्छ॒ भु॒वो॒ भु॒वो॒ य॒च्छ॒ य॒च्छ॒ भु॒व॒ स्प॒ते॒ प॒ते॒ भु॒वो॒ य॒च्छ॒ य॒च्छ॒ भु॒व॒ स्प॒ते॒ । \newline
20. भु॒व॒ स्प॒ते॒ प॒ते॒ भु॒वो॒ भु॒व॒ स्प॒ते॒ ध॒न॒दा ध॑न॒दा स्प॑ते भुवो भुव स्पते धन॒दाः । \newline
21. प॒ते॒ ध॒न॒दा ध॑न॒दा स्प॑ते पते धन॒दा अ॑स्यसि धन॒दा स्प॑ते पते धन॒दा अ॑सि । \newline
22. ध॒न॒दा अ॑स्यसि धन॒दा ध॑न॒दा अ॑सि नो नो ऽसि धन॒दा ध॑न॒दा अ॑सि नः । \newline
23. ध॒न॒दा इति॑ धन - दाः । \newline
24. अ॒सि॒ नो॒ नो॒ ऽस्य॒सि॒ न॒स्त्वम् त्वम् नो᳚ ऽस्यसि न॒स्त्वम् । \newline
25. न॒स्त्वम् त्वम् नो॑ न॒स्त्वम् । \newline
26. त्वमिति॒ त्वम् । \newline
27. प्र णो॑ नः॒ प्र प्र णो॑ यच्छतु यच्छतु नः॒ प्र प्र णो॑ यच्छतु । \newline
28. नो॒ य॒च्छ॒तु॒ य॒च्छ॒तु॒ नो॒ नो॒ य॒च्छ॒ त्व॒र्य॒मा ऽर्य॒मा य॑च्छतु नो नो यच्छ त्वर्य॒मा । \newline
29. य॒च्छ॒ त्व॒र्य॒मा ऽर्य॒मा य॑च्छतु यच्छ त्वर्य॒मा प्र प्रार्य॒मा य॑च्छतु यच्छ त्वर्य॒मा प्र । \newline
30. अ॒र्य॒मा प्र प्रार्य॒मा ऽर्य॒मा प्र भगो॒ भगः॒ प्रार्य॒मा ऽर्य॒मा प्र भगः॑ । \newline
31. प्र भगो॒ भगः॒ प्र प्र भगः॒ प्र प्र भगः॒ प्र प्र भगः॒ प्र । \newline
32. भगः॒ प्र प्र भगो॒ भगः॒ प्र बृह॒स्पति॒र् बृह॒स्पतिः॒ प्र भगो॒ भगः॒ प्र बृह॒स्पतिः॑ । \newline
33. प्र बृह॒स्पति॒र् बृह॒स्पतिः॒ प्र प्र बृह॒स्पतिः॑ । \newline
34. बृह॒स्पति॒रिति॒ बृह॒स्पतिः॑ । \newline
35. प्र दे॒वा दे॒वाः प्र प्र दे॒वाः प्र प्र दे॒वाः प्र प्र दे॒वाः प्र । \newline
36. दे॒वाः प्र प्र दे॒वा दे॒वाः प्रोतोत प्र दे॒वा दे॒वाः प्रोत । \newline
37. प्रोतोत प्र प्रोत सू॒नृता॑ सू॒नृतो॒त प्र प्रोत सू॒नृता᳚ । \newline
38. उ॒त सू॒नृता॑ सू॒नृतो॒तोत सू॒नृता॒ प्र प्र सू॒नृतो॒तोत सू॒नृता॒ प्र । \newline
39. सू॒नृता॒ प्र प्र सू॒नृता॑ सू॒नृता॒ प्र वाग् वाक् प्र सू॒नृता॑ सू॒नृता॒ प्र वाक् । \newline
40. प्र वाग् वाक् प्र प्र वाग् दे॒वी दे॒वी वाक् प्र प्र वाग् दे॒वी । \newline
41. वाग् दे॒वी दे॒वी वाग् वाग् दे॒वी द॑दातु ददातु दे॒वी वाग् वाग् दे॒वी द॑दातु । \newline
42. दे॒वी द॑दातु ददातु दे॒वी दे॒वी द॑दातु नो नो ददातु दे॒वी दे॒वी द॑दातु नः । \newline
43. द॒दा॒तु॒ नो॒ नो॒ द॒दा॒तु॒ द॒दा॒तु॒ नः॒ । \newline
44. न॒ इति॑ नः । \newline
45. अ॒र्य॒मण॒म् बृह॒स्पति॒म् बृह॒स्पति॑ मर्य॒मण॑ मर्य॒मण॒म् बृह॒स्पति॒ मिन्द्र॒ मिन्द्र॒म् बृह॒स्पति॑ मर्य॒मण॑ मर्य॒मण॒म् बृह॒स्पति॒ मिन्द्र᳚म् । \newline
46. बृह॒स्पति॒ मिन्द्र॒ मिन्द्र॒म् बृह॒स्पति॒म् बृह॒स्पति॒ मिन्द्र॒म् दाना॑य॒ दाना॒ये न्द्र॒म् बृह॒स्पति॒म् बृह॒स्पति॒ मिन्द्र॒म् दाना॑य । \newline
47. इन्द्र॒म् दाना॑य॒ दाना॒ये न्द्र॒ मिन्द्र॒म् दाना॑य चोदय चोदय॒ दाना॒ये न्द्र॒ मिन्द्र॒म् दाना॑य चोदय । \newline
48. दाना॑य चोदय चोदय॒ दाना॑य॒ दाना॑य चोदय । \newline
49. चो॒द॒येति॑ चोदय । \newline
50. वाचं॒ ॅविष्णुं॒ ॅविष्णुं॒ ॅवाचं॒ ॅवाचं॒ ॅविष्णु॒(ग्म्॒) सर॑स्वती॒(ग्म्॒) सर॑स्वतीं॒ ॅविष्णुं॒ ॅवाचं॒ ॅवाचं॒ ॅविष्णु॒(ग्म्॒) सर॑स्वतीम् । \newline
51. विष्णु॒(ग्म्॒) सर॑स्वती॒(ग्म्॒) सर॑स्वतीं॒ ॅविष्णुं॒ ॅविष्णु॒(ग्म्॒) सर॑स्वतीꣳ सवि॒तार(ग्म्॑) सवि॒तार॒(ग्म्॒) सर॑स्वतीं॒ ॅविष्णुं॒ ॅविष्णु॒(ग्म्॒) सर॑स्वतीꣳ सवि॒तार᳚म् । \newline
52. सर॑स्वतीꣳ सवि॒तार(ग्म्॑) सवि॒तार॒(ग्म्॒) सर॑स्वती॒(ग्म्॒) सर॑स्वतीꣳ सवि॒तार॑म् च च सवि॒तार॒(ग्म्॒) सर॑स्वती॒(ग्म्॒) सर॑स्वतीꣳ सवि॒तार॑म् च । \newline
53. स॒वि॒तार॑म् च च सवि॒तार(ग्म्॑) सवि॒तार॑म् च वा॒जिनं॑ ॅवा॒जिन॑म् च सवि॒तार(ग्म्॑) सवि॒तार॑म् च वा॒जिन᳚म् । \newline
\pagebreak
\markright{ TS 1.7.10.3  \hfill https://www.vedavms.in \hfill}
\addcontentsline{toc}{section}{ TS 1.7.10.3 }
\section*{ TS 1.7.10.3 }

\textbf{TS 1.7.10.3 } \newline
\textbf{Samhita Paata} \newline

च वा॒जिनं᳚ ॥ सोमꣳ॒॒ राजा॑नं॒ ॅवरु॑णम॒ग्नि-म॒न्वार॑भामहे । आ॒दि॒त्यान् विष्णुꣳ॒॒ सूर्यं॑ ब्र॒ह्माणं॑ च॒ बृह॒स्पतिं᳚ ॥ दे॒वस्य॑ त्वा सवि॒तुः प्र॑स॒वे᳚ऽश्विनो᳚र् बा॒हुभ्यां᳚ पू॒ष्णो हस्ता᳚भ्याꣳ॒॒ सर॑स्वत्यै वा॒चो य॒न्तुर् य॒न्त्रेणा॒ग्नेस्त्वा॒ आम्रा᳚ज्येना॒भिषि॑ञ्चा॒मीन्द्र॑स्य॒ बृह॒स्पते᳚स्त्वा॒ साम्रा᳚ज्येना॒भिषि॑ञ्चामि ॥ \newline

\textbf{Pada Paata} \newline

च॒ । वा॒जिन᳚म् ॥ सोम᳚म् । राजा॑नम् । वरु॑णम् । अ॒ग्निम् । अ॒न्वार॑भामह॒ इत्य॑नु - आर॑भामहे ॥ आ॒दि॒त्यान् । विष्णु᳚म् । सूर्य᳚म् । ब्र॒ह्माण᳚म् । च॒ । बृह॒स्पति᳚म् ॥ दे॒वस्य॑ । त्वा॒ । स॒वि॒तुः । प्र॒स॒व इति॑ प्र - स॒वे । अ॒श्विनोः᳚ । बा॒हुभ्या॒मिति॑ बा॒हु - भ्या॒म् । पू॒ष्णः । हस्ता᳚भ्याम् । सर॑स्वत्यै । वा॒चः । य॒न्तुः । य॒न्त्रेण॑ । अ॒ग्नेः । त्वा॒ । साम्रा᳚ज्ये॒नेति॒ सां - रा॒ज्ये॒न॒ । अ॒भीति॑ । सि॒ञ्चा॒मि॒ । इन्द्र॑स्य । बृह॒स्पतेः᳚ । त्वा॒ । साम्रा᳚ज्ये॒नेति॒ सां - रा॒ज्ये॒न॒ । अ॒भीति॑ । सि॒ञ्चा॒मि॒ ॥  \newline


\textbf{Krama Paata} \newline

च॒ वा॒जिन᳚म् । वा॒जिन॒मिति॑ वा॒जिन᳚म् ॥ सोमꣳ॒॒ राजा॑नम् । राजा॑नं॒ ॅवरु॑णम् । वरु॑णम॒ग्निम् । अ॒ग्निम॒न्वार॑भामहे । अ॒न्वार॑भामह॒ इत्य॑नु - आर॑भामहे ॥ आ॒दि॒त्यान्. विष्णु᳚म् । विष्णुꣳ॒॒ सूर्य᳚म् । सूर्य॑म् ब्र॒ह्माण᳚म् । ब्र॒ह्माण॑म् च । च॒ बृह॒स्पति᳚म् । बृह॒स्पति॒मिति॒ बृह॒स्पति᳚म् ॥ दे॒वस्य॑ त्वा । त्वा॒ स॒वि॒तुः । स॒वि॒तुः प्र॑स॒वे । प्र॒स॒वे᳚ऽश्विनोः᳚ । प्र॒स॒व इति॑ प्र - स॒वे । अ॒श्विनो᳚र्,बा॒हुभ्या᳚म् । बा॒हुभ्या᳚म् पू॒ष्णः । बा॒हुभ्या॒मिति॑ बा॒हु - भ्या॒म् । पू॒ष्णो हस्ता᳚भ्याम् । हस्ता᳚भ्याꣳ॒॒ सर॑स्वत्यै । सर॑स्वत्यै वा॒चः । वा॒चो य॒न्तुः । य॒न्तुर् य॒न्त्रेण॑ । य॒न्त्रेणा॒ग्नेः । अ॒ग्नेस्त्वा᳚ । त्वा॒ साम्रा᳚ज्येन । साम्रा᳚ज्येना॒भि । साम्रा᳚ज्ये॒नेति॒ साम् - रा॒ज्ये॒न॒ । अ॒भि षि॑ञ्चामि । सि॒ञ्चा॒मीन्द्र॑स्य । इन्द्र॑स्य॒ बृह॒स्पतेः᳚ । बृह॒स्पते᳚स्त्वा । त्वा॒ साम्रा᳚ज्येन । साम्रा᳚ज्येना॒भि । साम्रा᳚ज्ये॒नेति॒ साम् - रा॒ज्ये॒न॒ । अ॒भि षि॑ञ्चामि । सि॒ञ्चा॒मीति॑ सिञ्चामि । \newline

\textbf{Jatai Paata} \newline

1. च॒ वा॒जिनं॑ ॅवा॒जिन॑म् च च वा॒जिन᳚म् । \newline
2. वा॒जिन॒मिति॑ वा॒जिन᳚म् । \newline
3. सोमꣳ॒॒ राजा॑नꣳ॒॒ राजा॑नꣳ॒॒ सोमꣳ॒॒ सोमꣳ॒॒ राजा॑नम् । \newline
4. राजा॑नं॒ ॅवरु॑णं॒ ॅवरु॑णꣳ॒॒ राजा॑नꣳ॒॒ राजा॑नं॒ ॅवरु॑णम् । \newline
5. वरु॑ण म॒ग्नि म॒ग्निं ॅवरु॑णं॒ ॅवरु॑ण म॒ग्निम् । \newline
6. अ॒ग्नि म॒न्वार॑भामहे अ॒न्वार॑भामहे अ॒ग्नि म॒ग्नि म॒न्वार॑भामहे । \newline
7. अ॒न्वार॑भामह॒ इत्य॑नु - आर॑भामहे । \newline
8. आ॒दि॒त्यान्. विष्णुं॒ ॅविष्णु॑ मादि॒त्या ना॑दि॒त्यान्. विष्णु᳚म् । \newline
9. विष्णुꣳ॒॒ सूर्यꣳ॒॒ सूर्यं॒ ॅविष्णुं॒ ॅविष्णुꣳ॒॒ सूर्य᳚म् । \newline
10. सूर्य॑म् ब्र॒ह्माण॑म् ब्र॒ह्माणꣳ॒॒ सूर्यꣳ॒॒ सूर्य॑म् ब्र॒ह्माण᳚म् । \newline
11. ब्र॒ह्माण॑म् च च ब्र॒ह्माण॑म् ब्र॒ह्माण॑म् च । \newline
12. च॒ बृह॒स्पति॒म् बृह॒स्पति॑म् च च॒ बृह॒स्पति᳚म् । \newline
13. बृह॒स्पति॒मिति॒ बृह॒स्पति᳚म् । \newline
14. दे॒वस्य॑ त्वा त्वा दे॒वस्य॑ दे॒वस्य॑ त्वा । \newline
15. त्वा॒ स॒वि॒तुः स॑वि॒तु स्त्वा᳚ त्वा सवि॒तुः । \newline
16. स॒वि॒तुः प्र॑स॒वे प्र॑स॒वे स॑वि॒तुः स॑वि॒तुः प्र॑स॒वे । \newline
17. प्र॒स॒वे᳚ ऽश्विनो॑ र॒श्विनोः᳚ प्रस॒वे प्र॑स॒वे᳚ ऽश्विनोः᳚ । \newline
18. प्र॒स॒व इति॑ प्र - स॒वे । \newline
19. अ॒श्विनो᳚र् बा॒हुभ्या᳚म् बा॒हुभ्या॑ म॒श्विनो॑ र॒श्विनो᳚र् बा॒हुभ्या᳚म् । \newline
20. बा॒हुभ्या᳚म् पू॒ष्णः पू॒ष्णो बा॒हुभ्या᳚म् बा॒हुभ्या᳚म् पू॒ष्णः । \newline
21. बा॒हुभ्या॒मिति॑ बा॒हु - भ्या॒म् । \newline
22. पू॒ष्णो हस्ता᳚भ्याꣳ॒॒ हस्ता᳚भ्याम् पू॒ष्णः पू॒ष्णो हस्ता᳚भ्याम् । \newline
23. हस्ता᳚भ्याꣳ॒॒ सर॑स्वत्यै॒ सर॑स्वत्यै॒ हस्ता᳚भ्याꣳ॒॒ हस्ता᳚भ्याꣳ॒॒ सर॑स्वत्यै । \newline
24. सर॑स्वत्यै वा॒चो वा॒चः सर॑स्वत्यै॒ सर॑स्वत्यै वा॒चः । \newline
25. वा॒चो य॒न्तुर् य॒न्तुर् वा॒चो वा॒चो य॒न्तुः । \newline
26. य॒न्तुर् य॒न्त्रेण॑ य॒न्त्रेण॑ य॒न्तुर् य॒न्तुर् य॒न्त्रेण॑ । \newline
27. य॒न्त्रेणा॒ग्ने र॒ग्नेर् य॒न्त्रेण॑ य॒न्त्रेणा॒ग्नेः । \newline
28. अ॒ग्ने स्त्वा᳚ त्वा॒ ऽग्ने र॒ग्ने स्त्वा᳚ । \newline
29. त्वा॒ साम्रा᳚ज्येन॒ साम्रा᳚ज्येन त्वा त्वा॒ साम्रा᳚ज्येन । \newline
30. साम्रा᳚ज्येना॒भ्य॑भि साम्रा᳚ज्येन॒ साम्रा᳚ज्येना॒भि । \newline
31. साम्रा᳚ज्ये॒नेति॒ सां - रा॒ज्ये॒न॒ । \newline
32. अ॒भि षि॑ञ्चामि सिञ्चा म्य॒भ्य॑भि षि॑ञ्चामि । \newline
33. सि॒ञ्चा॒ मीन्द्र॒स्ये न्द्र॑स्य सिञ्चामि सिञ्चा॒ मीन्द्र॑स्य । \newline
34. इन्द्र॑स्य॒ बृह॒स्पते॒र् बृह॒स्पते॒ रिन्द्र॒स्ये न्द्र॑स्य॒ बृह॒स्पतेः᳚ । \newline
35. बृह॒स्पते᳚ स्त्वा त्वा॒ बृह॒स्पते॒र् बृह॒स्पते᳚ स्त्वा । \newline
36. त्वा॒ साम्रा᳚ज्येन॒ साम्रा᳚ज्येन त्वा त्वा॒ साम्रा᳚ज्येन । \newline
37. साम्रा᳚ज्ये ना॒भ्य॑भि साम्रा᳚ज्येन॒ साम्रा᳚ज्ये ना॒भि । \newline
38. साम्रा᳚ज्ये॒नेति॒ सां - रा॒ज्ये॒न॒ । \newline
39. अ॒भि षि॑ञ्चामि सिञ्चा म्य॒भ्य॑भि षि॑ञ्चामि । \newline
40. सि॒ञ्चा॒मीति॑ सिञ्चामि । \newline

\textbf{Ghana Paata } \newline

1. च॒ वा॒जिनं॑ ॅवा॒जिन॑म् च च वा॒जिन᳚म् । \newline
2. वा॒जिन॒मिति॑ वा॒जिन᳚म् । \newline
3. सोम॒(ग्म्॒) राजा॑न॒(ग्म्॒) राजा॑न॒(ग्म्॒) सोम॒(ग्म्॒) सोम॒(ग्म्॒) राजा॑नं॒ ॅवरु॑णं॒ ॅवरु॑ण॒(ग्म्॒) राजा॑न॒(ग्म्॒) सोम॒(ग्म्॒) सोम॒(ग्म्॒) राजा॑नं॒ ॅवरु॑णम् । \newline
4. राजा॑नं॒ ॅवरु॑णं॒ ॅवरु॑ण॒(ग्म्॒) राजा॑न॒(ग्म्॒) राजा॑नं॒ ॅवरु॑ण म॒ग्नि म॒ग्निं ॅवरु॑ण॒(ग्म्॒) राजा॑न॒(ग्म्॒) राजा॑नं॒ ॅवरु॑ण म॒ग्निम् । \newline
5. वरु॑ण म॒ग्नि म॒ग्निं ॅवरु॑णं॒ ॅवरु॑ण म॒ग्नि म॒न्वार॑भामहे अ॒न्वार॑भामहे अ॒ग्निं ॅवरु॑णं॒ ॅवरु॑ण म॒ग्नि म॒न्वार॑भामहे । \newline
6. अ॒ग्नि म॒न्वार॑भामहे अ॒न्वार॑भामहे अ॒ग्नि म॒ग्नि म॒न्वार॑भामहे । \newline
7. अ॒न्वार॑भामह॒ इत्य॑नु - आर॑भामहे । \newline
8. आ॒दि॒त्यान्. विष्णुं॒ ॅविष्णु॑ मादि॒त्या ना॑दि॒त्यान्. विष्णु॒(ग्म्॒) सूर्य॒(ग्म्॒) सूर्यं॒ ॅविष्णु॑ मादि॒त्या ना॑दि॒त्यान्. विष्णु॒(ग्म्॒) सूर्य᳚म् । \newline
9. विष्णु॒(ग्म्॒) सूर्य॒(ग्म्॒) सूर्यं॒ ॅविष्णुं॒ ॅविष्णु॒(ग्म्॒) सूर्य॑म् ब्र॒ह्माण॑म् ब्र॒ह्माण॒(ग्म्॒) सूर्यं॒ ॅविष्णुं॒ ॅविष्णु॒(ग्म्॒) सूर्य॑म् ब्र॒ह्माण᳚म् । \newline
10. सूर्य॑म् ब्र॒ह्माण॑म् ब्र॒ह्माण॒(ग्म्॒) सूर्य॒(ग्म्॒) सूर्य॑म् ब्र॒ह्माण॑म् च च ब्र॒ह्माण॒(ग्म्॒) सूर्य॒(ग्म्॒) सूर्य॑म् ब्र॒ह्माण॑म् च । \newline
11. ब्र॒ह्माण॑म् च च ब्र॒ह्माण॑म् ब्र॒ह्माण॑म् च॒ बृह॒स्पति॒म् बृह॒स्पति॑म् च ब्र॒ह्माण॑म् ब्र॒ह्माण॑म् च॒ बृह॒स्पति᳚म् । \newline
12. च॒ बृह॒स्पति॒म् बृह॒स्पति॑म् च च॒ बृह॒स्पति᳚म् । \newline
13. बृह॒स्पति॒मिति॒ बृह॒स्पति᳚म् । \newline
14. दे॒वस्य॑ त्वा त्वा दे॒वस्य॑ दे॒वस्य॑ त्वा सवि॒तुः स॑वि॒तु स्त्वा॑ दे॒वस्य॑ दे॒वस्य॑ त्वा सवि॒तुः । \newline
15. त्वा॒ स॒वि॒तुः स॑वि॒तु स्त्वा᳚ त्वा सवि॒तुः प्र॑स॒वे प्र॑स॒वे स॑वि॒तु स्त्वा᳚ त्वा सवि॒तुः प्र॑स॒वे । \newline
16. स॒वि॒तुः प्र॑स॒वे प्र॑स॒वे स॑वि॒तुः स॑वि॒तुः प्र॑स॒वे᳚ ऽश्विनो॑ र॒श्विनोः᳚ प्रस॒वे स॑वि॒तुः स॑वि॒तुः प्र॑स॒वे᳚ ऽश्विनोः᳚ । \newline
17. प्र॒स॒वे᳚ ऽश्विनो॑ र॒श्विनोः᳚ प्रस॒वे प्र॑स॒वे᳚ ऽश्विनो᳚र् बा॒हुभ्या᳚म् बा॒हुभ्या॑ म॒श्विनोः᳚ प्रस॒वे प्र॑स॒वे᳚ ऽश्विनो᳚र् बा॒हुभ्या᳚म् । \newline
18. प्र॒स॒व इति॑ प्र - स॒वे । \newline
19. अ॒श्विनो᳚र् बा॒हुभ्या᳚म् बा॒हुभ्या॑ म॒श्विनो॑ र॒श्विनो᳚र् बा॒हुभ्या᳚म् पू॒ष्णः पू॒ष्णो बा॒हुभ्या॑ म॒श्विनो॑ र॒श्विनो᳚र् बा॒हुभ्या᳚म् पू॒ष्णः । \newline
20. बा॒हुभ्या᳚म् पू॒ष्णः पू॒ष्णो बा॒हुभ्या᳚म् बा॒हुभ्या᳚म् पू॒ष्णो हस्ता᳚भ्या॒(ग्म्॒) हस्ता᳚भ्याम् पू॒ष्णो बा॒हुभ्या᳚म् बा॒हुभ्या᳚म् पू॒ष्णो हस्ता᳚भ्याम् । \newline
21. बा॒हुभ्या॒मिति॑ बा॒हु - भ्या॒म् । \newline
22. पू॒ष्णो हस्ता᳚भ्या॒(ग्म्॒) हस्ता᳚भ्याम् पू॒ष्णः पू॒ष्णो हस्ता᳚भ्या॒(ग्म्॒) सर॑स्वत्यै॒ सर॑स्वत्यै॒ हस्ता᳚भ्याम् पू॒ष्णः पू॒ष्णो हस्ता᳚भ्या॒(ग्म्॒) सर॑स्वत्यै । \newline
23. हस्ता᳚भ्या॒(ग्म्॒) सर॑स्वत्यै॒ सर॑स्वत्यै॒ हस्ता᳚भ्या॒(ग्म्॒) हस्ता᳚भ्या॒(ग्म्॒) सर॑स्वत्यै वा॒चो वा॒चः सर॑स्वत्यै॒ हस्ता᳚भ्या॒(ग्म्॒) हस्ता᳚भ्या॒(ग्म्॒) सर॑स्वत्यै वा॒चः । \newline
24. सर॑स्वत्यै वा॒चो वा॒चः सर॑स्वत्यै॒ सर॑स्वत्यै वा॒चो य॒न्तुर् य॒न्तुर् वा॒चः सर॑स्वत्यै॒ सर॑स्वत्यै वा॒चो य॒न्तुः । \newline
25. वा॒चो य॒न्तुर् य॒न्तुर् वा॒चो वा॒चो य॒न्तुर् य॒न्त्रेण॑ य॒न्त्रेण॑ य॒न्तुर् वा॒चो वा॒चो य॒न्तुर् य॒न्त्रेण॑ । \newline
26. य॒न्तुर् य॒न्त्रेण॑ य॒न्त्रेण॑ य॒न्तुर् य॒न्तुर् य॒न्त्रेणा॒ग्ने र॒ग्नेर् य॒न्त्रेण॑ य॒न्तुर् य॒न्तुर् य॒न्त्रेणा॒ग्नेः । \newline
27. य॒न्त्रेणा॒ग्ने र॒ग्नेर् य॒न्त्रेण॑ य॒न्त्रेणा॒ग्ने स्त्वा᳚ त्वा॒ ऽग्नेर् य॒न्त्रेण॑ य॒न्त्रेणा॒ग्ने स्त्वा᳚ । \newline
28. अ॒ग्नेस्त्वा᳚ त्वा॒ ऽग्ने र॒ग्ने स्त्वा॒ साम्रा᳚ज्येन॒ साम्रा᳚ज्येन त्वा॒ ऽग्नेर॒ग्ने स्त्वा॒ साम्रा᳚ज्येन । \newline
29. त्वा॒ साम्रा᳚ज्येन॒ साम्रा᳚ज्येन त्वा त्वा॒ साम्रा᳚ज्येना॒भ्य॑भि साम्रा᳚ज्येन त्वा त्वा॒ साम्रा᳚ज्येना॒भि । \newline
30. साम्रा᳚ज्येना॒भ्य॑भि साम्रा᳚ज्येन॒ साम्रा᳚ज्येना॒भि षि॑ञ्चामि सिञ्चाम्य॒भि साम्रा᳚ज्येन॒ साम्रा᳚ज्येना॒भि षि॑ञ्चामि । \newline
31. साम्रा᳚ज्ये॒नेति॒ सां - रा॒ज्ये॒न॒ । \newline
32. अ॒भि षि॑ञ्चामि सिञ्चाम्य॒भ्य॑भि षि॑ञ्चा॒ मीन्द्र॒स्ये न्द्र॑स्य सिञ्चाम्य॒ भ्य॑भि षि॑ञ्चा॒ मीन्द्र॑स्य । \newline
33. सि॒ञ्चा॒ मीन्द्र॒स्ये न्द्र॑स्य सिञ्चामि सिञ्चा॒ मीन्द्र॑स्य॒ बृह॒स्पते॒र् बृह॒स्पते॒ रिन्द्र॑स्य सिञ्चामि सिञ्चा॒ मीन्द्र॑स्य॒ बृह॒स्पतेः᳚ । \newline
34. इन्द्र॑स्य॒ बृह॒स्पते॒र् बृह॒स्पते॒ रिन्द्र॒स्ये न्द्र॑स्य॒ बृह॒स्पते᳚ स्त्वा त्वा॒ बृह॒स्पते॒ रिन्द्र॒स्ये न्द्र॑स्य॒ बृह॒स्पते᳚ स्त्वा । \newline
35. बृह॒स्पते᳚ स्त्वा त्वा॒ बृह॒स्पते॒र् बृह॒स्पते᳚ स्त्वा॒ साम्रा᳚ज्येन॒ साम्रा᳚ज्येन त्वा॒ बृह॒स्पते॒र् बृह॒स्पते᳚ स्त्वा॒ साम्रा᳚ज्येन । \newline
36. त्वा॒ साम्रा᳚ज्येन॒ साम्रा᳚ज्येन त्वा त्वा॒ साम्रा᳚ज्येना॒ भ्य॑भि साम्रा᳚ज्येन त्वा त्वा॒ साम्रा᳚ज्येना॒भि । \newline
37. साम्रा᳚ज्येना॒ भ्य॑भि साम्रा᳚ज्येन॒ साम्रा᳚ज्येना॒भि षि॑ञ्चामि सिञ्चाम्य॒भि साम्रा᳚ज्येन॒ साम्रा᳚ज्येना॒भि षि॑ञ्चामि । \newline
38. साम्रा᳚ज्ये॒नेति॒ सां - रा॒ज्ये॒न॒ । \newline
39. अ॒भि षि॑ञ्चामि सिञ्चाम्य॒ भ्य॑भि षि॑ञ्चामि । \newline
40. सि॒ञ्चा॒मीति॑ सिञ्चामि । \newline
\pagebreak
\markright{ TS 1.7.11.1  \hfill https://www.vedavms.in \hfill}
\addcontentsline{toc}{section}{ TS 1.7.11.1 }
\section*{ TS 1.7.11.1 }

\textbf{TS 1.7.11.1 } \newline
\textbf{Samhita Paata} \newline

अ॒ग्निरेका᳚क्षरेण॒ वाच॒मुद॑जयद॒श्विनौ॒ द्व्य॑क्षरेण प्राणापा॒नावुद॑जयतां॒ ॅविष्णु॒स्त्य्र॑क्षरेण॒ त्रीन् ॅलो॒कानुद॑जय॒थ् सोम॒श्चतु॑रक्षरेण॒ चतु॑ष्पदः प॒शूनुद॑जयत् पू॒षा पञ्चा᳚क्षरेण प॒ङ्क्तिमुद॑जयद् धा॒ता षड॑क्षरेण॒ षड्-ऋ॒तूनुद॑जयन् म॒रुतः॑ स॒प्ताक्ष॑रेण स॒प्तप॑दाꣳ॒॒ शक्व॑री॒मुद॑जय॒न् बृह॒स्पति॑-र॒ष्टाक्ष॑रेण गाय॒त्रीमुद॑जयन् मि॒त्रो नवा᳚क्षरेण त्रि॒वृतꣳ॒॒ स्तोम॒मुद॑जय॒द्-[ ] \newline

\textbf{Pada Paata} \newline

अ॒ग्निः । एका᳚क्षरे॒णेत्येक॑ -अ॒क्ष॒रे॒ण॒ । वाच᳚म् । उदिति॑ । अ॒ज॒य॒त् । अ॒श्विनौ᳚ । द्व्य॑क्षरे॒णेति॒ द्वि - अ॒क्ष॒रे॒ण॒ । प्रा॒णा॒पा॒नाविति॑ प्राण- अ॒पा॒नौ । उदिति॑ । अ॒ज॒य॒ता॒म् । विष्णुः॑ । त्र्य॑क्षरे॒णेति॒ त्रि - अ॒क्ष॒रे॒ण॒ । त्रीन् । लो॒कान् । उदिति॑ । अ॒ज॒य॒त् । सोमः॑ । चतु॑रक्षरे॒णेति॒ चतुः॑ - अ॒क्ष॒रे॒ण॒ । चतु॑ष्पद॒ इति॒ चतुः॑ - प॒दः॒ । प॒शून् । उदिति॑ । अ॒ज॒य॒त् । पू॒षा । पञ्चा᳚क्षरे॒णेति॒ पञ्च॑ - अ॒क्ष॒रे॒ण॒ । प॒ङ्क्तिम् । उदिति॑ । अ॒ज॒य॒त् । धा॒ता । षड॑क्षरे॒णेति॒ षट् - अ॒क्ष॒रे॒ण॒ । षट् । ऋ॒तून् । उदिति॑ । अ॒ज॒य॒त् । म॒रुतः॑ । स॒प्ताक्ष॑रे॒णेति॑ स॒प्त - अ॒क्ष॒रे॒ण॒ । स॒प्तप॑दा॒मिति॑ स॒प्त - प॒दा॒म् । शक्व॑रीम् । उदिति॑ । अ॒ज॒य॒न्न् । बृह॒स्पतिः॑ । अ॒ष्टाक्ष॑रे॒णेत्य॒ष्टा - अ॒क्ष॒रे॒ण॒ । गा॒य॒त्रीम् । उदिति॑ । अ॒ज॒य॒त् । मि॒त्रः । नवा᳚क्षरे॒णेति॒ नव॑ - अ॒क्ष॒रे॒ण॒ । त्रि॒वृत॒मिति॑ त्रि - वृत᳚म् । स्तोम᳚म् । उदिति॑ । अ॒ज॒य॒त् ।  \newline


\textbf{Krama Paata} \newline

अ॒ग्निरेका᳚क्षरेण । एका᳚क्षरेण॒ वाच᳚म् । एका᳚क्षरे॒णेत्येक॑ - अ॒क्ष॒रे॒ण॒ । वाच॒मुत् । उद॑जयत् । अ॒ज॒य॒द॒श्विनौ᳚ । अ॒श्विनौ॒ द्व्य॑क्षरेण । द्व्य॑क्षरेण प्राणापा॒नौ । द्व्य॑क्षरे॒णेति॒ द्वि - अ॒क्ष॒रे॒ण॒ । प्रा॒णा॒पा॒नावुत् । प्रा॒णा॒पा॒नाविति॑ प्राण - अ॒पा॒नौ । उद॑जयताम् । अ॒ज॒य॒तां॒ ॅविष्णुः॑ । विष्णुः॒ त्र्य॑क्षरेण । त्र्य॑क्षरेण॒ त्रीन् । त्र्य॑क्षरे॒णेति॒ त्रि - अ॒क्ष॒रे॒ण॒ । त्रीन् ॅलो॒कान् । लो॒कानुत् । उद॑जयत् । अ॒ज॒य॒थ् सोमः॑ । सोम॒श्चतु॑रक्षरेण । चतु॑रक्षरेण॒ चतु॑ष्पदः । चतु॑रक्षरे॒णेति॒ चतुः॑ - अ॒क्ष॒रे॒ण॒ । चतु॑ष्पदः प॒शून् । चतु॑ष्पद॒ इति॒ चतुः॑ - प॒दः॒ । प॒शूनुत् । उद॑जयत् । अ॒ज॒य॒त् पू॒षा । पू॒षा पञ्चा᳚क्षरेण । पञ्चा᳚क्षरेण प॒ङ्क्तिम् । पञ्चा᳚क्षरे॒णेति॒ पञ्च॑ - अ॒क्ष॒रे॒ण॒ । प॒ङ्क्तिमुत् । उद॑जयत् । अ॒ज॒य॒द् धा॒ता । धा॒ता षड॑क्षरेण । षड॑क्षरेण॒ षट् । षड॑क्षरे॒णेति॒ षट् - अ॒क्ष॒रे॒ण॒ । षडृ॒तून् । ऋ॒तूनुत् । उद॑जयत् । अ॒ज॒य॒न्म॒रुतः॑ । म॒रुतः॑ स॒प्ताक्ष॑रेण । स॒प्ताक्ष॑रेण स॒प्तप॑दाम् । स॒प्ताक्ष॑रे॒णेति॑ स॒प्त - अ॒क्ष॒रे॒ण॒ । स॒प्तप॑दाꣳ॒॒ शक्व॑रीम् । स॒प्तप॑दा॒मिति॑ स॒प्त - प॒दा॒म् । शक्व॑री॒मुत् । उद॑जयन्न् । अ॒ज॒य॒न्,बृह॒स्पतिः॑ । बृह॒स्पति॑र॒ष्टाक्ष॑रेण । अ॒ष्टाक्ष॑रेण गाय॒त्रीम् । अ॒ष्टाक्ष॑रे॒णेत्य॒ष्टा - अ॒क्ष॒रे॒ण॒ । गा॒य॒त्रीमुत् । उद॑जयत् । अ॒ज॒य॒न्मि॒त्रः । मि॒त्रो नवा᳚क्षरेण । नवा᳚क्षरेण त्रि॒वृत᳚म् । नवा᳚क्षरे॒णेति॒ नव॑ - अ॒क्ष॒रे॒ण॒ । त्रि॒वृतꣳ॒॒ स्तोम᳚म् । त्रि॒वृत॒मिति॑ त्रि - वृत᳚म् । स्तोम॒मुत् । उद॑जयत् । अ॒ज॒य॒द् वरु॑णः \newline

\textbf{Jatai Paata} \newline

1. अ॒ग्नि रेका᳚क्षरे॒ णैका᳚क्षरेणा॒ ग्नि र॒ग्निरेका᳚क्षरेण । \newline
2. एका᳚क्षरेण॒ वाचं॒ ॅवाच॒ मेका᳚क्षरे॒ णैका᳚क्षरेण॒ वाच᳚म् । \newline
3. एका᳚क्षरे॒णेत्येक॑ - अ॒क्ष॒रे॒ण॒ । \newline
4. वाच॒ मुदुद् वाचं॒ ॅवाच॒ मुत् । \newline
5. उ द॑जय दजय॒ दुदु द॑जयत् । \newline
6. अ॒ज॒य॒ द॒श्विना॑ व॒श्विना॑ वजय दजय द॒श्विनौ᳚ । \newline
7. अ॒श्विनौ॒ द्व्य॑क्षरेण॒ द्व्य॑क्षरेणा॒ श्विना॑ व॒श्विनौ॒ द्व्य॑क्षरेण । \newline
8. द्व्य॑क्षरेण प्राणापा॒नौ प्रा॑णापा॒नौ द्व्य॑क्षरेण॒ द्व्य॑क्षरेण प्राणापा॒नौ । \newline
9. द्व्य॑क्षरे॒णेति॒ द्वि - अ॒क्ष॒रे॒ण॒ । \newline
10. प्रा॒णा॒पा॒ना वुदुत् प्रा॑णापा॒नौ प्रा॑णापा॒ना वुत् । \newline
11. प्रा॒णा॒पा॒नाविति॑ प्राण - अ॒पा॒नौ । \newline
12. उद॑जयता मजयता॒ मुदु द॑जयताम् । \newline
13. अ॒ज॒य॒तां॒ ॅविष्णु॒र् विष्णु॑ रजयता मजयतां॒ ॅविष्णुः॑ । \newline
14. विष्णु॒ स्त्र्य॑क्षरेण॒ त्र्य॑क्षरे॒ण विष्णु॒र् विष्णु॒ स्त्र्य॑क्षरेण । \newline
15. त्र्य॑क्षरेण॒ त्रीꣳ स्त्रीꣳ स्त्र्य॑क्षरेण॒ त्र्य॑क्षरेण॒ त्रीन् । \newline
16. त्र्य॑क्षरे॒णेति॒ त्रि - अ॒क्ष॒रे॒ण॒ । \newline
17. त्रीन् ॅलो॒कान् ॅलो॒काꣳ स्त्रीꣳ स्त्रीन् ॅलो॒कान् । \newline
18. लो॒का नुदु ल्लो॒कान् ॅलो॒का नुत् । \newline
19. उद॑जय दजय॒ दुदु द॑जयत् । \newline
20. अ॒ज॒य॒थ् सोमः॒ सोमो॑ अजय दजय॒थ् सोमः॑ । \newline
21. सोम॒ श्चतु॑रक्षरेण॒ चतु॑रक्षरेण॒ सोमः॒ सोम॒ श्चतु॑रक्षरेण । \newline
22. चतु॑रक्षरेण॒ चतु॑ष्पद॒ श्चतु॑ष्पद॒ श्चतु॑रक्षरेण॒ चतु॑रक्षरेण॒ चतु॑ष्पदः । \newline
23. चतु॑रक्षरे॒णेति॒ चतुः॑ - अ॒क्ष॒रे॒ण॒ । \newline
24. चतु॑ष्पदः प॒शून् प॒शूꣳश् चतु॑ष्पद॒ श्चतु॑ष्पदः प॒शून् । \newline
25. चतु॑ष्पद॒ इति॒ चतुः॑ - प॒दः॒ । \newline
26. प॒शू नुदुत् प॒शून् प॒शू नुत् । \newline
27. उद॑जय दजय॒ दुदु द॑जयत् । \newline
28. अ॒ज॒य॒त् पू॒षा पू॒षा ऽज॑य दजयत् पू॒षा । \newline
29. पू॒षा पञ्चा᳚क्षरेण॒ पञ्चा᳚क्षरेण पू॒षा पू॒षा पञ्चा᳚क्षरेण । \newline
30. पञ्चा᳚क्षरेण प॒ङ्क्तिम् प॒ङ्क्तिम् पञ्चा᳚क्षरेण॒ 
पञ्चा᳚क्षरेण प॒ङ्क्तिम् । \newline
31. पञ्चा᳚क्षरे॒णेति॒ पञ्च॑ - अ॒क्ष॒रे॒ण॒ । \newline
32. प॒ङ्क्ति मुदुत् प॒ङ्क्तिम् प॒ङ्क्ति मुत् । \newline
33. उद॑जय दजय॒ दुदु द॑जयत् । \newline
34. अ॒ज॒य॒द् धा॒ता धा॒ता ऽज॑य दजयद् धा॒ता । \newline
35. धा॒ता षड॑क्षरेण॒ षड॑क्षरेण धा॒ता धा॒ता षड॑क्षरेण । \newline
36. षड॑क्षरेण॒ षट् थ्षट् थ्षड॑क्षरेण॒ षड॑क्षरेण॒ षट् । \newline
37. षड॑क्षरे॒णेति॒ षट् - अ॒क्ष॒रे॒ण॒ । \newline
38. षडृ॒तू नृ॒तून् षट् थ् षडृ॒तून् । \newline
39. ऋ॒तू नुदुदृ॒तू नृ॒तू नुत् । \newline
40. उद॑जय दजय॒ दुदु द॑जयत् । \newline
41. अ॒ज॒य॒न् म॒रुतो॑ म॒रुतो॑ अजय दजयन् म॒रुतः॑ । \newline
42. म॒रुतः॑ स॒प्ताक्ष॑रेण स॒प्ताक्ष॑रेण म॒रुतो॑ म॒रुतः॑ स॒प्ताक्ष॑रेण । \newline
43. स॒प्ताक्ष॑रेण स॒प्तप॑दाꣳ स॒प्तप॑दाꣳ स॒प्ताक्ष॑रेण स॒प्ताक्ष॑रेण स॒प्तप॑दाम् । \newline
44. स॒प्ताक्ष॑रे॒णेति॑ स॒प्त - अ॒क्ष॒रे॒ण॒ । \newline
45. स॒प्तप॑दाꣳ॒॒ शक्व॑रीꣳ॒॒ शक्व॑रीꣳ स॒प्तप॑दाꣳ स॒प्तप॑दाꣳ॒॒ शक्व॑रीम् । \newline
46. स॒प्तप॑दा॒मिति॑ स॒प्त - प॒दा॒म् । \newline
47. शक्व॑री॒ मुदुच् छक्व॑रीꣳ॒॒ शक्व॑री॒ मुत् । \newline
48. उद॑जयन् नजय॒न् नुदुद॑जयन्न् । \newline
49. अ॒ज॒य॒न् बृह॒स्पति॒र् बृह॒स्पति॑ रजयन् नजय॒न् बृह॒स्पतिः॑ । \newline
50. बृह॒स्पति॑ र॒ष्टाक्ष॑रेणा॒ ष्टाक्ष॑रेण॒ बृह॒स्पति॒र् बृह॒स्पति॑ र॒ष्टाक्ष॑रेण । \newline
51. अ॒ष्टाक्ष॑रेण गाय॒त्रीम् गा॑य॒त्री म॒ष्टाक्ष॑रेणा॒ ष्टाक्ष॑रेण गाय॒त्रीम् । \newline
52. अ॒ष्टाक्ष॑रे॒णेत्य॒ष्टा - अ॒क्ष॒रे॒ण॒ । \newline
53. गा॒य॒त्री मुदुद् गा॑य॒त्रीम् गा॑य॒त्री मुत् । \newline
54. उद॑जय दजय॒ दुदु द॑जयत् । \newline
55. अ॒ज॒य॒न् मि॒त्रो मि॒त्रो अ॑जय दजयन् मि॒त्रः । \newline
56. मि॒त्रो नवा᳚क्षरेण॒ नवा᳚क्षरेण मि॒त्रो मि॒त्रो नवा᳚क्षरेण । \newline
57. नवा᳚क्षरेण त्रि॒वृत॑म् त्रि॒वृत॒न् नवा᳚क्षरेण॒ नवा᳚क्षरेण त्रि॒वृत᳚म् । \newline
58. नवा᳚क्षरे॒णेति॒ नव॑ - अ॒क्ष॒रे॒ण॒ । \newline
59. त्रि॒वृतꣳ॒॒ स्तोमꣳ॒॒ स्तोम॑म् त्रि॒वृत॑म् त्रि॒वृतꣳ॒॒ स्तोम᳚म् । \newline
60. त्रि॒वृत॒मिति॑ त्रि - वृत᳚म् । \newline
61. स्तोम॒ मुदुथ् स्तोमꣳ॒॒ स्तोम॒ मुत् । \newline
62. उद॑जय दजय॒ दुदु द॑जयत् । \newline
63. अ॒ज॒य॒द् वरु॑णो॒ वरु॑णो अजय दजय॒द् वरु॑णः । \newline

\textbf{Ghana Paata } \newline

1. अ॒ग्नि रेका᳚क्षरे॒ णैका᳚क्षरेणा॒ ग्नि र॒ग्नि रेका᳚क्षरेण॒ वाचं॒ ॅवाच॒ मेका᳚क्षरेणा॒ ग्नि र॒ग्नि रेका᳚क्षरेण॒ वाच᳚म् । \newline
2. एका᳚क्षरेण॒ वाचं॒ ॅवाच॒ मेका᳚क्षरे॒ णैका᳚क्षरेण॒ वाच॒ मुदुद् वाच॒ मेका᳚क्षरे॒ णैका᳚क्षरेण॒ वाच॒ मुत् । \newline
3. एका᳚क्षरे॒णेत्येक॑ - अ॒क्ष॒रे॒ण॒ । \newline
4. वाच॒ मुदुद् वाचं॒ ॅवाच॒ मुद॑जय दजय॒ दुद् वाचं॒ ॅवाच॒ मुद॑जयत् । \newline
5. उद॑जय दजय॒ दुदु द॑जय द॒श्विना॑ व॒श्विना॑ वजय॒ दुदु द॑जय द॒श्विनौ᳚ । \newline
6. अ॒ज॒य॒ द॒श्विना॑ व॒श्विना॑ वजय दजय द॒श्विनौ॒ द्व्य॑क्षरेण॒ द्व्य॑क्षरेणा॒ श्विना॑ वजयदजय द॒श्विनौ॒ द्व्य॑क्षरेण । \newline
7. अ॒श्विनौ॒ द्व्य॑क्षरेण॒ द्व्य॑क्षरेणा॒ श्विना॑ व॒श्विनौ॒ द्व्य॑क्षरेण प्राणापा॒नौ प्रा॑णापा॒नौ द्व्य॑क्षरेणा॒ 
श्विना॑ व॒श्विनौ॒ द्व्य॑क्षरेण प्राणापा॒नौ । \newline
8. द्व्य॑क्षरेण प्राणापा॒नौ प्रा॑णापा॒नौ द्व्य॑क्षरेण॒ द्व्य॑क्षरेण प्राणापा॒ना वुदुत् प्रा॑णापा॒नौ द्व्य॑क्षरेण॒ द्व्य॑क्षरेण प्राणापा॒ना वुत् । \newline
9. द्व्य॑क्षरे॒णेति॒ द्वि - अ॒क्ष॒रे॒ण॒ । \newline
10. प्रा॒णा॒पा॒ना वुदुत् प्रा॑णापा॒नौ प्रा॑णापा॒ना वुद॑जयता मजयता॒ मुत् प्रा॑णापा॒नौ प्रा॑णापा॒ना वुद॑जयताम् । \newline
11. प्रा॒णा॒पा॒नाविति॑ प्राण - अ॒पा॒नौ । \newline
12. उद॑जयता मजयता॒ मुदु द॑जयतां॒ ॅविष्णु॒र् विष्णु॑ रजयता॒ मुदु द॑जयतां॒ ॅविष्णुः॑ । \newline
13. अ॒ज॒य॒तां॒ ॅविष्णु॒र् विष्णु॑ रजयता मजयतां॒ ॅविष्णु॒स्त्र्य॑क्षरेण॒ त्र्य॑क्षरेण॒ विष्णु॑रजयता मजयतां॒ ॅविष्णु॒स्त्र्य॑क्षरेण । \newline
14. विष्णु॒स्त्र्य॑क्षरेण॒ त्र्य॑क्षरेण॒ विष्णु॒र् विष्णु॒स्त्र्य॑क्षरेण॒ त्रीꣳ स्त्रीꣳ स्त्र्य॑क्षरेण॒ विष्णु॒र् विष्णु॒स्त्र्य॑क्षरेण॒ त्रीन् । \newline
15. त्र्य॑क्षरेण॒ त्रीꣳ स्त्रीꣳ स्त्र्य॑क्षरेण॒ त्र्य॑क्षरेण॒ त्रीन् ॅलो॒कान् ॅलो॒काꣳ स्त्रीꣳ स्त्र्य॑क्षरेण॒ त्र्य॑क्षरेण॒ त्रीन् ॅलो॒कान् । \newline
16. त्र्य॑क्षरे॒णेति॒ त्रि - अ॒क्ष॒रे॒ण॒ । \newline
17. त्रीन् ॅलो॒कान् ॅलो॒काꣳ स्त्रीꣳ स्त्रीन् ॅलो॒का नुदु ल्लो॒काꣳ स्त्रीꣳ स्त्रीन् ॅलो॒का नुत् । \newline
18. लो॒का नुदु ल्लो॒कान् ॅलो॒का नुद॑जय दजय॒दु ल्लो॒कान् ॅलो॒का नुद॑जयत् । \newline
19. उद॑जय दजय॒ दुदु द॑जय॒थ् सोमः॒ सोमो॑ अजय॒ दुदु द॑जय॒थ् सोमः॑ । \newline
20. अ॒ज॒य॒थ् सोमः॒ सोमो॑ अजय दजय॒थ् सोम॒ श्चतु॑रक्षरेण॒ चतु॑रक्षरेण॒ सोमो॑ अजय दजय॒थ् सोम॒ श्चतु॑रक्षरेण । \newline
21. सोम॒ श्चतु॑रक्षरेण॒ चतु॑ रक्षरेण॒ सोमः॒ सोम॒ श्चतु॑रक्षरेण॒ चतु॑ष्पद॒ श्चतु॑ष्पद॒ श्चतु॑रक्षरेण॒ सोमः॒ सोम॒ श्चतु॑रक्षरेण॒ चतु॑ष्पदः । \newline
22. चतु॑रक्षरेण॒ चतु॑ष्पद॒ श्चतु॑ष्पद॒ श्चतु॑रक्षरेण॒ चतु॑रक्षरेण॒ चतु॑ष्पदः प॒शून्  
प॒शूꣳ श्चतु॑ष्पद॒ श्चतु॑रक्षरेण॒ चतु॑रक्षरेण॒ चतु॑ष्पदः प॒शून् । \newline
23. चतु॑रक्षरे॒णेति॒ चतुः॑ - अ॒क्ष॒रे॒ण॒ । \newline
24. चतु॑ष्पदः प॒शून् प॒शूꣳ श्चतु॑ष्पद॒ श्चतु॑ष्पदः प॒शू नुदुत् प॒शूꣳ श्चतु॑ष्पद॒ श्चतु॑ष्पदः प॒शू नुत् । \newline
25. चतु॑ष्पद॒ इति॒ चतुः॑ - प॒दः॒ । \newline
26. प॒शू नुदुत् प॒शून् प॒शू नुद॑जय दजय॒दुत् प॒शून् प॒शू नुद॑जयत् । \newline
27. उद॑जय दजय॒ दुदु द॑जयत् पू॒षा पू॒षा ऽज॑य॒ दुदु द॑जयत् पू॒षा । \newline
28. अ॒ज॒य॒त् पू॒षा पू॒षा ऽज॑य दजयत् पू॒षा पञ्चा᳚क्षरेण॒ पञ्चा᳚क्षरेण पू॒षा ऽज॑य दजयत् पू॒षा 
पञ्चा᳚क्षरेण । \newline
29. पू॒षा पञ्चा᳚क्षरेण॒ पञ्चा᳚क्षरेण पू॒षा पू॒षा पञ्चा᳚क्षरेण प॒ङ्क्तिम् प॒ङ्क्तिम् पञ्चा᳚क्षरेण पू॒षा पू॒षा पञ्चा᳚क्षरेण प॒ङ्क्तिम् । \newline
30. पञ्चा᳚क्षरेण प॒ङ्क्तिम् प॒ङ्क्तिम् पञ्चा᳚क्षरेण॒ पञ्चा᳚क्षरेण प॒ङ्क्ति मुदुत् प॒ङ्क्तिम् पञ्चा᳚क्षरेण॒ 
पञ्चा᳚क्षरेण प॒ङ्क्ति मुत् । \newline
31. पञ्चा᳚क्षरे॒णेति॒ पञ्च॑ - अ॒क्ष॒रे॒ण॒ । \newline
32. प॒ङ्क्ति मुदुत् प॒ङ्क्तिम् प॒ङ्क्ति मु द॑जय दजय॒ दुत् प॒ङ्क्तिम् प॒ङ्क्ति मुद॑जयत् । \newline
33. उद॑जय दजय॒ दुदु द॑जयद् धा॒ता धा॒ता ऽज॑य॒ दुदु द॑जयद् धा॒ता । \newline
34. अ॒ज॒य॒द् धा॒ता धा॒ता ऽज॑य दजयद् धा॒ता षड॑क्षरेण॒ षड॑क्षरेण धा॒ता ऽज॑य दजयद् धा॒ता 
षड॑क्षरेण । \newline
35. धा॒ता षड॑क्षरेण॒ षड॑क्षरेण धा॒ता धा॒ता षड॑क्षरेण॒ षट् थ्षट् थ्षड॑क्षरेण धा॒ता धा॒ता षड॑क्षरेण॒ षट् । \newline
36. षड॑क्षरेण॒ षट् थ्षट् थ्षड॑क्षरेण॒ षड॑क्षरेण॒ षडृ॒तू नृ॒तून् षट् 
थ्षड॑क्षरेण॒ षड॑क्षरेण॒ षडृ॒तून् । \newline
37. षड॑क्षरे॒णेति॒ षट् - अ॒क्ष॒रे॒ण॒ । \newline
38. षडृ॒तू नृ॒तून् षट् थ्षडृ॒तू नुदुदृ॒तून् षट् थ्षडृ॒तू नुत् । \newline
39. ऋ॒तू नुदुदृ॒तू नृ॒तू नु द॑जय दजय॒ दुदृ॒तू नृ॒तू नु द॑जयत् । \newline
40. उद॑जय दजय॒ दुदु द॑जयन् म॒रुतो॑ म॒रुतो॑ अजय॒ दुदु द॑जयन् म॒रुतः॑ । \newline
41. अ॒ज॒य॒न् म॒रुतो॑ म॒रुतो॑ अजय दजयन् म॒रुतः॑ स॒प्ताक्ष॑रेण स॒प्ताक्ष॑रेण म॒रुतो॑ अजय दजयन् म॒रुतः॑ स॒प्ताक्ष॑रेण । \newline
42. म॒रुतः॑ स॒प्ताक्ष॑रेण स॒प्ताक्ष॑रेण म॒रुतो॑ म॒रुतः॑ स॒प्ताक्ष॑रेण स॒प्तप॑दाꣳ स॒प्तप॑दाꣳ स॒प्ताक्ष॑रेण म॒रुतो॑ म॒रुतः॑ स॒प्ताक्ष॑रेण स॒प्तप॑दाम् । \newline
43. स॒प्ताक्ष॑रेण स॒प्तप॑दाꣳ स॒प्तप॑दाꣳ स॒प्ताक्ष॑रेण स॒प्ताक्ष॑रेण स॒प्तप॑दा॒(ग्म्॒) शक्व॑री॒(ग्म्॒) शक्व॑रीꣳ स॒प्तप॑दाꣳ स॒प्ताक्ष॑रेण स॒प्ताक्ष॑रेण स॒प्तप॑दा॒(ग्म्॒) शक्व॑रीम् । \newline
44. स॒प्ताक्ष॑रे॒णेति॑ स॒प्त - अ॒क्ष॒रे॒ण॒ । \newline
45. स॒प्तप॑दा॒(ग्म्॒) शक्व॑री॒(ग्म्॒) शक्व॑रीꣳ स॒प्तप॑दाꣳ स॒प्तप॑दा॒(ग्म्॒) शक्व॑री॒ मुदु च्छक्व॑रीꣳ स॒प्तप॑दाꣳ स॒प्तप॑दा॒(ग्म्॒) शक्व॑री॒ मुत् । \newline
46. स॒प्तप॑दा॒मिति॑ स॒प्त - प॒दा॒म् । \newline
47. शक्व॑री॒ मुदु च्छक्व॑री॒(ग्म्॒) शक्व॑री॒ मुद॑जयन् नजय॒न् नु च्छक्व॑री॒(ग्म्॒) शक्व॑री॒ मुद॑जयन्न् । \newline
48. उद॑जयन् नजय॒न् नुदु द॑जय॒न् बृह॒स्पति॒र् बृह॒स्पति॑ रजय॒न् नुदु द॑जय॒न् बृह॒स्पतिः॑ । \newline
49. अ॒ज॒य॒न् बृह॒स्पति॒र् बृह॒स्पति॑ रजयन् नजय॒न् बृह॒स्पति॑ र॒ष्टाक्ष॑रेणा॒ ष्टाक्ष॑रेण॒ बृह॒स्पति॑ रजयन् नजय॒न् बृह॒स्पति॑ र॒ष्टाक्ष॑रेण । \newline
50. बृह॒स्पति॑ र॒ष्टाक्ष॑रेणा॒ ष्टाक्ष॑रेण॒ बृह॒स्पति॒र् बृह॒स्पति॑ र॒ष्टाक्ष॑रेण गाय॒त्रीम् गा॑य॒त्री म॒ष्टाक्ष॑रेण॒ बृह॒स्पति॒र् बृह॒स्पति॑ र॒ष्टाक्ष॑रेण गाय॒त्रीम् । \newline
51. अ॒ष्टाक्ष॑रेण गाय॒त्रीम् गा॑य॒त्री म॒ष्टाक्ष॑रेणा॒ ष्टाक्ष॑रेण गाय॒त्री मुदुद् गा॑य॒त्री म॒ष्टाक्ष॑रेणा॒ ष्टाक्ष॑रेण गाय॒त्री मुत् । \newline
52. अ॒ष्टाक्ष॑रे॒णेत्य॒ष्टा - अ॒क्ष॒रे॒ण॒ । \newline
53. गा॒य॒त्री मुदुद् गा॑य॒त्रीम् गा॑य॒त्री मु द॑जय दजय॒ दुद् गा॑य॒त्रीम् गा॑य॒त्री मुद॑जयत् । \newline
54. उद॑जय दजय॒ दुदु द॑जयन् मि॒त्रो मि॒त्रो अ॑जय॒ दुदु द॑जयन् मि॒त्रः । \newline
55. अ॒ज॒य॒न् मि॒त्रो मि॒त्रो अ॑जय दजयन् मि॒त्रो नवा᳚क्षरेण॒ नवा᳚क्षरेण मि॒त्रो अ॑जय दजयन् मि॒त्रो नवा᳚क्षरेण । \newline
56. मि॒त्रो नवा᳚क्षरेण॒ नवा᳚क्षरेण मि॒त्रो मि॒त्रो नवा᳚क्षरेण त्रि॒वृत॑म् त्रि॒वृत॒म् नवा᳚क्षरेण मि॒त्रो मि॒त्रो नवा᳚क्षरेण त्रि॒वृत᳚म् । \newline
57. नवा᳚क्षरेण त्रि॒वृत॑म् त्रि॒वृत॒म् नवा᳚क्षरेण॒ नवा᳚क्षरेण त्रि॒वृत॒(ग्ग्॒) स्तोम॒(ग्ग्॒) स्तोम॑म् त्रि॒वृत॒म् 
नवा᳚क्षरेण॒ नवा᳚क्षरेण त्रि॒वृत॒(ग्ग्॒) स्तोम᳚म् । \newline
58. नवा᳚क्षरे॒णेति॒ नव॑ - अ॒क्ष॒रे॒ण॒ । \newline
59. त्रि॒वृत॒(ग्ग्॒) स्तोम॒(ग्ग्॒) स्तोम॑म् त्रि॒वृत॑म् त्रि॒वृत॒(ग्ग्॒) स्तोम॒ मुदुथ् स्तोम॑म् त्रि॒वृत॑म् त्रि॒वृत॒(ग्ग्॒) स्तोम॒ मुत् । \newline
60. त्रि॒वृत॒मिति॑ त्रि - वृत᳚म् । \newline
61. स्तोम॒ मुदु थ्स्तोम॒(ग्ग्॒) स्तोम॒ मुद॑जय दजय॒दुथ् स्तोम॒(ग्ग्॒) स्तोम॒ मुद॑जयत् । \newline
62. उद॑जय दजय॒ दुदु द॑जय॒द् वरु॑णो॒ वरु॑णो अजय॒ दुदु द॑जय॒द् वरु॑णः । \newline
63. अ॒ज॒य॒द् वरु॑णो॒ वरु॑णो अजय दजय॒द् वरु॑णो॒ दशा᳚क्षरेण॒ दशा᳚क्षरेण॒ वरु॑णो अजय दजय॒द् वरु॑णो॒ दशा᳚क्षरेण । \newline
\pagebreak
\markright{ TS 1.7.11.2  \hfill https://www.vedavms.in \hfill}
\addcontentsline{toc}{section}{ TS 1.7.11.2 }
\section*{ TS 1.7.11.2 }

\textbf{TS 1.7.11.2 } \newline
\textbf{Samhita Paata} \newline

वरु॑णो॒ दशा᳚क्षरेण वि॒राज॒-मुद॑जय॒दिन्द्र॒ एका॑दशाक्षरेण त्रि॒ष्टुभ॒-मुद॑जय॒द् विश्वे॑ दे॒वा द्वाद॑शाक्षरेण॒ जग॑ती॒मुद॑जय॒न् वस॑व॒स्त्रयो॑ दशाक्षरेण त्रयोद॒शꣳ स्तोम॒मुद॑जयन् रु॒द्राश्चतु॑र्दशाक्षरेण चतुर्द॒शꣳ स्तोम॒मुद॑जयन्नादि॒त्याः पञ्च॑दशाक्षरेण पञ्चद॒शꣳ स्तोम॒मुद॑जय॒न्नदि॑तिः॒ षोड॑शाक्षरेण षोड॒शꣳ स्तोम॒मुद॑जयत् प्र॒जाप॑तिः स॒प्तद॑शाक्षरेण सप्तद॒शꣳ स्तोम॒मुद॑जयत् ॥ \newline

\textbf{Pada Paata} \newline

वरु॑णः । दशा᳚क्षरे॒णेति॒ दश॑ - अ॒क्ष॒रे॒ण॒ । वि॒राज॒मिति॑ वि - राज᳚म् । उदिति॑ । अ॒ज॒य॒त् । इन्द्रः॑ । एका॑दशाक्षरे॒णेत्येका॑दश - अ॒क्ष॒रे॒ण॒ । त्रि॒ष्टुभ᳚म् । उदिति॑ । अ॒ज॒य॒त् । विश्वे᳚ । दे॒वाः । द्वाद॑शाक्षरे॒णेति॒ द्वाद॑श - अ॒क्ष॒रे॒ण॒ । जग॑तीम् । उदिति॑ । अ॒ज॒य॒न्न् । वस॑वः । त्रयो॑दशाक्षरे॒णेति॒ त्रयो॑दश - अ॒क्ष॒रे॒ण॒ । त्र॒यो॒द॒शमिति॑ त्रयः - द॒शम् । स्तोम᳚म् । उदिति॑ । अ॒ज॒य॒न्न् । रु॒द्राः । चतु॑र्दशाक्षरे॒णेति॒ चतु॑र्दश - अ॒क्ष॒रे॒ण॒ । च॒तु॒र्द॒शमिति॑ चतुः-द॒शम् । स्तोम᳚म् । उदिति॑ । अ॒ज॒य॒न्न् । आ॒दि॒त्याः । पञ्च॑दशाक्षरे॒णेति॒ पञ्च॑दश - अ॒क्ष॒रे॒ण॒ । प॒ञ्च॒द॒शमिति॑ पञ्च - द॒शम् । स्तोम᳚म् । उदिति॑ । अ॒ज॒य॒॒न्न् । अदि॑तिः । षोड॑शाक्षरे॒णेति॒ षोड॑श - अ॒क्ष॒रे॒ण॒ । षो॒ड॒शम् । स्तोम᳚म् । उदिति॑ । अ॒ज॒य॒त् । प्र॒जाप॑ति॒रिति॑ प्र॒जा - प॒तिः॒ । स॒प्तद॑शाक्षरे॒णेति॑ स॒प्तद॑श-अ॒क्ष॒रे॒ण॒ । स॒प्त॒द॒शमिति॑ सप्त - द॒शम् । स्तोम᳚म् । उदिति॑ । अ॒ज॒य॒त् ॥  \newline


\textbf{Krama Paata} \newline

वरु॑णो॒ दशा᳚क्षरेण । दशा᳚क्षरेण वि॒राज᳚म् । दशा᳚क्षरे॒णेति॒ दश॑ - अ॒क्ष॒रे॒ण॒ । वि॒राज॒मुत् । वि॒राज॒मिति॑ वि - राज᳚म् । उद॑जयत् । अ॒ज॒य॒दिन्द्रः॑ । इन्द्र॒ एका॑दशाक्षरेण । एका॑दशाक्षरेण त्रि॒ष्टुभ᳚म् । एका॑दशाक्षरे॒णेत्येका॑दश - अ॒क्ष॒रे॒ण॒ । त्रि॒ष्टुभ॒मुत् । उद॑जयत् । अ॒ज॒य॒द् विश्वे᳚ । विश्वे॑ दे॒वाः । दे॒वा द्वाद॑शाक्षरेण । द्वाद॑शाक्षरेण॒ जग॑तीम् । द्वाद॑शाक्षरे॒णेति॒ द्वाद॑श - अ॒क्ष॒रे॒ण॒ । जग॑ती॒मुत् । उद॑जयन्न् । अ॒ज॒य॒न्॒. वस॑वः । वस॑व॒स्त्रयो॑दशाक्षरेण । त्रयो॑दशाक्षरेण त्रयोद॒शम् । त्रयो॑दशाक्षरे॒णेति॒ त्रयो॑दश - अ॒क्ष॒रे॒ण॒ । त्र॒यो॒द॒शꣳ स्तोम᳚म् । त्र॒यो॒द॒शमिति॑ त्रयः - द॒शम् । स्तोम॒मुत् । उद॑जयन्न् । अ॒ज॒य॒न्,रु॒द्राः । रु॒द्राश्चतु॑र्दशाक्षरेण । चतु॑र्दशाक्षरेण चतुर्द॒शम् । चतु॑र्दशाक्षरे॒णेति॒ चतु॑र्दश - अ॒क्ष॒रे॒ण॒ । च॒तु॒र्द॒शꣳ स्तोम᳚म् । च॒तु॒र्द॒शमिति॑ चतुः - द॒शम् । स्तोम॒मुत् । उद॑जयन्न् । अ॒ज॒य॒न्ना॒दि॒त्याः । आ॒दि॒त्याः पञ्च॑दशाक्षरेण । पञ्च॑दशाक्षरेण पञ्चद॒शम् । पञ्च॑दशाक्षरे॒णेति॒ पञ्च॑दश - अ॒क्ष॒रे॒ण॒ । प॒ञ्च॒द॒शꣳ स्तोम᳚म् । प॒ञ्च॒द॒शमिति॑ पञ्च - द॒शम् । स्तोम॒मुत् । उद॑जयन्न् । अ॒ज॒य॒न्नदि॑तिः । अदि॑तिः॒ षोड॑शाक्षरेण । षोड॑शाक्षरेण षोड॒शम् । षोड॑शाक्षरे॒णेति॒ षोड॑श - अ॒क्ष॒रे॒ण॒ । षो॒ड॒शꣳ स्तोम᳚म् । स्तोम॒मुत् । उद॑जयत् । अ॒ज॒य॒त्,प्र॒जाप॑तिः । प्र॒जाप॑तिः स॒प्तद॑शाक्षरेण । प्र॒जाप॑ति॒रिति॑ प्र॒जा - प॒तिः॒ । स॒प्तद॑शाक्षरेण सप्तद॒शम् । स॒प्तद॑शाक्षरे॒णेति॑ स॒प्तद॑श - अ॒क्ष॒रे॒ण॒ । स॒प्त॒द॒शꣳ स्तोम᳚म् । स॒प्त॒द॒शमिति॑ सप्त - द॒शम् । स्तोम॒मुत् । उद॑जयत् । अ॒ज॒य॒दि॒त्य॑जयत् । \newline

\textbf{Jatai Paata} \newline

1. वरु॑णो॒ दशा᳚क्षरेण॒ दशा᳚क्षरेण॒ वरु॑णो॒ वरु॑णो॒ दशा᳚क्षरेण । \newline
2. दशा᳚क्षरेण वि॒राजं॑ ॅवि॒राज॒म् दशा᳚क्षरेण॒ दशा᳚क्षरेण वि॒राज᳚म् । \newline
3. दशा᳚क्षरे॒णेति॒ दश॑ - अ॒क्ष॒रे॒ण॒ । \newline
4. वि॒राज॒ मुदुद् वि॒राजं॑ ॅवि॒राज॒ मुत् । \newline
5. वि॒राज॒मिति॑ वि - राज᳚म् । \newline
6. उद॑जय दजय॒ दुदु द॑जयत् । \newline
7. अ॒ज॒य॒ दिन्द्र॒ इन्द्रो॑ अजय दजय॒ दिन्द्रः॑ । \newline
8. इन्द्र॒ एका॑दशाक्षरे॒ णैका॑दशाक्षरे॒णे न्द्र॒ इन्द्र॒ एका॑दशाक्षरेण । \newline
9. एका॑दशाक्षरेण त्रि॒ष्टुभ॑म् त्रि॒ष्टुभ॒ मेका॑दशाक्षरे॒ णैका॑दशाक्षरेण त्रि॒ष्टुभ᳚म् । \newline
10. एका॑दशाक्षरे॒णेत्येका॑दश - अ॒क्ष॒रे॒ण॒ । \newline
11. त्रि॒ष्टुभ॒ मुदुत् त्रि॒ष्टुभ॑म् त्रि॒ष्टुभ॒ मुत् । \newline
12. उद॑जय दजय॒ दुदु द॑जयत् । \newline
13. अ॒ज॒य॒द् विश्वे॒ विश्वे॑ अजय दजय॒द् विश्वे᳚ । \newline
14. विश्वे॑ दे॒वा दे॒वा विश्वे॒ विश्वे॑ दे॒वाः । \newline
15. दे॒वा द्वाद॑शाक्षरेण॒ द्वाद॑शाक्षरेण दे॒वा दे॒वा द्वाद॑शाक्षरेण । \newline
16. द्वाद॑शाक्षरेण॒ जग॑ती॒म् जग॑ती॒म् द्वाद॑शाक्षरेण॒ द्वाद॑शाक्षरेण॒ जग॑तीम् । \newline
17. द्वाद॑शाक्षरे॒णेति॒ द्वाद॑श - अ॒क्ष॒रे॒ण॒ । \newline
18. जग॑ती॒ मुदुज् जग॑ती॒म् जग॑ती॒ मुत् । \newline
19. उद॑जयन् नजय॒न् नुदु द॑जयन्न् । \newline
20. अ॒ज॒य॒न्॒. वस॑वो॒ वस॑वो अजयन् नजय॒न्॒. वस॑वः । \newline
21. वस॑व॒ स्त्रयो॑दशाक्षरेण॒ त्रयो॑दशाक्षरेण॒ वस॑वो॒ वस॑व॒ स्त्रयो॑दशाक्षरेण । \newline
22. त्रयो॑दशाक्षरेण त्रयोद॒शम् त्र॑योद॒शम् त्रयो॑दशाक्षरेण॒ त्रयो॑दशाक्षरेण त्रयोद॒शम् । \newline
23. त्रयो॑दशाक्षरे॒णेति॒ त्रयो॑दश - अ॒क्ष॒रे॒ण॒ । \newline
24. त्र॒यो॒द॒शꣳ स्तोमꣳ॒॒ स्तोम॑म् त्रयोद॒शम् त्र॑योद॒शꣳ स्तोम᳚म् । \newline
25. त्र॒यो॒द॒शमिति॑ त्रयः - द॒शम् । \newline
26. स्तोम॒ मुदुथ् स्तोमꣳ॒॒ स्तोम॒ मुत् । \newline
27. उद॑जयन् नजय॒न् नुदु द॑जयन्न् । \newline
28. अ॒ज॒य॒न् रु॒द्रा रु॒द्रा अ॑जयन् नजयन् रु॒द्राः । \newline
29. रु॒द्रा श्चतु॑र्दशाक्षरेण॒ चतु॑र्दशाक्षरेण रु॒द्रा रु॒द्रा श्चतु॑र्दशाक्षरेण । \newline
30. चतु॑र्दशाक्षरेण चतुर्द॒शम् च॑तुर्द॒शम् चतु॑र्दशाक्षरेण॒ चतु॑र्दशाक्षरेण चतुर्द॒शम् । \newline
31. चतु॑र्दशाक्षरे॒णेति॒ चतु॑र्दश - अ॒क्ष॒रे॒ण॒ । \newline
32. च॒तु॒र्द॒शꣳ स्तोमꣳ॒॒ स्तोम॑म् चतुर्द॒शम् च॑तुर्द॒शꣳ स्तोम᳚म् । \newline
33. च॒तु॒र्द॒शमिति॑ चतुः - द॒शम् । \newline
34. स्तोम॒ मुदुथ् स्तोमꣳ॒॒ स्तोम॒ मुत् । \newline
35. उद॑जयन् नजय॒न् नुदु द॑जयन्न् । \newline
36. अ॒ज॒य॒न् ना॒दि॒त्या आ॑दि॒त्या अ॑जयन् नजयन् नादि॒त्याः । \newline
37. आ॒दि॒त्याः पञ्च॑दशाक्षरेण॒ पञ्च॑दशाक्षरे णादि॒त्या आ॑दि॒त्याः पञ्च॑दशाक्षरेण । \newline
38. पञ्च॑दशाक्षरेण पञ्चद॒शम् प॑ञ्चद॒शम् पञ्च॑दशाक्षरेण॒ पञ्च॑दशाक्षरेण पञ्चद॒शम् । \newline
39. पञ्च॑दशाक्षरे॒णेति॒ पञ्च॑दश - अ॒क्ष॒रे॒ण॒ । \newline
40. प॒ञ्च॒द॒शꣳ स्तोमꣳ॒॒ स्तोम॑म् पञ्चद॒शम् प॑ञ्चद॒शꣳ स्तोम᳚म् । \newline
41. प॒ञ्च॒द॒शमिति॑ पञ्च - द॒शम् । \newline
42. स्तोम॒ मुदुथ् स्तोमꣳ॒॒ स्तोम॒ मुत् । \newline
43. उद॑जयन् नजय॒न् नुदु द॑जयन्न् । \newline
44. अ॒ज॒य॒न् नदि॑ति॒ रदि॑ति रजयन् नजय॒न् नदि॑तिः । \newline
45. अदि॑ति॒ ष्षोड॑शाक्षरेण॒ षोड॑शाक्षरे॒णा दि॑ति॒ रदि॑ति॒ ष्षोड॑शाक्षरेण । \newline
46. षोड॑शाक्षरेण षोड॒शꣳ षो॑ड॒शꣳ षोड॑शाक्षरेण॒ षोड॑शाक्षरेण षोड॒शम् । \newline
47. षोड॑शाक्षरे॒णेति॒ षोड॑श - अ॒क्ष॒रे॒ण॒ । \newline
48. षो॒ड॒शꣳ स्तोमꣳ॒॒ स्तोमꣳ॑ षोड॒शꣳ षो॑ड॒शꣳ स्तोम᳚म् । \newline
49. स्तोम॒ मुदुथ् स्तोमꣳ॒॒ स्तोम॒ मुत् । \newline
50. उद॑जय दजय॒ दुदु द॑जयत् । \newline
51. अ॒ज॒य॒त् प्र॒जाप॑तिः प्र॒जाप॑ति रजय दजयत् प्र॒जाप॑तिः । \newline
52. प्र॒जाप॑तिः स॒प्तद॑शाक्षरेण स॒प्तद॑शाक्षरेण प्र॒जाप॑तिः प्र॒जाप॑तिः स॒प्तद॑शाक्षरेण । \newline
53. प्र॒जाप॑ति॒रिति॑ प्र॒जा - प॒तिः॒ । \newline
54. स॒प्तद॑शाक्षरेण सप्तद॒शꣳ स॑प्तद॒शꣳ स॒प्तद॑शाक्षरेण स॒प्तद॑शाक्षरेण सप्तद॒शम् । \newline
55. स॒प्तद॑शाक्षरे॒णेति॑ स॒प्तद॑श - अ॒क्ष॒रे॒ण॒ । \newline
56. स॒प्त॒द॒शꣳ स्तोमꣳ॒॒ स्तोमꣳ॑ सप्तद॒शꣳ स॑प्तद॒शꣳ स्तोम᳚म् । \newline
57. स॒प्त॒द॒शमिति॑ सप्त - द॒शम् । \newline
58. स्तोम॒ मुदुथ् स्तोमꣳ॒॒ स्तोम॒ मुत् । \newline
59. उद॑जय दजय॒ दुदु द॑जयत् । \newline
60. अ॒ज॒य॒दित्य॑जयत् । \newline

\textbf{Ghana Paata } \newline

1. वरु॑णो॒ दशा᳚क्षरेण॒ दशा᳚क्षरेण॒ वरु॑णो॒ वरु॑णो॒ दशा᳚क्षरेण वि॒राजं॑ ॅवि॒राज॒म् दशा᳚क्षरेण॒ वरु॑णो॒ वरु॑णो॒ दशा᳚क्षरेण वि॒राज᳚म् । \newline
2. दशा᳚क्षरेण वि॒राजं॑ ॅवि॒राज॒म् दशा᳚क्षरेण॒ दशा᳚क्षरेण वि॒राज॒ मुदुद् वि॒राज॒म् दशा᳚क्षरेण॒ दशा᳚क्षरेण वि॒राज॒ मुत् । \newline
3. दशा᳚क्षरे॒णेति॒ दश॑ - अ॒क्ष॒रे॒ण॒ । \newline
4. वि॒राज॒ मुदुद् वि॒राजं॑ ॅवि॒राज॒ मुद॑जय दजय॒ दुद् वि॒राजं॑ ॅवि॒राज॒ मुद॑जयत् । \newline
5. वि॒राज॒मिति॑ वि - राज᳚म् । \newline
6. उद॑जय दजय॒ दुदु द॑जय॒ दिन्द्र॒ इन्द्रो॑ अजय॒ दुदु द॑जय॒ दिन्द्रः॑ । \newline
7. अ॒ज॒य॒ दिन्द्र॒ इन्द्रो॑ अजय दजय॒ दिन्द्र॒ एका॑दशाक्षरे॒ णैका॑दशाक्षरे॒णे न्द्रो॑ अजय दजय॒ दिन्द्र॒ एका॑दशाक्षरेण । \newline
8. इन्द्र॒ एका॑दशाक्षरे॒ णैका॑दशाक्षरे॒णे न्द्र॒ इन्द्र॒ एका॑दशाक्षरेण त्रि॒ष्टुभ॑म् त्रि॒ष्टुभ॒ मेका॑दशाक्षरे॒णे न्द्र॒ इन्द्र॒ एका॑दशाक्षरेण त्रि॒ष्टुभ᳚म् । \newline
9. एका॑दशाक्षरेण त्रि॒ष्टुभ॑म् त्रि॒ष्टुभ॒ मेका॑दशाक्षरे॒ णैका॑दशाक्षरेण त्रि॒ष्टुभ॒ मुदुत् त्रि॒ष्टुभ॒ मेका॑दशाक्षरे॒ णैका॑दशाक्षरेण त्रि॒ष्टुभ॒ मुत् । \newline
10. एका॑दशाक्षरे॒णेत्येका॑दश - अ॒क्ष॒रे॒ण॒ । \newline
11. त्रि॒ष्टुभ॒ मुदुत् त्रि॒ष्टुभ॑म् त्रि॒ष्टुभ॒ मुद॑जय दजय॒ दुत् त्रि॒ष्टुभ॑म् त्रि॒ष्टुभ॒ मुद॑जयत् । \newline
12. उद॑जय दजय॒ दुदु द॑जय॒द् विश्वे॒ विश्वे॑ अजय॒ दुदु द॑जय॒द् विश्वे᳚ । \newline
13. अ॒ज॒य॒द् विश्वे॒ विश्वे॑ अजय दजय॒द् विश्वे॑ दे॒वा दे॒वा विश्वे॑ अजय दजय॒द् विश्वे॑ दे॒वाः । \newline
14. विश्वे॑ दे॒वा दे॒वा विश्वे॒ विश्वे॑ दे॒वा द्वाद॑शाक्षरेण॒ द्वाद॑शाक्षरेण दे॒वा विश्वे॒ विश्वे॑ दे॒वा द्वाद॑शाक्षरेण । \newline
15. दे॒वा द्वाद॑शाक्षरेण॒ द्वाद॑शाक्षरेण दे॒वा दे॒वा द्वाद॑शाक्षरेण॒ जग॑ती॒म् जग॑ती॒म् द्वाद॑शाक्षरेण दे॒वा दे॒वा द्वाद॑शाक्षरेण॒ जग॑तीम् । \newline
16. द्वाद॑शाक्षरेण॒ जग॑ती॒म् जग॑ती॒म् द्वाद॑शाक्षरेण॒ द्वाद॑शाक्षरेण॒ जग॑ती॒ मुदुज् जग॑ती॒म् द्वाद॑शाक्षरेण॒ द्वाद॑शाक्षरेण॒ जग॑ती॒ मुत् । \newline
17. द्वाद॑शाक्षरे॒णेति॒ द्वाद॑श - अ॒क्ष॒रे॒ण॒ । \newline
18. जग॑ती॒ मुदुज् जग॑ती॒म् जग॑ती॒ मुद॑जयन् नजय॒न् नुज् जग॑ती॒म् जग॑ती॒ मुद॑जयन्न् । \newline
19. उद॑जयन् नजय॒न् नुदु द॑जय॒न्॒. वस॑वो॒ वस॑वो अजय॒न् नुदु द॑जय॒न्॒. वस॑वः । \newline
20. अ॒ज॒य॒न्॒. वस॑वो॒ वस॑वो अजयन् नजय॒न्॒. वस॑व॒ स्त्रयो॑दशाक्षरेण॒ त्रयो॑दशाक्षरेण॒ वस॑वो अजयन् नजय॒न्॒. वस॑व॒ स्त्रयो॑दशाक्षरेण । \newline
21. वस॑व॒ स्त्रयो॑दशाक्षरेण॒ त्रयो॑दशाक्षरेण॒ वस॑वो॒ वस॑व॒ स्त्रयो॑दशाक्षरेण त्रयोद॒शम् त्र॑योद॒शम् त्रयो॑दशाक्षरेण॒ वस॑वो॒ वस॑व॒ स्त्रयो॑दशाक्षरेण त्रयोद॒शम् । \newline
22. त्रयो॑दशाक्षरेण त्रयोद॒शम् त्र॑योद॒शम् त्रयो॑दशाक्षरेण॒ त्रयो॑दशाक्षरेण त्रयोद॒शꣳ स्तोम॒(ग्ग्॒) स्तोम॑म् त्रयोद॒शम् त्रयो॑दशाक्षरेण॒ त्रयो॑दशाक्षरेण त्रयोद॒शꣳ स्तोम᳚म् । \newline
23. त्रयो॑दशाक्षरे॒णेति॒ त्रयो॑दश - अ॒क्ष॒रे॒ण॒ । \newline
24. त्र॒यो॒द॒शꣳ स्तोम॒(ग्ग्॒) स्तोम॑म् त्रयोद॒शम् त्र॑योद॒शꣳ स्तोम॒ मुदुथ् स्तोम॑म् त्रयोद॒शम् त्र॑योद॒शꣳ स्तोम॒ मुत् । \newline
25. त्र॒यो॒द॒शमिति॑ त्रयः - द॒शम् । \newline
26. स्तोम॒ मुदुथ् स्तोम॒(ग्ग्॒) स्तोम॒ मुद॑जयन् नजय॒न् नुथ्स्तोम॒(ग्ग्॒) स्तोम॒ मुद॑जयन्न् । \newline
27. उद॑जयन् नजय॒न् नुदु द॑जयन् रु॒द्रा रु॒द्रा अ॑जय॒न् नुदु द॑जयन् रु॒द्राः । \newline
28. अ॒ज॒य॒न् रु॒द्रा रु॒द्रा अ॑जयन् नजयन् रु॒द्रा श्चतु॑र्दशाक्षरेण॒ चतु॑र्दशाक्षरेण रु॒द्रा अ॑जयन् नजयन् रु॒द्रा श्चतु॑र्दशाक्षरेण । \newline
29. रु॒द्रा श्चतु॑र्दशाक्षरेण॒ चतु॑र्दशाक्षरेण रु॒द्रा रु॒द्रा श्चतु॑र्दशाक्षरेण चतुर्द॒शम् च॑तुर्द॒शम् चतु॑र्दशाक्षरेण रु॒द्रा रु॒द्रा श्चतु॑र्दशाक्षरेण चतुर्द॒शम् । \newline
30. चतु॑र्दशाक्षरेण चतुर्द॒शम् च॑तुर्द॒शम् चतु॑र्दशाक्षरेण॒ चतु॑र्दशाक्षरेण चतुर्द॒शꣳ स्तोम॒(ग्ग्॒) स्तोम॑म् चतुर्द॒शम् चतु॑र्दशाक्षरेण॒ चतु॑र्दशाक्षरेण चतुर्द॒शꣳ स्तोम᳚म् । \newline
31. चतु॑र्दशाक्षरे॒णेति॒ चतु॑र्दश - अ॒क्ष॒रे॒ण॒ । \newline
32. च॒तु॒र्द॒शꣳ स्तोम॒(ग्ग्॒) स्तोम॑म् चतुर्द॒शम् च॑तुर्द॒शꣳ स्तोम॒ मुदुथ् स्तोम॑म् चतुर्द॒शम् च॑तुर्द॒शꣳ स्तोम॒ मुत् । \newline
33. च॒तु॒र्द॒शमिति॑ चतुः - द॒शम् । \newline
34. स्तोम॒ मुदुथ् स्तोम॒(ग्ग्॒) स्तोम॒ मुद॑जयन् नजय॒न् नुथ् स्तोम॒(ग्ग्॒) स्तोम॒ मुद॑जयन्न् । \newline
35. उद॑जयन् नजय॒न् नुदु द॑जयन् नादि॒त्या आ॑दि॒त्या अ॑जय॒न् नुदु द॑जयन् नादि॒त्याः । \newline
36. अ॒ज॒य॒न् ना॒दि॒त्या आ॑दि॒त्या अ॑जयन् नजयन् नादि॒त्याः पञ्च॑दशाक्षरेण॒ पञ्च॑दशाक्षरे णादि॒त्या अ॑जयन् नजयन् नादि॒त्याः पञ्च॑दशाक्षरेण । \newline
37. आ॒दि॒त्याः पञ्च॑दशाक्षरेण॒ पञ्च॑दशाक्षरे णादि॒त्या आ॑दि॒त्याः पञ्च॑दशाक्षरेण पञ्चद॒शम् प॑ञ्चद॒शम् पञ्च॑दशाक्षरे णादि॒त्या आ॑दि॒त्याः पञ्च॑दशाक्षरेण पञ्चद॒शम् । \newline
38. पञ्च॑दशाक्षरेण पञ्चद॒शम् प॑ञ्चद॒शम् पञ्च॑दशाक्षरेण॒ पञ्च॑दशाक्षरेण पञ्चद॒शꣳ स्तोम॒(ग्ग्॒) स्तोम॑म् पञ्चद॒शम् पञ्च॑दशाक्षरेण॒ पञ्च॑दशाक्षरेण पञ्चद॒शꣳ स्तोम᳚म् । \newline
39. पञ्च॑दशाक्षरे॒णेति॒ पञ्च॑दश - अ॒क्ष॒रे॒ण॒ । \newline
40. प॒ञ्च॒द॒शꣳ स्तोम॒(ग्ग्॒) स्तोम॑म् पञ्चद॒शम् प॑ञ्चद॒शꣳ स्तोम॒ मुदुथ् स्तोम॑म् पञ्चद॒शम् प॑ञ्चद॒शꣳ स्तोम॒ मुत् । \newline
41. प॒ञ्च॒द॒शमिति॑ पञ्च - द॒शम् । \newline
42. स्तोम॒ मुदुथ् स्तोम॒(ग्ग्॒) स्तोम॒ मुद॑जयन् नजय॒न् नुथ् स्तोम॒(ग्ग्॒) स्तोम॒ मुद॑जयन्न् । \newline
43. उद॑जयन् नजय॒न् नुदु द॑जय॒न् नदि॑ति॒ रदि॑ति रजय॒न् नुदु द॑जय॒न् नदि॑तिः । \newline
44. अ॒ज॒य॒न् नदि॑ति॒ रदि॑ति रजयन् नजय॒न् नदि॑ति॒ ष्षोड॑शाक्षरेण॒ षोड॑शाक्षरे॒णा दि॑ति रजयन् नजय॒न् नदि॑ति॒ ष्षोड॑शाक्षरेण । \newline
45. अदि॑ति॒ ष्षोड॑शाक्षरेण॒ षोड॑शाक्षरे॒ णादि॑ति॒ रदि॑ति॒ ष्षोड॑शाक्षरेण षोड॒शꣳ षो॑ड॒शꣳ षोड॑शाक्षरे॒ णादि॑ति॒ रदि॑ति॒ ष्षोड॑शाक्षरेण षोड॒शम् । \newline
46. षोड॑शाक्षरेण षोड॒शꣳ षो॑ड॒शꣳ षोड॑शाक्षरेण॒ षोड॑शाक्षरेण षोड॒शꣳ स्तोम॒(ग्ग्॒) स्तोम(ग्म्॑) षोड॒शꣳ षोड॑शाक्षरेण॒ षोड॑शाक्षरेण षोड॒शꣳ स्तोम᳚म् । \newline
47. षोड॑शाक्षरे॒णेति॒ षोड॑श - अ॒क्ष॒रे॒ण॒ । \newline
48. षो॒ड॒शꣳ स्तोम॒(ग्ग्॒) स्तोम(ग्म्॑) षोड॒शꣳ षो॑ड॒शꣳ स्तोम॒ मुदुथ् स्तोम(ग्म्॑) षोड॒शꣳ षो॑ड॒शꣳ स्तोम॒ मुत् । \newline
49. स्तोम॒ मुदुथ् स्तोम॒(ग्ग्॒) स्तोम॒ मुद॑जय दजय॒दुथ् स्तोम॒(ग्ग्॒) स्तोम॒ मुद॑जयत् । \newline
50. उद॑जय दजय॒ दुदु द॑जयत् प्र॒जाप॑तिः प्र॒जाप॑ति रजय॒ दुदु द॑जयत् प्र॒जाप॑तिः । \newline
51. अ॒ज॒य॒त् प्र॒जाप॑तिः प्र॒जाप॑ति रजयदजयत् प्र॒जाप॑तिः स॒प्तद॑शाक्षरेण स॒प्तद॑शाक्षरेण प्र॒जाप॑ति रजयदजयत् प्र॒जाप॑तिः स॒प्तद॑शाक्षरेण । \newline
52. प्र॒जाप॑तिः स॒प्तद॑शाक्षरेण स॒प्तद॑शाक्षरेण प्र॒जाप॑तिः प्र॒जाप॑तिः स॒प्तद॑शाक्षरेण सप्तद॒शꣳ स॑प्तद॒शꣳ स॒प्तद॑शाक्षरेण प्र॒जाप॑तिः प्र॒जाप॑तिः स॒प्तद॑शाक्षरेण सप्तद॒शम् । \newline
53. प्र॒जाप॑ति॒रिति॑ प्र॒जा - प॒तिः॒ । \newline
54. स॒प्तद॑शाक्षरेण सप्तद॒शꣳ स॑प्तद॒शꣳ स॒प्तद॑शाक्षरेण स॒प्तद॑शाक्षरेण सप्तद॒शꣳ स्तोम॒(ग्ग्॒) स्तोम(ग्म्॑) सप्तद॒शꣳ स॒प्तद॑शाक्षरेण स॒प्तद॑शाक्षरेण सप्तद॒शꣳ स्तोम᳚म् । \newline
55. स॒प्तद॑शाक्षरे॒णेति॑ स॒प्तद॑श - अ॒क्ष॒रे॒ण॒ । \newline
56. स॒प्त॒द॒शꣳ स्तोम॒(ग्ग्॒) स्तोम(ग्म्॑) सप्तद॒शꣳ स॑प्तद॒शꣳ स्तोम॒ मुदुथ् स्तोम(ग्म्॑) सप्तद॒शꣳ स॑प्तद॒शꣳ स्तोम॒ मुत् । \newline
57. स॒प्त॒द॒शमिति॑ सप्त - द॒शम् । \newline
58. स्तोम॒ मुदुथ् स्तोम॒(ग्ग्॒) स्तोम॒ मुद॑जय दजय॒दुथ् स्तोम॒(ग्ग्॒) स्तोम॒ मुद॑जयत् । \newline
59. उद॑जय दजय॒ दुदु द॑जयत् । \newline
60. अ॒ज॒य॒दित्य॑जयत् । \newline
\pagebreak
\markright{ TS 1.7.12.1  \hfill https://www.vedavms.in \hfill}
\addcontentsline{toc}{section}{ TS 1.7.12.1 }
\section*{ TS 1.7.12.1 }

\textbf{TS 1.7.12.1 } \newline
\textbf{Samhita Paata} \newline

उ॒प॒या॒मगृ॑हीतोऽसि नृ॒षदं॑ त्वा द्रु॒षदं॑ भुवन॒सद॒मिन्द्रा॑य॒ जुष्टं॑ गृह्णाम्ये॒ष ते॒ योनि॒रिन्द्रा॑य त्वोपया॒मगृ॑हीतोऽस्यफ्सु॒षदं॑ त्वा घृत॒सदं॑ ॅव्योम॒सद॒मिन्द्रा॑य॒ जुष्टं॑ गृह्णाम्ये॒ष ते॒ योनि॒रिन्द्रा॑य त्वोपया॒मगृ॑हीतोऽसि पृथिवि॒षदं॑ त्वाऽन्तरिक्ष॒सदं॑ नाक॒सद॒मिन्द्रा॑य॒ जुष्टं॑ गृह्णाम्ये॒ष ते॒ योनि॒रिन्द्रा॑य त्वा ॥ ये ग्रहाः᳚ पञ्चज॒नीना॒ येषां᳚ ति॒स्रः प॑रम॒जाः । दैव्यः॒ कोशः॒ - [ ] \newline

\textbf{Pada Paata} \newline

उ॒प॒या॒मगृ॑हीत॒ इत्यु॑पया॒म - गृ॒ही॒तः॒ । अ॒सि॒ । नृ॒षद॒मिति॑ नृ-सद᳚म् । त्वा॒ । द्रु॒षद॒मिति॑ द्रु - सद᳚म् । भु॒व॒न॒सद॒मिति॑ भुवन - सद᳚म् । इन्द्रा॑य । जुष्ट᳚म् । गृ॒ह्णा॒मि॒ । ए॒षः । ते॒ । योनिः॑ । इन्द्रा॑य । त्वा॒ । उ॒प॒या॒मगृ॑हीत॒ इत्यु॑पया॒म - गृ॒ही॒तः॒ । अ॒सि॒ । अ॒फ्सु॒षद॒मित्य॑फ्सु - सद᳚म् । त्वा॒ । घृ॒त॒सद॒मिति॑ घृत - सद᳚म् । व्यो॒म॒सद॒मिति॑ व्योम - सद᳚म् । इन्द्रा॑य । जुष्ट᳚म् । गृ॒ह्णा॒मि॒ । ए॒षः । ते॒ । योनिः॑ । इन्द्रा॑य । त्वा॒ । उ॒प॒या॒मगृ॑हीत॒ इत्यु॑पया॒म - गृ॒ही॒तः॒ । अ॒सि॒ । पृ॒थि॒वि॒षद॒मिति॑ पृथिवि - सद᳚म् । त्वा॒ । अ॒न्त॒रि॒क्ष॒सद॒मित्य॑न्तरिक्ष - सद᳚म् । ना॒क॒सद॒मिति॑ नाक - सद᳚म् । इन्द्रा॑य । जुष्ट᳚म् । गृ॒ह्णा॒मि॒ । ए॒षः । ते॒ । योनिः॑ । इन्द्रा॑य । त्वा॒ ॥ ये । ग्रहाः᳚ । प॒ञ्च॒ज॒नीना॒ इति॑ पञ्च - ज॒नीनाः᳚ । येषा᳚म् । ति॒स्रः । प॒र॒म॒जा इति॑ परम - जाः ॥ दैव्यः॑ । कोशः॑ ।  \newline


\textbf{Krama Paata} \newline

उ॒प॒या॒मगृ॑हीतोऽसि । उ॒प॒या॒मगृ॑हीत॒ इत्यु॑पया॒म - गृ॒ही॒तः॒ । अ॒सि॒ नृ॒षद᳚म् । नृ॒षद॑म् त्वा । नृ॒षद॒मिति॑ नृ - सद᳚म् । त्वा॒ द्रु॒षद᳚म् । द्रु॒षद॑म् भुवन॒सद᳚म् । द्रु॒षद॒मिति॑ द्रु - सद᳚म् । भु॒व॒न॒सद॒मिन्द्रा॑य । भु॒व॒न॒सद॒मिति॑ भुवन - सद᳚म् । इन्द्रा॑य॒ जुष्ट᳚म् । जुष्ट॑म् गृह्णामि । गृ॒ह्णा॒म्ये॒षः । ए॒ष ते᳚ । ते॒ योनिः॑ । योनि॒रिन्द्रा॑य । इन्द्रा॑य त्वा । त्वो॒प॒या॒मगृ॑हीतः । उ॒प॒या॒मगृ॑हीतोऽसि । उ॒प॒या॒मगृ॑हीत॒ इत्यु॑पया॒म - गृ॒ही॒तः॒ । अ॒स्य॒फ्सु॒षद᳚म् । अ॒फ्सु॒षद॑म् त्वा । अ॒फ्सु॒षद॒मित्य॑फ्सु - सद᳚म् । त्वा॒ घृ॒त॒सद᳚म् । घृ॒त॒सदं॑ ॅव्योम॒सद᳚म् । घृ॒त॒सद॒मिति॑ घृत - सद᳚म् । व्यो॒म॒सद॒मिन्द्रा॑य । व्यो॒म॒सद॒मिति॑ व्योम - सद᳚म् । इन्द्रा॑य॒ जुष्ट᳚म् । जुष्ट॑म् गृह्णामि । गृ॒ह्णा॒म्ये॒षः । ए॒ष ते᳚ । ते॒ योनिः॑ । योनि॒रिन्द्रा॑य । इन्द्रा॑य त्वा । त्वो॒प॒या॒मगृ॑हीतः । उ॒प॒या॒मगृ॑हीतोऽसि । उ॒प॒या॒मगृ॑हीत॒ इत्यु॑पया॒म - गृ॒ही॒तः॒ । अ॒सि॒ पृ॒थि॒वि॒षद᳚म् । पृ॒थि॒वि॒षद॑म् त्वा । पृ॒थि॒वि॒षद॒मिति॑ पृथिवि - सद᳚म् । त्वा॒ऽन्त॒रि॒क्ष॒सद᳚म् । अ॒न्त॒रि॒क्ष॒सद॑म् नाक॒सद᳚म् । अ॒न्त॒रि॒क्ष॒सद॒मित्य॑न्तरिक्ष - सद᳚म् । ना॒क॒सद॒मिन्द्रा॑य । ना॒क॒सद॒मिति॑ नाक - सद᳚म् । इन्द्रा॑य॒ जुष्ट᳚म् । जुष्ट॑म् गृह्णामि । गृ॒ह्णा॒म्ये॒षः । ए॒ष ते᳚ । ते॒ योनिः॑ । योनि॒रिन्द्रा॑य । इन्द्रा॑य त्वा । त्वेति॑ त्वा ॥ ये ग्रहाः᳚ । ग्रहाः᳚ पञ्चज॒नीनाः᳚ । प॒ञ्च॒ज॒नीना॒ येषा᳚म् । प॒ञ्च॒ज॒नीना॒ इति॑ पञ्च - ज॒नीनाः᳚ । येषा᳚म् ति॒स्रः । ति॒स्रः प॑रम॒जाः । प॒र॒म॒जा इति॑ परम - जाः ॥ दैव्यः॒ कोशः॑ । कोशः॒ समु॑ब्जितः \newline

\textbf{Jatai Paata} \newline

1. उ॒प॒या॒मगृ॑हीतो ऽस्य स्युपया॒मगृ॑हीत उपया॒मगृ॑हीतो ऽसि । \newline
2. उ॒प॒या॒मगृ॑हीत॒ इत्यु॑पया॒म - गृ॒ही॒तः॒ । \newline
3. अ॒सि॒ नृ॒षद॑न् नृ॒षद॑ मस्यसि नृ॒षद᳚म् । \newline
4. नृ॒षद॑म् त्वा त्वा नृ॒षद॑न् नृ॒षद॑म् त्वा । \newline
5. नृ॒षद॒मिति॑ नृ - सद᳚म् । \newline
6. त्वा॒ द्रु॒षद॑म् द्रु॒षद॑म् त्वा त्वा द्रु॒षद᳚म् । \newline
7. द्रु॒षद॑म् भुवन॒सद॑म् भुवन॒सद॑म् द्रु॒षद॑म् द्रु॒षद॑म् भुवन॒सद᳚म् । \newline
8. द्रु॒षद॒मिति॑ द्रु - सद᳚म् । \newline
9. भु॒व॒न॒सद॒ मिन्द्रा॒ये न्द्रा॑य भुवन॒सद॑म् भुवन॒सद॒ मिन्द्रा॑य । \newline
10. भु॒व॒न॒सद॒मिति॑ भुवन - सद᳚म् । \newline
11. इन्द्रा॑य॒ जुष्ट॒म् जुष्ट॒ मिन्द्रा॒ये न्द्रा॑य॒ जुष्ट᳚म् । \newline
12. जुष्ट॑म् गृह्णामि गृह्णामि॒ जुष्ट॒म् जुष्ट॑म् गृह्णामि । \newline
13. गृ॒ह्णा॒म्ये॒ष ए॒ष गृ॑ह्णामि गृह्णाम्ये॒षः । \newline
14. ए॒ष ते॑ त ए॒ष ए॒ष ते᳚ । \newline
15. ते॒ योनि॒र् योनि॑ स्ते ते॒ योनिः॑ । \newline
16. योनि॒ रिन्द्रा॒ये न्द्रा॑य॒ योनि॒र् योनि॒ रिन्द्रा॑य । \newline
17. इन्द्रा॑य त्वा॒ त्वेन्द्रा॒ये न्द्रा॑य त्वा । \newline
18. त्वो॒प॒या॒मगृ॑हीत उपया॒मगृ॑हीत स्त्वा त्वोपया॒मगृ॑हीतः । \newline
19. उ॒प॒या॒मगृ॑हीतो ऽस्यस्युपया॒मगृ॑हीत उपया॒मगृ॑हीतो ऽसि । \newline
20. उ॒प॒या॒मगृ॑हीत॒ इत्यु॑पया॒म - गृ॒ही॒तः॒ । \newline
21. अ॒स्य॒ फ्सु॒षद॑ मफ्सु॒षद॑ मस्य स्यफ्सु॒षद᳚म् । \newline
22. अ॒फ्सु॒षद॑म् त्वा त्वा ऽफ्सु॒षद॑ मफ्सु॒षद॑म् त्वा । \newline
23. अ॒फ्सु॒षद॒मित्य॑फ्सु - सद᳚म् । \newline
24. त्वा॒ घृ॒त॒सद॑म् घृत॒सद॑म् त्वा त्वा घृत॒सद᳚म् । \newline
25. घृ॒त॒सदं॑ ॅव्योम॒सदं॑ ॅव्योम॒सद॑म् घृत॒सद॑म् घृत॒सदं॑ ॅव्योम॒सद᳚म् । \newline
26. घृ॒त॒सद॒मिति॑ घृत - सद᳚म् । \newline
27. व्यो॒म॒सद॒ मिन्द्रा॒ये न्द्रा॑य व्योम॒सदं॑ ॅव्योम॒सद॒ मिन्द्रा॑य । \newline
28. व्यो॒म॒सद॒मिति॑ व्योम - सद᳚म् । \newline
29. इन्द्रा॑य॒ जुष्ट॒म् जुष्ट॒ मिन्द्रा॒ये न्द्रा॑य॒ जुष्ट᳚म् । \newline
30. जुष्ट॑म् गृह्णामि गृह्णामि॒ जुष्ट॒म् जुष्ट॑म् गृह्णामि । \newline
31. गृ॒ह्णा॒ म्ये॒ष ए॒ष गृ॑ह्णामि गृह्णा म्ये॒षः । \newline
32. ए॒ष ते॑ त ए॒ष ए॒ष ते᳚ । \newline
33. ते॒ योनि॒र् योनि॑ स्ते ते॒ योनिः॑ । \newline
34. योनि॒ रिन्द्रा॒ये न्द्रा॑य॒ योनि॒र् योनि॒ रिन्द्रा॑य । \newline
35. इन्द्रा॑य त्वा॒ त्वेन्द्रा॒ये न्द्रा॑य त्वा । \newline
36. त्वो॒प॒या॒मगृ॑हीत उपया॒मगृ॑हीत स्त्वा त्वोपया॒मगृ॑हीतः । \newline
37. उ॒प॒या॒मगृ॑हीतो ऽस्यस्युपया॒मगृ॑हीत उपया॒मगृ॑हीतो ऽसि । \newline
38. उ॒प॒या॒मगृ॑हीत॒ इत्यु॑पया॒म - गृ॒ही॒तः॒ । \newline
39. अ॒सि॒ पृ॒थि॒वि॒षद॑म् पृथिवि॒षद॑ मस्यसि पृथिवि॒षद᳚म् । \newline
40. पृ॒थि॒वि॒षद॑म् त्वा त्वा पृथिवि॒षद॑म् पृथिवि॒षद॑म् त्वा । \newline
41. पृ॒थि॒वि॒षद॒मिति॑ पृथिवि - सद᳚म् । \newline
42. त्वा॒ ऽन्त॒रि॒क्ष॒सद॑ मन्तरिक्ष॒सद॑म् त्वा त्वा ऽन्तरिक्ष॒सद᳚म् । \newline
43. अ॒न्त॒रि॒क्ष॒सद॑म् नाक॒सद॑म् नाक॒सद॑ मन्तरिक्ष॒सद॑ मन्तरिक्ष॒सद॑म् नाक॒सद᳚म् । \newline
44. अ॒न्त॒रि॒क्ष॒सद॒मित्य॑न्तरिक्ष - सद᳚म् । \newline
45. ना॒क॒सद॒ मिन्द्रा॒ये न्द्रा॑य नाक॒सद॑म् नाक॒सद॒ मिन्द्रा॑य । \newline
46. ना॒क॒सद॒मिति॑ नाक - सद᳚म् । \newline
47. इन्द्रा॑य॒ जुष्ट॒म् जुष्ट॒ मिन्द्रा॒ये न्द्रा॑य॒ जुष्ट᳚म् । \newline
48. जुष्ट॑म् गृह्णामि गृह्णामि॒ जुष्ट॒म् जुष्ट॑म् गृह्णामि । \newline
49. गृ॒ह्णा॒ म्ये॒ष ए॒ष गृ॑ह्णामि गृह्णा म्ये॒षः । \newline
50. ए॒ष ते॑ त ए॒ष ए॒ष ते᳚ । \newline
51. ते॒ योनि॒र् योनि॑ स्ते ते॒ योनिः॑ । \newline
52. योनि॒ रिन्द्रा॒ये न्द्रा॑य॒ योनि॒र् योनि॒ रिन्द्रा॑य । \newline
53. इन्द्रा॑य त्वा॒ त्वेन्द्रा॒ये न्द्रा॑य त्वा । \newline
54. त्वेति॑ त्वा । \newline
55. ये ग्रहा॒ ग्रहा॒ ये ये ग्रहाः᳚ । \newline
56. ग्रहाः᳚ पञ्चज॒नीनाः᳚ पञ्चज॒नीना॒ ग्रहा॒ ग्रहाः᳚ पञ्चज॒नीनाः᳚ । \newline
57. प॒ञ्च॒ज॒नीना॒ येषां॒ ॅयेषा᳚म् पञ्चज॒नीनाः᳚ पञ्चज॒नीना॒ येषा᳚म् । \newline
58. प॒ञ्च॒ज॒नीना॒ इति॑ पञ्च - ज॒नीनाः᳚ । \newline
59. येषा᳚म् ति॒स्र स्ति॒स्रो येषां॒ ॅयेषा᳚म् ति॒स्रः । \newline
60. ति॒स्रः प॑रम॒जाः प॑रम॒जा स्ति॒स्र स्ति॒स्रः प॑रम॒जाः । \newline
61. प॒र॒म॒जा इति॑ परम - जाः । \newline
62. दैव्यः॒ कोशः॒ कोशो॒ दैव्यो॒ दैव्यः॒ कोशः॑ । \newline
63. कोशः॒ समु॑ब्जितः॒ समु॑ब्जितः॒ कोशः॒ कोशः॒ समु॑ब्जितः । \newline

\textbf{Ghana Paata } \newline

1. उ॒प॒या॒मगृ॑हीतो ऽस्यस्युपया॒मगृ॑हीत उपया॒मगृ॑हीतो ऽसि नृ॒षद॑म् नृ॒षद॑ मस्युपया॒मगृ॑हीत उपया॒मगृ॑हीतो ऽसि नृ॒षद᳚म् । \newline
2. उ॒प॒या॒मगृ॑हीत॒ इत्यु॑पया॒म - गृ॒ही॒तः॒ । \newline
3. अ॒सि॒ नृ॒षद॑म् नृ॒षद॑ मस्यसि नृ॒षद॑म् त्वा त्वा नृ॒षद॑ मस्यसि नृ॒षद॑म् त्वा । \newline
4. नृ॒षद॑म् त्वा त्वा नृ॒षद॑म् नृ॒षद॑म् त्वा द्रु॒षद॑म् द्रु॒षद॑म् त्वा नृ॒षद॑म् नृ॒षद॑म् त्वा द्रु॒षद᳚म् । \newline
5. नृ॒षद॒मिति॑ नृ - सद᳚म् । \newline
6. त्वा॒ द्रु॒षद॑म् द्रु॒षद॑म् त्वा त्वा द्रु॒षद॑म् भुवन॒सद॑म् भुवन॒सद॑म् द्रु॒षद॑म् त्वा त्वा द्रु॒षद॑म् भुवन॒सद᳚म् । \newline
7. द्रु॒षद॑म् भुवन॒सद॑म् भुवन॒सद॑म् द्रु॒षद॑म् द्रु॒षद॑म् भुवन॒सद॒ मिन्द्रा॒ये न्द्रा॑य भुवन॒सद॑म् द्रु॒षद॑म् द्रु॒षद॑म् भुवन॒सद॒ मिन्द्रा॑य । \newline
8. द्रु॒षद॒मिति॑ द्रु - सद᳚म् । \newline
9. भु॒व॒न॒सद॒ मिन्द्रा॒ये न्द्रा॑य भुवन॒सद॑म् भुवन॒सद॒ मिन्द्रा॑य॒ जुष्ट॒म् जुष्ट॒ मिन्द्रा॑य भुवन॒सद॑म् भुवन॒सद॒ मिन्द्रा॑य॒ जुष्ट᳚म् । \newline
10. भु॒व॒न॒सद॒मिति॑ भुवन - सद᳚म् । \newline
11. इन्द्रा॑य॒ जुष्ट॒म् जुष्ट॒ मिन्द्रा॒ये न्द्रा॑य॒ जुष्ट॑म् गृह्णामि गृह्णामि॒ जुष्ट॒ मिन्द्रा॒ये न्द्रा॑य॒ जुष्ट॑म् गृह्णामि । \newline
12. जुष्ट॑म् गृह्णामि गृह्णामि॒ जुष्ट॒म् जुष्ट॑म् गृह्णाम्ये॒ष ए॒ष गृ॑ह्णामि॒ जुष्ट॒म् जुष्ट॑म् गृह्णाम्ये॒षः । \newline
13. गृ॒ह्णा॒म्ये॒ष ए॒ष गृ॑ह्णामि गृह्णाम्ये॒ष ते॑ त ए॒ष गृ॑ह्णामि गृह्णाम्ये॒ष ते᳚ । \newline
14. ए॒ष ते॑ त ए॒ष ए॒ष ते॒ योनि॒र् योनि॑ स्त ए॒ष ए॒ष ते॒ योनिः॑ । \newline
15. ते॒ योनि॒र् योनि॑ स्ते ते॒ योनि॒ रिन्द्रा॒ये न्द्रा॑य॒ योनि॑स्ते ते॒ योनि॒ रिन्द्रा॑य । \newline
16. योनि॒ रिन्द्रा॒ये न्द्रा॑य॒ योनि॒र् योनि॒ रिन्द्रा॑य त्वा॒ त्वेन्द्रा॑य॒ योनि॒र् योनि॒ रिन्द्रा॑य त्वा । \newline
17. इन्द्रा॑य त्वा॒ त्वेन्द्रा॒ये न्द्रा॑य त्वोपया॒मगृ॑हीत उपया॒मगृ॑हीत॒ स्त्वेन्द्रा॒ये न्द्रा॑य त्वोपया॒मगृ॑हीतः । \newline
18. त्वो॒प॒या॒मगृ॑हीत उपया॒मगृ॑हीतस्त्वा त्वोपया॒मगृ॑हीतो ऽस्यस्युपया॒मगृ॑हीत स्त्वा त्वोपया॒मगृ॑हीतो ऽसि । \newline
19. उ॒प॒या॒मगृ॑हीतो ऽस्यस्युपया॒मगृ॑हीत उपया॒मगृ॑हीतो ऽस्यफ्सु॒षद॑ मफ्सु॒षद॑ मस्युपया॒मगृ॑हीत उपया॒मगृ॑हीतो ऽस्यफ्सु॒षद᳚म् । \newline
20. उ॒प॒या॒मगृ॑हीत॒ इत्यु॑पया॒म - गृ॒ही॒तः॒ । \newline
21. अ॒स्य॒फ्सु॒षद॑ मफ्सु॒षद॑ मस्य स्यफ्सु॒षद॑म् त्वा त्वा ऽफ्सु॒षद॑ मस्य स्यफ्सु॒षद॑म् त्वा । \newline
22. अ॒फ्सु॒षद॑म् त्वा त्वा ऽफ्सु॒षद॑ मफ्सु॒षद॑म् त्वा घृत॒सद॑म् घृत॒सद॑म् त्वा ऽफ्सु॒षद॑ मफ्सु॒षद॑म् त्वा घृत॒सद᳚म् । \newline
23. अ॒फ्सु॒षद॒मित्य॑फ्सु - सद᳚म् । \newline
24. त्वा॒ घृ॒त॒सद॑म् घृत॒सद॑म् त्वा त्वा घृत॒सदं॑ ॅव्योम॒सदं॑ ॅव्योम॒सद॑म् घृत॒सद॑म् त्वा त्वा घृत॒सदं॑ ॅव्योम॒सद᳚म् । \newline
25. घृ॒त॒सदं॑ ॅव्योम॒सदं॑ ॅव्योम॒सद॑म् घृत॒सद॑म् घृत॒सदं॑ ॅव्योम॒सद॒ मिन्द्रा॒ये न्द्रा॑य व्योम॒सद॑म् घृत॒सद॑म् घृत॒सदं॑ ॅव्योम॒सद॒ मिन्द्रा॑य । \newline
26. घृ॒त॒सद॒मिति॑ घृत - सद᳚म् । \newline
27. व्यो॒म॒सद॒ मिन्द्रा॒ये न्द्रा॑य व्योम॒सदं॑ ॅव्योम॒सद॒ मिन्द्रा॑य॒ जुष्ट॒म् जुष्ट॒ मिन्द्रा॑य व्योम॒सदं॑ ॅव्योम॒सद॒ मिन्द्रा॑य॒ जुष्ट᳚म् । \newline
28. व्यो॒म॒सद॒मिति॑ व्योम - सद᳚म् । \newline
29. इन्द्रा॑य॒ जुष्ट॒म् जुष्ट॒ मिन्द्रा॒ये न्द्रा॑य॒ जुष्ट॑म् गृह्णामि गृह्णामि॒ जुष्ट॒ मिन्द्रा॒ये न्द्रा॑य॒ जुष्ट॑म् गृह्णामि । \newline
30. जुष्ट॑म् गृह्णामि गृह्णामि॒ जुष्ट॒म् जुष्ट॑म् गृह्णाम्ये॒ष ए॒ष गृ॑ह्णामि॒ जुष्ट॒म् जुष्ट॑म् गृह्णाम्ये॒षः । \newline
31. गृ॒ह्णा॒म्ये॒ष ए॒ष गृ॑ह्णामि गृह्णाम्ये॒ष ते॑ त ए॒ष गृ॑ह्णामि गृह्णाम्ये॒ष ते᳚ । \newline
32. ए॒ष ते॑ त ए॒ष ए॒ष ते॒ योनि॒र् योनि॑स्त ए॒ष ए॒ष ते॒ योनिः॑ । \newline
33. ते॒ योनि॒र् योनि॑स्ते ते॒ योनि॒ रिन्द्रा॒ये न्द्रा॑य॒ योनि॑स्ते ते॒ योनि॒ रिन्द्रा॑य । \newline
34. योनि॒ रिन्द्रा॒ये न्द्रा॑य॒ योनि॒र् योनि॒ रिन्द्रा॑य त्वा॒ त्वेन्द्रा॑य॒ योनि॒र् योनि॒ रिन्द्रा॑य त्वा । \newline
35. इन्द्रा॑य त्वा॒ त्वेन्द्रा॒ये न्द्रा॑य त्वोपया॒मगृ॑हीत उपया॒मगृ॑हीत॒ स्त्वेन्द्रा॒ये न्द्रा॑य त्वोपया॒मगृ॑हीतः । \newline
36. त्वो॒प॒या॒मगृ॑हीत उपया॒मगृ॑हीत स्त्वा त्वोपया॒मगृ॑हीतो ऽस्यस्युपया॒मगृ॑हीत स्त्वा त्वोपया॒मगृ॑हीतो ऽसि । \newline
37. उ॒प॒या॒मगृ॑हीतो ऽस्यस्युपया॒मगृ॑हीत उपया॒मगृ॑हीतो ऽसि पृथिवि॒षद॑म् पृथिवि॒षद॑ मस्युपया॒मगृ॑हीत उपया॒मगृ॑हीतो ऽसि पृथिवि॒षद᳚म् । \newline
38. उ॒प॒या॒मगृ॑हीत॒ इत्यु॑पया॒म - गृ॒ही॒तः॒ । \newline
39. अ॒सि॒ पृ॒थि॒वि॒षद॑म् पृथिवि॒षद॑ मस्यसि पृथिवि॒षद॑म् त्वा त्वा पृथिवि॒षद॑ मस्यसि पृथिवि॒षद॑म् त्वा । \newline
40. पृ॒थि॒वि॒षद॑म् त्वा त्वा पृथिवि॒षद॑म् पृथिवि॒षद॑म् त्वा ऽन्तरिक्ष॒सद॑ मन्तरिक्ष॒सद॑म् त्वा पृथिवि॒षद॑म् पृथिवि॒षद॑म् त्वा ऽन्तरिक्ष॒सद᳚म् । \newline
41. पृ॒थि॒वि॒षद॒मिति॑ पृथिवि - सद᳚म् । \newline
42. त्वा॒ ऽन्त॒रि॒क्ष॒सद॑ मन्तरिक्ष॒सद॑म् त्वा त्वा ऽन्तरिक्ष॒सद॑म् नाक॒सद॑म् नाक॒सद॑ मन्तरिक्ष॒सद॑म् त्वा त्वा ऽन्तरिक्ष॒सद॑न्नाक॒सद᳚म् । \newline
43. अ॒न्त॒रि॒क्ष॒सद॑न् नाक॒सद॑न् नाक॒सद॑ मन्तरिक्ष॒सद॑ मन्तरिक्ष॒सदन् नाक॒सद॒ मिन्द्रा॒ये न्द्रा॑य नाक॒सद॑ मन्तरिक्ष॒सद॑ मन्तरिक्ष॒सद॑न् नाक॒सद॒ मिन्द्रा॑य । \newline
44. अ॒न्त॒रि॒क्ष॒सद॒मित्य॑न्तरिक्ष - सद᳚म् । \newline
45. ना॒क॒सद॒ मिन्द्रा॒ये न्द्रा॑य नाक॒सद॑न् नाक॒सद॒ मिन्द्रा॑य॒ जुष्ट॒म् जुष्ट॒ मिन्द्रा॑य नाक॒सद॑न् नाक॒सद॒ मिन्द्रा॑य॒ जुष्ट᳚म् । \newline
46. ना॒क॒सद॒मिति॑ नाक - सद᳚म् । \newline
47. इन्द्रा॑य॒ जुष्ट॒म् जुष्ट॒ मिन्द्रा॒ये न्द्रा॑य॒ जुष्ट॑म् गृह्णामि गृह्णामि॒ जुष्ट॒ मिन्द्रा॒ये न्द्रा॑य॒ जुष्ट॑म् गृह्णामि । \newline
48. जुष्ट॑म् गृह्णामि गृह्णामि॒ जुष्ट॒म् जुष्ट॑म् गृह्णाम्ये॒ष ए॒ष गृ॑ह्णामि॒ जुष्ट॒म् जुष्ट॑म् गृह्णाम्ये॒षः । \newline
49. गृ॒ह्णा॒म्ये॒ष ए॒ष गृ॑ह्णामि गृह्णाम्ये॒ष ते॑ त ए॒ष गृ॑ह्णामि गृह्णाम्ये॒ष ते᳚ । \newline
50. ए॒ष ते॑ त ए॒ष ए॒ष ते॒ योनि॒र् योनि॑ स्त ए॒ष ए॒ष ते॒ योनिः॑ । \newline
51. ते॒ योनि॒र् योनि॑ स्ते ते॒ योनि॒ रिन्द्रा॒ये न्द्रा॑य॒ योनि॑ स्ते ते॒ योनि॒ रिन्द्रा॑य । \newline
52. योनि॒ रिन्द्रा॒ये न्द्रा॑य॒ योनि॒र् योनि॒ रिन्द्रा॑य त्वा॒ त्वेन्द्रा॑य॒ योनि॒र् योनि॒ रिन्द्रा॑य त्वा । \newline
53. इन्द्रा॑य त्वा॒ त्वेन्द्रा॒ये न्द्रा॑य त्वा । \newline
54. त्वेति॑ त्वा । \newline
55. ये ग्रहा॒ ग्रहा॒ ये ये ग्रहाः᳚ पञ्चज॒नीनाः᳚ पञ्चज॒नीना॒ ग्रहा॒ ये ये ग्रहाः᳚ पञ्चज॒नीनाः᳚ । \newline
56. ग्रहाः᳚ पञ्चज॒नीनाः᳚ पञ्चज॒नीना॒ ग्रहा॒ ग्रहाः᳚ पञ्चज॒नीना॒ येषां॒ ॅयेषा᳚म् पञ्चज॒नीना॒ ग्रहा॒ ग्रहाः᳚ पञ्चज॒नीना॒ येषा᳚म् । \newline
57. प॒ञ्च॒ज॒नीना॒ येषां॒ ॅयेषा᳚म् पञ्चज॒नीनाः᳚ पञ्चज॒नीना॒ येषा᳚म् ति॒स्र स्ति॒स्रो येषा᳚म् पञ्चज॒नीनाः᳚ पञ्चज॒नीना॒ येषा᳚म् ति॒स्रः । \newline
58. प॒ञ्च॒ज॒नीना॒ इति॑ पञ्च - ज॒नीनाः᳚ । \newline
59. येषा᳚म् ति॒स्र स्ति॒स्रो येषां॒ ॅयेषा᳚म् ति॒स्रः प॑रम॒जाः प॑रम॒जा स्ति॒स्रो येषां॒ ॅयेषा᳚म् ति॒स्रः प॑रम॒जाः । \newline
60. ति॒स्रः प॑रम॒जाः प॑रम॒जा स्ति॒स्र स्ति॒स्रः प॑रम॒जाः । \newline
61. प॒र॒म॒जा इति॑ परम - जाः । \newline
62. दैव्यः॒ कोशः॒ कोशो॒ दैव्यो॒ दैव्यः॒ कोशः॒ समु॑ब्जितः॒ समु॑ब्जितः॒ कोशो॒ दैव्यो॒ दैव्यः॒ कोशः॒ समु॑ब्जितः । \newline
63. कोशः॒ समु॑ब्जितः॒ समु॑ब्जितः॒ कोशः॒ कोशः॒ समु॑ब्जितः । \newline
\pagebreak
\markright{ TS 1.7.12.2  \hfill https://www.vedavms.in \hfill}
\addcontentsline{toc}{section}{ TS 1.7.12.2 }
\section*{ TS 1.7.12.2 }

\textbf{TS 1.7.12.2 } \newline
\textbf{Samhita Paata} \newline

समु॑ब्जितः । तेषां॒ ॅविशि॑प्रियाणा॒-मिष॒मूर्जꣳ॒॒ सम॑ग्रभी-मे॒ष ते॒ योनि॒रिन्द्रा॑य त्वा ॥ अ॒पाꣳ रस॒मुद्व॑यसꣳ॒॒ सूर्य॑रश्मिꣳ स॒माभृ॑तं । अ॒पाꣳ रस॑स्य॒ यो रस॒स्तं ॅवो॑ गृह्णाम्युत्त॒ममे॒ष ते॒ योनि॒रिन्द्रा॑य त्वा ॥ अ॒या वि॒ष्ठा ज॒नय॒न् कर्व॑राणि॒ स हि घृणि॑रु॒रुर् वरा॑य गा॒तुः । स प्रत्युदै᳚द् ध॒रुणो मद्ध्वो॒ अग्रꣳ॒॒ स्वायां॒ ॅयत् त॒नुवां᳚ ( ) त॒नूमैर॑यत । उ॒प॒या॒मगृ॑हीतोऽसि प्र॒जाप॑तये त्वा॒ जुष्टं॑ गृह्णाम्ये॒ष ते॒ योनिः॑ प्र॒जाप॑तये त्वा ॥ \newline

\textbf{Pada Paata} \newline

समु॑ब्जित॒ इति॒ सं - उ॒ब्जि॒तः॒ ॥ तेषा᳚म् । विशि॑प्रियाणा॒मिति॒ वि - शि॒प्रि॒या॒णा॒म् । इष᳚म् । ऊर्ज᳚म् । समिति॑ । अ॒ग्र॒भी॒म् । ए॒षः । ते॒ । योनिः॑ । इन्द्रा॑य । त्वा॒ ॥ अ॒पाम् । रस᳚म् । उद्व॑यस॒मित्युत् - व॒य॒स॒म् । सूर्य॑रश्मि॒मिति॒ सूर्य॑ - र॒श्मि॒म् । स॒माभृ॑त॒मिति॑ सं - आभृ॑तम् ॥ अ॒पाम् । रस॑स्य । यः । रसः॑ । तम् । वः॒ । गृ॒ह्णा॒मी॒ । उ॒त्त॒ममित्यु॑त् - त॒मम् । ए॒षः । ते॒ । योनिः॑ । इन्द्रा॑य । त्वा॒ ॥ अ॒या । वि॒ष्ठा इति॑ वि-स्थाः । ज॒नयन्न्॑ । कर्व॑राणि । सः । हि । घृणिः॑ । उ॒रुः । वरा॑य । गा॒तुः ॥ सः । प्रति॑ । उदिति॑ । ऐ॒त् । ध॒रुणः॑ । मद्ध्वः॑ । अग्र᳚म् । स्वाया᳚म् । यत् । त॒नुवां᳚ ( ) । त॒नूम् । ऐर॑यत ॥ उ॒प॒या॒मगृ॑हीत॒ इत्यु॑पया॒म - गृ॒ही॒तः॒ । अ॒सि॒ । प्र॒जाप॑तय॒ इति॑ प्र॒जा - प॒त॒ये॒ । त्वा॒ । जुष्ट᳚म् । गृ॒ह्णा॒मि॒ । ए॒षः । ते॒ । योनिः॑ । प्र॒जाप॑तय॒ इति॑ प्र॒जा - प॒त॒ये॒ । त्वा॒ ॥  \newline


\textbf{Krama Paata} \newline

समु॑ब्जित॒ इति॒ सं - उ॒ब्जि॒तः॒ ॥ तेषां॒ ॅविशि॑प्रियाणाम् । विशि॑प्रियाणा॒मिष᳚म् । विशि॑प्रियाणा॒मिति॒ वि - शि॒प्रि॒या॒णा॒म् । इष॒मूर्ज᳚म् । ऊर्जꣳ॒॒ सम् । सम॑ग्रभीम् । अ॒ग्र॒भी॒मे॒षः । ए॒ष ते᳚ । ते॒ योनिः॑ । योनि॒रिन्द्रा॑य । इन्द्रा॑य त्वा । त्वेति॑ त्वा ॥ अ॒पाꣳ रस᳚म् । रस॒मुद्व॑यसम् । उद्व॑यसꣳ॒॒ सूर्य॑रश्मिम् । उद्व॑यस॒मित्युत् - व॒य॒स॒म् । सूर्य॑रश्मिꣳ स॒माभृ॑तम् । सूर्य॑रश्मि॒मिति॒ सूर्य॑ - र॒श्मि॒म् । स॒माभृ॑त॒मिति॑ सम् - आभृ॑तम् ॥ अ॒पाꣳ रस॑स्य । रस॑स्य॒ यः । यो रसः॑ । रस॒स्तम् । तं ॅवः॑ । वो॒ गृ॒ह्णा॒मि॒ । गृ॒ह्णा॒म्यु॒त्त॒मम् । उ॒त्त॒ममे॒षः । उ॒त्त॒ममित्यु॑त् - त॒मम् । ए॒ष ते᳚ । ते॒ योनिः॑ । योनि॒रिन्द्रा॑य । इन्द्रा॑य त्वा । त्वेति॑ त्वा ॥ अ॒या वि॒ष्ठाः । वि॒ष्ठा ज॒नयन्न्॑ । वि॒ष्ठा इति॑ वि - स्थाः । ज॒नय॒न् कर्व॑राणि । कर्व॑राणि॒ सः । स हि । हि घृणिः॑ । घृणि॑रु॒रुः । उ॒रुर् वरा॑य । वरा॑य गा॒तुः । गा॒तुरिति॑ गा॒तुः ॥ स प्रति॑ । प्रत्युत् । उदै᳚त् । ऐ॒द्ध॒रुणः॑ । ध॒रुणो॒ मद्ध्वः॑ । मद्ध्वो॒ अग्र᳚म् । अग्रꣳ॒॒ स्वाया᳚म् । स्वायां॒ ॅयत् । यत् त॒नुवा᳚म् ( ) । त॒नुवा᳚म् त॒नूम् । त॒नूमैर॑यत । ऐर॑य॒तेत्यैर॑यत ॥ उ॒प॒या॒मगृ॑हीतोऽसि । उ॒प॒या॒मगृ॑हीत॒ इत्यु॑पया॒म - गृ॒ही॒तः॒ । अ॒सि॒ प्र॒जाप॑तये । प्र॒जाप॑तये त्वा । प्र॒जाप॑तय॒ इति॑ प्र॒जा - प॒त॒ये॒ । त्वा॒ जुष्ट᳚म् । जुष्ट॑म् गृह्णामि । गृ॒ह्णा॒म्ये॒षः । ए॒ष ते᳚ । ते॒ योनिः॑ । योनिः॑ प्र॒जाप॑तये । प्र॒जाप॑तये त्वा । प्र॒जाप॑तय॒ इति॑ प्र॒जा - प॒त॒ये॒ । त्वेति॑ त्वा । \newline

\textbf{Jatai Paata} \newline

1. समु॑ब्जित॒ इति॒ सं - उ॒ब्जि॒तः॒ । \newline
2. तेषां॒ ॅविशि॑प्रियाणां॒ ॅविशि॑प्रियाणा॒म् तेषा॒म् तेषां॒ ॅविशि॑प्रियाणाम् । \newline
3. विशि॑प्रियाणा॒ मिष॒ मिषं॒ ॅविशि॑प्रियाणां॒ ॅविशि॑प्रियाणा॒ मिष᳚म् । \newline
4. विशि॑प्रियाणा॒मिति॒ वि - शि॒प्रि॒या॒णा॒म् । \newline
5. इष॒ मूर्ज॒ मूर्ज॒ मिष॒ मिष॒ मूर्ज᳚म् । \newline
6. ऊर्जꣳ॒॒ सꣳ स मूर्ज॒ मूर्जꣳ॒॒ सम् । \newline
7. स म॑ग्रभी मग्रभीꣳ॒॒ सꣳ स म॑ग्रभीम् । \newline
8. अ॒ग्र॒भी॒ मे॒ष ए॒षो अ॑ग्रभी मग्रभी मे॒षः । \newline
9. ए॒ष ते॑ त ए॒ष ए॒ष ते᳚ । \newline
10. ते॒ योनि॒र् योनि॑ स्ते ते॒ योनिः॑ । \newline
11. योनि॒ रिन्द्रा॒ये न्द्रा॑य॒ योनि॒र् योनि॒ रिन्द्रा॑य । \newline
12. इन्द्रा॑य त्वा॒ त्वेन्द्रा॒ये न्द्रा॑य त्वा । \newline
13. त्वेति॑ त्वा । \newline
14. अ॒पाꣳ रसꣳ॒॒ रस॑ म॒पा म॒पाꣳ रस᳚म् । \newline
15. रस॒ मुद्व॑यस॒ मुद्व॑यसꣳ॒॒ रसꣳ॒॒ रस॒ मुद्व॑यसम् । \newline
16. उद्व॑यसꣳ॒॒ सूर्य॑रश्मिꣳ॒॒ सूर्य॑रश्मि॒ मुद्व॑यस॒ मुद्व॑यसꣳ॒॒ सूर्य॑रश्मिम् । \newline
17. उद्व॑यस॒मित्युत् - व॒य॒स॒म् । \newline
18. सूर्य॑रश्मिꣳ स॒माभृ॑तꣳ स॒माभृ॑तꣳ॒॒ सूर्य॑रश्मिꣳ॒॒ सूर्य॑रश्मिꣳ स॒माभृ॑तम् । \newline
19. सूर्य॑रश्मि॒मिति॒ सूर्य॑ - र॒श्मि॒म् । \newline
20. स॒माभृ॑त॒मिति॑ सं - आभृ॑तम् । \newline
21. अ॒पाꣳ रस॑स्य॒ रस॑स्या॒पा म॒पाꣳ रस॑स्य । \newline
22. रस॑स्य॒ यो यो रस॑स्य॒ रस॑स्य॒ यः । \newline
23. यो रसो॒ रसो॒ यो यो रसः॑ । \newline
24. रस॒ स्तम् तꣳ रसो॒ रस॒ स्तम् । \newline
25. तं ॅवो॑ व॒ स्तम् तं ॅवः॑ । \newline
26. वो॒ गृ॒ह्णा॒मि॒ गृ॒ह्णा॒मि॒ वो॒ वो॒ गृ॒ह्णा॒मि॒ । \newline
27. गृ॒ह्णा॒ म्यु॒त्त॒म मु॑त्त॒मम् गृ॑ह्णामि गृह्णा म्युत्त॒मम् । \newline
28. उ॒त्त॒म मे॒ष ए॒ष उ॑त्त॒म मु॑त्त॒म मे॒षः । \newline
29. उ॒त्त॒ममित्यु॑त् - त॒मम् । \newline
30. ए॒ष ते॑ त ए॒ष ए॒ष ते᳚ । \newline
31. ते॒ योनि॒र् योनि॑ स्ते ते॒ योनिः॑ । \newline
32. योनि॒ रिन्द्रा॒ये न्द्रा॑य॒ योनि॒र् योनि॒ रिन्द्रा॑य । \newline
33. इन्द्रा॑य त्वा॒ त्वेन्द्रा॒ये न्द्रा॑य त्वा । \newline
34. त्वेति॑ त्वा । \newline
35. अ॒या वि॒ष्ठा वि॒ष्ठा अ॒या ऽया वि॒ष्ठाः । \newline
36. वि॒ष्ठा ज॒नय॑न् ज॒नय॑न्. वि॒ष्ठा वि॒ष्ठा ज॒नयन्न्॑ । \newline
37. वि॒ष्ठा इति॑ वि - स्थाः । \newline
38. ज॒नय॒न् कर्व॑राणि॒ कर्व॑राणि ज॒नय॑न् ज॒नय॒न् कर्व॑राणि । \newline
39. कर्व॑राणि॒ स स कर्व॑राणि॒ कर्व॑राणि॒ सः । \newline
40. स हि हि स स हि । \newline
41. हि घृणि॒र् घृणि॒र्॒. हि हि घृणिः॑ । \newline
42. घृणि॑ रु॒रु रु॒रुर् घृणि॒र् घृणि॑ रु॒रुः । \newline
43. उ॒रुर् वरा॑य॒ वरा॑ यो॒ रुरु॒ रुर् वरा॑य । \newline
44. वरा॑य गा॒तुर् गा॒तुर् वरा॑य॒ वरा॑य गा॒तुः । \newline
45. गा॒तुरिति॑ गा॒तुः । \newline
46. स प्रति॒ प्रति॒ स स प्रति॑ । \newline
47. प्रत्यु दुत् प्रति॒ प्रत्युत् । \newline
48. उ दै॑ दै॒ दुदु दै᳚त् । \newline
49. ऐ॒द् ध॒रुणो॑ ध॒रुण॑ ऐदैद् ध॒रुणः॑ । \newline
50. ध॒रुणो॒ मद्ध्वो॒ मद्ध्वो॑ ध॒रुणो॑ ध॒रुणो॒ मद्ध्वः॑ । \newline
51. मद्ध्वो॒ अग्र॒ मग्र॒म् मद्ध्वो॒ मद्ध्वो॒ अग्र᳚म् । \newline
52. अग्रꣳ॒॒ स्वायाꣳ॒॒ स्वाया॒ मग्र॒ मग्रꣳ॒॒ स्वाया᳚म् । \newline
53. स्वायां॒ ॅयद् यथ् स्वायाꣳ॒॒ स्वायां॒ ॅयत् । \newline
54. यत् त॒नुवां᳚ त॒नुवां॒ यद् यत् त॒नुवां᳚ । \newline
55. त॒नुवां᳚ त॒नूम् त॒नूम् त॒नुवां᳚ त॒नुवां᳚ त॒नूम् । \newline
56. त॒नू मैर॑य॒ तैर॑यत त॒नूम् त॒नू मैर॑यत । \newline
57. ऐर॑य॒ ते त्यैर॑यत । \newline
58. उ॒प॒या॒मगृ॑हीतो ऽस्य स्युपया॒मगृ॑हीत उपया॒मगृ॑हीतो ऽसि । \newline
59. उ॒प॒या॒मगृ॑हीत॒ इत्यु॑पया॒म - गृ॒ही॒तः॒ । \newline
60. अ॒सि॒ प्र॒जाप॑तये प्र॒जाप॑तये ऽस्यसि प्र॒जाप॑तये । \newline
61. प्र॒जाप॑तये त्वा त्वा प्र॒जाप॑तये प्र॒जाप॑तये त्वा । \newline
62. प्र॒जाप॑तय॒ इति॑ प्र॒जा - प॒त॒ये॒ । \newline
63. त्वा॒ जुष्ट॒म् जुष्ट॑म् त्वा त्वा॒ जुष्ट᳚म् । \newline
64. जुष्ट॑म् गृह्णामि गृह्णामि॒ जुष्ट॒म् जुष्ट॑म् गृह्णामि । \newline
65. गृ॒ह्णा॒ म्ये॒ष ए॒ष गृ॑ह्णामि गृह्णा म्ये॒षः । \newline
66. ए॒ष ते॑ त ए॒ष ए॒ष ते᳚ । \newline
67. ते॒ योनि॒र् योनि॑ स्ते ते॒ योनिः॑ । \newline
68. योनिः॑ प्र॒जाप॑तये प्र॒जाप॑तये॒ योनि॒र् योनिः॑ प्र॒जाप॑तये । \newline
69. प्र॒जाप॑तये त्वा त्वा प्र॒जाप॑तये प्र॒जाप॑तये त्वा । \newline
70. प्र॒जाप॑तय॒ इति॑ प्र॒जा - प॒त॒ये॒ । \newline
71. त्वेति॑ त्वा । \newline

\textbf{Ghana Paata } \newline

1. समु॑ब्जित॒ इति॒ सं - उ॒ब्जि॒तः॒ । \newline
2. तेषां॒ ॅविशि॑प्रियाणां॒ ॅविशि॑प्रियाणा॒म् तेषा॒म् तेषां॒ ॅविशि॑प्रियाणा॒ मिष॒ मिषं॒ ॅविशि॑प्रियाणा॒म् तेषा॒म् तेषां॒ ॅविशि॑प्रियाणा॒ मिष᳚म् । \newline
3. विशि॑प्रियाणा॒ मिष॒ मिषं॒ ॅविशि॑प्रियाणां॒ ॅविशि॑प्रियाणा॒ मिष॒ मूर्ज॒ मूर्ज॒ मिषं॒ ॅविशि॑प्रियाणां॒ ॅविशि॑प्रियाणा॒ मिष॒ मूर्ज᳚म् । \newline
4. विशि॑प्रियाणा॒मिति॒ वि - शि॒प्रि॒या॒णा॒म् । \newline
5. इष॒ मूर्ज॒ मूर्ज॒ मिष॒ मिष॒ मूर्ज॒(ग्म्॒) सꣳ स मूर्ज॒ मिष॒ मिष॒ मूर्ज॒(ग्म्॒) सम् । \newline
6. ऊर्ज॒(ग्म्॒) सꣳ स मूर्ज॒ मूर्ज॒(ग्म्॒) स म॑ग्रभी मग्रभी॒(ग्म्॒) स मूर्ज॒ मूर्ज॒(ग्म्॒) स म॑ग्रभीम् । \newline
7. स म॑ग्रभी मग्रभी॒(ग्म्॒) सꣳ स म॑ग्रभी मे॒ष ए॒षो अ॑ग्रभी॒(ग्म्॒) सꣳ स म॑ग्रभी मे॒षः । \newline
8. अ॒ग्र॒भी॒ मे॒ष ए॒षो अ॑ग्रभी मग्रभी मे॒ष ते॑ त ए॒षो अ॑ग्रभी मग्रभी मे॒ष ते᳚ । \newline
9. ए॒ष ते॑ त ए॒ष ए॒ष ते॒ योनि॒र् योनि॑ स्त ए॒ष ए॒ष ते॒ योनिः॑ । \newline
10. ते॒ योनि॒र् योनि॑ स्ते ते॒ योनि॒ रिन्द्रा॒ये न्द्रा॑य॒ योनि॑स्ते ते॒ योनि॒ रिन्द्रा॑य । \newline
11. योनि॒ रिन्द्रा॒ये न्द्रा॑य॒ योनि॒र् योनि॒ रिन्द्रा॑य त्वा॒ त्वेन्द्रा॑य॒ योनि॒र् योनि॒ रिन्द्रा॑य त्वा । \newline
12. इन्द्रा॑य त्वा॒ त्वेन्द्रा॒ये न्द्रा॑य त्वा । \newline
13. त्वेति॑ त्वा । \newline
14. अ॒पाꣳ रस॒(ग्म्॒) रस॑ म॒पा म॒पाꣳ रस॒ मुद्व॑यस॒ मुद्व॑यस॒(ग्म्॒) रस॑ म॒पा म॒पाꣳ रस॒ मुद्व॑यसम् । \newline
15. रस॒ मुद्व॑यस॒ मुद्व॑यस॒(ग्म्॒) रस॒(ग्म्॒) रस॒ मुद्व॑यस॒(ग्म्॒) सूर्य॑रश्मि॒(ग्म्॒) सूर्य॑रश्मि॒ मुद्व॑यस॒(ग्म्॒) रस॒(ग्म्॒) रस॒ मुद्व॑यस॒(ग्म्॒) सूर्य॑रश्मिम् । \newline
16. उद्व॑यस॒(ग्म्॒) सूर्य॑रश्मि॒(ग्म्॒) सूर्य॑रश्मि॒ मुद्व॑यस॒ मुद्व॑यस॒(ग्म्॒) सूर्य॑रश्मिꣳ स॒माभृ॑तꣳ स॒माभृ॑त॒(ग्म्॒) सूर्य॑रश्मि॒ मुद्व॑यस॒ मुद्व॑यस॒(ग्म्॒) सूर्य॑रश्मिꣳ स॒माभृ॑तम् । \newline
17. उद्व॑यस॒मित्युत् - व॒य॒स॒म् । \newline
18. सूर्य॑रश्मिꣳ स॒माभृ॑तꣳ स॒माभृ॑त॒(ग्म्॒) सूर्य॑रश्मि॒(ग्म्॒) सूर्य॑रश्मिꣳ स॒माभृ॑तम् । \newline
19. सूर्य॑रश्मि॒मिति॒ सूर्य॑ - र॒श्मि॒म् । \newline
20. स॒माभृ॑त॒मिति॑ सं - आभृ॑तम् । \newline
21. अ॒पाꣳ रस॑स्य॒ रस॑स्या॒पा म॒पाꣳ रस॑स्य॒ यो यो रस॑स्या॒पा म॒पाꣳ रस॑स्य॒ यः । \newline
22. रस॑स्य॒ यो यो रस॑स्य॒ रस॑स्य॒ यो रसो॒ रसो॒ यो रस॑स्य॒ रस॑स्य॒ यो रसः॑ । \newline
23. यो रसो॒ रसो॒ यो यो रस॒ स्तम् तꣳ रसो॒ यो यो रस॒ स्तम् । \newline
24. रस॒ स्तम् तꣳ रसो॒ रस॒ स्तं ॅवो॑ व॒स्तꣳ रसो॒ रस॒ स्तं ॅवः॑ । \newline
25. तं ॅवो॑ व॒स्तम् तं ॅवो॑ गृह्णामि गृह्णामि व॒स्तम् तं ॅवो॑ गृह्णामि । \newline
26. वो॒ गृ॒ह्णा॒मि॒ गृ॒ह्णा॒मि॒ वो॒ वो॒ गृ॒ह्णा॒ म्यु॒त्त॒म मु॑त्त॒मम् गृ॑ह्णामि वो वो गृह्णा म्युत्त॒मम् । \newline
27. गृ॒ह्णा॒ म्यु॒त्त॒म मु॑त्त॒मम् गृ॑ह्णामि गृह्णा म्युत्त॒म मे॒ष ए॒ष उ॑त्त॒मम् गृ॑ह्णामि गृह्णा म्युत्त॒म मे॒षः । \newline
28. उ॒त्त॒म मे॒ष ए॒ष उ॑त्त॒म मु॑त्त॒म मे॒ष ते॑ त ए॒ष उ॑त्त॒म मु॑त्त॒म मे॒ष ते᳚ । \newline
29. उ॒त्त॒ममित्यु॑त् - त॒मम् । \newline
30. ए॒ष ते॑ त ए॒ष ए॒ष ते॒ योनि॒र् योनि॑ स्त ए॒ष ए॒ष ते॒ योनिः॑ । \newline
31. ते॒ योनि॒र् योनि॑ स्ते ते॒ योनि॒ रिन्द्रा॒ये न्द्रा॑य॒ योनि॑ स्ते ते॒ योनि॒ रिन्द्रा॑य । \newline
32. योनि॒ रिन्द्रा॒ये न्द्रा॑य॒ योनि॒र् योनि॒ रिन्द्रा॑य त्वा॒ त्वेन्द्रा॑य॒ योनि॒र् योनि॒ रिन्द्रा॑य त्वा । \newline
33. इन्द्रा॑य त्वा॒ त्वेन्द्रा॒ये न्द्रा॑य त्वा । \newline
34. त्वेति॑ त्वा । \newline
35. अ॒या वि॒ष्ठा वि॒ष्ठा अ॒या ऽया वि॒ष्ठा ज॒नय॑न् ज॒नय॑न्. वि॒ष्ठा अ॒या ऽया वि॒ष्ठा ज॒नयन्न्॑ । \newline
36. वि॒ष्ठा ज॒नय॑न् ज॒नय॑न्. वि॒ष्ठा वि॒ष्ठा ज॒नय॒न् कर्व॑राणि॒ कर्व॑राणि ज॒नय॑न्. वि॒ष्ठा वि॒ष्ठा ज॒नय॒न् कर्व॑राणि । \newline
37. वि॒ष्ठा इति॑ वि - स्थाः । \newline
38. ज॒नय॒न् कर्व॑राणि॒ कर्व॑राणि ज॒नय॑न् ज॒नय॒न् कर्व॑राणि॒ स स कर्व॑राणि ज॒नय॑न् ज॒नय॒न् कर्व॑राणि॒ सः । \newline
39. कर्व॑राणि॒ स स कर्व॑राणि॒ कर्व॑राणि॒ स हि हि स कर्व॑राणि॒ कर्व॑राणि॒ स हि । \newline
40. स हि हि स स हि घृणि॒र् घृणि॒र्॒. हि स स हि घृणिः॑ । \newline
41. हि घृणि॒र् घृणि॒र्॒. हि हि घृणि॑ रु॒रु रु॒रुर् घृणि॒र्॒. हि हि घृणि॑ रु॒रुः । \newline
42. घृणि॑ रु॒रु रु॒रुर् घृणि॒र् घृणि॑ रु॒रुर् वरा॑य॒ वरा॑यो॒रुर् घृणि॒र् घृणि॑ रु॒रुर् वरा॑य । \newline
43. उ॒रुर् वरा॑य॒ वरा॑यो॒ रुरु॒रुर् वरा॑य गा॒तुर् गा॒तुर् वरा॑यो॒ रुरु॒रुर् वरा॑य गा॒तुः । \newline
44. वरा॑य गा॒तुर् गा॒तुर् वरा॑य॒ वरा॑य गा॒तुः । \newline
45. गा॒तुरिति॑ गा॒तुः । \newline
46. स प्रति॒ प्रति॒ स स प्रत्युदुत् प्रति॒ स स प्रत्युत् । \newline
47. प्रत्युदुत् प्रति॒ प्रत्यु दै॑दै॒ दुत् प्रति॒ प्रत्यु दै᳚त् । \newline
48. उ दै॑दै॒ दुदु दै᳚द् ध॒रुणो॑ ध॒रुण॑ ऐ॒दुदु दै᳚द् ध॒रुणः॑ । \newline
49. ऐ॒द् ध॒रुणो॑ ध॒रुण॑ ऐदैद् ध॒रुणो॒ मद्ध्वो॒ मद्ध्वो॑ ध॒रुण॑ ऐदैद् ध॒रुणो॒ मद्ध्वः॑ । \newline
50. ध॒रुणो॒ मद्ध्वो॒ मद्ध्वो॑ ध॒रुणो॑ ध॒रुणो॒ मद्ध्वो॒ अग्र॒ मग्र॒म् मद्ध्वो॑ ध॒रुणो॑ ध॒रुणो॒ मद्ध्वो॒ अग्र᳚म् । \newline
51. मद्ध्वो॒ अग्र॒ मग्र॒म् मद्ध्वो॒ मद्ध्वो॒ अग्र॒(ग्ग्॒) स्वाया॒(ग्ग्॒) स्वाया॒ मग्र॒म् मद्ध्वो॒ मद्ध्वो॒ अग्र॒(ग्ग्॒) स्वाया᳚म् । \newline
52. अग्र॒(ग्ग्॒) स्वाया॒(ग्ग्॒) स्वाया॒ मग्र॒ मग्र॒(ग्ग्॒) स्वायां॒ ॅयद् यथ् स्वाया॒ मग्र॒ मग्र॒(ग्ग्॒) स्वायां॒ ॅयत् । \newline
53. स्वायां॒ ॅयद् यथ् स्वाया॒(ग्ग्॒) स्वायां॒ ॅयत् त॒नुवां᳚ त॒नुवां॒ यथ् स्वाया॒(ग्ग्॒) स्वायां॒ ॅयत् त॒नुवां᳚ । \newline
54. यत् त॒नुवां᳚ त॒नुवां॒ यद् यत् त॒नुवां᳚ त॒नूम् त॒नूम् त॒नुवां॒ यद् यत् त॒नुवां᳚ त॒नूम् । \newline
55. त॒नुवां᳚ त॒नूम् त॒नूम् त॒नुवां᳚ त॒नुवां᳚ त॒नू मैर॑य॒ तैर॑यत त॒नूम् त॒नुवां᳚ त॒नुवां᳚ त॒नू मैर॑यत । \newline
56. त॒नू मैर॑य॒ तैर॑यत त॒नूम् त॒नू मैर॑यत । \newline
57. ऐर॑य॒ते त्यैर॑यत । \newline
58. उ॒प॒या॒मगृ॑हीतो ऽस्यस्युपया॒मगृ॑हीत उपया॒मगृ॑हीतो ऽसि प्र॒जाप॑तये प्र॒जाप॑तये ऽस्युपया॒मगृ॑हीत उपया॒मगृ॑हीतो ऽसि प्र॒जाप॑तये । \newline
59. उ॒प॒या॒मगृ॑हीत॒ इत्यु॑पया॒म - गृ॒ही॒तः॒ । \newline
60. अ॒सि॒ प्र॒जाप॑तये प्र॒जाप॑तये ऽस्यसि प्र॒जाप॑तये त्वा त्वा प्र॒जाप॑तये ऽस्यसि प्र॒जाप॑तये त्वा । \newline
61. प्र॒जाप॑तये त्वा त्वा प्र॒जाप॑तये प्र॒जाप॑तये त्वा॒ जुष्ट॒म् जुष्ट॑म् त्वा प्र॒जाप॑तये प्र॒जाप॑तये त्वा॒ जुष्ट᳚म् । \newline
62. प्र॒जाप॑तय॒ इति॑ प्र॒जा - प॒त॒ये॒ । \newline
63. त्वा॒ जुष्ट॒म् जुष्ट॑म् त्वा त्वा॒ जुष्ट॑म् गृह्णामि गृह्णामि॒ जुष्ट॑म् त्वा त्वा॒ जुष्ट॑म् गृह्णामि । \newline
64. जुष्ट॑म् गृह्णामि गृह्णामि॒ जुष्ट॒म् जुष्ट॑म् गृह्णाम्ये॒ष ए॒ष गृ॑ह्णामि॒ जुष्ट॒म् जुष्ट॑म् गृह्णाम्ये॒षः । \newline
65. गृ॒ह्णा॒म्ये॒ष ए॒ष गृ॑ह्णामि गृह्णाम्ये॒ष ते॑ त ए॒ष गृ॑ह्णामि गृह्णाम्ये॒ष ते᳚ । \newline
66. ए॒ष ते॑ त ए॒ष ए॒ष ते॒ योनि॒र् योनि॑ स्त ए॒ष ए॒ष ते॒ योनिः॑ । \newline
67. ते॒ योनि॒र् योनि॑ स्ते ते॒ योनिः॑ प्र॒जाप॑तये प्र॒जाप॑तये॒ योनि॑स्ते ते॒ योनिः॑ प्र॒जाप॑तये । \newline
68. योनिः॑ प्र॒जाप॑तये प्र॒जाप॑तये॒ योनि॒र् योनिः॑ प्र॒जाप॑तये त्वा त्वा प्र॒जाप॑तये॒ योनि॒र् योनिः॑ प्र॒जाप॑तये त्वा । \newline
69. प्र॒जाप॑तये त्वा त्वा प्र॒जाप॑तये प्र॒जाप॑तये त्वा । \newline
70. प्र॒जाप॑तय॒ इति॑ प्र॒जा - प॒त॒ये॒ । \newline
71. त्वेति॑ त्वा । \newline
\pagebreak
\markright{ TS 1.7.13.1  \hfill https://www.vedavms.in \hfill}
\addcontentsline{toc}{section}{ TS 1.7.13.1 }
\section*{ TS 1.7.13.1 }

\textbf{TS 1.7.13.1 } \newline
\textbf{Samhita Paata} \newline

अन्वह॒ मासा॒ अन्विद्वना॒न्यन्वोष॑धी॒रनु॒ पर्व॑तासः । अन्विन्द्रꣳ॒॒ रोद॑सी वावशा॒ने अन्वापो॑ अजिहत॒ जाय॑मानं ॥ अनु॑ ते दायि म॒ह इ॑न्द्रि॒याय॑ स॒त्रा ते॒ विश्व॒मनु॑ वृत्र॒हत्ये᳚ । अनु॑ क्ष॒त्रमनु॒ सहो॑ यज॒त्रेन्द्र॑ दे॒वेभि॒रनु॑ ते नृ॒षह्ये᳚ ॥ इ॒न्द्रा॒णीमा॒सु नारि॑षु सु॒पत्नी॑-म॒हम॑श्रवं । न ह्य॑स्या अप॒रं च॒न ज॒रसा॒ - [ ] \newline

\textbf{Pada Paata} \newline

अन्विति॑ । अह॑ । मासाः᳚ । अन्विति॑ । इत् । वना॑नि । अन्विति॑ । ओष॑धीः । अन्विति॑ । पर्व॑तासः ॥ अन्विति॑ । इन्द्र᳚म् । रोद॑सी॒ इति॑ । वा॒व॒शा॒ने इति॑ । अन्विति॑ । आपः॑ । अ॒जि॒ह॒त॒ । जाय॑मानम् ॥ अन्विति॑ । ते॒ । दा॒यि॒ । म॒हे । इ॒न्द्रि॒याय॑ । स॒त्रा । ते॒ । विश्व᳚म् । अन्विति॑ । वृ॒त्र॒हत्य॒ इति॑ वृत्र - हत्ये᳚ ॥ अन्विति॑ । क्ष॒त्रम् । अन्विति॑ । सहः॑ । य॒ज॒त्र॒ । इन्द्र॑ । दे॒वेभिः॑ । अन्विति॑ । ते॒ । नृ॒षह्य॒ इति॑ नृ - सह्ये᳚ ॥ इ॒न्द्रा॒णीम् । आ॒सु । नारि॑षु । सु॒पत्नी॒मिति॑ सु-पत्नी᳚म् । अ॒हम् । अ॒श्र॒व॒म् ॥ न । हि । अ॒स्याः॒ । अ॒प॒रम् । च॒न । ज॒रसा᳚ ।  \newline


\textbf{Krama Paata} \newline

अन्वह॑ । अह॒ मासाः᳚ । मासा॒ अनु॑ । अन्वित् । इद् वना॑नि । वना॒न्यनु॑ । अन्वोष॑धीः । ओष॑धी॒रनु॑ । अनु॒ पर्व॑तासः । पर्व॑तास॒ इति॒ पर्व॑तासः ॥ अन्विन्द्र᳚म् । इन्द्रꣳ॒॒ रोद॑सी । रोद॑सी वावशा॒ने । रोद॑सी॒ इति॒ रोद॑सी । वा॒व॒शा॒ने अनु॑ । वा॒व॒शा॒ने इति॑ वावशा॒ने । अन्वापः॑ । आपो॑ अजिहत । अ॒जि॒ह॒त॒ जाय॑मानम् । जाय॑मान॒मिति॒ जाय॑मानम् ॥ अनु॑ ते । ते॒ दा॒यि॒ । दा॒यि॒ म॒हे । म॒ह इ॑न्द्रि॒याय॑ । इ॒न्द्रि॒याय॑ स॒त्रा । स॒त्रा ते᳚ । ते॒ विश्व᳚म् । विश्व॒मनु॑ । अनु॑ वृत्र॒हत्ये᳚ । वृ॒त्र॒हत्य॒ इति॑ वृत्र - हत्ये᳚ । अनु॑ क्ष॒त्रम् । क्ष॒त्रमनु॑ । अनु॒ सहः॑ । सहो॑ यजत्र । य॒ज॒त्रेन्द्र॑ । इन्द्र॑ दे॒वेभिः॑ । दे॒वेभि॒रनु॑ । अनु॑ ते । ते॒ नृ॒षह्ये᳚ । नृ॒षह्य॒ इति॑ नृ - सह्ये᳚ ॥ इ॒न्द्रा॒णीमा॒सु । आ॒सु नारि॑षु । नारि॑षु सु॒पत्नी᳚म् । सु॒पत्नी॑म॒हम् । सु॒पत्नी॒मिति॑ सु - पत्नी᳚म् । अ॒हम॑श्रवम् । अ॒श्र॒व॒मित्य॑श्रवम् ॥ न हि । ह्य॑स्याः । अ॒स्या॒ अ॒प॒रम् । अ॒प॒रम् च॒न । च॒न ज॒रसा᳚ । ज॒रसा॒ मर॑ते \newline

\textbf{Jatai Paata} \newline

1. अन्व हाहा न्वन्वह॑ । \newline
2. अह॒ मासा॒ मासा॒ अहाह॒ मासाः᳚ । \newline
3. मासा॒ अन्वनु॒ मासा॒ मासा॒ अनु॑ । \newline
4. अन्वि दि दन्वन्वित् । \newline
5. इद् वना॑नि॒ वना॒ नीदिद् वना॑नि । \newline
6. वना॒ न्यन्वनु॒ वना॑नि॒ वना॒ न्यनु॑ । \newline
7. अन्वोष॑धी॒ रोष॑धी॒ रन्व न्वोष॑धीः । \newline
8. ओष॑धी॒ रन्व न्वोष॑धी॒ रोष॑धी॒ रनु॑ । \newline
9. अनु॒ पर्व॑तासः॒ पर्व॑तासो॒ अन्वनु॒ पर्व॑तासः । \newline
10. पर्व॑तास॒ इति॒ पर्व॑तासः । \newline
11. अन्विन्द्र॒ मिन्द्र॒ मन्व न्विन्द्र᳚म् । \newline
12. इन्द्रꣳ॒॒ रोद॑सी॒ रोद॑सी॒ इन्द्र॒ मिन्द्रꣳ॒॒ रोद॑सी । \newline
13. रोद॑सी वावशा॒ने वा॑वशा॒ने रोद॑सी॒ रोद॑सी वावशा॒ने । \newline
14. रोद॑सी॒ इति॒ रोद॑सी । \newline
15. वा॒व॒शा॒ने अन्वनु॑ वावशा॒ने वा॑वशा॒ने अनु॑ । \newline
16. वा॒व॒शा॒ने इति॑ वावशा॒ने । \newline
17. अन्वाप॒ आपो॒ अन्वन्वापः॑ । \newline
18. आपो॑ अजिहता जिह॒ताप॒ आपो॑ अजिहत । \newline
19. अ॒जि॒ह॒त॒ जाय॑मान॒म् जाय॑मान मजिहता जिहत॒ जाय॑मानम् । \newline
20. जाय॑मान॒मिति॒ जाय॑मानम् । \newline
21. अनु॑ ते ते॒ अन्वनु॑ ते । \newline
22. ते॒ दा॒यि॒ दा॒यि॒ ते॒ ते॒ दा॒यि॒ । \newline
23. दा॒यि॒ म॒हे म॒हे दा॑यि दायि म॒हे । \newline
24. म॒ह इ॑न्द्रि॒याये᳚ न्द्रि॒याय॑ म॒हे म॒ह इ॑न्द्रि॒याय॑ । \newline
25. इ॒न्द्रि॒याय॑ स॒त्रा स॒त्रेन्द्रि॒याये᳚ न्द्रि॒याय॑ स॒त्रा । \newline
26. स॒त्रा ते॑ ते स॒त्रा स॒त्रा ते᳚ । \newline
27. ते॒ विश्वं॒ ॅविश्व॑म् ते ते॒ विश्व᳚म् । \newline
28. विश्व॒ मन्वनु॒ विश्वं॒ ॅविश्व॒ मनु॑ । \newline
29. अनु॑ वृत्र॒हत्ये॑ वृत्र॒हत्ये॒ अन्वनु॑ वृत्र॒हत्ये᳚ । \newline
30. वृ॒त्र॒हत्य॒ इति॑ वृत्र - हत्ये᳚ । \newline
31. अनु॑ क्ष॒त्रम् क्ष॒त्र मन्वनु॑ क्ष॒त्रम् । \newline
32. क्ष॒त्र मन्वनु॑ क्ष॒त्रम् क्ष॒त्र मनु॑ । \newline
33. अनु॒ सहः॒ सहो॒ अन्वनु॒ सहः॑ । \newline
34. सहो॑ यजत्र यजत्र॒ सहः॒ सहो॑ यजत्र । \newline
35. य॒ज॒त्रे न्द्रे न्द्र॑ यजत्र यज॒त्रे न्द्र॑ । \newline
36. इन्द्र॑ दे॒वेभि॑र् दे॒वेभि॒ रिन्द्रे न्द्र॑ दे॒वेभिः॑ । \newline
37. दे॒वेभि॒ रन्वनु॑ दे॒वेभि॑र् दे॒वेभि॒ रनु॑ । \newline
38. अनु॑ ते ते॒ अन्वनु॑ ते । \newline
39. ते॒ नृ॒षह्ये॑ नृ॒षह्ये॑ ते ते नृ॒षह्ये᳚ । \newline
40. नृ॒षह्य॒ इति॑ नृ - सह्ये᳚ । \newline
41. इ॒न्द्रा॒णी मा॒स्वा᳚(1॒)स्वि॑न्द्रा॒णी मि॑न्द्रा॒णी मा॒सु । \newline
42. आ॒सु नारि॑षु॒ नारि॑ ष्वा॒स्वा॑सु नारि॑षु । \newline
43. नारि॑षु सु॒पत्नीꣳ॑ सु॒पत्नी॒म् नारि॑षु॒ नारि॑षु सु॒पत्नी᳚म् । \newline
44. सु॒पत्नी॑ म॒ह म॒हꣳ सु॒पत्नीꣳ॑ सु॒पत्नी॑ म॒हम् । \newline
45. सु॒पत्नी॒मिति॑ सु - पत्नी᳚म् । \newline
46. अ॒ह म॑श्रव मश्रव म॒ह म॒ह म॑श्रवम् । \newline
47. अ॒श्र॒व॒मित्य॑श्रवम् । \newline
48. न हि हि न न हि । \newline
49. ह्य॑स्या अस्या॒ हि ह्य॑स्याः । \newline
50. अ॒स्या॒ अ॒प॒र म॑प॒र म॑स्या अस्या अप॒रम् । \newline
51. अ॒प॒रम् च॒न च॒नाप॒र म॑प॒रम् च॒न । \newline
52. च॒न ज॒रसा॑ ज॒रसा॑ च॒न च॒न ज॒रसा᳚ । \newline
53. ज॒रसा॒ मर॑ते॒ मर॑ते ज॒रसा॑ ज॒रसा॒ मर॑ते । \newline

\textbf{Ghana Paata } \newline

1. अन्वहा हान्वन्वह॒ मासा॒ मासा॒ अहान्वन्वह॒ मासाः᳚ । \newline
2. अह॒ मासा॒ मासा॒ अहाह॒ मासा॒ अन्वनु॒ मासा॒ अहाह॒ मासा॒ अनु॑ । \newline
3. मासा॒ अन्वनु॒ मासा॒ मासा॒ अन्वि दिदनु॒ मासा॒ मासा॒ अन्वित् । \newline
4. अन्वि दि दन्वन्विद् वना॑नि॒ वना॒नी दन्वन्विद् वना॑नि । \newline
5. इद् वना॑नि॒ वना॒नी दिद् वना॒न्यन्वनु॒ वना॒नी दिद् वना॒न्यनु॑ । \newline
6. वना॒ न्यन्वनु॒ वना॑नि॒ वना॒ न्यन्वोष॑धी॒ रोष॑धी॒रनु॒ वना॑नि॒ वना॒न्य न्वोष॑धीः । \newline
7. अन्वोष॑धी॒ रोष॑धी॒ रन्व न्वोष॑धी॒ रन्व न्वोष॑धी॒ रन्व न्वोष॑धी॒ रनु॑ । \newline
8. ओष॑धी॒ रन्व न्वोष॑धी॒ रोष॑धी॒ रनु॒ पर्व॑तासः॒ पर्व॑तासो॒ अन्वोष॑धी॒ रोष॑धी॒ रनु॒ पर्व॑तासः । \newline
9. अनु॒ पर्व॑तासः॒ पर्व॑तासो॒ अन्वनु॒ पर्व॑तासः । \newline
10. पर्व॑तास॒ इति॒ पर्व॑तासः । \newline
11. अन्विन्द्र॒ मिन्द्र॒ मन्व न्विन्द्र॒(ग्म्॒) रोद॑सी॒ रोद॑सी॒ इन्द्र॒ मन्व न्विन्द्र॒(ग्म्॒) रोद॑सी । \newline
12. इन्द्र॒(ग्म्॒) रोद॑सी॒ रोद॑सी॒ इन्द्र॒ मिन्द्र॒(ग्म्॒) रोद॑सी वावशा॒ने वा॑वशा॒ने रोद॑सी॒ इन्द्र॒ मिन्द्र॒(ग्म्॒) रोद॑सी वावशा॒ने । \newline
13. रोद॑सी वावशा॒ने वा॑वशा॒ने रोद॑सी॒ रोद॑सी वावशा॒ने अन्वनु॑ वावशा॒ने रोद॑सी॒ रोद॑सी वावशा॒ने अनु॑ । \newline
14. रोद॑सी॒ इति॒ रोद॑सी । \newline
15. वा॒व॒शा॒ने अन्वनु॑ वावशा॒ने वा॑वशा॒ने अन्वाप॒ आपो॒ अनु॑ वावशा॒ने वा॑वशा॒ने अन्वापः॑ । \newline
16. वा॒व॒शा॒ने इति॑ वावशा॒ने । \newline
17. अन्वाप॒ आपो॒ अन्वन्वापो॑ अजिहता जिह॒तापो॒ अन्वन्वापो॑ अजिहत । \newline
18. आपो॑ अजिहता जिह॒ताप॒ आपो॑ अजिहत॒ जाय॑मान॒म् जाय॑मान मजिह॒ताप॒ आपो॑ अजिहत॒ जाय॑मानम् । \newline
19. अ॒जि॒ह॒त॒ जाय॑मान॒म् जाय॑मान मजिहता जिहत॒ जाय॑मानम् । \newline
20. जाय॑मान॒मिति॒ जाय॑मानम् । \newline
21. अनु॑ ते ते॒ अन्वनु॑ ते दायि दायि ते॒ अन्वनु॑ ते दायि । \newline
22. ते॒ दा॒यि॒ दा॒यि॒ ते॒ ते॒ दा॒यि॒ म॒हे म॒हे दा॑यि ते ते दायि म॒हे । \newline
23. दा॒यि॒ म॒हे म॒हे दा॑यि दायि म॒ह इ॑न्द्रि॒याये᳚ न्द्रि॒याय॑ म॒हे दा॑यि दायि म॒ह इ॑न्द्रि॒याय॑ । \newline
24. म॒ह इ॑न्द्रि॒याये᳚ न्द्रि॒याय॑ म॒हे म॒ह इ॑न्द्रि॒याय॑ स॒त्रा स॒त्रेन्द्रि॒याय॑ म॒हे म॒ह इ॑न्द्रि॒याय॑ स॒त्रा । \newline
25. इ॒न्द्रि॒याय॑ स॒त्रा स॒त्रेन्द्रि॒याये᳚ न्द्रि॒याय॑ स॒त्रा ते॑ ते स॒त्रेन्द्रि॒याये᳚ न्द्रि॒याय॑ स॒त्रा ते᳚ । \newline
26. स॒त्रा ते॑ ते स॒त्रा स॒त्रा ते॒ विश्वं॒ ॅविश्व॑म् ते स॒त्रा स॒त्रा ते॒ विश्व᳚म् । \newline
27. ते॒ विश्वं॒ ॅविश्व॑म् ते ते॒ विश्व॒ मन्वनु॒ विश्व॑म् ते ते॒ विश्व॒ मनु॑ । \newline
28. विश्व॒ मन्वनु॒ विश्वं॒ ॅविश्व॒ मनु॑ वृत्र॒हत्ये॑ वृत्र॒हत्ये ऽनु॒ विश्वं॒ ॅविश्व॒ मनु॑ वृत्र॒हत्ये᳚ । \newline
29. अनु॑ वृत्र॒हत्ये॑ वृत्र॒हत्ये॒ अन्वनु॑ वृत्र॒हत्ये᳚ । \newline
30. वृ॒त्र॒हत्य॒ इति॑ वृत्र - हत्ये᳚ । \newline
31. अनु॑ क्ष॒त्रम् क्ष॒त्र मन्वनु॑ क्ष॒त्र मन्वनु॑ क्ष॒त्र मन्वनु॑ क्ष॒त्र मनु॑ । \newline
32. क्ष॒त्र मन्वनु॑ क्ष॒त्रम् क्ष॒त्र मनु॒ सहः॒ सहो ऽनु॑ क्ष॒त्रम् क्ष॒त्र मनु॒ सहः॑ । \newline
33. अनु॒ सहः॒ सहो॒ अन्वनु॒ सहो॑ यजत्र यजत्र॒ सहो॒ अन्वनु॒ सहो॑ यजत्र । \newline
34. सहो॑ यजत्र यजत्र॒ सहः॒ सहो॑ यज॒त्रे न्द्रे न्द्र॑ यजत्र॒ सहः॒ सहो॑ यज॒त्रे न्द्र॑ । \newline
35. य॒ज॒त्रे न्द्रे न्द्र॑ यजत्र यज॒त्रे न्द्र॑ दे॒वेभि॑र् दे॒वेभि॒ रिन्द्र॑ यजत्र यज॒त्रे न्द्र॑ दे॒वेभिः॑ । \newline
36. इन्द्र॑ दे॒वेभि॑र् दे॒वेभि॒ रिन्द्रे न्द्र॑ दे॒वेभि॒ रन्वनु॑ दे॒वेभि॒ रिन्द्रे न्द्र॑ दे॒वेभि॒ रनु॑ । \newline
37. दे॒वेभि॒ रन्वनु॑ दे॒वेभि॑र् दे॒वेभि॒ रनु॑ ते ते॒ अनु॑ दे॒वेभि॑र् दे॒वेभि॒ रनु॑ ते । \newline
38. अनु॑ ते ते॒ अन्वनु॑ ते नृ॒षह्ये॑ नृ॒षह्ये॑ ते॒ अन्वनु॑ ते नृ॒षह्ये᳚ । \newline
39. ते॒ नृ॒षह्ये॑ नृ॒षह्ये॑ ते ते नृ॒षह्ये᳚ । \newline
40. नृ॒षह्य॒ इति॑ नृ - सह्ये᳚ । \newline
41. इ॒न्द्रा॒णी मा॒स्वा᳚(1॒)स्वि॑न्द्रा॒णी मि॑न्द्रा॒णी मा॒सु नारि॑षु॒ नारि॑ष्वा॒ स्वि॑न्द्रा॒णी मि॑न्द्रा॒णी मा॒सु नारि॑षु । \newline
42. आ॒सु नारि॑षु॒ नारि॑ष्वा॒ स्वा॑सु नारि॑षु सु॒पत्नी(ग्म्॑) सु॒पत्नी॒न् नारि॑ष्वा॒ स्वा॑सु नारि॑षु सु॒पत्नी᳚म् । \newline
43. नारि॑षु सु॒पत्नी(ग्म्॑) सु॒पत्नी॒न् नारि॑षु॒ नारि॑षु सु॒पत्नी॑ म॒ह म॒हꣳ सु॒पत्नी॒न् नारि॑षु॒ नारि॑षु सु॒पत्नी॑ म॒हम् । \newline
44. सु॒पत्नी॑ म॒ह म॒हꣳ सु॒पत्नी(ग्म्॑) सु॒पत्नी॑ म॒ह म॑श्रव मश्रव म॒हꣳ सु॒पत्नी(ग्म्॑) सु॒पत्नी॑ म॒ह म॑श्रवम् । \newline
45. सु॒पत्नी॒मिति॑ सु - पत्नी᳚म् । \newline
46. अ॒ह म॑श्रव मश्रव म॒ह म॒ह म॑श्रवम् । \newline
47. अ॒श्र॒व॒मित्य॑श्रवम् । \newline
48. न हि हि न न ह्य॑स्या अस्या॒ हि न न ह्य॑स्याः । \newline
49. ह्य॑स्या अस्या॒ हि ह्य॑स्या अप॒र म॑प॒र म॑स्या॒ हि ह्य॑स्या अप॒रम् । \newline
50. अ॒स्या॒ अ॒प॒र म॑प॒र म॑स्या अस्या अप॒रम् च॒न च॒नाप॒र म॑स्या अस्या अप॒रम् च॒न । \newline
51. अ॒प॒रम् च॒न च॒नाप॒र म॑प॒रम् च॒न ज॒रसा॑ ज॒रसा॑ च॒नाप॒र म॑प॒रम् च॒न ज॒रसा᳚ । \newline
52. च॒न ज॒रसा॑ ज॒रसा॑ च॒न च॒न ज॒रसा॒ मर॑ते॒ मर॑ते ज॒रसा॑ च॒न च॒न ज॒रसा॒ मर॑ते । \newline
53. ज॒रसा॒ मर॑ते॒ मर॑ते ज॒रसा॑ ज॒रसा॒ मर॑ते॒ पति॒ष् पति॒र् मर॑ते ज॒रसा॑ ज॒रसा॒ मर॑ते॒ पतिः॑ । \newline
\pagebreak
\markright{ TS 1.7.13.2  \hfill https://www.vedavms.in \hfill}
\addcontentsline{toc}{section}{ TS 1.7.13.2 }
\section*{ TS 1.7.13.2 }

\textbf{TS 1.7.13.2 } \newline
\textbf{Samhita Paata} \newline

मर॑ते॒ पतिः॑ ॥ नाहमि॑न्द्राणि रारण॒ सख्यु॑र् वृ॒षाक॑पेर्. ऋ॒ते । यस्ये॒दमप्यꣳ॑ ह॒विः प्रि॒यं दे॒वेषु॒ गच्छ॑ति ॥यो जा॒त ए॒व प्र॑थ॒मो मन॑स्वान् दे॒वो दे॒वान् क्रतु॑ना प॒र्यभू॑षत् । यस्य॒ शुष्मा॒द्रोद॑सी॒ अभ्य॑सेतां नृं॒णस्य॑ म॒ह्ना स ज॑नास॒ इन्द्रः॑ ॥ आ ते॑ म॒ह इ॑न्द्रो॒त्यु॑ग्र॒ सम॑न्यवो॒ यथ् स॒मर॑न्त॒ सेनाः᳚ । पता॑ति दि॒द्युन्नर्य॑स्य बाहु॒वोर् मा ते॒ - [ ] \newline

\textbf{Pada Paata} \newline

मर॑ते । पतिः॑ ॥ न । अ॒हम् । इ॒न्द्रा॒णि॒ । रा॒र॒ण॒ । सख्युः॑ । वृ॒षाक॑पे॒रिति॑ वृ॒षा - क॒पेः॒ । ऋ॒ते ॥ यस्य॑ । इ॒दम् । अप्य᳚म् । ह॒विः । प्रि॒यम् । दे॒वेषु॑ । गच्छ॑ति ॥ यः । जा॒तः । ए॒व । प्र॒थ॒मः । मन॑स्वान् । दे॒वः । दे॒वान् । क्रतु॑ना । प॒र्यभू॑ष॒दिति॑ परि - अभू॑षत् ॥ यस्य॑ । शुष्मा᳚त् । रोद॑सी॒ इति॑ । अभ्य॑सेताम् । नृ॒णंस्य॑ । म॒ह्ना । सः । ज॒ना॒सः॒ । इन्द्रः॑ ॥ ऐति॑ । ते॒ । म॒हः । इ॒न्द्र॒ । ऊ॒ती । उ॒ग्र॒ । सम॑न्यव॒ इति स - म॒न्य॒वः॒ । यत् । स॒मर॒न्तेति॑ सं - अर॑न्त । सेनाः᳚ ॥ पता॑ति । दि॒द्युत् । नर्य॑स्य । बा॒हु॒वोः । मा । ते॒ ।  \newline


\textbf{Krama Paata} \newline

मर॑ते॒ पतिः॑ । पति॒रिति॒ पतिः॑ ॥ नाहम् । अ॒हमि॑न्द्राणि । इ॒न्द्रा॒णि॒ रा॒र॒ण॒ । रा॒र॒ण॒ सख्युः॑ । सख्यु॑र्,वृ॒षाक॑पेः । वृ॒षाक॑पेर्. ऋ॒ते । वृ॒षाक॑पे॒रिति॑ वृ॒षा - क॒पेः॒ । ऋ॒त इत्यृ॒ते ॥ यस्ये॒दम् । इ॒दमप्य᳚म् । अप्यꣳ॑ ह॒विः । ह॒विः प्रि॒यम् । प्रि॒यम् दे॒वेषु॑ । दे॒वेषु॒ गच्छ॑ति । गच्छ॒तीति॒ गच्छ॑ति ॥ यो जा॒तः । जा॒त ए॒व । ए॒व प्र॑थ॒मः । प्र॒थ॒मो मन॑स्वान् । मन॑स्वान् दे॒वः । दे॒वो दे॒वान् । दे॒वान्,क्रतु॑ना । क्रतु॑ना प॒र्यभू॑षत् । प॒र्यभू॑ष॒दिति॑ परि - अभू॑षत् । यस्य॒ शुष्मा᳚त् । शुष्मा॒द् रोद॑सी । रोद॑सी॒ अभ्य॑सेताम् । रोद॑सी॒ इति॒ रोद॑सी । अभ्य॑सेताम् नृ॒म्णस्य॑ । नृ॒म्णस्य॑ म॒ह्ना । म॒ह्ना सः । स ज॑नासः । ज॒ना॒स॒ इन्द्रः॑ । इन्द्र॒ इतीन्द्रः॑ ॥ आ ते᳚ । ते॒ म॒हः । म॒ह इ॑न्द्र । इ॒न्द्रो॒ती । ऊ॒त्यु॑ग्र । उ॒ग्र॒ सम॑न्यवः । सम॑न्यवो॒ यत् । सम॑न्यव॒ इति॒ स - म॒न्य॒वः॒ । यथ् स॒मर॑न्त । स॒मर॑न्त॒ सेनाः᳚ । स॒मर॒न्तेति॑ सम् - अर॑न्तः । सेना॒ इति॒ सेनाः᳚ ॥ पता॑ति दि॒द्युत् । दि॒द्युन्नर्य॑स्य । नर्य॑स्य बाहु॒वोः । बा॒हु॒वोर् मा । मा ते᳚ । ते॒ मनः॑ \newline

\textbf{Jatai Paata} \newline

1. मर॑ते॒ पति॒ष् पति॒र् मर॑ते॒ मर॑ते॒ पतिः॑ । \newline
2. पति॒रिति॒ पतिः॑ । \newline
3. नाह म॒हम् न नाहम् । \newline
4. अ॒ह मि॑न्द्रा णीन्द्राण्य॒ह म॒ह मि॑न्द्राणि । \newline
5. इ॒न्द्रा॒णि॒ रा॒र॒ण॒ रा॒र॒णे॒ न्द्रा॒ णी॒न्द्रा॒णि॒ रा॒र॒ण॒ । \newline
6. रा॒र॒ण॒ सख्युः॒ सख्यू॑ रारण रारण॒ सख्युः॑ । \newline
7. सख्यु॑र् वृ॒षाक॑पेर् वृ॒षाक॑पेः॒ सख्युः॒ सख्यु॑र् वृ॒षाक॑पेः । \newline
8. वृ॒षाक॑पेर्. ऋ॒त ऋ॒ते वृ॒षाक॑पेर् वृ॒षाक॑पेर्. ऋ॒ते । \newline
9. वृ॒षाक॑पे॒रिति॑ वृ॒षा - क॒पेः॒ । \newline
10. ऋ॒त इत्यृ॒ते । \newline
11. यस्ये॒ द मि॒दं ॅयस्य॒ यस्ये॒ दम् । \newline
12. इ॒द मप्य॒ मप्य॑ मि॒द मि॒द मप्य᳚म् । \newline
13. अप्यꣳ॑ ह॒विर्. ह॒विरप्य॒ मप्यꣳ॑ ह॒विः । \newline
14. ह॒विः प्रि॒यम् प्रि॒यꣳ ह॒विर्. ह॒विः प्रि॒यम् । \newline
15. प्रि॒यम् दे॒वेषु॑ दे॒वेषु॑ प्रि॒यम् प्रि॒यम् दे॒वेषु॑ । \newline
16. दे॒वेषु॒ गच्छ॑ति॒ गच्छ॑ति दे॒वेषु॑ दे॒वेषु॒ गच्छ॑ति । \newline
17. गच्छ॒तीति॒ गच्छ॑ति । \newline
18. यो जा॒तो जा॒तो यो यो जा॒तः । \newline
19. जा॒त ए॒वैव जा॒तो जा॒त ए॒व । \newline
20. ए॒व प्र॑थ॒मः प्र॑थ॒म ए॒वैव प्र॑थ॒मः । \newline
21. प्र॒थ॒मो मन॑स्वा॒न् मन॑स्वान् प्रथ॒मः प्र॑थ॒मो मन॑स्वान् । \newline
22. मन॑स्वान् दे॒वो दे॒वो मन॑स्वा॒न् मन॑स्वान् दे॒वः । \newline
23. दे॒वो दे॒वान् दे॒वान् दे॒वो दे॒वो दे॒वान् । \newline
24. दे॒वान् क्रतु॑ना॒ क्रतु॑ना दे॒वान् दे॒वान् क्रतु॑ना । \newline
25. क्रतु॑ना प॒र्यभू॑षत् प॒र्यभू॑ष॒त् क्रतु॑ना॒ क्रतु॑ना प॒र्यभू॑षत् । \newline
26. प॒र्यभू॑ष॒दिति॑ परि - अभू॑षत् । \newline
27. यस्य॒ शुष्मा॒च् छुष्मा॒द् यस्य॒ यस्य॒ शुष्मा᳚त् । \newline
28. शुष्मा॒द् रोद॑सी॒ रोद॑सी॒ शुष्मा॒च् छुष्मा॒द् रोद॑सी । \newline
29. रोद॑सी॒ अभ्य॑सेता॒ मभ्य॑सेताꣳ॒॒ रोद॑सी॒ रोद॑सी॒ अभ्य॑सेताम् । \newline
30. रोद॑सी॒ इति॒ रोद॑सी । \newline
31. अभ्य॑सेताम् नृं॒णस्य॑ नृं॒णस्या भ्य॑सेता॒ मभ्य॑सेताम् नृं॒णस्य॑ । \newline
32. नृं॒णस्य॑ म॒ह्ना म॒ह्ना नृं॒णस्य॑ नृं॒णस्य॑ म॒ह्ना । \newline
33. म॒ह्ना स स म॒ह्ना म॒ह्ना सः । \newline
34. स ज॑नासो जनासः॒ स स ज॑नासः । \newline
35. ज॒ना॒स॒ इन्द्र॒ इन्द्रो॑ जनासो जनास॒ इन्द्रः॑ । \newline
36. इन्द्र॒ इतीन्द्रः॑ । \newline
37. आ ते॑ त॒ आ ते᳚ । \newline
38. ते॒ म॒हो म॒हस्ते॑ ते म॒हः । \newline
39. म॒ह इ॑न्द्रे न्द्र म॒हो म॒ह इ॑न्द्र । \newline
40. इ॒न्द्रो॒ त्यू॑तीन्द्रे᳚ न्द्रो॒ती । \newline
41. ऊ॒त्यु॑ग्रो ग्रो॒त्यू᳚(1॒)त्यु॑ग्र । \newline
42. उ॒ग्र॒ सम॑न्यवः॒ सम॑न्यव उग्रोग्र॒ सम॑न्यवः । \newline
43. सम॑न्यवो॒ यद् यथ् सम॑न्यवः॒ सम॑न्यवो॒ यत् । \newline
44. सम॑न्यव॒ इति स - म॒न्य॒वः॒ । \newline
45. यथ् स॒मर॑न्त स॒मर॑न्त॒ यद् यथ् स॒मर॑न्त । \newline
46. स॒मर॑न्त॒ सेनाः॒ सेनाः᳚ स॒मर॑न्त स॒मर॑न्त॒ सेनाः᳚ । \newline
47. स॒मर॒न्तेति॑ सं - अर॑न्त । \newline
48. सेना॒ इति॒ सेनाः᳚ । \newline
49. पता॑ति दि॒द्युद् दि॒द्युत् पता॑ति॒ पता॑ति दि॒द्युत् । \newline
50. दि॒द्युन् नर्य॑स्य॒ नर्य॑स्य दि॒द्युद् दि॒द्युन् नर्य॑स्य । \newline
51. नर्य॑स्य बाहु॒वोर् बा॑हु॒वोर् नर्य॑स्य॒ नर्य॑स्य बाहु॒वोः । \newline
52. बा॒हु॒वोर् मा मा बा॑हु॒वोर् बा॑हु॒वोर् मा । \newline
53. मा ते॑ ते॒ मा मा ते᳚ । \newline
54. ते॒ मनो॒ मन॑ स्ते ते॒ मनः॑ । \newline

\textbf{Ghana Paata } \newline

1. मर॑ते॒ पति॒ष् पति॒र् मर॑ते॒ मर॑ते॒ पतिः॑ । \newline
2. पति॒रिति॒ पतिः॑ । \newline
3. नाह म॒हन्न नाह मि॑न्द्राणी न्द्राण्य॒हन् न नाह मि॑न्द्राणि । \newline
4. अ॒ह मि॑न्द्राणी न्द्राण्य॒ह म॒ह मि॑न्द्राणि रारण रारणे न्द्राण्य॒ह म॒ह मि॑न्द्राणि रारण । \newline
5. इ॒न्द्रा॒णि॒ रा॒र॒ण॒ रा॒र॒णे॒ न्द्रा॒णी॒ न्द्रा॒णि॒ रा॒र॒ण॒ सख्युः॒ सख्यू॑ रारणे न्द्राणी न्द्राणि रारण॒ सख्युः॑ । \newline
6. रा॒र॒ण॒ सख्युः॒ सख्यू॑ रारण रारण॒ सख्यु॑र् वृ॒षाक॑पेर् वृ॒षाक॑पेः॒ सख्यू॑ रारण रारण॒ सख्यु॑र् वृ॒षाक॑पेः । \newline
7. सख्यु॑र् वृ॒षाक॑पेर् वृ॒षाक॑पेः॒ सख्युः॒ सख्यु॑र् वृ॒षाक॑पेर्. ऋ॒त ऋ॒ते वृ॒षाक॑पेः॒ सख्युः॒ सख्यु॑र् वृ॒षाक॑पेर्. ऋ॒ते । \newline
8. वृ॒षाक॑पेर्. ऋ॒त ऋ॒ते वृ॒षाक॑पेर् वृ॒षाक॑पेर्. ऋ॒ते । \newline
9. वृ॒षाक॑पे॒रिति॑ वृ॒षा - क॒पेः॒ । \newline
10. ऋ॒त इत्यृ॒ते । \newline
11. यस्ये॒ द मि॒दं ॅयस्य॒ यस्ये॒ द मप्य॒ मप्य॑ मि॒दं ॅयस्य॒ यस्ये॒ द मप्य᳚म् । \newline
12. इ॒द मप्य॒ मप्य॑ मि॒द मि॒द मप्य(ग्म्॑) ह॒विर्. ह॒विरप्य॑ मि॒द मि॒द मप्य(ग्म्॑) ह॒विः । \newline
13. अप्य(ग्म्॑) ह॒विर्. ह॒विरप्य॒ मप्य(ग्म्॑) ह॒विः प्रि॒यम् प्रि॒यꣳ ह॒विरप्य॒ मप्य(ग्म्॑) ह॒विः प्रि॒यम् । \newline
14. ह॒विः प्रि॒यम् प्रि॒यꣳ ह॒विर्. ह॒विः प्रि॒यम् दे॒वेषु॑ दे॒वेषु॑ प्रि॒यꣳ ह॒विर्. ह॒विः प्रि॒यम् दे॒वेषु॑ । \newline
15. प्रि॒यम् दे॒वेषु॑ दे॒वेषु॑ प्रि॒यम् प्रि॒यम् दे॒वेषु॒ गच्छ॑ति॒ गच्छ॑ति दे॒वेषु॑ प्रि॒यम् प्रि॒यम् दे॒वेषु॒ गच्छ॑ति । \newline
16. दे॒वेषु॒ गच्छ॑ति॒ गच्छ॑ति दे॒वेषु॑ दे॒वेषु॒ गच्छ॑ति । \newline
17. गच्छ॒तीति॒ गच्छ॑ति । \newline
18. यो जा॒तो जा॒तो यो यो जा॒त ए॒वैव जा॒तो यो यो जा॒त ए॒व । \newline
19. जा॒त ए॒वैव जा॒तो जा॒त ए॒व प्र॑थ॒मः प्र॑थ॒म ए॒व जा॒तो जा॒त ए॒व प्र॑थ॒मः । \newline
20. ए॒व प्र॑थ॒मः प्र॑थ॒म ए॒वैव प्र॑थ॒मो मन॑स्वा॒न् मन॑स्वान् प्रथ॒म ए॒वैव प्र॑थ॒मो मन॑स्वान् । \newline
21. प्र॒थ॒मो मन॑स्वा॒न् मन॑स्वान् प्रथ॒मः प्र॑थ॒मो मन॑स्वान् दे॒वो दे॒वो मन॑स्वान् प्रथ॒मः प्र॑थ॒मो मन॑स्वान् दे॒वः । \newline
22. मन॑स्वान् दे॒वो दे॒वो मन॑स्वा॒न् मन॑स्वान् दे॒वो दे॒वान् दे॒वान् दे॒वो मन॑स्वा॒न् मन॑स्वान् दे॒वो दे॒वान् । \newline
23. दे॒वो दे॒वान् दे॒वान् दे॒वो दे॒वो दे॒वान् क्रतु॑ना॒ क्रतु॑ना दे॒वान् दे॒वो दे॒वो दे॒वान् क्रतु॑ना । \newline
24. दे॒वान् क्रतु॑ना॒ क्रतु॑ना दे॒वान् दे॒वान् क्रतु॑ना प॒र्यभू॑षत् प॒र्यभू॑ष॒त् क्रतु॑ना दे॒वान् दे॒वान् क्रतु॑ना प॒र्यभू॑षत् । \newline
25. क्रतु॑ना प॒र्यभू॑षत् प॒र्यभू॑ष॒त् क्रतु॑ना॒ क्रतु॑ना प॒र्यभू॑षत् । \newline
26. प॒र्यभू॑ष॒दिति॑ परि - अभू॑षत् । \newline
27. यस्य॒ शुष्मा॒ च्छुष्मा॒द् यस्य॒ यस्य॒ शुष्मा॒द् रोद॑सी॒ रोद॑सी॒ शुष्मा॒द् यस्य॒ यस्य॒ शुष्मा॒द् रोद॑सी । \newline
28. शुष्मा॒द् रोद॑सी॒ रोद॑सी॒ शुष्मा॒ च्छुष्मा॒द् रोद॑सी॒ अभ्य॑सेता॒ मभ्य॑सेता॒(ग्म्॒) रोद॑सी॒ शुष्मा॒ च्छुष्मा॒द् 
रोद॑सी॒ अभ्य॑सेताम् । \newline
29. रोद॑सी॒ अभ्य॑सेता॒ मभ्य॑सेता॒(ग्म्॒) रोद॑सी॒ रोद॑सी॒ अभ्य॑सेतान् नृं॒णस्य॑ नृं॒णस्या भ्य॑सेता॒(ग्म्॒) 
रोद॑सी॒ रोद॑सी॒ अभ्य॑सेतान् नृं॒णस्य॑ । \newline
30. रोद॑सी॒ इति॒ रोद॑सी । \newline
31. अभ्य॑सेतान् नृं॒णस्य॑ नृं॒णस्या भ्य॑सेता॒ मभ्य॑सेतान् नृं॒णस्य॑ म॒ह्ना म॒ह्ना नृं॒णस्या भ्य॑सेता॒ मभ्य॑सेतान् नृं॒णस्य॑ म॒ह्ना । \newline
32. नृं॒णस्य॑ म॒ह्ना म॒ह्ना नृं॒णस्य॑ नृं॒णस्य॑ म॒ह्ना स स म॒ह्ना नृं॒णस्य॑ नृं॒णस्य॑ म॒ह्ना सः । \newline
33. म॒ह्ना स स म॒ह्ना म॒ह्ना स ज॑नासो जनासः॒ स म॒ह्ना म॒ह्ना स ज॑नासः । \newline
34. स ज॑नासो जनासः॒ स स ज॑नास॒ इन्द्र॒ इन्द्रो॑ जनासः॒ स स ज॑नास॒ इन्द्रः॑ । \newline
35. ज॒ना॒स॒ इन्द्र॒ इन्द्रो॑ जनासो जनास॒ इन्द्रः॑ । \newline
36. इन्द्र॒ इतीन्द्रः॑ । \newline
37. आ ते॑ त॒ आ ते॑ म॒हो म॒हस्त॒ आ ते॑ म॒हः । \newline
38. ते॒ म॒हो म॒ह स्ते॑ ते म॒ह इ॑न्द्रे न्द्र म॒ह स्ते॑ ते म॒ह इ॑न्द्र । \newline
39. म॒ह इ॑न्द्रे न्द्र म॒हो म॒ह इ॑न्द्रो॒ त्यू॑तीन्द्र॑ म॒हो म॒ह इ॑न्द्रो॒ती । \newline
40. इ॒न्द्रो॒ त्यू॑तीन्द्रे᳚ न्द्रो॒ त्यु॑ग्रोग्रो॒ तीन्द्रे᳚ न्द्रो॒त्यु॑ग्र । \newline
41. ऊ॒त्यु॑ग्रोग्रो॒त्यू᳚(1॒)त्यु॑ग्र॒ सम॑न्यवः॒ सम॑न्यव उग्रो॒त्यू᳚(1॒)त्यु॑ग्र॒ सम॑न्यवः । \newline
42. उ॒ग्र॒ सम॑न्यवः॒ सम॑न्यव उग्रोग्र॒ सम॑न्यवो॒ यद् यथ् सम॑न्यव उग्रोग्र॒ सम॑न्यवो॒ यत् । \newline
43. सम॑न्यवो॒ यद् यथ् सम॑न्यवः॒ सम॑न्यवो॒ यथ् स॒मर॑न्त स॒मर॑न्त॒ यथ् सम॑न्यवः॒ सम॑न्यवो॒ यथ् स॒मर॑न्त । \newline
44. सम॑न्यव॒ इति स - म॒न्य॒वः॒ । \newline
45. यथ् स॒मर॑न्त स॒मर॑न्त॒ यद् यथ् स॒मर॑न्त॒ सेनाः॒ सेनाः᳚ स॒मर॑न्त॒ यद् यथ् स॒मर॑न्त॒ सेनाः᳚ । \newline
46. स॒मर॑न्त॒ सेनाः॒ सेनाः᳚ स॒मर॑न्त स॒मर॑न्त॒ सेनाः᳚ । \newline
47. स॒मर॒न्तेति॑ सं - अर॑न्त । \newline
48. सेना॒ इति॒ सेनाः᳚ । \newline
49. पता॑ति दि॒द्युद् दि॒द्युत् पता॑ति॒ पता॑ति दि॒द्युन् नर्य॑स्य॒ नर्य॑स्य दि॒द्युत् पता॑ति॒ पता॑ति दि॒द्युन् नर्य॑स्य । \newline
50. दि॒द्युन् नर्य॑स्य॒ नर्य॑स्य दि॒द्युद् दि॒द्युन् नर्य॑स्य बाहु॒वोर् बा॑हु॒वोर् नर्य॑स्य दि॒द्युद् दि॒द्युन् नर्य॑स्य बाहु॒वोः । \newline
51. नर्य॑स्य बाहु॒वोर् बा॑हु॒वोर् नर्य॑स्य॒ नर्य॑स्य बाहु॒वोर् मा मा बा॑हु॒वोर् नर्य॑स्य॒ नर्य॑स्य बाहु॒वोर् मा । \newline
52. बा॒हु॒वोर् मा मा बा॑हु॒वोर् बा॑हु॒वोर् मा ते॑ ते॒ मा बा॑हु॒वोर् बा॑हु॒वोर् मा ते᳚ । \newline
53. मा ते॑ ते॒ मा मा ते॒ मनो॒ मन॑ स्ते॒ मा मा ते॒ मनः॑ । \newline
54. ते॒ मनो॒ मन॑स्ते ते॒ मनो॑ विष्व॒द्रिय॑ग् विष्व॒द्रिय॒ङ् मन॑स्ते ते॒ मनो॑ विष्व॒द्रिय॑क् । \newline
\pagebreak
\markright{ TS 1.7.13.3  \hfill https://www.vedavms.in \hfill}
\addcontentsline{toc}{section}{ TS 1.7.13.3 }
\section*{ TS 1.7.13.3 }

\textbf{TS 1.7.13.3 } \newline
\textbf{Samhita Paata} \newline

मनो॑ विष्व॒द्रिय॒ग् वि चा॑रीत् ॥ मा नो॑ मर्द्धी॒रा भ॑रा द॒द्धि तन्नः॒ प्र दा॒शुषे॒ दात॑वे॒ भूरि॒ यत् ते᳚ । नव्ये॑ दे॒ष्णे श॒स्ते अ॒स्मिन् त॑ उ॒क्थे प्र ब्र॑वाम व॒यमि॑न्द्र स्तु॒वन्तः॑ ॥ आ तू भ॑र॒ माकि॑रे॒तत् परि॑ ष्ठाद् वि॒द्मा हि त्वा॒ वसु॑पतिं॒ ॅवसू॑नां । इन्द्र॒ यत् ते॒ माहि॑नं॒ दत्र॒-मस्त्य॒स्मभ्यं॒ तद्ध॑र्यश्व॒ - [ ] \newline

\textbf{Pada Paata} \newline

मनः॑ । वि॒ष्व॒द्रिय॒गिति॑ विष्व - द्रिय॑क् । वीति॑ । चा॒री॒त् ॥ मा । नः॒ । म॒र्द्धीः॒ । एति॑ । भ॒र॒ । द॒द्धि । तत् । नः॒ । प्रेति॑ । दा॒शुषे᳚ । दात॑वे । भूरि॑ । यत् । ते॒ ॥ नव्ये᳚ । दे॒ष्णे । श॒स्ते । अ॒स्मिन्न् । ते॒ । उ॒क्थे । प्रेति॑ । ब्र॒वा॒म॒ । व॒यम् । इ॒न्द्र॒ । स्तु॒वन्तः॑ ॥ एति॑ । तु । भ॒र॒ । माकिः॑ । ए॒तत् । परीति॑ । स्था॒त् । वि॒द्म । हि । त्वा॒ । वसु॑पति॒मिति॒ वसु॑ - प॒ति॒म् । वसू॑नाम् ॥ इन्द्र॑ । यत् । ते॒ । माहि॑नम् । दत्र᳚म् । अस्ति॑ । अ॒स्मभ्य॒मित्य॒स्म - भ्य॒म् । तत् । ह॒र्य॒श्वेति॑ हरि - अ॒श्व॒ ।  \newline


\textbf{Krama Paata} \newline

मनो॑ विष्व॒द्रिय॑क् । वि॒ष्व॒द्रिय॒ग् वि । वि॒ष्व॒द्रिय॒गिति॑ विष्व - द्रिय॑क् । वि चा॑रीत् । चा॒री॒दिति॑ चारीत् ॥ मा नः॑ । नो॒ म॒र्द्धीः॒ । म॒र्द्धी॒रा । आ भ॑र । भ॒रा॒ द॒द्धि । द॒द्धि तत् । तन्नः॑ । नः॒ प्र । प्र दा॒शुषे᳚ । दा॒शुषे॒ दात॑वे । दात॑वे॒ भूरि॑ । भूरि॒ यत् । यत् ते᳚ । त॒ इति॑ ते ॥ नव्ये॑ दे॒ष्णे । दे॒ष्णे श॒स्ते । श॒स्ते अ॒स्मिन्न् । अ॒स्मिन् ते᳚ । त॒ उ॒क्थे । उ॒क्थे प्र । प्र ब्र॑वाम । ब्र॒वा॒म॒ व॒यम् । व॒यमि॑न्द्र । इ॒न्द्र॒ स्तु॒वन्तः॑ । स्तु॒वन्त॒ इति॑ स्तु॒वन्तः॑ ॥ आ तु । तू भ॑र । भ॒र॒ माकिः॑ । माकि॑रे॒तत् । ए॒तत्,परि॑ । परि॑ ष्ठात् । स्था॒द् वि॒द्म । वि॒द्मा हि । हि त्वा᳚ । त्वा॒ वसु॑पतिम् । वसु॑पतिं॒ ॅवसू॑नाम् । वसु॑पति॒मिति॒ वसु॑ - प॒ति॒म् । वसू॑ना॒मिति॒ वसू॑नाम् ॥ इन्द्र॒ यत् । यत् ते᳚ । ते॒ माहि॑नम् । माहि॑न॒म् दत्र᳚म् । दत्र॒मस्ति॑ । अस्त्य॒स्मभ्य᳚म् । अ॒स्मभ्य॒म् तत् । अ॒स्मभ्य॒मित्य॒स्म - भ्य॒म् । तद्ध॑र्यश्व । ह॒र्य॒श्व॒ प्र । 
ह॒र्य॒श्वेति॑ हरि - अ॒श्व॒ \newline

\textbf{Jatai Paata} \newline

1. मनो॑ विष्व॒द्रिय॑ग् विष्व॒द्रिय॒ङ् मनो॒ मनो॑ विष्व॒द्रिय॑क् । \newline
2. वि॒ष्व॒द्रिय॒ग् वि वि वि॑ष्व॒द्रिय॑ग् विष्व॒द्रिय॒ग् वि । \newline
3. वि॒ष्व॒द्रिय॒गिति॑ विष्व - द्रिय॑क् । \newline
4. वि चा॑रीच् चारी॒द् वि वि चा॑रीत् । \newline
5. चा॒री॒दिति॑ चारीत् । \newline
6. मा नो॑ नो॒ मा मा नः॑ । \newline
7. नो॒ म॒र्द्धी॒र् म॒र्द्धी॒र् नो॒ नो॒ म॒र्द्धीः॒ । \newline
8. म॒र्द्धी॒रा म॑र्द्धीर् मर्द्धी॒रा । \newline
9. आ भ॑र भ॒रा भ॑र । \newline
10. भ॒रा॒ द॒द्धि द॒द्धि भ॑र भरा द॒द्धि । \newline
11. द॒द्धि तत् तद् द॒द्धि द॒द्धि तत् । \newline
12. तन् नो॑ न॒ स्तत् तन् नः॑ । \newline
13. नः॒ प्र प्र णो॑ नः॒ प्र । \newline
14. प्र दा॒शुषे॑ दा॒शुषे॒ प्र प्र दा॒शुषे᳚ । \newline
15. दा॒शुषे॒ दात॑वे॒ दात॑वे दा॒शुषे॑ दा॒शुषे॒ दात॑वे । \newline
16. दात॑वे॒ भूरि॒ भूरि॒ दात॑वे॒ दात॑वे॒ भूरि॑ । \newline
17. भूरि॒ यद् यद् भूरि॒ भूरि॒ यत् । \newline
18. यत् ते॑ ते॒ यद् यत् ते᳚ । \newline
19. त॒ इति॑ ते । \newline
20. नव्ये॑ दे॒ष्णे दे॒ष्णे नव्ये॒ नव्ये॑ दे॒ष्णे । \newline
21. दे॒ष्णे श॒स्ते श॒स्ते दे॒ष्णे दे॒ष्णे श॒स्ते । \newline
22. श॒स्ते अ॒स्मिन् न॒स्मिञ् छ॒स्ते श॒स्ते अ॒स्मिन्न् । \newline
23. अ॒स्मिन् ते॑ ते अ॒स्मिन् न॒स्मिन् ते᳚ । \newline
24. त॒ उ॒क्थ उ॒क्थे ते॑ त उ॒क्थे । \newline
25. उ॒क्थे प्र प्रोक्थ उ॒क्थे प्र । \newline
26. प्र ब्र॑वाम ब्रवाम॒ प्र प्र ब्र॑वाम । \newline
27. ब्र॒वा॒म॒ व॒यं ॅव॒यम् ब्र॑वाम ब्रवाम व॒यम् । \newline
28. व॒य मि॑न्द्रे न्द्र व॒यं ॅव॒य मि॑न्द्र । \newline
29. इ॒न्द्र॒ स्तु॒वन्तः॑ स्तु॒वन्त॑ इन्द्रे न्द्र स्तु॒वन्तः॑ । \newline
30. स्तु॒वन्त॒ इति॑ स्तु॒वन्तः॑ । \newline
31. आ तु त्वा तु । \newline
32. तू भ॑र भर॒ तु तू भ॑र । \newline
33. भ॒र॒ माकि॒र् माकि॑र् भर भर॒ माकिः॑ । \newline
34. माकि॑ रे॒त दे॒तन् माकि॒र् माकि॑ रे॒तत् । \newline
35. ए॒तत् परि॒ पर्ये॒त दे॒तत् परि॑ । \newline
36. परि॑ ष्ठाथ् स्था॒त् परि॒ परि॑ ष्ठात् । \newline
37. स्था॒द् वि॒द्म वि॒द्म स्था᳚थ् स्थाद् वि॒द्म । \newline
38. वि॒द्मा हि हि वि॒द्म वि॒द्मा हि । \newline
39. हि त्वा᳚ त्वा॒ हि हि त्वा᳚ । \newline
40. त्वा॒ वसु॑पतिं॒ ॅवसु॑पतिम् त्वा त्वा॒ वसु॑पतिम् । \newline
41. वसु॑पतिं॒ ॅवसू॑नां॒ ॅवसू॑नां॒ ॅवसु॑पतिं॒ ॅवसु॑पतिं॒ ॅवसू॑नाम् । \newline
42. वसु॑पति॒मिति॒ वसु॑ - प॒ति॒म् । \newline
43. वसू॑ना॒मिति॒ वसू॑नाम् । \newline
44. इन्द्र॒ यद् यदिन्द्रे न्द्र॒ यत् । \newline
45. यत् ते॑ ते॒ यद् यत् ते᳚ । \newline
46. ते॒ माहि॑न॒म् माहि॑नम् ते ते॒ माहि॑नम् । \newline
47. माहि॑न॒म् दत्र॒म् दत्र॒म् माहि॑न॒म् माहि॑न॒म् दत्र᳚म् । \newline
48. दत्र॒ मस्त्यस्ति॒ दत्र॒म् दत्र॒ मस्ति॑ । \newline
49. अस्त्य॒ स्मभ्य॑ म॒स्मभ्य॒ मस्त्य स्त्य॒स्मभ्य᳚म् । \newline
50. अ॒स्मभ्य॒म् तत् तद॒स्मभ्य॑ म॒स्मभ्य॒म् तत् । \newline
51. अ॒स्मभ्य॒मित्य॒स्म - भ्य॒म् । \newline
52. तद्ध॑र्यश्व हर्यश्व॒ तत् तद्ध॑र्यश्व । \newline
53. ह॒र्य॒श्व॒ प्र प्र ह॑र्यश्व हर्यश्व॒ प्र । \newline
54. ह॒र्य॒श्वेति॑ हरि - अ॒श्व॒ । \newline

\textbf{Ghana Paata } \newline

1. मनो॑ विष्व॒द्रिय॑ग् विष्व॒द्रिय॒ङ् मनो॒ मनो॑ विष्व॒द्रिय॒ग् वि वि वि॑ष्व॒द्रिय॒ङ् मनो॒ मनो॑ विष्व॒द्रिय॒ग् वि । \newline
2. वि॒ष्व॒द्रिय॒ग् वि वि वि॑ष्व॒द्रिय॑ग् विष्व॒द्रिय॒ग् वि चा॑रीच् चारी॒द् वि वि॑ष्व॒द्रिय॑ग् विष्व॒द्रिय॒ग् वि चा॑रीत् । \newline
3. वि॒ष्व॒द्रिय॒गिति॑ विष्व - द्रिय॑क् । \newline
4. वि चा॑रीच् चारी॒द् वि वि चा॑रीत् । \newline
5. चा॒री॒दिति॑ चारीत् । \newline
6. मा नो॑ नो॒ मा मा नो॑ मर्द्धीर् मर्द्धीर् नो॒ मा मा नो॑ मर्द्धीः । \newline
7. नो॒ म॒र्द्धी॒र् म॒र्द्धी॒र् नो॒ नो॒ म॒र्द्धी॒रा म॑र्द्धीर् नो नो मर्द्धी॒रा । \newline
8. म॒र्द्धी॒रा म॑र्द्धीर् मर्द्धी॒रा भ॑र भ॒रा म॑र्द्धीर् मर्द्धी॒रा भ॑र । \newline
9. आ भ॑र भ॒रा भ॑रा द॒द्धि द॒द्धि भ॒रा भ॑रा द॒द्धि । \newline
10. भ॒रा॒ द॒द्धि द॒द्धि भ॑र भरा द॒द्धि तत् तद् द॒द्धि भ॑र भरा द॒द्धि तत् । \newline
11. द॒द्धि तत् तद् द॒द्धि द॒द्धि तन् नो॑ न॒स्तद् द॒द्धि द॒द्धि तन् नः॑ । \newline
12. तन् नो॑ न॒ स्तत् तन् नः॒ प्र प्र ण॒ स्तत् तन् नः॒ प्र । \newline
13. नः॒ प्र प्र णो॑ नः॒ प्र दा॒शुषे॑ दा॒शुषे॒ प्र णो॑ नः॒ प्र दा॒शुषे᳚ । \newline
14. प्र दा॒शुषे॑ दा॒शुषे॒ प्र प्र दा॒शुषे॒ दात॑वे॒ दात॑वे दा॒शुषे॒ प्र प्र दा॒शुषे॒ दात॑वे । \newline
15. दा॒शुषे॒ दात॑वे॒ दात॑वे दा॒शुषे॑ दा॒शुषे॒ दात॑वे॒ भूरि॒ भूरि॒ दात॑वे दा॒शुषे॑ दा॒शुषे॒ दात॑वे॒ भूरि॑ । \newline
16. दात॑वे॒ भूरि॒ भूरि॒ दात॑वे॒ दात॑वे॒ भूरि॒ यद् यद् भूरि॒ दात॑वे॒ दात॑वे॒ भूरि॒ यत् । \newline
17. भूरि॒ यद् यद् भूरि॒ भूरि॒ यत् ते॑ ते॒ यद् भूरि॒ भूरि॒ यत् ते᳚ । \newline
18. यत् ते॑ ते॒ यद् यत् ते᳚ । \newline
19. त॒ इति॑ ते । \newline
20. नव्ये॑ दे॒ष्णे दे॒ष्णे नव्ये॒ नव्ये॑ दे॒ष्णे श॒स्ते श॒स्ते दे॒ष्णे नव्ये॒ नव्ये॑ दे॒ष्णे श॒स्ते । \newline
21. दे॒ष्णे श॒स्ते श॒स्ते दे॒ष्णे दे॒ष्णे श॒स्ते अ॒स्मिन् न॒स्मिञ् छ॒स्ते दे॒ष्णे दे॒ष्णे श॒स्ते अ॒स्मिन्न् । \newline
22. श॒स्ते अ॒स्मिन् न॒स्मिञ् छ॒स्ते श॒स्ते अ॒स्मिन् ते॑ ते अ॒स्मिञ् छ॒स्ते श॒स्ते अ॒स्मिन् ते᳚ । \newline
23. अ॒स्मिन् ते॑ ते अ॒स्मिन् न॒स्मिन् त॑ उ॒क्थ उ॒क्थे ते॑ अ॒स्मिन् न॒स्मिन् त॑ उ॒क्थे । \newline
24. त॒ उ॒क्थ उ॒क्थे ते॑ त उ॒क्थे प्र प्रोक्थे ते॑ त उ॒क्थे प्र । \newline
25. उ॒क्थे प्र प्रोक्थ उ॒क्थे प्र ब्र॑वाम ब्रवाम॒ प्रोक्थ उ॒क्थे प्र ब्र॑वाम । \newline
26. प्र ब्र॑वाम ब्रवाम॒ प्र प्र ब्र॑वाम व॒यं ॅव॒यम् ब्र॑वाम॒ प्र प्र ब्र॑वाम व॒यम् । \newline
27. ब्र॒वा॒म॒ व॒यं ॅव॒यम् ब्र॑वाम ब्रवाम व॒य मि॑न्द्रे न्द्र व॒यम् ब्र॑वाम ब्रवाम व॒य मि॑न्द्र । \newline
28. व॒य मि॑न्द्रे न्द्र व॒यं ॅव॒य मि॑न्द्र स्तु॒वन्तः॑ स्तु॒वन्त॑ इन्द्र व॒यं ॅव॒य मि॑न्द्र स्तु॒वन्तः॑ । \newline
29. इ॒न्द्र॒ स्तु॒वन्तः॑ स्तु॒वन्त॑ इन्द्रे न्द्र स्तु॒वन्तः॑ । \newline
30. स्तु॒वन्त॒ इति॑ स्तु॒वन्तः॑ । \newline
31. आ तु त्वा तू भ॑र भर॒ त्वा तू भ॑र । \newline
32. तू भ॑र भर॒ तु तू भ॑र॒ माकि॒र् माकि॑र् भर॒ तु तू भ॑र॒ माकिः॑ । \newline
33. भ॒र॒ माकि॒र् माकि॑र् भर भर॒ माकि॑ रे॒त दे॒तन् माकि॑र् भर भर॒ माकि॑ रे॒तत् । \newline
34. माकि॑ रे॒त दे॒तन् माकि॒र् माकि॑ रे॒तत् परि॒ पर्ये॒तन् माकि॒र् माकि॑ रे॒तत् परि॑ । \newline
35. ए॒तत् परि॒ पर्ये॒त दे॒तत् परि॑ ष्ठाथ् स्था॒त् पर्ये॒त दे॒तत् परि॑ ष्ठात् । \newline
36. परि॑ ष्ठाथ् स्था॒त् परि॒ परि॑ ष्ठाद् वि॒द्म वि॒द्म स्था॒त् परि॒ परि॑ ष्ठाद् वि॒द्म । \newline
37. स्था॒द् वि॒द्म वि॒द्म स्था᳚थ् स्थाद् वि॒द्मा हि हि वि॒द्म स्था᳚थ् स्थाद् वि॒द्मा हि । \newline
38. वि॒द्मा हि हि वि॒द्म वि॒द्मा हि त्वा᳚ त्वा॒ हि वि॒द्म वि॒द्मा हि त्वा᳚ । \newline
39. हि त्वा᳚ त्वा॒ हि हि त्वा॒ वसु॑पतिं॒ ॅवसु॑पतिम् त्वा॒ हि हि त्वा॒ वसु॑पतिम् । \newline
40. त्वा॒ वसु॑पतिं॒ ॅवसु॑पतिम् त्वा त्वा॒ वसु॑पतिं॒ ॅवसू॑नां॒ ॅवसू॑नां॒ ॅवसु॑पतिम् त्वा त्वा॒ वसु॑पतिं॒ ॅवसू॑नाम् । \newline
41. वसु॑पतिं॒ ॅवसू॑नां॒ ॅवसू॑नां॒ ॅवसु॑पतिं॒ ॅवसु॑पतिं॒ ॅवसू॑नाम् । \newline
42. वसु॑पति॒मिति॒ वसु॑ - प॒ति॒म् । \newline
43. वसू॑ना॒मिति॒ वसू॑नाम् । \newline
44. इन्द्र॒ यद् यदिन्द्रे न्द्र॒ यत् ते॑ ते॒ यदिन्द्रे न्द्र॒ यत् ते᳚ । \newline
45. यत् ते॑ ते॒ यद् यत् ते॒ माहि॑न॒म् माहि॑नम् ते॒ यद् यत् ते॒ माहि॑नम् । \newline
46. ते॒ माहि॑न॒म् माहि॑नम् ते ते॒ माहि॑न॒म् दत्र॒म् दत्र॒म् माहि॑नम् ते ते॒ माहि॑न॒म् दत्र᳚म् । \newline
47. माहि॑न॒म् दत्र॒म् दत्र॒म् माहि॑न॒म् माहि॑न॒म् दत्र॒ मस्त्यस्ति॒ दत्र॒म् माहि॑न॒म् माहि॑न॒म् दत्र॒ मस्ति॑ । \newline
48. दत्र॒ मस्त्यस्ति॒ दत्र॒म् दत्र॒ मस्त्य॒ स्मभ्य॑ म॒स्मभ्य॒ मस्ति॒ दत्र॒म् दत्र॒ मस्त्य॒ स्मभ्य᳚म् । \newline
49. अस्त्य॒ स्मभ्य॑ म॒स्मभ्य॒ मस्त्य स्त्य॒स्मभ्य॒म् तत् तद॒स्मभ्य॒ मस्त्य स्त्य॒स्मभ्य॒म् तत् । \newline
50. अ॒स्मभ्य॒म् तत् तद॒स्मभ्य॑ म॒स्मभ्य॒म् तद्ध॑र्यश्व हर्यश्व॒ तद॒स्मभ्य॑ म॒स्मभ्य॒म् तद्ध॑र्यश्व । \newline
51. अ॒स्मभ्य॒मित्य॒स्म - भ्य॒म् । \newline
52. तद्ध॑र्यश्व हर्यश्व॒ तत् तद्ध॑र्यश्व॒ प्र प्र ह॑र्यश्व॒ तत् तद्ध॑र्यश्व॒ प्र । \newline
53. ह॒र्य॒श्व॒ प्र प्र ह॑र्यश्व हर्यश्व॒ प्र य॑न्धि यन्धि॒ प्र ह॑र्यश्व हर्यश्व॒ प्र य॑न्धि । \newline
54. ह॒र्य॒श्वेति॑ हरि - अ॒श्व॒ । \newline
\pagebreak
\markright{ TS 1.7.13.4  \hfill https://www.vedavms.in \hfill}
\addcontentsline{toc}{section}{ TS 1.7.13.4 }
\section*{ TS 1.7.13.4 }

\textbf{TS 1.7.13.4 } \newline
\textbf{Samhita Paata} \newline

प्र य॑न्धि ॥ प्र॒दा॒तारꣳ॑ हवामह॒ इन्द्र॒मा ह॒विषा॑ व॒यं । उ॒भा हि हस्ता॒ वसु॑ना पृ॒णस्वा ऽऽ प्र य॑च्छ॒ दक्षि॑णा॒दोत स॒व्यात् ॥ प्र॒दा॒ता व॒ज्री वृ॑ष॒भस्तु॑रा॒षाट्छु॒ष्मी राजा॑ वृत्र॒हा सो॑म॒पावा᳚ । अ॒स्मिन्. य॒ज्ञे ब॒र्॒.हिष्या नि॒षद्याथा॑ भव॒ यज॑मानाय॒ शं ॅयोः ॥ इन्द्रः॑ सु॒त्रामा॒ स्ववाꣳ॒॒ अवो॑भिः सुमृडी॒को भ॑वतु वि॒श्ववे॑दाः । बाध॑तां॒ द्वेषो॒ अभ॑यं कृणोतु सु॒वीर्य॑स्य॒ - [ ] \newline

\textbf{Pada Paata} \newline

प्रेति॑ । य॒न्धि॒ ॥ प्र॒दा॒तार॒मिति॑ प्र - दा॒तार᳚म् । ह॒वा॒म॒हे॒ । इन्द्र᳚म् । एति॑ । ह॒विषा᳚ । व॒यम् ॥ उ॒भा । हि । हस्ता᳚ । वसु॑ना । पृ॒णस्व॑ । आ । प्रेति॑ । य॒च्छ॒ । दक्षि॑णात् । एति॑ । उ॒त । स॒व्यात् ॥ प्र॒दा॒तेति॑ प्र - दा॒ता । व॒ज्री । वृ॒ष॒भः । तु॒रा॒षाट् । शु॒ष्मी । राजा᳚ । वृ॒त्र॒हेति॑ वृत्र - हा । सो॒म॒पावेति॑ सोम - पावा᳚ ॥ अ॒स्मिन्न् । य॒ज्ञे । ब॒र्॒.हिषि॑ । एति॑ । नि॒षद्येति॑ नि - सद्य॑ । अथ॑ । भ॒व॒ । यज॑मानाय । शम् । योः ॥ इन्द्रः॑ । सु॒त्रामेति॑ सु - त्रामा᳚ । स्ववा॒निति॒ स्व-वा॒न् । अवो॑भि॒रित्यवः॑ - भिः॒ । सु॒मृ॒डी॒क इति॑ सु - मृ॒डी॒कः । भ॒व॒तु॒ । वि॒श्ववे॑दा॒ इति॑ वि॒श्व-वे॒दाः॒ ॥ बाध॑ताम् । द्वेषः॑ । अभ॑यम् । कृ॒णो॒तु॒ । सु॒वीर्य॒स्येति॑ सु - वीर्य॑स्य ।  \newline


\textbf{Krama Paata} \newline

प्र य॑न्धि । य॒न्धीति॑ यन्धि ॥ प्र॒दा॒तारꣳ॑ हवामहे । प्र॒दा॒तार॒मिति॑ प्र - दा॒तार᳚म् । ह॒वा॒म॒ह॒ इन्द्र᳚म् । इन्द्र॒मा । आ ह॒विषा᳚ । ह॒विषा॑ व॒यम् । व॒यमिति॑ व॒यम् ॥ उ॒भा हि । हि हस्ता᳚ । हस्ता॒ वसु॑ना । वसु॑ना पृ॒णस्व॑ । पृ॒णस्वा । आ प्र । प्र य॑च्छ । य॒च्छ॒ दक्षि॑णात् । दक्षि॑णा॒दा । ओत । उ॒त स॒व्यात् । स॒व्यादिति॑ स॒व्यात् ॥ प्र॒दा॒ता व॒ज्री । प्र॒दा॒तेति॑ प्र - दा॒ता । व॒ज्री वृ॑ष॒भः । वृ॒ष॒भ,स्तु॑रा॒षाट् । तु॒रा॒षाट् छु॒ष्मी । शु॒ष्मी राजा᳚ । राजा॑ वृत्र॒हा । वृ॒त्र॒हा सो॑म॒पावा᳚ । वृ॒त्र॒हेति॑ वृत्र - हा । सो॒म॒पावेति॑ सोम - पावा᳚ ॥ अ॒स्मिन्. य॒ज्ञे । य॒ज्ञे ब॒र्.॒॒हिषि॑ । ब॒र्॒.हिष्या । आ नि॒षद्य॑ । नि॒षद्याथ॑ । नि॒षद्येति॑ नि - सद्य॑ । अथा॑ भव । भ॒व॒ यज॑मानाय । यज॑मानाय॒ शम् । शं ॅयोः । योरिति॒ योः ॥ इन्द्रः॑ सु॒त्रामा᳚ । सु॒त्रामा॒ स्ववान्॑ । सु॒त्रामेति॑ सु - त्रामा᳚ । स्ववाꣳ॒॒ अवो॑भिः । स्ववा॒निति॒ स्व - वा॒न्॒ । अवो॑भिः सुमृडी॒कः । अवो॑भि॒रित्यवः॑ - भिः॒ । सु॒मृ॒डी॒को भ॑वतु । सु॒मृ॒डी॒क इति॑ सु - मृ॒डी॒कः । भ॒व॒तु॒ वि॒श्ववे॑दाः । वि॒श्ववे॑दा॒ इति॑ वि॒श्व - वे॒दाः॒ ॥ बाध॑ता॒म् द्वेषः॑ । द्वेषो॒ अभ॑यम् । अभ॑यम् कृणोतु । कृ॒णो॒तु॒ सु॒वीर्य॑स्य । सु॒वीर्य॑स्य॒ पत॑यः । सु॒वीर्य॒स्येति॑ सु - वीर्य॑स्य \newline

\textbf{Jatai Paata} \newline

1. प्र य॑न्धि यन्धि॒ प्र प्र य॑न्धि । \newline
2. य॒न्धीति॑ यन्धि । \newline
3. प्र॒दा॒तारꣳ॑ हवामहे हवामहे प्रदा॒तार॑म् प्रदा॒तारꣳ॑ हवामहे । \newline
4. प्र॒दा॒तार॒मिति॑ प्र - दा॒तार᳚म् । \newline
5. ह॒वा॒म॒ह॒ इन्द्र॒ मिन्द्रꣳ॑ हवामहे हवामह॒ इन्द्र᳚म् । \newline
6. इन्द्र॒ मेन्द्र॒ मिन्द्र॒ मा । \newline
7. आ ह॒विषा॑ ह॒विषा ऽऽह॒विषा᳚ । \newline
8. ह॒विषा॑ व॒यं ॅव॒यꣳ ह॒विषा॑ ह॒विषा॑ व॒यम् । \newline
9. व॒यमिति॑ व॒यम् । \newline
10. उ॒भा हि ह्यु॑भोभा हि । \newline
11. हि हस्ता॒ हस्ता॒ हि हि हस्ता᳚ । \newline
12. हस्ता॒ वसु॑ना॒ वसु॑ना॒ हस्ता॒ हस्ता॒ वसु॑ना । \newline
13. वसु॑ना पृ॒णस्व॑ पृ॒णस्व॒ वसु॑ना॒ वसु॑ना पृ॒णस्व॑ । \newline
14. पृ॒णस्वा पृ॒णस्व॑ पृ॒णस्वा । \newline
15. आ प्र प्रा प्र । \newline
16. प्र य॑च्छ यच्छ॒ प्र प्र य॑च्छ । \newline
17. य॒च्छ॒ दक्षि॑णा॒द् दक्षि॑णाद् यच्छ यच्छ॒ दक्षि॑णात् । \newline
18. दक्षि॑णा॒दा दक्षि॑णा॒द् दक्षि॑णा॒दा । \newline
19. ओतोतोत । \newline
20. उ॒त स॒व्याथ् स॒व्या दु॒तोत स॒व्यात् । \newline
21. स॒व्यादिति॑ स॒व्यात् । \newline
22. प्र॒दा॒ता व॒ज्री व॒ज्री प्र॑दा॒ता प्र॑दा॒ता व॒ज्री । \newline
23. प्र॒दा॒तेति॑ प्र - दा॒ता । \newline
24. व॒ज्री वृ॑ष॒भो वृ॑ष॒भो व॒ज्री व॒ज्री वृ॑ष॒भः । \newline
25. वृ॒ष॒भ स्तु॑रा॒षाट् तु॑रा॒षाड् वृ॑ष॒भो वृ॑ष॒भ स्तु॑रा॒षाट् । \newline
26. तु॒रा॒षाट् छु॒ष्मी शु॒ष्मी तु॑रा॒षाट् तु॑रा॒षाट् छु॒ष्मी । \newline
27. शु॒ष्मी राजा॒ राजा॑ शु॒ष्मी शु॒ष्मी राजा᳚ । \newline
28. राजा॑ वृत्र॒हा वृ॑त्र॒हा राजा॒ राजा॑ वृत्र॒हा । \newline
29. वृ॒त्र॒हा सो॑म॒पावा॑ सोम॒पावा॑ वृत्र॒हा वृ॑त्र॒हा सो॑म॒पावा᳚ । \newline
30. वृ॒त्र॒हेति॑ वृत्र - हा । \newline
31. सो॒म॒पावेति॑ सोम - पावा᳚ । \newline
32. अ॒स्मिन्. य॒ज्ञे य॒ज्ञे अ॒स्मिन् न॒स्मिन्. य॒ज्ञे । \newline
33. य॒ज्ञे ब॒र्॒.हिषि॑ ब॒र्॒.हिषि॑ य॒ज्ञे य॒ज्ञे ब॒र्॒.हिषि॑ । \newline
34. ब॒र्॒.हिष्या ब॒र्॒.हिषि॑ ब॒र्॒.हिष्या । \newline
35. आ नि॒षद्य॑ नि॒षद्या नि॒षद्य॑ । \newline
36. नि॒षद्याथाथ॑ नि॒षद्य॑ नि॒षद्याथ॑ । \newline
37. नि॒षद्येति॑ नि - सद्य॑ । \newline
38. अथा॑ भव भ॒वाथाथा॑ भव । \newline
39. भ॒व॒ यज॑मानाय॒ यज॑मानाय भव भव॒ यज॑मानाय । \newline
40. यज॑मानाय॒ शꣳ शं ॅयज॑मानाय॒ यज॑मानाय॒ शम् । \newline
41. शं ॅयोर् योः शꣳ शं ॅयोः । \newline
42. योरिति॒ योः । \newline
43. इन्द्रः॑ सु॒त्रामा॑ सु॒त्रामेन्द्र॒ इन्द्रः॑ सु॒त्रामा᳚ । \newline
44. सु॒त्रामा॒ स्ववा॒न् थ्स्ववा᳚न् थ्सु॒त्रामा॑ सु॒त्रामा॒ स्ववान्॑ । \newline
45. सु॒त्रामेति॑ सु - त्रामा᳚ । \newline
46. स्ववाꣳ॒॒ अवो॑भि॒ रवो॑भिः॒ स्ववा॒न् थ्स्ववाꣳ॒॒ अवो॑भिः । \newline
47. स्ववा॒निति॒ स्व - वा॒न् । \newline
48. अवो॑भिः सुमृडी॒कः सु॑मृडी॒को ऽवो॑भि॒ रवो॑भिः सुमृडी॒कः । \newline
49. अवो॑भि॒रित्यवः॑ - भिः॒ । \newline
50. सु॒मृ॒डी॒को भ॑वतु भवतु सुमृडी॒कः सु॑मृडी॒को भ॑वतु । \newline
51. सु॒मृ॒डी॒क इति॑ सु - मृ॒डी॒कः । \newline
52. भ॒व॒तु॒ वि॒श्ववे॑दा वि॒श्ववे॑दा भवतु भवतु वि॒श्ववे॑दाः । \newline
53. वि॒श्ववे॑दा॒ इति॑ वि॒श्व - वे॒दाः॒ । \newline
54. बाध॑ता॒म् द्वेषो॒ द्वेषो॒ बाध॑ता॒म् बाध॑ता॒म् द्वेषः॑ । \newline
55. द्वेषो॒ अभ॑य॒ मभ॑य॒म् द्वेषो॒ द्वेषो॒ अभ॑यम् । \newline
56. अभ॑यम् कृणोतु कृणो॒त्वभ॑य॒ मभ॑यम् कृणोतु । \newline
57. कृ॒णो॒तु॒ सु॒वीर्य॑स्य सु॒वीर्य॑स्य कृणोतु कृणोतु सु॒वीर्य॑स्य । \newline
58. सु॒वीर्य॑स्य॒ पत॑यः॒ पत॑यः सु॒वीर्य॑स्य सु॒वीर्य॑स्य॒ पत॑यः । \newline
59. सु॒वीर्य॒स्येति॑ सु - वीर्य॑स्य । \newline

\textbf{Ghana Paata } \newline

1. प्र य॑न्धि यन्धि॒ प्र प्र य॑न्धि । \newline
2. य॒न्धीति॑ यन्धि । \newline
3. प्र॒दा॒तार(ग्म्॑) हवामहे हवामहे प्रदा॒तार॑म् प्रदा॒तार(ग्म्॑) हवामह॒ इन्द्र॒ मिन्द्र(ग्म्॑) हवामहे प्रदा॒तार॑म् प्रदा॒तार(ग्म्॑) हवामह॒ इन्द्र᳚म् । \newline
4. प्र॒दा॒तार॒मिति॑ प्र - दा॒तार᳚म् । \newline
5. ह॒वा॒म॒ह॒ इन्द्र॒ मिन्द्र(ग्म्॑) हवामहे हवामह॒ इन्द्र॒ मेन्द्र(ग्म्॑) हवामहे हवामह॒ इन्द्र॒ मा । \newline
6. इन्द्र॒ मेन्द्र॒ मिन्द्र॒ मा ह॒विषा॑ ह॒विषेन्द्र॒ मिन्द्र॒ मा ह॒विषा᳚ । \newline
7. आ ह॒विषा॑ ह॒विषा ऽऽह॒विषा॑ व॒यं ॅव॒यꣳ ह॒विषा ऽऽह॒विषा॑ व॒यम् । \newline
8. ह॒विषा॑ व॒यं ॅव॒यꣳ ह॒विषा॑ ह॒विषा॑ व॒यम् । \newline
9. व॒यमिति॑ व॒यम् । \newline
10. उ॒भा हि ह्यु॑भोभा हि हस्ता॒ हस्ता॒ ह्यु॑भोभा हि हस्ता᳚ । \newline
11. हि हस्ता॒ हस्ता॒ हि हि हस्ता॒ वसु॑ना॒ वसु॑ना॒ हस्ता॒ हि हि हस्ता॒ वसु॑ना । \newline
12. हस्ता॒ वसु॑ना॒ वसु॑ना॒ हस्ता॒ हस्ता॒ वसु॑ना पृ॒णस्व॑ पृ॒णस्व॒ वसु॑ना॒ हस्ता॒ हस्ता॒ वसु॑ना पृ॒णस्व॑ । \newline
13. वसु॑ना पृ॒णस्व॑ पृ॒णस्व॒ वसु॑ना॒ वसु॑ना पृ॒णस्वा पृ॒णस्व॒ वसु॑ना॒ वसु॑ना पृ॒णस्वा । \newline
14. पृ॒णस्वा पृ॒णस्व॑ पृ॒णस्वा प्र प्रा पृ॒णस्व॑ पृ॒णस्वा प्र । \newline
15. आ प्र प्रा प्र य॑च्छ यच्छ॒ प्रा प्र य॑च्छ । \newline
16. प्र य॑च्छ यच्छ॒ प्र प्र य॑च्छ॒ दक्षि॑णा॒द् दक्षि॑णाद् यच्छ॒ प्र प्र य॑च्छ॒ दक्षि॑णात् । \newline
17. य॒च्छ॒ दक्षि॑णा॒द् दक्षि॑णाद् यच्छ यच्छ॒ दक्षि॑णा॒दा दक्षि॑णाद् यच्छ यच्छ॒ दक्षि॑णा॒दा । \newline
18. दक्षि॑णा॒दा दक्षि॑णा॒द् दक्षि॑णा॒ दोतोता दक्षि॑णा॒द् दक्षि॑णा॒दोत । \newline
19. ओतो तोत स॒व्याथ् स॒व्या दु॒तोत स॒व्यात् । \newline
20. उ॒त स॒व्याथ् स॒व्या दु॒तोत स॒व्यात् । \newline
21. स॒व्यादिति॑ स॒व्यात् । \newline
22. प्र॒दा॒ता व॒ज्री व॒ज्री प्र॑दा॒ता प्र॑दा॒ता व॒ज्री वृ॑ष॒भो वृ॑ष॒भो व॒ज्री प्र॑दा॒ता प्र॑दा॒ता व॒ज्री वृ॑ष॒भः । \newline
23. प्र॒दा॒तेति॑ प्र - दा॒ता । \newline
24. व॒ज्री वृ॑ष॒भो वृ॑ष॒भो व॒ज्री व॒ज्री वृ॑ष॒भ स्तु॑रा॒षाट् तु॑रा॒षाड् वृ॑ष॒भो व॒ज्री व॒ज्री वृ॑ष॒भ स्तु॑रा॒षाट् । \newline
25. वृ॒ष॒भ स्तु॑रा॒षाट् तु॑रा॒षाड् वृ॑ष॒भो वृ॑ष॒भ स्तु॑रा॒षाट् छु॒ष्मी शु॒ष्मी तु॑रा॒षाड् वृ॑ष॒भो वृ॑ष॒भ स्तु॑रा॒षाट् छु॒ष्मी । \newline
26. तु॒रा॒षाट् छु॒ष्मी शु॒ष्मी तु॑रा॒षाट् तु॑रा॒षाट् छु॒ष्मी राजा॒ राजा॑ शु॒ष्मी तु॑रा॒षाट् तु॑रा॒षाट् छु॒ष्मी राजा᳚ । \newline
27. शु॒ष्मी राजा॒ राजा॑ शु॒ष्मी शु॒ष्मी राजा॑ वृत्र॒हा वृ॑त्र॒हा राजा॑ शु॒ष्मी शु॒ष्मी राजा॑ वृत्र॒हा । \newline
28. राजा॑ वृत्र॒हा वृ॑त्र॒हा राजा॒ राजा॑ वृत्र॒हा सो॑म॒पावा॑ सोम॒पावा॑ वृत्र॒हा राजा॒ राजा॑ वृत्र॒हा सो॑म॒पावा᳚ । \newline
29. वृ॒त्र॒हा सो॑म॒पावा॑ सोम॒पावा॑ वृत्र॒हा वृ॑त्र॒हा सो॑म॒पावा᳚ । \newline
30. वृ॒त्र॒हेति॑ वृत्र - हा । \newline
31. सो॒म॒पावेति॑ सोम - पावा᳚ । \newline
32. अ॒स्मिन्. य॒ज्ञे य॒ज्ञे अ॒स्मिन् न॒स्मिन्. य॒ज्ञे ब॒र्॒.हिषि॑ ब॒र्॒.हिषि॑ य॒ज्ञे अ॒स्मिन् न॒स्मिन्. य॒ज्ञे ब॒र्॒.हिषि॑ । \newline
33. य॒ज्ञे ब॒र्॒.हिषि॑ ब॒र्॒.हिषि॑ य॒ज्ञे य॒ज्ञे ब॒र्॒.हिष्या ब॒र्॒.हिषि॑ य॒ज्ञे य॒ज्ञे ब॒र्॒.हिष्या । \newline
34. ब॒र्॒.हिष्या ब॒र्॒.हिषि॑ ब॒र्॒.हिष्या नि॒षद्य॑ नि॒षद्या ब॒र्॒.हिषि॑ ब॒र्॒.हिष्या नि॒षद्य॑ । \newline
35. आ नि॒षद्य॑ नि॒षद्या नि॒षद्या थाथ॑ नि॒षद्या नि॒षद्याथ॑ । \newline
36. नि॒षद्या थाथ॑ नि॒षद्य॑ नि॒षद्याथा॑ भव भ॒वाथ॑ नि॒षद्य॑ नि॒षद्याथा॑ भव । \newline
37. नि॒षद्येति॑ नि - सद्य॑ । \newline
38. अथा॑ भव भ॒वाथाथा॑ भव॒ यज॑मानाय॒ यज॑मानाय भ॒वाथाथा॑ भव॒ यज॑मानाय । \newline
39. भ॒व॒ यज॑मानाय॒ यज॑मानाय भव भव॒ यज॑मानाय॒ शꣳ शं ॅयज॑मानाय भव भव॒ यज॑मानाय॒ शम् । \newline
40. यज॑मानाय॒ शꣳ शं ॅयज॑मानाय॒ यज॑मानाय॒ शं ॅयोर् योः शं ॅयज॑मानाय॒ यज॑मानाय॒ शं ॅयोः । \newline
41. शं ॅयोर् योः शꣳ शं ॅयोः । \newline
42. योरिति॒ योः । \newline
43. इन्द्रः॑ सु॒त्रामा॑ सु॒त्रामेन्द्र॒ इन्द्रः॑ सु॒त्रामा॒ स्ववा॒न् थ्स्ववा᳚न् थ्सु॒त्रामेन्द्र॒ इन्द्रः॑ सु॒त्रामा॒ स्ववान्॑ । \newline
44. सु॒त्रामा॒ स्ववा॒न् थ्स्ववा᳚न् थ्सु॒त्रामा॑ सु॒त्रामा॒ स्ववा॒(ग्म्॒) अवो॑भि॒ रवो॑भिः॒ स्ववा᳚न् थ्सु॒त्रामा॑ सु॒त्रामा॒ स्ववा॒(ग्म्॒) अवो॑भिः । \newline
45. सु॒त्रामेति॑ सु - त्रामा᳚ । \newline
46. स्ववा॒(ग्म्॒) अवो॑भि॒ रवो॑भिः॒ स्ववा॒न् थ्स्ववा॒(ग्म्॒) अवो॑भिः सुमृडी॒कः सु॑मृडी॒को ऽवो॑भिः॒ स्ववा॒न् थ्स्ववा॒(ग्म्॒) अवो॑भिः सुमृडी॒कः । \newline
47. स्ववा॒निति॒ स्व - वा॒न् । \newline
48. अवो॑भिः सुमृडी॒कः सु॑मृडी॒को ऽवो॑भि॒ रवो॑भिः सुमृडी॒को भ॑वतु भवतु सुमृडी॒को ऽवो॑भि॒ रवो॑भिः सुमृडी॒को भ॑वतु । \newline
49. अवो॑भि॒रित्यवः॑ - भिः॒ । \newline
50. सु॒मृ॒डी॒को भ॑वतु भवतु सुमृडी॒कः सु॑मृडी॒को भ॑वतु वि॒श्ववे॑दा वि॒श्ववे॑दा भवतु सुमृडी॒कः सु॑मृडी॒को भ॑वतु वि॒श्ववे॑दाः । \newline
51. सु॒मृ॒डी॒क इति॑ सु - मृ॒डी॒कः । \newline
52. भ॒व॒तु॒ वि॒श्ववे॑दा वि॒श्ववे॑दा भवतु भवतु वि॒श्ववे॑दाः । \newline
53. वि॒श्ववे॑दा॒ इति॑ वि॒श्व - वे॒दाः॒ । \newline
54. बाध॑ता॒म् द्वेषो॒ द्वेषो॒ बाध॑ता॒म् बाध॑ता॒म् द्वेषो॒ अभ॑य॒ मभ॑य॒म् द्वेषो॒ बाध॑ता॒म् बाध॑ता॒म् द्वेषो॒ अभ॑यम् । \newline
55. द्वेषो॒ अभ॑य॒ मभ॑य॒म् द्वेषो॒ द्वेषो॒ अभ॑यम् कृणोतु कृणो॒ त्वभ॑य॒म् द्वेषो॒ द्वेषो॒ अभ॑यम् कृणोतु । \newline
56. अभ॑यम् कृणोतु कृणो॒ त्वभ॑य॒ मभ॑यम् कृणोतु सु॒वीर्य॑स्य सु॒वीर्य॑स्य कृणो॒ त्वभ॑य॒ मभ॑यम् कृणोतु सु॒वीर्य॑स्य । \newline
57. कृ॒णो॒तु॒ सु॒वीर्य॑स्य सु॒वीर्य॑स्य कृणोतु कृणोतु सु॒वीर्य॑स्य॒ पत॑यः॒ पत॑यः सु॒वीर्य॑स्य कृणोतु कृणोतु सु॒वीर्य॑स्य॒ पत॑यः । \newline
58. सु॒वीर्य॑स्य॒ पत॑यः॒ पत॑यः सु॒वीर्य॑स्य सु॒वीर्य॑स्य॒ पत॑यः स्याम स्याम॒ पत॑यः सु॒वीर्य॑स्य सु॒वीर्य॑स्य॒ पत॑यः स्याम । \newline
59. सु॒वीर्य॒स्येति॑ सु - वीर्य॑स्य । \newline
\pagebreak
\markright{ TS 1.7.13.5  \hfill https://www.vedavms.in \hfill}
\addcontentsline{toc}{section}{ TS 1.7.13.5 }
\section*{ TS 1.7.13.5 }

\textbf{TS 1.7.13.5 } \newline
\textbf{Samhita Paata} \newline

पत॑यः स्याम ॥ तस्य॑ व॒यꣳ सु॑म॒तौ य॒ज्ञिय॒स्यापि॑ भ॒द्रे सौ॑मन॒से स्या॑म । स सु॒त्रामा॒ स्ववाꣳ॒॒ इन्द्रो॑ अ॒स्मे आ॒राच्चि॒द्-द्वेषः॑ सनु॒तर् यु॑योतु ॥ रे॒वती᳚र् नः सध॒माद॒ इन्द्रे॑ सन्तु तु॒विवा॑जाः । क्षु॒मन्तो॒ याभि॒र् मदे॑म ॥ प्रोष्व॑स्मै पुरोर॒थमिन्द्रा॑य शू॒षम॑र्चत । अ॒भीके॑ चिदु लोक॒कृथ् स॒ङ्गे स॒मथ्सु॑ वृत्र॒हा । अ॒स्माकं॑ बोधि चोदि॒ता नभ॑न्ता-मन्य॒केषां᳚ । ज्या॒का अधि॒ ( ) धन्व॑सु ॥ \newline

\textbf{Pada Paata} \newline

पत॑यः । स्या॒म॒ ॥ तस्य॑ । व॒यम् । सु॒म॒ताविति॑ सु-म॒तौ । य॒ज्ञिय॑स्य । अपीति॑ । भ॒द्रे । सौ॒म॒न॒से । स्या॒म॒ ॥ सः । सु॒त्रामेति॑ सु - त्रामा᳚ । स्ववा॒निति॒ स्व-वा॒न् । इन्द्रः॑ । अ॒स्मे इति॑ । आ॒रात् । चि॒त् । द्वेषः॑ । स॒नु॒तः । यु॒यो॒तु॒ ॥ रे॒वतीः᳚ । नः॒ । स॒ध॒माद॒ इति॑ सध - मादः॑ । इन्द्रे᳚ । स॒न्तु॒ । तु॒विवा॑जा॒ इति॑ तु॒वि - वा॒जाः॒ ॥ क्षु॒मन्तः॑ । याभिः॑ । मदे॑म ॥ प्रो इति॑ । स्विति॑ । अ॒स्मै॒ । पु॒रो॒र॒थमिति॑ पुरः - र॒थम् । इन्द्रा॑य । शू॒षम् । अ॒र्च॒त॒ ॥ अ॒भीके᳚ । चि॒त् । उ॒ । लो॒क॒कृदिति॑ लोक - कृत् । स॒ङ्गे । स॒मथ्स्विति॑ स॒मत् - सु॒ । वृ॒त्र॒हेति॑ वृत्र - हा ॥ अ॒स्माक᳚म् । बो॒धि॒ । चो॒दि॒ता । नभ॑न्ताम् । अ॒न्य॒केषा᳚म् ॥ ज्या॒काः । अधीति॑ ( ) । धन्व॒स्विति॒ धन्व॑ - सु॒ ॥  \newline


\textbf{Krama Paata} \newline

पत॑यः स्याम । स्या॒मेति॑ स्याम ॥ तस्य॑ व॒यम् । व॒यꣳ सु॑म॒तौ । सु॒म॒तौ य॒ज्ञिय॑स्य । सु॒म॒ताविति॑ सु - म॒तौ । य॒ज्ञिय॒स्यापि॑ । अपि॑ भ॒द्रे । भ॒द्रे सौ॑मन॒से । सौ॒म॒न॒से स्या॑म । स्या॒मेति॑ स्याम ॥ स सु॒त्रामा᳚ । सु॒त्रामा॒ स्ववान्॑ । सु॒त्रामेति॑ सु - त्रामा᳚ । स्ववाꣳ॒॒ इन्द्रः॑ । स्ववा॒निति॒ स्व - वा॒न्॒ । इन्द्रो॑ अ॒स्मे । अ॒स्मे आ॒रात् । अ॒स्मे इत्य॒स्मे । आ॒राच्चि॑त् । चि॒द् द्वेषः॑ । द्वेषः॑ सनु॒तः । स॒नु॒तर् यु॑योतु । यु॒यो॒त्विति॑ युयोतु ॥ रे॒वती᳚र् नः । नः॒ स॒ध॒मादः॑ । स॒ध॒माद॒ इन्द्रे᳚ । स॒ध॒माद॒ इति॑ सध - मादः॑ । इन्द्रे॑ सन्तु । स॒न्तु॒ तु॒विवा॑जाः । तु॒विवा॑जा॒ इति॑ तु॒वि - वा॒जाः॒ ॥ क्षु॒मन्तो॒ याभिः॑ । याभि॒र्,मदे॑म । मदे॒मेति॒ मदे॑म ॥ प्रो षु॑ । प्रो इति॒ प्रो । स्व॑स्मै । अ॒स्मै॒ पु॒रो॒र॒थम् । पु॒रो॒र॒थमिन्द्रा॑य । पु॒रो॒र॒थमिति॑ पुरः - र॒थम् । इन्द्रा॑य शू॒षम् । शू॒षम॑र्चत । अ॒र्च॒तेत्य॑र्चत ॥ अ॒भीके॑ चित् । चिदु॑ । उ॒ लो॒क॒कृत् । लो॒क॒कृथ् स॒ङ्गे । लो॒क॒कृदिति॑ लोक - कृत् । स॒ङ्गे स॒मथ्सु॑ । स॒मथ्सु॑ वृत्र॒हा । स॒मथ्स्विति॑ स॒मत् - सु॒ । वृ॒त्र॒हेति॑ वृत्र - हा ॥ अ॒स्माक॑म् बोधि । बो॒धि॒ चो॒दि॒ता । चो॒दि॒ता नभ॑न्ताम् । नभ॑न्तामन्य॒केषा᳚म् । अ॒न्य॒केषा॒मित्य॑न्य॒केषा᳚म् ॥ ज्या॒का अधि॑ ( ) । अधि॒ धन्व॑सु । धन्व॒स्विति॒ धन्व॑ - सु॒ । \newline

\textbf{Jatai Paata} \newline

1. पत॑यः स्याम स्याम॒ पत॑यः॒ पत॑यः स्याम । \newline
2. स्या॒मेति॑ स्याम । \newline
3. तस्य॑ व॒यं ॅव॒यम् तस्य॒ तस्य॑ व॒यम् । \newline
4. व॒यꣳ सु॑म॒तौ सु॑म॒तौ व॒यं ॅव॒यꣳ सु॑म॒तौ । \newline
5. सु॒म॒तौ य॒ज्ञिय॑स्य य॒ज्ञिय॑स्य सुम॒तौ सु॑म॒तौ य॒ज्ञिय॑स्य । \newline
6. सु॒म॒ताविति॑ सु - म॒तौ । \newline
7. य॒ज्ञिय॒ स्या प्यपि॑ य॒ज्ञिय॑स्य य॒ज्ञिय॒ स्यापि॑ । \newline
8. अपि॑ भ॒द्रे भ॒द्रे अप्यपि॑ भ॒द्रे । \newline
9. भ॒द्रे सौ॑मन॒से सौ॑मन॒से भ॒द्रे भ॒द्रे सौ॑मन॒से । \newline
10. सौ॒म॒न॒से स्या॑म स्याम सौमन॒से सौ॑मन॒से स्या॑म । \newline
11. स्या॒मेति॑ स्याम । \newline
12. स सु॒त्रामा॑ सु॒त्रामा॒ स स सु॒त्रामा᳚ । \newline
13. सु॒त्रामा॒ स्ववा॒न् थ्स्ववा᳚न् थ्सु॒त्रामा॑ सु॒त्रामा॒ स्ववान्॑ । \newline
14. सु॒त्रामेति॑ सु - त्रामा᳚ । \newline
15. स्ववाꣳ॒॒ इन्द्र॒ इन्द्रः॒ स्ववा॒न् थ्स्ववाꣳ॒॒ इन्द्रः॑ । \newline
16. स्ववा॒निति॒ स्व - वा॒न् । \newline
17. इन्द्रो॑ अ॒स्मे अ॒स्मे इन्द्र॒ इन्द्रो॑ अ॒स्मे । \newline
18. अ॒स्मे आ॒रा दा॒रा द॒स्मे अ॒स्मे आ॒रात् । \newline
19. अ॒स्मे इत्य॒स्मे । \newline
20. आ॒राच् चि॑च् चि दा॒रा दा॒राच् चि॑त् । \newline
21. चि॒द् द्वेषो॒ द्वेष॑ श्चिच् चि॒द् द्वेषः॑ । \newline
22. द्वेषः॑ सनु॒तः स॑नु॒तर् द्वेषो॒ द्वेषः॑ सनु॒तः । \newline
23. स॒नु॒तर् यु॑योतु युयोतु सनु॒तः स॑नु॒तर् यु॑योतु । \newline
24. यु॒यो॒त्विति॑ युयोतु । \newline
25. रे॒वती᳚र् नो नो रे॒वती॑ रे॒वती᳚र् नः । \newline
26. नः॒ स॒ध॒मादः॑ सध॒मादो॑ नो नः सध॒मादः॑ । \newline
27. स॒ध॒माद॒ इन्द्र॒ इन्द्रे॑ सध॒मादः॑ सध॒माद॒ इन्द्रे᳚ । \newline
28. स॒ध॒माद॒ इति॑ सध - मादः॑ । \newline
29. इन्द्रे॑ सन्तु स॒न्त्विन्द्र॒ इन्द्रे॑ सन्तु । \newline
30. स॒न्तु॒ तु॒विवा॑जा स्तु॒विवा॑जाः सन्तु सन्तु तु॒विवा॑जाः । \newline
31. तु॒विवा॑जा॒ इति॑ तु॒वि - वा॒जाः॒ । \newline
32. क्षु॒मन्तो॒ याभि॒र् याभिः॑ क्षु॒मन्तः॑ क्षु॒मन्तो॒ याभिः॑ । \newline
33. याभि॒र् मदे॑म॒ मदे॑म॒ याभि॒र् याभि॒र् मदे॑म । \newline
34. मदे॒मेति॒ मदे॑म । \newline
35. प्रो षु सु प्रो प्रो षु । \newline
36. प्रो इति॒ प्रो । \newline
37. स्व॑स्मा अस्मै॒ सु स्व॑स्मै । \newline
38. अ॒स्मै॒ पु॒रो॒र॒थम् पु॑रोर॒थ म॑स्मा अस्मै पुरोर॒थम् । \newline
39. पु॒रो॒र॒थ मिन्द्रा॒ये न्द्रा॑य पुरोर॒थम् पु॑रोर॒थ मिन्द्रा॑य । \newline
40. पु॒रो॒र॒थमिति॑ पुरः - र॒थम् । \newline
41. इन्द्रा॑य शू॒षꣳ शू॒ष मिन्द्रा॒ये न्द्रा॑य शू॒षम् । \newline
42. शू॒ष म॑र्चतार्चत शू॒षꣳ शू॒ष म॑र्चत । \newline
43. अ॒र्च॒तेत्य॑र्चत । \newline
44. अ॒भीके॑ चिच् चिद॒भीके॑ अ॒भीके॑ चित् । \newline
45. चि॒दु॒ वु॒ चि॒च् चि॒दु॒ । \newline
46. उ॒ लो॒क॒कृ ल्लो॑क॒कृ दु॑ वु लोक॒कृत् । \newline
47. लो॒क॒कृथ् स॒ङ्गे स॒ङ्गे लो॑क॒कृ ल्लो॑क॒कृथ् स॒ङ्गे । \newline
48. लो॒क॒कृदिति॑ लोक - कृत् । \newline
49. स॒ङ्गे स॒मथ्सु॑ स॒मथ्सु॑ स॒ङ्गे स॒ङ्गे स॒मथ्सु॑ । \newline
50. स॒मथ्सु॑ वृत्र॒हा वृ॑त्र॒हा स॒मथ्सु॑ स॒मथ्सु॑ वृत्र॒हा । \newline
51. स॒मथ्स्विति॑ स॒मत् - सु॒ । \newline
52. वृ॒त्र॒हेति॑ वृत्र - हा । \newline
53. अ॒स्माक॑म् बोधि बोध्य॒स्माक॑ म॒स्माक॑म् बोधि । \newline
54. बो॒धि॒ चो॒दि॒ता चो॑दि॒ता बो॑धि बोधि चोदि॒ता । \newline
55. चो॒दि॒ता नभ॑न्ता॒म् नभ॑न्ताम् चोदि॒ता चो॑दि॒ता नभ॑न्ताम् । \newline
56. नभ॑न्ता मन्य॒केषा॑ मन्य॒केषा॒म् नभ॑न्ता॒म् नभ॑न्ता मन्य॒केषा᳚म् । \newline
57. अ॒न्य॒केषा॒मित्य॑न्य॒केषा᳚म् । \newline
58. ज्या॒का अध्यधि॑ ज्या॒का ज्या॒का अधि॑ । \newline
59. अधि॒ धन्व॑सु॒ धन्व॒ स्वध्यधि॒ धन्व॑सु । \newline
60. धन्व॒स्विति॒ धन्व॑ - सु॒ । \newline

\textbf{Ghana Paata } \newline

1. पत॑यः स्याम स्याम॒ पत॑यः॒ पत॑यः स्याम । \newline
2. स्या॒मेति॑ स्याम । \newline
3. तस्य॑ व॒यं ॅव॒यम् तस्य॒ तस्य॑ व॒यꣳ सु॑म॒तौ सु॑म॒तौ व॒यम् तस्य॒ तस्य॑ व॒यꣳ सु॑म॒तौ । \newline
4. व॒यꣳ सु॑म॒तौ सु॑म॒तौ व॒यं ॅव॒यꣳ सु॑म॒तौ य॒ज्ञिय॑स्य य॒ज्ञिय॑स्य सुम॒तौ व॒यं ॅव॒यꣳ सु॑म॒तौ य॒ज्ञिय॑स्य । \newline
5. सु॒म॒तौ य॒ज्ञिय॑स्य य॒ज्ञिय॑स्य सुम॒तौ सु॑म॒तौ य॒ज्ञिय॒ स्याप्यपि॑ य॒ज्ञिय॑स्य सुम॒तौ सु॑म॒तौ य॒ज्ञिय॒स्यापि॑ । \newline
6. सु॒म॒ताविति॑ सु - म॒तौ । \newline
7. य॒ज्ञिय॒ स्याप्यपि॑ य॒ज्ञिय॑स्य य॒ज्ञिय॒ स्यापि॑ भ॒द्रे भ॒द्रे अपि॑ य॒ज्ञिय॑स्य य॒ज्ञिय॒ स्यापि॑ भ॒द्रे । \newline
8. अपि॑ भ॒द्रे भ॒द्रे अप्यपि॑ भ॒द्रे सौ॑मन॒से सौ॑मन॒से भ॒द्रे अप्यपि॑ भ॒द्रे सौ॑मन॒से । \newline
9. भ॒द्रे सौ॑मन॒से सौ॑मन॒से भ॒द्रे भ॒द्रे सौ॑मन॒से स्या॑म स्याम सौमन॒से भ॒द्रे भ॒द्रे सौ॑मन॒से स्या॑म । \newline
10. सौ॒म॒न॒से स्या॑म स्याम सौमन॒से सौ॑मन॒से स्या॑म । \newline
11. स्या॒मेति॑ स्याम । \newline
12. स सु॒त्रामा॑ सु॒त्रामा॒ स स सु॒त्रामा॒ स्ववा॒न् थ्स्ववा᳚न् थ्सु॒त्रामा॒ स स सु॒त्रामा॒ स्ववान्॑ । \newline
13. सु॒त्रामा॒ स्ववा॒न् थ्स्ववा᳚न् थ्सु॒त्रामा॑ सु॒त्रामा॒ स्ववा॒(ग्म्॒) इन्द्र॒ इन्द्रः॒ स्ववा᳚न् थ्सु॒त्रामा॑ सु॒त्रामा॒ स्ववा॒(ग्म्॒) इन्द्रः॑ । \newline
14. सु॒त्रामेति॑ सु - त्रामा᳚ । \newline
15. स्ववा॒(ग्म्॒) इन्द्र॒ इन्द्रः॒ स्ववा॒न् थ्स्ववा॒(ग्म्॒) इन्द्रो॑ अ॒स्मे अ॒स्मे इन्द्रः॒ स्ववा॒न् थ्स्ववा॒(ग्म्॒) इन्द्रो॑ अ॒स्मे । \newline
16. स्ववा॒निति॒ स्व - वा॒न् । \newline
17. इन्द्रो॑ अ॒स्मे अ॒स्मे इन्द्र॒ इन्द्रो॑ अ॒स्मे आ॒रा दा॒रा द॒स्मे इन्द्र॒ इन्द्रो॑ अ॒स्मे आ॒रात् । \newline
18. अ॒स्मे आ॒रा दा॒रा द॒स्मे अ॒स्मे आ॒राच् चि॑च् चिदा॒रा द॒स्मे अ॒स्मे आ॒राच् चि॑त् । \newline
19. अ॒स्मे इत्य॒स्मे । \newline
20. आ॒राच् चि॑च् चिदा॒रा दा॒राच् चि॒द् द्वेषो॒ द्वे ष॑ श्चि दा॒रा दा॒राच् चि॒द् द्वेषः॑ । \newline
21. चि॒द् द्वेषो॒ द्वेष॑ श्चिच् चि॒द् द्वेषः॑ सनु॒तः स॑नु॒तर् द्वेष॑ श्चिच् चि॒द् द्वेषः॑ सनु॒तः । \newline
22. द्वेषः॑ सनु॒तः स॑नु॒तर् द्वेषो॒ द्वेषः॑ सनु॒तर् यु॑योतु युयोतु सनु॒तर् द्वेषो॒ द्वेषः॑ सनु॒तर् यु॑योतु । \newline
23. स॒नु॒तर् यु॑योतु युयोतु सनु॒तः स॑नु॒तर् यु॑योतु । \newline
24. यु॒यो॒त्विति॑ युयोतु । \newline
25. रे॒वती᳚र् नो नो रे॒वती॑ रे॒वती᳚र् नः सध॒मादः॑ सध॒मादो॑ नो रे॒वती॑ रे॒वती᳚र् नः सध॒मादः॑ । \newline
26. नः॒ स॒ध॒मादः॑ सध॒मादो॑ नो नः सध॒माद॒ इन्द्र॒ इन्द्रे॑ सध॒मादो॑ नो नः सध॒माद॒ इन्द्रे᳚ । \newline
27. स॒ध॒माद॒ इन्द्र॒ इन्द्रे॑ सध॒मादः॑ सध॒माद॒ इन्द्रे॑ सन्तु स॒न्त्विन्द्रे॑ सध॒मादः॑ सध॒माद॒ इन्द्रे॑ सन्तु । \newline
28. स॒ध॒माद॒ इति॑ सध - मादः॑ । \newline
29. इन्द्रे॑ सन्तु स॒न्त्विन्द्र॒ इन्द्रे॑ सन्तु तु॒विवा॑जा स्तु॒विवा॑जाः स॒न्त्विन्द्र॒ इन्द्रे॑ सन्तु तु॒विवा॑जाः । \newline
30. स॒न्तु॒ तु॒विवा॑जा स्तु॒विवा॑जाः सन्तु सन्तु तु॒विवा॑जाः । \newline
31. तु॒विवा॑जा॒ इति॑ तु॒वि - वा॒जाः॒ । \newline
32. क्षु॒मन्तो॒ याभि॒र् याभिः॑ क्षु॒मन्तः॑ क्षु॒मन्तो॒ याभि॒र् मदे॑म॒ मदे॑म॒ याभिः॑ क्षु॒मन्तः॑ क्षु॒मन्तो॒ याभि॒र् मदे॑म । \newline
33. याभि॒र् मदे॑म॒ मदे॑म॒ याभि॒र् याभि॒र् मदे॑म । \newline
34. मदे॒मेति॒ मदे॑म । \newline
35. प्रो षु सु प्रो प्रो ष्व॑स्मा अस्मै॒ सु प्रो प्रो ष्व॑स्मै । \newline
36. प्रो इति॒ प्रो । \newline
37. स्व॑स्मा अस्मै॒ सु स्व॑स्मै पुरोर॒थम् पु॑रोर॒थ म॑स्मै॒ सु स्व॑स्मै पुरोर॒थम् । \newline
38. अ॒स्मै॒ पु॒रो॒र॒थम् पु॑रोर॒थ म॑स्मा अस्मै पुरोर॒थ मिन्द्रा॒ये न्द्रा॑य पुरोर॒थ म॑स्मा अस्मै पुरोर॒थ मिन्द्रा॑य । \newline
39. पु॒रो॒र॒थ मिन्द्रा॒ये न्द्रा॑य पुरोर॒थम् पु॑रोर॒थ मिन्द्रा॑य शू॒षꣳ शू॒ष मिन्द्रा॑य पुरोर॒थम् पु॑रोर॒थ मिन्द्रा॑य शू॒षम् । \newline
40. पु॒रो॒र॒थमिति॑ पुरः - र॒थम् । \newline
41. इन्द्रा॑य शू॒षꣳ शू॒ष मिन्द्रा॒ये न्द्रा॑य शू॒ष म॑र्चतार्चत शू॒ष मिन्द्रा॒ये न्द्रा॑य शू॒ष म॑र्चत । \newline
42. शू॒ष म॑र्चतार्चत शू॒षꣳ शू॒ष म॑र्चत । \newline
43. अ॒र्च॒तेत्य॑र्चत । \newline
44. अ॒भीके॑ चिच् चिद॒भीके॑ अ॒भीके॑ चिदु वु चिद॒भीके॑ अ॒भीके॑ चिदु । \newline
45. चि॒दु॒ वु॒ चि॒च् चि॒दु॒ लो॒क॒कृ ल्लो॑क॒कृ दु॑ चिच् चिदु लोक॒कृत् । \newline
46. उ॒ लो॒क॒कृ ल्लो॑क॒कृदु॑ वु लोक॒कृथ् स॒ङ्गे स॒ङ्गे लो॑क॒कृदु॑ वु लोक॒कृथ् स॒ङ्गे । \newline
47. लो॒क॒कृथ् स॒ङ्गे स॒ङ्गे लो॑क॒कृ ल्लो॑क॒कृथ् स॒ङ्गे स॒मथ्सु॑ स॒मथ्सु॑ स॒ङ्गे लो॑क॒कृ ल्लो॑क॒कृथ् स॒ङ्गे स॒मथ्सु॑ । \newline
48. लो॒क॒कृदिति॑ लोक - कृत् । \newline
49. स॒ङ्गे स॒मथ्सु॑ स॒मथ्सु॑ स॒ङ्गे स॒ङ्गे स॒मथ्सु॑ वृत्र॒हा वृ॑त्र॒हा स॒मथ्सु॑ स॒ङ्गे स॒ङ्गे स॒मथ्सु॑ वृत्र॒हा । \newline
50. स॒मथ्सु॑ वृत्र॒हा वृ॑त्र॒हा स॒मथ्सु॑ स॒मथ्सु॑ वृत्र॒हा । \newline
51. स॒मथ्स्विति॑ स॒मत् - सु॒ । \newline
52. वृ॒त्र॒हेति॑ वृत्र - हा । \newline
53. अ॒स्माक॑म् बोधि बोध्य॒स्माक॑ म॒स्माक॑म् बोधि चोदि॒ता चो॑दि॒ता बो᳚ध्य॒स्माक॑ म॒स्माक॑म् बोधि चोदि॒ता । \newline
54. बो॒धि॒ चो॒दि॒ता चो॑दि॒ता बो॑धि बोधि चोदि॒ता नभ॑न्ता॒न् नभ॑न्ताम् चोदि॒ता बो॑धि बोधि चोदि॒ता नभ॑न्ताम् । \newline
55. चो॒दि॒ता नभ॑न्ता॒न् नभ॑न्ताम् चोदि॒ता चो॑दि॒ता नभ॑न्ता मन्य॒केषा॑ मन्य॒केषा॒न् नभ॑न्ताम् चोदि॒ता चो॑दि॒ता नभ॑न्ता मन्य॒केषा᳚म् । \newline
56. नभ॑न्ता मन्य॒केषा॑ मन्य॒केषा॒न् नभ॑न्ता॒न् नभ॑न्ता मन्य॒केषा᳚म् । \newline
57. अ॒न्य॒केषा॒मित्य॑न्य॒केषा᳚म् । \newline
58. ज्या॒का अध्यधि॑ ज्या॒का ज्या॒का अधि॒ धन्व॑सु॒ धन्व॒स्वधि॑ ज्या॒का ज्या॒का अधि॒ धन्व॑सु । \newline
59. अधि॒ धन्व॑सु॒ धन्व॒स्वध्यधि॒ धन्व॑सु । \newline
60. धन्व॒स्विति॒ धन्व॑ - सु॒ । \newline
\pagebreak


\end{document}