\documentclass[17pt]{extarticle}
\usepackage{babel}
\usepackage{fontspec}
\usepackage{polyglossia}
\usepackage{extsizes}



\setmainlanguage{sanskrit}
\setotherlanguages{english} %% or other languages
\setlength{\parindent}{0pt}
\pagestyle{myheadings}
\newfontfamily\devanagarifont[Script=Devanagari]{AdishilaVedic}


\newcommand{\VAR}[1]{}
\newcommand{\BLOCK}[1]{}




\begin{document}
\begin{titlepage}
    \begin{center}
 
\begin{sanskrit}
    { \Huge
    कृष्ण यजुर्वेदीय तैत्तिरीय संहिता,पद,जटा,घन पाठः 
    }
    \\
    \vspace{2.5cm}
    \mbox{ \Huge
    4.5      चतुर्थकाण्डे पञ्चमः प्रश्नः - होमविधिनिरूपणं   }
\end{sanskrit}
\end{center}

\end{titlepage}
\tableofcontents
\pagebreak

\markright{ TS 4.5.1.1  \hfill https://www.vedavms.in \hfill}
\addcontentsline{toc}{section}{ TS 4.5.1.1 }
\section*{ TS 4.5.1.1 }

\textbf{TS 4.5.1.1 } \newline
\textbf{Samhita Paata} \newline

नम॑स्ते रुद्र म॒न्यव॑ उ॒तोत॒ इष॑वे॒ नमः॑ । नम॑स्ते अस्तु॒ धन्व॑ने बा॒हुभ्या॑मु॒त ते॒ नमः॑ ॥                                   या त॒ इषुः॑ शि॒वत॑मा शि॒वं ब॒भूव॑ ते॒ धनुः॑ । शि॒वा श॑र॒व्या॑ या तव॒ तया॑ नो रुद्र मृडय ॥     या ते॑ रुद्र शि॒वा त॒नूरघो॒रा ऽपा॑पकाशिनी । तया॑ नस्त॒नुवा॒ शन्त॑मया॒ गिरि॑शन्ता॒भि चा॑कशीहि ॥                                यामिषुं॑ गिरिशन्त॒ हस्ते॒ - [  ] \newline

\textbf{Pada Paata} \newline

नमः॑ । ते॒ । रु॒द्र॒ । म॒न्यवे᳚ । उ॒तो इति॑ । ते॒ । इष॑वे । नमः॑ ॥ नमः॑ । ते॒ । अ॒स्तु॒ । धन्व॑ने । बा॒हुभ्या॒मिति॑ बा॒हु-भ्या॒म् । उ॒त । ते॒ । नमः॑ ॥ या । ते॒ । इषुः॑ । शि॒वत॒मेति॑ शि॒व - त॒मा॒ । शि॒वम् । ब॒भूव॑ । ते॒ । धनुः॑ ॥ शि॒वा । श॒र॒व्या᳚ । या । तव॑ । तया᳚ । नः॒ । रु॒द्र॒ । मृ॒ड॒य॒ ॥ या । ते॒ । रु॒द्र॒ । शि॒वा । त॒नूः । अघो॑रा । अपा॑पकाशि॒नीत्यपा॑प-का॒शि॒नी॒ ॥ तया᳚ । नः॒ । त॒नुवा᳚ । शन्त॑म॒येति॒ शं - त॒म॒या॒ । गिरि॑श॒न्तेति॒ गिरि॑ - श॒न्त॒ । अ॒भीति॑ । चा॒क॒शी॒हि॒ ॥ याम् । इषु᳚म् । गि॒रि॒श॒न्तेति॑ गिरि - श॒न्त॒ । हस्ते᳚ ।  \newline




\markright{ TS 4.5.1.2  \hfill https://www.vedavms.in \hfill}
\addcontentsline{toc}{section}{ TS 4.5.1.2 }
\section*{ TS 4.5.1.2 }

\textbf{TS 4.5.1.2 } \newline
\textbf{Samhita Paata} \newline

बिभ॒र्ष्यस्त॑वे । शि॒वां गि॑रित्र॒ तां कु॑रु॒ मा हिꣳ॑सी॒ः॒ पुरु॑षं॒ जग॑त् ॥        शि॒वेन॒ वच॑सा त्वा॒ गिरि॒शाच्छा॑ वदामसि ।यथा॑ नः॒ सर्व॒मि-ज्जग॑दय॒क्ष्मꣳ सु॒मना॒ अस॑त् ॥                             अद्ध्य॑वोचदधिव॒क्ता प्र॑थ॒मो दैव्यो॑ भि॒षक् । अहीꣳ॑श्च॒॒ सर्वा᳚न् ज॒भंय॒न्थ् सर्वा᳚श्च यातु धा॒न्यः॑ ॥                          अ॒सौ यस्ता॒म्रो अ॑रु॒ण उ॒त ब॒भ्रुः सु॑म॒ङ्गलः॑ । ये चे॒माꣳ रु॒द्रा अ॒भितो॑ दि॒क्षु - [  ] \newline

\textbf{Pada Paata} \newline

बिभ॑र्.षि । अस्त॑वे ॥ शि॒वाम् । गि॒रि॒त्रेति॑ गिरि - त्र॒ । ताम् । कु॒रु॒ । मा । हिꣳ॒॒सीः॒ । पुरु॑षम् । जग॑त् ॥ शि॒वेन॑ । वच॑सा । त्वा॒ । गिरि॑श । अच्छ॑ । व॒दा॒म॒सि॒ ॥ यथा᳚ । नः॒ । सर्व᳚म् । इत् । जग॑त् । अ॒य॒क्ष्मम् । सु॒मना॒ इति॑ सु - मनाः᳚ । अस॑त् ॥ अधीति॑ । अ॒वो॒च॒त् । अ॒धि॒व॒क्तेत्य॑धि - व॒क्ता । प्र॒थ॒मः । दैव्यः॑ । भि॒षक् ॥ अहीन्॑ । च॒ । सर्वान्॑ । ज॒भंयन्न्॑ । सर्वाः᳚ । च॒ । या॒तु॒धा॒न्य॑ इति॑ यातु - धा॒न्यः॑ ॥ अ॒सौ । यः । ता॒म्रः । अ॒रु॒णः । उ॒त । ब॒भ्रुः । सु॒म॒ङ्गल॒ इति॑ सु - म॒ङ्गलः॑ ॥ ये । च॒ । इ॒माम् । रु॒द्राः । अ॒भितः॑ । दि॒क्षु ।  \newline




\markright{ TS 4.5.1.3  \hfill https://www.vedavms.in \hfill}
\addcontentsline{toc}{section}{ TS 4.5.1.3 }
\section*{ TS 4.5.1.3 }

\textbf{TS 4.5.1.3 } \newline
\textbf{Samhita Paata} \newline

श्रि॒ताः स॑हस्र॒शो ऽवै॑षाꣳ॒॒ हेड॑ ईमहे ॥ अ॒सौ यो॑ ऽव॒सर्प॑ति॒ नील॑ग्रीवो॒ विलो॑हितः । उ॒तैनं॑ गो॒पा अ॑दृश॒न्-नदृ॑शन्-नुदहा॒र्यः॑ । उ॒तैनं॒ ॅविश्वा॑ भू॒तानि॒ स दृ॒ष्टो मृ॑डयाति नः ॥                               नमो॑ अस्तु॒ नील॑ग्रीवाय सहस्रा॒क्षाय॑ मी॒ढुषे᳚ । अथो॒ ये अ॑स्य॒ सत्वा॑नो॒ऽहं तेभ्यो॑ ऽकर॒न्नमः॑ ॥                                   प्रमुं॑च॒ धन्व॑न॒स्त्व मु॒भयो॒-रार्त्नि॑यो॒र्ज्यां । याश्च॑ ते॒ हस्त॒ इष॑वः॒ - [  ] \newline

\textbf{Pada Paata} \newline

श्रि॒ताः । स॒ह॒स्र॒श इति॑ सहस्र - शः । अवेति॑ । ए॒षा॒म् । हेडः॑ । ई॒म॒हे॒ ॥ अ॒सौ । यः । अ॒व॒सर्प॒तीत्य॑व - सर्प॑ति । नील॑ग्रीव॒ इति॒ नील॑ - ग्री॒वः॒ । विलो॑हित॒ इति॒ वि - लो॒हि॒तः॒ ॥ उ॒त । ए॒न॒म् । गो॒पा इति॑ गो - पाः । अ॒दृ॒श॒न्न् । अदृ॑शन्न् । उ॒द॒हा॒र्य॑ इत्युद॑-हा॒र्यः॑ ॥ उ॒त । ए॒न॒म् । विश्वा᳚ । भू॒तानि॑ । सः । दृ॒ष्टः । मृ॒ड॒या॒ति॒ । नः॒ ॥ नमः॑ । अ॒स्तु॒ । नील॑ग्रीवा॒येति॒ नील॑ - ग्री॒वा॒य॒ । स॒ह॒स्रा॒क्षायेति॑ सहस्र - अ॒क्षाय॑ । मी॒ढुषे᳚ ॥ अथो॒ इति॑ । ये । अ॒स्य॒ । सत्वा॑नः । अ॒हम् । तेभ्यः॑ । अ॒क॒र॒म् । नमः॑ ॥ प्रेति॑ । मु॒ञ्च॒ । धन्व॑नः । त्वम् । उ॒भयोः᳚ । आर्त्नि॑योः । ज्याम् ॥ याः । च॒ । ते॒ । हस्ते᳚ । इष॑वः ।  \newline




\markright{ TS 4.5.1.4  \hfill https://www.vedavms.in \hfill}
\addcontentsline{toc}{section}{ TS 4.5.1.4 }
\section*{ TS 4.5.1.4 }

\textbf{TS 4.5.1.4 } \newline
\textbf{Samhita Paata} \newline

परा॒ ता भ॑गवो वप ॥                                        अ॒व॒तत्य॒ धनु॒स्त्वꣳ सह॑स्राक्ष॒ शते॑षुधे । नि॒शीर्य॑ श॒ल्यानां॒ मुखा॑ शि॒वो नः॑ सु॒मना॑ भव ॥                          विज्यं॒ धनुः॑ कप॒र्दिनो॒ विश॑ल्यो॒ बाण॑वाꣳ उ॒त । अने॑शन्न॒स्येष॑व आ॒भुर॑स्य निष॒ङ्गथिः॑ ॥                                               या ते॑ हे॒ति-र्मी॑ढुष्टम॒ हस्ते॑ ब॒भूव॑ ते॒ धनुः॑ । तया॒ऽस्मान्. वि॒श्वत॒ स्त्वम॑य॒क्ष्मया॒ परि॑ब्भुज ॥                                  नम॑स्ते अ॒स्त्वायु॑धा॒या-ना॑तताय धृ॒ष्णवे᳚ । उ॒भाभ्या॑ ( ) मु॒त ते॒ नमो॑ बा॒हुभ्यां॒ तव॒ धन्व॑ने ॥                      परि॑ ते॒ धन्व॑नो हे॒तिर॒स्मान्-वृ॑णक्तु वि॒श्वतः॑ । अथो॒ य इ॑षु॒धिस्तवा॒रे अ॒स्म-न्निधे॑हि॒ तं ॥ \newline

\textbf{Pada Paata} \newline

परेति॑ । ताः । भ॒ग॒व॒ इति॑ भग - वः॒ । व॒प॒ ॥ अ॒व॒तत्येत्य॑व - तत्य॑ । धनुः॑ । त्वम् । सह॑स्रा॒क्षेति॒ सह॑स्र - अ॒क्ष॒ । शते॑षुध॒ इति॒ शत॑ - इ॒षु॒धे॒ ॥ नि॒शीर्येति॑ नि-शीर्य॑ । श॒ल्याना᳚म् । मुखा᳚ । शि॒वः । नः॒ । सु॒मना॒ इति॑ सु - मनाः᳚ । भ॒व॒ ॥ विज्य॒मिति॒ वि - ज्य॒म् । धनुः॑ । क॒प॒र्दिनः॑ । विश॑ल्य॒ इति॒ वि - श॒ल्यः॒ । बाण॑वानिति॒ बाण॑ - वा॒न् । उ॒त ॥ अने॑शन्न् । अ॒स्य॒ । इष॑वः । आ॒भुः । अ॒स्य॒ । नि॒ष॒ङ्गथिः॑ ॥ या । ते॒ । हे॒तिः । मी॒ढु॒ष्ट॒मेति॑ मीढुः - त॒म॒ । हस्ते᳚ । ब॒भूव॑ । ते॒ । धनुः॑ ॥ तया᳚ । अ॒स्मान् । वि॒श्वतः॑ । त्वम् । अ॒य॒क्ष्मया᳚ । परीति॑ । भु॒ज॒ ॥ नमः॑ । ते॒ । अ॒स्तु॒ । आयु॑धाय । अना॑तता॒येत्यना᳚ - त॒ता॒य॒ । धृ॒ष्णवे᳚ ॥ उ॒भाभ्या᳚म् ( ) । उ॒त । ते॒ । नमः॑ । बा॒हुभ्या॒मिति॑ बा॒हु - भ्या॒म् । तव॑ । धन्व॑ने ॥ परीति॑ । ते॒ । धन्व॑नः । हे॒तिः । अ॒स्मान् । वृ॒ण॒क्तु॒ । वि॒श्वतः॑ ॥ अथो॒ इति॑ । यः । इ॒षु॒धिरिती॑षु-धिः । तव॑ । आ॒रे । अ॒स्मत् । नीति॑ । धे॒हि॒ । तम् ॥  \newline




\markright{ TS 4.5.2.1  \hfill https://www.vedavms.in \hfill}
\addcontentsline{toc}{section}{ TS 4.5.2.1 }
\section*{ TS 4.5.2.1 }

\textbf{TS 4.5.2.1 } \newline
\textbf{Samhita Paata} \newline

नमो॒ हिर॑ण्य बाहवे सेना॒न्ये॑ दि॒शांच॒ पत॑ये॒ नमो॒                        नमो॑ वृ॒क्षेभ्यो॒ हरि॑केशेभ्यः पशू॒नां पत॑ये॒ नमो॒                नमः॑ स॒स्पिञ्ज॑राय॒ त्विषी॑मते पथी॒नां पत॑ये॒ नमो॒                    नमो॑ बभ्लु॒शाय॑ विव्या॒धिने-ऽन्ना॑नां॒ पत॑ये॒ नमो॒                  नमो॒ हरि॑केशायो-पवी॒तिने॑ पु॒ष्टानां॒ पत॑ये॒ नमो॒                                 नमो॑ भ॒वस्य॑ हे॒त्यै जग॑तां॒ पत॑ये॒ नमो॒                                                नमो॑ रु॒द्राया॑-तता॒विने॒ क्षेत्रा॑णां॒ पत॑ये॒ नमो॒                          नमः॑ सू॒ताया-ह॑न्त्याय॒ वना॑नां॒ पत॑ये॒ नमो॒ नमो॒ - [  ] \newline

\textbf{Pada Paata} \newline

नमः॑ । हिर॑ण्यबाहव॒ इति॒ हिर॑ण्य - बा॒ह॒वे॒ । से॒ना॒न्य॑ इति॑ सेना - न्ये᳚ । दि॒शाम् । च॒ । पत॑ये । नमः॑ । नमः॑ । वृ॒क्षेभ्यः॑ । हरि॑केशेभ्य॒ इति॒ हरि॑ - के॒शे॒भ्यः॒ । प॒शू॒नाम् । पत॑ये । नमः॑ । नमः॑ । स॒स्पिञ्ज॑राय । त्विषी॑मत॒ इति॒ त्विषि॑ - म॒ते॒ । प॒थी॒नाम् । पत॑ये । नमः॑ । नमः॑ । ब॒भ्लु॒शाय॑ । वि॒व्या॒धिन॒ इति॑ वि - व्या॒धिने᳚ । अन्ना॑नाम् । पत॑ये । नमः॑ । नमः॑ । हरि॑केशा॒येति॒ हरि॑ - के॒शा॒य॒ । उ॒प॒वी॒तिन॒ इत्यु॑प - वी॒तिने᳚ । पु॒ष्टाना᳚म् । पत॑ये । नमः॑ । नमः॑ । भ॒वस्य॑ । हे॒त्यै । जग॑ताम् । पत॑ये । नमः॑ । नमः॑ । रु॒द्राय॑ । आ॒त॒ता॒विन॒ इत्या᳚ - त॒ता॒विने᳚ । क्षेत्रा॑णाम् । पत॑ये । नमः॑ । नमः॑ । सू॒ताय॑ । अह॑न्त्याय । वना॑नाम् । पत॑ये । नमः॑ । नमः॑ ।  \newline




\markright{ TS 4.5.2.2  \hfill https://www.vedavms.in \hfill}
\addcontentsline{toc}{section}{ TS 4.5.2.2 }
\section*{ TS 4.5.2.2 }

\textbf{TS 4.5.2.2 } \newline
\textbf{Samhita Paata} \newline

रोहि॑ताय स्थ॒पत॑ये वृ॒क्षाणां॒ पत॑ये॒ नमो॒                                      नमो॑ म॒न्त्रिणे॑ वाणि॒जाय॒ कक्षा॑णां॒ पत॑ये॒ नमो॒                          नमो॑ भुव॒न्तये॑ वारिवस्कृ॒ता-यौष॑धीनां॒ पत॑ये॒ नमो॒   नम॑ उ॒च्चै-र्घो॑षाया क्र॒न्दय॑ते पत्ती॒नां पत॑ये॒ नमो॒                    नमः॑ कृथ्स्नवी॒ताय॒ धाव॑ते॒ सत्त्व॑नां॒ पत॑ये॒ नमः॑ ॥ \newline

\textbf{Pada Paata} \newline

रोहि॑ताय । स्थ॒पत॑ये । वृ॒क्षाणा᳚म् । पत॑ये । नमः॑ । नमः॑ । म॒न्त्रिणे᳚ । वा॒णि॒जाय॑ । कक्षा॑णाम् । पत॑ये । नमः॑ । नमः॑ । भु॒व॒न्तये᳚ । वा॒रि॒व॒स्कृ॒तायेति॑ वारिवः - कृ॒ताय॑ । ओष॑धीनाम् । पत॑ये । नमः॑ । नमः॑ । उ॒च्चैर्घो॑षा॒येत्यु॒च्चैः - घो॒षा॒य॒ । आ॒क्र॒न्दय॑त॒ इत्या᳚-क्र॒न्दय॑ते । प॒त्ती॒नाम् । पत॑ये । नमः॑ । नमः॑ । कृ॒थ्स्न॒वी॒तायेति॑ कृथ्स्न-वी॒ताय॑ । धाव॑ते । सत्त्व॑नाम् । पत॑ये । नमः॑ ॥  \newline




\markright{ TS 4.5.3.1  \hfill https://www.vedavms.in \hfill}
\addcontentsline{toc}{section}{ TS 4.5.3.1 }
\section*{ TS 4.5.3.1 }

\textbf{TS 4.5.3.1 } \newline
\textbf{Samhita Paata} \newline

नमः॒ सह॑मानाय निव्या॒धिन॑ आव्या॒धिनी॑नां॒ पत॑ये॒ नमो॒          नमः॑ ककु॒भाय॑ निष॒ङ्गिणे᳚ स्ते॒नानां॒ पत॑ये॒ नमो॒                       नमो॑ निष॒ङ्गिण॑ इषुधि॒मते॒ तस्क॑राणां॒ पत॑ये॒ नमो॒           नमो॒ वञ्च॑ते परि॒वञ्च॑ते स्तायू॒नां पत॑ये॒ नमो॒                                         नमो॑ निचे॒रवे॑ परिच॒रायार॑ण्यानां॒ पत॑ये॒ नमो॒                                   नमः॑ सृका॒विभ्यो॒ जिघाꣳ॑सद्भ्यो मुष्ण॒तां पत॑ये॒ नमो॒                   नमो॑ ऽसि॒मद्भ्यो॒ नक्तं॒ चर॑द्भ्यः प्रकृ॒न्तानां॒ पत॑ये॒ नमो॒ नम॑ उष्णी॒षिणे॑ गिरिच॒राय॑ कुलु॒ञ्चानां॒ पत॑ये॒ नमो॒ नम॒ - [  ] \newline

\textbf{Pada Paata} \newline

नमः॑ । सह॑मानाय । नि॒व्या॒धिन॒ इति॑ नि - व्या॒धिने᳚ । आ॒व्या॒धिनी॑ना॒मित्या᳚ - व्या॒धिनी॑नाम् । पत॑ये । नमः॑ । नमः॑ । क॒कु॒भाय॑ । नि॒ष॒ङ्गिण॒ इति॑ नि-स॒ङ्गिने᳚ । स्ते॒नाना᳚म् । पत॑ये । नमः॑ । नमः॑ । नि॒ष॒ङ्गिण॒ इति॑ नि - स॒ङ्गिने᳚ । इ॒षु॒धि॒मत इती॑षुधि - मते᳚ । तस्क॑राणाम् । पत॑ये । नमः॑ । नमः॑ । वञ्च॑ते । प॒रि॒वञ्च॑त॒ इति॑ परि - वञ्च॑ते । स्ता॒यू॒नाम् । पत॑ये । नमः॑ । नमः॑ । नि॒चे॒रव॒ इति॑ नि - चे॒रवे᳚ । प॒रि॒च॒रायेति॑ परि - च॒राय॑ । अर॑ण्यानाम् । पत॑ये । नमः॑ । नमः॑ । सृ॒का॒विभ्य॒ इति॑ सृका॒वि - भ्यः॒ । जिघाꣳ॑सद्भ्य॒ इति॒ जिघाꣳ॑सत् - भ्यः॒ । मु॒ष्ण॒ताम् । पत॑ये । नमः॑ । नमः॑ । अ॒सि॒मद्भ्य॒ इत्य॑सि॒मत् - भ्यः॒ । नक्त᳚म् । चर॑द्भ्य॒ इति॒ चर॑त् - भ्यः॒ । प्र॒कृ॒न्ताना॒मिति॑ प्र - कृ॒न्ताना᳚म् । पत॑ये । नमः॑ । नमः॑ । उ॒ष्णी॒षिणे᳚ । गि॒रि॒च॒रायेति॑ गिरि - च॒राय॑ । कु॒लु॒ञ्चाना᳚म् । पत॑ये । नमः॑ । नमः॑ ।  \newline




\markright{ TS 4.5.3.2  \hfill https://www.vedavms.in \hfill}
\addcontentsline{toc}{section}{ TS 4.5.3.2 }
\section*{ TS 4.5.3.2 }

\textbf{TS 4.5.3.2 } \newline
\textbf{Samhita Paata} \newline

इषु॑मद्भ्यो धन्वा॒विभ्य॑श्च वो॒ नमो॒                  नम॑ आतन्वा॒नेभ्यः॑ प्रति॒दधा॑नेभ्यश्च वो॒ नमो॒                                     नम॑ आ॒यच्छ॑द्भ्यो विसृ॒जद्भ्य॑श्च वो॒ नमो॒   नमोऽस्य॑द्भ्यो॒ विद्ध्य॑द्भ्यश्च वो॒ नमो॒                नम॒ आसी॑नेभ्यः॒ शया॑नेभ्यश्च वो॒ नमो॒                                               नमः॑ स्व॒पद्भ्यो॒ जाग्र॑द्भ्यश्च वो॒ नमो॒                                              नम॒स्तिष्ठ॑द्भ्यो॒ धाव॑द्भ्यश्च वो॒ नमो॒     नमः॑ स॒भाभ्यः॑ स॒भाप॑तिभ्यश्च वो॒ नमो॒                                            नमो॒ अश्वे॒भ्यो ऽश्व॑पतिभ्य ( ) श्च वो॒ नमः॑ ॥ \newline

\textbf{Pada Paata} \newline

इषु॑मद्भ्य॒ इतीषु॑मत् - भ्यः॒ । ध॒न्वा॒विभ्य॒ इति॑ धन्वा॒वि-भ्यः॒ । च॒ । वः॒ । नमः॑ । नमः॑ । आ॒त॒न्वा॒नेभ्य॒ इत्या᳚ - त॒न्वा॒नेभ्यः॑ । प्र॒ति॒दधा॑नेभ्य॒ इति॑ प्रति - दधा॑नेभ्यः । च॒ । वः॒ । नमः॑ । नमः॑ । आ॒यच्छ॑द्भ्य॒ इत्या॒यच्छ॑त् - भ्यः॒ । वि॒सृ॒जद्भ्य॒ इति॑ विसृ॒जत् - भ्यः॒ । च॒ । वः॒ । नमः॑ । नमः॑ । अस्य॑द्भ्य॒ इत्यस्य॑त् - भ्यः॒ । विद्ध्य॑द्भ्य॒ इति॒ विद्ध्य॑त्-भ्यः॒ । च॒ । वः॒ । नमः॑ । नमः॑ । आसी॑नेभ्यः । शया॑नेभ्यः । च॒ । वः॒ । नमः॑ । नमः॑ । स्व॒पद्भ्य॒ इति॑ स्व॒पत् - भ्यः॒ । जाग्र॑द्भ्य॒ इति॒ जाग्र॑त्-भ्यः॒ । च॒ । वः॒ । नमः॑ । नमः॑ । तिष्ठ॑द्भ्य॒ इति॒ तिष्ठ॑त् - भ्यः॒ । धाव॑द्भ्य॒ इति॒ धाव॑त् - भ्यः॒ । च॒ । वः॒ । नमः॑ । नमः॑ । स॒भाभ्यः॑ । स॒भाप॑तिभ्य॒ इति॑ स॒भाप॑ति - भ्यः॒ । च॒ । वः॒ । नमः॑ । नमः॑ । अश्वे᳚भ्यः । अश्व॑पतिभ्य॒ इत्यश्व॑पति - भ्यः॒ ( ) । च॒ । वः॒ । नमः॑ ॥  \newline




\markright{ TS 4.5.4.1  \hfill https://www.vedavms.in \hfill}
\addcontentsline{toc}{section}{ TS 4.5.4.1 }
\section*{ TS 4.5.4.1 }

\textbf{TS 4.5.4.1 } \newline
\textbf{Samhita Paata} \newline

नम॑ आव्या॒धिनी᳚भ्यो वि॒विद्ध्य॑न्तीभ्यश्च वो॒ नमो॒                             नम॒ उग॑णाभ्य-स्तृꣳह॒तीभ्य॑श्च वो॒ नमो॒                                          नमो॑ गृ॒थ्सेभ्यो॑ गृ॒थ्सप॑तिभ्यश्च वो॒ नमो॒                                                नमो॒ व्राते᳚भ्यो॒ व्रात॑पतिभ्यश्च वो॒ नमो॒                                             नमो॑ ग॒णेभ्यो॑ ग॒णप॑तिभ्यश्च वो॒ नमो॒                                                      नमो॒ विरू॑पेभ्यो वि॒श्वरू॑पेभ्यश्च वो॒ नमो॒                                            नमो॑ म॒हद्भ्यः॑ क्षुल्ल॒केभ्य॑श्च वो॒ नमो॒                                           नमो॑ र॒थिभ्यो॑-ऽर॒थेभ्य॑श्च वो॒ नमो॒                                                       नमो॒ रथे᳚भ्यो॒ - [  ] \newline

\textbf{Pada Paata} \newline

नमः॑ । आ॒व्या॒धिनी᳚भ्य॒ इत्या᳚ - व्या॒धिनी᳚भ्यः । वि॒विद्ध्य॑न्तीभ्य॒ इति॑ वि - विद्ध्य॑न्तीभ्यः । च॒ । वः॒ । नमः॑ । नमः॑ । उग॑णाभ्यः । तृꣳ॒॒ह॒तीभ्यः॑ । च॒ । वः॒ । नमः॑ । नमः॑ । गृ॒थ्सेभ्यः॑ । गृ॒थ्सप॑तिभ्य॒ इति॑ गृ॒थ्सप॑ति - भ्यः॒ । च॒ । वः॒ । नमः॑ । नमः॑ । व्राते᳚भ्यः । व्रात॑पतिभ्य॒ इति॒ व्रात॑पति-भ्यः॒ । च॒ । वः॒ । नमः॑ । नमः॑ । ग॒णेभ्यः॑ । ग॒णप॑तिभ्य॒ इति॑ ग॒णप॑ति - भ्यः॒ । च॒ । वः॒ । नमः॑ । नमः॑ । विरू॑पेभ्य॒ इति॒ वि - रू॒पे॒भ्यः॒ । वि॒श्वरू॑पेभ्य॒ इति॑ वि॒श्व-रू॒पे॒भ्यः॒ । च॒ । वः॒ । नमः॑ । नमः॑ । म॒हद्भ्य॒ इति॑ म॒हत्-भ्यः॒ । क्षु॒ल्ल॒केभ्यः॑ । च॒ । वः॒ । नमः॑ । नमः॑ । र॒थिभ्य॒ इति॑ र॒थि-भ्यः॒ । अ॒र॒थेभ्यः॑ । च॒ । वः॒ । नमः॑ । नमः॑ । रथे᳚भ्यः ।  \newline




\markright{ TS 4.5.4.2  \hfill https://www.vedavms.in \hfill}
\addcontentsline{toc}{section}{ TS 4.5.4.2 }
\section*{ TS 4.5.4.2 }

\textbf{TS 4.5.4.2 } \newline
\textbf{Samhita Paata} \newline

रथ॑पतिभ्यश्च वो॒ नमो॒                                                      नमः॒ सेना᳚भ्यः सेना॒निभ्य॑श्च वो॒ नमो॒                                               नमः॑ क्ष॒त्तृभ्यः॑ संग्रही॒तृभ्य॑श्च वो॒ नमो॒                                     नम॒स्तक्ष॑भ्यो रथका॒रेभ्य॑श्च वो॒ नमो॒                                                    नमः॒ कुला॑लेभ्यः क॒र्मारे᳚भ्यश्च वो॒ नमो॒                                            नमः॑ पु॒ञ्जिष्टे᳚भ्यो निषा॒देभ्य॑श्च वो॒ नमो॒                                                नम॑ इषु॒कृद्भ्यो॑ धन्व॒कृद्भ्य॑श्च वो॒ नमो॒                                                नमो॑ मृग॒युभ्यः॑ श्व॒निभ्य॑श्च वो॒ नमो॒                                                     नमः᳡श्वभ्यः॒ श्वप॑तिभ्यश्च ( ) वो॒ नमः॑ ॥ \newline

\textbf{Pada Paata} \newline

रथ॑पतिभ्य॒ इति॒ रथ॑पति-भ्यः॒ । च॒ । वः॒ । नमः॑ । नमः॑ । सेना᳚भ्यः । से॒ना॒निभ्य॒ इति॑ सेना॒नि - भ्यः॒ । च॒ । वः॒ । नमः॑ । नमः॑ । क्ष॒त्तृभ्य॒ इति॑ क्ष॒त्तृ - भ्यः॒ । स॒ग्रं॒ही॒तृभ्य॒ इति॑ संग्रही॒तृ - भ्यः॒ । च॒ । वः॒ । नमः॑ । नमः॑ । तक्ष॑भ्य॒ इति॒ तक्ष॑ - भ्यः॒ । र॒थ॒का॒रेभ्य॒ इति॑ रथ-का॒रेभ्यः॑ । च॒ । वः॒ । नमः॑ । नमः॑ । कुला॑लेभ्यः । क॒र्मारे᳚भ्यः । च॒ । वः॒ । नमः॑ । नमः॑ । पु॒ञ्जिष्टे᳚भ्यः । नि॒षा॒देभ्यः॑ । च॒ । वः॒ । नमः॑ । नमः॑ । इ॒षु॒कृद्भ्य॒ इती॑षु॒कृत् - भ्यः॒ । ध॒न्व॒कृद्भ्य॒ इति॑ धन्व॒कृत् - भ्यः॒ । च॒ । वः॒ । नमः॑ । नमः॑ । मृ॒ग॒युभ्य॒ इति॑ मृग॒यु-भ्यः॒ । श्व॒निभ्य॒ इति॑ श्व॒नि - भ्यः॒ । च॒ । वः॒ । नमः॑ । नमः॑ । श्वभ्य॒ इति॒ श्व - भ्यः॒ । श्वप॑तिभ्य॒ इति॒ श्वप॑ति - भ्यः॒ । च॒ ( ) । वः॒ । नमः॑ ॥  \newline




\markright{ TS 4.5.5.1  \hfill https://www.vedavms.in \hfill}
\addcontentsline{toc}{section}{ TS 4.5.5.1 }
\section*{ TS 4.5.5.1 }

\textbf{TS 4.5.5.1 } \newline
\textbf{Samhita Paata} \newline

नमो॑ भ॒वाय॑ च रु॒द्राय॑ च॒ नमः॑ श॒र्वाय॑ च पशु॒पत॑ये च॒ नमो॒ नील॑ग्रीवाय च शिति॒कण्ठा॑य च॒ नमः॑ कप॒र्दिने॑ च॒ व्यु॑प्तकेशाय च॒ नमः॑ सहस्रा॒क्षाय॑ च श॒तध॑न्वने च॒ नमो॑ गिरि॒शाय॑ च शिपिवि॒ष्टाय॑ च॒ नमो॑ मी॒ढुष्ट॑माय॒ चेषु॑मते च॒ नमो᳚ ह्र॒स्वाय॑ च वाम॒नाय॑ च॒ नमो॑ बृह॒ते च॒ वर्.षी॑यसे च॒ नमो॑ वृ॒द्धाय॑ च सं॒ॅवृद्ध्व॑ने च॒-[  ] \newline

\textbf{Pada Paata} \newline

नमः॑ । भ॒वाय॑ । च॒ । रु॒द्राय॑ । च॒ । नमः॑ । श॒र्वाय॑ । च॒ । प॒शु॒पत॑य॒ इति॑ पशु-पत॑ये । च॒ । नमः॑ । नील॑ग्रीवा॒येति॒ नील॑ - ग्री॒वा॒य॒ । च॒ । शि॒ति॒कण्ठा॒येति॑ शिति - कण्ठा॑य । च॒ । नमः॑ । क॒प॒र्दिने᳚ । च॒ । व्यु॑प्तकेशा॒येति॒ व्यु॑प्त - के॒शा॒य॒ । च॒ । नमः॑ । स॒ह॒स्रा॒क्षायेति॑ सहस्र-अ॒क्षाय॑ । च॒ । श॒तध॑न्वन॒ इति॑ श॒त - ध॒न्व॒ने॒ । च॒ । नमः॑ । गि॒रि॒शाय॑ । च॒ । शि॒पि॒वि॒ष्टायेति॑ शिपि - वि॒ष्टाय॑ । च॒ । नमः॑ । मी॒ढुष्ट॑मा॒येति॑ मी॒ढुः - त॒मा॒य॒ । च॒ । इषु॑मत॒ इतीषु॑ - म॒ते॒ । च॒ । नमः॑ । ह्र॒स्वाय॑ । च॒ । वा॒म॒नाय॑ । च॒ । नमः॑ । बृ॒ह॒ते । च॒ । वर्.षी॑यसे । च॒ । नमः॑ । वृ॒द्धाय॑ । च॒ । सं॒ॅवृद्ध्व॑न॒ इति॑ सं - वृद्ध्व॑ने । च॒ ।  \newline




\markright{ TS 4.5.5.2  \hfill https://www.vedavms.in \hfill}
\addcontentsline{toc}{section}{ TS 4.5.5.2 }
\section*{ TS 4.5.5.2 }

\textbf{TS 4.5.5.2 } \newline
\textbf{Samhita Paata} \newline

नमो॒ अग्रि॑याय च प्रथ॒माय॑ च॒ नम॑ आ॒शवे॑ चाजि॒राय॑ च॒ नमः᳡शीघ्रि॑याय च॒ शीभ्या॑य च॒ नम॑ ऊ॒र्म्या॑य चावस्व॒न्या॑य च॒ नमः॑ स्रोत॒स्या॑य च॒ द्वीप्या॑य च ॥ \newline

\textbf{Pada Paata} \newline

नमः॑ । अग्रि॑याय । च॒ । प्र॒थ॒माय॑ । च॒ । नमः॑ । आ॒शवे᳚ । च॒ । अ॒जि॒राय॑ । च॒ । नमः॑ । शीघ्रि॑याय । च॒ । शीभ्या॑य । च॒ । नमः॑ । ऊ॒र्म्या॑य । च॒ । अ॒व॒स्व॒न्या॑येत्य॑व - स्व॒न्या॑य । च॒ । नमः॑ । स्रो॒त॒स्या॑य । च॒ । द्वीप्या॑य । च॒ ॥  \newline




\markright{ TS 4.5.6.1  \hfill https://www.vedavms.in \hfill}
\addcontentsline{toc}{section}{ TS 4.5.6.1 }
\section*{ TS 4.5.6.1 }

\textbf{TS 4.5.6.1 } \newline
\textbf{Samhita Paata} \newline

नमो᳚ ज्ये॒ष्ठाय॑ च कनि॒ष्ठाय॑ च॒ नमः॑ पूर्व॒जाय॑ चापर॒जाय॑ च॒ नमो॑ मद्ध्य॒माय॑ चापग॒ल्भाय॑ च॒ नमो॑ जघ॒न्या॑य च॒ बुद्ध्नि॑याय च॒ नमः॑ सो॒भ्या॑य च प्रतिस॒र्या॑य च॒ नमो॒ याम्या॑य च॒ क्षेम्या॑य च॒ नम॑ उर्व॒र्या॑य च॒ खल्या॑य च॒ नमः॒ श्लोक्या॑य चावसा॒न्या॑य च॒ नमो॒ वन्या॑य च॒ कक्ष्या॑य च॒ नमः॑ श्र॒वाय॑ च प्रतिश्र॒वाय॑ च॒-[  ] \newline

\textbf{Pada Paata} \newline

नमः॑ । ज्ये॒ष्ठाय॑ । च॒ । क॒नि॒ष्ठाय॑ । च॒ । नमः॑ । पू॒र्व॒जायेति॑ पूर्व - जाय॑ । च॒ । अ॒प॒र॒जायेत्य॑पर-जाय॑ । च॒ । नमः॑ । म॒द्ध्य॒माय॑ । च॒ । अ॒प॒ग॒ल्भायेत्य॑प - ग॒ल्भाय॑ । च॒ । नमः॑ । ज॒घ॒न्या॑य । च॒ । बुद्ध्नि॑याय । च॒ । नमः॑ । सो॒भ्या॑य । च॒ । प्र॒ति॒स॒र्या॑येति॑ प्रति - स॒र्या॑य । च॒ । नमः॑ । याम्या॑य । च॒ । क्षेम्या॑य । च॒ । नमः॑ । उ॒र्व॒र्या॑य । च॒ । खल्या॑य । च॒ । नमः॑ । श्लोक्या॑य । च॒ । अ॒व॒सा॒न्या॑येत्य॑व - सा॒न्या॑य । च॒ । नमः॑ । वन्या॑य । च॒ । कक्ष्या॑य । च॒ । नमः॑ । श्र॒वाय॑ । च॒ । प्र॒ति॒श्र॒वायेति॑ प्रति-श्र॒वाय॑ । च॒ ।  \newline




\markright{ TS 4.5.6.2  \hfill https://www.vedavms.in \hfill}
\addcontentsline{toc}{section}{ TS 4.5.6.2 }
\section*{ TS 4.5.6.2 }

\textbf{TS 4.5.6.2 } \newline
\textbf{Samhita Paata} \newline

नम॑ आ॒शुषे॑णाय चा॒शुर॑थाय च॒ नमः॒ शूरा॑य चावभिन्द॒ते च॒ नमो॑ व॒र्मिणे॑ च वरू॒थिने॑ च॒ नमो॑ बि॒ल्मिने॑ च कव॒चिने॑ च॒ नमः॑ श्रु॒ताय॑ च श्रुतसे॒नाय॑ च ॥ \newline

\textbf{Pada Paata} \newline

नमः॑ । आ॒शुषे॑णा॒येत्या॒शु - से॒ना॒य॒ । च॒ । आ॒शुर॑था॒येत्या॒शु-र॒था॒य॒ । च॒ । नमः॑ । शूरा॑य । च॒ । अ॒व॒भि॒न्द॒त इत्य॑व-भि॒न्द॒ते । च॒ । नमः॑ । व॒र्मिणे᳚ । च॒ । व॒रू॒थिने᳚ । च॒ । नमः॑ । बि॒ल्मिने᳚ । च॒ । क॒व॒चिने᳚ । च॒ । नमः॑ । श्रु॒ताय॑ । च॒ । श्रु॒त॒से॒नायेति॑ श्रुत-से॒नाय॑ । च॒ ॥  \newline




\markright{ TS 4.5.7.1  \hfill https://www.vedavms.in \hfill}
\addcontentsline{toc}{section}{ TS 4.5.7.1 }
\section*{ TS 4.5.7.1 }

\textbf{TS 4.5.7.1 } \newline
\textbf{Samhita Paata} \newline

नमो॑ दुन्दु॒भ्या॑य चाहन॒न्या॑य च॒ नमो॑ धृ॒ष्णवे॑ च प्रमृ॒शाय॑ च॒ नमो॑ दू॒ताय॑ च॒ प्रहि॑ताय च॒ नमो॑ निष॒ङ्गिणे॑ चेषुधि॒मते॑ च॒ नम॑ स्ती॒क्ष्णेष॑वे चायु॒धिने॑ च॒ नमः॑ स्वायु॒धाय॑ च सु॒धन्व॑ने च॒ नमः॒ स्रुत्या॑य च॒ पथ्या॑य च॒ नमः॑ का॒ट्या॑य च नी॒प्या॑य च॒ नमः॒ सूद्या॑य च सर॒स्या॑य च॒ नमो॑ ना॒द्याय॑ च वैश॒न्ताय॑ च॒ -[  ] \newline

\textbf{Pada Paata} \newline

नमः॑ । दु॒न्दु॒भ्या॑य । च॒ । आ॒ह॒न॒न्या॑येत्या᳚ - ह॒न॒न्या॑य । च॒ । नमः॑ । धृ॒ष्णवे᳚ । च॒ । प्र॒मृ॒शायेति॑ प्र - मृ॒शाय॑ । च॒ । नमः॑ । दू॒ताय॑ । च॒ । प्रहि॑ता॒येति॒ प्र - हि॒ता॒य॒ । च॒ । नमः॑ । नि॒ष॒ङ्गिण॒ इति॑ नि - स॒ङ्गिने᳚ । च॒ । इ॒षु॒धि॒मत॒ इती॑षुधि - मते᳚ । च॒ । नमः॑ । ती॒क्ष्णेष॑व॒ इति॑ ती॒क्ष्ण - इ॒ष॒वे॒ । च॒ । आ॒यु॒धिने᳚ । च॒ । नमः॑ । स्वा॒यु॒धायेति॑ सु - आ॒यु॒धाय॑ । च॒ । सु॒धन्व॑न॒ इति॑ सु - धन्व॑ने । च॒ । नमः॑ । स्रुत्या॑य । च॒ । पथ्या॑य । च॒ । नमः॑ । का॒ट्या॑य । च॒ । नी॒प्या॑य । च॒ । नमः॑ । सूद्या॑य । च॒ । स॒र॒स्या॑य । च॒ । नमः॑ । ना॒द्याय॑ । च॒ । वै॒श॒न्ताय॑ । च॒ ।  \newline




\markright{ TS 4.5.7.2  \hfill https://www.vedavms.in \hfill}
\addcontentsline{toc}{section}{ TS 4.5.7.2 }
\section*{ TS 4.5.7.2 }

\textbf{TS 4.5.7.2 } \newline
\textbf{Samhita Paata} \newline

नमः᳡कूप्या॑य चाव॒ट्या॑य च॒ नमो॒ वर्ष्या॑य चाव॒र्ष्याय॑ च॒ नमो॑ मे॒घ्या॑य च विद्यु॒त्या॑य च॒ नम॑ ई॒द्ध्रिया॑य चात॒प्या॑य च॒ नमो॒ वात्या॑य च॒ रेष्मि॑याय च॒ नमो॑ वास्त॒व्या॑य च वास्तु॒पाय॑ च ॥ \newline

\textbf{Pada Paata} \newline

नमः॑ । कूप्या॑य । च॒ । अ॒व॒ट्या॑य । च॒ । नमः॑ । वर्ष्या॑य । च॒ । अ॒व॒र्ष्याय॑ । च॒ । नमः॑ । मे॒घ्या॑य । च॒ । वि॒द्यु॒त्या॑येति॑ वि - द्यु॒त्या॑य । च॒ । नमः॑ । ई॒द्ध्रिया॑य । च॒ । आ॒त॒प्या॑येत्या᳚ - त॒प्या॑य । च॒ । नमः॑ । वात्या॑य । च॒ । रेष्मि॑याय । च॒ । नमः॑ । वा॒स्त॒व्या॑य । च॒ । वा॒स्तु॒पायेति॑ वास्तु - पाय॑ । च॒ ॥  \newline




\markright{ TS 4.5.8.1  \hfill https://www.vedavms.in \hfill}
\addcontentsline{toc}{section}{ TS 4.5.8.1 }
\section*{ TS 4.5.8.1 }

\textbf{TS 4.5.8.1 } \newline
\textbf{Samhita Paata} \newline

नमः॒ सोमा॑य च रु॒द्राय॑ च॒ नम॑स्ता॒म्राय॑ चारु॒णाय॑ च॒ नमः॑ श॒ङ्गाय॑ च पशु॒पत॑ये च॒ नम॑ उ॒ग्राय॑ च भी॒माय॑ च॒ नमो॑ अग्रेव॒धाय॑ च दूरेव॒धाय॑ च॒ नमो॑ ह॒न्त्रे च॒ हनी॑यसे च॒ नमो॑ वृ॒क्षेभ्यो॒ हरि॑केशेभ्यो॒ नम॑स्ता॒राय॒ नमः॑ श॒भंवे॑ च मयो॒भवे॑ च॒ नमः॑ शङ्क॒राय॑ च मयस्क॒राय॑ च॒ नमः॑ शि॒वाय॑ च शि॒वत॑राय च॒ - [  ] \newline

\textbf{Pada Paata} \newline

नमः॑ । सोमा॑य । च॒ । रु॒द्राय॑ । च॒ । नमः॑ । ता॒म्राय॑ । च॒ । अ॒रु॒णाय॑ । च॒ । नमः॑ । श॒ङ्गाय॑ । च॒ । प॒शु॒पत॑य॒ इति॑ पशु-पत॑ये । च॒ । नमः॑ । उ॒ग्राय॑ । च॒ । भी॒माय॑ । च॒ । नमः॑ । अ॒ग्रे॒व॒धायेत्य॑ग्रे - व॒धाय॑ । च॒ । दू॒रे॒व॒धायेति॑ दूरे - व॒धाय॑ । च॒ । नमः॑ । ह॒न्त्रे । च॒ । हनी॑यसे । च॒ । नमः॑ । वृ॒क्षेभ्यः॑ । हरि॑केशेभ्य॒ इति॒ हरि॑ - के॒शे॒भ्यः॒ । नमः॑ । ता॒राय॑ । नमः॑ । श॒भंव॒ इति॑ शं - भवे᳚ । च॒ । म॒यो॒भव॒ इति॑ मयः - भवे᳚ । च॒ । नमः॑ । श॒ङ्क॒रायेति॑ शं - क॒राय॑ । च॒ । म॒य॒स्क॒रायेति॑ मयः - क॒राय॑ । च॒ । नमः॑ । शि॒वाय॑ । च॒ । शि॒वत॑रा॒येति॑ शि॒व - त॒रा॒य॒ । च॒ ।  \newline




\markright{ TS 4.5.8.2  \hfill https://www.vedavms.in \hfill}
\addcontentsline{toc}{section}{ TS 4.5.8.2 }
\section*{ TS 4.5.8.2 }

\textbf{TS 4.5.8.2 } \newline
\textbf{Samhita Paata} \newline

नम॒स्तीर्थ्या॑य च॒ कूल्या॑य च॒ नमः॑ पा॒र्या॑य चावा॒र्या॑य च॒ नमः॑ प्र॒तर॑णाय चो॒त्तर॑णाय च॒ नम॑ आता॒र्या॑य चाला॒द्या॑य च॒ नमः॒ शष्प्या॑य च॒ फेन्या॑य च॒ नमः॑ सिक॒त्या॑य च प्रवा॒ह्या॑य च ॥ \newline

\textbf{Pada Paata} \newline

नमः॑ । तीर्थ्या॑य । च॒ । कूल्या॑य । च॒ । नमः॑ । पा॒र्या॑य । च॒ । अ॒वा॒र्या॑य । च॒ । नमः॑ । प्र॒तर॑णा॒येति॑ प्र - तर॑णाय । च॒ । उ॒त्तर॑णा॒येत्यु॑त् - तर॑णाय । च॒ । नमः॑ । आ॒ता॒र्या॑येत्या᳚-ता॒र्या॑य । च॒ । आ॒ला॒द्या॑येत्या᳚ - ला॒द्या॑य। च॒ । नमः॑ । शष्प्या॑य । च॒ । फेन्या॑य । च॒ । नमः॑ । सि॒क॒त्या॑य । च॒ । प्र॒वा॒ह्या॑येति॑ प्र - वा॒ह्या॑य । च॒ ।  \newline




\markright{ TS 4.5.9.1  \hfill https://www.vedavms.in \hfill}
\addcontentsline{toc}{section}{ TS 4.5.9.1 }
\section*{ TS 4.5.9.1 }

\textbf{TS 4.5.9.1 } \newline
\textbf{Samhita Paata} \newline

नम॑ इरि॒ण्या॑य च प्रप॒थ्या॑य च॒ नमः॑ किꣳशि॒लाय॑ च॒ क्षय॑णाय च॒ नमः॑ कप॒र्दिने॑ च पुल॒स्तये॑ च॒ नमो॒ गोष्ठ्या॑य च॒ गृह्या॑य च॒ नम॒स्तल्प्या॑य च॒ गेह्या॑य च॒ नमः॑ का॒ट्या॑य च गह्वरे॒ष्ठाय॑ च॒ नमो᳚ ह्रद॒य्या॑य च निवे॒ष्प्या॑य च॒ नमः॑ पाꣳस॒व्या॑य च रज॒स्या॑य च॒ नमः॒ शुष्क्या॑य च हरि॒त्या॑य च॒ नमो॒ लोप्या॑य चोल॒प्या॑य च॒- [  ] \newline

\textbf{Pada Paata} \newline

नमः॑ । इ॒रि॒ण्या॑य । च॒ । प्र॒प॒थ्या॑येति॑ प्र - प॒थ्या॑य । च॒ । नमः॑ । किꣳ॒॒शि॒लाय॑ । च॒ । क्षय॑णाय । च॒ । नमः॑ । क॒प॒र्दिने᳚ । च॒ । पु॒ल॒स्तये᳚ । च॒ । नमः॑ । गोष्ठ्या॒येति॒ गो - स्थ्या॒य॒ । च॒ । गृह्या॑य । च॒ । नमः॑ । तल्प्या॑य । च॒ । गेह्या॑य । च॒ । नमः॑ । का॒ट्या॑य । च॒ । ग॒ह्व॒रे॒ष्ठायेति॑ गह्वरे - स्थाय॑ । च॒ । नमः॑ । ह्र॒द॒य्या॑य । च॒ । नि॒वे॒ष्प्या॑येति॑ नि - वे॒ष्प्या॑य । च॒ । नमः॑ । पाꣳ॒॒स॒व्या॑य । च॒ । र॒ज॒स्या॑य । च॒ । नमः॑ । शुष्क्या॑य । च॒ । ह॒रि॒त्या॑य । च॒ । नमः॑ । लोप्या॑य । च॒ । उ॒ल॒प्या॑य । च॒ ।  \newline




\markright{ TS 4.5.9.2  \hfill https://www.vedavms.in \hfill}
\addcontentsline{toc}{section}{ TS 4.5.9.2 }
\section*{ TS 4.5.9.2 }

\textbf{TS 4.5.9.2 } \newline
\textbf{Samhita Paata} \newline

नम॑ ऊ॒र्व्या॑य च सू॒र्म्या॑य च॒ नमः॑ प॒र्ण्या॑य च पर्णश॒द्या॑य च॒ नमो॑ऽपगु॒रमा॑णाय चाभिघ्न॒ते च॒ नम॑ आक्खिद॒ते च॑ प्रक्खिद॒ते च॒ नमो॑ वः किरि॒केभ्यो॑ दे॒वानाꣳ॒॒ हृद॑येभ्यो॒ नमो॑ विक्षीण॒केभ्यो॒ नमो॑ विचिन्व॒त्केभ्यो॒ नम॑ आनिर्. ह॒तेभ्यो॒ नम॑ आमीव॒त्केभ्यः॑ ॥ \newline

\textbf{Pada Paata} \newline

नमः॑ । ऊ॒र्व्या॑य । च॒ । सू॒र्म्या॑य । च॒ । नमः॑ । प॒र्ण्या॑य । च॒ । प॒र्ण॒श॒द्या॑येति॑ पर्ण - श॒द्या॑य। च॒ । नमः॑ । अ॒प॒गु॒रमा॑णा॒येत्य॑प - गु॒रमा॑णाय । च॒ । अ॒भि॒घ्न॒त इत्य॑भि - घ्न॒ते । च॒ । नमः॑ । आ॒क्खि॒द॒त इत्या᳚ - खि॒द॒ते । च॒ । प्र॒क्खि॒द॒त इति॑ प्र-खि॒द॒ते । च॒ । नमः॑ । वः॒ । कि॒रि॒केभ्यः॑ । दे॒वाना᳚म् । हृद॑येभ्यः । नमः॑ । वि॒क्षी॒ण॒केभ्य॒ इति॑ वि - क्षी॒ण॒केभ्यः॑ । नमः॑ । वि॒चि॒न्व॒त्केभ्य॒ इति॑ वि - चि॒न्व॒त्केभ्यः॑ । नमः॑ । आ॒नि॒र्॒.ह॒तेभ्य॒ इत्या॑निः-ह॒तेभ्यः॑ । नमः॑ । आ॒मी॒व॒त्केभ्य॒ इत्या᳚ - मी॒व॒त्केभ्यः॑ ॥  \newline




\markright{ TS 4.5.10.1  \hfill https://www.vedavms.in \hfill}
\addcontentsline{toc}{section}{ TS 4.5.10.1 }
\section*{ TS 4.5.10.1 }

\textbf{TS 4.5.10.1 } \newline
\textbf{Samhita Paata} \newline

द्रापे॒ अन्ध॑सस्पते॒ दरि॑द्र॒न्नील॑लोहित । ए॒षां पुरु॑षाणामे॒षां प॑शू॒नां मा भे र्माऽरो॒ मो ए॑षां॒ किञ्च॒नाम॑मत् ॥ या ते॑ रुद्र शि॒वा त॒नूः शि॒वा वि॒श्वाह॑भेषजी । शि॒वा रु॒द्रस्य॑ भेष॒जी तया॑ नो मृड जी॒वसे᳚ ॥                                    इ॒माꣳ रु॒द्राय॑ त॒वसे॑ कप॒र्दिने᳚ क्ष॒यद्वी॑राय॒ प्रभ॑रामहे म॒तिं । यथा॑ नः॒ शमस॑द् द्वि॒पदे॒ चतु॑ष्पदे॒ विश्वं॑ पु॒ष्टं ग्रामे॑ अ॒स्मि - [  ] \newline

\textbf{Pada Paata} \newline

द्रापे᳚ । अन्ध॑सः । प॒ते॒ । दरि॑द्रत् । नील॑लोहि॒तेति॒ नील॑ - लो॒हि॒त॒ ॥ ए॒षाम् । पुरु॑षाणाम् । ए॒षाम् । प॒शू॒नाम् । मा । भेः । मा । अ॒रः॒ । मो इति॑ । ए॒षा॒म् । किम् । च॒न । आ॒म॒म॒त् ॥ या । ते॒ । रु॒द्र॒ । शि॒वा । त॒नूः । शि॒वा । वि॒श्वाह॑भेष॒जीति॑ वि॒श्वाह॑-भे॒ष॒जी॒ ॥ शि॒वा । रु॒द्रस्य॑ । भे॒ष॒जी । तया᳚ । नः॒ । मृ॒ड॒ । जी॒वसे᳚ ॥ इ॒माम् । रु॒द्राय॑ । त॒वसे᳚ । क॒प॒र्दिने᳚ । क्ष॒यद्वी॑रा॒येति॑ क्ष॒यत् - वी॒रा॒य॒ । प्रेति॑ । भ॒रा॒म॒हे॒ । म॒तिम् ॥ यथा᳚ । नः॒ । शम् । अस॑त् । द्वि॒पद॒ इति॑ द्वि - पदे᳚ । चतु॑ष्पद॒ इति॒ चतुः॑ - प॒दे॒ । विश्व᳚म् । पु॒ष्टम् । ग्रामे᳚ । अ॒स्मिन्न् ।  \newline




\markright{ TS 4.5.10.2  \hfill https://www.vedavms.in \hfill}
\addcontentsline{toc}{section}{ TS 4.5.10.2 }
\section*{ TS 4.5.10.2 }

\textbf{TS 4.5.10.2 } \newline
\textbf{Samhita Paata} \newline

न्नना॑तुरं ॥ मृ॒डा नो॑ रुद्रो॒ तनो॒ मय॑स्कृधि क्ष॒यद्वी॑राय॒ नम॑सा विधेम ते । यच्छं च॒ योश्च॒ मनु॑राय॒जे पि॒ता तद॑श्याम॒ तव॑ रुद्र॒ प्रणी॑तौ ॥ मा नो॑ म॒हान्त॑मु॒त मा नो॑ अर्भ॒कं मा न॒ उक्ष॑न्तमु॒त मा न॑ उक्षि॒तं । मा नो॑ वधीः पि॒तरं॒ मोत मा॒तरं॑ प्रि॒या मा न॑स्त॒नुवो॑ - [  ] \newline

\textbf{Pada Paata} \newline

अना॑तुर॒मित्यना᳚ - तु॒र॒म् ॥ मृ॒ड । नः॒ । रु॒द्र॒ । उ॒त । नः॒ । मयः॑ । कृ॒धि॒ । क्ष॒यद्वी॑रा॒येति॑ क्ष॒यत् - वी॒रा॒य॒ । नम॑सा । वि॒धे॒म॒ । ते॒ ॥ यत् । शम् । च॒ । योः । च॒ । मनुः॑ । आ॒य॒ज इत्या᳚ - य॒जे । पि॒ता । तत् । अ॒श्या॒म॒ । तव॑ । रु॒द्र॒ । प्रणी॑ता॒विति॒ प्र - नी॒तौ॒ ॥ मा । नः॒ । म॒हान्त᳚म् । उ॒त । मा । नः॒ । अ॒र्भ॒कम् । मा । नः॒ । उक्ष॑न्तम् । उ॒त । मा । नः॒ । उ॒क्षि॒तम् ॥ मा । नः॒ । व॒धीः॒ । पि॒तर᳚म् । मा । उ॒त । मा॒तर᳚म् । प्रि॒याः । मा । नः॒ । त॒नुवः॑ ।  \newline




\markright{ TS 4.5.10.3  \hfill https://www.vedavms.in \hfill}
\addcontentsline{toc}{section}{ TS 4.5.10.3 }
\section*{ TS 4.5.10.3 }

\textbf{TS 4.5.10.3 } \newline
\textbf{Samhita Paata} \newline

रुद्र रीरिषः ॥ मा न॑स्तो॒के तन॑ये॒ मा न॒ आयु॑षि॒ मा नो॒ गोषु॒ मा नो॒ अश्वे॑षु रीरिषः । वी॒रान् मानो॑ रुद्र भामि॒तो व॑धीर्. ह॒विष्म॑न्तो॒ नम॑सा विधेम ते ॥ आ॒रात्ते॑ गो॒घ्न उ॒त पू॑रुष॒घ्ने क्ष॒यद्वी॑राय सु॒म्नम॒स्मे ते॑ अस्तु । रक्षा॑ च नो॒ अधि॑ च देव ब्रू॒ह्यधा॑ च नः॒ शर्म॑ यच्छ द्वि॒बर्.हाः᳚ ॥ स्तु॒हि - [  ] \newline

\textbf{Pada Paata} \newline

रु॒द्र॒ । री॒रि॒षः॒ ॥ मा । नः॒ । तो॒के । तन॑ये । मा । नः॒ । आयु॑षि । मा । नः॒ । गोषु॑ । मा । नः॒ । अश्वे॑षु । री॒रि॒षः॒ ॥ वी॒रान् । मा । नः॒ । रु॒द्र॒ । भा॒मि॒तः । व॒धीः॒ । ह॒विष्म॑न्तः । नम॑सा । वि॒धे॒म॒ । ते॒ ॥ आ॒रात् । ते॒ । गो॒घ्न इति॑ गो - घ्ने । उ॒त । पू॒रु॒ष॒घ्न इति॑ पूरुष - घ्ने । क्ष॒यद्वी॑रा॒येति॑ क्ष॒यत् - वी॒रा॒य॒ । सु॒म्नम् । अ॒स्मे इति॑ । ते॒ । अ॒स्तु॒ ॥ रक्ष॑ । च॒ । नः॒ । अधीति॑ । च॒ । दे॒व॒ । ब्रू॒हि॒ । अध॑ । च॒ । नः॒ । शर्म॑ । य॒च्छ॒ । द्वि॒बर्.हा॒ इति॑ द्वि - बर्.हाः᳚ ॥ स्तु॒हि ।  \newline




\markright{ TS 4.5.10.4  \hfill https://www.vedavms.in \hfill}
\addcontentsline{toc}{section}{ TS 4.5.10.4 }
\section*{ TS 4.5.10.4 }

\textbf{TS 4.5.10.4 } \newline
\textbf{Samhita Paata} \newline

श्रु॒तं ग॑र्त्त॒सदं॒ ॅयुवा॑नं मृ॒गं न भी॒म-मु॑पह॒त्नु-मु॒ग्रं । मृ॒डा ज॑रि॒त्रे रु॑द्र॒ स्तवा॑नो अ॒न्यन्ते॑ अ॒स्मन्निव॑पन्तु॒ सेनाः᳚ ॥ परि॑णो रु॒द्रस्य॑ हे॒ति र्वृ॑णक्तु॒ परि॑त्वे॒षस्य॑ दुर्म॒तिर॑घा॒योः । अव॑ स्थि॒रा म॒घव॑द्भ्य-स्तनुष्व॒ मीढ्व॑स्तो॒काय॒ तन॑याय मृडय ॥ मीढु॑ष्टम॒ शिव॑तम शि॒वो नः॑ सु॒मना॑ भव ।प॒र॒मे वृ॒क्ष आयु॑धं नि॒धाय॒ कृत्तिं॒ ॅवसा॑न॒ आच॑र॒ पिना॑कं॒ - [  ] \newline

\textbf{Pada Paata} \newline

श्रु॒तम् । ग॒र्त्त॒सद॒मिति॑ गर्त्त-सद᳚म् । युवा॑नम् । मृ॒गम् । न । भी॒मम् । उ॒प॒ह॒त्नुम् । उ॒ग्रम् ॥ मृ॒ड । ज॒रि॒त्रे । रु॒द्र॒ । स्तवा॑नः । अ॒न्यम् । ते॒ । अ॒स्मत् । नीति॑ । व॒प॒न्तु॒ । सेनाः᳚ ॥ परीति॑ । नः॒ । रु॒द्रस्य॑ । हे॒तिः । वृ॒ण॒क्तु॒ । परीति॑ । त्वे॒षस्य॑ । दु॒र्म॒तिरिति॑ दुः - म॒तिः । अ॒घा॒योरित्य॑घ - योः ॥ अवेति॑ । स्थि॒रा । म॒घव॑द्भ्य॒ इति॑ म॒घव॑त् - भ्यः॒ । त॒नु॒ष्व॒ । मीढ्वः॑ । तो॒काय॑ । तन॑याय । मृ॒ड॒य॒ ॥ मीढु॑ष्ट॒मेति॒ मीढुः॑ - त॒म॒ । शिव॑त॒मेति॒ शिव॑ - त॒म॒ । शि॒वः । नः॒ । सु॒मना॒ इति॑ सु - मनाः᳚ । भ॒व॒ ॥ प॒र॒मे । वृ॒क्षे । आयु॑धम् । नि॒धायेति॑ नि - धाय॑ । कृत्ति᳚म् । वसा॑नः । एति॑ । च॒र॒ । पिना॑कम् ।  \newline




\markright{ TS 4.5.10.5  \hfill https://www.vedavms.in \hfill}
\addcontentsline{toc}{section}{ TS 4.5.10.5 }
\section*{ TS 4.5.10.5 }

\textbf{TS 4.5.10.5 } \newline
\textbf{Samhita Paata} \newline

बिभ्र॒दाग॑हि ॥ विकि॑रिद॒ विलो॑हित॒ नम॑स्ते अस्तु भगवः । यास्ते॑ स॒हस्रꣳ॑ हे॒तयो॒ऽन्य-म॒स्मन्नि व॑पन्तु॒ ताः ॥                            स॒हस्रा॑णि सहस्र॒धा बा॑हु॒वोस्तव॑ हे॒तयः॑ । तासा॒मीशा॑नो भगवः परा॒चीना॒ मुखा॑ कृधि ॥ \newline

\textbf{Pada Paata} \newline

बिभ्र॑त् । एति॑ । ग॒हि॒ ॥ विकि॑रि॒देति॒ वि - कि॒रि॒द॒ । विलो॑हि॒तेति॒ वि - लो॒हि॒त॒ । नमः॑ । ते॒ । अ॒स्तु॒ । भ॒ग॒व॒ इति॑ भग - वः॒ ॥ याः । ते॒ । स॒हस्र᳚म् । हे॒तयः॑ । अ॒न्यम् । अ॒स्मत् । नीति॑ । व॒प॒न्तु॒ । ताः ॥ स॒हस्रा॑णि । स॒ह॒स्र॒धेति॑ सहस्र-धा । बा॒हु॒वोः । तव॑ । हे॒तयः॑ ॥ तासा᳚म् । ईशा॑नः । भ॒ग॒व॒ इति॑ भग - वः॒ । प॒रा॒चीना᳚ । मुखा᳚ । कृ॒धि॒ ॥  \newline




\markright{ TS 4.5.11.1  \hfill https://www.vedavms.in \hfill}
\addcontentsline{toc}{section}{ TS 4.5.11.1 }
\section*{ TS 4.5.11.1 }

\textbf{TS 4.5.11.1 } \newline
\textbf{Samhita Paata} \newline

स॒हस्रा॑णि सहस्र॒शो ये रु॒द्रा अधि॒ भूम्यां᳚ । तेषाꣳ॑ सहस्रयोज॒ने ऽव॒धन्वा॑नि तन्मसि ॥                                      अ॒स्मिन्-म॑ह॒त्य॑र्ण॒वे᳚-ऽन्तरि॑क्षे भ॒वा अधि॑ ॥                                     नील॑ग्रीवाः शिति॒कण्ठाः᳚ श॒र्वा अ॒धः क्ष॑माच॒राः ॥                                नील॑ग्रीवाः शिति॒कण्ठा॒ दिवꣳ॑ रु॒द्रा उप॑श्रिताः ॥                         ये वृ॒क्षेषु॑ स॒स्पिञ्ज॑रा॒ नील॑ग्रीवा॒ विलो॑हिताः ॥                                        ये भू॒ताना॒-मधि॑पतयो विशि॒खासः॑ कप॒र्दि॑नः ॥                                    ये अन्ने॑षु वि॒विद्ध्य॑न्ति॒ पात्रे॑षु॒ पिब॑तो॒ जनान्॑ ॥                                       ये प॒थां प॑थि॒रक्ष॑य ऐलबृ॒दा य॒व्युधः॑ ॥ ये ती॒र्थानि॑ - [  ] \newline

\textbf{Pada Paata} \newline

स॒हस्रा॑णि । स॒ह॒स्र॒श इति॑ सहस्र - शः । ये । रु॒द्राः । अधीति॑ । भूम्या᳚म् ॥ तेषा᳚म् । स॒ह॒स्र॒यो॒ज॒न इति॑ सहस्र - यो॒ज॒ने । अवेति॑ । धन्वा॑नि । त॒न्म॒सि॒ ॥ अ॒स्मिन्न् । म॒ह॒ति । अ॒र्ण॒वे । अ॒न्तरि॑क्षे । भ॒वाः । अधि॑ ॥ नील॑ग्रीवा॒ इति॒ नील॑ - ग्री॒वाः॒ । शि॒ति॒कण्ठा॒ इति॑ शिति - कण्ठाः᳚ । श॒र्वाः । अ॒धः । क्ष॒मा॒च॒राः ॥ नील॑ग्रीवा॒ इति॒ नील॑-ग्री॒वाः॒ । शि॒ति॒कण्ठा॒ इति॑ शिति - कण्ठाः᳚ । दिव᳚म् । रु॒द्राः । उप॑श्रिता॒ इत्युप॑ - श्रि॒ताः॒ ॥ ये । वृ॒क्षेषु॑ । स॒स्पिञ्ज॑राः । नील॑ग्रीवा॒ इति॒ नील॑ - ग्री॒वाः॒ । विलो॑हिता॒ इति॒ वि - लो॒हि॒ताः॒ ॥ ये । भू॒ताना᳚म् । अधि॑पतय॒ इत्यधि॑ - प॒त॒यः॒ । वि॒शि॒खास॒ इति॑ वि -शि॒खासः॑ । क॒प॒र्दिनः॑ ॥ ये । अन्ने॑षु । वि॒विद्ध्य॒न्तीति॑ वि-विद्ध्य॑न्ति । पात्रे॑षु । पिब॑तः । जनान्॑ ॥ ये । प॒थाम् । प॒थि॒रक्ष॑य॒ इति॑ पथि - रक्ष॑यः । ऐ॒ल॒बृ॒दाः । य॒व्युधः॑ ॥ ये । ती॒र्थानि॑ ।  \newline




\markright{ TS 4.5.11.2  \hfill https://www.vedavms.in \hfill}
\addcontentsline{toc}{section}{ TS 4.5.11.2 }
\section*{ TS 4.5.11.2 }

\textbf{TS 4.5.11.2 } \newline
\textbf{Samhita Paata} \newline

प्र॒चर॑न्ति सृ॒काव॑न्तो निष॒ङ्गिणः॑ ॥                                     य ए॒ताव॑न्तश्च॒ भूयाꣳ॑सश्च॒ दिशो॑ रु॒द्रा वि॑तस्थि॒रे ॥ तेषाꣳ॑ सहस्रयोज॒ने ऽव॒धन्वा॑नि तन्मसि ॥                                              नमो॑ रु॒द्रेभ्यो॒ ये पृ॑थि॒व्यां ॅये᳚ऽन्तरि॑क्षे॒ ये दि॒वि येषा॒मन्नं॒ ॅवातो॑ व॒र्॒.षमिष॑व॒स्तेभ्यो॒ दश॒ प्राची॒ र्दश॑दक्षि॒णा दश॑प्र॒तीची॒ र्दशोदी॑ची॒ र्दशो॒र्द्ध्वा-स्तेभ्यो॒ नम॒स्ते नो॑ मृडयन्तु॒ ते यं द्वि॒ष्मो यश्च॑ ( ) नो॒ द्वेष्टि॒ तं ॅवो॒ जंभे॑ दधामि ॥ \newline

\textbf{Pada Paata} \newline

प्र॒चर॒न्तीति॑ प्र - चर॑न्ति । सृ॒काव॑न्त॒ इति॑ सृ॒का - व॒न्तः॒ । नि॒ष॒ङ्गिण॒ इति॑ नि - स॒ङ्गिनः॑ ॥ ये । ए॒ताव॑न्तः । च॒ । भूयाꣳ॑सः । च॒ । दिशः॑ । रु॒द्राः । वि॒त॒स्थि॒र इति॑ वि - त॒स्थि॒रे ॥ तेषा᳚म् । स॒ह॒स्र॒यो॒ज॒न इति॑ सहस्र - यो॒ज॒ने । अवेति॑ । धन्वा॑नि । त॒न्म॒सि॒ ॥ नमः॑ । रु॒द्रेभ्यः॑ । ये । पृ॒थि॒व्याम् । ये । अ॒न्तरि॑क्षे । ये । दि॒वि । येषा᳚म् । अन्न᳚म् । वातः॑ । व॒र्॒.षम् । इष॑वः । तेभ्यः॑ । दश॑ । प्राचीः᳚ । दश॑ । द॒क्षि॒णा । दश॑ । प्र॒तीचीः᳚ । दश॑ । उदी॑चीः । दश॑ । ऊ॒द्‌र्ध्वाः । तेभ्यः॑ । नमः॑ । ते । नः॒ । मृ॒ड॒य॒न्तु॒ । ते॒ । यम् । द्वि॒ष्मः । यः । च॒ ( ) । नः॒ । द्वेष्टि॑ । तम् । वः॒ । जंभे᳚ । द॒धा॒मि॒ ॥  \newline






\end{document}