\documentclass[17pt]{extarticle}
\usepackage{babel}
\usepackage{fontspec}
\usepackage{polyglossia}
\usepackage{extsizes}

\usepackage{color}   %May be necessary if you want to color links
\usepackage{hyperref}
\hypersetup{
    colorlinks=true, %set true if you want colored links
    linktoc=all,     %set to all if you want both sections and subsections linked
    linkcolor=black,  %choose some color if you want links to stand out
}

\setmainlanguage{sanskrit}
\setotherlanguages{english} %% or other languages
\setlength{\parindent}{0pt}
\pagestyle{myheadings}
\newfontfamily\devanagarifont[Script=Devanagari]{AdishilaVedic}
\renewcommand{\theHsection}{\thepart.section.\thesection}

\newcommand{\VAR}[1]{}
\newcommand{\BLOCK}[1]{}




\begin{document}
\begin{titlepage}
    \begin{center}
 
\begin{sanskrit}
    { \Large
    कृष्ण यजुर्वेदीय तैत्तिरीय संहिता,पद,जटा,घन पाठः 
    }
    \\
    \vspace{2.5cm}
    \mbox{ \Large
    4.5      चतुर्थकाण्डे पञ्चमः प्रश्नः - होमविधिनिरूपणं   }
\end{sanskrit}
\end{center}

\end{titlepage}
\tableofcontents
\phantomsection
\pagebreak

\markright{ TS 4.5.1.1  \hfill https://www.vedavms.in \hfill}

\section{ TS 4.5.1.1 }

\textbf{TS 4.5.1.1 } \newline
\textbf{Samhita Paata} \newline

नम॑स्ते रुद्र म॒न्यव॑ उ॒तोत॒ इष॑वे॒ नमः॑ । नम॑स्ते अस्तु॒ धन्व॑ने बा॒हुभ्या॑मु॒त ते॒ नमः॑ ॥                                   या त॒ इषुः॑ शि॒वत॑मा शि॒वं ब॒भूव॑ ते॒ धनुः॑ । शि॒वा श॑र॒व्या॑ या तव॒ तया॑ नो रुद्र मृडय ॥     या ते॑ रुद्र शि॒वा त॒नूरघो॒रा ऽपा॑पकाशिनी । तया॑ नस्त॒नुवा॒ शन्त॑मया॒ गिरि॑शन्ता॒भि चा॑कशीहि ॥                                यामिषुं॑ गिरिशन्त॒ हस्ते॒ - [  ] \newline

\textbf{Pada Paata} \newline

नमः॑ । ते॒ । रु॒द्र॒ । म॒न्यवे᳚ । उ॒तो इति॑ । ते॒ । इष॑वे । नमः॑ ॥ नमः॑ । ते॒ । अ॒स्तु॒ । धन्व॑ने । बा॒हुभ्या॒मिति॑ बा॒हु-भ्या॒म् । उ॒त । ते॒ । नमः॑ ॥ या । ते॒ । इषुः॑ । शि॒वत॒मेति॑ शि॒व - त॒मा॒ । शि॒वम् । ब॒भूव॑ । ते॒ । धनुः॑ ॥ शि॒वा । श॒र॒व्या᳚ । या । तव॑ । तया᳚ । नः॒ । रु॒द्र॒ । मृ॒ड॒य॒ ॥ या । ते॒ । रु॒द्र॒ । शि॒वा । त॒नूः । अघो॑रा । अपा॑पकाशि॒नीत्यपा॑प-का॒शि॒नी॒ ॥ तया᳚ । नः॒ । त॒नुवा᳚ । शन्त॑म॒येति॒ शं - त॒म॒या॒ । गिरि॑श॒न्तेति॒ गिरि॑ - श॒न्त॒ । अ॒भीति॑ । चा॒क॒शी॒हि॒ ॥ याम् । इषु᳚म् । गि॒रि॒श॒न्तेति॑ गिरि - श॒न्त॒ । हस्ते᳚ ।  \newline


\textbf{Krama Paata} \newline

नम॑स्ते । ते॒ रु॒द्र॒ । रु॒द्र॒ म॒न्यवे᳚ । म॒न्यव॑ उ॒तो । उ॒तो ते᳚ । उ॒तो इत्यु॒तो । त॒ इष॑वे । इष॑वे॒ नमः॑ । नम॒ इति॒ नमः॑ ॥ नम॑स्ते । ते॒ अ॒स्तु॒ । अ॒स्तु॒ धन्व॑ने । धन्व॑ने बा॒हुभ्या᳚म् । बा॒हुभ्या॑मु॒त । बा॒हुभ्या॒मिति॑ बा॒हु - भ्या॒म् । उ॒त ते᳚ । ते॒ नमः॑ । नम॒ इति॒ नमः॑ ॥ या ते᳚ । त॒ इषुः॑ । इषुः॑ शि॒वत॑मा । शि॒वत॑मा शि॒वम् । शि॒वत॒मेति॑ शि॒व - त॒मा॒ । शि॒वं ब॒भूव॑ । ब॒भूव॑ ते । ते॒ धनुः॑ । धनु॒रिति॒ धनुः॑ ॥ शि॒व श॑र॒व्या᳚ । श॒र॒व्या॑ या । या तव॑ । तव॒ तया᳚ । तया॑ नः । नो॒ रु॒द्र॒ । रु॒द्र॒ मृ॒ड॒य॒ । मृ॒ड॒येति॑ मृडय ॥ या ते᳚ । ते॒ रु॒द्र॒ । रु॒द्र॒ शि॒वा । शि॒वा त॒नूः । त॒नूरघो॑रा । अघो॒राऽपा॑पकाशिनी । अपा॑पकाशि॒नीत्यपा॑प - का॒शि॒नी॒ ॥ तया॑ नः । न॒स्त॒नुवा᳚ । त॒नुवा॒ शन्त॑मया । शन्त॑मया॒ गिरि॑शन्त । शन्त॑म॒येति॒ शं - त॒म॒या॒ । गिरि॑शन्ता॒भि । गिरि॑श॒न्तेति॒ गिरि॑ - श॒न्त॒ । अ॒भिचा॑कशीहि । चा॒क॒शी॒हीति॑चाकशीहि ॥ यामिषु᳚म् । इषुं॑ गिरिशन्त । गि॒रि॒श॒न्त॒ हस्ते᳚ । गि॒रि॒श॒न्तेति॑ गिरि - श॒न्त॒ । हस्ते॒ बिभ॑र्.षि \newline

\textbf{Jatai Paata} \newline

1. नम॑स्ते ते॒ नमो॒ नम॑स्ते । \newline
2. ते॒ रु॒द्र॒ रु॒द्र॒ ते॒ ते॒ रु॒द्र॒ । \newline
3. रु॒द्र॒ म॒न्यवे॑ म॒न्यवे॑ रुद्र रुद्र म॒न्यवे᳚ । \newline
4. म॒न्यव॑ उ॒तो उ॒तो म॒न्यवे॑ म॒न्यव॑ उ॒तो । \newline
5. उ॒तो ते॑ त उ॒तो उ॒तो ते᳚ । \newline
6. उ॒तो इत्यु॒तो । \newline
7. त॒ इष॑व॒ इष॑वे ते त॒ इष॑वे । \newline
8. इष॑वे॒ नमो॒ नम॒ इष॑व॒ इष॑वे॒ नमः॑ । \newline
9. नम॒ इति॒ नमः॑ । \newline
10. नम॑ स्ते ते॒ नमो॒ नम॑ स्ते । \newline
11. ते॒ अ॒स्त्व॒स्तु॒ ते॒ ते॒ अ॒स्तु॒ । \newline
12. अ॒स्तु॒ धन्व॑ने॒ धन्व॑ने अस्त्वस्तु॒ धन्व॑ने । \newline
13. धन्व॑ने बा॒हुभ्या᳚म् बा॒हुभ्या॒म् धन्व॑ने॒ धन्व॑ने बा॒हुभ्या᳚म् । \newline
14. बा॒हुभ्या॑ मु॒तोत बा॒हुभ्या᳚म् बा॒हुभ्या॑ मु॒त । \newline
15. बा॒हुभ्या॒मिति॑ बा॒हु - भ्या॒म् । \newline
16. उ॒त ते॑ त उ॒तोत ते᳚ । \newline
17. ते॒ नमो॒ नम॑ स्ते ते॒ नमः॑ । \newline
18. नम॒ इति॒ नमः॑ । \newline
19. या ते॑ ते॒ या या ते᳚ । \newline
20. त॒ इषु॒ रिषु॑ स्ते त॒ इषुः॑ । \newline
21. इषुः॑ शि॒वत॑मा शि॒वत॒मेषु॒ रिषुः॑ शि॒वत॑मा । \newline
22. शि॒वत॑मा शि॒वꣳ शि॒वꣳ शि॒वत॑मा शि॒वत॑मा शि॒वम् । \newline
23. शि॒वत॒मेति॑ शि॒व - त॒मा॒ । \newline
24. शि॒वम् ब॒भूव॑ ब॒भूव॑ शि॒वꣳ शि॒वम् ब॒भूव॑ । \newline
25. ब॒भूव॑ ते ते ब॒भूव॑ ब॒भूव॑ ते । \newline
26. ते॒ धनु॒र् धनु॑ स्ते ते॒ धनुः॑ । \newline
27. धनु॒रिति॒ धनुः॑ । \newline
28. शि॒वा श॑र॒व्या॑ शर॒व्या॑ शि॒वा शि॒वा श॑र॒व्या᳚ । \newline
29. श॒र॒व्या॑ या या श॑र॒व्या॑ शर॒व्या॑ या । \newline
30. या तव॒ तव॒ या या तव॑ । \newline
31. तव॒ तया॒ तया॒ तव॒ तव॒ तया᳚ । \newline
32. तया॑ नो न॒ स्तया॒ तया॑ नः । \newline
33. नो॒ रु॒द्र॒ रु॒द्र॒ नो॒ नो॒ रु॒द्र॒ । \newline
34. रु॒द्र॒ मृ॒ड॒य॒ मृ॒ड॒य॒ रु॒द्र॒ रु॒द्र॒ मृ॒ड॒य॒ । \newline
35. मृ॒ड॒येति॑ मृडय । \newline
36. या ते॑ ते॒ या या ते᳚ । \newline
37. ते॒ रु॒द्र॒ रु॒द्र॒ ते॒ ते॒ रु॒द्र॒ । \newline
38. रु॒द्र॒ शि॒वा शि॒वा रु॑द्र रुद्र शि॒वा । \newline
39. शि॒वा त॒नू स्त॒नूः शि॒वा शि॒वा त॒नूः । \newline
40. त॒नू रघो॒रा ऽघो॑रा त॒नू स्त॒नू रघो॑रा । \newline
41. अघो॒रा ऽपा॑पकाशि॒ न्यपा॑पकाशि॒ न्यघो॒रा ऽघो॒रा ऽपा॑पकाशिनी । \newline
42. अपा॑पकाशि॒नीत्यपा॑प - का॒शि॒नी॒ । \newline
43. तया॑ नो न॒ स्तया॒ तया॑ नः । \newline
44. न॒ स्त॒नुवा॑ त॒नुवा॑ नो न स्त॒नुवा᳚ । \newline
45. त॒नुवा॒ शन्त॑मया॒ शन्त॑मया त॒नुवा॑ त॒नुवा॒ शन्त॑मया । \newline
46. शन्त॑मया॒ गिरि॑शन्त॒ गिरि॑शन्त॒ शन्त॑मया॒ शन्त॑मया॒ गिरि॑शन्त । \newline
47. शन्त॑म॒येति॒ शं - त॒म॒या॒ । \newline
48. गिरि॑शन्ता॒ भ्य॑भि गिरि॑शन्त॒ गिरि॑शन्ता॒भि । \newline
49. गिरि॑श॒न्तेति॒ गिरि॑ - श॒न्त॒ । \newline
50. अ॒भि चा॑कशीहि चाकशी ह्य॒भ्य॑भि चा॑कशीहि । \newline
51. चा॒क॒शी॒हीति॑ चाकशीहि । \newline
52. या मिषु॒ मिषुं॒ ॅयां ॅया मिषु᳚म् । \newline
53. इषु॑म् गिरिशन्त गिरिश॒न्तेषु॒ मिषु॑म् गिरिशन्त । \newline
54. गि॒रि॒श॒न्त॒ हस्ते॒ हस्ते॑ गिरिशन्त गिरिशन्त॒ हस्ते᳚ । \newline
55. गि॒रि॒श॒न्तेति॑ गिरि - श॒न्त॒ । \newline
56. हस्ते॒ बिभ॑र्.षि॒ बिभ॑र्.षि॒ हस्ते॒ हस्ते॒ बिभ॑र्.षि । \newline

\textbf{Ghana Paata } \newline

1. नम॑स्ते ते॒ नमो॒ नम॑स्ते रुद्र रुद्र ते॒ नमो॒ नम॑स्ते रुद्र । \newline
2. ते॒ रु॒द्र॒ रु॒द्र॒ ते॒ ते॒ रु॒द्र॒ म॒न्यवे॑ म॒न्यवे॑ रुद्र ते ते रुद्र म॒न्यवे᳚ । \newline
3. रु॒द्र॒ म॒न्यवे॑ म॒न्यवे॑ रुद्र रुद्र म॒न्यव॑ उ॒तो उ॒तो म॒न्यवे॑ रुद्र रुद्र म॒न्यव॑ उ॒तो । \newline
4. म॒न्यव॑ उ॒तो उ॒तो म॒न्यवे॑ म॒न्यव॑ उ॒तो ते॑ त उ॒तो म॒न्यवे॑ म॒न्यव॑ उ॒तो ते᳚ । \newline
5. उ॒तो ते॑ त उ॒तो उ॒तो त॒ इष॑व॒ इष॑वे त उ॒तो उ॒तो त॒ इष॑वे । \newline
6. उ॒तो इत्यु॒तो । \newline
7. त॒ इष॑व॒ इष॑वे ते त॒ इष॑वे॒ नमो॒ नम॒ इष॑वे ते त॒ इष॑वे॒ नमः॑ । \newline
8. इष॑वे॒ नमो॒ नम॒ इष॑व॒ इष॑वे॒ नमः॑ । \newline
9. नम॒ इति॒ नमः॑ । \newline
10. नम॑स्ते ते॒ नमो॒ नम॑स्ते अस्त्वस्तु ते॒ नमो॒ नम॑स्ते अस्तु । \newline
11. ते॒ अ॒स्त्व॒स्तु॒ ते॒ ते॒ अ॒स्तु॒ धन्व॑ने॒ धन्व॑ने अस्तु ते ते अस्तु॒ धन्व॑ने । \newline
12. अ॒स्तु॒ धन्व॑ने॒ धन्व॑ने अस्त्वस्तु॒ धन्व॑ने बा॒हुभ्या᳚म् बा॒हुभ्या॒म् धन्व॑ने अस्त्वस्तु॒ धन्व॑ने बा॒हुभ्या᳚म् । \newline
13. धन्व॑ने बा॒हुभ्या᳚म् बा॒हुभ्या॒म् धन्व॑ने॒ धन्व॑ने बा॒हुभ्या॑ मु॒तोत बा॒हुभ्या॒म् धन्व॑ने॒ धन्व॑ने बा॒हुभ्या॑ मु॒त । \newline
14. बा॒हुभ्या॑ मु॒तोत बा॒हुभ्या᳚म् बा॒हुभ्या॑ मु॒त ते॑ त उ॒त बा॒हुभ्या᳚म् बा॒हुभ्या॑ मु॒त ते᳚ । \newline
15. बा॒हुभ्या॒मिति॑ बा॒हु - भ्या॒म् । \newline
16. उ॒त ते॑ त उ॒तोत ते॒ नमो॒ नम॑स्त उ॒तोत ते॒ नमः॑ । \newline
17. ते॒ नमो॒ नम॑ स्ते ते॒ नमः॑ । \newline
18. नम॒ इति॒ नमः॑ । \newline
19. या ते॑ ते॒ या या त॒ इषु॒ रिषु॑ स्ते॒ या या त॒ इषुः॑ । \newline
20. त॒ इषु॒रिषु॑ स्ते त॒ इषुः॑ शि॒वत॑मा शि॒वत॒मेषु॑ स्ते त॒ इषुः॑ शि॒वत॑मा । \newline
21. इषुः॑ शि॒वत॑मा शि॒वत॒मेषु॒ रिषुः॑ शि॒वत॑मा शि॒वꣳ शि॒वꣳ शि॒वत॒मेषु॒ रिषुः॑ शि॒वत॑मा शि॒वम् । \newline
22. शि॒वत॑मा शि॒वꣳ शि॒वꣳ शि॒वत॑मा शि॒वत॑मा शि॒वम् ब॒भूव॑ ब॒भूव॑ शि॒वꣳ शि॒वत॑मा शि॒वत॑मा शि॒वम् ब॒भूव॑ । \newline
23. शि॒वत॒मेति॑ शि॒व - त॒मा॒ । \newline
24. शि॒वम् ब॒भूव॑ ब॒भूव॑ शि॒वꣳ शि॒वम् ब॒भूव॑ ते ते ब॒भूव॑ शि॒वꣳ शि॒वम् ब॒भूव॑ ते । \newline
25. ब॒भूव॑ ते ते ब॒भूव॑ ब॒भूव॑ ते॒ धनु॒र् धनु॑ स्ते ब॒भूव॑ ब॒भूव॑ ते॒ धनुः॑ । \newline
26. ते॒ धनु॒र् धनु॑ स्ते ते॒ धनुः॑ । \newline
27. धनु॒रिति॒ धनुः॑ । \newline
28. शि॒वा श॑र॒व्या॑ शर॒व्या॑ शि॒वा शि॒वा श॑र॒व्या॑ या या श॑र॒व्या॑ शि॒वा शि॒वा श॑र॒व्या॑ या । \newline
29. श॒र॒व्या॑ या या श॑र॒व्या॑ शर॒व्या॑ या तव॒ तव॒ या श॑र॒व्या॑ शर॒व्या॑ या तव॑ । \newline
30. या तव॒ तव॒ या या तव॒ तया॒ तया॒ तव॒ या या तव॒ तया᳚ । \newline
31. तव॒ तया॒ तया॒ तव॒ तव॒ तया॑ नो न॒ स्तया॒ तव॒ तव॒ तया॑ नः । \newline
32. तया॑ नो न॒ स्तया॒ तया॑ नो रुद्र रुद्र न॒ स्तया॒ तया॑ नो रुद्र । \newline
33. नो॒ रु॒द्र॒ रु॒द्र॒ नो॒ नो॒ रु॒द्र॒ मृ॒ड॒य॒ मृ॒ड॒य॒ रु॒द्र॒ नो॒ नो॒ रु॒द्र॒ मृ॒ड॒य॒ । \newline
34. रु॒द्र॒ मृ॒ड॒य॒ मृ॒ड॒य॒ रु॒द्र॒ रु॒द्र॒ मृ॒ड॒य॒ । \newline
35. मृ॒ड॒येति॑ मृडय । \newline
36. या ते॑ ते॒ या या ते॑ रुद्र रुद्र ते॒ या या ते॑ रुद्र । \newline
37. ते॒ रु॒द्र॒ रु॒द्र॒ ते॒ ते॒ रु॒द्र॒ शि॒वा शि॒वा रु॑द्र ते ते रुद्र शि॒वा । \newline
38. रु॒द्र॒ शि॒वा शि॒वा रु॑द्र रुद्र शि॒वा त॒नू स्त॒नूः शि॒वा रु॑द्र रुद्र शि॒वा त॒नूः । \newline
39. शि॒वा त॒नू स्त॒नूः शि॒वा शि॒वा त॒नू रघो॒रा ऽघो॑रा त॒नूः शि॒वा शि॒वा त॒नू रघो॑रा । \newline
40. त॒नू रघो॒रा ऽघो॑रा त॒नू स्त॒नू रघो॒रा ऽपा॑पकाशि॒ न्यपा॑पकाशि॒ न्यघो॑रा त॒नू स्त॒नू रघो॒रा ऽपा॑पकाशिनी । \newline
41. अघो॒रा ऽपा॑पकाशि॒ न्यपा॑पकाशि॒ न्यघो॒रा ऽघो॒रा ऽपा॑पकाशिनी । \newline
42. अपा॑पकाशि॒नीत्यपा॑प - का॒शि॒नी॒ । \newline
43. तया॑ नो न॒ स्तया॒ तया॑ न स्त॒नुवा॑ त॒नुवा॑ न॒स्तया॒ तया॑ न स्त॒नुवा᳚ । \newline
44. न॒ स्त॒नुवा॑ त॒नुवा॑ नो न स्त॒नुवा॒ शन्त॑मया॒ शन्त॑मया त॒नुवा॑ नो न स्त॒नुवा॒ शन्त॑मया । \newline
45. त॒नुवा॒ शन्त॑मया॒ शन्त॑मया त॒नुवा॑ त॒नुवा॒ शन्त॑मया॒ गिरि॑शन्त॒ गिरि॑शन्त॒ शन्त॑मया त॒नुवा॑ त॒नुवा॒ शन्त॑मया॒ गिरि॑शन्त । \newline
46. शन्त॑मया॒ गिरि॑शन्त॒ गिरि॑शन्त॒ शन्त॑मया॒ शन्त॑मया॒ गिरि॑शन्ता॒भ्य॑भि गिरि॑शन्त॒ शन्त॑मया॒ शन्त॑मया॒ गिरि॑शन्ता॒भि । \newline
47. शन्त॑म॒येति॒ शं - त॒म॒या॒ । \newline
48. गिरि॑शन्ता॒भ्य॑भि गिरि॑शन्त॒ गिरि॑शन्ता॒भि चा॑कशीहि चाकशीह्य॒भि गिरि॑शन्त॒ गिरि॑शन्ता॒भि चा॑कशीहि । \newline
49. गिरि॑श॒न्तेति॒ गिरि॑ - श॒न्त॒ । \newline
50. अ॒भि चा॑कशीहि चाकशीह्य॒भ्य॑भि चा॑कशीहि । \newline
51. चा॒क॒शी॒हीति॑ चाकशीहि । \newline
52. या मिषु॒ मिषुं॒ ॅयां ॅया मिषु॑म् गिरिशन्त गिरिश॒न्तेषुं॒ ॅयां ॅया मिषु॑म् गिरिशन्त । \newline
53. इषु॑म् गिरिशन्त गिरिश॒न्तेषु॒ मिषु॑म् गिरिशन्त॒ हस्ते॒ हस्ते॑ गिरिश॒न्तेषु॒ मिषु॑म् गिरिशन्त॒ हस्ते᳚ । \newline
54. गि॒रि॒श॒न्त॒ हस्ते॒ हस्ते॑ गिरिशन्त गिरिशन्त॒ हस्ते॒ बिभ॑र्.षि॒ बिभ॑र्.षि॒ हस्ते॑ गिरिशन्त गिरिशन्त॒ हस्ते॒ बिभ॑र्.षि । \newline
55. गि॒रि॒श॒न्तेति॑ गिरि - श॒न्त॒ । \newline
56. हस्ते॒ बिभ॑र्.षि॒ बिभ॑र्.षि॒ हस्ते॒ हस्ते॒ बिभ॒र्ष्यस्त॑वे॒ अस्त॑वे॒ बिभ॑र्.षि॒ हस्ते॒ हस्ते॒ बिभ॒र्ष्यस्त॑वे । \newline
\pagebreak
\markright{ TS 4.5.1.2  \hfill https://www.vedavms.in \hfill}

\section{ TS 4.5.1.2 }

\textbf{TS 4.5.1.2 } \newline
\textbf{Samhita Paata} \newline

बिभ॒र्ष्यस्त॑वे । शि॒वां गि॑रित्र॒ तां कु॑रु॒ मा हिꣳ॑सी॒ः॒ पुरु॑षं॒ जग॑त् ॥        शि॒वेन॒ वच॑सा त्वा॒ गिरि॒शाच्छा॑ वदामसि ।यथा॑ नः॒ सर्व॒मि-ज्जग॑दय॒क्ष्मꣳ सु॒मना॒ अस॑त् ॥                             अद्ध्य॑वोचदधिव॒क्ता प्र॑थ॒मो दैव्यो॑ भि॒षक् । अहीꣳ॑श्च॒॒ सर्वा᳚न् ज॒भंय॒न्थ् सर्वा᳚श्च यातु धा॒न्यः॑ ॥                          अ॒सौ यस्ता॒म्रो अ॑रु॒ण उ॒त ब॒भ्रुः सु॑म॒ङ्गलः॑ । ये चे॒माꣳ रु॒द्रा अ॒भितो॑ दि॒क्षु - [  ] \newline

\textbf{Pada Paata} \newline

बिभ॑र्.षि । अस्त॑वे ॥ शि॒वाम् । गि॒रि॒त्रेति॑ गिरि - त्र॒ । ताम् । कु॒रु॒ । मा । हिꣳ॒॒सीः॒ । पुरु॑षम् । जग॑त् ॥ शि॒वेन॑ । वच॑सा । त्वा॒ । गिरि॑श । अच्छ॑ । व॒दा॒म॒सि॒ ॥ यथा᳚ । नः॒ । सर्व᳚म् । इत् । जग॑त् । अ॒य॒क्ष्मम् । सु॒मना॒ इति॑ सु - मनाः᳚ । अस॑त् ॥ अधीति॑ । अ॒वो॒च॒त् । अ॒धि॒व॒क्तेत्य॑धि - व॒क्ता । प्र॒थ॒मः । दैव्यः॑ । भि॒षक् ॥ अहीन्॑ । च॒ । सर्वान्॑ । ज॒भंयन्न्॑ । सर्वाः᳚ । च॒ । या॒तु॒धा॒न्य॑ इति॑ यातु - धा॒न्यः॑ ॥ अ॒सौ । यः । ता॒म्रः । अ॒रु॒णः । उ॒त । ब॒भ्रुः । सु॒म॒ङ्गल॒ इति॑ सु - म॒ङ्गलः॑ ॥ ये । च॒ । इ॒माम् । रु॒द्राः । अ॒भितः॑ । दि॒क्षु ।  \newline


\textbf{Krama Paata} \newline

बिभ॒र्ष्यस्त॑वे । अस्त॑व॒ इत्यस्त॑वे ॥ शि॒वां गि॑रित्र । गि॒रि॒त्र॒ ताम् । गि॒रि॒त्रेति॑ गिरि - त्र॒ । तां कु॑रु । कु॒रु॒ मा । मा हिꣳ॑सीः । हिꣳ॒॒सीः॒ पुरु॑षम् । पुरु॑ष॒म् जग॑त् । जग॒दिति॒ जग॑त् ॥ शि॒वेन॒ वच॑सा । वच॑सा त्वा । त्वा॒ गिरि॑श । गिरि॒शाच्छ॑ । अच्छा॑ वदामसि । व॒दा॒म॒सीति॑ वदामसि ॥ यथा॑ नः । नः॒ सर्व᳚म् । सर्व॒मित् । इज् जग॑त् । जग॑दय॒क्ष्मम् । अ॒य॒क्ष्मꣳ सु॒मनाः᳚ । सु॒मना॒ अस॑त् । सु॒मना॒ इति॑ सु - मनाः᳚ । अस॒दित्यस॑त् ॥ अद्ध्य॑वोचत् । अ॒वो॒च॒द॒धि॒व॒क्ता । अ॒धि॒व॒क्ता प्र॑थ॒मः । अ॒धि॒व॒क्तेत्य॑धि - व॒क्ता । प्र॒थ॒मो दैव्यः॑ । दैव्यो॑ भि॒षक् । भि॒षगिति॑ भि॒षक् ॥ अ॒हीꣳ॑श्च । च॒ सर्वान्॑ । सर्वा᳚न् ज॒म्भयन्न्॑ । ज॒म्भय॒न्थ् सर्वाः᳚ । सर्वा᳚श्च । च॒ या॒तु॒धा॒न्यः॑ । या॒तु॒धा॒न्य॑ इति॑ यातु - धा॒न्यः॑ ॥ अ॒सौ यः । यस्ता॒म्रः । ता॒म्रो अ॑रु॒णः । अ॒रु॒ण उ॒त । उ॒त ब॒भ्रुः । ब॒भ्रुः सु॑म॒ङ्गलः॑ । सु॒म॒ङ्गल॒ इति॑ सु - म॒ङ्गलः॑ ॥ ये च॑ । चे॒माम् । इ॒माꣳ रु॒द्राः । रु॒द्रा अ॒भितः॑ । अ॒भितो॑ दि॒क्षु । दि॒क्षु श्रि॒ताः \newline

\textbf{Jatai Paata} \newline

1. बिभ॒र्ष्यस्त॑वे॒ अस्त॑वे॒ बिभ॑र्.षि॒ बिभ॒र्ष्यस्त॑वे । \newline
2. अस्त॑व॒ इत्यस्त॑वे । \newline
3. शि॒वाम् गि॑रित्र गिरित्र शि॒वाꣳ शि॒वाम् गि॑रित्र । \newline
4. गि॒रि॒त्र॒ ताम् ताम् गि॑रित्र गिरित्र॒ ताम् । \newline
5. गि॒रि॒त्रेति॑ गिरि - त्र॒ । \newline
6. ताम् कु॑रु कुरु॒ ताम् ताम् कु॑रु । \newline
7. कु॒रु॒ मा मा कु॑रु कुरु॒ मा । \newline
8. मा हि(ग्म्॑)सीर्. हिꣳसी॒र् मा मा हि(ग्म्॑)सीः । \newline
9. हि॒(ग्म्॒)सीः॒ पुरु॑ष॒म् पुरु॑षꣳ हिꣳसीर्. हिꣳसीः॒ पुरु॑षम् । \newline
10. पुरु॑ष॒म् जग॒ज् जग॒त् पुरु॑ष॒म् पुरु॑ष॒म् जग॑त् । \newline
11. जग॒दिति॒ जग॑त् । \newline
12. शि॒वेन॒ वच॑सा॒ वच॑सा शि॒वेन॑ शि॒वेन॒ वच॑सा । \newline
13. वच॑सा त्वा त्वा॒ वच॑सा॒ वच॑सा त्वा । \newline
14. त्वा॒ गिरि॑श॒ गिरि॑श त्वा त्वा॒ गिरि॑श । \newline
15. गिरि॒शा च्छाच्छ॒ गिरि॑श॒ गिरि॒शाच्छ॑ । \newline
16. अच्छा॑ वदामसि वदाम॒ स्यच्छाच्छा॑ वदामसि । \newline
17. व॒दा॒म॒सीति॑ वदामसि । \newline
18. यथा॑ नो नो॒ यथा॒ यथा॑ नः । \newline
19. नः॒ सर्व॒(ग्म्॒) सर्व॑म् नो नः॒ सर्व᳚म् । \newline
20. सर्व॒ मिदिथ् सर्व॒(ग्म्॒) सर्व॒ मित् । \newline
21. इज् जग॒ज् जग॒ दिदिज् जग॑त् । \newline
22. जग॑दय॒क्ष्म म॑य॒क्ष्मम् जग॒ज् जग॑दय॒क्ष्मम् । \newline
23. अ॒य॒क्ष्मꣳ सु॒मनाः᳚ सु॒मना॑ अय॒क्ष्म म॑य॒क्ष्मꣳ सु॒मनाः᳚ । \newline
24. सु॒मना॒ अस॒ दस॑थ् सु॒मनाः᳚ सु॒मना॒ अस॑त् । \newline
25. सु॒मना॒ इति॑ सु - मनाः᳚ । \newline
26. अस॒दित्यस॑त् । \newline
27. अध्य॑वोच दवोच॒ दध्य ध्य॑वोचत् । \newline
28. अ॒वो॒च॒ द॒धि॒व॒क्ता ऽधि॑व॒क्ता ऽवो॑च दवोच दधिव॒क्ता । \newline
29. अ॒धि॒व॒क्ता प्र॑थ॒मः प्र॑थ॒मो अ॑धिव॒क्ता ऽधि॑व॒क्ता प्र॑थ॒मः । \newline
30. अ॒धि॒व॒क्तेत्य॑धि - व॒क्ता । \newline
31. प्र॒थ॒मो दैव्यो॒ दैव्यः॑ प्रथ॒मः प्र॑थ॒मो दैव्यः॑ । \newline
32. दैव्यो॑ भि॒षग् भि॒षग् दैव्यो॒ दैव्यो॑ भि॒षक् । \newline
33. भि॒षगिति॑ भि॒षक् । \newline
34. अही(ग्ग्॑)श्च॒ चाही॒ नही(ग्ग्॑)श्च । \newline
35. च॒ सर्वा॒न् थ्सर्वा(ग्ग्॑)श्च च॒ सर्वान्॑ । \newline
36. सर्वा᳚न् जं॒भय॑न् जं॒भय॒न् थ्सर्वा॒न् थ्सर्वा᳚न् जं॒भयन्न्॑ । \newline
37. जं॒भय॒न् थ्सर्वाः॒ सर्वा॑ जं॒भय॑न् जं॒भय॒न् थ्सर्वाः᳚ । \newline
38. सर्वा᳚श्च च॒ सर्वाः॒ सर्वा᳚श्च । \newline
39. च॒ या॒तु॒धा॒न्यो॑ यातुधा॒न्य॑श्च च यातुधा॒न्यः॑ । \newline
40. या॒तु॒धा॒न्य॑ इति॑ यातु - धा॒न्यः॑ । \newline
41. अ॒सौ यो यो अ॒सा व॒सौ यः । \newline
42. य स्ता॒म्र स्ता॒म्रो यो य स्ता॒म्रः । \newline
43. ता॒म्रो अ॑रु॒णो अ॑रु॒ण स्ता॒म्र स्ता॒म्रो अ॑रु॒णः । \newline
44. अ॒रु॒ण उ॒तोता रु॒णो अ॑रु॒ण उ॒त । \newline
45. उ॒त ब॒भ्रुर् ब॒भ्रु रु॒तोत ब॒भ्रुः । \newline
46. ब॒भ्रुः सु॑म॒ङ्गलः॑ सुम॒ङ्गलो॑ ब॒भ्रुर् ब॒भ्रुः सु॑म॒ङ्गलः॑ । \newline
47. सु॒म॒ङ्गल॒ इति॑ सु - म॒ङ्गलः॑ । \newline
48. ये च॑ च॒ ये ये च॑ । \newline
49. चे॒मा मि॒माम् च॑ चे॒माम् । \newline
50. इ॒माꣳ रु॒द्रा रु॒द्रा इ॒मा मि॒माꣳ रु॒द्राः । \newline
51. रु॒द्रा अ॒भितो॑ अ॒भितो॑ रु॒द्रा रु॒द्रा अ॒भितः॑ । \newline
52. अ॒भितो॑ दि॒क्षु दि॒क्ष्व॑भितो॑ अ॒भितो॑ दि॒क्षु । \newline
53. दि॒क्षु श्रि॒ताः श्रि॒ता दि॒क्षु दि॒क्षु श्रि॒ताः । \newline

\textbf{Ghana Paata } \newline

1. बिभ॒र्ष्यस्त॑वे॒ अस्त॑वे॒ बिभ॑र्.षि॒ बिभ॒र्ष्यस्त॑वे । \newline
2. अस्त॑व॒ इत्यस्त॑वे । \newline
3. शि॒वाम् गि॑रित्र गिरित्र शि॒वाꣳ शि॒वाम् गि॑रित्र॒ ताम् ताम् गि॑रित्र शि॒वाꣳ शि॒वाम् गि॑रित्र॒ ताम् । \newline
4. गि॒रि॒त्र॒ ताम् ताम् गि॑रित्र गिरित्र॒ ताम् कु॑रु कुरु॒ ताम् गि॑रित्र गिरित्र॒ ताम् कु॑रु । \newline
5. गि॒रि॒त्रेति॑ गिरि - त्र॒ । \newline
6. ताम् कु॑रु कुरु॒ ताम् ताम् कु॑रु॒ मा मा कु॑रु॒ ताम् ताम् कु॑रु॒ मा । \newline
7. कु॒रु॒ मा मा कु॑रु कुरु॒ मा हिꣳ॑सीर्. हिꣳसी॒र् मा कु॑रु कुरु॒ मा हिꣳ॑सीः । \newline
8. मा हिꣳ॑सीर्. हिꣳसी॒र् मा मा हिꣳ॑सीः॒ पुरु॑ष॒म् पुरु॑षꣳ हिꣳसी॒र् मा मा हिꣳ॑सीः॒ पुरु॑षम् । \newline
9. हिꣳ॒॒सीः॒ पुरु॑ष॒म् पुरु॑षꣳ हिꣳसीर्. हिꣳसीः॒ पुरु॑ष॒म् जग॒ज् जग॒त् पुरु॑षꣳ हिꣳसीर्. हिꣳसीः॒ पुरु॑ष॒म् जग॑त् । \newline
10. पुरु॑ष॒म् जग॒ज् जग॒त् पुरु॑ष॒म् पुरु॑ष॒म् जग॑त् । \newline
11. जग॒दिति॒ जग॑त् । \newline
12. शि॒वेन॒ वच॑सा॒ वच॑सा शि॒वेन॑ शि॒वेन॒ वच॑सा त्वा त्वा॒ वच॑सा शि॒वेन॑ शि॒वेन॒ वच॑सा त्वा । \newline
13. वच॑सा त्वा त्वा॒ वच॑सा॒ वच॑सा त्वा॒ गिरि॑श॒ गिरि॑श त्वा॒ वच॑सा॒ वच॑सा त्वा॒ गिरि॑श । \newline
14. त्वा॒ गिरि॑श॒ गिरि॑श त्वा त्वा॒ गिरि॒शा च्छाच्छ॒ गिरि॑श त्वा त्वा॒ गिरि॒शाच्छ॑ । \newline
15. गिरि॒शा च्छाच्छ॒ गिरि॑श॒ गिरि॒शाच्छा॑ वदामसि वदाम॒ स्यच्छ॒ गिरि॑श॒ गिरि॒शाच्छा॑ वदामसि । \newline
16. अच्छा॑ वदामसि वदाम॒ स्यच्छाच्छा॑ वदामसि । \newline
17. व॒दा॒म॒सीति॑ वदामसि । \newline
18. यथा॑ नो नो॒ यथा॒ यथा॑ नः॒ सर्वꣳ॒॒ सर्व॑म् नो॒ यथा॒ यथा॑ नः॒ सर्व᳚म् । \newline
19. नः॒ सर्वꣳ॒॒ सर्व॑म् नो नः॒ सर्व॒ मिदिथ् सर्व॑म् नो नः॒ सर्व॒ मित् । \newline
20. सर्व॒ मिदिथ् सर्वꣳ॒॒ सर्व॒ मिज् जग॒ज् जग॒दिथ् सर्वꣳ॒॒ सर्व॒ मिज् जग॑त् । \newline
21. इज् जग॒ज् जग॒ दिदिज् जग॑ दय॒क्ष्म म॑य॒क्ष्मम् जग॒ दिदिज् जग॑ दय॒क्ष्मम् । \newline
22. जग॑ दय॒क्ष्म म॑य॒क्ष्मम् जग॒ज् जग॑ दय॒क्ष्मꣳ सु॒मनाः᳚ सु॒मना॑ अय॒क्ष्मम् जग॒ज् जग॑ दय॒क्ष्मꣳ सु॒मनाः᳚ । \newline
23. अ॒य॒क्ष्मꣳ सु॒मनाः᳚ सु॒मना॑ अय॒क्ष्म म॑य॒क्ष्मꣳ सु॒मना॒ अस॒ दस॑थ् सु॒मना॑ अय॒क्ष्म म॑य॒क्ष्मꣳ सु॒मना॒ अस॑त् । \newline
24. सु॒मना॒ अस॒ दस॑थ् सु॒मनाः᳚ सु॒मना॒ अस॑त् । \newline
25. सु॒मना॒ इति॑ सु - मनाः᳚ । \newline
26. अस॒दित्यस॑त् । \newline
27. अध्य॑ वोचद वोच॒ दध्यध्य॑ वोच दधिव॒क्ता ऽधि॑व॒क्ता ऽवो॑च॒द ध्यध्य॑ वोच दधिव॒क्ता । \newline
28. अ॒वो॒च॒ द॒धि॒व॒क्ता ऽधि॑व॒क्ता ऽवो॑च दवोच दधिव॒क्ता प्र॑थ॒मः प्र॑थ॒मो अ॑धिव॒क्ता ऽवो॑च दवोच दधिव॒क्ता प्र॑थ॒मः । \newline
29. अ॒धि॒व॒क्ता प्र॑थ॒मः प्र॑थ॒मो अ॑धिव॒क्ता ऽधि॑व॒क्ता प्र॑थ॒मो दैव्यो॒ दैव्यः॑ प्रथ॒मो अ॑धिव॒क्ता ऽधि॑व॒क्ता प्र॑थ॒मो दैव्यः॑ । \newline
30. अ॒धि॒व॒क्तेत्य॑धि - व॒क्ता । \newline
31. प्र॒थ॒मो दैव्यो॒ दैव्यः॑ प्रथ॒मः प्र॑थ॒मो दैव्यो॑ भि॒षग् भि॒षग् दैव्यः॑ प्रथ॒मः प्र॑थ॒मो दैव्यो॑ भि॒षक् । \newline
32. दैव्यो॑ भि॒षग् भि॒षग् दैव्यो॒ दैव्यो॑ भि॒षक् । \newline
33. भि॒षगिति॑ भि॒षक् । \newline
34. अहीꣳ॑श्च॒ चाही॒ नहीꣳ॑श्च॒ सर्वा॒न् थ्सर्वाꣳ॒॒श्चाही॒ नहीꣳ॑श्च॒ सर्वान्॑ । \newline
35. च॒ सर्वा॒न् थ्सर्वाꣳ॑श्च च॒ सर्वा᳚न् जं॒भय॑न् जं॒भय॒न् थ्सर्वाꣳ॑श्च च॒ सर्वा᳚न् जं॒भयन्न्॑ । \newline
36. सर्वा᳚न् जं॒भय॑न् जं॒भय॒न् थ्सर्वा॒न् थ्सर्वा᳚न् जं॒भय॒न् थ्सर्वाः॒ सर्वा॑ जं॒भय॒न् थ्सर्वा॒न् थ्सर्वा᳚न् जं॒भय॒न् थ्सर्वाः᳚ । \newline
37. जं॒भय॒न् थ्सर्वाः॒ सर्वा॑ जं॒भय॑न् जं॒भय॒न् थ्सर्वा᳚श्च च॒ सर्वा॑ जं॒भय॑न् जं॒भय॒न् थ्सर्वा᳚श्च । \newline
38. सर्वा᳚श्च च॒ सर्वाः॒ सर्वा᳚श्च यातुधा॒न्यो॑ यातुधा॒न्य॑श्च॒ सर्वाः॒ सर्वा᳚श्च यातुधा॒न्यः॑ । \newline
39. च॒ या॒तु॒धा॒न्यो॑ यातुधा॒न्य॑श्च च यातुधा॒न्यः॑ । \newline
40. या॒तु॒धा॒न्य॑ इति॑ यातु - धा॒न्यः॑ । \newline
41. अ॒सौ यो यो अ॒सा व॒सौ य स्ता॒म्र स्ता॒म्रो यो अ॒सा व॒सौ य स्ता॒म्रः । \newline
42. य स्ता॒म्र स्ता॒म्रो यो यस्ता॒म्रो अ॑रु॒णो अ॑रु॒ण स्ता॒म्रो यो यस्ता॒म्रो अ॑रु॒णः । \newline
43. ता॒म्रो अ॑रु॒णो अ॑रु॒ण स्ता॒म्र स्ता॒म्रो अ॑रु॒ण उ॒तोतारु॒ण स्ता॒म्र स्ता॒म्रो अ॑रु॒ण उ॒त । \newline
44. अ॒रु॒ण उ॒तोतारु॒णो अ॑रु॒ण उ॒त ब॒भ्रुर् ब॒भ् रुरु॒ता रु॒णो अ॑रु॒ण उ॒त ब॒भ्रुः । \newline
45. उ॒त ब॒भ्रुर् ब॒भ्रु रु॒तोत ब॒भ्रुः सु॑म॒ङ्गलः॑ सुम॒ङ्गलो॑ ब॒भ्रुरु॒तोत ब॒भ्रुः सु॑म॒ङ्गलः॑ । \newline
46. ब॒भ्रुः सु॑म॒ङ्गलः॑ सुम॒ङ्गलो॑ ब॒भ्रुर् ब॒भ्रुः सु॑म॒ङ्गलः॑ । \newline
47. सु॒म॒ङ्गल॒ इति॑ सु - म॒ङ्गलः॑ । \newline
48. ये च॑ च॒ ये ये चे॒ मा मि॒माम् च॒ ये ये चे॒ माम् । \newline
49. चे॒ मा मि॒माम् च॑ चे॒ माꣳ रु॒द्रा रु॒द्रा इ॒माम् च॑ चे॒ माꣳ रु॒द्राः । \newline
50. इ॒माꣳ रु॒द्रा रु॒द्रा इ॒मा मि॒माꣳ रु॒द्रा अ॒भितो॑ अ॒भितो॑ रु॒द्रा इ॒मा मि॒माꣳ रु॒द्रा अ॒भितः॑ । \newline
51. रु॒द्रा अ॒भितो॑ अ॒भितो॑ रु॒द्रा रु॒द्रा अ॒भितो॑ दि॒क्षु दि॒क्ष्व॑भितो॑ रु॒द्रा रु॒द्रा अ॒भितो॑ दि॒क्षु । \newline
52. अ॒भितो॑ दि॒क्षु दि॒क्ष्व॑भितो॑ अ॒भितो॑ दि॒क्षु श्रि॒ताः श्रि॒ता दि॒क्ष्व॑भितो॑ अ॒भितो॑ दि॒क्षु श्रि॒ताः । \newline
53. दि॒क्षु श्रि॒ताः श्रि॒ता दि॒क्षु दि॒क्षु श्रि॒ताः स॑हस्र॒शः स॑हस्र॒शः श्रि॒ता दि॒क्षु दि॒क्षु श्रि॒ताः स॑हस्र॒शः । \newline
\pagebreak
\markright{ TS 4.5.1.3  \hfill https://www.vedavms.in \hfill}

\section{ TS 4.5.1.3 }

\textbf{TS 4.5.1.3 } \newline
\textbf{Samhita Paata} \newline

श्रि॒ताः स॑हस्र॒शो ऽवै॑षाꣳ॒॒ हेड॑ ईमहे ॥ अ॒सौ यो॑ ऽव॒सर्प॑ति॒ नील॑ग्रीवो॒ विलो॑हितः । उ॒तैनं॑ गो॒पा अ॑दृश॒न्-नदृ॑शन्-नुदहा॒र्यः॑ । उ॒तैनं॒ ॅविश्वा॑ भू॒तानि॒ स दृ॒ष्टो मृ॑डयाति नः ॥                               नमो॑ अस्तु॒ नील॑ग्रीवाय सहस्रा॒क्षाय॑ मी॒ढुषे᳚ । अथो॒ ये अ॑स्य॒ सत्वा॑नो॒ऽहं तेभ्यो॑ ऽकर॒न्नमः॑ ॥                                   प्रमुं॑च॒ धन्व॑न॒स्त्व मु॒भयो॒-रार्त्नि॑यो॒र्ज्यां । याश्च॑ ते॒ हस्त॒ इष॑वः॒ - [  ] \newline

\textbf{Pada Paata} \newline

श्रि॒ताः । स॒ह॒स्र॒श इति॑ सहस्र - शः । अवेति॑ । ए॒षा॒म् । हेडः॑ । ई॒म॒हे॒ ॥ अ॒सौ । यः । अ॒व॒सर्प॒तीत्य॑व - सर्प॑ति । नील॑ग्रीव॒ इति॒ नील॑ - ग्री॒वः॒ । विलो॑हित॒ इति॒ वि - लो॒हि॒तः॒ ॥ उ॒त । ए॒न॒म् । गो॒पा इति॑ गो - पाः । अ॒दृ॒श॒न्न् । अदृ॑शन्न् । उ॒द॒हा॒र्य॑ इत्युद॑-हा॒र्यः॑ ॥ उ॒त । ए॒न॒म् । विश्वा᳚ । भू॒तानि॑ । सः । दृ॒ष्टः । मृ॒ड॒या॒ति॒ । नः॒ ॥ नमः॑ । अ॒स्तु॒ । नील॑ग्रीवा॒येति॒ नील॑ - ग्री॒वा॒य॒ । स॒ह॒स्रा॒क्षायेति॑ सहस्र - अ॒क्षाय॑ । मी॒ढुषे᳚ ॥ अथो॒ इति॑ । ये । अ॒स्य॒ । सत्वा॑नः । अ॒हम् । तेभ्यः॑ । अ॒क॒र॒म् । नमः॑ ॥ प्रेति॑ । मु॒ञ्च॒ । धन्व॑नः । त्वम् । उ॒भयोः᳚ । आर्त्नि॑योः । ज्याम् ॥ याः । च॒ । ते॒ । हस्ते᳚ । इष॑वः ।  \newline


\textbf{Krama Paata} \newline

श्रि॒ताः स॑हस्र॒शः । स॒ह॒स्र॒शोऽव॑ । स॒ह॒स्र॒श इति॑ सहस्र - 
शः । अवै॑षाम् । ए॒षाꣳ॒॒ हेडः॑ । हेड॑ ईमहे । 

ई॒म॒ह॒ इती॑महे ॥ अ॒सौ यः । यो॑ऽव॒सर्प॑ति । अ॒व॒सर्प॑ति॒ नील॑ग्रीवः । अ॒व॒सर्प॒तीत्य॑व - सर्प॑ति । नील॑ग्रीवो॒ विलो॑हितः । नील॑ग्रीव॒ इति॒ नील॑ - ग्री॒वः॒ । विलो॑हित॒ इति॒ वि - लो॒हि॒तः॒ ॥ उ॒तैन᳚म् । ए॒न॒म् गो॒पाः । गो॒पा अ॑दृशन्न् । गो॒पा इति॑ गो - पाः । अ॒दृ॒श॒न्नदृ॑शन्न् । अदृ॑शन्नुदहा॒र्यः॑ । उ॒द॒हा॒र्य॑ इत्यु॑द - हा॒र्यः॑ ॥ उ॒तैन᳚म् । ए॒न॒म् ॅविश्वा᳚ । विश्वा॑ भू॒तानि॑ । भू॒तानि॒ सः । स दृ॒ष्टः । दृ॒ष्टो मृ॑डयाति । मृ॒ड॒या॒ति॒ नः॒ । न॒ इति॑ नः ॥ नमो॑ अस्तु । अ॒स्तु॒ नील॑ग्रीवाय । नील॑ग्रीवाय सहस्रा॒क्षाय॑ । नील॑ग्रीवा॒येति॒ नील॑ - ग्री॒वा॒य॒ । स॒ह॒स्रा॒क्षाय॑ मी॒ढुषे᳚ । स॒ह॒स्रा॒क्षायेति॑ सहस्र - अ॒क्षाय॑ । मी॒ढुष॒ इति॑ मी॒ढुषे᳚ ॥ अथो॒ ये । अथो॒ इत्यथो᳚ । ये अ॑स्य । अ॒स्य॒ सत्वा॑नः । सत्वा॑नो॒ऽहम् । अ॒हम् तेभ्यः॑ । तेभ्यो॑ऽकरम् । अ॒क॒र॒म् नमः॑ । नम॒ इति॒ नमः॑ ॥ प्र मु॑ञ्च । मु॒ञ्च॒ धन्व॑नः । धन्व॑न॒स्त्वम् । त्वमु॒भयोः᳚ । उ॒भयो॒रात्नि॑योः । आर्त्नि॑यो॒र्ज्याम् । ज्यामिति॒ज्याम् ॥ याश्च॑ । च॒ ते॒ । ते॒ हस्ते᳚ । हस्त॒ इष॑वः । इष॑वः॒ परा᳚ \newline

\textbf{Jatai Paata} \newline

1. श्रि॒ताः स॑हस्र॒शः स॑हस्र॒शः श्रि॒ताः श्रि॒ताः स॑हस्र॒शः । \newline
2. स॒ह॒स्र॒शो ऽवाव॑ सहस्र॒शः स॑हस्र॒शो ऽव॑ । \newline
3. स॒ह॒स्र॒श इति॑ सहस्र - शः । \newline
4. अवै॑षा मेषा॒ मवा वै॑षाम् । \newline
5. ए॒षा॒(ग्म्॒) हेडो॒ हेड॑ एषा मेषा॒(ग्म्॒) हेडः॑ । \newline
6. हेड॑ ईमह ईमहे॒ हेडो॒ हेड॑ ईमहे । \newline
7. ई॒म॒ह॒ इती॑महे । \newline
8. अ॒सौ यो यो अ॒सा व॒सौ यः । \newline
9. यो॑ ऽव॒सर्प॑ त्यव॒सर्प॑ति॒ यो यो॑ ऽव॒सर्प॑ति । \newline
10. अ॒व॒सर्प॑ति॒ नील॑ग्रीवो॒ नील॑ग्रीवो ऽव॒सर्प॑ त्यव॒सर्प॑ति॒ नील॑ग्रीवः । \newline
11. अ॒व॒सर्प॒तीत्य॑व - सर्प॑ति । \newline
12. नील॑ग्रीवो॒ विलो॑हितो॒ विलो॑हितो॒ नील॑ग्रीवो॒ नील॑ग्रीवो॒ विलो॑हितः । \newline
13. नील॑ग्रीव॒ इति॒ नील॑ - ग्री॒वः॒ । \newline
14. विलो॑हित॒ इति॒ वि - लो॒हि॒तः॒ । \newline
15. उ॒तैन॑ मेन मु॒तो तैन᳚म् । \newline
16. ए॒न॒म् गो॒पा गो॒पा ए॑न मेनम् गो॒पाः । \newline
17. गो॒पा अ॑दृशन् नदृशन् गो॒पा गो॒पा अ॑दृशन्न् । \newline
18. गो॒पा इति॑ गो - पाः । \newline
19. अ॒दृ॒श॒न् नदृ॑श॒न् नदृ॑शन् नदृशन् नदृश॒न् नदृ॑शन्न् । \newline
20. अदृ॑शन् नुदहा॒र्य॑ उदहा॒र्यो॑ अदृ॑श॒न् नदृ॑शन् नुदहा॒र्यः॑ । \newline
21. उ॒द॒हा॒र्य॑ इत्यु॑द - हा॒र्यः॑ । \newline
22. उ॒तैन॑ मेन मु॒तो तैन᳚म् । \newline
23. ए॒नं॒ ॅविश्वा॒ विश्वै॑न मेनं॒ ॅविश्वा᳚ । \newline
24. विश्वा॑ भू॒तानि॑ भू॒तानि॒ विश्वा॒ विश्वा॑ भू॒तानि॑ । \newline
25. भू॒तानि॒ स स भू॒तानि॑ भू॒तानि॒ सः । \newline
26. स दृ॒ष्टो दृ॒ष्टः स स दृ॒ष्टः । \newline
27. दृ॒ष्टो मृ॑डयाति मृडयाति दृ॒ष्टो दृ॒ष्टो मृ॑डयाति । \newline
28. मृ॒ड॒या॒ति॒ नो॒ नो॒ मृ॒ड॒या॒ति॒ मृ॒ड॒या॒ति॒ नः॒ । \newline
29. न॒ इति॑ नः । \newline
30. नमो॑ अस्त्वस्तु॒ नमो॒ नमो॑ अस्तु । \newline
31. अ॒स्तु॒ नील॑ग्रीवाय॒ नील॑ग्रीवाया स्त्वस्तु॒ नील॑ग्रीवाय । \newline
32. नील॑ग्रीवाय सहस्रा॒क्षाय॑ सहस्रा॒क्षाय॒ नील॑ग्रीवाय॒ नील॑ग्रीवाय सहस्रा॒क्षाय॑ । \newline
33. नील॑ग्रीवा॒येति॒ नील॑ - ग्री॒वा॒य॒ । \newline
34. स॒ह॒स्रा॒क्षाय॑ मी॒ढुषे॑ मी॒ढुषे॑ सहस्रा॒क्षाय॑ सहस्रा॒क्षाय॑ मी॒ढुषे᳚ । \newline
35. स॒ह॒स्रा॒क्षायेति॑ सहस्र - अ॒क्षाय॑ । \newline
36. मी॒ढुष॒ इति॑ मी॒ढुषे᳚ । \newline
37. अथो॒ ये ये ऽथो॒ अथो॒ ये । \newline
38. अथो॒ इत्यथो᳚ । \newline
39. ये अ॑स्यास्य॒ ये ये अ॑स्य । \newline
40. अ॒स्य॒ सत्वा॑नः॒ सत्वा॑नो अस्यास्य॒ सत्वा॑नः । \newline
41. सत्वा॑नो॒ ऽह म॒हꣳ सत्वा॑नः॒ सत्वा॑नो॒ ऽहम् । \newline
42. अ॒हम् तेभ्य॒ स्तेभ्यो॒ ऽह म॒हम् तेभ्यः॑ । \newline
43. तेभ्यो॑ ऽकर मकर॒म् तेभ्य॒ स्तेभ्यो॑ ऽकरम् । \newline
44. अ॒क॒र॒म् नमो॒ नमो॑ ऽकर मकर॒म् नमः॑ । \newline
45. नम॒ इति॒ नमः॑ । \newline
46. प्र मु॑ञ्च मुञ्च॒ प्र प्र मु॑ञ्च । \newline
47. मु॒ञ्च॒ धन्व॑नो॒ धन्व॑नो मुञ्च मुञ्च॒ धन्व॑नः । \newline
48. धन्व॑न॒ स्त्वम् त्वम् धन्व॑नो॒ धन्व॑न॒ स्त्वम् । \newline
49. त्व मु॒भयो॑ रु॒भयो॒ स्त्वम् त्व मु॒भयोः᳚ । \newline
50. उ॒भयो॒ रार्त्नि॑यो॒ रार्त्नि॑यो रु॒भयो॑ रु॒भयो॒ रार्त्नि॑योः । \newline
51. आर्त्नि॑यो॒र् ज्याम् ज्या मार्त्नि॑यो॒ रार्त्नि॑यो॒र् ज्याम् । \newline
52. ज्यामिति॒ ज्याम् । \newline
53. याश्च॑ च॒ या याश्च॑ । \newline
54. च॒ ते॒ ते॒ च॒ च॒ ते॒ । \newline
55. ते॒ हस्ते॒ हस्ते॑ ते ते॒ हस्ते᳚ । \newline
56. हस्त॒ इष॑व॒ इष॑वो॒ हस्ते॒ हस्त॒ इष॑वः । \newline
57. इष॑वः॒ परा॒ परेष॑व॒ इष॑वः॒ परा᳚ । \newline

\textbf{Ghana Paata } \newline

1. श्रि॒ताः स॑हस्र॒शः स॑हस्र॒शः श्रि॒ताः श्रि॒ताः स॑हस्र॒शो ऽवाव॑ सहस्र॒शः श्रि॒ताः श्रि॒ताः स॑हस्र॒शो ऽव॑ । \newline
2. स॒ह॒स्र॒शो ऽवाव॑ सहस्र॒शः स॑हस्र॒शो ऽवै॑षा मेषा॒ मव॑ सहस्र॒शः स॑हस्र॒शो ऽवै॑षाम् । \newline
3. स॒ह॒स्र॒श इति॑ सहस्र - शः । \newline
4. अवै॑षा मेषा॒ मवा वै॑षाꣳ॒॒ हेडो॒ हेड॑ एषा॒ मवा वै॑षाꣳ॒॒ हेडः॑ । \newline
5. ए॒षाꣳ॒॒ हेडो॒ हेड॑ एषा मेषाꣳ॒॒ हेड॑ ईमह ईमहे॒ हेड॑ एषा मेषाꣳ॒॒ हेड॑ ईमहे । \newline
6. हेड॑ ईमह ईमहे॒ हेडो॒ हेड॑ ईमहे । \newline
7. ई॒म॒ह॒ इती॑महे । \newline
8. अ॒सौ यो यो अ॒सा व॒सौ यो॑ ऽव॒सर्प॑ त्यव॒सर्प॑ति॒ यो अ॒सा व॒सौ यो॑ ऽव॒सर्प॑ति । \newline
9. यो॑ ऽव॒सर्प॑ त्यव॒सर्प॑ति॒ यो यो॑ ऽव॒सर्प॑ति॒ नील॑ग्रीवो॒ नील॑ग्रीवो ऽव॒सर्प॑ति॒ यो यो॑ ऽव॒सर्प॑ति॒ नील॑ग्रीवः । \newline
10. अ॒व॒सर्प॑ति॒ नील॑ग्रीवो॒ नील॑ग्रीवो ऽव॒सर्प॑ त्यव॒सर्प॑ति॒ नील॑ग्रीवो॒ विलो॑हितो॒ विलो॑हितो॒ नील॑ग्रीवो ऽव॒सर्प॑ त्यव॒सर्प॑ति॒ नील॑ग्रीवो॒ विलो॑हितः । \newline
11. अ॒व॒सर्प॒तीत्य॑व - सर्प॑ति । \newline
12. नील॑ग्रीवो॒ विलो॑हितो॒ विलो॑हितो॒ नील॑ग्रीवो॒ नील॑ग्रीवो॒ विलो॑हितः । \newline
13. नील॑ग्रीव॒ इति॒ नील॑ - ग्री॒वः॒ । \newline
14. विलो॑हित॒ इति॒ वि - लो॒हि॒तः॒ । \newline
15. उ॒तैन॑ मेन मु॒तोतैन॑म् गो॒पा गो॒पा ए॑न मु॒तोतैन॑म् गो॒पाः । \newline
16. ए॒न॒म् गो॒पा गो॒पा ए॑न मेनम् गो॒पा अ॑दृशन् नदृशन् गो॒पा ए॑न मेनम् गो॒पा अ॑दृशन्न् । \newline
17. गो॒पा अ॑दृशन् नदृशन् गो॒पा गो॒पा अ॑दृश॒न् नदृ॑श॒न् नदृ॑शन् नदृशन् गो॒पा गो॒पा अ॑दृश॒न् नदृ॑शन्न् । \newline
18. गो॒पा इति॑ गो - पाः । \newline
19. अ॒दृ॒श॒न् नदृ॑श॒न् नदृ॑शन् नदृशन् नदृश॒न् नदृ॑शन् नुदहा॒र्य॑ उदहा॒र्यो॑ अदृ॑शन् नदृशन् नदृश॒न् नदृ॑शन् नुदहा॒र्यः॑ । \newline
20. अदृ॑शन् नुदहा॒र्य॑ उदहा॒र्यो॑ अदृ॑श॒न् नदृ॑शन् नुदहा॒र्यः॑ । \newline
21. उ॒द॒हा॒र्य॑ इत्यु॑द - हा॒र्यः॑ । \newline
22. उ॒तैन॑ मेन मु॒तोतैनं॒ ॅविश्वा॒ विश्वै॑न मु॒तोतैनं॒ ॅविश्वा᳚ । \newline
23. ए॒नं॒ ॅविश्वा॒ विश्वै॑न मेनं॒ ॅविश्वा॑ भू॒तानि॑ भू॒तानि॒ विश्वै॑न मेनं॒ ॅविश्वा॑ भू॒तानि॑ । \newline
24. विश्वा॑ भू॒तानि॑ भू॒तानि॒ विश्वा॒ विश्वा॑ भू॒तानि॒ स स भू॒तानि॒ विश्वा॒ विश्वा॑ भू॒तानि॒ सः । \newline
25. भू॒तानि॒ स स भू॒तानि॑ भू॒तानि॒ स दृ॒ष्टो दृ॒ष्टः स भू॒तानि॑ भू॒तानि॒ स दृ॒ष्टः । \newline
26. स दृ॒ष्टो दृ॒ष्टः स स दृ॒ष्टो मृ॑डयाति मृडयाति दृ॒ष्टः स स दृ॒ष्टो मृ॑डयाति । \newline
27. दृ॒ष्टो मृ॑डयाति मृडयाति दृ॒ष्टो दृ॒ष्टो मृ॑डयाति नो नो मृडयाति दृ॒ष्टो दृ॒ष्टो मृ॑डयाति नः । \newline
28. मृ॒ड॒या॒ति॒ नो॒ नो॒ मृ॒ड॒या॒ति॒ मृ॒ड॒या॒ति॒ नः॒ । \newline
29. न॒ इति॑ नः । \newline
30. नमो॑ अस्त्वस्तु॒ नमो॒ नमो॑ अस्तु॒ नील॑ग्रीवाय॒ नील॑ग्रीवायास्तु॒ नमो॒ नमो॑ अस्तु॒ नील॑ग्रीवाय । \newline
31. अ॒स्तु॒ नील॑ग्रीवाय॒ नील॑ग्रीवायास्त्वस्तु॒ नील॑ग्रीवाय सहस्रा॒क्षाय॑ सहस्रा॒क्षाय॒ नील॑ग्रीवायास्त्वस्तु॒ नील॑ग्रीवाय सहस्रा॒क्षाय॑ । \newline
32. नील॑ग्रीवाय सहस्रा॒क्षाय॑ सहस्रा॒क्षाय॒ नील॑ग्रीवाय॒ नील॑ग्रीवाय सहस्रा॒क्षाय॑ मी॒ढुषे॑ मी॒ढुषे॑ सहस्रा॒क्षाय॒ नील॑ग्रीवाय॒ नील॑ग्रीवाय सहस्रा॒क्षाय॑ मी॒ढुषे᳚ । \newline
33. नील॑ग्रीवा॒येति॒ नील॑ - ग्री॒वा॒य॒ । \newline
34. स॒ह॒स्रा॒क्षाय॑ मी॒ढुषे॑ मी॒ढुषे॑ सहस्रा॒क्षाय॑ सहस्रा॒क्षाय॑ मी॒ढुषे᳚ । \newline
35. स॒ह॒स्रा॒क्षायेति॑ सहस्र - अ॒क्षाय॑ । \newline
36. मी॒ढुष॒ इति॑ मी॒ढुषे᳚ । \newline
37. अथो॒ ये ये ऽथो॒ अथो॒ ये अ॑स्यास्य॒ ये ऽथो॒ अथो॒ ये अ॑स्य । \newline
38. अथो॒ इत्यथो᳚ । \newline
39. ये अ॑स्यास्य॒ ये ये अ॑स्य॒ सत्वा॑नः॒ सत्वा॑नो अस्य॒ ये ये अ॑स्य॒ सत्वा॑नः । \newline
40. अ॒स्य॒ सत्वा॑नः॒ सत्वा॑नो अस्यास्य॒ सत्वा॑नो॒ ऽह म॒हꣳ सत्वा॑नो अस्यास्य॒ सत्वा॑नो॒ ऽहम् । \newline
41. सत्वा॑नो॒ ऽह म॒हꣳ सत्वा॑नः॒ सत्वा॑नो॒ ऽहम् तेभ्य॒ स्तेभ्यो॒ ऽहꣳ सत्वा॑नः॒ सत्वा॑नो॒ ऽहम् तेभ्यः॑ । \newline
42. अ॒हम् तेभ्य॒ स्तेभ्यो॒ ऽह म॒हम् तेभ्यो॑ ऽकर मकर॒म् तेभ्यो॒ ऽह म॒हम् तेभ्यो॑ ऽकरम् । \newline
43. तेभ्यो॑ ऽकर मकर॒म् तेभ्य॒ स्तेभ्यो॑ ऽकर॒म् नमो॒ नमो॑ ऽकर॒म् तेभ्य॒ स्तेभ्यो॑ ऽकर॒म् नमः॑ । \newline
44. अ॒क॒र॒म् नमो॒ नमो॑ ऽकर मकर॒म् नमः॑ । \newline
45. नम॒ इति॒ नमः॑ । \newline
46. प्र मु॑ञ्च मुञ्च॒ प्र प्र मु॑ञ्च॒ धन्व॑नो॒ धन्व॑नो मुञ्च॒ प्र प्र मु॑ञ्च॒ धन्व॑नः । \newline
47. मु॒ञ्च॒ धन्व॑नो॒ धन्व॑नो मुञ्च मुञ्च॒ धन्व॑न॒ स्त्वम् त्वम् धन्व॑नो मुञ्च मुञ्च॒ धन्व॑न॒ स्त्वम् । \newline
48. धन्व॑न॒ स्त्वम् त्वम् धन्व॑नो॒ धन्व॑न॒ स्त्व मु॒भयो॑ रु॒भयो॒ स्त्वम् धन्व॑नो॒ धन्व॑न॒ स्त्व मु॒भयोः᳚ । \newline
49. त्व मु॒भयो॑ रु॒भयो॒ स्त्वम् त्व मु॒भयो॒ रार्त्नि॑यो॒ रार्त्नि॑यो रु॒भयो॒ स्त्वम् त्व मु॒भयो॒ रार्त्नि॑योः । \newline
50. उ॒भयो॒ रार्त्नि॑यो॒ रार्त्नि॑यो रु॒भयो॑ रु॒भयो॒ रार्त्नि॑यो॒र् ज्याम् ज्या मार्त्नि॑यो रु॒भयो॑ रु॒भयो॒ रार्त्नि॑यो॒र् ज्याम् । \newline
51. आर्त्नि॑यो॒र् ज्याम् ज्या मार्त्नि॑यो॒ रार्त्नि॑यो॒र् ज्याम् । \newline
52. ज्यामिति॒ ज्याम् । \newline
53. याश्च॑ च॒ या याश्च॑ ते ते च॒ या याश्च॑ ते । \newline
54. च॒ ते॒ ते॒ च॒ च॒ ते॒ हस्ते॒ हस्ते॑ ते च च ते॒ हस्ते᳚ । \newline
55. ते॒ हस्ते॒ हस्ते॑ ते ते॒ हस्त॒ इष॑व॒ इष॑वो॒ हस्ते॑ ते ते॒ हस्त॒ इष॑वः । \newline
56. हस्त॒ इष॑व॒ इष॑वो॒ हस्ते॒ हस्त॒ इष॑वः॒ परा॒ परे ष॑वो॒ हस्ते॒ हस्त॒ इष॑वः॒ परा᳚ । \newline
57. इष॑वः॒ परा॒ परे ष॑व॒ इष॑वः॒ परा॒ तास्ताः परे ष॑व॒ इष॑वः॒ परा॒ ताः । \newline
\pagebreak
\markright{ TS 4.5.1.4  \hfill https://www.vedavms.in \hfill}

\section{ TS 4.5.1.4 }

\textbf{TS 4.5.1.4 } \newline
\textbf{Samhita Paata} \newline

परा॒ ता भ॑गवो वप ॥                                        अ॒व॒तत्य॒ धनु॒स्त्वꣳ सह॑स्राक्ष॒ शते॑षुधे । नि॒शीर्य॑ श॒ल्यानां॒ मुखा॑ शि॒वो नः॑ सु॒मना॑ भव ॥                          विज्यं॒ धनुः॑ कप॒र्दिनो॒ विश॑ल्यो॒ बाण॑वाꣳ उ॒त । अने॑शन्न॒स्येष॑व आ॒भुर॑स्य निष॒ङ्गथिः॑ ॥                                               या ते॑ हे॒ति-र्मी॑ढुष्टम॒ हस्ते॑ ब॒भूव॑ ते॒ धनुः॑ । तया॒ऽस्मान्. वि॒श्वत॒ स्त्वम॑य॒क्ष्मया॒ परि॑ब्भुज ॥                                  नम॑स्ते अ॒स्त्वायु॑धा॒या-ना॑तताय धृ॒ष्णवे᳚ । उ॒भाभ्या॑ ( ) मु॒त ते॒ नमो॑ बा॒हुभ्यां॒ तव॒ धन्व॑ने ॥                      परि॑ ते॒ धन्व॑नो हे॒तिर॒स्मान्-वृ॑णक्तु वि॒श्वतः॑ । अथो॒ य इ॑षु॒धिस्तवा॒रे अ॒स्म-न्निधे॑हि॒ तं ॥ \newline

\textbf{Pada Paata} \newline

परेति॑ । ताः । भ॒ग॒व॒ इति॑ भग - वः॒ । व॒प॒ ॥ अ॒व॒तत्येत्य॑व - तत्य॑ । धनुः॑ । त्वम् । सह॑स्रा॒क्षेति॒ सह॑स्र - अ॒क्ष॒ । शते॑षुध॒ इति॒ शत॑ - इ॒षु॒धे॒ ॥ नि॒शीर्येति॑ नि-शीर्य॑ । श॒ल्याना᳚म् । मुखा᳚ । शि॒वः । नः॒ । सु॒मना॒ इति॑ सु - मनाः᳚ । भ॒व॒ ॥ विज्य॒मिति॒ वि - ज्य॒म् । धनुः॑ । क॒प॒र्दिनः॑ । विश॑ल्य॒ इति॒ वि - श॒ल्यः॒ । बाण॑वानिति॒ बाण॑ - वा॒न् । उ॒त ॥ अने॑शन्न् । अ॒स्य॒ । इष॑वः । आ॒भुः । अ॒स्य॒ । नि॒ष॒ङ्गथिः॑ ॥ या । ते॒ । हे॒तिः । मी॒ढु॒ष्ट॒मेति॑ मीढुः - त॒म॒ । हस्ते᳚ । ब॒भूव॑ । ते॒ । धनुः॑ ॥ तया᳚ । अ॒स्मान् । वि॒श्वतः॑ । त्वम् । अ॒य॒क्ष्मया᳚ । परीति॑ । भु॒ज॒ ॥ नमः॑ । ते॒ । अ॒स्तु॒ । आयु॑धाय । अना॑तता॒येत्यना᳚ - त॒ता॒य॒ । धृ॒ष्णवे᳚ ॥ उ॒भाभ्या᳚म् ( ) । उ॒त । ते॒ । नमः॑ । बा॒हुभ्या॒मिति॑ बा॒हु - भ्या॒म् । तव॑ । धन्व॑ने ॥ परीति॑ । ते॒ । धन्व॑नः । हे॒तिः । अ॒स्मान् । वृ॒ण॒क्तु॒ । वि॒श्वतः॑ ॥ अथो॒ इति॑ । यः । इ॒षु॒धिरिती॑षु-धिः । तव॑ । आ॒रे । अ॒स्मत् । नीति॑ । धे॒हि॒ । तम् ॥  \newline


\textbf{Krama Paata} \newline

परा॒ ताः । ता भ॑गवः । भ॒ग॒वो॒ व॒प॒ । भ॒ग॒व॒ इति॑ भग - वः॒ । व॒पेति॑ वप ॥ अ॒व॒तत्य॒ धनुः॑ । अ॒व॒तत्येत्य॑व - तत्य॑ । धनु॒स्त्वम् । त्वꣳ सह॑स्राक्ष । सह॑स्राक्ष॒ शते॑षुधे । सह॑स्रा॒क्षेति॒ सह॑स्र - अ॒क्ष॒ । शते॑षुध॒ इति॒ शत॑ - इ॒षु॒धे॒ ॥ नि॒शीर्य॑ श॒ल्याना᳚म् । नि॒शीर्येति॑ नि - शीर्य॑ । श॒ल्याना॒म् मुखा᳚ । मुखा॑ शि॒वः । शि॒वो नः॑ । नः॒ सु॒मनाः᳚ । सु॒मना॑ भव । सु॒मना॒ इति॑ सु - मनाः᳚ । भ॒वेति॑ भव ॥ विज्य॒म् धनुः॑ । विज्य॒मिति॒ वि - ज्य॒म् । धनुः॑ कप॒र्दिनः॑ । क॒प॒र्दिनो॒ विश॑ल्यः । विश॑ल्यो॒ बाण॑वान् । विश॑ल्य॒ इति॒ वि - श॒ल्यः॒ । बाण॑वाꣳ उ॒त । बाण॑वा॒निति॒ बाण॑ - वा॒न्॒ । उ॒तेत्यु॒त ॥ अने॑शन्नस्य । अ॒स्येष॑वः । इष॑वः आ॒भुः । आ॒भुर॑स्य । अ॒स्य॒ नि॒ष॒ङ्गथिः॑ । नि॒ष॒ङ्गथि॒रिति॑ निष॒ङ्गथिः॑ ॥ या ते᳚ । ते॒ हे॒तिः । हे॒तिर् मी॑ढुष्टम । मी॒ढु॒ष्ट॒म॒ हस्ते᳚ । मी॒ढु॒ष्ट॒मेति॑ मीढुः - त॒म॒ । हस्ते॑ ब॒भूव॑ । ब॒भूव॑ ते । ते॒ धनुः॑ । धनु॒रिति॒ धनुः॑ ॥ तया॒ऽस्मान् । अ॒स्मान्. वि॒श्वतः॑ । वि॒श्वत॒स्त्वम् । त्वम॑य॒क्ष्मया᳚ । अ॒य॒क्ष्मया॒ परि॑ । परि॑ब्भुज । भु॒जेति॑ भुज ॥ नम॑स्ते । ते॒ अ॒स्तु॒ । अ॒स्त्वायु॑धाय । आयु॑धा॒याना॑तताय । अना॑तताय धृ॒ष्णवे᳚ । अना॑तता॒येत्यना᳚ - त॒ता॒य॒ । धृ॒ष्णव॒ इति॑ धृ॒ष्णवे᳚ ॥ उ॒भाभ्या॑मु॒त ( ) । उ॒त ते᳚ । ते॒ नमः॑ । नमो॑ बा॒हुभ्या᳚म् । बा॒हुभ्या॒म् तव॑ । बा॒हुभ्या॒मिति॑ बा॒हु - भ्या॒म् । तव॒ धन्व॑ने । धन्व॑न॒ इति॒ धन्व॑ने ॥ परि॑ ते । ते॒ धन्व॑नः । धन्व॑नो हे॒तिः । हे॒तिर॒स्मान् । अ॒स्मान् वृ॑णक्तु । वृ॒ण॒क्तु॒ वि॒श्वतः॑ । वि॒श्वत॒ इति॑ वि॒श्वतः॑ ॥ अथो॒ यः । अथो॒ इत्यथो᳚ । य इ॑षु॒धिः । इ॒षु॒धिस्तव॑ । इ॒षु॒धिरिती॑षु - धिः । तवा॒रे । आ॒रे अ॒स्मत् । अ॒स्मन् नि । नि धे॑हि । धे॒हि॒ तम् । तमिति॒ तम् । \newline

\textbf{Jatai Paata} \newline

1. परा॒ ता स्ताः परा॒ परा॒ ताः । \newline
2. ता भ॑गवो भगव॒ स्ता स्ता भ॑गवः । \newline
3. भ॒ग॒वो॒ व॒प॒ व॒प॒ भ॒ग॒वो॒ भ॒ग॒वो॒ व॒प॒ । \newline
4. भ॒ग॒व॒ इति॑ भग - वः॒ । \newline
5. व॒पेति॑ वप । \newline
6. अ॒व॒तत्य॒ धनु॒र् धनु॑ रव॒तत्या॑ व॒तत्य॒ धनुः॑ । \newline
7. अ॒व॒तत्येत्य॑व - तत्य॑ । \newline
8. धनु॒ स्त्वम् त्वम् धनु॒र् धनु॒ स्त्वम् । \newline
9. त्वꣳ सह॑स्राक्ष॒ सह॑स्राक्ष॒ त्वम् त्वꣳ सह॑स्राक्ष । \newline
10. सह॑स्राक्ष॒ शते॑षुधे॒ शते॑षुधे॒ सह॑स्राक्ष॒ सह॑स्राक्ष॒ शते॑षुधे । \newline
11. सह॑स्रा॒क्षेति॒ सह॑स्र - अ॒क्ष॒ । \newline
12. शते॑षुध॒ इति॒ शत॑ - इ॒षु॒धे॒ । \newline
13. नि॒शीर्य॑ श॒ल्याना(ग्म्॑) श॒ल्याना᳚म् नि॒शीर्य॑ नि॒शीर्य॑ श॒ल्याना᳚म् । \newline
14. नि॒शीर्येति॑ नि - शीर्य॑ । \newline
15. श॒ल्याना॒म् मुखा॒ मुखा॑ श॒ल्याना(ग्म्॑) श॒ल्याना॒म् मुखा᳚ । \newline
16. मुखा॑ शि॒वः शि॒वो मुखा॒ मुखा॑ शि॒वः । \newline
17. शि॒वो नो॑ नः शि॒वः शि॒वो नः॑ । \newline
18. नः॒ सु॒मनाः᳚ सु॒मना॑ नो नः सु॒मनाः᳚ । \newline
19. सु॒मना॑ भव भव सु॒मनाः᳚ सु॒मना॑ भव । \newline
20. सु॒मना॒ इति॑ सु - मनाः᳚ । \newline
21. भ॒वेति॑ भव । \newline
22. विज्य॒म् धनु॒र् धनु॒र् विज्यं॒ ॅविज्य॒म् धनुः॑ । \newline
23. विज्य॒मिति॒ वि - ज्य॒म् । \newline
24. धनुः॑ कप॒र्दिनः॑ कप॒र्दिनो॒ धनु॒र् धनुः॑ कप॒र्दिनः॑ । \newline
25. क॒प॒र्दिनो॒ विश॑ल्यो॒ विश॑ल्यः कप॒र्दिनः॑ कप॒र्दिनो॒ विश॑ल्यः । \newline
26. विश॑ल्यो॒ बाण॑वा॒न् बाण॑वा॒न्॒. विश॑ल्यो॒ विश॑ल्यो॒ बाण॑वान् । \newline
27. विश॑ल्य॒ इति॒ वि - श॒ल्यः॒ । \newline
28. बाण॑वाꣳ उ॒तोत बाण॑वा॒न् बाण॑वाꣳ उ॒त । \newline
29. बाण॑वा॒निति॒ बाण॑ - वा॒न् । \newline
30. उ॒तेत्यु॒त । \newline
31. अने॑शन् नस्या॒ स्याने॑श॒न् नने॑शन् नस्य । \newline
32. अ॒स्येष॑व॒ इष॑वो अस्या॒ स्येष॑वः । \newline
33. इष॑व आ॒भु रा॒भु रिष॑व॒ इष॑व आ॒भुः । \newline
34. आ॒भु र॑स्यास्या॒भु रा॒भु र॑स्य । \newline
35. अ॒स्य॒ नि॒ष॒ङ्गथि॑र् निष॒ङ्गथि॑ रस्यास्य निष॒ङ्गथिः॑ । \newline
36. नि॒ष॒ङ्गथि॒रिति॑ निष॒ङ्गथिः॑ । \newline
37. या ते॑ ते॒ या या ते᳚ । \newline
38. ते॒ हे॒तिर्. हे॒ति स्ते॑ ते हे॒तिः । \newline
39. हे॒तिर् मी॑ढुष्टम मीढुष्टम हे॒तिर्. हे॒तिर् मी॑ढुष्टम । \newline
40. मी॒ढु॒ष्ट॒म॒ हस्ते॒ हस्ते॑ मीढुष्टम मीढुष्टम॒ हस्ते᳚ । \newline
41. मी॒ढु॒ष्ट॒मेति॑ मीढुः - त॒म॒ । \newline
42. हस्ते॑ ब॒भूव॑ ब॒भूव॒ हस्ते॒ हस्ते॑ ब॒भूव॑ । \newline
43. ब॒भूव॑ ते ते ब॒भूव॑ ब॒भूव॑ ते । \newline
44. ते॒ धनु॒र् धनु॑ स्ते ते॒ धनुः॑ । \newline
45. धनु॒रिति॒ धनुः॑ । \newline
46. तया॒ ऽस्मा न॒स्मान् तया॒ तया॒ ऽस्मान् । \newline
47. अ॒स्मान्. वि॒श्वतो॑ वि॒श्वतो॑ अ॒स्मा न॒स्मान्. वि॒श्वतः॑ । \newline
48. वि॒श्वत॒ स्त्वम् त्वं ॅवि॒श्वतो॑ वि॒श्वत॒ स्त्वम् । \newline
49. त्व म॑य॒क्ष्मया॑ ऽय॒क्ष्मया॒ त्वम् त्व म॑य॒क्ष्मया᳚ । \newline
50. अ॒य॒क्ष्मया॒ परि॒ पर्य॑य॒क्ष्मया॑ ऽय॒क्ष्मया॒ परि॑ । \newline
51. परि॑ब्भुज भुज॒ परि॒ परि॑ब्भुज । \newline
52. भु॒जेति॑ भुज । \newline
53. नम॑ स्ते ते॒ नमो॒ नम॑ स्ते । \newline
54. ते॒ अ॒स्त्व॒स्तु॒ ते॒ ते॒ अ॒स्तु॒ । \newline
55. अ॒स्त्वा यु॑धा॒या यु॑धाया स्त्व॒ स्त्वायु॑धाय । \newline
56. आयु॑धा॒या ना॑तता॒या ना॑तता॒या यु॑धा॒या यु॑धा॒या ना॑तताय । \newline
57. अना॑तताय धृ॒ष्णवे॑ धृ॒ष्णवे ऽना॑तता॒या ना॑तताय धृ॒ष्णवे᳚ । \newline
58. अना॑तता॒येत्यना᳚ - त॒ता॒य॒ । \newline
59. धृ॒ष्णव॒ इति॑ धृ॒ष्णवे᳚ । \newline
60. उ॒भाभ्या॑ मु॒तोतोभाभ्या॑ मु॒भाभ्या॑ मु॒त । \newline
61. उ॒त ते॑ त उ॒तोत ते᳚ । \newline
62. ते॒ नमो॒ नम॑ स्ते ते॒ नमः॑ । \newline
63. नमो॑ बा॒हुभ्या᳚म् बा॒हुभ्या॒म् नमो॒ नमो॑ बा॒हुभ्या᳚म् । \newline
64. बा॒हुभ्या॒म् तव॒ तव॑ बा॒हुभ्या᳚म् बा॒हुभ्या॒म् तव॑ । \newline
65. बा॒हुभ्या॒मिति॑ बा॒हु - भ्या॒म् । \newline
66. तव॒ धन्व॑ने॒ धन्व॑ने॒ तव॒ तव॒ धन्व॑ने । \newline
67. धन्व॑न॒ इति॒ धन्व॑ने । \newline
68. परि॑ ते ते॒ परि॒ परि॑ ते । \newline
69. ते॒ धन्व॑नो॒ धन्व॑न स्ते ते॒ धन्व॑नः । \newline
70. धन्व॑नो हे॒तिर्. हे॒तिर् धन्व॑नो॒ धन्व॑नो हे॒तिः । \newline
71. हे॒ति र॒स्मा न॒स्मान्. हे॒तिर्. हे॒ति र॒स्मान् । \newline
72. अ॒स्मान् वृ॑णक्तु वृणक्त्व॒स्मा न॒स्मान् वृ॑णक्तु । \newline
73. वृ॒ण॒क्तु॒ वि॒श्वतो॑ वि॒श्वतो॑ वृणक्तु वृणक्तु वि॒श्वतः॑ । \newline
74. वि॒श्वत॒ इति॑ वि॒श्वतः॑ । \newline
75. अथो॒ यो यो ऽथो॒ अथो॒ यः । \newline
76. अथो॒ इत्यथो᳚ । \newline
77. य इ॑षु॒धि रि॑षु॒धिर् यो य इ॑षु॒धिः । \newline
78. इ॒षु॒धि स्तव॒ तवे॑षु॒धि रि॑षु॒धि स्तव॑ । \newline
79. इ॒षु॒धिरिती॑षु - धिः । \newline
80. तवा॒र आ॒रे तव॒ तवा॒रे । \newline
81. आ॒रे अ॒स्म द॒स्म दा॒र आ॒रे अ॒स्मत् । \newline
82. अ॒स्मन् नि न्य॑स्म द॒स्मन् नि । \newline
83. नि धे॑हि धेहि॒ नि नि धे॑हि । \newline
84. धे॒हि॒ तम् तम् धे॑हि धेहि॒ तम् । \newline
85. तमिति॒ तम् । \newline

\textbf{Ghana Paata } \newline

1. परा॒ ता स्ताः परा॒ परा॒ ता भ॑गवो भगव॒ स्ताः परा॒ परा॒ ता भ॑गवः । \newline
2. ता भ॑गवो भगव॒ स्ता स्ता भ॑गवो वप वप भगव॒ स्ता स्ता भ॑गवो वप । \newline
3. भ॒ग॒वो॒ व॒प॒ व॒प॒ भ॒ग॒वो॒ भ॒ग॒वो॒ व॒प॒ । \newline
4. भ॒ग॒व॒ इति॑ भग - वः॒ । \newline
5. व॒पेति॑ वप । \newline
6. अ॒व॒तत्य॒ धनु॒र् धनु॑ रव॒तत्या॑ व॒तत्य॒ धनु॒ स्त्वम् त्वम् धनु॑ रव॒तत्या॑ व॒तत्य॒ धनु॒ स्त्वम् । \newline
7. अ॒व॒तत्येत्य॑व - तत्य॑ । \newline
8. धनु॒ स्त्वम् त्वम् धनु॒र् धनु॒ स्त्वꣳ सह॑स्राक्ष॒ सह॑स्राक्ष॒ त्वम् धनु॒र् धनु॒ स्त्वꣳ सह॑स्राक्ष । \newline
9. त्वꣳ सह॑स्राक्ष॒ सह॑स्राक्ष॒ त्वम् त्वꣳ सह॑स्राक्ष॒ शते॑षुधे॒ शते॑षुधे॒ सह॑स्राक्ष॒ त्वम् 
त्वꣳ सह॑स्राक्ष॒ शते॑षुधे । \newline
10. सह॑स्राक्ष॒ शते॑षुधे॒ शते॑षुधे॒ सह॑स्राक्ष॒ सह॑स्राक्ष॒ शते॑षुधे । \newline
11. सह॑स्रा॒क्षेति॒ सह॑स्र - अ॒क्ष॒ । \newline
12. शते॑षुध॒ इति॒ शत॑ - इ॒षु॒धे॒ । \newline
13. नि॒शीर्य॑ श॒ल्यानाꣳ॑ श॒ल्याना᳚म् नि॒शीर्य॑ नि॒शीर्य॑ श॒ल्याना॒म् मुखा॒ मुखा॑ श॒ल्याना᳚म् नि॒शीर्य॑ नि॒शीर्य॑ श॒ल्याना॒म् मुखा᳚ । \newline
14. नि॒शीर्येति॑ नि - शीर्य॑ । \newline
15. श॒ल्याना॒म् मुखा॒ मुखा॑ श॒ल्यानाꣳ॑ श॒ल्याना॒म् मुखा॑ शि॒वः शि॒वो मुखा॑ श॒ल्यानाꣳ॑ श॒ल्याना॒म् मुखा॑ शि॒वः । \newline
16. मुखा॑ शि॒वः शि॒वो मुखा॒ मुखा॑ शि॒वो नो॑ नः शि॒वो मुखा॒ मुखा॑ शि॒वो नः॑ । \newline
17. शि॒वो नो॑ नः शि॒वः शि॒वो नः॑ सु॒मनाः᳚ सु॒मना॑ नः शि॒वः शि॒वो नः॑ सु॒मनाः᳚ । \newline
18. नः॒ सु॒मनाः᳚ सु॒मना॑ नो नः सु॒मना॑ भव भव सु॒मना॑ नो नः सु॒मना॑ भव । \newline
19. सु॒मना॑ भव भव सु॒मनाः᳚ सु॒मना॑ भव । \newline
20. सु॒मना॒ इति॑ सु - मनाः᳚ । \newline
21. भ॒वेति॑ भव । \newline
22. विज्य॒म् धनु॒र् धनु॒र् विज्यं॒ ॅविज्य॒म् धनुः॑ कप॒र्दिनः॑ कप॒र्दिनो॒ धनु॒र् विज्यं॒ ॅविज्य॒म् धनुः॑ कप॒र्दिनः॑ । \newline
23. विज्य॒मिति॒ वि - ज्य॒म् । \newline
24. धनुः॑ कप॒र्दिनः॑ कप॒र्दिनो॒ धनु॒र् धनुः॑ कप॒र्दिनो॒ विश॑ल्यो॒ विश॑ल्यः कप॒र्दिनो॒ धनु॒र् धनुः॑ कप॒र्दिनो॒ विश॑ल्यः । \newline
25. क॒प॒र्दिनो॒ विश॑ल्यो॒ विश॑ल्यः कप॒र्दिनः॑ कप॒र्दिनो॒ विश॑ल्यो॒ बाण॑वा॒न् बाण॑वा॒न्॒. विश॑ल्यः कप॒र्दिनः॑ कप॒र्दिनो॒ विश॑ल्यो॒ बाण॑वान् । \newline
26. विश॑ल्यो॒ बाण॑वा॒न् बाण॑वा॒न्॒. विश॑ल्यो॒ विश॑ल्यो॒ बाण॑वाꣳ उ॒तोत बाण॑वा॒न्॒. विश॑ल्यो॒ विश॑ल्यो॒ बाण॑वाꣳ उ॒त । \newline
27. विश॑ल्य॒ इति॒ वि - श॒ल्यः॒ । \newline
28. बाण॑वाꣳ उ॒तोत बाण॑वा॒न् बाण॑वाꣳ उ॒त । \newline
29. बाण॑वा॒निति॒ बाण॑ - वा॒न् । \newline
30. उ॒तेत्यु॒त । \newline
31. अने॑शन् नस्या॒स्या ने॑श॒न् नने॑शन् न॒स्ये ष॑व॒ इष॑वो अ॒स्या ने॑श॒न् नने॑शन् न॒स्ये ष॑वः । \newline
32. अ॒स्ये ष॑व॒ इष॑वो अस्या॒ स्येष॑व आ॒भु रा॒भु रिष॑वो अस्या॒ स्येष॑व आ॒भुः । \newline
33. इष॑व आ॒भु रा॒भु रिष॑व॒ इष॑व आ॒भु र॑स्या स्या॒भु रिष॑व॒ इष॑व आ॒भुर॑स्य । \newline
34. आ॒भु र॑स्यास्या॒भु रा॒भुर॑स्य निष॒ङ्गथि॑र् निष॒ङ्गथि॑ रस्या॒भु रा॒भुर॑स्य निष॒ङ्गथिः॑ । \newline
35. अ॒स्य॒ नि॒ष॒ङ्गथि॑र् निष॒ङ्गथि॑ रस्यास्य निष॒ङ्गथिः॑ । \newline
36. नि॒ष॒ङ्गथि॒रिति॑ निष॒ङ्गथिः॑ । \newline
37. या ते॑ ते॒ या या ते॑ हे॒तिर्. हे॒ति स्ते॒ या या ते॑ हे॒तिः । \newline
38. ते॒ हे॒तिर्. हे॒ति स्ते॑ ते हे॒तिर् मी॑ढुष्टम मीढुष्टम हे॒ति स्ते॑ ते हे॒तिर् मी॑ढुष्टम । \newline
39. हे॒तिर् मी॑ढुष्टम मीढुष्टम हे॒तिर्. हे॒तिर् मी॑ढुष्टम॒ हस्ते॒ हस्ते॑ मीढुष्टम हे॒तिर्. हे॒तिर् मी॑ढुष्टम॒ हस्ते᳚ । \newline
40. मी॒ढु॒ष्ट॒म॒ हस्ते॒ हस्ते॑ मीढुष्टम मीढुष्टम॒ हस्ते॑ ब॒भूव॑ ब॒भूव॒ हस्ते॑ मीढुष्टम मीढुष्टम॒ हस्ते॑ ब॒भूव॑ । \newline
41. मी॒ढु॒ष्ट॒मेति॑ मीढुः - त॒म॒ । \newline
42. हस्ते॑ ब॒भूव॑ ब॒भूव॒ हस्ते॒ हस्ते॑ ब॒भूव॑ ते ते ब॒भूव॒ हस्ते॒ हस्ते॑ ब॒भूव॑ ते । \newline
43. ब॒भूव॑ ते ते ब॒भूव॑ ब॒भूव॑ ते॒ धनु॒र् धनु॑ स्ते ब॒भूव॑ ब॒भूव॑ ते॒ धनुः॑ । \newline
44. ते॒ धनु॒र् धनु॑ स्ते ते॒ धनुः॑ । \newline
45. धनु॒रिति॒ धनुः॑ । \newline
46. तया॒ ऽस्मा न॒स्मान् तया॒ तया॒ ऽस्मान्. वि॒श्वतो॑ वि॒श्वतो॑ अ॒स्मान् तया॒ तया॒ ऽस्मान्. वि॒श्वतः॑ । \newline
47. अ॒स्मान्. वि॒श्वतो॑ वि॒श्वतो॑ अ॒स्मा न॒स्मान्. वि॒श्वत॒ स्त्वम् त्वं ॅवि॒श्वतो॑ अ॒स्मा न॒स्मान्. वि॒श्वत॒ स्त्वम् । \newline
48. वि॒श्वत॒ स्त्वम् त्वं ॅवि॒श्वतो॑ वि॒श्वत॒ स्त्व म॑य॒क्ष्मया॑ ऽय॒क्ष्मया॒ त्वं ॅवि॒श्वतो॑ वि॒श्वत॒ स्त्व म॑य॒क्ष्मया᳚ । \newline
49. त्व म॑य॒क्ष्मया॑ ऽय॒क्ष्मया॒ त्वम् त्व म॑य॒क्ष्मया॒ परि॒ पर्य॑य॒क्ष्मया॒ त्वम् त्व म॑य॒क्ष्मया॒ परि॑ । \newline
50. अ॒य॒क्ष्मया॒ परि॒ पर्य॑य॒क्ष्मया॑ ऽय॒क्ष्मया॒ परि॑ब्भुज भुज॒ पर्य॑य॒क्ष्मया॑ ऽय॒क्ष्मया॒ परि॑ब्भुज । \newline
51. परि॑ब्भुज भुज॒ परि॒ परि॑ब्भुज । \newline
52. भु॒जेति॑ भुज । \newline
53. नम॑स्ते ते॒ नमो॒ नम॑स्ते अस्त्वस्तु ते॒ नमो॒ नम॑स्ते अस्तु । \newline
54. ते॒ अ॒स्त्व॒स्तु॒ ते॒ ते॒ अ॒स्त्वा यु॑धा॒या यु॑धाया स्तु ते ते अ॒स्त्वायु॑धाय । \newline
55. अ॒स्त्वा यु॑धा॒या यु॑धाया स्त्व॒ स्त्वा यु॑धा॒या ना॑तता॒या ना॑तता॒या यु॑धाया स्त्व॒ स्त्वा यु॑धा॒या ना॑तताय । \newline
56. आयु॑धा॒या ना॑तता॒या ना॑तता॒या यु॑धा॒या यु॑धा॒या ना॑तताय धृ॒ष्णवे॑ धृ॒ष्णवे 
ऽना॑त ता॒यायु॑ धा॒यायु॑ धा॒या ना॑तताय धृ॒ष्णवे᳚ । \newline
57. अना॑तताय धृ॒ष्णवे॑ धृ॒ष्णवे ऽना॑तता॒या ना॑तताय धृ॒ष्णवे᳚ । \newline
58. अना॑तता॒येत्यना᳚ - त॒ता॒य॒ । \newline
59. धृ॒ष्णव॒ इति॑ धृ॒ष्णवे᳚ । \newline
60. उ॒भाभ्या॑ मु॒तोतो भाभ्या॑ मु॒भाभ्या॑ मु॒त ते॑ त उ॒तोभाभ्या॑ मु॒भाभ्या॑ मु॒त ते᳚ । \newline
61. उ॒त ते॑ त उ॒तोत ते॒ नमो॒ नम॑ स्त उ॒तोत ते॒ नमः॑ । \newline
62. ते॒ नमो॒ नम॑ स्ते ते॒ नमो॑ बा॒हुभ्या᳚म् बा॒हुभ्या॒म् नम॑ स्ते ते॒ नमो॑ बा॒हुभ्या᳚म् । \newline
63. नमो॑ बा॒हुभ्या᳚म् बा॒हुभ्या॒म् नमो॒ नमो॑ बा॒हुभ्या॒म् तव॒ तव॑ बा॒हुभ्या॒म् नमो॒ नमो॑ बा॒हुभ्या॒म् तव॑ । \newline
64. बा॒हुभ्या॒म् तव॒ तव॑ बा॒हुभ्या᳚म् बा॒हुभ्या॒म् तव॒ धन्व॑ने॒ धन्व॑ने॒ तव॑ बा॒हुभ्या᳚म् बा॒हुभ्या॒म् तव॒ धन्व॑ने । \newline
65. बा॒हुभ्या॒मिति॑ बा॒हु - भ्या॒म् । \newline
66. तव॒ धन्व॑ने॒ धन्व॑ने॒ तव॒ तव॒ धन्व॑ने । \newline
67. धन्व॑न॒ इति॒ धन्व॑ने । \newline
68. परि॑ ते ते॒ परि॒ परि॑ ते॒ धन्व॑नो॒ धन्व॑न स्ते॒ परि॒ परि॑ ते॒ धन्व॑नः । \newline
69. ते॒ धन्व॑नो॒ धन्व॑न स्ते ते॒ धन्व॑नो हे॒तिर्. हे॒तिर् धन्व॑न स्ते ते॒ धन्व॑नो हे॒तिः । \newline
70. धन्व॑नो हे॒तिर्. हे॒तिर् धन्व॑नो॒ धन्व॑नो हे॒ति र॒स्मा न॒स्मान्. हे॒तिर् धन्व॑नो॒ धन्व॑नो हे॒ति र॒स्मान् । \newline
71. हे॒ति र॒स्मा न॒स्मान्. हे॒तिर्. हे॒ति र॒स्मान्. वृ॑णक्तु वृणक्त्व॒स्मान्. हे॒तिर्. हे॒ति र॒स्मान् वृ॑णक्तु । \newline
72. अ॒स्मान् वृ॑णक्तु वृणक्त्व॒स्मा न॒स्मान् वृ॑णक्तु वि॒श्वतो॑ वि॒श्वतो॑ वृणक्त्व॒स्मा न॒स्मान् वृ॑णक्तु वि॒श्वतः॑ । \newline
73. वृ॒ण॒क्तु॒ वि॒श्वतो॑ वि॒श्वतो॑ वृणक्तु वृणक्तु वि॒श्वतः॑ । \newline
74. वि॒श्वत॒ इति॑ वि॒श्वतः॑ । \newline
75. अथो॒ यो यो ऽथो॒ अथो॒ य इ॑षु॒धि रि॑षु॒धिर् यो ऽथो॒ अथो॒ य इ॑षु॒धिः । \newline
76. अथो॒ इत्यथो᳚ । \newline
77. य इ॑षु॒धि रि॑षु॒धिर् यो य इ॑षु॒धि स्तव॒ तवे॑ षु॒धिर् यो य इ॑षु॒धि स्तव॑ । \newline
78. इ॒षु॒धि स्तव॒ तवे॑षु॒धि रि॑षु॒धि स्तवा॒र आ॒रे तवे॑षु॒धि रि॑षु॒धि स्तवा॒रे । \newline
79. इ॒षु॒धिरिती॑षु - धिः । \newline
80. तवा॒र आ॒रे तव॒ तवा॒रे अ॒स्म द॒स्म दा॒रे तव॒ तवा॒रे अ॒स्मत् । \newline
81. आ॒रे अ॒स्म द॒स्म दा॒र आ॒रे अ॒स्मन् नि न्य॑स्मदा॒र आ॒रे अ॒स्मन् नि । \newline
82. अ॒स्मन् नि न्य॑स्म द॒स्मन् नि धे॑हि धेहि॒ न्य॑स्म द॒स्मन् नि धे॑हि । \newline
83. नि धे॑हि धेहि॒ नि नि धे॑हि॒ तम् तम् धे॑हि॒ नि नि धे॑हि॒ तम् । \newline
84. धे॒हि॒ तम् तम् धे॑हि धेहि॒ तम् । \newline
85. तमिति॒ तम् । \newline
\pagebreak
\markright{ TS 4.5.2.1  \hfill https://www.vedavms.in \hfill}

\section{ TS 4.5.2.1 }

\textbf{TS 4.5.2.1 } \newline
\textbf{Samhita Paata} \newline

नमो॒ हिर॑ण्य बाहवे सेना॒न्ये॑ दि॒शांच॒ पत॑ये॒ नमो॒                        नमो॑ वृ॒क्षेभ्यो॒ हरि॑केशेभ्यः पशू॒नां पत॑ये॒ नमो॒                नमः॑ स॒स्पिञ्ज॑राय॒ त्विषी॑मते पथी॒नां पत॑ये॒ नमो॒                    नमो॑ बभ्लु॒शाय॑ विव्या॒धिने-ऽन्ना॑नां॒ पत॑ये॒ नमो॒                  नमो॒ हरि॑केशायो-पवी॒तिने॑ पु॒ष्टानां॒ पत॑ये॒ नमो॒                                 नमो॑ भ॒वस्य॑ हे॒त्यै जग॑तां॒ पत॑ये॒ नमो॒                                                नमो॑ रु॒द्राया॑-तता॒विने॒ क्षेत्रा॑णां॒ पत॑ये॒ नमो॒                          नमः॑ सू॒ताया-ह॑न्त्याय॒ वना॑नां॒ पत॑ये॒ नमो॒ नमो॒ - [  ] \newline

\textbf{Pada Paata} \newline

नमः॑ । हिर॑ण्यबाहव॒ इति॒ हिर॑ण्य - बा॒ह॒वे॒ । से॒ना॒न्य॑ इति॑ सेना - न्ये᳚ । दि॒शाम् । च॒ । पत॑ये । नमः॑ । नमः॑ । वृ॒क्षेभ्यः॑ । हरि॑केशेभ्य॒ इति॒ हरि॑ - के॒शे॒भ्यः॒ । प॒शू॒नाम् । पत॑ये । नमः॑ । नमः॑ । स॒स्पिञ्ज॑राय । त्विषी॑मत॒ इति॒ त्विषि॑ - म॒ते॒ । प॒थी॒नाम् । पत॑ये । नमः॑ । नमः॑ । ब॒भ्लु॒शाय॑ । वि॒व्या॒धिन॒ इति॑ वि - व्या॒धिने᳚ । अन्ना॑नाम् । पत॑ये । नमः॑ । नमः॑ । हरि॑केशा॒येति॒ हरि॑ - के॒शा॒य॒ । उ॒प॒वी॒तिन॒ इत्यु॑प - वी॒तिने᳚ । पु॒ष्टाना᳚म् । पत॑ये । नमः॑ । नमः॑ । भ॒वस्य॑ । हे॒त्यै । जग॑ताम् । पत॑ये । नमः॑ । नमः॑ । रु॒द्राय॑ । आ॒त॒ता॒विन॒ इत्या᳚ - त॒ता॒विने᳚ । क्षेत्रा॑णाम् । पत॑ये । नमः॑ । नमः॑ । सू॒ताय॑ । अह॑न्त्याय । वना॑नाम् । पत॑ये । नमः॑ । नमः॑ ।  \newline


\textbf{Krama Paata} \newline

नमो॒ हिर॑ण्यबाहवे । हिर॑ण्यबाहवे सेना॒न्ये᳚ । हिर॑ण्यबाहव॒ इति॒ हिर॑ण्य - बा॒ह॒वे॒ । से॒ना॒न्ये॑ दि॒शाम् । से॒ना॒न्य॑ इति॑ सेना - न्ये᳚ । दि॒शाञ्च॑ । च॒ पत॑ये । पत॑ये॒ नमः॑ । नमो॒ नमः॑ । नमो॑ वृ॒क्षेभ्यः॑ । वृ॒क्षेभ्यो॒ हरि॑केशेभ्यः । हरि॑केशेभ्यः पशू॒नाम् । हरि॑केशेभ्य॒ इति॒ हरि॑ - के॒शे॒भ्यः॒ । प॒शू॒नां पत॑ये । पत॑ये॒ नमः॑ । नमो॒ नमः॑ । नमः॑ स॒स्पिञ्ज॑राय । स॒स्पिञ्ज॑राय॒ त्विषी॑मते । त्विषी॑मते पथी॒नाम् । त्विषी॑मत॒ इति॒ त्विषि॑ - म॒ते॒ । प॒थी॒नां पत॑ये । पत॑ये॒ नमः॑ । नमो॒ नमः॑ । नमो॑ बभ्लु॒शाय॑ । ब॒भ्लु॒शाय॑ विव्या॒धिने᳚ । वि॒व्या॒धिनेऽन्ना॑नाम् । वि॒व्या॒धिन॒ इति॑ वि - व्या॒धिने᳚ । अन्ना॑ना॒म् पत॑ये । पत॑ये॒ नमः॑ । नमो॒ नमः॑ । नमो॒ हरि॑केशाय । हरि॑केशायोपवी॒तिने᳚ । हरि॑केशा॒येति॒ हरि॑ - के॒शा॒य॒ । उ॒प॒वी॒तिने॑ पु॒ष्टाना᳚म् । उ॒प॒वी॒तिन॒ इत्यु॑प - वी॒तिने᳚ । पु॒ष्टाना॒म् पत॑ये । पत॑ये॒ नमः॑ । नमो॒ नमः॑ । नमो॑ भ॒वस्य॑ । भ॒वस्य॑ हे॒त्यै । हे॒त्यै जग॑ताम् । जग॑ता॒म् पत॑ये । पत॑ये॒ नमः॑ । नमो॒ नमः॑ । नमो॑ रु॒द्राय॑ । रु॒द्राया॑तता॒विने᳚ । आ॒त॒ता॒विने॒ क्षेत्रा॑णाम् । आ॒त॒ता॒विन॒ इत्या᳚ - त॒ता॒विने᳚ । क्षेत्रा॑णा॒म् पत॑ये । पत॑ये॒ नमः॑ । नमो॒ नमः॑ । नमः॑ सू॒ताय॑ । सू॒तायाह॑न्त्याय । अह॑न्त्याय॒ वना॑नाम् । वना॑ना॒म् पत॑ये । पत॑ये॒ नमः॑ । नमो॒ नमः॑ ( ) । नमो॒ रोहि॑ताय \newline

\textbf{Jatai Paata} \newline

1. नमो॒ हिर॑ण्यबाहवे॒ हिर॑ण्यबाहवे॒ नमो॒ नमो॒ हिर॑ण्यबाहवे । \newline
2. हिर॑ण्यबाहवे सेना॒न्ये॑ सेना॒न्ये॑ हिर॑ण्यबाहवे॒ हिर॑ण्यबाहवे सेना॒न्ये᳚ । \newline
3. हिर॑ण्यबाहव॒ इति॒ हिर॑ण्य - बा॒ह॒वे॒ । \newline
4. से॒ना॒न्ये॑ दि॒शाम् दि॒शाꣳ से॑ना॒न्ये॑ सेना॒न्ये॑ दि॒शाम् । \newline
5. से॒ना॒न्य॑ इति॑ सेना - न्ये᳚ । \newline
6. दि॒शाम् च॑ च दि॒शाम् दि॒शाम् च॑ । \newline
7. च॒ पत॑ये॒ पत॑ये च च॒ पत॑ये । \newline
8. पत॑ये॒ नमो॒ नम॒ स्पत॑ये॒ पत॑ये॒ नमः॑ । \newline
9. नमो॒ नमः॑ । \newline
10. नमो॑ वृ॒क्षेभ्यो॑ वृ॒क्षेभ्यो॒ नमो॒ नमो॑ वृ॒क्षेभ्यः॑ । \newline
11. वृ॒क्षेभ्यो॒ हरि॑केशेभ्यो॒ हरि॑केशेभ्यो वृ॒क्षेभ्यो॑ वृ॒क्षेभ्यो॒ हरि॑केशेभ्यः । \newline
12. हरि॑केशेभ्यः पशू॒नाम् प॑शू॒नाꣳ हरि॑केशेभ्यो॒ हरि॑केशेभ्यः पशू॒नाम् । \newline
13. हरि॑केशेभ्य॒ इति॒ हरि॑ - के॒शे॒भ्यः॒ । \newline
14. प॒शू॒नाम् पत॑ये॒ पत॑ये पशू॒नाम् प॑शू॒नाम् पत॑ये । \newline
15. पत॑ये॒ नमो॒ नम॒ स्पत॑ये॒ पत॑ये॒ नमः॑ । \newline
16. नमो॒ नमः॑ । \newline
17. नमः॑ स॒स्पिञ्ज॑राय स॒स्पिञ्ज॑राय॒ नमो॒ नमः॑ स॒स्पिञ्ज॑राय । \newline
18. स॒स्पिञ्ज॑राय॒ त्विषी॑मते॒ त्विषी॑मते स॒स्पिञ्ज॑राय स॒स्पिञ्ज॑राय॒ त्विषी॑मते । \newline
19. त्विषी॑मते पथी॒नाम् प॑थी॒नाम् त्विषी॑मते॒ त्विषी॑मते पथी॒नाम् । \newline
20. त्विषी॑मत॒ इति॒ त्विषि॑ - म॒ते॒ । \newline
21. प॒थी॒नाम् पत॑ये॒ पत॑ये पथी॒नाम् प॑थी॒नाम् पत॑ये । \newline
22. पत॑ये॒ नमो॒ नम॒ स्पत॑ये॒ पत॑ये॒ नमः॑ । \newline
23. नमो॒ नमः॑ । \newline
24. नमो॑ बभ्लु॒शाय॑ बभ्लु॒शाय॒ नमो॒ नमो॑ बभ्लु॒शाय॑ । \newline
25. ब॒भ्लु॒शाय॑ विव्या॒धिने॑ विव्या॒धिने॑ बभ्लु॒शाय॑ बभ्लु॒शाय॑ विव्या॒धिने᳚ । \newline
26. वि॒व्या॒धिने ऽन्ना॑ना॒ मन्ना॑नां ॅविव्या॒धिने॑ विव्या॒धिने ऽन्ना॑नाम् । \newline
27. वि॒व्या॒धिन॒ इति॑ वि - व्या॒धिने᳚ । \newline
28. अन्ना॑ना॒म् पत॑ये॒ पत॒ये ऽन्ना॑ना॒ मन्ना॑ना॒म् पत॑ये । \newline
29. पत॑ये॒ नमो॒ नम॒ स्पत॑ये॒ पत॑ये॒ नमः॑ । \newline
30. नमो॒ नमः॑ । \newline
31. नमो॒ हरि॑केशाय॒ हरि॑केशाय॒ नमो॒ नमो॒ हरि॑केशाय । \newline
32. हरि॑केशा योपवी॒तिन॑ उपवी॒तिने॒ हरि॑केशाय॒ हरि॑केशा योपवी॒तिने᳚ । \newline
33. हरि॑केशा॒येति॒ हरि॑ - के॒शा॒य॒ । \newline
34. उ॒प॒वी॒तिने॑ पु॒ष्टाना᳚म् पु॒ष्टाना॑ मुपवी॒तिन॑ उपवी॒तिने॑ पु॒ष्टाना᳚म् । \newline
35. उ॒प॒वी॒तिन॒ इत्यु॑प - वी॒तिने᳚ । \newline
36. पु॒ष्टाना॒म् पत॑ये॒ पत॑ये पु॒ष्टाना᳚म् पु॒ष्टाना॒म् पत॑ये । \newline
37. पत॑ये॒ नमो॒ नम॒ स्पत॑ये॒ पत॑ये॒ नमः॑ । \newline
38. नमो॒ नमः॑ । \newline
39. नमो॑ भ॒वस्य॑ भ॒वस्य॒ नमो॒ नमो॑ भ॒वस्य॑ । \newline
40. भ॒वस्य॑ हे॒त्यै हे॒त्यै भ॒वस्य॑ भ॒वस्य॑ हे॒त्यै । \newline
41. हे॒त्यै जग॑ता॒म् जग॑ताꣳ हे॒त्यै हे॒त्यै जग॑ताम् । \newline
42. जग॑ता॒म् पत॑ये॒ पत॑ये॒ जग॑ता॒म् जग॑ता॒म् पत॑ये । \newline
43. पत॑ये॒ नमो॒ नम॒ स्पत॑ये॒ पत॑ये॒ नमः॑ । \newline
44. नमो॒ नमः॑ । \newline
45. नमो॑ रु॒द्राय॑ रु॒द्राय॒ नमो॒ नमो॑ रु॒द्राय॑ । \newline
46. रु॒द्राया॑त ता॒विन॑ आतता॒विने॑ रु॒द्राय॑ रु॒द्राया॑ तता॒विने᳚ । \newline
47. आ॒त॒ता॒विने॒ क्षेत्रा॑णा॒म् क्षेत्रा॑णा मातता॒विन॑ आतता॒विने॒ क्षेत्रा॑णाम् । \newline
48. आ॒त॒ता॒विन॒ इत्या᳚ - त॒ता॒विने᳚ । \newline
49. क्षेत्रा॑णा॒म् पत॑ये॒ पत॑ये॒ क्षेत्रा॑णा॒म् क्षेत्रा॑णा॒म् पत॑ये । \newline
50. पत॑ये॒ नमो॒ नम॒ स्पत॑ये॒ पत॑ये॒ नमः॑ । \newline
51. नमो॒ नमः॑ । \newline
52. नमः॑ सू॒ताय॑ सू॒ताय॒ नमो॒ नमः॑ सू॒ताय॑ । \newline
53. सू॒ताया ह॑न्त्या॒या ह॑न्त्याय सू॒ताय॑ सू॒ताया ह॑न्त्याय । \newline
54. अह॑न्त्याय॒ वना॑नां॒ ॅवना॑ना॒ मह॑न्त्या॒या ह॑न्त्याय॒ वना॑नाम् । \newline
55. वना॑ना॒म् पत॑ये॒ पत॑ये॒ वना॑नां॒ ॅवना॑ना॒म् पत॑ये । \newline
56. पत॑ये॒ नमो॒ नम॒ स्पत॑ये॒ पत॑ये॒ नमः॑ । \newline
57. नमो॒ नमः॑ । \newline
58. नमो॒ रोहि॑ताय॒ रोहि॑ताय॒ नमो॒ नमो॒ रोहि॑ताय । \newline

\textbf{Ghana Paata } \newline

1. नमो॒ हिर॑ण्यबाहवे॒ हिर॑ण्यबाहवे॒ नमो॒ नमो॒ हिर॑ण्यबाहवे सेना॒न्ये॑ सेना॒न्ये॑ हिर॑ण्यबाहवे॒ नमो॒ नमो॒ हिर॑ण्यबाहवे सेना॒न्ये᳚ । \newline
2. हिर॑ण्यबाहवे सेना॒न्ये॑ सेना॒न्ये॑ हिर॑ण्यबाहवे॒ हिर॑ण्यबाहवे सेना॒न्ये॑ दि॒शाम् दि॒शाꣳ से॑ना॒न्ये॑ हिर॑ण्यबाहवे॒ हिर॑ण्यबाहवे सेना॒न्ये॑ दि॒शाम् । \newline
3. हिर॑ण्यबाहव॒ इति॒ हिर॑ण्य - बा॒ह॒वे॒ । \newline
4. से॒ना॒न्ये॑ दि॒शाम् दि॒शाꣳ से॑ना॒न्ये॑ सेना॒न्ये॑ दि॒शाम् च॑ च दि॒शाꣳ से॑ना॒न्ये॑ सेना॒न्ये॑ दि॒शाम् च॑ । \newline
5. से॒ना॒न्य॑ इति॑ सेना - न्ये᳚ । \newline
6. दि॒शाम् च॑ च दि॒शाम् दि॒शाम् च॒ पत॑ये॒ पत॑ये च दि॒शाम् दि॒शाम् च॒ पत॑ये । \newline
7. च॒ पत॑ये॒ पत॑ये च च॒ पत॑ये॒ नमो॒ नम॒ स्पत॑ये च च॒ पत॑ये॒ नमः॑ । \newline
8. पत॑ये॒ नमो॒ नम॒ स्पत॑ये॒ पत॑ये॒ नमो॒ नमः॑ । \newline
9. नमो॒ नमः॑ । \newline
10. नमो॑ वृ॒क्षेभ्यो॑ वृ॒क्षेभ्यो॒ नमो॒ नमो॑ वृ॒क्षेभ्यो॒ हरि॑केशेभ्यो॒ हरि॑केशेभ्यो वृ॒क्षेभ्यो॒ नमो॒ नमो॑ वृ॒क्षेभ्यो॒ हरि॑केशेभ्यः । \newline
11. वृ॒क्षेभ्यो॒ हरि॑केशेभ्यो॒ हरि॑केशेभ्यो वृ॒क्षेभ्यो॑ वृ॒क्षेभ्यो॒ हरि॑केशेभ्यः पशू॒नाम् प॑शू॒नाꣳ हरि॑केशेभ्यो वृ॒क्षेभ्यो॑ वृ॒क्षेभ्यो॒ हरि॑केशेभ्यः पशू॒नाम् । \newline
12. हरि॑केशेभ्यः पशू॒नाम् प॑शू॒नाꣳ हरि॑केशेभ्यो॒ हरि॑केशेभ्यः पशू॒नाम् पत॑ये॒ पत॑ये पशू॒नाꣳ हरि॑केशेभ्यो॒ हरि॑केशेभ्यः पशू॒नाम् पत॑ये । \newline
13. हरि॑केशेभ्य॒ इति॒ हरि॑ - के॒शे॒भ्यः॒ । \newline
14. प॒शू॒नाम् पत॑ये॒ पत॑ये पशू॒नाम् प॑शू॒नाम् पत॑ये॒ नमो॒ नम॒ स्पत॑ये पशू॒नाम् प॑शू॒नाम् पत॑ये॒ नमः॑ । \newline
15. पत॑ये॒ नमो॒ नम॒ स्पत॑ये॒ पत॑ये॒ नमो॒ नमः॑ । \newline
16. नमो॒ नमः॑ । \newline
17. नमः॑ स॒स्पिञ्ज॑राय स॒स्पिञ्ज॑राय॒ नमो॒ नमः॑ स॒स्पिञ्ज॑राय॒ त्विषी॑मते॒ त्विषी॑मते स॒स्पिञ्ज॑राय॒ नमो॒ नमः॑ स॒स्पिञ्ज॑राय॒ त्विषी॑मते । \newline
18. स॒स्पिञ्ज॑राय॒ त्विषी॑मते॒ त्विषी॑मते स॒स्पिञ्ज॑राय स॒स्पिञ्ज॑राय॒ त्विषी॑मते पथी॒नाम् प॑थी॒नाम् त्विषी॑मते स॒स्पिञ्ज॑राय स॒स्पिञ्ज॑राय॒ त्विषी॑मते पथी॒नाम् । \newline
19. त्विषी॑मते पथी॒नाम् प॑थी॒नाम् त्विषी॑मते॒ त्विषी॑मते पथी॒नाम् पत॑ये॒ पत॑ये पथी॒नाम् त्विषी॑मते॒ त्विषी॑मते पथी॒नाम् पत॑ये । \newline
20. त्विषी॑मत॒ इति॒ त्विषि॑ - म॒ते॒ । \newline
21. प॒थी॒नाम् पत॑ये॒ पत॑ये पथी॒नाम् प॑थी॒नाम् पत॑ये॒ नमो॒ नम॒ स्पत॑ये पथी॒नाम् प॑थी॒नाम् पत॑ये॒ नमः॑ । \newline
22. पत॑ये॒ नमो॒ नम॒ स्पत॑ये॒ पत॑ये॒ नमो॒ नमः॑ । \newline
23. नमो॒ नमः॑ । \newline
24. नमो॑ बभ्लु॒शाय॑ बभ्लु॒शाय॒ नमो॒ नमो॑ बभ्लु॒शाय॑ विव्या॒धिने॑ विव्या॒धिने॑ बभ्लु॒शाय॒ नमो॒ नमो॑ बभ्लु॒शाय॑ विव्या॒धिने᳚ । \newline
25. ब॒भ्लु॒शाय॑ विव्या॒धिने॑ विव्या॒धिने॑ बभ्लु॒शाय॑ बभ्लु॒शाय॑ विव्या॒धिने ऽन्ना॑ना॒ मन्ना॑नां ॅविव्या॒धिने॑ बभ्लु॒शाय॑ बभ्लु॒शाय॑ विव्या॒धिने ऽन्ना॑नाम् । \newline
26. वि॒व्या॒धिने ऽन्ना॑ना॒ मन्ना॑नां ॅविव्या॒धिने॑ विव्या॒धिने ऽन्ना॑ना॒म् पत॑ये॒ पत॒ये ऽन्ना॑नां ॅविव्या॒धिने॑ विव्या॒धिने ऽन्ना॑ना॒म् पत॑ये । \newline
27. वि॒व्या॒धिन॒ इति॑ वि - व्या॒धिने᳚ । \newline
28. अन्ना॑ना॒म् पत॑ये॒ पत॒ये ऽन्ना॑ना॒ मन्ना॑ना॒म् पत॑ये॒ नमो॒ नम॒ स्पत॒ये ऽन्ना॑ना॒ मन्ना॑ना॒म् पत॑ये॒ नमः॑ । \newline
29. पत॑ये॒ नमो॒ नम॒ स्पत॑ये॒ पत॑ये॒ नमो॒ नमः॑ । \newline
30. नमो॒ नमः॑ । \newline
31. नमो॒ हरि॑केशाय॒ हरि॑केशाय॒ नमो॒ नमो॒ हरि॑केशा योपवी॒तिन॑ उपवी॒तिने॒ हरि॑केशाय॒ नमो॒ नमो॒ हरि॑केशा योपवी॒तिने᳚ । \newline
32. हरि॑केशा योपवी॒तिन॑ उपवी॒तिने॒ हरि॑केशाय॒ हरि॑केशा योपवी॒तिने॑ पु॒ष्टाना᳚म् पु॒ष्टाना॑ मुपवी॒तिने॒ हरि॑केशाय॒ हरि॑केशा योपवी॒तिने॑ पु॒ष्टाना᳚म् । \newline
33. हरि॑केशा॒येति॒ हरि॑ - के॒शा॒य॒ । \newline
34. उ॒प॒वी॒तिने॑ पु॒ष्टाना᳚म् पु॒ष्टाना॑ मुपवी॒तिन॑ उपवी॒तिने॑ पु॒ष्टाना॒म् पत॑ये॒ पत॑ये पु॒ष्टाना॑ मुपवी॒तिन॑ उपवी॒तिने॑ पु॒ष्टाना॒म् पत॑ये । \newline
35. उ॒प॒वी॒तिन॒ इत्यु॑प - वी॒तिने᳚ । \newline
36. पु॒ष्टाना॒म् पत॑ये॒ पत॑ये पु॒ष्टाना᳚म् पु॒ष्टाना॒म् पत॑ये॒ नमो॒ नम॒ स्पत॑ये पु॒ष्टाना᳚म् पु॒ष्टाना॒म् पत॑ये॒ नमः॑ । \newline
37. पत॑ये॒ नमो॒ नम॒ स्पत॑ये॒ पत॑ये॒ नमो॒ नमः॑ । \newline
38. नमो॒ नमः॑ । \newline
39. नमो॑ भ॒वस्य॑ भ॒वस्य॒ नमो॒ नमो॑ भ॒वस्य॑ हे॒त्यै हे॒त्यै भ॒वस्य॒ नमो॒ नमो॑ भ॒वस्य॑ हे॒त्यै । \newline
40. भ॒वस्य॑ हे॒त्यै हे॒त्यै भ॒वस्य॑ भ॒वस्य॑ हे॒त्यै जग॑ता॒म् जग॑ताꣳ हे॒त्यै भ॒वस्य॑ भ॒वस्य॑ हे॒त्यै जग॑ताम् । \newline
41. हे॒त्यै जग॑ता॒म् जग॑ताꣳ हे॒त्यै हे॒त्यै जग॑ता॒म् पत॑ये॒ पत॑ये॒ जग॑ताꣳ हे॒त्यै हे॒त्यै जग॑ता॒म् पत॑ये । \newline
42. जग॑ता॒म् पत॑ये॒ पत॑ये॒ जग॑ता॒म् जग॑ता॒म् पत॑ये॒ नमो॒ नम॒ स्पत॑ये॒ जग॑ता॒म् जग॑ता॒म् पत॑ये॒ नमः॑ । \newline
43. पत॑ये॒ नमो॒ नम॒ स्पत॑ये॒ पत॑ये॒ नमो॒ नमः॑ । \newline
44. नमो॒ नमः॑ । \newline
45. नमो॑ रु॒द्राय॑ रु॒द्राय॒ नमो॒ नमो॑ रु॒द्राया॑ तता॒विन॑ आतता॒विने॑ रु॒द्राय॒ नमो॒ नमो॑ रु॒द्राया॑ तता॒विने᳚ । \newline
46. रु॒द्राया॑ तता॒विन॑ आतता॒विने॑ रु॒द्राय॑ रु॒द्राया॑ तता॒विने॒ क्षेत्रा॑णा॒म् क्षेत्रा॑णा मातता॒विने॑ रु॒द्राय॑ रु॒द्राया॑ तता॒विने॒ क्षेत्रा॑णाम् । \newline
47. आ॒त॒ता॒विने॒ क्षेत्रा॑णा॒म् क्षेत्रा॑णा मातता॒विन॑ आतता॒विने॒ क्षेत्रा॑णा॒म् पत॑ये॒ पत॑ये॒ क्षेत्रा॑णा मातता॒विन॑ आतता॒विने॒ क्षेत्रा॑णा॒म् पत॑ये । \newline
48. आ॒त॒ता॒विन॒ इत्या᳚ - त॒ता॒विने᳚ । \newline
49. क्षेत्रा॑णा॒म् पत॑ये॒ पत॑ये॒ क्षेत्रा॑णा॒म् क्षेत्रा॑णा॒म् पत॑ये॒ नमो॒ नम॒ स्पत॑ये॒ क्षेत्रा॑णा॒म् क्षेत्रा॑णा॒म् पत॑ये॒ नमः॑ । \newline
50. पत॑ये॒ नमो॒ नम॒ स्पत॑ये॒ पत॑ये॒ नमो॒ नमः॑ । \newline
51. नमो॒ नमः॑ । \newline
52. नमः॑ सू॒ताय॑ सू॒ताय॒ नमो॒ नमः॑ सू॒ताया ह॑न्त्या॒या ह॑न्त्याय सू॒ताय॒ नमो॒ नमः॑ सू॒ताया ह॑न्त्याय । \newline
53. सू॒ताया ह॑न्त्या॒या ह॑न्त्याय सू॒ताय॑ सू॒ताया ह॑न्त्याय॒ वना॑नां॒ ॅवना॑ना॒ मह॑न्त्याय सू॒ताय॑ सू॒ताया ह॑न्त्याय॒ वना॑नाम् । \newline
54. अह॑न्त्याय॒ वना॑नां॒ ॅवना॑ना॒ मह॑न्त्या॒या ह॑न्त्याय॒ वना॑ना॒म् पत॑ये॒ पत॑ये॒ वना॑ना॒ मह॑न्त्या॒या ह॑न्त्याय॒ वना॑ना॒म् पत॑ये । \newline
55. वना॑ना॒म् पत॑ये॒ पत॑ये॒ वना॑नां॒ ॅवना॑ना॒म् पत॑ये॒ नमो॒ नम॒ स्पत॑ये॒ वना॑नां॒ ॅवना॑ना॒म् पत॑ये॒ नमः॑ । \newline
56. पत॑ये॒ नमो॒ नम॒ स्पत॑ये॒ पत॑ये॒ नमो॒ नमः॑ । \newline
57. नमो॒ नमः॑ । \newline
58. नमो॒ रोहि॑ताय॒ रोहि॑ताय॒ नमो॒ नमो॒ रोहि॑ताय स्थ॒पत॑ये स्थ॒पत॑ये॒ रोहि॑ताय॒ नमो॒ नमो॒ रोहि॑ताय स्थ॒पत॑ये । \newline
\pagebreak
\markright{ TS 4.5.2.2  \hfill https://www.vedavms.in \hfill}

\section{ TS 4.5.2.2 }

\textbf{TS 4.5.2.2 } \newline
\textbf{Samhita Paata} \newline

रोहि॑ताय स्थ॒पत॑ये वृ॒क्षाणां॒ पत॑ये॒ नमो॒                                      नमो॑ म॒न्त्रिणे॑ वाणि॒जाय॒ कक्षा॑णां॒ पत॑ये॒ नमो॒                          नमो॑ भुव॒न्तये॑ वारिवस्कृ॒ता-यौष॑धीनां॒ पत॑ये॒ नमो॒   नम॑ उ॒च्चै-र्घो॑षाया क्र॒न्दय॑ते पत्ती॒नां पत॑ये॒ नमो॒                    नमः॑ कृथ्स्नवी॒ताय॒ धाव॑ते॒ सत्त्व॑नां॒ पत॑ये॒ नमः॑ ॥ \newline

\textbf{Pada Paata} \newline

रोहि॑ताय । स्थ॒पत॑ये । वृ॒क्षाणा᳚म् । पत॑ये । नमः॑ । नमः॑ । म॒न्त्रिणे᳚ । वा॒णि॒जाय॑ । कक्षा॑णाम् । पत॑ये । नमः॑ । नमः॑ । भु॒व॒न्तये᳚ । वा॒रि॒व॒स्कृ॒तायेति॑ वारिवः - कृ॒ताय॑ । ओष॑धीनाम् । पत॑ये । नमः॑ । नमः॑ । उ॒च्चैर्घो॑षा॒येत्यु॒च्चैः - घो॒षा॒य॒ । आ॒क्र॒न्दय॑त॒ इत्या᳚-क्र॒न्दय॑ते । प॒त्ती॒नाम् । पत॑ये । नमः॑ । नमः॑ । कृ॒थ्स्न॒वी॒तायेति॑ कृथ्स्न-वी॒ताय॑ । धाव॑ते । सत्त्व॑नाम् । पत॑ये । नमः॑ ॥  \newline


\textbf{Krama Paata} \newline

रोहि॑ताय स्थ॒पत॑ये । स्थ॒पत॑ये वृ॒क्षाणा᳚म् । वृ॒क्षाणा॒म् पत॑ये । पत॑ये॒ नमः॑ । नमो॒ नमः॑ । नमो॑ म॒न्त्रिणे᳚ । म॒न्त्रिणे॑ वाणि॒जाय॑ । वा॒णि॒जाय॒ कक्षा॑णाम् । कक्षा॑णा॒म् पत॑ये । पत॑ये॒ नमः॑ । नमो॒ नमः॑ । नमो॑ भुव॒न्तये᳚ । भु॒व॒न्तये॑ वारिवस्कृ॒ताय॑ । वा॒रि॒व॒स्कृ॒तायौष॑धीनाम् । वा॒रि॒व॒स्कृ॒तायेति॑ वारिवः - कृ॒ताय॑ । ओष॑धीना॒म् पत॑ये । पत॑ये॒ नमः॑ । नमो॒ नमः॑ । नम॑ उ॒च्चैर्घो॑षाय । उ॒च्चैर्घो॑षायाक्र॒न्दय॑ते । उ॒च्चैर्घो॑षा॒येत्यु॒च्चैः - घो॒षा॒य॒ । आ॒क्र॒न्दय॑ते पत्ती॒नाम् । आ॒क्र॒न्दय॑त॒ इत्या᳚ - क्र॒न्दय॑ते । प॒त्ती॒नां पत॑ये । पत॑ये॒ नमः॑ । नमो॒ नमः॑ । नमः॑ कृथ्स्नवी॒ताय॑ । कृ॒थ्स्न॒वी॒ताय॒ धाव॑ते । कृ॒थ्स्न॒वी॒तायेति॑ कृथ्स्न - वी॒ताय॑ । धावे॑ते॒ सत्व॑नाम् । सत्व॑ना॒म् पत॑ये । पत॑ये॒ नमः॑ । नम॒ इति॒ नमः॑ । \newline

\textbf{Jatai Paata} \newline

1. रोहि॑ताय स्थ॒पत॑ये स्थ॒पत॑ये॒ रोहि॑ताय॒ रोहि॑ताय स्थ॒पत॑ये । \newline
2. स्थ॒पत॑ये वृ॒क्षाणां᳚ ॅवृ॒क्षाणा(ग्ग्॑) स्थ॒पत॑ये स्थ॒पत॑ये वृ॒क्षाणा᳚म् । \newline
3. वृ॒क्षाणा॒म् पत॑ये॒ पत॑ये वृ॒क्षाणां᳚ ॅवृ॒क्षाणा॒म् पत॑ये । \newline
4. पत॑ये॒ नमो॒ नम॒ स्पत॑ये॒ पत॑ये॒ नमः॑ । \newline
5. नमो॒ नमः॑ । \newline
6. नमो॑ म॒न्त्रिणे॑ म॒न्त्रिणे॒ नमो॒ नमो॑ म॒न्त्रिणे᳚ । \newline
7. म॒न्त्रिणे॑ वाणि॒जाय॑ वाणि॒जाय॑ म॒न्त्रिणे॑ म॒न्त्रिणे॑ वाणि॒जाय॑ । \newline
8. वा॒णि॒जाय॒ कक्षा॑णा॒म् कक्षा॑णां ॅवाणि॒जाय॑ वाणि॒जाय॒ कक्षा॑णाम् । \newline
9. कक्षा॑णा॒म् पत॑ये॒ पत॑ये॒ कक्षा॑णा॒म् कक्षा॑णा॒म् पत॑ये । \newline
10. पत॑ये॒ नमो॒ नम॒ स्पत॑ये॒ पत॑ये॒ नमः॑ । \newline
11. नमो॒ नमः॑ । \newline
12. नमो॑ भुव॒न्तये॑ भुव॒न्तये॒ नमो॒ नमो॑ भुव॒न्तये᳚ । \newline
13. भु॒व॒न्तये॑ वारिवस्कृ॒ताय॑ वारिवस्कृ॒ताय॑ भुव॒न्तये॑ भुव॒न्तये॑ वारिवस्कृ॒ताय॑ । \newline
14. वा॒रि॒व॒स्कृ॒ता यौष॑धीना॒ मोष॑धीनां ॅवारिवस्कृ॒ताय॑ वारिवस्कृ॒ता यौष॑धीनाम् । \newline
15. वा॒रि॒व॒स्कृ॒तायेति॑ वारिवः - कृ॒ताय॑ । \newline
16. ओष॑धीना॒म् पत॑ये॒ पत॑य॒ ओष॑धीना॒ मोष॑धीना॒म् पत॑ये । \newline
17. पत॑ये॒ नमो॒ नम॒ स्पत॑ये॒ पत॑ये॒ नमः॑ । \newline
18. नमो॒ नमः॑ । \newline
19. नम॑ उ॒च्चैर्घो॑षा यो॒च्चैर्घो॑षाय॒ नमो॒ नम॑ उ॒च्चैर्घो॑षाय । \newline
20. उ॒च्चैर्घो॑षाया क्र॒न्दय॑त आक्र॒न्दय॑त उ॒च्चैर्घो॑षा यो॒च्चैर्घो॑षाया क्र॒न्दय॑ते । \newline
21. उ॒च्चैर्घो॑षा॒येत्यु॒च्चैः - घो॒षा॒य॒ । \newline
22. आ॒क्र॒न्दय॑ते पत्ती॒नाम् प॑त्ती॒ना मा᳚क्र॒न्दय॑त आक्र॒न्दय॑ते पत्ती॒नाम् । \newline
23. आ॒क्र॒न्दय॑त॒ इत्या᳚ - क्र॒न्दय॑ते । \newline
24. प॒त्ती॒नाम् पत॑ये॒ पत॑ये पत्ती॒नाम् प॑त्ती॒नाम् पत॑ये । \newline
25. पत॑ये॒ नमो॒ नम॒ स्पत॑ये॒ पत॑ये॒ नमः॑ । \newline
26. नमो॒ नमः॑ । \newline
27. नमः॑ कृथ्स्नवी॒ताय॑ कृथ्स्नवी॒ताय॒ नमो॒ नमः॑ कृथ्स्नवी॒ताय॑ । \newline
28. कृ॒थ्स्न॒वी॒ताय॒ धाव॑ते॒ धाव॑ते कृथ्स्नवी॒ताय॑ कृथ्स्नवी॒ताय॒ धाव॑ते । \newline
29. कृ॒थ्स्न॒वी॒तायेति॑ कृथ्स्न - वी॒ताय॑ । \newline
30. धाव॑ते॒ सत्व॑ना॒(ग्म्॒) सत्व॑ना॒म् धाव॑ते॒ धाव॑ते॒ सत्व॑नाम् । \newline
31. सत्व॑ना॒म् पत॑ये॒ पत॑ये॒ सत्व॑ना॒(ग्म्॒) सत्व॑ना॒म् पत॑ये । \newline
32. पत॑ये॒ नमो॒ नम॒ स्पत॑ये॒ पत॑ये॒ नमः॑ । \newline
33. नम॒ इति॒ नमः॑ । \newline

\textbf{Ghana Paata } \newline

1. रोहि॑ताय स्थ॒पत॑ये स्थ॒पत॑ये॒ रोहि॑ताय॒ रोहि॑ताय स्थ॒पत॑ये वृ॒क्षाणां᳚ ॅवृ॒क्षाणाꣳ॑ स्थ॒पत॑ये॒ रोहि॑ताय॒ रोहि॑ताय स्थ॒पत॑ये वृ॒क्षाणा᳚म् । \newline
2. स्थ॒पत॑ये वृ॒क्षाणां᳚ ॅवृ॒क्षाणाꣳ॑ स्थ॒पत॑ये स्थ॒पत॑ये वृ॒क्षाणा॒म् पत॑ये॒ पत॑ये वृ॒क्षाणाꣳ॑ स्थ॒पत॑ये स्थ॒पत॑ये वृ॒क्षाणा॒म् पत॑ये । \newline
3. वृ॒क्षाणा॒म् पत॑ये॒ पत॑ये वृ॒क्षाणां᳚ ॅवृ॒क्षाणा॒म् पत॑ये॒ नमो॒ नम॒ स्पत॑ये वृ॒क्षाणां᳚ ॅवृ॒क्षाणा॒म् पत॑ये॒ नमः॑ । \newline
4. पत॑ये॒ नमो॒ नम॒ स्पत॑ये॒ पत॑ये॒ नमो॒ नमः॑ । \newline
5. नमो॒ नमः॑ । \newline
6. नमो॑ म॒न्त्रिणे॑ म॒न्त्रिणे॒ नमो॒ नमो॑ म॒न्त्रिणे॑ वाणि॒जाय॑ वाणि॒जाय॑ म॒न्त्रिणे॒ नमो॒ नमो॑ म॒न्त्रिणे॑ वाणि॒जाय॑ । \newline
7. म॒न्त्रिणे॑ वाणि॒जाय॑ वाणि॒जाय॑ म॒न्त्रिणे॑ म॒न्त्रिणे॑ वाणि॒जाय॒ कक्षा॑णा॒म् कक्षा॑णां ॅवाणि॒जाय॑ म॒न्त्रिणे॑ म॒न्त्रिणे॑ वाणि॒जाय॒ कक्षा॑णाम् । \newline
8. वा॒णि॒जाय॒ कक्षा॑णा॒म् कक्षा॑णां ॅवाणि॒जाय॑ वाणि॒जाय॒ कक्षा॑णा॒म् पत॑ये॒ पत॑ये॒ कक्षा॑णां ॅवाणि॒जाय॑ वाणि॒जाय॒ कक्षा॑णा॒म् पत॑ये । \newline
9. कक्षा॑णा॒म् पत॑ये॒ पत॑ये॒ कक्षा॑णा॒म् कक्षा॑णा॒म् पत॑ये॒ नमो॒ नम॒ स्पत॑ये॒ कक्षा॑णा॒म् कक्षा॑णा॒म् पत॑ये॒ नमः॑ । \newline
10. पत॑ये॒ नमो॒ नम॒ स्पत॑ये॒ पत॑ये॒ नमो॒ नमः॑ । \newline
11. नमो॒ नमः॑ । \newline
12. नमो॑ भुव॒न्तये॑ भुव॒न्तये॒ नमो॒ नमो॑ भुव॒न्तये॑ वारिवस्कृ॒ताय॑ वारिवस्कृ॒ताय॑ भुव॒न्तये॒ नमो॒ नमो॑ भुव॒न्तये॑ वारिवस्कृ॒ताय॑ । \newline
13. भु॒व॒न्तये॑ वारिवस्कृ॒ताय॑ वारिवस्कृ॒ताय॑ भुव॒न्तये॑ भुव॒न्तये॑ वारिवस्कृ॒ता यौष॑धीना॒ मोष॑धीनां ॅवारिवस्कृ॒ताय॑ भुव॒न्तये॑ भुव॒न्तये॑ वारिवस्कृ॒ता यौष॑धीनाम् । \newline
14. वा॒रि॒व॒स्कृ॒ता यौष॑धीना॒ मोष॑धीनां ॅवारिवस्कृ॒ताय॑ वारिवस्कृ॒ता यौष॑धीना॒म् पत॑ये॒ पत॑य॒ ओष॑धीनां ॅवारिवस्कृ॒ताय॑ वारिवस्कृ॒ता यौष॑धीना॒म् पत॑ये । \newline
15. वा॒रि॒व॒स्कृ॒तायेति॑ वारिवः - कृ॒ताय॑ । \newline
16. ओष॑धीना॒म् पत॑ये॒ पत॑य॒ ओष॑धीना॒ मोष॑धीना॒म् पत॑ये॒ नमो॒ नम॒ स्पत॑य॒ ओष॑धीना॒ मोष॑धीना॒म् पत॑ये॒ नमः॑ । \newline
17. पत॑ये॒ नमो॒ नम॒ स्पत॑ये॒ पत॑ये॒ नमो॒ नमः॑ । \newline
18. नमो॒ नमः॑ । \newline
19. नम॑ उ॒च्चैर्घो॑षा यो॒च्चैर्घो॑षाय॒ नमो॒ नम॑ उ॒च्चैर्घो॑षाया क्र॒न्दय॑त आक्र॒न्दय॑त उ॒च्चैर्घो॑षाय॒ नमो॒ नम॑ उ॒च्चैर्घो॑षाया क्र॒न्दय॑ते । \newline
20. उ॒च्चैर्घो॑षाया क्र॒न्दय॑त आक्र॒न्दय॑त उ॒च्चैर्घो॑षा यो॒च्चैर्घो॑षाया क्र॒न्दय॑ते पत्ती॒नाम् प॑त्ती॒ना मा᳚क्र॒न्दय॑त उ॒च्चैर्घो॑षा यो॒च्चैर्घो॑षाया क्र॒न्दय॑ते पत्ती॒नाम् । \newline
21. उ॒च्चैर्घो॑षा॒येत्यु॒च्चैः - घो॒षा॒य॒ । \newline
22. आ॒क्र॒न्दय॑ते पत्ती॒नाम् प॑त्ती॒ना मा᳚क्र॒न्दय॑त आक्र॒न्दय॑ते पत्ती॒नाम् पत॑ये॒ पत॑ये पत्ती॒ना मा᳚क्र॒न्दय॑त आक्र॒न्दय॑ते पत्ती॒नाम् पत॑ये । \newline
23. आ॒क्र॒न्दय॑त॒ इत्या᳚ - क्र॒न्दय॑ते । \newline
24. प॒त्ती॒नाम् पत॑ये॒ पत॑ये पत्ती॒नाम् प॑त्ती॒नाम् पत॑ये॒ नमो॒ नम॒ स्पत॑ये पत्ती॒नाम् प॑त्ती॒नाम् पत॑ये॒ नमः॑ । \newline
25. पत॑ये॒ नमो॒ नम॒ स्पत॑ये॒ पत॑ये॒ नमो॒ नमः॑ । \newline
26. नमो॒ नमः॑ । \newline
27. नमः॑ कृथ्स्नवी॒ताय॑ कृथ्स्नवी॒ताय॒ नमो॒ नमः॑ कृथ्स्नवी॒ताय॒ धाव॑ते॒ धाव॑ते कृथ्स्नवी॒ताय॒ नमो॒ नमः॑ कृथ्स्नवी॒ताय॒ धाव॑ते । \newline
28. कृ॒थ्स्न॒वी॒ताय॒ धाव॑ते॒ धाव॑ते कृथ्स्नवी॒ताय॑ कृथ्स्नवी॒ताय॒ धाव॑ते॒ सत्व॑नाꣳ॒॒ सत्व॑ना॒म् धाव॑ते कृथ्स्नवी॒ताय॑ कृथ्स्नवी॒ताय॒ धाव॑ते॒ सत्व॑नाम् । \newline
29. कृ॒थ्स्न॒वी॒तायेति॑ कृथ्स्न - वी॒ताय॑ । \newline
30. धाव॑ते॒ सत्व॑नाꣳ॒॒ सत्व॑ना॒म् धाव॑ते॒ धाव॑ते॒ सत्व॑ना॒म् पत॑ये॒ पत॑ये॒ सत्व॑ना॒म् धाव॑ते॒ धाव॑ते॒ सत्व॑ना॒म् पत॑ये । \newline
31. सत्व॑ना॒म् पत॑ये॒ पत॑ये॒ सत्व॑नाꣳ॒॒ सत्व॑ना॒म् पत॑ये॒ नमो॒ नम॒ स्पत॑ये॒ सत्व॑नाꣳ॒॒ सत्व॑ना॒म् पत॑ये॒ नमः॑ । \newline
32. पत॑ये॒ नमो॒ नम॒ स्पत॑ये॒ पत॑ये॒ नमः॑ । \newline
33. नम॒ इति॒ नमः॑ । \newline
\pagebreak
\markright{ TS 4.5.3.1  \hfill https://www.vedavms.in \hfill}

\section{ TS 4.5.3.1 }

\textbf{TS 4.5.3.1 } \newline
\textbf{Samhita Paata} \newline

नमः॒ सह॑मानाय निव्या॒धिन॑ आव्या॒धिनी॑नां॒ पत॑ये॒ नमो॒          नमः॑ ककु॒भाय॑ निष॒ङ्गिणे᳚ स्ते॒नानां॒ पत॑ये॒ नमो॒                       नमो॑ निष॒ङ्गिण॑ इषुधि॒मते॒ तस्क॑राणां॒ पत॑ये॒ नमो॒           नमो॒ वञ्च॑ते परि॒वञ्च॑ते स्तायू॒नां पत॑ये॒ नमो॒                                         नमो॑ निचे॒रवे॑ परिच॒रायार॑ण्यानां॒ पत॑ये॒ नमो॒                                   नमः॑ सृका॒विभ्यो॒ जिघाꣳ॑सद्भ्यो मुष्ण॒तां पत॑ये॒ नमो॒                   नमो॑ ऽसि॒मद्भ्यो॒ नक्तं॒ चर॑द्भ्यः प्रकृ॒न्तानां॒ पत॑ये॒ नमो॒ नम॑ उष्णी॒षिणे॑ गिरिच॒राय॑ कुलु॒ञ्चानां॒ पत॑ये॒ नमो॒ नम॒ - [  ] \newline

\textbf{Pada Paata} \newline

नमः॑ । सह॑मानाय । नि॒व्या॒धिन॒ इति॑ नि - व्या॒धिने᳚ । आ॒व्या॒धिनी॑ना॒मित्या᳚ - व्या॒धिनी॑नाम् । पत॑ये । नमः॑ । नमः॑ । क॒कु॒भाय॑ । नि॒ष॒ङ्गिण॒ इति॑ नि-स॒ङ्गिने᳚ । स्ते॒नाना᳚म् । पत॑ये । नमः॑ । नमः॑ । नि॒ष॒ङ्गिण॒ इति॑ नि - स॒ङ्गिने᳚ । इ॒षु॒धि॒मत इती॑षुधि - मते᳚ । तस्क॑राणाम् । पत॑ये । नमः॑ । नमः॑ । वञ्च॑ते । प॒रि॒वञ्च॑त॒ इति॑ परि - वञ्च॑ते । स्ता॒यू॒नाम् । पत॑ये । नमः॑ । नमः॑ । नि॒चे॒रव॒ इति॑ नि - चे॒रवे᳚ । प॒रि॒च॒रायेति॑ परि - च॒राय॑ । अर॑ण्यानाम् । पत॑ये । नमः॑ । नमः॑ । सृ॒का॒विभ्य॒ इति॑ सृका॒वि - भ्यः॒ । जिघाꣳ॑सद्भ्य॒ इति॒ जिघाꣳ॑सत् - भ्यः॒ । मु॒ष्ण॒ताम् । पत॑ये । नमः॑ । नमः॑ । अ॒सि॒मद्भ्य॒ इत्य॑सि॒मत् - भ्यः॒ । नक्त᳚म् । चर॑द्भ्य॒ इति॒ चर॑त् - भ्यः॒ । प्र॒कृ॒न्ताना॒मिति॑ प्र - कृ॒न्ताना᳚म् । पत॑ये । नमः॑ । नमः॑ । उ॒ष्णी॒षिणे᳚ । गि॒रि॒च॒रायेति॑ गिरि - च॒राय॑ । कु॒लु॒ञ्चाना᳚म् । पत॑ये । नमः॑ । नमः॑ ।  \newline


\textbf{Krama Paata} \newline

नमः॒ सह॑मानाय । सह॑मानाय निव्या॒धिने᳚ । नि॒व्या॒धिन॑ आव्या॒धिनी॑नाम् । नि॒व्या॒धिन॒ इति॑ नि - व्या॒धिने᳚ । आ॒व्या॒धिनी॑ना॒म् पत॑ये । आ॒व्या॒धिनी॑ना॒मित्या᳚ - व्या॒धिनी॑नाम् । पत॑ये॒ नमः॑ । नमो॒ नमः॑ । नमः॑ ककु॒भाय॑ । क॒कु॒भाय॑ निष॒ङ्गिणे᳚ । नि॒ष॒ङ्गिणे᳚ स्ते॒नाना᳚म् । नि॒ष॒ङ्गिण॒ इति॑ नि - स॒ङ्गिने᳚ । स्ते॒नाना॒म् पत॑ये । पत॑ये॒ नमः॑ । नमो॒ नमः॑ । नमो॑ निष॒ङ्गिणे᳚ । नि॒ष॒ङ्गिण॑ इषुधि॒मते᳚ । नि॒ष॒ङ्गिण॒ इति॑ नि - स॒ङ्गिने᳚ । इ॒षु॒धि॒मते॒ तस्क॑राणाम् । इ॒षु॒धि॒मत॒ इती॑षुधि - मते᳚ । तस्क॑राणा॒म् पत॑ये । पत॑ये॒ नमः॑ । नमो॒ नमः॑ । नमो॒ वञ्च॑ते । वञ्च॑ते परि॒वञ्च॑ते । प॒रि॒वञ्च॑ते स्तायू॒नाम् । प॒रि॒वञ्च॑त॒ इति॑ परि - वञ्च॑ते । स्ता॒यू॒नां पत॑ये । पत॑ये॒ नमः॑ । नमो॒ नमः॑ । नमो॑ निचे॒रवे᳚ । नि॒चे॒रवे॑ परिच॒राय॑ । नि॒चे॒रव॒ इति॑ नि - चे॒रवे᳚ । प॒रि॒च॒रायार॑ण्यानाम् । प॒रि॒च॒रायेति॑ परि - च॒राय॑ । अर॑ण्याना॒म् पत॑ये । पत॑ये॒ नमः॑ । नमो॒ नमः॑ । नमः॑ सृका॒विभ्यः॑ । सृ॒का॒विभ्यो॒ जिघाꣳ॑सद्भ्यः । सृ॒का॒विभ्य॒ इति॑ सृका॒वि - भ्यः॒ । जिघाꣳ॑सद्भ्यो मुष्ण॒ताम् । जिघाꣳ॑सद्भ्य॒ इति॒ जिघाꣳ॑सत् - भ्यः॒ । मु॒ष्ण॒तां पत॑ये । पत॑ये॒ नमः॑ । नमो॒ नमः॑ । नमो॑ऽसि॒मद्भ्यः॑ । अ॒सि॒मद्भ्यो॒ नक्त᳚म् । अ॒सि॒मद्भ्य॒ इत्य॑सि॒मत् - भ्यः॒ । नक्त॒म् चर॑द्भ्यः । चर॑द्भ्यः प्रकृ॒न्ताना᳚म् । चर॑द्भ्य॒ इति॒ चर॑त् - भ्यः॒ । प्र॒कृ॒न्ताना॒म् पत॑ये । प्र॒कृ॒न्ताना॒मिति॑ प्र - कृ॒न्ताना᳚म् । पत॑ये॒ नमः॑ । नमो॒ नमः॑ । नम॑ उष्णी॒षिणे᳚ । उ॒ष्णी॒षिणे॑ गिरिच॒राय॑ । गि॒रि॒च॒राय॑ कुलु॒ञ्चाना᳚म् । गि॒रि॒च॒रायेति॑ गिरि - च॒राय॑ । कु॒लु॒ञ्चाना॒म् पत॑ये । पत॑ये॒ नमः॑ । नमो॒ नमः॑ । नम॒ इषु॑मद्भ्यः \newline

\textbf{Jatai Paata} \newline

1. नमः॒ सह॑मानाय॒ सह॑मानाय॒ नमो॒ नमः॒ सह॑मानाय । \newline
2. सह॑मानाय निव्या॒धिने॑ निव्या॒धिने॒ सह॑मानाय॒ सह॑मानाय निव्या॒धिने᳚ । \newline
3. नि॒व्या॒धिन॑ आव्या॒धिनी॑ना माव्या॒धिनी॑ना न्निव्या॒धिने॑ निव्या॒धिन॑ आव्या॒धिनी॑नाम् । \newline
4. नि॒व्या॒धिन॒ इति॑ नि - व्या॒धिने᳚ । \newline
5. आ॒व्या॒धिनी॑ना॒म् पत॑ये॒ पत॑य आव्या॒धिनी॑ना माव्या॒धिनी॑ना॒म् पत॑ये । \newline
6. आ॒व्या॒धिनी॑ना॒मित्या᳚ - व्या॒धिनी॑नाम् । \newline
7. पत॑ये॒ नमो॒ नम॒ स्पत॑ये॒ पत॑ये॒ नमः॑ । \newline
8. नमो॒ नमः॑ । \newline
9. नमः॑ ककु॒भाय॑ ककु॒भाय॒ नमो॒ नमः॑ ककु॒भाय॑ । \newline
10. क॒कु॒भाय॑ निष॒ङ्गिणे॑ निष॒ङ्गिणे॑ ककु॒भाय॑ ककु॒भाय॑ निष॒ङ्गिणे᳚ । \newline
11. नि॒ष॒ङ्गिणे᳚ स्ते॒नाना(ग्ग्॑) स्ते॒नाना᳚ न्निष॒ङ्गिणे॑ निष॒ङ्गिणे᳚ स्ते॒नाना᳚म् । \newline
12. नि॒ष॒ङ्गिण॒ इति॑ नि - स॒ङ्गिने᳚ । \newline
13. स्ते॒नाना॒म् पत॑ये॒ पत॑ये स्ते॒नाना(ग्ग्॑) स्ते॒नाना॒म् पत॑ये । \newline
14. पत॑ये॒ नमो॒ नम॒ स्पत॑ये॒ पत॑ये॒ नमः॑ । \newline
15. नमो॒ नमः॑ । \newline
16. नमो॑ निष॒ङ्गिणे॑ निष॒ङ्गिणे॒ नमो॒ नमो॑ निष॒ङ्गिणे᳚ । \newline
17. नि॒ष॒ङ्गिण॑ इषुधि॒मत॑ इषुधि॒मते॑ निष॒ङ्गिणे॑ निष॒ङ्गिण॑ इषुधि॒मते᳚ । \newline
18. नि॒ष॒ङ्गिण॒ इति॑ नि - स॒ङ्गिने᳚ । \newline
19. इ॒षु॒धि॒मते॒ तस्क॑राणा॒म् तस्क॑राणा मिषुधि॒मत॑ इषुधि॒मते॒ तस्क॑राणाम् । \newline
20. इ॒षु॒धि॒मत॒ इती॑षुधि - मते᳚ । \newline
21. तस्क॑राणा॒म् पत॑ये॒ पत॑ये॒ तस्क॑राणा॒म् तस्क॑राणा॒म् पत॑ये । \newline
22. पत॑ये॒ नमो॒ नम॒ स्पत॑ये॒ पत॑ये॒ नमः॑ । \newline
23. नमो॒ नमः॑ । \newline
24. नमो॒ वञ्च॑ते॒ वञ्च॑ते॒ नमो॒ नमो॒ वञ्च॑ते । \newline
25. वञ्च॑ते परि॒वञ्च॑ते परि॒वञ्च॑ते॒ वञ्च॑ते॒ वञ्च॑ते परि॒वञ्च॑ते । \newline
26. प॒रि॒वञ्च॑ते स्तायू॒नाꣳ स्ता॑यू॒नाम् प॑रि॒वञ्च॑ते परि॒वञ्च॑ते स्तायू॒नाम् । \newline
27. प॒रि॒वञ्च॑त॒ इति॑ परि - वञ्च॑ते । \newline
28. स्ता॒यू॒नाम् पत॑ये॒ पत॑ये स्तायू॒नाꣳ स्ता॑यू॒नाम् पत॑ये । \newline
29. पत॑ये॒ नमो॒ नम॒ स्पत॑ये॒ पत॑ये॒ नमः॑ । \newline
30. नमो॒ नमः॑ । \newline
31. नमो॑ निचे॒रवे॑ निचे॒रवे॒ नमो॒ नमो॑ निचे॒रवे᳚ । \newline
32. नि॒चे॒रवे॑ परिच॒राय॑ परिच॒राय॑ निचे॒रवे॑ निचे॒रवे॑ परिच॒राय॑ । \newline
33. नि॒चे॒रव॒ इति॑ नि - चे॒रवे᳚ । \newline
34. प॒रि॒च॒राया र॑ण्याना॒ मर॑ण्यानाम् परिच॒राय॑ परिच॒राया र॑ण्यानाम् । \newline
35. प॒रि॒च॒रायेति॑ परि - च॒राय॑ । \newline
36. अर॑ण्याना॒म् पत॑ये॒ पत॒ये ऽर॑ण्याना॒ मर॑ण्याना॒म् पत॑ये । \newline
37. पत॑ये॒ नमो॒ नम॒ स्पत॑ये॒ पत॑ये॒ नमः॑ । \newline
38. नमो॒ नमः॑ । \newline
39. नमः॑ सृका॒विभ्यः॑ सृका॒विभ्यो॒ नमो॒ नमः॑ सृका॒विभ्यः॑ । \newline
40. सृ॒का॒विभ्यो॒ जिघा(ग्म्॑)सद्भ्यो॒ जिघा(ग्म्॑)सद्भ्यः सृका॒विभ्यः॑ सृका॒विभ्यो॒ जिघा(ग्म्॑)सद्भ्यः । \newline
41. सृ॒का॒विभ्य॒ इति॑ सृका॒वि - भ्यः॒ । \newline
42. जिघा(ग्म्॑)सद्भ्यो मुष्ण॒ताम् मु॑ष्ण॒ताम् जिघा(ग्म्॑)सद्भ्यो॒ जिघा(ग्म्॑)सद्भ्यो मुष्ण॒ताम् । \newline
43. जिघा(ग्म्॑)सद्भ्य॒ इति॒ जिघा(ग्म्॑)सत् - भ्यः॒ । \newline
44. मु॒ष्ण॒ताम् पत॑ये॒ पत॑ये मुष्ण॒ताम् मु॑ष्ण॒ताम् पत॑ये । \newline
45. पत॑ये॒ नमो॒ नम॒ स्पत॑ये॒ पत॑ये॒ नमः॑ । \newline
46. नमो॒ नमः॑ । \newline
47. नमो॑ ऽसि॒मद्भ्यो॑ ऽसि॒मद्भ्यो॒ नमो॒ नमो॑ ऽसि॒मद्भ्यः॑ । \newline
48. अ॒सि॒मद्भ्यो॒ नक्त॒म् नक्त॑ मसि॒मद्भ्यो॑ ऽसि॒मद्भ्यो॒ नक्त᳚म् । \newline
49. अ॒सि॒मद्भ्य॒ इत्य॑सि॒मत् - भ्यः॒ । \newline
50. नक्त॒म् चर॑द्भ्य॒ श्चर॑द्भ्यो॒ नक्त॒म् नक्त॒म् चर॑द्भ्यः । \newline
51. चर॑द्भ्यः प्रकृ॒न्ताना᳚म् प्रकृ॒न्ताना॒म् चर॑द्भ्य॒ श्चर॑द्भ्यः प्रकृ॒न्ताना᳚म् । \newline
52. चर॑द्भ्य॒ इति॒ चर॑त् - भ्यः॒ । \newline
53. प्र॒कृ॒न्ताना॒म् पत॑ये॒ पत॑ये प्रकृ॒न्ताना᳚म् प्रकृ॒न्ताना॒म् पत॑ये । \newline
54. प्र॒कृ॒न्ताना॒मिति॑ प्र - कृ॒न्ताना᳚म् । \newline
55. पत॑ये॒ नमो॒ नम॒ स्पत॑ये॒ पत॑ये॒ नमः॑ । \newline
56. नमो॒ नमः॑ । \newline
57. नम॑ उष्णी॒षिण॑ उष्णी॒षिणे॒ नमो॒ नम॑ उष्णी॒षिणे᳚ । \newline
58. उ॒ष्णी॒षिणे॑ गिरिच॒राय॑ गिरिच॒रा यो᳚ष्णी॒षिण॑ उष्णी॒षिणे॑ गिरिच॒राय॑ । \newline
59. गि॒रि॒च॒राय॑ कुलु॒ञ्चाना᳚म् कुलु॒ञ्चाना᳚म् गिरिच॒राय॑ गिरिच॒राय॑ कुलु॒ञ्चाना᳚म् । \newline
60. गि॒रि॒च॒रायेति॑ गिरि - च॒राय॑ । \newline
61. कु॒लु॒ञ्चाना॒म् पत॑ये॒ पत॑ये कुलु॒ञ्चाना᳚म् कुलु॒ञ्चाना॒म् पत॑ये । \newline
62. पत॑ये॒ नमो॒ नम॒ स्पत॑ये॒ पत॑ये॒ नमः॑ । \newline
63. नमो॒ नमः॑ । \newline
64. नम॒ इषु॑मद्भ्य॒ इषु॑मद्भ्यो॒ नमो॒ नम॒ इषु॑मद्भ्यः । \newline

\textbf{Ghana Paata } \newline

1. नमः॒ सह॑मानाय॒ सह॑मानाय॒ नमो॒ नमः॒ सह॑मानाय निव्या॒धिने॑ निव्या॒धिने॒ सह॑मानाय॒ नमो॒ नमः॒ सह॑मानाय निव्या॒धिने᳚ । \newline
2. सह॑मानाय निव्या॒धिने॑ निव्या॒धिने॒ सह॑मानाय॒ सह॑मानाय निव्या॒धिन॑ आव्या॒धिनी॑ना माव्या॒धिनी॑नां निव्या॒धिने॒ सह॑मानाय॒ सह॑मानाय निव्या॒धिन॑ आव्या॒धिनी॑नाम् । \newline
3. नि॒व्या॒धिन॑ आव्या॒धिनी॑ना माव्या॒धिनी॑नां निव्या॒धिने॑ निव्या॒धिन॑ आव्या॒धिनी॑ना॒म् पत॑ये॒ पत॑य आव्या॒धिनी॑नां निव्या॒धिने॑ निव्या॒धिन॑ आव्या॒धिनी॑ना॒म् पत॑ये । \newline
4. नि॒व्या॒धिन॒ इति॑ नि - व्या॒धिने᳚ । \newline
5. आ॒व्या॒धिनी॑ना॒म् पत॑ये॒ पत॑य आव्या॒धिनी॑ना माव्या॒धिनी॑ना॒म् पत॑ये॒ नमो॒ नम॒ स्पत॑य आव्या॒धिनी॑ना माव्या॒धिनी॑ना॒म् पत॑ये॒ नमः॑ । \newline
6. आ॒व्या॒धिनी॑ना॒मित्या᳚ - व्या॒धिनी॑नाम् । \newline
7. पत॑ये॒ नमो॒ नम॒ स्पत॑ये॒ पत॑ये॒ नमो॒ नमः॑ । \newline
8. नमो॒ नमः॑ । \newline
9. नमः॑ ककु॒भाय॑ ककु॒भाय॒ नमो॒ नमः॑ ककु॒भाय॑ निष॒ङ्गिणे॑ निष॒ङ्गिणे॑ ककु॒भाय॒ नमो॒ नमः॑ ककु॒भाय॑ निष॒ङ्गिणे᳚ । \newline
10. क॒कु॒भाय॑ निष॒ङ्गिणे॑ निष॒ङ्गिणे॑ ककु॒भाय॑ ककु॒भाय॑ निष॒ङ्गिणे᳚ स्ते॒नानाꣳ॑ स्ते॒नाना᳚न्निष॒ङ्गिणे॑ ककु॒भाय॑ ककु॒भाय॑ निष॒ङ्गिणे᳚ स्ते॒नाना᳚म् । \newline
11. नि॒ष॒ङ्गिणे᳚ स्ते॒नानाꣳ॑ स्ते॒नाना᳚म् निष॒ङ्गिणे॑ निष॒ङ्गिणे᳚ स्ते॒नाना॒म् पत॑ये॒ पत॑ये 
स्ते॒नाना᳚म् निष॒ङ्गिणे॑ निष॒ङ्गिणे᳚ स्ते॒नाना॒म् पत॑ये । \newline
12. नि॒ष॒ङ्गिण॒ इति॑ नि - स॒ङ्गिने᳚ । \newline
13. स्ते॒नाना॒म् पत॑ये॒ पत॑ये स्ते॒नानाꣳ॑ स्ते॒नाना॒म् पत॑ये॒ नमो॒ नम॒ स्पत॑ये स्ते॒नानाꣳ॑ स्ते॒नाना॒म् पत॑ये॒ नमः॑ । \newline
14. पत॑ये॒ नमो॒ नम॒ स्पत॑ये॒ पत॑ये॒ नमो॒ नमः॑ । \newline
15. नमो॒ नमः॑ । \newline
16. नमो॑ निष॒ङ्गिणे॑ निष॒ङ्गिणे॒ नमो॒ नमो॑ निष॒ङ्गिण॑ इषुधि॒मत॑ इषुधि॒मते॑ निष॒ङ्गिणे॒ नमो॒ नमो॑ निष॒ङ्गिण॑ इषुधि॒मते᳚ । \newline
17. नि॒ष॒ङ्गिण॑ इषुधि॒मत॑ इषुधि॒मते॑ निष॒ङ्गिणे॑ निष॒ङ्गिण॑ इषुधि॒मते॒ तस्क॑राणा॒म् तस्क॑राणा मिषुधि॒मते॑ निष॒ङ्गिणे॑ निष॒ङ्गिण॑ इषुधि॒मते॒ तस्क॑राणाम् । \newline
18. नि॒ष॒ङ्गिण॒ इति॑ नि - स॒ङ्गिने᳚ । \newline
19. इ॒षु॒धि॒मते॒ तस्क॑राणा॒म् तस्क॑राणा मिषुधि॒मत॑ इषुधि॒मते॒ तस्क॑राणा॒म् पत॑ये॒ पत॑ये॒ तस्क॑राणा मिषुधि॒मत॑ इषुधि॒मते॒ तस्क॑राणा॒म् पत॑ये । \newline
20. इ॒षु॒धि॒मत॒ इती॑षुधि - मते᳚ । \newline
21. तस्क॑राणा॒म् पत॑ये॒ पत॑ये॒ तस्क॑राणा॒म् तस्क॑राणा॒म् पत॑ये॒ नमो॒ नम॒ स्पत॑ये॒ तस्क॑राणा॒म् तस्क॑राणा॒म् पत॑ये॒ नमः॑ । \newline
22. पत॑ये॒ नमो॒ नम॒ स्पत॑ये॒ पत॑ये॒ नमो॒ नमः॑ । \newline
23. नमो॒ नमः॑ । \newline
24. नमो॒ वञ्च॑ते॒ वञ्च॑ते॒ नमो॒ नमो॒ वञ्च॑ते परि॒वञ्च॑ते परि॒वञ्च॑ते॒ वञ्च॑ते॒ नमो॒ नमो॒ वञ्च॑ते परि॒वञ्च॑ते । \newline
25. वञ्च॑ते परि॒वञ्च॑ते परि॒वञ्च॑ते॒ वञ्च॑ते॒ वञ्च॑ते परि॒वञ्च॑ते स्तायू॒नाꣳ स्ता॑यू॒नाम् प॑रि॒वञ्च॑ते॒ वञ्च॑ते॒ वञ्च॑ते परि॒वञ्च॑ते स्तायू॒नाम् । \newline
26. प॒रि॒वञ्च॑ते स्तायू॒नाꣳ स्ता॑यू॒नाम् प॑रि॒वञ्च॑ते परि॒वञ्च॑ते स्तायू॒नाम् पत॑ये॒ पत॑ये स्तायू॒नाम् प॑रि॒वञ्च॑ते परि॒वञ्च॑ते स्तायू॒नाम् पत॑ये । \newline
27. प॒रि॒वञ्च॑त॒ इति॑ परि - वञ्च॑ते । \newline
28. स्ता॒यू॒नाम् पत॑ये॒ पत॑ये स्तायू॒नाꣳ स्ता॑यू॒नाम् पत॑ये॒ नमो॒ नम॒ स्पत॑ये स्तायू॒नाꣳ स्ता॑यू॒नाम् पत॑ये॒ नमः॑ । \newline
29. पत॑ये॒ नमो॒ नम॒ स्पत॑ये॒ पत॑ये॒ नमो॒ नमः॑ । \newline
30. नमो॒ नमः॑ । \newline
31. नमो॑ निचे॒रवे॑ निचे॒रवे॒ नमो॒ नमो॑ निचे॒रवे॑ परिच॒राय॑ परिच॒राय॑ निचे॒रवे॒ नमो॒ नमो॑ निचे॒रवे॑ परिच॒राय॑ । \newline
32. नि॒चे॒रवे॑ परिच॒राय॑ परिच॒राय॑ निचे॒रवे॑ निचे॒रवे॑ परिच॒राया र॑ण्याना॒ मर॑ण्यानाम् परिच॒राय॑ निचे॒रवे॑ निचे॒रवे॑ परिच॒राया र॑ण्यानाम् । \newline
33. नि॒चे॒रव॒ इति॑ नि - चे॒रवे᳚ । \newline
34. प॒रि॒च॒राया र॑ण्याना॒ मर॑ण्यानाम् परिच॒राय॑ परिच॒राया र॑ण्याना॒म् पत॑ये॒ पत॒ये ऽर॑ण्यानाम् परिच॒राय॑ परिच॒राया र॑ण्याना॒म् पत॑ये । \newline
35. प॒रि॒च॒रायेति॑ परि - च॒राय॑ । \newline
36. अर॑ण्याना॒म् पत॑ये॒ पत॒ये ऽर॑ण्याना॒ मर॑ण्याना॒म् पत॑ये॒ नमो॒ नम॒ स्पत॒ये ऽर॑ण्याना॒ मर॑ण्याना॒म् पत॑ये॒ नमः॑ । \newline
37. पत॑ये॒ नमो॒ नम॒ स्पत॑ये॒ पत॑ये॒ नमो॒ नमः॑ । \newline
38. नमो॒ नमः॑ । \newline
39. नमः॑ सृका॒विभ्यः॑ सृका॒विभ्यो॒ नमो॒ नमः॑ सृका॒विभ्यो॒ जिघाꣳ॑सद्भ्यो॒ जिघाꣳ॑सद्भ्यः सृका॒विभ्यो॒ नमो॒ नमः॑ सृका॒विभ्यो॒ जिघाꣳ॑सद्भ्यः । \newline
40. सृ॒का॒विभ्यो॒ जिघाꣳ॑सद्भ्यो॒ जिघाꣳ॑सद्भ्यः सृका॒विभ्यः॑ सृका॒विभ्यो॒ जिघाꣳ॑सद्भ्यो मुष्ण॒ताम् मु॑ष्ण॒ताम् जिघाꣳ॑सद्भ्यः सृका॒विभ्यः॑ सृका॒विभ्यो॒ जिघाꣳ॑सद्भ्यो मुष्ण॒ताम् । \newline
41. सृ॒का॒विभ्य॒ इति॑ सृका॒वि - भ्यः॒ । \newline
42. जिघाꣳ॑सद्भ्यो मुष्ण॒ताम् मु॑ष्ण॒ताम् जिघाꣳ॑सद्भ्यो॒ जिघाꣳ॑सद्भ्यो मुष्ण॒ताम् पत॑ये॒ पत॑ये मुष्ण॒ताम् जिघाꣳ॑सद्भ्यो॒ जिघाꣳ॑सद्भ्यो मुष्ण॒ताम् पत॑ये । \newline
43. जिघाꣳ॑सद्भ्य॒ इति॒ जिघाꣳ॑सत् - भ्यः॒ । \newline
44. मु॒ष्ण॒ताम् पत॑ये॒ पत॑ये मुष्ण॒ताम् मु॑ष्ण॒ताम् पत॑ये॒ नमो॒ नम॒ स्पत॑ये मुष्ण॒ताम् मु॑ष्ण॒ताम् पत॑ये॒ नमः॑ । \newline
45. पत॑ये॒ नमो॒ नम॒ स्पत॑ये॒ पत॑ये॒ नमो॒ नमः॑ । \newline
46. नमो॒ नमः॑ । \newline
47. नमो॑ ऽसि॒मद्भ्यो॑ ऽसि॒मद्भ्यो॒ नमो॒ नमो॑ ऽसि॒मद्भ्यो॒ नक्त॒म् नक्त॑ मसि॒मद्भ्यो॒ नमो॒ नमो॑ ऽसि॒मद्भ्यो॒ नक्त᳚म् । \newline
48. अ॒सि॒मद्भ्यो॒ नक्त॒म् नक्त॑ मसि॒मद्भ्यो॑ ऽसि॒मद्भ्यो॒ नक्त॒म् चर॑द्भ्य॒ श्चर॑द्भ्यो॒ नक्त॑ मसि॒मद्भ्यो॑ ऽसि॒मद्भ्यो॒ नक्त॒म् चर॑द्भ्यः । \newline
49. अ॒सि॒मद्भ्य॒ इत्य॑सि॒मत् - भ्यः॒ । \newline
50. नक्त॒म् चर॑द्भ्य॒ श्चर॑द्भ्यो॒ नक्त॒म् नक्त॒म् चर॑द्भ्यः प्रकृ॒न्ताना᳚म् प्रकृ॒न्ताना॒म् चर॑द्भ्यो॒ नक्त॒म् नक्त॒म् चर॑द्भ्यः प्रकृ॒न्ताना᳚म् । \newline
51. चर॑द्भ्यः प्रकृ॒न्ताना᳚म् प्रकृ॒न्ताना॒म् चर॑द्भ्य॒ श्चर॑द्भ्यः प्रकृ॒न्ताना॒म् पत॑ये॒ पत॑ये प्रकृ॒न्ताना॒म् चर॑द्भ्य॒ श्चर॑द्भ्यः प्रकृ॒न्ताना॒म् पत॑ये । \newline
52. चर॑द्भ्य॒ इति॒ चर॑त् - भ्यः॒ । \newline
53. प्र॒कृ॒न्ताना॒म् पत॑ये॒ पत॑ये प्रकृ॒न्ताना᳚म् प्रकृ॒न्ताना॒म् पत॑ये॒ नमो॒ नम॒ स्पत॑ये प्रकृ॒न्ताना᳚म् प्रकृ॒न्ताना॒म् पत॑ये॒ नमः॑ । \newline
54. प्र॒कृ॒न्ताना॒मिति॑ प्र - कृ॒न्ताना᳚म् । \newline
55. पत॑ये॒ नमो॒ नम॒ स्पत॑ये॒ पत॑ये॒ नमो॒ नमः॑ । \newline
56. नमो॒ नमः॑ । \newline
57. नम॑ उष्णी॒षिण॑ उष्णी॒षिणे॒ नमो॒ नम॑ उष्णी॒षिणे॑ गिरिच॒राय॑ गिरिच॒रा यो᳚ष्णी॒षिणे॒ नमो॒ नम॑ उष्णी॒षिणे॑ गिरिच॒राय॑ । \newline
58. उ॒ष्णी॒षिणे॑ गिरिच॒राय॑ गिरिच॒रा यो᳚ष्णी॒षिण॑ उष्णी॒षिणे॑ गिरिच॒राय॑ कुलु॒ञ्चाना᳚म् कुलु॒ञ्चाना᳚म् गिरिच॒रा यो᳚ष्णी॒षिण॑ उष्णी॒षिणे॑ गिरिच॒राय॑ कुलु॒ञ्चाना᳚म् । \newline
59. गि॒रि॒च॒राय॑ कुलु॒ञ्चाना᳚म् कुलु॒ञ्चाना᳚म् गिरिच॒राय॑ गिरिच॒राय॑ कुलु॒ञ्चाना॒म् पत॑ये॒ पत॑ये कुलु॒ञ्चाना᳚म् गिरिच॒राय॑ गिरिच॒राय॑ कुलु॒ञ्चाना॒म् पत॑ये । \newline
60. गि॒रि॒च॒रायेति॑ गिरि - च॒राय॑ । \newline
61. कु॒लु॒ञ्चाना॒म् पत॑ये॒ पत॑ये कुलु॒ञ्चाना᳚म् कुलु॒ञ्चाना॒म् पत॑ये॒ नमो॒ नम॒ स्पत॑ये कुलु॒ञ्चाना᳚म् कुलु॒ञ्चाना॒म् पत॑ये॒ नमः॑ । \newline
62. पत॑ये॒ नमो॒ नम॒ स्पत॑ये॒ पत॑ये॒ नमो॒ नमः॑ । \newline
63. नमो॒ नमः॑ । \newline
64. नम॒ इषु॑मद्भ्य॒ इषु॑मद्भ्यो॒ नमो॒ नम॒ इषु॑मद्भ्यो धन्वा॒विभ्यो॑ धन्वा॒विभ्य॒ इषु॑मद्भ्यो॒ नमो॒ नम॒ इषु॑मद्भ्यो धन्वा॒विभ्यः॑ । \newline
\pagebreak
\markright{ TS 4.5.3.2  \hfill https://www.vedavms.in \hfill}

\section{ TS 4.5.3.2 }

\textbf{TS 4.5.3.2 } \newline
\textbf{Samhita Paata} \newline

इषु॑मद्भ्यो धन्वा॒विभ्य॑श्च वो॒ नमो॒                  नम॑ आतन्वा॒नेभ्यः॑ प्रति॒दधा॑नेभ्यश्च वो॒ नमो॒                                     नम॑ आ॒यच्छ॑द्भ्यो विसृ॒जद्भ्य॑श्च वो॒ नमो॒   नमोऽस्य॑द्भ्यो॒ विद्ध्य॑द्भ्यश्च वो॒ नमो॒                नम॒ आसी॑नेभ्यः॒ शया॑नेभ्यश्च वो॒ नमो॒                                               नमः॑ स्व॒पद्भ्यो॒ जाग्र॑द्भ्यश्च वो॒ नमो॒                                              नम॒स्तिष्ठ॑द्भ्यो॒ धाव॑द्भ्यश्च वो॒ नमो॒     नमः॑ स॒भाभ्यः॑ स॒भाप॑तिभ्यश्च वो॒ नमो॒                                            नमो॒ अश्वे॒भ्यो ऽश्व॑पतिभ्य ( ) श्च वो॒ नमः॑ ॥ \newline

\textbf{Pada Paata} \newline

इषु॑मद्भ्य॒ इतीषु॑मत् - भ्यः॒ । ध॒न्वा॒विभ्य॒ इति॑ धन्वा॒वि-भ्यः॒ । च॒ । वः॒ । नमः॑ । नमः॑ । आ॒त॒न्वा॒नेभ्य॒ इत्या᳚ - त॒न्वा॒नेभ्यः॑ । प्र॒ति॒दधा॑नेभ्य॒ इति॑ प्रति - दधा॑नेभ्यः । च॒ । वः॒ । नमः॑ । नमः॑ । आ॒यच्छ॑द्भ्य॒ इत्या॒यच्छ॑त् - भ्यः॒ । वि॒सृ॒जद्भ्य॒ इति॑ विसृ॒जत् - भ्यः॒ । च॒ । वः॒ । नमः॑ । नमः॑ । अस्य॑द्भ्य॒ इत्यस्य॑त् - भ्यः॒ । विद्ध्य॑द्भ्य॒ इति॒ विद्ध्य॑त्-भ्यः॒ । च॒ । वः॒ । नमः॑ । नमः॑ । आसी॑नेभ्यः । शया॑नेभ्यः । च॒ । वः॒ । नमः॑ । नमः॑ । स्व॒पद्भ्य॒ इति॑ स्व॒पत् - भ्यः॒ । जाग्र॑द्भ्य॒ इति॒ जाग्र॑त्-भ्यः॒ । च॒ । वः॒ । नमः॑ । नमः॑ । तिष्ठ॑द्भ्य॒ इति॒ तिष्ठ॑त् - भ्यः॒ । धाव॑द्भ्य॒ इति॒ धाव॑त् - भ्यः॒ । च॒ । वः॒ । नमः॑ । नमः॑ । स॒भाभ्यः॑ । स॒भाप॑तिभ्य॒ इति॑ स॒भाप॑ति - भ्यः॒ । च॒ । वः॒ । नमः॑ । नमः॑ । अश्वे᳚भ्यः । अश्व॑पतिभ्य॒ इत्यश्व॑पति - भ्यः॒ ( ) । च॒ । वः॒ । नमः॑ ॥  \newline


\textbf{Krama Paata} \newline

इषु॑मद्भ्यो धन्वा॒विभ्यः॑ । इषु॑मद्भ्य॒ इतीषु॑मत् - भ्यः॒ । ध॒न्वा॒विभ्य॑श्च । ध॒न्वा॒विभ्य॒ इति॑ धन्वा॒वि - भ्यः॒ । च॒ वः॒ । वो॒ नमः॑ । नमो॒ नमः॑ । नम॑ आतन्वा॒नेभ्यः॑ । आ॒त॒न्वा॒नेभ्यः॑ प्रति॒दधा॑नेभ्यः । आ॒त॒न्वा॒नेभ्य॒ इत्या᳚ - त॒न्वा॒नेभ्यः॑ । प्र॒ति॒दधा॑नेभ्यश्च । प्र॒ति॒दधा॑नेभ्य॒ इति॑ प्रति - दधा॑नेभ्यः । च॒ वः॒ । वो॒ नमः॑ । नमो॒ नमः॑ । नम॑ आ॒यच्छ॑द्भ्यः । आ॒यच्छ॑द्भ्यो विसृ॒जद्भ्यः॑ । आ॒यच्छ॑द्भ्य॒ इत्या॒यच्छ॑त् - भ्यः॒ । वि॒सृ॒जद्भ्य॑श्च । वि॒सृ॒जद्भ्य॒ इति॑ विसृ॒जत् - भ्यः॒ । च॒ वः॒ । वो॒ नमः॑ । नमो॒ नमः॑ । नमोऽस्य॑द्भ्यः । अस्य॑द्भ्यो॒ विद्ध्य॑द्भ्यः । अस्य॑द्भ्य॒ इत्यस्य॑त् - भ्यः॒ । विद्ध्य॑द्भ्यश्च । विद्ध्य॑द्भ्य॒ इति॒ विद्ध्य॑त् - भ्यः॒ । च॒ वः॒ । वो॒ नमः॑ । नमो॒ नमः॑ । नम॒ आसी॑नेभ्यः । आसी॑नेभ्यः॒ शया॑नेभ्यः । शया॑नेभ्यश्च । च॒ वः॒ । वो॒ नमः॑ । नमो॒ नमः॑ । नमः॑ स्व॒पद्भ्यः॑ । स्व॒पद्भ्यो॒ जाग्र॑द्भ्यः । स्व॒पद्भ्य॒ इति॑ स्व॒पत् - भ्यः॒ । जाग्र॑द्भ्यश्च । जाग्र॑द्भ्य॒ इति॒ जाग्र॑त् - भ्यः॒ । च॒ वः॒ । वो॒ नमः॑ । नमो॒ नमः॑ । नम॒स्तिष्ठ॑द्भ्यः । तिष्ठ॑द्भ्यो॒ धाव॑द्भ्यः । तिष्ठ॑द्भ्य॒ इति॒ तिष्ठ॑त् - भ्यः॒ । धाव॑द्भ्यश्च । धाव॑द्भ्य॒ इति॒ धाव॑त् - भ्यः॒ । च॒ वः॒ । वो॒ नमः॑ । नमो॒ नमः॑ । नमः॑ स॒भाभ्यः॑ । स॒भाभ्यः॑ स॒भाप॑तिभ्यः । स॒भाप॑तिभ्यश्च । स॒भाप॑तिभ्य॒ इति॑ स॒भाप॑ति - भ्यः॒ । च॒ वः॒ । वो॒ नमः॑ । नमो॒ नमः॑ । नमो॒ अश्वे᳚भ्यः । अश्वे॒भ्योऽश्व॑पतिभ्यः ( ) । अश्व॑पतिभ्यश्च । अश्व॑पतिभ्य॒ इत्यश्व॑पति - भ्यः॒ । च॒ वः॒ । वो॒ नमः॑ । नम॒ इति॒ नमः॑ । \newline

\textbf{Jatai Paata} \newline

1. इषु॑मद्भ्यो धन्वा॒विभ्यो॑ धन्वा॒विभ्य॒ इषु॑मद्भ्य॒ इषु॑मद्भ्यो धन्वा॒विभ्यः॑ । \newline
2. इषु॑मद्भ्य॒ इतीषु॑मत् - भ्यः॒ । \newline
3. ध॒न्वा॒विभ्य॑श्च च धन्वा॒विभ्यो॑ धन्वा॒विभ्य॑श्च । \newline
4. ध॒न्वा॒विभ्य॒ इति॑ धन्वा॒वि - भ्यः॒ । \newline
5. च॒ वो॒ व॒श्च॒ च॒ वः॒ । \newline
6. वो॒ नमो॒ नमो॑ वो वो॒ नमः॑ । \newline
7. नमो॒ नमः॑ । \newline
8. नम॑ आतन्वा॒नेभ्य॑ आतन्वा॒नेभ्यो॒ नमो॒ नम॑ आतन्वा॒नेभ्यः॑ । \newline
9. आ॒त॒न्वा॒नेभ्यः॑ प्रति॒दधा॑नेभ्यः प्रति॒दधा॑नेभ्य आतन्वा॒नेभ्य॑ आतन्वा॒नेभ्यः॑ प्रति॒दधा॑नेभ्यः । \newline
10. आ॒त॒न्वा॒नेभ्य॒ इत्या᳚ - त॒न्वा॒नेभ्यः॑ । \newline
11. प्र॒ति॒दधा॑नेभ्यश्च च प्रति॒दधा॑नेभ्यः प्रति॒दधा॑नेभ्यश्च । \newline
12. प्र॒ति॒दधा॑नेभ्य॒ इति॑ प्रति - दधा॑नेभ्यः । \newline
13. च॒ वो॒ व॒श्च॒ च॒ वः॒ । \newline
14. वो॒ नमो॒ नमो॑ वो वो॒ नमः॑ । \newline
15. नमो॒ नमः॑ । \newline
16. नम॑ आ॒यच्छ॑द्भ्य आ॒यच्छ॑द्भ्यो॒ नमो॒ नम॑ आ॒यच्छ॑द्भ्यः । \newline
17. आ॒यच्छ॑द्भ्यो विसृ॒जद्भ्यो॑ विसृ॒जद्भ्य॑ आ॒यच्छ॑द्भ्य आ॒यच्छ॑द्भ्यो विसृ॒जद्भ्यः॑ । \newline
18. आ॒यच्छ॑द्भ्य॒ इत्या॒यच्छ॑त् - भ्यः॒ । \newline
19. वि॒सृ॒जद्भ्य॑श्च च विसृ॒जद्भ्यो॑ विसृ॒जद्भ्य॑श्च । \newline
20. वि॒सृ॒जद्भ्य॒ इति॑ विसृ॒जत् - भ्यः॒ । \newline
21. च॒ वो॒ व॒श्च॒ च॒ वः॒ । \newline
22. वो॒ नमो॒ नमो॑ वो वो॒ नमः॑ । \newline
23. नमो॒ नमः॑ । \newline
24. नमो ऽस्य॒द्भ्यो ऽस्य॑द्भ्यो॒ नमो॒ नमो ऽस्य॑द्भ्यः । \newline
25. अस्य॑द्भ्यो॒ विद्ध्य॑द्भ्यो॒ विद्ध्य॒द्भ्यो ऽस्य॒द्भ्यो ऽस्य॑द्भ्यो॒ विद्ध्य॑द्भ्यः । \newline
26. अस्य॑द्भ्य॒ इत्यस्य॑त् - भ्यः॒ । \newline
27. विद्ध्य॑द्भ्यश्च च॒ विद्ध्य॑द्भ्यो॒ विद्ध्य॑द्भ्यश्च । \newline
28. विध्य॑द्भ्य॒ इति॒ विध्य॑त् - भ्यः॒ । \newline
29. च॒ वो॒ व॒श्च॒ च॒ वः॒ । \newline
30. वो॒ नमो॒ नमो॑ वो वो॒ नमः॑ । \newline
31. नमो॒ नमः॑ । \newline
32. नम॒ आसी॑नेभ्य॒ आसी॑नेभ्यो॒ नमो॒ नम॒ आसी॑नेभ्यः । \newline
33. आसी॑नेभ्यः॒ शया॑नेभ्यः॒ शया॑नेभ्य॒ आसी॑नेभ्य॒ आसी॑नेभ्यः॒ शया॑नेभ्यः । \newline
34. शया॑नेभ्यश्च च॒ शया॑नेभ्यः॒ शया॑नेभ्यश्च । \newline
35. च॒ वो॒ व॒श्च॒ च॒ वः॒ । \newline
36. वो॒ नमो॒ नमो॑ वो वो॒ नमः॑ । \newline
37. नमो॒ नमः॑ । \newline
38. नमः॑ स्व॒पद्भ्यः॑ स्व॒पद्भ्यो॒ नमो॒ नमः॑ स्व॒पद्भ्यः॑ । \newline
39. स्व॒पद्भ्यो॒ जाग्र॑द्भ्यो॒ जाग्र॑द्भ्यः स्व॒पद्भ्यः॑ स्व॒पद्भ्यो॒ जाग्र॑द्भ्यः । \newline
40. स्व॒पद्भ्य॒ इति॑ स्व॒पत् - भ्यः॒ । \newline
41. जाग्र॑द्भ्यश्च च॒ जाग्र॑द्भ्यो॒ जाग्र॑द्भ्यश्च । \newline
42. जाग्र॑द्भ्य॒ इति॒ जाग्र॑त् - भ्यः॒ । \newline
43. च॒ वो॒ व॒श्च॒ च॒ वः॒ । \newline
44. वो॒ नमो॒ नमो॑ वो वो॒ नमः॑ । \newline
45. नमो॒ नमः॑ । \newline
46. नम॒ स्तिष्ठ॑द्भ्य॒ स्तिष्ठ॑द्भ्यो॒ नमो॒ नम॒ स्तिष्ठ॑द्भ्यः । \newline
47. तिष्ठ॑द्भ्यो॒ धाव॑द्भ्यो॒ धाव॑द्भ्य॒ स्तिष्ठ॑द्भ्य॒ स्तिष्ठ॑द्भ्यो॒ धाव॑द्भ्यः । \newline
48. तिष्ठ॑द्भ्य॒ इति॒ तिष्ठ॑त् - भ्यः॒ । \newline
49. धाव॑द्भ्यश्च च॒ धाव॑द्भ्यो॒ धाव॑द्भ्यश्च । \newline
50. धाव॑द्भ्य॒ इति॒ धाव॑त् - भ्यः॒ । \newline
51. च॒ वो॒ व॒श्च॒ च॒ वः॒ । \newline
52. वो॒ नमो॒ नमो॑ वो वो॒ नमः॑ । \newline
53. नमो॒ नमः॑ । \newline
54. नमः॑ स॒भाभ्यः॑ स॒भाभ्यो॒ नमो॒ नमः॑ स॒भाभ्यः॑ । \newline
55. स॒भाभ्यः॑ स॒भाप॑तिभ्यः स॒भाप॑तिभ्यः स॒भाभ्यः॑ स॒भाभ्यः॑ स॒भाप॑तिभ्यः । \newline
56. स॒भाप॑तिभ्यश्च च स॒भाप॑तिभ्यः स॒भाप॑तिभ्यश्च । \newline
57. स॒भाप॑तिभ्य॒ इति॑ स॒भाप॑ति - भ्यः॒ । \newline
58. च॒ वो॒ व॒श्च॒ च॒ वः॒ । \newline
59. वो॒ नमो॒ नमो॑ वो वो॒ नमः॑ । \newline
60. नमो॒ नमः॑ । \newline
61. नमो॒ अश्वे॒भ्यो ऽश्वे᳚भ्यो॒ नमो॒ नमो॒ अश्वे᳚भ्यः । \newline
62. अश्वे॒भ्यो ऽश्व॑पति॒भ्यो ऽश्व॑पति॒भ्यो ऽश्वे॒भ्यो ऽश्वे॒भ्यो ऽश्व॑पतिभ्यः । \newline
63. अश्व॑पतिभ्यश्च॒ चाश्व॑पति॒भ्यो ऽश्व॑पतिभ्यश्च । \newline
64. अश्व॑पतिभ्य॒ इत्यश्व॑पति - भ्यः॒ । \newline
65. च॒ वो॒ व॒श्च॒ च॒ वः॒ । \newline
66. वो॒ नमो॒ नमो॑ वो वो॒ नमः॑ । \newline
67. नम॒ इति॒ नमः॑ । \newline

\textbf{Ghana Paata } \newline

1. इषु॑मद्भ्यो धन्वा॒विभ्यो॑ धन्वा॒विभ्य॒ इषु॑मद्भ्य॒ इषु॑मद्भ्यो धन्वा॒विभ्य॑श्च च धन्वा॒विभ्य॒ इषु॑मद्भ्य॒ इषु॑मद्भ्यो धन्वा॒विभ्य॑श्च । \newline
2. इषु॑मद्भ्य॒ इतीषु॑मत् - भ्यः॒ । \newline
3. ध॒न्वा॒विभ्य॑श्च च धन्वा॒विभ्यो॑ धन्वा॒विभ्य॑श्च वो वश्च धन्वा॒विभ्यो॑ धन्वा॒विभ्य॑श्च वः । \newline
4. ध॒न्वा॒विभ्य॒ इति॑ धन्वा॒वि - भ्यः॒ । \newline
5. च॒ वो॒ व॒श्च॒ च॒ वो॒ नमो॒ नमो॑ वश्च च वो॒ नमः॑ । \newline
6. वो॒ नमो॒ नमो॑ वो वो॒ नमो॒ नमः॑ । \newline
7. नमो॒ नमः॑ । \newline
8. नम॑ आतन्वा॒नेभ्य॑ आतन्वा॒नेभ्यो॒ नमो॒ नम॑ आतन्वा॒नेभ्यः॑ प्रति॒दधा॑नेभ्यः प्रति॒दधा॑नेभ्य आतन्वा॒नेभ्यो॒ नमो॒ नम॑ आतन्वा॒नेभ्यः॑ प्रति॒दधा॑नेभ्यः । \newline
9. आ॒त॒न्वा॒नेभ्यः॑ प्रति॒दधा॑नेभ्यः प्रति॒दधा॑नेभ्य आतन्वा॒नेभ्य॑ आतन्वा॒नेभ्यः॑ प्रति॒दधा॑नेभ्यश्च च प्रति॒दधा॑नेभ्य आतन्वा॒नेभ्य॑ आतन्वा॒नेभ्यः॑ प्रति॒दधा॑नेभ्यश्च । \newline
10. आ॒त॒न्वा॒नेभ्य॒ इत्या᳚ - त॒न्वा॒नेभ्यः॑ । \newline
11. प्र॒ति॒दधा॑नेभ्यश्च च प्रति॒दधा॑नेभ्यः प्रति॒दधा॑नेभ्यश्च वो वश्च प्रति॒दधा॑नेभ्यः प्रति॒दधा॑नेभ्यश्च वः । \newline
12. प्र॒ति॒दधा॑नेभ्य॒ इति॑ प्रति - दधा॑नेभ्यः । \newline
13. च॒ वो॒ व॒श्च॒ च॒ वो॒ नमो॒ नमो॑ वश्च च वो॒ नमः॑ । \newline
14. वो॒ नमो॒ नमो॑ वो वो॒ नमो॒ नमः॑ । \newline
15. नमो॒ नमः॑ । \newline
16. नम॑ आ॒यच्छ॑द्भ्य आ॒यच्छ॑द्भ्यो॒ नमो॒ नम॑ आ॒यच्छ॑द्भ्यो विसृ॒जद्भ्यो॑ विसृ॒जद्भ्य॑ आ॒यच्छ॑द्भ्यो॒ नमो॒ नम॑ आ॒यच्छ॑द्भ्यो विसृ॒जद्भ्यः॑ । \newline
17. आ॒यच्छ॑द्भ्यो विसृ॒जद्भ्यो॑ विसृ॒जद्भ्य॑ आ॒यच्छ॑द्भ्य आ॒यच्छ॑द्भ्यो विसृ॒जद्भ्य॑श्च च विसृ॒जद्भ्य॑ आ॒यच्छ॑द्भ्य आ॒यच्छ॑द्भ्यो विसृ॒जद्भ्य॑श्च । \newline
18. आ॒यच्छ॑द्भ्य॒ इत्या॒यच्छ॑त् - भ्यः॒ । \newline
19. वि॒सृ॒जद्भ्य॑श्च च विसृ॒जद्भ्यो॑ विसृ॒जद्भ्य॑श्च वो वश्च विसृ॒जद्भ्यो॑ विसृ॒जद्भ्य॑श्च वः । \newline
20. वि॒सृ॒जद्भ्य॒ इति॑ विसृ॒जत् - भ्यः॒ । \newline
21. च॒ वो॒ व॒श्च॒ च॒ वो॒ नमो॒ नमो॑ वश्च च वो॒ नमः॑ । \newline
22. वो॒ नमो॒ नमो॑ वो वो॒ नमो॒ नमः॑ । \newline
23. नमो॒ नमः॑ । \newline
24. नमो ऽस्य॒द्भ्यो ऽस्य॑द्भ्यो॒ नमो॒ नमो ऽस्य॑द्भ्यो॒ विद्ध्य॑द्भ्यो॒ विद्ध्य॒द्भ्यो ऽस्य॑द्भ्यो॒ नमो॒ नमो ऽस्य॑द्भ्यो॒ विद्ध्य॑द्भ्यः । \newline
25. अस्य॑द्भ्यो॒ विद्ध्य॑द्भ्यो॒ विद्ध्य॒द्भ्यो ऽस्य॒द्भ्यो ऽस्य॑द्भ्यो॒ विद्ध्य॑द्भ्यश्च च॒ विद्ध्य॒द्भ्यो ऽस्य॒द्भ्यो ऽस्य॑द्भ्यो॒ विद्ध्य॑द्भ्यश्च । \newline
26. अस्य॑द्भ्य॒ इत्यस्य॑त् - भ्यः॒ । \newline
27. विद्ध्य॑द्भ्यश्च च॒ विद्ध्य॑द्भ्यो॒ विद्ध्य॑द्भ्यश्च वो वश्च॒ विद्ध्य॑द्भ्यो॒ विद्ध्य॑द्भ्यश्च वः । \newline
28. विध्य॑द्भ्य॒ इति॒ विध्य॑त् - भ्यः॒ । \newline
29. च॒ वो॒ व॒श्च॒ च॒ वो॒ नमो॒ नमो॑ वश्च च वो॒ नमः॑ । \newline
30. वो॒ नमो॒ नमो॑ वो वो॒ नमो॒ नमः॑ । \newline
31. नमो॒ नमः॑ । \newline
32. नम॒ आसी॑नेभ्य॒ आसी॑नेभ्यो॒ नमो॒ नम॒ आसी॑नेभ्यः॒ शया॑नेभ्यः॒ शया॑नेभ्य॒ आसी॑नेभ्यो॒ नमो॒ नम॒ आसी॑नेभ्यः॒ शया॑नेभ्यः । \newline
33. आसी॑नेभ्यः॒ शया॑नेभ्यः॒ शया॑नेभ्य॒ आसी॑नेभ्य॒ आसी॑नेभ्यः॒ शया॑नेभ्यश्च च॒ शया॑नेभ्य॒ आसी॑नेभ्य॒ आसी॑नेभ्यः॒ शया॑नेभ्यश्च । \newline
34. शया॑नेभ्यश्च च॒ शया॑नेभ्यः॒ शया॑नेभ्यश्च वो वश्च॒ शया॑नेभ्यः॒ शया॑नेभ्यश्च वः । \newline
35. च॒ वो॒ व॒श्च॒ च॒ वो॒ नमो॒ नमो॑ वश्च च वो॒ नमः॑ । \newline
36. वो॒ नमो॒ नमो॑ वो वो॒ नमो॒ नमः॑ । \newline
37. नमो॒ नमः॑ । \newline
38. नमः॑ स्व॒पद्भ्यः॑ स्व॒पद्भ्यो॒ नमो॒ नमः॑ स्व॒पद्भ्यो॒ जाग्र॑द्भ्यो॒ जाग्र॑द्भ्यः स्व॒पद्भ्यो॒ नमो॒ नमः॑ स्व॒पद्भ्यो॒ जाग्र॑द्भ्यः । \newline
39. स्व॒पद्भ्यो॒ जाग्र॑द्भ्यो॒ जाग्र॑द्भ्यः स्व॒पद्भ्यः॑ स्व॒पद्भ्यो॒ जाग्र॑द्भ्यश्च च॒ जाग्र॑द्भ्यः स्व॒पद्भ्यः॑ स्व॒पद्भ्यो॒ जाग्र॑द्भ्यश्च । \newline
40. स्व॒पद्भ्य॒ इति॑ स्व॒पत् - भ्यः॒ । \newline
41. जाग्र॑द्भ्यश्च च॒ जाग्र॑द्भ्यो॒ जाग्र॑द्भ्यश्च वो वश्च॒ जाग्र॑द्भ्यो॒ जाग्र॑द्भ्यश्च वः । \newline
42. जाग्र॑द्भ्य॒ इति॒ जाग्र॑त् - भ्यः॒ । \newline
43. च॒ वो॒ व॒श्च॒ च॒ वो॒ नमो॒ नमो॑ वश्च च वो॒ नमः॑ । \newline
44. वो॒ नमो॒ नमो॑ वो वो॒ नमो॒ नमः॑ । \newline
45. नमो॒ नमः॑ । \newline
46. नम॒ स्तिष्ठ॑द्भ्य॒ स्तिष्ठ॑द्भ्यो॒ नमो॒ नम॒ स्तिष्ठ॑द्भ्यो॒ धाव॑द्भ्यो॒ धाव॑द्भ्य॒ स्तिष्ठ॑द्भ्यो॒ नमो॒ नम॒ स्तिष्ठ॑द्भ्यो॒ धाव॑द्भ्यः । \newline
47. तिष्ठ॑द्भ्यो॒ धाव॑द्भ्यो॒ धाव॑द्भ्य॒ स्तिष्ठ॑द्भ्य॒ स्तिष्ठ॑द्भ्यो॒ धाव॑द्भ्यश्च च॒ धाव॑द्भ्य॒ स्तिष्ठ॑द्भ्य॒ स्तिष्ठ॑द्भ्यो॒ धाव॑द्भ्यश्च । \newline
48. तिष्ठ॑द्भ्य॒ इति॒ तिष्ठ॑त् - भ्यः॒ । \newline
49. धाव॑द्भ्यश्च च॒ धाव॑द्भ्यो॒ धाव॑द्भ्यश्च वो वश्च॒ धाव॑द्भ्यो॒ धाव॑द्भ्यश्च वः । \newline
50. धाव॑द्भ्य॒ इति॒ धाव॑त् - भ्यः॒ । \newline
51. च॒ वो॒ व॒श्च॒ च॒ वो॒ नमो॒ नमो॑ वश्च च वो॒ नमः॑ । \newline
52. वो॒ नमो॒ नमो॑ वो वो॒ नमो॒ नमः॑ । \newline
53. नमो॒ नमः॑ । \newline
54. नमः॑ स॒भाभ्यः॑ स॒भाभ्यो॒ नमो॒ नमः॑ स॒भाभ्यः॑ स॒भाप॑तिभ्यः स॒भाप॑तिभ्यः स॒भाभ्यो॒ नमो॒ नमः॑ स॒भाभ्यः॑ स॒भाप॑तिभ्यः । \newline
55. स॒भाभ्यः॑ स॒भाप॑तिभ्यः स॒भाप॑तिभ्यः स॒भाभ्यः॑ स॒भाभ्यः॑ स॒भाप॑तिभ्यश्च च स॒भाप॑तिभ्यः स॒भाभ्यः॑ स॒भाभ्यः॑ स॒भाप॑तिभ्यश्च । \newline
56. स॒भाप॑तिभ्यश्च च स॒भाप॑तिभ्यः स॒भाप॑तिभ्यश्च वो वश्च स॒भाप॑तिभ्यः स॒भाप॑तिभ्यश्च वः । \newline
57. स॒भाप॑तिभ्य॒ इति॑ स॒भाप॑ति - भ्यः॒ । \newline
58. च॒ वो॒ व॒श्च॒ च॒ वो॒ नमो॒ नमो॑ वश्च च वो॒ नमः॑ । \newline
59. वो॒ नमो॒ नमो॑ वो वो॒ नमो॒ नमः॑ । \newline
60. नमो॒ नमः॑ । \newline
61. नमो॒ अश्वे॒भ्यो ऽश्वे᳚भ्यो॒ नमो॒ नमो॒ अश्वे॒भ्यो ऽश्व॑पति॒भ्यो ऽश्व॑पति॒भ्यो ऽश्वे᳚भ्यो॒ नमो॒ नमो॒ अश्वे॒भ्यो ऽश्व॑पतिभ्यः । \newline
62. अश्वे॒भ्यो ऽश्व॑पति॒भ्यो ऽश्व॑पति॒भ्यो ऽश्वे॒भ्यो ऽश्वे॒भ्यो ऽश्व॑पतिभ्यश्च॒ चाश्व॑पति॒भ्यो ऽश्वे॒भ्यो ऽश्वे॒भ्यो ऽश्व॑पतिभ्यश्च । \newline
63. अश्व॑पतिभ्यश्च॒ चाश्व॑पति॒भ्यो ऽश्व॑पतिभ्यश्च वो व॒ श्चाश्व॑पति॒भ्यो ऽश्व॑पतिभ्यश्च वः । \newline
64. अश्व॑पतिभ्य॒ इत्यश्व॑पति - भ्यः॒ । \newline
65. च॒ वो॒ व॒श्च॒ च॒ वो॒ नमो॒ नमो॑ वश्च च वो॒ नमः॑ । \newline
66. वो॒ नमो॒ नमो॑ वो वो॒ नमः॑ । \newline
67. नम॒ इति॒ नमः॑ । \newline
\pagebreak
\markright{ TS 4.5.4.1  \hfill https://www.vedavms.in \hfill}

\section{ TS 4.5.4.1 }

\textbf{TS 4.5.4.1 } \newline
\textbf{Samhita Paata} \newline

नम॑ आव्या॒धिनी᳚भ्यो वि॒विद्ध्य॑न्तीभ्यश्च वो॒ नमो॒                             नम॒ उग॑णाभ्य-स्तृꣳह॒तीभ्य॑श्च वो॒ नमो॒                                          नमो॑ गृ॒थ्सेभ्यो॑ गृ॒थ्सप॑तिभ्यश्च वो॒ नमो॒                                                नमो॒ व्राते᳚भ्यो॒ व्रात॑पतिभ्यश्च वो॒ नमो॒                                             नमो॑ ग॒णेभ्यो॑ ग॒णप॑तिभ्यश्च वो॒ नमो॒                                                      नमो॒ विरू॑पेभ्यो वि॒श्वरू॑पेभ्यश्च वो॒ नमो॒                                            नमो॑ म॒हद्भ्यः॑ क्षुल्ल॒केभ्य॑श्च वो॒ नमो॒                                           नमो॑ र॒थिभ्यो॑-ऽर॒थेभ्य॑श्च वो॒ नमो॒                                                       नमो॒ रथे᳚भ्यो॒ - [  ] \newline

\textbf{Pada Paata} \newline

नमः॑ । आ॒व्या॒धिनी᳚भ्य॒ इत्या᳚ - व्या॒धिनी᳚भ्यः । वि॒विद्ध्य॑न्तीभ्य॒ इति॑ वि - विद्ध्य॑न्तीभ्यः । च॒ । वः॒ । नमः॑ । नमः॑ । उग॑णाभ्यः । तृꣳ॒॒ह॒तीभ्यः॑ । च॒ । वः॒ । नमः॑ । नमः॑ । गृ॒थ्सेभ्यः॑ । गृ॒थ्सप॑तिभ्य॒ इति॑ गृ॒थ्सप॑ति - भ्यः॒ । च॒ । वः॒ । नमः॑ । नमः॑ । व्राते᳚भ्यः । व्रात॑पतिभ्य॒ इति॒ व्रात॑पति-भ्यः॒ । च॒ । वः॒ । नमः॑ । नमः॑ । ग॒णेभ्यः॑ । ग॒णप॑तिभ्य॒ इति॑ ग॒णप॑ति - भ्यः॒ । च॒ । वः॒ । नमः॑ । नमः॑ । विरू॑पेभ्य॒ इति॒ वि - रू॒पे॒भ्यः॒ । वि॒श्वरू॑पेभ्य॒ इति॑ वि॒श्व-रू॒पे॒भ्यः॒ । च॒ । वः॒ । नमः॑ । नमः॑ । म॒हद्भ्य॒ इति॑ म॒हत्-भ्यः॒ । क्षु॒ल्ल॒केभ्यः॑ । च॒ । वः॒ । नमः॑ । नमः॑ । र॒थिभ्य॒ इति॑ र॒थि-भ्यः॒ । अ॒र॒थेभ्यः॑ । च॒ । वः॒ । नमः॑ । नमः॑ । रथे᳚भ्यः ।  \newline


\textbf{Krama Paata} \newline

नम॑ आव्या॒धिनी᳚भ्यः । आ॒व्या॒धिनी᳚भ्यो वि॒विद्ध्य॑न्तीभ्यः । आ॒व्या॒धिनी᳚भ्य॒ इत्या᳚ - व्या॒धिनी᳚भ्यः । वि॒विद्ध्य॑न्तीभ्यश्च । वि॒विद्ध्य॑न्तीभ्य॒ इति॑ वि - विद्ध्य॑न्तीभ्यः । च॒ वः॒ । वो॒ नमः॑ । नमो॒ नमः॑ । नम॒ उग॑णाभ्यः । उग॑णाभ्यस्तृꣳह॒तीभ्यः॑ । तृꣳ॒॒ह॒तीभ्य॑श्च । च॒ वः॒ । वो॒ नमः॑ । नमो॒ नमः॑ । नमो॑ गृ॒थ्सेभ्यः॑ । गृ॒थ्सेभ्यो॑ गृ॒थ्सप॑तिभ्यः । गृ॒थ्सप॑तिभ्यश्च । गृ॒थ्सप॑तिभ्य॒ इति॑ गृ॒थ्सप॑ति - भ्यः॒ । च॒ वः॒ । वो॒ नमः॑ । नमो॒ नमः॑ । नमो॒ व्राते᳚भ्यः । व्राते᳚भ्यो॒ व्रात॑पतिभ्यः । व्रात॑पतिभ्यश्च । व्रात॑पतिभ्य॒ इति॒ व्रात॑पति - भ्यः॒ । च॒ वः॒ । वो॒ नमः॑ । नमो॒ नमः॑ । नमो॑ ग॒णेभ्यः॑ । ग॒णेभ्यो॑ ग॒णप॑तिभ्यः । ग॒णप॑तिभ्यश्च । ग॒णप॑तिभ्य॒ इति॑ ग॒णप॑ति - भ्यः॒ । च॒ वः॒ । वो॒ नमः॑ । नमो॒ नमः॑ । नमो॒ विरू॑पेभ्यः । विरू॑पेभ्यो वि॒श्वरू॑पेभ्यः । विरू॑पेभ्य॒ इति॒ वि - रू॒पे॒भ्यः॒ । वि॒श्वरू॑पेभ्यश्च । वि॒श्वरू॑पेभ्य॒ इति॑ वि॒श्व - रू॒पे॒भ्यः॒ । च॒ वः॒ । वो॒ नमः॑ । नमो॒ नमः॑ । नमो॑ म॒हद्भ्यः॑ । म॒हद्भ्यः॑ क्षुल्ल॒केभ्यः॑ । म॒हद्भ्य॒ इति॑ म॒हत् - भ्यः॒ । क्षु॒ल्ल॒केभ्य॑श्च । च॒ वः॒ । वो॒ नमः॑ । नमो॒ नमः॑ । नमो॑ र॒थिभ्यः॑ । र॒थिभ्यो॑ऽर॒थेभ्यः॑ । र॒थिभ्य॒ इति॑ र॒थि - भ्यः॒ । अ॒र॒थेभ्य॑श्च । च॒ वः॒ । वो॒ नमः॑ । नमो॒ नमः॑ । नमो॒ रथे᳚भ्यः । रथे᳚भ्यो॒ रथ॑पतिभ्यः \newline

\textbf{Jatai Paata} \newline

1. नम॑ आव्या॒धिनी᳚भ्य आव्या॒धिनी᳚भ्यो॒ नमो॒ नम॑ आव्या॒धिनी᳚भ्यः । \newline
2. आ॒व्या॒धिनी᳚भ्यो वि॒विद्ध्य॑न्तीभ्यो वि॒विद्ध्य॑न्तीभ्य आव्या॒धिनी᳚भ्य आव्या॒धिनी᳚भ्यो वि॒विद्ध्य॑न्तीभ्यः । \newline
3. आ॒व्या॒धिनी᳚भ्य॒ इत्या᳚ - व्या॒धिनी᳚भ्यः । \newline
4. वि॒विद्ध्य॑न्तीभ्यश्च च वि॒विद्ध्य॑न्तीभ्यो वि॒विद्ध्य॑न्तीभ्यश्च । \newline
5. वि॒विध्य॑न्तीभ्य॒ इति॑ वि - विध्य॑न्तीभ्यः । \newline
6. च॒ वो॒ व॒श्च॒ च॒ वः॒ । \newline
7. वो॒ नमो॒ नमो॑ वो वो॒ नमः॑ । \newline
8. नमो॒ नमः॑ । \newline
9. नम॒ उग॑णाभ्य॒ उग॑णाभ्यो॒ नमो॒ नम॒ उग॑णाभ्यः । \newline
10. उग॑णाभ्य स्तृꣳह॒तीभ्य॑ स्तृꣳह॒तीभ्य॒ उग॑णाभ्य॒ उग॑णाभ्य स्तृꣳह॒तीभ्यः॑ । \newline
11. तृ॒(ग्म्॒)ह॒तीभ्य॑श्च च तृꣳह॒तीभ्य॑ स्तृꣳह॒तीभ्य॑श्च । \newline
12. च॒ वो॒ व॒श्च॒ च॒ वः॒ । \newline
13. वो॒ नमो॒ नमो॑ वो वो॒ नमः॑ । \newline
14. नमो॒ नमः॑ । \newline
15. नमो॑ गृ॒थ्सेभ्यो॑ गृ॒थ्सेभ्यो॒ नमो॒ नमो॑ गृ॒थ्सेभ्यः॑ । \newline
16. गृ॒थ्सेभ्यो॑ गृ॒थ्सप॑तिभ्यो गृ॒थ्सप॑तिभ्यो गृ॒थ्सेभ्यो॑ गृ॒थ्सेभ्यो॑ गृ॒थ्सप॑तिभ्यः । \newline
17. गृ॒थ्सप॑तिभ्यश्च च गृ॒थ्सप॑तिभ्यो गृ॒थ्सप॑तिभ्यश्च । \newline
18. गृ॒थ्सप॑तिभ्य॒ इति॑ गृ॒थ्सप॑ति - भ्यः॒ । \newline
19. च॒ वो॒ व॒श्च॒ च॒ वः॒ । \newline
20. वो॒ नमो॒ नमो॑ वो वो॒ नमः॑ । \newline
21. नमो॒ नमः॑ । \newline
22. नमो॒ व्राते᳚भ्यो॒ व्राते᳚भ्यो॒ नमो॒ नमो॒ व्राते᳚भ्यः । \newline
23. व्राते᳚भ्यो॒ व्रात॑पतिभ्यो॒ व्रात॑पतिभ्यो॒ व्राते᳚भ्यो॒ व्राते᳚भ्यो॒ व्रात॑पतिभ्यः । \newline
24. व्रात॑पतिभ्यश्च च॒ व्रात॑पतिभ्यो॒ व्रात॑पतिभ्यश्च । \newline
25. व्रात॑पतिभ्य॒ इति॒ व्रात॑पति - भ्यः॒ । \newline
26. च॒ वो॒ व॒श्च॒ च॒ वः॒ । \newline
27. वो॒ नमो॒ नमो॑ वो वो॒ नमः॑ । \newline
28. नमो॒ नमः॑ । \newline
29. नमो॑ ग॒णेभ्यो॑ ग॒णेभ्यो॒ नमो॒ नमो॑ ग॒णेभ्यः॑ । \newline
30. ग॒णेभ्यो॑ ग॒णप॑तिभ्यो ग॒णप॑तिभ्यो ग॒णेभ्यो॑ ग॒णेभ्यो॑ ग॒णप॑तिभ्यः । \newline
31. ग॒णप॑तिभ्यश्च च ग॒णप॑तिभ्यो ग॒णप॑तिभ्यश्च । \newline
32. ग॒णप॑तिभ्य॒ इति॑ ग॒णप॑ति - भ्यः॒ । \newline
33. च॒ वो॒ व॒श्च॒ च॒ वः॒ । \newline
34. वो॒ नमो॒ नमो॑ वो वो॒ नमः॑ । \newline
35. नमो॒ नमः॑ । \newline
36. नमो॒ विरू॑पेभ्यो॒ विरू॑पेभ्यो॒ नमो॒ नमो॒ विरू॑पेभ्यः । \newline
37. विरू॑पेभ्यो वि॒श्वरू॑पेभ्यो वि॒श्वरू॑पेभ्यो॒ विरू॑पेभ्यो॒ विरू॑पेभ्यो वि॒श्वरू॑पेभ्यः । \newline
38. विरू॑पेभ्य॒ इति॒ वि - रू॒पे॒भ्यः॒ । \newline
39. वि॒श्वरू॑पेभ्यश्च च वि॒श्वरू॑पेभ्यो वि॒श्वरू॑पेभ्यश्च । \newline
40. वि॒श्वरू॑पेभ्य॒ इति॑ वि॒श्व - रू॒पे॒भ्यः॒ । \newline
41. च॒ वो॒ व॒श्च॒ च॒ वः॒ । \newline
42. वो॒ नमो॒ नमो॑ वो वो॒ नमः॑ । \newline
43. नमो॒ नमः॑ । \newline
44. नमो॑ म॒हद्भ्यो॑ म॒हद्भ्यो॒ नमो॒ नमो॑ म॒हद्भ्यः॑ । \newline
45. म॒हद्भ्यः॑ क्षुल्ल॒केभ्यः॑ क्षुल्ल॒केभ्यो॑ म॒हद्भ्यो॑ म॒हद्भ्यः॑ क्षुल्ल॒केभ्यः॑ । \newline
46. म॒हद्भ्य॒ इति॑ म॒हत् - भ्यः॒ । \newline
47. क्षु॒ल्ल॒केभ्य॑श्च च क्षुल्ल॒केभ्यः॑ क्षुल्ल॒केभ्य॑श्च । \newline
48. च॒ वो॒ व॒श्च॒ च॒ वः॒ । \newline
49. वो॒ नमो॒ नमो॑ वो वो॒ नमः॑ । \newline
50. नमो॒ नमः॑ । \newline
51. नमो॑ र॒थिभ्यो॑ र॒थिभ्यो॒ नमो॒ नमो॑ र॒थिभ्यः॑ । \newline
52. र॒थिभ्यो॑ ऽर॒थेभ्यो॑ ऽर॒थेभ्यो॑ र॒थिभ्यो॑ र॒थिभ्यो॑ ऽर॒थेभ्यः॑ । \newline
53. र॒थिभ्य॒ इति॑ र॒थि - भ्यः॒ । \newline
54. अ॒र॒थेभ्य॑श्च चार॒थेभ्यो॑ ऽर॒थेभ्य॑श्च । \newline
55. च॒ वो॒ व॒श्च॒ च॒ वः॒ । \newline
56. वो॒ नमो॒ नमो॑ वो वो॒ नमः॑ । \newline
57. नमो॒ नमः॑ । \newline
58. नमो॒ रथे᳚भ्यो॒ रथे᳚भ्यो॒ नमो॒ नमो॒ रथे᳚भ्यः । \newline
59. रथे᳚भ्यो॒ रथ॑पतिभ्यो॒ रथ॑पतिभ्यो॒ रथे᳚भ्यो॒ रथे᳚भ्यो॒ रथ॑पतिभ्यः । \newline

\textbf{Ghana Paata } \newline

1. नम॑ आव्या॒धिनी᳚भ्य आव्या॒धिनी᳚भ्यो॒ नमो॒ नम॑ आव्या॒धिनी᳚भ्यो वि॒विद्ध्य॑न्तीभ्यो वि॒विद्ध्य॑न्तीभ्य आव्या॒धिनी᳚भ्यो॒ नमो॒ नम॑ आव्या॒धिनी᳚भ्यो वि॒विद्ध्य॑न्तीभ्यः । \newline
2. आ॒व्या॒धिनी᳚भ्यो वि॒विद्ध्य॑न्तीभ्यो वि॒विद्ध्य॑न्तीभ्य आव्या॒धिनी᳚भ्य आव्या॒धिनी᳚भ्यो वि॒विद्ध्य॑न्तीभ्यश्च च वि॒विद्ध्य॑न्तीभ्य आव्या॒धिनी᳚भ्य आव्या॒धिनी᳚भ्यो वि॒विद्ध्य॑न्तीभ्यश्च । \newline
3. आ॒व्या॒धिनी᳚भ्य॒ इत्या᳚ - व्या॒धिनी᳚भ्यः । \newline
4. वि॒विद्ध्य॑न्तीभ्यश्च च वि॒विद्ध्य॑न्तीभ्यो वि॒विद्ध्य॑न्तीभ्यश्च वो वश्च वि॒विद्ध्य॑न्तीभ्यो वि॒विद्ध्य॑न्तीभ्यश्च वः । \newline
5. वि॒विध्य॑न्तीभ्य॒ इति॑ वि - विध्य॑न्तीभ्यः । \newline
6. च॒ वो॒ व॒श्च॒ च॒ वो॒ नमो॒ नमो॑ वश्च च वो॒ नमः॑ । \newline
7. वो॒ नमो॒ नमो॑ वो वो॒ नमो॒ नमः॑ । \newline
8. नमो॒ नमः॑ । \newline
9. नम॒ उग॑णाभ्य॒ उग॑णाभ्यो॒ नमो॒ नम॒ उग॑णाभ्य स्तृꣳह॒तीभ्य॑ स्तृꣳह॒तीभ्य॒ उग॑णाभ्यो॒ नमो॒ नम॒ उग॑णाभ्य स्तृꣳह॒तीभ्यः॑ । \newline
10. उग॑णाभ्य स्तृꣳह॒तीभ्य॑ स्तृꣳह॒तीभ्य॒ उग॑णाभ्य॒ उग॑णाभ्य स्तृꣳह॒तीभ्य॑श्च च तृꣳह॒तीभ्य॒ उग॑णाभ्य॒ उग॑णाभ्य स्तृꣳह॒तीभ्य॑श्च । \newline
11. तृꣳ॒॒ह॒तीभ्य॑श्च च तृꣳह॒तीभ्य॑ स्तृꣳह॒तीभ्य॑श्च वो वश्च तृꣳह॒तीभ्य॑ स्तृꣳह॒तीभ्य॑श्च वः । \newline
12. च॒ वो॒ व॒श्च॒ च॒ वो॒ नमो॒ नमो॑ वश्च च वो॒ नमः॑ । \newline
13. वो॒ नमो॒ नमो॑ वो वो॒ नमो॒ नमः॑ । \newline
14. नमो॒ नमः॑ । \newline
15. नमो॑ गृ॒थ्सेभ्यो॑ गृ॒थ्सेभ्यो॒ नमो॒ नमो॑ गृ॒थ्सेभ्यो॑ गृ॒थ्सप॑तिभ्यो गृ॒थ्सप॑तिभ्यो गृ॒थ्सेभ्यो॒ नमो॒ नमो॑ गृ॒थ्सेभ्यो॑ गृ॒थ्सप॑तिभ्यः । \newline
16. गृ॒थ्सेभ्यो॑ गृ॒थ्सप॑तिभ्यो गृ॒थ्सप॑तिभ्यो गृ॒थ्सेभ्यो॑ गृ॒थ्सेभ्यो॑ गृ॒थ्सप॑तिभ्यश्च च गृ॒थ्सप॑तिभ्यो गृ॒थ्सेभ्यो॑ गृ॒थ्सेभ्यो॑ गृ॒थ्सप॑तिभ्यश्च । \newline
17. गृ॒थ्सप॑तिभ्यश्च च गृ॒थ्सप॑तिभ्यो गृ॒थ्सप॑तिभ्यश्च वो वश्च गृ॒थ्सप॑तिभ्यो गृ॒थ्सप॑तिभ्यश्च वः । \newline
18. गृ॒थ्सप॑तिभ्य॒ इति॑ गृ॒थ्सप॑ति - भ्यः॒ । \newline
19. च॒ वो॒ व॒श्च॒ च॒ वो॒ नमो॒ नमो॑ वश्च च वो॒ नमः॑ । \newline
20. वो॒ नमो॒ नमो॑ वो वो॒ नमो॒ नमः॑ । \newline
21. नमो॒ नमः॑ । \newline
22. नमो॒ व्राते᳚भ्यो॒ व्राते᳚भ्यो॒ नमो॒ नमो॒ व्राते᳚भ्यो॒ व्रात॑पतिभ्यो॒ व्रात॑पतिभ्यो॒ व्राते᳚भ्यो॒ नमो॒ नमो॒ व्राते᳚भ्यो॒ व्रात॑पतिभ्यः । \newline
23. व्राते᳚भ्यो॒ व्रात॑पतिभ्यो॒ व्रात॑पतिभ्यो॒ व्राते᳚भ्यो॒ व्राते᳚भ्यो॒ व्रात॑पतिभ्यश्च च॒ व्रात॑पतिभ्यो॒ व्राते᳚भ्यो॒ व्राते᳚भ्यो॒ व्रात॑पतिभ्यश्च । \newline
24. व्रात॑पतिभ्यश्च च॒ व्रात॑पतिभ्यो॒ व्रात॑पतिभ्यश्च वो वश्च॒ व्रात॑पतिभ्यो॒ व्रात॑पतिभ्यश्च वः । \newline
25. व्रात॑पतिभ्य॒ इति॒ व्रात॑पति - भ्यः॒ । \newline
26. च॒ वो॒ व॒श्च॒ च॒ वो॒ नमो॒ नमो॑ वश्च च वो॒ नमः॑ । \newline
27. वो॒ नमो॒ नमो॑ वो वो॒ नमो॒ नमः॑ । \newline
28. नमो॒ नमः॑ । \newline
29. नमो॑ ग॒णेभ्यो॑ ग॒णेभ्यो॒ नमो॒ नमो॑ ग॒णेभ्यो॑ ग॒णप॑तिभ्यो ग॒णप॑तिभ्यो ग॒णेभ्यो॒ नमो॒ नमो॑ ग॒णेभ्यो॑ ग॒णप॑तिभ्यः । \newline
30. ग॒णेभ्यो॑ ग॒णप॑तिभ्यो ग॒णप॑तिभ्यो ग॒णेभ्यो॑ ग॒णेभ्यो॑ ग॒णप॑तिभ्यश्च च ग॒णप॑तिभ्यो ग॒णेभ्यो॑ ग॒णेभ्यो॑ ग॒णप॑तिभ्यश्च । \newline
31. ग॒णप॑तिभ्यश्च च ग॒णप॑तिभ्यो ग॒णप॑तिभ्यश्च वो वश्च ग॒णप॑तिभ्यो ग॒णप॑तिभ्यश्च वः । \newline
32. ग॒णप॑तिभ्य॒ इति॑ ग॒णप॑ति - भ्यः॒ । \newline
33. च॒ वो॒ व॒श्च॒ च॒ वो॒ नमो॒ नमो॑ वश्च च वो॒ नमः॑ । \newline
34. वो॒ नमो॒ नमो॑ वो वो॒ नमो॒ नमः॑ । \newline
35. नमो॒ नमः॑ । \newline
36. नमो॒ विरू॑पेभ्यो॒ विरू॑पेभ्यो॒ नमो॒ नमो॒ विरू॑पेभ्यो वि॒श्वरू॑पेभ्यो वि॒श्वरू॑पेभ्यो॒ विरू॑पेभ्यो॒ नमो॒ नमो॒ विरू॑पेभ्यो वि॒श्वरू॑पेभ्यः । \newline
37. विरू॑पेभ्यो वि॒श्वरू॑पेभ्यो वि॒श्वरू॑पेभ्यो॒ विरू॑पेभ्यो॒ विरू॑पेभ्यो वि॒श्वरू॑पेभ्यश्च च वि॒श्वरू॑पेभ्यो॒ विरू॑पेभ्यो॒ विरू॑पेभ्यो वि॒श्वरू॑पेभ्यश्च । \newline
38. विरू॑पेभ्य॒ इति॒ वि - रू॒पे॒भ्यः॒ । \newline
39. वि॒श्वरू॑पेभ्यश्च च वि॒श्वरू॑पेभ्यो वि॒श्वरू॑पेभ्यश्च वो वश्च वि॒श्वरू॑पेभ्यो वि॒श्वरू॑पेभ्यश्च वः । \newline
40. वि॒श्वरू॑पेभ्य॒ इति॑ वि॒श्व - रू॒पे॒भ्यः॒ । \newline
41. च॒ वो॒ व॒श्च॒ च॒ वो॒ नमो॒ नमो॑ वश्च च वो॒ नमः॑ । \newline
42. वो॒ नमो॒ नमो॑ वो वो॒ नमो॒ नमः॑ । \newline
43. नमो॒ नमः॑ । \newline
44. नमो॑ म॒हद्भ्यो॑ म॒हद्भ्यो॒ नमो॒ नमो॑ म॒हद्भ्यः॑ क्षुल्ल॒केभ्यः॑ क्षुल्ल॒केभ्यो॑ म॒हद्भ्यो॒ नमो॒ नमो॑ म॒हद्भ्यः॑ क्षुल्ल॒केभ्यः॑ । \newline
45. म॒हद्भ्यः॑ क्षुल्ल॒केभ्यः॑ क्षुल्ल॒केभ्यो॑ म॒हद्भ्यो॑ म॒हद्भ्यः॑ क्षुल्ल॒केभ्य॑श्च च क्षुल्ल॒केभ्यो॑ म॒हद्भ्यो॑ म॒हद्भ्यः॑ क्षुल्ल॒केभ्य॑श्च । \newline
46. म॒हद्भ्य॒ इति॑ म॒हत् - भ्यः॒ । \newline
47. क्षु॒ल्ल॒केभ्य॑श्च च क्षुल्ल॒केभ्यः॑ क्षुल्ल॒केभ्य॑श्च वो वश्च क्षुल्ल॒केभ्यः॑ क्षुल्ल॒केभ्य॑श्च वः । \newline
48. च॒ वो॒ व॒श्च॒ च॒ वो॒ नमो॒ नमो॑ वश्च च वो॒ नमः॑ । \newline
49. वो॒ नमो॒ नमो॑ वो वो॒ नमो॒ नमः॑ । \newline
50. नमो॒ नमः॑ । \newline
51. नमो॑ र॒थिभ्यो॑ र॒थिभ्यो॒ नमो॒ नमो॑ र॒थिभ्यो॑ ऽर॒थेभ्यो॑ ऽर॒थेभ्यो॑ र॒थिभ्यो॒ नमो॒ नमो॑ र॒थिभ्यो॑ ऽर॒थेभ्यः॑ । \newline
52. र॒थिभ्यो॑ ऽर॒थेभ्यो॑ ऽर॒थेभ्यो॑ र॒थिभ्यो॑ र॒थिभ्यो॑ ऽर॒थेभ्य॑श्च चार॒थेभ्यो॑ र॒थिभ्यो॑ र॒थिभ्यो॑ ऽर॒थेभ्य॑श्च । \newline
53. र॒थिभ्य॒ इति॑ र॒थि - भ्यः॒ । \newline
54. अ॒र॒थेभ्य॑श्च चार॒थेभ्यो॑ ऽर॒थेभ्य॑श्च वो वश्चार॒थेभ्यो॑ ऽर॒थेभ्य॑श्च वः । \newline
55. च॒ वो॒ व॒श्च॒ च॒ वो॒ नमो॒ नमो॑ वश्च च वो॒ नमः॑ । \newline
56. वो॒ नमो॒ नमो॑ वो वो॒ नमो॒ नमः॑ । \newline
57. नमो॒ नमः॑ । \newline
58. नमो॒ रथे᳚भ्यो॒ रथे᳚भ्यो॒ नमो॒ नमो॒ रथे᳚भ्यो॒ रथ॑पतिभ्यो॒ रथ॑पतिभ्यो॒ रथे᳚भ्यो॒ नमो॒ नमो॒ रथे᳚भ्यो॒ रथ॑पतिभ्यः । \newline
59. रथे᳚भ्यो॒ रथ॑पतिभ्यो॒ रथ॑पतिभ्यो॒ रथे᳚भ्यो॒ रथे᳚भ्यो॒ रथ॑पतिभ्यश्च च॒ रथ॑पतिभ्यो॒ रथे᳚भ्यो॒ रथे᳚भ्यो॒ रथ॑पतिभ्यश्च । \newline
\pagebreak
\markright{ TS 4.5.4.2  \hfill https://www.vedavms.in \hfill}

\section{ TS 4.5.4.2 }

\textbf{TS 4.5.4.2 } \newline
\textbf{Samhita Paata} \newline

रथ॑पतिभ्यश्च वो॒ नमो॒                                                      नमः॒ सेना᳚भ्यः सेना॒निभ्य॑श्च वो॒ नमो॒                                               नमः॑ क्ष॒त्तृभ्यः॑ संग्रही॒तृभ्य॑श्च वो॒ नमो॒                                     नम॒स्तक्ष॑भ्यो रथका॒रेभ्य॑श्च वो॒ नमो॒                                                    नमः॒ कुला॑लेभ्यः क॒र्मारे᳚भ्यश्च वो॒ नमो॒                                            नमः॑ पु॒ञ्जिष्टे᳚भ्यो निषा॒देभ्य॑श्च वो॒ नमो॒                                                नम॑ इषु॒कृद्भ्यो॑ धन्व॒कृद्भ्य॑श्च वो॒ नमो॒                                                नमो॑ मृग॒युभ्यः॑ श्व॒निभ्य॑श्च वो॒ नमो॒                                                     नमः᳡श्वभ्यः॒ श्वप॑तिभ्यश्च ( ) वो॒ नमः॑ ॥ \newline

\textbf{Pada Paata} \newline

रथ॑पतिभ्य॒ इति॒ रथ॑पति-भ्यः॒ । च॒ । वः॒ । नमः॑ । नमः॑ । सेना᳚भ्यः । से॒ना॒निभ्य॒ इति॑ सेना॒नि - भ्यः॒ । च॒ । वः॒ । नमः॑ । नमः॑ । क्ष॒त्तृभ्य॒ इति॑ क्ष॒त्तृ - भ्यः॒ । स॒ग्रं॒ही॒तृभ्य॒ इति॑ संग्रही॒तृ - भ्यः॒ । च॒ । वः॒ । नमः॑ । नमः॑ । तक्ष॑भ्य॒ इति॒ तक्ष॑ - भ्यः॒ । र॒थ॒का॒रेभ्य॒ इति॑ रथ-का॒रेभ्यः॑ । च॒ । वः॒ । नमः॑ । नमः॑ । कुला॑लेभ्यः । क॒र्मारे᳚भ्यः । च॒ । वः॒ । नमः॑ । नमः॑ । पु॒ञ्जिष्टे᳚भ्यः । नि॒षा॒देभ्यः॑ । च॒ । वः॒ । नमः॑ । नमः॑ । इ॒षु॒कृद्भ्य॒ इती॑षु॒कृत् - भ्यः॒ । ध॒न्व॒कृद्भ्य॒ इति॑ धन्व॒कृत् - भ्यः॒ । च॒ । वः॒ । नमः॑ । नमः॑ । मृ॒ग॒युभ्य॒ इति॑ मृग॒यु-भ्यः॒ । श्व॒निभ्य॒ इति॑ श्व॒नि - भ्यः॒ । च॒ । वः॒ । नमः॑ । नमः॑ । श्वभ्य॒ इति॒ श्व - भ्यः॒ । श्वप॑तिभ्य॒ इति॒ श्वप॑ति - भ्यः॒ । च॒ ( ) । वः॒ । नमः॑ ॥  \newline


\textbf{Krama Paata} \newline

रथ॑पतिभ्यश्च । रथ॑पतिभ्य॒ इति॒ रथ॑पति - भ्यः॒ । च॒ वः॒ । वो॒ नमः॑ । नमो॒ नमः॑ । नमः॒ सेना᳚भ्यः । सेना᳚भ्यः सेना॒निभ्यः॑ । से॒ना॒निभ्य॑श्च । से॒ना॒निभ्य॒ इति॑ सेना॒नि - भ्यः॒ । च॒ वः॒ । वो॒ नमः॑ । नमो॒ नमः॑ । नमः॑ क्ष॒त्तृभ्यः॑ । क्ष॒त्तृभ्यः॑ सङ्ग्रही॒तृभ्यः॑ । क्ष॒त्तृभ्यः॒ इति॑ क्ष॒त्तृ - भ्यः॒ । स॒ङ्ग्र॒ही॒तृभ्य॑श्च । स॒ङ्ग्र॒ही॒तृभ्य॒ इति॑ सङ्ग्रही॒तृ - भ्यः॒ । च॒ वः॒ । वो॒ नमः॑ । नमो॒ नमः॑ । नम॒स्तक्ष॑भ्यः । तक्ष॑भ्यो रथका॒रेभ्यः॑ । तक्ष॑भ्य॒ इति॒ तक्ष॑ - भ्यः॒ । र॒थ॒का॒रेभ्य॑श्च । र॒थ॒का॒रेभ्य॒ इति॑ रथ - का॒रेभ्यः॑ । च॒ वः॒ । वो॒ नमः॑ । नमो॒ नमः॑ । नमः॒ कुला॑लेभ्यः । कुला॑लेभ्यः क॒र्मारे᳚भ्यः । क॒र्मारे᳚भ्यश्च । च॒ वः॒ । वो॒ नमः॑ । नमो॒ नमः॑ । नमः॑ पु॒ञ्जिष्टे᳚भ्यः । पु॒ञ्जिष्टे᳚भ्यो निषा॒देभ्यः॑ । नि॒षा॒देभ्य॑श्च । च॒ वः॒ । वो॒ नमः॑ । नमो॒ नमः॑ । नम॑ इषु॒कृद्भ्यः॑ । इ॒षु॒कृद्भ्यो॑ धन्व॒कृद्भ्यः॑ । इ॒षु॒कृद्भ्य॒ इती॑षु॒कृत् - भ्यः॒ । ध॒न्व॒कृद्भ्य॑श्च । ध॒न्व॒कृद्भ्य॒ इति॑ धन्व॒कृत् - भ्यः॒ । च॒ वः॒ । वो॒ नमः॑ । नमो॒ नमः॑ । नमो॑ मृग॒युभ्यः॑ । मृ॒ग॒युभ्यः॑ श्व॒निभ्यः॑ । मृ॒ग॒युभ्य॒ इति॑ मृग॒यु - भ्यः॒ । श्व॒निभ्य॑श्च । श्व॒निभ्य॒ इति॑ श्व॒नि - भ्यः॒ । च॒ वः॒ । वो॒ नमः॑ । नमो॒ नमः॑ । नमः॒ श्वभ्यः॑ । श्वभ्यः॒ श्वप॑तिभ्यः । श्वभ्य॒ इति॒ श्व - भ्यः॒ । श्वप॑तिभ्यश्च ( ) । श्वप॑तिभ्य॒ इति॒ श्वप॑ति - भ्यः॒ । च॒ वः॒ । वो॒ नमः॑ । नम॒ इति॒ नमः॑ । \newline

\textbf{Jatai Paata} \newline

1. रथ॑पतिभ्यश्च च॒ रथ॑पतिभ्यो॒ रथ॑पतिभ्यश्च । \newline
2. रथ॑पतिभ्य॒ इति॒ रथ॑पति - भ्यः॒ । \newline
3. च॒ वो॒ व॒श्च॒ च॒ वः॒ । \newline
4. वो॒ नमो॒ नमो॑ वो वो॒ नमः॑ । \newline
5. नमो॒ नमः॑ । \newline
6. नमः॒ सेना᳚भ्यः॒ सेना᳚भ्यो॒ नमो॒ नमः॒ सेना᳚भ्यः । \newline
7. सेना᳚भ्यः सेना॒निभ्यः॑ सेना॒निभ्यः॒ सेना᳚भ्यः॒ सेना᳚भ्यः सेना॒निभ्यः॑ । \newline
8. से॒ना॒निभ्य॑श्च च सेना॒निभ्यः॑ सेना॒निभ्य॑श्च । \newline
9. से॒ना॒निभ्य॒ इति॑ सेना॒नि - भ्यः॒ । \newline
10. च॒ वो॒ व॒श्च॒ च॒ वः॒ । \newline
11. वो॒ नमो॒ नमो॑ वो वो॒ नमः॑ । \newline
12. नमो॒ नमः॑ । \newline
13. नमः॑ क्ष॒त्तृभ्यः॑ क्ष॒त्तृभ्यो॒ नमो॒ नमः॑ क्ष॒त्तृभ्यः॑ । \newline
14. क्ष॒त्तृभ्यः॑ सङ्ग्रही॒तृभ्यः॑ सङ्ग्रही॒तृभ्यः॑ क्ष॒त्तृभ्यः॑ क्ष॒त्तृभ्यः॑ सङ्ग्रही॒तृभ्यः॑ । \newline
15. क्ष॒त्तृभ्य॒ इति॑ क्ष॒त्तृ - भ्यः॒ । \newline
16. स॒ङ्ग्र॒ही॒तृभ्य॑श्च च सङ्ग्रही॒तृभ्यः॑ सङ्ग्रही॒तृभ्य॑श्च । \newline
17. स॒ङ्ग्र॒ही॒तृभ्य॒ इति॑ सङ्ग्रही॒तृ - भ्यः॒ । \newline
18. च॒ वो॒ व॒श्च॒ च॒ वः॒ । \newline
19. वो॒ नमो॒ नमो॑ वो वो॒ नमः॑ । \newline
20. नमो॒ नमः॑ । \newline
21. नम॒ स्तक्ष॑भ्य॒ स्तक्ष॑भ्यो॒ नमो॒ नम॒ स्तक्ष॑भ्यः । \newline
22. तक्ष॑भ्यो रथका॒रेभ्यो॑ रथका॒रेभ्य॒ स्तक्ष॑भ्य॒ स्तक्ष॑भ्यो रथका॒रेभ्यः॑ । \newline
23. तक्ष॑भ्य॒ इति॒ तक्ष॑ - भ्यः॒ । \newline
24. र॒थ॒का॒रेभ्य॑श्च च रथका॒रेभ्यो॑ रथका॒रेभ्य॑श्च । \newline
25. र॒थ॒का॒रेभ्य॒ इति॑ रथ - का॒रेभ्यः॑ । \newline
26. च॒ वो॒ व॒श्च॒ च॒ वः॒ । \newline
27. वो॒ नमो॒ नमो॑ वो वो॒ नमः॑ । \newline
28. नमो॒ नमः॑ । \newline
29. नमः॒ कुला॑लेभ्यः॒ कुला॑लेभ्यो॒ नमो॒ नमः॒ कुला॑लेभ्यः । \newline
30. कुला॑लेभ्यः क॒र्मारे᳚भ्यः क॒र्मारे᳚भ्यः॒ कुला॑लेभ्यः॒ कुला॑लेभ्यः क॒र्मारे᳚भ्यः । \newline
31. क॒र्मारे᳚भ्यश्च च क॒र्मारे᳚भ्यः क॒र्मारे᳚भ्यश्च । \newline
32. च॒ वो॒ व॒श्च॒ च॒ वः॒ । \newline
33. वो॒ नमो॒ नमो॑ वो वो॒ नमः॑ । \newline
34. नमो॒ नमः॑ । \newline
35. नमः॑ पु॒ञ्जिष्टे᳚भ्यः पु॒ञ्जिष्टे᳚भ्यो॒ नमो॒ नमः॑ पु॒ञ्जिष्टे᳚भ्यः । \newline
36. पु॒ञ्जिष्टे᳚भ्यो निषा॒देभ्यो॑ निषा॒देभ्यः॑ पु॒ञ्जिष्टे᳚भ्यः पु॒ञ्जिष्टे᳚भ्यो निषा॒देभ्यः॑ । \newline
37. नि॒षा॒देभ्य॑श्च च निषा॒देभ्यो॑ निषा॒देभ्य॑श्च । \newline
38. च॒ वो॒ व॒श्च॒ च॒ वः॒ । \newline
39. वो॒ नमो॒ नमो॑ वो वो॒ नमः॑ । \newline
40. नमो॒ नमः॑ । \newline
41. नम॑ इषु॒कृद्भ्य॑ इषु॒कृद्भ्यो॒ नमो॒ नम॑ इषु॒कृद्भ्यः॑ । \newline
42. इ॒षु॒कृद्भ्यो॑ धन्व॒कृद्भ्यो॑ धन्व॒कृद्भ्य॑ इषु॒कृद्भ्य॑ इषु॒कृद्भ्यो॑ धन्व॒कृद्भ्यः॑ । \newline
43. इ॒षु॒कृद्भ्य॒ इती॑षु॒कृत् - भ्यः॒ । \newline
44. ध॒न्व॒कृद्भ्य॑श्च च धन्व॒कृद्भ्यो॑ धन्व॒कृद्भ्य॑श्च । \newline
45. ध॒न्व॒कृद्भ्य॒ इति॑ धन्व॒कृत् - भ्यः॒ । \newline
46. च॒ वो॒ व॒श्च॒ च॒ वः॒ । \newline
47. वो॒ नमो॒ नमो॑ वो वो॒ नमः॑ । \newline
48. नमो॒ नमः॑ । \newline
49. नमो॑ मृग॒युभ्यो॑ मृग॒युभ्यो॒ नमो॒ नमो॑ मृग॒युभ्यः॑ । \newline
50. मृ॒ग॒युभ्यः॑ श्व॒निभ्यः॑ श्व॒निभ्यो॑ मृग॒युभ्यो॑ मृग॒युभ्यः॑ श्व॒निभ्यः॑ । \newline
51. मृ॒ग॒युभ्य॒ इति॑ मृग॒यु - भ्यः॒ । \newline
52. श्व॒निभ्य॑श्च च श्व॒निभ्यः॑ श्व॒निभ्य॑श्च । \newline
53. श्व॒निभ्य॒ इति॑ श्व॒नि - भ्यः॒ । \newline
54. च॒ वो॒ व॒श्च॒ च॒ वः॒ । \newline
55. वो॒ नमो॒ नमो॑ वो वो॒ नमः॑ । \newline
56. नमो॒ नमः॑ । \newline
57. नमः॒ श्वभ्यः॒ श्वभ्यो॒ नमो॒ नमः॒ श्वभ्यः॑ । \newline
58. श्वभ्यः॒ श्वप॑तिभ्यः॒ श्वप॑तिभ्यः॒ श्वभ्यः॒ श्वभ्यः॒ श्वप॑तिभ्यः । \newline
59. श्वभ्य॒ इति॒ श्व - भ्यः॒ । \newline
60. श्वप॑तिभ्यश्च च॒ श्वप॑तिभ्यः॒ श्वप॑तिभ्यश्च । \newline
61. श्वप॑तिभ्य॒ इति॒ श्वप॑ति - भ्यः॒ । \newline
62. च॒ वो॒ व॒श्च॒ च॒ वः॒ । \newline
63. वो॒ नमो॒ नमो॑ वो वो॒ नमः॑ । \newline
64. नम॒ इति॒ नमः॑ । \newline

\textbf{Ghana Paata } \newline

1. रथ॑पतिभ्यश्च च॒ रथ॑पतिभ्यो॒ रथ॑पतिभ्यश्च वो वश्च॒ रथ॑पतिभ्यो॒ रथ॑पतिभ्यश्च वः । \newline
2. रथ॑पतिभ्य॒ इति॒ रथ॑पति - भ्यः॒ । \newline
3. च॒ वो॒ व॒श्च॒ च॒ वो॒ नमो॒ नमो॑ वश्च च वो॒ नमः॑ । \newline
4. वो॒ नमो॒ नमो॑ वो वो॒ नमो॒ नमः॑ । \newline
5. नमो॒ नमः॑ । \newline
6. नमः॒ सेना᳚भ्यः॒ सेना᳚भ्यो॒ नमो॒ नमः॒ सेना᳚भ्यः सेना॒निभ्यः॑ सेना॒निभ्यः॒ सेना᳚भ्यो॒ नमो॒ नमः॒ सेना᳚भ्यः सेना॒निभ्यः॑ । \newline
7. सेना᳚भ्यः सेना॒निभ्यः॑ सेना॒निभ्यः॒ सेना᳚भ्यः॒ सेना᳚भ्यः सेना॒निभ्य॑श्च च सेना॒निभ्यः॒ सेना᳚भ्यः॒ सेना᳚भ्यः सेना॒निभ्य॑श्च । \newline
8. से॒ना॒निभ्य॑श्च च सेना॒निभ्यः॑ सेना॒निभ्य॑श्च वो वश्च सेना॒निभ्यः॑ सेना॒निभ्य॑श्च वः । \newline
9. से॒ना॒निभ्य॒ इति॑ सेना॒नि - भ्यः॒ । \newline
10. च॒ वो॒ व॒श्च॒ च॒ वो॒ नमो॒ नमो॑ वश्च च वो॒ नमः॑ । \newline
11. वो॒ नमो॒ नमो॑ वो वो॒ नमो॒ नमः॑ । \newline
12. नमो॒ नमः॑ । \newline
13. नमः॑ क्ष॒त्तृभ्यः॑ क्ष॒त्तृभ्यो॒ नमो॒ नमः॑ क्ष॒त्तृभ्यः॑ सङ्ग्रही॒तृभ्यः॑ सङ्ग्रही॒तृभ्यः॑ क्ष॒त्तृभ्यो॒ नमो॒ नमः॑ क्ष॒त्तृभ्यः॑ सङ्ग्रही॒तृभ्यः॑ । \newline
14. क्ष॒त्तृभ्यः॑ सङ्ग्रही॒तृभ्यः॑ सङ्ग्रही॒तृभ्यः॑ क्ष॒त्तृभ्यः॑ क्ष॒त्तृभ्यः॑ सङ्ग्रही॒तृभ्य॑श्च च सङ्ग्रही॒तृभ्यः॑ क्ष॒त्तृभ्यः॑ क्ष॒त्तृभ्यः॑ सङ्ग्रही॒तृभ्य॑श्च । \newline
15. क्ष॒त्तृभ्य॒ इति॑ क्ष॒त्तृ - भ्यः॒ । \newline
16. स॒ङ्ग्र॒ही॒तृभ्य॑श्च च सङ्ग्रही॒तृभ्यः॑ सङ्ग्रही॒तृभ्य॑श्च वो वश्च सङ्ग्रही॒तृभ्यः॑ सङ्ग्रही॒तृभ्य॑श्च वः । \newline
17. स॒ङ्ग्र॒ही॒तृभ्य॒ इति॑ सङ्ग्रही॒तृ - भ्यः॒ । \newline
18. च॒ वो॒ व॒श्च॒ च॒ वो॒ नमो॒ नमो॑ वश्च च वो॒ नमः॑ । \newline
19. वो॒ नमो॒ नमो॑ वो वो॒ नमो॒ नमः॑ । \newline
20. नमो॒ नमः॑ । \newline
21. नम॒ स्तक्ष॑भ्य॒ स्तक्ष॑भ्यो॒ नमो॒ नम॒ स्तक्ष॑भ्यो रथका॒रेभ्यो॑ रथका॒रेभ्य॒ स्तक्ष॑भ्यो॒ नमो॒ नम॒ स्तक्ष॑भ्यो रथका॒रेभ्यः॑ । \newline
22. तक्ष॑भ्यो रथका॒रेभ्यो॑ रथका॒रेभ्य॒ स्तक्ष॑भ्य॒ स्तक्ष॑भ्यो रथका॒रेभ्य॑श्च च रथका॒रेभ्य॒ स्तक्ष॑भ्य॒ स्तक्ष॑भ्यो रथका॒रेभ्य॑श्च । \newline
23. तक्ष॑भ्य॒ इति॒ तक्ष॑ - भ्यः॒ । \newline
24. र॒थ॒का॒रेभ्य॑श्च च रथका॒रेभ्यो॑ रथका॒रेभ्य॑श्च वो वश्च रथका॒रेभ्यो॑ रथका॒रेभ्य॑श्च वः । \newline
25. र॒थ॒का॒रेभ्य॒ इति॑ रथ - का॒रेभ्यः॑ । \newline
26. च॒ वो॒ व॒श्च॒ च॒ वो॒ नमो॒ नमो॑ वश्च च वो॒ नमः॑ । \newline
27. वो॒ नमो॒ नमो॑ वो वो॒ नमो॒ नमः॑ । \newline
28. नमो॒ नमः॑ । \newline
29. नमः॒ कुला॑लेभ्यः॒ कुला॑लेभ्यो॒ नमो॒ नमः॒ कुला॑लेभ्यः क॒र्मारे᳚भ्यः क॒र्मारे᳚भ्यः॒ कुला॑लेभ्यो॒ नमो॒ नमः॒ कुला॑लेभ्यः क॒र्मारे᳚भ्यः । \newline
30. कुला॑लेभ्यः क॒र्मारे᳚भ्यः क॒र्मारे᳚भ्यः॒ कुला॑लेभ्यः॒ कुला॑लेभ्यः क॒र्मारे᳚भ्यश्च च क॒र्मारे᳚भ्यः॒ कुला॑लेभ्यः॒ कुला॑लेभ्यः क॒र्मारे᳚भ्यश्च । \newline
31. क॒र्मारे᳚भ्यश्च च क॒र्मारे᳚भ्यः क॒र्मारे᳚भ्यश्च वो वश्च क॒र्मारे᳚भ्यः क॒र्मारे᳚भ्यश्च वः । \newline
32. च॒ वो॒ व॒श्च॒ च॒ वो॒ नमो॒ नमो॑ वश्च च वो॒ नमः॑ । \newline
33. वो॒ नमो॒ नमो॑ वो वो॒ नमो॒ नमः॑ । \newline
34. नमो॒ नमः॑ । \newline
35. नमः॑ पु॒ञ्जिष्टे᳚भ्यः पु॒ञ्जिष्टे᳚भ्यो॒ नमो॒ नमः॑ पु॒ञ्जिष्टे᳚भ्यो निषा॒देभ्यो॑ निषा॒देभ्यः॑ पु॒ञ्जिष्टे᳚भ्यो॒ नमो॒ नमः॑ पु॒ञ्जिष्टे᳚भ्यो निषा॒देभ्यः॑ । \newline
36. पु॒ञ्जिष्टे᳚भ्यो निषा॒देभ्यो॑ निषा॒देभ्यः॑ पु॒ञ्जिष्टे᳚भ्यः पु॒ञ्जिष्टे᳚भ्यो निषा॒देभ्य॑श्च च निषा॒देभ्यः॑ पु॒ञ्जिष्टे᳚भ्यः पु॒ञ्जिष्टे᳚भ्यो निषा॒देभ्य॑श्च । \newline
37. नि॒षा॒देभ्य॑श्च च निषा॒देभ्यो॑ निषा॒देभ्य॑श्च वो वश्च निषा॒देभ्यो॑ निषा॒देभ्य॑श्च वः । \newline
38. च॒ वो॒ व॒श्च॒ च॒ वो॒ नमो॒ नमो॑ वश्च च वो॒ नमः॑ । \newline
39. वो॒ नमो॒ नमो॑ वो वो॒ नमो॒ नमः॑ । \newline
40. नमो॒ नमः॑ । \newline
41. नम॑ इषु॒कृद्भ्य॑ इषु॒कृद्भ्यो॒ नमो॒ नम॑ इषु॒कृद्भ्यो॑ धन्व॒कृद्भ्यो॑ धन्व॒कृद्भ्य॑ इषु॒कृद्भ्यो॒ नमो॒ नम॑ इषु॒कृद्भ्यो॑ धन्व॒कृद्भ्यः॑ । \newline
42. इ॒षु॒कृद्भ्यो॑ धन्व॒कृद्भ्यो॑ धन्व॒कृद्भ्य॑ इषु॒कृद्भ्य॑ इषु॒कृद्भ्यो॑ धन्व॒कृद्भ्य॑श्च च धन्व॒कृद्भ्य॑ इषु॒कृद्भ्य॑ इषु॒कृद्भ्यो॑ धन्व॒कृद्भ्य॑श्च । \newline
43. इ॒षु॒कृद्भ्य॒ इती॑षु॒कृत् - भ्यः॒ । \newline
44. ध॒न्व॒कृद्भ्य॑श्च च धन्व॒कृद्भ्यो॑ धन्व॒कृद्भ्य॑श्च वो वश्च धन्व॒कृद्भ्यो॑ धन्व॒कृद्भ्य॑श्च वः । \newline
45. ध॒न्व॒कृद्भ्य॒ इति॑ धन्व॒कृत् - भ्यः॒ । \newline
46. च॒ वो॒ व॒श्च॒ च॒ वो॒ नमो॒ नमो॑ वश्च च वो॒ नमः॑ । \newline
47. वो॒ नमो॒ नमो॑ वो वो॒ नमो॒ नमः॑ । \newline
48. नमो॒ नमः॑ । \newline
49. नमो॑ मृग॒युभ्यो॑ मृग॒युभ्यो॒ नमो॒ नमो॑ मृग॒युभ्यः॑ श्व॒निभ्यः॑ श्व॒निभ्यो॑ मृग॒युभ्यो॒ नमो॒ नमो॑ मृग॒युभ्यः॑ श्व॒निभ्यः॑ । \newline
50. मृ॒ग॒युभ्यः॑ श्व॒निभ्यः॑ श्व॒निभ्यो॑ मृग॒युभ्यो॑ मृग॒युभ्यः॑ श्व॒निभ्य॑श्च च श्व॒निभ्यो॑ मृग॒युभ्यो॑ मृग॒युभ्यः॑ श्व॒निभ्य॑श्च । \newline
51. मृ॒ग॒युभ्य॒ इति॑ मृग॒यु - भ्यः॒ । \newline
52. श्व॒निभ्य॑श्च च श्व॒निभ्यः॑ श्व॒निभ्य॑श्च वो वश्च श्व॒निभ्यः॑ श्व॒निभ्य॑श्च वः । \newline
53. श्व॒निभ्य॒ इति॑ श्व॒नि - भ्यः॒ । \newline
54. च॒ वो॒ व॒श्च॒ च॒ वो॒ नमो॒ नमो॑ वश्च च वो॒ नमः॑ । \newline
55. वो॒ नमो॒ नमो॑ वो वो॒ नमो॒ नमः॑ । \newline
56. नमो॒ नमः॑ । \newline
57. नमः॒ श्वभ्यः॒ श्वभ्यो॒ नमो॒ नमः॒ श्वभ्यः॒ श्वप॑तिभ्यः॒ श्वप॑तिभ्यः॒ श्वभ्यो॒ नमो॒ नमः॒ श्वभ्यः॒ श्वप॑तिभ्यः । \newline
58. श्वभ्यः॒ श्वप॑तिभ्यः॒ श्वप॑तिभ्यः॒ श्वभ्यः॒ श्वभ्यः॒ श्वप॑तिभ्यश्च च॒ श्वप॑तिभ्यः॒ श्वभ्यः॒ श्वभ्यः॒ श्वप॑तिभ्यश्च । \newline
59. श्वभ्य॒ इति॒ श्व - भ्यः॒ । \newline
60. श्वप॑तिभ्यश्च च॒ श्वप॑तिभ्यः॒ श्वप॑तिभ्यश्च वो वश्च॒ श्वप॑तिभ्यः॒ श्वप॑तिभ्यश्च वः । \newline
61. श्वप॑तिभ्य॒ इति॒ श्वप॑ति - भ्यः॒ । \newline
62. च॒ वो॒ व॒श्च॒ च॒ वो॒ नमो॒ नमो॑ वश्च च वो॒ नमः॑ । \newline
63. वो॒ नमो॒ नमो॑ वो वो॒ नमः॑ । \newline
64. नम॒ इति॒ नमः॑ । \newline
\pagebreak
\markright{ TS 4.5.5.1  \hfill https://www.vedavms.in \hfill}

\section{ TS 4.5.5.1 }

\textbf{TS 4.5.5.1 } \newline
\textbf{Samhita Paata} \newline

नमो॑ भ॒वाय॑ च रु॒द्राय॑ च॒ नमः॑ श॒र्वाय॑ च पशु॒पत॑ये च॒ नमो॒ नील॑ग्रीवाय च शिति॒कण्ठा॑य च॒ नमः॑ कप॒र्दिने॑ च॒ व्यु॑प्तकेशाय च॒ नमः॑ सहस्रा॒क्षाय॑ च श॒तध॑न्वने च॒ नमो॑ गिरि॒शाय॑ च शिपिवि॒ष्टाय॑ च॒ नमो॑ मी॒ढुष्ट॑माय॒ चेषु॑मते च॒ नमो᳚ ह्र॒स्वाय॑ च वाम॒नाय॑ च॒ नमो॑ बृह॒ते च॒ वर्.षी॑यसे च॒ नमो॑ वृ॒द्धाय॑ च सं॒ॅवृद्ध्व॑ने च॒-[  ] \newline

\textbf{Pada Paata} \newline

नमः॑ । भ॒वाय॑ । च॒ । रु॒द्राय॑ । च॒ । नमः॑ । श॒र्वाय॑ । च॒ । प॒शु॒पत॑य॒ इति॑ पशु-पत॑ये । च॒ । नमः॑ । नील॑ग्रीवा॒येति॒ नील॑ - ग्री॒वा॒य॒ । च॒ । शि॒ति॒कण्ठा॒येति॑ शिति - कण्ठा॑य । च॒ । नमः॑ । क॒प॒र्दिने᳚ । च॒ । व्यु॑प्तकेशा॒येति॒ व्यु॑प्त - के॒शा॒य॒ । च॒ । नमः॑ । स॒ह॒स्रा॒क्षायेति॑ सहस्र-अ॒क्षाय॑ । च॒ । श॒तध॑न्वन॒ इति॑ श॒त - ध॒न्व॒ने॒ । च॒ । नमः॑ । गि॒रि॒शाय॑ । च॒ । शि॒पि॒वि॒ष्टायेति॑ शिपि - वि॒ष्टाय॑ । च॒ । नमः॑ । मी॒ढुष्ट॑मा॒येति॑ मी॒ढुः - त॒मा॒य॒ । च॒ । इषु॑मत॒ इतीषु॑ - म॒ते॒ । च॒ । नमः॑ । ह्र॒स्वाय॑ । च॒ । वा॒म॒नाय॑ । च॒ । नमः॑ । बृ॒ह॒ते । च॒ । वर्.षी॑यसे । च॒ । नमः॑ । वृ॒द्धाय॑ । च॒ । सं॒ॅवृद्ध्व॑न॒ इति॑ सं - वृद्ध्व॑ने । च॒ ।  \newline


\textbf{Krama Paata} \newline

नमो॑ भ॒वाय॑ । भ॒वाय॑ च । च॒ रु॒द्राय॑ । रु॒द्राय॑ च । च॒ नमः॑ । नमः॑ श॒र्वाय॑ । श॒र्वाय॑ च । च॒ प॒शु॒पत॑ये । प॒शु॒पत॑ये च । प॒शु॒पत॑य॒ इति॑ पशु - पत॑ये । च॒ नमः॑ । नमो॒ नील॑ग्रीवाय । नील॑ग्रीवाय च । नील॑ग्रीवा॒येति॒ नील॑ - ग्री॒वा॒य॒ । च॒ शि॒ति॒कण्ठा॑य । शि॒ति॒कण्ठा॑य च । शि॒ति॒कण्ठा॒येति॑ शिति - कण्ठा॑य । च॒ नमः॑ । नमः॑ कप॒र्दिने᳚ । क॒प॒र्दिने॑ च । च॒ व्यु॑प्तकेशाय । व्यु॑प्तकेशाय च । व्यु॑प्तकेशा॒येति॒ व्यु॑प्त - के॒शा॒य॒ । च॒ नमः॑ । नमः॑ सहस्रा॒क्षाय॑ । स॒ह॒स्रा॒क्षाय॑ च । स॒ह॒स्रा॒क्षायेति॑ सहस्र - अ॒क्षाय॑ । च॒ श॒तध॑न्वने । श॒तध॑न्वने च । श॒तध॑न्वन॒ इति॑ श॒त - ध॒न्व॒ने॒ । च॒ नमः॑ । नमो॑ गिरि॒शाय॑ । गि॒रि॒शाय॑ च । च॒ शि॒पि॒वि॒ष्टाय॑ । शि॒पि॒वि॒ष्टाय॑ च । शि॒पि॒वि॒ष्टायेति॑ शिपि - वि॒ष्टाय॑ । च॒ नमः॑ । नमो॑ मी॒ढुष्ट॑माय । मी॒ढुष्ट॑माय च । मी॒ढुष्ट॑मा॒येति॑ मी॒ढुः - त॒मा॒य॒ । चेषु॑मते । इषु॑मते च । इषु॑मत॒ इतीषु॑ - म॒ते॒ । च॒ नमः॑ । नमो᳚ ह्र॒स्वाय॑ । ह्र॒स्वाय॑ च । च॒ वा॒म॒नाय॑ । वा॒म॒नाय॑ च । च॒ नमः॑ । नमो॑ बृह॒ते । बृ॒ह॒ते च॑ । च॒ वर्.षी॑यसे । वर्.षी॑यसे च । च॒ नमः॑ । नमो॑ वृ॒द्धाय॑ । वृ॒द्धाय॑ च । च॒ स॒म्ॅवृध्द्व॑ने । स॒म्ॅवृध्द्व॑ने च ( ) । स॒म्ॅवृध्व॑न॒ इति॑ सं - वृद्ध्व॑ने । च॒ नमः॑ \newline

\textbf{Jatai Paata} \newline

1. नमो॑ भ॒वाय॑ भ॒वाय॒ नमो॒ नमो॑ भ॒वाय॑ । \newline
2. भ॒वाय॑ च च भ॒वाय॑ भ॒वाय॑ च । \newline
3. च॒ रु॒द्राय॑ रु॒द्राय॑ च च रु॒द्राय॑ । \newline
4. रु॒द्राय॑ च च रु॒द्राय॑ रु॒द्राय॑ च । \newline
5. च॒ नमो॒ नम॑श्च च॒ नमः॑ । \newline
6. नमः॑ श॒र्वाय॑ श॒र्वाय॒ नमो॒ नमः॑ श॒र्वाय॑ । \newline
7. श॒र्वाय॑ च च श॒र्वाय॑ श॒र्वाय॑ च । \newline
8. च॒ प॒शु॒पत॑ये पशु॒पत॑ये च च पशु॒पत॑ये । \newline
9. प॒शु॒पत॑ये च च पशु॒पत॑ये पशु॒पत॑ये च । \newline
10. प॒शु॒पत॑य॒ इति॑ पशु - पत॑ये । \newline
11. च॒ नमो॒ नम॑श्च च॒ नमः॑ । \newline
12. नमो॒ नील॑ग्रीवाय॒ नील॑ग्रीवाय॒ नमो॒ नमो॒ नील॑ग्रीवाय । \newline
13. नील॑ग्रीवाय च च॒ नील॑ग्रीवाय॒ नील॑ग्रीवाय च । \newline
14. नील॑ग्रीवा॒येति॒ नील॑ - ग्री॒वा॒य॒ । \newline
15. च॒ शि॒ति॒कण्ठा॑य शिति॒कण्ठा॑य च च शिति॒कण्ठा॑य । \newline
16. शि॒ति॒कण्ठा॑य च च शिति॒कण्ठा॑य शिति॒कण्ठा॑य च । \newline
17. शि॒ति॒कण्ठा॒येति॑ शिति - कण्ठा॑य । \newline
18. च॒ नमो॒ नम॑श्च च॒ नमः॑ । \newline
19. नमः॑ कप॒र्दिने॑ कप॒र्दिने॒ नमो॒ नमः॑ कप॒र्दिने᳚ । \newline
20. क॒प॒र्दिने॑ च च कप॒र्दिने॑ कप॒र्दिने॑ च । \newline
21. च॒ व्यु॑प्तकेशाय॒ व्यु॑प्तकेशाय च च॒ व्यु॑प्तकेशाय । \newline
22. व्यु॑प्तकेशाय च च॒ व्यु॑प्तकेशाय॒ व्यु॑प्तकेशाय च । \newline
23. व्यु॑प्तकेशा॒येति॒ व्यु॑प्त - के॒शा॒य॒ । \newline
24. च॒ नमो॒ नम॑श्च च॒ नमः॑ । \newline
25. नमः॑ सहस्रा॒क्षाय॑ सहस्रा॒क्षाय॒ नमो॒ नमः॑ सहस्रा॒क्षाय॑ । \newline
26. स॒ह॒स्रा॒क्षाय॑ च च सहस्रा॒क्षाय॑ सहस्रा॒क्षाय॑ च । \newline
27. स॒ह॒स्रा॒क्षायेति॑ सहस्र - अ॒क्षाय॑ । \newline
28. च॒ श॒तध॑न्वने श॒तध॑न्वने च च श॒तध॑न्वने । \newline
29. श॒तध॑न्वने च च श॒तध॑न्वने श॒तध॑न्वने च । \newline
30. श॒तध॑न्वन॒ इति॑ श॒त - ध॒न्व॒ने॒ । \newline
31. च॒ नमो॒ नम॑श्च च॒ नमः॑ । \newline
32. नमो॑ गिरि॒शाय॑ गिरि॒शाय॒ नमो॒ नमो॑ गिरि॒शाय॑ । \newline
33. गि॒रि॒शाय॑ च च गिरि॒शाय॑ गिरि॒शाय॑ च । \newline
34. च॒ शि॒पि॒वि॒ष्टाय॑ शिपिवि॒ष्टाय॑ च च शिपिवि॒ष्टाय॑ । \newline
35. शि॒पि॒वि॒ष्टाय॑ च च शिपिवि॒ष्टाय॑ शिपिवि॒ष्टाय॑ च । \newline
36. शि॒पि॒वि॒ष्टायेति॑ शिपि - वि॒ष्टाय॑ । \newline
37. च॒ नमो॒ नम॑श्च च॒ नमः॑ । \newline
38. नमो॑ मी॒ढुष्ट॑माय मी॒ढुष्ट॑माय॒ नमो॒ नमो॑ मी॒ढुष्ट॑माय । \newline
39. मी॒ढुष्ट॑माय च च मी॒ढुष्ट॑माय मी॒ढुष्ट॑माय च । \newline
40. मी॒ढुष्ट॑मा॒येति॑ मी॒ढुः - त॒मा॒य॒ । \newline
41. चेषु॑मत॒ इषु॑मते च॒ चेषु॑मते । \newline
42. इषु॑मते च॒ चेषु॑मत॒ इषु॑मते च । \newline
43. इषु॑मत॒ इतीषु॑ - म॒ते॒ । \newline
44. च॒ नमो॒ नम॑श्च च॒ नमः॑ । \newline
45. नमो᳚ ह्र॒स्वाय॑ ह्र॒स्वाय॒ नमो॒ नमो᳚ ह्र॒स्वाय॑ । \newline
46. ह्र॒स्वाय॑ च च ह्र॒स्वाय॑ ह्र॒स्वाय॑ च । \newline
47. च॒ वा॒म॒नाय॑ वाम॒नाय॑ च च वाम॒नाय॑ । \newline
48. वा॒म॒नाय॑ च च वाम॒नाय॑ वाम॒नाय॑ च । \newline
49. च॒ नमो॒ नम॑श्च च॒ नमः॑ । \newline
50. नमो॑ बृह॒ते बृ॑ह॒ते नमो॒ नमो॑ बृह॒ते । \newline
51. बृ॒ह॒ते च॑ च बृह॒ते बृ॑ह॒ते च॑ । \newline
52. च॒ वर्.षी॑यसे॒ वर्.षी॑यसे च च॒ वर्.षी॑यसे । \newline
53. वर्.षी॑यसे च च॒ वर्.षी॑यसे॒ वर्.षी॑यसे च । \newline
54. च॒ नमो॒ नम॑श्च च॒ नमः॑ । \newline
55. नमो॑ वृ॒द्धाय॑ वृ॒द्धाय॒ नमो॒ नमो॑ वृ॒द्धाय॑ । \newline
56. वृ॒द्धाय॑ च च वृ॒द्धाय॑ वृ॒द्धाय॑ च । \newline
57. च॒ स॒म्ॅवृद्ध्व॑ने स॒म्ॅवृद्ध्व॑ने च च स॒म्ॅवृद्ध्व॑ने । \newline
58. स॒म्ॅवृद्ध्व॑ने च च स॒म्ॅवृद्ध्व॑ने स॒म्ॅवृद्ध्व॑ने च । \newline
59. सं॒ॅवृध्व॑न॒ इति॑ सं - वृध्व॑ने । \newline
60. च॒ नमो॒ नम॑श्च च॒ नमः॑ । \newline

\textbf{Ghana Paata } \newline

1. नमो॑ भ॒वाय॑ भ॒वाय॒ नमो॒ नमो॑ भ॒वाय॑ च च भ॒वाय॒ नमो॒ नमो॑ भ॒वाय॑ च । \newline
2. भ॒वाय॑ च च भ॒वाय॑ भ॒वाय॑ च रु॒द्राय॑ रु॒द्राय॑ च भ॒वाय॑ भ॒वाय॑ च रु॒द्राय॑ । \newline
3. च॒ रु॒द्राय॑ रु॒द्राय॑ च च रु॒द्राय॑ च च रु॒द्राय॑ च च रु॒द्राय॑ च । \newline
4. रु॒द्राय॑ च च रु॒द्राय॑ रु॒द्राय॑ च॒ नमो॒ नम॑श्च रु॒द्राय॑ रु॒द्राय॑ च॒ नमः॑ । \newline
5. च॒ नमो॒ नम॑श्च च॒ नमः॑ श॒र्वाय॑ श॒र्वाय॒ नम॑श्च च॒ नमः॑ श॒र्वाय॑ । \newline
6. नमः॑ श॒र्वाय॑ श॒र्वाय॒ नमो॒ नमः॑ श॒र्वाय॑ च च श॒र्वाय॒ नमो॒ नमः॑ श॒र्वाय॑ च । \newline
7. श॒र्वाय॑ च च श॒र्वाय॑ श॒र्वाय॑ च पशु॒पत॑ये पशु॒पत॑ये च श॒र्वाय॑ श॒र्वाय॑ च पशु॒पत॑ये । \newline
8. च॒ प॒शु॒पत॑ये पशु॒पत॑ये च च पशु॒पत॑ये च च पशु॒पत॑ये च च पशु॒पत॑ये च । \newline
9. प॒शु॒पत॑ये च च पशु॒पत॑ये पशु॒पत॑ये च॒ नमो॒ नम॑श्च पशु॒पत॑ये पशु॒पत॑ये च॒ नमः॑ । \newline
10. प॒शु॒पत॑य॒ इति॑ पशु - पत॑ये । \newline
11. च॒ नमो॒ नम॑श्च च॒ नमो॒ नील॑ग्रीवाय॒ नील॑ग्रीवाय॒ नम॑श्च च॒ नमो॒ नील॑ग्रीवाय । \newline
12. नमो॒ नील॑ग्रीवाय॒ नील॑ग्रीवाय॒ नमो॒ नमो॒ नील॑ग्रीवाय च च॒ नील॑ग्रीवाय॒ नमो॒ नमो॒ नील॑ग्रीवाय च । \newline
13. नील॑ग्रीवाय च च॒ नील॑ग्रीवाय॒ नील॑ग्रीवाय च शिति॒कण्ठा॑य शिति॒कण्ठा॑य च॒ नील॑ग्रीवाय॒ नील॑ग्रीवाय च शिति॒कण्ठा॑य । \newline
14. नील॑ग्रीवा॒येति॒ नील॑ - ग्री॒वा॒य॒ । \newline
15. च॒ शि॒ति॒कण्ठा॑य शिति॒कण्ठा॑य च च शिति॒कण्ठा॑य च च शिति॒कण्ठा॑य च च शिति॒कण्ठा॑य च । \newline
16. शि॒ति॒कण्ठा॑य च च शिति॒कण्ठा॑य शिति॒कण्ठा॑य च॒ नमो॒ नम॑श्च शिति॒कण्ठा॑य शिति॒कण्ठा॑य च॒ नमः॑ । \newline
17. शि॒ति॒कण्ठा॒येति॑ शिति - कण्ठा॑य । \newline
18. च॒ नमो॒ नम॑श्च च॒ नमः॑ कप॒र्दिने॑ कप॒र्दिने॒ नम॑श्च च॒ नमः॑ कप॒र्दिने᳚ । \newline
19. नमः॑ कप॒र्दिने॑ कप॒र्दिने॒ नमो॒ नमः॑ कप॒र्दिने॑ च च कप॒र्दिने॒ नमो॒ नमः॑ कप॒र्दिने॑ च । \newline
20. क॒प॒र्दिने॑ च च कप॒र्दिने॑ कप॒र्दिने॑ च॒ व्यु॑प्तकेशाय॒ व्यु॑प्तकेशाय च कप॒र्दिने॑ कप॒र्दिने॑ च॒ व्यु॑प्तकेशाय । \newline
21. च॒ व्यु॑प्तकेशाय॒ व्यु॑प्तकेशाय च च॒ व्यु॑प्तकेशाय च च॒ व्यु॑प्तकेशाय च च॒ व्यु॑प्तकेशाय च । \newline
22. व्यु॑प्तकेशाय च च॒ व्यु॑प्तकेशाय॒ व्यु॑प्तकेशाय च॒ नमो॒ नम॑श्च॒ व्यु॑प्तकेशाय॒ व्यु॑प्तकेशाय च॒ नमः॑ । \newline
23. व्यु॑प्तकेशा॒येति॒ व्यु॑प्त - के॒शा॒य॒ । \newline
24. च॒ नमो॒ नम॑श्च च॒ नमः॑ सहस्रा॒क्षाय॑ सहस्रा॒क्षाय॒ नम॑श्च च॒ नमः॑ सहस्रा॒क्षाय॑ । \newline
25. नमः॑ सहस्रा॒क्षाय॑ सहस्रा॒क्षाय॒ नमो॒ नमः॑ सहस्रा॒क्षाय॑ च च सहस्रा॒क्षाय॒ नमो॒ नमः॑ सहस्रा॒क्षाय॑ च । \newline
26. स॒ह॒स्रा॒क्षाय॑ च च सहस्रा॒क्षाय॑ सहस्रा॒क्षाय॑ च श॒तध॑न्वने श॒तध॑न्वने च सहस्रा॒क्षाय॑ सहस्रा॒क्षाय॑ च श॒तध॑न्वने । \newline
27. स॒ह॒स्रा॒क्षायेति॑ सहस्र - अ॒क्षाय॑ । \newline
28. च॒ श॒तध॑न्वने श॒तध॑न्वने च च श॒तध॑न्वने च च श॒तध॑न्वने च च श॒तध॑न्वने च । \newline
29. श॒तध॑न्वने च च श॒तध॑न्वने श॒तध॑न्वने च॒ नमो॒ नम॑श्च श॒तध॑न्वने श॒तध॑न्वने च॒ नमः॑ । \newline
30. श॒तध॑न्वन॒ इति॑ श॒त - ध॒न्व॒ने॒ । \newline
31. च॒ नमो॒ नम॑श्च च॒ नमो॑ गिरि॒शाय॑ गिरि॒शाय॒ नम॑श्च च॒ नमो॑ गिरि॒शाय॑ । \newline
32. नमो॑ गिरि॒शाय॑ गिरि॒शाय॒ नमो॒ नमो॑ गिरि॒शाय॑ च च गिरि॒शाय॒ नमो॒ नमो॑ गिरि॒शाय॑ च । \newline
33. गि॒रि॒शाय॑ च च गिरि॒शाय॑ गिरि॒शाय॑ च शिपिवि॒ष्टाय॑ शिपिवि॒ष्टाय॑ च गिरि॒शाय॑ गिरि॒शाय॑ च शिपिवि॒ष्टाय॑ । \newline
34. च॒ शि॒पि॒वि॒ष्टाय॑ शिपिवि॒ष्टाय॑ च च शिपिवि॒ष्टाय॑ च च शिपिवि॒ष्टाय॑ च च शिपिवि॒ष्टाय॑ च । \newline
35. शि॒पि॒वि॒ष्टाय॑ च च शिपिवि॒ष्टाय॑ शिपिवि॒ष्टाय॑ च॒ नमो॒ नम॑श्च शिपिवि॒ष्टाय॑ शिपिवि॒ष्टाय॑ च॒ नमः॑ । \newline
36. शि॒पि॒वि॒ष्टायेति॑ शिपि - वि॒ष्टाय॑ । \newline
37. च॒ नमो॒ नम॑श्च च॒ नमो॑ मी॒ढुष्ट॑माय मी॒ढुष्ट॑माय॒ नम॑श्च च॒ नमो॑ मी॒ढुष्ट॑माय । \newline
38. नमो॑ मी॒ढुष्ट॑माय मी॒ढुष्ट॑माय॒ नमो॒ नमो॑ मी॒ढुष्ट॑माय च च मी॒ढुष्ट॑माय॒ नमो॒ नमो॑ मी॒ढुष्ट॑माय च । \newline
39. मी॒ढुष्ट॑माय च च मी॒ढुष्ट॑माय मी॒ढुष्ट॑माय॒ चेषु॑मत॒ इषु॑मते च मी॒ढुष्ट॑माय मी॒ढुष्ट॑माय॒ चेषु॑मते । \newline
40. मी॒ढुष्ट॑मा॒येति॑ मी॒ढुः - त॒मा॒य॒ । \newline
41. चेषु॑मत॒ इषु॑मते च॒ चेषु॑मते च॒ चेषु॑मते च॒ चेषु॑मते च । \newline
42. इषु॑मते च॒ चेषु॑मत॒ इषु॑मते च॒ नमो॒ नम॒ श्चेषु॑मत॒ इषु॑मते च॒ नमः॑ । \newline
43. इषु॑मत॒ इतीषु॑ - म॒ते॒ । \newline
44. च॒ नमो॒ नम॑श्च च॒ नमो᳚ ह्र॒स्वाय॑ ह्र॒स्वाय॒ नम॑श्च च॒ नमो᳚ ह्र॒स्वाय॑ । \newline
45. नमो᳚ ह्र॒स्वाय॑ ह्र॒स्वाय॒ नमो॒ नमो᳚ ह्र॒स्वाय॑ च च ह्र॒स्वाय॒ नमो॒ नमो᳚ ह्र॒स्वाय॑ च । \newline
46. ह्र॒स्वाय॑ च च ह्र॒स्वाय॑ ह्र॒स्वाय॑ च वाम॒नाय॑ वाम॒नाय॑ च ह्र॒स्वाय॑ ह्र॒स्वाय॑ च वाम॒नाय॑ । \newline
47. च॒ वा॒म॒नाय॑ वाम॒नाय॑ च च वाम॒नाय॑ च च वाम॒नाय॑ च च वाम॒नाय॑ च । \newline
48. वा॒म॒नाय॑ च च वाम॒नाय॑ वाम॒नाय॑ च॒ नमो॒ नम॑श्च वाम॒नाय॑ वाम॒नाय॑ च॒ नमः॑ । \newline
49. च॒ नमो॒ नम॑श्च च॒ नमो॑ बृह॒ते बृ॑ह॒ते नम॑श्च च॒ नमो॑ बृह॒ते । \newline
50. नमो॑ बृह॒ते बृ॑ह॒ते नमो॒ नमो॑ बृह॒ते च॑ च बृह॒ते नमो॒ नमो॑ बृह॒ते च॑ । \newline
51. बृ॒ह॒ते च॑ च बृह॒ते बृ॑ह॒ते च॒ वर्.षी॑यसे॒ वर्.षी॑यसे च बृह॒ते बृ॑ह॒ते च॒ वर्.षी॑यसे । \newline
52. च॒ वर्.षी॑यसे॒ वर्.षी॑यसे च च॒ वर्.षी॑यसे च च॒ वर्.षी॑यसे च च॒ वर्.षी॑यसे च । \newline
53. वर्.षी॑यसे च च॒ वर्.षी॑यसे॒ वर्.षी॑यसे च॒ नमो॒ नम॑श्च॒ वर्.षी॑यसे॒ वर्.षी॑यसे च॒ नमः॑ । \newline
54. च॒ नमो॒ नम॑श्च च॒ नमो॑ वृ॒द्धाय॑ वृ॒द्धाय॒ नम॑श्च च॒ नमो॑ वृ॒द्धाय॑ । \newline
55. नमो॑ वृ॒द्धाय॑ वृ॒द्धाय॒ नमो॒ नमो॑ वृ॒द्धाय॑ च च वृ॒द्धाय॒ नमो॒ नमो॑ वृ॒द्धाय॑ च । \newline
56. वृ॒द्धाय॑ च च वृ॒द्धाय॑ वृ॒द्धाय॑ च स॒म्ॅवृद्ध्व॑ने स॒म्ॅवृद्ध्व॑ने च वृ॒द्धाय॑ वृ॒द्धाय॑ च स॒म्ॅवृद्ध्व॑ने । \newline
57. च॒ स॒म्ॅवृद्ध्व॑ने स॒म्ॅवृद्ध्व॑ने च च स॒म्ॅवृद्ध्व॑ने च च स॒म्ॅवृद्ध्व॑ने च च स॒म्ॅवृद्ध्व॑ने च । \newline
58. स॒म्ॅवृद्ध्व॑ने च च स॒म्ॅवृद्ध्व॑ने स॒म्ॅवृद्ध्व॑ने च॒ नमो॒ नम॑श्च स॒म्ॅवृद्ध्व॑ने स॒म्ॅवृद्ध्व॑ने च॒ नमः॑ । \newline
59. सं॒ॅवृध्व॑न॒ इति॑ सं - वृध्व॑ने । \newline
60. च॒ नमो॒ नम॑श्च च॒ नमो॒ अग्रि॑या॒या ग्रि॑याय॒ नम॑श्च च॒ नमो॒ अग्रि॑याय । \newline
\pagebreak
\markright{ TS 4.5.5.2  \hfill https://www.vedavms.in \hfill}

\section{ TS 4.5.5.2 }

\textbf{TS 4.5.5.2 } \newline
\textbf{Samhita Paata} \newline

नमो॒ अग्रि॑याय च प्रथ॒माय॑ च॒ नम॑ आ॒शवे॑ चाजि॒राय॑ च॒ नमः᳡शीघ्रि॑याय च॒ शीभ्या॑य च॒ नम॑ ऊ॒र्म्या॑य चावस्व॒न्या॑य च॒ नमः॑ स्रोत॒स्या॑य च॒ द्वीप्या॑य च ॥ \newline

\textbf{Pada Paata} \newline

नमः॑ । अग्रि॑याय । च॒ । प्र॒थ॒माय॑ । च॒ । नमः॑ । आ॒शवे᳚ । च॒ । अ॒जि॒राय॑ । च॒ । नमः॑ । शीघ्रि॑याय । च॒ । शीभ्या॑य । च॒ । नमः॑ । ऊ॒र्म्या॑य । च॒ । अ॒व॒स्व॒न्या॑येत्य॑व - स्व॒न्या॑य । च॒ । नमः॑ । स्रो॒त॒स्या॑य । च॒ । द्वीप्या॑य । च॒ ॥  \newline


\textbf{Krama Paata} \newline

नमो॒ अग्रि॑याय । अग्रि॑याय च । च॒ प्र॒थ॒माय॑ । प्र॒थ॒माय॑ च । च॒ नमः॑ । नम॑ आ॒शवे᳚ । आ॒शवे॑ च । चा॒जि॒राय॑ । अ॒जि॒राय॑ च । च॒ नमः॑ । नमः॒ शीघ्रि॑याय । शीघ्रि॑याय च । च॒ शीभ्या॑य । शीभ्या॑य च । च॒ नमः॑ । नम॑ ऊ॒र्म्या॑य । ऊ॒र्म्या॑य च । चा॒व॒स्व॒न्या॑य । अ॒व॒स्व॒न्या॑य च । अ॒व॒स्व॒न्या॑येत्य॑व - स्व॒न्या॑य । च॒ नमः॑ । नमः॑ स्रोत॒स्या॑य । स्रो॒त॒स्या॑य च । च॒ द्वीप्या॑य । द्वीप्या॑य च । चेति॑ च । \newline

\textbf{Jatai Paata} \newline

1. नमो॒ अग्रि॑या॒या ग्रि॑याय॒ नमो॒ नमो॒ अग्रि॑याय । \newline
2. अग्रि॑याय च॒ चाग्रि॑या॒या ग्रि॑याय च । \newline
3. च॒ प्र॒थ॒माय॑ प्रथ॒माय॑ च च प्रथ॒माय॑ । \newline
4. प्र॒थ॒माय॑ च च प्रथ॒माय॑ प्रथ॒माय॑ च । \newline
5. च॒ नमो॒ नम॑श्च च॒ नमः॑ । \newline
6. नम॑ आ॒शव॑ आ॒शवे॒ नमो॒ नम॑ आ॒शवे᳚ । \newline
7. आ॒शवे॑ च चा॒शव॑ आ॒शवे॑ च । \newline
8. चा॒जि॒राया॑ जि॒राय॑ च चाजि॒राय॑ । \newline
9. अ॒जि॒राय॑ च चाजि॒राया॑ जि॒राय॑ च । \newline
10. च॒ नमो॒ नम॑श्च च॒ नमः॑ । \newline
11. नमः॒ शीघ्रि॑याय॒ शीघ्रि॑याय॒ नमो॒ नमः॒ शीघ्रि॑याय । \newline
12. शीघ्रि॑याय च च॒ शीघ्रि॑याय॒ शीघ्रि॑याय च । \newline
13. च॒ शीभ्या॑य॒ शीभ्या॑य च च॒ शीभ्या॑य । \newline
14. शीभ्या॑य च च॒ शीभ्या॑य॒ शीभ्या॑य च । \newline
15. च॒ नमो॒ नम॑श्च च॒ नमः॑ । \newline
16. नम॑ ऊ॒र्म्या॑ यो॒र्म्या॑य॒ नमो॒ नम॑ ऊ॒र्म्या॑य । \newline
17. ऊ॒र्म्या॑य च चो॒र्म्या॑ यो॒र्म्या॑य च । \newline
18. चा॒व॒स्व॒न्या॑या वस्व॒न्या॑य च चावस्व॒न्या॑य । \newline
19. अ॒व॒स्व॒न्या॑य च चावस्व॒न्या॑या वस्व॒न्या॑य च । \newline
20. अ॒व॒स्व॒न्या॑येत्य॑व - स्व॒न्या॑य । \newline
21. च॒ नमो॒ नम॑श्च च॒ नमः॑ । \newline
22. नमः॑ स्रोत॒स्या॑य स्रोत॒स्या॑य॒ नमो॒ नमः॑ स्रोत॒स्या॑य । \newline
23. स्रो॒त॒स्या॑य च च स्रोत॒स्या॑य स्रोत॒स्या॑य च । \newline
24. च॒ द्वीप्या॑य॒ द्वीप्या॑य च च॒ द्वीप्या॑य । \newline
25. द्वीप्या॑य च च॒ द्वीप्या॑य॒ द्वीप्या॑य च । \newline
26. चेति॑ च । \newline

\textbf{Ghana Paata } \newline

1. नमो॒ अग्रि॑या॒या ग्रि॑याय॒ नमो॒ नमो॒ अग्रि॑याय च॒ चाग्रि॑याय॒ नमो॒ नमो॒ अग्रि॑याय च । \newline
2. अग्रि॑याय च॒ चाग्रि॑या॒या ग्रि॑याय च प्रथ॒माय॑ प्रथ॒माय॒ चाग्रि॑या॒या ग्रि॑याय च प्रथ॒माय॑ । \newline
3. च॒ प्र॒थ॒माय॑ प्रथ॒माय॑ च च प्रथ॒माय॑ च च प्रथ॒माय॑ च च प्रथ॒माय॑ च । \newline
4. प्र॒थ॒माय॑ च च प्रथ॒माय॑ प्रथ॒माय॑ च॒ नमो॒ नम॑श्च प्रथ॒माय॑ प्रथ॒माय॑ च॒ नमः॑ । \newline
5. च॒ नमो॒ नम॑श्च च॒ नम॑ आ॒शव॑ आ॒शवे॒ नम॑श्च च॒ नम॑ आ॒शवे᳚ । \newline
6. नम॑ आ॒शव॑ आ॒शवे॒ नमो॒ नम॑ आ॒शवे॑ च चा॒शवे॒ नमो॒ नम॑ आ॒शवे॑ च । \newline
7. आ॒शवे॑ च चा॒शव॑ आ॒शवे॑ चाजि॒राया॑ जि॒राय॑ चा॒शव॑ आ॒शवे॑ चाजि॒राय॑ । \newline
8. चा॒जि॒राया॑ जि॒राय॑ च चाजि॒राय॑ च चाजि॒राय॑ च चाजि॒राय॑ च । \newline
9. अ॒जि॒राय॑ च चाजि॒राया॑ जि॒राय॑ च॒ नमो॒ नम॑ श्चाजि॒राया॑ जि॒राय॑ च॒ नमः॑ । \newline
10. च॒ नमो॒ नम॑श्च च॒ नमः॒ शीघ्रि॑याय॒ शीघ्रि॑याय॒ नम॑श्च च॒ नमः॒ शीघ्रि॑याय । \newline
11. नमः॒ शीघ्रि॑याय॒ शीघ्रि॑याय॒ नमो॒ नमः॒ शीघ्रि॑याय च च॒ शीघ्रि॑याय॒ नमो॒ नमः॒ शीघ्रि॑याय च । \newline
12. शीघ्रि॑याय च च॒ शीघ्रि॑याय॒ शीघ्रि॑याय च॒ शीभ्या॑य॒ शीभ्या॑य च॒ शीघ्रि॑याय॒ शीघ्रि॑याय च॒ शीभ्या॑य । \newline
13. च॒ शीभ्या॑य॒ शीभ्या॑य च च॒ शीभ्या॑य च च॒ शीभ्या॑य च च॒ शीभ्या॑य च । \newline
14. शीभ्या॑य च च॒ शीभ्या॑य॒ शीभ्या॑य च॒ नमो॒ नम॑श्च॒ शीभ्या॑य॒ शीभ्या॑य च॒ नमः॑ । \newline
15. च॒ नमो॒ नम॑श्च च॒ नम॑ ऊ॒र्म्या॑ यो॒र्म्या॑य॒ नम॑श्च च॒ नम॑ ऊ॒र्म्या॑य । \newline
16. नम॑ ऊ॒र्म्या॑ यो॒र्म्या॑य॒ नमो॒ नम॑ ऊ॒र्म्या॑य च चो॒र्म्या॑य॒ नमो॒ नम॑ ऊ॒र्म्या॑य च । \newline
17. ऊ॒र्म्या॑य च चो॒र्म्या॑ यो॒र्म्या॑य चावस्व॒न्या॑या वस्व॒न्या॑य चो॒र्म्या॑ यो॒र्म्या॑य चावस्व॒न्या॑य । \newline
18. चा॒व॒स्व॒न्या॑या वस्व॒न्या॑य च चावस्व॒न्या॑य च चावस्व॒न्या॑य च चावस्व॒न्या॑य च । \newline
19. अ॒व॒स्व॒न्या॑य च चावस्व॒न्या॑या वस्व॒न्या॑य च॒ नमो॒ नम॑ श्चावस्व॒न्या॑या वस्व॒न्या॑य च॒ नमः॑ । \newline
20. अ॒व॒स्व॒न्या॑येत्य॑व - स्व॒न्या॑य । \newline
21. च॒ नमो॒ नम॑श्च च॒ नमः॑ स्रोत॒स्या॑य स्रोत॒स्या॑य॒ नम॑श्च च॒ नमः॑ स्रोत॒स्या॑य । \newline
22. नमः॑ स्रोत॒स्या॑य स्रोत॒स्या॑य॒ नमो॒ नमः॑ स्रोत॒स्या॑य च च स्रोत॒स्या॑य॒ नमो॒ नमः॑ स्रोत॒स्या॑य च । \newline
23. स्रो॒त॒स्या॑य च च स्रोत॒स्या॑य स्रोत॒स्या॑य च॒ द्वीप्या॑य॒ द्वीप्या॑य च स्रोत॒स्या॑य स्रोत॒स्या॑य च॒ द्वीप्या॑य । \newline
24. च॒ द्वीप्या॑य॒ द्वीप्या॑य च च॒ द्वीप्या॑य च च॒ द्वीप्या॑य च च॒ द्वीप्या॑य च । \newline
25. द्वीप्या॑य च च॒ द्वीप्या॑य॒ द्वीप्या॑य च । \newline
26. चेति॑ च । \newline
\pagebreak
\markright{ TS 4.5.6.1  \hfill https://www.vedavms.in \hfill}

\section{ TS 4.5.6.1 }

\textbf{TS 4.5.6.1 } \newline
\textbf{Samhita Paata} \newline

नमो᳚ ज्ये॒ष्ठाय॑ च कनि॒ष्ठाय॑ च॒ नमः॑ पूर्व॒जाय॑ चापर॒जाय॑ च॒ नमो॑ मद्ध्य॒माय॑ चापग॒ल्भाय॑ च॒ नमो॑ जघ॒न्या॑य च॒ बुद्ध्नि॑याय च॒ नमः॑ सो॒भ्या॑य च प्रतिस॒र्या॑य च॒ नमो॒ याम्या॑य च॒ क्षेम्या॑य च॒ नम॑ उर्व॒र्या॑य च॒ खल्या॑य च॒ नमः॒ श्लोक्या॑य चावसा॒न्या॑य च॒ नमो॒ वन्या॑य च॒ कक्ष्या॑य च॒ नमः॑ श्र॒वाय॑ च प्रतिश्र॒वाय॑ च॒-[  ] \newline

\textbf{Pada Paata} \newline

नमः॑ । ज्ये॒ष्ठाय॑ । च॒ । क॒नि॒ष्ठाय॑ । च॒ । नमः॑ । पू॒र्व॒जायेति॑ पूर्व - जाय॑ । च॒ । अ॒प॒र॒जायेत्य॑पर-जाय॑ । च॒ । नमः॑ । म॒द्ध्य॒माय॑ । च॒ । अ॒प॒ग॒ल्भायेत्य॑प - ग॒ल्भाय॑ । च॒ । नमः॑ । ज॒घ॒न्या॑य । च॒ । बुद्ध्नि॑याय । च॒ । नमः॑ । सो॒भ्या॑य । च॒ । प्र॒ति॒स॒र्या॑येति॑ प्रति - स॒र्या॑य । च॒ । नमः॑ । याम्या॑य । च॒ । क्षेम्या॑य । च॒ । नमः॑ । उ॒र्व॒र्या॑य । च॒ । खल्या॑य । च॒ । नमः॑ । श्लोक्या॑य । च॒ । अ॒व॒सा॒न्या॑येत्य॑व - सा॒न्या॑य । च॒ । नमः॑ । वन्या॑य । च॒ । कक्ष्या॑य । च॒ । नमः॑ । श्र॒वाय॑ । च॒ । प्र॒ति॒श्र॒वायेति॑ प्रति-श्र॒वाय॑ । च॒ ।  \newline


\textbf{Krama Paata} \newline

नमो᳚ ज्ये॒ष्ठाय॑ । ज्ये॒ष्ठाय॑ च । च॒ क॒नि॒ष्ठाय॑ । क॒नि॒ष्ठाय॑ च । च॒ नमः॑ । नमः॑ पूर्व॒जाय॑ । पू॒र्व॒जाय॑ च । पू॒र्व॒जायेति॑ पूर्व - जाय॑ । चा॒प॒र॒जाय॑ । अ॒प॒र॒जाय॑ च । अ॒प॒र॒जायेत्य॑पर - जाय॑ । च॒ नमः॑ । नमो॑ मद्ध्य॒माय॑ । म॒द्ध्य॒माय॑ च । चा॒प॒ग॒ल्भाय॑ । अ॒प॒ग॒ल्भाय॑ च । अ॒प॒ग॒ल्भायेत्य॑प - ग॒ल्भाय॑ । च॒ नमः॑ । नमो॑ जघ॒न्या॑य । ज॒घ॒न्या॑य च । च॒ बुद्ध्नि॑याय । बुद्ध्नि॑याय च । च॒ नमः॑ । नमः॑ सो॒भ्या॑य । सो॒भ्या॑य च । च॒ प्र॒ति॒स॒र्या॑य । प्र॒ति॒स॒र्या॑य च । प्र॒ति॒स॒र्या॑येति॑ प्रति - स॒र्या॑य । च॒ नमः॑ । नमो॒ याम्या॑य । याम्या॑य च । च॒ क्षेम्या॑य । क्षेम्या॑य च । च॒ नमः॑ । नम॑ उर्व॒र्या॑य । उ॒र्व॒र्या॑य च । च॒ खल्या॑य । खल्या॑य च । च॒ नमः॑ । नमः॒ श्लोक्या॑य । श्लोक्या॑य च । चा॒व॒सा॒न्या॑य । अ॒व॒सा॒न्या॑य च । अ॒व॒सा॒न्या॑येत्य॑व - सा॒न्या॑य । च॒ नमः॑ । नमो॒ वन्या॑य । वन्या॑य च । च॒ कक्ष्या॑य । कक्ष्या॑य च । च॒ नमः॑ । नमः॑ श्र॒वाय॑ । श्र॒वाय॑ च । च॒ प्र॒ति॒श्र॒वाय॑ । प्र॒ति॒श्र॒वाय॑ च ( ) । प्र॒ति॒श्र॒वायेति॑ प्रति - श्र॒वाय॑ । च॒ नमः॑ \newline

\textbf{Jatai Paata} \newline

1. नमो᳚ ज्ये॒ष्ठाय॑ ज्ये॒ष्ठाय॒ नमो॒ नमो᳚ ज्ये॒ष्ठाय॑ । \newline
2. ज्ये॒ष्ठाय॑ च च ज्ये॒ष्ठाय॑ ज्ये॒ष्ठाय॑ च । \newline
3. च॒ क॒नि॒ष्ठाय॑ कनि॒ष्ठाय॑ च च कनि॒ष्ठाय॑ । \newline
4. क॒नि॒ष्ठाय॑ च च कनि॒ष्ठाय॑ कनि॒ष्ठाय॑ च । \newline
5. च॒ नमो॒ नम॑श्च च॒ नमः॑ । \newline
6. नमः॑ पूर्व॒जाय॑ पूर्व॒जाय॒ नमो॒ नमः॑ पूर्व॒जाय॑ । \newline
7. पू॒र्व॒जाय॑ च च पूर्व॒जाय॑ पूर्व॒जाय॑ च । \newline
8. पू॒र्व॒जायेति॑ पूर्व - जाय॑ । \newline
9. चा॒प॒र॒जाया॑ पर॒जाय॑ च चापर॒जाय॑ । \newline
10. अ॒प॒र॒जाय॑ च चापर॒जाया॑ पर॒जाय॑ च । \newline
11. अ॒प॒र॒जायेत्य॑ पर - जाय॑ । \newline
12. च॒ नमो॒ नम॑श्च च॒ नमः॑ । \newline
13. नमो॑ मद्ध्य॒माय॑ मद्ध्य॒माय॒ नमो॒ नमो॑ मद्ध्य॒माय॑ । \newline
14. म॒द्ध्य॒माय॑ च च मद्ध्य॒माय॑ मद्ध्य॒माय॑ च । \newline
15. चा॒प॒ग॒ल्भाया॑ पग॒ल्भाय॑ च चापग॒ल्भाय॑ । \newline
16. अ॒प॒ग॒ल्भाय॑ च चापग॒ल्भाया॑ पग॒ल्भाय॑ च । \newline
17. अ॒प॒ग॒ल्भायेत्य॑प - ग॒ल्भाय॑ । \newline
18. च॒ नमो॒ नम॑श्च च॒ नमः॑ । \newline
19. नमो॑ जघ॒न्या॑य जघ॒न्या॑य॒ नमो॒ नमो॑ जघ॒न्या॑य । \newline
20. ज॒घ॒न्या॑य च च जघ॒न्या॑य जघ॒न्या॑य च । \newline
21. च॒ बुद्ध्नि॑याय॒ बुद्ध्नि॑याय च च॒ बुद्ध्नि॑याय । \newline
22. बुद्ध्नि॑याय च च॒ बुद्ध्नि॑याय॒ बुद्ध्नि॑याय च । \newline
23. च॒ नमो॒ नम॑श्च च॒ नमः॑ । \newline
24. नमः॑ सो॒भ्या॑य सो॒भ्या॑य॒ नमो॒ नमः॑ सो॒भ्या॑य । \newline
25. सो॒भ्या॑य च च सो॒भ्या॑य सो॒भ्या॑य च । \newline
26. च॒ प्र॒ति॒स॒र्या॑य प्रतिस॒र्या॑य च च प्रतिस॒र्या॑य । \newline
27. प्र॒ति॒स॒र्या॑य च च प्रतिस॒र्या॑य प्रतिस॒र्या॑य च । \newline
28. प्र॒ति॒स॒र्या॑येति॑ प्रति - स॒र्या॑य । \newline
29. च॒ नमो॒ नम॑श्च च॒ नमः॑ । \newline
30. नमो॒ याम्या॑य॒ याम्या॑य॒ नमो॒ नमो॒ याम्या॑य । \newline
31. याम्या॑य च च॒ याम्या॑य॒ याम्या॑य च । \newline
32. च॒ क्षेम्या॑य॒ क्षेम्या॑य च च॒ क्षेम्या॑य । \newline
33. क्षेम्या॑य च च॒ क्षेम्या॑य॒ क्षेम्या॑य च । \newline
34. च॒ नमो॒ नम॑श्च च॒ नमः॑ । \newline
35. नम॑ उर्व॒र्या॑ योर्व॒र्या॑य॒ नमो॒ नम॑ उर्व॒र्या॑य । \newline
36. उ॒र्व॒र्या॑य च चोर्व॒र्या॑ योर्व॒र्या॑य च । \newline
37. च॒ खल्या॑य॒ खल्या॑य च च॒ खल्या॑य । \newline
38. खल्या॑य च च॒ खल्या॑य॒ खल्या॑य च । \newline
39. च॒ नमो॒ नम॑श्च च॒ नमः॑ । \newline
40. नमः॒ श्लोक्या॑य॒ श्लोक्या॑य॒ नमो॒ नमः॒ श्लोक्या॑य । \newline
41. श्लोक्या॑य च च॒ श्लोक्या॑य॒ श्लोक्या॑य च । \newline
42. चा॒व॒सा॒न्या॑या वसा॒न्या॑य च चावसा॒न्या॑य । \newline
43. अ॒व॒सा॒न्या॑य च चावसा॒न्या॑या वसा॒न्या॑य च । \newline
44. अ॒व॒सा॒न्या॑येत्य॑व - सा॒न्या॑य । \newline
45. च॒ नमो॒ नम॑श्च च॒ नमः॑ । \newline
46. नमो॒ वन्या॑य॒ वन्या॑य॒ नमो॒ नमो॒ वन्या॑य । \newline
47. वन्या॑य च च॒ वन्या॑य॒ वन्या॑य च । \newline
48. च॒ कक्ष्या॑य॒ कक्ष्या॑य च च॒ कक्ष्या॑य । \newline
49. कक्ष्या॑य च च॒ कक्ष्या॑य॒ कक्ष्या॑य च । \newline
50. च॒ नमो॒ नम॑श्च च॒ नमः॑ । \newline
51. नमः॑ श्र॒वाय॑ श्र॒वाय॒ नमो॒ नमः॑ श्र॒वाय॑ । \newline
52. श्र॒वाय॑ च च श्र॒वाय॑ श्र॒वाय॑ च । \newline
53. च॒ प्र॒ति॒श्र॒वाय॑ प्रतिश्र॒वाय॑ च च प्रतिश्र॒वाय॑ । \newline
54. प्र॒ति॒श्र॒वाय॑ च च प्रतिश्र॒वाय॑ प्रतिश्र॒वाय॑ च । \newline
55. प्र॒ति॒श्र॒वायेति॑ प्रति - श्र॒वाय॑ । \newline
56. च॒ नमो॒ नम॑श्च च॒ नमः॑ । \newline

\textbf{Ghana Paata } \newline

1. नमो᳚ ज्ये॒ष्ठाय॑ ज्ये॒ष्ठाय॒ नमो॒ नमो᳚ ज्ये॒ष्ठाय॑ च च ज्ये॒ष्ठाय॒ नमो॒ नमो᳚ ज्ये॒ष्ठाय॑ च । \newline
2. ज्ये॒ष्ठाय॑ च च ज्ये॒ष्ठाय॑ ज्ये॒ष्ठाय॑ च कनि॒ष्ठाय॑ कनि॒ष्ठाय॑ च ज्ये॒ष्ठाय॑ ज्ये॒ष्ठाय॑ च कनि॒ष्ठाय॑ । \newline
3. च॒ क॒नि॒ष्ठाय॑ कनि॒ष्ठाय॑ च च कनि॒ष्ठाय॑ च च कनि॒ष्ठाय॑ च च कनि॒ष्ठाय॑ च । \newline
4. क॒नि॒ष्ठाय॑ च च कनि॒ष्ठाय॑ कनि॒ष्ठाय॑ च॒ नमो॒ नम॑श्च कनि॒ष्ठाय॑ कनि॒ष्ठाय॑ च॒ नमः॑ । \newline
5. च॒ नमो॒ नम॑श्च च॒ नमः॑ पूर्व॒जाय॑ पूर्व॒जाय॒ नम॑श्च च॒ नमः॑ पूर्व॒जाय॑ । \newline
6. नमः॑ पूर्व॒जाय॑ पूर्व॒जाय॒ नमो॒ नमः॑ पूर्व॒जाय॑ च च पूर्व॒जाय॒ नमो॒ नमः॑ पूर्व॒जाय॑ च । \newline
7. पू॒र्व॒जाय॑ च च पूर्व॒जाय॑ पूर्व॒जाय॑ चापर॒जाया॑ पर॒जाय॑ च पूर्व॒जाय॑ पूर्व॒जाय॑ चापर॒जाय॑ । \newline
8. पू॒र्व॒जायेति॑ पूर्व - जाय॑ । \newline
9. चा॒प॒र॒जाया॑ पर॒जाय॑ च चापर॒जाय॑ च चापर॒जाय॑ च चापर॒जाय॑ च । \newline
10. अ॒प॒र॒जाय॑ च चापर॒जाया॑ पर॒जाय॑ च॒ नमो॒ नम॑ श्चापर॒जाया॑ पर॒जाय॑ च॒ नमः॑ । \newline
11. अ॒प॒र॒जायेत्य॑ पर - जाय॑ । \newline
12. च॒ नमो॒ नम॑श्च च॒ नमो॑ मद्ध्य॒माय॑ मद्ध्य॒माय॒ नम॑श्च च॒ नमो॑ मद्ध्य॒माय॑ । \newline
13. नमो॑ मद्ध्य॒माय॑ मद्ध्य॒माय॒ नमो॒ नमो॑ मद्ध्य॒माय॑ च च मद्ध्य॒माय॒ नमो॒ नमो॑ मद्ध्य॒माय॑ च । \newline
14. म॒द्ध्य॒माय॑ च च मद्ध्य॒माय॑ मद्ध्य॒माय॑ चापग॒ल्भाया॑ पग॒ल्भाय॑ च मद्ध्य॒माय॑ मद्ध्य॒माय॑ चापग॒ल्भाय॑ । \newline
15. चा॒प॒ग॒ल्भाया॑ पग॒ल्भाय॑ च चापग॒ल्भाय॑ च चापग॒ल्भाय॑ च चापग॒ल्भाय॑ च । \newline
16. अ॒प॒ग॒ल्भाय॑ च चापग॒ल्भाया॑ पग॒ल्भाय॑ च॒ नमो॒ नम॑ श्चापग॒ल्भाया॑ पग॒ल्भाय॑ च॒ नमः॑ । \newline
17. अ॒प॒ग॒ल्भायेत्य॑प - ग॒ल्भाय॑ । \newline
18. च॒ नमो॒ नम॑श्च च॒ नमो॑ जघ॒न्या॑य जघ॒न्या॑य॒ नम॑श्च च॒ नमो॑ जघ॒न्या॑य । \newline
19. नमो॑ जघ॒न्या॑य जघ॒न्या॑य॒ नमो॒ नमो॑ जघ॒न्या॑य च च जघ॒न्या॑य॒ नमो॒ नमो॑ जघ॒न्या॑य च । \newline
20. ज॒घ॒न्या॑य च च जघ॒न्या॑य जघ॒न्या॑य च॒ बुद्ध्नि॑याय॒ बुद्ध्नि॑याय च जघ॒न्या॑य जघ॒न्या॑य च॒ बुद्ध्नि॑याय । \newline
21. च॒ बुद्ध्नि॑याय॒ बुद्ध्नि॑याय च च॒ बुद्ध्नि॑याय च च॒ बुद्ध्नि॑याय च च॒ बुद्ध्नि॑याय च । \newline
22. बुद्ध्नि॑याय च च॒ बुद्ध्नि॑याय॒ बुद्ध्नि॑याय च॒ नमो॒ नम॑श्च॒ बुद्ध्नि॑याय॒ बुद्ध्नि॑याय च॒ नमः॑ । \newline
23. च॒ नमो॒ नम॑श्च च॒ नमः॑ सो॒भ्या॑य सो॒भ्या॑य॒ नम॑श्च च॒ नमः॑ सो॒भ्या॑य । \newline
24. नमः॑ सो॒भ्या॑य सो॒भ्या॑य॒ नमो॒ नमः॑ सो॒भ्या॑य च च सो॒भ्या॑य॒ नमो॒ नमः॑ सो॒भ्या॑य च । \newline
25. सो॒भ्या॑य च च सो॒भ्या॑य सो॒भ्या॑य च प्रतिस॒र्या॑य प्रतिस॒र्या॑य च सो॒भ्या॑य सो॒भ्या॑य च प्रतिस॒र्या॑य । \newline
26. च॒ प्र॒ति॒स॒र्या॑य प्रतिस॒र्या॑य च च प्रतिस॒र्या॑य च च प्रतिस॒र्या॑य च च प्रतिस॒र्या॑य च । \newline
27. प्र॒ति॒स॒र्या॑य च च प्रतिस॒र्या॑य प्रतिस॒र्या॑य च॒ नमो॒ नम॑श्च प्रतिस॒र्या॑य प्रतिस॒र्या॑य च॒ नमः॑ । \newline
28. प्र॒ति॒स॒र्या॑येति॑ प्रति - स॒र्या॑य । \newline
29. च॒ नमो॒ नम॑श्च च॒ नमो॒ याम्या॑य॒ याम्या॑य॒ नम॑श्च च॒ नमो॒ याम्या॑य । \newline
30. नमो॒ याम्या॑य॒ याम्या॑य॒ नमो॒ नमो॒ याम्या॑य च च॒ याम्या॑य॒ नमो॒ नमो॒ याम्या॑य च । \newline
31. याम्या॑य च च॒ याम्या॑य॒ याम्या॑य च॒ क्षेम्या॑य॒ क्षेम्या॑य च॒ याम्या॑य॒ याम्या॑य च॒ क्षेम्या॑य । \newline
32. च॒ क्षेम्या॑य॒ क्षेम्या॑य च च॒ क्षेम्या॑य च च॒ क्षेम्या॑य च च॒ क्षेम्या॑य च । \newline
33. क्षेम्या॑य च च॒ क्षेम्या॑य॒ क्षेम्या॑य च॒ नमो॒ नम॑श्च॒ क्षेम्या॑य॒ क्षेम्या॑य च॒ नमः॑ । \newline
34. च॒ नमो॒ नम॑श्च च॒ नम॑ उर्व॒र्या॑ योर्व॒र्या॑य॒ नम॑श्च च॒ नम॑ उर्व॒र्या॑य । \newline
35. नम॑ उर्व॒र्या॑ योर्व॒र्या॑य॒ नमो॒ नम॑ उर्व॒र्या॑य च चोर्व॒र्या॑य॒ नमो॒ नम॑ उर्व॒र्या॑य च । \newline
36. उ॒र्व॒र्या॑य च चोर्व॒र्या॑ योर्व॒र्या॑य च॒ खल्या॑य॒ खल्या॑य चोर्व॒र्या॑ योर्व॒र्या॑य च॒ खल्या॑य । \newline
37. च॒ खल्या॑य॒ खल्या॑य च च॒ खल्या॑य च च॒ खल्या॑य च च॒ खल्या॑य च । \newline
38. खल्या॑य च च॒ खल्या॑य॒ खल्या॑य च॒ नमो॒ नम॑श्च॒ खल्या॑य॒ खल्या॑य च॒ नमः॑ । \newline
39. च॒ नमो॒ नम॑श्च च॒ नमः॒ श्लोक्या॑य॒ श्लोक्या॑य॒ नम॑श्च च॒ नमः॒ श्लोक्या॑य । \newline
40. नमः॒ श्लोक्या॑य॒ श्लोक्या॑य॒ नमो॒ नमः॒ श्लोक्या॑य च च॒ श्लोक्या॑य॒ नमो॒ नमः॒ श्लोक्या॑य च । \newline
41. श्लोक्या॑य च च॒ श्लोक्या॑य॒ श्लोक्या॑य चावसा॒न्या॑या वसा॒न्या॑य च॒ श्लोक्या॑य॒ श्लोक्या॑य चावसा॒न्या॑य । \newline
42. चा॒व॒सा॒न्या॑या वसा॒न्या॑य च चावसा॒न्या॑य च चावसा॒न्या॑य च चावसा॒न्या॑य च । \newline
43. अ॒व॒सा॒न्या॑य च चावसा॒न्या॑या वसा॒न्या॑य च॒ नमो॒ नम॑ श्चावसा॒न्या॑या वसा॒न्या॑य च॒ नमः॑ । \newline
44. अ॒व॒सा॒न्या॑येत्य॑व - सा॒न्या॑य । \newline
45. च॒ नमो॒ नम॑श्च च॒ नमो॒ वन्या॑य॒ वन्या॑य॒ नम॑श्च च॒ नमो॒ वन्या॑य । \newline
46. नमो॒ वन्या॑य॒ वन्या॑य॒ नमो॒ नमो॒ वन्या॑य च च॒ वन्या॑य॒ नमो॒ नमो॒ वन्या॑य च । \newline
47. वन्या॑य च च॒ वन्या॑य॒ वन्या॑य च॒ कक्ष्या॑य॒ कक्ष्या॑य च॒ वन्या॑य॒ वन्या॑य च॒ कक्ष्या॑य । \newline
48. च॒ कक्ष्या॑य॒ कक्ष्या॑य च च॒ कक्ष्या॑य च च॒ कक्ष्या॑य च च॒ कक्ष्या॑य च । \newline
49. कक्ष्या॑य च च॒ कक्ष्या॑य॒ कक्ष्या॑य च॒ नमो॒ नम॑श्च॒ कक्ष्या॑य॒ कक्ष्या॑य च॒ नमः॑ । \newline
50. च॒ नमो॒ नम॑श्च च॒ नमः॑ श्र॒वाय॑ श्र॒वाय॒ नम॑श्च च॒ नमः॑ श्र॒वाय॑ । \newline
51. नमः॑ श्र॒वाय॑ श्र॒वाय॒ नमो॒ नमः॑ श्र॒वाय॑ च च श्र॒वाय॒ नमो॒ नमः॑ श्र॒वाय॑ च । \newline
52. श्र॒वाय॑ च च श्र॒वाय॑ श्र॒वाय॑ च प्रतिश्र॒वाय॑ प्रतिश्र॒वाय॑ च श्र॒वाय॑ श्र॒वाय॑ च प्रतिश्र॒वाय॑ । \newline
53. च॒ प्र॒ति॒श्र॒वाय॑ प्रतिश्र॒वाय॑ च च प्रतिश्र॒वाय॑ च च प्रतिश्र॒वाय॑ च च प्रतिश्र॒वाय॑ च । \newline
54. प्र॒ति॒श्र॒वाय॑ च च प्रतिश्र॒वाय॑ प्रतिश्र॒वाय॑ च॒ नमो॒ नम॑श्च प्रतिश्र॒वाय॑ प्रतिश्र॒वाय॑ च॒ नमः॑ । \newline
55. प्र॒ति॒श्र॒वायेति॑ प्रति - श्र॒वाय॑ । \newline
56. च॒ नमो॒ नम॑श्च च॒ नम॑ आ॒शुषे॑णाया॒ शुषे॑णाय॒ नम॑श्च च॒ नम॑ आ॒शुषे॑णाय । \newline
\pagebreak
\markright{ TS 4.5.6.2  \hfill https://www.vedavms.in \hfill}

\section{ TS 4.5.6.2 }

\textbf{TS 4.5.6.2 } \newline
\textbf{Samhita Paata} \newline

नम॑ आ॒शुषे॑णाय चा॒शुर॑थाय च॒ नमः॒ शूरा॑य चावभिन्द॒ते च॒ नमो॑ व॒र्मिणे॑ च वरू॒थिने॑ च॒ नमो॑ बि॒ल्मिने॑ च कव॒चिने॑ च॒ नमः॑ श्रु॒ताय॑ च श्रुतसे॒नाय॑ च ॥ \newline

\textbf{Pada Paata} \newline

नमः॑ । आ॒शुषे॑णा॒येत्या॒शु - से॒ना॒य॒ । च॒ । आ॒शुर॑था॒येत्या॒शु-र॒था॒य॒ । च॒ । नमः॑ । शूरा॑य । च॒ । अ॒व॒भि॒न्द॒त इत्य॑व-भि॒न्द॒ते । च॒ । नमः॑ । व॒र्मिणे᳚ । च॒ । व॒रू॒थिने᳚ । च॒ । नमः॑ । बि॒ल्मिने᳚ । च॒ । क॒व॒चिने᳚ । च॒ । नमः॑ । श्रु॒ताय॑ । च॒ । श्रु॒त॒से॒नायेति॑ श्रुत-से॒नाय॑ । च॒ ॥  \newline


\textbf{Krama Paata} \newline

नम॑ आ॒शुषे॑णाय । आ॒शुषे॑णाय च । आ॒शुषे॑णा॒येत्या॒शु - से॒ना॒य॒ । चा॒शुर॑थाय । आ॒शुर॑थाय च । आ॒शुर॑था॒येत्या॒शु - र॒था॒य॒ । च॒ नमः॑ । नमः॒ शूरा॑य । शूरा॑य च । चा॒व॒भि॒न्द॒ते । अ॒व॒भि॒न्द॒ते च॑ । अ॒व॒भि॒न्द॒त इत्य॑व - भि॒न्द॒ते । च॒ नमः॑ । नमो॑ व॒र्मिणे᳚ । व॒र्मिणे॑ च । च॒ व॒रू॒थिने᳚ । व॒रू॒थिने॑ च । च॒ नमः॑ । नमो॑ बि॒ल्मिने᳚ । बि॒ल्मिने॑ च । च॒ क॒व॒चिने᳚ । क॒व॒चिने॑ च । च॒ नमः॑ । नमः॑ श्रु॒ताय॑ । श्रु॒ताय॑ च । च॒ श्रु॒त॒से॒नाय॑ । श्रु॒त॒से॒नाय॑ च । श्रु॒त॒से॒नायेति॑ श्रुत - से॒नाय॑ । चेति॑ च । \newline

\textbf{Jatai Paata} \newline

1. नम॑ आ॒शुषे॑णाया॒ शुषे॑णाय॒ नमो॒ नम॑ आ॒शुषे॑णाय । \newline
2. आ॒शुषे॑णाय च चा॒शुषे॑णाया॒ शुषे॑णाय च । \newline
3. आ॒शुषे॑णा॒येत्या॒शु - से॒ना॒य॒ । \newline
4. चा॒शुर॑थाया॒ शुर॑थाय च चा॒शुर॑थाय । \newline
5. आ॒शुर॑थाय च चा॒शुर॑थाया॒ शुर॑थाय च । \newline
6. आ॒शुर॑था॒येत्या॒शु - र॒था॒य॒ । \newline
7. च॒ नमो॒ नम॑श्च च॒ नमः॑ । \newline
8. नमः॒ शूरा॑य॒ शूरा॑य॒ नमो॒ नमः॒ शूरा॑य । \newline
9. शूरा॑य च च॒ शूरा॑य॒ शूरा॑य च । \newline
10. चा॒व॒भि॒न्द॒ते॑ ऽवभिन्द॒ते च॑ चावभिन्द॒ते । \newline
11. अ॒व॒भि॒न्द॒ते च॑ चावभिन्द॒ते॑ ऽवभिन्द॒ते च॑ । \newline
12. अ॒व॒भि॒न्द॒त इत्य॑व - भि॒न्द॒ते । \newline
13. च॒ नमो॒ नम॑श्च च॒ नमः॑ । \newline
14. नमो॑ व॒र्मिणे॑ व॒र्मिणे॒ नमो॒ नमो॑ व॒र्मिणे᳚ । \newline
15. व॒र्मिणे॑ च च व॒र्मिणे॑ व॒र्मिणे॑ च । \newline
16. च॒ व॒रू॒थिने॑ वरू॒थिने॑ च च वरू॒थिने᳚ । \newline
17. व॒रू॒थिने॑ च च वरू॒थिने॑ वरू॒थिने॑ च । \newline
18. च॒ नमो॒ नम॑श्च च॒ नमः॑ । \newline
19. नमो॑ बि॒ल्मिने॑ बि॒ल्मिने॒ नमो॒ नमो॑ बि॒ल्मिने᳚ । \newline
20. बि॒ल्मिने॑ च च बि॒ल्मिने॑ बि॒ल्मिने॑ च । \newline
21. च॒ क॒व॒चिने॑ कव॒चिने॑ च च कव॒चिने᳚ । \newline
22. क॒व॒चिने॑ च च कव॒चिने॑ कव॒चिने॑ च । \newline
23. च॒ नमो॒ नम॑श्च च॒ नमः॑ । \newline
24. नमः॑ श्रु॒ताय॑ श्रु॒ताय॒ नमो॒ नमः॑ श्रु॒ताय॑ । \newline
25. श्रु॒ताय॑ च च श्रु॒ताय॑ श्रु॒ताय॑ च । \newline
26. च॒ श्रु॒त॒से॒नाय॑ श्रुतसे॒नाय॑ च च श्रुतसे॒नाय॑ । \newline
27. श्रु॒त॒से॒नाय॑ च च श्रुतसे॒नाय॑ श्रुतसे॒नाय॑ च । \newline
28. श्रु॒त॒से॒नायेति॑ श्रुत - से॒नाय॑ । \newline
29. चेति॑ च । \newline

\textbf{Ghana Paata } \newline

1. नम॑ आ॒शुषे॑णाया॒ शुषे॑णाय॒ नमो॒ नम॑ आ॒शुषे॑णाय च चा॒शुषे॑णाय॒ नमो॒ नम॑ आ॒शुषे॑णाय च । \newline
2. आ॒शुषे॑णाय च चा॒शुषे॑णाया॒ शुषे॑णाय चा॒शुर॑थाया॒ शुर॑थाय चा॒शुषे॑णाया॒ शुषे॑णाय चा॒शुर॑थाय । \newline
3. आ॒शुषे॑णा॒येत्या॒शु - से॒ना॒य॒ । \newline
4. चा॒शुर॑थाया॒ शुर॑थाय च चा॒शुर॑थाय च चा॒शुर॑थाय च चा॒शुर॑थाय च । \newline
5. आ॒शुर॑थाय च चा॒शुर॑थाया॒ शुर॑थाय च॒ नमो॒ नम॑ श्चा॒शुर॑थाया॒ शुर॑थाय च॒ नमः॑ । \newline
6. आ॒शुर॑था॒येत्या॒शु - र॒था॒य॒ । \newline
7. च॒ नमो॒ नम॑श्च च॒ नमः॒ शूरा॑य॒ शूरा॑य॒ नम॑श्च च॒ नमः॒ शूरा॑य । \newline
8. नमः॒ शूरा॑य॒ शूरा॑य॒ नमो॒ नमः॒ शूरा॑य च च॒ शूरा॑य॒ नमो॒ नमः॒ शूरा॑य च । \newline
9. शूरा॑य च च॒ शूरा॑य॒ शूरा॑य चावभिन्द॒ते॑ ऽवभिन्द॒ते च॒ शूरा॑य॒ शूरा॑य चावभिन्द॒ते । \newline
10. चा॒व॒भि॒न्द॒ते॑ ऽवभिन्द॒ते च॑ चावभिन्द॒ते च॑ चावभिन्द॒ते च॑ चावभिन्द॒ते च॑ । \newline
11. अ॒व॒भि॒न्द॒ते च॑ चावभिन्द॒ते॑ ऽवभिन्द॒ते च॒ नमो॒ नम॑ श्चावभिन्द॒ते॑ ऽवभिन्द॒ते च॒ नमः॑ । \newline
12. अ॒व॒भि॒न्द॒त इत्य॑व - भि॒न्द॒ते । \newline
13. च॒ नमो॒ नम॑श्च च॒ नमो॑ व॒र्मिणे॑ व॒र्मिणे॒ नम॑श्च च॒ नमो॑ व॒र्मिणे᳚ । \newline
14. नमो॑ व॒र्मिणे॑ व॒र्मिणे॒ नमो॒ नमो॑ व॒र्मिणे॑ च च व॒र्मिणे॒ नमो॒ नमो॑ व॒र्मिणे॑ च । \newline
15. व॒र्मिणे॑ च च व॒र्मिणे॑ व॒र्मिणे॑ च वरू॒थिने॑ वरू॒थिने॑ च व॒र्मिणे॑ व॒र्मिणे॑ च वरू॒थिने᳚ । \newline
16. च॒ व॒रू॒थिने॑ वरू॒थिने॑ च च वरू॒थिने॑ च च वरू॒थिने॑ च च वरू॒थिने॑ च । \newline
17. व॒रू॒थिने॑ च च वरू॒थिने॑ वरू॒थिने॑ च॒ नमो॒ नम॑श्च वरू॒थिने॑ वरू॒थिने॑ च॒ नमः॑ । \newline
18. च॒ नमो॒ नम॑श्च च॒ नमो॑ बि॒ल्मिने॑ बि॒ल्मिने॒ नम॑श्च च॒ नमो॑ बि॒ल्मिने᳚ । \newline
19. नमो॑ बि॒ल्मिने॑ बि॒ल्मिने॒ नमो॒ नमो॑ बि॒ल्मिने॑ च च बि॒ल्मिने॒ नमो॒ नमो॑ बि॒ल्मिने॑ च । \newline
20. बि॒ल्मिने॑ च च बि॒ल्मिने॑ बि॒ल्मिने॑ च कव॒चिने॑ कव॒चिने॑ च बि॒ल्मिने॑ बि॒ल्मिने॑ च कव॒चिने᳚ । \newline
21. च॒ क॒व॒चिने॑ कव॒चिने॑ च च कव॒चिने॑ च च कव॒चिने॑ च च कव॒चिने॑ च । \newline
22. क॒व॒चिने॑ च च कव॒चिने॑ कव॒चिने॑ च॒ नमो॒ नम॑श्च कव॒चिने॑ कव॒चिने॑ च॒ नमः॑ । \newline
23. च॒ नमो॒ नम॑श्च च॒ नमः॑ श्रु॒ताय॑ श्रु॒ताय॒ नम॑श्च च॒ नमः॑ श्रु॒ताय॑ । \newline
24. नमः॑ श्रु॒ताय॑ श्रु॒ताय॒ नमो॒ नमः॑ श्रु॒ताय॑ च च श्रु॒ताय॒ नमो॒ नमः॑ श्रु॒ताय॑ च । \newline
25. श्रु॒ताय॑ च च श्रु॒ताय॑ श्रु॒ताय॑ च श्रुतसे॒नाय॑ श्रुतसे॒नाय॑ च श्रु॒ताय॑ श्रु॒ताय॑ च श्रुतसे॒नाय॑ । \newline
26. च॒ श्रु॒त॒से॒नाय॑ श्रुतसे॒नाय॑ च च श्रुतसे॒नाय॑ च च श्रुतसे॒नाय॑ च च श्रुतसे॒नाय॑ च । \newline
27. श्रु॒त॒से॒नाय॑ च च श्रुतसे॒नाय॑ श्रुतसे॒नाय॑ च । \newline
28. श्रु॒त॒से॒नायेति॑ श्रुत - से॒नाय॑ । \newline
29. चेति॑ च । \newline
\pagebreak
\markright{ TS 4.5.7.1  \hfill https://www.vedavms.in \hfill}

\section{ TS 4.5.7.1 }

\textbf{TS 4.5.7.1 } \newline
\textbf{Samhita Paata} \newline

नमो॑ दुन्दु॒भ्या॑य चाहन॒न्या॑य च॒ नमो॑ धृ॒ष्णवे॑ च प्रमृ॒शाय॑ च॒ नमो॑ दू॒ताय॑ च॒ प्रहि॑ताय च॒ नमो॑ निष॒ङ्गिणे॑ चेषुधि॒मते॑ च॒ नम॑ स्ती॒क्ष्णेष॑वे चायु॒धिने॑ च॒ नमः॑ स्वायु॒धाय॑ च सु॒धन्व॑ने च॒ नमः॒ स्रुत्या॑य च॒ पथ्या॑य च॒ नमः॑ का॒ट्या॑य च नी॒प्या॑य च॒ नमः॒ सूद्या॑य च सर॒स्या॑य च॒ नमो॑ ना॒द्याय॑ च वैश॒न्ताय॑ च॒ -[  ] \newline

\textbf{Pada Paata} \newline

नमः॑ । दु॒न्दु॒भ्या॑य । च॒ । आ॒ह॒न॒न्या॑येत्या᳚ - ह॒न॒न्या॑य । च॒ । नमः॑ । धृ॒ष्णवे᳚ । च॒ । प्र॒मृ॒शायेति॑ प्र - मृ॒शाय॑ । च॒ । नमः॑ । दू॒ताय॑ । च॒ । प्रहि॑ता॒येति॒ प्र - हि॒ता॒य॒ । च॒ । नमः॑ । नि॒ष॒ङ्गिण॒ इति॑ नि - स॒ङ्गिने᳚ । च॒ । इ॒षु॒धि॒मत॒ इती॑षुधि - मते᳚ । च॒ । नमः॑ । ती॒क्ष्णेष॑व॒ इति॑ ती॒क्ष्ण - इ॒ष॒वे॒ । च॒ । आ॒यु॒धिने᳚ । च॒ । नमः॑ । स्वा॒यु॒धायेति॑ सु - आ॒यु॒धाय॑ । च॒ । सु॒धन्व॑न॒ इति॑ सु - धन्व॑ने । च॒ । नमः॑ । स्रुत्या॑य । च॒ । पथ्या॑य । च॒ । नमः॑ । का॒ट्या॑य । च॒ । नी॒प्या॑य । च॒ । नमः॑ । सूद्या॑य । च॒ । स॒र॒स्या॑य । च॒ । नमः॑ । ना॒द्याय॑ । च॒ । वै॒श॒न्ताय॑ । च॒ ।  \newline


\textbf{Krama Paata} \newline

नमो॑ दुन्दु॒भ्या॑य । दु॒न्दु॒भ्या॑य च । चा॒ह॒न॒न्या॑य । आ॒ह॒न॒न्या॑य च । आ॒ह॒न॒न्या॑येत्या᳚ - ह॒न॒न्या॑य । च॒ नमः॑ । नमो॑ धृ॒ष्णवे᳚ । धृ॒ष्णवे॑ च । च॒ प्र॒मृ॒शाय॑ । प्र॒मृ॒शाय॑ च । प्र॒मृ॒शायेति॑ प्र - मृ॒शाय॑ । च॒ नमः॑ । नमो॑ दू॒ताय॑ । दू॒ताय॑ च । च॒ प्रहि॑ताय । प्रहि॑ताय च । प्रहि॑ता॒येति॒ प्र - हि॒ता॒य॒ । च॒ नमः॑ । नमो॑ निष॒ङ्गिणे᳚ । नि॒ष॒ङ्गिणे॑ च । नि॒ष॒ङ्गिण॒ इति॑ नि - स॒ङ्गिने᳚ । चे॒षु॒धि॒मते᳚ । इ॒षु॒धि॒मते॑ च । इ॒षु॒धि॒मत॒ इती॑षुधि - मते᳚ । च॒ नमः॑ । नम॑ स्ती॒क्ष्णेष॑वे । ती॒क्ष्णेष॑वे च । ती॒क्ष्णेष॑व॒ इति॑ ती॒क्ष्ण - इ॒ष॒वे॒ । चा॒यु॒धिने᳚ । आ॒यु॒धिने॑ च । च॒ नमः॑ । नमः॑ स्वायु॒धाय॑ । स्वा॒यु॒धाय॑ च । स्वा॒यु॒धायेति॑ सु - आ॒यु॒धाय॑ । च॒ सु॒धन्व॑ने । सु॒धन्व॑ने च । सु॒धन्व॑न॒ इति॑ सु - धन्व॑ने । च॒ नमः॑ । नमः॒ स्रुत्या॑य । स्रुत्या॑य च । च॒ पथ्या॑य । पथ्या॑य च । च॒ नमः॑ । नमः॑ का॒ट्या॑य । का॒ट्या॑य च । च॒ नी॒प्या॑य । नी॒प्या॑य च । च॒ नमः॑ । नमः॒ सूद्या॑य । सूद्या॑य च । च॒ स॒र॒स्या॑य । स॒र॒स्या॑य च । च॒ नमः॑ । नमो॑ ना॒द्याय॑ । ना॒द्याय॑ च । च॒ वै॒श॒न्ताय॑ । वै॒श॒न्ताय॑ च ( ) । च॒ नमः॑ \newline

\textbf{Jatai Paata} \newline

1. नमो॑ दुन्दु॒भ्या॑य दुन्दु॒भ्या॑य॒ नमो॒ नमो॑ दुन्दु॒भ्या॑य । \newline
2. दु॒न्दु॒भ्या॑य च च दुन्दु॒भ्या॑य दुन्दु॒भ्या॑य च । \newline
3. चा॒ह॒न॒न्या॑या हन॒न्या॑य च चाहन॒न्या॑य । \newline
4. आ॒ह॒न॒न्या॑य च चाहन॒न्या॑या हन॒न्या॑य च । \newline
5. आ॒ह॒न॒न्या॑येत्या᳚ - ह॒न॒न्या॑य । \newline
6. च॒ नमो॒ नम॑श्च च॒ नमः॑ । \newline
7. नमो॑ धृ॒ष्णवे॑ धृ॒ष्णवे॒ नमो॒ नमो॑ धृ॒ष्णवे᳚ । \newline
8. धृ॒ष्णवे॑ च च धृ॒ष्णवे॑ धृ॒ष्णवे॑ च । \newline
9. च॒ प्र॒मृ॒शाय॑ प्रमृ॒शाय॑ च च प्रमृ॒शाय॑ । \newline
10. प्र॒मृ॒शाय॑ च च प्रमृ॒शाय॑ प्रमृ॒शाय॑ च । \newline
11. प्र॒मृ॒शायेति॑ प्र - मृ॒शाय॑ । \newline
12. च॒ नमो॒ नम॑श्च च॒ नमः॑ । \newline
13. नमो॑ दू॒ताय॑ दू॒ताय॒ नमो॒ नमो॑ दू॒ताय॑ । \newline
14. दू॒ताय॑ च च दू॒ताय॑ दू॒ताय॑ च । \newline
15. च॒ प्रहि॑ताय॒ प्रहि॑ताय च च॒ प्रहि॑ताय । \newline
16. प्रहि॑ताय च च॒ प्रहि॑ताय॒ प्रहि॑ताय च । \newline
17. प्रहि॑ता॒येति॒ प्र - हि॒ता॒य॒ । \newline
18. च॒ नमो॒ नम॑श्च च॒ नमः॑ । \newline
19. नमो॑ निष॒ङ्गिणे॑ निष॒ङ्गिणे॒ नमो॒ नमो॑ निष॒ङ्गिणे᳚ । \newline
20. नि॒ष॒ङ्गिणे॑ च च निष॒ङ्गिणे॑ निष॒ङ्गिणे॑ च । \newline
21. नि॒ष॒ङ्गिण॒ इति॑ नि - स॒ङ्गिने᳚ । \newline
22. चे॒षु॒धि॒मत॑ इषुधि॒मते॑ च चेषुधि॒मते᳚ । \newline
23. इ॒षु॒धि॒मते॑ च चेषुधि॒मत॑ इषुधि॒मते॑ च । \newline
24. इ॒षु॒धि॒मत॒ इती॑षुधि - मते᳚ । \newline
25. च॒ नमो॒ नम॑श्च च॒ नमः॑ । \newline
26. नम॑ स्ती॒क्ष्णेष॑वे ती॒क्ष्णेष॑वे॒ नमो॒ नम॑ स्ती॒क्ष्णेष॑वे । \newline
27. ती॒क्ष्णेष॑वे च च ती॒क्ष्णेष॑वे ती॒क्ष्णेष॑वे च । \newline
28. ती॒क्ष्णेष॑व॒ इति॑ ती॒क्ष्ण - इ॒ष॒वे॒ । \newline
29. चा॒यु॒धिन॑ आयु॒धिने॑ च चायु॒धिने᳚ । \newline
30. आ॒यु॒धिने॑ च चायु॒धिन॑ आयु॒धिने॑ च । \newline
31. च॒ नमो॒ नम॑श्च च॒ नमः॑ । \newline
32. नमः॑ स्वायु॒धाय॑ स्वायु॒धाय॒ नमो॒ नमः॑ स्वायु॒धाय॑ । \newline
33. स्वा॒यु॒धाय॑ च च स्वायु॒धाय॑ स्वायु॒धाय॑ च । \newline
34. स्वा॒यु॒धायेति॑ सु - आ॒यु॒धाय॑ । \newline
35. च॒ सु॒धन्व॑ने सु॒धन्व॑ने च च सु॒धन्व॑ने । \newline
36. सु॒धन्व॑ने च च सु॒धन्व॑ने सु॒धन्व॑ने च । \newline
37. सु॒धन्व॑न॒ इति॑ सु - धन्व॑ने । \newline
38. च॒ नमो॒ नम॑श्च च॒ नमः॑ । \newline
39. नमः॒ स्रुत्या॑य॒ स्रुत्या॑य॒ नमो॒ नमः॒ स्रुत्या॑य । \newline
40. स्रुत्या॑य च च॒ स्रुत्या॑य॒ स्रुत्या॑य च । \newline
41. च॒ पथ्या॑य॒ पथ्या॑य च च॒ पथ्या॑य । \newline
42. पथ्या॑य च च॒ पथ्या॑य॒ पथ्या॑य च । \newline
43. च॒ नमो॒ नम॑श्च च॒ नमः॑ । \newline
44. नमः॑ का॒ट्या॑य का॒ट्या॑य॒ नमो॒ नमः॑ का॒ट्या॑य । \newline
45. का॒ट्या॑य च च का॒ट्या॑य का॒ट्या॑य च । \newline
46. च॒ नी॒प्या॑य नी॒प्या॑य च च नी॒प्या॑य । \newline
47. नी॒प्या॑य च च नी॒प्या॑य नी॒प्या॑य च । \newline
48. च॒ नमो॒ नम॑श्च च॒ नमः॑ । \newline
49. नमः॒ सूद्या॑य॒ सूद्या॑य॒ नमो॒ नमः॒ सूद्या॑य । \newline
50. सूद्या॑य च च॒ सूद्या॑य॒ सूद्या॑य च । \newline
51. च॒ स॒र॒स्या॑य सर॒स्या॑य च च सर॒स्या॑य । \newline
52. स॒र॒स्या॑य च च सर॒स्या॑य सर॒स्या॑य च । \newline
53. च॒ नमो॒ नम॑श्च च॒ नमः॑ । \newline
54. नमो॑ ना॒द्याय॑ ना॒द्याय॒ नमो॒ नमो॑ ना॒द्याय॑ । \newline
55. ना॒द्याय॑ च च ना॒द्याय॑ ना॒द्याय॑ च । \newline
56. च॒ वै॒श॒न्ताय॑ वैश॒न्ताय॑ च च वैश॒न्ताय॑ । \newline
57. वै॒श॒न्ताय॑ च च वैश॒न्ताय॑ वैश॒न्ताय॑ च । \newline
58. च॒ नमो॒ नम॑श्च च॒ नमः॑ । \newline

\textbf{Ghana Paata } \newline

1. नमो॑ दुन्दु॒भ्या॑य दुन्दु॒भ्या॑य॒ नमो॒ नमो॑ दुन्दु॒भ्या॑य च च दुन्दु॒भ्या॑य॒ नमो॒ नमो॑ दुन्दु॒भ्या॑य च । \newline
2. दु॒न्दु॒भ्या॑य च च दुन्दु॒भ्या॑य दुन्दु॒भ्या॑य चाहन॒न्या॑या हन॒न्या॑य च दुन्दु॒भ्या॑य दुन्दु॒भ्या॑य चाहन॒न्या॑य । \newline
3. चा॒ह॒न॒न्या॑या हन॒न्या॑य च चाहन॒न्या॑य च चाहन॒न्या॑य च चाहन॒न्या॑य च । \newline
4. आ॒ह॒न॒न्या॑य च चाहन॒न्या॑या हन॒न्या॑य च॒ नमो॒ नम॑ श्चाहन॒न्या॑या हन॒न्या॑य च॒ नमः॑ । \newline
5. आ॒ह॒न॒न्या॑येत्या᳚ - ह॒न॒न्या॑य । \newline
6. च॒ नमो॒ नम॑श्च च॒ नमो॑ धृ॒ष्णवे॑ धृ॒ष्णवे॒ नम॑श्च च॒ नमो॑ धृ॒ष्णवे᳚ । \newline
7. नमो॑ धृ॒ष्णवे॑ धृ॒ष्णवे॒ नमो॒ नमो॑ धृ॒ष्णवे॑ च च धृ॒ष्णवे॒ नमो॒ नमो॑ धृ॒ष्णवे॑ च । \newline
8. धृ॒ष्णवे॑ च च धृ॒ष्णवे॑ धृ॒ष्णवे॑ च प्रमृ॒शाय॑ प्रमृ॒शाय॑ च धृ॒ष्णवे॑ धृ॒ष्णवे॑ च प्रमृ॒शाय॑ । \newline
9. च॒ प्र॒मृ॒शाय॑ प्रमृ॒शाय॑ च च प्रमृ॒शाय॑ च च प्रमृ॒शाय॑ च च प्रमृ॒शाय॑ च । \newline
10. प्र॒मृ॒शाय॑ च च प्रमृ॒शाय॑ प्रमृ॒शाय॑ च॒ नमो॒ नम॑श्च प्रमृ॒शाय॑ प्रमृ॒शाय॑ च॒ नमः॑ । \newline
11. प्र॒मृ॒शायेति॑ प्र - मृ॒शाय॑ । \newline
12. च॒ नमो॒ नम॑श्च च॒ नमो॑ दू॒ताय॑ दू॒ताय॒ नम॑श्च च॒ नमो॑ दू॒ताय॑ । \newline
13. नमो॑ दू॒ताय॑ दू॒ताय॒ नमो॒ नमो॑ दू॒ताय॑ च च दू॒ताय॒ नमो॒ नमो॑ दू॒ताय॑ च । \newline
14. दू॒ताय॑ च च दू॒ताय॑ दू॒ताय॑ च॒ प्रहि॑ताय॒ प्रहि॑ताय च दू॒ताय॑ दू॒ताय॑ च॒ प्रहि॑ताय । \newline
15. च॒ प्रहि॑ताय॒ प्रहि॑ताय च च॒ प्रहि॑ताय च च॒ प्रहि॑ताय च च॒ प्रहि॑ताय च । \newline
16. प्रहि॑ताय च च॒ प्रहि॑ताय॒ प्रहि॑ताय च॒ नमो॒ नम॑श्च॒ प्रहि॑ताय॒ प्रहि॑ताय च॒ नमः॑ । \newline
17. प्रहि॑ता॒येति॒ प्र - हि॒ता॒य॒ । \newline
18. च॒ नमो॒ नम॑श्च च॒ नमो॑ निष॒ङ्गिणे॑ निष॒ङ्गिणे॒ नम॑श्च च॒ नमो॑ निष॒ङ्गिणे᳚ । \newline
19. नमो॑ निष॒ङ्गिणे॑ निष॒ङ्गिणे॒ नमो॒ नमो॑ निष॒ङ्गिणे॑ च च निष॒ङ्गिणे॒ नमो॒ नमो॑ निष॒ङ्गिणे॑ च । \newline
20. नि॒ष॒ङ्गिणे॑ च च निष॒ङ्गिणे॑ निष॒ङ्गिणे॑ चेषुधि॒मत॑ इषुधि॒मते॑ च निष॒ङ्गिणे॑ निष॒ङ्गिणे॑ 
चेषुधि॒मते᳚ । \newline
21. नि॒ष॒ङ्गिण॒ इति॑ नि - स॒ङ्गिने᳚ । \newline
22. चे॒षु॒धि॒मत॑ इषुधि॒मते॑ च चेषुधि॒मते॑ च चेषुधि॒मते॑ च चेषुधि॒मते॑ च । \newline
23. इ॒षु॒धि॒मते॑ च चेषुधि॒मत॑ इषुधि॒मते॑ च॒ नमो॒ नम॑ श्चेषुधि॒मत॑ इषुधि॒मते॑ च॒ नमः॑ । \newline
24. इ॒षु॒धि॒मत॒ इती॑षुधि - मते᳚ । \newline
25. च॒ नमो॒ नम॑श्च च॒ नम॑ स्ती॒क्ष्णेष॑वे ती॒क्ष्णेष॑वे॒ नम॑श्च च॒ नम॑ स्ती॒क्ष्णेष॑वे । \newline
26. नम॑ स्ती॒क्ष्णेष॑वे ती॒क्ष्णेष॑वे॒ नमो॒ नम॑ स्ती॒क्ष्णेष॑वे च च ती॒क्ष्णेष॑वे॒ नमो॒ नम॑ स्ती॒क्ष्णेष॑वे च । \newline
27. ती॒क्ष्णेष॑वे च च ती॒क्ष्णेष॑वे ती॒क्ष्णेष॑वे चायु॒धिन॑ आयु॒धिने॑ च ती॒क्ष्णेष॑वे ती॒क्ष्णेष॑वे चायु॒धिने᳚ । \newline
28. ती॒क्ष्णेष॑व॒ इति॑ ती॒क्ष्ण - इ॒ष॒वे॒ । \newline
29. चा॒यु॒धिन॑ आयु॒धिने॑ च चायु॒धिने॑ च चायु॒धिने॑ च चायु॒धिने॑ च । \newline
30. आ॒यु॒धिने॑ च चायु॒धिन॑ आयु॒धिने॑ च॒ नमो॒ नम॑ श्चायु॒धिन॑ आयु॒धिने॑ च॒ नमः॑ । \newline
31. च॒ नमो॒ नम॑श्च च॒ नमः॑ स्वायु॒धाय॑ स्वायु॒धाय॒ नम॑श्च च॒ नमः॑ स्वायु॒धाय॑ । \newline
32. नमः॑ स्वायु॒धाय॑ स्वायु॒धाय॒ नमो॒ नमः॑ स्वायु॒धाय॑ च च स्वायु॒धाय॒ नमो॒ नमः॑ स्वायु॒धाय॑ च । \newline
33. स्वा॒यु॒धाय॑ च च स्वायु॒धाय॑ स्वायु॒धाय॑ च सु॒धन्व॑ने सु॒धन्व॑ने च स्वायु॒धाय॑ स्वायु॒धाय॑ च सु॒धन्व॑ने । \newline
34. स्वा॒यु॒धायेति॑ सु - आ॒यु॒धाय॑ । \newline
35. च॒ सु॒धन्व॑ने सु॒धन्व॑ने च च सु॒धन्व॑ने च च सु॒धन्व॑ने च च सु॒धन्व॑ने च । \newline
36. सु॒धन्व॑ने च च सु॒धन्व॑ने सु॒धन्व॑ने च॒ नमो॒ नम॑श्च सु॒धन्व॑ने सु॒धन्व॑ने च॒ नमः॑ । \newline
37. सु॒धन्व॑न॒ इति॑ सु - धन्व॑ने । \newline
38. च॒ नमो॒ नम॑श्च च॒ नमः॒ स्रुत्या॑य॒ स्रुत्या॑य॒ नम॑श्च च॒ नमः॒ स्रुत्या॑य । \newline
39. नमः॒ स्रुत्या॑य॒ स्रुत्या॑य॒ नमो॒ नमः॒ स्रुत्या॑य च च॒ स्रुत्या॑य॒ नमो॒ नमः॒ स्रुत्या॑य च । \newline
40. स्रुत्या॑य च च॒ स्रुत्या॑य॒ स्रुत्या॑य च॒ पथ्या॑य॒ पथ्या॑य च॒ स्रुत्या॑य॒ स्रुत्या॑य च॒ पथ्या॑य । \newline
41. च॒ पथ्या॑य॒ पथ्या॑य च च॒ पथ्या॑य च च॒ पथ्या॑य च च॒ पथ्या॑य च । \newline
42. पथ्या॑य च च॒ पथ्या॑य॒ पथ्या॑य च॒ नमो॒ नम॑श्च॒ पथ्या॑य॒ पथ्या॑य च॒ नमः॑ । \newline
43. च॒ नमो॒ नम॑श्च च॒ नमः॑ का॒ट्या॑य का॒ट्या॑य॒ नम॑श्च च॒ नमः॑ का॒ट्या॑य । \newline
44. नमः॑ का॒ट्या॑य का॒ट्या॑य॒ नमो॒ नमः॑ का॒ट्या॑य च च का॒ट्या॑य॒ नमो॒ नमः॑ का॒ट्या॑य च । \newline
45. का॒ट्या॑य च च का॒ट्या॑य का॒ट्या॑य च नी॒प्या॑य नी॒प्या॑य च का॒ट्या॑य का॒ट्या॑य च नी॒प्या॑य । \newline
46. च॒ नी॒प्या॑य नी॒प्या॑य च च नी॒प्या॑य च च नी॒प्या॑य च च नी॒प्या॑य च । \newline
47. नी॒प्या॑य च च नी॒प्या॑य नी॒प्या॑य च॒ नमो॒ नम॑श्च नी॒प्या॑य नी॒प्या॑य च॒ नमः॑ । \newline
48. च॒ नमो॒ नम॑श्च च॒ नमः॒ सूद्या॑य॒ सूद्या॑य॒ नम॑श्च च॒ नमः॒ सूद्या॑य । \newline
49. नमः॒ सूद्या॑य॒ सूद्या॑य॒ नमो॒ नमः॒ सूद्या॑य च च॒ सूद्या॑य॒ नमो॒ नमः॒ सूद्या॑य च । \newline
50. सूद्या॑य च च॒ सूद्या॑य॒ सूद्या॑य च सर॒स्या॑य सर॒स्या॑य च॒ सूद्या॑य॒ सूद्या॑य च सर॒स्या॑य । \newline
51. च॒ स॒र॒स्या॑य सर॒स्या॑य च च सर॒स्या॑य च च सर॒स्या॑य च च सर॒स्या॑य च । \newline
52. स॒र॒स्या॑य च च सर॒स्या॑य सर॒स्या॑य च॒ नमो॒ नम॑श्च सर॒स्या॑य सर॒स्या॑य च॒ नमः॑ । \newline
53. च॒ नमो॒ नम॑श्च च॒ नमो॑ ना॒द्याय॑ ना॒द्याय॒ नम॑श्च च॒ नमो॑ ना॒द्याय॑ । \newline
54. नमो॑ ना॒द्याय॑ ना॒द्याय॒ नमो॒ नमो॑ ना॒द्याय॑ च च ना॒द्याय॒ नमो॒ नमो॑ ना॒द्याय॑ च । \newline
55. ना॒द्याय॑ च च ना॒द्याय॑ ना॒द्याय॑ च वैश॒न्ताय॑ वैश॒न्ताय॑ च ना॒द्याय॑ ना॒द्याय॑ च वैश॒न्ताय॑ । \newline
56. च॒ वै॒श॒न्ताय॑ वैश॒न्ताय॑ च च वैश॒न्ताय॑ च च वैश॒न्ताय॑ च च वैश॒न्ताय॑ च । \newline
57. वै॒श॒न्ताय॑ च च वैश॒न्ताय॑ वैश॒न्ताय॑ च॒ नमो॒ नम॑श्च वैश॒न्ताय॑ वैश॒न्ताय॑ च॒ नमः॑ । \newline
58. च॒ नमो॒ नम॑श्च च॒ नमः॒ कूप्या॑य॒ कूप्या॑य॒ नम॑श्च च॒ नमः॒ कूप्या॑य । \newline
\pagebreak
\markright{ TS 4.5.7.2  \hfill https://www.vedavms.in \hfill}

\section{ TS 4.5.7.2 }

\textbf{TS 4.5.7.2 } \newline
\textbf{Samhita Paata} \newline

नमः᳡कूप्या॑य चाव॒ट्या॑य च॒ नमो॒ वर्ष्या॑य चाव॒र्ष्याय॑ च॒ नमो॑ मे॒घ्या॑य च विद्यु॒त्या॑य च॒ नम॑ ई॒द्ध्रिया॑य चात॒प्या॑य च॒ नमो॒ वात्या॑य च॒ रेष्मि॑याय च॒ नमो॑ वास्त॒व्या॑य च वास्तु॒पाय॑ च ॥ \newline

\textbf{Pada Paata} \newline

नमः॑ । कूप्या॑य । च॒ । अ॒व॒ट्या॑य । च॒ । नमः॑ । वर्ष्या॑य । च॒ । अ॒व॒र्ष्याय॑ । च॒ । नमः॑ । मे॒घ्या॑य । च॒ । वि॒द्यु॒त्या॑येति॑ वि - द्यु॒त्या॑य । च॒ । नमः॑ । ई॒द्ध्रिया॑य । च॒ । आ॒त॒प्या॑येत्या᳚ - त॒प्या॑य । च॒ । नमः॑ । वात्या॑य । च॒ । रेष्मि॑याय । च॒ । नमः॑ । वा॒स्त॒व्या॑य । च॒ । वा॒स्तु॒पायेति॑ वास्तु - पाय॑ । च॒ ॥  \newline


\textbf{Krama Paata} \newline

नमः॒ कूप्या॑य । कूप्या॑य च । चा॒व॒ट्या॑य । अ॒व॒ट्या॑य च । च॒ नमः॑ । नमो॒ वर्ष्या॑य । वर्ष्या॑य च । चा॒व॒र्ष्याय॑ । अ॒व॒र्ष्याय॑ च । च॒ नमः॑ । नमो॑ मे॒घ्या॑य । मे॒घ्या॑य च । च॒ वि॒द्यु॒त्या॑य । वि॒द्यु॒त्या॑य च । वि॒द्यु॒त्या॑येति॑ वि - द्यु॒त्या॑य । च॒ नमः॑ । नम॑ ई॒द्ध्रिया॑य । ई॒द्ध्रिया॑य च । चा॒त॒प्या॑य । आ॒त॒प्या॑य च । आ॒त॒प्या॑येत्या᳚ - त॒प्या॑य । च॒ नमः॑ । नमो॒ वात्या॑य । वात्या॑य च । च॒ रेष्मि॑याय । रेष्मि॑याय च । च॒ नमः॑ । नमो॑ वास्त॒व्या॑य । वा॒स्त॒व्या॑य च । च॒ वा॒स्तु॒पाय॑ । वा॒स्तु॒पाय॑ च । वा॒स्तु॒पायेति॑ वास्तु - पाय॑ । चेति॑ च । \newline

\textbf{Jatai Paata} \newline

1. नमः॒ कूप्या॑य॒ कूप्या॑य॒ नमो॒ नमः॒ कूप्या॑य । \newline
2. कूप्या॑य च च॒ कूप्या॑य॒ कूप्या॑य च । \newline
3. चा॒व॒ट्या॑या व॒ट्या॑य च चाव॒ट्या॑य । \newline
4. अ॒व॒ट्या॑य च चाव॒ट्या॑या व॒ट्या॑य च । \newline
5. च॒ नमो॒ नम॑श्च च॒ नमः॑ । \newline
6. नमो॒ वर्ष्या॑य॒ वर्ष्या॑य॒ नमो॒ नमो॒ वर्ष्या॑य । \newline
7. वर्ष्या॑य च च॒ वर्ष्या॑य॒ वर्ष्या॑य च । \newline
8. चा॒व॒र्ष्याया॑ व॒र्ष्याय॑ च चाव॒र्ष्याय॑ । \newline
9. अ॒व॒र्ष्याय॑ च चाव॒र्ष्याया॑ व॒र्ष्याय॑ च । \newline
10. च॒ नमो॒ नम॑श्च च॒ नमः॑ । \newline
11. नमो॑ मे॒घ्या॑य मे॒घ्या॑य॒ नमो॒ नमो॑ मे॒घ्या॑य । \newline
12. मे॒घ्या॑य च च मे॒घ्या॑य मे॒घ्या॑य च । \newline
13. च॒ वि॒द्यु॒त्या॑य विद्यु॒त्या॑य च च विद्यु॒त्या॑य । \newline
14. वि॒द्यु॒त्या॑य च च विद्यु॒त्या॑य विद्यु॒त्या॑य च । \newline
15. वि॒द्यु॒त्या॑येति॑ वि - द्यु॒त्या॑य । \newline
16. च॒ नमो॒ नम॑श्च च॒ नमः॑ । \newline
17. नम॑ ई॒द्ध्रिया॑ ये॒द्ध्रिया॑य॒ नमो॒ नम॑ ई॒द्ध्रिया॑य । \newline
18. ई॒द्ध्रिया॑य च चे॒द्ध्रिया॑ ये॒द्ध्रिया॑य च । \newline
19. चा॒त॒प्या॑या त॒प्या॑य च चात॒प्या॑य । \newline
20. आ॒त॒प्या॑य च चात॒प्या॑या त॒प्या॑य च । \newline
21. आ॒त॒प्या॑येत्या᳚ - त॒प्या॑य । \newline
22. च॒ नमो॒ नम॑श्च च॒ नमः॑ । \newline
23. नमो॒ वात्या॑य॒ वात्या॑य॒ नमो॒ नमो॒ वात्या॑य । \newline
24. वात्या॑य च च॒ वात्या॑य॒ वात्या॑य च । \newline
25. च॒ रेष्मि॑याय॒ रेष्मि॑याय च च॒ रेष्मि॑याय । \newline
26. रेष्मि॑याय च च॒ रेष्मि॑याय॒ रेष्मि॑याय च । \newline
27. च॒ नमो॒ नम॑श्च च॒ नमः॑ । \newline
28. नमो॑ वास्त॒व्या॑य वास्त॒व्या॑य॒ नमो॒ नमो॑ वास्त॒व्या॑य । \newline
29. वा॒स्त॒व्या॑य च च वास्त॒व्या॑य वास्त॒व्या॑य च । \newline
30. च॒ वा॒स्तु॒पाय॑ वास्तु॒पाय॑ च च वास्तु॒पाय॑ । \newline
31. वा॒स्तु॒पाय॑ च च वास्तु॒पाय॑ वास्तु॒पाय॑ च । \newline
32. वा॒स्तु॒पायेति॑ वास्तु - पाय॑ । \newline
33. चेति॑ च । \newline

\textbf{Ghana Paata } \newline

1. नमः॒ कूप्या॑य॒ कूप्या॑य॒ नमो॒ नमः॒ कूप्या॑य च च॒ कूप्या॑य॒ नमो॒ नमः॒ कूप्या॑य च । \newline
2. कूप्या॑य च च॒ कूप्या॑य॒ कूप्या॑य चाव॒ट्या॑या व॒ट्या॑य च॒ कूप्या॑य॒ कूप्या॑य चाव॒ट्या॑य । \newline
3. चा॒व॒ट्या॑या व॒ट्या॑य च चाव॒ट्या॑य च चाव॒ट्या॑य च चाव॒ट्या॑य च । \newline
4. अ॒व॒ट्या॑य च चाव॒ट्या॑या व॒ट्या॑य च॒ नमो॒ नम॑ श्चाव॒ट्या॑या व॒ट्या॑य च॒ नमः॑ । \newline
5. च॒ नमो॒ नम॑श्च च॒ नमो॒ वर्ष्या॑य॒ वर्ष्या॑य॒ नम॑श्च च॒ नमो॒ वर्ष्या॑य । \newline
6. नमो॒ वर्ष्या॑य॒ वर्ष्या॑य॒ नमो॒ नमो॒ वर्ष्या॑य च च॒ वर्ष्या॑य॒ नमो॒ नमो॒ वर्ष्या॑य च । \newline
7. वर्ष्या॑य च च॒ वर्ष्या॑य॒ वर्ष्या॑य चाव॒र्ष्याया॑ व॒र्ष्याय॑ च॒ वर्ष्या॑य॒ वर्ष्या॑य चाव॒र्ष्याय॑ । \newline
8. चा॒व॒र्ष्याया॑ व॒र्ष्याय॑ च चाव॒र्ष्याय॑ च चाव॒र्ष्याय॑ च चाव॒र्ष्याय॑ च । \newline
9. अ॒व॒र्ष्याय॑ च चाव॒र्ष्याया॑ व॒र्ष्याय॑ च॒ नमो॒ नम॑ श्चाव॒र्ष्याया॑ व॒र्ष्याय॑ च॒ नमः॑ । \newline
10. च॒ नमो॒ नम॑श्च च॒ नमो॑ मे॒घ्या॑य मे॒घ्या॑य॒ नम॑श्च च॒ नमो॑ मे॒घ्या॑य । \newline
11. नमो॑ मे॒घ्या॑य मे॒घ्या॑य॒ नमो॒ नमो॑ मे॒घ्या॑य च च मे॒घ्या॑य॒ नमो॒ नमो॑ मे॒घ्या॑य च । \newline
12. मे॒घ्या॑य च च मे॒घ्या॑य मे॒घ्या॑य च विद्यु॒त्या॑य विद्यु॒त्या॑य च मे॒घ्या॑य मे॒घ्या॑य च विद्यु॒त्या॑य । \newline
13. च॒ वि॒द्यु॒त्या॑य विद्यु॒त्या॑य च च विद्यु॒त्या॑य च च विद्यु॒त्या॑य च च विद्यु॒त्या॑य च । \newline
14. वि॒द्यु॒त्या॑य च च विद्यु॒त्या॑य विद्यु॒त्या॑य च॒ नमो॒ नम॑श्च विद्यु॒त्या॑य विद्यु॒त्या॑य च॒ नमः॑ । \newline
15. वि॒द्यु॒त्या॑येति॑ वि - द्यु॒त्या॑य । \newline
16. च॒ नमो॒ नम॑श्च च॒ नम॑ ई॒द्ध्रिया॑ ये॒द्ध्रिया॑य॒ नम॑श्च च॒ नम॑ ई॒द्ध्रिया॑य । \newline
17. नम॑ ई॒द्ध्रिया॑ ये॒द्ध्रिया॑य॒ नमो॒ नम॑ ई॒द्ध्रिया॑य च चे॒द्ध्रिया॑य॒ नमो॒ नम॑ ई॒द्ध्रिया॑य च । \newline
18. ई॒द्ध्रिया॑य च चे॒द्ध्रिया॑ ये॒द्ध्रिया॑य चात॒प्या॑या त॒प्या॑य चे॒द्ध्रिया॑ ये॒द्ध्रिया॑य चात॒प्या॑य । \newline
19. चा॒त॒प्या॑या त॒प्या॑य च चात॒प्या॑य च चात॒प्या॑य च चात॒प्या॑य च । \newline
20. आ॒त॒प्या॑य च चात॒प्या॑या त॒प्या॑य च॒ नमो॒ नम॑ श्चात॒प्या॑या त॒प्या॑य च॒ नमः॑ । \newline
21. आ॒त॒प्या॑येत्या᳚ - त॒प्या॑य । \newline
22. च॒ नमो॒ नम॑श्च च॒ नमो॒ वात्या॑य॒ वात्या॑य॒ नम॑श्च च॒ नमो॒ वात्या॑य । \newline
23. नमो॒ वात्या॑य॒ वात्या॑य॒ नमो॒ नमो॒ वात्या॑य च च॒ वात्या॑य॒ नमो॒ नमो॒ वात्या॑य च । \newline
24. वात्या॑य च च॒ वात्या॑य॒ वात्या॑य च॒ रेष्मि॑याय॒ रेष्मि॑याय च॒ वात्या॑य॒ वात्या॑य च॒ रेष्मि॑याय । \newline
25. च॒ रेष्मि॑याय॒ रेष्मि॑याय च च॒ रेष्मि॑याय च च॒ रेष्मि॑याय च च॒ रेष्मि॑याय च । \newline
26. रेष्मि॑याय च च॒ रेष्मि॑याय॒ रेष्मि॑याय च॒ नमो॒ नम॑श्च॒ रेष्मि॑याय॒ रेष्मि॑याय च॒ नमः॑ । \newline
27. च॒ नमो॒ नम॑श्च च॒ नमो॑ वास्त॒व्या॑य वास्त॒व्या॑य॒ नम॑श्च च॒ नमो॑ वास्त॒व्या॑य । \newline
28. नमो॑ वास्त॒व्या॑य वास्त॒व्या॑य॒ नमो॒ नमो॑ वास्त॒व्या॑य च च वास्त॒व्या॑य॒ नमो॒ नमो॑ वास्त॒व्या॑य च । \newline
29. वा॒स्त॒व्या॑य च च वास्त॒व्या॑य वास्त॒व्या॑य च वास्तु॒पाय॑ वास्तु॒पाय॑ च वास्त॒व्या॑य वास्त॒व्या॑य च वास्तु॒पाय॑ । \newline
30. च॒ वा॒स्तु॒पाय॑ वास्तु॒पाय॑ च च वास्तु॒पाय॑ च च वास्तु॒पाय॑ च च वास्तु॒पाय॑ च । \newline
31. वा॒स्तु॒पाय॑ च च वास्तु॒पाय॑ वास्तु॒पाय॑ च । \newline
32. वा॒स्तु॒पायेति॑ वास्तु - पाय॑ । \newline
33. चेति॑ च । \newline
\pagebreak
\markright{ TS 4.5.8.1  \hfill https://www.vedavms.in \hfill}

\section{ TS 4.5.8.1 }

\textbf{TS 4.5.8.1 } \newline
\textbf{Samhita Paata} \newline

नमः॒ सोमा॑य च रु॒द्राय॑ च॒ नम॑स्ता॒म्राय॑ चारु॒णाय॑ च॒ नमः॑ श॒ङ्गाय॑ च पशु॒पत॑ये च॒ नम॑ उ॒ग्राय॑ च भी॒माय॑ च॒ नमो॑ अग्रेव॒धाय॑ च दूरेव॒धाय॑ च॒ नमो॑ ह॒न्त्रे च॒ हनी॑यसे च॒ नमो॑ वृ॒क्षेभ्यो॒ हरि॑केशेभ्यो॒ नम॑स्ता॒राय॒ नमः॑ श॒भंवे॑ च मयो॒भवे॑ च॒ नमः॑ शङ्क॒राय॑ च मयस्क॒राय॑ च॒ नमः॑ शि॒वाय॑ च शि॒वत॑राय च॒ - [  ] \newline

\textbf{Pada Paata} \newline

नमः॑ । सोमा॑य । च॒ । रु॒द्राय॑ । च॒ । नमः॑ । ता॒म्राय॑ । च॒ । अ॒रु॒णाय॑ । च॒ । नमः॑ । श॒ङ्गाय॑ । च॒ । प॒शु॒पत॑य॒ इति॑ पशु-पत॑ये । च॒ । नमः॑ । उ॒ग्राय॑ । च॒ । भी॒माय॑ । च॒ । नमः॑ । अ॒ग्रे॒व॒धायेत्य॑ग्रे - व॒धाय॑ । च॒ । दू॒रे॒व॒धायेति॑ दूरे - व॒धाय॑ । च॒ । नमः॑ । ह॒न्त्रे । च॒ । हनी॑यसे । च॒ । नमः॑ । वृ॒क्षेभ्यः॑ । हरि॑केशेभ्य॒ इति॒ हरि॑ - के॒शे॒भ्यः॒ । नमः॑ । ता॒राय॑ । नमः॑ । श॒भंव॒ इति॑ शं - भवे᳚ । च॒ । म॒यो॒भव॒ इति॑ मयः - भवे᳚ । च॒ । नमः॑ । श॒ङ्क॒रायेति॑ शं - क॒राय॑ । च॒ । म॒य॒स्क॒रायेति॑ मयः - क॒राय॑ । च॒ । नमः॑ । शि॒वाय॑ । च॒ । शि॒वत॑रा॒येति॑ शि॒व - त॒रा॒य॒ । च॒ ।  \newline


\textbf{Krama Paata} \newline

नमः॒ सोमा॑य । सोमा॑य च । च॒ रु॒द्राय॑ । रु॒द्राय॑ च । च॒ नमः॑ । नम॑स्ता॒म्राय॑ । ता॒म्राय॑ च । चा॒रु॒णाय॑ । अ॒रु॒णाय॑ च । च॒ नमः॑ । नमः॑ श॒ङ्गाय॑ । श॒ङ्गाय॑ च । च॒ प॒शु॒पत॑ये । प॒शु॒पत॑ये च । प॒शु॒पत॑य॒ इति॑ पशु - पत॑ये । च॒ नमः॑ । नम॑ उ॒ग्राय॑ । उ॒ग्राय॑ च । च॒ भी॒माय॑ । भी॒माय॑ च । च॒ नमः॑ । नमो॑ अग्रेव॒धाय॑ । अ॒ग्रे॒व॒धाय॑ च । अ॒ग्रे॒व॒धायेत्य॑ग्रे - व॒धाय॑ । च॒ दू॒रे॒व॒धाय॑ । दू॒रे॒व॒धाय॑ च । दू॒रे॒व॒धायेति॑ दूरे - व॒धाय॑ । च॒ नमः॑ । नमो॑ ह॒न्त्रे । ह॒न्त्रे च॑ । च॒ हनी॑यसे । हनी॑यसे च । च॒ नमः॑ । नमो॑ वृ॒क्षेभ्यः॑ । वृ॒क्षेभ्यो॒ हरि॑केशेभ्यः । हरि॑केशेभ्यो॒ नमः॑ । हरि॑केशेभ्य॒ इति॒ हरि॑ - के॒शे॒भ्यः॒ । नम॑स्ता॒राय॑ । ता॒राय॒ नमः॑ । नमः॑ श॒म्भवे᳚ । श॒म्भवे॑ च । श॒म्भव॒ इति॑ शं - भवे᳚ । च॒ म॒यो॒भवे᳚ । म॒यो॒भवे॑ च । म॒यो॒भव॒ इति॑ मयः - भवे᳚ । च॒ नमः॑ । नमः॑ शङ्क॒राय॑ । श॒ङ्क॒राय॑ च । श॒ङ्क॒रायेति॑ शं - क॒राय॑ । च॒ म॒य॒स्क॒राय॑ । म॒य॒स्क॒राय॑ च । म॒य॒स्क॒रायेति॑ मयः - क॒राय॑ । च॒ नमः॑ । नमः॑ शि॒वाय॑ । शि॒वाय॑ च । च॒ शि॒वत॑राय । शि॒वत॑राय च ( ) । शि॒वत॑रा॒येति॑ शि॒व - त॒रा॒य॒ । च॒ नमः॑ \newline

\textbf{Jatai Paata} \newline

1. नमः॒ सोमा॑य॒ सोमा॑य॒ नमो॒ नमः॒ सोमा॑य । \newline
2. सोमा॑य च च॒ सोमा॑य॒ सोमा॑य च । \newline
3. च॒ रु॒द्राय॑ रु॒द्राय॑ च च रु॒द्राय॑ । \newline
4. रु॒द्राय॑ च च रु॒द्राय॑ रु॒द्राय॑ च । \newline
5. च॒ नमो॒ नम॑श्च च॒ नमः॑ । \newline
6. नम॑ स्ता॒म्राय॑ ता॒म्राय॒ नमो॒ नम॑ स्ता॒म्राय॑ । \newline
7. ता॒म्राय॑ च च ता॒म्राय॑ ता॒म्राय॑ च । \newline
8. चा॒रु॒णाया॑ रु॒णाय॑ च चारु॒णाय॑ । \newline
9. अ॒रु॒णाय॑ च चारु॒णाया॑ रु॒णाय॑ च । \newline
10. च॒ नमो॒ नम॑श्च च॒ नमः॑ । \newline
11. नमः॑ श॒ङ्गाय॑ श॒ङ्गाय॒ नमो॒ नमः॑ श॒ङ्गाय॑ । \newline
12. श॒ङ्गाय॑ च च श॒ङ्गाय॑ श॒ङ्गाय॑ च । \newline
13. च॒ प॒शु॒पत॑ये पशु॒पत॑ये च च पशु॒पत॑ये । \newline
14. प॒शु॒पत॑ये च च पशु॒पत॑ये पशु॒पत॑ये च । \newline
15. प॒शु॒पत॑य॒ इति॑ पशु - पत॑ये । \newline
16. च॒ नमो॒ नम॑श्च च॒ नमः॑ । \newline
17. नम॑ उ॒ग्रा यो॒ग्राय॒ नमो॒ नम॑ उ॒ग्राय॑ । \newline
18. उ॒ग्राय॑ च चो॒ग्रा यो॒ग्राय॑ च । \newline
19. च॒ भी॒माय॑ भी॒माय॑ च च भी॒माय॑ । \newline
20. भी॒माय॑ च च भी॒माय॑ भी॒माय॑ च । \newline
21. च॒ नमो॒ नम॑श्च च॒ नमः॑ । \newline
22. नमो॑ अग्रेव॒धाया᳚ ग्रेव॒धाय॒ नमो॒ नमो॑ अग्रेव॒धाय॑ । \newline
23. अ॒ग्रे॒व॒धाय॑ च चाग्रेव॒धाया᳚ ग्रेव॒धाय॑ च । \newline
24. अ॒ग्रे॒व॒धायेत्य॑ग्रे - व॒धाय॑ । \newline
25. च॒ दू॒रे॒व॒धाय॑ दूरेव॒धाय॑ च च दूरेव॒धाय॑ । \newline
26. दू॒रे॒व॒धाय॑ च च दूरेव॒धाय॑ दूरेव॒धाय॑ च । \newline
27. दू॒रे॒व॒धायेति॑ दूरे - व॒धाय॑ । \newline
28. च॒ नमो॒ नम॑श्च च॒ नमः॑ । \newline
29. नमो॑ ह॒न्त्रे ह॒न्त्रे नमो॒ नमो॑ ह॒न्त्रे । \newline
30. ह॒न्त्रे च॑ च ह॒न्त्रे ह॒न्त्रे च॑ । \newline
31. च॒ हनी॑यसे॒ हनी॑यसे च च॒ हनी॑यसे । \newline
32. हनी॑यसे च च॒ हनी॑यसे॒ हनी॑यसे च । \newline
33. च॒ नमो॒ नम॑श्च च॒ नमः॑ । \newline
34. नमो॑ वृ॒क्षेभ्यो॑ वृ॒क्षेभ्यो॒ नमो॒ नमो॑ वृ॒क्षेभ्यः॑ । \newline
35. वृ॒क्षेभ्यो॒ हरि॑केशेभ्यो॒ हरि॑केशेभ्यो वृ॒क्षेभ्यो॑ वृ॒क्षेभ्यो॒ हरि॑केशेभ्यः । \newline
36. हरि॑केशेभ्यो॒ नमो॒ नमो॒ हरि॑केशेभ्यो॒ हरि॑केशेभ्यो॒ नमः॑ । \newline
37. हरि॑केशेभ्य॒ इति॒ हरि॑ - के॒शे॒भ्यः॒ । \newline
38. नम॑ स्ता॒राय॑ ता॒राय॒ नमो॒ नम॑ स्ता॒राय॑ । \newline
39. ता॒राय॒ नमो॒ नम॑ स्ता॒राय॑ ता॒राय॒ नमः॑ । \newline
40. नमः॑ शं॒भवे॑ शं॒भवे॒ नमो॒ नमः॑ शं॒भवे᳚ । \newline
41. शं॒भवे॑ च च शं॒भवे॑ शं॒भवे॑ च । \newline
42. शं॒भव॒ इति॑ शं - भवे᳚ । \newline
43. च॒ म॒यो॒भवे॑ मयो॒भवे॑ च च मयो॒भवे᳚ । \newline
44. म॒यो॒भवे॑ च च मयो॒भवे॑ मयो॒भवे॑ च । \newline
45. म॒यो॒भव॒ इति॑ मयः - भवे᳚ । \newline
46. च॒ नमो॒ नम॑श्च च॒ नमः॑ । \newline
47. नमः॑ शङ्क॒राय॑ शङ्क॒राय॒ नमो॒ नमः॑ शङ्क॒राय॑ । \newline
48. श॒ङ्क॒राय॑ च च शङ्क॒राय॑ शङ्क॒राय॑ च । \newline
49. श॒ङ्क॒रायेति॑ शं - क॒राय॑ । \newline
50. च॒ म॒य॒स्क॒राय॑ मयस्क॒राय॑ च च मयस्क॒राय॑ । \newline
51. म॒य॒स्क॒राय॑ च च मयस्क॒राय॑ मयस्क॒राय॑ च । \newline
52. म॒य॒स्क॒रायेति॑ मयः - क॒राय॑ । \newline
53. च॒ नमो॒ नम॑श्च च॒ नमः॑ । \newline
54. नमः॑ शि॒वाय॑ शि॒वाय॒ नमो॒ नमः॑ शि॒वाय॑ । \newline
55. शि॒वाय॑ च च शि॒वाय॑ शि॒वाय॑ च । \newline
56. च॒ शि॒वत॑राय शि॒वत॑राय च च शि॒वत॑राय । \newline
57. शि॒वत॑राय च च शि॒वत॑राय शि॒वत॑राय च । \newline
58. शि॒वत॑रा॒येति॑ शि॒व - त॒रा॒य॒ । \newline
59. च॒ नमो॒ नम॑श्च च॒ नमः॑ । \newline

\textbf{Ghana Paata } \newline

1. नमः॒ सोमा॑य॒ सोमा॑य॒ नमो॒ नमः॒ सोमा॑य च च॒ सोमा॑य॒ नमो॒ नमः॒ सोमा॑य च । \newline
2. सोमा॑य च च॒ सोमा॑य॒ सोमा॑य च रु॒द्राय॑ रु॒द्राय॑ च॒ सोमा॑य॒ सोमा॑य च रु॒द्राय॑ । \newline
3. च॒ रु॒द्राय॑ रु॒द्राय॑ च च रु॒द्राय॑ च च रु॒द्राय॑ च च रु॒द्राय॑ च । \newline
4. रु॒द्राय॑ च च रु॒द्राय॑ रु॒द्राय॑ च॒ नमो॒ नम॑श्च रु॒द्राय॑ रु॒द्राय॑ च॒ नमः॑ । \newline
5. च॒ नमो॒ नम॑श्च च॒ नम॑ स्ता॒म्राय॑ ता॒म्राय॒ नम॑श्च च॒ नम॑ स्ता॒म्राय॑ । \newline
6. नम॑ स्ता॒म्राय॑ ता॒म्राय॒ नमो॒ नम॑ स्ता॒म्राय॑ च च ता॒म्राय॒ नमो॒ नम॑ स्ता॒म्राय॑ च । \newline
7. ता॒म्राय॑ च च ता॒म्राय॑ ता॒म्राय॑ चारु॒णाया॑ रु॒णाय॑ च ता॒म्राय॑ ता॒म्राय॑ चारु॒णाय॑ । \newline
8. चा॒रु॒णाया॑ रु॒णाय॑ च चारु॒णाय॑ च चारु॒णाय॑ च चारु॒णाय॑ च । \newline
9. अ॒रु॒णाय॑ च चारु॒णाया॑ रु॒णाय॑ च॒ नमो॒ नम॑ श्चारु॒णाया॑ रु॒णाय॑ च॒ नमः॑ । \newline
10. च॒ नमो॒ नम॑श्च च॒ नमः॑ श॒ङ्गाय॑ श॒ङ्गाय॒ नम॑श्च च॒ नमः॑ श॒ङ्गाय॑ । \newline
11. नमः॑ श॒ङ्गाय॑ श॒ङ्गाय॒ नमो॒ नमः॑ श॒ङ्गाय॑ च च श॒ङ्गाय॒ नमो॒ नमः॑ श॒ङ्गाय॑ च । \newline
12. श॒ङ्गाय॑ च च श॒ङ्गाय॑ श॒ङ्गाय॑ च पशु॒पत॑ये पशु॒पत॑ये च श॒ङ्गाय॑ श॒ङ्गाय॑ च पशु॒पत॑ये । \newline
13. च॒ प॒शु॒पत॑ये पशु॒पत॑ये च च पशु॒पत॑ये च च पशु॒पत॑ये च च पशु॒पत॑ये च । \newline
14. प॒शु॒पत॑ये च च पशु॒पत॑ये पशु॒पत॑ये च॒ नमो॒ नम॑श्च पशु॒पत॑ये पशु॒पत॑ये च॒ नमः॑ । \newline
15. प॒शु॒पत॑य॒ इति॑ पशु - पत॑ये । \newline
16. च॒ नमो॒ नम॑श्च च॒ नम॑ उ॒ग्रा यो॒ग्राय॒ नम॑श्च च॒ नम॑ उ॒ग्राय॑ । \newline
17. नम॑ उ॒ग्रा यो॒ग्राय॒ नमो॒ नम॑ उ॒ग्राय॑ च चो॒ग्राय॒ नमो॒ नम॑ उ॒ग्राय॑ च । \newline
18. उ॒ग्राय॑ च चो॒ग्रा यो॒ग्राय॑ च भी॒माय॑ भी॒माय॑ चो॒ग्रा यो॒ग्राय॑ च भी॒माय॑ । \newline
19. च॒ भी॒माय॑ भी॒माय॑ च च भी॒माय॑ च च भी॒माय॑ च च भी॒माय॑ च । \newline
20. भी॒माय॑ च च भी॒माय॑ भी॒माय॑ च॒ नमो॒ नम॑श्च भी॒माय॑ भी॒माय॑ च॒ नमः॑ । \newline
21. च॒ नमो॒ नम॑श्च च॒ नमो॑ अग्रेव॒धाया᳚ ग्रेव॒धाय॒ नम॑श्च च॒ नमो॑ अग्रेव॒धाय॑ । \newline
22. नमो॑ अग्रेव॒धाया᳚ ग्रेव॒धाय॒ नमो॒ नमो॑ अग्रेव॒धाय॑ च चाग्रेव॒धाय॒ नमो॒ नमो॑ अग्रेव॒धाय॑ च । \newline
23. अ॒ग्रे॒व॒धाय॑ च चाग्रेव॒धाया᳚ ग्रेव॒धाय॑ च दूरेव॒धाय॑ दूरेव॒धाय॑ चाग्रेव॒धाया᳚ ग्रेव॒धाय॑ च दूरेव॒धाय॑ । \newline
24. अ॒ग्रे॒व॒धायेत्य॑ग्रे - व॒धाय॑ । \newline
25. च॒ दू॒रे॒व॒धाय॑ दूरेव॒धाय॑ च च दूरेव॒धाय॑ च च दूरेव॒धाय॑ च च दूरेव॒धाय॑ च । \newline
26. दू॒रे॒व॒धाय॑ च च दूरेव॒धाय॑ दूरेव॒धाय॑ च॒ नमो॒ नम॑श्च दूरेव॒धाय॑ दूरेव॒धाय॑ च॒ नमः॑ । \newline
27. दू॒रे॒व॒धायेति॑ दूरे - व॒धाय॑ । \newline
28. च॒ नमो॒ नम॑श्च च॒ नमो॑ ह॒न्त्रे ह॒न्त्रे नम॑श्च च॒ नमो॑ ह॒न्त्रे । \newline
29. नमो॑ ह॒न्त्रे ह॒न्त्रे नमो॒ नमो॑ ह॒न्त्रे च॑ च ह॒न्त्रे नमो॒ नमो॑ ह॒न्त्रे च॑ । \newline
30. ह॒न्त्रे च॑ च ह॒न्त्रे ह॒न्त्रे च॒ हनी॑यसे॒ हनी॑यसे च ह॒न्त्रे ह॒न्त्रे च॒ हनी॑यसे । \newline
31. च॒ हनी॑यसे॒ हनी॑यसे च च॒ हनी॑यसे च च॒ हनी॑यसे च च॒ हनी॑यसे च । \newline
32. हनी॑यसे च च॒ हनी॑यसे॒ हनी॑यसे च॒ नमो॒ नम॑श्च॒ हनी॑यसे॒ हनी॑यसे च॒ नमः॑ । \newline
33. च॒ नमो॒ नम॑श्च च॒ नमो॑ वृ॒क्षेभ्यो॑ वृ॒क्षेभ्यो॒ नम॑श्च च॒ नमो॑ वृ॒क्षेभ्यः॑ । \newline
34. नमो॑ वृ॒क्षेभ्यो॑ वृ॒क्षेभ्यो॒ नमो॒ नमो॑ वृ॒क्षेभ्यो॒ हरि॑केशेभ्यो॒ हरि॑केशेभ्यो वृ॒क्षेभ्यो॒ नमो॒ नमो॑ वृ॒क्षेभ्यो॒ हरि॑केशेभ्यः । \newline
35. वृ॒क्षेभ्यो॒ हरि॑केशेभ्यो॒ हरि॑केशेभ्यो वृ॒क्षेभ्यो॑ वृ॒क्षेभ्यो॒ हरि॑केशेभ्यो॒ नमो॒ नमो॒ हरि॑केशेभ्यो वृ॒क्षेभ्यो॑ वृ॒क्षेभ्यो॒ हरि॑केशेभ्यो॒ नमः॑ । \newline
36. हरि॑केशेभ्यो॒ नमो॒ नमो॒ हरि॑केशेभ्यो॒ हरि॑केशेभ्यो॒ नम॑ स्ता॒राय॑ ता॒राय॒ नमो॒ हरि॑केशेभ्यो॒ हरि॑केशेभ्यो॒ नम॑ स्ता॒राय॑ । \newline
37. हरि॑केशेभ्य॒ इति॒ हरि॑ - के॒शे॒भ्यः॒ । \newline
38. नम॑ स्ता॒राय॑ ता॒राय॒ नमो॒ नम॑ स्ता॒राय॒ नमो॒ नम॑ स्ता॒राय॒ नमो॒ नम॑ स्ता॒राय॒ नमः॑ । \newline
39. ता॒राय॒ नमो॒ नम॑ स्ता॒राय॑ ता॒राय॒ नमः॑ शं॒भवे॑ शं॒भवे॒ नम॑ स्ता॒राय॑ ता॒राय॒ नमः॑ शं॒भवे᳚ । \newline
40. नमः॑ शं॒भवे॑ शं॒भवे॒ नमो॒ नमः॑ शं॒भवे॑ च च शं॒भवे॒ नमो॒ नमः॑ शं॒भवे॑ च । \newline
41. शं॒भवे॑ च च शं॒भवे॑ शं॒भवे॑ च मयो॒भवे॑ मयो॒भवे॑ च शं॒भवे॑ शं॒भवे॑ च मयो॒भवे᳚ । \newline
42. शं॒भव॒ इति॑ शं - भवे᳚ । \newline
43. च॒ म॒यो॒भवे॑ मयो॒भवे॑ च च मयो॒भवे॑ च च मयो॒भवे॑ च च मयो॒भवे॑ च । \newline
44. म॒यो॒भवे॑ च च मयो॒भवे॑ मयो॒भवे॑ च॒ नमो॒ नम॑श्च मयो॒भवे॑ मयो॒भवे॑ च॒ नमः॑ । \newline
45. म॒यो॒भव॒ इति॑ मयः - भवे᳚ । \newline
46. च॒ नमो॒ नम॑श्च च॒ नमः॑ शङ्क॒राय॑ शङ्क॒राय॒ नम॑श्च च॒ नमः॑ शङ्क॒राय॑ । \newline
47. नमः॑ शङ्क॒राय॑ शङ्क॒राय॒ नमो॒ नमः॑ शङ्क॒राय॑ च च शङ्क॒राय॒ नमो॒ नमः॑ शङ्क॒राय॑ च । \newline
48. श॒ङ्क॒राय॑ च च शङ्क॒राय॑ शङ्क॒राय॑ च मयस्क॒राय॑ मयस्क॒राय॑ च शङ्क॒राय॑ शङ्क॒राय॑ च मयस्क॒राय॑ । \newline
49. श॒ङ्क॒रायेति॑ शं - क॒राय॑ । \newline
50. च॒ म॒य॒स्क॒राय॑ मयस्क॒राय॑ च च मयस्क॒राय॑ च च मयस्क॒राय॑ च च मयस्क॒राय॑ च । \newline
51. म॒य॒स्क॒राय॑ च च मयस्क॒राय॑ मयस्क॒राय॑ च॒ नमो॒ नम॑श्च मयस्क॒राय॑ मयस्क॒राय॑ च॒ नमः॑ । \newline
52. म॒य॒स्क॒रायेति॑ मयः - क॒राय॑ । \newline
53. च॒ नमो॒ नम॑श्च च॒ नमः॑ शि॒वाय॑ शि॒वाय॒ नम॑श्च च॒ नमः॑ शि॒वाय॑ । \newline
54. नमः॑ शि॒वाय॑ शि॒वाय॒ नमो॒ नमः॑ शि॒वाय॑ च च शि॒वाय॒ नमो॒ नमः॑ शि॒वाय॑ च । \newline
55. शि॒वाय॑ च च शि॒वाय॑ शि॒वाय॑ च शि॒वत॑राय शि॒वत॑राय च शि॒वाय॑ शि॒वाय॑ च शि॒वत॑राय । \newline
56. च॒ शि॒वत॑राय शि॒वत॑राय च च शि॒वत॑राय च च शि॒वत॑राय च च शि॒वत॑राय च । \newline
57. शि॒वत॑राय च च शि॒वत॑राय शि॒वत॑राय च॒ नमो॒ नम॑श्च शि॒वत॑राय शि॒वत॑राय च॒ नमः॑ । \newline
58. शि॒वत॑रा॒येति॑ शि॒व - त॒रा॒य॒ । \newline
59. च॒ नमो॒ नम॑श्च च॒ नम॒ स्तीर्थ्या॑य॒ तीर्थ्या॑य॒ नम॑श्च च॒ नम॒ स्तीर्थ्या॑य । \newline
\pagebreak
\markright{ TS 4.5.8.2  \hfill https://www.vedavms.in \hfill}

\section{ TS 4.5.8.2 }

\textbf{TS 4.5.8.2 } \newline
\textbf{Samhita Paata} \newline

नम॒स्तीर्थ्या॑य च॒ कूल्या॑य च॒ नमः॑ पा॒र्या॑य चावा॒र्या॑य च॒ नमः॑ प्र॒तर॑णाय चो॒त्तर॑णाय च॒ नम॑ आता॒र्या॑य चाला॒द्या॑य च॒ नमः॒ शष्प्या॑य च॒ फेन्या॑य च॒ नमः॑ सिक॒त्या॑य च प्रवा॒ह्या॑य च ॥ \newline

\textbf{Pada Paata} \newline

नमः॑ । तीर्थ्या॑य । च॒ । कूल्या॑य । च॒ । नमः॑ । पा॒र्या॑य । च॒ । अ॒वा॒र्या॑य । च॒ । नमः॑ । प्र॒तर॑णा॒येति॑ प्र - तर॑णाय । च॒ । उ॒त्तर॑णा॒येत्यु॑त् - तर॑णाय । च॒ । नमः॑ । आ॒ता॒र्या॑येत्या᳚-ता॒र्या॑य । च॒ । आ॒ला॒द्या॑येत्या᳚ - ला॒द्या॑य। च॒ । नमः॑ । शष्प्या॑य । च॒ । फेन्या॑य । च॒ । नमः॑ । सि॒क॒त्या॑य । च॒ । प्र॒वा॒ह्या॑येति॑ प्र - वा॒ह्या॑य । च॒ ।  \newline


\textbf{Krama Paata} \newline

नम॒स्तीर्त्थ्या॑य । तीर्त्थ्या॑य च । च॒ कूल्या॑य । कूल्या॑य च । च॒ नमः॑ । नमः॑ पा॒र्या॑य । पा॒र्या॑य च । चा॒वा॒र्या॑य । अ॒वा॒र्या॑य च । च॒ नमः॑ । नमः॑ प्र॒तर॑णाय । प्र॒तर॑णाय च । प्र॒तर॑णा॒येति॑ प्र - तर॑णाय । चो॒त्तर॑णाय । उ॒त्तर॑णाय च । उ॒त्तर॑णा॒येत्यु॑त् - तर॑णाय । च॒ नमः॑ । नम॑ आता॒र्या॑य । आ॒ता॒र्या॑य च । आ॒ता॒र्या॑येत्या᳚ - ता॒र्या॑य । चा॒ला॒द्या॑य । आ॒ला॒द्या॑य च । आ॒ला॒द्या॑येत्या᳚ - ला॒द्या॑य । च॒ नमः॑ । नमः॒ शष्प्या॑य । शष्प्या॑य च । च॒ फेन्या॑य । फेन्या॑य च । च॒ नमः॑ । नमः॑ सिक॒त्या॑य । सि॒क॒त्या॑य च । च॒ प्र॒वा॒ह्या॑य । प्र॒वा॒ह्या॑य च । प्र॒वा॒ह्या॑येति॑ प्र - वा॒ह्या॑य । चेति॑ च । \newline

\textbf{Jatai Paata} \newline

1. नम॒ स्तीर्थ्या॑य॒ तीर्थ्या॑य॒ नमो॒ नम॒ स्तीर्थ्या॑य । \newline
2. तीर्थ्या॑य च च॒ तीर्थ्या॑य॒ तीर्थ्या॑य च । \newline
3. च॒ कूल्या॑य॒ कूल्या॑य च च॒ कूल्या॑य । \newline
4. कूल्या॑य च च॒ कूल्या॑य॒ कूल्या॑य च । \newline
5. च॒ नमो॒ नम॑श्च च॒ नमः॑ । \newline
6. नमः॑ पा॒र्या॑य पा॒र्या॑य॒ नमो॒ नमः॑ पा॒र्या॑य । \newline
7. पा॒र्या॑य च च पा॒र्या॑य पा॒र्या॑य च । \newline
8. चा॒वा॒र्या॑या वा॒र्या॑य च चावा॒र्या॑य । \newline
9. अ॒वा॒र्या॑य च चावा॒र्या॑या वा॒र्या॑य च । \newline
10. च॒ नमो॒ नम॑श्च च॒ नमः॑ । \newline
11. नमः॑ प्र॒तर॑णाय प्र॒तर॑णाय॒ नमो॒ नमः॑ प्र॒तर॑णाय । \newline
12. प्र॒तर॑णाय च च प्र॒तर॑णाय प्र॒तर॑णाय च । \newline
13. प्र॒तर॑णा॒येति॑ प्र - तर॑णाय । \newline
14. चो॒त्तर॑णा यो॒त्तर॑णाय च चो॒त्तर॑णाय । \newline
15. उ॒त्तर॑णाय च चो॒त्तर॑णा यो॒त्तर॑णाय च । \newline
16. उ॒त्तर॑णा॒येत्यु॑त् - तर॑णाय । \newline
17. च॒ नमो॒ नम॑श्च च॒ नमः॑ । \newline
18. नम॑ आता॒र्या॑या ता॒र्या॑य॒ नमो॒ नम॑ आता॒र्या॑य । \newline
19. आ॒ता॒र्या॑य च चाता॒र्या॑या ता॒र्या॑य च । \newline
20. आ॒ता॒र्या॑येत्या᳚ - ता॒र्या॑य । \newline
21. चा॒ला॒द्या॑या ला॒द्या॑य च चाला॒द्या॑य । \newline
22. आ॒ला॒द्या॑य च चाला॒द्या॑या ला॒द्या॑य च । \newline
23. आ॒ला॒द्या॑येत्या᳚ - ला॒द्या॑य । \newline
24. च॒ नमो॒ नम॑श्च च॒ नमः॑ । \newline
25. नमः॒ शष्प्या॑य॒ शष्प्या॑य॒ नमो॒ नमः॒ शष्प्या॑य । \newline
26. शष्प्या॑य च च॒ शष्प्या॑य॒ शष्प्या॑य च । \newline
27. च॒ फेन्या॑य॒ फेन्या॑य च च॒ फेन्या॑य । \newline
28. फेन्या॑य च च॒ फेन्या॑य॒ फेन्या॑य च । \newline
29. च॒ नमो॒ नम॑श्च च॒ नमः॑ । \newline
30. नमः॑ सिक॒त्या॑य सिक॒त्या॑य॒ नमो॒ नमः॑ सिक॒त्या॑य । \newline
31. सि॒क॒त्या॑य च च सिक॒त्या॑य सिक॒त्या॑य च । \newline
32. च॒ प्र॒वा॒ह्या॑य प्रवा॒ह्या॑य च च प्रवा॒ह्या॑य । \newline
33. प्र॒वा॒ह्या॑य च च प्रवा॒ह्या॑य प्रवा॒ह्या॑य च । \newline
34. प्र॒वा॒ह्या॑येति॑ प्र - वा॒ह्या॑य । \newline
35. चेति॑ च । \newline

\textbf{Ghana Paata } \newline

1. नम॒ स्तीर्थ्या॑य॒ तीर्थ्या॑य॒ नमो॒ नम॒ स्तीर्थ्या॑य च च॒ तीर्थ्या॑य॒ नमो॒ नम॒ स्तीर्थ्या॑य च । \newline
2. तीर्थ्या॑य च च॒ तीर्थ्या॑य॒ तीर्थ्या॑य च॒ कूल्या॑य॒ कूल्या॑य च॒ तीर्थ्या॑य॒ तीर्थ्या॑य च॒ कूल्या॑य । \newline
3. च॒ कूल्या॑य॒ कूल्या॑य च च॒ कूल्या॑य च च॒ कूल्या॑य च च॒ कूल्या॑य च । \newline
4. कूल्या॑य च च॒ कूल्या॑य॒ कूल्या॑य च॒ नमो॒ नम॑श्च॒ कूल्या॑य॒ कूल्या॑य च॒ नमः॑ । \newline
5. च॒ नमो॒ नम॑श्च च॒ नमः॑ पा॒र्या॑य पा॒र्या॑य॒ नम॑श्च च॒ नमः॑ पा॒र्या॑य । \newline
6. नमः॑ पा॒र्या॑य पा॒र्या॑य॒ नमो॒ नमः॑ पा॒र्या॑य च च पा॒र्या॑य॒ नमो॒ नमः॑ पा॒र्या॑य च । \newline
7. पा॒र्या॑य च च पा॒र्या॑य पा॒र्या॑य चावा॒र्या॑या वा॒र्या॑य च पा॒र्या॑य पा॒र्या॑य चावा॒र्या॑य । \newline
8. चा॒वा॒र्या॑या वा॒र्या॑य च चावा॒र्या॑य च चावा॒र्या॑य च चावा॒र्या॑य च । \newline
9. अ॒वा॒र्या॑य च चावा॒र्या॑या वा॒र्या॑य च॒ नमो॒ नम॑ श्चावा॒र्या॑या वा॒र्या॑य च॒ नमः॑ । \newline
10. च॒ नमो॒ नम॑श्च च॒ नमः॑ प्र॒तर॑णाय प्र॒तर॑णाय॒ नम॑श्च च॒ नमः॑ प्र॒तर॑णाय । \newline
11. नमः॑ प्र॒तर॑णाय प्र॒तर॑णाय॒ नमो॒ नमः॑ प्र॒तर॑णाय च च प्र॒तर॑णाय॒ नमो॒ नमः॑ प्र॒तर॑णाय च । \newline
12. प्र॒तर॑णाय च च प्र॒तर॑णाय प्र॒तर॑णाय चो॒त्तर॑णा यो॒त्तर॑णाय च प्र॒तर॑णाय प्र॒तर॑णाय चो॒त्तर॑णाय । \newline
13. प्र॒तर॑णा॒येति॑ प्र - तर॑णाय । \newline
14. चो॒त्तर॑णा यो॒त्तर॑णाय च चो॒त्तर॑णाय च चो॒त्तर॑णाय च चो॒त्तर॑णाय च । \newline
15. उ॒त्तर॑णाय च चो॒त्तर॑णा यो॒त्तर॑णाय च॒ नमो॒ नम॑ श्चो॒त्तर॑णा यो॒त्तर॑णाय च॒ नमः॑ । \newline
16. उ॒त्तर॑णा॒येत्यु॑त् - तर॑णाय । \newline
17. च॒ नमो॒ नम॑श्च च॒ नम॑ आता॒र्या॑या ता॒र्या॑य॒ नम॑श्च च॒ नम॑ आता॒र्या॑य । \newline
18. नम॑ आता॒र्या॑या ता॒र्या॑य॒ नमो॒ नम॑ आता॒र्या॑य च चाता॒र्या॑य॒ नमो॒ नम॑ आता॒र्या॑य च । \newline
19. आ॒ता॒र्या॑य च चाता॒र्या॑या ता॒र्या॑य चाला॒द्या॑या ला॒द्या॑य चाता॒र्या॑या ता॒र्या॑य चाला॒द्या॑य । \newline
20. आ॒ता॒र्या॑येत्या᳚ - ता॒र्या॑य । \newline
21. चा॒ला॒द्या॑या ला॒द्या॑य च चाला॒द्या॑य च चाला॒द्या॑य च चाला॒द्या॑य च । \newline
22. आ॒ला॒द्या॑य च चाला॒द्या॑या ला॒द्या॑य च॒ नमो॒ नम॑ श्चाला॒द्या॑या ला॒द्या॑य च॒ नमः॑ । \newline
23. आ॒ला॒द्या॑येत्या᳚ - ला॒द्या॑य । \newline
24. च॒ नमो॒ नम॑श्च च॒ नमः॒ शष्प्या॑य॒ शष्प्या॑य॒ नम॑श्च च॒ नमः॒ शष्प्या॑य । \newline
25. नमः॒ शष्प्या॑य॒ शष्प्या॑य॒ नमो॒ नमः॒ शष्प्या॑य च च॒ शष्प्या॑य॒ नमो॒ नमः॒ शष्प्या॑य च । \newline
26. शष्प्या॑य च च॒ शष्प्या॑य॒ शष्प्या॑य च॒ फेन्या॑य॒ फेन्या॑य च॒ शष्प्या॑य॒ शष्प्या॑य च॒ फेन्या॑य । \newline
27. च॒ फेन्या॑य॒ फेन्या॑य च च॒ फेन्या॑य च च॒ फेन्या॑य च च॒ फेन्या॑य च । \newline
28. फेन्या॑य च च॒ फेन्या॑य॒ फेन्या॑य च॒ नमो॒ नम॑श्च॒ फेन्या॑य॒ फेन्या॑य च॒ नमः॑ । \newline
29. च॒ नमो॒ नम॑श्च च॒ नमः॑ सिक॒त्या॑य सिक॒त्या॑य॒ नम॑श्च च॒ नमः॑ सिक॒त्या॑य । \newline
30. नमः॑ सिक॒त्या॑य सिक॒त्या॑य॒ नमो॒ नमः॑ सिक॒त्या॑य च च सिक॒त्या॑य॒ नमो॒ नमः॑ सिक॒त्या॑य च । \newline
31. सि॒क॒त्या॑य च च सिक॒त्या॑य सिक॒त्या॑य च प्रवा॒ह्या॑य प्रवा॒ह्या॑य च सिक॒त्या॑य सिक॒त्या॑य च प्रवा॒ह्या॑य । \newline
32. च॒ प्र॒वा॒ह्या॑य प्रवा॒ह्या॑य च च प्रवा॒ह्या॑य च च प्रवा॒ह्या॑य च च प्रवा॒ह्या॑य च । \newline
33. प्र॒वा॒ह्या॑य च च प्रवा॒ह्या॑य प्रवा॒ह्या॑य च । \newline
34. प्र॒वा॒ह्या॑येति॑ प्र - वा॒ह्या॑य । \newline
35. चेति॑ च । \newline
\pagebreak
\markright{ TS 4.5.9.1  \hfill https://www.vedavms.in \hfill}

\section{ TS 4.5.9.1 }

\textbf{TS 4.5.9.1 } \newline
\textbf{Samhita Paata} \newline

नम॑ इरि॒ण्या॑य च प्रप॒थ्या॑य च॒ नमः॑ किꣳशि॒लाय॑ च॒ क्षय॑णाय च॒ नमः॑ कप॒र्दिने॑ च पुल॒स्तये॑ च॒ नमो॒ गोष्ठ्या॑य च॒ गृह्या॑य च॒ नम॒स्तल्प्या॑य च॒ गेह्या॑य च॒ नमः॑ का॒ट्या॑य च गह्वरे॒ष्ठाय॑ च॒ नमो᳚ ह्रद॒य्या॑य च निवे॒ष्प्या॑य च॒ नमः॑ पाꣳस॒व्या॑य च रज॒स्या॑य च॒ नमः॒ शुष्क्या॑य च हरि॒त्या॑य च॒ नमो॒ लोप्या॑य चोल॒प्या॑य च॒- [  ] \newline

\textbf{Pada Paata} \newline

नमः॑ । इ॒रि॒ण्या॑य । च॒ । प्र॒प॒थ्या॑येति॑ प्र - प॒थ्या॑य । च॒ । नमः॑ । किꣳ॒॒शि॒लाय॑ । च॒ । क्षय॑णाय । च॒ । नमः॑ । क॒प॒र्दिने᳚ । च॒ । पु॒ल॒स्तये᳚ । च॒ । नमः॑ । गोष्ठ्या॒येति॒ गो - स्थ्या॒य॒ । च॒ । गृह्या॑य । च॒ । नमः॑ । तल्प्या॑य । च॒ । गेह्या॑य । च॒ । नमः॑ । का॒ट्या॑य । च॒ । ग॒ह्व॒रे॒ष्ठायेति॑ गह्वरे - स्थाय॑ । च॒ । नमः॑ । ह्र॒द॒य्या॑य । च॒ । नि॒वे॒ष्प्या॑येति॑ नि - वे॒ष्प्या॑य । च॒ । नमः॑ । पाꣳ॒॒स॒व्या॑य । च॒ । र॒ज॒स्या॑य । च॒ । नमः॑ । शुष्क्या॑य । च॒ । ह॒रि॒त्या॑य । च॒ । नमः॑ । लोप्या॑य । च॒ । उ॒ल॒प्या॑य । च॒ ।  \newline


\textbf{Krama Paata} \newline

नम॑ इरि॒ण्या॑य । इ॒रि॒ण्या॑य च । च॒ प्र॒प॒थ्या॑य । प्र॒प॒थ्या॑य च । प्र॒प॒थ्या॑येति॑ प्र - प॒थ्या॑य । च॒ नमः॑ । नमः॑ किꣳशि॒लाय॑ । किꣳ॒॒शि॒लाय॑ च । च॒ क्षय॑णाय । क्षय॑णाय च । च॒ नमः॑ । नमः॑ कप॒र्दिने᳚ । क॒प॒र्दिने॑ च । च॒ पु॒ल॒स्तये᳚ । पु॒ल॒स्तये॑ च । च॒ नमः॑ । नमो॒ गोष्ठ्या॑य । गोष्ठ्या॑य च । गोष्ठ्या॒येति॒ गो - स्थ्या॒य॒ । च॒ गृह्या॑य । गृह्या॑य च । च॒ नमः॑ । नम॒स्तल्प्या॑य । तल्प्या॑य च । च॒ गेह्या॑य । गेह्या॑य च । च॒ नमः॑ । नमः॑ का॒ट्या॑य । का॒ट्या॑य च । च॒ ग॒ह्व॒रे॒ष्ठाय॑ । ग॒ह्व॒रे॒ष्ठाय॑ च । ग॒ह्व॒रे॒ष्ठायेति॑ गह्वरे - स्थाय॑ । च॒ नमः॑ । नमो᳚ ह्रद॒य्या॑य । ह्र॒द॒य्या॑य च । च॒ नि॒वे॒ष्प्या॑य । नि॒वे॒ष्प्या॑य च । नि॒वे॒ष्प्या॑येति॑ नि - वे॒ष्प्या॑य । च॒ नमः॑ । नमः॑ पाꣳस॒व्या॑य । पाꣳ॒॒स॒व्या॑य च । च॒ र॒ज॒स्या॑य । र॒ज॒स्या॑य च । च॒ नमः॑ । नमः॒ शुष्क्या॑य । शुष्क्या॑य च । च॒ ह॒रि॒त्या॑य । ह॒रि॒त्या॑य च । च॒ नमः॑ । नमो॒ लोप्या॑य । लोप्या॑य च । चो॒ल॒प्या॑य । उ॒ल॒प्या॑य च ( ) । च॒ नमः॑ \newline

\textbf{Jatai Paata} \newline

1. नम॑ इरि॒ण्या॑ येरि॒ण्या॑य॒ नमो॒ नम॑ इरि॒ण्या॑य । \newline
2. इ॒रि॒ण्या॑य च चे रि॒ण्या॑ येरि॒ण्या॑य च । \newline
3. च॒ प्र॒प॒थ्या॑य प्रप॒थ्या॑य च च प्रप॒थ्या॑य । \newline
4. प्र॒प॒थ्या॑य च च प्रप॒थ्या॑य प्रप॒थ्या॑य च । \newline
5. प्र॒प॒थ्या॑येति॑ प्र - प॒थ्या॑य । \newline
6. च॒ नमो॒ नम॑श्च च॒ नमः॑ । \newline
7. नमः॑ किꣳशि॒लाय॑ किꣳशि॒लाय॒ नमो॒ नमः॑ किꣳशि॒लाय॑ । \newline
8. कि॒(ग्म्॒)शि॒लाय॑ च च किꣳशि॒लाय॑ किꣳशि॒लाय॑ च । \newline
9. च॒ क्षय॑णाय॒ क्षय॑णाय च च॒ क्षय॑णाय । \newline
10. क्षय॑णाय च च॒ क्षय॑णाय॒ क्षय॑णाय च । \newline
11. च॒ नमो॒ नम॑श्च च॒ नमः॑ । \newline
12. नमः॑ कप॒र्दिने॑ कप॒र्दिने॒ नमो॒ नमः॑ कप॒र्दिने᳚ । \newline
13. क॒प॒र्दिने॑ च च कप॒र्दिने॑ कप॒र्दिने॑ च । \newline
14. च॒ पु॒ल॒स्तये॑ पुल॒स्तये॑ च च पुल॒स्तये᳚ । \newline
15. पु॒ल॒स्तये॑ च च पुल॒स्तये॑ पुल॒स्तये॑ च । \newline
16. च॒ नमो॒ नम॑श्च च॒ नमः॑ । \newline
17. नमो॒ गोष्ठ्या॑य॒ गोष्ठ्या॑य॒ नमो॒ नमो॒ गोष्ठ्या॑य । \newline
18. गोष्ठ्या॑य च च॒ गोष्ठ्या॑य॒ गोष्ठ्या॑य च । \newline
19. गोष्ठ्या॒येति॒ गो - स्थ्या॒य॒ । \newline
20. च॒ गृह्या॑य॒ गृह्या॑य च च॒ गृह्या॑य । \newline
21. गृह्या॑य च च॒ गृह्या॑य॒ गृह्या॑य च । \newline
22. च॒ नमो॒ नम॑श्च च॒ नमः॑ । \newline
23. नम॒ स्तल्प्या॑य॒ तल्प्या॑य॒ नमो॒ नम॒ स्तल्प्या॑य । \newline
24. तल्प्या॑य च च॒ तल्प्या॑य॒ तल्प्या॑य च । \newline
25. च॒ गेह्या॑य॒ गेह्या॑य च च॒ गेह्या॑य । \newline
26. गेह्या॑य च च॒ गेह्या॑य॒ गेह्या॑य च । \newline
27. च॒ नमो॒ नम॑श्च च॒ नमः॑ । \newline
28. नमः॑ का॒ट्या॑य का॒ट्या॑य॒ नमो॒ नमः॑ का॒ट्या॑य । \newline
29. का॒ट्या॑य च च का॒ट्या॑य का॒ट्या॑य च । \newline
30. च॒ ग॒ह्व॒रे॒ष्ठाय॑ गह्वरे॒ष्ठाय॑ च च गह्वरे॒ष्ठाय॑ । \newline
31. ग॒ह्व॒रे॒ष्ठाय॑ च च गह्वरे॒ष्ठाय॑ गह्वरे॒ष्ठाय॑ च । \newline
32. ग॒ह्व॒रे॒ष्ठायेति॑ गह्वरे - स्थाय॑ । \newline
33. च॒ नमो॒ नम॑श्च च॒ नमः॑ । \newline
34. नमो᳚ ह्रद॒य्या॑य ह्रद॒य्या॑य॒ नमो॒ नमो᳚ ह्रद॒य्या॑य । \newline
35. ह्र॒द॒य्या॑य च च ह्रद॒य्या॑य ह्रद॒य्या॑य च । \newline
36. च॒ नि॒वे॒ष्प्या॑य निवे॒ष्प्या॑य च च निवे॒ष्प्या॑य । \newline
37. नि॒वे॒ष्प्या॑य च च निवे॒ष्प्या॑य निवे॒ष्प्या॑य च । \newline
38. नि॒वे॒ष्प्या॑येति॑ नि - वे॒ष्प्या॑य । \newline
39. च॒ नमो॒ नम॑श्च च॒ नमः॑ । \newline
40. नमः॑ पाꣳस॒व्या॑य पाꣳस॒व्या॑य॒ नमो॒ नमः॑ पाꣳस॒व्या॑य । \newline
41. पा॒(ग्म्॒)स॒व्या॑य च च पाꣳस॒व्या॑य पाꣳस॒व्या॑य च । \newline
42. च॒ र॒ज॒स्या॑य रज॒स्या॑य च च रज॒स्या॑य । \newline
43. र॒ज॒स्या॑य च च रज॒स्या॑य रज॒स्या॑य च । \newline
44. च॒ नमो॒ नम॑श्च च॒ नमः॑ । \newline
45. नमः॒ शुष्क्या॑य॒ शुष्क्या॑य॒ नमो॒ नमः॒ शुष्क्या॑य । \newline
46. शुष्क्या॑य च च॒ शुष्क्या॑य॒ शुष्क्या॑य च । \newline
47. च॒ ह॒रि॒त्या॑य हरि॒त्या॑य च च हरि॒त्या॑य । \newline
48. ह॒रि॒त्या॑य च च हरि॒त्या॑य हरि॒त्या॑य च । \newline
49. च॒ नमो॒ नम॑श्च च॒ नमः॑ । \newline
50. नमो॒ लोप्या॑य॒ लोप्या॑य॒ नमो॒ नमो॒ लोप्या॑य । \newline
51. लोप्या॑य च च॒ लोप्या॑य॒ लोप्या॑य च । \newline
52. चो॒ल॒प्या॑ योल॒प्या॑य च चोल॒प्या॑य । \newline
53. उ॒ल॒प्या॑य च चोल॒प्या॑ योल॒प्या॑य च । \newline
54. च॒ नमो॒ नम॑श्च च॒ नमः॑ । \newline

\textbf{Ghana Paata } \newline

1. नम॑ इरि॒ण्या॑ येरि॒ण्या॑य॒ नमो॒ नम॑ इरि॒ण्या॑य च चेरि॒ण्या॑य॒ नमो॒ नम॑ इरि॒ण्या॑य च । \newline
2. इ॒रि॒ण्या॑य च चेरि॒ण्या॑ये रि॒ण्या॑य च प्रप॒थ्या॑य प्रप॒थ्या॑य चेरि॒ण्या॑ येरि॒ण्या॑य च प्रप॒थ्या॑य । \newline
3. च॒ प्र॒प॒थ्या॑य प्रप॒थ्या॑य च च प्रप॒थ्या॑य च च प्रप॒थ्या॑य च च प्रप॒थ्या॑य च । \newline
4. प्र॒प॒थ्या॑य च च प्रप॒थ्या॑य प्रप॒थ्या॑य च॒ नमो॒ नम॑श्च प्रप॒थ्या॑य प्रप॒थ्या॑य च॒ नमः॑ । \newline
5. प्र॒प॒थ्या॑येति॑ प्र - प॒थ्या॑य । \newline
6. च॒ नमो॒ नम॑श्च च॒ नमः॑ किꣳशि॒लाय॑ किꣳशि॒लाय॒ नम॑श्च च॒ नमः॑ किꣳशि॒लाय॑ । \newline
7. नमः॑ किꣳशि॒लाय॑ किꣳशि॒लाय॒ नमो॒ नमः॑ किꣳशि॒लाय॑ च च किꣳशि॒लाय॒ नमो॒ नमः॑ किꣳशि॒लाय॑ च । \newline
8. किꣳ॒॒शि॒लाय॑ च च किꣳशि॒लाय॑ किꣳशि॒लाय॑ च॒ क्षय॑णाय॒ क्षय॑णाय च किꣳशि॒लाय॑ किꣳशि॒लाय॑ च॒ क्षय॑णाय । \newline
9. च॒ क्षय॑णाय॒ क्षय॑णाय च च॒ क्षय॑णाय च च॒ क्षय॑णाय च च॒ क्षय॑णाय च । \newline
10. क्षय॑णाय च च॒ क्षय॑णाय॒ क्षय॑णाय च॒ नमो॒ नम॑श्च॒ क्षय॑णाय॒ क्षय॑णाय च॒ नमः॑ । \newline
11. च॒ नमो॒ नम॑श्च च॒ नमः॑ कप॒र्दिने॑ कप॒र्दिने॒ नम॑श्च च॒ नमः॑ कप॒र्दिने᳚ । \newline
12. नमः॑ कप॒र्दिने॑ कप॒र्दिने॒ नमो॒ नमः॑ कप॒र्दिने॑ च च कप॒र्दिने॒ नमो॒ नमः॑ कप॒र्दिने॑ च । \newline
13. क॒प॒र्दिने॑ च च कप॒र्दिने॑ कप॒र्दिने॑ च पुल॒स्तये॑ पुल॒स्तये॑ च कप॒र्दिने॑ कप॒र्दिने॑ च पुल॒स्तये᳚ । \newline
14. च॒ पु॒ल॒स्तये॑ पुल॒स्तये॑ च च पुल॒स्तये॑ च च पुल॒स्तये॑ च च पुल॒स्तये॑ च । \newline
15. पु॒ल॒स्तये॑ च च पुल॒स्तये॑ पुल॒स्तये॑ च॒ नमो॒ नम॑श्च पुल॒स्तये॑ पुल॒स्तये॑ च॒ नमः॑ । \newline
16. च॒ नमो॒ नम॑श्च च॒ नमो॒ गोष्ठ्या॑य॒ गोष्ठ्या॑य॒ नम॑श्च च॒ नमो॒ गोष्ठ्या॑य । \newline
17. नमो॒ गोष्ठ्या॑य॒ गोष्ठ्या॑य॒ नमो॒ नमो॒ गोष्ठ्या॑य च च॒ गोष्ठ्या॑य॒ नमो॒ नमो॒ गोष्ठ्या॑य च । \newline
18. गोष्ठ्या॑य च च॒ गोष्ठ्या॑य॒ गोष्ठ्या॑य च॒ गृह्या॑य॒ गृह्या॑य च॒ गोष्ठ्या॑य॒ गोष्ठ्या॑य च॒ गृह्या॑य । \newline
19. गोष्ठ्या॒येति॒ गो - स्थ्या॒य॒ । \newline
20. च॒ गृह्या॑य॒ गृह्या॑य च च॒ गृह्या॑य च च॒ गृह्या॑य च च॒ गृह्या॑य च । \newline
21. गृह्या॑य च च॒ गृह्या॑य॒ गृह्या॑य च॒ नमो॒ नम॑श्च॒ गृह्या॑य॒ गृह्या॑य च॒ नमः॑ । \newline
22. च॒ नमो॒ नम॑श्च च॒ नम॒ स्तल्प्या॑य॒ तल्प्या॑य॒ नम॑श्च च॒ नम॒ स्तल्प्या॑य । \newline
23. नम॒ स्तल्प्या॑य॒ तल्प्या॑य॒ नमो॒ नम॒ स्तल्प्या॑य च च॒ तल्प्या॑य॒ नमो॒ नम॒ स्तल्प्या॑य च । \newline
24. तल्प्या॑य च च॒ तल्प्या॑य॒ तल्प्या॑य च॒ गेह्या॑य॒ गेह्या॑य च॒ तल्प्या॑य॒ तल्प्या॑य च॒ गेह्या॑य । \newline
25. च॒ गेह्या॑य॒ गेह्या॑य च च॒ गेह्या॑य च च॒ गेह्या॑य च च॒ गेह्या॑य च । \newline
26. गेह्या॑य च च॒ गेह्या॑य॒ गेह्या॑य च॒ नमो॒ नम॑श्च॒ गेह्या॑य॒ गेह्या॑य च॒ नमः॑ । \newline
27. च॒ नमो॒ नम॑श्च च॒ नमः॑ का॒ट्या॑य का॒ट्या॑य॒ नम॑श्च च॒ नमः॑ का॒ट्या॑य । \newline
28. नमः॑ का॒ट्या॑य का॒ट्या॑य॒ नमो॒ नमः॑ का॒ट्या॑य च च का॒ट्या॑य॒ नमो॒ नमः॑ का॒ट्या॑य च । \newline
29. का॒ट्या॑य च च का॒ट्या॑य का॒ट्या॑य च गह्वरे॒ष्ठाय॑ गह्वरे॒ष्ठाय॑ च का॒ट्या॑य का॒ट्या॑य च गह्वरे॒ष्ठाय॑ । \newline
30. च॒ ग॒ह्व॒रे॒ष्ठाय॑ गह्वरे॒ष्ठाय॑ च च गह्वरे॒ष्ठाय॑ च च गह्वरे॒ष्ठाय॑ च च गह्वरे॒ष्ठाय॑ च । \newline
31. ग॒ह्व॒रे॒ष्ठाय॑ च च गह्वरे॒ष्ठाय॑ गह्वरे॒ष्ठाय॑ च॒ नमो॒ नम॑श्च गह्वरे॒ष्ठाय॑ गह्वरे॒ष्ठाय॑ च॒ नमः॑ । \newline
32. ग॒ह्व॒रे॒ष्ठायेति॑ गह्वरे - स्थाय॑ । \newline
33. च॒ नमो॒ नम॑श्च च॒ नमो᳚ ह्रद॒य्या॑य ह्रद॒य्या॑य॒ नम॑श्च च॒ नमो᳚ ह्रद॒य्या॑य । \newline
34. नमो᳚ ह्रद॒य्या॑य ह्रद॒य्या॑य॒ नमो॒ नमो᳚ ह्रद॒य्या॑य च च ह्रद॒य्या॑य॒ नमो॒ नमो᳚ ह्रद॒य्या॑य च । \newline
35. ह्र॒द॒य्या॑य च च ह्रद॒य्या॑य ह्रद॒य्या॑य च निवे॒ष्प्या॑य निवे॒ष्प्या॑य च ह्रद॒य्या॑य ह्रद॒य्या॑य च निवे॒ष्प्या॑य । \newline
36. च॒ नि॒वे॒ष्प्या॑य निवे॒ष्प्या॑य च च निवे॒ष्प्या॑य च च निवे॒ष्प्या॑य च च निवे॒ष्प्या॑य च । \newline
37. नि॒वे॒ष्प्या॑य च च निवे॒ष्प्या॑य निवे॒ष्प्या॑य च॒ नमो॒ नम॑श्च निवे॒ष्प्या॑य निवे॒ष्प्या॑य च॒ नमः॑ । \newline
38. नि॒वे॒ष्प्या॑येति॑ नि - वे॒ष्प्या॑य । \newline
39. च॒ नमो॒ नम॑श्च च॒ नमः॑ पाꣳस॒व्या॑य पाꣳस॒व्या॑य॒ नम॑श्च च॒ नमः॑ पाꣳस॒व्या॑य । \newline
40. नमः॑ पाꣳस॒व्या॑य पाꣳस॒व्या॑य॒ नमो॒ नमः॑ पाꣳस॒व्या॑य च च पाꣳस॒व्या॑य॒ नमो॒ नमः॑ पाꣳस॒व्या॑य च । \newline
41. पाꣳ॒॒स॒व्या॑य च च पाꣳस॒व्या॑य पाꣳस॒व्या॑य च रज॒स्या॑य रज॒स्या॑य च पाꣳस॒व्या॑य पाꣳस॒व्या॑य च रज॒स्या॑य । \newline
42. च॒ र॒ज॒स्या॑य रज॒स्या॑य च च रज॒स्या॑य च च रज॒स्या॑य च च रज॒स्या॑य च । \newline
43. र॒ज॒स्या॑य च च रज॒स्या॑य रज॒स्या॑य च॒ नमो॒ नम॑श्च रज॒स्या॑य रज॒स्या॑य च॒ नमः॑ । \newline
44. च॒ नमो॒ नम॑श्च च॒ नमः॒ शुष्क्या॑य॒ शुष्क्या॑य॒ नम॑श्च च॒ नमः॒ शुष्क्या॑य । \newline
45. नमः॒ शुष्क्या॑य॒ शुष्क्या॑य॒ नमो॒ नमः॒ शुष्क्या॑य च च॒ शुष्क्या॑य॒ नमो॒ नमः॒ शुष्क्या॑य च । \newline
46. शुष्क्या॑य च च॒ शुष्क्या॑य॒ शुष्क्या॑य च हरि॒त्या॑य हरि॒त्या॑य च॒ शुष्क्या॑य॒ शुष्क्या॑य च हरि॒त्या॑य । \newline
47. च॒ ह॒रि॒त्या॑य हरि॒त्या॑य च च हरि॒त्या॑य च च हरि॒त्या॑य च च हरि॒त्या॑य च । \newline
48. ह॒रि॒त्या॑य च च हरि॒त्या॑य हरि॒त्या॑य च॒ नमो॒ नम॑श्च हरि॒त्या॑य हरि॒त्या॑य च॒ नमः॑ । \newline
49. च॒ नमो॒ नम॑श्च च॒ नमो॒ लोप्या॑य॒ लोप्या॑य॒ नम॑श्च च॒ नमो॒ लोप्या॑य । \newline
50. नमो॒ लोप्या॑य॒ लोप्या॑य॒ नमो॒ नमो॒ लोप्या॑य च च॒ लोप्या॑य॒ नमो॒ नमो॒ लोप्या॑य च । \newline
51. लोप्या॑य च च॒ लोप्या॑य॒ लोप्या॑य चोल॒प्या॑ योल॒प्या॑य च॒ लोप्या॑य॒ लोप्या॑य चोल॒प्या॑य । \newline
52. चो॒ल॒प्या॑ योल॒प्या॑य च चोल॒प्या॑य च चोल॒प्या॑य च चोल॒प्या॑य च । \newline
53. उ॒ल॒प्या॑य च चोल॒प्या॑ योल॒प्या॑य च॒ नमो॒ नम॑ श्चोल॒प्या॑ योल॒प्या॑य च॒ नमः॑ । \newline
54. च॒ नमो॒ नम॑श्च च॒ नम॑ ऊ॒र्व्या॑ यो॒र्व्या॑य॒ नम॑श्च च॒ नम॑ ऊ॒र्व्या॑य । \newline
\pagebreak
\markright{ TS 4.5.9.2  \hfill https://www.vedavms.in \hfill}

\section{ TS 4.5.9.2 }

\textbf{TS 4.5.9.2 } \newline
\textbf{Samhita Paata} \newline

नम॑ ऊ॒र्व्या॑य च सू॒र्म्या॑य च॒ नमः॑ प॒र्ण्या॑य च पर्णश॒द्या॑य च॒ नमो॑ऽपगु॒रमा॑णाय चाभिघ्न॒ते च॒ नम॑ आक्खिद॒ते च॑ प्रक्खिद॒ते च॒ नमो॑ वः किरि॒केभ्यो॑ दे॒वानाꣳ॒॒ हृद॑येभ्यो॒ नमो॑ विक्षीण॒केभ्यो॒ नमो॑ विचिन्व॒त्केभ्यो॒ नम॑ आनिर्. ह॒तेभ्यो॒ नम॑ आमीव॒त्केभ्यः॑ ॥ \newline

\textbf{Pada Paata} \newline

नमः॑ । ऊ॒र्व्या॑य । च॒ । सू॒र्म्या॑य । च॒ । नमः॑ । प॒र्ण्या॑य । च॒ । प॒र्ण॒श॒द्या॑येति॑ पर्ण - श॒द्या॑य। च॒ । नमः॑ । अ॒प॒गु॒रमा॑णा॒येत्य॑प - गु॒रमा॑णाय । च॒ । अ॒भि॒घ्न॒त इत्य॑भि - घ्न॒ते । च॒ । नमः॑ । आ॒क्खि॒द॒त इत्या᳚ - खि॒द॒ते । च॒ । प्र॒क्खि॒द॒त इति॑ प्र-खि॒द॒ते । च॒ । नमः॑ । वः॒ । कि॒रि॒केभ्यः॑ । दे॒वाना᳚म् । हृद॑येभ्यः । नमः॑ । वि॒क्षी॒ण॒केभ्य॒ इति॑ वि - क्षी॒ण॒केभ्यः॑ । नमः॑ । वि॒चि॒न्व॒त्केभ्य॒ इति॑ वि - चि॒न्व॒त्केभ्यः॑ । नमः॑ । आ॒नि॒र्॒.ह॒तेभ्य॒ इत्या॑निः-ह॒तेभ्यः॑ । नमः॑ । आ॒मी॒व॒त्केभ्य॒ इत्या᳚ - मी॒व॒त्केभ्यः॑ ॥  \newline


\textbf{Krama Paata} \newline

नम॑ ऊ॒र्व्या॑य । ऊ॒र्व्या॑य च । च॒ सू॒र्म्या॑य । सू॒र्म्या॑य च । च॒ नमः॑ । नमः॑ प॒र्ण्या॑य । प॒र्ण्या॑य च । च॒ प॒र्ण॒श॒द्या॑य । प॒र्ण॒श॒द्या॑य च । प॒र्ण॒श॒द्या॑येति॑ पर्ण - श॒द्या॑य । च॒ नमः॑ । नमो॑ऽपगु॒रमा॑णाय । अ॒प॒गु॒रमा॑णाय च । अ॒प॒गु॒रमा॑णा॒येत्य॑प - गु॒रमा॑णाय । चा॒भि॒घ्न॒ते । अ॒भि॒घ्न॒ते च॑ । अ॒भि॒घ्न॒त इत्य॑भि - घ्न॒ते । च॒ नमः॑ । नम॑ आक्खिद॒ते । आ॒क्खि॒द॒ते च॑ । आ॒क्खि॒द॒त इत्या᳚ - खि॒द॒ते । च॒ प्र॒क्खि॒द॒ते । प्र॒क्खि॒द॒ते च॑ । प्र॒क्खि॒द॒त इति॑ प्र - खि॒द॒ते । च॒ नमः॑ । नमो॑ वः । वः॒ कि॒रि॒केभ्यः॑ । कि॒रि॒केभ्यो॑ दे॒वाना᳚म् । दे॒वानाꣳ॒॒ हृद॑येभ्यः । हृद॑येभ्यो॒ नमः॑ । नमो॑ विक्षीण॒केभ्यः॑ । वि॒क्षी॒ण॒केभ्यो॒ नमः॑ । वि॒क्षी॒ण॒केभ्य॒ इति॑ वि - क्षी॒ण॒केभ्यः॑ । नमो॑ विचिन्व॒त्केभ्यः॑ । वि॒चि॒न्व॒त्केभ्यो॒ नमः॑ । वि॒चि॒न्व॒त्केभ्य॒ इति॑ वि - चि॒न्व॒त्केभ्यः॑ । नम॑ आनिर्.ह॒तेभ्यः॑ । आ॒नि॒र्॒.ह॒तेभ्यो॒ नमः॑ । आ॒नि॒र्॒.ह॒तेभ्य॒ इत्या॑निः - ह॒तेभ्यः॑ । नम॑ आमीव॒त्केभ्यः॑ । 
आ॒मी॒व॒त्केभ्य॒ इत्या᳚ - मी॒व॒त्केभ्यः॑ । \newline

\textbf{Jatai Paata} \newline

1. नम॑ ऊ॒र्व्या॑ यो॒र्व्या॑य॒ नमो॒ नम॑ ऊ॒र्व्या॑य । \newline
2. ऊ॒र्व्या॑य च चो॒र्व्या॑ यो॒र्व्या॑य च । \newline
3. च॒ सू॒र्म्या॑य सू॒र्म्या॑य च च सू॒र्म्या॑य । \newline
4. सू॒र्म्या॑य च च सू॒र्म्या॑य सू॒र्म्या॑य च । \newline
5. च॒ नमो॒ नम॑श्च च॒ नमः॑ । \newline
6. नमः॑ प॒र्ण्या॑य प॒र्ण्या॑य॒ नमो॒ नमः॑ प॒र्ण्या॑य । \newline
7. प॒र्ण्या॑य च च प॒र्ण्या॑य प॒र्ण्या॑य च । \newline
8. च॒ प॒र्ण॒श॒द्या॑य पर्णश॒द्या॑य च च पर्णश॒द्या॑य । \newline
9. प॒र्ण॒श॒द्या॑य च च पर्णश॒द्या॑य पर्णश॒द्या॑य च । \newline
10. प॒र्ण॒श॒द्या॑येति॑ पर्ण - श॒द्या॑य । \newline
11. च॒ नमो॒ नम॑श्च च॒ नमः॑ । \newline
12. नमो॑ ऽपगु॒रमा॑णाया पगु॒रमा॑णाय॒ नमो॒ नमो॑ ऽपगु॒रमा॑णाय । \newline
13. अ॒प॒गु॒रमा॑णाय च चापगु॒रमा॑णाया पगु॒रमा॑णाय च । \newline
14. अ॒प॒गु॒रमा॑णा॒येत्य॑प - गु॒रमा॑णाय । \newline
15. चा॒भि॒घ्न॒ते॑ ऽभिघ्न॒ते च॑ चाभिघ्न॒ते । \newline
16. अ॒भि॒घ्न॒ते च॑ चाभिघ्न॒ते॑ ऽभिघ्न॒ते च॑ । \newline
17. अ॒भि॒घ्न॒त इत्य॑भि - घ्न॒ते । \newline
18. च॒ नमो॒ नम॑श्च च॒ नमः॑ । \newline
19. नम॑ आक्खिद॒त आ᳚क्खिद॒ते नमो॒ नम॑ आक्खिद॒ते । \newline
20. आ॒क्खि॒द॒ते च॑ चाक्खिद॒त आ᳚क्खिद॒ते च॑ । \newline
21. आ॒क्खि॒द॒त इत्या᳚ - खि॒द॒ते । \newline
22. च॒ प्र॒क्खि॒द॒ते प्र॑क्खिद॒ते च॑ च प्रक्खिद॒ते । \newline
23. प्र॒क्खि॒द॒ते च॑ च प्रक्खिद॒ते प्र॑क्खिद॒ते च॑ । \newline
24. प्र॒क्खि॒द॒त इति॑ प्र - खि॒द॒ते । \newline
25. च॒ नमो॒ नम॑श्च च॒ नमः॑ । \newline
26. नमो॑ वो वो॒ नमो॒ नमो॑ वः । \newline
27. वः॒ कि॒रि॒केभ्यः॑ किरि॒केभ्यो॑ वो वः किरि॒केभ्यः॑ । \newline
28. कि॒रि॒केभ्यो॑ दे॒वाना᳚म् दे॒वाना᳚म् किरि॒केभ्यः॑ किरि॒केभ्यो॑ दे॒वाना᳚म् । \newline
29. दे॒वाना॒(ग्म्॒) हृद॑येभ्यो॒ हृद॑येभ्यो दे॒वाना᳚म् दे॒वाना॒(ग्म्॒) हृद॑येभ्यः । \newline
30. हृद॑येभ्यो॒ नमो॒ नमो॒ हृद॑येभ्यो॒ हृद॑येभ्यो॒ नमः॑ । \newline
31. नमो॑ विक्षीण॒केभ्यो॑ विक्षीण॒केभ्यो॒ नमो॒ नमो॑ विक्षीण॒केभ्यः॑ । \newline
32. वि॒क्षी॒ण॒केभ्यो॒ नमो॒ नमो॑ विक्षीण॒केभ्यो॑ विक्षीण॒केभ्यो॒ नमः॑ । \newline
33. वि॒क्षी॒ण॒केभ्य॒ इति॑ वि - क्षी॒ण॒केभ्यः॑ । \newline
34. नमो॑ विचिन्व॒त्केभ्यो॑ विचिन्व॒त्केभ्यो॒ नमो॒ नमो॑ विचिन्व॒त्केभ्यः॑ । \newline
35. वि॒चि॒न्व॒त्केभ्यो॒ नमो॒ नमो॑ विचिन्व॒त्केभ्यो॑ विचिन्व॒त्केभ्यो॒ नमः॑ । \newline
36. वि॒चि॒न्व॒त्केभ्य॒ इति॑ वि - चि॒न्व॒त्केभ्यः॑ । \newline
37. नम॑ आनिर्.ह॒तेभ्य॑ आनिर्.ह॒तेभ्यो॒ नमो॒ नम॑ आनिर्.ह॒तेभ्यः॑ । \newline
38. आ॒नि॒र्॒.ह॒तेभ्यो॒ नमो॒ नम॑ आनिर्.ह॒तेभ्य॑ आनिर्.ह॒तेभ्यो॒ नमः॑ । \newline
39. आ॒नि॒र्.॒ह॒तेभ्य॒इत्या॑निः - ह॒तेभ्यः॑ । \newline
40. नम॑ आमीव॒त्केभ्य॑ आमीव॒त्केभ्यो॒ नमो॒ नम॑ आमीव॒त्केभ्यः॑ । \newline
41. आ॒मी॒व॒त्केभ्य॒ इत्या᳚ - मी॒व॒त्केभ्यः॑ । \newline

\textbf{Ghana Paata } \newline

1. नम॑ ऊ॒र्व्या॑ यो॒र्व्या॑य॒ नमो॒ नम॑ ऊ॒र्व्या॑य च चो॒र्व्या॑य॒ नमो॒ नम॑ ऊ॒र्व्या॑य च । \newline
2. ऊ॒र्व्या॑य च चो॒र्व्या॑ यो॒र्व्या॑य च सू॒र्म्या॑य सू॒र्म्या॑य चो॒र्व्या॑ यो॒र्व्या॑य च सू॒र्म्या॑य । \newline
3. च॒ सू॒र्म्या॑य सू॒र्म्या॑य च च सू॒र्म्या॑य च च सू॒र्म्या॑य च च सू॒र्म्या॑य च । \newline
4. सू॒र्म्या॑य च च सू॒र्म्या॑य सू॒र्म्या॑य च॒ नमो॒ नम॑श्च सू॒र्म्या॑य सू॒र्म्या॑य च॒ नमः॑ । \newline
5. च॒ नमो॒ नम॑श्च च॒ नमः॑ प॒र्ण्या॑य प॒र्ण्या॑य॒ नम॑श्च च॒ नमः॑ प॒र्ण्या॑य । \newline
6. नमः॑ प॒र्ण्या॑य प॒र्ण्या॑य॒ नमो॒ नमः॑ प॒र्ण्या॑य च च प॒र्ण्या॑य॒ नमो॒ नमः॑ प॒र्ण्या॑य च । \newline
7. प॒र्ण्या॑य च च प॒र्ण्या॑य प॒र्ण्या॑य च पर्णश॒द्या॑य पर्णश॒द्या॑य च प॒र्ण्या॑य प॒र्ण्या॑य च पर्णश॒द्या॑य । \newline
8. च॒ प॒र्ण॒श॒द्या॑य पर्णश॒द्या॑य च च पर्णश॒द्या॑य च च पर्णश॒द्या॑य च च पर्णश॒द्या॑य च । \newline
9. प॒र्ण॒श॒द्या॑य च च पर्णश॒द्या॑य पर्णश॒द्या॑य च॒ नमो॒ नम॑श्च पर्णश॒द्या॑य पर्णश॒द्या॑य च॒ नमः॑ । \newline
10. प॒र्ण॒श॒द्या॑येति॑ पर्ण - श॒द्या॑य । \newline
11. च॒ नमो॒ नम॑श्च च॒ नमो॑ ऽपगु॒रमा॑णाया पगु॒रमा॑णाय॒ नम॑श्च च॒ नमो॑ ऽपगु॒रमा॑णाय । \newline
12. नमो॑ ऽपगु॒रमा॑णाया पगु॒रमा॑णाय॒ नमो॒ नमो॑ ऽपगु॒रमा॑णाय च चापगु॒रमा॑णाय॒ नमो॒ नमो॑ ऽपगु॒रमा॑णाय च । \newline
13. अ॒प॒गु॒रमा॑णाय च चापगु॒रमा॑णाया पगु॒रमा॑णाय चाभिघ्न॒ते॑ ऽभिघ्न॒ते चा॑पगु॒रमा॑णाया पगु॒रमा॑णाय चाभिघ्न॒ते । \newline
14. अ॒प॒गु॒रमा॑णा॒येत्य॑प - गु॒रमा॑णाय । \newline
15. चा॒भि॒घ्न॒ते॑ ऽभिघ्न॒ते च॑ चाभिघ्न॒ते च॑ चाभिघ्न॒ते च॑ चाभिघ्न॒ते च॑ । \newline
16. अ॒भि॒घ्न॒ते च॑ चाभिघ्न॒ते॑ ऽभिघ्न॒ते च॒ नमो॒ नम॑ श्चाभिघ्न॒ते॑ ऽभिघ्न॒ते च॒ नमः॑ । \newline
17. अ॒भि॒घ्न॒त इत्य॑भि - घ्न॒ते । \newline
18. च॒ नमो॒ नम॑श्च च॒ नम॑ आक्खिद॒त आ᳚क्खिद॒ते नम॑श्च च॒ नम॑ आक्खिद॒ते । \newline
19. नम॑ आक्खिद॒त आ᳚क्खिद॒ते नमो॒ नम॑ आक्खिद॒ते च॑ चाक्खिद॒ते नमो॒ नम॑ आक्खिद॒ते च॑ । \newline
20. आ॒क्खि॒द॒ते च॑ चाक्खिद॒त आ᳚क्खिद॒ते च॑ प्रक्खिद॒ते प्र॑क्खिद॒ते चा᳚क्खिद॒त आ᳚क्खिद॒ते च॑ प्रक्खिद॒ते । \newline
21. आ॒क्खि॒द॒त इत्या᳚ - खि॒द॒ते । \newline
22. च॒ प्र॒क्खि॒द॒ते प्र॑क्खिद॒ते च॑ च प्रक्खिद॒ते च॑ च प्रक्खिद॒ते च॑ च प्रक्खिद॒ते च॑ । \newline
23. प्र॒क्खि॒द॒ते च॑ च प्रक्खिद॒ते प्र॑क्खिद॒ते च॒ नमो॒ नम॑श्च प्रक्खिद॒ते प्र॑क्खिद॒ते च॒ नमः॑ । \newline
24. प्र॒क्खि॒द॒त इति॑ प्र - खि॒द॒ते । \newline
25. च॒ नमो॒ नम॑श्च च॒ नमो॑ वो वो॒ नम॑श्च च॒ नमो॑ वः । \newline
26. नमो॑ वो वो॒ नमो॒ नमो॑ वः किरि॒केभ्यः॑ किरि॒केभ्यो॑ वो॒ नमो॒ नमो॑ वः किरि॒केभ्यः॑ । \newline
27. वः॒ कि॒रि॒केभ्यः॑ किरि॒केभ्यो॑ वो वः किरि॒केभ्यो॑ दे॒वाना᳚म् दे॒वाना᳚म् किरि॒केभ्यो॑ वो वः किरि॒केभ्यो॑ दे॒वाना᳚म् । \newline
28. कि॒रि॒केभ्यो॑ दे॒वाना᳚म् दे॒वाना᳚म् किरि॒केभ्यः॑ किरि॒केभ्यो॑ दे॒वानाꣳ॒॒ हृद॑येभ्यो॒ हृद॑येभ्यो दे॒वाना᳚म् किरि॒केभ्यः॑ किरि॒केभ्यो॑ दे॒वानाꣳ॒॒ हृद॑येभ्यः । \newline
29. दे॒वानाꣳ॒॒ हृद॑येभ्यो॒ हृद॑येभ्यो दे॒वाना᳚म् दे॒वानाꣳ॒॒ हृद॑येभ्यो॒ नमो॒ नमो॒ हृद॑येभ्यो दे॒वाना᳚म् दे॒वानाꣳ॒॒ हृद॑येभ्यो॒ नमः॑ । \newline
30. हृद॑येभ्यो॒ नमो॒ नमो॒ हृद॑येभ्यो॒ हृद॑येभ्यो॒ नमो॑ विक्षीण॒केभ्यो॑ विक्षीण॒केभ्यो॒ नमो॒ हृद॑येभ्यो॒ हृद॑येभ्यो॒ नमो॑ विक्षीण॒केभ्यः॑ । \newline
31. नमो॑ विक्षीण॒केभ्यो॑ विक्षीण॒केभ्यो॒ नमो॒ नमो॑ विक्षीण॒केभ्यो॒ नमो॒ नमो॑ विक्षीण॒केभ्यो॒ नमो॒ नमो॑ विक्षीण॒केभ्यो॒ नमः॑ । \newline
32. वि॒क्षी॒ण॒केभ्यो॒ नमो॒ नमो॑ विक्षीण॒केभ्यो॑ विक्षीण॒केभ्यो॒ नमो॑ विचिन्व॒त्केभ्यो॑ विचिन्व॒त्केभ्यो॒ नमो॑ विक्षीण॒केभ्यो॑ विक्षीण॒केभ्यो॒ नमो॑ विचिन्व॒त्केभ्यः॑ । \newline
33. वि॒क्षी॒ण॒केभ्य॒ इति॑ वि - क्षी॒ण॒केभ्यः॑ । \newline
34. नमो॑ विचिन्व॒त्केभ्यो॑ विचिन्व॒त्केभ्यो॒ नमो॒ नमो॑ विचिन्व॒त्केभ्यो॒ नमो॒ नमो॑ विचिन्व॒त्केभ्यो॒ नमो॒ नमो॑ विचिन्व॒त्केभ्यो॒ नमः॑ । \newline
35. वि॒चि॒न्व॒त्केभ्यो॒ नमो॒ नमो॑ विचिन्व॒त्केभ्यो॑ विचिन्व॒त्केभ्यो॒ नम॑ आनिर्.ह॒तेभ्य॑ आनिर्.ह॒तेभ्यो॒ नमो॑ विचिन्व॒त्केभ्यो॑ विचिन्व॒त्केभ्यो॒ नम॑ आनिर्.ह॒तेभ्यः॑ । \newline
36. वि॒चि॒न्व॒त्केभ्य॒ इति॑ वि - चि॒न्व॒त्केभ्यः॑ । \newline
37. नम॑ आनिर्.ह॒तेभ्य॑ आनिर्.ह॒तेभ्यो॒ नमो॒ नम॑ आनिर्.ह॒तेभ्यो॒ नमो॒ नम॑ आनिर्.ह॒तेभ्यो॒ नमो॒ नम॑ आनिर्.ह॒तेभ्यो॒ नमः॑ । \newline
38. आ॒नि॒र्॒.ह॒तेभ्यो॒ नमो॒ नम॑ आनिर्.ह॒तेभ्य॑ आनिर्.ह॒तेभ्यो॒ नम॑ आमीव॒त्केभ्य॑ आमीव॒त्केभ्यो॒ नम॑ आनिर्.ह॒तेभ्य॑ आनिर्.ह॒तेभ्यो॒ नम॑ आमीव॒त्केभ्यः॑ । \newline
39. आ॒नि॒र्.॒ह॒तेभ्य॒इत्या॑निः - ह॒तेभ्यः॑ । \newline
40. नम॑ आमीव॒त्केभ्य॑ आमीव॒त्केभ्यो॒ नमो॒ नम॑ आमीव॒त्केभ्यः॑ । \newline
41. आ॒मी॒व॒त्केभ्य॒ इत्या᳚ - मी॒व॒त्केभ्यः॑ । \newline
\pagebreak
\markright{ TS 4.5.10.1  \hfill https://www.vedavms.in \hfill}

\section{ TS 4.5.10.1 }

\textbf{TS 4.5.10.1 } \newline
\textbf{Samhita Paata} \newline

द्रापे॒ अन्ध॑सस्पते॒ दरि॑द्र॒न्नील॑लोहित । ए॒षां पुरु॑षाणामे॒षां प॑शू॒नां मा भे र्माऽरो॒ मो ए॑षां॒ किञ्च॒नाम॑मत् ॥ या ते॑ रुद्र शि॒वा त॒नूः शि॒वा वि॒श्वाह॑भेषजी । शि॒वा रु॒द्रस्य॑ भेष॒जी तया॑ नो मृड जी॒वसे᳚ ॥                                    इ॒माꣳ रु॒द्राय॑ त॒वसे॑ कप॒र्दिने᳚ क्ष॒यद्वी॑राय॒ प्रभ॑रामहे म॒तिं । यथा॑ नः॒ शमस॑द् द्वि॒पदे॒ चतु॑ष्पदे॒ विश्वं॑ पु॒ष्टं ग्रामे॑ अ॒स्मि - [  ] \newline

\textbf{Pada Paata} \newline

द्रापे᳚ । अन्ध॑सः । प॒ते॒ । दरि॑द्रत् । नील॑लोहि॒तेति॒ नील॑ - लो॒हि॒त॒ ॥ ए॒षाम् । पुरु॑षाणाम् । ए॒षाम् । प॒शू॒नाम् । मा । भेः । मा । अ॒रः॒ । मो इति॑ । ए॒षा॒म् । किम् । च॒न । आ॒म॒म॒त् ॥ या । ते॒ । रु॒द्र॒ । शि॒वा । त॒नूः । शि॒वा । वि॒श्वाह॑भेष॒जीति॑ वि॒श्वाह॑-भे॒ष॒जी॒ ॥ शि॒वा । रु॒द्रस्य॑ । भे॒ष॒जी । तया᳚ । नः॒ । मृ॒ड॒ । जी॒वसे᳚ ॥ इ॒माम् । रु॒द्राय॑ । त॒वसे᳚ । क॒प॒र्दिने᳚ । क्ष॒यद्वी॑रा॒येति॑ क्ष॒यत् - वी॒रा॒य॒ । प्रेति॑ । भ॒रा॒म॒हे॒ । म॒तिम् ॥ यथा᳚ । नः॒ । शम् । अस॑त् । द्वि॒पद॒ इति॑ द्वि - पदे᳚ । चतु॑ष्पद॒ इति॒ चतुः॑ - प॒दे॒ । विश्व᳚म् । पु॒ष्टम् । ग्रामे᳚ । अ॒स्मिन्न् ।  \newline


\textbf{Krama Paata} \newline

द्रापे॒ अन्ध॑सः । अन्ध॑सस्पते । प॒ते॒ दरि॑द्रत् । दरि॑द्र॒न् नील॑लोहित । नील॑लोहि॒तेति॒ नील॑ - लो॒हि॒त॒ ॥ ए॒षां पुरु॑षाणाम् । पुरु॑षाणामे॒षाम् । ए॒षां प॑शू॒नाम् । प॒शू॒नां मा । मा भेः । भेर् मा । माऽरः॑ । अ॒रो॒ मो । मो ए॑षाम् । मो इति॒ मो । ए॒षा॒म् किम् । किम् च॒न । च॒नाम॑मत् । आ॒म॒म॒दित्या॑ममत् ॥ या ते᳚ । ते॒ रु॒द्र॒ । रु॒द्र॒ शि॒वा । शि॒वा त॒नूः । त॒नूः शि॒वा । शि॒वा वि॒श्वाह॑भेषजी । वि॒श्वाह॑भेष॒जीति॑ वि॒श्वाह॑ - भे॒ष॒जी॒ ॥ शि॒वा रु॒द्रस्य॑ । रु॒द्रस्य॑ भेष॒जी । भे॒ष॒जी तया᳚ । तया॑ नः । नो॒ मृ॒ड॒ । मृ॒ड॒ जी॒वसे᳚ । जी॒वस॒ इति॑ जी॒वसे᳚ ॥ इ॒माꣳ रु॒द्राय॑ । रु॒द्राय॑ त॒वसे᳚ । त॒वसे॑ कप॒र्दिने᳚ । क॒प॒र्दिने᳚ क्ष॒यद्वी॑राय । क्ष॒यद्वी॑राय॒ प्र । क्ष॒यद्वी॑रा॒येति॑ क्ष॒यत् - वी॒रा॒य॒ । प्र भ॑रामहे । भ॒रा॒म॒हे॒ म॒तिम् । म॒तिमिति॑ म॒तिम् ॥ यथा॑ नः । नः॒ शम् । शमस॑त् । अस॑द् द्वि॒पदे᳚ । द्वि॒पदे॒ चतु॑ष्पदे । द्वि॒पद॒ इति॑ द्वि - पदे᳚ । चतु॑ष्पदे॒ विश्व᳚म् । चतु॑ष्पद॒ इति॒ चतुः॑ - प॒दे॒ । विश्वं॑ पु॒ष्टम् । पु॒ष्टं ग्रामे᳚ । ग्रामे॑ अ॒स्मिन्न् । अ॒स्मिन्नना॑तुरम् \newline

\textbf{Jatai Paata} \newline

1. द्रापे॒ अन्ध॑सो॒ अन्ध॑सो॒ द्रापे॒ द्रापे॒ अन्ध॑सः । \newline
2. अन्ध॑स स्पते प॒ते ऽन्ध॑सो॒ अन्ध॑स स्पते । \newline
3. प॒ते॒ दरि॑द्र॒द् दरि॑द्रत् पते पते॒ दरि॑द्रत् । \newline
4. दरि॑द्र॒न् नील॑लोहित॒ नील॑लोहित॒ दरि॑द्र॒द् दरि॑द्र॒न् नील॑लोहित । \newline
5. नील॑लोहि॒तेति॒ नील॑ - लो॒हि॒त॒ । \newline
6. ए॒षाम् पुरु॑षाणा॒म् पुरु॑षाणा मे॒षा मे॒षाम् पुरु॑षाणाम् । \newline
7. पुरु॑षाणा मे॒षा मे॒षाम् पुरु॑षाणा॒म् पुरु॑षाणा मे॒षाम् । \newline
8. ए॒षाम् प॑शू॒नाम् प॑शू॒ना मे॒षा मे॒षाम् प॑शू॒नाम् । \newline
9. प॒शू॒नाम् मा मा प॑शू॒नाम् प॑शू॒नाम् मा । \newline
10. मा भेर् भेर् मा मा भेः । \newline
11. भेर् मा मा भेर् भेर् मा । \newline
12. मा ऽरो॑ अरो॒ मा मा ऽरः॑ । \newline
13. अ॒रो॒ मो मो अ॑रो अरो॒ मो । \newline
14. मो ए॑षा मेषा॒म् मो मो ए॑षाम् । \newline
15. मो इति॒ मो । \newline
16. ए॒षा॒म् किम् कि मे॑षा मेषा॒म् किम् । \newline
17. किम् च॒न च॒न किम् किम् च॒न । \newline
18. च॒ना म॑म दाममच् च॒न च॒ना म॑मत् । \newline
19. आ॒म॒म॒दित्या॑ ममत् । \newline
20. या ते॑ ते॒ या या ते᳚ । \newline
21. ते॒ रु॒द्र॒ रु॒द्र॒ ते॒ ते॒ रु॒द्र॒ । \newline
22. रु॒द्र॒ शि॒वा शि॒वा रु॑द्र रुद्र शि॒वा । \newline
23. शि॒वा त॒नू स्त॒नूः शि॒वा शि॒वा त॒नूः । \newline
24. त॒नूः शि॒वा शि॒वा त॒नू स्त॒नूः शि॒वा । \newline
25. शि॒वा वि॒श्वाह॑भेषजी वि॒श्वाह॑भेषजी शि॒वा शि॒वा वि॒श्वाह॑भेषजी । \newline
26. वि॒श्वाह॑भेष॒जीति॑ वि॒श्वाह॑ - भे॒ष॒जी॒ । \newline
27. शि॒वा रु॒द्रस्य॑ रु॒द्रस्य॑ शि॒वा शि॒वा रु॒द्रस्य॑ । \newline
28. रु॒द्रस्य॑ भेष॒जी भे॑ष॒जी रु॒द्रस्य॑ रु॒द्रस्य॑ भेष॒जी । \newline
29. भे॒ष॒जी तया॒ तया॑ भेष॒जी भे॑ष॒जी तया᳚ । \newline
30. तया॑ नो न॒ स्तया॒ तया॑ नः । \newline
31. नो॒ मृ॒ड॒ मृ॒ड॒ नो॒ नो॒ मृ॒ड॒ । \newline
32. मृ॒ड॒ जी॒वसे॑ जी॒वसे॑ मृड मृड जी॒वसे᳚ । \newline
33. जी॒वस॒ इति॑ जी॒वसे᳚ । \newline
34. इ॒माꣳ रु॒द्राय॑ रु॒द्राये॒ मा मि॒माꣳ रु॒द्राय॑ । \newline
35. रु॒द्राय॑ त॒वसे॑ त॒वसे॑ रु॒द्राय॑ रु॒द्राय॑ त॒वसे᳚ । \newline
36. त॒वसे॑ कप॒र्दिने॑ कप॒र्दिने॑ त॒वसे॑ त॒वसे॑ कप॒र्दिने᳚ । \newline
37. क॒प॒र्दिने᳚ क्ष॒यद्वी॑राय क्ष॒यद्वी॑राय कप॒र्दिने॑ कप॒र्दिने᳚ क्ष॒यद्वी॑राय । \newline
38. क्ष॒यद्वी॑राय॒ प्र प्र क्ष॒यद्वी॑राय क्ष॒यद्वी॑राय॒ प्र । \newline
39. क्ष॒यद्वी॑रा॒येति॑ क्ष॒यत् - वी॒रा॒य॒ । \newline
40. प्र भ॑रामहे भरामहे॒ प्र प्र भ॑रामहे । \newline
41. भ॒रा॒म॒हे॒ म॒तिम् म॒तिम् भ॑रामहे भरामहे म॒तिम् । \newline
42. म॒तिमिति॑ म॒तिम् । \newline
43. यथा॑ नो नो॒ यथा॒ यथा॑ नः । \newline
44. नः॒ शꣳ शम् नो॑ नः॒ शम् । \newline
45. शमस॒ दस॒ च्छꣳ शमस॑त् । \newline
46. अस॑द् द्वि॒पदे᳚ द्वि॒पदे॒ अस॒ दस॑द् द्वि॒पदे᳚ । \newline
47. द्वि॒पदे॒ चतु॑ष्पदे॒ चतु॑ष्पदे द्वि॒पदे᳚ द्वि॒पदे॒ चतु॑ष्पदे । \newline
48. द्वि॒पद॒ इति॑ द्वि - पदे᳚ । \newline
49. चतु॑ष्पदे॒ विश्वं॒ ॅविश्व॒म् चतु॑ष्पदे॒ चतु॑ष्पदे॒ विश्व᳚म् । \newline
50. चतु॑ष्पद॒ इति॒ चतुः॑ - प॒दे॒ । \newline
51. विश्व॑म् पु॒ष्टम् पु॒ष्टं ॅविश्वं॒ ॅविश्व॑म् पु॒ष्टम् । \newline
52. पु॒ष्टम् ग्रामे॒ ग्रामे॑ पु॒ष्टम् पु॒ष्टम् ग्रामे᳚ । \newline
53. ग्रामे॑ अ॒स्मिन् न॒स्मिन् ग्रामे॒ ग्रामे॑ अ॒स्मिन्न् । \newline
54. अ॒स्मिन् नना॑तुर॒ मना॑तुर म॒स्मिन् न॒स्मिन् नना॑तुरम् । \newline

\textbf{Ghana Paata } \newline

1. द्रापे॒ अन्ध॑सो॒ अन्ध॑सो॒ द्रापे॒ द्रापे॒ अन्ध॑स स्पते प॒ते ऽन्ध॑सो॒ द्रापे॒ द्रापे॒ अन्ध॑स स्पते । \newline
2. अन्ध॑स स्पते प॒ते ऽन्ध॑सो॒ अन्ध॑स स्पते॒ दरि॑द्र॒द् दरि॑द्रत् प॒ते ऽन्ध॑सो॒ अन्ध॑स स्पते॒ दरि॑द्रत् । \newline
3. प॒ते॒ दरि॑द्र॒द् दरि॑द्रत् पते पते॒ दरि॑द्र॒न् नील॑लोहित॒ नील॑लोहित॒ दरि॑द्रत् पते पते॒ दरि॑द्र॒न् नील॑लोहित । \newline
4. दरि॑द्र॒न् नील॑लोहित॒ नील॑लोहित॒ दरि॑द्र॒द् दरि॑द्र॒न् नील॑लोहित । \newline
5. नील॑लोहि॒तेति॒ नील॑ - लो॒हि॒त॒ । \newline
6. ए॒षाम् पुरु॑षाणा॒म् पुरु॑षाणा मे॒षा मे॒षाम् पुरु॑षाणा मे॒षा मे॒षाम् पुरु॑षाणा मे॒षा मे॒षाम् पुरु॑षाणा मे॒षाम् । \newline
7. पुरु॑षाणा मे॒षा मे॒षाम् पुरु॑षाणा॒म् पुरु॑षाणा मे॒षाम् प॑शू॒नाम् प॑शू॒ना मे॒षाम् पुरु॑षाणा॒म् पुरु॑षाणा मे॒षाम् प॑शू॒नाम् । \newline
8. ए॒षाम् प॑शू॒नाम् प॑शू॒ना मे॒षा मे॒षाम् प॑शू॒नाम् मा मा प॑शू॒ना मे॒षा मे॒षाम् प॑शू॒नाम् मा । \newline
9. प॒शू॒नाम् मा मा प॑शू॒नाम् प॑शू॒नाम् मा भेर् भेर् मा प॑शू॒नाम् प॑शू॒नाम् मा भेः । \newline
10. मा भेर् भेर् मा मा भेर् मा मा भेर् मा मा भेर् मा । \newline
11. भेर् मा मा भेर् भेर् मा ऽरो॑ अरो॒ मा भेर् भेर् मा ऽरः॑ । \newline
12. मा ऽरो॑ अरो॒ मा मा ऽरो॒ मो मो अ॑रो॒ मा मा ऽरो॒ मो । \newline
13. अ॒रो॒ मो मो अ॑रो अरो॒ मो ए॑षा मेषा॒म् मो अ॑रो अरो॒ मो ए॑षाम् । \newline
14. मो ए॑षा मेषा॒म् मो मो ए॑षा॒म् किम् कि मे॑षा॒म् मो मो ए॑षा॒म् किम् । \newline
15. मो इति॒ मो । \newline
16. ए॒षा॒म् किम् कि मे॑षा मेषा॒म् किम् च॒न च॒न कि मे॑षा मेषा॒म् किम् च॒न । \newline
17. किम् च॒न च॒न किम् किम् च॒नाम॑म दाममच् च॒न किम् किम् च॒नाम॑मत् । \newline
18. च॒नाम॑म दाममच् च॒न च॒नाम॑मत् । \newline
19. आ॒म॒म॒दित्या॑ ममत् । \newline
20. या ते॑ ते॒ या या ते॑ रुद्र रुद्र ते॒ या या ते॑ रुद्र । \newline
21. ते॒ रु॒द्र॒ रु॒द्र॒ ते॒ ते॒ रु॒द्र॒ शि॒वा शि॒वा रु॑द्र ते ते रुद्र शि॒वा । \newline
22. रु॒द्र॒ शि॒वा शि॒वा रु॑द्र रुद्र शि॒वा त॒नू स्त॒नूः शि॒वा रु॑द्र रुद्र शि॒वा त॒नूः । \newline
23. शि॒वा त॒नू स्त॒नूः शि॒वा शि॒वा त॒नूः शि॒वा शि॒वा त॒नूः शि॒वा शि॒वा त॒नूः शि॒वा । \newline
24. त॒नूः शि॒वा शि॒वा त॒नू स्त॒नूः शि॒वा वि॒श्वाह॑भेषजी वि॒श्वाह॑भेषजी शि॒वा त॒नू स्त॒नूः शि॒वा वि॒श्वाह॑भेषजी । \newline
25. शि॒वा वि॒श्वाह॑भेषजी वि॒श्वाह॑भेषजी शि॒वा शि॒वा वि॒श्वाह॑भेषजी । \newline
26. वि॒श्वाह॑भेष॒जीति॑ वि॒श्वाह॑ - भे॒ष॒जी॒ । \newline
27. शि॒वा रु॒द्रस्य॑ रु॒द्रस्य॑ शि॒वा शि॒वा रु॒द्रस्य॑ भेष॒जी भे॑ष॒जी रु॒द्रस्य॑ शि॒वा शि॒वा रु॒द्रस्य॑ भेष॒जी । \newline
28. रु॒द्रस्य॑ भेष॒जी भे॑ष॒जी रु॒द्रस्य॑ रु॒द्रस्य॑ भेष॒जी तया॒ तया॑ भेष॒जी रु॒द्रस्य॑ रु॒द्रस्य॑ भेष॒जी तया᳚ । \newline
29. भे॒ष॒जी तया॒ तया॑ भेष॒जी भे॑ष॒जी तया॑ नो न॒ स्तया॑ भेष॒जी भे॑ष॒जी तया॑ नः । \newline
30. तया॑ नो न॒ स्तया॒ तया॑ नो मृड मृड न॒ स्तया॒ तया॑ नो मृड । \newline
31. नो॒ मृ॒ड॒ मृ॒ड॒ नो॒ नो॒ मृ॒ड॒ जी॒वसे॑ जी॒वसे॑ मृड नो नो मृड जी॒वसे᳚ । \newline
32. मृ॒ड॒ जी॒वसे॑ जी॒वसे॑ मृड मृड जी॒वसे᳚ । \newline
33. जी॒वस॒ इति॑ जी॒वसे᳚ । \newline
34. इ॒माꣳ रु॒द्राय॑ रु॒द्राये॒मा मि॒माꣳ रु॒द्राय॑ त॒वसे॑ त॒वसे॑ रु॒द्राये॒मा मि॒माꣳ रु॒द्राय॑ त॒वसे᳚ । \newline
35. रु॒द्राय॑ त॒वसे॑ त॒वसे॑ रु॒द्राय॑ रु॒द्राय॑ त॒वसे॑ कप॒र्दिने॑ कप॒र्दिने॑ त॒वसे॑ रु॒द्राय॑ रु॒द्राय॑ त॒वसे॑ कप॒र्दिने᳚ । \newline
36. त॒वसे॑ कप॒र्दिने॑ कप॒र्दिने॑ त॒वसे॑ त॒वसे॑ कप॒र्दिने᳚ क्ष॒यद्वी॑राय क्ष॒यद्वी॑राय कप॒र्दिने॑ त॒वसे॑ त॒वसे॑ कप॒र्दिने᳚ क्ष॒यद्वी॑राय । \newline
37. क॒प॒र्दिने᳚ क्ष॒यद्वी॑राय क्ष॒यद्वी॑राय कप॒र्दिने॑ कप॒र्दिने᳚ क्ष॒यद्वी॑राय॒ प्र प्र क्ष॒यद्वी॑राय कप॒र्दिने॑ कप॒र्दिने᳚ क्ष॒यद्वी॑राय॒ प्र । \newline
38. क्ष॒यद्वी॑राय॒ प्र प्र क्ष॒यद्वी॑राय क्ष॒यद्वी॑राय॒ प्र भ॑रामहे भरामहे॒ प्र क्ष॒यद्वी॑राय क्ष॒यद्वी॑राय॒ प्र भ॑रामहे । \newline
39. क्ष॒यद्वी॑रा॒येति॑ क्ष॒यत् - वी॒रा॒य॒ । \newline
40. प्र भ॑रामहे भरामहे॒ प्र प्र भ॑रामहे म॒तिम् म॒तिम् भ॑रामहे॒ प्र प्र भ॑रामहे म॒तिम् । \newline
41. भ॒रा॒म॒हे॒ म॒तिम् म॒तिम् भ॑रामहे भरामहे म॒तिम् । \newline
42. म॒तिमिति॑ म॒तिम् । \newline
43. यथा॑ नो नो॒ यथा॒ यथा॑ नः॒ शꣳ शं नो॒ यथा॒ यथा॑ नः॒ शम् । \newline
44. नः॒ शꣳ शं नो॑ नः॒ श मस॒ दस॒च्छं नो॑ नः॒ श मस॑त् । \newline
45. शमस॒ दस॒च्छꣳ श मस॑द् द्वि॒पदे᳚ द्वि॒पदे॒ अस॒च्छꣳ शमस॑द् द्वि॒पदे᳚ । \newline
46. अस॑द् द्वि॒पदे᳚ द्वि॒पदे॒ अस॒ दस॑द् द्वि॒पदे॒ चतु॑ष्पदे॒ चतु॑ष्पदे द्वि॒पदे॒ अस॒ दस॑द् द्वि॒पदे॒ चतु॑ष्पदे । \newline
47. द्वि॒पदे॒ चतु॑ष्पदे॒ चतु॑ष्पदे द्वि॒पदे᳚ द्वि॒पदे॒ चतु॑ष्पदे॒ विश्वं॒ ॅविश्व॒म् चतु॑ष्पदे द्वि॒पदे᳚ द्वि॒पदे॒ चतु॑ष्पदे॒ विश्व᳚म् । \newline
48. द्वि॒पद॒ इति॑ द्वि - पदे᳚ । \newline
49. चतु॑ष्पदे॒ विश्वं॒ ॅविश्व॒म् चतु॑ष्पदे॒ चतु॑ष्पदे॒ विश्व॑म् पु॒ष्टम् पु॒ष्टं ॅविश्व॒म् चतु॑ष्पदे॒ चतु॑ष्पदे॒ विश्व॑म् पु॒ष्टम् । \newline
50. चतु॑ष्पद॒ इति॒ चतुः॑ - प॒दे॒ । \newline
51. विश्व॑म् पु॒ष्टम् पु॒ष्टं ॅविश्वं॒ ॅविश्व॑म् पु॒ष्टम् ग्रामे॒ ग्रामे॑ पु॒ष्टं ॅविश्वं॒ ॅविश्व॑म् पु॒ष्टम् ग्रामे᳚ । \newline
52. पु॒ष्टम् ग्रामे॒ ग्रामे॑ पु॒ष्टम् पु॒ष्टम् ग्रामे॑ अ॒स्मिन् न॒स्मिन् ग्रामे॑ पु॒ष्टम् पु॒ष्टम् ग्रामे॑ अ॒स्मिन्न् । \newline
53. ग्रामे॑ अ॒स्मिन् न॒स्मिन् ग्रामे॒ ग्रामे॑ अ॒स्मिन् नना॑तुर॒ मना॑तुर म॒स्मिन् ग्रामे॒ ग्रामे॑ अ॒स्मिन् नना॑तुरम् । \newline
54. अ॒स्मिन् नना॑तुर॒ मना॑तुर म॒स्मिन् न॒स्मिन् नना॑तुरम् । \newline
\pagebreak
\markright{ TS 4.5.10.2  \hfill https://www.vedavms.in \hfill}

\section{ TS 4.5.10.2 }

\textbf{TS 4.5.10.2 } \newline
\textbf{Samhita Paata} \newline

न्नना॑तुरं ॥ मृ॒डा नो॑ रुद्रो॒ तनो॒ मय॑स्कृधि क्ष॒यद्वी॑राय॒ नम॑सा विधेम ते । यच्छं च॒ योश्च॒ मनु॑राय॒जे पि॒ता तद॑श्याम॒ तव॑ रुद्र॒ प्रणी॑तौ ॥ मा नो॑ म॒हान्त॑मु॒त मा नो॑ अर्भ॒कं मा न॒ उक्ष॑न्तमु॒त मा न॑ उक्षि॒तं । मा नो॑ वधीः पि॒तरं॒ मोत मा॒तरं॑ प्रि॒या मा न॑स्त॒नुवो॑ - [  ] \newline

\textbf{Pada Paata} \newline

अना॑तुर॒मित्यना᳚ - तु॒र॒म् ॥ मृ॒ड । नः॒ । रु॒द्र॒ । उ॒त । नः॒ । मयः॑ । कृ॒धि॒ । क्ष॒यद्वी॑रा॒येति॑ क्ष॒यत् - वी॒रा॒य॒ । नम॑सा । वि॒धे॒म॒ । ते॒ ॥ यत् । शम् । च॒ । योः । च॒ । मनुः॑ । आ॒य॒ज इत्या᳚ - य॒जे । पि॒ता । तत् । अ॒श्या॒म॒ । तव॑ । रु॒द्र॒ । प्रणी॑ता॒विति॒ प्र - नी॒तौ॒ ॥ मा । नः॒ । म॒हान्त᳚म् । उ॒त । मा । नः॒ । अ॒र्भ॒कम् । मा । नः॒ । उक्ष॑न्तम् । उ॒त । मा । नः॒ । उ॒क्षि॒तम् ॥ मा । नः॒ । व॒धीः॒ । पि॒तर᳚म् । मा । उ॒त । मा॒तर᳚म् । प्रि॒याः । मा । नः॒ । त॒नुवः॑ ।  \newline


\textbf{Krama Paata} \newline

अना॑तुर॒मित्यना᳚ - तु॒र॒म् ॥ मृ॒डा नः॑ । नो॒ रु॒द्र॒ । रु॒द्रो॒त । उ॒त नः॑ । नो॒ मयः॑ । मय॑स्कृधि । कृ॒धि॒ क्ष॒यद्वी॑राय । क्ष॒यद्वी॑राय॒ नम॑सा । क्ष॒यद्वी॑रा॒येति॑ क्ष॒यत् - वी॒रा॒य॒ । नम॑सा विधेम । वि॒धे॒म॒ ते॒ । त॒ इति॑ ते ॥ यच्छम् । शम् च॑ । च॒ योः । योश्च॑ । च॒ मनुः॑ । मनु॑राय॒जे । आ॒य॒जे पि॒ता । आ॒य॒ज इत्या᳚ - य॒जे । पि॒ता तत् । तद॑श्याम । अ॒श्या॒म॒ तव॑ । तव॑ रुद्र । रु॒द्र॒ प्रणी॑तौ । प्रणी॑ता॒विति॒ प्र - नी॒तौ॒ ॥ मा नः॑ । नो॒ म॒हान्त᳚म् । म॒हान्त॑मु॒त । उ॒त मा । मा नः॑ । नो॒ अ॒र्भ॒कम् । अ॒र्भ॒कम् मा । मा नः॑ । न॒ उक्ष॑न्तम् । उक्ष॑न्तमु॒त । उ॒त मा । मा नः॑ । न॒ उ॒क्षि॒तम् । उ॒क्षि॒तमित्यु॑क्षि॒तम् ॥ मा नः॑ । नो॒ व॒धीः॒ । व॒धीः॒ पि॒तर᳚म् । पि॒तर॒म् मा । मोत । उ॒त मा॒तर᳚म् । मा॒तर॑म् प्रि॒याः । प्रि॒या मा । मा नः॑ । न॒स्त॒नुवः॑ । त॒नुवो॑ रुद्र \newline

\textbf{Jatai Paata} \newline

1. अना॑तुर॒मित्यना᳚ - तु॒र॒म् । \newline
2. मृ॒डा नो॑ नो मृ॒ड मृ॒डा नः॑ । \newline
3. नो॒ रु॒द्र॒ रु॒द्र॒ नो॒ नो॒ रु॒द्र॒ । \newline
4. रु॒द्रो॒तोत रु॑द्र रुद्रो॒त । \newline
5. उ॒त नो॑ न उ॒तोत नः॑ । \newline
6. नो॒ मयो॒ मयो॑ नो नो॒ मयः॑ । \newline
7. मय॑ स्कृधि कृधि॒ मयो॒ मय॑ स्कृधि । \newline
8. कृ॒धि॒ क्ष॒यद्वी॑राय क्ष॒यद्वी॑राय कृधि कृधि क्ष॒यद्वी॑राय । \newline
9. क्ष॒यद्वी॑राय॒ नम॑सा॒ नम॑सा क्ष॒यद्वी॑राय क्ष॒यद्वी॑राय॒ नम॑सा । \newline
10. क्ष॒यद्वी॑रा॒येति॑ क्ष॒यत् - वी॒रा॒य॒ । \newline
11. नम॑सा विधेम विधेम॒ नम॑सा॒ नम॑सा विधेम । \newline
12. वि॒धे॒म॒ ते॒ ते॒ वि॒धे॒म॒ वि॒धे॒म॒ ते॒ । \newline
13. त॒ इति॑ ते । \newline
14. यच्छꣳ शं ॅयद् यच्छम् । \newline
15. शम् च॑ च॒ शꣳ शम् च॑ । \newline
16. च॒ योर् योश्च॑ च॒ योः । \newline
17. योश्च॑ च॒ योर् योश्च॑ । \newline
18. च॒ मनु॒र् मनु॑श्च च॒ मनुः॑ । \newline
19. मनु॑ राय॒ज आ॑य॒जे मनु॒र् मनु॑ राय॒जे । \newline
20. आ॒य॒जे पि॒ता पि॒ता ऽऽय॒ज आ॑य॒जे पि॒ता । \newline
21. आ॒य॒ज इत्या᳚ - य॒जे । \newline
22. पि॒ता तत् तत् पि॒ता पि॒ता तत् । \newline
23. तद॑श्यामा श्याम॒ तत् तद॑श्याम । \newline
24. अ॒श्या॒म॒ तव॒ तवा᳚श्यामा श्याम॒ तव॑ । \newline
25. तव॑ रुद्र रुद्र॒ तव॒ तव॑ रुद्र । \newline
26. रु॒द्र॒ प्रणी॑तौ॒ प्रणी॑तौ रुद्र रुद्र॒ प्रणी॑तौ । \newline
27. प्रणी॑ता॒विति॒ प्र - नी॒तौ॒ । \newline
28. मा नो॑ नो॒ मा मा नः॑ । \newline
29. नो॒ म॒हान्त॑म् म॒हान्त॑म् नो नो म॒हान्त᳚म् । \newline
30. म॒हान्त॑ मु॒तोत म॒हान्त॑म् म॒हान्त॑ मु॒त । \newline
31. उ॒त मा मो तोत मा । \newline
32. मा नो॑ नो॒ मा मा नः॑ । \newline
33. नो॒ अ॒र्भ॒क म॑र्भ॒कम् नो॑ नो अर्भ॒कम् । \newline
34. अ॒र्भ॒कम् मा मा ऽर्भ॒क म॑र्भ॒कम् मा । \newline
35. मा नो॑ नो॒ मा मा नः॑ । \newline
36. न॒ उक्ष॑न्त॒ मुक्ष॑न्तम् नो न॒ उक्ष॑न्तम् । \newline
37. उक्ष॑न्त मु॒तोतोक्ष॑न्त॒ मुक्ष॑न्त मु॒त । \newline
38. उ॒त मा मो तोत मा । \newline
39. मा नो॑ नो॒ मा मा नः॑ । \newline
40. न॒ उ॒क्षि॒त मु॑क्षि॒तम् नो॑ न उक्षि॒तम् । \newline
41. उ॒क्षि॒तमित्यु॑ क्षि॒तम् । \newline
42. मा नो॑ नो॒ मा मा नः॑ । \newline
43. नो॒ व॒धी॒र् व॒धी॒र् नो॒ नो॒ व॒धीः॒ । \newline
44. व॒धीः॒ पि॒तर॑म् पि॒तरं॑ ॅवधीर् वधीः पि॒तर᳚म् । \newline
45. पि॒तर॒म् मा मा पि॒तर॑म् पि॒तर॒म् मा । \newline
46. मो तोत मा मोत । \newline
47. उ॒त मा॒तर॑म् मा॒तर॑ मु॒तोत मा॒तर᳚म् । \newline
48. मा॒तर॑म् प्रि॒याः प्रि॒या मा॒तर॑म् मा॒तर॑म् प्रि॒याः । \newline
49. प्रि॒या मा मा प्रि॒याः प्रि॒या मा । \newline
50. मा नो॑ नो॒ मा मा नः॑ । \newline
51. न॒ स्त॒नुव॑ स्त॒नुवो॑ नो न स्त॒नुवः॑ । \newline
52. त॒नुवो॑ रुद्र रुद्र त॒नुव॑ स्त॒नुवो॑ रुद्र । \newline

\textbf{Ghana Paata } \newline

1. अना॑तुर॒मित्यना᳚ - तु॒र॒म् । \newline
2. मृ॒डा नो॑ नो मृ॒ड मृ॒डा नो॑ रुद्र रुद्र नो मृ॒ड मृ॒डा नो॑ रुद्र । \newline
3. नो॒ रु॒द्र॒ रु॒द्र॒ नो॒ नो॒ रु॒द्रो॒तोत रु॑द्र नो नो रुद्रो॒त । \newline
4. रु॒द्रो॒तोत रु॑द्र रुद्रो॒त नो॑ न उ॒त रु॑द्र रुद्रो॒त नः॑ । \newline
5. उ॒त नो॑ न उ॒तोत नो॒ मयो॒ मयो॑ न उ॒तोत नो॒ मयः॑ । \newline
6. नो॒ मयो॒ मयो॑ नो नो॒ मय॑ स्कृधि कृधि॒ मयो॑ नो नो॒ मय॑ स्कृधि । \newline
7. मय॑ स्कृधि कृधि॒ मयो॒ मय॑ स्कृधि क्ष॒यद्वी॑राय क्ष॒यद्वी॑राय कृधि॒ मयो॒ मय॑ स्कृधि क्ष॒यद्वी॑राय । \newline
8. कृ॒धि॒ क्ष॒यद्वी॑राय क्ष॒यद्वी॑राय कृधि कृधि क्ष॒यद्वी॑राय॒ नम॑सा॒ नम॑सा क्ष॒यद्वी॑राय कृधि कृधि क्ष॒यद्वी॑राय॒ नम॑सा । \newline
9. क्ष॒यद्वी॑राय॒ नम॑सा॒ नम॑सा क्ष॒यद्वी॑राय क्ष॒यद्वी॑राय॒ नम॑सा विधेम विधेम॒ नम॑सा क्ष॒यद्वी॑राय क्ष॒यद्वी॑राय॒ नम॑सा विधेम । \newline
10. क्ष॒यद्वी॑रा॒येति॑ क्ष॒यत् - वी॒रा॒य॒ । \newline
11. नम॑सा विधेम विधेम॒ नम॑सा॒ नम॑सा विधेम ते ते विधेम॒ नम॑सा॒ नम॑सा विधेम ते । \newline
12. वि॒धे॒म॒ ते॒ ते॒ वि॒धे॒म॒ वि॒धे॒म॒ ते॒ । \newline
13. त॒ इति॑ ते । \newline
14. यच्छꣳ शं ॅयद् यच्छम् च॑ च॒ शं ॅयद् यच्छम् च॑ । \newline
15. शम् च॑ च॒ शꣳ शम् च॒ योर् योश्च॒ शꣳ शम् च॒ योः । \newline
16. च॒ योर् योश्च॑ च॒ योश्च॑ च॒ योश्च॑ च॒ योश्च॑ । \newline
17. योश्च॑ च॒ योर् योश्च॒ मनु॒र् मनु॑श्च॒ योर् योश्च॒ मनुः॑ । \newline
18. च॒ मनु॒र् मनु॑श्च च॒ मनु॑ राय॒ज आ॑य॒जे मनु॑श्च च॒ मनु॑ राय॒जे । \newline
19. मनु॑ राय॒ज आ॑य॒जे मनु॒र् मनु॑ राय॒जे पि॒ता पि॒ता ऽऽय॒जे मनु॒र् मनु॑ राय॒जे पि॒ता । \newline
20. आ॒य॒जे पि॒ता पि॒ता ऽऽय॒ज आ॑य॒जे पि॒ता तत् तत् पि॒ता ऽऽय॒ज आ॑य॒जे पि॒ता तत् । \newline
21. आ॒य॒ज इत्या᳚ - य॒जे । \newline
22. पि॒ता तत् तत् पि॒ता पि॒ता तद॑श्यामा श्याम॒ तत् पि॒ता पि॒ता तद॑श्याम । \newline
23. तद॑श्यामा श्याम॒ तत् तद॑श्याम॒ तव॒ तवा᳚श्याम॒ तत् तद॑श्याम॒ तव॑ । \newline
24. अ॒श्या॒म॒ तव॒ तवा᳚श्यामा श्याम॒ तव॑ रुद्र रुद्र॒ तवा᳚श्यामा श्याम॒ तव॑ रुद्र । \newline
25. तव॑ रुद्र रुद्र॒ तव॒ तव॑ रुद्र॒ प्रणी॑तौ॒ प्रणी॑तौ रुद्र॒ तव॒ तव॑ रुद्र॒ प्रणी॑तौ । \newline
26. रु॒द्र॒ प्रणी॑तौ॒ प्रणी॑तौ रुद्र रुद्र॒ प्रणी॑तौ । \newline
27. प्रणी॑ता॒विति॒ प्र - नी॒तौ॒ । \newline
28. मा नो॑ नो॒ मा मा नो॑ म॒हान्त॑म् म॒हान्त॑म् नो॒ मा मा नो॑ म॒हान्त᳚म् । \newline
29. नो॒ म॒हान्त॑म् म॒हान्तं॑ नो नो म॒हान्त॑ मु॒तोत म॒हान्तं॑ नो नो म॒हान्त॑ मु॒त । \newline
30. म॒हान्त॑ मु॒तोत म॒हान्त॑म् म॒हान्त॑ मु॒त मा मोत म॒हान्त॑म् म॒हान्त॑ मु॒त मा । \newline
31. उ॒त मा मोतोत मा नो॑ नो॒ मोतोत मा नः॑ । \newline
32. मा नो॑ नो॒ मा मा नो॑ अर्भ॒क म॑र्भ॒कं नो॒ मा मा नो॑ अर्भ॒कम् । \newline
33. नो॒ अ॒र्भ॒क म॑र्भ॒कं नो॑ नो अर्भ॒कम् मा मा ऽर्भ॒कं नो॑ नो अर्भ॒कम् मा । \newline
34. अ॒र्भ॒कम् मा मा ऽर्भ॒क म॑र्भ॒कम् मा नो॑ नो॒ मा ऽर्भ॒क म॑र्भ॒कम् मा नः॑ । \newline
35. मा नो॑ नो॒ मा मा न॒ उक्ष॑न्त॒ मुक्ष॑न्तम् नो॒ मा मा न॒ उक्ष॑न्तम् । \newline
36. न॒ उक्ष॑न्त॒ मुक्ष॑न्तम् नो न॒ उक्ष॑न्त मु॒तोतोक्ष॑न्तम् नो न॒ उक्ष॑न्त मु॒त । \newline
37. उक्ष॑न्त मु॒तोतोक्ष॑न्त॒ मुक्ष॑न्त मु॒त मा मोतोक्ष॑न्त॒ मुक्ष॑न्त मु॒त मा । \newline
38. उ॒त मा मोतोत मा नो॑ नो॒ मोतोत मा नः॑ । \newline
39. मा नो॑ नो॒ मा मा न॑ उक्षि॒त मु॑क्षि॒तम् नो॒ मा मा न॑ उक्षि॒तम् । \newline
40. न॒ उ॒क्षि॒त मु॑क्षि॒तम् नो॑ न उक्षि॒तम् । \newline
41. उ॒क्षि॒तमित्यु॑ क्षि॒तम् । \newline
42. मा नो॑ नो॒ मा मा नो॑ वधीर् वधीर् नो॒ मा मा नो॑ वधीः । \newline
43. नो॒ व॒धी॒र् व॒धी॒र् नो॒ नो॒ व॒धीः॒ पि॒तर॑म् पि॒तरं॑ ॅवधीर् नो नो वधीः पि॒तर᳚म् । \newline
44. व॒धीः॒ पि॒तर॑म् पि॒तरं॑ ॅवधीर् वधीः पि॒तर॒म् मा मा पि॒तरं॑ ॅवधीर् वधीः पि॒तर॒म् मा । \newline
45. पि॒तर॒म् मा मा पि॒तर॑म् पि॒तर॒म् मोतोत मा पि॒तर॑म् पि॒तर॒म् मोत । \newline
46. मोतोत मा मोत मा॒तर॑म् मा॒तर॑ मु॒त मा मोत मा॒तर᳚म् । \newline
47. उ॒त मा॒तर॑म् मा॒तर॑ मु॒तोत मा॒तर॑म् प्रि॒याः प्रि॒या मा॒तर॑ मु॒तोत मा॒तर॑म् प्रि॒याः । \newline
48. मा॒तर॑म् प्रि॒याः प्रि॒या मा॒तर॑म् मा॒तर॑म् प्रि॒या मा मा प्रि॒या मा॒तर॑म् मा॒तर॑म् प्रि॒या मा । \newline
49. प्रि॒या मा मा प्रि॒याः प्रि॒या मा नो॑ नो॒ मा प्रि॒याः प्रि॒या मा नः॑ । \newline
50. मा नो॑ नो॒ मा मा न॑ स्त॒नुव॑ स्त॒नुवो॑ नो॒ मा मा न॑ स्त॒नुवः॑ । \newline
51. न॒ स्त॒नुव॑ स्त॒नुवो॑ नो न स्त॒नुवो॑ रुद्र रुद्र त॒नुवो॑ नो न स्त॒नुवो॑ रुद्र । \newline
52. त॒नुवो॑ रुद्र रुद्र त॒नुव॑ स्त॒नुवो॑ रुद्र रीरिषो रीरिषो रुद्र त॒नुव॑ स्त॒नुवो॑ रुद्र रीरिषः । \newline
\pagebreak
\markright{ TS 4.5.10.3  \hfill https://www.vedavms.in \hfill}

\section{ TS 4.5.10.3 }

\textbf{TS 4.5.10.3 } \newline
\textbf{Samhita Paata} \newline

रुद्र रीरिषः ॥ मा न॑स्तो॒के तन॑ये॒ मा न॒ आयु॑षि॒ मा नो॒ गोषु॒ मा नो॒ अश्वे॑षु रीरिषः । वी॒रान् मानो॑ रुद्र भामि॒तो व॑धीर्. ह॒विष्म॑न्तो॒ नम॑सा विधेम ते ॥ आ॒रात्ते॑ गो॒घ्न उ॒त पू॑रुष॒घ्ने क्ष॒यद्वी॑राय सु॒म्नम॒स्मे ते॑ अस्तु । रक्षा॑ च नो॒ अधि॑ च देव ब्रू॒ह्यधा॑ च नः॒ शर्म॑ यच्छ द्वि॒बर्.हाः᳚ ॥ स्तु॒हि - [  ] \newline

\textbf{Pada Paata} \newline

रु॒द्र॒ । री॒रि॒षः॒ ॥ मा । नः॒ । तो॒के । तन॑ये । मा । नः॒ । आयु॑षि । मा । नः॒ । गोषु॑ । मा । नः॒ । अश्वे॑षु । री॒रि॒षः॒ ॥ वी॒रान् । मा । नः॒ । रु॒द्र॒ । भा॒मि॒तः । व॒धीः॒ । ह॒विष्म॑न्तः । नम॑सा । वि॒धे॒म॒ । ते॒ ॥ आ॒रात् । ते॒ । गो॒घ्न इति॑ गो - घ्ने । उ॒त । पू॒रु॒ष॒घ्न इति॑ पूरुष - घ्ने । क्ष॒यद्वी॑रा॒येति॑ क्ष॒यत् - वी॒रा॒य॒ । सु॒म्नम् । अ॒स्मे इति॑ । ते॒ । अ॒स्तु॒ ॥ रक्ष॑ । च॒ । नः॒ । अधीति॑ । च॒ । दे॒व॒ । ब्रू॒हि॒ । अध॑ । च॒ । नः॒ । शर्म॑ । य॒च्छ॒ । द्वि॒बर्.हा॒ इति॑ द्वि - बर्.हाः᳚ ॥ स्तु॒हि ।  \newline


\textbf{Krama Paata} \newline

रु॒द्र॒ री॒रि॒षः॒ । री॒रि॒ष॒ इति॑ रीरिषः ॥ मा नः॑ । न॒स्तो॒के । तो॒के तन॑ये । तन॑ये॒ मा । मा नः॑ । न॒ आयु॑षि । आयु॑षि॒ मा । मा नः॑ । नो॒ गोषु॑ । गोषु॒ मा । मा नः॑ । नो॒ अश्वे॑षु । अश्वे॑षु रीरिषः । री॒रि॒ष॒ इति॑ रीरिषः ॥ वी॒रान् मा । मा नः॑ । नो॒ रु॒द्र॒ । रु॒द्र॒ भा॒मि॒तः । भा॒मि॒तो व॑धीः । व॒धी॒र्.॒ ह॒विष्म॑न्तः । ह॒विष्म॑न्तो॒ नम॑सा । नम॑सा विधेम । वि॒धे॒म॒ ते॒ । त॒ इति॑ ते ॥ आ॒रात् ते᳚ । ते॒ गो॒घ्ने । गो॒घ्न उ॒त । गो॒घ्न इति॑ गो - घ्ने । उ॒त पू॑रुष॒घ्ने । पू॒रु॒ष॒घ्ने क्ष॒यद्वी॑राय । पू॒रु॒ष॒घ्न इति॑ पूरुष - घ्ने । क्ष॒यद्वी॑राय सु॒म्नम् । क्ष॒यद्वी॑रा॒येति॑ क्ष॒यत् - वी॒रा॒य॒ । सु॒म्नम॒स्मे । अ॒स्मे ते᳚ । अ॒स्मे इत्य॒स्मे । ते॒ अ॒स्तु॒ । अ॒स्त्वित्य॑स्तु ॥ रक्षा॑ च । च॒ नः॒ । नो॒ अधि॑ । अधि॑ च । च॒ दे॒व॒ । दे॒व॒ ब्रू॒हि॒ । ब्रू॒ह्यध॑ । अधा॑ च । च॒ नः॒ । नः॒ शर्म॑ । शर्म॑ यच्छ । य॒च्छ॒ द्वि॒बर्.हाः᳚ । द्वि॒बर्.हा॒ इति॑ द्वि - बर्.हाः᳚ ॥ स्तु॒हि श्रु॒तम् \newline

\textbf{Jatai Paata} \newline

1. रु॒द्र॒ री॒रि॒षो॒ री॒रि॒षो॒ रु॒द्र॒ रु॒द्र॒ री॒रि॒षः॒ । \newline
2. री॒रि॒ष॒ इति॑ रीरिषः । \newline
3. मा नो॑ नो॒ मा मा नः॑ । \newline
4. न॒ स्तो॒के तो॒के नो॑ न स्तो॒के । \newline
5. तो॒के तन॑ये॒ तन॑ये तो॒के तो॒के तन॑ये । \newline
6. तन॑ये॒ मा मा तन॑ये॒ तन॑ये॒ मा । \newline
7. मा नो॑ नो॒ मा मा नः॑ । \newline
8. न॒ आयु॒ ष्यायु॑षि नो न॒ आयु॑षि । \newline
9. आयु॑षि॒ मा मा ऽऽयु॒ ष्यायु॑षि॒ मा । \newline
10. मा नो॑ नो॒ मा मा नः॑ । \newline
11. नो॒ गोषु॒ गोषु॑ नो नो॒ गोषु॑ । \newline
12. गोषु॒ मा मा गोषु॒ गोषु॒ मा । \newline
13. मा नो॑ नो॒ मा मा नः॑ । \newline
14. नो॒ अश्वे॒ ष्वश्वे॑षु नो नो॒ अश्वे॑षु । \newline
15. अश्वे॑षु रीरिषो रीरिषो॒ अश्वे॒ ष्वश्वे॑षु रीरिषः । \newline
16. री॒रि॒ष॒ इति॑ रीरिषः । \newline
17. वी॒रान् मा मा वी॒रान्. वी॒रान् मा । \newline
18. मा नो॑ नो॒ मा मा नः॑ । \newline
19. नो॒ रु॒द्र॒ रु॒द्र॒ नो॒ नो॒ रु॒द्र॒ । \newline
20. रु॒द्र॒ भा॒मि॒तो भा॑मि॒तो रु॑द्र रुद्र भामि॒तः । \newline
21. भा॒मि॒तो व॑धीर् वधीर् भामि॒तो भा॑मि॒तो व॑धीः । \newline
22. व॒धी॒र्॒. ह॒विष्म॑न्तो ह॒विष्म॑न्तो वधीर् वधीर्. ह॒विष्म॑न्तः । \newline
23. ह॒विष्म॑न्तो॒ नम॑सा॒ नम॑सा ह॒विष्म॑न्तो ह॒विष्म॑न्तो॒ नम॑सा । \newline
24. नम॑सा विधेम विधेम॒ नम॑सा॒ नम॑सा विधेम । \newline
25. वि॒धे॒म॒ ते॒ ते॒ वि॒धे॒म॒ वि॒धे॒म॒ ते॒ । \newline
26. त॒ इति॑ ते । \newline
27. आ॒रात् ते॑ त आ॒रा दा॒रात् ते᳚ । \newline
28. ते॒ गो॒घ्ने गो॒घ्ने ते॑ ते गो॒घ्ने । \newline
29. गो॒घ्न उ॒तोत गो॒घ्ने गो॒घ्न उ॒त । \newline
30. गो॒घ्न इति॑ गो - घ्ने । \newline
31. उ॒त पू॑रुष॒घ्ने पू॑रुष॒घ्न उ॒तोत पू॑रुष॒घ्ने । \newline
32. पू॒रु॒ष॒घ्ने क्ष॒यद्वी॑राय क्ष॒यद्वी॑राय पूरुष॒घ्ने पू॑रुष॒घ्ने क्ष॒यद्वी॑राय । \newline
33. पू॒रु॒ष॒घ्न इति॑ पूरुष - घ्ने । \newline
34. क्ष॒यद्वी॑राय सु॒म्नꣳ सु॒म्नम् क्ष॒यद्वी॑राय क्ष॒यद्वी॑राय सु॒म्नम् । \newline
35. क्ष॒यद्वी॑रा॒येति॑ क्ष॒यत् - वी॒रा॒य॒ । \newline
36. सु॒म्न म॒स्मे अ॒स्मे सु॒म्नꣳ सु॒म्न म॒स्मे । \newline
37. अ॒स्मे ते॑ ते अ॒स्मे अ॒स्मे ते᳚ । \newline
38. अ॒स्मे इत्य॒स्मे । \newline
39. ते॒ अ॒स्त्व॒स्तु॒ ते॒ ते॒ अ॒स्तु॒ । \newline
40. अ॒स्त्वित्य॑स्तु । \newline
41. रक्षा॑ च च॒ रक्ष॒ रक्षा॑ च । \newline
42. च॒ नो॒ न॒श्च॒ च॒ नः॒ । \newline
43. नो॒ अध्यधि॑ नो नो॒ अधि॑ । \newline
44. अधि॑ च॒ चाध्यधि॑ च । \newline
45. च॒ दे॒व॒ दे॒व॒ च॒ च॒ दे॒व॒ । \newline
46. दे॒व॒ ब्रू॒हि॒ ब्रू॒हि॒ दे॒व॒ दे॒व॒ ब्रू॒हि॒ । \newline
47. ब्रू॒ह्यधाध॑ ब्रूहि ब्रू॒ह्यध॑ । \newline
48. अधा॑ च॒ चाधाधा॑ च । \newline
49. च॒ नो॒ न॒श्च॒ च॒ नः॒ । \newline
50. नः॒ शर्म॒ शर्म॑ नो नः॒ शर्म॑ । \newline
51. शर्म॑ यच्छ यच्छ॒ शर्म॒ शर्म॑ यच्छ । \newline
52. य॒च्छ॒ द्वि॒बर्.हा᳚ द्वि॒बर्.हा॑ यच्छ यच्छ द्वि॒बर्.हाः᳚ । \newline
53. द्वि॒बर्.हा॒ इति॑ द्वि - बर्हाः᳚ । \newline
54. स्तु॒हि श्रु॒तꣳ श्रु॒तꣳ स्तु॒हि स्तु॒हि श्रु॒तम् । \newline

\textbf{Ghana Paata } \newline

1. रु॒द्र॒ री॒रि॒षो॒ री॒रि॒षो॒ रु॒द्र॒ रु॒द्र॒ री॒रि॒षः॒ । \newline
2. री॒रि॒ष॒ इति॑ रीरिषः । \newline
3. मा नो॑ नो॒ मा मा न॑ स्तो॒के तो॒के नो॒ मा मा न॑ स्तो॒के । \newline
4. न॒ स्तो॒के तो॒के नो॑ न स्तो॒के तन॑ये॒ तन॑ये तो॒के नो॑ न स्तो॒के तन॑ये । \newline
5. तो॒के तन॑ये॒ तन॑ये तो॒के तो॒के तन॑ये॒ मा मा तन॑ये तो॒के तो॒के तन॑ये॒ मा । \newline
6. तन॑ये॒ मा मा तन॑ये॒ तन॑ये॒ मा नो॑ नो॒ मा तन॑ये॒ तन॑ये॒ मा नः॑ । \newline
7. मा नो॑ नो॒ मा मा न॒ आयु॒ ष्यायु॑षि नो॒ मा मा न॒ आयु॑षि । \newline
8. न॒ आयु॒ ष्यायु॑षि नो न॒ आयु॑षि॒ मा मा ऽऽयु॑षि नो न॒ आयु॑षि॒ मा । \newline
9. आयु॑षि॒ मा मा ऽऽयु॒ ष्यायु॑षि॒ मा नो॑ नो॒ मा ऽऽयु॒ ष्यायु॑षि॒ मा नः॑ । \newline
10. मा नो॑ नो॒ मा मा नो॒ गोषु॒ गोषु॑ नो॒ मा मा नो॒ गोषु॑ । \newline
11. नो॒ गोषु॒ गोषु॑ नो नो॒ गोषु॒ मा मा गोषु॑ नो नो॒ गोषु॒ मा । \newline
12. गोषु॒ मा मा गोषु॒ गोषु॒ मा नो॑ नो॒ मा गोषु॒ गोषु॒ मा नः॑ । \newline
13. मा नो॑ नो॒ मा मा नो॒ अश्वे॒ ष्वश्वे॑षु नो॒ मा मा नो॒ अश्वे॑षु । \newline
14. नो॒ अश्वे॒ ष्वश्वे॑षु नो नो॒ अश्वे॑षु रीरिषो रीरिषो॒ अश्वे॑षु नो नो॒ अश्वे॑षु रीरिषः । \newline
15. अश्वे॑षु रीरिषो रीरिषो॒ अश्वे॒ ष्वश्वे॑षु रीरिषः । \newline
16. री॒रि॒ष॒ इति॑ रीरिषः । \newline
17. वी॒रान् मा मा वी॒रान् वी॒रान् मा नो॑ नो॒ मा वी॒रान् वी॒रान् मा नः॑ । \newline
18. मा नो॑ नो॒ मा मा नो॑ रुद्र रुद्र नो॒ मा मा नो॑ रुद्र । \newline
19. नो॒ रु॒द्र॒ रु॒द्र॒ नो॒ नो॒ रु॒द्र॒ भा॒मि॒तो भा॑मि॒तो रु॑द्र नो नो रुद्र भामि॒तः । \newline
20. रु॒द्र॒ भा॒मि॒तो भा॑मि॒तो रु॑द्र रुद्र भामि॒तो व॑धीर् वधीर् भामि॒तो रु॑द्र रुद्र भामि॒तो व॑धीः । \newline
21. भा॒मि॒तो व॑धीर् वधीर् भामि॒तो भा॑मि॒तो व॑धीर्. ह॒विष्म॑न्तो ह॒विष्म॑न्तो वधीर् भामि॒तो भा॑मि॒तो व॑धीर्. ह॒विष्म॑न्तः । \newline
22. व॒धी॒र्॒. ह॒विष्म॑न्तो ह॒विष्म॑न्तो वधीर् वधीर्. ह॒विष्म॑न्तो॒ नम॑सा॒ नम॑सा ह॒विष्म॑न्तो वधीर् वधीर्. ह॒विष्म॑न्तो॒ नम॑सा । \newline
23. ह॒विष्म॑न्तो॒ नम॑सा॒ नम॑सा ह॒विष्म॑न्तो ह॒विष्म॑न्तो॒ नम॑सा विधेम विधेम॒ नम॑सा ह॒विष्म॑न्तो ह॒विष्म॑न्तो॒ नम॑सा विधेम । \newline
24. नम॑सा विधेम विधेम॒ नम॑सा॒ नम॑सा विधेम ते ते विधेम॒ नम॑सा॒ नम॑सा विधेम ते । \newline
25. वि॒धे॒म॒ ते॒ ते॒ वि॒धे॒म॒ वि॒धे॒म॒ ते॒ । \newline
26. त॒ इति॑ ते । \newline
27. आ॒रात् ते॑ त आ॒रा दा॒रात् ते॑ गो॒घ्ने गो॒घ्ने त॑ आ॒रा दा॒रात् ते॑ गो॒घ्ने । \newline
28. ते॒ गो॒घ्ने गो॒घ्ने ते॑ ते गो॒घ्न उ॒तोत गो॒घ्ने ते॑ ते गो॒घ्न उ॒त । \newline
29. गो॒घ्न उ॒तोत गो॒घ्ने गो॒घ्न उ॒त पू॑रुष॒घ्ने पू॑रुष॒घ्न उ॒त गो॒घ्ने गो॒घ्न उ॒त पू॑रुष॒घ्ने । \newline
30. गो॒घ्न इति॑ गो - घ्ने । \newline
31. उ॒त पू॑रुष॒घ्ने पू॑रुष॒घ्न उ॒तोत पू॑रुष॒घ्ने क्ष॒यद्वी॑राय क्ष॒यद्वी॑राय पूरुष॒घ्न उ॒तोत पू॑रुष॒घ्ने क्ष॒यद्वी॑राय । \newline
32. पू॒रु॒ष॒घ्ने क्ष॒यद्वी॑राय क्ष॒यद्वी॑राय पूरुष॒घ्ने पू॑रुष॒घ्ने क्ष॒यद्वी॑राय सु॒म्नꣳ सु॒म्नम् क्ष॒यद्वी॑राय पूरुष॒घ्ने पू॑रुष॒घ्ने क्ष॒यद्वी॑राय सु॒म्नम् । \newline
33. पू॒रु॒ष॒घ्न इति॑ पूरुष - घ्ने । \newline
34. क्ष॒यद्वी॑राय सु॒म्नꣳ सु॒म्नम् क्ष॒यद्वी॑राय क्ष॒यद्वी॑राय सु॒म्न म॒स्मे अ॒स्मे सु॒म्नम् क्ष॒यद्वी॑राय क्ष॒यद्वी॑राय सु॒म्न म॒स्मे । \newline
35. क्ष॒यद्वी॑रा॒येति॑ क्ष॒यत् - वी॒रा॒य॒ । \newline
36. सु॒म्न म॒स्मे अ॒स्मे सु॒म्नꣳ सु॒म्न म॒स्मे ते॑ ते अ॒स्मे सु॒म्नꣳ सु॒म्न म॒स्मे ते᳚ । \newline
37. अ॒स्मे ते॑ ते अ॒स्मे अ॒स्मे ते॑ अस्त्वस्तु ते अ॒स्मे अ॒स्मे ते॑ अस्तु । \newline
38. अ॒स्मे इत्य॒स्मे । \newline
39. ते॒ अ॒स्त्व॒स्तु॒ ते॒ ते॒ अ॒स्तु॒ । \newline
40. अ॒स्त्वित्य॑स्तु । \newline
41. रक्षा॑ च च॒ रक्ष॒ रक्षा॑ च नो नश्च॒ रक्ष॒ रक्षा॑ च नः । \newline
42. च॒ नो॒ न॒श्च॒ च॒ नो॒ अध्यधि॑ नश्च च नो॒ अधि॑ । \newline
43. नो॒ अध्यधि॑ नो नो॒ अधि॑ च॒ चाधि॑ नो नो॒ अधि॑ च । \newline
44. अधि॑ च॒ चाध्यधि॑ च देव देव॒ चाध्यधि॑ च देव । \newline
45. च॒ दे॒व॒ दे॒व॒ च॒ च॒ दे॒व॒ ब्रू॒हि॒ ब्रू॒हि॒ दे॒व॒ च॒ च॒ दे॒व॒ ब्रू॒हि॒ । \newline
46. दे॒व॒ ब्रू॒हि॒ ब्रू॒हि॒ दे॒व॒ दे॒व॒ ब्रू॒ह्यधाध॑ ब्रूहि देव देव ब्रू॒ह्यध॑ । \newline
47. ब्रू॒ह्यधाध॑ ब्रूहि ब्रू॒ह्यधा॑ च॒ चाध॑ ब्रूहि ब्रू॒ह्यधा॑ च । \newline
48. अधा॑ च॒ चाधाधा॑ च नो न॒श्चाधाधा॑ च नः । \newline
49. च॒ नो॒ न॒श्च॒ च॒ नः॒ शर्म॒ शर्म॑ नश्च च नः॒ शर्म॑ । \newline
50. नः॒ शर्म॒ शर्म॑ नो नः॒ शर्म॑ यच्छ यच्छ॒ शर्म॑ नो नः॒ शर्म॑ यच्छ । \newline
51. शर्म॑ यच्छ यच्छ॒ शर्म॒ शर्म॑ यच्छ द्वि॒बर्.हा᳚ द्वि॒बर्.हा॑ यच्छ॒ शर्म॒ शर्म॑ यच्छ द्वि॒बर्.हाः᳚ । \newline
52. य॒च्छ॒ द्वि॒बर्.हा᳚ द्वि॒बर्.हा॑ यच्छ यच्छ द्वि॒बर्.हाः᳚ । \newline
53. द्वि॒बर्.हा॒ इति॑ द्वि - बर्हाः᳚ । \newline
54. स्तु॒हि श्रु॒तꣳ श्रु॒तꣳ स्तु॒हि स्तु॒हि श्रु॒तम् ग॑र्त॒सद॑म् गर्त॒सदꣳ॑ श्रु॒तꣳ स्तु॒हि स्तु॒हि श्रु॒तम् ग॑र्त॒सद᳚म् । \newline
\pagebreak
\markright{ TS 4.5.10.4  \hfill https://www.vedavms.in \hfill}

\section{ TS 4.5.10.4 }

\textbf{TS 4.5.10.4 } \newline
\textbf{Samhita Paata} \newline

श्रु॒तं ग॑र्त्त॒सदं॒ ॅयुवा॑नं मृ॒गं न भी॒म-मु॑पह॒त्नु-मु॒ग्रं । मृ॒डा ज॑रि॒त्रे रु॑द्र॒ स्तवा॑नो अ॒न्यन्ते॑ अ॒स्मन्निव॑पन्तु॒ सेनाः᳚ ॥ परि॑णो रु॒द्रस्य॑ हे॒ति र्वृ॑णक्तु॒ परि॑त्वे॒षस्य॑ दुर्म॒तिर॑घा॒योः । अव॑ स्थि॒रा म॒घव॑द्भ्य-स्तनुष्व॒ मीढ्व॑स्तो॒काय॒ तन॑याय मृडय ॥ मीढु॑ष्टम॒ शिव॑तम शि॒वो नः॑ सु॒मना॑ भव ।प॒र॒मे वृ॒क्ष आयु॑धं नि॒धाय॒ कृत्तिं॒ ॅवसा॑न॒ आच॑र॒ पिना॑कं॒ - [  ] \newline

\textbf{Pada Paata} \newline

श्रु॒तम् । ग॒र्त्त॒सद॒मिति॑ गर्त्त-सद᳚म् । युवा॑नम् । मृ॒गम् । न । भी॒मम् । उ॒प॒ह॒त्नुम् । उ॒ग्रम् ॥ मृ॒ड । ज॒रि॒त्रे । रु॒द्र॒ । स्तवा॑नः । अ॒न्यम् । ते॒ । अ॒स्मत् । नीति॑ । व॒प॒न्तु॒ । सेनाः᳚ ॥ परीति॑ । नः॒ । रु॒द्रस्य॑ । हे॒तिः । वृ॒ण॒क्तु॒ । परीति॑ । त्वे॒षस्य॑ । दु॒र्म॒तिरिति॑ दुः - म॒तिः । अ॒घा॒योरित्य॑घ - योः ॥ अवेति॑ । स्थि॒रा । म॒घव॑द्भ्य॒ इति॑ म॒घव॑त् - भ्यः॒ । त॒नु॒ष्व॒ । मीढ्वः॑ । तो॒काय॑ । तन॑याय । मृ॒ड॒य॒ ॥ मीढु॑ष्ट॒मेति॒ मीढुः॑ - त॒म॒ । शिव॑त॒मेति॒ शिव॑ - त॒म॒ । शि॒वः । नः॒ । सु॒मना॒ इति॑ सु - मनाः᳚ । भ॒व॒ ॥ प॒र॒मे । वृ॒क्षे । आयु॑धम् । नि॒धायेति॑ नि - धाय॑ । कृत्ति᳚म् । वसा॑नः । एति॑ । च॒र॒ । पिना॑कम् ।  \newline


\textbf{Krama Paata} \newline

श्रु॒तं ग॑र्त्त॒सद᳚म् । ग॒र्त्त॒सद॒म् ॅयुवा॑नम् । ग॒र्त्त॒सद॒मिति॑ गर्त्त - सद᳚म् । युवा॑नं मृ॒गम् । मृ॒गम् न । न भी॒मम् । भी॒ममु॑पह॒त्नुम् । उ॒प॒ह॒त्नुमु॒ग्रम् । उ॒ग्रमित्यु॒ग्रम् ॥ मृ॒डा ज॑रि॒त्रे । ज॒रि॒त्रे रु॑द्र । रु॒द्र॒ स्तवा॑नः । स्तवा॑नो अ॒न्यम् । अ॒न्यम् ते᳚ । ते॒ अ॒स्मत् । अ॒स्मन् नि । नि व॑पन्तु । व॒प॒न्तु॒ सेनाः᳚ । सेना॒ इति॒ सेनाः᳚ ॥ परि॑ णः । नो॒ रु॒द्रस्य॑ । रु॒द्रस्य॑ हे॒तिः । हे॒तिर् वृ॑णक्तु । वृ॒ण॒क्तु॒ परि॑ । परि॑ त्वे॒षस्य॑ । त्वे॒षस्य॑ दुर्म॒तिः । दु॒र्म॒तिर॑घा॒योः । दु॒र्म॒तिरिति॑ दुः - म॒तिः । अ॒घा॒योरित्य॑घ - योः ॥ अव॑ स्थि॒रा । स्थि॒रा म॒घव॑द्भ्यः । म॒घव॑द्भ्य स्तनुष्व । म॒घव॑द्भ्य॒ इति॑ म॒घव॑त् - भ्यः॒ । त॒नु॒ष्व॒ मीढ्वः॑ । मीढ्व॑ स्तो॒काय॑ । तो॒काय॒ तन॑याय । तन॑याय मृडय । मृ॒ड॒येति॑ मृडय ॥ मीढु॑ष्टम॒ शिव॑तम । मीढु॑ष्ट॒मेति॒ मीढुः॑ - त॒म॒ । शिव॑तम शि॒वः । शिव॑त॒मेति॒ शिव॑ - त॒म॒ । शि॒वो नः॑ । नः॒ सु॒मनाः᳚ । सु॒मना॑ भव । सु॒मना॒ इति॑ सु - मनाः᳚ । भ॒वेति॑ भव ॥ प॒र॒मे वृ॒क्षे । वृ॒क्ष आयु॑धम् । आयु॑धम् नि॒धाय॑ । नि॒धाय॒ कृत्ति᳚म् । नि॒धायेति॑ नि - धाय॑ । कृत्ति॒म् ॅवसा॑नः । वसा॑न॒ आ । आ च॑र । च॒र॒ पिना॑कम् ( ) । पिना॑क॒म् बिभ्र॑त् \newline

\textbf{Jatai Paata} \newline

1. श्रु॒तम् ग॑र्त॒सद॑म् गर्त॒सद(ग्ग्॑) श्रु॒तꣳ श्रु॒तम् ग॑र्त॒सद᳚म् । \newline
2. ग॒र्त॒सदं॒ ॅयुवा॑नं॒ ॅयुवा॑नम् गर्त॒सद॑म् गर्त्त॒सदं॒ ॅयुवा॑नम् । \newline
3. ग॒र्त॒सद॒मिति॑ गर्त - सद᳚म् । \newline
4. युवा॑नम् मृ॒गम् मृ॒गं ॅयुवा॑नं॒ ॅयुवा॑नम् मृ॒गम् । \newline
5. मृ॒गम् न न मृ॒गम् मृ॒गन्न । \newline
6. न भी॒मम् भी॒मम् न न भी॒मम् । \newline
7. भी॒म मु॑पह॒त्नु मु॑पह॒त्नुम् भी॒मम् भी॒म मु॑पह॒त्नुम् । \newline
8. उ॒प॒ह॒त्नु मु॒ग्र मु॒ग्र मु॑पह॒त्नु मु॑पह॒त्नु मु॒ग्रम् । \newline
9. उ॒ग्रमित्यु॒ग्रम् । \newline
10. मृ॒डा ज॑रि॒त्रे ज॑रि॒त्रे मृ॒ड मृ॒डा ज॑रि॒त्रे । \newline
11. ज॒रि॒त्रे रु॑द्र रुद्र जरि॒त्रे ज॑रि॒त्रे रु॑द्र । \newline
12. रु॒द्र॒ स्तवा॑नः॒ स्तवा॑नो रुद्र रुद्र॒ स्तवा॑नः । \newline
13. स्तवा॑नो अ॒न्य म॒न्यꣳ स्तवा॑नः॒ स्तवा॑नो अ॒न्यम् । \newline
14. अ॒न्यम् ते॑ ते अ॒न्य म॒न्यम् ते᳚ । \newline
15. ते॒ अ॒स्म द॒स्मत् ते॑ ते अ॒स्मत् । \newline
16. अ॒स्मन् नि न्य॑स्म द॒स्मन् नि । \newline
17. नि व॑पन्तु वपन्तु॒ नि नि व॑पन्तु । \newline
18. व॒प॒न्तु॒ सेनाः॒ सेना॑ वपन्तु वपन्तु॒ सेनाः᳚ । \newline
19. सेना॒ इति॒ सेनाः᳚ । \newline
20. परि॑ णो नः॒ परि॒ परि॑ णः । \newline
21. नो॒ रु॒द्रस्य॑ रु॒द्रस्य॑ नो नो रु॒द्रस्य॑ । \newline
22. रु॒द्रस्य॑ हे॒तिर्. हे॒ती रु॒द्रस्य॑ रु॒द्रस्य॑ हे॒तिः । \newline
23. हे॒तिर् वृ॑णक्तु वृणक्तु हे॒तिर्. हे॒तिर् वृ॑णक्तु । \newline
24. वृ॒ण॒क्तु॒ परि॒ परि॑ वृणक्तु वृणक्तु॒ परि॑ । \newline
25. परि॑ त्वे॒षस्य॑ त्वे॒षस्य॒ परि॒ परि॑ त्वे॒षस्य॑ । \newline
26. त्वे॒षस्य॑ दुर्म॒तिर् दु॑र्म॒ति स्त्वे॒षस्य॑ त्वे॒षस्य॑ दुर्म॒तिः । \newline
27. दु॒र्म॒ति र॑घा॒यो र॑घा॒योर् दु॑र्म॒तिर् दु॑र्म॒ति र॑घा॒योः । \newline
28. दु॒र्म॒तिरिति॑ दुः - म॒तिः । \newline
29. अ॒घा॒योरित्य॑घ - योः । \newline
30. अव॑ स्थि॒रा स्थि॒रा ऽवाव॑ स्थि॒रा । \newline
31. स्थि॒रा म॒घव॑द्भ्यो म॒घव॑द्भ्यः स्थि॒रा स्थि॒रा म॒घव॑द्भ्यः । \newline
32. म॒घव॑द्भ्य स्तनुष्व तनुष्व म॒घव॑द्भ्यो म॒घव॑द्भ्य स्तनुष्व । \newline
33. म॒घव॑द्भ्य॒ इति॑ म॒घव॑त् - भ्यः॒ । \newline
34. त॒नु॒ष्व॒ मीढ्वो॒ मीढ्व॑ स्तनुष्व तनुष्व॒ मीढ्वः॑ । \newline
35. मीढ्व॑ स्तो॒काय॑ तो॒काय॒ मीढ्वो॒ मीढ्व॑ स्तो॒काय॑ । \newline
36. तो॒काय॒ तन॑याय॒ तन॑याय तो॒काय॑ तो॒काय॒ तन॑याय । \newline
37. तन॑याय मृडय मृडय॒ तन॑याय॒ तन॑याय मृडय । \newline
38. मृ॒ड॒येति॑ मृडय । \newline
39. मीढु॑ष्टम॒ शिव॑तम॒ शिव॑तम॒ मीढु॑ष्टम॒ मीढु॑ष्टम॒ शिव॑तम । \newline
40. मीढु॑ष्ट॒मेति॒ मीढुः॑ - त॒म॒ । \newline
41. शिव॑तम शि॒वः शि॒वः शिव॑तम॒ शिव॑तम शि॒वः । \newline
42. शिव॑त॒मेति॒ शिव॑ - त॒म॒ । \newline
43. शि॒वो नो॑ नः शि॒वः शि॒वो नः॑ । \newline
44. नः॒ सु॒मनाः᳚ सु॒मना॑ नो नः सु॒मनाः᳚ । \newline
45. सु॒मना॑ भव भव सु॒मनाः᳚ सु॒मना॑ भव । \newline
46. सु॒मना॒ इति॑ सु - मनाः᳚ । \newline
47. भ॒वेति॑ भव । \newline
48. प॒र॒मे वृ॒क्षे वृ॒क्षे प॑र॒मे प॑र॒मे वृ॒क्षे । \newline
49. वृ॒क्ष आयु॑ध॒ मायु॑धं ॅवृ॒क्षे वृ॒क्ष आयु॑धम् । \newline
50. आयु॑धम् नि॒धाय॑ नि॒धायायु॑ध॒ मायु॑धम् नि॒धाय॑ । \newline
51. नि॒धाय॒ कृत्ति॒म् कृत्ति॑म् नि॒धाय॑ नि॒धाय॒ कृत्ति᳚म् । \newline
52. नि॒धायेति॑ नि - धाय॑ । \newline
53. कृत्तिं॒ ॅवसा॑नो॒ वसा॑नः॒ कृत्ति॒म् कृत्तिं॒ ॅवसा॑नः । \newline
54. वसा॑न॒ आ वसा॑नो॒ वसा॑न॒ आ । \newline
55. आ च॑र च॒रा च॑र । \newline
56. च॒र॒ पिना॑क॒म् पिना॑कम् चर चर॒ पिना॑कम् । \newline
57. पिना॑क॒म् बिभ्र॒द् बिभ्र॒त् पिना॑क॒म् पिना॑क॒म् बिभ्र॑त् । \newline

\textbf{Ghana Paata } \newline

1. श्रु॒तम् ग॑र्त॒सद॑म् गर्त॒सदꣳ॑ श्रु॒तꣳ श्रु॒तम् ग॑र्त॒सदं॒ ॅयुवा॑नं॒ ॅयुवा॑नम् गर्त॒सदꣳ॑ श्रु॒तꣳ श्रु॒तम् ग॑र्त॒सदं॒ ॅयुवा॑नम् । \newline
2. ग॒र्त॒सदं॒ ॅयुवा॑नं॒ ॅयुवा॑नम् गर्त॒सद॑म् गर्त॒सदं॒ ॅयुवा॑नम् मृ॒गम् मृ॒गं ॅयुवा॑नम् गर्त॒सद॑म् गर्त॒सदं॒ ॅयुवा॑नम् मृ॒गम् । \newline
3. ग॒र्त॒सद॒मिति॑ गर्त - सद᳚म् । \newline
4. युवा॑नम् मृ॒गम् मृ॒गं ॅयुवा॑नं॒ ॅयुवा॑नम् मृ॒गम् न न मृ॒गं ॅयुवा॑नं॒ ॅयुवा॑नम् मृ॒गम् न । \newline
5. मृ॒गम् न न मृ॒गम् मृ॒गम् न भी॒मम् भी॒मम् न मृ॒गम् मृ॒गम् न भी॒मम् । \newline
6. न भी॒मम् भी॒मम् न न भी॒म मु॑पह॒त्नु मु॑पह॒त्नुम् भी॒मम् न न भी॒म मु॑पह॒त्नुम् । \newline
7. भी॒म मु॑पह॒त्नु मु॑पह॒त्नुम् भी॒मम् भी॒म मु॑पह॒त्नु मु॒ग्र मु॒ग्र मु॑पह॒त्नुम् भी॒मम् भी॒म मु॑पह॒त्नु मु॒ग्रम् । \newline
8. उ॒प॒ह॒त्नु मु॒ग्र मु॒ग्र मु॑पह॒त्नु मु॑पह॒त्नु मु॒ग्रम् । \newline
9. उ॒ग्रमित्यु॒ग्रम् । \newline
10. मृ॒डा ज॑रि॒त्रे ज॑रि॒त्रे मृ॒ड मृ॒डा ज॑रि॒त्रे रु॑द्र रुद्र जरि॒त्रे मृ॒ड मृ॒डा ज॑रि॒त्रे रु॑द्र । \newline
11. ज॒रि॒त्रे रु॑द्र रुद्र जरि॒त्रे ज॑रि॒त्रे रु॑द्र॒ स्तवा॑नः॒ स्तवा॑नो रुद्र जरि॒त्रे ज॑रि॒त्रे रु॑द्र॒ स्तवा॑नः । \newline
12. रु॒द्र॒ स्तवा॑नः॒ स्तवा॑नो रुद्र रुद्र॒ स्तवा॑नो अ॒न्य म॒न्यꣳ स्तवा॑नो रुद्र रुद्र॒ स्तवा॑नो अ॒न्यम् । \newline
13. स्तवा॑नो अ॒न्य म॒न्यꣳ स्तवा॑नः॒ स्तवा॑नो अ॒न्यम् ते॑ ते अ॒न्यꣳ स्तवा॑नः॒ स्तवा॑नो अ॒न्यम् ते᳚ । \newline
14. अ॒न्यम् ते॑ ते अ॒न्य म॒न्यम् ते॑ अ॒स्म द॒स्मत् ते॑ अ॒न्य म॒न्यम् ते॑ अ॒स्मत् । \newline
15. ते॒ अ॒स्म द॒स्मत् ते॑ ते अ॒स्मन् नि न्य॑स्मत् ते॑ ते अ॒स्मन् नि । \newline
16. अ॒स्मन् नि न्य॑स्म द॒स्मन् नि व॑पन्तु वपन्तु॒ न्य॑स्म द॒स्मन् नि व॑पन्तु । \newline
17. नि व॑पन्तु वपन्तु॒ नि नि व॑पन्तु॒ सेनाः॒ सेना॑ वपन्तु॒ नि नि व॑पन्तु॒ सेनाः᳚ । \newline
18. व॒प॒न्तु॒ सेनाः॒ सेना॑ वपन्तु वपन्तु॒ सेनाः᳚ । \newline
19. सेना॒ इति॒ सेनाः᳚ । \newline
20. परि॑ णो नः॒ परि॒ परि॑ णो रु॒द्रस्य॑ रु॒द्रस्य॑ नः॒ परि॒ परि॑ णो रु॒द्रस्य॑ । \newline
21. नो॒ रु॒द्रस्य॑ रु॒द्रस्य॑ नो नो रु॒द्रस्य॑ हे॒तिर्. हे॒ती रु॒द्रस्य॑ नो नो रु॒द्रस्य॑ हे॒तिः । \newline
22. रु॒द्रस्य॑ हे॒तिर्. हे॒ती रु॒द्रस्य॑ रु॒द्रस्य॑ हे॒तिर् वृ॑णक्तु वृणक्तु हे॒ती रु॒द्रस्य॑ रु॒द्रस्य॑ हे॒तिर् वृ॑णक्तु । \newline
23. हे॒तिर् वृ॑णक्तु वृणक्तु हे॒तिर्. हे॒तिर् वृ॑णक्तु॒ परि॒ परि॑ वृणक्तु हे॒तिर्. हे॒तिर् वृ॑णक्तु॒ परि॑ । \newline
24. वृ॒ण॒क्तु॒ परि॒ परि॑ वृणक्तु वृणक्तु॒ परि॑ त्वे॒षस्य॑ त्वे॒षस्य॒ परि॑ वृणक्तु वृणक्तु॒ परि॑ त्वे॒षस्य॑ । \newline
25. परि॑ त्वे॒षस्य॑ त्वे॒षस्य॒ परि॒ परि॑ त्वे॒षस्य॑ दुर्म॒तिर् दु॑र्म॒ति स्त्वे॒षस्य॒ परि॒ परि॑ त्वे॒षस्य॑ दुर्म॒तिः । \newline
26. त्वे॒षस्य॑ दुर्म॒तिर् दु॑र्म॒ति स्त्वे॒षस्य॑ त्वे॒षस्य॑ दुर्म॒ति र॑घा॒यो र॑घा॒योर् दु॑र्म॒ति स्त्वे॒षस्य॑ त्वे॒षस्य॑ दुर्म॒ति र॑घा॒योः । \newline
27. दु॒र्म॒ति र॑घा॒यो र॑घा॒योर् दु॑र्म॒तिर् दु॑र्म॒ति र॑घा॒योः । \newline
28. दु॒र्म॒तिरिति॑ दुः - म॒तिः । \newline
29. अ॒घा॒योरित्य॑घ - योः । \newline
30. अव॑ स्थि॒रा स्थि॒रा ऽवाव॑ स्थि॒रा म॒घव॑द्भ्यो म॒घव॑द्भ्यः स्थि॒रा ऽवाव॑ स्थि॒रा म॒घव॑द्भ्यः । \newline
31. स्थि॒रा म॒घव॑द्भ्यो म॒घव॑द्भ्यः स्थि॒रा स्थि॒रा म॒घव॑द्भ्य स्तनुष्व तनुष्व म॒घव॑द्भ्यः स्थि॒रा स्थि॒रा म॒घव॑द्भ्य स्तनुष्व । \newline
32. म॒घव॑द्भ्य स्तनुष्व तनुष्व म॒घव॑द्भ्यो म॒घव॑द्भ्य स्तनुष्व॒ मीढ्वो॒ मीढ्व॑ स्तनुष्व म॒घव॑द्भ्यो म॒घव॑द्भ्य स्तनुष्व॒ मीढ्वः॑ । \newline
33. म॒घव॑द्भ्य॒ इति॑ म॒घव॑त् - भ्यः॒ । \newline
34. त॒नु॒ष्व॒ मीढ्वो॒ मीढ्व॑ स्तनुष्व तनुष्व॒ मीढ्व॑ स्तो॒काय॑ तो॒काय॒ मीढ्व॑ स्तनुष्व तनुष्व॒ मीढ्व॑ स्तो॒काय॑ । \newline
35. मीढ्व॑ स्तो॒काय॑ तो॒काय॒ मीढ्वो॒ मीढ्व॑ स्तो॒काय॒ तन॑याय॒ तन॑याय तो॒काय॒ मीढ्वो॒ मीढ्व॑ स्तो॒काय॒ तन॑याय । \newline
36. तो॒काय॒ तन॑याय॒ तन॑याय तो॒काय॑ तो॒काय॒ तन॑याय मृडय मृडय॒ तन॑याय तो॒काय॑ तो॒काय॒ तन॑याय मृडय । \newline
37. तन॑याय मृडय मृडय॒ तन॑याय॒ तन॑याय मृडय । \newline
38. मृ॒ड॒येति॑ मृडय । \newline
39. मीढु॑ष्टम॒ शिव॑तम॒ शिव॑तम॒ मीढु॑ष्टम॒ मीढु॑ष्टम॒ शिव॑तम शि॒वः शि॒वः शिव॑तम॒ मीढु॑ष्टम॒ मीढु॑ष्टम॒ शिव॑तम शि॒वः । \newline
40. मीढु॑ष्ट॒मेति॒ मीढुः॑ - त॒म॒ । \newline
41. शिव॑तम शि॒वः शि॒वः शिव॑तम॒ शिव॑तम शि॒वो नो॑ नः शि॒वः शिव॑तम॒ शिव॑तम शि॒वो नः॑ । \newline
42. शिव॑त॒मेति॒ शिव॑ - त॒म॒ । \newline
43. शि॒वो नो॑ नः शि॒वः शि॒वो नः॑ सु॒मनाः᳚ सु॒मना॑ नः शि॒वः शि॒वो नः॑ सु॒मनाः᳚ । \newline
44. नः॒ सु॒मनाः᳚ सु॒मना॑ नो नः सु॒मना॑ भव भव सु॒मना॑ नो नः सु॒मना॑ भव । \newline
45. सु॒मना॑ भव भव सु॒मनाः᳚ सु॒मना॑ भव । \newline
46. सु॒मना॒ इति॑ सु - मनाः᳚ । \newline
47. भ॒वेति॑ भव । \newline
48. प॒र॒मे वृ॒क्षे वृ॒क्षे प॑र॒मे प॑र॒मे वृ॒क्ष आयु॑ध॒ मायु॑धं ॅवृ॒क्षे प॑र॒मे प॑र॒मे वृ॒क्ष आयु॑धम् । \newline
49. वृ॒क्ष आयु॑ध॒ मायु॑धं ॅवृ॒क्षे वृ॒क्ष आयु॑धम् नि॒धाय॑ नि॒धाया यु॑धं ॅवृ॒क्षे वृ॒क्ष आयु॑धम् नि॒धाय॑ । \newline
50. आयु॑धम् नि॒धाय॑ नि॒धाया यु॑ध॒ मायु॑धम् नि॒धाय॒ कृत्ति॒म् कृत्ति॑म् नि॒धाया यु॑ध॒ मायु॑धम् नि॒धाय॒ कृत्ति᳚म् । \newline
51. नि॒धाय॒ कृत्ति॒म् कृत्ति॑म् नि॒धाय॑ नि॒धाय॒ कृत्तिं॒ ॅवसा॑नो॒ वसा॑नः॒ कृत्ति॑म् नि॒धाय॑ नि॒धाय॒ कृत्तिं॒ ॅवसा॑नः । \newline
52. नि॒धायेति॑ नि - धाय॑ । \newline
53. कृत्तिं॒ ॅवसा॑नो॒ वसा॑नः॒ कृत्ति॒म् कृत्तिं॒ ॅवसा॑न॒ आ वसा॑नः॒ कृत्ति॒म् कृत्तिं॒ ॅवसा॑न॒ आ । \newline
54. वसा॑न॒ आ वसा॑नो॒ वसा॑न॒ आ च॑र च॒रा वसा॑नो॒ वसा॑न॒ आ च॑र । \newline
55. आ च॑र च॒रा च॑र॒ पिना॑क॒म् पिना॑कम् च॒रा च॑र॒ पिना॑कम् । \newline
56. च॒र॒ पिना॑क॒म् पिना॑कम् चर चर॒ पिना॑क॒म् बिभ्र॒द् बिभ्र॒त् पिना॑कम् चर चर॒ पिना॑क॒म् बिभ्र॑त् । \newline
57. पिना॑क॒म् बिभ्र॒द् बिभ्र॒त् पिना॑क॒म् पिना॑क॒म् बिभ्र॒दा बिभ्र॒त् पिना॑क॒म् पिना॑क॒म् बिभ्र॒दा । \newline
\pagebreak
\markright{ TS 4.5.10.5  \hfill https://www.vedavms.in \hfill}

\section{ TS 4.5.10.5 }

\textbf{TS 4.5.10.5 } \newline
\textbf{Samhita Paata} \newline

बिभ्र॒दाग॑हि ॥ विकि॑रिद॒ विलो॑हित॒ नम॑स्ते अस्तु भगवः । यास्ते॑ स॒हस्रꣳ॑ हे॒तयो॒ऽन्य-म॒स्मन्नि व॑पन्तु॒ ताः ॥                            स॒हस्रा॑णि सहस्र॒धा बा॑हु॒वोस्तव॑ हे॒तयः॑ । तासा॒मीशा॑नो भगवः परा॒चीना॒ मुखा॑ कृधि ॥ \newline

\textbf{Pada Paata} \newline

बिभ्र॑त् । एति॑ । ग॒हि॒ ॥ विकि॑रि॒देति॒ वि - कि॒रि॒द॒ । विलो॑हि॒तेति॒ वि - लो॒हि॒त॒ । नमः॑ । ते॒ । अ॒स्तु॒ । भ॒ग॒व॒ इति॑ भग - वः॒ ॥ याः । ते॒ । स॒हस्र᳚म् । हे॒तयः॑ । अ॒न्यम् । अ॒स्मत् । नीति॑ । व॒प॒न्तु॒ । ताः ॥ स॒हस्रा॑णि । स॒ह॒स्र॒धेति॑ सहस्र-धा । बा॒हु॒वोः । तव॑ । हे॒तयः॑ ॥ तासा᳚म् । ईशा॑नः । भ॒ग॒व॒ इति॑ भग - वः॒ । प॒रा॒चीना᳚ । मुखा᳚ । कृ॒धि॒ ॥  \newline


\textbf{Krama Paata} \newline

बिभ्र॒दा । आ ग॑हि । ग॒हीति॑ गहि ॥ विकि॑रिद॒ विलो॑हित । विकि॑रि॒देति॒ वि - कि॒रि॒द॒ । विलो॑हित॒ नमः॑ । विलो॑हि॒तेति॒ वि - लो॒हि॒त॒ । नम॑स्ते । ते॒ अ॒स्तु॒ । अ॒स्तु॒ भ॒ग॒वः॒ । भ॒ग॒व॒ इति॑ भग - वः॒ ॥ यास्ते᳚ । ते॒ स॒हस्र᳚म् । स॒हस्रꣳ॑ हे॒तयः॑ । हे॒तयो॒ऽन्यम् । अ॒न्यम॒स्मत् । अ॒स्मन् नि । नि व॑पन्तु । व॒प॒न्तु॒ ताः । ता इति॒ ताः ॥ स॒हस्रा॑णि सहस्र॒धा । स॒ह॒स्र॒धा बा॑हु॒वोः । स॒ह॒स्र॒धेति॑ सहस्र - धा । बा॒हु॒वोस्तव॑ । तव॑ हे॒तयः॑ । हे॒तय॒ इति॑ हे॒तयः॑ ॥ तासा॒मीशा॑नः । ईशा॑नो भगवः । भ॒ग॒वः॒ प॒रा॒चीना᳚ । भ॒ग॒व॒ इति॑ भग - वः॒ । प॒रा॒चीना॒ मुखा᳚ । मुखा॑ कृधि । कृ॒धीति॑ कृधि । \newline

\textbf{Jatai Paata} \newline

1. बिभ्र॒दा बिभ्र॒द् बिभ्र॒दा । \newline
2. आ ग॑हि ग॒ह्या ग॑हि । \newline
3. ग॒हीति॑ गहि । \newline
4. विकि॑रिद॒ विलो॑हित॒ विलो॑हित॒ विकि॑रिद॒ विकि॑रिद॒ विलो॑हित । \newline
5. विकि॑रि॒देति॒ वि - कि॒रि॒द॒ । \newline
6. विलो॑हित॒ नमो॒ नमो॒ विलो॑हित॒ विलो॑हित॒ नमः॑ । \newline
7. विलो॑हि॒तेति॒ वि - लो॒हि॒त॒ । \newline
8. नम॑ स्ते ते॒ नमो॒ नम॑ स्ते । \newline
9. ते॒ अ॒स्त्व॒स्तु॒ ते॒ ते॒ अ॒स्तु॒ । \newline
10. अ॒स्तु॒ भ॒ग॒वो॒ भ॒ग॒वो॒ अ॒स्त्व॒स्तु॒ भ॒ग॒वः॒ । \newline
11. भ॒ग॒व॒ इति॑ भग - वः॒ । \newline
12. या स्ते॑ ते॒ या या स्ते᳚ । \newline
13. ते॒ स॒हस्र(ग्म्॑) स॒हस्र॑म् ते ते स॒हस्र᳚म् । \newline
14. स॒हस्र(ग्म्॑) हे॒तयो॑ हे॒तयः॑ स॒हस्र(ग्म्॑) स॒हस्र(ग्म्॑) हे॒तयः॑ । \newline
15. हे॒तयो॒ ऽन्य म॒न्यꣳ हे॒तयो॑ हे॒तयो॒ ऽन्यम् । \newline
16. अ॒न्य म॒स्म द॒स्म द॒न्य म॒न्य म॒स्मत् । \newline
17. अ॒स्मन् नि न्य॑स्म द॒स्मन् नि । \newline
18. नि व॑पन्तु वपन्तु॒ नि नि व॑पन्तु । \newline
19. व॒प॒न्तु॒ तास्ता व॑पन्तु वपन्तु॒ ताः । \newline
20. ता इति॒ ताः । \newline
21. स॒हस्रा॑णि सहस्र॒धा स॑हस्र॒धा स॒हस्रा॑णि स॒हस्रा॑णि सहस्र॒धा । \newline
22. स॒ह॒स्र॒धा बा॑हु॒वोर् बा॑हु॒वोः स॑हस्र॒धा स॑हस्र॒धा बा॑हु॒वोः । \newline
23. स॒ह॒स्र॒धेति॑ सहस्र - धा । \newline
24. बा॒हु॒वो स्तव॒ तव॑ बाहु॒वोर् बा॑हु॒वो स्तव॑ । \newline
25. तव॑ हे॒तयो॑ हे॒तय॒ स्तव॒ तव॑ हे॒तयः॑ । \newline
26. हे॒तय॒ इति॑ हे॒तयः॑ । \newline
27. तासा॒ मीशा॑न॒ ईशा॑न॒ स्तासा॒म् तासा॒ मीशा॑नः । \newline
28. ईशा॑नो भगवो भगव॒ ईशा॑न॒ ईशा॑नो भगवः । \newline
29. भ॒ग॒वः॒ प॒रा॒चीना॑ परा॒चीना॑ भगवो भगवः परा॒चीना᳚ । \newline
30. भ॒ग॒व॒ इति॑ भग - वः॒ । \newline
31. प॒रा॒चीना॒ मुखा॒ मुखा॑ परा॒चीना॑ परा॒चीना॒ मुखा᳚ । \newline
32. मुखा॑ कृधि कृधि॒ मुखा॒ मुखा॑ कृधि । \newline
33. कृ॒धीति॑ कृधि । \newline

\textbf{Ghana Paata } \newline

1. बिभ्र॒दा बिभ्र॒द् बिभ्र॒दा ग॑हि ग॒ह्या बिभ्र॒द् बिभ्र॒दा ग॑हि । \newline
2. आ ग॑हि ग॒ह्या ग॑हि । \newline
3. ग॒हीति॑ गहि । \newline
4. विकि॑रिद॒ विलो॑हित॒ विलो॑हित॒ विकि॑रिद॒ विकि॑रिद॒ विलो॑हित॒ नमो॒ नमो॒ विलो॑हित॒ विकि॑रिद॒ विकि॑रिद॒ विलो॑हित॒ नमः॑ । \newline
5. विकि॑रि॒देति॒ वि - कि॒रि॒द॒ । \newline
6. विलो॑हित॒ नमो॒ नमो॒ विलो॑हित॒ विलो॑हित॒ नम॑ स्ते ते॒ नमो॒ विलो॑हित॒ विलो॑हित॒ नम॑स्ते । \newline
7. विलो॑हि॒तेति॒ वि - लो॒हि॒त॒ । \newline
8. नम॑स्ते ते॒ नमो॒ नम॑स्ते अस्त्वस्तु ते॒ नमो॒ नम॑स्ते अस्तु । \newline
9. ते॒ अ॒स्त्व॒स्तु॒ ते॒ ते॒ अ॒स्तु॒ भ॒ग॒वो॒ भ॒ग॒वो॒ अ॒स्तु॒ ते॒ ते॒ अ॒स्तु॒ भ॒ग॒वः॒ । \newline
10. अ॒स्तु॒ भ॒ग॒वो॒ भ॒ग॒वो॒ अ॒स्त्व॒स्तु॒ भ॒ग॒वः॒ । \newline
11. भ॒ग॒व॒ इति॑ भग - वः॒ । \newline
12. यास्ते॑ ते॒ या यास्ते॑ स॒हस्रꣳ॑ स॒हस्र॑म् ते॒ या यास्ते॑ स॒हस्र᳚म् । \newline
13. ते॒ स॒हस्रꣳ॑ स॒हस्र॑म् ते ते स॒हस्रꣳ॑ हे॒तयो॑ हे॒तयः॑ स॒हस्र॑म् ते ते स॒हस्रꣳ॑ हे॒तयः॑ । \newline
14. स॒हस्रꣳ॑ हे॒तयो॑ हे॒तयः॑ स॒हस्रꣳ॑ स॒हस्रꣳ॑ हे॒तयो॒ ऽन्य म॒न्यꣳ हे॒तयः॑ स॒हस्रꣳ॑ स॒हस्रꣳ॑ हे॒तयो॒ ऽन्यम् । \newline
15. हे॒तयो॒ ऽन्य म॒न्यꣳ हे॒तयो॑ हे॒तयो॒ ऽन्य म॒स्म द॒स्म द॒न्यꣳ हे॒तयो॑ हे॒तयो॒ ऽन्य म॒स्मत् । \newline
16. अ॒न्य म॒स्म द॒स्म द॒न्य म॒न्य म॒स्मन् नि न्य॑स्म द॒न्य म॒न्य म॒स्मन् नि । \newline
17. अ॒स्मन् नि न्य॑स्म द॒स्मन् नि व॑पन्तु वपन्तु॒ न्य॑स्म द॒स्मन् नि व॑पन्तु । \newline
18. नि व॑पन्तु वपन्तु॒ नि नि व॑पन्तु॒ ता स्ता व॑पन्तु॒ नि नि व॑पन्तु॒ ताः । \newline
19. व॒प॒न्तु॒ ता स्ता व॑पन्तु वपन्तु॒ ताः । \newline
20. ता इति॒ ताः । \newline
21. स॒हस्रा॑णि सहस्र॒धा स॑हस्र॒धा स॒हस्रा॑णि स॒हस्रा॑णि सहस्र॒धा बा॑हु॒वोर् बा॑हु॒वोः स॑हस्र॒धा स॒हस्रा॑णि स॒हस्रा॑णि सहस्र॒धा बा॑हु॒वोः । \newline
22. स॒ह॒स्र॒धा बा॑हु॒वोर् बा॑हु॒वोः स॑हस्र॒धा स॑हस्र॒धा बा॑हु॒वो स्तव॒ तव॑ बाहु॒वोः स॑हस्र॒धा स॑हस्र॒धा बा॑हु॒वो स्तव॑ । \newline
23. स॒ह॒स्र॒धेति॑ सहस्र - धा । \newline
24. बा॒हु॒वो स्तव॒ तव॑ बाहु॒वोर् बा॑हु॒वो स्तव॑ हे॒तयो॑ हे॒तय॒ स्तव॑ बाहु॒वोर् बा॑हु॒वो स्तव॑ हे॒तयः॑ । \newline
25. तव॑ हे॒तयो॑ हे॒तय॒ स्तव॒ तव॑ हे॒तयः॑ । \newline
26. हे॒तय॒ इति॑ हे॒तयः॑ । \newline
27. तासा॒ मीशा॑न॒ ईशा॑न॒ स्तासा॒म् तासा॒ मीशा॑नो भगवो भगव॒ ईशा॑न॒ स्तासा॒म् तासा॒ मीशा॑नो भगवः । \newline
28. ईशा॑नो भगवो भगव॒ ईशा॑न॒ ईशा॑नो भगवः परा॒चीना॑ परा॒चीना॑ भगव॒ ईशा॑न॒ ईशा॑नो भगवः परा॒चीना᳚ । \newline
29. भ॒ग॒वः॒ प॒रा॒चीना॑ परा॒चीना॑ भगवो भगवः परा॒चीना॒ मुखा॒ मुखा॑ परा॒चीना॑ भगवो भगवः परा॒चीना॒ मुखा᳚ । \newline
30. भ॒ग॒व॒ इति॑ भग - वः॒ । \newline
31. प॒रा॒चीना॒ मुखा॒ मुखा॑ परा॒चीना॑ परा॒चीना॒ मुखा॑ कृधि कृधि॒ मुखा॑ परा॒चीना॑ परा॒चीना॒ मुखा॑ कृधि । \newline
32. मुखा॑ कृधि कृधि॒ मुखा॒ मुखा॑ कृधि । \newline
33. कृ॒धीति॑ कृधि । \newline
\pagebreak
\markright{ TS 4.5.11.1  \hfill https://www.vedavms.in \hfill}

\section{ TS 4.5.11.1 }

\textbf{TS 4.5.11.1 } \newline
\textbf{Samhita Paata} \newline

स॒हस्रा॑णि सहस्र॒शो ये रु॒द्रा अधि॒ भूम्यां᳚ । तेषाꣳ॑ सहस्रयोज॒ने ऽव॒धन्वा॑नि तन्मसि ॥                                      अ॒स्मिन्-म॑ह॒त्य॑र्ण॒वे᳚-ऽन्तरि॑क्षे भ॒वा अधि॑ ॥                                     नील॑ग्रीवाः शिति॒कण्ठाः᳚ श॒र्वा अ॒धः क्ष॑माच॒राः ॥                                नील॑ग्रीवाः शिति॒कण्ठा॒ दिवꣳ॑ रु॒द्रा उप॑श्रिताः ॥                         ये वृ॒क्षेषु॑ स॒स्पिञ्ज॑रा॒ नील॑ग्रीवा॒ विलो॑हिताः ॥                                        ये भू॒ताना॒-मधि॑पतयो विशि॒खासः॑ कप॒र्दि॑नः ॥                                    ये अन्ने॑षु वि॒विद्ध्य॑न्ति॒ पात्रे॑षु॒ पिब॑तो॒ जनान्॑ ॥                                       ये प॒थां प॑थि॒रक्ष॑य ऐलबृ॒दा य॒व्युधः॑ ॥ ये ती॒र्थानि॑ - [  ] \newline

\textbf{Pada Paata} \newline

स॒हस्रा॑णि । स॒ह॒स्र॒श इति॑ सहस्र - शः । ये । रु॒द्राः । अधीति॑ । भूम्या᳚म् ॥ तेषा᳚म् । स॒ह॒स्र॒यो॒ज॒न इति॑ सहस्र - यो॒ज॒ने । अवेति॑ । धन्वा॑नि । त॒न्म॒सि॒ ॥ अ॒स्मिन्न् । म॒ह॒ति । अ॒र्ण॒वे । अ॒न्तरि॑क्षे । भ॒वाः । अधि॑ ॥ नील॑ग्रीवा॒ इति॒ नील॑ - ग्री॒वाः॒ । शि॒ति॒कण्ठा॒ इति॑ शिति - कण्ठाः᳚ । श॒र्वाः । अ॒धः । क्ष॒मा॒च॒राः ॥ नील॑ग्रीवा॒ इति॒ नील॑-ग्री॒वाः॒ । शि॒ति॒कण्ठा॒ इति॑ शिति - कण्ठाः᳚ । दिव᳚म् । रु॒द्राः । उप॑श्रिता॒ इत्युप॑ - श्रि॒ताः॒ ॥ ये । वृ॒क्षेषु॑ । स॒स्पिञ्ज॑राः । नील॑ग्रीवा॒ इति॒ नील॑ - ग्री॒वाः॒ । विलो॑हिता॒ इति॒ वि - लो॒हि॒ताः॒ ॥ ये । भू॒ताना᳚म् । अधि॑पतय॒ इत्यधि॑ - प॒त॒यः॒ । वि॒शि॒खास॒ इति॑ वि -शि॒खासः॑ । क॒प॒र्दिनः॑ ॥ ये । अन्ने॑षु । वि॒विद्ध्य॒न्तीति॑ वि-विद्ध्य॑न्ति । पात्रे॑षु । पिब॑तः । जनान्॑ ॥ ये । प॒थाम् । प॒थि॒रक्ष॑य॒ इति॑ पथि - रक्ष॑यः । ऐ॒ल॒बृ॒दाः । य॒व्युधः॑ ॥ ये । ती॒र्थानि॑ ।  \newline


\textbf{Krama Paata} \newline

स॒हस्रा॑णि सहस्र॒शः । स॒ह॒स्र॒शो ये । स॒ह॒स्र॒श इति॑ सहस्र - शः । ये रु॒द्राः । रु॒द्रा अधि॑ । अधि॒ भूम्या᳚म् । भूम्या॒मिति॒ भूम्या᳚म् ॥ तेषाꣳ॑ सहस्रयोज॒ने । स॒ह॒स्र॒यो॒ज॒नेऽव॑ । स॒ह॒स्र॒यो॒ज॒न इति॑ सहस्र - यो॒ज॒ने । अव॒ धन्वा॑नि । धन्वा॑नि तन्मसि । त॒न्म॒सीति॑ तन्मसि ॥ अ॒स्मिन् म॑ह॒ति । म॒ह॒त्य॑र्ण॒वे । 
अ॒र्ण॒वे᳚ऽन्तरि॑क्षे । अ॒न्तरि॑क्षे भ॒वाः । भ॒वा अधि॑ । अधी॒त्यधि॑ ॥ नील॑ग्रीवाः शिति॒कण्ठाः᳚ । नील॑ग्रीवा॒ इति॒ नील॑ - ग्री॒वाः॒ । शि॒ति॒कण्ठाः᳚ श॒र्वाः । शि॒ति॒कण्ठा॒ इति॑ शिति - कण्ठाः᳚ । श॒र्वा अ॒धः । अ॒धः क्ष॑माच॒राः । क्ष॒मा॒च॒रा इति॑ क्षमाच॒राः ॥ नील॑ग्रीवाः शिति॒कण्ठाः᳚ । नील॑ग्रीवा॒ इति॒ नील॑ - ग्री॒वाः॒ । शि॒ति॒कण्ठा॒ दिव᳚म् । शि॒ति॒कण्ठा॒ इति॑ शिति - कण्ठाः᳚ । दिवꣳ॑ रु॒द्राः । रु॒द्रा उप॑श्रिताः । उप॑श्रिता॒ इत्युप॑ - श्रि॒ताः॒ ॥ ये वृ॒क्षेषु॑ । वृ॒क्षेषु॑ स॒स्पिञ्ज॑राः । स॒स्पिञ्ज॑रा॒ नील॑ग्रीवाः । नील॑ग्रीवा॒ विलो॑हिताः । नील॑ग्रीवा॒ इति॒ नील॑ - ग्री॒वाः॒ । विलो॑हिता॒ इति॒ वि - लो॒हि॒ताः॒ ॥ ये भू॒ताना᳚म् । भू॒ताना॒मधि॑पतयः । अधि॑पतयो विशि॒खासः॑ । अधि॑पतय॒ इत्यधि॑ - प॒त॒यः॒ । वि॒शि॒खासः॑ कप॒र्दिनः॑ । वि॒शि॒खास॒ इति॑ वि - शि॒खासः॑ । क॒प॒र्दिन॒ इति॑ कप॒र्दिनः॑ ॥ ये अन्ने॑षु । अन्ने॑षु वि॒विद्ध्य॑न्ति । वि॒विद्ध्य॑न्ति॒ पात्रे॑षु । वि॒विद्ध्य॒न्तीति॑ वि - विद्ध्य॑न्ति । पात्रे॑षु॒ पिब॑तः । पिब॑तो॒ जनान्॑ । जना॒निति॒ जनान्॑ ॥ ये प॒थाम् । प॒थां प॑थि॒रक्ष॑यः । प॒थि॒रक्ष॑य ऐलबृ॒दाः । प॒थि॒रक्ष॑य॒ इति॑ पथि - रक्ष॑यः । ऐ॒ल॒बृ॒दा य॒व्युधः॑ । य॒व्युध॒ इति॑ य॒व्युधः॑ ॥ ये ती॒र्त्थानि॑ । ती॒र्त्थानि॑ प्र॒चर॑न्ति \newline

\textbf{Jatai Paata} \newline

1. स॒हस्रा॑णि सहस्र॒शः स॑हस्र॒शः स॒हस्रा॑णि स॒हस्रा॑णि सहस्र॒शः । \newline
2. स॒ह॒स्र॒शो ये ये स॑हस्र॒शः स॑हस्र॒शो ये । \newline
3. स॒ह॒स्र॒श इति॑ सहस्र - शः । \newline
4. ये रु॒द्रा रु॒द्रा ये ये रु॒द्राः । \newline
5. रु॒द्रा अध्यधि॑ रु॒द्रा रु॒द्रा अधि॑ । \newline
6. अधि॒ भूम्या॒म् भूम्या॒ मध्यधि॒ भूम्या᳚म् । \newline
7. भूम्या॒मिति॒ भूम्या᳚म् । \newline
8. तेषा(ग्म्॑) सहस्रयोज॒ने स॑हस्रयोज॒ने तेषा॒म् तेषा(ग्म्॑) सहस्रयोज॒ने । \newline
9. स॒ह॒स्र॒यो॒ज॒ने ऽवाव॑ सहस्रयोज॒ने स॑हस्रयोज॒ने ऽव॑ । \newline
10. स॒ह॒स्र॒यो॒ज॒न इति॑ सहस्र - यो॒ज॒ने । \newline
11. अव॒ धन्वा॑नि॒ धन्वा॒ न्यवाव॒ धन्वा॑नि । \newline
12. धन्वा॑नि तन्मसि तन्मसि॒ धन्वा॑नि॒ धन्वा॑नि तन्मसि । \newline
13. त॒न्म॒सीति॑ तन्मसि । \newline
14. अ॒स्मिन् म॑ह॒ति म॑ह॒ त्य॑स्मिन् न॒स्मिन् म॑ह॒ति । \newline
15. म॒ह॒त्य॑र्ण॒वे᳚ ऽर्ण॒वे म॑ह॒ति म॑ह॒त्य॑र्ण॒वे । \newline
16. अ॒र्ण॒वे᳚ ऽन्तरि॑क्षे॒ ऽन्तरि॑क्षे ऽर्ण॒वे᳚ ऽर्ण॒वे᳚ ऽन्तरि॑क्षे । \newline
17. अ॒न्तरि॑क्षे भ॒वा भ॒वा अ॒न्तरि॑क्षे॒ ऽन्तरि॑क्षे भ॒वाः । \newline
18. भ॒वा अध्यधि॑ भ॒वा भ॒वा अधि॑ । \newline
19. अधीत्यधि॑ । \newline
20. नील॑ग्रीवाः शिति॒कण्ठाः᳚ शिति॒कण्ठा॒ नील॑ग्रीवा॒ नील॑ग्रीवाः शिति॒कण्ठाः᳚ । \newline
21. नील॑ग्रीवा॒ इति॒ नील॑ - ग्री॒वाः॒ । \newline
22. शि॒ति॒कण्ठाः᳚ श॒र्वाः श॒र्वाः शि॑ति॒कण्ठाः᳚ शिति॒कण्ठाः᳚ श॒र्वाः । \newline
23. शि॒ति॒कण्ठा॒ इति॑ शिति - कण्ठाः᳚ । \newline
24. श॒र्वा अ॒धो॑ ऽधः श॒र्वाः श॒र्वा अ॒धः । \newline
25. अ॒धः क्ष॑माच॒राः क्ष॑माच॒रा अ॒धो॑ ऽधः क्ष॑माच॒राः । \newline
26. क्ष॒मा॒च॒रा इति॑ क्षमाच॒राः । \newline
27. नील॑ग्रीवाः शिति॒कण्ठाः᳚ शिति॒कण्ठा॒ नील॑ग्रीवा॒ नील॑ग्रीवाः शिति॒कण्ठाः᳚ । \newline
28. नील॑ग्रीवा॒ इति॒ नील॑ - ग्री॒वाः॒ । \newline
29. शि॒ति॒कण्ठा॒ दिव॒म् दिव(ग्म्॑) शिति॒कण्ठाः᳚ शिति॒कण्ठा॒ दिव᳚म् । \newline
30. शि॒ति॒कण्ठा॒ इति॑ शिति - कण्ठाः᳚ । \newline
31. दिव(ग्म्॑) रु॒द्रा रु॒द्रा दिव॒म् दिव(ग्म्॑) रु॒द्राः । \newline
32. रु॒द्रा उप॑श्रिता॒ उप॑श्रिता रु॒द्रा रु॒द्रा उप॑श्रिताः । \newline
33. उप॑श्रिता॒इत्युप॑ - श्रि॒ताः॒ । \newline
34. ये वृ॒क्षेषु॑ वृ॒क्षेषु॒ ये ये वृ॒क्षेषु॑ । \newline
35. वृ॒क्षेषु॑ स॒स्पिञ्ज॑राः स॒स्पिञ्ज॑रा वृ॒क्षेषु॑ वृ॒क्षेषु॑ स॒स्पिञ्ज॑राः । \newline
36. स॒स्पिञ्ज॑रा॒ नील॑ग्रीवा॒ नील॑ग्रीवाः स॒स्पिञ्ज॑राः स॒स्पिञ्ज॑रा॒ नील॑ग्रीवाः । \newline
37. नील॑ग्रीवा॒ विलो॑हिता॒ विलो॑हिता॒ नील॑ग्रीवा॒ नील॑ग्रीवा॒ विलो॑हिताः । \newline
38. नील॑ग्रीवा॒ इति॒ नील॑ - ग्री॒वाः॒ । \newline
39. विलो॑हिता॒ इति॒ वि - लो॒हि॒ताः॒ । \newline
40. ये भू॒ताना᳚म् भू॒तानां॒ ॅये ये भू॒ताना᳚म् । \newline
41. भू॒ताना॒ मधि॑पत॒यो ऽधि॑पतयो भू॒ताना᳚म् भू॒ताना॒ मधि॑पतयः । \newline
42. अधि॑पतयो विशि॒खासो॑ विशि॒खासो ऽधि॑पत॒यो ऽधि॑पतयो विशि॒खासः॑ । \newline
43. अधि॑पतय॒ इत्यधि॑ - प॒त॒यः॒ । \newline
44. वि॒शि॒खासः॑ कप॒र्दिनः॑ कप॒र्दिनो॑ विशि॒खासो॑ विशि॒खासः॑ कप॒र्दिनः॑ । \newline
45. वि॒शि॒खास॒ इति॑ वि - शि॒खासः॑ । \newline
46. क॒प॒र्दिन॒ इति॑ कप॒र्दिनः॑ । \newline
47. ये अन्ने॒ ष्वन्ने॑षु॒ ये ये अन्ने॑षु । \newline
48. अन्ने॑षु वि॒विद्ध्य॑न्ति वि॒विद्ध्य॒न् त्यन्ने॒ ष्वन्ने॑षु वि॒विद्ध्य॑न्ति । \newline
49. वि॒विद्ध्य॑न्ति॒ पात्रे॑षु॒ पात्रे॑षु वि॒विद्ध्य॑न्ति वि॒विद्ध्य॑न्ति॒ पात्रे॑षु । \newline
50. वि॒विद्ध्य॒न्तीति॑ वि - विद्ध्य॑न्ति । \newline
51. पात्रे॑षु॒ पिब॑तः॒ पिब॑तः॒ पात्रे॑षु॒ पात्रे॑षु॒ पिब॑तः । \newline
52. पिब॑तो॒ जना॒न् जना॒न् पिब॑तः॒ पिब॑तो॒ जनान्॑ । \newline
53. जना॒निति॒ जनान्॑ । \newline
54. ये प॒थाम् प॒थां ॅये ये प॒थाम् । \newline
55. प॒थाम् प॑थि॒रक्ष॑यः पथि॒रक्ष॑यः प॒थाम् प॒थाम् प॑थि॒रक्ष॑यः । \newline
56. प॒थि॒रक्ष॑य ऐलबृ॒दा ऐ॑लबृ॒दाः प॑थि॒रक्ष॑यः पथि॒रक्ष॑य ऐलबृ॒दाः । \newline
57. प॒थि॒रक्ष॑य॒ इति॑ पथि - रक्ष॑यः । \newline
58. ऐ॒ल॒बृ॒दा य॒व्युधो॑ य॒व्युध॑ ऐलबृ॒दा ऐ॑लबृ॒दा य॒व्युधः॑ । \newline
59. य॒व्युध॒ इति॑ य॒व्युधः॑ । \newline
60. ये ती॒र्थानि॑ ती॒र्थानि॒ ये ये ती॒र्थानि॑ । \newline
61. ती॒र्थानि॑ प्र॒चर॑न्ति प्र॒चर॑न्ति ती॒र्थानि॑ ती॒र्थानि॑ प्र॒चर॑न्ति । \newline

\textbf{Ghana Paata } \newline

1. स॒हस्रा॑णि सहस्र॒शः स॑हस्र॒शः स॒हस्रा॑णि स॒हस्रा॑णि सहस्र॒शो ये ये स॑हस्र॒शः स॒हस्रा॑णि स॒हस्रा॑णि सहस्र॒शो ये । \newline
2. स॒ह॒स्र॒शो ये ये स॑हस्र॒शः स॑हस्र॒शो ये रु॒द्रा रु॒द्रा ये स॑हस्र॒शः स॑हस्र॒शो ये रु॒द्राः । \newline
3. स॒ह॒स्र॒श इति॑ सहस्र - शः । \newline
4. ये रु॒द्रा रु॒द्रा ये ये रु॒द्रा अध्यधि॑ रु॒द्रा ये ये रु॒द्रा अधि॑ । \newline
5. रु॒द्रा अध्यधि॑ रु॒द्रा रु॒द्रा अधि॒ भूम्या॒म् भूम्या॒ मधि॑ रु॒द्रा रु॒द्रा अधि॒ भूम्या᳚म् । \newline
6. अधि॒ भूम्या॒म् भूम्या॒ मध्यधि॒ भूम्या᳚म् । \newline
7. भूम्या॒मिति॒ भूम्या᳚म् । \newline
8. तेषाꣳ॑ सहस्रयोज॒ने स॑हस्रयोज॒ने तेषा॒म् तेषाꣳ॑ सहस्रयोज॒ने ऽवाव॑ सहस्रयोज॒ने तेषा॒म् तेषाꣳ॑ सहस्रयोज॒ने ऽव॑ । \newline
9. स॒ह॒स्र॒यो॒ज॒ने ऽवाव॑ सहस्रयोज॒ने स॑हस्रयोज॒ने ऽव॒ धन्वा॑नि॒ धन्वा॒ न्यव॑ सहस्रयोज॒ने स॑हस्रयोज॒ने ऽव॒ धन्वा॑नि । \newline
10. स॒ह॒स्र॒यो॒ज॒न इति॑ सहस्र - यो॒ज॒ने । \newline
11. अव॒ धन्वा॑नि॒ धन्वा॒ न्यवाव॒ धन्वा॑नि तन्मसि तन्मसि॒ धन्वा॒ न्यवाव॒ धन्वा॑नि तन्मसि । \newline
12. धन्वा॑नि तन्मसि तन्मसि॒ धन्वा॑नि॒ धन्वा॑नि तन्मसि । \newline
13. त॒न्म॒सीति॑ तन्मसि । \newline
14. अ॒स्मिन् म॑ह॒ति म॑ह॒त्य॑स्मिन् न॒स्मिन् म॑ह॒त्य॑र्ण॒वे᳚ ऽर्ण॒वे म॑ह॒त्य॑स्मिन् न॒स्मिन् म॑ह॒त्य॑र्ण॒वे । \newline
15. म॒ह॒त्य॑र्ण॒वे᳚ ऽर्ण॒वे म॑ह॒ति म॑ह॒त्य॑र्ण॒वे᳚ ऽन्तरि॑क्षे॒ ऽन्तरि॑क्षे ऽर्ण॒वे म॑ह॒ति 
म॑ह॒त्य॑र्ण॒वे᳚ ऽन्तरि॑क्षे । \newline
16. अ॒र्ण॒वे᳚ ऽन्तरि॑क्षे॒ ऽन्तरि॑क्षे ऽर्ण॒वे᳚ ऽर्ण॒वे᳚ ऽन्तरि॑क्षे भ॒वा भ॒वा अ॒न्तरि॑क्षे ऽर्ण॒वे᳚ ऽर्ण॒वे᳚ ऽन्तरि॑क्षे भ॒वाः । \newline
17. अ॒न्तरि॑क्षे भ॒वा भ॒वा अ॒न्तरि॑क्षे॒ ऽन्तरि॑क्षे भ॒वा अध्यधि॑ भ॒वा अ॒न्तरि॑क्षे॒ ऽन्तरि॑क्षे भ॒वा अधि॑ । \newline
18. भ॒वा अध्यधि॑ भ॒वा भ॒वा अधि॑ । \newline
19. अधीत्यधि॑ । \newline
20. नील॑ग्रीवाः शिति॒कण्ठाः᳚ शिति॒कण्ठा॒ नील॑ग्रीवा॒ नील॑ग्रीवाः शिति॒कण्ठाः᳚ श॒र्वाः श॒र्वाः शि॑ति॒कण्ठा॒ नील॑ग्रीवा॒ नील॑ग्रीवाः शिति॒कण्ठाः᳚ श॒र्वाः । \newline
21. नील॑ग्रीवा॒ इति॒ नील॑ - ग्री॒वाः॒ । \newline
22. शि॒ति॒कण्ठाः᳚ श॒र्वाः श॒र्वाः शि॑ति॒कण्ठाः᳚ शिति॒कण्ठाः᳚ श॒र्वा अ॒धो॑ ऽधः श॒र्वाः शि॑ति॒कण्ठाः᳚ शिति॒कण्ठाः᳚ श॒र्वा अ॒धः । \newline
23. शि॒ति॒कण्ठा॒ इति॑ शिति - कण्ठाः᳚ । \newline
24. श॒र्वा अ॒धो॑ ऽधः श॒र्वाः श॒र्वा अ॒धः क्ष॑माच॒राः क्ष॑माच॒रा अ॒धः श॒र्वाः श॒र्वा अ॒धः क्ष॑माच॒राः । \newline
25. अ॒धः क्ष॑माच॒राः क्ष॑माच॒रा अ॒धो॑ ऽधः क्ष॑माच॒राः । \newline
26. क्ष॒मा॒च॒रा इति॑ क्षमाच॒राः । \newline
27. नील॑ग्रीवाः शिति॒कण्ठाः᳚ शिति॒कण्ठा॒ नील॑ग्रीवा॒ नील॑ग्रीवाः शिति॒कण्ठा॒ दिव॒म् दिवꣳ॑ शिति॒कण्ठा॒ नील॑ग्रीवा॒ नील॑ग्रीवाः शिति॒कण्ठा॒ दिव᳚म् । \newline
28. नील॑ग्रीवा॒ इति॒ नील॑ - ग्री॒वाः॒ । \newline
29. शि॒ति॒कण्ठा॒ दिव॒म् दिवꣳ॑ शिति॒कण्ठाः᳚ शिति॒कण्ठा॒ दिवꣳ॑ रु॒द्रा रु॒द्रा दिवꣳ॑ शिति॒कण्ठाः᳚ शिति॒कण्ठा॒ दिवꣳ॑ रु॒द्राः । \newline
30. शि॒ति॒कण्ठा॒ इति॑ शिति - कण्ठाः᳚ । \newline
31. दिवꣳ॑ रु॒द्रा रु॒द्रा दिव॒म् दिवꣳ॑ रु॒द्रा उप॑श्रिता॒ उप॑श्रिता रु॒द्रा दिव॒म् दिवꣳ॑ रु॒द्रा उप॑श्रिताः । \newline
32. रु॒द्रा उप॑श्रिता॒ उप॑श्रिता रु॒द्रा रु॒द्रा उप॑श्रिताः । \newline
33. उप॑श्रिता॒ इत्युप॑ - श्रि॒ताः॒ । \newline
34. ये वृ॒क्षेषु॑ वृ॒क्षेषु॒ ये ये वृ॒क्षेषु॑ स॒स्पिञ्ज॑राः स॒स्पिञ्ज॑रा वृ॒क्षेषु॒ ये ये वृ॒क्षेषु॑ स॒स्पिञ्ज॑राः । \newline
35. वृ॒क्षेषु॑ स॒स्पिञ्ज॑राः स॒स्पिञ्ज॑रा वृ॒क्षेषु॑ वृ॒क्षेषु॑ स॒स्पिञ्ज॑रा॒ नील॑ग्रीवा॒ नील॑ग्रीवाः स॒स्पिञ्ज॑रा वृ॒क्षेषु॑ वृ॒क्षेषु॑ स॒स्पिञ्ज॑रा॒ नील॑ग्रीवाः । \newline
36. स॒स्पिञ्ज॑रा॒ नील॑ग्रीवा॒ नील॑ग्रीवाः स॒स्पिञ्ज॑राः स॒स्पिञ्ज॑रा॒ नील॑ग्रीवा॒ विलो॑हिता॒ विलो॑हिता॒ नील॑ग्रीवाः स॒स्पिञ्ज॑राः स॒स्पिञ्ज॑रा॒ नील॑ग्रीवा॒ विलो॑हिताः । \newline
37. नील॑ग्रीवा॒ विलो॑हिता॒ विलो॑हिता॒ नील॑ग्रीवा॒ नील॑ग्रीवा॒ विलो॑हिताः । \newline
38. नील॑ग्रीवा॒ इति॒ नील॑ - ग्री॒वाः॒ । \newline
39. विलो॑हिता॒ इति॒ वि - लो॒हि॒ताः॒ । \newline
40. ये भू॒ताना᳚म् भू॒तानां॒ ॅये ये भू॒ताना॒ मधि॑पत॒यो ऽधि॑पतयो भू॒तानां॒ ॅये ये भू॒ताना॒ मधि॑पतयः । \newline
41. भू॒ताना॒ मधि॑पत॒यो ऽधि॑पतयो भू॒ताना᳚म् भू॒ताना॒ मधि॑पतयो विशि॒खासो॑ विशि॒खासो ऽधि॑पतयो भू॒ताना᳚म् भू॒ताना॒ मधि॑पतयो विशि॒खासः॑ । \newline
42. अधि॑पतयो विशि॒खासो॑ विशि॒खासो ऽधि॑पत॒यो ऽधि॑पतयो विशि॒खासः॑ कप॒र्दिनः॑ कप॒र्दिनो॑ विशि॒खासो ऽधि॑पत॒यो ऽधि॑पतयो विशि॒खासः॑ कप॒र्दिनः॑ । \newline
43. अधि॑पतय॒ इत्यधि॑ - प॒त॒यः॒ । \newline
44. वि॒शि॒खासः॑ कप॒र्दिनः॑ कप॒र्दिनो॑ विशि॒खासो॑ विशि॒खासः॑ कप॒र्दिनः॑ । \newline
45. वि॒शि॒खास॒ इति॑ वि - शि॒खासः॑ । \newline
46. क॒प॒र्दिन॒ इति॑ कप॒र्दिनः॑ । \newline
47. ये अन्ने॒ष्वन्ने॑षु॒ ये ये अन्ने॑षु वि॒विद्ध्य॑न्ति वि॒विद्ध्य॒न् त्यन्ने॑षु॒ ये ये अन्ने॑षु वि॒विद्ध्य॑न्ति । \newline
48. अन्ने॑षु वि॒विद्ध्य॑न्ति वि॒विद्ध्य॒न् त्यन्ने॒ ष्वन्ने॑षु वि॒विद्ध्य॑न्ति॒ पात्रे॑षु॒ पात्रे॑षु वि॒विद्ध्य॒न् त्यन्ने॒ ष्वन्ने॑षु वि॒विद्ध्य॑न्ति॒ पात्रे॑षु । \newline
49. वि॒विद्ध्य॑न्ति॒ पात्रे॑षु॒ पात्रे॑षु वि॒विद्ध्य॑न्ति वि॒विद्ध्य॑न्ति॒ पात्रे॑षु॒ पिब॑तः॒ पिब॑तः॒ पात्रे॑षु वि॒विद्ध्य॑न्ति वि॒विद्ध्य॑न्ति॒ पात्रे॑षु॒ पिब॑तः । \newline
50. वि॒विद्ध्य॒न्तीति॑ वि - विद्ध्य॑न्ति । \newline
51. पात्रे॑षु॒ पिब॑तः॒ पिब॑तः॒ पात्रे॑षु॒ पात्रे॑षु॒ पिब॑तो॒ जना॒न् जना॒न् पिब॑तः॒ पात्रे॑षु॒ पात्रे॑षु॒ पिब॑तो॒ जनान्॑ । \newline
52. पिब॑तो॒ जना॒न् जना॒न् पिब॑तः॒ पिब॑तो॒ जनान्॑ । \newline
53. जना॒निति॒ जनान्॑ । \newline
54. ये प॒थाम् प॒थां ॅये ये प॒थाम् प॑थि॒रक्ष॑यः पथि॒रक्ष॑यः प॒थां ॅये ये प॒थाम् प॑थि॒रक्ष॑यः । \newline
55. प॒थाम् प॑थि॒रक्ष॑यः पथि॒रक्ष॑यः प॒थाम् प॒थाम् प॑थि॒रक्ष॑य ऐलबृ॒दा ऐ॑लबृ॒दाः प॑थि॒रक्ष॑यः प॒थाम् प॒थाम् प॑थि॒रक्ष॑य ऐलबृ॒दाः । \newline
56. प॒थि॒रक्ष॑य ऐलबृ॒दा ऐ॑लबृ॒दाः प॑थि॒रक्ष॑यः पथि॒रक्ष॑य ऐलबृ॒दा य॒व्युधो॑ य॒व्युध॑ ऐलबृ॒दाः प॑थि॒रक्ष॑यः पथि॒रक्ष॑य ऐलबृ॒दा य॒व्युधः॑ । \newline
57. प॒थि॒रक्ष॑य॒ इति॑ पथि - रक्ष॑यः । \newline
58. ऐ॒ल॒बृ॒दा य॒व्युधो॑ य॒व्युध॑ ऐलबृ॒दा ऐ॑लबृ॒दा य॒व्युधः॑ । \newline
59. य॒व्युध॒ इति॑ य॒व्युधः॑ । \newline
60. ये ती॒र्थानि॑ ती॒र्थानि॒ ये ये ती॒र्थानि॑ प्र॒चर॑न्ति प्र॒चर॑न्ति ती॒र्थानि॒ ये ये ती॒र्थानि॑ प्र॒चर॑न्ति । \newline
61. ती॒र्थानि॑ प्र॒चर॑न्ति प्र॒चर॑न्ति ती॒र्थानि॑ ती॒र्थानि॑ प्र॒चर॑न्ति सृ॒काव॑न्तः सृ॒काव॑न्तः प्र॒चर॑न्ति ती॒र्थानि॑ ती॒र्थानि॑ प्र॒चर॑न्ति सृ॒काव॑न्तः । \newline
\pagebreak
\markright{ TS 4.5.11.2  \hfill https://www.vedavms.in \hfill}

\section{ TS 4.5.11.2 }

\textbf{TS 4.5.11.2 } \newline
\textbf{Samhita Paata} \newline

प्र॒चर॑न्ति सृ॒काव॑न्तो निष॒ङ्गिणः॑ ॥                                     य ए॒ताव॑न्तश्च॒ भूयाꣳ॑सश्च॒ दिशो॑ रु॒द्रा वि॑तस्थि॒रे ॥ तेषाꣳ॑ सहस्रयोज॒ने ऽव॒धन्वा॑नि तन्मसि ॥                                              नमो॑ रु॒द्रेभ्यो॒ ये पृ॑थि॒व्यां ॅये᳚ऽन्तरि॑क्षे॒ ये दि॒वि येषा॒मन्नं॒ ॅवातो॑ व॒र्॒.षमिष॑व॒स्तेभ्यो॒ दश॒ प्राची॒ र्दश॑दक्षि॒णा दश॑प्र॒तीची॒ र्दशोदी॑ची॒ र्दशो॒र्द्ध्वा-स्तेभ्यो॒ नम॒स्ते नो॑ मृडयन्तु॒ ते यं द्वि॒ष्मो यश्च॑ ( ) नो॒ द्वेष्टि॒ तं ॅवो॒ जंभे॑ दधामि ॥ \newline

\textbf{Pada Paata} \newline

प्र॒चर॒न्तीति॑ प्र - चर॑न्ति । सृ॒काव॑न्त॒ इति॑ सृ॒का - व॒न्तः॒ । नि॒ष॒ङ्गिण॒ इति॑ नि - स॒ङ्गिनः॑ ॥ ये । ए॒ताव॑न्तः । च॒ । भूयाꣳ॑सः । च॒ । दिशः॑ । रु॒द्राः । वि॒त॒स्थि॒र इति॑ वि - त॒स्थि॒रे ॥ तेषा᳚म् । स॒ह॒स्र॒यो॒ज॒न इति॑ सहस्र - यो॒ज॒ने । अवेति॑ । धन्वा॑नि । त॒न्म॒सि॒ ॥ नमः॑ । रु॒द्रेभ्यः॑ । ये । पृ॒थि॒व्याम् । ये । अ॒न्तरि॑क्षे । ये । दि॒वि । येषा᳚म् । अन्न᳚म् । वातः॑ । व॒र्॒.षम् । इष॑वः । तेभ्यः॑ । दश॑ । प्राचीः᳚ । दश॑ । द॒क्षि॒णा । दश॑ । प्र॒तीचीः᳚ । दश॑ । उदी॑चीः । दश॑ । ऊ॒द्‌र्ध्वाः । तेभ्यः॑ । नमः॑ । ते । नः॒ । मृ॒ड॒य॒न्तु॒ । ते॒ । यम् । द्वि॒ष्मः । यः । च॒ ( ) । नः॒ । द्वेष्टि॑ । तम् । वः॒ । जंभे᳚ । द॒धा॒मि॒ ॥  \newline


\textbf{Krama Paata} \newline

प्र॒चर॑न्ति सृ॒काव॑न्तः । प्र॒चर॒न्तीति॑ प्र - चर॑न्ति । सृ॒काव॑न्तो निष॒ङ्गिणः॑ । सृ॒काव॑न्त॒ इति॑ सृ॒का - व॒न्तः॒ । नि॒ष॒ङ्गिण॒ इति॑ नि - स॒ङ्गिनः॑ ॥ य ए॒ताव॑न्तः । ए॒ताव॑न्तश्च । च॒ भूयाꣳ॑सः । भूयाꣳ॑सश्च । च दिशः॑ । दिशो॑ रु॒द्राः । रु॒द्रा वि॑तस्थि॒रे । वि॒त॒स्थि॒र इति॑ वि - त॒स्थि॒रे ॥ तेषाꣳ॑ सहस्रयोज॒ने । स॒ह॒स्र॒यो॒ज॒नेऽव॑ । स॒ह॒स्र॒यो॒ज॒न इति॑ सहस्र - यो॒ज॒ने । अव॒ धन्वा॑नि । धन्वा॑नि तन्मसि । त॒न्म॒सीति॑ तन्मसि ॥ नमो॑ रु॒द्रेभ्यः॑ । रु॒द्रेभ्यो॒ ये । ये पृ॑थि॒व्याम् । पृ॒थि॒व्यां ॅये । ये᳚ऽन्तरि॑क्षे । अ॒न्तरि॑क्षे॒ ये । ये दि॒वि । दि॒वि येषा᳚म् । येषा॒मन्न᳚म् । अन्न॒म् ॅवातः॑ । वातो॑ व॒र्.॒षम् । व॒र्.॒षमिष॑वः । इष॑व॒स्तेभ्यः॑ । तेभ्यो॒ दश॑ । दश॒ प्राचीः᳚ । प्राची॒र् दश॑ । दश॑ दक्षि॒णा । द॒क्षि॒णा दश॑ । दश॑ प्र॒तीचीः᳚ । प्र॒तीची॒र् दश॑ । दशोदी॑चीः । उदी॑ची॒र् दश॑ । दशो॒र्द्ध्वाः । ऊ॒र्द्ध्वास्तेभ्यः॑ । तेभ्यो॒ नमः॑ । नम॒स्ते । ते नः॑ । नो॒ मृ॒ड॒य॒न्तु॒ । मृ॒ड॒य॒न्तु॒ ते । ते यम् । यं द्वि॒ष्मः । द्वि॒ष्मो यः । यश्च॑ ( ) । च॒ नः॒ । नो॒ द्वेष्टि॑ । द्वेष्टि॒ तम् । तं ॅवः॑ । वो॒ जंभे᳚ । जंभे॑ दधामि । द॒धा॒मीति॑ दधामि । \newline

\textbf{Jatai Paata} \newline

1. प्र॒चर॑न्ति सृ॒काव॑न्तः सृ॒काव॑न्तः प्र॒चर॑न्ति प्र॒चर॑न्ति सृ॒काव॑न्तः । \newline
2. प्र॒चर॒न्तीति॑ प्र - चर॑न्ति । \newline
3. सृ॒काव॑न्तो निष॒ङ्गिणो॑ निष॒ङ्गिणः॑ सृ॒काव॑न्तः सृ॒काव॑न्तो निष॒ङ्गिणः॑ । \newline
4. सृ॒काव॑न्त॒ इति॑ सृ॒का - व॒न्तः॒ । \newline
5. नि॒ष॒ङ्गिण॒ इति॑ नि - स॒ङ्गिनः॑ । \newline
6. य ए॒ताव॑न्त ए॒ताव॑न्तो॒ ये य ए॒ताव॑न्तः । \newline
7. ए॒ताव॑न्तश्च चै॒ताव॑न्त ए॒ताव॑न्तश्च । \newline
8. च॒ भूया(ग्म्॑)सो॒ भूया(ग्म्॑)सश्च च॒ भूया(ग्म्॑)सः । \newline
9. भूया(ग्म्॑)सश्च च॒ भूया(ग्म्॑)सो॒ भूया(ग्म्॑)सश्च । \newline
10. च॒ दिशो॒ दिश॑श्च च॒ दिशः॑ । \newline
11. दिशो॑ रु॒द्रा रु॒द्रा दिशो॒ दिशो॑ रु॒द्राः । \newline
12. रु॒द्रा वि॑तस्थि॒रे वि॑तस्थि॒रे रु॒द्रा रु॒द्रा वि॑तस्थि॒रे । \newline
13. वि॒त॒स्थि॒र इति॑ वि - त॒स्थि॒रे । \newline
14. तेषा(ग्म्॑) सहस्रयोज॒ने स॑हस्रयोज॒ने तेषा॒म् तेषा(ग्म्॑) सहस्रयोज॒ने । \newline
15. स॒ह॒स्र॒यो॒ज॒ने ऽवाव॑ सहस्रयोज॒ने स॑हस्रयोज॒ने ऽव॑ । \newline
16. स॒ह॒स्र॒यो॒ज॒न इति॑ सहस्र - यो॒ज॒ने । \newline
17. अव॒ धन्वा॑नि॒ धन्वा॒ न्यवाव॒ धन्वा॑नि । \newline
18. धन्वा॑नि तन्मसि तन्मसि॒ धन्वा॑नि॒ धन्वा॑नि तन्मसि । \newline
19. त॒न्म॒सीति॑ तन्मसि । \newline
20. नमो॑ रु॒द्रेभ्यो॑ रु॒द्रेभ्यो॒ नमो॒ नमो॑ रु॒द्रेभ्यः॑ । \newline
21. रु॒द्रेभ्यो॒ ये ये रु॒द्रेभ्यो॑ रु॒द्रेभ्यो॒ ये । \newline
22. ये पृ॑थि॒व्याम् पृ॑थि॒व्यां ॅये ये पृ॑थि॒व्याम् । \newline
23. पृ॒थि॒व्यां ॅये ये पृ॑थि॒व्याम् पृ॑थि॒व्यां ॅये । \newline
24. ये᳚ ऽन्तरि॑क्षे॒ ऽन्तरि॑क्षे॒ ये ये᳚ ऽन्तरि॑क्षे । \newline
25. अ॒न्तरि॑क्षे॒ ये ये᳚ ऽन्तरि॑क्षे॒ ऽन्तरि॑क्षे॒ ये । \newline
26. ये दि॒वि दि॒वि ये ये दि॒वि । \newline
27. दि॒वि येषां॒ ॅयेषा᳚म् दि॒वि दि॒वि येषा᳚म् । \newline
28. येषा॒ मन्न॒ मन्नं॒ ॅयेषां॒ ॅयेषा॒ मन्न᳚म् । \newline
29. अन्नं॒ ॅवातो॒ वातो ऽन्न॒ मन्नं॒ ॅवातः॑ । \newline
30. वातो॑ व॒र्॒.षं ॅव॒र्॒.षं ॅवातो॒ वातो॑ व॒र्॒.षम् । \newline
31. व॒र्॒.ष मिष॑व॒ इष॑वो व॒र्॒.षं ॅव॒र्॒.ष मिष॑वः । \newline
32. इष॑व॒ स्तेभ्य॒ स्तेभ्य॒ इष॑व॒ इष॑व॒ स्तेभ्यः॑ । \newline
33. तेभ्यो॒ दश॒ दश॒ तेभ्य॒ स्तेभ्यो॒ दश॑ । \newline
34. दश॒ प्राचीः॒ प्राची॒र् दश॒ दश॒ प्राचीः᳚ । \newline
35. प्राची॒र् दश॒ दश॒ प्राचीः॒ प्राची॒र् दश॑ । \newline
36. दश॑ दक्षि॒णा द॑क्षि॒णा दश॒ दश॑ दक्षि॒णा । \newline
37. द॒क्षि॒णा दश॒ दश॑ दक्षि॒णा द॑क्षि॒णा दश॑ । \newline
38. दश॑ प्र॒तीचीः᳚ प्र॒तीची॒र् दश॒ दश॑ प्र॒तीचीः᳚ । \newline
39. प्र॒तीची॒र् दश॒ दश॑ प्र॒तीचीः᳚ प्र॒तीची॒र् दश॑ । \newline
40. दशोदी॑ची॒ रुदी॑ची॒र् दश॒ दशोदी॑चीः । \newline
41. उदी॑ची॒र् दश॒ दशोदी॑ची॒ रुदी॑ची॒र् दश॑ । \newline
42. दशो॒र्द्ध्वा ऊ॒र्द्ध्वा दश॒ दशो॒र्द्ध्वाः । \newline
43. ऊ॒र्द्ध्वा स्तेभ्य॒ स्तेभ्य॑ ऊ॒र्द्ध्वा ऊ॒र्द्ध्वा स्तेभ्यः॑ । \newline
44. तेभ्यो॒ नमो॒ नम॒ स्तेभ्य॒ स्तेभ्यो॒ नमः॑ । \newline
45. नम॒ स्ते ते नमो॒ नम॒ स्ते । \newline
46. ते नो॑ न॒ स्ते ते नः॑ । \newline
47. नो॒ मृ॒ड॒य॒न्तु॒ मृ॒ड॒य॒न्तु॒ नो॒ नो॒ मृ॒ड॒य॒न्तु॒ । \newline
48. मृ॒ड॒य॒न्तु॒ ते ते मृ॑डयन्तु मृडयन्तु॒ ते । \newline
49. ते यं ॅयम् ते ते यम् । \newline
50. यम् द्वि॒ष्मो द्वि॒ष्मो यं ॅयम् द्वि॒ष्मः । \newline
51. द्वि॒ष्मो यो यो द्वि॒ष्मो द्वि॒ष्मो यः । \newline
52. यश्च॑ च॒ यो यश्च॑ । \newline
53. च॒ नो॒ न॒श्च॒ च॒ नः॒ । \newline
54. नो॒ द्वेष्टि॒ द्वेष्टि॑ नो नो॒ द्वेष्टि॑ । \newline
55. द्वेष्टि॒ तम् तम् द्वेष्टि॒ द्वेष्टि॒ तम् । \newline
56. तं ॅवो॑ व॒ स्तम् तं ॅवः॑ । \newline
57. वो॒ जंभे॒ जंभे॑ वो वो॒ जंभे᳚ । \newline
58. जंभे॑ दधामि दधामि॒ जंभे॒ जंभे॑ दधामि । \newline
59. द॒धा॒मीति॑ दधामि । \newline

\textbf{Ghana Paata } \newline

1. प्र॒चर॑न्ति सृ॒काव॑न्तः सृ॒काव॑न्तः प्र॒चर॑न्ति प्र॒चर॑न्ति सृ॒काव॑न्तो निष॒ङ्गिणो॑ निष॒ङ्गिणः॑ सृ॒काव॑न्तः प्र॒चर॑न्ति प्र॒चर॑न्ति सृ॒काव॑न्तो निष॒ङ्गिणः॑ । \newline
2. प्र॒चर॒न्तीति॑ प्र - चर॑न्ति । \newline
3. सृ॒काव॑न्तो निष॒ङ्गिणो॑ निष॒ङ्गिणः॑ सृ॒काव॑न्तः सृ॒काव॑न्तो निष॒ङ्गिणः॑ । \newline
4. सृ॒काव॑न्त॒ इति॑ सृ॒का - व॒न्तः॒ । \newline
5. नि॒ष॒ङ्गिण॒ इति॑ नि - स॒ङ्गिनः॑ । \newline
6. य ए॒ताव॑न्त ए॒ताव॑न्तो॒ ये य ए॒ताव॑न्तश्च चै॒ताव॑न्तो॒ ये य ए॒ताव॑न्तश्च । \newline
7. ए॒ताव॑न्तश्च चै॒ताव॑न्त ए॒ताव॑न्तश्च॒ भूयाꣳ॑सो॒ भूयाꣳ॑स श्चै॒ताव॑न्त ए॒ताव॑न्तश्च॒ भूयाꣳ॑सः । \newline
8. च॒ भूयाꣳ॑सो॒ भूयाꣳ॑सश्च च॒ भूयाꣳ॑सश्च च॒ भूयाꣳ॑सश्च च॒ भूयाꣳ॑सश्च । \newline
9. भूयाꣳ॑सश्च च॒ भूयाꣳ॑सो॒ भूयाꣳ॑सश्च॒ दिशो॒ दिश॑श्च॒ भूयाꣳ॑सो॒ भूयाꣳ॑सश्च॒ दिशः॑ । \newline
10. च॒ दिशो॒ दिश॑श्च च॒ दिशो॑ रु॒द्रा रु॒द्रा दिश॑श्च च॒ दिशो॑ रु॒द्राः । \newline
11. दिशो॑ रु॒द्रा रु॒द्रा दिशो॒ दिशो॑ रु॒द्रा वि॑तस्थि॒रे वि॑तस्थि॒रे रु॒द्रा दिशो॒ दिशो॑ रु॒द्रा वि॑तस्थि॒रे । \newline
12. रु॒द्रा वि॑तस्थि॒रे वि॑तस्थि॒रे रु॒द्रा रु॒द्रा वि॑तस्थि॒रे । \newline
13. वि॒त॒स्थि॒र इति॑ वि - त॒स्थि॒रे । \newline
14. तेषाꣳ॑ सहस्रयोज॒ने स॑हस्रयोज॒ने तेषा॒म् तेषाꣳ॑ सहस्रयोज॒ने ऽवाव॑ सहस्रयोज॒ने तेषा॒म् तेषाꣳ॑ सहस्रयोज॒ने ऽव॑ । \newline
15. स॒ह॒स्र॒यो॒ज॒ने ऽवाव॑ सहस्रयोज॒ने स॑हस्रयोज॒ने ऽव॒ धन्वा॑नि॒ धन्वा॒ न्यव॑ सहस्रयोज॒ने स॑हस्रयोज॒ने ऽव॒ धन्वा॑नि । \newline
16. स॒ह॒स्र॒यो॒ज॒न इति॑ सहस्र - यो॒ज॒ने । \newline
17. अव॒ धन्वा॑नि॒ धन्वा॒ न्यवाव॒ धन्वा॑नि तन्मसि तन्मसि॒ धन्वा॒ न्यवाव॒ धन्वा॑नि तन्मसि । \newline
18. धन्वा॑नि तन्मसि तन्मसि॒ धन्वा॑नि॒ धन्वा॑नि तन्मसि । \newline
19. त॒न्म॒सीति॑ तन्मसि । \newline
20. नमो॑ रु॒द्रेभ्यो॑ रु॒द्रेभ्यो॒ नमो॒ नमो॑ रु॒द्रेभ्यो॒ ये ये रु॒द्रेभ्यो॒ नमो॒ नमो॑ रु॒द्रेभ्यो॒ ये । \newline
21. रु॒द्रेभ्यो॒ ये ये रु॒द्रेभ्यो॑ रु॒द्रेभ्यो॒ ये पृ॑थि॒व्याम् पृ॑थि॒व्यां ॅये रु॒द्रेभ्यो॑ रु॒द्रेभ्यो॒ ये पृ॑थि॒व्याम् । \newline
22. ये पृ॑थि॒व्याम् पृ॑थि॒व्यां ॅये ये पृ॑थि॒व्यां ॅये ये पृ॑थि॒व्यां ॅये ये पृ॑थि॒व्यां ॅये । \newline
23. पृ॒थि॒व्यां ॅये ये पृ॑थि॒व्याम् पृ॑थि॒व्यां ॅये᳚ ऽन्तरि॑क्षे॒ ऽन्तरि॑क्षे॒ ये पृ॑थि॒व्याम् पृ॑थि॒व्यां ॅये᳚ ऽन्तरि॑क्षे । \newline
24. ये᳚ ऽन्तरि॑क्षे॒ ऽन्तरि॑क्षे॒ ये ये᳚ ऽन्तरि॑क्षे॒ ये ये᳚ ऽन्तरि॑क्षे॒ ये ये᳚ ऽन्तरि॑क्षे॒ ये । \newline
25. अ॒न्तरि॑क्षे॒ ये ये᳚ ऽन्तरि॑क्षे॒ ऽन्तरि॑क्षे॒ ये दि॒वि दि॒वि ये᳚ ऽन्तरि॑क्षे॒ ऽन्तरि॑क्षे॒ ये दि॒वि । \newline
26. ये दि॒वि दि॒वि ये ये दि॒वि येषां॒ ॅयेषा᳚म् दि॒वि ये ये दि॒वि येषा᳚म् । \newline
27. दि॒वि येषां॒ ॅयेषा᳚म् दि॒वि दि॒वि येषा॒ मन्न॒ मन्नं॒ ॅयेषा᳚म् दि॒वि दि॒वि येषा॒ मन्न᳚म् । \newline
28. येषा॒ मन्न॒ मन्नं॒ ॅयेषां॒ ॅयेषा॒ मन्नं॒ ॅवातो॒ वातो ऽन्नं॒ ॅयेषां॒ ॅयेषा॒ मन्नं॒ ॅवातः॑ । \newline
29. अन्नं॒ ॅवातो॒ वातो ऽन्न॒ मन्नं॒ ॅवातो॑ व॒र्॒.षं ॅव॒र्॒.षं ॅवातो ऽन्न॒ मन्नं॒ ॅवातो॑ व॒र्॒.षम् । \newline
30. वातो॑ व॒र्॒.षं ॅव॒र्॒.षं ॅवातो॒ वातो॑ व॒र्॒.ष मिष॑व॒ इष॑वो व॒र्॒.षं ॅवातो॒ वातो॑ व॒र्॒.ष मिष॑वः । \newline
31. व॒र्॒.ष मिष॑व॒ इष॑वो व॒र्॒.षं ॅव॒र्॒.ष मिष॑व॒ स्तेभ्य॒ स्तेभ्य॒ इष॑वो व॒र्॒.षं ॅव॒र्॒.ष मिष॑व॒ स्तेभ्यः॑ । \newline
32. इष॑व॒ स्तेभ्य॒ स्तेभ्य॒ इष॑व॒ इष॑व॒ स्तेभ्यो॒ दश॒ दश॒ तेभ्य॒ इष॑व॒ इष॑व॒ स्तेभ्यो॒ दश॑ । \newline
33. तेभ्यो॒ दश॒ दश॒ तेभ्य॒ स्तेभ्यो॒ दश॒ प्राचीः॒ प्राची॒र् दश॒ तेभ्य॒ स्तेभ्यो॒ दश॒ प्राचीः᳚ । \newline
34. दश॒ प्राचीः॒ प्राची॒र् दश॒ दश॒ प्राची॒र् दश॒ दश॒ प्राची॒र् दश॒ दश॒ प्राची॒र् दश॑ । \newline
35. प्राची॒र् दश॒ दश॒ प्राचीः॒ प्राची॒र् दश॑ दक्षि॒णा द॑क्षि॒णा दश॒ प्राचीः॒ प्राची॒र् दश॑ दक्षि॒णा । \newline
36. दश॑ दक्षि॒णा द॑क्षि॒णा दश॒ दश॑ दक्षि॒णा दश॒ दश॑ दक्षि॒णा दश॒ दश॑ दक्षि॒णा दश॑ । \newline
37. द॒क्षि॒णा दश॒ दश॑ दक्षि॒णा द॑क्षि॒णा दश॑ प्र॒तीचीः᳚ प्र॒तीची॒र् दश॑ दक्षि॒णा द॑क्षि॒णा दश॑ प्र॒तीचीः᳚ । \newline
38. दश॑ प्र॒तीचीः᳚ प्र॒तीची॒र् दश॒ दश॑ प्र॒तीची॒र् दश॒ दश॑ प्र॒तीची॒र् दश॒ दश॑ प्र॒तीची॒र् दश॑ । \newline
39. प्र॒तीची॒र् दश॒ दश॑ प्र॒तीचीः᳚ प्र॒तीची॒र् दशोदी॑ची॒ रुदी॑ची॒र् दश॑ प्र॒तीचीः᳚ प्र॒तीची॒र् दशोदी॑चीः । \newline
40. दशोदी॑ची॒ रुदी॑ची॒र् दश॒ दशोदी॑ची॒र् दश॒ दशोदी॑ची॒र् दश॒ दशोदी॑ची॒र् दश॑ । \newline
41. उदी॑ची॒र् दश॒ दशोदी॑ची॒ रुदी॑ची॒र् दशो॒र्द्ध्वा ऊ॒र्द्ध्वा दशोदी॑ची॒ रुदी॑ची॒र् दशो॒र्द्ध्वाः । \newline
42. दशो॒र्द्ध्वा ऊ॒र्द्ध्वा दश॒ दशो॒र्द्ध्वा स्तेभ्य॒ स्तेभ्य॑ ऊ॒र्द्ध्वा दश॒ दशो॒र्द्ध्वा स्तेभ्यः॑ । \newline
43. ऊ॒र्द्ध्वा स्तेभ्य॒ स्तेभ्य॑ ऊ॒र्द्ध्वा ऊ॒र्द्ध्वा स्तेभ्यो॒ नमो॒ नम॒ स्तेभ्य॑ ऊ॒र्द्ध्वा ऊ॒र्द्ध्वा स्तेभ्यो॒ नमः॑ । \newline
44. तेभ्यो॒ नमो॒ नम॒ स्तेभ्य॒ स्तेभ्यो॒ नम॒स्ते ते नम॒ स्तेभ्य॒ स्तेभ्यो॒ नम॒स्ते । \newline
45. नम॒स्ते ते नमो॒ नम॒स्ते नो॑ न॒स्ते नमो॒ नम॒स्ते नः॑ । \newline
46. ते नो॑ न॒स्ते ते नो॑ मृडयन्तु मृडयन्तु न॒स्ते ते नो॑ मृडयन्तु । \newline
47. नो॒ मृ॒ड॒य॒न्तु॒ मृ॒ड॒य॒न्तु॒ नो॒ नो॒ मृ॒ड॒य॒न्तु॒ ते ते मृ॑डयन्तु नो नो मृडयन्तु॒ ते । \newline
48. मृ॒ड॒य॒न्तु॒ ते ते मृ॑डयन्तु मृडयन्तु॒ ते यं ॅयम् ते मृ॑डयन्तु मृडयन्तु॒ ते यम् । \newline
49. ते यं ॅयम् ते ते यम् द्वि॒ष्मो द्वि॒ष्मो यम् ते ते यम् द्वि॒ष्मः । \newline
50. यम् द्वि॒ष्मो द्वि॒ष्मो यं ॅयम् द्वि॒ष्मो यो यो द्वि॒ष्मो यं ॅयम् द्वि॒ष्मो यः । \newline
51. द्वि॒ष्मो यो यो द्वि॒ष्मो द्वि॒ष्मो यश्च॑ च॒ यो द्वि॒ष्मो द्वि॒ष्मो यश्च॑ । \newline
52. यश्च॑ च॒ यो यश्च॑ नो नश्च॒ यो यश्च॑ नः । \newline
53. च॒ नो॒ न॒श्च॒ च॒ नो॒ द्वेष्टि॒ द्वेष्टि॑ नश्च च नो॒ द्वेष्टि॑ । \newline
54. नो॒ द्वेष्टि॒ द्वेष्टि॑ नो नो॒ द्वेष्टि॒ तम् तम् द्वेष्टि॑ नो नो॒ द्वेष्टि॒ तम् । \newline
55. द्वेष्टि॒ तम् तम् द्वेष्टि॒ द्वेष्टि॒ तं ॅवो॑ व॒ स्तम् द्वेष्टि॒ द्वेष्टि॒ तं ॅवः॑ । \newline
56. तं ॅवो॑ व॒स्तम् तं ॅवो॒ जंभे॒ जंभे॑ व॒ स्तम् तं ॅवो॒ जंभे᳚ । \newline
57. वो॒ जंभे॒ जंभे॑ वो वो॒ जंभे॑ दधामि दधामि॒ जंभे॑ वो वो॒ जंभे॑ दधामि । \newline
58. जंभे॑ दधामि दधामि॒ जंभे॒ जंभे॑ दधामि । \newline
59. द॒धा॒मीति॑ दधामि । \newline
\pagebreak


\end{document}